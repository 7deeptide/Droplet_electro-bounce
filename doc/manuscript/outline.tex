\documentclass[a4paper, 12pt]{article}
\usepackage{changepage}
\usepackage{color,soul}
\usepackage{listings}
\usepackage{pgf}
\usepackage{graphicx}
\graphicspath{ {../figures/} }
%\usepackage{svg}
\usepackage{verbatim}
\lstset{
basicstyle=\small\ttfamily,
columns=flexible,
breaklines=true
}
\title{\textsf{\textbf{Droplet Electro-Bouncing in Low-Gravity}}}
\vspace{-25mm}
\author{Erin S. Schmidt, Mark M. Weislogel}
\date{}
\usepackage{setspace}
\usepackage{abstract}
\renewcommand{\abstractnamefont}{\normalfont\bfseries}
\renewcommand{\abstracttextfont}{\normalfont\small\itshape}

\begin{document}
\maketitle

\begin{abstract}
\noindent
It's an abstract.
\end{abstract}
\doublespacing
\section{Introduction}


\section{Overview}
In design of fluidic systems for the environment of space we are often forced to deal with the inadequacy of our terrestrial born expectations about the behaviors of the physics of fluids.   

\begin{itemize}
\item In the absence of gravity the influence of usually negligible electrostatic forces can surprise the unready. We have observed electrostatic accelerations 0.5 $mL$ water droplets on the order of 10's of $cm/s^2$ in proximity to charged surfaces. It's quite easy to (accidentally) charge dielectric surfaces as well. Simply washing fluorinated surfaces with water can induce large surface potentials (hundreds of volts).

\item We can harness these forces for useful applications in a low-gravity environment. This ties into the dreaded ``phase-separation'' problem of low-gravity which arises in the absence of a body force like gravity. Some decades of research has focused on the various manifestations of this problem (tank venting, propellant settling, gas ingestion during tank draining or PMD rewetting, mixing, etc.) and their means of resolution, usually by artifice of a susbstitute body force (such as electric forces). Applications for means of manipulating large droplets in microgravity include electrostatic precipitators, condensing heat exchanges, droplet radiators, droplet column reactors, and container-less processing.

\item Mainly we are concerned about what parameters are important in electrostatic transport of relatively large (e.g. millimetric) droplets in low-gravity, and what the values of the respective dimensionless groups might be. This raises an interesting challenge; determination of net droplet electric charge beyond ideal cases we system capacitance is easily determined (such as between parallel plane electrodes).

\end{itemize}
\section{Methods}
To find typical vales of these parameters we employed spontaneous droplet jumps on charged dielectric super-hydrophobic surfaces under low-gravity conditions in a 2.1 $s$ drop tower. Using high-speed video and image analysis software we captured the trajectories of the droplets. Then we solved the inverse problem to find the key parameters by minimizing the $\chi^2$ goodness-of-fit statistic between an observed trajectory and the trajectory predicted by a dynamical model given that certain set parameters. The best fit parameters obtained by direct-search are those corresponding to the maximum likelihood experimental values of the parameters subject to the validity of the assumptions encoded in the model, and the constraints set by the measurement precision of measured independent variables.

\subsection{Experimental}
\begin{itemize}
\item A very low-tech superhydrophobic electret was produced, with surface potentials 0.7-4.0 $kV$ and contact angles $\sim 150^{\circ}$ with approximately $20^{\circ}$ contact angle hysteresis when uncharged. The dielectrics are a lamina of 3-4 corona charged 0.4 $mm$ PMMA sheets. The electric field scales with the number of dielectric lamina. The final, superhydropobic layer, is produced by laser etching PMMA, and depositing a thin layer of PTFE on the resulting roughness topology to increase the Young's angle. The surface charge density can be modulated during the experiment by means of a 0-2 $kV$ DC-DC converter, which can re-polarize the dielectric substrate by means of an embedded electrode, and the resulting bound charge partially or fully neutralizes the electric field due to the surface ions deposited by corona charging of the electret.
\end{itemize}   

\subsection{Parameter Estimation}
\begin{itemize}
\item The solution of the inverse problem  was achieved by direct search, using the \emph{Nelder-Mead} simplex algorithm. \emph{Nelder-Mead} is not robust to noise in the objective function so the experimental data and its approximate derivatives were interpolated by \emph{Savitsky-Golay} smoothing splines.
\end{itemize}

\section{Results}
\begin{itemize}
\item We found the distribution of mostly likely experimental net charges for a population of the droplets jumped in low-gravity. We found the charge to be a function of droplet volume and surface potential of the dielectric substrate. A two-ways T-test with a charge distribution determined by a corollary experiment suggests that the droplet charge is induced by the electric field (rather than through contact charging on the PTFE layer).

\item The bounce dynamics are controlled by a dimensionless ratio of electrostatic force to inertia. The dielectrophoretic force plays a very small role when droplets have net charge in a DC field.  

\item Using the unique capabilities of the low-gravity environment we obtained data on dimensionless contact time and coefficients of restitution at very low Ohnesorge numbers for a range of electric Bond numbers. Despite strong electric fields (20-30 $kV/cm$) we found little evidence for wetting transitions due to excession of a critical pressure (the ``Fakir impalement''). There is no obvious trend in dimensionless contact time or coefficient of restitution with electric Bond number.

\item Jump velocities are more strongly damped for relatively small droplet volumes in the presence of the electric fields than was shown by Attari \emph{et. al.}. This may be evidence for electrowetting paradoxically enhancing the effect of contact angle hysteresis pinning on sharp corners. (How does this tie into the coefficients of restitution problem?)

\item By scale arguments and perturbation of solutions to the equations of motion we find several simple rules of thumb in droplet ``escape velocity'' of impacts, length scales and time scales for returns.

\begin{figure}[htb]
    \centering
    %% Creator: Matplotlib, PGF backend
%%
%% To include the figure in your LaTeX document, write
%%   \input{<filename>.pgf}
%%
%% Make sure the required packages are loaded in your preamble
%%   \usepackage{pgf}
%%
%% Figures using additional raster images can only be included by \input if
%% they are in the same directory as the main LaTeX file. For loading figures
%% from other directories you can use the `import` package
%%   \usepackage{import}
%% and then include the figures with
%%   \import{<path to file>}{<filename>.pgf}
%%
%% Matplotlib used the following preamble
%%   \usepackage{fontspec}
%%   \setmainfont{DejaVu Serif}
%%   \setsansfont{DejaVu Sans}
%%   \setmonofont{DejaVu Sans Mono}
%%
\begingroup%
\makeatletter%
\begin{pgfpicture}%
\pgfpathrectangle{\pgfpointorigin}{\pgfqpoint{5.194240in}{3.788793in}}%
\pgfusepath{use as bounding box, clip}%
\begin{pgfscope}%
\pgfsetbuttcap%
\pgfsetmiterjoin%
\definecolor{currentfill}{rgb}{1.000000,1.000000,1.000000}%
\pgfsetfillcolor{currentfill}%
\pgfsetlinewidth{0.000000pt}%
\definecolor{currentstroke}{rgb}{1.000000,1.000000,1.000000}%
\pgfsetstrokecolor{currentstroke}%
\pgfsetdash{}{0pt}%
\pgfpathmoveto{\pgfqpoint{0.000000in}{0.000000in}}%
\pgfpathlineto{\pgfqpoint{5.194240in}{0.000000in}}%
\pgfpathlineto{\pgfqpoint{5.194240in}{3.788793in}}%
\pgfpathlineto{\pgfqpoint{0.000000in}{3.788793in}}%
\pgfpathclose%
\pgfusepath{fill}%
\end{pgfscope}%
\begin{pgfscope}%
\pgfsetbuttcap%
\pgfsetmiterjoin%
\definecolor{currentfill}{rgb}{1.000000,1.000000,1.000000}%
\pgfsetfillcolor{currentfill}%
\pgfsetlinewidth{0.000000pt}%
\definecolor{currentstroke}{rgb}{0.000000,0.000000,0.000000}%
\pgfsetstrokecolor{currentstroke}%
\pgfsetstrokeopacity{0.000000}%
\pgfsetdash{}{0pt}%
\pgfpathmoveto{\pgfqpoint{0.634105in}{0.521603in}}%
\pgfpathlineto{\pgfqpoint{4.354105in}{0.521603in}}%
\pgfpathlineto{\pgfqpoint{4.354105in}{3.541603in}}%
\pgfpathlineto{\pgfqpoint{0.634105in}{3.541603in}}%
\pgfpathclose%
\pgfusepath{fill}%
\end{pgfscope}%
\begin{pgfscope}%
\pgfpathrectangle{\pgfqpoint{0.634105in}{0.521603in}}{\pgfqpoint{3.720000in}{3.020000in}} %
\pgfusepath{clip}%
\pgfsetbuttcap%
\pgfsetroundjoin%
\definecolor{currentfill}{rgb}{0.061765,0.061765,0.085934}%
\pgfsetfillcolor{currentfill}%
\pgfsetlinewidth{0.000000pt}%
\definecolor{currentstroke}{rgb}{0.000000,0.000000,0.000000}%
\pgfsetstrokecolor{currentstroke}%
\pgfsetdash{}{0pt}%
\pgfpathmoveto{\pgfqpoint{1.369836in}{0.740556in}}%
\pgfpathlineto{\pgfqpoint{1.437974in}{0.740556in}}%
\pgfpathlineto{\pgfqpoint{1.506111in}{0.740556in}}%
\pgfpathlineto{\pgfqpoint{1.574249in}{0.740556in}}%
\pgfpathlineto{\pgfqpoint{1.642386in}{0.740556in}}%
\pgfpathlineto{\pgfqpoint{1.710524in}{0.740556in}}%
\pgfpathlineto{\pgfqpoint{1.778661in}{0.740556in}}%
\pgfpathlineto{\pgfqpoint{1.846799in}{0.795494in}}%
\pgfpathlineto{\pgfqpoint{1.914936in}{0.795494in}}%
\pgfpathlineto{\pgfqpoint{1.983074in}{0.795494in}}%
\pgfpathlineto{\pgfqpoint{2.051211in}{0.795494in}}%
\pgfpathlineto{\pgfqpoint{2.119349in}{0.795494in}}%
\pgfpathlineto{\pgfqpoint{2.187486in}{0.795494in}}%
\pgfpathlineto{\pgfqpoint{2.255624in}{0.850432in}}%
\pgfpathlineto{\pgfqpoint{2.323761in}{0.905370in}}%
\pgfpathlineto{\pgfqpoint{2.337491in}{0.916441in}}%
\pgfpathlineto{\pgfqpoint{2.323761in}{0.911739in}}%
\pgfpathlineto{\pgfqpoint{2.269121in}{0.905370in}}%
\pgfpathlineto{\pgfqpoint{2.255624in}{0.903185in}}%
\pgfpathlineto{\pgfqpoint{2.247795in}{0.905370in}}%
\pgfpathlineto{\pgfqpoint{2.187486in}{0.927654in}}%
\pgfpathlineto{\pgfqpoint{2.119349in}{0.940331in}}%
\pgfpathlineto{\pgfqpoint{2.051211in}{0.952591in}}%
\pgfpathlineto{\pgfqpoint{2.005953in}{0.960308in}}%
\pgfpathlineto{\pgfqpoint{1.983074in}{0.964247in}}%
\pgfpathlineto{\pgfqpoint{1.914936in}{0.972393in}}%
\pgfpathlineto{\pgfqpoint{1.846799in}{0.980237in}}%
\pgfpathlineto{\pgfqpoint{1.778661in}{0.988081in}}%
\pgfpathlineto{\pgfqpoint{1.710524in}{0.995925in}}%
\pgfpathlineto{\pgfqpoint{1.642386in}{1.000048in}}%
\pgfpathlineto{\pgfqpoint{1.574249in}{0.991101in}}%
\pgfpathlineto{\pgfqpoint{1.506111in}{0.980260in}}%
\pgfpathlineto{\pgfqpoint{1.437974in}{0.966635in}}%
\pgfpathlineto{\pgfqpoint{1.428669in}{0.960308in}}%
\pgfpathlineto{\pgfqpoint{1.369836in}{0.923458in}}%
\pgfpathlineto{\pgfqpoint{1.301699in}{0.933018in}}%
\pgfpathlineto{\pgfqpoint{1.233561in}{0.942578in}}%
\pgfpathlineto{\pgfqpoint{1.165424in}{0.952138in}}%
\pgfpathlineto{\pgfqpoint{1.107195in}{0.960308in}}%
\pgfpathlineto{\pgfqpoint{1.097286in}{0.961699in}}%
\pgfpathlineto{\pgfqpoint{1.095198in}{0.961992in}}%
\pgfpathlineto{\pgfqpoint{1.097286in}{0.960308in}}%
\pgfpathlineto{\pgfqpoint{1.165424in}{0.905370in}}%
\pgfpathlineto{\pgfqpoint{1.233561in}{0.850432in}}%
\pgfpathlineto{\pgfqpoint{1.301699in}{0.795494in}}%
\pgfpathclose%
\pgfusepath{fill}%
\end{pgfscope}%
\begin{pgfscope}%
\pgfpathrectangle{\pgfqpoint{0.634105in}{0.521603in}}{\pgfqpoint{3.720000in}{3.020000in}} %
\pgfusepath{clip}%
\pgfsetbuttcap%
\pgfsetroundjoin%
\definecolor{currentfill}{rgb}{0.185294,0.185294,0.257801}%
\pgfsetfillcolor{currentfill}%
\pgfsetlinewidth{0.000000pt}%
\definecolor{currentstroke}{rgb}{0.000000,0.000000,0.000000}%
\pgfsetstrokecolor{currentstroke}%
\pgfsetdash{}{0pt}%
\pgfpathmoveto{\pgfqpoint{2.255624in}{0.903185in}}%
\pgfpathlineto{\pgfqpoint{2.269121in}{0.905370in}}%
\pgfpathlineto{\pgfqpoint{2.323761in}{0.911739in}}%
\pgfpathlineto{\pgfqpoint{2.337491in}{0.916441in}}%
\pgfpathlineto{\pgfqpoint{2.391899in}{0.960308in}}%
\pgfpathlineto{\pgfqpoint{2.460036in}{1.015247in}}%
\pgfpathlineto{\pgfqpoint{2.528174in}{1.070185in}}%
\pgfpathlineto{\pgfqpoint{2.596311in}{1.125123in}}%
\pgfpathlineto{\pgfqpoint{2.664449in}{1.180061in}}%
\pgfpathlineto{\pgfqpoint{2.732586in}{1.234999in}}%
\pgfpathlineto{\pgfqpoint{2.800724in}{1.289938in}}%
\pgfpathlineto{\pgfqpoint{2.828867in}{1.312630in}}%
\pgfpathlineto{\pgfqpoint{2.800724in}{1.297732in}}%
\pgfpathlineto{\pgfqpoint{2.785999in}{1.289938in}}%
\pgfpathlineto{\pgfqpoint{2.732586in}{1.261664in}}%
\pgfpathlineto{\pgfqpoint{2.682214in}{1.234999in}}%
\pgfpathlineto{\pgfqpoint{2.664449in}{1.225595in}}%
\pgfpathlineto{\pgfqpoint{2.596311in}{1.197344in}}%
\pgfpathlineto{\pgfqpoint{2.528174in}{1.206434in}}%
\pgfpathlineto{\pgfqpoint{2.484547in}{1.234999in}}%
\pgfpathlineto{\pgfqpoint{2.460036in}{1.247206in}}%
\pgfpathlineto{\pgfqpoint{2.391899in}{1.269643in}}%
\pgfpathlineto{\pgfqpoint{2.323761in}{1.288767in}}%
\pgfpathlineto{\pgfqpoint{2.319592in}{1.289938in}}%
\pgfpathlineto{\pgfqpoint{2.255624in}{1.307892in}}%
\pgfpathlineto{\pgfqpoint{2.187486in}{1.327016in}}%
\pgfpathlineto{\pgfqpoint{2.123852in}{1.344876in}}%
\pgfpathlineto{\pgfqpoint{2.119349in}{1.346235in}}%
\pgfpathlineto{\pgfqpoint{2.051211in}{1.367605in}}%
\pgfpathlineto{\pgfqpoint{1.983074in}{1.386557in}}%
\pgfpathlineto{\pgfqpoint{1.917821in}{1.399814in}}%
\pgfpathlineto{\pgfqpoint{1.914936in}{1.400400in}}%
\pgfpathlineto{\pgfqpoint{1.846799in}{1.414243in}}%
\pgfpathlineto{\pgfqpoint{1.778661in}{1.428085in}}%
\pgfpathlineto{\pgfqpoint{1.710524in}{1.441928in}}%
\pgfpathlineto{\pgfqpoint{1.666521in}{1.399814in}}%
\pgfpathlineto{\pgfqpoint{1.642386in}{1.374990in}}%
\pgfpathlineto{\pgfqpoint{1.574249in}{1.357931in}}%
\pgfpathlineto{\pgfqpoint{1.506111in}{1.347090in}}%
\pgfpathlineto{\pgfqpoint{1.492193in}{1.344876in}}%
\pgfpathlineto{\pgfqpoint{1.437974in}{1.336250in}}%
\pgfpathlineto{\pgfqpoint{1.369836in}{1.325409in}}%
\pgfpathlineto{\pgfqpoint{1.301699in}{1.291509in}}%
\pgfpathlineto{\pgfqpoint{1.233561in}{1.301069in}}%
\pgfpathlineto{\pgfqpoint{1.165424in}{1.310629in}}%
\pgfpathlineto{\pgfqpoint{1.097286in}{1.320189in}}%
\pgfpathlineto{\pgfqpoint{1.080739in}{1.344876in}}%
\pgfpathlineto{\pgfqpoint{1.053387in}{1.399814in}}%
\pgfpathlineto{\pgfqpoint{1.034280in}{1.454752in}}%
\pgfpathlineto{\pgfqpoint{1.029149in}{1.472995in}}%
\pgfpathlineto{\pgfqpoint{0.999191in}{1.509690in}}%
\pgfpathlineto{\pgfqpoint{0.961011in}{1.556918in}}%
\pgfpathlineto{\pgfqpoint{0.961011in}{1.509690in}}%
\pgfpathlineto{\pgfqpoint{0.961011in}{1.454752in}}%
\pgfpathlineto{\pgfqpoint{0.961011in}{1.399814in}}%
\pgfpathlineto{\pgfqpoint{1.029149in}{1.344876in}}%
\pgfpathlineto{\pgfqpoint{1.029149in}{1.289938in}}%
\pgfpathlineto{\pgfqpoint{1.029149in}{1.234999in}}%
\pgfpathlineto{\pgfqpoint{1.029149in}{1.180061in}}%
\pgfpathlineto{\pgfqpoint{1.029149in}{1.125123in}}%
\pgfpathlineto{\pgfqpoint{1.029149in}{1.070185in}}%
\pgfpathlineto{\pgfqpoint{1.029149in}{1.015247in}}%
\pgfpathlineto{\pgfqpoint{1.095198in}{0.961992in}}%
\pgfpathlineto{\pgfqpoint{1.097286in}{0.961699in}}%
\pgfpathlineto{\pgfqpoint{1.107195in}{0.960308in}}%
\pgfpathlineto{\pgfqpoint{1.165424in}{0.952138in}}%
\pgfpathlineto{\pgfqpoint{1.233561in}{0.942578in}}%
\pgfpathlineto{\pgfqpoint{1.301699in}{0.933018in}}%
\pgfpathlineto{\pgfqpoint{1.369836in}{0.923458in}}%
\pgfpathlineto{\pgfqpoint{1.428669in}{0.960308in}}%
\pgfpathlineto{\pgfqpoint{1.437974in}{0.966635in}}%
\pgfpathlineto{\pgfqpoint{1.506111in}{0.980260in}}%
\pgfpathlineto{\pgfqpoint{1.574249in}{0.991101in}}%
\pgfpathlineto{\pgfqpoint{1.642386in}{1.000048in}}%
\pgfpathlineto{\pgfqpoint{1.710524in}{0.995925in}}%
\pgfpathlineto{\pgfqpoint{1.778661in}{0.988081in}}%
\pgfpathlineto{\pgfqpoint{1.846799in}{0.980237in}}%
\pgfpathlineto{\pgfqpoint{1.914936in}{0.972393in}}%
\pgfpathlineto{\pgfqpoint{1.983074in}{0.964247in}}%
\pgfpathlineto{\pgfqpoint{2.005953in}{0.960308in}}%
\pgfpathlineto{\pgfqpoint{2.051211in}{0.952591in}}%
\pgfpathlineto{\pgfqpoint{2.119349in}{0.940331in}}%
\pgfpathlineto{\pgfqpoint{2.187486in}{0.927654in}}%
\pgfpathlineto{\pgfqpoint{2.247795in}{0.905370in}}%
\pgfpathclose%
\pgfusepath{fill}%
\end{pgfscope}%
\begin{pgfscope}%
\pgfpathrectangle{\pgfqpoint{0.634105in}{0.521603in}}{\pgfqpoint{3.720000in}{3.020000in}} %
\pgfusepath{clip}%
\pgfsetbuttcap%
\pgfsetroundjoin%
\definecolor{currentfill}{rgb}{0.312255,0.312255,0.434442}%
\pgfsetfillcolor{currentfill}%
\pgfsetlinewidth{0.000000pt}%
\definecolor{currentstroke}{rgb}{0.000000,0.000000,0.000000}%
\pgfsetstrokecolor{currentstroke}%
\pgfsetdash{}{0pt}%
\pgfpathmoveto{\pgfqpoint{2.528174in}{1.206434in}}%
\pgfpathlineto{\pgfqpoint{2.596311in}{1.197344in}}%
\pgfpathlineto{\pgfqpoint{2.664449in}{1.225595in}}%
\pgfpathlineto{\pgfqpoint{2.682214in}{1.234999in}}%
\pgfpathlineto{\pgfqpoint{2.732586in}{1.261664in}}%
\pgfpathlineto{\pgfqpoint{2.785999in}{1.289938in}}%
\pgfpathlineto{\pgfqpoint{2.800724in}{1.297732in}}%
\pgfpathlineto{\pgfqpoint{2.828867in}{1.312630in}}%
\pgfpathlineto{\pgfqpoint{2.868861in}{1.344876in}}%
\pgfpathlineto{\pgfqpoint{2.936999in}{1.399814in}}%
\pgfpathlineto{\pgfqpoint{3.005136in}{1.454752in}}%
\pgfpathlineto{\pgfqpoint{3.073274in}{1.509690in}}%
\pgfpathlineto{\pgfqpoint{3.141411in}{1.564629in}}%
\pgfpathlineto{\pgfqpoint{3.209549in}{1.619567in}}%
\pgfpathlineto{\pgfqpoint{3.277686in}{1.674505in}}%
\pgfpathlineto{\pgfqpoint{3.320244in}{1.708818in}}%
\pgfpathlineto{\pgfqpoint{3.277686in}{1.686784in}}%
\pgfpathlineto{\pgfqpoint{3.255421in}{1.674505in}}%
\pgfpathlineto{\pgfqpoint{3.209549in}{1.650223in}}%
\pgfpathlineto{\pgfqpoint{3.151636in}{1.619567in}}%
\pgfpathlineto{\pgfqpoint{3.141411in}{1.614155in}}%
\pgfpathlineto{\pgfqpoint{3.073274in}{1.579343in}}%
\pgfpathlineto{\pgfqpoint{3.047850in}{1.564629in}}%
\pgfpathlineto{\pgfqpoint{3.005136in}{1.542018in}}%
\pgfpathlineto{\pgfqpoint{2.944065in}{1.509690in}}%
\pgfpathlineto{\pgfqpoint{2.936999in}{1.505950in}}%
\pgfpathlineto{\pgfqpoint{2.868861in}{1.483166in}}%
\pgfpathlineto{\pgfqpoint{2.800724in}{1.494202in}}%
\pgfpathlineto{\pgfqpoint{2.750992in}{1.509690in}}%
\pgfpathlineto{\pgfqpoint{2.732586in}{1.516233in}}%
\pgfpathlineto{\pgfqpoint{2.664449in}{1.540451in}}%
\pgfpathlineto{\pgfqpoint{2.596426in}{1.564629in}}%
\pgfpathlineto{\pgfqpoint{2.596311in}{1.564669in}}%
\pgfpathlineto{\pgfqpoint{2.528174in}{1.588888in}}%
\pgfpathlineto{\pgfqpoint{2.460036in}{1.613106in}}%
\pgfpathlineto{\pgfqpoint{2.441859in}{1.619567in}}%
\pgfpathlineto{\pgfqpoint{2.391899in}{1.637324in}}%
\pgfpathlineto{\pgfqpoint{2.323761in}{1.661543in}}%
\pgfpathlineto{\pgfqpoint{2.287292in}{1.674505in}}%
\pgfpathlineto{\pgfqpoint{2.255624in}{1.685761in}}%
\pgfpathlineto{\pgfqpoint{2.187486in}{1.709979in}}%
\pgfpathlineto{\pgfqpoint{2.132725in}{1.729443in}}%
\pgfpathlineto{\pgfqpoint{2.119349in}{1.734198in}}%
\pgfpathlineto{\pgfqpoint{2.051211in}{1.758416in}}%
\pgfpathlineto{\pgfqpoint{1.983074in}{1.782635in}}%
\pgfpathlineto{\pgfqpoint{1.978159in}{1.784381in}}%
\pgfpathlineto{\pgfqpoint{1.914936in}{1.806853in}}%
\pgfpathlineto{\pgfqpoint{1.846799in}{1.831071in}}%
\pgfpathlineto{\pgfqpoint{1.823592in}{1.839320in}}%
\pgfpathlineto{\pgfqpoint{1.778661in}{1.855290in}}%
\pgfpathlineto{\pgfqpoint{1.710524in}{1.872116in}}%
\pgfpathlineto{\pgfqpoint{1.642386in}{1.885959in}}%
\pgfpathlineto{\pgfqpoint{1.601535in}{1.894258in}}%
\pgfpathlineto{\pgfqpoint{1.574249in}{1.899801in}}%
\pgfpathlineto{\pgfqpoint{1.561047in}{1.894258in}}%
\pgfpathlineto{\pgfqpoint{1.506111in}{1.848486in}}%
\pgfpathlineto{\pgfqpoint{1.502632in}{1.839320in}}%
\pgfpathlineto{\pgfqpoint{1.481783in}{1.784381in}}%
\pgfpathlineto{\pgfqpoint{1.460934in}{1.729443in}}%
\pgfpathlineto{\pgfqpoint{1.437974in}{1.703270in}}%
\pgfpathlineto{\pgfqpoint{1.369836in}{1.692239in}}%
\pgfpathlineto{\pgfqpoint{1.301699in}{1.681399in}}%
\pgfpathlineto{\pgfqpoint{1.280527in}{1.674505in}}%
\pgfpathlineto{\pgfqpoint{1.233561in}{1.659560in}}%
\pgfpathlineto{\pgfqpoint{1.165424in}{1.669120in}}%
\pgfpathlineto{\pgfqpoint{1.127043in}{1.674505in}}%
\pgfpathlineto{\pgfqpoint{1.097286in}{1.679372in}}%
\pgfpathlineto{\pgfqpoint{1.085455in}{1.729443in}}%
\pgfpathlineto{\pgfqpoint{1.076365in}{1.784381in}}%
\pgfpathlineto{\pgfqpoint{1.067275in}{1.839320in}}%
\pgfpathlineto{\pgfqpoint{1.058184in}{1.894258in}}%
\pgfpathlineto{\pgfqpoint{1.049094in}{1.949196in}}%
\pgfpathlineto{\pgfqpoint{1.040004in}{2.004134in}}%
\pgfpathlineto{\pgfqpoint{1.030914in}{2.059072in}}%
\pgfpathlineto{\pgfqpoint{1.029149in}{2.069741in}}%
\pgfpathlineto{\pgfqpoint{0.995383in}{2.114011in}}%
\pgfpathlineto{\pgfqpoint{0.961011in}{2.159517in}}%
\pgfpathlineto{\pgfqpoint{0.904418in}{2.168949in}}%
\pgfpathlineto{\pgfqpoint{0.892874in}{2.170759in}}%
\pgfpathlineto{\pgfqpoint{0.892874in}{2.168949in}}%
\pgfpathlineto{\pgfqpoint{0.892874in}{2.114011in}}%
\pgfpathlineto{\pgfqpoint{0.892874in}{2.059072in}}%
\pgfpathlineto{\pgfqpoint{0.892874in}{2.004134in}}%
\pgfpathlineto{\pgfqpoint{0.892874in}{1.949196in}}%
\pgfpathlineto{\pgfqpoint{0.892874in}{1.894258in}}%
\pgfpathlineto{\pgfqpoint{0.892874in}{1.839320in}}%
\pgfpathlineto{\pgfqpoint{0.961011in}{1.784381in}}%
\pgfpathlineto{\pgfqpoint{0.961011in}{1.729443in}}%
\pgfpathlineto{\pgfqpoint{0.961011in}{1.674505in}}%
\pgfpathlineto{\pgfqpoint{0.961011in}{1.619567in}}%
\pgfpathlineto{\pgfqpoint{0.961011in}{1.564629in}}%
\pgfpathlineto{\pgfqpoint{0.961011in}{1.556918in}}%
\pgfpathlineto{\pgfqpoint{0.999191in}{1.509690in}}%
\pgfpathlineto{\pgfqpoint{1.029149in}{1.472995in}}%
\pgfpathlineto{\pgfqpoint{1.034280in}{1.454752in}}%
\pgfpathlineto{\pgfqpoint{1.053387in}{1.399814in}}%
\pgfpathlineto{\pgfqpoint{1.080739in}{1.344876in}}%
\pgfpathlineto{\pgfqpoint{1.097286in}{1.320189in}}%
\pgfpathlineto{\pgfqpoint{1.165424in}{1.310629in}}%
\pgfpathlineto{\pgfqpoint{1.233561in}{1.301069in}}%
\pgfpathlineto{\pgfqpoint{1.301699in}{1.291509in}}%
\pgfpathlineto{\pgfqpoint{1.369836in}{1.325409in}}%
\pgfpathlineto{\pgfqpoint{1.437974in}{1.336250in}}%
\pgfpathlineto{\pgfqpoint{1.492193in}{1.344876in}}%
\pgfpathlineto{\pgfqpoint{1.506111in}{1.347090in}}%
\pgfpathlineto{\pgfqpoint{1.574249in}{1.357931in}}%
\pgfpathlineto{\pgfqpoint{1.642386in}{1.374990in}}%
\pgfpathlineto{\pgfqpoint{1.666521in}{1.399814in}}%
\pgfpathlineto{\pgfqpoint{1.710524in}{1.441928in}}%
\pgfpathlineto{\pgfqpoint{1.778661in}{1.428085in}}%
\pgfpathlineto{\pgfqpoint{1.846799in}{1.414243in}}%
\pgfpathlineto{\pgfqpoint{1.914936in}{1.400400in}}%
\pgfpathlineto{\pgfqpoint{1.917821in}{1.399814in}}%
\pgfpathlineto{\pgfqpoint{1.983074in}{1.386557in}}%
\pgfpathlineto{\pgfqpoint{2.051211in}{1.367605in}}%
\pgfpathlineto{\pgfqpoint{2.119349in}{1.346235in}}%
\pgfpathlineto{\pgfqpoint{2.123852in}{1.344876in}}%
\pgfpathlineto{\pgfqpoint{2.187486in}{1.327016in}}%
\pgfpathlineto{\pgfqpoint{2.255624in}{1.307892in}}%
\pgfpathlineto{\pgfqpoint{2.319592in}{1.289938in}}%
\pgfpathlineto{\pgfqpoint{2.323761in}{1.288767in}}%
\pgfpathlineto{\pgfqpoint{2.391899in}{1.269643in}}%
\pgfpathlineto{\pgfqpoint{2.460036in}{1.247206in}}%
\pgfpathlineto{\pgfqpoint{2.484547in}{1.234999in}}%
\pgfpathclose%
\pgfusepath{fill}%
\end{pgfscope}%
\begin{pgfscope}%
\pgfpathrectangle{\pgfqpoint{0.634105in}{0.521603in}}{\pgfqpoint{3.720000in}{3.020000in}} %
\pgfusepath{clip}%
\pgfsetbuttcap%
\pgfsetroundjoin%
\definecolor{currentfill}{rgb}{0.439216,0.484130,0.564216}%
\pgfsetfillcolor{currentfill}%
\pgfsetlinewidth{0.000000pt}%
\definecolor{currentstroke}{rgb}{0.000000,0.000000,0.000000}%
\pgfsetstrokecolor{currentstroke}%
\pgfsetdash{}{0pt}%
\pgfpathmoveto{\pgfqpoint{2.800724in}{1.494202in}}%
\pgfpathlineto{\pgfqpoint{2.868861in}{1.483166in}}%
\pgfpathlineto{\pgfqpoint{2.936999in}{1.505950in}}%
\pgfpathlineto{\pgfqpoint{2.944065in}{1.509690in}}%
\pgfpathlineto{\pgfqpoint{3.005136in}{1.542018in}}%
\pgfpathlineto{\pgfqpoint{3.047850in}{1.564629in}}%
\pgfpathlineto{\pgfqpoint{3.073274in}{1.579343in}}%
\pgfpathlineto{\pgfqpoint{3.141411in}{1.614155in}}%
\pgfpathlineto{\pgfqpoint{3.151636in}{1.619567in}}%
\pgfpathlineto{\pgfqpoint{3.209549in}{1.650223in}}%
\pgfpathlineto{\pgfqpoint{3.255421in}{1.674505in}}%
\pgfpathlineto{\pgfqpoint{3.277686in}{1.686784in}}%
\pgfpathlineto{\pgfqpoint{3.320244in}{1.708818in}}%
\pgfpathlineto{\pgfqpoint{3.345824in}{1.729443in}}%
\pgfpathlineto{\pgfqpoint{3.413961in}{1.784381in}}%
\pgfpathlineto{\pgfqpoint{3.482099in}{1.839320in}}%
\pgfpathlineto{\pgfqpoint{3.550236in}{1.894258in}}%
\pgfpathlineto{\pgfqpoint{3.550236in}{1.949196in}}%
\pgfpathlineto{\pgfqpoint{3.618374in}{2.004134in}}%
\pgfpathlineto{\pgfqpoint{3.686511in}{2.059072in}}%
\pgfpathlineto{\pgfqpoint{3.686511in}{2.114011in}}%
\pgfpathlineto{\pgfqpoint{3.754649in}{2.168949in}}%
\pgfpathlineto{\pgfqpoint{3.754649in}{2.201612in}}%
\pgfpathlineto{\pgfqpoint{3.707448in}{2.168949in}}%
\pgfpathlineto{\pgfqpoint{3.686511in}{2.154460in}}%
\pgfpathlineto{\pgfqpoint{3.628058in}{2.114011in}}%
\pgfpathlineto{\pgfqpoint{3.618374in}{2.107309in}}%
\pgfpathlineto{\pgfqpoint{3.550236in}{2.069188in}}%
\pgfpathlineto{\pgfqpoint{3.529745in}{2.059072in}}%
\pgfpathlineto{\pgfqpoint{3.482099in}{2.035551in}}%
\pgfpathlineto{\pgfqpoint{3.418459in}{2.004134in}}%
\pgfpathlineto{\pgfqpoint{3.413961in}{2.001914in}}%
\pgfpathlineto{\pgfqpoint{3.345824in}{1.968276in}}%
\pgfpathlineto{\pgfqpoint{3.307173in}{1.949196in}}%
\pgfpathlineto{\pgfqpoint{3.277686in}{1.934639in}}%
\pgfpathlineto{\pgfqpoint{3.209549in}{1.901002in}}%
\pgfpathlineto{\pgfqpoint{3.195887in}{1.894258in}}%
\pgfpathlineto{\pgfqpoint{3.141411in}{1.867365in}}%
\pgfpathlineto{\pgfqpoint{3.076025in}{1.839320in}}%
\pgfpathlineto{\pgfqpoint{3.073274in}{1.838037in}}%
\pgfpathlineto{\pgfqpoint{3.067192in}{1.839320in}}%
\pgfpathlineto{\pgfqpoint{3.005136in}{1.857538in}}%
\pgfpathlineto{\pgfqpoint{2.936999in}{1.877542in}}%
\pgfpathlineto{\pgfqpoint{2.880062in}{1.894258in}}%
\pgfpathlineto{\pgfqpoint{2.868861in}{1.897546in}}%
\pgfpathlineto{\pgfqpoint{2.800724in}{1.917550in}}%
\pgfpathlineto{\pgfqpoint{2.732586in}{1.937554in}}%
\pgfpathlineto{\pgfqpoint{2.692931in}{1.949196in}}%
\pgfpathlineto{\pgfqpoint{2.664449in}{1.957558in}}%
\pgfpathlineto{\pgfqpoint{2.596311in}{1.977562in}}%
\pgfpathlineto{\pgfqpoint{2.528174in}{1.997566in}}%
\pgfpathlineto{\pgfqpoint{2.505801in}{2.004134in}}%
\pgfpathlineto{\pgfqpoint{2.460036in}{2.017570in}}%
\pgfpathlineto{\pgfqpoint{2.391899in}{2.037574in}}%
\pgfpathlineto{\pgfqpoint{2.323761in}{2.057645in}}%
\pgfpathlineto{\pgfqpoint{2.319278in}{2.059072in}}%
\pgfpathlineto{\pgfqpoint{2.255624in}{2.079292in}}%
\pgfpathlineto{\pgfqpoint{2.187486in}{2.102019in}}%
\pgfpathlineto{\pgfqpoint{2.153747in}{2.114011in}}%
\pgfpathlineto{\pgfqpoint{2.119349in}{2.126237in}}%
\pgfpathlineto{\pgfqpoint{2.051211in}{2.150455in}}%
\pgfpathlineto{\pgfqpoint{1.999180in}{2.168949in}}%
\pgfpathlineto{\pgfqpoint{1.983074in}{2.174674in}}%
\pgfpathlineto{\pgfqpoint{1.914936in}{2.198892in}}%
\pgfpathlineto{\pgfqpoint{1.846799in}{2.223110in}}%
\pgfpathlineto{\pgfqpoint{1.844613in}{2.223887in}}%
\pgfpathlineto{\pgfqpoint{1.778661in}{2.247329in}}%
\pgfpathlineto{\pgfqpoint{1.710524in}{2.271547in}}%
\pgfpathlineto{\pgfqpoint{1.690047in}{2.278825in}}%
\pgfpathlineto{\pgfqpoint{1.642386in}{2.295765in}}%
\pgfpathlineto{\pgfqpoint{1.574249in}{2.319984in}}%
\pgfpathlineto{\pgfqpoint{1.533347in}{2.333763in}}%
\pgfpathlineto{\pgfqpoint{1.506111in}{2.343155in}}%
\pgfpathlineto{\pgfqpoint{1.437974in}{2.357675in}}%
\pgfpathlineto{\pgfqpoint{1.369836in}{2.342991in}}%
\pgfpathlineto{\pgfqpoint{1.366334in}{2.333763in}}%
\pgfpathlineto{\pgfqpoint{1.345485in}{2.278825in}}%
\pgfpathlineto{\pgfqpoint{1.324635in}{2.223887in}}%
\pgfpathlineto{\pgfqpoint{1.303786in}{2.168949in}}%
\pgfpathlineto{\pgfqpoint{1.301699in}{2.163449in}}%
\pgfpathlineto{\pgfqpoint{1.282936in}{2.114011in}}%
\pgfpathlineto{\pgfqpoint{1.255766in}{2.059072in}}%
\pgfpathlineto{\pgfqpoint{1.233561in}{2.037389in}}%
\pgfpathlineto{\pgfqpoint{1.165424in}{2.027611in}}%
\pgfpathlineto{\pgfqpoint{1.151034in}{2.059072in}}%
\pgfpathlineto{\pgfqpoint{1.130978in}{2.114011in}}%
\pgfpathlineto{\pgfqpoint{1.115692in}{2.168949in}}%
\pgfpathlineto{\pgfqpoint{1.104839in}{2.223887in}}%
\pgfpathlineto{\pgfqpoint{1.100966in}{2.278825in}}%
\pgfpathlineto{\pgfqpoint{1.097526in}{2.333763in}}%
\pgfpathlineto{\pgfqpoint{1.097286in}{2.337842in}}%
\pgfpathlineto{\pgfqpoint{1.092891in}{2.388702in}}%
\pgfpathlineto{\pgfqpoint{1.084182in}{2.443640in}}%
\pgfpathlineto{\pgfqpoint{1.062163in}{2.498578in}}%
\pgfpathlineto{\pgfqpoint{1.029149in}{2.526883in}}%
\pgfpathlineto{\pgfqpoint{1.029149in}{2.498578in}}%
\pgfpathlineto{\pgfqpoint{0.961011in}{2.443640in}}%
\pgfpathlineto{\pgfqpoint{0.961011in}{2.388702in}}%
\pgfpathlineto{\pgfqpoint{0.892874in}{2.333763in}}%
\pgfpathlineto{\pgfqpoint{0.892874in}{2.278825in}}%
\pgfpathlineto{\pgfqpoint{0.892874in}{2.223887in}}%
\pgfpathlineto{\pgfqpoint{0.892874in}{2.170759in}}%
\pgfpathlineto{\pgfqpoint{0.904418in}{2.168949in}}%
\pgfpathlineto{\pgfqpoint{0.961011in}{2.159517in}}%
\pgfpathlineto{\pgfqpoint{0.995383in}{2.114011in}}%
\pgfpathlineto{\pgfqpoint{1.029149in}{2.069741in}}%
\pgfpathlineto{\pgfqpoint{1.030914in}{2.059072in}}%
\pgfpathlineto{\pgfqpoint{1.040004in}{2.004134in}}%
\pgfpathlineto{\pgfqpoint{1.049094in}{1.949196in}}%
\pgfpathlineto{\pgfqpoint{1.058184in}{1.894258in}}%
\pgfpathlineto{\pgfqpoint{1.067275in}{1.839320in}}%
\pgfpathlineto{\pgfqpoint{1.076365in}{1.784381in}}%
\pgfpathlineto{\pgfqpoint{1.085455in}{1.729443in}}%
\pgfpathlineto{\pgfqpoint{1.097286in}{1.679372in}}%
\pgfpathlineto{\pgfqpoint{1.127043in}{1.674505in}}%
\pgfpathlineto{\pgfqpoint{1.165424in}{1.669120in}}%
\pgfpathlineto{\pgfqpoint{1.233561in}{1.659560in}}%
\pgfpathlineto{\pgfqpoint{1.280527in}{1.674505in}}%
\pgfpathlineto{\pgfqpoint{1.301699in}{1.681399in}}%
\pgfpathlineto{\pgfqpoint{1.369836in}{1.692239in}}%
\pgfpathlineto{\pgfqpoint{1.437974in}{1.703270in}}%
\pgfpathlineto{\pgfqpoint{1.460934in}{1.729443in}}%
\pgfpathlineto{\pgfqpoint{1.481783in}{1.784381in}}%
\pgfpathlineto{\pgfqpoint{1.502632in}{1.839320in}}%
\pgfpathlineto{\pgfqpoint{1.506111in}{1.848486in}}%
\pgfpathlineto{\pgfqpoint{1.561047in}{1.894258in}}%
\pgfpathlineto{\pgfqpoint{1.574249in}{1.899801in}}%
\pgfpathlineto{\pgfqpoint{1.601535in}{1.894258in}}%
\pgfpathlineto{\pgfqpoint{1.642386in}{1.885959in}}%
\pgfpathlineto{\pgfqpoint{1.710524in}{1.872116in}}%
\pgfpathlineto{\pgfqpoint{1.778661in}{1.855290in}}%
\pgfpathlineto{\pgfqpoint{1.823592in}{1.839320in}}%
\pgfpathlineto{\pgfqpoint{1.846799in}{1.831071in}}%
\pgfpathlineto{\pgfqpoint{1.914936in}{1.806853in}}%
\pgfpathlineto{\pgfqpoint{1.978159in}{1.784381in}}%
\pgfpathlineto{\pgfqpoint{1.983074in}{1.782635in}}%
\pgfpathlineto{\pgfqpoint{2.051211in}{1.758416in}}%
\pgfpathlineto{\pgfqpoint{2.119349in}{1.734198in}}%
\pgfpathlineto{\pgfqpoint{2.132725in}{1.729443in}}%
\pgfpathlineto{\pgfqpoint{2.187486in}{1.709979in}}%
\pgfpathlineto{\pgfqpoint{2.255624in}{1.685761in}}%
\pgfpathlineto{\pgfqpoint{2.287292in}{1.674505in}}%
\pgfpathlineto{\pgfqpoint{2.323761in}{1.661543in}}%
\pgfpathlineto{\pgfqpoint{2.391899in}{1.637324in}}%
\pgfpathlineto{\pgfqpoint{2.441859in}{1.619567in}}%
\pgfpathlineto{\pgfqpoint{2.460036in}{1.613106in}}%
\pgfpathlineto{\pgfqpoint{2.528174in}{1.588888in}}%
\pgfpathlineto{\pgfqpoint{2.596311in}{1.564669in}}%
\pgfpathlineto{\pgfqpoint{2.596426in}{1.564629in}}%
\pgfpathlineto{\pgfqpoint{2.664449in}{1.540451in}}%
\pgfpathlineto{\pgfqpoint{2.732586in}{1.516233in}}%
\pgfpathlineto{\pgfqpoint{2.750992in}{1.509690in}}%
\pgfpathclose%
\pgfusepath{fill}%
\end{pgfscope}%
\begin{pgfscope}%
\pgfpathrectangle{\pgfqpoint{0.634105in}{0.521603in}}{\pgfqpoint{3.720000in}{3.020000in}} %
\pgfusepath{clip}%
\pgfsetbuttcap%
\pgfsetroundjoin%
\definecolor{currentfill}{rgb}{0.562745,0.653983,0.687745}%
\pgfsetfillcolor{currentfill}%
\pgfsetlinewidth{0.000000pt}%
\definecolor{currentstroke}{rgb}{0.000000,0.000000,0.000000}%
\pgfsetstrokecolor{currentstroke}%
\pgfsetdash{}{0pt}%
\pgfpathmoveto{\pgfqpoint{3.073274in}{1.838037in}}%
\pgfpathlineto{\pgfqpoint{3.076025in}{1.839320in}}%
\pgfpathlineto{\pgfqpoint{3.141411in}{1.867365in}}%
\pgfpathlineto{\pgfqpoint{3.195887in}{1.894258in}}%
\pgfpathlineto{\pgfqpoint{3.209549in}{1.901002in}}%
\pgfpathlineto{\pgfqpoint{3.277686in}{1.934639in}}%
\pgfpathlineto{\pgfqpoint{3.307173in}{1.949196in}}%
\pgfpathlineto{\pgfqpoint{3.345824in}{1.968276in}}%
\pgfpathlineto{\pgfqpoint{3.413961in}{2.001914in}}%
\pgfpathlineto{\pgfqpoint{3.418459in}{2.004134in}}%
\pgfpathlineto{\pgfqpoint{3.482099in}{2.035551in}}%
\pgfpathlineto{\pgfqpoint{3.529745in}{2.059072in}}%
\pgfpathlineto{\pgfqpoint{3.550236in}{2.069188in}}%
\pgfpathlineto{\pgfqpoint{3.618374in}{2.107309in}}%
\pgfpathlineto{\pgfqpoint{3.628058in}{2.114011in}}%
\pgfpathlineto{\pgfqpoint{3.686511in}{2.154460in}}%
\pgfpathlineto{\pgfqpoint{3.707448in}{2.168949in}}%
\pgfpathlineto{\pgfqpoint{3.754649in}{2.201612in}}%
\pgfpathlineto{\pgfqpoint{3.754649in}{2.223887in}}%
\pgfpathlineto{\pgfqpoint{3.822786in}{2.278825in}}%
\pgfpathlineto{\pgfqpoint{3.890924in}{2.333763in}}%
\pgfpathlineto{\pgfqpoint{3.890924in}{2.388702in}}%
\pgfpathlineto{\pgfqpoint{3.959061in}{2.443640in}}%
\pgfpathlineto{\pgfqpoint{4.027199in}{2.498578in}}%
\pgfpathlineto{\pgfqpoint{4.027199in}{2.553516in}}%
\pgfpathlineto{\pgfqpoint{4.095336in}{2.608454in}}%
\pgfpathlineto{\pgfqpoint{4.095336in}{2.663393in}}%
\pgfpathlineto{\pgfqpoint{4.095336in}{2.718331in}}%
\pgfpathlineto{\pgfqpoint{4.095336in}{2.718774in}}%
\pgfpathlineto{\pgfqpoint{4.094696in}{2.718331in}}%
\pgfpathlineto{\pgfqpoint{4.027199in}{2.671622in}}%
\pgfpathlineto{\pgfqpoint{4.015306in}{2.663393in}}%
\pgfpathlineto{\pgfqpoint{3.959061in}{2.624471in}}%
\pgfpathlineto{\pgfqpoint{3.935916in}{2.608454in}}%
\pgfpathlineto{\pgfqpoint{3.890924in}{2.577320in}}%
\pgfpathlineto{\pgfqpoint{3.856525in}{2.553516in}}%
\pgfpathlineto{\pgfqpoint{3.822786in}{2.530169in}}%
\pgfpathlineto{\pgfqpoint{3.777135in}{2.498578in}}%
\pgfpathlineto{\pgfqpoint{3.754649in}{2.483017in}}%
\pgfpathlineto{\pgfqpoint{3.697745in}{2.443640in}}%
\pgfpathlineto{\pgfqpoint{3.686511in}{2.435866in}}%
\pgfpathlineto{\pgfqpoint{3.618374in}{2.396333in}}%
\pgfpathlineto{\pgfqpoint{3.602916in}{2.388702in}}%
\pgfpathlineto{\pgfqpoint{3.550236in}{2.362695in}}%
\pgfpathlineto{\pgfqpoint{3.491630in}{2.333763in}}%
\pgfpathlineto{\pgfqpoint{3.482099in}{2.329058in}}%
\pgfpathlineto{\pgfqpoint{3.413961in}{2.295421in}}%
\pgfpathlineto{\pgfqpoint{3.380344in}{2.278825in}}%
\pgfpathlineto{\pgfqpoint{3.345824in}{2.261784in}}%
\pgfpathlineto{\pgfqpoint{3.277686in}{2.229450in}}%
\pgfpathlineto{\pgfqpoint{3.209549in}{2.241630in}}%
\pgfpathlineto{\pgfqpoint{3.141411in}{2.261634in}}%
\pgfpathlineto{\pgfqpoint{3.082854in}{2.278825in}}%
\pgfpathlineto{\pgfqpoint{3.073274in}{2.281638in}}%
\pgfpathlineto{\pgfqpoint{3.005136in}{2.301642in}}%
\pgfpathlineto{\pgfqpoint{2.936999in}{2.321646in}}%
\pgfpathlineto{\pgfqpoint{2.895724in}{2.333763in}}%
\pgfpathlineto{\pgfqpoint{2.868861in}{2.341650in}}%
\pgfpathlineto{\pgfqpoint{2.800724in}{2.361654in}}%
\pgfpathlineto{\pgfqpoint{2.732586in}{2.381658in}}%
\pgfpathlineto{\pgfqpoint{2.708594in}{2.388702in}}%
\pgfpathlineto{\pgfqpoint{2.664449in}{2.401662in}}%
\pgfpathlineto{\pgfqpoint{2.596311in}{2.421666in}}%
\pgfpathlineto{\pgfqpoint{2.528174in}{2.441670in}}%
\pgfpathlineto{\pgfqpoint{2.521464in}{2.443640in}}%
\pgfpathlineto{\pgfqpoint{2.460036in}{2.461674in}}%
\pgfpathlineto{\pgfqpoint{2.391899in}{2.481678in}}%
\pgfpathlineto{\pgfqpoint{2.334334in}{2.498578in}}%
\pgfpathlineto{\pgfqpoint{2.323761in}{2.501682in}}%
\pgfpathlineto{\pgfqpoint{2.255624in}{2.521686in}}%
\pgfpathlineto{\pgfqpoint{2.187486in}{2.541690in}}%
\pgfpathlineto{\pgfqpoint{2.147204in}{2.553516in}}%
\pgfpathlineto{\pgfqpoint{2.119349in}{2.561694in}}%
\pgfpathlineto{\pgfqpoint{2.051211in}{2.581698in}}%
\pgfpathlineto{\pgfqpoint{1.983074in}{2.601702in}}%
\pgfpathlineto{\pgfqpoint{1.960073in}{2.608454in}}%
\pgfpathlineto{\pgfqpoint{1.914936in}{2.621706in}}%
\pgfpathlineto{\pgfqpoint{1.846799in}{2.641710in}}%
\pgfpathlineto{\pgfqpoint{1.778661in}{2.661714in}}%
\pgfpathlineto{\pgfqpoint{1.772943in}{2.663393in}}%
\pgfpathlineto{\pgfqpoint{1.710524in}{2.681718in}}%
\pgfpathlineto{\pgfqpoint{1.642386in}{2.701722in}}%
\pgfpathlineto{\pgfqpoint{1.585813in}{2.718331in}}%
\pgfpathlineto{\pgfqpoint{1.574249in}{2.721726in}}%
\pgfpathlineto{\pgfqpoint{1.506111in}{2.741730in}}%
\pgfpathlineto{\pgfqpoint{1.437974in}{2.762575in}}%
\pgfpathlineto{\pgfqpoint{1.405840in}{2.773269in}}%
\pgfpathlineto{\pgfqpoint{1.369836in}{2.785224in}}%
\pgfpathlineto{\pgfqpoint{1.301699in}{2.808896in}}%
\pgfpathlineto{\pgfqpoint{1.237165in}{2.828207in}}%
\pgfpathlineto{\pgfqpoint{1.233561in}{2.829366in}}%
\pgfpathlineto{\pgfqpoint{1.165424in}{2.853515in}}%
\pgfpathlineto{\pgfqpoint{1.165424in}{2.828207in}}%
\pgfpathlineto{\pgfqpoint{1.165424in}{2.773269in}}%
\pgfpathlineto{\pgfqpoint{1.097286in}{2.718331in}}%
\pgfpathlineto{\pgfqpoint{1.097286in}{2.663393in}}%
\pgfpathlineto{\pgfqpoint{1.029149in}{2.608454in}}%
\pgfpathlineto{\pgfqpoint{1.029149in}{2.553516in}}%
\pgfpathlineto{\pgfqpoint{1.029149in}{2.526883in}}%
\pgfpathlineto{\pgfqpoint{1.062163in}{2.498578in}}%
\pgfpathlineto{\pgfqpoint{1.084182in}{2.443640in}}%
\pgfpathlineto{\pgfqpoint{1.092891in}{2.388702in}}%
\pgfpathlineto{\pgfqpoint{1.097286in}{2.337842in}}%
\pgfpathlineto{\pgfqpoint{1.097526in}{2.333763in}}%
\pgfpathlineto{\pgfqpoint{1.100966in}{2.278825in}}%
\pgfpathlineto{\pgfqpoint{1.104839in}{2.223887in}}%
\pgfpathlineto{\pgfqpoint{1.115692in}{2.168949in}}%
\pgfpathlineto{\pgfqpoint{1.130978in}{2.114011in}}%
\pgfpathlineto{\pgfqpoint{1.151034in}{2.059072in}}%
\pgfpathlineto{\pgfqpoint{1.165424in}{2.027611in}}%
\pgfpathlineto{\pgfqpoint{1.233561in}{2.037389in}}%
\pgfpathlineto{\pgfqpoint{1.255766in}{2.059072in}}%
\pgfpathlineto{\pgfqpoint{1.282936in}{2.114011in}}%
\pgfpathlineto{\pgfqpoint{1.301699in}{2.163449in}}%
\pgfpathlineto{\pgfqpoint{1.303786in}{2.168949in}}%
\pgfpathlineto{\pgfqpoint{1.324635in}{2.223887in}}%
\pgfpathlineto{\pgfqpoint{1.345485in}{2.278825in}}%
\pgfpathlineto{\pgfqpoint{1.366334in}{2.333763in}}%
\pgfpathlineto{\pgfqpoint{1.369836in}{2.342991in}}%
\pgfpathlineto{\pgfqpoint{1.437974in}{2.357675in}}%
\pgfpathlineto{\pgfqpoint{1.506111in}{2.343155in}}%
\pgfpathlineto{\pgfqpoint{1.533347in}{2.333763in}}%
\pgfpathlineto{\pgfqpoint{1.574249in}{2.319984in}}%
\pgfpathlineto{\pgfqpoint{1.642386in}{2.295765in}}%
\pgfpathlineto{\pgfqpoint{1.690047in}{2.278825in}}%
\pgfpathlineto{\pgfqpoint{1.710524in}{2.271547in}}%
\pgfpathlineto{\pgfqpoint{1.778661in}{2.247329in}}%
\pgfpathlineto{\pgfqpoint{1.844613in}{2.223887in}}%
\pgfpathlineto{\pgfqpoint{1.846799in}{2.223110in}}%
\pgfpathlineto{\pgfqpoint{1.914936in}{2.198892in}}%
\pgfpathlineto{\pgfqpoint{1.983074in}{2.174674in}}%
\pgfpathlineto{\pgfqpoint{1.999180in}{2.168949in}}%
\pgfpathlineto{\pgfqpoint{2.051211in}{2.150455in}}%
\pgfpathlineto{\pgfqpoint{2.119349in}{2.126237in}}%
\pgfpathlineto{\pgfqpoint{2.153747in}{2.114011in}}%
\pgfpathlineto{\pgfqpoint{2.187486in}{2.102019in}}%
\pgfpathlineto{\pgfqpoint{2.255624in}{2.079292in}}%
\pgfpathlineto{\pgfqpoint{2.319278in}{2.059072in}}%
\pgfpathlineto{\pgfqpoint{2.323761in}{2.057645in}}%
\pgfpathlineto{\pgfqpoint{2.391899in}{2.037574in}}%
\pgfpathlineto{\pgfqpoint{2.460036in}{2.017570in}}%
\pgfpathlineto{\pgfqpoint{2.505801in}{2.004134in}}%
\pgfpathlineto{\pgfqpoint{2.528174in}{1.997566in}}%
\pgfpathlineto{\pgfqpoint{2.596311in}{1.977562in}}%
\pgfpathlineto{\pgfqpoint{2.664449in}{1.957558in}}%
\pgfpathlineto{\pgfqpoint{2.692931in}{1.949196in}}%
\pgfpathlineto{\pgfqpoint{2.732586in}{1.937554in}}%
\pgfpathlineto{\pgfqpoint{2.800724in}{1.917550in}}%
\pgfpathlineto{\pgfqpoint{2.868861in}{1.897546in}}%
\pgfpathlineto{\pgfqpoint{2.880062in}{1.894258in}}%
\pgfpathlineto{\pgfqpoint{2.936999in}{1.877542in}}%
\pgfpathlineto{\pgfqpoint{3.005136in}{1.857538in}}%
\pgfpathlineto{\pgfqpoint{3.067192in}{1.839320in}}%
\pgfpathclose%
\pgfusepath{fill}%
\end{pgfscope}%
\begin{pgfscope}%
\pgfpathrectangle{\pgfqpoint{0.634105in}{0.521603in}}{\pgfqpoint{3.720000in}{3.020000in}} %
\pgfusepath{clip}%
\pgfsetbuttcap%
\pgfsetroundjoin%
\definecolor{currentfill}{rgb}{0.710478,0.814706,0.814706}%
\pgfsetfillcolor{currentfill}%
\pgfsetlinewidth{0.000000pt}%
\definecolor{currentstroke}{rgb}{0.000000,0.000000,0.000000}%
\pgfsetstrokecolor{currentstroke}%
\pgfsetdash{}{0pt}%
\pgfpathmoveto{\pgfqpoint{3.141411in}{2.261634in}}%
\pgfpathlineto{\pgfqpoint{3.209549in}{2.241630in}}%
\pgfpathlineto{\pgfqpoint{3.277686in}{2.229450in}}%
\pgfpathlineto{\pgfqpoint{3.345824in}{2.261784in}}%
\pgfpathlineto{\pgfqpoint{3.380344in}{2.278825in}}%
\pgfpathlineto{\pgfqpoint{3.413961in}{2.295421in}}%
\pgfpathlineto{\pgfqpoint{3.482099in}{2.329058in}}%
\pgfpathlineto{\pgfqpoint{3.491630in}{2.333763in}}%
\pgfpathlineto{\pgfqpoint{3.550236in}{2.362695in}}%
\pgfpathlineto{\pgfqpoint{3.602916in}{2.388702in}}%
\pgfpathlineto{\pgfqpoint{3.618374in}{2.396333in}}%
\pgfpathlineto{\pgfqpoint{3.686511in}{2.435866in}}%
\pgfpathlineto{\pgfqpoint{3.697745in}{2.443640in}}%
\pgfpathlineto{\pgfqpoint{3.754649in}{2.483017in}}%
\pgfpathlineto{\pgfqpoint{3.777135in}{2.498578in}}%
\pgfpathlineto{\pgfqpoint{3.822786in}{2.530169in}}%
\pgfpathlineto{\pgfqpoint{3.856525in}{2.553516in}}%
\pgfpathlineto{\pgfqpoint{3.890924in}{2.577320in}}%
\pgfpathlineto{\pgfqpoint{3.935916in}{2.608454in}}%
\pgfpathlineto{\pgfqpoint{3.959061in}{2.624471in}}%
\pgfpathlineto{\pgfqpoint{4.015306in}{2.663393in}}%
\pgfpathlineto{\pgfqpoint{4.027199in}{2.671622in}}%
\pgfpathlineto{\pgfqpoint{4.094696in}{2.718331in}}%
\pgfpathlineto{\pgfqpoint{4.095336in}{2.718774in}}%
\pgfpathlineto{\pgfqpoint{4.095336in}{2.773269in}}%
\pgfpathlineto{\pgfqpoint{4.027199in}{2.828207in}}%
\pgfpathlineto{\pgfqpoint{4.027199in}{2.883145in}}%
\pgfpathlineto{\pgfqpoint{4.027199in}{2.938084in}}%
\pgfpathlineto{\pgfqpoint{4.017224in}{2.946126in}}%
\pgfpathlineto{\pgfqpoint{4.005603in}{2.938084in}}%
\pgfpathlineto{\pgfqpoint{3.959061in}{2.905877in}}%
\pgfpathlineto{\pgfqpoint{3.926213in}{2.883145in}}%
\pgfpathlineto{\pgfqpoint{3.890924in}{2.858725in}}%
\pgfpathlineto{\pgfqpoint{3.846822in}{2.828207in}}%
\pgfpathlineto{\pgfqpoint{3.822786in}{2.811574in}}%
\pgfpathlineto{\pgfqpoint{3.767432in}{2.773269in}}%
\pgfpathlineto{\pgfqpoint{3.754649in}{2.764423in}}%
\pgfpathlineto{\pgfqpoint{3.686511in}{2.723477in}}%
\pgfpathlineto{\pgfqpoint{3.676087in}{2.718331in}}%
\pgfpathlineto{\pgfqpoint{3.618374in}{2.689840in}}%
\pgfpathlineto{\pgfqpoint{3.564801in}{2.663393in}}%
\pgfpathlineto{\pgfqpoint{3.550236in}{2.656202in}}%
\pgfpathlineto{\pgfqpoint{3.482099in}{2.623297in}}%
\pgfpathlineto{\pgfqpoint{3.413961in}{2.625722in}}%
\pgfpathlineto{\pgfqpoint{3.345824in}{2.645726in}}%
\pgfpathlineto{\pgfqpoint{3.285647in}{2.663393in}}%
\pgfpathlineto{\pgfqpoint{3.277686in}{2.665730in}}%
\pgfpathlineto{\pgfqpoint{3.209549in}{2.685734in}}%
\pgfpathlineto{\pgfqpoint{3.141411in}{2.705738in}}%
\pgfpathlineto{\pgfqpoint{3.098517in}{2.718331in}}%
\pgfpathlineto{\pgfqpoint{3.073274in}{2.725742in}}%
\pgfpathlineto{\pgfqpoint{3.005136in}{2.745746in}}%
\pgfpathlineto{\pgfqpoint{2.936999in}{2.765750in}}%
\pgfpathlineto{\pgfqpoint{2.911387in}{2.773269in}}%
\pgfpathlineto{\pgfqpoint{2.868861in}{2.785754in}}%
\pgfpathlineto{\pgfqpoint{2.800724in}{2.805758in}}%
\pgfpathlineto{\pgfqpoint{2.732586in}{2.825762in}}%
\pgfpathlineto{\pgfqpoint{2.724257in}{2.828207in}}%
\pgfpathlineto{\pgfqpoint{2.664449in}{2.845766in}}%
\pgfpathlineto{\pgfqpoint{2.596311in}{2.865770in}}%
\pgfpathlineto{\pgfqpoint{2.537126in}{2.883145in}}%
\pgfpathlineto{\pgfqpoint{2.528174in}{2.885774in}}%
\pgfpathlineto{\pgfqpoint{2.460036in}{2.905778in}}%
\pgfpathlineto{\pgfqpoint{2.391899in}{2.925782in}}%
\pgfpathlineto{\pgfqpoint{2.349996in}{2.938084in}}%
\pgfpathlineto{\pgfqpoint{2.323761in}{2.945786in}}%
\pgfpathlineto{\pgfqpoint{2.255624in}{2.965790in}}%
\pgfpathlineto{\pgfqpoint{2.187486in}{2.985794in}}%
\pgfpathlineto{\pgfqpoint{2.162866in}{2.993022in}}%
\pgfpathlineto{\pgfqpoint{2.119349in}{3.005798in}}%
\pgfpathlineto{\pgfqpoint{2.051211in}{3.025802in}}%
\pgfpathlineto{\pgfqpoint{1.983074in}{3.045806in}}%
\pgfpathlineto{\pgfqpoint{1.975736in}{3.047960in}}%
\pgfpathlineto{\pgfqpoint{1.914936in}{3.047960in}}%
\pgfpathlineto{\pgfqpoint{1.846799in}{3.047960in}}%
\pgfpathlineto{\pgfqpoint{1.778661in}{3.047960in}}%
\pgfpathlineto{\pgfqpoint{1.710524in}{2.993022in}}%
\pgfpathlineto{\pgfqpoint{1.642386in}{2.993022in}}%
\pgfpathlineto{\pgfqpoint{1.574249in}{2.993022in}}%
\pgfpathlineto{\pgfqpoint{1.506111in}{2.993022in}}%
\pgfpathlineto{\pgfqpoint{1.437974in}{2.993022in}}%
\pgfpathlineto{\pgfqpoint{1.369836in}{2.938084in}}%
\pgfpathlineto{\pgfqpoint{1.301699in}{2.938084in}}%
\pgfpathlineto{\pgfqpoint{1.233561in}{2.938084in}}%
\pgfpathlineto{\pgfqpoint{1.165424in}{2.883145in}}%
\pgfpathlineto{\pgfqpoint{1.165424in}{2.853515in}}%
\pgfpathlineto{\pgfqpoint{1.233561in}{2.829366in}}%
\pgfpathlineto{\pgfqpoint{1.237165in}{2.828207in}}%
\pgfpathlineto{\pgfqpoint{1.301699in}{2.808896in}}%
\pgfpathlineto{\pgfqpoint{1.369836in}{2.785224in}}%
\pgfpathlineto{\pgfqpoint{1.405840in}{2.773269in}}%
\pgfpathlineto{\pgfqpoint{1.437974in}{2.762575in}}%
\pgfpathlineto{\pgfqpoint{1.506111in}{2.741730in}}%
\pgfpathlineto{\pgfqpoint{1.574249in}{2.721726in}}%
\pgfpathlineto{\pgfqpoint{1.585813in}{2.718331in}}%
\pgfpathlineto{\pgfqpoint{1.642386in}{2.701722in}}%
\pgfpathlineto{\pgfqpoint{1.710524in}{2.681718in}}%
\pgfpathlineto{\pgfqpoint{1.772943in}{2.663393in}}%
\pgfpathlineto{\pgfqpoint{1.778661in}{2.661714in}}%
\pgfpathlineto{\pgfqpoint{1.846799in}{2.641710in}}%
\pgfpathlineto{\pgfqpoint{1.914936in}{2.621706in}}%
\pgfpathlineto{\pgfqpoint{1.960073in}{2.608454in}}%
\pgfpathlineto{\pgfqpoint{1.983074in}{2.601702in}}%
\pgfpathlineto{\pgfqpoint{2.051211in}{2.581698in}}%
\pgfpathlineto{\pgfqpoint{2.119349in}{2.561694in}}%
\pgfpathlineto{\pgfqpoint{2.147204in}{2.553516in}}%
\pgfpathlineto{\pgfqpoint{2.187486in}{2.541690in}}%
\pgfpathlineto{\pgfqpoint{2.255624in}{2.521686in}}%
\pgfpathlineto{\pgfqpoint{2.323761in}{2.501682in}}%
\pgfpathlineto{\pgfqpoint{2.334334in}{2.498578in}}%
\pgfpathlineto{\pgfqpoint{2.391899in}{2.481678in}}%
\pgfpathlineto{\pgfqpoint{2.460036in}{2.461674in}}%
\pgfpathlineto{\pgfqpoint{2.521464in}{2.443640in}}%
\pgfpathlineto{\pgfqpoint{2.528174in}{2.441670in}}%
\pgfpathlineto{\pgfqpoint{2.596311in}{2.421666in}}%
\pgfpathlineto{\pgfqpoint{2.664449in}{2.401662in}}%
\pgfpathlineto{\pgfqpoint{2.708594in}{2.388702in}}%
\pgfpathlineto{\pgfqpoint{2.732586in}{2.381658in}}%
\pgfpathlineto{\pgfqpoint{2.800724in}{2.361654in}}%
\pgfpathlineto{\pgfqpoint{2.868861in}{2.341650in}}%
\pgfpathlineto{\pgfqpoint{2.895724in}{2.333763in}}%
\pgfpathlineto{\pgfqpoint{2.936999in}{2.321646in}}%
\pgfpathlineto{\pgfqpoint{3.005136in}{2.301642in}}%
\pgfpathlineto{\pgfqpoint{3.073274in}{2.281638in}}%
\pgfpathlineto{\pgfqpoint{3.082854in}{2.278825in}}%
\pgfpathclose%
\pgfusepath{fill}%
\end{pgfscope}%
\begin{pgfscope}%
\pgfpathrectangle{\pgfqpoint{0.634105in}{0.521603in}}{\pgfqpoint{3.720000in}{3.020000in}} %
\pgfusepath{clip}%
\pgfsetbuttcap%
\pgfsetroundjoin%
\definecolor{currentfill}{rgb}{0.903493,0.938235,0.938235}%
\pgfsetfillcolor{currentfill}%
\pgfsetlinewidth{0.000000pt}%
\definecolor{currentstroke}{rgb}{0.000000,0.000000,0.000000}%
\pgfsetstrokecolor{currentstroke}%
\pgfsetdash{}{0pt}%
\pgfpathmoveto{\pgfqpoint{3.345824in}{2.645726in}}%
\pgfpathlineto{\pgfqpoint{3.413961in}{2.625722in}}%
\pgfpathlineto{\pgfqpoint{3.482099in}{2.623297in}}%
\pgfpathlineto{\pgfqpoint{3.550236in}{2.656202in}}%
\pgfpathlineto{\pgfqpoint{3.564801in}{2.663393in}}%
\pgfpathlineto{\pgfqpoint{3.618374in}{2.689840in}}%
\pgfpathlineto{\pgfqpoint{3.676087in}{2.718331in}}%
\pgfpathlineto{\pgfqpoint{3.686511in}{2.723477in}}%
\pgfpathlineto{\pgfqpoint{3.754649in}{2.764423in}}%
\pgfpathlineto{\pgfqpoint{3.767432in}{2.773269in}}%
\pgfpathlineto{\pgfqpoint{3.822786in}{2.811574in}}%
\pgfpathlineto{\pgfqpoint{3.846822in}{2.828207in}}%
\pgfpathlineto{\pgfqpoint{3.890924in}{2.858725in}}%
\pgfpathlineto{\pgfqpoint{3.926213in}{2.883145in}}%
\pgfpathlineto{\pgfqpoint{3.959061in}{2.905877in}}%
\pgfpathlineto{\pgfqpoint{4.005603in}{2.938084in}}%
\pgfpathlineto{\pgfqpoint{4.017224in}{2.946126in}}%
\pgfpathlineto{\pgfqpoint{3.959061in}{2.993022in}}%
\pgfpathlineto{\pgfqpoint{3.959061in}{3.047960in}}%
\pgfpathlineto{\pgfqpoint{3.959061in}{3.102898in}}%
\pgfpathlineto{\pgfqpoint{3.902741in}{3.148308in}}%
\pgfpathlineto{\pgfqpoint{3.890924in}{3.140131in}}%
\pgfpathlineto{\pgfqpoint{3.837119in}{3.102898in}}%
\pgfpathlineto{\pgfqpoint{3.822786in}{3.092980in}}%
\pgfpathlineto{\pgfqpoint{3.754649in}{3.050621in}}%
\pgfpathlineto{\pgfqpoint{3.749258in}{3.047960in}}%
\pgfpathlineto{\pgfqpoint{3.686511in}{3.016984in}}%
\pgfpathlineto{\pgfqpoint{3.618374in}{3.009814in}}%
\pgfpathlineto{\pgfqpoint{3.550236in}{3.029818in}}%
\pgfpathlineto{\pgfqpoint{3.488440in}{3.047960in}}%
\pgfpathlineto{\pgfqpoint{3.482099in}{3.049822in}}%
\pgfpathlineto{\pgfqpoint{3.413961in}{3.069826in}}%
\pgfpathlineto{\pgfqpoint{3.345824in}{3.089830in}}%
\pgfpathlineto{\pgfqpoint{3.301309in}{3.102898in}}%
\pgfpathlineto{\pgfqpoint{3.277686in}{3.109834in}}%
\pgfpathlineto{\pgfqpoint{3.209549in}{3.129838in}}%
\pgfpathlineto{\pgfqpoint{3.141411in}{3.149842in}}%
\pgfpathlineto{\pgfqpoint{3.114179in}{3.157836in}}%
\pgfpathlineto{\pgfqpoint{3.073274in}{3.169846in}}%
\pgfpathlineto{\pgfqpoint{3.005136in}{3.189850in}}%
\pgfpathlineto{\pgfqpoint{2.936999in}{3.209854in}}%
\pgfpathlineto{\pgfqpoint{2.927049in}{3.212775in}}%
\pgfpathlineto{\pgfqpoint{2.868861in}{3.212775in}}%
\pgfpathlineto{\pgfqpoint{2.800724in}{3.157836in}}%
\pgfpathlineto{\pgfqpoint{2.732586in}{3.157836in}}%
\pgfpathlineto{\pgfqpoint{2.664449in}{3.157836in}}%
\pgfpathlineto{\pgfqpoint{2.596311in}{3.157836in}}%
\pgfpathlineto{\pgfqpoint{2.528174in}{3.157836in}}%
\pgfpathlineto{\pgfqpoint{2.460036in}{3.102898in}}%
\pgfpathlineto{\pgfqpoint{2.391899in}{3.102898in}}%
\pgfpathlineto{\pgfqpoint{2.323761in}{3.102898in}}%
\pgfpathlineto{\pgfqpoint{2.255624in}{3.102898in}}%
\pgfpathlineto{\pgfqpoint{2.187486in}{3.102898in}}%
\pgfpathlineto{\pgfqpoint{2.119349in}{3.047960in}}%
\pgfpathlineto{\pgfqpoint{2.051211in}{3.047960in}}%
\pgfpathlineto{\pgfqpoint{1.983074in}{3.047960in}}%
\pgfpathlineto{\pgfqpoint{1.975736in}{3.047960in}}%
\pgfpathlineto{\pgfqpoint{1.983074in}{3.045806in}}%
\pgfpathlineto{\pgfqpoint{2.051211in}{3.025802in}}%
\pgfpathlineto{\pgfqpoint{2.119349in}{3.005798in}}%
\pgfpathlineto{\pgfqpoint{2.162866in}{2.993022in}}%
\pgfpathlineto{\pgfqpoint{2.187486in}{2.985794in}}%
\pgfpathlineto{\pgfqpoint{2.255624in}{2.965790in}}%
\pgfpathlineto{\pgfqpoint{2.323761in}{2.945786in}}%
\pgfpathlineto{\pgfqpoint{2.349996in}{2.938084in}}%
\pgfpathlineto{\pgfqpoint{2.391899in}{2.925782in}}%
\pgfpathlineto{\pgfqpoint{2.460036in}{2.905778in}}%
\pgfpathlineto{\pgfqpoint{2.528174in}{2.885774in}}%
\pgfpathlineto{\pgfqpoint{2.537126in}{2.883145in}}%
\pgfpathlineto{\pgfqpoint{2.596311in}{2.865770in}}%
\pgfpathlineto{\pgfqpoint{2.664449in}{2.845766in}}%
\pgfpathlineto{\pgfqpoint{2.724257in}{2.828207in}}%
\pgfpathlineto{\pgfqpoint{2.732586in}{2.825762in}}%
\pgfpathlineto{\pgfqpoint{2.800724in}{2.805758in}}%
\pgfpathlineto{\pgfqpoint{2.868861in}{2.785754in}}%
\pgfpathlineto{\pgfqpoint{2.911387in}{2.773269in}}%
\pgfpathlineto{\pgfqpoint{2.936999in}{2.765750in}}%
\pgfpathlineto{\pgfqpoint{3.005136in}{2.745746in}}%
\pgfpathlineto{\pgfqpoint{3.073274in}{2.725742in}}%
\pgfpathlineto{\pgfqpoint{3.098517in}{2.718331in}}%
\pgfpathlineto{\pgfqpoint{3.141411in}{2.705738in}}%
\pgfpathlineto{\pgfqpoint{3.209549in}{2.685734in}}%
\pgfpathlineto{\pgfqpoint{3.277686in}{2.665730in}}%
\pgfpathlineto{\pgfqpoint{3.285647in}{2.663393in}}%
\pgfpathclose%
\pgfusepath{fill}%
\end{pgfscope}%
\begin{pgfscope}%
\pgfpathrectangle{\pgfqpoint{0.634105in}{0.521603in}}{\pgfqpoint{3.720000in}{3.020000in}} %
\pgfusepath{clip}%
\pgfsetbuttcap%
\pgfsetroundjoin%
\definecolor{currentfill}{rgb}{1.000000,1.000000,1.000000}%
\pgfsetfillcolor{currentfill}%
\pgfsetlinewidth{1.003750pt}%
\definecolor{currentstroke}{rgb}{0.000000,0.000000,0.000000}%
\pgfsetstrokecolor{currentstroke}%
\pgfsetdash{}{0pt}%
\pgfpathmoveto{\pgfqpoint{3.838210in}{3.335922in}}%
\pgfpathcurveto{\pgfqpoint{3.849260in}{3.335922in}}{\pgfqpoint{3.859859in}{3.340313in}}{\pgfqpoint{3.867673in}{3.348126in}}%
\pgfpathcurveto{\pgfqpoint{3.875486in}{3.355940in}}{\pgfqpoint{3.879877in}{3.366539in}}{\pgfqpoint{3.879877in}{3.377589in}}%
\pgfpathcurveto{\pgfqpoint{3.879877in}{3.388639in}}{\pgfqpoint{3.875486in}{3.399238in}}{\pgfqpoint{3.867673in}{3.407052in}}%
\pgfpathcurveto{\pgfqpoint{3.859859in}{3.414866in}}{\pgfqpoint{3.849260in}{3.419256in}}{\pgfqpoint{3.838210in}{3.419256in}}%
\pgfpathcurveto{\pgfqpoint{3.827160in}{3.419256in}}{\pgfqpoint{3.816561in}{3.414866in}}{\pgfqpoint{3.808747in}{3.407052in}}%
\pgfpathcurveto{\pgfqpoint{3.800934in}{3.399238in}}{\pgfqpoint{3.796543in}{3.388639in}}{\pgfqpoint{3.796543in}{3.377589in}}%
\pgfpathcurveto{\pgfqpoint{3.796543in}{3.366539in}}{\pgfqpoint{3.800934in}{3.355940in}}{\pgfqpoint{3.808747in}{3.348126in}}%
\pgfpathcurveto{\pgfqpoint{3.816561in}{3.340313in}}{\pgfqpoint{3.827160in}{3.335922in}}{\pgfqpoint{3.838210in}{3.335922in}}%
\pgfpathclose%
\pgfusepath{stroke,fill}%
\end{pgfscope}%
\begin{pgfscope}%
\pgfpathrectangle{\pgfqpoint{0.634105in}{0.521603in}}{\pgfqpoint{3.720000in}{3.020000in}} %
\pgfusepath{clip}%
\pgfsetbuttcap%
\pgfsetroundjoin%
\definecolor{currentfill}{rgb}{1.000000,1.000000,1.000000}%
\pgfsetfillcolor{currentfill}%
\pgfsetlinewidth{1.003750pt}%
\definecolor{currentstroke}{rgb}{0.000000,0.000000,0.000000}%
\pgfsetstrokecolor{currentstroke}%
\pgfsetdash{}{0pt}%
\pgfpathmoveto{\pgfqpoint{1.015690in}{2.178249in}}%
\pgfpathcurveto{\pgfqpoint{1.026740in}{2.178249in}}{\pgfqpoint{1.037339in}{2.182639in}}{\pgfqpoint{1.045153in}{2.190453in}}%
\pgfpathcurveto{\pgfqpoint{1.052966in}{2.198266in}}{\pgfqpoint{1.057357in}{2.208866in}}{\pgfqpoint{1.057357in}{2.219916in}}%
\pgfpathcurveto{\pgfqpoint{1.057357in}{2.230966in}}{\pgfqpoint{1.052966in}{2.241565in}}{\pgfqpoint{1.045153in}{2.249378in}}%
\pgfpathcurveto{\pgfqpoint{1.037339in}{2.257192in}}{\pgfqpoint{1.026740in}{2.261582in}}{\pgfqpoint{1.015690in}{2.261582in}}%
\pgfpathcurveto{\pgfqpoint{1.004640in}{2.261582in}}{\pgfqpoint{0.994041in}{2.257192in}}{\pgfqpoint{0.986227in}{2.249378in}}%
\pgfpathcurveto{\pgfqpoint{0.978414in}{2.241565in}}{\pgfqpoint{0.974023in}{2.230966in}}{\pgfqpoint{0.974023in}{2.219916in}}%
\pgfpathcurveto{\pgfqpoint{0.974023in}{2.208866in}}{\pgfqpoint{0.978414in}{2.198266in}}{\pgfqpoint{0.986227in}{2.190453in}}%
\pgfpathcurveto{\pgfqpoint{0.994041in}{2.182639in}}{\pgfqpoint{1.004640in}{2.178249in}}{\pgfqpoint{1.015690in}{2.178249in}}%
\pgfpathclose%
\pgfusepath{stroke,fill}%
\end{pgfscope}%
\begin{pgfscope}%
\pgfpathrectangle{\pgfqpoint{0.634105in}{0.521603in}}{\pgfqpoint{3.720000in}{3.020000in}} %
\pgfusepath{clip}%
\pgfsetbuttcap%
\pgfsetroundjoin%
\definecolor{currentfill}{rgb}{1.000000,1.000000,1.000000}%
\pgfsetfillcolor{currentfill}%
\pgfsetlinewidth{1.003750pt}%
\definecolor{currentstroke}{rgb}{0.000000,0.000000,0.000000}%
\pgfsetstrokecolor{currentstroke}%
\pgfsetdash{}{0pt}%
\pgfpathmoveto{\pgfqpoint{0.824736in}{2.177069in}}%
\pgfpathcurveto{\pgfqpoint{0.835786in}{2.177069in}}{\pgfqpoint{0.846385in}{2.181459in}}{\pgfqpoint{0.854199in}{2.189273in}}%
\pgfpathcurveto{\pgfqpoint{0.862012in}{2.197087in}}{\pgfqpoint{0.866403in}{2.207686in}}{\pgfqpoint{0.866403in}{2.218736in}}%
\pgfpathcurveto{\pgfqpoint{0.866403in}{2.229786in}}{\pgfqpoint{0.862012in}{2.240385in}}{\pgfqpoint{0.854199in}{2.248199in}}%
\pgfpathcurveto{\pgfqpoint{0.846385in}{2.256012in}}{\pgfqpoint{0.835786in}{2.260402in}}{\pgfqpoint{0.824736in}{2.260402in}}%
\pgfpathcurveto{\pgfqpoint{0.813686in}{2.260402in}}{\pgfqpoint{0.803087in}{2.256012in}}{\pgfqpoint{0.795273in}{2.248199in}}%
\pgfpathcurveto{\pgfqpoint{0.787460in}{2.240385in}}{\pgfqpoint{0.783069in}{2.229786in}}{\pgfqpoint{0.783069in}{2.218736in}}%
\pgfpathcurveto{\pgfqpoint{0.783069in}{2.207686in}}{\pgfqpoint{0.787460in}{2.197087in}}{\pgfqpoint{0.795273in}{2.189273in}}%
\pgfpathcurveto{\pgfqpoint{0.803087in}{2.181459in}}{\pgfqpoint{0.813686in}{2.177069in}}{\pgfqpoint{0.824736in}{2.177069in}}%
\pgfpathclose%
\pgfusepath{stroke,fill}%
\end{pgfscope}%
\begin{pgfscope}%
\pgfpathrectangle{\pgfqpoint{0.634105in}{0.521603in}}{\pgfqpoint{3.720000in}{3.020000in}} %
\pgfusepath{clip}%
\pgfsetbuttcap%
\pgfsetroundjoin%
\definecolor{currentfill}{rgb}{1.000000,1.000000,1.000000}%
\pgfsetfillcolor{currentfill}%
\pgfsetlinewidth{1.003750pt}%
\definecolor{currentstroke}{rgb}{0.000000,0.000000,0.000000}%
\pgfsetstrokecolor{currentstroke}%
\pgfsetdash{}{0pt}%
\pgfpathmoveto{\pgfqpoint{1.772294in}{1.116526in}}%
\pgfpathcurveto{\pgfqpoint{1.783344in}{1.116526in}}{\pgfqpoint{1.793943in}{1.120916in}}{\pgfqpoint{1.801757in}{1.128730in}}%
\pgfpathcurveto{\pgfqpoint{1.809570in}{1.136543in}}{\pgfqpoint{1.813960in}{1.147142in}}{\pgfqpoint{1.813960in}{1.158192in}}%
\pgfpathcurveto{\pgfqpoint{1.813960in}{1.169243in}}{\pgfqpoint{1.809570in}{1.179842in}}{\pgfqpoint{1.801757in}{1.187655in}}%
\pgfpathcurveto{\pgfqpoint{1.793943in}{1.195469in}}{\pgfqpoint{1.783344in}{1.199859in}}{\pgfqpoint{1.772294in}{1.199859in}}%
\pgfpathcurveto{\pgfqpoint{1.761244in}{1.199859in}}{\pgfqpoint{1.750645in}{1.195469in}}{\pgfqpoint{1.742831in}{1.187655in}}%
\pgfpathcurveto{\pgfqpoint{1.735017in}{1.179842in}}{\pgfqpoint{1.730627in}{1.169243in}}{\pgfqpoint{1.730627in}{1.158192in}}%
\pgfpathcurveto{\pgfqpoint{1.730627in}{1.147142in}}{\pgfqpoint{1.735017in}{1.136543in}}{\pgfqpoint{1.742831in}{1.128730in}}%
\pgfpathcurveto{\pgfqpoint{1.750645in}{1.120916in}}{\pgfqpoint{1.761244in}{1.116526in}}{\pgfqpoint{1.772294in}{1.116526in}}%
\pgfpathclose%
\pgfusepath{stroke,fill}%
\end{pgfscope}%
\begin{pgfscope}%
\pgfpathrectangle{\pgfqpoint{0.634105in}{0.521603in}}{\pgfqpoint{3.720000in}{3.020000in}} %
\pgfusepath{clip}%
\pgfsetbuttcap%
\pgfsetroundjoin%
\definecolor{currentfill}{rgb}{1.000000,1.000000,1.000000}%
\pgfsetfillcolor{currentfill}%
\pgfsetlinewidth{1.003750pt}%
\definecolor{currentstroke}{rgb}{0.000000,0.000000,0.000000}%
\pgfsetstrokecolor{currentstroke}%
\pgfsetdash{}{0pt}%
\pgfpathmoveto{\pgfqpoint{1.017683in}{0.972313in}}%
\pgfpathcurveto{\pgfqpoint{1.028733in}{0.972313in}}{\pgfqpoint{1.039332in}{0.976703in}}{\pgfqpoint{1.047146in}{0.984517in}}%
\pgfpathcurveto{\pgfqpoint{1.054959in}{0.992331in}}{\pgfqpoint{1.059349in}{1.002930in}}{\pgfqpoint{1.059349in}{1.013980in}}%
\pgfpathcurveto{\pgfqpoint{1.059349in}{1.025030in}}{\pgfqpoint{1.054959in}{1.035629in}}{\pgfqpoint{1.047146in}{1.043443in}}%
\pgfpathcurveto{\pgfqpoint{1.039332in}{1.051256in}}{\pgfqpoint{1.028733in}{1.055647in}}{\pgfqpoint{1.017683in}{1.055647in}}%
\pgfpathcurveto{\pgfqpoint{1.006633in}{1.055647in}}{\pgfqpoint{0.996034in}{1.051256in}}{\pgfqpoint{0.988220in}{1.043443in}}%
\pgfpathcurveto{\pgfqpoint{0.980406in}{1.035629in}}{\pgfqpoint{0.976016in}{1.025030in}}{\pgfqpoint{0.976016in}{1.013980in}}%
\pgfpathcurveto{\pgfqpoint{0.976016in}{1.002930in}}{\pgfqpoint{0.980406in}{0.992331in}}{\pgfqpoint{0.988220in}{0.984517in}}%
\pgfpathcurveto{\pgfqpoint{0.996034in}{0.976703in}}{\pgfqpoint{1.006633in}{0.972313in}}{\pgfqpoint{1.017683in}{0.972313in}}%
\pgfpathclose%
\pgfusepath{stroke,fill}%
\end{pgfscope}%
\begin{pgfscope}%
\pgfpathrectangle{\pgfqpoint{0.634105in}{0.521603in}}{\pgfqpoint{3.720000in}{3.020000in}} %
\pgfusepath{clip}%
\pgfsetbuttcap%
\pgfsetroundjoin%
\definecolor{currentfill}{rgb}{1.000000,1.000000,1.000000}%
\pgfsetfillcolor{currentfill}%
\pgfsetlinewidth{1.003750pt}%
\definecolor{currentstroke}{rgb}{0.000000,0.000000,0.000000}%
\pgfsetstrokecolor{currentstroke}%
\pgfsetdash{}{0pt}%
\pgfpathmoveto{\pgfqpoint{1.482493in}{0.751603in}}%
\pgfpathcurveto{\pgfqpoint{1.493543in}{0.751603in}}{\pgfqpoint{1.504142in}{0.755993in}}{\pgfqpoint{1.511956in}{0.763806in}}%
\pgfpathcurveto{\pgfqpoint{1.519770in}{0.771620in}}{\pgfqpoint{1.524160in}{0.782219in}}{\pgfqpoint{1.524160in}{0.793269in}}%
\pgfpathcurveto{\pgfqpoint{1.524160in}{0.804319in}}{\pgfqpoint{1.519770in}{0.814918in}}{\pgfqpoint{1.511956in}{0.822732in}}%
\pgfpathcurveto{\pgfqpoint{1.504142in}{0.830546in}}{\pgfqpoint{1.493543in}{0.834936in}}{\pgfqpoint{1.482493in}{0.834936in}}%
\pgfpathcurveto{\pgfqpoint{1.471443in}{0.834936in}}{\pgfqpoint{1.460844in}{0.830546in}}{\pgfqpoint{1.453030in}{0.822732in}}%
\pgfpathcurveto{\pgfqpoint{1.445217in}{0.814918in}}{\pgfqpoint{1.440826in}{0.804319in}}{\pgfqpoint{1.440826in}{0.793269in}}%
\pgfpathcurveto{\pgfqpoint{1.440826in}{0.782219in}}{\pgfqpoint{1.445217in}{0.771620in}}{\pgfqpoint{1.453030in}{0.763806in}}%
\pgfpathcurveto{\pgfqpoint{1.460844in}{0.755993in}}{\pgfqpoint{1.471443in}{0.751603in}}{\pgfqpoint{1.482493in}{0.751603in}}%
\pgfpathclose%
\pgfusepath{stroke,fill}%
\end{pgfscope}%
\begin{pgfscope}%
\pgfpathrectangle{\pgfqpoint{0.634105in}{0.521603in}}{\pgfqpoint{3.720000in}{3.020000in}} %
\pgfusepath{clip}%
\pgfsetbuttcap%
\pgfsetroundjoin%
\definecolor{currentfill}{rgb}{1.000000,1.000000,1.000000}%
\pgfsetfillcolor{currentfill}%
\pgfsetlinewidth{1.003750pt}%
\definecolor{currentstroke}{rgb}{0.000000,0.000000,0.000000}%
\pgfsetstrokecolor{currentstroke}%
\pgfsetdash{}{0pt}%
\pgfpathmoveto{\pgfqpoint{1.421851in}{0.643951in}}%
\pgfpathcurveto{\pgfqpoint{1.432901in}{0.643951in}}{\pgfqpoint{1.443501in}{0.648341in}}{\pgfqpoint{1.451314in}{0.656155in}}%
\pgfpathcurveto{\pgfqpoint{1.459128in}{0.663968in}}{\pgfqpoint{1.463518in}{0.674567in}}{\pgfqpoint{1.463518in}{0.685618in}}%
\pgfpathcurveto{\pgfqpoint{1.463518in}{0.696668in}}{\pgfqpoint{1.459128in}{0.707267in}}{\pgfqpoint{1.451314in}{0.715080in}}%
\pgfpathcurveto{\pgfqpoint{1.443501in}{0.722894in}}{\pgfqpoint{1.432901in}{0.727284in}}{\pgfqpoint{1.421851in}{0.727284in}}%
\pgfpathcurveto{\pgfqpoint{1.410801in}{0.727284in}}{\pgfqpoint{1.400202in}{0.722894in}}{\pgfqpoint{1.392389in}{0.715080in}}%
\pgfpathcurveto{\pgfqpoint{1.384575in}{0.707267in}}{\pgfqpoint{1.380185in}{0.696668in}}{\pgfqpoint{1.380185in}{0.685618in}}%
\pgfpathcurveto{\pgfqpoint{1.380185in}{0.674567in}}{\pgfqpoint{1.384575in}{0.663968in}}{\pgfqpoint{1.392389in}{0.656155in}}%
\pgfpathcurveto{\pgfqpoint{1.400202in}{0.648341in}}{\pgfqpoint{1.410801in}{0.643951in}}{\pgfqpoint{1.421851in}{0.643951in}}%
\pgfpathclose%
\pgfusepath{stroke,fill}%
\end{pgfscope}%
\begin{pgfscope}%
\pgfpathrectangle{\pgfqpoint{0.634105in}{0.521603in}}{\pgfqpoint{3.720000in}{3.020000in}} %
\pgfusepath{clip}%
\pgfsetbuttcap%
\pgfsetroundjoin%
\definecolor{currentfill}{rgb}{1.000000,1.000000,1.000000}%
\pgfsetfillcolor{currentfill}%
\pgfsetlinewidth{1.003750pt}%
\definecolor{currentstroke}{rgb}{0.000000,0.000000,0.000000}%
\pgfsetstrokecolor{currentstroke}%
\pgfsetdash{}{0pt}%
\pgfpathmoveto{\pgfqpoint{4.163474in}{2.623161in}}%
\pgfpathcurveto{\pgfqpoint{4.174524in}{2.623161in}}{\pgfqpoint{4.185123in}{2.627551in}}{\pgfqpoint{4.192936in}{2.635365in}}%
\pgfpathcurveto{\pgfqpoint{4.200750in}{2.643178in}}{\pgfqpoint{4.205140in}{2.653777in}}{\pgfqpoint{4.205140in}{2.664828in}}%
\pgfpathcurveto{\pgfqpoint{4.205140in}{2.675878in}}{\pgfqpoint{4.200750in}{2.686477in}}{\pgfqpoint{4.192936in}{2.694290in}}%
\pgfpathcurveto{\pgfqpoint{4.185123in}{2.702104in}}{\pgfqpoint{4.174524in}{2.706494in}}{\pgfqpoint{4.163474in}{2.706494in}}%
\pgfpathcurveto{\pgfqpoint{4.152424in}{2.706494in}}{\pgfqpoint{4.141825in}{2.702104in}}{\pgfqpoint{4.134011in}{2.694290in}}%
\pgfpathcurveto{\pgfqpoint{4.126197in}{2.686477in}}{\pgfqpoint{4.121807in}{2.675878in}}{\pgfqpoint{4.121807in}{2.664828in}}%
\pgfpathcurveto{\pgfqpoint{4.121807in}{2.653777in}}{\pgfqpoint{4.126197in}{2.643178in}}{\pgfqpoint{4.134011in}{2.635365in}}%
\pgfpathcurveto{\pgfqpoint{4.141825in}{2.627551in}}{\pgfqpoint{4.152424in}{2.623161in}}{\pgfqpoint{4.163474in}{2.623161in}}%
\pgfpathclose%
\pgfusepath{stroke,fill}%
\end{pgfscope}%
\begin{pgfscope}%
\pgfpathrectangle{\pgfqpoint{0.634105in}{0.521603in}}{\pgfqpoint{3.720000in}{3.020000in}} %
\pgfusepath{clip}%
\pgfsetbuttcap%
\pgfsetroundjoin%
\definecolor{currentfill}{rgb}{1.000000,1.000000,1.000000}%
\pgfsetfillcolor{currentfill}%
\pgfsetlinewidth{1.003750pt}%
\definecolor{currentstroke}{rgb}{0.000000,0.000000,0.000000}%
\pgfsetstrokecolor{currentstroke}%
\pgfsetdash{}{0pt}%
\pgfpathmoveto{\pgfqpoint{3.556105in}{1.839626in}}%
\pgfpathcurveto{\pgfqpoint{3.567155in}{1.839626in}}{\pgfqpoint{3.577754in}{1.844017in}}{\pgfqpoint{3.585568in}{1.851830in}}%
\pgfpathcurveto{\pgfqpoint{3.593381in}{1.859644in}}{\pgfqpoint{3.597772in}{1.870243in}}{\pgfqpoint{3.597772in}{1.881293in}}%
\pgfpathcurveto{\pgfqpoint{3.597772in}{1.892343in}}{\pgfqpoint{3.593381in}{1.902942in}}{\pgfqpoint{3.585568in}{1.910756in}}%
\pgfpathcurveto{\pgfqpoint{3.577754in}{1.918569in}}{\pgfqpoint{3.567155in}{1.922960in}}{\pgfqpoint{3.556105in}{1.922960in}}%
\pgfpathcurveto{\pgfqpoint{3.545055in}{1.922960in}}{\pgfqpoint{3.534456in}{1.918569in}}{\pgfqpoint{3.526642in}{1.910756in}}%
\pgfpathcurveto{\pgfqpoint{3.518829in}{1.902942in}}{\pgfqpoint{3.514438in}{1.892343in}}{\pgfqpoint{3.514438in}{1.881293in}}%
\pgfpathcurveto{\pgfqpoint{3.514438in}{1.870243in}}{\pgfqpoint{3.518829in}{1.859644in}}{\pgfqpoint{3.526642in}{1.851830in}}%
\pgfpathcurveto{\pgfqpoint{3.534456in}{1.844017in}}{\pgfqpoint{3.545055in}{1.839626in}}{\pgfqpoint{3.556105in}{1.839626in}}%
\pgfpathclose%
\pgfusepath{stroke,fill}%
\end{pgfscope}%
\begin{pgfscope}%
\pgfpathrectangle{\pgfqpoint{0.634105in}{0.521603in}}{\pgfqpoint{3.720000in}{3.020000in}} %
\pgfusepath{clip}%
\pgfsetbuttcap%
\pgfsetroundjoin%
\definecolor{currentfill}{rgb}{1.000000,1.000000,1.000000}%
\pgfsetfillcolor{currentfill}%
\pgfsetlinewidth{1.003750pt}%
\definecolor{currentstroke}{rgb}{0.000000,0.000000,0.000000}%
\pgfsetstrokecolor{currentstroke}%
\pgfsetdash{}{0pt}%
\pgfpathmoveto{\pgfqpoint{2.926462in}{1.486911in}}%
\pgfpathcurveto{\pgfqpoint{2.937513in}{1.486911in}}{\pgfqpoint{2.948112in}{1.491301in}}{\pgfqpoint{2.955925in}{1.499115in}}%
\pgfpathcurveto{\pgfqpoint{2.963739in}{1.506928in}}{\pgfqpoint{2.968129in}{1.517527in}}{\pgfqpoint{2.968129in}{1.528577in}}%
\pgfpathcurveto{\pgfqpoint{2.968129in}{1.539628in}}{\pgfqpoint{2.963739in}{1.550227in}}{\pgfqpoint{2.955925in}{1.558040in}}%
\pgfpathcurveto{\pgfqpoint{2.948112in}{1.565854in}}{\pgfqpoint{2.937513in}{1.570244in}}{\pgfqpoint{2.926462in}{1.570244in}}%
\pgfpathcurveto{\pgfqpoint{2.915412in}{1.570244in}}{\pgfqpoint{2.904813in}{1.565854in}}{\pgfqpoint{2.897000in}{1.558040in}}%
\pgfpathcurveto{\pgfqpoint{2.889186in}{1.550227in}}{\pgfqpoint{2.884796in}{1.539628in}}{\pgfqpoint{2.884796in}{1.528577in}}%
\pgfpathcurveto{\pgfqpoint{2.884796in}{1.517527in}}{\pgfqpoint{2.889186in}{1.506928in}}{\pgfqpoint{2.897000in}{1.499115in}}%
\pgfpathcurveto{\pgfqpoint{2.904813in}{1.491301in}}{\pgfqpoint{2.915412in}{1.486911in}}{\pgfqpoint{2.926462in}{1.486911in}}%
\pgfpathclose%
\pgfusepath{stroke,fill}%
\end{pgfscope}%
\begin{pgfscope}%
\pgfpathrectangle{\pgfqpoint{0.634105in}{0.521603in}}{\pgfqpoint{3.720000in}{3.020000in}} %
\pgfusepath{clip}%
\pgfsetbuttcap%
\pgfsetroundjoin%
\definecolor{currentfill}{rgb}{1.000000,1.000000,1.000000}%
\pgfsetfillcolor{currentfill}%
\pgfsetlinewidth{1.003750pt}%
\definecolor{currentstroke}{rgb}{0.000000,0.000000,0.000000}%
\pgfsetstrokecolor{currentstroke}%
\pgfsetdash{}{0pt}%
\pgfpathmoveto{\pgfqpoint{1.190499in}{2.912413in}}%
\pgfpathcurveto{\pgfqpoint{1.201549in}{2.912413in}}{\pgfqpoint{1.212148in}{2.916804in}}{\pgfqpoint{1.219962in}{2.924617in}}%
\pgfpathcurveto{\pgfqpoint{1.227775in}{2.932431in}}{\pgfqpoint{1.232166in}{2.943030in}}{\pgfqpoint{1.232166in}{2.954080in}}%
\pgfpathcurveto{\pgfqpoint{1.232166in}{2.965130in}}{\pgfqpoint{1.227775in}{2.975729in}}{\pgfqpoint{1.219962in}{2.983543in}}%
\pgfpathcurveto{\pgfqpoint{1.212148in}{2.991356in}}{\pgfqpoint{1.201549in}{2.995747in}}{\pgfqpoint{1.190499in}{2.995747in}}%
\pgfpathcurveto{\pgfqpoint{1.179449in}{2.995747in}}{\pgfqpoint{1.168850in}{2.991356in}}{\pgfqpoint{1.161036in}{2.983543in}}%
\pgfpathcurveto{\pgfqpoint{1.153222in}{2.975729in}}{\pgfqpoint{1.148832in}{2.965130in}}{\pgfqpoint{1.148832in}{2.954080in}}%
\pgfpathcurveto{\pgfqpoint{1.148832in}{2.943030in}}{\pgfqpoint{1.153222in}{2.932431in}}{\pgfqpoint{1.161036in}{2.924617in}}%
\pgfpathcurveto{\pgfqpoint{1.168850in}{2.916804in}}{\pgfqpoint{1.179449in}{2.912413in}}{\pgfqpoint{1.190499in}{2.912413in}}%
\pgfpathclose%
\pgfusepath{stroke,fill}%
\end{pgfscope}%
\begin{pgfscope}%
\pgfpathrectangle{\pgfqpoint{0.634105in}{0.521603in}}{\pgfqpoint{3.720000in}{3.020000in}} %
\pgfusepath{clip}%
\pgfsetbuttcap%
\pgfsetroundjoin%
\definecolor{currentfill}{rgb}{1.000000,1.000000,1.000000}%
\pgfsetfillcolor{currentfill}%
\pgfsetlinewidth{1.003750pt}%
\definecolor{currentstroke}{rgb}{0.000000,0.000000,0.000000}%
\pgfsetstrokecolor{currentstroke}%
\pgfsetdash{}{0pt}%
\pgfpathmoveto{\pgfqpoint{2.312047in}{1.118990in}}%
\pgfpathcurveto{\pgfqpoint{2.323097in}{1.118990in}}{\pgfqpoint{2.333696in}{1.123380in}}{\pgfqpoint{2.341510in}{1.131194in}}%
\pgfpathcurveto{\pgfqpoint{2.349323in}{1.139007in}}{\pgfqpoint{2.353714in}{1.149606in}}{\pgfqpoint{2.353714in}{1.160656in}}%
\pgfpathcurveto{\pgfqpoint{2.353714in}{1.171707in}}{\pgfqpoint{2.349323in}{1.182306in}}{\pgfqpoint{2.341510in}{1.190119in}}%
\pgfpathcurveto{\pgfqpoint{2.333696in}{1.197933in}}{\pgfqpoint{2.323097in}{1.202323in}}{\pgfqpoint{2.312047in}{1.202323in}}%
\pgfpathcurveto{\pgfqpoint{2.300997in}{1.202323in}}{\pgfqpoint{2.290398in}{1.197933in}}{\pgfqpoint{2.282584in}{1.190119in}}%
\pgfpathcurveto{\pgfqpoint{2.274770in}{1.182306in}}{\pgfqpoint{2.270380in}{1.171707in}}{\pgfqpoint{2.270380in}{1.160656in}}%
\pgfpathcurveto{\pgfqpoint{2.270380in}{1.149606in}}{\pgfqpoint{2.274770in}{1.139007in}}{\pgfqpoint{2.282584in}{1.131194in}}%
\pgfpathcurveto{\pgfqpoint{2.290398in}{1.123380in}}{\pgfqpoint{2.300997in}{1.118990in}}{\pgfqpoint{2.312047in}{1.118990in}}%
\pgfpathclose%
\pgfusepath{stroke,fill}%
\end{pgfscope}%
\begin{pgfscope}%
\pgfpathrectangle{\pgfqpoint{0.634105in}{0.521603in}}{\pgfqpoint{3.720000in}{3.020000in}} %
\pgfusepath{clip}%
\pgfsetbuttcap%
\pgfsetroundjoin%
\definecolor{currentfill}{rgb}{1.000000,1.000000,1.000000}%
\pgfsetfillcolor{currentfill}%
\pgfsetlinewidth{1.003750pt}%
\definecolor{currentstroke}{rgb}{0.000000,0.000000,0.000000}%
\pgfsetstrokecolor{currentstroke}%
\pgfsetdash{}{0pt}%
\pgfpathmoveto{\pgfqpoint{2.024287in}{1.311281in}}%
\pgfpathcurveto{\pgfqpoint{2.035338in}{1.311281in}}{\pgfqpoint{2.045937in}{1.315671in}}{\pgfqpoint{2.053750in}{1.323484in}}%
\pgfpathcurveto{\pgfqpoint{2.061564in}{1.331298in}}{\pgfqpoint{2.065954in}{1.341897in}}{\pgfqpoint{2.065954in}{1.352947in}}%
\pgfpathcurveto{\pgfqpoint{2.065954in}{1.363997in}}{\pgfqpoint{2.061564in}{1.374596in}}{\pgfqpoint{2.053750in}{1.382410in}}%
\pgfpathcurveto{\pgfqpoint{2.045937in}{1.390224in}}{\pgfqpoint{2.035338in}{1.394614in}}{\pgfqpoint{2.024287in}{1.394614in}}%
\pgfpathcurveto{\pgfqpoint{2.013237in}{1.394614in}}{\pgfqpoint{2.002638in}{1.390224in}}{\pgfqpoint{1.994825in}{1.382410in}}%
\pgfpathcurveto{\pgfqpoint{1.987011in}{1.374596in}}{\pgfqpoint{1.982621in}{1.363997in}}{\pgfqpoint{1.982621in}{1.352947in}}%
\pgfpathcurveto{\pgfqpoint{1.982621in}{1.341897in}}{\pgfqpoint{1.987011in}{1.331298in}}{\pgfqpoint{1.994825in}{1.323484in}}%
\pgfpathcurveto{\pgfqpoint{2.002638in}{1.315671in}}{\pgfqpoint{2.013237in}{1.311281in}}{\pgfqpoint{2.024287in}{1.311281in}}%
\pgfpathclose%
\pgfusepath{stroke,fill}%
\end{pgfscope}%
\begin{pgfscope}%
\pgfpathrectangle{\pgfqpoint{0.634105in}{0.521603in}}{\pgfqpoint{3.720000in}{3.020000in}} %
\pgfusepath{clip}%
\pgfsetbuttcap%
\pgfsetroundjoin%
\definecolor{currentfill}{rgb}{1.000000,1.000000,1.000000}%
\pgfsetfillcolor{currentfill}%
\pgfsetlinewidth{1.003750pt}%
\definecolor{currentstroke}{rgb}{0.000000,0.000000,0.000000}%
\pgfsetstrokecolor{currentstroke}%
\pgfsetdash{}{0pt}%
\pgfpathmoveto{\pgfqpoint{1.154746in}{2.170547in}}%
\pgfpathcurveto{\pgfqpoint{1.165796in}{2.170547in}}{\pgfqpoint{1.176395in}{2.174937in}}{\pgfqpoint{1.184209in}{2.182751in}}%
\pgfpathcurveto{\pgfqpoint{1.192022in}{2.190564in}}{\pgfqpoint{1.196413in}{2.201163in}}{\pgfqpoint{1.196413in}{2.212213in}}%
\pgfpathcurveto{\pgfqpoint{1.196413in}{2.223264in}}{\pgfqpoint{1.192022in}{2.233863in}}{\pgfqpoint{1.184209in}{2.241676in}}%
\pgfpathcurveto{\pgfqpoint{1.176395in}{2.249490in}}{\pgfqpoint{1.165796in}{2.253880in}}{\pgfqpoint{1.154746in}{2.253880in}}%
\pgfpathcurveto{\pgfqpoint{1.143696in}{2.253880in}}{\pgfqpoint{1.133097in}{2.249490in}}{\pgfqpoint{1.125283in}{2.241676in}}%
\pgfpathcurveto{\pgfqpoint{1.117470in}{2.233863in}}{\pgfqpoint{1.113079in}{2.223264in}}{\pgfqpoint{1.113079in}{2.212213in}}%
\pgfpathcurveto{\pgfqpoint{1.113079in}{2.201163in}}{\pgfqpoint{1.117470in}{2.190564in}}{\pgfqpoint{1.125283in}{2.182751in}}%
\pgfpathcurveto{\pgfqpoint{1.133097in}{2.174937in}}{\pgfqpoint{1.143696in}{2.170547in}}{\pgfqpoint{1.154746in}{2.170547in}}%
\pgfpathclose%
\pgfusepath{stroke,fill}%
\end{pgfscope}%
\begin{pgfscope}%
\pgfpathrectangle{\pgfqpoint{0.634105in}{0.521603in}}{\pgfqpoint{3.720000in}{3.020000in}} %
\pgfusepath{clip}%
\pgfsetbuttcap%
\pgfsetroundjoin%
\definecolor{currentfill}{rgb}{1.000000,1.000000,1.000000}%
\pgfsetfillcolor{currentfill}%
\pgfsetlinewidth{1.003750pt}%
\definecolor{currentstroke}{rgb}{0.000000,0.000000,0.000000}%
\pgfsetstrokecolor{currentstroke}%
\pgfsetdash{}{0pt}%
\pgfpathmoveto{\pgfqpoint{2.182791in}{0.747687in}}%
\pgfpathcurveto{\pgfqpoint{2.193841in}{0.747687in}}{\pgfqpoint{2.204440in}{0.752077in}}{\pgfqpoint{2.212254in}{0.759891in}}%
\pgfpathcurveto{\pgfqpoint{2.220067in}{0.767704in}}{\pgfqpoint{2.224457in}{0.778303in}}{\pgfqpoint{2.224457in}{0.789353in}}%
\pgfpathcurveto{\pgfqpoint{2.224457in}{0.800403in}}{\pgfqpoint{2.220067in}{0.811002in}}{\pgfqpoint{2.212254in}{0.818816in}}%
\pgfpathcurveto{\pgfqpoint{2.204440in}{0.826630in}}{\pgfqpoint{2.193841in}{0.831020in}}{\pgfqpoint{2.182791in}{0.831020in}}%
\pgfpathcurveto{\pgfqpoint{2.171741in}{0.831020in}}{\pgfqpoint{2.161142in}{0.826630in}}{\pgfqpoint{2.153328in}{0.818816in}}%
\pgfpathcurveto{\pgfqpoint{2.145514in}{0.811002in}}{\pgfqpoint{2.141124in}{0.800403in}}{\pgfqpoint{2.141124in}{0.789353in}}%
\pgfpathcurveto{\pgfqpoint{2.141124in}{0.778303in}}{\pgfqpoint{2.145514in}{0.767704in}}{\pgfqpoint{2.153328in}{0.759891in}}%
\pgfpathcurveto{\pgfqpoint{2.161142in}{0.752077in}}{\pgfqpoint{2.171741in}{0.747687in}}{\pgfqpoint{2.182791in}{0.747687in}}%
\pgfpathclose%
\pgfusepath{stroke,fill}%
\end{pgfscope}%
\begin{pgfscope}%
\pgfsetbuttcap%
\pgfsetroundjoin%
\definecolor{currentfill}{rgb}{0.000000,0.000000,0.000000}%
\pgfsetfillcolor{currentfill}%
\pgfsetlinewidth{0.803000pt}%
\definecolor{currentstroke}{rgb}{0.000000,0.000000,0.000000}%
\pgfsetstrokecolor{currentstroke}%
\pgfsetdash{}{0pt}%
\pgfsys@defobject{currentmarker}{\pgfqpoint{0.000000in}{-0.048611in}}{\pgfqpoint{0.000000in}{0.000000in}}{%
\pgfpathmoveto{\pgfqpoint{0.000000in}{0.000000in}}%
\pgfpathlineto{\pgfqpoint{0.000000in}{-0.048611in}}%
\pgfusepath{stroke,fill}%
}%
\begin{pgfscope}%
\pgfsys@transformshift{0.674322in}{0.521603in}%
\pgfsys@useobject{currentmarker}{}%
\end{pgfscope}%
\end{pgfscope}%
\begin{pgfscope}%
\pgftext[x=0.674322in,y=0.424381in,,top]{\rmfamily\fontsize{10.000000}{12.000000}\selectfont \(\displaystyle 0.4\)}%
\end{pgfscope}%
\begin{pgfscope}%
\pgfsetbuttcap%
\pgfsetroundjoin%
\definecolor{currentfill}{rgb}{0.000000,0.000000,0.000000}%
\pgfsetfillcolor{currentfill}%
\pgfsetlinewidth{0.803000pt}%
\definecolor{currentstroke}{rgb}{0.000000,0.000000,0.000000}%
\pgfsetstrokecolor{currentstroke}%
\pgfsetdash{}{0pt}%
\pgfsys@defobject{currentmarker}{\pgfqpoint{0.000000in}{-0.048611in}}{\pgfqpoint{0.000000in}{0.000000in}}{%
\pgfpathmoveto{\pgfqpoint{0.000000in}{0.000000in}}%
\pgfpathlineto{\pgfqpoint{0.000000in}{-0.048611in}}%
\pgfusepath{stroke,fill}%
}%
\begin{pgfscope}%
\pgfsys@transformshift{1.156200in}{0.521603in}%
\pgfsys@useobject{currentmarker}{}%
\end{pgfscope}%
\end{pgfscope}%
\begin{pgfscope}%
\pgftext[x=1.156200in,y=0.424381in,,top]{\rmfamily\fontsize{10.000000}{12.000000}\selectfont \(\displaystyle 0.6\)}%
\end{pgfscope}%
\begin{pgfscope}%
\pgfsetbuttcap%
\pgfsetroundjoin%
\definecolor{currentfill}{rgb}{0.000000,0.000000,0.000000}%
\pgfsetfillcolor{currentfill}%
\pgfsetlinewidth{0.803000pt}%
\definecolor{currentstroke}{rgb}{0.000000,0.000000,0.000000}%
\pgfsetstrokecolor{currentstroke}%
\pgfsetdash{}{0pt}%
\pgfsys@defobject{currentmarker}{\pgfqpoint{0.000000in}{-0.048611in}}{\pgfqpoint{0.000000in}{0.000000in}}{%
\pgfpathmoveto{\pgfqpoint{0.000000in}{0.000000in}}%
\pgfpathlineto{\pgfqpoint{0.000000in}{-0.048611in}}%
\pgfusepath{stroke,fill}%
}%
\begin{pgfscope}%
\pgfsys@transformshift{1.638079in}{0.521603in}%
\pgfsys@useobject{currentmarker}{}%
\end{pgfscope}%
\end{pgfscope}%
\begin{pgfscope}%
\pgftext[x=1.638079in,y=0.424381in,,top]{\rmfamily\fontsize{10.000000}{12.000000}\selectfont \(\displaystyle 0.8\)}%
\end{pgfscope}%
\begin{pgfscope}%
\pgfsetbuttcap%
\pgfsetroundjoin%
\definecolor{currentfill}{rgb}{0.000000,0.000000,0.000000}%
\pgfsetfillcolor{currentfill}%
\pgfsetlinewidth{0.803000pt}%
\definecolor{currentstroke}{rgb}{0.000000,0.000000,0.000000}%
\pgfsetstrokecolor{currentstroke}%
\pgfsetdash{}{0pt}%
\pgfsys@defobject{currentmarker}{\pgfqpoint{0.000000in}{-0.048611in}}{\pgfqpoint{0.000000in}{0.000000in}}{%
\pgfpathmoveto{\pgfqpoint{0.000000in}{0.000000in}}%
\pgfpathlineto{\pgfqpoint{0.000000in}{-0.048611in}}%
\pgfusepath{stroke,fill}%
}%
\begin{pgfscope}%
\pgfsys@transformshift{2.119958in}{0.521603in}%
\pgfsys@useobject{currentmarker}{}%
\end{pgfscope}%
\end{pgfscope}%
\begin{pgfscope}%
\pgftext[x=2.119958in,y=0.424381in,,top]{\rmfamily\fontsize{10.000000}{12.000000}\selectfont \(\displaystyle 1.0\)}%
\end{pgfscope}%
\begin{pgfscope}%
\pgfsetbuttcap%
\pgfsetroundjoin%
\definecolor{currentfill}{rgb}{0.000000,0.000000,0.000000}%
\pgfsetfillcolor{currentfill}%
\pgfsetlinewidth{0.803000pt}%
\definecolor{currentstroke}{rgb}{0.000000,0.000000,0.000000}%
\pgfsetstrokecolor{currentstroke}%
\pgfsetdash{}{0pt}%
\pgfsys@defobject{currentmarker}{\pgfqpoint{0.000000in}{-0.048611in}}{\pgfqpoint{0.000000in}{0.000000in}}{%
\pgfpathmoveto{\pgfqpoint{0.000000in}{0.000000in}}%
\pgfpathlineto{\pgfqpoint{0.000000in}{-0.048611in}}%
\pgfusepath{stroke,fill}%
}%
\begin{pgfscope}%
\pgfsys@transformshift{2.601837in}{0.521603in}%
\pgfsys@useobject{currentmarker}{}%
\end{pgfscope}%
\end{pgfscope}%
\begin{pgfscope}%
\pgftext[x=2.601837in,y=0.424381in,,top]{\rmfamily\fontsize{10.000000}{12.000000}\selectfont \(\displaystyle 1.2\)}%
\end{pgfscope}%
\begin{pgfscope}%
\pgfsetbuttcap%
\pgfsetroundjoin%
\definecolor{currentfill}{rgb}{0.000000,0.000000,0.000000}%
\pgfsetfillcolor{currentfill}%
\pgfsetlinewidth{0.803000pt}%
\definecolor{currentstroke}{rgb}{0.000000,0.000000,0.000000}%
\pgfsetstrokecolor{currentstroke}%
\pgfsetdash{}{0pt}%
\pgfsys@defobject{currentmarker}{\pgfqpoint{0.000000in}{-0.048611in}}{\pgfqpoint{0.000000in}{0.000000in}}{%
\pgfpathmoveto{\pgfqpoint{0.000000in}{0.000000in}}%
\pgfpathlineto{\pgfqpoint{0.000000in}{-0.048611in}}%
\pgfusepath{stroke,fill}%
}%
\begin{pgfscope}%
\pgfsys@transformshift{3.083716in}{0.521603in}%
\pgfsys@useobject{currentmarker}{}%
\end{pgfscope}%
\end{pgfscope}%
\begin{pgfscope}%
\pgftext[x=3.083716in,y=0.424381in,,top]{\rmfamily\fontsize{10.000000}{12.000000}\selectfont \(\displaystyle 1.4\)}%
\end{pgfscope}%
\begin{pgfscope}%
\pgfsetbuttcap%
\pgfsetroundjoin%
\definecolor{currentfill}{rgb}{0.000000,0.000000,0.000000}%
\pgfsetfillcolor{currentfill}%
\pgfsetlinewidth{0.803000pt}%
\definecolor{currentstroke}{rgb}{0.000000,0.000000,0.000000}%
\pgfsetstrokecolor{currentstroke}%
\pgfsetdash{}{0pt}%
\pgfsys@defobject{currentmarker}{\pgfqpoint{0.000000in}{-0.048611in}}{\pgfqpoint{0.000000in}{0.000000in}}{%
\pgfpathmoveto{\pgfqpoint{0.000000in}{0.000000in}}%
\pgfpathlineto{\pgfqpoint{0.000000in}{-0.048611in}}%
\pgfusepath{stroke,fill}%
}%
\begin{pgfscope}%
\pgfsys@transformshift{3.565595in}{0.521603in}%
\pgfsys@useobject{currentmarker}{}%
\end{pgfscope}%
\end{pgfscope}%
\begin{pgfscope}%
\pgftext[x=3.565595in,y=0.424381in,,top]{\rmfamily\fontsize{10.000000}{12.000000}\selectfont \(\displaystyle 1.6\)}%
\end{pgfscope}%
\begin{pgfscope}%
\pgfsetbuttcap%
\pgfsetroundjoin%
\definecolor{currentfill}{rgb}{0.000000,0.000000,0.000000}%
\pgfsetfillcolor{currentfill}%
\pgfsetlinewidth{0.803000pt}%
\definecolor{currentstroke}{rgb}{0.000000,0.000000,0.000000}%
\pgfsetstrokecolor{currentstroke}%
\pgfsetdash{}{0pt}%
\pgfsys@defobject{currentmarker}{\pgfqpoint{0.000000in}{-0.048611in}}{\pgfqpoint{0.000000in}{0.000000in}}{%
\pgfpathmoveto{\pgfqpoint{0.000000in}{0.000000in}}%
\pgfpathlineto{\pgfqpoint{0.000000in}{-0.048611in}}%
\pgfusepath{stroke,fill}%
}%
\begin{pgfscope}%
\pgfsys@transformshift{4.047474in}{0.521603in}%
\pgfsys@useobject{currentmarker}{}%
\end{pgfscope}%
\end{pgfscope}%
\begin{pgfscope}%
\pgftext[x=4.047474in,y=0.424381in,,top]{\rmfamily\fontsize{10.000000}{12.000000}\selectfont \(\displaystyle 1.8\)}%
\end{pgfscope}%
\begin{pgfscope}%
\pgftext[x=2.494105in,y=0.234413in,,top]{\rmfamily\fontsize{10.000000}{12.000000}\selectfont \(\displaystyle \varphi_s\) (kV)}%
\end{pgfscope}%
\begin{pgfscope}%
\pgfsetbuttcap%
\pgfsetroundjoin%
\definecolor{currentfill}{rgb}{0.000000,0.000000,0.000000}%
\pgfsetfillcolor{currentfill}%
\pgfsetlinewidth{0.803000pt}%
\definecolor{currentstroke}{rgb}{0.000000,0.000000,0.000000}%
\pgfsetstrokecolor{currentstroke}%
\pgfsetdash{}{0pt}%
\pgfsys@defobject{currentmarker}{\pgfqpoint{-0.048611in}{0.000000in}}{\pgfqpoint{0.000000in}{0.000000in}}{%
\pgfpathmoveto{\pgfqpoint{0.000000in}{0.000000in}}%
\pgfpathlineto{\pgfqpoint{-0.048611in}{0.000000in}}%
\pgfusepath{stroke,fill}%
}%
\begin{pgfscope}%
\pgfsys@transformshift{0.634105in}{0.571135in}%
\pgfsys@useobject{currentmarker}{}%
\end{pgfscope}%
\end{pgfscope}%
\begin{pgfscope}%
\pgftext[x=0.289968in,y=0.518373in,left,base]{\rmfamily\fontsize{10.000000}{12.000000}\selectfont \(\displaystyle 0.00\)}%
\end{pgfscope}%
\begin{pgfscope}%
\pgfsetbuttcap%
\pgfsetroundjoin%
\definecolor{currentfill}{rgb}{0.000000,0.000000,0.000000}%
\pgfsetfillcolor{currentfill}%
\pgfsetlinewidth{0.803000pt}%
\definecolor{currentstroke}{rgb}{0.000000,0.000000,0.000000}%
\pgfsetstrokecolor{currentstroke}%
\pgfsetdash{}{0pt}%
\pgfsys@defobject{currentmarker}{\pgfqpoint{-0.048611in}{0.000000in}}{\pgfqpoint{0.000000in}{0.000000in}}{%
\pgfpathmoveto{\pgfqpoint{0.000000in}{0.000000in}}%
\pgfpathlineto{\pgfqpoint{-0.048611in}{0.000000in}}%
\pgfusepath{stroke,fill}%
}%
\begin{pgfscope}%
\pgfsys@transformshift{0.634105in}{0.942149in}%
\pgfsys@useobject{currentmarker}{}%
\end{pgfscope}%
\end{pgfscope}%
\begin{pgfscope}%
\pgftext[x=0.289968in,y=0.889387in,left,base]{\rmfamily\fontsize{10.000000}{12.000000}\selectfont \(\displaystyle 0.05\)}%
\end{pgfscope}%
\begin{pgfscope}%
\pgfsetbuttcap%
\pgfsetroundjoin%
\definecolor{currentfill}{rgb}{0.000000,0.000000,0.000000}%
\pgfsetfillcolor{currentfill}%
\pgfsetlinewidth{0.803000pt}%
\definecolor{currentstroke}{rgb}{0.000000,0.000000,0.000000}%
\pgfsetstrokecolor{currentstroke}%
\pgfsetdash{}{0pt}%
\pgfsys@defobject{currentmarker}{\pgfqpoint{-0.048611in}{0.000000in}}{\pgfqpoint{0.000000in}{0.000000in}}{%
\pgfpathmoveto{\pgfqpoint{0.000000in}{0.000000in}}%
\pgfpathlineto{\pgfqpoint{-0.048611in}{0.000000in}}%
\pgfusepath{stroke,fill}%
}%
\begin{pgfscope}%
\pgfsys@transformshift{0.634105in}{1.313162in}%
\pgfsys@useobject{currentmarker}{}%
\end{pgfscope}%
\end{pgfscope}%
\begin{pgfscope}%
\pgftext[x=0.289968in,y=1.260401in,left,base]{\rmfamily\fontsize{10.000000}{12.000000}\selectfont \(\displaystyle 0.10\)}%
\end{pgfscope}%
\begin{pgfscope}%
\pgfsetbuttcap%
\pgfsetroundjoin%
\definecolor{currentfill}{rgb}{0.000000,0.000000,0.000000}%
\pgfsetfillcolor{currentfill}%
\pgfsetlinewidth{0.803000pt}%
\definecolor{currentstroke}{rgb}{0.000000,0.000000,0.000000}%
\pgfsetstrokecolor{currentstroke}%
\pgfsetdash{}{0pt}%
\pgfsys@defobject{currentmarker}{\pgfqpoint{-0.048611in}{0.000000in}}{\pgfqpoint{0.000000in}{0.000000in}}{%
\pgfpathmoveto{\pgfqpoint{0.000000in}{0.000000in}}%
\pgfpathlineto{\pgfqpoint{-0.048611in}{0.000000in}}%
\pgfusepath{stroke,fill}%
}%
\begin{pgfscope}%
\pgfsys@transformshift{0.634105in}{1.684176in}%
\pgfsys@useobject{currentmarker}{}%
\end{pgfscope}%
\end{pgfscope}%
\begin{pgfscope}%
\pgftext[x=0.289968in,y=1.631415in,left,base]{\rmfamily\fontsize{10.000000}{12.000000}\selectfont \(\displaystyle 0.15\)}%
\end{pgfscope}%
\begin{pgfscope}%
\pgfsetbuttcap%
\pgfsetroundjoin%
\definecolor{currentfill}{rgb}{0.000000,0.000000,0.000000}%
\pgfsetfillcolor{currentfill}%
\pgfsetlinewidth{0.803000pt}%
\definecolor{currentstroke}{rgb}{0.000000,0.000000,0.000000}%
\pgfsetstrokecolor{currentstroke}%
\pgfsetdash{}{0pt}%
\pgfsys@defobject{currentmarker}{\pgfqpoint{-0.048611in}{0.000000in}}{\pgfqpoint{0.000000in}{0.000000in}}{%
\pgfpathmoveto{\pgfqpoint{0.000000in}{0.000000in}}%
\pgfpathlineto{\pgfqpoint{-0.048611in}{0.000000in}}%
\pgfusepath{stroke,fill}%
}%
\begin{pgfscope}%
\pgfsys@transformshift{0.634105in}{2.055190in}%
\pgfsys@useobject{currentmarker}{}%
\end{pgfscope}%
\end{pgfscope}%
\begin{pgfscope}%
\pgftext[x=0.289968in,y=2.002429in,left,base]{\rmfamily\fontsize{10.000000}{12.000000}\selectfont \(\displaystyle 0.20\)}%
\end{pgfscope}%
\begin{pgfscope}%
\pgfsetbuttcap%
\pgfsetroundjoin%
\definecolor{currentfill}{rgb}{0.000000,0.000000,0.000000}%
\pgfsetfillcolor{currentfill}%
\pgfsetlinewidth{0.803000pt}%
\definecolor{currentstroke}{rgb}{0.000000,0.000000,0.000000}%
\pgfsetstrokecolor{currentstroke}%
\pgfsetdash{}{0pt}%
\pgfsys@defobject{currentmarker}{\pgfqpoint{-0.048611in}{0.000000in}}{\pgfqpoint{0.000000in}{0.000000in}}{%
\pgfpathmoveto{\pgfqpoint{0.000000in}{0.000000in}}%
\pgfpathlineto{\pgfqpoint{-0.048611in}{0.000000in}}%
\pgfusepath{stroke,fill}%
}%
\begin{pgfscope}%
\pgfsys@transformshift{0.634105in}{2.426204in}%
\pgfsys@useobject{currentmarker}{}%
\end{pgfscope}%
\end{pgfscope}%
\begin{pgfscope}%
\pgftext[x=0.289968in,y=2.373443in,left,base]{\rmfamily\fontsize{10.000000}{12.000000}\selectfont \(\displaystyle 0.25\)}%
\end{pgfscope}%
\begin{pgfscope}%
\pgfsetbuttcap%
\pgfsetroundjoin%
\definecolor{currentfill}{rgb}{0.000000,0.000000,0.000000}%
\pgfsetfillcolor{currentfill}%
\pgfsetlinewidth{0.803000pt}%
\definecolor{currentstroke}{rgb}{0.000000,0.000000,0.000000}%
\pgfsetstrokecolor{currentstroke}%
\pgfsetdash{}{0pt}%
\pgfsys@defobject{currentmarker}{\pgfqpoint{-0.048611in}{0.000000in}}{\pgfqpoint{0.000000in}{0.000000in}}{%
\pgfpathmoveto{\pgfqpoint{0.000000in}{0.000000in}}%
\pgfpathlineto{\pgfqpoint{-0.048611in}{0.000000in}}%
\pgfusepath{stroke,fill}%
}%
\begin{pgfscope}%
\pgfsys@transformshift{0.634105in}{2.797218in}%
\pgfsys@useobject{currentmarker}{}%
\end{pgfscope}%
\end{pgfscope}%
\begin{pgfscope}%
\pgftext[x=0.289968in,y=2.744457in,left,base]{\rmfamily\fontsize{10.000000}{12.000000}\selectfont \(\displaystyle 0.30\)}%
\end{pgfscope}%
\begin{pgfscope}%
\pgfsetbuttcap%
\pgfsetroundjoin%
\definecolor{currentfill}{rgb}{0.000000,0.000000,0.000000}%
\pgfsetfillcolor{currentfill}%
\pgfsetlinewidth{0.803000pt}%
\definecolor{currentstroke}{rgb}{0.000000,0.000000,0.000000}%
\pgfsetstrokecolor{currentstroke}%
\pgfsetdash{}{0pt}%
\pgfsys@defobject{currentmarker}{\pgfqpoint{-0.048611in}{0.000000in}}{\pgfqpoint{0.000000in}{0.000000in}}{%
\pgfpathmoveto{\pgfqpoint{0.000000in}{0.000000in}}%
\pgfpathlineto{\pgfqpoint{-0.048611in}{0.000000in}}%
\pgfusepath{stroke,fill}%
}%
\begin{pgfscope}%
\pgfsys@transformshift{0.634105in}{3.168232in}%
\pgfsys@useobject{currentmarker}{}%
\end{pgfscope}%
\end{pgfscope}%
\begin{pgfscope}%
\pgftext[x=0.289968in,y=3.115470in,left,base]{\rmfamily\fontsize{10.000000}{12.000000}\selectfont \(\displaystyle 0.35\)}%
\end{pgfscope}%
\begin{pgfscope}%
\pgfsetbuttcap%
\pgfsetroundjoin%
\definecolor{currentfill}{rgb}{0.000000,0.000000,0.000000}%
\pgfsetfillcolor{currentfill}%
\pgfsetlinewidth{0.803000pt}%
\definecolor{currentstroke}{rgb}{0.000000,0.000000,0.000000}%
\pgfsetstrokecolor{currentstroke}%
\pgfsetdash{}{0pt}%
\pgfsys@defobject{currentmarker}{\pgfqpoint{-0.048611in}{0.000000in}}{\pgfqpoint{0.000000in}{0.000000in}}{%
\pgfpathmoveto{\pgfqpoint{0.000000in}{0.000000in}}%
\pgfpathlineto{\pgfqpoint{-0.048611in}{0.000000in}}%
\pgfusepath{stroke,fill}%
}%
\begin{pgfscope}%
\pgfsys@transformshift{0.634105in}{3.539246in}%
\pgfsys@useobject{currentmarker}{}%
\end{pgfscope}%
\end{pgfscope}%
\begin{pgfscope}%
\pgftext[x=0.289968in,y=3.486484in,left,base]{\rmfamily\fontsize{10.000000}{12.000000}\selectfont \(\displaystyle 0.40\)}%
\end{pgfscope}%
\begin{pgfscope}%
\pgftext[x=0.234413in,y=2.031603in,,bottom,rotate=90.000000]{\rmfamily\fontsize{10.000000}{12.000000}\selectfont \(\displaystyle V_d\) (mL)}%
\end{pgfscope}%
\begin{pgfscope}%
\pgfsetrectcap%
\pgfsetmiterjoin%
\pgfsetlinewidth{0.803000pt}%
\definecolor{currentstroke}{rgb}{0.000000,0.000000,0.000000}%
\pgfsetstrokecolor{currentstroke}%
\pgfsetdash{}{0pt}%
\pgfpathmoveto{\pgfqpoint{0.634105in}{0.521603in}}%
\pgfpathlineto{\pgfqpoint{0.634105in}{3.541603in}}%
\pgfusepath{stroke}%
\end{pgfscope}%
\begin{pgfscope}%
\pgfsetrectcap%
\pgfsetmiterjoin%
\pgfsetlinewidth{0.803000pt}%
\definecolor{currentstroke}{rgb}{0.000000,0.000000,0.000000}%
\pgfsetstrokecolor{currentstroke}%
\pgfsetdash{}{0pt}%
\pgfpathmoveto{\pgfqpoint{4.354105in}{0.521603in}}%
\pgfpathlineto{\pgfqpoint{4.354105in}{3.541603in}}%
\pgfusepath{stroke}%
\end{pgfscope}%
\begin{pgfscope}%
\pgfsetrectcap%
\pgfsetmiterjoin%
\pgfsetlinewidth{0.803000pt}%
\definecolor{currentstroke}{rgb}{0.000000,0.000000,0.000000}%
\pgfsetstrokecolor{currentstroke}%
\pgfsetdash{}{0pt}%
\pgfpathmoveto{\pgfqpoint{0.634105in}{0.521603in}}%
\pgfpathlineto{\pgfqpoint{4.354105in}{0.521603in}}%
\pgfusepath{stroke}%
\end{pgfscope}%
\begin{pgfscope}%
\pgfsetrectcap%
\pgfsetmiterjoin%
\pgfsetlinewidth{0.803000pt}%
\definecolor{currentstroke}{rgb}{0.000000,0.000000,0.000000}%
\pgfsetstrokecolor{currentstroke}%
\pgfsetdash{}{0pt}%
\pgfpathmoveto{\pgfqpoint{0.634105in}{3.541603in}}%
\pgfpathlineto{\pgfqpoint{4.354105in}{3.541603in}}%
\pgfusepath{stroke}%
\end{pgfscope}%
\begin{pgfscope}%
\pgfpathrectangle{\pgfqpoint{4.586605in}{0.521603in}}{\pgfqpoint{0.151000in}{3.020000in}} %
\pgfusepath{clip}%
\pgfsetbuttcap%
\pgfsetmiterjoin%
\definecolor{currentfill}{rgb}{1.000000,1.000000,1.000000}%
\pgfsetfillcolor{currentfill}%
\pgfsetlinewidth{0.010037pt}%
\definecolor{currentstroke}{rgb}{1.000000,1.000000,1.000000}%
\pgfsetstrokecolor{currentstroke}%
\pgfsetdash{}{0pt}%
\pgfpathmoveto{\pgfqpoint{4.586605in}{0.521603in}}%
\pgfpathlineto{\pgfqpoint{4.586605in}{0.953032in}}%
\pgfpathlineto{\pgfqpoint{4.586605in}{3.110175in}}%
\pgfpathlineto{\pgfqpoint{4.586605in}{3.541603in}}%
\pgfpathlineto{\pgfqpoint{4.737605in}{3.541603in}}%
\pgfpathlineto{\pgfqpoint{4.737605in}{3.110175in}}%
\pgfpathlineto{\pgfqpoint{4.737605in}{0.953032in}}%
\pgfpathlineto{\pgfqpoint{4.737605in}{0.521603in}}%
\pgfpathclose%
\pgfusepath{stroke,fill}%
\end{pgfscope}%
\begin{pgfscope}%
\pgfpathrectangle{\pgfqpoint{4.586605in}{0.521603in}}{\pgfqpoint{0.151000in}{3.020000in}} %
\pgfusepath{clip}%
\pgfsetbuttcap%
\pgfsetroundjoin%
\definecolor{currentfill}{rgb}{0.061765,0.061765,0.085934}%
\pgfsetfillcolor{currentfill}%
\pgfsetlinewidth{0.000000pt}%
\definecolor{currentstroke}{rgb}{0.000000,0.000000,0.000000}%
\pgfsetstrokecolor{currentstroke}%
\pgfsetdash{}{0pt}%
\pgfpathmoveto{\pgfqpoint{4.586605in}{0.521603in}}%
\pgfpathlineto{\pgfqpoint{4.737605in}{0.521603in}}%
\pgfpathlineto{\pgfqpoint{4.737605in}{0.953032in}}%
\pgfpathlineto{\pgfqpoint{4.586605in}{0.953032in}}%
\pgfpathlineto{\pgfqpoint{4.586605in}{0.521603in}}%
\pgfusepath{fill}%
\end{pgfscope}%
\begin{pgfscope}%
\pgfpathrectangle{\pgfqpoint{4.586605in}{0.521603in}}{\pgfqpoint{0.151000in}{3.020000in}} %
\pgfusepath{clip}%
\pgfsetbuttcap%
\pgfsetroundjoin%
\definecolor{currentfill}{rgb}{0.185294,0.185294,0.257801}%
\pgfsetfillcolor{currentfill}%
\pgfsetlinewidth{0.000000pt}%
\definecolor{currentstroke}{rgb}{0.000000,0.000000,0.000000}%
\pgfsetstrokecolor{currentstroke}%
\pgfsetdash{}{0pt}%
\pgfpathmoveto{\pgfqpoint{4.586605in}{0.953032in}}%
\pgfpathlineto{\pgfqpoint{4.737605in}{0.953032in}}%
\pgfpathlineto{\pgfqpoint{4.737605in}{1.384460in}}%
\pgfpathlineto{\pgfqpoint{4.586605in}{1.384460in}}%
\pgfpathlineto{\pgfqpoint{4.586605in}{0.953032in}}%
\pgfusepath{fill}%
\end{pgfscope}%
\begin{pgfscope}%
\pgfpathrectangle{\pgfqpoint{4.586605in}{0.521603in}}{\pgfqpoint{0.151000in}{3.020000in}} %
\pgfusepath{clip}%
\pgfsetbuttcap%
\pgfsetroundjoin%
\definecolor{currentfill}{rgb}{0.312255,0.312255,0.434442}%
\pgfsetfillcolor{currentfill}%
\pgfsetlinewidth{0.000000pt}%
\definecolor{currentstroke}{rgb}{0.000000,0.000000,0.000000}%
\pgfsetstrokecolor{currentstroke}%
\pgfsetdash{}{0pt}%
\pgfpathmoveto{\pgfqpoint{4.586605in}{1.384460in}}%
\pgfpathlineto{\pgfqpoint{4.737605in}{1.384460in}}%
\pgfpathlineto{\pgfqpoint{4.737605in}{1.815889in}}%
\pgfpathlineto{\pgfqpoint{4.586605in}{1.815889in}}%
\pgfpathlineto{\pgfqpoint{4.586605in}{1.384460in}}%
\pgfusepath{fill}%
\end{pgfscope}%
\begin{pgfscope}%
\pgfpathrectangle{\pgfqpoint{4.586605in}{0.521603in}}{\pgfqpoint{0.151000in}{3.020000in}} %
\pgfusepath{clip}%
\pgfsetbuttcap%
\pgfsetroundjoin%
\definecolor{currentfill}{rgb}{0.439216,0.484130,0.564216}%
\pgfsetfillcolor{currentfill}%
\pgfsetlinewidth{0.000000pt}%
\definecolor{currentstroke}{rgb}{0.000000,0.000000,0.000000}%
\pgfsetstrokecolor{currentstroke}%
\pgfsetdash{}{0pt}%
\pgfpathmoveto{\pgfqpoint{4.586605in}{1.815889in}}%
\pgfpathlineto{\pgfqpoint{4.737605in}{1.815889in}}%
\pgfpathlineto{\pgfqpoint{4.737605in}{2.247318in}}%
\pgfpathlineto{\pgfqpoint{4.586605in}{2.247318in}}%
\pgfpathlineto{\pgfqpoint{4.586605in}{1.815889in}}%
\pgfusepath{fill}%
\end{pgfscope}%
\begin{pgfscope}%
\pgfpathrectangle{\pgfqpoint{4.586605in}{0.521603in}}{\pgfqpoint{0.151000in}{3.020000in}} %
\pgfusepath{clip}%
\pgfsetbuttcap%
\pgfsetroundjoin%
\definecolor{currentfill}{rgb}{0.562745,0.653983,0.687745}%
\pgfsetfillcolor{currentfill}%
\pgfsetlinewidth{0.000000pt}%
\definecolor{currentstroke}{rgb}{0.000000,0.000000,0.000000}%
\pgfsetstrokecolor{currentstroke}%
\pgfsetdash{}{0pt}%
\pgfpathmoveto{\pgfqpoint{4.586605in}{2.247318in}}%
\pgfpathlineto{\pgfqpoint{4.737605in}{2.247318in}}%
\pgfpathlineto{\pgfqpoint{4.737605in}{2.678746in}}%
\pgfpathlineto{\pgfqpoint{4.586605in}{2.678746in}}%
\pgfpathlineto{\pgfqpoint{4.586605in}{2.247318in}}%
\pgfusepath{fill}%
\end{pgfscope}%
\begin{pgfscope}%
\pgfpathrectangle{\pgfqpoint{4.586605in}{0.521603in}}{\pgfqpoint{0.151000in}{3.020000in}} %
\pgfusepath{clip}%
\pgfsetbuttcap%
\pgfsetroundjoin%
\definecolor{currentfill}{rgb}{0.710478,0.814706,0.814706}%
\pgfsetfillcolor{currentfill}%
\pgfsetlinewidth{0.000000pt}%
\definecolor{currentstroke}{rgb}{0.000000,0.000000,0.000000}%
\pgfsetstrokecolor{currentstroke}%
\pgfsetdash{}{0pt}%
\pgfpathmoveto{\pgfqpoint{4.586605in}{2.678746in}}%
\pgfpathlineto{\pgfqpoint{4.737605in}{2.678746in}}%
\pgfpathlineto{\pgfqpoint{4.737605in}{3.110175in}}%
\pgfpathlineto{\pgfqpoint{4.586605in}{3.110175in}}%
\pgfpathlineto{\pgfqpoint{4.586605in}{2.678746in}}%
\pgfusepath{fill}%
\end{pgfscope}%
\begin{pgfscope}%
\pgfpathrectangle{\pgfqpoint{4.586605in}{0.521603in}}{\pgfqpoint{0.151000in}{3.020000in}} %
\pgfusepath{clip}%
\pgfsetbuttcap%
\pgfsetroundjoin%
\definecolor{currentfill}{rgb}{0.903493,0.938235,0.938235}%
\pgfsetfillcolor{currentfill}%
\pgfsetlinewidth{0.000000pt}%
\definecolor{currentstroke}{rgb}{0.000000,0.000000,0.000000}%
\pgfsetstrokecolor{currentstroke}%
\pgfsetdash{}{0pt}%
\pgfpathmoveto{\pgfqpoint{4.586605in}{3.110175in}}%
\pgfpathlineto{\pgfqpoint{4.737605in}{3.110175in}}%
\pgfpathlineto{\pgfqpoint{4.737605in}{3.541603in}}%
\pgfpathlineto{\pgfqpoint{4.586605in}{3.541603in}}%
\pgfpathlineto{\pgfqpoint{4.586605in}{3.110175in}}%
\pgfusepath{fill}%
\end{pgfscope}%
\begin{pgfscope}%
\pgfsetbuttcap%
\pgfsetroundjoin%
\definecolor{currentfill}{rgb}{0.000000,0.000000,0.000000}%
\pgfsetfillcolor{currentfill}%
\pgfsetlinewidth{0.803000pt}%
\definecolor{currentstroke}{rgb}{0.000000,0.000000,0.000000}%
\pgfsetstrokecolor{currentstroke}%
\pgfsetdash{}{0pt}%
\pgfsys@defobject{currentmarker}{\pgfqpoint{0.000000in}{0.000000in}}{\pgfqpoint{0.048611in}{0.000000in}}{%
\pgfpathmoveto{\pgfqpoint{0.000000in}{0.000000in}}%
\pgfpathlineto{\pgfqpoint{0.048611in}{0.000000in}}%
\pgfusepath{stroke,fill}%
}%
\begin{pgfscope}%
\pgfsys@transformshift{4.737605in}{0.521603in}%
\pgfsys@useobject{currentmarker}{}%
\end{pgfscope}%
\end{pgfscope}%
\begin{pgfscope}%
\pgftext[x=4.834827in,y=0.468842in,left,base]{\rmfamily\fontsize{10.000000}{12.000000}\selectfont \(\displaystyle 0\)}%
\end{pgfscope}%
\begin{pgfscope}%
\pgfsetbuttcap%
\pgfsetroundjoin%
\definecolor{currentfill}{rgb}{0.000000,0.000000,0.000000}%
\pgfsetfillcolor{currentfill}%
\pgfsetlinewidth{0.803000pt}%
\definecolor{currentstroke}{rgb}{0.000000,0.000000,0.000000}%
\pgfsetstrokecolor{currentstroke}%
\pgfsetdash{}{0pt}%
\pgfsys@defobject{currentmarker}{\pgfqpoint{0.000000in}{0.000000in}}{\pgfqpoint{0.048611in}{0.000000in}}{%
\pgfpathmoveto{\pgfqpoint{0.000000in}{0.000000in}}%
\pgfpathlineto{\pgfqpoint{0.048611in}{0.000000in}}%
\pgfusepath{stroke,fill}%
}%
\begin{pgfscope}%
\pgfsys@transformshift{4.737605in}{0.953032in}%
\pgfsys@useobject{currentmarker}{}%
\end{pgfscope}%
\end{pgfscope}%
\begin{pgfscope}%
\pgftext[x=4.834827in,y=0.900270in,left,base]{\rmfamily\fontsize{10.000000}{12.000000}\selectfont \(\displaystyle 1\)}%
\end{pgfscope}%
\begin{pgfscope}%
\pgfsetbuttcap%
\pgfsetroundjoin%
\definecolor{currentfill}{rgb}{0.000000,0.000000,0.000000}%
\pgfsetfillcolor{currentfill}%
\pgfsetlinewidth{0.803000pt}%
\definecolor{currentstroke}{rgb}{0.000000,0.000000,0.000000}%
\pgfsetstrokecolor{currentstroke}%
\pgfsetdash{}{0pt}%
\pgfsys@defobject{currentmarker}{\pgfqpoint{0.000000in}{0.000000in}}{\pgfqpoint{0.048611in}{0.000000in}}{%
\pgfpathmoveto{\pgfqpoint{0.000000in}{0.000000in}}%
\pgfpathlineto{\pgfqpoint{0.048611in}{0.000000in}}%
\pgfusepath{stroke,fill}%
}%
\begin{pgfscope}%
\pgfsys@transformshift{4.737605in}{1.384460in}%
\pgfsys@useobject{currentmarker}{}%
\end{pgfscope}%
\end{pgfscope}%
\begin{pgfscope}%
\pgftext[x=4.834827in,y=1.331699in,left,base]{\rmfamily\fontsize{10.000000}{12.000000}\selectfont \(\displaystyle 2\)}%
\end{pgfscope}%
\begin{pgfscope}%
\pgfsetbuttcap%
\pgfsetroundjoin%
\definecolor{currentfill}{rgb}{0.000000,0.000000,0.000000}%
\pgfsetfillcolor{currentfill}%
\pgfsetlinewidth{0.803000pt}%
\definecolor{currentstroke}{rgb}{0.000000,0.000000,0.000000}%
\pgfsetstrokecolor{currentstroke}%
\pgfsetdash{}{0pt}%
\pgfsys@defobject{currentmarker}{\pgfqpoint{0.000000in}{0.000000in}}{\pgfqpoint{0.048611in}{0.000000in}}{%
\pgfpathmoveto{\pgfqpoint{0.000000in}{0.000000in}}%
\pgfpathlineto{\pgfqpoint{0.048611in}{0.000000in}}%
\pgfusepath{stroke,fill}%
}%
\begin{pgfscope}%
\pgfsys@transformshift{4.737605in}{1.815889in}%
\pgfsys@useobject{currentmarker}{}%
\end{pgfscope}%
\end{pgfscope}%
\begin{pgfscope}%
\pgftext[x=4.834827in,y=1.763128in,left,base]{\rmfamily\fontsize{10.000000}{12.000000}\selectfont \(\displaystyle 3\)}%
\end{pgfscope}%
\begin{pgfscope}%
\pgfsetbuttcap%
\pgfsetroundjoin%
\definecolor{currentfill}{rgb}{0.000000,0.000000,0.000000}%
\pgfsetfillcolor{currentfill}%
\pgfsetlinewidth{0.803000pt}%
\definecolor{currentstroke}{rgb}{0.000000,0.000000,0.000000}%
\pgfsetstrokecolor{currentstroke}%
\pgfsetdash{}{0pt}%
\pgfsys@defobject{currentmarker}{\pgfqpoint{0.000000in}{0.000000in}}{\pgfqpoint{0.048611in}{0.000000in}}{%
\pgfpathmoveto{\pgfqpoint{0.000000in}{0.000000in}}%
\pgfpathlineto{\pgfqpoint{0.048611in}{0.000000in}}%
\pgfusepath{stroke,fill}%
}%
\begin{pgfscope}%
\pgfsys@transformshift{4.737605in}{2.247318in}%
\pgfsys@useobject{currentmarker}{}%
\end{pgfscope}%
\end{pgfscope}%
\begin{pgfscope}%
\pgftext[x=4.834827in,y=2.194556in,left,base]{\rmfamily\fontsize{10.000000}{12.000000}\selectfont \(\displaystyle 4\)}%
\end{pgfscope}%
\begin{pgfscope}%
\pgfsetbuttcap%
\pgfsetroundjoin%
\definecolor{currentfill}{rgb}{0.000000,0.000000,0.000000}%
\pgfsetfillcolor{currentfill}%
\pgfsetlinewidth{0.803000pt}%
\definecolor{currentstroke}{rgb}{0.000000,0.000000,0.000000}%
\pgfsetstrokecolor{currentstroke}%
\pgfsetdash{}{0pt}%
\pgfsys@defobject{currentmarker}{\pgfqpoint{0.000000in}{0.000000in}}{\pgfqpoint{0.048611in}{0.000000in}}{%
\pgfpathmoveto{\pgfqpoint{0.000000in}{0.000000in}}%
\pgfpathlineto{\pgfqpoint{0.048611in}{0.000000in}}%
\pgfusepath{stroke,fill}%
}%
\begin{pgfscope}%
\pgfsys@transformshift{4.737605in}{2.678746in}%
\pgfsys@useobject{currentmarker}{}%
\end{pgfscope}%
\end{pgfscope}%
\begin{pgfscope}%
\pgftext[x=4.834827in,y=2.625985in,left,base]{\rmfamily\fontsize{10.000000}{12.000000}\selectfont \(\displaystyle 5\)}%
\end{pgfscope}%
\begin{pgfscope}%
\pgfsetbuttcap%
\pgfsetroundjoin%
\definecolor{currentfill}{rgb}{0.000000,0.000000,0.000000}%
\pgfsetfillcolor{currentfill}%
\pgfsetlinewidth{0.803000pt}%
\definecolor{currentstroke}{rgb}{0.000000,0.000000,0.000000}%
\pgfsetstrokecolor{currentstroke}%
\pgfsetdash{}{0pt}%
\pgfsys@defobject{currentmarker}{\pgfqpoint{0.000000in}{0.000000in}}{\pgfqpoint{0.048611in}{0.000000in}}{%
\pgfpathmoveto{\pgfqpoint{0.000000in}{0.000000in}}%
\pgfpathlineto{\pgfqpoint{0.048611in}{0.000000in}}%
\pgfusepath{stroke,fill}%
}%
\begin{pgfscope}%
\pgfsys@transformshift{4.737605in}{3.110175in}%
\pgfsys@useobject{currentmarker}{}%
\end{pgfscope}%
\end{pgfscope}%
\begin{pgfscope}%
\pgftext[x=4.834827in,y=3.057413in,left,base]{\rmfamily\fontsize{10.000000}{12.000000}\selectfont \(\displaystyle 6\)}%
\end{pgfscope}%
\begin{pgfscope}%
\pgfsetbuttcap%
\pgfsetroundjoin%
\definecolor{currentfill}{rgb}{0.000000,0.000000,0.000000}%
\pgfsetfillcolor{currentfill}%
\pgfsetlinewidth{0.803000pt}%
\definecolor{currentstroke}{rgb}{0.000000,0.000000,0.000000}%
\pgfsetstrokecolor{currentstroke}%
\pgfsetdash{}{0pt}%
\pgfsys@defobject{currentmarker}{\pgfqpoint{0.000000in}{0.000000in}}{\pgfqpoint{0.048611in}{0.000000in}}{%
\pgfpathmoveto{\pgfqpoint{0.000000in}{0.000000in}}%
\pgfpathlineto{\pgfqpoint{0.048611in}{0.000000in}}%
\pgfusepath{stroke,fill}%
}%
\begin{pgfscope}%
\pgfsys@transformshift{4.737605in}{3.541603in}%
\pgfsys@useobject{currentmarker}{}%
\end{pgfscope}%
\end{pgfscope}%
\begin{pgfscope}%
\pgftext[x=4.834827in,y=3.488842in,left,base]{\rmfamily\fontsize{10.000000}{12.000000}\selectfont \(\displaystyle 7\)}%
\end{pgfscope}%
\begin{pgfscope}%
\pgftext[x=4.959827in,y=2.031603in,,top,rotate=90.000000]{\rmfamily\fontsize{10.000000}{12.000000}\selectfont \(\displaystyle q\) (C)}%
\end{pgfscope}%
\begin{pgfscope}%
\pgftext[x=4.737605in,y=3.583270in,right,base]{\rmfamily\fontsize{10.000000}{12.000000}\selectfont \(\displaystyle \times10^{-10}\)}%
\end{pgfscope}%
\begin{pgfscope}%
\pgfsetbuttcap%
\pgfsetmiterjoin%
\pgfsetlinewidth{0.803000pt}%
\definecolor{currentstroke}{rgb}{0.000000,0.000000,0.000000}%
\pgfsetstrokecolor{currentstroke}%
\pgfsetdash{}{0pt}%
\pgfpathmoveto{\pgfqpoint{4.586605in}{0.521603in}}%
\pgfpathlineto{\pgfqpoint{4.586605in}{0.953032in}}%
\pgfpathlineto{\pgfqpoint{4.586605in}{3.110175in}}%
\pgfpathlineto{\pgfqpoint{4.586605in}{3.541603in}}%
\pgfpathlineto{\pgfqpoint{4.737605in}{3.541603in}}%
\pgfpathlineto{\pgfqpoint{4.737605in}{3.110175in}}%
\pgfpathlineto{\pgfqpoint{4.737605in}{0.953032in}}%
\pgfpathlineto{\pgfqpoint{4.737605in}{0.521603in}}%
\pgfpathclose%
\pgfusepath{stroke}%
\end{pgfscope}%
\end{pgfpicture}%
\makeatother%
\endgroup%

    \caption{A simple EMA plot.\label{fig:charge}}
\end{figure}

\begin{figure}[htb]
    \centering
    %% Creator: Matplotlib, PGF backend
%%
%% To include the figure in your LaTeX document, write
%%   \input{<filename>.pgf}
%%
%% Make sure the required packages are loaded in your preamble
%%   \usepackage{pgf}
%%
%% Figures using additional raster images can only be included by \input if
%% they are in the same directory as the main LaTeX file. For loading figures
%% from other directories you can use the `import` package
%%   \usepackage{import}
%% and then include the figures with
%%   \import{<path to file>}{<filename>.pgf}
%%
%% Matplotlib used the following preamble
%%   \usepackage{fontspec}
%%   \setmainfont{DejaVu Serif}
%%   \setsansfont{DejaVu Sans}
%%   \setmonofont{DejaVu Sans Mono}
%%
\begingroup%
\makeatletter%
\begin{pgfpicture}%
\pgfpathrectangle{\pgfpointorigin}{\pgfqpoint{5.430024in}{3.877527in}}%
\pgfusepath{use as bounding box, clip}%
\begin{pgfscope}%
\pgfsetbuttcap%
\pgfsetmiterjoin%
\definecolor{currentfill}{rgb}{1.000000,1.000000,1.000000}%
\pgfsetfillcolor{currentfill}%
\pgfsetlinewidth{0.000000pt}%
\definecolor{currentstroke}{rgb}{1.000000,1.000000,1.000000}%
\pgfsetstrokecolor{currentstroke}%
\pgfsetdash{}{0pt}%
\pgfpathmoveto{\pgfqpoint{0.000000in}{0.000000in}}%
\pgfpathlineto{\pgfqpoint{5.430024in}{0.000000in}}%
\pgfpathlineto{\pgfqpoint{5.430024in}{3.877527in}}%
\pgfpathlineto{\pgfqpoint{0.000000in}{3.877527in}}%
\pgfpathclose%
\pgfusepath{fill}%
\end{pgfscope}%
\begin{pgfscope}%
\pgfsetbuttcap%
\pgfsetmiterjoin%
\definecolor{currentfill}{rgb}{1.000000,1.000000,1.000000}%
\pgfsetfillcolor{currentfill}%
\pgfsetlinewidth{0.000000pt}%
\definecolor{currentstroke}{rgb}{0.000000,0.000000,0.000000}%
\pgfsetstrokecolor{currentstroke}%
\pgfsetstrokeopacity{0.000000}%
\pgfsetdash{}{0pt}%
\pgfpathmoveto{\pgfqpoint{0.578349in}{0.682899in}}%
\pgfpathlineto{\pgfqpoint{4.298349in}{0.682899in}}%
\pgfpathlineto{\pgfqpoint{4.298349in}{3.702899in}}%
\pgfpathlineto{\pgfqpoint{0.578349in}{3.702899in}}%
\pgfpathclose%
\pgfusepath{fill}%
\end{pgfscope}%
\begin{pgfscope}%
\pgfsetbuttcap%
\pgfsetroundjoin%
\definecolor{currentfill}{rgb}{0.000000,0.000000,0.000000}%
\pgfsetfillcolor{currentfill}%
\pgfsetlinewidth{0.803000pt}%
\definecolor{currentstroke}{rgb}{0.000000,0.000000,0.000000}%
\pgfsetstrokecolor{currentstroke}%
\pgfsetdash{}{0pt}%
\pgfsys@defobject{currentmarker}{\pgfqpoint{0.000000in}{-0.048611in}}{\pgfqpoint{0.000000in}{0.000000in}}{%
\pgfpathmoveto{\pgfqpoint{0.000000in}{0.000000in}}%
\pgfpathlineto{\pgfqpoint{0.000000in}{-0.048611in}}%
\pgfusepath{stroke,fill}%
}%
\begin{pgfscope}%
\pgfsys@transformshift{0.703520in}{0.682899in}%
\pgfsys@useobject{currentmarker}{}%
\end{pgfscope}%
\end{pgfscope}%
\begin{pgfscope}%
\pgftext[x=0.703520in,y=0.585677in,,top]{\rmfamily\fontsize{16.000000}{19.200000}\selectfont \(\displaystyle 0.0\)}%
\end{pgfscope}%
\begin{pgfscope}%
\pgfsetbuttcap%
\pgfsetroundjoin%
\definecolor{currentfill}{rgb}{0.000000,0.000000,0.000000}%
\pgfsetfillcolor{currentfill}%
\pgfsetlinewidth{0.803000pt}%
\definecolor{currentstroke}{rgb}{0.000000,0.000000,0.000000}%
\pgfsetstrokecolor{currentstroke}%
\pgfsetdash{}{0pt}%
\pgfsys@defobject{currentmarker}{\pgfqpoint{0.000000in}{-0.048611in}}{\pgfqpoint{0.000000in}{0.000000in}}{%
\pgfpathmoveto{\pgfqpoint{0.000000in}{0.000000in}}%
\pgfpathlineto{\pgfqpoint{0.000000in}{-0.048611in}}%
\pgfusepath{stroke,fill}%
}%
\begin{pgfscope}%
\pgfsys@transformshift{1.581915in}{0.682899in}%
\pgfsys@useobject{currentmarker}{}%
\end{pgfscope}%
\end{pgfscope}%
\begin{pgfscope}%
\pgftext[x=1.581915in,y=0.585677in,,top]{\rmfamily\fontsize{16.000000}{19.200000}\selectfont \(\displaystyle 0.5\)}%
\end{pgfscope}%
\begin{pgfscope}%
\pgfsetbuttcap%
\pgfsetroundjoin%
\definecolor{currentfill}{rgb}{0.000000,0.000000,0.000000}%
\pgfsetfillcolor{currentfill}%
\pgfsetlinewidth{0.803000pt}%
\definecolor{currentstroke}{rgb}{0.000000,0.000000,0.000000}%
\pgfsetstrokecolor{currentstroke}%
\pgfsetdash{}{0pt}%
\pgfsys@defobject{currentmarker}{\pgfqpoint{0.000000in}{-0.048611in}}{\pgfqpoint{0.000000in}{0.000000in}}{%
\pgfpathmoveto{\pgfqpoint{0.000000in}{0.000000in}}%
\pgfpathlineto{\pgfqpoint{0.000000in}{-0.048611in}}%
\pgfusepath{stroke,fill}%
}%
\begin{pgfscope}%
\pgfsys@transformshift{2.460309in}{0.682899in}%
\pgfsys@useobject{currentmarker}{}%
\end{pgfscope}%
\end{pgfscope}%
\begin{pgfscope}%
\pgftext[x=2.460309in,y=0.585677in,,top]{\rmfamily\fontsize{16.000000}{19.200000}\selectfont \(\displaystyle 1.0\)}%
\end{pgfscope}%
\begin{pgfscope}%
\pgfsetbuttcap%
\pgfsetroundjoin%
\definecolor{currentfill}{rgb}{0.000000,0.000000,0.000000}%
\pgfsetfillcolor{currentfill}%
\pgfsetlinewidth{0.803000pt}%
\definecolor{currentstroke}{rgb}{0.000000,0.000000,0.000000}%
\pgfsetstrokecolor{currentstroke}%
\pgfsetdash{}{0pt}%
\pgfsys@defobject{currentmarker}{\pgfqpoint{0.000000in}{-0.048611in}}{\pgfqpoint{0.000000in}{0.000000in}}{%
\pgfpathmoveto{\pgfqpoint{0.000000in}{0.000000in}}%
\pgfpathlineto{\pgfqpoint{0.000000in}{-0.048611in}}%
\pgfusepath{stroke,fill}%
}%
\begin{pgfscope}%
\pgfsys@transformshift{3.338703in}{0.682899in}%
\pgfsys@useobject{currentmarker}{}%
\end{pgfscope}%
\end{pgfscope}%
\begin{pgfscope}%
\pgftext[x=3.338703in,y=0.585677in,,top]{\rmfamily\fontsize{16.000000}{19.200000}\selectfont \(\displaystyle 1.5\)}%
\end{pgfscope}%
\begin{pgfscope}%
\pgfsetbuttcap%
\pgfsetroundjoin%
\definecolor{currentfill}{rgb}{0.000000,0.000000,0.000000}%
\pgfsetfillcolor{currentfill}%
\pgfsetlinewidth{0.803000pt}%
\definecolor{currentstroke}{rgb}{0.000000,0.000000,0.000000}%
\pgfsetstrokecolor{currentstroke}%
\pgfsetdash{}{0pt}%
\pgfsys@defobject{currentmarker}{\pgfqpoint{0.000000in}{-0.048611in}}{\pgfqpoint{0.000000in}{0.000000in}}{%
\pgfpathmoveto{\pgfqpoint{0.000000in}{0.000000in}}%
\pgfpathlineto{\pgfqpoint{0.000000in}{-0.048611in}}%
\pgfusepath{stroke,fill}%
}%
\begin{pgfscope}%
\pgfsys@transformshift{4.217098in}{0.682899in}%
\pgfsys@useobject{currentmarker}{}%
\end{pgfscope}%
\end{pgfscope}%
\begin{pgfscope}%
\pgftext[x=4.217098in,y=0.585677in,,top]{\rmfamily\fontsize{16.000000}{19.200000}\selectfont \(\displaystyle 2.0\)}%
\end{pgfscope}%
\begin{pgfscope}%
\pgftext[x=2.438349in,y=0.315061in,,top]{\rmfamily\fontsize{16.000000}{19.200000}\selectfont \(\displaystyle t\) (s)}%
\end{pgfscope}%
\begin{pgfscope}%
\pgfsetbuttcap%
\pgfsetroundjoin%
\definecolor{currentfill}{rgb}{0.000000,0.000000,0.000000}%
\pgfsetfillcolor{currentfill}%
\pgfsetlinewidth{0.803000pt}%
\definecolor{currentstroke}{rgb}{0.000000,0.000000,0.000000}%
\pgfsetstrokecolor{currentstroke}%
\pgfsetdash{}{0pt}%
\pgfsys@defobject{currentmarker}{\pgfqpoint{-0.048611in}{0.000000in}}{\pgfqpoint{0.000000in}{0.000000in}}{%
\pgfpathmoveto{\pgfqpoint{0.000000in}{0.000000in}}%
\pgfpathlineto{\pgfqpoint{-0.048611in}{0.000000in}}%
\pgfusepath{stroke,fill}%
}%
\begin{pgfscope}%
\pgfsys@transformshift{0.578349in}{0.759278in}%
\pgfsys@useobject{currentmarker}{}%
\end{pgfscope}%
\end{pgfscope}%
\begin{pgfscope}%
\pgftext[x=0.371059in,y=0.674860in,left,base]{\rmfamily\fontsize{16.000000}{19.200000}\selectfont \(\displaystyle 0\)}%
\end{pgfscope}%
\begin{pgfscope}%
\pgfsetbuttcap%
\pgfsetroundjoin%
\definecolor{currentfill}{rgb}{0.000000,0.000000,0.000000}%
\pgfsetfillcolor{currentfill}%
\pgfsetlinewidth{0.803000pt}%
\definecolor{currentstroke}{rgb}{0.000000,0.000000,0.000000}%
\pgfsetstrokecolor{currentstroke}%
\pgfsetdash{}{0pt}%
\pgfsys@defobject{currentmarker}{\pgfqpoint{-0.048611in}{0.000000in}}{\pgfqpoint{0.000000in}{0.000000in}}{%
\pgfpathmoveto{\pgfqpoint{0.000000in}{0.000000in}}%
\pgfpathlineto{\pgfqpoint{-0.048611in}{0.000000in}}%
\pgfusepath{stroke,fill}%
}%
\begin{pgfscope}%
\pgfsys@transformshift{0.578349in}{1.492736in}%
\pgfsys@useobject{currentmarker}{}%
\end{pgfscope}%
\end{pgfscope}%
\begin{pgfscope}%
\pgftext[x=0.371059in,y=1.408317in,left,base]{\rmfamily\fontsize{16.000000}{19.200000}\selectfont \(\displaystyle 2\)}%
\end{pgfscope}%
\begin{pgfscope}%
\pgfsetbuttcap%
\pgfsetroundjoin%
\definecolor{currentfill}{rgb}{0.000000,0.000000,0.000000}%
\pgfsetfillcolor{currentfill}%
\pgfsetlinewidth{0.803000pt}%
\definecolor{currentstroke}{rgb}{0.000000,0.000000,0.000000}%
\pgfsetstrokecolor{currentstroke}%
\pgfsetdash{}{0pt}%
\pgfsys@defobject{currentmarker}{\pgfqpoint{-0.048611in}{0.000000in}}{\pgfqpoint{0.000000in}{0.000000in}}{%
\pgfpathmoveto{\pgfqpoint{0.000000in}{0.000000in}}%
\pgfpathlineto{\pgfqpoint{-0.048611in}{0.000000in}}%
\pgfusepath{stroke,fill}%
}%
\begin{pgfscope}%
\pgfsys@transformshift{0.578349in}{2.226193in}%
\pgfsys@useobject{currentmarker}{}%
\end{pgfscope}%
\end{pgfscope}%
\begin{pgfscope}%
\pgftext[x=0.371059in,y=2.141775in,left,base]{\rmfamily\fontsize{16.000000}{19.200000}\selectfont \(\displaystyle 4\)}%
\end{pgfscope}%
\begin{pgfscope}%
\pgfsetbuttcap%
\pgfsetroundjoin%
\definecolor{currentfill}{rgb}{0.000000,0.000000,0.000000}%
\pgfsetfillcolor{currentfill}%
\pgfsetlinewidth{0.803000pt}%
\definecolor{currentstroke}{rgb}{0.000000,0.000000,0.000000}%
\pgfsetstrokecolor{currentstroke}%
\pgfsetdash{}{0pt}%
\pgfsys@defobject{currentmarker}{\pgfqpoint{-0.048611in}{0.000000in}}{\pgfqpoint{0.000000in}{0.000000in}}{%
\pgfpathmoveto{\pgfqpoint{0.000000in}{0.000000in}}%
\pgfpathlineto{\pgfqpoint{-0.048611in}{0.000000in}}%
\pgfusepath{stroke,fill}%
}%
\begin{pgfscope}%
\pgfsys@transformshift{0.578349in}{2.959651in}%
\pgfsys@useobject{currentmarker}{}%
\end{pgfscope}%
\end{pgfscope}%
\begin{pgfscope}%
\pgftext[x=0.371059in,y=2.875232in,left,base]{\rmfamily\fontsize{16.000000}{19.200000}\selectfont \(\displaystyle 6\)}%
\end{pgfscope}%
\begin{pgfscope}%
\pgfsetbuttcap%
\pgfsetroundjoin%
\definecolor{currentfill}{rgb}{0.000000,0.000000,0.000000}%
\pgfsetfillcolor{currentfill}%
\pgfsetlinewidth{0.803000pt}%
\definecolor{currentstroke}{rgb}{0.000000,0.000000,0.000000}%
\pgfsetstrokecolor{currentstroke}%
\pgfsetdash{}{0pt}%
\pgfsys@defobject{currentmarker}{\pgfqpoint{-0.048611in}{0.000000in}}{\pgfqpoint{0.000000in}{0.000000in}}{%
\pgfpathmoveto{\pgfqpoint{0.000000in}{0.000000in}}%
\pgfpathlineto{\pgfqpoint{-0.048611in}{0.000000in}}%
\pgfusepath{stroke,fill}%
}%
\begin{pgfscope}%
\pgfsys@transformshift{0.578349in}{3.693108in}%
\pgfsys@useobject{currentmarker}{}%
\end{pgfscope}%
\end{pgfscope}%
\begin{pgfscope}%
\pgftext[x=0.371059in,y=3.608690in,left,base]{\rmfamily\fontsize{16.000000}{19.200000}\selectfont \(\displaystyle 8\)}%
\end{pgfscope}%
\begin{pgfscope}%
\pgftext[x=0.315503in,y=2.192899in,,bottom,rotate=90.000000]{\rmfamily\fontsize{16.000000}{19.200000}\selectfont \(\displaystyle y\) (cm)}%
\end{pgfscope}%
\begin{pgfscope}%
\pgfpathrectangle{\pgfqpoint{0.578349in}{0.682899in}}{\pgfqpoint{3.720000in}{3.020000in}}%
\pgfusepath{clip}%
\pgfsetrectcap%
\pgfsetroundjoin%
\pgfsetlinewidth{1.505625pt}%
\definecolor{currentstroke}{rgb}{0.267004,0.004874,0.329415}%
\pgfsetstrokecolor{currentstroke}%
\pgfsetdash{}{0pt}%
\pgfpathmoveto{\pgfqpoint{0.835280in}{1.115285in}}%
\pgfpathlineto{\pgfqpoint{0.849920in}{1.137544in}}%
\pgfpathlineto{\pgfqpoint{0.864559in}{1.159705in}}%
\pgfpathlineto{\pgfqpoint{0.879199in}{1.190646in}}%
\pgfpathlineto{\pgfqpoint{0.893839in}{1.218954in}}%
\pgfpathlineto{\pgfqpoint{0.908479in}{1.243083in}}%
\pgfpathlineto{\pgfqpoint{0.923119in}{1.268237in}}%
\pgfpathlineto{\pgfqpoint{0.937759in}{1.290856in}}%
\pgfpathlineto{\pgfqpoint{0.952399in}{1.317725in}}%
\pgfpathlineto{\pgfqpoint{0.967039in}{1.346420in}}%
\pgfpathlineto{\pgfqpoint{0.981679in}{1.368244in}}%
\pgfpathlineto{\pgfqpoint{0.996319in}{1.392193in}}%
\pgfpathlineto{\pgfqpoint{1.010959in}{1.421446in}}%
\pgfpathlineto{\pgfqpoint{1.025598in}{1.447674in}}%
\pgfpathlineto{\pgfqpoint{1.040238in}{1.470717in}}%
\pgfpathlineto{\pgfqpoint{1.054878in}{1.490649in}}%
\pgfpathlineto{\pgfqpoint{1.069518in}{1.516795in}}%
\pgfpathlineto{\pgfqpoint{1.084158in}{1.543934in}}%
\pgfpathlineto{\pgfqpoint{1.098798in}{1.569350in}}%
\pgfpathlineto{\pgfqpoint{1.113438in}{1.591092in}}%
\pgfpathlineto{\pgfqpoint{1.128078in}{1.615391in}}%
\pgfpathlineto{\pgfqpoint{1.142718in}{1.641668in}}%
\pgfpathlineto{\pgfqpoint{1.157358in}{1.665137in}}%
\pgfpathlineto{\pgfqpoint{1.171997in}{1.686140in}}%
\pgfpathlineto{\pgfqpoint{1.186637in}{1.709834in}}%
\pgfpathlineto{\pgfqpoint{1.201277in}{1.734633in}}%
\pgfpathlineto{\pgfqpoint{1.215917in}{1.759690in}}%
\pgfpathlineto{\pgfqpoint{1.230557in}{1.779620in}}%
\pgfpathlineto{\pgfqpoint{1.245197in}{1.803435in}}%
\pgfpathlineto{\pgfqpoint{1.259837in}{1.828302in}}%
\pgfpathlineto{\pgfqpoint{1.274477in}{1.851896in}}%
\pgfpathlineto{\pgfqpoint{1.289117in}{1.873185in}}%
\pgfpathlineto{\pgfqpoint{1.303757in}{1.894875in}}%
\pgfpathlineto{\pgfqpoint{1.318397in}{1.918401in}}%
\pgfpathlineto{\pgfqpoint{1.333036in}{1.942620in}}%
\pgfpathlineto{\pgfqpoint{1.347676in}{1.962038in}}%
\pgfpathlineto{\pgfqpoint{1.362316in}{1.986096in}}%
\pgfpathlineto{\pgfqpoint{1.376956in}{2.008754in}}%
\pgfpathlineto{\pgfqpoint{1.391596in}{2.032267in}}%
\pgfpathlineto{\pgfqpoint{1.406236in}{2.053794in}}%
\pgfpathlineto{\pgfqpoint{1.420876in}{2.074799in}}%
\pgfpathlineto{\pgfqpoint{1.435516in}{2.094807in}}%
\pgfpathlineto{\pgfqpoint{1.450156in}{2.118870in}}%
\pgfpathlineto{\pgfqpoint{1.464796in}{2.139836in}}%
\pgfpathlineto{\pgfqpoint{1.479435in}{2.163363in}}%
\pgfpathlineto{\pgfqpoint{1.494075in}{2.185735in}}%
\pgfpathlineto{\pgfqpoint{1.508715in}{2.208808in}}%
\pgfpathlineto{\pgfqpoint{1.523355in}{2.230685in}}%
\pgfpathlineto{\pgfqpoint{1.537995in}{2.250134in}}%
\pgfpathlineto{\pgfqpoint{1.552635in}{2.269312in}}%
\pgfpathlineto{\pgfqpoint{1.567275in}{2.292216in}}%
\pgfpathlineto{\pgfqpoint{1.581915in}{2.313982in}}%
\pgfpathlineto{\pgfqpoint{1.596555in}{2.336390in}}%
\pgfpathlineto{\pgfqpoint{1.611195in}{2.358995in}}%
\pgfpathlineto{\pgfqpoint{1.625835in}{2.381377in}}%
\pgfpathlineto{\pgfqpoint{1.640474in}{2.403393in}}%
\pgfpathlineto{\pgfqpoint{1.655114in}{2.422605in}}%
\pgfpathlineto{\pgfqpoint{1.669754in}{2.441282in}}%
\pgfpathlineto{\pgfqpoint{1.684394in}{2.461906in}}%
\pgfpathlineto{\pgfqpoint{1.699034in}{2.484512in}}%
\pgfpathlineto{\pgfqpoint{1.713674in}{2.506671in}}%
\pgfpathlineto{\pgfqpoint{1.728314in}{2.530160in}}%
\pgfpathlineto{\pgfqpoint{1.742954in}{2.551860in}}%
\pgfpathlineto{\pgfqpoint{1.757594in}{2.573580in}}%
\pgfpathlineto{\pgfqpoint{1.772234in}{2.592289in}}%
\pgfpathlineto{\pgfqpoint{1.786874in}{2.610974in}}%
\pgfpathlineto{\pgfqpoint{1.801513in}{2.630618in}}%
\pgfpathlineto{\pgfqpoint{1.816153in}{2.652250in}}%
\pgfpathlineto{\pgfqpoint{1.830793in}{2.675197in}}%
\pgfpathlineto{\pgfqpoint{1.845433in}{2.698928in}}%
\pgfpathlineto{\pgfqpoint{1.860073in}{2.720083in}}%
\pgfpathlineto{\pgfqpoint{1.874713in}{2.741151in}}%
\pgfpathlineto{\pgfqpoint{1.889353in}{2.759934in}}%
\pgfpathlineto{\pgfqpoint{1.903993in}{2.777568in}}%
\pgfpathlineto{\pgfqpoint{1.918633in}{2.796936in}}%
\pgfpathlineto{\pgfqpoint{1.933273in}{2.817393in}}%
\pgfpathlineto{\pgfqpoint{1.947912in}{2.840632in}}%
\pgfpathlineto{\pgfqpoint{1.962552in}{2.863786in}}%
\pgfpathlineto{\pgfqpoint{1.977192in}{2.885562in}}%
\pgfpathlineto{\pgfqpoint{1.991832in}{2.906663in}}%
\pgfpathlineto{\pgfqpoint{2.006472in}{2.924599in}}%
\pgfpathlineto{\pgfqpoint{2.021112in}{2.941886in}}%
\pgfpathlineto{\pgfqpoint{2.035752in}{2.961192in}}%
\pgfpathlineto{\pgfqpoint{2.050392in}{2.981749in}}%
\pgfpathlineto{\pgfqpoint{2.065032in}{3.004117in}}%
\pgfpathlineto{\pgfqpoint{2.079672in}{3.027709in}}%
\pgfpathlineto{\pgfqpoint{2.094312in}{3.048794in}}%
\pgfpathlineto{\pgfqpoint{2.108951in}{3.069204in}}%
\pgfpathlineto{\pgfqpoint{2.123591in}{3.087860in}}%
\pgfpathlineto{\pgfqpoint{2.138231in}{3.104197in}}%
\pgfpathlineto{\pgfqpoint{2.152871in}{3.122989in}}%
\pgfpathlineto{\pgfqpoint{2.167511in}{3.142988in}}%
\pgfpathlineto{\pgfqpoint{2.182151in}{3.165055in}}%
\pgfpathlineto{\pgfqpoint{2.196791in}{3.188984in}}%
\pgfpathlineto{\pgfqpoint{2.211431in}{3.209907in}}%
\pgfpathlineto{\pgfqpoint{2.226071in}{3.229763in}}%
\pgfpathlineto{\pgfqpoint{2.240711in}{3.249327in}}%
\pgfpathlineto{\pgfqpoint{2.255350in}{3.265223in}}%
\pgfpathlineto{\pgfqpoint{2.269990in}{3.283412in}}%
\pgfpathlineto{\pgfqpoint{2.284630in}{3.303364in}}%
\pgfpathlineto{\pgfqpoint{2.299270in}{3.325801in}}%
\pgfpathlineto{\pgfqpoint{2.313910in}{3.348918in}}%
\pgfpathlineto{\pgfqpoint{2.328550in}{3.369231in}}%
\pgfpathlineto{\pgfqpoint{2.343190in}{3.388670in}}%
\pgfpathlineto{\pgfqpoint{2.357830in}{3.408480in}}%
\pgfpathlineto{\pgfqpoint{2.372470in}{3.424781in}}%
\pgfpathlineto{\pgfqpoint{2.387110in}{3.442126in}}%
\pgfpathlineto{\pgfqpoint{2.401750in}{3.462295in}}%
\pgfpathlineto{\pgfqpoint{2.416389in}{3.484889in}}%
\pgfpathlineto{\pgfqpoint{2.431029in}{3.507743in}}%
\pgfpathlineto{\pgfqpoint{2.445669in}{3.526630in}}%
\pgfpathlineto{\pgfqpoint{2.460309in}{3.547823in}}%
\pgfpathlineto{\pgfqpoint{2.474949in}{3.565626in}}%
\pgfusepath{stroke}%
\end{pgfscope}%
\begin{pgfscope}%
\pgfpathrectangle{\pgfqpoint{0.578349in}{0.682899in}}{\pgfqpoint{3.720000in}{3.020000in}}%
\pgfusepath{clip}%
\pgfsetrectcap%
\pgfsetroundjoin%
\pgfsetlinewidth{1.505625pt}%
\definecolor{currentstroke}{rgb}{0.283187,0.125848,0.444960}%
\pgfsetstrokecolor{currentstroke}%
\pgfsetdash{}{0pt}%
\pgfpathmoveto{\pgfqpoint{0.791360in}{0.972820in}}%
\pgfpathlineto{\pgfqpoint{0.820640in}{1.027214in}}%
\pgfpathlineto{\pgfqpoint{0.849920in}{1.070021in}}%
\pgfpathlineto{\pgfqpoint{0.879199in}{1.124777in}}%
\pgfpathlineto{\pgfqpoint{0.923119in}{1.189075in}}%
\pgfpathlineto{\pgfqpoint{0.952399in}{1.241867in}}%
\pgfpathlineto{\pgfqpoint{0.967039in}{1.259788in}}%
\pgfpathlineto{\pgfqpoint{0.981679in}{1.280745in}}%
\pgfpathlineto{\pgfqpoint{1.010959in}{1.331982in}}%
\pgfpathlineto{\pgfqpoint{1.040238in}{1.369836in}}%
\pgfpathlineto{\pgfqpoint{1.054878in}{1.391840in}}%
\pgfpathlineto{\pgfqpoint{1.069518in}{1.417882in}}%
\pgfpathlineto{\pgfqpoint{1.084158in}{1.439486in}}%
\pgfpathlineto{\pgfqpoint{1.098798in}{1.456817in}}%
\pgfpathlineto{\pgfqpoint{1.113438in}{1.477330in}}%
\pgfpathlineto{\pgfqpoint{1.128078in}{1.502190in}}%
\pgfpathlineto{\pgfqpoint{1.142718in}{1.524273in}}%
\pgfpathlineto{\pgfqpoint{1.171997in}{1.560310in}}%
\pgfpathlineto{\pgfqpoint{1.215917in}{1.625159in}}%
\pgfpathlineto{\pgfqpoint{1.230557in}{1.642037in}}%
\pgfpathlineto{\pgfqpoint{1.245197in}{1.662645in}}%
\pgfpathlineto{\pgfqpoint{1.259837in}{1.685347in}}%
\pgfpathlineto{\pgfqpoint{1.274477in}{1.705369in}}%
\pgfpathlineto{\pgfqpoint{1.303757in}{1.740440in}}%
\pgfpathlineto{\pgfqpoint{1.333036in}{1.783206in}}%
\pgfpathlineto{\pgfqpoint{1.362316in}{1.817440in}}%
\pgfpathlineto{\pgfqpoint{1.406236in}{1.877683in}}%
\pgfpathlineto{\pgfqpoint{1.435516in}{1.912257in}}%
\pgfpathlineto{\pgfqpoint{1.464796in}{1.952276in}}%
\pgfpathlineto{\pgfqpoint{1.494075in}{1.985719in}}%
\pgfpathlineto{\pgfqpoint{1.523355in}{2.025702in}}%
\pgfpathlineto{\pgfqpoint{1.567275in}{2.076967in}}%
\pgfpathlineto{\pgfqpoint{1.581915in}{2.097003in}}%
\pgfpathlineto{\pgfqpoint{1.728314in}{2.271255in}}%
\pgfpathlineto{\pgfqpoint{1.742954in}{2.287137in}}%
\pgfpathlineto{\pgfqpoint{1.801513in}{2.356354in}}%
\pgfpathlineto{\pgfqpoint{1.830793in}{2.390718in}}%
\pgfpathlineto{\pgfqpoint{1.845433in}{2.408293in}}%
\pgfpathlineto{\pgfqpoint{1.874713in}{2.439913in}}%
\pgfpathlineto{\pgfqpoint{1.918633in}{2.491614in}}%
\pgfpathlineto{\pgfqpoint{1.947912in}{2.523691in}}%
\pgfpathlineto{\pgfqpoint{1.977192in}{2.557771in}}%
\pgfpathlineto{\pgfqpoint{2.006472in}{2.589279in}}%
\pgfpathlineto{\pgfqpoint{2.035752in}{2.623732in}}%
\pgfpathlineto{\pgfqpoint{2.079672in}{2.671590in}}%
\pgfpathlineto{\pgfqpoint{2.108951in}{2.704229in}}%
\pgfpathlineto{\pgfqpoint{2.138231in}{2.735830in}}%
\pgfpathlineto{\pgfqpoint{2.167511in}{2.769128in}}%
\pgfpathlineto{\pgfqpoint{2.196791in}{2.800159in}}%
\pgfpathlineto{\pgfqpoint{2.240711in}{2.848220in}}%
\pgfpathlineto{\pgfqpoint{2.269990in}{2.879752in}}%
\pgfpathlineto{\pgfqpoint{2.299270in}{2.911896in}}%
\pgfpathlineto{\pgfqpoint{2.343190in}{2.959035in}}%
\pgfpathlineto{\pgfqpoint{2.431029in}{3.052580in}}%
\pgfpathlineto{\pgfqpoint{2.811667in}{3.449251in}}%
\pgfpathlineto{\pgfqpoint{2.826307in}{3.463954in}}%
\pgfpathlineto{\pgfqpoint{2.826307in}{3.463954in}}%
\pgfusepath{stroke}%
\end{pgfscope}%
\begin{pgfscope}%
\pgfpathrectangle{\pgfqpoint{0.578349in}{0.682899in}}{\pgfqpoint{3.720000in}{3.020000in}}%
\pgfusepath{clip}%
\pgfsetrectcap%
\pgfsetroundjoin%
\pgfsetlinewidth{1.505625pt}%
\definecolor{currentstroke}{rgb}{0.192357,0.403199,0.555836}%
\pgfsetstrokecolor{currentstroke}%
\pgfsetdash{}{0pt}%
\pgfpathmoveto{\pgfqpoint{0.864559in}{1.095273in}}%
\pgfpathlineto{\pgfqpoint{0.879199in}{1.102314in}}%
\pgfpathlineto{\pgfqpoint{0.893839in}{1.134639in}}%
\pgfpathlineto{\pgfqpoint{0.908479in}{1.161664in}}%
\pgfpathlineto{\pgfqpoint{0.923119in}{1.177296in}}%
\pgfpathlineto{\pgfqpoint{0.937759in}{1.197119in}}%
\pgfpathlineto{\pgfqpoint{0.952399in}{1.221408in}}%
\pgfpathlineto{\pgfqpoint{0.967039in}{1.249372in}}%
\pgfpathlineto{\pgfqpoint{0.981679in}{1.268597in}}%
\pgfpathlineto{\pgfqpoint{0.996319in}{1.284784in}}%
\pgfpathlineto{\pgfqpoint{1.010959in}{1.306630in}}%
\pgfpathlineto{\pgfqpoint{1.025598in}{1.332337in}}%
\pgfpathlineto{\pgfqpoint{1.040238in}{1.351802in}}%
\pgfpathlineto{\pgfqpoint{1.069518in}{1.385242in}}%
\pgfpathlineto{\pgfqpoint{1.084158in}{1.406322in}}%
\pgfpathlineto{\pgfqpoint{1.098798in}{1.430591in}}%
\pgfpathlineto{\pgfqpoint{1.113438in}{1.448401in}}%
\pgfpathlineto{\pgfqpoint{1.128078in}{1.461329in}}%
\pgfpathlineto{\pgfqpoint{1.142718in}{1.481095in}}%
\pgfpathlineto{\pgfqpoint{1.157358in}{1.503667in}}%
\pgfpathlineto{\pgfqpoint{1.171997in}{1.521452in}}%
\pgfpathlineto{\pgfqpoint{1.201277in}{1.550396in}}%
\pgfpathlineto{\pgfqpoint{1.230557in}{1.590500in}}%
\pgfpathlineto{\pgfqpoint{1.245197in}{1.606425in}}%
\pgfpathlineto{\pgfqpoint{1.259837in}{1.618037in}}%
\pgfpathlineto{\pgfqpoint{1.303757in}{1.672193in}}%
\pgfpathlineto{\pgfqpoint{1.333036in}{1.697973in}}%
\pgfpathlineto{\pgfqpoint{1.347676in}{1.714904in}}%
\pgfpathlineto{\pgfqpoint{1.362316in}{1.734343in}}%
\pgfpathlineto{\pgfqpoint{1.376956in}{1.747134in}}%
\pgfpathlineto{\pgfqpoint{1.391596in}{1.758434in}}%
\pgfpathlineto{\pgfqpoint{1.406236in}{1.774315in}}%
\pgfpathlineto{\pgfqpoint{1.420876in}{1.792316in}}%
\pgfpathlineto{\pgfqpoint{1.435516in}{1.806904in}}%
\pgfpathlineto{\pgfqpoint{1.464796in}{1.830028in}}%
\pgfpathlineto{\pgfqpoint{1.479435in}{1.845684in}}%
\pgfpathlineto{\pgfqpoint{1.494075in}{1.863064in}}%
\pgfpathlineto{\pgfqpoint{1.508715in}{1.874786in}}%
\pgfpathlineto{\pgfqpoint{1.523355in}{1.884483in}}%
\pgfpathlineto{\pgfqpoint{1.567275in}{1.929019in}}%
\pgfpathlineto{\pgfqpoint{1.596555in}{1.950001in}}%
\pgfpathlineto{\pgfqpoint{1.611195in}{1.964139in}}%
\pgfpathlineto{\pgfqpoint{1.625835in}{1.980070in}}%
\pgfpathlineto{\pgfqpoint{1.655114in}{1.999888in}}%
\pgfpathlineto{\pgfqpoint{1.699034in}{2.040283in}}%
\pgfpathlineto{\pgfqpoint{1.728314in}{2.059525in}}%
\pgfpathlineto{\pgfqpoint{1.742954in}{2.072528in}}%
\pgfpathlineto{\pgfqpoint{1.757594in}{2.087192in}}%
\pgfpathlineto{\pgfqpoint{1.786874in}{2.105388in}}%
\pgfpathlineto{\pgfqpoint{1.801513in}{2.117442in}}%
\pgfpathlineto{\pgfqpoint{1.816153in}{2.131022in}}%
\pgfpathlineto{\pgfqpoint{1.830793in}{2.142358in}}%
\pgfpathlineto{\pgfqpoint{1.860073in}{2.160162in}}%
\pgfpathlineto{\pgfqpoint{1.889353in}{2.185625in}}%
\pgfpathlineto{\pgfqpoint{1.918633in}{2.202592in}}%
\pgfpathlineto{\pgfqpoint{1.962552in}{2.236774in}}%
\pgfpathlineto{\pgfqpoint{1.991832in}{2.253186in}}%
\pgfpathlineto{\pgfqpoint{2.021112in}{2.276686in}}%
\pgfpathlineto{\pgfqpoint{2.050392in}{2.292404in}}%
\pgfpathlineto{\pgfqpoint{2.094312in}{2.323918in}}%
\pgfpathlineto{\pgfqpoint{2.123591in}{2.339094in}}%
\pgfpathlineto{\pgfqpoint{2.152871in}{2.360656in}}%
\pgfpathlineto{\pgfqpoint{2.182151in}{2.375453in}}%
\pgfpathlineto{\pgfqpoint{2.226071in}{2.404597in}}%
\pgfpathlineto{\pgfqpoint{2.240711in}{2.411018in}}%
\pgfpathlineto{\pgfqpoint{2.255350in}{2.418978in}}%
\pgfpathlineto{\pgfqpoint{2.269990in}{2.429445in}}%
\pgfpathlineto{\pgfqpoint{2.284630in}{2.438469in}}%
\pgfpathlineto{\pgfqpoint{2.313910in}{2.452597in}}%
\pgfpathlineto{\pgfqpoint{2.357830in}{2.479273in}}%
\pgfpathlineto{\pgfqpoint{2.372470in}{2.485348in}}%
\pgfpathlineto{\pgfqpoint{2.387110in}{2.493184in}}%
\pgfpathlineto{\pgfqpoint{2.416389in}{2.510836in}}%
\pgfpathlineto{\pgfqpoint{2.445669in}{2.524169in}}%
\pgfpathlineto{\pgfqpoint{2.489589in}{2.548636in}}%
\pgfpathlineto{\pgfqpoint{2.504229in}{2.554421in}}%
\pgfpathlineto{\pgfqpoint{2.518869in}{2.561682in}}%
\pgfpathlineto{\pgfqpoint{2.548149in}{2.578274in}}%
\pgfpathlineto{\pgfqpoint{2.577428in}{2.590626in}}%
\pgfpathlineto{\pgfqpoint{2.621348in}{2.613485in}}%
\pgfpathlineto{\pgfqpoint{2.635988in}{2.618933in}}%
\pgfpathlineto{\pgfqpoint{2.650628in}{2.625700in}}%
\pgfpathlineto{\pgfqpoint{2.665268in}{2.634176in}}%
\pgfpathlineto{\pgfqpoint{2.679908in}{2.641287in}}%
\pgfpathlineto{\pgfqpoint{2.709188in}{2.653008in}}%
\pgfpathlineto{\pgfqpoint{2.738467in}{2.668120in}}%
\pgfpathlineto{\pgfqpoint{2.782387in}{2.685850in}}%
\pgfpathlineto{\pgfqpoint{2.797027in}{2.693879in}}%
\pgfpathlineto{\pgfqpoint{2.811667in}{2.700450in}}%
\pgfpathlineto{\pgfqpoint{2.840947in}{2.711309in}}%
\pgfpathlineto{\pgfqpoint{2.870227in}{2.725535in}}%
\pgfpathlineto{\pgfqpoint{2.914146in}{2.742144in}}%
\pgfpathlineto{\pgfqpoint{2.928786in}{2.749683in}}%
\pgfpathlineto{\pgfqpoint{2.958066in}{2.760422in}}%
\pgfpathlineto{\pgfqpoint{2.972706in}{2.765719in}}%
\pgfpathlineto{\pgfqpoint{3.001986in}{2.778930in}}%
\pgfpathlineto{\pgfqpoint{3.031265in}{2.788310in}}%
\pgfpathlineto{\pgfqpoint{3.075185in}{2.806864in}}%
\pgfpathlineto{\pgfqpoint{3.104465in}{2.816436in}}%
\pgfpathlineto{\pgfqpoint{3.133745in}{2.828704in}}%
\pgfpathlineto{\pgfqpoint{3.163025in}{2.837488in}}%
\pgfpathlineto{\pgfqpoint{3.192304in}{2.850198in}}%
\pgfpathlineto{\pgfqpoint{3.250864in}{2.869739in}}%
\pgfpathlineto{\pgfqpoint{3.265504in}{2.875275in}}%
\pgfpathlineto{\pgfqpoint{3.294784in}{2.883773in}}%
\pgfpathlineto{\pgfqpoint{3.324064in}{2.895432in}}%
\pgfpathlineto{\pgfqpoint{3.382623in}{2.913850in}}%
\pgfpathlineto{\pgfqpoint{3.397263in}{2.918736in}}%
\pgfpathlineto{\pgfqpoint{3.426543in}{2.926894in}}%
\pgfpathlineto{\pgfqpoint{3.455823in}{2.937766in}}%
\pgfpathlineto{\pgfqpoint{3.499742in}{2.949809in}}%
\pgfpathlineto{\pgfqpoint{3.529022in}{2.959240in}}%
\pgfpathlineto{\pgfqpoint{3.558302in}{2.967058in}}%
\pgfpathlineto{\pgfqpoint{3.587582in}{2.977004in}}%
\pgfpathlineto{\pgfqpoint{3.675421in}{3.000114in}}%
\pgfpathlineto{\pgfqpoint{3.733981in}{3.016874in}}%
\pgfpathlineto{\pgfqpoint{3.763261in}{3.024067in}}%
\pgfpathlineto{\pgfqpoint{3.792541in}{3.031951in}}%
\pgfpathlineto{\pgfqpoint{3.821820in}{3.039315in}}%
\pgfpathlineto{\pgfqpoint{3.851100in}{3.047408in}}%
\pgfpathlineto{\pgfqpoint{3.880380in}{3.053348in}}%
\pgfpathlineto{\pgfqpoint{3.909660in}{3.061385in}}%
\pgfpathlineto{\pgfqpoint{3.953580in}{3.071361in}}%
\pgfpathlineto{\pgfqpoint{3.982859in}{3.078860in}}%
\pgfpathlineto{\pgfqpoint{4.026779in}{3.088256in}}%
\pgfpathlineto{\pgfqpoint{4.114618in}{3.108012in}}%
\pgfpathlineto{\pgfqpoint{4.129258in}{3.110225in}}%
\pgfpathlineto{\pgfqpoint{4.129258in}{3.110225in}}%
\pgfusepath{stroke}%
\end{pgfscope}%
\begin{pgfscope}%
\pgfpathrectangle{\pgfqpoint{0.578349in}{0.682899in}}{\pgfqpoint{3.720000in}{3.020000in}}%
\pgfusepath{clip}%
\pgfsetrectcap%
\pgfsetroundjoin%
\pgfsetlinewidth{1.505625pt}%
\definecolor{currentstroke}{rgb}{0.182256,0.426184,0.557120}%
\pgfsetstrokecolor{currentstroke}%
\pgfsetdash{}{0pt}%
\pgfpathmoveto{\pgfqpoint{0.820640in}{1.013153in}}%
\pgfpathlineto{\pgfqpoint{0.835280in}{1.025591in}}%
\pgfpathlineto{\pgfqpoint{0.849920in}{1.044354in}}%
\pgfpathlineto{\pgfqpoint{0.864559in}{1.079472in}}%
\pgfpathlineto{\pgfqpoint{0.879199in}{1.099440in}}%
\pgfpathlineto{\pgfqpoint{0.893839in}{1.107457in}}%
\pgfpathlineto{\pgfqpoint{0.923119in}{1.162096in}}%
\pgfpathlineto{\pgfqpoint{0.937759in}{1.178576in}}%
\pgfpathlineto{\pgfqpoint{0.952399in}{1.192666in}}%
\pgfpathlineto{\pgfqpoint{0.981679in}{1.236941in}}%
\pgfpathlineto{\pgfqpoint{1.025598in}{1.286165in}}%
\pgfpathlineto{\pgfqpoint{1.040238in}{1.307228in}}%
\pgfpathlineto{\pgfqpoint{1.054878in}{1.322604in}}%
\pgfpathlineto{\pgfqpoint{1.069518in}{1.336195in}}%
\pgfpathlineto{\pgfqpoint{1.098798in}{1.372888in}}%
\pgfpathlineto{\pgfqpoint{1.128078in}{1.400029in}}%
\pgfpathlineto{\pgfqpoint{1.157358in}{1.434274in}}%
\pgfpathlineto{\pgfqpoint{1.186637in}{1.459903in}}%
\pgfpathlineto{\pgfqpoint{1.201277in}{1.476527in}}%
\pgfpathlineto{\pgfqpoint{1.215917in}{1.491395in}}%
\pgfpathlineto{\pgfqpoint{1.245197in}{1.515217in}}%
\pgfpathlineto{\pgfqpoint{1.274477in}{1.544763in}}%
\pgfpathlineto{\pgfqpoint{1.289117in}{1.554629in}}%
\pgfpathlineto{\pgfqpoint{1.303757in}{1.567725in}}%
\pgfpathlineto{\pgfqpoint{1.318397in}{1.582316in}}%
\pgfpathlineto{\pgfqpoint{1.333036in}{1.594322in}}%
\pgfpathlineto{\pgfqpoint{1.347676in}{1.604269in}}%
\pgfpathlineto{\pgfqpoint{1.362316in}{1.616003in}}%
\pgfpathlineto{\pgfqpoint{1.376956in}{1.630041in}}%
\pgfpathlineto{\pgfqpoint{1.406236in}{1.649787in}}%
\pgfpathlineto{\pgfqpoint{1.435516in}{1.674277in}}%
\pgfpathlineto{\pgfqpoint{1.479435in}{1.704307in}}%
\pgfpathlineto{\pgfqpoint{1.494075in}{1.715727in}}%
\pgfpathlineto{\pgfqpoint{1.508715in}{1.723715in}}%
\pgfpathlineto{\pgfqpoint{1.523355in}{1.733143in}}%
\pgfpathlineto{\pgfqpoint{1.552635in}{1.753968in}}%
\pgfpathlineto{\pgfqpoint{1.581915in}{1.770256in}}%
\pgfpathlineto{\pgfqpoint{1.596555in}{1.780886in}}%
\pgfpathlineto{\pgfqpoint{1.611195in}{1.789737in}}%
\pgfpathlineto{\pgfqpoint{1.625835in}{1.796672in}}%
\pgfpathlineto{\pgfqpoint{1.669754in}{1.822995in}}%
\pgfpathlineto{\pgfqpoint{1.684394in}{1.829600in}}%
\pgfpathlineto{\pgfqpoint{1.728314in}{1.853429in}}%
\pgfpathlineto{\pgfqpoint{1.742954in}{1.859937in}}%
\pgfpathlineto{\pgfqpoint{1.772234in}{1.875857in}}%
\pgfpathlineto{\pgfqpoint{1.801513in}{1.887945in}}%
\pgfpathlineto{\pgfqpoint{1.830793in}{1.902721in}}%
\pgfpathlineto{\pgfqpoint{1.860073in}{1.913871in}}%
\pgfpathlineto{\pgfqpoint{1.889353in}{1.927212in}}%
\pgfpathlineto{\pgfqpoint{1.918633in}{1.937615in}}%
\pgfpathlineto{\pgfqpoint{1.933273in}{1.944211in}}%
\pgfpathlineto{\pgfqpoint{2.006472in}{1.969957in}}%
\pgfpathlineto{\pgfqpoint{2.035752in}{1.978907in}}%
\pgfpathlineto{\pgfqpoint{2.050392in}{1.984250in}}%
\pgfpathlineto{\pgfqpoint{2.094312in}{1.996452in}}%
\pgfpathlineto{\pgfqpoint{2.123591in}{2.004515in}}%
\pgfpathlineto{\pgfqpoint{2.138231in}{2.007704in}}%
\pgfpathlineto{\pgfqpoint{2.167511in}{2.015813in}}%
\pgfpathlineto{\pgfqpoint{2.211431in}{2.025596in}}%
\pgfpathlineto{\pgfqpoint{2.240711in}{2.031132in}}%
\pgfpathlineto{\pgfqpoint{2.328550in}{2.046887in}}%
\pgfpathlineto{\pgfqpoint{2.387110in}{2.054864in}}%
\pgfpathlineto{\pgfqpoint{2.518869in}{2.065301in}}%
\pgfpathlineto{\pgfqpoint{2.577428in}{2.066954in}}%
\pgfpathlineto{\pgfqpoint{2.635988in}{2.067234in}}%
\pgfpathlineto{\pgfqpoint{2.679908in}{2.066138in}}%
\pgfpathlineto{\pgfqpoint{2.782387in}{2.059742in}}%
\pgfpathlineto{\pgfqpoint{2.899506in}{2.045125in}}%
\pgfpathlineto{\pgfqpoint{3.045905in}{2.015772in}}%
\pgfpathlineto{\pgfqpoint{3.075185in}{2.008303in}}%
\pgfpathlineto{\pgfqpoint{3.163025in}{1.983073in}}%
\pgfpathlineto{\pgfqpoint{3.221584in}{1.963297in}}%
\pgfpathlineto{\pgfqpoint{3.294784in}{1.935065in}}%
\pgfpathlineto{\pgfqpoint{3.382623in}{1.896173in}}%
\pgfpathlineto{\pgfqpoint{3.441183in}{1.866988in}}%
\pgfpathlineto{\pgfqpoint{3.499742in}{1.835100in}}%
\pgfpathlineto{\pgfqpoint{3.572942in}{1.791180in}}%
\pgfpathlineto{\pgfqpoint{3.646142in}{1.742905in}}%
\pgfpathlineto{\pgfqpoint{3.690061in}{1.711731in}}%
\pgfpathlineto{\pgfqpoint{3.733981in}{1.679043in}}%
\pgfpathlineto{\pgfqpoint{3.792541in}{1.632559in}}%
\pgfpathlineto{\pgfqpoint{3.836460in}{1.595490in}}%
\pgfpathlineto{\pgfqpoint{3.895020in}{1.543517in}}%
\pgfpathlineto{\pgfqpoint{3.953580in}{1.487721in}}%
\pgfpathlineto{\pgfqpoint{4.012139in}{1.428330in}}%
\pgfpathlineto{\pgfqpoint{4.070699in}{1.364802in}}%
\pgfpathlineto{\pgfqpoint{4.114618in}{1.314294in}}%
\pgfpathlineto{\pgfqpoint{4.114618in}{1.314294in}}%
\pgfusepath{stroke}%
\end{pgfscope}%
\begin{pgfscope}%
\pgfpathrectangle{\pgfqpoint{0.578349in}{0.682899in}}{\pgfqpoint{3.720000in}{3.020000in}}%
\pgfusepath{clip}%
\pgfsetrectcap%
\pgfsetroundjoin%
\pgfsetlinewidth{1.505625pt}%
\definecolor{currentstroke}{rgb}{0.122312,0.633153,0.530398}%
\pgfsetstrokecolor{currentstroke}%
\pgfsetdash{}{0pt}%
\pgfpathmoveto{\pgfqpoint{0.747440in}{0.885286in}}%
\pgfpathlineto{\pgfqpoint{0.762080in}{0.930922in}}%
\pgfpathlineto{\pgfqpoint{0.776720in}{0.960287in}}%
\pgfpathlineto{\pgfqpoint{0.791360in}{0.979516in}}%
\pgfpathlineto{\pgfqpoint{0.806000in}{1.002903in}}%
\pgfpathlineto{\pgfqpoint{0.820640in}{1.038450in}}%
\pgfpathlineto{\pgfqpoint{0.835280in}{1.068519in}}%
\pgfpathlineto{\pgfqpoint{0.864559in}{1.107425in}}%
\pgfpathlineto{\pgfqpoint{0.879199in}{1.134308in}}%
\pgfpathlineto{\pgfqpoint{0.893839in}{1.152477in}}%
\pgfpathlineto{\pgfqpoint{0.908479in}{1.176495in}}%
\pgfpathlineto{\pgfqpoint{0.923119in}{1.193513in}}%
\pgfpathlineto{\pgfqpoint{0.937759in}{1.207168in}}%
\pgfpathlineto{\pgfqpoint{0.952399in}{1.226173in}}%
\pgfpathlineto{\pgfqpoint{0.967039in}{1.250299in}}%
\pgfpathlineto{\pgfqpoint{0.981679in}{1.269861in}}%
\pgfpathlineto{\pgfqpoint{1.010959in}{1.296542in}}%
\pgfpathlineto{\pgfqpoint{1.040238in}{1.335652in}}%
\pgfpathlineto{\pgfqpoint{1.054878in}{1.348278in}}%
\pgfpathlineto{\pgfqpoint{1.069518in}{1.358063in}}%
\pgfpathlineto{\pgfqpoint{1.084158in}{1.369337in}}%
\pgfpathlineto{\pgfqpoint{1.113438in}{1.403738in}}%
\pgfpathlineto{\pgfqpoint{1.128078in}{1.413707in}}%
\pgfpathlineto{\pgfqpoint{1.142718in}{1.420781in}}%
\pgfpathlineto{\pgfqpoint{1.157358in}{1.433085in}}%
\pgfpathlineto{\pgfqpoint{1.171997in}{1.448428in}}%
\pgfpathlineto{\pgfqpoint{1.186637in}{1.459812in}}%
\pgfpathlineto{\pgfqpoint{1.215917in}{1.472196in}}%
\pgfpathlineto{\pgfqpoint{1.230557in}{1.482965in}}%
\pgfpathlineto{\pgfqpoint{1.245197in}{1.495977in}}%
\pgfpathlineto{\pgfqpoint{1.259837in}{1.504993in}}%
\pgfpathlineto{\pgfqpoint{1.274477in}{1.507857in}}%
\pgfpathlineto{\pgfqpoint{1.289117in}{1.514736in}}%
\pgfpathlineto{\pgfqpoint{1.303757in}{1.525523in}}%
\pgfpathlineto{\pgfqpoint{1.318397in}{1.534164in}}%
\pgfpathlineto{\pgfqpoint{1.333036in}{1.538567in}}%
\pgfpathlineto{\pgfqpoint{1.347676in}{1.540616in}}%
\pgfpathlineto{\pgfqpoint{1.362316in}{1.545710in}}%
\pgfpathlineto{\pgfqpoint{1.391596in}{1.560561in}}%
\pgfpathlineto{\pgfqpoint{1.406236in}{1.561027in}}%
\pgfpathlineto{\pgfqpoint{1.420876in}{1.562802in}}%
\pgfpathlineto{\pgfqpoint{1.450156in}{1.574368in}}%
\pgfpathlineto{\pgfqpoint{1.464796in}{1.576600in}}%
\pgfpathlineto{\pgfqpoint{1.479435in}{1.575508in}}%
\pgfpathlineto{\pgfqpoint{1.494075in}{1.575799in}}%
\pgfpathlineto{\pgfqpoint{1.523355in}{1.583437in}}%
\pgfpathlineto{\pgfqpoint{1.552635in}{1.579613in}}%
\pgfpathlineto{\pgfqpoint{1.567275in}{1.580524in}}%
\pgfpathlineto{\pgfqpoint{1.581915in}{1.582849in}}%
\pgfpathlineto{\pgfqpoint{1.596555in}{1.582729in}}%
\pgfpathlineto{\pgfqpoint{1.625835in}{1.575370in}}%
\pgfpathlineto{\pgfqpoint{1.640474in}{1.574310in}}%
\pgfpathlineto{\pgfqpoint{1.655114in}{1.575010in}}%
\pgfpathlineto{\pgfqpoint{1.669754in}{1.571310in}}%
\pgfpathlineto{\pgfqpoint{1.684394in}{1.565221in}}%
\pgfpathlineto{\pgfqpoint{1.699034in}{1.561665in}}%
\pgfpathlineto{\pgfqpoint{1.713674in}{1.560270in}}%
\pgfpathlineto{\pgfqpoint{1.728314in}{1.557256in}}%
\pgfpathlineto{\pgfqpoint{1.772234in}{1.537944in}}%
\pgfpathlineto{\pgfqpoint{1.786874in}{1.534703in}}%
\pgfpathlineto{\pgfqpoint{1.801513in}{1.528533in}}%
\pgfpathlineto{\pgfqpoint{1.816153in}{1.519297in}}%
\pgfpathlineto{\pgfqpoint{1.830793in}{1.511492in}}%
\pgfpathlineto{\pgfqpoint{1.860073in}{1.499386in}}%
\pgfpathlineto{\pgfqpoint{1.874713in}{1.490252in}}%
\pgfpathlineto{\pgfqpoint{1.903993in}{1.469262in}}%
\pgfpathlineto{\pgfqpoint{1.918633in}{1.461937in}}%
\pgfpathlineto{\pgfqpoint{1.933273in}{1.452660in}}%
\pgfpathlineto{\pgfqpoint{1.962552in}{1.428354in}}%
\pgfpathlineto{\pgfqpoint{2.006472in}{1.395219in}}%
\pgfpathlineto{\pgfqpoint{2.035752in}{1.365825in}}%
\pgfpathlineto{\pgfqpoint{2.065032in}{1.340215in}}%
\pgfpathlineto{\pgfqpoint{2.138231in}{1.260456in}}%
\pgfpathlineto{\pgfqpoint{2.211431in}{1.164830in}}%
\pgfpathlineto{\pgfqpoint{2.269990in}{1.077765in}}%
\pgfpathlineto{\pgfqpoint{2.328550in}{0.977311in}}%
\pgfpathlineto{\pgfqpoint{2.357830in}{0.920057in}}%
\pgfpathlineto{\pgfqpoint{2.372470in}{0.901048in}}%
\pgfpathlineto{\pgfqpoint{2.401750in}{0.867626in}}%
\pgfpathlineto{\pgfqpoint{2.416389in}{0.861357in}}%
\pgfpathlineto{\pgfqpoint{2.431029in}{0.865882in}}%
\pgfpathlineto{\pgfqpoint{2.445669in}{0.875010in}}%
\pgfpathlineto{\pgfqpoint{2.460309in}{0.891184in}}%
\pgfpathlineto{\pgfqpoint{2.474949in}{0.913407in}}%
\pgfpathlineto{\pgfqpoint{2.489589in}{0.932109in}}%
\pgfpathlineto{\pgfqpoint{2.504229in}{0.961756in}}%
\pgfpathlineto{\pgfqpoint{2.533509in}{1.006132in}}%
\pgfpathlineto{\pgfqpoint{2.548149in}{1.023253in}}%
\pgfpathlineto{\pgfqpoint{2.562789in}{1.038491in}}%
\pgfpathlineto{\pgfqpoint{2.592068in}{1.073175in}}%
\pgfpathlineto{\pgfqpoint{2.606708in}{1.088056in}}%
\pgfpathlineto{\pgfqpoint{2.650628in}{1.127772in}}%
\pgfpathlineto{\pgfqpoint{2.665268in}{1.141186in}}%
\pgfpathlineto{\pgfqpoint{2.694548in}{1.160151in}}%
\pgfpathlineto{\pgfqpoint{2.709188in}{1.169327in}}%
\pgfpathlineto{\pgfqpoint{2.723827in}{1.180059in}}%
\pgfpathlineto{\pgfqpoint{2.738467in}{1.188204in}}%
\pgfpathlineto{\pgfqpoint{2.782387in}{1.208296in}}%
\pgfpathlineto{\pgfqpoint{2.797027in}{1.215372in}}%
\pgfpathlineto{\pgfqpoint{2.811667in}{1.219831in}}%
\pgfpathlineto{\pgfqpoint{2.840947in}{1.225647in}}%
\pgfpathlineto{\pgfqpoint{2.855587in}{1.230130in}}%
\pgfpathlineto{\pgfqpoint{2.870227in}{1.233240in}}%
\pgfpathlineto{\pgfqpoint{2.914146in}{1.235497in}}%
\pgfpathlineto{\pgfqpoint{2.928786in}{1.236427in}}%
\pgfpathlineto{\pgfqpoint{2.943426in}{1.236181in}}%
\pgfpathlineto{\pgfqpoint{2.972706in}{1.230691in}}%
\pgfpathlineto{\pgfqpoint{3.001986in}{1.226646in}}%
\pgfpathlineto{\pgfqpoint{3.075185in}{1.200766in}}%
\pgfpathlineto{\pgfqpoint{3.133745in}{1.167525in}}%
\pgfpathlineto{\pgfqpoint{3.163025in}{1.145294in}}%
\pgfpathlineto{\pgfqpoint{3.206944in}{1.109811in}}%
\pgfpathlineto{\pgfqpoint{3.236224in}{1.079470in}}%
\pgfpathlineto{\pgfqpoint{3.265504in}{1.047415in}}%
\pgfpathlineto{\pgfqpoint{3.294784in}{1.009569in}}%
\pgfpathlineto{\pgfqpoint{3.324064in}{0.967418in}}%
\pgfpathlineto{\pgfqpoint{3.353343in}{0.920045in}}%
\pgfpathlineto{\pgfqpoint{3.367983in}{0.898519in}}%
\pgfpathlineto{\pgfqpoint{3.382623in}{0.881122in}}%
\pgfpathlineto{\pgfqpoint{3.397263in}{0.867419in}}%
\pgfpathlineto{\pgfqpoint{3.411903in}{0.866623in}}%
\pgfpathlineto{\pgfqpoint{3.426543in}{0.874639in}}%
\pgfpathlineto{\pgfqpoint{3.441183in}{0.891217in}}%
\pgfpathlineto{\pgfqpoint{3.470463in}{0.932664in}}%
\pgfpathlineto{\pgfqpoint{3.485103in}{0.955013in}}%
\pgfpathlineto{\pgfqpoint{3.499742in}{0.973795in}}%
\pgfpathlineto{\pgfqpoint{3.514382in}{0.996816in}}%
\pgfpathlineto{\pgfqpoint{3.529022in}{1.015552in}}%
\pgfpathlineto{\pgfqpoint{3.558302in}{1.045727in}}%
\pgfpathlineto{\pgfqpoint{3.572942in}{1.057120in}}%
\pgfpathlineto{\pgfqpoint{3.602222in}{1.082517in}}%
\pgfpathlineto{\pgfqpoint{3.616862in}{1.093687in}}%
\pgfpathlineto{\pgfqpoint{3.675421in}{1.127367in}}%
\pgfpathlineto{\pgfqpoint{3.690061in}{1.132792in}}%
\pgfpathlineto{\pgfqpoint{3.719341in}{1.141157in}}%
\pgfpathlineto{\pgfqpoint{3.733981in}{1.146606in}}%
\pgfpathlineto{\pgfqpoint{3.748621in}{1.150046in}}%
\pgfpathlineto{\pgfqpoint{3.807180in}{1.155058in}}%
\pgfpathlineto{\pgfqpoint{3.821820in}{1.153897in}}%
\pgfpathlineto{\pgfqpoint{3.851100in}{1.148525in}}%
\pgfpathlineto{\pgfqpoint{3.865740in}{1.147281in}}%
\pgfpathlineto{\pgfqpoint{3.880380in}{1.144097in}}%
\pgfpathlineto{\pgfqpoint{3.895020in}{1.139369in}}%
\pgfpathlineto{\pgfqpoint{3.953580in}{1.114173in}}%
\pgfpathlineto{\pgfqpoint{3.997499in}{1.085697in}}%
\pgfpathlineto{\pgfqpoint{4.012139in}{1.075614in}}%
\pgfpathlineto{\pgfqpoint{4.026779in}{1.063707in}}%
\pgfpathlineto{\pgfqpoint{4.070699in}{1.021112in}}%
\pgfpathlineto{\pgfqpoint{4.085339in}{1.005721in}}%
\pgfpathlineto{\pgfqpoint{4.085339in}{1.005721in}}%
\pgfusepath{stroke}%
\end{pgfscope}%
\begin{pgfscope}%
\pgfpathrectangle{\pgfqpoint{0.578349in}{0.682899in}}{\pgfqpoint{3.720000in}{3.020000in}}%
\pgfusepath{clip}%
\pgfsetrectcap%
\pgfsetroundjoin%
\pgfsetlinewidth{1.505625pt}%
\definecolor{currentstroke}{rgb}{0.288921,0.758394,0.428426}%
\pgfsetstrokecolor{currentstroke}%
\pgfsetdash{}{0pt}%
\pgfpathmoveto{\pgfqpoint{0.835280in}{0.996766in}}%
\pgfpathlineto{\pgfqpoint{0.849920in}{1.008786in}}%
\pgfpathlineto{\pgfqpoint{0.879199in}{1.050432in}}%
\pgfpathlineto{\pgfqpoint{0.893839in}{1.066529in}}%
\pgfpathlineto{\pgfqpoint{0.908479in}{1.077595in}}%
\pgfpathlineto{\pgfqpoint{0.937759in}{1.112443in}}%
\pgfpathlineto{\pgfqpoint{0.967039in}{1.133272in}}%
\pgfpathlineto{\pgfqpoint{0.981679in}{1.148526in}}%
\pgfpathlineto{\pgfqpoint{0.996319in}{1.161379in}}%
\pgfpathlineto{\pgfqpoint{1.010959in}{1.167947in}}%
\pgfpathlineto{\pgfqpoint{1.025598in}{1.175833in}}%
\pgfpathlineto{\pgfqpoint{1.040238in}{1.188283in}}%
\pgfpathlineto{\pgfqpoint{1.054878in}{1.196702in}}%
\pgfpathlineto{\pgfqpoint{1.069518in}{1.200408in}}%
\pgfpathlineto{\pgfqpoint{1.098798in}{1.217618in}}%
\pgfpathlineto{\pgfqpoint{1.113438in}{1.221501in}}%
\pgfpathlineto{\pgfqpoint{1.128078in}{1.222623in}}%
\pgfpathlineto{\pgfqpoint{1.157358in}{1.234665in}}%
\pgfpathlineto{\pgfqpoint{1.171997in}{1.234451in}}%
\pgfpathlineto{\pgfqpoint{1.186637in}{1.235434in}}%
\pgfpathlineto{\pgfqpoint{1.201277in}{1.239862in}}%
\pgfpathlineto{\pgfqpoint{1.215917in}{1.241613in}}%
\pgfpathlineto{\pgfqpoint{1.230557in}{1.238326in}}%
\pgfpathlineto{\pgfqpoint{1.245197in}{1.237349in}}%
\pgfpathlineto{\pgfqpoint{1.259837in}{1.239174in}}%
\pgfpathlineto{\pgfqpoint{1.274477in}{1.236427in}}%
\pgfpathlineto{\pgfqpoint{1.289117in}{1.231217in}}%
\pgfpathlineto{\pgfqpoint{1.318397in}{1.228214in}}%
\pgfpathlineto{\pgfqpoint{1.347676in}{1.213542in}}%
\pgfpathlineto{\pgfqpoint{1.362316in}{1.209750in}}%
\pgfpathlineto{\pgfqpoint{1.376956in}{1.204334in}}%
\pgfpathlineto{\pgfqpoint{1.391596in}{1.193868in}}%
\pgfpathlineto{\pgfqpoint{1.406236in}{1.185676in}}%
\pgfpathlineto{\pgfqpoint{1.420876in}{1.179687in}}%
\pgfpathlineto{\pgfqpoint{1.435516in}{1.169630in}}%
\pgfpathlineto{\pgfqpoint{1.450156in}{1.156057in}}%
\pgfpathlineto{\pgfqpoint{1.479435in}{1.135824in}}%
\pgfpathlineto{\pgfqpoint{1.508715in}{1.105945in}}%
\pgfpathlineto{\pgfqpoint{1.523355in}{1.093518in}}%
\pgfpathlineto{\pgfqpoint{1.537995in}{1.079251in}}%
\pgfpathlineto{\pgfqpoint{1.567275in}{1.041752in}}%
\pgfpathlineto{\pgfqpoint{1.581915in}{1.025367in}}%
\pgfpathlineto{\pgfqpoint{1.596555in}{1.005757in}}%
\pgfpathlineto{\pgfqpoint{1.640474in}{0.936933in}}%
\pgfpathlineto{\pgfqpoint{1.655114in}{0.906548in}}%
\pgfpathlineto{\pgfqpoint{1.669754in}{0.880651in}}%
\pgfpathlineto{\pgfqpoint{1.684394in}{0.859309in}}%
\pgfpathlineto{\pgfqpoint{1.699034in}{0.840254in}}%
\pgfpathlineto{\pgfqpoint{1.713674in}{0.840391in}}%
\pgfpathlineto{\pgfqpoint{1.728314in}{0.847262in}}%
\pgfpathlineto{\pgfqpoint{1.742954in}{0.869654in}}%
\pgfpathlineto{\pgfqpoint{1.757594in}{0.889089in}}%
\pgfpathlineto{\pgfqpoint{1.772234in}{0.910876in}}%
\pgfpathlineto{\pgfqpoint{1.801513in}{0.962037in}}%
\pgfpathlineto{\pgfqpoint{1.816153in}{0.981890in}}%
\pgfpathlineto{\pgfqpoint{1.830793in}{0.997869in}}%
\pgfpathlineto{\pgfqpoint{1.874713in}{1.038852in}}%
\pgfpathlineto{\pgfqpoint{1.903993in}{1.059239in}}%
\pgfpathlineto{\pgfqpoint{1.933273in}{1.075399in}}%
\pgfpathlineto{\pgfqpoint{1.962552in}{1.086377in}}%
\pgfpathlineto{\pgfqpoint{1.977192in}{1.091450in}}%
\pgfpathlineto{\pgfqpoint{2.006472in}{1.096059in}}%
\pgfpathlineto{\pgfqpoint{2.035752in}{1.097451in}}%
\pgfpathlineto{\pgfqpoint{2.050392in}{1.096222in}}%
\pgfpathlineto{\pgfqpoint{2.094312in}{1.087127in}}%
\pgfpathlineto{\pgfqpoint{2.123591in}{1.075244in}}%
\pgfpathlineto{\pgfqpoint{2.138231in}{1.067934in}}%
\pgfpathlineto{\pgfqpoint{2.152871in}{1.059328in}}%
\pgfpathlineto{\pgfqpoint{2.182151in}{1.038630in}}%
\pgfpathlineto{\pgfqpoint{2.196791in}{1.027070in}}%
\pgfpathlineto{\pgfqpoint{2.226071in}{0.998144in}}%
\pgfpathlineto{\pgfqpoint{2.255350in}{0.964161in}}%
\pgfpathlineto{\pgfqpoint{2.269990in}{0.943716in}}%
\pgfpathlineto{\pgfqpoint{2.299270in}{0.893650in}}%
\pgfpathlineto{\pgfqpoint{2.313910in}{0.877155in}}%
\pgfpathlineto{\pgfqpoint{2.328550in}{0.864862in}}%
\pgfpathlineto{\pgfqpoint{2.343190in}{0.850397in}}%
\pgfpathlineto{\pgfqpoint{2.357830in}{0.852101in}}%
\pgfpathlineto{\pgfqpoint{2.372470in}{0.859488in}}%
\pgfpathlineto{\pgfqpoint{2.387110in}{0.873975in}}%
\pgfpathlineto{\pgfqpoint{2.401750in}{0.890910in}}%
\pgfpathlineto{\pgfqpoint{2.416389in}{0.914542in}}%
\pgfpathlineto{\pgfqpoint{2.431029in}{0.931926in}}%
\pgfpathlineto{\pgfqpoint{2.445669in}{0.947202in}}%
\pgfpathlineto{\pgfqpoint{2.460309in}{0.957503in}}%
\pgfpathlineto{\pgfqpoint{2.474949in}{0.972997in}}%
\pgfpathlineto{\pgfqpoint{2.489589in}{0.982672in}}%
\pgfpathlineto{\pgfqpoint{2.533509in}{0.998626in}}%
\pgfpathlineto{\pgfqpoint{2.548149in}{1.002524in}}%
\pgfpathlineto{\pgfqpoint{2.577428in}{1.000181in}}%
\pgfpathlineto{\pgfqpoint{2.592068in}{1.000128in}}%
\pgfpathlineto{\pgfqpoint{2.606708in}{0.997843in}}%
\pgfpathlineto{\pgfqpoint{2.621348in}{0.991156in}}%
\pgfpathlineto{\pgfqpoint{2.650628in}{0.980889in}}%
\pgfpathlineto{\pgfqpoint{2.665268in}{0.970700in}}%
\pgfpathlineto{\pgfqpoint{2.679908in}{0.954011in}}%
\pgfpathlineto{\pgfqpoint{2.694548in}{0.943031in}}%
\pgfpathlineto{\pgfqpoint{2.709188in}{0.929860in}}%
\pgfpathlineto{\pgfqpoint{2.723827in}{0.911930in}}%
\pgfpathlineto{\pgfqpoint{2.738467in}{0.890989in}}%
\pgfpathlineto{\pgfqpoint{2.753107in}{0.878438in}}%
\pgfpathlineto{\pgfqpoint{2.767747in}{0.870683in}}%
\pgfpathlineto{\pgfqpoint{2.782387in}{0.859038in}}%
\pgfpathlineto{\pgfqpoint{2.797027in}{0.859836in}}%
\pgfpathlineto{\pgfqpoint{2.811667in}{0.865712in}}%
\pgfpathlineto{\pgfqpoint{2.840947in}{0.890233in}}%
\pgfpathlineto{\pgfqpoint{2.855587in}{0.903775in}}%
\pgfpathlineto{\pgfqpoint{2.870227in}{0.920598in}}%
\pgfpathlineto{\pgfqpoint{2.884866in}{0.934342in}}%
\pgfpathlineto{\pgfqpoint{2.914146in}{0.945572in}}%
\pgfpathlineto{\pgfqpoint{2.928786in}{0.954672in}}%
\pgfpathlineto{\pgfqpoint{2.943426in}{0.958048in}}%
\pgfpathlineto{\pgfqpoint{2.958066in}{0.954406in}}%
\pgfpathlineto{\pgfqpoint{2.972706in}{0.953644in}}%
\pgfpathlineto{\pgfqpoint{2.987346in}{0.954437in}}%
\pgfpathlineto{\pgfqpoint{3.001986in}{0.943321in}}%
\pgfpathlineto{\pgfqpoint{3.016626in}{0.935451in}}%
\pgfpathlineto{\pgfqpoint{3.031265in}{0.924344in}}%
\pgfpathlineto{\pgfqpoint{3.075185in}{0.880190in}}%
\pgfpathlineto{\pgfqpoint{3.089825in}{0.871336in}}%
\pgfpathlineto{\pgfqpoint{3.104465in}{0.860965in}}%
\pgfpathlineto{\pgfqpoint{3.119105in}{0.860462in}}%
\pgfpathlineto{\pgfqpoint{3.133745in}{0.864675in}}%
\pgfpathlineto{\pgfqpoint{3.148385in}{0.875045in}}%
\pgfpathlineto{\pgfqpoint{3.163025in}{0.890014in}}%
\pgfpathlineto{\pgfqpoint{3.177665in}{0.899921in}}%
\pgfpathlineto{\pgfqpoint{3.192304in}{0.916571in}}%
\pgfpathlineto{\pgfqpoint{3.206944in}{0.931476in}}%
\pgfpathlineto{\pgfqpoint{3.236224in}{0.945830in}}%
\pgfpathlineto{\pgfqpoint{3.265504in}{0.956895in}}%
\pgfpathlineto{\pgfqpoint{3.294784in}{0.956179in}}%
\pgfpathlineto{\pgfqpoint{3.309424in}{0.955043in}}%
\pgfpathlineto{\pgfqpoint{3.338703in}{0.945378in}}%
\pgfpathlineto{\pgfqpoint{3.367983in}{0.929832in}}%
\pgfpathlineto{\pgfqpoint{3.382623in}{0.911705in}}%
\pgfpathlineto{\pgfqpoint{3.411903in}{0.889044in}}%
\pgfpathlineto{\pgfqpoint{3.441183in}{0.866513in}}%
\pgfpathlineto{\pgfqpoint{3.455823in}{0.863488in}}%
\pgfpathlineto{\pgfqpoint{3.470463in}{0.865572in}}%
\pgfpathlineto{\pgfqpoint{3.485103in}{0.873140in}}%
\pgfpathlineto{\pgfqpoint{3.499742in}{0.882758in}}%
\pgfpathlineto{\pgfqpoint{3.543662in}{0.915416in}}%
\pgfpathlineto{\pgfqpoint{3.572942in}{0.931867in}}%
\pgfpathlineto{\pgfqpoint{3.587582in}{0.938659in}}%
\pgfpathlineto{\pgfqpoint{3.602222in}{0.939354in}}%
\pgfpathlineto{\pgfqpoint{3.631502in}{0.937959in}}%
\pgfpathlineto{\pgfqpoint{3.646142in}{0.933975in}}%
\pgfpathlineto{\pgfqpoint{3.675421in}{0.914842in}}%
\pgfpathlineto{\pgfqpoint{3.704701in}{0.895514in}}%
\pgfpathlineto{\pgfqpoint{3.719341in}{0.883770in}}%
\pgfpathlineto{\pgfqpoint{3.748621in}{0.869665in}}%
\pgfpathlineto{\pgfqpoint{3.763261in}{0.867311in}}%
\pgfpathlineto{\pgfqpoint{3.777901in}{0.870637in}}%
\pgfpathlineto{\pgfqpoint{3.792541in}{0.878785in}}%
\pgfpathlineto{\pgfqpoint{3.821820in}{0.898094in}}%
\pgfpathlineto{\pgfqpoint{3.851100in}{0.916846in}}%
\pgfpathlineto{\pgfqpoint{3.865740in}{0.923112in}}%
\pgfpathlineto{\pgfqpoint{3.880380in}{0.934863in}}%
\pgfpathlineto{\pgfqpoint{3.895020in}{0.938862in}}%
\pgfpathlineto{\pgfqpoint{3.924300in}{0.941293in}}%
\pgfpathlineto{\pgfqpoint{3.938940in}{0.939936in}}%
\pgfpathlineto{\pgfqpoint{3.968219in}{0.933413in}}%
\pgfpathlineto{\pgfqpoint{3.982859in}{0.921977in}}%
\pgfpathlineto{\pgfqpoint{3.997499in}{0.915132in}}%
\pgfpathlineto{\pgfqpoint{4.026779in}{0.896594in}}%
\pgfpathlineto{\pgfqpoint{4.056059in}{0.878979in}}%
\pgfpathlineto{\pgfqpoint{4.070699in}{0.875032in}}%
\pgfpathlineto{\pgfqpoint{4.085339in}{0.872895in}}%
\pgfpathlineto{\pgfqpoint{4.099979in}{0.876288in}}%
\pgfpathlineto{\pgfqpoint{4.114618in}{0.882637in}}%
\pgfpathlineto{\pgfqpoint{4.114618in}{0.882637in}}%
\pgfusepath{stroke}%
\end{pgfscope}%
\begin{pgfscope}%
\pgfpathrectangle{\pgfqpoint{0.578349in}{0.682899in}}{\pgfqpoint{3.720000in}{3.020000in}}%
\pgfusepath{clip}%
\pgfsetrectcap%
\pgfsetroundjoin%
\pgfsetlinewidth{1.505625pt}%
\definecolor{currentstroke}{rgb}{0.751884,0.874951,0.143228}%
\pgfsetstrokecolor{currentstroke}%
\pgfsetdash{}{0pt}%
\pgfpathmoveto{\pgfqpoint{0.820640in}{0.955271in}}%
\pgfpathlineto{\pgfqpoint{0.835280in}{0.971972in}}%
\pgfpathlineto{\pgfqpoint{0.849920in}{0.985214in}}%
\pgfpathlineto{\pgfqpoint{0.864559in}{0.993810in}}%
\pgfpathlineto{\pgfqpoint{0.893839in}{1.017637in}}%
\pgfpathlineto{\pgfqpoint{0.908479in}{1.023148in}}%
\pgfpathlineto{\pgfqpoint{0.923119in}{1.032583in}}%
\pgfpathlineto{\pgfqpoint{0.937759in}{1.039129in}}%
\pgfpathlineto{\pgfqpoint{0.952399in}{1.042151in}}%
\pgfpathlineto{\pgfqpoint{0.967039in}{1.049168in}}%
\pgfpathlineto{\pgfqpoint{0.981679in}{1.051439in}}%
\pgfpathlineto{\pgfqpoint{0.996319in}{1.050906in}}%
\pgfpathlineto{\pgfqpoint{1.010959in}{1.055950in}}%
\pgfpathlineto{\pgfqpoint{1.025598in}{1.055131in}}%
\pgfpathlineto{\pgfqpoint{1.040238in}{1.052205in}}%
\pgfpathlineto{\pgfqpoint{1.054878in}{1.053635in}}%
\pgfpathlineto{\pgfqpoint{1.084158in}{1.044488in}}%
\pgfpathlineto{\pgfqpoint{1.098798in}{1.042096in}}%
\pgfpathlineto{\pgfqpoint{1.128078in}{1.026627in}}%
\pgfpathlineto{\pgfqpoint{1.142718in}{1.021322in}}%
\pgfpathlineto{\pgfqpoint{1.157358in}{1.009968in}}%
\pgfpathlineto{\pgfqpoint{1.186637in}{0.991347in}}%
\pgfpathlineto{\pgfqpoint{1.215917in}{0.962830in}}%
\pgfpathlineto{\pgfqpoint{1.230557in}{0.950203in}}%
\pgfpathlineto{\pgfqpoint{1.274477in}{0.894011in}}%
\pgfpathlineto{\pgfqpoint{1.289117in}{0.872114in}}%
\pgfpathlineto{\pgfqpoint{1.303757in}{0.857078in}}%
\pgfpathlineto{\pgfqpoint{1.318397in}{0.839215in}}%
\pgfpathlineto{\pgfqpoint{1.333036in}{0.839368in}}%
\pgfpathlineto{\pgfqpoint{1.347676in}{0.845951in}}%
\pgfpathlineto{\pgfqpoint{1.362316in}{0.862524in}}%
\pgfpathlineto{\pgfqpoint{1.376956in}{0.881872in}}%
\pgfpathlineto{\pgfqpoint{1.391596in}{0.898622in}}%
\pgfpathlineto{\pgfqpoint{1.435516in}{0.934445in}}%
\pgfpathlineto{\pgfqpoint{1.450156in}{0.936839in}}%
\pgfpathlineto{\pgfqpoint{1.464796in}{0.947472in}}%
\pgfpathlineto{\pgfqpoint{1.479435in}{0.952417in}}%
\pgfpathlineto{\pgfqpoint{1.494075in}{0.950737in}}%
\pgfpathlineto{\pgfqpoint{1.508715in}{0.955389in}}%
\pgfpathlineto{\pgfqpoint{1.523355in}{0.957401in}}%
\pgfpathlineto{\pgfqpoint{1.537995in}{0.951910in}}%
\pgfpathlineto{\pgfqpoint{1.552635in}{0.953210in}}%
\pgfpathlineto{\pgfqpoint{1.567275in}{0.948072in}}%
\pgfpathlineto{\pgfqpoint{1.581915in}{0.940068in}}%
\pgfpathlineto{\pgfqpoint{1.596555in}{0.936752in}}%
\pgfpathlineto{\pgfqpoint{1.611195in}{0.926723in}}%
\pgfpathlineto{\pgfqpoint{1.625835in}{0.913802in}}%
\pgfpathlineto{\pgfqpoint{1.640474in}{0.904404in}}%
\pgfpathlineto{\pgfqpoint{1.669754in}{0.872004in}}%
\pgfpathlineto{\pgfqpoint{1.684394in}{0.857998in}}%
\pgfpathlineto{\pgfqpoint{1.699034in}{0.845635in}}%
\pgfpathlineto{\pgfqpoint{1.713674in}{0.844708in}}%
\pgfpathlineto{\pgfqpoint{1.728314in}{0.848759in}}%
\pgfpathlineto{\pgfqpoint{1.757594in}{0.874322in}}%
\pgfpathlineto{\pgfqpoint{1.772234in}{0.890006in}}%
\pgfpathlineto{\pgfqpoint{1.786874in}{0.899112in}}%
\pgfpathlineto{\pgfqpoint{1.801513in}{0.910695in}}%
\pgfpathlineto{\pgfqpoint{1.816153in}{0.914978in}}%
\pgfpathlineto{\pgfqpoint{1.830793in}{0.915289in}}%
\pgfpathlineto{\pgfqpoint{1.845433in}{0.918116in}}%
\pgfpathlineto{\pgfqpoint{1.860073in}{0.915901in}}%
\pgfpathlineto{\pgfqpoint{1.889353in}{0.901854in}}%
\pgfpathlineto{\pgfqpoint{1.903993in}{0.891011in}}%
\pgfpathlineto{\pgfqpoint{1.918633in}{0.875047in}}%
\pgfpathlineto{\pgfqpoint{1.933273in}{0.865328in}}%
\pgfpathlineto{\pgfqpoint{1.947912in}{0.852120in}}%
\pgfpathlineto{\pgfqpoint{1.962552in}{0.847370in}}%
\pgfpathlineto{\pgfqpoint{1.977192in}{0.849202in}}%
\pgfpathlineto{\pgfqpoint{2.006472in}{0.868820in}}%
\pgfpathlineto{\pgfqpoint{2.021112in}{0.884538in}}%
\pgfpathlineto{\pgfqpoint{2.035752in}{0.894928in}}%
\pgfpathlineto{\pgfqpoint{2.050392in}{0.901483in}}%
\pgfpathlineto{\pgfqpoint{2.065032in}{0.906486in}}%
\pgfpathlineto{\pgfqpoint{2.079672in}{0.907775in}}%
\pgfpathlineto{\pgfqpoint{2.108951in}{0.901955in}}%
\pgfpathlineto{\pgfqpoint{2.123591in}{0.895837in}}%
\pgfpathlineto{\pgfqpoint{2.138231in}{0.884851in}}%
\pgfpathlineto{\pgfqpoint{2.152871in}{0.877035in}}%
\pgfpathlineto{\pgfqpoint{2.167511in}{0.863305in}}%
\pgfpathlineto{\pgfqpoint{2.182151in}{0.856478in}}%
\pgfpathlineto{\pgfqpoint{2.196791in}{0.852366in}}%
\pgfpathlineto{\pgfqpoint{2.211431in}{0.855912in}}%
\pgfpathlineto{\pgfqpoint{2.226071in}{0.862171in}}%
\pgfpathlineto{\pgfqpoint{2.240711in}{0.874483in}}%
\pgfpathlineto{\pgfqpoint{2.269990in}{0.892073in}}%
\pgfpathlineto{\pgfqpoint{2.284630in}{0.897797in}}%
\pgfpathlineto{\pgfqpoint{2.299270in}{0.900326in}}%
\pgfpathlineto{\pgfqpoint{2.313910in}{0.898821in}}%
\pgfpathlineto{\pgfqpoint{2.328550in}{0.893168in}}%
\pgfpathlineto{\pgfqpoint{2.357830in}{0.874636in}}%
\pgfpathlineto{\pgfqpoint{2.372470in}{0.864471in}}%
\pgfpathlineto{\pgfqpoint{2.387110in}{0.856753in}}%
\pgfpathlineto{\pgfqpoint{2.401750in}{0.854264in}}%
\pgfpathlineto{\pgfqpoint{2.416389in}{0.857236in}}%
\pgfpathlineto{\pgfqpoint{2.474949in}{0.893766in}}%
\pgfpathlineto{\pgfqpoint{2.489589in}{0.898402in}}%
\pgfpathlineto{\pgfqpoint{2.504229in}{0.900213in}}%
\pgfpathlineto{\pgfqpoint{2.518869in}{0.897508in}}%
\pgfpathlineto{\pgfqpoint{2.533509in}{0.891649in}}%
\pgfpathlineto{\pgfqpoint{2.577428in}{0.864535in}}%
\pgfpathlineto{\pgfqpoint{2.592068in}{0.858511in}}%
\pgfpathlineto{\pgfqpoint{2.606708in}{0.858491in}}%
\pgfpathlineto{\pgfqpoint{2.621348in}{0.864013in}}%
\pgfpathlineto{\pgfqpoint{2.665268in}{0.891794in}}%
\pgfpathlineto{\pgfqpoint{2.679908in}{0.897854in}}%
\pgfpathlineto{\pgfqpoint{2.694548in}{0.900606in}}%
\pgfpathlineto{\pgfqpoint{2.709188in}{0.899542in}}%
\pgfpathlineto{\pgfqpoint{2.723827in}{0.894854in}}%
\pgfpathlineto{\pgfqpoint{2.753107in}{0.879101in}}%
\pgfpathlineto{\pgfqpoint{2.767747in}{0.870485in}}%
\pgfpathlineto{\pgfqpoint{2.782387in}{0.863614in}}%
\pgfpathlineto{\pgfqpoint{2.797027in}{0.862482in}}%
\pgfpathlineto{\pgfqpoint{2.811667in}{0.865548in}}%
\pgfpathlineto{\pgfqpoint{2.826307in}{0.872811in}}%
\pgfpathlineto{\pgfqpoint{2.855587in}{0.890508in}}%
\pgfpathlineto{\pgfqpoint{2.870227in}{0.897672in}}%
\pgfpathlineto{\pgfqpoint{2.884866in}{0.901038in}}%
\pgfpathlineto{\pgfqpoint{2.899506in}{0.902148in}}%
\pgfpathlineto{\pgfqpoint{2.914146in}{0.899480in}}%
\pgfpathlineto{\pgfqpoint{2.943426in}{0.886519in}}%
\pgfpathlineto{\pgfqpoint{2.972706in}{0.870853in}}%
\pgfpathlineto{\pgfqpoint{2.987346in}{0.866293in}}%
\pgfpathlineto{\pgfqpoint{3.001986in}{0.866093in}}%
\pgfpathlineto{\pgfqpoint{3.016626in}{0.870353in}}%
\pgfpathlineto{\pgfqpoint{3.060545in}{0.893635in}}%
\pgfpathlineto{\pgfqpoint{3.075185in}{0.899520in}}%
\pgfpathlineto{\pgfqpoint{3.089825in}{0.902055in}}%
\pgfpathlineto{\pgfqpoint{3.104465in}{0.901651in}}%
\pgfpathlineto{\pgfqpoint{3.119105in}{0.898215in}}%
\pgfpathlineto{\pgfqpoint{3.177665in}{0.871245in}}%
\pgfpathlineto{\pgfqpoint{3.192304in}{0.869369in}}%
\pgfpathlineto{\pgfqpoint{3.206944in}{0.870651in}}%
\pgfpathlineto{\pgfqpoint{3.221584in}{0.876601in}}%
\pgfpathlineto{\pgfqpoint{3.250864in}{0.890954in}}%
\pgfpathlineto{\pgfqpoint{3.265504in}{0.898000in}}%
\pgfpathlineto{\pgfqpoint{3.280144in}{0.902986in}}%
\pgfpathlineto{\pgfqpoint{3.294784in}{0.903527in}}%
\pgfpathlineto{\pgfqpoint{3.309424in}{0.902354in}}%
\pgfpathlineto{\pgfqpoint{3.324064in}{0.897714in}}%
\pgfpathlineto{\pgfqpoint{3.382623in}{0.872931in}}%
\pgfpathlineto{\pgfqpoint{3.397263in}{0.871766in}}%
\pgfpathlineto{\pgfqpoint{3.411903in}{0.873956in}}%
\pgfpathlineto{\pgfqpoint{3.426543in}{0.878640in}}%
\pgfpathlineto{\pgfqpoint{3.470463in}{0.896342in}}%
\pgfpathlineto{\pgfqpoint{3.485103in}{0.899949in}}%
\pgfpathlineto{\pgfqpoint{3.499742in}{0.900769in}}%
\pgfpathlineto{\pgfqpoint{3.514382in}{0.898075in}}%
\pgfpathlineto{\pgfqpoint{3.529022in}{0.893988in}}%
\pgfpathlineto{\pgfqpoint{3.572942in}{0.877456in}}%
\pgfpathlineto{\pgfqpoint{3.587582in}{0.874504in}}%
\pgfpathlineto{\pgfqpoint{3.602222in}{0.874255in}}%
\pgfpathlineto{\pgfqpoint{3.616862in}{0.876880in}}%
\pgfpathlineto{\pgfqpoint{3.631502in}{0.881027in}}%
\pgfpathlineto{\pgfqpoint{3.660781in}{0.891807in}}%
\pgfpathlineto{\pgfqpoint{3.675421in}{0.895114in}}%
\pgfpathlineto{\pgfqpoint{3.690061in}{0.896723in}}%
\pgfpathlineto{\pgfqpoint{3.704701in}{0.896242in}}%
\pgfpathlineto{\pgfqpoint{3.719341in}{0.893279in}}%
\pgfpathlineto{\pgfqpoint{3.763261in}{0.879172in}}%
\pgfpathlineto{\pgfqpoint{3.777901in}{0.876355in}}%
\pgfpathlineto{\pgfqpoint{3.792541in}{0.875646in}}%
\pgfpathlineto{\pgfqpoint{3.807180in}{0.877035in}}%
\pgfpathlineto{\pgfqpoint{3.821820in}{0.880673in}}%
\pgfpathlineto{\pgfqpoint{3.836460in}{0.885470in}}%
\pgfpathlineto{\pgfqpoint{3.865740in}{0.892606in}}%
\pgfpathlineto{\pgfqpoint{3.880380in}{0.894168in}}%
\pgfpathlineto{\pgfqpoint{3.895020in}{0.893481in}}%
\pgfpathlineto{\pgfqpoint{3.909660in}{0.891292in}}%
\pgfpathlineto{\pgfqpoint{3.938940in}{0.883685in}}%
\pgfpathlineto{\pgfqpoint{3.953580in}{0.879958in}}%
\pgfpathlineto{\pgfqpoint{3.968219in}{0.877573in}}%
\pgfpathlineto{\pgfqpoint{3.982859in}{0.876828in}}%
\pgfpathlineto{\pgfqpoint{4.012139in}{0.881079in}}%
\pgfpathlineto{\pgfqpoint{4.056059in}{0.890789in}}%
\pgfpathlineto{\pgfqpoint{4.070699in}{0.891736in}}%
\pgfpathlineto{\pgfqpoint{4.085339in}{0.891255in}}%
\pgfpathlineto{\pgfqpoint{4.129258in}{0.882978in}}%
\pgfpathlineto{\pgfqpoint{4.129258in}{0.882978in}}%
\pgfusepath{stroke}%
\end{pgfscope}%
\begin{pgfscope}%
\pgfpathrectangle{\pgfqpoint{0.578349in}{0.682899in}}{\pgfqpoint{3.720000in}{3.020000in}}%
\pgfusepath{clip}%
\pgfsetrectcap%
\pgfsetroundjoin%
\pgfsetlinewidth{1.505625pt}%
\definecolor{currentstroke}{rgb}{0.993248,0.906157,0.143936}%
\pgfsetstrokecolor{currentstroke}%
\pgfsetdash{}{0pt}%
\pgfpathmoveto{\pgfqpoint{0.835280in}{0.905809in}}%
\pgfpathlineto{\pgfqpoint{0.864559in}{0.916139in}}%
\pgfpathlineto{\pgfqpoint{0.879199in}{0.920033in}}%
\pgfpathlineto{\pgfqpoint{0.893839in}{0.925614in}}%
\pgfpathlineto{\pgfqpoint{0.908479in}{0.929848in}}%
\pgfpathlineto{\pgfqpoint{0.923119in}{0.931974in}}%
\pgfpathlineto{\pgfqpoint{0.952399in}{0.940202in}}%
\pgfpathlineto{\pgfqpoint{0.967039in}{0.939762in}}%
\pgfpathlineto{\pgfqpoint{0.996319in}{0.943861in}}%
\pgfpathlineto{\pgfqpoint{1.010959in}{0.943427in}}%
\pgfpathlineto{\pgfqpoint{1.040238in}{0.944328in}}%
\pgfpathlineto{\pgfqpoint{1.098798in}{0.938311in}}%
\pgfpathlineto{\pgfqpoint{1.142718in}{0.928692in}}%
\pgfpathlineto{\pgfqpoint{1.186637in}{0.914415in}}%
\pgfpathlineto{\pgfqpoint{1.230557in}{0.895283in}}%
\pgfpathlineto{\pgfqpoint{1.245197in}{0.887254in}}%
\pgfpathlineto{\pgfqpoint{1.274477in}{0.868428in}}%
\pgfpathlineto{\pgfqpoint{1.318397in}{0.829120in}}%
\pgfpathlineto{\pgfqpoint{1.333036in}{0.822068in}}%
\pgfpathlineto{\pgfqpoint{1.347676in}{0.824795in}}%
\pgfpathlineto{\pgfqpoint{1.362316in}{0.835084in}}%
\pgfpathlineto{\pgfqpoint{1.391596in}{0.857587in}}%
\pgfpathlineto{\pgfqpoint{1.406236in}{0.864875in}}%
\pgfpathlineto{\pgfqpoint{1.435516in}{0.876323in}}%
\pgfpathlineto{\pgfqpoint{1.450156in}{0.879313in}}%
\pgfpathlineto{\pgfqpoint{1.479435in}{0.882612in}}%
\pgfpathlineto{\pgfqpoint{1.494075in}{0.882317in}}%
\pgfpathlineto{\pgfqpoint{1.523355in}{0.878008in}}%
\pgfpathlineto{\pgfqpoint{1.552635in}{0.869713in}}%
\pgfpathlineto{\pgfqpoint{1.581915in}{0.854673in}}%
\pgfpathlineto{\pgfqpoint{1.596555in}{0.845235in}}%
\pgfpathlineto{\pgfqpoint{1.611195in}{0.833535in}}%
\pgfpathlineto{\pgfqpoint{1.625835in}{0.823745in}}%
\pgfpathlineto{\pgfqpoint{1.640474in}{0.821016in}}%
\pgfpathlineto{\pgfqpoint{1.655114in}{0.825591in}}%
\pgfpathlineto{\pgfqpoint{1.684394in}{0.844181in}}%
\pgfpathlineto{\pgfqpoint{1.699034in}{0.851796in}}%
\pgfpathlineto{\pgfqpoint{1.713674in}{0.856530in}}%
\pgfpathlineto{\pgfqpoint{1.742954in}{0.863305in}}%
\pgfpathlineto{\pgfqpoint{1.772234in}{0.862452in}}%
\pgfpathlineto{\pgfqpoint{1.786874in}{0.861170in}}%
\pgfpathlineto{\pgfqpoint{1.816153in}{0.852455in}}%
\pgfpathlineto{\pgfqpoint{1.830793in}{0.845673in}}%
\pgfpathlineto{\pgfqpoint{1.860073in}{0.827995in}}%
\pgfpathlineto{\pgfqpoint{1.874713in}{0.821387in}}%
\pgfpathlineto{\pgfqpoint{1.889353in}{0.821637in}}%
\pgfpathlineto{\pgfqpoint{1.903993in}{0.826893in}}%
\pgfpathlineto{\pgfqpoint{1.918633in}{0.835801in}}%
\pgfpathlineto{\pgfqpoint{1.933273in}{0.842043in}}%
\pgfpathlineto{\pgfqpoint{1.947912in}{0.846969in}}%
\pgfpathlineto{\pgfqpoint{1.962552in}{0.849019in}}%
\pgfpathlineto{\pgfqpoint{1.977192in}{0.848924in}}%
\pgfpathlineto{\pgfqpoint{1.991832in}{0.847153in}}%
\pgfpathlineto{\pgfqpoint{2.006472in}{0.843416in}}%
\pgfpathlineto{\pgfqpoint{2.021112in}{0.837197in}}%
\pgfpathlineto{\pgfqpoint{2.050392in}{0.822587in}}%
\pgfpathlineto{\pgfqpoint{2.065032in}{0.820172in}}%
\pgfpathlineto{\pgfqpoint{2.079672in}{0.823726in}}%
\pgfpathlineto{\pgfqpoint{2.108951in}{0.837403in}}%
\pgfpathlineto{\pgfqpoint{2.123591in}{0.841034in}}%
\pgfpathlineto{\pgfqpoint{2.138231in}{0.841277in}}%
\pgfpathlineto{\pgfqpoint{2.152871in}{0.838656in}}%
\pgfpathlineto{\pgfqpoint{2.196791in}{0.824466in}}%
\pgfpathlineto{\pgfqpoint{2.211431in}{0.823370in}}%
\pgfpathlineto{\pgfqpoint{2.226071in}{0.825212in}}%
\pgfpathlineto{\pgfqpoint{2.240711in}{0.830550in}}%
\pgfpathlineto{\pgfqpoint{2.255350in}{0.834525in}}%
\pgfpathlineto{\pgfqpoint{2.269990in}{0.837145in}}%
\pgfpathlineto{\pgfqpoint{2.284630in}{0.836485in}}%
\pgfpathlineto{\pgfqpoint{2.299270in}{0.832490in}}%
\pgfpathlineto{\pgfqpoint{2.313910in}{0.827322in}}%
\pgfpathlineto{\pgfqpoint{2.328550in}{0.823353in}}%
\pgfpathlineto{\pgfqpoint{2.343190in}{0.823122in}}%
\pgfpathlineto{\pgfqpoint{2.357830in}{0.826156in}}%
\pgfpathlineto{\pgfqpoint{2.387110in}{0.834250in}}%
\pgfpathlineto{\pgfqpoint{2.401750in}{0.835591in}}%
\pgfpathlineto{\pgfqpoint{2.416389in}{0.833906in}}%
\pgfpathlineto{\pgfqpoint{2.460309in}{0.825106in}}%
\pgfpathlineto{\pgfqpoint{2.474949in}{0.825453in}}%
\pgfpathlineto{\pgfqpoint{2.518869in}{0.833287in}}%
\pgfpathlineto{\pgfqpoint{2.533509in}{0.833687in}}%
\pgfpathlineto{\pgfqpoint{2.548149in}{0.829930in}}%
\pgfpathlineto{\pgfqpoint{2.562789in}{0.828663in}}%
\pgfpathlineto{\pgfqpoint{2.577428in}{0.826222in}}%
\pgfpathlineto{\pgfqpoint{2.606708in}{0.828206in}}%
\pgfpathlineto{\pgfqpoint{2.635988in}{0.832366in}}%
\pgfpathlineto{\pgfqpoint{2.650628in}{0.833053in}}%
\pgfpathlineto{\pgfqpoint{2.679908in}{0.830717in}}%
\pgfpathlineto{\pgfqpoint{2.709188in}{0.827387in}}%
\pgfpathlineto{\pgfqpoint{2.738467in}{0.828313in}}%
\pgfpathlineto{\pgfqpoint{2.782387in}{0.832679in}}%
\pgfpathlineto{\pgfqpoint{2.797027in}{0.832586in}}%
\pgfpathlineto{\pgfqpoint{2.855587in}{0.828133in}}%
\pgfpathlineto{\pgfqpoint{2.928786in}{0.831031in}}%
\pgfpathlineto{\pgfqpoint{2.958066in}{0.828477in}}%
\pgfpathlineto{\pgfqpoint{2.987346in}{0.828419in}}%
\pgfpathlineto{\pgfqpoint{3.031265in}{0.831395in}}%
\pgfpathlineto{\pgfqpoint{3.060545in}{0.829639in}}%
\pgfpathlineto{\pgfqpoint{3.089825in}{0.827793in}}%
\pgfpathlineto{\pgfqpoint{3.119105in}{0.829072in}}%
\pgfpathlineto{\pgfqpoint{3.148385in}{0.831153in}}%
\pgfpathlineto{\pgfqpoint{3.177665in}{0.830523in}}%
\pgfpathlineto{\pgfqpoint{3.206944in}{0.828449in}}%
\pgfpathlineto{\pgfqpoint{3.236224in}{0.828476in}}%
\pgfpathlineto{\pgfqpoint{3.294784in}{0.830867in}}%
\pgfpathlineto{\pgfqpoint{3.353343in}{0.828019in}}%
\pgfpathlineto{\pgfqpoint{3.411903in}{0.830085in}}%
\pgfpathlineto{\pgfqpoint{3.499742in}{0.828889in}}%
\pgfpathlineto{\pgfqpoint{3.529022in}{0.829475in}}%
\pgfpathlineto{\pgfqpoint{3.587582in}{0.827936in}}%
\pgfpathlineto{\pgfqpoint{3.660781in}{0.828726in}}%
\pgfpathlineto{\pgfqpoint{3.719341in}{0.828390in}}%
\pgfpathlineto{\pgfqpoint{3.777901in}{0.828916in}}%
\pgfpathlineto{\pgfqpoint{3.836460in}{0.828394in}}%
\pgfpathlineto{\pgfqpoint{3.880380in}{0.829067in}}%
\pgfpathlineto{\pgfqpoint{3.938940in}{0.828177in}}%
\pgfpathlineto{\pgfqpoint{4.012139in}{0.829068in}}%
\pgfpathlineto{\pgfqpoint{4.070699in}{0.828546in}}%
\pgfpathlineto{\pgfqpoint{4.099979in}{0.829283in}}%
\pgfpathlineto{\pgfqpoint{4.099979in}{0.829283in}}%
\pgfusepath{stroke}%
\end{pgfscope}%
\begin{pgfscope}%
\pgfsetrectcap%
\pgfsetmiterjoin%
\pgfsetlinewidth{0.803000pt}%
\definecolor{currentstroke}{rgb}{0.501961,0.501961,0.501961}%
\pgfsetstrokecolor{currentstroke}%
\pgfsetdash{}{0pt}%
\pgfpathmoveto{\pgfqpoint{0.578349in}{0.682899in}}%
\pgfpathlineto{\pgfqpoint{0.578349in}{3.702899in}}%
\pgfusepath{stroke}%
\end{pgfscope}%
\begin{pgfscope}%
\pgfsetrectcap%
\pgfsetmiterjoin%
\pgfsetlinewidth{0.803000pt}%
\definecolor{currentstroke}{rgb}{0.501961,0.501961,0.501961}%
\pgfsetstrokecolor{currentstroke}%
\pgfsetdash{}{0pt}%
\pgfpathmoveto{\pgfqpoint{4.298349in}{0.682899in}}%
\pgfpathlineto{\pgfqpoint{4.298349in}{3.702899in}}%
\pgfusepath{stroke}%
\end{pgfscope}%
\begin{pgfscope}%
\pgfsetrectcap%
\pgfsetmiterjoin%
\pgfsetlinewidth{0.803000pt}%
\definecolor{currentstroke}{rgb}{0.501961,0.501961,0.501961}%
\pgfsetstrokecolor{currentstroke}%
\pgfsetdash{}{0pt}%
\pgfpathmoveto{\pgfqpoint{0.578349in}{0.682899in}}%
\pgfpathlineto{\pgfqpoint{4.298349in}{0.682899in}}%
\pgfusepath{stroke}%
\end{pgfscope}%
\begin{pgfscope}%
\pgfsetrectcap%
\pgfsetmiterjoin%
\pgfsetlinewidth{0.803000pt}%
\definecolor{currentstroke}{rgb}{0.501961,0.501961,0.501961}%
\pgfsetstrokecolor{currentstroke}%
\pgfsetdash{}{0pt}%
\pgfpathmoveto{\pgfqpoint{0.578349in}{3.702899in}}%
\pgfpathlineto{\pgfqpoint{4.298349in}{3.702899in}}%
\pgfusepath{stroke}%
\end{pgfscope}%
\begin{pgfscope}%
\pgfpathrectangle{\pgfqpoint{4.530849in}{0.682899in}}{\pgfqpoint{0.151000in}{3.020000in}}%
\pgfusepath{clip}%
\pgfsetbuttcap%
\pgfsetmiterjoin%
\definecolor{currentfill}{rgb}{1.000000,1.000000,1.000000}%
\pgfsetfillcolor{currentfill}%
\pgfsetlinewidth{0.010037pt}%
\definecolor{currentstroke}{rgb}{1.000000,1.000000,1.000000}%
\pgfsetstrokecolor{currentstroke}%
\pgfsetdash{}{0pt}%
\pgfpathmoveto{\pgfqpoint{4.530849in}{0.682899in}}%
\pgfpathlineto{\pgfqpoint{4.530849in}{0.694696in}}%
\pgfpathlineto{\pgfqpoint{4.530849in}{3.691102in}}%
\pgfpathlineto{\pgfqpoint{4.530849in}{3.702899in}}%
\pgfpathlineto{\pgfqpoint{4.681849in}{3.702899in}}%
\pgfpathlineto{\pgfqpoint{4.681849in}{3.691102in}}%
\pgfpathlineto{\pgfqpoint{4.681849in}{0.694696in}}%
\pgfpathlineto{\pgfqpoint{4.681849in}{0.682899in}}%
\pgfpathclose%
\pgfusepath{stroke,fill}%
\end{pgfscope}%
\begin{pgfscope}%
\pgfsys@transformshift{4.530000in}{0.687527in}%
\pgftext[left,bottom]{\pgfimage[interpolate=true,width=0.150000in,height=3.020000in]{series-img0.png}}%
\end{pgfscope}%
\begin{pgfscope}%
\pgfsetbuttcap%
\pgfsetroundjoin%
\definecolor{currentfill}{rgb}{0.000000,0.000000,0.000000}%
\pgfsetfillcolor{currentfill}%
\pgfsetlinewidth{0.803000pt}%
\definecolor{currentstroke}{rgb}{0.000000,0.000000,0.000000}%
\pgfsetstrokecolor{currentstroke}%
\pgfsetdash{}{0pt}%
\pgfsys@defobject{currentmarker}{\pgfqpoint{0.000000in}{0.000000in}}{\pgfqpoint{0.048611in}{0.000000in}}{%
\pgfpathmoveto{\pgfqpoint{0.000000in}{0.000000in}}%
\pgfpathlineto{\pgfqpoint{0.048611in}{0.000000in}}%
\pgfusepath{stroke,fill}%
}%
\begin{pgfscope}%
\pgfsys@transformshift{4.681849in}{0.966221in}%
\pgfsys@useobject{currentmarker}{}%
\end{pgfscope}%
\end{pgfscope}%
\begin{pgfscope}%
\pgfsetbuttcap%
\pgfsetroundjoin%
\definecolor{currentfill}{rgb}{0.000000,0.000000,0.000000}%
\pgfsetfillcolor{currentfill}%
\pgfsetlinewidth{0.803000pt}%
\definecolor{currentstroke}{rgb}{0.000000,0.000000,0.000000}%
\pgfsetstrokecolor{currentstroke}%
\pgfsetdash{}{0pt}%
\pgfsys@defobject{currentmarker}{\pgfqpoint{0.000000in}{0.000000in}}{\pgfqpoint{0.048611in}{0.000000in}}{%
\pgfpathmoveto{\pgfqpoint{0.000000in}{0.000000in}}%
\pgfpathlineto{\pgfqpoint{0.048611in}{0.000000in}}%
\pgfusepath{stroke,fill}%
}%
\begin{pgfscope}%
\pgfsys@transformshift{4.681849in}{1.352218in}%
\pgfsys@useobject{currentmarker}{}%
\end{pgfscope}%
\end{pgfscope}%
\begin{pgfscope}%
\pgfsetbuttcap%
\pgfsetroundjoin%
\definecolor{currentfill}{rgb}{0.000000,0.000000,0.000000}%
\pgfsetfillcolor{currentfill}%
\pgfsetlinewidth{0.803000pt}%
\definecolor{currentstroke}{rgb}{0.000000,0.000000,0.000000}%
\pgfsetstrokecolor{currentstroke}%
\pgfsetdash{}{0pt}%
\pgfsys@defobject{currentmarker}{\pgfqpoint{0.000000in}{0.000000in}}{\pgfqpoint{0.048611in}{0.000000in}}{%
\pgfpathmoveto{\pgfqpoint{0.000000in}{0.000000in}}%
\pgfpathlineto{\pgfqpoint{0.048611in}{0.000000in}}%
\pgfusepath{stroke,fill}%
}%
\begin{pgfscope}%
\pgfsys@transformshift{4.681849in}{1.626088in}%
\pgfsys@useobject{currentmarker}{}%
\end{pgfscope}%
\end{pgfscope}%
\begin{pgfscope}%
\pgfsetbuttcap%
\pgfsetroundjoin%
\definecolor{currentfill}{rgb}{0.000000,0.000000,0.000000}%
\pgfsetfillcolor{currentfill}%
\pgfsetlinewidth{0.803000pt}%
\definecolor{currentstroke}{rgb}{0.000000,0.000000,0.000000}%
\pgfsetstrokecolor{currentstroke}%
\pgfsetdash{}{0pt}%
\pgfsys@defobject{currentmarker}{\pgfqpoint{0.000000in}{0.000000in}}{\pgfqpoint{0.048611in}{0.000000in}}{%
\pgfpathmoveto{\pgfqpoint{0.000000in}{0.000000in}}%
\pgfpathlineto{\pgfqpoint{0.048611in}{0.000000in}}%
\pgfusepath{stroke,fill}%
}%
\begin{pgfscope}%
\pgfsys@transformshift{4.681849in}{1.838517in}%
\pgfsys@useobject{currentmarker}{}%
\end{pgfscope}%
\end{pgfscope}%
\begin{pgfscope}%
\pgfsetbuttcap%
\pgfsetroundjoin%
\definecolor{currentfill}{rgb}{0.000000,0.000000,0.000000}%
\pgfsetfillcolor{currentfill}%
\pgfsetlinewidth{0.803000pt}%
\definecolor{currentstroke}{rgb}{0.000000,0.000000,0.000000}%
\pgfsetstrokecolor{currentstroke}%
\pgfsetdash{}{0pt}%
\pgfsys@defobject{currentmarker}{\pgfqpoint{0.000000in}{0.000000in}}{\pgfqpoint{0.048611in}{0.000000in}}{%
\pgfpathmoveto{\pgfqpoint{0.000000in}{0.000000in}}%
\pgfpathlineto{\pgfqpoint{0.048611in}{0.000000in}}%
\pgfusepath{stroke,fill}%
}%
\begin{pgfscope}%
\pgfsys@transformshift{4.681849in}{2.012085in}%
\pgfsys@useobject{currentmarker}{}%
\end{pgfscope}%
\end{pgfscope}%
\begin{pgfscope}%
\pgfsetbuttcap%
\pgfsetroundjoin%
\definecolor{currentfill}{rgb}{0.000000,0.000000,0.000000}%
\pgfsetfillcolor{currentfill}%
\pgfsetlinewidth{0.803000pt}%
\definecolor{currentstroke}{rgb}{0.000000,0.000000,0.000000}%
\pgfsetstrokecolor{currentstroke}%
\pgfsetdash{}{0pt}%
\pgfsys@defobject{currentmarker}{\pgfqpoint{0.000000in}{0.000000in}}{\pgfqpoint{0.048611in}{0.000000in}}{%
\pgfpathmoveto{\pgfqpoint{0.000000in}{0.000000in}}%
\pgfpathlineto{\pgfqpoint{0.048611in}{0.000000in}}%
\pgfusepath{stroke,fill}%
}%
\begin{pgfscope}%
\pgfsys@transformshift{4.681849in}{2.158834in}%
\pgfsys@useobject{currentmarker}{}%
\end{pgfscope}%
\end{pgfscope}%
\begin{pgfscope}%
\pgfsetbuttcap%
\pgfsetroundjoin%
\definecolor{currentfill}{rgb}{0.000000,0.000000,0.000000}%
\pgfsetfillcolor{currentfill}%
\pgfsetlinewidth{0.803000pt}%
\definecolor{currentstroke}{rgb}{0.000000,0.000000,0.000000}%
\pgfsetstrokecolor{currentstroke}%
\pgfsetdash{}{0pt}%
\pgfsys@defobject{currentmarker}{\pgfqpoint{0.000000in}{0.000000in}}{\pgfqpoint{0.048611in}{0.000000in}}{%
\pgfpathmoveto{\pgfqpoint{0.000000in}{0.000000in}}%
\pgfpathlineto{\pgfqpoint{0.048611in}{0.000000in}}%
\pgfusepath{stroke,fill}%
}%
\begin{pgfscope}%
\pgfsys@transformshift{4.681849in}{2.285954in}%
\pgfsys@useobject{currentmarker}{}%
\end{pgfscope}%
\end{pgfscope}%
\begin{pgfscope}%
\pgfsetbuttcap%
\pgfsetroundjoin%
\definecolor{currentfill}{rgb}{0.000000,0.000000,0.000000}%
\pgfsetfillcolor{currentfill}%
\pgfsetlinewidth{0.803000pt}%
\definecolor{currentstroke}{rgb}{0.000000,0.000000,0.000000}%
\pgfsetstrokecolor{currentstroke}%
\pgfsetdash{}{0pt}%
\pgfsys@defobject{currentmarker}{\pgfqpoint{0.000000in}{0.000000in}}{\pgfqpoint{0.048611in}{0.000000in}}{%
\pgfpathmoveto{\pgfqpoint{0.000000in}{0.000000in}}%
\pgfpathlineto{\pgfqpoint{0.048611in}{0.000000in}}%
\pgfusepath{stroke,fill}%
}%
\begin{pgfscope}%
\pgfsys@transformshift{4.681849in}{2.398082in}%
\pgfsys@useobject{currentmarker}{}%
\end{pgfscope}%
\end{pgfscope}%
\begin{pgfscope}%
\pgfsetbuttcap%
\pgfsetroundjoin%
\definecolor{currentfill}{rgb}{0.000000,0.000000,0.000000}%
\pgfsetfillcolor{currentfill}%
\pgfsetlinewidth{0.803000pt}%
\definecolor{currentstroke}{rgb}{0.000000,0.000000,0.000000}%
\pgfsetstrokecolor{currentstroke}%
\pgfsetdash{}{0pt}%
\pgfsys@defobject{currentmarker}{\pgfqpoint{0.000000in}{0.000000in}}{\pgfqpoint{0.048611in}{0.000000in}}{%
\pgfpathmoveto{\pgfqpoint{0.000000in}{0.000000in}}%
\pgfpathlineto{\pgfqpoint{0.048611in}{0.000000in}}%
\pgfusepath{stroke,fill}%
}%
\begin{pgfscope}%
\pgfsys@transformshift{4.681849in}{2.498383in}%
\pgfsys@useobject{currentmarker}{}%
\end{pgfscope}%
\end{pgfscope}%
\begin{pgfscope}%
\pgftext[x=4.779072in,y=2.413965in,left,base]{\rmfamily\fontsize{16.000000}{19.200000}\selectfont \(\displaystyle 10^{1}\)}%
\end{pgfscope}%
\begin{pgfscope}%
\pgfsetbuttcap%
\pgfsetroundjoin%
\definecolor{currentfill}{rgb}{0.000000,0.000000,0.000000}%
\pgfsetfillcolor{currentfill}%
\pgfsetlinewidth{0.803000pt}%
\definecolor{currentstroke}{rgb}{0.000000,0.000000,0.000000}%
\pgfsetstrokecolor{currentstroke}%
\pgfsetdash{}{0pt}%
\pgfsys@defobject{currentmarker}{\pgfqpoint{0.000000in}{0.000000in}}{\pgfqpoint{0.048611in}{0.000000in}}{%
\pgfpathmoveto{\pgfqpoint{0.000000in}{0.000000in}}%
\pgfpathlineto{\pgfqpoint{0.048611in}{0.000000in}}%
\pgfusepath{stroke,fill}%
}%
\begin{pgfscope}%
\pgfsys@transformshift{4.681849in}{3.158250in}%
\pgfsys@useobject{currentmarker}{}%
\end{pgfscope}%
\end{pgfscope}%
\begin{pgfscope}%
\pgfsetbuttcap%
\pgfsetroundjoin%
\definecolor{currentfill}{rgb}{0.000000,0.000000,0.000000}%
\pgfsetfillcolor{currentfill}%
\pgfsetlinewidth{0.803000pt}%
\definecolor{currentstroke}{rgb}{0.000000,0.000000,0.000000}%
\pgfsetstrokecolor{currentstroke}%
\pgfsetdash{}{0pt}%
\pgfsys@defobject{currentmarker}{\pgfqpoint{0.000000in}{0.000000in}}{\pgfqpoint{0.048611in}{0.000000in}}{%
\pgfpathmoveto{\pgfqpoint{0.000000in}{0.000000in}}%
\pgfpathlineto{\pgfqpoint{0.048611in}{0.000000in}}%
\pgfusepath{stroke,fill}%
}%
\begin{pgfscope}%
\pgfsys@transformshift{4.681849in}{3.544247in}%
\pgfsys@useobject{currentmarker}{}%
\end{pgfscope}%
\end{pgfscope}%
\begin{pgfscope}%
\pgftext[x=5.193305in,y=2.192899in,,top]{\rmfamily\fontsize{14.000000}{16.800000}\selectfont \(\displaystyle {\mathbf{E} \mbox{u}}\)}%
\end{pgfscope}%
\begin{pgfscope}%
\pgfsetbuttcap%
\pgfsetmiterjoin%
\pgfsetlinewidth{0.803000pt}%
\definecolor{currentstroke}{rgb}{0.501961,0.501961,0.501961}%
\pgfsetstrokecolor{currentstroke}%
\pgfsetdash{}{0pt}%
\pgfpathmoveto{\pgfqpoint{4.530849in}{0.682899in}}%
\pgfpathlineto{\pgfqpoint{4.530849in}{0.694696in}}%
\pgfpathlineto{\pgfqpoint{4.530849in}{3.691102in}}%
\pgfpathlineto{\pgfqpoint{4.530849in}{3.702899in}}%
\pgfpathlineto{\pgfqpoint{4.681849in}{3.702899in}}%
\pgfpathlineto{\pgfqpoint{4.681849in}{3.691102in}}%
\pgfpathlineto{\pgfqpoint{4.681849in}{0.694696in}}%
\pgfpathlineto{\pgfqpoint{4.681849in}{0.682899in}}%
\pgfpathclose%
\pgfusepath{stroke}%
\end{pgfscope}%
\end{pgfpicture}%
\makeatother%
\endgroup%

    \caption{A simple EMA plot.\label{fig:series}}
\end{figure}

\begin{figure}[htb]
    \centering
    \resizebox{14cm}{!}{%% Creator: Matplotlib, PGF backend
%%
%% To include the figure in your LaTeX document, write
%%   \input{<filename>.pgf}
%%
%% Make sure the required packages are loaded in your preamble
%%   \usepackage{pgf}
%%
%% Figures using additional raster images can only be included by \input if
%% they are in the same directory as the main LaTeX file. For loading figures
%% from other directories you can use the `import` package
%%   \usepackage{import}
%% and then include the figures with
%%   \import{<path to file>}{<filename>.pgf}
%%
%% Matplotlib used the following preamble
%%   \usepackage{fontspec}
%%   \setmainfont{DejaVu Serif}
%%   \setsansfont{DejaVu Sans}
%%   \setmonofont{DejaVu Sans Mono}
%%
\begingroup%
\makeatletter%
\begin{pgfpicture}%
\pgfpathrectangle{\pgfpointorigin}{\pgfqpoint{12.806532in}{18.099281in}}%
\pgfusepath{use as bounding box, clip}%
\begin{pgfscope}%
\pgfsetbuttcap%
\pgfsetmiterjoin%
\definecolor{currentfill}{rgb}{1.000000,1.000000,1.000000}%
\pgfsetfillcolor{currentfill}%
\pgfsetlinewidth{0.000000pt}%
\definecolor{currentstroke}{rgb}{1.000000,1.000000,1.000000}%
\pgfsetstrokecolor{currentstroke}%
\pgfsetdash{}{0pt}%
\pgfpathmoveto{\pgfqpoint{0.000000in}{0.000000in}}%
\pgfpathlineto{\pgfqpoint{12.806532in}{0.000000in}}%
\pgfpathlineto{\pgfqpoint{12.806532in}{18.099281in}}%
\pgfpathlineto{\pgfqpoint{0.000000in}{18.099281in}}%
\pgfpathclose%
\pgfusepath{fill}%
\end{pgfscope}%
\begin{pgfscope}%
\pgfsetbuttcap%
\pgfsetmiterjoin%
\definecolor{currentfill}{rgb}{1.000000,1.000000,1.000000}%
\pgfsetfillcolor{currentfill}%
\pgfsetlinewidth{0.000000pt}%
\definecolor{currentstroke}{rgb}{0.000000,0.000000,0.000000}%
\pgfsetstrokecolor{currentstroke}%
\pgfsetstrokeopacity{0.000000}%
\pgfsetdash{}{0pt}%
\pgfpathmoveto{\pgfqpoint{0.880000in}{6.967719in}}%
\pgfpathlineto{\pgfqpoint{2.777959in}{6.967719in}}%
\pgfpathlineto{\pgfqpoint{2.777959in}{8.340446in}}%
\pgfpathlineto{\pgfqpoint{0.880000in}{8.340446in}}%
\pgfpathclose%
\pgfusepath{fill}%
\end{pgfscope}%
\begin{pgfscope}%
\pgfsetbuttcap%
\pgfsetroundjoin%
\definecolor{currentfill}{rgb}{0.000000,0.000000,0.000000}%
\pgfsetfillcolor{currentfill}%
\pgfsetlinewidth{0.803000pt}%
\definecolor{currentstroke}{rgb}{0.000000,0.000000,0.000000}%
\pgfsetstrokecolor{currentstroke}%
\pgfsetdash{}{0pt}%
\pgfsys@defobject{currentmarker}{\pgfqpoint{0.000000in}{-0.048611in}}{\pgfqpoint{0.000000in}{0.000000in}}{%
\pgfpathmoveto{\pgfqpoint{0.000000in}{0.000000in}}%
\pgfpathlineto{\pgfqpoint{0.000000in}{-0.048611in}}%
\pgfusepath{stroke,fill}%
}%
\begin{pgfscope}%
\pgfsys@transformshift{0.880000in}{6.967719in}%
\pgfsys@useobject{currentmarker}{}%
\end{pgfscope}%
\end{pgfscope}%
\begin{pgfscope}%
\pgftext[x=0.880000in,y=6.870496in,,top]{\rmfamily\fontsize{12.000000}{14.400000}\selectfont \(\displaystyle 0.0\)}%
\end{pgfscope}%
\begin{pgfscope}%
\pgfsetbuttcap%
\pgfsetroundjoin%
\definecolor{currentfill}{rgb}{0.000000,0.000000,0.000000}%
\pgfsetfillcolor{currentfill}%
\pgfsetlinewidth{0.803000pt}%
\definecolor{currentstroke}{rgb}{0.000000,0.000000,0.000000}%
\pgfsetstrokecolor{currentstroke}%
\pgfsetdash{}{0pt}%
\pgfsys@defobject{currentmarker}{\pgfqpoint{0.000000in}{-0.048611in}}{\pgfqpoint{0.000000in}{0.000000in}}{%
\pgfpathmoveto{\pgfqpoint{0.000000in}{0.000000in}}%
\pgfpathlineto{\pgfqpoint{0.000000in}{-0.048611in}}%
\pgfusepath{stroke,fill}%
}%
\begin{pgfscope}%
\pgfsys@transformshift{1.828980in}{6.967719in}%
\pgfsys@useobject{currentmarker}{}%
\end{pgfscope}%
\end{pgfscope}%
\begin{pgfscope}%
\pgftext[x=1.828980in,y=6.870496in,,top]{\rmfamily\fontsize{12.000000}{14.400000}\selectfont \(\displaystyle 0.5\)}%
\end{pgfscope}%
\begin{pgfscope}%
\pgfsetbuttcap%
\pgfsetroundjoin%
\definecolor{currentfill}{rgb}{0.000000,0.000000,0.000000}%
\pgfsetfillcolor{currentfill}%
\pgfsetlinewidth{0.803000pt}%
\definecolor{currentstroke}{rgb}{0.000000,0.000000,0.000000}%
\pgfsetstrokecolor{currentstroke}%
\pgfsetdash{}{0pt}%
\pgfsys@defobject{currentmarker}{\pgfqpoint{0.000000in}{-0.048611in}}{\pgfqpoint{0.000000in}{0.000000in}}{%
\pgfpathmoveto{\pgfqpoint{0.000000in}{0.000000in}}%
\pgfpathlineto{\pgfqpoint{0.000000in}{-0.048611in}}%
\pgfusepath{stroke,fill}%
}%
\begin{pgfscope}%
\pgfsys@transformshift{2.777959in}{6.967719in}%
\pgfsys@useobject{currentmarker}{}%
\end{pgfscope}%
\end{pgfscope}%
\begin{pgfscope}%
\pgftext[x=2.777959in,y=6.870496in,,top]{\rmfamily\fontsize{12.000000}{14.400000}\selectfont \(\displaystyle 1.0\)}%
\end{pgfscope}%
\begin{pgfscope}%
\pgfsetbuttcap%
\pgfsetroundjoin%
\definecolor{currentfill}{rgb}{0.000000,0.000000,0.000000}%
\pgfsetfillcolor{currentfill}%
\pgfsetlinewidth{0.803000pt}%
\definecolor{currentstroke}{rgb}{0.000000,0.000000,0.000000}%
\pgfsetstrokecolor{currentstroke}%
\pgfsetdash{}{0pt}%
\pgfsys@defobject{currentmarker}{\pgfqpoint{-0.048611in}{0.000000in}}{\pgfqpoint{0.000000in}{0.000000in}}{%
\pgfpathmoveto{\pgfqpoint{0.000000in}{0.000000in}}%
\pgfpathlineto{\pgfqpoint{-0.048611in}{0.000000in}}%
\pgfusepath{stroke,fill}%
}%
\begin{pgfscope}%
\pgfsys@transformshift{0.880000in}{6.967719in}%
\pgfsys@useobject{currentmarker}{}%
\end{pgfscope}%
\end{pgfscope}%
\begin{pgfscope}%
\pgftext[x=0.492657in,y=6.904405in,left,base]{\rmfamily\fontsize{12.000000}{14.400000}\selectfont \(\displaystyle 0.00\)}%
\end{pgfscope}%
\begin{pgfscope}%
\pgfsetbuttcap%
\pgfsetroundjoin%
\definecolor{currentfill}{rgb}{0.000000,0.000000,0.000000}%
\pgfsetfillcolor{currentfill}%
\pgfsetlinewidth{0.803000pt}%
\definecolor{currentstroke}{rgb}{0.000000,0.000000,0.000000}%
\pgfsetstrokecolor{currentstroke}%
\pgfsetdash{}{0pt}%
\pgfsys@defobject{currentmarker}{\pgfqpoint{-0.048611in}{0.000000in}}{\pgfqpoint{0.000000in}{0.000000in}}{%
\pgfpathmoveto{\pgfqpoint{0.000000in}{0.000000in}}%
\pgfpathlineto{\pgfqpoint{-0.048611in}{0.000000in}}%
\pgfusepath{stroke,fill}%
}%
\begin{pgfscope}%
\pgfsys@transformshift{0.880000in}{7.310900in}%
\pgfsys@useobject{currentmarker}{}%
\end{pgfscope}%
\end{pgfscope}%
\begin{pgfscope}%
\pgftext[x=0.492657in,y=7.247587in,left,base]{\rmfamily\fontsize{12.000000}{14.400000}\selectfont \(\displaystyle 0.25\)}%
\end{pgfscope}%
\begin{pgfscope}%
\pgfsetbuttcap%
\pgfsetroundjoin%
\definecolor{currentfill}{rgb}{0.000000,0.000000,0.000000}%
\pgfsetfillcolor{currentfill}%
\pgfsetlinewidth{0.803000pt}%
\definecolor{currentstroke}{rgb}{0.000000,0.000000,0.000000}%
\pgfsetstrokecolor{currentstroke}%
\pgfsetdash{}{0pt}%
\pgfsys@defobject{currentmarker}{\pgfqpoint{-0.048611in}{0.000000in}}{\pgfqpoint{0.000000in}{0.000000in}}{%
\pgfpathmoveto{\pgfqpoint{0.000000in}{0.000000in}}%
\pgfpathlineto{\pgfqpoint{-0.048611in}{0.000000in}}%
\pgfusepath{stroke,fill}%
}%
\begin{pgfscope}%
\pgfsys@transformshift{0.880000in}{7.654082in}%
\pgfsys@useobject{currentmarker}{}%
\end{pgfscope}%
\end{pgfscope}%
\begin{pgfscope}%
\pgftext[x=0.492657in,y=7.590768in,left,base]{\rmfamily\fontsize{12.000000}{14.400000}\selectfont \(\displaystyle 0.50\)}%
\end{pgfscope}%
\begin{pgfscope}%
\pgfsetbuttcap%
\pgfsetroundjoin%
\definecolor{currentfill}{rgb}{0.000000,0.000000,0.000000}%
\pgfsetfillcolor{currentfill}%
\pgfsetlinewidth{0.803000pt}%
\definecolor{currentstroke}{rgb}{0.000000,0.000000,0.000000}%
\pgfsetstrokecolor{currentstroke}%
\pgfsetdash{}{0pt}%
\pgfsys@defobject{currentmarker}{\pgfqpoint{-0.048611in}{0.000000in}}{\pgfqpoint{0.000000in}{0.000000in}}{%
\pgfpathmoveto{\pgfqpoint{0.000000in}{0.000000in}}%
\pgfpathlineto{\pgfqpoint{-0.048611in}{0.000000in}}%
\pgfusepath{stroke,fill}%
}%
\begin{pgfscope}%
\pgfsys@transformshift{0.880000in}{7.997264in}%
\pgfsys@useobject{currentmarker}{}%
\end{pgfscope}%
\end{pgfscope}%
\begin{pgfscope}%
\pgftext[x=0.492657in,y=7.933950in,left,base]{\rmfamily\fontsize{12.000000}{14.400000}\selectfont \(\displaystyle 0.75\)}%
\end{pgfscope}%
\begin{pgfscope}%
\pgfsetbuttcap%
\pgfsetroundjoin%
\definecolor{currentfill}{rgb}{0.000000,0.000000,0.000000}%
\pgfsetfillcolor{currentfill}%
\pgfsetlinewidth{0.803000pt}%
\definecolor{currentstroke}{rgb}{0.000000,0.000000,0.000000}%
\pgfsetstrokecolor{currentstroke}%
\pgfsetdash{}{0pt}%
\pgfsys@defobject{currentmarker}{\pgfqpoint{-0.048611in}{0.000000in}}{\pgfqpoint{0.000000in}{0.000000in}}{%
\pgfpathmoveto{\pgfqpoint{0.000000in}{0.000000in}}%
\pgfpathlineto{\pgfqpoint{-0.048611in}{0.000000in}}%
\pgfusepath{stroke,fill}%
}%
\begin{pgfscope}%
\pgfsys@transformshift{0.880000in}{8.340446in}%
\pgfsys@useobject{currentmarker}{}%
\end{pgfscope}%
\end{pgfscope}%
\begin{pgfscope}%
\pgftext[x=0.492657in,y=8.277132in,left,base]{\rmfamily\fontsize{12.000000}{14.400000}\selectfont \(\displaystyle 1.00\)}%
\end{pgfscope}%
\begin{pgfscope}%
\pgfsetrectcap%
\pgfsetmiterjoin%
\pgfsetlinewidth{0.803000pt}%
\definecolor{currentstroke}{rgb}{0.501961,0.501961,0.501961}%
\pgfsetstrokecolor{currentstroke}%
\pgfsetdash{}{0pt}%
\pgfpathmoveto{\pgfqpoint{0.880000in}{6.967719in}}%
\pgfpathlineto{\pgfqpoint{0.880000in}{8.340446in}}%
\pgfusepath{stroke}%
\end{pgfscope}%
\begin{pgfscope}%
\pgfsetrectcap%
\pgfsetmiterjoin%
\pgfsetlinewidth{0.803000pt}%
\definecolor{currentstroke}{rgb}{0.501961,0.501961,0.501961}%
\pgfsetstrokecolor{currentstroke}%
\pgfsetdash{}{0pt}%
\pgfpathmoveto{\pgfqpoint{2.777959in}{6.967719in}}%
\pgfpathlineto{\pgfqpoint{2.777959in}{8.340446in}}%
\pgfusepath{stroke}%
\end{pgfscope}%
\begin{pgfscope}%
\pgfsetrectcap%
\pgfsetmiterjoin%
\pgfsetlinewidth{0.803000pt}%
\definecolor{currentstroke}{rgb}{0.501961,0.501961,0.501961}%
\pgfsetstrokecolor{currentstroke}%
\pgfsetdash{}{0pt}%
\pgfpathmoveto{\pgfqpoint{0.880000in}{6.967719in}}%
\pgfpathlineto{\pgfqpoint{2.777959in}{6.967719in}}%
\pgfusepath{stroke}%
\end{pgfscope}%
\begin{pgfscope}%
\pgfsetrectcap%
\pgfsetmiterjoin%
\pgfsetlinewidth{0.803000pt}%
\definecolor{currentstroke}{rgb}{0.501961,0.501961,0.501961}%
\pgfsetstrokecolor{currentstroke}%
\pgfsetdash{}{0pt}%
\pgfpathmoveto{\pgfqpoint{0.880000in}{8.340446in}}%
\pgfpathlineto{\pgfqpoint{2.777959in}{8.340446in}}%
\pgfusepath{stroke}%
\end{pgfscope}%
\begin{pgfscope}%
\pgfsetbuttcap%
\pgfsetmiterjoin%
\definecolor{currentfill}{rgb}{1.000000,1.000000,1.000000}%
\pgfsetfillcolor{currentfill}%
\pgfsetlinewidth{0.000000pt}%
\definecolor{currentstroke}{rgb}{0.000000,0.000000,0.000000}%
\pgfsetstrokecolor{currentstroke}%
\pgfsetstrokeopacity{0.000000}%
\pgfsetdash{}{0pt}%
\pgfpathmoveto{\pgfqpoint{3.347347in}{6.967719in}}%
\pgfpathlineto{\pgfqpoint{5.245306in}{6.967719in}}%
\pgfpathlineto{\pgfqpoint{5.245306in}{8.340446in}}%
\pgfpathlineto{\pgfqpoint{3.347347in}{8.340446in}}%
\pgfpathclose%
\pgfusepath{fill}%
\end{pgfscope}%
\begin{pgfscope}%
\pgfsetbuttcap%
\pgfsetroundjoin%
\definecolor{currentfill}{rgb}{0.000000,0.000000,0.000000}%
\pgfsetfillcolor{currentfill}%
\pgfsetlinewidth{0.803000pt}%
\definecolor{currentstroke}{rgb}{0.000000,0.000000,0.000000}%
\pgfsetstrokecolor{currentstroke}%
\pgfsetdash{}{0pt}%
\pgfsys@defobject{currentmarker}{\pgfqpoint{0.000000in}{-0.048611in}}{\pgfqpoint{0.000000in}{0.000000in}}{%
\pgfpathmoveto{\pgfqpoint{0.000000in}{0.000000in}}%
\pgfpathlineto{\pgfqpoint{0.000000in}{-0.048611in}}%
\pgfusepath{stroke,fill}%
}%
\begin{pgfscope}%
\pgfsys@transformshift{3.347347in}{6.967719in}%
\pgfsys@useobject{currentmarker}{}%
\end{pgfscope}%
\end{pgfscope}%
\begin{pgfscope}%
\pgftext[x=3.347347in,y=6.870496in,,top]{\rmfamily\fontsize{12.000000}{14.400000}\selectfont \(\displaystyle 0.0\)}%
\end{pgfscope}%
\begin{pgfscope}%
\pgfsetbuttcap%
\pgfsetroundjoin%
\definecolor{currentfill}{rgb}{0.000000,0.000000,0.000000}%
\pgfsetfillcolor{currentfill}%
\pgfsetlinewidth{0.803000pt}%
\definecolor{currentstroke}{rgb}{0.000000,0.000000,0.000000}%
\pgfsetstrokecolor{currentstroke}%
\pgfsetdash{}{0pt}%
\pgfsys@defobject{currentmarker}{\pgfqpoint{0.000000in}{-0.048611in}}{\pgfqpoint{0.000000in}{0.000000in}}{%
\pgfpathmoveto{\pgfqpoint{0.000000in}{0.000000in}}%
\pgfpathlineto{\pgfqpoint{0.000000in}{-0.048611in}}%
\pgfusepath{stroke,fill}%
}%
\begin{pgfscope}%
\pgfsys@transformshift{4.296327in}{6.967719in}%
\pgfsys@useobject{currentmarker}{}%
\end{pgfscope}%
\end{pgfscope}%
\begin{pgfscope}%
\pgftext[x=4.296327in,y=6.870496in,,top]{\rmfamily\fontsize{12.000000}{14.400000}\selectfont \(\displaystyle 0.5\)}%
\end{pgfscope}%
\begin{pgfscope}%
\pgfsetbuttcap%
\pgfsetroundjoin%
\definecolor{currentfill}{rgb}{0.000000,0.000000,0.000000}%
\pgfsetfillcolor{currentfill}%
\pgfsetlinewidth{0.803000pt}%
\definecolor{currentstroke}{rgb}{0.000000,0.000000,0.000000}%
\pgfsetstrokecolor{currentstroke}%
\pgfsetdash{}{0pt}%
\pgfsys@defobject{currentmarker}{\pgfqpoint{0.000000in}{-0.048611in}}{\pgfqpoint{0.000000in}{0.000000in}}{%
\pgfpathmoveto{\pgfqpoint{0.000000in}{0.000000in}}%
\pgfpathlineto{\pgfqpoint{0.000000in}{-0.048611in}}%
\pgfusepath{stroke,fill}%
}%
\begin{pgfscope}%
\pgfsys@transformshift{5.245306in}{6.967719in}%
\pgfsys@useobject{currentmarker}{}%
\end{pgfscope}%
\end{pgfscope}%
\begin{pgfscope}%
\pgftext[x=5.245306in,y=6.870496in,,top]{\rmfamily\fontsize{12.000000}{14.400000}\selectfont \(\displaystyle 1.0\)}%
\end{pgfscope}%
\begin{pgfscope}%
\pgfsetbuttcap%
\pgfsetroundjoin%
\definecolor{currentfill}{rgb}{0.000000,0.000000,0.000000}%
\pgfsetfillcolor{currentfill}%
\pgfsetlinewidth{0.803000pt}%
\definecolor{currentstroke}{rgb}{0.000000,0.000000,0.000000}%
\pgfsetstrokecolor{currentstroke}%
\pgfsetdash{}{0pt}%
\pgfsys@defobject{currentmarker}{\pgfqpoint{-0.048611in}{0.000000in}}{\pgfqpoint{0.000000in}{0.000000in}}{%
\pgfpathmoveto{\pgfqpoint{0.000000in}{0.000000in}}%
\pgfpathlineto{\pgfqpoint{-0.048611in}{0.000000in}}%
\pgfusepath{stroke,fill}%
}%
\begin{pgfscope}%
\pgfsys@transformshift{3.347347in}{6.967719in}%
\pgfsys@useobject{currentmarker}{}%
\end{pgfscope}%
\end{pgfscope}%
\begin{pgfscope}%
\pgftext[x=2.960004in,y=6.904405in,left,base]{\rmfamily\fontsize{12.000000}{14.400000}\selectfont \(\displaystyle 0.00\)}%
\end{pgfscope}%
\begin{pgfscope}%
\pgfsetbuttcap%
\pgfsetroundjoin%
\definecolor{currentfill}{rgb}{0.000000,0.000000,0.000000}%
\pgfsetfillcolor{currentfill}%
\pgfsetlinewidth{0.803000pt}%
\definecolor{currentstroke}{rgb}{0.000000,0.000000,0.000000}%
\pgfsetstrokecolor{currentstroke}%
\pgfsetdash{}{0pt}%
\pgfsys@defobject{currentmarker}{\pgfqpoint{-0.048611in}{0.000000in}}{\pgfqpoint{0.000000in}{0.000000in}}{%
\pgfpathmoveto{\pgfqpoint{0.000000in}{0.000000in}}%
\pgfpathlineto{\pgfqpoint{-0.048611in}{0.000000in}}%
\pgfusepath{stroke,fill}%
}%
\begin{pgfscope}%
\pgfsys@transformshift{3.347347in}{7.310900in}%
\pgfsys@useobject{currentmarker}{}%
\end{pgfscope}%
\end{pgfscope}%
\begin{pgfscope}%
\pgftext[x=2.960004in,y=7.247587in,left,base]{\rmfamily\fontsize{12.000000}{14.400000}\selectfont \(\displaystyle 0.25\)}%
\end{pgfscope}%
\begin{pgfscope}%
\pgfsetbuttcap%
\pgfsetroundjoin%
\definecolor{currentfill}{rgb}{0.000000,0.000000,0.000000}%
\pgfsetfillcolor{currentfill}%
\pgfsetlinewidth{0.803000pt}%
\definecolor{currentstroke}{rgb}{0.000000,0.000000,0.000000}%
\pgfsetstrokecolor{currentstroke}%
\pgfsetdash{}{0pt}%
\pgfsys@defobject{currentmarker}{\pgfqpoint{-0.048611in}{0.000000in}}{\pgfqpoint{0.000000in}{0.000000in}}{%
\pgfpathmoveto{\pgfqpoint{0.000000in}{0.000000in}}%
\pgfpathlineto{\pgfqpoint{-0.048611in}{0.000000in}}%
\pgfusepath{stroke,fill}%
}%
\begin{pgfscope}%
\pgfsys@transformshift{3.347347in}{7.654082in}%
\pgfsys@useobject{currentmarker}{}%
\end{pgfscope}%
\end{pgfscope}%
\begin{pgfscope}%
\pgftext[x=2.960004in,y=7.590768in,left,base]{\rmfamily\fontsize{12.000000}{14.400000}\selectfont \(\displaystyle 0.50\)}%
\end{pgfscope}%
\begin{pgfscope}%
\pgfsetbuttcap%
\pgfsetroundjoin%
\definecolor{currentfill}{rgb}{0.000000,0.000000,0.000000}%
\pgfsetfillcolor{currentfill}%
\pgfsetlinewidth{0.803000pt}%
\definecolor{currentstroke}{rgb}{0.000000,0.000000,0.000000}%
\pgfsetstrokecolor{currentstroke}%
\pgfsetdash{}{0pt}%
\pgfsys@defobject{currentmarker}{\pgfqpoint{-0.048611in}{0.000000in}}{\pgfqpoint{0.000000in}{0.000000in}}{%
\pgfpathmoveto{\pgfqpoint{0.000000in}{0.000000in}}%
\pgfpathlineto{\pgfqpoint{-0.048611in}{0.000000in}}%
\pgfusepath{stroke,fill}%
}%
\begin{pgfscope}%
\pgfsys@transformshift{3.347347in}{7.997264in}%
\pgfsys@useobject{currentmarker}{}%
\end{pgfscope}%
\end{pgfscope}%
\begin{pgfscope}%
\pgftext[x=2.960004in,y=7.933950in,left,base]{\rmfamily\fontsize{12.000000}{14.400000}\selectfont \(\displaystyle 0.75\)}%
\end{pgfscope}%
\begin{pgfscope}%
\pgfsetbuttcap%
\pgfsetroundjoin%
\definecolor{currentfill}{rgb}{0.000000,0.000000,0.000000}%
\pgfsetfillcolor{currentfill}%
\pgfsetlinewidth{0.803000pt}%
\definecolor{currentstroke}{rgb}{0.000000,0.000000,0.000000}%
\pgfsetstrokecolor{currentstroke}%
\pgfsetdash{}{0pt}%
\pgfsys@defobject{currentmarker}{\pgfqpoint{-0.048611in}{0.000000in}}{\pgfqpoint{0.000000in}{0.000000in}}{%
\pgfpathmoveto{\pgfqpoint{0.000000in}{0.000000in}}%
\pgfpathlineto{\pgfqpoint{-0.048611in}{0.000000in}}%
\pgfusepath{stroke,fill}%
}%
\begin{pgfscope}%
\pgfsys@transformshift{3.347347in}{8.340446in}%
\pgfsys@useobject{currentmarker}{}%
\end{pgfscope}%
\end{pgfscope}%
\begin{pgfscope}%
\pgftext[x=2.960004in,y=8.277132in,left,base]{\rmfamily\fontsize{12.000000}{14.400000}\selectfont \(\displaystyle 1.00\)}%
\end{pgfscope}%
\begin{pgfscope}%
\pgfsetrectcap%
\pgfsetmiterjoin%
\pgfsetlinewidth{0.803000pt}%
\definecolor{currentstroke}{rgb}{0.501961,0.501961,0.501961}%
\pgfsetstrokecolor{currentstroke}%
\pgfsetdash{}{0pt}%
\pgfpathmoveto{\pgfqpoint{3.347347in}{6.967719in}}%
\pgfpathlineto{\pgfqpoint{3.347347in}{8.340446in}}%
\pgfusepath{stroke}%
\end{pgfscope}%
\begin{pgfscope}%
\pgfsetrectcap%
\pgfsetmiterjoin%
\pgfsetlinewidth{0.803000pt}%
\definecolor{currentstroke}{rgb}{0.501961,0.501961,0.501961}%
\pgfsetstrokecolor{currentstroke}%
\pgfsetdash{}{0pt}%
\pgfpathmoveto{\pgfqpoint{5.245306in}{6.967719in}}%
\pgfpathlineto{\pgfqpoint{5.245306in}{8.340446in}}%
\pgfusepath{stroke}%
\end{pgfscope}%
\begin{pgfscope}%
\pgfsetrectcap%
\pgfsetmiterjoin%
\pgfsetlinewidth{0.803000pt}%
\definecolor{currentstroke}{rgb}{0.501961,0.501961,0.501961}%
\pgfsetstrokecolor{currentstroke}%
\pgfsetdash{}{0pt}%
\pgfpathmoveto{\pgfqpoint{3.347347in}{6.967719in}}%
\pgfpathlineto{\pgfqpoint{5.245306in}{6.967719in}}%
\pgfusepath{stroke}%
\end{pgfscope}%
\begin{pgfscope}%
\pgfsetrectcap%
\pgfsetmiterjoin%
\pgfsetlinewidth{0.803000pt}%
\definecolor{currentstroke}{rgb}{0.501961,0.501961,0.501961}%
\pgfsetstrokecolor{currentstroke}%
\pgfsetdash{}{0pt}%
\pgfpathmoveto{\pgfqpoint{3.347347in}{8.340446in}}%
\pgfpathlineto{\pgfqpoint{5.245306in}{8.340446in}}%
\pgfusepath{stroke}%
\end{pgfscope}%
\begin{pgfscope}%
\pgfsetbuttcap%
\pgfsetmiterjoin%
\definecolor{currentfill}{rgb}{1.000000,1.000000,1.000000}%
\pgfsetfillcolor{currentfill}%
\pgfsetlinewidth{0.000000pt}%
\definecolor{currentstroke}{rgb}{0.000000,0.000000,0.000000}%
\pgfsetstrokecolor{currentstroke}%
\pgfsetstrokeopacity{0.000000}%
\pgfsetdash{}{0pt}%
\pgfpathmoveto{\pgfqpoint{5.814694in}{6.967719in}}%
\pgfpathlineto{\pgfqpoint{7.712653in}{6.967719in}}%
\pgfpathlineto{\pgfqpoint{7.712653in}{8.340446in}}%
\pgfpathlineto{\pgfqpoint{5.814694in}{8.340446in}}%
\pgfpathclose%
\pgfusepath{fill}%
\end{pgfscope}%
\begin{pgfscope}%
\pgfsetbuttcap%
\pgfsetroundjoin%
\definecolor{currentfill}{rgb}{0.000000,0.000000,0.000000}%
\pgfsetfillcolor{currentfill}%
\pgfsetlinewidth{0.803000pt}%
\definecolor{currentstroke}{rgb}{0.000000,0.000000,0.000000}%
\pgfsetstrokecolor{currentstroke}%
\pgfsetdash{}{0pt}%
\pgfsys@defobject{currentmarker}{\pgfqpoint{0.000000in}{-0.048611in}}{\pgfqpoint{0.000000in}{0.000000in}}{%
\pgfpathmoveto{\pgfqpoint{0.000000in}{0.000000in}}%
\pgfpathlineto{\pgfqpoint{0.000000in}{-0.048611in}}%
\pgfusepath{stroke,fill}%
}%
\begin{pgfscope}%
\pgfsys@transformshift{5.814694in}{6.967719in}%
\pgfsys@useobject{currentmarker}{}%
\end{pgfscope}%
\end{pgfscope}%
\begin{pgfscope}%
\pgftext[x=5.814694in,y=6.870496in,,top]{\rmfamily\fontsize{12.000000}{14.400000}\selectfont \(\displaystyle 0.0\)}%
\end{pgfscope}%
\begin{pgfscope}%
\pgfsetbuttcap%
\pgfsetroundjoin%
\definecolor{currentfill}{rgb}{0.000000,0.000000,0.000000}%
\pgfsetfillcolor{currentfill}%
\pgfsetlinewidth{0.803000pt}%
\definecolor{currentstroke}{rgb}{0.000000,0.000000,0.000000}%
\pgfsetstrokecolor{currentstroke}%
\pgfsetdash{}{0pt}%
\pgfsys@defobject{currentmarker}{\pgfqpoint{0.000000in}{-0.048611in}}{\pgfqpoint{0.000000in}{0.000000in}}{%
\pgfpathmoveto{\pgfqpoint{0.000000in}{0.000000in}}%
\pgfpathlineto{\pgfqpoint{0.000000in}{-0.048611in}}%
\pgfusepath{stroke,fill}%
}%
\begin{pgfscope}%
\pgfsys@transformshift{6.763673in}{6.967719in}%
\pgfsys@useobject{currentmarker}{}%
\end{pgfscope}%
\end{pgfscope}%
\begin{pgfscope}%
\pgftext[x=6.763673in,y=6.870496in,,top]{\rmfamily\fontsize{12.000000}{14.400000}\selectfont \(\displaystyle 0.5\)}%
\end{pgfscope}%
\begin{pgfscope}%
\pgfsetbuttcap%
\pgfsetroundjoin%
\definecolor{currentfill}{rgb}{0.000000,0.000000,0.000000}%
\pgfsetfillcolor{currentfill}%
\pgfsetlinewidth{0.803000pt}%
\definecolor{currentstroke}{rgb}{0.000000,0.000000,0.000000}%
\pgfsetstrokecolor{currentstroke}%
\pgfsetdash{}{0pt}%
\pgfsys@defobject{currentmarker}{\pgfqpoint{0.000000in}{-0.048611in}}{\pgfqpoint{0.000000in}{0.000000in}}{%
\pgfpathmoveto{\pgfqpoint{0.000000in}{0.000000in}}%
\pgfpathlineto{\pgfqpoint{0.000000in}{-0.048611in}}%
\pgfusepath{stroke,fill}%
}%
\begin{pgfscope}%
\pgfsys@transformshift{7.712653in}{6.967719in}%
\pgfsys@useobject{currentmarker}{}%
\end{pgfscope}%
\end{pgfscope}%
\begin{pgfscope}%
\pgftext[x=7.712653in,y=6.870496in,,top]{\rmfamily\fontsize{12.000000}{14.400000}\selectfont \(\displaystyle 1.0\)}%
\end{pgfscope}%
\begin{pgfscope}%
\pgfsetbuttcap%
\pgfsetroundjoin%
\definecolor{currentfill}{rgb}{0.000000,0.000000,0.000000}%
\pgfsetfillcolor{currentfill}%
\pgfsetlinewidth{0.803000pt}%
\definecolor{currentstroke}{rgb}{0.000000,0.000000,0.000000}%
\pgfsetstrokecolor{currentstroke}%
\pgfsetdash{}{0pt}%
\pgfsys@defobject{currentmarker}{\pgfqpoint{-0.048611in}{0.000000in}}{\pgfqpoint{0.000000in}{0.000000in}}{%
\pgfpathmoveto{\pgfqpoint{0.000000in}{0.000000in}}%
\pgfpathlineto{\pgfqpoint{-0.048611in}{0.000000in}}%
\pgfusepath{stroke,fill}%
}%
\begin{pgfscope}%
\pgfsys@transformshift{5.814694in}{6.967719in}%
\pgfsys@useobject{currentmarker}{}%
\end{pgfscope}%
\end{pgfscope}%
\begin{pgfscope}%
\pgftext[x=5.427351in,y=6.904405in,left,base]{\rmfamily\fontsize{12.000000}{14.400000}\selectfont \(\displaystyle 0.00\)}%
\end{pgfscope}%
\begin{pgfscope}%
\pgfsetbuttcap%
\pgfsetroundjoin%
\definecolor{currentfill}{rgb}{0.000000,0.000000,0.000000}%
\pgfsetfillcolor{currentfill}%
\pgfsetlinewidth{0.803000pt}%
\definecolor{currentstroke}{rgb}{0.000000,0.000000,0.000000}%
\pgfsetstrokecolor{currentstroke}%
\pgfsetdash{}{0pt}%
\pgfsys@defobject{currentmarker}{\pgfqpoint{-0.048611in}{0.000000in}}{\pgfqpoint{0.000000in}{0.000000in}}{%
\pgfpathmoveto{\pgfqpoint{0.000000in}{0.000000in}}%
\pgfpathlineto{\pgfqpoint{-0.048611in}{0.000000in}}%
\pgfusepath{stroke,fill}%
}%
\begin{pgfscope}%
\pgfsys@transformshift{5.814694in}{7.310900in}%
\pgfsys@useobject{currentmarker}{}%
\end{pgfscope}%
\end{pgfscope}%
\begin{pgfscope}%
\pgftext[x=5.427351in,y=7.247587in,left,base]{\rmfamily\fontsize{12.000000}{14.400000}\selectfont \(\displaystyle 0.25\)}%
\end{pgfscope}%
\begin{pgfscope}%
\pgfsetbuttcap%
\pgfsetroundjoin%
\definecolor{currentfill}{rgb}{0.000000,0.000000,0.000000}%
\pgfsetfillcolor{currentfill}%
\pgfsetlinewidth{0.803000pt}%
\definecolor{currentstroke}{rgb}{0.000000,0.000000,0.000000}%
\pgfsetstrokecolor{currentstroke}%
\pgfsetdash{}{0pt}%
\pgfsys@defobject{currentmarker}{\pgfqpoint{-0.048611in}{0.000000in}}{\pgfqpoint{0.000000in}{0.000000in}}{%
\pgfpathmoveto{\pgfqpoint{0.000000in}{0.000000in}}%
\pgfpathlineto{\pgfqpoint{-0.048611in}{0.000000in}}%
\pgfusepath{stroke,fill}%
}%
\begin{pgfscope}%
\pgfsys@transformshift{5.814694in}{7.654082in}%
\pgfsys@useobject{currentmarker}{}%
\end{pgfscope}%
\end{pgfscope}%
\begin{pgfscope}%
\pgftext[x=5.427351in,y=7.590768in,left,base]{\rmfamily\fontsize{12.000000}{14.400000}\selectfont \(\displaystyle 0.50\)}%
\end{pgfscope}%
\begin{pgfscope}%
\pgfsetbuttcap%
\pgfsetroundjoin%
\definecolor{currentfill}{rgb}{0.000000,0.000000,0.000000}%
\pgfsetfillcolor{currentfill}%
\pgfsetlinewidth{0.803000pt}%
\definecolor{currentstroke}{rgb}{0.000000,0.000000,0.000000}%
\pgfsetstrokecolor{currentstroke}%
\pgfsetdash{}{0pt}%
\pgfsys@defobject{currentmarker}{\pgfqpoint{-0.048611in}{0.000000in}}{\pgfqpoint{0.000000in}{0.000000in}}{%
\pgfpathmoveto{\pgfqpoint{0.000000in}{0.000000in}}%
\pgfpathlineto{\pgfqpoint{-0.048611in}{0.000000in}}%
\pgfusepath{stroke,fill}%
}%
\begin{pgfscope}%
\pgfsys@transformshift{5.814694in}{7.997264in}%
\pgfsys@useobject{currentmarker}{}%
\end{pgfscope}%
\end{pgfscope}%
\begin{pgfscope}%
\pgftext[x=5.427351in,y=7.933950in,left,base]{\rmfamily\fontsize{12.000000}{14.400000}\selectfont \(\displaystyle 0.75\)}%
\end{pgfscope}%
\begin{pgfscope}%
\pgfsetbuttcap%
\pgfsetroundjoin%
\definecolor{currentfill}{rgb}{0.000000,0.000000,0.000000}%
\pgfsetfillcolor{currentfill}%
\pgfsetlinewidth{0.803000pt}%
\definecolor{currentstroke}{rgb}{0.000000,0.000000,0.000000}%
\pgfsetstrokecolor{currentstroke}%
\pgfsetdash{}{0pt}%
\pgfsys@defobject{currentmarker}{\pgfqpoint{-0.048611in}{0.000000in}}{\pgfqpoint{0.000000in}{0.000000in}}{%
\pgfpathmoveto{\pgfqpoint{0.000000in}{0.000000in}}%
\pgfpathlineto{\pgfqpoint{-0.048611in}{0.000000in}}%
\pgfusepath{stroke,fill}%
}%
\begin{pgfscope}%
\pgfsys@transformshift{5.814694in}{8.340446in}%
\pgfsys@useobject{currentmarker}{}%
\end{pgfscope}%
\end{pgfscope}%
\begin{pgfscope}%
\pgftext[x=5.427351in,y=8.277132in,left,base]{\rmfamily\fontsize{12.000000}{14.400000}\selectfont \(\displaystyle 1.00\)}%
\end{pgfscope}%
\begin{pgfscope}%
\pgfsetrectcap%
\pgfsetmiterjoin%
\pgfsetlinewidth{0.803000pt}%
\definecolor{currentstroke}{rgb}{0.501961,0.501961,0.501961}%
\pgfsetstrokecolor{currentstroke}%
\pgfsetdash{}{0pt}%
\pgfpathmoveto{\pgfqpoint{5.814694in}{6.967719in}}%
\pgfpathlineto{\pgfqpoint{5.814694in}{8.340446in}}%
\pgfusepath{stroke}%
\end{pgfscope}%
\begin{pgfscope}%
\pgfsetrectcap%
\pgfsetmiterjoin%
\pgfsetlinewidth{0.803000pt}%
\definecolor{currentstroke}{rgb}{0.501961,0.501961,0.501961}%
\pgfsetstrokecolor{currentstroke}%
\pgfsetdash{}{0pt}%
\pgfpathmoveto{\pgfqpoint{7.712653in}{6.967719in}}%
\pgfpathlineto{\pgfqpoint{7.712653in}{8.340446in}}%
\pgfusepath{stroke}%
\end{pgfscope}%
\begin{pgfscope}%
\pgfsetrectcap%
\pgfsetmiterjoin%
\pgfsetlinewidth{0.803000pt}%
\definecolor{currentstroke}{rgb}{0.501961,0.501961,0.501961}%
\pgfsetstrokecolor{currentstroke}%
\pgfsetdash{}{0pt}%
\pgfpathmoveto{\pgfqpoint{5.814694in}{6.967719in}}%
\pgfpathlineto{\pgfqpoint{7.712653in}{6.967719in}}%
\pgfusepath{stroke}%
\end{pgfscope}%
\begin{pgfscope}%
\pgfsetrectcap%
\pgfsetmiterjoin%
\pgfsetlinewidth{0.803000pt}%
\definecolor{currentstroke}{rgb}{0.501961,0.501961,0.501961}%
\pgfsetstrokecolor{currentstroke}%
\pgfsetdash{}{0pt}%
\pgfpathmoveto{\pgfqpoint{5.814694in}{8.340446in}}%
\pgfpathlineto{\pgfqpoint{7.712653in}{8.340446in}}%
\pgfusepath{stroke}%
\end{pgfscope}%
\begin{pgfscope}%
\pgfsetbuttcap%
\pgfsetmiterjoin%
\definecolor{currentfill}{rgb}{1.000000,1.000000,1.000000}%
\pgfsetfillcolor{currentfill}%
\pgfsetlinewidth{0.000000pt}%
\definecolor{currentstroke}{rgb}{0.000000,0.000000,0.000000}%
\pgfsetstrokecolor{currentstroke}%
\pgfsetstrokeopacity{0.000000}%
\pgfsetdash{}{0pt}%
\pgfpathmoveto{\pgfqpoint{8.282041in}{6.967719in}}%
\pgfpathlineto{\pgfqpoint{10.180000in}{6.967719in}}%
\pgfpathlineto{\pgfqpoint{10.180000in}{8.340446in}}%
\pgfpathlineto{\pgfqpoint{8.282041in}{8.340446in}}%
\pgfpathclose%
\pgfusepath{fill}%
\end{pgfscope}%
\begin{pgfscope}%
\pgfsetbuttcap%
\pgfsetroundjoin%
\definecolor{currentfill}{rgb}{0.000000,0.000000,0.000000}%
\pgfsetfillcolor{currentfill}%
\pgfsetlinewidth{0.803000pt}%
\definecolor{currentstroke}{rgb}{0.000000,0.000000,0.000000}%
\pgfsetstrokecolor{currentstroke}%
\pgfsetdash{}{0pt}%
\pgfsys@defobject{currentmarker}{\pgfqpoint{0.000000in}{-0.048611in}}{\pgfqpoint{0.000000in}{0.000000in}}{%
\pgfpathmoveto{\pgfqpoint{0.000000in}{0.000000in}}%
\pgfpathlineto{\pgfqpoint{0.000000in}{-0.048611in}}%
\pgfusepath{stroke,fill}%
}%
\begin{pgfscope}%
\pgfsys@transformshift{8.282041in}{6.967719in}%
\pgfsys@useobject{currentmarker}{}%
\end{pgfscope}%
\end{pgfscope}%
\begin{pgfscope}%
\pgftext[x=8.282041in,y=6.870496in,,top]{\rmfamily\fontsize{12.000000}{14.400000}\selectfont \(\displaystyle 0.0\)}%
\end{pgfscope}%
\begin{pgfscope}%
\pgfsetbuttcap%
\pgfsetroundjoin%
\definecolor{currentfill}{rgb}{0.000000,0.000000,0.000000}%
\pgfsetfillcolor{currentfill}%
\pgfsetlinewidth{0.803000pt}%
\definecolor{currentstroke}{rgb}{0.000000,0.000000,0.000000}%
\pgfsetstrokecolor{currentstroke}%
\pgfsetdash{}{0pt}%
\pgfsys@defobject{currentmarker}{\pgfqpoint{0.000000in}{-0.048611in}}{\pgfqpoint{0.000000in}{0.000000in}}{%
\pgfpathmoveto{\pgfqpoint{0.000000in}{0.000000in}}%
\pgfpathlineto{\pgfqpoint{0.000000in}{-0.048611in}}%
\pgfusepath{stroke,fill}%
}%
\begin{pgfscope}%
\pgfsys@transformshift{9.231020in}{6.967719in}%
\pgfsys@useobject{currentmarker}{}%
\end{pgfscope}%
\end{pgfscope}%
\begin{pgfscope}%
\pgftext[x=9.231020in,y=6.870496in,,top]{\rmfamily\fontsize{12.000000}{14.400000}\selectfont \(\displaystyle 0.5\)}%
\end{pgfscope}%
\begin{pgfscope}%
\pgfsetbuttcap%
\pgfsetroundjoin%
\definecolor{currentfill}{rgb}{0.000000,0.000000,0.000000}%
\pgfsetfillcolor{currentfill}%
\pgfsetlinewidth{0.803000pt}%
\definecolor{currentstroke}{rgb}{0.000000,0.000000,0.000000}%
\pgfsetstrokecolor{currentstroke}%
\pgfsetdash{}{0pt}%
\pgfsys@defobject{currentmarker}{\pgfqpoint{0.000000in}{-0.048611in}}{\pgfqpoint{0.000000in}{0.000000in}}{%
\pgfpathmoveto{\pgfqpoint{0.000000in}{0.000000in}}%
\pgfpathlineto{\pgfqpoint{0.000000in}{-0.048611in}}%
\pgfusepath{stroke,fill}%
}%
\begin{pgfscope}%
\pgfsys@transformshift{10.180000in}{6.967719in}%
\pgfsys@useobject{currentmarker}{}%
\end{pgfscope}%
\end{pgfscope}%
\begin{pgfscope}%
\pgftext[x=10.180000in,y=6.870496in,,top]{\rmfamily\fontsize{12.000000}{14.400000}\selectfont \(\displaystyle 1.0\)}%
\end{pgfscope}%
\begin{pgfscope}%
\pgfsetbuttcap%
\pgfsetroundjoin%
\definecolor{currentfill}{rgb}{0.000000,0.000000,0.000000}%
\pgfsetfillcolor{currentfill}%
\pgfsetlinewidth{0.803000pt}%
\definecolor{currentstroke}{rgb}{0.000000,0.000000,0.000000}%
\pgfsetstrokecolor{currentstroke}%
\pgfsetdash{}{0pt}%
\pgfsys@defobject{currentmarker}{\pgfqpoint{-0.048611in}{0.000000in}}{\pgfqpoint{0.000000in}{0.000000in}}{%
\pgfpathmoveto{\pgfqpoint{0.000000in}{0.000000in}}%
\pgfpathlineto{\pgfqpoint{-0.048611in}{0.000000in}}%
\pgfusepath{stroke,fill}%
}%
\begin{pgfscope}%
\pgfsys@transformshift{8.282041in}{6.967719in}%
\pgfsys@useobject{currentmarker}{}%
\end{pgfscope}%
\end{pgfscope}%
\begin{pgfscope}%
\pgftext[x=7.894698in,y=6.904405in,left,base]{\rmfamily\fontsize{12.000000}{14.400000}\selectfont \(\displaystyle 0.00\)}%
\end{pgfscope}%
\begin{pgfscope}%
\pgfsetbuttcap%
\pgfsetroundjoin%
\definecolor{currentfill}{rgb}{0.000000,0.000000,0.000000}%
\pgfsetfillcolor{currentfill}%
\pgfsetlinewidth{0.803000pt}%
\definecolor{currentstroke}{rgb}{0.000000,0.000000,0.000000}%
\pgfsetstrokecolor{currentstroke}%
\pgfsetdash{}{0pt}%
\pgfsys@defobject{currentmarker}{\pgfqpoint{-0.048611in}{0.000000in}}{\pgfqpoint{0.000000in}{0.000000in}}{%
\pgfpathmoveto{\pgfqpoint{0.000000in}{0.000000in}}%
\pgfpathlineto{\pgfqpoint{-0.048611in}{0.000000in}}%
\pgfusepath{stroke,fill}%
}%
\begin{pgfscope}%
\pgfsys@transformshift{8.282041in}{7.310900in}%
\pgfsys@useobject{currentmarker}{}%
\end{pgfscope}%
\end{pgfscope}%
\begin{pgfscope}%
\pgftext[x=7.894698in,y=7.247587in,left,base]{\rmfamily\fontsize{12.000000}{14.400000}\selectfont \(\displaystyle 0.25\)}%
\end{pgfscope}%
\begin{pgfscope}%
\pgfsetbuttcap%
\pgfsetroundjoin%
\definecolor{currentfill}{rgb}{0.000000,0.000000,0.000000}%
\pgfsetfillcolor{currentfill}%
\pgfsetlinewidth{0.803000pt}%
\definecolor{currentstroke}{rgb}{0.000000,0.000000,0.000000}%
\pgfsetstrokecolor{currentstroke}%
\pgfsetdash{}{0pt}%
\pgfsys@defobject{currentmarker}{\pgfqpoint{-0.048611in}{0.000000in}}{\pgfqpoint{0.000000in}{0.000000in}}{%
\pgfpathmoveto{\pgfqpoint{0.000000in}{0.000000in}}%
\pgfpathlineto{\pgfqpoint{-0.048611in}{0.000000in}}%
\pgfusepath{stroke,fill}%
}%
\begin{pgfscope}%
\pgfsys@transformshift{8.282041in}{7.654082in}%
\pgfsys@useobject{currentmarker}{}%
\end{pgfscope}%
\end{pgfscope}%
\begin{pgfscope}%
\pgftext[x=7.894698in,y=7.590768in,left,base]{\rmfamily\fontsize{12.000000}{14.400000}\selectfont \(\displaystyle 0.50\)}%
\end{pgfscope}%
\begin{pgfscope}%
\pgfsetbuttcap%
\pgfsetroundjoin%
\definecolor{currentfill}{rgb}{0.000000,0.000000,0.000000}%
\pgfsetfillcolor{currentfill}%
\pgfsetlinewidth{0.803000pt}%
\definecolor{currentstroke}{rgb}{0.000000,0.000000,0.000000}%
\pgfsetstrokecolor{currentstroke}%
\pgfsetdash{}{0pt}%
\pgfsys@defobject{currentmarker}{\pgfqpoint{-0.048611in}{0.000000in}}{\pgfqpoint{0.000000in}{0.000000in}}{%
\pgfpathmoveto{\pgfqpoint{0.000000in}{0.000000in}}%
\pgfpathlineto{\pgfqpoint{-0.048611in}{0.000000in}}%
\pgfusepath{stroke,fill}%
}%
\begin{pgfscope}%
\pgfsys@transformshift{8.282041in}{7.997264in}%
\pgfsys@useobject{currentmarker}{}%
\end{pgfscope}%
\end{pgfscope}%
\begin{pgfscope}%
\pgftext[x=7.894698in,y=7.933950in,left,base]{\rmfamily\fontsize{12.000000}{14.400000}\selectfont \(\displaystyle 0.75\)}%
\end{pgfscope}%
\begin{pgfscope}%
\pgfsetbuttcap%
\pgfsetroundjoin%
\definecolor{currentfill}{rgb}{0.000000,0.000000,0.000000}%
\pgfsetfillcolor{currentfill}%
\pgfsetlinewidth{0.803000pt}%
\definecolor{currentstroke}{rgb}{0.000000,0.000000,0.000000}%
\pgfsetstrokecolor{currentstroke}%
\pgfsetdash{}{0pt}%
\pgfsys@defobject{currentmarker}{\pgfqpoint{-0.048611in}{0.000000in}}{\pgfqpoint{0.000000in}{0.000000in}}{%
\pgfpathmoveto{\pgfqpoint{0.000000in}{0.000000in}}%
\pgfpathlineto{\pgfqpoint{-0.048611in}{0.000000in}}%
\pgfusepath{stroke,fill}%
}%
\begin{pgfscope}%
\pgfsys@transformshift{8.282041in}{8.340446in}%
\pgfsys@useobject{currentmarker}{}%
\end{pgfscope}%
\end{pgfscope}%
\begin{pgfscope}%
\pgftext[x=7.894698in,y=8.277132in,left,base]{\rmfamily\fontsize{12.000000}{14.400000}\selectfont \(\displaystyle 1.00\)}%
\end{pgfscope}%
\begin{pgfscope}%
\pgfsetrectcap%
\pgfsetmiterjoin%
\pgfsetlinewidth{0.803000pt}%
\definecolor{currentstroke}{rgb}{0.501961,0.501961,0.501961}%
\pgfsetstrokecolor{currentstroke}%
\pgfsetdash{}{0pt}%
\pgfpathmoveto{\pgfqpoint{8.282041in}{6.967719in}}%
\pgfpathlineto{\pgfqpoint{8.282041in}{8.340446in}}%
\pgfusepath{stroke}%
\end{pgfscope}%
\begin{pgfscope}%
\pgfsetrectcap%
\pgfsetmiterjoin%
\pgfsetlinewidth{0.803000pt}%
\definecolor{currentstroke}{rgb}{0.501961,0.501961,0.501961}%
\pgfsetstrokecolor{currentstroke}%
\pgfsetdash{}{0pt}%
\pgfpathmoveto{\pgfqpoint{10.180000in}{6.967719in}}%
\pgfpathlineto{\pgfqpoint{10.180000in}{8.340446in}}%
\pgfusepath{stroke}%
\end{pgfscope}%
\begin{pgfscope}%
\pgfsetrectcap%
\pgfsetmiterjoin%
\pgfsetlinewidth{0.803000pt}%
\definecolor{currentstroke}{rgb}{0.501961,0.501961,0.501961}%
\pgfsetstrokecolor{currentstroke}%
\pgfsetdash{}{0pt}%
\pgfpathmoveto{\pgfqpoint{8.282041in}{6.967719in}}%
\pgfpathlineto{\pgfqpoint{10.180000in}{6.967719in}}%
\pgfusepath{stroke}%
\end{pgfscope}%
\begin{pgfscope}%
\pgfsetrectcap%
\pgfsetmiterjoin%
\pgfsetlinewidth{0.803000pt}%
\definecolor{currentstroke}{rgb}{0.501961,0.501961,0.501961}%
\pgfsetstrokecolor{currentstroke}%
\pgfsetdash{}{0pt}%
\pgfpathmoveto{\pgfqpoint{8.282041in}{8.340446in}}%
\pgfpathlineto{\pgfqpoint{10.180000in}{8.340446in}}%
\pgfusepath{stroke}%
\end{pgfscope}%
\begin{pgfscope}%
\pgfsetbuttcap%
\pgfsetmiterjoin%
\definecolor{currentfill}{rgb}{1.000000,1.000000,1.000000}%
\pgfsetfillcolor{currentfill}%
\pgfsetlinewidth{0.000000pt}%
\definecolor{currentstroke}{rgb}{0.000000,0.000000,0.000000}%
\pgfsetstrokecolor{currentstroke}%
\pgfsetstrokeopacity{0.000000}%
\pgfsetdash{}{0pt}%
\pgfpathmoveto{\pgfqpoint{0.880000in}{4.908628in}}%
\pgfpathlineto{\pgfqpoint{2.777959in}{4.908628in}}%
\pgfpathlineto{\pgfqpoint{2.777959in}{6.281355in}}%
\pgfpathlineto{\pgfqpoint{0.880000in}{6.281355in}}%
\pgfpathclose%
\pgfusepath{fill}%
\end{pgfscope}%
\begin{pgfscope}%
\pgfsetbuttcap%
\pgfsetroundjoin%
\definecolor{currentfill}{rgb}{0.000000,0.000000,0.000000}%
\pgfsetfillcolor{currentfill}%
\pgfsetlinewidth{0.803000pt}%
\definecolor{currentstroke}{rgb}{0.000000,0.000000,0.000000}%
\pgfsetstrokecolor{currentstroke}%
\pgfsetdash{}{0pt}%
\pgfsys@defobject{currentmarker}{\pgfqpoint{0.000000in}{-0.048611in}}{\pgfqpoint{0.000000in}{0.000000in}}{%
\pgfpathmoveto{\pgfqpoint{0.000000in}{0.000000in}}%
\pgfpathlineto{\pgfqpoint{0.000000in}{-0.048611in}}%
\pgfusepath{stroke,fill}%
}%
\begin{pgfscope}%
\pgfsys@transformshift{0.880000in}{4.908628in}%
\pgfsys@useobject{currentmarker}{}%
\end{pgfscope}%
\end{pgfscope}%
\begin{pgfscope}%
\pgftext[x=0.880000in,y=4.811405in,,top]{\rmfamily\fontsize{12.000000}{14.400000}\selectfont \(\displaystyle 0.0\)}%
\end{pgfscope}%
\begin{pgfscope}%
\pgfsetbuttcap%
\pgfsetroundjoin%
\definecolor{currentfill}{rgb}{0.000000,0.000000,0.000000}%
\pgfsetfillcolor{currentfill}%
\pgfsetlinewidth{0.803000pt}%
\definecolor{currentstroke}{rgb}{0.000000,0.000000,0.000000}%
\pgfsetstrokecolor{currentstroke}%
\pgfsetdash{}{0pt}%
\pgfsys@defobject{currentmarker}{\pgfqpoint{0.000000in}{-0.048611in}}{\pgfqpoint{0.000000in}{0.000000in}}{%
\pgfpathmoveto{\pgfqpoint{0.000000in}{0.000000in}}%
\pgfpathlineto{\pgfqpoint{0.000000in}{-0.048611in}}%
\pgfusepath{stroke,fill}%
}%
\begin{pgfscope}%
\pgfsys@transformshift{1.828980in}{4.908628in}%
\pgfsys@useobject{currentmarker}{}%
\end{pgfscope}%
\end{pgfscope}%
\begin{pgfscope}%
\pgftext[x=1.828980in,y=4.811405in,,top]{\rmfamily\fontsize{12.000000}{14.400000}\selectfont \(\displaystyle 0.5\)}%
\end{pgfscope}%
\begin{pgfscope}%
\pgfsetbuttcap%
\pgfsetroundjoin%
\definecolor{currentfill}{rgb}{0.000000,0.000000,0.000000}%
\pgfsetfillcolor{currentfill}%
\pgfsetlinewidth{0.803000pt}%
\definecolor{currentstroke}{rgb}{0.000000,0.000000,0.000000}%
\pgfsetstrokecolor{currentstroke}%
\pgfsetdash{}{0pt}%
\pgfsys@defobject{currentmarker}{\pgfqpoint{0.000000in}{-0.048611in}}{\pgfqpoint{0.000000in}{0.000000in}}{%
\pgfpathmoveto{\pgfqpoint{0.000000in}{0.000000in}}%
\pgfpathlineto{\pgfqpoint{0.000000in}{-0.048611in}}%
\pgfusepath{stroke,fill}%
}%
\begin{pgfscope}%
\pgfsys@transformshift{2.777959in}{4.908628in}%
\pgfsys@useobject{currentmarker}{}%
\end{pgfscope}%
\end{pgfscope}%
\begin{pgfscope}%
\pgftext[x=2.777959in,y=4.811405in,,top]{\rmfamily\fontsize{12.000000}{14.400000}\selectfont \(\displaystyle 1.0\)}%
\end{pgfscope}%
\begin{pgfscope}%
\pgfsetbuttcap%
\pgfsetroundjoin%
\definecolor{currentfill}{rgb}{0.000000,0.000000,0.000000}%
\pgfsetfillcolor{currentfill}%
\pgfsetlinewidth{0.803000pt}%
\definecolor{currentstroke}{rgb}{0.000000,0.000000,0.000000}%
\pgfsetstrokecolor{currentstroke}%
\pgfsetdash{}{0pt}%
\pgfsys@defobject{currentmarker}{\pgfqpoint{-0.048611in}{0.000000in}}{\pgfqpoint{0.000000in}{0.000000in}}{%
\pgfpathmoveto{\pgfqpoint{0.000000in}{0.000000in}}%
\pgfpathlineto{\pgfqpoint{-0.048611in}{0.000000in}}%
\pgfusepath{stroke,fill}%
}%
\begin{pgfscope}%
\pgfsys@transformshift{0.880000in}{4.908628in}%
\pgfsys@useobject{currentmarker}{}%
\end{pgfscope}%
\end{pgfscope}%
\begin{pgfscope}%
\pgftext[x=0.492657in,y=4.845314in,left,base]{\rmfamily\fontsize{12.000000}{14.400000}\selectfont \(\displaystyle 0.00\)}%
\end{pgfscope}%
\begin{pgfscope}%
\pgfsetbuttcap%
\pgfsetroundjoin%
\definecolor{currentfill}{rgb}{0.000000,0.000000,0.000000}%
\pgfsetfillcolor{currentfill}%
\pgfsetlinewidth{0.803000pt}%
\definecolor{currentstroke}{rgb}{0.000000,0.000000,0.000000}%
\pgfsetstrokecolor{currentstroke}%
\pgfsetdash{}{0pt}%
\pgfsys@defobject{currentmarker}{\pgfqpoint{-0.048611in}{0.000000in}}{\pgfqpoint{0.000000in}{0.000000in}}{%
\pgfpathmoveto{\pgfqpoint{0.000000in}{0.000000in}}%
\pgfpathlineto{\pgfqpoint{-0.048611in}{0.000000in}}%
\pgfusepath{stroke,fill}%
}%
\begin{pgfscope}%
\pgfsys@transformshift{0.880000in}{5.251809in}%
\pgfsys@useobject{currentmarker}{}%
\end{pgfscope}%
\end{pgfscope}%
\begin{pgfscope}%
\pgftext[x=0.492657in,y=5.188496in,left,base]{\rmfamily\fontsize{12.000000}{14.400000}\selectfont \(\displaystyle 0.25\)}%
\end{pgfscope}%
\begin{pgfscope}%
\pgfsetbuttcap%
\pgfsetroundjoin%
\definecolor{currentfill}{rgb}{0.000000,0.000000,0.000000}%
\pgfsetfillcolor{currentfill}%
\pgfsetlinewidth{0.803000pt}%
\definecolor{currentstroke}{rgb}{0.000000,0.000000,0.000000}%
\pgfsetstrokecolor{currentstroke}%
\pgfsetdash{}{0pt}%
\pgfsys@defobject{currentmarker}{\pgfqpoint{-0.048611in}{0.000000in}}{\pgfqpoint{0.000000in}{0.000000in}}{%
\pgfpathmoveto{\pgfqpoint{0.000000in}{0.000000in}}%
\pgfpathlineto{\pgfqpoint{-0.048611in}{0.000000in}}%
\pgfusepath{stroke,fill}%
}%
\begin{pgfscope}%
\pgfsys@transformshift{0.880000in}{5.594991in}%
\pgfsys@useobject{currentmarker}{}%
\end{pgfscope}%
\end{pgfscope}%
\begin{pgfscope}%
\pgftext[x=0.492657in,y=5.531677in,left,base]{\rmfamily\fontsize{12.000000}{14.400000}\selectfont \(\displaystyle 0.50\)}%
\end{pgfscope}%
\begin{pgfscope}%
\pgfsetbuttcap%
\pgfsetroundjoin%
\definecolor{currentfill}{rgb}{0.000000,0.000000,0.000000}%
\pgfsetfillcolor{currentfill}%
\pgfsetlinewidth{0.803000pt}%
\definecolor{currentstroke}{rgb}{0.000000,0.000000,0.000000}%
\pgfsetstrokecolor{currentstroke}%
\pgfsetdash{}{0pt}%
\pgfsys@defobject{currentmarker}{\pgfqpoint{-0.048611in}{0.000000in}}{\pgfqpoint{0.000000in}{0.000000in}}{%
\pgfpathmoveto{\pgfqpoint{0.000000in}{0.000000in}}%
\pgfpathlineto{\pgfqpoint{-0.048611in}{0.000000in}}%
\pgfusepath{stroke,fill}%
}%
\begin{pgfscope}%
\pgfsys@transformshift{0.880000in}{5.938173in}%
\pgfsys@useobject{currentmarker}{}%
\end{pgfscope}%
\end{pgfscope}%
\begin{pgfscope}%
\pgftext[x=0.492657in,y=5.874859in,left,base]{\rmfamily\fontsize{12.000000}{14.400000}\selectfont \(\displaystyle 0.75\)}%
\end{pgfscope}%
\begin{pgfscope}%
\pgfsetbuttcap%
\pgfsetroundjoin%
\definecolor{currentfill}{rgb}{0.000000,0.000000,0.000000}%
\pgfsetfillcolor{currentfill}%
\pgfsetlinewidth{0.803000pt}%
\definecolor{currentstroke}{rgb}{0.000000,0.000000,0.000000}%
\pgfsetstrokecolor{currentstroke}%
\pgfsetdash{}{0pt}%
\pgfsys@defobject{currentmarker}{\pgfqpoint{-0.048611in}{0.000000in}}{\pgfqpoint{0.000000in}{0.000000in}}{%
\pgfpathmoveto{\pgfqpoint{0.000000in}{0.000000in}}%
\pgfpathlineto{\pgfqpoint{-0.048611in}{0.000000in}}%
\pgfusepath{stroke,fill}%
}%
\begin{pgfscope}%
\pgfsys@transformshift{0.880000in}{6.281355in}%
\pgfsys@useobject{currentmarker}{}%
\end{pgfscope}%
\end{pgfscope}%
\begin{pgfscope}%
\pgftext[x=0.492657in,y=6.218041in,left,base]{\rmfamily\fontsize{12.000000}{14.400000}\selectfont \(\displaystyle 1.00\)}%
\end{pgfscope}%
\begin{pgfscope}%
\pgfsetrectcap%
\pgfsetmiterjoin%
\pgfsetlinewidth{0.803000pt}%
\definecolor{currentstroke}{rgb}{0.501961,0.501961,0.501961}%
\pgfsetstrokecolor{currentstroke}%
\pgfsetdash{}{0pt}%
\pgfpathmoveto{\pgfqpoint{0.880000in}{4.908628in}}%
\pgfpathlineto{\pgfqpoint{0.880000in}{6.281355in}}%
\pgfusepath{stroke}%
\end{pgfscope}%
\begin{pgfscope}%
\pgfsetrectcap%
\pgfsetmiterjoin%
\pgfsetlinewidth{0.803000pt}%
\definecolor{currentstroke}{rgb}{0.501961,0.501961,0.501961}%
\pgfsetstrokecolor{currentstroke}%
\pgfsetdash{}{0pt}%
\pgfpathmoveto{\pgfqpoint{2.777959in}{4.908628in}}%
\pgfpathlineto{\pgfqpoint{2.777959in}{6.281355in}}%
\pgfusepath{stroke}%
\end{pgfscope}%
\begin{pgfscope}%
\pgfsetrectcap%
\pgfsetmiterjoin%
\pgfsetlinewidth{0.803000pt}%
\definecolor{currentstroke}{rgb}{0.501961,0.501961,0.501961}%
\pgfsetstrokecolor{currentstroke}%
\pgfsetdash{}{0pt}%
\pgfpathmoveto{\pgfqpoint{0.880000in}{4.908628in}}%
\pgfpathlineto{\pgfqpoint{2.777959in}{4.908628in}}%
\pgfusepath{stroke}%
\end{pgfscope}%
\begin{pgfscope}%
\pgfsetrectcap%
\pgfsetmiterjoin%
\pgfsetlinewidth{0.803000pt}%
\definecolor{currentstroke}{rgb}{0.501961,0.501961,0.501961}%
\pgfsetstrokecolor{currentstroke}%
\pgfsetdash{}{0pt}%
\pgfpathmoveto{\pgfqpoint{0.880000in}{6.281355in}}%
\pgfpathlineto{\pgfqpoint{2.777959in}{6.281355in}}%
\pgfusepath{stroke}%
\end{pgfscope}%
\begin{pgfscope}%
\pgfsetbuttcap%
\pgfsetmiterjoin%
\definecolor{currentfill}{rgb}{1.000000,1.000000,1.000000}%
\pgfsetfillcolor{currentfill}%
\pgfsetlinewidth{0.000000pt}%
\definecolor{currentstroke}{rgb}{0.000000,0.000000,0.000000}%
\pgfsetstrokecolor{currentstroke}%
\pgfsetstrokeopacity{0.000000}%
\pgfsetdash{}{0pt}%
\pgfpathmoveto{\pgfqpoint{3.347347in}{4.908628in}}%
\pgfpathlineto{\pgfqpoint{5.245306in}{4.908628in}}%
\pgfpathlineto{\pgfqpoint{5.245306in}{6.281355in}}%
\pgfpathlineto{\pgfqpoint{3.347347in}{6.281355in}}%
\pgfpathclose%
\pgfusepath{fill}%
\end{pgfscope}%
\begin{pgfscope}%
\pgfsetbuttcap%
\pgfsetroundjoin%
\definecolor{currentfill}{rgb}{0.000000,0.000000,0.000000}%
\pgfsetfillcolor{currentfill}%
\pgfsetlinewidth{0.803000pt}%
\definecolor{currentstroke}{rgb}{0.000000,0.000000,0.000000}%
\pgfsetstrokecolor{currentstroke}%
\pgfsetdash{}{0pt}%
\pgfsys@defobject{currentmarker}{\pgfqpoint{0.000000in}{-0.048611in}}{\pgfqpoint{0.000000in}{0.000000in}}{%
\pgfpathmoveto{\pgfqpoint{0.000000in}{0.000000in}}%
\pgfpathlineto{\pgfqpoint{0.000000in}{-0.048611in}}%
\pgfusepath{stroke,fill}%
}%
\begin{pgfscope}%
\pgfsys@transformshift{3.347347in}{4.908628in}%
\pgfsys@useobject{currentmarker}{}%
\end{pgfscope}%
\end{pgfscope}%
\begin{pgfscope}%
\pgftext[x=3.347347in,y=4.811405in,,top]{\rmfamily\fontsize{12.000000}{14.400000}\selectfont \(\displaystyle 0.0\)}%
\end{pgfscope}%
\begin{pgfscope}%
\pgfsetbuttcap%
\pgfsetroundjoin%
\definecolor{currentfill}{rgb}{0.000000,0.000000,0.000000}%
\pgfsetfillcolor{currentfill}%
\pgfsetlinewidth{0.803000pt}%
\definecolor{currentstroke}{rgb}{0.000000,0.000000,0.000000}%
\pgfsetstrokecolor{currentstroke}%
\pgfsetdash{}{0pt}%
\pgfsys@defobject{currentmarker}{\pgfqpoint{0.000000in}{-0.048611in}}{\pgfqpoint{0.000000in}{0.000000in}}{%
\pgfpathmoveto{\pgfqpoint{0.000000in}{0.000000in}}%
\pgfpathlineto{\pgfqpoint{0.000000in}{-0.048611in}}%
\pgfusepath{stroke,fill}%
}%
\begin{pgfscope}%
\pgfsys@transformshift{4.296327in}{4.908628in}%
\pgfsys@useobject{currentmarker}{}%
\end{pgfscope}%
\end{pgfscope}%
\begin{pgfscope}%
\pgftext[x=4.296327in,y=4.811405in,,top]{\rmfamily\fontsize{12.000000}{14.400000}\selectfont \(\displaystyle 0.5\)}%
\end{pgfscope}%
\begin{pgfscope}%
\pgfsetbuttcap%
\pgfsetroundjoin%
\definecolor{currentfill}{rgb}{0.000000,0.000000,0.000000}%
\pgfsetfillcolor{currentfill}%
\pgfsetlinewidth{0.803000pt}%
\definecolor{currentstroke}{rgb}{0.000000,0.000000,0.000000}%
\pgfsetstrokecolor{currentstroke}%
\pgfsetdash{}{0pt}%
\pgfsys@defobject{currentmarker}{\pgfqpoint{0.000000in}{-0.048611in}}{\pgfqpoint{0.000000in}{0.000000in}}{%
\pgfpathmoveto{\pgfqpoint{0.000000in}{0.000000in}}%
\pgfpathlineto{\pgfqpoint{0.000000in}{-0.048611in}}%
\pgfusepath{stroke,fill}%
}%
\begin{pgfscope}%
\pgfsys@transformshift{5.245306in}{4.908628in}%
\pgfsys@useobject{currentmarker}{}%
\end{pgfscope}%
\end{pgfscope}%
\begin{pgfscope}%
\pgftext[x=5.245306in,y=4.811405in,,top]{\rmfamily\fontsize{12.000000}{14.400000}\selectfont \(\displaystyle 1.0\)}%
\end{pgfscope}%
\begin{pgfscope}%
\pgfsetbuttcap%
\pgfsetroundjoin%
\definecolor{currentfill}{rgb}{0.000000,0.000000,0.000000}%
\pgfsetfillcolor{currentfill}%
\pgfsetlinewidth{0.803000pt}%
\definecolor{currentstroke}{rgb}{0.000000,0.000000,0.000000}%
\pgfsetstrokecolor{currentstroke}%
\pgfsetdash{}{0pt}%
\pgfsys@defobject{currentmarker}{\pgfqpoint{-0.048611in}{0.000000in}}{\pgfqpoint{0.000000in}{0.000000in}}{%
\pgfpathmoveto{\pgfqpoint{0.000000in}{0.000000in}}%
\pgfpathlineto{\pgfqpoint{-0.048611in}{0.000000in}}%
\pgfusepath{stroke,fill}%
}%
\begin{pgfscope}%
\pgfsys@transformshift{3.347347in}{4.908628in}%
\pgfsys@useobject{currentmarker}{}%
\end{pgfscope}%
\end{pgfscope}%
\begin{pgfscope}%
\pgftext[x=2.960004in,y=4.845314in,left,base]{\rmfamily\fontsize{12.000000}{14.400000}\selectfont \(\displaystyle 0.00\)}%
\end{pgfscope}%
\begin{pgfscope}%
\pgfsetbuttcap%
\pgfsetroundjoin%
\definecolor{currentfill}{rgb}{0.000000,0.000000,0.000000}%
\pgfsetfillcolor{currentfill}%
\pgfsetlinewidth{0.803000pt}%
\definecolor{currentstroke}{rgb}{0.000000,0.000000,0.000000}%
\pgfsetstrokecolor{currentstroke}%
\pgfsetdash{}{0pt}%
\pgfsys@defobject{currentmarker}{\pgfqpoint{-0.048611in}{0.000000in}}{\pgfqpoint{0.000000in}{0.000000in}}{%
\pgfpathmoveto{\pgfqpoint{0.000000in}{0.000000in}}%
\pgfpathlineto{\pgfqpoint{-0.048611in}{0.000000in}}%
\pgfusepath{stroke,fill}%
}%
\begin{pgfscope}%
\pgfsys@transformshift{3.347347in}{5.251809in}%
\pgfsys@useobject{currentmarker}{}%
\end{pgfscope}%
\end{pgfscope}%
\begin{pgfscope}%
\pgftext[x=2.960004in,y=5.188496in,left,base]{\rmfamily\fontsize{12.000000}{14.400000}\selectfont \(\displaystyle 0.25\)}%
\end{pgfscope}%
\begin{pgfscope}%
\pgfsetbuttcap%
\pgfsetroundjoin%
\definecolor{currentfill}{rgb}{0.000000,0.000000,0.000000}%
\pgfsetfillcolor{currentfill}%
\pgfsetlinewidth{0.803000pt}%
\definecolor{currentstroke}{rgb}{0.000000,0.000000,0.000000}%
\pgfsetstrokecolor{currentstroke}%
\pgfsetdash{}{0pt}%
\pgfsys@defobject{currentmarker}{\pgfqpoint{-0.048611in}{0.000000in}}{\pgfqpoint{0.000000in}{0.000000in}}{%
\pgfpathmoveto{\pgfqpoint{0.000000in}{0.000000in}}%
\pgfpathlineto{\pgfqpoint{-0.048611in}{0.000000in}}%
\pgfusepath{stroke,fill}%
}%
\begin{pgfscope}%
\pgfsys@transformshift{3.347347in}{5.594991in}%
\pgfsys@useobject{currentmarker}{}%
\end{pgfscope}%
\end{pgfscope}%
\begin{pgfscope}%
\pgftext[x=2.960004in,y=5.531677in,left,base]{\rmfamily\fontsize{12.000000}{14.400000}\selectfont \(\displaystyle 0.50\)}%
\end{pgfscope}%
\begin{pgfscope}%
\pgfsetbuttcap%
\pgfsetroundjoin%
\definecolor{currentfill}{rgb}{0.000000,0.000000,0.000000}%
\pgfsetfillcolor{currentfill}%
\pgfsetlinewidth{0.803000pt}%
\definecolor{currentstroke}{rgb}{0.000000,0.000000,0.000000}%
\pgfsetstrokecolor{currentstroke}%
\pgfsetdash{}{0pt}%
\pgfsys@defobject{currentmarker}{\pgfqpoint{-0.048611in}{0.000000in}}{\pgfqpoint{0.000000in}{0.000000in}}{%
\pgfpathmoveto{\pgfqpoint{0.000000in}{0.000000in}}%
\pgfpathlineto{\pgfqpoint{-0.048611in}{0.000000in}}%
\pgfusepath{stroke,fill}%
}%
\begin{pgfscope}%
\pgfsys@transformshift{3.347347in}{5.938173in}%
\pgfsys@useobject{currentmarker}{}%
\end{pgfscope}%
\end{pgfscope}%
\begin{pgfscope}%
\pgftext[x=2.960004in,y=5.874859in,left,base]{\rmfamily\fontsize{12.000000}{14.400000}\selectfont \(\displaystyle 0.75\)}%
\end{pgfscope}%
\begin{pgfscope}%
\pgfsetbuttcap%
\pgfsetroundjoin%
\definecolor{currentfill}{rgb}{0.000000,0.000000,0.000000}%
\pgfsetfillcolor{currentfill}%
\pgfsetlinewidth{0.803000pt}%
\definecolor{currentstroke}{rgb}{0.000000,0.000000,0.000000}%
\pgfsetstrokecolor{currentstroke}%
\pgfsetdash{}{0pt}%
\pgfsys@defobject{currentmarker}{\pgfqpoint{-0.048611in}{0.000000in}}{\pgfqpoint{0.000000in}{0.000000in}}{%
\pgfpathmoveto{\pgfqpoint{0.000000in}{0.000000in}}%
\pgfpathlineto{\pgfqpoint{-0.048611in}{0.000000in}}%
\pgfusepath{stroke,fill}%
}%
\begin{pgfscope}%
\pgfsys@transformshift{3.347347in}{6.281355in}%
\pgfsys@useobject{currentmarker}{}%
\end{pgfscope}%
\end{pgfscope}%
\begin{pgfscope}%
\pgftext[x=2.960004in,y=6.218041in,left,base]{\rmfamily\fontsize{12.000000}{14.400000}\selectfont \(\displaystyle 1.00\)}%
\end{pgfscope}%
\begin{pgfscope}%
\pgfsetrectcap%
\pgfsetmiterjoin%
\pgfsetlinewidth{0.803000pt}%
\definecolor{currentstroke}{rgb}{0.501961,0.501961,0.501961}%
\pgfsetstrokecolor{currentstroke}%
\pgfsetdash{}{0pt}%
\pgfpathmoveto{\pgfqpoint{3.347347in}{4.908628in}}%
\pgfpathlineto{\pgfqpoint{3.347347in}{6.281355in}}%
\pgfusepath{stroke}%
\end{pgfscope}%
\begin{pgfscope}%
\pgfsetrectcap%
\pgfsetmiterjoin%
\pgfsetlinewidth{0.803000pt}%
\definecolor{currentstroke}{rgb}{0.501961,0.501961,0.501961}%
\pgfsetstrokecolor{currentstroke}%
\pgfsetdash{}{0pt}%
\pgfpathmoveto{\pgfqpoint{5.245306in}{4.908628in}}%
\pgfpathlineto{\pgfqpoint{5.245306in}{6.281355in}}%
\pgfusepath{stroke}%
\end{pgfscope}%
\begin{pgfscope}%
\pgfsetrectcap%
\pgfsetmiterjoin%
\pgfsetlinewidth{0.803000pt}%
\definecolor{currentstroke}{rgb}{0.501961,0.501961,0.501961}%
\pgfsetstrokecolor{currentstroke}%
\pgfsetdash{}{0pt}%
\pgfpathmoveto{\pgfqpoint{3.347347in}{4.908628in}}%
\pgfpathlineto{\pgfqpoint{5.245306in}{4.908628in}}%
\pgfusepath{stroke}%
\end{pgfscope}%
\begin{pgfscope}%
\pgfsetrectcap%
\pgfsetmiterjoin%
\pgfsetlinewidth{0.803000pt}%
\definecolor{currentstroke}{rgb}{0.501961,0.501961,0.501961}%
\pgfsetstrokecolor{currentstroke}%
\pgfsetdash{}{0pt}%
\pgfpathmoveto{\pgfqpoint{3.347347in}{6.281355in}}%
\pgfpathlineto{\pgfqpoint{5.245306in}{6.281355in}}%
\pgfusepath{stroke}%
\end{pgfscope}%
\begin{pgfscope}%
\pgfsetbuttcap%
\pgfsetmiterjoin%
\definecolor{currentfill}{rgb}{1.000000,1.000000,1.000000}%
\pgfsetfillcolor{currentfill}%
\pgfsetlinewidth{0.000000pt}%
\definecolor{currentstroke}{rgb}{0.000000,0.000000,0.000000}%
\pgfsetstrokecolor{currentstroke}%
\pgfsetstrokeopacity{0.000000}%
\pgfsetdash{}{0pt}%
\pgfpathmoveto{\pgfqpoint{5.814694in}{4.908628in}}%
\pgfpathlineto{\pgfqpoint{7.712653in}{4.908628in}}%
\pgfpathlineto{\pgfqpoint{7.712653in}{6.281355in}}%
\pgfpathlineto{\pgfqpoint{5.814694in}{6.281355in}}%
\pgfpathclose%
\pgfusepath{fill}%
\end{pgfscope}%
\begin{pgfscope}%
\pgfsetbuttcap%
\pgfsetroundjoin%
\definecolor{currentfill}{rgb}{0.000000,0.000000,0.000000}%
\pgfsetfillcolor{currentfill}%
\pgfsetlinewidth{0.803000pt}%
\definecolor{currentstroke}{rgb}{0.000000,0.000000,0.000000}%
\pgfsetstrokecolor{currentstroke}%
\pgfsetdash{}{0pt}%
\pgfsys@defobject{currentmarker}{\pgfqpoint{0.000000in}{-0.048611in}}{\pgfqpoint{0.000000in}{0.000000in}}{%
\pgfpathmoveto{\pgfqpoint{0.000000in}{0.000000in}}%
\pgfpathlineto{\pgfqpoint{0.000000in}{-0.048611in}}%
\pgfusepath{stroke,fill}%
}%
\begin{pgfscope}%
\pgfsys@transformshift{5.814694in}{4.908628in}%
\pgfsys@useobject{currentmarker}{}%
\end{pgfscope}%
\end{pgfscope}%
\begin{pgfscope}%
\pgftext[x=5.814694in,y=4.811405in,,top]{\rmfamily\fontsize{12.000000}{14.400000}\selectfont \(\displaystyle 0.0\)}%
\end{pgfscope}%
\begin{pgfscope}%
\pgfsetbuttcap%
\pgfsetroundjoin%
\definecolor{currentfill}{rgb}{0.000000,0.000000,0.000000}%
\pgfsetfillcolor{currentfill}%
\pgfsetlinewidth{0.803000pt}%
\definecolor{currentstroke}{rgb}{0.000000,0.000000,0.000000}%
\pgfsetstrokecolor{currentstroke}%
\pgfsetdash{}{0pt}%
\pgfsys@defobject{currentmarker}{\pgfqpoint{0.000000in}{-0.048611in}}{\pgfqpoint{0.000000in}{0.000000in}}{%
\pgfpathmoveto{\pgfqpoint{0.000000in}{0.000000in}}%
\pgfpathlineto{\pgfqpoint{0.000000in}{-0.048611in}}%
\pgfusepath{stroke,fill}%
}%
\begin{pgfscope}%
\pgfsys@transformshift{6.763673in}{4.908628in}%
\pgfsys@useobject{currentmarker}{}%
\end{pgfscope}%
\end{pgfscope}%
\begin{pgfscope}%
\pgftext[x=6.763673in,y=4.811405in,,top]{\rmfamily\fontsize{12.000000}{14.400000}\selectfont \(\displaystyle 0.5\)}%
\end{pgfscope}%
\begin{pgfscope}%
\pgfsetbuttcap%
\pgfsetroundjoin%
\definecolor{currentfill}{rgb}{0.000000,0.000000,0.000000}%
\pgfsetfillcolor{currentfill}%
\pgfsetlinewidth{0.803000pt}%
\definecolor{currentstroke}{rgb}{0.000000,0.000000,0.000000}%
\pgfsetstrokecolor{currentstroke}%
\pgfsetdash{}{0pt}%
\pgfsys@defobject{currentmarker}{\pgfqpoint{0.000000in}{-0.048611in}}{\pgfqpoint{0.000000in}{0.000000in}}{%
\pgfpathmoveto{\pgfqpoint{0.000000in}{0.000000in}}%
\pgfpathlineto{\pgfqpoint{0.000000in}{-0.048611in}}%
\pgfusepath{stroke,fill}%
}%
\begin{pgfscope}%
\pgfsys@transformshift{7.712653in}{4.908628in}%
\pgfsys@useobject{currentmarker}{}%
\end{pgfscope}%
\end{pgfscope}%
\begin{pgfscope}%
\pgftext[x=7.712653in,y=4.811405in,,top]{\rmfamily\fontsize{12.000000}{14.400000}\selectfont \(\displaystyle 1.0\)}%
\end{pgfscope}%
\begin{pgfscope}%
\pgfsetbuttcap%
\pgfsetroundjoin%
\definecolor{currentfill}{rgb}{0.000000,0.000000,0.000000}%
\pgfsetfillcolor{currentfill}%
\pgfsetlinewidth{0.803000pt}%
\definecolor{currentstroke}{rgb}{0.000000,0.000000,0.000000}%
\pgfsetstrokecolor{currentstroke}%
\pgfsetdash{}{0pt}%
\pgfsys@defobject{currentmarker}{\pgfqpoint{-0.048611in}{0.000000in}}{\pgfqpoint{0.000000in}{0.000000in}}{%
\pgfpathmoveto{\pgfqpoint{0.000000in}{0.000000in}}%
\pgfpathlineto{\pgfqpoint{-0.048611in}{0.000000in}}%
\pgfusepath{stroke,fill}%
}%
\begin{pgfscope}%
\pgfsys@transformshift{5.814694in}{4.908628in}%
\pgfsys@useobject{currentmarker}{}%
\end{pgfscope}%
\end{pgfscope}%
\begin{pgfscope}%
\pgftext[x=5.427351in,y=4.845314in,left,base]{\rmfamily\fontsize{12.000000}{14.400000}\selectfont \(\displaystyle 0.00\)}%
\end{pgfscope}%
\begin{pgfscope}%
\pgfsetbuttcap%
\pgfsetroundjoin%
\definecolor{currentfill}{rgb}{0.000000,0.000000,0.000000}%
\pgfsetfillcolor{currentfill}%
\pgfsetlinewidth{0.803000pt}%
\definecolor{currentstroke}{rgb}{0.000000,0.000000,0.000000}%
\pgfsetstrokecolor{currentstroke}%
\pgfsetdash{}{0pt}%
\pgfsys@defobject{currentmarker}{\pgfqpoint{-0.048611in}{0.000000in}}{\pgfqpoint{0.000000in}{0.000000in}}{%
\pgfpathmoveto{\pgfqpoint{0.000000in}{0.000000in}}%
\pgfpathlineto{\pgfqpoint{-0.048611in}{0.000000in}}%
\pgfusepath{stroke,fill}%
}%
\begin{pgfscope}%
\pgfsys@transformshift{5.814694in}{5.251809in}%
\pgfsys@useobject{currentmarker}{}%
\end{pgfscope}%
\end{pgfscope}%
\begin{pgfscope}%
\pgftext[x=5.427351in,y=5.188496in,left,base]{\rmfamily\fontsize{12.000000}{14.400000}\selectfont \(\displaystyle 0.25\)}%
\end{pgfscope}%
\begin{pgfscope}%
\pgfsetbuttcap%
\pgfsetroundjoin%
\definecolor{currentfill}{rgb}{0.000000,0.000000,0.000000}%
\pgfsetfillcolor{currentfill}%
\pgfsetlinewidth{0.803000pt}%
\definecolor{currentstroke}{rgb}{0.000000,0.000000,0.000000}%
\pgfsetstrokecolor{currentstroke}%
\pgfsetdash{}{0pt}%
\pgfsys@defobject{currentmarker}{\pgfqpoint{-0.048611in}{0.000000in}}{\pgfqpoint{0.000000in}{0.000000in}}{%
\pgfpathmoveto{\pgfqpoint{0.000000in}{0.000000in}}%
\pgfpathlineto{\pgfqpoint{-0.048611in}{0.000000in}}%
\pgfusepath{stroke,fill}%
}%
\begin{pgfscope}%
\pgfsys@transformshift{5.814694in}{5.594991in}%
\pgfsys@useobject{currentmarker}{}%
\end{pgfscope}%
\end{pgfscope}%
\begin{pgfscope}%
\pgftext[x=5.427351in,y=5.531677in,left,base]{\rmfamily\fontsize{12.000000}{14.400000}\selectfont \(\displaystyle 0.50\)}%
\end{pgfscope}%
\begin{pgfscope}%
\pgfsetbuttcap%
\pgfsetroundjoin%
\definecolor{currentfill}{rgb}{0.000000,0.000000,0.000000}%
\pgfsetfillcolor{currentfill}%
\pgfsetlinewidth{0.803000pt}%
\definecolor{currentstroke}{rgb}{0.000000,0.000000,0.000000}%
\pgfsetstrokecolor{currentstroke}%
\pgfsetdash{}{0pt}%
\pgfsys@defobject{currentmarker}{\pgfqpoint{-0.048611in}{0.000000in}}{\pgfqpoint{0.000000in}{0.000000in}}{%
\pgfpathmoveto{\pgfqpoint{0.000000in}{0.000000in}}%
\pgfpathlineto{\pgfqpoint{-0.048611in}{0.000000in}}%
\pgfusepath{stroke,fill}%
}%
\begin{pgfscope}%
\pgfsys@transformshift{5.814694in}{5.938173in}%
\pgfsys@useobject{currentmarker}{}%
\end{pgfscope}%
\end{pgfscope}%
\begin{pgfscope}%
\pgftext[x=5.427351in,y=5.874859in,left,base]{\rmfamily\fontsize{12.000000}{14.400000}\selectfont \(\displaystyle 0.75\)}%
\end{pgfscope}%
\begin{pgfscope}%
\pgfsetbuttcap%
\pgfsetroundjoin%
\definecolor{currentfill}{rgb}{0.000000,0.000000,0.000000}%
\pgfsetfillcolor{currentfill}%
\pgfsetlinewidth{0.803000pt}%
\definecolor{currentstroke}{rgb}{0.000000,0.000000,0.000000}%
\pgfsetstrokecolor{currentstroke}%
\pgfsetdash{}{0pt}%
\pgfsys@defobject{currentmarker}{\pgfqpoint{-0.048611in}{0.000000in}}{\pgfqpoint{0.000000in}{0.000000in}}{%
\pgfpathmoveto{\pgfqpoint{0.000000in}{0.000000in}}%
\pgfpathlineto{\pgfqpoint{-0.048611in}{0.000000in}}%
\pgfusepath{stroke,fill}%
}%
\begin{pgfscope}%
\pgfsys@transformshift{5.814694in}{6.281355in}%
\pgfsys@useobject{currentmarker}{}%
\end{pgfscope}%
\end{pgfscope}%
\begin{pgfscope}%
\pgftext[x=5.427351in,y=6.218041in,left,base]{\rmfamily\fontsize{12.000000}{14.400000}\selectfont \(\displaystyle 1.00\)}%
\end{pgfscope}%
\begin{pgfscope}%
\pgfsetrectcap%
\pgfsetmiterjoin%
\pgfsetlinewidth{0.803000pt}%
\definecolor{currentstroke}{rgb}{0.501961,0.501961,0.501961}%
\pgfsetstrokecolor{currentstroke}%
\pgfsetdash{}{0pt}%
\pgfpathmoveto{\pgfqpoint{5.814694in}{4.908628in}}%
\pgfpathlineto{\pgfqpoint{5.814694in}{6.281355in}}%
\pgfusepath{stroke}%
\end{pgfscope}%
\begin{pgfscope}%
\pgfsetrectcap%
\pgfsetmiterjoin%
\pgfsetlinewidth{0.803000pt}%
\definecolor{currentstroke}{rgb}{0.501961,0.501961,0.501961}%
\pgfsetstrokecolor{currentstroke}%
\pgfsetdash{}{0pt}%
\pgfpathmoveto{\pgfqpoint{7.712653in}{4.908628in}}%
\pgfpathlineto{\pgfqpoint{7.712653in}{6.281355in}}%
\pgfusepath{stroke}%
\end{pgfscope}%
\begin{pgfscope}%
\pgfsetrectcap%
\pgfsetmiterjoin%
\pgfsetlinewidth{0.803000pt}%
\definecolor{currentstroke}{rgb}{0.501961,0.501961,0.501961}%
\pgfsetstrokecolor{currentstroke}%
\pgfsetdash{}{0pt}%
\pgfpathmoveto{\pgfqpoint{5.814694in}{4.908628in}}%
\pgfpathlineto{\pgfqpoint{7.712653in}{4.908628in}}%
\pgfusepath{stroke}%
\end{pgfscope}%
\begin{pgfscope}%
\pgfsetrectcap%
\pgfsetmiterjoin%
\pgfsetlinewidth{0.803000pt}%
\definecolor{currentstroke}{rgb}{0.501961,0.501961,0.501961}%
\pgfsetstrokecolor{currentstroke}%
\pgfsetdash{}{0pt}%
\pgfpathmoveto{\pgfqpoint{5.814694in}{6.281355in}}%
\pgfpathlineto{\pgfqpoint{7.712653in}{6.281355in}}%
\pgfusepath{stroke}%
\end{pgfscope}%
\begin{pgfscope}%
\pgfsetbuttcap%
\pgfsetmiterjoin%
\definecolor{currentfill}{rgb}{1.000000,1.000000,1.000000}%
\pgfsetfillcolor{currentfill}%
\pgfsetlinewidth{0.000000pt}%
\definecolor{currentstroke}{rgb}{0.000000,0.000000,0.000000}%
\pgfsetstrokecolor{currentstroke}%
\pgfsetstrokeopacity{0.000000}%
\pgfsetdash{}{0pt}%
\pgfpathmoveto{\pgfqpoint{8.282041in}{4.908628in}}%
\pgfpathlineto{\pgfqpoint{10.180000in}{4.908628in}}%
\pgfpathlineto{\pgfqpoint{10.180000in}{6.281355in}}%
\pgfpathlineto{\pgfqpoint{8.282041in}{6.281355in}}%
\pgfpathclose%
\pgfusepath{fill}%
\end{pgfscope}%
\begin{pgfscope}%
\pgfsetbuttcap%
\pgfsetroundjoin%
\definecolor{currentfill}{rgb}{0.000000,0.000000,0.000000}%
\pgfsetfillcolor{currentfill}%
\pgfsetlinewidth{0.803000pt}%
\definecolor{currentstroke}{rgb}{0.000000,0.000000,0.000000}%
\pgfsetstrokecolor{currentstroke}%
\pgfsetdash{}{0pt}%
\pgfsys@defobject{currentmarker}{\pgfqpoint{0.000000in}{-0.048611in}}{\pgfqpoint{0.000000in}{0.000000in}}{%
\pgfpathmoveto{\pgfqpoint{0.000000in}{0.000000in}}%
\pgfpathlineto{\pgfqpoint{0.000000in}{-0.048611in}}%
\pgfusepath{stroke,fill}%
}%
\begin{pgfscope}%
\pgfsys@transformshift{8.282041in}{4.908628in}%
\pgfsys@useobject{currentmarker}{}%
\end{pgfscope}%
\end{pgfscope}%
\begin{pgfscope}%
\pgftext[x=8.282041in,y=4.811405in,,top]{\rmfamily\fontsize{12.000000}{14.400000}\selectfont \(\displaystyle 0.0\)}%
\end{pgfscope}%
\begin{pgfscope}%
\pgfsetbuttcap%
\pgfsetroundjoin%
\definecolor{currentfill}{rgb}{0.000000,0.000000,0.000000}%
\pgfsetfillcolor{currentfill}%
\pgfsetlinewidth{0.803000pt}%
\definecolor{currentstroke}{rgb}{0.000000,0.000000,0.000000}%
\pgfsetstrokecolor{currentstroke}%
\pgfsetdash{}{0pt}%
\pgfsys@defobject{currentmarker}{\pgfqpoint{0.000000in}{-0.048611in}}{\pgfqpoint{0.000000in}{0.000000in}}{%
\pgfpathmoveto{\pgfqpoint{0.000000in}{0.000000in}}%
\pgfpathlineto{\pgfqpoint{0.000000in}{-0.048611in}}%
\pgfusepath{stroke,fill}%
}%
\begin{pgfscope}%
\pgfsys@transformshift{9.231020in}{4.908628in}%
\pgfsys@useobject{currentmarker}{}%
\end{pgfscope}%
\end{pgfscope}%
\begin{pgfscope}%
\pgftext[x=9.231020in,y=4.811405in,,top]{\rmfamily\fontsize{12.000000}{14.400000}\selectfont \(\displaystyle 0.5\)}%
\end{pgfscope}%
\begin{pgfscope}%
\pgfsetbuttcap%
\pgfsetroundjoin%
\definecolor{currentfill}{rgb}{0.000000,0.000000,0.000000}%
\pgfsetfillcolor{currentfill}%
\pgfsetlinewidth{0.803000pt}%
\definecolor{currentstroke}{rgb}{0.000000,0.000000,0.000000}%
\pgfsetstrokecolor{currentstroke}%
\pgfsetdash{}{0pt}%
\pgfsys@defobject{currentmarker}{\pgfqpoint{0.000000in}{-0.048611in}}{\pgfqpoint{0.000000in}{0.000000in}}{%
\pgfpathmoveto{\pgfqpoint{0.000000in}{0.000000in}}%
\pgfpathlineto{\pgfqpoint{0.000000in}{-0.048611in}}%
\pgfusepath{stroke,fill}%
}%
\begin{pgfscope}%
\pgfsys@transformshift{10.180000in}{4.908628in}%
\pgfsys@useobject{currentmarker}{}%
\end{pgfscope}%
\end{pgfscope}%
\begin{pgfscope}%
\pgftext[x=10.180000in,y=4.811405in,,top]{\rmfamily\fontsize{12.000000}{14.400000}\selectfont \(\displaystyle 1.0\)}%
\end{pgfscope}%
\begin{pgfscope}%
\pgfsetbuttcap%
\pgfsetroundjoin%
\definecolor{currentfill}{rgb}{0.000000,0.000000,0.000000}%
\pgfsetfillcolor{currentfill}%
\pgfsetlinewidth{0.803000pt}%
\definecolor{currentstroke}{rgb}{0.000000,0.000000,0.000000}%
\pgfsetstrokecolor{currentstroke}%
\pgfsetdash{}{0pt}%
\pgfsys@defobject{currentmarker}{\pgfqpoint{-0.048611in}{0.000000in}}{\pgfqpoint{0.000000in}{0.000000in}}{%
\pgfpathmoveto{\pgfqpoint{0.000000in}{0.000000in}}%
\pgfpathlineto{\pgfqpoint{-0.048611in}{0.000000in}}%
\pgfusepath{stroke,fill}%
}%
\begin{pgfscope}%
\pgfsys@transformshift{8.282041in}{4.908628in}%
\pgfsys@useobject{currentmarker}{}%
\end{pgfscope}%
\end{pgfscope}%
\begin{pgfscope}%
\pgftext[x=7.894698in,y=4.845314in,left,base]{\rmfamily\fontsize{12.000000}{14.400000}\selectfont \(\displaystyle 0.00\)}%
\end{pgfscope}%
\begin{pgfscope}%
\pgfsetbuttcap%
\pgfsetroundjoin%
\definecolor{currentfill}{rgb}{0.000000,0.000000,0.000000}%
\pgfsetfillcolor{currentfill}%
\pgfsetlinewidth{0.803000pt}%
\definecolor{currentstroke}{rgb}{0.000000,0.000000,0.000000}%
\pgfsetstrokecolor{currentstroke}%
\pgfsetdash{}{0pt}%
\pgfsys@defobject{currentmarker}{\pgfqpoint{-0.048611in}{0.000000in}}{\pgfqpoint{0.000000in}{0.000000in}}{%
\pgfpathmoveto{\pgfqpoint{0.000000in}{0.000000in}}%
\pgfpathlineto{\pgfqpoint{-0.048611in}{0.000000in}}%
\pgfusepath{stroke,fill}%
}%
\begin{pgfscope}%
\pgfsys@transformshift{8.282041in}{5.251809in}%
\pgfsys@useobject{currentmarker}{}%
\end{pgfscope}%
\end{pgfscope}%
\begin{pgfscope}%
\pgftext[x=7.894698in,y=5.188496in,left,base]{\rmfamily\fontsize{12.000000}{14.400000}\selectfont \(\displaystyle 0.25\)}%
\end{pgfscope}%
\begin{pgfscope}%
\pgfsetbuttcap%
\pgfsetroundjoin%
\definecolor{currentfill}{rgb}{0.000000,0.000000,0.000000}%
\pgfsetfillcolor{currentfill}%
\pgfsetlinewidth{0.803000pt}%
\definecolor{currentstroke}{rgb}{0.000000,0.000000,0.000000}%
\pgfsetstrokecolor{currentstroke}%
\pgfsetdash{}{0pt}%
\pgfsys@defobject{currentmarker}{\pgfqpoint{-0.048611in}{0.000000in}}{\pgfqpoint{0.000000in}{0.000000in}}{%
\pgfpathmoveto{\pgfqpoint{0.000000in}{0.000000in}}%
\pgfpathlineto{\pgfqpoint{-0.048611in}{0.000000in}}%
\pgfusepath{stroke,fill}%
}%
\begin{pgfscope}%
\pgfsys@transformshift{8.282041in}{5.594991in}%
\pgfsys@useobject{currentmarker}{}%
\end{pgfscope}%
\end{pgfscope}%
\begin{pgfscope}%
\pgftext[x=7.894698in,y=5.531677in,left,base]{\rmfamily\fontsize{12.000000}{14.400000}\selectfont \(\displaystyle 0.50\)}%
\end{pgfscope}%
\begin{pgfscope}%
\pgfsetbuttcap%
\pgfsetroundjoin%
\definecolor{currentfill}{rgb}{0.000000,0.000000,0.000000}%
\pgfsetfillcolor{currentfill}%
\pgfsetlinewidth{0.803000pt}%
\definecolor{currentstroke}{rgb}{0.000000,0.000000,0.000000}%
\pgfsetstrokecolor{currentstroke}%
\pgfsetdash{}{0pt}%
\pgfsys@defobject{currentmarker}{\pgfqpoint{-0.048611in}{0.000000in}}{\pgfqpoint{0.000000in}{0.000000in}}{%
\pgfpathmoveto{\pgfqpoint{0.000000in}{0.000000in}}%
\pgfpathlineto{\pgfqpoint{-0.048611in}{0.000000in}}%
\pgfusepath{stroke,fill}%
}%
\begin{pgfscope}%
\pgfsys@transformshift{8.282041in}{5.938173in}%
\pgfsys@useobject{currentmarker}{}%
\end{pgfscope}%
\end{pgfscope}%
\begin{pgfscope}%
\pgftext[x=7.894698in,y=5.874859in,left,base]{\rmfamily\fontsize{12.000000}{14.400000}\selectfont \(\displaystyle 0.75\)}%
\end{pgfscope}%
\begin{pgfscope}%
\pgfsetbuttcap%
\pgfsetroundjoin%
\definecolor{currentfill}{rgb}{0.000000,0.000000,0.000000}%
\pgfsetfillcolor{currentfill}%
\pgfsetlinewidth{0.803000pt}%
\definecolor{currentstroke}{rgb}{0.000000,0.000000,0.000000}%
\pgfsetstrokecolor{currentstroke}%
\pgfsetdash{}{0pt}%
\pgfsys@defobject{currentmarker}{\pgfqpoint{-0.048611in}{0.000000in}}{\pgfqpoint{0.000000in}{0.000000in}}{%
\pgfpathmoveto{\pgfqpoint{0.000000in}{0.000000in}}%
\pgfpathlineto{\pgfqpoint{-0.048611in}{0.000000in}}%
\pgfusepath{stroke,fill}%
}%
\begin{pgfscope}%
\pgfsys@transformshift{8.282041in}{6.281355in}%
\pgfsys@useobject{currentmarker}{}%
\end{pgfscope}%
\end{pgfscope}%
\begin{pgfscope}%
\pgftext[x=7.894698in,y=6.218041in,left,base]{\rmfamily\fontsize{12.000000}{14.400000}\selectfont \(\displaystyle 1.00\)}%
\end{pgfscope}%
\begin{pgfscope}%
\pgfsetrectcap%
\pgfsetmiterjoin%
\pgfsetlinewidth{0.803000pt}%
\definecolor{currentstroke}{rgb}{0.501961,0.501961,0.501961}%
\pgfsetstrokecolor{currentstroke}%
\pgfsetdash{}{0pt}%
\pgfpathmoveto{\pgfqpoint{8.282041in}{4.908628in}}%
\pgfpathlineto{\pgfqpoint{8.282041in}{6.281355in}}%
\pgfusepath{stroke}%
\end{pgfscope}%
\begin{pgfscope}%
\pgfsetrectcap%
\pgfsetmiterjoin%
\pgfsetlinewidth{0.803000pt}%
\definecolor{currentstroke}{rgb}{0.501961,0.501961,0.501961}%
\pgfsetstrokecolor{currentstroke}%
\pgfsetdash{}{0pt}%
\pgfpathmoveto{\pgfqpoint{10.180000in}{4.908628in}}%
\pgfpathlineto{\pgfqpoint{10.180000in}{6.281355in}}%
\pgfusepath{stroke}%
\end{pgfscope}%
\begin{pgfscope}%
\pgfsetrectcap%
\pgfsetmiterjoin%
\pgfsetlinewidth{0.803000pt}%
\definecolor{currentstroke}{rgb}{0.501961,0.501961,0.501961}%
\pgfsetstrokecolor{currentstroke}%
\pgfsetdash{}{0pt}%
\pgfpathmoveto{\pgfqpoint{8.282041in}{4.908628in}}%
\pgfpathlineto{\pgfqpoint{10.180000in}{4.908628in}}%
\pgfusepath{stroke}%
\end{pgfscope}%
\begin{pgfscope}%
\pgfsetrectcap%
\pgfsetmiterjoin%
\pgfsetlinewidth{0.803000pt}%
\definecolor{currentstroke}{rgb}{0.501961,0.501961,0.501961}%
\pgfsetstrokecolor{currentstroke}%
\pgfsetdash{}{0pt}%
\pgfpathmoveto{\pgfqpoint{8.282041in}{6.281355in}}%
\pgfpathlineto{\pgfqpoint{10.180000in}{6.281355in}}%
\pgfusepath{stroke}%
\end{pgfscope}%
\begin{pgfscope}%
\pgfsetbuttcap%
\pgfsetmiterjoin%
\definecolor{currentfill}{rgb}{1.000000,1.000000,1.000000}%
\pgfsetfillcolor{currentfill}%
\pgfsetlinewidth{0.000000pt}%
\definecolor{currentstroke}{rgb}{0.000000,0.000000,0.000000}%
\pgfsetstrokecolor{currentstroke}%
\pgfsetstrokeopacity{0.000000}%
\pgfsetdash{}{0pt}%
\pgfpathmoveto{\pgfqpoint{0.880000in}{2.849537in}}%
\pgfpathlineto{\pgfqpoint{2.777959in}{2.849537in}}%
\pgfpathlineto{\pgfqpoint{2.777959in}{4.222264in}}%
\pgfpathlineto{\pgfqpoint{0.880000in}{4.222264in}}%
\pgfpathclose%
\pgfusepath{fill}%
\end{pgfscope}%
\begin{pgfscope}%
\pgfsetbuttcap%
\pgfsetroundjoin%
\definecolor{currentfill}{rgb}{0.000000,0.000000,0.000000}%
\pgfsetfillcolor{currentfill}%
\pgfsetlinewidth{0.803000pt}%
\definecolor{currentstroke}{rgb}{0.000000,0.000000,0.000000}%
\pgfsetstrokecolor{currentstroke}%
\pgfsetdash{}{0pt}%
\pgfsys@defobject{currentmarker}{\pgfqpoint{0.000000in}{-0.048611in}}{\pgfqpoint{0.000000in}{0.000000in}}{%
\pgfpathmoveto{\pgfqpoint{0.000000in}{0.000000in}}%
\pgfpathlineto{\pgfqpoint{0.000000in}{-0.048611in}}%
\pgfusepath{stroke,fill}%
}%
\begin{pgfscope}%
\pgfsys@transformshift{0.880000in}{2.849537in}%
\pgfsys@useobject{currentmarker}{}%
\end{pgfscope}%
\end{pgfscope}%
\begin{pgfscope}%
\pgftext[x=0.880000in,y=2.752315in,,top]{\rmfamily\fontsize{12.000000}{14.400000}\selectfont \(\displaystyle 0.0\)}%
\end{pgfscope}%
\begin{pgfscope}%
\pgfsetbuttcap%
\pgfsetroundjoin%
\definecolor{currentfill}{rgb}{0.000000,0.000000,0.000000}%
\pgfsetfillcolor{currentfill}%
\pgfsetlinewidth{0.803000pt}%
\definecolor{currentstroke}{rgb}{0.000000,0.000000,0.000000}%
\pgfsetstrokecolor{currentstroke}%
\pgfsetdash{}{0pt}%
\pgfsys@defobject{currentmarker}{\pgfqpoint{0.000000in}{-0.048611in}}{\pgfqpoint{0.000000in}{0.000000in}}{%
\pgfpathmoveto{\pgfqpoint{0.000000in}{0.000000in}}%
\pgfpathlineto{\pgfqpoint{0.000000in}{-0.048611in}}%
\pgfusepath{stroke,fill}%
}%
\begin{pgfscope}%
\pgfsys@transformshift{1.828980in}{2.849537in}%
\pgfsys@useobject{currentmarker}{}%
\end{pgfscope}%
\end{pgfscope}%
\begin{pgfscope}%
\pgftext[x=1.828980in,y=2.752315in,,top]{\rmfamily\fontsize{12.000000}{14.400000}\selectfont \(\displaystyle 0.5\)}%
\end{pgfscope}%
\begin{pgfscope}%
\pgfsetbuttcap%
\pgfsetroundjoin%
\definecolor{currentfill}{rgb}{0.000000,0.000000,0.000000}%
\pgfsetfillcolor{currentfill}%
\pgfsetlinewidth{0.803000pt}%
\definecolor{currentstroke}{rgb}{0.000000,0.000000,0.000000}%
\pgfsetstrokecolor{currentstroke}%
\pgfsetdash{}{0pt}%
\pgfsys@defobject{currentmarker}{\pgfqpoint{0.000000in}{-0.048611in}}{\pgfqpoint{0.000000in}{0.000000in}}{%
\pgfpathmoveto{\pgfqpoint{0.000000in}{0.000000in}}%
\pgfpathlineto{\pgfqpoint{0.000000in}{-0.048611in}}%
\pgfusepath{stroke,fill}%
}%
\begin{pgfscope}%
\pgfsys@transformshift{2.777959in}{2.849537in}%
\pgfsys@useobject{currentmarker}{}%
\end{pgfscope}%
\end{pgfscope}%
\begin{pgfscope}%
\pgftext[x=2.777959in,y=2.752315in,,top]{\rmfamily\fontsize{12.000000}{14.400000}\selectfont \(\displaystyle 1.0\)}%
\end{pgfscope}%
\begin{pgfscope}%
\pgfsetbuttcap%
\pgfsetroundjoin%
\definecolor{currentfill}{rgb}{0.000000,0.000000,0.000000}%
\pgfsetfillcolor{currentfill}%
\pgfsetlinewidth{0.803000pt}%
\definecolor{currentstroke}{rgb}{0.000000,0.000000,0.000000}%
\pgfsetstrokecolor{currentstroke}%
\pgfsetdash{}{0pt}%
\pgfsys@defobject{currentmarker}{\pgfqpoint{-0.048611in}{0.000000in}}{\pgfqpoint{0.000000in}{0.000000in}}{%
\pgfpathmoveto{\pgfqpoint{0.000000in}{0.000000in}}%
\pgfpathlineto{\pgfqpoint{-0.048611in}{0.000000in}}%
\pgfusepath{stroke,fill}%
}%
\begin{pgfscope}%
\pgfsys@transformshift{0.880000in}{2.849537in}%
\pgfsys@useobject{currentmarker}{}%
\end{pgfscope}%
\end{pgfscope}%
\begin{pgfscope}%
\pgftext[x=0.492657in,y=2.786223in,left,base]{\rmfamily\fontsize{12.000000}{14.400000}\selectfont \(\displaystyle 0.00\)}%
\end{pgfscope}%
\begin{pgfscope}%
\pgfsetbuttcap%
\pgfsetroundjoin%
\definecolor{currentfill}{rgb}{0.000000,0.000000,0.000000}%
\pgfsetfillcolor{currentfill}%
\pgfsetlinewidth{0.803000pt}%
\definecolor{currentstroke}{rgb}{0.000000,0.000000,0.000000}%
\pgfsetstrokecolor{currentstroke}%
\pgfsetdash{}{0pt}%
\pgfsys@defobject{currentmarker}{\pgfqpoint{-0.048611in}{0.000000in}}{\pgfqpoint{0.000000in}{0.000000in}}{%
\pgfpathmoveto{\pgfqpoint{0.000000in}{0.000000in}}%
\pgfpathlineto{\pgfqpoint{-0.048611in}{0.000000in}}%
\pgfusepath{stroke,fill}%
}%
\begin{pgfscope}%
\pgfsys@transformshift{0.880000in}{3.192719in}%
\pgfsys@useobject{currentmarker}{}%
\end{pgfscope}%
\end{pgfscope}%
\begin{pgfscope}%
\pgftext[x=0.492657in,y=3.129405in,left,base]{\rmfamily\fontsize{12.000000}{14.400000}\selectfont \(\displaystyle 0.25\)}%
\end{pgfscope}%
\begin{pgfscope}%
\pgfsetbuttcap%
\pgfsetroundjoin%
\definecolor{currentfill}{rgb}{0.000000,0.000000,0.000000}%
\pgfsetfillcolor{currentfill}%
\pgfsetlinewidth{0.803000pt}%
\definecolor{currentstroke}{rgb}{0.000000,0.000000,0.000000}%
\pgfsetstrokecolor{currentstroke}%
\pgfsetdash{}{0pt}%
\pgfsys@defobject{currentmarker}{\pgfqpoint{-0.048611in}{0.000000in}}{\pgfqpoint{0.000000in}{0.000000in}}{%
\pgfpathmoveto{\pgfqpoint{0.000000in}{0.000000in}}%
\pgfpathlineto{\pgfqpoint{-0.048611in}{0.000000in}}%
\pgfusepath{stroke,fill}%
}%
\begin{pgfscope}%
\pgfsys@transformshift{0.880000in}{3.535900in}%
\pgfsys@useobject{currentmarker}{}%
\end{pgfscope}%
\end{pgfscope}%
\begin{pgfscope}%
\pgftext[x=0.492657in,y=3.472587in,left,base]{\rmfamily\fontsize{12.000000}{14.400000}\selectfont \(\displaystyle 0.50\)}%
\end{pgfscope}%
\begin{pgfscope}%
\pgfsetbuttcap%
\pgfsetroundjoin%
\definecolor{currentfill}{rgb}{0.000000,0.000000,0.000000}%
\pgfsetfillcolor{currentfill}%
\pgfsetlinewidth{0.803000pt}%
\definecolor{currentstroke}{rgb}{0.000000,0.000000,0.000000}%
\pgfsetstrokecolor{currentstroke}%
\pgfsetdash{}{0pt}%
\pgfsys@defobject{currentmarker}{\pgfqpoint{-0.048611in}{0.000000in}}{\pgfqpoint{0.000000in}{0.000000in}}{%
\pgfpathmoveto{\pgfqpoint{0.000000in}{0.000000in}}%
\pgfpathlineto{\pgfqpoint{-0.048611in}{0.000000in}}%
\pgfusepath{stroke,fill}%
}%
\begin{pgfscope}%
\pgfsys@transformshift{0.880000in}{3.879082in}%
\pgfsys@useobject{currentmarker}{}%
\end{pgfscope}%
\end{pgfscope}%
\begin{pgfscope}%
\pgftext[x=0.492657in,y=3.815768in,left,base]{\rmfamily\fontsize{12.000000}{14.400000}\selectfont \(\displaystyle 0.75\)}%
\end{pgfscope}%
\begin{pgfscope}%
\pgfsetbuttcap%
\pgfsetroundjoin%
\definecolor{currentfill}{rgb}{0.000000,0.000000,0.000000}%
\pgfsetfillcolor{currentfill}%
\pgfsetlinewidth{0.803000pt}%
\definecolor{currentstroke}{rgb}{0.000000,0.000000,0.000000}%
\pgfsetstrokecolor{currentstroke}%
\pgfsetdash{}{0pt}%
\pgfsys@defobject{currentmarker}{\pgfqpoint{-0.048611in}{0.000000in}}{\pgfqpoint{0.000000in}{0.000000in}}{%
\pgfpathmoveto{\pgfqpoint{0.000000in}{0.000000in}}%
\pgfpathlineto{\pgfqpoint{-0.048611in}{0.000000in}}%
\pgfusepath{stroke,fill}%
}%
\begin{pgfscope}%
\pgfsys@transformshift{0.880000in}{4.222264in}%
\pgfsys@useobject{currentmarker}{}%
\end{pgfscope}%
\end{pgfscope}%
\begin{pgfscope}%
\pgftext[x=0.492657in,y=4.158950in,left,base]{\rmfamily\fontsize{12.000000}{14.400000}\selectfont \(\displaystyle 1.00\)}%
\end{pgfscope}%
\begin{pgfscope}%
\pgfsetrectcap%
\pgfsetmiterjoin%
\pgfsetlinewidth{0.803000pt}%
\definecolor{currentstroke}{rgb}{0.501961,0.501961,0.501961}%
\pgfsetstrokecolor{currentstroke}%
\pgfsetdash{}{0pt}%
\pgfpathmoveto{\pgfqpoint{0.880000in}{2.849537in}}%
\pgfpathlineto{\pgfqpoint{0.880000in}{4.222264in}}%
\pgfusepath{stroke}%
\end{pgfscope}%
\begin{pgfscope}%
\pgfsetrectcap%
\pgfsetmiterjoin%
\pgfsetlinewidth{0.803000pt}%
\definecolor{currentstroke}{rgb}{0.501961,0.501961,0.501961}%
\pgfsetstrokecolor{currentstroke}%
\pgfsetdash{}{0pt}%
\pgfpathmoveto{\pgfqpoint{2.777959in}{2.849537in}}%
\pgfpathlineto{\pgfqpoint{2.777959in}{4.222264in}}%
\pgfusepath{stroke}%
\end{pgfscope}%
\begin{pgfscope}%
\pgfsetrectcap%
\pgfsetmiterjoin%
\pgfsetlinewidth{0.803000pt}%
\definecolor{currentstroke}{rgb}{0.501961,0.501961,0.501961}%
\pgfsetstrokecolor{currentstroke}%
\pgfsetdash{}{0pt}%
\pgfpathmoveto{\pgfqpoint{0.880000in}{2.849537in}}%
\pgfpathlineto{\pgfqpoint{2.777959in}{2.849537in}}%
\pgfusepath{stroke}%
\end{pgfscope}%
\begin{pgfscope}%
\pgfsetrectcap%
\pgfsetmiterjoin%
\pgfsetlinewidth{0.803000pt}%
\definecolor{currentstroke}{rgb}{0.501961,0.501961,0.501961}%
\pgfsetstrokecolor{currentstroke}%
\pgfsetdash{}{0pt}%
\pgfpathmoveto{\pgfqpoint{0.880000in}{4.222264in}}%
\pgfpathlineto{\pgfqpoint{2.777959in}{4.222264in}}%
\pgfusepath{stroke}%
\end{pgfscope}%
\begin{pgfscope}%
\pgfsetbuttcap%
\pgfsetmiterjoin%
\definecolor{currentfill}{rgb}{1.000000,1.000000,1.000000}%
\pgfsetfillcolor{currentfill}%
\pgfsetlinewidth{0.000000pt}%
\definecolor{currentstroke}{rgb}{0.000000,0.000000,0.000000}%
\pgfsetstrokecolor{currentstroke}%
\pgfsetstrokeopacity{0.000000}%
\pgfsetdash{}{0pt}%
\pgfpathmoveto{\pgfqpoint{3.347347in}{2.849537in}}%
\pgfpathlineto{\pgfqpoint{5.245306in}{2.849537in}}%
\pgfpathlineto{\pgfqpoint{5.245306in}{4.222264in}}%
\pgfpathlineto{\pgfqpoint{3.347347in}{4.222264in}}%
\pgfpathclose%
\pgfusepath{fill}%
\end{pgfscope}%
\begin{pgfscope}%
\pgfsetbuttcap%
\pgfsetroundjoin%
\definecolor{currentfill}{rgb}{0.000000,0.000000,0.000000}%
\pgfsetfillcolor{currentfill}%
\pgfsetlinewidth{0.803000pt}%
\definecolor{currentstroke}{rgb}{0.000000,0.000000,0.000000}%
\pgfsetstrokecolor{currentstroke}%
\pgfsetdash{}{0pt}%
\pgfsys@defobject{currentmarker}{\pgfqpoint{0.000000in}{-0.048611in}}{\pgfqpoint{0.000000in}{0.000000in}}{%
\pgfpathmoveto{\pgfqpoint{0.000000in}{0.000000in}}%
\pgfpathlineto{\pgfqpoint{0.000000in}{-0.048611in}}%
\pgfusepath{stroke,fill}%
}%
\begin{pgfscope}%
\pgfsys@transformshift{3.347347in}{2.849537in}%
\pgfsys@useobject{currentmarker}{}%
\end{pgfscope}%
\end{pgfscope}%
\begin{pgfscope}%
\pgftext[x=3.347347in,y=2.752315in,,top]{\rmfamily\fontsize{12.000000}{14.400000}\selectfont \(\displaystyle 0.0\)}%
\end{pgfscope}%
\begin{pgfscope}%
\pgfsetbuttcap%
\pgfsetroundjoin%
\definecolor{currentfill}{rgb}{0.000000,0.000000,0.000000}%
\pgfsetfillcolor{currentfill}%
\pgfsetlinewidth{0.803000pt}%
\definecolor{currentstroke}{rgb}{0.000000,0.000000,0.000000}%
\pgfsetstrokecolor{currentstroke}%
\pgfsetdash{}{0pt}%
\pgfsys@defobject{currentmarker}{\pgfqpoint{0.000000in}{-0.048611in}}{\pgfqpoint{0.000000in}{0.000000in}}{%
\pgfpathmoveto{\pgfqpoint{0.000000in}{0.000000in}}%
\pgfpathlineto{\pgfqpoint{0.000000in}{-0.048611in}}%
\pgfusepath{stroke,fill}%
}%
\begin{pgfscope}%
\pgfsys@transformshift{4.296327in}{2.849537in}%
\pgfsys@useobject{currentmarker}{}%
\end{pgfscope}%
\end{pgfscope}%
\begin{pgfscope}%
\pgftext[x=4.296327in,y=2.752315in,,top]{\rmfamily\fontsize{12.000000}{14.400000}\selectfont \(\displaystyle 0.5\)}%
\end{pgfscope}%
\begin{pgfscope}%
\pgfsetbuttcap%
\pgfsetroundjoin%
\definecolor{currentfill}{rgb}{0.000000,0.000000,0.000000}%
\pgfsetfillcolor{currentfill}%
\pgfsetlinewidth{0.803000pt}%
\definecolor{currentstroke}{rgb}{0.000000,0.000000,0.000000}%
\pgfsetstrokecolor{currentstroke}%
\pgfsetdash{}{0pt}%
\pgfsys@defobject{currentmarker}{\pgfqpoint{0.000000in}{-0.048611in}}{\pgfqpoint{0.000000in}{0.000000in}}{%
\pgfpathmoveto{\pgfqpoint{0.000000in}{0.000000in}}%
\pgfpathlineto{\pgfqpoint{0.000000in}{-0.048611in}}%
\pgfusepath{stroke,fill}%
}%
\begin{pgfscope}%
\pgfsys@transformshift{5.245306in}{2.849537in}%
\pgfsys@useobject{currentmarker}{}%
\end{pgfscope}%
\end{pgfscope}%
\begin{pgfscope}%
\pgftext[x=5.245306in,y=2.752315in,,top]{\rmfamily\fontsize{12.000000}{14.400000}\selectfont \(\displaystyle 1.0\)}%
\end{pgfscope}%
\begin{pgfscope}%
\pgfsetbuttcap%
\pgfsetroundjoin%
\definecolor{currentfill}{rgb}{0.000000,0.000000,0.000000}%
\pgfsetfillcolor{currentfill}%
\pgfsetlinewidth{0.803000pt}%
\definecolor{currentstroke}{rgb}{0.000000,0.000000,0.000000}%
\pgfsetstrokecolor{currentstroke}%
\pgfsetdash{}{0pt}%
\pgfsys@defobject{currentmarker}{\pgfqpoint{-0.048611in}{0.000000in}}{\pgfqpoint{0.000000in}{0.000000in}}{%
\pgfpathmoveto{\pgfqpoint{0.000000in}{0.000000in}}%
\pgfpathlineto{\pgfqpoint{-0.048611in}{0.000000in}}%
\pgfusepath{stroke,fill}%
}%
\begin{pgfscope}%
\pgfsys@transformshift{3.347347in}{2.849537in}%
\pgfsys@useobject{currentmarker}{}%
\end{pgfscope}%
\end{pgfscope}%
\begin{pgfscope}%
\pgftext[x=2.960004in,y=2.786223in,left,base]{\rmfamily\fontsize{12.000000}{14.400000}\selectfont \(\displaystyle 0.00\)}%
\end{pgfscope}%
\begin{pgfscope}%
\pgfsetbuttcap%
\pgfsetroundjoin%
\definecolor{currentfill}{rgb}{0.000000,0.000000,0.000000}%
\pgfsetfillcolor{currentfill}%
\pgfsetlinewidth{0.803000pt}%
\definecolor{currentstroke}{rgb}{0.000000,0.000000,0.000000}%
\pgfsetstrokecolor{currentstroke}%
\pgfsetdash{}{0pt}%
\pgfsys@defobject{currentmarker}{\pgfqpoint{-0.048611in}{0.000000in}}{\pgfqpoint{0.000000in}{0.000000in}}{%
\pgfpathmoveto{\pgfqpoint{0.000000in}{0.000000in}}%
\pgfpathlineto{\pgfqpoint{-0.048611in}{0.000000in}}%
\pgfusepath{stroke,fill}%
}%
\begin{pgfscope}%
\pgfsys@transformshift{3.347347in}{3.192719in}%
\pgfsys@useobject{currentmarker}{}%
\end{pgfscope}%
\end{pgfscope}%
\begin{pgfscope}%
\pgftext[x=2.960004in,y=3.129405in,left,base]{\rmfamily\fontsize{12.000000}{14.400000}\selectfont \(\displaystyle 0.25\)}%
\end{pgfscope}%
\begin{pgfscope}%
\pgfsetbuttcap%
\pgfsetroundjoin%
\definecolor{currentfill}{rgb}{0.000000,0.000000,0.000000}%
\pgfsetfillcolor{currentfill}%
\pgfsetlinewidth{0.803000pt}%
\definecolor{currentstroke}{rgb}{0.000000,0.000000,0.000000}%
\pgfsetstrokecolor{currentstroke}%
\pgfsetdash{}{0pt}%
\pgfsys@defobject{currentmarker}{\pgfqpoint{-0.048611in}{0.000000in}}{\pgfqpoint{0.000000in}{0.000000in}}{%
\pgfpathmoveto{\pgfqpoint{0.000000in}{0.000000in}}%
\pgfpathlineto{\pgfqpoint{-0.048611in}{0.000000in}}%
\pgfusepath{stroke,fill}%
}%
\begin{pgfscope}%
\pgfsys@transformshift{3.347347in}{3.535900in}%
\pgfsys@useobject{currentmarker}{}%
\end{pgfscope}%
\end{pgfscope}%
\begin{pgfscope}%
\pgftext[x=2.960004in,y=3.472587in,left,base]{\rmfamily\fontsize{12.000000}{14.400000}\selectfont \(\displaystyle 0.50\)}%
\end{pgfscope}%
\begin{pgfscope}%
\pgfsetbuttcap%
\pgfsetroundjoin%
\definecolor{currentfill}{rgb}{0.000000,0.000000,0.000000}%
\pgfsetfillcolor{currentfill}%
\pgfsetlinewidth{0.803000pt}%
\definecolor{currentstroke}{rgb}{0.000000,0.000000,0.000000}%
\pgfsetstrokecolor{currentstroke}%
\pgfsetdash{}{0pt}%
\pgfsys@defobject{currentmarker}{\pgfqpoint{-0.048611in}{0.000000in}}{\pgfqpoint{0.000000in}{0.000000in}}{%
\pgfpathmoveto{\pgfqpoint{0.000000in}{0.000000in}}%
\pgfpathlineto{\pgfqpoint{-0.048611in}{0.000000in}}%
\pgfusepath{stroke,fill}%
}%
\begin{pgfscope}%
\pgfsys@transformshift{3.347347in}{3.879082in}%
\pgfsys@useobject{currentmarker}{}%
\end{pgfscope}%
\end{pgfscope}%
\begin{pgfscope}%
\pgftext[x=2.960004in,y=3.815768in,left,base]{\rmfamily\fontsize{12.000000}{14.400000}\selectfont \(\displaystyle 0.75\)}%
\end{pgfscope}%
\begin{pgfscope}%
\pgfsetbuttcap%
\pgfsetroundjoin%
\definecolor{currentfill}{rgb}{0.000000,0.000000,0.000000}%
\pgfsetfillcolor{currentfill}%
\pgfsetlinewidth{0.803000pt}%
\definecolor{currentstroke}{rgb}{0.000000,0.000000,0.000000}%
\pgfsetstrokecolor{currentstroke}%
\pgfsetdash{}{0pt}%
\pgfsys@defobject{currentmarker}{\pgfqpoint{-0.048611in}{0.000000in}}{\pgfqpoint{0.000000in}{0.000000in}}{%
\pgfpathmoveto{\pgfqpoint{0.000000in}{0.000000in}}%
\pgfpathlineto{\pgfqpoint{-0.048611in}{0.000000in}}%
\pgfusepath{stroke,fill}%
}%
\begin{pgfscope}%
\pgfsys@transformshift{3.347347in}{4.222264in}%
\pgfsys@useobject{currentmarker}{}%
\end{pgfscope}%
\end{pgfscope}%
\begin{pgfscope}%
\pgftext[x=2.960004in,y=4.158950in,left,base]{\rmfamily\fontsize{12.000000}{14.400000}\selectfont \(\displaystyle 1.00\)}%
\end{pgfscope}%
\begin{pgfscope}%
\pgfsetrectcap%
\pgfsetmiterjoin%
\pgfsetlinewidth{0.803000pt}%
\definecolor{currentstroke}{rgb}{0.501961,0.501961,0.501961}%
\pgfsetstrokecolor{currentstroke}%
\pgfsetdash{}{0pt}%
\pgfpathmoveto{\pgfqpoint{3.347347in}{2.849537in}}%
\pgfpathlineto{\pgfqpoint{3.347347in}{4.222264in}}%
\pgfusepath{stroke}%
\end{pgfscope}%
\begin{pgfscope}%
\pgfsetrectcap%
\pgfsetmiterjoin%
\pgfsetlinewidth{0.803000pt}%
\definecolor{currentstroke}{rgb}{0.501961,0.501961,0.501961}%
\pgfsetstrokecolor{currentstroke}%
\pgfsetdash{}{0pt}%
\pgfpathmoveto{\pgfqpoint{5.245306in}{2.849537in}}%
\pgfpathlineto{\pgfqpoint{5.245306in}{4.222264in}}%
\pgfusepath{stroke}%
\end{pgfscope}%
\begin{pgfscope}%
\pgfsetrectcap%
\pgfsetmiterjoin%
\pgfsetlinewidth{0.803000pt}%
\definecolor{currentstroke}{rgb}{0.501961,0.501961,0.501961}%
\pgfsetstrokecolor{currentstroke}%
\pgfsetdash{}{0pt}%
\pgfpathmoveto{\pgfqpoint{3.347347in}{2.849537in}}%
\pgfpathlineto{\pgfqpoint{5.245306in}{2.849537in}}%
\pgfusepath{stroke}%
\end{pgfscope}%
\begin{pgfscope}%
\pgfsetrectcap%
\pgfsetmiterjoin%
\pgfsetlinewidth{0.803000pt}%
\definecolor{currentstroke}{rgb}{0.501961,0.501961,0.501961}%
\pgfsetstrokecolor{currentstroke}%
\pgfsetdash{}{0pt}%
\pgfpathmoveto{\pgfqpoint{3.347347in}{4.222264in}}%
\pgfpathlineto{\pgfqpoint{5.245306in}{4.222264in}}%
\pgfusepath{stroke}%
\end{pgfscope}%
\begin{pgfscope}%
\pgfsetbuttcap%
\pgfsetmiterjoin%
\definecolor{currentfill}{rgb}{1.000000,1.000000,1.000000}%
\pgfsetfillcolor{currentfill}%
\pgfsetlinewidth{0.000000pt}%
\definecolor{currentstroke}{rgb}{0.000000,0.000000,0.000000}%
\pgfsetstrokecolor{currentstroke}%
\pgfsetstrokeopacity{0.000000}%
\pgfsetdash{}{0pt}%
\pgfpathmoveto{\pgfqpoint{5.814694in}{2.849537in}}%
\pgfpathlineto{\pgfqpoint{7.712653in}{2.849537in}}%
\pgfpathlineto{\pgfqpoint{7.712653in}{4.222264in}}%
\pgfpathlineto{\pgfqpoint{5.814694in}{4.222264in}}%
\pgfpathclose%
\pgfusepath{fill}%
\end{pgfscope}%
\begin{pgfscope}%
\pgfsetbuttcap%
\pgfsetroundjoin%
\definecolor{currentfill}{rgb}{0.000000,0.000000,0.000000}%
\pgfsetfillcolor{currentfill}%
\pgfsetlinewidth{0.803000pt}%
\definecolor{currentstroke}{rgb}{0.000000,0.000000,0.000000}%
\pgfsetstrokecolor{currentstroke}%
\pgfsetdash{}{0pt}%
\pgfsys@defobject{currentmarker}{\pgfqpoint{0.000000in}{-0.048611in}}{\pgfqpoint{0.000000in}{0.000000in}}{%
\pgfpathmoveto{\pgfqpoint{0.000000in}{0.000000in}}%
\pgfpathlineto{\pgfqpoint{0.000000in}{-0.048611in}}%
\pgfusepath{stroke,fill}%
}%
\begin{pgfscope}%
\pgfsys@transformshift{5.814694in}{2.849537in}%
\pgfsys@useobject{currentmarker}{}%
\end{pgfscope}%
\end{pgfscope}%
\begin{pgfscope}%
\pgftext[x=5.814694in,y=2.752315in,,top]{\rmfamily\fontsize{12.000000}{14.400000}\selectfont \(\displaystyle 0.0\)}%
\end{pgfscope}%
\begin{pgfscope}%
\pgfsetbuttcap%
\pgfsetroundjoin%
\definecolor{currentfill}{rgb}{0.000000,0.000000,0.000000}%
\pgfsetfillcolor{currentfill}%
\pgfsetlinewidth{0.803000pt}%
\definecolor{currentstroke}{rgb}{0.000000,0.000000,0.000000}%
\pgfsetstrokecolor{currentstroke}%
\pgfsetdash{}{0pt}%
\pgfsys@defobject{currentmarker}{\pgfqpoint{0.000000in}{-0.048611in}}{\pgfqpoint{0.000000in}{0.000000in}}{%
\pgfpathmoveto{\pgfqpoint{0.000000in}{0.000000in}}%
\pgfpathlineto{\pgfqpoint{0.000000in}{-0.048611in}}%
\pgfusepath{stroke,fill}%
}%
\begin{pgfscope}%
\pgfsys@transformshift{6.763673in}{2.849537in}%
\pgfsys@useobject{currentmarker}{}%
\end{pgfscope}%
\end{pgfscope}%
\begin{pgfscope}%
\pgftext[x=6.763673in,y=2.752315in,,top]{\rmfamily\fontsize{12.000000}{14.400000}\selectfont \(\displaystyle 0.5\)}%
\end{pgfscope}%
\begin{pgfscope}%
\pgfsetbuttcap%
\pgfsetroundjoin%
\definecolor{currentfill}{rgb}{0.000000,0.000000,0.000000}%
\pgfsetfillcolor{currentfill}%
\pgfsetlinewidth{0.803000pt}%
\definecolor{currentstroke}{rgb}{0.000000,0.000000,0.000000}%
\pgfsetstrokecolor{currentstroke}%
\pgfsetdash{}{0pt}%
\pgfsys@defobject{currentmarker}{\pgfqpoint{0.000000in}{-0.048611in}}{\pgfqpoint{0.000000in}{0.000000in}}{%
\pgfpathmoveto{\pgfqpoint{0.000000in}{0.000000in}}%
\pgfpathlineto{\pgfqpoint{0.000000in}{-0.048611in}}%
\pgfusepath{stroke,fill}%
}%
\begin{pgfscope}%
\pgfsys@transformshift{7.712653in}{2.849537in}%
\pgfsys@useobject{currentmarker}{}%
\end{pgfscope}%
\end{pgfscope}%
\begin{pgfscope}%
\pgftext[x=7.712653in,y=2.752315in,,top]{\rmfamily\fontsize{12.000000}{14.400000}\selectfont \(\displaystyle 1.0\)}%
\end{pgfscope}%
\begin{pgfscope}%
\pgfsetbuttcap%
\pgfsetroundjoin%
\definecolor{currentfill}{rgb}{0.000000,0.000000,0.000000}%
\pgfsetfillcolor{currentfill}%
\pgfsetlinewidth{0.803000pt}%
\definecolor{currentstroke}{rgb}{0.000000,0.000000,0.000000}%
\pgfsetstrokecolor{currentstroke}%
\pgfsetdash{}{0pt}%
\pgfsys@defobject{currentmarker}{\pgfqpoint{-0.048611in}{0.000000in}}{\pgfqpoint{0.000000in}{0.000000in}}{%
\pgfpathmoveto{\pgfqpoint{0.000000in}{0.000000in}}%
\pgfpathlineto{\pgfqpoint{-0.048611in}{0.000000in}}%
\pgfusepath{stroke,fill}%
}%
\begin{pgfscope}%
\pgfsys@transformshift{5.814694in}{2.849537in}%
\pgfsys@useobject{currentmarker}{}%
\end{pgfscope}%
\end{pgfscope}%
\begin{pgfscope}%
\pgftext[x=5.427351in,y=2.786223in,left,base]{\rmfamily\fontsize{12.000000}{14.400000}\selectfont \(\displaystyle 0.00\)}%
\end{pgfscope}%
\begin{pgfscope}%
\pgfsetbuttcap%
\pgfsetroundjoin%
\definecolor{currentfill}{rgb}{0.000000,0.000000,0.000000}%
\pgfsetfillcolor{currentfill}%
\pgfsetlinewidth{0.803000pt}%
\definecolor{currentstroke}{rgb}{0.000000,0.000000,0.000000}%
\pgfsetstrokecolor{currentstroke}%
\pgfsetdash{}{0pt}%
\pgfsys@defobject{currentmarker}{\pgfqpoint{-0.048611in}{0.000000in}}{\pgfqpoint{0.000000in}{0.000000in}}{%
\pgfpathmoveto{\pgfqpoint{0.000000in}{0.000000in}}%
\pgfpathlineto{\pgfqpoint{-0.048611in}{0.000000in}}%
\pgfusepath{stroke,fill}%
}%
\begin{pgfscope}%
\pgfsys@transformshift{5.814694in}{3.192719in}%
\pgfsys@useobject{currentmarker}{}%
\end{pgfscope}%
\end{pgfscope}%
\begin{pgfscope}%
\pgftext[x=5.427351in,y=3.129405in,left,base]{\rmfamily\fontsize{12.000000}{14.400000}\selectfont \(\displaystyle 0.25\)}%
\end{pgfscope}%
\begin{pgfscope}%
\pgfsetbuttcap%
\pgfsetroundjoin%
\definecolor{currentfill}{rgb}{0.000000,0.000000,0.000000}%
\pgfsetfillcolor{currentfill}%
\pgfsetlinewidth{0.803000pt}%
\definecolor{currentstroke}{rgb}{0.000000,0.000000,0.000000}%
\pgfsetstrokecolor{currentstroke}%
\pgfsetdash{}{0pt}%
\pgfsys@defobject{currentmarker}{\pgfqpoint{-0.048611in}{0.000000in}}{\pgfqpoint{0.000000in}{0.000000in}}{%
\pgfpathmoveto{\pgfqpoint{0.000000in}{0.000000in}}%
\pgfpathlineto{\pgfqpoint{-0.048611in}{0.000000in}}%
\pgfusepath{stroke,fill}%
}%
\begin{pgfscope}%
\pgfsys@transformshift{5.814694in}{3.535900in}%
\pgfsys@useobject{currentmarker}{}%
\end{pgfscope}%
\end{pgfscope}%
\begin{pgfscope}%
\pgftext[x=5.427351in,y=3.472587in,left,base]{\rmfamily\fontsize{12.000000}{14.400000}\selectfont \(\displaystyle 0.50\)}%
\end{pgfscope}%
\begin{pgfscope}%
\pgfsetbuttcap%
\pgfsetroundjoin%
\definecolor{currentfill}{rgb}{0.000000,0.000000,0.000000}%
\pgfsetfillcolor{currentfill}%
\pgfsetlinewidth{0.803000pt}%
\definecolor{currentstroke}{rgb}{0.000000,0.000000,0.000000}%
\pgfsetstrokecolor{currentstroke}%
\pgfsetdash{}{0pt}%
\pgfsys@defobject{currentmarker}{\pgfqpoint{-0.048611in}{0.000000in}}{\pgfqpoint{0.000000in}{0.000000in}}{%
\pgfpathmoveto{\pgfqpoint{0.000000in}{0.000000in}}%
\pgfpathlineto{\pgfqpoint{-0.048611in}{0.000000in}}%
\pgfusepath{stroke,fill}%
}%
\begin{pgfscope}%
\pgfsys@transformshift{5.814694in}{3.879082in}%
\pgfsys@useobject{currentmarker}{}%
\end{pgfscope}%
\end{pgfscope}%
\begin{pgfscope}%
\pgftext[x=5.427351in,y=3.815768in,left,base]{\rmfamily\fontsize{12.000000}{14.400000}\selectfont \(\displaystyle 0.75\)}%
\end{pgfscope}%
\begin{pgfscope}%
\pgfsetbuttcap%
\pgfsetroundjoin%
\definecolor{currentfill}{rgb}{0.000000,0.000000,0.000000}%
\pgfsetfillcolor{currentfill}%
\pgfsetlinewidth{0.803000pt}%
\definecolor{currentstroke}{rgb}{0.000000,0.000000,0.000000}%
\pgfsetstrokecolor{currentstroke}%
\pgfsetdash{}{0pt}%
\pgfsys@defobject{currentmarker}{\pgfqpoint{-0.048611in}{0.000000in}}{\pgfqpoint{0.000000in}{0.000000in}}{%
\pgfpathmoveto{\pgfqpoint{0.000000in}{0.000000in}}%
\pgfpathlineto{\pgfqpoint{-0.048611in}{0.000000in}}%
\pgfusepath{stroke,fill}%
}%
\begin{pgfscope}%
\pgfsys@transformshift{5.814694in}{4.222264in}%
\pgfsys@useobject{currentmarker}{}%
\end{pgfscope}%
\end{pgfscope}%
\begin{pgfscope}%
\pgftext[x=5.427351in,y=4.158950in,left,base]{\rmfamily\fontsize{12.000000}{14.400000}\selectfont \(\displaystyle 1.00\)}%
\end{pgfscope}%
\begin{pgfscope}%
\pgfsetrectcap%
\pgfsetmiterjoin%
\pgfsetlinewidth{0.803000pt}%
\definecolor{currentstroke}{rgb}{0.501961,0.501961,0.501961}%
\pgfsetstrokecolor{currentstroke}%
\pgfsetdash{}{0pt}%
\pgfpathmoveto{\pgfqpoint{5.814694in}{2.849537in}}%
\pgfpathlineto{\pgfqpoint{5.814694in}{4.222264in}}%
\pgfusepath{stroke}%
\end{pgfscope}%
\begin{pgfscope}%
\pgfsetrectcap%
\pgfsetmiterjoin%
\pgfsetlinewidth{0.803000pt}%
\definecolor{currentstroke}{rgb}{0.501961,0.501961,0.501961}%
\pgfsetstrokecolor{currentstroke}%
\pgfsetdash{}{0pt}%
\pgfpathmoveto{\pgfqpoint{7.712653in}{2.849537in}}%
\pgfpathlineto{\pgfqpoint{7.712653in}{4.222264in}}%
\pgfusepath{stroke}%
\end{pgfscope}%
\begin{pgfscope}%
\pgfsetrectcap%
\pgfsetmiterjoin%
\pgfsetlinewidth{0.803000pt}%
\definecolor{currentstroke}{rgb}{0.501961,0.501961,0.501961}%
\pgfsetstrokecolor{currentstroke}%
\pgfsetdash{}{0pt}%
\pgfpathmoveto{\pgfqpoint{5.814694in}{2.849537in}}%
\pgfpathlineto{\pgfqpoint{7.712653in}{2.849537in}}%
\pgfusepath{stroke}%
\end{pgfscope}%
\begin{pgfscope}%
\pgfsetrectcap%
\pgfsetmiterjoin%
\pgfsetlinewidth{0.803000pt}%
\definecolor{currentstroke}{rgb}{0.501961,0.501961,0.501961}%
\pgfsetstrokecolor{currentstroke}%
\pgfsetdash{}{0pt}%
\pgfpathmoveto{\pgfqpoint{5.814694in}{4.222264in}}%
\pgfpathlineto{\pgfqpoint{7.712653in}{4.222264in}}%
\pgfusepath{stroke}%
\end{pgfscope}%
\begin{pgfscope}%
\pgfsetbuttcap%
\pgfsetmiterjoin%
\definecolor{currentfill}{rgb}{1.000000,1.000000,1.000000}%
\pgfsetfillcolor{currentfill}%
\pgfsetlinewidth{0.000000pt}%
\definecolor{currentstroke}{rgb}{0.000000,0.000000,0.000000}%
\pgfsetstrokecolor{currentstroke}%
\pgfsetstrokeopacity{0.000000}%
\pgfsetdash{}{0pt}%
\pgfpathmoveto{\pgfqpoint{8.282041in}{2.849537in}}%
\pgfpathlineto{\pgfqpoint{10.180000in}{2.849537in}}%
\pgfpathlineto{\pgfqpoint{10.180000in}{4.222264in}}%
\pgfpathlineto{\pgfqpoint{8.282041in}{4.222264in}}%
\pgfpathclose%
\pgfusepath{fill}%
\end{pgfscope}%
\begin{pgfscope}%
\pgfsetbuttcap%
\pgfsetroundjoin%
\definecolor{currentfill}{rgb}{0.000000,0.000000,0.000000}%
\pgfsetfillcolor{currentfill}%
\pgfsetlinewidth{0.803000pt}%
\definecolor{currentstroke}{rgb}{0.000000,0.000000,0.000000}%
\pgfsetstrokecolor{currentstroke}%
\pgfsetdash{}{0pt}%
\pgfsys@defobject{currentmarker}{\pgfqpoint{0.000000in}{-0.048611in}}{\pgfqpoint{0.000000in}{0.000000in}}{%
\pgfpathmoveto{\pgfqpoint{0.000000in}{0.000000in}}%
\pgfpathlineto{\pgfqpoint{0.000000in}{-0.048611in}}%
\pgfusepath{stroke,fill}%
}%
\begin{pgfscope}%
\pgfsys@transformshift{8.282041in}{2.849537in}%
\pgfsys@useobject{currentmarker}{}%
\end{pgfscope}%
\end{pgfscope}%
\begin{pgfscope}%
\pgftext[x=8.282041in,y=2.752315in,,top]{\rmfamily\fontsize{12.000000}{14.400000}\selectfont \(\displaystyle 0.0\)}%
\end{pgfscope}%
\begin{pgfscope}%
\pgfsetbuttcap%
\pgfsetroundjoin%
\definecolor{currentfill}{rgb}{0.000000,0.000000,0.000000}%
\pgfsetfillcolor{currentfill}%
\pgfsetlinewidth{0.803000pt}%
\definecolor{currentstroke}{rgb}{0.000000,0.000000,0.000000}%
\pgfsetstrokecolor{currentstroke}%
\pgfsetdash{}{0pt}%
\pgfsys@defobject{currentmarker}{\pgfqpoint{0.000000in}{-0.048611in}}{\pgfqpoint{0.000000in}{0.000000in}}{%
\pgfpathmoveto{\pgfqpoint{0.000000in}{0.000000in}}%
\pgfpathlineto{\pgfqpoint{0.000000in}{-0.048611in}}%
\pgfusepath{stroke,fill}%
}%
\begin{pgfscope}%
\pgfsys@transformshift{9.231020in}{2.849537in}%
\pgfsys@useobject{currentmarker}{}%
\end{pgfscope}%
\end{pgfscope}%
\begin{pgfscope}%
\pgftext[x=9.231020in,y=2.752315in,,top]{\rmfamily\fontsize{12.000000}{14.400000}\selectfont \(\displaystyle 0.5\)}%
\end{pgfscope}%
\begin{pgfscope}%
\pgfsetbuttcap%
\pgfsetroundjoin%
\definecolor{currentfill}{rgb}{0.000000,0.000000,0.000000}%
\pgfsetfillcolor{currentfill}%
\pgfsetlinewidth{0.803000pt}%
\definecolor{currentstroke}{rgb}{0.000000,0.000000,0.000000}%
\pgfsetstrokecolor{currentstroke}%
\pgfsetdash{}{0pt}%
\pgfsys@defobject{currentmarker}{\pgfqpoint{0.000000in}{-0.048611in}}{\pgfqpoint{0.000000in}{0.000000in}}{%
\pgfpathmoveto{\pgfqpoint{0.000000in}{0.000000in}}%
\pgfpathlineto{\pgfqpoint{0.000000in}{-0.048611in}}%
\pgfusepath{stroke,fill}%
}%
\begin{pgfscope}%
\pgfsys@transformshift{10.180000in}{2.849537in}%
\pgfsys@useobject{currentmarker}{}%
\end{pgfscope}%
\end{pgfscope}%
\begin{pgfscope}%
\pgftext[x=10.180000in,y=2.752315in,,top]{\rmfamily\fontsize{12.000000}{14.400000}\selectfont \(\displaystyle 1.0\)}%
\end{pgfscope}%
\begin{pgfscope}%
\pgfsetbuttcap%
\pgfsetroundjoin%
\definecolor{currentfill}{rgb}{0.000000,0.000000,0.000000}%
\pgfsetfillcolor{currentfill}%
\pgfsetlinewidth{0.803000pt}%
\definecolor{currentstroke}{rgb}{0.000000,0.000000,0.000000}%
\pgfsetstrokecolor{currentstroke}%
\pgfsetdash{}{0pt}%
\pgfsys@defobject{currentmarker}{\pgfqpoint{-0.048611in}{0.000000in}}{\pgfqpoint{0.000000in}{0.000000in}}{%
\pgfpathmoveto{\pgfqpoint{0.000000in}{0.000000in}}%
\pgfpathlineto{\pgfqpoint{-0.048611in}{0.000000in}}%
\pgfusepath{stroke,fill}%
}%
\begin{pgfscope}%
\pgfsys@transformshift{8.282041in}{2.849537in}%
\pgfsys@useobject{currentmarker}{}%
\end{pgfscope}%
\end{pgfscope}%
\begin{pgfscope}%
\pgftext[x=7.894698in,y=2.786223in,left,base]{\rmfamily\fontsize{12.000000}{14.400000}\selectfont \(\displaystyle 0.00\)}%
\end{pgfscope}%
\begin{pgfscope}%
\pgfsetbuttcap%
\pgfsetroundjoin%
\definecolor{currentfill}{rgb}{0.000000,0.000000,0.000000}%
\pgfsetfillcolor{currentfill}%
\pgfsetlinewidth{0.803000pt}%
\definecolor{currentstroke}{rgb}{0.000000,0.000000,0.000000}%
\pgfsetstrokecolor{currentstroke}%
\pgfsetdash{}{0pt}%
\pgfsys@defobject{currentmarker}{\pgfqpoint{-0.048611in}{0.000000in}}{\pgfqpoint{0.000000in}{0.000000in}}{%
\pgfpathmoveto{\pgfqpoint{0.000000in}{0.000000in}}%
\pgfpathlineto{\pgfqpoint{-0.048611in}{0.000000in}}%
\pgfusepath{stroke,fill}%
}%
\begin{pgfscope}%
\pgfsys@transformshift{8.282041in}{3.192719in}%
\pgfsys@useobject{currentmarker}{}%
\end{pgfscope}%
\end{pgfscope}%
\begin{pgfscope}%
\pgftext[x=7.894698in,y=3.129405in,left,base]{\rmfamily\fontsize{12.000000}{14.400000}\selectfont \(\displaystyle 0.25\)}%
\end{pgfscope}%
\begin{pgfscope}%
\pgfsetbuttcap%
\pgfsetroundjoin%
\definecolor{currentfill}{rgb}{0.000000,0.000000,0.000000}%
\pgfsetfillcolor{currentfill}%
\pgfsetlinewidth{0.803000pt}%
\definecolor{currentstroke}{rgb}{0.000000,0.000000,0.000000}%
\pgfsetstrokecolor{currentstroke}%
\pgfsetdash{}{0pt}%
\pgfsys@defobject{currentmarker}{\pgfqpoint{-0.048611in}{0.000000in}}{\pgfqpoint{0.000000in}{0.000000in}}{%
\pgfpathmoveto{\pgfqpoint{0.000000in}{0.000000in}}%
\pgfpathlineto{\pgfqpoint{-0.048611in}{0.000000in}}%
\pgfusepath{stroke,fill}%
}%
\begin{pgfscope}%
\pgfsys@transformshift{8.282041in}{3.535900in}%
\pgfsys@useobject{currentmarker}{}%
\end{pgfscope}%
\end{pgfscope}%
\begin{pgfscope}%
\pgftext[x=7.894698in,y=3.472587in,left,base]{\rmfamily\fontsize{12.000000}{14.400000}\selectfont \(\displaystyle 0.50\)}%
\end{pgfscope}%
\begin{pgfscope}%
\pgfsetbuttcap%
\pgfsetroundjoin%
\definecolor{currentfill}{rgb}{0.000000,0.000000,0.000000}%
\pgfsetfillcolor{currentfill}%
\pgfsetlinewidth{0.803000pt}%
\definecolor{currentstroke}{rgb}{0.000000,0.000000,0.000000}%
\pgfsetstrokecolor{currentstroke}%
\pgfsetdash{}{0pt}%
\pgfsys@defobject{currentmarker}{\pgfqpoint{-0.048611in}{0.000000in}}{\pgfqpoint{0.000000in}{0.000000in}}{%
\pgfpathmoveto{\pgfqpoint{0.000000in}{0.000000in}}%
\pgfpathlineto{\pgfqpoint{-0.048611in}{0.000000in}}%
\pgfusepath{stroke,fill}%
}%
\begin{pgfscope}%
\pgfsys@transformshift{8.282041in}{3.879082in}%
\pgfsys@useobject{currentmarker}{}%
\end{pgfscope}%
\end{pgfscope}%
\begin{pgfscope}%
\pgftext[x=7.894698in,y=3.815768in,left,base]{\rmfamily\fontsize{12.000000}{14.400000}\selectfont \(\displaystyle 0.75\)}%
\end{pgfscope}%
\begin{pgfscope}%
\pgfsetbuttcap%
\pgfsetroundjoin%
\definecolor{currentfill}{rgb}{0.000000,0.000000,0.000000}%
\pgfsetfillcolor{currentfill}%
\pgfsetlinewidth{0.803000pt}%
\definecolor{currentstroke}{rgb}{0.000000,0.000000,0.000000}%
\pgfsetstrokecolor{currentstroke}%
\pgfsetdash{}{0pt}%
\pgfsys@defobject{currentmarker}{\pgfqpoint{-0.048611in}{0.000000in}}{\pgfqpoint{0.000000in}{0.000000in}}{%
\pgfpathmoveto{\pgfqpoint{0.000000in}{0.000000in}}%
\pgfpathlineto{\pgfqpoint{-0.048611in}{0.000000in}}%
\pgfusepath{stroke,fill}%
}%
\begin{pgfscope}%
\pgfsys@transformshift{8.282041in}{4.222264in}%
\pgfsys@useobject{currentmarker}{}%
\end{pgfscope}%
\end{pgfscope}%
\begin{pgfscope}%
\pgftext[x=7.894698in,y=4.158950in,left,base]{\rmfamily\fontsize{12.000000}{14.400000}\selectfont \(\displaystyle 1.00\)}%
\end{pgfscope}%
\begin{pgfscope}%
\pgfsetrectcap%
\pgfsetmiterjoin%
\pgfsetlinewidth{0.803000pt}%
\definecolor{currentstroke}{rgb}{0.501961,0.501961,0.501961}%
\pgfsetstrokecolor{currentstroke}%
\pgfsetdash{}{0pt}%
\pgfpathmoveto{\pgfqpoint{8.282041in}{2.849537in}}%
\pgfpathlineto{\pgfqpoint{8.282041in}{4.222264in}}%
\pgfusepath{stroke}%
\end{pgfscope}%
\begin{pgfscope}%
\pgfsetrectcap%
\pgfsetmiterjoin%
\pgfsetlinewidth{0.803000pt}%
\definecolor{currentstroke}{rgb}{0.501961,0.501961,0.501961}%
\pgfsetstrokecolor{currentstroke}%
\pgfsetdash{}{0pt}%
\pgfpathmoveto{\pgfqpoint{10.180000in}{2.849537in}}%
\pgfpathlineto{\pgfqpoint{10.180000in}{4.222264in}}%
\pgfusepath{stroke}%
\end{pgfscope}%
\begin{pgfscope}%
\pgfsetrectcap%
\pgfsetmiterjoin%
\pgfsetlinewidth{0.803000pt}%
\definecolor{currentstroke}{rgb}{0.501961,0.501961,0.501961}%
\pgfsetstrokecolor{currentstroke}%
\pgfsetdash{}{0pt}%
\pgfpathmoveto{\pgfqpoint{8.282041in}{2.849537in}}%
\pgfpathlineto{\pgfqpoint{10.180000in}{2.849537in}}%
\pgfusepath{stroke}%
\end{pgfscope}%
\begin{pgfscope}%
\pgfsetrectcap%
\pgfsetmiterjoin%
\pgfsetlinewidth{0.803000pt}%
\definecolor{currentstroke}{rgb}{0.501961,0.501961,0.501961}%
\pgfsetstrokecolor{currentstroke}%
\pgfsetdash{}{0pt}%
\pgfpathmoveto{\pgfqpoint{8.282041in}{4.222264in}}%
\pgfpathlineto{\pgfqpoint{10.180000in}{4.222264in}}%
\pgfusepath{stroke}%
\end{pgfscope}%
\begin{pgfscope}%
\pgfsetbuttcap%
\pgfsetmiterjoin%
\definecolor{currentfill}{rgb}{1.000000,1.000000,1.000000}%
\pgfsetfillcolor{currentfill}%
\pgfsetlinewidth{0.000000pt}%
\definecolor{currentstroke}{rgb}{0.000000,0.000000,0.000000}%
\pgfsetstrokecolor{currentstroke}%
\pgfsetstrokeopacity{0.000000}%
\pgfsetdash{}{0pt}%
\pgfpathmoveto{\pgfqpoint{0.880000in}{0.790446in}}%
\pgfpathlineto{\pgfqpoint{2.777959in}{0.790446in}}%
\pgfpathlineto{\pgfqpoint{2.777959in}{2.163173in}}%
\pgfpathlineto{\pgfqpoint{0.880000in}{2.163173in}}%
\pgfpathclose%
\pgfusepath{fill}%
\end{pgfscope}%
\begin{pgfscope}%
\pgfsetbuttcap%
\pgfsetroundjoin%
\definecolor{currentfill}{rgb}{0.000000,0.000000,0.000000}%
\pgfsetfillcolor{currentfill}%
\pgfsetlinewidth{0.803000pt}%
\definecolor{currentstroke}{rgb}{0.000000,0.000000,0.000000}%
\pgfsetstrokecolor{currentstroke}%
\pgfsetdash{}{0pt}%
\pgfsys@defobject{currentmarker}{\pgfqpoint{0.000000in}{-0.048611in}}{\pgfqpoint{0.000000in}{0.000000in}}{%
\pgfpathmoveto{\pgfqpoint{0.000000in}{0.000000in}}%
\pgfpathlineto{\pgfqpoint{0.000000in}{-0.048611in}}%
\pgfusepath{stroke,fill}%
}%
\begin{pgfscope}%
\pgfsys@transformshift{0.880000in}{0.790446in}%
\pgfsys@useobject{currentmarker}{}%
\end{pgfscope}%
\end{pgfscope}%
\begin{pgfscope}%
\pgftext[x=0.880000in,y=0.693224in,,top]{\rmfamily\fontsize{12.000000}{14.400000}\selectfont \(\displaystyle 0.0\)}%
\end{pgfscope}%
\begin{pgfscope}%
\pgfsetbuttcap%
\pgfsetroundjoin%
\definecolor{currentfill}{rgb}{0.000000,0.000000,0.000000}%
\pgfsetfillcolor{currentfill}%
\pgfsetlinewidth{0.803000pt}%
\definecolor{currentstroke}{rgb}{0.000000,0.000000,0.000000}%
\pgfsetstrokecolor{currentstroke}%
\pgfsetdash{}{0pt}%
\pgfsys@defobject{currentmarker}{\pgfqpoint{0.000000in}{-0.048611in}}{\pgfqpoint{0.000000in}{0.000000in}}{%
\pgfpathmoveto{\pgfqpoint{0.000000in}{0.000000in}}%
\pgfpathlineto{\pgfqpoint{0.000000in}{-0.048611in}}%
\pgfusepath{stroke,fill}%
}%
\begin{pgfscope}%
\pgfsys@transformshift{1.828980in}{0.790446in}%
\pgfsys@useobject{currentmarker}{}%
\end{pgfscope}%
\end{pgfscope}%
\begin{pgfscope}%
\pgftext[x=1.828980in,y=0.693224in,,top]{\rmfamily\fontsize{12.000000}{14.400000}\selectfont \(\displaystyle 0.5\)}%
\end{pgfscope}%
\begin{pgfscope}%
\pgfsetbuttcap%
\pgfsetroundjoin%
\definecolor{currentfill}{rgb}{0.000000,0.000000,0.000000}%
\pgfsetfillcolor{currentfill}%
\pgfsetlinewidth{0.803000pt}%
\definecolor{currentstroke}{rgb}{0.000000,0.000000,0.000000}%
\pgfsetstrokecolor{currentstroke}%
\pgfsetdash{}{0pt}%
\pgfsys@defobject{currentmarker}{\pgfqpoint{0.000000in}{-0.048611in}}{\pgfqpoint{0.000000in}{0.000000in}}{%
\pgfpathmoveto{\pgfqpoint{0.000000in}{0.000000in}}%
\pgfpathlineto{\pgfqpoint{0.000000in}{-0.048611in}}%
\pgfusepath{stroke,fill}%
}%
\begin{pgfscope}%
\pgfsys@transformshift{2.777959in}{0.790446in}%
\pgfsys@useobject{currentmarker}{}%
\end{pgfscope}%
\end{pgfscope}%
\begin{pgfscope}%
\pgftext[x=2.777959in,y=0.693224in,,top]{\rmfamily\fontsize{12.000000}{14.400000}\selectfont \(\displaystyle 1.0\)}%
\end{pgfscope}%
\begin{pgfscope}%
\pgfsetbuttcap%
\pgfsetroundjoin%
\definecolor{currentfill}{rgb}{0.000000,0.000000,0.000000}%
\pgfsetfillcolor{currentfill}%
\pgfsetlinewidth{0.803000pt}%
\definecolor{currentstroke}{rgb}{0.000000,0.000000,0.000000}%
\pgfsetstrokecolor{currentstroke}%
\pgfsetdash{}{0pt}%
\pgfsys@defobject{currentmarker}{\pgfqpoint{-0.048611in}{0.000000in}}{\pgfqpoint{0.000000in}{0.000000in}}{%
\pgfpathmoveto{\pgfqpoint{0.000000in}{0.000000in}}%
\pgfpathlineto{\pgfqpoint{-0.048611in}{0.000000in}}%
\pgfusepath{stroke,fill}%
}%
\begin{pgfscope}%
\pgfsys@transformshift{0.880000in}{0.790446in}%
\pgfsys@useobject{currentmarker}{}%
\end{pgfscope}%
\end{pgfscope}%
\begin{pgfscope}%
\pgftext[x=0.492657in,y=0.727132in,left,base]{\rmfamily\fontsize{12.000000}{14.400000}\selectfont \(\displaystyle 0.00\)}%
\end{pgfscope}%
\begin{pgfscope}%
\pgfsetbuttcap%
\pgfsetroundjoin%
\definecolor{currentfill}{rgb}{0.000000,0.000000,0.000000}%
\pgfsetfillcolor{currentfill}%
\pgfsetlinewidth{0.803000pt}%
\definecolor{currentstroke}{rgb}{0.000000,0.000000,0.000000}%
\pgfsetstrokecolor{currentstroke}%
\pgfsetdash{}{0pt}%
\pgfsys@defobject{currentmarker}{\pgfqpoint{-0.048611in}{0.000000in}}{\pgfqpoint{0.000000in}{0.000000in}}{%
\pgfpathmoveto{\pgfqpoint{0.000000in}{0.000000in}}%
\pgfpathlineto{\pgfqpoint{-0.048611in}{0.000000in}}%
\pgfusepath{stroke,fill}%
}%
\begin{pgfscope}%
\pgfsys@transformshift{0.880000in}{1.133628in}%
\pgfsys@useobject{currentmarker}{}%
\end{pgfscope}%
\end{pgfscope}%
\begin{pgfscope}%
\pgftext[x=0.492657in,y=1.070314in,left,base]{\rmfamily\fontsize{12.000000}{14.400000}\selectfont \(\displaystyle 0.25\)}%
\end{pgfscope}%
\begin{pgfscope}%
\pgfsetbuttcap%
\pgfsetroundjoin%
\definecolor{currentfill}{rgb}{0.000000,0.000000,0.000000}%
\pgfsetfillcolor{currentfill}%
\pgfsetlinewidth{0.803000pt}%
\definecolor{currentstroke}{rgb}{0.000000,0.000000,0.000000}%
\pgfsetstrokecolor{currentstroke}%
\pgfsetdash{}{0pt}%
\pgfsys@defobject{currentmarker}{\pgfqpoint{-0.048611in}{0.000000in}}{\pgfqpoint{0.000000in}{0.000000in}}{%
\pgfpathmoveto{\pgfqpoint{0.000000in}{0.000000in}}%
\pgfpathlineto{\pgfqpoint{-0.048611in}{0.000000in}}%
\pgfusepath{stroke,fill}%
}%
\begin{pgfscope}%
\pgfsys@transformshift{0.880000in}{1.476809in}%
\pgfsys@useobject{currentmarker}{}%
\end{pgfscope}%
\end{pgfscope}%
\begin{pgfscope}%
\pgftext[x=0.492657in,y=1.413496in,left,base]{\rmfamily\fontsize{12.000000}{14.400000}\selectfont \(\displaystyle 0.50\)}%
\end{pgfscope}%
\begin{pgfscope}%
\pgfsetbuttcap%
\pgfsetroundjoin%
\definecolor{currentfill}{rgb}{0.000000,0.000000,0.000000}%
\pgfsetfillcolor{currentfill}%
\pgfsetlinewidth{0.803000pt}%
\definecolor{currentstroke}{rgb}{0.000000,0.000000,0.000000}%
\pgfsetstrokecolor{currentstroke}%
\pgfsetdash{}{0pt}%
\pgfsys@defobject{currentmarker}{\pgfqpoint{-0.048611in}{0.000000in}}{\pgfqpoint{0.000000in}{0.000000in}}{%
\pgfpathmoveto{\pgfqpoint{0.000000in}{0.000000in}}%
\pgfpathlineto{\pgfqpoint{-0.048611in}{0.000000in}}%
\pgfusepath{stroke,fill}%
}%
\begin{pgfscope}%
\pgfsys@transformshift{0.880000in}{1.819991in}%
\pgfsys@useobject{currentmarker}{}%
\end{pgfscope}%
\end{pgfscope}%
\begin{pgfscope}%
\pgftext[x=0.492657in,y=1.756677in,left,base]{\rmfamily\fontsize{12.000000}{14.400000}\selectfont \(\displaystyle 0.75\)}%
\end{pgfscope}%
\begin{pgfscope}%
\pgfsetbuttcap%
\pgfsetroundjoin%
\definecolor{currentfill}{rgb}{0.000000,0.000000,0.000000}%
\pgfsetfillcolor{currentfill}%
\pgfsetlinewidth{0.803000pt}%
\definecolor{currentstroke}{rgb}{0.000000,0.000000,0.000000}%
\pgfsetstrokecolor{currentstroke}%
\pgfsetdash{}{0pt}%
\pgfsys@defobject{currentmarker}{\pgfqpoint{-0.048611in}{0.000000in}}{\pgfqpoint{0.000000in}{0.000000in}}{%
\pgfpathmoveto{\pgfqpoint{0.000000in}{0.000000in}}%
\pgfpathlineto{\pgfqpoint{-0.048611in}{0.000000in}}%
\pgfusepath{stroke,fill}%
}%
\begin{pgfscope}%
\pgfsys@transformshift{0.880000in}{2.163173in}%
\pgfsys@useobject{currentmarker}{}%
\end{pgfscope}%
\end{pgfscope}%
\begin{pgfscope}%
\pgftext[x=0.492657in,y=2.099859in,left,base]{\rmfamily\fontsize{12.000000}{14.400000}\selectfont \(\displaystyle 1.00\)}%
\end{pgfscope}%
\begin{pgfscope}%
\pgfsetrectcap%
\pgfsetmiterjoin%
\pgfsetlinewidth{0.803000pt}%
\definecolor{currentstroke}{rgb}{0.501961,0.501961,0.501961}%
\pgfsetstrokecolor{currentstroke}%
\pgfsetdash{}{0pt}%
\pgfpathmoveto{\pgfqpoint{0.880000in}{0.790446in}}%
\pgfpathlineto{\pgfqpoint{0.880000in}{2.163173in}}%
\pgfusepath{stroke}%
\end{pgfscope}%
\begin{pgfscope}%
\pgfsetrectcap%
\pgfsetmiterjoin%
\pgfsetlinewidth{0.803000pt}%
\definecolor{currentstroke}{rgb}{0.501961,0.501961,0.501961}%
\pgfsetstrokecolor{currentstroke}%
\pgfsetdash{}{0pt}%
\pgfpathmoveto{\pgfqpoint{2.777959in}{0.790446in}}%
\pgfpathlineto{\pgfqpoint{2.777959in}{2.163173in}}%
\pgfusepath{stroke}%
\end{pgfscope}%
\begin{pgfscope}%
\pgfsetrectcap%
\pgfsetmiterjoin%
\pgfsetlinewidth{0.803000pt}%
\definecolor{currentstroke}{rgb}{0.501961,0.501961,0.501961}%
\pgfsetstrokecolor{currentstroke}%
\pgfsetdash{}{0pt}%
\pgfpathmoveto{\pgfqpoint{0.880000in}{0.790446in}}%
\pgfpathlineto{\pgfqpoint{2.777959in}{0.790446in}}%
\pgfusepath{stroke}%
\end{pgfscope}%
\begin{pgfscope}%
\pgfsetrectcap%
\pgfsetmiterjoin%
\pgfsetlinewidth{0.803000pt}%
\definecolor{currentstroke}{rgb}{0.501961,0.501961,0.501961}%
\pgfsetstrokecolor{currentstroke}%
\pgfsetdash{}{0pt}%
\pgfpathmoveto{\pgfqpoint{0.880000in}{2.163173in}}%
\pgfpathlineto{\pgfqpoint{2.777959in}{2.163173in}}%
\pgfusepath{stroke}%
\end{pgfscope}%
\begin{pgfscope}%
\pgfsetbuttcap%
\pgfsetmiterjoin%
\definecolor{currentfill}{rgb}{1.000000,1.000000,1.000000}%
\pgfsetfillcolor{currentfill}%
\pgfsetlinewidth{0.000000pt}%
\definecolor{currentstroke}{rgb}{0.000000,0.000000,0.000000}%
\pgfsetstrokecolor{currentstroke}%
\pgfsetstrokeopacity{0.000000}%
\pgfsetdash{}{0pt}%
\pgfpathmoveto{\pgfqpoint{3.347347in}{0.790446in}}%
\pgfpathlineto{\pgfqpoint{5.245306in}{0.790446in}}%
\pgfpathlineto{\pgfqpoint{5.245306in}{2.163173in}}%
\pgfpathlineto{\pgfqpoint{3.347347in}{2.163173in}}%
\pgfpathclose%
\pgfusepath{fill}%
\end{pgfscope}%
\begin{pgfscope}%
\pgfsetbuttcap%
\pgfsetroundjoin%
\definecolor{currentfill}{rgb}{0.000000,0.000000,0.000000}%
\pgfsetfillcolor{currentfill}%
\pgfsetlinewidth{0.803000pt}%
\definecolor{currentstroke}{rgb}{0.000000,0.000000,0.000000}%
\pgfsetstrokecolor{currentstroke}%
\pgfsetdash{}{0pt}%
\pgfsys@defobject{currentmarker}{\pgfqpoint{0.000000in}{-0.048611in}}{\pgfqpoint{0.000000in}{0.000000in}}{%
\pgfpathmoveto{\pgfqpoint{0.000000in}{0.000000in}}%
\pgfpathlineto{\pgfqpoint{0.000000in}{-0.048611in}}%
\pgfusepath{stroke,fill}%
}%
\begin{pgfscope}%
\pgfsys@transformshift{3.347347in}{0.790446in}%
\pgfsys@useobject{currentmarker}{}%
\end{pgfscope}%
\end{pgfscope}%
\begin{pgfscope}%
\pgftext[x=3.347347in,y=0.693224in,,top]{\rmfamily\fontsize{12.000000}{14.400000}\selectfont \(\displaystyle 0.0\)}%
\end{pgfscope}%
\begin{pgfscope}%
\pgfsetbuttcap%
\pgfsetroundjoin%
\definecolor{currentfill}{rgb}{0.000000,0.000000,0.000000}%
\pgfsetfillcolor{currentfill}%
\pgfsetlinewidth{0.803000pt}%
\definecolor{currentstroke}{rgb}{0.000000,0.000000,0.000000}%
\pgfsetstrokecolor{currentstroke}%
\pgfsetdash{}{0pt}%
\pgfsys@defobject{currentmarker}{\pgfqpoint{0.000000in}{-0.048611in}}{\pgfqpoint{0.000000in}{0.000000in}}{%
\pgfpathmoveto{\pgfqpoint{0.000000in}{0.000000in}}%
\pgfpathlineto{\pgfqpoint{0.000000in}{-0.048611in}}%
\pgfusepath{stroke,fill}%
}%
\begin{pgfscope}%
\pgfsys@transformshift{4.296327in}{0.790446in}%
\pgfsys@useobject{currentmarker}{}%
\end{pgfscope}%
\end{pgfscope}%
\begin{pgfscope}%
\pgftext[x=4.296327in,y=0.693224in,,top]{\rmfamily\fontsize{12.000000}{14.400000}\selectfont \(\displaystyle 0.5\)}%
\end{pgfscope}%
\begin{pgfscope}%
\pgfsetbuttcap%
\pgfsetroundjoin%
\definecolor{currentfill}{rgb}{0.000000,0.000000,0.000000}%
\pgfsetfillcolor{currentfill}%
\pgfsetlinewidth{0.803000pt}%
\definecolor{currentstroke}{rgb}{0.000000,0.000000,0.000000}%
\pgfsetstrokecolor{currentstroke}%
\pgfsetdash{}{0pt}%
\pgfsys@defobject{currentmarker}{\pgfqpoint{0.000000in}{-0.048611in}}{\pgfqpoint{0.000000in}{0.000000in}}{%
\pgfpathmoveto{\pgfqpoint{0.000000in}{0.000000in}}%
\pgfpathlineto{\pgfqpoint{0.000000in}{-0.048611in}}%
\pgfusepath{stroke,fill}%
}%
\begin{pgfscope}%
\pgfsys@transformshift{5.245306in}{0.790446in}%
\pgfsys@useobject{currentmarker}{}%
\end{pgfscope}%
\end{pgfscope}%
\begin{pgfscope}%
\pgftext[x=5.245306in,y=0.693224in,,top]{\rmfamily\fontsize{12.000000}{14.400000}\selectfont \(\displaystyle 1.0\)}%
\end{pgfscope}%
\begin{pgfscope}%
\pgfsetbuttcap%
\pgfsetroundjoin%
\definecolor{currentfill}{rgb}{0.000000,0.000000,0.000000}%
\pgfsetfillcolor{currentfill}%
\pgfsetlinewidth{0.803000pt}%
\definecolor{currentstroke}{rgb}{0.000000,0.000000,0.000000}%
\pgfsetstrokecolor{currentstroke}%
\pgfsetdash{}{0pt}%
\pgfsys@defobject{currentmarker}{\pgfqpoint{-0.048611in}{0.000000in}}{\pgfqpoint{0.000000in}{0.000000in}}{%
\pgfpathmoveto{\pgfqpoint{0.000000in}{0.000000in}}%
\pgfpathlineto{\pgfqpoint{-0.048611in}{0.000000in}}%
\pgfusepath{stroke,fill}%
}%
\begin{pgfscope}%
\pgfsys@transformshift{3.347347in}{0.790446in}%
\pgfsys@useobject{currentmarker}{}%
\end{pgfscope}%
\end{pgfscope}%
\begin{pgfscope}%
\pgftext[x=2.960004in,y=0.727132in,left,base]{\rmfamily\fontsize{12.000000}{14.400000}\selectfont \(\displaystyle 0.00\)}%
\end{pgfscope}%
\begin{pgfscope}%
\pgfsetbuttcap%
\pgfsetroundjoin%
\definecolor{currentfill}{rgb}{0.000000,0.000000,0.000000}%
\pgfsetfillcolor{currentfill}%
\pgfsetlinewidth{0.803000pt}%
\definecolor{currentstroke}{rgb}{0.000000,0.000000,0.000000}%
\pgfsetstrokecolor{currentstroke}%
\pgfsetdash{}{0pt}%
\pgfsys@defobject{currentmarker}{\pgfqpoint{-0.048611in}{0.000000in}}{\pgfqpoint{0.000000in}{0.000000in}}{%
\pgfpathmoveto{\pgfqpoint{0.000000in}{0.000000in}}%
\pgfpathlineto{\pgfqpoint{-0.048611in}{0.000000in}}%
\pgfusepath{stroke,fill}%
}%
\begin{pgfscope}%
\pgfsys@transformshift{3.347347in}{1.133628in}%
\pgfsys@useobject{currentmarker}{}%
\end{pgfscope}%
\end{pgfscope}%
\begin{pgfscope}%
\pgftext[x=2.960004in,y=1.070314in,left,base]{\rmfamily\fontsize{12.000000}{14.400000}\selectfont \(\displaystyle 0.25\)}%
\end{pgfscope}%
\begin{pgfscope}%
\pgfsetbuttcap%
\pgfsetroundjoin%
\definecolor{currentfill}{rgb}{0.000000,0.000000,0.000000}%
\pgfsetfillcolor{currentfill}%
\pgfsetlinewidth{0.803000pt}%
\definecolor{currentstroke}{rgb}{0.000000,0.000000,0.000000}%
\pgfsetstrokecolor{currentstroke}%
\pgfsetdash{}{0pt}%
\pgfsys@defobject{currentmarker}{\pgfqpoint{-0.048611in}{0.000000in}}{\pgfqpoint{0.000000in}{0.000000in}}{%
\pgfpathmoveto{\pgfqpoint{0.000000in}{0.000000in}}%
\pgfpathlineto{\pgfqpoint{-0.048611in}{0.000000in}}%
\pgfusepath{stroke,fill}%
}%
\begin{pgfscope}%
\pgfsys@transformshift{3.347347in}{1.476809in}%
\pgfsys@useobject{currentmarker}{}%
\end{pgfscope}%
\end{pgfscope}%
\begin{pgfscope}%
\pgftext[x=2.960004in,y=1.413496in,left,base]{\rmfamily\fontsize{12.000000}{14.400000}\selectfont \(\displaystyle 0.50\)}%
\end{pgfscope}%
\begin{pgfscope}%
\pgfsetbuttcap%
\pgfsetroundjoin%
\definecolor{currentfill}{rgb}{0.000000,0.000000,0.000000}%
\pgfsetfillcolor{currentfill}%
\pgfsetlinewidth{0.803000pt}%
\definecolor{currentstroke}{rgb}{0.000000,0.000000,0.000000}%
\pgfsetstrokecolor{currentstroke}%
\pgfsetdash{}{0pt}%
\pgfsys@defobject{currentmarker}{\pgfqpoint{-0.048611in}{0.000000in}}{\pgfqpoint{0.000000in}{0.000000in}}{%
\pgfpathmoveto{\pgfqpoint{0.000000in}{0.000000in}}%
\pgfpathlineto{\pgfqpoint{-0.048611in}{0.000000in}}%
\pgfusepath{stroke,fill}%
}%
\begin{pgfscope}%
\pgfsys@transformshift{3.347347in}{1.819991in}%
\pgfsys@useobject{currentmarker}{}%
\end{pgfscope}%
\end{pgfscope}%
\begin{pgfscope}%
\pgftext[x=2.960004in,y=1.756677in,left,base]{\rmfamily\fontsize{12.000000}{14.400000}\selectfont \(\displaystyle 0.75\)}%
\end{pgfscope}%
\begin{pgfscope}%
\pgfsetbuttcap%
\pgfsetroundjoin%
\definecolor{currentfill}{rgb}{0.000000,0.000000,0.000000}%
\pgfsetfillcolor{currentfill}%
\pgfsetlinewidth{0.803000pt}%
\definecolor{currentstroke}{rgb}{0.000000,0.000000,0.000000}%
\pgfsetstrokecolor{currentstroke}%
\pgfsetdash{}{0pt}%
\pgfsys@defobject{currentmarker}{\pgfqpoint{-0.048611in}{0.000000in}}{\pgfqpoint{0.000000in}{0.000000in}}{%
\pgfpathmoveto{\pgfqpoint{0.000000in}{0.000000in}}%
\pgfpathlineto{\pgfqpoint{-0.048611in}{0.000000in}}%
\pgfusepath{stroke,fill}%
}%
\begin{pgfscope}%
\pgfsys@transformshift{3.347347in}{2.163173in}%
\pgfsys@useobject{currentmarker}{}%
\end{pgfscope}%
\end{pgfscope}%
\begin{pgfscope}%
\pgftext[x=2.960004in,y=2.099859in,left,base]{\rmfamily\fontsize{12.000000}{14.400000}\selectfont \(\displaystyle 1.00\)}%
\end{pgfscope}%
\begin{pgfscope}%
\pgfsetrectcap%
\pgfsetmiterjoin%
\pgfsetlinewidth{0.803000pt}%
\definecolor{currentstroke}{rgb}{0.501961,0.501961,0.501961}%
\pgfsetstrokecolor{currentstroke}%
\pgfsetdash{}{0pt}%
\pgfpathmoveto{\pgfqpoint{3.347347in}{0.790446in}}%
\pgfpathlineto{\pgfqpoint{3.347347in}{2.163173in}}%
\pgfusepath{stroke}%
\end{pgfscope}%
\begin{pgfscope}%
\pgfsetrectcap%
\pgfsetmiterjoin%
\pgfsetlinewidth{0.803000pt}%
\definecolor{currentstroke}{rgb}{0.501961,0.501961,0.501961}%
\pgfsetstrokecolor{currentstroke}%
\pgfsetdash{}{0pt}%
\pgfpathmoveto{\pgfqpoint{5.245306in}{0.790446in}}%
\pgfpathlineto{\pgfqpoint{5.245306in}{2.163173in}}%
\pgfusepath{stroke}%
\end{pgfscope}%
\begin{pgfscope}%
\pgfsetrectcap%
\pgfsetmiterjoin%
\pgfsetlinewidth{0.803000pt}%
\definecolor{currentstroke}{rgb}{0.501961,0.501961,0.501961}%
\pgfsetstrokecolor{currentstroke}%
\pgfsetdash{}{0pt}%
\pgfpathmoveto{\pgfqpoint{3.347347in}{0.790446in}}%
\pgfpathlineto{\pgfqpoint{5.245306in}{0.790446in}}%
\pgfusepath{stroke}%
\end{pgfscope}%
\begin{pgfscope}%
\pgfsetrectcap%
\pgfsetmiterjoin%
\pgfsetlinewidth{0.803000pt}%
\definecolor{currentstroke}{rgb}{0.501961,0.501961,0.501961}%
\pgfsetstrokecolor{currentstroke}%
\pgfsetdash{}{0pt}%
\pgfpathmoveto{\pgfqpoint{3.347347in}{2.163173in}}%
\pgfpathlineto{\pgfqpoint{5.245306in}{2.163173in}}%
\pgfusepath{stroke}%
\end{pgfscope}%
\begin{pgfscope}%
\pgfsetbuttcap%
\pgfsetmiterjoin%
\definecolor{currentfill}{rgb}{1.000000,1.000000,1.000000}%
\pgfsetfillcolor{currentfill}%
\pgfsetlinewidth{0.000000pt}%
\definecolor{currentstroke}{rgb}{0.000000,0.000000,0.000000}%
\pgfsetstrokecolor{currentstroke}%
\pgfsetstrokeopacity{0.000000}%
\pgfsetdash{}{0pt}%
\pgfpathmoveto{\pgfqpoint{5.814694in}{0.790446in}}%
\pgfpathlineto{\pgfqpoint{7.712653in}{0.790446in}}%
\pgfpathlineto{\pgfqpoint{7.712653in}{2.163173in}}%
\pgfpathlineto{\pgfqpoint{5.814694in}{2.163173in}}%
\pgfpathclose%
\pgfusepath{fill}%
\end{pgfscope}%
\begin{pgfscope}%
\pgfsetbuttcap%
\pgfsetroundjoin%
\definecolor{currentfill}{rgb}{0.000000,0.000000,0.000000}%
\pgfsetfillcolor{currentfill}%
\pgfsetlinewidth{0.803000pt}%
\definecolor{currentstroke}{rgb}{0.000000,0.000000,0.000000}%
\pgfsetstrokecolor{currentstroke}%
\pgfsetdash{}{0pt}%
\pgfsys@defobject{currentmarker}{\pgfqpoint{0.000000in}{-0.048611in}}{\pgfqpoint{0.000000in}{0.000000in}}{%
\pgfpathmoveto{\pgfqpoint{0.000000in}{0.000000in}}%
\pgfpathlineto{\pgfqpoint{0.000000in}{-0.048611in}}%
\pgfusepath{stroke,fill}%
}%
\begin{pgfscope}%
\pgfsys@transformshift{5.814694in}{0.790446in}%
\pgfsys@useobject{currentmarker}{}%
\end{pgfscope}%
\end{pgfscope}%
\begin{pgfscope}%
\pgftext[x=5.814694in,y=0.693224in,,top]{\rmfamily\fontsize{12.000000}{14.400000}\selectfont \(\displaystyle 0.0\)}%
\end{pgfscope}%
\begin{pgfscope}%
\pgfsetbuttcap%
\pgfsetroundjoin%
\definecolor{currentfill}{rgb}{0.000000,0.000000,0.000000}%
\pgfsetfillcolor{currentfill}%
\pgfsetlinewidth{0.803000pt}%
\definecolor{currentstroke}{rgb}{0.000000,0.000000,0.000000}%
\pgfsetstrokecolor{currentstroke}%
\pgfsetdash{}{0pt}%
\pgfsys@defobject{currentmarker}{\pgfqpoint{0.000000in}{-0.048611in}}{\pgfqpoint{0.000000in}{0.000000in}}{%
\pgfpathmoveto{\pgfqpoint{0.000000in}{0.000000in}}%
\pgfpathlineto{\pgfqpoint{0.000000in}{-0.048611in}}%
\pgfusepath{stroke,fill}%
}%
\begin{pgfscope}%
\pgfsys@transformshift{6.763673in}{0.790446in}%
\pgfsys@useobject{currentmarker}{}%
\end{pgfscope}%
\end{pgfscope}%
\begin{pgfscope}%
\pgftext[x=6.763673in,y=0.693224in,,top]{\rmfamily\fontsize{12.000000}{14.400000}\selectfont \(\displaystyle 0.5\)}%
\end{pgfscope}%
\begin{pgfscope}%
\pgfsetbuttcap%
\pgfsetroundjoin%
\definecolor{currentfill}{rgb}{0.000000,0.000000,0.000000}%
\pgfsetfillcolor{currentfill}%
\pgfsetlinewidth{0.803000pt}%
\definecolor{currentstroke}{rgb}{0.000000,0.000000,0.000000}%
\pgfsetstrokecolor{currentstroke}%
\pgfsetdash{}{0pt}%
\pgfsys@defobject{currentmarker}{\pgfqpoint{0.000000in}{-0.048611in}}{\pgfqpoint{0.000000in}{0.000000in}}{%
\pgfpathmoveto{\pgfqpoint{0.000000in}{0.000000in}}%
\pgfpathlineto{\pgfqpoint{0.000000in}{-0.048611in}}%
\pgfusepath{stroke,fill}%
}%
\begin{pgfscope}%
\pgfsys@transformshift{7.712653in}{0.790446in}%
\pgfsys@useobject{currentmarker}{}%
\end{pgfscope}%
\end{pgfscope}%
\begin{pgfscope}%
\pgftext[x=7.712653in,y=0.693224in,,top]{\rmfamily\fontsize{12.000000}{14.400000}\selectfont \(\displaystyle 1.0\)}%
\end{pgfscope}%
\begin{pgfscope}%
\pgfsetbuttcap%
\pgfsetroundjoin%
\definecolor{currentfill}{rgb}{0.000000,0.000000,0.000000}%
\pgfsetfillcolor{currentfill}%
\pgfsetlinewidth{0.803000pt}%
\definecolor{currentstroke}{rgb}{0.000000,0.000000,0.000000}%
\pgfsetstrokecolor{currentstroke}%
\pgfsetdash{}{0pt}%
\pgfsys@defobject{currentmarker}{\pgfqpoint{-0.048611in}{0.000000in}}{\pgfqpoint{0.000000in}{0.000000in}}{%
\pgfpathmoveto{\pgfqpoint{0.000000in}{0.000000in}}%
\pgfpathlineto{\pgfqpoint{-0.048611in}{0.000000in}}%
\pgfusepath{stroke,fill}%
}%
\begin{pgfscope}%
\pgfsys@transformshift{5.814694in}{0.790446in}%
\pgfsys@useobject{currentmarker}{}%
\end{pgfscope}%
\end{pgfscope}%
\begin{pgfscope}%
\pgftext[x=5.427351in,y=0.727132in,left,base]{\rmfamily\fontsize{12.000000}{14.400000}\selectfont \(\displaystyle 0.00\)}%
\end{pgfscope}%
\begin{pgfscope}%
\pgfsetbuttcap%
\pgfsetroundjoin%
\definecolor{currentfill}{rgb}{0.000000,0.000000,0.000000}%
\pgfsetfillcolor{currentfill}%
\pgfsetlinewidth{0.803000pt}%
\definecolor{currentstroke}{rgb}{0.000000,0.000000,0.000000}%
\pgfsetstrokecolor{currentstroke}%
\pgfsetdash{}{0pt}%
\pgfsys@defobject{currentmarker}{\pgfqpoint{-0.048611in}{0.000000in}}{\pgfqpoint{0.000000in}{0.000000in}}{%
\pgfpathmoveto{\pgfqpoint{0.000000in}{0.000000in}}%
\pgfpathlineto{\pgfqpoint{-0.048611in}{0.000000in}}%
\pgfusepath{stroke,fill}%
}%
\begin{pgfscope}%
\pgfsys@transformshift{5.814694in}{1.133628in}%
\pgfsys@useobject{currentmarker}{}%
\end{pgfscope}%
\end{pgfscope}%
\begin{pgfscope}%
\pgftext[x=5.427351in,y=1.070314in,left,base]{\rmfamily\fontsize{12.000000}{14.400000}\selectfont \(\displaystyle 0.25\)}%
\end{pgfscope}%
\begin{pgfscope}%
\pgfsetbuttcap%
\pgfsetroundjoin%
\definecolor{currentfill}{rgb}{0.000000,0.000000,0.000000}%
\pgfsetfillcolor{currentfill}%
\pgfsetlinewidth{0.803000pt}%
\definecolor{currentstroke}{rgb}{0.000000,0.000000,0.000000}%
\pgfsetstrokecolor{currentstroke}%
\pgfsetdash{}{0pt}%
\pgfsys@defobject{currentmarker}{\pgfqpoint{-0.048611in}{0.000000in}}{\pgfqpoint{0.000000in}{0.000000in}}{%
\pgfpathmoveto{\pgfqpoint{0.000000in}{0.000000in}}%
\pgfpathlineto{\pgfqpoint{-0.048611in}{0.000000in}}%
\pgfusepath{stroke,fill}%
}%
\begin{pgfscope}%
\pgfsys@transformshift{5.814694in}{1.476809in}%
\pgfsys@useobject{currentmarker}{}%
\end{pgfscope}%
\end{pgfscope}%
\begin{pgfscope}%
\pgftext[x=5.427351in,y=1.413496in,left,base]{\rmfamily\fontsize{12.000000}{14.400000}\selectfont \(\displaystyle 0.50\)}%
\end{pgfscope}%
\begin{pgfscope}%
\pgfsetbuttcap%
\pgfsetroundjoin%
\definecolor{currentfill}{rgb}{0.000000,0.000000,0.000000}%
\pgfsetfillcolor{currentfill}%
\pgfsetlinewidth{0.803000pt}%
\definecolor{currentstroke}{rgb}{0.000000,0.000000,0.000000}%
\pgfsetstrokecolor{currentstroke}%
\pgfsetdash{}{0pt}%
\pgfsys@defobject{currentmarker}{\pgfqpoint{-0.048611in}{0.000000in}}{\pgfqpoint{0.000000in}{0.000000in}}{%
\pgfpathmoveto{\pgfqpoint{0.000000in}{0.000000in}}%
\pgfpathlineto{\pgfqpoint{-0.048611in}{0.000000in}}%
\pgfusepath{stroke,fill}%
}%
\begin{pgfscope}%
\pgfsys@transformshift{5.814694in}{1.819991in}%
\pgfsys@useobject{currentmarker}{}%
\end{pgfscope}%
\end{pgfscope}%
\begin{pgfscope}%
\pgftext[x=5.427351in,y=1.756677in,left,base]{\rmfamily\fontsize{12.000000}{14.400000}\selectfont \(\displaystyle 0.75\)}%
\end{pgfscope}%
\begin{pgfscope}%
\pgfsetbuttcap%
\pgfsetroundjoin%
\definecolor{currentfill}{rgb}{0.000000,0.000000,0.000000}%
\pgfsetfillcolor{currentfill}%
\pgfsetlinewidth{0.803000pt}%
\definecolor{currentstroke}{rgb}{0.000000,0.000000,0.000000}%
\pgfsetstrokecolor{currentstroke}%
\pgfsetdash{}{0pt}%
\pgfsys@defobject{currentmarker}{\pgfqpoint{-0.048611in}{0.000000in}}{\pgfqpoint{0.000000in}{0.000000in}}{%
\pgfpathmoveto{\pgfqpoint{0.000000in}{0.000000in}}%
\pgfpathlineto{\pgfqpoint{-0.048611in}{0.000000in}}%
\pgfusepath{stroke,fill}%
}%
\begin{pgfscope}%
\pgfsys@transformshift{5.814694in}{2.163173in}%
\pgfsys@useobject{currentmarker}{}%
\end{pgfscope}%
\end{pgfscope}%
\begin{pgfscope}%
\pgftext[x=5.427351in,y=2.099859in,left,base]{\rmfamily\fontsize{12.000000}{14.400000}\selectfont \(\displaystyle 1.00\)}%
\end{pgfscope}%
\begin{pgfscope}%
\pgfsetrectcap%
\pgfsetmiterjoin%
\pgfsetlinewidth{0.803000pt}%
\definecolor{currentstroke}{rgb}{0.501961,0.501961,0.501961}%
\pgfsetstrokecolor{currentstroke}%
\pgfsetdash{}{0pt}%
\pgfpathmoveto{\pgfqpoint{5.814694in}{0.790446in}}%
\pgfpathlineto{\pgfqpoint{5.814694in}{2.163173in}}%
\pgfusepath{stroke}%
\end{pgfscope}%
\begin{pgfscope}%
\pgfsetrectcap%
\pgfsetmiterjoin%
\pgfsetlinewidth{0.803000pt}%
\definecolor{currentstroke}{rgb}{0.501961,0.501961,0.501961}%
\pgfsetstrokecolor{currentstroke}%
\pgfsetdash{}{0pt}%
\pgfpathmoveto{\pgfqpoint{7.712653in}{0.790446in}}%
\pgfpathlineto{\pgfqpoint{7.712653in}{2.163173in}}%
\pgfusepath{stroke}%
\end{pgfscope}%
\begin{pgfscope}%
\pgfsetrectcap%
\pgfsetmiterjoin%
\pgfsetlinewidth{0.803000pt}%
\definecolor{currentstroke}{rgb}{0.501961,0.501961,0.501961}%
\pgfsetstrokecolor{currentstroke}%
\pgfsetdash{}{0pt}%
\pgfpathmoveto{\pgfqpoint{5.814694in}{0.790446in}}%
\pgfpathlineto{\pgfqpoint{7.712653in}{0.790446in}}%
\pgfusepath{stroke}%
\end{pgfscope}%
\begin{pgfscope}%
\pgfsetrectcap%
\pgfsetmiterjoin%
\pgfsetlinewidth{0.803000pt}%
\definecolor{currentstroke}{rgb}{0.501961,0.501961,0.501961}%
\pgfsetstrokecolor{currentstroke}%
\pgfsetdash{}{0pt}%
\pgfpathmoveto{\pgfqpoint{5.814694in}{2.163173in}}%
\pgfpathlineto{\pgfqpoint{7.712653in}{2.163173in}}%
\pgfusepath{stroke}%
\end{pgfscope}%
\begin{pgfscope}%
\pgfsetbuttcap%
\pgfsetmiterjoin%
\definecolor{currentfill}{rgb}{1.000000,1.000000,1.000000}%
\pgfsetfillcolor{currentfill}%
\pgfsetlinewidth{0.000000pt}%
\definecolor{currentstroke}{rgb}{0.000000,0.000000,0.000000}%
\pgfsetstrokecolor{currentstroke}%
\pgfsetstrokeopacity{0.000000}%
\pgfsetdash{}{0pt}%
\pgfpathmoveto{\pgfqpoint{8.282041in}{0.790446in}}%
\pgfpathlineto{\pgfqpoint{10.180000in}{0.790446in}}%
\pgfpathlineto{\pgfqpoint{10.180000in}{2.163173in}}%
\pgfpathlineto{\pgfqpoint{8.282041in}{2.163173in}}%
\pgfpathclose%
\pgfusepath{fill}%
\end{pgfscope}%
\begin{pgfscope}%
\pgfsetbuttcap%
\pgfsetroundjoin%
\definecolor{currentfill}{rgb}{0.000000,0.000000,0.000000}%
\pgfsetfillcolor{currentfill}%
\pgfsetlinewidth{0.803000pt}%
\definecolor{currentstroke}{rgb}{0.000000,0.000000,0.000000}%
\pgfsetstrokecolor{currentstroke}%
\pgfsetdash{}{0pt}%
\pgfsys@defobject{currentmarker}{\pgfqpoint{0.000000in}{-0.048611in}}{\pgfqpoint{0.000000in}{0.000000in}}{%
\pgfpathmoveto{\pgfqpoint{0.000000in}{0.000000in}}%
\pgfpathlineto{\pgfqpoint{0.000000in}{-0.048611in}}%
\pgfusepath{stroke,fill}%
}%
\begin{pgfscope}%
\pgfsys@transformshift{9.166562in}{0.790446in}%
\pgfsys@useobject{currentmarker}{}%
\end{pgfscope}%
\end{pgfscope}%
\begin{pgfscope}%
\pgftext[x=9.166562in,y=0.693224in,,top]{\rmfamily\fontsize{12.000000}{14.400000}\selectfont \(\displaystyle 0.5\)}%
\end{pgfscope}%
\begin{pgfscope}%
\pgfsetbuttcap%
\pgfsetroundjoin%
\definecolor{currentfill}{rgb}{0.000000,0.000000,0.000000}%
\pgfsetfillcolor{currentfill}%
\pgfsetlinewidth{0.803000pt}%
\definecolor{currentstroke}{rgb}{0.000000,0.000000,0.000000}%
\pgfsetstrokecolor{currentstroke}%
\pgfsetdash{}{0pt}%
\pgfsys@defobject{currentmarker}{\pgfqpoint{0.000000in}{-0.048611in}}{\pgfqpoint{0.000000in}{0.000000in}}{%
\pgfpathmoveto{\pgfqpoint{0.000000in}{0.000000in}}%
\pgfpathlineto{\pgfqpoint{0.000000in}{-0.048611in}}%
\pgfusepath{stroke,fill}%
}%
\begin{pgfscope}%
\pgfsys@transformshift{10.082468in}{0.790446in}%
\pgfsys@useobject{currentmarker}{}%
\end{pgfscope}%
\end{pgfscope}%
\begin{pgfscope}%
\pgftext[x=10.082468in,y=0.693224in,,top]{\rmfamily\fontsize{12.000000}{14.400000}\selectfont \(\displaystyle 1.0\)}%
\end{pgfscope}%
\begin{pgfscope}%
\pgfsetbuttcap%
\pgfsetroundjoin%
\definecolor{currentfill}{rgb}{0.000000,0.000000,0.000000}%
\pgfsetfillcolor{currentfill}%
\pgfsetlinewidth{0.803000pt}%
\definecolor{currentstroke}{rgb}{0.000000,0.000000,0.000000}%
\pgfsetstrokecolor{currentstroke}%
\pgfsetdash{}{0pt}%
\pgfsys@defobject{currentmarker}{\pgfqpoint{-0.048611in}{0.000000in}}{\pgfqpoint{0.000000in}{0.000000in}}{%
\pgfpathmoveto{\pgfqpoint{0.000000in}{0.000000in}}%
\pgfpathlineto{\pgfqpoint{-0.048611in}{0.000000in}}%
\pgfusepath{stroke,fill}%
}%
\begin{pgfscope}%
\pgfsys@transformshift{8.282041in}{1.007426in}%
\pgfsys@useobject{currentmarker}{}%
\end{pgfscope}%
\end{pgfscope}%
\begin{pgfscope}%
\pgftext[x=7.863830in,y=0.944112in,left,base]{\rmfamily\fontsize{12.000000}{14.400000}\selectfont \(\displaystyle 10^{-2}\)}%
\end{pgfscope}%
\begin{pgfscope}%
\pgfsetbuttcap%
\pgfsetroundjoin%
\definecolor{currentfill}{rgb}{0.000000,0.000000,0.000000}%
\pgfsetfillcolor{currentfill}%
\pgfsetlinewidth{0.803000pt}%
\definecolor{currentstroke}{rgb}{0.000000,0.000000,0.000000}%
\pgfsetstrokecolor{currentstroke}%
\pgfsetdash{}{0pt}%
\pgfsys@defobject{currentmarker}{\pgfqpoint{-0.048611in}{0.000000in}}{\pgfqpoint{0.000000in}{0.000000in}}{%
\pgfpathmoveto{\pgfqpoint{0.000000in}{0.000000in}}%
\pgfpathlineto{\pgfqpoint{-0.048611in}{0.000000in}}%
\pgfusepath{stroke,fill}%
}%
\begin{pgfscope}%
\pgfsys@transformshift{8.282041in}{1.558552in}%
\pgfsys@useobject{currentmarker}{}%
\end{pgfscope}%
\end{pgfscope}%
\begin{pgfscope}%
\pgftext[x=7.863830in,y=1.495238in,left,base]{\rmfamily\fontsize{12.000000}{14.400000}\selectfont \(\displaystyle 10^{-1}\)}%
\end{pgfscope}%
\begin{pgfscope}%
\pgfsetbuttcap%
\pgfsetroundjoin%
\definecolor{currentfill}{rgb}{0.000000,0.000000,0.000000}%
\pgfsetfillcolor{currentfill}%
\pgfsetlinewidth{0.803000pt}%
\definecolor{currentstroke}{rgb}{0.000000,0.000000,0.000000}%
\pgfsetstrokecolor{currentstroke}%
\pgfsetdash{}{0pt}%
\pgfsys@defobject{currentmarker}{\pgfqpoint{-0.048611in}{0.000000in}}{\pgfqpoint{0.000000in}{0.000000in}}{%
\pgfpathmoveto{\pgfqpoint{0.000000in}{0.000000in}}%
\pgfpathlineto{\pgfqpoint{-0.048611in}{0.000000in}}%
\pgfusepath{stroke,fill}%
}%
\begin{pgfscope}%
\pgfsys@transformshift{8.282041in}{2.109678in}%
\pgfsys@useobject{currentmarker}{}%
\end{pgfscope}%
\end{pgfscope}%
\begin{pgfscope}%
\pgftext[x=7.955653in,y=2.046364in,left,base]{\rmfamily\fontsize{12.000000}{14.400000}\selectfont \(\displaystyle 10^{0}\)}%
\end{pgfscope}%
\begin{pgfscope}%
\pgfsetbuttcap%
\pgfsetroundjoin%
\definecolor{currentfill}{rgb}{0.000000,0.000000,0.000000}%
\pgfsetfillcolor{currentfill}%
\pgfsetlinewidth{0.602250pt}%
\definecolor{currentstroke}{rgb}{0.000000,0.000000,0.000000}%
\pgfsetstrokecolor{currentstroke}%
\pgfsetdash{}{0pt}%
\pgfsys@defobject{currentmarker}{\pgfqpoint{-0.027778in}{0.000000in}}{\pgfqpoint{0.000000in}{0.000000in}}{%
\pgfpathmoveto{\pgfqpoint{0.000000in}{0.000000in}}%
\pgfpathlineto{\pgfqpoint{-0.027778in}{0.000000in}}%
\pgfusepath{stroke,fill}%
}%
\begin{pgfscope}%
\pgfsys@transformshift{8.282041in}{0.788111in}%
\pgfsys@useobject{currentmarker}{}%
\end{pgfscope}%
\end{pgfscope}%
\begin{pgfscope}%
\pgfsetbuttcap%
\pgfsetroundjoin%
\definecolor{currentfill}{rgb}{0.000000,0.000000,0.000000}%
\pgfsetfillcolor{currentfill}%
\pgfsetlinewidth{0.602250pt}%
\definecolor{currentstroke}{rgb}{0.000000,0.000000,0.000000}%
\pgfsetstrokecolor{currentstroke}%
\pgfsetdash{}{0pt}%
\pgfsys@defobject{currentmarker}{\pgfqpoint{-0.027778in}{0.000000in}}{\pgfqpoint{0.000000in}{0.000000in}}{%
\pgfpathmoveto{\pgfqpoint{0.000000in}{0.000000in}}%
\pgfpathlineto{\pgfqpoint{-0.027778in}{0.000000in}}%
\pgfusepath{stroke,fill}%
}%
\begin{pgfscope}%
\pgfsys@transformshift{8.282041in}{0.841520in}%
\pgfsys@useobject{currentmarker}{}%
\end{pgfscope}%
\end{pgfscope}%
\begin{pgfscope}%
\pgfsetbuttcap%
\pgfsetroundjoin%
\definecolor{currentfill}{rgb}{0.000000,0.000000,0.000000}%
\pgfsetfillcolor{currentfill}%
\pgfsetlinewidth{0.602250pt}%
\definecolor{currentstroke}{rgb}{0.000000,0.000000,0.000000}%
\pgfsetstrokecolor{currentstroke}%
\pgfsetdash{}{0pt}%
\pgfsys@defobject{currentmarker}{\pgfqpoint{-0.027778in}{0.000000in}}{\pgfqpoint{0.000000in}{0.000000in}}{%
\pgfpathmoveto{\pgfqpoint{0.000000in}{0.000000in}}%
\pgfpathlineto{\pgfqpoint{-0.027778in}{0.000000in}}%
\pgfusepath{stroke,fill}%
}%
\begin{pgfscope}%
\pgfsys@transformshift{8.282041in}{0.885159in}%
\pgfsys@useobject{currentmarker}{}%
\end{pgfscope}%
\end{pgfscope}%
\begin{pgfscope}%
\pgfsetbuttcap%
\pgfsetroundjoin%
\definecolor{currentfill}{rgb}{0.000000,0.000000,0.000000}%
\pgfsetfillcolor{currentfill}%
\pgfsetlinewidth{0.602250pt}%
\definecolor{currentstroke}{rgb}{0.000000,0.000000,0.000000}%
\pgfsetstrokecolor{currentstroke}%
\pgfsetdash{}{0pt}%
\pgfsys@defobject{currentmarker}{\pgfqpoint{-0.027778in}{0.000000in}}{\pgfqpoint{0.000000in}{0.000000in}}{%
\pgfpathmoveto{\pgfqpoint{0.000000in}{0.000000in}}%
\pgfpathlineto{\pgfqpoint{-0.027778in}{0.000000in}}%
\pgfusepath{stroke,fill}%
}%
\begin{pgfscope}%
\pgfsys@transformshift{8.282041in}{0.922055in}%
\pgfsys@useobject{currentmarker}{}%
\end{pgfscope}%
\end{pgfscope}%
\begin{pgfscope}%
\pgfsetbuttcap%
\pgfsetroundjoin%
\definecolor{currentfill}{rgb}{0.000000,0.000000,0.000000}%
\pgfsetfillcolor{currentfill}%
\pgfsetlinewidth{0.602250pt}%
\definecolor{currentstroke}{rgb}{0.000000,0.000000,0.000000}%
\pgfsetstrokecolor{currentstroke}%
\pgfsetdash{}{0pt}%
\pgfsys@defobject{currentmarker}{\pgfqpoint{-0.027778in}{0.000000in}}{\pgfqpoint{0.000000in}{0.000000in}}{%
\pgfpathmoveto{\pgfqpoint{0.000000in}{0.000000in}}%
\pgfpathlineto{\pgfqpoint{-0.027778in}{0.000000in}}%
\pgfusepath{stroke,fill}%
}%
\begin{pgfscope}%
\pgfsys@transformshift{8.282041in}{0.954016in}%
\pgfsys@useobject{currentmarker}{}%
\end{pgfscope}%
\end{pgfscope}%
\begin{pgfscope}%
\pgfsetbuttcap%
\pgfsetroundjoin%
\definecolor{currentfill}{rgb}{0.000000,0.000000,0.000000}%
\pgfsetfillcolor{currentfill}%
\pgfsetlinewidth{0.602250pt}%
\definecolor{currentstroke}{rgb}{0.000000,0.000000,0.000000}%
\pgfsetstrokecolor{currentstroke}%
\pgfsetdash{}{0pt}%
\pgfsys@defobject{currentmarker}{\pgfqpoint{-0.027778in}{0.000000in}}{\pgfqpoint{0.000000in}{0.000000in}}{%
\pgfpathmoveto{\pgfqpoint{0.000000in}{0.000000in}}%
\pgfpathlineto{\pgfqpoint{-0.027778in}{0.000000in}}%
\pgfusepath{stroke,fill}%
}%
\begin{pgfscope}%
\pgfsys@transformshift{8.282041in}{0.982208in}%
\pgfsys@useobject{currentmarker}{}%
\end{pgfscope}%
\end{pgfscope}%
\begin{pgfscope}%
\pgfsetbuttcap%
\pgfsetroundjoin%
\definecolor{currentfill}{rgb}{0.000000,0.000000,0.000000}%
\pgfsetfillcolor{currentfill}%
\pgfsetlinewidth{0.602250pt}%
\definecolor{currentstroke}{rgb}{0.000000,0.000000,0.000000}%
\pgfsetstrokecolor{currentstroke}%
\pgfsetdash{}{0pt}%
\pgfsys@defobject{currentmarker}{\pgfqpoint{-0.027778in}{0.000000in}}{\pgfqpoint{0.000000in}{0.000000in}}{%
\pgfpathmoveto{\pgfqpoint{0.000000in}{0.000000in}}%
\pgfpathlineto{\pgfqpoint{-0.027778in}{0.000000in}}%
\pgfusepath{stroke,fill}%
}%
\begin{pgfscope}%
\pgfsys@transformshift{8.282041in}{1.173331in}%
\pgfsys@useobject{currentmarker}{}%
\end{pgfscope}%
\end{pgfscope}%
\begin{pgfscope}%
\pgfsetbuttcap%
\pgfsetroundjoin%
\definecolor{currentfill}{rgb}{0.000000,0.000000,0.000000}%
\pgfsetfillcolor{currentfill}%
\pgfsetlinewidth{0.602250pt}%
\definecolor{currentstroke}{rgb}{0.000000,0.000000,0.000000}%
\pgfsetstrokecolor{currentstroke}%
\pgfsetdash{}{0pt}%
\pgfsys@defobject{currentmarker}{\pgfqpoint{-0.027778in}{0.000000in}}{\pgfqpoint{0.000000in}{0.000000in}}{%
\pgfpathmoveto{\pgfqpoint{0.000000in}{0.000000in}}%
\pgfpathlineto{\pgfqpoint{-0.027778in}{0.000000in}}%
\pgfusepath{stroke,fill}%
}%
\begin{pgfscope}%
\pgfsys@transformshift{8.282041in}{1.270380in}%
\pgfsys@useobject{currentmarker}{}%
\end{pgfscope}%
\end{pgfscope}%
\begin{pgfscope}%
\pgfsetbuttcap%
\pgfsetroundjoin%
\definecolor{currentfill}{rgb}{0.000000,0.000000,0.000000}%
\pgfsetfillcolor{currentfill}%
\pgfsetlinewidth{0.602250pt}%
\definecolor{currentstroke}{rgb}{0.000000,0.000000,0.000000}%
\pgfsetstrokecolor{currentstroke}%
\pgfsetdash{}{0pt}%
\pgfsys@defobject{currentmarker}{\pgfqpoint{-0.027778in}{0.000000in}}{\pgfqpoint{0.000000in}{0.000000in}}{%
\pgfpathmoveto{\pgfqpoint{0.000000in}{0.000000in}}%
\pgfpathlineto{\pgfqpoint{-0.027778in}{0.000000in}}%
\pgfusepath{stroke,fill}%
}%
\begin{pgfscope}%
\pgfsys@transformshift{8.282041in}{1.339237in}%
\pgfsys@useobject{currentmarker}{}%
\end{pgfscope}%
\end{pgfscope}%
\begin{pgfscope}%
\pgfsetbuttcap%
\pgfsetroundjoin%
\definecolor{currentfill}{rgb}{0.000000,0.000000,0.000000}%
\pgfsetfillcolor{currentfill}%
\pgfsetlinewidth{0.602250pt}%
\definecolor{currentstroke}{rgb}{0.000000,0.000000,0.000000}%
\pgfsetstrokecolor{currentstroke}%
\pgfsetdash{}{0pt}%
\pgfsys@defobject{currentmarker}{\pgfqpoint{-0.027778in}{0.000000in}}{\pgfqpoint{0.000000in}{0.000000in}}{%
\pgfpathmoveto{\pgfqpoint{0.000000in}{0.000000in}}%
\pgfpathlineto{\pgfqpoint{-0.027778in}{0.000000in}}%
\pgfusepath{stroke,fill}%
}%
\begin{pgfscope}%
\pgfsys@transformshift{8.282041in}{1.392646in}%
\pgfsys@useobject{currentmarker}{}%
\end{pgfscope}%
\end{pgfscope}%
\begin{pgfscope}%
\pgfsetbuttcap%
\pgfsetroundjoin%
\definecolor{currentfill}{rgb}{0.000000,0.000000,0.000000}%
\pgfsetfillcolor{currentfill}%
\pgfsetlinewidth{0.602250pt}%
\definecolor{currentstroke}{rgb}{0.000000,0.000000,0.000000}%
\pgfsetstrokecolor{currentstroke}%
\pgfsetdash{}{0pt}%
\pgfsys@defobject{currentmarker}{\pgfqpoint{-0.027778in}{0.000000in}}{\pgfqpoint{0.000000in}{0.000000in}}{%
\pgfpathmoveto{\pgfqpoint{0.000000in}{0.000000in}}%
\pgfpathlineto{\pgfqpoint{-0.027778in}{0.000000in}}%
\pgfusepath{stroke,fill}%
}%
\begin{pgfscope}%
\pgfsys@transformshift{8.282041in}{1.436285in}%
\pgfsys@useobject{currentmarker}{}%
\end{pgfscope}%
\end{pgfscope}%
\begin{pgfscope}%
\pgfsetbuttcap%
\pgfsetroundjoin%
\definecolor{currentfill}{rgb}{0.000000,0.000000,0.000000}%
\pgfsetfillcolor{currentfill}%
\pgfsetlinewidth{0.602250pt}%
\definecolor{currentstroke}{rgb}{0.000000,0.000000,0.000000}%
\pgfsetstrokecolor{currentstroke}%
\pgfsetdash{}{0pt}%
\pgfsys@defobject{currentmarker}{\pgfqpoint{-0.027778in}{0.000000in}}{\pgfqpoint{0.000000in}{0.000000in}}{%
\pgfpathmoveto{\pgfqpoint{0.000000in}{0.000000in}}%
\pgfpathlineto{\pgfqpoint{-0.027778in}{0.000000in}}%
\pgfusepath{stroke,fill}%
}%
\begin{pgfscope}%
\pgfsys@transformshift{8.282041in}{1.473181in}%
\pgfsys@useobject{currentmarker}{}%
\end{pgfscope}%
\end{pgfscope}%
\begin{pgfscope}%
\pgfsetbuttcap%
\pgfsetroundjoin%
\definecolor{currentfill}{rgb}{0.000000,0.000000,0.000000}%
\pgfsetfillcolor{currentfill}%
\pgfsetlinewidth{0.602250pt}%
\definecolor{currentstroke}{rgb}{0.000000,0.000000,0.000000}%
\pgfsetstrokecolor{currentstroke}%
\pgfsetdash{}{0pt}%
\pgfsys@defobject{currentmarker}{\pgfqpoint{-0.027778in}{0.000000in}}{\pgfqpoint{0.000000in}{0.000000in}}{%
\pgfpathmoveto{\pgfqpoint{0.000000in}{0.000000in}}%
\pgfpathlineto{\pgfqpoint{-0.027778in}{0.000000in}}%
\pgfusepath{stroke,fill}%
}%
\begin{pgfscope}%
\pgfsys@transformshift{8.282041in}{1.505142in}%
\pgfsys@useobject{currentmarker}{}%
\end{pgfscope}%
\end{pgfscope}%
\begin{pgfscope}%
\pgfsetbuttcap%
\pgfsetroundjoin%
\definecolor{currentfill}{rgb}{0.000000,0.000000,0.000000}%
\pgfsetfillcolor{currentfill}%
\pgfsetlinewidth{0.602250pt}%
\definecolor{currentstroke}{rgb}{0.000000,0.000000,0.000000}%
\pgfsetstrokecolor{currentstroke}%
\pgfsetdash{}{0pt}%
\pgfsys@defobject{currentmarker}{\pgfqpoint{-0.027778in}{0.000000in}}{\pgfqpoint{0.000000in}{0.000000in}}{%
\pgfpathmoveto{\pgfqpoint{0.000000in}{0.000000in}}%
\pgfpathlineto{\pgfqpoint{-0.027778in}{0.000000in}}%
\pgfusepath{stroke,fill}%
}%
\begin{pgfscope}%
\pgfsys@transformshift{8.282041in}{1.533334in}%
\pgfsys@useobject{currentmarker}{}%
\end{pgfscope}%
\end{pgfscope}%
\begin{pgfscope}%
\pgfsetbuttcap%
\pgfsetroundjoin%
\definecolor{currentfill}{rgb}{0.000000,0.000000,0.000000}%
\pgfsetfillcolor{currentfill}%
\pgfsetlinewidth{0.602250pt}%
\definecolor{currentstroke}{rgb}{0.000000,0.000000,0.000000}%
\pgfsetstrokecolor{currentstroke}%
\pgfsetdash{}{0pt}%
\pgfsys@defobject{currentmarker}{\pgfqpoint{-0.027778in}{0.000000in}}{\pgfqpoint{0.000000in}{0.000000in}}{%
\pgfpathmoveto{\pgfqpoint{0.000000in}{0.000000in}}%
\pgfpathlineto{\pgfqpoint{-0.027778in}{0.000000in}}%
\pgfusepath{stroke,fill}%
}%
\begin{pgfscope}%
\pgfsys@transformshift{8.282041in}{1.724457in}%
\pgfsys@useobject{currentmarker}{}%
\end{pgfscope}%
\end{pgfscope}%
\begin{pgfscope}%
\pgfsetbuttcap%
\pgfsetroundjoin%
\definecolor{currentfill}{rgb}{0.000000,0.000000,0.000000}%
\pgfsetfillcolor{currentfill}%
\pgfsetlinewidth{0.602250pt}%
\definecolor{currentstroke}{rgb}{0.000000,0.000000,0.000000}%
\pgfsetstrokecolor{currentstroke}%
\pgfsetdash{}{0pt}%
\pgfsys@defobject{currentmarker}{\pgfqpoint{-0.027778in}{0.000000in}}{\pgfqpoint{0.000000in}{0.000000in}}{%
\pgfpathmoveto{\pgfqpoint{0.000000in}{0.000000in}}%
\pgfpathlineto{\pgfqpoint{-0.027778in}{0.000000in}}%
\pgfusepath{stroke,fill}%
}%
\begin{pgfscope}%
\pgfsys@transformshift{8.282041in}{1.821506in}%
\pgfsys@useobject{currentmarker}{}%
\end{pgfscope}%
\end{pgfscope}%
\begin{pgfscope}%
\pgfsetbuttcap%
\pgfsetroundjoin%
\definecolor{currentfill}{rgb}{0.000000,0.000000,0.000000}%
\pgfsetfillcolor{currentfill}%
\pgfsetlinewidth{0.602250pt}%
\definecolor{currentstroke}{rgb}{0.000000,0.000000,0.000000}%
\pgfsetstrokecolor{currentstroke}%
\pgfsetdash{}{0pt}%
\pgfsys@defobject{currentmarker}{\pgfqpoint{-0.027778in}{0.000000in}}{\pgfqpoint{0.000000in}{0.000000in}}{%
\pgfpathmoveto{\pgfqpoint{0.000000in}{0.000000in}}%
\pgfpathlineto{\pgfqpoint{-0.027778in}{0.000000in}}%
\pgfusepath{stroke,fill}%
}%
\begin{pgfscope}%
\pgfsys@transformshift{8.282041in}{1.890363in}%
\pgfsys@useobject{currentmarker}{}%
\end{pgfscope}%
\end{pgfscope}%
\begin{pgfscope}%
\pgfsetbuttcap%
\pgfsetroundjoin%
\definecolor{currentfill}{rgb}{0.000000,0.000000,0.000000}%
\pgfsetfillcolor{currentfill}%
\pgfsetlinewidth{0.602250pt}%
\definecolor{currentstroke}{rgb}{0.000000,0.000000,0.000000}%
\pgfsetstrokecolor{currentstroke}%
\pgfsetdash{}{0pt}%
\pgfsys@defobject{currentmarker}{\pgfqpoint{-0.027778in}{0.000000in}}{\pgfqpoint{0.000000in}{0.000000in}}{%
\pgfpathmoveto{\pgfqpoint{0.000000in}{0.000000in}}%
\pgfpathlineto{\pgfqpoint{-0.027778in}{0.000000in}}%
\pgfusepath{stroke,fill}%
}%
\begin{pgfscope}%
\pgfsys@transformshift{8.282041in}{1.943772in}%
\pgfsys@useobject{currentmarker}{}%
\end{pgfscope}%
\end{pgfscope}%
\begin{pgfscope}%
\pgfsetbuttcap%
\pgfsetroundjoin%
\definecolor{currentfill}{rgb}{0.000000,0.000000,0.000000}%
\pgfsetfillcolor{currentfill}%
\pgfsetlinewidth{0.602250pt}%
\definecolor{currentstroke}{rgb}{0.000000,0.000000,0.000000}%
\pgfsetstrokecolor{currentstroke}%
\pgfsetdash{}{0pt}%
\pgfsys@defobject{currentmarker}{\pgfqpoint{-0.027778in}{0.000000in}}{\pgfqpoint{0.000000in}{0.000000in}}{%
\pgfpathmoveto{\pgfqpoint{0.000000in}{0.000000in}}%
\pgfpathlineto{\pgfqpoint{-0.027778in}{0.000000in}}%
\pgfusepath{stroke,fill}%
}%
\begin{pgfscope}%
\pgfsys@transformshift{8.282041in}{1.987411in}%
\pgfsys@useobject{currentmarker}{}%
\end{pgfscope}%
\end{pgfscope}%
\begin{pgfscope}%
\pgfsetbuttcap%
\pgfsetroundjoin%
\definecolor{currentfill}{rgb}{0.000000,0.000000,0.000000}%
\pgfsetfillcolor{currentfill}%
\pgfsetlinewidth{0.602250pt}%
\definecolor{currentstroke}{rgb}{0.000000,0.000000,0.000000}%
\pgfsetstrokecolor{currentstroke}%
\pgfsetdash{}{0pt}%
\pgfsys@defobject{currentmarker}{\pgfqpoint{-0.027778in}{0.000000in}}{\pgfqpoint{0.000000in}{0.000000in}}{%
\pgfpathmoveto{\pgfqpoint{0.000000in}{0.000000in}}%
\pgfpathlineto{\pgfqpoint{-0.027778in}{0.000000in}}%
\pgfusepath{stroke,fill}%
}%
\begin{pgfscope}%
\pgfsys@transformshift{8.282041in}{2.024307in}%
\pgfsys@useobject{currentmarker}{}%
\end{pgfscope}%
\end{pgfscope}%
\begin{pgfscope}%
\pgfsetbuttcap%
\pgfsetroundjoin%
\definecolor{currentfill}{rgb}{0.000000,0.000000,0.000000}%
\pgfsetfillcolor{currentfill}%
\pgfsetlinewidth{0.602250pt}%
\definecolor{currentstroke}{rgb}{0.000000,0.000000,0.000000}%
\pgfsetstrokecolor{currentstroke}%
\pgfsetdash{}{0pt}%
\pgfsys@defobject{currentmarker}{\pgfqpoint{-0.027778in}{0.000000in}}{\pgfqpoint{0.000000in}{0.000000in}}{%
\pgfpathmoveto{\pgfqpoint{0.000000in}{0.000000in}}%
\pgfpathlineto{\pgfqpoint{-0.027778in}{0.000000in}}%
\pgfusepath{stroke,fill}%
}%
\begin{pgfscope}%
\pgfsys@transformshift{8.282041in}{2.056268in}%
\pgfsys@useobject{currentmarker}{}%
\end{pgfscope}%
\end{pgfscope}%
\begin{pgfscope}%
\pgfsetbuttcap%
\pgfsetroundjoin%
\definecolor{currentfill}{rgb}{0.000000,0.000000,0.000000}%
\pgfsetfillcolor{currentfill}%
\pgfsetlinewidth{0.602250pt}%
\definecolor{currentstroke}{rgb}{0.000000,0.000000,0.000000}%
\pgfsetstrokecolor{currentstroke}%
\pgfsetdash{}{0pt}%
\pgfsys@defobject{currentmarker}{\pgfqpoint{-0.027778in}{0.000000in}}{\pgfqpoint{0.000000in}{0.000000in}}{%
\pgfpathmoveto{\pgfqpoint{0.000000in}{0.000000in}}%
\pgfpathlineto{\pgfqpoint{-0.027778in}{0.000000in}}%
\pgfusepath{stroke,fill}%
}%
\begin{pgfscope}%
\pgfsys@transformshift{8.282041in}{2.084460in}%
\pgfsys@useobject{currentmarker}{}%
\end{pgfscope}%
\end{pgfscope}%
\begin{pgfscope}%
\pgfpathrectangle{\pgfqpoint{8.282041in}{0.790446in}}{\pgfqpoint{1.897959in}{1.372727in}}%
\pgfusepath{clip}%
\pgfsetrectcap%
\pgfsetroundjoin%
\pgfsetlinewidth{1.505625pt}%
\definecolor{currentstroke}{rgb}{1.000000,0.000000,0.000000}%
\pgfsetstrokecolor{currentstroke}%
\pgfsetstrokeopacity{0.750000}%
\pgfsetdash{}{0pt}%
\pgfpathmoveto{\pgfqpoint{8.411294in}{2.052880in}}%
\pgfpathlineto{\pgfqpoint{8.421904in}{2.057666in}}%
\pgfpathlineto{\pgfqpoint{8.432058in}{2.062209in}}%
\pgfpathlineto{\pgfqpoint{8.441757in}{2.065973in}}%
\pgfusepath{stroke}%
\end{pgfscope}%
\begin{pgfscope}%
\pgfpathrectangle{\pgfqpoint{8.282041in}{0.790446in}}{\pgfqpoint{1.897959in}{1.372727in}}%
\pgfusepath{clip}%
\pgfsetrectcap%
\pgfsetroundjoin%
\pgfsetlinewidth{1.505625pt}%
\definecolor{currentstroke}{rgb}{0.000000,0.000000,1.000000}%
\pgfsetstrokecolor{currentstroke}%
\pgfsetstrokeopacity{0.750000}%
\pgfsetdash{}{0pt}%
\pgfpathmoveto{\pgfqpoint{8.411294in}{1.128135in}}%
\pgfpathlineto{\pgfqpoint{8.421904in}{1.120198in}}%
\pgfpathlineto{\pgfqpoint{8.432058in}{1.111678in}}%
\pgfpathlineto{\pgfqpoint{8.441757in}{1.101956in}}%
\pgfusepath{stroke}%
\end{pgfscope}%
\begin{pgfscope}%
\pgfpathrectangle{\pgfqpoint{8.282041in}{0.790446in}}{\pgfqpoint{1.897959in}{1.372727in}}%
\pgfusepath{clip}%
\pgfsetrectcap%
\pgfsetroundjoin%
\pgfsetlinewidth{1.505625pt}%
\definecolor{currentstroke}{rgb}{0.000000,0.750000,0.750000}%
\pgfsetstrokecolor{currentstroke}%
\pgfsetstrokeopacity{0.750000}%
\pgfsetdash{}{0pt}%
\pgfpathmoveto{\pgfqpoint{8.411294in}{1.720466in}}%
\pgfpathlineto{\pgfqpoint{8.421904in}{1.697019in}}%
\pgfpathlineto{\pgfqpoint{8.432058in}{1.676502in}}%
\pgfpathlineto{\pgfqpoint{8.441757in}{1.657900in}}%
\pgfusepath{stroke}%
\end{pgfscope}%
\begin{pgfscope}%
\pgfpathrectangle{\pgfqpoint{8.282041in}{0.790446in}}{\pgfqpoint{1.897959in}{1.372727in}}%
\pgfusepath{clip}%
\pgfsetrectcap%
\pgfsetroundjoin%
\pgfsetlinewidth{1.505625pt}%
\definecolor{currentstroke}{rgb}{1.000000,0.000000,0.000000}%
\pgfsetstrokecolor{currentstroke}%
\pgfsetstrokeopacity{0.750000}%
\pgfsetdash{}{0pt}%
\pgfpathmoveto{\pgfqpoint{8.400709in}{2.073914in}}%
\pgfpathlineto{\pgfqpoint{8.412792in}{2.077111in}}%
\pgfpathlineto{\pgfqpoint{8.424381in}{2.080143in}}%
\pgfpathlineto{\pgfqpoint{8.435482in}{2.082627in}}%
\pgfpathlineto{\pgfqpoint{8.446099in}{2.084706in}}%
\pgfpathlineto{\pgfqpoint{8.456237in}{2.086476in}}%
\pgfpathlineto{\pgfqpoint{8.465901in}{2.088008in}}%
\pgfpathlineto{\pgfqpoint{8.475095in}{2.089351in}}%
\pgfusepath{stroke}%
\end{pgfscope}%
\begin{pgfscope}%
\pgfpathrectangle{\pgfqpoint{8.282041in}{0.790446in}}{\pgfqpoint{1.897959in}{1.372727in}}%
\pgfusepath{clip}%
\pgfsetrectcap%
\pgfsetroundjoin%
\pgfsetlinewidth{1.505625pt}%
\definecolor{currentstroke}{rgb}{0.000000,0.000000,1.000000}%
\pgfsetstrokecolor{currentstroke}%
\pgfsetstrokeopacity{0.750000}%
\pgfsetdash{}{0pt}%
\pgfpathmoveto{\pgfqpoint{8.400709in}{1.302243in}}%
\pgfpathlineto{\pgfqpoint{8.412792in}{1.295197in}}%
\pgfpathlineto{\pgfqpoint{8.424381in}{1.287687in}}%
\pgfpathlineto{\pgfqpoint{8.435482in}{1.279273in}}%
\pgfpathlineto{\pgfqpoint{8.446099in}{1.270045in}}%
\pgfpathlineto{\pgfqpoint{8.456237in}{1.260049in}}%
\pgfpathlineto{\pgfqpoint{8.465901in}{1.249301in}}%
\pgfpathlineto{\pgfqpoint{8.475095in}{1.237798in}}%
\pgfusepath{stroke}%
\end{pgfscope}%
\begin{pgfscope}%
\pgfpathrectangle{\pgfqpoint{8.282041in}{0.790446in}}{\pgfqpoint{1.897959in}{1.372727in}}%
\pgfusepath{clip}%
\pgfsetrectcap%
\pgfsetroundjoin%
\pgfsetlinewidth{1.505625pt}%
\definecolor{currentstroke}{rgb}{0.000000,0.750000,0.750000}%
\pgfsetstrokecolor{currentstroke}%
\pgfsetstrokeopacity{0.750000}%
\pgfsetdash{}{0pt}%
\pgfpathmoveto{\pgfqpoint{8.400709in}{1.572577in}}%
\pgfpathlineto{\pgfqpoint{8.412792in}{1.541753in}}%
\pgfpathlineto{\pgfqpoint{8.424381in}{1.514786in}}%
\pgfpathlineto{\pgfqpoint{8.435482in}{1.490625in}}%
\pgfpathlineto{\pgfqpoint{8.446099in}{1.468897in}}%
\pgfpathlineto{\pgfqpoint{8.456237in}{1.449295in}}%
\pgfpathlineto{\pgfqpoint{8.465901in}{1.431566in}}%
\pgfpathlineto{\pgfqpoint{8.475095in}{1.415499in}}%
\pgfusepath{stroke}%
\end{pgfscope}%
\begin{pgfscope}%
\pgfpathrectangle{\pgfqpoint{8.282041in}{0.790446in}}{\pgfqpoint{1.897959in}{1.372727in}}%
\pgfusepath{clip}%
\pgfsetrectcap%
\pgfsetroundjoin%
\pgfsetlinewidth{1.505625pt}%
\definecolor{currentstroke}{rgb}{1.000000,0.000000,0.000000}%
\pgfsetstrokecolor{currentstroke}%
\pgfsetstrokeopacity{0.750000}%
\pgfsetdash{}{0pt}%
\pgfpathmoveto{\pgfqpoint{8.431628in}{2.047988in}}%
\pgfpathlineto{\pgfqpoint{8.446301in}{2.054002in}}%
\pgfpathlineto{\pgfqpoint{8.460331in}{2.059692in}}%
\pgfpathlineto{\pgfqpoint{8.473725in}{2.064271in}}%
\pgfpathlineto{\pgfqpoint{8.486490in}{2.068020in}}%
\pgfpathlineto{\pgfqpoint{8.498633in}{2.071135in}}%
\pgfpathlineto{\pgfqpoint{8.510160in}{2.073755in}}%
\pgfusepath{stroke}%
\end{pgfscope}%
\begin{pgfscope}%
\pgfpathrectangle{\pgfqpoint{8.282041in}{0.790446in}}{\pgfqpoint{1.897959in}{1.372727in}}%
\pgfusepath{clip}%
\pgfsetrectcap%
\pgfsetroundjoin%
\pgfsetlinewidth{1.505625pt}%
\definecolor{currentstroke}{rgb}{0.000000,0.000000,1.000000}%
\pgfsetstrokecolor{currentstroke}%
\pgfsetstrokeopacity{0.750000}%
\pgfsetdash{}{0pt}%
\pgfpathmoveto{\pgfqpoint{8.431628in}{1.048023in}}%
\pgfpathlineto{\pgfqpoint{8.446301in}{1.041717in}}%
\pgfpathlineto{\pgfqpoint{8.460331in}{1.034878in}}%
\pgfpathlineto{\pgfqpoint{8.473725in}{1.026611in}}%
\pgfpathlineto{\pgfqpoint{8.486490in}{1.017108in}}%
\pgfpathlineto{\pgfqpoint{8.498633in}{1.006481in}}%
\pgfpathlineto{\pgfqpoint{8.510160in}{0.994792in}}%
\pgfusepath{stroke}%
\end{pgfscope}%
\begin{pgfscope}%
\pgfpathrectangle{\pgfqpoint{8.282041in}{0.790446in}}{\pgfqpoint{1.897959in}{1.372727in}}%
\pgfusepath{clip}%
\pgfsetrectcap%
\pgfsetroundjoin%
\pgfsetlinewidth{1.505625pt}%
\definecolor{currentstroke}{rgb}{0.000000,0.750000,0.750000}%
\pgfsetstrokecolor{currentstroke}%
\pgfsetstrokeopacity{0.750000}%
\pgfsetdash{}{0pt}%
\pgfpathmoveto{\pgfqpoint{8.431628in}{1.745525in}}%
\pgfpathlineto{\pgfqpoint{8.446301in}{1.717843in}}%
\pgfpathlineto{\pgfqpoint{8.460331in}{1.694082in}}%
\pgfpathlineto{\pgfqpoint{8.473725in}{1.672706in}}%
\pgfpathlineto{\pgfqpoint{8.486490in}{1.653433in}}%
\pgfpathlineto{\pgfqpoint{8.498633in}{1.636019in}}%
\pgfpathlineto{\pgfqpoint{8.510160in}{1.620259in}}%
\pgfusepath{stroke}%
\end{pgfscope}%
\begin{pgfscope}%
\pgfpathrectangle{\pgfqpoint{8.282041in}{0.790446in}}{\pgfqpoint{1.897959in}{1.372727in}}%
\pgfusepath{clip}%
\pgfsetrectcap%
\pgfsetroundjoin%
\pgfsetlinewidth{1.505625pt}%
\definecolor{currentstroke}{rgb}{1.000000,0.000000,0.000000}%
\pgfsetstrokecolor{currentstroke}%
\pgfsetstrokeopacity{0.750000}%
\pgfsetdash{}{0pt}%
\pgfpathmoveto{\pgfqpoint{8.368312in}{2.041640in}}%
\pgfpathlineto{\pgfqpoint{8.380745in}{2.048401in}}%
\pgfpathlineto{\pgfqpoint{8.393002in}{2.054616in}}%
\pgfpathlineto{\pgfqpoint{8.405081in}{2.059459in}}%
\pgfpathlineto{\pgfqpoint{8.416983in}{2.063327in}}%
\pgfpathlineto{\pgfqpoint{8.428704in}{2.066477in}}%
\pgfpathlineto{\pgfqpoint{8.440245in}{2.069092in}}%
\pgfpathlineto{\pgfqpoint{8.451604in}{2.071296in}}%
\pgfpathlineto{\pgfqpoint{8.462779in}{2.073181in}}%
\pgfpathlineto{\pgfqpoint{8.473770in}{2.074814in}}%
\pgfpathlineto{\pgfqpoint{8.484575in}{2.076245in}}%
\pgfpathlineto{\pgfqpoint{8.495193in}{2.077513in}}%
\pgfpathlineto{\pgfqpoint{8.505623in}{2.078648in}}%
\pgfpathlineto{\pgfqpoint{8.515859in}{2.079671in}}%
\pgfpathlineto{\pgfqpoint{8.525902in}{2.080603in}}%
\pgfpathlineto{\pgfqpoint{8.535751in}{2.081457in}}%
\pgfpathlineto{\pgfqpoint{8.545405in}{2.082245in}}%
\pgfpathlineto{\pgfqpoint{8.554865in}{2.082979in}}%
\pgfpathlineto{\pgfqpoint{8.564130in}{2.083664in}}%
\pgfpathlineto{\pgfqpoint{8.573202in}{2.084309in}}%
\pgfpathlineto{\pgfqpoint{8.582083in}{2.084918in}}%
\pgfpathlineto{\pgfqpoint{8.590772in}{2.085496in}}%
\pgfpathlineto{\pgfqpoint{8.599274in}{2.086048in}}%
\pgfpathlineto{\pgfqpoint{8.607590in}{2.086576in}}%
\pgfpathlineto{\pgfqpoint{8.615724in}{2.087083in}}%
\pgfpathlineto{\pgfqpoint{8.623677in}{2.087572in}}%
\pgfpathlineto{\pgfqpoint{8.631452in}{2.088045in}}%
\pgfpathlineto{\pgfqpoint{8.639053in}{2.088504in}}%
\pgfpathlineto{\pgfqpoint{8.646481in}{2.088950in}}%
\pgfpathlineto{\pgfqpoint{8.653738in}{2.089385in}}%
\pgfpathlineto{\pgfqpoint{8.660828in}{2.089809in}}%
\pgfpathlineto{\pgfqpoint{8.667751in}{2.090224in}}%
\pgfpathlineto{\pgfqpoint{8.674509in}{2.090631in}}%
\pgfpathlineto{\pgfqpoint{8.681104in}{2.091031in}}%
\pgfpathlineto{\pgfqpoint{8.687537in}{2.091423in}}%
\pgfpathlineto{\pgfqpoint{8.693809in}{2.091810in}}%
\pgfpathlineto{\pgfqpoint{8.699921in}{2.092191in}}%
\pgfpathlineto{\pgfqpoint{8.705874in}{2.092566in}}%
\pgfpathlineto{\pgfqpoint{8.711670in}{2.092937in}}%
\pgfpathlineto{\pgfqpoint{8.717309in}{2.093304in}}%
\pgfpathlineto{\pgfqpoint{8.722792in}{2.093666in}}%
\pgfpathlineto{\pgfqpoint{8.728120in}{2.094024in}}%
\pgfpathlineto{\pgfqpoint{8.733295in}{2.094379in}}%
\pgfpathlineto{\pgfqpoint{8.738318in}{2.094730in}}%
\pgfpathlineto{\pgfqpoint{8.743189in}{2.095078in}}%
\pgfpathlineto{\pgfqpoint{8.747910in}{2.095423in}}%
\pgfpathlineto{\pgfqpoint{8.752483in}{2.095765in}}%
\pgfpathlineto{\pgfqpoint{8.756907in}{2.096103in}}%
\pgfpathlineto{\pgfqpoint{8.761185in}{2.096439in}}%
\pgfpathlineto{\pgfqpoint{8.765317in}{2.096772in}}%
\pgfpathlineto{\pgfqpoint{8.769304in}{2.097103in}}%
\pgfpathlineto{\pgfqpoint{8.773147in}{2.097430in}}%
\pgfpathlineto{\pgfqpoint{8.776848in}{2.097755in}}%
\pgfpathlineto{\pgfqpoint{8.780406in}{2.098077in}}%
\pgfpathlineto{\pgfqpoint{8.783822in}{2.098397in}}%
\pgfpathlineto{\pgfqpoint{8.787097in}{2.098714in}}%
\pgfpathlineto{\pgfqpoint{8.790232in}{2.099028in}}%
\pgfpathlineto{\pgfqpoint{8.793227in}{2.099338in}}%
\pgfpathlineto{\pgfqpoint{8.796082in}{2.099646in}}%
\pgfusepath{stroke}%
\end{pgfscope}%
\begin{pgfscope}%
\pgfpathrectangle{\pgfqpoint{8.282041in}{0.790446in}}{\pgfqpoint{1.897959in}{1.372727in}}%
\pgfusepath{clip}%
\pgfsetrectcap%
\pgfsetroundjoin%
\pgfsetlinewidth{1.505625pt}%
\definecolor{currentstroke}{rgb}{0.000000,0.000000,1.000000}%
\pgfsetstrokecolor{currentstroke}%
\pgfsetstrokeopacity{0.750000}%
\pgfsetdash{}{0pt}%
\pgfpathmoveto{\pgfqpoint{8.368312in}{1.470268in}}%
\pgfpathlineto{\pgfqpoint{8.380745in}{1.473781in}}%
\pgfpathlineto{\pgfqpoint{8.393002in}{1.476863in}}%
\pgfpathlineto{\pgfqpoint{8.405081in}{1.478644in}}%
\pgfpathlineto{\pgfqpoint{8.416983in}{1.479489in}}%
\pgfpathlineto{\pgfqpoint{8.428704in}{1.479633in}}%
\pgfpathlineto{\pgfqpoint{8.440245in}{1.479241in}}%
\pgfpathlineto{\pgfqpoint{8.451604in}{1.478425in}}%
\pgfpathlineto{\pgfqpoint{8.462779in}{1.477266in}}%
\pgfpathlineto{\pgfqpoint{8.473770in}{1.475822in}}%
\pgfpathlineto{\pgfqpoint{8.484575in}{1.474135in}}%
\pgfpathlineto{\pgfqpoint{8.495193in}{1.472239in}}%
\pgfpathlineto{\pgfqpoint{8.505623in}{1.470156in}}%
\pgfpathlineto{\pgfqpoint{8.515859in}{1.467906in}}%
\pgfpathlineto{\pgfqpoint{8.525902in}{1.465501in}}%
\pgfpathlineto{\pgfqpoint{8.535751in}{1.462952in}}%
\pgfpathlineto{\pgfqpoint{8.545405in}{1.460268in}}%
\pgfpathlineto{\pgfqpoint{8.554865in}{1.457455in}}%
\pgfpathlineto{\pgfqpoint{8.564130in}{1.454515in}}%
\pgfpathlineto{\pgfqpoint{8.573202in}{1.451454in}}%
\pgfpathlineto{\pgfqpoint{8.582083in}{1.448272in}}%
\pgfpathlineto{\pgfqpoint{8.590772in}{1.444971in}}%
\pgfpathlineto{\pgfqpoint{8.599274in}{1.441552in}}%
\pgfpathlineto{\pgfqpoint{8.607590in}{1.438013in}}%
\pgfpathlineto{\pgfqpoint{8.615724in}{1.434356in}}%
\pgfpathlineto{\pgfqpoint{8.623677in}{1.430578in}}%
\pgfpathlineto{\pgfqpoint{8.631452in}{1.426678in}}%
\pgfpathlineto{\pgfqpoint{8.639053in}{1.422654in}}%
\pgfpathlineto{\pgfqpoint{8.646481in}{1.418504in}}%
\pgfpathlineto{\pgfqpoint{8.653738in}{1.414225in}}%
\pgfpathlineto{\pgfqpoint{8.660828in}{1.409813in}}%
\pgfpathlineto{\pgfqpoint{8.667751in}{1.405267in}}%
\pgfpathlineto{\pgfqpoint{8.674509in}{1.400582in}}%
\pgfpathlineto{\pgfqpoint{8.681104in}{1.395754in}}%
\pgfpathlineto{\pgfqpoint{8.687537in}{1.390778in}}%
\pgfpathlineto{\pgfqpoint{8.693809in}{1.385650in}}%
\pgfpathlineto{\pgfqpoint{8.699921in}{1.380366in}}%
\pgfpathlineto{\pgfqpoint{8.705874in}{1.374919in}}%
\pgfpathlineto{\pgfqpoint{8.711670in}{1.369303in}}%
\pgfpathlineto{\pgfqpoint{8.717309in}{1.363512in}}%
\pgfpathlineto{\pgfqpoint{8.722792in}{1.357540in}}%
\pgfpathlineto{\pgfqpoint{8.728120in}{1.351378in}}%
\pgfpathlineto{\pgfqpoint{8.733295in}{1.345018in}}%
\pgfpathlineto{\pgfqpoint{8.738318in}{1.338452in}}%
\pgfpathlineto{\pgfqpoint{8.743189in}{1.331671in}}%
\pgfpathlineto{\pgfqpoint{8.747910in}{1.324663in}}%
\pgfpathlineto{\pgfqpoint{8.752483in}{1.317418in}}%
\pgfpathlineto{\pgfqpoint{8.756907in}{1.309923in}}%
\pgfpathlineto{\pgfqpoint{8.761185in}{1.302166in}}%
\pgfpathlineto{\pgfqpoint{8.765317in}{1.294131in}}%
\pgfpathlineto{\pgfqpoint{8.769304in}{1.285802in}}%
\pgfpathlineto{\pgfqpoint{8.773147in}{1.277163in}}%
\pgfpathlineto{\pgfqpoint{8.776848in}{1.268193in}}%
\pgfpathlineto{\pgfqpoint{8.780406in}{1.258871in}}%
\pgfpathlineto{\pgfqpoint{8.783822in}{1.249173in}}%
\pgfpathlineto{\pgfqpoint{8.787097in}{1.239073in}}%
\pgfpathlineto{\pgfqpoint{8.790232in}{1.228541in}}%
\pgfpathlineto{\pgfqpoint{8.793227in}{1.217542in}}%
\pgfpathlineto{\pgfqpoint{8.796082in}{1.206039in}}%
\pgfusepath{stroke}%
\end{pgfscope}%
\begin{pgfscope}%
\pgfpathrectangle{\pgfqpoint{8.282041in}{0.790446in}}{\pgfqpoint{1.897959in}{1.372727in}}%
\pgfusepath{clip}%
\pgfsetrectcap%
\pgfsetroundjoin%
\pgfsetlinewidth{1.505625pt}%
\definecolor{currentstroke}{rgb}{0.000000,0.750000,0.750000}%
\pgfsetstrokecolor{currentstroke}%
\pgfsetstrokeopacity{0.750000}%
\pgfsetdash{}{0pt}%
\pgfpathmoveto{\pgfqpoint{8.368312in}{1.701666in}}%
\pgfpathlineto{\pgfqpoint{8.380745in}{1.661949in}}%
\pgfpathlineto{\pgfqpoint{8.393002in}{1.627256in}}%
\pgfpathlineto{\pgfqpoint{8.405081in}{1.595697in}}%
\pgfpathlineto{\pgfqpoint{8.416983in}{1.566880in}}%
\pgfpathlineto{\pgfqpoint{8.428704in}{1.540468in}}%
\pgfpathlineto{\pgfqpoint{8.440245in}{1.516173in}}%
\pgfpathlineto{\pgfqpoint{8.451604in}{1.493755in}}%
\pgfpathlineto{\pgfqpoint{8.462779in}{1.473004in}}%
\pgfpathlineto{\pgfqpoint{8.473770in}{1.453744in}}%
\pgfpathlineto{\pgfqpoint{8.484575in}{1.435821in}}%
\pgfpathlineto{\pgfqpoint{8.495193in}{1.419106in}}%
\pgfpathlineto{\pgfqpoint{8.505623in}{1.403484in}}%
\pgfpathlineto{\pgfqpoint{8.515859in}{1.388854in}}%
\pgfpathlineto{\pgfqpoint{8.525902in}{1.375129in}}%
\pgfpathlineto{\pgfqpoint{8.535751in}{1.362232in}}%
\pgfpathlineto{\pgfqpoint{8.545405in}{1.350095in}}%
\pgfpathlineto{\pgfqpoint{8.554865in}{1.338659in}}%
\pgfpathlineto{\pgfqpoint{8.564130in}{1.327868in}}%
\pgfpathlineto{\pgfqpoint{8.573202in}{1.317674in}}%
\pgfpathlineto{\pgfqpoint{8.582083in}{1.308035in}}%
\pgfpathlineto{\pgfqpoint{8.590772in}{1.298910in}}%
\pgfpathlineto{\pgfqpoint{8.599274in}{1.290264in}}%
\pgfpathlineto{\pgfqpoint{8.607590in}{1.282066in}}%
\pgfpathlineto{\pgfqpoint{8.615724in}{1.274287in}}%
\pgfpathlineto{\pgfqpoint{8.623677in}{1.266899in}}%
\pgfpathlineto{\pgfqpoint{8.631452in}{1.259879in}}%
\pgfpathlineto{\pgfqpoint{8.639053in}{1.253205in}}%
\pgfpathlineto{\pgfqpoint{8.646481in}{1.246856in}}%
\pgfpathlineto{\pgfqpoint{8.653738in}{1.240814in}}%
\pgfpathlineto{\pgfqpoint{8.660828in}{1.235061in}}%
\pgfpathlineto{\pgfqpoint{8.667751in}{1.229581in}}%
\pgfpathlineto{\pgfqpoint{8.674509in}{1.224361in}}%
\pgfpathlineto{\pgfqpoint{8.681104in}{1.219385in}}%
\pgfpathlineto{\pgfqpoint{8.687537in}{1.214642in}}%
\pgfpathlineto{\pgfqpoint{8.693809in}{1.210120in}}%
\pgfpathlineto{\pgfqpoint{8.699921in}{1.205809in}}%
\pgfpathlineto{\pgfqpoint{8.705874in}{1.201698in}}%
\pgfpathlineto{\pgfqpoint{8.711670in}{1.197777in}}%
\pgfpathlineto{\pgfqpoint{8.717309in}{1.194039in}}%
\pgfpathlineto{\pgfqpoint{8.722792in}{1.190474in}}%
\pgfpathlineto{\pgfqpoint{8.728120in}{1.187075in}}%
\pgfpathlineto{\pgfqpoint{8.733295in}{1.183835in}}%
\pgfpathlineto{\pgfqpoint{8.738318in}{1.180748in}}%
\pgfpathlineto{\pgfqpoint{8.743189in}{1.177806in}}%
\pgfpathlineto{\pgfqpoint{8.747910in}{1.175005in}}%
\pgfpathlineto{\pgfqpoint{8.752483in}{1.172338in}}%
\pgfpathlineto{\pgfqpoint{8.756907in}{1.169800in}}%
\pgfpathlineto{\pgfqpoint{8.761185in}{1.167388in}}%
\pgfpathlineto{\pgfqpoint{8.765317in}{1.165095in}}%
\pgfpathlineto{\pgfqpoint{8.769304in}{1.162917in}}%
\pgfpathlineto{\pgfqpoint{8.773147in}{1.160852in}}%
\pgfpathlineto{\pgfqpoint{8.776848in}{1.158894in}}%
\pgfpathlineto{\pgfqpoint{8.780406in}{1.157040in}}%
\pgfpathlineto{\pgfqpoint{8.783822in}{1.155287in}}%
\pgfpathlineto{\pgfqpoint{8.787097in}{1.153632in}}%
\pgfpathlineto{\pgfqpoint{8.790232in}{1.152072in}}%
\pgfpathlineto{\pgfqpoint{8.793227in}{1.150603in}}%
\pgfpathlineto{\pgfqpoint{8.796082in}{1.149223in}}%
\pgfusepath{stroke}%
\end{pgfscope}%
\begin{pgfscope}%
\pgfpathrectangle{\pgfqpoint{8.282041in}{0.790446in}}{\pgfqpoint{1.897959in}{1.372727in}}%
\pgfusepath{clip}%
\pgfsetrectcap%
\pgfsetroundjoin%
\pgfsetlinewidth{1.505625pt}%
\definecolor{currentstroke}{rgb}{1.000000,0.000000,0.000000}%
\pgfsetstrokecolor{currentstroke}%
\pgfsetstrokeopacity{0.750000}%
\pgfsetdash{}{0pt}%
\pgfpathmoveto{\pgfqpoint{8.396039in}{2.024213in}}%
\pgfpathlineto{\pgfqpoint{8.409842in}{2.033072in}}%
\pgfpathlineto{\pgfqpoint{8.423392in}{2.041374in}}%
\pgfpathlineto{\pgfqpoint{8.436693in}{2.048018in}}%
\pgfpathlineto{\pgfqpoint{8.449746in}{2.053430in}}%
\pgfpathlineto{\pgfqpoint{8.462553in}{2.057908in}}%
\pgfpathlineto{\pgfqpoint{8.475117in}{2.061665in}}%
\pgfpathlineto{\pgfqpoint{8.487440in}{2.064854in}}%
\pgfpathlineto{\pgfqpoint{8.499523in}{2.067592in}}%
\pgfpathlineto{\pgfqpoint{8.511370in}{2.069966in}}%
\pgfpathlineto{\pgfqpoint{8.522982in}{2.072043in}}%
\pgfpathlineto{\pgfqpoint{8.534362in}{2.073875in}}%
\pgfpathlineto{\pgfqpoint{8.545511in}{2.075502in}}%
\pgfpathlineto{\pgfqpoint{8.556433in}{2.076958in}}%
\pgfpathlineto{\pgfqpoint{8.567130in}{2.078270in}}%
\pgfpathlineto{\pgfqpoint{8.577604in}{2.079458in}}%
\pgfpathlineto{\pgfqpoint{8.587857in}{2.080540in}}%
\pgfpathlineto{\pgfqpoint{8.597890in}{2.081531in}}%
\pgfpathlineto{\pgfqpoint{8.607705in}{2.082443in}}%
\pgfpathlineto{\pgfqpoint{8.617304in}{2.083285in}}%
\pgfpathlineto{\pgfqpoint{8.626689in}{2.084068in}}%
\pgfpathlineto{\pgfqpoint{8.635861in}{2.084796in}}%
\pgfpathlineto{\pgfqpoint{8.644822in}{2.085478in}}%
\pgfpathlineto{\pgfqpoint{8.653573in}{2.086118in}}%
\pgfpathlineto{\pgfqpoint{8.662116in}{2.086721in}}%
\pgfpathlineto{\pgfqpoint{8.670453in}{2.087290in}}%
\pgfpathlineto{\pgfqpoint{8.678585in}{2.087830in}}%
\pgfpathlineto{\pgfqpoint{8.686516in}{2.088342in}}%
\pgfpathlineto{\pgfqpoint{8.694246in}{2.088831in}}%
\pgfpathlineto{\pgfqpoint{8.701779in}{2.089297in}}%
\pgfpathlineto{\pgfqpoint{8.709117in}{2.089743in}}%
\pgfpathlineto{\pgfqpoint{8.716263in}{2.090172in}}%
\pgfpathlineto{\pgfqpoint{8.723218in}{2.090583in}}%
\pgfpathlineto{\pgfqpoint{8.729985in}{2.090979in}}%
\pgfpathlineto{\pgfqpoint{8.736568in}{2.091361in}}%
\pgfpathlineto{\pgfqpoint{8.742967in}{2.091730in}}%
\pgfpathlineto{\pgfqpoint{8.749185in}{2.092087in}}%
\pgfpathlineto{\pgfqpoint{8.755225in}{2.092433in}}%
\pgfpathlineto{\pgfqpoint{8.761088in}{2.092768in}}%
\pgfpathlineto{\pgfqpoint{8.766775in}{2.093094in}}%
\pgfpathlineto{\pgfqpoint{8.772290in}{2.093410in}}%
\pgfpathlineto{\pgfqpoint{8.777632in}{2.093718in}}%
\pgfpathlineto{\pgfqpoint{8.782803in}{2.094018in}}%
\pgfpathlineto{\pgfqpoint{8.787805in}{2.094311in}}%
\pgfpathlineto{\pgfqpoint{8.792638in}{2.094596in}}%
\pgfpathlineto{\pgfqpoint{8.797303in}{2.094874in}}%
\pgfpathlineto{\pgfqpoint{8.801802in}{2.095146in}}%
\pgfpathlineto{\pgfqpoint{8.806134in}{2.095411in}}%
\pgfpathlineto{\pgfqpoint{8.810300in}{2.095671in}}%
\pgfpathlineto{\pgfqpoint{8.814302in}{2.095924in}}%
\pgfpathlineto{\pgfqpoint{8.818140in}{2.096172in}}%
\pgfpathlineto{\pgfqpoint{8.821815in}{2.096415in}}%
\pgfpathlineto{\pgfqpoint{8.825328in}{2.096652in}}%
\pgfpathlineto{\pgfqpoint{8.828679in}{2.096884in}}%
\pgfusepath{stroke}%
\end{pgfscope}%
\begin{pgfscope}%
\pgfpathrectangle{\pgfqpoint{8.282041in}{0.790446in}}{\pgfqpoint{1.897959in}{1.372727in}}%
\pgfusepath{clip}%
\pgfsetrectcap%
\pgfsetroundjoin%
\pgfsetlinewidth{1.505625pt}%
\definecolor{currentstroke}{rgb}{0.000000,0.000000,1.000000}%
\pgfsetstrokecolor{currentstroke}%
\pgfsetstrokeopacity{0.750000}%
\pgfsetdash{}{0pt}%
\pgfpathmoveto{\pgfqpoint{8.396039in}{1.314994in}}%
\pgfpathlineto{\pgfqpoint{8.409842in}{1.319833in}}%
\pgfpathlineto{\pgfqpoint{8.423392in}{1.324308in}}%
\pgfpathlineto{\pgfqpoint{8.436693in}{1.327254in}}%
\pgfpathlineto{\pgfqpoint{8.449746in}{1.329052in}}%
\pgfpathlineto{\pgfqpoint{8.462553in}{1.329968in}}%
\pgfpathlineto{\pgfqpoint{8.475117in}{1.330187in}}%
\pgfpathlineto{\pgfqpoint{8.487440in}{1.329846in}}%
\pgfpathlineto{\pgfqpoint{8.499523in}{1.329043in}}%
\pgfpathlineto{\pgfqpoint{8.511370in}{1.327853in}}%
\pgfpathlineto{\pgfqpoint{8.522982in}{1.326332in}}%
\pgfpathlineto{\pgfqpoint{8.534362in}{1.324521in}}%
\pgfpathlineto{\pgfqpoint{8.545511in}{1.322454in}}%
\pgfpathlineto{\pgfqpoint{8.556433in}{1.320156in}}%
\pgfpathlineto{\pgfqpoint{8.567130in}{1.317647in}}%
\pgfpathlineto{\pgfqpoint{8.577604in}{1.314942in}}%
\pgfpathlineto{\pgfqpoint{8.587857in}{1.312053in}}%
\pgfpathlineto{\pgfqpoint{8.597890in}{1.308990in}}%
\pgfpathlineto{\pgfqpoint{8.607705in}{1.305759in}}%
\pgfpathlineto{\pgfqpoint{8.617304in}{1.302365in}}%
\pgfpathlineto{\pgfqpoint{8.626689in}{1.298812in}}%
\pgfpathlineto{\pgfqpoint{8.635861in}{1.295102in}}%
\pgfpathlineto{\pgfqpoint{8.644822in}{1.291238in}}%
\pgfpathlineto{\pgfqpoint{8.653573in}{1.287219in}}%
\pgfpathlineto{\pgfqpoint{8.662116in}{1.283045in}}%
\pgfpathlineto{\pgfqpoint{8.670453in}{1.278715in}}%
\pgfpathlineto{\pgfqpoint{8.678585in}{1.274229in}}%
\pgfpathlineto{\pgfqpoint{8.686516in}{1.269582in}}%
\pgfpathlineto{\pgfqpoint{8.694246in}{1.264774in}}%
\pgfpathlineto{\pgfqpoint{8.701779in}{1.259800in}}%
\pgfpathlineto{\pgfqpoint{8.709117in}{1.254656in}}%
\pgfpathlineto{\pgfqpoint{8.716263in}{1.249339in}}%
\pgfpathlineto{\pgfqpoint{8.723218in}{1.243844in}}%
\pgfpathlineto{\pgfqpoint{8.729985in}{1.238164in}}%
\pgfpathlineto{\pgfqpoint{8.736568in}{1.232295in}}%
\pgfpathlineto{\pgfqpoint{8.742967in}{1.226229in}}%
\pgfpathlineto{\pgfqpoint{8.749185in}{1.219959in}}%
\pgfpathlineto{\pgfqpoint{8.755225in}{1.213478in}}%
\pgfpathlineto{\pgfqpoint{8.761088in}{1.206777in}}%
\pgfpathlineto{\pgfqpoint{8.766775in}{1.199846in}}%
\pgfpathlineto{\pgfqpoint{8.772290in}{1.192676in}}%
\pgfpathlineto{\pgfqpoint{8.777632in}{1.185254in}}%
\pgfpathlineto{\pgfqpoint{8.782803in}{1.177570in}}%
\pgfpathlineto{\pgfqpoint{8.787805in}{1.169609in}}%
\pgfpathlineto{\pgfqpoint{8.792638in}{1.161357in}}%
\pgfpathlineto{\pgfqpoint{8.797303in}{1.152797in}}%
\pgfpathlineto{\pgfqpoint{8.801802in}{1.143911in}}%
\pgfpathlineto{\pgfqpoint{8.806134in}{1.134680in}}%
\pgfpathlineto{\pgfqpoint{8.810300in}{1.125082in}}%
\pgfpathlineto{\pgfqpoint{8.814302in}{1.115093in}}%
\pgfpathlineto{\pgfqpoint{8.818140in}{1.104685in}}%
\pgfpathlineto{\pgfqpoint{8.821815in}{1.093827in}}%
\pgfpathlineto{\pgfqpoint{8.825328in}{1.082486in}}%
\pgfpathlineto{\pgfqpoint{8.828679in}{1.070623in}}%
\pgfusepath{stroke}%
\end{pgfscope}%
\begin{pgfscope}%
\pgfpathrectangle{\pgfqpoint{8.282041in}{0.790446in}}{\pgfqpoint{1.897959in}{1.372727in}}%
\pgfusepath{clip}%
\pgfsetrectcap%
\pgfsetroundjoin%
\pgfsetlinewidth{1.505625pt}%
\definecolor{currentstroke}{rgb}{0.000000,0.750000,0.750000}%
\pgfsetstrokecolor{currentstroke}%
\pgfsetstrokeopacity{0.750000}%
\pgfsetdash{}{0pt}%
\pgfpathmoveto{\pgfqpoint{8.396039in}{1.794929in}}%
\pgfpathlineto{\pgfqpoint{8.409842in}{1.762999in}}%
\pgfpathlineto{\pgfqpoint{8.423392in}{1.735185in}}%
\pgfpathlineto{\pgfqpoint{8.436693in}{1.709540in}}%
\pgfpathlineto{\pgfqpoint{8.449746in}{1.685858in}}%
\pgfpathlineto{\pgfqpoint{8.462553in}{1.663944in}}%
\pgfpathlineto{\pgfqpoint{8.475117in}{1.643624in}}%
\pgfpathlineto{\pgfqpoint{8.487440in}{1.624744in}}%
\pgfpathlineto{\pgfqpoint{8.499523in}{1.607167in}}%
\pgfpathlineto{\pgfqpoint{8.511370in}{1.590770in}}%
\pgfpathlineto{\pgfqpoint{8.522982in}{1.575448in}}%
\pgfpathlineto{\pgfqpoint{8.534362in}{1.561104in}}%
\pgfpathlineto{\pgfqpoint{8.545511in}{1.547654in}}%
\pgfpathlineto{\pgfqpoint{8.556433in}{1.535024in}}%
\pgfpathlineto{\pgfqpoint{8.567130in}{1.523147in}}%
\pgfpathlineto{\pgfqpoint{8.577604in}{1.511964in}}%
\pgfpathlineto{\pgfqpoint{8.587857in}{1.501421in}}%
\pgfpathlineto{\pgfqpoint{8.597890in}{1.491471in}}%
\pgfpathlineto{\pgfqpoint{8.607705in}{1.482069in}}%
\pgfpathlineto{\pgfqpoint{8.617304in}{1.473177in}}%
\pgfpathlineto{\pgfqpoint{8.626689in}{1.464760in}}%
\pgfpathlineto{\pgfqpoint{8.635861in}{1.456787in}}%
\pgfpathlineto{\pgfqpoint{8.644822in}{1.449226in}}%
\pgfpathlineto{\pgfqpoint{8.653573in}{1.442054in}}%
\pgfpathlineto{\pgfqpoint{8.662116in}{1.435244in}}%
\pgfpathlineto{\pgfqpoint{8.670453in}{1.428774in}}%
\pgfpathlineto{\pgfqpoint{8.678585in}{1.422625in}}%
\pgfpathlineto{\pgfqpoint{8.686516in}{1.416778in}}%
\pgfpathlineto{\pgfqpoint{8.694246in}{1.411216in}}%
\pgfpathlineto{\pgfqpoint{8.701779in}{1.405922in}}%
\pgfpathlineto{\pgfqpoint{8.709117in}{1.400881in}}%
\pgfpathlineto{\pgfqpoint{8.716263in}{1.396082in}}%
\pgfpathlineto{\pgfqpoint{8.723218in}{1.391509in}}%
\pgfpathlineto{\pgfqpoint{8.729985in}{1.387154in}}%
\pgfpathlineto{\pgfqpoint{8.736568in}{1.383003in}}%
\pgfpathlineto{\pgfqpoint{8.742967in}{1.379048in}}%
\pgfpathlineto{\pgfqpoint{8.749185in}{1.375280in}}%
\pgfpathlineto{\pgfqpoint{8.755225in}{1.371688in}}%
\pgfpathlineto{\pgfqpoint{8.761088in}{1.368266in}}%
\pgfpathlineto{\pgfqpoint{8.766775in}{1.365006in}}%
\pgfpathlineto{\pgfqpoint{8.772290in}{1.361900in}}%
\pgfpathlineto{\pgfqpoint{8.777632in}{1.358942in}}%
\pgfpathlineto{\pgfqpoint{8.782803in}{1.356126in}}%
\pgfpathlineto{\pgfqpoint{8.787805in}{1.353447in}}%
\pgfpathlineto{\pgfqpoint{8.792638in}{1.350898in}}%
\pgfpathlineto{\pgfqpoint{8.797303in}{1.348475in}}%
\pgfpathlineto{\pgfqpoint{8.801802in}{1.346173in}}%
\pgfpathlineto{\pgfqpoint{8.806134in}{1.343988in}}%
\pgfpathlineto{\pgfqpoint{8.810300in}{1.341914in}}%
\pgfpathlineto{\pgfqpoint{8.814302in}{1.339950in}}%
\pgfpathlineto{\pgfqpoint{8.818140in}{1.338090in}}%
\pgfpathlineto{\pgfqpoint{8.821815in}{1.336331in}}%
\pgfpathlineto{\pgfqpoint{8.825328in}{1.334671in}}%
\pgfpathlineto{\pgfqpoint{8.828679in}{1.333106in}}%
\pgfusepath{stroke}%
\end{pgfscope}%
\begin{pgfscope}%
\pgfpathrectangle{\pgfqpoint{8.282041in}{0.790446in}}{\pgfqpoint{1.897959in}{1.372727in}}%
\pgfusepath{clip}%
\pgfsetrectcap%
\pgfsetroundjoin%
\pgfsetlinewidth{1.505625pt}%
\definecolor{currentstroke}{rgb}{1.000000,0.000000,0.000000}%
\pgfsetstrokecolor{currentstroke}%
\pgfsetstrokeopacity{0.750000}%
\pgfsetdash{}{0pt}%
\pgfpathmoveto{\pgfqpoint{8.468496in}{2.045612in}}%
\pgfpathlineto{\pgfqpoint{8.486633in}{2.051707in}}%
\pgfpathlineto{\pgfqpoint{8.504281in}{2.057520in}}%
\pgfpathlineto{\pgfqpoint{8.521446in}{2.062232in}}%
\pgfpathlineto{\pgfqpoint{8.538133in}{2.066116in}}%
\pgfpathlineto{\pgfqpoint{8.554349in}{2.069364in}}%
\pgfpathlineto{\pgfqpoint{8.570099in}{2.072112in}}%
\pgfpathlineto{\pgfqpoint{8.585391in}{2.074464in}}%
\pgfpathlineto{\pgfqpoint{8.600230in}{2.076497in}}%
\pgfpathlineto{\pgfqpoint{8.614621in}{2.078271in}}%
\pgfpathlineto{\pgfqpoint{8.628572in}{2.079830in}}%
\pgfpathlineto{\pgfqpoint{8.642088in}{2.081210in}}%
\pgfpathlineto{\pgfqpoint{8.655176in}{2.082440in}}%
\pgfpathlineto{\pgfqpoint{8.667840in}{2.083543in}}%
\pgfpathlineto{\pgfqpoint{8.680089in}{2.084538in}}%
\pgfpathlineto{\pgfqpoint{8.691927in}{2.085440in}}%
\pgfpathlineto{\pgfqpoint{8.703361in}{2.086260in}}%
\pgfpathlineto{\pgfqpoint{8.714397in}{2.087011in}}%
\pgfpathlineto{\pgfqpoint{8.725041in}{2.087700in}}%
\pgfpathlineto{\pgfqpoint{8.735300in}{2.088334in}}%
\pgfpathlineto{\pgfqpoint{8.745179in}{2.088921in}}%
\pgfpathlineto{\pgfqpoint{8.754683in}{2.089465in}}%
\pgfpathlineto{\pgfqpoint{8.763819in}{2.089970in}}%
\pgfpathlineto{\pgfqpoint{8.772591in}{2.090441in}}%
\pgfpathlineto{\pgfqpoint{8.781005in}{2.090881in}}%
\pgfpathlineto{\pgfqpoint{8.789066in}{2.091293in}}%
\pgfpathlineto{\pgfqpoint{8.796779in}{2.091679in}}%
\pgfpathlineto{\pgfqpoint{8.804149in}{2.092042in}}%
\pgfpathlineto{\pgfqpoint{8.811179in}{2.092384in}}%
\pgfpathlineto{\pgfqpoint{8.817874in}{2.092705in}}%
\pgfusepath{stroke}%
\end{pgfscope}%
\begin{pgfscope}%
\pgfpathrectangle{\pgfqpoint{8.282041in}{0.790446in}}{\pgfqpoint{1.897959in}{1.372727in}}%
\pgfusepath{clip}%
\pgfsetrectcap%
\pgfsetroundjoin%
\pgfsetlinewidth{1.505625pt}%
\definecolor{currentstroke}{rgb}{0.000000,0.000000,1.000000}%
\pgfsetstrokecolor{currentstroke}%
\pgfsetstrokeopacity{0.750000}%
\pgfsetdash{}{0pt}%
\pgfpathmoveto{\pgfqpoint{8.468496in}{1.107182in}}%
\pgfpathlineto{\pgfqpoint{8.486633in}{1.107382in}}%
\pgfpathlineto{\pgfqpoint{8.504281in}{1.107384in}}%
\pgfpathlineto{\pgfqpoint{8.521446in}{1.106312in}}%
\pgfpathlineto{\pgfqpoint{8.538133in}{1.104391in}}%
\pgfpathlineto{\pgfqpoint{8.554349in}{1.101777in}}%
\pgfpathlineto{\pgfqpoint{8.570099in}{1.098579in}}%
\pgfpathlineto{\pgfqpoint{8.585391in}{1.094874in}}%
\pgfpathlineto{\pgfqpoint{8.600230in}{1.090719in}}%
\pgfpathlineto{\pgfqpoint{8.614621in}{1.086153in}}%
\pgfpathlineto{\pgfqpoint{8.628572in}{1.081205in}}%
\pgfpathlineto{\pgfqpoint{8.642088in}{1.075894in}}%
\pgfpathlineto{\pgfqpoint{8.655176in}{1.070232in}}%
\pgfpathlineto{\pgfqpoint{8.667840in}{1.064229in}}%
\pgfpathlineto{\pgfqpoint{8.680089in}{1.057887in}}%
\pgfpathlineto{\pgfqpoint{8.691927in}{1.051206in}}%
\pgfpathlineto{\pgfqpoint{8.703361in}{1.044183in}}%
\pgfpathlineto{\pgfqpoint{8.714397in}{1.036812in}}%
\pgfpathlineto{\pgfqpoint{8.725041in}{1.029085in}}%
\pgfpathlineto{\pgfqpoint{8.735300in}{1.020991in}}%
\pgfpathlineto{\pgfqpoint{8.745179in}{1.012518in}}%
\pgfpathlineto{\pgfqpoint{8.754683in}{1.003649in}}%
\pgfpathlineto{\pgfqpoint{8.763819in}{0.994367in}}%
\pgfpathlineto{\pgfqpoint{8.772591in}{0.984651in}}%
\pgfpathlineto{\pgfqpoint{8.781005in}{0.974478in}}%
\pgfpathlineto{\pgfqpoint{8.789066in}{0.963820in}}%
\pgfpathlineto{\pgfqpoint{8.796779in}{0.952648in}}%
\pgfpathlineto{\pgfqpoint{8.804149in}{0.940926in}}%
\pgfpathlineto{\pgfqpoint{8.811179in}{0.928616in}}%
\pgfpathlineto{\pgfqpoint{8.817874in}{0.915670in}}%
\pgfusepath{stroke}%
\end{pgfscope}%
\begin{pgfscope}%
\pgfpathrectangle{\pgfqpoint{8.282041in}{0.790446in}}{\pgfqpoint{1.897959in}{1.372727in}}%
\pgfusepath{clip}%
\pgfsetrectcap%
\pgfsetroundjoin%
\pgfsetlinewidth{1.505625pt}%
\definecolor{currentstroke}{rgb}{0.000000,0.750000,0.750000}%
\pgfsetstrokecolor{currentstroke}%
\pgfsetstrokeopacity{0.750000}%
\pgfsetdash{}{0pt}%
\pgfpathmoveto{\pgfqpoint{8.468496in}{1.750232in}}%
\pgfpathlineto{\pgfqpoint{8.486633in}{1.722589in}}%
\pgfpathlineto{\pgfqpoint{8.504281in}{1.698569in}}%
\pgfpathlineto{\pgfqpoint{8.521446in}{1.676685in}}%
\pgfpathlineto{\pgfqpoint{8.538133in}{1.656695in}}%
\pgfpathlineto{\pgfqpoint{8.554349in}{1.638386in}}%
\pgfpathlineto{\pgfqpoint{8.570099in}{1.621574in}}%
\pgfpathlineto{\pgfqpoint{8.585391in}{1.606100in}}%
\pgfpathlineto{\pgfqpoint{8.600230in}{1.591828in}}%
\pgfpathlineto{\pgfqpoint{8.614621in}{1.578636in}}%
\pgfpathlineto{\pgfqpoint{8.628572in}{1.566420in}}%
\pgfpathlineto{\pgfqpoint{8.642088in}{1.555090in}}%
\pgfpathlineto{\pgfqpoint{8.655176in}{1.544563in}}%
\pgfpathlineto{\pgfqpoint{8.667840in}{1.534770in}}%
\pgfpathlineto{\pgfqpoint{8.680089in}{1.525649in}}%
\pgfpathlineto{\pgfqpoint{8.691927in}{1.517143in}}%
\pgfpathlineto{\pgfqpoint{8.703361in}{1.509203in}}%
\pgfpathlineto{\pgfqpoint{8.714397in}{1.501785in}}%
\pgfpathlineto{\pgfqpoint{8.725041in}{1.494849in}}%
\pgfpathlineto{\pgfqpoint{8.735300in}{1.488360in}}%
\pgfpathlineto{\pgfqpoint{8.745179in}{1.482287in}}%
\pgfpathlineto{\pgfqpoint{8.754683in}{1.476600in}}%
\pgfpathlineto{\pgfqpoint{8.763819in}{1.471275in}}%
\pgfpathlineto{\pgfqpoint{8.772591in}{1.466286in}}%
\pgfpathlineto{\pgfqpoint{8.781005in}{1.461613in}}%
\pgfpathlineto{\pgfqpoint{8.789066in}{1.457237in}}%
\pgfpathlineto{\pgfqpoint{8.796779in}{1.453140in}}%
\pgfpathlineto{\pgfqpoint{8.804149in}{1.449307in}}%
\pgfpathlineto{\pgfqpoint{8.811179in}{1.445722in}}%
\pgfpathlineto{\pgfqpoint{8.817874in}{1.442372in}}%
\pgfusepath{stroke}%
\end{pgfscope}%
\begin{pgfscope}%
\pgfpathrectangle{\pgfqpoint{8.282041in}{0.790446in}}{\pgfqpoint{1.897959in}{1.372727in}}%
\pgfusepath{clip}%
\pgfsetrectcap%
\pgfsetroundjoin%
\pgfsetlinewidth{1.505625pt}%
\definecolor{currentstroke}{rgb}{1.000000,0.000000,0.000000}%
\pgfsetstrokecolor{currentstroke}%
\pgfsetstrokeopacity{0.750000}%
\pgfsetdash{}{0pt}%
\pgfpathmoveto{\pgfqpoint{8.505852in}{1.856732in}}%
\pgfpathlineto{\pgfqpoint{8.523987in}{1.872188in}}%
\pgfpathlineto{\pgfqpoint{8.541573in}{1.887531in}}%
\pgfpathlineto{\pgfqpoint{8.558623in}{1.900726in}}%
\pgfpathlineto{\pgfqpoint{8.575150in}{1.912181in}}%
\pgfpathlineto{\pgfqpoint{8.591169in}{1.922205in}}%
\pgfpathlineto{\pgfqpoint{8.606692in}{1.931040in}}%
\pgfpathlineto{\pgfqpoint{8.621733in}{1.938878in}}%
\pgfpathlineto{\pgfqpoint{8.636305in}{1.945872in}}%
\pgfpathlineto{\pgfqpoint{8.650421in}{1.952144in}}%
\pgfpathlineto{\pgfqpoint{8.664095in}{1.957795in}}%
\pgfpathlineto{\pgfqpoint{8.677341in}{1.962909in}}%
\pgfpathlineto{\pgfqpoint{8.690171in}{1.967555in}}%
\pgfpathlineto{\pgfqpoint{8.702606in}{1.971790in}}%
\pgfpathlineto{\pgfqpoint{8.714656in}{1.975664in}}%
\pgfpathlineto{\pgfqpoint{8.726335in}{1.979218in}}%
\pgfpathlineto{\pgfqpoint{8.737653in}{1.982488in}}%
\pgfpathlineto{\pgfqpoint{8.748624in}{1.985504in}}%
\pgfpathlineto{\pgfqpoint{8.759259in}{1.988292in}}%
\pgfpathlineto{\pgfqpoint{8.769569in}{1.990876in}}%
\pgfpathlineto{\pgfqpoint{8.779562in}{1.993275in}}%
\pgfpathlineto{\pgfqpoint{8.789248in}{1.995508in}}%
\pgfpathlineto{\pgfqpoint{8.798635in}{1.997588in}}%
\pgfpathlineto{\pgfqpoint{8.807732in}{1.999529in}}%
\pgfpathlineto{\pgfqpoint{8.816545in}{2.001344in}}%
\pgfpathlineto{\pgfqpoint{8.825081in}{2.003042in}}%
\pgfpathlineto{\pgfqpoint{8.833344in}{2.004634in}}%
\pgfpathlineto{\pgfqpoint{8.841339in}{2.006128in}}%
\pgfpathlineto{\pgfqpoint{8.849071in}{2.007531in}}%
\pgfpathlineto{\pgfqpoint{8.856545in}{2.008850in}}%
\pgfpathlineto{\pgfqpoint{8.863766in}{2.010091in}}%
\pgfpathlineto{\pgfqpoint{8.870737in}{2.011259in}}%
\pgfpathlineto{\pgfqpoint{8.877462in}{2.012360in}}%
\pgfpathlineto{\pgfqpoint{8.883946in}{2.013397in}}%
\pgfpathlineto{\pgfqpoint{8.890192in}{2.014375in}}%
\pgfpathlineto{\pgfqpoint{8.896206in}{2.015298in}}%
\pgfpathlineto{\pgfqpoint{8.901990in}{2.016168in}}%
\pgfpathlineto{\pgfqpoint{8.907548in}{2.016988in}}%
\pgfpathlineto{\pgfqpoint{8.912882in}{2.017762in}}%
\pgfusepath{stroke}%
\end{pgfscope}%
\begin{pgfscope}%
\pgfpathrectangle{\pgfqpoint{8.282041in}{0.790446in}}{\pgfqpoint{1.897959in}{1.372727in}}%
\pgfusepath{clip}%
\pgfsetrectcap%
\pgfsetroundjoin%
\pgfsetlinewidth{1.505625pt}%
\definecolor{currentstroke}{rgb}{0.000000,0.000000,1.000000}%
\pgfsetstrokecolor{currentstroke}%
\pgfsetstrokeopacity{0.750000}%
\pgfsetdash{}{0pt}%
\pgfpathmoveto{\pgfqpoint{8.505852in}{0.989473in}}%
\pgfpathlineto{\pgfqpoint{8.523987in}{0.996628in}}%
\pgfpathlineto{\pgfqpoint{8.541573in}{1.004242in}}%
\pgfpathlineto{\pgfqpoint{8.558623in}{1.010142in}}%
\pgfpathlineto{\pgfqpoint{8.575150in}{1.014630in}}%
\pgfpathlineto{\pgfqpoint{8.591169in}{1.017939in}}%
\pgfpathlineto{\pgfqpoint{8.606692in}{1.020246in}}%
\pgfpathlineto{\pgfqpoint{8.621733in}{1.021691in}}%
\pgfpathlineto{\pgfqpoint{8.636305in}{1.022386in}}%
\pgfpathlineto{\pgfqpoint{8.650421in}{1.022418in}}%
\pgfpathlineto{\pgfqpoint{8.664095in}{1.021859in}}%
\pgfpathlineto{\pgfqpoint{8.677341in}{1.020766in}}%
\pgfpathlineto{\pgfqpoint{8.690171in}{1.019188in}}%
\pgfpathlineto{\pgfqpoint{8.702606in}{1.017162in}}%
\pgfpathlineto{\pgfqpoint{8.714656in}{1.014719in}}%
\pgfpathlineto{\pgfqpoint{8.726335in}{1.011885in}}%
\pgfpathlineto{\pgfqpoint{8.737653in}{1.008682in}}%
\pgfpathlineto{\pgfqpoint{8.748624in}{1.005125in}}%
\pgfpathlineto{\pgfqpoint{8.759259in}{1.001228in}}%
\pgfpathlineto{\pgfqpoint{8.769569in}{0.997001in}}%
\pgfpathlineto{\pgfqpoint{8.779562in}{0.992452in}}%
\pgfpathlineto{\pgfqpoint{8.789248in}{0.987587in}}%
\pgfpathlineto{\pgfqpoint{8.798635in}{0.982408in}}%
\pgfpathlineto{\pgfqpoint{8.807732in}{0.976917in}}%
\pgfpathlineto{\pgfqpoint{8.816545in}{0.971114in}}%
\pgfpathlineto{\pgfqpoint{8.825081in}{0.964996in}}%
\pgfpathlineto{\pgfqpoint{8.833344in}{0.958560in}}%
\pgfpathlineto{\pgfqpoint{8.841339in}{0.951801in}}%
\pgfpathlineto{\pgfqpoint{8.849071in}{0.944711in}}%
\pgfpathlineto{\pgfqpoint{8.856545in}{0.937283in}}%
\pgfpathlineto{\pgfqpoint{8.863766in}{0.929506in}}%
\pgfpathlineto{\pgfqpoint{8.870737in}{0.921368in}}%
\pgfpathlineto{\pgfqpoint{8.877462in}{0.912856in}}%
\pgfpathlineto{\pgfqpoint{8.883946in}{0.903953in}}%
\pgfpathlineto{\pgfqpoint{8.890192in}{0.894640in}}%
\pgfpathlineto{\pgfqpoint{8.896206in}{0.884898in}}%
\pgfpathlineto{\pgfqpoint{8.901990in}{0.874703in}}%
\pgfpathlineto{\pgfqpoint{8.907548in}{0.864028in}}%
\pgfpathlineto{\pgfqpoint{8.912882in}{0.852843in}}%
\pgfusepath{stroke}%
\end{pgfscope}%
\begin{pgfscope}%
\pgfpathrectangle{\pgfqpoint{8.282041in}{0.790446in}}{\pgfqpoint{1.897959in}{1.372727in}}%
\pgfusepath{clip}%
\pgfsetrectcap%
\pgfsetroundjoin%
\pgfsetlinewidth{1.505625pt}%
\definecolor{currentstroke}{rgb}{0.000000,0.750000,0.750000}%
\pgfsetstrokecolor{currentstroke}%
\pgfsetstrokeopacity{0.750000}%
\pgfsetdash{}{0pt}%
\pgfpathmoveto{\pgfqpoint{8.505852in}{2.006475in}}%
\pgfpathlineto{\pgfqpoint{8.523987in}{1.993865in}}%
\pgfpathlineto{\pgfqpoint{8.541573in}{1.984283in}}%
\pgfpathlineto{\pgfqpoint{8.558623in}{1.975156in}}%
\pgfpathlineto{\pgfqpoint{8.575150in}{1.966482in}}%
\pgfpathlineto{\pgfqpoint{8.591169in}{1.958246in}}%
\pgfpathlineto{\pgfqpoint{8.606692in}{1.950433in}}%
\pgfpathlineto{\pgfqpoint{8.621733in}{1.943026in}}%
\pgfpathlineto{\pgfqpoint{8.636305in}{1.936005in}}%
\pgfpathlineto{\pgfqpoint{8.650421in}{1.929348in}}%
\pgfpathlineto{\pgfqpoint{8.664095in}{1.923038in}}%
\pgfpathlineto{\pgfqpoint{8.677341in}{1.917055in}}%
\pgfpathlineto{\pgfqpoint{8.690171in}{1.911380in}}%
\pgfpathlineto{\pgfqpoint{8.702606in}{1.905997in}}%
\pgfpathlineto{\pgfqpoint{8.714656in}{1.900889in}}%
\pgfpathlineto{\pgfqpoint{8.726335in}{1.896040in}}%
\pgfpathlineto{\pgfqpoint{8.737653in}{1.891436in}}%
\pgfpathlineto{\pgfqpoint{8.748624in}{1.887064in}}%
\pgfpathlineto{\pgfqpoint{8.759259in}{1.882910in}}%
\pgfpathlineto{\pgfqpoint{8.769569in}{1.878963in}}%
\pgfpathlineto{\pgfqpoint{8.779562in}{1.875212in}}%
\pgfpathlineto{\pgfqpoint{8.789248in}{1.871646in}}%
\pgfpathlineto{\pgfqpoint{8.798635in}{1.868256in}}%
\pgfpathlineto{\pgfqpoint{8.807732in}{1.865032in}}%
\pgfpathlineto{\pgfqpoint{8.816545in}{1.861966in}}%
\pgfpathlineto{\pgfqpoint{8.825081in}{1.859051in}}%
\pgfpathlineto{\pgfqpoint{8.833344in}{1.856279in}}%
\pgfpathlineto{\pgfqpoint{8.841339in}{1.853643in}}%
\pgfpathlineto{\pgfqpoint{8.849071in}{1.851137in}}%
\pgfpathlineto{\pgfqpoint{8.856545in}{1.848755in}}%
\pgfpathlineto{\pgfqpoint{8.863766in}{1.846490in}}%
\pgfpathlineto{\pgfqpoint{8.870737in}{1.844340in}}%
\pgfpathlineto{\pgfqpoint{8.877462in}{1.842297in}}%
\pgfpathlineto{\pgfqpoint{8.883946in}{1.840358in}}%
\pgfpathlineto{\pgfqpoint{8.890192in}{1.838518in}}%
\pgfpathlineto{\pgfqpoint{8.896206in}{1.836773in}}%
\pgfpathlineto{\pgfqpoint{8.901990in}{1.835121in}}%
\pgfpathlineto{\pgfqpoint{8.907548in}{1.833556in}}%
\pgfpathlineto{\pgfqpoint{8.912882in}{1.832076in}}%
\pgfusepath{stroke}%
\end{pgfscope}%
\begin{pgfscope}%
\pgfpathrectangle{\pgfqpoint{8.282041in}{0.790446in}}{\pgfqpoint{1.897959in}{1.372727in}}%
\pgfusepath{clip}%
\pgfsetrectcap%
\pgfsetroundjoin%
\pgfsetlinewidth{1.505625pt}%
\definecolor{currentstroke}{rgb}{1.000000,0.000000,0.000000}%
\pgfsetstrokecolor{currentstroke}%
\pgfsetstrokeopacity{0.750000}%
\pgfsetdash{}{0pt}%
\pgfpathmoveto{\pgfqpoint{8.442158in}{1.950474in}}%
\pgfpathlineto{\pgfqpoint{8.459568in}{1.964769in}}%
\pgfpathlineto{\pgfqpoint{8.476656in}{1.978606in}}%
\pgfpathlineto{\pgfqpoint{8.493428in}{1.990049in}}%
\pgfpathlineto{\pgfqpoint{8.509889in}{1.999634in}}%
\pgfpathlineto{\pgfqpoint{8.526044in}{2.007753in}}%
\pgfpathlineto{\pgfqpoint{8.541898in}{2.014697in}}%
\pgfpathlineto{\pgfqpoint{8.557456in}{2.020690in}}%
\pgfpathlineto{\pgfqpoint{8.572724in}{2.025904in}}%
\pgfpathlineto{\pgfqpoint{8.587707in}{2.030473in}}%
\pgfpathlineto{\pgfqpoint{8.602409in}{2.034503in}}%
\pgfpathlineto{\pgfqpoint{8.616836in}{2.038078in}}%
\pgfpathlineto{\pgfqpoint{8.630993in}{2.041268in}}%
\pgfpathlineto{\pgfqpoint{8.644884in}{2.044129in}}%
\pgfpathlineto{\pgfqpoint{8.658514in}{2.046706in}}%
\pgfpathlineto{\pgfqpoint{8.671889in}{2.049039in}}%
\pgfpathlineto{\pgfqpoint{8.685015in}{2.051158in}}%
\pgfpathlineto{\pgfqpoint{8.697896in}{2.053091in}}%
\pgfpathlineto{\pgfqpoint{8.710540in}{2.054860in}}%
\pgfpathlineto{\pgfqpoint{8.722953in}{2.056485in}}%
\pgfpathlineto{\pgfqpoint{8.735139in}{2.057982in}}%
\pgfpathlineto{\pgfqpoint{8.747106in}{2.059365in}}%
\pgfpathlineto{\pgfqpoint{8.758860in}{2.060647in}}%
\pgfpathlineto{\pgfqpoint{8.770406in}{2.061838in}}%
\pgfpathlineto{\pgfqpoint{8.781751in}{2.062948in}}%
\pgfpathlineto{\pgfqpoint{8.792900in}{2.063983in}}%
\pgfpathlineto{\pgfqpoint{8.803857in}{2.064953in}}%
\pgfpathlineto{\pgfqpoint{8.814628in}{2.065862in}}%
\pgfpathlineto{\pgfqpoint{8.825217in}{2.066717in}}%
\pgfpathlineto{\pgfqpoint{8.835628in}{2.067522in}}%
\pgfpathlineto{\pgfqpoint{8.845864in}{2.068282in}}%
\pgfpathlineto{\pgfqpoint{8.855929in}{2.069000in}}%
\pgfpathlineto{\pgfqpoint{8.865826in}{2.069680in}}%
\pgfpathlineto{\pgfqpoint{8.875558in}{2.070325in}}%
\pgfpathlineto{\pgfqpoint{8.885126in}{2.070938in}}%
\pgfpathlineto{\pgfqpoint{8.894534in}{2.071522in}}%
\pgfpathlineto{\pgfqpoint{8.903784in}{2.072078in}}%
\pgfpathlineto{\pgfqpoint{8.912877in}{2.072609in}}%
\pgfpathlineto{\pgfqpoint{8.921817in}{2.073117in}}%
\pgfpathlineto{\pgfqpoint{8.930606in}{2.073603in}}%
\pgfpathlineto{\pgfqpoint{8.939244in}{2.074068in}}%
\pgfpathlineto{\pgfqpoint{8.947736in}{2.074515in}}%
\pgfpathlineto{\pgfqpoint{8.956082in}{2.074945in}}%
\pgfpathlineto{\pgfqpoint{8.964285in}{2.075358in}}%
\pgfpathlineto{\pgfqpoint{8.972347in}{2.075755in}}%
\pgfpathlineto{\pgfqpoint{8.980269in}{2.076139in}}%
\pgfpathlineto{\pgfqpoint{8.988055in}{2.076509in}}%
\pgfpathlineto{\pgfqpoint{8.995706in}{2.076867in}}%
\pgfpathlineto{\pgfqpoint{9.003224in}{2.077213in}}%
\pgfpathlineto{\pgfqpoint{9.010610in}{2.077548in}}%
\pgfpathlineto{\pgfqpoint{9.017867in}{2.077872in}}%
\pgfpathlineto{\pgfqpoint{9.024996in}{2.078187in}}%
\pgfpathlineto{\pgfqpoint{9.031998in}{2.078492in}}%
\pgfpathlineto{\pgfqpoint{9.038874in}{2.078788in}}%
\pgfpathlineto{\pgfqpoint{9.045628in}{2.079077in}}%
\pgfpathlineto{\pgfqpoint{9.052258in}{2.079357in}}%
\pgfpathlineto{\pgfqpoint{9.058767in}{2.079630in}}%
\pgfpathlineto{\pgfqpoint{9.065157in}{2.079895in}}%
\pgfpathlineto{\pgfqpoint{9.071427in}{2.080154in}}%
\pgfpathlineto{\pgfqpoint{9.077579in}{2.080407in}}%
\pgfpathlineto{\pgfqpoint{9.083614in}{2.080654in}}%
\pgfpathlineto{\pgfqpoint{9.089534in}{2.080894in}}%
\pgfpathlineto{\pgfqpoint{9.095339in}{2.081129in}}%
\pgfpathlineto{\pgfqpoint{9.101029in}{2.081359in}}%
\pgfpathlineto{\pgfqpoint{9.106607in}{2.081584in}}%
\pgfpathlineto{\pgfqpoint{9.112073in}{2.081804in}}%
\pgfpathlineto{\pgfqpoint{9.117428in}{2.082019in}}%
\pgfpathlineto{\pgfqpoint{9.122673in}{2.082229in}}%
\pgfpathlineto{\pgfqpoint{9.127810in}{2.082436in}}%
\pgfpathlineto{\pgfqpoint{9.132838in}{2.082638in}}%
\pgfpathlineto{\pgfqpoint{9.137759in}{2.082836in}}%
\pgfpathlineto{\pgfqpoint{9.142574in}{2.083030in}}%
\pgfpathlineto{\pgfqpoint{9.147284in}{2.083220in}}%
\pgfpathlineto{\pgfqpoint{9.151890in}{2.083407in}}%
\pgfpathlineto{\pgfqpoint{9.156393in}{2.083590in}}%
\pgfpathlineto{\pgfqpoint{9.160792in}{2.083770in}}%
\pgfpathlineto{\pgfqpoint{9.165090in}{2.083947in}}%
\pgfpathlineto{\pgfqpoint{9.169287in}{2.084120in}}%
\pgfpathlineto{\pgfqpoint{9.173384in}{2.084290in}}%
\pgfpathlineto{\pgfqpoint{9.177381in}{2.084457in}}%
\pgfpathlineto{\pgfqpoint{9.181278in}{2.084621in}}%
\pgfpathlineto{\pgfqpoint{9.185077in}{2.084782in}}%
\pgfpathlineto{\pgfqpoint{9.188777in}{2.084941in}}%
\pgfpathlineto{\pgfqpoint{9.192379in}{2.085096in}}%
\pgfpathlineto{\pgfqpoint{9.195885in}{2.085248in}}%
\pgfpathlineto{\pgfqpoint{9.199293in}{2.085398in}}%
\pgfpathlineto{\pgfqpoint{9.202605in}{2.085545in}}%
\pgfpathlineto{\pgfqpoint{9.205821in}{2.085690in}}%
\pgfpathlineto{\pgfqpoint{9.208942in}{2.085832in}}%
\pgfpathlineto{\pgfqpoint{9.211967in}{2.085971in}}%
\pgfpathlineto{\pgfqpoint{9.214897in}{2.086107in}}%
\pgfpathlineto{\pgfqpoint{9.217734in}{2.086242in}}%
\pgfpathlineto{\pgfqpoint{9.220476in}{2.086373in}}%
\pgfpathlineto{\pgfqpoint{9.223125in}{2.086503in}}%
\pgfpathlineto{\pgfqpoint{9.225680in}{2.086629in}}%
\pgfpathlineto{\pgfqpoint{9.228144in}{2.086753in}}%
\pgfpathlineto{\pgfqpoint{9.230514in}{2.086875in}}%
\pgfpathlineto{\pgfqpoint{9.232794in}{2.086994in}}%
\pgfpathlineto{\pgfqpoint{9.234981in}{2.087111in}}%
\pgfpathlineto{\pgfqpoint{9.237078in}{2.087226in}}%
\pgfpathlineto{\pgfqpoint{9.239084in}{2.087338in}}%
\pgfpathlineto{\pgfqpoint{9.240999in}{2.087447in}}%
\pgfusepath{stroke}%
\end{pgfscope}%
\begin{pgfscope}%
\pgfpathrectangle{\pgfqpoint{8.282041in}{0.790446in}}{\pgfqpoint{1.897959in}{1.372727in}}%
\pgfusepath{clip}%
\pgfsetrectcap%
\pgfsetroundjoin%
\pgfsetlinewidth{1.505625pt}%
\definecolor{currentstroke}{rgb}{0.000000,0.000000,1.000000}%
\pgfsetstrokecolor{currentstroke}%
\pgfsetstrokeopacity{0.750000}%
\pgfsetdash{}{0pt}%
\pgfpathmoveto{\pgfqpoint{8.442158in}{1.192116in}}%
\pgfpathlineto{\pgfqpoint{8.459568in}{1.202878in}}%
\pgfpathlineto{\pgfqpoint{8.476656in}{1.213607in}}%
\pgfpathlineto{\pgfqpoint{8.493428in}{1.222264in}}%
\pgfpathlineto{\pgfqpoint{8.509889in}{1.229313in}}%
\pgfpathlineto{\pgfqpoint{8.526044in}{1.235088in}}%
\pgfpathlineto{\pgfqpoint{8.541898in}{1.239841in}}%
\pgfpathlineto{\pgfqpoint{8.557456in}{1.243762in}}%
\pgfpathlineto{\pgfqpoint{8.572724in}{1.246997in}}%
\pgfpathlineto{\pgfqpoint{8.587707in}{1.249659in}}%
\pgfpathlineto{\pgfqpoint{8.602409in}{1.251838in}}%
\pgfpathlineto{\pgfqpoint{8.616836in}{1.253605in}}%
\pgfpathlineto{\pgfqpoint{8.630993in}{1.255016in}}%
\pgfpathlineto{\pgfqpoint{8.644884in}{1.256117in}}%
\pgfpathlineto{\pgfqpoint{8.658514in}{1.256947in}}%
\pgfpathlineto{\pgfqpoint{8.671889in}{1.257536in}}%
\pgfpathlineto{\pgfqpoint{8.685015in}{1.257909in}}%
\pgfpathlineto{\pgfqpoint{8.697896in}{1.258089in}}%
\pgfpathlineto{\pgfqpoint{8.710540in}{1.258091in}}%
\pgfpathlineto{\pgfqpoint{8.722953in}{1.257932in}}%
\pgfpathlineto{\pgfqpoint{8.735139in}{1.257624in}}%
\pgfpathlineto{\pgfqpoint{8.747106in}{1.257178in}}%
\pgfpathlineto{\pgfqpoint{8.758860in}{1.256602in}}%
\pgfpathlineto{\pgfqpoint{8.770406in}{1.255905in}}%
\pgfpathlineto{\pgfqpoint{8.781751in}{1.255093in}}%
\pgfpathlineto{\pgfqpoint{8.792900in}{1.254172in}}%
\pgfpathlineto{\pgfqpoint{8.803857in}{1.253146in}}%
\pgfpathlineto{\pgfqpoint{8.814628in}{1.252020in}}%
\pgfpathlineto{\pgfqpoint{8.825217in}{1.250798in}}%
\pgfpathlineto{\pgfqpoint{8.835628in}{1.249482in}}%
\pgfpathlineto{\pgfqpoint{8.845864in}{1.248075in}}%
\pgfpathlineto{\pgfqpoint{8.855929in}{1.246580in}}%
\pgfpathlineto{\pgfqpoint{8.865826in}{1.244998in}}%
\pgfpathlineto{\pgfqpoint{8.875558in}{1.243331in}}%
\pgfpathlineto{\pgfqpoint{8.885126in}{1.241580in}}%
\pgfpathlineto{\pgfqpoint{8.894534in}{1.239746in}}%
\pgfpathlineto{\pgfqpoint{8.903784in}{1.237831in}}%
\pgfpathlineto{\pgfqpoint{8.912877in}{1.235835in}}%
\pgfpathlineto{\pgfqpoint{8.921817in}{1.233758in}}%
\pgfpathlineto{\pgfqpoint{8.930606in}{1.231602in}}%
\pgfpathlineto{\pgfqpoint{8.939244in}{1.229365in}}%
\pgfpathlineto{\pgfqpoint{8.947736in}{1.227048in}}%
\pgfpathlineto{\pgfqpoint{8.956082in}{1.224652in}}%
\pgfpathlineto{\pgfqpoint{8.964285in}{1.222175in}}%
\pgfpathlineto{\pgfqpoint{8.972347in}{1.219619in}}%
\pgfpathlineto{\pgfqpoint{8.980269in}{1.216982in}}%
\pgfpathlineto{\pgfqpoint{8.988055in}{1.214263in}}%
\pgfpathlineto{\pgfqpoint{8.995706in}{1.211463in}}%
\pgfpathlineto{\pgfqpoint{9.003224in}{1.208580in}}%
\pgfpathlineto{\pgfqpoint{9.010610in}{1.205614in}}%
\pgfpathlineto{\pgfqpoint{9.017867in}{1.202564in}}%
\pgfpathlineto{\pgfqpoint{9.024996in}{1.199428in}}%
\pgfpathlineto{\pgfqpoint{9.031998in}{1.196207in}}%
\pgfpathlineto{\pgfqpoint{9.038874in}{1.192898in}}%
\pgfpathlineto{\pgfqpoint{9.045628in}{1.189500in}}%
\pgfpathlineto{\pgfqpoint{9.052258in}{1.186011in}}%
\pgfpathlineto{\pgfqpoint{9.058767in}{1.182431in}}%
\pgfpathlineto{\pgfqpoint{9.065157in}{1.178757in}}%
\pgfpathlineto{\pgfqpoint{9.071427in}{1.174988in}}%
\pgfpathlineto{\pgfqpoint{9.077579in}{1.171122in}}%
\pgfpathlineto{\pgfqpoint{9.083614in}{1.167157in}}%
\pgfpathlineto{\pgfqpoint{9.089534in}{1.163090in}}%
\pgfpathlineto{\pgfqpoint{9.095339in}{1.158920in}}%
\pgfpathlineto{\pgfqpoint{9.101029in}{1.154643in}}%
\pgfpathlineto{\pgfqpoint{9.106607in}{1.150258in}}%
\pgfpathlineto{\pgfqpoint{9.112073in}{1.145761in}}%
\pgfpathlineto{\pgfqpoint{9.117428in}{1.141150in}}%
\pgfpathlineto{\pgfqpoint{9.122673in}{1.136421in}}%
\pgfpathlineto{\pgfqpoint{9.127810in}{1.131572in}}%
\pgfpathlineto{\pgfqpoint{9.132838in}{1.126597in}}%
\pgfpathlineto{\pgfqpoint{9.137759in}{1.121494in}}%
\pgfpathlineto{\pgfqpoint{9.142574in}{1.116259in}}%
\pgfpathlineto{\pgfqpoint{9.147284in}{1.110887in}}%
\pgfpathlineto{\pgfqpoint{9.151890in}{1.105373in}}%
\pgfpathlineto{\pgfqpoint{9.156393in}{1.099713in}}%
\pgfpathlineto{\pgfqpoint{9.160792in}{1.093902in}}%
\pgfpathlineto{\pgfqpoint{9.165090in}{1.087933in}}%
\pgfpathlineto{\pgfqpoint{9.169287in}{1.081800in}}%
\pgfpathlineto{\pgfqpoint{9.173384in}{1.075498in}}%
\pgfpathlineto{\pgfqpoint{9.177381in}{1.069019in}}%
\pgfpathlineto{\pgfqpoint{9.181278in}{1.062355in}}%
\pgfpathlineto{\pgfqpoint{9.185077in}{1.055498in}}%
\pgfpathlineto{\pgfqpoint{9.188777in}{1.048441in}}%
\pgfpathlineto{\pgfqpoint{9.192379in}{1.041171in}}%
\pgfpathlineto{\pgfqpoint{9.195885in}{1.033681in}}%
\pgfpathlineto{\pgfqpoint{9.199293in}{1.025959in}}%
\pgfpathlineto{\pgfqpoint{9.202605in}{1.017992in}}%
\pgfpathlineto{\pgfqpoint{9.205821in}{1.009768in}}%
\pgfpathlineto{\pgfqpoint{9.208942in}{1.001271in}}%
\pgfpathlineto{\pgfqpoint{9.211967in}{0.992486in}}%
\pgfpathlineto{\pgfqpoint{9.214897in}{0.983396in}}%
\pgfpathlineto{\pgfqpoint{9.217734in}{0.973981in}}%
\pgfpathlineto{\pgfqpoint{9.220476in}{0.964220in}}%
\pgfpathlineto{\pgfqpoint{9.223125in}{0.954090in}}%
\pgfpathlineto{\pgfqpoint{9.225680in}{0.943564in}}%
\pgfpathlineto{\pgfqpoint{9.228144in}{0.932613in}}%
\pgfpathlineto{\pgfqpoint{9.230514in}{0.921203in}}%
\pgfpathlineto{\pgfqpoint{9.232794in}{0.909299in}}%
\pgfpathlineto{\pgfqpoint{9.234981in}{0.896857in}}%
\pgfpathlineto{\pgfqpoint{9.237078in}{0.883829in}}%
\pgfpathlineto{\pgfqpoint{9.239084in}{0.870161in}}%
\pgfpathlineto{\pgfqpoint{9.240999in}{0.855788in}}%
\pgfusepath{stroke}%
\end{pgfscope}%
\begin{pgfscope}%
\pgfpathrectangle{\pgfqpoint{8.282041in}{0.790446in}}{\pgfqpoint{1.897959in}{1.372727in}}%
\pgfusepath{clip}%
\pgfsetrectcap%
\pgfsetroundjoin%
\pgfsetlinewidth{1.505625pt}%
\definecolor{currentstroke}{rgb}{0.000000,0.750000,0.750000}%
\pgfsetstrokecolor{currentstroke}%
\pgfsetstrokeopacity{0.750000}%
\pgfsetdash{}{0pt}%
\pgfpathmoveto{\pgfqpoint{8.442158in}{1.929433in}}%
\pgfpathlineto{\pgfqpoint{8.459568in}{1.906676in}}%
\pgfpathlineto{\pgfqpoint{8.476656in}{1.887644in}}%
\pgfpathlineto{\pgfqpoint{8.493428in}{1.869655in}}%
\pgfpathlineto{\pgfqpoint{8.509889in}{1.852681in}}%
\pgfpathlineto{\pgfqpoint{8.526044in}{1.836678in}}%
\pgfpathlineto{\pgfqpoint{8.541898in}{1.821591in}}%
\pgfpathlineto{\pgfqpoint{8.557456in}{1.807367in}}%
\pgfpathlineto{\pgfqpoint{8.572724in}{1.793951in}}%
\pgfpathlineto{\pgfqpoint{8.587707in}{1.781288in}}%
\pgfpathlineto{\pgfqpoint{8.602409in}{1.769327in}}%
\pgfpathlineto{\pgfqpoint{8.616836in}{1.758020in}}%
\pgfpathlineto{\pgfqpoint{8.630993in}{1.747323in}}%
\pgfpathlineto{\pgfqpoint{8.644884in}{1.737194in}}%
\pgfpathlineto{\pgfqpoint{8.658514in}{1.727595in}}%
\pgfpathlineto{\pgfqpoint{8.671889in}{1.718489in}}%
\pgfpathlineto{\pgfqpoint{8.685015in}{1.709844in}}%
\pgfpathlineto{\pgfqpoint{8.697896in}{1.701630in}}%
\pgfpathlineto{\pgfqpoint{8.710540in}{1.693819in}}%
\pgfpathlineto{\pgfqpoint{8.722953in}{1.686386in}}%
\pgfpathlineto{\pgfqpoint{8.735139in}{1.679307in}}%
\pgfpathlineto{\pgfqpoint{8.747106in}{1.672559in}}%
\pgfpathlineto{\pgfqpoint{8.758860in}{1.666123in}}%
\pgfpathlineto{\pgfqpoint{8.770406in}{1.659981in}}%
\pgfpathlineto{\pgfqpoint{8.781751in}{1.654114in}}%
\pgfpathlineto{\pgfqpoint{8.792900in}{1.648506in}}%
\pgfpathlineto{\pgfqpoint{8.803857in}{1.643144in}}%
\pgfpathlineto{\pgfqpoint{8.814628in}{1.638014in}}%
\pgfpathlineto{\pgfqpoint{8.825217in}{1.633101in}}%
\pgfpathlineto{\pgfqpoint{8.835628in}{1.628396in}}%
\pgfpathlineto{\pgfqpoint{8.845864in}{1.623885in}}%
\pgfpathlineto{\pgfqpoint{8.855929in}{1.619560in}}%
\pgfpathlineto{\pgfqpoint{8.865826in}{1.615411in}}%
\pgfpathlineto{\pgfqpoint{8.875558in}{1.611428in}}%
\pgfpathlineto{\pgfqpoint{8.885126in}{1.607604in}}%
\pgfpathlineto{\pgfqpoint{8.894534in}{1.603930in}}%
\pgfpathlineto{\pgfqpoint{8.903784in}{1.600399in}}%
\pgfpathlineto{\pgfqpoint{8.912877in}{1.597004in}}%
\pgfpathlineto{\pgfqpoint{8.921817in}{1.593739in}}%
\pgfpathlineto{\pgfqpoint{8.930606in}{1.590597in}}%
\pgfpathlineto{\pgfqpoint{8.939244in}{1.587573in}}%
\pgfpathlineto{\pgfqpoint{8.947736in}{1.584662in}}%
\pgfpathlineto{\pgfqpoint{8.956082in}{1.581858in}}%
\pgfpathlineto{\pgfqpoint{8.964285in}{1.579157in}}%
\pgfpathlineto{\pgfqpoint{8.972347in}{1.576554in}}%
\pgfpathlineto{\pgfqpoint{8.980269in}{1.574046in}}%
\pgfpathlineto{\pgfqpoint{8.988055in}{1.571627in}}%
\pgfpathlineto{\pgfqpoint{8.995706in}{1.569294in}}%
\pgfpathlineto{\pgfqpoint{9.003224in}{1.567045in}}%
\pgfpathlineto{\pgfqpoint{9.010610in}{1.564874in}}%
\pgfpathlineto{\pgfqpoint{9.017867in}{1.562780in}}%
\pgfpathlineto{\pgfqpoint{9.024996in}{1.560759in}}%
\pgfpathlineto{\pgfqpoint{9.031998in}{1.558808in}}%
\pgfpathlineto{\pgfqpoint{9.038874in}{1.556925in}}%
\pgfpathlineto{\pgfqpoint{9.045628in}{1.555107in}}%
\pgfpathlineto{\pgfqpoint{9.052258in}{1.553352in}}%
\pgfpathlineto{\pgfqpoint{9.058767in}{1.551656in}}%
\pgfpathlineto{\pgfqpoint{9.065157in}{1.550019in}}%
\pgfpathlineto{\pgfqpoint{9.071427in}{1.548438in}}%
\pgfpathlineto{\pgfqpoint{9.077579in}{1.546911in}}%
\pgfpathlineto{\pgfqpoint{9.083614in}{1.545436in}}%
\pgfpathlineto{\pgfqpoint{9.089534in}{1.544011in}}%
\pgfpathlineto{\pgfqpoint{9.095339in}{1.542635in}}%
\pgfpathlineto{\pgfqpoint{9.101029in}{1.541306in}}%
\pgfpathlineto{\pgfqpoint{9.106607in}{1.540022in}}%
\pgfpathlineto{\pgfqpoint{9.112073in}{1.538782in}}%
\pgfpathlineto{\pgfqpoint{9.117428in}{1.537585in}}%
\pgfpathlineto{\pgfqpoint{9.122673in}{1.536429in}}%
\pgfpathlineto{\pgfqpoint{9.127810in}{1.535312in}}%
\pgfpathlineto{\pgfqpoint{9.132838in}{1.534235in}}%
\pgfpathlineto{\pgfqpoint{9.137759in}{1.533194in}}%
\pgfpathlineto{\pgfqpoint{9.142574in}{1.532191in}}%
\pgfpathlineto{\pgfqpoint{9.147284in}{1.531222in}}%
\pgfpathlineto{\pgfqpoint{9.151890in}{1.530288in}}%
\pgfpathlineto{\pgfqpoint{9.156393in}{1.529387in}}%
\pgfpathlineto{\pgfqpoint{9.160792in}{1.528519in}}%
\pgfpathlineto{\pgfqpoint{9.165090in}{1.527682in}}%
\pgfpathlineto{\pgfqpoint{9.169287in}{1.526875in}}%
\pgfpathlineto{\pgfqpoint{9.173384in}{1.526099in}}%
\pgfpathlineto{\pgfqpoint{9.177381in}{1.525351in}}%
\pgfpathlineto{\pgfqpoint{9.181278in}{1.524632in}}%
\pgfpathlineto{\pgfqpoint{9.185077in}{1.523940in}}%
\pgfpathlineto{\pgfqpoint{9.188777in}{1.523275in}}%
\pgfpathlineto{\pgfqpoint{9.192379in}{1.522636in}}%
\pgfpathlineto{\pgfqpoint{9.195885in}{1.522023in}}%
\pgfpathlineto{\pgfqpoint{9.199293in}{1.521434in}}%
\pgfpathlineto{\pgfqpoint{9.202605in}{1.520871in}}%
\pgfpathlineto{\pgfqpoint{9.205821in}{1.520331in}}%
\pgfpathlineto{\pgfqpoint{9.208942in}{1.519814in}}%
\pgfpathlineto{\pgfqpoint{9.211967in}{1.519320in}}%
\pgfpathlineto{\pgfqpoint{9.214897in}{1.518848in}}%
\pgfpathlineto{\pgfqpoint{9.217734in}{1.518398in}}%
\pgfpathlineto{\pgfqpoint{9.220476in}{1.517970in}}%
\pgfpathlineto{\pgfqpoint{9.223125in}{1.517563in}}%
\pgfpathlineto{\pgfqpoint{9.225680in}{1.517176in}}%
\pgfpathlineto{\pgfqpoint{9.228144in}{1.516810in}}%
\pgfpathlineto{\pgfqpoint{9.230514in}{1.516463in}}%
\pgfpathlineto{\pgfqpoint{9.232794in}{1.516136in}}%
\pgfpathlineto{\pgfqpoint{9.234981in}{1.515828in}}%
\pgfpathlineto{\pgfqpoint{9.237078in}{1.515539in}}%
\pgfpathlineto{\pgfqpoint{9.239084in}{1.515268in}}%
\pgfpathlineto{\pgfqpoint{9.240999in}{1.515016in}}%
\pgfusepath{stroke}%
\end{pgfscope}%
\begin{pgfscope}%
\pgfpathrectangle{\pgfqpoint{8.282041in}{0.790446in}}{\pgfqpoint{1.897959in}{1.372727in}}%
\pgfusepath{clip}%
\pgfsetrectcap%
\pgfsetroundjoin%
\pgfsetlinewidth{1.505625pt}%
\definecolor{currentstroke}{rgb}{1.000000,0.000000,0.000000}%
\pgfsetstrokecolor{currentstroke}%
\pgfsetstrokeopacity{0.750000}%
\pgfsetdash{}{0pt}%
\pgfpathmoveto{\pgfqpoint{8.391651in}{1.996016in}}%
\pgfpathlineto{\pgfqpoint{8.421603in}{2.018488in}}%
\pgfpathlineto{\pgfqpoint{8.436279in}{2.027102in}}%
\pgfpathlineto{\pgfqpoint{8.465051in}{2.039730in}}%
\pgfpathlineto{\pgfqpoint{8.493066in}{2.048406in}}%
\pgfpathlineto{\pgfqpoint{8.520347in}{2.054647in}}%
\pgfpathlineto{\pgfqpoint{8.559945in}{2.061207in}}%
\pgfpathlineto{\pgfqpoint{8.610401in}{2.066910in}}%
\pgfpathlineto{\pgfqpoint{8.681432in}{2.072117in}}%
\pgfpathlineto{\pgfqpoint{8.778462in}{2.076559in}}%
\pgfpathlineto{\pgfqpoint{8.942831in}{2.081391in}}%
\pgfpathlineto{\pgfqpoint{9.212438in}{2.089327in}}%
\pgfpathlineto{\pgfqpoint{9.323357in}{2.094842in}}%
\pgfpathlineto{\pgfqpoint{9.380491in}{2.099628in}}%
\pgfpathlineto{\pgfqpoint{9.389283in}{2.100776in}}%
\pgfpathlineto{\pgfqpoint{9.389283in}{2.100776in}}%
\pgfusepath{stroke}%
\end{pgfscope}%
\begin{pgfscope}%
\pgfpathrectangle{\pgfqpoint{8.282041in}{0.790446in}}{\pgfqpoint{1.897959in}{1.372727in}}%
\pgfusepath{clip}%
\pgfsetrectcap%
\pgfsetroundjoin%
\pgfsetlinewidth{1.505625pt}%
\definecolor{currentstroke}{rgb}{0.000000,0.000000,1.000000}%
\pgfsetstrokecolor{currentstroke}%
\pgfsetstrokeopacity{0.750000}%
\pgfsetdash{}{0pt}%
\pgfpathmoveto{\pgfqpoint{8.391651in}{1.436066in}}%
\pgfpathlineto{\pgfqpoint{8.421603in}{1.455776in}}%
\pgfpathlineto{\pgfqpoint{8.436279in}{1.463325in}}%
\pgfpathlineto{\pgfqpoint{8.465051in}{1.474221in}}%
\pgfpathlineto{\pgfqpoint{8.493066in}{1.481517in}}%
\pgfpathlineto{\pgfqpoint{8.533720in}{1.488559in}}%
\pgfpathlineto{\pgfqpoint{8.572804in}{1.492855in}}%
\pgfpathlineto{\pgfqpoint{8.622616in}{1.496132in}}%
\pgfpathlineto{\pgfqpoint{8.681432in}{1.497865in}}%
\pgfpathlineto{\pgfqpoint{8.747322in}{1.497745in}}%
\pgfpathlineto{\pgfqpoint{8.818208in}{1.495431in}}%
\pgfpathlineto{\pgfqpoint{8.883187in}{1.491184in}}%
\pgfpathlineto{\pgfqpoint{8.942831in}{1.485170in}}%
\pgfpathlineto{\pgfqpoint{8.997581in}{1.477433in}}%
\pgfpathlineto{\pgfqpoint{9.047721in}{1.467961in}}%
\pgfpathlineto{\pgfqpoint{9.087239in}{1.458421in}}%
\pgfpathlineto{\pgfqpoint{9.123766in}{1.447508in}}%
\pgfpathlineto{\pgfqpoint{9.157474in}{1.435143in}}%
\pgfpathlineto{\pgfqpoint{9.188508in}{1.421222in}}%
\pgfpathlineto{\pgfqpoint{9.212438in}{1.408334in}}%
\pgfpathlineto{\pgfqpoint{9.234710in}{1.394173in}}%
\pgfpathlineto{\pgfqpoint{9.255403in}{1.378617in}}%
\pgfpathlineto{\pgfqpoint{9.274572in}{1.361516in}}%
\pgfpathlineto{\pgfqpoint{9.292261in}{1.342680in}}%
\pgfpathlineto{\pgfqpoint{9.308508in}{1.321872in}}%
\pgfpathlineto{\pgfqpoint{9.323357in}{1.298786in}}%
\pgfpathlineto{\pgfqpoint{9.336840in}{1.273023in}}%
\pgfpathlineto{\pgfqpoint{9.348976in}{1.244048in}}%
\pgfpathlineto{\pgfqpoint{9.359781in}{1.211121in}}%
\pgfpathlineto{\pgfqpoint{9.369287in}{1.173185in}}%
\pgfpathlineto{\pgfqpoint{9.377536in}{1.128643in}}%
\pgfpathlineto{\pgfqpoint{9.383250in}{1.086577in}}%
\pgfpathlineto{\pgfqpoint{9.388176in}{1.036527in}}%
\pgfpathlineto{\pgfqpoint{9.389283in}{1.022390in}}%
\pgfpathlineto{\pgfqpoint{9.389283in}{1.022390in}}%
\pgfusepath{stroke}%
\end{pgfscope}%
\begin{pgfscope}%
\pgfpathrectangle{\pgfqpoint{8.282041in}{0.790446in}}{\pgfqpoint{1.897959in}{1.372727in}}%
\pgfusepath{clip}%
\pgfsetrectcap%
\pgfsetroundjoin%
\pgfsetlinewidth{1.505625pt}%
\definecolor{currentstroke}{rgb}{0.000000,0.750000,0.750000}%
\pgfsetstrokecolor{currentstroke}%
\pgfsetstrokeopacity{0.750000}%
\pgfsetdash{}{0pt}%
\pgfpathmoveto{\pgfqpoint{8.391651in}{1.840019in}}%
\pgfpathlineto{\pgfqpoint{8.406727in}{1.806492in}}%
\pgfpathlineto{\pgfqpoint{8.436279in}{1.750572in}}%
\pgfpathlineto{\pgfqpoint{8.465051in}{1.702133in}}%
\pgfpathlineto{\pgfqpoint{8.493066in}{1.659994in}}%
\pgfpathlineto{\pgfqpoint{8.520347in}{1.623102in}}%
\pgfpathlineto{\pgfqpoint{8.546918in}{1.590586in}}%
\pgfpathlineto{\pgfqpoint{8.572804in}{1.561744in}}%
\pgfpathlineto{\pgfqpoint{8.610401in}{1.524166in}}%
\pgfpathlineto{\pgfqpoint{8.646587in}{1.492133in}}%
\pgfpathlineto{\pgfqpoint{8.681432in}{1.464553in}}%
\pgfpathlineto{\pgfqpoint{8.714993in}{1.440606in}}%
\pgfpathlineto{\pgfqpoint{8.757832in}{1.413256in}}%
\pgfpathlineto{\pgfqpoint{8.798583in}{1.390123in}}%
\pgfpathlineto{\pgfqpoint{8.837351in}{1.370376in}}%
\pgfpathlineto{\pgfqpoint{8.883187in}{1.349510in}}%
\pgfpathlineto{\pgfqpoint{8.934618in}{1.328895in}}%
\pgfpathlineto{\pgfqpoint{8.982419in}{1.312047in}}%
\pgfpathlineto{\pgfqpoint{9.033851in}{1.296074in}}%
\pgfpathlineto{\pgfqpoint{9.093530in}{1.279987in}}%
\pgfpathlineto{\pgfqpoint{9.152045in}{1.266494in}}%
\pgfpathlineto{\pgfqpoint{9.212438in}{1.254744in}}%
\pgfpathlineto{\pgfqpoint{9.274572in}{1.244928in}}%
\pgfpathlineto{\pgfqpoint{9.331609in}{1.238133in}}%
\pgfpathlineto{\pgfqpoint{9.375985in}{1.234946in}}%
\pgfpathlineto{\pgfqpoint{9.389283in}{1.234764in}}%
\pgfpathlineto{\pgfqpoint{9.389283in}{1.234764in}}%
\pgfusepath{stroke}%
\end{pgfscope}%
\begin{pgfscope}%
\pgfpathrectangle{\pgfqpoint{8.282041in}{0.790446in}}{\pgfqpoint{1.897959in}{1.372727in}}%
\pgfusepath{clip}%
\pgfsetrectcap%
\pgfsetroundjoin%
\pgfsetlinewidth{1.505625pt}%
\definecolor{currentstroke}{rgb}{1.000000,0.000000,0.000000}%
\pgfsetstrokecolor{currentstroke}%
\pgfsetstrokeopacity{0.750000}%
\pgfsetdash{}{0pt}%
\pgfpathmoveto{\pgfqpoint{8.504795in}{1.985954in}}%
\pgfpathlineto{\pgfqpoint{8.540288in}{2.003085in}}%
\pgfpathlineto{\pgfqpoint{8.574979in}{2.016558in}}%
\pgfpathlineto{\pgfqpoint{8.608867in}{2.026737in}}%
\pgfpathlineto{\pgfqpoint{8.641949in}{2.034628in}}%
\pgfpathlineto{\pgfqpoint{8.690054in}{2.043535in}}%
\pgfpathlineto{\pgfqpoint{8.751318in}{2.051866in}}%
\pgfpathlineto{\pgfqpoint{8.823277in}{2.058868in}}%
\pgfpathlineto{\pgfqpoint{8.916090in}{2.065132in}}%
\pgfpathlineto{\pgfqpoint{9.024088in}{2.070031in}}%
\pgfpathlineto{\pgfqpoint{9.171757in}{2.074397in}}%
\pgfpathlineto{\pgfqpoint{9.397769in}{2.078606in}}%
\pgfpathlineto{\pgfqpoint{10.010993in}{2.089039in}}%
\pgfpathlineto{\pgfqpoint{10.010993in}{2.089039in}}%
\pgfusepath{stroke}%
\end{pgfscope}%
\begin{pgfscope}%
\pgfpathrectangle{\pgfqpoint{8.282041in}{0.790446in}}{\pgfqpoint{1.897959in}{1.372727in}}%
\pgfusepath{clip}%
\pgfsetrectcap%
\pgfsetroundjoin%
\pgfsetlinewidth{1.505625pt}%
\definecolor{currentstroke}{rgb}{0.000000,0.000000,1.000000}%
\pgfsetstrokecolor{currentstroke}%
\pgfsetstrokeopacity{0.750000}%
\pgfsetdash{}{0pt}%
\pgfpathmoveto{\pgfqpoint{8.504795in}{1.242610in}}%
\pgfpathlineto{\pgfqpoint{8.557734in}{1.265543in}}%
\pgfpathlineto{\pgfqpoint{8.592024in}{1.276985in}}%
\pgfpathlineto{\pgfqpoint{8.625509in}{1.286065in}}%
\pgfpathlineto{\pgfqpoint{8.674222in}{1.296698in}}%
\pgfpathlineto{\pgfqpoint{8.736314in}{1.307284in}}%
\pgfpathlineto{\pgfqpoint{8.809285in}{1.316984in}}%
\pgfpathlineto{\pgfqpoint{8.890438in}{1.325445in}}%
\pgfpathlineto{\pgfqpoint{8.989357in}{1.333271in}}%
\pgfpathlineto{\pgfqpoint{9.090112in}{1.338791in}}%
\pgfpathlineto{\pgfqpoint{9.181499in}{1.341639in}}%
\pgfpathlineto{\pgfqpoint{9.273923in}{1.342195in}}%
\pgfpathlineto{\pgfqpoint{9.358292in}{1.340334in}}%
\pgfpathlineto{\pgfqpoint{9.435548in}{1.336269in}}%
\pgfpathlineto{\pgfqpoint{9.499568in}{1.330833in}}%
\pgfpathlineto{\pgfqpoint{9.558923in}{1.323768in}}%
\pgfpathlineto{\pgfqpoint{9.614015in}{1.315094in}}%
\pgfpathlineto{\pgfqpoint{9.665311in}{1.304800in}}%
\pgfpathlineto{\pgfqpoint{9.713338in}{1.292844in}}%
\pgfpathlineto{\pgfqpoint{9.753466in}{1.280763in}}%
\pgfpathlineto{\pgfqpoint{9.791193in}{1.267236in}}%
\pgfpathlineto{\pgfqpoint{9.826640in}{1.252162in}}%
\pgfpathlineto{\pgfqpoint{9.860040in}{1.235409in}}%
\pgfpathlineto{\pgfqpoint{9.887690in}{1.219240in}}%
\pgfpathlineto{\pgfqpoint{9.913897in}{1.201510in}}%
\pgfpathlineto{\pgfqpoint{9.938686in}{1.182035in}}%
\pgfpathlineto{\pgfqpoint{9.962112in}{1.160585in}}%
\pgfpathlineto{\pgfqpoint{9.984268in}{1.136864in}}%
\pgfpathlineto{\pgfqpoint{10.002303in}{1.114431in}}%
\pgfpathlineto{\pgfqpoint{10.010993in}{1.102387in}}%
\pgfpathlineto{\pgfqpoint{10.010993in}{1.102387in}}%
\pgfusepath{stroke}%
\end{pgfscope}%
\begin{pgfscope}%
\pgfpathrectangle{\pgfqpoint{8.282041in}{0.790446in}}{\pgfqpoint{1.897959in}{1.372727in}}%
\pgfusepath{clip}%
\pgfsetrectcap%
\pgfsetroundjoin%
\pgfsetlinewidth{1.505625pt}%
\definecolor{currentstroke}{rgb}{0.000000,0.750000,0.750000}%
\pgfsetstrokecolor{currentstroke}%
\pgfsetstrokeopacity{0.750000}%
\pgfsetdash{}{0pt}%
\pgfpathmoveto{\pgfqpoint{8.504795in}{1.878246in}}%
\pgfpathlineto{\pgfqpoint{8.522641in}{1.860009in}}%
\pgfpathlineto{\pgfqpoint{8.557734in}{1.828770in}}%
\pgfpathlineto{\pgfqpoint{8.592024in}{1.800803in}}%
\pgfpathlineto{\pgfqpoint{8.625509in}{1.775734in}}%
\pgfpathlineto{\pgfqpoint{8.674222in}{1.742800in}}%
\pgfpathlineto{\pgfqpoint{8.721102in}{1.714562in}}%
\pgfpathlineto{\pgfqpoint{8.766115in}{1.690177in}}%
\pgfpathlineto{\pgfqpoint{8.809285in}{1.668974in}}%
\pgfpathlineto{\pgfqpoint{8.864114in}{1.644742in}}%
\pgfpathlineto{\pgfqpoint{8.916090in}{1.624243in}}%
\pgfpathlineto{\pgfqpoint{8.977508in}{1.602758in}}%
\pgfpathlineto{\pgfqpoint{9.035403in}{1.584891in}}%
\pgfpathlineto{\pgfqpoint{9.100698in}{1.567146in}}%
\pgfpathlineto{\pgfqpoint{9.171757in}{1.550341in}}%
\pgfpathlineto{\pgfqpoint{9.247103in}{1.534988in}}%
\pgfpathlineto{\pgfqpoint{9.325437in}{1.521342in}}%
\pgfpathlineto{\pgfqpoint{9.413079in}{1.508439in}}%
\pgfpathlineto{\pgfqpoint{9.506384in}{1.496991in}}%
\pgfpathlineto{\pgfqpoint{9.608092in}{1.486793in}}%
\pgfpathlineto{\pgfqpoint{9.713338in}{1.478435in}}%
\pgfpathlineto{\pgfqpoint{9.826640in}{1.471676in}}%
\pgfpathlineto{\pgfqpoint{9.942114in}{1.467006in}}%
\pgfpathlineto{\pgfqpoint{10.010993in}{1.465332in}}%
\pgfpathlineto{\pgfqpoint{10.010993in}{1.465332in}}%
\pgfusepath{stroke}%
\end{pgfscope}%
\begin{pgfscope}%
\pgfpathrectangle{\pgfqpoint{8.282041in}{0.790446in}}{\pgfqpoint{1.897959in}{1.372727in}}%
\pgfusepath{clip}%
\pgfsetrectcap%
\pgfsetroundjoin%
\pgfsetlinewidth{1.505625pt}%
\definecolor{currentstroke}{rgb}{1.000000,0.000000,0.000000}%
\pgfsetstrokecolor{currentstroke}%
\pgfsetstrokeopacity{0.750000}%
\pgfsetdash{}{0pt}%
\pgfpathmoveto{\pgfqpoint{8.416676in}{1.926541in}}%
\pgfpathlineto{\pgfqpoint{8.436050in}{1.945318in}}%
\pgfpathlineto{\pgfqpoint{8.455246in}{1.963431in}}%
\pgfpathlineto{\pgfqpoint{8.474267in}{1.977943in}}%
\pgfpathlineto{\pgfqpoint{8.493116in}{1.989743in}}%
\pgfpathlineto{\pgfqpoint{8.511797in}{1.999456in}}%
\pgfpathlineto{\pgfqpoint{8.530312in}{2.007540in}}%
\pgfpathlineto{\pgfqpoint{8.548666in}{2.014335in}}%
\pgfpathlineto{\pgfqpoint{8.566862in}{2.020095in}}%
\pgfpathlineto{\pgfqpoint{8.584903in}{2.025013in}}%
\pgfpathlineto{\pgfqpoint{8.602792in}{2.029241in}}%
\pgfpathlineto{\pgfqpoint{8.620533in}{2.032896in}}%
\pgfpathlineto{\pgfqpoint{8.638130in}{2.036074in}}%
\pgfpathlineto{\pgfqpoint{8.655587in}{2.038849in}}%
\pgfpathlineto{\pgfqpoint{8.672907in}{2.041282in}}%
\pgfpathlineto{\pgfqpoint{8.690093in}{2.043421in}}%
\pgfpathlineto{\pgfqpoint{8.707149in}{2.045309in}}%
\pgfpathlineto{\pgfqpoint{8.724077in}{2.046978in}}%
\pgfpathlineto{\pgfqpoint{8.740880in}{2.048458in}}%
\pgfpathlineto{\pgfqpoint{8.757561in}{2.049772in}}%
\pgfpathlineto{\pgfqpoint{8.774122in}{2.050940in}}%
\pgfpathlineto{\pgfqpoint{8.790566in}{2.051978in}}%
\pgfpathlineto{\pgfqpoint{8.806895in}{2.052903in}}%
\pgfpathlineto{\pgfqpoint{8.823110in}{2.053726in}}%
\pgfpathlineto{\pgfqpoint{8.839216in}{2.054457in}}%
\pgfpathlineto{\pgfqpoint{8.855212in}{2.055107in}}%
\pgfpathlineto{\pgfqpoint{8.871103in}{2.055683in}}%
\pgfpathlineto{\pgfqpoint{8.886890in}{2.056192in}}%
\pgfpathlineto{\pgfqpoint{8.902577in}{2.056642in}}%
\pgfpathlineto{\pgfqpoint{8.918165in}{2.057036in}}%
\pgfpathlineto{\pgfqpoint{8.933658in}{2.057381in}}%
\pgfpathlineto{\pgfqpoint{8.949058in}{2.057681in}}%
\pgfpathlineto{\pgfqpoint{8.964368in}{2.057939in}}%
\pgfpathlineto{\pgfqpoint{8.979590in}{2.058159in}}%
\pgfpathlineto{\pgfqpoint{8.994727in}{2.058344in}}%
\pgfpathlineto{\pgfqpoint{9.009782in}{2.058497in}}%
\pgfpathlineto{\pgfqpoint{9.024757in}{2.058620in}}%
\pgfpathlineto{\pgfqpoint{9.039655in}{2.058715in}}%
\pgfpathlineto{\pgfqpoint{9.054477in}{2.058784in}}%
\pgfpathlineto{\pgfqpoint{9.069226in}{2.058830in}}%
\pgfpathlineto{\pgfqpoint{9.083903in}{2.058854in}}%
\pgfpathlineto{\pgfqpoint{9.098511in}{2.058857in}}%
\pgfpathlineto{\pgfqpoint{9.113049in}{2.058841in}}%
\pgfpathlineto{\pgfqpoint{9.127521in}{2.058807in}}%
\pgfpathlineto{\pgfqpoint{9.141927in}{2.058756in}}%
\pgfpathlineto{\pgfqpoint{9.156268in}{2.058690in}}%
\pgfpathlineto{\pgfqpoint{9.170545in}{2.058609in}}%
\pgfpathlineto{\pgfqpoint{9.184759in}{2.058514in}}%
\pgfpathlineto{\pgfqpoint{9.198911in}{2.058406in}}%
\pgfpathlineto{\pgfqpoint{9.213003in}{2.058286in}}%
\pgfpathlineto{\pgfqpoint{9.227035in}{2.058155in}}%
\pgfpathlineto{\pgfqpoint{9.241008in}{2.058013in}}%
\pgfpathlineto{\pgfqpoint{9.254923in}{2.057860in}}%
\pgfpathlineto{\pgfqpoint{9.268781in}{2.057698in}}%
\pgfpathlineto{\pgfqpoint{9.282584in}{2.057526in}}%
\pgfpathlineto{\pgfqpoint{9.296333in}{2.057346in}}%
\pgfpathlineto{\pgfqpoint{9.310029in}{2.057159in}}%
\pgfpathlineto{\pgfqpoint{9.323673in}{2.056963in}}%
\pgfpathlineto{\pgfqpoint{9.337267in}{2.056759in}}%
\pgfpathlineto{\pgfqpoint{9.350812in}{2.056549in}}%
\pgfpathlineto{\pgfqpoint{9.364310in}{2.056331in}}%
\pgfpathlineto{\pgfqpoint{9.377762in}{2.056108in}}%
\pgfpathlineto{\pgfqpoint{9.391168in}{2.055879in}}%
\pgfpathlineto{\pgfqpoint{9.404532in}{2.055644in}}%
\pgfpathlineto{\pgfqpoint{9.417853in}{2.055403in}}%
\pgfpathlineto{\pgfqpoint{9.431133in}{2.055157in}}%
\pgfpathlineto{\pgfqpoint{9.444373in}{2.054907in}}%
\pgfpathlineto{\pgfqpoint{9.457574in}{2.054651in}}%
\pgfpathlineto{\pgfqpoint{9.470737in}{2.054391in}}%
\pgfpathlineto{\pgfqpoint{9.483862in}{2.054127in}}%
\pgfpathlineto{\pgfqpoint{9.496952in}{2.053858in}}%
\pgfpathlineto{\pgfqpoint{9.510006in}{2.053586in}}%
\pgfpathlineto{\pgfqpoint{9.523026in}{2.053310in}}%
\pgfpathlineto{\pgfqpoint{9.536011in}{2.053031in}}%
\pgfpathlineto{\pgfqpoint{9.548963in}{2.052748in}}%
\pgfpathlineto{\pgfqpoint{9.561882in}{2.052461in}}%
\pgfpathlineto{\pgfqpoint{9.574769in}{2.052172in}}%
\pgfpathlineto{\pgfqpoint{9.587624in}{2.051880in}}%
\pgfpathlineto{\pgfqpoint{9.600448in}{2.051584in}}%
\pgfpathlineto{\pgfqpoint{9.613242in}{2.051286in}}%
\pgfpathlineto{\pgfqpoint{9.626006in}{2.050985in}}%
\pgfpathlineto{\pgfqpoint{9.638740in}{2.050682in}}%
\pgfpathlineto{\pgfqpoint{9.651445in}{2.050377in}}%
\pgfpathlineto{\pgfqpoint{9.664121in}{2.050068in}}%
\pgfpathlineto{\pgfqpoint{9.676770in}{2.049758in}}%
\pgfpathlineto{\pgfqpoint{9.689391in}{2.049446in}}%
\pgfpathlineto{\pgfqpoint{9.701986in}{2.049132in}}%
\pgfpathlineto{\pgfqpoint{9.714553in}{2.048816in}}%
\pgfpathlineto{\pgfqpoint{9.727095in}{2.048498in}}%
\pgfpathlineto{\pgfqpoint{9.739611in}{2.048178in}}%
\pgfpathlineto{\pgfqpoint{9.752102in}{2.047855in}}%
\pgfpathlineto{\pgfqpoint{9.764568in}{2.047532in}}%
\pgfpathlineto{\pgfqpoint{9.777009in}{2.047207in}}%
\pgfpathlineto{\pgfqpoint{9.789426in}{2.046881in}}%
\pgfpathlineto{\pgfqpoint{9.801818in}{2.046553in}}%
\pgfpathlineto{\pgfqpoint{9.814187in}{2.046224in}}%
\pgfpathlineto{\pgfqpoint{9.826533in}{2.045893in}}%
\pgfpathlineto{\pgfqpoint{9.838855in}{2.045560in}}%
\pgfpathlineto{\pgfqpoint{9.851154in}{2.045227in}}%
\pgfpathlineto{\pgfqpoint{9.863430in}{2.044893in}}%
\pgfpathlineto{\pgfqpoint{9.875684in}{2.044558in}}%
\pgfpathlineto{\pgfqpoint{9.887916in}{2.044221in}}%
\pgfpathlineto{\pgfqpoint{9.900125in}{2.043883in}}%
\pgfpathlineto{\pgfqpoint{9.912313in}{2.043544in}}%
\pgfpathlineto{\pgfqpoint{9.924479in}{2.043205in}}%
\pgfpathlineto{\pgfqpoint{9.936625in}{2.042864in}}%
\pgfpathlineto{\pgfqpoint{9.948749in}{2.042523in}}%
\pgfpathlineto{\pgfqpoint{9.960853in}{2.042181in}}%
\pgfpathlineto{\pgfqpoint{9.972938in}{2.041838in}}%
\pgfpathlineto{\pgfqpoint{9.985002in}{2.041494in}}%
\pgfpathlineto{\pgfqpoint{9.997047in}{2.041150in}}%
\pgfpathlineto{\pgfqpoint{10.009074in}{2.040804in}}%
\pgfpathlineto{\pgfqpoint{10.021082in}{2.040459in}}%
\pgfpathlineto{\pgfqpoint{10.033071in}{2.040113in}}%
\pgfpathlineto{\pgfqpoint{10.045043in}{2.039766in}}%
\pgfpathlineto{\pgfqpoint{10.056997in}{2.039418in}}%
\pgfpathlineto{\pgfqpoint{10.068934in}{2.039070in}}%
\pgfpathlineto{\pgfqpoint{10.080854in}{2.038721in}}%
\pgfpathlineto{\pgfqpoint{10.092757in}{2.038372in}}%
\pgfusepath{stroke}%
\end{pgfscope}%
\begin{pgfscope}%
\pgfpathrectangle{\pgfqpoint{8.282041in}{0.790446in}}{\pgfqpoint{1.897959in}{1.372727in}}%
\pgfusepath{clip}%
\pgfsetrectcap%
\pgfsetroundjoin%
\pgfsetlinewidth{1.505625pt}%
\definecolor{currentstroke}{rgb}{0.000000,0.000000,1.000000}%
\pgfsetstrokecolor{currentstroke}%
\pgfsetstrokeopacity{0.750000}%
\pgfsetdash{}{0pt}%
\pgfpathmoveto{\pgfqpoint{8.416676in}{1.349579in}}%
\pgfpathlineto{\pgfqpoint{8.436050in}{1.369152in}}%
\pgfpathlineto{\pgfqpoint{8.455246in}{1.388455in}}%
\pgfpathlineto{\pgfqpoint{8.474267in}{1.404458in}}%
\pgfpathlineto{\pgfqpoint{8.493116in}{1.417985in}}%
\pgfpathlineto{\pgfqpoint{8.511797in}{1.429615in}}%
\pgfpathlineto{\pgfqpoint{8.530312in}{1.439770in}}%
\pgfpathlineto{\pgfqpoint{8.548666in}{1.448766in}}%
\pgfpathlineto{\pgfqpoint{8.566862in}{1.456835in}}%
\pgfpathlineto{\pgfqpoint{8.584903in}{1.464155in}}%
\pgfpathlineto{\pgfqpoint{8.602792in}{1.470864in}}%
\pgfpathlineto{\pgfqpoint{8.620533in}{1.477070in}}%
\pgfpathlineto{\pgfqpoint{8.638130in}{1.482856in}}%
\pgfpathlineto{\pgfqpoint{8.655587in}{1.488291in}}%
\pgfpathlineto{\pgfqpoint{8.672907in}{1.493429in}}%
\pgfpathlineto{\pgfqpoint{8.690093in}{1.498312in}}%
\pgfpathlineto{\pgfqpoint{8.707149in}{1.502977in}}%
\pgfpathlineto{\pgfqpoint{8.724077in}{1.507452in}}%
\pgfpathlineto{\pgfqpoint{8.740880in}{1.511763in}}%
\pgfpathlineto{\pgfqpoint{8.757561in}{1.515928in}}%
\pgfpathlineto{\pgfqpoint{8.774122in}{1.519965in}}%
\pgfpathlineto{\pgfqpoint{8.790566in}{1.523887in}}%
\pgfpathlineto{\pgfqpoint{8.806895in}{1.527707in}}%
\pgfpathlineto{\pgfqpoint{8.823110in}{1.531436in}}%
\pgfpathlineto{\pgfqpoint{8.839216in}{1.535079in}}%
\pgfpathlineto{\pgfqpoint{8.855212in}{1.538646in}}%
\pgfpathlineto{\pgfqpoint{8.871103in}{1.542143in}}%
\pgfpathlineto{\pgfqpoint{8.886890in}{1.545574in}}%
\pgfpathlineto{\pgfqpoint{8.902577in}{1.548945in}}%
\pgfpathlineto{\pgfqpoint{8.918165in}{1.552260in}}%
\pgfpathlineto{\pgfqpoint{8.933658in}{1.555523in}}%
\pgfpathlineto{\pgfqpoint{8.949058in}{1.558735in}}%
\pgfpathlineto{\pgfqpoint{8.964368in}{1.561901in}}%
\pgfpathlineto{\pgfqpoint{8.979590in}{1.565022in}}%
\pgfpathlineto{\pgfqpoint{8.994727in}{1.568101in}}%
\pgfpathlineto{\pgfqpoint{9.009782in}{1.571139in}}%
\pgfpathlineto{\pgfqpoint{9.024757in}{1.574139in}}%
\pgfpathlineto{\pgfqpoint{9.039655in}{1.577101in}}%
\pgfpathlineto{\pgfqpoint{9.054477in}{1.580026in}}%
\pgfpathlineto{\pgfqpoint{9.069226in}{1.582917in}}%
\pgfpathlineto{\pgfqpoint{9.083903in}{1.585775in}}%
\pgfpathlineto{\pgfqpoint{9.098511in}{1.588599in}}%
\pgfpathlineto{\pgfqpoint{9.113049in}{1.591391in}}%
\pgfpathlineto{\pgfqpoint{9.127521in}{1.594153in}}%
\pgfpathlineto{\pgfqpoint{9.141927in}{1.596885in}}%
\pgfpathlineto{\pgfqpoint{9.156268in}{1.599587in}}%
\pgfpathlineto{\pgfqpoint{9.170545in}{1.602260in}}%
\pgfpathlineto{\pgfqpoint{9.184759in}{1.604905in}}%
\pgfpathlineto{\pgfqpoint{9.198911in}{1.607522in}}%
\pgfpathlineto{\pgfqpoint{9.213003in}{1.610112in}}%
\pgfpathlineto{\pgfqpoint{9.227035in}{1.612676in}}%
\pgfpathlineto{\pgfqpoint{9.241008in}{1.615214in}}%
\pgfpathlineto{\pgfqpoint{9.254923in}{1.617725in}}%
\pgfpathlineto{\pgfqpoint{9.268781in}{1.620212in}}%
\pgfpathlineto{\pgfqpoint{9.282584in}{1.622674in}}%
\pgfpathlineto{\pgfqpoint{9.296333in}{1.625112in}}%
\pgfpathlineto{\pgfqpoint{9.310029in}{1.627526in}}%
\pgfpathlineto{\pgfqpoint{9.323673in}{1.629916in}}%
\pgfpathlineto{\pgfqpoint{9.337267in}{1.632282in}}%
\pgfpathlineto{\pgfqpoint{9.350812in}{1.634625in}}%
\pgfpathlineto{\pgfqpoint{9.364310in}{1.636946in}}%
\pgfpathlineto{\pgfqpoint{9.377762in}{1.639245in}}%
\pgfpathlineto{\pgfqpoint{9.391168in}{1.641522in}}%
\pgfpathlineto{\pgfqpoint{9.404532in}{1.643777in}}%
\pgfpathlineto{\pgfqpoint{9.417853in}{1.646010in}}%
\pgfpathlineto{\pgfqpoint{9.431133in}{1.648223in}}%
\pgfpathlineto{\pgfqpoint{9.444373in}{1.650415in}}%
\pgfpathlineto{\pgfqpoint{9.457574in}{1.652586in}}%
\pgfpathlineto{\pgfqpoint{9.470737in}{1.654737in}}%
\pgfpathlineto{\pgfqpoint{9.483862in}{1.656867in}}%
\pgfpathlineto{\pgfqpoint{9.496952in}{1.658978in}}%
\pgfpathlineto{\pgfqpoint{9.510006in}{1.661070in}}%
\pgfpathlineto{\pgfqpoint{9.523026in}{1.663143in}}%
\pgfpathlineto{\pgfqpoint{9.536011in}{1.665196in}}%
\pgfpathlineto{\pgfqpoint{9.548963in}{1.667230in}}%
\pgfpathlineto{\pgfqpoint{9.561882in}{1.669246in}}%
\pgfpathlineto{\pgfqpoint{9.574769in}{1.671244in}}%
\pgfpathlineto{\pgfqpoint{9.587624in}{1.673224in}}%
\pgfpathlineto{\pgfqpoint{9.600448in}{1.675186in}}%
\pgfpathlineto{\pgfqpoint{9.613242in}{1.677130in}}%
\pgfpathlineto{\pgfqpoint{9.626006in}{1.679057in}}%
\pgfpathlineto{\pgfqpoint{9.638740in}{1.680966in}}%
\pgfpathlineto{\pgfqpoint{9.651445in}{1.682859in}}%
\pgfpathlineto{\pgfqpoint{9.664121in}{1.684735in}}%
\pgfpathlineto{\pgfqpoint{9.676770in}{1.686594in}}%
\pgfpathlineto{\pgfqpoint{9.689391in}{1.688437in}}%
\pgfpathlineto{\pgfqpoint{9.701986in}{1.690265in}}%
\pgfpathlineto{\pgfqpoint{9.714553in}{1.692076in}}%
\pgfpathlineto{\pgfqpoint{9.727095in}{1.693871in}}%
\pgfpathlineto{\pgfqpoint{9.739611in}{1.695650in}}%
\pgfpathlineto{\pgfqpoint{9.752102in}{1.697413in}}%
\pgfpathlineto{\pgfqpoint{9.764568in}{1.699161in}}%
\pgfpathlineto{\pgfqpoint{9.777009in}{1.700895in}}%
\pgfpathlineto{\pgfqpoint{9.789426in}{1.702614in}}%
\pgfpathlineto{\pgfqpoint{9.801818in}{1.704318in}}%
\pgfpathlineto{\pgfqpoint{9.814187in}{1.706008in}}%
\pgfpathlineto{\pgfqpoint{9.826533in}{1.707683in}}%
\pgfpathlineto{\pgfqpoint{9.838855in}{1.709344in}}%
\pgfpathlineto{\pgfqpoint{9.851154in}{1.710991in}}%
\pgfpathlineto{\pgfqpoint{9.863430in}{1.712624in}}%
\pgfpathlineto{\pgfqpoint{9.875684in}{1.714244in}}%
\pgfpathlineto{\pgfqpoint{9.887916in}{1.715850in}}%
\pgfpathlineto{\pgfqpoint{9.900125in}{1.717442in}}%
\pgfpathlineto{\pgfqpoint{9.912313in}{1.719021in}}%
\pgfpathlineto{\pgfqpoint{9.924479in}{1.720587in}}%
\pgfpathlineto{\pgfqpoint{9.936625in}{1.722141in}}%
\pgfpathlineto{\pgfqpoint{9.948749in}{1.723681in}}%
\pgfpathlineto{\pgfqpoint{9.960853in}{1.725210in}}%
\pgfpathlineto{\pgfqpoint{9.972938in}{1.726725in}}%
\pgfpathlineto{\pgfqpoint{9.985002in}{1.728228in}}%
\pgfpathlineto{\pgfqpoint{9.997047in}{1.729719in}}%
\pgfpathlineto{\pgfqpoint{10.009074in}{1.731198in}}%
\pgfpathlineto{\pgfqpoint{10.021082in}{1.732665in}}%
\pgfpathlineto{\pgfqpoint{10.033071in}{1.734120in}}%
\pgfpathlineto{\pgfqpoint{10.045043in}{1.735564in}}%
\pgfpathlineto{\pgfqpoint{10.056997in}{1.736996in}}%
\pgfpathlineto{\pgfqpoint{10.068934in}{1.738417in}}%
\pgfpathlineto{\pgfqpoint{10.080854in}{1.739826in}}%
\pgfpathlineto{\pgfqpoint{10.092757in}{1.741224in}}%
\pgfusepath{stroke}%
\end{pgfscope}%
\begin{pgfscope}%
\pgfpathrectangle{\pgfqpoint{8.282041in}{0.790446in}}{\pgfqpoint{1.897959in}{1.372727in}}%
\pgfusepath{clip}%
\pgfsetrectcap%
\pgfsetroundjoin%
\pgfsetlinewidth{1.505625pt}%
\definecolor{currentstroke}{rgb}{0.000000,0.750000,0.750000}%
\pgfsetstrokecolor{currentstroke}%
\pgfsetstrokeopacity{0.750000}%
\pgfsetdash{}{0pt}%
\pgfpathmoveto{\pgfqpoint{8.416676in}{1.945381in}}%
\pgfpathlineto{\pgfqpoint{8.436050in}{1.916587in}}%
\pgfpathlineto{\pgfqpoint{8.455246in}{1.892942in}}%
\pgfpathlineto{\pgfqpoint{8.474267in}{1.870367in}}%
\pgfpathlineto{\pgfqpoint{8.493116in}{1.848915in}}%
\pgfpathlineto{\pgfqpoint{8.511797in}{1.828583in}}%
\pgfpathlineto{\pgfqpoint{8.530312in}{1.809341in}}%
\pgfpathlineto{\pgfqpoint{8.548666in}{1.791145in}}%
\pgfpathlineto{\pgfqpoint{8.566862in}{1.773941in}}%
\pgfpathlineto{\pgfqpoint{8.584903in}{1.757670in}}%
\pgfpathlineto{\pgfqpoint{8.602792in}{1.742275in}}%
\pgfpathlineto{\pgfqpoint{8.620533in}{1.727699in}}%
\pgfpathlineto{\pgfqpoint{8.638130in}{1.713888in}}%
\pgfpathlineto{\pgfqpoint{8.655587in}{1.700793in}}%
\pgfpathlineto{\pgfqpoint{8.672907in}{1.688364in}}%
\pgfpathlineto{\pgfqpoint{8.690093in}{1.676558in}}%
\pgfpathlineto{\pgfqpoint{8.707149in}{1.665333in}}%
\pgfpathlineto{\pgfqpoint{8.724077in}{1.654651in}}%
\pgfpathlineto{\pgfqpoint{8.740880in}{1.644478in}}%
\pgfpathlineto{\pgfqpoint{8.757561in}{1.634779in}}%
\pgfpathlineto{\pgfqpoint{8.774122in}{1.625526in}}%
\pgfpathlineto{\pgfqpoint{8.790566in}{1.616690in}}%
\pgfpathlineto{\pgfqpoint{8.806895in}{1.608246in}}%
\pgfpathlineto{\pgfqpoint{8.823110in}{1.600169in}}%
\pgfpathlineto{\pgfqpoint{8.839216in}{1.592438in}}%
\pgfpathlineto{\pgfqpoint{8.855212in}{1.585031in}}%
\pgfpathlineto{\pgfqpoint{8.871103in}{1.577930in}}%
\pgfpathlineto{\pgfqpoint{8.886890in}{1.571118in}}%
\pgfpathlineto{\pgfqpoint{8.902577in}{1.564578in}}%
\pgfpathlineto{\pgfqpoint{8.918165in}{1.558294in}}%
\pgfpathlineto{\pgfqpoint{8.933658in}{1.552253in}}%
\pgfpathlineto{\pgfqpoint{8.949058in}{1.546441in}}%
\pgfpathlineto{\pgfqpoint{8.964368in}{1.540847in}}%
\pgfpathlineto{\pgfqpoint{8.979590in}{1.535457in}}%
\pgfpathlineto{\pgfqpoint{8.994727in}{1.530262in}}%
\pgfpathlineto{\pgfqpoint{9.009782in}{1.525252in}}%
\pgfpathlineto{\pgfqpoint{9.024757in}{1.520418in}}%
\pgfpathlineto{\pgfqpoint{9.039655in}{1.515750in}}%
\pgfpathlineto{\pgfqpoint{9.054477in}{1.511239in}}%
\pgfpathlineto{\pgfqpoint{9.069226in}{1.506878in}}%
\pgfpathlineto{\pgfqpoint{9.083903in}{1.502662in}}%
\pgfpathlineto{\pgfqpoint{9.098511in}{1.498580in}}%
\pgfpathlineto{\pgfqpoint{9.113049in}{1.494629in}}%
\pgfpathlineto{\pgfqpoint{9.127521in}{1.490801in}}%
\pgfpathlineto{\pgfqpoint{9.141927in}{1.487091in}}%
\pgfpathlineto{\pgfqpoint{9.156268in}{1.483494in}}%
\pgfpathlineto{\pgfqpoint{9.170545in}{1.480004in}}%
\pgfpathlineto{\pgfqpoint{9.184759in}{1.476617in}}%
\pgfpathlineto{\pgfqpoint{9.198911in}{1.473327in}}%
\pgfpathlineto{\pgfqpoint{9.213003in}{1.470132in}}%
\pgfpathlineto{\pgfqpoint{9.227035in}{1.467026in}}%
\pgfpathlineto{\pgfqpoint{9.241008in}{1.464006in}}%
\pgfpathlineto{\pgfqpoint{9.254923in}{1.461069in}}%
\pgfpathlineto{\pgfqpoint{9.268781in}{1.458210in}}%
\pgfpathlineto{\pgfqpoint{9.282584in}{1.455427in}}%
\pgfpathlineto{\pgfqpoint{9.296333in}{1.452716in}}%
\pgfpathlineto{\pgfqpoint{9.310029in}{1.450075in}}%
\pgfpathlineto{\pgfqpoint{9.323673in}{1.447501in}}%
\pgfpathlineto{\pgfqpoint{9.337267in}{1.444991in}}%
\pgfpathlineto{\pgfqpoint{9.350812in}{1.442541in}}%
\pgfpathlineto{\pgfqpoint{9.364310in}{1.440152in}}%
\pgfpathlineto{\pgfqpoint{9.377762in}{1.437819in}}%
\pgfpathlineto{\pgfqpoint{9.391168in}{1.435541in}}%
\pgfpathlineto{\pgfqpoint{9.404532in}{1.433316in}}%
\pgfpathlineto{\pgfqpoint{9.417853in}{1.431140in}}%
\pgfpathlineto{\pgfqpoint{9.431133in}{1.429014in}}%
\pgfpathlineto{\pgfqpoint{9.444373in}{1.426936in}}%
\pgfpathlineto{\pgfqpoint{9.457574in}{1.424902in}}%
\pgfpathlineto{\pgfqpoint{9.470737in}{1.422912in}}%
\pgfpathlineto{\pgfqpoint{9.483862in}{1.420964in}}%
\pgfpathlineto{\pgfqpoint{9.496952in}{1.419057in}}%
\pgfpathlineto{\pgfqpoint{9.510006in}{1.417189in}}%
\pgfpathlineto{\pgfqpoint{9.523026in}{1.415359in}}%
\pgfpathlineto{\pgfqpoint{9.536011in}{1.413565in}}%
\pgfpathlineto{\pgfqpoint{9.548963in}{1.411806in}}%
\pgfpathlineto{\pgfqpoint{9.561882in}{1.410082in}}%
\pgfpathlineto{\pgfqpoint{9.574769in}{1.408391in}}%
\pgfpathlineto{\pgfqpoint{9.587624in}{1.406732in}}%
\pgfpathlineto{\pgfqpoint{9.600448in}{1.405103in}}%
\pgfpathlineto{\pgfqpoint{9.613242in}{1.403504in}}%
\pgfpathlineto{\pgfqpoint{9.626006in}{1.401934in}}%
\pgfpathlineto{\pgfqpoint{9.638740in}{1.400392in}}%
\pgfpathlineto{\pgfqpoint{9.651445in}{1.398878in}}%
\pgfpathlineto{\pgfqpoint{9.664121in}{1.397389in}}%
\pgfpathlineto{\pgfqpoint{9.676770in}{1.395926in}}%
\pgfpathlineto{\pgfqpoint{9.689391in}{1.394488in}}%
\pgfpathlineto{\pgfqpoint{9.701986in}{1.393073in}}%
\pgfpathlineto{\pgfqpoint{9.714553in}{1.391682in}}%
\pgfpathlineto{\pgfqpoint{9.727095in}{1.390313in}}%
\pgfpathlineto{\pgfqpoint{9.739611in}{1.388965in}}%
\pgfpathlineto{\pgfqpoint{9.752102in}{1.387638in}}%
\pgfpathlineto{\pgfqpoint{9.764568in}{1.386332in}}%
\pgfpathlineto{\pgfqpoint{9.777009in}{1.385047in}}%
\pgfpathlineto{\pgfqpoint{9.789426in}{1.383781in}}%
\pgfpathlineto{\pgfqpoint{9.801818in}{1.382533in}}%
\pgfpathlineto{\pgfqpoint{9.814187in}{1.381304in}}%
\pgfpathlineto{\pgfqpoint{9.826533in}{1.380092in}}%
\pgfpathlineto{\pgfqpoint{9.838855in}{1.378897in}}%
\pgfpathlineto{\pgfqpoint{9.851154in}{1.377720in}}%
\pgfpathlineto{\pgfqpoint{9.863430in}{1.376559in}}%
\pgfpathlineto{\pgfqpoint{9.875684in}{1.375413in}}%
\pgfpathlineto{\pgfqpoint{9.887916in}{1.374284in}}%
\pgfpathlineto{\pgfqpoint{9.900125in}{1.373169in}}%
\pgfpathlineto{\pgfqpoint{9.912313in}{1.372068in}}%
\pgfpathlineto{\pgfqpoint{9.924479in}{1.370982in}}%
\pgfpathlineto{\pgfqpoint{9.936625in}{1.369910in}}%
\pgfpathlineto{\pgfqpoint{9.948749in}{1.368852in}}%
\pgfpathlineto{\pgfqpoint{9.960853in}{1.367806in}}%
\pgfpathlineto{\pgfqpoint{9.972938in}{1.366774in}}%
\pgfpathlineto{\pgfqpoint{9.985002in}{1.365754in}}%
\pgfpathlineto{\pgfqpoint{9.997047in}{1.364746in}}%
\pgfpathlineto{\pgfqpoint{10.009074in}{1.363750in}}%
\pgfpathlineto{\pgfqpoint{10.021082in}{1.362766in}}%
\pgfpathlineto{\pgfqpoint{10.033071in}{1.361793in}}%
\pgfpathlineto{\pgfqpoint{10.045043in}{1.360831in}}%
\pgfpathlineto{\pgfqpoint{10.056997in}{1.359880in}}%
\pgfpathlineto{\pgfqpoint{10.068934in}{1.358939in}}%
\pgfpathlineto{\pgfqpoint{10.080854in}{1.358009in}}%
\pgfpathlineto{\pgfqpoint{10.092757in}{1.357089in}}%
\pgfusepath{stroke}%
\end{pgfscope}%
\begin{pgfscope}%
\pgfpathrectangle{\pgfqpoint{8.282041in}{0.790446in}}{\pgfqpoint{1.897959in}{1.372727in}}%
\pgfusepath{clip}%
\pgfsetrectcap%
\pgfsetroundjoin%
\pgfsetlinewidth{1.505625pt}%
\definecolor{currentstroke}{rgb}{1.000000,0.000000,0.000000}%
\pgfsetstrokecolor{currentstroke}%
\pgfsetstrokeopacity{0.750000}%
\pgfsetdash{}{0pt}%
\pgfpathmoveto{\pgfqpoint{8.523435in}{1.990240in}}%
\pgfpathlineto{\pgfqpoint{8.543513in}{1.997571in}}%
\pgfpathlineto{\pgfqpoint{8.563540in}{2.004647in}}%
\pgfpathlineto{\pgfqpoint{8.583512in}{2.010561in}}%
\pgfpathlineto{\pgfqpoint{8.603426in}{2.015544in}}%
\pgfpathlineto{\pgfqpoint{8.623279in}{2.019765in}}%
\pgfpathlineto{\pgfqpoint{8.643070in}{2.023358in}}%
\pgfpathlineto{\pgfqpoint{8.662795in}{2.026426in}}%
\pgfpathlineto{\pgfqpoint{8.682451in}{2.029056in}}%
\pgfpathlineto{\pgfqpoint{8.702036in}{2.031312in}}%
\pgfpathlineto{\pgfqpoint{8.721547in}{2.033250in}}%
\pgfpathlineto{\pgfqpoint{8.740981in}{2.034915in}}%
\pgfpathlineto{\pgfqpoint{8.760335in}{2.036345in}}%
\pgfpathlineto{\pgfqpoint{8.779604in}{2.037569in}}%
\pgfpathlineto{\pgfqpoint{8.798784in}{2.038614in}}%
\pgfpathlineto{\pgfqpoint{8.817875in}{2.039500in}}%
\pgfpathlineto{\pgfqpoint{8.836873in}{2.040247in}}%
\pgfpathlineto{\pgfqpoint{8.855777in}{2.040869in}}%
\pgfpathlineto{\pgfqpoint{8.874586in}{2.041381in}}%
\pgfpathlineto{\pgfqpoint{8.893298in}{2.041793in}}%
\pgfpathlineto{\pgfqpoint{8.911913in}{2.042116in}}%
\pgfpathlineto{\pgfqpoint{8.930432in}{2.042359in}}%
\pgfpathlineto{\pgfqpoint{8.948855in}{2.042529in}}%
\pgfpathlineto{\pgfqpoint{8.967183in}{2.042632in}}%
\pgfpathlineto{\pgfqpoint{8.985417in}{2.042675in}}%
\pgfpathlineto{\pgfqpoint{9.003560in}{2.042662in}}%
\pgfpathlineto{\pgfqpoint{9.021614in}{2.042599in}}%
\pgfpathlineto{\pgfqpoint{9.039581in}{2.042489in}}%
\pgfpathlineto{\pgfqpoint{9.057464in}{2.042335in}}%
\pgfpathlineto{\pgfqpoint{9.075266in}{2.042142in}}%
\pgfpathlineto{\pgfqpoint{9.092989in}{2.041911in}}%
\pgfpathlineto{\pgfqpoint{9.110637in}{2.041647in}}%
\pgfpathlineto{\pgfqpoint{9.128213in}{2.041350in}}%
\pgfpathlineto{\pgfqpoint{9.145719in}{2.041024in}}%
\pgfpathlineto{\pgfqpoint{9.163158in}{2.040670in}}%
\pgfpathlineto{\pgfqpoint{9.180533in}{2.040289in}}%
\pgfpathlineto{\pgfqpoint{9.197848in}{2.039885in}}%
\pgfpathlineto{\pgfqpoint{9.215105in}{2.039457in}}%
\pgfpathlineto{\pgfqpoint{9.232306in}{2.039008in}}%
\pgfpathlineto{\pgfqpoint{9.249454in}{2.038538in}}%
\pgfpathlineto{\pgfqpoint{9.266551in}{2.038049in}}%
\pgfpathlineto{\pgfqpoint{9.283598in}{2.037542in}}%
\pgfpathlineto{\pgfqpoint{9.300599in}{2.037019in}}%
\pgfpathlineto{\pgfqpoint{9.317555in}{2.036479in}}%
\pgfpathlineto{\pgfqpoint{9.334468in}{2.035923in}}%
\pgfpathlineto{\pgfqpoint{9.351339in}{2.035353in}}%
\pgfpathlineto{\pgfqpoint{9.368170in}{2.034768in}}%
\pgfpathlineto{\pgfqpoint{9.384961in}{2.034171in}}%
\pgfpathlineto{\pgfqpoint{9.401715in}{2.033561in}}%
\pgfpathlineto{\pgfqpoint{9.418432in}{2.032939in}}%
\pgfpathlineto{\pgfqpoint{9.435113in}{2.032305in}}%
\pgfpathlineto{\pgfqpoint{9.451761in}{2.031661in}}%
\pgfpathlineto{\pgfqpoint{9.468375in}{2.031006in}}%
\pgfpathlineto{\pgfqpoint{9.484958in}{2.030342in}}%
\pgfpathlineto{\pgfqpoint{9.501509in}{2.029667in}}%
\pgfpathlineto{\pgfqpoint{9.518029in}{2.028984in}}%
\pgfpathlineto{\pgfqpoint{9.534518in}{2.028292in}}%
\pgfpathlineto{\pgfqpoint{9.550978in}{2.027592in}}%
\pgfpathlineto{\pgfqpoint{9.567408in}{2.026884in}}%
\pgfpathlineto{\pgfqpoint{9.583810in}{2.026168in}}%
\pgfpathlineto{\pgfqpoint{9.600182in}{2.025445in}}%
\pgfpathlineto{\pgfqpoint{9.616527in}{2.024715in}}%
\pgfpathlineto{\pgfqpoint{9.632842in}{2.023978in}}%
\pgfpathlineto{\pgfqpoint{9.649129in}{2.023235in}}%
\pgfpathlineto{\pgfqpoint{9.665386in}{2.022486in}}%
\pgfpathlineto{\pgfqpoint{9.681613in}{2.021731in}}%
\pgfpathlineto{\pgfqpoint{9.697812in}{2.020969in}}%
\pgfpathlineto{\pgfqpoint{9.713981in}{2.020203in}}%
\pgfpathlineto{\pgfqpoint{9.730120in}{2.019431in}}%
\pgfpathlineto{\pgfqpoint{9.746231in}{2.018655in}}%
\pgfpathlineto{\pgfqpoint{9.762311in}{2.017874in}}%
\pgfpathlineto{\pgfqpoint{9.778361in}{2.017088in}}%
\pgfpathlineto{\pgfqpoint{9.794382in}{2.016296in}}%
\pgfpathlineto{\pgfqpoint{9.810374in}{2.015500in}}%
\pgfpathlineto{\pgfqpoint{9.826337in}{2.014701in}}%
\pgfpathlineto{\pgfqpoint{9.842273in}{2.013898in}}%
\pgfpathlineto{\pgfqpoint{9.858181in}{2.013092in}}%
\pgfpathlineto{\pgfqpoint{9.874063in}{2.012281in}}%
\pgfpathlineto{\pgfqpoint{9.889917in}{2.011466in}}%
\pgfpathlineto{\pgfqpoint{9.905745in}{2.010648in}}%
\pgfpathlineto{\pgfqpoint{9.921548in}{2.009827in}}%
\pgfpathlineto{\pgfqpoint{9.937324in}{2.009004in}}%
\pgfpathlineto{\pgfqpoint{9.953077in}{2.008177in}}%
\pgfpathlineto{\pgfqpoint{9.968804in}{2.007348in}}%
\pgfpathlineto{\pgfqpoint{9.984508in}{2.006515in}}%
\pgfpathlineto{\pgfqpoint{10.000186in}{2.005679in}}%
\pgfpathlineto{\pgfqpoint{10.015840in}{2.004842in}}%
\pgfpathlineto{\pgfqpoint{10.031468in}{2.004002in}}%
\pgfpathlineto{\pgfqpoint{10.047071in}{2.003160in}}%
\pgfpathlineto{\pgfqpoint{10.062648in}{2.002316in}}%
\pgfpathlineto{\pgfqpoint{10.078201in}{2.001469in}}%
\pgfpathlineto{\pgfqpoint{10.093729in}{2.000620in}}%
\pgfusepath{stroke}%
\end{pgfscope}%
\begin{pgfscope}%
\pgfpathrectangle{\pgfqpoint{8.282041in}{0.790446in}}{\pgfqpoint{1.897959in}{1.372727in}}%
\pgfusepath{clip}%
\pgfsetrectcap%
\pgfsetroundjoin%
\pgfsetlinewidth{1.505625pt}%
\definecolor{currentstroke}{rgb}{0.000000,0.000000,1.000000}%
\pgfsetstrokecolor{currentstroke}%
\pgfsetstrokeopacity{0.750000}%
\pgfsetdash{}{0pt}%
\pgfpathmoveto{\pgfqpoint{8.523435in}{1.502193in}}%
\pgfpathlineto{\pgfqpoint{8.543513in}{1.512919in}}%
\pgfpathlineto{\pgfqpoint{8.563540in}{1.523516in}}%
\pgfpathlineto{\pgfqpoint{8.583512in}{1.533059in}}%
\pgfpathlineto{\pgfqpoint{8.603426in}{1.541762in}}%
\pgfpathlineto{\pgfqpoint{8.623279in}{1.549783in}}%
\pgfpathlineto{\pgfqpoint{8.643070in}{1.557241in}}%
\pgfpathlineto{\pgfqpoint{8.662795in}{1.564234in}}%
\pgfpathlineto{\pgfqpoint{8.682451in}{1.570837in}}%
\pgfpathlineto{\pgfqpoint{8.702036in}{1.577109in}}%
\pgfpathlineto{\pgfqpoint{8.721547in}{1.583100in}}%
\pgfpathlineto{\pgfqpoint{8.740981in}{1.588849in}}%
\pgfpathlineto{\pgfqpoint{8.760335in}{1.594388in}}%
\pgfpathlineto{\pgfqpoint{8.779604in}{1.599743in}}%
\pgfpathlineto{\pgfqpoint{8.798784in}{1.604936in}}%
\pgfpathlineto{\pgfqpoint{8.817875in}{1.609984in}}%
\pgfpathlineto{\pgfqpoint{8.836873in}{1.614902in}}%
\pgfpathlineto{\pgfqpoint{8.855777in}{1.619702in}}%
\pgfpathlineto{\pgfqpoint{8.874586in}{1.624396in}}%
\pgfpathlineto{\pgfqpoint{8.893298in}{1.628992in}}%
\pgfpathlineto{\pgfqpoint{8.911913in}{1.633498in}}%
\pgfpathlineto{\pgfqpoint{8.930432in}{1.637920in}}%
\pgfpathlineto{\pgfqpoint{8.948855in}{1.642264in}}%
\pgfpathlineto{\pgfqpoint{8.967183in}{1.646535in}}%
\pgfpathlineto{\pgfqpoint{8.985417in}{1.650736in}}%
\pgfpathlineto{\pgfqpoint{9.003560in}{1.654871in}}%
\pgfpathlineto{\pgfqpoint{9.021614in}{1.658944in}}%
\pgfpathlineto{\pgfqpoint{9.039581in}{1.662956in}}%
\pgfpathlineto{\pgfqpoint{9.057464in}{1.666911in}}%
\pgfpathlineto{\pgfqpoint{9.075266in}{1.670810in}}%
\pgfpathlineto{\pgfqpoint{9.092989in}{1.674656in}}%
\pgfpathlineto{\pgfqpoint{9.110637in}{1.678450in}}%
\pgfpathlineto{\pgfqpoint{9.128213in}{1.682194in}}%
\pgfpathlineto{\pgfqpoint{9.145719in}{1.685888in}}%
\pgfpathlineto{\pgfqpoint{9.163158in}{1.689536in}}%
\pgfpathlineto{\pgfqpoint{9.180533in}{1.693136in}}%
\pgfpathlineto{\pgfqpoint{9.197848in}{1.696691in}}%
\pgfpathlineto{\pgfqpoint{9.215105in}{1.700202in}}%
\pgfpathlineto{\pgfqpoint{9.232306in}{1.703669in}}%
\pgfpathlineto{\pgfqpoint{9.249454in}{1.707094in}}%
\pgfpathlineto{\pgfqpoint{9.266551in}{1.710477in}}%
\pgfpathlineto{\pgfqpoint{9.283598in}{1.713818in}}%
\pgfpathlineto{\pgfqpoint{9.300599in}{1.717119in}}%
\pgfpathlineto{\pgfqpoint{9.317555in}{1.720381in}}%
\pgfpathlineto{\pgfqpoint{9.334468in}{1.723603in}}%
\pgfpathlineto{\pgfqpoint{9.351339in}{1.726786in}}%
\pgfpathlineto{\pgfqpoint{9.368170in}{1.729932in}}%
\pgfpathlineto{\pgfqpoint{9.384961in}{1.733040in}}%
\pgfpathlineto{\pgfqpoint{9.401715in}{1.736112in}}%
\pgfpathlineto{\pgfqpoint{9.418432in}{1.739147in}}%
\pgfpathlineto{\pgfqpoint{9.435113in}{1.742146in}}%
\pgfpathlineto{\pgfqpoint{9.451761in}{1.745110in}}%
\pgfpathlineto{\pgfqpoint{9.468375in}{1.748040in}}%
\pgfpathlineto{\pgfqpoint{9.484958in}{1.750935in}}%
\pgfpathlineto{\pgfqpoint{9.501509in}{1.753796in}}%
\pgfpathlineto{\pgfqpoint{9.518029in}{1.756624in}}%
\pgfpathlineto{\pgfqpoint{9.534518in}{1.759420in}}%
\pgfpathlineto{\pgfqpoint{9.550978in}{1.762183in}}%
\pgfpathlineto{\pgfqpoint{9.567408in}{1.764913in}}%
\pgfpathlineto{\pgfqpoint{9.583810in}{1.767613in}}%
\pgfpathlineto{\pgfqpoint{9.600182in}{1.770281in}}%
\pgfpathlineto{\pgfqpoint{9.616527in}{1.772919in}}%
\pgfpathlineto{\pgfqpoint{9.632842in}{1.775526in}}%
\pgfpathlineto{\pgfqpoint{9.649129in}{1.778103in}}%
\pgfpathlineto{\pgfqpoint{9.665386in}{1.780651in}}%
\pgfpathlineto{\pgfqpoint{9.681613in}{1.783170in}}%
\pgfpathlineto{\pgfqpoint{9.697812in}{1.785660in}}%
\pgfpathlineto{\pgfqpoint{9.713981in}{1.788122in}}%
\pgfpathlineto{\pgfqpoint{9.730120in}{1.790556in}}%
\pgfpathlineto{\pgfqpoint{9.746231in}{1.792963in}}%
\pgfpathlineto{\pgfqpoint{9.762311in}{1.795342in}}%
\pgfpathlineto{\pgfqpoint{9.778361in}{1.797694in}}%
\pgfpathlineto{\pgfqpoint{9.794382in}{1.800020in}}%
\pgfpathlineto{\pgfqpoint{9.810374in}{1.802319in}}%
\pgfpathlineto{\pgfqpoint{9.826337in}{1.804592in}}%
\pgfpathlineto{\pgfqpoint{9.842273in}{1.806841in}}%
\pgfpathlineto{\pgfqpoint{9.858181in}{1.809065in}}%
\pgfpathlineto{\pgfqpoint{9.874063in}{1.811263in}}%
\pgfpathlineto{\pgfqpoint{9.889917in}{1.813437in}}%
\pgfpathlineto{\pgfqpoint{9.905745in}{1.815587in}}%
\pgfpathlineto{\pgfqpoint{9.921548in}{1.817713in}}%
\pgfpathlineto{\pgfqpoint{9.937324in}{1.819816in}}%
\pgfpathlineto{\pgfqpoint{9.953077in}{1.821896in}}%
\pgfpathlineto{\pgfqpoint{9.968804in}{1.823953in}}%
\pgfpathlineto{\pgfqpoint{9.984508in}{1.825987in}}%
\pgfpathlineto{\pgfqpoint{10.000186in}{1.827998in}}%
\pgfpathlineto{\pgfqpoint{10.015840in}{1.829988in}}%
\pgfpathlineto{\pgfqpoint{10.031468in}{1.831957in}}%
\pgfpathlineto{\pgfqpoint{10.047071in}{1.833904in}}%
\pgfpathlineto{\pgfqpoint{10.062648in}{1.835830in}}%
\pgfpathlineto{\pgfqpoint{10.078201in}{1.837735in}}%
\pgfpathlineto{\pgfqpoint{10.093729in}{1.839619in}}%
\pgfusepath{stroke}%
\end{pgfscope}%
\begin{pgfscope}%
\pgfpathrectangle{\pgfqpoint{8.282041in}{0.790446in}}{\pgfqpoint{1.897959in}{1.372727in}}%
\pgfusepath{clip}%
\pgfsetrectcap%
\pgfsetroundjoin%
\pgfsetlinewidth{1.505625pt}%
\definecolor{currentstroke}{rgb}{0.000000,0.750000,0.750000}%
\pgfsetstrokecolor{currentstroke}%
\pgfsetstrokeopacity{0.750000}%
\pgfsetdash{}{0pt}%
\pgfpathmoveto{\pgfqpoint{8.523435in}{1.834646in}}%
\pgfpathlineto{\pgfqpoint{8.543513in}{1.813246in}}%
\pgfpathlineto{\pgfqpoint{8.563540in}{1.794148in}}%
\pgfpathlineto{\pgfqpoint{8.583512in}{1.776098in}}%
\pgfpathlineto{\pgfqpoint{8.603426in}{1.759044in}}%
\pgfpathlineto{\pgfqpoint{8.623279in}{1.742924in}}%
\pgfpathlineto{\pgfqpoint{8.643070in}{1.727676in}}%
\pgfpathlineto{\pgfqpoint{8.662795in}{1.713242in}}%
\pgfpathlineto{\pgfqpoint{8.682451in}{1.699568in}}%
\pgfpathlineto{\pgfqpoint{8.702036in}{1.686602in}}%
\pgfpathlineto{\pgfqpoint{8.721547in}{1.674294in}}%
\pgfpathlineto{\pgfqpoint{8.740981in}{1.662603in}}%
\pgfpathlineto{\pgfqpoint{8.760335in}{1.651485in}}%
\pgfpathlineto{\pgfqpoint{8.779604in}{1.640902in}}%
\pgfpathlineto{\pgfqpoint{8.798784in}{1.630820in}}%
\pgfpathlineto{\pgfqpoint{8.817875in}{1.621204in}}%
\pgfpathlineto{\pgfqpoint{8.836873in}{1.612024in}}%
\pgfpathlineto{\pgfqpoint{8.855777in}{1.603254in}}%
\pgfpathlineto{\pgfqpoint{8.874586in}{1.594866in}}%
\pgfpathlineto{\pgfqpoint{8.893298in}{1.586838in}}%
\pgfpathlineto{\pgfqpoint{8.911913in}{1.579146in}}%
\pgfpathlineto{\pgfqpoint{8.930432in}{1.571770in}}%
\pgfpathlineto{\pgfqpoint{8.948855in}{1.564692in}}%
\pgfpathlineto{\pgfqpoint{8.967183in}{1.557894in}}%
\pgfpathlineto{\pgfqpoint{8.985417in}{1.551358in}}%
\pgfpathlineto{\pgfqpoint{9.003560in}{1.545072in}}%
\pgfpathlineto{\pgfqpoint{9.021614in}{1.539019in}}%
\pgfpathlineto{\pgfqpoint{9.039581in}{1.533186in}}%
\pgfpathlineto{\pgfqpoint{9.057464in}{1.527561in}}%
\pgfpathlineto{\pgfqpoint{9.075266in}{1.522134in}}%
\pgfpathlineto{\pgfqpoint{9.092989in}{1.516893in}}%
\pgfpathlineto{\pgfqpoint{9.110637in}{1.511827in}}%
\pgfpathlineto{\pgfqpoint{9.128213in}{1.506929in}}%
\pgfpathlineto{\pgfqpoint{9.145719in}{1.502189in}}%
\pgfpathlineto{\pgfqpoint{9.163158in}{1.497598in}}%
\pgfpathlineto{\pgfqpoint{9.180533in}{1.493150in}}%
\pgfpathlineto{\pgfqpoint{9.197848in}{1.488836in}}%
\pgfpathlineto{\pgfqpoint{9.215105in}{1.484651in}}%
\pgfpathlineto{\pgfqpoint{9.232306in}{1.480588in}}%
\pgfpathlineto{\pgfqpoint{9.249454in}{1.476640in}}%
\pgfpathlineto{\pgfqpoint{9.266551in}{1.472802in}}%
\pgfpathlineto{\pgfqpoint{9.283598in}{1.469070in}}%
\pgfpathlineto{\pgfqpoint{9.300599in}{1.465438in}}%
\pgfpathlineto{\pgfqpoint{9.317555in}{1.461901in}}%
\pgfpathlineto{\pgfqpoint{9.334468in}{1.458454in}}%
\pgfpathlineto{\pgfqpoint{9.351339in}{1.455094in}}%
\pgfpathlineto{\pgfqpoint{9.368170in}{1.451816in}}%
\pgfpathlineto{\pgfqpoint{9.384961in}{1.448618in}}%
\pgfpathlineto{\pgfqpoint{9.401715in}{1.445495in}}%
\pgfpathlineto{\pgfqpoint{9.418432in}{1.442444in}}%
\pgfpathlineto{\pgfqpoint{9.435113in}{1.439462in}}%
\pgfpathlineto{\pgfqpoint{9.451761in}{1.436546in}}%
\pgfpathlineto{\pgfqpoint{9.468375in}{1.433693in}}%
\pgfpathlineto{\pgfqpoint{9.484958in}{1.430901in}}%
\pgfpathlineto{\pgfqpoint{9.501509in}{1.428166in}}%
\pgfpathlineto{\pgfqpoint{9.518029in}{1.425487in}}%
\pgfpathlineto{\pgfqpoint{9.534518in}{1.422862in}}%
\pgfpathlineto{\pgfqpoint{9.550978in}{1.420287in}}%
\pgfpathlineto{\pgfqpoint{9.567408in}{1.417761in}}%
\pgfpathlineto{\pgfqpoint{9.583810in}{1.415283in}}%
\pgfpathlineto{\pgfqpoint{9.600182in}{1.412850in}}%
\pgfpathlineto{\pgfqpoint{9.616527in}{1.410460in}}%
\pgfpathlineto{\pgfqpoint{9.632842in}{1.408112in}}%
\pgfpathlineto{\pgfqpoint{9.649129in}{1.405804in}}%
\pgfpathlineto{\pgfqpoint{9.665386in}{1.403535in}}%
\pgfpathlineto{\pgfqpoint{9.681613in}{1.401303in}}%
\pgfpathlineto{\pgfqpoint{9.697812in}{1.399106in}}%
\pgfpathlineto{\pgfqpoint{9.713981in}{1.396945in}}%
\pgfpathlineto{\pgfqpoint{9.730120in}{1.394817in}}%
\pgfpathlineto{\pgfqpoint{9.746231in}{1.392721in}}%
\pgfpathlineto{\pgfqpoint{9.762311in}{1.390656in}}%
\pgfpathlineto{\pgfqpoint{9.778361in}{1.388620in}}%
\pgfpathlineto{\pgfqpoint{9.794382in}{1.386613in}}%
\pgfpathlineto{\pgfqpoint{9.810374in}{1.384633in}}%
\pgfpathlineto{\pgfqpoint{9.826337in}{1.382680in}}%
\pgfpathlineto{\pgfqpoint{9.842273in}{1.380754in}}%
\pgfpathlineto{\pgfqpoint{9.858181in}{1.378853in}}%
\pgfpathlineto{\pgfqpoint{9.874063in}{1.376976in}}%
\pgfpathlineto{\pgfqpoint{9.889917in}{1.375122in}}%
\pgfpathlineto{\pgfqpoint{9.905745in}{1.373290in}}%
\pgfpathlineto{\pgfqpoint{9.921548in}{1.371481in}}%
\pgfpathlineto{\pgfqpoint{9.937324in}{1.369694in}}%
\pgfpathlineto{\pgfqpoint{9.953077in}{1.367926in}}%
\pgfpathlineto{\pgfqpoint{9.968804in}{1.366179in}}%
\pgfpathlineto{\pgfqpoint{9.984508in}{1.364451in}}%
\pgfpathlineto{\pgfqpoint{10.000186in}{1.362741in}}%
\pgfpathlineto{\pgfqpoint{10.015840in}{1.361050in}}%
\pgfpathlineto{\pgfqpoint{10.031468in}{1.359376in}}%
\pgfpathlineto{\pgfqpoint{10.047071in}{1.357720in}}%
\pgfpathlineto{\pgfqpoint{10.062648in}{1.356080in}}%
\pgfpathlineto{\pgfqpoint{10.078201in}{1.354456in}}%
\pgfpathlineto{\pgfqpoint{10.093729in}{1.352848in}}%
\pgfusepath{stroke}%
\end{pgfscope}%
\begin{pgfscope}%
\pgfsetrectcap%
\pgfsetmiterjoin%
\pgfsetlinewidth{0.803000pt}%
\definecolor{currentstroke}{rgb}{0.501961,0.501961,0.501961}%
\pgfsetstrokecolor{currentstroke}%
\pgfsetdash{}{0pt}%
\pgfpathmoveto{\pgfqpoint{8.282041in}{0.790446in}}%
\pgfpathlineto{\pgfqpoint{8.282041in}{2.163173in}}%
\pgfusepath{stroke}%
\end{pgfscope}%
\begin{pgfscope}%
\pgfsetrectcap%
\pgfsetmiterjoin%
\pgfsetlinewidth{0.803000pt}%
\definecolor{currentstroke}{rgb}{0.501961,0.501961,0.501961}%
\pgfsetstrokecolor{currentstroke}%
\pgfsetdash{}{0pt}%
\pgfpathmoveto{\pgfqpoint{10.180000in}{0.790446in}}%
\pgfpathlineto{\pgfqpoint{10.180000in}{2.163173in}}%
\pgfusepath{stroke}%
\end{pgfscope}%
\begin{pgfscope}%
\pgfsetrectcap%
\pgfsetmiterjoin%
\pgfsetlinewidth{0.803000pt}%
\definecolor{currentstroke}{rgb}{0.501961,0.501961,0.501961}%
\pgfsetstrokecolor{currentstroke}%
\pgfsetdash{}{0pt}%
\pgfpathmoveto{\pgfqpoint{8.282041in}{0.790446in}}%
\pgfpathlineto{\pgfqpoint{10.180000in}{0.790446in}}%
\pgfusepath{stroke}%
\end{pgfscope}%
\begin{pgfscope}%
\pgfsetrectcap%
\pgfsetmiterjoin%
\pgfsetlinewidth{0.803000pt}%
\definecolor{currentstroke}{rgb}{0.501961,0.501961,0.501961}%
\pgfsetstrokecolor{currentstroke}%
\pgfsetdash{}{0pt}%
\pgfpathmoveto{\pgfqpoint{8.282041in}{2.163173in}}%
\pgfpathlineto{\pgfqpoint{10.180000in}{2.163173in}}%
\pgfusepath{stroke}%
\end{pgfscope}%
\begin{pgfscope}%
\pgfsetrectcap%
\pgfsetroundjoin%
\pgfsetlinewidth{1.505625pt}%
\definecolor{currentstroke}{rgb}{1.000000,0.000000,0.000000}%
\pgfsetstrokecolor{currentstroke}%
\pgfsetstrokeopacity{0.750000}%
\pgfsetdash{}{0pt}%
\pgfpathmoveto{\pgfqpoint{10.637370in}{17.841827in}}%
\pgfpathlineto{\pgfqpoint{11.026259in}{17.841827in}}%
\pgfusepath{stroke}%
\end{pgfscope}%
\begin{pgfscope}%
\definecolor{textcolor}{rgb}{0.501961,0.501961,0.501961}%
\pgfsetstrokecolor{textcolor}%
\pgfsetfillcolor{textcolor}%
\pgftext[x=11.181814in,y=17.773771in,left,base]{\color{textcolor}\rmfamily\fontsize{14.000000}{16.800000}\selectfont Coulomb force}%
\end{pgfscope}%
\begin{pgfscope}%
\pgfsetrectcap%
\pgfsetroundjoin%
\pgfsetlinewidth{1.505625pt}%
\definecolor{currentstroke}{rgb}{0.000000,0.000000,1.000000}%
\pgfsetstrokecolor{currentstroke}%
\pgfsetstrokeopacity{0.750000}%
\pgfsetdash{}{0pt}%
\pgfpathmoveto{\pgfqpoint{10.637370in}{17.556426in}}%
\pgfpathlineto{\pgfqpoint{11.026259in}{17.556426in}}%
\pgfusepath{stroke}%
\end{pgfscope}%
\begin{pgfscope}%
\definecolor{textcolor}{rgb}{0.501961,0.501961,0.501961}%
\pgfsetstrokecolor{textcolor}%
\pgfsetfillcolor{textcolor}%
\pgftext[x=11.181814in,y=17.488371in,left,base]{\color{textcolor}\rmfamily\fontsize{14.000000}{16.800000}\selectfont Drag force}%
\end{pgfscope}%
\begin{pgfscope}%
\pgfsetrectcap%
\pgfsetroundjoin%
\pgfsetlinewidth{1.505625pt}%
\definecolor{currentstroke}{rgb}{0.000000,0.750000,0.750000}%
\pgfsetstrokecolor{currentstroke}%
\pgfsetstrokeopacity{0.750000}%
\pgfsetdash{}{0pt}%
\pgfpathmoveto{\pgfqpoint{10.637370in}{17.268273in}}%
\pgfpathlineto{\pgfqpoint{11.026259in}{17.268273in}}%
\pgfusepath{stroke}%
\end{pgfscope}%
\begin{pgfscope}%
\definecolor{textcolor}{rgb}{0.501961,0.501961,0.501961}%
\pgfsetstrokecolor{textcolor}%
\pgfsetfillcolor{textcolor}%
\pgftext[x=11.181814in,y=17.200217in,left,base]{\color{textcolor}\rmfamily\fontsize{14.000000}{16.800000}\selectfont Image force}%
\end{pgfscope}%
\begin{pgfscope}%
\pgfsetrectcap%
\pgfsetroundjoin%
\pgfsetlinewidth{1.505625pt}%
\definecolor{currentstroke}{rgb}{1.000000,0.000000,0.000000}%
\pgfsetstrokecolor{currentstroke}%
\pgfsetstrokeopacity{0.750000}%
\pgfsetdash{}{0pt}%
\pgfpathmoveto{\pgfqpoint{10.637370in}{16.980119in}}%
\pgfpathlineto{\pgfqpoint{11.026259in}{16.980119in}}%
\pgfusepath{stroke}%
\end{pgfscope}%
\begin{pgfscope}%
\definecolor{textcolor}{rgb}{0.501961,0.501961,0.501961}%
\pgfsetstrokecolor{textcolor}%
\pgfsetfillcolor{textcolor}%
\pgftext[x=11.181814in,y=16.912063in,left,base]{\color{textcolor}\rmfamily\fontsize{14.000000}{16.800000}\selectfont Coulomb force}%
\end{pgfscope}%
\begin{pgfscope}%
\pgfsetrectcap%
\pgfsetroundjoin%
\pgfsetlinewidth{1.505625pt}%
\definecolor{currentstroke}{rgb}{0.000000,0.000000,1.000000}%
\pgfsetstrokecolor{currentstroke}%
\pgfsetstrokeopacity{0.750000}%
\pgfsetdash{}{0pt}%
\pgfpathmoveto{\pgfqpoint{10.637370in}{16.694719in}}%
\pgfpathlineto{\pgfqpoint{11.026259in}{16.694719in}}%
\pgfusepath{stroke}%
\end{pgfscope}%
\begin{pgfscope}%
\definecolor{textcolor}{rgb}{0.501961,0.501961,0.501961}%
\pgfsetstrokecolor{textcolor}%
\pgfsetfillcolor{textcolor}%
\pgftext[x=11.181814in,y=16.626663in,left,base]{\color{textcolor}\rmfamily\fontsize{14.000000}{16.800000}\selectfont Drag force}%
\end{pgfscope}%
\begin{pgfscope}%
\pgfsetrectcap%
\pgfsetroundjoin%
\pgfsetlinewidth{1.505625pt}%
\definecolor{currentstroke}{rgb}{0.000000,0.750000,0.750000}%
\pgfsetstrokecolor{currentstroke}%
\pgfsetstrokeopacity{0.750000}%
\pgfsetdash{}{0pt}%
\pgfpathmoveto{\pgfqpoint{10.637370in}{16.406565in}}%
\pgfpathlineto{\pgfqpoint{11.026259in}{16.406565in}}%
\pgfusepath{stroke}%
\end{pgfscope}%
\begin{pgfscope}%
\definecolor{textcolor}{rgb}{0.501961,0.501961,0.501961}%
\pgfsetstrokecolor{textcolor}%
\pgfsetfillcolor{textcolor}%
\pgftext[x=11.181814in,y=16.338510in,left,base]{\color{textcolor}\rmfamily\fontsize{14.000000}{16.800000}\selectfont Image force}%
\end{pgfscope}%
\begin{pgfscope}%
\pgfsetrectcap%
\pgfsetroundjoin%
\pgfsetlinewidth{1.505625pt}%
\definecolor{currentstroke}{rgb}{1.000000,0.000000,0.000000}%
\pgfsetstrokecolor{currentstroke}%
\pgfsetstrokeopacity{0.750000}%
\pgfsetdash{}{0pt}%
\pgfpathmoveto{\pgfqpoint{10.637370in}{16.118412in}}%
\pgfpathlineto{\pgfqpoint{11.026259in}{16.118412in}}%
\pgfusepath{stroke}%
\end{pgfscope}%
\begin{pgfscope}%
\definecolor{textcolor}{rgb}{0.501961,0.501961,0.501961}%
\pgfsetstrokecolor{textcolor}%
\pgfsetfillcolor{textcolor}%
\pgftext[x=11.181814in,y=16.050356in,left,base]{\color{textcolor}\rmfamily\fontsize{14.000000}{16.800000}\selectfont Coulomb force}%
\end{pgfscope}%
\begin{pgfscope}%
\pgfsetrectcap%
\pgfsetroundjoin%
\pgfsetlinewidth{1.505625pt}%
\definecolor{currentstroke}{rgb}{0.000000,0.000000,1.000000}%
\pgfsetstrokecolor{currentstroke}%
\pgfsetstrokeopacity{0.750000}%
\pgfsetdash{}{0pt}%
\pgfpathmoveto{\pgfqpoint{10.637370in}{15.833011in}}%
\pgfpathlineto{\pgfqpoint{11.026259in}{15.833011in}}%
\pgfusepath{stroke}%
\end{pgfscope}%
\begin{pgfscope}%
\definecolor{textcolor}{rgb}{0.501961,0.501961,0.501961}%
\pgfsetstrokecolor{textcolor}%
\pgfsetfillcolor{textcolor}%
\pgftext[x=11.181814in,y=15.764956in,left,base]{\color{textcolor}\rmfamily\fontsize{14.000000}{16.800000}\selectfont Drag force}%
\end{pgfscope}%
\begin{pgfscope}%
\pgfsetrectcap%
\pgfsetroundjoin%
\pgfsetlinewidth{1.505625pt}%
\definecolor{currentstroke}{rgb}{0.000000,0.750000,0.750000}%
\pgfsetstrokecolor{currentstroke}%
\pgfsetstrokeopacity{0.750000}%
\pgfsetdash{}{0pt}%
\pgfpathmoveto{\pgfqpoint{10.637370in}{15.544858in}}%
\pgfpathlineto{\pgfqpoint{11.026259in}{15.544858in}}%
\pgfusepath{stroke}%
\end{pgfscope}%
\begin{pgfscope}%
\definecolor{textcolor}{rgb}{0.501961,0.501961,0.501961}%
\pgfsetstrokecolor{textcolor}%
\pgfsetfillcolor{textcolor}%
\pgftext[x=11.181814in,y=15.476802in,left,base]{\color{textcolor}\rmfamily\fontsize{14.000000}{16.800000}\selectfont Image force}%
\end{pgfscope}%
\begin{pgfscope}%
\pgfsetrectcap%
\pgfsetroundjoin%
\pgfsetlinewidth{1.505625pt}%
\definecolor{currentstroke}{rgb}{1.000000,0.000000,0.000000}%
\pgfsetstrokecolor{currentstroke}%
\pgfsetstrokeopacity{0.750000}%
\pgfsetdash{}{0pt}%
\pgfpathmoveto{\pgfqpoint{10.637370in}{15.256704in}}%
\pgfpathlineto{\pgfqpoint{11.026259in}{15.256704in}}%
\pgfusepath{stroke}%
\end{pgfscope}%
\begin{pgfscope}%
\definecolor{textcolor}{rgb}{0.501961,0.501961,0.501961}%
\pgfsetstrokecolor{textcolor}%
\pgfsetfillcolor{textcolor}%
\pgftext[x=11.181814in,y=15.188648in,left,base]{\color{textcolor}\rmfamily\fontsize{14.000000}{16.800000}\selectfont Coulomb force}%
\end{pgfscope}%
\begin{pgfscope}%
\pgfsetrectcap%
\pgfsetroundjoin%
\pgfsetlinewidth{1.505625pt}%
\definecolor{currentstroke}{rgb}{0.000000,0.000000,1.000000}%
\pgfsetstrokecolor{currentstroke}%
\pgfsetstrokeopacity{0.750000}%
\pgfsetdash{}{0pt}%
\pgfpathmoveto{\pgfqpoint{10.637370in}{14.971304in}}%
\pgfpathlineto{\pgfqpoint{11.026259in}{14.971304in}}%
\pgfusepath{stroke}%
\end{pgfscope}%
\begin{pgfscope}%
\definecolor{textcolor}{rgb}{0.501961,0.501961,0.501961}%
\pgfsetstrokecolor{textcolor}%
\pgfsetfillcolor{textcolor}%
\pgftext[x=11.181814in,y=14.903248in,left,base]{\color{textcolor}\rmfamily\fontsize{14.000000}{16.800000}\selectfont Drag force}%
\end{pgfscope}%
\begin{pgfscope}%
\pgfsetrectcap%
\pgfsetroundjoin%
\pgfsetlinewidth{1.505625pt}%
\definecolor{currentstroke}{rgb}{0.000000,0.750000,0.750000}%
\pgfsetstrokecolor{currentstroke}%
\pgfsetstrokeopacity{0.750000}%
\pgfsetdash{}{0pt}%
\pgfpathmoveto{\pgfqpoint{10.637370in}{14.683150in}}%
\pgfpathlineto{\pgfqpoint{11.026259in}{14.683150in}}%
\pgfusepath{stroke}%
\end{pgfscope}%
\begin{pgfscope}%
\definecolor{textcolor}{rgb}{0.501961,0.501961,0.501961}%
\pgfsetstrokecolor{textcolor}%
\pgfsetfillcolor{textcolor}%
\pgftext[x=11.181814in,y=14.615095in,left,base]{\color{textcolor}\rmfamily\fontsize{14.000000}{16.800000}\selectfont Image force}%
\end{pgfscope}%
\begin{pgfscope}%
\pgfsetrectcap%
\pgfsetroundjoin%
\pgfsetlinewidth{1.505625pt}%
\definecolor{currentstroke}{rgb}{1.000000,0.000000,0.000000}%
\pgfsetstrokecolor{currentstroke}%
\pgfsetstrokeopacity{0.750000}%
\pgfsetdash{}{0pt}%
\pgfpathmoveto{\pgfqpoint{10.637370in}{14.394997in}}%
\pgfpathlineto{\pgfqpoint{11.026259in}{14.394997in}}%
\pgfusepath{stroke}%
\end{pgfscope}%
\begin{pgfscope}%
\definecolor{textcolor}{rgb}{0.501961,0.501961,0.501961}%
\pgfsetstrokecolor{textcolor}%
\pgfsetfillcolor{textcolor}%
\pgftext[x=11.181814in,y=14.326941in,left,base]{\color{textcolor}\rmfamily\fontsize{14.000000}{16.800000}\selectfont Coulomb force}%
\end{pgfscope}%
\begin{pgfscope}%
\pgfsetrectcap%
\pgfsetroundjoin%
\pgfsetlinewidth{1.505625pt}%
\definecolor{currentstroke}{rgb}{0.000000,0.000000,1.000000}%
\pgfsetstrokecolor{currentstroke}%
\pgfsetstrokeopacity{0.750000}%
\pgfsetdash{}{0pt}%
\pgfpathmoveto{\pgfqpoint{10.637370in}{14.109596in}}%
\pgfpathlineto{\pgfqpoint{11.026259in}{14.109596in}}%
\pgfusepath{stroke}%
\end{pgfscope}%
\begin{pgfscope}%
\definecolor{textcolor}{rgb}{0.501961,0.501961,0.501961}%
\pgfsetstrokecolor{textcolor}%
\pgfsetfillcolor{textcolor}%
\pgftext[x=11.181814in,y=14.041541in,left,base]{\color{textcolor}\rmfamily\fontsize{14.000000}{16.800000}\selectfont Drag force}%
\end{pgfscope}%
\begin{pgfscope}%
\pgfsetrectcap%
\pgfsetroundjoin%
\pgfsetlinewidth{1.505625pt}%
\definecolor{currentstroke}{rgb}{0.000000,0.750000,0.750000}%
\pgfsetstrokecolor{currentstroke}%
\pgfsetstrokeopacity{0.750000}%
\pgfsetdash{}{0pt}%
\pgfpathmoveto{\pgfqpoint{10.637370in}{13.821443in}}%
\pgfpathlineto{\pgfqpoint{11.026259in}{13.821443in}}%
\pgfusepath{stroke}%
\end{pgfscope}%
\begin{pgfscope}%
\definecolor{textcolor}{rgb}{0.501961,0.501961,0.501961}%
\pgfsetstrokecolor{textcolor}%
\pgfsetfillcolor{textcolor}%
\pgftext[x=11.181814in,y=13.753387in,left,base]{\color{textcolor}\rmfamily\fontsize{14.000000}{16.800000}\selectfont Image force}%
\end{pgfscope}%
\begin{pgfscope}%
\pgfsetrectcap%
\pgfsetroundjoin%
\pgfsetlinewidth{1.505625pt}%
\definecolor{currentstroke}{rgb}{1.000000,0.000000,0.000000}%
\pgfsetstrokecolor{currentstroke}%
\pgfsetstrokeopacity{0.750000}%
\pgfsetdash{}{0pt}%
\pgfpathmoveto{\pgfqpoint{10.637370in}{13.533289in}}%
\pgfpathlineto{\pgfqpoint{11.026259in}{13.533289in}}%
\pgfusepath{stroke}%
\end{pgfscope}%
\begin{pgfscope}%
\definecolor{textcolor}{rgb}{0.501961,0.501961,0.501961}%
\pgfsetstrokecolor{textcolor}%
\pgfsetfillcolor{textcolor}%
\pgftext[x=11.181814in,y=13.465233in,left,base]{\color{textcolor}\rmfamily\fontsize{14.000000}{16.800000}\selectfont Coulomb force}%
\end{pgfscope}%
\begin{pgfscope}%
\pgfsetrectcap%
\pgfsetroundjoin%
\pgfsetlinewidth{1.505625pt}%
\definecolor{currentstroke}{rgb}{0.000000,0.000000,1.000000}%
\pgfsetstrokecolor{currentstroke}%
\pgfsetstrokeopacity{0.750000}%
\pgfsetdash{}{0pt}%
\pgfpathmoveto{\pgfqpoint{10.637370in}{13.247889in}}%
\pgfpathlineto{\pgfqpoint{11.026259in}{13.247889in}}%
\pgfusepath{stroke}%
\end{pgfscope}%
\begin{pgfscope}%
\definecolor{textcolor}{rgb}{0.501961,0.501961,0.501961}%
\pgfsetstrokecolor{textcolor}%
\pgfsetfillcolor{textcolor}%
\pgftext[x=11.181814in,y=13.179833in,left,base]{\color{textcolor}\rmfamily\fontsize{14.000000}{16.800000}\selectfont Drag force}%
\end{pgfscope}%
\begin{pgfscope}%
\pgfsetrectcap%
\pgfsetroundjoin%
\pgfsetlinewidth{1.505625pt}%
\definecolor{currentstroke}{rgb}{0.000000,0.750000,0.750000}%
\pgfsetstrokecolor{currentstroke}%
\pgfsetstrokeopacity{0.750000}%
\pgfsetdash{}{0pt}%
\pgfpathmoveto{\pgfqpoint{10.637370in}{12.959735in}}%
\pgfpathlineto{\pgfqpoint{11.026259in}{12.959735in}}%
\pgfusepath{stroke}%
\end{pgfscope}%
\begin{pgfscope}%
\definecolor{textcolor}{rgb}{0.501961,0.501961,0.501961}%
\pgfsetstrokecolor{textcolor}%
\pgfsetfillcolor{textcolor}%
\pgftext[x=11.181814in,y=12.891680in,left,base]{\color{textcolor}\rmfamily\fontsize{14.000000}{16.800000}\selectfont Image force}%
\end{pgfscope}%
\begin{pgfscope}%
\pgfsetrectcap%
\pgfsetroundjoin%
\pgfsetlinewidth{1.505625pt}%
\definecolor{currentstroke}{rgb}{1.000000,0.000000,0.000000}%
\pgfsetstrokecolor{currentstroke}%
\pgfsetstrokeopacity{0.750000}%
\pgfsetdash{}{0pt}%
\pgfpathmoveto{\pgfqpoint{10.637370in}{12.671582in}}%
\pgfpathlineto{\pgfqpoint{11.026259in}{12.671582in}}%
\pgfusepath{stroke}%
\end{pgfscope}%
\begin{pgfscope}%
\definecolor{textcolor}{rgb}{0.501961,0.501961,0.501961}%
\pgfsetstrokecolor{textcolor}%
\pgfsetfillcolor{textcolor}%
\pgftext[x=11.181814in,y=12.603526in,left,base]{\color{textcolor}\rmfamily\fontsize{14.000000}{16.800000}\selectfont Coulomb force}%
\end{pgfscope}%
\begin{pgfscope}%
\pgfsetrectcap%
\pgfsetroundjoin%
\pgfsetlinewidth{1.505625pt}%
\definecolor{currentstroke}{rgb}{0.000000,0.000000,1.000000}%
\pgfsetstrokecolor{currentstroke}%
\pgfsetstrokeopacity{0.750000}%
\pgfsetdash{}{0pt}%
\pgfpathmoveto{\pgfqpoint{10.637370in}{12.386181in}}%
\pgfpathlineto{\pgfqpoint{11.026259in}{12.386181in}}%
\pgfusepath{stroke}%
\end{pgfscope}%
\begin{pgfscope}%
\definecolor{textcolor}{rgb}{0.501961,0.501961,0.501961}%
\pgfsetstrokecolor{textcolor}%
\pgfsetfillcolor{textcolor}%
\pgftext[x=11.181814in,y=12.318126in,left,base]{\color{textcolor}\rmfamily\fontsize{14.000000}{16.800000}\selectfont Drag force}%
\end{pgfscope}%
\begin{pgfscope}%
\pgfsetrectcap%
\pgfsetroundjoin%
\pgfsetlinewidth{1.505625pt}%
\definecolor{currentstroke}{rgb}{0.000000,0.750000,0.750000}%
\pgfsetstrokecolor{currentstroke}%
\pgfsetstrokeopacity{0.750000}%
\pgfsetdash{}{0pt}%
\pgfpathmoveto{\pgfqpoint{10.637370in}{12.098028in}}%
\pgfpathlineto{\pgfqpoint{11.026259in}{12.098028in}}%
\pgfusepath{stroke}%
\end{pgfscope}%
\begin{pgfscope}%
\definecolor{textcolor}{rgb}{0.501961,0.501961,0.501961}%
\pgfsetstrokecolor{textcolor}%
\pgfsetfillcolor{textcolor}%
\pgftext[x=11.181814in,y=12.029972in,left,base]{\color{textcolor}\rmfamily\fontsize{14.000000}{16.800000}\selectfont Image force}%
\end{pgfscope}%
\begin{pgfscope}%
\pgfsetrectcap%
\pgfsetroundjoin%
\pgfsetlinewidth{1.505625pt}%
\definecolor{currentstroke}{rgb}{1.000000,0.000000,0.000000}%
\pgfsetstrokecolor{currentstroke}%
\pgfsetstrokeopacity{0.750000}%
\pgfsetdash{}{0pt}%
\pgfpathmoveto{\pgfqpoint{10.637370in}{11.809874in}}%
\pgfpathlineto{\pgfqpoint{11.026259in}{11.809874in}}%
\pgfusepath{stroke}%
\end{pgfscope}%
\begin{pgfscope}%
\definecolor{textcolor}{rgb}{0.501961,0.501961,0.501961}%
\pgfsetstrokecolor{textcolor}%
\pgfsetfillcolor{textcolor}%
\pgftext[x=11.181814in,y=11.741818in,left,base]{\color{textcolor}\rmfamily\fontsize{14.000000}{16.800000}\selectfont Coulomb force}%
\end{pgfscope}%
\begin{pgfscope}%
\pgfsetrectcap%
\pgfsetroundjoin%
\pgfsetlinewidth{1.505625pt}%
\definecolor{currentstroke}{rgb}{0.000000,0.000000,1.000000}%
\pgfsetstrokecolor{currentstroke}%
\pgfsetstrokeopacity{0.750000}%
\pgfsetdash{}{0pt}%
\pgfpathmoveto{\pgfqpoint{10.637370in}{11.524474in}}%
\pgfpathlineto{\pgfqpoint{11.026259in}{11.524474in}}%
\pgfusepath{stroke}%
\end{pgfscope}%
\begin{pgfscope}%
\definecolor{textcolor}{rgb}{0.501961,0.501961,0.501961}%
\pgfsetstrokecolor{textcolor}%
\pgfsetfillcolor{textcolor}%
\pgftext[x=11.181814in,y=11.456418in,left,base]{\color{textcolor}\rmfamily\fontsize{14.000000}{16.800000}\selectfont Drag force}%
\end{pgfscope}%
\begin{pgfscope}%
\pgfsetrectcap%
\pgfsetroundjoin%
\pgfsetlinewidth{1.505625pt}%
\definecolor{currentstroke}{rgb}{0.000000,0.750000,0.750000}%
\pgfsetstrokecolor{currentstroke}%
\pgfsetstrokeopacity{0.750000}%
\pgfsetdash{}{0pt}%
\pgfpathmoveto{\pgfqpoint{10.637370in}{11.236320in}}%
\pgfpathlineto{\pgfqpoint{11.026259in}{11.236320in}}%
\pgfusepath{stroke}%
\end{pgfscope}%
\begin{pgfscope}%
\definecolor{textcolor}{rgb}{0.501961,0.501961,0.501961}%
\pgfsetstrokecolor{textcolor}%
\pgfsetfillcolor{textcolor}%
\pgftext[x=11.181814in,y=11.168265in,left,base]{\color{textcolor}\rmfamily\fontsize{14.000000}{16.800000}\selectfont Image force}%
\end{pgfscope}%
\begin{pgfscope}%
\pgfsetrectcap%
\pgfsetroundjoin%
\pgfsetlinewidth{1.505625pt}%
\definecolor{currentstroke}{rgb}{1.000000,0.000000,0.000000}%
\pgfsetstrokecolor{currentstroke}%
\pgfsetstrokeopacity{0.750000}%
\pgfsetdash{}{0pt}%
\pgfpathmoveto{\pgfqpoint{10.637370in}{10.948167in}}%
\pgfpathlineto{\pgfqpoint{11.026259in}{10.948167in}}%
\pgfusepath{stroke}%
\end{pgfscope}%
\begin{pgfscope}%
\definecolor{textcolor}{rgb}{0.501961,0.501961,0.501961}%
\pgfsetstrokecolor{textcolor}%
\pgfsetfillcolor{textcolor}%
\pgftext[x=11.181814in,y=10.880111in,left,base]{\color{textcolor}\rmfamily\fontsize{14.000000}{16.800000}\selectfont Coulomb force}%
\end{pgfscope}%
\begin{pgfscope}%
\pgfsetrectcap%
\pgfsetroundjoin%
\pgfsetlinewidth{1.505625pt}%
\definecolor{currentstroke}{rgb}{0.000000,0.000000,1.000000}%
\pgfsetstrokecolor{currentstroke}%
\pgfsetstrokeopacity{0.750000}%
\pgfsetdash{}{0pt}%
\pgfpathmoveto{\pgfqpoint{10.637370in}{10.662766in}}%
\pgfpathlineto{\pgfqpoint{11.026259in}{10.662766in}}%
\pgfusepath{stroke}%
\end{pgfscope}%
\begin{pgfscope}%
\definecolor{textcolor}{rgb}{0.501961,0.501961,0.501961}%
\pgfsetstrokecolor{textcolor}%
\pgfsetfillcolor{textcolor}%
\pgftext[x=11.181814in,y=10.594711in,left,base]{\color{textcolor}\rmfamily\fontsize{14.000000}{16.800000}\selectfont Drag force}%
\end{pgfscope}%
\begin{pgfscope}%
\pgfsetrectcap%
\pgfsetroundjoin%
\pgfsetlinewidth{1.505625pt}%
\definecolor{currentstroke}{rgb}{0.000000,0.750000,0.750000}%
\pgfsetstrokecolor{currentstroke}%
\pgfsetstrokeopacity{0.750000}%
\pgfsetdash{}{0pt}%
\pgfpathmoveto{\pgfqpoint{10.637370in}{10.374613in}}%
\pgfpathlineto{\pgfqpoint{11.026259in}{10.374613in}}%
\pgfusepath{stroke}%
\end{pgfscope}%
\begin{pgfscope}%
\definecolor{textcolor}{rgb}{0.501961,0.501961,0.501961}%
\pgfsetstrokecolor{textcolor}%
\pgfsetfillcolor{textcolor}%
\pgftext[x=11.181814in,y=10.306557in,left,base]{\color{textcolor}\rmfamily\fontsize{14.000000}{16.800000}\selectfont Image force}%
\end{pgfscope}%
\begin{pgfscope}%
\pgfsetrectcap%
\pgfsetroundjoin%
\pgfsetlinewidth{1.505625pt}%
\definecolor{currentstroke}{rgb}{1.000000,0.000000,0.000000}%
\pgfsetstrokecolor{currentstroke}%
\pgfsetstrokeopacity{0.750000}%
\pgfsetdash{}{0pt}%
\pgfpathmoveto{\pgfqpoint{10.637370in}{10.086459in}}%
\pgfpathlineto{\pgfqpoint{11.026259in}{10.086459in}}%
\pgfusepath{stroke}%
\end{pgfscope}%
\begin{pgfscope}%
\definecolor{textcolor}{rgb}{0.501961,0.501961,0.501961}%
\pgfsetstrokecolor{textcolor}%
\pgfsetfillcolor{textcolor}%
\pgftext[x=11.181814in,y=10.018403in,left,base]{\color{textcolor}\rmfamily\fontsize{14.000000}{16.800000}\selectfont Coulomb force}%
\end{pgfscope}%
\begin{pgfscope}%
\pgfsetrectcap%
\pgfsetroundjoin%
\pgfsetlinewidth{1.505625pt}%
\definecolor{currentstroke}{rgb}{0.000000,0.000000,1.000000}%
\pgfsetstrokecolor{currentstroke}%
\pgfsetstrokeopacity{0.750000}%
\pgfsetdash{}{0pt}%
\pgfpathmoveto{\pgfqpoint{10.637370in}{9.801059in}}%
\pgfpathlineto{\pgfqpoint{11.026259in}{9.801059in}}%
\pgfusepath{stroke}%
\end{pgfscope}%
\begin{pgfscope}%
\definecolor{textcolor}{rgb}{0.501961,0.501961,0.501961}%
\pgfsetstrokecolor{textcolor}%
\pgfsetfillcolor{textcolor}%
\pgftext[x=11.181814in,y=9.733003in,left,base]{\color{textcolor}\rmfamily\fontsize{14.000000}{16.800000}\selectfont Drag force}%
\end{pgfscope}%
\begin{pgfscope}%
\pgfsetrectcap%
\pgfsetroundjoin%
\pgfsetlinewidth{1.505625pt}%
\definecolor{currentstroke}{rgb}{0.000000,0.750000,0.750000}%
\pgfsetstrokecolor{currentstroke}%
\pgfsetstrokeopacity{0.750000}%
\pgfsetdash{}{0pt}%
\pgfpathmoveto{\pgfqpoint{10.637370in}{9.512905in}}%
\pgfpathlineto{\pgfqpoint{11.026259in}{9.512905in}}%
\pgfusepath{stroke}%
\end{pgfscope}%
\begin{pgfscope}%
\definecolor{textcolor}{rgb}{0.501961,0.501961,0.501961}%
\pgfsetstrokecolor{textcolor}%
\pgfsetfillcolor{textcolor}%
\pgftext[x=11.181814in,y=9.444850in,left,base]{\color{textcolor}\rmfamily\fontsize{14.000000}{16.800000}\selectfont Image force}%
\end{pgfscope}%
\begin{pgfscope}%
\pgfsetrectcap%
\pgfsetroundjoin%
\pgfsetlinewidth{1.505625pt}%
\definecolor{currentstroke}{rgb}{1.000000,0.000000,0.000000}%
\pgfsetstrokecolor{currentstroke}%
\pgfsetstrokeopacity{0.750000}%
\pgfsetdash{}{0pt}%
\pgfpathmoveto{\pgfqpoint{10.637370in}{9.224752in}}%
\pgfpathlineto{\pgfqpoint{11.026259in}{9.224752in}}%
\pgfusepath{stroke}%
\end{pgfscope}%
\begin{pgfscope}%
\definecolor{textcolor}{rgb}{0.501961,0.501961,0.501961}%
\pgfsetstrokecolor{textcolor}%
\pgfsetfillcolor{textcolor}%
\pgftext[x=11.181814in,y=9.156696in,left,base]{\color{textcolor}\rmfamily\fontsize{14.000000}{16.800000}\selectfont Coulomb force}%
\end{pgfscope}%
\begin{pgfscope}%
\pgfsetrectcap%
\pgfsetroundjoin%
\pgfsetlinewidth{1.505625pt}%
\definecolor{currentstroke}{rgb}{0.000000,0.000000,1.000000}%
\pgfsetstrokecolor{currentstroke}%
\pgfsetstrokeopacity{0.750000}%
\pgfsetdash{}{0pt}%
\pgfpathmoveto{\pgfqpoint{10.637370in}{8.939351in}}%
\pgfpathlineto{\pgfqpoint{11.026259in}{8.939351in}}%
\pgfusepath{stroke}%
\end{pgfscope}%
\begin{pgfscope}%
\definecolor{textcolor}{rgb}{0.501961,0.501961,0.501961}%
\pgfsetstrokecolor{textcolor}%
\pgfsetfillcolor{textcolor}%
\pgftext[x=11.181814in,y=8.871296in,left,base]{\color{textcolor}\rmfamily\fontsize{14.000000}{16.800000}\selectfont Drag force}%
\end{pgfscope}%
\begin{pgfscope}%
\pgfsetrectcap%
\pgfsetroundjoin%
\pgfsetlinewidth{1.505625pt}%
\definecolor{currentstroke}{rgb}{0.000000,0.750000,0.750000}%
\pgfsetstrokecolor{currentstroke}%
\pgfsetstrokeopacity{0.750000}%
\pgfsetdash{}{0pt}%
\pgfpathmoveto{\pgfqpoint{10.637370in}{8.651198in}}%
\pgfpathlineto{\pgfqpoint{11.026259in}{8.651198in}}%
\pgfusepath{stroke}%
\end{pgfscope}%
\begin{pgfscope}%
\definecolor{textcolor}{rgb}{0.501961,0.501961,0.501961}%
\pgfsetstrokecolor{textcolor}%
\pgfsetfillcolor{textcolor}%
\pgftext[x=11.181814in,y=8.583142in,left,base]{\color{textcolor}\rmfamily\fontsize{14.000000}{16.800000}\selectfont Image force}%
\end{pgfscope}%
\begin{pgfscope}%
\pgfsetrectcap%
\pgfsetroundjoin%
\pgfsetlinewidth{1.505625pt}%
\definecolor{currentstroke}{rgb}{1.000000,0.000000,0.000000}%
\pgfsetstrokecolor{currentstroke}%
\pgfsetstrokeopacity{0.750000}%
\pgfsetdash{}{0pt}%
\pgfpathmoveto{\pgfqpoint{10.637370in}{8.363044in}}%
\pgfpathlineto{\pgfqpoint{11.026259in}{8.363044in}}%
\pgfusepath{stroke}%
\end{pgfscope}%
\begin{pgfscope}%
\definecolor{textcolor}{rgb}{0.501961,0.501961,0.501961}%
\pgfsetstrokecolor{textcolor}%
\pgfsetfillcolor{textcolor}%
\pgftext[x=11.181814in,y=8.294988in,left,base]{\color{textcolor}\rmfamily\fontsize{14.000000}{16.800000}\selectfont Coulomb force}%
\end{pgfscope}%
\begin{pgfscope}%
\pgfsetrectcap%
\pgfsetroundjoin%
\pgfsetlinewidth{1.505625pt}%
\definecolor{currentstroke}{rgb}{0.000000,0.000000,1.000000}%
\pgfsetstrokecolor{currentstroke}%
\pgfsetstrokeopacity{0.750000}%
\pgfsetdash{}{0pt}%
\pgfpathmoveto{\pgfqpoint{10.637370in}{8.077644in}}%
\pgfpathlineto{\pgfqpoint{11.026259in}{8.077644in}}%
\pgfusepath{stroke}%
\end{pgfscope}%
\begin{pgfscope}%
\definecolor{textcolor}{rgb}{0.501961,0.501961,0.501961}%
\pgfsetstrokecolor{textcolor}%
\pgfsetfillcolor{textcolor}%
\pgftext[x=11.181814in,y=8.009588in,left,base]{\color{textcolor}\rmfamily\fontsize{14.000000}{16.800000}\selectfont Drag force}%
\end{pgfscope}%
\begin{pgfscope}%
\pgfsetrectcap%
\pgfsetroundjoin%
\pgfsetlinewidth{1.505625pt}%
\definecolor{currentstroke}{rgb}{0.000000,0.750000,0.750000}%
\pgfsetstrokecolor{currentstroke}%
\pgfsetstrokeopacity{0.750000}%
\pgfsetdash{}{0pt}%
\pgfpathmoveto{\pgfqpoint{10.637370in}{7.789490in}}%
\pgfpathlineto{\pgfqpoint{11.026259in}{7.789490in}}%
\pgfusepath{stroke}%
\end{pgfscope}%
\begin{pgfscope}%
\definecolor{textcolor}{rgb}{0.501961,0.501961,0.501961}%
\pgfsetstrokecolor{textcolor}%
\pgfsetfillcolor{textcolor}%
\pgftext[x=11.181814in,y=7.721435in,left,base]{\color{textcolor}\rmfamily\fontsize{14.000000}{16.800000}\selectfont Image force}%
\end{pgfscope}%
\begin{pgfscope}%
\pgfsetrectcap%
\pgfsetroundjoin%
\pgfsetlinewidth{1.505625pt}%
\definecolor{currentstroke}{rgb}{1.000000,0.000000,0.000000}%
\pgfsetstrokecolor{currentstroke}%
\pgfsetstrokeopacity{0.750000}%
\pgfsetdash{}{0pt}%
\pgfpathmoveto{\pgfqpoint{10.637370in}{7.501337in}}%
\pgfpathlineto{\pgfqpoint{11.026259in}{7.501337in}}%
\pgfusepath{stroke}%
\end{pgfscope}%
\begin{pgfscope}%
\definecolor{textcolor}{rgb}{0.501961,0.501961,0.501961}%
\pgfsetstrokecolor{textcolor}%
\pgfsetfillcolor{textcolor}%
\pgftext[x=11.181814in,y=7.433281in,left,base]{\color{textcolor}\rmfamily\fontsize{14.000000}{16.800000}\selectfont Coulomb force}%
\end{pgfscope}%
\begin{pgfscope}%
\pgfsetrectcap%
\pgfsetroundjoin%
\pgfsetlinewidth{1.505625pt}%
\definecolor{currentstroke}{rgb}{0.000000,0.000000,1.000000}%
\pgfsetstrokecolor{currentstroke}%
\pgfsetstrokeopacity{0.750000}%
\pgfsetdash{}{0pt}%
\pgfpathmoveto{\pgfqpoint{10.637370in}{7.215936in}}%
\pgfpathlineto{\pgfqpoint{11.026259in}{7.215936in}}%
\pgfusepath{stroke}%
\end{pgfscope}%
\begin{pgfscope}%
\definecolor{textcolor}{rgb}{0.501961,0.501961,0.501961}%
\pgfsetstrokecolor{textcolor}%
\pgfsetfillcolor{textcolor}%
\pgftext[x=11.181814in,y=7.147881in,left,base]{\color{textcolor}\rmfamily\fontsize{14.000000}{16.800000}\selectfont Drag force}%
\end{pgfscope}%
\begin{pgfscope}%
\pgfsetrectcap%
\pgfsetroundjoin%
\pgfsetlinewidth{1.505625pt}%
\definecolor{currentstroke}{rgb}{0.000000,0.750000,0.750000}%
\pgfsetstrokecolor{currentstroke}%
\pgfsetstrokeopacity{0.750000}%
\pgfsetdash{}{0pt}%
\pgfpathmoveto{\pgfqpoint{10.637370in}{6.927783in}}%
\pgfpathlineto{\pgfqpoint{11.026259in}{6.927783in}}%
\pgfusepath{stroke}%
\end{pgfscope}%
\begin{pgfscope}%
\definecolor{textcolor}{rgb}{0.501961,0.501961,0.501961}%
\pgfsetstrokecolor{textcolor}%
\pgfsetfillcolor{textcolor}%
\pgftext[x=11.181814in,y=6.859727in,left,base]{\color{textcolor}\rmfamily\fontsize{14.000000}{16.800000}\selectfont Image force}%
\end{pgfscope}%
\begin{pgfscope}%
\pgfsetrectcap%
\pgfsetroundjoin%
\pgfsetlinewidth{1.505625pt}%
\definecolor{currentstroke}{rgb}{1.000000,0.000000,0.000000}%
\pgfsetstrokecolor{currentstroke}%
\pgfsetstrokeopacity{0.750000}%
\pgfsetdash{}{0pt}%
\pgfpathmoveto{\pgfqpoint{10.637370in}{6.639629in}}%
\pgfpathlineto{\pgfqpoint{11.026259in}{6.639629in}}%
\pgfusepath{stroke}%
\end{pgfscope}%
\begin{pgfscope}%
\definecolor{textcolor}{rgb}{0.501961,0.501961,0.501961}%
\pgfsetstrokecolor{textcolor}%
\pgfsetfillcolor{textcolor}%
\pgftext[x=11.181814in,y=6.571573in,left,base]{\color{textcolor}\rmfamily\fontsize{14.000000}{16.800000}\selectfont Coulomb force}%
\end{pgfscope}%
\begin{pgfscope}%
\pgfsetrectcap%
\pgfsetroundjoin%
\pgfsetlinewidth{1.505625pt}%
\definecolor{currentstroke}{rgb}{0.000000,0.000000,1.000000}%
\pgfsetstrokecolor{currentstroke}%
\pgfsetstrokeopacity{0.750000}%
\pgfsetdash{}{0pt}%
\pgfpathmoveto{\pgfqpoint{10.637370in}{6.354229in}}%
\pgfpathlineto{\pgfqpoint{11.026259in}{6.354229in}}%
\pgfusepath{stroke}%
\end{pgfscope}%
\begin{pgfscope}%
\definecolor{textcolor}{rgb}{0.501961,0.501961,0.501961}%
\pgfsetstrokecolor{textcolor}%
\pgfsetfillcolor{textcolor}%
\pgftext[x=11.181814in,y=6.286173in,left,base]{\color{textcolor}\rmfamily\fontsize{14.000000}{16.800000}\selectfont Drag force}%
\end{pgfscope}%
\begin{pgfscope}%
\pgfsetrectcap%
\pgfsetroundjoin%
\pgfsetlinewidth{1.505625pt}%
\definecolor{currentstroke}{rgb}{0.000000,0.750000,0.750000}%
\pgfsetstrokecolor{currentstroke}%
\pgfsetstrokeopacity{0.750000}%
\pgfsetdash{}{0pt}%
\pgfpathmoveto{\pgfqpoint{10.637370in}{6.066075in}}%
\pgfpathlineto{\pgfqpoint{11.026259in}{6.066075in}}%
\pgfusepath{stroke}%
\end{pgfscope}%
\begin{pgfscope}%
\definecolor{textcolor}{rgb}{0.501961,0.501961,0.501961}%
\pgfsetstrokecolor{textcolor}%
\pgfsetfillcolor{textcolor}%
\pgftext[x=11.181814in,y=5.998020in,left,base]{\color{textcolor}\rmfamily\fontsize{14.000000}{16.800000}\selectfont Image force}%
\end{pgfscope}%
\begin{pgfscope}%
\pgfsetrectcap%
\pgfsetroundjoin%
\pgfsetlinewidth{1.505625pt}%
\definecolor{currentstroke}{rgb}{1.000000,0.000000,0.000000}%
\pgfsetstrokecolor{currentstroke}%
\pgfsetstrokeopacity{0.750000}%
\pgfsetdash{}{0pt}%
\pgfpathmoveto{\pgfqpoint{10.637370in}{5.777922in}}%
\pgfpathlineto{\pgfqpoint{11.026259in}{5.777922in}}%
\pgfusepath{stroke}%
\end{pgfscope}%
\begin{pgfscope}%
\definecolor{textcolor}{rgb}{0.501961,0.501961,0.501961}%
\pgfsetstrokecolor{textcolor}%
\pgfsetfillcolor{textcolor}%
\pgftext[x=11.181814in,y=5.709866in,left,base]{\color{textcolor}\rmfamily\fontsize{14.000000}{16.800000}\selectfont Coulomb force}%
\end{pgfscope}%
\begin{pgfscope}%
\pgfsetrectcap%
\pgfsetroundjoin%
\pgfsetlinewidth{1.505625pt}%
\definecolor{currentstroke}{rgb}{0.000000,0.000000,1.000000}%
\pgfsetstrokecolor{currentstroke}%
\pgfsetstrokeopacity{0.750000}%
\pgfsetdash{}{0pt}%
\pgfpathmoveto{\pgfqpoint{10.637370in}{5.492521in}}%
\pgfpathlineto{\pgfqpoint{11.026259in}{5.492521in}}%
\pgfusepath{stroke}%
\end{pgfscope}%
\begin{pgfscope}%
\definecolor{textcolor}{rgb}{0.501961,0.501961,0.501961}%
\pgfsetstrokecolor{textcolor}%
\pgfsetfillcolor{textcolor}%
\pgftext[x=11.181814in,y=5.424466in,left,base]{\color{textcolor}\rmfamily\fontsize{14.000000}{16.800000}\selectfont Drag force}%
\end{pgfscope}%
\begin{pgfscope}%
\pgfsetrectcap%
\pgfsetroundjoin%
\pgfsetlinewidth{1.505625pt}%
\definecolor{currentstroke}{rgb}{0.000000,0.750000,0.750000}%
\pgfsetstrokecolor{currentstroke}%
\pgfsetstrokeopacity{0.750000}%
\pgfsetdash{}{0pt}%
\pgfpathmoveto{\pgfqpoint{10.637370in}{5.204368in}}%
\pgfpathlineto{\pgfqpoint{11.026259in}{5.204368in}}%
\pgfusepath{stroke}%
\end{pgfscope}%
\begin{pgfscope}%
\definecolor{textcolor}{rgb}{0.501961,0.501961,0.501961}%
\pgfsetstrokecolor{textcolor}%
\pgfsetfillcolor{textcolor}%
\pgftext[x=11.181814in,y=5.136312in,left,base]{\color{textcolor}\rmfamily\fontsize{14.000000}{16.800000}\selectfont Image force}%
\end{pgfscope}%
\begin{pgfscope}%
\pgfsetrectcap%
\pgfsetroundjoin%
\pgfsetlinewidth{1.505625pt}%
\definecolor{currentstroke}{rgb}{1.000000,0.000000,0.000000}%
\pgfsetstrokecolor{currentstroke}%
\pgfsetstrokeopacity{0.750000}%
\pgfsetdash{}{0pt}%
\pgfpathmoveto{\pgfqpoint{10.637370in}{4.916214in}}%
\pgfpathlineto{\pgfqpoint{11.026259in}{4.916214in}}%
\pgfusepath{stroke}%
\end{pgfscope}%
\begin{pgfscope}%
\definecolor{textcolor}{rgb}{0.501961,0.501961,0.501961}%
\pgfsetstrokecolor{textcolor}%
\pgfsetfillcolor{textcolor}%
\pgftext[x=11.181814in,y=4.848158in,left,base]{\color{textcolor}\rmfamily\fontsize{14.000000}{16.800000}\selectfont Coulomb force}%
\end{pgfscope}%
\begin{pgfscope}%
\pgfsetrectcap%
\pgfsetroundjoin%
\pgfsetlinewidth{1.505625pt}%
\definecolor{currentstroke}{rgb}{0.000000,0.000000,1.000000}%
\pgfsetstrokecolor{currentstroke}%
\pgfsetstrokeopacity{0.750000}%
\pgfsetdash{}{0pt}%
\pgfpathmoveto{\pgfqpoint{10.637370in}{4.630814in}}%
\pgfpathlineto{\pgfqpoint{11.026259in}{4.630814in}}%
\pgfusepath{stroke}%
\end{pgfscope}%
\begin{pgfscope}%
\definecolor{textcolor}{rgb}{0.501961,0.501961,0.501961}%
\pgfsetstrokecolor{textcolor}%
\pgfsetfillcolor{textcolor}%
\pgftext[x=11.181814in,y=4.562758in,left,base]{\color{textcolor}\rmfamily\fontsize{14.000000}{16.800000}\selectfont Drag force}%
\end{pgfscope}%
\begin{pgfscope}%
\pgfsetrectcap%
\pgfsetroundjoin%
\pgfsetlinewidth{1.505625pt}%
\definecolor{currentstroke}{rgb}{0.000000,0.750000,0.750000}%
\pgfsetstrokecolor{currentstroke}%
\pgfsetstrokeopacity{0.750000}%
\pgfsetdash{}{0pt}%
\pgfpathmoveto{\pgfqpoint{10.637370in}{4.342660in}}%
\pgfpathlineto{\pgfqpoint{11.026259in}{4.342660in}}%
\pgfusepath{stroke}%
\end{pgfscope}%
\begin{pgfscope}%
\definecolor{textcolor}{rgb}{0.501961,0.501961,0.501961}%
\pgfsetstrokecolor{textcolor}%
\pgfsetfillcolor{textcolor}%
\pgftext[x=11.181814in,y=4.274605in,left,base]{\color{textcolor}\rmfamily\fontsize{14.000000}{16.800000}\selectfont Image force}%
\end{pgfscope}%
\begin{pgfscope}%
\definecolor{textcolor}{rgb}{0.501961,0.501961,0.501961}%
\pgfsetstrokecolor{textcolor}%
\pgfsetfillcolor{textcolor}%
\pgftext[x=5.380000in,y=0.140446in,,base]{\color{textcolor}\rmfamily\fontsize{14.000000}{16.800000}\selectfont \(\displaystyle t\) (s)}%
\end{pgfscope}%
\begin{pgfscope}%
\definecolor{textcolor}{rgb}{0.501961,0.501961,0.501961}%
\pgfsetstrokecolor{textcolor}%
\pgfsetfillcolor{textcolor}%
\pgftext[x=0.247732in,y=4.069478in,left,base,rotate=90.000000]{\color{textcolor}\rmfamily\fontsize{14.000000}{16.800000}\selectfont Force (N)}%
\end{pgfscope}%
\end{pgfpicture}%
\makeatother%
\endgroup%
}
    \caption{A simple EMA plot.\label{fig:forces}}
\end{figure}

\begin{figure}[htb]
    \centering
    %% Creator: Matplotlib, PGF backend
%%
%% To include the figure in your LaTeX document, write
%%   \input{<filename>.pgf}
%%
%% Make sure the required packages are loaded in your preamble
%%   \usepackage{pgf}
%%
%% Figures using additional raster images can only be included by \input if
%% they are in the same directory as the main LaTeX file. For loading figures
%% from other directories you can use the `import` package
%%   \usepackage{import}
%% and then include the figures with
%%   \import{<path to file>}{<filename>.pgf}
%%
%% Matplotlib used the following preamble
%%   \usepackage{fontspec}
%%   \setmainfont{DejaVu Serif}
%%   \setsansfont{DejaVu Sans}
%%   \setmonofont{DejaVu Sans Mono}
%%
\begingroup%
\makeatletter%
\begin{pgfpicture}%
\pgfpathrectangle{\pgfpointorigin}{\pgfqpoint{5.437191in}{3.676603in}}%
\pgfusepath{use as bounding box, clip}%
\begin{pgfscope}%
\pgfsetbuttcap%
\pgfsetmiterjoin%
\definecolor{currentfill}{rgb}{1.000000,1.000000,1.000000}%
\pgfsetfillcolor{currentfill}%
\pgfsetlinewidth{0.000000pt}%
\definecolor{currentstroke}{rgb}{1.000000,1.000000,1.000000}%
\pgfsetstrokecolor{currentstroke}%
\pgfsetdash{}{0pt}%
\pgfpathmoveto{\pgfqpoint{0.000000in}{0.000000in}}%
\pgfpathlineto{\pgfqpoint{5.437191in}{0.000000in}}%
\pgfpathlineto{\pgfqpoint{5.437191in}{3.676603in}}%
\pgfpathlineto{\pgfqpoint{0.000000in}{3.676603in}}%
\pgfpathclose%
\pgfusepath{fill}%
\end{pgfscope}%
\begin{pgfscope}%
\pgfsetbuttcap%
\pgfsetmiterjoin%
\definecolor{currentfill}{rgb}{1.000000,1.000000,1.000000}%
\pgfsetfillcolor{currentfill}%
\pgfsetlinewidth{0.000000pt}%
\definecolor{currentstroke}{rgb}{0.000000,0.000000,0.000000}%
\pgfsetstrokecolor{currentstroke}%
\pgfsetstrokeopacity{0.000000}%
\pgfsetdash{}{0pt}%
\pgfpathmoveto{\pgfqpoint{0.574151in}{0.521603in}}%
\pgfpathlineto{\pgfqpoint{4.294151in}{0.521603in}}%
\pgfpathlineto{\pgfqpoint{4.294151in}{3.541603in}}%
\pgfpathlineto{\pgfqpoint{0.574151in}{3.541603in}}%
\pgfpathclose%
\pgfusepath{fill}%
\end{pgfscope}%
\begin{pgfscope}%
\pgfpathrectangle{\pgfqpoint{0.574151in}{0.521603in}}{\pgfqpoint{3.720000in}{3.020000in}} %
\pgfusepath{clip}%
\pgfsetbuttcap%
\pgfsetroundjoin%
\definecolor{currentfill}{rgb}{0.441176,0.995734,0.739009}%
\pgfsetfillcolor{currentfill}%
\pgfsetlinewidth{1.003750pt}%
\definecolor{currentstroke}{rgb}{0.441176,0.995734,0.739009}%
\pgfsetstrokecolor{currentstroke}%
\pgfsetdash{}{0pt}%
\pgfpathmoveto{\pgfqpoint{3.151854in}{1.797941in}}%
\pgfpathcurveto{\pgfqpoint{3.162905in}{1.797941in}}{\pgfqpoint{3.173504in}{1.802331in}}{\pgfqpoint{3.181317in}{1.810145in}}%
\pgfpathcurveto{\pgfqpoint{3.189131in}{1.817958in}}{\pgfqpoint{3.193521in}{1.828557in}}{\pgfqpoint{3.193521in}{1.839607in}}%
\pgfpathcurveto{\pgfqpoint{3.193521in}{1.850657in}}{\pgfqpoint{3.189131in}{1.861257in}}{\pgfqpoint{3.181317in}{1.869070in}}%
\pgfpathcurveto{\pgfqpoint{3.173504in}{1.876884in}}{\pgfqpoint{3.162905in}{1.881274in}}{\pgfqpoint{3.151854in}{1.881274in}}%
\pgfpathcurveto{\pgfqpoint{3.140804in}{1.881274in}}{\pgfqpoint{3.130205in}{1.876884in}}{\pgfqpoint{3.122392in}{1.869070in}}%
\pgfpathcurveto{\pgfqpoint{3.114578in}{1.861257in}}{\pgfqpoint{3.110188in}{1.850657in}}{\pgfqpoint{3.110188in}{1.839607in}}%
\pgfpathcurveto{\pgfqpoint{3.110188in}{1.828557in}}{\pgfqpoint{3.114578in}{1.817958in}}{\pgfqpoint{3.122392in}{1.810145in}}%
\pgfpathcurveto{\pgfqpoint{3.130205in}{1.802331in}}{\pgfqpoint{3.140804in}{1.797941in}}{\pgfqpoint{3.151854in}{1.797941in}}%
\pgfpathclose%
\pgfusepath{stroke,fill}%
\end{pgfscope}%
\begin{pgfscope}%
\pgfpathrectangle{\pgfqpoint{0.574151in}{0.521603in}}{\pgfqpoint{3.720000in}{3.020000in}} %
\pgfusepath{clip}%
\pgfsetbuttcap%
\pgfsetroundjoin%
\definecolor{currentfill}{rgb}{0.209804,0.440216,0.974139}%
\pgfsetfillcolor{currentfill}%
\pgfsetlinewidth{1.003750pt}%
\definecolor{currentstroke}{rgb}{0.209804,0.440216,0.974139}%
\pgfsetstrokecolor{currentstroke}%
\pgfsetdash{}{0pt}%
\pgfpathmoveto{\pgfqpoint{3.179149in}{2.208860in}}%
\pgfpathcurveto{\pgfqpoint{3.190199in}{2.208860in}}{\pgfqpoint{3.200798in}{2.213251in}}{\pgfqpoint{3.208612in}{2.221064in}}%
\pgfpathcurveto{\pgfqpoint{3.216426in}{2.228878in}}{\pgfqpoint{3.220816in}{2.239477in}}{\pgfqpoint{3.220816in}{2.250527in}}%
\pgfpathcurveto{\pgfqpoint{3.220816in}{2.261577in}}{\pgfqpoint{3.216426in}{2.272176in}}{\pgfqpoint{3.208612in}{2.279990in}}%
\pgfpathcurveto{\pgfqpoint{3.200798in}{2.287803in}}{\pgfqpoint{3.190199in}{2.292194in}}{\pgfqpoint{3.179149in}{2.292194in}}%
\pgfpathcurveto{\pgfqpoint{3.168099in}{2.292194in}}{\pgfqpoint{3.157500in}{2.287803in}}{\pgfqpoint{3.149686in}{2.279990in}}%
\pgfpathcurveto{\pgfqpoint{3.141873in}{2.272176in}}{\pgfqpoint{3.137483in}{2.261577in}}{\pgfqpoint{3.137483in}{2.250527in}}%
\pgfpathcurveto{\pgfqpoint{3.137483in}{2.239477in}}{\pgfqpoint{3.141873in}{2.228878in}}{\pgfqpoint{3.149686in}{2.221064in}}%
\pgfpathcurveto{\pgfqpoint{3.157500in}{2.213251in}}{\pgfqpoint{3.168099in}{2.208860in}}{\pgfqpoint{3.179149in}{2.208860in}}%
\pgfpathclose%
\pgfusepath{stroke,fill}%
\end{pgfscope}%
\begin{pgfscope}%
\pgfpathrectangle{\pgfqpoint{0.574151in}{0.521603in}}{\pgfqpoint{3.720000in}{3.020000in}} %
\pgfusepath{clip}%
\pgfsetbuttcap%
\pgfsetroundjoin%
\definecolor{currentfill}{rgb}{0.178431,0.483911,0.968276}%
\pgfsetfillcolor{currentfill}%
\pgfsetlinewidth{1.003750pt}%
\definecolor{currentstroke}{rgb}{0.178431,0.483911,0.968276}%
\pgfsetstrokecolor{currentstroke}%
\pgfsetdash{}{0pt}%
\pgfpathmoveto{\pgfqpoint{2.797680in}{3.329817in}}%
\pgfpathcurveto{\pgfqpoint{2.808730in}{3.329817in}}{\pgfqpoint{2.819329in}{3.334207in}}{\pgfqpoint{2.827142in}{3.342021in}}%
\pgfpathcurveto{\pgfqpoint{2.834956in}{3.349835in}}{\pgfqpoint{2.839346in}{3.360434in}}{\pgfqpoint{2.839346in}{3.371484in}}%
\pgfpathcurveto{\pgfqpoint{2.839346in}{3.382534in}}{\pgfqpoint{2.834956in}{3.393133in}}{\pgfqpoint{2.827142in}{3.400947in}}%
\pgfpathcurveto{\pgfqpoint{2.819329in}{3.408760in}}{\pgfqpoint{2.808730in}{3.413151in}}{\pgfqpoint{2.797680in}{3.413151in}}%
\pgfpathcurveto{\pgfqpoint{2.786629in}{3.413151in}}{\pgfqpoint{2.776030in}{3.408760in}}{\pgfqpoint{2.768217in}{3.400947in}}%
\pgfpathcurveto{\pgfqpoint{2.760403in}{3.393133in}}{\pgfqpoint{2.756013in}{3.382534in}}{\pgfqpoint{2.756013in}{3.371484in}}%
\pgfpathcurveto{\pgfqpoint{2.756013in}{3.360434in}}{\pgfqpoint{2.760403in}{3.349835in}}{\pgfqpoint{2.768217in}{3.342021in}}%
\pgfpathcurveto{\pgfqpoint{2.776030in}{3.334207in}}{\pgfqpoint{2.786629in}{3.329817in}}{\pgfqpoint{2.797680in}{3.329817in}}%
\pgfpathclose%
\pgfusepath{stroke,fill}%
\end{pgfscope}%
\begin{pgfscope}%
\pgfpathrectangle{\pgfqpoint{0.574151in}{0.521603in}}{\pgfqpoint{3.720000in}{3.020000in}} %
\pgfusepath{clip}%
\pgfsetbuttcap%
\pgfsetroundjoin%
\definecolor{currentfill}{rgb}{0.468627,0.049260,0.999696}%
\pgfsetfillcolor{currentfill}%
\pgfsetlinewidth{1.003750pt}%
\definecolor{currentstroke}{rgb}{0.468627,0.049260,0.999696}%
\pgfsetstrokecolor{currentstroke}%
\pgfsetdash{}{0pt}%
\pgfpathmoveto{\pgfqpoint{2.011631in}{3.351296in}}%
\pgfpathcurveto{\pgfqpoint{2.022681in}{3.351296in}}{\pgfqpoint{2.033280in}{3.355687in}}{\pgfqpoint{2.041094in}{3.363500in}}%
\pgfpathcurveto{\pgfqpoint{2.048907in}{3.371314in}}{\pgfqpoint{2.053297in}{3.381913in}}{\pgfqpoint{2.053297in}{3.392963in}}%
\pgfpathcurveto{\pgfqpoint{2.053297in}{3.404013in}}{\pgfqpoint{2.048907in}{3.414612in}}{\pgfqpoint{2.041094in}{3.422426in}}%
\pgfpathcurveto{\pgfqpoint{2.033280in}{3.430239in}}{\pgfqpoint{2.022681in}{3.434630in}}{\pgfqpoint{2.011631in}{3.434630in}}%
\pgfpathcurveto{\pgfqpoint{2.000581in}{3.434630in}}{\pgfqpoint{1.989982in}{3.430239in}}{\pgfqpoint{1.982168in}{3.422426in}}%
\pgfpathcurveto{\pgfqpoint{1.974354in}{3.414612in}}{\pgfqpoint{1.969964in}{3.404013in}}{\pgfqpoint{1.969964in}{3.392963in}}%
\pgfpathcurveto{\pgfqpoint{1.969964in}{3.381913in}}{\pgfqpoint{1.974354in}{3.371314in}}{\pgfqpoint{1.982168in}{3.363500in}}%
\pgfpathcurveto{\pgfqpoint{1.989982in}{3.355687in}}{\pgfqpoint{2.000581in}{3.351296in}}{\pgfqpoint{2.011631in}{3.351296in}}%
\pgfpathclose%
\pgfusepath{stroke,fill}%
\end{pgfscope}%
\begin{pgfscope}%
\pgfpathrectangle{\pgfqpoint{0.574151in}{0.521603in}}{\pgfqpoint{3.720000in}{3.020000in}} %
\pgfusepath{clip}%
\pgfsetbuttcap%
\pgfsetroundjoin%
\definecolor{currentfill}{rgb}{1.000000,0.505325,0.261793}%
\pgfsetfillcolor{currentfill}%
\pgfsetlinewidth{1.003750pt}%
\definecolor{currentstroke}{rgb}{1.000000,0.505325,0.261793}%
\pgfsetstrokecolor{currentstroke}%
\pgfsetdash{}{0pt}%
\pgfpathmoveto{\pgfqpoint{3.917459in}{0.946599in}}%
\pgfpathcurveto{\pgfqpoint{3.928509in}{0.946599in}}{\pgfqpoint{3.939108in}{0.950989in}}{\pgfqpoint{3.946922in}{0.958803in}}%
\pgfpathcurveto{\pgfqpoint{3.954735in}{0.966616in}}{\pgfqpoint{3.959125in}{0.977215in}}{\pgfqpoint{3.959125in}{0.988265in}}%
\pgfpathcurveto{\pgfqpoint{3.959125in}{0.999315in}}{\pgfqpoint{3.954735in}{1.009914in}}{\pgfqpoint{3.946922in}{1.017728in}}%
\pgfpathcurveto{\pgfqpoint{3.939108in}{1.025542in}}{\pgfqpoint{3.928509in}{1.029932in}}{\pgfqpoint{3.917459in}{1.029932in}}%
\pgfpathcurveto{\pgfqpoint{3.906409in}{1.029932in}}{\pgfqpoint{3.895810in}{1.025542in}}{\pgfqpoint{3.887996in}{1.017728in}}%
\pgfpathcurveto{\pgfqpoint{3.880182in}{1.009914in}}{\pgfqpoint{3.875792in}{0.999315in}}{\pgfqpoint{3.875792in}{0.988265in}}%
\pgfpathcurveto{\pgfqpoint{3.875792in}{0.977215in}}{\pgfqpoint{3.880182in}{0.966616in}}{\pgfqpoint{3.887996in}{0.958803in}}%
\pgfpathcurveto{\pgfqpoint{3.895810in}{0.950989in}}{\pgfqpoint{3.906409in}{0.946599in}}{\pgfqpoint{3.917459in}{0.946599in}}%
\pgfpathclose%
\pgfusepath{stroke,fill}%
\end{pgfscope}%
\begin{pgfscope}%
\pgfpathrectangle{\pgfqpoint{0.574151in}{0.521603in}}{\pgfqpoint{3.720000in}{3.020000in}} %
\pgfusepath{clip}%
\pgfsetbuttcap%
\pgfsetroundjoin%
\definecolor{currentfill}{rgb}{1.000000,0.505325,0.261793}%
\pgfsetfillcolor{currentfill}%
\pgfsetlinewidth{1.003750pt}%
\definecolor{currentstroke}{rgb}{1.000000,0.505325,0.261793}%
\pgfsetstrokecolor{currentstroke}%
\pgfsetdash{}{0pt}%
\pgfpathmoveto{\pgfqpoint{3.571021in}{0.946599in}}%
\pgfpathcurveto{\pgfqpoint{3.582071in}{0.946599in}}{\pgfqpoint{3.592670in}{0.950989in}}{\pgfqpoint{3.600484in}{0.958803in}}%
\pgfpathcurveto{\pgfqpoint{3.608298in}{0.966616in}}{\pgfqpoint{3.612688in}{0.977215in}}{\pgfqpoint{3.612688in}{0.988265in}}%
\pgfpathcurveto{\pgfqpoint{3.612688in}{0.999315in}}{\pgfqpoint{3.608298in}{1.009914in}}{\pgfqpoint{3.600484in}{1.017728in}}%
\pgfpathcurveto{\pgfqpoint{3.592670in}{1.025542in}}{\pgfqpoint{3.582071in}{1.029932in}}{\pgfqpoint{3.571021in}{1.029932in}}%
\pgfpathcurveto{\pgfqpoint{3.559971in}{1.029932in}}{\pgfqpoint{3.549372in}{1.025542in}}{\pgfqpoint{3.541558in}{1.017728in}}%
\pgfpathcurveto{\pgfqpoint{3.533745in}{1.009914in}}{\pgfqpoint{3.529355in}{0.999315in}}{\pgfqpoint{3.529355in}{0.988265in}}%
\pgfpathcurveto{\pgfqpoint{3.529355in}{0.977215in}}{\pgfqpoint{3.533745in}{0.966616in}}{\pgfqpoint{3.541558in}{0.958803in}}%
\pgfpathcurveto{\pgfqpoint{3.549372in}{0.950989in}}{\pgfqpoint{3.559971in}{0.946599in}}{\pgfqpoint{3.571021in}{0.946599in}}%
\pgfpathclose%
\pgfusepath{stroke,fill}%
\end{pgfscope}%
\begin{pgfscope}%
\pgfpathrectangle{\pgfqpoint{0.574151in}{0.521603in}}{\pgfqpoint{3.720000in}{3.020000in}} %
\pgfusepath{clip}%
\pgfsetbuttcap%
\pgfsetroundjoin%
\definecolor{currentfill}{rgb}{1.000000,0.645928,0.343949}%
\pgfsetfillcolor{currentfill}%
\pgfsetlinewidth{1.003750pt}%
\definecolor{currentstroke}{rgb}{1.000000,0.645928,0.343949}%
\pgfsetstrokecolor{currentstroke}%
\pgfsetdash{}{0pt}%
\pgfpathmoveto{\pgfqpoint{3.526117in}{1.586433in}}%
\pgfpathcurveto{\pgfqpoint{3.537167in}{1.586433in}}{\pgfqpoint{3.547766in}{1.590823in}}{\pgfqpoint{3.555580in}{1.598637in}}%
\pgfpathcurveto{\pgfqpoint{3.563393in}{1.606450in}}{\pgfqpoint{3.567784in}{1.617049in}}{\pgfqpoint{3.567784in}{1.628099in}}%
\pgfpathcurveto{\pgfqpoint{3.567784in}{1.639149in}}{\pgfqpoint{3.563393in}{1.649749in}}{\pgfqpoint{3.555580in}{1.657562in}}%
\pgfpathcurveto{\pgfqpoint{3.547766in}{1.665376in}}{\pgfqpoint{3.537167in}{1.669766in}}{\pgfqpoint{3.526117in}{1.669766in}}%
\pgfpathcurveto{\pgfqpoint{3.515067in}{1.669766in}}{\pgfqpoint{3.504468in}{1.665376in}}{\pgfqpoint{3.496654in}{1.657562in}}%
\pgfpathcurveto{\pgfqpoint{3.488841in}{1.649749in}}{\pgfqpoint{3.484450in}{1.639149in}}{\pgfqpoint{3.484450in}{1.628099in}}%
\pgfpathcurveto{\pgfqpoint{3.484450in}{1.617049in}}{\pgfqpoint{3.488841in}{1.606450in}}{\pgfqpoint{3.496654in}{1.598637in}}%
\pgfpathcurveto{\pgfqpoint{3.504468in}{1.590823in}}{\pgfqpoint{3.515067in}{1.586433in}}{\pgfqpoint{3.526117in}{1.586433in}}%
\pgfpathclose%
\pgfusepath{stroke,fill}%
\end{pgfscope}%
\begin{pgfscope}%
\pgfpathrectangle{\pgfqpoint{0.574151in}{0.521603in}}{\pgfqpoint{3.720000in}{3.020000in}} %
\pgfusepath{clip}%
\pgfsetbuttcap%
\pgfsetroundjoin%
\definecolor{currentfill}{rgb}{1.000000,0.645928,0.343949}%
\pgfsetfillcolor{currentfill}%
\pgfsetlinewidth{1.003750pt}%
\definecolor{currentstroke}{rgb}{1.000000,0.645928,0.343949}%
\pgfsetstrokecolor{currentstroke}%
\pgfsetdash{}{0pt}%
\pgfpathmoveto{\pgfqpoint{3.302686in}{2.167561in}}%
\pgfpathcurveto{\pgfqpoint{3.313736in}{2.167561in}}{\pgfqpoint{3.324335in}{2.171951in}}{\pgfqpoint{3.332149in}{2.179765in}}%
\pgfpathcurveto{\pgfqpoint{3.339962in}{2.187578in}}{\pgfqpoint{3.344352in}{2.198177in}}{\pgfqpoint{3.344352in}{2.209228in}}%
\pgfpathcurveto{\pgfqpoint{3.344352in}{2.220278in}}{\pgfqpoint{3.339962in}{2.230877in}}{\pgfqpoint{3.332149in}{2.238690in}}%
\pgfpathcurveto{\pgfqpoint{3.324335in}{2.246504in}}{\pgfqpoint{3.313736in}{2.250894in}}{\pgfqpoint{3.302686in}{2.250894in}}%
\pgfpathcurveto{\pgfqpoint{3.291636in}{2.250894in}}{\pgfqpoint{3.281037in}{2.246504in}}{\pgfqpoint{3.273223in}{2.238690in}}%
\pgfpathcurveto{\pgfqpoint{3.265409in}{2.230877in}}{\pgfqpoint{3.261019in}{2.220278in}}{\pgfqpoint{3.261019in}{2.209228in}}%
\pgfpathcurveto{\pgfqpoint{3.261019in}{2.198177in}}{\pgfqpoint{3.265409in}{2.187578in}}{\pgfqpoint{3.273223in}{2.179765in}}%
\pgfpathcurveto{\pgfqpoint{3.281037in}{2.171951in}}{\pgfqpoint{3.291636in}{2.167561in}}{\pgfqpoint{3.302686in}{2.167561in}}%
\pgfpathclose%
\pgfusepath{stroke,fill}%
\end{pgfscope}%
\begin{pgfscope}%
\pgfpathrectangle{\pgfqpoint{0.574151in}{0.521603in}}{\pgfqpoint{3.720000in}{3.020000in}} %
\pgfusepath{clip}%
\pgfsetbuttcap%
\pgfsetroundjoin%
\definecolor{currentfill}{rgb}{0.739216,0.930229,0.562593}%
\pgfsetfillcolor{currentfill}%
\pgfsetlinewidth{1.003750pt}%
\definecolor{currentstroke}{rgb}{0.739216,0.930229,0.562593}%
\pgfsetstrokecolor{currentstroke}%
\pgfsetdash{}{0pt}%
\pgfpathmoveto{\pgfqpoint{3.115516in}{1.790323in}}%
\pgfpathcurveto{\pgfqpoint{3.126566in}{1.790323in}}{\pgfqpoint{3.137165in}{1.794713in}}{\pgfqpoint{3.144979in}{1.802527in}}%
\pgfpathcurveto{\pgfqpoint{3.152793in}{1.810341in}}{\pgfqpoint{3.157183in}{1.820940in}}{\pgfqpoint{3.157183in}{1.831990in}}%
\pgfpathcurveto{\pgfqpoint{3.157183in}{1.843040in}}{\pgfqpoint{3.152793in}{1.853639in}}{\pgfqpoint{3.144979in}{1.861453in}}%
\pgfpathcurveto{\pgfqpoint{3.137165in}{1.869266in}}{\pgfqpoint{3.126566in}{1.873657in}}{\pgfqpoint{3.115516in}{1.873657in}}%
\pgfpathcurveto{\pgfqpoint{3.104466in}{1.873657in}}{\pgfqpoint{3.093867in}{1.869266in}}{\pgfqpoint{3.086053in}{1.861453in}}%
\pgfpathcurveto{\pgfqpoint{3.078240in}{1.853639in}}{\pgfqpoint{3.073850in}{1.843040in}}{\pgfqpoint{3.073850in}{1.831990in}}%
\pgfpathcurveto{\pgfqpoint{3.073850in}{1.820940in}}{\pgfqpoint{3.078240in}{1.810341in}}{\pgfqpoint{3.086053in}{1.802527in}}%
\pgfpathcurveto{\pgfqpoint{3.093867in}{1.794713in}}{\pgfqpoint{3.104466in}{1.790323in}}{\pgfqpoint{3.115516in}{1.790323in}}%
\pgfpathclose%
\pgfusepath{stroke,fill}%
\end{pgfscope}%
\begin{pgfscope}%
\pgfpathrectangle{\pgfqpoint{0.574151in}{0.521603in}}{\pgfqpoint{3.720000in}{3.020000in}} %
\pgfusepath{clip}%
\pgfsetbuttcap%
\pgfsetroundjoin%
\definecolor{currentfill}{rgb}{0.500000,0.000000,1.000000}%
\pgfsetfillcolor{currentfill}%
\pgfsetlinewidth{1.003750pt}%
\definecolor{currentstroke}{rgb}{0.500000,0.000000,1.000000}%
\pgfsetstrokecolor{currentstroke}%
\pgfsetdash{}{0pt}%
\pgfpathmoveto{\pgfqpoint{3.074734in}{2.208860in}}%
\pgfpathcurveto{\pgfqpoint{3.085784in}{2.208860in}}{\pgfqpoint{3.096383in}{2.213251in}}{\pgfqpoint{3.104197in}{2.221064in}}%
\pgfpathcurveto{\pgfqpoint{3.112011in}{2.228878in}}{\pgfqpoint{3.116401in}{2.239477in}}{\pgfqpoint{3.116401in}{2.250527in}}%
\pgfpathcurveto{\pgfqpoint{3.116401in}{2.261577in}}{\pgfqpoint{3.112011in}{2.272176in}}{\pgfqpoint{3.104197in}{2.279990in}}%
\pgfpathcurveto{\pgfqpoint{3.096383in}{2.287803in}}{\pgfqpoint{3.085784in}{2.292194in}}{\pgfqpoint{3.074734in}{2.292194in}}%
\pgfpathcurveto{\pgfqpoint{3.063684in}{2.292194in}}{\pgfqpoint{3.053085in}{2.287803in}}{\pgfqpoint{3.045271in}{2.279990in}}%
\pgfpathcurveto{\pgfqpoint{3.037458in}{2.272176in}}{\pgfqpoint{3.033068in}{2.261577in}}{\pgfqpoint{3.033068in}{2.250527in}}%
\pgfpathcurveto{\pgfqpoint{3.033068in}{2.239477in}}{\pgfqpoint{3.037458in}{2.228878in}}{\pgfqpoint{3.045271in}{2.221064in}}%
\pgfpathcurveto{\pgfqpoint{3.053085in}{2.213251in}}{\pgfqpoint{3.063684in}{2.208860in}}{\pgfqpoint{3.074734in}{2.208860in}}%
\pgfpathclose%
\pgfusepath{stroke,fill}%
\end{pgfscope}%
\begin{pgfscope}%
\pgfpathrectangle{\pgfqpoint{0.574151in}{0.521603in}}{\pgfqpoint{3.720000in}{3.020000in}} %
\pgfusepath{clip}%
\pgfsetbuttcap%
\pgfsetroundjoin%
\definecolor{currentfill}{rgb}{0.319608,0.279583,0.989980}%
\pgfsetfillcolor{currentfill}%
\pgfsetlinewidth{1.003750pt}%
\definecolor{currentstroke}{rgb}{0.319608,0.279583,0.989980}%
\pgfsetstrokecolor{currentstroke}%
\pgfsetdash{}{0pt}%
\pgfpathmoveto{\pgfqpoint{2.630750in}{1.387021in}}%
\pgfpathcurveto{\pgfqpoint{2.641800in}{1.387021in}}{\pgfqpoint{2.652399in}{1.391411in}}{\pgfqpoint{2.660213in}{1.399225in}}%
\pgfpathcurveto{\pgfqpoint{2.668027in}{1.407039in}}{\pgfqpoint{2.672417in}{1.417638in}}{\pgfqpoint{2.672417in}{1.428688in}}%
\pgfpathcurveto{\pgfqpoint{2.672417in}{1.439738in}}{\pgfqpoint{2.668027in}{1.450337in}}{\pgfqpoint{2.660213in}{1.458150in}}%
\pgfpathcurveto{\pgfqpoint{2.652399in}{1.465964in}}{\pgfqpoint{2.641800in}{1.470354in}}{\pgfqpoint{2.630750in}{1.470354in}}%
\pgfpathcurveto{\pgfqpoint{2.619700in}{1.470354in}}{\pgfqpoint{2.609101in}{1.465964in}}{\pgfqpoint{2.601288in}{1.458150in}}%
\pgfpathcurveto{\pgfqpoint{2.593474in}{1.450337in}}{\pgfqpoint{2.589084in}{1.439738in}}{\pgfqpoint{2.589084in}{1.428688in}}%
\pgfpathcurveto{\pgfqpoint{2.589084in}{1.417638in}}{\pgfqpoint{2.593474in}{1.407039in}}{\pgfqpoint{2.601288in}{1.399225in}}%
\pgfpathcurveto{\pgfqpoint{2.609101in}{1.391411in}}{\pgfqpoint{2.619700in}{1.387021in}}{\pgfqpoint{2.630750in}{1.387021in}}%
\pgfpathclose%
\pgfusepath{stroke,fill}%
\end{pgfscope}%
\begin{pgfscope}%
\pgfpathrectangle{\pgfqpoint{0.574151in}{0.521603in}}{\pgfqpoint{3.720000in}{3.020000in}} %
\pgfusepath{clip}%
\pgfsetbuttcap%
\pgfsetroundjoin%
\definecolor{currentfill}{rgb}{0.319608,0.279583,0.989980}%
\pgfsetfillcolor{currentfill}%
\pgfsetlinewidth{1.003750pt}%
\definecolor{currentstroke}{rgb}{0.319608,0.279583,0.989980}%
\pgfsetstrokecolor{currentstroke}%
\pgfsetdash{}{0pt}%
\pgfpathmoveto{\pgfqpoint{2.181758in}{0.976101in}}%
\pgfpathcurveto{\pgfqpoint{2.192808in}{0.976101in}}{\pgfqpoint{2.203407in}{0.980492in}}{\pgfqpoint{2.211220in}{0.988305in}}%
\pgfpathcurveto{\pgfqpoint{2.219034in}{0.996119in}}{\pgfqpoint{2.223424in}{1.006718in}}{\pgfqpoint{2.223424in}{1.017768in}}%
\pgfpathcurveto{\pgfqpoint{2.223424in}{1.028818in}}{\pgfqpoint{2.219034in}{1.039417in}}{\pgfqpoint{2.211220in}{1.047231in}}%
\pgfpathcurveto{\pgfqpoint{2.203407in}{1.055044in}}{\pgfqpoint{2.192808in}{1.059435in}}{\pgfqpoint{2.181758in}{1.059435in}}%
\pgfpathcurveto{\pgfqpoint{2.170707in}{1.059435in}}{\pgfqpoint{2.160108in}{1.055044in}}{\pgfqpoint{2.152295in}{1.047231in}}%
\pgfpathcurveto{\pgfqpoint{2.144481in}{1.039417in}}{\pgfqpoint{2.140091in}{1.028818in}}{\pgfqpoint{2.140091in}{1.017768in}}%
\pgfpathcurveto{\pgfqpoint{2.140091in}{1.006718in}}{\pgfqpoint{2.144481in}{0.996119in}}{\pgfqpoint{2.152295in}{0.988305in}}%
\pgfpathcurveto{\pgfqpoint{2.160108in}{0.980492in}}{\pgfqpoint{2.170707in}{0.976101in}}{\pgfqpoint{2.181758in}{0.976101in}}%
\pgfpathclose%
\pgfusepath{stroke,fill}%
\end{pgfscope}%
\begin{pgfscope}%
\pgfpathrectangle{\pgfqpoint{0.574151in}{0.521603in}}{\pgfqpoint{3.720000in}{3.020000in}} %
\pgfusepath{clip}%
\pgfsetbuttcap%
\pgfsetroundjoin%
\definecolor{currentfill}{rgb}{1.000000,0.000000,0.000000}%
\pgfsetfillcolor{currentfill}%
\pgfsetlinewidth{1.003750pt}%
\definecolor{currentstroke}{rgb}{1.000000,0.000000,0.000000}%
\pgfsetstrokecolor{currentstroke}%
\pgfsetdash{}{0pt}%
\pgfpathmoveto{\pgfqpoint{4.121408in}{1.223734in}}%
\pgfpathcurveto{\pgfqpoint{4.132458in}{1.223734in}}{\pgfqpoint{4.143057in}{1.228125in}}{\pgfqpoint{4.150871in}{1.235938in}}%
\pgfpathcurveto{\pgfqpoint{4.158684in}{1.243752in}}{\pgfqpoint{4.163075in}{1.254351in}}{\pgfqpoint{4.163075in}{1.265401in}}%
\pgfpathcurveto{\pgfqpoint{4.163075in}{1.276451in}}{\pgfqpoint{4.158684in}{1.287050in}}{\pgfqpoint{4.150871in}{1.294864in}}%
\pgfpathcurveto{\pgfqpoint{4.143057in}{1.302677in}}{\pgfqpoint{4.132458in}{1.307068in}}{\pgfqpoint{4.121408in}{1.307068in}}%
\pgfpathcurveto{\pgfqpoint{4.110358in}{1.307068in}}{\pgfqpoint{4.099759in}{1.302677in}}{\pgfqpoint{4.091945in}{1.294864in}}%
\pgfpathcurveto{\pgfqpoint{4.084132in}{1.287050in}}{\pgfqpoint{4.079741in}{1.276451in}}{\pgfqpoint{4.079741in}{1.265401in}}%
\pgfpathcurveto{\pgfqpoint{4.079741in}{1.254351in}}{\pgfqpoint{4.084132in}{1.243752in}}{\pgfqpoint{4.091945in}{1.235938in}}%
\pgfpathcurveto{\pgfqpoint{4.099759in}{1.228125in}}{\pgfqpoint{4.110358in}{1.223734in}}{\pgfqpoint{4.121408in}{1.223734in}}%
\pgfpathclose%
\pgfusepath{stroke,fill}%
\end{pgfscope}%
\begin{pgfscope}%
\pgfpathrectangle{\pgfqpoint{0.574151in}{0.521603in}}{\pgfqpoint{3.720000in}{3.020000in}} %
\pgfusepath{clip}%
\pgfsetbuttcap%
\pgfsetroundjoin%
\definecolor{currentfill}{rgb}{1.000000,0.000000,0.000000}%
\pgfsetfillcolor{currentfill}%
\pgfsetlinewidth{1.003750pt}%
\definecolor{currentstroke}{rgb}{1.000000,0.000000,0.000000}%
\pgfsetstrokecolor{currentstroke}%
\pgfsetdash{}{0pt}%
\pgfpathmoveto{\pgfqpoint{2.883396in}{2.326348in}}%
\pgfpathcurveto{\pgfqpoint{2.894447in}{2.326348in}}{\pgfqpoint{2.905046in}{2.330738in}}{\pgfqpoint{2.912859in}{2.338551in}}%
\pgfpathcurveto{\pgfqpoint{2.920673in}{2.346365in}}{\pgfqpoint{2.925063in}{2.356964in}}{\pgfqpoint{2.925063in}{2.368014in}}%
\pgfpathcurveto{\pgfqpoint{2.925063in}{2.379064in}}{\pgfqpoint{2.920673in}{2.389663in}}{\pgfqpoint{2.912859in}{2.397477in}}%
\pgfpathcurveto{\pgfqpoint{2.905046in}{2.405291in}}{\pgfqpoint{2.894447in}{2.409681in}}{\pgfqpoint{2.883396in}{2.409681in}}%
\pgfpathcurveto{\pgfqpoint{2.872346in}{2.409681in}}{\pgfqpoint{2.861747in}{2.405291in}}{\pgfqpoint{2.853934in}{2.397477in}}%
\pgfpathcurveto{\pgfqpoint{2.846120in}{2.389663in}}{\pgfqpoint{2.841730in}{2.379064in}}{\pgfqpoint{2.841730in}{2.368014in}}%
\pgfpathcurveto{\pgfqpoint{2.841730in}{2.356964in}}{\pgfqpoint{2.846120in}{2.346365in}}{\pgfqpoint{2.853934in}{2.338551in}}%
\pgfpathcurveto{\pgfqpoint{2.861747in}{2.330738in}}{\pgfqpoint{2.872346in}{2.326348in}}{\pgfqpoint{2.883396in}{2.326348in}}%
\pgfpathclose%
\pgfusepath{stroke,fill}%
\end{pgfscope}%
\begin{pgfscope}%
\pgfpathrectangle{\pgfqpoint{0.574151in}{0.521603in}}{\pgfqpoint{3.720000in}{3.020000in}} %
\pgfusepath{clip}%
\pgfsetbuttcap%
\pgfsetroundjoin%
\definecolor{currentfill}{rgb}{0.484314,0.024637,0.999924}%
\pgfsetfillcolor{currentfill}%
\pgfsetlinewidth{1.003750pt}%
\definecolor{currentstroke}{rgb}{0.484314,0.024637,0.999924}%
\pgfsetstrokecolor{currentstroke}%
\pgfsetdash{}{0pt}%
\pgfpathmoveto{\pgfqpoint{2.989684in}{2.167561in}}%
\pgfpathcurveto{\pgfqpoint{3.000734in}{2.167561in}}{\pgfqpoint{3.011333in}{2.171951in}}{\pgfqpoint{3.019147in}{2.179765in}}%
\pgfpathcurveto{\pgfqpoint{3.026961in}{2.187578in}}{\pgfqpoint{3.031351in}{2.198177in}}{\pgfqpoint{3.031351in}{2.209228in}}%
\pgfpathcurveto{\pgfqpoint{3.031351in}{2.220278in}}{\pgfqpoint{3.026961in}{2.230877in}}{\pgfqpoint{3.019147in}{2.238690in}}%
\pgfpathcurveto{\pgfqpoint{3.011333in}{2.246504in}}{\pgfqpoint{3.000734in}{2.250894in}}{\pgfqpoint{2.989684in}{2.250894in}}%
\pgfpathcurveto{\pgfqpoint{2.978634in}{2.250894in}}{\pgfqpoint{2.968035in}{2.246504in}}{\pgfqpoint{2.960221in}{2.238690in}}%
\pgfpathcurveto{\pgfqpoint{2.952408in}{2.230877in}}{\pgfqpoint{2.948017in}{2.220278in}}{\pgfqpoint{2.948017in}{2.209228in}}%
\pgfpathcurveto{\pgfqpoint{2.948017in}{2.198177in}}{\pgfqpoint{2.952408in}{2.187578in}}{\pgfqpoint{2.960221in}{2.179765in}}%
\pgfpathcurveto{\pgfqpoint{2.968035in}{2.171951in}}{\pgfqpoint{2.978634in}{2.167561in}}{\pgfqpoint{2.989684in}{2.167561in}}%
\pgfpathclose%
\pgfusepath{stroke,fill}%
\end{pgfscope}%
\begin{pgfscope}%
\pgfpathrectangle{\pgfqpoint{0.574151in}{0.521603in}}{\pgfqpoint{3.720000in}{3.020000in}} %
\pgfusepath{clip}%
\pgfsetbuttcap%
\pgfsetroundjoin%
\definecolor{currentfill}{rgb}{0.484314,0.024637,0.999924}%
\pgfsetfillcolor{currentfill}%
\pgfsetlinewidth{1.003750pt}%
\definecolor{currentstroke}{rgb}{0.484314,0.024637,0.999924}%
\pgfsetstrokecolor{currentstroke}%
\pgfsetdash{}{0pt}%
\pgfpathmoveto{\pgfqpoint{2.724122in}{2.167561in}}%
\pgfpathcurveto{\pgfqpoint{2.735172in}{2.167561in}}{\pgfqpoint{2.745771in}{2.171951in}}{\pgfqpoint{2.753585in}{2.179765in}}%
\pgfpathcurveto{\pgfqpoint{2.761398in}{2.187578in}}{\pgfqpoint{2.765788in}{2.198177in}}{\pgfqpoint{2.765788in}{2.209228in}}%
\pgfpathcurveto{\pgfqpoint{2.765788in}{2.220278in}}{\pgfqpoint{2.761398in}{2.230877in}}{\pgfqpoint{2.753585in}{2.238690in}}%
\pgfpathcurveto{\pgfqpoint{2.745771in}{2.246504in}}{\pgfqpoint{2.735172in}{2.250894in}}{\pgfqpoint{2.724122in}{2.250894in}}%
\pgfpathcurveto{\pgfqpoint{2.713072in}{2.250894in}}{\pgfqpoint{2.702473in}{2.246504in}}{\pgfqpoint{2.694659in}{2.238690in}}%
\pgfpathcurveto{\pgfqpoint{2.686845in}{2.230877in}}{\pgfqpoint{2.682455in}{2.220278in}}{\pgfqpoint{2.682455in}{2.209228in}}%
\pgfpathcurveto{\pgfqpoint{2.682455in}{2.198177in}}{\pgfqpoint{2.686845in}{2.187578in}}{\pgfqpoint{2.694659in}{2.179765in}}%
\pgfpathcurveto{\pgfqpoint{2.702473in}{2.171951in}}{\pgfqpoint{2.713072in}{2.167561in}}{\pgfqpoint{2.724122in}{2.167561in}}%
\pgfpathclose%
\pgfusepath{stroke,fill}%
\end{pgfscope}%
\begin{pgfscope}%
\pgfpathrectangle{\pgfqpoint{0.574151in}{0.521603in}}{\pgfqpoint{3.720000in}{3.020000in}} %
\pgfusepath{clip}%
\pgfsetbuttcap%
\pgfsetroundjoin%
\definecolor{currentfill}{rgb}{0.484314,0.024637,0.999924}%
\pgfsetfillcolor{currentfill}%
\pgfsetlinewidth{1.003750pt}%
\definecolor{currentstroke}{rgb}{0.484314,0.024637,0.999924}%
\pgfsetstrokecolor{currentstroke}%
\pgfsetdash{}{0pt}%
\pgfpathmoveto{\pgfqpoint{2.567617in}{2.167561in}}%
\pgfpathcurveto{\pgfqpoint{2.578667in}{2.167561in}}{\pgfqpoint{2.589266in}{2.171951in}}{\pgfqpoint{2.597080in}{2.179765in}}%
\pgfpathcurveto{\pgfqpoint{2.604893in}{2.187578in}}{\pgfqpoint{2.609284in}{2.198177in}}{\pgfqpoint{2.609284in}{2.209228in}}%
\pgfpathcurveto{\pgfqpoint{2.609284in}{2.220278in}}{\pgfqpoint{2.604893in}{2.230877in}}{\pgfqpoint{2.597080in}{2.238690in}}%
\pgfpathcurveto{\pgfqpoint{2.589266in}{2.246504in}}{\pgfqpoint{2.578667in}{2.250894in}}{\pgfqpoint{2.567617in}{2.250894in}}%
\pgfpathcurveto{\pgfqpoint{2.556567in}{2.250894in}}{\pgfqpoint{2.545968in}{2.246504in}}{\pgfqpoint{2.538154in}{2.238690in}}%
\pgfpathcurveto{\pgfqpoint{2.530341in}{2.230877in}}{\pgfqpoint{2.525950in}{2.220278in}}{\pgfqpoint{2.525950in}{2.209228in}}%
\pgfpathcurveto{\pgfqpoint{2.525950in}{2.198177in}}{\pgfqpoint{2.530341in}{2.187578in}}{\pgfqpoint{2.538154in}{2.179765in}}%
\pgfpathcurveto{\pgfqpoint{2.545968in}{2.171951in}}{\pgfqpoint{2.556567in}{2.167561in}}{\pgfqpoint{2.567617in}{2.167561in}}%
\pgfpathclose%
\pgfusepath{stroke,fill}%
\end{pgfscope}%
\begin{pgfscope}%
\pgfpathrectangle{\pgfqpoint{0.574151in}{0.521603in}}{\pgfqpoint{3.720000in}{3.020000in}} %
\pgfusepath{clip}%
\pgfsetbuttcap%
\pgfsetroundjoin%
\definecolor{currentfill}{rgb}{0.484314,0.024637,0.999924}%
\pgfsetfillcolor{currentfill}%
\pgfsetlinewidth{1.003750pt}%
\definecolor{currentstroke}{rgb}{0.484314,0.024637,0.999924}%
\pgfsetstrokecolor{currentstroke}%
\pgfsetdash{}{0pt}%
\pgfpathmoveto{\pgfqpoint{2.421665in}{2.167561in}}%
\pgfpathcurveto{\pgfqpoint{2.432715in}{2.167561in}}{\pgfqpoint{2.443314in}{2.171951in}}{\pgfqpoint{2.451128in}{2.179765in}}%
\pgfpathcurveto{\pgfqpoint{2.458941in}{2.187578in}}{\pgfqpoint{2.463331in}{2.198177in}}{\pgfqpoint{2.463331in}{2.209228in}}%
\pgfpathcurveto{\pgfqpoint{2.463331in}{2.220278in}}{\pgfqpoint{2.458941in}{2.230877in}}{\pgfqpoint{2.451128in}{2.238690in}}%
\pgfpathcurveto{\pgfqpoint{2.443314in}{2.246504in}}{\pgfqpoint{2.432715in}{2.250894in}}{\pgfqpoint{2.421665in}{2.250894in}}%
\pgfpathcurveto{\pgfqpoint{2.410615in}{2.250894in}}{\pgfqpoint{2.400016in}{2.246504in}}{\pgfqpoint{2.392202in}{2.238690in}}%
\pgfpathcurveto{\pgfqpoint{2.384388in}{2.230877in}}{\pgfqpoint{2.379998in}{2.220278in}}{\pgfqpoint{2.379998in}{2.209228in}}%
\pgfpathcurveto{\pgfqpoint{2.379998in}{2.198177in}}{\pgfqpoint{2.384388in}{2.187578in}}{\pgfqpoint{2.392202in}{2.179765in}}%
\pgfpathcurveto{\pgfqpoint{2.400016in}{2.171951in}}{\pgfqpoint{2.410615in}{2.167561in}}{\pgfqpoint{2.421665in}{2.167561in}}%
\pgfpathclose%
\pgfusepath{stroke,fill}%
\end{pgfscope}%
\begin{pgfscope}%
\pgfpathrectangle{\pgfqpoint{0.574151in}{0.521603in}}{\pgfqpoint{3.720000in}{3.020000in}} %
\pgfusepath{clip}%
\pgfsetbuttcap%
\pgfsetroundjoin%
\definecolor{currentfill}{rgb}{0.484314,0.024637,0.999924}%
\pgfsetfillcolor{currentfill}%
\pgfsetlinewidth{1.003750pt}%
\definecolor{currentstroke}{rgb}{0.484314,0.024637,0.999924}%
\pgfsetstrokecolor{currentstroke}%
\pgfsetdash{}{0pt}%
\pgfpathmoveto{\pgfqpoint{2.018634in}{1.586433in}}%
\pgfpathcurveto{\pgfqpoint{2.029684in}{1.586433in}}{\pgfqpoint{2.040283in}{1.590823in}}{\pgfqpoint{2.048097in}{1.598637in}}%
\pgfpathcurveto{\pgfqpoint{2.055911in}{1.606450in}}{\pgfqpoint{2.060301in}{1.617049in}}{\pgfqpoint{2.060301in}{1.628099in}}%
\pgfpathcurveto{\pgfqpoint{2.060301in}{1.639149in}}{\pgfqpoint{2.055911in}{1.649749in}}{\pgfqpoint{2.048097in}{1.657562in}}%
\pgfpathcurveto{\pgfqpoint{2.040283in}{1.665376in}}{\pgfqpoint{2.029684in}{1.669766in}}{\pgfqpoint{2.018634in}{1.669766in}}%
\pgfpathcurveto{\pgfqpoint{2.007584in}{1.669766in}}{\pgfqpoint{1.996985in}{1.665376in}}{\pgfqpoint{1.989171in}{1.657562in}}%
\pgfpathcurveto{\pgfqpoint{1.981358in}{1.649749in}}{\pgfqpoint{1.976967in}{1.639149in}}{\pgfqpoint{1.976967in}{1.628099in}}%
\pgfpathcurveto{\pgfqpoint{1.976967in}{1.617049in}}{\pgfqpoint{1.981358in}{1.606450in}}{\pgfqpoint{1.989171in}{1.598637in}}%
\pgfpathcurveto{\pgfqpoint{1.996985in}{1.590823in}}{\pgfqpoint{2.007584in}{1.586433in}}{\pgfqpoint{2.018634in}{1.586433in}}%
\pgfpathclose%
\pgfusepath{stroke,fill}%
\end{pgfscope}%
\begin{pgfscope}%
\pgfsetbuttcap%
\pgfsetroundjoin%
\definecolor{currentfill}{rgb}{0.000000,0.000000,0.000000}%
\pgfsetfillcolor{currentfill}%
\pgfsetlinewidth{0.803000pt}%
\definecolor{currentstroke}{rgb}{0.000000,0.000000,0.000000}%
\pgfsetstrokecolor{currentstroke}%
\pgfsetdash{}{0pt}%
\pgfsys@defobject{currentmarker}{\pgfqpoint{0.000000in}{-0.048611in}}{\pgfqpoint{0.000000in}{0.000000in}}{%
\pgfpathmoveto{\pgfqpoint{0.000000in}{0.000000in}}%
\pgfpathlineto{\pgfqpoint{0.000000in}{-0.048611in}}%
\pgfusepath{stroke,fill}%
}%
\begin{pgfscope}%
\pgfsys@transformshift{0.743242in}{0.521603in}%
\pgfsys@useobject{currentmarker}{}%
\end{pgfscope}%
\end{pgfscope}%
\begin{pgfscope}%
\pgftext[x=0.743242in,y=0.424381in,,top]{\rmfamily\fontsize{10.000000}{12.000000}\selectfont \(\displaystyle 10^{-2}\)}%
\end{pgfscope}%
\begin{pgfscope}%
\pgfsetbuttcap%
\pgfsetroundjoin%
\definecolor{currentfill}{rgb}{0.000000,0.000000,0.000000}%
\pgfsetfillcolor{currentfill}%
\pgfsetlinewidth{0.803000pt}%
\definecolor{currentstroke}{rgb}{0.000000,0.000000,0.000000}%
\pgfsetstrokecolor{currentstroke}%
\pgfsetdash{}{0pt}%
\pgfsys@defobject{currentmarker}{\pgfqpoint{0.000000in}{-0.048611in}}{\pgfqpoint{0.000000in}{0.000000in}}{%
\pgfpathmoveto{\pgfqpoint{0.000000in}{0.000000in}}%
\pgfpathlineto{\pgfqpoint{0.000000in}{-0.048611in}}%
\pgfusepath{stroke,fill}%
}%
\begin{pgfscope}%
\pgfsys@transformshift{2.358141in}{0.521603in}%
\pgfsys@useobject{currentmarker}{}%
\end{pgfscope}%
\end{pgfscope}%
\begin{pgfscope}%
\pgftext[x=2.358141in,y=0.424381in,,top]{\rmfamily\fontsize{10.000000}{12.000000}\selectfont \(\displaystyle 10^{-1}\)}%
\end{pgfscope}%
\begin{pgfscope}%
\pgfsetbuttcap%
\pgfsetroundjoin%
\definecolor{currentfill}{rgb}{0.000000,0.000000,0.000000}%
\pgfsetfillcolor{currentfill}%
\pgfsetlinewidth{0.803000pt}%
\definecolor{currentstroke}{rgb}{0.000000,0.000000,0.000000}%
\pgfsetstrokecolor{currentstroke}%
\pgfsetdash{}{0pt}%
\pgfsys@defobject{currentmarker}{\pgfqpoint{0.000000in}{-0.048611in}}{\pgfqpoint{0.000000in}{0.000000in}}{%
\pgfpathmoveto{\pgfqpoint{0.000000in}{0.000000in}}%
\pgfpathlineto{\pgfqpoint{0.000000in}{-0.048611in}}%
\pgfusepath{stroke,fill}%
}%
\begin{pgfscope}%
\pgfsys@transformshift{3.973040in}{0.521603in}%
\pgfsys@useobject{currentmarker}{}%
\end{pgfscope}%
\end{pgfscope}%
\begin{pgfscope}%
\pgftext[x=3.973040in,y=0.424381in,,top]{\rmfamily\fontsize{10.000000}{12.000000}\selectfont \(\displaystyle 10^{0}\)}%
\end{pgfscope}%
\begin{pgfscope}%
\pgfsetbuttcap%
\pgfsetroundjoin%
\definecolor{currentfill}{rgb}{0.000000,0.000000,0.000000}%
\pgfsetfillcolor{currentfill}%
\pgfsetlinewidth{0.602250pt}%
\definecolor{currentstroke}{rgb}{0.000000,0.000000,0.000000}%
\pgfsetstrokecolor{currentstroke}%
\pgfsetdash{}{0pt}%
\pgfsys@defobject{currentmarker}{\pgfqpoint{0.000000in}{-0.027778in}}{\pgfqpoint{0.000000in}{0.000000in}}{%
\pgfpathmoveto{\pgfqpoint{0.000000in}{0.000000in}}%
\pgfpathlineto{\pgfqpoint{0.000000in}{-0.027778in}}%
\pgfusepath{stroke,fill}%
}%
\begin{pgfscope}%
\pgfsys@transformshift{0.586742in}{0.521603in}%
\pgfsys@useobject{currentmarker}{}%
\end{pgfscope}%
\end{pgfscope}%
\begin{pgfscope}%
\pgfsetbuttcap%
\pgfsetroundjoin%
\definecolor{currentfill}{rgb}{0.000000,0.000000,0.000000}%
\pgfsetfillcolor{currentfill}%
\pgfsetlinewidth{0.602250pt}%
\definecolor{currentstroke}{rgb}{0.000000,0.000000,0.000000}%
\pgfsetstrokecolor{currentstroke}%
\pgfsetdash{}{0pt}%
\pgfsys@defobject{currentmarker}{\pgfqpoint{0.000000in}{-0.027778in}}{\pgfqpoint{0.000000in}{0.000000in}}{%
\pgfpathmoveto{\pgfqpoint{0.000000in}{0.000000in}}%
\pgfpathlineto{\pgfqpoint{0.000000in}{-0.027778in}}%
\pgfusepath{stroke,fill}%
}%
\begin{pgfscope}%
\pgfsys@transformshift{0.669349in}{0.521603in}%
\pgfsys@useobject{currentmarker}{}%
\end{pgfscope}%
\end{pgfscope}%
\begin{pgfscope}%
\pgfsetbuttcap%
\pgfsetroundjoin%
\definecolor{currentfill}{rgb}{0.000000,0.000000,0.000000}%
\pgfsetfillcolor{currentfill}%
\pgfsetlinewidth{0.602250pt}%
\definecolor{currentstroke}{rgb}{0.000000,0.000000,0.000000}%
\pgfsetstrokecolor{currentstroke}%
\pgfsetdash{}{0pt}%
\pgfsys@defobject{currentmarker}{\pgfqpoint{0.000000in}{-0.027778in}}{\pgfqpoint{0.000000in}{0.000000in}}{%
\pgfpathmoveto{\pgfqpoint{0.000000in}{0.000000in}}%
\pgfpathlineto{\pgfqpoint{0.000000in}{-0.027778in}}%
\pgfusepath{stroke,fill}%
}%
\begin{pgfscope}%
\pgfsys@transformshift{1.229375in}{0.521603in}%
\pgfsys@useobject{currentmarker}{}%
\end{pgfscope}%
\end{pgfscope}%
\begin{pgfscope}%
\pgfsetbuttcap%
\pgfsetroundjoin%
\definecolor{currentfill}{rgb}{0.000000,0.000000,0.000000}%
\pgfsetfillcolor{currentfill}%
\pgfsetlinewidth{0.602250pt}%
\definecolor{currentstroke}{rgb}{0.000000,0.000000,0.000000}%
\pgfsetstrokecolor{currentstroke}%
\pgfsetdash{}{0pt}%
\pgfsys@defobject{currentmarker}{\pgfqpoint{0.000000in}{-0.027778in}}{\pgfqpoint{0.000000in}{0.000000in}}{%
\pgfpathmoveto{\pgfqpoint{0.000000in}{0.000000in}}%
\pgfpathlineto{\pgfqpoint{0.000000in}{-0.027778in}}%
\pgfusepath{stroke,fill}%
}%
\begin{pgfscope}%
\pgfsys@transformshift{1.513745in}{0.521603in}%
\pgfsys@useobject{currentmarker}{}%
\end{pgfscope}%
\end{pgfscope}%
\begin{pgfscope}%
\pgfsetbuttcap%
\pgfsetroundjoin%
\definecolor{currentfill}{rgb}{0.000000,0.000000,0.000000}%
\pgfsetfillcolor{currentfill}%
\pgfsetlinewidth{0.602250pt}%
\definecolor{currentstroke}{rgb}{0.000000,0.000000,0.000000}%
\pgfsetstrokecolor{currentstroke}%
\pgfsetdash{}{0pt}%
\pgfsys@defobject{currentmarker}{\pgfqpoint{0.000000in}{-0.027778in}}{\pgfqpoint{0.000000in}{0.000000in}}{%
\pgfpathmoveto{\pgfqpoint{0.000000in}{0.000000in}}%
\pgfpathlineto{\pgfqpoint{0.000000in}{-0.027778in}}%
\pgfusepath{stroke,fill}%
}%
\begin{pgfscope}%
\pgfsys@transformshift{1.715508in}{0.521603in}%
\pgfsys@useobject{currentmarker}{}%
\end{pgfscope}%
\end{pgfscope}%
\begin{pgfscope}%
\pgfsetbuttcap%
\pgfsetroundjoin%
\definecolor{currentfill}{rgb}{0.000000,0.000000,0.000000}%
\pgfsetfillcolor{currentfill}%
\pgfsetlinewidth{0.602250pt}%
\definecolor{currentstroke}{rgb}{0.000000,0.000000,0.000000}%
\pgfsetstrokecolor{currentstroke}%
\pgfsetdash{}{0pt}%
\pgfsys@defobject{currentmarker}{\pgfqpoint{0.000000in}{-0.027778in}}{\pgfqpoint{0.000000in}{0.000000in}}{%
\pgfpathmoveto{\pgfqpoint{0.000000in}{0.000000in}}%
\pgfpathlineto{\pgfqpoint{0.000000in}{-0.027778in}}%
\pgfusepath{stroke,fill}%
}%
\begin{pgfscope}%
\pgfsys@transformshift{1.872008in}{0.521603in}%
\pgfsys@useobject{currentmarker}{}%
\end{pgfscope}%
\end{pgfscope}%
\begin{pgfscope}%
\pgfsetbuttcap%
\pgfsetroundjoin%
\definecolor{currentfill}{rgb}{0.000000,0.000000,0.000000}%
\pgfsetfillcolor{currentfill}%
\pgfsetlinewidth{0.602250pt}%
\definecolor{currentstroke}{rgb}{0.000000,0.000000,0.000000}%
\pgfsetstrokecolor{currentstroke}%
\pgfsetdash{}{0pt}%
\pgfsys@defobject{currentmarker}{\pgfqpoint{0.000000in}{-0.027778in}}{\pgfqpoint{0.000000in}{0.000000in}}{%
\pgfpathmoveto{\pgfqpoint{0.000000in}{0.000000in}}%
\pgfpathlineto{\pgfqpoint{0.000000in}{-0.027778in}}%
\pgfusepath{stroke,fill}%
}%
\begin{pgfscope}%
\pgfsys@transformshift{1.999878in}{0.521603in}%
\pgfsys@useobject{currentmarker}{}%
\end{pgfscope}%
\end{pgfscope}%
\begin{pgfscope}%
\pgfsetbuttcap%
\pgfsetroundjoin%
\definecolor{currentfill}{rgb}{0.000000,0.000000,0.000000}%
\pgfsetfillcolor{currentfill}%
\pgfsetlinewidth{0.602250pt}%
\definecolor{currentstroke}{rgb}{0.000000,0.000000,0.000000}%
\pgfsetstrokecolor{currentstroke}%
\pgfsetdash{}{0pt}%
\pgfsys@defobject{currentmarker}{\pgfqpoint{0.000000in}{-0.027778in}}{\pgfqpoint{0.000000in}{0.000000in}}{%
\pgfpathmoveto{\pgfqpoint{0.000000in}{0.000000in}}%
\pgfpathlineto{\pgfqpoint{0.000000in}{-0.027778in}}%
\pgfusepath{stroke,fill}%
}%
\begin{pgfscope}%
\pgfsys@transformshift{2.107990in}{0.521603in}%
\pgfsys@useobject{currentmarker}{}%
\end{pgfscope}%
\end{pgfscope}%
\begin{pgfscope}%
\pgfsetbuttcap%
\pgfsetroundjoin%
\definecolor{currentfill}{rgb}{0.000000,0.000000,0.000000}%
\pgfsetfillcolor{currentfill}%
\pgfsetlinewidth{0.602250pt}%
\definecolor{currentstroke}{rgb}{0.000000,0.000000,0.000000}%
\pgfsetstrokecolor{currentstroke}%
\pgfsetdash{}{0pt}%
\pgfsys@defobject{currentmarker}{\pgfqpoint{0.000000in}{-0.027778in}}{\pgfqpoint{0.000000in}{0.000000in}}{%
\pgfpathmoveto{\pgfqpoint{0.000000in}{0.000000in}}%
\pgfpathlineto{\pgfqpoint{0.000000in}{-0.027778in}}%
\pgfusepath{stroke,fill}%
}%
\begin{pgfscope}%
\pgfsys@transformshift{2.201641in}{0.521603in}%
\pgfsys@useobject{currentmarker}{}%
\end{pgfscope}%
\end{pgfscope}%
\begin{pgfscope}%
\pgfsetbuttcap%
\pgfsetroundjoin%
\definecolor{currentfill}{rgb}{0.000000,0.000000,0.000000}%
\pgfsetfillcolor{currentfill}%
\pgfsetlinewidth{0.602250pt}%
\definecolor{currentstroke}{rgb}{0.000000,0.000000,0.000000}%
\pgfsetstrokecolor{currentstroke}%
\pgfsetdash{}{0pt}%
\pgfsys@defobject{currentmarker}{\pgfqpoint{0.000000in}{-0.027778in}}{\pgfqpoint{0.000000in}{0.000000in}}{%
\pgfpathmoveto{\pgfqpoint{0.000000in}{0.000000in}}%
\pgfpathlineto{\pgfqpoint{0.000000in}{-0.027778in}}%
\pgfusepath{stroke,fill}%
}%
\begin{pgfscope}%
\pgfsys@transformshift{2.284247in}{0.521603in}%
\pgfsys@useobject{currentmarker}{}%
\end{pgfscope}%
\end{pgfscope}%
\begin{pgfscope}%
\pgfsetbuttcap%
\pgfsetroundjoin%
\definecolor{currentfill}{rgb}{0.000000,0.000000,0.000000}%
\pgfsetfillcolor{currentfill}%
\pgfsetlinewidth{0.602250pt}%
\definecolor{currentstroke}{rgb}{0.000000,0.000000,0.000000}%
\pgfsetstrokecolor{currentstroke}%
\pgfsetdash{}{0pt}%
\pgfsys@defobject{currentmarker}{\pgfqpoint{0.000000in}{-0.027778in}}{\pgfqpoint{0.000000in}{0.000000in}}{%
\pgfpathmoveto{\pgfqpoint{0.000000in}{0.000000in}}%
\pgfpathlineto{\pgfqpoint{0.000000in}{-0.027778in}}%
\pgfusepath{stroke,fill}%
}%
\begin{pgfscope}%
\pgfsys@transformshift{2.844274in}{0.521603in}%
\pgfsys@useobject{currentmarker}{}%
\end{pgfscope}%
\end{pgfscope}%
\begin{pgfscope}%
\pgfsetbuttcap%
\pgfsetroundjoin%
\definecolor{currentfill}{rgb}{0.000000,0.000000,0.000000}%
\pgfsetfillcolor{currentfill}%
\pgfsetlinewidth{0.602250pt}%
\definecolor{currentstroke}{rgb}{0.000000,0.000000,0.000000}%
\pgfsetstrokecolor{currentstroke}%
\pgfsetdash{}{0pt}%
\pgfsys@defobject{currentmarker}{\pgfqpoint{0.000000in}{-0.027778in}}{\pgfqpoint{0.000000in}{0.000000in}}{%
\pgfpathmoveto{\pgfqpoint{0.000000in}{0.000000in}}%
\pgfpathlineto{\pgfqpoint{0.000000in}{-0.027778in}}%
\pgfusepath{stroke,fill}%
}%
\begin{pgfscope}%
\pgfsys@transformshift{3.128644in}{0.521603in}%
\pgfsys@useobject{currentmarker}{}%
\end{pgfscope}%
\end{pgfscope}%
\begin{pgfscope}%
\pgfsetbuttcap%
\pgfsetroundjoin%
\definecolor{currentfill}{rgb}{0.000000,0.000000,0.000000}%
\pgfsetfillcolor{currentfill}%
\pgfsetlinewidth{0.602250pt}%
\definecolor{currentstroke}{rgb}{0.000000,0.000000,0.000000}%
\pgfsetstrokecolor{currentstroke}%
\pgfsetdash{}{0pt}%
\pgfsys@defobject{currentmarker}{\pgfqpoint{0.000000in}{-0.027778in}}{\pgfqpoint{0.000000in}{0.000000in}}{%
\pgfpathmoveto{\pgfqpoint{0.000000in}{0.000000in}}%
\pgfpathlineto{\pgfqpoint{0.000000in}{-0.027778in}}%
\pgfusepath{stroke,fill}%
}%
\begin{pgfscope}%
\pgfsys@transformshift{3.330407in}{0.521603in}%
\pgfsys@useobject{currentmarker}{}%
\end{pgfscope}%
\end{pgfscope}%
\begin{pgfscope}%
\pgfsetbuttcap%
\pgfsetroundjoin%
\definecolor{currentfill}{rgb}{0.000000,0.000000,0.000000}%
\pgfsetfillcolor{currentfill}%
\pgfsetlinewidth{0.602250pt}%
\definecolor{currentstroke}{rgb}{0.000000,0.000000,0.000000}%
\pgfsetstrokecolor{currentstroke}%
\pgfsetdash{}{0pt}%
\pgfsys@defobject{currentmarker}{\pgfqpoint{0.000000in}{-0.027778in}}{\pgfqpoint{0.000000in}{0.000000in}}{%
\pgfpathmoveto{\pgfqpoint{0.000000in}{0.000000in}}%
\pgfpathlineto{\pgfqpoint{0.000000in}{-0.027778in}}%
\pgfusepath{stroke,fill}%
}%
\begin{pgfscope}%
\pgfsys@transformshift{3.486907in}{0.521603in}%
\pgfsys@useobject{currentmarker}{}%
\end{pgfscope}%
\end{pgfscope}%
\begin{pgfscope}%
\pgfsetbuttcap%
\pgfsetroundjoin%
\definecolor{currentfill}{rgb}{0.000000,0.000000,0.000000}%
\pgfsetfillcolor{currentfill}%
\pgfsetlinewidth{0.602250pt}%
\definecolor{currentstroke}{rgb}{0.000000,0.000000,0.000000}%
\pgfsetstrokecolor{currentstroke}%
\pgfsetdash{}{0pt}%
\pgfsys@defobject{currentmarker}{\pgfqpoint{0.000000in}{-0.027778in}}{\pgfqpoint{0.000000in}{0.000000in}}{%
\pgfpathmoveto{\pgfqpoint{0.000000in}{0.000000in}}%
\pgfpathlineto{\pgfqpoint{0.000000in}{-0.027778in}}%
\pgfusepath{stroke,fill}%
}%
\begin{pgfscope}%
\pgfsys@transformshift{3.614777in}{0.521603in}%
\pgfsys@useobject{currentmarker}{}%
\end{pgfscope}%
\end{pgfscope}%
\begin{pgfscope}%
\pgfsetbuttcap%
\pgfsetroundjoin%
\definecolor{currentfill}{rgb}{0.000000,0.000000,0.000000}%
\pgfsetfillcolor{currentfill}%
\pgfsetlinewidth{0.602250pt}%
\definecolor{currentstroke}{rgb}{0.000000,0.000000,0.000000}%
\pgfsetstrokecolor{currentstroke}%
\pgfsetdash{}{0pt}%
\pgfsys@defobject{currentmarker}{\pgfqpoint{0.000000in}{-0.027778in}}{\pgfqpoint{0.000000in}{0.000000in}}{%
\pgfpathmoveto{\pgfqpoint{0.000000in}{0.000000in}}%
\pgfpathlineto{\pgfqpoint{0.000000in}{-0.027778in}}%
\pgfusepath{stroke,fill}%
}%
\begin{pgfscope}%
\pgfsys@transformshift{3.722889in}{0.521603in}%
\pgfsys@useobject{currentmarker}{}%
\end{pgfscope}%
\end{pgfscope}%
\begin{pgfscope}%
\pgfsetbuttcap%
\pgfsetroundjoin%
\definecolor{currentfill}{rgb}{0.000000,0.000000,0.000000}%
\pgfsetfillcolor{currentfill}%
\pgfsetlinewidth{0.602250pt}%
\definecolor{currentstroke}{rgb}{0.000000,0.000000,0.000000}%
\pgfsetstrokecolor{currentstroke}%
\pgfsetdash{}{0pt}%
\pgfsys@defobject{currentmarker}{\pgfqpoint{0.000000in}{-0.027778in}}{\pgfqpoint{0.000000in}{0.000000in}}{%
\pgfpathmoveto{\pgfqpoint{0.000000in}{0.000000in}}%
\pgfpathlineto{\pgfqpoint{0.000000in}{-0.027778in}}%
\pgfusepath{stroke,fill}%
}%
\begin{pgfscope}%
\pgfsys@transformshift{3.816540in}{0.521603in}%
\pgfsys@useobject{currentmarker}{}%
\end{pgfscope}%
\end{pgfscope}%
\begin{pgfscope}%
\pgfsetbuttcap%
\pgfsetroundjoin%
\definecolor{currentfill}{rgb}{0.000000,0.000000,0.000000}%
\pgfsetfillcolor{currentfill}%
\pgfsetlinewidth{0.602250pt}%
\definecolor{currentstroke}{rgb}{0.000000,0.000000,0.000000}%
\pgfsetstrokecolor{currentstroke}%
\pgfsetdash{}{0pt}%
\pgfsys@defobject{currentmarker}{\pgfqpoint{0.000000in}{-0.027778in}}{\pgfqpoint{0.000000in}{0.000000in}}{%
\pgfpathmoveto{\pgfqpoint{0.000000in}{0.000000in}}%
\pgfpathlineto{\pgfqpoint{0.000000in}{-0.027778in}}%
\pgfusepath{stroke,fill}%
}%
\begin{pgfscope}%
\pgfsys@transformshift{3.899146in}{0.521603in}%
\pgfsys@useobject{currentmarker}{}%
\end{pgfscope}%
\end{pgfscope}%
\begin{pgfscope}%
\pgftext[x=2.434151in,y=0.234413in,,top]{\rmfamily\fontsize{10.000000}{12.000000}\selectfont \(\displaystyle We\)}%
\end{pgfscope}%
\begin{pgfscope}%
\pgfsetbuttcap%
\pgfsetroundjoin%
\definecolor{currentfill}{rgb}{0.000000,0.000000,0.000000}%
\pgfsetfillcolor{currentfill}%
\pgfsetlinewidth{0.803000pt}%
\definecolor{currentstroke}{rgb}{0.000000,0.000000,0.000000}%
\pgfsetstrokecolor{currentstroke}%
\pgfsetdash{}{0pt}%
\pgfsys@defobject{currentmarker}{\pgfqpoint{-0.048611in}{0.000000in}}{\pgfqpoint{0.000000in}{0.000000in}}{%
\pgfpathmoveto{\pgfqpoint{0.000000in}{0.000000in}}%
\pgfpathlineto{\pgfqpoint{-0.048611in}{0.000000in}}%
\pgfusepath{stroke,fill}%
}%
\begin{pgfscope}%
\pgfsys@transformshift{0.574151in}{1.002178in}%
\pgfsys@useobject{currentmarker}{}%
\end{pgfscope}%
\end{pgfscope}%
\begin{pgfscope}%
\pgftext[x=0.299459in,y=0.949416in,left,base]{\rmfamily\fontsize{10.000000}{12.000000}\selectfont \(\displaystyle 2.5\)}%
\end{pgfscope}%
\begin{pgfscope}%
\pgfsetbuttcap%
\pgfsetroundjoin%
\definecolor{currentfill}{rgb}{0.000000,0.000000,0.000000}%
\pgfsetfillcolor{currentfill}%
\pgfsetlinewidth{0.803000pt}%
\definecolor{currentstroke}{rgb}{0.000000,0.000000,0.000000}%
\pgfsetstrokecolor{currentstroke}%
\pgfsetdash{}{0pt}%
\pgfsys@defobject{currentmarker}{\pgfqpoint{-0.048611in}{0.000000in}}{\pgfqpoint{0.000000in}{0.000000in}}{%
\pgfpathmoveto{\pgfqpoint{0.000000in}{0.000000in}}%
\pgfpathlineto{\pgfqpoint{-0.048611in}{0.000000in}}%
\pgfusepath{stroke,fill}%
}%
\begin{pgfscope}%
\pgfsys@transformshift{0.574151in}{1.574347in}%
\pgfsys@useobject{currentmarker}{}%
\end{pgfscope}%
\end{pgfscope}%
\begin{pgfscope}%
\pgftext[x=0.299459in,y=1.521586in,left,base]{\rmfamily\fontsize{10.000000}{12.000000}\selectfont \(\displaystyle 3.0\)}%
\end{pgfscope}%
\begin{pgfscope}%
\pgfsetbuttcap%
\pgfsetroundjoin%
\definecolor{currentfill}{rgb}{0.000000,0.000000,0.000000}%
\pgfsetfillcolor{currentfill}%
\pgfsetlinewidth{0.803000pt}%
\definecolor{currentstroke}{rgb}{0.000000,0.000000,0.000000}%
\pgfsetstrokecolor{currentstroke}%
\pgfsetdash{}{0pt}%
\pgfsys@defobject{currentmarker}{\pgfqpoint{-0.048611in}{0.000000in}}{\pgfqpoint{0.000000in}{0.000000in}}{%
\pgfpathmoveto{\pgfqpoint{0.000000in}{0.000000in}}%
\pgfpathlineto{\pgfqpoint{-0.048611in}{0.000000in}}%
\pgfusepath{stroke,fill}%
}%
\begin{pgfscope}%
\pgfsys@transformshift{0.574151in}{2.146517in}%
\pgfsys@useobject{currentmarker}{}%
\end{pgfscope}%
\end{pgfscope}%
\begin{pgfscope}%
\pgftext[x=0.299459in,y=2.093755in,left,base]{\rmfamily\fontsize{10.000000}{12.000000}\selectfont \(\displaystyle 3.5\)}%
\end{pgfscope}%
\begin{pgfscope}%
\pgfsetbuttcap%
\pgfsetroundjoin%
\definecolor{currentfill}{rgb}{0.000000,0.000000,0.000000}%
\pgfsetfillcolor{currentfill}%
\pgfsetlinewidth{0.803000pt}%
\definecolor{currentstroke}{rgb}{0.000000,0.000000,0.000000}%
\pgfsetstrokecolor{currentstroke}%
\pgfsetdash{}{0pt}%
\pgfsys@defobject{currentmarker}{\pgfqpoint{-0.048611in}{0.000000in}}{\pgfqpoint{0.000000in}{0.000000in}}{%
\pgfpathmoveto{\pgfqpoint{0.000000in}{0.000000in}}%
\pgfpathlineto{\pgfqpoint{-0.048611in}{0.000000in}}%
\pgfusepath{stroke,fill}%
}%
\begin{pgfscope}%
\pgfsys@transformshift{0.574151in}{2.718686in}%
\pgfsys@useobject{currentmarker}{}%
\end{pgfscope}%
\end{pgfscope}%
\begin{pgfscope}%
\pgftext[x=0.299459in,y=2.665925in,left,base]{\rmfamily\fontsize{10.000000}{12.000000}\selectfont \(\displaystyle 4.0\)}%
\end{pgfscope}%
\begin{pgfscope}%
\pgfsetbuttcap%
\pgfsetroundjoin%
\definecolor{currentfill}{rgb}{0.000000,0.000000,0.000000}%
\pgfsetfillcolor{currentfill}%
\pgfsetlinewidth{0.803000pt}%
\definecolor{currentstroke}{rgb}{0.000000,0.000000,0.000000}%
\pgfsetstrokecolor{currentstroke}%
\pgfsetdash{}{0pt}%
\pgfsys@defobject{currentmarker}{\pgfqpoint{-0.048611in}{0.000000in}}{\pgfqpoint{0.000000in}{0.000000in}}{%
\pgfpathmoveto{\pgfqpoint{0.000000in}{0.000000in}}%
\pgfpathlineto{\pgfqpoint{-0.048611in}{0.000000in}}%
\pgfusepath{stroke,fill}%
}%
\begin{pgfscope}%
\pgfsys@transformshift{0.574151in}{3.290856in}%
\pgfsys@useobject{currentmarker}{}%
\end{pgfscope}%
\end{pgfscope}%
\begin{pgfscope}%
\pgftext[x=0.299459in,y=3.238094in,left,base]{\rmfamily\fontsize{10.000000}{12.000000}\selectfont \(\displaystyle 4.5\)}%
\end{pgfscope}%
\begin{pgfscope}%
\pgftext[x=0.243904in,y=2.031603in,,bottom,rotate=90.000000]{\rmfamily\fontsize{10.000000}{12.000000}\selectfont \(\displaystyle t_j/ \tau\)}%
\end{pgfscope}%
\begin{pgfscope}%
\pgfpathrectangle{\pgfqpoint{0.574151in}{0.521603in}}{\pgfqpoint{3.720000in}{3.020000in}} %
\pgfusepath{clip}%
\pgfsetrectcap%
\pgfsetroundjoin%
\pgfsetlinewidth{1.505625pt}%
\definecolor{currentstroke}{rgb}{0.121569,0.466667,0.705882}%
\pgfsetstrokecolor{currentstroke}%
\pgfsetdash{}{0pt}%
\pgfpathmoveto{\pgfqpoint{0.743242in}{0.658876in}}%
\pgfpathlineto{\pgfqpoint{1.229375in}{0.658876in}}%
\pgfpathlineto{\pgfqpoint{1.513745in}{0.658876in}}%
\pgfpathlineto{\pgfqpoint{1.715508in}{0.658876in}}%
\pgfpathlineto{\pgfqpoint{1.872008in}{0.658876in}}%
\pgfpathlineto{\pgfqpoint{1.999878in}{0.658876in}}%
\pgfpathlineto{\pgfqpoint{2.107990in}{0.658876in}}%
\pgfpathlineto{\pgfqpoint{2.201641in}{0.658876in}}%
\pgfpathlineto{\pgfqpoint{2.284247in}{0.658876in}}%
\pgfpathlineto{\pgfqpoint{2.358141in}{0.658876in}}%
\pgfpathlineto{\pgfqpoint{2.424986in}{0.658876in}}%
\pgfpathlineto{\pgfqpoint{2.486011in}{0.658876in}}%
\pgfpathlineto{\pgfqpoint{2.542148in}{0.658876in}}%
\pgfpathlineto{\pgfqpoint{2.594123in}{0.658876in}}%
\pgfpathlineto{\pgfqpoint{2.642511in}{0.658876in}}%
\pgfpathlineto{\pgfqpoint{2.687774in}{0.658876in}}%
\pgfpathlineto{\pgfqpoint{2.730293in}{0.658876in}}%
\pgfpathlineto{\pgfqpoint{2.770380in}{0.658876in}}%
\pgfpathlineto{\pgfqpoint{2.808300in}{0.658876in}}%
\pgfpathlineto{\pgfqpoint{2.844274in}{0.658876in}}%
\pgfpathlineto{\pgfqpoint{2.878493in}{0.658876in}}%
\pgfpathlineto{\pgfqpoint{2.911119in}{0.658876in}}%
\pgfpathlineto{\pgfqpoint{2.942295in}{0.658876in}}%
\pgfpathlineto{\pgfqpoint{2.972144in}{0.658876in}}%
\pgfpathlineto{\pgfqpoint{3.000774in}{0.658876in}}%
\pgfpathlineto{\pgfqpoint{3.028281in}{0.658876in}}%
\pgfpathlineto{\pgfqpoint{3.054750in}{0.658876in}}%
\pgfpathlineto{\pgfqpoint{3.080256in}{0.658876in}}%
\pgfpathlineto{\pgfqpoint{3.104867in}{0.658876in}}%
\pgfpathlineto{\pgfqpoint{3.128644in}{0.658876in}}%
\pgfpathlineto{\pgfqpoint{3.151641in}{0.658876in}}%
\pgfpathlineto{\pgfqpoint{3.173907in}{0.658876in}}%
\pgfpathlineto{\pgfqpoint{3.195489in}{0.658876in}}%
\pgfpathlineto{\pgfqpoint{3.216426in}{0.658876in}}%
\pgfpathlineto{\pgfqpoint{3.236756in}{0.658876in}}%
\pgfpathlineto{\pgfqpoint{3.256513in}{0.658876in}}%
\pgfpathlineto{\pgfqpoint{3.275729in}{0.658876in}}%
\pgfpathlineto{\pgfqpoint{3.294433in}{0.658876in}}%
\pgfpathlineto{\pgfqpoint{3.312651in}{0.658876in}}%
\pgfpathlineto{\pgfqpoint{3.330407in}{0.658876in}}%
\pgfpathlineto{\pgfqpoint{3.347725in}{0.658876in}}%
\pgfpathlineto{\pgfqpoint{3.364626in}{0.658876in}}%
\pgfpathlineto{\pgfqpoint{3.381129in}{0.658876in}}%
\pgfpathlineto{\pgfqpoint{3.397252in}{0.658876in}}%
\pgfpathlineto{\pgfqpoint{3.413013in}{0.658876in}}%
\pgfpathlineto{\pgfqpoint{3.428428in}{0.658876in}}%
\pgfpathlineto{\pgfqpoint{3.443511in}{0.658876in}}%
\pgfpathlineto{\pgfqpoint{3.458277in}{0.658876in}}%
\pgfpathlineto{\pgfqpoint{3.472738in}{0.658876in}}%
\pgfpathlineto{\pgfqpoint{3.486907in}{0.658876in}}%
\pgfpathlineto{\pgfqpoint{3.500795in}{0.658876in}}%
\pgfpathlineto{\pgfqpoint{3.514414in}{0.658876in}}%
\pgfpathlineto{\pgfqpoint{3.527773in}{0.658876in}}%
\pgfpathlineto{\pgfqpoint{3.540883in}{0.658876in}}%
\pgfpathlineto{\pgfqpoint{3.553752in}{0.658876in}}%
\pgfpathlineto{\pgfqpoint{3.566389in}{0.658876in}}%
\pgfpathlineto{\pgfqpoint{3.578803in}{0.658876in}}%
\pgfpathlineto{\pgfqpoint{3.591000in}{0.658876in}}%
\pgfpathlineto{\pgfqpoint{3.602989in}{0.658876in}}%
\pgfpathlineto{\pgfqpoint{3.614777in}{0.658876in}}%
\pgfpathlineto{\pgfqpoint{3.626369in}{0.658876in}}%
\pgfpathlineto{\pgfqpoint{3.637774in}{0.658876in}}%
\pgfpathlineto{\pgfqpoint{3.648995in}{0.658876in}}%
\pgfpathlineto{\pgfqpoint{3.660040in}{0.658876in}}%
\pgfpathlineto{\pgfqpoint{3.670914in}{0.658876in}}%
\pgfpathlineto{\pgfqpoint{3.681622in}{0.658876in}}%
\pgfpathlineto{\pgfqpoint{3.692168in}{0.658876in}}%
\pgfpathlineto{\pgfqpoint{3.702559in}{0.658876in}}%
\pgfpathlineto{\pgfqpoint{3.712798in}{0.658876in}}%
\pgfpathlineto{\pgfqpoint{3.722889in}{0.658876in}}%
\pgfpathlineto{\pgfqpoint{3.732837in}{0.658876in}}%
\pgfpathlineto{\pgfqpoint{3.742646in}{0.658876in}}%
\pgfpathlineto{\pgfqpoint{3.752320in}{0.658876in}}%
\pgfpathlineto{\pgfqpoint{3.761862in}{0.658876in}}%
\pgfpathlineto{\pgfqpoint{3.771277in}{0.658876in}}%
\pgfpathlineto{\pgfqpoint{3.780566in}{0.658876in}}%
\pgfpathlineto{\pgfqpoint{3.789734in}{0.658876in}}%
\pgfpathlineto{\pgfqpoint{3.798784in}{0.658876in}}%
\pgfpathlineto{\pgfqpoint{3.807718in}{0.658876in}}%
\pgfpathlineto{\pgfqpoint{3.816540in}{0.658876in}}%
\pgfpathlineto{\pgfqpoint{3.825253in}{0.658876in}}%
\pgfpathlineto{\pgfqpoint{3.833858in}{0.658876in}}%
\pgfpathlineto{\pgfqpoint{3.842359in}{0.658876in}}%
\pgfpathlineto{\pgfqpoint{3.850759in}{0.658876in}}%
\pgfpathlineto{\pgfqpoint{3.859059in}{0.658876in}}%
\pgfpathlineto{\pgfqpoint{3.867262in}{0.658876in}}%
\pgfpathlineto{\pgfqpoint{3.875370in}{0.658876in}}%
\pgfpathlineto{\pgfqpoint{3.883385in}{0.658876in}}%
\pgfpathlineto{\pgfqpoint{3.891310in}{0.658876in}}%
\pgfpathlineto{\pgfqpoint{3.899146in}{0.658876in}}%
\pgfpathlineto{\pgfqpoint{3.906896in}{0.658876in}}%
\pgfpathlineto{\pgfqpoint{3.914561in}{0.658876in}}%
\pgfpathlineto{\pgfqpoint{3.922143in}{0.658876in}}%
\pgfpathlineto{\pgfqpoint{3.929644in}{0.658876in}}%
\pgfpathlineto{\pgfqpoint{3.937066in}{0.658876in}}%
\pgfpathlineto{\pgfqpoint{3.944410in}{0.658876in}}%
\pgfpathlineto{\pgfqpoint{3.951678in}{0.658876in}}%
\pgfpathlineto{\pgfqpoint{3.958871in}{0.658876in}}%
\pgfpathlineto{\pgfqpoint{3.965991in}{0.658876in}}%
\pgfpathlineto{\pgfqpoint{3.973040in}{0.658876in}}%
\pgfusepath{stroke}%
\end{pgfscope}%
\begin{pgfscope}%
\pgfsetrectcap%
\pgfsetmiterjoin%
\pgfsetlinewidth{0.803000pt}%
\definecolor{currentstroke}{rgb}{0.000000,0.000000,0.000000}%
\pgfsetstrokecolor{currentstroke}%
\pgfsetdash{}{0pt}%
\pgfpathmoveto{\pgfqpoint{0.574151in}{0.521603in}}%
\pgfpathlineto{\pgfqpoint{0.574151in}{3.541603in}}%
\pgfusepath{stroke}%
\end{pgfscope}%
\begin{pgfscope}%
\pgfsetrectcap%
\pgfsetmiterjoin%
\pgfsetlinewidth{0.803000pt}%
\definecolor{currentstroke}{rgb}{0.000000,0.000000,0.000000}%
\pgfsetstrokecolor{currentstroke}%
\pgfsetdash{}{0pt}%
\pgfpathmoveto{\pgfqpoint{4.294151in}{0.521603in}}%
\pgfpathlineto{\pgfqpoint{4.294151in}{3.541603in}}%
\pgfusepath{stroke}%
\end{pgfscope}%
\begin{pgfscope}%
\pgfsetrectcap%
\pgfsetmiterjoin%
\pgfsetlinewidth{0.803000pt}%
\definecolor{currentstroke}{rgb}{0.000000,0.000000,0.000000}%
\pgfsetstrokecolor{currentstroke}%
\pgfsetdash{}{0pt}%
\pgfpathmoveto{\pgfqpoint{0.574151in}{0.521603in}}%
\pgfpathlineto{\pgfqpoint{4.294151in}{0.521603in}}%
\pgfusepath{stroke}%
\end{pgfscope}%
\begin{pgfscope}%
\pgfsetrectcap%
\pgfsetmiterjoin%
\pgfsetlinewidth{0.803000pt}%
\definecolor{currentstroke}{rgb}{0.000000,0.000000,0.000000}%
\pgfsetstrokecolor{currentstroke}%
\pgfsetdash{}{0pt}%
\pgfpathmoveto{\pgfqpoint{0.574151in}{3.541603in}}%
\pgfpathlineto{\pgfqpoint{4.294151in}{3.541603in}}%
\pgfusepath{stroke}%
\end{pgfscope}%
\begin{pgfscope}%
\pgfsetbuttcap%
\pgfsetmiterjoin%
\definecolor{currentfill}{rgb}{1.000000,1.000000,1.000000}%
\pgfsetfillcolor{currentfill}%
\pgfsetfillopacity{0.800000}%
\pgfsetlinewidth{1.003750pt}%
\definecolor{currentstroke}{rgb}{0.800000,0.800000,0.800000}%
\pgfsetstrokecolor{currentstroke}%
\pgfsetstrokeopacity{0.800000}%
\pgfsetdash{}{0pt}%
\pgfpathmoveto{\pgfqpoint{2.729264in}{1.908841in}}%
\pgfpathlineto{\pgfqpoint{4.196929in}{1.908841in}}%
\pgfpathquadraticcurveto{\pgfqpoint{4.224707in}{1.908841in}}{\pgfqpoint{4.224707in}{1.936619in}}%
\pgfpathlineto{\pgfqpoint{4.224707in}{2.126587in}}%
\pgfpathquadraticcurveto{\pgfqpoint{4.224707in}{2.154365in}}{\pgfqpoint{4.196929in}{2.154365in}}%
\pgfpathlineto{\pgfqpoint{2.729264in}{2.154365in}}%
\pgfpathquadraticcurveto{\pgfqpoint{2.701486in}{2.154365in}}{\pgfqpoint{2.701486in}{2.126587in}}%
\pgfpathlineto{\pgfqpoint{2.701486in}{1.936619in}}%
\pgfpathquadraticcurveto{\pgfqpoint{2.701486in}{1.908841in}}{\pgfqpoint{2.729264in}{1.908841in}}%
\pgfpathclose%
\pgfusepath{stroke,fill}%
\end{pgfscope}%
\begin{pgfscope}%
\pgfsetrectcap%
\pgfsetroundjoin%
\pgfsetlinewidth{1.505625pt}%
\definecolor{currentstroke}{rgb}{0.121569,0.466667,0.705882}%
\pgfsetstrokecolor{currentstroke}%
\pgfsetdash{}{0pt}%
\pgfpathmoveto{\pgfqpoint{2.757042in}{2.041898in}}%
\pgfpathlineto{\pgfqpoint{3.034820in}{2.041898in}}%
\pgfusepath{stroke}%
\end{pgfscope}%
\begin{pgfscope}%
\pgftext[x=3.145931in,y=1.993287in,left,base]{\rmfamily\fontsize{10.000000}{12.000000}\selectfont Richards 2001}%
\end{pgfscope}%
\begin{pgfscope}%
\pgfpathrectangle{\pgfqpoint{4.526651in}{0.521603in}}{\pgfqpoint{0.151000in}{3.020000in}} %
\pgfusepath{clip}%
\pgfsetbuttcap%
\pgfsetmiterjoin%
\definecolor{currentfill}{rgb}{1.000000,1.000000,1.000000}%
\pgfsetfillcolor{currentfill}%
\pgfsetlinewidth{0.010037pt}%
\definecolor{currentstroke}{rgb}{1.000000,1.000000,1.000000}%
\pgfsetstrokecolor{currentstroke}%
\pgfsetdash{}{0pt}%
\pgfpathmoveto{\pgfqpoint{4.526651in}{0.521603in}}%
\pgfpathlineto{\pgfqpoint{4.526651in}{0.533400in}}%
\pgfpathlineto{\pgfqpoint{4.526651in}{3.529806in}}%
\pgfpathlineto{\pgfqpoint{4.526651in}{3.541603in}}%
\pgfpathlineto{\pgfqpoint{4.677651in}{3.541603in}}%
\pgfpathlineto{\pgfqpoint{4.677651in}{3.529806in}}%
\pgfpathlineto{\pgfqpoint{4.677651in}{0.533400in}}%
\pgfpathlineto{\pgfqpoint{4.677651in}{0.521603in}}%
\pgfpathclose%
\pgfusepath{stroke,fill}%
\end{pgfscope}%
\begin{pgfscope}%
\pgfsys@transformshift{4.530000in}{0.526603in}%
\pgftext[left,bottom]{\pgfimage[interpolate=true,width=0.150000in,height=3.020000in]{contact-img0.png}}%
\end{pgfscope}%
\begin{pgfscope}%
\pgfsetbuttcap%
\pgfsetroundjoin%
\definecolor{currentfill}{rgb}{0.000000,0.000000,0.000000}%
\pgfsetfillcolor{currentfill}%
\pgfsetlinewidth{0.803000pt}%
\definecolor{currentstroke}{rgb}{0.000000,0.000000,0.000000}%
\pgfsetstrokecolor{currentstroke}%
\pgfsetdash{}{0pt}%
\pgfsys@defobject{currentmarker}{\pgfqpoint{0.000000in}{0.000000in}}{\pgfqpoint{0.048611in}{0.000000in}}{%
\pgfpathmoveto{\pgfqpoint{0.000000in}{0.000000in}}%
\pgfpathlineto{\pgfqpoint{0.048611in}{0.000000in}}%
\pgfusepath{stroke,fill}%
}%
\begin{pgfscope}%
\pgfsys@transformshift{4.677651in}{0.874491in}%
\pgfsys@useobject{currentmarker}{}%
\end{pgfscope}%
\end{pgfscope}%
\begin{pgfscope}%
\pgftext[x=4.774874in,y=0.821729in,left,base]{\rmfamily\fontsize{10.000000}{12.000000}\selectfont \(\displaystyle 0.1\)}%
\end{pgfscope}%
\begin{pgfscope}%
\pgfsetbuttcap%
\pgfsetroundjoin%
\definecolor{currentfill}{rgb}{0.000000,0.000000,0.000000}%
\pgfsetfillcolor{currentfill}%
\pgfsetlinewidth{0.803000pt}%
\definecolor{currentstroke}{rgb}{0.000000,0.000000,0.000000}%
\pgfsetstrokecolor{currentstroke}%
\pgfsetdash{}{0pt}%
\pgfsys@defobject{currentmarker}{\pgfqpoint{0.000000in}{0.000000in}}{\pgfqpoint{0.048611in}{0.000000in}}{%
\pgfpathmoveto{\pgfqpoint{0.000000in}{0.000000in}}%
\pgfpathlineto{\pgfqpoint{0.048611in}{0.000000in}}%
\pgfusepath{stroke,fill}%
}%
\begin{pgfscope}%
\pgfsys@transformshift{4.677651in}{1.358178in}%
\pgfsys@useobject{currentmarker}{}%
\end{pgfscope}%
\end{pgfscope}%
\begin{pgfscope}%
\pgftext[x=4.774874in,y=1.305417in,left,base]{\rmfamily\fontsize{10.000000}{12.000000}\selectfont \(\displaystyle 0.2\)}%
\end{pgfscope}%
\begin{pgfscope}%
\pgfsetbuttcap%
\pgfsetroundjoin%
\definecolor{currentfill}{rgb}{0.000000,0.000000,0.000000}%
\pgfsetfillcolor{currentfill}%
\pgfsetlinewidth{0.803000pt}%
\definecolor{currentstroke}{rgb}{0.000000,0.000000,0.000000}%
\pgfsetstrokecolor{currentstroke}%
\pgfsetdash{}{0pt}%
\pgfsys@defobject{currentmarker}{\pgfqpoint{0.000000in}{0.000000in}}{\pgfqpoint{0.048611in}{0.000000in}}{%
\pgfpathmoveto{\pgfqpoint{0.000000in}{0.000000in}}%
\pgfpathlineto{\pgfqpoint{0.048611in}{0.000000in}}%
\pgfusepath{stroke,fill}%
}%
\begin{pgfscope}%
\pgfsys@transformshift{4.677651in}{1.841866in}%
\pgfsys@useobject{currentmarker}{}%
\end{pgfscope}%
\end{pgfscope}%
\begin{pgfscope}%
\pgftext[x=4.774874in,y=1.789104in,left,base]{\rmfamily\fontsize{10.000000}{12.000000}\selectfont \(\displaystyle 0.3\)}%
\end{pgfscope}%
\begin{pgfscope}%
\pgfsetbuttcap%
\pgfsetroundjoin%
\definecolor{currentfill}{rgb}{0.000000,0.000000,0.000000}%
\pgfsetfillcolor{currentfill}%
\pgfsetlinewidth{0.803000pt}%
\definecolor{currentstroke}{rgb}{0.000000,0.000000,0.000000}%
\pgfsetstrokecolor{currentstroke}%
\pgfsetdash{}{0pt}%
\pgfsys@defobject{currentmarker}{\pgfqpoint{0.000000in}{0.000000in}}{\pgfqpoint{0.048611in}{0.000000in}}{%
\pgfpathmoveto{\pgfqpoint{0.000000in}{0.000000in}}%
\pgfpathlineto{\pgfqpoint{0.048611in}{0.000000in}}%
\pgfusepath{stroke,fill}%
}%
\begin{pgfscope}%
\pgfsys@transformshift{4.677651in}{2.325553in}%
\pgfsys@useobject{currentmarker}{}%
\end{pgfscope}%
\end{pgfscope}%
\begin{pgfscope}%
\pgftext[x=4.774874in,y=2.272791in,left,base]{\rmfamily\fontsize{10.000000}{12.000000}\selectfont \(\displaystyle 0.4\)}%
\end{pgfscope}%
\begin{pgfscope}%
\pgfsetbuttcap%
\pgfsetroundjoin%
\definecolor{currentfill}{rgb}{0.000000,0.000000,0.000000}%
\pgfsetfillcolor{currentfill}%
\pgfsetlinewidth{0.803000pt}%
\definecolor{currentstroke}{rgb}{0.000000,0.000000,0.000000}%
\pgfsetstrokecolor{currentstroke}%
\pgfsetdash{}{0pt}%
\pgfsys@defobject{currentmarker}{\pgfqpoint{0.000000in}{0.000000in}}{\pgfqpoint{0.048611in}{0.000000in}}{%
\pgfpathmoveto{\pgfqpoint{0.000000in}{0.000000in}}%
\pgfpathlineto{\pgfqpoint{0.048611in}{0.000000in}}%
\pgfusepath{stroke,fill}%
}%
\begin{pgfscope}%
\pgfsys@transformshift{4.677651in}{2.809240in}%
\pgfsys@useobject{currentmarker}{}%
\end{pgfscope}%
\end{pgfscope}%
\begin{pgfscope}%
\pgftext[x=4.774874in,y=2.756479in,left,base]{\rmfamily\fontsize{10.000000}{12.000000}\selectfont \(\displaystyle 0.5\)}%
\end{pgfscope}%
\begin{pgfscope}%
\pgfsetbuttcap%
\pgfsetroundjoin%
\definecolor{currentfill}{rgb}{0.000000,0.000000,0.000000}%
\pgfsetfillcolor{currentfill}%
\pgfsetlinewidth{0.803000pt}%
\definecolor{currentstroke}{rgb}{0.000000,0.000000,0.000000}%
\pgfsetstrokecolor{currentstroke}%
\pgfsetdash{}{0pt}%
\pgfsys@defobject{currentmarker}{\pgfqpoint{0.000000in}{0.000000in}}{\pgfqpoint{0.048611in}{0.000000in}}{%
\pgfpathmoveto{\pgfqpoint{0.000000in}{0.000000in}}%
\pgfpathlineto{\pgfqpoint{0.048611in}{0.000000in}}%
\pgfusepath{stroke,fill}%
}%
\begin{pgfscope}%
\pgfsys@transformshift{4.677651in}{3.292928in}%
\pgfsys@useobject{currentmarker}{}%
\end{pgfscope}%
\end{pgfscope}%
\begin{pgfscope}%
\pgftext[x=4.774874in,y=3.240166in,left,base]{\rmfamily\fontsize{10.000000}{12.000000}\selectfont \(\displaystyle 0.6\)}%
\end{pgfscope}%
\begin{pgfscope}%
\pgftext[x=5.007899in,y=2.031603in,,top,rotate=90.000000]{\rmfamily\fontsize{10.000000}{12.000000}\selectfont \(\displaystyle \mathrm{\mathit{Bo_e}} \equiv \frac{\epsilon E_0^2 R_0}{\gamma}\)}%
\end{pgfscope}%
\begin{pgfscope}%
\pgfsetbuttcap%
\pgfsetmiterjoin%
\pgfsetlinewidth{0.803000pt}%
\definecolor{currentstroke}{rgb}{0.000000,0.000000,0.000000}%
\pgfsetstrokecolor{currentstroke}%
\pgfsetdash{}{0pt}%
\pgfpathmoveto{\pgfqpoint{4.526651in}{0.521603in}}%
\pgfpathlineto{\pgfqpoint{4.526651in}{0.533400in}}%
\pgfpathlineto{\pgfqpoint{4.526651in}{3.529806in}}%
\pgfpathlineto{\pgfqpoint{4.526651in}{3.541603in}}%
\pgfpathlineto{\pgfqpoint{4.677651in}{3.541603in}}%
\pgfpathlineto{\pgfqpoint{4.677651in}{3.529806in}}%
\pgfpathlineto{\pgfqpoint{4.677651in}{0.533400in}}%
\pgfpathlineto{\pgfqpoint{4.677651in}{0.521603in}}%
\pgfpathclose%
\pgfusepath{stroke}%
\end{pgfscope}%
\end{pgfpicture}%
\makeatother%
\endgroup%

    \caption{A simple EMA plot.\label{fig:contact}}
\end{figure}

\begin{figure}[htb]
    \centering
    %% Creator: Matplotlib, PGF backend
%%
%% To include the figure in your LaTeX document, write
%%   \input{<filename>.pgf}
%%
%% Make sure the required packages are loaded in your preamble
%%   \usepackage{pgf}
%%
%% Figures using additional raster images can only be included by \input if
%% they are in the same directory as the main LaTeX file. For loading figures
%% from other directories you can use the `import` package
%%   \usepackage{import}
%% and then include the figures with
%%   \import{<path to file>}{<filename>.pgf}
%%
%% Matplotlib used the following preamble
%%   \usepackage{fontspec}
%%   \setmainfont{DejaVu Serif}
%%   \setsansfont{DejaVu Sans}
%%   \setmonofont{DejaVu Sans Mono}
%%
\begingroup%
\makeatletter%
\begin{pgfpicture}%
\pgfpathrectangle{\pgfpointorigin}{\pgfqpoint{5.427700in}{3.676603in}}%
\pgfusepath{use as bounding box, clip}%
\begin{pgfscope}%
\pgfsetbuttcap%
\pgfsetmiterjoin%
\definecolor{currentfill}{rgb}{1.000000,1.000000,1.000000}%
\pgfsetfillcolor{currentfill}%
\pgfsetlinewidth{0.000000pt}%
\definecolor{currentstroke}{rgb}{1.000000,1.000000,1.000000}%
\pgfsetstrokecolor{currentstroke}%
\pgfsetdash{}{0pt}%
\pgfpathmoveto{\pgfqpoint{0.000000in}{0.000000in}}%
\pgfpathlineto{\pgfqpoint{5.427700in}{0.000000in}}%
\pgfpathlineto{\pgfqpoint{5.427700in}{3.676603in}}%
\pgfpathlineto{\pgfqpoint{0.000000in}{3.676603in}}%
\pgfpathclose%
\pgfusepath{fill}%
\end{pgfscope}%
\begin{pgfscope}%
\pgfsetbuttcap%
\pgfsetmiterjoin%
\definecolor{currentfill}{rgb}{1.000000,1.000000,1.000000}%
\pgfsetfillcolor{currentfill}%
\pgfsetlinewidth{0.000000pt}%
\definecolor{currentstroke}{rgb}{0.000000,0.000000,0.000000}%
\pgfsetstrokecolor{currentstroke}%
\pgfsetstrokeopacity{0.000000}%
\pgfsetdash{}{0pt}%
\pgfpathmoveto{\pgfqpoint{0.564660in}{0.521603in}}%
\pgfpathlineto{\pgfqpoint{4.284660in}{0.521603in}}%
\pgfpathlineto{\pgfqpoint{4.284660in}{3.541603in}}%
\pgfpathlineto{\pgfqpoint{0.564660in}{3.541603in}}%
\pgfpathclose%
\pgfusepath{fill}%
\end{pgfscope}%
\begin{pgfscope}%
\pgfpathrectangle{\pgfqpoint{0.564660in}{0.521603in}}{\pgfqpoint{3.720000in}{3.020000in}} %
\pgfusepath{clip}%
\pgfsetbuttcap%
\pgfsetroundjoin%
\definecolor{currentfill}{rgb}{0.264706,0.361242,0.982973}%
\pgfsetfillcolor{currentfill}%
\pgfsetlinewidth{1.003750pt}%
\definecolor{currentstroke}{rgb}{0.264706,0.361242,0.982973}%
\pgfsetstrokecolor{currentstroke}%
\pgfsetdash{}{0pt}%
\pgfpathmoveto{\pgfqpoint{3.328555in}{1.872073in}}%
\pgfpathcurveto{\pgfqpoint{3.339606in}{1.872073in}}{\pgfqpoint{3.350205in}{1.876464in}}{\pgfqpoint{3.358018in}{1.884277in}}%
\pgfpathcurveto{\pgfqpoint{3.365832in}{1.892091in}}{\pgfqpoint{3.370222in}{1.902690in}}{\pgfqpoint{3.370222in}{1.913740in}}%
\pgfpathcurveto{\pgfqpoint{3.370222in}{1.924790in}}{\pgfqpoint{3.365832in}{1.935389in}}{\pgfqpoint{3.358018in}{1.943203in}}%
\pgfpathcurveto{\pgfqpoint{3.350205in}{1.951017in}}{\pgfqpoint{3.339606in}{1.955407in}}{\pgfqpoint{3.328555in}{1.955407in}}%
\pgfpathcurveto{\pgfqpoint{3.317505in}{1.955407in}}{\pgfqpoint{3.306906in}{1.951017in}}{\pgfqpoint{3.299093in}{1.943203in}}%
\pgfpathcurveto{\pgfqpoint{3.291279in}{1.935389in}}{\pgfqpoint{3.286889in}{1.924790in}}{\pgfqpoint{3.286889in}{1.913740in}}%
\pgfpathcurveto{\pgfqpoint{3.286889in}{1.902690in}}{\pgfqpoint{3.291279in}{1.892091in}}{\pgfqpoint{3.299093in}{1.884277in}}%
\pgfpathcurveto{\pgfqpoint{3.306906in}{1.876464in}}{\pgfqpoint{3.317505in}{1.872073in}}{\pgfqpoint{3.328555in}{1.872073in}}%
\pgfpathclose%
\pgfusepath{stroke,fill}%
\end{pgfscope}%
\begin{pgfscope}%
\pgfpathrectangle{\pgfqpoint{0.564660in}{0.521603in}}{\pgfqpoint{3.720000in}{3.020000in}} %
\pgfusepath{clip}%
\pgfsetbuttcap%
\pgfsetroundjoin%
\definecolor{currentfill}{rgb}{0.413725,0.135105,0.997705}%
\pgfsetfillcolor{currentfill}%
\pgfsetlinewidth{1.003750pt}%
\definecolor{currentstroke}{rgb}{0.413725,0.135105,0.997705}%
\pgfsetstrokecolor{currentstroke}%
\pgfsetdash{}{0pt}%
\pgfpathmoveto{\pgfqpoint{2.895722in}{1.797307in}}%
\pgfpathcurveto{\pgfqpoint{2.906772in}{1.797307in}}{\pgfqpoint{2.917371in}{1.801697in}}{\pgfqpoint{2.925185in}{1.809511in}}%
\pgfpathcurveto{\pgfqpoint{2.932998in}{1.817324in}}{\pgfqpoint{2.937389in}{1.827923in}}{\pgfqpoint{2.937389in}{1.838973in}}%
\pgfpathcurveto{\pgfqpoint{2.937389in}{1.850023in}}{\pgfqpoint{2.932998in}{1.860622in}}{\pgfqpoint{2.925185in}{1.868436in}}%
\pgfpathcurveto{\pgfqpoint{2.917371in}{1.876250in}}{\pgfqpoint{2.906772in}{1.880640in}}{\pgfqpoint{2.895722in}{1.880640in}}%
\pgfpathcurveto{\pgfqpoint{2.884672in}{1.880640in}}{\pgfqpoint{2.874073in}{1.876250in}}{\pgfqpoint{2.866259in}{1.868436in}}%
\pgfpathcurveto{\pgfqpoint{2.858445in}{1.860622in}}{\pgfqpoint{2.854055in}{1.850023in}}{\pgfqpoint{2.854055in}{1.838973in}}%
\pgfpathcurveto{\pgfqpoint{2.854055in}{1.827923in}}{\pgfqpoint{2.858445in}{1.817324in}}{\pgfqpoint{2.866259in}{1.809511in}}%
\pgfpathcurveto{\pgfqpoint{2.874073in}{1.801697in}}{\pgfqpoint{2.884672in}{1.797307in}}{\pgfqpoint{2.895722in}{1.797307in}}%
\pgfpathclose%
\pgfusepath{stroke,fill}%
\end{pgfscope}%
\begin{pgfscope}%
\pgfpathrectangle{\pgfqpoint{0.564660in}{0.521603in}}{\pgfqpoint{3.720000in}{3.020000in}} %
\pgfusepath{clip}%
\pgfsetbuttcap%
\pgfsetroundjoin%
\definecolor{currentfill}{rgb}{0.500000,0.000000,1.000000}%
\pgfsetfillcolor{currentfill}%
\pgfsetlinewidth{1.003750pt}%
\definecolor{currentstroke}{rgb}{0.500000,0.000000,1.000000}%
\pgfsetstrokecolor{currentstroke}%
\pgfsetdash{}{0pt}%
\pgfpathmoveto{\pgfqpoint{2.003834in}{2.171024in}}%
\pgfpathcurveto{\pgfqpoint{2.014884in}{2.171024in}}{\pgfqpoint{2.025483in}{2.175415in}}{\pgfqpoint{2.033297in}{2.183228in}}%
\pgfpathcurveto{\pgfqpoint{2.041110in}{2.191042in}}{\pgfqpoint{2.045500in}{2.201641in}}{\pgfqpoint{2.045500in}{2.212691in}}%
\pgfpathcurveto{\pgfqpoint{2.045500in}{2.223741in}}{\pgfqpoint{2.041110in}{2.234340in}}{\pgfqpoint{2.033297in}{2.242154in}}%
\pgfpathcurveto{\pgfqpoint{2.025483in}{2.249968in}}{\pgfqpoint{2.014884in}{2.254358in}}{\pgfqpoint{2.003834in}{2.254358in}}%
\pgfpathcurveto{\pgfqpoint{1.992784in}{2.254358in}}{\pgfqpoint{1.982185in}{2.249968in}}{\pgfqpoint{1.974371in}{2.242154in}}%
\pgfpathcurveto{\pgfqpoint{1.966557in}{2.234340in}}{\pgfqpoint{1.962167in}{2.223741in}}{\pgfqpoint{1.962167in}{2.212691in}}%
\pgfpathcurveto{\pgfqpoint{1.962167in}{2.201641in}}{\pgfqpoint{1.966557in}{2.191042in}}{\pgfqpoint{1.974371in}{2.183228in}}%
\pgfpathcurveto{\pgfqpoint{1.982185in}{2.175415in}}{\pgfqpoint{1.992784in}{2.171024in}}{\pgfqpoint{2.003834in}{2.171024in}}%
\pgfpathclose%
\pgfusepath{stroke,fill}%
\end{pgfscope}%
\begin{pgfscope}%
\pgfpathrectangle{\pgfqpoint{0.564660in}{0.521603in}}{\pgfqpoint{3.720000in}{3.020000in}} %
\pgfusepath{clip}%
\pgfsetbuttcap%
\pgfsetroundjoin%
\definecolor{currentfill}{rgb}{1.000000,0.000000,0.000000}%
\pgfsetfillcolor{currentfill}%
\pgfsetlinewidth{1.003750pt}%
\definecolor{currentstroke}{rgb}{1.000000,0.000000,0.000000}%
\pgfsetstrokecolor{currentstroke}%
\pgfsetdash{}{0pt}%
\pgfpathmoveto{\pgfqpoint{4.166276in}{1.931929in}}%
\pgfpathcurveto{\pgfqpoint{4.177326in}{1.931929in}}{\pgfqpoint{4.187926in}{1.936319in}}{\pgfqpoint{4.195739in}{1.944133in}}%
\pgfpathcurveto{\pgfqpoint{4.203553in}{1.951947in}}{\pgfqpoint{4.207943in}{1.962546in}}{\pgfqpoint{4.207943in}{1.973596in}}%
\pgfpathcurveto{\pgfqpoint{4.207943in}{1.984646in}}{\pgfqpoint{4.203553in}{1.995245in}}{\pgfqpoint{4.195739in}{2.003059in}}%
\pgfpathcurveto{\pgfqpoint{4.187926in}{2.010872in}}{\pgfqpoint{4.177326in}{2.015263in}}{\pgfqpoint{4.166276in}{2.015263in}}%
\pgfpathcurveto{\pgfqpoint{4.155226in}{2.015263in}}{\pgfqpoint{4.144627in}{2.010872in}}{\pgfqpoint{4.136814in}{2.003059in}}%
\pgfpathcurveto{\pgfqpoint{4.129000in}{1.995245in}}{\pgfqpoint{4.124610in}{1.984646in}}{\pgfqpoint{4.124610in}{1.973596in}}%
\pgfpathcurveto{\pgfqpoint{4.124610in}{1.962546in}}{\pgfqpoint{4.129000in}{1.951947in}}{\pgfqpoint{4.136814in}{1.944133in}}%
\pgfpathcurveto{\pgfqpoint{4.144627in}{1.936319in}}{\pgfqpoint{4.155226in}{1.931929in}}{\pgfqpoint{4.166276in}{1.931929in}}%
\pgfpathclose%
\pgfusepath{stroke,fill}%
\end{pgfscope}%
\begin{pgfscope}%
\pgfpathrectangle{\pgfqpoint{0.564660in}{0.521603in}}{\pgfqpoint{3.720000in}{3.020000in}} %
\pgfusepath{clip}%
\pgfsetbuttcap%
\pgfsetroundjoin%
\definecolor{currentfill}{rgb}{1.000000,0.000000,0.000000}%
\pgfsetfillcolor{currentfill}%
\pgfsetlinewidth{1.003750pt}%
\definecolor{currentstroke}{rgb}{1.000000,0.000000,0.000000}%
\pgfsetstrokecolor{currentstroke}%
\pgfsetdash{}{0pt}%
\pgfpathmoveto{\pgfqpoint{3.773192in}{2.699740in}}%
\pgfpathcurveto{\pgfqpoint{3.784242in}{2.699740in}}{\pgfqpoint{3.794841in}{2.704131in}}{\pgfqpoint{3.802655in}{2.711944in}}%
\pgfpathcurveto{\pgfqpoint{3.810468in}{2.719758in}}{\pgfqpoint{3.814859in}{2.730357in}}{\pgfqpoint{3.814859in}{2.741407in}}%
\pgfpathcurveto{\pgfqpoint{3.814859in}{2.752457in}}{\pgfqpoint{3.810468in}{2.763056in}}{\pgfqpoint{3.802655in}{2.770870in}}%
\pgfpathcurveto{\pgfqpoint{3.794841in}{2.778684in}}{\pgfqpoint{3.784242in}{2.783074in}}{\pgfqpoint{3.773192in}{2.783074in}}%
\pgfpathcurveto{\pgfqpoint{3.762142in}{2.783074in}}{\pgfqpoint{3.751543in}{2.778684in}}{\pgfqpoint{3.743729in}{2.770870in}}%
\pgfpathcurveto{\pgfqpoint{3.735915in}{2.763056in}}{\pgfqpoint{3.731525in}{2.752457in}}{\pgfqpoint{3.731525in}{2.741407in}}%
\pgfpathcurveto{\pgfqpoint{3.731525in}{2.730357in}}{\pgfqpoint{3.735915in}{2.719758in}}{\pgfqpoint{3.743729in}{2.711944in}}%
\pgfpathcurveto{\pgfqpoint{3.751543in}{2.704131in}}{\pgfqpoint{3.762142in}{2.699740in}}{\pgfqpoint{3.773192in}{2.699740in}}%
\pgfpathclose%
\pgfusepath{stroke,fill}%
\end{pgfscope}%
\begin{pgfscope}%
\pgfpathrectangle{\pgfqpoint{0.564660in}{0.521603in}}{\pgfqpoint{3.720000in}{3.020000in}} %
\pgfusepath{clip}%
\pgfsetbuttcap%
\pgfsetroundjoin%
\definecolor{currentfill}{rgb}{0.778431,0.905873,0.536867}%
\pgfsetfillcolor{currentfill}%
\pgfsetlinewidth{1.003750pt}%
\definecolor{currentstroke}{rgb}{0.778431,0.905873,0.536867}%
\pgfsetstrokecolor{currentstroke}%
\pgfsetdash{}{0pt}%
\pgfpathmoveto{\pgfqpoint{3.722241in}{2.305818in}}%
\pgfpathcurveto{\pgfqpoint{3.733292in}{2.305818in}}{\pgfqpoint{3.743891in}{2.310209in}}{\pgfqpoint{3.751704in}{2.318022in}}%
\pgfpathcurveto{\pgfqpoint{3.759518in}{2.325836in}}{\pgfqpoint{3.763908in}{2.336435in}}{\pgfqpoint{3.763908in}{2.347485in}}%
\pgfpathcurveto{\pgfqpoint{3.763908in}{2.358535in}}{\pgfqpoint{3.759518in}{2.369134in}}{\pgfqpoint{3.751704in}{2.376948in}}%
\pgfpathcurveto{\pgfqpoint{3.743891in}{2.384761in}}{\pgfqpoint{3.733292in}{2.389152in}}{\pgfqpoint{3.722241in}{2.389152in}}%
\pgfpathcurveto{\pgfqpoint{3.711191in}{2.389152in}}{\pgfqpoint{3.700592in}{2.384761in}}{\pgfqpoint{3.692779in}{2.376948in}}%
\pgfpathcurveto{\pgfqpoint{3.684965in}{2.369134in}}{\pgfqpoint{3.680575in}{2.358535in}}{\pgfqpoint{3.680575in}{2.347485in}}%
\pgfpathcurveto{\pgfqpoint{3.680575in}{2.336435in}}{\pgfqpoint{3.684965in}{2.325836in}}{\pgfqpoint{3.692779in}{2.318022in}}%
\pgfpathcurveto{\pgfqpoint{3.700592in}{2.310209in}}{\pgfqpoint{3.711191in}{2.305818in}}{\pgfqpoint{3.722241in}{2.305818in}}%
\pgfpathclose%
\pgfusepath{stroke,fill}%
\end{pgfscope}%
\begin{pgfscope}%
\pgfpathrectangle{\pgfqpoint{0.564660in}{0.521603in}}{\pgfqpoint{3.720000in}{3.020000in}} %
\pgfusepath{clip}%
\pgfsetbuttcap%
\pgfsetroundjoin%
\definecolor{currentfill}{rgb}{0.778431,0.905873,0.536867}%
\pgfsetfillcolor{currentfill}%
\pgfsetlinewidth{1.003750pt}%
\definecolor{currentstroke}{rgb}{0.778431,0.905873,0.536867}%
\pgfsetstrokecolor{currentstroke}%
\pgfsetdash{}{0pt}%
\pgfpathmoveto{\pgfqpoint{3.468726in}{0.665598in}}%
\pgfpathcurveto{\pgfqpoint{3.479776in}{0.665598in}}{\pgfqpoint{3.490375in}{0.669988in}}{\pgfqpoint{3.498189in}{0.677802in}}%
\pgfpathcurveto{\pgfqpoint{3.506002in}{0.685615in}}{\pgfqpoint{3.510392in}{0.696214in}}{\pgfqpoint{3.510392in}{0.707265in}}%
\pgfpathcurveto{\pgfqpoint{3.510392in}{0.718315in}}{\pgfqpoint{3.506002in}{0.728914in}}{\pgfqpoint{3.498189in}{0.736727in}}%
\pgfpathcurveto{\pgfqpoint{3.490375in}{0.744541in}}{\pgfqpoint{3.479776in}{0.748931in}}{\pgfqpoint{3.468726in}{0.748931in}}%
\pgfpathcurveto{\pgfqpoint{3.457676in}{0.748931in}}{\pgfqpoint{3.447077in}{0.744541in}}{\pgfqpoint{3.439263in}{0.736727in}}%
\pgfpathcurveto{\pgfqpoint{3.431449in}{0.728914in}}{\pgfqpoint{3.427059in}{0.718315in}}{\pgfqpoint{3.427059in}{0.707265in}}%
\pgfpathcurveto{\pgfqpoint{3.427059in}{0.696214in}}{\pgfqpoint{3.431449in}{0.685615in}}{\pgfqpoint{3.439263in}{0.677802in}}%
\pgfpathcurveto{\pgfqpoint{3.447077in}{0.669988in}}{\pgfqpoint{3.457676in}{0.665598in}}{\pgfqpoint{3.468726in}{0.665598in}}%
\pgfpathclose%
\pgfusepath{stroke,fill}%
\end{pgfscope}%
\begin{pgfscope}%
\pgfpathrectangle{\pgfqpoint{0.564660in}{0.521603in}}{\pgfqpoint{3.720000in}{3.020000in}} %
\pgfusepath{clip}%
\pgfsetbuttcap%
\pgfsetroundjoin%
\definecolor{currentfill}{rgb}{0.292157,0.947177,0.812622}%
\pgfsetfillcolor{currentfill}%
\pgfsetlinewidth{1.003750pt}%
\definecolor{currentstroke}{rgb}{0.292157,0.947177,0.812622}%
\pgfsetstrokecolor{currentstroke}%
\pgfsetdash{}{0pt}%
\pgfpathmoveto{\pgfqpoint{3.256354in}{0.933468in}}%
\pgfpathcurveto{\pgfqpoint{3.267405in}{0.933468in}}{\pgfqpoint{3.278004in}{0.937859in}}{\pgfqpoint{3.285817in}{0.945672in}}%
\pgfpathcurveto{\pgfqpoint{3.293631in}{0.953486in}}{\pgfqpoint{3.298021in}{0.964085in}}{\pgfqpoint{3.298021in}{0.975135in}}%
\pgfpathcurveto{\pgfqpoint{3.298021in}{0.986185in}}{\pgfqpoint{3.293631in}{0.996784in}}{\pgfqpoint{3.285817in}{1.004598in}}%
\pgfpathcurveto{\pgfqpoint{3.278004in}{1.012411in}}{\pgfqpoint{3.267405in}{1.016802in}}{\pgfqpoint{3.256354in}{1.016802in}}%
\pgfpathcurveto{\pgfqpoint{3.245304in}{1.016802in}}{\pgfqpoint{3.234705in}{1.012411in}}{\pgfqpoint{3.226892in}{1.004598in}}%
\pgfpathcurveto{\pgfqpoint{3.219078in}{0.996784in}}{\pgfqpoint{3.214688in}{0.986185in}}{\pgfqpoint{3.214688in}{0.975135in}}%
\pgfpathcurveto{\pgfqpoint{3.214688in}{0.964085in}}{\pgfqpoint{3.219078in}{0.953486in}}{\pgfqpoint{3.226892in}{0.945672in}}%
\pgfpathcurveto{\pgfqpoint{3.234705in}{0.937859in}}{\pgfqpoint{3.245304in}{0.933468in}}{\pgfqpoint{3.256354in}{0.933468in}}%
\pgfpathclose%
\pgfusepath{stroke,fill}%
\end{pgfscope}%
\begin{pgfscope}%
\pgfpathrectangle{\pgfqpoint{0.564660in}{0.521603in}}{\pgfqpoint{3.720000in}{3.020000in}} %
\pgfusepath{clip}%
\pgfsetbuttcap%
\pgfsetroundjoin%
\definecolor{currentfill}{rgb}{0.025490,0.734845,0.916034}%
\pgfsetfillcolor{currentfill}%
\pgfsetlinewidth{1.003750pt}%
\definecolor{currentstroke}{rgb}{0.025490,0.734845,0.916034}%
\pgfsetstrokecolor{currentstroke}%
\pgfsetdash{}{0pt}%
\pgfpathmoveto{\pgfqpoint{3.210081in}{0.992260in}}%
\pgfpathcurveto{\pgfqpoint{3.221131in}{0.992260in}}{\pgfqpoint{3.231730in}{0.996651in}}{\pgfqpoint{3.239544in}{1.004464in}}%
\pgfpathcurveto{\pgfqpoint{3.247358in}{1.012278in}}{\pgfqpoint{3.251748in}{1.022877in}}{\pgfqpoint{3.251748in}{1.033927in}}%
\pgfpathcurveto{\pgfqpoint{3.251748in}{1.044977in}}{\pgfqpoint{3.247358in}{1.055576in}}{\pgfqpoint{3.239544in}{1.063390in}}%
\pgfpathcurveto{\pgfqpoint{3.231730in}{1.071203in}}{\pgfqpoint{3.221131in}{1.075594in}}{\pgfqpoint{3.210081in}{1.075594in}}%
\pgfpathcurveto{\pgfqpoint{3.199031in}{1.075594in}}{\pgfqpoint{3.188432in}{1.071203in}}{\pgfqpoint{3.180618in}{1.063390in}}%
\pgfpathcurveto{\pgfqpoint{3.172805in}{1.055576in}}{\pgfqpoint{3.168415in}{1.044977in}}{\pgfqpoint{3.168415in}{1.033927in}}%
\pgfpathcurveto{\pgfqpoint{3.168415in}{1.022877in}}{\pgfqpoint{3.172805in}{1.012278in}}{\pgfqpoint{3.180618in}{1.004464in}}%
\pgfpathcurveto{\pgfqpoint{3.188432in}{0.996651in}}{\pgfqpoint{3.199031in}{0.992260in}}{\pgfqpoint{3.210081in}{0.992260in}}%
\pgfpathclose%
\pgfusepath{stroke,fill}%
\end{pgfscope}%
\begin{pgfscope}%
\pgfpathrectangle{\pgfqpoint{0.564660in}{0.521603in}}{\pgfqpoint{3.720000in}{3.020000in}} %
\pgfusepath{clip}%
\pgfsetbuttcap%
\pgfsetroundjoin%
\definecolor{currentfill}{rgb}{0.178431,0.483911,0.968276}%
\pgfsetfillcolor{currentfill}%
\pgfsetlinewidth{1.003750pt}%
\definecolor{currentstroke}{rgb}{0.178431,0.483911,0.968276}%
\pgfsetstrokecolor{currentstroke}%
\pgfsetdash{}{0pt}%
\pgfpathmoveto{\pgfqpoint{2.706316in}{1.508649in}}%
\pgfpathcurveto{\pgfqpoint{2.717366in}{1.508649in}}{\pgfqpoint{2.727965in}{1.513039in}}{\pgfqpoint{2.735779in}{1.520853in}}%
\pgfpathcurveto{\pgfqpoint{2.743593in}{1.528667in}}{\pgfqpoint{2.747983in}{1.539266in}}{\pgfqpoint{2.747983in}{1.550316in}}%
\pgfpathcurveto{\pgfqpoint{2.747983in}{1.561366in}}{\pgfqpoint{2.743593in}{1.571965in}}{\pgfqpoint{2.735779in}{1.579779in}}%
\pgfpathcurveto{\pgfqpoint{2.727965in}{1.587592in}}{\pgfqpoint{2.717366in}{1.591982in}}{\pgfqpoint{2.706316in}{1.591982in}}%
\pgfpathcurveto{\pgfqpoint{2.695266in}{1.591982in}}{\pgfqpoint{2.684667in}{1.587592in}}{\pgfqpoint{2.676853in}{1.579779in}}%
\pgfpathcurveto{\pgfqpoint{2.669040in}{1.571965in}}{\pgfqpoint{2.664649in}{1.561366in}}{\pgfqpoint{2.664649in}{1.550316in}}%
\pgfpathcurveto{\pgfqpoint{2.664649in}{1.539266in}}{\pgfqpoint{2.669040in}{1.528667in}}{\pgfqpoint{2.676853in}{1.520853in}}%
\pgfpathcurveto{\pgfqpoint{2.684667in}{1.513039in}}{\pgfqpoint{2.695266in}{1.508649in}}{\pgfqpoint{2.706316in}{1.508649in}}%
\pgfpathclose%
\pgfusepath{stroke,fill}%
\end{pgfscope}%
\begin{pgfscope}%
\pgfpathrectangle{\pgfqpoint{0.564660in}{0.521603in}}{\pgfqpoint{3.720000in}{3.020000in}} %
\pgfusepath{clip}%
\pgfsetbuttcap%
\pgfsetroundjoin%
\definecolor{currentfill}{rgb}{0.178431,0.483911,0.968276}%
\pgfsetfillcolor{currentfill}%
\pgfsetlinewidth{1.003750pt}%
\definecolor{currentstroke}{rgb}{0.178431,0.483911,0.968276}%
\pgfsetstrokecolor{currentstroke}%
\pgfsetdash{}{0pt}%
\pgfpathmoveto{\pgfqpoint{2.196868in}{2.811735in}}%
\pgfpathcurveto{\pgfqpoint{2.207918in}{2.811735in}}{\pgfqpoint{2.218517in}{2.816125in}}{\pgfqpoint{2.226330in}{2.823939in}}%
\pgfpathcurveto{\pgfqpoint{2.234144in}{2.831753in}}{\pgfqpoint{2.238534in}{2.842352in}}{\pgfqpoint{2.238534in}{2.853402in}}%
\pgfpathcurveto{\pgfqpoint{2.238534in}{2.864452in}}{\pgfqpoint{2.234144in}{2.875051in}}{\pgfqpoint{2.226330in}{2.882865in}}%
\pgfpathcurveto{\pgfqpoint{2.218517in}{2.890678in}}{\pgfqpoint{2.207918in}{2.895069in}}{\pgfqpoint{2.196868in}{2.895069in}}%
\pgfpathcurveto{\pgfqpoint{2.185818in}{2.895069in}}{\pgfqpoint{2.175219in}{2.890678in}}{\pgfqpoint{2.167405in}{2.882865in}}%
\pgfpathcurveto{\pgfqpoint{2.159591in}{2.875051in}}{\pgfqpoint{2.155201in}{2.864452in}}{\pgfqpoint{2.155201in}{2.853402in}}%
\pgfpathcurveto{\pgfqpoint{2.155201in}{2.842352in}}{\pgfqpoint{2.159591in}{2.831753in}}{\pgfqpoint{2.167405in}{2.823939in}}%
\pgfpathcurveto{\pgfqpoint{2.175219in}{2.816125in}}{\pgfqpoint{2.185818in}{2.811735in}}{\pgfqpoint{2.196868in}{2.811735in}}%
\pgfpathclose%
\pgfusepath{stroke,fill}%
\end{pgfscope}%
\begin{pgfscope}%
\pgfpathrectangle{\pgfqpoint{0.564660in}{0.521603in}}{\pgfqpoint{3.720000in}{3.020000in}} %
\pgfusepath{clip}%
\pgfsetbuttcap%
\pgfsetroundjoin%
\definecolor{currentfill}{rgb}{0.264706,0.361242,0.982973}%
\pgfsetfillcolor{currentfill}%
\pgfsetlinewidth{1.003750pt}%
\definecolor{currentstroke}{rgb}{0.264706,0.361242,0.982973}%
\pgfsetstrokecolor{currentstroke}%
\pgfsetdash{}{0pt}%
\pgfpathmoveto{\pgfqpoint{3.113579in}{2.730222in}}%
\pgfpathcurveto{\pgfqpoint{3.124629in}{2.730222in}}{\pgfqpoint{3.135229in}{2.734613in}}{\pgfqpoint{3.143042in}{2.742426in}}%
\pgfpathcurveto{\pgfqpoint{3.150856in}{2.750240in}}{\pgfqpoint{3.155246in}{2.760839in}}{\pgfqpoint{3.155246in}{2.771889in}}%
\pgfpathcurveto{\pgfqpoint{3.155246in}{2.782939in}}{\pgfqpoint{3.150856in}{2.793538in}}{\pgfqpoint{3.143042in}{2.801352in}}%
\pgfpathcurveto{\pgfqpoint{3.135229in}{2.809165in}}{\pgfqpoint{3.124629in}{2.813556in}}{\pgfqpoint{3.113579in}{2.813556in}}%
\pgfpathcurveto{\pgfqpoint{3.102529in}{2.813556in}}{\pgfqpoint{3.091930in}{2.809165in}}{\pgfqpoint{3.084117in}{2.801352in}}%
\pgfpathcurveto{\pgfqpoint{3.076303in}{2.793538in}}{\pgfqpoint{3.071913in}{2.782939in}}{\pgfqpoint{3.071913in}{2.771889in}}%
\pgfpathcurveto{\pgfqpoint{3.071913in}{2.760839in}}{\pgfqpoint{3.076303in}{2.750240in}}{\pgfqpoint{3.084117in}{2.742426in}}%
\pgfpathcurveto{\pgfqpoint{3.091930in}{2.734613in}}{\pgfqpoint{3.102529in}{2.730222in}}{\pgfqpoint{3.113579in}{2.730222in}}%
\pgfpathclose%
\pgfusepath{stroke,fill}%
\end{pgfscope}%
\begin{pgfscope}%
\pgfpathrectangle{\pgfqpoint{0.564660in}{0.521603in}}{\pgfqpoint{3.720000in}{3.020000in}} %
\pgfusepath{clip}%
\pgfsetbuttcap%
\pgfsetroundjoin%
\definecolor{currentfill}{rgb}{0.264706,0.361242,0.982973}%
\pgfsetfillcolor{currentfill}%
\pgfsetlinewidth{1.003750pt}%
\definecolor{currentstroke}{rgb}{0.264706,0.361242,0.982973}%
\pgfsetstrokecolor{currentstroke}%
\pgfsetdash{}{0pt}%
\pgfpathmoveto{\pgfqpoint{2.812260in}{2.292217in}}%
\pgfpathcurveto{\pgfqpoint{2.823310in}{2.292217in}}{\pgfqpoint{2.833909in}{2.296608in}}{\pgfqpoint{2.841723in}{2.304421in}}%
\pgfpathcurveto{\pgfqpoint{2.849536in}{2.312235in}}{\pgfqpoint{2.853926in}{2.322834in}}{\pgfqpoint{2.853926in}{2.333884in}}%
\pgfpathcurveto{\pgfqpoint{2.853926in}{2.344934in}}{\pgfqpoint{2.849536in}{2.355533in}}{\pgfqpoint{2.841723in}{2.363347in}}%
\pgfpathcurveto{\pgfqpoint{2.833909in}{2.371160in}}{\pgfqpoint{2.823310in}{2.375551in}}{\pgfqpoint{2.812260in}{2.375551in}}%
\pgfpathcurveto{\pgfqpoint{2.801210in}{2.375551in}}{\pgfqpoint{2.790611in}{2.371160in}}{\pgfqpoint{2.782797in}{2.363347in}}%
\pgfpathcurveto{\pgfqpoint{2.774983in}{2.355533in}}{\pgfqpoint{2.770593in}{2.344934in}}{\pgfqpoint{2.770593in}{2.333884in}}%
\pgfpathcurveto{\pgfqpoint{2.770593in}{2.322834in}}{\pgfqpoint{2.774983in}{2.312235in}}{\pgfqpoint{2.782797in}{2.304421in}}%
\pgfpathcurveto{\pgfqpoint{2.790611in}{2.296608in}}{\pgfqpoint{2.801210in}{2.292217in}}{\pgfqpoint{2.812260in}{2.292217in}}%
\pgfpathclose%
\pgfusepath{stroke,fill}%
\end{pgfscope}%
\begin{pgfscope}%
\pgfpathrectangle{\pgfqpoint{0.564660in}{0.521603in}}{\pgfqpoint{3.720000in}{3.020000in}} %
\pgfusepath{clip}%
\pgfsetbuttcap%
\pgfsetroundjoin%
\definecolor{currentfill}{rgb}{0.264706,0.361242,0.982973}%
\pgfsetfillcolor{currentfill}%
\pgfsetlinewidth{1.003750pt}%
\definecolor{currentstroke}{rgb}{0.264706,0.361242,0.982973}%
\pgfsetstrokecolor{currentstroke}%
\pgfsetdash{}{0pt}%
\pgfpathmoveto{\pgfqpoint{2.634682in}{2.192622in}}%
\pgfpathcurveto{\pgfqpoint{2.645732in}{2.192622in}}{\pgfqpoint{2.656331in}{2.197012in}}{\pgfqpoint{2.664145in}{2.204826in}}%
\pgfpathcurveto{\pgfqpoint{2.671958in}{2.212640in}}{\pgfqpoint{2.676349in}{2.223239in}}{\pgfqpoint{2.676349in}{2.234289in}}%
\pgfpathcurveto{\pgfqpoint{2.676349in}{2.245339in}}{\pgfqpoint{2.671958in}{2.255938in}}{\pgfqpoint{2.664145in}{2.263752in}}%
\pgfpathcurveto{\pgfqpoint{2.656331in}{2.271565in}}{\pgfqpoint{2.645732in}{2.275956in}}{\pgfqpoint{2.634682in}{2.275956in}}%
\pgfpathcurveto{\pgfqpoint{2.623632in}{2.275956in}}{\pgfqpoint{2.613033in}{2.271565in}}{\pgfqpoint{2.605219in}{2.263752in}}%
\pgfpathcurveto{\pgfqpoint{2.597406in}{2.255938in}}{\pgfqpoint{2.593015in}{2.245339in}}{\pgfqpoint{2.593015in}{2.234289in}}%
\pgfpathcurveto{\pgfqpoint{2.593015in}{2.223239in}}{\pgfqpoint{2.597406in}{2.212640in}}{\pgfqpoint{2.605219in}{2.204826in}}%
\pgfpathcurveto{\pgfqpoint{2.613033in}{2.197012in}}{\pgfqpoint{2.623632in}{2.192622in}}{\pgfqpoint{2.634682in}{2.192622in}}%
\pgfpathclose%
\pgfusepath{stroke,fill}%
\end{pgfscope}%
\begin{pgfscope}%
\pgfpathrectangle{\pgfqpoint{0.564660in}{0.521603in}}{\pgfqpoint{3.720000in}{3.020000in}} %
\pgfusepath{clip}%
\pgfsetbuttcap%
\pgfsetroundjoin%
\definecolor{currentfill}{rgb}{0.264706,0.361242,0.982973}%
\pgfsetfillcolor{currentfill}%
\pgfsetlinewidth{1.003750pt}%
\definecolor{currentstroke}{rgb}{0.264706,0.361242,0.982973}%
\pgfsetstrokecolor{currentstroke}%
\pgfsetdash{}{0pt}%
\pgfpathmoveto{\pgfqpoint{2.469078in}{1.376730in}}%
\pgfpathcurveto{\pgfqpoint{2.480128in}{1.376730in}}{\pgfqpoint{2.490727in}{1.381120in}}{\pgfqpoint{2.498541in}{1.388934in}}%
\pgfpathcurveto{\pgfqpoint{2.506354in}{1.396747in}}{\pgfqpoint{2.510744in}{1.407346in}}{\pgfqpoint{2.510744in}{1.418396in}}%
\pgfpathcurveto{\pgfqpoint{2.510744in}{1.429447in}}{\pgfqpoint{2.506354in}{1.440046in}}{\pgfqpoint{2.498541in}{1.447859in}}%
\pgfpathcurveto{\pgfqpoint{2.490727in}{1.455673in}}{\pgfqpoint{2.480128in}{1.460063in}}{\pgfqpoint{2.469078in}{1.460063in}}%
\pgfpathcurveto{\pgfqpoint{2.458028in}{1.460063in}}{\pgfqpoint{2.447429in}{1.455673in}}{\pgfqpoint{2.439615in}{1.447859in}}%
\pgfpathcurveto{\pgfqpoint{2.431801in}{1.440046in}}{\pgfqpoint{2.427411in}{1.429447in}}{\pgfqpoint{2.427411in}{1.418396in}}%
\pgfpathcurveto{\pgfqpoint{2.427411in}{1.407346in}}{\pgfqpoint{2.431801in}{1.396747in}}{\pgfqpoint{2.439615in}{1.388934in}}%
\pgfpathcurveto{\pgfqpoint{2.447429in}{1.381120in}}{\pgfqpoint{2.458028in}{1.376730in}}{\pgfqpoint{2.469078in}{1.376730in}}%
\pgfpathclose%
\pgfusepath{stroke,fill}%
\end{pgfscope}%
\begin{pgfscope}%
\pgfpathrectangle{\pgfqpoint{0.564660in}{0.521603in}}{\pgfqpoint{3.720000in}{3.020000in}} %
\pgfusepath{clip}%
\pgfsetbuttcap%
\pgfsetroundjoin%
\definecolor{currentfill}{rgb}{0.264706,0.361242,0.982973}%
\pgfsetfillcolor{currentfill}%
\pgfsetlinewidth{1.003750pt}%
\definecolor{currentstroke}{rgb}{0.264706,0.361242,0.982973}%
\pgfsetstrokecolor{currentstroke}%
\pgfsetdash{}{0pt}%
\pgfpathmoveto{\pgfqpoint{2.011780in}{3.314275in}}%
\pgfpathcurveto{\pgfqpoint{2.022830in}{3.314275in}}{\pgfqpoint{2.033429in}{3.318666in}}{\pgfqpoint{2.041243in}{3.326479in}}%
\pgfpathcurveto{\pgfqpoint{2.049057in}{3.334293in}}{\pgfqpoint{2.053447in}{3.344892in}}{\pgfqpoint{2.053447in}{3.355942in}}%
\pgfpathcurveto{\pgfqpoint{2.053447in}{3.366992in}}{\pgfqpoint{2.049057in}{3.377591in}}{\pgfqpoint{2.041243in}{3.385405in}}%
\pgfpathcurveto{\pgfqpoint{2.033429in}{3.393219in}}{\pgfqpoint{2.022830in}{3.397609in}}{\pgfqpoint{2.011780in}{3.397609in}}%
\pgfpathcurveto{\pgfqpoint{2.000730in}{3.397609in}}{\pgfqpoint{1.990131in}{3.393219in}}{\pgfqpoint{1.982317in}{3.385405in}}%
\pgfpathcurveto{\pgfqpoint{1.974504in}{3.377591in}}{\pgfqpoint{1.970113in}{3.366992in}}{\pgfqpoint{1.970113in}{3.355942in}}%
\pgfpathcurveto{\pgfqpoint{1.970113in}{3.344892in}}{\pgfqpoint{1.974504in}{3.334293in}}{\pgfqpoint{1.982317in}{3.326479in}}%
\pgfpathcurveto{\pgfqpoint{1.990131in}{3.318666in}}{\pgfqpoint{2.000730in}{3.314275in}}{\pgfqpoint{2.011780in}{3.314275in}}%
\pgfpathclose%
\pgfusepath{stroke,fill}%
\end{pgfscope}%
\begin{pgfscope}%
\pgfpathrectangle{\pgfqpoint{0.564660in}{0.521603in}}{\pgfqpoint{3.720000in}{3.020000in}} %
\pgfusepath{clip}%
\pgfsetbuttcap%
\pgfsetroundjoin%
\definecolor{currentfill}{rgb}{0.645098,0.974139,0.622113}%
\pgfsetfillcolor{currentfill}%
\pgfsetlinewidth{1.003750pt}%
\definecolor{currentstroke}{rgb}{0.645098,0.974139,0.622113}%
\pgfsetstrokecolor{currentstroke}%
\pgfsetdash{}{0pt}%
\pgfpathmoveto{\pgfqpoint{3.297585in}{2.473743in}}%
\pgfpathcurveto{\pgfqpoint{3.308636in}{2.473743in}}{\pgfqpoint{3.319235in}{2.478133in}}{\pgfqpoint{3.327048in}{2.485947in}}%
\pgfpathcurveto{\pgfqpoint{3.334862in}{2.493761in}}{\pgfqpoint{3.339252in}{2.504360in}}{\pgfqpoint{3.339252in}{2.515410in}}%
\pgfpathcurveto{\pgfqpoint{3.339252in}{2.526460in}}{\pgfqpoint{3.334862in}{2.537059in}}{\pgfqpoint{3.327048in}{2.544872in}}%
\pgfpathcurveto{\pgfqpoint{3.319235in}{2.552686in}}{\pgfqpoint{3.308636in}{2.557076in}}{\pgfqpoint{3.297585in}{2.557076in}}%
\pgfpathcurveto{\pgfqpoint{3.286535in}{2.557076in}}{\pgfqpoint{3.275936in}{2.552686in}}{\pgfqpoint{3.268123in}{2.544872in}}%
\pgfpathcurveto{\pgfqpoint{3.260309in}{2.537059in}}{\pgfqpoint{3.255919in}{2.526460in}}{\pgfqpoint{3.255919in}{2.515410in}}%
\pgfpathcurveto{\pgfqpoint{3.255919in}{2.504360in}}{\pgfqpoint{3.260309in}{2.493761in}}{\pgfqpoint{3.268123in}{2.485947in}}%
\pgfpathcurveto{\pgfqpoint{3.275936in}{2.478133in}}{\pgfqpoint{3.286535in}{2.473743in}}{\pgfqpoint{3.297585in}{2.473743in}}%
\pgfpathclose%
\pgfusepath{stroke,fill}%
\end{pgfscope}%
\begin{pgfscope}%
\pgfsetbuttcap%
\pgfsetroundjoin%
\definecolor{currentfill}{rgb}{0.000000,0.000000,0.000000}%
\pgfsetfillcolor{currentfill}%
\pgfsetlinewidth{0.803000pt}%
\definecolor{currentstroke}{rgb}{0.000000,0.000000,0.000000}%
\pgfsetstrokecolor{currentstroke}%
\pgfsetdash{}{0pt}%
\pgfsys@defobject{currentmarker}{\pgfqpoint{0.000000in}{-0.048611in}}{\pgfqpoint{0.000000in}{0.000000in}}{%
\pgfpathmoveto{\pgfqpoint{0.000000in}{0.000000in}}%
\pgfpathlineto{\pgfqpoint{0.000000in}{-0.048611in}}%
\pgfusepath{stroke,fill}%
}%
\begin{pgfscope}%
\pgfsys@transformshift{0.564660in}{0.521603in}%
\pgfsys@useobject{currentmarker}{}%
\end{pgfscope}%
\end{pgfscope}%
\begin{pgfscope}%
\pgftext[x=0.564660in,y=0.424381in,,top]{\rmfamily\fontsize{10.000000}{12.000000}\selectfont \(\displaystyle 10^{-2}\)}%
\end{pgfscope}%
\begin{pgfscope}%
\pgfsetbuttcap%
\pgfsetroundjoin%
\definecolor{currentfill}{rgb}{0.000000,0.000000,0.000000}%
\pgfsetfillcolor{currentfill}%
\pgfsetlinewidth{0.803000pt}%
\definecolor{currentstroke}{rgb}{0.000000,0.000000,0.000000}%
\pgfsetstrokecolor{currentstroke}%
\pgfsetdash{}{0pt}%
\pgfsys@defobject{currentmarker}{\pgfqpoint{0.000000in}{-0.048611in}}{\pgfqpoint{0.000000in}{0.000000in}}{%
\pgfpathmoveto{\pgfqpoint{0.000000in}{0.000000in}}%
\pgfpathlineto{\pgfqpoint{0.000000in}{-0.048611in}}%
\pgfusepath{stroke,fill}%
}%
\begin{pgfscope}%
\pgfsys@transformshift{2.397001in}{0.521603in}%
\pgfsys@useobject{currentmarker}{}%
\end{pgfscope}%
\end{pgfscope}%
\begin{pgfscope}%
\pgftext[x=2.397001in,y=0.424381in,,top]{\rmfamily\fontsize{10.000000}{12.000000}\selectfont \(\displaystyle 10^{-1}\)}%
\end{pgfscope}%
\begin{pgfscope}%
\pgfsetbuttcap%
\pgfsetroundjoin%
\definecolor{currentfill}{rgb}{0.000000,0.000000,0.000000}%
\pgfsetfillcolor{currentfill}%
\pgfsetlinewidth{0.803000pt}%
\definecolor{currentstroke}{rgb}{0.000000,0.000000,0.000000}%
\pgfsetstrokecolor{currentstroke}%
\pgfsetdash{}{0pt}%
\pgfsys@defobject{currentmarker}{\pgfqpoint{0.000000in}{-0.048611in}}{\pgfqpoint{0.000000in}{0.000000in}}{%
\pgfpathmoveto{\pgfqpoint{0.000000in}{0.000000in}}%
\pgfpathlineto{\pgfqpoint{0.000000in}{-0.048611in}}%
\pgfusepath{stroke,fill}%
}%
\begin{pgfscope}%
\pgfsys@transformshift{4.229341in}{0.521603in}%
\pgfsys@useobject{currentmarker}{}%
\end{pgfscope}%
\end{pgfscope}%
\begin{pgfscope}%
\pgftext[x=4.229341in,y=0.424381in,,top]{\rmfamily\fontsize{10.000000}{12.000000}\selectfont \(\displaystyle 10^{0}\)}%
\end{pgfscope}%
\begin{pgfscope}%
\pgfsetbuttcap%
\pgfsetroundjoin%
\definecolor{currentfill}{rgb}{0.000000,0.000000,0.000000}%
\pgfsetfillcolor{currentfill}%
\pgfsetlinewidth{0.602250pt}%
\definecolor{currentstroke}{rgb}{0.000000,0.000000,0.000000}%
\pgfsetstrokecolor{currentstroke}%
\pgfsetdash{}{0pt}%
\pgfsys@defobject{currentmarker}{\pgfqpoint{0.000000in}{-0.027778in}}{\pgfqpoint{0.000000in}{0.000000in}}{%
\pgfpathmoveto{\pgfqpoint{0.000000in}{0.000000in}}%
\pgfpathlineto{\pgfqpoint{0.000000in}{-0.027778in}}%
\pgfusepath{stroke,fill}%
}%
\begin{pgfscope}%
\pgfsys@transformshift{1.116250in}{0.521603in}%
\pgfsys@useobject{currentmarker}{}%
\end{pgfscope}%
\end{pgfscope}%
\begin{pgfscope}%
\pgfsetbuttcap%
\pgfsetroundjoin%
\definecolor{currentfill}{rgb}{0.000000,0.000000,0.000000}%
\pgfsetfillcolor{currentfill}%
\pgfsetlinewidth{0.602250pt}%
\definecolor{currentstroke}{rgb}{0.000000,0.000000,0.000000}%
\pgfsetstrokecolor{currentstroke}%
\pgfsetdash{}{0pt}%
\pgfsys@defobject{currentmarker}{\pgfqpoint{0.000000in}{-0.027778in}}{\pgfqpoint{0.000000in}{0.000000in}}{%
\pgfpathmoveto{\pgfqpoint{0.000000in}{0.000000in}}%
\pgfpathlineto{\pgfqpoint{0.000000in}{-0.027778in}}%
\pgfusepath{stroke,fill}%
}%
\begin{pgfscope}%
\pgfsys@transformshift{1.438909in}{0.521603in}%
\pgfsys@useobject{currentmarker}{}%
\end{pgfscope}%
\end{pgfscope}%
\begin{pgfscope}%
\pgfsetbuttcap%
\pgfsetroundjoin%
\definecolor{currentfill}{rgb}{0.000000,0.000000,0.000000}%
\pgfsetfillcolor{currentfill}%
\pgfsetlinewidth{0.602250pt}%
\definecolor{currentstroke}{rgb}{0.000000,0.000000,0.000000}%
\pgfsetstrokecolor{currentstroke}%
\pgfsetdash{}{0pt}%
\pgfsys@defobject{currentmarker}{\pgfqpoint{0.000000in}{-0.027778in}}{\pgfqpoint{0.000000in}{0.000000in}}{%
\pgfpathmoveto{\pgfqpoint{0.000000in}{0.000000in}}%
\pgfpathlineto{\pgfqpoint{0.000000in}{-0.027778in}}%
\pgfusepath{stroke,fill}%
}%
\begin{pgfscope}%
\pgfsys@transformshift{1.667839in}{0.521603in}%
\pgfsys@useobject{currentmarker}{}%
\end{pgfscope}%
\end{pgfscope}%
\begin{pgfscope}%
\pgfsetbuttcap%
\pgfsetroundjoin%
\definecolor{currentfill}{rgb}{0.000000,0.000000,0.000000}%
\pgfsetfillcolor{currentfill}%
\pgfsetlinewidth{0.602250pt}%
\definecolor{currentstroke}{rgb}{0.000000,0.000000,0.000000}%
\pgfsetstrokecolor{currentstroke}%
\pgfsetdash{}{0pt}%
\pgfsys@defobject{currentmarker}{\pgfqpoint{0.000000in}{-0.027778in}}{\pgfqpoint{0.000000in}{0.000000in}}{%
\pgfpathmoveto{\pgfqpoint{0.000000in}{0.000000in}}%
\pgfpathlineto{\pgfqpoint{0.000000in}{-0.027778in}}%
\pgfusepath{stroke,fill}%
}%
\begin{pgfscope}%
\pgfsys@transformshift{1.845411in}{0.521603in}%
\pgfsys@useobject{currentmarker}{}%
\end{pgfscope}%
\end{pgfscope}%
\begin{pgfscope}%
\pgfsetbuttcap%
\pgfsetroundjoin%
\definecolor{currentfill}{rgb}{0.000000,0.000000,0.000000}%
\pgfsetfillcolor{currentfill}%
\pgfsetlinewidth{0.602250pt}%
\definecolor{currentstroke}{rgb}{0.000000,0.000000,0.000000}%
\pgfsetstrokecolor{currentstroke}%
\pgfsetdash{}{0pt}%
\pgfsys@defobject{currentmarker}{\pgfqpoint{0.000000in}{-0.027778in}}{\pgfqpoint{0.000000in}{0.000000in}}{%
\pgfpathmoveto{\pgfqpoint{0.000000in}{0.000000in}}%
\pgfpathlineto{\pgfqpoint{0.000000in}{-0.027778in}}%
\pgfusepath{stroke,fill}%
}%
\begin{pgfscope}%
\pgfsys@transformshift{1.990498in}{0.521603in}%
\pgfsys@useobject{currentmarker}{}%
\end{pgfscope}%
\end{pgfscope}%
\begin{pgfscope}%
\pgfsetbuttcap%
\pgfsetroundjoin%
\definecolor{currentfill}{rgb}{0.000000,0.000000,0.000000}%
\pgfsetfillcolor{currentfill}%
\pgfsetlinewidth{0.602250pt}%
\definecolor{currentstroke}{rgb}{0.000000,0.000000,0.000000}%
\pgfsetstrokecolor{currentstroke}%
\pgfsetdash{}{0pt}%
\pgfsys@defobject{currentmarker}{\pgfqpoint{0.000000in}{-0.027778in}}{\pgfqpoint{0.000000in}{0.000000in}}{%
\pgfpathmoveto{\pgfqpoint{0.000000in}{0.000000in}}%
\pgfpathlineto{\pgfqpoint{0.000000in}{-0.027778in}}%
\pgfusepath{stroke,fill}%
}%
\begin{pgfscope}%
\pgfsys@transformshift{2.113168in}{0.521603in}%
\pgfsys@useobject{currentmarker}{}%
\end{pgfscope}%
\end{pgfscope}%
\begin{pgfscope}%
\pgfsetbuttcap%
\pgfsetroundjoin%
\definecolor{currentfill}{rgb}{0.000000,0.000000,0.000000}%
\pgfsetfillcolor{currentfill}%
\pgfsetlinewidth{0.602250pt}%
\definecolor{currentstroke}{rgb}{0.000000,0.000000,0.000000}%
\pgfsetstrokecolor{currentstroke}%
\pgfsetdash{}{0pt}%
\pgfsys@defobject{currentmarker}{\pgfqpoint{0.000000in}{-0.027778in}}{\pgfqpoint{0.000000in}{0.000000in}}{%
\pgfpathmoveto{\pgfqpoint{0.000000in}{0.000000in}}%
\pgfpathlineto{\pgfqpoint{0.000000in}{-0.027778in}}%
\pgfusepath{stroke,fill}%
}%
\begin{pgfscope}%
\pgfsys@transformshift{2.219429in}{0.521603in}%
\pgfsys@useobject{currentmarker}{}%
\end{pgfscope}%
\end{pgfscope}%
\begin{pgfscope}%
\pgfsetbuttcap%
\pgfsetroundjoin%
\definecolor{currentfill}{rgb}{0.000000,0.000000,0.000000}%
\pgfsetfillcolor{currentfill}%
\pgfsetlinewidth{0.602250pt}%
\definecolor{currentstroke}{rgb}{0.000000,0.000000,0.000000}%
\pgfsetstrokecolor{currentstroke}%
\pgfsetdash{}{0pt}%
\pgfsys@defobject{currentmarker}{\pgfqpoint{0.000000in}{-0.027778in}}{\pgfqpoint{0.000000in}{0.000000in}}{%
\pgfpathmoveto{\pgfqpoint{0.000000in}{0.000000in}}%
\pgfpathlineto{\pgfqpoint{0.000000in}{-0.027778in}}%
\pgfusepath{stroke,fill}%
}%
\begin{pgfscope}%
\pgfsys@transformshift{2.313157in}{0.521603in}%
\pgfsys@useobject{currentmarker}{}%
\end{pgfscope}%
\end{pgfscope}%
\begin{pgfscope}%
\pgfsetbuttcap%
\pgfsetroundjoin%
\definecolor{currentfill}{rgb}{0.000000,0.000000,0.000000}%
\pgfsetfillcolor{currentfill}%
\pgfsetlinewidth{0.602250pt}%
\definecolor{currentstroke}{rgb}{0.000000,0.000000,0.000000}%
\pgfsetstrokecolor{currentstroke}%
\pgfsetdash{}{0pt}%
\pgfsys@defobject{currentmarker}{\pgfqpoint{0.000000in}{-0.027778in}}{\pgfqpoint{0.000000in}{0.000000in}}{%
\pgfpathmoveto{\pgfqpoint{0.000000in}{0.000000in}}%
\pgfpathlineto{\pgfqpoint{0.000000in}{-0.027778in}}%
\pgfusepath{stroke,fill}%
}%
\begin{pgfscope}%
\pgfsys@transformshift{2.948590in}{0.521603in}%
\pgfsys@useobject{currentmarker}{}%
\end{pgfscope}%
\end{pgfscope}%
\begin{pgfscope}%
\pgfsetbuttcap%
\pgfsetroundjoin%
\definecolor{currentfill}{rgb}{0.000000,0.000000,0.000000}%
\pgfsetfillcolor{currentfill}%
\pgfsetlinewidth{0.602250pt}%
\definecolor{currentstroke}{rgb}{0.000000,0.000000,0.000000}%
\pgfsetstrokecolor{currentstroke}%
\pgfsetdash{}{0pt}%
\pgfsys@defobject{currentmarker}{\pgfqpoint{0.000000in}{-0.027778in}}{\pgfqpoint{0.000000in}{0.000000in}}{%
\pgfpathmoveto{\pgfqpoint{0.000000in}{0.000000in}}%
\pgfpathlineto{\pgfqpoint{0.000000in}{-0.027778in}}%
\pgfusepath{stroke,fill}%
}%
\begin{pgfscope}%
\pgfsys@transformshift{3.271249in}{0.521603in}%
\pgfsys@useobject{currentmarker}{}%
\end{pgfscope}%
\end{pgfscope}%
\begin{pgfscope}%
\pgfsetbuttcap%
\pgfsetroundjoin%
\definecolor{currentfill}{rgb}{0.000000,0.000000,0.000000}%
\pgfsetfillcolor{currentfill}%
\pgfsetlinewidth{0.602250pt}%
\definecolor{currentstroke}{rgb}{0.000000,0.000000,0.000000}%
\pgfsetstrokecolor{currentstroke}%
\pgfsetdash{}{0pt}%
\pgfsys@defobject{currentmarker}{\pgfqpoint{0.000000in}{-0.027778in}}{\pgfqpoint{0.000000in}{0.000000in}}{%
\pgfpathmoveto{\pgfqpoint{0.000000in}{0.000000in}}%
\pgfpathlineto{\pgfqpoint{0.000000in}{-0.027778in}}%
\pgfusepath{stroke,fill}%
}%
\begin{pgfscope}%
\pgfsys@transformshift{3.500180in}{0.521603in}%
\pgfsys@useobject{currentmarker}{}%
\end{pgfscope}%
\end{pgfscope}%
\begin{pgfscope}%
\pgfsetbuttcap%
\pgfsetroundjoin%
\definecolor{currentfill}{rgb}{0.000000,0.000000,0.000000}%
\pgfsetfillcolor{currentfill}%
\pgfsetlinewidth{0.602250pt}%
\definecolor{currentstroke}{rgb}{0.000000,0.000000,0.000000}%
\pgfsetstrokecolor{currentstroke}%
\pgfsetdash{}{0pt}%
\pgfsys@defobject{currentmarker}{\pgfqpoint{0.000000in}{-0.027778in}}{\pgfqpoint{0.000000in}{0.000000in}}{%
\pgfpathmoveto{\pgfqpoint{0.000000in}{0.000000in}}%
\pgfpathlineto{\pgfqpoint{0.000000in}{-0.027778in}}%
\pgfusepath{stroke,fill}%
}%
\begin{pgfscope}%
\pgfsys@transformshift{3.677752in}{0.521603in}%
\pgfsys@useobject{currentmarker}{}%
\end{pgfscope}%
\end{pgfscope}%
\begin{pgfscope}%
\pgfsetbuttcap%
\pgfsetroundjoin%
\definecolor{currentfill}{rgb}{0.000000,0.000000,0.000000}%
\pgfsetfillcolor{currentfill}%
\pgfsetlinewidth{0.602250pt}%
\definecolor{currentstroke}{rgb}{0.000000,0.000000,0.000000}%
\pgfsetstrokecolor{currentstroke}%
\pgfsetdash{}{0pt}%
\pgfsys@defobject{currentmarker}{\pgfqpoint{0.000000in}{-0.027778in}}{\pgfqpoint{0.000000in}{0.000000in}}{%
\pgfpathmoveto{\pgfqpoint{0.000000in}{0.000000in}}%
\pgfpathlineto{\pgfqpoint{0.000000in}{-0.027778in}}%
\pgfusepath{stroke,fill}%
}%
\begin{pgfscope}%
\pgfsys@transformshift{3.822839in}{0.521603in}%
\pgfsys@useobject{currentmarker}{}%
\end{pgfscope}%
\end{pgfscope}%
\begin{pgfscope}%
\pgfsetbuttcap%
\pgfsetroundjoin%
\definecolor{currentfill}{rgb}{0.000000,0.000000,0.000000}%
\pgfsetfillcolor{currentfill}%
\pgfsetlinewidth{0.602250pt}%
\definecolor{currentstroke}{rgb}{0.000000,0.000000,0.000000}%
\pgfsetstrokecolor{currentstroke}%
\pgfsetdash{}{0pt}%
\pgfsys@defobject{currentmarker}{\pgfqpoint{0.000000in}{-0.027778in}}{\pgfqpoint{0.000000in}{0.000000in}}{%
\pgfpathmoveto{\pgfqpoint{0.000000in}{0.000000in}}%
\pgfpathlineto{\pgfqpoint{0.000000in}{-0.027778in}}%
\pgfusepath{stroke,fill}%
}%
\begin{pgfscope}%
\pgfsys@transformshift{3.945508in}{0.521603in}%
\pgfsys@useobject{currentmarker}{}%
\end{pgfscope}%
\end{pgfscope}%
\begin{pgfscope}%
\pgfsetbuttcap%
\pgfsetroundjoin%
\definecolor{currentfill}{rgb}{0.000000,0.000000,0.000000}%
\pgfsetfillcolor{currentfill}%
\pgfsetlinewidth{0.602250pt}%
\definecolor{currentstroke}{rgb}{0.000000,0.000000,0.000000}%
\pgfsetstrokecolor{currentstroke}%
\pgfsetdash{}{0pt}%
\pgfsys@defobject{currentmarker}{\pgfqpoint{0.000000in}{-0.027778in}}{\pgfqpoint{0.000000in}{0.000000in}}{%
\pgfpathmoveto{\pgfqpoint{0.000000in}{0.000000in}}%
\pgfpathlineto{\pgfqpoint{0.000000in}{-0.027778in}}%
\pgfusepath{stroke,fill}%
}%
\begin{pgfscope}%
\pgfsys@transformshift{4.051769in}{0.521603in}%
\pgfsys@useobject{currentmarker}{}%
\end{pgfscope}%
\end{pgfscope}%
\begin{pgfscope}%
\pgfsetbuttcap%
\pgfsetroundjoin%
\definecolor{currentfill}{rgb}{0.000000,0.000000,0.000000}%
\pgfsetfillcolor{currentfill}%
\pgfsetlinewidth{0.602250pt}%
\definecolor{currentstroke}{rgb}{0.000000,0.000000,0.000000}%
\pgfsetstrokecolor{currentstroke}%
\pgfsetdash{}{0pt}%
\pgfsys@defobject{currentmarker}{\pgfqpoint{0.000000in}{-0.027778in}}{\pgfqpoint{0.000000in}{0.000000in}}{%
\pgfpathmoveto{\pgfqpoint{0.000000in}{0.000000in}}%
\pgfpathlineto{\pgfqpoint{0.000000in}{-0.027778in}}%
\pgfusepath{stroke,fill}%
}%
\begin{pgfscope}%
\pgfsys@transformshift{4.145498in}{0.521603in}%
\pgfsys@useobject{currentmarker}{}%
\end{pgfscope}%
\end{pgfscope}%
\begin{pgfscope}%
\pgftext[x=2.424660in,y=0.234413in,,top]{\rmfamily\fontsize{10.000000}{12.000000}\selectfont \(\displaystyle We\)}%
\end{pgfscope}%
\begin{pgfscope}%
\pgfsetbuttcap%
\pgfsetroundjoin%
\definecolor{currentfill}{rgb}{0.000000,0.000000,0.000000}%
\pgfsetfillcolor{currentfill}%
\pgfsetlinewidth{0.803000pt}%
\definecolor{currentstroke}{rgb}{0.000000,0.000000,0.000000}%
\pgfsetstrokecolor{currentstroke}%
\pgfsetdash{}{0pt}%
\pgfsys@defobject{currentmarker}{\pgfqpoint{-0.048611in}{0.000000in}}{\pgfqpoint{0.000000in}{0.000000in}}{%
\pgfpathmoveto{\pgfqpoint{0.000000in}{0.000000in}}%
\pgfpathlineto{\pgfqpoint{-0.048611in}{0.000000in}}%
\pgfusepath{stroke,fill}%
}%
\begin{pgfscope}%
\pgfsys@transformshift{0.564660in}{0.952279in}%
\pgfsys@useobject{currentmarker}{}%
\end{pgfscope}%
\end{pgfscope}%
\begin{pgfscope}%
\pgftext[x=0.289968in,y=0.899518in,left,base]{\rmfamily\fontsize{10.000000}{12.000000}\selectfont \(\displaystyle 0.5\)}%
\end{pgfscope}%
\begin{pgfscope}%
\pgfsetbuttcap%
\pgfsetroundjoin%
\definecolor{currentfill}{rgb}{0.000000,0.000000,0.000000}%
\pgfsetfillcolor{currentfill}%
\pgfsetlinewidth{0.803000pt}%
\definecolor{currentstroke}{rgb}{0.000000,0.000000,0.000000}%
\pgfsetstrokecolor{currentstroke}%
\pgfsetdash{}{0pt}%
\pgfsys@defobject{currentmarker}{\pgfqpoint{-0.048611in}{0.000000in}}{\pgfqpoint{0.000000in}{0.000000in}}{%
\pgfpathmoveto{\pgfqpoint{0.000000in}{0.000000in}}%
\pgfpathlineto{\pgfqpoint{-0.048611in}{0.000000in}}%
\pgfusepath{stroke,fill}%
}%
\begin{pgfscope}%
\pgfsys@transformshift{0.564660in}{1.439390in}%
\pgfsys@useobject{currentmarker}{}%
\end{pgfscope}%
\end{pgfscope}%
\begin{pgfscope}%
\pgftext[x=0.289968in,y=1.386628in,left,base]{\rmfamily\fontsize{10.000000}{12.000000}\selectfont \(\displaystyle 0.6\)}%
\end{pgfscope}%
\begin{pgfscope}%
\pgfsetbuttcap%
\pgfsetroundjoin%
\definecolor{currentfill}{rgb}{0.000000,0.000000,0.000000}%
\pgfsetfillcolor{currentfill}%
\pgfsetlinewidth{0.803000pt}%
\definecolor{currentstroke}{rgb}{0.000000,0.000000,0.000000}%
\pgfsetstrokecolor{currentstroke}%
\pgfsetdash{}{0pt}%
\pgfsys@defobject{currentmarker}{\pgfqpoint{-0.048611in}{0.000000in}}{\pgfqpoint{0.000000in}{0.000000in}}{%
\pgfpathmoveto{\pgfqpoint{0.000000in}{0.000000in}}%
\pgfpathlineto{\pgfqpoint{-0.048611in}{0.000000in}}%
\pgfusepath{stroke,fill}%
}%
\begin{pgfscope}%
\pgfsys@transformshift{0.564660in}{1.926500in}%
\pgfsys@useobject{currentmarker}{}%
\end{pgfscope}%
\end{pgfscope}%
\begin{pgfscope}%
\pgftext[x=0.289968in,y=1.873739in,left,base]{\rmfamily\fontsize{10.000000}{12.000000}\selectfont \(\displaystyle 0.7\)}%
\end{pgfscope}%
\begin{pgfscope}%
\pgfsetbuttcap%
\pgfsetroundjoin%
\definecolor{currentfill}{rgb}{0.000000,0.000000,0.000000}%
\pgfsetfillcolor{currentfill}%
\pgfsetlinewidth{0.803000pt}%
\definecolor{currentstroke}{rgb}{0.000000,0.000000,0.000000}%
\pgfsetstrokecolor{currentstroke}%
\pgfsetdash{}{0pt}%
\pgfsys@defobject{currentmarker}{\pgfqpoint{-0.048611in}{0.000000in}}{\pgfqpoint{0.000000in}{0.000000in}}{%
\pgfpathmoveto{\pgfqpoint{0.000000in}{0.000000in}}%
\pgfpathlineto{\pgfqpoint{-0.048611in}{0.000000in}}%
\pgfusepath{stroke,fill}%
}%
\begin{pgfscope}%
\pgfsys@transformshift{0.564660in}{2.413610in}%
\pgfsys@useobject{currentmarker}{}%
\end{pgfscope}%
\end{pgfscope}%
\begin{pgfscope}%
\pgftext[x=0.289968in,y=2.360849in,left,base]{\rmfamily\fontsize{10.000000}{12.000000}\selectfont \(\displaystyle 0.8\)}%
\end{pgfscope}%
\begin{pgfscope}%
\pgfsetbuttcap%
\pgfsetroundjoin%
\definecolor{currentfill}{rgb}{0.000000,0.000000,0.000000}%
\pgfsetfillcolor{currentfill}%
\pgfsetlinewidth{0.803000pt}%
\definecolor{currentstroke}{rgb}{0.000000,0.000000,0.000000}%
\pgfsetstrokecolor{currentstroke}%
\pgfsetdash{}{0pt}%
\pgfsys@defobject{currentmarker}{\pgfqpoint{-0.048611in}{0.000000in}}{\pgfqpoint{0.000000in}{0.000000in}}{%
\pgfpathmoveto{\pgfqpoint{0.000000in}{0.000000in}}%
\pgfpathlineto{\pgfqpoint{-0.048611in}{0.000000in}}%
\pgfusepath{stroke,fill}%
}%
\begin{pgfscope}%
\pgfsys@transformshift{0.564660in}{2.900721in}%
\pgfsys@useobject{currentmarker}{}%
\end{pgfscope}%
\end{pgfscope}%
\begin{pgfscope}%
\pgftext[x=0.289968in,y=2.847959in,left,base]{\rmfamily\fontsize{10.000000}{12.000000}\selectfont \(\displaystyle 0.9\)}%
\end{pgfscope}%
\begin{pgfscope}%
\pgfsetbuttcap%
\pgfsetroundjoin%
\definecolor{currentfill}{rgb}{0.000000,0.000000,0.000000}%
\pgfsetfillcolor{currentfill}%
\pgfsetlinewidth{0.803000pt}%
\definecolor{currentstroke}{rgb}{0.000000,0.000000,0.000000}%
\pgfsetstrokecolor{currentstroke}%
\pgfsetdash{}{0pt}%
\pgfsys@defobject{currentmarker}{\pgfqpoint{-0.048611in}{0.000000in}}{\pgfqpoint{0.000000in}{0.000000in}}{%
\pgfpathmoveto{\pgfqpoint{0.000000in}{0.000000in}}%
\pgfpathlineto{\pgfqpoint{-0.048611in}{0.000000in}}%
\pgfusepath{stroke,fill}%
}%
\begin{pgfscope}%
\pgfsys@transformshift{0.564660in}{3.387831in}%
\pgfsys@useobject{currentmarker}{}%
\end{pgfscope}%
\end{pgfscope}%
\begin{pgfscope}%
\pgftext[x=0.289968in,y=3.335069in,left,base]{\rmfamily\fontsize{10.000000}{12.000000}\selectfont \(\displaystyle 1.0\)}%
\end{pgfscope}%
\begin{pgfscope}%
\pgftext[x=0.234413in,y=2.031603in,,bottom,rotate=90.000000]{\rmfamily\fontsize{10.000000}{12.000000}\selectfont \(\displaystyle C_r\)}%
\end{pgfscope}%
\begin{pgfscope}%
\pgfsetrectcap%
\pgfsetmiterjoin%
\pgfsetlinewidth{0.803000pt}%
\definecolor{currentstroke}{rgb}{0.000000,0.000000,0.000000}%
\pgfsetstrokecolor{currentstroke}%
\pgfsetdash{}{0pt}%
\pgfpathmoveto{\pgfqpoint{0.564660in}{0.521603in}}%
\pgfpathlineto{\pgfqpoint{0.564660in}{3.541603in}}%
\pgfusepath{stroke}%
\end{pgfscope}%
\begin{pgfscope}%
\pgfsetrectcap%
\pgfsetmiterjoin%
\pgfsetlinewidth{0.803000pt}%
\definecolor{currentstroke}{rgb}{0.000000,0.000000,0.000000}%
\pgfsetstrokecolor{currentstroke}%
\pgfsetdash{}{0pt}%
\pgfpathmoveto{\pgfqpoint{4.284660in}{0.521603in}}%
\pgfpathlineto{\pgfqpoint{4.284660in}{3.541603in}}%
\pgfusepath{stroke}%
\end{pgfscope}%
\begin{pgfscope}%
\pgfsetrectcap%
\pgfsetmiterjoin%
\pgfsetlinewidth{0.803000pt}%
\definecolor{currentstroke}{rgb}{0.000000,0.000000,0.000000}%
\pgfsetstrokecolor{currentstroke}%
\pgfsetdash{}{0pt}%
\pgfpathmoveto{\pgfqpoint{0.564660in}{0.521603in}}%
\pgfpathlineto{\pgfqpoint{4.284660in}{0.521603in}}%
\pgfusepath{stroke}%
\end{pgfscope}%
\begin{pgfscope}%
\pgfsetrectcap%
\pgfsetmiterjoin%
\pgfsetlinewidth{0.803000pt}%
\definecolor{currentstroke}{rgb}{0.000000,0.000000,0.000000}%
\pgfsetstrokecolor{currentstroke}%
\pgfsetdash{}{0pt}%
\pgfpathmoveto{\pgfqpoint{0.564660in}{3.541603in}}%
\pgfpathlineto{\pgfqpoint{4.284660in}{3.541603in}}%
\pgfusepath{stroke}%
\end{pgfscope}%
\begin{pgfscope}%
\pgfpathrectangle{\pgfqpoint{4.517160in}{0.521603in}}{\pgfqpoint{0.151000in}{3.020000in}} %
\pgfusepath{clip}%
\pgfsetbuttcap%
\pgfsetmiterjoin%
\definecolor{currentfill}{rgb}{1.000000,1.000000,1.000000}%
\pgfsetfillcolor{currentfill}%
\pgfsetlinewidth{0.010037pt}%
\definecolor{currentstroke}{rgb}{1.000000,1.000000,1.000000}%
\pgfsetstrokecolor{currentstroke}%
\pgfsetdash{}{0pt}%
\pgfpathmoveto{\pgfqpoint{4.517160in}{0.521603in}}%
\pgfpathlineto{\pgfqpoint{4.517160in}{0.533400in}}%
\pgfpathlineto{\pgfqpoint{4.517160in}{3.529806in}}%
\pgfpathlineto{\pgfqpoint{4.517160in}{3.541603in}}%
\pgfpathlineto{\pgfqpoint{4.668160in}{3.541603in}}%
\pgfpathlineto{\pgfqpoint{4.668160in}{3.529806in}}%
\pgfpathlineto{\pgfqpoint{4.668160in}{0.533400in}}%
\pgfpathlineto{\pgfqpoint{4.668160in}{0.521603in}}%
\pgfpathclose%
\pgfusepath{stroke,fill}%
\end{pgfscope}%
\begin{pgfscope}%
\pgfsys@transformshift{4.520000in}{0.526603in}%
\pgftext[left,bottom]{\pgfimage[interpolate=true,width=0.150000in,height=3.020000in]{restitution-img0.png}}%
\end{pgfscope}%
\begin{pgfscope}%
\pgfsetbuttcap%
\pgfsetroundjoin%
\definecolor{currentfill}{rgb}{0.000000,0.000000,0.000000}%
\pgfsetfillcolor{currentfill}%
\pgfsetlinewidth{0.803000pt}%
\definecolor{currentstroke}{rgb}{0.000000,0.000000,0.000000}%
\pgfsetstrokecolor{currentstroke}%
\pgfsetdash{}{0pt}%
\pgfsys@defobject{currentmarker}{\pgfqpoint{0.000000in}{0.000000in}}{\pgfqpoint{0.048611in}{0.000000in}}{%
\pgfpathmoveto{\pgfqpoint{0.000000in}{0.000000in}}%
\pgfpathlineto{\pgfqpoint{0.048611in}{0.000000in}}%
\pgfusepath{stroke,fill}%
}%
\begin{pgfscope}%
\pgfsys@transformshift{4.668160in}{0.991682in}%
\pgfsys@useobject{currentmarker}{}%
\end{pgfscope}%
\end{pgfscope}%
\begin{pgfscope}%
\pgftext[x=4.765383in,y=0.938921in,left,base]{\rmfamily\fontsize{10.000000}{12.000000}\selectfont \(\displaystyle 0.2\)}%
\end{pgfscope}%
\begin{pgfscope}%
\pgfsetbuttcap%
\pgfsetroundjoin%
\definecolor{currentfill}{rgb}{0.000000,0.000000,0.000000}%
\pgfsetfillcolor{currentfill}%
\pgfsetlinewidth{0.803000pt}%
\definecolor{currentstroke}{rgb}{0.000000,0.000000,0.000000}%
\pgfsetstrokecolor{currentstroke}%
\pgfsetdash{}{0pt}%
\pgfsys@defobject{currentmarker}{\pgfqpoint{0.000000in}{0.000000in}}{\pgfqpoint{0.048611in}{0.000000in}}{%
\pgfpathmoveto{\pgfqpoint{0.000000in}{0.000000in}}%
\pgfpathlineto{\pgfqpoint{0.048611in}{0.000000in}}%
\pgfusepath{stroke,fill}%
}%
\begin{pgfscope}%
\pgfsys@transformshift{4.668160in}{1.583298in}%
\pgfsys@useobject{currentmarker}{}%
\end{pgfscope}%
\end{pgfscope}%
\begin{pgfscope}%
\pgftext[x=4.765383in,y=1.530536in,left,base]{\rmfamily\fontsize{10.000000}{12.000000}\selectfont \(\displaystyle 0.4\)}%
\end{pgfscope}%
\begin{pgfscope}%
\pgfsetbuttcap%
\pgfsetroundjoin%
\definecolor{currentfill}{rgb}{0.000000,0.000000,0.000000}%
\pgfsetfillcolor{currentfill}%
\pgfsetlinewidth{0.803000pt}%
\definecolor{currentstroke}{rgb}{0.000000,0.000000,0.000000}%
\pgfsetstrokecolor{currentstroke}%
\pgfsetdash{}{0pt}%
\pgfsys@defobject{currentmarker}{\pgfqpoint{0.000000in}{0.000000in}}{\pgfqpoint{0.048611in}{0.000000in}}{%
\pgfpathmoveto{\pgfqpoint{0.000000in}{0.000000in}}%
\pgfpathlineto{\pgfqpoint{0.048611in}{0.000000in}}%
\pgfusepath{stroke,fill}%
}%
\begin{pgfscope}%
\pgfsys@transformshift{4.668160in}{2.174913in}%
\pgfsys@useobject{currentmarker}{}%
\end{pgfscope}%
\end{pgfscope}%
\begin{pgfscope}%
\pgftext[x=4.765383in,y=2.122152in,left,base]{\rmfamily\fontsize{10.000000}{12.000000}\selectfont \(\displaystyle 0.6\)}%
\end{pgfscope}%
\begin{pgfscope}%
\pgfsetbuttcap%
\pgfsetroundjoin%
\definecolor{currentfill}{rgb}{0.000000,0.000000,0.000000}%
\pgfsetfillcolor{currentfill}%
\pgfsetlinewidth{0.803000pt}%
\definecolor{currentstroke}{rgb}{0.000000,0.000000,0.000000}%
\pgfsetstrokecolor{currentstroke}%
\pgfsetdash{}{0pt}%
\pgfsys@defobject{currentmarker}{\pgfqpoint{0.000000in}{0.000000in}}{\pgfqpoint{0.048611in}{0.000000in}}{%
\pgfpathmoveto{\pgfqpoint{0.000000in}{0.000000in}}%
\pgfpathlineto{\pgfqpoint{0.048611in}{0.000000in}}%
\pgfusepath{stroke,fill}%
}%
\begin{pgfscope}%
\pgfsys@transformshift{4.668160in}{2.766529in}%
\pgfsys@useobject{currentmarker}{}%
\end{pgfscope}%
\end{pgfscope}%
\begin{pgfscope}%
\pgftext[x=4.765383in,y=2.713767in,left,base]{\rmfamily\fontsize{10.000000}{12.000000}\selectfont \(\displaystyle 0.8\)}%
\end{pgfscope}%
\begin{pgfscope}%
\pgfsetbuttcap%
\pgfsetroundjoin%
\definecolor{currentfill}{rgb}{0.000000,0.000000,0.000000}%
\pgfsetfillcolor{currentfill}%
\pgfsetlinewidth{0.803000pt}%
\definecolor{currentstroke}{rgb}{0.000000,0.000000,0.000000}%
\pgfsetstrokecolor{currentstroke}%
\pgfsetdash{}{0pt}%
\pgfsys@defobject{currentmarker}{\pgfqpoint{0.000000in}{0.000000in}}{\pgfqpoint{0.048611in}{0.000000in}}{%
\pgfpathmoveto{\pgfqpoint{0.000000in}{0.000000in}}%
\pgfpathlineto{\pgfqpoint{0.048611in}{0.000000in}}%
\pgfusepath{stroke,fill}%
}%
\begin{pgfscope}%
\pgfsys@transformshift{4.668160in}{3.358144in}%
\pgfsys@useobject{currentmarker}{}%
\end{pgfscope}%
\end{pgfscope}%
\begin{pgfscope}%
\pgftext[x=4.765383in,y=3.305382in,left,base]{\rmfamily\fontsize{10.000000}{12.000000}\selectfont \(\displaystyle 1.0\)}%
\end{pgfscope}%
\begin{pgfscope}%
\pgftext[x=4.998408in,y=2.031603in,,top,rotate=90.000000]{\rmfamily\fontsize{10.000000}{12.000000}\selectfont \(\displaystyle \mathrm{\mathit{Bo_e}} \equiv \frac{\epsilon E_0^2 R_0}{\gamma}\)}%
\end{pgfscope}%
\begin{pgfscope}%
\pgfsetbuttcap%
\pgfsetmiterjoin%
\pgfsetlinewidth{0.803000pt}%
\definecolor{currentstroke}{rgb}{0.000000,0.000000,0.000000}%
\pgfsetstrokecolor{currentstroke}%
\pgfsetdash{}{0pt}%
\pgfpathmoveto{\pgfqpoint{4.517160in}{0.521603in}}%
\pgfpathlineto{\pgfqpoint{4.517160in}{0.533400in}}%
\pgfpathlineto{\pgfqpoint{4.517160in}{3.529806in}}%
\pgfpathlineto{\pgfqpoint{4.517160in}{3.541603in}}%
\pgfpathlineto{\pgfqpoint{4.668160in}{3.541603in}}%
\pgfpathlineto{\pgfqpoint{4.668160in}{3.529806in}}%
\pgfpathlineto{\pgfqpoint{4.668160in}{0.533400in}}%
\pgfpathlineto{\pgfqpoint{4.668160in}{0.521603in}}%
\pgfpathclose%
\pgfusepath{stroke}%
\end{pgfscope}%
\end{pgfpicture}%
\makeatother%
\endgroup%

    \caption{A simple EMA plot.\label{fig:restitution}}
\end{figure}

\end{itemize} 
\end{document}

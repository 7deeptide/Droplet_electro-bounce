\documentclass[10pt,a4paper]{article}
\usepackage[utf8]{inputenc}
%\usepackage{fontspec} % This line only for XeLaTeX and LuaLaTeX
\usepackage{pgfplots}
\usepackage{pgf}
\usepackage[english]{babel}
\usepackage{amsmath}
\usepackage{amsfonts}
\usepackage{amssymb}
\usepackage{graphicx}
\graphicspath{ {../figures/} }
%\usepackage{svg}
\usepackage{verbatim}
\usepackage{color,soul}
\usepackage{listings}
\usepackage{setspace}
\author{Erin Schmidt}

\newlength\figureheight
\newlength\figurewidth
\setlength\figureheight{7cm}
\setlength\figurewidth{10cm}

\let\pgfimageWithoutPath\pgfimage 
\renewcommand{\pgfimage}[2][]{\pgfimageWithoutPath[#1]{../figures/#2}}

\usepackage[
backend=biber,
style=phys,
sorting=none
]{biblatex}
\addbibresource{thesis.bib}

\begin{document}

\doublespacing
\section{Charge Estimates}
We found the distribution of mostly likely experimental net charges for a population of the droplets jumped in low-gravity. A covariance plot of the model variables is shown in Figure \ref{fig:scatter}. The multicollinear dependence of charge on droplet surface area, $A$, and the characteristic electric field, $E_0$, is evident. Assuming the main effect is the interaction between charge and electric field, a Robust Least Squares model fit $q \sim kAE_0$ (using the Python \verb|statsmodels.formula.api.RLM| function), with the non-linear transformation $A = V_d^{2/3}$, found that $k=5.01 \times 10^{-11} \pm  2.85 \times 10^{-11}$ with $R^2 = 0.946$. This model uses Huber's T norm, median absolute scaling, and H1 covariance estimation. A contour plot showing the estimated droplet free charge as a function of $V_d$ and $\varphi_s$ is shown in Figure \ref{fig:charge}.
\begin{figure}[htb]
    \centering
    \resizebox{12cm}{!}{%% Creator: Matplotlib, PGF backend
%%
%% To include the figure in your LaTeX document, write
%%   \input{<filename>.pgf}
%%
%% Make sure the required packages are loaded in your preamble
%%   \usepackage{pgf}
%%
%% Figures using additional raster images can only be included by \input if
%% they are in the same directory as the main LaTeX file. For loading figures
%% from other directories you can use the `import` package
%%   \usepackage{import}
%% and then include the figures with
%%   \import{<path to file>}{<filename>.pgf}
%%
%% Matplotlib used the following preamble
%%   \usepackage{fontspec}
%%   \setmainfont{DejaVu Serif}
%%   \setsansfont{DejaVu Sans}
%%   \setmonofont{DejaVu Sans Mono}
%%
\begingroup%
\makeatletter%
\begin{pgfpicture}%
\pgfpathrectangle{\pgfpointorigin}{\pgfqpoint{5.618185in}{3.988185in}}%
\pgfusepath{use as bounding box, clip}%
\begin{pgfscope}%
\pgfsetbuttcap%
\pgfsetmiterjoin%
\definecolor{currentfill}{rgb}{1.000000,1.000000,1.000000}%
\pgfsetfillcolor{currentfill}%
\pgfsetlinewidth{0.000000pt}%
\definecolor{currentstroke}{rgb}{1.000000,1.000000,1.000000}%
\pgfsetstrokecolor{currentstroke}%
\pgfsetdash{}{0pt}%
\pgfpathmoveto{\pgfqpoint{0.000000in}{0.000000in}}%
\pgfpathlineto{\pgfqpoint{5.618185in}{0.000000in}}%
\pgfpathlineto{\pgfqpoint{5.618185in}{3.988185in}}%
\pgfpathlineto{\pgfqpoint{0.000000in}{3.988185in}}%
\pgfpathclose%
\pgfusepath{fill}%
\end{pgfscope}%
\begin{pgfscope}%
\pgfsetbuttcap%
\pgfsetmiterjoin%
\definecolor{currentfill}{rgb}{1.000000,1.000000,1.000000}%
\pgfsetfillcolor{currentfill}%
\pgfsetlinewidth{0.000000pt}%
\definecolor{currentstroke}{rgb}{0.000000,0.000000,0.000000}%
\pgfsetstrokecolor{currentstroke}%
\pgfsetstrokeopacity{0.000000}%
\pgfsetdash{}{0pt}%
\pgfpathmoveto{\pgfqpoint{0.833185in}{3.098185in}}%
\pgfpathlineto{\pgfqpoint{1.995685in}{3.098185in}}%
\pgfpathlineto{\pgfqpoint{1.995685in}{3.853185in}}%
\pgfpathlineto{\pgfqpoint{0.833185in}{3.853185in}}%
\pgfpathclose%
\pgfusepath{fill}%
\end{pgfscope}%
\begin{pgfscope}%
\pgfsetbuttcap%
\pgfsetroundjoin%
\definecolor{currentfill}{rgb}{0.000000,0.000000,0.000000}%
\pgfsetfillcolor{currentfill}%
\pgfsetlinewidth{0.803000pt}%
\definecolor{currentstroke}{rgb}{0.000000,0.000000,0.000000}%
\pgfsetstrokecolor{currentstroke}%
\pgfsetdash{}{0pt}%
\pgfsys@defobject{currentmarker}{\pgfqpoint{-0.048611in}{0.000000in}}{\pgfqpoint{0.000000in}{0.000000in}}{%
\pgfpathmoveto{\pgfqpoint{0.000000in}{0.000000in}}%
\pgfpathlineto{\pgfqpoint{-0.048611in}{0.000000in}}%
\pgfusepath{stroke,fill}%
}%
\begin{pgfscope}%
\pgfsys@transformshift{0.833185in}{3.164777in}%
\pgfsys@useobject{currentmarker}{}%
\end{pgfscope}%
\end{pgfscope}%
\begin{pgfscope}%
\pgftext[x=0.559259in,y=3.122567in,left,base]{\rmfamily\fontsize{8.000000}{9.600000}\selectfont 0.5}%
\end{pgfscope}%
\begin{pgfscope}%
\pgfsetbuttcap%
\pgfsetroundjoin%
\definecolor{currentfill}{rgb}{0.000000,0.000000,0.000000}%
\pgfsetfillcolor{currentfill}%
\pgfsetlinewidth{0.803000pt}%
\definecolor{currentstroke}{rgb}{0.000000,0.000000,0.000000}%
\pgfsetstrokecolor{currentstroke}%
\pgfsetdash{}{0pt}%
\pgfsys@defobject{currentmarker}{\pgfqpoint{-0.048611in}{0.000000in}}{\pgfqpoint{0.000000in}{0.000000in}}{%
\pgfpathmoveto{\pgfqpoint{0.000000in}{0.000000in}}%
\pgfpathlineto{\pgfqpoint{-0.048611in}{0.000000in}}%
\pgfusepath{stroke,fill}%
}%
\begin{pgfscope}%
\pgfsys@transformshift{0.833185in}{3.440784in}%
\pgfsys@useobject{currentmarker}{}%
\end{pgfscope}%
\end{pgfscope}%
\begin{pgfscope}%
\pgftext[x=0.559259in,y=3.398574in,left,base]{\rmfamily\fontsize{8.000000}{9.600000}\selectfont 1.0}%
\end{pgfscope}%
\begin{pgfscope}%
\pgfsetbuttcap%
\pgfsetroundjoin%
\definecolor{currentfill}{rgb}{0.000000,0.000000,0.000000}%
\pgfsetfillcolor{currentfill}%
\pgfsetlinewidth{0.803000pt}%
\definecolor{currentstroke}{rgb}{0.000000,0.000000,0.000000}%
\pgfsetstrokecolor{currentstroke}%
\pgfsetdash{}{0pt}%
\pgfsys@defobject{currentmarker}{\pgfqpoint{-0.048611in}{0.000000in}}{\pgfqpoint{0.000000in}{0.000000in}}{%
\pgfpathmoveto{\pgfqpoint{0.000000in}{0.000000in}}%
\pgfpathlineto{\pgfqpoint{-0.048611in}{0.000000in}}%
\pgfusepath{stroke,fill}%
}%
\begin{pgfscope}%
\pgfsys@transformshift{0.833185in}{3.716791in}%
\pgfsys@useobject{currentmarker}{}%
\end{pgfscope}%
\end{pgfscope}%
\begin{pgfscope}%
\pgftext[x=0.559259in,y=3.674581in,left,base]{\rmfamily\fontsize{8.000000}{9.600000}\selectfont 1.5}%
\end{pgfscope}%
\begin{pgfscope}%
\pgftext[x=0.503703in,y=3.475685in,,bottom,rotate=90.000000]{\rmfamily\fontsize{10.000000}{12.000000}\selectfont Ef0}%
\end{pgfscope}%
\begin{pgfscope}%
\pgfpathrectangle{\pgfqpoint{0.833185in}{3.098185in}}{\pgfqpoint{1.162500in}{0.755000in}} %
\pgfusepath{clip}%
\pgfsetrectcap%
\pgfsetroundjoin%
\pgfsetlinewidth{1.505625pt}%
\definecolor{currentstroke}{rgb}{0.121569,0.466667,0.705882}%
\pgfsetstrokecolor{currentstroke}%
\pgfsetdash{}{0pt}%
\pgfpathmoveto{\pgfqpoint{0.860863in}{3.508850in}}%
\pgfpathlineto{\pgfqpoint{0.889678in}{3.584059in}}%
\pgfpathlineto{\pgfqpoint{0.914059in}{3.641668in}}%
\pgfpathlineto{\pgfqpoint{0.935116in}{3.685989in}}%
\pgfpathlineto{\pgfqpoint{0.953956in}{3.720805in}}%
\pgfpathlineto{\pgfqpoint{0.971688in}{3.749038in}}%
\pgfpathlineto{\pgfqpoint{0.988312in}{3.771288in}}%
\pgfpathlineto{\pgfqpoint{1.003828in}{3.788247in}}%
\pgfpathlineto{\pgfqpoint{1.018235in}{3.800646in}}%
\pgfpathlineto{\pgfqpoint{1.031534in}{3.809221in}}%
\pgfpathlineto{\pgfqpoint{1.044833in}{3.815062in}}%
\pgfpathlineto{\pgfqpoint{1.057024in}{3.818055in}}%
\pgfpathlineto{\pgfqpoint{1.069214in}{3.818850in}}%
\pgfpathlineto{\pgfqpoint{1.081405in}{3.817519in}}%
\pgfpathlineto{\pgfqpoint{1.094704in}{3.813751in}}%
\pgfpathlineto{\pgfqpoint{1.108003in}{3.807704in}}%
\pgfpathlineto{\pgfqpoint{1.122410in}{3.798763in}}%
\pgfpathlineto{\pgfqpoint{1.139034in}{3.785647in}}%
\pgfpathlineto{\pgfqpoint{1.156766in}{3.768754in}}%
\pgfpathlineto{\pgfqpoint{1.177823in}{3.745460in}}%
\pgfpathlineto{\pgfqpoint{1.203313in}{3.713710in}}%
\pgfpathlineto{\pgfqpoint{1.239885in}{3.664179in}}%
\pgfpathlineto{\pgfqpoint{1.305272in}{3.575335in}}%
\pgfpathlineto{\pgfqpoint{1.336303in}{3.537262in}}%
\pgfpathlineto{\pgfqpoint{1.362901in}{3.508055in}}%
\pgfpathlineto{\pgfqpoint{1.387282in}{3.484432in}}%
\pgfpathlineto{\pgfqpoint{1.410556in}{3.464808in}}%
\pgfpathlineto{\pgfqpoint{1.433829in}{3.448013in}}%
\pgfpathlineto{\pgfqpoint{1.457102in}{3.433906in}}%
\pgfpathlineto{\pgfqpoint{1.481484in}{3.421768in}}%
\pgfpathlineto{\pgfqpoint{1.506974in}{3.411611in}}%
\pgfpathlineto{\pgfqpoint{1.534680in}{3.403002in}}%
\pgfpathlineto{\pgfqpoint{1.566819in}{3.395392in}}%
\pgfpathlineto{\pgfqpoint{1.613366in}{3.386905in}}%
\pgfpathlineto{\pgfqpoint{1.674319in}{3.375490in}}%
\pgfpathlineto{\pgfqpoint{1.706459in}{3.367112in}}%
\pgfpathlineto{\pgfqpoint{1.733057in}{3.357976in}}%
\pgfpathlineto{\pgfqpoint{1.757438in}{3.347350in}}%
\pgfpathlineto{\pgfqpoint{1.780712in}{3.334841in}}%
\pgfpathlineto{\pgfqpoint{1.803985in}{3.319740in}}%
\pgfpathlineto{\pgfqpoint{1.826150in}{3.302753in}}%
\pgfpathlineto{\pgfqpoint{1.849423in}{3.282046in}}%
\pgfpathlineto{\pgfqpoint{1.872696in}{3.258348in}}%
\pgfpathlineto{\pgfqpoint{1.897078in}{3.230355in}}%
\pgfpathlineto{\pgfqpoint{1.923676in}{3.196315in}}%
\pgfpathlineto{\pgfqpoint{1.952490in}{3.155732in}}%
\pgfpathlineto{\pgfqpoint{1.968006in}{3.132503in}}%
\pgfpathlineto{\pgfqpoint{1.968006in}{3.132503in}}%
\pgfusepath{stroke}%
\end{pgfscope}%
\begin{pgfscope}%
\pgfsetrectcap%
\pgfsetmiterjoin%
\pgfsetlinewidth{0.803000pt}%
\definecolor{currentstroke}{rgb}{0.000000,0.000000,0.000000}%
\pgfsetstrokecolor{currentstroke}%
\pgfsetdash{}{0pt}%
\pgfpathmoveto{\pgfqpoint{0.833185in}{3.098185in}}%
\pgfpathlineto{\pgfqpoint{0.833185in}{3.853185in}}%
\pgfusepath{stroke}%
\end{pgfscope}%
\begin{pgfscope}%
\pgfsetrectcap%
\pgfsetmiterjoin%
\pgfsetlinewidth{0.803000pt}%
\definecolor{currentstroke}{rgb}{0.000000,0.000000,0.000000}%
\pgfsetstrokecolor{currentstroke}%
\pgfsetdash{}{0pt}%
\pgfpathmoveto{\pgfqpoint{1.995685in}{3.098185in}}%
\pgfpathlineto{\pgfqpoint{1.995685in}{3.853185in}}%
\pgfusepath{stroke}%
\end{pgfscope}%
\begin{pgfscope}%
\pgfsetrectcap%
\pgfsetmiterjoin%
\pgfsetlinewidth{0.803000pt}%
\definecolor{currentstroke}{rgb}{0.000000,0.000000,0.000000}%
\pgfsetstrokecolor{currentstroke}%
\pgfsetdash{}{0pt}%
\pgfpathmoveto{\pgfqpoint{0.833185in}{3.098185in}}%
\pgfpathlineto{\pgfqpoint{1.995685in}{3.098185in}}%
\pgfusepath{stroke}%
\end{pgfscope}%
\begin{pgfscope}%
\pgfsetrectcap%
\pgfsetmiterjoin%
\pgfsetlinewidth{0.803000pt}%
\definecolor{currentstroke}{rgb}{0.000000,0.000000,0.000000}%
\pgfsetstrokecolor{currentstroke}%
\pgfsetdash{}{0pt}%
\pgfpathmoveto{\pgfqpoint{0.833185in}{3.853185in}}%
\pgfpathlineto{\pgfqpoint{1.995685in}{3.853185in}}%
\pgfusepath{stroke}%
\end{pgfscope}%
\begin{pgfscope}%
\pgfsetbuttcap%
\pgfsetmiterjoin%
\definecolor{currentfill}{rgb}{1.000000,1.000000,1.000000}%
\pgfsetfillcolor{currentfill}%
\pgfsetlinewidth{0.000000pt}%
\definecolor{currentstroke}{rgb}{0.000000,0.000000,0.000000}%
\pgfsetstrokecolor{currentstroke}%
\pgfsetstrokeopacity{0.000000}%
\pgfsetdash{}{0pt}%
\pgfpathmoveto{\pgfqpoint{1.995685in}{3.098185in}}%
\pgfpathlineto{\pgfqpoint{3.158185in}{3.098185in}}%
\pgfpathlineto{\pgfqpoint{3.158185in}{3.853185in}}%
\pgfpathlineto{\pgfqpoint{1.995685in}{3.853185in}}%
\pgfpathclose%
\pgfusepath{fill}%
\end{pgfscope}%
\begin{pgfscope}%
\pgfpathrectangle{\pgfqpoint{1.995685in}{3.098185in}}{\pgfqpoint{1.162500in}{0.755000in}} %
\pgfusepath{clip}%
\pgfsetbuttcap%
\pgfsetroundjoin%
\definecolor{currentfill}{rgb}{0.000000,0.000000,0.000000}%
\pgfsetfillcolor{currentfill}%
\pgfsetfillopacity{0.500000}%
\pgfsetlinewidth{0.000000pt}%
\definecolor{currentstroke}{rgb}{0.000000,0.000000,0.000000}%
\pgfsetstrokecolor{currentstroke}%
\pgfsetdash{}{0pt}%
\pgfpathmoveto{\pgfqpoint{3.130506in}{3.691634in}}%
\pgfpathcurveto{\pgfqpoint{3.136031in}{3.691634in}}{\pgfqpoint{3.141331in}{3.693829in}}{\pgfqpoint{3.145237in}{3.697736in}}%
\pgfpathcurveto{\pgfqpoint{3.149144in}{3.701643in}}{\pgfqpoint{3.151339in}{3.706942in}}{\pgfqpoint{3.151339in}{3.712467in}}%
\pgfpathcurveto{\pgfqpoint{3.151339in}{3.717993in}}{\pgfqpoint{3.149144in}{3.723292in}}{\pgfqpoint{3.145237in}{3.727199in}}%
\pgfpathcurveto{\pgfqpoint{3.141331in}{3.731106in}}{\pgfqpoint{3.136031in}{3.733301in}}{\pgfqpoint{3.130506in}{3.733301in}}%
\pgfpathcurveto{\pgfqpoint{3.124981in}{3.733301in}}{\pgfqpoint{3.119681in}{3.731106in}}{\pgfqpoint{3.115775in}{3.727199in}}%
\pgfpathcurveto{\pgfqpoint{3.111868in}{3.723292in}}{\pgfqpoint{3.109673in}{3.717993in}}{\pgfqpoint{3.109673in}{3.712467in}}%
\pgfpathcurveto{\pgfqpoint{3.109673in}{3.706942in}}{\pgfqpoint{3.111868in}{3.701643in}}{\pgfqpoint{3.115775in}{3.697736in}}%
\pgfpathcurveto{\pgfqpoint{3.119681in}{3.693829in}}{\pgfqpoint{3.124981in}{3.691634in}}{\pgfqpoint{3.130506in}{3.691634in}}%
\pgfpathclose%
\pgfusepath{fill}%
\end{pgfscope}%
\begin{pgfscope}%
\pgfpathrectangle{\pgfqpoint{1.995685in}{3.098185in}}{\pgfqpoint{1.162500in}{0.755000in}} %
\pgfusepath{clip}%
\pgfsetbuttcap%
\pgfsetroundjoin%
\definecolor{currentfill}{rgb}{0.000000,0.000000,0.000000}%
\pgfsetfillcolor{currentfill}%
\pgfsetfillopacity{0.500000}%
\pgfsetlinewidth{0.000000pt}%
\definecolor{currentstroke}{rgb}{0.000000,0.000000,0.000000}%
\pgfsetstrokecolor{currentstroke}%
\pgfsetdash{}{0pt}%
\pgfpathmoveto{\pgfqpoint{2.755547in}{3.134292in}}%
\pgfpathcurveto{\pgfqpoint{2.761072in}{3.134292in}}{\pgfqpoint{2.766372in}{3.136487in}}{\pgfqpoint{2.770279in}{3.140394in}}%
\pgfpathcurveto{\pgfqpoint{2.774186in}{3.144301in}}{\pgfqpoint{2.776381in}{3.149600in}}{\pgfqpoint{2.776381in}{3.155125in}}%
\pgfpathcurveto{\pgfqpoint{2.776381in}{3.160650in}}{\pgfqpoint{2.774186in}{3.165950in}}{\pgfqpoint{2.770279in}{3.169857in}}%
\pgfpathcurveto{\pgfqpoint{2.766372in}{3.173763in}}{\pgfqpoint{2.761072in}{3.175959in}}{\pgfqpoint{2.755547in}{3.175959in}}%
\pgfpathcurveto{\pgfqpoint{2.750022in}{3.175959in}}{\pgfqpoint{2.744723in}{3.173763in}}{\pgfqpoint{2.740816in}{3.169857in}}%
\pgfpathcurveto{\pgfqpoint{2.736909in}{3.165950in}}{\pgfqpoint{2.734714in}{3.160650in}}{\pgfqpoint{2.734714in}{3.155125in}}%
\pgfpathcurveto{\pgfqpoint{2.734714in}{3.149600in}}{\pgfqpoint{2.736909in}{3.144301in}}{\pgfqpoint{2.740816in}{3.140394in}}%
\pgfpathcurveto{\pgfqpoint{2.744723in}{3.136487in}}{\pgfqpoint{2.750022in}{3.134292in}}{\pgfqpoint{2.755547in}{3.134292in}}%
\pgfpathclose%
\pgfusepath{fill}%
\end{pgfscope}%
\begin{pgfscope}%
\pgfpathrectangle{\pgfqpoint{1.995685in}{3.098185in}}{\pgfqpoint{1.162500in}{0.755000in}} %
\pgfusepath{clip}%
\pgfsetbuttcap%
\pgfsetroundjoin%
\definecolor{currentfill}{rgb}{0.000000,0.000000,0.000000}%
\pgfsetfillcolor{currentfill}%
\pgfsetfillopacity{0.500000}%
\pgfsetlinewidth{0.000000pt}%
\definecolor{currentstroke}{rgb}{0.000000,0.000000,0.000000}%
\pgfsetstrokecolor{currentstroke}%
\pgfsetdash{}{0pt}%
\pgfpathmoveto{\pgfqpoint{2.755127in}{3.095327in}}%
\pgfpathcurveto{\pgfqpoint{2.760652in}{3.095327in}}{\pgfqpoint{2.765952in}{3.097523in}}{\pgfqpoint{2.769858in}{3.101429in}}%
\pgfpathcurveto{\pgfqpoint{2.773765in}{3.105336in}}{\pgfqpoint{2.775960in}{3.110636in}}{\pgfqpoint{2.775960in}{3.116161in}}%
\pgfpathcurveto{\pgfqpoint{2.775960in}{3.121686in}}{\pgfqpoint{2.773765in}{3.126985in}}{\pgfqpoint{2.769858in}{3.130892in}}%
\pgfpathcurveto{\pgfqpoint{2.765952in}{3.134799in}}{\pgfqpoint{2.760652in}{3.136994in}}{\pgfqpoint{2.755127in}{3.136994in}}%
\pgfpathcurveto{\pgfqpoint{2.749602in}{3.136994in}}{\pgfqpoint{2.744302in}{3.134799in}}{\pgfqpoint{2.740396in}{3.130892in}}%
\pgfpathcurveto{\pgfqpoint{2.736489in}{3.126985in}}{\pgfqpoint{2.734294in}{3.121686in}}{\pgfqpoint{2.734294in}{3.116161in}}%
\pgfpathcurveto{\pgfqpoint{2.734294in}{3.110636in}}{\pgfqpoint{2.736489in}{3.105336in}}{\pgfqpoint{2.740396in}{3.101429in}}%
\pgfpathcurveto{\pgfqpoint{2.744302in}{3.097523in}}{\pgfqpoint{2.749602in}{3.095327in}}{\pgfqpoint{2.755127in}{3.095327in}}%
\pgfpathclose%
\pgfusepath{fill}%
\end{pgfscope}%
\begin{pgfscope}%
\pgfpathrectangle{\pgfqpoint{1.995685in}{3.098185in}}{\pgfqpoint{1.162500in}{0.755000in}} %
\pgfusepath{clip}%
\pgfsetbuttcap%
\pgfsetroundjoin%
\definecolor{currentfill}{rgb}{0.000000,0.000000,0.000000}%
\pgfsetfillcolor{currentfill}%
\pgfsetfillopacity{0.500000}%
\pgfsetlinewidth{0.000000pt}%
\definecolor{currentstroke}{rgb}{0.000000,0.000000,0.000000}%
\pgfsetstrokecolor{currentstroke}%
\pgfsetdash{}{0pt}%
\pgfpathmoveto{\pgfqpoint{2.317114in}{3.302828in}}%
\pgfpathcurveto{\pgfqpoint{2.322639in}{3.302828in}}{\pgfqpoint{2.327939in}{3.305023in}}{\pgfqpoint{2.331845in}{3.308930in}}%
\pgfpathcurveto{\pgfqpoint{2.335752in}{3.312837in}}{\pgfqpoint{2.337947in}{3.318136in}}{\pgfqpoint{2.337947in}{3.323661in}}%
\pgfpathcurveto{\pgfqpoint{2.337947in}{3.329186in}}{\pgfqpoint{2.335752in}{3.334486in}}{\pgfqpoint{2.331845in}{3.338393in}}%
\pgfpathcurveto{\pgfqpoint{2.327939in}{3.342300in}}{\pgfqpoint{2.322639in}{3.344495in}}{\pgfqpoint{2.317114in}{3.344495in}}%
\pgfpathcurveto{\pgfqpoint{2.311589in}{3.344495in}}{\pgfqpoint{2.306289in}{3.342300in}}{\pgfqpoint{2.302383in}{3.338393in}}%
\pgfpathcurveto{\pgfqpoint{2.298476in}{3.334486in}}{\pgfqpoint{2.296281in}{3.329186in}}{\pgfqpoint{2.296281in}{3.323661in}}%
\pgfpathcurveto{\pgfqpoint{2.296281in}{3.318136in}}{\pgfqpoint{2.298476in}{3.312837in}}{\pgfqpoint{2.302383in}{3.308930in}}%
\pgfpathcurveto{\pgfqpoint{2.306289in}{3.305023in}}{\pgfqpoint{2.311589in}{3.302828in}}{\pgfqpoint{2.317114in}{3.302828in}}%
\pgfpathclose%
\pgfusepath{fill}%
\end{pgfscope}%
\begin{pgfscope}%
\pgfpathrectangle{\pgfqpoint{1.995685in}{3.098185in}}{\pgfqpoint{1.162500in}{0.755000in}} %
\pgfusepath{clip}%
\pgfsetbuttcap%
\pgfsetroundjoin%
\definecolor{currentfill}{rgb}{0.000000,0.000000,0.000000}%
\pgfsetfillcolor{currentfill}%
\pgfsetfillopacity{0.500000}%
\pgfsetlinewidth{0.000000pt}%
\definecolor{currentstroke}{rgb}{0.000000,0.000000,0.000000}%
\pgfsetstrokecolor{currentstroke}%
\pgfsetdash{}{0pt}%
\pgfpathmoveto{\pgfqpoint{2.241285in}{3.145618in}}%
\pgfpathcurveto{\pgfqpoint{2.246810in}{3.145618in}}{\pgfqpoint{2.252110in}{3.147813in}}{\pgfqpoint{2.256017in}{3.151720in}}%
\pgfpathcurveto{\pgfqpoint{2.259923in}{3.155626in}}{\pgfqpoint{2.262118in}{3.160926in}}{\pgfqpoint{2.262118in}{3.166451in}}%
\pgfpathcurveto{\pgfqpoint{2.262118in}{3.171976in}}{\pgfqpoint{2.259923in}{3.177276in}}{\pgfqpoint{2.256017in}{3.181182in}}%
\pgfpathcurveto{\pgfqpoint{2.252110in}{3.185089in}}{\pgfqpoint{2.246810in}{3.187284in}}{\pgfqpoint{2.241285in}{3.187284in}}%
\pgfpathcurveto{\pgfqpoint{2.235760in}{3.187284in}}{\pgfqpoint{2.230461in}{3.185089in}}{\pgfqpoint{2.226554in}{3.181182in}}%
\pgfpathcurveto{\pgfqpoint{2.222647in}{3.177276in}}{\pgfqpoint{2.220452in}{3.171976in}}{\pgfqpoint{2.220452in}{3.166451in}}%
\pgfpathcurveto{\pgfqpoint{2.220452in}{3.160926in}}{\pgfqpoint{2.222647in}{3.155626in}}{\pgfqpoint{2.226554in}{3.151720in}}%
\pgfpathcurveto{\pgfqpoint{2.230461in}{3.147813in}}{\pgfqpoint{2.235760in}{3.145618in}}{\pgfqpoint{2.241285in}{3.145618in}}%
\pgfpathclose%
\pgfusepath{fill}%
\end{pgfscope}%
\begin{pgfscope}%
\pgfpathrectangle{\pgfqpoint{1.995685in}{3.098185in}}{\pgfqpoint{1.162500in}{0.755000in}} %
\pgfusepath{clip}%
\pgfsetbuttcap%
\pgfsetroundjoin%
\definecolor{currentfill}{rgb}{0.000000,0.000000,0.000000}%
\pgfsetfillcolor{currentfill}%
\pgfsetfillopacity{0.500000}%
\pgfsetlinewidth{0.000000pt}%
\definecolor{currentstroke}{rgb}{0.000000,0.000000,0.000000}%
\pgfsetstrokecolor{currentstroke}%
\pgfsetdash{}{0pt}%
\pgfpathmoveto{\pgfqpoint{2.106066in}{3.249924in}}%
\pgfpathcurveto{\pgfqpoint{2.111591in}{3.249924in}}{\pgfqpoint{2.116890in}{3.252119in}}{\pgfqpoint{2.120797in}{3.256026in}}%
\pgfpathcurveto{\pgfqpoint{2.124704in}{3.259932in}}{\pgfqpoint{2.126899in}{3.265232in}}{\pgfqpoint{2.126899in}{3.270757in}}%
\pgfpathcurveto{\pgfqpoint{2.126899in}{3.276282in}}{\pgfqpoint{2.124704in}{3.281582in}}{\pgfqpoint{2.120797in}{3.285488in}}%
\pgfpathcurveto{\pgfqpoint{2.116890in}{3.289395in}}{\pgfqpoint{2.111591in}{3.291590in}}{\pgfqpoint{2.106066in}{3.291590in}}%
\pgfpathcurveto{\pgfqpoint{2.100541in}{3.291590in}}{\pgfqpoint{2.095241in}{3.289395in}}{\pgfqpoint{2.091334in}{3.285488in}}%
\pgfpathcurveto{\pgfqpoint{2.087428in}{3.281582in}}{\pgfqpoint{2.085232in}{3.276282in}}{\pgfqpoint{2.085232in}{3.270757in}}%
\pgfpathcurveto{\pgfqpoint{2.085232in}{3.265232in}}{\pgfqpoint{2.087428in}{3.259932in}}{\pgfqpoint{2.091334in}{3.256026in}}%
\pgfpathcurveto{\pgfqpoint{2.095241in}{3.252119in}}{\pgfqpoint{2.100541in}{3.249924in}}{\pgfqpoint{2.106066in}{3.249924in}}%
\pgfpathclose%
\pgfusepath{fill}%
\end{pgfscope}%
\begin{pgfscope}%
\pgfpathrectangle{\pgfqpoint{1.995685in}{3.098185in}}{\pgfqpoint{1.162500in}{0.755000in}} %
\pgfusepath{clip}%
\pgfsetbuttcap%
\pgfsetroundjoin%
\definecolor{currentfill}{rgb}{0.000000,0.000000,0.000000}%
\pgfsetfillcolor{currentfill}%
\pgfsetfillopacity{0.500000}%
\pgfsetlinewidth{0.000000pt}%
\definecolor{currentstroke}{rgb}{0.000000,0.000000,0.000000}%
\pgfsetstrokecolor{currentstroke}%
\pgfsetdash{}{0pt}%
\pgfpathmoveto{\pgfqpoint{2.023363in}{3.240978in}}%
\pgfpathcurveto{\pgfqpoint{2.028888in}{3.240978in}}{\pgfqpoint{2.034188in}{3.243173in}}{\pgfqpoint{2.038095in}{3.247080in}}%
\pgfpathcurveto{\pgfqpoint{2.042001in}{3.250987in}}{\pgfqpoint{2.044196in}{3.256286in}}{\pgfqpoint{2.044196in}{3.261812in}}%
\pgfpathcurveto{\pgfqpoint{2.044196in}{3.267337in}}{\pgfqpoint{2.042001in}{3.272636in}}{\pgfqpoint{2.038095in}{3.276543in}}%
\pgfpathcurveto{\pgfqpoint{2.034188in}{3.280450in}}{\pgfqpoint{2.028888in}{3.282645in}}{\pgfqpoint{2.023363in}{3.282645in}}%
\pgfpathcurveto{\pgfqpoint{2.017838in}{3.282645in}}{\pgfqpoint{2.012539in}{3.280450in}}{\pgfqpoint{2.008632in}{3.276543in}}%
\pgfpathcurveto{\pgfqpoint{2.004725in}{3.272636in}}{\pgfqpoint{2.002530in}{3.267337in}}{\pgfqpoint{2.002530in}{3.261812in}}%
\pgfpathcurveto{\pgfqpoint{2.002530in}{3.256286in}}{\pgfqpoint{2.004725in}{3.250987in}}{\pgfqpoint{2.008632in}{3.247080in}}%
\pgfpathcurveto{\pgfqpoint{2.012539in}{3.243173in}}{\pgfqpoint{2.017838in}{3.240978in}}{\pgfqpoint{2.023363in}{3.240978in}}%
\pgfpathclose%
\pgfusepath{fill}%
\end{pgfscope}%
\begin{pgfscope}%
\pgfpathrectangle{\pgfqpoint{1.995685in}{3.098185in}}{\pgfqpoint{1.162500in}{0.755000in}} %
\pgfusepath{clip}%
\pgfsetbuttcap%
\pgfsetroundjoin%
\definecolor{currentfill}{rgb}{0.000000,0.000000,0.000000}%
\pgfsetfillcolor{currentfill}%
\pgfsetfillopacity{0.500000}%
\pgfsetlinewidth{0.000000pt}%
\definecolor{currentstroke}{rgb}{0.000000,0.000000,0.000000}%
\pgfsetstrokecolor{currentstroke}%
\pgfsetdash{}{0pt}%
\pgfpathmoveto{\pgfqpoint{2.907651in}{3.768072in}}%
\pgfpathcurveto{\pgfqpoint{2.913176in}{3.768072in}}{\pgfqpoint{2.918476in}{3.770267in}}{\pgfqpoint{2.922382in}{3.774174in}}%
\pgfpathcurveto{\pgfqpoint{2.926289in}{3.778081in}}{\pgfqpoint{2.928484in}{3.783381in}}{\pgfqpoint{2.928484in}{3.788906in}}%
\pgfpathcurveto{\pgfqpoint{2.928484in}{3.794431in}}{\pgfqpoint{2.926289in}{3.799730in}}{\pgfqpoint{2.922382in}{3.803637in}}%
\pgfpathcurveto{\pgfqpoint{2.918476in}{3.807544in}}{\pgfqpoint{2.913176in}{3.809739in}}{\pgfqpoint{2.907651in}{3.809739in}}%
\pgfpathcurveto{\pgfqpoint{2.902126in}{3.809739in}}{\pgfqpoint{2.896827in}{3.807544in}}{\pgfqpoint{2.892920in}{3.803637in}}%
\pgfpathcurveto{\pgfqpoint{2.889013in}{3.799730in}}{\pgfqpoint{2.886818in}{3.794431in}}{\pgfqpoint{2.886818in}{3.788906in}}%
\pgfpathcurveto{\pgfqpoint{2.886818in}{3.783381in}}{\pgfqpoint{2.889013in}{3.778081in}}{\pgfqpoint{2.892920in}{3.774174in}}%
\pgfpathcurveto{\pgfqpoint{2.896827in}{3.770267in}}{\pgfqpoint{2.902126in}{3.768072in}}{\pgfqpoint{2.907651in}{3.768072in}}%
\pgfpathclose%
\pgfusepath{fill}%
\end{pgfscope}%
\begin{pgfscope}%
\pgfpathrectangle{\pgfqpoint{1.995685in}{3.098185in}}{\pgfqpoint{1.162500in}{0.755000in}} %
\pgfusepath{clip}%
\pgfsetbuttcap%
\pgfsetroundjoin%
\definecolor{currentfill}{rgb}{0.000000,0.000000,0.000000}%
\pgfsetfillcolor{currentfill}%
\pgfsetfillopacity{0.500000}%
\pgfsetlinewidth{0.000000pt}%
\definecolor{currentstroke}{rgb}{0.000000,0.000000,0.000000}%
\pgfsetstrokecolor{currentstroke}%
\pgfsetdash{}{0pt}%
\pgfpathmoveto{\pgfqpoint{2.630361in}{3.814375in}}%
\pgfpathcurveto{\pgfqpoint{2.635886in}{3.814375in}}{\pgfqpoint{2.641186in}{3.816570in}}{\pgfqpoint{2.645093in}{3.820477in}}%
\pgfpathcurveto{\pgfqpoint{2.649000in}{3.824384in}}{\pgfqpoint{2.651195in}{3.829683in}}{\pgfqpoint{2.651195in}{3.835208in}}%
\pgfpathcurveto{\pgfqpoint{2.651195in}{3.840733in}}{\pgfqpoint{2.649000in}{3.846033in}}{\pgfqpoint{2.645093in}{3.849940in}}%
\pgfpathcurveto{\pgfqpoint{2.641186in}{3.853847in}}{\pgfqpoint{2.635886in}{3.856042in}}{\pgfqpoint{2.630361in}{3.856042in}}%
\pgfpathcurveto{\pgfqpoint{2.624836in}{3.856042in}}{\pgfqpoint{2.619537in}{3.853847in}}{\pgfqpoint{2.615630in}{3.849940in}}%
\pgfpathcurveto{\pgfqpoint{2.611723in}{3.846033in}}{\pgfqpoint{2.609528in}{3.840733in}}{\pgfqpoint{2.609528in}{3.835208in}}%
\pgfpathcurveto{\pgfqpoint{2.609528in}{3.829683in}}{\pgfqpoint{2.611723in}{3.824384in}}{\pgfqpoint{2.615630in}{3.820477in}}%
\pgfpathcurveto{\pgfqpoint{2.619537in}{3.816570in}}{\pgfqpoint{2.624836in}{3.814375in}}{\pgfqpoint{2.630361in}{3.814375in}}%
\pgfpathclose%
\pgfusepath{fill}%
\end{pgfscope}%
\begin{pgfscope}%
\pgfpathrectangle{\pgfqpoint{1.995685in}{3.098185in}}{\pgfqpoint{1.162500in}{0.755000in}} %
\pgfusepath{clip}%
\pgfsetbuttcap%
\pgfsetroundjoin%
\definecolor{currentfill}{rgb}{0.000000,0.000000,0.000000}%
\pgfsetfillcolor{currentfill}%
\pgfsetfillopacity{0.500000}%
\pgfsetlinewidth{0.000000pt}%
\definecolor{currentstroke}{rgb}{0.000000,0.000000,0.000000}%
\pgfsetstrokecolor{currentstroke}%
\pgfsetdash{}{0pt}%
\pgfpathmoveto{\pgfqpoint{2.487752in}{3.667724in}}%
\pgfpathcurveto{\pgfqpoint{2.493277in}{3.667724in}}{\pgfqpoint{2.498576in}{3.669919in}}{\pgfqpoint{2.502483in}{3.673826in}}%
\pgfpathcurveto{\pgfqpoint{2.506390in}{3.677733in}}{\pgfqpoint{2.508585in}{3.683032in}}{\pgfqpoint{2.508585in}{3.688558in}}%
\pgfpathcurveto{\pgfqpoint{2.508585in}{3.694083in}}{\pgfqpoint{2.506390in}{3.699382in}}{\pgfqpoint{2.502483in}{3.703289in}}%
\pgfpathcurveto{\pgfqpoint{2.498576in}{3.707196in}}{\pgfqpoint{2.493277in}{3.709391in}}{\pgfqpoint{2.487752in}{3.709391in}}%
\pgfpathcurveto{\pgfqpoint{2.482227in}{3.709391in}}{\pgfqpoint{2.476927in}{3.707196in}}{\pgfqpoint{2.473020in}{3.703289in}}%
\pgfpathcurveto{\pgfqpoint{2.469113in}{3.699382in}}{\pgfqpoint{2.466918in}{3.694083in}}{\pgfqpoint{2.466918in}{3.688558in}}%
\pgfpathcurveto{\pgfqpoint{2.466918in}{3.683032in}}{\pgfqpoint{2.469113in}{3.677733in}}{\pgfqpoint{2.473020in}{3.673826in}}%
\pgfpathcurveto{\pgfqpoint{2.476927in}{3.669919in}}{\pgfqpoint{2.482227in}{3.667724in}}{\pgfqpoint{2.487752in}{3.667724in}}%
\pgfpathclose%
\pgfusepath{fill}%
\end{pgfscope}%
\begin{pgfscope}%
\pgfpathrectangle{\pgfqpoint{1.995685in}{3.098185in}}{\pgfqpoint{1.162500in}{0.755000in}} %
\pgfusepath{clip}%
\pgfsetbuttcap%
\pgfsetroundjoin%
\definecolor{currentfill}{rgb}{0.000000,0.000000,0.000000}%
\pgfsetfillcolor{currentfill}%
\pgfsetfillopacity{0.500000}%
\pgfsetlinewidth{0.000000pt}%
\definecolor{currentstroke}{rgb}{0.000000,0.000000,0.000000}%
\pgfsetstrokecolor{currentstroke}%
\pgfsetdash{}{0pt}%
\pgfpathmoveto{\pgfqpoint{3.000738in}{3.223705in}}%
\pgfpathcurveto{\pgfqpoint{3.006263in}{3.223705in}}{\pgfqpoint{3.011563in}{3.225900in}}{\pgfqpoint{3.015470in}{3.229807in}}%
\pgfpathcurveto{\pgfqpoint{3.019376in}{3.233713in}}{\pgfqpoint{3.021572in}{3.239013in}}{\pgfqpoint{3.021572in}{3.244538in}}%
\pgfpathcurveto{\pgfqpoint{3.021572in}{3.250063in}}{\pgfqpoint{3.019376in}{3.255363in}}{\pgfqpoint{3.015470in}{3.259269in}}%
\pgfpathcurveto{\pgfqpoint{3.011563in}{3.263176in}}{\pgfqpoint{3.006263in}{3.265371in}}{\pgfqpoint{3.000738in}{3.265371in}}%
\pgfpathcurveto{\pgfqpoint{2.995213in}{3.265371in}}{\pgfqpoint{2.989914in}{3.263176in}}{\pgfqpoint{2.986007in}{3.259269in}}%
\pgfpathcurveto{\pgfqpoint{2.982100in}{3.255363in}}{\pgfqpoint{2.979905in}{3.250063in}}{\pgfqpoint{2.979905in}{3.244538in}}%
\pgfpathcurveto{\pgfqpoint{2.979905in}{3.239013in}}{\pgfqpoint{2.982100in}{3.233713in}}{\pgfqpoint{2.986007in}{3.229807in}}%
\pgfpathcurveto{\pgfqpoint{2.989914in}{3.225900in}}{\pgfqpoint{2.995213in}{3.223705in}}{\pgfqpoint{3.000738in}{3.223705in}}%
\pgfpathclose%
\pgfusepath{fill}%
\end{pgfscope}%
\begin{pgfscope}%
\pgfpathrectangle{\pgfqpoint{1.995685in}{3.098185in}}{\pgfqpoint{1.162500in}{0.755000in}} %
\pgfusepath{clip}%
\pgfsetbuttcap%
\pgfsetroundjoin%
\definecolor{currentfill}{rgb}{0.000000,0.000000,0.000000}%
\pgfsetfillcolor{currentfill}%
\pgfsetfillopacity{0.500000}%
\pgfsetlinewidth{0.000000pt}%
\definecolor{currentstroke}{rgb}{0.000000,0.000000,0.000000}%
\pgfsetstrokecolor{currentstroke}%
\pgfsetdash{}{0pt}%
\pgfpathmoveto{\pgfqpoint{2.318352in}{3.524055in}}%
\pgfpathcurveto{\pgfqpoint{2.323877in}{3.524055in}}{\pgfqpoint{2.329176in}{3.526250in}}{\pgfqpoint{2.333083in}{3.530157in}}%
\pgfpathcurveto{\pgfqpoint{2.336990in}{3.534064in}}{\pgfqpoint{2.339185in}{3.539363in}}{\pgfqpoint{2.339185in}{3.544888in}}%
\pgfpathcurveto{\pgfqpoint{2.339185in}{3.550413in}}{\pgfqpoint{2.336990in}{3.555713in}}{\pgfqpoint{2.333083in}{3.559620in}}%
\pgfpathcurveto{\pgfqpoint{2.329176in}{3.563526in}}{\pgfqpoint{2.323877in}{3.565721in}}{\pgfqpoint{2.318352in}{3.565721in}}%
\pgfpathcurveto{\pgfqpoint{2.312827in}{3.565721in}}{\pgfqpoint{2.307527in}{3.563526in}}{\pgfqpoint{2.303620in}{3.559620in}}%
\pgfpathcurveto{\pgfqpoint{2.299713in}{3.555713in}}{\pgfqpoint{2.297518in}{3.550413in}}{\pgfqpoint{2.297518in}{3.544888in}}%
\pgfpathcurveto{\pgfqpoint{2.297518in}{3.539363in}}{\pgfqpoint{2.299713in}{3.534064in}}{\pgfqpoint{2.303620in}{3.530157in}}%
\pgfpathcurveto{\pgfqpoint{2.307527in}{3.526250in}}{\pgfqpoint{2.312827in}{3.524055in}}{\pgfqpoint{2.318352in}{3.524055in}}%
\pgfpathclose%
\pgfusepath{fill}%
\end{pgfscope}%
\begin{pgfscope}%
\pgfpathrectangle{\pgfqpoint{1.995685in}{3.098185in}}{\pgfqpoint{1.162500in}{0.755000in}} %
\pgfusepath{clip}%
\pgfsetbuttcap%
\pgfsetroundjoin%
\definecolor{currentfill}{rgb}{0.000000,0.000000,0.000000}%
\pgfsetfillcolor{currentfill}%
\pgfsetfillopacity{0.500000}%
\pgfsetlinewidth{0.000000pt}%
\definecolor{currentstroke}{rgb}{0.000000,0.000000,0.000000}%
\pgfsetstrokecolor{currentstroke}%
\pgfsetdash{}{0pt}%
\pgfpathmoveto{\pgfqpoint{2.410257in}{3.446907in}}%
\pgfpathcurveto{\pgfqpoint{2.415782in}{3.446907in}}{\pgfqpoint{2.421082in}{3.449102in}}{\pgfqpoint{2.424989in}{3.453009in}}%
\pgfpathcurveto{\pgfqpoint{2.428896in}{3.456916in}}{\pgfqpoint{2.431091in}{3.462215in}}{\pgfqpoint{2.431091in}{3.467740in}}%
\pgfpathcurveto{\pgfqpoint{2.431091in}{3.473265in}}{\pgfqpoint{2.428896in}{3.478565in}}{\pgfqpoint{2.424989in}{3.482472in}}%
\pgfpathcurveto{\pgfqpoint{2.421082in}{3.486379in}}{\pgfqpoint{2.415782in}{3.488574in}}{\pgfqpoint{2.410257in}{3.488574in}}%
\pgfpathcurveto{\pgfqpoint{2.404732in}{3.488574in}}{\pgfqpoint{2.399433in}{3.486379in}}{\pgfqpoint{2.395526in}{3.482472in}}%
\pgfpathcurveto{\pgfqpoint{2.391619in}{3.478565in}}{\pgfqpoint{2.389424in}{3.473265in}}{\pgfqpoint{2.389424in}{3.467740in}}%
\pgfpathcurveto{\pgfqpoint{2.389424in}{3.462215in}}{\pgfqpoint{2.391619in}{3.456916in}}{\pgfqpoint{2.395526in}{3.453009in}}%
\pgfpathcurveto{\pgfqpoint{2.399433in}{3.449102in}}{\pgfqpoint{2.404732in}{3.446907in}}{\pgfqpoint{2.410257in}{3.446907in}}%
\pgfpathclose%
\pgfusepath{fill}%
\end{pgfscope}%
\begin{pgfscope}%
\pgfpathrectangle{\pgfqpoint{1.995685in}{3.098185in}}{\pgfqpoint{1.162500in}{0.755000in}} %
\pgfusepath{clip}%
\pgfsetbuttcap%
\pgfsetroundjoin%
\definecolor{currentfill}{rgb}{0.000000,0.000000,0.000000}%
\pgfsetfillcolor{currentfill}%
\pgfsetfillopacity{0.500000}%
\pgfsetlinewidth{0.000000pt}%
\definecolor{currentstroke}{rgb}{0.000000,0.000000,0.000000}%
\pgfsetstrokecolor{currentstroke}%
\pgfsetdash{}{0pt}%
\pgfpathmoveto{\pgfqpoint{2.752801in}{3.220444in}}%
\pgfpathcurveto{\pgfqpoint{2.758327in}{3.220444in}}{\pgfqpoint{2.763626in}{3.222640in}}{\pgfqpoint{2.767533in}{3.226546in}}%
\pgfpathcurveto{\pgfqpoint{2.771440in}{3.230453in}}{\pgfqpoint{2.773635in}{3.235753in}}{\pgfqpoint{2.773635in}{3.241278in}}%
\pgfpathcurveto{\pgfqpoint{2.773635in}{3.246803in}}{\pgfqpoint{2.771440in}{3.252102in}}{\pgfqpoint{2.767533in}{3.256009in}}%
\pgfpathcurveto{\pgfqpoint{2.763626in}{3.259916in}}{\pgfqpoint{2.758327in}{3.262111in}}{\pgfqpoint{2.752801in}{3.262111in}}%
\pgfpathcurveto{\pgfqpoint{2.747276in}{3.262111in}}{\pgfqpoint{2.741977in}{3.259916in}}{\pgfqpoint{2.738070in}{3.256009in}}%
\pgfpathcurveto{\pgfqpoint{2.734163in}{3.252102in}}{\pgfqpoint{2.731968in}{3.246803in}}{\pgfqpoint{2.731968in}{3.241278in}}%
\pgfpathcurveto{\pgfqpoint{2.731968in}{3.235753in}}{\pgfqpoint{2.734163in}{3.230453in}}{\pgfqpoint{2.738070in}{3.226546in}}%
\pgfpathcurveto{\pgfqpoint{2.741977in}{3.222640in}}{\pgfqpoint{2.747276in}{3.220444in}}{\pgfqpoint{2.752801in}{3.220444in}}%
\pgfpathclose%
\pgfusepath{fill}%
\end{pgfscope}%
\begin{pgfscope}%
\pgfpathrectangle{\pgfqpoint{1.995685in}{3.098185in}}{\pgfqpoint{1.162500in}{0.755000in}} %
\pgfusepath{clip}%
\pgfsetbuttcap%
\pgfsetroundjoin%
\definecolor{currentfill}{rgb}{0.000000,0.000000,0.000000}%
\pgfsetfillcolor{currentfill}%
\pgfsetfillopacity{0.500000}%
\pgfsetlinewidth{0.000000pt}%
\definecolor{currentstroke}{rgb}{0.000000,0.000000,0.000000}%
\pgfsetstrokecolor{currentstroke}%
\pgfsetdash{}{0pt}%
\pgfpathmoveto{\pgfqpoint{2.103337in}{3.505459in}}%
\pgfpathcurveto{\pgfqpoint{2.108862in}{3.505459in}}{\pgfqpoint{2.114161in}{3.507654in}}{\pgfqpoint{2.118068in}{3.511561in}}%
\pgfpathcurveto{\pgfqpoint{2.121975in}{3.515468in}}{\pgfqpoint{2.124170in}{3.520768in}}{\pgfqpoint{2.124170in}{3.526293in}}%
\pgfpathcurveto{\pgfqpoint{2.124170in}{3.531818in}}{\pgfqpoint{2.121975in}{3.537117in}}{\pgfqpoint{2.118068in}{3.541024in}}%
\pgfpathcurveto{\pgfqpoint{2.114161in}{3.544931in}}{\pgfqpoint{2.108862in}{3.547126in}}{\pgfqpoint{2.103337in}{3.547126in}}%
\pgfpathcurveto{\pgfqpoint{2.097812in}{3.547126in}}{\pgfqpoint{2.092512in}{3.544931in}}{\pgfqpoint{2.088605in}{3.541024in}}%
\pgfpathcurveto{\pgfqpoint{2.084698in}{3.537117in}}{\pgfqpoint{2.082503in}{3.531818in}}{\pgfqpoint{2.082503in}{3.526293in}}%
\pgfpathcurveto{\pgfqpoint{2.082503in}{3.520768in}}{\pgfqpoint{2.084698in}{3.515468in}}{\pgfqpoint{2.088605in}{3.511561in}}%
\pgfpathcurveto{\pgfqpoint{2.092512in}{3.507654in}}{\pgfqpoint{2.097812in}{3.505459in}}{\pgfqpoint{2.103337in}{3.505459in}}%
\pgfpathclose%
\pgfusepath{fill}%
\end{pgfscope}%
\begin{pgfscope}%
\pgfsetrectcap%
\pgfsetmiterjoin%
\pgfsetlinewidth{0.803000pt}%
\definecolor{currentstroke}{rgb}{0.000000,0.000000,0.000000}%
\pgfsetstrokecolor{currentstroke}%
\pgfsetdash{}{0pt}%
\pgfpathmoveto{\pgfqpoint{1.995685in}{3.098185in}}%
\pgfpathlineto{\pgfqpoint{1.995685in}{3.853185in}}%
\pgfusepath{stroke}%
\end{pgfscope}%
\begin{pgfscope}%
\pgfsetrectcap%
\pgfsetmiterjoin%
\pgfsetlinewidth{0.803000pt}%
\definecolor{currentstroke}{rgb}{0.000000,0.000000,0.000000}%
\pgfsetstrokecolor{currentstroke}%
\pgfsetdash{}{0pt}%
\pgfpathmoveto{\pgfqpoint{3.158185in}{3.098185in}}%
\pgfpathlineto{\pgfqpoint{3.158185in}{3.853185in}}%
\pgfusepath{stroke}%
\end{pgfscope}%
\begin{pgfscope}%
\pgfsetrectcap%
\pgfsetmiterjoin%
\pgfsetlinewidth{0.803000pt}%
\definecolor{currentstroke}{rgb}{0.000000,0.000000,0.000000}%
\pgfsetstrokecolor{currentstroke}%
\pgfsetdash{}{0pt}%
\pgfpathmoveto{\pgfqpoint{1.995685in}{3.098185in}}%
\pgfpathlineto{\pgfqpoint{3.158185in}{3.098185in}}%
\pgfusepath{stroke}%
\end{pgfscope}%
\begin{pgfscope}%
\pgfsetrectcap%
\pgfsetmiterjoin%
\pgfsetlinewidth{0.803000pt}%
\definecolor{currentstroke}{rgb}{0.000000,0.000000,0.000000}%
\pgfsetstrokecolor{currentstroke}%
\pgfsetdash{}{0pt}%
\pgfpathmoveto{\pgfqpoint{1.995685in}{3.853185in}}%
\pgfpathlineto{\pgfqpoint{3.158185in}{3.853185in}}%
\pgfusepath{stroke}%
\end{pgfscope}%
\begin{pgfscope}%
\pgfsetbuttcap%
\pgfsetmiterjoin%
\definecolor{currentfill}{rgb}{1.000000,1.000000,1.000000}%
\pgfsetfillcolor{currentfill}%
\pgfsetlinewidth{0.000000pt}%
\definecolor{currentstroke}{rgb}{0.000000,0.000000,0.000000}%
\pgfsetstrokecolor{currentstroke}%
\pgfsetstrokeopacity{0.000000}%
\pgfsetdash{}{0pt}%
\pgfpathmoveto{\pgfqpoint{3.158185in}{3.098185in}}%
\pgfpathlineto{\pgfqpoint{4.320685in}{3.098185in}}%
\pgfpathlineto{\pgfqpoint{4.320685in}{3.853185in}}%
\pgfpathlineto{\pgfqpoint{3.158185in}{3.853185in}}%
\pgfpathclose%
\pgfusepath{fill}%
\end{pgfscope}%
\begin{pgfscope}%
\pgfpathrectangle{\pgfqpoint{3.158185in}{3.098185in}}{\pgfqpoint{1.162500in}{0.755000in}} %
\pgfusepath{clip}%
\pgfsetbuttcap%
\pgfsetroundjoin%
\definecolor{currentfill}{rgb}{0.000000,0.000000,0.000000}%
\pgfsetfillcolor{currentfill}%
\pgfsetfillopacity{0.500000}%
\pgfsetlinewidth{0.000000pt}%
\definecolor{currentstroke}{rgb}{0.000000,0.000000,0.000000}%
\pgfsetstrokecolor{currentstroke}%
\pgfsetdash{}{0pt}%
\pgfpathmoveto{\pgfqpoint{4.293006in}{3.691634in}}%
\pgfpathcurveto{\pgfqpoint{4.298531in}{3.691634in}}{\pgfqpoint{4.303831in}{3.693829in}}{\pgfqpoint{4.307737in}{3.697736in}}%
\pgfpathcurveto{\pgfqpoint{4.311644in}{3.701643in}}{\pgfqpoint{4.313839in}{3.706942in}}{\pgfqpoint{4.313839in}{3.712467in}}%
\pgfpathcurveto{\pgfqpoint{4.313839in}{3.717993in}}{\pgfqpoint{4.311644in}{3.723292in}}{\pgfqpoint{4.307737in}{3.727199in}}%
\pgfpathcurveto{\pgfqpoint{4.303831in}{3.731106in}}{\pgfqpoint{4.298531in}{3.733301in}}{\pgfqpoint{4.293006in}{3.733301in}}%
\pgfpathcurveto{\pgfqpoint{4.287481in}{3.733301in}}{\pgfqpoint{4.282181in}{3.731106in}}{\pgfqpoint{4.278275in}{3.727199in}}%
\pgfpathcurveto{\pgfqpoint{4.274368in}{3.723292in}}{\pgfqpoint{4.272173in}{3.717993in}}{\pgfqpoint{4.272173in}{3.712467in}}%
\pgfpathcurveto{\pgfqpoint{4.272173in}{3.706942in}}{\pgfqpoint{4.274368in}{3.701643in}}{\pgfqpoint{4.278275in}{3.697736in}}%
\pgfpathcurveto{\pgfqpoint{4.282181in}{3.693829in}}{\pgfqpoint{4.287481in}{3.691634in}}{\pgfqpoint{4.293006in}{3.691634in}}%
\pgfpathclose%
\pgfusepath{fill}%
\end{pgfscope}%
\begin{pgfscope}%
\pgfpathrectangle{\pgfqpoint{3.158185in}{3.098185in}}{\pgfqpoint{1.162500in}{0.755000in}} %
\pgfusepath{clip}%
\pgfsetbuttcap%
\pgfsetroundjoin%
\definecolor{currentfill}{rgb}{0.000000,0.000000,0.000000}%
\pgfsetfillcolor{currentfill}%
\pgfsetfillopacity{0.500000}%
\pgfsetlinewidth{0.000000pt}%
\definecolor{currentstroke}{rgb}{0.000000,0.000000,0.000000}%
\pgfsetstrokecolor{currentstroke}%
\pgfsetdash{}{0pt}%
\pgfpathmoveto{\pgfqpoint{3.586831in}{3.134292in}}%
\pgfpathcurveto{\pgfqpoint{3.592356in}{3.134292in}}{\pgfqpoint{3.597655in}{3.136487in}}{\pgfqpoint{3.601562in}{3.140394in}}%
\pgfpathcurveto{\pgfqpoint{3.605469in}{3.144301in}}{\pgfqpoint{3.607664in}{3.149600in}}{\pgfqpoint{3.607664in}{3.155125in}}%
\pgfpathcurveto{\pgfqpoint{3.607664in}{3.160650in}}{\pgfqpoint{3.605469in}{3.165950in}}{\pgfqpoint{3.601562in}{3.169857in}}%
\pgfpathcurveto{\pgfqpoint{3.597655in}{3.173763in}}{\pgfqpoint{3.592356in}{3.175959in}}{\pgfqpoint{3.586831in}{3.175959in}}%
\pgfpathcurveto{\pgfqpoint{3.581306in}{3.175959in}}{\pgfqpoint{3.576006in}{3.173763in}}{\pgfqpoint{3.572099in}{3.169857in}}%
\pgfpathcurveto{\pgfqpoint{3.568193in}{3.165950in}}{\pgfqpoint{3.565998in}{3.160650in}}{\pgfqpoint{3.565998in}{3.155125in}}%
\pgfpathcurveto{\pgfqpoint{3.565998in}{3.149600in}}{\pgfqpoint{3.568193in}{3.144301in}}{\pgfqpoint{3.572099in}{3.140394in}}%
\pgfpathcurveto{\pgfqpoint{3.576006in}{3.136487in}}{\pgfqpoint{3.581306in}{3.134292in}}{\pgfqpoint{3.586831in}{3.134292in}}%
\pgfpathclose%
\pgfusepath{fill}%
\end{pgfscope}%
\begin{pgfscope}%
\pgfpathrectangle{\pgfqpoint{3.158185in}{3.098185in}}{\pgfqpoint{1.162500in}{0.755000in}} %
\pgfusepath{clip}%
\pgfsetbuttcap%
\pgfsetroundjoin%
\definecolor{currentfill}{rgb}{0.000000,0.000000,0.000000}%
\pgfsetfillcolor{currentfill}%
\pgfsetfillopacity{0.500000}%
\pgfsetlinewidth{0.000000pt}%
\definecolor{currentstroke}{rgb}{0.000000,0.000000,0.000000}%
\pgfsetstrokecolor{currentstroke}%
\pgfsetdash{}{0pt}%
\pgfpathmoveto{\pgfqpoint{3.578870in}{3.095327in}}%
\pgfpathcurveto{\pgfqpoint{3.584395in}{3.095327in}}{\pgfqpoint{3.589694in}{3.097523in}}{\pgfqpoint{3.593601in}{3.101429in}}%
\pgfpathcurveto{\pgfqpoint{3.597508in}{3.105336in}}{\pgfqpoint{3.599703in}{3.110636in}}{\pgfqpoint{3.599703in}{3.116161in}}%
\pgfpathcurveto{\pgfqpoint{3.599703in}{3.121686in}}{\pgfqpoint{3.597508in}{3.126985in}}{\pgfqpoint{3.593601in}{3.130892in}}%
\pgfpathcurveto{\pgfqpoint{3.589694in}{3.134799in}}{\pgfqpoint{3.584395in}{3.136994in}}{\pgfqpoint{3.578870in}{3.136994in}}%
\pgfpathcurveto{\pgfqpoint{3.573345in}{3.136994in}}{\pgfqpoint{3.568045in}{3.134799in}}{\pgfqpoint{3.564138in}{3.130892in}}%
\pgfpathcurveto{\pgfqpoint{3.560231in}{3.126985in}}{\pgfqpoint{3.558036in}{3.121686in}}{\pgfqpoint{3.558036in}{3.116161in}}%
\pgfpathcurveto{\pgfqpoint{3.558036in}{3.110636in}}{\pgfqpoint{3.560231in}{3.105336in}}{\pgfqpoint{3.564138in}{3.101429in}}%
\pgfpathcurveto{\pgfqpoint{3.568045in}{3.097523in}}{\pgfqpoint{3.573345in}{3.095327in}}{\pgfqpoint{3.578870in}{3.095327in}}%
\pgfpathclose%
\pgfusepath{fill}%
\end{pgfscope}%
\begin{pgfscope}%
\pgfpathrectangle{\pgfqpoint{3.158185in}{3.098185in}}{\pgfqpoint{1.162500in}{0.755000in}} %
\pgfusepath{clip}%
\pgfsetbuttcap%
\pgfsetroundjoin%
\definecolor{currentfill}{rgb}{0.000000,0.000000,0.000000}%
\pgfsetfillcolor{currentfill}%
\pgfsetfillopacity{0.500000}%
\pgfsetlinewidth{0.000000pt}%
\definecolor{currentstroke}{rgb}{0.000000,0.000000,0.000000}%
\pgfsetstrokecolor{currentstroke}%
\pgfsetdash{}{0pt}%
\pgfpathmoveto{\pgfqpoint{3.333069in}{3.302828in}}%
\pgfpathcurveto{\pgfqpoint{3.338594in}{3.302828in}}{\pgfqpoint{3.343894in}{3.305023in}}{\pgfqpoint{3.347801in}{3.308930in}}%
\pgfpathcurveto{\pgfqpoint{3.351708in}{3.312837in}}{\pgfqpoint{3.353903in}{3.318136in}}{\pgfqpoint{3.353903in}{3.323661in}}%
\pgfpathcurveto{\pgfqpoint{3.353903in}{3.329186in}}{\pgfqpoint{3.351708in}{3.334486in}}{\pgfqpoint{3.347801in}{3.338393in}}%
\pgfpathcurveto{\pgfqpoint{3.343894in}{3.342300in}}{\pgfqpoint{3.338594in}{3.344495in}}{\pgfqpoint{3.333069in}{3.344495in}}%
\pgfpathcurveto{\pgfqpoint{3.327544in}{3.344495in}}{\pgfqpoint{3.322245in}{3.342300in}}{\pgfqpoint{3.318338in}{3.338393in}}%
\pgfpathcurveto{\pgfqpoint{3.314431in}{3.334486in}}{\pgfqpoint{3.312236in}{3.329186in}}{\pgfqpoint{3.312236in}{3.323661in}}%
\pgfpathcurveto{\pgfqpoint{3.312236in}{3.318136in}}{\pgfqpoint{3.314431in}{3.312837in}}{\pgfqpoint{3.318338in}{3.308930in}}%
\pgfpathcurveto{\pgfqpoint{3.322245in}{3.305023in}}{\pgfqpoint{3.327544in}{3.302828in}}{\pgfqpoint{3.333069in}{3.302828in}}%
\pgfpathclose%
\pgfusepath{fill}%
\end{pgfscope}%
\begin{pgfscope}%
\pgfpathrectangle{\pgfqpoint{3.158185in}{3.098185in}}{\pgfqpoint{1.162500in}{0.755000in}} %
\pgfusepath{clip}%
\pgfsetbuttcap%
\pgfsetroundjoin%
\definecolor{currentfill}{rgb}{0.000000,0.000000,0.000000}%
\pgfsetfillcolor{currentfill}%
\pgfsetfillopacity{0.500000}%
\pgfsetlinewidth{0.000000pt}%
\definecolor{currentstroke}{rgb}{0.000000,0.000000,0.000000}%
\pgfsetstrokecolor{currentstroke}%
\pgfsetdash{}{0pt}%
\pgfpathmoveto{\pgfqpoint{3.296013in}{3.145618in}}%
\pgfpathcurveto{\pgfqpoint{3.301538in}{3.145618in}}{\pgfqpoint{3.306838in}{3.147813in}}{\pgfqpoint{3.310744in}{3.151720in}}%
\pgfpathcurveto{\pgfqpoint{3.314651in}{3.155626in}}{\pgfqpoint{3.316846in}{3.160926in}}{\pgfqpoint{3.316846in}{3.166451in}}%
\pgfpathcurveto{\pgfqpoint{3.316846in}{3.171976in}}{\pgfqpoint{3.314651in}{3.177276in}}{\pgfqpoint{3.310744in}{3.181182in}}%
\pgfpathcurveto{\pgfqpoint{3.306838in}{3.185089in}}{\pgfqpoint{3.301538in}{3.187284in}}{\pgfqpoint{3.296013in}{3.187284in}}%
\pgfpathcurveto{\pgfqpoint{3.290488in}{3.187284in}}{\pgfqpoint{3.285188in}{3.185089in}}{\pgfqpoint{3.281282in}{3.181182in}}%
\pgfpathcurveto{\pgfqpoint{3.277375in}{3.177276in}}{\pgfqpoint{3.275180in}{3.171976in}}{\pgfqpoint{3.275180in}{3.166451in}}%
\pgfpathcurveto{\pgfqpoint{3.275180in}{3.160926in}}{\pgfqpoint{3.277375in}{3.155626in}}{\pgfqpoint{3.281282in}{3.151720in}}%
\pgfpathcurveto{\pgfqpoint{3.285188in}{3.147813in}}{\pgfqpoint{3.290488in}{3.145618in}}{\pgfqpoint{3.296013in}{3.145618in}}%
\pgfpathclose%
\pgfusepath{fill}%
\end{pgfscope}%
\begin{pgfscope}%
\pgfpathrectangle{\pgfqpoint{3.158185in}{3.098185in}}{\pgfqpoint{1.162500in}{0.755000in}} %
\pgfusepath{clip}%
\pgfsetbuttcap%
\pgfsetroundjoin%
\definecolor{currentfill}{rgb}{0.000000,0.000000,0.000000}%
\pgfsetfillcolor{currentfill}%
\pgfsetfillopacity{0.500000}%
\pgfsetlinewidth{0.000000pt}%
\definecolor{currentstroke}{rgb}{0.000000,0.000000,0.000000}%
\pgfsetstrokecolor{currentstroke}%
\pgfsetdash{}{0pt}%
\pgfpathmoveto{\pgfqpoint{3.206735in}{3.249924in}}%
\pgfpathcurveto{\pgfqpoint{3.212260in}{3.249924in}}{\pgfqpoint{3.217559in}{3.252119in}}{\pgfqpoint{3.221466in}{3.256026in}}%
\pgfpathcurveto{\pgfqpoint{3.225373in}{3.259932in}}{\pgfqpoint{3.227568in}{3.265232in}}{\pgfqpoint{3.227568in}{3.270757in}}%
\pgfpathcurveto{\pgfqpoint{3.227568in}{3.276282in}}{\pgfqpoint{3.225373in}{3.281582in}}{\pgfqpoint{3.221466in}{3.285488in}}%
\pgfpathcurveto{\pgfqpoint{3.217559in}{3.289395in}}{\pgfqpoint{3.212260in}{3.291590in}}{\pgfqpoint{3.206735in}{3.291590in}}%
\pgfpathcurveto{\pgfqpoint{3.201210in}{3.291590in}}{\pgfqpoint{3.195910in}{3.289395in}}{\pgfqpoint{3.192003in}{3.285488in}}%
\pgfpathcurveto{\pgfqpoint{3.188097in}{3.281582in}}{\pgfqpoint{3.185901in}{3.276282in}}{\pgfqpoint{3.185901in}{3.270757in}}%
\pgfpathcurveto{\pgfqpoint{3.185901in}{3.265232in}}{\pgfqpoint{3.188097in}{3.259932in}}{\pgfqpoint{3.192003in}{3.256026in}}%
\pgfpathcurveto{\pgfqpoint{3.195910in}{3.252119in}}{\pgfqpoint{3.201210in}{3.249924in}}{\pgfqpoint{3.206735in}{3.249924in}}%
\pgfpathclose%
\pgfusepath{fill}%
\end{pgfscope}%
\begin{pgfscope}%
\pgfpathrectangle{\pgfqpoint{3.158185in}{3.098185in}}{\pgfqpoint{1.162500in}{0.755000in}} %
\pgfusepath{clip}%
\pgfsetbuttcap%
\pgfsetroundjoin%
\definecolor{currentfill}{rgb}{0.000000,0.000000,0.000000}%
\pgfsetfillcolor{currentfill}%
\pgfsetfillopacity{0.500000}%
\pgfsetlinewidth{0.000000pt}%
\definecolor{currentstroke}{rgb}{0.000000,0.000000,0.000000}%
\pgfsetstrokecolor{currentstroke}%
\pgfsetdash{}{0pt}%
\pgfpathmoveto{\pgfqpoint{3.185863in}{3.240978in}}%
\pgfpathcurveto{\pgfqpoint{3.191388in}{3.240978in}}{\pgfqpoint{3.196688in}{3.243173in}}{\pgfqpoint{3.200595in}{3.247080in}}%
\pgfpathcurveto{\pgfqpoint{3.204501in}{3.250987in}}{\pgfqpoint{3.206696in}{3.256286in}}{\pgfqpoint{3.206696in}{3.261812in}}%
\pgfpathcurveto{\pgfqpoint{3.206696in}{3.267337in}}{\pgfqpoint{3.204501in}{3.272636in}}{\pgfqpoint{3.200595in}{3.276543in}}%
\pgfpathcurveto{\pgfqpoint{3.196688in}{3.280450in}}{\pgfqpoint{3.191388in}{3.282645in}}{\pgfqpoint{3.185863in}{3.282645in}}%
\pgfpathcurveto{\pgfqpoint{3.180338in}{3.282645in}}{\pgfqpoint{3.175039in}{3.280450in}}{\pgfqpoint{3.171132in}{3.276543in}}%
\pgfpathcurveto{\pgfqpoint{3.167225in}{3.272636in}}{\pgfqpoint{3.165030in}{3.267337in}}{\pgfqpoint{3.165030in}{3.261812in}}%
\pgfpathcurveto{\pgfqpoint{3.165030in}{3.256286in}}{\pgfqpoint{3.167225in}{3.250987in}}{\pgfqpoint{3.171132in}{3.247080in}}%
\pgfpathcurveto{\pgfqpoint{3.175039in}{3.243173in}}{\pgfqpoint{3.180338in}{3.240978in}}{\pgfqpoint{3.185863in}{3.240978in}}%
\pgfpathclose%
\pgfusepath{fill}%
\end{pgfscope}%
\begin{pgfscope}%
\pgfpathrectangle{\pgfqpoint{3.158185in}{3.098185in}}{\pgfqpoint{1.162500in}{0.755000in}} %
\pgfusepath{clip}%
\pgfsetbuttcap%
\pgfsetroundjoin%
\definecolor{currentfill}{rgb}{0.000000,0.000000,0.000000}%
\pgfsetfillcolor{currentfill}%
\pgfsetfillopacity{0.500000}%
\pgfsetlinewidth{0.000000pt}%
\definecolor{currentstroke}{rgb}{0.000000,0.000000,0.000000}%
\pgfsetstrokecolor{currentstroke}%
\pgfsetdash{}{0pt}%
\pgfpathmoveto{\pgfqpoint{3.808562in}{3.768072in}}%
\pgfpathcurveto{\pgfqpoint{3.814087in}{3.768072in}}{\pgfqpoint{3.819386in}{3.770267in}}{\pgfqpoint{3.823293in}{3.774174in}}%
\pgfpathcurveto{\pgfqpoint{3.827200in}{3.778081in}}{\pgfqpoint{3.829395in}{3.783381in}}{\pgfqpoint{3.829395in}{3.788906in}}%
\pgfpathcurveto{\pgfqpoint{3.829395in}{3.794431in}}{\pgfqpoint{3.827200in}{3.799730in}}{\pgfqpoint{3.823293in}{3.803637in}}%
\pgfpathcurveto{\pgfqpoint{3.819386in}{3.807544in}}{\pgfqpoint{3.814087in}{3.809739in}}{\pgfqpoint{3.808562in}{3.809739in}}%
\pgfpathcurveto{\pgfqpoint{3.803037in}{3.809739in}}{\pgfqpoint{3.797737in}{3.807544in}}{\pgfqpoint{3.793830in}{3.803637in}}%
\pgfpathcurveto{\pgfqpoint{3.789924in}{3.799730in}}{\pgfqpoint{3.787728in}{3.794431in}}{\pgfqpoint{3.787728in}{3.788906in}}%
\pgfpathcurveto{\pgfqpoint{3.787728in}{3.783381in}}{\pgfqpoint{3.789924in}{3.778081in}}{\pgfqpoint{3.793830in}{3.774174in}}%
\pgfpathcurveto{\pgfqpoint{3.797737in}{3.770267in}}{\pgfqpoint{3.803037in}{3.768072in}}{\pgfqpoint{3.808562in}{3.768072in}}%
\pgfpathclose%
\pgfusepath{fill}%
\end{pgfscope}%
\begin{pgfscope}%
\pgfpathrectangle{\pgfqpoint{3.158185in}{3.098185in}}{\pgfqpoint{1.162500in}{0.755000in}} %
\pgfusepath{clip}%
\pgfsetbuttcap%
\pgfsetroundjoin%
\definecolor{currentfill}{rgb}{0.000000,0.000000,0.000000}%
\pgfsetfillcolor{currentfill}%
\pgfsetfillopacity{0.500000}%
\pgfsetlinewidth{0.000000pt}%
\definecolor{currentstroke}{rgb}{0.000000,0.000000,0.000000}%
\pgfsetstrokecolor{currentstroke}%
\pgfsetdash{}{0pt}%
\pgfpathmoveto{\pgfqpoint{3.620933in}{3.814375in}}%
\pgfpathcurveto{\pgfqpoint{3.626458in}{3.814375in}}{\pgfqpoint{3.631758in}{3.816570in}}{\pgfqpoint{3.635664in}{3.820477in}}%
\pgfpathcurveto{\pgfqpoint{3.639571in}{3.824384in}}{\pgfqpoint{3.641766in}{3.829683in}}{\pgfqpoint{3.641766in}{3.835208in}}%
\pgfpathcurveto{\pgfqpoint{3.641766in}{3.840733in}}{\pgfqpoint{3.639571in}{3.846033in}}{\pgfqpoint{3.635664in}{3.849940in}}%
\pgfpathcurveto{\pgfqpoint{3.631758in}{3.853847in}}{\pgfqpoint{3.626458in}{3.856042in}}{\pgfqpoint{3.620933in}{3.856042in}}%
\pgfpathcurveto{\pgfqpoint{3.615408in}{3.856042in}}{\pgfqpoint{3.610108in}{3.853847in}}{\pgfqpoint{3.606202in}{3.849940in}}%
\pgfpathcurveto{\pgfqpoint{3.602295in}{3.846033in}}{\pgfqpoint{3.600100in}{3.840733in}}{\pgfqpoint{3.600100in}{3.835208in}}%
\pgfpathcurveto{\pgfqpoint{3.600100in}{3.829683in}}{\pgfqpoint{3.602295in}{3.824384in}}{\pgfqpoint{3.606202in}{3.820477in}}%
\pgfpathcurveto{\pgfqpoint{3.610108in}{3.816570in}}{\pgfqpoint{3.615408in}{3.814375in}}{\pgfqpoint{3.620933in}{3.814375in}}%
\pgfpathclose%
\pgfusepath{fill}%
\end{pgfscope}%
\begin{pgfscope}%
\pgfpathrectangle{\pgfqpoint{3.158185in}{3.098185in}}{\pgfqpoint{1.162500in}{0.755000in}} %
\pgfusepath{clip}%
\pgfsetbuttcap%
\pgfsetroundjoin%
\definecolor{currentfill}{rgb}{0.000000,0.000000,0.000000}%
\pgfsetfillcolor{currentfill}%
\pgfsetfillopacity{0.500000}%
\pgfsetlinewidth{0.000000pt}%
\definecolor{currentstroke}{rgb}{0.000000,0.000000,0.000000}%
\pgfsetstrokecolor{currentstroke}%
\pgfsetdash{}{0pt}%
\pgfpathmoveto{\pgfqpoint{3.600191in}{3.667724in}}%
\pgfpathcurveto{\pgfqpoint{3.605716in}{3.667724in}}{\pgfqpoint{3.611016in}{3.669919in}}{\pgfqpoint{3.614922in}{3.673826in}}%
\pgfpathcurveto{\pgfqpoint{3.618829in}{3.677733in}}{\pgfqpoint{3.621024in}{3.683032in}}{\pgfqpoint{3.621024in}{3.688558in}}%
\pgfpathcurveto{\pgfqpoint{3.621024in}{3.694083in}}{\pgfqpoint{3.618829in}{3.699382in}}{\pgfqpoint{3.614922in}{3.703289in}}%
\pgfpathcurveto{\pgfqpoint{3.611016in}{3.707196in}}{\pgfqpoint{3.605716in}{3.709391in}}{\pgfqpoint{3.600191in}{3.709391in}}%
\pgfpathcurveto{\pgfqpoint{3.594666in}{3.709391in}}{\pgfqpoint{3.589366in}{3.707196in}}{\pgfqpoint{3.585460in}{3.703289in}}%
\pgfpathcurveto{\pgfqpoint{3.581553in}{3.699382in}}{\pgfqpoint{3.579358in}{3.694083in}}{\pgfqpoint{3.579358in}{3.688558in}}%
\pgfpathcurveto{\pgfqpoint{3.579358in}{3.683032in}}{\pgfqpoint{3.581553in}{3.677733in}}{\pgfqpoint{3.585460in}{3.673826in}}%
\pgfpathcurveto{\pgfqpoint{3.589366in}{3.669919in}}{\pgfqpoint{3.594666in}{3.667724in}}{\pgfqpoint{3.600191in}{3.667724in}}%
\pgfpathclose%
\pgfusepath{fill}%
\end{pgfscope}%
\begin{pgfscope}%
\pgfpathrectangle{\pgfqpoint{3.158185in}{3.098185in}}{\pgfqpoint{1.162500in}{0.755000in}} %
\pgfusepath{clip}%
\pgfsetbuttcap%
\pgfsetroundjoin%
\definecolor{currentfill}{rgb}{0.000000,0.000000,0.000000}%
\pgfsetfillcolor{currentfill}%
\pgfsetfillopacity{0.500000}%
\pgfsetlinewidth{0.000000pt}%
\definecolor{currentstroke}{rgb}{0.000000,0.000000,0.000000}%
\pgfsetstrokecolor{currentstroke}%
\pgfsetdash{}{0pt}%
\pgfpathmoveto{\pgfqpoint{3.899960in}{3.223705in}}%
\pgfpathcurveto{\pgfqpoint{3.905485in}{3.223705in}}{\pgfqpoint{3.910785in}{3.225900in}}{\pgfqpoint{3.914692in}{3.229807in}}%
\pgfpathcurveto{\pgfqpoint{3.918598in}{3.233713in}}{\pgfqpoint{3.920794in}{3.239013in}}{\pgfqpoint{3.920794in}{3.244538in}}%
\pgfpathcurveto{\pgfqpoint{3.920794in}{3.250063in}}{\pgfqpoint{3.918598in}{3.255363in}}{\pgfqpoint{3.914692in}{3.259269in}}%
\pgfpathcurveto{\pgfqpoint{3.910785in}{3.263176in}}{\pgfqpoint{3.905485in}{3.265371in}}{\pgfqpoint{3.899960in}{3.265371in}}%
\pgfpathcurveto{\pgfqpoint{3.894435in}{3.265371in}}{\pgfqpoint{3.889136in}{3.263176in}}{\pgfqpoint{3.885229in}{3.259269in}}%
\pgfpathcurveto{\pgfqpoint{3.881322in}{3.255363in}}{\pgfqpoint{3.879127in}{3.250063in}}{\pgfqpoint{3.879127in}{3.244538in}}%
\pgfpathcurveto{\pgfqpoint{3.879127in}{3.239013in}}{\pgfqpoint{3.881322in}{3.233713in}}{\pgfqpoint{3.885229in}{3.229807in}}%
\pgfpathcurveto{\pgfqpoint{3.889136in}{3.225900in}}{\pgfqpoint{3.894435in}{3.223705in}}{\pgfqpoint{3.899960in}{3.223705in}}%
\pgfpathclose%
\pgfusepath{fill}%
\end{pgfscope}%
\begin{pgfscope}%
\pgfpathrectangle{\pgfqpoint{3.158185in}{3.098185in}}{\pgfqpoint{1.162500in}{0.755000in}} %
\pgfusepath{clip}%
\pgfsetbuttcap%
\pgfsetroundjoin%
\definecolor{currentfill}{rgb}{0.000000,0.000000,0.000000}%
\pgfsetfillcolor{currentfill}%
\pgfsetfillopacity{0.500000}%
\pgfsetlinewidth{0.000000pt}%
\definecolor{currentstroke}{rgb}{0.000000,0.000000,0.000000}%
\pgfsetstrokecolor{currentstroke}%
\pgfsetdash{}{0pt}%
\pgfpathmoveto{\pgfqpoint{3.368320in}{3.524055in}}%
\pgfpathcurveto{\pgfqpoint{3.373845in}{3.524055in}}{\pgfqpoint{3.379145in}{3.526250in}}{\pgfqpoint{3.383052in}{3.530157in}}%
\pgfpathcurveto{\pgfqpoint{3.386958in}{3.534064in}}{\pgfqpoint{3.389154in}{3.539363in}}{\pgfqpoint{3.389154in}{3.544888in}}%
\pgfpathcurveto{\pgfqpoint{3.389154in}{3.550413in}}{\pgfqpoint{3.386958in}{3.555713in}}{\pgfqpoint{3.383052in}{3.559620in}}%
\pgfpathcurveto{\pgfqpoint{3.379145in}{3.563526in}}{\pgfqpoint{3.373845in}{3.565721in}}{\pgfqpoint{3.368320in}{3.565721in}}%
\pgfpathcurveto{\pgfqpoint{3.362795in}{3.565721in}}{\pgfqpoint{3.357496in}{3.563526in}}{\pgfqpoint{3.353589in}{3.559620in}}%
\pgfpathcurveto{\pgfqpoint{3.349682in}{3.555713in}}{\pgfqpoint{3.347487in}{3.550413in}}{\pgfqpoint{3.347487in}{3.544888in}}%
\pgfpathcurveto{\pgfqpoint{3.347487in}{3.539363in}}{\pgfqpoint{3.349682in}{3.534064in}}{\pgfqpoint{3.353589in}{3.530157in}}%
\pgfpathcurveto{\pgfqpoint{3.357496in}{3.526250in}}{\pgfqpoint{3.362795in}{3.524055in}}{\pgfqpoint{3.368320in}{3.524055in}}%
\pgfpathclose%
\pgfusepath{fill}%
\end{pgfscope}%
\begin{pgfscope}%
\pgfpathrectangle{\pgfqpoint{3.158185in}{3.098185in}}{\pgfqpoint{1.162500in}{0.755000in}} %
\pgfusepath{clip}%
\pgfsetbuttcap%
\pgfsetroundjoin%
\definecolor{currentfill}{rgb}{0.000000,0.000000,0.000000}%
\pgfsetfillcolor{currentfill}%
\pgfsetfillopacity{0.500000}%
\pgfsetlinewidth{0.000000pt}%
\definecolor{currentstroke}{rgb}{0.000000,0.000000,0.000000}%
\pgfsetstrokecolor{currentstroke}%
\pgfsetdash{}{0pt}%
\pgfpathmoveto{\pgfqpoint{3.416175in}{3.446907in}}%
\pgfpathcurveto{\pgfqpoint{3.421700in}{3.446907in}}{\pgfqpoint{3.427000in}{3.449102in}}{\pgfqpoint{3.430907in}{3.453009in}}%
\pgfpathcurveto{\pgfqpoint{3.434814in}{3.456916in}}{\pgfqpoint{3.437009in}{3.462215in}}{\pgfqpoint{3.437009in}{3.467740in}}%
\pgfpathcurveto{\pgfqpoint{3.437009in}{3.473265in}}{\pgfqpoint{3.434814in}{3.478565in}}{\pgfqpoint{3.430907in}{3.482472in}}%
\pgfpathcurveto{\pgfqpoint{3.427000in}{3.486379in}}{\pgfqpoint{3.421700in}{3.488574in}}{\pgfqpoint{3.416175in}{3.488574in}}%
\pgfpathcurveto{\pgfqpoint{3.410650in}{3.488574in}}{\pgfqpoint{3.405351in}{3.486379in}}{\pgfqpoint{3.401444in}{3.482472in}}%
\pgfpathcurveto{\pgfqpoint{3.397537in}{3.478565in}}{\pgfqpoint{3.395342in}{3.473265in}}{\pgfqpoint{3.395342in}{3.467740in}}%
\pgfpathcurveto{\pgfqpoint{3.395342in}{3.462215in}}{\pgfqpoint{3.397537in}{3.456916in}}{\pgfqpoint{3.401444in}{3.453009in}}%
\pgfpathcurveto{\pgfqpoint{3.405351in}{3.449102in}}{\pgfqpoint{3.410650in}{3.446907in}}{\pgfqpoint{3.416175in}{3.446907in}}%
\pgfpathclose%
\pgfusepath{fill}%
\end{pgfscope}%
\begin{pgfscope}%
\pgfpathrectangle{\pgfqpoint{3.158185in}{3.098185in}}{\pgfqpoint{1.162500in}{0.755000in}} %
\pgfusepath{clip}%
\pgfsetbuttcap%
\pgfsetroundjoin%
\definecolor{currentfill}{rgb}{0.000000,0.000000,0.000000}%
\pgfsetfillcolor{currentfill}%
\pgfsetfillopacity{0.500000}%
\pgfsetlinewidth{0.000000pt}%
\definecolor{currentstroke}{rgb}{0.000000,0.000000,0.000000}%
\pgfsetstrokecolor{currentstroke}%
\pgfsetdash{}{0pt}%
\pgfpathmoveto{\pgfqpoint{3.789669in}{3.220444in}}%
\pgfpathcurveto{\pgfqpoint{3.795194in}{3.220444in}}{\pgfqpoint{3.800493in}{3.222640in}}{\pgfqpoint{3.804400in}{3.226546in}}%
\pgfpathcurveto{\pgfqpoint{3.808307in}{3.230453in}}{\pgfqpoint{3.810502in}{3.235753in}}{\pgfqpoint{3.810502in}{3.241278in}}%
\pgfpathcurveto{\pgfqpoint{3.810502in}{3.246803in}}{\pgfqpoint{3.808307in}{3.252102in}}{\pgfqpoint{3.804400in}{3.256009in}}%
\pgfpathcurveto{\pgfqpoint{3.800493in}{3.259916in}}{\pgfqpoint{3.795194in}{3.262111in}}{\pgfqpoint{3.789669in}{3.262111in}}%
\pgfpathcurveto{\pgfqpoint{3.784143in}{3.262111in}}{\pgfqpoint{3.778844in}{3.259916in}}{\pgfqpoint{3.774937in}{3.256009in}}%
\pgfpathcurveto{\pgfqpoint{3.771030in}{3.252102in}}{\pgfqpoint{3.768835in}{3.246803in}}{\pgfqpoint{3.768835in}{3.241278in}}%
\pgfpathcurveto{\pgfqpoint{3.768835in}{3.235753in}}{\pgfqpoint{3.771030in}{3.230453in}}{\pgfqpoint{3.774937in}{3.226546in}}%
\pgfpathcurveto{\pgfqpoint{3.778844in}{3.222640in}}{\pgfqpoint{3.784143in}{3.220444in}}{\pgfqpoint{3.789669in}{3.220444in}}%
\pgfpathclose%
\pgfusepath{fill}%
\end{pgfscope}%
\begin{pgfscope}%
\pgfpathrectangle{\pgfqpoint{3.158185in}{3.098185in}}{\pgfqpoint{1.162500in}{0.755000in}} %
\pgfusepath{clip}%
\pgfsetbuttcap%
\pgfsetroundjoin%
\definecolor{currentfill}{rgb}{0.000000,0.000000,0.000000}%
\pgfsetfillcolor{currentfill}%
\pgfsetfillopacity{0.500000}%
\pgfsetlinewidth{0.000000pt}%
\definecolor{currentstroke}{rgb}{0.000000,0.000000,0.000000}%
\pgfsetstrokecolor{currentstroke}%
\pgfsetdash{}{0pt}%
\pgfpathmoveto{\pgfqpoint{3.231064in}{3.505459in}}%
\pgfpathcurveto{\pgfqpoint{3.236589in}{3.505459in}}{\pgfqpoint{3.241888in}{3.507654in}}{\pgfqpoint{3.245795in}{3.511561in}}%
\pgfpathcurveto{\pgfqpoint{3.249702in}{3.515468in}}{\pgfqpoint{3.251897in}{3.520768in}}{\pgfqpoint{3.251897in}{3.526293in}}%
\pgfpathcurveto{\pgfqpoint{3.251897in}{3.531818in}}{\pgfqpoint{3.249702in}{3.537117in}}{\pgfqpoint{3.245795in}{3.541024in}}%
\pgfpathcurveto{\pgfqpoint{3.241888in}{3.544931in}}{\pgfqpoint{3.236589in}{3.547126in}}{\pgfqpoint{3.231064in}{3.547126in}}%
\pgfpathcurveto{\pgfqpoint{3.225539in}{3.547126in}}{\pgfqpoint{3.220239in}{3.544931in}}{\pgfqpoint{3.216332in}{3.541024in}}%
\pgfpathcurveto{\pgfqpoint{3.212425in}{3.537117in}}{\pgfqpoint{3.210230in}{3.531818in}}{\pgfqpoint{3.210230in}{3.526293in}}%
\pgfpathcurveto{\pgfqpoint{3.210230in}{3.520768in}}{\pgfqpoint{3.212425in}{3.515468in}}{\pgfqpoint{3.216332in}{3.511561in}}%
\pgfpathcurveto{\pgfqpoint{3.220239in}{3.507654in}}{\pgfqpoint{3.225539in}{3.505459in}}{\pgfqpoint{3.231064in}{3.505459in}}%
\pgfpathclose%
\pgfusepath{fill}%
\end{pgfscope}%
\begin{pgfscope}%
\pgfsetrectcap%
\pgfsetmiterjoin%
\pgfsetlinewidth{0.803000pt}%
\definecolor{currentstroke}{rgb}{0.000000,0.000000,0.000000}%
\pgfsetstrokecolor{currentstroke}%
\pgfsetdash{}{0pt}%
\pgfpathmoveto{\pgfqpoint{3.158185in}{3.098185in}}%
\pgfpathlineto{\pgfqpoint{3.158185in}{3.853185in}}%
\pgfusepath{stroke}%
\end{pgfscope}%
\begin{pgfscope}%
\pgfsetrectcap%
\pgfsetmiterjoin%
\pgfsetlinewidth{0.803000pt}%
\definecolor{currentstroke}{rgb}{0.000000,0.000000,0.000000}%
\pgfsetstrokecolor{currentstroke}%
\pgfsetdash{}{0pt}%
\pgfpathmoveto{\pgfqpoint{4.320685in}{3.098185in}}%
\pgfpathlineto{\pgfqpoint{4.320685in}{3.853185in}}%
\pgfusepath{stroke}%
\end{pgfscope}%
\begin{pgfscope}%
\pgfsetrectcap%
\pgfsetmiterjoin%
\pgfsetlinewidth{0.803000pt}%
\definecolor{currentstroke}{rgb}{0.000000,0.000000,0.000000}%
\pgfsetstrokecolor{currentstroke}%
\pgfsetdash{}{0pt}%
\pgfpathmoveto{\pgfqpoint{3.158185in}{3.098185in}}%
\pgfpathlineto{\pgfqpoint{4.320685in}{3.098185in}}%
\pgfusepath{stroke}%
\end{pgfscope}%
\begin{pgfscope}%
\pgfsetrectcap%
\pgfsetmiterjoin%
\pgfsetlinewidth{0.803000pt}%
\definecolor{currentstroke}{rgb}{0.000000,0.000000,0.000000}%
\pgfsetstrokecolor{currentstroke}%
\pgfsetdash{}{0pt}%
\pgfpathmoveto{\pgfqpoint{3.158185in}{3.853185in}}%
\pgfpathlineto{\pgfqpoint{4.320685in}{3.853185in}}%
\pgfusepath{stroke}%
\end{pgfscope}%
\begin{pgfscope}%
\pgfsetbuttcap%
\pgfsetmiterjoin%
\definecolor{currentfill}{rgb}{1.000000,1.000000,1.000000}%
\pgfsetfillcolor{currentfill}%
\pgfsetlinewidth{0.000000pt}%
\definecolor{currentstroke}{rgb}{0.000000,0.000000,0.000000}%
\pgfsetstrokecolor{currentstroke}%
\pgfsetstrokeopacity{0.000000}%
\pgfsetdash{}{0pt}%
\pgfpathmoveto{\pgfqpoint{4.320685in}{3.098185in}}%
\pgfpathlineto{\pgfqpoint{5.483185in}{3.098185in}}%
\pgfpathlineto{\pgfqpoint{5.483185in}{3.853185in}}%
\pgfpathlineto{\pgfqpoint{4.320685in}{3.853185in}}%
\pgfpathclose%
\pgfusepath{fill}%
\end{pgfscope}%
\begin{pgfscope}%
\pgfpathrectangle{\pgfqpoint{4.320685in}{3.098185in}}{\pgfqpoint{1.162500in}{0.755000in}} %
\pgfusepath{clip}%
\pgfsetbuttcap%
\pgfsetroundjoin%
\definecolor{currentfill}{rgb}{0.000000,0.000000,0.000000}%
\pgfsetfillcolor{currentfill}%
\pgfsetfillopacity{0.500000}%
\pgfsetlinewidth{0.000000pt}%
\definecolor{currentstroke}{rgb}{0.000000,0.000000,0.000000}%
\pgfsetstrokecolor{currentstroke}%
\pgfsetdash{}{0pt}%
\pgfpathmoveto{\pgfqpoint{4.763543in}{3.691634in}}%
\pgfpathcurveto{\pgfqpoint{4.769068in}{3.691634in}}{\pgfqpoint{4.774367in}{3.693829in}}{\pgfqpoint{4.778274in}{3.697736in}}%
\pgfpathcurveto{\pgfqpoint{4.782181in}{3.701643in}}{\pgfqpoint{4.784376in}{3.706942in}}{\pgfqpoint{4.784376in}{3.712467in}}%
\pgfpathcurveto{\pgfqpoint{4.784376in}{3.717993in}}{\pgfqpoint{4.782181in}{3.723292in}}{\pgfqpoint{4.778274in}{3.727199in}}%
\pgfpathcurveto{\pgfqpoint{4.774367in}{3.731106in}}{\pgfqpoint{4.769068in}{3.733301in}}{\pgfqpoint{4.763543in}{3.733301in}}%
\pgfpathcurveto{\pgfqpoint{4.758018in}{3.733301in}}{\pgfqpoint{4.752718in}{3.731106in}}{\pgfqpoint{4.748811in}{3.727199in}}%
\pgfpathcurveto{\pgfqpoint{4.744905in}{3.723292in}}{\pgfqpoint{4.742709in}{3.717993in}}{\pgfqpoint{4.742709in}{3.712467in}}%
\pgfpathcurveto{\pgfqpoint{4.742709in}{3.706942in}}{\pgfqpoint{4.744905in}{3.701643in}}{\pgfqpoint{4.748811in}{3.697736in}}%
\pgfpathcurveto{\pgfqpoint{4.752718in}{3.693829in}}{\pgfqpoint{4.758018in}{3.691634in}}{\pgfqpoint{4.763543in}{3.691634in}}%
\pgfpathclose%
\pgfusepath{fill}%
\end{pgfscope}%
\begin{pgfscope}%
\pgfpathrectangle{\pgfqpoint{4.320685in}{3.098185in}}{\pgfqpoint{1.162500in}{0.755000in}} %
\pgfusepath{clip}%
\pgfsetbuttcap%
\pgfsetroundjoin%
\definecolor{currentfill}{rgb}{0.000000,0.000000,0.000000}%
\pgfsetfillcolor{currentfill}%
\pgfsetfillopacity{0.500000}%
\pgfsetlinewidth{0.000000pt}%
\definecolor{currentstroke}{rgb}{0.000000,0.000000,0.000000}%
\pgfsetstrokecolor{currentstroke}%
\pgfsetdash{}{0pt}%
\pgfpathmoveto{\pgfqpoint{5.385388in}{3.134292in}}%
\pgfpathcurveto{\pgfqpoint{5.390913in}{3.134292in}}{\pgfqpoint{5.396213in}{3.136487in}}{\pgfqpoint{5.400119in}{3.140394in}}%
\pgfpathcurveto{\pgfqpoint{5.404026in}{3.144301in}}{\pgfqpoint{5.406221in}{3.149600in}}{\pgfqpoint{5.406221in}{3.155125in}}%
\pgfpathcurveto{\pgfqpoint{5.406221in}{3.160650in}}{\pgfqpoint{5.404026in}{3.165950in}}{\pgfqpoint{5.400119in}{3.169857in}}%
\pgfpathcurveto{\pgfqpoint{5.396213in}{3.173763in}}{\pgfqpoint{5.390913in}{3.175959in}}{\pgfqpoint{5.385388in}{3.175959in}}%
\pgfpathcurveto{\pgfqpoint{5.379863in}{3.175959in}}{\pgfqpoint{5.374564in}{3.173763in}}{\pgfqpoint{5.370657in}{3.169857in}}%
\pgfpathcurveto{\pgfqpoint{5.366750in}{3.165950in}}{\pgfqpoint{5.364555in}{3.160650in}}{\pgfqpoint{5.364555in}{3.155125in}}%
\pgfpathcurveto{\pgfqpoint{5.364555in}{3.149600in}}{\pgfqpoint{5.366750in}{3.144301in}}{\pgfqpoint{5.370657in}{3.140394in}}%
\pgfpathcurveto{\pgfqpoint{5.374564in}{3.136487in}}{\pgfqpoint{5.379863in}{3.134292in}}{\pgfqpoint{5.385388in}{3.134292in}}%
\pgfpathclose%
\pgfusepath{fill}%
\end{pgfscope}%
\begin{pgfscope}%
\pgfpathrectangle{\pgfqpoint{4.320685in}{3.098185in}}{\pgfqpoint{1.162500in}{0.755000in}} %
\pgfusepath{clip}%
\pgfsetbuttcap%
\pgfsetroundjoin%
\definecolor{currentfill}{rgb}{0.000000,0.000000,0.000000}%
\pgfsetfillcolor{currentfill}%
\pgfsetfillopacity{0.500000}%
\pgfsetlinewidth{0.000000pt}%
\definecolor{currentstroke}{rgb}{0.000000,0.000000,0.000000}%
\pgfsetstrokecolor{currentstroke}%
\pgfsetdash{}{0pt}%
\pgfpathmoveto{\pgfqpoint{5.455506in}{3.095327in}}%
\pgfpathcurveto{\pgfqpoint{5.461031in}{3.095327in}}{\pgfqpoint{5.466331in}{3.097523in}}{\pgfqpoint{5.470237in}{3.101429in}}%
\pgfpathcurveto{\pgfqpoint{5.474144in}{3.105336in}}{\pgfqpoint{5.476339in}{3.110636in}}{\pgfqpoint{5.476339in}{3.116161in}}%
\pgfpathcurveto{\pgfqpoint{5.476339in}{3.121686in}}{\pgfqpoint{5.474144in}{3.126985in}}{\pgfqpoint{5.470237in}{3.130892in}}%
\pgfpathcurveto{\pgfqpoint{5.466331in}{3.134799in}}{\pgfqpoint{5.461031in}{3.136994in}}{\pgfqpoint{5.455506in}{3.136994in}}%
\pgfpathcurveto{\pgfqpoint{5.449981in}{3.136994in}}{\pgfqpoint{5.444681in}{3.134799in}}{\pgfqpoint{5.440775in}{3.130892in}}%
\pgfpathcurveto{\pgfqpoint{5.436868in}{3.126985in}}{\pgfqpoint{5.434673in}{3.121686in}}{\pgfqpoint{5.434673in}{3.116161in}}%
\pgfpathcurveto{\pgfqpoint{5.434673in}{3.110636in}}{\pgfqpoint{5.436868in}{3.105336in}}{\pgfqpoint{5.440775in}{3.101429in}}%
\pgfpathcurveto{\pgfqpoint{5.444681in}{3.097523in}}{\pgfqpoint{5.449981in}{3.095327in}}{\pgfqpoint{5.455506in}{3.095327in}}%
\pgfpathclose%
\pgfusepath{fill}%
\end{pgfscope}%
\begin{pgfscope}%
\pgfpathrectangle{\pgfqpoint{4.320685in}{3.098185in}}{\pgfqpoint{1.162500in}{0.755000in}} %
\pgfusepath{clip}%
\pgfsetbuttcap%
\pgfsetroundjoin%
\definecolor{currentfill}{rgb}{0.000000,0.000000,0.000000}%
\pgfsetfillcolor{currentfill}%
\pgfsetfillopacity{0.500000}%
\pgfsetlinewidth{0.000000pt}%
\definecolor{currentstroke}{rgb}{0.000000,0.000000,0.000000}%
\pgfsetstrokecolor{currentstroke}%
\pgfsetdash{}{0pt}%
\pgfpathmoveto{\pgfqpoint{5.031128in}{3.302828in}}%
\pgfpathcurveto{\pgfqpoint{5.036653in}{3.302828in}}{\pgfqpoint{5.041953in}{3.305023in}}{\pgfqpoint{5.045859in}{3.308930in}}%
\pgfpathcurveto{\pgfqpoint{5.049766in}{3.312837in}}{\pgfqpoint{5.051961in}{3.318136in}}{\pgfqpoint{5.051961in}{3.323661in}}%
\pgfpathcurveto{\pgfqpoint{5.051961in}{3.329186in}}{\pgfqpoint{5.049766in}{3.334486in}}{\pgfqpoint{5.045859in}{3.338393in}}%
\pgfpathcurveto{\pgfqpoint{5.041953in}{3.342300in}}{\pgfqpoint{5.036653in}{3.344495in}}{\pgfqpoint{5.031128in}{3.344495in}}%
\pgfpathcurveto{\pgfqpoint{5.025603in}{3.344495in}}{\pgfqpoint{5.020303in}{3.342300in}}{\pgfqpoint{5.016397in}{3.338393in}}%
\pgfpathcurveto{\pgfqpoint{5.012490in}{3.334486in}}{\pgfqpoint{5.010295in}{3.329186in}}{\pgfqpoint{5.010295in}{3.323661in}}%
\pgfpathcurveto{\pgfqpoint{5.010295in}{3.318136in}}{\pgfqpoint{5.012490in}{3.312837in}}{\pgfqpoint{5.016397in}{3.308930in}}%
\pgfpathcurveto{\pgfqpoint{5.020303in}{3.305023in}}{\pgfqpoint{5.025603in}{3.302828in}}{\pgfqpoint{5.031128in}{3.302828in}}%
\pgfpathclose%
\pgfusepath{fill}%
\end{pgfscope}%
\begin{pgfscope}%
\pgfpathrectangle{\pgfqpoint{4.320685in}{3.098185in}}{\pgfqpoint{1.162500in}{0.755000in}} %
\pgfusepath{clip}%
\pgfsetbuttcap%
\pgfsetroundjoin%
\definecolor{currentfill}{rgb}{0.000000,0.000000,0.000000}%
\pgfsetfillcolor{currentfill}%
\pgfsetfillopacity{0.500000}%
\pgfsetlinewidth{0.000000pt}%
\definecolor{currentstroke}{rgb}{0.000000,0.000000,0.000000}%
\pgfsetstrokecolor{currentstroke}%
\pgfsetdash{}{0pt}%
\pgfpathmoveto{\pgfqpoint{5.102087in}{3.145618in}}%
\pgfpathcurveto{\pgfqpoint{5.107612in}{3.145618in}}{\pgfqpoint{5.112912in}{3.147813in}}{\pgfqpoint{5.116819in}{3.151720in}}%
\pgfpathcurveto{\pgfqpoint{5.120726in}{3.155626in}}{\pgfqpoint{5.122921in}{3.160926in}}{\pgfqpoint{5.122921in}{3.166451in}}%
\pgfpathcurveto{\pgfqpoint{5.122921in}{3.171976in}}{\pgfqpoint{5.120726in}{3.177276in}}{\pgfqpoint{5.116819in}{3.181182in}}%
\pgfpathcurveto{\pgfqpoint{5.112912in}{3.185089in}}{\pgfqpoint{5.107612in}{3.187284in}}{\pgfqpoint{5.102087in}{3.187284in}}%
\pgfpathcurveto{\pgfqpoint{5.096562in}{3.187284in}}{\pgfqpoint{5.091263in}{3.185089in}}{\pgfqpoint{5.087356in}{3.181182in}}%
\pgfpathcurveto{\pgfqpoint{5.083449in}{3.177276in}}{\pgfqpoint{5.081254in}{3.171976in}}{\pgfqpoint{5.081254in}{3.166451in}}%
\pgfpathcurveto{\pgfqpoint{5.081254in}{3.160926in}}{\pgfqpoint{5.083449in}{3.155626in}}{\pgfqpoint{5.087356in}{3.151720in}}%
\pgfpathcurveto{\pgfqpoint{5.091263in}{3.147813in}}{\pgfqpoint{5.096562in}{3.145618in}}{\pgfqpoint{5.102087in}{3.145618in}}%
\pgfpathclose%
\pgfusepath{fill}%
\end{pgfscope}%
\begin{pgfscope}%
\pgfpathrectangle{\pgfqpoint{4.320685in}{3.098185in}}{\pgfqpoint{1.162500in}{0.755000in}} %
\pgfusepath{clip}%
\pgfsetbuttcap%
\pgfsetroundjoin%
\definecolor{currentfill}{rgb}{0.000000,0.000000,0.000000}%
\pgfsetfillcolor{currentfill}%
\pgfsetfillopacity{0.500000}%
\pgfsetlinewidth{0.000000pt}%
\definecolor{currentstroke}{rgb}{0.000000,0.000000,0.000000}%
\pgfsetstrokecolor{currentstroke}%
\pgfsetdash{}{0pt}%
\pgfpathmoveto{\pgfqpoint{4.940844in}{3.249924in}}%
\pgfpathcurveto{\pgfqpoint{4.946369in}{3.249924in}}{\pgfqpoint{4.951669in}{3.252119in}}{\pgfqpoint{4.955576in}{3.256026in}}%
\pgfpathcurveto{\pgfqpoint{4.959482in}{3.259932in}}{\pgfqpoint{4.961678in}{3.265232in}}{\pgfqpoint{4.961678in}{3.270757in}}%
\pgfpathcurveto{\pgfqpoint{4.961678in}{3.276282in}}{\pgfqpoint{4.959482in}{3.281582in}}{\pgfqpoint{4.955576in}{3.285488in}}%
\pgfpathcurveto{\pgfqpoint{4.951669in}{3.289395in}}{\pgfqpoint{4.946369in}{3.291590in}}{\pgfqpoint{4.940844in}{3.291590in}}%
\pgfpathcurveto{\pgfqpoint{4.935319in}{3.291590in}}{\pgfqpoint{4.930020in}{3.289395in}}{\pgfqpoint{4.926113in}{3.285488in}}%
\pgfpathcurveto{\pgfqpoint{4.922206in}{3.281582in}}{\pgfqpoint{4.920011in}{3.276282in}}{\pgfqpoint{4.920011in}{3.270757in}}%
\pgfpathcurveto{\pgfqpoint{4.920011in}{3.265232in}}{\pgfqpoint{4.922206in}{3.259932in}}{\pgfqpoint{4.926113in}{3.256026in}}%
\pgfpathcurveto{\pgfqpoint{4.930020in}{3.252119in}}{\pgfqpoint{4.935319in}{3.249924in}}{\pgfqpoint{4.940844in}{3.249924in}}%
\pgfpathclose%
\pgfusepath{fill}%
\end{pgfscope}%
\begin{pgfscope}%
\pgfpathrectangle{\pgfqpoint{4.320685in}{3.098185in}}{\pgfqpoint{1.162500in}{0.755000in}} %
\pgfusepath{clip}%
\pgfsetbuttcap%
\pgfsetroundjoin%
\definecolor{currentfill}{rgb}{0.000000,0.000000,0.000000}%
\pgfsetfillcolor{currentfill}%
\pgfsetfillopacity{0.500000}%
\pgfsetlinewidth{0.000000pt}%
\definecolor{currentstroke}{rgb}{0.000000,0.000000,0.000000}%
\pgfsetstrokecolor{currentstroke}%
\pgfsetdash{}{0pt}%
\pgfpathmoveto{\pgfqpoint{4.348363in}{3.240978in}}%
\pgfpathcurveto{\pgfqpoint{4.353888in}{3.240978in}}{\pgfqpoint{4.359188in}{3.243173in}}{\pgfqpoint{4.363095in}{3.247080in}}%
\pgfpathcurveto{\pgfqpoint{4.367001in}{3.250987in}}{\pgfqpoint{4.369196in}{3.256286in}}{\pgfqpoint{4.369196in}{3.261812in}}%
\pgfpathcurveto{\pgfqpoint{4.369196in}{3.267337in}}{\pgfqpoint{4.367001in}{3.272636in}}{\pgfqpoint{4.363095in}{3.276543in}}%
\pgfpathcurveto{\pgfqpoint{4.359188in}{3.280450in}}{\pgfqpoint{4.353888in}{3.282645in}}{\pgfqpoint{4.348363in}{3.282645in}}%
\pgfpathcurveto{\pgfqpoint{4.342838in}{3.282645in}}{\pgfqpoint{4.337539in}{3.280450in}}{\pgfqpoint{4.333632in}{3.276543in}}%
\pgfpathcurveto{\pgfqpoint{4.329725in}{3.272636in}}{\pgfqpoint{4.327530in}{3.267337in}}{\pgfqpoint{4.327530in}{3.261812in}}%
\pgfpathcurveto{\pgfqpoint{4.327530in}{3.256286in}}{\pgfqpoint{4.329725in}{3.250987in}}{\pgfqpoint{4.333632in}{3.247080in}}%
\pgfpathcurveto{\pgfqpoint{4.337539in}{3.243173in}}{\pgfqpoint{4.342838in}{3.240978in}}{\pgfqpoint{4.348363in}{3.240978in}}%
\pgfpathclose%
\pgfusepath{fill}%
\end{pgfscope}%
\begin{pgfscope}%
\pgfpathrectangle{\pgfqpoint{4.320685in}{3.098185in}}{\pgfqpoint{1.162500in}{0.755000in}} %
\pgfusepath{clip}%
\pgfsetbuttcap%
\pgfsetroundjoin%
\definecolor{currentfill}{rgb}{0.000000,0.000000,0.000000}%
\pgfsetfillcolor{currentfill}%
\pgfsetfillopacity{0.500000}%
\pgfsetlinewidth{0.000000pt}%
\definecolor{currentstroke}{rgb}{0.000000,0.000000,0.000000}%
\pgfsetstrokecolor{currentstroke}%
\pgfsetdash{}{0pt}%
\pgfpathmoveto{\pgfqpoint{5.332823in}{3.768072in}}%
\pgfpathcurveto{\pgfqpoint{5.338348in}{3.768072in}}{\pgfqpoint{5.343648in}{3.770267in}}{\pgfqpoint{5.347555in}{3.774174in}}%
\pgfpathcurveto{\pgfqpoint{5.351461in}{3.778081in}}{\pgfqpoint{5.353656in}{3.783381in}}{\pgfqpoint{5.353656in}{3.788906in}}%
\pgfpathcurveto{\pgfqpoint{5.353656in}{3.794431in}}{\pgfqpoint{5.351461in}{3.799730in}}{\pgfqpoint{5.347555in}{3.803637in}}%
\pgfpathcurveto{\pgfqpoint{5.343648in}{3.807544in}}{\pgfqpoint{5.338348in}{3.809739in}}{\pgfqpoint{5.332823in}{3.809739in}}%
\pgfpathcurveto{\pgfqpoint{5.327298in}{3.809739in}}{\pgfqpoint{5.321999in}{3.807544in}}{\pgfqpoint{5.318092in}{3.803637in}}%
\pgfpathcurveto{\pgfqpoint{5.314185in}{3.799730in}}{\pgfqpoint{5.311990in}{3.794431in}}{\pgfqpoint{5.311990in}{3.788906in}}%
\pgfpathcurveto{\pgfqpoint{5.311990in}{3.783381in}}{\pgfqpoint{5.314185in}{3.778081in}}{\pgfqpoint{5.318092in}{3.774174in}}%
\pgfpathcurveto{\pgfqpoint{5.321999in}{3.770267in}}{\pgfqpoint{5.327298in}{3.768072in}}{\pgfqpoint{5.332823in}{3.768072in}}%
\pgfpathclose%
\pgfusepath{fill}%
\end{pgfscope}%
\begin{pgfscope}%
\pgfpathrectangle{\pgfqpoint{4.320685in}{3.098185in}}{\pgfqpoint{1.162500in}{0.755000in}} %
\pgfusepath{clip}%
\pgfsetbuttcap%
\pgfsetroundjoin%
\definecolor{currentfill}{rgb}{0.000000,0.000000,0.000000}%
\pgfsetfillcolor{currentfill}%
\pgfsetfillopacity{0.500000}%
\pgfsetlinewidth{0.000000pt}%
\definecolor{currentstroke}{rgb}{0.000000,0.000000,0.000000}%
\pgfsetstrokecolor{currentstroke}%
\pgfsetdash{}{0pt}%
\pgfpathmoveto{\pgfqpoint{5.090412in}{3.814375in}}%
\pgfpathcurveto{\pgfqpoint{5.095937in}{3.814375in}}{\pgfqpoint{5.101237in}{3.816570in}}{\pgfqpoint{5.105143in}{3.820477in}}%
\pgfpathcurveto{\pgfqpoint{5.109050in}{3.824384in}}{\pgfqpoint{5.111245in}{3.829683in}}{\pgfqpoint{5.111245in}{3.835208in}}%
\pgfpathcurveto{\pgfqpoint{5.111245in}{3.840733in}}{\pgfqpoint{5.109050in}{3.846033in}}{\pgfqpoint{5.105143in}{3.849940in}}%
\pgfpathcurveto{\pgfqpoint{5.101237in}{3.853847in}}{\pgfqpoint{5.095937in}{3.856042in}}{\pgfqpoint{5.090412in}{3.856042in}}%
\pgfpathcurveto{\pgfqpoint{5.084887in}{3.856042in}}{\pgfqpoint{5.079587in}{3.853847in}}{\pgfqpoint{5.075681in}{3.849940in}}%
\pgfpathcurveto{\pgfqpoint{5.071774in}{3.846033in}}{\pgfqpoint{5.069579in}{3.840733in}}{\pgfqpoint{5.069579in}{3.835208in}}%
\pgfpathcurveto{\pgfqpoint{5.069579in}{3.829683in}}{\pgfqpoint{5.071774in}{3.824384in}}{\pgfqpoint{5.075681in}{3.820477in}}%
\pgfpathcurveto{\pgfqpoint{5.079587in}{3.816570in}}{\pgfqpoint{5.084887in}{3.814375in}}{\pgfqpoint{5.090412in}{3.814375in}}%
\pgfpathclose%
\pgfusepath{fill}%
\end{pgfscope}%
\begin{pgfscope}%
\pgfpathrectangle{\pgfqpoint{4.320685in}{3.098185in}}{\pgfqpoint{1.162500in}{0.755000in}} %
\pgfusepath{clip}%
\pgfsetbuttcap%
\pgfsetroundjoin%
\definecolor{currentfill}{rgb}{0.000000,0.000000,0.000000}%
\pgfsetfillcolor{currentfill}%
\pgfsetfillopacity{0.500000}%
\pgfsetlinewidth{0.000000pt}%
\definecolor{currentstroke}{rgb}{0.000000,0.000000,0.000000}%
\pgfsetstrokecolor{currentstroke}%
\pgfsetdash{}{0pt}%
\pgfpathmoveto{\pgfqpoint{4.817426in}{3.667724in}}%
\pgfpathcurveto{\pgfqpoint{4.822951in}{3.667724in}}{\pgfqpoint{4.828251in}{3.669919in}}{\pgfqpoint{4.832158in}{3.673826in}}%
\pgfpathcurveto{\pgfqpoint{4.836065in}{3.677733in}}{\pgfqpoint{4.838260in}{3.683032in}}{\pgfqpoint{4.838260in}{3.688558in}}%
\pgfpathcurveto{\pgfqpoint{4.838260in}{3.694083in}}{\pgfqpoint{4.836065in}{3.699382in}}{\pgfqpoint{4.832158in}{3.703289in}}%
\pgfpathcurveto{\pgfqpoint{4.828251in}{3.707196in}}{\pgfqpoint{4.822951in}{3.709391in}}{\pgfqpoint{4.817426in}{3.709391in}}%
\pgfpathcurveto{\pgfqpoint{4.811901in}{3.709391in}}{\pgfqpoint{4.806602in}{3.707196in}}{\pgfqpoint{4.802695in}{3.703289in}}%
\pgfpathcurveto{\pgfqpoint{4.798788in}{3.699382in}}{\pgfqpoint{4.796593in}{3.694083in}}{\pgfqpoint{4.796593in}{3.688558in}}%
\pgfpathcurveto{\pgfqpoint{4.796593in}{3.683032in}}{\pgfqpoint{4.798788in}{3.677733in}}{\pgfqpoint{4.802695in}{3.673826in}}%
\pgfpathcurveto{\pgfqpoint{4.806602in}{3.669919in}}{\pgfqpoint{4.811901in}{3.667724in}}{\pgfqpoint{4.817426in}{3.667724in}}%
\pgfpathclose%
\pgfusepath{fill}%
\end{pgfscope}%
\begin{pgfscope}%
\pgfpathrectangle{\pgfqpoint{4.320685in}{3.098185in}}{\pgfqpoint{1.162500in}{0.755000in}} %
\pgfusepath{clip}%
\pgfsetbuttcap%
\pgfsetroundjoin%
\definecolor{currentfill}{rgb}{0.000000,0.000000,0.000000}%
\pgfsetfillcolor{currentfill}%
\pgfsetfillopacity{0.500000}%
\pgfsetlinewidth{0.000000pt}%
\definecolor{currentstroke}{rgb}{0.000000,0.000000,0.000000}%
\pgfsetstrokecolor{currentstroke}%
\pgfsetdash{}{0pt}%
\pgfpathmoveto{\pgfqpoint{5.258215in}{3.223705in}}%
\pgfpathcurveto{\pgfqpoint{5.263740in}{3.223705in}}{\pgfqpoint{5.269040in}{3.225900in}}{\pgfqpoint{5.272947in}{3.229807in}}%
\pgfpathcurveto{\pgfqpoint{5.276853in}{3.233713in}}{\pgfqpoint{5.279048in}{3.239013in}}{\pgfqpoint{5.279048in}{3.244538in}}%
\pgfpathcurveto{\pgfqpoint{5.279048in}{3.250063in}}{\pgfqpoint{5.276853in}{3.255363in}}{\pgfqpoint{5.272947in}{3.259269in}}%
\pgfpathcurveto{\pgfqpoint{5.269040in}{3.263176in}}{\pgfqpoint{5.263740in}{3.265371in}}{\pgfqpoint{5.258215in}{3.265371in}}%
\pgfpathcurveto{\pgfqpoint{5.252690in}{3.265371in}}{\pgfqpoint{5.247391in}{3.263176in}}{\pgfqpoint{5.243484in}{3.259269in}}%
\pgfpathcurveto{\pgfqpoint{5.239577in}{3.255363in}}{\pgfqpoint{5.237382in}{3.250063in}}{\pgfqpoint{5.237382in}{3.244538in}}%
\pgfpathcurveto{\pgfqpoint{5.237382in}{3.239013in}}{\pgfqpoint{5.239577in}{3.233713in}}{\pgfqpoint{5.243484in}{3.229807in}}%
\pgfpathcurveto{\pgfqpoint{5.247391in}{3.225900in}}{\pgfqpoint{5.252690in}{3.223705in}}{\pgfqpoint{5.258215in}{3.223705in}}%
\pgfpathclose%
\pgfusepath{fill}%
\end{pgfscope}%
\begin{pgfscope}%
\pgfpathrectangle{\pgfqpoint{4.320685in}{3.098185in}}{\pgfqpoint{1.162500in}{0.755000in}} %
\pgfusepath{clip}%
\pgfsetbuttcap%
\pgfsetroundjoin%
\definecolor{currentfill}{rgb}{0.000000,0.000000,0.000000}%
\pgfsetfillcolor{currentfill}%
\pgfsetfillopacity{0.500000}%
\pgfsetlinewidth{0.000000pt}%
\definecolor{currentstroke}{rgb}{0.000000,0.000000,0.000000}%
\pgfsetstrokecolor{currentstroke}%
\pgfsetdash{}{0pt}%
\pgfpathmoveto{\pgfqpoint{4.803827in}{3.524055in}}%
\pgfpathcurveto{\pgfqpoint{4.809352in}{3.524055in}}{\pgfqpoint{4.814651in}{3.526250in}}{\pgfqpoint{4.818558in}{3.530157in}}%
\pgfpathcurveto{\pgfqpoint{4.822465in}{3.534064in}}{\pgfqpoint{4.824660in}{3.539363in}}{\pgfqpoint{4.824660in}{3.544888in}}%
\pgfpathcurveto{\pgfqpoint{4.824660in}{3.550413in}}{\pgfqpoint{4.822465in}{3.555713in}}{\pgfqpoint{4.818558in}{3.559620in}}%
\pgfpathcurveto{\pgfqpoint{4.814651in}{3.563526in}}{\pgfqpoint{4.809352in}{3.565721in}}{\pgfqpoint{4.803827in}{3.565721in}}%
\pgfpathcurveto{\pgfqpoint{4.798302in}{3.565721in}}{\pgfqpoint{4.793002in}{3.563526in}}{\pgfqpoint{4.789096in}{3.559620in}}%
\pgfpathcurveto{\pgfqpoint{4.785189in}{3.555713in}}{\pgfqpoint{4.782994in}{3.550413in}}{\pgfqpoint{4.782994in}{3.544888in}}%
\pgfpathcurveto{\pgfqpoint{4.782994in}{3.539363in}}{\pgfqpoint{4.785189in}{3.534064in}}{\pgfqpoint{4.789096in}{3.530157in}}%
\pgfpathcurveto{\pgfqpoint{4.793002in}{3.526250in}}{\pgfqpoint{4.798302in}{3.524055in}}{\pgfqpoint{4.803827in}{3.524055in}}%
\pgfpathclose%
\pgfusepath{fill}%
\end{pgfscope}%
\begin{pgfscope}%
\pgfpathrectangle{\pgfqpoint{4.320685in}{3.098185in}}{\pgfqpoint{1.162500in}{0.755000in}} %
\pgfusepath{clip}%
\pgfsetbuttcap%
\pgfsetroundjoin%
\definecolor{currentfill}{rgb}{0.000000,0.000000,0.000000}%
\pgfsetfillcolor{currentfill}%
\pgfsetfillopacity{0.500000}%
\pgfsetlinewidth{0.000000pt}%
\definecolor{currentstroke}{rgb}{0.000000,0.000000,0.000000}%
\pgfsetstrokecolor{currentstroke}%
\pgfsetdash{}{0pt}%
\pgfpathmoveto{\pgfqpoint{4.437194in}{3.446907in}}%
\pgfpathcurveto{\pgfqpoint{4.442719in}{3.446907in}}{\pgfqpoint{4.448018in}{3.449102in}}{\pgfqpoint{4.451925in}{3.453009in}}%
\pgfpathcurveto{\pgfqpoint{4.455832in}{3.456916in}}{\pgfqpoint{4.458027in}{3.462215in}}{\pgfqpoint{4.458027in}{3.467740in}}%
\pgfpathcurveto{\pgfqpoint{4.458027in}{3.473265in}}{\pgfqpoint{4.455832in}{3.478565in}}{\pgfqpoint{4.451925in}{3.482472in}}%
\pgfpathcurveto{\pgfqpoint{4.448018in}{3.486379in}}{\pgfqpoint{4.442719in}{3.488574in}}{\pgfqpoint{4.437194in}{3.488574in}}%
\pgfpathcurveto{\pgfqpoint{4.431669in}{3.488574in}}{\pgfqpoint{4.426369in}{3.486379in}}{\pgfqpoint{4.422462in}{3.482472in}}%
\pgfpathcurveto{\pgfqpoint{4.418556in}{3.478565in}}{\pgfqpoint{4.416360in}{3.473265in}}{\pgfqpoint{4.416360in}{3.467740in}}%
\pgfpathcurveto{\pgfqpoint{4.416360in}{3.462215in}}{\pgfqpoint{4.418556in}{3.456916in}}{\pgfqpoint{4.422462in}{3.453009in}}%
\pgfpathcurveto{\pgfqpoint{4.426369in}{3.449102in}}{\pgfqpoint{4.431669in}{3.446907in}}{\pgfqpoint{4.437194in}{3.446907in}}%
\pgfpathclose%
\pgfusepath{fill}%
\end{pgfscope}%
\begin{pgfscope}%
\pgfpathrectangle{\pgfqpoint{4.320685in}{3.098185in}}{\pgfqpoint{1.162500in}{0.755000in}} %
\pgfusepath{clip}%
\pgfsetbuttcap%
\pgfsetroundjoin%
\definecolor{currentfill}{rgb}{0.000000,0.000000,0.000000}%
\pgfsetfillcolor{currentfill}%
\pgfsetfillopacity{0.500000}%
\pgfsetlinewidth{0.000000pt}%
\definecolor{currentstroke}{rgb}{0.000000,0.000000,0.000000}%
\pgfsetstrokecolor{currentstroke}%
\pgfsetdash{}{0pt}%
\pgfpathmoveto{\pgfqpoint{5.262967in}{3.220444in}}%
\pgfpathcurveto{\pgfqpoint{5.268492in}{3.220444in}}{\pgfqpoint{5.273791in}{3.222640in}}{\pgfqpoint{5.277698in}{3.226546in}}%
\pgfpathcurveto{\pgfqpoint{5.281605in}{3.230453in}}{\pgfqpoint{5.283800in}{3.235753in}}{\pgfqpoint{5.283800in}{3.241278in}}%
\pgfpathcurveto{\pgfqpoint{5.283800in}{3.246803in}}{\pgfqpoint{5.281605in}{3.252102in}}{\pgfqpoint{5.277698in}{3.256009in}}%
\pgfpathcurveto{\pgfqpoint{5.273791in}{3.259916in}}{\pgfqpoint{5.268492in}{3.262111in}}{\pgfqpoint{5.262967in}{3.262111in}}%
\pgfpathcurveto{\pgfqpoint{5.257442in}{3.262111in}}{\pgfqpoint{5.252142in}{3.259916in}}{\pgfqpoint{5.248235in}{3.256009in}}%
\pgfpathcurveto{\pgfqpoint{5.244329in}{3.252102in}}{\pgfqpoint{5.242134in}{3.246803in}}{\pgfqpoint{5.242134in}{3.241278in}}%
\pgfpathcurveto{\pgfqpoint{5.242134in}{3.235753in}}{\pgfqpoint{5.244329in}{3.230453in}}{\pgfqpoint{5.248235in}{3.226546in}}%
\pgfpathcurveto{\pgfqpoint{5.252142in}{3.222640in}}{\pgfqpoint{5.257442in}{3.220444in}}{\pgfqpoint{5.262967in}{3.220444in}}%
\pgfpathclose%
\pgfusepath{fill}%
\end{pgfscope}%
\begin{pgfscope}%
\pgfpathrectangle{\pgfqpoint{4.320685in}{3.098185in}}{\pgfqpoint{1.162500in}{0.755000in}} %
\pgfusepath{clip}%
\pgfsetbuttcap%
\pgfsetroundjoin%
\definecolor{currentfill}{rgb}{0.000000,0.000000,0.000000}%
\pgfsetfillcolor{currentfill}%
\pgfsetfillopacity{0.500000}%
\pgfsetlinewidth{0.000000pt}%
\definecolor{currentstroke}{rgb}{0.000000,0.000000,0.000000}%
\pgfsetstrokecolor{currentstroke}%
\pgfsetdash{}{0pt}%
\pgfpathmoveto{\pgfqpoint{4.896922in}{3.505459in}}%
\pgfpathcurveto{\pgfqpoint{4.902447in}{3.505459in}}{\pgfqpoint{4.907746in}{3.507654in}}{\pgfqpoint{4.911653in}{3.511561in}}%
\pgfpathcurveto{\pgfqpoint{4.915560in}{3.515468in}}{\pgfqpoint{4.917755in}{3.520768in}}{\pgfqpoint{4.917755in}{3.526293in}}%
\pgfpathcurveto{\pgfqpoint{4.917755in}{3.531818in}}{\pgfqpoint{4.915560in}{3.537117in}}{\pgfqpoint{4.911653in}{3.541024in}}%
\pgfpathcurveto{\pgfqpoint{4.907746in}{3.544931in}}{\pgfqpoint{4.902447in}{3.547126in}}{\pgfqpoint{4.896922in}{3.547126in}}%
\pgfpathcurveto{\pgfqpoint{4.891397in}{3.547126in}}{\pgfqpoint{4.886097in}{3.544931in}}{\pgfqpoint{4.882190in}{3.541024in}}%
\pgfpathcurveto{\pgfqpoint{4.878284in}{3.537117in}}{\pgfqpoint{4.876088in}{3.531818in}}{\pgfqpoint{4.876088in}{3.526293in}}%
\pgfpathcurveto{\pgfqpoint{4.876088in}{3.520768in}}{\pgfqpoint{4.878284in}{3.515468in}}{\pgfqpoint{4.882190in}{3.511561in}}%
\pgfpathcurveto{\pgfqpoint{4.886097in}{3.507654in}}{\pgfqpoint{4.891397in}{3.505459in}}{\pgfqpoint{4.896922in}{3.505459in}}%
\pgfpathclose%
\pgfusepath{fill}%
\end{pgfscope}%
\begin{pgfscope}%
\pgfsetrectcap%
\pgfsetmiterjoin%
\pgfsetlinewidth{0.803000pt}%
\definecolor{currentstroke}{rgb}{0.000000,0.000000,0.000000}%
\pgfsetstrokecolor{currentstroke}%
\pgfsetdash{}{0pt}%
\pgfpathmoveto{\pgfqpoint{4.320685in}{3.098185in}}%
\pgfpathlineto{\pgfqpoint{4.320685in}{3.853185in}}%
\pgfusepath{stroke}%
\end{pgfscope}%
\begin{pgfscope}%
\pgfsetrectcap%
\pgfsetmiterjoin%
\pgfsetlinewidth{0.803000pt}%
\definecolor{currentstroke}{rgb}{0.000000,0.000000,0.000000}%
\pgfsetstrokecolor{currentstroke}%
\pgfsetdash{}{0pt}%
\pgfpathmoveto{\pgfqpoint{5.483185in}{3.098185in}}%
\pgfpathlineto{\pgfqpoint{5.483185in}{3.853185in}}%
\pgfusepath{stroke}%
\end{pgfscope}%
\begin{pgfscope}%
\pgfsetrectcap%
\pgfsetmiterjoin%
\pgfsetlinewidth{0.803000pt}%
\definecolor{currentstroke}{rgb}{0.000000,0.000000,0.000000}%
\pgfsetstrokecolor{currentstroke}%
\pgfsetdash{}{0pt}%
\pgfpathmoveto{\pgfqpoint{4.320685in}{3.098185in}}%
\pgfpathlineto{\pgfqpoint{5.483185in}{3.098185in}}%
\pgfusepath{stroke}%
\end{pgfscope}%
\begin{pgfscope}%
\pgfsetrectcap%
\pgfsetmiterjoin%
\pgfsetlinewidth{0.803000pt}%
\definecolor{currentstroke}{rgb}{0.000000,0.000000,0.000000}%
\pgfsetstrokecolor{currentstroke}%
\pgfsetdash{}{0pt}%
\pgfpathmoveto{\pgfqpoint{4.320685in}{3.853185in}}%
\pgfpathlineto{\pgfqpoint{5.483185in}{3.853185in}}%
\pgfusepath{stroke}%
\end{pgfscope}%
\begin{pgfscope}%
\pgfsetbuttcap%
\pgfsetmiterjoin%
\definecolor{currentfill}{rgb}{1.000000,1.000000,1.000000}%
\pgfsetfillcolor{currentfill}%
\pgfsetlinewidth{0.000000pt}%
\definecolor{currentstroke}{rgb}{0.000000,0.000000,0.000000}%
\pgfsetstrokecolor{currentstroke}%
\pgfsetstrokeopacity{0.000000}%
\pgfsetdash{}{0pt}%
\pgfpathmoveto{\pgfqpoint{0.833185in}{2.343185in}}%
\pgfpathlineto{\pgfqpoint{1.995685in}{2.343185in}}%
\pgfpathlineto{\pgfqpoint{1.995685in}{3.098185in}}%
\pgfpathlineto{\pgfqpoint{0.833185in}{3.098185in}}%
\pgfpathclose%
\pgfusepath{fill}%
\end{pgfscope}%
\begin{pgfscope}%
\pgfpathrectangle{\pgfqpoint{0.833185in}{2.343185in}}{\pgfqpoint{1.162500in}{0.755000in}} %
\pgfusepath{clip}%
\pgfsetbuttcap%
\pgfsetroundjoin%
\definecolor{currentfill}{rgb}{0.000000,0.000000,0.000000}%
\pgfsetfillcolor{currentfill}%
\pgfsetfillopacity{0.500000}%
\pgfsetlinewidth{0.000000pt}%
\definecolor{currentstroke}{rgb}{0.000000,0.000000,0.000000}%
\pgfsetstrokecolor{currentstroke}%
\pgfsetdash{}{0pt}%
\pgfpathmoveto{\pgfqpoint{1.779018in}{3.059375in}}%
\pgfpathcurveto{\pgfqpoint{1.784543in}{3.059375in}}{\pgfqpoint{1.789842in}{3.061570in}}{\pgfqpoint{1.793749in}{3.065477in}}%
\pgfpathcurveto{\pgfqpoint{1.797656in}{3.069384in}}{\pgfqpoint{1.799851in}{3.074683in}}{\pgfqpoint{1.799851in}{3.080208in}}%
\pgfpathcurveto{\pgfqpoint{1.799851in}{3.085733in}}{\pgfqpoint{1.797656in}{3.091033in}}{\pgfqpoint{1.793749in}{3.094940in}}%
\pgfpathcurveto{\pgfqpoint{1.789842in}{3.098847in}}{\pgfqpoint{1.784543in}{3.101042in}}{\pgfqpoint{1.779018in}{3.101042in}}%
\pgfpathcurveto{\pgfqpoint{1.773492in}{3.101042in}}{\pgfqpoint{1.768193in}{3.098847in}}{\pgfqpoint{1.764286in}{3.094940in}}%
\pgfpathcurveto{\pgfqpoint{1.760379in}{3.091033in}}{\pgfqpoint{1.758184in}{3.085733in}}{\pgfqpoint{1.758184in}{3.080208in}}%
\pgfpathcurveto{\pgfqpoint{1.758184in}{3.074683in}}{\pgfqpoint{1.760379in}{3.069384in}}{\pgfqpoint{1.764286in}{3.065477in}}%
\pgfpathcurveto{\pgfqpoint{1.768193in}{3.061570in}}{\pgfqpoint{1.773492in}{3.059375in}}{\pgfqpoint{1.779018in}{3.059375in}}%
\pgfpathclose%
\pgfusepath{fill}%
\end{pgfscope}%
\begin{pgfscope}%
\pgfpathrectangle{\pgfqpoint{0.833185in}{2.343185in}}{\pgfqpoint{1.162500in}{0.755000in}} %
\pgfusepath{clip}%
\pgfsetbuttcap%
\pgfsetroundjoin%
\definecolor{currentfill}{rgb}{0.000000,0.000000,0.000000}%
\pgfsetfillcolor{currentfill}%
\pgfsetfillopacity{0.500000}%
\pgfsetlinewidth{0.000000pt}%
\definecolor{currentstroke}{rgb}{0.000000,0.000000,0.000000}%
\pgfsetstrokecolor{currentstroke}%
\pgfsetdash{}{0pt}%
\pgfpathmoveto{\pgfqpoint{0.920858in}{2.815854in}}%
\pgfpathcurveto{\pgfqpoint{0.926383in}{2.815854in}}{\pgfqpoint{0.931683in}{2.818049in}}{\pgfqpoint{0.935589in}{2.821955in}}%
\pgfpathcurveto{\pgfqpoint{0.939496in}{2.825862in}}{\pgfqpoint{0.941691in}{2.831162in}}{\pgfqpoint{0.941691in}{2.836687in}}%
\pgfpathcurveto{\pgfqpoint{0.941691in}{2.842212in}}{\pgfqpoint{0.939496in}{2.847511in}}{\pgfqpoint{0.935589in}{2.851418in}}%
\pgfpathcurveto{\pgfqpoint{0.931683in}{2.855325in}}{\pgfqpoint{0.926383in}{2.857520in}}{\pgfqpoint{0.920858in}{2.857520in}}%
\pgfpathcurveto{\pgfqpoint{0.915333in}{2.857520in}}{\pgfqpoint{0.910034in}{2.855325in}}{\pgfqpoint{0.906127in}{2.851418in}}%
\pgfpathcurveto{\pgfqpoint{0.902220in}{2.847511in}}{\pgfqpoint{0.900025in}{2.842212in}}{\pgfqpoint{0.900025in}{2.836687in}}%
\pgfpathcurveto{\pgfqpoint{0.900025in}{2.831162in}}{\pgfqpoint{0.902220in}{2.825862in}}{\pgfqpoint{0.906127in}{2.821955in}}%
\pgfpathcurveto{\pgfqpoint{0.910034in}{2.818049in}}{\pgfqpoint{0.915333in}{2.815854in}}{\pgfqpoint{0.920858in}{2.815854in}}%
\pgfpathclose%
\pgfusepath{fill}%
\end{pgfscope}%
\begin{pgfscope}%
\pgfpathrectangle{\pgfqpoint{0.833185in}{2.343185in}}{\pgfqpoint{1.162500in}{0.755000in}} %
\pgfusepath{clip}%
\pgfsetbuttcap%
\pgfsetroundjoin%
\definecolor{currentfill}{rgb}{0.000000,0.000000,0.000000}%
\pgfsetfillcolor{currentfill}%
\pgfsetfillopacity{0.500000}%
\pgfsetlinewidth{0.000000pt}%
\definecolor{currentstroke}{rgb}{0.000000,0.000000,0.000000}%
\pgfsetstrokecolor{currentstroke}%
\pgfsetdash{}{0pt}%
\pgfpathmoveto{\pgfqpoint{0.860863in}{2.815581in}}%
\pgfpathcurveto{\pgfqpoint{0.866388in}{2.815581in}}{\pgfqpoint{0.871688in}{2.817776in}}{\pgfqpoint{0.875595in}{2.821682in}}%
\pgfpathcurveto{\pgfqpoint{0.879501in}{2.825589in}}{\pgfqpoint{0.881696in}{2.830889in}}{\pgfqpoint{0.881696in}{2.836414in}}%
\pgfpathcurveto{\pgfqpoint{0.881696in}{2.841939in}}{\pgfqpoint{0.879501in}{2.847238in}}{\pgfqpoint{0.875595in}{2.851145in}}%
\pgfpathcurveto{\pgfqpoint{0.871688in}{2.855052in}}{\pgfqpoint{0.866388in}{2.857247in}}{\pgfqpoint{0.860863in}{2.857247in}}%
\pgfpathcurveto{\pgfqpoint{0.855338in}{2.857247in}}{\pgfqpoint{0.850039in}{2.855052in}}{\pgfqpoint{0.846132in}{2.851145in}}%
\pgfpathcurveto{\pgfqpoint{0.842225in}{2.847238in}}{\pgfqpoint{0.840030in}{2.841939in}}{\pgfqpoint{0.840030in}{2.836414in}}%
\pgfpathcurveto{\pgfqpoint{0.840030in}{2.830889in}}{\pgfqpoint{0.842225in}{2.825589in}}{\pgfqpoint{0.846132in}{2.821682in}}%
\pgfpathcurveto{\pgfqpoint{0.850039in}{2.817776in}}{\pgfqpoint{0.855338in}{2.815581in}}{\pgfqpoint{0.860863in}{2.815581in}}%
\pgfpathclose%
\pgfusepath{fill}%
\end{pgfscope}%
\begin{pgfscope}%
\pgfpathrectangle{\pgfqpoint{0.833185in}{2.343185in}}{\pgfqpoint{1.162500in}{0.755000in}} %
\pgfusepath{clip}%
\pgfsetbuttcap%
\pgfsetroundjoin%
\definecolor{currentfill}{rgb}{0.000000,0.000000,0.000000}%
\pgfsetfillcolor{currentfill}%
\pgfsetfillopacity{0.500000}%
\pgfsetlinewidth{0.000000pt}%
\definecolor{currentstroke}{rgb}{0.000000,0.000000,0.000000}%
\pgfsetstrokecolor{currentstroke}%
\pgfsetdash{}{0pt}%
\pgfpathmoveto{\pgfqpoint{1.180359in}{2.531108in}}%
\pgfpathcurveto{\pgfqpoint{1.185884in}{2.531108in}}{\pgfqpoint{1.191184in}{2.533303in}}{\pgfqpoint{1.195090in}{2.537210in}}%
\pgfpathcurveto{\pgfqpoint{1.198997in}{2.541116in}}{\pgfqpoint{1.201192in}{2.546416in}}{\pgfqpoint{1.201192in}{2.551941in}}%
\pgfpathcurveto{\pgfqpoint{1.201192in}{2.557466in}}{\pgfqpoint{1.198997in}{2.562766in}}{\pgfqpoint{1.195090in}{2.566672in}}%
\pgfpathcurveto{\pgfqpoint{1.191184in}{2.570579in}}{\pgfqpoint{1.185884in}{2.572774in}}{\pgfqpoint{1.180359in}{2.572774in}}%
\pgfpathcurveto{\pgfqpoint{1.174834in}{2.572774in}}{\pgfqpoint{1.169535in}{2.570579in}}{\pgfqpoint{1.165628in}{2.566672in}}%
\pgfpathcurveto{\pgfqpoint{1.161721in}{2.562766in}}{\pgfqpoint{1.159526in}{2.557466in}}{\pgfqpoint{1.159526in}{2.551941in}}%
\pgfpathcurveto{\pgfqpoint{1.159526in}{2.546416in}}{\pgfqpoint{1.161721in}{2.541116in}}{\pgfqpoint{1.165628in}{2.537210in}}%
\pgfpathcurveto{\pgfqpoint{1.169535in}{2.533303in}}{\pgfqpoint{1.174834in}{2.531108in}}{\pgfqpoint{1.180359in}{2.531108in}}%
\pgfpathclose%
\pgfusepath{fill}%
\end{pgfscope}%
\begin{pgfscope}%
\pgfpathrectangle{\pgfqpoint{0.833185in}{2.343185in}}{\pgfqpoint{1.162500in}{0.755000in}} %
\pgfusepath{clip}%
\pgfsetbuttcap%
\pgfsetroundjoin%
\definecolor{currentfill}{rgb}{0.000000,0.000000,0.000000}%
\pgfsetfillcolor{currentfill}%
\pgfsetfillopacity{0.500000}%
\pgfsetlinewidth{0.000000pt}%
\definecolor{currentstroke}{rgb}{0.000000,0.000000,0.000000}%
\pgfsetstrokecolor{currentstroke}%
\pgfsetdash{}{0pt}%
\pgfpathmoveto{\pgfqpoint{0.938297in}{2.481860in}}%
\pgfpathcurveto{\pgfqpoint{0.943822in}{2.481860in}}{\pgfqpoint{0.949121in}{2.484055in}}{\pgfqpoint{0.953028in}{2.487962in}}%
\pgfpathcurveto{\pgfqpoint{0.956935in}{2.491868in}}{\pgfqpoint{0.959130in}{2.497168in}}{\pgfqpoint{0.959130in}{2.502693in}}%
\pgfpathcurveto{\pgfqpoint{0.959130in}{2.508218in}}{\pgfqpoint{0.956935in}{2.513517in}}{\pgfqpoint{0.953028in}{2.517424in}}%
\pgfpathcurveto{\pgfqpoint{0.949121in}{2.521331in}}{\pgfqpoint{0.943822in}{2.523526in}}{\pgfqpoint{0.938297in}{2.523526in}}%
\pgfpathcurveto{\pgfqpoint{0.932772in}{2.523526in}}{\pgfqpoint{0.927472in}{2.521331in}}{\pgfqpoint{0.923565in}{2.517424in}}%
\pgfpathcurveto{\pgfqpoint{0.919659in}{2.513517in}}{\pgfqpoint{0.917464in}{2.508218in}}{\pgfqpoint{0.917464in}{2.502693in}}%
\pgfpathcurveto{\pgfqpoint{0.917464in}{2.497168in}}{\pgfqpoint{0.919659in}{2.491868in}}{\pgfqpoint{0.923565in}{2.487962in}}%
\pgfpathcurveto{\pgfqpoint{0.927472in}{2.484055in}}{\pgfqpoint{0.932772in}{2.481860in}}{\pgfqpoint{0.938297in}{2.481860in}}%
\pgfpathclose%
\pgfusepath{fill}%
\end{pgfscope}%
\begin{pgfscope}%
\pgfpathrectangle{\pgfqpoint{0.833185in}{2.343185in}}{\pgfqpoint{1.162500in}{0.755000in}} %
\pgfusepath{clip}%
\pgfsetbuttcap%
\pgfsetroundjoin%
\definecolor{currentfill}{rgb}{0.000000,0.000000,0.000000}%
\pgfsetfillcolor{currentfill}%
\pgfsetfillopacity{0.500000}%
\pgfsetlinewidth{0.000000pt}%
\definecolor{currentstroke}{rgb}{0.000000,0.000000,0.000000}%
\pgfsetstrokecolor{currentstroke}%
\pgfsetdash{}{0pt}%
\pgfpathmoveto{\pgfqpoint{1.098900in}{2.394040in}}%
\pgfpathcurveto{\pgfqpoint{1.104426in}{2.394040in}}{\pgfqpoint{1.109725in}{2.396235in}}{\pgfqpoint{1.113632in}{2.400142in}}%
\pgfpathcurveto{\pgfqpoint{1.117539in}{2.404048in}}{\pgfqpoint{1.119734in}{2.409348in}}{\pgfqpoint{1.119734in}{2.414873in}}%
\pgfpathcurveto{\pgfqpoint{1.119734in}{2.420398in}}{\pgfqpoint{1.117539in}{2.425698in}}{\pgfqpoint{1.113632in}{2.429604in}}%
\pgfpathcurveto{\pgfqpoint{1.109725in}{2.433511in}}{\pgfqpoint{1.104426in}{2.435706in}}{\pgfqpoint{1.098900in}{2.435706in}}%
\pgfpathcurveto{\pgfqpoint{1.093375in}{2.435706in}}{\pgfqpoint{1.088076in}{2.433511in}}{\pgfqpoint{1.084169in}{2.429604in}}%
\pgfpathcurveto{\pgfqpoint{1.080262in}{2.425698in}}{\pgfqpoint{1.078067in}{2.420398in}}{\pgfqpoint{1.078067in}{2.414873in}}%
\pgfpathcurveto{\pgfqpoint{1.078067in}{2.409348in}}{\pgfqpoint{1.080262in}{2.404048in}}{\pgfqpoint{1.084169in}{2.400142in}}%
\pgfpathcurveto{\pgfqpoint{1.088076in}{2.396235in}}{\pgfqpoint{1.093375in}{2.394040in}}{\pgfqpoint{1.098900in}{2.394040in}}%
\pgfpathclose%
\pgfusepath{fill}%
\end{pgfscope}%
\begin{pgfscope}%
\pgfpathrectangle{\pgfqpoint{0.833185in}{2.343185in}}{\pgfqpoint{1.162500in}{0.755000in}} %
\pgfusepath{clip}%
\pgfsetbuttcap%
\pgfsetroundjoin%
\definecolor{currentfill}{rgb}{0.000000,0.000000,0.000000}%
\pgfsetfillcolor{currentfill}%
\pgfsetfillopacity{0.500000}%
\pgfsetlinewidth{0.000000pt}%
\definecolor{currentstroke}{rgb}{0.000000,0.000000,0.000000}%
\pgfsetstrokecolor{currentstroke}%
\pgfsetdash{}{0pt}%
\pgfpathmoveto{\pgfqpoint{1.085127in}{2.340327in}}%
\pgfpathcurveto{\pgfqpoint{1.090652in}{2.340327in}}{\pgfqpoint{1.095951in}{2.342523in}}{\pgfqpoint{1.099858in}{2.346429in}}%
\pgfpathcurveto{\pgfqpoint{1.103765in}{2.350336in}}{\pgfqpoint{1.105960in}{2.355636in}}{\pgfqpoint{1.105960in}{2.361161in}}%
\pgfpathcurveto{\pgfqpoint{1.105960in}{2.366686in}}{\pgfqpoint{1.103765in}{2.371985in}}{\pgfqpoint{1.099858in}{2.375892in}}%
\pgfpathcurveto{\pgfqpoint{1.095951in}{2.379799in}}{\pgfqpoint{1.090652in}{2.381994in}}{\pgfqpoint{1.085127in}{2.381994in}}%
\pgfpathcurveto{\pgfqpoint{1.079602in}{2.381994in}}{\pgfqpoint{1.074302in}{2.379799in}}{\pgfqpoint{1.070395in}{2.375892in}}%
\pgfpathcurveto{\pgfqpoint{1.066488in}{2.371985in}}{\pgfqpoint{1.064293in}{2.366686in}}{\pgfqpoint{1.064293in}{2.361161in}}%
\pgfpathcurveto{\pgfqpoint{1.064293in}{2.355636in}}{\pgfqpoint{1.066488in}{2.350336in}}{\pgfqpoint{1.070395in}{2.346429in}}%
\pgfpathcurveto{\pgfqpoint{1.074302in}{2.342523in}}{\pgfqpoint{1.079602in}{2.340327in}}{\pgfqpoint{1.085127in}{2.340327in}}%
\pgfpathclose%
\pgfusepath{fill}%
\end{pgfscope}%
\begin{pgfscope}%
\pgfpathrectangle{\pgfqpoint{0.833185in}{2.343185in}}{\pgfqpoint{1.162500in}{0.755000in}} %
\pgfusepath{clip}%
\pgfsetbuttcap%
\pgfsetroundjoin%
\definecolor{currentfill}{rgb}{0.000000,0.000000,0.000000}%
\pgfsetfillcolor{currentfill}%
\pgfsetfillopacity{0.500000}%
\pgfsetlinewidth{0.000000pt}%
\definecolor{currentstroke}{rgb}{0.000000,0.000000,0.000000}%
\pgfsetstrokecolor{currentstroke}%
\pgfsetdash{}{0pt}%
\pgfpathmoveto{\pgfqpoint{1.896712in}{2.914639in}}%
\pgfpathcurveto{\pgfqpoint{1.902237in}{2.914639in}}{\pgfqpoint{1.907537in}{2.916834in}}{\pgfqpoint{1.911443in}{2.920741in}}%
\pgfpathcurveto{\pgfqpoint{1.915350in}{2.924648in}}{\pgfqpoint{1.917545in}{2.929947in}}{\pgfqpoint{1.917545in}{2.935473in}}%
\pgfpathcurveto{\pgfqpoint{1.917545in}{2.940998in}}{\pgfqpoint{1.915350in}{2.946297in}}{\pgfqpoint{1.911443in}{2.950204in}}%
\pgfpathcurveto{\pgfqpoint{1.907537in}{2.954111in}}{\pgfqpoint{1.902237in}{2.956306in}}{\pgfqpoint{1.896712in}{2.956306in}}%
\pgfpathcurveto{\pgfqpoint{1.891187in}{2.956306in}}{\pgfqpoint{1.885887in}{2.954111in}}{\pgfqpoint{1.881981in}{2.950204in}}%
\pgfpathcurveto{\pgfqpoint{1.878074in}{2.946297in}}{\pgfqpoint{1.875879in}{2.940998in}}{\pgfqpoint{1.875879in}{2.935473in}}%
\pgfpathcurveto{\pgfqpoint{1.875879in}{2.929947in}}{\pgfqpoint{1.878074in}{2.924648in}}{\pgfqpoint{1.881981in}{2.920741in}}%
\pgfpathcurveto{\pgfqpoint{1.885887in}{2.916834in}}{\pgfqpoint{1.891187in}{2.914639in}}{\pgfqpoint{1.896712in}{2.914639in}}%
\pgfpathclose%
\pgfusepath{fill}%
\end{pgfscope}%
\begin{pgfscope}%
\pgfpathrectangle{\pgfqpoint{0.833185in}{2.343185in}}{\pgfqpoint{1.162500in}{0.755000in}} %
\pgfusepath{clip}%
\pgfsetbuttcap%
\pgfsetroundjoin%
\definecolor{currentfill}{rgb}{0.000000,0.000000,0.000000}%
\pgfsetfillcolor{currentfill}%
\pgfsetfillopacity{0.500000}%
\pgfsetlinewidth{0.000000pt}%
\definecolor{currentstroke}{rgb}{0.000000,0.000000,0.000000}%
\pgfsetstrokecolor{currentstroke}%
\pgfsetdash{}{0pt}%
\pgfpathmoveto{\pgfqpoint{1.968006in}{2.734550in}}%
\pgfpathcurveto{\pgfqpoint{1.973531in}{2.734550in}}{\pgfqpoint{1.978831in}{2.736745in}}{\pgfqpoint{1.982737in}{2.740652in}}%
\pgfpathcurveto{\pgfqpoint{1.986644in}{2.744559in}}{\pgfqpoint{1.988839in}{2.749858in}}{\pgfqpoint{1.988839in}{2.755383in}}%
\pgfpathcurveto{\pgfqpoint{1.988839in}{2.760908in}}{\pgfqpoint{1.986644in}{2.766208in}}{\pgfqpoint{1.982737in}{2.770115in}}%
\pgfpathcurveto{\pgfqpoint{1.978831in}{2.774021in}}{\pgfqpoint{1.973531in}{2.776217in}}{\pgfqpoint{1.968006in}{2.776217in}}%
\pgfpathcurveto{\pgfqpoint{1.962481in}{2.776217in}}{\pgfqpoint{1.957181in}{2.774021in}}{\pgfqpoint{1.953275in}{2.770115in}}%
\pgfpathcurveto{\pgfqpoint{1.949368in}{2.766208in}}{\pgfqpoint{1.947173in}{2.760908in}}{\pgfqpoint{1.947173in}{2.755383in}}%
\pgfpathcurveto{\pgfqpoint{1.947173in}{2.749858in}}{\pgfqpoint{1.949368in}{2.744559in}}{\pgfqpoint{1.953275in}{2.740652in}}%
\pgfpathcurveto{\pgfqpoint{1.957181in}{2.736745in}}{\pgfqpoint{1.962481in}{2.734550in}}{\pgfqpoint{1.968006in}{2.734550in}}%
\pgfpathclose%
\pgfusepath{fill}%
\end{pgfscope}%
\begin{pgfscope}%
\pgfpathrectangle{\pgfqpoint{0.833185in}{2.343185in}}{\pgfqpoint{1.162500in}{0.755000in}} %
\pgfusepath{clip}%
\pgfsetbuttcap%
\pgfsetroundjoin%
\definecolor{currentfill}{rgb}{0.000000,0.000000,0.000000}%
\pgfsetfillcolor{currentfill}%
\pgfsetfillopacity{0.500000}%
\pgfsetlinewidth{0.000000pt}%
\definecolor{currentstroke}{rgb}{0.000000,0.000000,0.000000}%
\pgfsetstrokecolor{currentstroke}%
\pgfsetdash{}{0pt}%
\pgfpathmoveto{\pgfqpoint{1.742203in}{2.641930in}}%
\pgfpathcurveto{\pgfqpoint{1.747728in}{2.641930in}}{\pgfqpoint{1.753027in}{2.644125in}}{\pgfqpoint{1.756934in}{2.648032in}}%
\pgfpathcurveto{\pgfqpoint{1.760841in}{2.651939in}}{\pgfqpoint{1.763036in}{2.657239in}}{\pgfqpoint{1.763036in}{2.662764in}}%
\pgfpathcurveto{\pgfqpoint{1.763036in}{2.668289in}}{\pgfqpoint{1.760841in}{2.673588in}}{\pgfqpoint{1.756934in}{2.677495in}}%
\pgfpathcurveto{\pgfqpoint{1.753027in}{2.681402in}}{\pgfqpoint{1.747728in}{2.683597in}}{\pgfqpoint{1.742203in}{2.683597in}}%
\pgfpathcurveto{\pgfqpoint{1.736677in}{2.683597in}}{\pgfqpoint{1.731378in}{2.681402in}}{\pgfqpoint{1.727471in}{2.677495in}}%
\pgfpathcurveto{\pgfqpoint{1.723564in}{2.673588in}}{\pgfqpoint{1.721369in}{2.668289in}}{\pgfqpoint{1.721369in}{2.662764in}}%
\pgfpathcurveto{\pgfqpoint{1.721369in}{2.657239in}}{\pgfqpoint{1.723564in}{2.651939in}}{\pgfqpoint{1.727471in}{2.648032in}}%
\pgfpathcurveto{\pgfqpoint{1.731378in}{2.644125in}}{\pgfqpoint{1.736677in}{2.641930in}}{\pgfqpoint{1.742203in}{2.641930in}}%
\pgfpathclose%
\pgfusepath{fill}%
\end{pgfscope}%
\begin{pgfscope}%
\pgfpathrectangle{\pgfqpoint{0.833185in}{2.343185in}}{\pgfqpoint{1.162500in}{0.755000in}} %
\pgfusepath{clip}%
\pgfsetbuttcap%
\pgfsetroundjoin%
\definecolor{currentfill}{rgb}{0.000000,0.000000,0.000000}%
\pgfsetfillcolor{currentfill}%
\pgfsetfillopacity{0.500000}%
\pgfsetlinewidth{0.000000pt}%
\definecolor{currentstroke}{rgb}{0.000000,0.000000,0.000000}%
\pgfsetstrokecolor{currentstroke}%
\pgfsetdash{}{0pt}%
\pgfpathmoveto{\pgfqpoint{1.058530in}{2.975096in}}%
\pgfpathcurveto{\pgfqpoint{1.064055in}{2.975096in}}{\pgfqpoint{1.069355in}{2.977291in}}{\pgfqpoint{1.073261in}{2.981198in}}%
\pgfpathcurveto{\pgfqpoint{1.077168in}{2.985104in}}{\pgfqpoint{1.079363in}{2.990404in}}{\pgfqpoint{1.079363in}{2.995929in}}%
\pgfpathcurveto{\pgfqpoint{1.079363in}{3.001454in}}{\pgfqpoint{1.077168in}{3.006754in}}{\pgfqpoint{1.073261in}{3.010660in}}%
\pgfpathcurveto{\pgfqpoint{1.069355in}{3.014567in}}{\pgfqpoint{1.064055in}{3.016762in}}{\pgfqpoint{1.058530in}{3.016762in}}%
\pgfpathcurveto{\pgfqpoint{1.053005in}{3.016762in}}{\pgfqpoint{1.047706in}{3.014567in}}{\pgfqpoint{1.043799in}{3.010660in}}%
\pgfpathcurveto{\pgfqpoint{1.039892in}{3.006754in}}{\pgfqpoint{1.037697in}{3.001454in}}{\pgfqpoint{1.037697in}{2.995929in}}%
\pgfpathcurveto{\pgfqpoint{1.037697in}{2.990404in}}{\pgfqpoint{1.039892in}{2.985104in}}{\pgfqpoint{1.043799in}{2.981198in}}%
\pgfpathcurveto{\pgfqpoint{1.047706in}{2.977291in}}{\pgfqpoint{1.053005in}{2.975096in}}{\pgfqpoint{1.058530in}{2.975096in}}%
\pgfpathclose%
\pgfusepath{fill}%
\end{pgfscope}%
\begin{pgfscope}%
\pgfpathrectangle{\pgfqpoint{0.833185in}{2.343185in}}{\pgfqpoint{1.162500in}{0.755000in}} %
\pgfusepath{clip}%
\pgfsetbuttcap%
\pgfsetroundjoin%
\definecolor{currentfill}{rgb}{0.000000,0.000000,0.000000}%
\pgfsetfillcolor{currentfill}%
\pgfsetfillopacity{0.500000}%
\pgfsetlinewidth{0.000000pt}%
\definecolor{currentstroke}{rgb}{0.000000,0.000000,0.000000}%
\pgfsetstrokecolor{currentstroke}%
\pgfsetdash{}{0pt}%
\pgfpathmoveto{\pgfqpoint{1.520990in}{2.531911in}}%
\pgfpathcurveto{\pgfqpoint{1.526515in}{2.531911in}}{\pgfqpoint{1.531814in}{2.534106in}}{\pgfqpoint{1.535721in}{2.538013in}}%
\pgfpathcurveto{\pgfqpoint{1.539628in}{2.541920in}}{\pgfqpoint{1.541823in}{2.547220in}}{\pgfqpoint{1.541823in}{2.552745in}}%
\pgfpathcurveto{\pgfqpoint{1.541823in}{2.558270in}}{\pgfqpoint{1.539628in}{2.563569in}}{\pgfqpoint{1.535721in}{2.567476in}}%
\pgfpathcurveto{\pgfqpoint{1.531814in}{2.571383in}}{\pgfqpoint{1.526515in}{2.573578in}}{\pgfqpoint{1.520990in}{2.573578in}}%
\pgfpathcurveto{\pgfqpoint{1.515465in}{2.573578in}}{\pgfqpoint{1.510165in}{2.571383in}}{\pgfqpoint{1.506258in}{2.567476in}}%
\pgfpathcurveto{\pgfqpoint{1.502352in}{2.563569in}}{\pgfqpoint{1.500156in}{2.558270in}}{\pgfqpoint{1.500156in}{2.552745in}}%
\pgfpathcurveto{\pgfqpoint{1.500156in}{2.547220in}}{\pgfqpoint{1.502352in}{2.541920in}}{\pgfqpoint{1.506258in}{2.538013in}}%
\pgfpathcurveto{\pgfqpoint{1.510165in}{2.534106in}}{\pgfqpoint{1.515465in}{2.531911in}}{\pgfqpoint{1.520990in}{2.531911in}}%
\pgfpathclose%
\pgfusepath{fill}%
\end{pgfscope}%
\begin{pgfscope}%
\pgfpathrectangle{\pgfqpoint{0.833185in}{2.343185in}}{\pgfqpoint{1.162500in}{0.755000in}} %
\pgfusepath{clip}%
\pgfsetbuttcap%
\pgfsetroundjoin%
\definecolor{currentfill}{rgb}{0.000000,0.000000,0.000000}%
\pgfsetfillcolor{currentfill}%
\pgfsetfillopacity{0.500000}%
\pgfsetlinewidth{0.000000pt}%
\definecolor{currentstroke}{rgb}{0.000000,0.000000,0.000000}%
\pgfsetstrokecolor{currentstroke}%
\pgfsetdash{}{0pt}%
\pgfpathmoveto{\pgfqpoint{1.402203in}{2.591601in}}%
\pgfpathcurveto{\pgfqpoint{1.407728in}{2.591601in}}{\pgfqpoint{1.413027in}{2.593796in}}{\pgfqpoint{1.416934in}{2.597703in}}%
\pgfpathcurveto{\pgfqpoint{1.420841in}{2.601609in}}{\pgfqpoint{1.423036in}{2.606909in}}{\pgfqpoint{1.423036in}{2.612434in}}%
\pgfpathcurveto{\pgfqpoint{1.423036in}{2.617959in}}{\pgfqpoint{1.420841in}{2.623259in}}{\pgfqpoint{1.416934in}{2.627165in}}%
\pgfpathcurveto{\pgfqpoint{1.413027in}{2.631072in}}{\pgfqpoint{1.407728in}{2.633267in}}{\pgfqpoint{1.402203in}{2.633267in}}%
\pgfpathcurveto{\pgfqpoint{1.396677in}{2.633267in}}{\pgfqpoint{1.391378in}{2.631072in}}{\pgfqpoint{1.387471in}{2.627165in}}%
\pgfpathcurveto{\pgfqpoint{1.383564in}{2.623259in}}{\pgfqpoint{1.381369in}{2.617959in}}{\pgfqpoint{1.381369in}{2.612434in}}%
\pgfpathcurveto{\pgfqpoint{1.381369in}{2.606909in}}{\pgfqpoint{1.383564in}{2.601609in}}{\pgfqpoint{1.387471in}{2.597703in}}%
\pgfpathcurveto{\pgfqpoint{1.391378in}{2.593796in}}{\pgfqpoint{1.396677in}{2.591601in}}{\pgfqpoint{1.402203in}{2.591601in}}%
\pgfpathclose%
\pgfusepath{fill}%
\end{pgfscope}%
\begin{pgfscope}%
\pgfpathrectangle{\pgfqpoint{0.833185in}{2.343185in}}{\pgfqpoint{1.162500in}{0.755000in}} %
\pgfusepath{clip}%
\pgfsetbuttcap%
\pgfsetroundjoin%
\definecolor{currentfill}{rgb}{0.000000,0.000000,0.000000}%
\pgfsetfillcolor{currentfill}%
\pgfsetfillopacity{0.500000}%
\pgfsetlinewidth{0.000000pt}%
\definecolor{currentstroke}{rgb}{0.000000,0.000000,0.000000}%
\pgfsetstrokecolor{currentstroke}%
\pgfsetdash{}{0pt}%
\pgfpathmoveto{\pgfqpoint{1.053510in}{2.814070in}}%
\pgfpathcurveto{\pgfqpoint{1.059035in}{2.814070in}}{\pgfqpoint{1.064335in}{2.816265in}}{\pgfqpoint{1.068242in}{2.820172in}}%
\pgfpathcurveto{\pgfqpoint{1.072148in}{2.824079in}}{\pgfqpoint{1.074344in}{2.829378in}}{\pgfqpoint{1.074344in}{2.834904in}}%
\pgfpathcurveto{\pgfqpoint{1.074344in}{2.840429in}}{\pgfqpoint{1.072148in}{2.845728in}}{\pgfqpoint{1.068242in}{2.849635in}}%
\pgfpathcurveto{\pgfqpoint{1.064335in}{2.853542in}}{\pgfqpoint{1.059035in}{2.855737in}}{\pgfqpoint{1.053510in}{2.855737in}}%
\pgfpathcurveto{\pgfqpoint{1.047985in}{2.855737in}}{\pgfqpoint{1.042686in}{2.853542in}}{\pgfqpoint{1.038779in}{2.849635in}}%
\pgfpathcurveto{\pgfqpoint{1.034872in}{2.845728in}}{\pgfqpoint{1.032677in}{2.840429in}}{\pgfqpoint{1.032677in}{2.834904in}}%
\pgfpathcurveto{\pgfqpoint{1.032677in}{2.829378in}}{\pgfqpoint{1.034872in}{2.824079in}}{\pgfqpoint{1.038779in}{2.820172in}}%
\pgfpathcurveto{\pgfqpoint{1.042686in}{2.816265in}}{\pgfqpoint{1.047985in}{2.814070in}}{\pgfqpoint{1.053510in}{2.814070in}}%
\pgfpathclose%
\pgfusepath{fill}%
\end{pgfscope}%
\begin{pgfscope}%
\pgfpathrectangle{\pgfqpoint{0.833185in}{2.343185in}}{\pgfqpoint{1.162500in}{0.755000in}} %
\pgfusepath{clip}%
\pgfsetbuttcap%
\pgfsetroundjoin%
\definecolor{currentfill}{rgb}{0.000000,0.000000,0.000000}%
\pgfsetfillcolor{currentfill}%
\pgfsetfillopacity{0.500000}%
\pgfsetlinewidth{0.000000pt}%
\definecolor{currentstroke}{rgb}{0.000000,0.000000,0.000000}%
\pgfsetstrokecolor{currentstroke}%
\pgfsetdash{}{0pt}%
\pgfpathmoveto{\pgfqpoint{1.492358in}{2.392267in}}%
\pgfpathcurveto{\pgfqpoint{1.497883in}{2.392267in}}{\pgfqpoint{1.503182in}{2.394462in}}{\pgfqpoint{1.507089in}{2.398369in}}%
\pgfpathcurveto{\pgfqpoint{1.510996in}{2.402276in}}{\pgfqpoint{1.513191in}{2.407575in}}{\pgfqpoint{1.513191in}{2.413101in}}%
\pgfpathcurveto{\pgfqpoint{1.513191in}{2.418626in}}{\pgfqpoint{1.510996in}{2.423925in}}{\pgfqpoint{1.507089in}{2.427832in}}%
\pgfpathcurveto{\pgfqpoint{1.503182in}{2.431739in}}{\pgfqpoint{1.497883in}{2.433934in}}{\pgfqpoint{1.492358in}{2.433934in}}%
\pgfpathcurveto{\pgfqpoint{1.486832in}{2.433934in}}{\pgfqpoint{1.481533in}{2.431739in}}{\pgfqpoint{1.477626in}{2.427832in}}%
\pgfpathcurveto{\pgfqpoint{1.473719in}{2.423925in}}{\pgfqpoint{1.471524in}{2.418626in}}{\pgfqpoint{1.471524in}{2.413101in}}%
\pgfpathcurveto{\pgfqpoint{1.471524in}{2.407575in}}{\pgfqpoint{1.473719in}{2.402276in}}{\pgfqpoint{1.477626in}{2.398369in}}%
\pgfpathcurveto{\pgfqpoint{1.481533in}{2.394462in}}{\pgfqpoint{1.486832in}{2.392267in}}{\pgfqpoint{1.492358in}{2.392267in}}%
\pgfpathclose%
\pgfusepath{fill}%
\end{pgfscope}%
\begin{pgfscope}%
\pgfsetbuttcap%
\pgfsetroundjoin%
\definecolor{currentfill}{rgb}{0.000000,0.000000,0.000000}%
\pgfsetfillcolor{currentfill}%
\pgfsetlinewidth{0.803000pt}%
\definecolor{currentstroke}{rgb}{0.000000,0.000000,0.000000}%
\pgfsetstrokecolor{currentstroke}%
\pgfsetdash{}{0pt}%
\pgfsys@defobject{currentmarker}{\pgfqpoint{-0.048611in}{0.000000in}}{\pgfqpoint{0.000000in}{0.000000in}}{%
\pgfpathmoveto{\pgfqpoint{0.000000in}{0.000000in}}%
\pgfpathlineto{\pgfqpoint{-0.048611in}{0.000000in}}%
\pgfusepath{stroke,fill}%
}%
\begin{pgfscope}%
\pgfsys@transformshift{0.833185in}{2.654426in}%
\pgfsys@useobject{currentmarker}{}%
\end{pgfscope}%
\end{pgfscope}%
\begin{pgfscope}%
\pgftext[x=0.289968in,y=2.612217in,left,base]{\rmfamily\fontsize{8.000000}{9.600000}\selectfont \(\displaystyle 0.000025\)}%
\end{pgfscope}%
\begin{pgfscope}%
\pgfsetbuttcap%
\pgfsetroundjoin%
\definecolor{currentfill}{rgb}{0.000000,0.000000,0.000000}%
\pgfsetfillcolor{currentfill}%
\pgfsetlinewidth{0.803000pt}%
\definecolor{currentstroke}{rgb}{0.000000,0.000000,0.000000}%
\pgfsetstrokecolor{currentstroke}%
\pgfsetdash{}{0pt}%
\pgfsys@defobject{currentmarker}{\pgfqpoint{-0.048611in}{0.000000in}}{\pgfqpoint{0.000000in}{0.000000in}}{%
\pgfpathmoveto{\pgfqpoint{0.000000in}{0.000000in}}%
\pgfpathlineto{\pgfqpoint{-0.048611in}{0.000000in}}%
\pgfusepath{stroke,fill}%
}%
\begin{pgfscope}%
\pgfsys@transformshift{0.833185in}{3.044353in}%
\pgfsys@useobject{currentmarker}{}%
\end{pgfscope}%
\end{pgfscope}%
\begin{pgfscope}%
\pgftext[x=0.289968in,y=3.002144in,left,base]{\rmfamily\fontsize{8.000000}{9.600000}\selectfont \(\displaystyle 0.000050\)}%
\end{pgfscope}%
\begin{pgfscope}%
\pgftext[x=0.234413in,y=2.720685in,,bottom,rotate=90.000000]{\rmfamily\fontsize{10.000000}{12.000000}\selectfont area}%
\end{pgfscope}%
\begin{pgfscope}%
\pgfsetrectcap%
\pgfsetmiterjoin%
\pgfsetlinewidth{0.803000pt}%
\definecolor{currentstroke}{rgb}{0.000000,0.000000,0.000000}%
\pgfsetstrokecolor{currentstroke}%
\pgfsetdash{}{0pt}%
\pgfpathmoveto{\pgfqpoint{0.833185in}{2.343185in}}%
\pgfpathlineto{\pgfqpoint{0.833185in}{3.098185in}}%
\pgfusepath{stroke}%
\end{pgfscope}%
\begin{pgfscope}%
\pgfsetrectcap%
\pgfsetmiterjoin%
\pgfsetlinewidth{0.803000pt}%
\definecolor{currentstroke}{rgb}{0.000000,0.000000,0.000000}%
\pgfsetstrokecolor{currentstroke}%
\pgfsetdash{}{0pt}%
\pgfpathmoveto{\pgfqpoint{1.995685in}{2.343185in}}%
\pgfpathlineto{\pgfqpoint{1.995685in}{3.098185in}}%
\pgfusepath{stroke}%
\end{pgfscope}%
\begin{pgfscope}%
\pgfsetrectcap%
\pgfsetmiterjoin%
\pgfsetlinewidth{0.803000pt}%
\definecolor{currentstroke}{rgb}{0.000000,0.000000,0.000000}%
\pgfsetstrokecolor{currentstroke}%
\pgfsetdash{}{0pt}%
\pgfpathmoveto{\pgfqpoint{0.833185in}{2.343185in}}%
\pgfpathlineto{\pgfqpoint{1.995685in}{2.343185in}}%
\pgfusepath{stroke}%
\end{pgfscope}%
\begin{pgfscope}%
\pgfsetrectcap%
\pgfsetmiterjoin%
\pgfsetlinewidth{0.803000pt}%
\definecolor{currentstroke}{rgb}{0.000000,0.000000,0.000000}%
\pgfsetstrokecolor{currentstroke}%
\pgfsetdash{}{0pt}%
\pgfpathmoveto{\pgfqpoint{0.833185in}{3.098185in}}%
\pgfpathlineto{\pgfqpoint{1.995685in}{3.098185in}}%
\pgfusepath{stroke}%
\end{pgfscope}%
\begin{pgfscope}%
\pgfsetbuttcap%
\pgfsetmiterjoin%
\definecolor{currentfill}{rgb}{1.000000,1.000000,1.000000}%
\pgfsetfillcolor{currentfill}%
\pgfsetlinewidth{0.000000pt}%
\definecolor{currentstroke}{rgb}{0.000000,0.000000,0.000000}%
\pgfsetstrokecolor{currentstroke}%
\pgfsetstrokeopacity{0.000000}%
\pgfsetdash{}{0pt}%
\pgfpathmoveto{\pgfqpoint{1.995685in}{2.343185in}}%
\pgfpathlineto{\pgfqpoint{3.158185in}{2.343185in}}%
\pgfpathlineto{\pgfqpoint{3.158185in}{3.098185in}}%
\pgfpathlineto{\pgfqpoint{1.995685in}{3.098185in}}%
\pgfpathclose%
\pgfusepath{fill}%
\end{pgfscope}%
\begin{pgfscope}%
\pgfpathrectangle{\pgfqpoint{1.995685in}{2.343185in}}{\pgfqpoint{1.162500in}{0.755000in}} %
\pgfusepath{clip}%
\pgfsetrectcap%
\pgfsetroundjoin%
\pgfsetlinewidth{1.505625pt}%
\definecolor{currentstroke}{rgb}{0.121569,0.466667,0.705882}%
\pgfsetstrokecolor{currentstroke}%
\pgfsetdash{}{0pt}%
\pgfpathmoveto{\pgfqpoint{2.023363in}{2.636992in}}%
\pgfpathlineto{\pgfqpoint{2.061044in}{2.733201in}}%
\pgfpathlineto{\pgfqpoint{2.090966in}{2.803557in}}%
\pgfpathlineto{\pgfqpoint{2.116456in}{2.858016in}}%
\pgfpathlineto{\pgfqpoint{2.139730in}{2.902658in}}%
\pgfpathlineto{\pgfqpoint{2.160786in}{2.938493in}}%
\pgfpathlineto{\pgfqpoint{2.180735in}{2.968245in}}%
\pgfpathlineto{\pgfqpoint{2.199575in}{2.992499in}}%
\pgfpathlineto{\pgfqpoint{2.217307in}{3.011894in}}%
\pgfpathlineto{\pgfqpoint{2.235039in}{3.027999in}}%
\pgfpathlineto{\pgfqpoint{2.251663in}{3.040184in}}%
\pgfpathlineto{\pgfqpoint{2.268287in}{3.049660in}}%
\pgfpathlineto{\pgfqpoint{2.284910in}{3.056569in}}%
\pgfpathlineto{\pgfqpoint{2.301534in}{3.061086in}}%
\pgfpathlineto{\pgfqpoint{2.319266in}{3.063499in}}%
\pgfpathlineto{\pgfqpoint{2.336998in}{3.063703in}}%
\pgfpathlineto{\pgfqpoint{2.356947in}{3.061671in}}%
\pgfpathlineto{\pgfqpoint{2.379112in}{3.057152in}}%
\pgfpathlineto{\pgfqpoint{2.406818in}{3.049123in}}%
\pgfpathlineto{\pgfqpoint{2.455581in}{3.032151in}}%
\pgfpathlineto{\pgfqpoint{2.499911in}{3.017708in}}%
\pgfpathlineto{\pgfqpoint{2.532050in}{3.009550in}}%
\pgfpathlineto{\pgfqpoint{2.563081in}{3.003938in}}%
\pgfpathlineto{\pgfqpoint{2.598545in}{2.999866in}}%
\pgfpathlineto{\pgfqpoint{2.687206in}{2.990950in}}%
\pgfpathlineto{\pgfqpoint{2.711587in}{2.985549in}}%
\pgfpathlineto{\pgfqpoint{2.733752in}{2.978361in}}%
\pgfpathlineto{\pgfqpoint{2.753701in}{2.969630in}}%
\pgfpathlineto{\pgfqpoint{2.773649in}{2.958460in}}%
\pgfpathlineto{\pgfqpoint{2.793598in}{2.944633in}}%
\pgfpathlineto{\pgfqpoint{2.813546in}{2.928005in}}%
\pgfpathlineto{\pgfqpoint{2.833495in}{2.908505in}}%
\pgfpathlineto{\pgfqpoint{2.854551in}{2.884806in}}%
\pgfpathlineto{\pgfqpoint{2.876717in}{2.856493in}}%
\pgfpathlineto{\pgfqpoint{2.901098in}{2.821570in}}%
\pgfpathlineto{\pgfqpoint{2.927696in}{2.779342in}}%
\pgfpathlineto{\pgfqpoint{2.956511in}{2.729334in}}%
\pgfpathlineto{\pgfqpoint{2.989758in}{2.667097in}}%
\pgfpathlineto{\pgfqpoint{3.030763in}{2.585401in}}%
\pgfpathlineto{\pgfqpoint{3.095042in}{2.451609in}}%
\pgfpathlineto{\pgfqpoint{3.130506in}{2.377503in}}%
\pgfpathlineto{\pgfqpoint{3.130506in}{2.377503in}}%
\pgfusepath{stroke}%
\end{pgfscope}%
\begin{pgfscope}%
\pgfsetrectcap%
\pgfsetmiterjoin%
\pgfsetlinewidth{0.803000pt}%
\definecolor{currentstroke}{rgb}{0.000000,0.000000,0.000000}%
\pgfsetstrokecolor{currentstroke}%
\pgfsetdash{}{0pt}%
\pgfpathmoveto{\pgfqpoint{1.995685in}{2.343185in}}%
\pgfpathlineto{\pgfqpoint{1.995685in}{3.098185in}}%
\pgfusepath{stroke}%
\end{pgfscope}%
\begin{pgfscope}%
\pgfsetrectcap%
\pgfsetmiterjoin%
\pgfsetlinewidth{0.803000pt}%
\definecolor{currentstroke}{rgb}{0.000000,0.000000,0.000000}%
\pgfsetstrokecolor{currentstroke}%
\pgfsetdash{}{0pt}%
\pgfpathmoveto{\pgfqpoint{3.158185in}{2.343185in}}%
\pgfpathlineto{\pgfqpoint{3.158185in}{3.098185in}}%
\pgfusepath{stroke}%
\end{pgfscope}%
\begin{pgfscope}%
\pgfsetrectcap%
\pgfsetmiterjoin%
\pgfsetlinewidth{0.803000pt}%
\definecolor{currentstroke}{rgb}{0.000000,0.000000,0.000000}%
\pgfsetstrokecolor{currentstroke}%
\pgfsetdash{}{0pt}%
\pgfpathmoveto{\pgfqpoint{1.995685in}{2.343185in}}%
\pgfpathlineto{\pgfqpoint{3.158185in}{2.343185in}}%
\pgfusepath{stroke}%
\end{pgfscope}%
\begin{pgfscope}%
\pgfsetrectcap%
\pgfsetmiterjoin%
\pgfsetlinewidth{0.803000pt}%
\definecolor{currentstroke}{rgb}{0.000000,0.000000,0.000000}%
\pgfsetstrokecolor{currentstroke}%
\pgfsetdash{}{0pt}%
\pgfpathmoveto{\pgfqpoint{1.995685in}{3.098185in}}%
\pgfpathlineto{\pgfqpoint{3.158185in}{3.098185in}}%
\pgfusepath{stroke}%
\end{pgfscope}%
\begin{pgfscope}%
\pgfsetbuttcap%
\pgfsetmiterjoin%
\definecolor{currentfill}{rgb}{1.000000,1.000000,1.000000}%
\pgfsetfillcolor{currentfill}%
\pgfsetlinewidth{0.000000pt}%
\definecolor{currentstroke}{rgb}{0.000000,0.000000,0.000000}%
\pgfsetstrokecolor{currentstroke}%
\pgfsetstrokeopacity{0.000000}%
\pgfsetdash{}{0pt}%
\pgfpathmoveto{\pgfqpoint{3.158185in}{2.343185in}}%
\pgfpathlineto{\pgfqpoint{4.320685in}{2.343185in}}%
\pgfpathlineto{\pgfqpoint{4.320685in}{3.098185in}}%
\pgfpathlineto{\pgfqpoint{3.158185in}{3.098185in}}%
\pgfpathclose%
\pgfusepath{fill}%
\end{pgfscope}%
\begin{pgfscope}%
\pgfpathrectangle{\pgfqpoint{3.158185in}{2.343185in}}{\pgfqpoint{1.162500in}{0.755000in}} %
\pgfusepath{clip}%
\pgfsetbuttcap%
\pgfsetroundjoin%
\definecolor{currentfill}{rgb}{0.000000,0.000000,0.000000}%
\pgfsetfillcolor{currentfill}%
\pgfsetfillopacity{0.500000}%
\pgfsetlinewidth{0.000000pt}%
\definecolor{currentstroke}{rgb}{0.000000,0.000000,0.000000}%
\pgfsetstrokecolor{currentstroke}%
\pgfsetdash{}{0pt}%
\pgfpathmoveto{\pgfqpoint{4.293006in}{3.059375in}}%
\pgfpathcurveto{\pgfqpoint{4.298531in}{3.059375in}}{\pgfqpoint{4.303831in}{3.061570in}}{\pgfqpoint{4.307737in}{3.065477in}}%
\pgfpathcurveto{\pgfqpoint{4.311644in}{3.069384in}}{\pgfqpoint{4.313839in}{3.074683in}}{\pgfqpoint{4.313839in}{3.080208in}}%
\pgfpathcurveto{\pgfqpoint{4.313839in}{3.085733in}}{\pgfqpoint{4.311644in}{3.091033in}}{\pgfqpoint{4.307737in}{3.094940in}}%
\pgfpathcurveto{\pgfqpoint{4.303831in}{3.098847in}}{\pgfqpoint{4.298531in}{3.101042in}}{\pgfqpoint{4.293006in}{3.101042in}}%
\pgfpathcurveto{\pgfqpoint{4.287481in}{3.101042in}}{\pgfqpoint{4.282181in}{3.098847in}}{\pgfqpoint{4.278275in}{3.094940in}}%
\pgfpathcurveto{\pgfqpoint{4.274368in}{3.091033in}}{\pgfqpoint{4.272173in}{3.085733in}}{\pgfqpoint{4.272173in}{3.080208in}}%
\pgfpathcurveto{\pgfqpoint{4.272173in}{3.074683in}}{\pgfqpoint{4.274368in}{3.069384in}}{\pgfqpoint{4.278275in}{3.065477in}}%
\pgfpathcurveto{\pgfqpoint{4.282181in}{3.061570in}}{\pgfqpoint{4.287481in}{3.059375in}}{\pgfqpoint{4.293006in}{3.059375in}}%
\pgfpathclose%
\pgfusepath{fill}%
\end{pgfscope}%
\begin{pgfscope}%
\pgfpathrectangle{\pgfqpoint{3.158185in}{2.343185in}}{\pgfqpoint{1.162500in}{0.755000in}} %
\pgfusepath{clip}%
\pgfsetbuttcap%
\pgfsetroundjoin%
\definecolor{currentfill}{rgb}{0.000000,0.000000,0.000000}%
\pgfsetfillcolor{currentfill}%
\pgfsetfillopacity{0.500000}%
\pgfsetlinewidth{0.000000pt}%
\definecolor{currentstroke}{rgb}{0.000000,0.000000,0.000000}%
\pgfsetstrokecolor{currentstroke}%
\pgfsetdash{}{0pt}%
\pgfpathmoveto{\pgfqpoint{3.586831in}{2.815854in}}%
\pgfpathcurveto{\pgfqpoint{3.592356in}{2.815854in}}{\pgfqpoint{3.597655in}{2.818049in}}{\pgfqpoint{3.601562in}{2.821955in}}%
\pgfpathcurveto{\pgfqpoint{3.605469in}{2.825862in}}{\pgfqpoint{3.607664in}{2.831162in}}{\pgfqpoint{3.607664in}{2.836687in}}%
\pgfpathcurveto{\pgfqpoint{3.607664in}{2.842212in}}{\pgfqpoint{3.605469in}{2.847511in}}{\pgfqpoint{3.601562in}{2.851418in}}%
\pgfpathcurveto{\pgfqpoint{3.597655in}{2.855325in}}{\pgfqpoint{3.592356in}{2.857520in}}{\pgfqpoint{3.586831in}{2.857520in}}%
\pgfpathcurveto{\pgfqpoint{3.581306in}{2.857520in}}{\pgfqpoint{3.576006in}{2.855325in}}{\pgfqpoint{3.572099in}{2.851418in}}%
\pgfpathcurveto{\pgfqpoint{3.568193in}{2.847511in}}{\pgfqpoint{3.565998in}{2.842212in}}{\pgfqpoint{3.565998in}{2.836687in}}%
\pgfpathcurveto{\pgfqpoint{3.565998in}{2.831162in}}{\pgfqpoint{3.568193in}{2.825862in}}{\pgfqpoint{3.572099in}{2.821955in}}%
\pgfpathcurveto{\pgfqpoint{3.576006in}{2.818049in}}{\pgfqpoint{3.581306in}{2.815854in}}{\pgfqpoint{3.586831in}{2.815854in}}%
\pgfpathclose%
\pgfusepath{fill}%
\end{pgfscope}%
\begin{pgfscope}%
\pgfpathrectangle{\pgfqpoint{3.158185in}{2.343185in}}{\pgfqpoint{1.162500in}{0.755000in}} %
\pgfusepath{clip}%
\pgfsetbuttcap%
\pgfsetroundjoin%
\definecolor{currentfill}{rgb}{0.000000,0.000000,0.000000}%
\pgfsetfillcolor{currentfill}%
\pgfsetfillopacity{0.500000}%
\pgfsetlinewidth{0.000000pt}%
\definecolor{currentstroke}{rgb}{0.000000,0.000000,0.000000}%
\pgfsetstrokecolor{currentstroke}%
\pgfsetdash{}{0pt}%
\pgfpathmoveto{\pgfqpoint{3.578870in}{2.815581in}}%
\pgfpathcurveto{\pgfqpoint{3.584395in}{2.815581in}}{\pgfqpoint{3.589694in}{2.817776in}}{\pgfqpoint{3.593601in}{2.821682in}}%
\pgfpathcurveto{\pgfqpoint{3.597508in}{2.825589in}}{\pgfqpoint{3.599703in}{2.830889in}}{\pgfqpoint{3.599703in}{2.836414in}}%
\pgfpathcurveto{\pgfqpoint{3.599703in}{2.841939in}}{\pgfqpoint{3.597508in}{2.847238in}}{\pgfqpoint{3.593601in}{2.851145in}}%
\pgfpathcurveto{\pgfqpoint{3.589694in}{2.855052in}}{\pgfqpoint{3.584395in}{2.857247in}}{\pgfqpoint{3.578870in}{2.857247in}}%
\pgfpathcurveto{\pgfqpoint{3.573345in}{2.857247in}}{\pgfqpoint{3.568045in}{2.855052in}}{\pgfqpoint{3.564138in}{2.851145in}}%
\pgfpathcurveto{\pgfqpoint{3.560231in}{2.847238in}}{\pgfqpoint{3.558036in}{2.841939in}}{\pgfqpoint{3.558036in}{2.836414in}}%
\pgfpathcurveto{\pgfqpoint{3.558036in}{2.830889in}}{\pgfqpoint{3.560231in}{2.825589in}}{\pgfqpoint{3.564138in}{2.821682in}}%
\pgfpathcurveto{\pgfqpoint{3.568045in}{2.817776in}}{\pgfqpoint{3.573345in}{2.815581in}}{\pgfqpoint{3.578870in}{2.815581in}}%
\pgfpathclose%
\pgfusepath{fill}%
\end{pgfscope}%
\begin{pgfscope}%
\pgfpathrectangle{\pgfqpoint{3.158185in}{2.343185in}}{\pgfqpoint{1.162500in}{0.755000in}} %
\pgfusepath{clip}%
\pgfsetbuttcap%
\pgfsetroundjoin%
\definecolor{currentfill}{rgb}{0.000000,0.000000,0.000000}%
\pgfsetfillcolor{currentfill}%
\pgfsetfillopacity{0.500000}%
\pgfsetlinewidth{0.000000pt}%
\definecolor{currentstroke}{rgb}{0.000000,0.000000,0.000000}%
\pgfsetstrokecolor{currentstroke}%
\pgfsetdash{}{0pt}%
\pgfpathmoveto{\pgfqpoint{3.333069in}{2.531108in}}%
\pgfpathcurveto{\pgfqpoint{3.338594in}{2.531108in}}{\pgfqpoint{3.343894in}{2.533303in}}{\pgfqpoint{3.347801in}{2.537210in}}%
\pgfpathcurveto{\pgfqpoint{3.351708in}{2.541116in}}{\pgfqpoint{3.353903in}{2.546416in}}{\pgfqpoint{3.353903in}{2.551941in}}%
\pgfpathcurveto{\pgfqpoint{3.353903in}{2.557466in}}{\pgfqpoint{3.351708in}{2.562766in}}{\pgfqpoint{3.347801in}{2.566672in}}%
\pgfpathcurveto{\pgfqpoint{3.343894in}{2.570579in}}{\pgfqpoint{3.338594in}{2.572774in}}{\pgfqpoint{3.333069in}{2.572774in}}%
\pgfpathcurveto{\pgfqpoint{3.327544in}{2.572774in}}{\pgfqpoint{3.322245in}{2.570579in}}{\pgfqpoint{3.318338in}{2.566672in}}%
\pgfpathcurveto{\pgfqpoint{3.314431in}{2.562766in}}{\pgfqpoint{3.312236in}{2.557466in}}{\pgfqpoint{3.312236in}{2.551941in}}%
\pgfpathcurveto{\pgfqpoint{3.312236in}{2.546416in}}{\pgfqpoint{3.314431in}{2.541116in}}{\pgfqpoint{3.318338in}{2.537210in}}%
\pgfpathcurveto{\pgfqpoint{3.322245in}{2.533303in}}{\pgfqpoint{3.327544in}{2.531108in}}{\pgfqpoint{3.333069in}{2.531108in}}%
\pgfpathclose%
\pgfusepath{fill}%
\end{pgfscope}%
\begin{pgfscope}%
\pgfpathrectangle{\pgfqpoint{3.158185in}{2.343185in}}{\pgfqpoint{1.162500in}{0.755000in}} %
\pgfusepath{clip}%
\pgfsetbuttcap%
\pgfsetroundjoin%
\definecolor{currentfill}{rgb}{0.000000,0.000000,0.000000}%
\pgfsetfillcolor{currentfill}%
\pgfsetfillopacity{0.500000}%
\pgfsetlinewidth{0.000000pt}%
\definecolor{currentstroke}{rgb}{0.000000,0.000000,0.000000}%
\pgfsetstrokecolor{currentstroke}%
\pgfsetdash{}{0pt}%
\pgfpathmoveto{\pgfqpoint{3.296013in}{2.481860in}}%
\pgfpathcurveto{\pgfqpoint{3.301538in}{2.481860in}}{\pgfqpoint{3.306838in}{2.484055in}}{\pgfqpoint{3.310744in}{2.487962in}}%
\pgfpathcurveto{\pgfqpoint{3.314651in}{2.491868in}}{\pgfqpoint{3.316846in}{2.497168in}}{\pgfqpoint{3.316846in}{2.502693in}}%
\pgfpathcurveto{\pgfqpoint{3.316846in}{2.508218in}}{\pgfqpoint{3.314651in}{2.513517in}}{\pgfqpoint{3.310744in}{2.517424in}}%
\pgfpathcurveto{\pgfqpoint{3.306838in}{2.521331in}}{\pgfqpoint{3.301538in}{2.523526in}}{\pgfqpoint{3.296013in}{2.523526in}}%
\pgfpathcurveto{\pgfqpoint{3.290488in}{2.523526in}}{\pgfqpoint{3.285188in}{2.521331in}}{\pgfqpoint{3.281282in}{2.517424in}}%
\pgfpathcurveto{\pgfqpoint{3.277375in}{2.513517in}}{\pgfqpoint{3.275180in}{2.508218in}}{\pgfqpoint{3.275180in}{2.502693in}}%
\pgfpathcurveto{\pgfqpoint{3.275180in}{2.497168in}}{\pgfqpoint{3.277375in}{2.491868in}}{\pgfqpoint{3.281282in}{2.487962in}}%
\pgfpathcurveto{\pgfqpoint{3.285188in}{2.484055in}}{\pgfqpoint{3.290488in}{2.481860in}}{\pgfqpoint{3.296013in}{2.481860in}}%
\pgfpathclose%
\pgfusepath{fill}%
\end{pgfscope}%
\begin{pgfscope}%
\pgfpathrectangle{\pgfqpoint{3.158185in}{2.343185in}}{\pgfqpoint{1.162500in}{0.755000in}} %
\pgfusepath{clip}%
\pgfsetbuttcap%
\pgfsetroundjoin%
\definecolor{currentfill}{rgb}{0.000000,0.000000,0.000000}%
\pgfsetfillcolor{currentfill}%
\pgfsetfillopacity{0.500000}%
\pgfsetlinewidth{0.000000pt}%
\definecolor{currentstroke}{rgb}{0.000000,0.000000,0.000000}%
\pgfsetstrokecolor{currentstroke}%
\pgfsetdash{}{0pt}%
\pgfpathmoveto{\pgfqpoint{3.206735in}{2.394040in}}%
\pgfpathcurveto{\pgfqpoint{3.212260in}{2.394040in}}{\pgfqpoint{3.217559in}{2.396235in}}{\pgfqpoint{3.221466in}{2.400142in}}%
\pgfpathcurveto{\pgfqpoint{3.225373in}{2.404048in}}{\pgfqpoint{3.227568in}{2.409348in}}{\pgfqpoint{3.227568in}{2.414873in}}%
\pgfpathcurveto{\pgfqpoint{3.227568in}{2.420398in}}{\pgfqpoint{3.225373in}{2.425698in}}{\pgfqpoint{3.221466in}{2.429604in}}%
\pgfpathcurveto{\pgfqpoint{3.217559in}{2.433511in}}{\pgfqpoint{3.212260in}{2.435706in}}{\pgfqpoint{3.206735in}{2.435706in}}%
\pgfpathcurveto{\pgfqpoint{3.201210in}{2.435706in}}{\pgfqpoint{3.195910in}{2.433511in}}{\pgfqpoint{3.192003in}{2.429604in}}%
\pgfpathcurveto{\pgfqpoint{3.188097in}{2.425698in}}{\pgfqpoint{3.185901in}{2.420398in}}{\pgfqpoint{3.185901in}{2.414873in}}%
\pgfpathcurveto{\pgfqpoint{3.185901in}{2.409348in}}{\pgfqpoint{3.188097in}{2.404048in}}{\pgfqpoint{3.192003in}{2.400142in}}%
\pgfpathcurveto{\pgfqpoint{3.195910in}{2.396235in}}{\pgfqpoint{3.201210in}{2.394040in}}{\pgfqpoint{3.206735in}{2.394040in}}%
\pgfpathclose%
\pgfusepath{fill}%
\end{pgfscope}%
\begin{pgfscope}%
\pgfpathrectangle{\pgfqpoint{3.158185in}{2.343185in}}{\pgfqpoint{1.162500in}{0.755000in}} %
\pgfusepath{clip}%
\pgfsetbuttcap%
\pgfsetroundjoin%
\definecolor{currentfill}{rgb}{0.000000,0.000000,0.000000}%
\pgfsetfillcolor{currentfill}%
\pgfsetfillopacity{0.500000}%
\pgfsetlinewidth{0.000000pt}%
\definecolor{currentstroke}{rgb}{0.000000,0.000000,0.000000}%
\pgfsetstrokecolor{currentstroke}%
\pgfsetdash{}{0pt}%
\pgfpathmoveto{\pgfqpoint{3.185863in}{2.340327in}}%
\pgfpathcurveto{\pgfqpoint{3.191388in}{2.340327in}}{\pgfqpoint{3.196688in}{2.342523in}}{\pgfqpoint{3.200595in}{2.346429in}}%
\pgfpathcurveto{\pgfqpoint{3.204501in}{2.350336in}}{\pgfqpoint{3.206696in}{2.355636in}}{\pgfqpoint{3.206696in}{2.361161in}}%
\pgfpathcurveto{\pgfqpoint{3.206696in}{2.366686in}}{\pgfqpoint{3.204501in}{2.371985in}}{\pgfqpoint{3.200595in}{2.375892in}}%
\pgfpathcurveto{\pgfqpoint{3.196688in}{2.379799in}}{\pgfqpoint{3.191388in}{2.381994in}}{\pgfqpoint{3.185863in}{2.381994in}}%
\pgfpathcurveto{\pgfqpoint{3.180338in}{2.381994in}}{\pgfqpoint{3.175039in}{2.379799in}}{\pgfqpoint{3.171132in}{2.375892in}}%
\pgfpathcurveto{\pgfqpoint{3.167225in}{2.371985in}}{\pgfqpoint{3.165030in}{2.366686in}}{\pgfqpoint{3.165030in}{2.361161in}}%
\pgfpathcurveto{\pgfqpoint{3.165030in}{2.355636in}}{\pgfqpoint{3.167225in}{2.350336in}}{\pgfqpoint{3.171132in}{2.346429in}}%
\pgfpathcurveto{\pgfqpoint{3.175039in}{2.342523in}}{\pgfqpoint{3.180338in}{2.340327in}}{\pgfqpoint{3.185863in}{2.340327in}}%
\pgfpathclose%
\pgfusepath{fill}%
\end{pgfscope}%
\begin{pgfscope}%
\pgfpathrectangle{\pgfqpoint{3.158185in}{2.343185in}}{\pgfqpoint{1.162500in}{0.755000in}} %
\pgfusepath{clip}%
\pgfsetbuttcap%
\pgfsetroundjoin%
\definecolor{currentfill}{rgb}{0.000000,0.000000,0.000000}%
\pgfsetfillcolor{currentfill}%
\pgfsetfillopacity{0.500000}%
\pgfsetlinewidth{0.000000pt}%
\definecolor{currentstroke}{rgb}{0.000000,0.000000,0.000000}%
\pgfsetstrokecolor{currentstroke}%
\pgfsetdash{}{0pt}%
\pgfpathmoveto{\pgfqpoint{3.808562in}{2.914639in}}%
\pgfpathcurveto{\pgfqpoint{3.814087in}{2.914639in}}{\pgfqpoint{3.819386in}{2.916834in}}{\pgfqpoint{3.823293in}{2.920741in}}%
\pgfpathcurveto{\pgfqpoint{3.827200in}{2.924648in}}{\pgfqpoint{3.829395in}{2.929947in}}{\pgfqpoint{3.829395in}{2.935473in}}%
\pgfpathcurveto{\pgfqpoint{3.829395in}{2.940998in}}{\pgfqpoint{3.827200in}{2.946297in}}{\pgfqpoint{3.823293in}{2.950204in}}%
\pgfpathcurveto{\pgfqpoint{3.819386in}{2.954111in}}{\pgfqpoint{3.814087in}{2.956306in}}{\pgfqpoint{3.808562in}{2.956306in}}%
\pgfpathcurveto{\pgfqpoint{3.803037in}{2.956306in}}{\pgfqpoint{3.797737in}{2.954111in}}{\pgfqpoint{3.793830in}{2.950204in}}%
\pgfpathcurveto{\pgfqpoint{3.789924in}{2.946297in}}{\pgfqpoint{3.787728in}{2.940998in}}{\pgfqpoint{3.787728in}{2.935473in}}%
\pgfpathcurveto{\pgfqpoint{3.787728in}{2.929947in}}{\pgfqpoint{3.789924in}{2.924648in}}{\pgfqpoint{3.793830in}{2.920741in}}%
\pgfpathcurveto{\pgfqpoint{3.797737in}{2.916834in}}{\pgfqpoint{3.803037in}{2.914639in}}{\pgfqpoint{3.808562in}{2.914639in}}%
\pgfpathclose%
\pgfusepath{fill}%
\end{pgfscope}%
\begin{pgfscope}%
\pgfpathrectangle{\pgfqpoint{3.158185in}{2.343185in}}{\pgfqpoint{1.162500in}{0.755000in}} %
\pgfusepath{clip}%
\pgfsetbuttcap%
\pgfsetroundjoin%
\definecolor{currentfill}{rgb}{0.000000,0.000000,0.000000}%
\pgfsetfillcolor{currentfill}%
\pgfsetfillopacity{0.500000}%
\pgfsetlinewidth{0.000000pt}%
\definecolor{currentstroke}{rgb}{0.000000,0.000000,0.000000}%
\pgfsetstrokecolor{currentstroke}%
\pgfsetdash{}{0pt}%
\pgfpathmoveto{\pgfqpoint{3.620933in}{2.734550in}}%
\pgfpathcurveto{\pgfqpoint{3.626458in}{2.734550in}}{\pgfqpoint{3.631758in}{2.736745in}}{\pgfqpoint{3.635664in}{2.740652in}}%
\pgfpathcurveto{\pgfqpoint{3.639571in}{2.744559in}}{\pgfqpoint{3.641766in}{2.749858in}}{\pgfqpoint{3.641766in}{2.755383in}}%
\pgfpathcurveto{\pgfqpoint{3.641766in}{2.760908in}}{\pgfqpoint{3.639571in}{2.766208in}}{\pgfqpoint{3.635664in}{2.770115in}}%
\pgfpathcurveto{\pgfqpoint{3.631758in}{2.774021in}}{\pgfqpoint{3.626458in}{2.776217in}}{\pgfqpoint{3.620933in}{2.776217in}}%
\pgfpathcurveto{\pgfqpoint{3.615408in}{2.776217in}}{\pgfqpoint{3.610108in}{2.774021in}}{\pgfqpoint{3.606202in}{2.770115in}}%
\pgfpathcurveto{\pgfqpoint{3.602295in}{2.766208in}}{\pgfqpoint{3.600100in}{2.760908in}}{\pgfqpoint{3.600100in}{2.755383in}}%
\pgfpathcurveto{\pgfqpoint{3.600100in}{2.749858in}}{\pgfqpoint{3.602295in}{2.744559in}}{\pgfqpoint{3.606202in}{2.740652in}}%
\pgfpathcurveto{\pgfqpoint{3.610108in}{2.736745in}}{\pgfqpoint{3.615408in}{2.734550in}}{\pgfqpoint{3.620933in}{2.734550in}}%
\pgfpathclose%
\pgfusepath{fill}%
\end{pgfscope}%
\begin{pgfscope}%
\pgfpathrectangle{\pgfqpoint{3.158185in}{2.343185in}}{\pgfqpoint{1.162500in}{0.755000in}} %
\pgfusepath{clip}%
\pgfsetbuttcap%
\pgfsetroundjoin%
\definecolor{currentfill}{rgb}{0.000000,0.000000,0.000000}%
\pgfsetfillcolor{currentfill}%
\pgfsetfillopacity{0.500000}%
\pgfsetlinewidth{0.000000pt}%
\definecolor{currentstroke}{rgb}{0.000000,0.000000,0.000000}%
\pgfsetstrokecolor{currentstroke}%
\pgfsetdash{}{0pt}%
\pgfpathmoveto{\pgfqpoint{3.600191in}{2.641930in}}%
\pgfpathcurveto{\pgfqpoint{3.605716in}{2.641930in}}{\pgfqpoint{3.611016in}{2.644125in}}{\pgfqpoint{3.614922in}{2.648032in}}%
\pgfpathcurveto{\pgfqpoint{3.618829in}{2.651939in}}{\pgfqpoint{3.621024in}{2.657239in}}{\pgfqpoint{3.621024in}{2.662764in}}%
\pgfpathcurveto{\pgfqpoint{3.621024in}{2.668289in}}{\pgfqpoint{3.618829in}{2.673588in}}{\pgfqpoint{3.614922in}{2.677495in}}%
\pgfpathcurveto{\pgfqpoint{3.611016in}{2.681402in}}{\pgfqpoint{3.605716in}{2.683597in}}{\pgfqpoint{3.600191in}{2.683597in}}%
\pgfpathcurveto{\pgfqpoint{3.594666in}{2.683597in}}{\pgfqpoint{3.589366in}{2.681402in}}{\pgfqpoint{3.585460in}{2.677495in}}%
\pgfpathcurveto{\pgfqpoint{3.581553in}{2.673588in}}{\pgfqpoint{3.579358in}{2.668289in}}{\pgfqpoint{3.579358in}{2.662764in}}%
\pgfpathcurveto{\pgfqpoint{3.579358in}{2.657239in}}{\pgfqpoint{3.581553in}{2.651939in}}{\pgfqpoint{3.585460in}{2.648032in}}%
\pgfpathcurveto{\pgfqpoint{3.589366in}{2.644125in}}{\pgfqpoint{3.594666in}{2.641930in}}{\pgfqpoint{3.600191in}{2.641930in}}%
\pgfpathclose%
\pgfusepath{fill}%
\end{pgfscope}%
\begin{pgfscope}%
\pgfpathrectangle{\pgfqpoint{3.158185in}{2.343185in}}{\pgfqpoint{1.162500in}{0.755000in}} %
\pgfusepath{clip}%
\pgfsetbuttcap%
\pgfsetroundjoin%
\definecolor{currentfill}{rgb}{0.000000,0.000000,0.000000}%
\pgfsetfillcolor{currentfill}%
\pgfsetfillopacity{0.500000}%
\pgfsetlinewidth{0.000000pt}%
\definecolor{currentstroke}{rgb}{0.000000,0.000000,0.000000}%
\pgfsetstrokecolor{currentstroke}%
\pgfsetdash{}{0pt}%
\pgfpathmoveto{\pgfqpoint{3.899960in}{2.975096in}}%
\pgfpathcurveto{\pgfqpoint{3.905485in}{2.975096in}}{\pgfqpoint{3.910785in}{2.977291in}}{\pgfqpoint{3.914692in}{2.981198in}}%
\pgfpathcurveto{\pgfqpoint{3.918598in}{2.985104in}}{\pgfqpoint{3.920794in}{2.990404in}}{\pgfqpoint{3.920794in}{2.995929in}}%
\pgfpathcurveto{\pgfqpoint{3.920794in}{3.001454in}}{\pgfqpoint{3.918598in}{3.006754in}}{\pgfqpoint{3.914692in}{3.010660in}}%
\pgfpathcurveto{\pgfqpoint{3.910785in}{3.014567in}}{\pgfqpoint{3.905485in}{3.016762in}}{\pgfqpoint{3.899960in}{3.016762in}}%
\pgfpathcurveto{\pgfqpoint{3.894435in}{3.016762in}}{\pgfqpoint{3.889136in}{3.014567in}}{\pgfqpoint{3.885229in}{3.010660in}}%
\pgfpathcurveto{\pgfqpoint{3.881322in}{3.006754in}}{\pgfqpoint{3.879127in}{3.001454in}}{\pgfqpoint{3.879127in}{2.995929in}}%
\pgfpathcurveto{\pgfqpoint{3.879127in}{2.990404in}}{\pgfqpoint{3.881322in}{2.985104in}}{\pgfqpoint{3.885229in}{2.981198in}}%
\pgfpathcurveto{\pgfqpoint{3.889136in}{2.977291in}}{\pgfqpoint{3.894435in}{2.975096in}}{\pgfqpoint{3.899960in}{2.975096in}}%
\pgfpathclose%
\pgfusepath{fill}%
\end{pgfscope}%
\begin{pgfscope}%
\pgfpathrectangle{\pgfqpoint{3.158185in}{2.343185in}}{\pgfqpoint{1.162500in}{0.755000in}} %
\pgfusepath{clip}%
\pgfsetbuttcap%
\pgfsetroundjoin%
\definecolor{currentfill}{rgb}{0.000000,0.000000,0.000000}%
\pgfsetfillcolor{currentfill}%
\pgfsetfillopacity{0.500000}%
\pgfsetlinewidth{0.000000pt}%
\definecolor{currentstroke}{rgb}{0.000000,0.000000,0.000000}%
\pgfsetstrokecolor{currentstroke}%
\pgfsetdash{}{0pt}%
\pgfpathmoveto{\pgfqpoint{3.368320in}{2.531911in}}%
\pgfpathcurveto{\pgfqpoint{3.373845in}{2.531911in}}{\pgfqpoint{3.379145in}{2.534106in}}{\pgfqpoint{3.383052in}{2.538013in}}%
\pgfpathcurveto{\pgfqpoint{3.386958in}{2.541920in}}{\pgfqpoint{3.389154in}{2.547220in}}{\pgfqpoint{3.389154in}{2.552745in}}%
\pgfpathcurveto{\pgfqpoint{3.389154in}{2.558270in}}{\pgfqpoint{3.386958in}{2.563569in}}{\pgfqpoint{3.383052in}{2.567476in}}%
\pgfpathcurveto{\pgfqpoint{3.379145in}{2.571383in}}{\pgfqpoint{3.373845in}{2.573578in}}{\pgfqpoint{3.368320in}{2.573578in}}%
\pgfpathcurveto{\pgfqpoint{3.362795in}{2.573578in}}{\pgfqpoint{3.357496in}{2.571383in}}{\pgfqpoint{3.353589in}{2.567476in}}%
\pgfpathcurveto{\pgfqpoint{3.349682in}{2.563569in}}{\pgfqpoint{3.347487in}{2.558270in}}{\pgfqpoint{3.347487in}{2.552745in}}%
\pgfpathcurveto{\pgfqpoint{3.347487in}{2.547220in}}{\pgfqpoint{3.349682in}{2.541920in}}{\pgfqpoint{3.353589in}{2.538013in}}%
\pgfpathcurveto{\pgfqpoint{3.357496in}{2.534106in}}{\pgfqpoint{3.362795in}{2.531911in}}{\pgfqpoint{3.368320in}{2.531911in}}%
\pgfpathclose%
\pgfusepath{fill}%
\end{pgfscope}%
\begin{pgfscope}%
\pgfpathrectangle{\pgfqpoint{3.158185in}{2.343185in}}{\pgfqpoint{1.162500in}{0.755000in}} %
\pgfusepath{clip}%
\pgfsetbuttcap%
\pgfsetroundjoin%
\definecolor{currentfill}{rgb}{0.000000,0.000000,0.000000}%
\pgfsetfillcolor{currentfill}%
\pgfsetfillopacity{0.500000}%
\pgfsetlinewidth{0.000000pt}%
\definecolor{currentstroke}{rgb}{0.000000,0.000000,0.000000}%
\pgfsetstrokecolor{currentstroke}%
\pgfsetdash{}{0pt}%
\pgfpathmoveto{\pgfqpoint{3.416175in}{2.591601in}}%
\pgfpathcurveto{\pgfqpoint{3.421700in}{2.591601in}}{\pgfqpoint{3.427000in}{2.593796in}}{\pgfqpoint{3.430907in}{2.597703in}}%
\pgfpathcurveto{\pgfqpoint{3.434814in}{2.601609in}}{\pgfqpoint{3.437009in}{2.606909in}}{\pgfqpoint{3.437009in}{2.612434in}}%
\pgfpathcurveto{\pgfqpoint{3.437009in}{2.617959in}}{\pgfqpoint{3.434814in}{2.623259in}}{\pgfqpoint{3.430907in}{2.627165in}}%
\pgfpathcurveto{\pgfqpoint{3.427000in}{2.631072in}}{\pgfqpoint{3.421700in}{2.633267in}}{\pgfqpoint{3.416175in}{2.633267in}}%
\pgfpathcurveto{\pgfqpoint{3.410650in}{2.633267in}}{\pgfqpoint{3.405351in}{2.631072in}}{\pgfqpoint{3.401444in}{2.627165in}}%
\pgfpathcurveto{\pgfqpoint{3.397537in}{2.623259in}}{\pgfqpoint{3.395342in}{2.617959in}}{\pgfqpoint{3.395342in}{2.612434in}}%
\pgfpathcurveto{\pgfqpoint{3.395342in}{2.606909in}}{\pgfqpoint{3.397537in}{2.601609in}}{\pgfqpoint{3.401444in}{2.597703in}}%
\pgfpathcurveto{\pgfqpoint{3.405351in}{2.593796in}}{\pgfqpoint{3.410650in}{2.591601in}}{\pgfqpoint{3.416175in}{2.591601in}}%
\pgfpathclose%
\pgfusepath{fill}%
\end{pgfscope}%
\begin{pgfscope}%
\pgfpathrectangle{\pgfqpoint{3.158185in}{2.343185in}}{\pgfqpoint{1.162500in}{0.755000in}} %
\pgfusepath{clip}%
\pgfsetbuttcap%
\pgfsetroundjoin%
\definecolor{currentfill}{rgb}{0.000000,0.000000,0.000000}%
\pgfsetfillcolor{currentfill}%
\pgfsetfillopacity{0.500000}%
\pgfsetlinewidth{0.000000pt}%
\definecolor{currentstroke}{rgb}{0.000000,0.000000,0.000000}%
\pgfsetstrokecolor{currentstroke}%
\pgfsetdash{}{0pt}%
\pgfpathmoveto{\pgfqpoint{3.789669in}{2.814070in}}%
\pgfpathcurveto{\pgfqpoint{3.795194in}{2.814070in}}{\pgfqpoint{3.800493in}{2.816265in}}{\pgfqpoint{3.804400in}{2.820172in}}%
\pgfpathcurveto{\pgfqpoint{3.808307in}{2.824079in}}{\pgfqpoint{3.810502in}{2.829378in}}{\pgfqpoint{3.810502in}{2.834904in}}%
\pgfpathcurveto{\pgfqpoint{3.810502in}{2.840429in}}{\pgfqpoint{3.808307in}{2.845728in}}{\pgfqpoint{3.804400in}{2.849635in}}%
\pgfpathcurveto{\pgfqpoint{3.800493in}{2.853542in}}{\pgfqpoint{3.795194in}{2.855737in}}{\pgfqpoint{3.789669in}{2.855737in}}%
\pgfpathcurveto{\pgfqpoint{3.784143in}{2.855737in}}{\pgfqpoint{3.778844in}{2.853542in}}{\pgfqpoint{3.774937in}{2.849635in}}%
\pgfpathcurveto{\pgfqpoint{3.771030in}{2.845728in}}{\pgfqpoint{3.768835in}{2.840429in}}{\pgfqpoint{3.768835in}{2.834904in}}%
\pgfpathcurveto{\pgfqpoint{3.768835in}{2.829378in}}{\pgfqpoint{3.771030in}{2.824079in}}{\pgfqpoint{3.774937in}{2.820172in}}%
\pgfpathcurveto{\pgfqpoint{3.778844in}{2.816265in}}{\pgfqpoint{3.784143in}{2.814070in}}{\pgfqpoint{3.789669in}{2.814070in}}%
\pgfpathclose%
\pgfusepath{fill}%
\end{pgfscope}%
\begin{pgfscope}%
\pgfpathrectangle{\pgfqpoint{3.158185in}{2.343185in}}{\pgfqpoint{1.162500in}{0.755000in}} %
\pgfusepath{clip}%
\pgfsetbuttcap%
\pgfsetroundjoin%
\definecolor{currentfill}{rgb}{0.000000,0.000000,0.000000}%
\pgfsetfillcolor{currentfill}%
\pgfsetfillopacity{0.500000}%
\pgfsetlinewidth{0.000000pt}%
\definecolor{currentstroke}{rgb}{0.000000,0.000000,0.000000}%
\pgfsetstrokecolor{currentstroke}%
\pgfsetdash{}{0pt}%
\pgfpathmoveto{\pgfqpoint{3.231064in}{2.392267in}}%
\pgfpathcurveto{\pgfqpoint{3.236589in}{2.392267in}}{\pgfqpoint{3.241888in}{2.394462in}}{\pgfqpoint{3.245795in}{2.398369in}}%
\pgfpathcurveto{\pgfqpoint{3.249702in}{2.402276in}}{\pgfqpoint{3.251897in}{2.407575in}}{\pgfqpoint{3.251897in}{2.413101in}}%
\pgfpathcurveto{\pgfqpoint{3.251897in}{2.418626in}}{\pgfqpoint{3.249702in}{2.423925in}}{\pgfqpoint{3.245795in}{2.427832in}}%
\pgfpathcurveto{\pgfqpoint{3.241888in}{2.431739in}}{\pgfqpoint{3.236589in}{2.433934in}}{\pgfqpoint{3.231064in}{2.433934in}}%
\pgfpathcurveto{\pgfqpoint{3.225539in}{2.433934in}}{\pgfqpoint{3.220239in}{2.431739in}}{\pgfqpoint{3.216332in}{2.427832in}}%
\pgfpathcurveto{\pgfqpoint{3.212425in}{2.423925in}}{\pgfqpoint{3.210230in}{2.418626in}}{\pgfqpoint{3.210230in}{2.413101in}}%
\pgfpathcurveto{\pgfqpoint{3.210230in}{2.407575in}}{\pgfqpoint{3.212425in}{2.402276in}}{\pgfqpoint{3.216332in}{2.398369in}}%
\pgfpathcurveto{\pgfqpoint{3.220239in}{2.394462in}}{\pgfqpoint{3.225539in}{2.392267in}}{\pgfqpoint{3.231064in}{2.392267in}}%
\pgfpathclose%
\pgfusepath{fill}%
\end{pgfscope}%
\begin{pgfscope}%
\pgfsetrectcap%
\pgfsetmiterjoin%
\pgfsetlinewidth{0.803000pt}%
\definecolor{currentstroke}{rgb}{0.000000,0.000000,0.000000}%
\pgfsetstrokecolor{currentstroke}%
\pgfsetdash{}{0pt}%
\pgfpathmoveto{\pgfqpoint{3.158185in}{2.343185in}}%
\pgfpathlineto{\pgfqpoint{3.158185in}{3.098185in}}%
\pgfusepath{stroke}%
\end{pgfscope}%
\begin{pgfscope}%
\pgfsetrectcap%
\pgfsetmiterjoin%
\pgfsetlinewidth{0.803000pt}%
\definecolor{currentstroke}{rgb}{0.000000,0.000000,0.000000}%
\pgfsetstrokecolor{currentstroke}%
\pgfsetdash{}{0pt}%
\pgfpathmoveto{\pgfqpoint{4.320685in}{2.343185in}}%
\pgfpathlineto{\pgfqpoint{4.320685in}{3.098185in}}%
\pgfusepath{stroke}%
\end{pgfscope}%
\begin{pgfscope}%
\pgfsetrectcap%
\pgfsetmiterjoin%
\pgfsetlinewidth{0.803000pt}%
\definecolor{currentstroke}{rgb}{0.000000,0.000000,0.000000}%
\pgfsetstrokecolor{currentstroke}%
\pgfsetdash{}{0pt}%
\pgfpathmoveto{\pgfqpoint{3.158185in}{2.343185in}}%
\pgfpathlineto{\pgfqpoint{4.320685in}{2.343185in}}%
\pgfusepath{stroke}%
\end{pgfscope}%
\begin{pgfscope}%
\pgfsetrectcap%
\pgfsetmiterjoin%
\pgfsetlinewidth{0.803000pt}%
\definecolor{currentstroke}{rgb}{0.000000,0.000000,0.000000}%
\pgfsetstrokecolor{currentstroke}%
\pgfsetdash{}{0pt}%
\pgfpathmoveto{\pgfqpoint{3.158185in}{3.098185in}}%
\pgfpathlineto{\pgfqpoint{4.320685in}{3.098185in}}%
\pgfusepath{stroke}%
\end{pgfscope}%
\begin{pgfscope}%
\pgfsetbuttcap%
\pgfsetmiterjoin%
\definecolor{currentfill}{rgb}{1.000000,1.000000,1.000000}%
\pgfsetfillcolor{currentfill}%
\pgfsetlinewidth{0.000000pt}%
\definecolor{currentstroke}{rgb}{0.000000,0.000000,0.000000}%
\pgfsetstrokecolor{currentstroke}%
\pgfsetstrokeopacity{0.000000}%
\pgfsetdash{}{0pt}%
\pgfpathmoveto{\pgfqpoint{4.320685in}{2.343185in}}%
\pgfpathlineto{\pgfqpoint{5.483185in}{2.343185in}}%
\pgfpathlineto{\pgfqpoint{5.483185in}{3.098185in}}%
\pgfpathlineto{\pgfqpoint{4.320685in}{3.098185in}}%
\pgfpathclose%
\pgfusepath{fill}%
\end{pgfscope}%
\begin{pgfscope}%
\pgfpathrectangle{\pgfqpoint{4.320685in}{2.343185in}}{\pgfqpoint{1.162500in}{0.755000in}} %
\pgfusepath{clip}%
\pgfsetbuttcap%
\pgfsetroundjoin%
\definecolor{currentfill}{rgb}{0.000000,0.000000,0.000000}%
\pgfsetfillcolor{currentfill}%
\pgfsetfillopacity{0.500000}%
\pgfsetlinewidth{0.000000pt}%
\definecolor{currentstroke}{rgb}{0.000000,0.000000,0.000000}%
\pgfsetstrokecolor{currentstroke}%
\pgfsetdash{}{0pt}%
\pgfpathmoveto{\pgfqpoint{4.763543in}{3.059375in}}%
\pgfpathcurveto{\pgfqpoint{4.769068in}{3.059375in}}{\pgfqpoint{4.774367in}{3.061570in}}{\pgfqpoint{4.778274in}{3.065477in}}%
\pgfpathcurveto{\pgfqpoint{4.782181in}{3.069384in}}{\pgfqpoint{4.784376in}{3.074683in}}{\pgfqpoint{4.784376in}{3.080208in}}%
\pgfpathcurveto{\pgfqpoint{4.784376in}{3.085733in}}{\pgfqpoint{4.782181in}{3.091033in}}{\pgfqpoint{4.778274in}{3.094940in}}%
\pgfpathcurveto{\pgfqpoint{4.774367in}{3.098847in}}{\pgfqpoint{4.769068in}{3.101042in}}{\pgfqpoint{4.763543in}{3.101042in}}%
\pgfpathcurveto{\pgfqpoint{4.758018in}{3.101042in}}{\pgfqpoint{4.752718in}{3.098847in}}{\pgfqpoint{4.748811in}{3.094940in}}%
\pgfpathcurveto{\pgfqpoint{4.744905in}{3.091033in}}{\pgfqpoint{4.742709in}{3.085733in}}{\pgfqpoint{4.742709in}{3.080208in}}%
\pgfpathcurveto{\pgfqpoint{4.742709in}{3.074683in}}{\pgfqpoint{4.744905in}{3.069384in}}{\pgfqpoint{4.748811in}{3.065477in}}%
\pgfpathcurveto{\pgfqpoint{4.752718in}{3.061570in}}{\pgfqpoint{4.758018in}{3.059375in}}{\pgfqpoint{4.763543in}{3.059375in}}%
\pgfpathclose%
\pgfusepath{fill}%
\end{pgfscope}%
\begin{pgfscope}%
\pgfpathrectangle{\pgfqpoint{4.320685in}{2.343185in}}{\pgfqpoint{1.162500in}{0.755000in}} %
\pgfusepath{clip}%
\pgfsetbuttcap%
\pgfsetroundjoin%
\definecolor{currentfill}{rgb}{0.000000,0.000000,0.000000}%
\pgfsetfillcolor{currentfill}%
\pgfsetfillopacity{0.500000}%
\pgfsetlinewidth{0.000000pt}%
\definecolor{currentstroke}{rgb}{0.000000,0.000000,0.000000}%
\pgfsetstrokecolor{currentstroke}%
\pgfsetdash{}{0pt}%
\pgfpathmoveto{\pgfqpoint{5.385388in}{2.815854in}}%
\pgfpathcurveto{\pgfqpoint{5.390913in}{2.815854in}}{\pgfqpoint{5.396213in}{2.818049in}}{\pgfqpoint{5.400119in}{2.821955in}}%
\pgfpathcurveto{\pgfqpoint{5.404026in}{2.825862in}}{\pgfqpoint{5.406221in}{2.831162in}}{\pgfqpoint{5.406221in}{2.836687in}}%
\pgfpathcurveto{\pgfqpoint{5.406221in}{2.842212in}}{\pgfqpoint{5.404026in}{2.847511in}}{\pgfqpoint{5.400119in}{2.851418in}}%
\pgfpathcurveto{\pgfqpoint{5.396213in}{2.855325in}}{\pgfqpoint{5.390913in}{2.857520in}}{\pgfqpoint{5.385388in}{2.857520in}}%
\pgfpathcurveto{\pgfqpoint{5.379863in}{2.857520in}}{\pgfqpoint{5.374564in}{2.855325in}}{\pgfqpoint{5.370657in}{2.851418in}}%
\pgfpathcurveto{\pgfqpoint{5.366750in}{2.847511in}}{\pgfqpoint{5.364555in}{2.842212in}}{\pgfqpoint{5.364555in}{2.836687in}}%
\pgfpathcurveto{\pgfqpoint{5.364555in}{2.831162in}}{\pgfqpoint{5.366750in}{2.825862in}}{\pgfqpoint{5.370657in}{2.821955in}}%
\pgfpathcurveto{\pgfqpoint{5.374564in}{2.818049in}}{\pgfqpoint{5.379863in}{2.815854in}}{\pgfqpoint{5.385388in}{2.815854in}}%
\pgfpathclose%
\pgfusepath{fill}%
\end{pgfscope}%
\begin{pgfscope}%
\pgfpathrectangle{\pgfqpoint{4.320685in}{2.343185in}}{\pgfqpoint{1.162500in}{0.755000in}} %
\pgfusepath{clip}%
\pgfsetbuttcap%
\pgfsetroundjoin%
\definecolor{currentfill}{rgb}{0.000000,0.000000,0.000000}%
\pgfsetfillcolor{currentfill}%
\pgfsetfillopacity{0.500000}%
\pgfsetlinewidth{0.000000pt}%
\definecolor{currentstroke}{rgb}{0.000000,0.000000,0.000000}%
\pgfsetstrokecolor{currentstroke}%
\pgfsetdash{}{0pt}%
\pgfpathmoveto{\pgfqpoint{5.455506in}{2.815581in}}%
\pgfpathcurveto{\pgfqpoint{5.461031in}{2.815581in}}{\pgfqpoint{5.466331in}{2.817776in}}{\pgfqpoint{5.470237in}{2.821682in}}%
\pgfpathcurveto{\pgfqpoint{5.474144in}{2.825589in}}{\pgfqpoint{5.476339in}{2.830889in}}{\pgfqpoint{5.476339in}{2.836414in}}%
\pgfpathcurveto{\pgfqpoint{5.476339in}{2.841939in}}{\pgfqpoint{5.474144in}{2.847238in}}{\pgfqpoint{5.470237in}{2.851145in}}%
\pgfpathcurveto{\pgfqpoint{5.466331in}{2.855052in}}{\pgfqpoint{5.461031in}{2.857247in}}{\pgfqpoint{5.455506in}{2.857247in}}%
\pgfpathcurveto{\pgfqpoint{5.449981in}{2.857247in}}{\pgfqpoint{5.444681in}{2.855052in}}{\pgfqpoint{5.440775in}{2.851145in}}%
\pgfpathcurveto{\pgfqpoint{5.436868in}{2.847238in}}{\pgfqpoint{5.434673in}{2.841939in}}{\pgfqpoint{5.434673in}{2.836414in}}%
\pgfpathcurveto{\pgfqpoint{5.434673in}{2.830889in}}{\pgfqpoint{5.436868in}{2.825589in}}{\pgfqpoint{5.440775in}{2.821682in}}%
\pgfpathcurveto{\pgfqpoint{5.444681in}{2.817776in}}{\pgfqpoint{5.449981in}{2.815581in}}{\pgfqpoint{5.455506in}{2.815581in}}%
\pgfpathclose%
\pgfusepath{fill}%
\end{pgfscope}%
\begin{pgfscope}%
\pgfpathrectangle{\pgfqpoint{4.320685in}{2.343185in}}{\pgfqpoint{1.162500in}{0.755000in}} %
\pgfusepath{clip}%
\pgfsetbuttcap%
\pgfsetroundjoin%
\definecolor{currentfill}{rgb}{0.000000,0.000000,0.000000}%
\pgfsetfillcolor{currentfill}%
\pgfsetfillopacity{0.500000}%
\pgfsetlinewidth{0.000000pt}%
\definecolor{currentstroke}{rgb}{0.000000,0.000000,0.000000}%
\pgfsetstrokecolor{currentstroke}%
\pgfsetdash{}{0pt}%
\pgfpathmoveto{\pgfqpoint{5.031128in}{2.531108in}}%
\pgfpathcurveto{\pgfqpoint{5.036653in}{2.531108in}}{\pgfqpoint{5.041953in}{2.533303in}}{\pgfqpoint{5.045859in}{2.537210in}}%
\pgfpathcurveto{\pgfqpoint{5.049766in}{2.541116in}}{\pgfqpoint{5.051961in}{2.546416in}}{\pgfqpoint{5.051961in}{2.551941in}}%
\pgfpathcurveto{\pgfqpoint{5.051961in}{2.557466in}}{\pgfqpoint{5.049766in}{2.562766in}}{\pgfqpoint{5.045859in}{2.566672in}}%
\pgfpathcurveto{\pgfqpoint{5.041953in}{2.570579in}}{\pgfqpoint{5.036653in}{2.572774in}}{\pgfqpoint{5.031128in}{2.572774in}}%
\pgfpathcurveto{\pgfqpoint{5.025603in}{2.572774in}}{\pgfqpoint{5.020303in}{2.570579in}}{\pgfqpoint{5.016397in}{2.566672in}}%
\pgfpathcurveto{\pgfqpoint{5.012490in}{2.562766in}}{\pgfqpoint{5.010295in}{2.557466in}}{\pgfqpoint{5.010295in}{2.551941in}}%
\pgfpathcurveto{\pgfqpoint{5.010295in}{2.546416in}}{\pgfqpoint{5.012490in}{2.541116in}}{\pgfqpoint{5.016397in}{2.537210in}}%
\pgfpathcurveto{\pgfqpoint{5.020303in}{2.533303in}}{\pgfqpoint{5.025603in}{2.531108in}}{\pgfqpoint{5.031128in}{2.531108in}}%
\pgfpathclose%
\pgfusepath{fill}%
\end{pgfscope}%
\begin{pgfscope}%
\pgfpathrectangle{\pgfqpoint{4.320685in}{2.343185in}}{\pgfqpoint{1.162500in}{0.755000in}} %
\pgfusepath{clip}%
\pgfsetbuttcap%
\pgfsetroundjoin%
\definecolor{currentfill}{rgb}{0.000000,0.000000,0.000000}%
\pgfsetfillcolor{currentfill}%
\pgfsetfillopacity{0.500000}%
\pgfsetlinewidth{0.000000pt}%
\definecolor{currentstroke}{rgb}{0.000000,0.000000,0.000000}%
\pgfsetstrokecolor{currentstroke}%
\pgfsetdash{}{0pt}%
\pgfpathmoveto{\pgfqpoint{5.102087in}{2.481860in}}%
\pgfpathcurveto{\pgfqpoint{5.107612in}{2.481860in}}{\pgfqpoint{5.112912in}{2.484055in}}{\pgfqpoint{5.116819in}{2.487962in}}%
\pgfpathcurveto{\pgfqpoint{5.120726in}{2.491868in}}{\pgfqpoint{5.122921in}{2.497168in}}{\pgfqpoint{5.122921in}{2.502693in}}%
\pgfpathcurveto{\pgfqpoint{5.122921in}{2.508218in}}{\pgfqpoint{5.120726in}{2.513517in}}{\pgfqpoint{5.116819in}{2.517424in}}%
\pgfpathcurveto{\pgfqpoint{5.112912in}{2.521331in}}{\pgfqpoint{5.107612in}{2.523526in}}{\pgfqpoint{5.102087in}{2.523526in}}%
\pgfpathcurveto{\pgfqpoint{5.096562in}{2.523526in}}{\pgfqpoint{5.091263in}{2.521331in}}{\pgfqpoint{5.087356in}{2.517424in}}%
\pgfpathcurveto{\pgfqpoint{5.083449in}{2.513517in}}{\pgfqpoint{5.081254in}{2.508218in}}{\pgfqpoint{5.081254in}{2.502693in}}%
\pgfpathcurveto{\pgfqpoint{5.081254in}{2.497168in}}{\pgfqpoint{5.083449in}{2.491868in}}{\pgfqpoint{5.087356in}{2.487962in}}%
\pgfpathcurveto{\pgfqpoint{5.091263in}{2.484055in}}{\pgfqpoint{5.096562in}{2.481860in}}{\pgfqpoint{5.102087in}{2.481860in}}%
\pgfpathclose%
\pgfusepath{fill}%
\end{pgfscope}%
\begin{pgfscope}%
\pgfpathrectangle{\pgfqpoint{4.320685in}{2.343185in}}{\pgfqpoint{1.162500in}{0.755000in}} %
\pgfusepath{clip}%
\pgfsetbuttcap%
\pgfsetroundjoin%
\definecolor{currentfill}{rgb}{0.000000,0.000000,0.000000}%
\pgfsetfillcolor{currentfill}%
\pgfsetfillopacity{0.500000}%
\pgfsetlinewidth{0.000000pt}%
\definecolor{currentstroke}{rgb}{0.000000,0.000000,0.000000}%
\pgfsetstrokecolor{currentstroke}%
\pgfsetdash{}{0pt}%
\pgfpathmoveto{\pgfqpoint{4.940844in}{2.394040in}}%
\pgfpathcurveto{\pgfqpoint{4.946369in}{2.394040in}}{\pgfqpoint{4.951669in}{2.396235in}}{\pgfqpoint{4.955576in}{2.400142in}}%
\pgfpathcurveto{\pgfqpoint{4.959482in}{2.404048in}}{\pgfqpoint{4.961678in}{2.409348in}}{\pgfqpoint{4.961678in}{2.414873in}}%
\pgfpathcurveto{\pgfqpoint{4.961678in}{2.420398in}}{\pgfqpoint{4.959482in}{2.425698in}}{\pgfqpoint{4.955576in}{2.429604in}}%
\pgfpathcurveto{\pgfqpoint{4.951669in}{2.433511in}}{\pgfqpoint{4.946369in}{2.435706in}}{\pgfqpoint{4.940844in}{2.435706in}}%
\pgfpathcurveto{\pgfqpoint{4.935319in}{2.435706in}}{\pgfqpoint{4.930020in}{2.433511in}}{\pgfqpoint{4.926113in}{2.429604in}}%
\pgfpathcurveto{\pgfqpoint{4.922206in}{2.425698in}}{\pgfqpoint{4.920011in}{2.420398in}}{\pgfqpoint{4.920011in}{2.414873in}}%
\pgfpathcurveto{\pgfqpoint{4.920011in}{2.409348in}}{\pgfqpoint{4.922206in}{2.404048in}}{\pgfqpoint{4.926113in}{2.400142in}}%
\pgfpathcurveto{\pgfqpoint{4.930020in}{2.396235in}}{\pgfqpoint{4.935319in}{2.394040in}}{\pgfqpoint{4.940844in}{2.394040in}}%
\pgfpathclose%
\pgfusepath{fill}%
\end{pgfscope}%
\begin{pgfscope}%
\pgfpathrectangle{\pgfqpoint{4.320685in}{2.343185in}}{\pgfqpoint{1.162500in}{0.755000in}} %
\pgfusepath{clip}%
\pgfsetbuttcap%
\pgfsetroundjoin%
\definecolor{currentfill}{rgb}{0.000000,0.000000,0.000000}%
\pgfsetfillcolor{currentfill}%
\pgfsetfillopacity{0.500000}%
\pgfsetlinewidth{0.000000pt}%
\definecolor{currentstroke}{rgb}{0.000000,0.000000,0.000000}%
\pgfsetstrokecolor{currentstroke}%
\pgfsetdash{}{0pt}%
\pgfpathmoveto{\pgfqpoint{4.348363in}{2.340327in}}%
\pgfpathcurveto{\pgfqpoint{4.353888in}{2.340327in}}{\pgfqpoint{4.359188in}{2.342523in}}{\pgfqpoint{4.363095in}{2.346429in}}%
\pgfpathcurveto{\pgfqpoint{4.367001in}{2.350336in}}{\pgfqpoint{4.369196in}{2.355636in}}{\pgfqpoint{4.369196in}{2.361161in}}%
\pgfpathcurveto{\pgfqpoint{4.369196in}{2.366686in}}{\pgfqpoint{4.367001in}{2.371985in}}{\pgfqpoint{4.363095in}{2.375892in}}%
\pgfpathcurveto{\pgfqpoint{4.359188in}{2.379799in}}{\pgfqpoint{4.353888in}{2.381994in}}{\pgfqpoint{4.348363in}{2.381994in}}%
\pgfpathcurveto{\pgfqpoint{4.342838in}{2.381994in}}{\pgfqpoint{4.337539in}{2.379799in}}{\pgfqpoint{4.333632in}{2.375892in}}%
\pgfpathcurveto{\pgfqpoint{4.329725in}{2.371985in}}{\pgfqpoint{4.327530in}{2.366686in}}{\pgfqpoint{4.327530in}{2.361161in}}%
\pgfpathcurveto{\pgfqpoint{4.327530in}{2.355636in}}{\pgfqpoint{4.329725in}{2.350336in}}{\pgfqpoint{4.333632in}{2.346429in}}%
\pgfpathcurveto{\pgfqpoint{4.337539in}{2.342523in}}{\pgfqpoint{4.342838in}{2.340327in}}{\pgfqpoint{4.348363in}{2.340327in}}%
\pgfpathclose%
\pgfusepath{fill}%
\end{pgfscope}%
\begin{pgfscope}%
\pgfpathrectangle{\pgfqpoint{4.320685in}{2.343185in}}{\pgfqpoint{1.162500in}{0.755000in}} %
\pgfusepath{clip}%
\pgfsetbuttcap%
\pgfsetroundjoin%
\definecolor{currentfill}{rgb}{0.000000,0.000000,0.000000}%
\pgfsetfillcolor{currentfill}%
\pgfsetfillopacity{0.500000}%
\pgfsetlinewidth{0.000000pt}%
\definecolor{currentstroke}{rgb}{0.000000,0.000000,0.000000}%
\pgfsetstrokecolor{currentstroke}%
\pgfsetdash{}{0pt}%
\pgfpathmoveto{\pgfqpoint{5.332823in}{2.914639in}}%
\pgfpathcurveto{\pgfqpoint{5.338348in}{2.914639in}}{\pgfqpoint{5.343648in}{2.916834in}}{\pgfqpoint{5.347555in}{2.920741in}}%
\pgfpathcurveto{\pgfqpoint{5.351461in}{2.924648in}}{\pgfqpoint{5.353656in}{2.929947in}}{\pgfqpoint{5.353656in}{2.935473in}}%
\pgfpathcurveto{\pgfqpoint{5.353656in}{2.940998in}}{\pgfqpoint{5.351461in}{2.946297in}}{\pgfqpoint{5.347555in}{2.950204in}}%
\pgfpathcurveto{\pgfqpoint{5.343648in}{2.954111in}}{\pgfqpoint{5.338348in}{2.956306in}}{\pgfqpoint{5.332823in}{2.956306in}}%
\pgfpathcurveto{\pgfqpoint{5.327298in}{2.956306in}}{\pgfqpoint{5.321999in}{2.954111in}}{\pgfqpoint{5.318092in}{2.950204in}}%
\pgfpathcurveto{\pgfqpoint{5.314185in}{2.946297in}}{\pgfqpoint{5.311990in}{2.940998in}}{\pgfqpoint{5.311990in}{2.935473in}}%
\pgfpathcurveto{\pgfqpoint{5.311990in}{2.929947in}}{\pgfqpoint{5.314185in}{2.924648in}}{\pgfqpoint{5.318092in}{2.920741in}}%
\pgfpathcurveto{\pgfqpoint{5.321999in}{2.916834in}}{\pgfqpoint{5.327298in}{2.914639in}}{\pgfqpoint{5.332823in}{2.914639in}}%
\pgfpathclose%
\pgfusepath{fill}%
\end{pgfscope}%
\begin{pgfscope}%
\pgfpathrectangle{\pgfqpoint{4.320685in}{2.343185in}}{\pgfqpoint{1.162500in}{0.755000in}} %
\pgfusepath{clip}%
\pgfsetbuttcap%
\pgfsetroundjoin%
\definecolor{currentfill}{rgb}{0.000000,0.000000,0.000000}%
\pgfsetfillcolor{currentfill}%
\pgfsetfillopacity{0.500000}%
\pgfsetlinewidth{0.000000pt}%
\definecolor{currentstroke}{rgb}{0.000000,0.000000,0.000000}%
\pgfsetstrokecolor{currentstroke}%
\pgfsetdash{}{0pt}%
\pgfpathmoveto{\pgfqpoint{5.090412in}{2.734550in}}%
\pgfpathcurveto{\pgfqpoint{5.095937in}{2.734550in}}{\pgfqpoint{5.101237in}{2.736745in}}{\pgfqpoint{5.105143in}{2.740652in}}%
\pgfpathcurveto{\pgfqpoint{5.109050in}{2.744559in}}{\pgfqpoint{5.111245in}{2.749858in}}{\pgfqpoint{5.111245in}{2.755383in}}%
\pgfpathcurveto{\pgfqpoint{5.111245in}{2.760908in}}{\pgfqpoint{5.109050in}{2.766208in}}{\pgfqpoint{5.105143in}{2.770115in}}%
\pgfpathcurveto{\pgfqpoint{5.101237in}{2.774021in}}{\pgfqpoint{5.095937in}{2.776217in}}{\pgfqpoint{5.090412in}{2.776217in}}%
\pgfpathcurveto{\pgfqpoint{5.084887in}{2.776217in}}{\pgfqpoint{5.079587in}{2.774021in}}{\pgfqpoint{5.075681in}{2.770115in}}%
\pgfpathcurveto{\pgfqpoint{5.071774in}{2.766208in}}{\pgfqpoint{5.069579in}{2.760908in}}{\pgfqpoint{5.069579in}{2.755383in}}%
\pgfpathcurveto{\pgfqpoint{5.069579in}{2.749858in}}{\pgfqpoint{5.071774in}{2.744559in}}{\pgfqpoint{5.075681in}{2.740652in}}%
\pgfpathcurveto{\pgfqpoint{5.079587in}{2.736745in}}{\pgfqpoint{5.084887in}{2.734550in}}{\pgfqpoint{5.090412in}{2.734550in}}%
\pgfpathclose%
\pgfusepath{fill}%
\end{pgfscope}%
\begin{pgfscope}%
\pgfpathrectangle{\pgfqpoint{4.320685in}{2.343185in}}{\pgfqpoint{1.162500in}{0.755000in}} %
\pgfusepath{clip}%
\pgfsetbuttcap%
\pgfsetroundjoin%
\definecolor{currentfill}{rgb}{0.000000,0.000000,0.000000}%
\pgfsetfillcolor{currentfill}%
\pgfsetfillopacity{0.500000}%
\pgfsetlinewidth{0.000000pt}%
\definecolor{currentstroke}{rgb}{0.000000,0.000000,0.000000}%
\pgfsetstrokecolor{currentstroke}%
\pgfsetdash{}{0pt}%
\pgfpathmoveto{\pgfqpoint{4.817426in}{2.641930in}}%
\pgfpathcurveto{\pgfqpoint{4.822951in}{2.641930in}}{\pgfqpoint{4.828251in}{2.644125in}}{\pgfqpoint{4.832158in}{2.648032in}}%
\pgfpathcurveto{\pgfqpoint{4.836065in}{2.651939in}}{\pgfqpoint{4.838260in}{2.657239in}}{\pgfqpoint{4.838260in}{2.662764in}}%
\pgfpathcurveto{\pgfqpoint{4.838260in}{2.668289in}}{\pgfqpoint{4.836065in}{2.673588in}}{\pgfqpoint{4.832158in}{2.677495in}}%
\pgfpathcurveto{\pgfqpoint{4.828251in}{2.681402in}}{\pgfqpoint{4.822951in}{2.683597in}}{\pgfqpoint{4.817426in}{2.683597in}}%
\pgfpathcurveto{\pgfqpoint{4.811901in}{2.683597in}}{\pgfqpoint{4.806602in}{2.681402in}}{\pgfqpoint{4.802695in}{2.677495in}}%
\pgfpathcurveto{\pgfqpoint{4.798788in}{2.673588in}}{\pgfqpoint{4.796593in}{2.668289in}}{\pgfqpoint{4.796593in}{2.662764in}}%
\pgfpathcurveto{\pgfqpoint{4.796593in}{2.657239in}}{\pgfqpoint{4.798788in}{2.651939in}}{\pgfqpoint{4.802695in}{2.648032in}}%
\pgfpathcurveto{\pgfqpoint{4.806602in}{2.644125in}}{\pgfqpoint{4.811901in}{2.641930in}}{\pgfqpoint{4.817426in}{2.641930in}}%
\pgfpathclose%
\pgfusepath{fill}%
\end{pgfscope}%
\begin{pgfscope}%
\pgfpathrectangle{\pgfqpoint{4.320685in}{2.343185in}}{\pgfqpoint{1.162500in}{0.755000in}} %
\pgfusepath{clip}%
\pgfsetbuttcap%
\pgfsetroundjoin%
\definecolor{currentfill}{rgb}{0.000000,0.000000,0.000000}%
\pgfsetfillcolor{currentfill}%
\pgfsetfillopacity{0.500000}%
\pgfsetlinewidth{0.000000pt}%
\definecolor{currentstroke}{rgb}{0.000000,0.000000,0.000000}%
\pgfsetstrokecolor{currentstroke}%
\pgfsetdash{}{0pt}%
\pgfpathmoveto{\pgfqpoint{5.258215in}{2.975096in}}%
\pgfpathcurveto{\pgfqpoint{5.263740in}{2.975096in}}{\pgfqpoint{5.269040in}{2.977291in}}{\pgfqpoint{5.272947in}{2.981198in}}%
\pgfpathcurveto{\pgfqpoint{5.276853in}{2.985104in}}{\pgfqpoint{5.279048in}{2.990404in}}{\pgfqpoint{5.279048in}{2.995929in}}%
\pgfpathcurveto{\pgfqpoint{5.279048in}{3.001454in}}{\pgfqpoint{5.276853in}{3.006754in}}{\pgfqpoint{5.272947in}{3.010660in}}%
\pgfpathcurveto{\pgfqpoint{5.269040in}{3.014567in}}{\pgfqpoint{5.263740in}{3.016762in}}{\pgfqpoint{5.258215in}{3.016762in}}%
\pgfpathcurveto{\pgfqpoint{5.252690in}{3.016762in}}{\pgfqpoint{5.247391in}{3.014567in}}{\pgfqpoint{5.243484in}{3.010660in}}%
\pgfpathcurveto{\pgfqpoint{5.239577in}{3.006754in}}{\pgfqpoint{5.237382in}{3.001454in}}{\pgfqpoint{5.237382in}{2.995929in}}%
\pgfpathcurveto{\pgfqpoint{5.237382in}{2.990404in}}{\pgfqpoint{5.239577in}{2.985104in}}{\pgfqpoint{5.243484in}{2.981198in}}%
\pgfpathcurveto{\pgfqpoint{5.247391in}{2.977291in}}{\pgfqpoint{5.252690in}{2.975096in}}{\pgfqpoint{5.258215in}{2.975096in}}%
\pgfpathclose%
\pgfusepath{fill}%
\end{pgfscope}%
\begin{pgfscope}%
\pgfpathrectangle{\pgfqpoint{4.320685in}{2.343185in}}{\pgfqpoint{1.162500in}{0.755000in}} %
\pgfusepath{clip}%
\pgfsetbuttcap%
\pgfsetroundjoin%
\definecolor{currentfill}{rgb}{0.000000,0.000000,0.000000}%
\pgfsetfillcolor{currentfill}%
\pgfsetfillopacity{0.500000}%
\pgfsetlinewidth{0.000000pt}%
\definecolor{currentstroke}{rgb}{0.000000,0.000000,0.000000}%
\pgfsetstrokecolor{currentstroke}%
\pgfsetdash{}{0pt}%
\pgfpathmoveto{\pgfqpoint{4.803827in}{2.531911in}}%
\pgfpathcurveto{\pgfqpoint{4.809352in}{2.531911in}}{\pgfqpoint{4.814651in}{2.534106in}}{\pgfqpoint{4.818558in}{2.538013in}}%
\pgfpathcurveto{\pgfqpoint{4.822465in}{2.541920in}}{\pgfqpoint{4.824660in}{2.547220in}}{\pgfqpoint{4.824660in}{2.552745in}}%
\pgfpathcurveto{\pgfqpoint{4.824660in}{2.558270in}}{\pgfqpoint{4.822465in}{2.563569in}}{\pgfqpoint{4.818558in}{2.567476in}}%
\pgfpathcurveto{\pgfqpoint{4.814651in}{2.571383in}}{\pgfqpoint{4.809352in}{2.573578in}}{\pgfqpoint{4.803827in}{2.573578in}}%
\pgfpathcurveto{\pgfqpoint{4.798302in}{2.573578in}}{\pgfqpoint{4.793002in}{2.571383in}}{\pgfqpoint{4.789096in}{2.567476in}}%
\pgfpathcurveto{\pgfqpoint{4.785189in}{2.563569in}}{\pgfqpoint{4.782994in}{2.558270in}}{\pgfqpoint{4.782994in}{2.552745in}}%
\pgfpathcurveto{\pgfqpoint{4.782994in}{2.547220in}}{\pgfqpoint{4.785189in}{2.541920in}}{\pgfqpoint{4.789096in}{2.538013in}}%
\pgfpathcurveto{\pgfqpoint{4.793002in}{2.534106in}}{\pgfqpoint{4.798302in}{2.531911in}}{\pgfqpoint{4.803827in}{2.531911in}}%
\pgfpathclose%
\pgfusepath{fill}%
\end{pgfscope}%
\begin{pgfscope}%
\pgfpathrectangle{\pgfqpoint{4.320685in}{2.343185in}}{\pgfqpoint{1.162500in}{0.755000in}} %
\pgfusepath{clip}%
\pgfsetbuttcap%
\pgfsetroundjoin%
\definecolor{currentfill}{rgb}{0.000000,0.000000,0.000000}%
\pgfsetfillcolor{currentfill}%
\pgfsetfillopacity{0.500000}%
\pgfsetlinewidth{0.000000pt}%
\definecolor{currentstroke}{rgb}{0.000000,0.000000,0.000000}%
\pgfsetstrokecolor{currentstroke}%
\pgfsetdash{}{0pt}%
\pgfpathmoveto{\pgfqpoint{4.437194in}{2.591601in}}%
\pgfpathcurveto{\pgfqpoint{4.442719in}{2.591601in}}{\pgfqpoint{4.448018in}{2.593796in}}{\pgfqpoint{4.451925in}{2.597703in}}%
\pgfpathcurveto{\pgfqpoint{4.455832in}{2.601609in}}{\pgfqpoint{4.458027in}{2.606909in}}{\pgfqpoint{4.458027in}{2.612434in}}%
\pgfpathcurveto{\pgfqpoint{4.458027in}{2.617959in}}{\pgfqpoint{4.455832in}{2.623259in}}{\pgfqpoint{4.451925in}{2.627165in}}%
\pgfpathcurveto{\pgfqpoint{4.448018in}{2.631072in}}{\pgfqpoint{4.442719in}{2.633267in}}{\pgfqpoint{4.437194in}{2.633267in}}%
\pgfpathcurveto{\pgfqpoint{4.431669in}{2.633267in}}{\pgfqpoint{4.426369in}{2.631072in}}{\pgfqpoint{4.422462in}{2.627165in}}%
\pgfpathcurveto{\pgfqpoint{4.418556in}{2.623259in}}{\pgfqpoint{4.416360in}{2.617959in}}{\pgfqpoint{4.416360in}{2.612434in}}%
\pgfpathcurveto{\pgfqpoint{4.416360in}{2.606909in}}{\pgfqpoint{4.418556in}{2.601609in}}{\pgfqpoint{4.422462in}{2.597703in}}%
\pgfpathcurveto{\pgfqpoint{4.426369in}{2.593796in}}{\pgfqpoint{4.431669in}{2.591601in}}{\pgfqpoint{4.437194in}{2.591601in}}%
\pgfpathclose%
\pgfusepath{fill}%
\end{pgfscope}%
\begin{pgfscope}%
\pgfpathrectangle{\pgfqpoint{4.320685in}{2.343185in}}{\pgfqpoint{1.162500in}{0.755000in}} %
\pgfusepath{clip}%
\pgfsetbuttcap%
\pgfsetroundjoin%
\definecolor{currentfill}{rgb}{0.000000,0.000000,0.000000}%
\pgfsetfillcolor{currentfill}%
\pgfsetfillopacity{0.500000}%
\pgfsetlinewidth{0.000000pt}%
\definecolor{currentstroke}{rgb}{0.000000,0.000000,0.000000}%
\pgfsetstrokecolor{currentstroke}%
\pgfsetdash{}{0pt}%
\pgfpathmoveto{\pgfqpoint{5.262967in}{2.814070in}}%
\pgfpathcurveto{\pgfqpoint{5.268492in}{2.814070in}}{\pgfqpoint{5.273791in}{2.816265in}}{\pgfqpoint{5.277698in}{2.820172in}}%
\pgfpathcurveto{\pgfqpoint{5.281605in}{2.824079in}}{\pgfqpoint{5.283800in}{2.829378in}}{\pgfqpoint{5.283800in}{2.834904in}}%
\pgfpathcurveto{\pgfqpoint{5.283800in}{2.840429in}}{\pgfqpoint{5.281605in}{2.845728in}}{\pgfqpoint{5.277698in}{2.849635in}}%
\pgfpathcurveto{\pgfqpoint{5.273791in}{2.853542in}}{\pgfqpoint{5.268492in}{2.855737in}}{\pgfqpoint{5.262967in}{2.855737in}}%
\pgfpathcurveto{\pgfqpoint{5.257442in}{2.855737in}}{\pgfqpoint{5.252142in}{2.853542in}}{\pgfqpoint{5.248235in}{2.849635in}}%
\pgfpathcurveto{\pgfqpoint{5.244329in}{2.845728in}}{\pgfqpoint{5.242134in}{2.840429in}}{\pgfqpoint{5.242134in}{2.834904in}}%
\pgfpathcurveto{\pgfqpoint{5.242134in}{2.829378in}}{\pgfqpoint{5.244329in}{2.824079in}}{\pgfqpoint{5.248235in}{2.820172in}}%
\pgfpathcurveto{\pgfqpoint{5.252142in}{2.816265in}}{\pgfqpoint{5.257442in}{2.814070in}}{\pgfqpoint{5.262967in}{2.814070in}}%
\pgfpathclose%
\pgfusepath{fill}%
\end{pgfscope}%
\begin{pgfscope}%
\pgfpathrectangle{\pgfqpoint{4.320685in}{2.343185in}}{\pgfqpoint{1.162500in}{0.755000in}} %
\pgfusepath{clip}%
\pgfsetbuttcap%
\pgfsetroundjoin%
\definecolor{currentfill}{rgb}{0.000000,0.000000,0.000000}%
\pgfsetfillcolor{currentfill}%
\pgfsetfillopacity{0.500000}%
\pgfsetlinewidth{0.000000pt}%
\definecolor{currentstroke}{rgb}{0.000000,0.000000,0.000000}%
\pgfsetstrokecolor{currentstroke}%
\pgfsetdash{}{0pt}%
\pgfpathmoveto{\pgfqpoint{4.896922in}{2.392267in}}%
\pgfpathcurveto{\pgfqpoint{4.902447in}{2.392267in}}{\pgfqpoint{4.907746in}{2.394462in}}{\pgfqpoint{4.911653in}{2.398369in}}%
\pgfpathcurveto{\pgfqpoint{4.915560in}{2.402276in}}{\pgfqpoint{4.917755in}{2.407575in}}{\pgfqpoint{4.917755in}{2.413101in}}%
\pgfpathcurveto{\pgfqpoint{4.917755in}{2.418626in}}{\pgfqpoint{4.915560in}{2.423925in}}{\pgfqpoint{4.911653in}{2.427832in}}%
\pgfpathcurveto{\pgfqpoint{4.907746in}{2.431739in}}{\pgfqpoint{4.902447in}{2.433934in}}{\pgfqpoint{4.896922in}{2.433934in}}%
\pgfpathcurveto{\pgfqpoint{4.891397in}{2.433934in}}{\pgfqpoint{4.886097in}{2.431739in}}{\pgfqpoint{4.882190in}{2.427832in}}%
\pgfpathcurveto{\pgfqpoint{4.878284in}{2.423925in}}{\pgfqpoint{4.876088in}{2.418626in}}{\pgfqpoint{4.876088in}{2.413101in}}%
\pgfpathcurveto{\pgfqpoint{4.876088in}{2.407575in}}{\pgfqpoint{4.878284in}{2.402276in}}{\pgfqpoint{4.882190in}{2.398369in}}%
\pgfpathcurveto{\pgfqpoint{4.886097in}{2.394462in}}{\pgfqpoint{4.891397in}{2.392267in}}{\pgfqpoint{4.896922in}{2.392267in}}%
\pgfpathclose%
\pgfusepath{fill}%
\end{pgfscope}%
\begin{pgfscope}%
\pgfsetrectcap%
\pgfsetmiterjoin%
\pgfsetlinewidth{0.803000pt}%
\definecolor{currentstroke}{rgb}{0.000000,0.000000,0.000000}%
\pgfsetstrokecolor{currentstroke}%
\pgfsetdash{}{0pt}%
\pgfpathmoveto{\pgfqpoint{4.320685in}{2.343185in}}%
\pgfpathlineto{\pgfqpoint{4.320685in}{3.098185in}}%
\pgfusepath{stroke}%
\end{pgfscope}%
\begin{pgfscope}%
\pgfsetrectcap%
\pgfsetmiterjoin%
\pgfsetlinewidth{0.803000pt}%
\definecolor{currentstroke}{rgb}{0.000000,0.000000,0.000000}%
\pgfsetstrokecolor{currentstroke}%
\pgfsetdash{}{0pt}%
\pgfpathmoveto{\pgfqpoint{5.483185in}{2.343185in}}%
\pgfpathlineto{\pgfqpoint{5.483185in}{3.098185in}}%
\pgfusepath{stroke}%
\end{pgfscope}%
\begin{pgfscope}%
\pgfsetrectcap%
\pgfsetmiterjoin%
\pgfsetlinewidth{0.803000pt}%
\definecolor{currentstroke}{rgb}{0.000000,0.000000,0.000000}%
\pgfsetstrokecolor{currentstroke}%
\pgfsetdash{}{0pt}%
\pgfpathmoveto{\pgfqpoint{4.320685in}{2.343185in}}%
\pgfpathlineto{\pgfqpoint{5.483185in}{2.343185in}}%
\pgfusepath{stroke}%
\end{pgfscope}%
\begin{pgfscope}%
\pgfsetrectcap%
\pgfsetmiterjoin%
\pgfsetlinewidth{0.803000pt}%
\definecolor{currentstroke}{rgb}{0.000000,0.000000,0.000000}%
\pgfsetstrokecolor{currentstroke}%
\pgfsetdash{}{0pt}%
\pgfpathmoveto{\pgfqpoint{4.320685in}{3.098185in}}%
\pgfpathlineto{\pgfqpoint{5.483185in}{3.098185in}}%
\pgfusepath{stroke}%
\end{pgfscope}%
\begin{pgfscope}%
\pgfsetbuttcap%
\pgfsetmiterjoin%
\definecolor{currentfill}{rgb}{1.000000,1.000000,1.000000}%
\pgfsetfillcolor{currentfill}%
\pgfsetlinewidth{0.000000pt}%
\definecolor{currentstroke}{rgb}{0.000000,0.000000,0.000000}%
\pgfsetstrokecolor{currentstroke}%
\pgfsetstrokeopacity{0.000000}%
\pgfsetdash{}{0pt}%
\pgfpathmoveto{\pgfqpoint{0.833185in}{1.588185in}}%
\pgfpathlineto{\pgfqpoint{1.995685in}{1.588185in}}%
\pgfpathlineto{\pgfqpoint{1.995685in}{2.343185in}}%
\pgfpathlineto{\pgfqpoint{0.833185in}{2.343185in}}%
\pgfpathclose%
\pgfusepath{fill}%
\end{pgfscope}%
\begin{pgfscope}%
\pgfpathrectangle{\pgfqpoint{0.833185in}{1.588185in}}{\pgfqpoint{1.162500in}{0.755000in}} %
\pgfusepath{clip}%
\pgfsetbuttcap%
\pgfsetroundjoin%
\definecolor{currentfill}{rgb}{0.000000,0.000000,0.000000}%
\pgfsetfillcolor{currentfill}%
\pgfsetfillopacity{0.500000}%
\pgfsetlinewidth{0.000000pt}%
\definecolor{currentstroke}{rgb}{0.000000,0.000000,0.000000}%
\pgfsetstrokecolor{currentstroke}%
\pgfsetdash{}{0pt}%
\pgfpathmoveto{\pgfqpoint{1.779018in}{2.304375in}}%
\pgfpathcurveto{\pgfqpoint{1.784543in}{2.304375in}}{\pgfqpoint{1.789842in}{2.306570in}}{\pgfqpoint{1.793749in}{2.310477in}}%
\pgfpathcurveto{\pgfqpoint{1.797656in}{2.314384in}}{\pgfqpoint{1.799851in}{2.319683in}}{\pgfqpoint{1.799851in}{2.325208in}}%
\pgfpathcurveto{\pgfqpoint{1.799851in}{2.330733in}}{\pgfqpoint{1.797656in}{2.336033in}}{\pgfqpoint{1.793749in}{2.339940in}}%
\pgfpathcurveto{\pgfqpoint{1.789842in}{2.343847in}}{\pgfqpoint{1.784543in}{2.346042in}}{\pgfqpoint{1.779018in}{2.346042in}}%
\pgfpathcurveto{\pgfqpoint{1.773492in}{2.346042in}}{\pgfqpoint{1.768193in}{2.343847in}}{\pgfqpoint{1.764286in}{2.339940in}}%
\pgfpathcurveto{\pgfqpoint{1.760379in}{2.336033in}}{\pgfqpoint{1.758184in}{2.330733in}}{\pgfqpoint{1.758184in}{2.325208in}}%
\pgfpathcurveto{\pgfqpoint{1.758184in}{2.319683in}}{\pgfqpoint{1.760379in}{2.314384in}}{\pgfqpoint{1.764286in}{2.310477in}}%
\pgfpathcurveto{\pgfqpoint{1.768193in}{2.306570in}}{\pgfqpoint{1.773492in}{2.304375in}}{\pgfqpoint{1.779018in}{2.304375in}}%
\pgfpathclose%
\pgfusepath{fill}%
\end{pgfscope}%
\begin{pgfscope}%
\pgfpathrectangle{\pgfqpoint{0.833185in}{1.588185in}}{\pgfqpoint{1.162500in}{0.755000in}} %
\pgfusepath{clip}%
\pgfsetbuttcap%
\pgfsetroundjoin%
\definecolor{currentfill}{rgb}{0.000000,0.000000,0.000000}%
\pgfsetfillcolor{currentfill}%
\pgfsetfillopacity{0.500000}%
\pgfsetlinewidth{0.000000pt}%
\definecolor{currentstroke}{rgb}{0.000000,0.000000,0.000000}%
\pgfsetstrokecolor{currentstroke}%
\pgfsetdash{}{0pt}%
\pgfpathmoveto{\pgfqpoint{0.920858in}{1.845741in}}%
\pgfpathcurveto{\pgfqpoint{0.926383in}{1.845741in}}{\pgfqpoint{0.931683in}{1.847936in}}{\pgfqpoint{0.935589in}{1.851843in}}%
\pgfpathcurveto{\pgfqpoint{0.939496in}{1.855750in}}{\pgfqpoint{0.941691in}{1.861049in}}{\pgfqpoint{0.941691in}{1.866574in}}%
\pgfpathcurveto{\pgfqpoint{0.941691in}{1.872099in}}{\pgfqpoint{0.939496in}{1.877399in}}{\pgfqpoint{0.935589in}{1.881306in}}%
\pgfpathcurveto{\pgfqpoint{0.931683in}{1.885212in}}{\pgfqpoint{0.926383in}{1.887408in}}{\pgfqpoint{0.920858in}{1.887408in}}%
\pgfpathcurveto{\pgfqpoint{0.915333in}{1.887408in}}{\pgfqpoint{0.910034in}{1.885212in}}{\pgfqpoint{0.906127in}{1.881306in}}%
\pgfpathcurveto{\pgfqpoint{0.902220in}{1.877399in}}{\pgfqpoint{0.900025in}{1.872099in}}{\pgfqpoint{0.900025in}{1.866574in}}%
\pgfpathcurveto{\pgfqpoint{0.900025in}{1.861049in}}{\pgfqpoint{0.902220in}{1.855750in}}{\pgfqpoint{0.906127in}{1.851843in}}%
\pgfpathcurveto{\pgfqpoint{0.910034in}{1.847936in}}{\pgfqpoint{0.915333in}{1.845741in}}{\pgfqpoint{0.920858in}{1.845741in}}%
\pgfpathclose%
\pgfusepath{fill}%
\end{pgfscope}%
\begin{pgfscope}%
\pgfpathrectangle{\pgfqpoint{0.833185in}{1.588185in}}{\pgfqpoint{1.162500in}{0.755000in}} %
\pgfusepath{clip}%
\pgfsetbuttcap%
\pgfsetroundjoin%
\definecolor{currentfill}{rgb}{0.000000,0.000000,0.000000}%
\pgfsetfillcolor{currentfill}%
\pgfsetfillopacity{0.500000}%
\pgfsetlinewidth{0.000000pt}%
\definecolor{currentstroke}{rgb}{0.000000,0.000000,0.000000}%
\pgfsetstrokecolor{currentstroke}%
\pgfsetdash{}{0pt}%
\pgfpathmoveto{\pgfqpoint{0.860863in}{1.840570in}}%
\pgfpathcurveto{\pgfqpoint{0.866388in}{1.840570in}}{\pgfqpoint{0.871688in}{1.842766in}}{\pgfqpoint{0.875595in}{1.846672in}}%
\pgfpathcurveto{\pgfqpoint{0.879501in}{1.850579in}}{\pgfqpoint{0.881696in}{1.855879in}}{\pgfqpoint{0.881696in}{1.861404in}}%
\pgfpathcurveto{\pgfqpoint{0.881696in}{1.866929in}}{\pgfqpoint{0.879501in}{1.872228in}}{\pgfqpoint{0.875595in}{1.876135in}}%
\pgfpathcurveto{\pgfqpoint{0.871688in}{1.880042in}}{\pgfqpoint{0.866388in}{1.882237in}}{\pgfqpoint{0.860863in}{1.882237in}}%
\pgfpathcurveto{\pgfqpoint{0.855338in}{1.882237in}}{\pgfqpoint{0.850039in}{1.880042in}}{\pgfqpoint{0.846132in}{1.876135in}}%
\pgfpathcurveto{\pgfqpoint{0.842225in}{1.872228in}}{\pgfqpoint{0.840030in}{1.866929in}}{\pgfqpoint{0.840030in}{1.861404in}}%
\pgfpathcurveto{\pgfqpoint{0.840030in}{1.855879in}}{\pgfqpoint{0.842225in}{1.850579in}}{\pgfqpoint{0.846132in}{1.846672in}}%
\pgfpathcurveto{\pgfqpoint{0.850039in}{1.842766in}}{\pgfqpoint{0.855338in}{1.840570in}}{\pgfqpoint{0.860863in}{1.840570in}}%
\pgfpathclose%
\pgfusepath{fill}%
\end{pgfscope}%
\begin{pgfscope}%
\pgfpathrectangle{\pgfqpoint{0.833185in}{1.588185in}}{\pgfqpoint{1.162500in}{0.755000in}} %
\pgfusepath{clip}%
\pgfsetbuttcap%
\pgfsetroundjoin%
\definecolor{currentfill}{rgb}{0.000000,0.000000,0.000000}%
\pgfsetfillcolor{currentfill}%
\pgfsetfillopacity{0.500000}%
\pgfsetlinewidth{0.000000pt}%
\definecolor{currentstroke}{rgb}{0.000000,0.000000,0.000000}%
\pgfsetstrokecolor{currentstroke}%
\pgfsetdash{}{0pt}%
\pgfpathmoveto{\pgfqpoint{1.180359in}{1.680932in}}%
\pgfpathcurveto{\pgfqpoint{1.185884in}{1.680932in}}{\pgfqpoint{1.191184in}{1.683127in}}{\pgfqpoint{1.195090in}{1.687034in}}%
\pgfpathcurveto{\pgfqpoint{1.198997in}{1.690941in}}{\pgfqpoint{1.201192in}{1.696241in}}{\pgfqpoint{1.201192in}{1.701766in}}%
\pgfpathcurveto{\pgfqpoint{1.201192in}{1.707291in}}{\pgfqpoint{1.198997in}{1.712590in}}{\pgfqpoint{1.195090in}{1.716497in}}%
\pgfpathcurveto{\pgfqpoint{1.191184in}{1.720404in}}{\pgfqpoint{1.185884in}{1.722599in}}{\pgfqpoint{1.180359in}{1.722599in}}%
\pgfpathcurveto{\pgfqpoint{1.174834in}{1.722599in}}{\pgfqpoint{1.169535in}{1.720404in}}{\pgfqpoint{1.165628in}{1.716497in}}%
\pgfpathcurveto{\pgfqpoint{1.161721in}{1.712590in}}{\pgfqpoint{1.159526in}{1.707291in}}{\pgfqpoint{1.159526in}{1.701766in}}%
\pgfpathcurveto{\pgfqpoint{1.159526in}{1.696241in}}{\pgfqpoint{1.161721in}{1.690941in}}{\pgfqpoint{1.165628in}{1.687034in}}%
\pgfpathcurveto{\pgfqpoint{1.169535in}{1.683127in}}{\pgfqpoint{1.174834in}{1.680932in}}{\pgfqpoint{1.180359in}{1.680932in}}%
\pgfpathclose%
\pgfusepath{fill}%
\end{pgfscope}%
\begin{pgfscope}%
\pgfpathrectangle{\pgfqpoint{0.833185in}{1.588185in}}{\pgfqpoint{1.162500in}{0.755000in}} %
\pgfusepath{clip}%
\pgfsetbuttcap%
\pgfsetroundjoin%
\definecolor{currentfill}{rgb}{0.000000,0.000000,0.000000}%
\pgfsetfillcolor{currentfill}%
\pgfsetfillopacity{0.500000}%
\pgfsetlinewidth{0.000000pt}%
\definecolor{currentstroke}{rgb}{0.000000,0.000000,0.000000}%
\pgfsetstrokecolor{currentstroke}%
\pgfsetdash{}{0pt}%
\pgfpathmoveto{\pgfqpoint{0.938297in}{1.656866in}}%
\pgfpathcurveto{\pgfqpoint{0.943822in}{1.656866in}}{\pgfqpoint{0.949121in}{1.659061in}}{\pgfqpoint{0.953028in}{1.662968in}}%
\pgfpathcurveto{\pgfqpoint{0.956935in}{1.666874in}}{\pgfqpoint{0.959130in}{1.672174in}}{\pgfqpoint{0.959130in}{1.677699in}}%
\pgfpathcurveto{\pgfqpoint{0.959130in}{1.683224in}}{\pgfqpoint{0.956935in}{1.688523in}}{\pgfqpoint{0.953028in}{1.692430in}}%
\pgfpathcurveto{\pgfqpoint{0.949121in}{1.696337in}}{\pgfqpoint{0.943822in}{1.698532in}}{\pgfqpoint{0.938297in}{1.698532in}}%
\pgfpathcurveto{\pgfqpoint{0.932772in}{1.698532in}}{\pgfqpoint{0.927472in}{1.696337in}}{\pgfqpoint{0.923565in}{1.692430in}}%
\pgfpathcurveto{\pgfqpoint{0.919659in}{1.688523in}}{\pgfqpoint{0.917464in}{1.683224in}}{\pgfqpoint{0.917464in}{1.677699in}}%
\pgfpathcurveto{\pgfqpoint{0.917464in}{1.672174in}}{\pgfqpoint{0.919659in}{1.666874in}}{\pgfqpoint{0.923565in}{1.662968in}}%
\pgfpathcurveto{\pgfqpoint{0.927472in}{1.659061in}}{\pgfqpoint{0.932772in}{1.656866in}}{\pgfqpoint{0.938297in}{1.656866in}}%
\pgfpathclose%
\pgfusepath{fill}%
\end{pgfscope}%
\begin{pgfscope}%
\pgfpathrectangle{\pgfqpoint{0.833185in}{1.588185in}}{\pgfqpoint{1.162500in}{0.755000in}} %
\pgfusepath{clip}%
\pgfsetbuttcap%
\pgfsetroundjoin%
\definecolor{currentfill}{rgb}{0.000000,0.000000,0.000000}%
\pgfsetfillcolor{currentfill}%
\pgfsetfillopacity{0.500000}%
\pgfsetlinewidth{0.000000pt}%
\definecolor{currentstroke}{rgb}{0.000000,0.000000,0.000000}%
\pgfsetstrokecolor{currentstroke}%
\pgfsetdash{}{0pt}%
\pgfpathmoveto{\pgfqpoint{1.098900in}{1.598883in}}%
\pgfpathcurveto{\pgfqpoint{1.104426in}{1.598883in}}{\pgfqpoint{1.109725in}{1.601078in}}{\pgfqpoint{1.113632in}{1.604985in}}%
\pgfpathcurveto{\pgfqpoint{1.117539in}{1.608892in}}{\pgfqpoint{1.119734in}{1.614191in}}{\pgfqpoint{1.119734in}{1.619716in}}%
\pgfpathcurveto{\pgfqpoint{1.119734in}{1.625241in}}{\pgfqpoint{1.117539in}{1.630541in}}{\pgfqpoint{1.113632in}{1.634447in}}%
\pgfpathcurveto{\pgfqpoint{1.109725in}{1.638354in}}{\pgfqpoint{1.104426in}{1.640549in}}{\pgfqpoint{1.098900in}{1.640549in}}%
\pgfpathcurveto{\pgfqpoint{1.093375in}{1.640549in}}{\pgfqpoint{1.088076in}{1.638354in}}{\pgfqpoint{1.084169in}{1.634447in}}%
\pgfpathcurveto{\pgfqpoint{1.080262in}{1.630541in}}{\pgfqpoint{1.078067in}{1.625241in}}{\pgfqpoint{1.078067in}{1.619716in}}%
\pgfpathcurveto{\pgfqpoint{1.078067in}{1.614191in}}{\pgfqpoint{1.080262in}{1.608892in}}{\pgfqpoint{1.084169in}{1.604985in}}%
\pgfpathcurveto{\pgfqpoint{1.088076in}{1.601078in}}{\pgfqpoint{1.093375in}{1.598883in}}{\pgfqpoint{1.098900in}{1.598883in}}%
\pgfpathclose%
\pgfusepath{fill}%
\end{pgfscope}%
\begin{pgfscope}%
\pgfpathrectangle{\pgfqpoint{0.833185in}{1.588185in}}{\pgfqpoint{1.162500in}{0.755000in}} %
\pgfusepath{clip}%
\pgfsetbuttcap%
\pgfsetroundjoin%
\definecolor{currentfill}{rgb}{0.000000,0.000000,0.000000}%
\pgfsetfillcolor{currentfill}%
\pgfsetfillopacity{0.500000}%
\pgfsetlinewidth{0.000000pt}%
\definecolor{currentstroke}{rgb}{0.000000,0.000000,0.000000}%
\pgfsetstrokecolor{currentstroke}%
\pgfsetdash{}{0pt}%
\pgfpathmoveto{\pgfqpoint{1.085127in}{1.585327in}}%
\pgfpathcurveto{\pgfqpoint{1.090652in}{1.585327in}}{\pgfqpoint{1.095951in}{1.587523in}}{\pgfqpoint{1.099858in}{1.591429in}}%
\pgfpathcurveto{\pgfqpoint{1.103765in}{1.595336in}}{\pgfqpoint{1.105960in}{1.600636in}}{\pgfqpoint{1.105960in}{1.606161in}}%
\pgfpathcurveto{\pgfqpoint{1.105960in}{1.611686in}}{\pgfqpoint{1.103765in}{1.616985in}}{\pgfqpoint{1.099858in}{1.620892in}}%
\pgfpathcurveto{\pgfqpoint{1.095951in}{1.624799in}}{\pgfqpoint{1.090652in}{1.626994in}}{\pgfqpoint{1.085127in}{1.626994in}}%
\pgfpathcurveto{\pgfqpoint{1.079602in}{1.626994in}}{\pgfqpoint{1.074302in}{1.624799in}}{\pgfqpoint{1.070395in}{1.620892in}}%
\pgfpathcurveto{\pgfqpoint{1.066488in}{1.616985in}}{\pgfqpoint{1.064293in}{1.611686in}}{\pgfqpoint{1.064293in}{1.606161in}}%
\pgfpathcurveto{\pgfqpoint{1.064293in}{1.600636in}}{\pgfqpoint{1.066488in}{1.595336in}}{\pgfqpoint{1.070395in}{1.591429in}}%
\pgfpathcurveto{\pgfqpoint{1.074302in}{1.587523in}}{\pgfqpoint{1.079602in}{1.585327in}}{\pgfqpoint{1.085127in}{1.585327in}}%
\pgfpathclose%
\pgfusepath{fill}%
\end{pgfscope}%
\begin{pgfscope}%
\pgfpathrectangle{\pgfqpoint{0.833185in}{1.588185in}}{\pgfqpoint{1.162500in}{0.755000in}} %
\pgfusepath{clip}%
\pgfsetbuttcap%
\pgfsetroundjoin%
\definecolor{currentfill}{rgb}{0.000000,0.000000,0.000000}%
\pgfsetfillcolor{currentfill}%
\pgfsetfillopacity{0.500000}%
\pgfsetlinewidth{0.000000pt}%
\definecolor{currentstroke}{rgb}{0.000000,0.000000,0.000000}%
\pgfsetstrokecolor{currentstroke}%
\pgfsetdash{}{0pt}%
\pgfpathmoveto{\pgfqpoint{1.896712in}{1.989747in}}%
\pgfpathcurveto{\pgfqpoint{1.902237in}{1.989747in}}{\pgfqpoint{1.907537in}{1.991942in}}{\pgfqpoint{1.911443in}{1.995849in}}%
\pgfpathcurveto{\pgfqpoint{1.915350in}{1.999756in}}{\pgfqpoint{1.917545in}{2.005055in}}{\pgfqpoint{1.917545in}{2.010580in}}%
\pgfpathcurveto{\pgfqpoint{1.917545in}{2.016105in}}{\pgfqpoint{1.915350in}{2.021405in}}{\pgfqpoint{1.911443in}{2.025311in}}%
\pgfpathcurveto{\pgfqpoint{1.907537in}{2.029218in}}{\pgfqpoint{1.902237in}{2.031413in}}{\pgfqpoint{1.896712in}{2.031413in}}%
\pgfpathcurveto{\pgfqpoint{1.891187in}{2.031413in}}{\pgfqpoint{1.885887in}{2.029218in}}{\pgfqpoint{1.881981in}{2.025311in}}%
\pgfpathcurveto{\pgfqpoint{1.878074in}{2.021405in}}{\pgfqpoint{1.875879in}{2.016105in}}{\pgfqpoint{1.875879in}{2.010580in}}%
\pgfpathcurveto{\pgfqpoint{1.875879in}{2.005055in}}{\pgfqpoint{1.878074in}{1.999756in}}{\pgfqpoint{1.881981in}{1.995849in}}%
\pgfpathcurveto{\pgfqpoint{1.885887in}{1.991942in}}{\pgfqpoint{1.891187in}{1.989747in}}{\pgfqpoint{1.896712in}{1.989747in}}%
\pgfpathclose%
\pgfusepath{fill}%
\end{pgfscope}%
\begin{pgfscope}%
\pgfpathrectangle{\pgfqpoint{0.833185in}{1.588185in}}{\pgfqpoint{1.162500in}{0.755000in}} %
\pgfusepath{clip}%
\pgfsetbuttcap%
\pgfsetroundjoin%
\definecolor{currentfill}{rgb}{0.000000,0.000000,0.000000}%
\pgfsetfillcolor{currentfill}%
\pgfsetfillopacity{0.500000}%
\pgfsetlinewidth{0.000000pt}%
\definecolor{currentstroke}{rgb}{0.000000,0.000000,0.000000}%
\pgfsetstrokecolor{currentstroke}%
\pgfsetdash{}{0pt}%
\pgfpathmoveto{\pgfqpoint{1.968006in}{1.867889in}}%
\pgfpathcurveto{\pgfqpoint{1.973531in}{1.867889in}}{\pgfqpoint{1.978831in}{1.870084in}}{\pgfqpoint{1.982737in}{1.873991in}}%
\pgfpathcurveto{\pgfqpoint{1.986644in}{1.877898in}}{\pgfqpoint{1.988839in}{1.883197in}}{\pgfqpoint{1.988839in}{1.888722in}}%
\pgfpathcurveto{\pgfqpoint{1.988839in}{1.894247in}}{\pgfqpoint{1.986644in}{1.899547in}}{\pgfqpoint{1.982737in}{1.903454in}}%
\pgfpathcurveto{\pgfqpoint{1.978831in}{1.907360in}}{\pgfqpoint{1.973531in}{1.909556in}}{\pgfqpoint{1.968006in}{1.909556in}}%
\pgfpathcurveto{\pgfqpoint{1.962481in}{1.909556in}}{\pgfqpoint{1.957181in}{1.907360in}}{\pgfqpoint{1.953275in}{1.903454in}}%
\pgfpathcurveto{\pgfqpoint{1.949368in}{1.899547in}}{\pgfqpoint{1.947173in}{1.894247in}}{\pgfqpoint{1.947173in}{1.888722in}}%
\pgfpathcurveto{\pgfqpoint{1.947173in}{1.883197in}}{\pgfqpoint{1.949368in}{1.877898in}}{\pgfqpoint{1.953275in}{1.873991in}}%
\pgfpathcurveto{\pgfqpoint{1.957181in}{1.870084in}}{\pgfqpoint{1.962481in}{1.867889in}}{\pgfqpoint{1.968006in}{1.867889in}}%
\pgfpathclose%
\pgfusepath{fill}%
\end{pgfscope}%
\begin{pgfscope}%
\pgfpathrectangle{\pgfqpoint{0.833185in}{1.588185in}}{\pgfqpoint{1.162500in}{0.755000in}} %
\pgfusepath{clip}%
\pgfsetbuttcap%
\pgfsetroundjoin%
\definecolor{currentfill}{rgb}{0.000000,0.000000,0.000000}%
\pgfsetfillcolor{currentfill}%
\pgfsetfillopacity{0.500000}%
\pgfsetlinewidth{0.000000pt}%
\definecolor{currentstroke}{rgb}{0.000000,0.000000,0.000000}%
\pgfsetstrokecolor{currentstroke}%
\pgfsetdash{}{0pt}%
\pgfpathmoveto{\pgfqpoint{1.742203in}{1.854418in}}%
\pgfpathcurveto{\pgfqpoint{1.747728in}{1.854418in}}{\pgfqpoint{1.753027in}{1.856613in}}{\pgfqpoint{1.756934in}{1.860520in}}%
\pgfpathcurveto{\pgfqpoint{1.760841in}{1.864427in}}{\pgfqpoint{1.763036in}{1.869726in}}{\pgfqpoint{1.763036in}{1.875251in}}%
\pgfpathcurveto{\pgfqpoint{1.763036in}{1.880776in}}{\pgfqpoint{1.760841in}{1.886076in}}{\pgfqpoint{1.756934in}{1.889982in}}%
\pgfpathcurveto{\pgfqpoint{1.753027in}{1.893889in}}{\pgfqpoint{1.747728in}{1.896084in}}{\pgfqpoint{1.742203in}{1.896084in}}%
\pgfpathcurveto{\pgfqpoint{1.736677in}{1.896084in}}{\pgfqpoint{1.731378in}{1.893889in}}{\pgfqpoint{1.727471in}{1.889982in}}%
\pgfpathcurveto{\pgfqpoint{1.723564in}{1.886076in}}{\pgfqpoint{1.721369in}{1.880776in}}{\pgfqpoint{1.721369in}{1.875251in}}%
\pgfpathcurveto{\pgfqpoint{1.721369in}{1.869726in}}{\pgfqpoint{1.723564in}{1.864427in}}{\pgfqpoint{1.727471in}{1.860520in}}%
\pgfpathcurveto{\pgfqpoint{1.731378in}{1.856613in}}{\pgfqpoint{1.736677in}{1.854418in}}{\pgfqpoint{1.742203in}{1.854418in}}%
\pgfpathclose%
\pgfusepath{fill}%
\end{pgfscope}%
\begin{pgfscope}%
\pgfpathrectangle{\pgfqpoint{0.833185in}{1.588185in}}{\pgfqpoint{1.162500in}{0.755000in}} %
\pgfusepath{clip}%
\pgfsetbuttcap%
\pgfsetroundjoin%
\definecolor{currentfill}{rgb}{0.000000,0.000000,0.000000}%
\pgfsetfillcolor{currentfill}%
\pgfsetfillopacity{0.500000}%
\pgfsetlinewidth{0.000000pt}%
\definecolor{currentstroke}{rgb}{0.000000,0.000000,0.000000}%
\pgfsetstrokecolor{currentstroke}%
\pgfsetdash{}{0pt}%
\pgfpathmoveto{\pgfqpoint{1.058530in}{2.049107in}}%
\pgfpathcurveto{\pgfqpoint{1.064055in}{2.049107in}}{\pgfqpoint{1.069355in}{2.051302in}}{\pgfqpoint{1.073261in}{2.055209in}}%
\pgfpathcurveto{\pgfqpoint{1.077168in}{2.059115in}}{\pgfqpoint{1.079363in}{2.064415in}}{\pgfqpoint{1.079363in}{2.069940in}}%
\pgfpathcurveto{\pgfqpoint{1.079363in}{2.075465in}}{\pgfqpoint{1.077168in}{2.080764in}}{\pgfqpoint{1.073261in}{2.084671in}}%
\pgfpathcurveto{\pgfqpoint{1.069355in}{2.088578in}}{\pgfqpoint{1.064055in}{2.090773in}}{\pgfqpoint{1.058530in}{2.090773in}}%
\pgfpathcurveto{\pgfqpoint{1.053005in}{2.090773in}}{\pgfqpoint{1.047706in}{2.088578in}}{\pgfqpoint{1.043799in}{2.084671in}}%
\pgfpathcurveto{\pgfqpoint{1.039892in}{2.080764in}}{\pgfqpoint{1.037697in}{2.075465in}}{\pgfqpoint{1.037697in}{2.069940in}}%
\pgfpathcurveto{\pgfqpoint{1.037697in}{2.064415in}}{\pgfqpoint{1.039892in}{2.059115in}}{\pgfqpoint{1.043799in}{2.055209in}}%
\pgfpathcurveto{\pgfqpoint{1.047706in}{2.051302in}}{\pgfqpoint{1.053005in}{2.049107in}}{\pgfqpoint{1.058530in}{2.049107in}}%
\pgfpathclose%
\pgfusepath{fill}%
\end{pgfscope}%
\begin{pgfscope}%
\pgfpathrectangle{\pgfqpoint{0.833185in}{1.588185in}}{\pgfqpoint{1.162500in}{0.755000in}} %
\pgfusepath{clip}%
\pgfsetbuttcap%
\pgfsetroundjoin%
\definecolor{currentfill}{rgb}{0.000000,0.000000,0.000000}%
\pgfsetfillcolor{currentfill}%
\pgfsetfillopacity{0.500000}%
\pgfsetlinewidth{0.000000pt}%
\definecolor{currentstroke}{rgb}{0.000000,0.000000,0.000000}%
\pgfsetstrokecolor{currentstroke}%
\pgfsetdash{}{0pt}%
\pgfpathmoveto{\pgfqpoint{1.520990in}{1.703826in}}%
\pgfpathcurveto{\pgfqpoint{1.526515in}{1.703826in}}{\pgfqpoint{1.531814in}{1.706022in}}{\pgfqpoint{1.535721in}{1.709928in}}%
\pgfpathcurveto{\pgfqpoint{1.539628in}{1.713835in}}{\pgfqpoint{1.541823in}{1.719135in}}{\pgfqpoint{1.541823in}{1.724660in}}%
\pgfpathcurveto{\pgfqpoint{1.541823in}{1.730185in}}{\pgfqpoint{1.539628in}{1.735484in}}{\pgfqpoint{1.535721in}{1.739391in}}%
\pgfpathcurveto{\pgfqpoint{1.531814in}{1.743298in}}{\pgfqpoint{1.526515in}{1.745493in}}{\pgfqpoint{1.520990in}{1.745493in}}%
\pgfpathcurveto{\pgfqpoint{1.515465in}{1.745493in}}{\pgfqpoint{1.510165in}{1.743298in}}{\pgfqpoint{1.506258in}{1.739391in}}%
\pgfpathcurveto{\pgfqpoint{1.502352in}{1.735484in}}{\pgfqpoint{1.500156in}{1.730185in}}{\pgfqpoint{1.500156in}{1.724660in}}%
\pgfpathcurveto{\pgfqpoint{1.500156in}{1.719135in}}{\pgfqpoint{1.502352in}{1.713835in}}{\pgfqpoint{1.506258in}{1.709928in}}%
\pgfpathcurveto{\pgfqpoint{1.510165in}{1.706022in}}{\pgfqpoint{1.515465in}{1.703826in}}{\pgfqpoint{1.520990in}{1.703826in}}%
\pgfpathclose%
\pgfusepath{fill}%
\end{pgfscope}%
\begin{pgfscope}%
\pgfpathrectangle{\pgfqpoint{0.833185in}{1.588185in}}{\pgfqpoint{1.162500in}{0.755000in}} %
\pgfusepath{clip}%
\pgfsetbuttcap%
\pgfsetroundjoin%
\definecolor{currentfill}{rgb}{0.000000,0.000000,0.000000}%
\pgfsetfillcolor{currentfill}%
\pgfsetfillopacity{0.500000}%
\pgfsetlinewidth{0.000000pt}%
\definecolor{currentstroke}{rgb}{0.000000,0.000000,0.000000}%
\pgfsetstrokecolor{currentstroke}%
\pgfsetdash{}{0pt}%
\pgfpathmoveto{\pgfqpoint{1.402203in}{1.734907in}}%
\pgfpathcurveto{\pgfqpoint{1.407728in}{1.734907in}}{\pgfqpoint{1.413027in}{1.737102in}}{\pgfqpoint{1.416934in}{1.741009in}}%
\pgfpathcurveto{\pgfqpoint{1.420841in}{1.744915in}}{\pgfqpoint{1.423036in}{1.750215in}}{\pgfqpoint{1.423036in}{1.755740in}}%
\pgfpathcurveto{\pgfqpoint{1.423036in}{1.761265in}}{\pgfqpoint{1.420841in}{1.766565in}}{\pgfqpoint{1.416934in}{1.770471in}}%
\pgfpathcurveto{\pgfqpoint{1.413027in}{1.774378in}}{\pgfqpoint{1.407728in}{1.776573in}}{\pgfqpoint{1.402203in}{1.776573in}}%
\pgfpathcurveto{\pgfqpoint{1.396677in}{1.776573in}}{\pgfqpoint{1.391378in}{1.774378in}}{\pgfqpoint{1.387471in}{1.770471in}}%
\pgfpathcurveto{\pgfqpoint{1.383564in}{1.766565in}}{\pgfqpoint{1.381369in}{1.761265in}}{\pgfqpoint{1.381369in}{1.755740in}}%
\pgfpathcurveto{\pgfqpoint{1.381369in}{1.750215in}}{\pgfqpoint{1.383564in}{1.744915in}}{\pgfqpoint{1.387471in}{1.741009in}}%
\pgfpathcurveto{\pgfqpoint{1.391378in}{1.737102in}}{\pgfqpoint{1.396677in}{1.734907in}}{\pgfqpoint{1.402203in}{1.734907in}}%
\pgfpathclose%
\pgfusepath{fill}%
\end{pgfscope}%
\begin{pgfscope}%
\pgfpathrectangle{\pgfqpoint{0.833185in}{1.588185in}}{\pgfqpoint{1.162500in}{0.755000in}} %
\pgfusepath{clip}%
\pgfsetbuttcap%
\pgfsetroundjoin%
\definecolor{currentfill}{rgb}{0.000000,0.000000,0.000000}%
\pgfsetfillcolor{currentfill}%
\pgfsetfillopacity{0.500000}%
\pgfsetlinewidth{0.000000pt}%
\definecolor{currentstroke}{rgb}{0.000000,0.000000,0.000000}%
\pgfsetstrokecolor{currentstroke}%
\pgfsetdash{}{0pt}%
\pgfpathmoveto{\pgfqpoint{1.053510in}{1.977476in}}%
\pgfpathcurveto{\pgfqpoint{1.059035in}{1.977476in}}{\pgfqpoint{1.064335in}{1.979671in}}{\pgfqpoint{1.068242in}{1.983578in}}%
\pgfpathcurveto{\pgfqpoint{1.072148in}{1.987485in}}{\pgfqpoint{1.074344in}{1.992785in}}{\pgfqpoint{1.074344in}{1.998310in}}%
\pgfpathcurveto{\pgfqpoint{1.074344in}{2.003835in}}{\pgfqpoint{1.072148in}{2.009134in}}{\pgfqpoint{1.068242in}{2.013041in}}%
\pgfpathcurveto{\pgfqpoint{1.064335in}{2.016948in}}{\pgfqpoint{1.059035in}{2.019143in}}{\pgfqpoint{1.053510in}{2.019143in}}%
\pgfpathcurveto{\pgfqpoint{1.047985in}{2.019143in}}{\pgfqpoint{1.042686in}{2.016948in}}{\pgfqpoint{1.038779in}{2.013041in}}%
\pgfpathcurveto{\pgfqpoint{1.034872in}{2.009134in}}{\pgfqpoint{1.032677in}{2.003835in}}{\pgfqpoint{1.032677in}{1.998310in}}%
\pgfpathcurveto{\pgfqpoint{1.032677in}{1.992785in}}{\pgfqpoint{1.034872in}{1.987485in}}{\pgfqpoint{1.038779in}{1.983578in}}%
\pgfpathcurveto{\pgfqpoint{1.042686in}{1.979671in}}{\pgfqpoint{1.047985in}{1.977476in}}{\pgfqpoint{1.053510in}{1.977476in}}%
\pgfpathclose%
\pgfusepath{fill}%
\end{pgfscope}%
\begin{pgfscope}%
\pgfpathrectangle{\pgfqpoint{0.833185in}{1.588185in}}{\pgfqpoint{1.162500in}{0.755000in}} %
\pgfusepath{clip}%
\pgfsetbuttcap%
\pgfsetroundjoin%
\definecolor{currentfill}{rgb}{0.000000,0.000000,0.000000}%
\pgfsetfillcolor{currentfill}%
\pgfsetfillopacity{0.500000}%
\pgfsetlinewidth{0.000000pt}%
\definecolor{currentstroke}{rgb}{0.000000,0.000000,0.000000}%
\pgfsetstrokecolor{currentstroke}%
\pgfsetdash{}{0pt}%
\pgfpathmoveto{\pgfqpoint{1.492358in}{1.614683in}}%
\pgfpathcurveto{\pgfqpoint{1.497883in}{1.614683in}}{\pgfqpoint{1.503182in}{1.616879in}}{\pgfqpoint{1.507089in}{1.620785in}}%
\pgfpathcurveto{\pgfqpoint{1.510996in}{1.624692in}}{\pgfqpoint{1.513191in}{1.629992in}}{\pgfqpoint{1.513191in}{1.635517in}}%
\pgfpathcurveto{\pgfqpoint{1.513191in}{1.641042in}}{\pgfqpoint{1.510996in}{1.646341in}}{\pgfqpoint{1.507089in}{1.650248in}}%
\pgfpathcurveto{\pgfqpoint{1.503182in}{1.654155in}}{\pgfqpoint{1.497883in}{1.656350in}}{\pgfqpoint{1.492358in}{1.656350in}}%
\pgfpathcurveto{\pgfqpoint{1.486832in}{1.656350in}}{\pgfqpoint{1.481533in}{1.654155in}}{\pgfqpoint{1.477626in}{1.650248in}}%
\pgfpathcurveto{\pgfqpoint{1.473719in}{1.646341in}}{\pgfqpoint{1.471524in}{1.641042in}}{\pgfqpoint{1.471524in}{1.635517in}}%
\pgfpathcurveto{\pgfqpoint{1.471524in}{1.629992in}}{\pgfqpoint{1.473719in}{1.624692in}}{\pgfqpoint{1.477626in}{1.620785in}}%
\pgfpathcurveto{\pgfqpoint{1.481533in}{1.616879in}}{\pgfqpoint{1.486832in}{1.614683in}}{\pgfqpoint{1.492358in}{1.614683in}}%
\pgfpathclose%
\pgfusepath{fill}%
\end{pgfscope}%
\begin{pgfscope}%
\pgfsetbuttcap%
\pgfsetroundjoin%
\definecolor{currentfill}{rgb}{0.000000,0.000000,0.000000}%
\pgfsetfillcolor{currentfill}%
\pgfsetlinewidth{0.803000pt}%
\definecolor{currentstroke}{rgb}{0.000000,0.000000,0.000000}%
\pgfsetstrokecolor{currentstroke}%
\pgfsetdash{}{0pt}%
\pgfsys@defobject{currentmarker}{\pgfqpoint{-0.048611in}{0.000000in}}{\pgfqpoint{0.000000in}{0.000000in}}{%
\pgfpathmoveto{\pgfqpoint{0.000000in}{0.000000in}}%
\pgfpathlineto{\pgfqpoint{-0.048611in}{0.000000in}}%
\pgfusepath{stroke,fill}%
}%
\begin{pgfscope}%
\pgfsys@transformshift{0.833185in}{1.808481in}%
\pgfsys@useobject{currentmarker}{}%
\end{pgfscope}%
\end{pgfscope}%
\begin{pgfscope}%
\pgftext[x=0.585111in,y=1.766272in,left,base]{\rmfamily\fontsize{8.000000}{9.600000}\selectfont \(\displaystyle 2.5\)}%
\end{pgfscope}%
\begin{pgfscope}%
\pgfsetbuttcap%
\pgfsetroundjoin%
\definecolor{currentfill}{rgb}{0.000000,0.000000,0.000000}%
\pgfsetfillcolor{currentfill}%
\pgfsetlinewidth{0.803000pt}%
\definecolor{currentstroke}{rgb}{0.000000,0.000000,0.000000}%
\pgfsetstrokecolor{currentstroke}%
\pgfsetdash{}{0pt}%
\pgfsys@defobject{currentmarker}{\pgfqpoint{-0.048611in}{0.000000in}}{\pgfqpoint{0.000000in}{0.000000in}}{%
\pgfpathmoveto{\pgfqpoint{0.000000in}{0.000000in}}%
\pgfpathlineto{\pgfqpoint{-0.048611in}{0.000000in}}%
\pgfusepath{stroke,fill}%
}%
\begin{pgfscope}%
\pgfsys@transformshift{0.833185in}{2.044496in}%
\pgfsys@useobject{currentmarker}{}%
\end{pgfscope}%
\end{pgfscope}%
\begin{pgfscope}%
\pgftext[x=0.585111in,y=2.002287in,left,base]{\rmfamily\fontsize{8.000000}{9.600000}\selectfont \(\displaystyle 5.0\)}%
\end{pgfscope}%
\begin{pgfscope}%
\pgfsetbuttcap%
\pgfsetroundjoin%
\definecolor{currentfill}{rgb}{0.000000,0.000000,0.000000}%
\pgfsetfillcolor{currentfill}%
\pgfsetlinewidth{0.803000pt}%
\definecolor{currentstroke}{rgb}{0.000000,0.000000,0.000000}%
\pgfsetstrokecolor{currentstroke}%
\pgfsetdash{}{0pt}%
\pgfsys@defobject{currentmarker}{\pgfqpoint{-0.048611in}{0.000000in}}{\pgfqpoint{0.000000in}{0.000000in}}{%
\pgfpathmoveto{\pgfqpoint{0.000000in}{0.000000in}}%
\pgfpathlineto{\pgfqpoint{-0.048611in}{0.000000in}}%
\pgfusepath{stroke,fill}%
}%
\begin{pgfscope}%
\pgfsys@transformshift{0.833185in}{2.280511in}%
\pgfsys@useobject{currentmarker}{}%
\end{pgfscope}%
\end{pgfscope}%
\begin{pgfscope}%
\pgftext[x=0.585111in,y=2.238302in,left,base]{\rmfamily\fontsize{8.000000}{9.600000}\selectfont \(\displaystyle 7.5\)}%
\end{pgfscope}%
\begin{pgfscope}%
\pgftext[x=0.529556in,y=1.965685in,,bottom,rotate=90.000000]{\rmfamily\fontsize{10.000000}{12.000000}\selectfont charge}%
\end{pgfscope}%
\begin{pgfscope}%
\pgftext[x=0.833185in,y=2.384851in,left,base]{\rmfamily\fontsize{10.000000}{12.000000}\selectfont \(\displaystyle \times10^{-10}\)}%
\end{pgfscope}%
\begin{pgfscope}%
\pgfsetrectcap%
\pgfsetmiterjoin%
\pgfsetlinewidth{0.803000pt}%
\definecolor{currentstroke}{rgb}{0.000000,0.000000,0.000000}%
\pgfsetstrokecolor{currentstroke}%
\pgfsetdash{}{0pt}%
\pgfpathmoveto{\pgfqpoint{0.833185in}{1.588185in}}%
\pgfpathlineto{\pgfqpoint{0.833185in}{2.343185in}}%
\pgfusepath{stroke}%
\end{pgfscope}%
\begin{pgfscope}%
\pgfsetrectcap%
\pgfsetmiterjoin%
\pgfsetlinewidth{0.803000pt}%
\definecolor{currentstroke}{rgb}{0.000000,0.000000,0.000000}%
\pgfsetstrokecolor{currentstroke}%
\pgfsetdash{}{0pt}%
\pgfpathmoveto{\pgfqpoint{1.995685in}{1.588185in}}%
\pgfpathlineto{\pgfqpoint{1.995685in}{2.343185in}}%
\pgfusepath{stroke}%
\end{pgfscope}%
\begin{pgfscope}%
\pgfsetrectcap%
\pgfsetmiterjoin%
\pgfsetlinewidth{0.803000pt}%
\definecolor{currentstroke}{rgb}{0.000000,0.000000,0.000000}%
\pgfsetstrokecolor{currentstroke}%
\pgfsetdash{}{0pt}%
\pgfpathmoveto{\pgfqpoint{0.833185in}{1.588185in}}%
\pgfpathlineto{\pgfqpoint{1.995685in}{1.588185in}}%
\pgfusepath{stroke}%
\end{pgfscope}%
\begin{pgfscope}%
\pgfsetrectcap%
\pgfsetmiterjoin%
\pgfsetlinewidth{0.803000pt}%
\definecolor{currentstroke}{rgb}{0.000000,0.000000,0.000000}%
\pgfsetstrokecolor{currentstroke}%
\pgfsetdash{}{0pt}%
\pgfpathmoveto{\pgfqpoint{0.833185in}{2.343185in}}%
\pgfpathlineto{\pgfqpoint{1.995685in}{2.343185in}}%
\pgfusepath{stroke}%
\end{pgfscope}%
\begin{pgfscope}%
\pgfsetbuttcap%
\pgfsetmiterjoin%
\definecolor{currentfill}{rgb}{1.000000,1.000000,1.000000}%
\pgfsetfillcolor{currentfill}%
\pgfsetlinewidth{0.000000pt}%
\definecolor{currentstroke}{rgb}{0.000000,0.000000,0.000000}%
\pgfsetstrokecolor{currentstroke}%
\pgfsetstrokeopacity{0.000000}%
\pgfsetdash{}{0pt}%
\pgfpathmoveto{\pgfqpoint{1.995685in}{1.588185in}}%
\pgfpathlineto{\pgfqpoint{3.158185in}{1.588185in}}%
\pgfpathlineto{\pgfqpoint{3.158185in}{2.343185in}}%
\pgfpathlineto{\pgfqpoint{1.995685in}{2.343185in}}%
\pgfpathclose%
\pgfusepath{fill}%
\end{pgfscope}%
\begin{pgfscope}%
\pgfpathrectangle{\pgfqpoint{1.995685in}{1.588185in}}{\pgfqpoint{1.162500in}{0.755000in}} %
\pgfusepath{clip}%
\pgfsetbuttcap%
\pgfsetroundjoin%
\definecolor{currentfill}{rgb}{0.000000,0.000000,0.000000}%
\pgfsetfillcolor{currentfill}%
\pgfsetfillopacity{0.500000}%
\pgfsetlinewidth{0.000000pt}%
\definecolor{currentstroke}{rgb}{0.000000,0.000000,0.000000}%
\pgfsetstrokecolor{currentstroke}%
\pgfsetdash{}{0pt}%
\pgfpathmoveto{\pgfqpoint{3.130506in}{2.304375in}}%
\pgfpathcurveto{\pgfqpoint{3.136031in}{2.304375in}}{\pgfqpoint{3.141331in}{2.306570in}}{\pgfqpoint{3.145237in}{2.310477in}}%
\pgfpathcurveto{\pgfqpoint{3.149144in}{2.314384in}}{\pgfqpoint{3.151339in}{2.319683in}}{\pgfqpoint{3.151339in}{2.325208in}}%
\pgfpathcurveto{\pgfqpoint{3.151339in}{2.330733in}}{\pgfqpoint{3.149144in}{2.336033in}}{\pgfqpoint{3.145237in}{2.339940in}}%
\pgfpathcurveto{\pgfqpoint{3.141331in}{2.343847in}}{\pgfqpoint{3.136031in}{2.346042in}}{\pgfqpoint{3.130506in}{2.346042in}}%
\pgfpathcurveto{\pgfqpoint{3.124981in}{2.346042in}}{\pgfqpoint{3.119681in}{2.343847in}}{\pgfqpoint{3.115775in}{2.339940in}}%
\pgfpathcurveto{\pgfqpoint{3.111868in}{2.336033in}}{\pgfqpoint{3.109673in}{2.330733in}}{\pgfqpoint{3.109673in}{2.325208in}}%
\pgfpathcurveto{\pgfqpoint{3.109673in}{2.319683in}}{\pgfqpoint{3.111868in}{2.314384in}}{\pgfqpoint{3.115775in}{2.310477in}}%
\pgfpathcurveto{\pgfqpoint{3.119681in}{2.306570in}}{\pgfqpoint{3.124981in}{2.304375in}}{\pgfqpoint{3.130506in}{2.304375in}}%
\pgfpathclose%
\pgfusepath{fill}%
\end{pgfscope}%
\begin{pgfscope}%
\pgfpathrectangle{\pgfqpoint{1.995685in}{1.588185in}}{\pgfqpoint{1.162500in}{0.755000in}} %
\pgfusepath{clip}%
\pgfsetbuttcap%
\pgfsetroundjoin%
\definecolor{currentfill}{rgb}{0.000000,0.000000,0.000000}%
\pgfsetfillcolor{currentfill}%
\pgfsetfillopacity{0.500000}%
\pgfsetlinewidth{0.000000pt}%
\definecolor{currentstroke}{rgb}{0.000000,0.000000,0.000000}%
\pgfsetstrokecolor{currentstroke}%
\pgfsetdash{}{0pt}%
\pgfpathmoveto{\pgfqpoint{2.755547in}{1.845741in}}%
\pgfpathcurveto{\pgfqpoint{2.761072in}{1.845741in}}{\pgfqpoint{2.766372in}{1.847936in}}{\pgfqpoint{2.770279in}{1.851843in}}%
\pgfpathcurveto{\pgfqpoint{2.774186in}{1.855750in}}{\pgfqpoint{2.776381in}{1.861049in}}{\pgfqpoint{2.776381in}{1.866574in}}%
\pgfpathcurveto{\pgfqpoint{2.776381in}{1.872099in}}{\pgfqpoint{2.774186in}{1.877399in}}{\pgfqpoint{2.770279in}{1.881306in}}%
\pgfpathcurveto{\pgfqpoint{2.766372in}{1.885212in}}{\pgfqpoint{2.761072in}{1.887408in}}{\pgfqpoint{2.755547in}{1.887408in}}%
\pgfpathcurveto{\pgfqpoint{2.750022in}{1.887408in}}{\pgfqpoint{2.744723in}{1.885212in}}{\pgfqpoint{2.740816in}{1.881306in}}%
\pgfpathcurveto{\pgfqpoint{2.736909in}{1.877399in}}{\pgfqpoint{2.734714in}{1.872099in}}{\pgfqpoint{2.734714in}{1.866574in}}%
\pgfpathcurveto{\pgfqpoint{2.734714in}{1.861049in}}{\pgfqpoint{2.736909in}{1.855750in}}{\pgfqpoint{2.740816in}{1.851843in}}%
\pgfpathcurveto{\pgfqpoint{2.744723in}{1.847936in}}{\pgfqpoint{2.750022in}{1.845741in}}{\pgfqpoint{2.755547in}{1.845741in}}%
\pgfpathclose%
\pgfusepath{fill}%
\end{pgfscope}%
\begin{pgfscope}%
\pgfpathrectangle{\pgfqpoint{1.995685in}{1.588185in}}{\pgfqpoint{1.162500in}{0.755000in}} %
\pgfusepath{clip}%
\pgfsetbuttcap%
\pgfsetroundjoin%
\definecolor{currentfill}{rgb}{0.000000,0.000000,0.000000}%
\pgfsetfillcolor{currentfill}%
\pgfsetfillopacity{0.500000}%
\pgfsetlinewidth{0.000000pt}%
\definecolor{currentstroke}{rgb}{0.000000,0.000000,0.000000}%
\pgfsetstrokecolor{currentstroke}%
\pgfsetdash{}{0pt}%
\pgfpathmoveto{\pgfqpoint{2.755127in}{1.840570in}}%
\pgfpathcurveto{\pgfqpoint{2.760652in}{1.840570in}}{\pgfqpoint{2.765952in}{1.842766in}}{\pgfqpoint{2.769858in}{1.846672in}}%
\pgfpathcurveto{\pgfqpoint{2.773765in}{1.850579in}}{\pgfqpoint{2.775960in}{1.855879in}}{\pgfqpoint{2.775960in}{1.861404in}}%
\pgfpathcurveto{\pgfqpoint{2.775960in}{1.866929in}}{\pgfqpoint{2.773765in}{1.872228in}}{\pgfqpoint{2.769858in}{1.876135in}}%
\pgfpathcurveto{\pgfqpoint{2.765952in}{1.880042in}}{\pgfqpoint{2.760652in}{1.882237in}}{\pgfqpoint{2.755127in}{1.882237in}}%
\pgfpathcurveto{\pgfqpoint{2.749602in}{1.882237in}}{\pgfqpoint{2.744302in}{1.880042in}}{\pgfqpoint{2.740396in}{1.876135in}}%
\pgfpathcurveto{\pgfqpoint{2.736489in}{1.872228in}}{\pgfqpoint{2.734294in}{1.866929in}}{\pgfqpoint{2.734294in}{1.861404in}}%
\pgfpathcurveto{\pgfqpoint{2.734294in}{1.855879in}}{\pgfqpoint{2.736489in}{1.850579in}}{\pgfqpoint{2.740396in}{1.846672in}}%
\pgfpathcurveto{\pgfqpoint{2.744302in}{1.842766in}}{\pgfqpoint{2.749602in}{1.840570in}}{\pgfqpoint{2.755127in}{1.840570in}}%
\pgfpathclose%
\pgfusepath{fill}%
\end{pgfscope}%
\begin{pgfscope}%
\pgfpathrectangle{\pgfqpoint{1.995685in}{1.588185in}}{\pgfqpoint{1.162500in}{0.755000in}} %
\pgfusepath{clip}%
\pgfsetbuttcap%
\pgfsetroundjoin%
\definecolor{currentfill}{rgb}{0.000000,0.000000,0.000000}%
\pgfsetfillcolor{currentfill}%
\pgfsetfillopacity{0.500000}%
\pgfsetlinewidth{0.000000pt}%
\definecolor{currentstroke}{rgb}{0.000000,0.000000,0.000000}%
\pgfsetstrokecolor{currentstroke}%
\pgfsetdash{}{0pt}%
\pgfpathmoveto{\pgfqpoint{2.317114in}{1.680932in}}%
\pgfpathcurveto{\pgfqpoint{2.322639in}{1.680932in}}{\pgfqpoint{2.327939in}{1.683127in}}{\pgfqpoint{2.331845in}{1.687034in}}%
\pgfpathcurveto{\pgfqpoint{2.335752in}{1.690941in}}{\pgfqpoint{2.337947in}{1.696241in}}{\pgfqpoint{2.337947in}{1.701766in}}%
\pgfpathcurveto{\pgfqpoint{2.337947in}{1.707291in}}{\pgfqpoint{2.335752in}{1.712590in}}{\pgfqpoint{2.331845in}{1.716497in}}%
\pgfpathcurveto{\pgfqpoint{2.327939in}{1.720404in}}{\pgfqpoint{2.322639in}{1.722599in}}{\pgfqpoint{2.317114in}{1.722599in}}%
\pgfpathcurveto{\pgfqpoint{2.311589in}{1.722599in}}{\pgfqpoint{2.306289in}{1.720404in}}{\pgfqpoint{2.302383in}{1.716497in}}%
\pgfpathcurveto{\pgfqpoint{2.298476in}{1.712590in}}{\pgfqpoint{2.296281in}{1.707291in}}{\pgfqpoint{2.296281in}{1.701766in}}%
\pgfpathcurveto{\pgfqpoint{2.296281in}{1.696241in}}{\pgfqpoint{2.298476in}{1.690941in}}{\pgfqpoint{2.302383in}{1.687034in}}%
\pgfpathcurveto{\pgfqpoint{2.306289in}{1.683127in}}{\pgfqpoint{2.311589in}{1.680932in}}{\pgfqpoint{2.317114in}{1.680932in}}%
\pgfpathclose%
\pgfusepath{fill}%
\end{pgfscope}%
\begin{pgfscope}%
\pgfpathrectangle{\pgfqpoint{1.995685in}{1.588185in}}{\pgfqpoint{1.162500in}{0.755000in}} %
\pgfusepath{clip}%
\pgfsetbuttcap%
\pgfsetroundjoin%
\definecolor{currentfill}{rgb}{0.000000,0.000000,0.000000}%
\pgfsetfillcolor{currentfill}%
\pgfsetfillopacity{0.500000}%
\pgfsetlinewidth{0.000000pt}%
\definecolor{currentstroke}{rgb}{0.000000,0.000000,0.000000}%
\pgfsetstrokecolor{currentstroke}%
\pgfsetdash{}{0pt}%
\pgfpathmoveto{\pgfqpoint{2.241285in}{1.656866in}}%
\pgfpathcurveto{\pgfqpoint{2.246810in}{1.656866in}}{\pgfqpoint{2.252110in}{1.659061in}}{\pgfqpoint{2.256017in}{1.662968in}}%
\pgfpathcurveto{\pgfqpoint{2.259923in}{1.666874in}}{\pgfqpoint{2.262118in}{1.672174in}}{\pgfqpoint{2.262118in}{1.677699in}}%
\pgfpathcurveto{\pgfqpoint{2.262118in}{1.683224in}}{\pgfqpoint{2.259923in}{1.688523in}}{\pgfqpoint{2.256017in}{1.692430in}}%
\pgfpathcurveto{\pgfqpoint{2.252110in}{1.696337in}}{\pgfqpoint{2.246810in}{1.698532in}}{\pgfqpoint{2.241285in}{1.698532in}}%
\pgfpathcurveto{\pgfqpoint{2.235760in}{1.698532in}}{\pgfqpoint{2.230461in}{1.696337in}}{\pgfqpoint{2.226554in}{1.692430in}}%
\pgfpathcurveto{\pgfqpoint{2.222647in}{1.688523in}}{\pgfqpoint{2.220452in}{1.683224in}}{\pgfqpoint{2.220452in}{1.677699in}}%
\pgfpathcurveto{\pgfqpoint{2.220452in}{1.672174in}}{\pgfqpoint{2.222647in}{1.666874in}}{\pgfqpoint{2.226554in}{1.662968in}}%
\pgfpathcurveto{\pgfqpoint{2.230461in}{1.659061in}}{\pgfqpoint{2.235760in}{1.656866in}}{\pgfqpoint{2.241285in}{1.656866in}}%
\pgfpathclose%
\pgfusepath{fill}%
\end{pgfscope}%
\begin{pgfscope}%
\pgfpathrectangle{\pgfqpoint{1.995685in}{1.588185in}}{\pgfqpoint{1.162500in}{0.755000in}} %
\pgfusepath{clip}%
\pgfsetbuttcap%
\pgfsetroundjoin%
\definecolor{currentfill}{rgb}{0.000000,0.000000,0.000000}%
\pgfsetfillcolor{currentfill}%
\pgfsetfillopacity{0.500000}%
\pgfsetlinewidth{0.000000pt}%
\definecolor{currentstroke}{rgb}{0.000000,0.000000,0.000000}%
\pgfsetstrokecolor{currentstroke}%
\pgfsetdash{}{0pt}%
\pgfpathmoveto{\pgfqpoint{2.106066in}{1.598883in}}%
\pgfpathcurveto{\pgfqpoint{2.111591in}{1.598883in}}{\pgfqpoint{2.116890in}{1.601078in}}{\pgfqpoint{2.120797in}{1.604985in}}%
\pgfpathcurveto{\pgfqpoint{2.124704in}{1.608892in}}{\pgfqpoint{2.126899in}{1.614191in}}{\pgfqpoint{2.126899in}{1.619716in}}%
\pgfpathcurveto{\pgfqpoint{2.126899in}{1.625241in}}{\pgfqpoint{2.124704in}{1.630541in}}{\pgfqpoint{2.120797in}{1.634447in}}%
\pgfpathcurveto{\pgfqpoint{2.116890in}{1.638354in}}{\pgfqpoint{2.111591in}{1.640549in}}{\pgfqpoint{2.106066in}{1.640549in}}%
\pgfpathcurveto{\pgfqpoint{2.100541in}{1.640549in}}{\pgfqpoint{2.095241in}{1.638354in}}{\pgfqpoint{2.091334in}{1.634447in}}%
\pgfpathcurveto{\pgfqpoint{2.087428in}{1.630541in}}{\pgfqpoint{2.085232in}{1.625241in}}{\pgfqpoint{2.085232in}{1.619716in}}%
\pgfpathcurveto{\pgfqpoint{2.085232in}{1.614191in}}{\pgfqpoint{2.087428in}{1.608892in}}{\pgfqpoint{2.091334in}{1.604985in}}%
\pgfpathcurveto{\pgfqpoint{2.095241in}{1.601078in}}{\pgfqpoint{2.100541in}{1.598883in}}{\pgfqpoint{2.106066in}{1.598883in}}%
\pgfpathclose%
\pgfusepath{fill}%
\end{pgfscope}%
\begin{pgfscope}%
\pgfpathrectangle{\pgfqpoint{1.995685in}{1.588185in}}{\pgfqpoint{1.162500in}{0.755000in}} %
\pgfusepath{clip}%
\pgfsetbuttcap%
\pgfsetroundjoin%
\definecolor{currentfill}{rgb}{0.000000,0.000000,0.000000}%
\pgfsetfillcolor{currentfill}%
\pgfsetfillopacity{0.500000}%
\pgfsetlinewidth{0.000000pt}%
\definecolor{currentstroke}{rgb}{0.000000,0.000000,0.000000}%
\pgfsetstrokecolor{currentstroke}%
\pgfsetdash{}{0pt}%
\pgfpathmoveto{\pgfqpoint{2.023363in}{1.585327in}}%
\pgfpathcurveto{\pgfqpoint{2.028888in}{1.585327in}}{\pgfqpoint{2.034188in}{1.587523in}}{\pgfqpoint{2.038095in}{1.591429in}}%
\pgfpathcurveto{\pgfqpoint{2.042001in}{1.595336in}}{\pgfqpoint{2.044196in}{1.600636in}}{\pgfqpoint{2.044196in}{1.606161in}}%
\pgfpathcurveto{\pgfqpoint{2.044196in}{1.611686in}}{\pgfqpoint{2.042001in}{1.616985in}}{\pgfqpoint{2.038095in}{1.620892in}}%
\pgfpathcurveto{\pgfqpoint{2.034188in}{1.624799in}}{\pgfqpoint{2.028888in}{1.626994in}}{\pgfqpoint{2.023363in}{1.626994in}}%
\pgfpathcurveto{\pgfqpoint{2.017838in}{1.626994in}}{\pgfqpoint{2.012539in}{1.624799in}}{\pgfqpoint{2.008632in}{1.620892in}}%
\pgfpathcurveto{\pgfqpoint{2.004725in}{1.616985in}}{\pgfqpoint{2.002530in}{1.611686in}}{\pgfqpoint{2.002530in}{1.606161in}}%
\pgfpathcurveto{\pgfqpoint{2.002530in}{1.600636in}}{\pgfqpoint{2.004725in}{1.595336in}}{\pgfqpoint{2.008632in}{1.591429in}}%
\pgfpathcurveto{\pgfqpoint{2.012539in}{1.587523in}}{\pgfqpoint{2.017838in}{1.585327in}}{\pgfqpoint{2.023363in}{1.585327in}}%
\pgfpathclose%
\pgfusepath{fill}%
\end{pgfscope}%
\begin{pgfscope}%
\pgfpathrectangle{\pgfqpoint{1.995685in}{1.588185in}}{\pgfqpoint{1.162500in}{0.755000in}} %
\pgfusepath{clip}%
\pgfsetbuttcap%
\pgfsetroundjoin%
\definecolor{currentfill}{rgb}{0.000000,0.000000,0.000000}%
\pgfsetfillcolor{currentfill}%
\pgfsetfillopacity{0.500000}%
\pgfsetlinewidth{0.000000pt}%
\definecolor{currentstroke}{rgb}{0.000000,0.000000,0.000000}%
\pgfsetstrokecolor{currentstroke}%
\pgfsetdash{}{0pt}%
\pgfpathmoveto{\pgfqpoint{2.907651in}{1.989747in}}%
\pgfpathcurveto{\pgfqpoint{2.913176in}{1.989747in}}{\pgfqpoint{2.918476in}{1.991942in}}{\pgfqpoint{2.922382in}{1.995849in}}%
\pgfpathcurveto{\pgfqpoint{2.926289in}{1.999756in}}{\pgfqpoint{2.928484in}{2.005055in}}{\pgfqpoint{2.928484in}{2.010580in}}%
\pgfpathcurveto{\pgfqpoint{2.928484in}{2.016105in}}{\pgfqpoint{2.926289in}{2.021405in}}{\pgfqpoint{2.922382in}{2.025311in}}%
\pgfpathcurveto{\pgfqpoint{2.918476in}{2.029218in}}{\pgfqpoint{2.913176in}{2.031413in}}{\pgfqpoint{2.907651in}{2.031413in}}%
\pgfpathcurveto{\pgfqpoint{2.902126in}{2.031413in}}{\pgfqpoint{2.896827in}{2.029218in}}{\pgfqpoint{2.892920in}{2.025311in}}%
\pgfpathcurveto{\pgfqpoint{2.889013in}{2.021405in}}{\pgfqpoint{2.886818in}{2.016105in}}{\pgfqpoint{2.886818in}{2.010580in}}%
\pgfpathcurveto{\pgfqpoint{2.886818in}{2.005055in}}{\pgfqpoint{2.889013in}{1.999756in}}{\pgfqpoint{2.892920in}{1.995849in}}%
\pgfpathcurveto{\pgfqpoint{2.896827in}{1.991942in}}{\pgfqpoint{2.902126in}{1.989747in}}{\pgfqpoint{2.907651in}{1.989747in}}%
\pgfpathclose%
\pgfusepath{fill}%
\end{pgfscope}%
\begin{pgfscope}%
\pgfpathrectangle{\pgfqpoint{1.995685in}{1.588185in}}{\pgfqpoint{1.162500in}{0.755000in}} %
\pgfusepath{clip}%
\pgfsetbuttcap%
\pgfsetroundjoin%
\definecolor{currentfill}{rgb}{0.000000,0.000000,0.000000}%
\pgfsetfillcolor{currentfill}%
\pgfsetfillopacity{0.500000}%
\pgfsetlinewidth{0.000000pt}%
\definecolor{currentstroke}{rgb}{0.000000,0.000000,0.000000}%
\pgfsetstrokecolor{currentstroke}%
\pgfsetdash{}{0pt}%
\pgfpathmoveto{\pgfqpoint{2.630361in}{1.867889in}}%
\pgfpathcurveto{\pgfqpoint{2.635886in}{1.867889in}}{\pgfqpoint{2.641186in}{1.870084in}}{\pgfqpoint{2.645093in}{1.873991in}}%
\pgfpathcurveto{\pgfqpoint{2.649000in}{1.877898in}}{\pgfqpoint{2.651195in}{1.883197in}}{\pgfqpoint{2.651195in}{1.888722in}}%
\pgfpathcurveto{\pgfqpoint{2.651195in}{1.894247in}}{\pgfqpoint{2.649000in}{1.899547in}}{\pgfqpoint{2.645093in}{1.903454in}}%
\pgfpathcurveto{\pgfqpoint{2.641186in}{1.907360in}}{\pgfqpoint{2.635886in}{1.909556in}}{\pgfqpoint{2.630361in}{1.909556in}}%
\pgfpathcurveto{\pgfqpoint{2.624836in}{1.909556in}}{\pgfqpoint{2.619537in}{1.907360in}}{\pgfqpoint{2.615630in}{1.903454in}}%
\pgfpathcurveto{\pgfqpoint{2.611723in}{1.899547in}}{\pgfqpoint{2.609528in}{1.894247in}}{\pgfqpoint{2.609528in}{1.888722in}}%
\pgfpathcurveto{\pgfqpoint{2.609528in}{1.883197in}}{\pgfqpoint{2.611723in}{1.877898in}}{\pgfqpoint{2.615630in}{1.873991in}}%
\pgfpathcurveto{\pgfqpoint{2.619537in}{1.870084in}}{\pgfqpoint{2.624836in}{1.867889in}}{\pgfqpoint{2.630361in}{1.867889in}}%
\pgfpathclose%
\pgfusepath{fill}%
\end{pgfscope}%
\begin{pgfscope}%
\pgfpathrectangle{\pgfqpoint{1.995685in}{1.588185in}}{\pgfqpoint{1.162500in}{0.755000in}} %
\pgfusepath{clip}%
\pgfsetbuttcap%
\pgfsetroundjoin%
\definecolor{currentfill}{rgb}{0.000000,0.000000,0.000000}%
\pgfsetfillcolor{currentfill}%
\pgfsetfillopacity{0.500000}%
\pgfsetlinewidth{0.000000pt}%
\definecolor{currentstroke}{rgb}{0.000000,0.000000,0.000000}%
\pgfsetstrokecolor{currentstroke}%
\pgfsetdash{}{0pt}%
\pgfpathmoveto{\pgfqpoint{2.487752in}{1.854418in}}%
\pgfpathcurveto{\pgfqpoint{2.493277in}{1.854418in}}{\pgfqpoint{2.498576in}{1.856613in}}{\pgfqpoint{2.502483in}{1.860520in}}%
\pgfpathcurveto{\pgfqpoint{2.506390in}{1.864427in}}{\pgfqpoint{2.508585in}{1.869726in}}{\pgfqpoint{2.508585in}{1.875251in}}%
\pgfpathcurveto{\pgfqpoint{2.508585in}{1.880776in}}{\pgfqpoint{2.506390in}{1.886076in}}{\pgfqpoint{2.502483in}{1.889982in}}%
\pgfpathcurveto{\pgfqpoint{2.498576in}{1.893889in}}{\pgfqpoint{2.493277in}{1.896084in}}{\pgfqpoint{2.487752in}{1.896084in}}%
\pgfpathcurveto{\pgfqpoint{2.482227in}{1.896084in}}{\pgfqpoint{2.476927in}{1.893889in}}{\pgfqpoint{2.473020in}{1.889982in}}%
\pgfpathcurveto{\pgfqpoint{2.469113in}{1.886076in}}{\pgfqpoint{2.466918in}{1.880776in}}{\pgfqpoint{2.466918in}{1.875251in}}%
\pgfpathcurveto{\pgfqpoint{2.466918in}{1.869726in}}{\pgfqpoint{2.469113in}{1.864427in}}{\pgfqpoint{2.473020in}{1.860520in}}%
\pgfpathcurveto{\pgfqpoint{2.476927in}{1.856613in}}{\pgfqpoint{2.482227in}{1.854418in}}{\pgfqpoint{2.487752in}{1.854418in}}%
\pgfpathclose%
\pgfusepath{fill}%
\end{pgfscope}%
\begin{pgfscope}%
\pgfpathrectangle{\pgfqpoint{1.995685in}{1.588185in}}{\pgfqpoint{1.162500in}{0.755000in}} %
\pgfusepath{clip}%
\pgfsetbuttcap%
\pgfsetroundjoin%
\definecolor{currentfill}{rgb}{0.000000,0.000000,0.000000}%
\pgfsetfillcolor{currentfill}%
\pgfsetfillopacity{0.500000}%
\pgfsetlinewidth{0.000000pt}%
\definecolor{currentstroke}{rgb}{0.000000,0.000000,0.000000}%
\pgfsetstrokecolor{currentstroke}%
\pgfsetdash{}{0pt}%
\pgfpathmoveto{\pgfqpoint{3.000738in}{2.049107in}}%
\pgfpathcurveto{\pgfqpoint{3.006263in}{2.049107in}}{\pgfqpoint{3.011563in}{2.051302in}}{\pgfqpoint{3.015470in}{2.055209in}}%
\pgfpathcurveto{\pgfqpoint{3.019376in}{2.059115in}}{\pgfqpoint{3.021572in}{2.064415in}}{\pgfqpoint{3.021572in}{2.069940in}}%
\pgfpathcurveto{\pgfqpoint{3.021572in}{2.075465in}}{\pgfqpoint{3.019376in}{2.080764in}}{\pgfqpoint{3.015470in}{2.084671in}}%
\pgfpathcurveto{\pgfqpoint{3.011563in}{2.088578in}}{\pgfqpoint{3.006263in}{2.090773in}}{\pgfqpoint{3.000738in}{2.090773in}}%
\pgfpathcurveto{\pgfqpoint{2.995213in}{2.090773in}}{\pgfqpoint{2.989914in}{2.088578in}}{\pgfqpoint{2.986007in}{2.084671in}}%
\pgfpathcurveto{\pgfqpoint{2.982100in}{2.080764in}}{\pgfqpoint{2.979905in}{2.075465in}}{\pgfqpoint{2.979905in}{2.069940in}}%
\pgfpathcurveto{\pgfqpoint{2.979905in}{2.064415in}}{\pgfqpoint{2.982100in}{2.059115in}}{\pgfqpoint{2.986007in}{2.055209in}}%
\pgfpathcurveto{\pgfqpoint{2.989914in}{2.051302in}}{\pgfqpoint{2.995213in}{2.049107in}}{\pgfqpoint{3.000738in}{2.049107in}}%
\pgfpathclose%
\pgfusepath{fill}%
\end{pgfscope}%
\begin{pgfscope}%
\pgfpathrectangle{\pgfqpoint{1.995685in}{1.588185in}}{\pgfqpoint{1.162500in}{0.755000in}} %
\pgfusepath{clip}%
\pgfsetbuttcap%
\pgfsetroundjoin%
\definecolor{currentfill}{rgb}{0.000000,0.000000,0.000000}%
\pgfsetfillcolor{currentfill}%
\pgfsetfillopacity{0.500000}%
\pgfsetlinewidth{0.000000pt}%
\definecolor{currentstroke}{rgb}{0.000000,0.000000,0.000000}%
\pgfsetstrokecolor{currentstroke}%
\pgfsetdash{}{0pt}%
\pgfpathmoveto{\pgfqpoint{2.318352in}{1.703826in}}%
\pgfpathcurveto{\pgfqpoint{2.323877in}{1.703826in}}{\pgfqpoint{2.329176in}{1.706022in}}{\pgfqpoint{2.333083in}{1.709928in}}%
\pgfpathcurveto{\pgfqpoint{2.336990in}{1.713835in}}{\pgfqpoint{2.339185in}{1.719135in}}{\pgfqpoint{2.339185in}{1.724660in}}%
\pgfpathcurveto{\pgfqpoint{2.339185in}{1.730185in}}{\pgfqpoint{2.336990in}{1.735484in}}{\pgfqpoint{2.333083in}{1.739391in}}%
\pgfpathcurveto{\pgfqpoint{2.329176in}{1.743298in}}{\pgfqpoint{2.323877in}{1.745493in}}{\pgfqpoint{2.318352in}{1.745493in}}%
\pgfpathcurveto{\pgfqpoint{2.312827in}{1.745493in}}{\pgfqpoint{2.307527in}{1.743298in}}{\pgfqpoint{2.303620in}{1.739391in}}%
\pgfpathcurveto{\pgfqpoint{2.299713in}{1.735484in}}{\pgfqpoint{2.297518in}{1.730185in}}{\pgfqpoint{2.297518in}{1.724660in}}%
\pgfpathcurveto{\pgfqpoint{2.297518in}{1.719135in}}{\pgfqpoint{2.299713in}{1.713835in}}{\pgfqpoint{2.303620in}{1.709928in}}%
\pgfpathcurveto{\pgfqpoint{2.307527in}{1.706022in}}{\pgfqpoint{2.312827in}{1.703826in}}{\pgfqpoint{2.318352in}{1.703826in}}%
\pgfpathclose%
\pgfusepath{fill}%
\end{pgfscope}%
\begin{pgfscope}%
\pgfpathrectangle{\pgfqpoint{1.995685in}{1.588185in}}{\pgfqpoint{1.162500in}{0.755000in}} %
\pgfusepath{clip}%
\pgfsetbuttcap%
\pgfsetroundjoin%
\definecolor{currentfill}{rgb}{0.000000,0.000000,0.000000}%
\pgfsetfillcolor{currentfill}%
\pgfsetfillopacity{0.500000}%
\pgfsetlinewidth{0.000000pt}%
\definecolor{currentstroke}{rgb}{0.000000,0.000000,0.000000}%
\pgfsetstrokecolor{currentstroke}%
\pgfsetdash{}{0pt}%
\pgfpathmoveto{\pgfqpoint{2.410257in}{1.734907in}}%
\pgfpathcurveto{\pgfqpoint{2.415782in}{1.734907in}}{\pgfqpoint{2.421082in}{1.737102in}}{\pgfqpoint{2.424989in}{1.741009in}}%
\pgfpathcurveto{\pgfqpoint{2.428896in}{1.744915in}}{\pgfqpoint{2.431091in}{1.750215in}}{\pgfqpoint{2.431091in}{1.755740in}}%
\pgfpathcurveto{\pgfqpoint{2.431091in}{1.761265in}}{\pgfqpoint{2.428896in}{1.766565in}}{\pgfqpoint{2.424989in}{1.770471in}}%
\pgfpathcurveto{\pgfqpoint{2.421082in}{1.774378in}}{\pgfqpoint{2.415782in}{1.776573in}}{\pgfqpoint{2.410257in}{1.776573in}}%
\pgfpathcurveto{\pgfqpoint{2.404732in}{1.776573in}}{\pgfqpoint{2.399433in}{1.774378in}}{\pgfqpoint{2.395526in}{1.770471in}}%
\pgfpathcurveto{\pgfqpoint{2.391619in}{1.766565in}}{\pgfqpoint{2.389424in}{1.761265in}}{\pgfqpoint{2.389424in}{1.755740in}}%
\pgfpathcurveto{\pgfqpoint{2.389424in}{1.750215in}}{\pgfqpoint{2.391619in}{1.744915in}}{\pgfqpoint{2.395526in}{1.741009in}}%
\pgfpathcurveto{\pgfqpoint{2.399433in}{1.737102in}}{\pgfqpoint{2.404732in}{1.734907in}}{\pgfqpoint{2.410257in}{1.734907in}}%
\pgfpathclose%
\pgfusepath{fill}%
\end{pgfscope}%
\begin{pgfscope}%
\pgfpathrectangle{\pgfqpoint{1.995685in}{1.588185in}}{\pgfqpoint{1.162500in}{0.755000in}} %
\pgfusepath{clip}%
\pgfsetbuttcap%
\pgfsetroundjoin%
\definecolor{currentfill}{rgb}{0.000000,0.000000,0.000000}%
\pgfsetfillcolor{currentfill}%
\pgfsetfillopacity{0.500000}%
\pgfsetlinewidth{0.000000pt}%
\definecolor{currentstroke}{rgb}{0.000000,0.000000,0.000000}%
\pgfsetstrokecolor{currentstroke}%
\pgfsetdash{}{0pt}%
\pgfpathmoveto{\pgfqpoint{2.752801in}{1.977476in}}%
\pgfpathcurveto{\pgfqpoint{2.758327in}{1.977476in}}{\pgfqpoint{2.763626in}{1.979671in}}{\pgfqpoint{2.767533in}{1.983578in}}%
\pgfpathcurveto{\pgfqpoint{2.771440in}{1.987485in}}{\pgfqpoint{2.773635in}{1.992785in}}{\pgfqpoint{2.773635in}{1.998310in}}%
\pgfpathcurveto{\pgfqpoint{2.773635in}{2.003835in}}{\pgfqpoint{2.771440in}{2.009134in}}{\pgfqpoint{2.767533in}{2.013041in}}%
\pgfpathcurveto{\pgfqpoint{2.763626in}{2.016948in}}{\pgfqpoint{2.758327in}{2.019143in}}{\pgfqpoint{2.752801in}{2.019143in}}%
\pgfpathcurveto{\pgfqpoint{2.747276in}{2.019143in}}{\pgfqpoint{2.741977in}{2.016948in}}{\pgfqpoint{2.738070in}{2.013041in}}%
\pgfpathcurveto{\pgfqpoint{2.734163in}{2.009134in}}{\pgfqpoint{2.731968in}{2.003835in}}{\pgfqpoint{2.731968in}{1.998310in}}%
\pgfpathcurveto{\pgfqpoint{2.731968in}{1.992785in}}{\pgfqpoint{2.734163in}{1.987485in}}{\pgfqpoint{2.738070in}{1.983578in}}%
\pgfpathcurveto{\pgfqpoint{2.741977in}{1.979671in}}{\pgfqpoint{2.747276in}{1.977476in}}{\pgfqpoint{2.752801in}{1.977476in}}%
\pgfpathclose%
\pgfusepath{fill}%
\end{pgfscope}%
\begin{pgfscope}%
\pgfpathrectangle{\pgfqpoint{1.995685in}{1.588185in}}{\pgfqpoint{1.162500in}{0.755000in}} %
\pgfusepath{clip}%
\pgfsetbuttcap%
\pgfsetroundjoin%
\definecolor{currentfill}{rgb}{0.000000,0.000000,0.000000}%
\pgfsetfillcolor{currentfill}%
\pgfsetfillopacity{0.500000}%
\pgfsetlinewidth{0.000000pt}%
\definecolor{currentstroke}{rgb}{0.000000,0.000000,0.000000}%
\pgfsetstrokecolor{currentstroke}%
\pgfsetdash{}{0pt}%
\pgfpathmoveto{\pgfqpoint{2.103337in}{1.614683in}}%
\pgfpathcurveto{\pgfqpoint{2.108862in}{1.614683in}}{\pgfqpoint{2.114161in}{1.616879in}}{\pgfqpoint{2.118068in}{1.620785in}}%
\pgfpathcurveto{\pgfqpoint{2.121975in}{1.624692in}}{\pgfqpoint{2.124170in}{1.629992in}}{\pgfqpoint{2.124170in}{1.635517in}}%
\pgfpathcurveto{\pgfqpoint{2.124170in}{1.641042in}}{\pgfqpoint{2.121975in}{1.646341in}}{\pgfqpoint{2.118068in}{1.650248in}}%
\pgfpathcurveto{\pgfqpoint{2.114161in}{1.654155in}}{\pgfqpoint{2.108862in}{1.656350in}}{\pgfqpoint{2.103337in}{1.656350in}}%
\pgfpathcurveto{\pgfqpoint{2.097812in}{1.656350in}}{\pgfqpoint{2.092512in}{1.654155in}}{\pgfqpoint{2.088605in}{1.650248in}}%
\pgfpathcurveto{\pgfqpoint{2.084698in}{1.646341in}}{\pgfqpoint{2.082503in}{1.641042in}}{\pgfqpoint{2.082503in}{1.635517in}}%
\pgfpathcurveto{\pgfqpoint{2.082503in}{1.629992in}}{\pgfqpoint{2.084698in}{1.624692in}}{\pgfqpoint{2.088605in}{1.620785in}}%
\pgfpathcurveto{\pgfqpoint{2.092512in}{1.616879in}}{\pgfqpoint{2.097812in}{1.614683in}}{\pgfqpoint{2.103337in}{1.614683in}}%
\pgfpathclose%
\pgfusepath{fill}%
\end{pgfscope}%
\begin{pgfscope}%
\pgfsetrectcap%
\pgfsetmiterjoin%
\pgfsetlinewidth{0.803000pt}%
\definecolor{currentstroke}{rgb}{0.000000,0.000000,0.000000}%
\pgfsetstrokecolor{currentstroke}%
\pgfsetdash{}{0pt}%
\pgfpathmoveto{\pgfqpoint{1.995685in}{1.588185in}}%
\pgfpathlineto{\pgfqpoint{1.995685in}{2.343185in}}%
\pgfusepath{stroke}%
\end{pgfscope}%
\begin{pgfscope}%
\pgfsetrectcap%
\pgfsetmiterjoin%
\pgfsetlinewidth{0.803000pt}%
\definecolor{currentstroke}{rgb}{0.000000,0.000000,0.000000}%
\pgfsetstrokecolor{currentstroke}%
\pgfsetdash{}{0pt}%
\pgfpathmoveto{\pgfqpoint{3.158185in}{1.588185in}}%
\pgfpathlineto{\pgfqpoint{3.158185in}{2.343185in}}%
\pgfusepath{stroke}%
\end{pgfscope}%
\begin{pgfscope}%
\pgfsetrectcap%
\pgfsetmiterjoin%
\pgfsetlinewidth{0.803000pt}%
\definecolor{currentstroke}{rgb}{0.000000,0.000000,0.000000}%
\pgfsetstrokecolor{currentstroke}%
\pgfsetdash{}{0pt}%
\pgfpathmoveto{\pgfqpoint{1.995685in}{1.588185in}}%
\pgfpathlineto{\pgfqpoint{3.158185in}{1.588185in}}%
\pgfusepath{stroke}%
\end{pgfscope}%
\begin{pgfscope}%
\pgfsetrectcap%
\pgfsetmiterjoin%
\pgfsetlinewidth{0.803000pt}%
\definecolor{currentstroke}{rgb}{0.000000,0.000000,0.000000}%
\pgfsetstrokecolor{currentstroke}%
\pgfsetdash{}{0pt}%
\pgfpathmoveto{\pgfqpoint{1.995685in}{2.343185in}}%
\pgfpathlineto{\pgfqpoint{3.158185in}{2.343185in}}%
\pgfusepath{stroke}%
\end{pgfscope}%
\begin{pgfscope}%
\pgfsetbuttcap%
\pgfsetmiterjoin%
\definecolor{currentfill}{rgb}{1.000000,1.000000,1.000000}%
\pgfsetfillcolor{currentfill}%
\pgfsetlinewidth{0.000000pt}%
\definecolor{currentstroke}{rgb}{0.000000,0.000000,0.000000}%
\pgfsetstrokecolor{currentstroke}%
\pgfsetstrokeopacity{0.000000}%
\pgfsetdash{}{0pt}%
\pgfpathmoveto{\pgfqpoint{3.158185in}{1.588185in}}%
\pgfpathlineto{\pgfqpoint{4.320685in}{1.588185in}}%
\pgfpathlineto{\pgfqpoint{4.320685in}{2.343185in}}%
\pgfpathlineto{\pgfqpoint{3.158185in}{2.343185in}}%
\pgfpathclose%
\pgfusepath{fill}%
\end{pgfscope}%
\begin{pgfscope}%
\pgfpathrectangle{\pgfqpoint{3.158185in}{1.588185in}}{\pgfqpoint{1.162500in}{0.755000in}} %
\pgfusepath{clip}%
\pgfsetrectcap%
\pgfsetroundjoin%
\pgfsetlinewidth{1.505625pt}%
\definecolor{currentstroke}{rgb}{0.121569,0.466667,0.705882}%
\pgfsetstrokecolor{currentstroke}%
\pgfsetdash{}{0pt}%
\pgfpathmoveto{\pgfqpoint{3.185863in}{2.104447in}}%
\pgfpathlineto{\pgfqpoint{3.211353in}{2.150561in}}%
\pgfpathlineto{\pgfqpoint{3.233518in}{2.186403in}}%
\pgfpathlineto{\pgfqpoint{3.254575in}{2.216289in}}%
\pgfpathlineto{\pgfqpoint{3.273415in}{2.239364in}}%
\pgfpathlineto{\pgfqpoint{3.292255in}{2.258893in}}%
\pgfpathlineto{\pgfqpoint{3.309987in}{2.274058in}}%
\pgfpathlineto{\pgfqpoint{3.327719in}{2.286206in}}%
\pgfpathlineto{\pgfqpoint{3.344343in}{2.294999in}}%
\pgfpathlineto{\pgfqpoint{3.362075in}{2.301807in}}%
\pgfpathlineto{\pgfqpoint{3.379807in}{2.306200in}}%
\pgfpathlineto{\pgfqpoint{3.398647in}{2.308514in}}%
\pgfpathlineto{\pgfqpoint{3.418596in}{2.308664in}}%
\pgfpathlineto{\pgfqpoint{3.440761in}{2.306475in}}%
\pgfpathlineto{\pgfqpoint{3.465142in}{2.301687in}}%
\pgfpathlineto{\pgfqpoint{3.491740in}{2.294102in}}%
\pgfpathlineto{\pgfqpoint{3.520555in}{2.283496in}}%
\pgfpathlineto{\pgfqpoint{3.550478in}{2.270030in}}%
\pgfpathlineto{\pgfqpoint{3.579292in}{2.254637in}}%
\pgfpathlineto{\pgfqpoint{3.606999in}{2.237399in}}%
\pgfpathlineto{\pgfqpoint{3.634705in}{2.217533in}}%
\pgfpathlineto{\pgfqpoint{3.662411in}{2.194829in}}%
\pgfpathlineto{\pgfqpoint{3.690117in}{2.169146in}}%
\pgfpathlineto{\pgfqpoint{3.717824in}{2.140453in}}%
\pgfpathlineto{\pgfqpoint{3.747746in}{2.106189in}}%
\pgfpathlineto{\pgfqpoint{3.779886in}{2.065923in}}%
\pgfpathlineto{\pgfqpoint{3.817566in}{2.014928in}}%
\pgfpathlineto{\pgfqpoint{3.869654in}{1.940191in}}%
\pgfpathlineto{\pgfqpoint{3.940582in}{1.838745in}}%
\pgfpathlineto{\pgfqpoint{3.974938in}{1.793588in}}%
\pgfpathlineto{\pgfqpoint{4.003752in}{1.759310in}}%
\pgfpathlineto{\pgfqpoint{4.029242in}{1.732355in}}%
\pgfpathlineto{\pgfqpoint{4.052516in}{1.710813in}}%
\pgfpathlineto{\pgfqpoint{4.074681in}{1.693141in}}%
\pgfpathlineto{\pgfqpoint{4.096846in}{1.678234in}}%
\pgfpathlineto{\pgfqpoint{4.119011in}{1.665964in}}%
\pgfpathlineto{\pgfqpoint{4.142284in}{1.655659in}}%
\pgfpathlineto{\pgfqpoint{4.166665in}{1.647288in}}%
\pgfpathlineto{\pgfqpoint{4.195480in}{1.639872in}}%
\pgfpathlineto{\pgfqpoint{4.233160in}{1.632692in}}%
\pgfpathlineto{\pgfqpoint{4.293006in}{1.622503in}}%
\pgfpathlineto{\pgfqpoint{4.293006in}{1.622503in}}%
\pgfusepath{stroke}%
\end{pgfscope}%
\begin{pgfscope}%
\pgfsetrectcap%
\pgfsetmiterjoin%
\pgfsetlinewidth{0.803000pt}%
\definecolor{currentstroke}{rgb}{0.000000,0.000000,0.000000}%
\pgfsetstrokecolor{currentstroke}%
\pgfsetdash{}{0pt}%
\pgfpathmoveto{\pgfqpoint{3.158185in}{1.588185in}}%
\pgfpathlineto{\pgfqpoint{3.158185in}{2.343185in}}%
\pgfusepath{stroke}%
\end{pgfscope}%
\begin{pgfscope}%
\pgfsetrectcap%
\pgfsetmiterjoin%
\pgfsetlinewidth{0.803000pt}%
\definecolor{currentstroke}{rgb}{0.000000,0.000000,0.000000}%
\pgfsetstrokecolor{currentstroke}%
\pgfsetdash{}{0pt}%
\pgfpathmoveto{\pgfqpoint{4.320685in}{1.588185in}}%
\pgfpathlineto{\pgfqpoint{4.320685in}{2.343185in}}%
\pgfusepath{stroke}%
\end{pgfscope}%
\begin{pgfscope}%
\pgfsetrectcap%
\pgfsetmiterjoin%
\pgfsetlinewidth{0.803000pt}%
\definecolor{currentstroke}{rgb}{0.000000,0.000000,0.000000}%
\pgfsetstrokecolor{currentstroke}%
\pgfsetdash{}{0pt}%
\pgfpathmoveto{\pgfqpoint{3.158185in}{1.588185in}}%
\pgfpathlineto{\pgfqpoint{4.320685in}{1.588185in}}%
\pgfusepath{stroke}%
\end{pgfscope}%
\begin{pgfscope}%
\pgfsetrectcap%
\pgfsetmiterjoin%
\pgfsetlinewidth{0.803000pt}%
\definecolor{currentstroke}{rgb}{0.000000,0.000000,0.000000}%
\pgfsetstrokecolor{currentstroke}%
\pgfsetdash{}{0pt}%
\pgfpathmoveto{\pgfqpoint{3.158185in}{2.343185in}}%
\pgfpathlineto{\pgfqpoint{4.320685in}{2.343185in}}%
\pgfusepath{stroke}%
\end{pgfscope}%
\begin{pgfscope}%
\pgfsetbuttcap%
\pgfsetmiterjoin%
\definecolor{currentfill}{rgb}{1.000000,1.000000,1.000000}%
\pgfsetfillcolor{currentfill}%
\pgfsetlinewidth{0.000000pt}%
\definecolor{currentstroke}{rgb}{0.000000,0.000000,0.000000}%
\pgfsetstrokecolor{currentstroke}%
\pgfsetstrokeopacity{0.000000}%
\pgfsetdash{}{0pt}%
\pgfpathmoveto{\pgfqpoint{4.320685in}{1.588185in}}%
\pgfpathlineto{\pgfqpoint{5.483185in}{1.588185in}}%
\pgfpathlineto{\pgfqpoint{5.483185in}{2.343185in}}%
\pgfpathlineto{\pgfqpoint{4.320685in}{2.343185in}}%
\pgfpathclose%
\pgfusepath{fill}%
\end{pgfscope}%
\begin{pgfscope}%
\pgfpathrectangle{\pgfqpoint{4.320685in}{1.588185in}}{\pgfqpoint{1.162500in}{0.755000in}} %
\pgfusepath{clip}%
\pgfsetbuttcap%
\pgfsetroundjoin%
\definecolor{currentfill}{rgb}{0.000000,0.000000,0.000000}%
\pgfsetfillcolor{currentfill}%
\pgfsetfillopacity{0.500000}%
\pgfsetlinewidth{0.000000pt}%
\definecolor{currentstroke}{rgb}{0.000000,0.000000,0.000000}%
\pgfsetstrokecolor{currentstroke}%
\pgfsetdash{}{0pt}%
\pgfpathmoveto{\pgfqpoint{4.763543in}{2.304375in}}%
\pgfpathcurveto{\pgfqpoint{4.769068in}{2.304375in}}{\pgfqpoint{4.774367in}{2.306570in}}{\pgfqpoint{4.778274in}{2.310477in}}%
\pgfpathcurveto{\pgfqpoint{4.782181in}{2.314384in}}{\pgfqpoint{4.784376in}{2.319683in}}{\pgfqpoint{4.784376in}{2.325208in}}%
\pgfpathcurveto{\pgfqpoint{4.784376in}{2.330733in}}{\pgfqpoint{4.782181in}{2.336033in}}{\pgfqpoint{4.778274in}{2.339940in}}%
\pgfpathcurveto{\pgfqpoint{4.774367in}{2.343847in}}{\pgfqpoint{4.769068in}{2.346042in}}{\pgfqpoint{4.763543in}{2.346042in}}%
\pgfpathcurveto{\pgfqpoint{4.758018in}{2.346042in}}{\pgfqpoint{4.752718in}{2.343847in}}{\pgfqpoint{4.748811in}{2.339940in}}%
\pgfpathcurveto{\pgfqpoint{4.744905in}{2.336033in}}{\pgfqpoint{4.742709in}{2.330733in}}{\pgfqpoint{4.742709in}{2.325208in}}%
\pgfpathcurveto{\pgfqpoint{4.742709in}{2.319683in}}{\pgfqpoint{4.744905in}{2.314384in}}{\pgfqpoint{4.748811in}{2.310477in}}%
\pgfpathcurveto{\pgfqpoint{4.752718in}{2.306570in}}{\pgfqpoint{4.758018in}{2.304375in}}{\pgfqpoint{4.763543in}{2.304375in}}%
\pgfpathclose%
\pgfusepath{fill}%
\end{pgfscope}%
\begin{pgfscope}%
\pgfpathrectangle{\pgfqpoint{4.320685in}{1.588185in}}{\pgfqpoint{1.162500in}{0.755000in}} %
\pgfusepath{clip}%
\pgfsetbuttcap%
\pgfsetroundjoin%
\definecolor{currentfill}{rgb}{0.000000,0.000000,0.000000}%
\pgfsetfillcolor{currentfill}%
\pgfsetfillopacity{0.500000}%
\pgfsetlinewidth{0.000000pt}%
\definecolor{currentstroke}{rgb}{0.000000,0.000000,0.000000}%
\pgfsetstrokecolor{currentstroke}%
\pgfsetdash{}{0pt}%
\pgfpathmoveto{\pgfqpoint{5.385388in}{1.845741in}}%
\pgfpathcurveto{\pgfqpoint{5.390913in}{1.845741in}}{\pgfqpoint{5.396213in}{1.847936in}}{\pgfqpoint{5.400119in}{1.851843in}}%
\pgfpathcurveto{\pgfqpoint{5.404026in}{1.855750in}}{\pgfqpoint{5.406221in}{1.861049in}}{\pgfqpoint{5.406221in}{1.866574in}}%
\pgfpathcurveto{\pgfqpoint{5.406221in}{1.872099in}}{\pgfqpoint{5.404026in}{1.877399in}}{\pgfqpoint{5.400119in}{1.881306in}}%
\pgfpathcurveto{\pgfqpoint{5.396213in}{1.885212in}}{\pgfqpoint{5.390913in}{1.887408in}}{\pgfqpoint{5.385388in}{1.887408in}}%
\pgfpathcurveto{\pgfqpoint{5.379863in}{1.887408in}}{\pgfqpoint{5.374564in}{1.885212in}}{\pgfqpoint{5.370657in}{1.881306in}}%
\pgfpathcurveto{\pgfqpoint{5.366750in}{1.877399in}}{\pgfqpoint{5.364555in}{1.872099in}}{\pgfqpoint{5.364555in}{1.866574in}}%
\pgfpathcurveto{\pgfqpoint{5.364555in}{1.861049in}}{\pgfqpoint{5.366750in}{1.855750in}}{\pgfqpoint{5.370657in}{1.851843in}}%
\pgfpathcurveto{\pgfqpoint{5.374564in}{1.847936in}}{\pgfqpoint{5.379863in}{1.845741in}}{\pgfqpoint{5.385388in}{1.845741in}}%
\pgfpathclose%
\pgfusepath{fill}%
\end{pgfscope}%
\begin{pgfscope}%
\pgfpathrectangle{\pgfqpoint{4.320685in}{1.588185in}}{\pgfqpoint{1.162500in}{0.755000in}} %
\pgfusepath{clip}%
\pgfsetbuttcap%
\pgfsetroundjoin%
\definecolor{currentfill}{rgb}{0.000000,0.000000,0.000000}%
\pgfsetfillcolor{currentfill}%
\pgfsetfillopacity{0.500000}%
\pgfsetlinewidth{0.000000pt}%
\definecolor{currentstroke}{rgb}{0.000000,0.000000,0.000000}%
\pgfsetstrokecolor{currentstroke}%
\pgfsetdash{}{0pt}%
\pgfpathmoveto{\pgfqpoint{5.455506in}{1.840570in}}%
\pgfpathcurveto{\pgfqpoint{5.461031in}{1.840570in}}{\pgfqpoint{5.466331in}{1.842766in}}{\pgfqpoint{5.470237in}{1.846672in}}%
\pgfpathcurveto{\pgfqpoint{5.474144in}{1.850579in}}{\pgfqpoint{5.476339in}{1.855879in}}{\pgfqpoint{5.476339in}{1.861404in}}%
\pgfpathcurveto{\pgfqpoint{5.476339in}{1.866929in}}{\pgfqpoint{5.474144in}{1.872228in}}{\pgfqpoint{5.470237in}{1.876135in}}%
\pgfpathcurveto{\pgfqpoint{5.466331in}{1.880042in}}{\pgfqpoint{5.461031in}{1.882237in}}{\pgfqpoint{5.455506in}{1.882237in}}%
\pgfpathcurveto{\pgfqpoint{5.449981in}{1.882237in}}{\pgfqpoint{5.444681in}{1.880042in}}{\pgfqpoint{5.440775in}{1.876135in}}%
\pgfpathcurveto{\pgfqpoint{5.436868in}{1.872228in}}{\pgfqpoint{5.434673in}{1.866929in}}{\pgfqpoint{5.434673in}{1.861404in}}%
\pgfpathcurveto{\pgfqpoint{5.434673in}{1.855879in}}{\pgfqpoint{5.436868in}{1.850579in}}{\pgfqpoint{5.440775in}{1.846672in}}%
\pgfpathcurveto{\pgfqpoint{5.444681in}{1.842766in}}{\pgfqpoint{5.449981in}{1.840570in}}{\pgfqpoint{5.455506in}{1.840570in}}%
\pgfpathclose%
\pgfusepath{fill}%
\end{pgfscope}%
\begin{pgfscope}%
\pgfpathrectangle{\pgfqpoint{4.320685in}{1.588185in}}{\pgfqpoint{1.162500in}{0.755000in}} %
\pgfusepath{clip}%
\pgfsetbuttcap%
\pgfsetroundjoin%
\definecolor{currentfill}{rgb}{0.000000,0.000000,0.000000}%
\pgfsetfillcolor{currentfill}%
\pgfsetfillopacity{0.500000}%
\pgfsetlinewidth{0.000000pt}%
\definecolor{currentstroke}{rgb}{0.000000,0.000000,0.000000}%
\pgfsetstrokecolor{currentstroke}%
\pgfsetdash{}{0pt}%
\pgfpathmoveto{\pgfqpoint{5.031128in}{1.680932in}}%
\pgfpathcurveto{\pgfqpoint{5.036653in}{1.680932in}}{\pgfqpoint{5.041953in}{1.683127in}}{\pgfqpoint{5.045859in}{1.687034in}}%
\pgfpathcurveto{\pgfqpoint{5.049766in}{1.690941in}}{\pgfqpoint{5.051961in}{1.696241in}}{\pgfqpoint{5.051961in}{1.701766in}}%
\pgfpathcurveto{\pgfqpoint{5.051961in}{1.707291in}}{\pgfqpoint{5.049766in}{1.712590in}}{\pgfqpoint{5.045859in}{1.716497in}}%
\pgfpathcurveto{\pgfqpoint{5.041953in}{1.720404in}}{\pgfqpoint{5.036653in}{1.722599in}}{\pgfqpoint{5.031128in}{1.722599in}}%
\pgfpathcurveto{\pgfqpoint{5.025603in}{1.722599in}}{\pgfqpoint{5.020303in}{1.720404in}}{\pgfqpoint{5.016397in}{1.716497in}}%
\pgfpathcurveto{\pgfqpoint{5.012490in}{1.712590in}}{\pgfqpoint{5.010295in}{1.707291in}}{\pgfqpoint{5.010295in}{1.701766in}}%
\pgfpathcurveto{\pgfqpoint{5.010295in}{1.696241in}}{\pgfqpoint{5.012490in}{1.690941in}}{\pgfqpoint{5.016397in}{1.687034in}}%
\pgfpathcurveto{\pgfqpoint{5.020303in}{1.683127in}}{\pgfqpoint{5.025603in}{1.680932in}}{\pgfqpoint{5.031128in}{1.680932in}}%
\pgfpathclose%
\pgfusepath{fill}%
\end{pgfscope}%
\begin{pgfscope}%
\pgfpathrectangle{\pgfqpoint{4.320685in}{1.588185in}}{\pgfqpoint{1.162500in}{0.755000in}} %
\pgfusepath{clip}%
\pgfsetbuttcap%
\pgfsetroundjoin%
\definecolor{currentfill}{rgb}{0.000000,0.000000,0.000000}%
\pgfsetfillcolor{currentfill}%
\pgfsetfillopacity{0.500000}%
\pgfsetlinewidth{0.000000pt}%
\definecolor{currentstroke}{rgb}{0.000000,0.000000,0.000000}%
\pgfsetstrokecolor{currentstroke}%
\pgfsetdash{}{0pt}%
\pgfpathmoveto{\pgfqpoint{5.102087in}{1.656866in}}%
\pgfpathcurveto{\pgfqpoint{5.107612in}{1.656866in}}{\pgfqpoint{5.112912in}{1.659061in}}{\pgfqpoint{5.116819in}{1.662968in}}%
\pgfpathcurveto{\pgfqpoint{5.120726in}{1.666874in}}{\pgfqpoint{5.122921in}{1.672174in}}{\pgfqpoint{5.122921in}{1.677699in}}%
\pgfpathcurveto{\pgfqpoint{5.122921in}{1.683224in}}{\pgfqpoint{5.120726in}{1.688523in}}{\pgfqpoint{5.116819in}{1.692430in}}%
\pgfpathcurveto{\pgfqpoint{5.112912in}{1.696337in}}{\pgfqpoint{5.107612in}{1.698532in}}{\pgfqpoint{5.102087in}{1.698532in}}%
\pgfpathcurveto{\pgfqpoint{5.096562in}{1.698532in}}{\pgfqpoint{5.091263in}{1.696337in}}{\pgfqpoint{5.087356in}{1.692430in}}%
\pgfpathcurveto{\pgfqpoint{5.083449in}{1.688523in}}{\pgfqpoint{5.081254in}{1.683224in}}{\pgfqpoint{5.081254in}{1.677699in}}%
\pgfpathcurveto{\pgfqpoint{5.081254in}{1.672174in}}{\pgfqpoint{5.083449in}{1.666874in}}{\pgfqpoint{5.087356in}{1.662968in}}%
\pgfpathcurveto{\pgfqpoint{5.091263in}{1.659061in}}{\pgfqpoint{5.096562in}{1.656866in}}{\pgfqpoint{5.102087in}{1.656866in}}%
\pgfpathclose%
\pgfusepath{fill}%
\end{pgfscope}%
\begin{pgfscope}%
\pgfpathrectangle{\pgfqpoint{4.320685in}{1.588185in}}{\pgfqpoint{1.162500in}{0.755000in}} %
\pgfusepath{clip}%
\pgfsetbuttcap%
\pgfsetroundjoin%
\definecolor{currentfill}{rgb}{0.000000,0.000000,0.000000}%
\pgfsetfillcolor{currentfill}%
\pgfsetfillopacity{0.500000}%
\pgfsetlinewidth{0.000000pt}%
\definecolor{currentstroke}{rgb}{0.000000,0.000000,0.000000}%
\pgfsetstrokecolor{currentstroke}%
\pgfsetdash{}{0pt}%
\pgfpathmoveto{\pgfqpoint{4.940844in}{1.598883in}}%
\pgfpathcurveto{\pgfqpoint{4.946369in}{1.598883in}}{\pgfqpoint{4.951669in}{1.601078in}}{\pgfqpoint{4.955576in}{1.604985in}}%
\pgfpathcurveto{\pgfqpoint{4.959482in}{1.608892in}}{\pgfqpoint{4.961678in}{1.614191in}}{\pgfqpoint{4.961678in}{1.619716in}}%
\pgfpathcurveto{\pgfqpoint{4.961678in}{1.625241in}}{\pgfqpoint{4.959482in}{1.630541in}}{\pgfqpoint{4.955576in}{1.634447in}}%
\pgfpathcurveto{\pgfqpoint{4.951669in}{1.638354in}}{\pgfqpoint{4.946369in}{1.640549in}}{\pgfqpoint{4.940844in}{1.640549in}}%
\pgfpathcurveto{\pgfqpoint{4.935319in}{1.640549in}}{\pgfqpoint{4.930020in}{1.638354in}}{\pgfqpoint{4.926113in}{1.634447in}}%
\pgfpathcurveto{\pgfqpoint{4.922206in}{1.630541in}}{\pgfqpoint{4.920011in}{1.625241in}}{\pgfqpoint{4.920011in}{1.619716in}}%
\pgfpathcurveto{\pgfqpoint{4.920011in}{1.614191in}}{\pgfqpoint{4.922206in}{1.608892in}}{\pgfqpoint{4.926113in}{1.604985in}}%
\pgfpathcurveto{\pgfqpoint{4.930020in}{1.601078in}}{\pgfqpoint{4.935319in}{1.598883in}}{\pgfqpoint{4.940844in}{1.598883in}}%
\pgfpathclose%
\pgfusepath{fill}%
\end{pgfscope}%
\begin{pgfscope}%
\pgfpathrectangle{\pgfqpoint{4.320685in}{1.588185in}}{\pgfqpoint{1.162500in}{0.755000in}} %
\pgfusepath{clip}%
\pgfsetbuttcap%
\pgfsetroundjoin%
\definecolor{currentfill}{rgb}{0.000000,0.000000,0.000000}%
\pgfsetfillcolor{currentfill}%
\pgfsetfillopacity{0.500000}%
\pgfsetlinewidth{0.000000pt}%
\definecolor{currentstroke}{rgb}{0.000000,0.000000,0.000000}%
\pgfsetstrokecolor{currentstroke}%
\pgfsetdash{}{0pt}%
\pgfpathmoveto{\pgfqpoint{4.348363in}{1.585327in}}%
\pgfpathcurveto{\pgfqpoint{4.353888in}{1.585327in}}{\pgfqpoint{4.359188in}{1.587523in}}{\pgfqpoint{4.363095in}{1.591429in}}%
\pgfpathcurveto{\pgfqpoint{4.367001in}{1.595336in}}{\pgfqpoint{4.369196in}{1.600636in}}{\pgfqpoint{4.369196in}{1.606161in}}%
\pgfpathcurveto{\pgfqpoint{4.369196in}{1.611686in}}{\pgfqpoint{4.367001in}{1.616985in}}{\pgfqpoint{4.363095in}{1.620892in}}%
\pgfpathcurveto{\pgfqpoint{4.359188in}{1.624799in}}{\pgfqpoint{4.353888in}{1.626994in}}{\pgfqpoint{4.348363in}{1.626994in}}%
\pgfpathcurveto{\pgfqpoint{4.342838in}{1.626994in}}{\pgfqpoint{4.337539in}{1.624799in}}{\pgfqpoint{4.333632in}{1.620892in}}%
\pgfpathcurveto{\pgfqpoint{4.329725in}{1.616985in}}{\pgfqpoint{4.327530in}{1.611686in}}{\pgfqpoint{4.327530in}{1.606161in}}%
\pgfpathcurveto{\pgfqpoint{4.327530in}{1.600636in}}{\pgfqpoint{4.329725in}{1.595336in}}{\pgfqpoint{4.333632in}{1.591429in}}%
\pgfpathcurveto{\pgfqpoint{4.337539in}{1.587523in}}{\pgfqpoint{4.342838in}{1.585327in}}{\pgfqpoint{4.348363in}{1.585327in}}%
\pgfpathclose%
\pgfusepath{fill}%
\end{pgfscope}%
\begin{pgfscope}%
\pgfpathrectangle{\pgfqpoint{4.320685in}{1.588185in}}{\pgfqpoint{1.162500in}{0.755000in}} %
\pgfusepath{clip}%
\pgfsetbuttcap%
\pgfsetroundjoin%
\definecolor{currentfill}{rgb}{0.000000,0.000000,0.000000}%
\pgfsetfillcolor{currentfill}%
\pgfsetfillopacity{0.500000}%
\pgfsetlinewidth{0.000000pt}%
\definecolor{currentstroke}{rgb}{0.000000,0.000000,0.000000}%
\pgfsetstrokecolor{currentstroke}%
\pgfsetdash{}{0pt}%
\pgfpathmoveto{\pgfqpoint{5.332823in}{1.989747in}}%
\pgfpathcurveto{\pgfqpoint{5.338348in}{1.989747in}}{\pgfqpoint{5.343648in}{1.991942in}}{\pgfqpoint{5.347555in}{1.995849in}}%
\pgfpathcurveto{\pgfqpoint{5.351461in}{1.999756in}}{\pgfqpoint{5.353656in}{2.005055in}}{\pgfqpoint{5.353656in}{2.010580in}}%
\pgfpathcurveto{\pgfqpoint{5.353656in}{2.016105in}}{\pgfqpoint{5.351461in}{2.021405in}}{\pgfqpoint{5.347555in}{2.025311in}}%
\pgfpathcurveto{\pgfqpoint{5.343648in}{2.029218in}}{\pgfqpoint{5.338348in}{2.031413in}}{\pgfqpoint{5.332823in}{2.031413in}}%
\pgfpathcurveto{\pgfqpoint{5.327298in}{2.031413in}}{\pgfqpoint{5.321999in}{2.029218in}}{\pgfqpoint{5.318092in}{2.025311in}}%
\pgfpathcurveto{\pgfqpoint{5.314185in}{2.021405in}}{\pgfqpoint{5.311990in}{2.016105in}}{\pgfqpoint{5.311990in}{2.010580in}}%
\pgfpathcurveto{\pgfqpoint{5.311990in}{2.005055in}}{\pgfqpoint{5.314185in}{1.999756in}}{\pgfqpoint{5.318092in}{1.995849in}}%
\pgfpathcurveto{\pgfqpoint{5.321999in}{1.991942in}}{\pgfqpoint{5.327298in}{1.989747in}}{\pgfqpoint{5.332823in}{1.989747in}}%
\pgfpathclose%
\pgfusepath{fill}%
\end{pgfscope}%
\begin{pgfscope}%
\pgfpathrectangle{\pgfqpoint{4.320685in}{1.588185in}}{\pgfqpoint{1.162500in}{0.755000in}} %
\pgfusepath{clip}%
\pgfsetbuttcap%
\pgfsetroundjoin%
\definecolor{currentfill}{rgb}{0.000000,0.000000,0.000000}%
\pgfsetfillcolor{currentfill}%
\pgfsetfillopacity{0.500000}%
\pgfsetlinewidth{0.000000pt}%
\definecolor{currentstroke}{rgb}{0.000000,0.000000,0.000000}%
\pgfsetstrokecolor{currentstroke}%
\pgfsetdash{}{0pt}%
\pgfpathmoveto{\pgfqpoint{5.090412in}{1.867889in}}%
\pgfpathcurveto{\pgfqpoint{5.095937in}{1.867889in}}{\pgfqpoint{5.101237in}{1.870084in}}{\pgfqpoint{5.105143in}{1.873991in}}%
\pgfpathcurveto{\pgfqpoint{5.109050in}{1.877898in}}{\pgfqpoint{5.111245in}{1.883197in}}{\pgfqpoint{5.111245in}{1.888722in}}%
\pgfpathcurveto{\pgfqpoint{5.111245in}{1.894247in}}{\pgfqpoint{5.109050in}{1.899547in}}{\pgfqpoint{5.105143in}{1.903454in}}%
\pgfpathcurveto{\pgfqpoint{5.101237in}{1.907360in}}{\pgfqpoint{5.095937in}{1.909556in}}{\pgfqpoint{5.090412in}{1.909556in}}%
\pgfpathcurveto{\pgfqpoint{5.084887in}{1.909556in}}{\pgfqpoint{5.079587in}{1.907360in}}{\pgfqpoint{5.075681in}{1.903454in}}%
\pgfpathcurveto{\pgfqpoint{5.071774in}{1.899547in}}{\pgfqpoint{5.069579in}{1.894247in}}{\pgfqpoint{5.069579in}{1.888722in}}%
\pgfpathcurveto{\pgfqpoint{5.069579in}{1.883197in}}{\pgfqpoint{5.071774in}{1.877898in}}{\pgfqpoint{5.075681in}{1.873991in}}%
\pgfpathcurveto{\pgfqpoint{5.079587in}{1.870084in}}{\pgfqpoint{5.084887in}{1.867889in}}{\pgfqpoint{5.090412in}{1.867889in}}%
\pgfpathclose%
\pgfusepath{fill}%
\end{pgfscope}%
\begin{pgfscope}%
\pgfpathrectangle{\pgfqpoint{4.320685in}{1.588185in}}{\pgfqpoint{1.162500in}{0.755000in}} %
\pgfusepath{clip}%
\pgfsetbuttcap%
\pgfsetroundjoin%
\definecolor{currentfill}{rgb}{0.000000,0.000000,0.000000}%
\pgfsetfillcolor{currentfill}%
\pgfsetfillopacity{0.500000}%
\pgfsetlinewidth{0.000000pt}%
\definecolor{currentstroke}{rgb}{0.000000,0.000000,0.000000}%
\pgfsetstrokecolor{currentstroke}%
\pgfsetdash{}{0pt}%
\pgfpathmoveto{\pgfqpoint{4.817426in}{1.854418in}}%
\pgfpathcurveto{\pgfqpoint{4.822951in}{1.854418in}}{\pgfqpoint{4.828251in}{1.856613in}}{\pgfqpoint{4.832158in}{1.860520in}}%
\pgfpathcurveto{\pgfqpoint{4.836065in}{1.864427in}}{\pgfqpoint{4.838260in}{1.869726in}}{\pgfqpoint{4.838260in}{1.875251in}}%
\pgfpathcurveto{\pgfqpoint{4.838260in}{1.880776in}}{\pgfqpoint{4.836065in}{1.886076in}}{\pgfqpoint{4.832158in}{1.889982in}}%
\pgfpathcurveto{\pgfqpoint{4.828251in}{1.893889in}}{\pgfqpoint{4.822951in}{1.896084in}}{\pgfqpoint{4.817426in}{1.896084in}}%
\pgfpathcurveto{\pgfqpoint{4.811901in}{1.896084in}}{\pgfqpoint{4.806602in}{1.893889in}}{\pgfqpoint{4.802695in}{1.889982in}}%
\pgfpathcurveto{\pgfqpoint{4.798788in}{1.886076in}}{\pgfqpoint{4.796593in}{1.880776in}}{\pgfqpoint{4.796593in}{1.875251in}}%
\pgfpathcurveto{\pgfqpoint{4.796593in}{1.869726in}}{\pgfqpoint{4.798788in}{1.864427in}}{\pgfqpoint{4.802695in}{1.860520in}}%
\pgfpathcurveto{\pgfqpoint{4.806602in}{1.856613in}}{\pgfqpoint{4.811901in}{1.854418in}}{\pgfqpoint{4.817426in}{1.854418in}}%
\pgfpathclose%
\pgfusepath{fill}%
\end{pgfscope}%
\begin{pgfscope}%
\pgfpathrectangle{\pgfqpoint{4.320685in}{1.588185in}}{\pgfqpoint{1.162500in}{0.755000in}} %
\pgfusepath{clip}%
\pgfsetbuttcap%
\pgfsetroundjoin%
\definecolor{currentfill}{rgb}{0.000000,0.000000,0.000000}%
\pgfsetfillcolor{currentfill}%
\pgfsetfillopacity{0.500000}%
\pgfsetlinewidth{0.000000pt}%
\definecolor{currentstroke}{rgb}{0.000000,0.000000,0.000000}%
\pgfsetstrokecolor{currentstroke}%
\pgfsetdash{}{0pt}%
\pgfpathmoveto{\pgfqpoint{5.258215in}{2.049107in}}%
\pgfpathcurveto{\pgfqpoint{5.263740in}{2.049107in}}{\pgfqpoint{5.269040in}{2.051302in}}{\pgfqpoint{5.272947in}{2.055209in}}%
\pgfpathcurveto{\pgfqpoint{5.276853in}{2.059115in}}{\pgfqpoint{5.279048in}{2.064415in}}{\pgfqpoint{5.279048in}{2.069940in}}%
\pgfpathcurveto{\pgfqpoint{5.279048in}{2.075465in}}{\pgfqpoint{5.276853in}{2.080764in}}{\pgfqpoint{5.272947in}{2.084671in}}%
\pgfpathcurveto{\pgfqpoint{5.269040in}{2.088578in}}{\pgfqpoint{5.263740in}{2.090773in}}{\pgfqpoint{5.258215in}{2.090773in}}%
\pgfpathcurveto{\pgfqpoint{5.252690in}{2.090773in}}{\pgfqpoint{5.247391in}{2.088578in}}{\pgfqpoint{5.243484in}{2.084671in}}%
\pgfpathcurveto{\pgfqpoint{5.239577in}{2.080764in}}{\pgfqpoint{5.237382in}{2.075465in}}{\pgfqpoint{5.237382in}{2.069940in}}%
\pgfpathcurveto{\pgfqpoint{5.237382in}{2.064415in}}{\pgfqpoint{5.239577in}{2.059115in}}{\pgfqpoint{5.243484in}{2.055209in}}%
\pgfpathcurveto{\pgfqpoint{5.247391in}{2.051302in}}{\pgfqpoint{5.252690in}{2.049107in}}{\pgfqpoint{5.258215in}{2.049107in}}%
\pgfpathclose%
\pgfusepath{fill}%
\end{pgfscope}%
\begin{pgfscope}%
\pgfpathrectangle{\pgfqpoint{4.320685in}{1.588185in}}{\pgfqpoint{1.162500in}{0.755000in}} %
\pgfusepath{clip}%
\pgfsetbuttcap%
\pgfsetroundjoin%
\definecolor{currentfill}{rgb}{0.000000,0.000000,0.000000}%
\pgfsetfillcolor{currentfill}%
\pgfsetfillopacity{0.500000}%
\pgfsetlinewidth{0.000000pt}%
\definecolor{currentstroke}{rgb}{0.000000,0.000000,0.000000}%
\pgfsetstrokecolor{currentstroke}%
\pgfsetdash{}{0pt}%
\pgfpathmoveto{\pgfqpoint{4.803827in}{1.703826in}}%
\pgfpathcurveto{\pgfqpoint{4.809352in}{1.703826in}}{\pgfqpoint{4.814651in}{1.706022in}}{\pgfqpoint{4.818558in}{1.709928in}}%
\pgfpathcurveto{\pgfqpoint{4.822465in}{1.713835in}}{\pgfqpoint{4.824660in}{1.719135in}}{\pgfqpoint{4.824660in}{1.724660in}}%
\pgfpathcurveto{\pgfqpoint{4.824660in}{1.730185in}}{\pgfqpoint{4.822465in}{1.735484in}}{\pgfqpoint{4.818558in}{1.739391in}}%
\pgfpathcurveto{\pgfqpoint{4.814651in}{1.743298in}}{\pgfqpoint{4.809352in}{1.745493in}}{\pgfqpoint{4.803827in}{1.745493in}}%
\pgfpathcurveto{\pgfqpoint{4.798302in}{1.745493in}}{\pgfqpoint{4.793002in}{1.743298in}}{\pgfqpoint{4.789096in}{1.739391in}}%
\pgfpathcurveto{\pgfqpoint{4.785189in}{1.735484in}}{\pgfqpoint{4.782994in}{1.730185in}}{\pgfqpoint{4.782994in}{1.724660in}}%
\pgfpathcurveto{\pgfqpoint{4.782994in}{1.719135in}}{\pgfqpoint{4.785189in}{1.713835in}}{\pgfqpoint{4.789096in}{1.709928in}}%
\pgfpathcurveto{\pgfqpoint{4.793002in}{1.706022in}}{\pgfqpoint{4.798302in}{1.703826in}}{\pgfqpoint{4.803827in}{1.703826in}}%
\pgfpathclose%
\pgfusepath{fill}%
\end{pgfscope}%
\begin{pgfscope}%
\pgfpathrectangle{\pgfqpoint{4.320685in}{1.588185in}}{\pgfqpoint{1.162500in}{0.755000in}} %
\pgfusepath{clip}%
\pgfsetbuttcap%
\pgfsetroundjoin%
\definecolor{currentfill}{rgb}{0.000000,0.000000,0.000000}%
\pgfsetfillcolor{currentfill}%
\pgfsetfillopacity{0.500000}%
\pgfsetlinewidth{0.000000pt}%
\definecolor{currentstroke}{rgb}{0.000000,0.000000,0.000000}%
\pgfsetstrokecolor{currentstroke}%
\pgfsetdash{}{0pt}%
\pgfpathmoveto{\pgfqpoint{4.437194in}{1.734907in}}%
\pgfpathcurveto{\pgfqpoint{4.442719in}{1.734907in}}{\pgfqpoint{4.448018in}{1.737102in}}{\pgfqpoint{4.451925in}{1.741009in}}%
\pgfpathcurveto{\pgfqpoint{4.455832in}{1.744915in}}{\pgfqpoint{4.458027in}{1.750215in}}{\pgfqpoint{4.458027in}{1.755740in}}%
\pgfpathcurveto{\pgfqpoint{4.458027in}{1.761265in}}{\pgfqpoint{4.455832in}{1.766565in}}{\pgfqpoint{4.451925in}{1.770471in}}%
\pgfpathcurveto{\pgfqpoint{4.448018in}{1.774378in}}{\pgfqpoint{4.442719in}{1.776573in}}{\pgfqpoint{4.437194in}{1.776573in}}%
\pgfpathcurveto{\pgfqpoint{4.431669in}{1.776573in}}{\pgfqpoint{4.426369in}{1.774378in}}{\pgfqpoint{4.422462in}{1.770471in}}%
\pgfpathcurveto{\pgfqpoint{4.418556in}{1.766565in}}{\pgfqpoint{4.416360in}{1.761265in}}{\pgfqpoint{4.416360in}{1.755740in}}%
\pgfpathcurveto{\pgfqpoint{4.416360in}{1.750215in}}{\pgfqpoint{4.418556in}{1.744915in}}{\pgfqpoint{4.422462in}{1.741009in}}%
\pgfpathcurveto{\pgfqpoint{4.426369in}{1.737102in}}{\pgfqpoint{4.431669in}{1.734907in}}{\pgfqpoint{4.437194in}{1.734907in}}%
\pgfpathclose%
\pgfusepath{fill}%
\end{pgfscope}%
\begin{pgfscope}%
\pgfpathrectangle{\pgfqpoint{4.320685in}{1.588185in}}{\pgfqpoint{1.162500in}{0.755000in}} %
\pgfusepath{clip}%
\pgfsetbuttcap%
\pgfsetroundjoin%
\definecolor{currentfill}{rgb}{0.000000,0.000000,0.000000}%
\pgfsetfillcolor{currentfill}%
\pgfsetfillopacity{0.500000}%
\pgfsetlinewidth{0.000000pt}%
\definecolor{currentstroke}{rgb}{0.000000,0.000000,0.000000}%
\pgfsetstrokecolor{currentstroke}%
\pgfsetdash{}{0pt}%
\pgfpathmoveto{\pgfqpoint{5.262967in}{1.977476in}}%
\pgfpathcurveto{\pgfqpoint{5.268492in}{1.977476in}}{\pgfqpoint{5.273791in}{1.979671in}}{\pgfqpoint{5.277698in}{1.983578in}}%
\pgfpathcurveto{\pgfqpoint{5.281605in}{1.987485in}}{\pgfqpoint{5.283800in}{1.992785in}}{\pgfqpoint{5.283800in}{1.998310in}}%
\pgfpathcurveto{\pgfqpoint{5.283800in}{2.003835in}}{\pgfqpoint{5.281605in}{2.009134in}}{\pgfqpoint{5.277698in}{2.013041in}}%
\pgfpathcurveto{\pgfqpoint{5.273791in}{2.016948in}}{\pgfqpoint{5.268492in}{2.019143in}}{\pgfqpoint{5.262967in}{2.019143in}}%
\pgfpathcurveto{\pgfqpoint{5.257442in}{2.019143in}}{\pgfqpoint{5.252142in}{2.016948in}}{\pgfqpoint{5.248235in}{2.013041in}}%
\pgfpathcurveto{\pgfqpoint{5.244329in}{2.009134in}}{\pgfqpoint{5.242134in}{2.003835in}}{\pgfqpoint{5.242134in}{1.998310in}}%
\pgfpathcurveto{\pgfqpoint{5.242134in}{1.992785in}}{\pgfqpoint{5.244329in}{1.987485in}}{\pgfqpoint{5.248235in}{1.983578in}}%
\pgfpathcurveto{\pgfqpoint{5.252142in}{1.979671in}}{\pgfqpoint{5.257442in}{1.977476in}}{\pgfqpoint{5.262967in}{1.977476in}}%
\pgfpathclose%
\pgfusepath{fill}%
\end{pgfscope}%
\begin{pgfscope}%
\pgfpathrectangle{\pgfqpoint{4.320685in}{1.588185in}}{\pgfqpoint{1.162500in}{0.755000in}} %
\pgfusepath{clip}%
\pgfsetbuttcap%
\pgfsetroundjoin%
\definecolor{currentfill}{rgb}{0.000000,0.000000,0.000000}%
\pgfsetfillcolor{currentfill}%
\pgfsetfillopacity{0.500000}%
\pgfsetlinewidth{0.000000pt}%
\definecolor{currentstroke}{rgb}{0.000000,0.000000,0.000000}%
\pgfsetstrokecolor{currentstroke}%
\pgfsetdash{}{0pt}%
\pgfpathmoveto{\pgfqpoint{4.896922in}{1.614683in}}%
\pgfpathcurveto{\pgfqpoint{4.902447in}{1.614683in}}{\pgfqpoint{4.907746in}{1.616879in}}{\pgfqpoint{4.911653in}{1.620785in}}%
\pgfpathcurveto{\pgfqpoint{4.915560in}{1.624692in}}{\pgfqpoint{4.917755in}{1.629992in}}{\pgfqpoint{4.917755in}{1.635517in}}%
\pgfpathcurveto{\pgfqpoint{4.917755in}{1.641042in}}{\pgfqpoint{4.915560in}{1.646341in}}{\pgfqpoint{4.911653in}{1.650248in}}%
\pgfpathcurveto{\pgfqpoint{4.907746in}{1.654155in}}{\pgfqpoint{4.902447in}{1.656350in}}{\pgfqpoint{4.896922in}{1.656350in}}%
\pgfpathcurveto{\pgfqpoint{4.891397in}{1.656350in}}{\pgfqpoint{4.886097in}{1.654155in}}{\pgfqpoint{4.882190in}{1.650248in}}%
\pgfpathcurveto{\pgfqpoint{4.878284in}{1.646341in}}{\pgfqpoint{4.876088in}{1.641042in}}{\pgfqpoint{4.876088in}{1.635517in}}%
\pgfpathcurveto{\pgfqpoint{4.876088in}{1.629992in}}{\pgfqpoint{4.878284in}{1.624692in}}{\pgfqpoint{4.882190in}{1.620785in}}%
\pgfpathcurveto{\pgfqpoint{4.886097in}{1.616879in}}{\pgfqpoint{4.891397in}{1.614683in}}{\pgfqpoint{4.896922in}{1.614683in}}%
\pgfpathclose%
\pgfusepath{fill}%
\end{pgfscope}%
\begin{pgfscope}%
\pgfsetrectcap%
\pgfsetmiterjoin%
\pgfsetlinewidth{0.803000pt}%
\definecolor{currentstroke}{rgb}{0.000000,0.000000,0.000000}%
\pgfsetstrokecolor{currentstroke}%
\pgfsetdash{}{0pt}%
\pgfpathmoveto{\pgfqpoint{4.320685in}{1.588185in}}%
\pgfpathlineto{\pgfqpoint{4.320685in}{2.343185in}}%
\pgfusepath{stroke}%
\end{pgfscope}%
\begin{pgfscope}%
\pgfsetrectcap%
\pgfsetmiterjoin%
\pgfsetlinewidth{0.803000pt}%
\definecolor{currentstroke}{rgb}{0.000000,0.000000,0.000000}%
\pgfsetstrokecolor{currentstroke}%
\pgfsetdash{}{0pt}%
\pgfpathmoveto{\pgfqpoint{5.483185in}{1.588185in}}%
\pgfpathlineto{\pgfqpoint{5.483185in}{2.343185in}}%
\pgfusepath{stroke}%
\end{pgfscope}%
\begin{pgfscope}%
\pgfsetrectcap%
\pgfsetmiterjoin%
\pgfsetlinewidth{0.803000pt}%
\definecolor{currentstroke}{rgb}{0.000000,0.000000,0.000000}%
\pgfsetstrokecolor{currentstroke}%
\pgfsetdash{}{0pt}%
\pgfpathmoveto{\pgfqpoint{4.320685in}{1.588185in}}%
\pgfpathlineto{\pgfqpoint{5.483185in}{1.588185in}}%
\pgfusepath{stroke}%
\end{pgfscope}%
\begin{pgfscope}%
\pgfsetrectcap%
\pgfsetmiterjoin%
\pgfsetlinewidth{0.803000pt}%
\definecolor{currentstroke}{rgb}{0.000000,0.000000,0.000000}%
\pgfsetstrokecolor{currentstroke}%
\pgfsetdash{}{0pt}%
\pgfpathmoveto{\pgfqpoint{4.320685in}{2.343185in}}%
\pgfpathlineto{\pgfqpoint{5.483185in}{2.343185in}}%
\pgfusepath{stroke}%
\end{pgfscope}%
\begin{pgfscope}%
\pgfsetbuttcap%
\pgfsetmiterjoin%
\definecolor{currentfill}{rgb}{1.000000,1.000000,1.000000}%
\pgfsetfillcolor{currentfill}%
\pgfsetlinewidth{0.000000pt}%
\definecolor{currentstroke}{rgb}{0.000000,0.000000,0.000000}%
\pgfsetstrokecolor{currentstroke}%
\pgfsetstrokeopacity{0.000000}%
\pgfsetdash{}{0pt}%
\pgfpathmoveto{\pgfqpoint{0.833185in}{0.833185in}}%
\pgfpathlineto{\pgfqpoint{1.995685in}{0.833185in}}%
\pgfpathlineto{\pgfqpoint{1.995685in}{1.588185in}}%
\pgfpathlineto{\pgfqpoint{0.833185in}{1.588185in}}%
\pgfpathclose%
\pgfusepath{fill}%
\end{pgfscope}%
\begin{pgfscope}%
\pgfpathrectangle{\pgfqpoint{0.833185in}{0.833185in}}{\pgfqpoint{1.162500in}{0.755000in}} %
\pgfusepath{clip}%
\pgfsetbuttcap%
\pgfsetroundjoin%
\definecolor{currentfill}{rgb}{0.000000,0.000000,0.000000}%
\pgfsetfillcolor{currentfill}%
\pgfsetfillopacity{0.500000}%
\pgfsetlinewidth{0.000000pt}%
\definecolor{currentstroke}{rgb}{0.000000,0.000000,0.000000}%
\pgfsetstrokecolor{currentstroke}%
\pgfsetdash{}{0pt}%
\pgfpathmoveto{\pgfqpoint{1.779018in}{1.099971in}}%
\pgfpathcurveto{\pgfqpoint{1.784543in}{1.099971in}}{\pgfqpoint{1.789842in}{1.102166in}}{\pgfqpoint{1.793749in}{1.106073in}}%
\pgfpathcurveto{\pgfqpoint{1.797656in}{1.109980in}}{\pgfqpoint{1.799851in}{1.115279in}}{\pgfqpoint{1.799851in}{1.120804in}}%
\pgfpathcurveto{\pgfqpoint{1.799851in}{1.126329in}}{\pgfqpoint{1.797656in}{1.131629in}}{\pgfqpoint{1.793749in}{1.135536in}}%
\pgfpathcurveto{\pgfqpoint{1.789842in}{1.139442in}}{\pgfqpoint{1.784543in}{1.141638in}}{\pgfqpoint{1.779018in}{1.141638in}}%
\pgfpathcurveto{\pgfqpoint{1.773492in}{1.141638in}}{\pgfqpoint{1.768193in}{1.139442in}}{\pgfqpoint{1.764286in}{1.135536in}}%
\pgfpathcurveto{\pgfqpoint{1.760379in}{1.131629in}}{\pgfqpoint{1.758184in}{1.126329in}}{\pgfqpoint{1.758184in}{1.120804in}}%
\pgfpathcurveto{\pgfqpoint{1.758184in}{1.115279in}}{\pgfqpoint{1.760379in}{1.109980in}}{\pgfqpoint{1.764286in}{1.106073in}}%
\pgfpathcurveto{\pgfqpoint{1.768193in}{1.102166in}}{\pgfqpoint{1.773492in}{1.099971in}}{\pgfqpoint{1.779018in}{1.099971in}}%
\pgfpathclose%
\pgfusepath{fill}%
\end{pgfscope}%
\begin{pgfscope}%
\pgfpathrectangle{\pgfqpoint{0.833185in}{0.833185in}}{\pgfqpoint{1.162500in}{0.755000in}} %
\pgfusepath{clip}%
\pgfsetbuttcap%
\pgfsetroundjoin%
\definecolor{currentfill}{rgb}{0.000000,0.000000,0.000000}%
\pgfsetfillcolor{currentfill}%
\pgfsetfillopacity{0.500000}%
\pgfsetlinewidth{0.000000pt}%
\definecolor{currentstroke}{rgb}{0.000000,0.000000,0.000000}%
\pgfsetstrokecolor{currentstroke}%
\pgfsetdash{}{0pt}%
\pgfpathmoveto{\pgfqpoint{0.920858in}{1.503836in}}%
\pgfpathcurveto{\pgfqpoint{0.926383in}{1.503836in}}{\pgfqpoint{0.931683in}{1.506031in}}{\pgfqpoint{0.935589in}{1.509938in}}%
\pgfpathcurveto{\pgfqpoint{0.939496in}{1.513845in}}{\pgfqpoint{0.941691in}{1.519144in}}{\pgfqpoint{0.941691in}{1.524669in}}%
\pgfpathcurveto{\pgfqpoint{0.941691in}{1.530195in}}{\pgfqpoint{0.939496in}{1.535494in}}{\pgfqpoint{0.935589in}{1.539401in}}%
\pgfpathcurveto{\pgfqpoint{0.931683in}{1.543308in}}{\pgfqpoint{0.926383in}{1.545503in}}{\pgfqpoint{0.920858in}{1.545503in}}%
\pgfpathcurveto{\pgfqpoint{0.915333in}{1.545503in}}{\pgfqpoint{0.910034in}{1.543308in}}{\pgfqpoint{0.906127in}{1.539401in}}%
\pgfpathcurveto{\pgfqpoint{0.902220in}{1.535494in}}{\pgfqpoint{0.900025in}{1.530195in}}{\pgfqpoint{0.900025in}{1.524669in}}%
\pgfpathcurveto{\pgfqpoint{0.900025in}{1.519144in}}{\pgfqpoint{0.902220in}{1.513845in}}{\pgfqpoint{0.906127in}{1.509938in}}%
\pgfpathcurveto{\pgfqpoint{0.910034in}{1.506031in}}{\pgfqpoint{0.915333in}{1.503836in}}{\pgfqpoint{0.920858in}{1.503836in}}%
\pgfpathclose%
\pgfusepath{fill}%
\end{pgfscope}%
\begin{pgfscope}%
\pgfpathrectangle{\pgfqpoint{0.833185in}{0.833185in}}{\pgfqpoint{1.162500in}{0.755000in}} %
\pgfusepath{clip}%
\pgfsetbuttcap%
\pgfsetroundjoin%
\definecolor{currentfill}{rgb}{0.000000,0.000000,0.000000}%
\pgfsetfillcolor{currentfill}%
\pgfsetfillopacity{0.500000}%
\pgfsetlinewidth{0.000000pt}%
\definecolor{currentstroke}{rgb}{0.000000,0.000000,0.000000}%
\pgfsetstrokecolor{currentstroke}%
\pgfsetdash{}{0pt}%
\pgfpathmoveto{\pgfqpoint{0.860863in}{1.549375in}}%
\pgfpathcurveto{\pgfqpoint{0.866388in}{1.549375in}}{\pgfqpoint{0.871688in}{1.551570in}}{\pgfqpoint{0.875595in}{1.555477in}}%
\pgfpathcurveto{\pgfqpoint{0.879501in}{1.559384in}}{\pgfqpoint{0.881696in}{1.564683in}}{\pgfqpoint{0.881696in}{1.570208in}}%
\pgfpathcurveto{\pgfqpoint{0.881696in}{1.575733in}}{\pgfqpoint{0.879501in}{1.581033in}}{\pgfqpoint{0.875595in}{1.584940in}}%
\pgfpathcurveto{\pgfqpoint{0.871688in}{1.588847in}}{\pgfqpoint{0.866388in}{1.591042in}}{\pgfqpoint{0.860863in}{1.591042in}}%
\pgfpathcurveto{\pgfqpoint{0.855338in}{1.591042in}}{\pgfqpoint{0.850039in}{1.588847in}}{\pgfqpoint{0.846132in}{1.584940in}}%
\pgfpathcurveto{\pgfqpoint{0.842225in}{1.581033in}}{\pgfqpoint{0.840030in}{1.575733in}}{\pgfqpoint{0.840030in}{1.570208in}}%
\pgfpathcurveto{\pgfqpoint{0.840030in}{1.564683in}}{\pgfqpoint{0.842225in}{1.559384in}}{\pgfqpoint{0.846132in}{1.555477in}}%
\pgfpathcurveto{\pgfqpoint{0.850039in}{1.551570in}}{\pgfqpoint{0.855338in}{1.549375in}}{\pgfqpoint{0.860863in}{1.549375in}}%
\pgfpathclose%
\pgfusepath{fill}%
\end{pgfscope}%
\begin{pgfscope}%
\pgfpathrectangle{\pgfqpoint{0.833185in}{0.833185in}}{\pgfqpoint{1.162500in}{0.755000in}} %
\pgfusepath{clip}%
\pgfsetbuttcap%
\pgfsetroundjoin%
\definecolor{currentfill}{rgb}{0.000000,0.000000,0.000000}%
\pgfsetfillcolor{currentfill}%
\pgfsetfillopacity{0.500000}%
\pgfsetlinewidth{0.000000pt}%
\definecolor{currentstroke}{rgb}{0.000000,0.000000,0.000000}%
\pgfsetstrokecolor{currentstroke}%
\pgfsetdash{}{0pt}%
\pgfpathmoveto{\pgfqpoint{1.180359in}{1.273758in}}%
\pgfpathcurveto{\pgfqpoint{1.185884in}{1.273758in}}{\pgfqpoint{1.191184in}{1.275953in}}{\pgfqpoint{1.195090in}{1.279859in}}%
\pgfpathcurveto{\pgfqpoint{1.198997in}{1.283766in}}{\pgfqpoint{1.201192in}{1.289066in}}{\pgfqpoint{1.201192in}{1.294591in}}%
\pgfpathcurveto{\pgfqpoint{1.201192in}{1.300116in}}{\pgfqpoint{1.198997in}{1.305415in}}{\pgfqpoint{1.195090in}{1.309322in}}%
\pgfpathcurveto{\pgfqpoint{1.191184in}{1.313229in}}{\pgfqpoint{1.185884in}{1.315424in}}{\pgfqpoint{1.180359in}{1.315424in}}%
\pgfpathcurveto{\pgfqpoint{1.174834in}{1.315424in}}{\pgfqpoint{1.169535in}{1.313229in}}{\pgfqpoint{1.165628in}{1.309322in}}%
\pgfpathcurveto{\pgfqpoint{1.161721in}{1.305415in}}{\pgfqpoint{1.159526in}{1.300116in}}{\pgfqpoint{1.159526in}{1.294591in}}%
\pgfpathcurveto{\pgfqpoint{1.159526in}{1.289066in}}{\pgfqpoint{1.161721in}{1.283766in}}{\pgfqpoint{1.165628in}{1.279859in}}%
\pgfpathcurveto{\pgfqpoint{1.169535in}{1.275953in}}{\pgfqpoint{1.174834in}{1.273758in}}{\pgfqpoint{1.180359in}{1.273758in}}%
\pgfpathclose%
\pgfusepath{fill}%
\end{pgfscope}%
\begin{pgfscope}%
\pgfpathrectangle{\pgfqpoint{0.833185in}{0.833185in}}{\pgfqpoint{1.162500in}{0.755000in}} %
\pgfusepath{clip}%
\pgfsetbuttcap%
\pgfsetroundjoin%
\definecolor{currentfill}{rgb}{0.000000,0.000000,0.000000}%
\pgfsetfillcolor{currentfill}%
\pgfsetfillopacity{0.500000}%
\pgfsetlinewidth{0.000000pt}%
\definecolor{currentstroke}{rgb}{0.000000,0.000000,0.000000}%
\pgfsetstrokecolor{currentstroke}%
\pgfsetdash{}{0pt}%
\pgfpathmoveto{\pgfqpoint{0.938297in}{1.319843in}}%
\pgfpathcurveto{\pgfqpoint{0.943822in}{1.319843in}}{\pgfqpoint{0.949121in}{1.322038in}}{\pgfqpoint{0.953028in}{1.325945in}}%
\pgfpathcurveto{\pgfqpoint{0.956935in}{1.329852in}}{\pgfqpoint{0.959130in}{1.335151in}}{\pgfqpoint{0.959130in}{1.340676in}}%
\pgfpathcurveto{\pgfqpoint{0.959130in}{1.346201in}}{\pgfqpoint{0.956935in}{1.351501in}}{\pgfqpoint{0.953028in}{1.355408in}}%
\pgfpathcurveto{\pgfqpoint{0.949121in}{1.359314in}}{\pgfqpoint{0.943822in}{1.361510in}}{\pgfqpoint{0.938297in}{1.361510in}}%
\pgfpathcurveto{\pgfqpoint{0.932772in}{1.361510in}}{\pgfqpoint{0.927472in}{1.359314in}}{\pgfqpoint{0.923565in}{1.355408in}}%
\pgfpathcurveto{\pgfqpoint{0.919659in}{1.351501in}}{\pgfqpoint{0.917464in}{1.346201in}}{\pgfqpoint{0.917464in}{1.340676in}}%
\pgfpathcurveto{\pgfqpoint{0.917464in}{1.335151in}}{\pgfqpoint{0.919659in}{1.329852in}}{\pgfqpoint{0.923565in}{1.325945in}}%
\pgfpathcurveto{\pgfqpoint{0.927472in}{1.322038in}}{\pgfqpoint{0.932772in}{1.319843in}}{\pgfqpoint{0.938297in}{1.319843in}}%
\pgfpathclose%
\pgfusepath{fill}%
\end{pgfscope}%
\begin{pgfscope}%
\pgfpathrectangle{\pgfqpoint{0.833185in}{0.833185in}}{\pgfqpoint{1.162500in}{0.755000in}} %
\pgfusepath{clip}%
\pgfsetbuttcap%
\pgfsetroundjoin%
\definecolor{currentfill}{rgb}{0.000000,0.000000,0.000000}%
\pgfsetfillcolor{currentfill}%
\pgfsetfillopacity{0.500000}%
\pgfsetlinewidth{0.000000pt}%
\definecolor{currentstroke}{rgb}{0.000000,0.000000,0.000000}%
\pgfsetstrokecolor{currentstroke}%
\pgfsetdash{}{0pt}%
\pgfpathmoveto{\pgfqpoint{1.098900in}{1.215122in}}%
\pgfpathcurveto{\pgfqpoint{1.104426in}{1.215122in}}{\pgfqpoint{1.109725in}{1.217317in}}{\pgfqpoint{1.113632in}{1.221224in}}%
\pgfpathcurveto{\pgfqpoint{1.117539in}{1.225130in}}{\pgfqpoint{1.119734in}{1.230430in}}{\pgfqpoint{1.119734in}{1.235955in}}%
\pgfpathcurveto{\pgfqpoint{1.119734in}{1.241480in}}{\pgfqpoint{1.117539in}{1.246780in}}{\pgfqpoint{1.113632in}{1.250686in}}%
\pgfpathcurveto{\pgfqpoint{1.109725in}{1.254593in}}{\pgfqpoint{1.104426in}{1.256788in}}{\pgfqpoint{1.098900in}{1.256788in}}%
\pgfpathcurveto{\pgfqpoint{1.093375in}{1.256788in}}{\pgfqpoint{1.088076in}{1.254593in}}{\pgfqpoint{1.084169in}{1.250686in}}%
\pgfpathcurveto{\pgfqpoint{1.080262in}{1.246780in}}{\pgfqpoint{1.078067in}{1.241480in}}{\pgfqpoint{1.078067in}{1.235955in}}%
\pgfpathcurveto{\pgfqpoint{1.078067in}{1.230430in}}{\pgfqpoint{1.080262in}{1.225130in}}{\pgfqpoint{1.084169in}{1.221224in}}%
\pgfpathcurveto{\pgfqpoint{1.088076in}{1.217317in}}{\pgfqpoint{1.093375in}{1.215122in}}{\pgfqpoint{1.098900in}{1.215122in}}%
\pgfpathclose%
\pgfusepath{fill}%
\end{pgfscope}%
\begin{pgfscope}%
\pgfpathrectangle{\pgfqpoint{0.833185in}{0.833185in}}{\pgfqpoint{1.162500in}{0.755000in}} %
\pgfusepath{clip}%
\pgfsetbuttcap%
\pgfsetroundjoin%
\definecolor{currentfill}{rgb}{0.000000,0.000000,0.000000}%
\pgfsetfillcolor{currentfill}%
\pgfsetfillopacity{0.500000}%
\pgfsetlinewidth{0.000000pt}%
\definecolor{currentstroke}{rgb}{0.000000,0.000000,0.000000}%
\pgfsetstrokecolor{currentstroke}%
\pgfsetdash{}{0pt}%
\pgfpathmoveto{\pgfqpoint{1.085127in}{0.830327in}}%
\pgfpathcurveto{\pgfqpoint{1.090652in}{0.830327in}}{\pgfqpoint{1.095951in}{0.832523in}}{\pgfqpoint{1.099858in}{0.836429in}}%
\pgfpathcurveto{\pgfqpoint{1.103765in}{0.840336in}}{\pgfqpoint{1.105960in}{0.845636in}}{\pgfqpoint{1.105960in}{0.851161in}}%
\pgfpathcurveto{\pgfqpoint{1.105960in}{0.856686in}}{\pgfqpoint{1.103765in}{0.861985in}}{\pgfqpoint{1.099858in}{0.865892in}}%
\pgfpathcurveto{\pgfqpoint{1.095951in}{0.869799in}}{\pgfqpoint{1.090652in}{0.871994in}}{\pgfqpoint{1.085127in}{0.871994in}}%
\pgfpathcurveto{\pgfqpoint{1.079602in}{0.871994in}}{\pgfqpoint{1.074302in}{0.869799in}}{\pgfqpoint{1.070395in}{0.865892in}}%
\pgfpathcurveto{\pgfqpoint{1.066488in}{0.861985in}}{\pgfqpoint{1.064293in}{0.856686in}}{\pgfqpoint{1.064293in}{0.851161in}}%
\pgfpathcurveto{\pgfqpoint{1.064293in}{0.845636in}}{\pgfqpoint{1.066488in}{0.840336in}}{\pgfqpoint{1.070395in}{0.836429in}}%
\pgfpathcurveto{\pgfqpoint{1.074302in}{0.832523in}}{\pgfqpoint{1.079602in}{0.830327in}}{\pgfqpoint{1.085127in}{0.830327in}}%
\pgfpathclose%
\pgfusepath{fill}%
\end{pgfscope}%
\begin{pgfscope}%
\pgfpathrectangle{\pgfqpoint{0.833185in}{0.833185in}}{\pgfqpoint{1.162500in}{0.755000in}} %
\pgfusepath{clip}%
\pgfsetbuttcap%
\pgfsetroundjoin%
\definecolor{currentfill}{rgb}{0.000000,0.000000,0.000000}%
\pgfsetfillcolor{currentfill}%
\pgfsetfillopacity{0.500000}%
\pgfsetlinewidth{0.000000pt}%
\definecolor{currentstroke}{rgb}{0.000000,0.000000,0.000000}%
\pgfsetstrokecolor{currentstroke}%
\pgfsetdash{}{0pt}%
\pgfpathmoveto{\pgfqpoint{1.896712in}{1.469697in}}%
\pgfpathcurveto{\pgfqpoint{1.902237in}{1.469697in}}{\pgfqpoint{1.907537in}{1.471892in}}{\pgfqpoint{1.911443in}{1.475799in}}%
\pgfpathcurveto{\pgfqpoint{1.915350in}{1.479706in}}{\pgfqpoint{1.917545in}{1.485005in}}{\pgfqpoint{1.917545in}{1.490530in}}%
\pgfpathcurveto{\pgfqpoint{1.917545in}{1.496056in}}{\pgfqpoint{1.915350in}{1.501355in}}{\pgfqpoint{1.911443in}{1.505262in}}%
\pgfpathcurveto{\pgfqpoint{1.907537in}{1.509169in}}{\pgfqpoint{1.902237in}{1.511364in}}{\pgfqpoint{1.896712in}{1.511364in}}%
\pgfpathcurveto{\pgfqpoint{1.891187in}{1.511364in}}{\pgfqpoint{1.885887in}{1.509169in}}{\pgfqpoint{1.881981in}{1.505262in}}%
\pgfpathcurveto{\pgfqpoint{1.878074in}{1.501355in}}{\pgfqpoint{1.875879in}{1.496056in}}{\pgfqpoint{1.875879in}{1.490530in}}%
\pgfpathcurveto{\pgfqpoint{1.875879in}{1.485005in}}{\pgfqpoint{1.878074in}{1.479706in}}{\pgfqpoint{1.881981in}{1.475799in}}%
\pgfpathcurveto{\pgfqpoint{1.885887in}{1.471892in}}{\pgfqpoint{1.891187in}{1.469697in}}{\pgfqpoint{1.896712in}{1.469697in}}%
\pgfpathclose%
\pgfusepath{fill}%
\end{pgfscope}%
\begin{pgfscope}%
\pgfpathrectangle{\pgfqpoint{0.833185in}{0.833185in}}{\pgfqpoint{1.162500in}{0.755000in}} %
\pgfusepath{clip}%
\pgfsetbuttcap%
\pgfsetroundjoin%
\definecolor{currentfill}{rgb}{0.000000,0.000000,0.000000}%
\pgfsetfillcolor{currentfill}%
\pgfsetfillopacity{0.500000}%
\pgfsetlinewidth{0.000000pt}%
\definecolor{currentstroke}{rgb}{0.000000,0.000000,0.000000}%
\pgfsetstrokecolor{currentstroke}%
\pgfsetdash{}{0pt}%
\pgfpathmoveto{\pgfqpoint{1.968006in}{1.312260in}}%
\pgfpathcurveto{\pgfqpoint{1.973531in}{1.312260in}}{\pgfqpoint{1.978831in}{1.314455in}}{\pgfqpoint{1.982737in}{1.318362in}}%
\pgfpathcurveto{\pgfqpoint{1.986644in}{1.322269in}}{\pgfqpoint{1.988839in}{1.327569in}}{\pgfqpoint{1.988839in}{1.333094in}}%
\pgfpathcurveto{\pgfqpoint{1.988839in}{1.338619in}}{\pgfqpoint{1.986644in}{1.343918in}}{\pgfqpoint{1.982737in}{1.347825in}}%
\pgfpathcurveto{\pgfqpoint{1.978831in}{1.351732in}}{\pgfqpoint{1.973531in}{1.353927in}}{\pgfqpoint{1.968006in}{1.353927in}}%
\pgfpathcurveto{\pgfqpoint{1.962481in}{1.353927in}}{\pgfqpoint{1.957181in}{1.351732in}}{\pgfqpoint{1.953275in}{1.347825in}}%
\pgfpathcurveto{\pgfqpoint{1.949368in}{1.343918in}}{\pgfqpoint{1.947173in}{1.338619in}}{\pgfqpoint{1.947173in}{1.333094in}}%
\pgfpathcurveto{\pgfqpoint{1.947173in}{1.327569in}}{\pgfqpoint{1.949368in}{1.322269in}}{\pgfqpoint{1.953275in}{1.318362in}}%
\pgfpathcurveto{\pgfqpoint{1.957181in}{1.314455in}}{\pgfqpoint{1.962481in}{1.312260in}}{\pgfqpoint{1.968006in}{1.312260in}}%
\pgfpathclose%
\pgfusepath{fill}%
\end{pgfscope}%
\begin{pgfscope}%
\pgfpathrectangle{\pgfqpoint{0.833185in}{0.833185in}}{\pgfqpoint{1.162500in}{0.755000in}} %
\pgfusepath{clip}%
\pgfsetbuttcap%
\pgfsetroundjoin%
\definecolor{currentfill}{rgb}{0.000000,0.000000,0.000000}%
\pgfsetfillcolor{currentfill}%
\pgfsetfillopacity{0.500000}%
\pgfsetlinewidth{0.000000pt}%
\definecolor{currentstroke}{rgb}{0.000000,0.000000,0.000000}%
\pgfsetstrokecolor{currentstroke}%
\pgfsetdash{}{0pt}%
\pgfpathmoveto{\pgfqpoint{1.742203in}{1.134966in}}%
\pgfpathcurveto{\pgfqpoint{1.747728in}{1.134966in}}{\pgfqpoint{1.753027in}{1.137161in}}{\pgfqpoint{1.756934in}{1.141068in}}%
\pgfpathcurveto{\pgfqpoint{1.760841in}{1.144975in}}{\pgfqpoint{1.763036in}{1.150275in}}{\pgfqpoint{1.763036in}{1.155800in}}%
\pgfpathcurveto{\pgfqpoint{1.763036in}{1.161325in}}{\pgfqpoint{1.760841in}{1.166624in}}{\pgfqpoint{1.756934in}{1.170531in}}%
\pgfpathcurveto{\pgfqpoint{1.753027in}{1.174438in}}{\pgfqpoint{1.747728in}{1.176633in}}{\pgfqpoint{1.742203in}{1.176633in}}%
\pgfpathcurveto{\pgfqpoint{1.736677in}{1.176633in}}{\pgfqpoint{1.731378in}{1.174438in}}{\pgfqpoint{1.727471in}{1.170531in}}%
\pgfpathcurveto{\pgfqpoint{1.723564in}{1.166624in}}{\pgfqpoint{1.721369in}{1.161325in}}{\pgfqpoint{1.721369in}{1.155800in}}%
\pgfpathcurveto{\pgfqpoint{1.721369in}{1.150275in}}{\pgfqpoint{1.723564in}{1.144975in}}{\pgfqpoint{1.727471in}{1.141068in}}%
\pgfpathcurveto{\pgfqpoint{1.731378in}{1.137161in}}{\pgfqpoint{1.736677in}{1.134966in}}{\pgfqpoint{1.742203in}{1.134966in}}%
\pgfpathclose%
\pgfusepath{fill}%
\end{pgfscope}%
\begin{pgfscope}%
\pgfpathrectangle{\pgfqpoint{0.833185in}{0.833185in}}{\pgfqpoint{1.162500in}{0.755000in}} %
\pgfusepath{clip}%
\pgfsetbuttcap%
\pgfsetroundjoin%
\definecolor{currentfill}{rgb}{0.000000,0.000000,0.000000}%
\pgfsetfillcolor{currentfill}%
\pgfsetfillopacity{0.500000}%
\pgfsetlinewidth{0.000000pt}%
\definecolor{currentstroke}{rgb}{0.000000,0.000000,0.000000}%
\pgfsetstrokecolor{currentstroke}%
\pgfsetdash{}{0pt}%
\pgfpathmoveto{\pgfqpoint{1.058530in}{1.421242in}}%
\pgfpathcurveto{\pgfqpoint{1.064055in}{1.421242in}}{\pgfqpoint{1.069355in}{1.423437in}}{\pgfqpoint{1.073261in}{1.427344in}}%
\pgfpathcurveto{\pgfqpoint{1.077168in}{1.431251in}}{\pgfqpoint{1.079363in}{1.436550in}}{\pgfqpoint{1.079363in}{1.442075in}}%
\pgfpathcurveto{\pgfqpoint{1.079363in}{1.447600in}}{\pgfqpoint{1.077168in}{1.452900in}}{\pgfqpoint{1.073261in}{1.456807in}}%
\pgfpathcurveto{\pgfqpoint{1.069355in}{1.460714in}}{\pgfqpoint{1.064055in}{1.462909in}}{\pgfqpoint{1.058530in}{1.462909in}}%
\pgfpathcurveto{\pgfqpoint{1.053005in}{1.462909in}}{\pgfqpoint{1.047706in}{1.460714in}}{\pgfqpoint{1.043799in}{1.456807in}}%
\pgfpathcurveto{\pgfqpoint{1.039892in}{1.452900in}}{\pgfqpoint{1.037697in}{1.447600in}}{\pgfqpoint{1.037697in}{1.442075in}}%
\pgfpathcurveto{\pgfqpoint{1.037697in}{1.436550in}}{\pgfqpoint{1.039892in}{1.431251in}}{\pgfqpoint{1.043799in}{1.427344in}}%
\pgfpathcurveto{\pgfqpoint{1.047706in}{1.423437in}}{\pgfqpoint{1.053005in}{1.421242in}}{\pgfqpoint{1.058530in}{1.421242in}}%
\pgfpathclose%
\pgfusepath{fill}%
\end{pgfscope}%
\begin{pgfscope}%
\pgfpathrectangle{\pgfqpoint{0.833185in}{0.833185in}}{\pgfqpoint{1.162500in}{0.755000in}} %
\pgfusepath{clip}%
\pgfsetbuttcap%
\pgfsetroundjoin%
\definecolor{currentfill}{rgb}{0.000000,0.000000,0.000000}%
\pgfsetfillcolor{currentfill}%
\pgfsetfillopacity{0.500000}%
\pgfsetlinewidth{0.000000pt}%
\definecolor{currentstroke}{rgb}{0.000000,0.000000,0.000000}%
\pgfsetstrokecolor{currentstroke}%
\pgfsetdash{}{0pt}%
\pgfpathmoveto{\pgfqpoint{1.520990in}{1.126134in}}%
\pgfpathcurveto{\pgfqpoint{1.526515in}{1.126134in}}{\pgfqpoint{1.531814in}{1.128329in}}{\pgfqpoint{1.535721in}{1.132236in}}%
\pgfpathcurveto{\pgfqpoint{1.539628in}{1.136143in}}{\pgfqpoint{1.541823in}{1.141442in}}{\pgfqpoint{1.541823in}{1.146967in}}%
\pgfpathcurveto{\pgfqpoint{1.541823in}{1.152492in}}{\pgfqpoint{1.539628in}{1.157792in}}{\pgfqpoint{1.535721in}{1.161699in}}%
\pgfpathcurveto{\pgfqpoint{1.531814in}{1.165606in}}{\pgfqpoint{1.526515in}{1.167801in}}{\pgfqpoint{1.520990in}{1.167801in}}%
\pgfpathcurveto{\pgfqpoint{1.515465in}{1.167801in}}{\pgfqpoint{1.510165in}{1.165606in}}{\pgfqpoint{1.506258in}{1.161699in}}%
\pgfpathcurveto{\pgfqpoint{1.502352in}{1.157792in}}{\pgfqpoint{1.500156in}{1.152492in}}{\pgfqpoint{1.500156in}{1.146967in}}%
\pgfpathcurveto{\pgfqpoint{1.500156in}{1.141442in}}{\pgfqpoint{1.502352in}{1.136143in}}{\pgfqpoint{1.506258in}{1.132236in}}%
\pgfpathcurveto{\pgfqpoint{1.510165in}{1.128329in}}{\pgfqpoint{1.515465in}{1.126134in}}{\pgfqpoint{1.520990in}{1.126134in}}%
\pgfpathclose%
\pgfusepath{fill}%
\end{pgfscope}%
\begin{pgfscope}%
\pgfpathrectangle{\pgfqpoint{0.833185in}{0.833185in}}{\pgfqpoint{1.162500in}{0.755000in}} %
\pgfusepath{clip}%
\pgfsetbuttcap%
\pgfsetroundjoin%
\definecolor{currentfill}{rgb}{0.000000,0.000000,0.000000}%
\pgfsetfillcolor{currentfill}%
\pgfsetfillopacity{0.500000}%
\pgfsetlinewidth{0.000000pt}%
\definecolor{currentstroke}{rgb}{0.000000,0.000000,0.000000}%
\pgfsetstrokecolor{currentstroke}%
\pgfsetdash{}{0pt}%
\pgfpathmoveto{\pgfqpoint{1.402203in}{0.888020in}}%
\pgfpathcurveto{\pgfqpoint{1.407728in}{0.888020in}}{\pgfqpoint{1.413027in}{0.890215in}}{\pgfqpoint{1.416934in}{0.894122in}}%
\pgfpathcurveto{\pgfqpoint{1.420841in}{0.898028in}}{\pgfqpoint{1.423036in}{0.903328in}}{\pgfqpoint{1.423036in}{0.908853in}}%
\pgfpathcurveto{\pgfqpoint{1.423036in}{0.914378in}}{\pgfqpoint{1.420841in}{0.919677in}}{\pgfqpoint{1.416934in}{0.923584in}}%
\pgfpathcurveto{\pgfqpoint{1.413027in}{0.927491in}}{\pgfqpoint{1.407728in}{0.929686in}}{\pgfqpoint{1.402203in}{0.929686in}}%
\pgfpathcurveto{\pgfqpoint{1.396677in}{0.929686in}}{\pgfqpoint{1.391378in}{0.927491in}}{\pgfqpoint{1.387471in}{0.923584in}}%
\pgfpathcurveto{\pgfqpoint{1.383564in}{0.919677in}}{\pgfqpoint{1.381369in}{0.914378in}}{\pgfqpoint{1.381369in}{0.908853in}}%
\pgfpathcurveto{\pgfqpoint{1.381369in}{0.903328in}}{\pgfqpoint{1.383564in}{0.898028in}}{\pgfqpoint{1.387471in}{0.894122in}}%
\pgfpathcurveto{\pgfqpoint{1.391378in}{0.890215in}}{\pgfqpoint{1.396677in}{0.888020in}}{\pgfqpoint{1.402203in}{0.888020in}}%
\pgfpathclose%
\pgfusepath{fill}%
\end{pgfscope}%
\begin{pgfscope}%
\pgfpathrectangle{\pgfqpoint{0.833185in}{0.833185in}}{\pgfqpoint{1.162500in}{0.755000in}} %
\pgfusepath{clip}%
\pgfsetbuttcap%
\pgfsetroundjoin%
\definecolor{currentfill}{rgb}{0.000000,0.000000,0.000000}%
\pgfsetfillcolor{currentfill}%
\pgfsetfillopacity{0.500000}%
\pgfsetlinewidth{0.000000pt}%
\definecolor{currentstroke}{rgb}{0.000000,0.000000,0.000000}%
\pgfsetstrokecolor{currentstroke}%
\pgfsetdash{}{0pt}%
\pgfpathmoveto{\pgfqpoint{1.053510in}{1.424328in}}%
\pgfpathcurveto{\pgfqpoint{1.059035in}{1.424328in}}{\pgfqpoint{1.064335in}{1.426523in}}{\pgfqpoint{1.068242in}{1.430430in}}%
\pgfpathcurveto{\pgfqpoint{1.072148in}{1.434337in}}{\pgfqpoint{1.074344in}{1.439636in}}{\pgfqpoint{1.074344in}{1.445161in}}%
\pgfpathcurveto{\pgfqpoint{1.074344in}{1.450687in}}{\pgfqpoint{1.072148in}{1.455986in}}{\pgfqpoint{1.068242in}{1.459893in}}%
\pgfpathcurveto{\pgfqpoint{1.064335in}{1.463800in}}{\pgfqpoint{1.059035in}{1.465995in}}{\pgfqpoint{1.053510in}{1.465995in}}%
\pgfpathcurveto{\pgfqpoint{1.047985in}{1.465995in}}{\pgfqpoint{1.042686in}{1.463800in}}{\pgfqpoint{1.038779in}{1.459893in}}%
\pgfpathcurveto{\pgfqpoint{1.034872in}{1.455986in}}{\pgfqpoint{1.032677in}{1.450687in}}{\pgfqpoint{1.032677in}{1.445161in}}%
\pgfpathcurveto{\pgfqpoint{1.032677in}{1.439636in}}{\pgfqpoint{1.034872in}{1.434337in}}{\pgfqpoint{1.038779in}{1.430430in}}%
\pgfpathcurveto{\pgfqpoint{1.042686in}{1.426523in}}{\pgfqpoint{1.047985in}{1.424328in}}{\pgfqpoint{1.053510in}{1.424328in}}%
\pgfpathclose%
\pgfusepath{fill}%
\end{pgfscope}%
\begin{pgfscope}%
\pgfpathrectangle{\pgfqpoint{0.833185in}{0.833185in}}{\pgfqpoint{1.162500in}{0.755000in}} %
\pgfusepath{clip}%
\pgfsetbuttcap%
\pgfsetroundjoin%
\definecolor{currentfill}{rgb}{0.000000,0.000000,0.000000}%
\pgfsetfillcolor{currentfill}%
\pgfsetfillopacity{0.500000}%
\pgfsetlinewidth{0.000000pt}%
\definecolor{currentstroke}{rgb}{0.000000,0.000000,0.000000}%
\pgfsetstrokecolor{currentstroke}%
\pgfsetdash{}{0pt}%
\pgfpathmoveto{\pgfqpoint{1.492358in}{1.186596in}}%
\pgfpathcurveto{\pgfqpoint{1.497883in}{1.186596in}}{\pgfqpoint{1.503182in}{1.188791in}}{\pgfqpoint{1.507089in}{1.192698in}}%
\pgfpathcurveto{\pgfqpoint{1.510996in}{1.196604in}}{\pgfqpoint{1.513191in}{1.201904in}}{\pgfqpoint{1.513191in}{1.207429in}}%
\pgfpathcurveto{\pgfqpoint{1.513191in}{1.212954in}}{\pgfqpoint{1.510996in}{1.218254in}}{\pgfqpoint{1.507089in}{1.222160in}}%
\pgfpathcurveto{\pgfqpoint{1.503182in}{1.226067in}}{\pgfqpoint{1.497883in}{1.228262in}}{\pgfqpoint{1.492358in}{1.228262in}}%
\pgfpathcurveto{\pgfqpoint{1.486832in}{1.228262in}}{\pgfqpoint{1.481533in}{1.226067in}}{\pgfqpoint{1.477626in}{1.222160in}}%
\pgfpathcurveto{\pgfqpoint{1.473719in}{1.218254in}}{\pgfqpoint{1.471524in}{1.212954in}}{\pgfqpoint{1.471524in}{1.207429in}}%
\pgfpathcurveto{\pgfqpoint{1.471524in}{1.201904in}}{\pgfqpoint{1.473719in}{1.196604in}}{\pgfqpoint{1.477626in}{1.192698in}}%
\pgfpathcurveto{\pgfqpoint{1.481533in}{1.188791in}}{\pgfqpoint{1.486832in}{1.186596in}}{\pgfqpoint{1.492358in}{1.186596in}}%
\pgfpathclose%
\pgfusepath{fill}%
\end{pgfscope}%
\begin{pgfscope}%
\pgfsetbuttcap%
\pgfsetroundjoin%
\definecolor{currentfill}{rgb}{0.000000,0.000000,0.000000}%
\pgfsetfillcolor{currentfill}%
\pgfsetlinewidth{0.803000pt}%
\definecolor{currentstroke}{rgb}{0.000000,0.000000,0.000000}%
\pgfsetstrokecolor{currentstroke}%
\pgfsetdash{}{0pt}%
\pgfsys@defobject{currentmarker}{\pgfqpoint{0.000000in}{-0.048611in}}{\pgfqpoint{0.000000in}{0.000000in}}{%
\pgfpathmoveto{\pgfqpoint{0.000000in}{0.000000in}}%
\pgfpathlineto{\pgfqpoint{0.000000in}{-0.048611in}}%
\pgfusepath{stroke,fill}%
}%
\begin{pgfscope}%
\pgfsys@transformshift{0.935719in}{0.833185in}%
\pgfsys@useobject{currentmarker}{}%
\end{pgfscope}%
\end{pgfscope}%
\begin{pgfscope}%
\pgftext[x=0.966372in,y=0.585111in,left,base,rotate=90.000000]{\rmfamily\fontsize{8.000000}{9.600000}\selectfont \(\displaystyle 0.5\)}%
\end{pgfscope}%
\begin{pgfscope}%
\pgfsetbuttcap%
\pgfsetroundjoin%
\definecolor{currentfill}{rgb}{0.000000,0.000000,0.000000}%
\pgfsetfillcolor{currentfill}%
\pgfsetlinewidth{0.803000pt}%
\definecolor{currentstroke}{rgb}{0.000000,0.000000,0.000000}%
\pgfsetstrokecolor{currentstroke}%
\pgfsetdash{}{0pt}%
\pgfsys@defobject{currentmarker}{\pgfqpoint{0.000000in}{-0.048611in}}{\pgfqpoint{0.000000in}{0.000000in}}{%
\pgfpathmoveto{\pgfqpoint{0.000000in}{0.000000in}}%
\pgfpathlineto{\pgfqpoint{0.000000in}{-0.048611in}}%
\pgfusepath{stroke,fill}%
}%
\begin{pgfscope}%
\pgfsys@transformshift{1.360696in}{0.833185in}%
\pgfsys@useobject{currentmarker}{}%
\end{pgfscope}%
\end{pgfscope}%
\begin{pgfscope}%
\pgftext[x=1.391350in,y=0.585111in,left,base,rotate=90.000000]{\rmfamily\fontsize{8.000000}{9.600000}\selectfont \(\displaystyle 1.0\)}%
\end{pgfscope}%
\begin{pgfscope}%
\pgfsetbuttcap%
\pgfsetroundjoin%
\definecolor{currentfill}{rgb}{0.000000,0.000000,0.000000}%
\pgfsetfillcolor{currentfill}%
\pgfsetlinewidth{0.803000pt}%
\definecolor{currentstroke}{rgb}{0.000000,0.000000,0.000000}%
\pgfsetstrokecolor{currentstroke}%
\pgfsetdash{}{0pt}%
\pgfsys@defobject{currentmarker}{\pgfqpoint{0.000000in}{-0.048611in}}{\pgfqpoint{0.000000in}{0.000000in}}{%
\pgfpathmoveto{\pgfqpoint{0.000000in}{0.000000in}}%
\pgfpathlineto{\pgfqpoint{0.000000in}{-0.048611in}}%
\pgfusepath{stroke,fill}%
}%
\begin{pgfscope}%
\pgfsys@transformshift{1.785674in}{0.833185in}%
\pgfsys@useobject{currentmarker}{}%
\end{pgfscope}%
\end{pgfscope}%
\begin{pgfscope}%
\pgftext[x=1.816328in,y=0.585111in,left,base,rotate=90.000000]{\rmfamily\fontsize{8.000000}{9.600000}\selectfont \(\displaystyle 1.5\)}%
\end{pgfscope}%
\begin{pgfscope}%
\pgftext[x=1.414435in,y=0.529556in,,top]{\rmfamily\fontsize{10.000000}{12.000000}\selectfont Ef0}%
\end{pgfscope}%
\begin{pgfscope}%
\pgfsetbuttcap%
\pgfsetroundjoin%
\definecolor{currentfill}{rgb}{0.000000,0.000000,0.000000}%
\pgfsetfillcolor{currentfill}%
\pgfsetlinewidth{0.803000pt}%
\definecolor{currentstroke}{rgb}{0.000000,0.000000,0.000000}%
\pgfsetstrokecolor{currentstroke}%
\pgfsetdash{}{0pt}%
\pgfsys@defobject{currentmarker}{\pgfqpoint{-0.048611in}{0.000000in}}{\pgfqpoint{0.000000in}{0.000000in}}{%
\pgfpathmoveto{\pgfqpoint{0.000000in}{0.000000in}}%
\pgfpathlineto{\pgfqpoint{-0.048611in}{0.000000in}}%
\pgfusepath{stroke,fill}%
}%
\begin{pgfscope}%
\pgfsys@transformshift{0.833185in}{0.915277in}%
\pgfsys@useobject{currentmarker}{}%
\end{pgfscope}%
\end{pgfscope}%
\begin{pgfscope}%
\pgftext[x=0.467054in,y=0.873068in,left,base]{\rmfamily\fontsize{8.000000}{9.600000}\selectfont \(\displaystyle 0.025\)}%
\end{pgfscope}%
\begin{pgfscope}%
\pgfsetbuttcap%
\pgfsetroundjoin%
\definecolor{currentfill}{rgb}{0.000000,0.000000,0.000000}%
\pgfsetfillcolor{currentfill}%
\pgfsetlinewidth{0.803000pt}%
\definecolor{currentstroke}{rgb}{0.000000,0.000000,0.000000}%
\pgfsetstrokecolor{currentstroke}%
\pgfsetdash{}{0pt}%
\pgfsys@defobject{currentmarker}{\pgfqpoint{-0.048611in}{0.000000in}}{\pgfqpoint{0.000000in}{0.000000in}}{%
\pgfpathmoveto{\pgfqpoint{0.000000in}{0.000000in}}%
\pgfpathlineto{\pgfqpoint{-0.048611in}{0.000000in}}%
\pgfusepath{stroke,fill}%
}%
\begin{pgfscope}%
\pgfsys@transformshift{0.833185in}{1.179603in}%
\pgfsys@useobject{currentmarker}{}%
\end{pgfscope}%
\end{pgfscope}%
\begin{pgfscope}%
\pgftext[x=0.467054in,y=1.137394in,left,base]{\rmfamily\fontsize{8.000000}{9.600000}\selectfont \(\displaystyle 0.050\)}%
\end{pgfscope}%
\begin{pgfscope}%
\pgfsetbuttcap%
\pgfsetroundjoin%
\definecolor{currentfill}{rgb}{0.000000,0.000000,0.000000}%
\pgfsetfillcolor{currentfill}%
\pgfsetlinewidth{0.803000pt}%
\definecolor{currentstroke}{rgb}{0.000000,0.000000,0.000000}%
\pgfsetstrokecolor{currentstroke}%
\pgfsetdash{}{0pt}%
\pgfsys@defobject{currentmarker}{\pgfqpoint{-0.048611in}{0.000000in}}{\pgfqpoint{0.000000in}{0.000000in}}{%
\pgfpathmoveto{\pgfqpoint{0.000000in}{0.000000in}}%
\pgfpathlineto{\pgfqpoint{-0.048611in}{0.000000in}}%
\pgfusepath{stroke,fill}%
}%
\begin{pgfscope}%
\pgfsys@transformshift{0.833185in}{1.443929in}%
\pgfsys@useobject{currentmarker}{}%
\end{pgfscope}%
\end{pgfscope}%
\begin{pgfscope}%
\pgftext[x=0.467054in,y=1.401719in,left,base]{\rmfamily\fontsize{8.000000}{9.600000}\selectfont \(\displaystyle 0.075\)}%
\end{pgfscope}%
\begin{pgfscope}%
\pgftext[x=0.411499in,y=1.210685in,,bottom,rotate=90.000000]{\rmfamily\fontsize{10.000000}{12.000000}\selectfont u0}%
\end{pgfscope}%
\begin{pgfscope}%
\pgfsetrectcap%
\pgfsetmiterjoin%
\pgfsetlinewidth{0.803000pt}%
\definecolor{currentstroke}{rgb}{0.000000,0.000000,0.000000}%
\pgfsetstrokecolor{currentstroke}%
\pgfsetdash{}{0pt}%
\pgfpathmoveto{\pgfqpoint{0.833185in}{0.833185in}}%
\pgfpathlineto{\pgfqpoint{0.833185in}{1.588185in}}%
\pgfusepath{stroke}%
\end{pgfscope}%
\begin{pgfscope}%
\pgfsetrectcap%
\pgfsetmiterjoin%
\pgfsetlinewidth{0.803000pt}%
\definecolor{currentstroke}{rgb}{0.000000,0.000000,0.000000}%
\pgfsetstrokecolor{currentstroke}%
\pgfsetdash{}{0pt}%
\pgfpathmoveto{\pgfqpoint{1.995685in}{0.833185in}}%
\pgfpathlineto{\pgfqpoint{1.995685in}{1.588185in}}%
\pgfusepath{stroke}%
\end{pgfscope}%
\begin{pgfscope}%
\pgfsetrectcap%
\pgfsetmiterjoin%
\pgfsetlinewidth{0.803000pt}%
\definecolor{currentstroke}{rgb}{0.000000,0.000000,0.000000}%
\pgfsetstrokecolor{currentstroke}%
\pgfsetdash{}{0pt}%
\pgfpathmoveto{\pgfqpoint{0.833185in}{0.833185in}}%
\pgfpathlineto{\pgfqpoint{1.995685in}{0.833185in}}%
\pgfusepath{stroke}%
\end{pgfscope}%
\begin{pgfscope}%
\pgfsetrectcap%
\pgfsetmiterjoin%
\pgfsetlinewidth{0.803000pt}%
\definecolor{currentstroke}{rgb}{0.000000,0.000000,0.000000}%
\pgfsetstrokecolor{currentstroke}%
\pgfsetdash{}{0pt}%
\pgfpathmoveto{\pgfqpoint{0.833185in}{1.588185in}}%
\pgfpathlineto{\pgfqpoint{1.995685in}{1.588185in}}%
\pgfusepath{stroke}%
\end{pgfscope}%
\begin{pgfscope}%
\pgfsetbuttcap%
\pgfsetmiterjoin%
\definecolor{currentfill}{rgb}{1.000000,1.000000,1.000000}%
\pgfsetfillcolor{currentfill}%
\pgfsetlinewidth{0.000000pt}%
\definecolor{currentstroke}{rgb}{0.000000,0.000000,0.000000}%
\pgfsetstrokecolor{currentstroke}%
\pgfsetstrokeopacity{0.000000}%
\pgfsetdash{}{0pt}%
\pgfpathmoveto{\pgfqpoint{1.995685in}{0.833185in}}%
\pgfpathlineto{\pgfqpoint{3.158185in}{0.833185in}}%
\pgfpathlineto{\pgfqpoint{3.158185in}{1.588185in}}%
\pgfpathlineto{\pgfqpoint{1.995685in}{1.588185in}}%
\pgfpathclose%
\pgfusepath{fill}%
\end{pgfscope}%
\begin{pgfscope}%
\pgfpathrectangle{\pgfqpoint{1.995685in}{0.833185in}}{\pgfqpoint{1.162500in}{0.755000in}} %
\pgfusepath{clip}%
\pgfsetbuttcap%
\pgfsetroundjoin%
\definecolor{currentfill}{rgb}{0.000000,0.000000,0.000000}%
\pgfsetfillcolor{currentfill}%
\pgfsetfillopacity{0.500000}%
\pgfsetlinewidth{0.000000pt}%
\definecolor{currentstroke}{rgb}{0.000000,0.000000,0.000000}%
\pgfsetstrokecolor{currentstroke}%
\pgfsetdash{}{0pt}%
\pgfpathmoveto{\pgfqpoint{3.130506in}{1.099971in}}%
\pgfpathcurveto{\pgfqpoint{3.136031in}{1.099971in}}{\pgfqpoint{3.141331in}{1.102166in}}{\pgfqpoint{3.145237in}{1.106073in}}%
\pgfpathcurveto{\pgfqpoint{3.149144in}{1.109980in}}{\pgfqpoint{3.151339in}{1.115279in}}{\pgfqpoint{3.151339in}{1.120804in}}%
\pgfpathcurveto{\pgfqpoint{3.151339in}{1.126329in}}{\pgfqpoint{3.149144in}{1.131629in}}{\pgfqpoint{3.145237in}{1.135536in}}%
\pgfpathcurveto{\pgfqpoint{3.141331in}{1.139442in}}{\pgfqpoint{3.136031in}{1.141638in}}{\pgfqpoint{3.130506in}{1.141638in}}%
\pgfpathcurveto{\pgfqpoint{3.124981in}{1.141638in}}{\pgfqpoint{3.119681in}{1.139442in}}{\pgfqpoint{3.115775in}{1.135536in}}%
\pgfpathcurveto{\pgfqpoint{3.111868in}{1.131629in}}{\pgfqpoint{3.109673in}{1.126329in}}{\pgfqpoint{3.109673in}{1.120804in}}%
\pgfpathcurveto{\pgfqpoint{3.109673in}{1.115279in}}{\pgfqpoint{3.111868in}{1.109980in}}{\pgfqpoint{3.115775in}{1.106073in}}%
\pgfpathcurveto{\pgfqpoint{3.119681in}{1.102166in}}{\pgfqpoint{3.124981in}{1.099971in}}{\pgfqpoint{3.130506in}{1.099971in}}%
\pgfpathclose%
\pgfusepath{fill}%
\end{pgfscope}%
\begin{pgfscope}%
\pgfpathrectangle{\pgfqpoint{1.995685in}{0.833185in}}{\pgfqpoint{1.162500in}{0.755000in}} %
\pgfusepath{clip}%
\pgfsetbuttcap%
\pgfsetroundjoin%
\definecolor{currentfill}{rgb}{0.000000,0.000000,0.000000}%
\pgfsetfillcolor{currentfill}%
\pgfsetfillopacity{0.500000}%
\pgfsetlinewidth{0.000000pt}%
\definecolor{currentstroke}{rgb}{0.000000,0.000000,0.000000}%
\pgfsetstrokecolor{currentstroke}%
\pgfsetdash{}{0pt}%
\pgfpathmoveto{\pgfqpoint{2.755547in}{1.503836in}}%
\pgfpathcurveto{\pgfqpoint{2.761072in}{1.503836in}}{\pgfqpoint{2.766372in}{1.506031in}}{\pgfqpoint{2.770279in}{1.509938in}}%
\pgfpathcurveto{\pgfqpoint{2.774186in}{1.513845in}}{\pgfqpoint{2.776381in}{1.519144in}}{\pgfqpoint{2.776381in}{1.524669in}}%
\pgfpathcurveto{\pgfqpoint{2.776381in}{1.530195in}}{\pgfqpoint{2.774186in}{1.535494in}}{\pgfqpoint{2.770279in}{1.539401in}}%
\pgfpathcurveto{\pgfqpoint{2.766372in}{1.543308in}}{\pgfqpoint{2.761072in}{1.545503in}}{\pgfqpoint{2.755547in}{1.545503in}}%
\pgfpathcurveto{\pgfqpoint{2.750022in}{1.545503in}}{\pgfqpoint{2.744723in}{1.543308in}}{\pgfqpoint{2.740816in}{1.539401in}}%
\pgfpathcurveto{\pgfqpoint{2.736909in}{1.535494in}}{\pgfqpoint{2.734714in}{1.530195in}}{\pgfqpoint{2.734714in}{1.524669in}}%
\pgfpathcurveto{\pgfqpoint{2.734714in}{1.519144in}}{\pgfqpoint{2.736909in}{1.513845in}}{\pgfqpoint{2.740816in}{1.509938in}}%
\pgfpathcurveto{\pgfqpoint{2.744723in}{1.506031in}}{\pgfqpoint{2.750022in}{1.503836in}}{\pgfqpoint{2.755547in}{1.503836in}}%
\pgfpathclose%
\pgfusepath{fill}%
\end{pgfscope}%
\begin{pgfscope}%
\pgfpathrectangle{\pgfqpoint{1.995685in}{0.833185in}}{\pgfqpoint{1.162500in}{0.755000in}} %
\pgfusepath{clip}%
\pgfsetbuttcap%
\pgfsetroundjoin%
\definecolor{currentfill}{rgb}{0.000000,0.000000,0.000000}%
\pgfsetfillcolor{currentfill}%
\pgfsetfillopacity{0.500000}%
\pgfsetlinewidth{0.000000pt}%
\definecolor{currentstroke}{rgb}{0.000000,0.000000,0.000000}%
\pgfsetstrokecolor{currentstroke}%
\pgfsetdash{}{0pt}%
\pgfpathmoveto{\pgfqpoint{2.755127in}{1.549375in}}%
\pgfpathcurveto{\pgfqpoint{2.760652in}{1.549375in}}{\pgfqpoint{2.765952in}{1.551570in}}{\pgfqpoint{2.769858in}{1.555477in}}%
\pgfpathcurveto{\pgfqpoint{2.773765in}{1.559384in}}{\pgfqpoint{2.775960in}{1.564683in}}{\pgfqpoint{2.775960in}{1.570208in}}%
\pgfpathcurveto{\pgfqpoint{2.775960in}{1.575733in}}{\pgfqpoint{2.773765in}{1.581033in}}{\pgfqpoint{2.769858in}{1.584940in}}%
\pgfpathcurveto{\pgfqpoint{2.765952in}{1.588847in}}{\pgfqpoint{2.760652in}{1.591042in}}{\pgfqpoint{2.755127in}{1.591042in}}%
\pgfpathcurveto{\pgfqpoint{2.749602in}{1.591042in}}{\pgfqpoint{2.744302in}{1.588847in}}{\pgfqpoint{2.740396in}{1.584940in}}%
\pgfpathcurveto{\pgfqpoint{2.736489in}{1.581033in}}{\pgfqpoint{2.734294in}{1.575733in}}{\pgfqpoint{2.734294in}{1.570208in}}%
\pgfpathcurveto{\pgfqpoint{2.734294in}{1.564683in}}{\pgfqpoint{2.736489in}{1.559384in}}{\pgfqpoint{2.740396in}{1.555477in}}%
\pgfpathcurveto{\pgfqpoint{2.744302in}{1.551570in}}{\pgfqpoint{2.749602in}{1.549375in}}{\pgfqpoint{2.755127in}{1.549375in}}%
\pgfpathclose%
\pgfusepath{fill}%
\end{pgfscope}%
\begin{pgfscope}%
\pgfpathrectangle{\pgfqpoint{1.995685in}{0.833185in}}{\pgfqpoint{1.162500in}{0.755000in}} %
\pgfusepath{clip}%
\pgfsetbuttcap%
\pgfsetroundjoin%
\definecolor{currentfill}{rgb}{0.000000,0.000000,0.000000}%
\pgfsetfillcolor{currentfill}%
\pgfsetfillopacity{0.500000}%
\pgfsetlinewidth{0.000000pt}%
\definecolor{currentstroke}{rgb}{0.000000,0.000000,0.000000}%
\pgfsetstrokecolor{currentstroke}%
\pgfsetdash{}{0pt}%
\pgfpathmoveto{\pgfqpoint{2.317114in}{1.273758in}}%
\pgfpathcurveto{\pgfqpoint{2.322639in}{1.273758in}}{\pgfqpoint{2.327939in}{1.275953in}}{\pgfqpoint{2.331845in}{1.279859in}}%
\pgfpathcurveto{\pgfqpoint{2.335752in}{1.283766in}}{\pgfqpoint{2.337947in}{1.289066in}}{\pgfqpoint{2.337947in}{1.294591in}}%
\pgfpathcurveto{\pgfqpoint{2.337947in}{1.300116in}}{\pgfqpoint{2.335752in}{1.305415in}}{\pgfqpoint{2.331845in}{1.309322in}}%
\pgfpathcurveto{\pgfqpoint{2.327939in}{1.313229in}}{\pgfqpoint{2.322639in}{1.315424in}}{\pgfqpoint{2.317114in}{1.315424in}}%
\pgfpathcurveto{\pgfqpoint{2.311589in}{1.315424in}}{\pgfqpoint{2.306289in}{1.313229in}}{\pgfqpoint{2.302383in}{1.309322in}}%
\pgfpathcurveto{\pgfqpoint{2.298476in}{1.305415in}}{\pgfqpoint{2.296281in}{1.300116in}}{\pgfqpoint{2.296281in}{1.294591in}}%
\pgfpathcurveto{\pgfqpoint{2.296281in}{1.289066in}}{\pgfqpoint{2.298476in}{1.283766in}}{\pgfqpoint{2.302383in}{1.279859in}}%
\pgfpathcurveto{\pgfqpoint{2.306289in}{1.275953in}}{\pgfqpoint{2.311589in}{1.273758in}}{\pgfqpoint{2.317114in}{1.273758in}}%
\pgfpathclose%
\pgfusepath{fill}%
\end{pgfscope}%
\begin{pgfscope}%
\pgfpathrectangle{\pgfqpoint{1.995685in}{0.833185in}}{\pgfqpoint{1.162500in}{0.755000in}} %
\pgfusepath{clip}%
\pgfsetbuttcap%
\pgfsetroundjoin%
\definecolor{currentfill}{rgb}{0.000000,0.000000,0.000000}%
\pgfsetfillcolor{currentfill}%
\pgfsetfillopacity{0.500000}%
\pgfsetlinewidth{0.000000pt}%
\definecolor{currentstroke}{rgb}{0.000000,0.000000,0.000000}%
\pgfsetstrokecolor{currentstroke}%
\pgfsetdash{}{0pt}%
\pgfpathmoveto{\pgfqpoint{2.241285in}{1.319843in}}%
\pgfpathcurveto{\pgfqpoint{2.246810in}{1.319843in}}{\pgfqpoint{2.252110in}{1.322038in}}{\pgfqpoint{2.256017in}{1.325945in}}%
\pgfpathcurveto{\pgfqpoint{2.259923in}{1.329852in}}{\pgfqpoint{2.262118in}{1.335151in}}{\pgfqpoint{2.262118in}{1.340676in}}%
\pgfpathcurveto{\pgfqpoint{2.262118in}{1.346201in}}{\pgfqpoint{2.259923in}{1.351501in}}{\pgfqpoint{2.256017in}{1.355408in}}%
\pgfpathcurveto{\pgfqpoint{2.252110in}{1.359314in}}{\pgfqpoint{2.246810in}{1.361510in}}{\pgfqpoint{2.241285in}{1.361510in}}%
\pgfpathcurveto{\pgfqpoint{2.235760in}{1.361510in}}{\pgfqpoint{2.230461in}{1.359314in}}{\pgfqpoint{2.226554in}{1.355408in}}%
\pgfpathcurveto{\pgfqpoint{2.222647in}{1.351501in}}{\pgfqpoint{2.220452in}{1.346201in}}{\pgfqpoint{2.220452in}{1.340676in}}%
\pgfpathcurveto{\pgfqpoint{2.220452in}{1.335151in}}{\pgfqpoint{2.222647in}{1.329852in}}{\pgfqpoint{2.226554in}{1.325945in}}%
\pgfpathcurveto{\pgfqpoint{2.230461in}{1.322038in}}{\pgfqpoint{2.235760in}{1.319843in}}{\pgfqpoint{2.241285in}{1.319843in}}%
\pgfpathclose%
\pgfusepath{fill}%
\end{pgfscope}%
\begin{pgfscope}%
\pgfpathrectangle{\pgfqpoint{1.995685in}{0.833185in}}{\pgfqpoint{1.162500in}{0.755000in}} %
\pgfusepath{clip}%
\pgfsetbuttcap%
\pgfsetroundjoin%
\definecolor{currentfill}{rgb}{0.000000,0.000000,0.000000}%
\pgfsetfillcolor{currentfill}%
\pgfsetfillopacity{0.500000}%
\pgfsetlinewidth{0.000000pt}%
\definecolor{currentstroke}{rgb}{0.000000,0.000000,0.000000}%
\pgfsetstrokecolor{currentstroke}%
\pgfsetdash{}{0pt}%
\pgfpathmoveto{\pgfqpoint{2.106066in}{1.215122in}}%
\pgfpathcurveto{\pgfqpoint{2.111591in}{1.215122in}}{\pgfqpoint{2.116890in}{1.217317in}}{\pgfqpoint{2.120797in}{1.221224in}}%
\pgfpathcurveto{\pgfqpoint{2.124704in}{1.225130in}}{\pgfqpoint{2.126899in}{1.230430in}}{\pgfqpoint{2.126899in}{1.235955in}}%
\pgfpathcurveto{\pgfqpoint{2.126899in}{1.241480in}}{\pgfqpoint{2.124704in}{1.246780in}}{\pgfqpoint{2.120797in}{1.250686in}}%
\pgfpathcurveto{\pgfqpoint{2.116890in}{1.254593in}}{\pgfqpoint{2.111591in}{1.256788in}}{\pgfqpoint{2.106066in}{1.256788in}}%
\pgfpathcurveto{\pgfqpoint{2.100541in}{1.256788in}}{\pgfqpoint{2.095241in}{1.254593in}}{\pgfqpoint{2.091334in}{1.250686in}}%
\pgfpathcurveto{\pgfqpoint{2.087428in}{1.246780in}}{\pgfqpoint{2.085232in}{1.241480in}}{\pgfqpoint{2.085232in}{1.235955in}}%
\pgfpathcurveto{\pgfqpoint{2.085232in}{1.230430in}}{\pgfqpoint{2.087428in}{1.225130in}}{\pgfqpoint{2.091334in}{1.221224in}}%
\pgfpathcurveto{\pgfqpoint{2.095241in}{1.217317in}}{\pgfqpoint{2.100541in}{1.215122in}}{\pgfqpoint{2.106066in}{1.215122in}}%
\pgfpathclose%
\pgfusepath{fill}%
\end{pgfscope}%
\begin{pgfscope}%
\pgfpathrectangle{\pgfqpoint{1.995685in}{0.833185in}}{\pgfqpoint{1.162500in}{0.755000in}} %
\pgfusepath{clip}%
\pgfsetbuttcap%
\pgfsetroundjoin%
\definecolor{currentfill}{rgb}{0.000000,0.000000,0.000000}%
\pgfsetfillcolor{currentfill}%
\pgfsetfillopacity{0.500000}%
\pgfsetlinewidth{0.000000pt}%
\definecolor{currentstroke}{rgb}{0.000000,0.000000,0.000000}%
\pgfsetstrokecolor{currentstroke}%
\pgfsetdash{}{0pt}%
\pgfpathmoveto{\pgfqpoint{2.023363in}{0.830327in}}%
\pgfpathcurveto{\pgfqpoint{2.028888in}{0.830327in}}{\pgfqpoint{2.034188in}{0.832523in}}{\pgfqpoint{2.038095in}{0.836429in}}%
\pgfpathcurveto{\pgfqpoint{2.042001in}{0.840336in}}{\pgfqpoint{2.044196in}{0.845636in}}{\pgfqpoint{2.044196in}{0.851161in}}%
\pgfpathcurveto{\pgfqpoint{2.044196in}{0.856686in}}{\pgfqpoint{2.042001in}{0.861985in}}{\pgfqpoint{2.038095in}{0.865892in}}%
\pgfpathcurveto{\pgfqpoint{2.034188in}{0.869799in}}{\pgfqpoint{2.028888in}{0.871994in}}{\pgfqpoint{2.023363in}{0.871994in}}%
\pgfpathcurveto{\pgfqpoint{2.017838in}{0.871994in}}{\pgfqpoint{2.012539in}{0.869799in}}{\pgfqpoint{2.008632in}{0.865892in}}%
\pgfpathcurveto{\pgfqpoint{2.004725in}{0.861985in}}{\pgfqpoint{2.002530in}{0.856686in}}{\pgfqpoint{2.002530in}{0.851161in}}%
\pgfpathcurveto{\pgfqpoint{2.002530in}{0.845636in}}{\pgfqpoint{2.004725in}{0.840336in}}{\pgfqpoint{2.008632in}{0.836429in}}%
\pgfpathcurveto{\pgfqpoint{2.012539in}{0.832523in}}{\pgfqpoint{2.017838in}{0.830327in}}{\pgfqpoint{2.023363in}{0.830327in}}%
\pgfpathclose%
\pgfusepath{fill}%
\end{pgfscope}%
\begin{pgfscope}%
\pgfpathrectangle{\pgfqpoint{1.995685in}{0.833185in}}{\pgfqpoint{1.162500in}{0.755000in}} %
\pgfusepath{clip}%
\pgfsetbuttcap%
\pgfsetroundjoin%
\definecolor{currentfill}{rgb}{0.000000,0.000000,0.000000}%
\pgfsetfillcolor{currentfill}%
\pgfsetfillopacity{0.500000}%
\pgfsetlinewidth{0.000000pt}%
\definecolor{currentstroke}{rgb}{0.000000,0.000000,0.000000}%
\pgfsetstrokecolor{currentstroke}%
\pgfsetdash{}{0pt}%
\pgfpathmoveto{\pgfqpoint{2.907651in}{1.469697in}}%
\pgfpathcurveto{\pgfqpoint{2.913176in}{1.469697in}}{\pgfqpoint{2.918476in}{1.471892in}}{\pgfqpoint{2.922382in}{1.475799in}}%
\pgfpathcurveto{\pgfqpoint{2.926289in}{1.479706in}}{\pgfqpoint{2.928484in}{1.485005in}}{\pgfqpoint{2.928484in}{1.490530in}}%
\pgfpathcurveto{\pgfqpoint{2.928484in}{1.496056in}}{\pgfqpoint{2.926289in}{1.501355in}}{\pgfqpoint{2.922382in}{1.505262in}}%
\pgfpathcurveto{\pgfqpoint{2.918476in}{1.509169in}}{\pgfqpoint{2.913176in}{1.511364in}}{\pgfqpoint{2.907651in}{1.511364in}}%
\pgfpathcurveto{\pgfqpoint{2.902126in}{1.511364in}}{\pgfqpoint{2.896827in}{1.509169in}}{\pgfqpoint{2.892920in}{1.505262in}}%
\pgfpathcurveto{\pgfqpoint{2.889013in}{1.501355in}}{\pgfqpoint{2.886818in}{1.496056in}}{\pgfqpoint{2.886818in}{1.490530in}}%
\pgfpathcurveto{\pgfqpoint{2.886818in}{1.485005in}}{\pgfqpoint{2.889013in}{1.479706in}}{\pgfqpoint{2.892920in}{1.475799in}}%
\pgfpathcurveto{\pgfqpoint{2.896827in}{1.471892in}}{\pgfqpoint{2.902126in}{1.469697in}}{\pgfqpoint{2.907651in}{1.469697in}}%
\pgfpathclose%
\pgfusepath{fill}%
\end{pgfscope}%
\begin{pgfscope}%
\pgfpathrectangle{\pgfqpoint{1.995685in}{0.833185in}}{\pgfqpoint{1.162500in}{0.755000in}} %
\pgfusepath{clip}%
\pgfsetbuttcap%
\pgfsetroundjoin%
\definecolor{currentfill}{rgb}{0.000000,0.000000,0.000000}%
\pgfsetfillcolor{currentfill}%
\pgfsetfillopacity{0.500000}%
\pgfsetlinewidth{0.000000pt}%
\definecolor{currentstroke}{rgb}{0.000000,0.000000,0.000000}%
\pgfsetstrokecolor{currentstroke}%
\pgfsetdash{}{0pt}%
\pgfpathmoveto{\pgfqpoint{2.630361in}{1.312260in}}%
\pgfpathcurveto{\pgfqpoint{2.635886in}{1.312260in}}{\pgfqpoint{2.641186in}{1.314455in}}{\pgfqpoint{2.645093in}{1.318362in}}%
\pgfpathcurveto{\pgfqpoint{2.649000in}{1.322269in}}{\pgfqpoint{2.651195in}{1.327569in}}{\pgfqpoint{2.651195in}{1.333094in}}%
\pgfpathcurveto{\pgfqpoint{2.651195in}{1.338619in}}{\pgfqpoint{2.649000in}{1.343918in}}{\pgfqpoint{2.645093in}{1.347825in}}%
\pgfpathcurveto{\pgfqpoint{2.641186in}{1.351732in}}{\pgfqpoint{2.635886in}{1.353927in}}{\pgfqpoint{2.630361in}{1.353927in}}%
\pgfpathcurveto{\pgfqpoint{2.624836in}{1.353927in}}{\pgfqpoint{2.619537in}{1.351732in}}{\pgfqpoint{2.615630in}{1.347825in}}%
\pgfpathcurveto{\pgfqpoint{2.611723in}{1.343918in}}{\pgfqpoint{2.609528in}{1.338619in}}{\pgfqpoint{2.609528in}{1.333094in}}%
\pgfpathcurveto{\pgfqpoint{2.609528in}{1.327569in}}{\pgfqpoint{2.611723in}{1.322269in}}{\pgfqpoint{2.615630in}{1.318362in}}%
\pgfpathcurveto{\pgfqpoint{2.619537in}{1.314455in}}{\pgfqpoint{2.624836in}{1.312260in}}{\pgfqpoint{2.630361in}{1.312260in}}%
\pgfpathclose%
\pgfusepath{fill}%
\end{pgfscope}%
\begin{pgfscope}%
\pgfpathrectangle{\pgfqpoint{1.995685in}{0.833185in}}{\pgfqpoint{1.162500in}{0.755000in}} %
\pgfusepath{clip}%
\pgfsetbuttcap%
\pgfsetroundjoin%
\definecolor{currentfill}{rgb}{0.000000,0.000000,0.000000}%
\pgfsetfillcolor{currentfill}%
\pgfsetfillopacity{0.500000}%
\pgfsetlinewidth{0.000000pt}%
\definecolor{currentstroke}{rgb}{0.000000,0.000000,0.000000}%
\pgfsetstrokecolor{currentstroke}%
\pgfsetdash{}{0pt}%
\pgfpathmoveto{\pgfqpoint{2.487752in}{1.134966in}}%
\pgfpathcurveto{\pgfqpoint{2.493277in}{1.134966in}}{\pgfqpoint{2.498576in}{1.137161in}}{\pgfqpoint{2.502483in}{1.141068in}}%
\pgfpathcurveto{\pgfqpoint{2.506390in}{1.144975in}}{\pgfqpoint{2.508585in}{1.150275in}}{\pgfqpoint{2.508585in}{1.155800in}}%
\pgfpathcurveto{\pgfqpoint{2.508585in}{1.161325in}}{\pgfqpoint{2.506390in}{1.166624in}}{\pgfqpoint{2.502483in}{1.170531in}}%
\pgfpathcurveto{\pgfqpoint{2.498576in}{1.174438in}}{\pgfqpoint{2.493277in}{1.176633in}}{\pgfqpoint{2.487752in}{1.176633in}}%
\pgfpathcurveto{\pgfqpoint{2.482227in}{1.176633in}}{\pgfqpoint{2.476927in}{1.174438in}}{\pgfqpoint{2.473020in}{1.170531in}}%
\pgfpathcurveto{\pgfqpoint{2.469113in}{1.166624in}}{\pgfqpoint{2.466918in}{1.161325in}}{\pgfqpoint{2.466918in}{1.155800in}}%
\pgfpathcurveto{\pgfqpoint{2.466918in}{1.150275in}}{\pgfqpoint{2.469113in}{1.144975in}}{\pgfqpoint{2.473020in}{1.141068in}}%
\pgfpathcurveto{\pgfqpoint{2.476927in}{1.137161in}}{\pgfqpoint{2.482227in}{1.134966in}}{\pgfqpoint{2.487752in}{1.134966in}}%
\pgfpathclose%
\pgfusepath{fill}%
\end{pgfscope}%
\begin{pgfscope}%
\pgfpathrectangle{\pgfqpoint{1.995685in}{0.833185in}}{\pgfqpoint{1.162500in}{0.755000in}} %
\pgfusepath{clip}%
\pgfsetbuttcap%
\pgfsetroundjoin%
\definecolor{currentfill}{rgb}{0.000000,0.000000,0.000000}%
\pgfsetfillcolor{currentfill}%
\pgfsetfillopacity{0.500000}%
\pgfsetlinewidth{0.000000pt}%
\definecolor{currentstroke}{rgb}{0.000000,0.000000,0.000000}%
\pgfsetstrokecolor{currentstroke}%
\pgfsetdash{}{0pt}%
\pgfpathmoveto{\pgfqpoint{3.000738in}{1.421242in}}%
\pgfpathcurveto{\pgfqpoint{3.006263in}{1.421242in}}{\pgfqpoint{3.011563in}{1.423437in}}{\pgfqpoint{3.015470in}{1.427344in}}%
\pgfpathcurveto{\pgfqpoint{3.019376in}{1.431251in}}{\pgfqpoint{3.021572in}{1.436550in}}{\pgfqpoint{3.021572in}{1.442075in}}%
\pgfpathcurveto{\pgfqpoint{3.021572in}{1.447600in}}{\pgfqpoint{3.019376in}{1.452900in}}{\pgfqpoint{3.015470in}{1.456807in}}%
\pgfpathcurveto{\pgfqpoint{3.011563in}{1.460714in}}{\pgfqpoint{3.006263in}{1.462909in}}{\pgfqpoint{3.000738in}{1.462909in}}%
\pgfpathcurveto{\pgfqpoint{2.995213in}{1.462909in}}{\pgfqpoint{2.989914in}{1.460714in}}{\pgfqpoint{2.986007in}{1.456807in}}%
\pgfpathcurveto{\pgfqpoint{2.982100in}{1.452900in}}{\pgfqpoint{2.979905in}{1.447600in}}{\pgfqpoint{2.979905in}{1.442075in}}%
\pgfpathcurveto{\pgfqpoint{2.979905in}{1.436550in}}{\pgfqpoint{2.982100in}{1.431251in}}{\pgfqpoint{2.986007in}{1.427344in}}%
\pgfpathcurveto{\pgfqpoint{2.989914in}{1.423437in}}{\pgfqpoint{2.995213in}{1.421242in}}{\pgfqpoint{3.000738in}{1.421242in}}%
\pgfpathclose%
\pgfusepath{fill}%
\end{pgfscope}%
\begin{pgfscope}%
\pgfpathrectangle{\pgfqpoint{1.995685in}{0.833185in}}{\pgfqpoint{1.162500in}{0.755000in}} %
\pgfusepath{clip}%
\pgfsetbuttcap%
\pgfsetroundjoin%
\definecolor{currentfill}{rgb}{0.000000,0.000000,0.000000}%
\pgfsetfillcolor{currentfill}%
\pgfsetfillopacity{0.500000}%
\pgfsetlinewidth{0.000000pt}%
\definecolor{currentstroke}{rgb}{0.000000,0.000000,0.000000}%
\pgfsetstrokecolor{currentstroke}%
\pgfsetdash{}{0pt}%
\pgfpathmoveto{\pgfqpoint{2.318352in}{1.126134in}}%
\pgfpathcurveto{\pgfqpoint{2.323877in}{1.126134in}}{\pgfqpoint{2.329176in}{1.128329in}}{\pgfqpoint{2.333083in}{1.132236in}}%
\pgfpathcurveto{\pgfqpoint{2.336990in}{1.136143in}}{\pgfqpoint{2.339185in}{1.141442in}}{\pgfqpoint{2.339185in}{1.146967in}}%
\pgfpathcurveto{\pgfqpoint{2.339185in}{1.152492in}}{\pgfqpoint{2.336990in}{1.157792in}}{\pgfqpoint{2.333083in}{1.161699in}}%
\pgfpathcurveto{\pgfqpoint{2.329176in}{1.165606in}}{\pgfqpoint{2.323877in}{1.167801in}}{\pgfqpoint{2.318352in}{1.167801in}}%
\pgfpathcurveto{\pgfqpoint{2.312827in}{1.167801in}}{\pgfqpoint{2.307527in}{1.165606in}}{\pgfqpoint{2.303620in}{1.161699in}}%
\pgfpathcurveto{\pgfqpoint{2.299713in}{1.157792in}}{\pgfqpoint{2.297518in}{1.152492in}}{\pgfqpoint{2.297518in}{1.146967in}}%
\pgfpathcurveto{\pgfqpoint{2.297518in}{1.141442in}}{\pgfqpoint{2.299713in}{1.136143in}}{\pgfqpoint{2.303620in}{1.132236in}}%
\pgfpathcurveto{\pgfqpoint{2.307527in}{1.128329in}}{\pgfqpoint{2.312827in}{1.126134in}}{\pgfqpoint{2.318352in}{1.126134in}}%
\pgfpathclose%
\pgfusepath{fill}%
\end{pgfscope}%
\begin{pgfscope}%
\pgfpathrectangle{\pgfqpoint{1.995685in}{0.833185in}}{\pgfqpoint{1.162500in}{0.755000in}} %
\pgfusepath{clip}%
\pgfsetbuttcap%
\pgfsetroundjoin%
\definecolor{currentfill}{rgb}{0.000000,0.000000,0.000000}%
\pgfsetfillcolor{currentfill}%
\pgfsetfillopacity{0.500000}%
\pgfsetlinewidth{0.000000pt}%
\definecolor{currentstroke}{rgb}{0.000000,0.000000,0.000000}%
\pgfsetstrokecolor{currentstroke}%
\pgfsetdash{}{0pt}%
\pgfpathmoveto{\pgfqpoint{2.410257in}{0.888020in}}%
\pgfpathcurveto{\pgfqpoint{2.415782in}{0.888020in}}{\pgfqpoint{2.421082in}{0.890215in}}{\pgfqpoint{2.424989in}{0.894122in}}%
\pgfpathcurveto{\pgfqpoint{2.428896in}{0.898028in}}{\pgfqpoint{2.431091in}{0.903328in}}{\pgfqpoint{2.431091in}{0.908853in}}%
\pgfpathcurveto{\pgfqpoint{2.431091in}{0.914378in}}{\pgfqpoint{2.428896in}{0.919677in}}{\pgfqpoint{2.424989in}{0.923584in}}%
\pgfpathcurveto{\pgfqpoint{2.421082in}{0.927491in}}{\pgfqpoint{2.415782in}{0.929686in}}{\pgfqpoint{2.410257in}{0.929686in}}%
\pgfpathcurveto{\pgfqpoint{2.404732in}{0.929686in}}{\pgfqpoint{2.399433in}{0.927491in}}{\pgfqpoint{2.395526in}{0.923584in}}%
\pgfpathcurveto{\pgfqpoint{2.391619in}{0.919677in}}{\pgfqpoint{2.389424in}{0.914378in}}{\pgfqpoint{2.389424in}{0.908853in}}%
\pgfpathcurveto{\pgfqpoint{2.389424in}{0.903328in}}{\pgfqpoint{2.391619in}{0.898028in}}{\pgfqpoint{2.395526in}{0.894122in}}%
\pgfpathcurveto{\pgfqpoint{2.399433in}{0.890215in}}{\pgfqpoint{2.404732in}{0.888020in}}{\pgfqpoint{2.410257in}{0.888020in}}%
\pgfpathclose%
\pgfusepath{fill}%
\end{pgfscope}%
\begin{pgfscope}%
\pgfpathrectangle{\pgfqpoint{1.995685in}{0.833185in}}{\pgfqpoint{1.162500in}{0.755000in}} %
\pgfusepath{clip}%
\pgfsetbuttcap%
\pgfsetroundjoin%
\definecolor{currentfill}{rgb}{0.000000,0.000000,0.000000}%
\pgfsetfillcolor{currentfill}%
\pgfsetfillopacity{0.500000}%
\pgfsetlinewidth{0.000000pt}%
\definecolor{currentstroke}{rgb}{0.000000,0.000000,0.000000}%
\pgfsetstrokecolor{currentstroke}%
\pgfsetdash{}{0pt}%
\pgfpathmoveto{\pgfqpoint{2.752801in}{1.424328in}}%
\pgfpathcurveto{\pgfqpoint{2.758327in}{1.424328in}}{\pgfqpoint{2.763626in}{1.426523in}}{\pgfqpoint{2.767533in}{1.430430in}}%
\pgfpathcurveto{\pgfqpoint{2.771440in}{1.434337in}}{\pgfqpoint{2.773635in}{1.439636in}}{\pgfqpoint{2.773635in}{1.445161in}}%
\pgfpathcurveto{\pgfqpoint{2.773635in}{1.450687in}}{\pgfqpoint{2.771440in}{1.455986in}}{\pgfqpoint{2.767533in}{1.459893in}}%
\pgfpathcurveto{\pgfqpoint{2.763626in}{1.463800in}}{\pgfqpoint{2.758327in}{1.465995in}}{\pgfqpoint{2.752801in}{1.465995in}}%
\pgfpathcurveto{\pgfqpoint{2.747276in}{1.465995in}}{\pgfqpoint{2.741977in}{1.463800in}}{\pgfqpoint{2.738070in}{1.459893in}}%
\pgfpathcurveto{\pgfqpoint{2.734163in}{1.455986in}}{\pgfqpoint{2.731968in}{1.450687in}}{\pgfqpoint{2.731968in}{1.445161in}}%
\pgfpathcurveto{\pgfqpoint{2.731968in}{1.439636in}}{\pgfqpoint{2.734163in}{1.434337in}}{\pgfqpoint{2.738070in}{1.430430in}}%
\pgfpathcurveto{\pgfqpoint{2.741977in}{1.426523in}}{\pgfqpoint{2.747276in}{1.424328in}}{\pgfqpoint{2.752801in}{1.424328in}}%
\pgfpathclose%
\pgfusepath{fill}%
\end{pgfscope}%
\begin{pgfscope}%
\pgfpathrectangle{\pgfqpoint{1.995685in}{0.833185in}}{\pgfqpoint{1.162500in}{0.755000in}} %
\pgfusepath{clip}%
\pgfsetbuttcap%
\pgfsetroundjoin%
\definecolor{currentfill}{rgb}{0.000000,0.000000,0.000000}%
\pgfsetfillcolor{currentfill}%
\pgfsetfillopacity{0.500000}%
\pgfsetlinewidth{0.000000pt}%
\definecolor{currentstroke}{rgb}{0.000000,0.000000,0.000000}%
\pgfsetstrokecolor{currentstroke}%
\pgfsetdash{}{0pt}%
\pgfpathmoveto{\pgfqpoint{2.103337in}{1.186596in}}%
\pgfpathcurveto{\pgfqpoint{2.108862in}{1.186596in}}{\pgfqpoint{2.114161in}{1.188791in}}{\pgfqpoint{2.118068in}{1.192698in}}%
\pgfpathcurveto{\pgfqpoint{2.121975in}{1.196604in}}{\pgfqpoint{2.124170in}{1.201904in}}{\pgfqpoint{2.124170in}{1.207429in}}%
\pgfpathcurveto{\pgfqpoint{2.124170in}{1.212954in}}{\pgfqpoint{2.121975in}{1.218254in}}{\pgfqpoint{2.118068in}{1.222160in}}%
\pgfpathcurveto{\pgfqpoint{2.114161in}{1.226067in}}{\pgfqpoint{2.108862in}{1.228262in}}{\pgfqpoint{2.103337in}{1.228262in}}%
\pgfpathcurveto{\pgfqpoint{2.097812in}{1.228262in}}{\pgfqpoint{2.092512in}{1.226067in}}{\pgfqpoint{2.088605in}{1.222160in}}%
\pgfpathcurveto{\pgfqpoint{2.084698in}{1.218254in}}{\pgfqpoint{2.082503in}{1.212954in}}{\pgfqpoint{2.082503in}{1.207429in}}%
\pgfpathcurveto{\pgfqpoint{2.082503in}{1.201904in}}{\pgfqpoint{2.084698in}{1.196604in}}{\pgfqpoint{2.088605in}{1.192698in}}%
\pgfpathcurveto{\pgfqpoint{2.092512in}{1.188791in}}{\pgfqpoint{2.097812in}{1.186596in}}{\pgfqpoint{2.103337in}{1.186596in}}%
\pgfpathclose%
\pgfusepath{fill}%
\end{pgfscope}%
\begin{pgfscope}%
\pgfsetbuttcap%
\pgfsetroundjoin%
\definecolor{currentfill}{rgb}{0.000000,0.000000,0.000000}%
\pgfsetfillcolor{currentfill}%
\pgfsetlinewidth{0.803000pt}%
\definecolor{currentstroke}{rgb}{0.000000,0.000000,0.000000}%
\pgfsetstrokecolor{currentstroke}%
\pgfsetdash{}{0pt}%
\pgfsys@defobject{currentmarker}{\pgfqpoint{0.000000in}{-0.048611in}}{\pgfqpoint{0.000000in}{0.000000in}}{%
\pgfpathmoveto{\pgfqpoint{0.000000in}{0.000000in}}%
\pgfpathlineto{\pgfqpoint{0.000000in}{-0.048611in}}%
\pgfusepath{stroke,fill}%
}%
\begin{pgfscope}%
\pgfsys@transformshift{2.474914in}{0.833185in}%
\pgfsys@useobject{currentmarker}{}%
\end{pgfscope}%
\end{pgfscope}%
\begin{pgfscope}%
\pgftext[x=2.505567in,y=0.289968in,left,base,rotate=90.000000]{\rmfamily\fontsize{8.000000}{9.600000}\selectfont \(\displaystyle 0.000025\)}%
\end{pgfscope}%
\begin{pgfscope}%
\pgfsetbuttcap%
\pgfsetroundjoin%
\definecolor{currentfill}{rgb}{0.000000,0.000000,0.000000}%
\pgfsetfillcolor{currentfill}%
\pgfsetlinewidth{0.803000pt}%
\definecolor{currentstroke}{rgb}{0.000000,0.000000,0.000000}%
\pgfsetstrokecolor{currentstroke}%
\pgfsetdash{}{0pt}%
\pgfsys@defobject{currentmarker}{\pgfqpoint{0.000000in}{-0.048611in}}{\pgfqpoint{0.000000in}{0.000000in}}{%
\pgfpathmoveto{\pgfqpoint{0.000000in}{0.000000in}}%
\pgfpathlineto{\pgfqpoint{0.000000in}{-0.048611in}}%
\pgfusepath{stroke,fill}%
}%
\begin{pgfscope}%
\pgfsys@transformshift{3.075299in}{0.833185in}%
\pgfsys@useobject{currentmarker}{}%
\end{pgfscope}%
\end{pgfscope}%
\begin{pgfscope}%
\pgftext[x=3.105952in,y=0.289968in,left,base,rotate=90.000000]{\rmfamily\fontsize{8.000000}{9.600000}\selectfont \(\displaystyle 0.000050\)}%
\end{pgfscope}%
\begin{pgfscope}%
\pgftext[x=2.576935in,y=0.234413in,,top]{\rmfamily\fontsize{10.000000}{12.000000}\selectfont area}%
\end{pgfscope}%
\begin{pgfscope}%
\pgfsetrectcap%
\pgfsetmiterjoin%
\pgfsetlinewidth{0.803000pt}%
\definecolor{currentstroke}{rgb}{0.000000,0.000000,0.000000}%
\pgfsetstrokecolor{currentstroke}%
\pgfsetdash{}{0pt}%
\pgfpathmoveto{\pgfqpoint{1.995685in}{0.833185in}}%
\pgfpathlineto{\pgfqpoint{1.995685in}{1.588185in}}%
\pgfusepath{stroke}%
\end{pgfscope}%
\begin{pgfscope}%
\pgfsetrectcap%
\pgfsetmiterjoin%
\pgfsetlinewidth{0.803000pt}%
\definecolor{currentstroke}{rgb}{0.000000,0.000000,0.000000}%
\pgfsetstrokecolor{currentstroke}%
\pgfsetdash{}{0pt}%
\pgfpathmoveto{\pgfqpoint{3.158185in}{0.833185in}}%
\pgfpathlineto{\pgfqpoint{3.158185in}{1.588185in}}%
\pgfusepath{stroke}%
\end{pgfscope}%
\begin{pgfscope}%
\pgfsetrectcap%
\pgfsetmiterjoin%
\pgfsetlinewidth{0.803000pt}%
\definecolor{currentstroke}{rgb}{0.000000,0.000000,0.000000}%
\pgfsetstrokecolor{currentstroke}%
\pgfsetdash{}{0pt}%
\pgfpathmoveto{\pgfqpoint{1.995685in}{0.833185in}}%
\pgfpathlineto{\pgfqpoint{3.158185in}{0.833185in}}%
\pgfusepath{stroke}%
\end{pgfscope}%
\begin{pgfscope}%
\pgfsetrectcap%
\pgfsetmiterjoin%
\pgfsetlinewidth{0.803000pt}%
\definecolor{currentstroke}{rgb}{0.000000,0.000000,0.000000}%
\pgfsetstrokecolor{currentstroke}%
\pgfsetdash{}{0pt}%
\pgfpathmoveto{\pgfqpoint{1.995685in}{1.588185in}}%
\pgfpathlineto{\pgfqpoint{3.158185in}{1.588185in}}%
\pgfusepath{stroke}%
\end{pgfscope}%
\begin{pgfscope}%
\pgfsetbuttcap%
\pgfsetmiterjoin%
\definecolor{currentfill}{rgb}{1.000000,1.000000,1.000000}%
\pgfsetfillcolor{currentfill}%
\pgfsetlinewidth{0.000000pt}%
\definecolor{currentstroke}{rgb}{0.000000,0.000000,0.000000}%
\pgfsetstrokecolor{currentstroke}%
\pgfsetstrokeopacity{0.000000}%
\pgfsetdash{}{0pt}%
\pgfpathmoveto{\pgfqpoint{3.158185in}{0.833185in}}%
\pgfpathlineto{\pgfqpoint{4.320685in}{0.833185in}}%
\pgfpathlineto{\pgfqpoint{4.320685in}{1.588185in}}%
\pgfpathlineto{\pgfqpoint{3.158185in}{1.588185in}}%
\pgfpathclose%
\pgfusepath{fill}%
\end{pgfscope}%
\begin{pgfscope}%
\pgfpathrectangle{\pgfqpoint{3.158185in}{0.833185in}}{\pgfqpoint{1.162500in}{0.755000in}} %
\pgfusepath{clip}%
\pgfsetbuttcap%
\pgfsetroundjoin%
\definecolor{currentfill}{rgb}{0.000000,0.000000,0.000000}%
\pgfsetfillcolor{currentfill}%
\pgfsetfillopacity{0.500000}%
\pgfsetlinewidth{0.000000pt}%
\definecolor{currentstroke}{rgb}{0.000000,0.000000,0.000000}%
\pgfsetstrokecolor{currentstroke}%
\pgfsetdash{}{0pt}%
\pgfpathmoveto{\pgfqpoint{4.293006in}{1.099971in}}%
\pgfpathcurveto{\pgfqpoint{4.298531in}{1.099971in}}{\pgfqpoint{4.303831in}{1.102166in}}{\pgfqpoint{4.307737in}{1.106073in}}%
\pgfpathcurveto{\pgfqpoint{4.311644in}{1.109980in}}{\pgfqpoint{4.313839in}{1.115279in}}{\pgfqpoint{4.313839in}{1.120804in}}%
\pgfpathcurveto{\pgfqpoint{4.313839in}{1.126329in}}{\pgfqpoint{4.311644in}{1.131629in}}{\pgfqpoint{4.307737in}{1.135536in}}%
\pgfpathcurveto{\pgfqpoint{4.303831in}{1.139442in}}{\pgfqpoint{4.298531in}{1.141638in}}{\pgfqpoint{4.293006in}{1.141638in}}%
\pgfpathcurveto{\pgfqpoint{4.287481in}{1.141638in}}{\pgfqpoint{4.282181in}{1.139442in}}{\pgfqpoint{4.278275in}{1.135536in}}%
\pgfpathcurveto{\pgfqpoint{4.274368in}{1.131629in}}{\pgfqpoint{4.272173in}{1.126329in}}{\pgfqpoint{4.272173in}{1.120804in}}%
\pgfpathcurveto{\pgfqpoint{4.272173in}{1.115279in}}{\pgfqpoint{4.274368in}{1.109980in}}{\pgfqpoint{4.278275in}{1.106073in}}%
\pgfpathcurveto{\pgfqpoint{4.282181in}{1.102166in}}{\pgfqpoint{4.287481in}{1.099971in}}{\pgfqpoint{4.293006in}{1.099971in}}%
\pgfpathclose%
\pgfusepath{fill}%
\end{pgfscope}%
\begin{pgfscope}%
\pgfpathrectangle{\pgfqpoint{3.158185in}{0.833185in}}{\pgfqpoint{1.162500in}{0.755000in}} %
\pgfusepath{clip}%
\pgfsetbuttcap%
\pgfsetroundjoin%
\definecolor{currentfill}{rgb}{0.000000,0.000000,0.000000}%
\pgfsetfillcolor{currentfill}%
\pgfsetfillopacity{0.500000}%
\pgfsetlinewidth{0.000000pt}%
\definecolor{currentstroke}{rgb}{0.000000,0.000000,0.000000}%
\pgfsetstrokecolor{currentstroke}%
\pgfsetdash{}{0pt}%
\pgfpathmoveto{\pgfqpoint{3.586831in}{1.503836in}}%
\pgfpathcurveto{\pgfqpoint{3.592356in}{1.503836in}}{\pgfqpoint{3.597655in}{1.506031in}}{\pgfqpoint{3.601562in}{1.509938in}}%
\pgfpathcurveto{\pgfqpoint{3.605469in}{1.513845in}}{\pgfqpoint{3.607664in}{1.519144in}}{\pgfqpoint{3.607664in}{1.524669in}}%
\pgfpathcurveto{\pgfqpoint{3.607664in}{1.530195in}}{\pgfqpoint{3.605469in}{1.535494in}}{\pgfqpoint{3.601562in}{1.539401in}}%
\pgfpathcurveto{\pgfqpoint{3.597655in}{1.543308in}}{\pgfqpoint{3.592356in}{1.545503in}}{\pgfqpoint{3.586831in}{1.545503in}}%
\pgfpathcurveto{\pgfqpoint{3.581306in}{1.545503in}}{\pgfqpoint{3.576006in}{1.543308in}}{\pgfqpoint{3.572099in}{1.539401in}}%
\pgfpathcurveto{\pgfqpoint{3.568193in}{1.535494in}}{\pgfqpoint{3.565998in}{1.530195in}}{\pgfqpoint{3.565998in}{1.524669in}}%
\pgfpathcurveto{\pgfqpoint{3.565998in}{1.519144in}}{\pgfqpoint{3.568193in}{1.513845in}}{\pgfqpoint{3.572099in}{1.509938in}}%
\pgfpathcurveto{\pgfqpoint{3.576006in}{1.506031in}}{\pgfqpoint{3.581306in}{1.503836in}}{\pgfqpoint{3.586831in}{1.503836in}}%
\pgfpathclose%
\pgfusepath{fill}%
\end{pgfscope}%
\begin{pgfscope}%
\pgfpathrectangle{\pgfqpoint{3.158185in}{0.833185in}}{\pgfqpoint{1.162500in}{0.755000in}} %
\pgfusepath{clip}%
\pgfsetbuttcap%
\pgfsetroundjoin%
\definecolor{currentfill}{rgb}{0.000000,0.000000,0.000000}%
\pgfsetfillcolor{currentfill}%
\pgfsetfillopacity{0.500000}%
\pgfsetlinewidth{0.000000pt}%
\definecolor{currentstroke}{rgb}{0.000000,0.000000,0.000000}%
\pgfsetstrokecolor{currentstroke}%
\pgfsetdash{}{0pt}%
\pgfpathmoveto{\pgfqpoint{3.578870in}{1.549375in}}%
\pgfpathcurveto{\pgfqpoint{3.584395in}{1.549375in}}{\pgfqpoint{3.589694in}{1.551570in}}{\pgfqpoint{3.593601in}{1.555477in}}%
\pgfpathcurveto{\pgfqpoint{3.597508in}{1.559384in}}{\pgfqpoint{3.599703in}{1.564683in}}{\pgfqpoint{3.599703in}{1.570208in}}%
\pgfpathcurveto{\pgfqpoint{3.599703in}{1.575733in}}{\pgfqpoint{3.597508in}{1.581033in}}{\pgfqpoint{3.593601in}{1.584940in}}%
\pgfpathcurveto{\pgfqpoint{3.589694in}{1.588847in}}{\pgfqpoint{3.584395in}{1.591042in}}{\pgfqpoint{3.578870in}{1.591042in}}%
\pgfpathcurveto{\pgfqpoint{3.573345in}{1.591042in}}{\pgfqpoint{3.568045in}{1.588847in}}{\pgfqpoint{3.564138in}{1.584940in}}%
\pgfpathcurveto{\pgfqpoint{3.560231in}{1.581033in}}{\pgfqpoint{3.558036in}{1.575733in}}{\pgfqpoint{3.558036in}{1.570208in}}%
\pgfpathcurveto{\pgfqpoint{3.558036in}{1.564683in}}{\pgfqpoint{3.560231in}{1.559384in}}{\pgfqpoint{3.564138in}{1.555477in}}%
\pgfpathcurveto{\pgfqpoint{3.568045in}{1.551570in}}{\pgfqpoint{3.573345in}{1.549375in}}{\pgfqpoint{3.578870in}{1.549375in}}%
\pgfpathclose%
\pgfusepath{fill}%
\end{pgfscope}%
\begin{pgfscope}%
\pgfpathrectangle{\pgfqpoint{3.158185in}{0.833185in}}{\pgfqpoint{1.162500in}{0.755000in}} %
\pgfusepath{clip}%
\pgfsetbuttcap%
\pgfsetroundjoin%
\definecolor{currentfill}{rgb}{0.000000,0.000000,0.000000}%
\pgfsetfillcolor{currentfill}%
\pgfsetfillopacity{0.500000}%
\pgfsetlinewidth{0.000000pt}%
\definecolor{currentstroke}{rgb}{0.000000,0.000000,0.000000}%
\pgfsetstrokecolor{currentstroke}%
\pgfsetdash{}{0pt}%
\pgfpathmoveto{\pgfqpoint{3.333069in}{1.273758in}}%
\pgfpathcurveto{\pgfqpoint{3.338594in}{1.273758in}}{\pgfqpoint{3.343894in}{1.275953in}}{\pgfqpoint{3.347801in}{1.279859in}}%
\pgfpathcurveto{\pgfqpoint{3.351708in}{1.283766in}}{\pgfqpoint{3.353903in}{1.289066in}}{\pgfqpoint{3.353903in}{1.294591in}}%
\pgfpathcurveto{\pgfqpoint{3.353903in}{1.300116in}}{\pgfqpoint{3.351708in}{1.305415in}}{\pgfqpoint{3.347801in}{1.309322in}}%
\pgfpathcurveto{\pgfqpoint{3.343894in}{1.313229in}}{\pgfqpoint{3.338594in}{1.315424in}}{\pgfqpoint{3.333069in}{1.315424in}}%
\pgfpathcurveto{\pgfqpoint{3.327544in}{1.315424in}}{\pgfqpoint{3.322245in}{1.313229in}}{\pgfqpoint{3.318338in}{1.309322in}}%
\pgfpathcurveto{\pgfqpoint{3.314431in}{1.305415in}}{\pgfqpoint{3.312236in}{1.300116in}}{\pgfqpoint{3.312236in}{1.294591in}}%
\pgfpathcurveto{\pgfqpoint{3.312236in}{1.289066in}}{\pgfqpoint{3.314431in}{1.283766in}}{\pgfqpoint{3.318338in}{1.279859in}}%
\pgfpathcurveto{\pgfqpoint{3.322245in}{1.275953in}}{\pgfqpoint{3.327544in}{1.273758in}}{\pgfqpoint{3.333069in}{1.273758in}}%
\pgfpathclose%
\pgfusepath{fill}%
\end{pgfscope}%
\begin{pgfscope}%
\pgfpathrectangle{\pgfqpoint{3.158185in}{0.833185in}}{\pgfqpoint{1.162500in}{0.755000in}} %
\pgfusepath{clip}%
\pgfsetbuttcap%
\pgfsetroundjoin%
\definecolor{currentfill}{rgb}{0.000000,0.000000,0.000000}%
\pgfsetfillcolor{currentfill}%
\pgfsetfillopacity{0.500000}%
\pgfsetlinewidth{0.000000pt}%
\definecolor{currentstroke}{rgb}{0.000000,0.000000,0.000000}%
\pgfsetstrokecolor{currentstroke}%
\pgfsetdash{}{0pt}%
\pgfpathmoveto{\pgfqpoint{3.296013in}{1.319843in}}%
\pgfpathcurveto{\pgfqpoint{3.301538in}{1.319843in}}{\pgfqpoint{3.306838in}{1.322038in}}{\pgfqpoint{3.310744in}{1.325945in}}%
\pgfpathcurveto{\pgfqpoint{3.314651in}{1.329852in}}{\pgfqpoint{3.316846in}{1.335151in}}{\pgfqpoint{3.316846in}{1.340676in}}%
\pgfpathcurveto{\pgfqpoint{3.316846in}{1.346201in}}{\pgfqpoint{3.314651in}{1.351501in}}{\pgfqpoint{3.310744in}{1.355408in}}%
\pgfpathcurveto{\pgfqpoint{3.306838in}{1.359314in}}{\pgfqpoint{3.301538in}{1.361510in}}{\pgfqpoint{3.296013in}{1.361510in}}%
\pgfpathcurveto{\pgfqpoint{3.290488in}{1.361510in}}{\pgfqpoint{3.285188in}{1.359314in}}{\pgfqpoint{3.281282in}{1.355408in}}%
\pgfpathcurveto{\pgfqpoint{3.277375in}{1.351501in}}{\pgfqpoint{3.275180in}{1.346201in}}{\pgfqpoint{3.275180in}{1.340676in}}%
\pgfpathcurveto{\pgfqpoint{3.275180in}{1.335151in}}{\pgfqpoint{3.277375in}{1.329852in}}{\pgfqpoint{3.281282in}{1.325945in}}%
\pgfpathcurveto{\pgfqpoint{3.285188in}{1.322038in}}{\pgfqpoint{3.290488in}{1.319843in}}{\pgfqpoint{3.296013in}{1.319843in}}%
\pgfpathclose%
\pgfusepath{fill}%
\end{pgfscope}%
\begin{pgfscope}%
\pgfpathrectangle{\pgfqpoint{3.158185in}{0.833185in}}{\pgfqpoint{1.162500in}{0.755000in}} %
\pgfusepath{clip}%
\pgfsetbuttcap%
\pgfsetroundjoin%
\definecolor{currentfill}{rgb}{0.000000,0.000000,0.000000}%
\pgfsetfillcolor{currentfill}%
\pgfsetfillopacity{0.500000}%
\pgfsetlinewidth{0.000000pt}%
\definecolor{currentstroke}{rgb}{0.000000,0.000000,0.000000}%
\pgfsetstrokecolor{currentstroke}%
\pgfsetdash{}{0pt}%
\pgfpathmoveto{\pgfqpoint{3.206735in}{1.215122in}}%
\pgfpathcurveto{\pgfqpoint{3.212260in}{1.215122in}}{\pgfqpoint{3.217559in}{1.217317in}}{\pgfqpoint{3.221466in}{1.221224in}}%
\pgfpathcurveto{\pgfqpoint{3.225373in}{1.225130in}}{\pgfqpoint{3.227568in}{1.230430in}}{\pgfqpoint{3.227568in}{1.235955in}}%
\pgfpathcurveto{\pgfqpoint{3.227568in}{1.241480in}}{\pgfqpoint{3.225373in}{1.246780in}}{\pgfqpoint{3.221466in}{1.250686in}}%
\pgfpathcurveto{\pgfqpoint{3.217559in}{1.254593in}}{\pgfqpoint{3.212260in}{1.256788in}}{\pgfqpoint{3.206735in}{1.256788in}}%
\pgfpathcurveto{\pgfqpoint{3.201210in}{1.256788in}}{\pgfqpoint{3.195910in}{1.254593in}}{\pgfqpoint{3.192003in}{1.250686in}}%
\pgfpathcurveto{\pgfqpoint{3.188097in}{1.246780in}}{\pgfqpoint{3.185901in}{1.241480in}}{\pgfqpoint{3.185901in}{1.235955in}}%
\pgfpathcurveto{\pgfqpoint{3.185901in}{1.230430in}}{\pgfqpoint{3.188097in}{1.225130in}}{\pgfqpoint{3.192003in}{1.221224in}}%
\pgfpathcurveto{\pgfqpoint{3.195910in}{1.217317in}}{\pgfqpoint{3.201210in}{1.215122in}}{\pgfqpoint{3.206735in}{1.215122in}}%
\pgfpathclose%
\pgfusepath{fill}%
\end{pgfscope}%
\begin{pgfscope}%
\pgfpathrectangle{\pgfqpoint{3.158185in}{0.833185in}}{\pgfqpoint{1.162500in}{0.755000in}} %
\pgfusepath{clip}%
\pgfsetbuttcap%
\pgfsetroundjoin%
\definecolor{currentfill}{rgb}{0.000000,0.000000,0.000000}%
\pgfsetfillcolor{currentfill}%
\pgfsetfillopacity{0.500000}%
\pgfsetlinewidth{0.000000pt}%
\definecolor{currentstroke}{rgb}{0.000000,0.000000,0.000000}%
\pgfsetstrokecolor{currentstroke}%
\pgfsetdash{}{0pt}%
\pgfpathmoveto{\pgfqpoint{3.185863in}{0.830327in}}%
\pgfpathcurveto{\pgfqpoint{3.191388in}{0.830327in}}{\pgfqpoint{3.196688in}{0.832523in}}{\pgfqpoint{3.200595in}{0.836429in}}%
\pgfpathcurveto{\pgfqpoint{3.204501in}{0.840336in}}{\pgfqpoint{3.206696in}{0.845636in}}{\pgfqpoint{3.206696in}{0.851161in}}%
\pgfpathcurveto{\pgfqpoint{3.206696in}{0.856686in}}{\pgfqpoint{3.204501in}{0.861985in}}{\pgfqpoint{3.200595in}{0.865892in}}%
\pgfpathcurveto{\pgfqpoint{3.196688in}{0.869799in}}{\pgfqpoint{3.191388in}{0.871994in}}{\pgfqpoint{3.185863in}{0.871994in}}%
\pgfpathcurveto{\pgfqpoint{3.180338in}{0.871994in}}{\pgfqpoint{3.175039in}{0.869799in}}{\pgfqpoint{3.171132in}{0.865892in}}%
\pgfpathcurveto{\pgfqpoint{3.167225in}{0.861985in}}{\pgfqpoint{3.165030in}{0.856686in}}{\pgfqpoint{3.165030in}{0.851161in}}%
\pgfpathcurveto{\pgfqpoint{3.165030in}{0.845636in}}{\pgfqpoint{3.167225in}{0.840336in}}{\pgfqpoint{3.171132in}{0.836429in}}%
\pgfpathcurveto{\pgfqpoint{3.175039in}{0.832523in}}{\pgfqpoint{3.180338in}{0.830327in}}{\pgfqpoint{3.185863in}{0.830327in}}%
\pgfpathclose%
\pgfusepath{fill}%
\end{pgfscope}%
\begin{pgfscope}%
\pgfpathrectangle{\pgfqpoint{3.158185in}{0.833185in}}{\pgfqpoint{1.162500in}{0.755000in}} %
\pgfusepath{clip}%
\pgfsetbuttcap%
\pgfsetroundjoin%
\definecolor{currentfill}{rgb}{0.000000,0.000000,0.000000}%
\pgfsetfillcolor{currentfill}%
\pgfsetfillopacity{0.500000}%
\pgfsetlinewidth{0.000000pt}%
\definecolor{currentstroke}{rgb}{0.000000,0.000000,0.000000}%
\pgfsetstrokecolor{currentstroke}%
\pgfsetdash{}{0pt}%
\pgfpathmoveto{\pgfqpoint{3.808562in}{1.469697in}}%
\pgfpathcurveto{\pgfqpoint{3.814087in}{1.469697in}}{\pgfqpoint{3.819386in}{1.471892in}}{\pgfqpoint{3.823293in}{1.475799in}}%
\pgfpathcurveto{\pgfqpoint{3.827200in}{1.479706in}}{\pgfqpoint{3.829395in}{1.485005in}}{\pgfqpoint{3.829395in}{1.490530in}}%
\pgfpathcurveto{\pgfqpoint{3.829395in}{1.496056in}}{\pgfqpoint{3.827200in}{1.501355in}}{\pgfqpoint{3.823293in}{1.505262in}}%
\pgfpathcurveto{\pgfqpoint{3.819386in}{1.509169in}}{\pgfqpoint{3.814087in}{1.511364in}}{\pgfqpoint{3.808562in}{1.511364in}}%
\pgfpathcurveto{\pgfqpoint{3.803037in}{1.511364in}}{\pgfqpoint{3.797737in}{1.509169in}}{\pgfqpoint{3.793830in}{1.505262in}}%
\pgfpathcurveto{\pgfqpoint{3.789924in}{1.501355in}}{\pgfqpoint{3.787728in}{1.496056in}}{\pgfqpoint{3.787728in}{1.490530in}}%
\pgfpathcurveto{\pgfqpoint{3.787728in}{1.485005in}}{\pgfqpoint{3.789924in}{1.479706in}}{\pgfqpoint{3.793830in}{1.475799in}}%
\pgfpathcurveto{\pgfqpoint{3.797737in}{1.471892in}}{\pgfqpoint{3.803037in}{1.469697in}}{\pgfqpoint{3.808562in}{1.469697in}}%
\pgfpathclose%
\pgfusepath{fill}%
\end{pgfscope}%
\begin{pgfscope}%
\pgfpathrectangle{\pgfqpoint{3.158185in}{0.833185in}}{\pgfqpoint{1.162500in}{0.755000in}} %
\pgfusepath{clip}%
\pgfsetbuttcap%
\pgfsetroundjoin%
\definecolor{currentfill}{rgb}{0.000000,0.000000,0.000000}%
\pgfsetfillcolor{currentfill}%
\pgfsetfillopacity{0.500000}%
\pgfsetlinewidth{0.000000pt}%
\definecolor{currentstroke}{rgb}{0.000000,0.000000,0.000000}%
\pgfsetstrokecolor{currentstroke}%
\pgfsetdash{}{0pt}%
\pgfpathmoveto{\pgfqpoint{3.620933in}{1.312260in}}%
\pgfpathcurveto{\pgfqpoint{3.626458in}{1.312260in}}{\pgfqpoint{3.631758in}{1.314455in}}{\pgfqpoint{3.635664in}{1.318362in}}%
\pgfpathcurveto{\pgfqpoint{3.639571in}{1.322269in}}{\pgfqpoint{3.641766in}{1.327569in}}{\pgfqpoint{3.641766in}{1.333094in}}%
\pgfpathcurveto{\pgfqpoint{3.641766in}{1.338619in}}{\pgfqpoint{3.639571in}{1.343918in}}{\pgfqpoint{3.635664in}{1.347825in}}%
\pgfpathcurveto{\pgfqpoint{3.631758in}{1.351732in}}{\pgfqpoint{3.626458in}{1.353927in}}{\pgfqpoint{3.620933in}{1.353927in}}%
\pgfpathcurveto{\pgfqpoint{3.615408in}{1.353927in}}{\pgfqpoint{3.610108in}{1.351732in}}{\pgfqpoint{3.606202in}{1.347825in}}%
\pgfpathcurveto{\pgfqpoint{3.602295in}{1.343918in}}{\pgfqpoint{3.600100in}{1.338619in}}{\pgfqpoint{3.600100in}{1.333094in}}%
\pgfpathcurveto{\pgfqpoint{3.600100in}{1.327569in}}{\pgfqpoint{3.602295in}{1.322269in}}{\pgfqpoint{3.606202in}{1.318362in}}%
\pgfpathcurveto{\pgfqpoint{3.610108in}{1.314455in}}{\pgfqpoint{3.615408in}{1.312260in}}{\pgfqpoint{3.620933in}{1.312260in}}%
\pgfpathclose%
\pgfusepath{fill}%
\end{pgfscope}%
\begin{pgfscope}%
\pgfpathrectangle{\pgfqpoint{3.158185in}{0.833185in}}{\pgfqpoint{1.162500in}{0.755000in}} %
\pgfusepath{clip}%
\pgfsetbuttcap%
\pgfsetroundjoin%
\definecolor{currentfill}{rgb}{0.000000,0.000000,0.000000}%
\pgfsetfillcolor{currentfill}%
\pgfsetfillopacity{0.500000}%
\pgfsetlinewidth{0.000000pt}%
\definecolor{currentstroke}{rgb}{0.000000,0.000000,0.000000}%
\pgfsetstrokecolor{currentstroke}%
\pgfsetdash{}{0pt}%
\pgfpathmoveto{\pgfqpoint{3.600191in}{1.134966in}}%
\pgfpathcurveto{\pgfqpoint{3.605716in}{1.134966in}}{\pgfqpoint{3.611016in}{1.137161in}}{\pgfqpoint{3.614922in}{1.141068in}}%
\pgfpathcurveto{\pgfqpoint{3.618829in}{1.144975in}}{\pgfqpoint{3.621024in}{1.150275in}}{\pgfqpoint{3.621024in}{1.155800in}}%
\pgfpathcurveto{\pgfqpoint{3.621024in}{1.161325in}}{\pgfqpoint{3.618829in}{1.166624in}}{\pgfqpoint{3.614922in}{1.170531in}}%
\pgfpathcurveto{\pgfqpoint{3.611016in}{1.174438in}}{\pgfqpoint{3.605716in}{1.176633in}}{\pgfqpoint{3.600191in}{1.176633in}}%
\pgfpathcurveto{\pgfqpoint{3.594666in}{1.176633in}}{\pgfqpoint{3.589366in}{1.174438in}}{\pgfqpoint{3.585460in}{1.170531in}}%
\pgfpathcurveto{\pgfqpoint{3.581553in}{1.166624in}}{\pgfqpoint{3.579358in}{1.161325in}}{\pgfqpoint{3.579358in}{1.155800in}}%
\pgfpathcurveto{\pgfqpoint{3.579358in}{1.150275in}}{\pgfqpoint{3.581553in}{1.144975in}}{\pgfqpoint{3.585460in}{1.141068in}}%
\pgfpathcurveto{\pgfqpoint{3.589366in}{1.137161in}}{\pgfqpoint{3.594666in}{1.134966in}}{\pgfqpoint{3.600191in}{1.134966in}}%
\pgfpathclose%
\pgfusepath{fill}%
\end{pgfscope}%
\begin{pgfscope}%
\pgfpathrectangle{\pgfqpoint{3.158185in}{0.833185in}}{\pgfqpoint{1.162500in}{0.755000in}} %
\pgfusepath{clip}%
\pgfsetbuttcap%
\pgfsetroundjoin%
\definecolor{currentfill}{rgb}{0.000000,0.000000,0.000000}%
\pgfsetfillcolor{currentfill}%
\pgfsetfillopacity{0.500000}%
\pgfsetlinewidth{0.000000pt}%
\definecolor{currentstroke}{rgb}{0.000000,0.000000,0.000000}%
\pgfsetstrokecolor{currentstroke}%
\pgfsetdash{}{0pt}%
\pgfpathmoveto{\pgfqpoint{3.899960in}{1.421242in}}%
\pgfpathcurveto{\pgfqpoint{3.905485in}{1.421242in}}{\pgfqpoint{3.910785in}{1.423437in}}{\pgfqpoint{3.914692in}{1.427344in}}%
\pgfpathcurveto{\pgfqpoint{3.918598in}{1.431251in}}{\pgfqpoint{3.920794in}{1.436550in}}{\pgfqpoint{3.920794in}{1.442075in}}%
\pgfpathcurveto{\pgfqpoint{3.920794in}{1.447600in}}{\pgfqpoint{3.918598in}{1.452900in}}{\pgfqpoint{3.914692in}{1.456807in}}%
\pgfpathcurveto{\pgfqpoint{3.910785in}{1.460714in}}{\pgfqpoint{3.905485in}{1.462909in}}{\pgfqpoint{3.899960in}{1.462909in}}%
\pgfpathcurveto{\pgfqpoint{3.894435in}{1.462909in}}{\pgfqpoint{3.889136in}{1.460714in}}{\pgfqpoint{3.885229in}{1.456807in}}%
\pgfpathcurveto{\pgfqpoint{3.881322in}{1.452900in}}{\pgfqpoint{3.879127in}{1.447600in}}{\pgfqpoint{3.879127in}{1.442075in}}%
\pgfpathcurveto{\pgfqpoint{3.879127in}{1.436550in}}{\pgfqpoint{3.881322in}{1.431251in}}{\pgfqpoint{3.885229in}{1.427344in}}%
\pgfpathcurveto{\pgfqpoint{3.889136in}{1.423437in}}{\pgfqpoint{3.894435in}{1.421242in}}{\pgfqpoint{3.899960in}{1.421242in}}%
\pgfpathclose%
\pgfusepath{fill}%
\end{pgfscope}%
\begin{pgfscope}%
\pgfpathrectangle{\pgfqpoint{3.158185in}{0.833185in}}{\pgfqpoint{1.162500in}{0.755000in}} %
\pgfusepath{clip}%
\pgfsetbuttcap%
\pgfsetroundjoin%
\definecolor{currentfill}{rgb}{0.000000,0.000000,0.000000}%
\pgfsetfillcolor{currentfill}%
\pgfsetfillopacity{0.500000}%
\pgfsetlinewidth{0.000000pt}%
\definecolor{currentstroke}{rgb}{0.000000,0.000000,0.000000}%
\pgfsetstrokecolor{currentstroke}%
\pgfsetdash{}{0pt}%
\pgfpathmoveto{\pgfqpoint{3.368320in}{1.126134in}}%
\pgfpathcurveto{\pgfqpoint{3.373845in}{1.126134in}}{\pgfqpoint{3.379145in}{1.128329in}}{\pgfqpoint{3.383052in}{1.132236in}}%
\pgfpathcurveto{\pgfqpoint{3.386958in}{1.136143in}}{\pgfqpoint{3.389154in}{1.141442in}}{\pgfqpoint{3.389154in}{1.146967in}}%
\pgfpathcurveto{\pgfqpoint{3.389154in}{1.152492in}}{\pgfqpoint{3.386958in}{1.157792in}}{\pgfqpoint{3.383052in}{1.161699in}}%
\pgfpathcurveto{\pgfqpoint{3.379145in}{1.165606in}}{\pgfqpoint{3.373845in}{1.167801in}}{\pgfqpoint{3.368320in}{1.167801in}}%
\pgfpathcurveto{\pgfqpoint{3.362795in}{1.167801in}}{\pgfqpoint{3.357496in}{1.165606in}}{\pgfqpoint{3.353589in}{1.161699in}}%
\pgfpathcurveto{\pgfqpoint{3.349682in}{1.157792in}}{\pgfqpoint{3.347487in}{1.152492in}}{\pgfqpoint{3.347487in}{1.146967in}}%
\pgfpathcurveto{\pgfqpoint{3.347487in}{1.141442in}}{\pgfqpoint{3.349682in}{1.136143in}}{\pgfqpoint{3.353589in}{1.132236in}}%
\pgfpathcurveto{\pgfqpoint{3.357496in}{1.128329in}}{\pgfqpoint{3.362795in}{1.126134in}}{\pgfqpoint{3.368320in}{1.126134in}}%
\pgfpathclose%
\pgfusepath{fill}%
\end{pgfscope}%
\begin{pgfscope}%
\pgfpathrectangle{\pgfqpoint{3.158185in}{0.833185in}}{\pgfqpoint{1.162500in}{0.755000in}} %
\pgfusepath{clip}%
\pgfsetbuttcap%
\pgfsetroundjoin%
\definecolor{currentfill}{rgb}{0.000000,0.000000,0.000000}%
\pgfsetfillcolor{currentfill}%
\pgfsetfillopacity{0.500000}%
\pgfsetlinewidth{0.000000pt}%
\definecolor{currentstroke}{rgb}{0.000000,0.000000,0.000000}%
\pgfsetstrokecolor{currentstroke}%
\pgfsetdash{}{0pt}%
\pgfpathmoveto{\pgfqpoint{3.416175in}{0.888020in}}%
\pgfpathcurveto{\pgfqpoint{3.421700in}{0.888020in}}{\pgfqpoint{3.427000in}{0.890215in}}{\pgfqpoint{3.430907in}{0.894122in}}%
\pgfpathcurveto{\pgfqpoint{3.434814in}{0.898028in}}{\pgfqpoint{3.437009in}{0.903328in}}{\pgfqpoint{3.437009in}{0.908853in}}%
\pgfpathcurveto{\pgfqpoint{3.437009in}{0.914378in}}{\pgfqpoint{3.434814in}{0.919677in}}{\pgfqpoint{3.430907in}{0.923584in}}%
\pgfpathcurveto{\pgfqpoint{3.427000in}{0.927491in}}{\pgfqpoint{3.421700in}{0.929686in}}{\pgfqpoint{3.416175in}{0.929686in}}%
\pgfpathcurveto{\pgfqpoint{3.410650in}{0.929686in}}{\pgfqpoint{3.405351in}{0.927491in}}{\pgfqpoint{3.401444in}{0.923584in}}%
\pgfpathcurveto{\pgfqpoint{3.397537in}{0.919677in}}{\pgfqpoint{3.395342in}{0.914378in}}{\pgfqpoint{3.395342in}{0.908853in}}%
\pgfpathcurveto{\pgfqpoint{3.395342in}{0.903328in}}{\pgfqpoint{3.397537in}{0.898028in}}{\pgfqpoint{3.401444in}{0.894122in}}%
\pgfpathcurveto{\pgfqpoint{3.405351in}{0.890215in}}{\pgfqpoint{3.410650in}{0.888020in}}{\pgfqpoint{3.416175in}{0.888020in}}%
\pgfpathclose%
\pgfusepath{fill}%
\end{pgfscope}%
\begin{pgfscope}%
\pgfpathrectangle{\pgfqpoint{3.158185in}{0.833185in}}{\pgfqpoint{1.162500in}{0.755000in}} %
\pgfusepath{clip}%
\pgfsetbuttcap%
\pgfsetroundjoin%
\definecolor{currentfill}{rgb}{0.000000,0.000000,0.000000}%
\pgfsetfillcolor{currentfill}%
\pgfsetfillopacity{0.500000}%
\pgfsetlinewidth{0.000000pt}%
\definecolor{currentstroke}{rgb}{0.000000,0.000000,0.000000}%
\pgfsetstrokecolor{currentstroke}%
\pgfsetdash{}{0pt}%
\pgfpathmoveto{\pgfqpoint{3.789669in}{1.424328in}}%
\pgfpathcurveto{\pgfqpoint{3.795194in}{1.424328in}}{\pgfqpoint{3.800493in}{1.426523in}}{\pgfqpoint{3.804400in}{1.430430in}}%
\pgfpathcurveto{\pgfqpoint{3.808307in}{1.434337in}}{\pgfqpoint{3.810502in}{1.439636in}}{\pgfqpoint{3.810502in}{1.445161in}}%
\pgfpathcurveto{\pgfqpoint{3.810502in}{1.450687in}}{\pgfqpoint{3.808307in}{1.455986in}}{\pgfqpoint{3.804400in}{1.459893in}}%
\pgfpathcurveto{\pgfqpoint{3.800493in}{1.463800in}}{\pgfqpoint{3.795194in}{1.465995in}}{\pgfqpoint{3.789669in}{1.465995in}}%
\pgfpathcurveto{\pgfqpoint{3.784143in}{1.465995in}}{\pgfqpoint{3.778844in}{1.463800in}}{\pgfqpoint{3.774937in}{1.459893in}}%
\pgfpathcurveto{\pgfqpoint{3.771030in}{1.455986in}}{\pgfqpoint{3.768835in}{1.450687in}}{\pgfqpoint{3.768835in}{1.445161in}}%
\pgfpathcurveto{\pgfqpoint{3.768835in}{1.439636in}}{\pgfqpoint{3.771030in}{1.434337in}}{\pgfqpoint{3.774937in}{1.430430in}}%
\pgfpathcurveto{\pgfqpoint{3.778844in}{1.426523in}}{\pgfqpoint{3.784143in}{1.424328in}}{\pgfqpoint{3.789669in}{1.424328in}}%
\pgfpathclose%
\pgfusepath{fill}%
\end{pgfscope}%
\begin{pgfscope}%
\pgfpathrectangle{\pgfqpoint{3.158185in}{0.833185in}}{\pgfqpoint{1.162500in}{0.755000in}} %
\pgfusepath{clip}%
\pgfsetbuttcap%
\pgfsetroundjoin%
\definecolor{currentfill}{rgb}{0.000000,0.000000,0.000000}%
\pgfsetfillcolor{currentfill}%
\pgfsetfillopacity{0.500000}%
\pgfsetlinewidth{0.000000pt}%
\definecolor{currentstroke}{rgb}{0.000000,0.000000,0.000000}%
\pgfsetstrokecolor{currentstroke}%
\pgfsetdash{}{0pt}%
\pgfpathmoveto{\pgfqpoint{3.231064in}{1.186596in}}%
\pgfpathcurveto{\pgfqpoint{3.236589in}{1.186596in}}{\pgfqpoint{3.241888in}{1.188791in}}{\pgfqpoint{3.245795in}{1.192698in}}%
\pgfpathcurveto{\pgfqpoint{3.249702in}{1.196604in}}{\pgfqpoint{3.251897in}{1.201904in}}{\pgfqpoint{3.251897in}{1.207429in}}%
\pgfpathcurveto{\pgfqpoint{3.251897in}{1.212954in}}{\pgfqpoint{3.249702in}{1.218254in}}{\pgfqpoint{3.245795in}{1.222160in}}%
\pgfpathcurveto{\pgfqpoint{3.241888in}{1.226067in}}{\pgfqpoint{3.236589in}{1.228262in}}{\pgfqpoint{3.231064in}{1.228262in}}%
\pgfpathcurveto{\pgfqpoint{3.225539in}{1.228262in}}{\pgfqpoint{3.220239in}{1.226067in}}{\pgfqpoint{3.216332in}{1.222160in}}%
\pgfpathcurveto{\pgfqpoint{3.212425in}{1.218254in}}{\pgfqpoint{3.210230in}{1.212954in}}{\pgfqpoint{3.210230in}{1.207429in}}%
\pgfpathcurveto{\pgfqpoint{3.210230in}{1.201904in}}{\pgfqpoint{3.212425in}{1.196604in}}{\pgfqpoint{3.216332in}{1.192698in}}%
\pgfpathcurveto{\pgfqpoint{3.220239in}{1.188791in}}{\pgfqpoint{3.225539in}{1.186596in}}{\pgfqpoint{3.231064in}{1.186596in}}%
\pgfpathclose%
\pgfusepath{fill}%
\end{pgfscope}%
\begin{pgfscope}%
\pgfsetbuttcap%
\pgfsetroundjoin%
\definecolor{currentfill}{rgb}{0.000000,0.000000,0.000000}%
\pgfsetfillcolor{currentfill}%
\pgfsetlinewidth{0.803000pt}%
\definecolor{currentstroke}{rgb}{0.000000,0.000000,0.000000}%
\pgfsetstrokecolor{currentstroke}%
\pgfsetdash{}{0pt}%
\pgfsys@defobject{currentmarker}{\pgfqpoint{0.000000in}{-0.048611in}}{\pgfqpoint{0.000000in}{0.000000in}}{%
\pgfpathmoveto{\pgfqpoint{0.000000in}{0.000000in}}%
\pgfpathlineto{\pgfqpoint{0.000000in}{-0.048611in}}%
\pgfusepath{stroke,fill}%
}%
\begin{pgfscope}%
\pgfsys@transformshift{3.497383in}{0.833185in}%
\pgfsys@useobject{currentmarker}{}%
\end{pgfscope}%
\end{pgfscope}%
\begin{pgfscope}%
\pgftext[x=3.528037in,y=0.585111in,left,base,rotate=90.000000]{\rmfamily\fontsize{8.000000}{9.600000}\selectfont \(\displaystyle 2.5\)}%
\end{pgfscope}%
\begin{pgfscope}%
\pgfsetbuttcap%
\pgfsetroundjoin%
\definecolor{currentfill}{rgb}{0.000000,0.000000,0.000000}%
\pgfsetfillcolor{currentfill}%
\pgfsetlinewidth{0.803000pt}%
\definecolor{currentstroke}{rgb}{0.000000,0.000000,0.000000}%
\pgfsetstrokecolor{currentstroke}%
\pgfsetdash{}{0pt}%
\pgfsys@defobject{currentmarker}{\pgfqpoint{0.000000in}{-0.048611in}}{\pgfqpoint{0.000000in}{0.000000in}}{%
\pgfpathmoveto{\pgfqpoint{0.000000in}{0.000000in}}%
\pgfpathlineto{\pgfqpoint{0.000000in}{-0.048611in}}%
\pgfusepath{stroke,fill}%
}%
\begin{pgfscope}%
\pgfsys@transformshift{3.860784in}{0.833185in}%
\pgfsys@useobject{currentmarker}{}%
\end{pgfscope}%
\end{pgfscope}%
\begin{pgfscope}%
\pgftext[x=3.891437in,y=0.585111in,left,base,rotate=90.000000]{\rmfamily\fontsize{8.000000}{9.600000}\selectfont \(\displaystyle 5.0\)}%
\end{pgfscope}%
\begin{pgfscope}%
\pgfsetbuttcap%
\pgfsetroundjoin%
\definecolor{currentfill}{rgb}{0.000000,0.000000,0.000000}%
\pgfsetfillcolor{currentfill}%
\pgfsetlinewidth{0.803000pt}%
\definecolor{currentstroke}{rgb}{0.000000,0.000000,0.000000}%
\pgfsetstrokecolor{currentstroke}%
\pgfsetdash{}{0pt}%
\pgfsys@defobject{currentmarker}{\pgfqpoint{0.000000in}{-0.048611in}}{\pgfqpoint{0.000000in}{0.000000in}}{%
\pgfpathmoveto{\pgfqpoint{0.000000in}{0.000000in}}%
\pgfpathlineto{\pgfqpoint{0.000000in}{-0.048611in}}%
\pgfusepath{stroke,fill}%
}%
\begin{pgfscope}%
\pgfsys@transformshift{4.224184in}{0.833185in}%
\pgfsys@useobject{currentmarker}{}%
\end{pgfscope}%
\end{pgfscope}%
\begin{pgfscope}%
\pgftext[x=4.254838in,y=0.585111in,left,base,rotate=90.000000]{\rmfamily\fontsize{8.000000}{9.600000}\selectfont \(\displaystyle 7.5\)}%
\end{pgfscope}%
\begin{pgfscope}%
\pgftext[x=3.739435in,y=0.529556in,,top]{\rmfamily\fontsize{10.000000}{12.000000}\selectfont charge}%
\end{pgfscope}%
\begin{pgfscope}%
\pgftext[x=4.320685in,y=0.543445in,right,top]{\rmfamily\fontsize{10.000000}{12.000000}\selectfont \(\displaystyle \times10^{-10}\)}%
\end{pgfscope}%
\begin{pgfscope}%
\pgfsetrectcap%
\pgfsetmiterjoin%
\pgfsetlinewidth{0.803000pt}%
\definecolor{currentstroke}{rgb}{0.000000,0.000000,0.000000}%
\pgfsetstrokecolor{currentstroke}%
\pgfsetdash{}{0pt}%
\pgfpathmoveto{\pgfqpoint{3.158185in}{0.833185in}}%
\pgfpathlineto{\pgfqpoint{3.158185in}{1.588185in}}%
\pgfusepath{stroke}%
\end{pgfscope}%
\begin{pgfscope}%
\pgfsetrectcap%
\pgfsetmiterjoin%
\pgfsetlinewidth{0.803000pt}%
\definecolor{currentstroke}{rgb}{0.000000,0.000000,0.000000}%
\pgfsetstrokecolor{currentstroke}%
\pgfsetdash{}{0pt}%
\pgfpathmoveto{\pgfqpoint{4.320685in}{0.833185in}}%
\pgfpathlineto{\pgfqpoint{4.320685in}{1.588185in}}%
\pgfusepath{stroke}%
\end{pgfscope}%
\begin{pgfscope}%
\pgfsetrectcap%
\pgfsetmiterjoin%
\pgfsetlinewidth{0.803000pt}%
\definecolor{currentstroke}{rgb}{0.000000,0.000000,0.000000}%
\pgfsetstrokecolor{currentstroke}%
\pgfsetdash{}{0pt}%
\pgfpathmoveto{\pgfqpoint{3.158185in}{0.833185in}}%
\pgfpathlineto{\pgfqpoint{4.320685in}{0.833185in}}%
\pgfusepath{stroke}%
\end{pgfscope}%
\begin{pgfscope}%
\pgfsetrectcap%
\pgfsetmiterjoin%
\pgfsetlinewidth{0.803000pt}%
\definecolor{currentstroke}{rgb}{0.000000,0.000000,0.000000}%
\pgfsetstrokecolor{currentstroke}%
\pgfsetdash{}{0pt}%
\pgfpathmoveto{\pgfqpoint{3.158185in}{1.588185in}}%
\pgfpathlineto{\pgfqpoint{4.320685in}{1.588185in}}%
\pgfusepath{stroke}%
\end{pgfscope}%
\begin{pgfscope}%
\pgfsetbuttcap%
\pgfsetmiterjoin%
\definecolor{currentfill}{rgb}{1.000000,1.000000,1.000000}%
\pgfsetfillcolor{currentfill}%
\pgfsetlinewidth{0.000000pt}%
\definecolor{currentstroke}{rgb}{0.000000,0.000000,0.000000}%
\pgfsetstrokecolor{currentstroke}%
\pgfsetstrokeopacity{0.000000}%
\pgfsetdash{}{0pt}%
\pgfpathmoveto{\pgfqpoint{4.320685in}{0.833185in}}%
\pgfpathlineto{\pgfqpoint{5.483185in}{0.833185in}}%
\pgfpathlineto{\pgfqpoint{5.483185in}{1.588185in}}%
\pgfpathlineto{\pgfqpoint{4.320685in}{1.588185in}}%
\pgfpathclose%
\pgfusepath{fill}%
\end{pgfscope}%
\begin{pgfscope}%
\pgfsetbuttcap%
\pgfsetroundjoin%
\definecolor{currentfill}{rgb}{0.000000,0.000000,0.000000}%
\pgfsetfillcolor{currentfill}%
\pgfsetlinewidth{0.803000pt}%
\definecolor{currentstroke}{rgb}{0.000000,0.000000,0.000000}%
\pgfsetstrokecolor{currentstroke}%
\pgfsetdash{}{0pt}%
\pgfsys@defobject{currentmarker}{\pgfqpoint{0.000000in}{-0.048611in}}{\pgfqpoint{0.000000in}{0.000000in}}{%
\pgfpathmoveto{\pgfqpoint{0.000000in}{0.000000in}}%
\pgfpathlineto{\pgfqpoint{0.000000in}{-0.048611in}}%
\pgfusepath{stroke,fill}%
}%
\begin{pgfscope}%
\pgfsys@transformshift{4.447086in}{0.833185in}%
\pgfsys@useobject{currentmarker}{}%
\end{pgfscope}%
\end{pgfscope}%
\begin{pgfscope}%
\pgftext[x=4.477739in,y=0.467054in,left,base,rotate=90.000000]{\rmfamily\fontsize{8.000000}{9.600000}\selectfont \(\displaystyle 0.025\)}%
\end{pgfscope}%
\begin{pgfscope}%
\pgfsetbuttcap%
\pgfsetroundjoin%
\definecolor{currentfill}{rgb}{0.000000,0.000000,0.000000}%
\pgfsetfillcolor{currentfill}%
\pgfsetlinewidth{0.803000pt}%
\definecolor{currentstroke}{rgb}{0.000000,0.000000,0.000000}%
\pgfsetstrokecolor{currentstroke}%
\pgfsetdash{}{0pt}%
\pgfsys@defobject{currentmarker}{\pgfqpoint{0.000000in}{-0.048611in}}{\pgfqpoint{0.000000in}{0.000000in}}{%
\pgfpathmoveto{\pgfqpoint{0.000000in}{0.000000in}}%
\pgfpathlineto{\pgfqpoint{0.000000in}{-0.048611in}}%
\pgfusepath{stroke,fill}%
}%
\begin{pgfscope}%
\pgfsys@transformshift{4.854077in}{0.833185in}%
\pgfsys@useobject{currentmarker}{}%
\end{pgfscope}%
\end{pgfscope}%
\begin{pgfscope}%
\pgftext[x=4.884731in,y=0.467054in,left,base,rotate=90.000000]{\rmfamily\fontsize{8.000000}{9.600000}\selectfont \(\displaystyle 0.050\)}%
\end{pgfscope}%
\begin{pgfscope}%
\pgfsetbuttcap%
\pgfsetroundjoin%
\definecolor{currentfill}{rgb}{0.000000,0.000000,0.000000}%
\pgfsetfillcolor{currentfill}%
\pgfsetlinewidth{0.803000pt}%
\definecolor{currentstroke}{rgb}{0.000000,0.000000,0.000000}%
\pgfsetstrokecolor{currentstroke}%
\pgfsetdash{}{0pt}%
\pgfsys@defobject{currentmarker}{\pgfqpoint{0.000000in}{-0.048611in}}{\pgfqpoint{0.000000in}{0.000000in}}{%
\pgfpathmoveto{\pgfqpoint{0.000000in}{0.000000in}}%
\pgfpathlineto{\pgfqpoint{0.000000in}{-0.048611in}}%
\pgfusepath{stroke,fill}%
}%
\begin{pgfscope}%
\pgfsys@transformshift{5.261069in}{0.833185in}%
\pgfsys@useobject{currentmarker}{}%
\end{pgfscope}%
\end{pgfscope}%
\begin{pgfscope}%
\pgftext[x=5.291722in,y=0.467054in,left,base,rotate=90.000000]{\rmfamily\fontsize{8.000000}{9.600000}\selectfont \(\displaystyle 0.075\)}%
\end{pgfscope}%
\begin{pgfscope}%
\pgftext[x=4.901935in,y=0.411499in,,top]{\rmfamily\fontsize{10.000000}{12.000000}\selectfont u0}%
\end{pgfscope}%
\begin{pgfscope}%
\pgfpathrectangle{\pgfqpoint{4.320685in}{0.833185in}}{\pgfqpoint{1.162500in}{0.755000in}} %
\pgfusepath{clip}%
\pgfsetrectcap%
\pgfsetroundjoin%
\pgfsetlinewidth{1.505625pt}%
\definecolor{currentstroke}{rgb}{0.121569,0.466667,0.705882}%
\pgfsetstrokecolor{currentstroke}%
\pgfsetdash{}{0pt}%
\pgfpathmoveto{\pgfqpoint{4.348363in}{0.867503in}}%
\pgfpathlineto{\pgfqpoint{4.384935in}{0.889698in}}%
\pgfpathlineto{\pgfqpoint{4.438131in}{0.918946in}}%
\pgfpathlineto{\pgfqpoint{4.481353in}{0.943686in}}%
\pgfpathlineto{\pgfqpoint{4.509060in}{0.962052in}}%
\pgfpathlineto{\pgfqpoint{4.533441in}{0.980861in}}%
\pgfpathlineto{\pgfqpoint{4.556714in}{1.001727in}}%
\pgfpathlineto{\pgfqpoint{4.578879in}{1.024610in}}%
\pgfpathlineto{\pgfqpoint{4.602153in}{1.051996in}}%
\pgfpathlineto{\pgfqpoint{4.626534in}{1.084371in}}%
\pgfpathlineto{\pgfqpoint{4.653132in}{1.123666in}}%
\pgfpathlineto{\pgfqpoint{4.684163in}{1.173821in}}%
\pgfpathlineto{\pgfqpoint{4.728493in}{1.250270in}}%
\pgfpathlineto{\pgfqpoint{4.786122in}{1.348907in}}%
\pgfpathlineto{\pgfqpoint{4.816045in}{1.395699in}}%
\pgfpathlineto{\pgfqpoint{4.841535in}{1.431586in}}%
\pgfpathlineto{\pgfqpoint{4.864808in}{1.460534in}}%
\pgfpathlineto{\pgfqpoint{4.885865in}{1.483292in}}%
\pgfpathlineto{\pgfqpoint{4.905813in}{1.501731in}}%
\pgfpathlineto{\pgfqpoint{4.925762in}{1.517130in}}%
\pgfpathlineto{\pgfqpoint{4.944602in}{1.528953in}}%
\pgfpathlineto{\pgfqpoint{4.963443in}{1.538257in}}%
\pgfpathlineto{\pgfqpoint{4.982283in}{1.545206in}}%
\pgfpathlineto{\pgfqpoint{5.002231in}{1.550207in}}%
\pgfpathlineto{\pgfqpoint{5.023288in}{1.553112in}}%
\pgfpathlineto{\pgfqpoint{5.045453in}{1.553833in}}%
\pgfpathlineto{\pgfqpoint{5.068726in}{1.552312in}}%
\pgfpathlineto{\pgfqpoint{5.093108in}{1.548490in}}%
\pgfpathlineto{\pgfqpoint{5.118598in}{1.542249in}}%
\pgfpathlineto{\pgfqpoint{5.144087in}{1.533767in}}%
\pgfpathlineto{\pgfqpoint{5.169577in}{1.522946in}}%
\pgfpathlineto{\pgfqpoint{5.193959in}{1.510184in}}%
\pgfpathlineto{\pgfqpoint{5.217232in}{1.495522in}}%
\pgfpathlineto{\pgfqpoint{5.239397in}{1.479020in}}%
\pgfpathlineto{\pgfqpoint{5.261562in}{1.459760in}}%
\pgfpathlineto{\pgfqpoint{5.283727in}{1.437493in}}%
\pgfpathlineto{\pgfqpoint{5.305892in}{1.412020in}}%
\pgfpathlineto{\pgfqpoint{5.329165in}{1.381694in}}%
\pgfpathlineto{\pgfqpoint{5.352439in}{1.347687in}}%
\pgfpathlineto{\pgfqpoint{5.377928in}{1.306375in}}%
\pgfpathlineto{\pgfqpoint{5.405635in}{1.257087in}}%
\pgfpathlineto{\pgfqpoint{5.437774in}{1.195202in}}%
\pgfpathlineto{\pgfqpoint{5.455506in}{1.159442in}}%
\pgfpathlineto{\pgfqpoint{5.455506in}{1.159442in}}%
\pgfusepath{stroke}%
\end{pgfscope}%
\begin{pgfscope}%
\pgfsetrectcap%
\pgfsetmiterjoin%
\pgfsetlinewidth{0.803000pt}%
\definecolor{currentstroke}{rgb}{0.000000,0.000000,0.000000}%
\pgfsetstrokecolor{currentstroke}%
\pgfsetdash{}{0pt}%
\pgfpathmoveto{\pgfqpoint{4.320685in}{0.833185in}}%
\pgfpathlineto{\pgfqpoint{4.320685in}{1.588185in}}%
\pgfusepath{stroke}%
\end{pgfscope}%
\begin{pgfscope}%
\pgfsetrectcap%
\pgfsetmiterjoin%
\pgfsetlinewidth{0.803000pt}%
\definecolor{currentstroke}{rgb}{0.000000,0.000000,0.000000}%
\pgfsetstrokecolor{currentstroke}%
\pgfsetdash{}{0pt}%
\pgfpathmoveto{\pgfqpoint{5.483185in}{0.833185in}}%
\pgfpathlineto{\pgfqpoint{5.483185in}{1.588185in}}%
\pgfusepath{stroke}%
\end{pgfscope}%
\begin{pgfscope}%
\pgfsetrectcap%
\pgfsetmiterjoin%
\pgfsetlinewidth{0.803000pt}%
\definecolor{currentstroke}{rgb}{0.000000,0.000000,0.000000}%
\pgfsetstrokecolor{currentstroke}%
\pgfsetdash{}{0pt}%
\pgfpathmoveto{\pgfqpoint{4.320685in}{0.833185in}}%
\pgfpathlineto{\pgfqpoint{5.483185in}{0.833185in}}%
\pgfusepath{stroke}%
\end{pgfscope}%
\begin{pgfscope}%
\pgfsetrectcap%
\pgfsetmiterjoin%
\pgfsetlinewidth{0.803000pt}%
\definecolor{currentstroke}{rgb}{0.000000,0.000000,0.000000}%
\pgfsetstrokecolor{currentstroke}%
\pgfsetdash{}{0pt}%
\pgfpathmoveto{\pgfqpoint{4.320685in}{1.588185in}}%
\pgfpathlineto{\pgfqpoint{5.483185in}{1.588185in}}%
\pgfusepath{stroke}%
\end{pgfscope}%
\end{pgfpicture}%
\makeatother%
\endgroup%
}
    \caption{A simple EMA plot.\label{fig:scatter}}
\end{figure}
\begin{figure}[htb]
    \centering
    %% Creator: Matplotlib, PGF backend
%%
%% To include the figure in your LaTeX document, write
%%   \input{<filename>.pgf}
%%
%% Make sure the required packages are loaded in your preamble
%%   \usepackage{pgf}
%%
%% Figures using additional raster images can only be included by \input if
%% they are in the same directory as the main LaTeX file. For loading figures
%% from other directories you can use the `import` package
%%   \usepackage{import}
%% and then include the figures with
%%   \import{<path to file>}{<filename>.pgf}
%%
%% Matplotlib used the following preamble
%%   \usepackage{fontspec}
%%   \setmainfont{DejaVu Serif}
%%   \setsansfont{DejaVu Sans}
%%   \setmonofont{DejaVu Sans Mono}
%%
\begingroup%
\makeatletter%
\begin{pgfpicture}%
\pgfpathrectangle{\pgfpointorigin}{\pgfqpoint{5.194240in}{3.788793in}}%
\pgfusepath{use as bounding box, clip}%
\begin{pgfscope}%
\pgfsetbuttcap%
\pgfsetmiterjoin%
\definecolor{currentfill}{rgb}{1.000000,1.000000,1.000000}%
\pgfsetfillcolor{currentfill}%
\pgfsetlinewidth{0.000000pt}%
\definecolor{currentstroke}{rgb}{1.000000,1.000000,1.000000}%
\pgfsetstrokecolor{currentstroke}%
\pgfsetdash{}{0pt}%
\pgfpathmoveto{\pgfqpoint{0.000000in}{0.000000in}}%
\pgfpathlineto{\pgfqpoint{5.194240in}{0.000000in}}%
\pgfpathlineto{\pgfqpoint{5.194240in}{3.788793in}}%
\pgfpathlineto{\pgfqpoint{0.000000in}{3.788793in}}%
\pgfpathclose%
\pgfusepath{fill}%
\end{pgfscope}%
\begin{pgfscope}%
\pgfsetbuttcap%
\pgfsetmiterjoin%
\definecolor{currentfill}{rgb}{1.000000,1.000000,1.000000}%
\pgfsetfillcolor{currentfill}%
\pgfsetlinewidth{0.000000pt}%
\definecolor{currentstroke}{rgb}{0.000000,0.000000,0.000000}%
\pgfsetstrokecolor{currentstroke}%
\pgfsetstrokeopacity{0.000000}%
\pgfsetdash{}{0pt}%
\pgfpathmoveto{\pgfqpoint{0.634105in}{0.521603in}}%
\pgfpathlineto{\pgfqpoint{4.354105in}{0.521603in}}%
\pgfpathlineto{\pgfqpoint{4.354105in}{3.541603in}}%
\pgfpathlineto{\pgfqpoint{0.634105in}{3.541603in}}%
\pgfpathclose%
\pgfusepath{fill}%
\end{pgfscope}%
\begin{pgfscope}%
\pgfpathrectangle{\pgfqpoint{0.634105in}{0.521603in}}{\pgfqpoint{3.720000in}{3.020000in}} %
\pgfusepath{clip}%
\pgfsetbuttcap%
\pgfsetroundjoin%
\definecolor{currentfill}{rgb}{0.061765,0.061765,0.085934}%
\pgfsetfillcolor{currentfill}%
\pgfsetlinewidth{0.000000pt}%
\definecolor{currentstroke}{rgb}{0.000000,0.000000,0.000000}%
\pgfsetstrokecolor{currentstroke}%
\pgfsetdash{}{0pt}%
\pgfpathmoveto{\pgfqpoint{1.369836in}{0.740556in}}%
\pgfpathlineto{\pgfqpoint{1.437974in}{0.740556in}}%
\pgfpathlineto{\pgfqpoint{1.506111in}{0.740556in}}%
\pgfpathlineto{\pgfqpoint{1.574249in}{0.740556in}}%
\pgfpathlineto{\pgfqpoint{1.642386in}{0.740556in}}%
\pgfpathlineto{\pgfqpoint{1.710524in}{0.740556in}}%
\pgfpathlineto{\pgfqpoint{1.778661in}{0.740556in}}%
\pgfpathlineto{\pgfqpoint{1.846799in}{0.795494in}}%
\pgfpathlineto{\pgfqpoint{1.914936in}{0.795494in}}%
\pgfpathlineto{\pgfqpoint{1.983074in}{0.795494in}}%
\pgfpathlineto{\pgfqpoint{2.051211in}{0.795494in}}%
\pgfpathlineto{\pgfqpoint{2.119349in}{0.795494in}}%
\pgfpathlineto{\pgfqpoint{2.187486in}{0.795494in}}%
\pgfpathlineto{\pgfqpoint{2.255624in}{0.850432in}}%
\pgfpathlineto{\pgfqpoint{2.323761in}{0.905370in}}%
\pgfpathlineto{\pgfqpoint{2.337491in}{0.916441in}}%
\pgfpathlineto{\pgfqpoint{2.323761in}{0.911739in}}%
\pgfpathlineto{\pgfqpoint{2.269121in}{0.905370in}}%
\pgfpathlineto{\pgfqpoint{2.255624in}{0.903185in}}%
\pgfpathlineto{\pgfqpoint{2.247795in}{0.905370in}}%
\pgfpathlineto{\pgfqpoint{2.187486in}{0.927654in}}%
\pgfpathlineto{\pgfqpoint{2.119349in}{0.940331in}}%
\pgfpathlineto{\pgfqpoint{2.051211in}{0.952591in}}%
\pgfpathlineto{\pgfqpoint{2.005953in}{0.960308in}}%
\pgfpathlineto{\pgfqpoint{1.983074in}{0.964247in}}%
\pgfpathlineto{\pgfqpoint{1.914936in}{0.972393in}}%
\pgfpathlineto{\pgfqpoint{1.846799in}{0.980237in}}%
\pgfpathlineto{\pgfqpoint{1.778661in}{0.988081in}}%
\pgfpathlineto{\pgfqpoint{1.710524in}{0.995925in}}%
\pgfpathlineto{\pgfqpoint{1.642386in}{1.000048in}}%
\pgfpathlineto{\pgfqpoint{1.574249in}{0.991101in}}%
\pgfpathlineto{\pgfqpoint{1.506111in}{0.980260in}}%
\pgfpathlineto{\pgfqpoint{1.437974in}{0.966635in}}%
\pgfpathlineto{\pgfqpoint{1.428669in}{0.960308in}}%
\pgfpathlineto{\pgfqpoint{1.369836in}{0.923458in}}%
\pgfpathlineto{\pgfqpoint{1.301699in}{0.933018in}}%
\pgfpathlineto{\pgfqpoint{1.233561in}{0.942578in}}%
\pgfpathlineto{\pgfqpoint{1.165424in}{0.952138in}}%
\pgfpathlineto{\pgfqpoint{1.107195in}{0.960308in}}%
\pgfpathlineto{\pgfqpoint{1.097286in}{0.961699in}}%
\pgfpathlineto{\pgfqpoint{1.095198in}{0.961992in}}%
\pgfpathlineto{\pgfqpoint{1.097286in}{0.960308in}}%
\pgfpathlineto{\pgfqpoint{1.165424in}{0.905370in}}%
\pgfpathlineto{\pgfqpoint{1.233561in}{0.850432in}}%
\pgfpathlineto{\pgfqpoint{1.301699in}{0.795494in}}%
\pgfpathclose%
\pgfusepath{fill}%
\end{pgfscope}%
\begin{pgfscope}%
\pgfpathrectangle{\pgfqpoint{0.634105in}{0.521603in}}{\pgfqpoint{3.720000in}{3.020000in}} %
\pgfusepath{clip}%
\pgfsetbuttcap%
\pgfsetroundjoin%
\definecolor{currentfill}{rgb}{0.185294,0.185294,0.257801}%
\pgfsetfillcolor{currentfill}%
\pgfsetlinewidth{0.000000pt}%
\definecolor{currentstroke}{rgb}{0.000000,0.000000,0.000000}%
\pgfsetstrokecolor{currentstroke}%
\pgfsetdash{}{0pt}%
\pgfpathmoveto{\pgfqpoint{2.255624in}{0.903185in}}%
\pgfpathlineto{\pgfqpoint{2.269121in}{0.905370in}}%
\pgfpathlineto{\pgfqpoint{2.323761in}{0.911739in}}%
\pgfpathlineto{\pgfqpoint{2.337491in}{0.916441in}}%
\pgfpathlineto{\pgfqpoint{2.391899in}{0.960308in}}%
\pgfpathlineto{\pgfqpoint{2.460036in}{1.015247in}}%
\pgfpathlineto{\pgfqpoint{2.528174in}{1.070185in}}%
\pgfpathlineto{\pgfqpoint{2.596311in}{1.125123in}}%
\pgfpathlineto{\pgfqpoint{2.664449in}{1.180061in}}%
\pgfpathlineto{\pgfqpoint{2.732586in}{1.234999in}}%
\pgfpathlineto{\pgfqpoint{2.800724in}{1.289938in}}%
\pgfpathlineto{\pgfqpoint{2.828867in}{1.312630in}}%
\pgfpathlineto{\pgfqpoint{2.800724in}{1.297732in}}%
\pgfpathlineto{\pgfqpoint{2.785999in}{1.289938in}}%
\pgfpathlineto{\pgfqpoint{2.732586in}{1.261664in}}%
\pgfpathlineto{\pgfqpoint{2.682214in}{1.234999in}}%
\pgfpathlineto{\pgfqpoint{2.664449in}{1.225595in}}%
\pgfpathlineto{\pgfqpoint{2.596311in}{1.197344in}}%
\pgfpathlineto{\pgfqpoint{2.528174in}{1.206434in}}%
\pgfpathlineto{\pgfqpoint{2.484547in}{1.234999in}}%
\pgfpathlineto{\pgfqpoint{2.460036in}{1.247206in}}%
\pgfpathlineto{\pgfqpoint{2.391899in}{1.269643in}}%
\pgfpathlineto{\pgfqpoint{2.323761in}{1.288767in}}%
\pgfpathlineto{\pgfqpoint{2.319592in}{1.289938in}}%
\pgfpathlineto{\pgfqpoint{2.255624in}{1.307892in}}%
\pgfpathlineto{\pgfqpoint{2.187486in}{1.327016in}}%
\pgfpathlineto{\pgfqpoint{2.123852in}{1.344876in}}%
\pgfpathlineto{\pgfqpoint{2.119349in}{1.346235in}}%
\pgfpathlineto{\pgfqpoint{2.051211in}{1.367605in}}%
\pgfpathlineto{\pgfqpoint{1.983074in}{1.386557in}}%
\pgfpathlineto{\pgfqpoint{1.917821in}{1.399814in}}%
\pgfpathlineto{\pgfqpoint{1.914936in}{1.400400in}}%
\pgfpathlineto{\pgfqpoint{1.846799in}{1.414243in}}%
\pgfpathlineto{\pgfqpoint{1.778661in}{1.428085in}}%
\pgfpathlineto{\pgfqpoint{1.710524in}{1.441928in}}%
\pgfpathlineto{\pgfqpoint{1.666521in}{1.399814in}}%
\pgfpathlineto{\pgfqpoint{1.642386in}{1.374990in}}%
\pgfpathlineto{\pgfqpoint{1.574249in}{1.357931in}}%
\pgfpathlineto{\pgfqpoint{1.506111in}{1.347090in}}%
\pgfpathlineto{\pgfqpoint{1.492193in}{1.344876in}}%
\pgfpathlineto{\pgfqpoint{1.437974in}{1.336250in}}%
\pgfpathlineto{\pgfqpoint{1.369836in}{1.325409in}}%
\pgfpathlineto{\pgfqpoint{1.301699in}{1.291509in}}%
\pgfpathlineto{\pgfqpoint{1.233561in}{1.301069in}}%
\pgfpathlineto{\pgfqpoint{1.165424in}{1.310629in}}%
\pgfpathlineto{\pgfqpoint{1.097286in}{1.320189in}}%
\pgfpathlineto{\pgfqpoint{1.080739in}{1.344876in}}%
\pgfpathlineto{\pgfqpoint{1.053387in}{1.399814in}}%
\pgfpathlineto{\pgfqpoint{1.034280in}{1.454752in}}%
\pgfpathlineto{\pgfqpoint{1.029149in}{1.472995in}}%
\pgfpathlineto{\pgfqpoint{0.999191in}{1.509690in}}%
\pgfpathlineto{\pgfqpoint{0.961011in}{1.556918in}}%
\pgfpathlineto{\pgfqpoint{0.961011in}{1.509690in}}%
\pgfpathlineto{\pgfqpoint{0.961011in}{1.454752in}}%
\pgfpathlineto{\pgfqpoint{0.961011in}{1.399814in}}%
\pgfpathlineto{\pgfqpoint{1.029149in}{1.344876in}}%
\pgfpathlineto{\pgfqpoint{1.029149in}{1.289938in}}%
\pgfpathlineto{\pgfqpoint{1.029149in}{1.234999in}}%
\pgfpathlineto{\pgfqpoint{1.029149in}{1.180061in}}%
\pgfpathlineto{\pgfqpoint{1.029149in}{1.125123in}}%
\pgfpathlineto{\pgfqpoint{1.029149in}{1.070185in}}%
\pgfpathlineto{\pgfqpoint{1.029149in}{1.015247in}}%
\pgfpathlineto{\pgfqpoint{1.095198in}{0.961992in}}%
\pgfpathlineto{\pgfqpoint{1.097286in}{0.961699in}}%
\pgfpathlineto{\pgfqpoint{1.107195in}{0.960308in}}%
\pgfpathlineto{\pgfqpoint{1.165424in}{0.952138in}}%
\pgfpathlineto{\pgfqpoint{1.233561in}{0.942578in}}%
\pgfpathlineto{\pgfqpoint{1.301699in}{0.933018in}}%
\pgfpathlineto{\pgfqpoint{1.369836in}{0.923458in}}%
\pgfpathlineto{\pgfqpoint{1.428669in}{0.960308in}}%
\pgfpathlineto{\pgfqpoint{1.437974in}{0.966635in}}%
\pgfpathlineto{\pgfqpoint{1.506111in}{0.980260in}}%
\pgfpathlineto{\pgfqpoint{1.574249in}{0.991101in}}%
\pgfpathlineto{\pgfqpoint{1.642386in}{1.000048in}}%
\pgfpathlineto{\pgfqpoint{1.710524in}{0.995925in}}%
\pgfpathlineto{\pgfqpoint{1.778661in}{0.988081in}}%
\pgfpathlineto{\pgfqpoint{1.846799in}{0.980237in}}%
\pgfpathlineto{\pgfqpoint{1.914936in}{0.972393in}}%
\pgfpathlineto{\pgfqpoint{1.983074in}{0.964247in}}%
\pgfpathlineto{\pgfqpoint{2.005953in}{0.960308in}}%
\pgfpathlineto{\pgfqpoint{2.051211in}{0.952591in}}%
\pgfpathlineto{\pgfqpoint{2.119349in}{0.940331in}}%
\pgfpathlineto{\pgfqpoint{2.187486in}{0.927654in}}%
\pgfpathlineto{\pgfqpoint{2.247795in}{0.905370in}}%
\pgfpathclose%
\pgfusepath{fill}%
\end{pgfscope}%
\begin{pgfscope}%
\pgfpathrectangle{\pgfqpoint{0.634105in}{0.521603in}}{\pgfqpoint{3.720000in}{3.020000in}} %
\pgfusepath{clip}%
\pgfsetbuttcap%
\pgfsetroundjoin%
\definecolor{currentfill}{rgb}{0.312255,0.312255,0.434442}%
\pgfsetfillcolor{currentfill}%
\pgfsetlinewidth{0.000000pt}%
\definecolor{currentstroke}{rgb}{0.000000,0.000000,0.000000}%
\pgfsetstrokecolor{currentstroke}%
\pgfsetdash{}{0pt}%
\pgfpathmoveto{\pgfqpoint{2.528174in}{1.206434in}}%
\pgfpathlineto{\pgfqpoint{2.596311in}{1.197344in}}%
\pgfpathlineto{\pgfqpoint{2.664449in}{1.225595in}}%
\pgfpathlineto{\pgfqpoint{2.682214in}{1.234999in}}%
\pgfpathlineto{\pgfqpoint{2.732586in}{1.261664in}}%
\pgfpathlineto{\pgfqpoint{2.785999in}{1.289938in}}%
\pgfpathlineto{\pgfqpoint{2.800724in}{1.297732in}}%
\pgfpathlineto{\pgfqpoint{2.828867in}{1.312630in}}%
\pgfpathlineto{\pgfqpoint{2.868861in}{1.344876in}}%
\pgfpathlineto{\pgfqpoint{2.936999in}{1.399814in}}%
\pgfpathlineto{\pgfqpoint{3.005136in}{1.454752in}}%
\pgfpathlineto{\pgfqpoint{3.073274in}{1.509690in}}%
\pgfpathlineto{\pgfqpoint{3.141411in}{1.564629in}}%
\pgfpathlineto{\pgfqpoint{3.209549in}{1.619567in}}%
\pgfpathlineto{\pgfqpoint{3.277686in}{1.674505in}}%
\pgfpathlineto{\pgfqpoint{3.320244in}{1.708818in}}%
\pgfpathlineto{\pgfqpoint{3.277686in}{1.686784in}}%
\pgfpathlineto{\pgfqpoint{3.255421in}{1.674505in}}%
\pgfpathlineto{\pgfqpoint{3.209549in}{1.650223in}}%
\pgfpathlineto{\pgfqpoint{3.151636in}{1.619567in}}%
\pgfpathlineto{\pgfqpoint{3.141411in}{1.614155in}}%
\pgfpathlineto{\pgfqpoint{3.073274in}{1.579343in}}%
\pgfpathlineto{\pgfqpoint{3.047850in}{1.564629in}}%
\pgfpathlineto{\pgfqpoint{3.005136in}{1.542018in}}%
\pgfpathlineto{\pgfqpoint{2.944065in}{1.509690in}}%
\pgfpathlineto{\pgfqpoint{2.936999in}{1.505950in}}%
\pgfpathlineto{\pgfqpoint{2.868861in}{1.483166in}}%
\pgfpathlineto{\pgfqpoint{2.800724in}{1.494202in}}%
\pgfpathlineto{\pgfqpoint{2.750992in}{1.509690in}}%
\pgfpathlineto{\pgfqpoint{2.732586in}{1.516233in}}%
\pgfpathlineto{\pgfqpoint{2.664449in}{1.540451in}}%
\pgfpathlineto{\pgfqpoint{2.596426in}{1.564629in}}%
\pgfpathlineto{\pgfqpoint{2.596311in}{1.564669in}}%
\pgfpathlineto{\pgfqpoint{2.528174in}{1.588888in}}%
\pgfpathlineto{\pgfqpoint{2.460036in}{1.613106in}}%
\pgfpathlineto{\pgfqpoint{2.441859in}{1.619567in}}%
\pgfpathlineto{\pgfqpoint{2.391899in}{1.637324in}}%
\pgfpathlineto{\pgfqpoint{2.323761in}{1.661543in}}%
\pgfpathlineto{\pgfqpoint{2.287292in}{1.674505in}}%
\pgfpathlineto{\pgfqpoint{2.255624in}{1.685761in}}%
\pgfpathlineto{\pgfqpoint{2.187486in}{1.709979in}}%
\pgfpathlineto{\pgfqpoint{2.132725in}{1.729443in}}%
\pgfpathlineto{\pgfqpoint{2.119349in}{1.734198in}}%
\pgfpathlineto{\pgfqpoint{2.051211in}{1.758416in}}%
\pgfpathlineto{\pgfqpoint{1.983074in}{1.782635in}}%
\pgfpathlineto{\pgfqpoint{1.978159in}{1.784381in}}%
\pgfpathlineto{\pgfqpoint{1.914936in}{1.806853in}}%
\pgfpathlineto{\pgfqpoint{1.846799in}{1.831071in}}%
\pgfpathlineto{\pgfqpoint{1.823592in}{1.839320in}}%
\pgfpathlineto{\pgfqpoint{1.778661in}{1.855290in}}%
\pgfpathlineto{\pgfqpoint{1.710524in}{1.872116in}}%
\pgfpathlineto{\pgfqpoint{1.642386in}{1.885959in}}%
\pgfpathlineto{\pgfqpoint{1.601535in}{1.894258in}}%
\pgfpathlineto{\pgfqpoint{1.574249in}{1.899801in}}%
\pgfpathlineto{\pgfqpoint{1.561047in}{1.894258in}}%
\pgfpathlineto{\pgfqpoint{1.506111in}{1.848486in}}%
\pgfpathlineto{\pgfqpoint{1.502632in}{1.839320in}}%
\pgfpathlineto{\pgfqpoint{1.481783in}{1.784381in}}%
\pgfpathlineto{\pgfqpoint{1.460934in}{1.729443in}}%
\pgfpathlineto{\pgfqpoint{1.437974in}{1.703270in}}%
\pgfpathlineto{\pgfqpoint{1.369836in}{1.692239in}}%
\pgfpathlineto{\pgfqpoint{1.301699in}{1.681399in}}%
\pgfpathlineto{\pgfqpoint{1.280527in}{1.674505in}}%
\pgfpathlineto{\pgfqpoint{1.233561in}{1.659560in}}%
\pgfpathlineto{\pgfqpoint{1.165424in}{1.669120in}}%
\pgfpathlineto{\pgfqpoint{1.127043in}{1.674505in}}%
\pgfpathlineto{\pgfqpoint{1.097286in}{1.679372in}}%
\pgfpathlineto{\pgfqpoint{1.085455in}{1.729443in}}%
\pgfpathlineto{\pgfqpoint{1.076365in}{1.784381in}}%
\pgfpathlineto{\pgfqpoint{1.067275in}{1.839320in}}%
\pgfpathlineto{\pgfqpoint{1.058184in}{1.894258in}}%
\pgfpathlineto{\pgfqpoint{1.049094in}{1.949196in}}%
\pgfpathlineto{\pgfqpoint{1.040004in}{2.004134in}}%
\pgfpathlineto{\pgfqpoint{1.030914in}{2.059072in}}%
\pgfpathlineto{\pgfqpoint{1.029149in}{2.069741in}}%
\pgfpathlineto{\pgfqpoint{0.995383in}{2.114011in}}%
\pgfpathlineto{\pgfqpoint{0.961011in}{2.159517in}}%
\pgfpathlineto{\pgfqpoint{0.904418in}{2.168949in}}%
\pgfpathlineto{\pgfqpoint{0.892874in}{2.170759in}}%
\pgfpathlineto{\pgfqpoint{0.892874in}{2.168949in}}%
\pgfpathlineto{\pgfqpoint{0.892874in}{2.114011in}}%
\pgfpathlineto{\pgfqpoint{0.892874in}{2.059072in}}%
\pgfpathlineto{\pgfqpoint{0.892874in}{2.004134in}}%
\pgfpathlineto{\pgfqpoint{0.892874in}{1.949196in}}%
\pgfpathlineto{\pgfqpoint{0.892874in}{1.894258in}}%
\pgfpathlineto{\pgfqpoint{0.892874in}{1.839320in}}%
\pgfpathlineto{\pgfqpoint{0.961011in}{1.784381in}}%
\pgfpathlineto{\pgfqpoint{0.961011in}{1.729443in}}%
\pgfpathlineto{\pgfqpoint{0.961011in}{1.674505in}}%
\pgfpathlineto{\pgfqpoint{0.961011in}{1.619567in}}%
\pgfpathlineto{\pgfqpoint{0.961011in}{1.564629in}}%
\pgfpathlineto{\pgfqpoint{0.961011in}{1.556918in}}%
\pgfpathlineto{\pgfqpoint{0.999191in}{1.509690in}}%
\pgfpathlineto{\pgfqpoint{1.029149in}{1.472995in}}%
\pgfpathlineto{\pgfqpoint{1.034280in}{1.454752in}}%
\pgfpathlineto{\pgfqpoint{1.053387in}{1.399814in}}%
\pgfpathlineto{\pgfqpoint{1.080739in}{1.344876in}}%
\pgfpathlineto{\pgfqpoint{1.097286in}{1.320189in}}%
\pgfpathlineto{\pgfqpoint{1.165424in}{1.310629in}}%
\pgfpathlineto{\pgfqpoint{1.233561in}{1.301069in}}%
\pgfpathlineto{\pgfqpoint{1.301699in}{1.291509in}}%
\pgfpathlineto{\pgfqpoint{1.369836in}{1.325409in}}%
\pgfpathlineto{\pgfqpoint{1.437974in}{1.336250in}}%
\pgfpathlineto{\pgfqpoint{1.492193in}{1.344876in}}%
\pgfpathlineto{\pgfqpoint{1.506111in}{1.347090in}}%
\pgfpathlineto{\pgfqpoint{1.574249in}{1.357931in}}%
\pgfpathlineto{\pgfqpoint{1.642386in}{1.374990in}}%
\pgfpathlineto{\pgfqpoint{1.666521in}{1.399814in}}%
\pgfpathlineto{\pgfqpoint{1.710524in}{1.441928in}}%
\pgfpathlineto{\pgfqpoint{1.778661in}{1.428085in}}%
\pgfpathlineto{\pgfqpoint{1.846799in}{1.414243in}}%
\pgfpathlineto{\pgfqpoint{1.914936in}{1.400400in}}%
\pgfpathlineto{\pgfqpoint{1.917821in}{1.399814in}}%
\pgfpathlineto{\pgfqpoint{1.983074in}{1.386557in}}%
\pgfpathlineto{\pgfqpoint{2.051211in}{1.367605in}}%
\pgfpathlineto{\pgfqpoint{2.119349in}{1.346235in}}%
\pgfpathlineto{\pgfqpoint{2.123852in}{1.344876in}}%
\pgfpathlineto{\pgfqpoint{2.187486in}{1.327016in}}%
\pgfpathlineto{\pgfqpoint{2.255624in}{1.307892in}}%
\pgfpathlineto{\pgfqpoint{2.319592in}{1.289938in}}%
\pgfpathlineto{\pgfqpoint{2.323761in}{1.288767in}}%
\pgfpathlineto{\pgfqpoint{2.391899in}{1.269643in}}%
\pgfpathlineto{\pgfqpoint{2.460036in}{1.247206in}}%
\pgfpathlineto{\pgfqpoint{2.484547in}{1.234999in}}%
\pgfpathclose%
\pgfusepath{fill}%
\end{pgfscope}%
\begin{pgfscope}%
\pgfpathrectangle{\pgfqpoint{0.634105in}{0.521603in}}{\pgfqpoint{3.720000in}{3.020000in}} %
\pgfusepath{clip}%
\pgfsetbuttcap%
\pgfsetroundjoin%
\definecolor{currentfill}{rgb}{0.439216,0.484130,0.564216}%
\pgfsetfillcolor{currentfill}%
\pgfsetlinewidth{0.000000pt}%
\definecolor{currentstroke}{rgb}{0.000000,0.000000,0.000000}%
\pgfsetstrokecolor{currentstroke}%
\pgfsetdash{}{0pt}%
\pgfpathmoveto{\pgfqpoint{2.800724in}{1.494202in}}%
\pgfpathlineto{\pgfqpoint{2.868861in}{1.483166in}}%
\pgfpathlineto{\pgfqpoint{2.936999in}{1.505950in}}%
\pgfpathlineto{\pgfqpoint{2.944065in}{1.509690in}}%
\pgfpathlineto{\pgfqpoint{3.005136in}{1.542018in}}%
\pgfpathlineto{\pgfqpoint{3.047850in}{1.564629in}}%
\pgfpathlineto{\pgfqpoint{3.073274in}{1.579343in}}%
\pgfpathlineto{\pgfqpoint{3.141411in}{1.614155in}}%
\pgfpathlineto{\pgfqpoint{3.151636in}{1.619567in}}%
\pgfpathlineto{\pgfqpoint{3.209549in}{1.650223in}}%
\pgfpathlineto{\pgfqpoint{3.255421in}{1.674505in}}%
\pgfpathlineto{\pgfqpoint{3.277686in}{1.686784in}}%
\pgfpathlineto{\pgfqpoint{3.320244in}{1.708818in}}%
\pgfpathlineto{\pgfqpoint{3.345824in}{1.729443in}}%
\pgfpathlineto{\pgfqpoint{3.413961in}{1.784381in}}%
\pgfpathlineto{\pgfqpoint{3.482099in}{1.839320in}}%
\pgfpathlineto{\pgfqpoint{3.550236in}{1.894258in}}%
\pgfpathlineto{\pgfqpoint{3.550236in}{1.949196in}}%
\pgfpathlineto{\pgfqpoint{3.618374in}{2.004134in}}%
\pgfpathlineto{\pgfqpoint{3.686511in}{2.059072in}}%
\pgfpathlineto{\pgfqpoint{3.686511in}{2.114011in}}%
\pgfpathlineto{\pgfqpoint{3.754649in}{2.168949in}}%
\pgfpathlineto{\pgfqpoint{3.754649in}{2.201612in}}%
\pgfpathlineto{\pgfqpoint{3.707448in}{2.168949in}}%
\pgfpathlineto{\pgfqpoint{3.686511in}{2.154460in}}%
\pgfpathlineto{\pgfqpoint{3.628058in}{2.114011in}}%
\pgfpathlineto{\pgfqpoint{3.618374in}{2.107309in}}%
\pgfpathlineto{\pgfqpoint{3.550236in}{2.069188in}}%
\pgfpathlineto{\pgfqpoint{3.529745in}{2.059072in}}%
\pgfpathlineto{\pgfqpoint{3.482099in}{2.035551in}}%
\pgfpathlineto{\pgfqpoint{3.418459in}{2.004134in}}%
\pgfpathlineto{\pgfqpoint{3.413961in}{2.001914in}}%
\pgfpathlineto{\pgfqpoint{3.345824in}{1.968276in}}%
\pgfpathlineto{\pgfqpoint{3.307173in}{1.949196in}}%
\pgfpathlineto{\pgfqpoint{3.277686in}{1.934639in}}%
\pgfpathlineto{\pgfqpoint{3.209549in}{1.901002in}}%
\pgfpathlineto{\pgfqpoint{3.195887in}{1.894258in}}%
\pgfpathlineto{\pgfqpoint{3.141411in}{1.867365in}}%
\pgfpathlineto{\pgfqpoint{3.076025in}{1.839320in}}%
\pgfpathlineto{\pgfqpoint{3.073274in}{1.838037in}}%
\pgfpathlineto{\pgfqpoint{3.067192in}{1.839320in}}%
\pgfpathlineto{\pgfqpoint{3.005136in}{1.857538in}}%
\pgfpathlineto{\pgfqpoint{2.936999in}{1.877542in}}%
\pgfpathlineto{\pgfqpoint{2.880062in}{1.894258in}}%
\pgfpathlineto{\pgfqpoint{2.868861in}{1.897546in}}%
\pgfpathlineto{\pgfqpoint{2.800724in}{1.917550in}}%
\pgfpathlineto{\pgfqpoint{2.732586in}{1.937554in}}%
\pgfpathlineto{\pgfqpoint{2.692931in}{1.949196in}}%
\pgfpathlineto{\pgfqpoint{2.664449in}{1.957558in}}%
\pgfpathlineto{\pgfqpoint{2.596311in}{1.977562in}}%
\pgfpathlineto{\pgfqpoint{2.528174in}{1.997566in}}%
\pgfpathlineto{\pgfqpoint{2.505801in}{2.004134in}}%
\pgfpathlineto{\pgfqpoint{2.460036in}{2.017570in}}%
\pgfpathlineto{\pgfqpoint{2.391899in}{2.037574in}}%
\pgfpathlineto{\pgfqpoint{2.323761in}{2.057645in}}%
\pgfpathlineto{\pgfqpoint{2.319278in}{2.059072in}}%
\pgfpathlineto{\pgfqpoint{2.255624in}{2.079292in}}%
\pgfpathlineto{\pgfqpoint{2.187486in}{2.102019in}}%
\pgfpathlineto{\pgfqpoint{2.153747in}{2.114011in}}%
\pgfpathlineto{\pgfqpoint{2.119349in}{2.126237in}}%
\pgfpathlineto{\pgfqpoint{2.051211in}{2.150455in}}%
\pgfpathlineto{\pgfqpoint{1.999180in}{2.168949in}}%
\pgfpathlineto{\pgfqpoint{1.983074in}{2.174674in}}%
\pgfpathlineto{\pgfqpoint{1.914936in}{2.198892in}}%
\pgfpathlineto{\pgfqpoint{1.846799in}{2.223110in}}%
\pgfpathlineto{\pgfqpoint{1.844613in}{2.223887in}}%
\pgfpathlineto{\pgfqpoint{1.778661in}{2.247329in}}%
\pgfpathlineto{\pgfqpoint{1.710524in}{2.271547in}}%
\pgfpathlineto{\pgfqpoint{1.690047in}{2.278825in}}%
\pgfpathlineto{\pgfqpoint{1.642386in}{2.295765in}}%
\pgfpathlineto{\pgfqpoint{1.574249in}{2.319984in}}%
\pgfpathlineto{\pgfqpoint{1.533347in}{2.333763in}}%
\pgfpathlineto{\pgfqpoint{1.506111in}{2.343155in}}%
\pgfpathlineto{\pgfqpoint{1.437974in}{2.357675in}}%
\pgfpathlineto{\pgfqpoint{1.369836in}{2.342991in}}%
\pgfpathlineto{\pgfqpoint{1.366334in}{2.333763in}}%
\pgfpathlineto{\pgfqpoint{1.345485in}{2.278825in}}%
\pgfpathlineto{\pgfqpoint{1.324635in}{2.223887in}}%
\pgfpathlineto{\pgfqpoint{1.303786in}{2.168949in}}%
\pgfpathlineto{\pgfqpoint{1.301699in}{2.163449in}}%
\pgfpathlineto{\pgfqpoint{1.282936in}{2.114011in}}%
\pgfpathlineto{\pgfqpoint{1.255766in}{2.059072in}}%
\pgfpathlineto{\pgfqpoint{1.233561in}{2.037389in}}%
\pgfpathlineto{\pgfqpoint{1.165424in}{2.027611in}}%
\pgfpathlineto{\pgfqpoint{1.151034in}{2.059072in}}%
\pgfpathlineto{\pgfqpoint{1.130978in}{2.114011in}}%
\pgfpathlineto{\pgfqpoint{1.115692in}{2.168949in}}%
\pgfpathlineto{\pgfqpoint{1.104839in}{2.223887in}}%
\pgfpathlineto{\pgfqpoint{1.100966in}{2.278825in}}%
\pgfpathlineto{\pgfqpoint{1.097526in}{2.333763in}}%
\pgfpathlineto{\pgfqpoint{1.097286in}{2.337842in}}%
\pgfpathlineto{\pgfqpoint{1.092891in}{2.388702in}}%
\pgfpathlineto{\pgfqpoint{1.084182in}{2.443640in}}%
\pgfpathlineto{\pgfqpoint{1.062163in}{2.498578in}}%
\pgfpathlineto{\pgfqpoint{1.029149in}{2.526883in}}%
\pgfpathlineto{\pgfqpoint{1.029149in}{2.498578in}}%
\pgfpathlineto{\pgfqpoint{0.961011in}{2.443640in}}%
\pgfpathlineto{\pgfqpoint{0.961011in}{2.388702in}}%
\pgfpathlineto{\pgfqpoint{0.892874in}{2.333763in}}%
\pgfpathlineto{\pgfqpoint{0.892874in}{2.278825in}}%
\pgfpathlineto{\pgfqpoint{0.892874in}{2.223887in}}%
\pgfpathlineto{\pgfqpoint{0.892874in}{2.170759in}}%
\pgfpathlineto{\pgfqpoint{0.904418in}{2.168949in}}%
\pgfpathlineto{\pgfqpoint{0.961011in}{2.159517in}}%
\pgfpathlineto{\pgfqpoint{0.995383in}{2.114011in}}%
\pgfpathlineto{\pgfqpoint{1.029149in}{2.069741in}}%
\pgfpathlineto{\pgfqpoint{1.030914in}{2.059072in}}%
\pgfpathlineto{\pgfqpoint{1.040004in}{2.004134in}}%
\pgfpathlineto{\pgfqpoint{1.049094in}{1.949196in}}%
\pgfpathlineto{\pgfqpoint{1.058184in}{1.894258in}}%
\pgfpathlineto{\pgfqpoint{1.067275in}{1.839320in}}%
\pgfpathlineto{\pgfqpoint{1.076365in}{1.784381in}}%
\pgfpathlineto{\pgfqpoint{1.085455in}{1.729443in}}%
\pgfpathlineto{\pgfqpoint{1.097286in}{1.679372in}}%
\pgfpathlineto{\pgfqpoint{1.127043in}{1.674505in}}%
\pgfpathlineto{\pgfqpoint{1.165424in}{1.669120in}}%
\pgfpathlineto{\pgfqpoint{1.233561in}{1.659560in}}%
\pgfpathlineto{\pgfqpoint{1.280527in}{1.674505in}}%
\pgfpathlineto{\pgfqpoint{1.301699in}{1.681399in}}%
\pgfpathlineto{\pgfqpoint{1.369836in}{1.692239in}}%
\pgfpathlineto{\pgfqpoint{1.437974in}{1.703270in}}%
\pgfpathlineto{\pgfqpoint{1.460934in}{1.729443in}}%
\pgfpathlineto{\pgfqpoint{1.481783in}{1.784381in}}%
\pgfpathlineto{\pgfqpoint{1.502632in}{1.839320in}}%
\pgfpathlineto{\pgfqpoint{1.506111in}{1.848486in}}%
\pgfpathlineto{\pgfqpoint{1.561047in}{1.894258in}}%
\pgfpathlineto{\pgfqpoint{1.574249in}{1.899801in}}%
\pgfpathlineto{\pgfqpoint{1.601535in}{1.894258in}}%
\pgfpathlineto{\pgfqpoint{1.642386in}{1.885959in}}%
\pgfpathlineto{\pgfqpoint{1.710524in}{1.872116in}}%
\pgfpathlineto{\pgfqpoint{1.778661in}{1.855290in}}%
\pgfpathlineto{\pgfqpoint{1.823592in}{1.839320in}}%
\pgfpathlineto{\pgfqpoint{1.846799in}{1.831071in}}%
\pgfpathlineto{\pgfqpoint{1.914936in}{1.806853in}}%
\pgfpathlineto{\pgfqpoint{1.978159in}{1.784381in}}%
\pgfpathlineto{\pgfqpoint{1.983074in}{1.782635in}}%
\pgfpathlineto{\pgfqpoint{2.051211in}{1.758416in}}%
\pgfpathlineto{\pgfqpoint{2.119349in}{1.734198in}}%
\pgfpathlineto{\pgfqpoint{2.132725in}{1.729443in}}%
\pgfpathlineto{\pgfqpoint{2.187486in}{1.709979in}}%
\pgfpathlineto{\pgfqpoint{2.255624in}{1.685761in}}%
\pgfpathlineto{\pgfqpoint{2.287292in}{1.674505in}}%
\pgfpathlineto{\pgfqpoint{2.323761in}{1.661543in}}%
\pgfpathlineto{\pgfqpoint{2.391899in}{1.637324in}}%
\pgfpathlineto{\pgfqpoint{2.441859in}{1.619567in}}%
\pgfpathlineto{\pgfqpoint{2.460036in}{1.613106in}}%
\pgfpathlineto{\pgfqpoint{2.528174in}{1.588888in}}%
\pgfpathlineto{\pgfqpoint{2.596311in}{1.564669in}}%
\pgfpathlineto{\pgfqpoint{2.596426in}{1.564629in}}%
\pgfpathlineto{\pgfqpoint{2.664449in}{1.540451in}}%
\pgfpathlineto{\pgfqpoint{2.732586in}{1.516233in}}%
\pgfpathlineto{\pgfqpoint{2.750992in}{1.509690in}}%
\pgfpathclose%
\pgfusepath{fill}%
\end{pgfscope}%
\begin{pgfscope}%
\pgfpathrectangle{\pgfqpoint{0.634105in}{0.521603in}}{\pgfqpoint{3.720000in}{3.020000in}} %
\pgfusepath{clip}%
\pgfsetbuttcap%
\pgfsetroundjoin%
\definecolor{currentfill}{rgb}{0.562745,0.653983,0.687745}%
\pgfsetfillcolor{currentfill}%
\pgfsetlinewidth{0.000000pt}%
\definecolor{currentstroke}{rgb}{0.000000,0.000000,0.000000}%
\pgfsetstrokecolor{currentstroke}%
\pgfsetdash{}{0pt}%
\pgfpathmoveto{\pgfqpoint{3.073274in}{1.838037in}}%
\pgfpathlineto{\pgfqpoint{3.076025in}{1.839320in}}%
\pgfpathlineto{\pgfqpoint{3.141411in}{1.867365in}}%
\pgfpathlineto{\pgfqpoint{3.195887in}{1.894258in}}%
\pgfpathlineto{\pgfqpoint{3.209549in}{1.901002in}}%
\pgfpathlineto{\pgfqpoint{3.277686in}{1.934639in}}%
\pgfpathlineto{\pgfqpoint{3.307173in}{1.949196in}}%
\pgfpathlineto{\pgfqpoint{3.345824in}{1.968276in}}%
\pgfpathlineto{\pgfqpoint{3.413961in}{2.001914in}}%
\pgfpathlineto{\pgfqpoint{3.418459in}{2.004134in}}%
\pgfpathlineto{\pgfqpoint{3.482099in}{2.035551in}}%
\pgfpathlineto{\pgfqpoint{3.529745in}{2.059072in}}%
\pgfpathlineto{\pgfqpoint{3.550236in}{2.069188in}}%
\pgfpathlineto{\pgfqpoint{3.618374in}{2.107309in}}%
\pgfpathlineto{\pgfqpoint{3.628058in}{2.114011in}}%
\pgfpathlineto{\pgfqpoint{3.686511in}{2.154460in}}%
\pgfpathlineto{\pgfqpoint{3.707448in}{2.168949in}}%
\pgfpathlineto{\pgfqpoint{3.754649in}{2.201612in}}%
\pgfpathlineto{\pgfqpoint{3.754649in}{2.223887in}}%
\pgfpathlineto{\pgfqpoint{3.822786in}{2.278825in}}%
\pgfpathlineto{\pgfqpoint{3.890924in}{2.333763in}}%
\pgfpathlineto{\pgfqpoint{3.890924in}{2.388702in}}%
\pgfpathlineto{\pgfqpoint{3.959061in}{2.443640in}}%
\pgfpathlineto{\pgfqpoint{4.027199in}{2.498578in}}%
\pgfpathlineto{\pgfqpoint{4.027199in}{2.553516in}}%
\pgfpathlineto{\pgfqpoint{4.095336in}{2.608454in}}%
\pgfpathlineto{\pgfqpoint{4.095336in}{2.663393in}}%
\pgfpathlineto{\pgfqpoint{4.095336in}{2.718331in}}%
\pgfpathlineto{\pgfqpoint{4.095336in}{2.718774in}}%
\pgfpathlineto{\pgfqpoint{4.094696in}{2.718331in}}%
\pgfpathlineto{\pgfqpoint{4.027199in}{2.671622in}}%
\pgfpathlineto{\pgfqpoint{4.015306in}{2.663393in}}%
\pgfpathlineto{\pgfqpoint{3.959061in}{2.624471in}}%
\pgfpathlineto{\pgfqpoint{3.935916in}{2.608454in}}%
\pgfpathlineto{\pgfqpoint{3.890924in}{2.577320in}}%
\pgfpathlineto{\pgfqpoint{3.856525in}{2.553516in}}%
\pgfpathlineto{\pgfqpoint{3.822786in}{2.530169in}}%
\pgfpathlineto{\pgfqpoint{3.777135in}{2.498578in}}%
\pgfpathlineto{\pgfqpoint{3.754649in}{2.483017in}}%
\pgfpathlineto{\pgfqpoint{3.697745in}{2.443640in}}%
\pgfpathlineto{\pgfqpoint{3.686511in}{2.435866in}}%
\pgfpathlineto{\pgfqpoint{3.618374in}{2.396333in}}%
\pgfpathlineto{\pgfqpoint{3.602916in}{2.388702in}}%
\pgfpathlineto{\pgfqpoint{3.550236in}{2.362695in}}%
\pgfpathlineto{\pgfqpoint{3.491630in}{2.333763in}}%
\pgfpathlineto{\pgfqpoint{3.482099in}{2.329058in}}%
\pgfpathlineto{\pgfqpoint{3.413961in}{2.295421in}}%
\pgfpathlineto{\pgfqpoint{3.380344in}{2.278825in}}%
\pgfpathlineto{\pgfqpoint{3.345824in}{2.261784in}}%
\pgfpathlineto{\pgfqpoint{3.277686in}{2.229450in}}%
\pgfpathlineto{\pgfqpoint{3.209549in}{2.241630in}}%
\pgfpathlineto{\pgfqpoint{3.141411in}{2.261634in}}%
\pgfpathlineto{\pgfqpoint{3.082854in}{2.278825in}}%
\pgfpathlineto{\pgfqpoint{3.073274in}{2.281638in}}%
\pgfpathlineto{\pgfqpoint{3.005136in}{2.301642in}}%
\pgfpathlineto{\pgfqpoint{2.936999in}{2.321646in}}%
\pgfpathlineto{\pgfqpoint{2.895724in}{2.333763in}}%
\pgfpathlineto{\pgfqpoint{2.868861in}{2.341650in}}%
\pgfpathlineto{\pgfqpoint{2.800724in}{2.361654in}}%
\pgfpathlineto{\pgfqpoint{2.732586in}{2.381658in}}%
\pgfpathlineto{\pgfqpoint{2.708594in}{2.388702in}}%
\pgfpathlineto{\pgfqpoint{2.664449in}{2.401662in}}%
\pgfpathlineto{\pgfqpoint{2.596311in}{2.421666in}}%
\pgfpathlineto{\pgfqpoint{2.528174in}{2.441670in}}%
\pgfpathlineto{\pgfqpoint{2.521464in}{2.443640in}}%
\pgfpathlineto{\pgfqpoint{2.460036in}{2.461674in}}%
\pgfpathlineto{\pgfqpoint{2.391899in}{2.481678in}}%
\pgfpathlineto{\pgfqpoint{2.334334in}{2.498578in}}%
\pgfpathlineto{\pgfqpoint{2.323761in}{2.501682in}}%
\pgfpathlineto{\pgfqpoint{2.255624in}{2.521686in}}%
\pgfpathlineto{\pgfqpoint{2.187486in}{2.541690in}}%
\pgfpathlineto{\pgfqpoint{2.147204in}{2.553516in}}%
\pgfpathlineto{\pgfqpoint{2.119349in}{2.561694in}}%
\pgfpathlineto{\pgfqpoint{2.051211in}{2.581698in}}%
\pgfpathlineto{\pgfqpoint{1.983074in}{2.601702in}}%
\pgfpathlineto{\pgfqpoint{1.960073in}{2.608454in}}%
\pgfpathlineto{\pgfqpoint{1.914936in}{2.621706in}}%
\pgfpathlineto{\pgfqpoint{1.846799in}{2.641710in}}%
\pgfpathlineto{\pgfqpoint{1.778661in}{2.661714in}}%
\pgfpathlineto{\pgfqpoint{1.772943in}{2.663393in}}%
\pgfpathlineto{\pgfqpoint{1.710524in}{2.681718in}}%
\pgfpathlineto{\pgfqpoint{1.642386in}{2.701722in}}%
\pgfpathlineto{\pgfqpoint{1.585813in}{2.718331in}}%
\pgfpathlineto{\pgfqpoint{1.574249in}{2.721726in}}%
\pgfpathlineto{\pgfqpoint{1.506111in}{2.741730in}}%
\pgfpathlineto{\pgfqpoint{1.437974in}{2.762575in}}%
\pgfpathlineto{\pgfqpoint{1.405840in}{2.773269in}}%
\pgfpathlineto{\pgfqpoint{1.369836in}{2.785224in}}%
\pgfpathlineto{\pgfqpoint{1.301699in}{2.808896in}}%
\pgfpathlineto{\pgfqpoint{1.237165in}{2.828207in}}%
\pgfpathlineto{\pgfqpoint{1.233561in}{2.829366in}}%
\pgfpathlineto{\pgfqpoint{1.165424in}{2.853515in}}%
\pgfpathlineto{\pgfqpoint{1.165424in}{2.828207in}}%
\pgfpathlineto{\pgfqpoint{1.165424in}{2.773269in}}%
\pgfpathlineto{\pgfqpoint{1.097286in}{2.718331in}}%
\pgfpathlineto{\pgfqpoint{1.097286in}{2.663393in}}%
\pgfpathlineto{\pgfqpoint{1.029149in}{2.608454in}}%
\pgfpathlineto{\pgfqpoint{1.029149in}{2.553516in}}%
\pgfpathlineto{\pgfqpoint{1.029149in}{2.526883in}}%
\pgfpathlineto{\pgfqpoint{1.062163in}{2.498578in}}%
\pgfpathlineto{\pgfqpoint{1.084182in}{2.443640in}}%
\pgfpathlineto{\pgfqpoint{1.092891in}{2.388702in}}%
\pgfpathlineto{\pgfqpoint{1.097286in}{2.337842in}}%
\pgfpathlineto{\pgfqpoint{1.097526in}{2.333763in}}%
\pgfpathlineto{\pgfqpoint{1.100966in}{2.278825in}}%
\pgfpathlineto{\pgfqpoint{1.104839in}{2.223887in}}%
\pgfpathlineto{\pgfqpoint{1.115692in}{2.168949in}}%
\pgfpathlineto{\pgfqpoint{1.130978in}{2.114011in}}%
\pgfpathlineto{\pgfqpoint{1.151034in}{2.059072in}}%
\pgfpathlineto{\pgfqpoint{1.165424in}{2.027611in}}%
\pgfpathlineto{\pgfqpoint{1.233561in}{2.037389in}}%
\pgfpathlineto{\pgfqpoint{1.255766in}{2.059072in}}%
\pgfpathlineto{\pgfqpoint{1.282936in}{2.114011in}}%
\pgfpathlineto{\pgfqpoint{1.301699in}{2.163449in}}%
\pgfpathlineto{\pgfqpoint{1.303786in}{2.168949in}}%
\pgfpathlineto{\pgfqpoint{1.324635in}{2.223887in}}%
\pgfpathlineto{\pgfqpoint{1.345485in}{2.278825in}}%
\pgfpathlineto{\pgfqpoint{1.366334in}{2.333763in}}%
\pgfpathlineto{\pgfqpoint{1.369836in}{2.342991in}}%
\pgfpathlineto{\pgfqpoint{1.437974in}{2.357675in}}%
\pgfpathlineto{\pgfqpoint{1.506111in}{2.343155in}}%
\pgfpathlineto{\pgfqpoint{1.533347in}{2.333763in}}%
\pgfpathlineto{\pgfqpoint{1.574249in}{2.319984in}}%
\pgfpathlineto{\pgfqpoint{1.642386in}{2.295765in}}%
\pgfpathlineto{\pgfqpoint{1.690047in}{2.278825in}}%
\pgfpathlineto{\pgfqpoint{1.710524in}{2.271547in}}%
\pgfpathlineto{\pgfqpoint{1.778661in}{2.247329in}}%
\pgfpathlineto{\pgfqpoint{1.844613in}{2.223887in}}%
\pgfpathlineto{\pgfqpoint{1.846799in}{2.223110in}}%
\pgfpathlineto{\pgfqpoint{1.914936in}{2.198892in}}%
\pgfpathlineto{\pgfqpoint{1.983074in}{2.174674in}}%
\pgfpathlineto{\pgfqpoint{1.999180in}{2.168949in}}%
\pgfpathlineto{\pgfqpoint{2.051211in}{2.150455in}}%
\pgfpathlineto{\pgfqpoint{2.119349in}{2.126237in}}%
\pgfpathlineto{\pgfqpoint{2.153747in}{2.114011in}}%
\pgfpathlineto{\pgfqpoint{2.187486in}{2.102019in}}%
\pgfpathlineto{\pgfqpoint{2.255624in}{2.079292in}}%
\pgfpathlineto{\pgfqpoint{2.319278in}{2.059072in}}%
\pgfpathlineto{\pgfqpoint{2.323761in}{2.057645in}}%
\pgfpathlineto{\pgfqpoint{2.391899in}{2.037574in}}%
\pgfpathlineto{\pgfqpoint{2.460036in}{2.017570in}}%
\pgfpathlineto{\pgfqpoint{2.505801in}{2.004134in}}%
\pgfpathlineto{\pgfqpoint{2.528174in}{1.997566in}}%
\pgfpathlineto{\pgfqpoint{2.596311in}{1.977562in}}%
\pgfpathlineto{\pgfqpoint{2.664449in}{1.957558in}}%
\pgfpathlineto{\pgfqpoint{2.692931in}{1.949196in}}%
\pgfpathlineto{\pgfqpoint{2.732586in}{1.937554in}}%
\pgfpathlineto{\pgfqpoint{2.800724in}{1.917550in}}%
\pgfpathlineto{\pgfqpoint{2.868861in}{1.897546in}}%
\pgfpathlineto{\pgfqpoint{2.880062in}{1.894258in}}%
\pgfpathlineto{\pgfqpoint{2.936999in}{1.877542in}}%
\pgfpathlineto{\pgfqpoint{3.005136in}{1.857538in}}%
\pgfpathlineto{\pgfqpoint{3.067192in}{1.839320in}}%
\pgfpathclose%
\pgfusepath{fill}%
\end{pgfscope}%
\begin{pgfscope}%
\pgfpathrectangle{\pgfqpoint{0.634105in}{0.521603in}}{\pgfqpoint{3.720000in}{3.020000in}} %
\pgfusepath{clip}%
\pgfsetbuttcap%
\pgfsetroundjoin%
\definecolor{currentfill}{rgb}{0.710478,0.814706,0.814706}%
\pgfsetfillcolor{currentfill}%
\pgfsetlinewidth{0.000000pt}%
\definecolor{currentstroke}{rgb}{0.000000,0.000000,0.000000}%
\pgfsetstrokecolor{currentstroke}%
\pgfsetdash{}{0pt}%
\pgfpathmoveto{\pgfqpoint{3.141411in}{2.261634in}}%
\pgfpathlineto{\pgfqpoint{3.209549in}{2.241630in}}%
\pgfpathlineto{\pgfqpoint{3.277686in}{2.229450in}}%
\pgfpathlineto{\pgfqpoint{3.345824in}{2.261784in}}%
\pgfpathlineto{\pgfqpoint{3.380344in}{2.278825in}}%
\pgfpathlineto{\pgfqpoint{3.413961in}{2.295421in}}%
\pgfpathlineto{\pgfqpoint{3.482099in}{2.329058in}}%
\pgfpathlineto{\pgfqpoint{3.491630in}{2.333763in}}%
\pgfpathlineto{\pgfqpoint{3.550236in}{2.362695in}}%
\pgfpathlineto{\pgfqpoint{3.602916in}{2.388702in}}%
\pgfpathlineto{\pgfqpoint{3.618374in}{2.396333in}}%
\pgfpathlineto{\pgfqpoint{3.686511in}{2.435866in}}%
\pgfpathlineto{\pgfqpoint{3.697745in}{2.443640in}}%
\pgfpathlineto{\pgfqpoint{3.754649in}{2.483017in}}%
\pgfpathlineto{\pgfqpoint{3.777135in}{2.498578in}}%
\pgfpathlineto{\pgfqpoint{3.822786in}{2.530169in}}%
\pgfpathlineto{\pgfqpoint{3.856525in}{2.553516in}}%
\pgfpathlineto{\pgfqpoint{3.890924in}{2.577320in}}%
\pgfpathlineto{\pgfqpoint{3.935916in}{2.608454in}}%
\pgfpathlineto{\pgfqpoint{3.959061in}{2.624471in}}%
\pgfpathlineto{\pgfqpoint{4.015306in}{2.663393in}}%
\pgfpathlineto{\pgfqpoint{4.027199in}{2.671622in}}%
\pgfpathlineto{\pgfqpoint{4.094696in}{2.718331in}}%
\pgfpathlineto{\pgfqpoint{4.095336in}{2.718774in}}%
\pgfpathlineto{\pgfqpoint{4.095336in}{2.773269in}}%
\pgfpathlineto{\pgfqpoint{4.027199in}{2.828207in}}%
\pgfpathlineto{\pgfqpoint{4.027199in}{2.883145in}}%
\pgfpathlineto{\pgfqpoint{4.027199in}{2.938084in}}%
\pgfpathlineto{\pgfqpoint{4.017224in}{2.946126in}}%
\pgfpathlineto{\pgfqpoint{4.005603in}{2.938084in}}%
\pgfpathlineto{\pgfqpoint{3.959061in}{2.905877in}}%
\pgfpathlineto{\pgfqpoint{3.926213in}{2.883145in}}%
\pgfpathlineto{\pgfqpoint{3.890924in}{2.858725in}}%
\pgfpathlineto{\pgfqpoint{3.846822in}{2.828207in}}%
\pgfpathlineto{\pgfqpoint{3.822786in}{2.811574in}}%
\pgfpathlineto{\pgfqpoint{3.767432in}{2.773269in}}%
\pgfpathlineto{\pgfqpoint{3.754649in}{2.764423in}}%
\pgfpathlineto{\pgfqpoint{3.686511in}{2.723477in}}%
\pgfpathlineto{\pgfqpoint{3.676087in}{2.718331in}}%
\pgfpathlineto{\pgfqpoint{3.618374in}{2.689840in}}%
\pgfpathlineto{\pgfqpoint{3.564801in}{2.663393in}}%
\pgfpathlineto{\pgfqpoint{3.550236in}{2.656202in}}%
\pgfpathlineto{\pgfqpoint{3.482099in}{2.623297in}}%
\pgfpathlineto{\pgfqpoint{3.413961in}{2.625722in}}%
\pgfpathlineto{\pgfqpoint{3.345824in}{2.645726in}}%
\pgfpathlineto{\pgfqpoint{3.285647in}{2.663393in}}%
\pgfpathlineto{\pgfqpoint{3.277686in}{2.665730in}}%
\pgfpathlineto{\pgfqpoint{3.209549in}{2.685734in}}%
\pgfpathlineto{\pgfqpoint{3.141411in}{2.705738in}}%
\pgfpathlineto{\pgfqpoint{3.098517in}{2.718331in}}%
\pgfpathlineto{\pgfqpoint{3.073274in}{2.725742in}}%
\pgfpathlineto{\pgfqpoint{3.005136in}{2.745746in}}%
\pgfpathlineto{\pgfqpoint{2.936999in}{2.765750in}}%
\pgfpathlineto{\pgfqpoint{2.911387in}{2.773269in}}%
\pgfpathlineto{\pgfqpoint{2.868861in}{2.785754in}}%
\pgfpathlineto{\pgfqpoint{2.800724in}{2.805758in}}%
\pgfpathlineto{\pgfqpoint{2.732586in}{2.825762in}}%
\pgfpathlineto{\pgfqpoint{2.724257in}{2.828207in}}%
\pgfpathlineto{\pgfqpoint{2.664449in}{2.845766in}}%
\pgfpathlineto{\pgfqpoint{2.596311in}{2.865770in}}%
\pgfpathlineto{\pgfqpoint{2.537126in}{2.883145in}}%
\pgfpathlineto{\pgfqpoint{2.528174in}{2.885774in}}%
\pgfpathlineto{\pgfqpoint{2.460036in}{2.905778in}}%
\pgfpathlineto{\pgfqpoint{2.391899in}{2.925782in}}%
\pgfpathlineto{\pgfqpoint{2.349996in}{2.938084in}}%
\pgfpathlineto{\pgfqpoint{2.323761in}{2.945786in}}%
\pgfpathlineto{\pgfqpoint{2.255624in}{2.965790in}}%
\pgfpathlineto{\pgfqpoint{2.187486in}{2.985794in}}%
\pgfpathlineto{\pgfqpoint{2.162866in}{2.993022in}}%
\pgfpathlineto{\pgfqpoint{2.119349in}{3.005798in}}%
\pgfpathlineto{\pgfqpoint{2.051211in}{3.025802in}}%
\pgfpathlineto{\pgfqpoint{1.983074in}{3.045806in}}%
\pgfpathlineto{\pgfqpoint{1.975736in}{3.047960in}}%
\pgfpathlineto{\pgfqpoint{1.914936in}{3.047960in}}%
\pgfpathlineto{\pgfqpoint{1.846799in}{3.047960in}}%
\pgfpathlineto{\pgfqpoint{1.778661in}{3.047960in}}%
\pgfpathlineto{\pgfqpoint{1.710524in}{2.993022in}}%
\pgfpathlineto{\pgfqpoint{1.642386in}{2.993022in}}%
\pgfpathlineto{\pgfqpoint{1.574249in}{2.993022in}}%
\pgfpathlineto{\pgfqpoint{1.506111in}{2.993022in}}%
\pgfpathlineto{\pgfqpoint{1.437974in}{2.993022in}}%
\pgfpathlineto{\pgfqpoint{1.369836in}{2.938084in}}%
\pgfpathlineto{\pgfqpoint{1.301699in}{2.938084in}}%
\pgfpathlineto{\pgfqpoint{1.233561in}{2.938084in}}%
\pgfpathlineto{\pgfqpoint{1.165424in}{2.883145in}}%
\pgfpathlineto{\pgfqpoint{1.165424in}{2.853515in}}%
\pgfpathlineto{\pgfqpoint{1.233561in}{2.829366in}}%
\pgfpathlineto{\pgfqpoint{1.237165in}{2.828207in}}%
\pgfpathlineto{\pgfqpoint{1.301699in}{2.808896in}}%
\pgfpathlineto{\pgfqpoint{1.369836in}{2.785224in}}%
\pgfpathlineto{\pgfqpoint{1.405840in}{2.773269in}}%
\pgfpathlineto{\pgfqpoint{1.437974in}{2.762575in}}%
\pgfpathlineto{\pgfqpoint{1.506111in}{2.741730in}}%
\pgfpathlineto{\pgfqpoint{1.574249in}{2.721726in}}%
\pgfpathlineto{\pgfqpoint{1.585813in}{2.718331in}}%
\pgfpathlineto{\pgfqpoint{1.642386in}{2.701722in}}%
\pgfpathlineto{\pgfqpoint{1.710524in}{2.681718in}}%
\pgfpathlineto{\pgfqpoint{1.772943in}{2.663393in}}%
\pgfpathlineto{\pgfqpoint{1.778661in}{2.661714in}}%
\pgfpathlineto{\pgfqpoint{1.846799in}{2.641710in}}%
\pgfpathlineto{\pgfqpoint{1.914936in}{2.621706in}}%
\pgfpathlineto{\pgfqpoint{1.960073in}{2.608454in}}%
\pgfpathlineto{\pgfqpoint{1.983074in}{2.601702in}}%
\pgfpathlineto{\pgfqpoint{2.051211in}{2.581698in}}%
\pgfpathlineto{\pgfqpoint{2.119349in}{2.561694in}}%
\pgfpathlineto{\pgfqpoint{2.147204in}{2.553516in}}%
\pgfpathlineto{\pgfqpoint{2.187486in}{2.541690in}}%
\pgfpathlineto{\pgfqpoint{2.255624in}{2.521686in}}%
\pgfpathlineto{\pgfqpoint{2.323761in}{2.501682in}}%
\pgfpathlineto{\pgfqpoint{2.334334in}{2.498578in}}%
\pgfpathlineto{\pgfqpoint{2.391899in}{2.481678in}}%
\pgfpathlineto{\pgfqpoint{2.460036in}{2.461674in}}%
\pgfpathlineto{\pgfqpoint{2.521464in}{2.443640in}}%
\pgfpathlineto{\pgfqpoint{2.528174in}{2.441670in}}%
\pgfpathlineto{\pgfqpoint{2.596311in}{2.421666in}}%
\pgfpathlineto{\pgfqpoint{2.664449in}{2.401662in}}%
\pgfpathlineto{\pgfqpoint{2.708594in}{2.388702in}}%
\pgfpathlineto{\pgfqpoint{2.732586in}{2.381658in}}%
\pgfpathlineto{\pgfqpoint{2.800724in}{2.361654in}}%
\pgfpathlineto{\pgfqpoint{2.868861in}{2.341650in}}%
\pgfpathlineto{\pgfqpoint{2.895724in}{2.333763in}}%
\pgfpathlineto{\pgfqpoint{2.936999in}{2.321646in}}%
\pgfpathlineto{\pgfqpoint{3.005136in}{2.301642in}}%
\pgfpathlineto{\pgfqpoint{3.073274in}{2.281638in}}%
\pgfpathlineto{\pgfqpoint{3.082854in}{2.278825in}}%
\pgfpathclose%
\pgfusepath{fill}%
\end{pgfscope}%
\begin{pgfscope}%
\pgfpathrectangle{\pgfqpoint{0.634105in}{0.521603in}}{\pgfqpoint{3.720000in}{3.020000in}} %
\pgfusepath{clip}%
\pgfsetbuttcap%
\pgfsetroundjoin%
\definecolor{currentfill}{rgb}{0.903493,0.938235,0.938235}%
\pgfsetfillcolor{currentfill}%
\pgfsetlinewidth{0.000000pt}%
\definecolor{currentstroke}{rgb}{0.000000,0.000000,0.000000}%
\pgfsetstrokecolor{currentstroke}%
\pgfsetdash{}{0pt}%
\pgfpathmoveto{\pgfqpoint{3.345824in}{2.645726in}}%
\pgfpathlineto{\pgfqpoint{3.413961in}{2.625722in}}%
\pgfpathlineto{\pgfqpoint{3.482099in}{2.623297in}}%
\pgfpathlineto{\pgfqpoint{3.550236in}{2.656202in}}%
\pgfpathlineto{\pgfqpoint{3.564801in}{2.663393in}}%
\pgfpathlineto{\pgfqpoint{3.618374in}{2.689840in}}%
\pgfpathlineto{\pgfqpoint{3.676087in}{2.718331in}}%
\pgfpathlineto{\pgfqpoint{3.686511in}{2.723477in}}%
\pgfpathlineto{\pgfqpoint{3.754649in}{2.764423in}}%
\pgfpathlineto{\pgfqpoint{3.767432in}{2.773269in}}%
\pgfpathlineto{\pgfqpoint{3.822786in}{2.811574in}}%
\pgfpathlineto{\pgfqpoint{3.846822in}{2.828207in}}%
\pgfpathlineto{\pgfqpoint{3.890924in}{2.858725in}}%
\pgfpathlineto{\pgfqpoint{3.926213in}{2.883145in}}%
\pgfpathlineto{\pgfqpoint{3.959061in}{2.905877in}}%
\pgfpathlineto{\pgfqpoint{4.005603in}{2.938084in}}%
\pgfpathlineto{\pgfqpoint{4.017224in}{2.946126in}}%
\pgfpathlineto{\pgfqpoint{3.959061in}{2.993022in}}%
\pgfpathlineto{\pgfqpoint{3.959061in}{3.047960in}}%
\pgfpathlineto{\pgfqpoint{3.959061in}{3.102898in}}%
\pgfpathlineto{\pgfqpoint{3.902741in}{3.148308in}}%
\pgfpathlineto{\pgfqpoint{3.890924in}{3.140131in}}%
\pgfpathlineto{\pgfqpoint{3.837119in}{3.102898in}}%
\pgfpathlineto{\pgfqpoint{3.822786in}{3.092980in}}%
\pgfpathlineto{\pgfqpoint{3.754649in}{3.050621in}}%
\pgfpathlineto{\pgfqpoint{3.749258in}{3.047960in}}%
\pgfpathlineto{\pgfqpoint{3.686511in}{3.016984in}}%
\pgfpathlineto{\pgfqpoint{3.618374in}{3.009814in}}%
\pgfpathlineto{\pgfqpoint{3.550236in}{3.029818in}}%
\pgfpathlineto{\pgfqpoint{3.488440in}{3.047960in}}%
\pgfpathlineto{\pgfqpoint{3.482099in}{3.049822in}}%
\pgfpathlineto{\pgfqpoint{3.413961in}{3.069826in}}%
\pgfpathlineto{\pgfqpoint{3.345824in}{3.089830in}}%
\pgfpathlineto{\pgfqpoint{3.301309in}{3.102898in}}%
\pgfpathlineto{\pgfqpoint{3.277686in}{3.109834in}}%
\pgfpathlineto{\pgfqpoint{3.209549in}{3.129838in}}%
\pgfpathlineto{\pgfqpoint{3.141411in}{3.149842in}}%
\pgfpathlineto{\pgfqpoint{3.114179in}{3.157836in}}%
\pgfpathlineto{\pgfqpoint{3.073274in}{3.169846in}}%
\pgfpathlineto{\pgfqpoint{3.005136in}{3.189850in}}%
\pgfpathlineto{\pgfqpoint{2.936999in}{3.209854in}}%
\pgfpathlineto{\pgfqpoint{2.927049in}{3.212775in}}%
\pgfpathlineto{\pgfqpoint{2.868861in}{3.212775in}}%
\pgfpathlineto{\pgfqpoint{2.800724in}{3.157836in}}%
\pgfpathlineto{\pgfqpoint{2.732586in}{3.157836in}}%
\pgfpathlineto{\pgfqpoint{2.664449in}{3.157836in}}%
\pgfpathlineto{\pgfqpoint{2.596311in}{3.157836in}}%
\pgfpathlineto{\pgfqpoint{2.528174in}{3.157836in}}%
\pgfpathlineto{\pgfqpoint{2.460036in}{3.102898in}}%
\pgfpathlineto{\pgfqpoint{2.391899in}{3.102898in}}%
\pgfpathlineto{\pgfqpoint{2.323761in}{3.102898in}}%
\pgfpathlineto{\pgfqpoint{2.255624in}{3.102898in}}%
\pgfpathlineto{\pgfqpoint{2.187486in}{3.102898in}}%
\pgfpathlineto{\pgfqpoint{2.119349in}{3.047960in}}%
\pgfpathlineto{\pgfqpoint{2.051211in}{3.047960in}}%
\pgfpathlineto{\pgfqpoint{1.983074in}{3.047960in}}%
\pgfpathlineto{\pgfqpoint{1.975736in}{3.047960in}}%
\pgfpathlineto{\pgfqpoint{1.983074in}{3.045806in}}%
\pgfpathlineto{\pgfqpoint{2.051211in}{3.025802in}}%
\pgfpathlineto{\pgfqpoint{2.119349in}{3.005798in}}%
\pgfpathlineto{\pgfqpoint{2.162866in}{2.993022in}}%
\pgfpathlineto{\pgfqpoint{2.187486in}{2.985794in}}%
\pgfpathlineto{\pgfqpoint{2.255624in}{2.965790in}}%
\pgfpathlineto{\pgfqpoint{2.323761in}{2.945786in}}%
\pgfpathlineto{\pgfqpoint{2.349996in}{2.938084in}}%
\pgfpathlineto{\pgfqpoint{2.391899in}{2.925782in}}%
\pgfpathlineto{\pgfqpoint{2.460036in}{2.905778in}}%
\pgfpathlineto{\pgfqpoint{2.528174in}{2.885774in}}%
\pgfpathlineto{\pgfqpoint{2.537126in}{2.883145in}}%
\pgfpathlineto{\pgfqpoint{2.596311in}{2.865770in}}%
\pgfpathlineto{\pgfqpoint{2.664449in}{2.845766in}}%
\pgfpathlineto{\pgfqpoint{2.724257in}{2.828207in}}%
\pgfpathlineto{\pgfqpoint{2.732586in}{2.825762in}}%
\pgfpathlineto{\pgfqpoint{2.800724in}{2.805758in}}%
\pgfpathlineto{\pgfqpoint{2.868861in}{2.785754in}}%
\pgfpathlineto{\pgfqpoint{2.911387in}{2.773269in}}%
\pgfpathlineto{\pgfqpoint{2.936999in}{2.765750in}}%
\pgfpathlineto{\pgfqpoint{3.005136in}{2.745746in}}%
\pgfpathlineto{\pgfqpoint{3.073274in}{2.725742in}}%
\pgfpathlineto{\pgfqpoint{3.098517in}{2.718331in}}%
\pgfpathlineto{\pgfqpoint{3.141411in}{2.705738in}}%
\pgfpathlineto{\pgfqpoint{3.209549in}{2.685734in}}%
\pgfpathlineto{\pgfqpoint{3.277686in}{2.665730in}}%
\pgfpathlineto{\pgfqpoint{3.285647in}{2.663393in}}%
\pgfpathclose%
\pgfusepath{fill}%
\end{pgfscope}%
\begin{pgfscope}%
\pgfpathrectangle{\pgfqpoint{0.634105in}{0.521603in}}{\pgfqpoint{3.720000in}{3.020000in}} %
\pgfusepath{clip}%
\pgfsetbuttcap%
\pgfsetroundjoin%
\definecolor{currentfill}{rgb}{1.000000,1.000000,1.000000}%
\pgfsetfillcolor{currentfill}%
\pgfsetlinewidth{1.003750pt}%
\definecolor{currentstroke}{rgb}{0.000000,0.000000,0.000000}%
\pgfsetstrokecolor{currentstroke}%
\pgfsetdash{}{0pt}%
\pgfpathmoveto{\pgfqpoint{3.838210in}{3.335922in}}%
\pgfpathcurveto{\pgfqpoint{3.849260in}{3.335922in}}{\pgfqpoint{3.859859in}{3.340313in}}{\pgfqpoint{3.867673in}{3.348126in}}%
\pgfpathcurveto{\pgfqpoint{3.875486in}{3.355940in}}{\pgfqpoint{3.879877in}{3.366539in}}{\pgfqpoint{3.879877in}{3.377589in}}%
\pgfpathcurveto{\pgfqpoint{3.879877in}{3.388639in}}{\pgfqpoint{3.875486in}{3.399238in}}{\pgfqpoint{3.867673in}{3.407052in}}%
\pgfpathcurveto{\pgfqpoint{3.859859in}{3.414866in}}{\pgfqpoint{3.849260in}{3.419256in}}{\pgfqpoint{3.838210in}{3.419256in}}%
\pgfpathcurveto{\pgfqpoint{3.827160in}{3.419256in}}{\pgfqpoint{3.816561in}{3.414866in}}{\pgfqpoint{3.808747in}{3.407052in}}%
\pgfpathcurveto{\pgfqpoint{3.800934in}{3.399238in}}{\pgfqpoint{3.796543in}{3.388639in}}{\pgfqpoint{3.796543in}{3.377589in}}%
\pgfpathcurveto{\pgfqpoint{3.796543in}{3.366539in}}{\pgfqpoint{3.800934in}{3.355940in}}{\pgfqpoint{3.808747in}{3.348126in}}%
\pgfpathcurveto{\pgfqpoint{3.816561in}{3.340313in}}{\pgfqpoint{3.827160in}{3.335922in}}{\pgfqpoint{3.838210in}{3.335922in}}%
\pgfpathclose%
\pgfusepath{stroke,fill}%
\end{pgfscope}%
\begin{pgfscope}%
\pgfpathrectangle{\pgfqpoint{0.634105in}{0.521603in}}{\pgfqpoint{3.720000in}{3.020000in}} %
\pgfusepath{clip}%
\pgfsetbuttcap%
\pgfsetroundjoin%
\definecolor{currentfill}{rgb}{1.000000,1.000000,1.000000}%
\pgfsetfillcolor{currentfill}%
\pgfsetlinewidth{1.003750pt}%
\definecolor{currentstroke}{rgb}{0.000000,0.000000,0.000000}%
\pgfsetstrokecolor{currentstroke}%
\pgfsetdash{}{0pt}%
\pgfpathmoveto{\pgfqpoint{1.015690in}{2.178249in}}%
\pgfpathcurveto{\pgfqpoint{1.026740in}{2.178249in}}{\pgfqpoint{1.037339in}{2.182639in}}{\pgfqpoint{1.045153in}{2.190453in}}%
\pgfpathcurveto{\pgfqpoint{1.052966in}{2.198266in}}{\pgfqpoint{1.057357in}{2.208866in}}{\pgfqpoint{1.057357in}{2.219916in}}%
\pgfpathcurveto{\pgfqpoint{1.057357in}{2.230966in}}{\pgfqpoint{1.052966in}{2.241565in}}{\pgfqpoint{1.045153in}{2.249378in}}%
\pgfpathcurveto{\pgfqpoint{1.037339in}{2.257192in}}{\pgfqpoint{1.026740in}{2.261582in}}{\pgfqpoint{1.015690in}{2.261582in}}%
\pgfpathcurveto{\pgfqpoint{1.004640in}{2.261582in}}{\pgfqpoint{0.994041in}{2.257192in}}{\pgfqpoint{0.986227in}{2.249378in}}%
\pgfpathcurveto{\pgfqpoint{0.978414in}{2.241565in}}{\pgfqpoint{0.974023in}{2.230966in}}{\pgfqpoint{0.974023in}{2.219916in}}%
\pgfpathcurveto{\pgfqpoint{0.974023in}{2.208866in}}{\pgfqpoint{0.978414in}{2.198266in}}{\pgfqpoint{0.986227in}{2.190453in}}%
\pgfpathcurveto{\pgfqpoint{0.994041in}{2.182639in}}{\pgfqpoint{1.004640in}{2.178249in}}{\pgfqpoint{1.015690in}{2.178249in}}%
\pgfpathclose%
\pgfusepath{stroke,fill}%
\end{pgfscope}%
\begin{pgfscope}%
\pgfpathrectangle{\pgfqpoint{0.634105in}{0.521603in}}{\pgfqpoint{3.720000in}{3.020000in}} %
\pgfusepath{clip}%
\pgfsetbuttcap%
\pgfsetroundjoin%
\definecolor{currentfill}{rgb}{1.000000,1.000000,1.000000}%
\pgfsetfillcolor{currentfill}%
\pgfsetlinewidth{1.003750pt}%
\definecolor{currentstroke}{rgb}{0.000000,0.000000,0.000000}%
\pgfsetstrokecolor{currentstroke}%
\pgfsetdash{}{0pt}%
\pgfpathmoveto{\pgfqpoint{0.824736in}{2.177069in}}%
\pgfpathcurveto{\pgfqpoint{0.835786in}{2.177069in}}{\pgfqpoint{0.846385in}{2.181459in}}{\pgfqpoint{0.854199in}{2.189273in}}%
\pgfpathcurveto{\pgfqpoint{0.862012in}{2.197087in}}{\pgfqpoint{0.866403in}{2.207686in}}{\pgfqpoint{0.866403in}{2.218736in}}%
\pgfpathcurveto{\pgfqpoint{0.866403in}{2.229786in}}{\pgfqpoint{0.862012in}{2.240385in}}{\pgfqpoint{0.854199in}{2.248199in}}%
\pgfpathcurveto{\pgfqpoint{0.846385in}{2.256012in}}{\pgfqpoint{0.835786in}{2.260402in}}{\pgfqpoint{0.824736in}{2.260402in}}%
\pgfpathcurveto{\pgfqpoint{0.813686in}{2.260402in}}{\pgfqpoint{0.803087in}{2.256012in}}{\pgfqpoint{0.795273in}{2.248199in}}%
\pgfpathcurveto{\pgfqpoint{0.787460in}{2.240385in}}{\pgfqpoint{0.783069in}{2.229786in}}{\pgfqpoint{0.783069in}{2.218736in}}%
\pgfpathcurveto{\pgfqpoint{0.783069in}{2.207686in}}{\pgfqpoint{0.787460in}{2.197087in}}{\pgfqpoint{0.795273in}{2.189273in}}%
\pgfpathcurveto{\pgfqpoint{0.803087in}{2.181459in}}{\pgfqpoint{0.813686in}{2.177069in}}{\pgfqpoint{0.824736in}{2.177069in}}%
\pgfpathclose%
\pgfusepath{stroke,fill}%
\end{pgfscope}%
\begin{pgfscope}%
\pgfpathrectangle{\pgfqpoint{0.634105in}{0.521603in}}{\pgfqpoint{3.720000in}{3.020000in}} %
\pgfusepath{clip}%
\pgfsetbuttcap%
\pgfsetroundjoin%
\definecolor{currentfill}{rgb}{1.000000,1.000000,1.000000}%
\pgfsetfillcolor{currentfill}%
\pgfsetlinewidth{1.003750pt}%
\definecolor{currentstroke}{rgb}{0.000000,0.000000,0.000000}%
\pgfsetstrokecolor{currentstroke}%
\pgfsetdash{}{0pt}%
\pgfpathmoveto{\pgfqpoint{1.772294in}{1.116526in}}%
\pgfpathcurveto{\pgfqpoint{1.783344in}{1.116526in}}{\pgfqpoint{1.793943in}{1.120916in}}{\pgfqpoint{1.801757in}{1.128730in}}%
\pgfpathcurveto{\pgfqpoint{1.809570in}{1.136543in}}{\pgfqpoint{1.813960in}{1.147142in}}{\pgfqpoint{1.813960in}{1.158192in}}%
\pgfpathcurveto{\pgfqpoint{1.813960in}{1.169243in}}{\pgfqpoint{1.809570in}{1.179842in}}{\pgfqpoint{1.801757in}{1.187655in}}%
\pgfpathcurveto{\pgfqpoint{1.793943in}{1.195469in}}{\pgfqpoint{1.783344in}{1.199859in}}{\pgfqpoint{1.772294in}{1.199859in}}%
\pgfpathcurveto{\pgfqpoint{1.761244in}{1.199859in}}{\pgfqpoint{1.750645in}{1.195469in}}{\pgfqpoint{1.742831in}{1.187655in}}%
\pgfpathcurveto{\pgfqpoint{1.735017in}{1.179842in}}{\pgfqpoint{1.730627in}{1.169243in}}{\pgfqpoint{1.730627in}{1.158192in}}%
\pgfpathcurveto{\pgfqpoint{1.730627in}{1.147142in}}{\pgfqpoint{1.735017in}{1.136543in}}{\pgfqpoint{1.742831in}{1.128730in}}%
\pgfpathcurveto{\pgfqpoint{1.750645in}{1.120916in}}{\pgfqpoint{1.761244in}{1.116526in}}{\pgfqpoint{1.772294in}{1.116526in}}%
\pgfpathclose%
\pgfusepath{stroke,fill}%
\end{pgfscope}%
\begin{pgfscope}%
\pgfpathrectangle{\pgfqpoint{0.634105in}{0.521603in}}{\pgfqpoint{3.720000in}{3.020000in}} %
\pgfusepath{clip}%
\pgfsetbuttcap%
\pgfsetroundjoin%
\definecolor{currentfill}{rgb}{1.000000,1.000000,1.000000}%
\pgfsetfillcolor{currentfill}%
\pgfsetlinewidth{1.003750pt}%
\definecolor{currentstroke}{rgb}{0.000000,0.000000,0.000000}%
\pgfsetstrokecolor{currentstroke}%
\pgfsetdash{}{0pt}%
\pgfpathmoveto{\pgfqpoint{1.017683in}{0.972313in}}%
\pgfpathcurveto{\pgfqpoint{1.028733in}{0.972313in}}{\pgfqpoint{1.039332in}{0.976703in}}{\pgfqpoint{1.047146in}{0.984517in}}%
\pgfpathcurveto{\pgfqpoint{1.054959in}{0.992331in}}{\pgfqpoint{1.059349in}{1.002930in}}{\pgfqpoint{1.059349in}{1.013980in}}%
\pgfpathcurveto{\pgfqpoint{1.059349in}{1.025030in}}{\pgfqpoint{1.054959in}{1.035629in}}{\pgfqpoint{1.047146in}{1.043443in}}%
\pgfpathcurveto{\pgfqpoint{1.039332in}{1.051256in}}{\pgfqpoint{1.028733in}{1.055647in}}{\pgfqpoint{1.017683in}{1.055647in}}%
\pgfpathcurveto{\pgfqpoint{1.006633in}{1.055647in}}{\pgfqpoint{0.996034in}{1.051256in}}{\pgfqpoint{0.988220in}{1.043443in}}%
\pgfpathcurveto{\pgfqpoint{0.980406in}{1.035629in}}{\pgfqpoint{0.976016in}{1.025030in}}{\pgfqpoint{0.976016in}{1.013980in}}%
\pgfpathcurveto{\pgfqpoint{0.976016in}{1.002930in}}{\pgfqpoint{0.980406in}{0.992331in}}{\pgfqpoint{0.988220in}{0.984517in}}%
\pgfpathcurveto{\pgfqpoint{0.996034in}{0.976703in}}{\pgfqpoint{1.006633in}{0.972313in}}{\pgfqpoint{1.017683in}{0.972313in}}%
\pgfpathclose%
\pgfusepath{stroke,fill}%
\end{pgfscope}%
\begin{pgfscope}%
\pgfpathrectangle{\pgfqpoint{0.634105in}{0.521603in}}{\pgfqpoint{3.720000in}{3.020000in}} %
\pgfusepath{clip}%
\pgfsetbuttcap%
\pgfsetroundjoin%
\definecolor{currentfill}{rgb}{1.000000,1.000000,1.000000}%
\pgfsetfillcolor{currentfill}%
\pgfsetlinewidth{1.003750pt}%
\definecolor{currentstroke}{rgb}{0.000000,0.000000,0.000000}%
\pgfsetstrokecolor{currentstroke}%
\pgfsetdash{}{0pt}%
\pgfpathmoveto{\pgfqpoint{1.482493in}{0.751603in}}%
\pgfpathcurveto{\pgfqpoint{1.493543in}{0.751603in}}{\pgfqpoint{1.504142in}{0.755993in}}{\pgfqpoint{1.511956in}{0.763806in}}%
\pgfpathcurveto{\pgfqpoint{1.519770in}{0.771620in}}{\pgfqpoint{1.524160in}{0.782219in}}{\pgfqpoint{1.524160in}{0.793269in}}%
\pgfpathcurveto{\pgfqpoint{1.524160in}{0.804319in}}{\pgfqpoint{1.519770in}{0.814918in}}{\pgfqpoint{1.511956in}{0.822732in}}%
\pgfpathcurveto{\pgfqpoint{1.504142in}{0.830546in}}{\pgfqpoint{1.493543in}{0.834936in}}{\pgfqpoint{1.482493in}{0.834936in}}%
\pgfpathcurveto{\pgfqpoint{1.471443in}{0.834936in}}{\pgfqpoint{1.460844in}{0.830546in}}{\pgfqpoint{1.453030in}{0.822732in}}%
\pgfpathcurveto{\pgfqpoint{1.445217in}{0.814918in}}{\pgfqpoint{1.440826in}{0.804319in}}{\pgfqpoint{1.440826in}{0.793269in}}%
\pgfpathcurveto{\pgfqpoint{1.440826in}{0.782219in}}{\pgfqpoint{1.445217in}{0.771620in}}{\pgfqpoint{1.453030in}{0.763806in}}%
\pgfpathcurveto{\pgfqpoint{1.460844in}{0.755993in}}{\pgfqpoint{1.471443in}{0.751603in}}{\pgfqpoint{1.482493in}{0.751603in}}%
\pgfpathclose%
\pgfusepath{stroke,fill}%
\end{pgfscope}%
\begin{pgfscope}%
\pgfpathrectangle{\pgfqpoint{0.634105in}{0.521603in}}{\pgfqpoint{3.720000in}{3.020000in}} %
\pgfusepath{clip}%
\pgfsetbuttcap%
\pgfsetroundjoin%
\definecolor{currentfill}{rgb}{1.000000,1.000000,1.000000}%
\pgfsetfillcolor{currentfill}%
\pgfsetlinewidth{1.003750pt}%
\definecolor{currentstroke}{rgb}{0.000000,0.000000,0.000000}%
\pgfsetstrokecolor{currentstroke}%
\pgfsetdash{}{0pt}%
\pgfpathmoveto{\pgfqpoint{1.421851in}{0.643951in}}%
\pgfpathcurveto{\pgfqpoint{1.432901in}{0.643951in}}{\pgfqpoint{1.443501in}{0.648341in}}{\pgfqpoint{1.451314in}{0.656155in}}%
\pgfpathcurveto{\pgfqpoint{1.459128in}{0.663968in}}{\pgfqpoint{1.463518in}{0.674567in}}{\pgfqpoint{1.463518in}{0.685618in}}%
\pgfpathcurveto{\pgfqpoint{1.463518in}{0.696668in}}{\pgfqpoint{1.459128in}{0.707267in}}{\pgfqpoint{1.451314in}{0.715080in}}%
\pgfpathcurveto{\pgfqpoint{1.443501in}{0.722894in}}{\pgfqpoint{1.432901in}{0.727284in}}{\pgfqpoint{1.421851in}{0.727284in}}%
\pgfpathcurveto{\pgfqpoint{1.410801in}{0.727284in}}{\pgfqpoint{1.400202in}{0.722894in}}{\pgfqpoint{1.392389in}{0.715080in}}%
\pgfpathcurveto{\pgfqpoint{1.384575in}{0.707267in}}{\pgfqpoint{1.380185in}{0.696668in}}{\pgfqpoint{1.380185in}{0.685618in}}%
\pgfpathcurveto{\pgfqpoint{1.380185in}{0.674567in}}{\pgfqpoint{1.384575in}{0.663968in}}{\pgfqpoint{1.392389in}{0.656155in}}%
\pgfpathcurveto{\pgfqpoint{1.400202in}{0.648341in}}{\pgfqpoint{1.410801in}{0.643951in}}{\pgfqpoint{1.421851in}{0.643951in}}%
\pgfpathclose%
\pgfusepath{stroke,fill}%
\end{pgfscope}%
\begin{pgfscope}%
\pgfpathrectangle{\pgfqpoint{0.634105in}{0.521603in}}{\pgfqpoint{3.720000in}{3.020000in}} %
\pgfusepath{clip}%
\pgfsetbuttcap%
\pgfsetroundjoin%
\definecolor{currentfill}{rgb}{1.000000,1.000000,1.000000}%
\pgfsetfillcolor{currentfill}%
\pgfsetlinewidth{1.003750pt}%
\definecolor{currentstroke}{rgb}{0.000000,0.000000,0.000000}%
\pgfsetstrokecolor{currentstroke}%
\pgfsetdash{}{0pt}%
\pgfpathmoveto{\pgfqpoint{4.163474in}{2.623161in}}%
\pgfpathcurveto{\pgfqpoint{4.174524in}{2.623161in}}{\pgfqpoint{4.185123in}{2.627551in}}{\pgfqpoint{4.192936in}{2.635365in}}%
\pgfpathcurveto{\pgfqpoint{4.200750in}{2.643178in}}{\pgfqpoint{4.205140in}{2.653777in}}{\pgfqpoint{4.205140in}{2.664828in}}%
\pgfpathcurveto{\pgfqpoint{4.205140in}{2.675878in}}{\pgfqpoint{4.200750in}{2.686477in}}{\pgfqpoint{4.192936in}{2.694290in}}%
\pgfpathcurveto{\pgfqpoint{4.185123in}{2.702104in}}{\pgfqpoint{4.174524in}{2.706494in}}{\pgfqpoint{4.163474in}{2.706494in}}%
\pgfpathcurveto{\pgfqpoint{4.152424in}{2.706494in}}{\pgfqpoint{4.141825in}{2.702104in}}{\pgfqpoint{4.134011in}{2.694290in}}%
\pgfpathcurveto{\pgfqpoint{4.126197in}{2.686477in}}{\pgfqpoint{4.121807in}{2.675878in}}{\pgfqpoint{4.121807in}{2.664828in}}%
\pgfpathcurveto{\pgfqpoint{4.121807in}{2.653777in}}{\pgfqpoint{4.126197in}{2.643178in}}{\pgfqpoint{4.134011in}{2.635365in}}%
\pgfpathcurveto{\pgfqpoint{4.141825in}{2.627551in}}{\pgfqpoint{4.152424in}{2.623161in}}{\pgfqpoint{4.163474in}{2.623161in}}%
\pgfpathclose%
\pgfusepath{stroke,fill}%
\end{pgfscope}%
\begin{pgfscope}%
\pgfpathrectangle{\pgfqpoint{0.634105in}{0.521603in}}{\pgfqpoint{3.720000in}{3.020000in}} %
\pgfusepath{clip}%
\pgfsetbuttcap%
\pgfsetroundjoin%
\definecolor{currentfill}{rgb}{1.000000,1.000000,1.000000}%
\pgfsetfillcolor{currentfill}%
\pgfsetlinewidth{1.003750pt}%
\definecolor{currentstroke}{rgb}{0.000000,0.000000,0.000000}%
\pgfsetstrokecolor{currentstroke}%
\pgfsetdash{}{0pt}%
\pgfpathmoveto{\pgfqpoint{3.556105in}{1.839626in}}%
\pgfpathcurveto{\pgfqpoint{3.567155in}{1.839626in}}{\pgfqpoint{3.577754in}{1.844017in}}{\pgfqpoint{3.585568in}{1.851830in}}%
\pgfpathcurveto{\pgfqpoint{3.593381in}{1.859644in}}{\pgfqpoint{3.597772in}{1.870243in}}{\pgfqpoint{3.597772in}{1.881293in}}%
\pgfpathcurveto{\pgfqpoint{3.597772in}{1.892343in}}{\pgfqpoint{3.593381in}{1.902942in}}{\pgfqpoint{3.585568in}{1.910756in}}%
\pgfpathcurveto{\pgfqpoint{3.577754in}{1.918569in}}{\pgfqpoint{3.567155in}{1.922960in}}{\pgfqpoint{3.556105in}{1.922960in}}%
\pgfpathcurveto{\pgfqpoint{3.545055in}{1.922960in}}{\pgfqpoint{3.534456in}{1.918569in}}{\pgfqpoint{3.526642in}{1.910756in}}%
\pgfpathcurveto{\pgfqpoint{3.518829in}{1.902942in}}{\pgfqpoint{3.514438in}{1.892343in}}{\pgfqpoint{3.514438in}{1.881293in}}%
\pgfpathcurveto{\pgfqpoint{3.514438in}{1.870243in}}{\pgfqpoint{3.518829in}{1.859644in}}{\pgfqpoint{3.526642in}{1.851830in}}%
\pgfpathcurveto{\pgfqpoint{3.534456in}{1.844017in}}{\pgfqpoint{3.545055in}{1.839626in}}{\pgfqpoint{3.556105in}{1.839626in}}%
\pgfpathclose%
\pgfusepath{stroke,fill}%
\end{pgfscope}%
\begin{pgfscope}%
\pgfpathrectangle{\pgfqpoint{0.634105in}{0.521603in}}{\pgfqpoint{3.720000in}{3.020000in}} %
\pgfusepath{clip}%
\pgfsetbuttcap%
\pgfsetroundjoin%
\definecolor{currentfill}{rgb}{1.000000,1.000000,1.000000}%
\pgfsetfillcolor{currentfill}%
\pgfsetlinewidth{1.003750pt}%
\definecolor{currentstroke}{rgb}{0.000000,0.000000,0.000000}%
\pgfsetstrokecolor{currentstroke}%
\pgfsetdash{}{0pt}%
\pgfpathmoveto{\pgfqpoint{2.926462in}{1.486911in}}%
\pgfpathcurveto{\pgfqpoint{2.937513in}{1.486911in}}{\pgfqpoint{2.948112in}{1.491301in}}{\pgfqpoint{2.955925in}{1.499115in}}%
\pgfpathcurveto{\pgfqpoint{2.963739in}{1.506928in}}{\pgfqpoint{2.968129in}{1.517527in}}{\pgfqpoint{2.968129in}{1.528577in}}%
\pgfpathcurveto{\pgfqpoint{2.968129in}{1.539628in}}{\pgfqpoint{2.963739in}{1.550227in}}{\pgfqpoint{2.955925in}{1.558040in}}%
\pgfpathcurveto{\pgfqpoint{2.948112in}{1.565854in}}{\pgfqpoint{2.937513in}{1.570244in}}{\pgfqpoint{2.926462in}{1.570244in}}%
\pgfpathcurveto{\pgfqpoint{2.915412in}{1.570244in}}{\pgfqpoint{2.904813in}{1.565854in}}{\pgfqpoint{2.897000in}{1.558040in}}%
\pgfpathcurveto{\pgfqpoint{2.889186in}{1.550227in}}{\pgfqpoint{2.884796in}{1.539628in}}{\pgfqpoint{2.884796in}{1.528577in}}%
\pgfpathcurveto{\pgfqpoint{2.884796in}{1.517527in}}{\pgfqpoint{2.889186in}{1.506928in}}{\pgfqpoint{2.897000in}{1.499115in}}%
\pgfpathcurveto{\pgfqpoint{2.904813in}{1.491301in}}{\pgfqpoint{2.915412in}{1.486911in}}{\pgfqpoint{2.926462in}{1.486911in}}%
\pgfpathclose%
\pgfusepath{stroke,fill}%
\end{pgfscope}%
\begin{pgfscope}%
\pgfpathrectangle{\pgfqpoint{0.634105in}{0.521603in}}{\pgfqpoint{3.720000in}{3.020000in}} %
\pgfusepath{clip}%
\pgfsetbuttcap%
\pgfsetroundjoin%
\definecolor{currentfill}{rgb}{1.000000,1.000000,1.000000}%
\pgfsetfillcolor{currentfill}%
\pgfsetlinewidth{1.003750pt}%
\definecolor{currentstroke}{rgb}{0.000000,0.000000,0.000000}%
\pgfsetstrokecolor{currentstroke}%
\pgfsetdash{}{0pt}%
\pgfpathmoveto{\pgfqpoint{1.190499in}{2.912413in}}%
\pgfpathcurveto{\pgfqpoint{1.201549in}{2.912413in}}{\pgfqpoint{1.212148in}{2.916804in}}{\pgfqpoint{1.219962in}{2.924617in}}%
\pgfpathcurveto{\pgfqpoint{1.227775in}{2.932431in}}{\pgfqpoint{1.232166in}{2.943030in}}{\pgfqpoint{1.232166in}{2.954080in}}%
\pgfpathcurveto{\pgfqpoint{1.232166in}{2.965130in}}{\pgfqpoint{1.227775in}{2.975729in}}{\pgfqpoint{1.219962in}{2.983543in}}%
\pgfpathcurveto{\pgfqpoint{1.212148in}{2.991356in}}{\pgfqpoint{1.201549in}{2.995747in}}{\pgfqpoint{1.190499in}{2.995747in}}%
\pgfpathcurveto{\pgfqpoint{1.179449in}{2.995747in}}{\pgfqpoint{1.168850in}{2.991356in}}{\pgfqpoint{1.161036in}{2.983543in}}%
\pgfpathcurveto{\pgfqpoint{1.153222in}{2.975729in}}{\pgfqpoint{1.148832in}{2.965130in}}{\pgfqpoint{1.148832in}{2.954080in}}%
\pgfpathcurveto{\pgfqpoint{1.148832in}{2.943030in}}{\pgfqpoint{1.153222in}{2.932431in}}{\pgfqpoint{1.161036in}{2.924617in}}%
\pgfpathcurveto{\pgfqpoint{1.168850in}{2.916804in}}{\pgfqpoint{1.179449in}{2.912413in}}{\pgfqpoint{1.190499in}{2.912413in}}%
\pgfpathclose%
\pgfusepath{stroke,fill}%
\end{pgfscope}%
\begin{pgfscope}%
\pgfpathrectangle{\pgfqpoint{0.634105in}{0.521603in}}{\pgfqpoint{3.720000in}{3.020000in}} %
\pgfusepath{clip}%
\pgfsetbuttcap%
\pgfsetroundjoin%
\definecolor{currentfill}{rgb}{1.000000,1.000000,1.000000}%
\pgfsetfillcolor{currentfill}%
\pgfsetlinewidth{1.003750pt}%
\definecolor{currentstroke}{rgb}{0.000000,0.000000,0.000000}%
\pgfsetstrokecolor{currentstroke}%
\pgfsetdash{}{0pt}%
\pgfpathmoveto{\pgfqpoint{2.312047in}{1.118990in}}%
\pgfpathcurveto{\pgfqpoint{2.323097in}{1.118990in}}{\pgfqpoint{2.333696in}{1.123380in}}{\pgfqpoint{2.341510in}{1.131194in}}%
\pgfpathcurveto{\pgfqpoint{2.349323in}{1.139007in}}{\pgfqpoint{2.353714in}{1.149606in}}{\pgfqpoint{2.353714in}{1.160656in}}%
\pgfpathcurveto{\pgfqpoint{2.353714in}{1.171707in}}{\pgfqpoint{2.349323in}{1.182306in}}{\pgfqpoint{2.341510in}{1.190119in}}%
\pgfpathcurveto{\pgfqpoint{2.333696in}{1.197933in}}{\pgfqpoint{2.323097in}{1.202323in}}{\pgfqpoint{2.312047in}{1.202323in}}%
\pgfpathcurveto{\pgfqpoint{2.300997in}{1.202323in}}{\pgfqpoint{2.290398in}{1.197933in}}{\pgfqpoint{2.282584in}{1.190119in}}%
\pgfpathcurveto{\pgfqpoint{2.274770in}{1.182306in}}{\pgfqpoint{2.270380in}{1.171707in}}{\pgfqpoint{2.270380in}{1.160656in}}%
\pgfpathcurveto{\pgfqpoint{2.270380in}{1.149606in}}{\pgfqpoint{2.274770in}{1.139007in}}{\pgfqpoint{2.282584in}{1.131194in}}%
\pgfpathcurveto{\pgfqpoint{2.290398in}{1.123380in}}{\pgfqpoint{2.300997in}{1.118990in}}{\pgfqpoint{2.312047in}{1.118990in}}%
\pgfpathclose%
\pgfusepath{stroke,fill}%
\end{pgfscope}%
\begin{pgfscope}%
\pgfpathrectangle{\pgfqpoint{0.634105in}{0.521603in}}{\pgfqpoint{3.720000in}{3.020000in}} %
\pgfusepath{clip}%
\pgfsetbuttcap%
\pgfsetroundjoin%
\definecolor{currentfill}{rgb}{1.000000,1.000000,1.000000}%
\pgfsetfillcolor{currentfill}%
\pgfsetlinewidth{1.003750pt}%
\definecolor{currentstroke}{rgb}{0.000000,0.000000,0.000000}%
\pgfsetstrokecolor{currentstroke}%
\pgfsetdash{}{0pt}%
\pgfpathmoveto{\pgfqpoint{2.024287in}{1.311281in}}%
\pgfpathcurveto{\pgfqpoint{2.035338in}{1.311281in}}{\pgfqpoint{2.045937in}{1.315671in}}{\pgfqpoint{2.053750in}{1.323484in}}%
\pgfpathcurveto{\pgfqpoint{2.061564in}{1.331298in}}{\pgfqpoint{2.065954in}{1.341897in}}{\pgfqpoint{2.065954in}{1.352947in}}%
\pgfpathcurveto{\pgfqpoint{2.065954in}{1.363997in}}{\pgfqpoint{2.061564in}{1.374596in}}{\pgfqpoint{2.053750in}{1.382410in}}%
\pgfpathcurveto{\pgfqpoint{2.045937in}{1.390224in}}{\pgfqpoint{2.035338in}{1.394614in}}{\pgfqpoint{2.024287in}{1.394614in}}%
\pgfpathcurveto{\pgfqpoint{2.013237in}{1.394614in}}{\pgfqpoint{2.002638in}{1.390224in}}{\pgfqpoint{1.994825in}{1.382410in}}%
\pgfpathcurveto{\pgfqpoint{1.987011in}{1.374596in}}{\pgfqpoint{1.982621in}{1.363997in}}{\pgfqpoint{1.982621in}{1.352947in}}%
\pgfpathcurveto{\pgfqpoint{1.982621in}{1.341897in}}{\pgfqpoint{1.987011in}{1.331298in}}{\pgfqpoint{1.994825in}{1.323484in}}%
\pgfpathcurveto{\pgfqpoint{2.002638in}{1.315671in}}{\pgfqpoint{2.013237in}{1.311281in}}{\pgfqpoint{2.024287in}{1.311281in}}%
\pgfpathclose%
\pgfusepath{stroke,fill}%
\end{pgfscope}%
\begin{pgfscope}%
\pgfpathrectangle{\pgfqpoint{0.634105in}{0.521603in}}{\pgfqpoint{3.720000in}{3.020000in}} %
\pgfusepath{clip}%
\pgfsetbuttcap%
\pgfsetroundjoin%
\definecolor{currentfill}{rgb}{1.000000,1.000000,1.000000}%
\pgfsetfillcolor{currentfill}%
\pgfsetlinewidth{1.003750pt}%
\definecolor{currentstroke}{rgb}{0.000000,0.000000,0.000000}%
\pgfsetstrokecolor{currentstroke}%
\pgfsetdash{}{0pt}%
\pgfpathmoveto{\pgfqpoint{1.154746in}{2.170547in}}%
\pgfpathcurveto{\pgfqpoint{1.165796in}{2.170547in}}{\pgfqpoint{1.176395in}{2.174937in}}{\pgfqpoint{1.184209in}{2.182751in}}%
\pgfpathcurveto{\pgfqpoint{1.192022in}{2.190564in}}{\pgfqpoint{1.196413in}{2.201163in}}{\pgfqpoint{1.196413in}{2.212213in}}%
\pgfpathcurveto{\pgfqpoint{1.196413in}{2.223264in}}{\pgfqpoint{1.192022in}{2.233863in}}{\pgfqpoint{1.184209in}{2.241676in}}%
\pgfpathcurveto{\pgfqpoint{1.176395in}{2.249490in}}{\pgfqpoint{1.165796in}{2.253880in}}{\pgfqpoint{1.154746in}{2.253880in}}%
\pgfpathcurveto{\pgfqpoint{1.143696in}{2.253880in}}{\pgfqpoint{1.133097in}{2.249490in}}{\pgfqpoint{1.125283in}{2.241676in}}%
\pgfpathcurveto{\pgfqpoint{1.117470in}{2.233863in}}{\pgfqpoint{1.113079in}{2.223264in}}{\pgfqpoint{1.113079in}{2.212213in}}%
\pgfpathcurveto{\pgfqpoint{1.113079in}{2.201163in}}{\pgfqpoint{1.117470in}{2.190564in}}{\pgfqpoint{1.125283in}{2.182751in}}%
\pgfpathcurveto{\pgfqpoint{1.133097in}{2.174937in}}{\pgfqpoint{1.143696in}{2.170547in}}{\pgfqpoint{1.154746in}{2.170547in}}%
\pgfpathclose%
\pgfusepath{stroke,fill}%
\end{pgfscope}%
\begin{pgfscope}%
\pgfpathrectangle{\pgfqpoint{0.634105in}{0.521603in}}{\pgfqpoint{3.720000in}{3.020000in}} %
\pgfusepath{clip}%
\pgfsetbuttcap%
\pgfsetroundjoin%
\definecolor{currentfill}{rgb}{1.000000,1.000000,1.000000}%
\pgfsetfillcolor{currentfill}%
\pgfsetlinewidth{1.003750pt}%
\definecolor{currentstroke}{rgb}{0.000000,0.000000,0.000000}%
\pgfsetstrokecolor{currentstroke}%
\pgfsetdash{}{0pt}%
\pgfpathmoveto{\pgfqpoint{2.182791in}{0.747687in}}%
\pgfpathcurveto{\pgfqpoint{2.193841in}{0.747687in}}{\pgfqpoint{2.204440in}{0.752077in}}{\pgfqpoint{2.212254in}{0.759891in}}%
\pgfpathcurveto{\pgfqpoint{2.220067in}{0.767704in}}{\pgfqpoint{2.224457in}{0.778303in}}{\pgfqpoint{2.224457in}{0.789353in}}%
\pgfpathcurveto{\pgfqpoint{2.224457in}{0.800403in}}{\pgfqpoint{2.220067in}{0.811002in}}{\pgfqpoint{2.212254in}{0.818816in}}%
\pgfpathcurveto{\pgfqpoint{2.204440in}{0.826630in}}{\pgfqpoint{2.193841in}{0.831020in}}{\pgfqpoint{2.182791in}{0.831020in}}%
\pgfpathcurveto{\pgfqpoint{2.171741in}{0.831020in}}{\pgfqpoint{2.161142in}{0.826630in}}{\pgfqpoint{2.153328in}{0.818816in}}%
\pgfpathcurveto{\pgfqpoint{2.145514in}{0.811002in}}{\pgfqpoint{2.141124in}{0.800403in}}{\pgfqpoint{2.141124in}{0.789353in}}%
\pgfpathcurveto{\pgfqpoint{2.141124in}{0.778303in}}{\pgfqpoint{2.145514in}{0.767704in}}{\pgfqpoint{2.153328in}{0.759891in}}%
\pgfpathcurveto{\pgfqpoint{2.161142in}{0.752077in}}{\pgfqpoint{2.171741in}{0.747687in}}{\pgfqpoint{2.182791in}{0.747687in}}%
\pgfpathclose%
\pgfusepath{stroke,fill}%
\end{pgfscope}%
\begin{pgfscope}%
\pgfsetbuttcap%
\pgfsetroundjoin%
\definecolor{currentfill}{rgb}{0.000000,0.000000,0.000000}%
\pgfsetfillcolor{currentfill}%
\pgfsetlinewidth{0.803000pt}%
\definecolor{currentstroke}{rgb}{0.000000,0.000000,0.000000}%
\pgfsetstrokecolor{currentstroke}%
\pgfsetdash{}{0pt}%
\pgfsys@defobject{currentmarker}{\pgfqpoint{0.000000in}{-0.048611in}}{\pgfqpoint{0.000000in}{0.000000in}}{%
\pgfpathmoveto{\pgfqpoint{0.000000in}{0.000000in}}%
\pgfpathlineto{\pgfqpoint{0.000000in}{-0.048611in}}%
\pgfusepath{stroke,fill}%
}%
\begin{pgfscope}%
\pgfsys@transformshift{0.674322in}{0.521603in}%
\pgfsys@useobject{currentmarker}{}%
\end{pgfscope}%
\end{pgfscope}%
\begin{pgfscope}%
\pgftext[x=0.674322in,y=0.424381in,,top]{\rmfamily\fontsize{10.000000}{12.000000}\selectfont \(\displaystyle 0.4\)}%
\end{pgfscope}%
\begin{pgfscope}%
\pgfsetbuttcap%
\pgfsetroundjoin%
\definecolor{currentfill}{rgb}{0.000000,0.000000,0.000000}%
\pgfsetfillcolor{currentfill}%
\pgfsetlinewidth{0.803000pt}%
\definecolor{currentstroke}{rgb}{0.000000,0.000000,0.000000}%
\pgfsetstrokecolor{currentstroke}%
\pgfsetdash{}{0pt}%
\pgfsys@defobject{currentmarker}{\pgfqpoint{0.000000in}{-0.048611in}}{\pgfqpoint{0.000000in}{0.000000in}}{%
\pgfpathmoveto{\pgfqpoint{0.000000in}{0.000000in}}%
\pgfpathlineto{\pgfqpoint{0.000000in}{-0.048611in}}%
\pgfusepath{stroke,fill}%
}%
\begin{pgfscope}%
\pgfsys@transformshift{1.156200in}{0.521603in}%
\pgfsys@useobject{currentmarker}{}%
\end{pgfscope}%
\end{pgfscope}%
\begin{pgfscope}%
\pgftext[x=1.156200in,y=0.424381in,,top]{\rmfamily\fontsize{10.000000}{12.000000}\selectfont \(\displaystyle 0.6\)}%
\end{pgfscope}%
\begin{pgfscope}%
\pgfsetbuttcap%
\pgfsetroundjoin%
\definecolor{currentfill}{rgb}{0.000000,0.000000,0.000000}%
\pgfsetfillcolor{currentfill}%
\pgfsetlinewidth{0.803000pt}%
\definecolor{currentstroke}{rgb}{0.000000,0.000000,0.000000}%
\pgfsetstrokecolor{currentstroke}%
\pgfsetdash{}{0pt}%
\pgfsys@defobject{currentmarker}{\pgfqpoint{0.000000in}{-0.048611in}}{\pgfqpoint{0.000000in}{0.000000in}}{%
\pgfpathmoveto{\pgfqpoint{0.000000in}{0.000000in}}%
\pgfpathlineto{\pgfqpoint{0.000000in}{-0.048611in}}%
\pgfusepath{stroke,fill}%
}%
\begin{pgfscope}%
\pgfsys@transformshift{1.638079in}{0.521603in}%
\pgfsys@useobject{currentmarker}{}%
\end{pgfscope}%
\end{pgfscope}%
\begin{pgfscope}%
\pgftext[x=1.638079in,y=0.424381in,,top]{\rmfamily\fontsize{10.000000}{12.000000}\selectfont \(\displaystyle 0.8\)}%
\end{pgfscope}%
\begin{pgfscope}%
\pgfsetbuttcap%
\pgfsetroundjoin%
\definecolor{currentfill}{rgb}{0.000000,0.000000,0.000000}%
\pgfsetfillcolor{currentfill}%
\pgfsetlinewidth{0.803000pt}%
\definecolor{currentstroke}{rgb}{0.000000,0.000000,0.000000}%
\pgfsetstrokecolor{currentstroke}%
\pgfsetdash{}{0pt}%
\pgfsys@defobject{currentmarker}{\pgfqpoint{0.000000in}{-0.048611in}}{\pgfqpoint{0.000000in}{0.000000in}}{%
\pgfpathmoveto{\pgfqpoint{0.000000in}{0.000000in}}%
\pgfpathlineto{\pgfqpoint{0.000000in}{-0.048611in}}%
\pgfusepath{stroke,fill}%
}%
\begin{pgfscope}%
\pgfsys@transformshift{2.119958in}{0.521603in}%
\pgfsys@useobject{currentmarker}{}%
\end{pgfscope}%
\end{pgfscope}%
\begin{pgfscope}%
\pgftext[x=2.119958in,y=0.424381in,,top]{\rmfamily\fontsize{10.000000}{12.000000}\selectfont \(\displaystyle 1.0\)}%
\end{pgfscope}%
\begin{pgfscope}%
\pgfsetbuttcap%
\pgfsetroundjoin%
\definecolor{currentfill}{rgb}{0.000000,0.000000,0.000000}%
\pgfsetfillcolor{currentfill}%
\pgfsetlinewidth{0.803000pt}%
\definecolor{currentstroke}{rgb}{0.000000,0.000000,0.000000}%
\pgfsetstrokecolor{currentstroke}%
\pgfsetdash{}{0pt}%
\pgfsys@defobject{currentmarker}{\pgfqpoint{0.000000in}{-0.048611in}}{\pgfqpoint{0.000000in}{0.000000in}}{%
\pgfpathmoveto{\pgfqpoint{0.000000in}{0.000000in}}%
\pgfpathlineto{\pgfqpoint{0.000000in}{-0.048611in}}%
\pgfusepath{stroke,fill}%
}%
\begin{pgfscope}%
\pgfsys@transformshift{2.601837in}{0.521603in}%
\pgfsys@useobject{currentmarker}{}%
\end{pgfscope}%
\end{pgfscope}%
\begin{pgfscope}%
\pgftext[x=2.601837in,y=0.424381in,,top]{\rmfamily\fontsize{10.000000}{12.000000}\selectfont \(\displaystyle 1.2\)}%
\end{pgfscope}%
\begin{pgfscope}%
\pgfsetbuttcap%
\pgfsetroundjoin%
\definecolor{currentfill}{rgb}{0.000000,0.000000,0.000000}%
\pgfsetfillcolor{currentfill}%
\pgfsetlinewidth{0.803000pt}%
\definecolor{currentstroke}{rgb}{0.000000,0.000000,0.000000}%
\pgfsetstrokecolor{currentstroke}%
\pgfsetdash{}{0pt}%
\pgfsys@defobject{currentmarker}{\pgfqpoint{0.000000in}{-0.048611in}}{\pgfqpoint{0.000000in}{0.000000in}}{%
\pgfpathmoveto{\pgfqpoint{0.000000in}{0.000000in}}%
\pgfpathlineto{\pgfqpoint{0.000000in}{-0.048611in}}%
\pgfusepath{stroke,fill}%
}%
\begin{pgfscope}%
\pgfsys@transformshift{3.083716in}{0.521603in}%
\pgfsys@useobject{currentmarker}{}%
\end{pgfscope}%
\end{pgfscope}%
\begin{pgfscope}%
\pgftext[x=3.083716in,y=0.424381in,,top]{\rmfamily\fontsize{10.000000}{12.000000}\selectfont \(\displaystyle 1.4\)}%
\end{pgfscope}%
\begin{pgfscope}%
\pgfsetbuttcap%
\pgfsetroundjoin%
\definecolor{currentfill}{rgb}{0.000000,0.000000,0.000000}%
\pgfsetfillcolor{currentfill}%
\pgfsetlinewidth{0.803000pt}%
\definecolor{currentstroke}{rgb}{0.000000,0.000000,0.000000}%
\pgfsetstrokecolor{currentstroke}%
\pgfsetdash{}{0pt}%
\pgfsys@defobject{currentmarker}{\pgfqpoint{0.000000in}{-0.048611in}}{\pgfqpoint{0.000000in}{0.000000in}}{%
\pgfpathmoveto{\pgfqpoint{0.000000in}{0.000000in}}%
\pgfpathlineto{\pgfqpoint{0.000000in}{-0.048611in}}%
\pgfusepath{stroke,fill}%
}%
\begin{pgfscope}%
\pgfsys@transformshift{3.565595in}{0.521603in}%
\pgfsys@useobject{currentmarker}{}%
\end{pgfscope}%
\end{pgfscope}%
\begin{pgfscope}%
\pgftext[x=3.565595in,y=0.424381in,,top]{\rmfamily\fontsize{10.000000}{12.000000}\selectfont \(\displaystyle 1.6\)}%
\end{pgfscope}%
\begin{pgfscope}%
\pgfsetbuttcap%
\pgfsetroundjoin%
\definecolor{currentfill}{rgb}{0.000000,0.000000,0.000000}%
\pgfsetfillcolor{currentfill}%
\pgfsetlinewidth{0.803000pt}%
\definecolor{currentstroke}{rgb}{0.000000,0.000000,0.000000}%
\pgfsetstrokecolor{currentstroke}%
\pgfsetdash{}{0pt}%
\pgfsys@defobject{currentmarker}{\pgfqpoint{0.000000in}{-0.048611in}}{\pgfqpoint{0.000000in}{0.000000in}}{%
\pgfpathmoveto{\pgfqpoint{0.000000in}{0.000000in}}%
\pgfpathlineto{\pgfqpoint{0.000000in}{-0.048611in}}%
\pgfusepath{stroke,fill}%
}%
\begin{pgfscope}%
\pgfsys@transformshift{4.047474in}{0.521603in}%
\pgfsys@useobject{currentmarker}{}%
\end{pgfscope}%
\end{pgfscope}%
\begin{pgfscope}%
\pgftext[x=4.047474in,y=0.424381in,,top]{\rmfamily\fontsize{10.000000}{12.000000}\selectfont \(\displaystyle 1.8\)}%
\end{pgfscope}%
\begin{pgfscope}%
\pgftext[x=2.494105in,y=0.234413in,,top]{\rmfamily\fontsize{10.000000}{12.000000}\selectfont \(\displaystyle \varphi_s\) (kV)}%
\end{pgfscope}%
\begin{pgfscope}%
\pgfsetbuttcap%
\pgfsetroundjoin%
\definecolor{currentfill}{rgb}{0.000000,0.000000,0.000000}%
\pgfsetfillcolor{currentfill}%
\pgfsetlinewidth{0.803000pt}%
\definecolor{currentstroke}{rgb}{0.000000,0.000000,0.000000}%
\pgfsetstrokecolor{currentstroke}%
\pgfsetdash{}{0pt}%
\pgfsys@defobject{currentmarker}{\pgfqpoint{-0.048611in}{0.000000in}}{\pgfqpoint{0.000000in}{0.000000in}}{%
\pgfpathmoveto{\pgfqpoint{0.000000in}{0.000000in}}%
\pgfpathlineto{\pgfqpoint{-0.048611in}{0.000000in}}%
\pgfusepath{stroke,fill}%
}%
\begin{pgfscope}%
\pgfsys@transformshift{0.634105in}{0.571135in}%
\pgfsys@useobject{currentmarker}{}%
\end{pgfscope}%
\end{pgfscope}%
\begin{pgfscope}%
\pgftext[x=0.289968in,y=0.518373in,left,base]{\rmfamily\fontsize{10.000000}{12.000000}\selectfont \(\displaystyle 0.00\)}%
\end{pgfscope}%
\begin{pgfscope}%
\pgfsetbuttcap%
\pgfsetroundjoin%
\definecolor{currentfill}{rgb}{0.000000,0.000000,0.000000}%
\pgfsetfillcolor{currentfill}%
\pgfsetlinewidth{0.803000pt}%
\definecolor{currentstroke}{rgb}{0.000000,0.000000,0.000000}%
\pgfsetstrokecolor{currentstroke}%
\pgfsetdash{}{0pt}%
\pgfsys@defobject{currentmarker}{\pgfqpoint{-0.048611in}{0.000000in}}{\pgfqpoint{0.000000in}{0.000000in}}{%
\pgfpathmoveto{\pgfqpoint{0.000000in}{0.000000in}}%
\pgfpathlineto{\pgfqpoint{-0.048611in}{0.000000in}}%
\pgfusepath{stroke,fill}%
}%
\begin{pgfscope}%
\pgfsys@transformshift{0.634105in}{0.942149in}%
\pgfsys@useobject{currentmarker}{}%
\end{pgfscope}%
\end{pgfscope}%
\begin{pgfscope}%
\pgftext[x=0.289968in,y=0.889387in,left,base]{\rmfamily\fontsize{10.000000}{12.000000}\selectfont \(\displaystyle 0.05\)}%
\end{pgfscope}%
\begin{pgfscope}%
\pgfsetbuttcap%
\pgfsetroundjoin%
\definecolor{currentfill}{rgb}{0.000000,0.000000,0.000000}%
\pgfsetfillcolor{currentfill}%
\pgfsetlinewidth{0.803000pt}%
\definecolor{currentstroke}{rgb}{0.000000,0.000000,0.000000}%
\pgfsetstrokecolor{currentstroke}%
\pgfsetdash{}{0pt}%
\pgfsys@defobject{currentmarker}{\pgfqpoint{-0.048611in}{0.000000in}}{\pgfqpoint{0.000000in}{0.000000in}}{%
\pgfpathmoveto{\pgfqpoint{0.000000in}{0.000000in}}%
\pgfpathlineto{\pgfqpoint{-0.048611in}{0.000000in}}%
\pgfusepath{stroke,fill}%
}%
\begin{pgfscope}%
\pgfsys@transformshift{0.634105in}{1.313162in}%
\pgfsys@useobject{currentmarker}{}%
\end{pgfscope}%
\end{pgfscope}%
\begin{pgfscope}%
\pgftext[x=0.289968in,y=1.260401in,left,base]{\rmfamily\fontsize{10.000000}{12.000000}\selectfont \(\displaystyle 0.10\)}%
\end{pgfscope}%
\begin{pgfscope}%
\pgfsetbuttcap%
\pgfsetroundjoin%
\definecolor{currentfill}{rgb}{0.000000,0.000000,0.000000}%
\pgfsetfillcolor{currentfill}%
\pgfsetlinewidth{0.803000pt}%
\definecolor{currentstroke}{rgb}{0.000000,0.000000,0.000000}%
\pgfsetstrokecolor{currentstroke}%
\pgfsetdash{}{0pt}%
\pgfsys@defobject{currentmarker}{\pgfqpoint{-0.048611in}{0.000000in}}{\pgfqpoint{0.000000in}{0.000000in}}{%
\pgfpathmoveto{\pgfqpoint{0.000000in}{0.000000in}}%
\pgfpathlineto{\pgfqpoint{-0.048611in}{0.000000in}}%
\pgfusepath{stroke,fill}%
}%
\begin{pgfscope}%
\pgfsys@transformshift{0.634105in}{1.684176in}%
\pgfsys@useobject{currentmarker}{}%
\end{pgfscope}%
\end{pgfscope}%
\begin{pgfscope}%
\pgftext[x=0.289968in,y=1.631415in,left,base]{\rmfamily\fontsize{10.000000}{12.000000}\selectfont \(\displaystyle 0.15\)}%
\end{pgfscope}%
\begin{pgfscope}%
\pgfsetbuttcap%
\pgfsetroundjoin%
\definecolor{currentfill}{rgb}{0.000000,0.000000,0.000000}%
\pgfsetfillcolor{currentfill}%
\pgfsetlinewidth{0.803000pt}%
\definecolor{currentstroke}{rgb}{0.000000,0.000000,0.000000}%
\pgfsetstrokecolor{currentstroke}%
\pgfsetdash{}{0pt}%
\pgfsys@defobject{currentmarker}{\pgfqpoint{-0.048611in}{0.000000in}}{\pgfqpoint{0.000000in}{0.000000in}}{%
\pgfpathmoveto{\pgfqpoint{0.000000in}{0.000000in}}%
\pgfpathlineto{\pgfqpoint{-0.048611in}{0.000000in}}%
\pgfusepath{stroke,fill}%
}%
\begin{pgfscope}%
\pgfsys@transformshift{0.634105in}{2.055190in}%
\pgfsys@useobject{currentmarker}{}%
\end{pgfscope}%
\end{pgfscope}%
\begin{pgfscope}%
\pgftext[x=0.289968in,y=2.002429in,left,base]{\rmfamily\fontsize{10.000000}{12.000000}\selectfont \(\displaystyle 0.20\)}%
\end{pgfscope}%
\begin{pgfscope}%
\pgfsetbuttcap%
\pgfsetroundjoin%
\definecolor{currentfill}{rgb}{0.000000,0.000000,0.000000}%
\pgfsetfillcolor{currentfill}%
\pgfsetlinewidth{0.803000pt}%
\definecolor{currentstroke}{rgb}{0.000000,0.000000,0.000000}%
\pgfsetstrokecolor{currentstroke}%
\pgfsetdash{}{0pt}%
\pgfsys@defobject{currentmarker}{\pgfqpoint{-0.048611in}{0.000000in}}{\pgfqpoint{0.000000in}{0.000000in}}{%
\pgfpathmoveto{\pgfqpoint{0.000000in}{0.000000in}}%
\pgfpathlineto{\pgfqpoint{-0.048611in}{0.000000in}}%
\pgfusepath{stroke,fill}%
}%
\begin{pgfscope}%
\pgfsys@transformshift{0.634105in}{2.426204in}%
\pgfsys@useobject{currentmarker}{}%
\end{pgfscope}%
\end{pgfscope}%
\begin{pgfscope}%
\pgftext[x=0.289968in,y=2.373443in,left,base]{\rmfamily\fontsize{10.000000}{12.000000}\selectfont \(\displaystyle 0.25\)}%
\end{pgfscope}%
\begin{pgfscope}%
\pgfsetbuttcap%
\pgfsetroundjoin%
\definecolor{currentfill}{rgb}{0.000000,0.000000,0.000000}%
\pgfsetfillcolor{currentfill}%
\pgfsetlinewidth{0.803000pt}%
\definecolor{currentstroke}{rgb}{0.000000,0.000000,0.000000}%
\pgfsetstrokecolor{currentstroke}%
\pgfsetdash{}{0pt}%
\pgfsys@defobject{currentmarker}{\pgfqpoint{-0.048611in}{0.000000in}}{\pgfqpoint{0.000000in}{0.000000in}}{%
\pgfpathmoveto{\pgfqpoint{0.000000in}{0.000000in}}%
\pgfpathlineto{\pgfqpoint{-0.048611in}{0.000000in}}%
\pgfusepath{stroke,fill}%
}%
\begin{pgfscope}%
\pgfsys@transformshift{0.634105in}{2.797218in}%
\pgfsys@useobject{currentmarker}{}%
\end{pgfscope}%
\end{pgfscope}%
\begin{pgfscope}%
\pgftext[x=0.289968in,y=2.744457in,left,base]{\rmfamily\fontsize{10.000000}{12.000000}\selectfont \(\displaystyle 0.30\)}%
\end{pgfscope}%
\begin{pgfscope}%
\pgfsetbuttcap%
\pgfsetroundjoin%
\definecolor{currentfill}{rgb}{0.000000,0.000000,0.000000}%
\pgfsetfillcolor{currentfill}%
\pgfsetlinewidth{0.803000pt}%
\definecolor{currentstroke}{rgb}{0.000000,0.000000,0.000000}%
\pgfsetstrokecolor{currentstroke}%
\pgfsetdash{}{0pt}%
\pgfsys@defobject{currentmarker}{\pgfqpoint{-0.048611in}{0.000000in}}{\pgfqpoint{0.000000in}{0.000000in}}{%
\pgfpathmoveto{\pgfqpoint{0.000000in}{0.000000in}}%
\pgfpathlineto{\pgfqpoint{-0.048611in}{0.000000in}}%
\pgfusepath{stroke,fill}%
}%
\begin{pgfscope}%
\pgfsys@transformshift{0.634105in}{3.168232in}%
\pgfsys@useobject{currentmarker}{}%
\end{pgfscope}%
\end{pgfscope}%
\begin{pgfscope}%
\pgftext[x=0.289968in,y=3.115470in,left,base]{\rmfamily\fontsize{10.000000}{12.000000}\selectfont \(\displaystyle 0.35\)}%
\end{pgfscope}%
\begin{pgfscope}%
\pgfsetbuttcap%
\pgfsetroundjoin%
\definecolor{currentfill}{rgb}{0.000000,0.000000,0.000000}%
\pgfsetfillcolor{currentfill}%
\pgfsetlinewidth{0.803000pt}%
\definecolor{currentstroke}{rgb}{0.000000,0.000000,0.000000}%
\pgfsetstrokecolor{currentstroke}%
\pgfsetdash{}{0pt}%
\pgfsys@defobject{currentmarker}{\pgfqpoint{-0.048611in}{0.000000in}}{\pgfqpoint{0.000000in}{0.000000in}}{%
\pgfpathmoveto{\pgfqpoint{0.000000in}{0.000000in}}%
\pgfpathlineto{\pgfqpoint{-0.048611in}{0.000000in}}%
\pgfusepath{stroke,fill}%
}%
\begin{pgfscope}%
\pgfsys@transformshift{0.634105in}{3.539246in}%
\pgfsys@useobject{currentmarker}{}%
\end{pgfscope}%
\end{pgfscope}%
\begin{pgfscope}%
\pgftext[x=0.289968in,y=3.486484in,left,base]{\rmfamily\fontsize{10.000000}{12.000000}\selectfont \(\displaystyle 0.40\)}%
\end{pgfscope}%
\begin{pgfscope}%
\pgftext[x=0.234413in,y=2.031603in,,bottom,rotate=90.000000]{\rmfamily\fontsize{10.000000}{12.000000}\selectfont \(\displaystyle V_d\) (mL)}%
\end{pgfscope}%
\begin{pgfscope}%
\pgfsetrectcap%
\pgfsetmiterjoin%
\pgfsetlinewidth{0.803000pt}%
\definecolor{currentstroke}{rgb}{0.000000,0.000000,0.000000}%
\pgfsetstrokecolor{currentstroke}%
\pgfsetdash{}{0pt}%
\pgfpathmoveto{\pgfqpoint{0.634105in}{0.521603in}}%
\pgfpathlineto{\pgfqpoint{0.634105in}{3.541603in}}%
\pgfusepath{stroke}%
\end{pgfscope}%
\begin{pgfscope}%
\pgfsetrectcap%
\pgfsetmiterjoin%
\pgfsetlinewidth{0.803000pt}%
\definecolor{currentstroke}{rgb}{0.000000,0.000000,0.000000}%
\pgfsetstrokecolor{currentstroke}%
\pgfsetdash{}{0pt}%
\pgfpathmoveto{\pgfqpoint{4.354105in}{0.521603in}}%
\pgfpathlineto{\pgfqpoint{4.354105in}{3.541603in}}%
\pgfusepath{stroke}%
\end{pgfscope}%
\begin{pgfscope}%
\pgfsetrectcap%
\pgfsetmiterjoin%
\pgfsetlinewidth{0.803000pt}%
\definecolor{currentstroke}{rgb}{0.000000,0.000000,0.000000}%
\pgfsetstrokecolor{currentstroke}%
\pgfsetdash{}{0pt}%
\pgfpathmoveto{\pgfqpoint{0.634105in}{0.521603in}}%
\pgfpathlineto{\pgfqpoint{4.354105in}{0.521603in}}%
\pgfusepath{stroke}%
\end{pgfscope}%
\begin{pgfscope}%
\pgfsetrectcap%
\pgfsetmiterjoin%
\pgfsetlinewidth{0.803000pt}%
\definecolor{currentstroke}{rgb}{0.000000,0.000000,0.000000}%
\pgfsetstrokecolor{currentstroke}%
\pgfsetdash{}{0pt}%
\pgfpathmoveto{\pgfqpoint{0.634105in}{3.541603in}}%
\pgfpathlineto{\pgfqpoint{4.354105in}{3.541603in}}%
\pgfusepath{stroke}%
\end{pgfscope}%
\begin{pgfscope}%
\pgfpathrectangle{\pgfqpoint{4.586605in}{0.521603in}}{\pgfqpoint{0.151000in}{3.020000in}} %
\pgfusepath{clip}%
\pgfsetbuttcap%
\pgfsetmiterjoin%
\definecolor{currentfill}{rgb}{1.000000,1.000000,1.000000}%
\pgfsetfillcolor{currentfill}%
\pgfsetlinewidth{0.010037pt}%
\definecolor{currentstroke}{rgb}{1.000000,1.000000,1.000000}%
\pgfsetstrokecolor{currentstroke}%
\pgfsetdash{}{0pt}%
\pgfpathmoveto{\pgfqpoint{4.586605in}{0.521603in}}%
\pgfpathlineto{\pgfqpoint{4.586605in}{0.953032in}}%
\pgfpathlineto{\pgfqpoint{4.586605in}{3.110175in}}%
\pgfpathlineto{\pgfqpoint{4.586605in}{3.541603in}}%
\pgfpathlineto{\pgfqpoint{4.737605in}{3.541603in}}%
\pgfpathlineto{\pgfqpoint{4.737605in}{3.110175in}}%
\pgfpathlineto{\pgfqpoint{4.737605in}{0.953032in}}%
\pgfpathlineto{\pgfqpoint{4.737605in}{0.521603in}}%
\pgfpathclose%
\pgfusepath{stroke,fill}%
\end{pgfscope}%
\begin{pgfscope}%
\pgfpathrectangle{\pgfqpoint{4.586605in}{0.521603in}}{\pgfqpoint{0.151000in}{3.020000in}} %
\pgfusepath{clip}%
\pgfsetbuttcap%
\pgfsetroundjoin%
\definecolor{currentfill}{rgb}{0.061765,0.061765,0.085934}%
\pgfsetfillcolor{currentfill}%
\pgfsetlinewidth{0.000000pt}%
\definecolor{currentstroke}{rgb}{0.000000,0.000000,0.000000}%
\pgfsetstrokecolor{currentstroke}%
\pgfsetdash{}{0pt}%
\pgfpathmoveto{\pgfqpoint{4.586605in}{0.521603in}}%
\pgfpathlineto{\pgfqpoint{4.737605in}{0.521603in}}%
\pgfpathlineto{\pgfqpoint{4.737605in}{0.953032in}}%
\pgfpathlineto{\pgfqpoint{4.586605in}{0.953032in}}%
\pgfpathlineto{\pgfqpoint{4.586605in}{0.521603in}}%
\pgfusepath{fill}%
\end{pgfscope}%
\begin{pgfscope}%
\pgfpathrectangle{\pgfqpoint{4.586605in}{0.521603in}}{\pgfqpoint{0.151000in}{3.020000in}} %
\pgfusepath{clip}%
\pgfsetbuttcap%
\pgfsetroundjoin%
\definecolor{currentfill}{rgb}{0.185294,0.185294,0.257801}%
\pgfsetfillcolor{currentfill}%
\pgfsetlinewidth{0.000000pt}%
\definecolor{currentstroke}{rgb}{0.000000,0.000000,0.000000}%
\pgfsetstrokecolor{currentstroke}%
\pgfsetdash{}{0pt}%
\pgfpathmoveto{\pgfqpoint{4.586605in}{0.953032in}}%
\pgfpathlineto{\pgfqpoint{4.737605in}{0.953032in}}%
\pgfpathlineto{\pgfqpoint{4.737605in}{1.384460in}}%
\pgfpathlineto{\pgfqpoint{4.586605in}{1.384460in}}%
\pgfpathlineto{\pgfqpoint{4.586605in}{0.953032in}}%
\pgfusepath{fill}%
\end{pgfscope}%
\begin{pgfscope}%
\pgfpathrectangle{\pgfqpoint{4.586605in}{0.521603in}}{\pgfqpoint{0.151000in}{3.020000in}} %
\pgfusepath{clip}%
\pgfsetbuttcap%
\pgfsetroundjoin%
\definecolor{currentfill}{rgb}{0.312255,0.312255,0.434442}%
\pgfsetfillcolor{currentfill}%
\pgfsetlinewidth{0.000000pt}%
\definecolor{currentstroke}{rgb}{0.000000,0.000000,0.000000}%
\pgfsetstrokecolor{currentstroke}%
\pgfsetdash{}{0pt}%
\pgfpathmoveto{\pgfqpoint{4.586605in}{1.384460in}}%
\pgfpathlineto{\pgfqpoint{4.737605in}{1.384460in}}%
\pgfpathlineto{\pgfqpoint{4.737605in}{1.815889in}}%
\pgfpathlineto{\pgfqpoint{4.586605in}{1.815889in}}%
\pgfpathlineto{\pgfqpoint{4.586605in}{1.384460in}}%
\pgfusepath{fill}%
\end{pgfscope}%
\begin{pgfscope}%
\pgfpathrectangle{\pgfqpoint{4.586605in}{0.521603in}}{\pgfqpoint{0.151000in}{3.020000in}} %
\pgfusepath{clip}%
\pgfsetbuttcap%
\pgfsetroundjoin%
\definecolor{currentfill}{rgb}{0.439216,0.484130,0.564216}%
\pgfsetfillcolor{currentfill}%
\pgfsetlinewidth{0.000000pt}%
\definecolor{currentstroke}{rgb}{0.000000,0.000000,0.000000}%
\pgfsetstrokecolor{currentstroke}%
\pgfsetdash{}{0pt}%
\pgfpathmoveto{\pgfqpoint{4.586605in}{1.815889in}}%
\pgfpathlineto{\pgfqpoint{4.737605in}{1.815889in}}%
\pgfpathlineto{\pgfqpoint{4.737605in}{2.247318in}}%
\pgfpathlineto{\pgfqpoint{4.586605in}{2.247318in}}%
\pgfpathlineto{\pgfqpoint{4.586605in}{1.815889in}}%
\pgfusepath{fill}%
\end{pgfscope}%
\begin{pgfscope}%
\pgfpathrectangle{\pgfqpoint{4.586605in}{0.521603in}}{\pgfqpoint{0.151000in}{3.020000in}} %
\pgfusepath{clip}%
\pgfsetbuttcap%
\pgfsetroundjoin%
\definecolor{currentfill}{rgb}{0.562745,0.653983,0.687745}%
\pgfsetfillcolor{currentfill}%
\pgfsetlinewidth{0.000000pt}%
\definecolor{currentstroke}{rgb}{0.000000,0.000000,0.000000}%
\pgfsetstrokecolor{currentstroke}%
\pgfsetdash{}{0pt}%
\pgfpathmoveto{\pgfqpoint{4.586605in}{2.247318in}}%
\pgfpathlineto{\pgfqpoint{4.737605in}{2.247318in}}%
\pgfpathlineto{\pgfqpoint{4.737605in}{2.678746in}}%
\pgfpathlineto{\pgfqpoint{4.586605in}{2.678746in}}%
\pgfpathlineto{\pgfqpoint{4.586605in}{2.247318in}}%
\pgfusepath{fill}%
\end{pgfscope}%
\begin{pgfscope}%
\pgfpathrectangle{\pgfqpoint{4.586605in}{0.521603in}}{\pgfqpoint{0.151000in}{3.020000in}} %
\pgfusepath{clip}%
\pgfsetbuttcap%
\pgfsetroundjoin%
\definecolor{currentfill}{rgb}{0.710478,0.814706,0.814706}%
\pgfsetfillcolor{currentfill}%
\pgfsetlinewidth{0.000000pt}%
\definecolor{currentstroke}{rgb}{0.000000,0.000000,0.000000}%
\pgfsetstrokecolor{currentstroke}%
\pgfsetdash{}{0pt}%
\pgfpathmoveto{\pgfqpoint{4.586605in}{2.678746in}}%
\pgfpathlineto{\pgfqpoint{4.737605in}{2.678746in}}%
\pgfpathlineto{\pgfqpoint{4.737605in}{3.110175in}}%
\pgfpathlineto{\pgfqpoint{4.586605in}{3.110175in}}%
\pgfpathlineto{\pgfqpoint{4.586605in}{2.678746in}}%
\pgfusepath{fill}%
\end{pgfscope}%
\begin{pgfscope}%
\pgfpathrectangle{\pgfqpoint{4.586605in}{0.521603in}}{\pgfqpoint{0.151000in}{3.020000in}} %
\pgfusepath{clip}%
\pgfsetbuttcap%
\pgfsetroundjoin%
\definecolor{currentfill}{rgb}{0.903493,0.938235,0.938235}%
\pgfsetfillcolor{currentfill}%
\pgfsetlinewidth{0.000000pt}%
\definecolor{currentstroke}{rgb}{0.000000,0.000000,0.000000}%
\pgfsetstrokecolor{currentstroke}%
\pgfsetdash{}{0pt}%
\pgfpathmoveto{\pgfqpoint{4.586605in}{3.110175in}}%
\pgfpathlineto{\pgfqpoint{4.737605in}{3.110175in}}%
\pgfpathlineto{\pgfqpoint{4.737605in}{3.541603in}}%
\pgfpathlineto{\pgfqpoint{4.586605in}{3.541603in}}%
\pgfpathlineto{\pgfqpoint{4.586605in}{3.110175in}}%
\pgfusepath{fill}%
\end{pgfscope}%
\begin{pgfscope}%
\pgfsetbuttcap%
\pgfsetroundjoin%
\definecolor{currentfill}{rgb}{0.000000,0.000000,0.000000}%
\pgfsetfillcolor{currentfill}%
\pgfsetlinewidth{0.803000pt}%
\definecolor{currentstroke}{rgb}{0.000000,0.000000,0.000000}%
\pgfsetstrokecolor{currentstroke}%
\pgfsetdash{}{0pt}%
\pgfsys@defobject{currentmarker}{\pgfqpoint{0.000000in}{0.000000in}}{\pgfqpoint{0.048611in}{0.000000in}}{%
\pgfpathmoveto{\pgfqpoint{0.000000in}{0.000000in}}%
\pgfpathlineto{\pgfqpoint{0.048611in}{0.000000in}}%
\pgfusepath{stroke,fill}%
}%
\begin{pgfscope}%
\pgfsys@transformshift{4.737605in}{0.521603in}%
\pgfsys@useobject{currentmarker}{}%
\end{pgfscope}%
\end{pgfscope}%
\begin{pgfscope}%
\pgftext[x=4.834827in,y=0.468842in,left,base]{\rmfamily\fontsize{10.000000}{12.000000}\selectfont \(\displaystyle 0\)}%
\end{pgfscope}%
\begin{pgfscope}%
\pgfsetbuttcap%
\pgfsetroundjoin%
\definecolor{currentfill}{rgb}{0.000000,0.000000,0.000000}%
\pgfsetfillcolor{currentfill}%
\pgfsetlinewidth{0.803000pt}%
\definecolor{currentstroke}{rgb}{0.000000,0.000000,0.000000}%
\pgfsetstrokecolor{currentstroke}%
\pgfsetdash{}{0pt}%
\pgfsys@defobject{currentmarker}{\pgfqpoint{0.000000in}{0.000000in}}{\pgfqpoint{0.048611in}{0.000000in}}{%
\pgfpathmoveto{\pgfqpoint{0.000000in}{0.000000in}}%
\pgfpathlineto{\pgfqpoint{0.048611in}{0.000000in}}%
\pgfusepath{stroke,fill}%
}%
\begin{pgfscope}%
\pgfsys@transformshift{4.737605in}{0.953032in}%
\pgfsys@useobject{currentmarker}{}%
\end{pgfscope}%
\end{pgfscope}%
\begin{pgfscope}%
\pgftext[x=4.834827in,y=0.900270in,left,base]{\rmfamily\fontsize{10.000000}{12.000000}\selectfont \(\displaystyle 1\)}%
\end{pgfscope}%
\begin{pgfscope}%
\pgfsetbuttcap%
\pgfsetroundjoin%
\definecolor{currentfill}{rgb}{0.000000,0.000000,0.000000}%
\pgfsetfillcolor{currentfill}%
\pgfsetlinewidth{0.803000pt}%
\definecolor{currentstroke}{rgb}{0.000000,0.000000,0.000000}%
\pgfsetstrokecolor{currentstroke}%
\pgfsetdash{}{0pt}%
\pgfsys@defobject{currentmarker}{\pgfqpoint{0.000000in}{0.000000in}}{\pgfqpoint{0.048611in}{0.000000in}}{%
\pgfpathmoveto{\pgfqpoint{0.000000in}{0.000000in}}%
\pgfpathlineto{\pgfqpoint{0.048611in}{0.000000in}}%
\pgfusepath{stroke,fill}%
}%
\begin{pgfscope}%
\pgfsys@transformshift{4.737605in}{1.384460in}%
\pgfsys@useobject{currentmarker}{}%
\end{pgfscope}%
\end{pgfscope}%
\begin{pgfscope}%
\pgftext[x=4.834827in,y=1.331699in,left,base]{\rmfamily\fontsize{10.000000}{12.000000}\selectfont \(\displaystyle 2\)}%
\end{pgfscope}%
\begin{pgfscope}%
\pgfsetbuttcap%
\pgfsetroundjoin%
\definecolor{currentfill}{rgb}{0.000000,0.000000,0.000000}%
\pgfsetfillcolor{currentfill}%
\pgfsetlinewidth{0.803000pt}%
\definecolor{currentstroke}{rgb}{0.000000,0.000000,0.000000}%
\pgfsetstrokecolor{currentstroke}%
\pgfsetdash{}{0pt}%
\pgfsys@defobject{currentmarker}{\pgfqpoint{0.000000in}{0.000000in}}{\pgfqpoint{0.048611in}{0.000000in}}{%
\pgfpathmoveto{\pgfqpoint{0.000000in}{0.000000in}}%
\pgfpathlineto{\pgfqpoint{0.048611in}{0.000000in}}%
\pgfusepath{stroke,fill}%
}%
\begin{pgfscope}%
\pgfsys@transformshift{4.737605in}{1.815889in}%
\pgfsys@useobject{currentmarker}{}%
\end{pgfscope}%
\end{pgfscope}%
\begin{pgfscope}%
\pgftext[x=4.834827in,y=1.763128in,left,base]{\rmfamily\fontsize{10.000000}{12.000000}\selectfont \(\displaystyle 3\)}%
\end{pgfscope}%
\begin{pgfscope}%
\pgfsetbuttcap%
\pgfsetroundjoin%
\definecolor{currentfill}{rgb}{0.000000,0.000000,0.000000}%
\pgfsetfillcolor{currentfill}%
\pgfsetlinewidth{0.803000pt}%
\definecolor{currentstroke}{rgb}{0.000000,0.000000,0.000000}%
\pgfsetstrokecolor{currentstroke}%
\pgfsetdash{}{0pt}%
\pgfsys@defobject{currentmarker}{\pgfqpoint{0.000000in}{0.000000in}}{\pgfqpoint{0.048611in}{0.000000in}}{%
\pgfpathmoveto{\pgfqpoint{0.000000in}{0.000000in}}%
\pgfpathlineto{\pgfqpoint{0.048611in}{0.000000in}}%
\pgfusepath{stroke,fill}%
}%
\begin{pgfscope}%
\pgfsys@transformshift{4.737605in}{2.247318in}%
\pgfsys@useobject{currentmarker}{}%
\end{pgfscope}%
\end{pgfscope}%
\begin{pgfscope}%
\pgftext[x=4.834827in,y=2.194556in,left,base]{\rmfamily\fontsize{10.000000}{12.000000}\selectfont \(\displaystyle 4\)}%
\end{pgfscope}%
\begin{pgfscope}%
\pgfsetbuttcap%
\pgfsetroundjoin%
\definecolor{currentfill}{rgb}{0.000000,0.000000,0.000000}%
\pgfsetfillcolor{currentfill}%
\pgfsetlinewidth{0.803000pt}%
\definecolor{currentstroke}{rgb}{0.000000,0.000000,0.000000}%
\pgfsetstrokecolor{currentstroke}%
\pgfsetdash{}{0pt}%
\pgfsys@defobject{currentmarker}{\pgfqpoint{0.000000in}{0.000000in}}{\pgfqpoint{0.048611in}{0.000000in}}{%
\pgfpathmoveto{\pgfqpoint{0.000000in}{0.000000in}}%
\pgfpathlineto{\pgfqpoint{0.048611in}{0.000000in}}%
\pgfusepath{stroke,fill}%
}%
\begin{pgfscope}%
\pgfsys@transformshift{4.737605in}{2.678746in}%
\pgfsys@useobject{currentmarker}{}%
\end{pgfscope}%
\end{pgfscope}%
\begin{pgfscope}%
\pgftext[x=4.834827in,y=2.625985in,left,base]{\rmfamily\fontsize{10.000000}{12.000000}\selectfont \(\displaystyle 5\)}%
\end{pgfscope}%
\begin{pgfscope}%
\pgfsetbuttcap%
\pgfsetroundjoin%
\definecolor{currentfill}{rgb}{0.000000,0.000000,0.000000}%
\pgfsetfillcolor{currentfill}%
\pgfsetlinewidth{0.803000pt}%
\definecolor{currentstroke}{rgb}{0.000000,0.000000,0.000000}%
\pgfsetstrokecolor{currentstroke}%
\pgfsetdash{}{0pt}%
\pgfsys@defobject{currentmarker}{\pgfqpoint{0.000000in}{0.000000in}}{\pgfqpoint{0.048611in}{0.000000in}}{%
\pgfpathmoveto{\pgfqpoint{0.000000in}{0.000000in}}%
\pgfpathlineto{\pgfqpoint{0.048611in}{0.000000in}}%
\pgfusepath{stroke,fill}%
}%
\begin{pgfscope}%
\pgfsys@transformshift{4.737605in}{3.110175in}%
\pgfsys@useobject{currentmarker}{}%
\end{pgfscope}%
\end{pgfscope}%
\begin{pgfscope}%
\pgftext[x=4.834827in,y=3.057413in,left,base]{\rmfamily\fontsize{10.000000}{12.000000}\selectfont \(\displaystyle 6\)}%
\end{pgfscope}%
\begin{pgfscope}%
\pgfsetbuttcap%
\pgfsetroundjoin%
\definecolor{currentfill}{rgb}{0.000000,0.000000,0.000000}%
\pgfsetfillcolor{currentfill}%
\pgfsetlinewidth{0.803000pt}%
\definecolor{currentstroke}{rgb}{0.000000,0.000000,0.000000}%
\pgfsetstrokecolor{currentstroke}%
\pgfsetdash{}{0pt}%
\pgfsys@defobject{currentmarker}{\pgfqpoint{0.000000in}{0.000000in}}{\pgfqpoint{0.048611in}{0.000000in}}{%
\pgfpathmoveto{\pgfqpoint{0.000000in}{0.000000in}}%
\pgfpathlineto{\pgfqpoint{0.048611in}{0.000000in}}%
\pgfusepath{stroke,fill}%
}%
\begin{pgfscope}%
\pgfsys@transformshift{4.737605in}{3.541603in}%
\pgfsys@useobject{currentmarker}{}%
\end{pgfscope}%
\end{pgfscope}%
\begin{pgfscope}%
\pgftext[x=4.834827in,y=3.488842in,left,base]{\rmfamily\fontsize{10.000000}{12.000000}\selectfont \(\displaystyle 7\)}%
\end{pgfscope}%
\begin{pgfscope}%
\pgftext[x=4.959827in,y=2.031603in,,top,rotate=90.000000]{\rmfamily\fontsize{10.000000}{12.000000}\selectfont \(\displaystyle q\) (C)}%
\end{pgfscope}%
\begin{pgfscope}%
\pgftext[x=4.737605in,y=3.583270in,right,base]{\rmfamily\fontsize{10.000000}{12.000000}\selectfont \(\displaystyle \times10^{-10}\)}%
\end{pgfscope}%
\begin{pgfscope}%
\pgfsetbuttcap%
\pgfsetmiterjoin%
\pgfsetlinewidth{0.803000pt}%
\definecolor{currentstroke}{rgb}{0.000000,0.000000,0.000000}%
\pgfsetstrokecolor{currentstroke}%
\pgfsetdash{}{0pt}%
\pgfpathmoveto{\pgfqpoint{4.586605in}{0.521603in}}%
\pgfpathlineto{\pgfqpoint{4.586605in}{0.953032in}}%
\pgfpathlineto{\pgfqpoint{4.586605in}{3.110175in}}%
\pgfpathlineto{\pgfqpoint{4.586605in}{3.541603in}}%
\pgfpathlineto{\pgfqpoint{4.737605in}{3.541603in}}%
\pgfpathlineto{\pgfqpoint{4.737605in}{3.110175in}}%
\pgfpathlineto{\pgfqpoint{4.737605in}{0.953032in}}%
\pgfpathlineto{\pgfqpoint{4.737605in}{0.521603in}}%
\pgfpathclose%
\pgfusepath{stroke}%
\end{pgfscope}%
\end{pgfpicture}%
\makeatother%
\endgroup%

    \caption{A simple EMA plot.\label{fig:charge}}
\end{figure}

A two-ways T-test comparison of charge distributions between the droplet bounce experiment and a corollary experiment with zero electric field at the time of droplet deposition on the superhydrophobic surface suggests that the droplet charge is induced by the electric field, rather than through contact charging on the PTFE layer ($t = 5.11, p = 0.0002$). The T-test informs us that the charge distribution  are about 5 times more different from each other as they are within each other, and there is a 0.02$\%$ probability that this result happened by chance. This corollary experiment is documented in Appendix \ref{sec.drop_charge}.

The model $q \sim kAE_0$ is incidentally very similar to the classical solution for the surface charge density of a half-spherical conductor with a field induced dipole \cite{david_j._griffiths_introduction_1999}
\begin{eqnarray*}
q &=& 3 \epsilon_0 E_0 \int_A \cos \theta dA \\
&=& 3 \pi^{1/3} 6 \left(6 V_d \right)^{2/3} \epsilon_0 E_0 \int^{4 \pi/2}_{\pi / 2} \cos \theta d\theta \\
&=& k E_0 V_d^{2/3}
\end{eqnarray*}
with $k \approx 1.3 \times 10^{-10}$. This is also of a similar form to the charge found by Takamatsu and coauthors for droplets falling from a grounded nozzle in an external electric field \cite{takamatsu_theoretical_1981}
\[q = 4 \pi \epsilon_0 \beta E_0 R_d^2 \]
with $\beta \approx 2.63$.

\section{Scale Quantities}
The dielectrophoretic force plays a very small role when droplets have net charge in a DC field; the condition to neglect the DEP force was satisfied for all experiments in the dataset. Dimensional droplet apoapses scale closely with $\mathbb{E}\mbox{u}$ as seen in Figure \ref{fig:series_s_eu}. The relative magnitudes of the simulated forces felt by the droplets is shown in Figure \ref{fig:forces}. Forces acting on the drops vary in magnitude between $\mathcal{O}(10^{-6})-\mathcal{O}(10^{-4})$ N. We see that, of the drops in the experimental dataset only the two with the largest $\mathbb{E}\mbox{u}$, $\mathbb{E}\mbox{u} \sim \mathcal{O}(1)$ could appropriately be said to be in the inertial electro-viscous regime. In all other cases image forces are much stronger than drag. For these drops $\mathbb{E}\mbox{u} \gg 1/8 \pi$, and are likely on escape trajectories. The image forces themselves rapidly become small compared to Coulomb forces for drops with apoapses $\mbox{max}\left( y\right) \gtrapprox L$, thus it is reasonable to claim that for intermediate drops Coulomb force scales as inertia, and we can neglect the effects of drag and image forces.

\begin{figure}[htb]
    \centering
    %% Creator: Matplotlib, PGF backend
%%
%% To include the figure in your LaTeX document, write
%%   \input{<filename>.pgf}
%%
%% Make sure the required packages are loaded in your preamble
%%   \usepackage{pgf}
%%
%% Figures using additional raster images can only be included by \input if
%% they are in the same directory as the main LaTeX file. For loading figures
%% from other directories you can use the `import` package
%%   \usepackage{import}
%% and then include the figures with
%%   \import{<path to file>}{<filename>.pgf}
%%
%% Matplotlib used the following preamble
%%   \usepackage{fontspec}
%%   \setmainfont{DejaVu Serif}
%%   \setsansfont{DejaVu Sans}
%%   \setmonofont{DejaVu Sans Mono}
%%
\begingroup%
\makeatletter%
\begin{pgfpicture}%
\pgfpathrectangle{\pgfpointorigin}{\pgfqpoint{5.270186in}{3.684574in}}%
\pgfusepath{use as bounding box, clip}%
\begin{pgfscope}%
\pgfsetbuttcap%
\pgfsetmiterjoin%
\definecolor{currentfill}{rgb}{1.000000,1.000000,1.000000}%
\pgfsetfillcolor{currentfill}%
\pgfsetlinewidth{0.000000pt}%
\definecolor{currentstroke}{rgb}{1.000000,1.000000,1.000000}%
\pgfsetstrokecolor{currentstroke}%
\pgfsetdash{}{0pt}%
\pgfpathmoveto{\pgfqpoint{0.000000in}{0.000000in}}%
\pgfpathlineto{\pgfqpoint{5.270186in}{0.000000in}}%
\pgfpathlineto{\pgfqpoint{5.270186in}{3.684574in}}%
\pgfpathlineto{\pgfqpoint{0.000000in}{3.684574in}}%
\pgfpathclose%
\pgfusepath{fill}%
\end{pgfscope}%
\begin{pgfscope}%
\pgfsetbuttcap%
\pgfsetmiterjoin%
\definecolor{currentfill}{rgb}{1.000000,1.000000,1.000000}%
\pgfsetfillcolor{currentfill}%
\pgfsetlinewidth{0.000000pt}%
\definecolor{currentstroke}{rgb}{0.000000,0.000000,0.000000}%
\pgfsetstrokecolor{currentstroke}%
\pgfsetstrokeopacity{0.000000}%
\pgfsetdash{}{0pt}%
\pgfpathmoveto{\pgfqpoint{0.456635in}{0.521603in}}%
\pgfpathlineto{\pgfqpoint{4.176635in}{0.521603in}}%
\pgfpathlineto{\pgfqpoint{4.176635in}{3.541603in}}%
\pgfpathlineto{\pgfqpoint{0.456635in}{3.541603in}}%
\pgfpathclose%
\pgfusepath{fill}%
\end{pgfscope}%
\begin{pgfscope}%
\pgfsetbuttcap%
\pgfsetroundjoin%
\definecolor{currentfill}{rgb}{0.000000,0.000000,0.000000}%
\pgfsetfillcolor{currentfill}%
\pgfsetlinewidth{0.803000pt}%
\definecolor{currentstroke}{rgb}{0.000000,0.000000,0.000000}%
\pgfsetstrokecolor{currentstroke}%
\pgfsetdash{}{0pt}%
\pgfsys@defobject{currentmarker}{\pgfqpoint{0.000000in}{-0.048611in}}{\pgfqpoint{0.000000in}{0.000000in}}{%
\pgfpathmoveto{\pgfqpoint{0.000000in}{0.000000in}}%
\pgfpathlineto{\pgfqpoint{0.000000in}{-0.048611in}}%
\pgfusepath{stroke,fill}%
}%
\begin{pgfscope}%
\pgfsys@transformshift{0.581806in}{0.521603in}%
\pgfsys@useobject{currentmarker}{}%
\end{pgfscope}%
\end{pgfscope}%
\begin{pgfscope}%
\pgftext[x=0.581806in,y=0.424381in,,top]{\rmfamily\fontsize{10.000000}{12.000000}\selectfont \(\displaystyle 0.0\)}%
\end{pgfscope}%
\begin{pgfscope}%
\pgfsetbuttcap%
\pgfsetroundjoin%
\definecolor{currentfill}{rgb}{0.000000,0.000000,0.000000}%
\pgfsetfillcolor{currentfill}%
\pgfsetlinewidth{0.803000pt}%
\definecolor{currentstroke}{rgb}{0.000000,0.000000,0.000000}%
\pgfsetstrokecolor{currentstroke}%
\pgfsetdash{}{0pt}%
\pgfsys@defobject{currentmarker}{\pgfqpoint{0.000000in}{-0.048611in}}{\pgfqpoint{0.000000in}{0.000000in}}{%
\pgfpathmoveto{\pgfqpoint{0.000000in}{0.000000in}}%
\pgfpathlineto{\pgfqpoint{0.000000in}{-0.048611in}}%
\pgfusepath{stroke,fill}%
}%
\begin{pgfscope}%
\pgfsys@transformshift{1.460201in}{0.521603in}%
\pgfsys@useobject{currentmarker}{}%
\end{pgfscope}%
\end{pgfscope}%
\begin{pgfscope}%
\pgftext[x=1.460201in,y=0.424381in,,top]{\rmfamily\fontsize{10.000000}{12.000000}\selectfont \(\displaystyle 0.5\)}%
\end{pgfscope}%
\begin{pgfscope}%
\pgfsetbuttcap%
\pgfsetroundjoin%
\definecolor{currentfill}{rgb}{0.000000,0.000000,0.000000}%
\pgfsetfillcolor{currentfill}%
\pgfsetlinewidth{0.803000pt}%
\definecolor{currentstroke}{rgb}{0.000000,0.000000,0.000000}%
\pgfsetstrokecolor{currentstroke}%
\pgfsetdash{}{0pt}%
\pgfsys@defobject{currentmarker}{\pgfqpoint{0.000000in}{-0.048611in}}{\pgfqpoint{0.000000in}{0.000000in}}{%
\pgfpathmoveto{\pgfqpoint{0.000000in}{0.000000in}}%
\pgfpathlineto{\pgfqpoint{0.000000in}{-0.048611in}}%
\pgfusepath{stroke,fill}%
}%
\begin{pgfscope}%
\pgfsys@transformshift{2.338595in}{0.521603in}%
\pgfsys@useobject{currentmarker}{}%
\end{pgfscope}%
\end{pgfscope}%
\begin{pgfscope}%
\pgftext[x=2.338595in,y=0.424381in,,top]{\rmfamily\fontsize{10.000000}{12.000000}\selectfont \(\displaystyle 1.0\)}%
\end{pgfscope}%
\begin{pgfscope}%
\pgfsetbuttcap%
\pgfsetroundjoin%
\definecolor{currentfill}{rgb}{0.000000,0.000000,0.000000}%
\pgfsetfillcolor{currentfill}%
\pgfsetlinewidth{0.803000pt}%
\definecolor{currentstroke}{rgb}{0.000000,0.000000,0.000000}%
\pgfsetstrokecolor{currentstroke}%
\pgfsetdash{}{0pt}%
\pgfsys@defobject{currentmarker}{\pgfqpoint{0.000000in}{-0.048611in}}{\pgfqpoint{0.000000in}{0.000000in}}{%
\pgfpathmoveto{\pgfqpoint{0.000000in}{0.000000in}}%
\pgfpathlineto{\pgfqpoint{0.000000in}{-0.048611in}}%
\pgfusepath{stroke,fill}%
}%
\begin{pgfscope}%
\pgfsys@transformshift{3.216989in}{0.521603in}%
\pgfsys@useobject{currentmarker}{}%
\end{pgfscope}%
\end{pgfscope}%
\begin{pgfscope}%
\pgftext[x=3.216989in,y=0.424381in,,top]{\rmfamily\fontsize{10.000000}{12.000000}\selectfont \(\displaystyle 1.5\)}%
\end{pgfscope}%
\begin{pgfscope}%
\pgfsetbuttcap%
\pgfsetroundjoin%
\definecolor{currentfill}{rgb}{0.000000,0.000000,0.000000}%
\pgfsetfillcolor{currentfill}%
\pgfsetlinewidth{0.803000pt}%
\definecolor{currentstroke}{rgb}{0.000000,0.000000,0.000000}%
\pgfsetstrokecolor{currentstroke}%
\pgfsetdash{}{0pt}%
\pgfsys@defobject{currentmarker}{\pgfqpoint{0.000000in}{-0.048611in}}{\pgfqpoint{0.000000in}{0.000000in}}{%
\pgfpathmoveto{\pgfqpoint{0.000000in}{0.000000in}}%
\pgfpathlineto{\pgfqpoint{0.000000in}{-0.048611in}}%
\pgfusepath{stroke,fill}%
}%
\begin{pgfscope}%
\pgfsys@transformshift{4.095384in}{0.521603in}%
\pgfsys@useobject{currentmarker}{}%
\end{pgfscope}%
\end{pgfscope}%
\begin{pgfscope}%
\pgftext[x=4.095384in,y=0.424381in,,top]{\rmfamily\fontsize{10.000000}{12.000000}\selectfont \(\displaystyle 2.0\)}%
\end{pgfscope}%
\begin{pgfscope}%
\pgftext[x=2.316635in,y=0.234413in,,top]{\rmfamily\fontsize{10.000000}{12.000000}\selectfont \(\displaystyle t\) (s)}%
\end{pgfscope}%
\begin{pgfscope}%
\pgfsetbuttcap%
\pgfsetroundjoin%
\definecolor{currentfill}{rgb}{0.000000,0.000000,0.000000}%
\pgfsetfillcolor{currentfill}%
\pgfsetlinewidth{0.803000pt}%
\definecolor{currentstroke}{rgb}{0.000000,0.000000,0.000000}%
\pgfsetstrokecolor{currentstroke}%
\pgfsetdash{}{0pt}%
\pgfsys@defobject{currentmarker}{\pgfqpoint{-0.048611in}{0.000000in}}{\pgfqpoint{0.000000in}{0.000000in}}{%
\pgfpathmoveto{\pgfqpoint{0.000000in}{0.000000in}}%
\pgfpathlineto{\pgfqpoint{-0.048611in}{0.000000in}}%
\pgfusepath{stroke,fill}%
}%
\begin{pgfscope}%
\pgfsys@transformshift{0.456635in}{0.597983in}%
\pgfsys@useobject{currentmarker}{}%
\end{pgfscope}%
\end{pgfscope}%
\begin{pgfscope}%
\pgftext[x=0.289968in,y=0.545221in,left,base]{\rmfamily\fontsize{10.000000}{12.000000}\selectfont \(\displaystyle 0\)}%
\end{pgfscope}%
\begin{pgfscope}%
\pgfsetbuttcap%
\pgfsetroundjoin%
\definecolor{currentfill}{rgb}{0.000000,0.000000,0.000000}%
\pgfsetfillcolor{currentfill}%
\pgfsetlinewidth{0.803000pt}%
\definecolor{currentstroke}{rgb}{0.000000,0.000000,0.000000}%
\pgfsetstrokecolor{currentstroke}%
\pgfsetdash{}{0pt}%
\pgfsys@defobject{currentmarker}{\pgfqpoint{-0.048611in}{0.000000in}}{\pgfqpoint{0.000000in}{0.000000in}}{%
\pgfpathmoveto{\pgfqpoint{0.000000in}{0.000000in}}%
\pgfpathlineto{\pgfqpoint{-0.048611in}{0.000000in}}%
\pgfusepath{stroke,fill}%
}%
\begin{pgfscope}%
\pgfsys@transformshift{0.456635in}{0.964711in}%
\pgfsys@useobject{currentmarker}{}%
\end{pgfscope}%
\end{pgfscope}%
\begin{pgfscope}%
\pgftext[x=0.289968in,y=0.911950in,left,base]{\rmfamily\fontsize{10.000000}{12.000000}\selectfont \(\displaystyle 1\)}%
\end{pgfscope}%
\begin{pgfscope}%
\pgfsetbuttcap%
\pgfsetroundjoin%
\definecolor{currentfill}{rgb}{0.000000,0.000000,0.000000}%
\pgfsetfillcolor{currentfill}%
\pgfsetlinewidth{0.803000pt}%
\definecolor{currentstroke}{rgb}{0.000000,0.000000,0.000000}%
\pgfsetstrokecolor{currentstroke}%
\pgfsetdash{}{0pt}%
\pgfsys@defobject{currentmarker}{\pgfqpoint{-0.048611in}{0.000000in}}{\pgfqpoint{0.000000in}{0.000000in}}{%
\pgfpathmoveto{\pgfqpoint{0.000000in}{0.000000in}}%
\pgfpathlineto{\pgfqpoint{-0.048611in}{0.000000in}}%
\pgfusepath{stroke,fill}%
}%
\begin{pgfscope}%
\pgfsys@transformshift{0.456635in}{1.331440in}%
\pgfsys@useobject{currentmarker}{}%
\end{pgfscope}%
\end{pgfscope}%
\begin{pgfscope}%
\pgftext[x=0.289968in,y=1.278679in,left,base]{\rmfamily\fontsize{10.000000}{12.000000}\selectfont \(\displaystyle 2\)}%
\end{pgfscope}%
\begin{pgfscope}%
\pgfsetbuttcap%
\pgfsetroundjoin%
\definecolor{currentfill}{rgb}{0.000000,0.000000,0.000000}%
\pgfsetfillcolor{currentfill}%
\pgfsetlinewidth{0.803000pt}%
\definecolor{currentstroke}{rgb}{0.000000,0.000000,0.000000}%
\pgfsetstrokecolor{currentstroke}%
\pgfsetdash{}{0pt}%
\pgfsys@defobject{currentmarker}{\pgfqpoint{-0.048611in}{0.000000in}}{\pgfqpoint{0.000000in}{0.000000in}}{%
\pgfpathmoveto{\pgfqpoint{0.000000in}{0.000000in}}%
\pgfpathlineto{\pgfqpoint{-0.048611in}{0.000000in}}%
\pgfusepath{stroke,fill}%
}%
\begin{pgfscope}%
\pgfsys@transformshift{0.456635in}{1.698169in}%
\pgfsys@useobject{currentmarker}{}%
\end{pgfscope}%
\end{pgfscope}%
\begin{pgfscope}%
\pgftext[x=0.289968in,y=1.645407in,left,base]{\rmfamily\fontsize{10.000000}{12.000000}\selectfont \(\displaystyle 3\)}%
\end{pgfscope}%
\begin{pgfscope}%
\pgfsetbuttcap%
\pgfsetroundjoin%
\definecolor{currentfill}{rgb}{0.000000,0.000000,0.000000}%
\pgfsetfillcolor{currentfill}%
\pgfsetlinewidth{0.803000pt}%
\definecolor{currentstroke}{rgb}{0.000000,0.000000,0.000000}%
\pgfsetstrokecolor{currentstroke}%
\pgfsetdash{}{0pt}%
\pgfsys@defobject{currentmarker}{\pgfqpoint{-0.048611in}{0.000000in}}{\pgfqpoint{0.000000in}{0.000000in}}{%
\pgfpathmoveto{\pgfqpoint{0.000000in}{0.000000in}}%
\pgfpathlineto{\pgfqpoint{-0.048611in}{0.000000in}}%
\pgfusepath{stroke,fill}%
}%
\begin{pgfscope}%
\pgfsys@transformshift{0.456635in}{2.064898in}%
\pgfsys@useobject{currentmarker}{}%
\end{pgfscope}%
\end{pgfscope}%
\begin{pgfscope}%
\pgftext[x=0.289968in,y=2.012136in,left,base]{\rmfamily\fontsize{10.000000}{12.000000}\selectfont \(\displaystyle 4\)}%
\end{pgfscope}%
\begin{pgfscope}%
\pgfsetbuttcap%
\pgfsetroundjoin%
\definecolor{currentfill}{rgb}{0.000000,0.000000,0.000000}%
\pgfsetfillcolor{currentfill}%
\pgfsetlinewidth{0.803000pt}%
\definecolor{currentstroke}{rgb}{0.000000,0.000000,0.000000}%
\pgfsetstrokecolor{currentstroke}%
\pgfsetdash{}{0pt}%
\pgfsys@defobject{currentmarker}{\pgfqpoint{-0.048611in}{0.000000in}}{\pgfqpoint{0.000000in}{0.000000in}}{%
\pgfpathmoveto{\pgfqpoint{0.000000in}{0.000000in}}%
\pgfpathlineto{\pgfqpoint{-0.048611in}{0.000000in}}%
\pgfusepath{stroke,fill}%
}%
\begin{pgfscope}%
\pgfsys@transformshift{0.456635in}{2.431627in}%
\pgfsys@useobject{currentmarker}{}%
\end{pgfscope}%
\end{pgfscope}%
\begin{pgfscope}%
\pgftext[x=0.289968in,y=2.378865in,left,base]{\rmfamily\fontsize{10.000000}{12.000000}\selectfont \(\displaystyle 5\)}%
\end{pgfscope}%
\begin{pgfscope}%
\pgfsetbuttcap%
\pgfsetroundjoin%
\definecolor{currentfill}{rgb}{0.000000,0.000000,0.000000}%
\pgfsetfillcolor{currentfill}%
\pgfsetlinewidth{0.803000pt}%
\definecolor{currentstroke}{rgb}{0.000000,0.000000,0.000000}%
\pgfsetstrokecolor{currentstroke}%
\pgfsetdash{}{0pt}%
\pgfsys@defobject{currentmarker}{\pgfqpoint{-0.048611in}{0.000000in}}{\pgfqpoint{0.000000in}{0.000000in}}{%
\pgfpathmoveto{\pgfqpoint{0.000000in}{0.000000in}}%
\pgfpathlineto{\pgfqpoint{-0.048611in}{0.000000in}}%
\pgfusepath{stroke,fill}%
}%
\begin{pgfscope}%
\pgfsys@transformshift{0.456635in}{2.798355in}%
\pgfsys@useobject{currentmarker}{}%
\end{pgfscope}%
\end{pgfscope}%
\begin{pgfscope}%
\pgftext[x=0.289968in,y=2.745594in,left,base]{\rmfamily\fontsize{10.000000}{12.000000}\selectfont \(\displaystyle 6\)}%
\end{pgfscope}%
\begin{pgfscope}%
\pgfsetbuttcap%
\pgfsetroundjoin%
\definecolor{currentfill}{rgb}{0.000000,0.000000,0.000000}%
\pgfsetfillcolor{currentfill}%
\pgfsetlinewidth{0.803000pt}%
\definecolor{currentstroke}{rgb}{0.000000,0.000000,0.000000}%
\pgfsetstrokecolor{currentstroke}%
\pgfsetdash{}{0pt}%
\pgfsys@defobject{currentmarker}{\pgfqpoint{-0.048611in}{0.000000in}}{\pgfqpoint{0.000000in}{0.000000in}}{%
\pgfpathmoveto{\pgfqpoint{0.000000in}{0.000000in}}%
\pgfpathlineto{\pgfqpoint{-0.048611in}{0.000000in}}%
\pgfusepath{stroke,fill}%
}%
\begin{pgfscope}%
\pgfsys@transformshift{0.456635in}{3.165084in}%
\pgfsys@useobject{currentmarker}{}%
\end{pgfscope}%
\end{pgfscope}%
\begin{pgfscope}%
\pgftext[x=0.289968in,y=3.112323in,left,base]{\rmfamily\fontsize{10.000000}{12.000000}\selectfont \(\displaystyle 7\)}%
\end{pgfscope}%
\begin{pgfscope}%
\pgfsetbuttcap%
\pgfsetroundjoin%
\definecolor{currentfill}{rgb}{0.000000,0.000000,0.000000}%
\pgfsetfillcolor{currentfill}%
\pgfsetlinewidth{0.803000pt}%
\definecolor{currentstroke}{rgb}{0.000000,0.000000,0.000000}%
\pgfsetstrokecolor{currentstroke}%
\pgfsetdash{}{0pt}%
\pgfsys@defobject{currentmarker}{\pgfqpoint{-0.048611in}{0.000000in}}{\pgfqpoint{0.000000in}{0.000000in}}{%
\pgfpathmoveto{\pgfqpoint{0.000000in}{0.000000in}}%
\pgfpathlineto{\pgfqpoint{-0.048611in}{0.000000in}}%
\pgfusepath{stroke,fill}%
}%
\begin{pgfscope}%
\pgfsys@transformshift{0.456635in}{3.531813in}%
\pgfsys@useobject{currentmarker}{}%
\end{pgfscope}%
\end{pgfscope}%
\begin{pgfscope}%
\pgftext[x=0.289968in,y=3.479051in,left,base]{\rmfamily\fontsize{10.000000}{12.000000}\selectfont \(\displaystyle 8\)}%
\end{pgfscope}%
\begin{pgfscope}%
\pgftext[x=0.234413in,y=2.031603in,,bottom,rotate=90.000000]{\rmfamily\fontsize{10.000000}{12.000000}\selectfont \(\displaystyle y\) (cm)}%
\end{pgfscope}%
\begin{pgfscope}%
\pgfpathrectangle{\pgfqpoint{0.456635in}{0.521603in}}{\pgfqpoint{3.720000in}{3.020000in}} %
\pgfusepath{clip}%
\pgfsetrectcap%
\pgfsetroundjoin%
\pgfsetlinewidth{1.505625pt}%
\definecolor{currentstroke}{rgb}{0.500000,0.000000,1.000000}%
\pgfsetstrokecolor{currentstroke}%
\pgfsetdash{}{0pt}%
\pgfpathmoveto{\pgfqpoint{0.713566in}{0.953990in}}%
\pgfpathlineto{\pgfqpoint{0.728206in}{0.976249in}}%
\pgfpathlineto{\pgfqpoint{0.742845in}{0.998409in}}%
\pgfpathlineto{\pgfqpoint{0.757485in}{1.029351in}}%
\pgfpathlineto{\pgfqpoint{0.772125in}{1.057658in}}%
\pgfpathlineto{\pgfqpoint{0.786765in}{1.081788in}}%
\pgfpathlineto{\pgfqpoint{0.801405in}{1.106941in}}%
\pgfpathlineto{\pgfqpoint{0.816045in}{1.129560in}}%
\pgfpathlineto{\pgfqpoint{0.830685in}{1.156430in}}%
\pgfpathlineto{\pgfqpoint{0.845325in}{1.185125in}}%
\pgfpathlineto{\pgfqpoint{0.859965in}{1.206949in}}%
\pgfpathlineto{\pgfqpoint{0.874605in}{1.230897in}}%
\pgfpathlineto{\pgfqpoint{0.889244in}{1.260151in}}%
\pgfpathlineto{\pgfqpoint{0.903884in}{1.286379in}}%
\pgfpathlineto{\pgfqpoint{0.918524in}{1.309421in}}%
\pgfpathlineto{\pgfqpoint{0.933164in}{1.329354in}}%
\pgfpathlineto{\pgfqpoint{0.947804in}{1.355500in}}%
\pgfpathlineto{\pgfqpoint{0.962444in}{1.382638in}}%
\pgfpathlineto{\pgfqpoint{0.977084in}{1.408055in}}%
\pgfpathlineto{\pgfqpoint{0.991724in}{1.429796in}}%
\pgfpathlineto{\pgfqpoint{1.006364in}{1.454095in}}%
\pgfpathlineto{\pgfqpoint{1.021004in}{1.480372in}}%
\pgfpathlineto{\pgfqpoint{1.035644in}{1.503842in}}%
\pgfpathlineto{\pgfqpoint{1.050283in}{1.524845in}}%
\pgfpathlineto{\pgfqpoint{1.064923in}{1.548538in}}%
\pgfpathlineto{\pgfqpoint{1.079563in}{1.573337in}}%
\pgfpathlineto{\pgfqpoint{1.094203in}{1.598394in}}%
\pgfpathlineto{\pgfqpoint{1.108843in}{1.618325in}}%
\pgfpathlineto{\pgfqpoint{1.123483in}{1.642140in}}%
\pgfpathlineto{\pgfqpoint{1.138123in}{1.667006in}}%
\pgfpathlineto{\pgfqpoint{1.152763in}{1.690600in}}%
\pgfpathlineto{\pgfqpoint{1.167403in}{1.711890in}}%
\pgfpathlineto{\pgfqpoint{1.182043in}{1.733580in}}%
\pgfpathlineto{\pgfqpoint{1.196683in}{1.757106in}}%
\pgfpathlineto{\pgfqpoint{1.211322in}{1.781324in}}%
\pgfpathlineto{\pgfqpoint{1.225962in}{1.800742in}}%
\pgfpathlineto{\pgfqpoint{1.240602in}{1.824801in}}%
\pgfpathlineto{\pgfqpoint{1.255242in}{1.847458in}}%
\pgfpathlineto{\pgfqpoint{1.269882in}{1.870972in}}%
\pgfpathlineto{\pgfqpoint{1.284522in}{1.892498in}}%
\pgfpathlineto{\pgfqpoint{1.299162in}{1.913503in}}%
\pgfpathlineto{\pgfqpoint{1.313802in}{1.933511in}}%
\pgfpathlineto{\pgfqpoint{1.328442in}{1.957575in}}%
\pgfpathlineto{\pgfqpoint{1.343082in}{1.978541in}}%
\pgfpathlineto{\pgfqpoint{1.357721in}{2.002067in}}%
\pgfpathlineto{\pgfqpoint{1.372361in}{2.024440in}}%
\pgfpathlineto{\pgfqpoint{1.387001in}{2.047513in}}%
\pgfpathlineto{\pgfqpoint{1.401641in}{2.069389in}}%
\pgfpathlineto{\pgfqpoint{1.416281in}{2.088838in}}%
\pgfpathlineto{\pgfqpoint{1.430921in}{2.108017in}}%
\pgfpathlineto{\pgfqpoint{1.445561in}{2.130921in}}%
\pgfpathlineto{\pgfqpoint{1.460201in}{2.152687in}}%
\pgfpathlineto{\pgfqpoint{1.474841in}{2.175094in}}%
\pgfpathlineto{\pgfqpoint{1.489481in}{2.197700in}}%
\pgfpathlineto{\pgfqpoint{1.504121in}{2.220082in}}%
\pgfpathlineto{\pgfqpoint{1.518760in}{2.242097in}}%
\pgfpathlineto{\pgfqpoint{1.533400in}{2.261309in}}%
\pgfpathlineto{\pgfqpoint{1.548040in}{2.279987in}}%
\pgfpathlineto{\pgfqpoint{1.562680in}{2.300611in}}%
\pgfpathlineto{\pgfqpoint{1.577320in}{2.323216in}}%
\pgfpathlineto{\pgfqpoint{1.591960in}{2.345376in}}%
\pgfpathlineto{\pgfqpoint{1.606600in}{2.368865in}}%
\pgfpathlineto{\pgfqpoint{1.621240in}{2.390565in}}%
\pgfpathlineto{\pgfqpoint{1.635880in}{2.412284in}}%
\pgfpathlineto{\pgfqpoint{1.650520in}{2.430994in}}%
\pgfpathlineto{\pgfqpoint{1.665159in}{2.449679in}}%
\pgfpathlineto{\pgfqpoint{1.679799in}{2.469323in}}%
\pgfpathlineto{\pgfqpoint{1.694439in}{2.490954in}}%
\pgfpathlineto{\pgfqpoint{1.709079in}{2.513901in}}%
\pgfpathlineto{\pgfqpoint{1.723719in}{2.537633in}}%
\pgfpathlineto{\pgfqpoint{1.738359in}{2.558787in}}%
\pgfpathlineto{\pgfqpoint{1.752999in}{2.579855in}}%
\pgfpathlineto{\pgfqpoint{1.767639in}{2.598638in}}%
\pgfpathlineto{\pgfqpoint{1.782279in}{2.616273in}}%
\pgfpathlineto{\pgfqpoint{1.796919in}{2.635641in}}%
\pgfpathlineto{\pgfqpoint{1.811559in}{2.656098in}}%
\pgfpathlineto{\pgfqpoint{1.826198in}{2.679336in}}%
\pgfpathlineto{\pgfqpoint{1.840838in}{2.702490in}}%
\pgfpathlineto{\pgfqpoint{1.855478in}{2.724267in}}%
\pgfpathlineto{\pgfqpoint{1.870118in}{2.745367in}}%
\pgfpathlineto{\pgfqpoint{1.884758in}{2.763303in}}%
\pgfpathlineto{\pgfqpoint{1.899398in}{2.780590in}}%
\pgfpathlineto{\pgfqpoint{1.914038in}{2.799896in}}%
\pgfpathlineto{\pgfqpoint{1.928678in}{2.820454in}}%
\pgfpathlineto{\pgfqpoint{1.943318in}{2.842822in}}%
\pgfpathlineto{\pgfqpoint{1.957958in}{2.866414in}}%
\pgfpathlineto{\pgfqpoint{1.972597in}{2.887499in}}%
\pgfpathlineto{\pgfqpoint{1.987237in}{2.907908in}}%
\pgfpathlineto{\pgfqpoint{2.001877in}{2.926565in}}%
\pgfpathlineto{\pgfqpoint{2.016517in}{2.942901in}}%
\pgfpathlineto{\pgfqpoint{2.031157in}{2.961694in}}%
\pgfpathlineto{\pgfqpoint{2.045797in}{2.981693in}}%
\pgfpathlineto{\pgfqpoint{2.060437in}{3.003759in}}%
\pgfpathlineto{\pgfqpoint{2.075077in}{3.027689in}}%
\pgfpathlineto{\pgfqpoint{2.089717in}{3.048612in}}%
\pgfpathlineto{\pgfqpoint{2.104357in}{3.068468in}}%
\pgfpathlineto{\pgfqpoint{2.118997in}{3.088032in}}%
\pgfpathlineto{\pgfqpoint{2.133636in}{3.103928in}}%
\pgfpathlineto{\pgfqpoint{2.148276in}{3.122116in}}%
\pgfpathlineto{\pgfqpoint{2.162916in}{3.142069in}}%
\pgfpathlineto{\pgfqpoint{2.177556in}{3.164506in}}%
\pgfpathlineto{\pgfqpoint{2.192196in}{3.187622in}}%
\pgfpathlineto{\pgfqpoint{2.206836in}{3.207936in}}%
\pgfpathlineto{\pgfqpoint{2.221476in}{3.227375in}}%
\pgfpathlineto{\pgfqpoint{2.236116in}{3.247184in}}%
\pgfpathlineto{\pgfqpoint{2.250756in}{3.263485in}}%
\pgfpathlineto{\pgfqpoint{2.265396in}{3.280831in}}%
\pgfpathlineto{\pgfqpoint{2.280036in}{3.301000in}}%
\pgfpathlineto{\pgfqpoint{2.294675in}{3.323594in}}%
\pgfpathlineto{\pgfqpoint{2.309315in}{3.346447in}}%
\pgfpathlineto{\pgfqpoint{2.323955in}{3.365335in}}%
\pgfpathlineto{\pgfqpoint{2.338595in}{3.386528in}}%
\pgfpathlineto{\pgfqpoint{2.353235in}{3.404331in}}%
\pgfusepath{stroke}%
\end{pgfscope}%
\begin{pgfscope}%
\pgfpathrectangle{\pgfqpoint{0.456635in}{0.521603in}}{\pgfqpoint{3.720000in}{3.020000in}} %
\pgfusepath{clip}%
\pgfsetrectcap%
\pgfsetroundjoin%
\pgfsetlinewidth{1.505625pt}%
\definecolor{currentstroke}{rgb}{0.182353,0.878081,0.859800}%
\pgfsetstrokecolor{currentstroke}%
\pgfsetdash{}{0pt}%
\pgfpathmoveto{\pgfqpoint{0.742845in}{0.933978in}}%
\pgfpathlineto{\pgfqpoint{0.757485in}{0.941019in}}%
\pgfpathlineto{\pgfqpoint{0.772125in}{0.973343in}}%
\pgfpathlineto{\pgfqpoint{0.786765in}{1.000368in}}%
\pgfpathlineto{\pgfqpoint{0.801405in}{1.016000in}}%
\pgfpathlineto{\pgfqpoint{0.816045in}{1.035824in}}%
\pgfpathlineto{\pgfqpoint{0.830685in}{1.060113in}}%
\pgfpathlineto{\pgfqpoint{0.845325in}{1.088076in}}%
\pgfpathlineto{\pgfqpoint{0.859965in}{1.107301in}}%
\pgfpathlineto{\pgfqpoint{0.874605in}{1.123489in}}%
\pgfpathlineto{\pgfqpoint{0.889244in}{1.145334in}}%
\pgfpathlineto{\pgfqpoint{0.903884in}{1.171042in}}%
\pgfpathlineto{\pgfqpoint{0.918524in}{1.190507in}}%
\pgfpathlineto{\pgfqpoint{0.947804in}{1.223946in}}%
\pgfpathlineto{\pgfqpoint{0.962444in}{1.245026in}}%
\pgfpathlineto{\pgfqpoint{0.977084in}{1.269296in}}%
\pgfpathlineto{\pgfqpoint{0.991724in}{1.287106in}}%
\pgfpathlineto{\pgfqpoint{1.006364in}{1.300033in}}%
\pgfpathlineto{\pgfqpoint{1.021004in}{1.319799in}}%
\pgfpathlineto{\pgfqpoint{1.035644in}{1.342371in}}%
\pgfpathlineto{\pgfqpoint{1.050283in}{1.360156in}}%
\pgfpathlineto{\pgfqpoint{1.079563in}{1.389101in}}%
\pgfpathlineto{\pgfqpoint{1.108843in}{1.429204in}}%
\pgfpathlineto{\pgfqpoint{1.123483in}{1.445130in}}%
\pgfpathlineto{\pgfqpoint{1.138123in}{1.456741in}}%
\pgfpathlineto{\pgfqpoint{1.182043in}{1.510897in}}%
\pgfpathlineto{\pgfqpoint{1.211322in}{1.536678in}}%
\pgfpathlineto{\pgfqpoint{1.225962in}{1.553608in}}%
\pgfpathlineto{\pgfqpoint{1.240602in}{1.573048in}}%
\pgfpathlineto{\pgfqpoint{1.255242in}{1.585839in}}%
\pgfpathlineto{\pgfqpoint{1.269882in}{1.597138in}}%
\pgfpathlineto{\pgfqpoint{1.284522in}{1.613020in}}%
\pgfpathlineto{\pgfqpoint{1.299162in}{1.631021in}}%
\pgfpathlineto{\pgfqpoint{1.313802in}{1.645608in}}%
\pgfpathlineto{\pgfqpoint{1.343082in}{1.668732in}}%
\pgfpathlineto{\pgfqpoint{1.357721in}{1.684388in}}%
\pgfpathlineto{\pgfqpoint{1.372361in}{1.701768in}}%
\pgfpathlineto{\pgfqpoint{1.387001in}{1.713490in}}%
\pgfpathlineto{\pgfqpoint{1.401641in}{1.723187in}}%
\pgfpathlineto{\pgfqpoint{1.445561in}{1.767723in}}%
\pgfpathlineto{\pgfqpoint{1.474841in}{1.788706in}}%
\pgfpathlineto{\pgfqpoint{1.489481in}{1.802843in}}%
\pgfpathlineto{\pgfqpoint{1.504121in}{1.818774in}}%
\pgfpathlineto{\pgfqpoint{1.533400in}{1.838593in}}%
\pgfpathlineto{\pgfqpoint{1.577320in}{1.878987in}}%
\pgfpathlineto{\pgfqpoint{1.606600in}{1.898229in}}%
\pgfpathlineto{\pgfqpoint{1.621240in}{1.911232in}}%
\pgfpathlineto{\pgfqpoint{1.635880in}{1.925897in}}%
\pgfpathlineto{\pgfqpoint{1.665159in}{1.944092in}}%
\pgfpathlineto{\pgfqpoint{1.679799in}{1.956146in}}%
\pgfpathlineto{\pgfqpoint{1.694439in}{1.969727in}}%
\pgfpathlineto{\pgfqpoint{1.709079in}{1.981062in}}%
\pgfpathlineto{\pgfqpoint{1.738359in}{1.998867in}}%
\pgfpathlineto{\pgfqpoint{1.767639in}{2.024330in}}%
\pgfpathlineto{\pgfqpoint{1.796919in}{2.041296in}}%
\pgfpathlineto{\pgfqpoint{1.840838in}{2.075478in}}%
\pgfpathlineto{\pgfqpoint{1.870118in}{2.091891in}}%
\pgfpathlineto{\pgfqpoint{1.899398in}{2.115390in}}%
\pgfpathlineto{\pgfqpoint{1.928678in}{2.131108in}}%
\pgfpathlineto{\pgfqpoint{1.972597in}{2.162623in}}%
\pgfpathlineto{\pgfqpoint{2.001877in}{2.177798in}}%
\pgfpathlineto{\pgfqpoint{2.031157in}{2.199360in}}%
\pgfpathlineto{\pgfqpoint{2.060437in}{2.214157in}}%
\pgfpathlineto{\pgfqpoint{2.104357in}{2.243302in}}%
\pgfpathlineto{\pgfqpoint{2.118997in}{2.249723in}}%
\pgfpathlineto{\pgfqpoint{2.133636in}{2.257683in}}%
\pgfpathlineto{\pgfqpoint{2.148276in}{2.268149in}}%
\pgfpathlineto{\pgfqpoint{2.162916in}{2.277173in}}%
\pgfpathlineto{\pgfqpoint{2.192196in}{2.291301in}}%
\pgfpathlineto{\pgfqpoint{2.236116in}{2.317978in}}%
\pgfpathlineto{\pgfqpoint{2.250756in}{2.324052in}}%
\pgfpathlineto{\pgfqpoint{2.265396in}{2.331888in}}%
\pgfpathlineto{\pgfqpoint{2.294675in}{2.349540in}}%
\pgfpathlineto{\pgfqpoint{2.323955in}{2.362874in}}%
\pgfpathlineto{\pgfqpoint{2.367875in}{2.387341in}}%
\pgfpathlineto{\pgfqpoint{2.382515in}{2.393125in}}%
\pgfpathlineto{\pgfqpoint{2.397155in}{2.400387in}}%
\pgfpathlineto{\pgfqpoint{2.426435in}{2.416978in}}%
\pgfpathlineto{\pgfqpoint{2.455714in}{2.429330in}}%
\pgfpathlineto{\pgfqpoint{2.499634in}{2.452189in}}%
\pgfpathlineto{\pgfqpoint{2.514274in}{2.457637in}}%
\pgfpathlineto{\pgfqpoint{2.528914in}{2.464404in}}%
\pgfpathlineto{\pgfqpoint{2.543554in}{2.472880in}}%
\pgfpathlineto{\pgfqpoint{2.558194in}{2.479991in}}%
\pgfpathlineto{\pgfqpoint{2.587474in}{2.491712in}}%
\pgfpathlineto{\pgfqpoint{2.616753in}{2.506825in}}%
\pgfpathlineto{\pgfqpoint{2.660673in}{2.524554in}}%
\pgfpathlineto{\pgfqpoint{2.675313in}{2.532584in}}%
\pgfpathlineto{\pgfqpoint{2.689953in}{2.539155in}}%
\pgfpathlineto{\pgfqpoint{2.719233in}{2.550014in}}%
\pgfpathlineto{\pgfqpoint{2.748512in}{2.564240in}}%
\pgfpathlineto{\pgfqpoint{2.792432in}{2.580848in}}%
\pgfpathlineto{\pgfqpoint{2.807072in}{2.588388in}}%
\pgfpathlineto{\pgfqpoint{2.836352in}{2.599127in}}%
\pgfpathlineto{\pgfqpoint{2.850992in}{2.604424in}}%
\pgfpathlineto{\pgfqpoint{2.880272in}{2.617634in}}%
\pgfpathlineto{\pgfqpoint{2.909551in}{2.627015in}}%
\pgfpathlineto{\pgfqpoint{2.953471in}{2.645568in}}%
\pgfpathlineto{\pgfqpoint{2.982751in}{2.655140in}}%
\pgfpathlineto{\pgfqpoint{3.012031in}{2.667409in}}%
\pgfpathlineto{\pgfqpoint{3.041311in}{2.676193in}}%
\pgfpathlineto{\pgfqpoint{3.070590in}{2.688902in}}%
\pgfpathlineto{\pgfqpoint{3.129150in}{2.708443in}}%
\pgfpathlineto{\pgfqpoint{3.143790in}{2.713979in}}%
\pgfpathlineto{\pgfqpoint{3.173070in}{2.722477in}}%
\pgfpathlineto{\pgfqpoint{3.202350in}{2.734136in}}%
\pgfpathlineto{\pgfqpoint{3.260909in}{2.752555in}}%
\pgfpathlineto{\pgfqpoint{3.275549in}{2.757441in}}%
\pgfpathlineto{\pgfqpoint{3.304829in}{2.765599in}}%
\pgfpathlineto{\pgfqpoint{3.334109in}{2.776470in}}%
\pgfpathlineto{\pgfqpoint{3.378028in}{2.788514in}}%
\pgfpathlineto{\pgfqpoint{3.407308in}{2.797945in}}%
\pgfpathlineto{\pgfqpoint{3.436588in}{2.805763in}}%
\pgfpathlineto{\pgfqpoint{3.465868in}{2.815708in}}%
\pgfpathlineto{\pgfqpoint{3.553707in}{2.838818in}}%
\pgfpathlineto{\pgfqpoint{3.612267in}{2.855578in}}%
\pgfpathlineto{\pgfqpoint{3.641547in}{2.862772in}}%
\pgfpathlineto{\pgfqpoint{3.670827in}{2.870655in}}%
\pgfpathlineto{\pgfqpoint{3.700106in}{2.878019in}}%
\pgfpathlineto{\pgfqpoint{3.729386in}{2.886112in}}%
\pgfpathlineto{\pgfqpoint{3.758666in}{2.892053in}}%
\pgfpathlineto{\pgfqpoint{3.787946in}{2.900089in}}%
\pgfpathlineto{\pgfqpoint{3.831866in}{2.910066in}}%
\pgfpathlineto{\pgfqpoint{3.861145in}{2.917565in}}%
\pgfpathlineto{\pgfqpoint{3.905065in}{2.926960in}}%
\pgfpathlineto{\pgfqpoint{3.992904in}{2.946717in}}%
\pgfpathlineto{\pgfqpoint{4.007544in}{2.948929in}}%
\pgfpathlineto{\pgfqpoint{4.007544in}{2.948929in}}%
\pgfusepath{stroke}%
\end{pgfscope}%
\begin{pgfscope}%
\pgfpathrectangle{\pgfqpoint{0.456635in}{0.521603in}}{\pgfqpoint{3.720000in}{3.020000in}} %
\pgfusepath{clip}%
\pgfsetrectcap%
\pgfsetroundjoin%
\pgfsetlinewidth{1.505625pt}%
\definecolor{currentstroke}{rgb}{0.327451,0.267733,0.990831}%
\pgfsetstrokecolor{currentstroke}%
\pgfsetdash{}{0pt}%
\pgfpathmoveto{\pgfqpoint{0.669646in}{0.811525in}}%
\pgfpathlineto{\pgfqpoint{0.698926in}{0.865918in}}%
\pgfpathlineto{\pgfqpoint{0.728206in}{0.908725in}}%
\pgfpathlineto{\pgfqpoint{0.757485in}{0.963482in}}%
\pgfpathlineto{\pgfqpoint{0.801405in}{1.027780in}}%
\pgfpathlineto{\pgfqpoint{0.830685in}{1.080571in}}%
\pgfpathlineto{\pgfqpoint{0.845325in}{1.098492in}}%
\pgfpathlineto{\pgfqpoint{0.859965in}{1.119449in}}%
\pgfpathlineto{\pgfqpoint{0.889244in}{1.170686in}}%
\pgfpathlineto{\pgfqpoint{0.918524in}{1.208540in}}%
\pgfpathlineto{\pgfqpoint{0.933164in}{1.230544in}}%
\pgfpathlineto{\pgfqpoint{0.947804in}{1.256586in}}%
\pgfpathlineto{\pgfqpoint{0.962444in}{1.278190in}}%
\pgfpathlineto{\pgfqpoint{0.977084in}{1.295522in}}%
\pgfpathlineto{\pgfqpoint{0.991724in}{1.316035in}}%
\pgfpathlineto{\pgfqpoint{1.006364in}{1.340894in}}%
\pgfpathlineto{\pgfqpoint{1.021004in}{1.362977in}}%
\pgfpathlineto{\pgfqpoint{1.050283in}{1.399015in}}%
\pgfpathlineto{\pgfqpoint{1.094203in}{1.463863in}}%
\pgfpathlineto{\pgfqpoint{1.108843in}{1.480742in}}%
\pgfpathlineto{\pgfqpoint{1.123483in}{1.501349in}}%
\pgfpathlineto{\pgfqpoint{1.138123in}{1.524051in}}%
\pgfpathlineto{\pgfqpoint{1.152763in}{1.544073in}}%
\pgfpathlineto{\pgfqpoint{1.182043in}{1.579144in}}%
\pgfpathlineto{\pgfqpoint{1.211322in}{1.621910in}}%
\pgfpathlineto{\pgfqpoint{1.240602in}{1.656144in}}%
\pgfpathlineto{\pgfqpoint{1.284522in}{1.716388in}}%
\pgfpathlineto{\pgfqpoint{1.313802in}{1.750962in}}%
\pgfpathlineto{\pgfqpoint{1.343082in}{1.790981in}}%
\pgfpathlineto{\pgfqpoint{1.372361in}{1.824423in}}%
\pgfpathlineto{\pgfqpoint{1.401641in}{1.864406in}}%
\pgfpathlineto{\pgfqpoint{1.445561in}{1.915671in}}%
\pgfpathlineto{\pgfqpoint{1.460201in}{1.935707in}}%
\pgfpathlineto{\pgfqpoint{1.606600in}{2.109960in}}%
\pgfpathlineto{\pgfqpoint{1.621240in}{2.125842in}}%
\pgfpathlineto{\pgfqpoint{1.679799in}{2.195059in}}%
\pgfpathlineto{\pgfqpoint{1.709079in}{2.229423in}}%
\pgfpathlineto{\pgfqpoint{1.723719in}{2.246998in}}%
\pgfpathlineto{\pgfqpoint{1.752999in}{2.278617in}}%
\pgfpathlineto{\pgfqpoint{1.796919in}{2.330318in}}%
\pgfpathlineto{\pgfqpoint{1.826198in}{2.362395in}}%
\pgfpathlineto{\pgfqpoint{1.855478in}{2.396475in}}%
\pgfpathlineto{\pgfqpoint{1.884758in}{2.427983in}}%
\pgfpathlineto{\pgfqpoint{1.914038in}{2.462437in}}%
\pgfpathlineto{\pgfqpoint{1.957958in}{2.510295in}}%
\pgfpathlineto{\pgfqpoint{1.987237in}{2.542933in}}%
\pgfpathlineto{\pgfqpoint{2.016517in}{2.574534in}}%
\pgfpathlineto{\pgfqpoint{2.045797in}{2.607833in}}%
\pgfpathlineto{\pgfqpoint{2.075077in}{2.638864in}}%
\pgfpathlineto{\pgfqpoint{2.118997in}{2.686924in}}%
\pgfpathlineto{\pgfqpoint{2.148276in}{2.718456in}}%
\pgfpathlineto{\pgfqpoint{2.177556in}{2.750600in}}%
\pgfpathlineto{\pgfqpoint{2.221476in}{2.797740in}}%
\pgfpathlineto{\pgfqpoint{2.309315in}{2.891284in}}%
\pgfpathlineto{\pgfqpoint{2.689953in}{3.287956in}}%
\pgfpathlineto{\pgfqpoint{2.704593in}{3.302659in}}%
\pgfpathlineto{\pgfqpoint{2.704593in}{3.302659in}}%
\pgfusepath{stroke}%
\end{pgfscope}%
\begin{pgfscope}%
\pgfpathrectangle{\pgfqpoint{0.456635in}{0.521603in}}{\pgfqpoint{3.720000in}{3.020000in}} %
\pgfusepath{clip}%
\pgfsetrectcap%
\pgfsetroundjoin%
\pgfsetlinewidth{1.505625pt}%
\definecolor{currentstroke}{rgb}{0.221569,0.905873,0.843667}%
\pgfsetstrokecolor{currentstroke}%
\pgfsetdash{}{0pt}%
\pgfpathmoveto{\pgfqpoint{0.698926in}{0.851858in}}%
\pgfpathlineto{\pgfqpoint{0.713566in}{0.864296in}}%
\pgfpathlineto{\pgfqpoint{0.728206in}{0.883058in}}%
\pgfpathlineto{\pgfqpoint{0.742845in}{0.918177in}}%
\pgfpathlineto{\pgfqpoint{0.757485in}{0.938144in}}%
\pgfpathlineto{\pgfqpoint{0.772125in}{0.946161in}}%
\pgfpathlineto{\pgfqpoint{0.801405in}{1.000800in}}%
\pgfpathlineto{\pgfqpoint{0.816045in}{1.017281in}}%
\pgfpathlineto{\pgfqpoint{0.830685in}{1.031370in}}%
\pgfpathlineto{\pgfqpoint{0.859965in}{1.075645in}}%
\pgfpathlineto{\pgfqpoint{0.903884in}{1.124870in}}%
\pgfpathlineto{\pgfqpoint{0.918524in}{1.145932in}}%
\pgfpathlineto{\pgfqpoint{0.933164in}{1.161308in}}%
\pgfpathlineto{\pgfqpoint{0.947804in}{1.174899in}}%
\pgfpathlineto{\pgfqpoint{0.977084in}{1.211593in}}%
\pgfpathlineto{\pgfqpoint{1.006364in}{1.238733in}}%
\pgfpathlineto{\pgfqpoint{1.035644in}{1.272978in}}%
\pgfpathlineto{\pgfqpoint{1.064923in}{1.298607in}}%
\pgfpathlineto{\pgfqpoint{1.079563in}{1.315231in}}%
\pgfpathlineto{\pgfqpoint{1.094203in}{1.330100in}}%
\pgfpathlineto{\pgfqpoint{1.123483in}{1.353921in}}%
\pgfpathlineto{\pgfqpoint{1.152763in}{1.383467in}}%
\pgfpathlineto{\pgfqpoint{1.167403in}{1.393333in}}%
\pgfpathlineto{\pgfqpoint{1.182043in}{1.406430in}}%
\pgfpathlineto{\pgfqpoint{1.196683in}{1.421021in}}%
\pgfpathlineto{\pgfqpoint{1.211322in}{1.433027in}}%
\pgfpathlineto{\pgfqpoint{1.225962in}{1.442974in}}%
\pgfpathlineto{\pgfqpoint{1.240602in}{1.454708in}}%
\pgfpathlineto{\pgfqpoint{1.255242in}{1.468745in}}%
\pgfpathlineto{\pgfqpoint{1.284522in}{1.488491in}}%
\pgfpathlineto{\pgfqpoint{1.313802in}{1.512982in}}%
\pgfpathlineto{\pgfqpoint{1.357721in}{1.543011in}}%
\pgfpathlineto{\pgfqpoint{1.372361in}{1.554431in}}%
\pgfpathlineto{\pgfqpoint{1.387001in}{1.562419in}}%
\pgfpathlineto{\pgfqpoint{1.401641in}{1.571848in}}%
\pgfpathlineto{\pgfqpoint{1.430921in}{1.592672in}}%
\pgfpathlineto{\pgfqpoint{1.460201in}{1.608960in}}%
\pgfpathlineto{\pgfqpoint{1.474841in}{1.619591in}}%
\pgfpathlineto{\pgfqpoint{1.489481in}{1.628442in}}%
\pgfpathlineto{\pgfqpoint{1.504121in}{1.635377in}}%
\pgfpathlineto{\pgfqpoint{1.548040in}{1.661700in}}%
\pgfpathlineto{\pgfqpoint{1.562680in}{1.668304in}}%
\pgfpathlineto{\pgfqpoint{1.606600in}{1.692133in}}%
\pgfpathlineto{\pgfqpoint{1.621240in}{1.698642in}}%
\pgfpathlineto{\pgfqpoint{1.650520in}{1.714561in}}%
\pgfpathlineto{\pgfqpoint{1.679799in}{1.726649in}}%
\pgfpathlineto{\pgfqpoint{1.709079in}{1.741425in}}%
\pgfpathlineto{\pgfqpoint{1.738359in}{1.752576in}}%
\pgfpathlineto{\pgfqpoint{1.767639in}{1.765916in}}%
\pgfpathlineto{\pgfqpoint{1.796919in}{1.776319in}}%
\pgfpathlineto{\pgfqpoint{1.811559in}{1.782916in}}%
\pgfpathlineto{\pgfqpoint{1.884758in}{1.808662in}}%
\pgfpathlineto{\pgfqpoint{1.914038in}{1.817611in}}%
\pgfpathlineto{\pgfqpoint{1.928678in}{1.822955in}}%
\pgfpathlineto{\pgfqpoint{1.972597in}{1.835156in}}%
\pgfpathlineto{\pgfqpoint{2.001877in}{1.843219in}}%
\pgfpathlineto{\pgfqpoint{2.016517in}{1.846409in}}%
\pgfpathlineto{\pgfqpoint{2.045797in}{1.854518in}}%
\pgfpathlineto{\pgfqpoint{2.089717in}{1.864300in}}%
\pgfpathlineto{\pgfqpoint{2.118997in}{1.869836in}}%
\pgfpathlineto{\pgfqpoint{2.206836in}{1.885592in}}%
\pgfpathlineto{\pgfqpoint{2.265396in}{1.893569in}}%
\pgfpathlineto{\pgfqpoint{2.397155in}{1.904005in}}%
\pgfpathlineto{\pgfqpoint{2.455714in}{1.905659in}}%
\pgfpathlineto{\pgfqpoint{2.514274in}{1.905938in}}%
\pgfpathlineto{\pgfqpoint{2.558194in}{1.904842in}}%
\pgfpathlineto{\pgfqpoint{2.660673in}{1.898446in}}%
\pgfpathlineto{\pgfqpoint{2.777792in}{1.883829in}}%
\pgfpathlineto{\pgfqpoint{2.924191in}{1.854476in}}%
\pgfpathlineto{\pgfqpoint{2.953471in}{1.847007in}}%
\pgfpathlineto{\pgfqpoint{3.041311in}{1.821777in}}%
\pgfpathlineto{\pgfqpoint{3.099870in}{1.802002in}}%
\pgfpathlineto{\pgfqpoint{3.173070in}{1.773770in}}%
\pgfpathlineto{\pgfqpoint{3.260909in}{1.734878in}}%
\pgfpathlineto{\pgfqpoint{3.319469in}{1.705692in}}%
\pgfpathlineto{\pgfqpoint{3.378028in}{1.673804in}}%
\pgfpathlineto{\pgfqpoint{3.451228in}{1.629885in}}%
\pgfpathlineto{\pgfqpoint{3.524427in}{1.581609in}}%
\pgfpathlineto{\pgfqpoint{3.568347in}{1.550435in}}%
\pgfpathlineto{\pgfqpoint{3.612267in}{1.517747in}}%
\pgfpathlineto{\pgfqpoint{3.670827in}{1.471263in}}%
\pgfpathlineto{\pgfqpoint{3.714746in}{1.434195in}}%
\pgfpathlineto{\pgfqpoint{3.773306in}{1.382222in}}%
\pgfpathlineto{\pgfqpoint{3.831866in}{1.326426in}}%
\pgfpathlineto{\pgfqpoint{3.890425in}{1.267035in}}%
\pgfpathlineto{\pgfqpoint{3.948985in}{1.203506in}}%
\pgfpathlineto{\pgfqpoint{3.992904in}{1.152998in}}%
\pgfpathlineto{\pgfqpoint{3.992904in}{1.152998in}}%
\pgfusepath{stroke}%
\end{pgfscope}%
\begin{pgfscope}%
\pgfpathrectangle{\pgfqpoint{0.456635in}{0.521603in}}{\pgfqpoint{3.720000in}{3.020000in}} %
\pgfusepath{clip}%
\pgfsetrectcap%
\pgfsetroundjoin%
\pgfsetlinewidth{1.505625pt}%
\definecolor{currentstroke}{rgb}{0.676471,0.961826,0.602635}%
\pgfsetstrokecolor{currentstroke}%
\pgfsetdash{}{0pt}%
\pgfpathmoveto{\pgfqpoint{0.625726in}{0.723991in}}%
\pgfpathlineto{\pgfqpoint{0.640366in}{0.769626in}}%
\pgfpathlineto{\pgfqpoint{0.655006in}{0.798991in}}%
\pgfpathlineto{\pgfqpoint{0.669646in}{0.818221in}}%
\pgfpathlineto{\pgfqpoint{0.684286in}{0.841608in}}%
\pgfpathlineto{\pgfqpoint{0.698926in}{0.877155in}}%
\pgfpathlineto{\pgfqpoint{0.713566in}{0.907224in}}%
\pgfpathlineto{\pgfqpoint{0.742845in}{0.946129in}}%
\pgfpathlineto{\pgfqpoint{0.757485in}{0.973013in}}%
\pgfpathlineto{\pgfqpoint{0.772125in}{0.991182in}}%
\pgfpathlineto{\pgfqpoint{0.786765in}{1.015200in}}%
\pgfpathlineto{\pgfqpoint{0.801405in}{1.032217in}}%
\pgfpathlineto{\pgfqpoint{0.816045in}{1.045872in}}%
\pgfpathlineto{\pgfqpoint{0.830685in}{1.064878in}}%
\pgfpathlineto{\pgfqpoint{0.845325in}{1.089004in}}%
\pgfpathlineto{\pgfqpoint{0.859965in}{1.108565in}}%
\pgfpathlineto{\pgfqpoint{0.889244in}{1.135246in}}%
\pgfpathlineto{\pgfqpoint{0.918524in}{1.174356in}}%
\pgfpathlineto{\pgfqpoint{0.933164in}{1.186982in}}%
\pgfpathlineto{\pgfqpoint{0.947804in}{1.196767in}}%
\pgfpathlineto{\pgfqpoint{0.962444in}{1.208042in}}%
\pgfpathlineto{\pgfqpoint{0.991724in}{1.242442in}}%
\pgfpathlineto{\pgfqpoint{1.006364in}{1.252412in}}%
\pgfpathlineto{\pgfqpoint{1.021004in}{1.259485in}}%
\pgfpathlineto{\pgfqpoint{1.035644in}{1.271790in}}%
\pgfpathlineto{\pgfqpoint{1.050283in}{1.287133in}}%
\pgfpathlineto{\pgfqpoint{1.064923in}{1.298516in}}%
\pgfpathlineto{\pgfqpoint{1.094203in}{1.310900in}}%
\pgfpathlineto{\pgfqpoint{1.108843in}{1.321669in}}%
\pgfpathlineto{\pgfqpoint{1.123483in}{1.334681in}}%
\pgfpathlineto{\pgfqpoint{1.138123in}{1.343697in}}%
\pgfpathlineto{\pgfqpoint{1.152763in}{1.346561in}}%
\pgfpathlineto{\pgfqpoint{1.167403in}{1.353440in}}%
\pgfpathlineto{\pgfqpoint{1.182043in}{1.364228in}}%
\pgfpathlineto{\pgfqpoint{1.196683in}{1.372868in}}%
\pgfpathlineto{\pgfqpoint{1.211322in}{1.377271in}}%
\pgfpathlineto{\pgfqpoint{1.225962in}{1.379321in}}%
\pgfpathlineto{\pgfqpoint{1.240602in}{1.384414in}}%
\pgfpathlineto{\pgfqpoint{1.269882in}{1.399266in}}%
\pgfpathlineto{\pgfqpoint{1.284522in}{1.399731in}}%
\pgfpathlineto{\pgfqpoint{1.299162in}{1.401507in}}%
\pgfpathlineto{\pgfqpoint{1.328442in}{1.413073in}}%
\pgfpathlineto{\pgfqpoint{1.343082in}{1.415304in}}%
\pgfpathlineto{\pgfqpoint{1.357721in}{1.414213in}}%
\pgfpathlineto{\pgfqpoint{1.372361in}{1.414503in}}%
\pgfpathlineto{\pgfqpoint{1.401641in}{1.422141in}}%
\pgfpathlineto{\pgfqpoint{1.430921in}{1.418317in}}%
\pgfpathlineto{\pgfqpoint{1.445561in}{1.419228in}}%
\pgfpathlineto{\pgfqpoint{1.460201in}{1.421553in}}%
\pgfpathlineto{\pgfqpoint{1.474841in}{1.421434in}}%
\pgfpathlineto{\pgfqpoint{1.504121in}{1.414074in}}%
\pgfpathlineto{\pgfqpoint{1.518760in}{1.413015in}}%
\pgfpathlineto{\pgfqpoint{1.533400in}{1.413715in}}%
\pgfpathlineto{\pgfqpoint{1.548040in}{1.410014in}}%
\pgfpathlineto{\pgfqpoint{1.562680in}{1.403926in}}%
\pgfpathlineto{\pgfqpoint{1.577320in}{1.400369in}}%
\pgfpathlineto{\pgfqpoint{1.591960in}{1.398974in}}%
\pgfpathlineto{\pgfqpoint{1.606600in}{1.395960in}}%
\pgfpathlineto{\pgfqpoint{1.650520in}{1.376649in}}%
\pgfpathlineto{\pgfqpoint{1.665159in}{1.373407in}}%
\pgfpathlineto{\pgfqpoint{1.679799in}{1.367237in}}%
\pgfpathlineto{\pgfqpoint{1.694439in}{1.358001in}}%
\pgfpathlineto{\pgfqpoint{1.709079in}{1.350196in}}%
\pgfpathlineto{\pgfqpoint{1.738359in}{1.338091in}}%
\pgfpathlineto{\pgfqpoint{1.752999in}{1.328956in}}%
\pgfpathlineto{\pgfqpoint{1.782279in}{1.307966in}}%
\pgfpathlineto{\pgfqpoint{1.796919in}{1.300642in}}%
\pgfpathlineto{\pgfqpoint{1.811559in}{1.291365in}}%
\pgfpathlineto{\pgfqpoint{1.840838in}{1.267059in}}%
\pgfpathlineto{\pgfqpoint{1.884758in}{1.233923in}}%
\pgfpathlineto{\pgfqpoint{1.914038in}{1.204529in}}%
\pgfpathlineto{\pgfqpoint{1.943318in}{1.178919in}}%
\pgfpathlineto{\pgfqpoint{2.016517in}{1.099160in}}%
\pgfpathlineto{\pgfqpoint{2.089717in}{1.003535in}}%
\pgfpathlineto{\pgfqpoint{2.148276in}{0.916469in}}%
\pgfpathlineto{\pgfqpoint{2.206836in}{0.816016in}}%
\pgfpathlineto{\pgfqpoint{2.236116in}{0.758762in}}%
\pgfpathlineto{\pgfqpoint{2.250756in}{0.739752in}}%
\pgfpathlineto{\pgfqpoint{2.280036in}{0.706330in}}%
\pgfpathlineto{\pgfqpoint{2.294675in}{0.700062in}}%
\pgfpathlineto{\pgfqpoint{2.309315in}{0.704587in}}%
\pgfpathlineto{\pgfqpoint{2.323955in}{0.713714in}}%
\pgfpathlineto{\pgfqpoint{2.338595in}{0.729888in}}%
\pgfpathlineto{\pgfqpoint{2.353235in}{0.752111in}}%
\pgfpathlineto{\pgfqpoint{2.367875in}{0.770814in}}%
\pgfpathlineto{\pgfqpoint{2.382515in}{0.800461in}}%
\pgfpathlineto{\pgfqpoint{2.411795in}{0.844837in}}%
\pgfpathlineto{\pgfqpoint{2.426435in}{0.861958in}}%
\pgfpathlineto{\pgfqpoint{2.441074in}{0.877195in}}%
\pgfpathlineto{\pgfqpoint{2.470354in}{0.911880in}}%
\pgfpathlineto{\pgfqpoint{2.484994in}{0.926760in}}%
\pgfpathlineto{\pgfqpoint{2.528914in}{0.966476in}}%
\pgfpathlineto{\pgfqpoint{2.543554in}{0.979890in}}%
\pgfpathlineto{\pgfqpoint{2.572834in}{0.998856in}}%
\pgfpathlineto{\pgfqpoint{2.587474in}{1.008032in}}%
\pgfpathlineto{\pgfqpoint{2.602113in}{1.018763in}}%
\pgfpathlineto{\pgfqpoint{2.616753in}{1.026908in}}%
\pgfpathlineto{\pgfqpoint{2.660673in}{1.047001in}}%
\pgfpathlineto{\pgfqpoint{2.675313in}{1.054076in}}%
\pgfpathlineto{\pgfqpoint{2.689953in}{1.058535in}}%
\pgfpathlineto{\pgfqpoint{2.719233in}{1.064352in}}%
\pgfpathlineto{\pgfqpoint{2.733873in}{1.068835in}}%
\pgfpathlineto{\pgfqpoint{2.748512in}{1.071944in}}%
\pgfpathlineto{\pgfqpoint{2.792432in}{1.074202in}}%
\pgfpathlineto{\pgfqpoint{2.807072in}{1.075131in}}%
\pgfpathlineto{\pgfqpoint{2.821712in}{1.074885in}}%
\pgfpathlineto{\pgfqpoint{2.850992in}{1.069395in}}%
\pgfpathlineto{\pgfqpoint{2.880272in}{1.065350in}}%
\pgfpathlineto{\pgfqpoint{2.953471in}{1.039470in}}%
\pgfpathlineto{\pgfqpoint{3.012031in}{1.006229in}}%
\pgfpathlineto{\pgfqpoint{3.041311in}{0.983998in}}%
\pgfpathlineto{\pgfqpoint{3.085230in}{0.948516in}}%
\pgfpathlineto{\pgfqpoint{3.114510in}{0.918175in}}%
\pgfpathlineto{\pgfqpoint{3.143790in}{0.886120in}}%
\pgfpathlineto{\pgfqpoint{3.173070in}{0.848274in}}%
\pgfpathlineto{\pgfqpoint{3.202350in}{0.806122in}}%
\pgfpathlineto{\pgfqpoint{3.231629in}{0.758749in}}%
\pgfpathlineto{\pgfqpoint{3.246269in}{0.737224in}}%
\pgfpathlineto{\pgfqpoint{3.260909in}{0.719826in}}%
\pgfpathlineto{\pgfqpoint{3.275549in}{0.706124in}}%
\pgfpathlineto{\pgfqpoint{3.290189in}{0.705328in}}%
\pgfpathlineto{\pgfqpoint{3.304829in}{0.713343in}}%
\pgfpathlineto{\pgfqpoint{3.319469in}{0.729921in}}%
\pgfpathlineto{\pgfqpoint{3.348749in}{0.771368in}}%
\pgfpathlineto{\pgfqpoint{3.363389in}{0.793717in}}%
\pgfpathlineto{\pgfqpoint{3.378028in}{0.812499in}}%
\pgfpathlineto{\pgfqpoint{3.392668in}{0.835521in}}%
\pgfpathlineto{\pgfqpoint{3.407308in}{0.854257in}}%
\pgfpathlineto{\pgfqpoint{3.436588in}{0.884432in}}%
\pgfpathlineto{\pgfqpoint{3.451228in}{0.895825in}}%
\pgfpathlineto{\pgfqpoint{3.480508in}{0.921222in}}%
\pgfpathlineto{\pgfqpoint{3.495148in}{0.932392in}}%
\pgfpathlineto{\pgfqpoint{3.553707in}{0.966072in}}%
\pgfpathlineto{\pgfqpoint{3.568347in}{0.971496in}}%
\pgfpathlineto{\pgfqpoint{3.597627in}{0.979862in}}%
\pgfpathlineto{\pgfqpoint{3.612267in}{0.985311in}}%
\pgfpathlineto{\pgfqpoint{3.626907in}{0.988750in}}%
\pgfpathlineto{\pgfqpoint{3.685466in}{0.993763in}}%
\pgfpathlineto{\pgfqpoint{3.700106in}{0.992602in}}%
\pgfpathlineto{\pgfqpoint{3.729386in}{0.987230in}}%
\pgfpathlineto{\pgfqpoint{3.744026in}{0.985985in}}%
\pgfpathlineto{\pgfqpoint{3.758666in}{0.982802in}}%
\pgfpathlineto{\pgfqpoint{3.773306in}{0.978073in}}%
\pgfpathlineto{\pgfqpoint{3.831866in}{0.952877in}}%
\pgfpathlineto{\pgfqpoint{3.875785in}{0.924402in}}%
\pgfpathlineto{\pgfqpoint{3.890425in}{0.914319in}}%
\pgfpathlineto{\pgfqpoint{3.905065in}{0.902412in}}%
\pgfpathlineto{\pgfqpoint{3.948985in}{0.859816in}}%
\pgfpathlineto{\pgfqpoint{3.963625in}{0.844425in}}%
\pgfpathlineto{\pgfqpoint{3.963625in}{0.844425in}}%
\pgfusepath{stroke}%
\end{pgfscope}%
\begin{pgfscope}%
\pgfpathrectangle{\pgfqpoint{0.456635in}{0.521603in}}{\pgfqpoint{3.720000in}{3.020000in}} %
\pgfusepath{clip}%
\pgfsetrectcap%
\pgfsetroundjoin%
\pgfsetlinewidth{1.505625pt}%
\definecolor{currentstroke}{rgb}{1.000000,0.000000,0.000000}%
\pgfsetstrokecolor{currentstroke}%
\pgfsetdash{}{0pt}%
\pgfpathmoveto{\pgfqpoint{0.713566in}{0.744513in}}%
\pgfpathlineto{\pgfqpoint{0.742845in}{0.754844in}}%
\pgfpathlineto{\pgfqpoint{0.757485in}{0.758737in}}%
\pgfpathlineto{\pgfqpoint{0.772125in}{0.764319in}}%
\pgfpathlineto{\pgfqpoint{0.786765in}{0.768553in}}%
\pgfpathlineto{\pgfqpoint{0.801405in}{0.770678in}}%
\pgfpathlineto{\pgfqpoint{0.830685in}{0.778906in}}%
\pgfpathlineto{\pgfqpoint{0.845325in}{0.778466in}}%
\pgfpathlineto{\pgfqpoint{0.874605in}{0.782565in}}%
\pgfpathlineto{\pgfqpoint{0.889244in}{0.782131in}}%
\pgfpathlineto{\pgfqpoint{0.918524in}{0.783032in}}%
\pgfpathlineto{\pgfqpoint{0.977084in}{0.777015in}}%
\pgfpathlineto{\pgfqpoint{1.021004in}{0.767397in}}%
\pgfpathlineto{\pgfqpoint{1.064923in}{0.753119in}}%
\pgfpathlineto{\pgfqpoint{1.108843in}{0.733987in}}%
\pgfpathlineto{\pgfqpoint{1.123483in}{0.725958in}}%
\pgfpathlineto{\pgfqpoint{1.152763in}{0.707133in}}%
\pgfpathlineto{\pgfqpoint{1.196683in}{0.667825in}}%
\pgfpathlineto{\pgfqpoint{1.211322in}{0.660773in}}%
\pgfpathlineto{\pgfqpoint{1.225962in}{0.663499in}}%
\pgfpathlineto{\pgfqpoint{1.240602in}{0.673788in}}%
\pgfpathlineto{\pgfqpoint{1.269882in}{0.696291in}}%
\pgfpathlineto{\pgfqpoint{1.284522in}{0.703579in}}%
\pgfpathlineto{\pgfqpoint{1.313802in}{0.715027in}}%
\pgfpathlineto{\pgfqpoint{1.328442in}{0.718017in}}%
\pgfpathlineto{\pgfqpoint{1.357721in}{0.721316in}}%
\pgfpathlineto{\pgfqpoint{1.372361in}{0.721022in}}%
\pgfpathlineto{\pgfqpoint{1.401641in}{0.716713in}}%
\pgfpathlineto{\pgfqpoint{1.430921in}{0.708418in}}%
\pgfpathlineto{\pgfqpoint{1.460201in}{0.693378in}}%
\pgfpathlineto{\pgfqpoint{1.474841in}{0.683940in}}%
\pgfpathlineto{\pgfqpoint{1.489481in}{0.672239in}}%
\pgfpathlineto{\pgfqpoint{1.504121in}{0.662450in}}%
\pgfpathlineto{\pgfqpoint{1.518760in}{0.659721in}}%
\pgfpathlineto{\pgfqpoint{1.533400in}{0.664295in}}%
\pgfpathlineto{\pgfqpoint{1.562680in}{0.682885in}}%
\pgfpathlineto{\pgfqpoint{1.577320in}{0.690500in}}%
\pgfpathlineto{\pgfqpoint{1.591960in}{0.695235in}}%
\pgfpathlineto{\pgfqpoint{1.621240in}{0.702010in}}%
\pgfpathlineto{\pgfqpoint{1.650520in}{0.701157in}}%
\pgfpathlineto{\pgfqpoint{1.665159in}{0.699875in}}%
\pgfpathlineto{\pgfqpoint{1.694439in}{0.691160in}}%
\pgfpathlineto{\pgfqpoint{1.709079in}{0.684377in}}%
\pgfpathlineto{\pgfqpoint{1.738359in}{0.666700in}}%
\pgfpathlineto{\pgfqpoint{1.752999in}{0.660091in}}%
\pgfpathlineto{\pgfqpoint{1.767639in}{0.660342in}}%
\pgfpathlineto{\pgfqpoint{1.782279in}{0.665597in}}%
\pgfpathlineto{\pgfqpoint{1.796919in}{0.674506in}}%
\pgfpathlineto{\pgfqpoint{1.811559in}{0.680747in}}%
\pgfpathlineto{\pgfqpoint{1.826198in}{0.685673in}}%
\pgfpathlineto{\pgfqpoint{1.840838in}{0.687723in}}%
\pgfpathlineto{\pgfqpoint{1.855478in}{0.687629in}}%
\pgfpathlineto{\pgfqpoint{1.870118in}{0.685857in}}%
\pgfpathlineto{\pgfqpoint{1.884758in}{0.682120in}}%
\pgfpathlineto{\pgfqpoint{1.899398in}{0.675901in}}%
\pgfpathlineto{\pgfqpoint{1.928678in}{0.661291in}}%
\pgfpathlineto{\pgfqpoint{1.943318in}{0.658876in}}%
\pgfpathlineto{\pgfqpoint{1.957958in}{0.662430in}}%
\pgfpathlineto{\pgfqpoint{1.987237in}{0.676108in}}%
\pgfpathlineto{\pgfqpoint{2.001877in}{0.679739in}}%
\pgfpathlineto{\pgfqpoint{2.016517in}{0.679982in}}%
\pgfpathlineto{\pgfqpoint{2.031157in}{0.677360in}}%
\pgfpathlineto{\pgfqpoint{2.075077in}{0.663171in}}%
\pgfpathlineto{\pgfqpoint{2.089717in}{0.662075in}}%
\pgfpathlineto{\pgfqpoint{2.104357in}{0.663916in}}%
\pgfpathlineto{\pgfqpoint{2.118997in}{0.669254in}}%
\pgfpathlineto{\pgfqpoint{2.133636in}{0.673229in}}%
\pgfpathlineto{\pgfqpoint{2.148276in}{0.675850in}}%
\pgfpathlineto{\pgfqpoint{2.162916in}{0.675189in}}%
\pgfpathlineto{\pgfqpoint{2.177556in}{0.671194in}}%
\pgfpathlineto{\pgfqpoint{2.192196in}{0.666026in}}%
\pgfpathlineto{\pgfqpoint{2.206836in}{0.662058in}}%
\pgfpathlineto{\pgfqpoint{2.221476in}{0.661826in}}%
\pgfpathlineto{\pgfqpoint{2.236116in}{0.664860in}}%
\pgfpathlineto{\pgfqpoint{2.265396in}{0.672954in}}%
\pgfpathlineto{\pgfqpoint{2.280036in}{0.674296in}}%
\pgfpathlineto{\pgfqpoint{2.294675in}{0.672611in}}%
\pgfpathlineto{\pgfqpoint{2.338595in}{0.663810in}}%
\pgfpathlineto{\pgfqpoint{2.353235in}{0.664157in}}%
\pgfpathlineto{\pgfqpoint{2.397155in}{0.671991in}}%
\pgfpathlineto{\pgfqpoint{2.411795in}{0.672391in}}%
\pgfpathlineto{\pgfqpoint{2.426435in}{0.668634in}}%
\pgfpathlineto{\pgfqpoint{2.441074in}{0.667368in}}%
\pgfpathlineto{\pgfqpoint{2.455714in}{0.664927in}}%
\pgfpathlineto{\pgfqpoint{2.484994in}{0.666911in}}%
\pgfpathlineto{\pgfqpoint{2.514274in}{0.671071in}}%
\pgfpathlineto{\pgfqpoint{2.528914in}{0.671758in}}%
\pgfpathlineto{\pgfqpoint{2.558194in}{0.669422in}}%
\pgfpathlineto{\pgfqpoint{2.587474in}{0.666092in}}%
\pgfpathlineto{\pgfqpoint{2.616753in}{0.667017in}}%
\pgfpathlineto{\pgfqpoint{2.660673in}{0.671383in}}%
\pgfpathlineto{\pgfqpoint{2.675313in}{0.671291in}}%
\pgfpathlineto{\pgfqpoint{2.733873in}{0.666837in}}%
\pgfpathlineto{\pgfqpoint{2.807072in}{0.669736in}}%
\pgfpathlineto{\pgfqpoint{2.836352in}{0.667182in}}%
\pgfpathlineto{\pgfqpoint{2.865632in}{0.667123in}}%
\pgfpathlineto{\pgfqpoint{2.909551in}{0.670099in}}%
\pgfpathlineto{\pgfqpoint{2.938831in}{0.668344in}}%
\pgfpathlineto{\pgfqpoint{2.968111in}{0.666498in}}%
\pgfpathlineto{\pgfqpoint{2.997391in}{0.667777in}}%
\pgfpathlineto{\pgfqpoint{3.026671in}{0.669857in}}%
\pgfpathlineto{\pgfqpoint{3.055951in}{0.669227in}}%
\pgfpathlineto{\pgfqpoint{3.085230in}{0.667153in}}%
\pgfpathlineto{\pgfqpoint{3.114510in}{0.667180in}}%
\pgfpathlineto{\pgfqpoint{3.173070in}{0.669572in}}%
\pgfpathlineto{\pgfqpoint{3.231629in}{0.666723in}}%
\pgfpathlineto{\pgfqpoint{3.290189in}{0.668790in}}%
\pgfpathlineto{\pgfqpoint{3.378028in}{0.667594in}}%
\pgfpathlineto{\pgfqpoint{3.407308in}{0.668180in}}%
\pgfpathlineto{\pgfqpoint{3.465868in}{0.666640in}}%
\pgfpathlineto{\pgfqpoint{3.539067in}{0.667431in}}%
\pgfpathlineto{\pgfqpoint{3.597627in}{0.667095in}}%
\pgfpathlineto{\pgfqpoint{3.656187in}{0.667620in}}%
\pgfpathlineto{\pgfqpoint{3.714746in}{0.667098in}}%
\pgfpathlineto{\pgfqpoint{3.758666in}{0.667772in}}%
\pgfpathlineto{\pgfqpoint{3.817226in}{0.666881in}}%
\pgfpathlineto{\pgfqpoint{3.890425in}{0.667773in}}%
\pgfpathlineto{\pgfqpoint{3.948985in}{0.667250in}}%
\pgfpathlineto{\pgfqpoint{3.978265in}{0.667987in}}%
\pgfpathlineto{\pgfqpoint{3.978265in}{0.667987in}}%
\pgfusepath{stroke}%
\end{pgfscope}%
\begin{pgfscope}%
\pgfpathrectangle{\pgfqpoint{0.456635in}{0.521603in}}{\pgfqpoint{3.720000in}{3.020000in}} %
\pgfusepath{clip}%
\pgfsetrectcap%
\pgfsetroundjoin%
\pgfsetlinewidth{1.505625pt}%
\definecolor{currentstroke}{rgb}{0.958824,0.751332,0.412356}%
\pgfsetstrokecolor{currentstroke}%
\pgfsetdash{}{0pt}%
\pgfpathmoveto{\pgfqpoint{0.713566in}{0.835471in}}%
\pgfpathlineto{\pgfqpoint{0.728206in}{0.847490in}}%
\pgfpathlineto{\pgfqpoint{0.757485in}{0.889137in}}%
\pgfpathlineto{\pgfqpoint{0.772125in}{0.905233in}}%
\pgfpathlineto{\pgfqpoint{0.786765in}{0.916299in}}%
\pgfpathlineto{\pgfqpoint{0.816045in}{0.951147in}}%
\pgfpathlineto{\pgfqpoint{0.845325in}{0.971976in}}%
\pgfpathlineto{\pgfqpoint{0.859965in}{0.987231in}}%
\pgfpathlineto{\pgfqpoint{0.874605in}{1.000083in}}%
\pgfpathlineto{\pgfqpoint{0.889244in}{1.006651in}}%
\pgfpathlineto{\pgfqpoint{0.903884in}{1.014538in}}%
\pgfpathlineto{\pgfqpoint{0.918524in}{1.026987in}}%
\pgfpathlineto{\pgfqpoint{0.933164in}{1.035406in}}%
\pgfpathlineto{\pgfqpoint{0.947804in}{1.039112in}}%
\pgfpathlineto{\pgfqpoint{0.977084in}{1.056323in}}%
\pgfpathlineto{\pgfqpoint{0.991724in}{1.060205in}}%
\pgfpathlineto{\pgfqpoint{1.006364in}{1.061328in}}%
\pgfpathlineto{\pgfqpoint{1.035644in}{1.073370in}}%
\pgfpathlineto{\pgfqpoint{1.050283in}{1.073156in}}%
\pgfpathlineto{\pgfqpoint{1.064923in}{1.074139in}}%
\pgfpathlineto{\pgfqpoint{1.079563in}{1.078566in}}%
\pgfpathlineto{\pgfqpoint{1.094203in}{1.080318in}}%
\pgfpathlineto{\pgfqpoint{1.108843in}{1.077031in}}%
\pgfpathlineto{\pgfqpoint{1.123483in}{1.076053in}}%
\pgfpathlineto{\pgfqpoint{1.138123in}{1.077879in}}%
\pgfpathlineto{\pgfqpoint{1.152763in}{1.075132in}}%
\pgfpathlineto{\pgfqpoint{1.167403in}{1.069922in}}%
\pgfpathlineto{\pgfqpoint{1.196683in}{1.066918in}}%
\pgfpathlineto{\pgfqpoint{1.225962in}{1.052246in}}%
\pgfpathlineto{\pgfqpoint{1.240602in}{1.048455in}}%
\pgfpathlineto{\pgfqpoint{1.255242in}{1.043038in}}%
\pgfpathlineto{\pgfqpoint{1.269882in}{1.032572in}}%
\pgfpathlineto{\pgfqpoint{1.284522in}{1.024380in}}%
\pgfpathlineto{\pgfqpoint{1.299162in}{1.018391in}}%
\pgfpathlineto{\pgfqpoint{1.313802in}{1.008334in}}%
\pgfpathlineto{\pgfqpoint{1.328442in}{0.994761in}}%
\pgfpathlineto{\pgfqpoint{1.357721in}{0.974528in}}%
\pgfpathlineto{\pgfqpoint{1.387001in}{0.944649in}}%
\pgfpathlineto{\pgfqpoint{1.401641in}{0.932223in}}%
\pgfpathlineto{\pgfqpoint{1.416281in}{0.917956in}}%
\pgfpathlineto{\pgfqpoint{1.445561in}{0.880457in}}%
\pgfpathlineto{\pgfqpoint{1.460201in}{0.864071in}}%
\pgfpathlineto{\pgfqpoint{1.474841in}{0.844462in}}%
\pgfpathlineto{\pgfqpoint{1.518760in}{0.775638in}}%
\pgfpathlineto{\pgfqpoint{1.533400in}{0.745253in}}%
\pgfpathlineto{\pgfqpoint{1.548040in}{0.719356in}}%
\pgfpathlineto{\pgfqpoint{1.562680in}{0.698014in}}%
\pgfpathlineto{\pgfqpoint{1.577320in}{0.678958in}}%
\pgfpathlineto{\pgfqpoint{1.591960in}{0.679096in}}%
\pgfpathlineto{\pgfqpoint{1.606600in}{0.685966in}}%
\pgfpathlineto{\pgfqpoint{1.621240in}{0.708358in}}%
\pgfpathlineto{\pgfqpoint{1.635880in}{0.727794in}}%
\pgfpathlineto{\pgfqpoint{1.650520in}{0.749581in}}%
\pgfpathlineto{\pgfqpoint{1.679799in}{0.800741in}}%
\pgfpathlineto{\pgfqpoint{1.694439in}{0.820594in}}%
\pgfpathlineto{\pgfqpoint{1.709079in}{0.836573in}}%
\pgfpathlineto{\pgfqpoint{1.752999in}{0.877556in}}%
\pgfpathlineto{\pgfqpoint{1.782279in}{0.897944in}}%
\pgfpathlineto{\pgfqpoint{1.811559in}{0.914104in}}%
\pgfpathlineto{\pgfqpoint{1.840838in}{0.925081in}}%
\pgfpathlineto{\pgfqpoint{1.855478in}{0.930155in}}%
\pgfpathlineto{\pgfqpoint{1.884758in}{0.934764in}}%
\pgfpathlineto{\pgfqpoint{1.914038in}{0.936155in}}%
\pgfpathlineto{\pgfqpoint{1.928678in}{0.934927in}}%
\pgfpathlineto{\pgfqpoint{1.972597in}{0.925832in}}%
\pgfpathlineto{\pgfqpoint{2.001877in}{0.913948in}}%
\pgfpathlineto{\pgfqpoint{2.016517in}{0.906638in}}%
\pgfpathlineto{\pgfqpoint{2.031157in}{0.898033in}}%
\pgfpathlineto{\pgfqpoint{2.060437in}{0.877335in}}%
\pgfpathlineto{\pgfqpoint{2.075077in}{0.865775in}}%
\pgfpathlineto{\pgfqpoint{2.104357in}{0.836848in}}%
\pgfpathlineto{\pgfqpoint{2.133636in}{0.802866in}}%
\pgfpathlineto{\pgfqpoint{2.148276in}{0.782420in}}%
\pgfpathlineto{\pgfqpoint{2.177556in}{0.732354in}}%
\pgfpathlineto{\pgfqpoint{2.192196in}{0.715859in}}%
\pgfpathlineto{\pgfqpoint{2.206836in}{0.703566in}}%
\pgfpathlineto{\pgfqpoint{2.221476in}{0.689101in}}%
\pgfpathlineto{\pgfqpoint{2.236116in}{0.690806in}}%
\pgfpathlineto{\pgfqpoint{2.250756in}{0.698193in}}%
\pgfpathlineto{\pgfqpoint{2.265396in}{0.712680in}}%
\pgfpathlineto{\pgfqpoint{2.280036in}{0.729615in}}%
\pgfpathlineto{\pgfqpoint{2.294675in}{0.753247in}}%
\pgfpathlineto{\pgfqpoint{2.309315in}{0.770631in}}%
\pgfpathlineto{\pgfqpoint{2.323955in}{0.785907in}}%
\pgfpathlineto{\pgfqpoint{2.338595in}{0.796207in}}%
\pgfpathlineto{\pgfqpoint{2.353235in}{0.811701in}}%
\pgfpathlineto{\pgfqpoint{2.367875in}{0.821377in}}%
\pgfpathlineto{\pgfqpoint{2.411795in}{0.837330in}}%
\pgfpathlineto{\pgfqpoint{2.426435in}{0.841228in}}%
\pgfpathlineto{\pgfqpoint{2.455714in}{0.838885in}}%
\pgfpathlineto{\pgfqpoint{2.470354in}{0.838832in}}%
\pgfpathlineto{\pgfqpoint{2.484994in}{0.836547in}}%
\pgfpathlineto{\pgfqpoint{2.499634in}{0.829860in}}%
\pgfpathlineto{\pgfqpoint{2.528914in}{0.819593in}}%
\pgfpathlineto{\pgfqpoint{2.543554in}{0.809405in}}%
\pgfpathlineto{\pgfqpoint{2.558194in}{0.792715in}}%
\pgfpathlineto{\pgfqpoint{2.572834in}{0.781736in}}%
\pgfpathlineto{\pgfqpoint{2.587474in}{0.768565in}}%
\pgfpathlineto{\pgfqpoint{2.602113in}{0.750635in}}%
\pgfpathlineto{\pgfqpoint{2.616753in}{0.729694in}}%
\pgfpathlineto{\pgfqpoint{2.631393in}{0.717143in}}%
\pgfpathlineto{\pgfqpoint{2.646033in}{0.709388in}}%
\pgfpathlineto{\pgfqpoint{2.660673in}{0.697743in}}%
\pgfpathlineto{\pgfqpoint{2.675313in}{0.698540in}}%
\pgfpathlineto{\pgfqpoint{2.689953in}{0.704416in}}%
\pgfpathlineto{\pgfqpoint{2.719233in}{0.728937in}}%
\pgfpathlineto{\pgfqpoint{2.733873in}{0.742480in}}%
\pgfpathlineto{\pgfqpoint{2.748512in}{0.759303in}}%
\pgfpathlineto{\pgfqpoint{2.763152in}{0.773047in}}%
\pgfpathlineto{\pgfqpoint{2.792432in}{0.784277in}}%
\pgfpathlineto{\pgfqpoint{2.807072in}{0.793376in}}%
\pgfpathlineto{\pgfqpoint{2.821712in}{0.796753in}}%
\pgfpathlineto{\pgfqpoint{2.836352in}{0.793110in}}%
\pgfpathlineto{\pgfqpoint{2.850992in}{0.792348in}}%
\pgfpathlineto{\pgfqpoint{2.865632in}{0.793142in}}%
\pgfpathlineto{\pgfqpoint{2.880272in}{0.782026in}}%
\pgfpathlineto{\pgfqpoint{2.894912in}{0.774155in}}%
\pgfpathlineto{\pgfqpoint{2.909551in}{0.763049in}}%
\pgfpathlineto{\pgfqpoint{2.953471in}{0.718895in}}%
\pgfpathlineto{\pgfqpoint{2.968111in}{0.710040in}}%
\pgfpathlineto{\pgfqpoint{2.982751in}{0.699669in}}%
\pgfpathlineto{\pgfqpoint{2.997391in}{0.699166in}}%
\pgfpathlineto{\pgfqpoint{3.012031in}{0.703379in}}%
\pgfpathlineto{\pgfqpoint{3.026671in}{0.713750in}}%
\pgfpathlineto{\pgfqpoint{3.041311in}{0.728719in}}%
\pgfpathlineto{\pgfqpoint{3.055951in}{0.738625in}}%
\pgfpathlineto{\pgfqpoint{3.070590in}{0.755275in}}%
\pgfpathlineto{\pgfqpoint{3.085230in}{0.770180in}}%
\pgfpathlineto{\pgfqpoint{3.114510in}{0.784535in}}%
\pgfpathlineto{\pgfqpoint{3.143790in}{0.795600in}}%
\pgfpathlineto{\pgfqpoint{3.173070in}{0.794884in}}%
\pgfpathlineto{\pgfqpoint{3.187710in}{0.793747in}}%
\pgfpathlineto{\pgfqpoint{3.216989in}{0.784082in}}%
\pgfpathlineto{\pgfqpoint{3.246269in}{0.768536in}}%
\pgfpathlineto{\pgfqpoint{3.260909in}{0.750410in}}%
\pgfpathlineto{\pgfqpoint{3.290189in}{0.727749in}}%
\pgfpathlineto{\pgfqpoint{3.319469in}{0.705218in}}%
\pgfpathlineto{\pgfqpoint{3.334109in}{0.702192in}}%
\pgfpathlineto{\pgfqpoint{3.348749in}{0.704276in}}%
\pgfpathlineto{\pgfqpoint{3.363389in}{0.711844in}}%
\pgfpathlineto{\pgfqpoint{3.378028in}{0.721462in}}%
\pgfpathlineto{\pgfqpoint{3.421948in}{0.754120in}}%
\pgfpathlineto{\pgfqpoint{3.451228in}{0.770571in}}%
\pgfpathlineto{\pgfqpoint{3.465868in}{0.777363in}}%
\pgfpathlineto{\pgfqpoint{3.480508in}{0.778059in}}%
\pgfpathlineto{\pgfqpoint{3.509788in}{0.776663in}}%
\pgfpathlineto{\pgfqpoint{3.524427in}{0.772679in}}%
\pgfpathlineto{\pgfqpoint{3.553707in}{0.753547in}}%
\pgfpathlineto{\pgfqpoint{3.582987in}{0.734218in}}%
\pgfpathlineto{\pgfqpoint{3.597627in}{0.722475in}}%
\pgfpathlineto{\pgfqpoint{3.626907in}{0.708369in}}%
\pgfpathlineto{\pgfqpoint{3.641547in}{0.706015in}}%
\pgfpathlineto{\pgfqpoint{3.656187in}{0.709342in}}%
\pgfpathlineto{\pgfqpoint{3.670827in}{0.717489in}}%
\pgfpathlineto{\pgfqpoint{3.700106in}{0.736799in}}%
\pgfpathlineto{\pgfqpoint{3.729386in}{0.755551in}}%
\pgfpathlineto{\pgfqpoint{3.744026in}{0.761817in}}%
\pgfpathlineto{\pgfqpoint{3.758666in}{0.773568in}}%
\pgfpathlineto{\pgfqpoint{3.773306in}{0.777566in}}%
\pgfpathlineto{\pgfqpoint{3.802586in}{0.779997in}}%
\pgfpathlineto{\pgfqpoint{3.817226in}{0.778640in}}%
\pgfpathlineto{\pgfqpoint{3.846505in}{0.772118in}}%
\pgfpathlineto{\pgfqpoint{3.861145in}{0.760681in}}%
\pgfpathlineto{\pgfqpoint{3.875785in}{0.753836in}}%
\pgfpathlineto{\pgfqpoint{3.905065in}{0.735299in}}%
\pgfpathlineto{\pgfqpoint{3.934345in}{0.717684in}}%
\pgfpathlineto{\pgfqpoint{3.948985in}{0.713737in}}%
\pgfpathlineto{\pgfqpoint{3.963625in}{0.711600in}}%
\pgfpathlineto{\pgfqpoint{3.978265in}{0.714993in}}%
\pgfpathlineto{\pgfqpoint{3.992904in}{0.721341in}}%
\pgfpathlineto{\pgfqpoint{3.992904in}{0.721341in}}%
\pgfusepath{stroke}%
\end{pgfscope}%
\begin{pgfscope}%
\pgfpathrectangle{\pgfqpoint{0.456635in}{0.521603in}}{\pgfqpoint{3.720000in}{3.020000in}} %
\pgfusepath{clip}%
\pgfsetrectcap%
\pgfsetroundjoin%
\pgfsetlinewidth{1.505625pt}%
\definecolor{currentstroke}{rgb}{1.000000,0.255843,0.128999}%
\pgfsetstrokecolor{currentstroke}%
\pgfsetdash{}{0pt}%
\pgfpathmoveto{\pgfqpoint{0.698926in}{0.793976in}}%
\pgfpathlineto{\pgfqpoint{0.713566in}{0.810676in}}%
\pgfpathlineto{\pgfqpoint{0.728206in}{0.823919in}}%
\pgfpathlineto{\pgfqpoint{0.742845in}{0.832514in}}%
\pgfpathlineto{\pgfqpoint{0.772125in}{0.856341in}}%
\pgfpathlineto{\pgfqpoint{0.786765in}{0.861852in}}%
\pgfpathlineto{\pgfqpoint{0.801405in}{0.871288in}}%
\pgfpathlineto{\pgfqpoint{0.816045in}{0.877833in}}%
\pgfpathlineto{\pgfqpoint{0.830685in}{0.880855in}}%
\pgfpathlineto{\pgfqpoint{0.845325in}{0.887872in}}%
\pgfpathlineto{\pgfqpoint{0.859965in}{0.890144in}}%
\pgfpathlineto{\pgfqpoint{0.874605in}{0.889611in}}%
\pgfpathlineto{\pgfqpoint{0.889244in}{0.894655in}}%
\pgfpathlineto{\pgfqpoint{0.903884in}{0.893836in}}%
\pgfpathlineto{\pgfqpoint{0.918524in}{0.890909in}}%
\pgfpathlineto{\pgfqpoint{0.933164in}{0.892340in}}%
\pgfpathlineto{\pgfqpoint{0.962444in}{0.883192in}}%
\pgfpathlineto{\pgfqpoint{0.977084in}{0.880800in}}%
\pgfpathlineto{\pgfqpoint{1.006364in}{0.865331in}}%
\pgfpathlineto{\pgfqpoint{1.021004in}{0.860026in}}%
\pgfpathlineto{\pgfqpoint{1.035644in}{0.848672in}}%
\pgfpathlineto{\pgfqpoint{1.064923in}{0.830052in}}%
\pgfpathlineto{\pgfqpoint{1.094203in}{0.801535in}}%
\pgfpathlineto{\pgfqpoint{1.108843in}{0.788907in}}%
\pgfpathlineto{\pgfqpoint{1.152763in}{0.732715in}}%
\pgfpathlineto{\pgfqpoint{1.167403in}{0.710818in}}%
\pgfpathlineto{\pgfqpoint{1.182043in}{0.695783in}}%
\pgfpathlineto{\pgfqpoint{1.196683in}{0.677920in}}%
\pgfpathlineto{\pgfqpoint{1.211322in}{0.678072in}}%
\pgfpathlineto{\pgfqpoint{1.225962in}{0.684655in}}%
\pgfpathlineto{\pgfqpoint{1.240602in}{0.701228in}}%
\pgfpathlineto{\pgfqpoint{1.255242in}{0.720577in}}%
\pgfpathlineto{\pgfqpoint{1.269882in}{0.737327in}}%
\pgfpathlineto{\pgfqpoint{1.313802in}{0.773150in}}%
\pgfpathlineto{\pgfqpoint{1.328442in}{0.775543in}}%
\pgfpathlineto{\pgfqpoint{1.343082in}{0.786176in}}%
\pgfpathlineto{\pgfqpoint{1.357721in}{0.791121in}}%
\pgfpathlineto{\pgfqpoint{1.372361in}{0.789441in}}%
\pgfpathlineto{\pgfqpoint{1.387001in}{0.794093in}}%
\pgfpathlineto{\pgfqpoint{1.401641in}{0.796106in}}%
\pgfpathlineto{\pgfqpoint{1.416281in}{0.790614in}}%
\pgfpathlineto{\pgfqpoint{1.430921in}{0.791915in}}%
\pgfpathlineto{\pgfqpoint{1.445561in}{0.786776in}}%
\pgfpathlineto{\pgfqpoint{1.460201in}{0.778773in}}%
\pgfpathlineto{\pgfqpoint{1.474841in}{0.775456in}}%
\pgfpathlineto{\pgfqpoint{1.489481in}{0.765427in}}%
\pgfpathlineto{\pgfqpoint{1.504121in}{0.752507in}}%
\pgfpathlineto{\pgfqpoint{1.518760in}{0.743108in}}%
\pgfpathlineto{\pgfqpoint{1.548040in}{0.710709in}}%
\pgfpathlineto{\pgfqpoint{1.562680in}{0.696702in}}%
\pgfpathlineto{\pgfqpoint{1.577320in}{0.684339in}}%
\pgfpathlineto{\pgfqpoint{1.591960in}{0.683412in}}%
\pgfpathlineto{\pgfqpoint{1.606600in}{0.687464in}}%
\pgfpathlineto{\pgfqpoint{1.635880in}{0.713026in}}%
\pgfpathlineto{\pgfqpoint{1.650520in}{0.728710in}}%
\pgfpathlineto{\pgfqpoint{1.665159in}{0.737816in}}%
\pgfpathlineto{\pgfqpoint{1.679799in}{0.749399in}}%
\pgfpathlineto{\pgfqpoint{1.694439in}{0.753682in}}%
\pgfpathlineto{\pgfqpoint{1.709079in}{0.753994in}}%
\pgfpathlineto{\pgfqpoint{1.723719in}{0.756821in}}%
\pgfpathlineto{\pgfqpoint{1.738359in}{0.754605in}}%
\pgfpathlineto{\pgfqpoint{1.767639in}{0.740558in}}%
\pgfpathlineto{\pgfqpoint{1.782279in}{0.729716in}}%
\pgfpathlineto{\pgfqpoint{1.796919in}{0.713752in}}%
\pgfpathlineto{\pgfqpoint{1.811559in}{0.704032in}}%
\pgfpathlineto{\pgfqpoint{1.826198in}{0.690824in}}%
\pgfpathlineto{\pgfqpoint{1.840838in}{0.686075in}}%
\pgfpathlineto{\pgfqpoint{1.855478in}{0.687906in}}%
\pgfpathlineto{\pgfqpoint{1.884758in}{0.707524in}}%
\pgfpathlineto{\pgfqpoint{1.899398in}{0.723243in}}%
\pgfpathlineto{\pgfqpoint{1.914038in}{0.733632in}}%
\pgfpathlineto{\pgfqpoint{1.928678in}{0.740188in}}%
\pgfpathlineto{\pgfqpoint{1.943318in}{0.745190in}}%
\pgfpathlineto{\pgfqpoint{1.957958in}{0.746480in}}%
\pgfpathlineto{\pgfqpoint{1.987237in}{0.740660in}}%
\pgfpathlineto{\pgfqpoint{2.001877in}{0.734541in}}%
\pgfpathlineto{\pgfqpoint{2.016517in}{0.723556in}}%
\pgfpathlineto{\pgfqpoint{2.031157in}{0.715739in}}%
\pgfpathlineto{\pgfqpoint{2.045797in}{0.702010in}}%
\pgfpathlineto{\pgfqpoint{2.060437in}{0.695183in}}%
\pgfpathlineto{\pgfqpoint{2.075077in}{0.691070in}}%
\pgfpathlineto{\pgfqpoint{2.089717in}{0.694616in}}%
\pgfpathlineto{\pgfqpoint{2.104357in}{0.700876in}}%
\pgfpathlineto{\pgfqpoint{2.118997in}{0.713187in}}%
\pgfpathlineto{\pgfqpoint{2.148276in}{0.730778in}}%
\pgfpathlineto{\pgfqpoint{2.162916in}{0.736501in}}%
\pgfpathlineto{\pgfqpoint{2.177556in}{0.739030in}}%
\pgfpathlineto{\pgfqpoint{2.192196in}{0.737526in}}%
\pgfpathlineto{\pgfqpoint{2.206836in}{0.731872in}}%
\pgfpathlineto{\pgfqpoint{2.236116in}{0.713340in}}%
\pgfpathlineto{\pgfqpoint{2.250756in}{0.703175in}}%
\pgfpathlineto{\pgfqpoint{2.265396in}{0.695458in}}%
\pgfpathlineto{\pgfqpoint{2.280036in}{0.692968in}}%
\pgfpathlineto{\pgfqpoint{2.294675in}{0.695941in}}%
\pgfpathlineto{\pgfqpoint{2.353235in}{0.732471in}}%
\pgfpathlineto{\pgfqpoint{2.367875in}{0.737107in}}%
\pgfpathlineto{\pgfqpoint{2.382515in}{0.738917in}}%
\pgfpathlineto{\pgfqpoint{2.397155in}{0.736212in}}%
\pgfpathlineto{\pgfqpoint{2.411795in}{0.730353in}}%
\pgfpathlineto{\pgfqpoint{2.455714in}{0.703239in}}%
\pgfpathlineto{\pgfqpoint{2.470354in}{0.697216in}}%
\pgfpathlineto{\pgfqpoint{2.484994in}{0.697196in}}%
\pgfpathlineto{\pgfqpoint{2.499634in}{0.702717in}}%
\pgfpathlineto{\pgfqpoint{2.543554in}{0.730498in}}%
\pgfpathlineto{\pgfqpoint{2.558194in}{0.736559in}}%
\pgfpathlineto{\pgfqpoint{2.572834in}{0.739310in}}%
\pgfpathlineto{\pgfqpoint{2.587474in}{0.738246in}}%
\pgfpathlineto{\pgfqpoint{2.602113in}{0.733559in}}%
\pgfpathlineto{\pgfqpoint{2.631393in}{0.717805in}}%
\pgfpathlineto{\pgfqpoint{2.646033in}{0.709189in}}%
\pgfpathlineto{\pgfqpoint{2.660673in}{0.702319in}}%
\pgfpathlineto{\pgfqpoint{2.675313in}{0.701187in}}%
\pgfpathlineto{\pgfqpoint{2.689953in}{0.704252in}}%
\pgfpathlineto{\pgfqpoint{2.704593in}{0.711515in}}%
\pgfpathlineto{\pgfqpoint{2.733873in}{0.729212in}}%
\pgfpathlineto{\pgfqpoint{2.748512in}{0.736376in}}%
\pgfpathlineto{\pgfqpoint{2.763152in}{0.739742in}}%
\pgfpathlineto{\pgfqpoint{2.777792in}{0.740852in}}%
\pgfpathlineto{\pgfqpoint{2.792432in}{0.738184in}}%
\pgfpathlineto{\pgfqpoint{2.821712in}{0.725223in}}%
\pgfpathlineto{\pgfqpoint{2.850992in}{0.709557in}}%
\pgfpathlineto{\pgfqpoint{2.865632in}{0.704998in}}%
\pgfpathlineto{\pgfqpoint{2.880272in}{0.704797in}}%
\pgfpathlineto{\pgfqpoint{2.894912in}{0.709057in}}%
\pgfpathlineto{\pgfqpoint{2.938831in}{0.732340in}}%
\pgfpathlineto{\pgfqpoint{2.953471in}{0.738224in}}%
\pgfpathlineto{\pgfqpoint{2.968111in}{0.740760in}}%
\pgfpathlineto{\pgfqpoint{2.982751in}{0.740355in}}%
\pgfpathlineto{\pgfqpoint{2.997391in}{0.736920in}}%
\pgfpathlineto{\pgfqpoint{3.055951in}{0.709950in}}%
\pgfpathlineto{\pgfqpoint{3.070590in}{0.708073in}}%
\pgfpathlineto{\pgfqpoint{3.085230in}{0.709356in}}%
\pgfpathlineto{\pgfqpoint{3.099870in}{0.715305in}}%
\pgfpathlineto{\pgfqpoint{3.129150in}{0.729659in}}%
\pgfpathlineto{\pgfqpoint{3.143790in}{0.736704in}}%
\pgfpathlineto{\pgfqpoint{3.158430in}{0.741691in}}%
\pgfpathlineto{\pgfqpoint{3.173070in}{0.742231in}}%
\pgfpathlineto{\pgfqpoint{3.187710in}{0.741058in}}%
\pgfpathlineto{\pgfqpoint{3.202350in}{0.736418in}}%
\pgfpathlineto{\pgfqpoint{3.260909in}{0.711635in}}%
\pgfpathlineto{\pgfqpoint{3.275549in}{0.710470in}}%
\pgfpathlineto{\pgfqpoint{3.290189in}{0.712661in}}%
\pgfpathlineto{\pgfqpoint{3.304829in}{0.717345in}}%
\pgfpathlineto{\pgfqpoint{3.348749in}{0.735047in}}%
\pgfpathlineto{\pgfqpoint{3.363389in}{0.738654in}}%
\pgfpathlineto{\pgfqpoint{3.378028in}{0.739474in}}%
\pgfpathlineto{\pgfqpoint{3.392668in}{0.736780in}}%
\pgfpathlineto{\pgfqpoint{3.407308in}{0.732693in}}%
\pgfpathlineto{\pgfqpoint{3.451228in}{0.716161in}}%
\pgfpathlineto{\pgfqpoint{3.465868in}{0.713208in}}%
\pgfpathlineto{\pgfqpoint{3.480508in}{0.712959in}}%
\pgfpathlineto{\pgfqpoint{3.495148in}{0.715585in}}%
\pgfpathlineto{\pgfqpoint{3.509788in}{0.719731in}}%
\pgfpathlineto{\pgfqpoint{3.539067in}{0.730512in}}%
\pgfpathlineto{\pgfqpoint{3.553707in}{0.733819in}}%
\pgfpathlineto{\pgfqpoint{3.568347in}{0.735427in}}%
\pgfpathlineto{\pgfqpoint{3.582987in}{0.734946in}}%
\pgfpathlineto{\pgfqpoint{3.597627in}{0.731983in}}%
\pgfpathlineto{\pgfqpoint{3.641547in}{0.717876in}}%
\pgfpathlineto{\pgfqpoint{3.656187in}{0.715059in}}%
\pgfpathlineto{\pgfqpoint{3.670827in}{0.714350in}}%
\pgfpathlineto{\pgfqpoint{3.685466in}{0.715740in}}%
\pgfpathlineto{\pgfqpoint{3.700106in}{0.719377in}}%
\pgfpathlineto{\pgfqpoint{3.714746in}{0.724174in}}%
\pgfpathlineto{\pgfqpoint{3.744026in}{0.731310in}}%
\pgfpathlineto{\pgfqpoint{3.758666in}{0.732873in}}%
\pgfpathlineto{\pgfqpoint{3.773306in}{0.732185in}}%
\pgfpathlineto{\pgfqpoint{3.787946in}{0.729996in}}%
\pgfpathlineto{\pgfqpoint{3.817226in}{0.722389in}}%
\pgfpathlineto{\pgfqpoint{3.831866in}{0.718663in}}%
\pgfpathlineto{\pgfqpoint{3.846505in}{0.716278in}}%
\pgfpathlineto{\pgfqpoint{3.861145in}{0.715532in}}%
\pgfpathlineto{\pgfqpoint{3.890425in}{0.719783in}}%
\pgfpathlineto{\pgfqpoint{3.934345in}{0.729494in}}%
\pgfpathlineto{\pgfqpoint{3.948985in}{0.730440in}}%
\pgfpathlineto{\pgfqpoint{3.963625in}{0.729959in}}%
\pgfpathlineto{\pgfqpoint{4.007544in}{0.721682in}}%
\pgfpathlineto{\pgfqpoint{4.007544in}{0.721682in}}%
\pgfusepath{stroke}%
\end{pgfscope}%
\begin{pgfscope}%
\pgfsetrectcap%
\pgfsetmiterjoin%
\pgfsetlinewidth{0.803000pt}%
\definecolor{currentstroke}{rgb}{0.000000,0.000000,0.000000}%
\pgfsetstrokecolor{currentstroke}%
\pgfsetdash{}{0pt}%
\pgfpathmoveto{\pgfqpoint{0.456635in}{0.521603in}}%
\pgfpathlineto{\pgfqpoint{0.456635in}{3.541603in}}%
\pgfusepath{stroke}%
\end{pgfscope}%
\begin{pgfscope}%
\pgfsetrectcap%
\pgfsetmiterjoin%
\pgfsetlinewidth{0.803000pt}%
\definecolor{currentstroke}{rgb}{0.000000,0.000000,0.000000}%
\pgfsetstrokecolor{currentstroke}%
\pgfsetdash{}{0pt}%
\pgfpathmoveto{\pgfqpoint{4.176635in}{0.521603in}}%
\pgfpathlineto{\pgfqpoint{4.176635in}{3.541603in}}%
\pgfusepath{stroke}%
\end{pgfscope}%
\begin{pgfscope}%
\pgfsetrectcap%
\pgfsetmiterjoin%
\pgfsetlinewidth{0.803000pt}%
\definecolor{currentstroke}{rgb}{0.000000,0.000000,0.000000}%
\pgfsetstrokecolor{currentstroke}%
\pgfsetdash{}{0pt}%
\pgfpathmoveto{\pgfqpoint{0.456635in}{0.521603in}}%
\pgfpathlineto{\pgfqpoint{4.176635in}{0.521603in}}%
\pgfusepath{stroke}%
\end{pgfscope}%
\begin{pgfscope}%
\pgfsetrectcap%
\pgfsetmiterjoin%
\pgfsetlinewidth{0.803000pt}%
\definecolor{currentstroke}{rgb}{0.000000,0.000000,0.000000}%
\pgfsetstrokecolor{currentstroke}%
\pgfsetdash{}{0pt}%
\pgfpathmoveto{\pgfqpoint{0.456635in}{3.541603in}}%
\pgfpathlineto{\pgfqpoint{4.176635in}{3.541603in}}%
\pgfusepath{stroke}%
\end{pgfscope}%
\begin{pgfscope}%
\pgfpathrectangle{\pgfqpoint{4.409135in}{0.521603in}}{\pgfqpoint{0.151000in}{3.020000in}} %
\pgfusepath{clip}%
\pgfsetbuttcap%
\pgfsetmiterjoin%
\definecolor{currentfill}{rgb}{1.000000,1.000000,1.000000}%
\pgfsetfillcolor{currentfill}%
\pgfsetlinewidth{0.010037pt}%
\definecolor{currentstroke}{rgb}{1.000000,1.000000,1.000000}%
\pgfsetstrokecolor{currentstroke}%
\pgfsetdash{}{0pt}%
\pgfpathmoveto{\pgfqpoint{4.409135in}{0.521603in}}%
\pgfpathlineto{\pgfqpoint{4.409135in}{0.533400in}}%
\pgfpathlineto{\pgfqpoint{4.409135in}{3.529806in}}%
\pgfpathlineto{\pgfqpoint{4.409135in}{3.541603in}}%
\pgfpathlineto{\pgfqpoint{4.560135in}{3.541603in}}%
\pgfpathlineto{\pgfqpoint{4.560135in}{3.529806in}}%
\pgfpathlineto{\pgfqpoint{4.560135in}{0.533400in}}%
\pgfpathlineto{\pgfqpoint{4.560135in}{0.521603in}}%
\pgfpathclose%
\pgfusepath{stroke,fill}%
\end{pgfscope}%
\begin{pgfscope}%
\pgfsys@transformshift{4.410000in}{0.524574in}%
\pgftext[left,bottom]{\pgfimage[interpolate=true,width=0.150000in,height=3.020000in]{series_s_eu-img0.png}}%
\end{pgfscope}%
\begin{pgfscope}%
\pgfsetbuttcap%
\pgfsetroundjoin%
\definecolor{currentfill}{rgb}{0.000000,0.000000,0.000000}%
\pgfsetfillcolor{currentfill}%
\pgfsetlinewidth{0.803000pt}%
\definecolor{currentstroke}{rgb}{0.000000,0.000000,0.000000}%
\pgfsetstrokecolor{currentstroke}%
\pgfsetdash{}{0pt}%
\pgfsys@defobject{currentmarker}{\pgfqpoint{0.000000in}{0.000000in}}{\pgfqpoint{0.048611in}{0.000000in}}{%
\pgfpathmoveto{\pgfqpoint{0.000000in}{0.000000in}}%
\pgfpathlineto{\pgfqpoint{0.048611in}{0.000000in}}%
\pgfusepath{stroke,fill}%
}%
\begin{pgfscope}%
\pgfsys@transformshift{4.560135in}{0.853575in}%
\pgfsys@useobject{currentmarker}{}%
\end{pgfscope}%
\end{pgfscope}%
\begin{pgfscope}%
\pgfsetbuttcap%
\pgfsetroundjoin%
\definecolor{currentfill}{rgb}{0.000000,0.000000,0.000000}%
\pgfsetfillcolor{currentfill}%
\pgfsetlinewidth{0.803000pt}%
\definecolor{currentstroke}{rgb}{0.000000,0.000000,0.000000}%
\pgfsetstrokecolor{currentstroke}%
\pgfsetdash{}{0pt}%
\pgfsys@defobject{currentmarker}{\pgfqpoint{0.000000in}{0.000000in}}{\pgfqpoint{0.048611in}{0.000000in}}{%
\pgfpathmoveto{\pgfqpoint{0.000000in}{0.000000in}}%
\pgfpathlineto{\pgfqpoint{0.048611in}{0.000000in}}%
\pgfusepath{stroke,fill}%
}%
\begin{pgfscope}%
\pgfsys@transformshift{4.560135in}{1.247759in}%
\pgfsys@useobject{currentmarker}{}%
\end{pgfscope}%
\end{pgfscope}%
\begin{pgfscope}%
\pgfsetbuttcap%
\pgfsetroundjoin%
\definecolor{currentfill}{rgb}{0.000000,0.000000,0.000000}%
\pgfsetfillcolor{currentfill}%
\pgfsetlinewidth{0.803000pt}%
\definecolor{currentstroke}{rgb}{0.000000,0.000000,0.000000}%
\pgfsetstrokecolor{currentstroke}%
\pgfsetdash{}{0pt}%
\pgfsys@defobject{currentmarker}{\pgfqpoint{0.000000in}{0.000000in}}{\pgfqpoint{0.048611in}{0.000000in}}{%
\pgfpathmoveto{\pgfqpoint{0.000000in}{0.000000in}}%
\pgfpathlineto{\pgfqpoint{0.048611in}{0.000000in}}%
\pgfusepath{stroke,fill}%
}%
\begin{pgfscope}%
\pgfsys@transformshift{4.560135in}{1.527436in}%
\pgfsys@useobject{currentmarker}{}%
\end{pgfscope}%
\end{pgfscope}%
\begin{pgfscope}%
\pgfsetbuttcap%
\pgfsetroundjoin%
\definecolor{currentfill}{rgb}{0.000000,0.000000,0.000000}%
\pgfsetfillcolor{currentfill}%
\pgfsetlinewidth{0.803000pt}%
\definecolor{currentstroke}{rgb}{0.000000,0.000000,0.000000}%
\pgfsetstrokecolor{currentstroke}%
\pgfsetdash{}{0pt}%
\pgfsys@defobject{currentmarker}{\pgfqpoint{0.000000in}{0.000000in}}{\pgfqpoint{0.048611in}{0.000000in}}{%
\pgfpathmoveto{\pgfqpoint{0.000000in}{0.000000in}}%
\pgfpathlineto{\pgfqpoint{0.048611in}{0.000000in}}%
\pgfusepath{stroke,fill}%
}%
\begin{pgfscope}%
\pgfsys@transformshift{4.560135in}{1.744371in}%
\pgfsys@useobject{currentmarker}{}%
\end{pgfscope}%
\end{pgfscope}%
\begin{pgfscope}%
\pgfsetbuttcap%
\pgfsetroundjoin%
\definecolor{currentfill}{rgb}{0.000000,0.000000,0.000000}%
\pgfsetfillcolor{currentfill}%
\pgfsetlinewidth{0.803000pt}%
\definecolor{currentstroke}{rgb}{0.000000,0.000000,0.000000}%
\pgfsetstrokecolor{currentstroke}%
\pgfsetdash{}{0pt}%
\pgfsys@defobject{currentmarker}{\pgfqpoint{0.000000in}{0.000000in}}{\pgfqpoint{0.048611in}{0.000000in}}{%
\pgfpathmoveto{\pgfqpoint{0.000000in}{0.000000in}}%
\pgfpathlineto{\pgfqpoint{0.048611in}{0.000000in}}%
\pgfusepath{stroke,fill}%
}%
\begin{pgfscope}%
\pgfsys@transformshift{4.560135in}{1.921620in}%
\pgfsys@useobject{currentmarker}{}%
\end{pgfscope}%
\end{pgfscope}%
\begin{pgfscope}%
\pgfsetbuttcap%
\pgfsetroundjoin%
\definecolor{currentfill}{rgb}{0.000000,0.000000,0.000000}%
\pgfsetfillcolor{currentfill}%
\pgfsetlinewidth{0.803000pt}%
\definecolor{currentstroke}{rgb}{0.000000,0.000000,0.000000}%
\pgfsetstrokecolor{currentstroke}%
\pgfsetdash{}{0pt}%
\pgfsys@defobject{currentmarker}{\pgfqpoint{0.000000in}{0.000000in}}{\pgfqpoint{0.048611in}{0.000000in}}{%
\pgfpathmoveto{\pgfqpoint{0.000000in}{0.000000in}}%
\pgfpathlineto{\pgfqpoint{0.048611in}{0.000000in}}%
\pgfusepath{stroke,fill}%
}%
\begin{pgfscope}%
\pgfsys@transformshift{4.560135in}{2.071482in}%
\pgfsys@useobject{currentmarker}{}%
\end{pgfscope}%
\end{pgfscope}%
\begin{pgfscope}%
\pgfsetbuttcap%
\pgfsetroundjoin%
\definecolor{currentfill}{rgb}{0.000000,0.000000,0.000000}%
\pgfsetfillcolor{currentfill}%
\pgfsetlinewidth{0.803000pt}%
\definecolor{currentstroke}{rgb}{0.000000,0.000000,0.000000}%
\pgfsetstrokecolor{currentstroke}%
\pgfsetdash{}{0pt}%
\pgfsys@defobject{currentmarker}{\pgfqpoint{0.000000in}{0.000000in}}{\pgfqpoint{0.048611in}{0.000000in}}{%
\pgfpathmoveto{\pgfqpoint{0.000000in}{0.000000in}}%
\pgfpathlineto{\pgfqpoint{0.048611in}{0.000000in}}%
\pgfusepath{stroke,fill}%
}%
\begin{pgfscope}%
\pgfsys@transformshift{4.560135in}{2.201298in}%
\pgfsys@useobject{currentmarker}{}%
\end{pgfscope}%
\end{pgfscope}%
\begin{pgfscope}%
\pgfsetbuttcap%
\pgfsetroundjoin%
\definecolor{currentfill}{rgb}{0.000000,0.000000,0.000000}%
\pgfsetfillcolor{currentfill}%
\pgfsetlinewidth{0.803000pt}%
\definecolor{currentstroke}{rgb}{0.000000,0.000000,0.000000}%
\pgfsetstrokecolor{currentstroke}%
\pgfsetdash{}{0pt}%
\pgfsys@defobject{currentmarker}{\pgfqpoint{0.000000in}{0.000000in}}{\pgfqpoint{0.048611in}{0.000000in}}{%
\pgfpathmoveto{\pgfqpoint{0.000000in}{0.000000in}}%
\pgfpathlineto{\pgfqpoint{0.048611in}{0.000000in}}%
\pgfusepath{stroke,fill}%
}%
\begin{pgfscope}%
\pgfsys@transformshift{4.560135in}{2.315804in}%
\pgfsys@useobject{currentmarker}{}%
\end{pgfscope}%
\end{pgfscope}%
\begin{pgfscope}%
\pgfsetbuttcap%
\pgfsetroundjoin%
\definecolor{currentfill}{rgb}{0.000000,0.000000,0.000000}%
\pgfsetfillcolor{currentfill}%
\pgfsetlinewidth{0.803000pt}%
\definecolor{currentstroke}{rgb}{0.000000,0.000000,0.000000}%
\pgfsetstrokecolor{currentstroke}%
\pgfsetdash{}{0pt}%
\pgfsys@defobject{currentmarker}{\pgfqpoint{0.000000in}{0.000000in}}{\pgfqpoint{0.048611in}{0.000000in}}{%
\pgfpathmoveto{\pgfqpoint{0.000000in}{0.000000in}}%
\pgfpathlineto{\pgfqpoint{0.048611in}{0.000000in}}%
\pgfusepath{stroke,fill}%
}%
\begin{pgfscope}%
\pgfsys@transformshift{4.560135in}{2.418233in}%
\pgfsys@useobject{currentmarker}{}%
\end{pgfscope}%
\end{pgfscope}%
\begin{pgfscope}%
\pgftext[x=4.657357in,y=2.365471in,left,base]{\rmfamily\fontsize{10.000000}{12.000000}\selectfont \(\displaystyle 10^{1}\)}%
\end{pgfscope}%
\begin{pgfscope}%
\pgfsetbuttcap%
\pgfsetroundjoin%
\definecolor{currentfill}{rgb}{0.000000,0.000000,0.000000}%
\pgfsetfillcolor{currentfill}%
\pgfsetlinewidth{0.803000pt}%
\definecolor{currentstroke}{rgb}{0.000000,0.000000,0.000000}%
\pgfsetstrokecolor{currentstroke}%
\pgfsetdash{}{0pt}%
\pgfsys@defobject{currentmarker}{\pgfqpoint{0.000000in}{0.000000in}}{\pgfqpoint{0.048611in}{0.000000in}}{%
\pgfpathmoveto{\pgfqpoint{0.000000in}{0.000000in}}%
\pgfpathlineto{\pgfqpoint{0.048611in}{0.000000in}}%
\pgfusepath{stroke,fill}%
}%
\begin{pgfscope}%
\pgfsys@transformshift{4.560135in}{3.092094in}%
\pgfsys@useobject{currentmarker}{}%
\end{pgfscope}%
\end{pgfscope}%
\begin{pgfscope}%
\pgfsetbuttcap%
\pgfsetroundjoin%
\definecolor{currentfill}{rgb}{0.000000,0.000000,0.000000}%
\pgfsetfillcolor{currentfill}%
\pgfsetlinewidth{0.803000pt}%
\definecolor{currentstroke}{rgb}{0.000000,0.000000,0.000000}%
\pgfsetstrokecolor{currentstroke}%
\pgfsetdash{}{0pt}%
\pgfsys@defobject{currentmarker}{\pgfqpoint{0.000000in}{0.000000in}}{\pgfqpoint{0.048611in}{0.000000in}}{%
\pgfpathmoveto{\pgfqpoint{0.000000in}{0.000000in}}%
\pgfpathlineto{\pgfqpoint{0.048611in}{0.000000in}}%
\pgfusepath{stroke,fill}%
}%
\begin{pgfscope}%
\pgfsys@transformshift{4.560135in}{3.486278in}%
\pgfsys@useobject{currentmarker}{}%
\end{pgfscope}%
\end{pgfscope}%
\begin{pgfscope}%
\pgftext[x=5.052999in,y=2.031603in,,top]{\rmfamily\fontsize{12.000000}{14.400000}\selectfont \(\displaystyle {\mathbf{E} \mbox{u}}\)}%
\end{pgfscope}%
\begin{pgfscope}%
\pgfsetbuttcap%
\pgfsetmiterjoin%
\pgfsetlinewidth{0.803000pt}%
\definecolor{currentstroke}{rgb}{0.000000,0.000000,0.000000}%
\pgfsetstrokecolor{currentstroke}%
\pgfsetdash{}{0pt}%
\pgfpathmoveto{\pgfqpoint{4.409135in}{0.521603in}}%
\pgfpathlineto{\pgfqpoint{4.409135in}{0.533400in}}%
\pgfpathlineto{\pgfqpoint{4.409135in}{3.529806in}}%
\pgfpathlineto{\pgfqpoint{4.409135in}{3.541603in}}%
\pgfpathlineto{\pgfqpoint{4.560135in}{3.541603in}}%
\pgfpathlineto{\pgfqpoint{4.560135in}{3.529806in}}%
\pgfpathlineto{\pgfqpoint{4.560135in}{0.533400in}}%
\pgfpathlineto{\pgfqpoint{4.560135in}{0.521603in}}%
\pgfpathclose%
\pgfusepath{stroke}%
\end{pgfscope}%
\end{pgfpicture}%
\makeatother%
\endgroup%

    \caption{Droplet trajectories as a function of $\mathbb{E}\mbox{u}$.\label{fig:series_s_eu}}
\end{figure}
\begin{figure}[htb]
    \centering
    \resizebox{14cm}{!}{%% Creator: Matplotlib, PGF backend
%%
%% To include the figure in your LaTeX document, write
%%   \input{<filename>.pgf}
%%
%% Make sure the required packages are loaded in your preamble
%%   \usepackage{pgf}
%%
%% Figures using additional raster images can only be included by \input if
%% they are in the same directory as the main LaTeX file. For loading figures
%% from other directories you can use the `import` package
%%   \usepackage{import}
%% and then include the figures with
%%   \import{<path to file>}{<filename>.pgf}
%%
%% Matplotlib used the following preamble
%%   \usepackage{fontspec}
%%   \setmainfont{DejaVu Serif}
%%   \setsansfont{DejaVu Sans}
%%   \setmonofont{DejaVu Sans Mono}
%%
\begingroup%
\makeatletter%
\begin{pgfpicture}%
\pgfpathrectangle{\pgfpointorigin}{\pgfqpoint{12.806532in}{18.099281in}}%
\pgfusepath{use as bounding box, clip}%
\begin{pgfscope}%
\pgfsetbuttcap%
\pgfsetmiterjoin%
\definecolor{currentfill}{rgb}{1.000000,1.000000,1.000000}%
\pgfsetfillcolor{currentfill}%
\pgfsetlinewidth{0.000000pt}%
\definecolor{currentstroke}{rgb}{1.000000,1.000000,1.000000}%
\pgfsetstrokecolor{currentstroke}%
\pgfsetdash{}{0pt}%
\pgfpathmoveto{\pgfqpoint{0.000000in}{0.000000in}}%
\pgfpathlineto{\pgfqpoint{12.806532in}{0.000000in}}%
\pgfpathlineto{\pgfqpoint{12.806532in}{18.099281in}}%
\pgfpathlineto{\pgfqpoint{0.000000in}{18.099281in}}%
\pgfpathclose%
\pgfusepath{fill}%
\end{pgfscope}%
\begin{pgfscope}%
\pgfsetbuttcap%
\pgfsetmiterjoin%
\definecolor{currentfill}{rgb}{1.000000,1.000000,1.000000}%
\pgfsetfillcolor{currentfill}%
\pgfsetlinewidth{0.000000pt}%
\definecolor{currentstroke}{rgb}{0.000000,0.000000,0.000000}%
\pgfsetstrokecolor{currentstroke}%
\pgfsetstrokeopacity{0.000000}%
\pgfsetdash{}{0pt}%
\pgfpathmoveto{\pgfqpoint{0.880000in}{6.967719in}}%
\pgfpathlineto{\pgfqpoint{2.777959in}{6.967719in}}%
\pgfpathlineto{\pgfqpoint{2.777959in}{8.340446in}}%
\pgfpathlineto{\pgfqpoint{0.880000in}{8.340446in}}%
\pgfpathclose%
\pgfusepath{fill}%
\end{pgfscope}%
\begin{pgfscope}%
\pgfsetbuttcap%
\pgfsetroundjoin%
\definecolor{currentfill}{rgb}{0.000000,0.000000,0.000000}%
\pgfsetfillcolor{currentfill}%
\pgfsetlinewidth{0.803000pt}%
\definecolor{currentstroke}{rgb}{0.000000,0.000000,0.000000}%
\pgfsetstrokecolor{currentstroke}%
\pgfsetdash{}{0pt}%
\pgfsys@defobject{currentmarker}{\pgfqpoint{0.000000in}{-0.048611in}}{\pgfqpoint{0.000000in}{0.000000in}}{%
\pgfpathmoveto{\pgfqpoint{0.000000in}{0.000000in}}%
\pgfpathlineto{\pgfqpoint{0.000000in}{-0.048611in}}%
\pgfusepath{stroke,fill}%
}%
\begin{pgfscope}%
\pgfsys@transformshift{0.880000in}{6.967719in}%
\pgfsys@useobject{currentmarker}{}%
\end{pgfscope}%
\end{pgfscope}%
\begin{pgfscope}%
\pgftext[x=0.880000in,y=6.870496in,,top]{\rmfamily\fontsize{12.000000}{14.400000}\selectfont \(\displaystyle 0.0\)}%
\end{pgfscope}%
\begin{pgfscope}%
\pgfsetbuttcap%
\pgfsetroundjoin%
\definecolor{currentfill}{rgb}{0.000000,0.000000,0.000000}%
\pgfsetfillcolor{currentfill}%
\pgfsetlinewidth{0.803000pt}%
\definecolor{currentstroke}{rgb}{0.000000,0.000000,0.000000}%
\pgfsetstrokecolor{currentstroke}%
\pgfsetdash{}{0pt}%
\pgfsys@defobject{currentmarker}{\pgfqpoint{0.000000in}{-0.048611in}}{\pgfqpoint{0.000000in}{0.000000in}}{%
\pgfpathmoveto{\pgfqpoint{0.000000in}{0.000000in}}%
\pgfpathlineto{\pgfqpoint{0.000000in}{-0.048611in}}%
\pgfusepath{stroke,fill}%
}%
\begin{pgfscope}%
\pgfsys@transformshift{1.828980in}{6.967719in}%
\pgfsys@useobject{currentmarker}{}%
\end{pgfscope}%
\end{pgfscope}%
\begin{pgfscope}%
\pgftext[x=1.828980in,y=6.870496in,,top]{\rmfamily\fontsize{12.000000}{14.400000}\selectfont \(\displaystyle 0.5\)}%
\end{pgfscope}%
\begin{pgfscope}%
\pgfsetbuttcap%
\pgfsetroundjoin%
\definecolor{currentfill}{rgb}{0.000000,0.000000,0.000000}%
\pgfsetfillcolor{currentfill}%
\pgfsetlinewidth{0.803000pt}%
\definecolor{currentstroke}{rgb}{0.000000,0.000000,0.000000}%
\pgfsetstrokecolor{currentstroke}%
\pgfsetdash{}{0pt}%
\pgfsys@defobject{currentmarker}{\pgfqpoint{0.000000in}{-0.048611in}}{\pgfqpoint{0.000000in}{0.000000in}}{%
\pgfpathmoveto{\pgfqpoint{0.000000in}{0.000000in}}%
\pgfpathlineto{\pgfqpoint{0.000000in}{-0.048611in}}%
\pgfusepath{stroke,fill}%
}%
\begin{pgfscope}%
\pgfsys@transformshift{2.777959in}{6.967719in}%
\pgfsys@useobject{currentmarker}{}%
\end{pgfscope}%
\end{pgfscope}%
\begin{pgfscope}%
\pgftext[x=2.777959in,y=6.870496in,,top]{\rmfamily\fontsize{12.000000}{14.400000}\selectfont \(\displaystyle 1.0\)}%
\end{pgfscope}%
\begin{pgfscope}%
\pgfsetbuttcap%
\pgfsetroundjoin%
\definecolor{currentfill}{rgb}{0.000000,0.000000,0.000000}%
\pgfsetfillcolor{currentfill}%
\pgfsetlinewidth{0.803000pt}%
\definecolor{currentstroke}{rgb}{0.000000,0.000000,0.000000}%
\pgfsetstrokecolor{currentstroke}%
\pgfsetdash{}{0pt}%
\pgfsys@defobject{currentmarker}{\pgfqpoint{-0.048611in}{0.000000in}}{\pgfqpoint{0.000000in}{0.000000in}}{%
\pgfpathmoveto{\pgfqpoint{0.000000in}{0.000000in}}%
\pgfpathlineto{\pgfqpoint{-0.048611in}{0.000000in}}%
\pgfusepath{stroke,fill}%
}%
\begin{pgfscope}%
\pgfsys@transformshift{0.880000in}{6.967719in}%
\pgfsys@useobject{currentmarker}{}%
\end{pgfscope}%
\end{pgfscope}%
\begin{pgfscope}%
\pgftext[x=0.492657in,y=6.904405in,left,base]{\rmfamily\fontsize{12.000000}{14.400000}\selectfont \(\displaystyle 0.00\)}%
\end{pgfscope}%
\begin{pgfscope}%
\pgfsetbuttcap%
\pgfsetroundjoin%
\definecolor{currentfill}{rgb}{0.000000,0.000000,0.000000}%
\pgfsetfillcolor{currentfill}%
\pgfsetlinewidth{0.803000pt}%
\definecolor{currentstroke}{rgb}{0.000000,0.000000,0.000000}%
\pgfsetstrokecolor{currentstroke}%
\pgfsetdash{}{0pt}%
\pgfsys@defobject{currentmarker}{\pgfqpoint{-0.048611in}{0.000000in}}{\pgfqpoint{0.000000in}{0.000000in}}{%
\pgfpathmoveto{\pgfqpoint{0.000000in}{0.000000in}}%
\pgfpathlineto{\pgfqpoint{-0.048611in}{0.000000in}}%
\pgfusepath{stroke,fill}%
}%
\begin{pgfscope}%
\pgfsys@transformshift{0.880000in}{7.310900in}%
\pgfsys@useobject{currentmarker}{}%
\end{pgfscope}%
\end{pgfscope}%
\begin{pgfscope}%
\pgftext[x=0.492657in,y=7.247587in,left,base]{\rmfamily\fontsize{12.000000}{14.400000}\selectfont \(\displaystyle 0.25\)}%
\end{pgfscope}%
\begin{pgfscope}%
\pgfsetbuttcap%
\pgfsetroundjoin%
\definecolor{currentfill}{rgb}{0.000000,0.000000,0.000000}%
\pgfsetfillcolor{currentfill}%
\pgfsetlinewidth{0.803000pt}%
\definecolor{currentstroke}{rgb}{0.000000,0.000000,0.000000}%
\pgfsetstrokecolor{currentstroke}%
\pgfsetdash{}{0pt}%
\pgfsys@defobject{currentmarker}{\pgfqpoint{-0.048611in}{0.000000in}}{\pgfqpoint{0.000000in}{0.000000in}}{%
\pgfpathmoveto{\pgfqpoint{0.000000in}{0.000000in}}%
\pgfpathlineto{\pgfqpoint{-0.048611in}{0.000000in}}%
\pgfusepath{stroke,fill}%
}%
\begin{pgfscope}%
\pgfsys@transformshift{0.880000in}{7.654082in}%
\pgfsys@useobject{currentmarker}{}%
\end{pgfscope}%
\end{pgfscope}%
\begin{pgfscope}%
\pgftext[x=0.492657in,y=7.590768in,left,base]{\rmfamily\fontsize{12.000000}{14.400000}\selectfont \(\displaystyle 0.50\)}%
\end{pgfscope}%
\begin{pgfscope}%
\pgfsetbuttcap%
\pgfsetroundjoin%
\definecolor{currentfill}{rgb}{0.000000,0.000000,0.000000}%
\pgfsetfillcolor{currentfill}%
\pgfsetlinewidth{0.803000pt}%
\definecolor{currentstroke}{rgb}{0.000000,0.000000,0.000000}%
\pgfsetstrokecolor{currentstroke}%
\pgfsetdash{}{0pt}%
\pgfsys@defobject{currentmarker}{\pgfqpoint{-0.048611in}{0.000000in}}{\pgfqpoint{0.000000in}{0.000000in}}{%
\pgfpathmoveto{\pgfqpoint{0.000000in}{0.000000in}}%
\pgfpathlineto{\pgfqpoint{-0.048611in}{0.000000in}}%
\pgfusepath{stroke,fill}%
}%
\begin{pgfscope}%
\pgfsys@transformshift{0.880000in}{7.997264in}%
\pgfsys@useobject{currentmarker}{}%
\end{pgfscope}%
\end{pgfscope}%
\begin{pgfscope}%
\pgftext[x=0.492657in,y=7.933950in,left,base]{\rmfamily\fontsize{12.000000}{14.400000}\selectfont \(\displaystyle 0.75\)}%
\end{pgfscope}%
\begin{pgfscope}%
\pgfsetbuttcap%
\pgfsetroundjoin%
\definecolor{currentfill}{rgb}{0.000000,0.000000,0.000000}%
\pgfsetfillcolor{currentfill}%
\pgfsetlinewidth{0.803000pt}%
\definecolor{currentstroke}{rgb}{0.000000,0.000000,0.000000}%
\pgfsetstrokecolor{currentstroke}%
\pgfsetdash{}{0pt}%
\pgfsys@defobject{currentmarker}{\pgfqpoint{-0.048611in}{0.000000in}}{\pgfqpoint{0.000000in}{0.000000in}}{%
\pgfpathmoveto{\pgfqpoint{0.000000in}{0.000000in}}%
\pgfpathlineto{\pgfqpoint{-0.048611in}{0.000000in}}%
\pgfusepath{stroke,fill}%
}%
\begin{pgfscope}%
\pgfsys@transformshift{0.880000in}{8.340446in}%
\pgfsys@useobject{currentmarker}{}%
\end{pgfscope}%
\end{pgfscope}%
\begin{pgfscope}%
\pgftext[x=0.492657in,y=8.277132in,left,base]{\rmfamily\fontsize{12.000000}{14.400000}\selectfont \(\displaystyle 1.00\)}%
\end{pgfscope}%
\begin{pgfscope}%
\pgfsetrectcap%
\pgfsetmiterjoin%
\pgfsetlinewidth{0.803000pt}%
\definecolor{currentstroke}{rgb}{0.501961,0.501961,0.501961}%
\pgfsetstrokecolor{currentstroke}%
\pgfsetdash{}{0pt}%
\pgfpathmoveto{\pgfqpoint{0.880000in}{6.967719in}}%
\pgfpathlineto{\pgfqpoint{0.880000in}{8.340446in}}%
\pgfusepath{stroke}%
\end{pgfscope}%
\begin{pgfscope}%
\pgfsetrectcap%
\pgfsetmiterjoin%
\pgfsetlinewidth{0.803000pt}%
\definecolor{currentstroke}{rgb}{0.501961,0.501961,0.501961}%
\pgfsetstrokecolor{currentstroke}%
\pgfsetdash{}{0pt}%
\pgfpathmoveto{\pgfqpoint{2.777959in}{6.967719in}}%
\pgfpathlineto{\pgfqpoint{2.777959in}{8.340446in}}%
\pgfusepath{stroke}%
\end{pgfscope}%
\begin{pgfscope}%
\pgfsetrectcap%
\pgfsetmiterjoin%
\pgfsetlinewidth{0.803000pt}%
\definecolor{currentstroke}{rgb}{0.501961,0.501961,0.501961}%
\pgfsetstrokecolor{currentstroke}%
\pgfsetdash{}{0pt}%
\pgfpathmoveto{\pgfqpoint{0.880000in}{6.967719in}}%
\pgfpathlineto{\pgfqpoint{2.777959in}{6.967719in}}%
\pgfusepath{stroke}%
\end{pgfscope}%
\begin{pgfscope}%
\pgfsetrectcap%
\pgfsetmiterjoin%
\pgfsetlinewidth{0.803000pt}%
\definecolor{currentstroke}{rgb}{0.501961,0.501961,0.501961}%
\pgfsetstrokecolor{currentstroke}%
\pgfsetdash{}{0pt}%
\pgfpathmoveto{\pgfqpoint{0.880000in}{8.340446in}}%
\pgfpathlineto{\pgfqpoint{2.777959in}{8.340446in}}%
\pgfusepath{stroke}%
\end{pgfscope}%
\begin{pgfscope}%
\pgfsetbuttcap%
\pgfsetmiterjoin%
\definecolor{currentfill}{rgb}{1.000000,1.000000,1.000000}%
\pgfsetfillcolor{currentfill}%
\pgfsetlinewidth{0.000000pt}%
\definecolor{currentstroke}{rgb}{0.000000,0.000000,0.000000}%
\pgfsetstrokecolor{currentstroke}%
\pgfsetstrokeopacity{0.000000}%
\pgfsetdash{}{0pt}%
\pgfpathmoveto{\pgfqpoint{3.347347in}{6.967719in}}%
\pgfpathlineto{\pgfqpoint{5.245306in}{6.967719in}}%
\pgfpathlineto{\pgfqpoint{5.245306in}{8.340446in}}%
\pgfpathlineto{\pgfqpoint{3.347347in}{8.340446in}}%
\pgfpathclose%
\pgfusepath{fill}%
\end{pgfscope}%
\begin{pgfscope}%
\pgfsetbuttcap%
\pgfsetroundjoin%
\definecolor{currentfill}{rgb}{0.000000,0.000000,0.000000}%
\pgfsetfillcolor{currentfill}%
\pgfsetlinewidth{0.803000pt}%
\definecolor{currentstroke}{rgb}{0.000000,0.000000,0.000000}%
\pgfsetstrokecolor{currentstroke}%
\pgfsetdash{}{0pt}%
\pgfsys@defobject{currentmarker}{\pgfqpoint{0.000000in}{-0.048611in}}{\pgfqpoint{0.000000in}{0.000000in}}{%
\pgfpathmoveto{\pgfqpoint{0.000000in}{0.000000in}}%
\pgfpathlineto{\pgfqpoint{0.000000in}{-0.048611in}}%
\pgfusepath{stroke,fill}%
}%
\begin{pgfscope}%
\pgfsys@transformshift{3.347347in}{6.967719in}%
\pgfsys@useobject{currentmarker}{}%
\end{pgfscope}%
\end{pgfscope}%
\begin{pgfscope}%
\pgftext[x=3.347347in,y=6.870496in,,top]{\rmfamily\fontsize{12.000000}{14.400000}\selectfont \(\displaystyle 0.0\)}%
\end{pgfscope}%
\begin{pgfscope}%
\pgfsetbuttcap%
\pgfsetroundjoin%
\definecolor{currentfill}{rgb}{0.000000,0.000000,0.000000}%
\pgfsetfillcolor{currentfill}%
\pgfsetlinewidth{0.803000pt}%
\definecolor{currentstroke}{rgb}{0.000000,0.000000,0.000000}%
\pgfsetstrokecolor{currentstroke}%
\pgfsetdash{}{0pt}%
\pgfsys@defobject{currentmarker}{\pgfqpoint{0.000000in}{-0.048611in}}{\pgfqpoint{0.000000in}{0.000000in}}{%
\pgfpathmoveto{\pgfqpoint{0.000000in}{0.000000in}}%
\pgfpathlineto{\pgfqpoint{0.000000in}{-0.048611in}}%
\pgfusepath{stroke,fill}%
}%
\begin{pgfscope}%
\pgfsys@transformshift{4.296327in}{6.967719in}%
\pgfsys@useobject{currentmarker}{}%
\end{pgfscope}%
\end{pgfscope}%
\begin{pgfscope}%
\pgftext[x=4.296327in,y=6.870496in,,top]{\rmfamily\fontsize{12.000000}{14.400000}\selectfont \(\displaystyle 0.5\)}%
\end{pgfscope}%
\begin{pgfscope}%
\pgfsetbuttcap%
\pgfsetroundjoin%
\definecolor{currentfill}{rgb}{0.000000,0.000000,0.000000}%
\pgfsetfillcolor{currentfill}%
\pgfsetlinewidth{0.803000pt}%
\definecolor{currentstroke}{rgb}{0.000000,0.000000,0.000000}%
\pgfsetstrokecolor{currentstroke}%
\pgfsetdash{}{0pt}%
\pgfsys@defobject{currentmarker}{\pgfqpoint{0.000000in}{-0.048611in}}{\pgfqpoint{0.000000in}{0.000000in}}{%
\pgfpathmoveto{\pgfqpoint{0.000000in}{0.000000in}}%
\pgfpathlineto{\pgfqpoint{0.000000in}{-0.048611in}}%
\pgfusepath{stroke,fill}%
}%
\begin{pgfscope}%
\pgfsys@transformshift{5.245306in}{6.967719in}%
\pgfsys@useobject{currentmarker}{}%
\end{pgfscope}%
\end{pgfscope}%
\begin{pgfscope}%
\pgftext[x=5.245306in,y=6.870496in,,top]{\rmfamily\fontsize{12.000000}{14.400000}\selectfont \(\displaystyle 1.0\)}%
\end{pgfscope}%
\begin{pgfscope}%
\pgfsetbuttcap%
\pgfsetroundjoin%
\definecolor{currentfill}{rgb}{0.000000,0.000000,0.000000}%
\pgfsetfillcolor{currentfill}%
\pgfsetlinewidth{0.803000pt}%
\definecolor{currentstroke}{rgb}{0.000000,0.000000,0.000000}%
\pgfsetstrokecolor{currentstroke}%
\pgfsetdash{}{0pt}%
\pgfsys@defobject{currentmarker}{\pgfqpoint{-0.048611in}{0.000000in}}{\pgfqpoint{0.000000in}{0.000000in}}{%
\pgfpathmoveto{\pgfqpoint{0.000000in}{0.000000in}}%
\pgfpathlineto{\pgfqpoint{-0.048611in}{0.000000in}}%
\pgfusepath{stroke,fill}%
}%
\begin{pgfscope}%
\pgfsys@transformshift{3.347347in}{6.967719in}%
\pgfsys@useobject{currentmarker}{}%
\end{pgfscope}%
\end{pgfscope}%
\begin{pgfscope}%
\pgftext[x=2.960004in,y=6.904405in,left,base]{\rmfamily\fontsize{12.000000}{14.400000}\selectfont \(\displaystyle 0.00\)}%
\end{pgfscope}%
\begin{pgfscope}%
\pgfsetbuttcap%
\pgfsetroundjoin%
\definecolor{currentfill}{rgb}{0.000000,0.000000,0.000000}%
\pgfsetfillcolor{currentfill}%
\pgfsetlinewidth{0.803000pt}%
\definecolor{currentstroke}{rgb}{0.000000,0.000000,0.000000}%
\pgfsetstrokecolor{currentstroke}%
\pgfsetdash{}{0pt}%
\pgfsys@defobject{currentmarker}{\pgfqpoint{-0.048611in}{0.000000in}}{\pgfqpoint{0.000000in}{0.000000in}}{%
\pgfpathmoveto{\pgfqpoint{0.000000in}{0.000000in}}%
\pgfpathlineto{\pgfqpoint{-0.048611in}{0.000000in}}%
\pgfusepath{stroke,fill}%
}%
\begin{pgfscope}%
\pgfsys@transformshift{3.347347in}{7.310900in}%
\pgfsys@useobject{currentmarker}{}%
\end{pgfscope}%
\end{pgfscope}%
\begin{pgfscope}%
\pgftext[x=2.960004in,y=7.247587in,left,base]{\rmfamily\fontsize{12.000000}{14.400000}\selectfont \(\displaystyle 0.25\)}%
\end{pgfscope}%
\begin{pgfscope}%
\pgfsetbuttcap%
\pgfsetroundjoin%
\definecolor{currentfill}{rgb}{0.000000,0.000000,0.000000}%
\pgfsetfillcolor{currentfill}%
\pgfsetlinewidth{0.803000pt}%
\definecolor{currentstroke}{rgb}{0.000000,0.000000,0.000000}%
\pgfsetstrokecolor{currentstroke}%
\pgfsetdash{}{0pt}%
\pgfsys@defobject{currentmarker}{\pgfqpoint{-0.048611in}{0.000000in}}{\pgfqpoint{0.000000in}{0.000000in}}{%
\pgfpathmoveto{\pgfqpoint{0.000000in}{0.000000in}}%
\pgfpathlineto{\pgfqpoint{-0.048611in}{0.000000in}}%
\pgfusepath{stroke,fill}%
}%
\begin{pgfscope}%
\pgfsys@transformshift{3.347347in}{7.654082in}%
\pgfsys@useobject{currentmarker}{}%
\end{pgfscope}%
\end{pgfscope}%
\begin{pgfscope}%
\pgftext[x=2.960004in,y=7.590768in,left,base]{\rmfamily\fontsize{12.000000}{14.400000}\selectfont \(\displaystyle 0.50\)}%
\end{pgfscope}%
\begin{pgfscope}%
\pgfsetbuttcap%
\pgfsetroundjoin%
\definecolor{currentfill}{rgb}{0.000000,0.000000,0.000000}%
\pgfsetfillcolor{currentfill}%
\pgfsetlinewidth{0.803000pt}%
\definecolor{currentstroke}{rgb}{0.000000,0.000000,0.000000}%
\pgfsetstrokecolor{currentstroke}%
\pgfsetdash{}{0pt}%
\pgfsys@defobject{currentmarker}{\pgfqpoint{-0.048611in}{0.000000in}}{\pgfqpoint{0.000000in}{0.000000in}}{%
\pgfpathmoveto{\pgfqpoint{0.000000in}{0.000000in}}%
\pgfpathlineto{\pgfqpoint{-0.048611in}{0.000000in}}%
\pgfusepath{stroke,fill}%
}%
\begin{pgfscope}%
\pgfsys@transformshift{3.347347in}{7.997264in}%
\pgfsys@useobject{currentmarker}{}%
\end{pgfscope}%
\end{pgfscope}%
\begin{pgfscope}%
\pgftext[x=2.960004in,y=7.933950in,left,base]{\rmfamily\fontsize{12.000000}{14.400000}\selectfont \(\displaystyle 0.75\)}%
\end{pgfscope}%
\begin{pgfscope}%
\pgfsetbuttcap%
\pgfsetroundjoin%
\definecolor{currentfill}{rgb}{0.000000,0.000000,0.000000}%
\pgfsetfillcolor{currentfill}%
\pgfsetlinewidth{0.803000pt}%
\definecolor{currentstroke}{rgb}{0.000000,0.000000,0.000000}%
\pgfsetstrokecolor{currentstroke}%
\pgfsetdash{}{0pt}%
\pgfsys@defobject{currentmarker}{\pgfqpoint{-0.048611in}{0.000000in}}{\pgfqpoint{0.000000in}{0.000000in}}{%
\pgfpathmoveto{\pgfqpoint{0.000000in}{0.000000in}}%
\pgfpathlineto{\pgfqpoint{-0.048611in}{0.000000in}}%
\pgfusepath{stroke,fill}%
}%
\begin{pgfscope}%
\pgfsys@transformshift{3.347347in}{8.340446in}%
\pgfsys@useobject{currentmarker}{}%
\end{pgfscope}%
\end{pgfscope}%
\begin{pgfscope}%
\pgftext[x=2.960004in,y=8.277132in,left,base]{\rmfamily\fontsize{12.000000}{14.400000}\selectfont \(\displaystyle 1.00\)}%
\end{pgfscope}%
\begin{pgfscope}%
\pgfsetrectcap%
\pgfsetmiterjoin%
\pgfsetlinewidth{0.803000pt}%
\definecolor{currentstroke}{rgb}{0.501961,0.501961,0.501961}%
\pgfsetstrokecolor{currentstroke}%
\pgfsetdash{}{0pt}%
\pgfpathmoveto{\pgfqpoint{3.347347in}{6.967719in}}%
\pgfpathlineto{\pgfqpoint{3.347347in}{8.340446in}}%
\pgfusepath{stroke}%
\end{pgfscope}%
\begin{pgfscope}%
\pgfsetrectcap%
\pgfsetmiterjoin%
\pgfsetlinewidth{0.803000pt}%
\definecolor{currentstroke}{rgb}{0.501961,0.501961,0.501961}%
\pgfsetstrokecolor{currentstroke}%
\pgfsetdash{}{0pt}%
\pgfpathmoveto{\pgfqpoint{5.245306in}{6.967719in}}%
\pgfpathlineto{\pgfqpoint{5.245306in}{8.340446in}}%
\pgfusepath{stroke}%
\end{pgfscope}%
\begin{pgfscope}%
\pgfsetrectcap%
\pgfsetmiterjoin%
\pgfsetlinewidth{0.803000pt}%
\definecolor{currentstroke}{rgb}{0.501961,0.501961,0.501961}%
\pgfsetstrokecolor{currentstroke}%
\pgfsetdash{}{0pt}%
\pgfpathmoveto{\pgfqpoint{3.347347in}{6.967719in}}%
\pgfpathlineto{\pgfqpoint{5.245306in}{6.967719in}}%
\pgfusepath{stroke}%
\end{pgfscope}%
\begin{pgfscope}%
\pgfsetrectcap%
\pgfsetmiterjoin%
\pgfsetlinewidth{0.803000pt}%
\definecolor{currentstroke}{rgb}{0.501961,0.501961,0.501961}%
\pgfsetstrokecolor{currentstroke}%
\pgfsetdash{}{0pt}%
\pgfpathmoveto{\pgfqpoint{3.347347in}{8.340446in}}%
\pgfpathlineto{\pgfqpoint{5.245306in}{8.340446in}}%
\pgfusepath{stroke}%
\end{pgfscope}%
\begin{pgfscope}%
\pgfsetbuttcap%
\pgfsetmiterjoin%
\definecolor{currentfill}{rgb}{1.000000,1.000000,1.000000}%
\pgfsetfillcolor{currentfill}%
\pgfsetlinewidth{0.000000pt}%
\definecolor{currentstroke}{rgb}{0.000000,0.000000,0.000000}%
\pgfsetstrokecolor{currentstroke}%
\pgfsetstrokeopacity{0.000000}%
\pgfsetdash{}{0pt}%
\pgfpathmoveto{\pgfqpoint{5.814694in}{6.967719in}}%
\pgfpathlineto{\pgfqpoint{7.712653in}{6.967719in}}%
\pgfpathlineto{\pgfqpoint{7.712653in}{8.340446in}}%
\pgfpathlineto{\pgfqpoint{5.814694in}{8.340446in}}%
\pgfpathclose%
\pgfusepath{fill}%
\end{pgfscope}%
\begin{pgfscope}%
\pgfsetbuttcap%
\pgfsetroundjoin%
\definecolor{currentfill}{rgb}{0.000000,0.000000,0.000000}%
\pgfsetfillcolor{currentfill}%
\pgfsetlinewidth{0.803000pt}%
\definecolor{currentstroke}{rgb}{0.000000,0.000000,0.000000}%
\pgfsetstrokecolor{currentstroke}%
\pgfsetdash{}{0pt}%
\pgfsys@defobject{currentmarker}{\pgfqpoint{0.000000in}{-0.048611in}}{\pgfqpoint{0.000000in}{0.000000in}}{%
\pgfpathmoveto{\pgfqpoint{0.000000in}{0.000000in}}%
\pgfpathlineto{\pgfqpoint{0.000000in}{-0.048611in}}%
\pgfusepath{stroke,fill}%
}%
\begin{pgfscope}%
\pgfsys@transformshift{5.814694in}{6.967719in}%
\pgfsys@useobject{currentmarker}{}%
\end{pgfscope}%
\end{pgfscope}%
\begin{pgfscope}%
\pgftext[x=5.814694in,y=6.870496in,,top]{\rmfamily\fontsize{12.000000}{14.400000}\selectfont \(\displaystyle 0.0\)}%
\end{pgfscope}%
\begin{pgfscope}%
\pgfsetbuttcap%
\pgfsetroundjoin%
\definecolor{currentfill}{rgb}{0.000000,0.000000,0.000000}%
\pgfsetfillcolor{currentfill}%
\pgfsetlinewidth{0.803000pt}%
\definecolor{currentstroke}{rgb}{0.000000,0.000000,0.000000}%
\pgfsetstrokecolor{currentstroke}%
\pgfsetdash{}{0pt}%
\pgfsys@defobject{currentmarker}{\pgfqpoint{0.000000in}{-0.048611in}}{\pgfqpoint{0.000000in}{0.000000in}}{%
\pgfpathmoveto{\pgfqpoint{0.000000in}{0.000000in}}%
\pgfpathlineto{\pgfqpoint{0.000000in}{-0.048611in}}%
\pgfusepath{stroke,fill}%
}%
\begin{pgfscope}%
\pgfsys@transformshift{6.763673in}{6.967719in}%
\pgfsys@useobject{currentmarker}{}%
\end{pgfscope}%
\end{pgfscope}%
\begin{pgfscope}%
\pgftext[x=6.763673in,y=6.870496in,,top]{\rmfamily\fontsize{12.000000}{14.400000}\selectfont \(\displaystyle 0.5\)}%
\end{pgfscope}%
\begin{pgfscope}%
\pgfsetbuttcap%
\pgfsetroundjoin%
\definecolor{currentfill}{rgb}{0.000000,0.000000,0.000000}%
\pgfsetfillcolor{currentfill}%
\pgfsetlinewidth{0.803000pt}%
\definecolor{currentstroke}{rgb}{0.000000,0.000000,0.000000}%
\pgfsetstrokecolor{currentstroke}%
\pgfsetdash{}{0pt}%
\pgfsys@defobject{currentmarker}{\pgfqpoint{0.000000in}{-0.048611in}}{\pgfqpoint{0.000000in}{0.000000in}}{%
\pgfpathmoveto{\pgfqpoint{0.000000in}{0.000000in}}%
\pgfpathlineto{\pgfqpoint{0.000000in}{-0.048611in}}%
\pgfusepath{stroke,fill}%
}%
\begin{pgfscope}%
\pgfsys@transformshift{7.712653in}{6.967719in}%
\pgfsys@useobject{currentmarker}{}%
\end{pgfscope}%
\end{pgfscope}%
\begin{pgfscope}%
\pgftext[x=7.712653in,y=6.870496in,,top]{\rmfamily\fontsize{12.000000}{14.400000}\selectfont \(\displaystyle 1.0\)}%
\end{pgfscope}%
\begin{pgfscope}%
\pgfsetbuttcap%
\pgfsetroundjoin%
\definecolor{currentfill}{rgb}{0.000000,0.000000,0.000000}%
\pgfsetfillcolor{currentfill}%
\pgfsetlinewidth{0.803000pt}%
\definecolor{currentstroke}{rgb}{0.000000,0.000000,0.000000}%
\pgfsetstrokecolor{currentstroke}%
\pgfsetdash{}{0pt}%
\pgfsys@defobject{currentmarker}{\pgfqpoint{-0.048611in}{0.000000in}}{\pgfqpoint{0.000000in}{0.000000in}}{%
\pgfpathmoveto{\pgfqpoint{0.000000in}{0.000000in}}%
\pgfpathlineto{\pgfqpoint{-0.048611in}{0.000000in}}%
\pgfusepath{stroke,fill}%
}%
\begin{pgfscope}%
\pgfsys@transformshift{5.814694in}{6.967719in}%
\pgfsys@useobject{currentmarker}{}%
\end{pgfscope}%
\end{pgfscope}%
\begin{pgfscope}%
\pgftext[x=5.427351in,y=6.904405in,left,base]{\rmfamily\fontsize{12.000000}{14.400000}\selectfont \(\displaystyle 0.00\)}%
\end{pgfscope}%
\begin{pgfscope}%
\pgfsetbuttcap%
\pgfsetroundjoin%
\definecolor{currentfill}{rgb}{0.000000,0.000000,0.000000}%
\pgfsetfillcolor{currentfill}%
\pgfsetlinewidth{0.803000pt}%
\definecolor{currentstroke}{rgb}{0.000000,0.000000,0.000000}%
\pgfsetstrokecolor{currentstroke}%
\pgfsetdash{}{0pt}%
\pgfsys@defobject{currentmarker}{\pgfqpoint{-0.048611in}{0.000000in}}{\pgfqpoint{0.000000in}{0.000000in}}{%
\pgfpathmoveto{\pgfqpoint{0.000000in}{0.000000in}}%
\pgfpathlineto{\pgfqpoint{-0.048611in}{0.000000in}}%
\pgfusepath{stroke,fill}%
}%
\begin{pgfscope}%
\pgfsys@transformshift{5.814694in}{7.310900in}%
\pgfsys@useobject{currentmarker}{}%
\end{pgfscope}%
\end{pgfscope}%
\begin{pgfscope}%
\pgftext[x=5.427351in,y=7.247587in,left,base]{\rmfamily\fontsize{12.000000}{14.400000}\selectfont \(\displaystyle 0.25\)}%
\end{pgfscope}%
\begin{pgfscope}%
\pgfsetbuttcap%
\pgfsetroundjoin%
\definecolor{currentfill}{rgb}{0.000000,0.000000,0.000000}%
\pgfsetfillcolor{currentfill}%
\pgfsetlinewidth{0.803000pt}%
\definecolor{currentstroke}{rgb}{0.000000,0.000000,0.000000}%
\pgfsetstrokecolor{currentstroke}%
\pgfsetdash{}{0pt}%
\pgfsys@defobject{currentmarker}{\pgfqpoint{-0.048611in}{0.000000in}}{\pgfqpoint{0.000000in}{0.000000in}}{%
\pgfpathmoveto{\pgfqpoint{0.000000in}{0.000000in}}%
\pgfpathlineto{\pgfqpoint{-0.048611in}{0.000000in}}%
\pgfusepath{stroke,fill}%
}%
\begin{pgfscope}%
\pgfsys@transformshift{5.814694in}{7.654082in}%
\pgfsys@useobject{currentmarker}{}%
\end{pgfscope}%
\end{pgfscope}%
\begin{pgfscope}%
\pgftext[x=5.427351in,y=7.590768in,left,base]{\rmfamily\fontsize{12.000000}{14.400000}\selectfont \(\displaystyle 0.50\)}%
\end{pgfscope}%
\begin{pgfscope}%
\pgfsetbuttcap%
\pgfsetroundjoin%
\definecolor{currentfill}{rgb}{0.000000,0.000000,0.000000}%
\pgfsetfillcolor{currentfill}%
\pgfsetlinewidth{0.803000pt}%
\definecolor{currentstroke}{rgb}{0.000000,0.000000,0.000000}%
\pgfsetstrokecolor{currentstroke}%
\pgfsetdash{}{0pt}%
\pgfsys@defobject{currentmarker}{\pgfqpoint{-0.048611in}{0.000000in}}{\pgfqpoint{0.000000in}{0.000000in}}{%
\pgfpathmoveto{\pgfqpoint{0.000000in}{0.000000in}}%
\pgfpathlineto{\pgfqpoint{-0.048611in}{0.000000in}}%
\pgfusepath{stroke,fill}%
}%
\begin{pgfscope}%
\pgfsys@transformshift{5.814694in}{7.997264in}%
\pgfsys@useobject{currentmarker}{}%
\end{pgfscope}%
\end{pgfscope}%
\begin{pgfscope}%
\pgftext[x=5.427351in,y=7.933950in,left,base]{\rmfamily\fontsize{12.000000}{14.400000}\selectfont \(\displaystyle 0.75\)}%
\end{pgfscope}%
\begin{pgfscope}%
\pgfsetbuttcap%
\pgfsetroundjoin%
\definecolor{currentfill}{rgb}{0.000000,0.000000,0.000000}%
\pgfsetfillcolor{currentfill}%
\pgfsetlinewidth{0.803000pt}%
\definecolor{currentstroke}{rgb}{0.000000,0.000000,0.000000}%
\pgfsetstrokecolor{currentstroke}%
\pgfsetdash{}{0pt}%
\pgfsys@defobject{currentmarker}{\pgfqpoint{-0.048611in}{0.000000in}}{\pgfqpoint{0.000000in}{0.000000in}}{%
\pgfpathmoveto{\pgfqpoint{0.000000in}{0.000000in}}%
\pgfpathlineto{\pgfqpoint{-0.048611in}{0.000000in}}%
\pgfusepath{stroke,fill}%
}%
\begin{pgfscope}%
\pgfsys@transformshift{5.814694in}{8.340446in}%
\pgfsys@useobject{currentmarker}{}%
\end{pgfscope}%
\end{pgfscope}%
\begin{pgfscope}%
\pgftext[x=5.427351in,y=8.277132in,left,base]{\rmfamily\fontsize{12.000000}{14.400000}\selectfont \(\displaystyle 1.00\)}%
\end{pgfscope}%
\begin{pgfscope}%
\pgfsetrectcap%
\pgfsetmiterjoin%
\pgfsetlinewidth{0.803000pt}%
\definecolor{currentstroke}{rgb}{0.501961,0.501961,0.501961}%
\pgfsetstrokecolor{currentstroke}%
\pgfsetdash{}{0pt}%
\pgfpathmoveto{\pgfqpoint{5.814694in}{6.967719in}}%
\pgfpathlineto{\pgfqpoint{5.814694in}{8.340446in}}%
\pgfusepath{stroke}%
\end{pgfscope}%
\begin{pgfscope}%
\pgfsetrectcap%
\pgfsetmiterjoin%
\pgfsetlinewidth{0.803000pt}%
\definecolor{currentstroke}{rgb}{0.501961,0.501961,0.501961}%
\pgfsetstrokecolor{currentstroke}%
\pgfsetdash{}{0pt}%
\pgfpathmoveto{\pgfqpoint{7.712653in}{6.967719in}}%
\pgfpathlineto{\pgfqpoint{7.712653in}{8.340446in}}%
\pgfusepath{stroke}%
\end{pgfscope}%
\begin{pgfscope}%
\pgfsetrectcap%
\pgfsetmiterjoin%
\pgfsetlinewidth{0.803000pt}%
\definecolor{currentstroke}{rgb}{0.501961,0.501961,0.501961}%
\pgfsetstrokecolor{currentstroke}%
\pgfsetdash{}{0pt}%
\pgfpathmoveto{\pgfqpoint{5.814694in}{6.967719in}}%
\pgfpathlineto{\pgfqpoint{7.712653in}{6.967719in}}%
\pgfusepath{stroke}%
\end{pgfscope}%
\begin{pgfscope}%
\pgfsetrectcap%
\pgfsetmiterjoin%
\pgfsetlinewidth{0.803000pt}%
\definecolor{currentstroke}{rgb}{0.501961,0.501961,0.501961}%
\pgfsetstrokecolor{currentstroke}%
\pgfsetdash{}{0pt}%
\pgfpathmoveto{\pgfqpoint{5.814694in}{8.340446in}}%
\pgfpathlineto{\pgfqpoint{7.712653in}{8.340446in}}%
\pgfusepath{stroke}%
\end{pgfscope}%
\begin{pgfscope}%
\pgfsetbuttcap%
\pgfsetmiterjoin%
\definecolor{currentfill}{rgb}{1.000000,1.000000,1.000000}%
\pgfsetfillcolor{currentfill}%
\pgfsetlinewidth{0.000000pt}%
\definecolor{currentstroke}{rgb}{0.000000,0.000000,0.000000}%
\pgfsetstrokecolor{currentstroke}%
\pgfsetstrokeopacity{0.000000}%
\pgfsetdash{}{0pt}%
\pgfpathmoveto{\pgfqpoint{8.282041in}{6.967719in}}%
\pgfpathlineto{\pgfqpoint{10.180000in}{6.967719in}}%
\pgfpathlineto{\pgfqpoint{10.180000in}{8.340446in}}%
\pgfpathlineto{\pgfqpoint{8.282041in}{8.340446in}}%
\pgfpathclose%
\pgfusepath{fill}%
\end{pgfscope}%
\begin{pgfscope}%
\pgfsetbuttcap%
\pgfsetroundjoin%
\definecolor{currentfill}{rgb}{0.000000,0.000000,0.000000}%
\pgfsetfillcolor{currentfill}%
\pgfsetlinewidth{0.803000pt}%
\definecolor{currentstroke}{rgb}{0.000000,0.000000,0.000000}%
\pgfsetstrokecolor{currentstroke}%
\pgfsetdash{}{0pt}%
\pgfsys@defobject{currentmarker}{\pgfqpoint{0.000000in}{-0.048611in}}{\pgfqpoint{0.000000in}{0.000000in}}{%
\pgfpathmoveto{\pgfqpoint{0.000000in}{0.000000in}}%
\pgfpathlineto{\pgfqpoint{0.000000in}{-0.048611in}}%
\pgfusepath{stroke,fill}%
}%
\begin{pgfscope}%
\pgfsys@transformshift{8.282041in}{6.967719in}%
\pgfsys@useobject{currentmarker}{}%
\end{pgfscope}%
\end{pgfscope}%
\begin{pgfscope}%
\pgftext[x=8.282041in,y=6.870496in,,top]{\rmfamily\fontsize{12.000000}{14.400000}\selectfont \(\displaystyle 0.0\)}%
\end{pgfscope}%
\begin{pgfscope}%
\pgfsetbuttcap%
\pgfsetroundjoin%
\definecolor{currentfill}{rgb}{0.000000,0.000000,0.000000}%
\pgfsetfillcolor{currentfill}%
\pgfsetlinewidth{0.803000pt}%
\definecolor{currentstroke}{rgb}{0.000000,0.000000,0.000000}%
\pgfsetstrokecolor{currentstroke}%
\pgfsetdash{}{0pt}%
\pgfsys@defobject{currentmarker}{\pgfqpoint{0.000000in}{-0.048611in}}{\pgfqpoint{0.000000in}{0.000000in}}{%
\pgfpathmoveto{\pgfqpoint{0.000000in}{0.000000in}}%
\pgfpathlineto{\pgfqpoint{0.000000in}{-0.048611in}}%
\pgfusepath{stroke,fill}%
}%
\begin{pgfscope}%
\pgfsys@transformshift{9.231020in}{6.967719in}%
\pgfsys@useobject{currentmarker}{}%
\end{pgfscope}%
\end{pgfscope}%
\begin{pgfscope}%
\pgftext[x=9.231020in,y=6.870496in,,top]{\rmfamily\fontsize{12.000000}{14.400000}\selectfont \(\displaystyle 0.5\)}%
\end{pgfscope}%
\begin{pgfscope}%
\pgfsetbuttcap%
\pgfsetroundjoin%
\definecolor{currentfill}{rgb}{0.000000,0.000000,0.000000}%
\pgfsetfillcolor{currentfill}%
\pgfsetlinewidth{0.803000pt}%
\definecolor{currentstroke}{rgb}{0.000000,0.000000,0.000000}%
\pgfsetstrokecolor{currentstroke}%
\pgfsetdash{}{0pt}%
\pgfsys@defobject{currentmarker}{\pgfqpoint{0.000000in}{-0.048611in}}{\pgfqpoint{0.000000in}{0.000000in}}{%
\pgfpathmoveto{\pgfqpoint{0.000000in}{0.000000in}}%
\pgfpathlineto{\pgfqpoint{0.000000in}{-0.048611in}}%
\pgfusepath{stroke,fill}%
}%
\begin{pgfscope}%
\pgfsys@transformshift{10.180000in}{6.967719in}%
\pgfsys@useobject{currentmarker}{}%
\end{pgfscope}%
\end{pgfscope}%
\begin{pgfscope}%
\pgftext[x=10.180000in,y=6.870496in,,top]{\rmfamily\fontsize{12.000000}{14.400000}\selectfont \(\displaystyle 1.0\)}%
\end{pgfscope}%
\begin{pgfscope}%
\pgfsetbuttcap%
\pgfsetroundjoin%
\definecolor{currentfill}{rgb}{0.000000,0.000000,0.000000}%
\pgfsetfillcolor{currentfill}%
\pgfsetlinewidth{0.803000pt}%
\definecolor{currentstroke}{rgb}{0.000000,0.000000,0.000000}%
\pgfsetstrokecolor{currentstroke}%
\pgfsetdash{}{0pt}%
\pgfsys@defobject{currentmarker}{\pgfqpoint{-0.048611in}{0.000000in}}{\pgfqpoint{0.000000in}{0.000000in}}{%
\pgfpathmoveto{\pgfqpoint{0.000000in}{0.000000in}}%
\pgfpathlineto{\pgfqpoint{-0.048611in}{0.000000in}}%
\pgfusepath{stroke,fill}%
}%
\begin{pgfscope}%
\pgfsys@transformshift{8.282041in}{6.967719in}%
\pgfsys@useobject{currentmarker}{}%
\end{pgfscope}%
\end{pgfscope}%
\begin{pgfscope}%
\pgftext[x=7.894698in,y=6.904405in,left,base]{\rmfamily\fontsize{12.000000}{14.400000}\selectfont \(\displaystyle 0.00\)}%
\end{pgfscope}%
\begin{pgfscope}%
\pgfsetbuttcap%
\pgfsetroundjoin%
\definecolor{currentfill}{rgb}{0.000000,0.000000,0.000000}%
\pgfsetfillcolor{currentfill}%
\pgfsetlinewidth{0.803000pt}%
\definecolor{currentstroke}{rgb}{0.000000,0.000000,0.000000}%
\pgfsetstrokecolor{currentstroke}%
\pgfsetdash{}{0pt}%
\pgfsys@defobject{currentmarker}{\pgfqpoint{-0.048611in}{0.000000in}}{\pgfqpoint{0.000000in}{0.000000in}}{%
\pgfpathmoveto{\pgfqpoint{0.000000in}{0.000000in}}%
\pgfpathlineto{\pgfqpoint{-0.048611in}{0.000000in}}%
\pgfusepath{stroke,fill}%
}%
\begin{pgfscope}%
\pgfsys@transformshift{8.282041in}{7.310900in}%
\pgfsys@useobject{currentmarker}{}%
\end{pgfscope}%
\end{pgfscope}%
\begin{pgfscope}%
\pgftext[x=7.894698in,y=7.247587in,left,base]{\rmfamily\fontsize{12.000000}{14.400000}\selectfont \(\displaystyle 0.25\)}%
\end{pgfscope}%
\begin{pgfscope}%
\pgfsetbuttcap%
\pgfsetroundjoin%
\definecolor{currentfill}{rgb}{0.000000,0.000000,0.000000}%
\pgfsetfillcolor{currentfill}%
\pgfsetlinewidth{0.803000pt}%
\definecolor{currentstroke}{rgb}{0.000000,0.000000,0.000000}%
\pgfsetstrokecolor{currentstroke}%
\pgfsetdash{}{0pt}%
\pgfsys@defobject{currentmarker}{\pgfqpoint{-0.048611in}{0.000000in}}{\pgfqpoint{0.000000in}{0.000000in}}{%
\pgfpathmoveto{\pgfqpoint{0.000000in}{0.000000in}}%
\pgfpathlineto{\pgfqpoint{-0.048611in}{0.000000in}}%
\pgfusepath{stroke,fill}%
}%
\begin{pgfscope}%
\pgfsys@transformshift{8.282041in}{7.654082in}%
\pgfsys@useobject{currentmarker}{}%
\end{pgfscope}%
\end{pgfscope}%
\begin{pgfscope}%
\pgftext[x=7.894698in,y=7.590768in,left,base]{\rmfamily\fontsize{12.000000}{14.400000}\selectfont \(\displaystyle 0.50\)}%
\end{pgfscope}%
\begin{pgfscope}%
\pgfsetbuttcap%
\pgfsetroundjoin%
\definecolor{currentfill}{rgb}{0.000000,0.000000,0.000000}%
\pgfsetfillcolor{currentfill}%
\pgfsetlinewidth{0.803000pt}%
\definecolor{currentstroke}{rgb}{0.000000,0.000000,0.000000}%
\pgfsetstrokecolor{currentstroke}%
\pgfsetdash{}{0pt}%
\pgfsys@defobject{currentmarker}{\pgfqpoint{-0.048611in}{0.000000in}}{\pgfqpoint{0.000000in}{0.000000in}}{%
\pgfpathmoveto{\pgfqpoint{0.000000in}{0.000000in}}%
\pgfpathlineto{\pgfqpoint{-0.048611in}{0.000000in}}%
\pgfusepath{stroke,fill}%
}%
\begin{pgfscope}%
\pgfsys@transformshift{8.282041in}{7.997264in}%
\pgfsys@useobject{currentmarker}{}%
\end{pgfscope}%
\end{pgfscope}%
\begin{pgfscope}%
\pgftext[x=7.894698in,y=7.933950in,left,base]{\rmfamily\fontsize{12.000000}{14.400000}\selectfont \(\displaystyle 0.75\)}%
\end{pgfscope}%
\begin{pgfscope}%
\pgfsetbuttcap%
\pgfsetroundjoin%
\definecolor{currentfill}{rgb}{0.000000,0.000000,0.000000}%
\pgfsetfillcolor{currentfill}%
\pgfsetlinewidth{0.803000pt}%
\definecolor{currentstroke}{rgb}{0.000000,0.000000,0.000000}%
\pgfsetstrokecolor{currentstroke}%
\pgfsetdash{}{0pt}%
\pgfsys@defobject{currentmarker}{\pgfqpoint{-0.048611in}{0.000000in}}{\pgfqpoint{0.000000in}{0.000000in}}{%
\pgfpathmoveto{\pgfqpoint{0.000000in}{0.000000in}}%
\pgfpathlineto{\pgfqpoint{-0.048611in}{0.000000in}}%
\pgfusepath{stroke,fill}%
}%
\begin{pgfscope}%
\pgfsys@transformshift{8.282041in}{8.340446in}%
\pgfsys@useobject{currentmarker}{}%
\end{pgfscope}%
\end{pgfscope}%
\begin{pgfscope}%
\pgftext[x=7.894698in,y=8.277132in,left,base]{\rmfamily\fontsize{12.000000}{14.400000}\selectfont \(\displaystyle 1.00\)}%
\end{pgfscope}%
\begin{pgfscope}%
\pgfsetrectcap%
\pgfsetmiterjoin%
\pgfsetlinewidth{0.803000pt}%
\definecolor{currentstroke}{rgb}{0.501961,0.501961,0.501961}%
\pgfsetstrokecolor{currentstroke}%
\pgfsetdash{}{0pt}%
\pgfpathmoveto{\pgfqpoint{8.282041in}{6.967719in}}%
\pgfpathlineto{\pgfqpoint{8.282041in}{8.340446in}}%
\pgfusepath{stroke}%
\end{pgfscope}%
\begin{pgfscope}%
\pgfsetrectcap%
\pgfsetmiterjoin%
\pgfsetlinewidth{0.803000pt}%
\definecolor{currentstroke}{rgb}{0.501961,0.501961,0.501961}%
\pgfsetstrokecolor{currentstroke}%
\pgfsetdash{}{0pt}%
\pgfpathmoveto{\pgfqpoint{10.180000in}{6.967719in}}%
\pgfpathlineto{\pgfqpoint{10.180000in}{8.340446in}}%
\pgfusepath{stroke}%
\end{pgfscope}%
\begin{pgfscope}%
\pgfsetrectcap%
\pgfsetmiterjoin%
\pgfsetlinewidth{0.803000pt}%
\definecolor{currentstroke}{rgb}{0.501961,0.501961,0.501961}%
\pgfsetstrokecolor{currentstroke}%
\pgfsetdash{}{0pt}%
\pgfpathmoveto{\pgfqpoint{8.282041in}{6.967719in}}%
\pgfpathlineto{\pgfqpoint{10.180000in}{6.967719in}}%
\pgfusepath{stroke}%
\end{pgfscope}%
\begin{pgfscope}%
\pgfsetrectcap%
\pgfsetmiterjoin%
\pgfsetlinewidth{0.803000pt}%
\definecolor{currentstroke}{rgb}{0.501961,0.501961,0.501961}%
\pgfsetstrokecolor{currentstroke}%
\pgfsetdash{}{0pt}%
\pgfpathmoveto{\pgfqpoint{8.282041in}{8.340446in}}%
\pgfpathlineto{\pgfqpoint{10.180000in}{8.340446in}}%
\pgfusepath{stroke}%
\end{pgfscope}%
\begin{pgfscope}%
\pgfsetbuttcap%
\pgfsetmiterjoin%
\definecolor{currentfill}{rgb}{1.000000,1.000000,1.000000}%
\pgfsetfillcolor{currentfill}%
\pgfsetlinewidth{0.000000pt}%
\definecolor{currentstroke}{rgb}{0.000000,0.000000,0.000000}%
\pgfsetstrokecolor{currentstroke}%
\pgfsetstrokeopacity{0.000000}%
\pgfsetdash{}{0pt}%
\pgfpathmoveto{\pgfqpoint{0.880000in}{4.908628in}}%
\pgfpathlineto{\pgfqpoint{2.777959in}{4.908628in}}%
\pgfpathlineto{\pgfqpoint{2.777959in}{6.281355in}}%
\pgfpathlineto{\pgfqpoint{0.880000in}{6.281355in}}%
\pgfpathclose%
\pgfusepath{fill}%
\end{pgfscope}%
\begin{pgfscope}%
\pgfsetbuttcap%
\pgfsetroundjoin%
\definecolor{currentfill}{rgb}{0.000000,0.000000,0.000000}%
\pgfsetfillcolor{currentfill}%
\pgfsetlinewidth{0.803000pt}%
\definecolor{currentstroke}{rgb}{0.000000,0.000000,0.000000}%
\pgfsetstrokecolor{currentstroke}%
\pgfsetdash{}{0pt}%
\pgfsys@defobject{currentmarker}{\pgfqpoint{0.000000in}{-0.048611in}}{\pgfqpoint{0.000000in}{0.000000in}}{%
\pgfpathmoveto{\pgfqpoint{0.000000in}{0.000000in}}%
\pgfpathlineto{\pgfqpoint{0.000000in}{-0.048611in}}%
\pgfusepath{stroke,fill}%
}%
\begin{pgfscope}%
\pgfsys@transformshift{0.880000in}{4.908628in}%
\pgfsys@useobject{currentmarker}{}%
\end{pgfscope}%
\end{pgfscope}%
\begin{pgfscope}%
\pgftext[x=0.880000in,y=4.811405in,,top]{\rmfamily\fontsize{12.000000}{14.400000}\selectfont \(\displaystyle 0.0\)}%
\end{pgfscope}%
\begin{pgfscope}%
\pgfsetbuttcap%
\pgfsetroundjoin%
\definecolor{currentfill}{rgb}{0.000000,0.000000,0.000000}%
\pgfsetfillcolor{currentfill}%
\pgfsetlinewidth{0.803000pt}%
\definecolor{currentstroke}{rgb}{0.000000,0.000000,0.000000}%
\pgfsetstrokecolor{currentstroke}%
\pgfsetdash{}{0pt}%
\pgfsys@defobject{currentmarker}{\pgfqpoint{0.000000in}{-0.048611in}}{\pgfqpoint{0.000000in}{0.000000in}}{%
\pgfpathmoveto{\pgfqpoint{0.000000in}{0.000000in}}%
\pgfpathlineto{\pgfqpoint{0.000000in}{-0.048611in}}%
\pgfusepath{stroke,fill}%
}%
\begin{pgfscope}%
\pgfsys@transformshift{1.828980in}{4.908628in}%
\pgfsys@useobject{currentmarker}{}%
\end{pgfscope}%
\end{pgfscope}%
\begin{pgfscope}%
\pgftext[x=1.828980in,y=4.811405in,,top]{\rmfamily\fontsize{12.000000}{14.400000}\selectfont \(\displaystyle 0.5\)}%
\end{pgfscope}%
\begin{pgfscope}%
\pgfsetbuttcap%
\pgfsetroundjoin%
\definecolor{currentfill}{rgb}{0.000000,0.000000,0.000000}%
\pgfsetfillcolor{currentfill}%
\pgfsetlinewidth{0.803000pt}%
\definecolor{currentstroke}{rgb}{0.000000,0.000000,0.000000}%
\pgfsetstrokecolor{currentstroke}%
\pgfsetdash{}{0pt}%
\pgfsys@defobject{currentmarker}{\pgfqpoint{0.000000in}{-0.048611in}}{\pgfqpoint{0.000000in}{0.000000in}}{%
\pgfpathmoveto{\pgfqpoint{0.000000in}{0.000000in}}%
\pgfpathlineto{\pgfqpoint{0.000000in}{-0.048611in}}%
\pgfusepath{stroke,fill}%
}%
\begin{pgfscope}%
\pgfsys@transformshift{2.777959in}{4.908628in}%
\pgfsys@useobject{currentmarker}{}%
\end{pgfscope}%
\end{pgfscope}%
\begin{pgfscope}%
\pgftext[x=2.777959in,y=4.811405in,,top]{\rmfamily\fontsize{12.000000}{14.400000}\selectfont \(\displaystyle 1.0\)}%
\end{pgfscope}%
\begin{pgfscope}%
\pgfsetbuttcap%
\pgfsetroundjoin%
\definecolor{currentfill}{rgb}{0.000000,0.000000,0.000000}%
\pgfsetfillcolor{currentfill}%
\pgfsetlinewidth{0.803000pt}%
\definecolor{currentstroke}{rgb}{0.000000,0.000000,0.000000}%
\pgfsetstrokecolor{currentstroke}%
\pgfsetdash{}{0pt}%
\pgfsys@defobject{currentmarker}{\pgfqpoint{-0.048611in}{0.000000in}}{\pgfqpoint{0.000000in}{0.000000in}}{%
\pgfpathmoveto{\pgfqpoint{0.000000in}{0.000000in}}%
\pgfpathlineto{\pgfqpoint{-0.048611in}{0.000000in}}%
\pgfusepath{stroke,fill}%
}%
\begin{pgfscope}%
\pgfsys@transformshift{0.880000in}{4.908628in}%
\pgfsys@useobject{currentmarker}{}%
\end{pgfscope}%
\end{pgfscope}%
\begin{pgfscope}%
\pgftext[x=0.492657in,y=4.845314in,left,base]{\rmfamily\fontsize{12.000000}{14.400000}\selectfont \(\displaystyle 0.00\)}%
\end{pgfscope}%
\begin{pgfscope}%
\pgfsetbuttcap%
\pgfsetroundjoin%
\definecolor{currentfill}{rgb}{0.000000,0.000000,0.000000}%
\pgfsetfillcolor{currentfill}%
\pgfsetlinewidth{0.803000pt}%
\definecolor{currentstroke}{rgb}{0.000000,0.000000,0.000000}%
\pgfsetstrokecolor{currentstroke}%
\pgfsetdash{}{0pt}%
\pgfsys@defobject{currentmarker}{\pgfqpoint{-0.048611in}{0.000000in}}{\pgfqpoint{0.000000in}{0.000000in}}{%
\pgfpathmoveto{\pgfqpoint{0.000000in}{0.000000in}}%
\pgfpathlineto{\pgfqpoint{-0.048611in}{0.000000in}}%
\pgfusepath{stroke,fill}%
}%
\begin{pgfscope}%
\pgfsys@transformshift{0.880000in}{5.251809in}%
\pgfsys@useobject{currentmarker}{}%
\end{pgfscope}%
\end{pgfscope}%
\begin{pgfscope}%
\pgftext[x=0.492657in,y=5.188496in,left,base]{\rmfamily\fontsize{12.000000}{14.400000}\selectfont \(\displaystyle 0.25\)}%
\end{pgfscope}%
\begin{pgfscope}%
\pgfsetbuttcap%
\pgfsetroundjoin%
\definecolor{currentfill}{rgb}{0.000000,0.000000,0.000000}%
\pgfsetfillcolor{currentfill}%
\pgfsetlinewidth{0.803000pt}%
\definecolor{currentstroke}{rgb}{0.000000,0.000000,0.000000}%
\pgfsetstrokecolor{currentstroke}%
\pgfsetdash{}{0pt}%
\pgfsys@defobject{currentmarker}{\pgfqpoint{-0.048611in}{0.000000in}}{\pgfqpoint{0.000000in}{0.000000in}}{%
\pgfpathmoveto{\pgfqpoint{0.000000in}{0.000000in}}%
\pgfpathlineto{\pgfqpoint{-0.048611in}{0.000000in}}%
\pgfusepath{stroke,fill}%
}%
\begin{pgfscope}%
\pgfsys@transformshift{0.880000in}{5.594991in}%
\pgfsys@useobject{currentmarker}{}%
\end{pgfscope}%
\end{pgfscope}%
\begin{pgfscope}%
\pgftext[x=0.492657in,y=5.531677in,left,base]{\rmfamily\fontsize{12.000000}{14.400000}\selectfont \(\displaystyle 0.50\)}%
\end{pgfscope}%
\begin{pgfscope}%
\pgfsetbuttcap%
\pgfsetroundjoin%
\definecolor{currentfill}{rgb}{0.000000,0.000000,0.000000}%
\pgfsetfillcolor{currentfill}%
\pgfsetlinewidth{0.803000pt}%
\definecolor{currentstroke}{rgb}{0.000000,0.000000,0.000000}%
\pgfsetstrokecolor{currentstroke}%
\pgfsetdash{}{0pt}%
\pgfsys@defobject{currentmarker}{\pgfqpoint{-0.048611in}{0.000000in}}{\pgfqpoint{0.000000in}{0.000000in}}{%
\pgfpathmoveto{\pgfqpoint{0.000000in}{0.000000in}}%
\pgfpathlineto{\pgfqpoint{-0.048611in}{0.000000in}}%
\pgfusepath{stroke,fill}%
}%
\begin{pgfscope}%
\pgfsys@transformshift{0.880000in}{5.938173in}%
\pgfsys@useobject{currentmarker}{}%
\end{pgfscope}%
\end{pgfscope}%
\begin{pgfscope}%
\pgftext[x=0.492657in,y=5.874859in,left,base]{\rmfamily\fontsize{12.000000}{14.400000}\selectfont \(\displaystyle 0.75\)}%
\end{pgfscope}%
\begin{pgfscope}%
\pgfsetbuttcap%
\pgfsetroundjoin%
\definecolor{currentfill}{rgb}{0.000000,0.000000,0.000000}%
\pgfsetfillcolor{currentfill}%
\pgfsetlinewidth{0.803000pt}%
\definecolor{currentstroke}{rgb}{0.000000,0.000000,0.000000}%
\pgfsetstrokecolor{currentstroke}%
\pgfsetdash{}{0pt}%
\pgfsys@defobject{currentmarker}{\pgfqpoint{-0.048611in}{0.000000in}}{\pgfqpoint{0.000000in}{0.000000in}}{%
\pgfpathmoveto{\pgfqpoint{0.000000in}{0.000000in}}%
\pgfpathlineto{\pgfqpoint{-0.048611in}{0.000000in}}%
\pgfusepath{stroke,fill}%
}%
\begin{pgfscope}%
\pgfsys@transformshift{0.880000in}{6.281355in}%
\pgfsys@useobject{currentmarker}{}%
\end{pgfscope}%
\end{pgfscope}%
\begin{pgfscope}%
\pgftext[x=0.492657in,y=6.218041in,left,base]{\rmfamily\fontsize{12.000000}{14.400000}\selectfont \(\displaystyle 1.00\)}%
\end{pgfscope}%
\begin{pgfscope}%
\pgfsetrectcap%
\pgfsetmiterjoin%
\pgfsetlinewidth{0.803000pt}%
\definecolor{currentstroke}{rgb}{0.501961,0.501961,0.501961}%
\pgfsetstrokecolor{currentstroke}%
\pgfsetdash{}{0pt}%
\pgfpathmoveto{\pgfqpoint{0.880000in}{4.908628in}}%
\pgfpathlineto{\pgfqpoint{0.880000in}{6.281355in}}%
\pgfusepath{stroke}%
\end{pgfscope}%
\begin{pgfscope}%
\pgfsetrectcap%
\pgfsetmiterjoin%
\pgfsetlinewidth{0.803000pt}%
\definecolor{currentstroke}{rgb}{0.501961,0.501961,0.501961}%
\pgfsetstrokecolor{currentstroke}%
\pgfsetdash{}{0pt}%
\pgfpathmoveto{\pgfqpoint{2.777959in}{4.908628in}}%
\pgfpathlineto{\pgfqpoint{2.777959in}{6.281355in}}%
\pgfusepath{stroke}%
\end{pgfscope}%
\begin{pgfscope}%
\pgfsetrectcap%
\pgfsetmiterjoin%
\pgfsetlinewidth{0.803000pt}%
\definecolor{currentstroke}{rgb}{0.501961,0.501961,0.501961}%
\pgfsetstrokecolor{currentstroke}%
\pgfsetdash{}{0pt}%
\pgfpathmoveto{\pgfqpoint{0.880000in}{4.908628in}}%
\pgfpathlineto{\pgfqpoint{2.777959in}{4.908628in}}%
\pgfusepath{stroke}%
\end{pgfscope}%
\begin{pgfscope}%
\pgfsetrectcap%
\pgfsetmiterjoin%
\pgfsetlinewidth{0.803000pt}%
\definecolor{currentstroke}{rgb}{0.501961,0.501961,0.501961}%
\pgfsetstrokecolor{currentstroke}%
\pgfsetdash{}{0pt}%
\pgfpathmoveto{\pgfqpoint{0.880000in}{6.281355in}}%
\pgfpathlineto{\pgfqpoint{2.777959in}{6.281355in}}%
\pgfusepath{stroke}%
\end{pgfscope}%
\begin{pgfscope}%
\pgfsetbuttcap%
\pgfsetmiterjoin%
\definecolor{currentfill}{rgb}{1.000000,1.000000,1.000000}%
\pgfsetfillcolor{currentfill}%
\pgfsetlinewidth{0.000000pt}%
\definecolor{currentstroke}{rgb}{0.000000,0.000000,0.000000}%
\pgfsetstrokecolor{currentstroke}%
\pgfsetstrokeopacity{0.000000}%
\pgfsetdash{}{0pt}%
\pgfpathmoveto{\pgfqpoint{3.347347in}{4.908628in}}%
\pgfpathlineto{\pgfqpoint{5.245306in}{4.908628in}}%
\pgfpathlineto{\pgfqpoint{5.245306in}{6.281355in}}%
\pgfpathlineto{\pgfqpoint{3.347347in}{6.281355in}}%
\pgfpathclose%
\pgfusepath{fill}%
\end{pgfscope}%
\begin{pgfscope}%
\pgfsetbuttcap%
\pgfsetroundjoin%
\definecolor{currentfill}{rgb}{0.000000,0.000000,0.000000}%
\pgfsetfillcolor{currentfill}%
\pgfsetlinewidth{0.803000pt}%
\definecolor{currentstroke}{rgb}{0.000000,0.000000,0.000000}%
\pgfsetstrokecolor{currentstroke}%
\pgfsetdash{}{0pt}%
\pgfsys@defobject{currentmarker}{\pgfqpoint{0.000000in}{-0.048611in}}{\pgfqpoint{0.000000in}{0.000000in}}{%
\pgfpathmoveto{\pgfqpoint{0.000000in}{0.000000in}}%
\pgfpathlineto{\pgfqpoint{0.000000in}{-0.048611in}}%
\pgfusepath{stroke,fill}%
}%
\begin{pgfscope}%
\pgfsys@transformshift{3.347347in}{4.908628in}%
\pgfsys@useobject{currentmarker}{}%
\end{pgfscope}%
\end{pgfscope}%
\begin{pgfscope}%
\pgftext[x=3.347347in,y=4.811405in,,top]{\rmfamily\fontsize{12.000000}{14.400000}\selectfont \(\displaystyle 0.0\)}%
\end{pgfscope}%
\begin{pgfscope}%
\pgfsetbuttcap%
\pgfsetroundjoin%
\definecolor{currentfill}{rgb}{0.000000,0.000000,0.000000}%
\pgfsetfillcolor{currentfill}%
\pgfsetlinewidth{0.803000pt}%
\definecolor{currentstroke}{rgb}{0.000000,0.000000,0.000000}%
\pgfsetstrokecolor{currentstroke}%
\pgfsetdash{}{0pt}%
\pgfsys@defobject{currentmarker}{\pgfqpoint{0.000000in}{-0.048611in}}{\pgfqpoint{0.000000in}{0.000000in}}{%
\pgfpathmoveto{\pgfqpoint{0.000000in}{0.000000in}}%
\pgfpathlineto{\pgfqpoint{0.000000in}{-0.048611in}}%
\pgfusepath{stroke,fill}%
}%
\begin{pgfscope}%
\pgfsys@transformshift{4.296327in}{4.908628in}%
\pgfsys@useobject{currentmarker}{}%
\end{pgfscope}%
\end{pgfscope}%
\begin{pgfscope}%
\pgftext[x=4.296327in,y=4.811405in,,top]{\rmfamily\fontsize{12.000000}{14.400000}\selectfont \(\displaystyle 0.5\)}%
\end{pgfscope}%
\begin{pgfscope}%
\pgfsetbuttcap%
\pgfsetroundjoin%
\definecolor{currentfill}{rgb}{0.000000,0.000000,0.000000}%
\pgfsetfillcolor{currentfill}%
\pgfsetlinewidth{0.803000pt}%
\definecolor{currentstroke}{rgb}{0.000000,0.000000,0.000000}%
\pgfsetstrokecolor{currentstroke}%
\pgfsetdash{}{0pt}%
\pgfsys@defobject{currentmarker}{\pgfqpoint{0.000000in}{-0.048611in}}{\pgfqpoint{0.000000in}{0.000000in}}{%
\pgfpathmoveto{\pgfqpoint{0.000000in}{0.000000in}}%
\pgfpathlineto{\pgfqpoint{0.000000in}{-0.048611in}}%
\pgfusepath{stroke,fill}%
}%
\begin{pgfscope}%
\pgfsys@transformshift{5.245306in}{4.908628in}%
\pgfsys@useobject{currentmarker}{}%
\end{pgfscope}%
\end{pgfscope}%
\begin{pgfscope}%
\pgftext[x=5.245306in,y=4.811405in,,top]{\rmfamily\fontsize{12.000000}{14.400000}\selectfont \(\displaystyle 1.0\)}%
\end{pgfscope}%
\begin{pgfscope}%
\pgfsetbuttcap%
\pgfsetroundjoin%
\definecolor{currentfill}{rgb}{0.000000,0.000000,0.000000}%
\pgfsetfillcolor{currentfill}%
\pgfsetlinewidth{0.803000pt}%
\definecolor{currentstroke}{rgb}{0.000000,0.000000,0.000000}%
\pgfsetstrokecolor{currentstroke}%
\pgfsetdash{}{0pt}%
\pgfsys@defobject{currentmarker}{\pgfqpoint{-0.048611in}{0.000000in}}{\pgfqpoint{0.000000in}{0.000000in}}{%
\pgfpathmoveto{\pgfqpoint{0.000000in}{0.000000in}}%
\pgfpathlineto{\pgfqpoint{-0.048611in}{0.000000in}}%
\pgfusepath{stroke,fill}%
}%
\begin{pgfscope}%
\pgfsys@transformshift{3.347347in}{4.908628in}%
\pgfsys@useobject{currentmarker}{}%
\end{pgfscope}%
\end{pgfscope}%
\begin{pgfscope}%
\pgftext[x=2.960004in,y=4.845314in,left,base]{\rmfamily\fontsize{12.000000}{14.400000}\selectfont \(\displaystyle 0.00\)}%
\end{pgfscope}%
\begin{pgfscope}%
\pgfsetbuttcap%
\pgfsetroundjoin%
\definecolor{currentfill}{rgb}{0.000000,0.000000,0.000000}%
\pgfsetfillcolor{currentfill}%
\pgfsetlinewidth{0.803000pt}%
\definecolor{currentstroke}{rgb}{0.000000,0.000000,0.000000}%
\pgfsetstrokecolor{currentstroke}%
\pgfsetdash{}{0pt}%
\pgfsys@defobject{currentmarker}{\pgfqpoint{-0.048611in}{0.000000in}}{\pgfqpoint{0.000000in}{0.000000in}}{%
\pgfpathmoveto{\pgfqpoint{0.000000in}{0.000000in}}%
\pgfpathlineto{\pgfqpoint{-0.048611in}{0.000000in}}%
\pgfusepath{stroke,fill}%
}%
\begin{pgfscope}%
\pgfsys@transformshift{3.347347in}{5.251809in}%
\pgfsys@useobject{currentmarker}{}%
\end{pgfscope}%
\end{pgfscope}%
\begin{pgfscope}%
\pgftext[x=2.960004in,y=5.188496in,left,base]{\rmfamily\fontsize{12.000000}{14.400000}\selectfont \(\displaystyle 0.25\)}%
\end{pgfscope}%
\begin{pgfscope}%
\pgfsetbuttcap%
\pgfsetroundjoin%
\definecolor{currentfill}{rgb}{0.000000,0.000000,0.000000}%
\pgfsetfillcolor{currentfill}%
\pgfsetlinewidth{0.803000pt}%
\definecolor{currentstroke}{rgb}{0.000000,0.000000,0.000000}%
\pgfsetstrokecolor{currentstroke}%
\pgfsetdash{}{0pt}%
\pgfsys@defobject{currentmarker}{\pgfqpoint{-0.048611in}{0.000000in}}{\pgfqpoint{0.000000in}{0.000000in}}{%
\pgfpathmoveto{\pgfqpoint{0.000000in}{0.000000in}}%
\pgfpathlineto{\pgfqpoint{-0.048611in}{0.000000in}}%
\pgfusepath{stroke,fill}%
}%
\begin{pgfscope}%
\pgfsys@transformshift{3.347347in}{5.594991in}%
\pgfsys@useobject{currentmarker}{}%
\end{pgfscope}%
\end{pgfscope}%
\begin{pgfscope}%
\pgftext[x=2.960004in,y=5.531677in,left,base]{\rmfamily\fontsize{12.000000}{14.400000}\selectfont \(\displaystyle 0.50\)}%
\end{pgfscope}%
\begin{pgfscope}%
\pgfsetbuttcap%
\pgfsetroundjoin%
\definecolor{currentfill}{rgb}{0.000000,0.000000,0.000000}%
\pgfsetfillcolor{currentfill}%
\pgfsetlinewidth{0.803000pt}%
\definecolor{currentstroke}{rgb}{0.000000,0.000000,0.000000}%
\pgfsetstrokecolor{currentstroke}%
\pgfsetdash{}{0pt}%
\pgfsys@defobject{currentmarker}{\pgfqpoint{-0.048611in}{0.000000in}}{\pgfqpoint{0.000000in}{0.000000in}}{%
\pgfpathmoveto{\pgfqpoint{0.000000in}{0.000000in}}%
\pgfpathlineto{\pgfqpoint{-0.048611in}{0.000000in}}%
\pgfusepath{stroke,fill}%
}%
\begin{pgfscope}%
\pgfsys@transformshift{3.347347in}{5.938173in}%
\pgfsys@useobject{currentmarker}{}%
\end{pgfscope}%
\end{pgfscope}%
\begin{pgfscope}%
\pgftext[x=2.960004in,y=5.874859in,left,base]{\rmfamily\fontsize{12.000000}{14.400000}\selectfont \(\displaystyle 0.75\)}%
\end{pgfscope}%
\begin{pgfscope}%
\pgfsetbuttcap%
\pgfsetroundjoin%
\definecolor{currentfill}{rgb}{0.000000,0.000000,0.000000}%
\pgfsetfillcolor{currentfill}%
\pgfsetlinewidth{0.803000pt}%
\definecolor{currentstroke}{rgb}{0.000000,0.000000,0.000000}%
\pgfsetstrokecolor{currentstroke}%
\pgfsetdash{}{0pt}%
\pgfsys@defobject{currentmarker}{\pgfqpoint{-0.048611in}{0.000000in}}{\pgfqpoint{0.000000in}{0.000000in}}{%
\pgfpathmoveto{\pgfqpoint{0.000000in}{0.000000in}}%
\pgfpathlineto{\pgfqpoint{-0.048611in}{0.000000in}}%
\pgfusepath{stroke,fill}%
}%
\begin{pgfscope}%
\pgfsys@transformshift{3.347347in}{6.281355in}%
\pgfsys@useobject{currentmarker}{}%
\end{pgfscope}%
\end{pgfscope}%
\begin{pgfscope}%
\pgftext[x=2.960004in,y=6.218041in,left,base]{\rmfamily\fontsize{12.000000}{14.400000}\selectfont \(\displaystyle 1.00\)}%
\end{pgfscope}%
\begin{pgfscope}%
\pgfsetrectcap%
\pgfsetmiterjoin%
\pgfsetlinewidth{0.803000pt}%
\definecolor{currentstroke}{rgb}{0.501961,0.501961,0.501961}%
\pgfsetstrokecolor{currentstroke}%
\pgfsetdash{}{0pt}%
\pgfpathmoveto{\pgfqpoint{3.347347in}{4.908628in}}%
\pgfpathlineto{\pgfqpoint{3.347347in}{6.281355in}}%
\pgfusepath{stroke}%
\end{pgfscope}%
\begin{pgfscope}%
\pgfsetrectcap%
\pgfsetmiterjoin%
\pgfsetlinewidth{0.803000pt}%
\definecolor{currentstroke}{rgb}{0.501961,0.501961,0.501961}%
\pgfsetstrokecolor{currentstroke}%
\pgfsetdash{}{0pt}%
\pgfpathmoveto{\pgfqpoint{5.245306in}{4.908628in}}%
\pgfpathlineto{\pgfqpoint{5.245306in}{6.281355in}}%
\pgfusepath{stroke}%
\end{pgfscope}%
\begin{pgfscope}%
\pgfsetrectcap%
\pgfsetmiterjoin%
\pgfsetlinewidth{0.803000pt}%
\definecolor{currentstroke}{rgb}{0.501961,0.501961,0.501961}%
\pgfsetstrokecolor{currentstroke}%
\pgfsetdash{}{0pt}%
\pgfpathmoveto{\pgfqpoint{3.347347in}{4.908628in}}%
\pgfpathlineto{\pgfqpoint{5.245306in}{4.908628in}}%
\pgfusepath{stroke}%
\end{pgfscope}%
\begin{pgfscope}%
\pgfsetrectcap%
\pgfsetmiterjoin%
\pgfsetlinewidth{0.803000pt}%
\definecolor{currentstroke}{rgb}{0.501961,0.501961,0.501961}%
\pgfsetstrokecolor{currentstroke}%
\pgfsetdash{}{0pt}%
\pgfpathmoveto{\pgfqpoint{3.347347in}{6.281355in}}%
\pgfpathlineto{\pgfqpoint{5.245306in}{6.281355in}}%
\pgfusepath{stroke}%
\end{pgfscope}%
\begin{pgfscope}%
\pgfsetbuttcap%
\pgfsetmiterjoin%
\definecolor{currentfill}{rgb}{1.000000,1.000000,1.000000}%
\pgfsetfillcolor{currentfill}%
\pgfsetlinewidth{0.000000pt}%
\definecolor{currentstroke}{rgb}{0.000000,0.000000,0.000000}%
\pgfsetstrokecolor{currentstroke}%
\pgfsetstrokeopacity{0.000000}%
\pgfsetdash{}{0pt}%
\pgfpathmoveto{\pgfqpoint{5.814694in}{4.908628in}}%
\pgfpathlineto{\pgfqpoint{7.712653in}{4.908628in}}%
\pgfpathlineto{\pgfqpoint{7.712653in}{6.281355in}}%
\pgfpathlineto{\pgfqpoint{5.814694in}{6.281355in}}%
\pgfpathclose%
\pgfusepath{fill}%
\end{pgfscope}%
\begin{pgfscope}%
\pgfsetbuttcap%
\pgfsetroundjoin%
\definecolor{currentfill}{rgb}{0.000000,0.000000,0.000000}%
\pgfsetfillcolor{currentfill}%
\pgfsetlinewidth{0.803000pt}%
\definecolor{currentstroke}{rgb}{0.000000,0.000000,0.000000}%
\pgfsetstrokecolor{currentstroke}%
\pgfsetdash{}{0pt}%
\pgfsys@defobject{currentmarker}{\pgfqpoint{0.000000in}{-0.048611in}}{\pgfqpoint{0.000000in}{0.000000in}}{%
\pgfpathmoveto{\pgfqpoint{0.000000in}{0.000000in}}%
\pgfpathlineto{\pgfqpoint{0.000000in}{-0.048611in}}%
\pgfusepath{stroke,fill}%
}%
\begin{pgfscope}%
\pgfsys@transformshift{5.814694in}{4.908628in}%
\pgfsys@useobject{currentmarker}{}%
\end{pgfscope}%
\end{pgfscope}%
\begin{pgfscope}%
\pgftext[x=5.814694in,y=4.811405in,,top]{\rmfamily\fontsize{12.000000}{14.400000}\selectfont \(\displaystyle 0.0\)}%
\end{pgfscope}%
\begin{pgfscope}%
\pgfsetbuttcap%
\pgfsetroundjoin%
\definecolor{currentfill}{rgb}{0.000000,0.000000,0.000000}%
\pgfsetfillcolor{currentfill}%
\pgfsetlinewidth{0.803000pt}%
\definecolor{currentstroke}{rgb}{0.000000,0.000000,0.000000}%
\pgfsetstrokecolor{currentstroke}%
\pgfsetdash{}{0pt}%
\pgfsys@defobject{currentmarker}{\pgfqpoint{0.000000in}{-0.048611in}}{\pgfqpoint{0.000000in}{0.000000in}}{%
\pgfpathmoveto{\pgfqpoint{0.000000in}{0.000000in}}%
\pgfpathlineto{\pgfqpoint{0.000000in}{-0.048611in}}%
\pgfusepath{stroke,fill}%
}%
\begin{pgfscope}%
\pgfsys@transformshift{6.763673in}{4.908628in}%
\pgfsys@useobject{currentmarker}{}%
\end{pgfscope}%
\end{pgfscope}%
\begin{pgfscope}%
\pgftext[x=6.763673in,y=4.811405in,,top]{\rmfamily\fontsize{12.000000}{14.400000}\selectfont \(\displaystyle 0.5\)}%
\end{pgfscope}%
\begin{pgfscope}%
\pgfsetbuttcap%
\pgfsetroundjoin%
\definecolor{currentfill}{rgb}{0.000000,0.000000,0.000000}%
\pgfsetfillcolor{currentfill}%
\pgfsetlinewidth{0.803000pt}%
\definecolor{currentstroke}{rgb}{0.000000,0.000000,0.000000}%
\pgfsetstrokecolor{currentstroke}%
\pgfsetdash{}{0pt}%
\pgfsys@defobject{currentmarker}{\pgfqpoint{0.000000in}{-0.048611in}}{\pgfqpoint{0.000000in}{0.000000in}}{%
\pgfpathmoveto{\pgfqpoint{0.000000in}{0.000000in}}%
\pgfpathlineto{\pgfqpoint{0.000000in}{-0.048611in}}%
\pgfusepath{stroke,fill}%
}%
\begin{pgfscope}%
\pgfsys@transformshift{7.712653in}{4.908628in}%
\pgfsys@useobject{currentmarker}{}%
\end{pgfscope}%
\end{pgfscope}%
\begin{pgfscope}%
\pgftext[x=7.712653in,y=4.811405in,,top]{\rmfamily\fontsize{12.000000}{14.400000}\selectfont \(\displaystyle 1.0\)}%
\end{pgfscope}%
\begin{pgfscope}%
\pgfsetbuttcap%
\pgfsetroundjoin%
\definecolor{currentfill}{rgb}{0.000000,0.000000,0.000000}%
\pgfsetfillcolor{currentfill}%
\pgfsetlinewidth{0.803000pt}%
\definecolor{currentstroke}{rgb}{0.000000,0.000000,0.000000}%
\pgfsetstrokecolor{currentstroke}%
\pgfsetdash{}{0pt}%
\pgfsys@defobject{currentmarker}{\pgfqpoint{-0.048611in}{0.000000in}}{\pgfqpoint{0.000000in}{0.000000in}}{%
\pgfpathmoveto{\pgfqpoint{0.000000in}{0.000000in}}%
\pgfpathlineto{\pgfqpoint{-0.048611in}{0.000000in}}%
\pgfusepath{stroke,fill}%
}%
\begin{pgfscope}%
\pgfsys@transformshift{5.814694in}{4.908628in}%
\pgfsys@useobject{currentmarker}{}%
\end{pgfscope}%
\end{pgfscope}%
\begin{pgfscope}%
\pgftext[x=5.427351in,y=4.845314in,left,base]{\rmfamily\fontsize{12.000000}{14.400000}\selectfont \(\displaystyle 0.00\)}%
\end{pgfscope}%
\begin{pgfscope}%
\pgfsetbuttcap%
\pgfsetroundjoin%
\definecolor{currentfill}{rgb}{0.000000,0.000000,0.000000}%
\pgfsetfillcolor{currentfill}%
\pgfsetlinewidth{0.803000pt}%
\definecolor{currentstroke}{rgb}{0.000000,0.000000,0.000000}%
\pgfsetstrokecolor{currentstroke}%
\pgfsetdash{}{0pt}%
\pgfsys@defobject{currentmarker}{\pgfqpoint{-0.048611in}{0.000000in}}{\pgfqpoint{0.000000in}{0.000000in}}{%
\pgfpathmoveto{\pgfqpoint{0.000000in}{0.000000in}}%
\pgfpathlineto{\pgfqpoint{-0.048611in}{0.000000in}}%
\pgfusepath{stroke,fill}%
}%
\begin{pgfscope}%
\pgfsys@transformshift{5.814694in}{5.251809in}%
\pgfsys@useobject{currentmarker}{}%
\end{pgfscope}%
\end{pgfscope}%
\begin{pgfscope}%
\pgftext[x=5.427351in,y=5.188496in,left,base]{\rmfamily\fontsize{12.000000}{14.400000}\selectfont \(\displaystyle 0.25\)}%
\end{pgfscope}%
\begin{pgfscope}%
\pgfsetbuttcap%
\pgfsetroundjoin%
\definecolor{currentfill}{rgb}{0.000000,0.000000,0.000000}%
\pgfsetfillcolor{currentfill}%
\pgfsetlinewidth{0.803000pt}%
\definecolor{currentstroke}{rgb}{0.000000,0.000000,0.000000}%
\pgfsetstrokecolor{currentstroke}%
\pgfsetdash{}{0pt}%
\pgfsys@defobject{currentmarker}{\pgfqpoint{-0.048611in}{0.000000in}}{\pgfqpoint{0.000000in}{0.000000in}}{%
\pgfpathmoveto{\pgfqpoint{0.000000in}{0.000000in}}%
\pgfpathlineto{\pgfqpoint{-0.048611in}{0.000000in}}%
\pgfusepath{stroke,fill}%
}%
\begin{pgfscope}%
\pgfsys@transformshift{5.814694in}{5.594991in}%
\pgfsys@useobject{currentmarker}{}%
\end{pgfscope}%
\end{pgfscope}%
\begin{pgfscope}%
\pgftext[x=5.427351in,y=5.531677in,left,base]{\rmfamily\fontsize{12.000000}{14.400000}\selectfont \(\displaystyle 0.50\)}%
\end{pgfscope}%
\begin{pgfscope}%
\pgfsetbuttcap%
\pgfsetroundjoin%
\definecolor{currentfill}{rgb}{0.000000,0.000000,0.000000}%
\pgfsetfillcolor{currentfill}%
\pgfsetlinewidth{0.803000pt}%
\definecolor{currentstroke}{rgb}{0.000000,0.000000,0.000000}%
\pgfsetstrokecolor{currentstroke}%
\pgfsetdash{}{0pt}%
\pgfsys@defobject{currentmarker}{\pgfqpoint{-0.048611in}{0.000000in}}{\pgfqpoint{0.000000in}{0.000000in}}{%
\pgfpathmoveto{\pgfqpoint{0.000000in}{0.000000in}}%
\pgfpathlineto{\pgfqpoint{-0.048611in}{0.000000in}}%
\pgfusepath{stroke,fill}%
}%
\begin{pgfscope}%
\pgfsys@transformshift{5.814694in}{5.938173in}%
\pgfsys@useobject{currentmarker}{}%
\end{pgfscope}%
\end{pgfscope}%
\begin{pgfscope}%
\pgftext[x=5.427351in,y=5.874859in,left,base]{\rmfamily\fontsize{12.000000}{14.400000}\selectfont \(\displaystyle 0.75\)}%
\end{pgfscope}%
\begin{pgfscope}%
\pgfsetbuttcap%
\pgfsetroundjoin%
\definecolor{currentfill}{rgb}{0.000000,0.000000,0.000000}%
\pgfsetfillcolor{currentfill}%
\pgfsetlinewidth{0.803000pt}%
\definecolor{currentstroke}{rgb}{0.000000,0.000000,0.000000}%
\pgfsetstrokecolor{currentstroke}%
\pgfsetdash{}{0pt}%
\pgfsys@defobject{currentmarker}{\pgfqpoint{-0.048611in}{0.000000in}}{\pgfqpoint{0.000000in}{0.000000in}}{%
\pgfpathmoveto{\pgfqpoint{0.000000in}{0.000000in}}%
\pgfpathlineto{\pgfqpoint{-0.048611in}{0.000000in}}%
\pgfusepath{stroke,fill}%
}%
\begin{pgfscope}%
\pgfsys@transformshift{5.814694in}{6.281355in}%
\pgfsys@useobject{currentmarker}{}%
\end{pgfscope}%
\end{pgfscope}%
\begin{pgfscope}%
\pgftext[x=5.427351in,y=6.218041in,left,base]{\rmfamily\fontsize{12.000000}{14.400000}\selectfont \(\displaystyle 1.00\)}%
\end{pgfscope}%
\begin{pgfscope}%
\pgfsetrectcap%
\pgfsetmiterjoin%
\pgfsetlinewidth{0.803000pt}%
\definecolor{currentstroke}{rgb}{0.501961,0.501961,0.501961}%
\pgfsetstrokecolor{currentstroke}%
\pgfsetdash{}{0pt}%
\pgfpathmoveto{\pgfqpoint{5.814694in}{4.908628in}}%
\pgfpathlineto{\pgfqpoint{5.814694in}{6.281355in}}%
\pgfusepath{stroke}%
\end{pgfscope}%
\begin{pgfscope}%
\pgfsetrectcap%
\pgfsetmiterjoin%
\pgfsetlinewidth{0.803000pt}%
\definecolor{currentstroke}{rgb}{0.501961,0.501961,0.501961}%
\pgfsetstrokecolor{currentstroke}%
\pgfsetdash{}{0pt}%
\pgfpathmoveto{\pgfqpoint{7.712653in}{4.908628in}}%
\pgfpathlineto{\pgfqpoint{7.712653in}{6.281355in}}%
\pgfusepath{stroke}%
\end{pgfscope}%
\begin{pgfscope}%
\pgfsetrectcap%
\pgfsetmiterjoin%
\pgfsetlinewidth{0.803000pt}%
\definecolor{currentstroke}{rgb}{0.501961,0.501961,0.501961}%
\pgfsetstrokecolor{currentstroke}%
\pgfsetdash{}{0pt}%
\pgfpathmoveto{\pgfqpoint{5.814694in}{4.908628in}}%
\pgfpathlineto{\pgfqpoint{7.712653in}{4.908628in}}%
\pgfusepath{stroke}%
\end{pgfscope}%
\begin{pgfscope}%
\pgfsetrectcap%
\pgfsetmiterjoin%
\pgfsetlinewidth{0.803000pt}%
\definecolor{currentstroke}{rgb}{0.501961,0.501961,0.501961}%
\pgfsetstrokecolor{currentstroke}%
\pgfsetdash{}{0pt}%
\pgfpathmoveto{\pgfqpoint{5.814694in}{6.281355in}}%
\pgfpathlineto{\pgfqpoint{7.712653in}{6.281355in}}%
\pgfusepath{stroke}%
\end{pgfscope}%
\begin{pgfscope}%
\pgfsetbuttcap%
\pgfsetmiterjoin%
\definecolor{currentfill}{rgb}{1.000000,1.000000,1.000000}%
\pgfsetfillcolor{currentfill}%
\pgfsetlinewidth{0.000000pt}%
\definecolor{currentstroke}{rgb}{0.000000,0.000000,0.000000}%
\pgfsetstrokecolor{currentstroke}%
\pgfsetstrokeopacity{0.000000}%
\pgfsetdash{}{0pt}%
\pgfpathmoveto{\pgfqpoint{8.282041in}{4.908628in}}%
\pgfpathlineto{\pgfqpoint{10.180000in}{4.908628in}}%
\pgfpathlineto{\pgfqpoint{10.180000in}{6.281355in}}%
\pgfpathlineto{\pgfqpoint{8.282041in}{6.281355in}}%
\pgfpathclose%
\pgfusepath{fill}%
\end{pgfscope}%
\begin{pgfscope}%
\pgfsetbuttcap%
\pgfsetroundjoin%
\definecolor{currentfill}{rgb}{0.000000,0.000000,0.000000}%
\pgfsetfillcolor{currentfill}%
\pgfsetlinewidth{0.803000pt}%
\definecolor{currentstroke}{rgb}{0.000000,0.000000,0.000000}%
\pgfsetstrokecolor{currentstroke}%
\pgfsetdash{}{0pt}%
\pgfsys@defobject{currentmarker}{\pgfqpoint{0.000000in}{-0.048611in}}{\pgfqpoint{0.000000in}{0.000000in}}{%
\pgfpathmoveto{\pgfqpoint{0.000000in}{0.000000in}}%
\pgfpathlineto{\pgfqpoint{0.000000in}{-0.048611in}}%
\pgfusepath{stroke,fill}%
}%
\begin{pgfscope}%
\pgfsys@transformshift{8.282041in}{4.908628in}%
\pgfsys@useobject{currentmarker}{}%
\end{pgfscope}%
\end{pgfscope}%
\begin{pgfscope}%
\pgftext[x=8.282041in,y=4.811405in,,top]{\rmfamily\fontsize{12.000000}{14.400000}\selectfont \(\displaystyle 0.0\)}%
\end{pgfscope}%
\begin{pgfscope}%
\pgfsetbuttcap%
\pgfsetroundjoin%
\definecolor{currentfill}{rgb}{0.000000,0.000000,0.000000}%
\pgfsetfillcolor{currentfill}%
\pgfsetlinewidth{0.803000pt}%
\definecolor{currentstroke}{rgb}{0.000000,0.000000,0.000000}%
\pgfsetstrokecolor{currentstroke}%
\pgfsetdash{}{0pt}%
\pgfsys@defobject{currentmarker}{\pgfqpoint{0.000000in}{-0.048611in}}{\pgfqpoint{0.000000in}{0.000000in}}{%
\pgfpathmoveto{\pgfqpoint{0.000000in}{0.000000in}}%
\pgfpathlineto{\pgfqpoint{0.000000in}{-0.048611in}}%
\pgfusepath{stroke,fill}%
}%
\begin{pgfscope}%
\pgfsys@transformshift{9.231020in}{4.908628in}%
\pgfsys@useobject{currentmarker}{}%
\end{pgfscope}%
\end{pgfscope}%
\begin{pgfscope}%
\pgftext[x=9.231020in,y=4.811405in,,top]{\rmfamily\fontsize{12.000000}{14.400000}\selectfont \(\displaystyle 0.5\)}%
\end{pgfscope}%
\begin{pgfscope}%
\pgfsetbuttcap%
\pgfsetroundjoin%
\definecolor{currentfill}{rgb}{0.000000,0.000000,0.000000}%
\pgfsetfillcolor{currentfill}%
\pgfsetlinewidth{0.803000pt}%
\definecolor{currentstroke}{rgb}{0.000000,0.000000,0.000000}%
\pgfsetstrokecolor{currentstroke}%
\pgfsetdash{}{0pt}%
\pgfsys@defobject{currentmarker}{\pgfqpoint{0.000000in}{-0.048611in}}{\pgfqpoint{0.000000in}{0.000000in}}{%
\pgfpathmoveto{\pgfqpoint{0.000000in}{0.000000in}}%
\pgfpathlineto{\pgfqpoint{0.000000in}{-0.048611in}}%
\pgfusepath{stroke,fill}%
}%
\begin{pgfscope}%
\pgfsys@transformshift{10.180000in}{4.908628in}%
\pgfsys@useobject{currentmarker}{}%
\end{pgfscope}%
\end{pgfscope}%
\begin{pgfscope}%
\pgftext[x=10.180000in,y=4.811405in,,top]{\rmfamily\fontsize{12.000000}{14.400000}\selectfont \(\displaystyle 1.0\)}%
\end{pgfscope}%
\begin{pgfscope}%
\pgfsetbuttcap%
\pgfsetroundjoin%
\definecolor{currentfill}{rgb}{0.000000,0.000000,0.000000}%
\pgfsetfillcolor{currentfill}%
\pgfsetlinewidth{0.803000pt}%
\definecolor{currentstroke}{rgb}{0.000000,0.000000,0.000000}%
\pgfsetstrokecolor{currentstroke}%
\pgfsetdash{}{0pt}%
\pgfsys@defobject{currentmarker}{\pgfqpoint{-0.048611in}{0.000000in}}{\pgfqpoint{0.000000in}{0.000000in}}{%
\pgfpathmoveto{\pgfqpoint{0.000000in}{0.000000in}}%
\pgfpathlineto{\pgfqpoint{-0.048611in}{0.000000in}}%
\pgfusepath{stroke,fill}%
}%
\begin{pgfscope}%
\pgfsys@transformshift{8.282041in}{4.908628in}%
\pgfsys@useobject{currentmarker}{}%
\end{pgfscope}%
\end{pgfscope}%
\begin{pgfscope}%
\pgftext[x=7.894698in,y=4.845314in,left,base]{\rmfamily\fontsize{12.000000}{14.400000}\selectfont \(\displaystyle 0.00\)}%
\end{pgfscope}%
\begin{pgfscope}%
\pgfsetbuttcap%
\pgfsetroundjoin%
\definecolor{currentfill}{rgb}{0.000000,0.000000,0.000000}%
\pgfsetfillcolor{currentfill}%
\pgfsetlinewidth{0.803000pt}%
\definecolor{currentstroke}{rgb}{0.000000,0.000000,0.000000}%
\pgfsetstrokecolor{currentstroke}%
\pgfsetdash{}{0pt}%
\pgfsys@defobject{currentmarker}{\pgfqpoint{-0.048611in}{0.000000in}}{\pgfqpoint{0.000000in}{0.000000in}}{%
\pgfpathmoveto{\pgfqpoint{0.000000in}{0.000000in}}%
\pgfpathlineto{\pgfqpoint{-0.048611in}{0.000000in}}%
\pgfusepath{stroke,fill}%
}%
\begin{pgfscope}%
\pgfsys@transformshift{8.282041in}{5.251809in}%
\pgfsys@useobject{currentmarker}{}%
\end{pgfscope}%
\end{pgfscope}%
\begin{pgfscope}%
\pgftext[x=7.894698in,y=5.188496in,left,base]{\rmfamily\fontsize{12.000000}{14.400000}\selectfont \(\displaystyle 0.25\)}%
\end{pgfscope}%
\begin{pgfscope}%
\pgfsetbuttcap%
\pgfsetroundjoin%
\definecolor{currentfill}{rgb}{0.000000,0.000000,0.000000}%
\pgfsetfillcolor{currentfill}%
\pgfsetlinewidth{0.803000pt}%
\definecolor{currentstroke}{rgb}{0.000000,0.000000,0.000000}%
\pgfsetstrokecolor{currentstroke}%
\pgfsetdash{}{0pt}%
\pgfsys@defobject{currentmarker}{\pgfqpoint{-0.048611in}{0.000000in}}{\pgfqpoint{0.000000in}{0.000000in}}{%
\pgfpathmoveto{\pgfqpoint{0.000000in}{0.000000in}}%
\pgfpathlineto{\pgfqpoint{-0.048611in}{0.000000in}}%
\pgfusepath{stroke,fill}%
}%
\begin{pgfscope}%
\pgfsys@transformshift{8.282041in}{5.594991in}%
\pgfsys@useobject{currentmarker}{}%
\end{pgfscope}%
\end{pgfscope}%
\begin{pgfscope}%
\pgftext[x=7.894698in,y=5.531677in,left,base]{\rmfamily\fontsize{12.000000}{14.400000}\selectfont \(\displaystyle 0.50\)}%
\end{pgfscope}%
\begin{pgfscope}%
\pgfsetbuttcap%
\pgfsetroundjoin%
\definecolor{currentfill}{rgb}{0.000000,0.000000,0.000000}%
\pgfsetfillcolor{currentfill}%
\pgfsetlinewidth{0.803000pt}%
\definecolor{currentstroke}{rgb}{0.000000,0.000000,0.000000}%
\pgfsetstrokecolor{currentstroke}%
\pgfsetdash{}{0pt}%
\pgfsys@defobject{currentmarker}{\pgfqpoint{-0.048611in}{0.000000in}}{\pgfqpoint{0.000000in}{0.000000in}}{%
\pgfpathmoveto{\pgfqpoint{0.000000in}{0.000000in}}%
\pgfpathlineto{\pgfqpoint{-0.048611in}{0.000000in}}%
\pgfusepath{stroke,fill}%
}%
\begin{pgfscope}%
\pgfsys@transformshift{8.282041in}{5.938173in}%
\pgfsys@useobject{currentmarker}{}%
\end{pgfscope}%
\end{pgfscope}%
\begin{pgfscope}%
\pgftext[x=7.894698in,y=5.874859in,left,base]{\rmfamily\fontsize{12.000000}{14.400000}\selectfont \(\displaystyle 0.75\)}%
\end{pgfscope}%
\begin{pgfscope}%
\pgfsetbuttcap%
\pgfsetroundjoin%
\definecolor{currentfill}{rgb}{0.000000,0.000000,0.000000}%
\pgfsetfillcolor{currentfill}%
\pgfsetlinewidth{0.803000pt}%
\definecolor{currentstroke}{rgb}{0.000000,0.000000,0.000000}%
\pgfsetstrokecolor{currentstroke}%
\pgfsetdash{}{0pt}%
\pgfsys@defobject{currentmarker}{\pgfqpoint{-0.048611in}{0.000000in}}{\pgfqpoint{0.000000in}{0.000000in}}{%
\pgfpathmoveto{\pgfqpoint{0.000000in}{0.000000in}}%
\pgfpathlineto{\pgfqpoint{-0.048611in}{0.000000in}}%
\pgfusepath{stroke,fill}%
}%
\begin{pgfscope}%
\pgfsys@transformshift{8.282041in}{6.281355in}%
\pgfsys@useobject{currentmarker}{}%
\end{pgfscope}%
\end{pgfscope}%
\begin{pgfscope}%
\pgftext[x=7.894698in,y=6.218041in,left,base]{\rmfamily\fontsize{12.000000}{14.400000}\selectfont \(\displaystyle 1.00\)}%
\end{pgfscope}%
\begin{pgfscope}%
\pgfsetrectcap%
\pgfsetmiterjoin%
\pgfsetlinewidth{0.803000pt}%
\definecolor{currentstroke}{rgb}{0.501961,0.501961,0.501961}%
\pgfsetstrokecolor{currentstroke}%
\pgfsetdash{}{0pt}%
\pgfpathmoveto{\pgfqpoint{8.282041in}{4.908628in}}%
\pgfpathlineto{\pgfqpoint{8.282041in}{6.281355in}}%
\pgfusepath{stroke}%
\end{pgfscope}%
\begin{pgfscope}%
\pgfsetrectcap%
\pgfsetmiterjoin%
\pgfsetlinewidth{0.803000pt}%
\definecolor{currentstroke}{rgb}{0.501961,0.501961,0.501961}%
\pgfsetstrokecolor{currentstroke}%
\pgfsetdash{}{0pt}%
\pgfpathmoveto{\pgfqpoint{10.180000in}{4.908628in}}%
\pgfpathlineto{\pgfqpoint{10.180000in}{6.281355in}}%
\pgfusepath{stroke}%
\end{pgfscope}%
\begin{pgfscope}%
\pgfsetrectcap%
\pgfsetmiterjoin%
\pgfsetlinewidth{0.803000pt}%
\definecolor{currentstroke}{rgb}{0.501961,0.501961,0.501961}%
\pgfsetstrokecolor{currentstroke}%
\pgfsetdash{}{0pt}%
\pgfpathmoveto{\pgfqpoint{8.282041in}{4.908628in}}%
\pgfpathlineto{\pgfqpoint{10.180000in}{4.908628in}}%
\pgfusepath{stroke}%
\end{pgfscope}%
\begin{pgfscope}%
\pgfsetrectcap%
\pgfsetmiterjoin%
\pgfsetlinewidth{0.803000pt}%
\definecolor{currentstroke}{rgb}{0.501961,0.501961,0.501961}%
\pgfsetstrokecolor{currentstroke}%
\pgfsetdash{}{0pt}%
\pgfpathmoveto{\pgfqpoint{8.282041in}{6.281355in}}%
\pgfpathlineto{\pgfqpoint{10.180000in}{6.281355in}}%
\pgfusepath{stroke}%
\end{pgfscope}%
\begin{pgfscope}%
\pgfsetbuttcap%
\pgfsetmiterjoin%
\definecolor{currentfill}{rgb}{1.000000,1.000000,1.000000}%
\pgfsetfillcolor{currentfill}%
\pgfsetlinewidth{0.000000pt}%
\definecolor{currentstroke}{rgb}{0.000000,0.000000,0.000000}%
\pgfsetstrokecolor{currentstroke}%
\pgfsetstrokeopacity{0.000000}%
\pgfsetdash{}{0pt}%
\pgfpathmoveto{\pgfqpoint{0.880000in}{2.849537in}}%
\pgfpathlineto{\pgfqpoint{2.777959in}{2.849537in}}%
\pgfpathlineto{\pgfqpoint{2.777959in}{4.222264in}}%
\pgfpathlineto{\pgfqpoint{0.880000in}{4.222264in}}%
\pgfpathclose%
\pgfusepath{fill}%
\end{pgfscope}%
\begin{pgfscope}%
\pgfsetbuttcap%
\pgfsetroundjoin%
\definecolor{currentfill}{rgb}{0.000000,0.000000,0.000000}%
\pgfsetfillcolor{currentfill}%
\pgfsetlinewidth{0.803000pt}%
\definecolor{currentstroke}{rgb}{0.000000,0.000000,0.000000}%
\pgfsetstrokecolor{currentstroke}%
\pgfsetdash{}{0pt}%
\pgfsys@defobject{currentmarker}{\pgfqpoint{0.000000in}{-0.048611in}}{\pgfqpoint{0.000000in}{0.000000in}}{%
\pgfpathmoveto{\pgfqpoint{0.000000in}{0.000000in}}%
\pgfpathlineto{\pgfqpoint{0.000000in}{-0.048611in}}%
\pgfusepath{stroke,fill}%
}%
\begin{pgfscope}%
\pgfsys@transformshift{0.880000in}{2.849537in}%
\pgfsys@useobject{currentmarker}{}%
\end{pgfscope}%
\end{pgfscope}%
\begin{pgfscope}%
\pgftext[x=0.880000in,y=2.752315in,,top]{\rmfamily\fontsize{12.000000}{14.400000}\selectfont \(\displaystyle 0.0\)}%
\end{pgfscope}%
\begin{pgfscope}%
\pgfsetbuttcap%
\pgfsetroundjoin%
\definecolor{currentfill}{rgb}{0.000000,0.000000,0.000000}%
\pgfsetfillcolor{currentfill}%
\pgfsetlinewidth{0.803000pt}%
\definecolor{currentstroke}{rgb}{0.000000,0.000000,0.000000}%
\pgfsetstrokecolor{currentstroke}%
\pgfsetdash{}{0pt}%
\pgfsys@defobject{currentmarker}{\pgfqpoint{0.000000in}{-0.048611in}}{\pgfqpoint{0.000000in}{0.000000in}}{%
\pgfpathmoveto{\pgfqpoint{0.000000in}{0.000000in}}%
\pgfpathlineto{\pgfqpoint{0.000000in}{-0.048611in}}%
\pgfusepath{stroke,fill}%
}%
\begin{pgfscope}%
\pgfsys@transformshift{1.828980in}{2.849537in}%
\pgfsys@useobject{currentmarker}{}%
\end{pgfscope}%
\end{pgfscope}%
\begin{pgfscope}%
\pgftext[x=1.828980in,y=2.752315in,,top]{\rmfamily\fontsize{12.000000}{14.400000}\selectfont \(\displaystyle 0.5\)}%
\end{pgfscope}%
\begin{pgfscope}%
\pgfsetbuttcap%
\pgfsetroundjoin%
\definecolor{currentfill}{rgb}{0.000000,0.000000,0.000000}%
\pgfsetfillcolor{currentfill}%
\pgfsetlinewidth{0.803000pt}%
\definecolor{currentstroke}{rgb}{0.000000,0.000000,0.000000}%
\pgfsetstrokecolor{currentstroke}%
\pgfsetdash{}{0pt}%
\pgfsys@defobject{currentmarker}{\pgfqpoint{0.000000in}{-0.048611in}}{\pgfqpoint{0.000000in}{0.000000in}}{%
\pgfpathmoveto{\pgfqpoint{0.000000in}{0.000000in}}%
\pgfpathlineto{\pgfqpoint{0.000000in}{-0.048611in}}%
\pgfusepath{stroke,fill}%
}%
\begin{pgfscope}%
\pgfsys@transformshift{2.777959in}{2.849537in}%
\pgfsys@useobject{currentmarker}{}%
\end{pgfscope}%
\end{pgfscope}%
\begin{pgfscope}%
\pgftext[x=2.777959in,y=2.752315in,,top]{\rmfamily\fontsize{12.000000}{14.400000}\selectfont \(\displaystyle 1.0\)}%
\end{pgfscope}%
\begin{pgfscope}%
\pgfsetbuttcap%
\pgfsetroundjoin%
\definecolor{currentfill}{rgb}{0.000000,0.000000,0.000000}%
\pgfsetfillcolor{currentfill}%
\pgfsetlinewidth{0.803000pt}%
\definecolor{currentstroke}{rgb}{0.000000,0.000000,0.000000}%
\pgfsetstrokecolor{currentstroke}%
\pgfsetdash{}{0pt}%
\pgfsys@defobject{currentmarker}{\pgfqpoint{-0.048611in}{0.000000in}}{\pgfqpoint{0.000000in}{0.000000in}}{%
\pgfpathmoveto{\pgfqpoint{0.000000in}{0.000000in}}%
\pgfpathlineto{\pgfqpoint{-0.048611in}{0.000000in}}%
\pgfusepath{stroke,fill}%
}%
\begin{pgfscope}%
\pgfsys@transformshift{0.880000in}{2.849537in}%
\pgfsys@useobject{currentmarker}{}%
\end{pgfscope}%
\end{pgfscope}%
\begin{pgfscope}%
\pgftext[x=0.492657in,y=2.786223in,left,base]{\rmfamily\fontsize{12.000000}{14.400000}\selectfont \(\displaystyle 0.00\)}%
\end{pgfscope}%
\begin{pgfscope}%
\pgfsetbuttcap%
\pgfsetroundjoin%
\definecolor{currentfill}{rgb}{0.000000,0.000000,0.000000}%
\pgfsetfillcolor{currentfill}%
\pgfsetlinewidth{0.803000pt}%
\definecolor{currentstroke}{rgb}{0.000000,0.000000,0.000000}%
\pgfsetstrokecolor{currentstroke}%
\pgfsetdash{}{0pt}%
\pgfsys@defobject{currentmarker}{\pgfqpoint{-0.048611in}{0.000000in}}{\pgfqpoint{0.000000in}{0.000000in}}{%
\pgfpathmoveto{\pgfqpoint{0.000000in}{0.000000in}}%
\pgfpathlineto{\pgfqpoint{-0.048611in}{0.000000in}}%
\pgfusepath{stroke,fill}%
}%
\begin{pgfscope}%
\pgfsys@transformshift{0.880000in}{3.192719in}%
\pgfsys@useobject{currentmarker}{}%
\end{pgfscope}%
\end{pgfscope}%
\begin{pgfscope}%
\pgftext[x=0.492657in,y=3.129405in,left,base]{\rmfamily\fontsize{12.000000}{14.400000}\selectfont \(\displaystyle 0.25\)}%
\end{pgfscope}%
\begin{pgfscope}%
\pgfsetbuttcap%
\pgfsetroundjoin%
\definecolor{currentfill}{rgb}{0.000000,0.000000,0.000000}%
\pgfsetfillcolor{currentfill}%
\pgfsetlinewidth{0.803000pt}%
\definecolor{currentstroke}{rgb}{0.000000,0.000000,0.000000}%
\pgfsetstrokecolor{currentstroke}%
\pgfsetdash{}{0pt}%
\pgfsys@defobject{currentmarker}{\pgfqpoint{-0.048611in}{0.000000in}}{\pgfqpoint{0.000000in}{0.000000in}}{%
\pgfpathmoveto{\pgfqpoint{0.000000in}{0.000000in}}%
\pgfpathlineto{\pgfqpoint{-0.048611in}{0.000000in}}%
\pgfusepath{stroke,fill}%
}%
\begin{pgfscope}%
\pgfsys@transformshift{0.880000in}{3.535900in}%
\pgfsys@useobject{currentmarker}{}%
\end{pgfscope}%
\end{pgfscope}%
\begin{pgfscope}%
\pgftext[x=0.492657in,y=3.472587in,left,base]{\rmfamily\fontsize{12.000000}{14.400000}\selectfont \(\displaystyle 0.50\)}%
\end{pgfscope}%
\begin{pgfscope}%
\pgfsetbuttcap%
\pgfsetroundjoin%
\definecolor{currentfill}{rgb}{0.000000,0.000000,0.000000}%
\pgfsetfillcolor{currentfill}%
\pgfsetlinewidth{0.803000pt}%
\definecolor{currentstroke}{rgb}{0.000000,0.000000,0.000000}%
\pgfsetstrokecolor{currentstroke}%
\pgfsetdash{}{0pt}%
\pgfsys@defobject{currentmarker}{\pgfqpoint{-0.048611in}{0.000000in}}{\pgfqpoint{0.000000in}{0.000000in}}{%
\pgfpathmoveto{\pgfqpoint{0.000000in}{0.000000in}}%
\pgfpathlineto{\pgfqpoint{-0.048611in}{0.000000in}}%
\pgfusepath{stroke,fill}%
}%
\begin{pgfscope}%
\pgfsys@transformshift{0.880000in}{3.879082in}%
\pgfsys@useobject{currentmarker}{}%
\end{pgfscope}%
\end{pgfscope}%
\begin{pgfscope}%
\pgftext[x=0.492657in,y=3.815768in,left,base]{\rmfamily\fontsize{12.000000}{14.400000}\selectfont \(\displaystyle 0.75\)}%
\end{pgfscope}%
\begin{pgfscope}%
\pgfsetbuttcap%
\pgfsetroundjoin%
\definecolor{currentfill}{rgb}{0.000000,0.000000,0.000000}%
\pgfsetfillcolor{currentfill}%
\pgfsetlinewidth{0.803000pt}%
\definecolor{currentstroke}{rgb}{0.000000,0.000000,0.000000}%
\pgfsetstrokecolor{currentstroke}%
\pgfsetdash{}{0pt}%
\pgfsys@defobject{currentmarker}{\pgfqpoint{-0.048611in}{0.000000in}}{\pgfqpoint{0.000000in}{0.000000in}}{%
\pgfpathmoveto{\pgfqpoint{0.000000in}{0.000000in}}%
\pgfpathlineto{\pgfqpoint{-0.048611in}{0.000000in}}%
\pgfusepath{stroke,fill}%
}%
\begin{pgfscope}%
\pgfsys@transformshift{0.880000in}{4.222264in}%
\pgfsys@useobject{currentmarker}{}%
\end{pgfscope}%
\end{pgfscope}%
\begin{pgfscope}%
\pgftext[x=0.492657in,y=4.158950in,left,base]{\rmfamily\fontsize{12.000000}{14.400000}\selectfont \(\displaystyle 1.00\)}%
\end{pgfscope}%
\begin{pgfscope}%
\pgfsetrectcap%
\pgfsetmiterjoin%
\pgfsetlinewidth{0.803000pt}%
\definecolor{currentstroke}{rgb}{0.501961,0.501961,0.501961}%
\pgfsetstrokecolor{currentstroke}%
\pgfsetdash{}{0pt}%
\pgfpathmoveto{\pgfqpoint{0.880000in}{2.849537in}}%
\pgfpathlineto{\pgfqpoint{0.880000in}{4.222264in}}%
\pgfusepath{stroke}%
\end{pgfscope}%
\begin{pgfscope}%
\pgfsetrectcap%
\pgfsetmiterjoin%
\pgfsetlinewidth{0.803000pt}%
\definecolor{currentstroke}{rgb}{0.501961,0.501961,0.501961}%
\pgfsetstrokecolor{currentstroke}%
\pgfsetdash{}{0pt}%
\pgfpathmoveto{\pgfqpoint{2.777959in}{2.849537in}}%
\pgfpathlineto{\pgfqpoint{2.777959in}{4.222264in}}%
\pgfusepath{stroke}%
\end{pgfscope}%
\begin{pgfscope}%
\pgfsetrectcap%
\pgfsetmiterjoin%
\pgfsetlinewidth{0.803000pt}%
\definecolor{currentstroke}{rgb}{0.501961,0.501961,0.501961}%
\pgfsetstrokecolor{currentstroke}%
\pgfsetdash{}{0pt}%
\pgfpathmoveto{\pgfqpoint{0.880000in}{2.849537in}}%
\pgfpathlineto{\pgfqpoint{2.777959in}{2.849537in}}%
\pgfusepath{stroke}%
\end{pgfscope}%
\begin{pgfscope}%
\pgfsetrectcap%
\pgfsetmiterjoin%
\pgfsetlinewidth{0.803000pt}%
\definecolor{currentstroke}{rgb}{0.501961,0.501961,0.501961}%
\pgfsetstrokecolor{currentstroke}%
\pgfsetdash{}{0pt}%
\pgfpathmoveto{\pgfqpoint{0.880000in}{4.222264in}}%
\pgfpathlineto{\pgfqpoint{2.777959in}{4.222264in}}%
\pgfusepath{stroke}%
\end{pgfscope}%
\begin{pgfscope}%
\pgfsetbuttcap%
\pgfsetmiterjoin%
\definecolor{currentfill}{rgb}{1.000000,1.000000,1.000000}%
\pgfsetfillcolor{currentfill}%
\pgfsetlinewidth{0.000000pt}%
\definecolor{currentstroke}{rgb}{0.000000,0.000000,0.000000}%
\pgfsetstrokecolor{currentstroke}%
\pgfsetstrokeopacity{0.000000}%
\pgfsetdash{}{0pt}%
\pgfpathmoveto{\pgfqpoint{3.347347in}{2.849537in}}%
\pgfpathlineto{\pgfqpoint{5.245306in}{2.849537in}}%
\pgfpathlineto{\pgfqpoint{5.245306in}{4.222264in}}%
\pgfpathlineto{\pgfqpoint{3.347347in}{4.222264in}}%
\pgfpathclose%
\pgfusepath{fill}%
\end{pgfscope}%
\begin{pgfscope}%
\pgfsetbuttcap%
\pgfsetroundjoin%
\definecolor{currentfill}{rgb}{0.000000,0.000000,0.000000}%
\pgfsetfillcolor{currentfill}%
\pgfsetlinewidth{0.803000pt}%
\definecolor{currentstroke}{rgb}{0.000000,0.000000,0.000000}%
\pgfsetstrokecolor{currentstroke}%
\pgfsetdash{}{0pt}%
\pgfsys@defobject{currentmarker}{\pgfqpoint{0.000000in}{-0.048611in}}{\pgfqpoint{0.000000in}{0.000000in}}{%
\pgfpathmoveto{\pgfqpoint{0.000000in}{0.000000in}}%
\pgfpathlineto{\pgfqpoint{0.000000in}{-0.048611in}}%
\pgfusepath{stroke,fill}%
}%
\begin{pgfscope}%
\pgfsys@transformshift{3.347347in}{2.849537in}%
\pgfsys@useobject{currentmarker}{}%
\end{pgfscope}%
\end{pgfscope}%
\begin{pgfscope}%
\pgftext[x=3.347347in,y=2.752315in,,top]{\rmfamily\fontsize{12.000000}{14.400000}\selectfont \(\displaystyle 0.0\)}%
\end{pgfscope}%
\begin{pgfscope}%
\pgfsetbuttcap%
\pgfsetroundjoin%
\definecolor{currentfill}{rgb}{0.000000,0.000000,0.000000}%
\pgfsetfillcolor{currentfill}%
\pgfsetlinewidth{0.803000pt}%
\definecolor{currentstroke}{rgb}{0.000000,0.000000,0.000000}%
\pgfsetstrokecolor{currentstroke}%
\pgfsetdash{}{0pt}%
\pgfsys@defobject{currentmarker}{\pgfqpoint{0.000000in}{-0.048611in}}{\pgfqpoint{0.000000in}{0.000000in}}{%
\pgfpathmoveto{\pgfqpoint{0.000000in}{0.000000in}}%
\pgfpathlineto{\pgfqpoint{0.000000in}{-0.048611in}}%
\pgfusepath{stroke,fill}%
}%
\begin{pgfscope}%
\pgfsys@transformshift{4.296327in}{2.849537in}%
\pgfsys@useobject{currentmarker}{}%
\end{pgfscope}%
\end{pgfscope}%
\begin{pgfscope}%
\pgftext[x=4.296327in,y=2.752315in,,top]{\rmfamily\fontsize{12.000000}{14.400000}\selectfont \(\displaystyle 0.5\)}%
\end{pgfscope}%
\begin{pgfscope}%
\pgfsetbuttcap%
\pgfsetroundjoin%
\definecolor{currentfill}{rgb}{0.000000,0.000000,0.000000}%
\pgfsetfillcolor{currentfill}%
\pgfsetlinewidth{0.803000pt}%
\definecolor{currentstroke}{rgb}{0.000000,0.000000,0.000000}%
\pgfsetstrokecolor{currentstroke}%
\pgfsetdash{}{0pt}%
\pgfsys@defobject{currentmarker}{\pgfqpoint{0.000000in}{-0.048611in}}{\pgfqpoint{0.000000in}{0.000000in}}{%
\pgfpathmoveto{\pgfqpoint{0.000000in}{0.000000in}}%
\pgfpathlineto{\pgfqpoint{0.000000in}{-0.048611in}}%
\pgfusepath{stroke,fill}%
}%
\begin{pgfscope}%
\pgfsys@transformshift{5.245306in}{2.849537in}%
\pgfsys@useobject{currentmarker}{}%
\end{pgfscope}%
\end{pgfscope}%
\begin{pgfscope}%
\pgftext[x=5.245306in,y=2.752315in,,top]{\rmfamily\fontsize{12.000000}{14.400000}\selectfont \(\displaystyle 1.0\)}%
\end{pgfscope}%
\begin{pgfscope}%
\pgfsetbuttcap%
\pgfsetroundjoin%
\definecolor{currentfill}{rgb}{0.000000,0.000000,0.000000}%
\pgfsetfillcolor{currentfill}%
\pgfsetlinewidth{0.803000pt}%
\definecolor{currentstroke}{rgb}{0.000000,0.000000,0.000000}%
\pgfsetstrokecolor{currentstroke}%
\pgfsetdash{}{0pt}%
\pgfsys@defobject{currentmarker}{\pgfqpoint{-0.048611in}{0.000000in}}{\pgfqpoint{0.000000in}{0.000000in}}{%
\pgfpathmoveto{\pgfqpoint{0.000000in}{0.000000in}}%
\pgfpathlineto{\pgfqpoint{-0.048611in}{0.000000in}}%
\pgfusepath{stroke,fill}%
}%
\begin{pgfscope}%
\pgfsys@transformshift{3.347347in}{2.849537in}%
\pgfsys@useobject{currentmarker}{}%
\end{pgfscope}%
\end{pgfscope}%
\begin{pgfscope}%
\pgftext[x=2.960004in,y=2.786223in,left,base]{\rmfamily\fontsize{12.000000}{14.400000}\selectfont \(\displaystyle 0.00\)}%
\end{pgfscope}%
\begin{pgfscope}%
\pgfsetbuttcap%
\pgfsetroundjoin%
\definecolor{currentfill}{rgb}{0.000000,0.000000,0.000000}%
\pgfsetfillcolor{currentfill}%
\pgfsetlinewidth{0.803000pt}%
\definecolor{currentstroke}{rgb}{0.000000,0.000000,0.000000}%
\pgfsetstrokecolor{currentstroke}%
\pgfsetdash{}{0pt}%
\pgfsys@defobject{currentmarker}{\pgfqpoint{-0.048611in}{0.000000in}}{\pgfqpoint{0.000000in}{0.000000in}}{%
\pgfpathmoveto{\pgfqpoint{0.000000in}{0.000000in}}%
\pgfpathlineto{\pgfqpoint{-0.048611in}{0.000000in}}%
\pgfusepath{stroke,fill}%
}%
\begin{pgfscope}%
\pgfsys@transformshift{3.347347in}{3.192719in}%
\pgfsys@useobject{currentmarker}{}%
\end{pgfscope}%
\end{pgfscope}%
\begin{pgfscope}%
\pgftext[x=2.960004in,y=3.129405in,left,base]{\rmfamily\fontsize{12.000000}{14.400000}\selectfont \(\displaystyle 0.25\)}%
\end{pgfscope}%
\begin{pgfscope}%
\pgfsetbuttcap%
\pgfsetroundjoin%
\definecolor{currentfill}{rgb}{0.000000,0.000000,0.000000}%
\pgfsetfillcolor{currentfill}%
\pgfsetlinewidth{0.803000pt}%
\definecolor{currentstroke}{rgb}{0.000000,0.000000,0.000000}%
\pgfsetstrokecolor{currentstroke}%
\pgfsetdash{}{0pt}%
\pgfsys@defobject{currentmarker}{\pgfqpoint{-0.048611in}{0.000000in}}{\pgfqpoint{0.000000in}{0.000000in}}{%
\pgfpathmoveto{\pgfqpoint{0.000000in}{0.000000in}}%
\pgfpathlineto{\pgfqpoint{-0.048611in}{0.000000in}}%
\pgfusepath{stroke,fill}%
}%
\begin{pgfscope}%
\pgfsys@transformshift{3.347347in}{3.535900in}%
\pgfsys@useobject{currentmarker}{}%
\end{pgfscope}%
\end{pgfscope}%
\begin{pgfscope}%
\pgftext[x=2.960004in,y=3.472587in,left,base]{\rmfamily\fontsize{12.000000}{14.400000}\selectfont \(\displaystyle 0.50\)}%
\end{pgfscope}%
\begin{pgfscope}%
\pgfsetbuttcap%
\pgfsetroundjoin%
\definecolor{currentfill}{rgb}{0.000000,0.000000,0.000000}%
\pgfsetfillcolor{currentfill}%
\pgfsetlinewidth{0.803000pt}%
\definecolor{currentstroke}{rgb}{0.000000,0.000000,0.000000}%
\pgfsetstrokecolor{currentstroke}%
\pgfsetdash{}{0pt}%
\pgfsys@defobject{currentmarker}{\pgfqpoint{-0.048611in}{0.000000in}}{\pgfqpoint{0.000000in}{0.000000in}}{%
\pgfpathmoveto{\pgfqpoint{0.000000in}{0.000000in}}%
\pgfpathlineto{\pgfqpoint{-0.048611in}{0.000000in}}%
\pgfusepath{stroke,fill}%
}%
\begin{pgfscope}%
\pgfsys@transformshift{3.347347in}{3.879082in}%
\pgfsys@useobject{currentmarker}{}%
\end{pgfscope}%
\end{pgfscope}%
\begin{pgfscope}%
\pgftext[x=2.960004in,y=3.815768in,left,base]{\rmfamily\fontsize{12.000000}{14.400000}\selectfont \(\displaystyle 0.75\)}%
\end{pgfscope}%
\begin{pgfscope}%
\pgfsetbuttcap%
\pgfsetroundjoin%
\definecolor{currentfill}{rgb}{0.000000,0.000000,0.000000}%
\pgfsetfillcolor{currentfill}%
\pgfsetlinewidth{0.803000pt}%
\definecolor{currentstroke}{rgb}{0.000000,0.000000,0.000000}%
\pgfsetstrokecolor{currentstroke}%
\pgfsetdash{}{0pt}%
\pgfsys@defobject{currentmarker}{\pgfqpoint{-0.048611in}{0.000000in}}{\pgfqpoint{0.000000in}{0.000000in}}{%
\pgfpathmoveto{\pgfqpoint{0.000000in}{0.000000in}}%
\pgfpathlineto{\pgfqpoint{-0.048611in}{0.000000in}}%
\pgfusepath{stroke,fill}%
}%
\begin{pgfscope}%
\pgfsys@transformshift{3.347347in}{4.222264in}%
\pgfsys@useobject{currentmarker}{}%
\end{pgfscope}%
\end{pgfscope}%
\begin{pgfscope}%
\pgftext[x=2.960004in,y=4.158950in,left,base]{\rmfamily\fontsize{12.000000}{14.400000}\selectfont \(\displaystyle 1.00\)}%
\end{pgfscope}%
\begin{pgfscope}%
\pgfsetrectcap%
\pgfsetmiterjoin%
\pgfsetlinewidth{0.803000pt}%
\definecolor{currentstroke}{rgb}{0.501961,0.501961,0.501961}%
\pgfsetstrokecolor{currentstroke}%
\pgfsetdash{}{0pt}%
\pgfpathmoveto{\pgfqpoint{3.347347in}{2.849537in}}%
\pgfpathlineto{\pgfqpoint{3.347347in}{4.222264in}}%
\pgfusepath{stroke}%
\end{pgfscope}%
\begin{pgfscope}%
\pgfsetrectcap%
\pgfsetmiterjoin%
\pgfsetlinewidth{0.803000pt}%
\definecolor{currentstroke}{rgb}{0.501961,0.501961,0.501961}%
\pgfsetstrokecolor{currentstroke}%
\pgfsetdash{}{0pt}%
\pgfpathmoveto{\pgfqpoint{5.245306in}{2.849537in}}%
\pgfpathlineto{\pgfqpoint{5.245306in}{4.222264in}}%
\pgfusepath{stroke}%
\end{pgfscope}%
\begin{pgfscope}%
\pgfsetrectcap%
\pgfsetmiterjoin%
\pgfsetlinewidth{0.803000pt}%
\definecolor{currentstroke}{rgb}{0.501961,0.501961,0.501961}%
\pgfsetstrokecolor{currentstroke}%
\pgfsetdash{}{0pt}%
\pgfpathmoveto{\pgfqpoint{3.347347in}{2.849537in}}%
\pgfpathlineto{\pgfqpoint{5.245306in}{2.849537in}}%
\pgfusepath{stroke}%
\end{pgfscope}%
\begin{pgfscope}%
\pgfsetrectcap%
\pgfsetmiterjoin%
\pgfsetlinewidth{0.803000pt}%
\definecolor{currentstroke}{rgb}{0.501961,0.501961,0.501961}%
\pgfsetstrokecolor{currentstroke}%
\pgfsetdash{}{0pt}%
\pgfpathmoveto{\pgfqpoint{3.347347in}{4.222264in}}%
\pgfpathlineto{\pgfqpoint{5.245306in}{4.222264in}}%
\pgfusepath{stroke}%
\end{pgfscope}%
\begin{pgfscope}%
\pgfsetbuttcap%
\pgfsetmiterjoin%
\definecolor{currentfill}{rgb}{1.000000,1.000000,1.000000}%
\pgfsetfillcolor{currentfill}%
\pgfsetlinewidth{0.000000pt}%
\definecolor{currentstroke}{rgb}{0.000000,0.000000,0.000000}%
\pgfsetstrokecolor{currentstroke}%
\pgfsetstrokeopacity{0.000000}%
\pgfsetdash{}{0pt}%
\pgfpathmoveto{\pgfqpoint{5.814694in}{2.849537in}}%
\pgfpathlineto{\pgfqpoint{7.712653in}{2.849537in}}%
\pgfpathlineto{\pgfqpoint{7.712653in}{4.222264in}}%
\pgfpathlineto{\pgfqpoint{5.814694in}{4.222264in}}%
\pgfpathclose%
\pgfusepath{fill}%
\end{pgfscope}%
\begin{pgfscope}%
\pgfsetbuttcap%
\pgfsetroundjoin%
\definecolor{currentfill}{rgb}{0.000000,0.000000,0.000000}%
\pgfsetfillcolor{currentfill}%
\pgfsetlinewidth{0.803000pt}%
\definecolor{currentstroke}{rgb}{0.000000,0.000000,0.000000}%
\pgfsetstrokecolor{currentstroke}%
\pgfsetdash{}{0pt}%
\pgfsys@defobject{currentmarker}{\pgfqpoint{0.000000in}{-0.048611in}}{\pgfqpoint{0.000000in}{0.000000in}}{%
\pgfpathmoveto{\pgfqpoint{0.000000in}{0.000000in}}%
\pgfpathlineto{\pgfqpoint{0.000000in}{-0.048611in}}%
\pgfusepath{stroke,fill}%
}%
\begin{pgfscope}%
\pgfsys@transformshift{5.814694in}{2.849537in}%
\pgfsys@useobject{currentmarker}{}%
\end{pgfscope}%
\end{pgfscope}%
\begin{pgfscope}%
\pgftext[x=5.814694in,y=2.752315in,,top]{\rmfamily\fontsize{12.000000}{14.400000}\selectfont \(\displaystyle 0.0\)}%
\end{pgfscope}%
\begin{pgfscope}%
\pgfsetbuttcap%
\pgfsetroundjoin%
\definecolor{currentfill}{rgb}{0.000000,0.000000,0.000000}%
\pgfsetfillcolor{currentfill}%
\pgfsetlinewidth{0.803000pt}%
\definecolor{currentstroke}{rgb}{0.000000,0.000000,0.000000}%
\pgfsetstrokecolor{currentstroke}%
\pgfsetdash{}{0pt}%
\pgfsys@defobject{currentmarker}{\pgfqpoint{0.000000in}{-0.048611in}}{\pgfqpoint{0.000000in}{0.000000in}}{%
\pgfpathmoveto{\pgfqpoint{0.000000in}{0.000000in}}%
\pgfpathlineto{\pgfqpoint{0.000000in}{-0.048611in}}%
\pgfusepath{stroke,fill}%
}%
\begin{pgfscope}%
\pgfsys@transformshift{6.763673in}{2.849537in}%
\pgfsys@useobject{currentmarker}{}%
\end{pgfscope}%
\end{pgfscope}%
\begin{pgfscope}%
\pgftext[x=6.763673in,y=2.752315in,,top]{\rmfamily\fontsize{12.000000}{14.400000}\selectfont \(\displaystyle 0.5\)}%
\end{pgfscope}%
\begin{pgfscope}%
\pgfsetbuttcap%
\pgfsetroundjoin%
\definecolor{currentfill}{rgb}{0.000000,0.000000,0.000000}%
\pgfsetfillcolor{currentfill}%
\pgfsetlinewidth{0.803000pt}%
\definecolor{currentstroke}{rgb}{0.000000,0.000000,0.000000}%
\pgfsetstrokecolor{currentstroke}%
\pgfsetdash{}{0pt}%
\pgfsys@defobject{currentmarker}{\pgfqpoint{0.000000in}{-0.048611in}}{\pgfqpoint{0.000000in}{0.000000in}}{%
\pgfpathmoveto{\pgfqpoint{0.000000in}{0.000000in}}%
\pgfpathlineto{\pgfqpoint{0.000000in}{-0.048611in}}%
\pgfusepath{stroke,fill}%
}%
\begin{pgfscope}%
\pgfsys@transformshift{7.712653in}{2.849537in}%
\pgfsys@useobject{currentmarker}{}%
\end{pgfscope}%
\end{pgfscope}%
\begin{pgfscope}%
\pgftext[x=7.712653in,y=2.752315in,,top]{\rmfamily\fontsize{12.000000}{14.400000}\selectfont \(\displaystyle 1.0\)}%
\end{pgfscope}%
\begin{pgfscope}%
\pgfsetbuttcap%
\pgfsetroundjoin%
\definecolor{currentfill}{rgb}{0.000000,0.000000,0.000000}%
\pgfsetfillcolor{currentfill}%
\pgfsetlinewidth{0.803000pt}%
\definecolor{currentstroke}{rgb}{0.000000,0.000000,0.000000}%
\pgfsetstrokecolor{currentstroke}%
\pgfsetdash{}{0pt}%
\pgfsys@defobject{currentmarker}{\pgfqpoint{-0.048611in}{0.000000in}}{\pgfqpoint{0.000000in}{0.000000in}}{%
\pgfpathmoveto{\pgfqpoint{0.000000in}{0.000000in}}%
\pgfpathlineto{\pgfqpoint{-0.048611in}{0.000000in}}%
\pgfusepath{stroke,fill}%
}%
\begin{pgfscope}%
\pgfsys@transformshift{5.814694in}{2.849537in}%
\pgfsys@useobject{currentmarker}{}%
\end{pgfscope}%
\end{pgfscope}%
\begin{pgfscope}%
\pgftext[x=5.427351in,y=2.786223in,left,base]{\rmfamily\fontsize{12.000000}{14.400000}\selectfont \(\displaystyle 0.00\)}%
\end{pgfscope}%
\begin{pgfscope}%
\pgfsetbuttcap%
\pgfsetroundjoin%
\definecolor{currentfill}{rgb}{0.000000,0.000000,0.000000}%
\pgfsetfillcolor{currentfill}%
\pgfsetlinewidth{0.803000pt}%
\definecolor{currentstroke}{rgb}{0.000000,0.000000,0.000000}%
\pgfsetstrokecolor{currentstroke}%
\pgfsetdash{}{0pt}%
\pgfsys@defobject{currentmarker}{\pgfqpoint{-0.048611in}{0.000000in}}{\pgfqpoint{0.000000in}{0.000000in}}{%
\pgfpathmoveto{\pgfqpoint{0.000000in}{0.000000in}}%
\pgfpathlineto{\pgfqpoint{-0.048611in}{0.000000in}}%
\pgfusepath{stroke,fill}%
}%
\begin{pgfscope}%
\pgfsys@transformshift{5.814694in}{3.192719in}%
\pgfsys@useobject{currentmarker}{}%
\end{pgfscope}%
\end{pgfscope}%
\begin{pgfscope}%
\pgftext[x=5.427351in,y=3.129405in,left,base]{\rmfamily\fontsize{12.000000}{14.400000}\selectfont \(\displaystyle 0.25\)}%
\end{pgfscope}%
\begin{pgfscope}%
\pgfsetbuttcap%
\pgfsetroundjoin%
\definecolor{currentfill}{rgb}{0.000000,0.000000,0.000000}%
\pgfsetfillcolor{currentfill}%
\pgfsetlinewidth{0.803000pt}%
\definecolor{currentstroke}{rgb}{0.000000,0.000000,0.000000}%
\pgfsetstrokecolor{currentstroke}%
\pgfsetdash{}{0pt}%
\pgfsys@defobject{currentmarker}{\pgfqpoint{-0.048611in}{0.000000in}}{\pgfqpoint{0.000000in}{0.000000in}}{%
\pgfpathmoveto{\pgfqpoint{0.000000in}{0.000000in}}%
\pgfpathlineto{\pgfqpoint{-0.048611in}{0.000000in}}%
\pgfusepath{stroke,fill}%
}%
\begin{pgfscope}%
\pgfsys@transformshift{5.814694in}{3.535900in}%
\pgfsys@useobject{currentmarker}{}%
\end{pgfscope}%
\end{pgfscope}%
\begin{pgfscope}%
\pgftext[x=5.427351in,y=3.472587in,left,base]{\rmfamily\fontsize{12.000000}{14.400000}\selectfont \(\displaystyle 0.50\)}%
\end{pgfscope}%
\begin{pgfscope}%
\pgfsetbuttcap%
\pgfsetroundjoin%
\definecolor{currentfill}{rgb}{0.000000,0.000000,0.000000}%
\pgfsetfillcolor{currentfill}%
\pgfsetlinewidth{0.803000pt}%
\definecolor{currentstroke}{rgb}{0.000000,0.000000,0.000000}%
\pgfsetstrokecolor{currentstroke}%
\pgfsetdash{}{0pt}%
\pgfsys@defobject{currentmarker}{\pgfqpoint{-0.048611in}{0.000000in}}{\pgfqpoint{0.000000in}{0.000000in}}{%
\pgfpathmoveto{\pgfqpoint{0.000000in}{0.000000in}}%
\pgfpathlineto{\pgfqpoint{-0.048611in}{0.000000in}}%
\pgfusepath{stroke,fill}%
}%
\begin{pgfscope}%
\pgfsys@transformshift{5.814694in}{3.879082in}%
\pgfsys@useobject{currentmarker}{}%
\end{pgfscope}%
\end{pgfscope}%
\begin{pgfscope}%
\pgftext[x=5.427351in,y=3.815768in,left,base]{\rmfamily\fontsize{12.000000}{14.400000}\selectfont \(\displaystyle 0.75\)}%
\end{pgfscope}%
\begin{pgfscope}%
\pgfsetbuttcap%
\pgfsetroundjoin%
\definecolor{currentfill}{rgb}{0.000000,0.000000,0.000000}%
\pgfsetfillcolor{currentfill}%
\pgfsetlinewidth{0.803000pt}%
\definecolor{currentstroke}{rgb}{0.000000,0.000000,0.000000}%
\pgfsetstrokecolor{currentstroke}%
\pgfsetdash{}{0pt}%
\pgfsys@defobject{currentmarker}{\pgfqpoint{-0.048611in}{0.000000in}}{\pgfqpoint{0.000000in}{0.000000in}}{%
\pgfpathmoveto{\pgfqpoint{0.000000in}{0.000000in}}%
\pgfpathlineto{\pgfqpoint{-0.048611in}{0.000000in}}%
\pgfusepath{stroke,fill}%
}%
\begin{pgfscope}%
\pgfsys@transformshift{5.814694in}{4.222264in}%
\pgfsys@useobject{currentmarker}{}%
\end{pgfscope}%
\end{pgfscope}%
\begin{pgfscope}%
\pgftext[x=5.427351in,y=4.158950in,left,base]{\rmfamily\fontsize{12.000000}{14.400000}\selectfont \(\displaystyle 1.00\)}%
\end{pgfscope}%
\begin{pgfscope}%
\pgfsetrectcap%
\pgfsetmiterjoin%
\pgfsetlinewidth{0.803000pt}%
\definecolor{currentstroke}{rgb}{0.501961,0.501961,0.501961}%
\pgfsetstrokecolor{currentstroke}%
\pgfsetdash{}{0pt}%
\pgfpathmoveto{\pgfqpoint{5.814694in}{2.849537in}}%
\pgfpathlineto{\pgfqpoint{5.814694in}{4.222264in}}%
\pgfusepath{stroke}%
\end{pgfscope}%
\begin{pgfscope}%
\pgfsetrectcap%
\pgfsetmiterjoin%
\pgfsetlinewidth{0.803000pt}%
\definecolor{currentstroke}{rgb}{0.501961,0.501961,0.501961}%
\pgfsetstrokecolor{currentstroke}%
\pgfsetdash{}{0pt}%
\pgfpathmoveto{\pgfqpoint{7.712653in}{2.849537in}}%
\pgfpathlineto{\pgfqpoint{7.712653in}{4.222264in}}%
\pgfusepath{stroke}%
\end{pgfscope}%
\begin{pgfscope}%
\pgfsetrectcap%
\pgfsetmiterjoin%
\pgfsetlinewidth{0.803000pt}%
\definecolor{currentstroke}{rgb}{0.501961,0.501961,0.501961}%
\pgfsetstrokecolor{currentstroke}%
\pgfsetdash{}{0pt}%
\pgfpathmoveto{\pgfqpoint{5.814694in}{2.849537in}}%
\pgfpathlineto{\pgfqpoint{7.712653in}{2.849537in}}%
\pgfusepath{stroke}%
\end{pgfscope}%
\begin{pgfscope}%
\pgfsetrectcap%
\pgfsetmiterjoin%
\pgfsetlinewidth{0.803000pt}%
\definecolor{currentstroke}{rgb}{0.501961,0.501961,0.501961}%
\pgfsetstrokecolor{currentstroke}%
\pgfsetdash{}{0pt}%
\pgfpathmoveto{\pgfqpoint{5.814694in}{4.222264in}}%
\pgfpathlineto{\pgfqpoint{7.712653in}{4.222264in}}%
\pgfusepath{stroke}%
\end{pgfscope}%
\begin{pgfscope}%
\pgfsetbuttcap%
\pgfsetmiterjoin%
\definecolor{currentfill}{rgb}{1.000000,1.000000,1.000000}%
\pgfsetfillcolor{currentfill}%
\pgfsetlinewidth{0.000000pt}%
\definecolor{currentstroke}{rgb}{0.000000,0.000000,0.000000}%
\pgfsetstrokecolor{currentstroke}%
\pgfsetstrokeopacity{0.000000}%
\pgfsetdash{}{0pt}%
\pgfpathmoveto{\pgfqpoint{8.282041in}{2.849537in}}%
\pgfpathlineto{\pgfqpoint{10.180000in}{2.849537in}}%
\pgfpathlineto{\pgfqpoint{10.180000in}{4.222264in}}%
\pgfpathlineto{\pgfqpoint{8.282041in}{4.222264in}}%
\pgfpathclose%
\pgfusepath{fill}%
\end{pgfscope}%
\begin{pgfscope}%
\pgfsetbuttcap%
\pgfsetroundjoin%
\definecolor{currentfill}{rgb}{0.000000,0.000000,0.000000}%
\pgfsetfillcolor{currentfill}%
\pgfsetlinewidth{0.803000pt}%
\definecolor{currentstroke}{rgb}{0.000000,0.000000,0.000000}%
\pgfsetstrokecolor{currentstroke}%
\pgfsetdash{}{0pt}%
\pgfsys@defobject{currentmarker}{\pgfqpoint{0.000000in}{-0.048611in}}{\pgfqpoint{0.000000in}{0.000000in}}{%
\pgfpathmoveto{\pgfqpoint{0.000000in}{0.000000in}}%
\pgfpathlineto{\pgfqpoint{0.000000in}{-0.048611in}}%
\pgfusepath{stroke,fill}%
}%
\begin{pgfscope}%
\pgfsys@transformshift{8.282041in}{2.849537in}%
\pgfsys@useobject{currentmarker}{}%
\end{pgfscope}%
\end{pgfscope}%
\begin{pgfscope}%
\pgftext[x=8.282041in,y=2.752315in,,top]{\rmfamily\fontsize{12.000000}{14.400000}\selectfont \(\displaystyle 0.0\)}%
\end{pgfscope}%
\begin{pgfscope}%
\pgfsetbuttcap%
\pgfsetroundjoin%
\definecolor{currentfill}{rgb}{0.000000,0.000000,0.000000}%
\pgfsetfillcolor{currentfill}%
\pgfsetlinewidth{0.803000pt}%
\definecolor{currentstroke}{rgb}{0.000000,0.000000,0.000000}%
\pgfsetstrokecolor{currentstroke}%
\pgfsetdash{}{0pt}%
\pgfsys@defobject{currentmarker}{\pgfqpoint{0.000000in}{-0.048611in}}{\pgfqpoint{0.000000in}{0.000000in}}{%
\pgfpathmoveto{\pgfqpoint{0.000000in}{0.000000in}}%
\pgfpathlineto{\pgfqpoint{0.000000in}{-0.048611in}}%
\pgfusepath{stroke,fill}%
}%
\begin{pgfscope}%
\pgfsys@transformshift{9.231020in}{2.849537in}%
\pgfsys@useobject{currentmarker}{}%
\end{pgfscope}%
\end{pgfscope}%
\begin{pgfscope}%
\pgftext[x=9.231020in,y=2.752315in,,top]{\rmfamily\fontsize{12.000000}{14.400000}\selectfont \(\displaystyle 0.5\)}%
\end{pgfscope}%
\begin{pgfscope}%
\pgfsetbuttcap%
\pgfsetroundjoin%
\definecolor{currentfill}{rgb}{0.000000,0.000000,0.000000}%
\pgfsetfillcolor{currentfill}%
\pgfsetlinewidth{0.803000pt}%
\definecolor{currentstroke}{rgb}{0.000000,0.000000,0.000000}%
\pgfsetstrokecolor{currentstroke}%
\pgfsetdash{}{0pt}%
\pgfsys@defobject{currentmarker}{\pgfqpoint{0.000000in}{-0.048611in}}{\pgfqpoint{0.000000in}{0.000000in}}{%
\pgfpathmoveto{\pgfqpoint{0.000000in}{0.000000in}}%
\pgfpathlineto{\pgfqpoint{0.000000in}{-0.048611in}}%
\pgfusepath{stroke,fill}%
}%
\begin{pgfscope}%
\pgfsys@transformshift{10.180000in}{2.849537in}%
\pgfsys@useobject{currentmarker}{}%
\end{pgfscope}%
\end{pgfscope}%
\begin{pgfscope}%
\pgftext[x=10.180000in,y=2.752315in,,top]{\rmfamily\fontsize{12.000000}{14.400000}\selectfont \(\displaystyle 1.0\)}%
\end{pgfscope}%
\begin{pgfscope}%
\pgfsetbuttcap%
\pgfsetroundjoin%
\definecolor{currentfill}{rgb}{0.000000,0.000000,0.000000}%
\pgfsetfillcolor{currentfill}%
\pgfsetlinewidth{0.803000pt}%
\definecolor{currentstroke}{rgb}{0.000000,0.000000,0.000000}%
\pgfsetstrokecolor{currentstroke}%
\pgfsetdash{}{0pt}%
\pgfsys@defobject{currentmarker}{\pgfqpoint{-0.048611in}{0.000000in}}{\pgfqpoint{0.000000in}{0.000000in}}{%
\pgfpathmoveto{\pgfqpoint{0.000000in}{0.000000in}}%
\pgfpathlineto{\pgfqpoint{-0.048611in}{0.000000in}}%
\pgfusepath{stroke,fill}%
}%
\begin{pgfscope}%
\pgfsys@transformshift{8.282041in}{2.849537in}%
\pgfsys@useobject{currentmarker}{}%
\end{pgfscope}%
\end{pgfscope}%
\begin{pgfscope}%
\pgftext[x=7.894698in,y=2.786223in,left,base]{\rmfamily\fontsize{12.000000}{14.400000}\selectfont \(\displaystyle 0.00\)}%
\end{pgfscope}%
\begin{pgfscope}%
\pgfsetbuttcap%
\pgfsetroundjoin%
\definecolor{currentfill}{rgb}{0.000000,0.000000,0.000000}%
\pgfsetfillcolor{currentfill}%
\pgfsetlinewidth{0.803000pt}%
\definecolor{currentstroke}{rgb}{0.000000,0.000000,0.000000}%
\pgfsetstrokecolor{currentstroke}%
\pgfsetdash{}{0pt}%
\pgfsys@defobject{currentmarker}{\pgfqpoint{-0.048611in}{0.000000in}}{\pgfqpoint{0.000000in}{0.000000in}}{%
\pgfpathmoveto{\pgfqpoint{0.000000in}{0.000000in}}%
\pgfpathlineto{\pgfqpoint{-0.048611in}{0.000000in}}%
\pgfusepath{stroke,fill}%
}%
\begin{pgfscope}%
\pgfsys@transformshift{8.282041in}{3.192719in}%
\pgfsys@useobject{currentmarker}{}%
\end{pgfscope}%
\end{pgfscope}%
\begin{pgfscope}%
\pgftext[x=7.894698in,y=3.129405in,left,base]{\rmfamily\fontsize{12.000000}{14.400000}\selectfont \(\displaystyle 0.25\)}%
\end{pgfscope}%
\begin{pgfscope}%
\pgfsetbuttcap%
\pgfsetroundjoin%
\definecolor{currentfill}{rgb}{0.000000,0.000000,0.000000}%
\pgfsetfillcolor{currentfill}%
\pgfsetlinewidth{0.803000pt}%
\definecolor{currentstroke}{rgb}{0.000000,0.000000,0.000000}%
\pgfsetstrokecolor{currentstroke}%
\pgfsetdash{}{0pt}%
\pgfsys@defobject{currentmarker}{\pgfqpoint{-0.048611in}{0.000000in}}{\pgfqpoint{0.000000in}{0.000000in}}{%
\pgfpathmoveto{\pgfqpoint{0.000000in}{0.000000in}}%
\pgfpathlineto{\pgfqpoint{-0.048611in}{0.000000in}}%
\pgfusepath{stroke,fill}%
}%
\begin{pgfscope}%
\pgfsys@transformshift{8.282041in}{3.535900in}%
\pgfsys@useobject{currentmarker}{}%
\end{pgfscope}%
\end{pgfscope}%
\begin{pgfscope}%
\pgftext[x=7.894698in,y=3.472587in,left,base]{\rmfamily\fontsize{12.000000}{14.400000}\selectfont \(\displaystyle 0.50\)}%
\end{pgfscope}%
\begin{pgfscope}%
\pgfsetbuttcap%
\pgfsetroundjoin%
\definecolor{currentfill}{rgb}{0.000000,0.000000,0.000000}%
\pgfsetfillcolor{currentfill}%
\pgfsetlinewidth{0.803000pt}%
\definecolor{currentstroke}{rgb}{0.000000,0.000000,0.000000}%
\pgfsetstrokecolor{currentstroke}%
\pgfsetdash{}{0pt}%
\pgfsys@defobject{currentmarker}{\pgfqpoint{-0.048611in}{0.000000in}}{\pgfqpoint{0.000000in}{0.000000in}}{%
\pgfpathmoveto{\pgfqpoint{0.000000in}{0.000000in}}%
\pgfpathlineto{\pgfqpoint{-0.048611in}{0.000000in}}%
\pgfusepath{stroke,fill}%
}%
\begin{pgfscope}%
\pgfsys@transformshift{8.282041in}{3.879082in}%
\pgfsys@useobject{currentmarker}{}%
\end{pgfscope}%
\end{pgfscope}%
\begin{pgfscope}%
\pgftext[x=7.894698in,y=3.815768in,left,base]{\rmfamily\fontsize{12.000000}{14.400000}\selectfont \(\displaystyle 0.75\)}%
\end{pgfscope}%
\begin{pgfscope}%
\pgfsetbuttcap%
\pgfsetroundjoin%
\definecolor{currentfill}{rgb}{0.000000,0.000000,0.000000}%
\pgfsetfillcolor{currentfill}%
\pgfsetlinewidth{0.803000pt}%
\definecolor{currentstroke}{rgb}{0.000000,0.000000,0.000000}%
\pgfsetstrokecolor{currentstroke}%
\pgfsetdash{}{0pt}%
\pgfsys@defobject{currentmarker}{\pgfqpoint{-0.048611in}{0.000000in}}{\pgfqpoint{0.000000in}{0.000000in}}{%
\pgfpathmoveto{\pgfqpoint{0.000000in}{0.000000in}}%
\pgfpathlineto{\pgfqpoint{-0.048611in}{0.000000in}}%
\pgfusepath{stroke,fill}%
}%
\begin{pgfscope}%
\pgfsys@transformshift{8.282041in}{4.222264in}%
\pgfsys@useobject{currentmarker}{}%
\end{pgfscope}%
\end{pgfscope}%
\begin{pgfscope}%
\pgftext[x=7.894698in,y=4.158950in,left,base]{\rmfamily\fontsize{12.000000}{14.400000}\selectfont \(\displaystyle 1.00\)}%
\end{pgfscope}%
\begin{pgfscope}%
\pgfsetrectcap%
\pgfsetmiterjoin%
\pgfsetlinewidth{0.803000pt}%
\definecolor{currentstroke}{rgb}{0.501961,0.501961,0.501961}%
\pgfsetstrokecolor{currentstroke}%
\pgfsetdash{}{0pt}%
\pgfpathmoveto{\pgfqpoint{8.282041in}{2.849537in}}%
\pgfpathlineto{\pgfqpoint{8.282041in}{4.222264in}}%
\pgfusepath{stroke}%
\end{pgfscope}%
\begin{pgfscope}%
\pgfsetrectcap%
\pgfsetmiterjoin%
\pgfsetlinewidth{0.803000pt}%
\definecolor{currentstroke}{rgb}{0.501961,0.501961,0.501961}%
\pgfsetstrokecolor{currentstroke}%
\pgfsetdash{}{0pt}%
\pgfpathmoveto{\pgfqpoint{10.180000in}{2.849537in}}%
\pgfpathlineto{\pgfqpoint{10.180000in}{4.222264in}}%
\pgfusepath{stroke}%
\end{pgfscope}%
\begin{pgfscope}%
\pgfsetrectcap%
\pgfsetmiterjoin%
\pgfsetlinewidth{0.803000pt}%
\definecolor{currentstroke}{rgb}{0.501961,0.501961,0.501961}%
\pgfsetstrokecolor{currentstroke}%
\pgfsetdash{}{0pt}%
\pgfpathmoveto{\pgfqpoint{8.282041in}{2.849537in}}%
\pgfpathlineto{\pgfqpoint{10.180000in}{2.849537in}}%
\pgfusepath{stroke}%
\end{pgfscope}%
\begin{pgfscope}%
\pgfsetrectcap%
\pgfsetmiterjoin%
\pgfsetlinewidth{0.803000pt}%
\definecolor{currentstroke}{rgb}{0.501961,0.501961,0.501961}%
\pgfsetstrokecolor{currentstroke}%
\pgfsetdash{}{0pt}%
\pgfpathmoveto{\pgfqpoint{8.282041in}{4.222264in}}%
\pgfpathlineto{\pgfqpoint{10.180000in}{4.222264in}}%
\pgfusepath{stroke}%
\end{pgfscope}%
\begin{pgfscope}%
\pgfsetbuttcap%
\pgfsetmiterjoin%
\definecolor{currentfill}{rgb}{1.000000,1.000000,1.000000}%
\pgfsetfillcolor{currentfill}%
\pgfsetlinewidth{0.000000pt}%
\definecolor{currentstroke}{rgb}{0.000000,0.000000,0.000000}%
\pgfsetstrokecolor{currentstroke}%
\pgfsetstrokeopacity{0.000000}%
\pgfsetdash{}{0pt}%
\pgfpathmoveto{\pgfqpoint{0.880000in}{0.790446in}}%
\pgfpathlineto{\pgfqpoint{2.777959in}{0.790446in}}%
\pgfpathlineto{\pgfqpoint{2.777959in}{2.163173in}}%
\pgfpathlineto{\pgfqpoint{0.880000in}{2.163173in}}%
\pgfpathclose%
\pgfusepath{fill}%
\end{pgfscope}%
\begin{pgfscope}%
\pgfsetbuttcap%
\pgfsetroundjoin%
\definecolor{currentfill}{rgb}{0.000000,0.000000,0.000000}%
\pgfsetfillcolor{currentfill}%
\pgfsetlinewidth{0.803000pt}%
\definecolor{currentstroke}{rgb}{0.000000,0.000000,0.000000}%
\pgfsetstrokecolor{currentstroke}%
\pgfsetdash{}{0pt}%
\pgfsys@defobject{currentmarker}{\pgfqpoint{0.000000in}{-0.048611in}}{\pgfqpoint{0.000000in}{0.000000in}}{%
\pgfpathmoveto{\pgfqpoint{0.000000in}{0.000000in}}%
\pgfpathlineto{\pgfqpoint{0.000000in}{-0.048611in}}%
\pgfusepath{stroke,fill}%
}%
\begin{pgfscope}%
\pgfsys@transformshift{0.880000in}{0.790446in}%
\pgfsys@useobject{currentmarker}{}%
\end{pgfscope}%
\end{pgfscope}%
\begin{pgfscope}%
\pgftext[x=0.880000in,y=0.693224in,,top]{\rmfamily\fontsize{12.000000}{14.400000}\selectfont \(\displaystyle 0.0\)}%
\end{pgfscope}%
\begin{pgfscope}%
\pgfsetbuttcap%
\pgfsetroundjoin%
\definecolor{currentfill}{rgb}{0.000000,0.000000,0.000000}%
\pgfsetfillcolor{currentfill}%
\pgfsetlinewidth{0.803000pt}%
\definecolor{currentstroke}{rgb}{0.000000,0.000000,0.000000}%
\pgfsetstrokecolor{currentstroke}%
\pgfsetdash{}{0pt}%
\pgfsys@defobject{currentmarker}{\pgfqpoint{0.000000in}{-0.048611in}}{\pgfqpoint{0.000000in}{0.000000in}}{%
\pgfpathmoveto{\pgfqpoint{0.000000in}{0.000000in}}%
\pgfpathlineto{\pgfqpoint{0.000000in}{-0.048611in}}%
\pgfusepath{stroke,fill}%
}%
\begin{pgfscope}%
\pgfsys@transformshift{1.828980in}{0.790446in}%
\pgfsys@useobject{currentmarker}{}%
\end{pgfscope}%
\end{pgfscope}%
\begin{pgfscope}%
\pgftext[x=1.828980in,y=0.693224in,,top]{\rmfamily\fontsize{12.000000}{14.400000}\selectfont \(\displaystyle 0.5\)}%
\end{pgfscope}%
\begin{pgfscope}%
\pgfsetbuttcap%
\pgfsetroundjoin%
\definecolor{currentfill}{rgb}{0.000000,0.000000,0.000000}%
\pgfsetfillcolor{currentfill}%
\pgfsetlinewidth{0.803000pt}%
\definecolor{currentstroke}{rgb}{0.000000,0.000000,0.000000}%
\pgfsetstrokecolor{currentstroke}%
\pgfsetdash{}{0pt}%
\pgfsys@defobject{currentmarker}{\pgfqpoint{0.000000in}{-0.048611in}}{\pgfqpoint{0.000000in}{0.000000in}}{%
\pgfpathmoveto{\pgfqpoint{0.000000in}{0.000000in}}%
\pgfpathlineto{\pgfqpoint{0.000000in}{-0.048611in}}%
\pgfusepath{stroke,fill}%
}%
\begin{pgfscope}%
\pgfsys@transformshift{2.777959in}{0.790446in}%
\pgfsys@useobject{currentmarker}{}%
\end{pgfscope}%
\end{pgfscope}%
\begin{pgfscope}%
\pgftext[x=2.777959in,y=0.693224in,,top]{\rmfamily\fontsize{12.000000}{14.400000}\selectfont \(\displaystyle 1.0\)}%
\end{pgfscope}%
\begin{pgfscope}%
\pgfsetbuttcap%
\pgfsetroundjoin%
\definecolor{currentfill}{rgb}{0.000000,0.000000,0.000000}%
\pgfsetfillcolor{currentfill}%
\pgfsetlinewidth{0.803000pt}%
\definecolor{currentstroke}{rgb}{0.000000,0.000000,0.000000}%
\pgfsetstrokecolor{currentstroke}%
\pgfsetdash{}{0pt}%
\pgfsys@defobject{currentmarker}{\pgfqpoint{-0.048611in}{0.000000in}}{\pgfqpoint{0.000000in}{0.000000in}}{%
\pgfpathmoveto{\pgfqpoint{0.000000in}{0.000000in}}%
\pgfpathlineto{\pgfqpoint{-0.048611in}{0.000000in}}%
\pgfusepath{stroke,fill}%
}%
\begin{pgfscope}%
\pgfsys@transformshift{0.880000in}{0.790446in}%
\pgfsys@useobject{currentmarker}{}%
\end{pgfscope}%
\end{pgfscope}%
\begin{pgfscope}%
\pgftext[x=0.492657in,y=0.727132in,left,base]{\rmfamily\fontsize{12.000000}{14.400000}\selectfont \(\displaystyle 0.00\)}%
\end{pgfscope}%
\begin{pgfscope}%
\pgfsetbuttcap%
\pgfsetroundjoin%
\definecolor{currentfill}{rgb}{0.000000,0.000000,0.000000}%
\pgfsetfillcolor{currentfill}%
\pgfsetlinewidth{0.803000pt}%
\definecolor{currentstroke}{rgb}{0.000000,0.000000,0.000000}%
\pgfsetstrokecolor{currentstroke}%
\pgfsetdash{}{0pt}%
\pgfsys@defobject{currentmarker}{\pgfqpoint{-0.048611in}{0.000000in}}{\pgfqpoint{0.000000in}{0.000000in}}{%
\pgfpathmoveto{\pgfqpoint{0.000000in}{0.000000in}}%
\pgfpathlineto{\pgfqpoint{-0.048611in}{0.000000in}}%
\pgfusepath{stroke,fill}%
}%
\begin{pgfscope}%
\pgfsys@transformshift{0.880000in}{1.133628in}%
\pgfsys@useobject{currentmarker}{}%
\end{pgfscope}%
\end{pgfscope}%
\begin{pgfscope}%
\pgftext[x=0.492657in,y=1.070314in,left,base]{\rmfamily\fontsize{12.000000}{14.400000}\selectfont \(\displaystyle 0.25\)}%
\end{pgfscope}%
\begin{pgfscope}%
\pgfsetbuttcap%
\pgfsetroundjoin%
\definecolor{currentfill}{rgb}{0.000000,0.000000,0.000000}%
\pgfsetfillcolor{currentfill}%
\pgfsetlinewidth{0.803000pt}%
\definecolor{currentstroke}{rgb}{0.000000,0.000000,0.000000}%
\pgfsetstrokecolor{currentstroke}%
\pgfsetdash{}{0pt}%
\pgfsys@defobject{currentmarker}{\pgfqpoint{-0.048611in}{0.000000in}}{\pgfqpoint{0.000000in}{0.000000in}}{%
\pgfpathmoveto{\pgfqpoint{0.000000in}{0.000000in}}%
\pgfpathlineto{\pgfqpoint{-0.048611in}{0.000000in}}%
\pgfusepath{stroke,fill}%
}%
\begin{pgfscope}%
\pgfsys@transformshift{0.880000in}{1.476809in}%
\pgfsys@useobject{currentmarker}{}%
\end{pgfscope}%
\end{pgfscope}%
\begin{pgfscope}%
\pgftext[x=0.492657in,y=1.413496in,left,base]{\rmfamily\fontsize{12.000000}{14.400000}\selectfont \(\displaystyle 0.50\)}%
\end{pgfscope}%
\begin{pgfscope}%
\pgfsetbuttcap%
\pgfsetroundjoin%
\definecolor{currentfill}{rgb}{0.000000,0.000000,0.000000}%
\pgfsetfillcolor{currentfill}%
\pgfsetlinewidth{0.803000pt}%
\definecolor{currentstroke}{rgb}{0.000000,0.000000,0.000000}%
\pgfsetstrokecolor{currentstroke}%
\pgfsetdash{}{0pt}%
\pgfsys@defobject{currentmarker}{\pgfqpoint{-0.048611in}{0.000000in}}{\pgfqpoint{0.000000in}{0.000000in}}{%
\pgfpathmoveto{\pgfqpoint{0.000000in}{0.000000in}}%
\pgfpathlineto{\pgfqpoint{-0.048611in}{0.000000in}}%
\pgfusepath{stroke,fill}%
}%
\begin{pgfscope}%
\pgfsys@transformshift{0.880000in}{1.819991in}%
\pgfsys@useobject{currentmarker}{}%
\end{pgfscope}%
\end{pgfscope}%
\begin{pgfscope}%
\pgftext[x=0.492657in,y=1.756677in,left,base]{\rmfamily\fontsize{12.000000}{14.400000}\selectfont \(\displaystyle 0.75\)}%
\end{pgfscope}%
\begin{pgfscope}%
\pgfsetbuttcap%
\pgfsetroundjoin%
\definecolor{currentfill}{rgb}{0.000000,0.000000,0.000000}%
\pgfsetfillcolor{currentfill}%
\pgfsetlinewidth{0.803000pt}%
\definecolor{currentstroke}{rgb}{0.000000,0.000000,0.000000}%
\pgfsetstrokecolor{currentstroke}%
\pgfsetdash{}{0pt}%
\pgfsys@defobject{currentmarker}{\pgfqpoint{-0.048611in}{0.000000in}}{\pgfqpoint{0.000000in}{0.000000in}}{%
\pgfpathmoveto{\pgfqpoint{0.000000in}{0.000000in}}%
\pgfpathlineto{\pgfqpoint{-0.048611in}{0.000000in}}%
\pgfusepath{stroke,fill}%
}%
\begin{pgfscope}%
\pgfsys@transformshift{0.880000in}{2.163173in}%
\pgfsys@useobject{currentmarker}{}%
\end{pgfscope}%
\end{pgfscope}%
\begin{pgfscope}%
\pgftext[x=0.492657in,y=2.099859in,left,base]{\rmfamily\fontsize{12.000000}{14.400000}\selectfont \(\displaystyle 1.00\)}%
\end{pgfscope}%
\begin{pgfscope}%
\pgfsetrectcap%
\pgfsetmiterjoin%
\pgfsetlinewidth{0.803000pt}%
\definecolor{currentstroke}{rgb}{0.501961,0.501961,0.501961}%
\pgfsetstrokecolor{currentstroke}%
\pgfsetdash{}{0pt}%
\pgfpathmoveto{\pgfqpoint{0.880000in}{0.790446in}}%
\pgfpathlineto{\pgfqpoint{0.880000in}{2.163173in}}%
\pgfusepath{stroke}%
\end{pgfscope}%
\begin{pgfscope}%
\pgfsetrectcap%
\pgfsetmiterjoin%
\pgfsetlinewidth{0.803000pt}%
\definecolor{currentstroke}{rgb}{0.501961,0.501961,0.501961}%
\pgfsetstrokecolor{currentstroke}%
\pgfsetdash{}{0pt}%
\pgfpathmoveto{\pgfqpoint{2.777959in}{0.790446in}}%
\pgfpathlineto{\pgfqpoint{2.777959in}{2.163173in}}%
\pgfusepath{stroke}%
\end{pgfscope}%
\begin{pgfscope}%
\pgfsetrectcap%
\pgfsetmiterjoin%
\pgfsetlinewidth{0.803000pt}%
\definecolor{currentstroke}{rgb}{0.501961,0.501961,0.501961}%
\pgfsetstrokecolor{currentstroke}%
\pgfsetdash{}{0pt}%
\pgfpathmoveto{\pgfqpoint{0.880000in}{0.790446in}}%
\pgfpathlineto{\pgfqpoint{2.777959in}{0.790446in}}%
\pgfusepath{stroke}%
\end{pgfscope}%
\begin{pgfscope}%
\pgfsetrectcap%
\pgfsetmiterjoin%
\pgfsetlinewidth{0.803000pt}%
\definecolor{currentstroke}{rgb}{0.501961,0.501961,0.501961}%
\pgfsetstrokecolor{currentstroke}%
\pgfsetdash{}{0pt}%
\pgfpathmoveto{\pgfqpoint{0.880000in}{2.163173in}}%
\pgfpathlineto{\pgfqpoint{2.777959in}{2.163173in}}%
\pgfusepath{stroke}%
\end{pgfscope}%
\begin{pgfscope}%
\pgfsetbuttcap%
\pgfsetmiterjoin%
\definecolor{currentfill}{rgb}{1.000000,1.000000,1.000000}%
\pgfsetfillcolor{currentfill}%
\pgfsetlinewidth{0.000000pt}%
\definecolor{currentstroke}{rgb}{0.000000,0.000000,0.000000}%
\pgfsetstrokecolor{currentstroke}%
\pgfsetstrokeopacity{0.000000}%
\pgfsetdash{}{0pt}%
\pgfpathmoveto{\pgfqpoint{3.347347in}{0.790446in}}%
\pgfpathlineto{\pgfqpoint{5.245306in}{0.790446in}}%
\pgfpathlineto{\pgfqpoint{5.245306in}{2.163173in}}%
\pgfpathlineto{\pgfqpoint{3.347347in}{2.163173in}}%
\pgfpathclose%
\pgfusepath{fill}%
\end{pgfscope}%
\begin{pgfscope}%
\pgfsetbuttcap%
\pgfsetroundjoin%
\definecolor{currentfill}{rgb}{0.000000,0.000000,0.000000}%
\pgfsetfillcolor{currentfill}%
\pgfsetlinewidth{0.803000pt}%
\definecolor{currentstroke}{rgb}{0.000000,0.000000,0.000000}%
\pgfsetstrokecolor{currentstroke}%
\pgfsetdash{}{0pt}%
\pgfsys@defobject{currentmarker}{\pgfqpoint{0.000000in}{-0.048611in}}{\pgfqpoint{0.000000in}{0.000000in}}{%
\pgfpathmoveto{\pgfqpoint{0.000000in}{0.000000in}}%
\pgfpathlineto{\pgfqpoint{0.000000in}{-0.048611in}}%
\pgfusepath{stroke,fill}%
}%
\begin{pgfscope}%
\pgfsys@transformshift{3.347347in}{0.790446in}%
\pgfsys@useobject{currentmarker}{}%
\end{pgfscope}%
\end{pgfscope}%
\begin{pgfscope}%
\pgftext[x=3.347347in,y=0.693224in,,top]{\rmfamily\fontsize{12.000000}{14.400000}\selectfont \(\displaystyle 0.0\)}%
\end{pgfscope}%
\begin{pgfscope}%
\pgfsetbuttcap%
\pgfsetroundjoin%
\definecolor{currentfill}{rgb}{0.000000,0.000000,0.000000}%
\pgfsetfillcolor{currentfill}%
\pgfsetlinewidth{0.803000pt}%
\definecolor{currentstroke}{rgb}{0.000000,0.000000,0.000000}%
\pgfsetstrokecolor{currentstroke}%
\pgfsetdash{}{0pt}%
\pgfsys@defobject{currentmarker}{\pgfqpoint{0.000000in}{-0.048611in}}{\pgfqpoint{0.000000in}{0.000000in}}{%
\pgfpathmoveto{\pgfqpoint{0.000000in}{0.000000in}}%
\pgfpathlineto{\pgfqpoint{0.000000in}{-0.048611in}}%
\pgfusepath{stroke,fill}%
}%
\begin{pgfscope}%
\pgfsys@transformshift{4.296327in}{0.790446in}%
\pgfsys@useobject{currentmarker}{}%
\end{pgfscope}%
\end{pgfscope}%
\begin{pgfscope}%
\pgftext[x=4.296327in,y=0.693224in,,top]{\rmfamily\fontsize{12.000000}{14.400000}\selectfont \(\displaystyle 0.5\)}%
\end{pgfscope}%
\begin{pgfscope}%
\pgfsetbuttcap%
\pgfsetroundjoin%
\definecolor{currentfill}{rgb}{0.000000,0.000000,0.000000}%
\pgfsetfillcolor{currentfill}%
\pgfsetlinewidth{0.803000pt}%
\definecolor{currentstroke}{rgb}{0.000000,0.000000,0.000000}%
\pgfsetstrokecolor{currentstroke}%
\pgfsetdash{}{0pt}%
\pgfsys@defobject{currentmarker}{\pgfqpoint{0.000000in}{-0.048611in}}{\pgfqpoint{0.000000in}{0.000000in}}{%
\pgfpathmoveto{\pgfqpoint{0.000000in}{0.000000in}}%
\pgfpathlineto{\pgfqpoint{0.000000in}{-0.048611in}}%
\pgfusepath{stroke,fill}%
}%
\begin{pgfscope}%
\pgfsys@transformshift{5.245306in}{0.790446in}%
\pgfsys@useobject{currentmarker}{}%
\end{pgfscope}%
\end{pgfscope}%
\begin{pgfscope}%
\pgftext[x=5.245306in,y=0.693224in,,top]{\rmfamily\fontsize{12.000000}{14.400000}\selectfont \(\displaystyle 1.0\)}%
\end{pgfscope}%
\begin{pgfscope}%
\pgfsetbuttcap%
\pgfsetroundjoin%
\definecolor{currentfill}{rgb}{0.000000,0.000000,0.000000}%
\pgfsetfillcolor{currentfill}%
\pgfsetlinewidth{0.803000pt}%
\definecolor{currentstroke}{rgb}{0.000000,0.000000,0.000000}%
\pgfsetstrokecolor{currentstroke}%
\pgfsetdash{}{0pt}%
\pgfsys@defobject{currentmarker}{\pgfqpoint{-0.048611in}{0.000000in}}{\pgfqpoint{0.000000in}{0.000000in}}{%
\pgfpathmoveto{\pgfqpoint{0.000000in}{0.000000in}}%
\pgfpathlineto{\pgfqpoint{-0.048611in}{0.000000in}}%
\pgfusepath{stroke,fill}%
}%
\begin{pgfscope}%
\pgfsys@transformshift{3.347347in}{0.790446in}%
\pgfsys@useobject{currentmarker}{}%
\end{pgfscope}%
\end{pgfscope}%
\begin{pgfscope}%
\pgftext[x=2.960004in,y=0.727132in,left,base]{\rmfamily\fontsize{12.000000}{14.400000}\selectfont \(\displaystyle 0.00\)}%
\end{pgfscope}%
\begin{pgfscope}%
\pgfsetbuttcap%
\pgfsetroundjoin%
\definecolor{currentfill}{rgb}{0.000000,0.000000,0.000000}%
\pgfsetfillcolor{currentfill}%
\pgfsetlinewidth{0.803000pt}%
\definecolor{currentstroke}{rgb}{0.000000,0.000000,0.000000}%
\pgfsetstrokecolor{currentstroke}%
\pgfsetdash{}{0pt}%
\pgfsys@defobject{currentmarker}{\pgfqpoint{-0.048611in}{0.000000in}}{\pgfqpoint{0.000000in}{0.000000in}}{%
\pgfpathmoveto{\pgfqpoint{0.000000in}{0.000000in}}%
\pgfpathlineto{\pgfqpoint{-0.048611in}{0.000000in}}%
\pgfusepath{stroke,fill}%
}%
\begin{pgfscope}%
\pgfsys@transformshift{3.347347in}{1.133628in}%
\pgfsys@useobject{currentmarker}{}%
\end{pgfscope}%
\end{pgfscope}%
\begin{pgfscope}%
\pgftext[x=2.960004in,y=1.070314in,left,base]{\rmfamily\fontsize{12.000000}{14.400000}\selectfont \(\displaystyle 0.25\)}%
\end{pgfscope}%
\begin{pgfscope}%
\pgfsetbuttcap%
\pgfsetroundjoin%
\definecolor{currentfill}{rgb}{0.000000,0.000000,0.000000}%
\pgfsetfillcolor{currentfill}%
\pgfsetlinewidth{0.803000pt}%
\definecolor{currentstroke}{rgb}{0.000000,0.000000,0.000000}%
\pgfsetstrokecolor{currentstroke}%
\pgfsetdash{}{0pt}%
\pgfsys@defobject{currentmarker}{\pgfqpoint{-0.048611in}{0.000000in}}{\pgfqpoint{0.000000in}{0.000000in}}{%
\pgfpathmoveto{\pgfqpoint{0.000000in}{0.000000in}}%
\pgfpathlineto{\pgfqpoint{-0.048611in}{0.000000in}}%
\pgfusepath{stroke,fill}%
}%
\begin{pgfscope}%
\pgfsys@transformshift{3.347347in}{1.476809in}%
\pgfsys@useobject{currentmarker}{}%
\end{pgfscope}%
\end{pgfscope}%
\begin{pgfscope}%
\pgftext[x=2.960004in,y=1.413496in,left,base]{\rmfamily\fontsize{12.000000}{14.400000}\selectfont \(\displaystyle 0.50\)}%
\end{pgfscope}%
\begin{pgfscope}%
\pgfsetbuttcap%
\pgfsetroundjoin%
\definecolor{currentfill}{rgb}{0.000000,0.000000,0.000000}%
\pgfsetfillcolor{currentfill}%
\pgfsetlinewidth{0.803000pt}%
\definecolor{currentstroke}{rgb}{0.000000,0.000000,0.000000}%
\pgfsetstrokecolor{currentstroke}%
\pgfsetdash{}{0pt}%
\pgfsys@defobject{currentmarker}{\pgfqpoint{-0.048611in}{0.000000in}}{\pgfqpoint{0.000000in}{0.000000in}}{%
\pgfpathmoveto{\pgfqpoint{0.000000in}{0.000000in}}%
\pgfpathlineto{\pgfqpoint{-0.048611in}{0.000000in}}%
\pgfusepath{stroke,fill}%
}%
\begin{pgfscope}%
\pgfsys@transformshift{3.347347in}{1.819991in}%
\pgfsys@useobject{currentmarker}{}%
\end{pgfscope}%
\end{pgfscope}%
\begin{pgfscope}%
\pgftext[x=2.960004in,y=1.756677in,left,base]{\rmfamily\fontsize{12.000000}{14.400000}\selectfont \(\displaystyle 0.75\)}%
\end{pgfscope}%
\begin{pgfscope}%
\pgfsetbuttcap%
\pgfsetroundjoin%
\definecolor{currentfill}{rgb}{0.000000,0.000000,0.000000}%
\pgfsetfillcolor{currentfill}%
\pgfsetlinewidth{0.803000pt}%
\definecolor{currentstroke}{rgb}{0.000000,0.000000,0.000000}%
\pgfsetstrokecolor{currentstroke}%
\pgfsetdash{}{0pt}%
\pgfsys@defobject{currentmarker}{\pgfqpoint{-0.048611in}{0.000000in}}{\pgfqpoint{0.000000in}{0.000000in}}{%
\pgfpathmoveto{\pgfqpoint{0.000000in}{0.000000in}}%
\pgfpathlineto{\pgfqpoint{-0.048611in}{0.000000in}}%
\pgfusepath{stroke,fill}%
}%
\begin{pgfscope}%
\pgfsys@transformshift{3.347347in}{2.163173in}%
\pgfsys@useobject{currentmarker}{}%
\end{pgfscope}%
\end{pgfscope}%
\begin{pgfscope}%
\pgftext[x=2.960004in,y=2.099859in,left,base]{\rmfamily\fontsize{12.000000}{14.400000}\selectfont \(\displaystyle 1.00\)}%
\end{pgfscope}%
\begin{pgfscope}%
\pgfsetrectcap%
\pgfsetmiterjoin%
\pgfsetlinewidth{0.803000pt}%
\definecolor{currentstroke}{rgb}{0.501961,0.501961,0.501961}%
\pgfsetstrokecolor{currentstroke}%
\pgfsetdash{}{0pt}%
\pgfpathmoveto{\pgfqpoint{3.347347in}{0.790446in}}%
\pgfpathlineto{\pgfqpoint{3.347347in}{2.163173in}}%
\pgfusepath{stroke}%
\end{pgfscope}%
\begin{pgfscope}%
\pgfsetrectcap%
\pgfsetmiterjoin%
\pgfsetlinewidth{0.803000pt}%
\definecolor{currentstroke}{rgb}{0.501961,0.501961,0.501961}%
\pgfsetstrokecolor{currentstroke}%
\pgfsetdash{}{0pt}%
\pgfpathmoveto{\pgfqpoint{5.245306in}{0.790446in}}%
\pgfpathlineto{\pgfqpoint{5.245306in}{2.163173in}}%
\pgfusepath{stroke}%
\end{pgfscope}%
\begin{pgfscope}%
\pgfsetrectcap%
\pgfsetmiterjoin%
\pgfsetlinewidth{0.803000pt}%
\definecolor{currentstroke}{rgb}{0.501961,0.501961,0.501961}%
\pgfsetstrokecolor{currentstroke}%
\pgfsetdash{}{0pt}%
\pgfpathmoveto{\pgfqpoint{3.347347in}{0.790446in}}%
\pgfpathlineto{\pgfqpoint{5.245306in}{0.790446in}}%
\pgfusepath{stroke}%
\end{pgfscope}%
\begin{pgfscope}%
\pgfsetrectcap%
\pgfsetmiterjoin%
\pgfsetlinewidth{0.803000pt}%
\definecolor{currentstroke}{rgb}{0.501961,0.501961,0.501961}%
\pgfsetstrokecolor{currentstroke}%
\pgfsetdash{}{0pt}%
\pgfpathmoveto{\pgfqpoint{3.347347in}{2.163173in}}%
\pgfpathlineto{\pgfqpoint{5.245306in}{2.163173in}}%
\pgfusepath{stroke}%
\end{pgfscope}%
\begin{pgfscope}%
\pgfsetbuttcap%
\pgfsetmiterjoin%
\definecolor{currentfill}{rgb}{1.000000,1.000000,1.000000}%
\pgfsetfillcolor{currentfill}%
\pgfsetlinewidth{0.000000pt}%
\definecolor{currentstroke}{rgb}{0.000000,0.000000,0.000000}%
\pgfsetstrokecolor{currentstroke}%
\pgfsetstrokeopacity{0.000000}%
\pgfsetdash{}{0pt}%
\pgfpathmoveto{\pgfqpoint{5.814694in}{0.790446in}}%
\pgfpathlineto{\pgfqpoint{7.712653in}{0.790446in}}%
\pgfpathlineto{\pgfqpoint{7.712653in}{2.163173in}}%
\pgfpathlineto{\pgfqpoint{5.814694in}{2.163173in}}%
\pgfpathclose%
\pgfusepath{fill}%
\end{pgfscope}%
\begin{pgfscope}%
\pgfsetbuttcap%
\pgfsetroundjoin%
\definecolor{currentfill}{rgb}{0.000000,0.000000,0.000000}%
\pgfsetfillcolor{currentfill}%
\pgfsetlinewidth{0.803000pt}%
\definecolor{currentstroke}{rgb}{0.000000,0.000000,0.000000}%
\pgfsetstrokecolor{currentstroke}%
\pgfsetdash{}{0pt}%
\pgfsys@defobject{currentmarker}{\pgfqpoint{0.000000in}{-0.048611in}}{\pgfqpoint{0.000000in}{0.000000in}}{%
\pgfpathmoveto{\pgfqpoint{0.000000in}{0.000000in}}%
\pgfpathlineto{\pgfqpoint{0.000000in}{-0.048611in}}%
\pgfusepath{stroke,fill}%
}%
\begin{pgfscope}%
\pgfsys@transformshift{5.814694in}{0.790446in}%
\pgfsys@useobject{currentmarker}{}%
\end{pgfscope}%
\end{pgfscope}%
\begin{pgfscope}%
\pgftext[x=5.814694in,y=0.693224in,,top]{\rmfamily\fontsize{12.000000}{14.400000}\selectfont \(\displaystyle 0.0\)}%
\end{pgfscope}%
\begin{pgfscope}%
\pgfsetbuttcap%
\pgfsetroundjoin%
\definecolor{currentfill}{rgb}{0.000000,0.000000,0.000000}%
\pgfsetfillcolor{currentfill}%
\pgfsetlinewidth{0.803000pt}%
\definecolor{currentstroke}{rgb}{0.000000,0.000000,0.000000}%
\pgfsetstrokecolor{currentstroke}%
\pgfsetdash{}{0pt}%
\pgfsys@defobject{currentmarker}{\pgfqpoint{0.000000in}{-0.048611in}}{\pgfqpoint{0.000000in}{0.000000in}}{%
\pgfpathmoveto{\pgfqpoint{0.000000in}{0.000000in}}%
\pgfpathlineto{\pgfqpoint{0.000000in}{-0.048611in}}%
\pgfusepath{stroke,fill}%
}%
\begin{pgfscope}%
\pgfsys@transformshift{6.763673in}{0.790446in}%
\pgfsys@useobject{currentmarker}{}%
\end{pgfscope}%
\end{pgfscope}%
\begin{pgfscope}%
\pgftext[x=6.763673in,y=0.693224in,,top]{\rmfamily\fontsize{12.000000}{14.400000}\selectfont \(\displaystyle 0.5\)}%
\end{pgfscope}%
\begin{pgfscope}%
\pgfsetbuttcap%
\pgfsetroundjoin%
\definecolor{currentfill}{rgb}{0.000000,0.000000,0.000000}%
\pgfsetfillcolor{currentfill}%
\pgfsetlinewidth{0.803000pt}%
\definecolor{currentstroke}{rgb}{0.000000,0.000000,0.000000}%
\pgfsetstrokecolor{currentstroke}%
\pgfsetdash{}{0pt}%
\pgfsys@defobject{currentmarker}{\pgfqpoint{0.000000in}{-0.048611in}}{\pgfqpoint{0.000000in}{0.000000in}}{%
\pgfpathmoveto{\pgfqpoint{0.000000in}{0.000000in}}%
\pgfpathlineto{\pgfqpoint{0.000000in}{-0.048611in}}%
\pgfusepath{stroke,fill}%
}%
\begin{pgfscope}%
\pgfsys@transformshift{7.712653in}{0.790446in}%
\pgfsys@useobject{currentmarker}{}%
\end{pgfscope}%
\end{pgfscope}%
\begin{pgfscope}%
\pgftext[x=7.712653in,y=0.693224in,,top]{\rmfamily\fontsize{12.000000}{14.400000}\selectfont \(\displaystyle 1.0\)}%
\end{pgfscope}%
\begin{pgfscope}%
\pgfsetbuttcap%
\pgfsetroundjoin%
\definecolor{currentfill}{rgb}{0.000000,0.000000,0.000000}%
\pgfsetfillcolor{currentfill}%
\pgfsetlinewidth{0.803000pt}%
\definecolor{currentstroke}{rgb}{0.000000,0.000000,0.000000}%
\pgfsetstrokecolor{currentstroke}%
\pgfsetdash{}{0pt}%
\pgfsys@defobject{currentmarker}{\pgfqpoint{-0.048611in}{0.000000in}}{\pgfqpoint{0.000000in}{0.000000in}}{%
\pgfpathmoveto{\pgfqpoint{0.000000in}{0.000000in}}%
\pgfpathlineto{\pgfqpoint{-0.048611in}{0.000000in}}%
\pgfusepath{stroke,fill}%
}%
\begin{pgfscope}%
\pgfsys@transformshift{5.814694in}{0.790446in}%
\pgfsys@useobject{currentmarker}{}%
\end{pgfscope}%
\end{pgfscope}%
\begin{pgfscope}%
\pgftext[x=5.427351in,y=0.727132in,left,base]{\rmfamily\fontsize{12.000000}{14.400000}\selectfont \(\displaystyle 0.00\)}%
\end{pgfscope}%
\begin{pgfscope}%
\pgfsetbuttcap%
\pgfsetroundjoin%
\definecolor{currentfill}{rgb}{0.000000,0.000000,0.000000}%
\pgfsetfillcolor{currentfill}%
\pgfsetlinewidth{0.803000pt}%
\definecolor{currentstroke}{rgb}{0.000000,0.000000,0.000000}%
\pgfsetstrokecolor{currentstroke}%
\pgfsetdash{}{0pt}%
\pgfsys@defobject{currentmarker}{\pgfqpoint{-0.048611in}{0.000000in}}{\pgfqpoint{0.000000in}{0.000000in}}{%
\pgfpathmoveto{\pgfqpoint{0.000000in}{0.000000in}}%
\pgfpathlineto{\pgfqpoint{-0.048611in}{0.000000in}}%
\pgfusepath{stroke,fill}%
}%
\begin{pgfscope}%
\pgfsys@transformshift{5.814694in}{1.133628in}%
\pgfsys@useobject{currentmarker}{}%
\end{pgfscope}%
\end{pgfscope}%
\begin{pgfscope}%
\pgftext[x=5.427351in,y=1.070314in,left,base]{\rmfamily\fontsize{12.000000}{14.400000}\selectfont \(\displaystyle 0.25\)}%
\end{pgfscope}%
\begin{pgfscope}%
\pgfsetbuttcap%
\pgfsetroundjoin%
\definecolor{currentfill}{rgb}{0.000000,0.000000,0.000000}%
\pgfsetfillcolor{currentfill}%
\pgfsetlinewidth{0.803000pt}%
\definecolor{currentstroke}{rgb}{0.000000,0.000000,0.000000}%
\pgfsetstrokecolor{currentstroke}%
\pgfsetdash{}{0pt}%
\pgfsys@defobject{currentmarker}{\pgfqpoint{-0.048611in}{0.000000in}}{\pgfqpoint{0.000000in}{0.000000in}}{%
\pgfpathmoveto{\pgfqpoint{0.000000in}{0.000000in}}%
\pgfpathlineto{\pgfqpoint{-0.048611in}{0.000000in}}%
\pgfusepath{stroke,fill}%
}%
\begin{pgfscope}%
\pgfsys@transformshift{5.814694in}{1.476809in}%
\pgfsys@useobject{currentmarker}{}%
\end{pgfscope}%
\end{pgfscope}%
\begin{pgfscope}%
\pgftext[x=5.427351in,y=1.413496in,left,base]{\rmfamily\fontsize{12.000000}{14.400000}\selectfont \(\displaystyle 0.50\)}%
\end{pgfscope}%
\begin{pgfscope}%
\pgfsetbuttcap%
\pgfsetroundjoin%
\definecolor{currentfill}{rgb}{0.000000,0.000000,0.000000}%
\pgfsetfillcolor{currentfill}%
\pgfsetlinewidth{0.803000pt}%
\definecolor{currentstroke}{rgb}{0.000000,0.000000,0.000000}%
\pgfsetstrokecolor{currentstroke}%
\pgfsetdash{}{0pt}%
\pgfsys@defobject{currentmarker}{\pgfqpoint{-0.048611in}{0.000000in}}{\pgfqpoint{0.000000in}{0.000000in}}{%
\pgfpathmoveto{\pgfqpoint{0.000000in}{0.000000in}}%
\pgfpathlineto{\pgfqpoint{-0.048611in}{0.000000in}}%
\pgfusepath{stroke,fill}%
}%
\begin{pgfscope}%
\pgfsys@transformshift{5.814694in}{1.819991in}%
\pgfsys@useobject{currentmarker}{}%
\end{pgfscope}%
\end{pgfscope}%
\begin{pgfscope}%
\pgftext[x=5.427351in,y=1.756677in,left,base]{\rmfamily\fontsize{12.000000}{14.400000}\selectfont \(\displaystyle 0.75\)}%
\end{pgfscope}%
\begin{pgfscope}%
\pgfsetbuttcap%
\pgfsetroundjoin%
\definecolor{currentfill}{rgb}{0.000000,0.000000,0.000000}%
\pgfsetfillcolor{currentfill}%
\pgfsetlinewidth{0.803000pt}%
\definecolor{currentstroke}{rgb}{0.000000,0.000000,0.000000}%
\pgfsetstrokecolor{currentstroke}%
\pgfsetdash{}{0pt}%
\pgfsys@defobject{currentmarker}{\pgfqpoint{-0.048611in}{0.000000in}}{\pgfqpoint{0.000000in}{0.000000in}}{%
\pgfpathmoveto{\pgfqpoint{0.000000in}{0.000000in}}%
\pgfpathlineto{\pgfqpoint{-0.048611in}{0.000000in}}%
\pgfusepath{stroke,fill}%
}%
\begin{pgfscope}%
\pgfsys@transformshift{5.814694in}{2.163173in}%
\pgfsys@useobject{currentmarker}{}%
\end{pgfscope}%
\end{pgfscope}%
\begin{pgfscope}%
\pgftext[x=5.427351in,y=2.099859in,left,base]{\rmfamily\fontsize{12.000000}{14.400000}\selectfont \(\displaystyle 1.00\)}%
\end{pgfscope}%
\begin{pgfscope}%
\pgfsetrectcap%
\pgfsetmiterjoin%
\pgfsetlinewidth{0.803000pt}%
\definecolor{currentstroke}{rgb}{0.501961,0.501961,0.501961}%
\pgfsetstrokecolor{currentstroke}%
\pgfsetdash{}{0pt}%
\pgfpathmoveto{\pgfqpoint{5.814694in}{0.790446in}}%
\pgfpathlineto{\pgfqpoint{5.814694in}{2.163173in}}%
\pgfusepath{stroke}%
\end{pgfscope}%
\begin{pgfscope}%
\pgfsetrectcap%
\pgfsetmiterjoin%
\pgfsetlinewidth{0.803000pt}%
\definecolor{currentstroke}{rgb}{0.501961,0.501961,0.501961}%
\pgfsetstrokecolor{currentstroke}%
\pgfsetdash{}{0pt}%
\pgfpathmoveto{\pgfqpoint{7.712653in}{0.790446in}}%
\pgfpathlineto{\pgfqpoint{7.712653in}{2.163173in}}%
\pgfusepath{stroke}%
\end{pgfscope}%
\begin{pgfscope}%
\pgfsetrectcap%
\pgfsetmiterjoin%
\pgfsetlinewidth{0.803000pt}%
\definecolor{currentstroke}{rgb}{0.501961,0.501961,0.501961}%
\pgfsetstrokecolor{currentstroke}%
\pgfsetdash{}{0pt}%
\pgfpathmoveto{\pgfqpoint{5.814694in}{0.790446in}}%
\pgfpathlineto{\pgfqpoint{7.712653in}{0.790446in}}%
\pgfusepath{stroke}%
\end{pgfscope}%
\begin{pgfscope}%
\pgfsetrectcap%
\pgfsetmiterjoin%
\pgfsetlinewidth{0.803000pt}%
\definecolor{currentstroke}{rgb}{0.501961,0.501961,0.501961}%
\pgfsetstrokecolor{currentstroke}%
\pgfsetdash{}{0pt}%
\pgfpathmoveto{\pgfqpoint{5.814694in}{2.163173in}}%
\pgfpathlineto{\pgfqpoint{7.712653in}{2.163173in}}%
\pgfusepath{stroke}%
\end{pgfscope}%
\begin{pgfscope}%
\pgfsetbuttcap%
\pgfsetmiterjoin%
\definecolor{currentfill}{rgb}{1.000000,1.000000,1.000000}%
\pgfsetfillcolor{currentfill}%
\pgfsetlinewidth{0.000000pt}%
\definecolor{currentstroke}{rgb}{0.000000,0.000000,0.000000}%
\pgfsetstrokecolor{currentstroke}%
\pgfsetstrokeopacity{0.000000}%
\pgfsetdash{}{0pt}%
\pgfpathmoveto{\pgfqpoint{8.282041in}{0.790446in}}%
\pgfpathlineto{\pgfqpoint{10.180000in}{0.790446in}}%
\pgfpathlineto{\pgfqpoint{10.180000in}{2.163173in}}%
\pgfpathlineto{\pgfqpoint{8.282041in}{2.163173in}}%
\pgfpathclose%
\pgfusepath{fill}%
\end{pgfscope}%
\begin{pgfscope}%
\pgfsetbuttcap%
\pgfsetroundjoin%
\definecolor{currentfill}{rgb}{0.000000,0.000000,0.000000}%
\pgfsetfillcolor{currentfill}%
\pgfsetlinewidth{0.803000pt}%
\definecolor{currentstroke}{rgb}{0.000000,0.000000,0.000000}%
\pgfsetstrokecolor{currentstroke}%
\pgfsetdash{}{0pt}%
\pgfsys@defobject{currentmarker}{\pgfqpoint{0.000000in}{-0.048611in}}{\pgfqpoint{0.000000in}{0.000000in}}{%
\pgfpathmoveto{\pgfqpoint{0.000000in}{0.000000in}}%
\pgfpathlineto{\pgfqpoint{0.000000in}{-0.048611in}}%
\pgfusepath{stroke,fill}%
}%
\begin{pgfscope}%
\pgfsys@transformshift{9.166562in}{0.790446in}%
\pgfsys@useobject{currentmarker}{}%
\end{pgfscope}%
\end{pgfscope}%
\begin{pgfscope}%
\pgftext[x=9.166562in,y=0.693224in,,top]{\rmfamily\fontsize{12.000000}{14.400000}\selectfont \(\displaystyle 0.5\)}%
\end{pgfscope}%
\begin{pgfscope}%
\pgfsetbuttcap%
\pgfsetroundjoin%
\definecolor{currentfill}{rgb}{0.000000,0.000000,0.000000}%
\pgfsetfillcolor{currentfill}%
\pgfsetlinewidth{0.803000pt}%
\definecolor{currentstroke}{rgb}{0.000000,0.000000,0.000000}%
\pgfsetstrokecolor{currentstroke}%
\pgfsetdash{}{0pt}%
\pgfsys@defobject{currentmarker}{\pgfqpoint{0.000000in}{-0.048611in}}{\pgfqpoint{0.000000in}{0.000000in}}{%
\pgfpathmoveto{\pgfqpoint{0.000000in}{0.000000in}}%
\pgfpathlineto{\pgfqpoint{0.000000in}{-0.048611in}}%
\pgfusepath{stroke,fill}%
}%
\begin{pgfscope}%
\pgfsys@transformshift{10.082468in}{0.790446in}%
\pgfsys@useobject{currentmarker}{}%
\end{pgfscope}%
\end{pgfscope}%
\begin{pgfscope}%
\pgftext[x=10.082468in,y=0.693224in,,top]{\rmfamily\fontsize{12.000000}{14.400000}\selectfont \(\displaystyle 1.0\)}%
\end{pgfscope}%
\begin{pgfscope}%
\pgfsetbuttcap%
\pgfsetroundjoin%
\definecolor{currentfill}{rgb}{0.000000,0.000000,0.000000}%
\pgfsetfillcolor{currentfill}%
\pgfsetlinewidth{0.803000pt}%
\definecolor{currentstroke}{rgb}{0.000000,0.000000,0.000000}%
\pgfsetstrokecolor{currentstroke}%
\pgfsetdash{}{0pt}%
\pgfsys@defobject{currentmarker}{\pgfqpoint{-0.048611in}{0.000000in}}{\pgfqpoint{0.000000in}{0.000000in}}{%
\pgfpathmoveto{\pgfqpoint{0.000000in}{0.000000in}}%
\pgfpathlineto{\pgfqpoint{-0.048611in}{0.000000in}}%
\pgfusepath{stroke,fill}%
}%
\begin{pgfscope}%
\pgfsys@transformshift{8.282041in}{1.007426in}%
\pgfsys@useobject{currentmarker}{}%
\end{pgfscope}%
\end{pgfscope}%
\begin{pgfscope}%
\pgftext[x=7.863830in,y=0.944112in,left,base]{\rmfamily\fontsize{12.000000}{14.400000}\selectfont \(\displaystyle 10^{-2}\)}%
\end{pgfscope}%
\begin{pgfscope}%
\pgfsetbuttcap%
\pgfsetroundjoin%
\definecolor{currentfill}{rgb}{0.000000,0.000000,0.000000}%
\pgfsetfillcolor{currentfill}%
\pgfsetlinewidth{0.803000pt}%
\definecolor{currentstroke}{rgb}{0.000000,0.000000,0.000000}%
\pgfsetstrokecolor{currentstroke}%
\pgfsetdash{}{0pt}%
\pgfsys@defobject{currentmarker}{\pgfqpoint{-0.048611in}{0.000000in}}{\pgfqpoint{0.000000in}{0.000000in}}{%
\pgfpathmoveto{\pgfqpoint{0.000000in}{0.000000in}}%
\pgfpathlineto{\pgfqpoint{-0.048611in}{0.000000in}}%
\pgfusepath{stroke,fill}%
}%
\begin{pgfscope}%
\pgfsys@transformshift{8.282041in}{1.558552in}%
\pgfsys@useobject{currentmarker}{}%
\end{pgfscope}%
\end{pgfscope}%
\begin{pgfscope}%
\pgftext[x=7.863830in,y=1.495238in,left,base]{\rmfamily\fontsize{12.000000}{14.400000}\selectfont \(\displaystyle 10^{-1}\)}%
\end{pgfscope}%
\begin{pgfscope}%
\pgfsetbuttcap%
\pgfsetroundjoin%
\definecolor{currentfill}{rgb}{0.000000,0.000000,0.000000}%
\pgfsetfillcolor{currentfill}%
\pgfsetlinewidth{0.803000pt}%
\definecolor{currentstroke}{rgb}{0.000000,0.000000,0.000000}%
\pgfsetstrokecolor{currentstroke}%
\pgfsetdash{}{0pt}%
\pgfsys@defobject{currentmarker}{\pgfqpoint{-0.048611in}{0.000000in}}{\pgfqpoint{0.000000in}{0.000000in}}{%
\pgfpathmoveto{\pgfqpoint{0.000000in}{0.000000in}}%
\pgfpathlineto{\pgfqpoint{-0.048611in}{0.000000in}}%
\pgfusepath{stroke,fill}%
}%
\begin{pgfscope}%
\pgfsys@transformshift{8.282041in}{2.109678in}%
\pgfsys@useobject{currentmarker}{}%
\end{pgfscope}%
\end{pgfscope}%
\begin{pgfscope}%
\pgftext[x=7.955653in,y=2.046364in,left,base]{\rmfamily\fontsize{12.000000}{14.400000}\selectfont \(\displaystyle 10^{0}\)}%
\end{pgfscope}%
\begin{pgfscope}%
\pgfsetbuttcap%
\pgfsetroundjoin%
\definecolor{currentfill}{rgb}{0.000000,0.000000,0.000000}%
\pgfsetfillcolor{currentfill}%
\pgfsetlinewidth{0.602250pt}%
\definecolor{currentstroke}{rgb}{0.000000,0.000000,0.000000}%
\pgfsetstrokecolor{currentstroke}%
\pgfsetdash{}{0pt}%
\pgfsys@defobject{currentmarker}{\pgfqpoint{-0.027778in}{0.000000in}}{\pgfqpoint{0.000000in}{0.000000in}}{%
\pgfpathmoveto{\pgfqpoint{0.000000in}{0.000000in}}%
\pgfpathlineto{\pgfqpoint{-0.027778in}{0.000000in}}%
\pgfusepath{stroke,fill}%
}%
\begin{pgfscope}%
\pgfsys@transformshift{8.282041in}{0.788111in}%
\pgfsys@useobject{currentmarker}{}%
\end{pgfscope}%
\end{pgfscope}%
\begin{pgfscope}%
\pgfsetbuttcap%
\pgfsetroundjoin%
\definecolor{currentfill}{rgb}{0.000000,0.000000,0.000000}%
\pgfsetfillcolor{currentfill}%
\pgfsetlinewidth{0.602250pt}%
\definecolor{currentstroke}{rgb}{0.000000,0.000000,0.000000}%
\pgfsetstrokecolor{currentstroke}%
\pgfsetdash{}{0pt}%
\pgfsys@defobject{currentmarker}{\pgfqpoint{-0.027778in}{0.000000in}}{\pgfqpoint{0.000000in}{0.000000in}}{%
\pgfpathmoveto{\pgfqpoint{0.000000in}{0.000000in}}%
\pgfpathlineto{\pgfqpoint{-0.027778in}{0.000000in}}%
\pgfusepath{stroke,fill}%
}%
\begin{pgfscope}%
\pgfsys@transformshift{8.282041in}{0.841520in}%
\pgfsys@useobject{currentmarker}{}%
\end{pgfscope}%
\end{pgfscope}%
\begin{pgfscope}%
\pgfsetbuttcap%
\pgfsetroundjoin%
\definecolor{currentfill}{rgb}{0.000000,0.000000,0.000000}%
\pgfsetfillcolor{currentfill}%
\pgfsetlinewidth{0.602250pt}%
\definecolor{currentstroke}{rgb}{0.000000,0.000000,0.000000}%
\pgfsetstrokecolor{currentstroke}%
\pgfsetdash{}{0pt}%
\pgfsys@defobject{currentmarker}{\pgfqpoint{-0.027778in}{0.000000in}}{\pgfqpoint{0.000000in}{0.000000in}}{%
\pgfpathmoveto{\pgfqpoint{0.000000in}{0.000000in}}%
\pgfpathlineto{\pgfqpoint{-0.027778in}{0.000000in}}%
\pgfusepath{stroke,fill}%
}%
\begin{pgfscope}%
\pgfsys@transformshift{8.282041in}{0.885159in}%
\pgfsys@useobject{currentmarker}{}%
\end{pgfscope}%
\end{pgfscope}%
\begin{pgfscope}%
\pgfsetbuttcap%
\pgfsetroundjoin%
\definecolor{currentfill}{rgb}{0.000000,0.000000,0.000000}%
\pgfsetfillcolor{currentfill}%
\pgfsetlinewidth{0.602250pt}%
\definecolor{currentstroke}{rgb}{0.000000,0.000000,0.000000}%
\pgfsetstrokecolor{currentstroke}%
\pgfsetdash{}{0pt}%
\pgfsys@defobject{currentmarker}{\pgfqpoint{-0.027778in}{0.000000in}}{\pgfqpoint{0.000000in}{0.000000in}}{%
\pgfpathmoveto{\pgfqpoint{0.000000in}{0.000000in}}%
\pgfpathlineto{\pgfqpoint{-0.027778in}{0.000000in}}%
\pgfusepath{stroke,fill}%
}%
\begin{pgfscope}%
\pgfsys@transformshift{8.282041in}{0.922055in}%
\pgfsys@useobject{currentmarker}{}%
\end{pgfscope}%
\end{pgfscope}%
\begin{pgfscope}%
\pgfsetbuttcap%
\pgfsetroundjoin%
\definecolor{currentfill}{rgb}{0.000000,0.000000,0.000000}%
\pgfsetfillcolor{currentfill}%
\pgfsetlinewidth{0.602250pt}%
\definecolor{currentstroke}{rgb}{0.000000,0.000000,0.000000}%
\pgfsetstrokecolor{currentstroke}%
\pgfsetdash{}{0pt}%
\pgfsys@defobject{currentmarker}{\pgfqpoint{-0.027778in}{0.000000in}}{\pgfqpoint{0.000000in}{0.000000in}}{%
\pgfpathmoveto{\pgfqpoint{0.000000in}{0.000000in}}%
\pgfpathlineto{\pgfqpoint{-0.027778in}{0.000000in}}%
\pgfusepath{stroke,fill}%
}%
\begin{pgfscope}%
\pgfsys@transformshift{8.282041in}{0.954016in}%
\pgfsys@useobject{currentmarker}{}%
\end{pgfscope}%
\end{pgfscope}%
\begin{pgfscope}%
\pgfsetbuttcap%
\pgfsetroundjoin%
\definecolor{currentfill}{rgb}{0.000000,0.000000,0.000000}%
\pgfsetfillcolor{currentfill}%
\pgfsetlinewidth{0.602250pt}%
\definecolor{currentstroke}{rgb}{0.000000,0.000000,0.000000}%
\pgfsetstrokecolor{currentstroke}%
\pgfsetdash{}{0pt}%
\pgfsys@defobject{currentmarker}{\pgfqpoint{-0.027778in}{0.000000in}}{\pgfqpoint{0.000000in}{0.000000in}}{%
\pgfpathmoveto{\pgfqpoint{0.000000in}{0.000000in}}%
\pgfpathlineto{\pgfqpoint{-0.027778in}{0.000000in}}%
\pgfusepath{stroke,fill}%
}%
\begin{pgfscope}%
\pgfsys@transformshift{8.282041in}{0.982208in}%
\pgfsys@useobject{currentmarker}{}%
\end{pgfscope}%
\end{pgfscope}%
\begin{pgfscope}%
\pgfsetbuttcap%
\pgfsetroundjoin%
\definecolor{currentfill}{rgb}{0.000000,0.000000,0.000000}%
\pgfsetfillcolor{currentfill}%
\pgfsetlinewidth{0.602250pt}%
\definecolor{currentstroke}{rgb}{0.000000,0.000000,0.000000}%
\pgfsetstrokecolor{currentstroke}%
\pgfsetdash{}{0pt}%
\pgfsys@defobject{currentmarker}{\pgfqpoint{-0.027778in}{0.000000in}}{\pgfqpoint{0.000000in}{0.000000in}}{%
\pgfpathmoveto{\pgfqpoint{0.000000in}{0.000000in}}%
\pgfpathlineto{\pgfqpoint{-0.027778in}{0.000000in}}%
\pgfusepath{stroke,fill}%
}%
\begin{pgfscope}%
\pgfsys@transformshift{8.282041in}{1.173331in}%
\pgfsys@useobject{currentmarker}{}%
\end{pgfscope}%
\end{pgfscope}%
\begin{pgfscope}%
\pgfsetbuttcap%
\pgfsetroundjoin%
\definecolor{currentfill}{rgb}{0.000000,0.000000,0.000000}%
\pgfsetfillcolor{currentfill}%
\pgfsetlinewidth{0.602250pt}%
\definecolor{currentstroke}{rgb}{0.000000,0.000000,0.000000}%
\pgfsetstrokecolor{currentstroke}%
\pgfsetdash{}{0pt}%
\pgfsys@defobject{currentmarker}{\pgfqpoint{-0.027778in}{0.000000in}}{\pgfqpoint{0.000000in}{0.000000in}}{%
\pgfpathmoveto{\pgfqpoint{0.000000in}{0.000000in}}%
\pgfpathlineto{\pgfqpoint{-0.027778in}{0.000000in}}%
\pgfusepath{stroke,fill}%
}%
\begin{pgfscope}%
\pgfsys@transformshift{8.282041in}{1.270380in}%
\pgfsys@useobject{currentmarker}{}%
\end{pgfscope}%
\end{pgfscope}%
\begin{pgfscope}%
\pgfsetbuttcap%
\pgfsetroundjoin%
\definecolor{currentfill}{rgb}{0.000000,0.000000,0.000000}%
\pgfsetfillcolor{currentfill}%
\pgfsetlinewidth{0.602250pt}%
\definecolor{currentstroke}{rgb}{0.000000,0.000000,0.000000}%
\pgfsetstrokecolor{currentstroke}%
\pgfsetdash{}{0pt}%
\pgfsys@defobject{currentmarker}{\pgfqpoint{-0.027778in}{0.000000in}}{\pgfqpoint{0.000000in}{0.000000in}}{%
\pgfpathmoveto{\pgfqpoint{0.000000in}{0.000000in}}%
\pgfpathlineto{\pgfqpoint{-0.027778in}{0.000000in}}%
\pgfusepath{stroke,fill}%
}%
\begin{pgfscope}%
\pgfsys@transformshift{8.282041in}{1.339237in}%
\pgfsys@useobject{currentmarker}{}%
\end{pgfscope}%
\end{pgfscope}%
\begin{pgfscope}%
\pgfsetbuttcap%
\pgfsetroundjoin%
\definecolor{currentfill}{rgb}{0.000000,0.000000,0.000000}%
\pgfsetfillcolor{currentfill}%
\pgfsetlinewidth{0.602250pt}%
\definecolor{currentstroke}{rgb}{0.000000,0.000000,0.000000}%
\pgfsetstrokecolor{currentstroke}%
\pgfsetdash{}{0pt}%
\pgfsys@defobject{currentmarker}{\pgfqpoint{-0.027778in}{0.000000in}}{\pgfqpoint{0.000000in}{0.000000in}}{%
\pgfpathmoveto{\pgfqpoint{0.000000in}{0.000000in}}%
\pgfpathlineto{\pgfqpoint{-0.027778in}{0.000000in}}%
\pgfusepath{stroke,fill}%
}%
\begin{pgfscope}%
\pgfsys@transformshift{8.282041in}{1.392646in}%
\pgfsys@useobject{currentmarker}{}%
\end{pgfscope}%
\end{pgfscope}%
\begin{pgfscope}%
\pgfsetbuttcap%
\pgfsetroundjoin%
\definecolor{currentfill}{rgb}{0.000000,0.000000,0.000000}%
\pgfsetfillcolor{currentfill}%
\pgfsetlinewidth{0.602250pt}%
\definecolor{currentstroke}{rgb}{0.000000,0.000000,0.000000}%
\pgfsetstrokecolor{currentstroke}%
\pgfsetdash{}{0pt}%
\pgfsys@defobject{currentmarker}{\pgfqpoint{-0.027778in}{0.000000in}}{\pgfqpoint{0.000000in}{0.000000in}}{%
\pgfpathmoveto{\pgfqpoint{0.000000in}{0.000000in}}%
\pgfpathlineto{\pgfqpoint{-0.027778in}{0.000000in}}%
\pgfusepath{stroke,fill}%
}%
\begin{pgfscope}%
\pgfsys@transformshift{8.282041in}{1.436285in}%
\pgfsys@useobject{currentmarker}{}%
\end{pgfscope}%
\end{pgfscope}%
\begin{pgfscope}%
\pgfsetbuttcap%
\pgfsetroundjoin%
\definecolor{currentfill}{rgb}{0.000000,0.000000,0.000000}%
\pgfsetfillcolor{currentfill}%
\pgfsetlinewidth{0.602250pt}%
\definecolor{currentstroke}{rgb}{0.000000,0.000000,0.000000}%
\pgfsetstrokecolor{currentstroke}%
\pgfsetdash{}{0pt}%
\pgfsys@defobject{currentmarker}{\pgfqpoint{-0.027778in}{0.000000in}}{\pgfqpoint{0.000000in}{0.000000in}}{%
\pgfpathmoveto{\pgfqpoint{0.000000in}{0.000000in}}%
\pgfpathlineto{\pgfqpoint{-0.027778in}{0.000000in}}%
\pgfusepath{stroke,fill}%
}%
\begin{pgfscope}%
\pgfsys@transformshift{8.282041in}{1.473181in}%
\pgfsys@useobject{currentmarker}{}%
\end{pgfscope}%
\end{pgfscope}%
\begin{pgfscope}%
\pgfsetbuttcap%
\pgfsetroundjoin%
\definecolor{currentfill}{rgb}{0.000000,0.000000,0.000000}%
\pgfsetfillcolor{currentfill}%
\pgfsetlinewidth{0.602250pt}%
\definecolor{currentstroke}{rgb}{0.000000,0.000000,0.000000}%
\pgfsetstrokecolor{currentstroke}%
\pgfsetdash{}{0pt}%
\pgfsys@defobject{currentmarker}{\pgfqpoint{-0.027778in}{0.000000in}}{\pgfqpoint{0.000000in}{0.000000in}}{%
\pgfpathmoveto{\pgfqpoint{0.000000in}{0.000000in}}%
\pgfpathlineto{\pgfqpoint{-0.027778in}{0.000000in}}%
\pgfusepath{stroke,fill}%
}%
\begin{pgfscope}%
\pgfsys@transformshift{8.282041in}{1.505142in}%
\pgfsys@useobject{currentmarker}{}%
\end{pgfscope}%
\end{pgfscope}%
\begin{pgfscope}%
\pgfsetbuttcap%
\pgfsetroundjoin%
\definecolor{currentfill}{rgb}{0.000000,0.000000,0.000000}%
\pgfsetfillcolor{currentfill}%
\pgfsetlinewidth{0.602250pt}%
\definecolor{currentstroke}{rgb}{0.000000,0.000000,0.000000}%
\pgfsetstrokecolor{currentstroke}%
\pgfsetdash{}{0pt}%
\pgfsys@defobject{currentmarker}{\pgfqpoint{-0.027778in}{0.000000in}}{\pgfqpoint{0.000000in}{0.000000in}}{%
\pgfpathmoveto{\pgfqpoint{0.000000in}{0.000000in}}%
\pgfpathlineto{\pgfqpoint{-0.027778in}{0.000000in}}%
\pgfusepath{stroke,fill}%
}%
\begin{pgfscope}%
\pgfsys@transformshift{8.282041in}{1.533334in}%
\pgfsys@useobject{currentmarker}{}%
\end{pgfscope}%
\end{pgfscope}%
\begin{pgfscope}%
\pgfsetbuttcap%
\pgfsetroundjoin%
\definecolor{currentfill}{rgb}{0.000000,0.000000,0.000000}%
\pgfsetfillcolor{currentfill}%
\pgfsetlinewidth{0.602250pt}%
\definecolor{currentstroke}{rgb}{0.000000,0.000000,0.000000}%
\pgfsetstrokecolor{currentstroke}%
\pgfsetdash{}{0pt}%
\pgfsys@defobject{currentmarker}{\pgfqpoint{-0.027778in}{0.000000in}}{\pgfqpoint{0.000000in}{0.000000in}}{%
\pgfpathmoveto{\pgfqpoint{0.000000in}{0.000000in}}%
\pgfpathlineto{\pgfqpoint{-0.027778in}{0.000000in}}%
\pgfusepath{stroke,fill}%
}%
\begin{pgfscope}%
\pgfsys@transformshift{8.282041in}{1.724457in}%
\pgfsys@useobject{currentmarker}{}%
\end{pgfscope}%
\end{pgfscope}%
\begin{pgfscope}%
\pgfsetbuttcap%
\pgfsetroundjoin%
\definecolor{currentfill}{rgb}{0.000000,0.000000,0.000000}%
\pgfsetfillcolor{currentfill}%
\pgfsetlinewidth{0.602250pt}%
\definecolor{currentstroke}{rgb}{0.000000,0.000000,0.000000}%
\pgfsetstrokecolor{currentstroke}%
\pgfsetdash{}{0pt}%
\pgfsys@defobject{currentmarker}{\pgfqpoint{-0.027778in}{0.000000in}}{\pgfqpoint{0.000000in}{0.000000in}}{%
\pgfpathmoveto{\pgfqpoint{0.000000in}{0.000000in}}%
\pgfpathlineto{\pgfqpoint{-0.027778in}{0.000000in}}%
\pgfusepath{stroke,fill}%
}%
\begin{pgfscope}%
\pgfsys@transformshift{8.282041in}{1.821506in}%
\pgfsys@useobject{currentmarker}{}%
\end{pgfscope}%
\end{pgfscope}%
\begin{pgfscope}%
\pgfsetbuttcap%
\pgfsetroundjoin%
\definecolor{currentfill}{rgb}{0.000000,0.000000,0.000000}%
\pgfsetfillcolor{currentfill}%
\pgfsetlinewidth{0.602250pt}%
\definecolor{currentstroke}{rgb}{0.000000,0.000000,0.000000}%
\pgfsetstrokecolor{currentstroke}%
\pgfsetdash{}{0pt}%
\pgfsys@defobject{currentmarker}{\pgfqpoint{-0.027778in}{0.000000in}}{\pgfqpoint{0.000000in}{0.000000in}}{%
\pgfpathmoveto{\pgfqpoint{0.000000in}{0.000000in}}%
\pgfpathlineto{\pgfqpoint{-0.027778in}{0.000000in}}%
\pgfusepath{stroke,fill}%
}%
\begin{pgfscope}%
\pgfsys@transformshift{8.282041in}{1.890363in}%
\pgfsys@useobject{currentmarker}{}%
\end{pgfscope}%
\end{pgfscope}%
\begin{pgfscope}%
\pgfsetbuttcap%
\pgfsetroundjoin%
\definecolor{currentfill}{rgb}{0.000000,0.000000,0.000000}%
\pgfsetfillcolor{currentfill}%
\pgfsetlinewidth{0.602250pt}%
\definecolor{currentstroke}{rgb}{0.000000,0.000000,0.000000}%
\pgfsetstrokecolor{currentstroke}%
\pgfsetdash{}{0pt}%
\pgfsys@defobject{currentmarker}{\pgfqpoint{-0.027778in}{0.000000in}}{\pgfqpoint{0.000000in}{0.000000in}}{%
\pgfpathmoveto{\pgfqpoint{0.000000in}{0.000000in}}%
\pgfpathlineto{\pgfqpoint{-0.027778in}{0.000000in}}%
\pgfusepath{stroke,fill}%
}%
\begin{pgfscope}%
\pgfsys@transformshift{8.282041in}{1.943772in}%
\pgfsys@useobject{currentmarker}{}%
\end{pgfscope}%
\end{pgfscope}%
\begin{pgfscope}%
\pgfsetbuttcap%
\pgfsetroundjoin%
\definecolor{currentfill}{rgb}{0.000000,0.000000,0.000000}%
\pgfsetfillcolor{currentfill}%
\pgfsetlinewidth{0.602250pt}%
\definecolor{currentstroke}{rgb}{0.000000,0.000000,0.000000}%
\pgfsetstrokecolor{currentstroke}%
\pgfsetdash{}{0pt}%
\pgfsys@defobject{currentmarker}{\pgfqpoint{-0.027778in}{0.000000in}}{\pgfqpoint{0.000000in}{0.000000in}}{%
\pgfpathmoveto{\pgfqpoint{0.000000in}{0.000000in}}%
\pgfpathlineto{\pgfqpoint{-0.027778in}{0.000000in}}%
\pgfusepath{stroke,fill}%
}%
\begin{pgfscope}%
\pgfsys@transformshift{8.282041in}{1.987411in}%
\pgfsys@useobject{currentmarker}{}%
\end{pgfscope}%
\end{pgfscope}%
\begin{pgfscope}%
\pgfsetbuttcap%
\pgfsetroundjoin%
\definecolor{currentfill}{rgb}{0.000000,0.000000,0.000000}%
\pgfsetfillcolor{currentfill}%
\pgfsetlinewidth{0.602250pt}%
\definecolor{currentstroke}{rgb}{0.000000,0.000000,0.000000}%
\pgfsetstrokecolor{currentstroke}%
\pgfsetdash{}{0pt}%
\pgfsys@defobject{currentmarker}{\pgfqpoint{-0.027778in}{0.000000in}}{\pgfqpoint{0.000000in}{0.000000in}}{%
\pgfpathmoveto{\pgfqpoint{0.000000in}{0.000000in}}%
\pgfpathlineto{\pgfqpoint{-0.027778in}{0.000000in}}%
\pgfusepath{stroke,fill}%
}%
\begin{pgfscope}%
\pgfsys@transformshift{8.282041in}{2.024307in}%
\pgfsys@useobject{currentmarker}{}%
\end{pgfscope}%
\end{pgfscope}%
\begin{pgfscope}%
\pgfsetbuttcap%
\pgfsetroundjoin%
\definecolor{currentfill}{rgb}{0.000000,0.000000,0.000000}%
\pgfsetfillcolor{currentfill}%
\pgfsetlinewidth{0.602250pt}%
\definecolor{currentstroke}{rgb}{0.000000,0.000000,0.000000}%
\pgfsetstrokecolor{currentstroke}%
\pgfsetdash{}{0pt}%
\pgfsys@defobject{currentmarker}{\pgfqpoint{-0.027778in}{0.000000in}}{\pgfqpoint{0.000000in}{0.000000in}}{%
\pgfpathmoveto{\pgfqpoint{0.000000in}{0.000000in}}%
\pgfpathlineto{\pgfqpoint{-0.027778in}{0.000000in}}%
\pgfusepath{stroke,fill}%
}%
\begin{pgfscope}%
\pgfsys@transformshift{8.282041in}{2.056268in}%
\pgfsys@useobject{currentmarker}{}%
\end{pgfscope}%
\end{pgfscope}%
\begin{pgfscope}%
\pgfsetbuttcap%
\pgfsetroundjoin%
\definecolor{currentfill}{rgb}{0.000000,0.000000,0.000000}%
\pgfsetfillcolor{currentfill}%
\pgfsetlinewidth{0.602250pt}%
\definecolor{currentstroke}{rgb}{0.000000,0.000000,0.000000}%
\pgfsetstrokecolor{currentstroke}%
\pgfsetdash{}{0pt}%
\pgfsys@defobject{currentmarker}{\pgfqpoint{-0.027778in}{0.000000in}}{\pgfqpoint{0.000000in}{0.000000in}}{%
\pgfpathmoveto{\pgfqpoint{0.000000in}{0.000000in}}%
\pgfpathlineto{\pgfqpoint{-0.027778in}{0.000000in}}%
\pgfusepath{stroke,fill}%
}%
\begin{pgfscope}%
\pgfsys@transformshift{8.282041in}{2.084460in}%
\pgfsys@useobject{currentmarker}{}%
\end{pgfscope}%
\end{pgfscope}%
\begin{pgfscope}%
\pgfpathrectangle{\pgfqpoint{8.282041in}{0.790446in}}{\pgfqpoint{1.897959in}{1.372727in}}%
\pgfusepath{clip}%
\pgfsetrectcap%
\pgfsetroundjoin%
\pgfsetlinewidth{1.505625pt}%
\definecolor{currentstroke}{rgb}{1.000000,0.000000,0.000000}%
\pgfsetstrokecolor{currentstroke}%
\pgfsetstrokeopacity{0.750000}%
\pgfsetdash{}{0pt}%
\pgfpathmoveto{\pgfqpoint{8.411294in}{2.052880in}}%
\pgfpathlineto{\pgfqpoint{8.421904in}{2.057666in}}%
\pgfpathlineto{\pgfqpoint{8.432058in}{2.062209in}}%
\pgfpathlineto{\pgfqpoint{8.441757in}{2.065973in}}%
\pgfusepath{stroke}%
\end{pgfscope}%
\begin{pgfscope}%
\pgfpathrectangle{\pgfqpoint{8.282041in}{0.790446in}}{\pgfqpoint{1.897959in}{1.372727in}}%
\pgfusepath{clip}%
\pgfsetrectcap%
\pgfsetroundjoin%
\pgfsetlinewidth{1.505625pt}%
\definecolor{currentstroke}{rgb}{0.000000,0.000000,1.000000}%
\pgfsetstrokecolor{currentstroke}%
\pgfsetstrokeopacity{0.750000}%
\pgfsetdash{}{0pt}%
\pgfpathmoveto{\pgfqpoint{8.411294in}{1.128135in}}%
\pgfpathlineto{\pgfqpoint{8.421904in}{1.120198in}}%
\pgfpathlineto{\pgfqpoint{8.432058in}{1.111678in}}%
\pgfpathlineto{\pgfqpoint{8.441757in}{1.101956in}}%
\pgfusepath{stroke}%
\end{pgfscope}%
\begin{pgfscope}%
\pgfpathrectangle{\pgfqpoint{8.282041in}{0.790446in}}{\pgfqpoint{1.897959in}{1.372727in}}%
\pgfusepath{clip}%
\pgfsetrectcap%
\pgfsetroundjoin%
\pgfsetlinewidth{1.505625pt}%
\definecolor{currentstroke}{rgb}{0.000000,0.750000,0.750000}%
\pgfsetstrokecolor{currentstroke}%
\pgfsetstrokeopacity{0.750000}%
\pgfsetdash{}{0pt}%
\pgfpathmoveto{\pgfqpoint{8.411294in}{1.720466in}}%
\pgfpathlineto{\pgfqpoint{8.421904in}{1.697019in}}%
\pgfpathlineto{\pgfqpoint{8.432058in}{1.676502in}}%
\pgfpathlineto{\pgfqpoint{8.441757in}{1.657900in}}%
\pgfusepath{stroke}%
\end{pgfscope}%
\begin{pgfscope}%
\pgfpathrectangle{\pgfqpoint{8.282041in}{0.790446in}}{\pgfqpoint{1.897959in}{1.372727in}}%
\pgfusepath{clip}%
\pgfsetrectcap%
\pgfsetroundjoin%
\pgfsetlinewidth{1.505625pt}%
\definecolor{currentstroke}{rgb}{1.000000,0.000000,0.000000}%
\pgfsetstrokecolor{currentstroke}%
\pgfsetstrokeopacity{0.750000}%
\pgfsetdash{}{0pt}%
\pgfpathmoveto{\pgfqpoint{8.400709in}{2.073914in}}%
\pgfpathlineto{\pgfqpoint{8.412792in}{2.077111in}}%
\pgfpathlineto{\pgfqpoint{8.424381in}{2.080143in}}%
\pgfpathlineto{\pgfqpoint{8.435482in}{2.082627in}}%
\pgfpathlineto{\pgfqpoint{8.446099in}{2.084706in}}%
\pgfpathlineto{\pgfqpoint{8.456237in}{2.086476in}}%
\pgfpathlineto{\pgfqpoint{8.465901in}{2.088008in}}%
\pgfpathlineto{\pgfqpoint{8.475095in}{2.089351in}}%
\pgfusepath{stroke}%
\end{pgfscope}%
\begin{pgfscope}%
\pgfpathrectangle{\pgfqpoint{8.282041in}{0.790446in}}{\pgfqpoint{1.897959in}{1.372727in}}%
\pgfusepath{clip}%
\pgfsetrectcap%
\pgfsetroundjoin%
\pgfsetlinewidth{1.505625pt}%
\definecolor{currentstroke}{rgb}{0.000000,0.000000,1.000000}%
\pgfsetstrokecolor{currentstroke}%
\pgfsetstrokeopacity{0.750000}%
\pgfsetdash{}{0pt}%
\pgfpathmoveto{\pgfqpoint{8.400709in}{1.302243in}}%
\pgfpathlineto{\pgfqpoint{8.412792in}{1.295197in}}%
\pgfpathlineto{\pgfqpoint{8.424381in}{1.287687in}}%
\pgfpathlineto{\pgfqpoint{8.435482in}{1.279273in}}%
\pgfpathlineto{\pgfqpoint{8.446099in}{1.270045in}}%
\pgfpathlineto{\pgfqpoint{8.456237in}{1.260049in}}%
\pgfpathlineto{\pgfqpoint{8.465901in}{1.249301in}}%
\pgfpathlineto{\pgfqpoint{8.475095in}{1.237798in}}%
\pgfusepath{stroke}%
\end{pgfscope}%
\begin{pgfscope}%
\pgfpathrectangle{\pgfqpoint{8.282041in}{0.790446in}}{\pgfqpoint{1.897959in}{1.372727in}}%
\pgfusepath{clip}%
\pgfsetrectcap%
\pgfsetroundjoin%
\pgfsetlinewidth{1.505625pt}%
\definecolor{currentstroke}{rgb}{0.000000,0.750000,0.750000}%
\pgfsetstrokecolor{currentstroke}%
\pgfsetstrokeopacity{0.750000}%
\pgfsetdash{}{0pt}%
\pgfpathmoveto{\pgfqpoint{8.400709in}{1.572577in}}%
\pgfpathlineto{\pgfqpoint{8.412792in}{1.541753in}}%
\pgfpathlineto{\pgfqpoint{8.424381in}{1.514786in}}%
\pgfpathlineto{\pgfqpoint{8.435482in}{1.490625in}}%
\pgfpathlineto{\pgfqpoint{8.446099in}{1.468897in}}%
\pgfpathlineto{\pgfqpoint{8.456237in}{1.449295in}}%
\pgfpathlineto{\pgfqpoint{8.465901in}{1.431566in}}%
\pgfpathlineto{\pgfqpoint{8.475095in}{1.415499in}}%
\pgfusepath{stroke}%
\end{pgfscope}%
\begin{pgfscope}%
\pgfpathrectangle{\pgfqpoint{8.282041in}{0.790446in}}{\pgfqpoint{1.897959in}{1.372727in}}%
\pgfusepath{clip}%
\pgfsetrectcap%
\pgfsetroundjoin%
\pgfsetlinewidth{1.505625pt}%
\definecolor{currentstroke}{rgb}{1.000000,0.000000,0.000000}%
\pgfsetstrokecolor{currentstroke}%
\pgfsetstrokeopacity{0.750000}%
\pgfsetdash{}{0pt}%
\pgfpathmoveto{\pgfqpoint{8.431628in}{2.047988in}}%
\pgfpathlineto{\pgfqpoint{8.446301in}{2.054002in}}%
\pgfpathlineto{\pgfqpoint{8.460331in}{2.059692in}}%
\pgfpathlineto{\pgfqpoint{8.473725in}{2.064271in}}%
\pgfpathlineto{\pgfqpoint{8.486490in}{2.068020in}}%
\pgfpathlineto{\pgfqpoint{8.498633in}{2.071135in}}%
\pgfpathlineto{\pgfqpoint{8.510160in}{2.073755in}}%
\pgfusepath{stroke}%
\end{pgfscope}%
\begin{pgfscope}%
\pgfpathrectangle{\pgfqpoint{8.282041in}{0.790446in}}{\pgfqpoint{1.897959in}{1.372727in}}%
\pgfusepath{clip}%
\pgfsetrectcap%
\pgfsetroundjoin%
\pgfsetlinewidth{1.505625pt}%
\definecolor{currentstroke}{rgb}{0.000000,0.000000,1.000000}%
\pgfsetstrokecolor{currentstroke}%
\pgfsetstrokeopacity{0.750000}%
\pgfsetdash{}{0pt}%
\pgfpathmoveto{\pgfqpoint{8.431628in}{1.048023in}}%
\pgfpathlineto{\pgfqpoint{8.446301in}{1.041717in}}%
\pgfpathlineto{\pgfqpoint{8.460331in}{1.034878in}}%
\pgfpathlineto{\pgfqpoint{8.473725in}{1.026611in}}%
\pgfpathlineto{\pgfqpoint{8.486490in}{1.017108in}}%
\pgfpathlineto{\pgfqpoint{8.498633in}{1.006481in}}%
\pgfpathlineto{\pgfqpoint{8.510160in}{0.994792in}}%
\pgfusepath{stroke}%
\end{pgfscope}%
\begin{pgfscope}%
\pgfpathrectangle{\pgfqpoint{8.282041in}{0.790446in}}{\pgfqpoint{1.897959in}{1.372727in}}%
\pgfusepath{clip}%
\pgfsetrectcap%
\pgfsetroundjoin%
\pgfsetlinewidth{1.505625pt}%
\definecolor{currentstroke}{rgb}{0.000000,0.750000,0.750000}%
\pgfsetstrokecolor{currentstroke}%
\pgfsetstrokeopacity{0.750000}%
\pgfsetdash{}{0pt}%
\pgfpathmoveto{\pgfqpoint{8.431628in}{1.745525in}}%
\pgfpathlineto{\pgfqpoint{8.446301in}{1.717843in}}%
\pgfpathlineto{\pgfqpoint{8.460331in}{1.694082in}}%
\pgfpathlineto{\pgfqpoint{8.473725in}{1.672706in}}%
\pgfpathlineto{\pgfqpoint{8.486490in}{1.653433in}}%
\pgfpathlineto{\pgfqpoint{8.498633in}{1.636019in}}%
\pgfpathlineto{\pgfqpoint{8.510160in}{1.620259in}}%
\pgfusepath{stroke}%
\end{pgfscope}%
\begin{pgfscope}%
\pgfpathrectangle{\pgfqpoint{8.282041in}{0.790446in}}{\pgfqpoint{1.897959in}{1.372727in}}%
\pgfusepath{clip}%
\pgfsetrectcap%
\pgfsetroundjoin%
\pgfsetlinewidth{1.505625pt}%
\definecolor{currentstroke}{rgb}{1.000000,0.000000,0.000000}%
\pgfsetstrokecolor{currentstroke}%
\pgfsetstrokeopacity{0.750000}%
\pgfsetdash{}{0pt}%
\pgfpathmoveto{\pgfqpoint{8.368312in}{2.041640in}}%
\pgfpathlineto{\pgfqpoint{8.380745in}{2.048401in}}%
\pgfpathlineto{\pgfqpoint{8.393002in}{2.054616in}}%
\pgfpathlineto{\pgfqpoint{8.405081in}{2.059459in}}%
\pgfpathlineto{\pgfqpoint{8.416983in}{2.063327in}}%
\pgfpathlineto{\pgfqpoint{8.428704in}{2.066477in}}%
\pgfpathlineto{\pgfqpoint{8.440245in}{2.069092in}}%
\pgfpathlineto{\pgfqpoint{8.451604in}{2.071296in}}%
\pgfpathlineto{\pgfqpoint{8.462779in}{2.073181in}}%
\pgfpathlineto{\pgfqpoint{8.473770in}{2.074814in}}%
\pgfpathlineto{\pgfqpoint{8.484575in}{2.076245in}}%
\pgfpathlineto{\pgfqpoint{8.495193in}{2.077513in}}%
\pgfpathlineto{\pgfqpoint{8.505623in}{2.078648in}}%
\pgfpathlineto{\pgfqpoint{8.515859in}{2.079671in}}%
\pgfpathlineto{\pgfqpoint{8.525902in}{2.080603in}}%
\pgfpathlineto{\pgfqpoint{8.535751in}{2.081457in}}%
\pgfpathlineto{\pgfqpoint{8.545405in}{2.082245in}}%
\pgfpathlineto{\pgfqpoint{8.554865in}{2.082979in}}%
\pgfpathlineto{\pgfqpoint{8.564130in}{2.083664in}}%
\pgfpathlineto{\pgfqpoint{8.573202in}{2.084309in}}%
\pgfpathlineto{\pgfqpoint{8.582083in}{2.084918in}}%
\pgfpathlineto{\pgfqpoint{8.590772in}{2.085496in}}%
\pgfpathlineto{\pgfqpoint{8.599274in}{2.086048in}}%
\pgfpathlineto{\pgfqpoint{8.607590in}{2.086576in}}%
\pgfpathlineto{\pgfqpoint{8.615724in}{2.087083in}}%
\pgfpathlineto{\pgfqpoint{8.623677in}{2.087572in}}%
\pgfpathlineto{\pgfqpoint{8.631452in}{2.088045in}}%
\pgfpathlineto{\pgfqpoint{8.639053in}{2.088504in}}%
\pgfpathlineto{\pgfqpoint{8.646481in}{2.088950in}}%
\pgfpathlineto{\pgfqpoint{8.653738in}{2.089385in}}%
\pgfpathlineto{\pgfqpoint{8.660828in}{2.089809in}}%
\pgfpathlineto{\pgfqpoint{8.667751in}{2.090224in}}%
\pgfpathlineto{\pgfqpoint{8.674509in}{2.090631in}}%
\pgfpathlineto{\pgfqpoint{8.681104in}{2.091031in}}%
\pgfpathlineto{\pgfqpoint{8.687537in}{2.091423in}}%
\pgfpathlineto{\pgfqpoint{8.693809in}{2.091810in}}%
\pgfpathlineto{\pgfqpoint{8.699921in}{2.092191in}}%
\pgfpathlineto{\pgfqpoint{8.705874in}{2.092566in}}%
\pgfpathlineto{\pgfqpoint{8.711670in}{2.092937in}}%
\pgfpathlineto{\pgfqpoint{8.717309in}{2.093304in}}%
\pgfpathlineto{\pgfqpoint{8.722792in}{2.093666in}}%
\pgfpathlineto{\pgfqpoint{8.728120in}{2.094024in}}%
\pgfpathlineto{\pgfqpoint{8.733295in}{2.094379in}}%
\pgfpathlineto{\pgfqpoint{8.738318in}{2.094730in}}%
\pgfpathlineto{\pgfqpoint{8.743189in}{2.095078in}}%
\pgfpathlineto{\pgfqpoint{8.747910in}{2.095423in}}%
\pgfpathlineto{\pgfqpoint{8.752483in}{2.095765in}}%
\pgfpathlineto{\pgfqpoint{8.756907in}{2.096103in}}%
\pgfpathlineto{\pgfqpoint{8.761185in}{2.096439in}}%
\pgfpathlineto{\pgfqpoint{8.765317in}{2.096772in}}%
\pgfpathlineto{\pgfqpoint{8.769304in}{2.097103in}}%
\pgfpathlineto{\pgfqpoint{8.773147in}{2.097430in}}%
\pgfpathlineto{\pgfqpoint{8.776848in}{2.097755in}}%
\pgfpathlineto{\pgfqpoint{8.780406in}{2.098077in}}%
\pgfpathlineto{\pgfqpoint{8.783822in}{2.098397in}}%
\pgfpathlineto{\pgfqpoint{8.787097in}{2.098714in}}%
\pgfpathlineto{\pgfqpoint{8.790232in}{2.099028in}}%
\pgfpathlineto{\pgfqpoint{8.793227in}{2.099338in}}%
\pgfpathlineto{\pgfqpoint{8.796082in}{2.099646in}}%
\pgfusepath{stroke}%
\end{pgfscope}%
\begin{pgfscope}%
\pgfpathrectangle{\pgfqpoint{8.282041in}{0.790446in}}{\pgfqpoint{1.897959in}{1.372727in}}%
\pgfusepath{clip}%
\pgfsetrectcap%
\pgfsetroundjoin%
\pgfsetlinewidth{1.505625pt}%
\definecolor{currentstroke}{rgb}{0.000000,0.000000,1.000000}%
\pgfsetstrokecolor{currentstroke}%
\pgfsetstrokeopacity{0.750000}%
\pgfsetdash{}{0pt}%
\pgfpathmoveto{\pgfqpoint{8.368312in}{1.470268in}}%
\pgfpathlineto{\pgfqpoint{8.380745in}{1.473781in}}%
\pgfpathlineto{\pgfqpoint{8.393002in}{1.476863in}}%
\pgfpathlineto{\pgfqpoint{8.405081in}{1.478644in}}%
\pgfpathlineto{\pgfqpoint{8.416983in}{1.479489in}}%
\pgfpathlineto{\pgfqpoint{8.428704in}{1.479633in}}%
\pgfpathlineto{\pgfqpoint{8.440245in}{1.479241in}}%
\pgfpathlineto{\pgfqpoint{8.451604in}{1.478425in}}%
\pgfpathlineto{\pgfqpoint{8.462779in}{1.477266in}}%
\pgfpathlineto{\pgfqpoint{8.473770in}{1.475822in}}%
\pgfpathlineto{\pgfqpoint{8.484575in}{1.474135in}}%
\pgfpathlineto{\pgfqpoint{8.495193in}{1.472239in}}%
\pgfpathlineto{\pgfqpoint{8.505623in}{1.470156in}}%
\pgfpathlineto{\pgfqpoint{8.515859in}{1.467906in}}%
\pgfpathlineto{\pgfqpoint{8.525902in}{1.465501in}}%
\pgfpathlineto{\pgfqpoint{8.535751in}{1.462952in}}%
\pgfpathlineto{\pgfqpoint{8.545405in}{1.460268in}}%
\pgfpathlineto{\pgfqpoint{8.554865in}{1.457455in}}%
\pgfpathlineto{\pgfqpoint{8.564130in}{1.454515in}}%
\pgfpathlineto{\pgfqpoint{8.573202in}{1.451454in}}%
\pgfpathlineto{\pgfqpoint{8.582083in}{1.448272in}}%
\pgfpathlineto{\pgfqpoint{8.590772in}{1.444971in}}%
\pgfpathlineto{\pgfqpoint{8.599274in}{1.441552in}}%
\pgfpathlineto{\pgfqpoint{8.607590in}{1.438013in}}%
\pgfpathlineto{\pgfqpoint{8.615724in}{1.434356in}}%
\pgfpathlineto{\pgfqpoint{8.623677in}{1.430578in}}%
\pgfpathlineto{\pgfqpoint{8.631452in}{1.426678in}}%
\pgfpathlineto{\pgfqpoint{8.639053in}{1.422654in}}%
\pgfpathlineto{\pgfqpoint{8.646481in}{1.418504in}}%
\pgfpathlineto{\pgfqpoint{8.653738in}{1.414225in}}%
\pgfpathlineto{\pgfqpoint{8.660828in}{1.409813in}}%
\pgfpathlineto{\pgfqpoint{8.667751in}{1.405267in}}%
\pgfpathlineto{\pgfqpoint{8.674509in}{1.400582in}}%
\pgfpathlineto{\pgfqpoint{8.681104in}{1.395754in}}%
\pgfpathlineto{\pgfqpoint{8.687537in}{1.390778in}}%
\pgfpathlineto{\pgfqpoint{8.693809in}{1.385650in}}%
\pgfpathlineto{\pgfqpoint{8.699921in}{1.380366in}}%
\pgfpathlineto{\pgfqpoint{8.705874in}{1.374919in}}%
\pgfpathlineto{\pgfqpoint{8.711670in}{1.369303in}}%
\pgfpathlineto{\pgfqpoint{8.717309in}{1.363512in}}%
\pgfpathlineto{\pgfqpoint{8.722792in}{1.357540in}}%
\pgfpathlineto{\pgfqpoint{8.728120in}{1.351378in}}%
\pgfpathlineto{\pgfqpoint{8.733295in}{1.345018in}}%
\pgfpathlineto{\pgfqpoint{8.738318in}{1.338452in}}%
\pgfpathlineto{\pgfqpoint{8.743189in}{1.331671in}}%
\pgfpathlineto{\pgfqpoint{8.747910in}{1.324663in}}%
\pgfpathlineto{\pgfqpoint{8.752483in}{1.317418in}}%
\pgfpathlineto{\pgfqpoint{8.756907in}{1.309923in}}%
\pgfpathlineto{\pgfqpoint{8.761185in}{1.302166in}}%
\pgfpathlineto{\pgfqpoint{8.765317in}{1.294131in}}%
\pgfpathlineto{\pgfqpoint{8.769304in}{1.285802in}}%
\pgfpathlineto{\pgfqpoint{8.773147in}{1.277163in}}%
\pgfpathlineto{\pgfqpoint{8.776848in}{1.268193in}}%
\pgfpathlineto{\pgfqpoint{8.780406in}{1.258871in}}%
\pgfpathlineto{\pgfqpoint{8.783822in}{1.249173in}}%
\pgfpathlineto{\pgfqpoint{8.787097in}{1.239073in}}%
\pgfpathlineto{\pgfqpoint{8.790232in}{1.228541in}}%
\pgfpathlineto{\pgfqpoint{8.793227in}{1.217542in}}%
\pgfpathlineto{\pgfqpoint{8.796082in}{1.206039in}}%
\pgfusepath{stroke}%
\end{pgfscope}%
\begin{pgfscope}%
\pgfpathrectangle{\pgfqpoint{8.282041in}{0.790446in}}{\pgfqpoint{1.897959in}{1.372727in}}%
\pgfusepath{clip}%
\pgfsetrectcap%
\pgfsetroundjoin%
\pgfsetlinewidth{1.505625pt}%
\definecolor{currentstroke}{rgb}{0.000000,0.750000,0.750000}%
\pgfsetstrokecolor{currentstroke}%
\pgfsetstrokeopacity{0.750000}%
\pgfsetdash{}{0pt}%
\pgfpathmoveto{\pgfqpoint{8.368312in}{1.701666in}}%
\pgfpathlineto{\pgfqpoint{8.380745in}{1.661949in}}%
\pgfpathlineto{\pgfqpoint{8.393002in}{1.627256in}}%
\pgfpathlineto{\pgfqpoint{8.405081in}{1.595697in}}%
\pgfpathlineto{\pgfqpoint{8.416983in}{1.566880in}}%
\pgfpathlineto{\pgfqpoint{8.428704in}{1.540468in}}%
\pgfpathlineto{\pgfqpoint{8.440245in}{1.516173in}}%
\pgfpathlineto{\pgfqpoint{8.451604in}{1.493755in}}%
\pgfpathlineto{\pgfqpoint{8.462779in}{1.473004in}}%
\pgfpathlineto{\pgfqpoint{8.473770in}{1.453744in}}%
\pgfpathlineto{\pgfqpoint{8.484575in}{1.435821in}}%
\pgfpathlineto{\pgfqpoint{8.495193in}{1.419106in}}%
\pgfpathlineto{\pgfqpoint{8.505623in}{1.403484in}}%
\pgfpathlineto{\pgfqpoint{8.515859in}{1.388854in}}%
\pgfpathlineto{\pgfqpoint{8.525902in}{1.375129in}}%
\pgfpathlineto{\pgfqpoint{8.535751in}{1.362232in}}%
\pgfpathlineto{\pgfqpoint{8.545405in}{1.350095in}}%
\pgfpathlineto{\pgfqpoint{8.554865in}{1.338659in}}%
\pgfpathlineto{\pgfqpoint{8.564130in}{1.327868in}}%
\pgfpathlineto{\pgfqpoint{8.573202in}{1.317674in}}%
\pgfpathlineto{\pgfqpoint{8.582083in}{1.308035in}}%
\pgfpathlineto{\pgfqpoint{8.590772in}{1.298910in}}%
\pgfpathlineto{\pgfqpoint{8.599274in}{1.290264in}}%
\pgfpathlineto{\pgfqpoint{8.607590in}{1.282066in}}%
\pgfpathlineto{\pgfqpoint{8.615724in}{1.274287in}}%
\pgfpathlineto{\pgfqpoint{8.623677in}{1.266899in}}%
\pgfpathlineto{\pgfqpoint{8.631452in}{1.259879in}}%
\pgfpathlineto{\pgfqpoint{8.639053in}{1.253205in}}%
\pgfpathlineto{\pgfqpoint{8.646481in}{1.246856in}}%
\pgfpathlineto{\pgfqpoint{8.653738in}{1.240814in}}%
\pgfpathlineto{\pgfqpoint{8.660828in}{1.235061in}}%
\pgfpathlineto{\pgfqpoint{8.667751in}{1.229581in}}%
\pgfpathlineto{\pgfqpoint{8.674509in}{1.224361in}}%
\pgfpathlineto{\pgfqpoint{8.681104in}{1.219385in}}%
\pgfpathlineto{\pgfqpoint{8.687537in}{1.214642in}}%
\pgfpathlineto{\pgfqpoint{8.693809in}{1.210120in}}%
\pgfpathlineto{\pgfqpoint{8.699921in}{1.205809in}}%
\pgfpathlineto{\pgfqpoint{8.705874in}{1.201698in}}%
\pgfpathlineto{\pgfqpoint{8.711670in}{1.197777in}}%
\pgfpathlineto{\pgfqpoint{8.717309in}{1.194039in}}%
\pgfpathlineto{\pgfqpoint{8.722792in}{1.190474in}}%
\pgfpathlineto{\pgfqpoint{8.728120in}{1.187075in}}%
\pgfpathlineto{\pgfqpoint{8.733295in}{1.183835in}}%
\pgfpathlineto{\pgfqpoint{8.738318in}{1.180748in}}%
\pgfpathlineto{\pgfqpoint{8.743189in}{1.177806in}}%
\pgfpathlineto{\pgfqpoint{8.747910in}{1.175005in}}%
\pgfpathlineto{\pgfqpoint{8.752483in}{1.172338in}}%
\pgfpathlineto{\pgfqpoint{8.756907in}{1.169800in}}%
\pgfpathlineto{\pgfqpoint{8.761185in}{1.167388in}}%
\pgfpathlineto{\pgfqpoint{8.765317in}{1.165095in}}%
\pgfpathlineto{\pgfqpoint{8.769304in}{1.162917in}}%
\pgfpathlineto{\pgfqpoint{8.773147in}{1.160852in}}%
\pgfpathlineto{\pgfqpoint{8.776848in}{1.158894in}}%
\pgfpathlineto{\pgfqpoint{8.780406in}{1.157040in}}%
\pgfpathlineto{\pgfqpoint{8.783822in}{1.155287in}}%
\pgfpathlineto{\pgfqpoint{8.787097in}{1.153632in}}%
\pgfpathlineto{\pgfqpoint{8.790232in}{1.152072in}}%
\pgfpathlineto{\pgfqpoint{8.793227in}{1.150603in}}%
\pgfpathlineto{\pgfqpoint{8.796082in}{1.149223in}}%
\pgfusepath{stroke}%
\end{pgfscope}%
\begin{pgfscope}%
\pgfpathrectangle{\pgfqpoint{8.282041in}{0.790446in}}{\pgfqpoint{1.897959in}{1.372727in}}%
\pgfusepath{clip}%
\pgfsetrectcap%
\pgfsetroundjoin%
\pgfsetlinewidth{1.505625pt}%
\definecolor{currentstroke}{rgb}{1.000000,0.000000,0.000000}%
\pgfsetstrokecolor{currentstroke}%
\pgfsetstrokeopacity{0.750000}%
\pgfsetdash{}{0pt}%
\pgfpathmoveto{\pgfqpoint{8.396039in}{2.024213in}}%
\pgfpathlineto{\pgfqpoint{8.409842in}{2.033072in}}%
\pgfpathlineto{\pgfqpoint{8.423392in}{2.041374in}}%
\pgfpathlineto{\pgfqpoint{8.436693in}{2.048018in}}%
\pgfpathlineto{\pgfqpoint{8.449746in}{2.053430in}}%
\pgfpathlineto{\pgfqpoint{8.462553in}{2.057908in}}%
\pgfpathlineto{\pgfqpoint{8.475117in}{2.061665in}}%
\pgfpathlineto{\pgfqpoint{8.487440in}{2.064854in}}%
\pgfpathlineto{\pgfqpoint{8.499523in}{2.067592in}}%
\pgfpathlineto{\pgfqpoint{8.511370in}{2.069966in}}%
\pgfpathlineto{\pgfqpoint{8.522982in}{2.072043in}}%
\pgfpathlineto{\pgfqpoint{8.534362in}{2.073875in}}%
\pgfpathlineto{\pgfqpoint{8.545511in}{2.075502in}}%
\pgfpathlineto{\pgfqpoint{8.556433in}{2.076958in}}%
\pgfpathlineto{\pgfqpoint{8.567130in}{2.078270in}}%
\pgfpathlineto{\pgfqpoint{8.577604in}{2.079458in}}%
\pgfpathlineto{\pgfqpoint{8.587857in}{2.080540in}}%
\pgfpathlineto{\pgfqpoint{8.597890in}{2.081531in}}%
\pgfpathlineto{\pgfqpoint{8.607705in}{2.082443in}}%
\pgfpathlineto{\pgfqpoint{8.617304in}{2.083285in}}%
\pgfpathlineto{\pgfqpoint{8.626689in}{2.084068in}}%
\pgfpathlineto{\pgfqpoint{8.635861in}{2.084796in}}%
\pgfpathlineto{\pgfqpoint{8.644822in}{2.085478in}}%
\pgfpathlineto{\pgfqpoint{8.653573in}{2.086118in}}%
\pgfpathlineto{\pgfqpoint{8.662116in}{2.086721in}}%
\pgfpathlineto{\pgfqpoint{8.670453in}{2.087290in}}%
\pgfpathlineto{\pgfqpoint{8.678585in}{2.087830in}}%
\pgfpathlineto{\pgfqpoint{8.686516in}{2.088342in}}%
\pgfpathlineto{\pgfqpoint{8.694246in}{2.088831in}}%
\pgfpathlineto{\pgfqpoint{8.701779in}{2.089297in}}%
\pgfpathlineto{\pgfqpoint{8.709117in}{2.089743in}}%
\pgfpathlineto{\pgfqpoint{8.716263in}{2.090172in}}%
\pgfpathlineto{\pgfqpoint{8.723218in}{2.090583in}}%
\pgfpathlineto{\pgfqpoint{8.729985in}{2.090979in}}%
\pgfpathlineto{\pgfqpoint{8.736568in}{2.091361in}}%
\pgfpathlineto{\pgfqpoint{8.742967in}{2.091730in}}%
\pgfpathlineto{\pgfqpoint{8.749185in}{2.092087in}}%
\pgfpathlineto{\pgfqpoint{8.755225in}{2.092433in}}%
\pgfpathlineto{\pgfqpoint{8.761088in}{2.092768in}}%
\pgfpathlineto{\pgfqpoint{8.766775in}{2.093094in}}%
\pgfpathlineto{\pgfqpoint{8.772290in}{2.093410in}}%
\pgfpathlineto{\pgfqpoint{8.777632in}{2.093718in}}%
\pgfpathlineto{\pgfqpoint{8.782803in}{2.094018in}}%
\pgfpathlineto{\pgfqpoint{8.787805in}{2.094311in}}%
\pgfpathlineto{\pgfqpoint{8.792638in}{2.094596in}}%
\pgfpathlineto{\pgfqpoint{8.797303in}{2.094874in}}%
\pgfpathlineto{\pgfqpoint{8.801802in}{2.095146in}}%
\pgfpathlineto{\pgfqpoint{8.806134in}{2.095411in}}%
\pgfpathlineto{\pgfqpoint{8.810300in}{2.095671in}}%
\pgfpathlineto{\pgfqpoint{8.814302in}{2.095924in}}%
\pgfpathlineto{\pgfqpoint{8.818140in}{2.096172in}}%
\pgfpathlineto{\pgfqpoint{8.821815in}{2.096415in}}%
\pgfpathlineto{\pgfqpoint{8.825328in}{2.096652in}}%
\pgfpathlineto{\pgfqpoint{8.828679in}{2.096884in}}%
\pgfusepath{stroke}%
\end{pgfscope}%
\begin{pgfscope}%
\pgfpathrectangle{\pgfqpoint{8.282041in}{0.790446in}}{\pgfqpoint{1.897959in}{1.372727in}}%
\pgfusepath{clip}%
\pgfsetrectcap%
\pgfsetroundjoin%
\pgfsetlinewidth{1.505625pt}%
\definecolor{currentstroke}{rgb}{0.000000,0.000000,1.000000}%
\pgfsetstrokecolor{currentstroke}%
\pgfsetstrokeopacity{0.750000}%
\pgfsetdash{}{0pt}%
\pgfpathmoveto{\pgfqpoint{8.396039in}{1.314994in}}%
\pgfpathlineto{\pgfqpoint{8.409842in}{1.319833in}}%
\pgfpathlineto{\pgfqpoint{8.423392in}{1.324308in}}%
\pgfpathlineto{\pgfqpoint{8.436693in}{1.327254in}}%
\pgfpathlineto{\pgfqpoint{8.449746in}{1.329052in}}%
\pgfpathlineto{\pgfqpoint{8.462553in}{1.329968in}}%
\pgfpathlineto{\pgfqpoint{8.475117in}{1.330187in}}%
\pgfpathlineto{\pgfqpoint{8.487440in}{1.329846in}}%
\pgfpathlineto{\pgfqpoint{8.499523in}{1.329043in}}%
\pgfpathlineto{\pgfqpoint{8.511370in}{1.327853in}}%
\pgfpathlineto{\pgfqpoint{8.522982in}{1.326332in}}%
\pgfpathlineto{\pgfqpoint{8.534362in}{1.324521in}}%
\pgfpathlineto{\pgfqpoint{8.545511in}{1.322454in}}%
\pgfpathlineto{\pgfqpoint{8.556433in}{1.320156in}}%
\pgfpathlineto{\pgfqpoint{8.567130in}{1.317647in}}%
\pgfpathlineto{\pgfqpoint{8.577604in}{1.314942in}}%
\pgfpathlineto{\pgfqpoint{8.587857in}{1.312053in}}%
\pgfpathlineto{\pgfqpoint{8.597890in}{1.308990in}}%
\pgfpathlineto{\pgfqpoint{8.607705in}{1.305759in}}%
\pgfpathlineto{\pgfqpoint{8.617304in}{1.302365in}}%
\pgfpathlineto{\pgfqpoint{8.626689in}{1.298812in}}%
\pgfpathlineto{\pgfqpoint{8.635861in}{1.295102in}}%
\pgfpathlineto{\pgfqpoint{8.644822in}{1.291238in}}%
\pgfpathlineto{\pgfqpoint{8.653573in}{1.287219in}}%
\pgfpathlineto{\pgfqpoint{8.662116in}{1.283045in}}%
\pgfpathlineto{\pgfqpoint{8.670453in}{1.278715in}}%
\pgfpathlineto{\pgfqpoint{8.678585in}{1.274229in}}%
\pgfpathlineto{\pgfqpoint{8.686516in}{1.269582in}}%
\pgfpathlineto{\pgfqpoint{8.694246in}{1.264774in}}%
\pgfpathlineto{\pgfqpoint{8.701779in}{1.259800in}}%
\pgfpathlineto{\pgfqpoint{8.709117in}{1.254656in}}%
\pgfpathlineto{\pgfqpoint{8.716263in}{1.249339in}}%
\pgfpathlineto{\pgfqpoint{8.723218in}{1.243844in}}%
\pgfpathlineto{\pgfqpoint{8.729985in}{1.238164in}}%
\pgfpathlineto{\pgfqpoint{8.736568in}{1.232295in}}%
\pgfpathlineto{\pgfqpoint{8.742967in}{1.226229in}}%
\pgfpathlineto{\pgfqpoint{8.749185in}{1.219959in}}%
\pgfpathlineto{\pgfqpoint{8.755225in}{1.213478in}}%
\pgfpathlineto{\pgfqpoint{8.761088in}{1.206777in}}%
\pgfpathlineto{\pgfqpoint{8.766775in}{1.199846in}}%
\pgfpathlineto{\pgfqpoint{8.772290in}{1.192676in}}%
\pgfpathlineto{\pgfqpoint{8.777632in}{1.185254in}}%
\pgfpathlineto{\pgfqpoint{8.782803in}{1.177570in}}%
\pgfpathlineto{\pgfqpoint{8.787805in}{1.169609in}}%
\pgfpathlineto{\pgfqpoint{8.792638in}{1.161357in}}%
\pgfpathlineto{\pgfqpoint{8.797303in}{1.152797in}}%
\pgfpathlineto{\pgfqpoint{8.801802in}{1.143911in}}%
\pgfpathlineto{\pgfqpoint{8.806134in}{1.134680in}}%
\pgfpathlineto{\pgfqpoint{8.810300in}{1.125082in}}%
\pgfpathlineto{\pgfqpoint{8.814302in}{1.115093in}}%
\pgfpathlineto{\pgfqpoint{8.818140in}{1.104685in}}%
\pgfpathlineto{\pgfqpoint{8.821815in}{1.093827in}}%
\pgfpathlineto{\pgfqpoint{8.825328in}{1.082486in}}%
\pgfpathlineto{\pgfqpoint{8.828679in}{1.070623in}}%
\pgfusepath{stroke}%
\end{pgfscope}%
\begin{pgfscope}%
\pgfpathrectangle{\pgfqpoint{8.282041in}{0.790446in}}{\pgfqpoint{1.897959in}{1.372727in}}%
\pgfusepath{clip}%
\pgfsetrectcap%
\pgfsetroundjoin%
\pgfsetlinewidth{1.505625pt}%
\definecolor{currentstroke}{rgb}{0.000000,0.750000,0.750000}%
\pgfsetstrokecolor{currentstroke}%
\pgfsetstrokeopacity{0.750000}%
\pgfsetdash{}{0pt}%
\pgfpathmoveto{\pgfqpoint{8.396039in}{1.794929in}}%
\pgfpathlineto{\pgfqpoint{8.409842in}{1.762999in}}%
\pgfpathlineto{\pgfqpoint{8.423392in}{1.735185in}}%
\pgfpathlineto{\pgfqpoint{8.436693in}{1.709540in}}%
\pgfpathlineto{\pgfqpoint{8.449746in}{1.685858in}}%
\pgfpathlineto{\pgfqpoint{8.462553in}{1.663944in}}%
\pgfpathlineto{\pgfqpoint{8.475117in}{1.643624in}}%
\pgfpathlineto{\pgfqpoint{8.487440in}{1.624744in}}%
\pgfpathlineto{\pgfqpoint{8.499523in}{1.607167in}}%
\pgfpathlineto{\pgfqpoint{8.511370in}{1.590770in}}%
\pgfpathlineto{\pgfqpoint{8.522982in}{1.575448in}}%
\pgfpathlineto{\pgfqpoint{8.534362in}{1.561104in}}%
\pgfpathlineto{\pgfqpoint{8.545511in}{1.547654in}}%
\pgfpathlineto{\pgfqpoint{8.556433in}{1.535024in}}%
\pgfpathlineto{\pgfqpoint{8.567130in}{1.523147in}}%
\pgfpathlineto{\pgfqpoint{8.577604in}{1.511964in}}%
\pgfpathlineto{\pgfqpoint{8.587857in}{1.501421in}}%
\pgfpathlineto{\pgfqpoint{8.597890in}{1.491471in}}%
\pgfpathlineto{\pgfqpoint{8.607705in}{1.482069in}}%
\pgfpathlineto{\pgfqpoint{8.617304in}{1.473177in}}%
\pgfpathlineto{\pgfqpoint{8.626689in}{1.464760in}}%
\pgfpathlineto{\pgfqpoint{8.635861in}{1.456787in}}%
\pgfpathlineto{\pgfqpoint{8.644822in}{1.449226in}}%
\pgfpathlineto{\pgfqpoint{8.653573in}{1.442054in}}%
\pgfpathlineto{\pgfqpoint{8.662116in}{1.435244in}}%
\pgfpathlineto{\pgfqpoint{8.670453in}{1.428774in}}%
\pgfpathlineto{\pgfqpoint{8.678585in}{1.422625in}}%
\pgfpathlineto{\pgfqpoint{8.686516in}{1.416778in}}%
\pgfpathlineto{\pgfqpoint{8.694246in}{1.411216in}}%
\pgfpathlineto{\pgfqpoint{8.701779in}{1.405922in}}%
\pgfpathlineto{\pgfqpoint{8.709117in}{1.400881in}}%
\pgfpathlineto{\pgfqpoint{8.716263in}{1.396082in}}%
\pgfpathlineto{\pgfqpoint{8.723218in}{1.391509in}}%
\pgfpathlineto{\pgfqpoint{8.729985in}{1.387154in}}%
\pgfpathlineto{\pgfqpoint{8.736568in}{1.383003in}}%
\pgfpathlineto{\pgfqpoint{8.742967in}{1.379048in}}%
\pgfpathlineto{\pgfqpoint{8.749185in}{1.375280in}}%
\pgfpathlineto{\pgfqpoint{8.755225in}{1.371688in}}%
\pgfpathlineto{\pgfqpoint{8.761088in}{1.368266in}}%
\pgfpathlineto{\pgfqpoint{8.766775in}{1.365006in}}%
\pgfpathlineto{\pgfqpoint{8.772290in}{1.361900in}}%
\pgfpathlineto{\pgfqpoint{8.777632in}{1.358942in}}%
\pgfpathlineto{\pgfqpoint{8.782803in}{1.356126in}}%
\pgfpathlineto{\pgfqpoint{8.787805in}{1.353447in}}%
\pgfpathlineto{\pgfqpoint{8.792638in}{1.350898in}}%
\pgfpathlineto{\pgfqpoint{8.797303in}{1.348475in}}%
\pgfpathlineto{\pgfqpoint{8.801802in}{1.346173in}}%
\pgfpathlineto{\pgfqpoint{8.806134in}{1.343988in}}%
\pgfpathlineto{\pgfqpoint{8.810300in}{1.341914in}}%
\pgfpathlineto{\pgfqpoint{8.814302in}{1.339950in}}%
\pgfpathlineto{\pgfqpoint{8.818140in}{1.338090in}}%
\pgfpathlineto{\pgfqpoint{8.821815in}{1.336331in}}%
\pgfpathlineto{\pgfqpoint{8.825328in}{1.334671in}}%
\pgfpathlineto{\pgfqpoint{8.828679in}{1.333106in}}%
\pgfusepath{stroke}%
\end{pgfscope}%
\begin{pgfscope}%
\pgfpathrectangle{\pgfqpoint{8.282041in}{0.790446in}}{\pgfqpoint{1.897959in}{1.372727in}}%
\pgfusepath{clip}%
\pgfsetrectcap%
\pgfsetroundjoin%
\pgfsetlinewidth{1.505625pt}%
\definecolor{currentstroke}{rgb}{1.000000,0.000000,0.000000}%
\pgfsetstrokecolor{currentstroke}%
\pgfsetstrokeopacity{0.750000}%
\pgfsetdash{}{0pt}%
\pgfpathmoveto{\pgfqpoint{8.468496in}{2.045612in}}%
\pgfpathlineto{\pgfqpoint{8.486633in}{2.051707in}}%
\pgfpathlineto{\pgfqpoint{8.504281in}{2.057520in}}%
\pgfpathlineto{\pgfqpoint{8.521446in}{2.062232in}}%
\pgfpathlineto{\pgfqpoint{8.538133in}{2.066116in}}%
\pgfpathlineto{\pgfqpoint{8.554349in}{2.069364in}}%
\pgfpathlineto{\pgfqpoint{8.570099in}{2.072112in}}%
\pgfpathlineto{\pgfqpoint{8.585391in}{2.074464in}}%
\pgfpathlineto{\pgfqpoint{8.600230in}{2.076497in}}%
\pgfpathlineto{\pgfqpoint{8.614621in}{2.078271in}}%
\pgfpathlineto{\pgfqpoint{8.628572in}{2.079830in}}%
\pgfpathlineto{\pgfqpoint{8.642088in}{2.081210in}}%
\pgfpathlineto{\pgfqpoint{8.655176in}{2.082440in}}%
\pgfpathlineto{\pgfqpoint{8.667840in}{2.083543in}}%
\pgfpathlineto{\pgfqpoint{8.680089in}{2.084538in}}%
\pgfpathlineto{\pgfqpoint{8.691927in}{2.085440in}}%
\pgfpathlineto{\pgfqpoint{8.703361in}{2.086260in}}%
\pgfpathlineto{\pgfqpoint{8.714397in}{2.087011in}}%
\pgfpathlineto{\pgfqpoint{8.725041in}{2.087700in}}%
\pgfpathlineto{\pgfqpoint{8.735300in}{2.088334in}}%
\pgfpathlineto{\pgfqpoint{8.745179in}{2.088921in}}%
\pgfpathlineto{\pgfqpoint{8.754683in}{2.089465in}}%
\pgfpathlineto{\pgfqpoint{8.763819in}{2.089970in}}%
\pgfpathlineto{\pgfqpoint{8.772591in}{2.090441in}}%
\pgfpathlineto{\pgfqpoint{8.781005in}{2.090881in}}%
\pgfpathlineto{\pgfqpoint{8.789066in}{2.091293in}}%
\pgfpathlineto{\pgfqpoint{8.796779in}{2.091679in}}%
\pgfpathlineto{\pgfqpoint{8.804149in}{2.092042in}}%
\pgfpathlineto{\pgfqpoint{8.811179in}{2.092384in}}%
\pgfpathlineto{\pgfqpoint{8.817874in}{2.092705in}}%
\pgfusepath{stroke}%
\end{pgfscope}%
\begin{pgfscope}%
\pgfpathrectangle{\pgfqpoint{8.282041in}{0.790446in}}{\pgfqpoint{1.897959in}{1.372727in}}%
\pgfusepath{clip}%
\pgfsetrectcap%
\pgfsetroundjoin%
\pgfsetlinewidth{1.505625pt}%
\definecolor{currentstroke}{rgb}{0.000000,0.000000,1.000000}%
\pgfsetstrokecolor{currentstroke}%
\pgfsetstrokeopacity{0.750000}%
\pgfsetdash{}{0pt}%
\pgfpathmoveto{\pgfqpoint{8.468496in}{1.107182in}}%
\pgfpathlineto{\pgfqpoint{8.486633in}{1.107382in}}%
\pgfpathlineto{\pgfqpoint{8.504281in}{1.107384in}}%
\pgfpathlineto{\pgfqpoint{8.521446in}{1.106312in}}%
\pgfpathlineto{\pgfqpoint{8.538133in}{1.104391in}}%
\pgfpathlineto{\pgfqpoint{8.554349in}{1.101777in}}%
\pgfpathlineto{\pgfqpoint{8.570099in}{1.098579in}}%
\pgfpathlineto{\pgfqpoint{8.585391in}{1.094874in}}%
\pgfpathlineto{\pgfqpoint{8.600230in}{1.090719in}}%
\pgfpathlineto{\pgfqpoint{8.614621in}{1.086153in}}%
\pgfpathlineto{\pgfqpoint{8.628572in}{1.081205in}}%
\pgfpathlineto{\pgfqpoint{8.642088in}{1.075894in}}%
\pgfpathlineto{\pgfqpoint{8.655176in}{1.070232in}}%
\pgfpathlineto{\pgfqpoint{8.667840in}{1.064229in}}%
\pgfpathlineto{\pgfqpoint{8.680089in}{1.057887in}}%
\pgfpathlineto{\pgfqpoint{8.691927in}{1.051206in}}%
\pgfpathlineto{\pgfqpoint{8.703361in}{1.044183in}}%
\pgfpathlineto{\pgfqpoint{8.714397in}{1.036812in}}%
\pgfpathlineto{\pgfqpoint{8.725041in}{1.029085in}}%
\pgfpathlineto{\pgfqpoint{8.735300in}{1.020991in}}%
\pgfpathlineto{\pgfqpoint{8.745179in}{1.012518in}}%
\pgfpathlineto{\pgfqpoint{8.754683in}{1.003649in}}%
\pgfpathlineto{\pgfqpoint{8.763819in}{0.994367in}}%
\pgfpathlineto{\pgfqpoint{8.772591in}{0.984651in}}%
\pgfpathlineto{\pgfqpoint{8.781005in}{0.974478in}}%
\pgfpathlineto{\pgfqpoint{8.789066in}{0.963820in}}%
\pgfpathlineto{\pgfqpoint{8.796779in}{0.952648in}}%
\pgfpathlineto{\pgfqpoint{8.804149in}{0.940926in}}%
\pgfpathlineto{\pgfqpoint{8.811179in}{0.928616in}}%
\pgfpathlineto{\pgfqpoint{8.817874in}{0.915670in}}%
\pgfusepath{stroke}%
\end{pgfscope}%
\begin{pgfscope}%
\pgfpathrectangle{\pgfqpoint{8.282041in}{0.790446in}}{\pgfqpoint{1.897959in}{1.372727in}}%
\pgfusepath{clip}%
\pgfsetrectcap%
\pgfsetroundjoin%
\pgfsetlinewidth{1.505625pt}%
\definecolor{currentstroke}{rgb}{0.000000,0.750000,0.750000}%
\pgfsetstrokecolor{currentstroke}%
\pgfsetstrokeopacity{0.750000}%
\pgfsetdash{}{0pt}%
\pgfpathmoveto{\pgfqpoint{8.468496in}{1.750232in}}%
\pgfpathlineto{\pgfqpoint{8.486633in}{1.722589in}}%
\pgfpathlineto{\pgfqpoint{8.504281in}{1.698569in}}%
\pgfpathlineto{\pgfqpoint{8.521446in}{1.676685in}}%
\pgfpathlineto{\pgfqpoint{8.538133in}{1.656695in}}%
\pgfpathlineto{\pgfqpoint{8.554349in}{1.638386in}}%
\pgfpathlineto{\pgfqpoint{8.570099in}{1.621574in}}%
\pgfpathlineto{\pgfqpoint{8.585391in}{1.606100in}}%
\pgfpathlineto{\pgfqpoint{8.600230in}{1.591828in}}%
\pgfpathlineto{\pgfqpoint{8.614621in}{1.578636in}}%
\pgfpathlineto{\pgfqpoint{8.628572in}{1.566420in}}%
\pgfpathlineto{\pgfqpoint{8.642088in}{1.555090in}}%
\pgfpathlineto{\pgfqpoint{8.655176in}{1.544563in}}%
\pgfpathlineto{\pgfqpoint{8.667840in}{1.534770in}}%
\pgfpathlineto{\pgfqpoint{8.680089in}{1.525649in}}%
\pgfpathlineto{\pgfqpoint{8.691927in}{1.517143in}}%
\pgfpathlineto{\pgfqpoint{8.703361in}{1.509203in}}%
\pgfpathlineto{\pgfqpoint{8.714397in}{1.501785in}}%
\pgfpathlineto{\pgfqpoint{8.725041in}{1.494849in}}%
\pgfpathlineto{\pgfqpoint{8.735300in}{1.488360in}}%
\pgfpathlineto{\pgfqpoint{8.745179in}{1.482287in}}%
\pgfpathlineto{\pgfqpoint{8.754683in}{1.476600in}}%
\pgfpathlineto{\pgfqpoint{8.763819in}{1.471275in}}%
\pgfpathlineto{\pgfqpoint{8.772591in}{1.466286in}}%
\pgfpathlineto{\pgfqpoint{8.781005in}{1.461613in}}%
\pgfpathlineto{\pgfqpoint{8.789066in}{1.457237in}}%
\pgfpathlineto{\pgfqpoint{8.796779in}{1.453140in}}%
\pgfpathlineto{\pgfqpoint{8.804149in}{1.449307in}}%
\pgfpathlineto{\pgfqpoint{8.811179in}{1.445722in}}%
\pgfpathlineto{\pgfqpoint{8.817874in}{1.442372in}}%
\pgfusepath{stroke}%
\end{pgfscope}%
\begin{pgfscope}%
\pgfpathrectangle{\pgfqpoint{8.282041in}{0.790446in}}{\pgfqpoint{1.897959in}{1.372727in}}%
\pgfusepath{clip}%
\pgfsetrectcap%
\pgfsetroundjoin%
\pgfsetlinewidth{1.505625pt}%
\definecolor{currentstroke}{rgb}{1.000000,0.000000,0.000000}%
\pgfsetstrokecolor{currentstroke}%
\pgfsetstrokeopacity{0.750000}%
\pgfsetdash{}{0pt}%
\pgfpathmoveto{\pgfqpoint{8.505852in}{1.856732in}}%
\pgfpathlineto{\pgfqpoint{8.523987in}{1.872188in}}%
\pgfpathlineto{\pgfqpoint{8.541573in}{1.887531in}}%
\pgfpathlineto{\pgfqpoint{8.558623in}{1.900726in}}%
\pgfpathlineto{\pgfqpoint{8.575150in}{1.912181in}}%
\pgfpathlineto{\pgfqpoint{8.591169in}{1.922205in}}%
\pgfpathlineto{\pgfqpoint{8.606692in}{1.931040in}}%
\pgfpathlineto{\pgfqpoint{8.621733in}{1.938878in}}%
\pgfpathlineto{\pgfqpoint{8.636305in}{1.945872in}}%
\pgfpathlineto{\pgfqpoint{8.650421in}{1.952144in}}%
\pgfpathlineto{\pgfqpoint{8.664095in}{1.957795in}}%
\pgfpathlineto{\pgfqpoint{8.677341in}{1.962909in}}%
\pgfpathlineto{\pgfqpoint{8.690171in}{1.967555in}}%
\pgfpathlineto{\pgfqpoint{8.702606in}{1.971790in}}%
\pgfpathlineto{\pgfqpoint{8.714656in}{1.975664in}}%
\pgfpathlineto{\pgfqpoint{8.726335in}{1.979218in}}%
\pgfpathlineto{\pgfqpoint{8.737653in}{1.982488in}}%
\pgfpathlineto{\pgfqpoint{8.748624in}{1.985504in}}%
\pgfpathlineto{\pgfqpoint{8.759259in}{1.988292in}}%
\pgfpathlineto{\pgfqpoint{8.769569in}{1.990876in}}%
\pgfpathlineto{\pgfqpoint{8.779562in}{1.993275in}}%
\pgfpathlineto{\pgfqpoint{8.789248in}{1.995508in}}%
\pgfpathlineto{\pgfqpoint{8.798635in}{1.997588in}}%
\pgfpathlineto{\pgfqpoint{8.807732in}{1.999529in}}%
\pgfpathlineto{\pgfqpoint{8.816545in}{2.001344in}}%
\pgfpathlineto{\pgfqpoint{8.825081in}{2.003042in}}%
\pgfpathlineto{\pgfqpoint{8.833344in}{2.004634in}}%
\pgfpathlineto{\pgfqpoint{8.841339in}{2.006128in}}%
\pgfpathlineto{\pgfqpoint{8.849071in}{2.007531in}}%
\pgfpathlineto{\pgfqpoint{8.856545in}{2.008850in}}%
\pgfpathlineto{\pgfqpoint{8.863766in}{2.010091in}}%
\pgfpathlineto{\pgfqpoint{8.870737in}{2.011259in}}%
\pgfpathlineto{\pgfqpoint{8.877462in}{2.012360in}}%
\pgfpathlineto{\pgfqpoint{8.883946in}{2.013397in}}%
\pgfpathlineto{\pgfqpoint{8.890192in}{2.014375in}}%
\pgfpathlineto{\pgfqpoint{8.896206in}{2.015298in}}%
\pgfpathlineto{\pgfqpoint{8.901990in}{2.016168in}}%
\pgfpathlineto{\pgfqpoint{8.907548in}{2.016988in}}%
\pgfpathlineto{\pgfqpoint{8.912882in}{2.017762in}}%
\pgfusepath{stroke}%
\end{pgfscope}%
\begin{pgfscope}%
\pgfpathrectangle{\pgfqpoint{8.282041in}{0.790446in}}{\pgfqpoint{1.897959in}{1.372727in}}%
\pgfusepath{clip}%
\pgfsetrectcap%
\pgfsetroundjoin%
\pgfsetlinewidth{1.505625pt}%
\definecolor{currentstroke}{rgb}{0.000000,0.000000,1.000000}%
\pgfsetstrokecolor{currentstroke}%
\pgfsetstrokeopacity{0.750000}%
\pgfsetdash{}{0pt}%
\pgfpathmoveto{\pgfqpoint{8.505852in}{0.989473in}}%
\pgfpathlineto{\pgfqpoint{8.523987in}{0.996628in}}%
\pgfpathlineto{\pgfqpoint{8.541573in}{1.004242in}}%
\pgfpathlineto{\pgfqpoint{8.558623in}{1.010142in}}%
\pgfpathlineto{\pgfqpoint{8.575150in}{1.014630in}}%
\pgfpathlineto{\pgfqpoint{8.591169in}{1.017939in}}%
\pgfpathlineto{\pgfqpoint{8.606692in}{1.020246in}}%
\pgfpathlineto{\pgfqpoint{8.621733in}{1.021691in}}%
\pgfpathlineto{\pgfqpoint{8.636305in}{1.022386in}}%
\pgfpathlineto{\pgfqpoint{8.650421in}{1.022418in}}%
\pgfpathlineto{\pgfqpoint{8.664095in}{1.021859in}}%
\pgfpathlineto{\pgfqpoint{8.677341in}{1.020766in}}%
\pgfpathlineto{\pgfqpoint{8.690171in}{1.019188in}}%
\pgfpathlineto{\pgfqpoint{8.702606in}{1.017162in}}%
\pgfpathlineto{\pgfqpoint{8.714656in}{1.014719in}}%
\pgfpathlineto{\pgfqpoint{8.726335in}{1.011885in}}%
\pgfpathlineto{\pgfqpoint{8.737653in}{1.008682in}}%
\pgfpathlineto{\pgfqpoint{8.748624in}{1.005125in}}%
\pgfpathlineto{\pgfqpoint{8.759259in}{1.001228in}}%
\pgfpathlineto{\pgfqpoint{8.769569in}{0.997001in}}%
\pgfpathlineto{\pgfqpoint{8.779562in}{0.992452in}}%
\pgfpathlineto{\pgfqpoint{8.789248in}{0.987587in}}%
\pgfpathlineto{\pgfqpoint{8.798635in}{0.982408in}}%
\pgfpathlineto{\pgfqpoint{8.807732in}{0.976917in}}%
\pgfpathlineto{\pgfqpoint{8.816545in}{0.971114in}}%
\pgfpathlineto{\pgfqpoint{8.825081in}{0.964996in}}%
\pgfpathlineto{\pgfqpoint{8.833344in}{0.958560in}}%
\pgfpathlineto{\pgfqpoint{8.841339in}{0.951801in}}%
\pgfpathlineto{\pgfqpoint{8.849071in}{0.944711in}}%
\pgfpathlineto{\pgfqpoint{8.856545in}{0.937283in}}%
\pgfpathlineto{\pgfqpoint{8.863766in}{0.929506in}}%
\pgfpathlineto{\pgfqpoint{8.870737in}{0.921368in}}%
\pgfpathlineto{\pgfqpoint{8.877462in}{0.912856in}}%
\pgfpathlineto{\pgfqpoint{8.883946in}{0.903953in}}%
\pgfpathlineto{\pgfqpoint{8.890192in}{0.894640in}}%
\pgfpathlineto{\pgfqpoint{8.896206in}{0.884898in}}%
\pgfpathlineto{\pgfqpoint{8.901990in}{0.874703in}}%
\pgfpathlineto{\pgfqpoint{8.907548in}{0.864028in}}%
\pgfpathlineto{\pgfqpoint{8.912882in}{0.852843in}}%
\pgfusepath{stroke}%
\end{pgfscope}%
\begin{pgfscope}%
\pgfpathrectangle{\pgfqpoint{8.282041in}{0.790446in}}{\pgfqpoint{1.897959in}{1.372727in}}%
\pgfusepath{clip}%
\pgfsetrectcap%
\pgfsetroundjoin%
\pgfsetlinewidth{1.505625pt}%
\definecolor{currentstroke}{rgb}{0.000000,0.750000,0.750000}%
\pgfsetstrokecolor{currentstroke}%
\pgfsetstrokeopacity{0.750000}%
\pgfsetdash{}{0pt}%
\pgfpathmoveto{\pgfqpoint{8.505852in}{2.006475in}}%
\pgfpathlineto{\pgfqpoint{8.523987in}{1.993865in}}%
\pgfpathlineto{\pgfqpoint{8.541573in}{1.984283in}}%
\pgfpathlineto{\pgfqpoint{8.558623in}{1.975156in}}%
\pgfpathlineto{\pgfqpoint{8.575150in}{1.966482in}}%
\pgfpathlineto{\pgfqpoint{8.591169in}{1.958246in}}%
\pgfpathlineto{\pgfqpoint{8.606692in}{1.950433in}}%
\pgfpathlineto{\pgfqpoint{8.621733in}{1.943026in}}%
\pgfpathlineto{\pgfqpoint{8.636305in}{1.936005in}}%
\pgfpathlineto{\pgfqpoint{8.650421in}{1.929348in}}%
\pgfpathlineto{\pgfqpoint{8.664095in}{1.923038in}}%
\pgfpathlineto{\pgfqpoint{8.677341in}{1.917055in}}%
\pgfpathlineto{\pgfqpoint{8.690171in}{1.911380in}}%
\pgfpathlineto{\pgfqpoint{8.702606in}{1.905997in}}%
\pgfpathlineto{\pgfqpoint{8.714656in}{1.900889in}}%
\pgfpathlineto{\pgfqpoint{8.726335in}{1.896040in}}%
\pgfpathlineto{\pgfqpoint{8.737653in}{1.891436in}}%
\pgfpathlineto{\pgfqpoint{8.748624in}{1.887064in}}%
\pgfpathlineto{\pgfqpoint{8.759259in}{1.882910in}}%
\pgfpathlineto{\pgfqpoint{8.769569in}{1.878963in}}%
\pgfpathlineto{\pgfqpoint{8.779562in}{1.875212in}}%
\pgfpathlineto{\pgfqpoint{8.789248in}{1.871646in}}%
\pgfpathlineto{\pgfqpoint{8.798635in}{1.868256in}}%
\pgfpathlineto{\pgfqpoint{8.807732in}{1.865032in}}%
\pgfpathlineto{\pgfqpoint{8.816545in}{1.861966in}}%
\pgfpathlineto{\pgfqpoint{8.825081in}{1.859051in}}%
\pgfpathlineto{\pgfqpoint{8.833344in}{1.856279in}}%
\pgfpathlineto{\pgfqpoint{8.841339in}{1.853643in}}%
\pgfpathlineto{\pgfqpoint{8.849071in}{1.851137in}}%
\pgfpathlineto{\pgfqpoint{8.856545in}{1.848755in}}%
\pgfpathlineto{\pgfqpoint{8.863766in}{1.846490in}}%
\pgfpathlineto{\pgfqpoint{8.870737in}{1.844340in}}%
\pgfpathlineto{\pgfqpoint{8.877462in}{1.842297in}}%
\pgfpathlineto{\pgfqpoint{8.883946in}{1.840358in}}%
\pgfpathlineto{\pgfqpoint{8.890192in}{1.838518in}}%
\pgfpathlineto{\pgfqpoint{8.896206in}{1.836773in}}%
\pgfpathlineto{\pgfqpoint{8.901990in}{1.835121in}}%
\pgfpathlineto{\pgfqpoint{8.907548in}{1.833556in}}%
\pgfpathlineto{\pgfqpoint{8.912882in}{1.832076in}}%
\pgfusepath{stroke}%
\end{pgfscope}%
\begin{pgfscope}%
\pgfpathrectangle{\pgfqpoint{8.282041in}{0.790446in}}{\pgfqpoint{1.897959in}{1.372727in}}%
\pgfusepath{clip}%
\pgfsetrectcap%
\pgfsetroundjoin%
\pgfsetlinewidth{1.505625pt}%
\definecolor{currentstroke}{rgb}{1.000000,0.000000,0.000000}%
\pgfsetstrokecolor{currentstroke}%
\pgfsetstrokeopacity{0.750000}%
\pgfsetdash{}{0pt}%
\pgfpathmoveto{\pgfqpoint{8.442158in}{1.950474in}}%
\pgfpathlineto{\pgfqpoint{8.459568in}{1.964769in}}%
\pgfpathlineto{\pgfqpoint{8.476656in}{1.978606in}}%
\pgfpathlineto{\pgfqpoint{8.493428in}{1.990049in}}%
\pgfpathlineto{\pgfqpoint{8.509889in}{1.999634in}}%
\pgfpathlineto{\pgfqpoint{8.526044in}{2.007753in}}%
\pgfpathlineto{\pgfqpoint{8.541898in}{2.014697in}}%
\pgfpathlineto{\pgfqpoint{8.557456in}{2.020690in}}%
\pgfpathlineto{\pgfqpoint{8.572724in}{2.025904in}}%
\pgfpathlineto{\pgfqpoint{8.587707in}{2.030473in}}%
\pgfpathlineto{\pgfqpoint{8.602409in}{2.034503in}}%
\pgfpathlineto{\pgfqpoint{8.616836in}{2.038078in}}%
\pgfpathlineto{\pgfqpoint{8.630993in}{2.041268in}}%
\pgfpathlineto{\pgfqpoint{8.644884in}{2.044129in}}%
\pgfpathlineto{\pgfqpoint{8.658514in}{2.046706in}}%
\pgfpathlineto{\pgfqpoint{8.671889in}{2.049039in}}%
\pgfpathlineto{\pgfqpoint{8.685015in}{2.051158in}}%
\pgfpathlineto{\pgfqpoint{8.697896in}{2.053091in}}%
\pgfpathlineto{\pgfqpoint{8.710540in}{2.054860in}}%
\pgfpathlineto{\pgfqpoint{8.722953in}{2.056485in}}%
\pgfpathlineto{\pgfqpoint{8.735139in}{2.057982in}}%
\pgfpathlineto{\pgfqpoint{8.747106in}{2.059365in}}%
\pgfpathlineto{\pgfqpoint{8.758860in}{2.060647in}}%
\pgfpathlineto{\pgfqpoint{8.770406in}{2.061838in}}%
\pgfpathlineto{\pgfqpoint{8.781751in}{2.062948in}}%
\pgfpathlineto{\pgfqpoint{8.792900in}{2.063983in}}%
\pgfpathlineto{\pgfqpoint{8.803857in}{2.064953in}}%
\pgfpathlineto{\pgfqpoint{8.814628in}{2.065862in}}%
\pgfpathlineto{\pgfqpoint{8.825217in}{2.066717in}}%
\pgfpathlineto{\pgfqpoint{8.835628in}{2.067522in}}%
\pgfpathlineto{\pgfqpoint{8.845864in}{2.068282in}}%
\pgfpathlineto{\pgfqpoint{8.855929in}{2.069000in}}%
\pgfpathlineto{\pgfqpoint{8.865826in}{2.069680in}}%
\pgfpathlineto{\pgfqpoint{8.875558in}{2.070325in}}%
\pgfpathlineto{\pgfqpoint{8.885126in}{2.070938in}}%
\pgfpathlineto{\pgfqpoint{8.894534in}{2.071522in}}%
\pgfpathlineto{\pgfqpoint{8.903784in}{2.072078in}}%
\pgfpathlineto{\pgfqpoint{8.912877in}{2.072609in}}%
\pgfpathlineto{\pgfqpoint{8.921817in}{2.073117in}}%
\pgfpathlineto{\pgfqpoint{8.930606in}{2.073603in}}%
\pgfpathlineto{\pgfqpoint{8.939244in}{2.074068in}}%
\pgfpathlineto{\pgfqpoint{8.947736in}{2.074515in}}%
\pgfpathlineto{\pgfqpoint{8.956082in}{2.074945in}}%
\pgfpathlineto{\pgfqpoint{8.964285in}{2.075358in}}%
\pgfpathlineto{\pgfqpoint{8.972347in}{2.075755in}}%
\pgfpathlineto{\pgfqpoint{8.980269in}{2.076139in}}%
\pgfpathlineto{\pgfqpoint{8.988055in}{2.076509in}}%
\pgfpathlineto{\pgfqpoint{8.995706in}{2.076867in}}%
\pgfpathlineto{\pgfqpoint{9.003224in}{2.077213in}}%
\pgfpathlineto{\pgfqpoint{9.010610in}{2.077548in}}%
\pgfpathlineto{\pgfqpoint{9.017867in}{2.077872in}}%
\pgfpathlineto{\pgfqpoint{9.024996in}{2.078187in}}%
\pgfpathlineto{\pgfqpoint{9.031998in}{2.078492in}}%
\pgfpathlineto{\pgfqpoint{9.038874in}{2.078788in}}%
\pgfpathlineto{\pgfqpoint{9.045628in}{2.079077in}}%
\pgfpathlineto{\pgfqpoint{9.052258in}{2.079357in}}%
\pgfpathlineto{\pgfqpoint{9.058767in}{2.079630in}}%
\pgfpathlineto{\pgfqpoint{9.065157in}{2.079895in}}%
\pgfpathlineto{\pgfqpoint{9.071427in}{2.080154in}}%
\pgfpathlineto{\pgfqpoint{9.077579in}{2.080407in}}%
\pgfpathlineto{\pgfqpoint{9.083614in}{2.080654in}}%
\pgfpathlineto{\pgfqpoint{9.089534in}{2.080894in}}%
\pgfpathlineto{\pgfqpoint{9.095339in}{2.081129in}}%
\pgfpathlineto{\pgfqpoint{9.101029in}{2.081359in}}%
\pgfpathlineto{\pgfqpoint{9.106607in}{2.081584in}}%
\pgfpathlineto{\pgfqpoint{9.112073in}{2.081804in}}%
\pgfpathlineto{\pgfqpoint{9.117428in}{2.082019in}}%
\pgfpathlineto{\pgfqpoint{9.122673in}{2.082229in}}%
\pgfpathlineto{\pgfqpoint{9.127810in}{2.082436in}}%
\pgfpathlineto{\pgfqpoint{9.132838in}{2.082638in}}%
\pgfpathlineto{\pgfqpoint{9.137759in}{2.082836in}}%
\pgfpathlineto{\pgfqpoint{9.142574in}{2.083030in}}%
\pgfpathlineto{\pgfqpoint{9.147284in}{2.083220in}}%
\pgfpathlineto{\pgfqpoint{9.151890in}{2.083407in}}%
\pgfpathlineto{\pgfqpoint{9.156393in}{2.083590in}}%
\pgfpathlineto{\pgfqpoint{9.160792in}{2.083770in}}%
\pgfpathlineto{\pgfqpoint{9.165090in}{2.083947in}}%
\pgfpathlineto{\pgfqpoint{9.169287in}{2.084120in}}%
\pgfpathlineto{\pgfqpoint{9.173384in}{2.084290in}}%
\pgfpathlineto{\pgfqpoint{9.177381in}{2.084457in}}%
\pgfpathlineto{\pgfqpoint{9.181278in}{2.084621in}}%
\pgfpathlineto{\pgfqpoint{9.185077in}{2.084782in}}%
\pgfpathlineto{\pgfqpoint{9.188777in}{2.084941in}}%
\pgfpathlineto{\pgfqpoint{9.192379in}{2.085096in}}%
\pgfpathlineto{\pgfqpoint{9.195885in}{2.085248in}}%
\pgfpathlineto{\pgfqpoint{9.199293in}{2.085398in}}%
\pgfpathlineto{\pgfqpoint{9.202605in}{2.085545in}}%
\pgfpathlineto{\pgfqpoint{9.205821in}{2.085690in}}%
\pgfpathlineto{\pgfqpoint{9.208942in}{2.085832in}}%
\pgfpathlineto{\pgfqpoint{9.211967in}{2.085971in}}%
\pgfpathlineto{\pgfqpoint{9.214897in}{2.086107in}}%
\pgfpathlineto{\pgfqpoint{9.217734in}{2.086242in}}%
\pgfpathlineto{\pgfqpoint{9.220476in}{2.086373in}}%
\pgfpathlineto{\pgfqpoint{9.223125in}{2.086503in}}%
\pgfpathlineto{\pgfqpoint{9.225680in}{2.086629in}}%
\pgfpathlineto{\pgfqpoint{9.228144in}{2.086753in}}%
\pgfpathlineto{\pgfqpoint{9.230514in}{2.086875in}}%
\pgfpathlineto{\pgfqpoint{9.232794in}{2.086994in}}%
\pgfpathlineto{\pgfqpoint{9.234981in}{2.087111in}}%
\pgfpathlineto{\pgfqpoint{9.237078in}{2.087226in}}%
\pgfpathlineto{\pgfqpoint{9.239084in}{2.087338in}}%
\pgfpathlineto{\pgfqpoint{9.240999in}{2.087447in}}%
\pgfusepath{stroke}%
\end{pgfscope}%
\begin{pgfscope}%
\pgfpathrectangle{\pgfqpoint{8.282041in}{0.790446in}}{\pgfqpoint{1.897959in}{1.372727in}}%
\pgfusepath{clip}%
\pgfsetrectcap%
\pgfsetroundjoin%
\pgfsetlinewidth{1.505625pt}%
\definecolor{currentstroke}{rgb}{0.000000,0.000000,1.000000}%
\pgfsetstrokecolor{currentstroke}%
\pgfsetstrokeopacity{0.750000}%
\pgfsetdash{}{0pt}%
\pgfpathmoveto{\pgfqpoint{8.442158in}{1.192116in}}%
\pgfpathlineto{\pgfqpoint{8.459568in}{1.202878in}}%
\pgfpathlineto{\pgfqpoint{8.476656in}{1.213607in}}%
\pgfpathlineto{\pgfqpoint{8.493428in}{1.222264in}}%
\pgfpathlineto{\pgfqpoint{8.509889in}{1.229313in}}%
\pgfpathlineto{\pgfqpoint{8.526044in}{1.235088in}}%
\pgfpathlineto{\pgfqpoint{8.541898in}{1.239841in}}%
\pgfpathlineto{\pgfqpoint{8.557456in}{1.243762in}}%
\pgfpathlineto{\pgfqpoint{8.572724in}{1.246997in}}%
\pgfpathlineto{\pgfqpoint{8.587707in}{1.249659in}}%
\pgfpathlineto{\pgfqpoint{8.602409in}{1.251838in}}%
\pgfpathlineto{\pgfqpoint{8.616836in}{1.253605in}}%
\pgfpathlineto{\pgfqpoint{8.630993in}{1.255016in}}%
\pgfpathlineto{\pgfqpoint{8.644884in}{1.256117in}}%
\pgfpathlineto{\pgfqpoint{8.658514in}{1.256947in}}%
\pgfpathlineto{\pgfqpoint{8.671889in}{1.257536in}}%
\pgfpathlineto{\pgfqpoint{8.685015in}{1.257909in}}%
\pgfpathlineto{\pgfqpoint{8.697896in}{1.258089in}}%
\pgfpathlineto{\pgfqpoint{8.710540in}{1.258091in}}%
\pgfpathlineto{\pgfqpoint{8.722953in}{1.257932in}}%
\pgfpathlineto{\pgfqpoint{8.735139in}{1.257624in}}%
\pgfpathlineto{\pgfqpoint{8.747106in}{1.257178in}}%
\pgfpathlineto{\pgfqpoint{8.758860in}{1.256602in}}%
\pgfpathlineto{\pgfqpoint{8.770406in}{1.255905in}}%
\pgfpathlineto{\pgfqpoint{8.781751in}{1.255093in}}%
\pgfpathlineto{\pgfqpoint{8.792900in}{1.254172in}}%
\pgfpathlineto{\pgfqpoint{8.803857in}{1.253146in}}%
\pgfpathlineto{\pgfqpoint{8.814628in}{1.252020in}}%
\pgfpathlineto{\pgfqpoint{8.825217in}{1.250798in}}%
\pgfpathlineto{\pgfqpoint{8.835628in}{1.249482in}}%
\pgfpathlineto{\pgfqpoint{8.845864in}{1.248075in}}%
\pgfpathlineto{\pgfqpoint{8.855929in}{1.246580in}}%
\pgfpathlineto{\pgfqpoint{8.865826in}{1.244998in}}%
\pgfpathlineto{\pgfqpoint{8.875558in}{1.243331in}}%
\pgfpathlineto{\pgfqpoint{8.885126in}{1.241580in}}%
\pgfpathlineto{\pgfqpoint{8.894534in}{1.239746in}}%
\pgfpathlineto{\pgfqpoint{8.903784in}{1.237831in}}%
\pgfpathlineto{\pgfqpoint{8.912877in}{1.235835in}}%
\pgfpathlineto{\pgfqpoint{8.921817in}{1.233758in}}%
\pgfpathlineto{\pgfqpoint{8.930606in}{1.231602in}}%
\pgfpathlineto{\pgfqpoint{8.939244in}{1.229365in}}%
\pgfpathlineto{\pgfqpoint{8.947736in}{1.227048in}}%
\pgfpathlineto{\pgfqpoint{8.956082in}{1.224652in}}%
\pgfpathlineto{\pgfqpoint{8.964285in}{1.222175in}}%
\pgfpathlineto{\pgfqpoint{8.972347in}{1.219619in}}%
\pgfpathlineto{\pgfqpoint{8.980269in}{1.216982in}}%
\pgfpathlineto{\pgfqpoint{8.988055in}{1.214263in}}%
\pgfpathlineto{\pgfqpoint{8.995706in}{1.211463in}}%
\pgfpathlineto{\pgfqpoint{9.003224in}{1.208580in}}%
\pgfpathlineto{\pgfqpoint{9.010610in}{1.205614in}}%
\pgfpathlineto{\pgfqpoint{9.017867in}{1.202564in}}%
\pgfpathlineto{\pgfqpoint{9.024996in}{1.199428in}}%
\pgfpathlineto{\pgfqpoint{9.031998in}{1.196207in}}%
\pgfpathlineto{\pgfqpoint{9.038874in}{1.192898in}}%
\pgfpathlineto{\pgfqpoint{9.045628in}{1.189500in}}%
\pgfpathlineto{\pgfqpoint{9.052258in}{1.186011in}}%
\pgfpathlineto{\pgfqpoint{9.058767in}{1.182431in}}%
\pgfpathlineto{\pgfqpoint{9.065157in}{1.178757in}}%
\pgfpathlineto{\pgfqpoint{9.071427in}{1.174988in}}%
\pgfpathlineto{\pgfqpoint{9.077579in}{1.171122in}}%
\pgfpathlineto{\pgfqpoint{9.083614in}{1.167157in}}%
\pgfpathlineto{\pgfqpoint{9.089534in}{1.163090in}}%
\pgfpathlineto{\pgfqpoint{9.095339in}{1.158920in}}%
\pgfpathlineto{\pgfqpoint{9.101029in}{1.154643in}}%
\pgfpathlineto{\pgfqpoint{9.106607in}{1.150258in}}%
\pgfpathlineto{\pgfqpoint{9.112073in}{1.145761in}}%
\pgfpathlineto{\pgfqpoint{9.117428in}{1.141150in}}%
\pgfpathlineto{\pgfqpoint{9.122673in}{1.136421in}}%
\pgfpathlineto{\pgfqpoint{9.127810in}{1.131572in}}%
\pgfpathlineto{\pgfqpoint{9.132838in}{1.126597in}}%
\pgfpathlineto{\pgfqpoint{9.137759in}{1.121494in}}%
\pgfpathlineto{\pgfqpoint{9.142574in}{1.116259in}}%
\pgfpathlineto{\pgfqpoint{9.147284in}{1.110887in}}%
\pgfpathlineto{\pgfqpoint{9.151890in}{1.105373in}}%
\pgfpathlineto{\pgfqpoint{9.156393in}{1.099713in}}%
\pgfpathlineto{\pgfqpoint{9.160792in}{1.093902in}}%
\pgfpathlineto{\pgfqpoint{9.165090in}{1.087933in}}%
\pgfpathlineto{\pgfqpoint{9.169287in}{1.081800in}}%
\pgfpathlineto{\pgfqpoint{9.173384in}{1.075498in}}%
\pgfpathlineto{\pgfqpoint{9.177381in}{1.069019in}}%
\pgfpathlineto{\pgfqpoint{9.181278in}{1.062355in}}%
\pgfpathlineto{\pgfqpoint{9.185077in}{1.055498in}}%
\pgfpathlineto{\pgfqpoint{9.188777in}{1.048441in}}%
\pgfpathlineto{\pgfqpoint{9.192379in}{1.041171in}}%
\pgfpathlineto{\pgfqpoint{9.195885in}{1.033681in}}%
\pgfpathlineto{\pgfqpoint{9.199293in}{1.025959in}}%
\pgfpathlineto{\pgfqpoint{9.202605in}{1.017992in}}%
\pgfpathlineto{\pgfqpoint{9.205821in}{1.009768in}}%
\pgfpathlineto{\pgfqpoint{9.208942in}{1.001271in}}%
\pgfpathlineto{\pgfqpoint{9.211967in}{0.992486in}}%
\pgfpathlineto{\pgfqpoint{9.214897in}{0.983396in}}%
\pgfpathlineto{\pgfqpoint{9.217734in}{0.973981in}}%
\pgfpathlineto{\pgfqpoint{9.220476in}{0.964220in}}%
\pgfpathlineto{\pgfqpoint{9.223125in}{0.954090in}}%
\pgfpathlineto{\pgfqpoint{9.225680in}{0.943564in}}%
\pgfpathlineto{\pgfqpoint{9.228144in}{0.932613in}}%
\pgfpathlineto{\pgfqpoint{9.230514in}{0.921203in}}%
\pgfpathlineto{\pgfqpoint{9.232794in}{0.909299in}}%
\pgfpathlineto{\pgfqpoint{9.234981in}{0.896857in}}%
\pgfpathlineto{\pgfqpoint{9.237078in}{0.883829in}}%
\pgfpathlineto{\pgfqpoint{9.239084in}{0.870161in}}%
\pgfpathlineto{\pgfqpoint{9.240999in}{0.855788in}}%
\pgfusepath{stroke}%
\end{pgfscope}%
\begin{pgfscope}%
\pgfpathrectangle{\pgfqpoint{8.282041in}{0.790446in}}{\pgfqpoint{1.897959in}{1.372727in}}%
\pgfusepath{clip}%
\pgfsetrectcap%
\pgfsetroundjoin%
\pgfsetlinewidth{1.505625pt}%
\definecolor{currentstroke}{rgb}{0.000000,0.750000,0.750000}%
\pgfsetstrokecolor{currentstroke}%
\pgfsetstrokeopacity{0.750000}%
\pgfsetdash{}{0pt}%
\pgfpathmoveto{\pgfqpoint{8.442158in}{1.929433in}}%
\pgfpathlineto{\pgfqpoint{8.459568in}{1.906676in}}%
\pgfpathlineto{\pgfqpoint{8.476656in}{1.887644in}}%
\pgfpathlineto{\pgfqpoint{8.493428in}{1.869655in}}%
\pgfpathlineto{\pgfqpoint{8.509889in}{1.852681in}}%
\pgfpathlineto{\pgfqpoint{8.526044in}{1.836678in}}%
\pgfpathlineto{\pgfqpoint{8.541898in}{1.821591in}}%
\pgfpathlineto{\pgfqpoint{8.557456in}{1.807367in}}%
\pgfpathlineto{\pgfqpoint{8.572724in}{1.793951in}}%
\pgfpathlineto{\pgfqpoint{8.587707in}{1.781288in}}%
\pgfpathlineto{\pgfqpoint{8.602409in}{1.769327in}}%
\pgfpathlineto{\pgfqpoint{8.616836in}{1.758020in}}%
\pgfpathlineto{\pgfqpoint{8.630993in}{1.747323in}}%
\pgfpathlineto{\pgfqpoint{8.644884in}{1.737194in}}%
\pgfpathlineto{\pgfqpoint{8.658514in}{1.727595in}}%
\pgfpathlineto{\pgfqpoint{8.671889in}{1.718489in}}%
\pgfpathlineto{\pgfqpoint{8.685015in}{1.709844in}}%
\pgfpathlineto{\pgfqpoint{8.697896in}{1.701630in}}%
\pgfpathlineto{\pgfqpoint{8.710540in}{1.693819in}}%
\pgfpathlineto{\pgfqpoint{8.722953in}{1.686386in}}%
\pgfpathlineto{\pgfqpoint{8.735139in}{1.679307in}}%
\pgfpathlineto{\pgfqpoint{8.747106in}{1.672559in}}%
\pgfpathlineto{\pgfqpoint{8.758860in}{1.666123in}}%
\pgfpathlineto{\pgfqpoint{8.770406in}{1.659981in}}%
\pgfpathlineto{\pgfqpoint{8.781751in}{1.654114in}}%
\pgfpathlineto{\pgfqpoint{8.792900in}{1.648506in}}%
\pgfpathlineto{\pgfqpoint{8.803857in}{1.643144in}}%
\pgfpathlineto{\pgfqpoint{8.814628in}{1.638014in}}%
\pgfpathlineto{\pgfqpoint{8.825217in}{1.633101in}}%
\pgfpathlineto{\pgfqpoint{8.835628in}{1.628396in}}%
\pgfpathlineto{\pgfqpoint{8.845864in}{1.623885in}}%
\pgfpathlineto{\pgfqpoint{8.855929in}{1.619560in}}%
\pgfpathlineto{\pgfqpoint{8.865826in}{1.615411in}}%
\pgfpathlineto{\pgfqpoint{8.875558in}{1.611428in}}%
\pgfpathlineto{\pgfqpoint{8.885126in}{1.607604in}}%
\pgfpathlineto{\pgfqpoint{8.894534in}{1.603930in}}%
\pgfpathlineto{\pgfqpoint{8.903784in}{1.600399in}}%
\pgfpathlineto{\pgfqpoint{8.912877in}{1.597004in}}%
\pgfpathlineto{\pgfqpoint{8.921817in}{1.593739in}}%
\pgfpathlineto{\pgfqpoint{8.930606in}{1.590597in}}%
\pgfpathlineto{\pgfqpoint{8.939244in}{1.587573in}}%
\pgfpathlineto{\pgfqpoint{8.947736in}{1.584662in}}%
\pgfpathlineto{\pgfqpoint{8.956082in}{1.581858in}}%
\pgfpathlineto{\pgfqpoint{8.964285in}{1.579157in}}%
\pgfpathlineto{\pgfqpoint{8.972347in}{1.576554in}}%
\pgfpathlineto{\pgfqpoint{8.980269in}{1.574046in}}%
\pgfpathlineto{\pgfqpoint{8.988055in}{1.571627in}}%
\pgfpathlineto{\pgfqpoint{8.995706in}{1.569294in}}%
\pgfpathlineto{\pgfqpoint{9.003224in}{1.567045in}}%
\pgfpathlineto{\pgfqpoint{9.010610in}{1.564874in}}%
\pgfpathlineto{\pgfqpoint{9.017867in}{1.562780in}}%
\pgfpathlineto{\pgfqpoint{9.024996in}{1.560759in}}%
\pgfpathlineto{\pgfqpoint{9.031998in}{1.558808in}}%
\pgfpathlineto{\pgfqpoint{9.038874in}{1.556925in}}%
\pgfpathlineto{\pgfqpoint{9.045628in}{1.555107in}}%
\pgfpathlineto{\pgfqpoint{9.052258in}{1.553352in}}%
\pgfpathlineto{\pgfqpoint{9.058767in}{1.551656in}}%
\pgfpathlineto{\pgfqpoint{9.065157in}{1.550019in}}%
\pgfpathlineto{\pgfqpoint{9.071427in}{1.548438in}}%
\pgfpathlineto{\pgfqpoint{9.077579in}{1.546911in}}%
\pgfpathlineto{\pgfqpoint{9.083614in}{1.545436in}}%
\pgfpathlineto{\pgfqpoint{9.089534in}{1.544011in}}%
\pgfpathlineto{\pgfqpoint{9.095339in}{1.542635in}}%
\pgfpathlineto{\pgfqpoint{9.101029in}{1.541306in}}%
\pgfpathlineto{\pgfqpoint{9.106607in}{1.540022in}}%
\pgfpathlineto{\pgfqpoint{9.112073in}{1.538782in}}%
\pgfpathlineto{\pgfqpoint{9.117428in}{1.537585in}}%
\pgfpathlineto{\pgfqpoint{9.122673in}{1.536429in}}%
\pgfpathlineto{\pgfqpoint{9.127810in}{1.535312in}}%
\pgfpathlineto{\pgfqpoint{9.132838in}{1.534235in}}%
\pgfpathlineto{\pgfqpoint{9.137759in}{1.533194in}}%
\pgfpathlineto{\pgfqpoint{9.142574in}{1.532191in}}%
\pgfpathlineto{\pgfqpoint{9.147284in}{1.531222in}}%
\pgfpathlineto{\pgfqpoint{9.151890in}{1.530288in}}%
\pgfpathlineto{\pgfqpoint{9.156393in}{1.529387in}}%
\pgfpathlineto{\pgfqpoint{9.160792in}{1.528519in}}%
\pgfpathlineto{\pgfqpoint{9.165090in}{1.527682in}}%
\pgfpathlineto{\pgfqpoint{9.169287in}{1.526875in}}%
\pgfpathlineto{\pgfqpoint{9.173384in}{1.526099in}}%
\pgfpathlineto{\pgfqpoint{9.177381in}{1.525351in}}%
\pgfpathlineto{\pgfqpoint{9.181278in}{1.524632in}}%
\pgfpathlineto{\pgfqpoint{9.185077in}{1.523940in}}%
\pgfpathlineto{\pgfqpoint{9.188777in}{1.523275in}}%
\pgfpathlineto{\pgfqpoint{9.192379in}{1.522636in}}%
\pgfpathlineto{\pgfqpoint{9.195885in}{1.522023in}}%
\pgfpathlineto{\pgfqpoint{9.199293in}{1.521434in}}%
\pgfpathlineto{\pgfqpoint{9.202605in}{1.520871in}}%
\pgfpathlineto{\pgfqpoint{9.205821in}{1.520331in}}%
\pgfpathlineto{\pgfqpoint{9.208942in}{1.519814in}}%
\pgfpathlineto{\pgfqpoint{9.211967in}{1.519320in}}%
\pgfpathlineto{\pgfqpoint{9.214897in}{1.518848in}}%
\pgfpathlineto{\pgfqpoint{9.217734in}{1.518398in}}%
\pgfpathlineto{\pgfqpoint{9.220476in}{1.517970in}}%
\pgfpathlineto{\pgfqpoint{9.223125in}{1.517563in}}%
\pgfpathlineto{\pgfqpoint{9.225680in}{1.517176in}}%
\pgfpathlineto{\pgfqpoint{9.228144in}{1.516810in}}%
\pgfpathlineto{\pgfqpoint{9.230514in}{1.516463in}}%
\pgfpathlineto{\pgfqpoint{9.232794in}{1.516136in}}%
\pgfpathlineto{\pgfqpoint{9.234981in}{1.515828in}}%
\pgfpathlineto{\pgfqpoint{9.237078in}{1.515539in}}%
\pgfpathlineto{\pgfqpoint{9.239084in}{1.515268in}}%
\pgfpathlineto{\pgfqpoint{9.240999in}{1.515016in}}%
\pgfusepath{stroke}%
\end{pgfscope}%
\begin{pgfscope}%
\pgfpathrectangle{\pgfqpoint{8.282041in}{0.790446in}}{\pgfqpoint{1.897959in}{1.372727in}}%
\pgfusepath{clip}%
\pgfsetrectcap%
\pgfsetroundjoin%
\pgfsetlinewidth{1.505625pt}%
\definecolor{currentstroke}{rgb}{1.000000,0.000000,0.000000}%
\pgfsetstrokecolor{currentstroke}%
\pgfsetstrokeopacity{0.750000}%
\pgfsetdash{}{0pt}%
\pgfpathmoveto{\pgfqpoint{8.391651in}{1.996016in}}%
\pgfpathlineto{\pgfqpoint{8.421603in}{2.018488in}}%
\pgfpathlineto{\pgfqpoint{8.436279in}{2.027102in}}%
\pgfpathlineto{\pgfqpoint{8.465051in}{2.039730in}}%
\pgfpathlineto{\pgfqpoint{8.493066in}{2.048406in}}%
\pgfpathlineto{\pgfqpoint{8.520347in}{2.054647in}}%
\pgfpathlineto{\pgfqpoint{8.559945in}{2.061207in}}%
\pgfpathlineto{\pgfqpoint{8.610401in}{2.066910in}}%
\pgfpathlineto{\pgfqpoint{8.681432in}{2.072117in}}%
\pgfpathlineto{\pgfqpoint{8.778462in}{2.076559in}}%
\pgfpathlineto{\pgfqpoint{8.942831in}{2.081391in}}%
\pgfpathlineto{\pgfqpoint{9.212438in}{2.089327in}}%
\pgfpathlineto{\pgfqpoint{9.323357in}{2.094842in}}%
\pgfpathlineto{\pgfqpoint{9.380491in}{2.099628in}}%
\pgfpathlineto{\pgfqpoint{9.389283in}{2.100776in}}%
\pgfpathlineto{\pgfqpoint{9.389283in}{2.100776in}}%
\pgfusepath{stroke}%
\end{pgfscope}%
\begin{pgfscope}%
\pgfpathrectangle{\pgfqpoint{8.282041in}{0.790446in}}{\pgfqpoint{1.897959in}{1.372727in}}%
\pgfusepath{clip}%
\pgfsetrectcap%
\pgfsetroundjoin%
\pgfsetlinewidth{1.505625pt}%
\definecolor{currentstroke}{rgb}{0.000000,0.000000,1.000000}%
\pgfsetstrokecolor{currentstroke}%
\pgfsetstrokeopacity{0.750000}%
\pgfsetdash{}{0pt}%
\pgfpathmoveto{\pgfqpoint{8.391651in}{1.436066in}}%
\pgfpathlineto{\pgfqpoint{8.421603in}{1.455776in}}%
\pgfpathlineto{\pgfqpoint{8.436279in}{1.463325in}}%
\pgfpathlineto{\pgfqpoint{8.465051in}{1.474221in}}%
\pgfpathlineto{\pgfqpoint{8.493066in}{1.481517in}}%
\pgfpathlineto{\pgfqpoint{8.533720in}{1.488559in}}%
\pgfpathlineto{\pgfqpoint{8.572804in}{1.492855in}}%
\pgfpathlineto{\pgfqpoint{8.622616in}{1.496132in}}%
\pgfpathlineto{\pgfqpoint{8.681432in}{1.497865in}}%
\pgfpathlineto{\pgfqpoint{8.747322in}{1.497745in}}%
\pgfpathlineto{\pgfqpoint{8.818208in}{1.495431in}}%
\pgfpathlineto{\pgfqpoint{8.883187in}{1.491184in}}%
\pgfpathlineto{\pgfqpoint{8.942831in}{1.485170in}}%
\pgfpathlineto{\pgfqpoint{8.997581in}{1.477433in}}%
\pgfpathlineto{\pgfqpoint{9.047721in}{1.467961in}}%
\pgfpathlineto{\pgfqpoint{9.087239in}{1.458421in}}%
\pgfpathlineto{\pgfqpoint{9.123766in}{1.447508in}}%
\pgfpathlineto{\pgfqpoint{9.157474in}{1.435143in}}%
\pgfpathlineto{\pgfqpoint{9.188508in}{1.421222in}}%
\pgfpathlineto{\pgfqpoint{9.212438in}{1.408334in}}%
\pgfpathlineto{\pgfqpoint{9.234710in}{1.394173in}}%
\pgfpathlineto{\pgfqpoint{9.255403in}{1.378617in}}%
\pgfpathlineto{\pgfqpoint{9.274572in}{1.361516in}}%
\pgfpathlineto{\pgfqpoint{9.292261in}{1.342680in}}%
\pgfpathlineto{\pgfqpoint{9.308508in}{1.321872in}}%
\pgfpathlineto{\pgfqpoint{9.323357in}{1.298786in}}%
\pgfpathlineto{\pgfqpoint{9.336840in}{1.273023in}}%
\pgfpathlineto{\pgfqpoint{9.348976in}{1.244048in}}%
\pgfpathlineto{\pgfqpoint{9.359781in}{1.211121in}}%
\pgfpathlineto{\pgfqpoint{9.369287in}{1.173185in}}%
\pgfpathlineto{\pgfqpoint{9.377536in}{1.128643in}}%
\pgfpathlineto{\pgfqpoint{9.383250in}{1.086577in}}%
\pgfpathlineto{\pgfqpoint{9.388176in}{1.036527in}}%
\pgfpathlineto{\pgfqpoint{9.389283in}{1.022390in}}%
\pgfpathlineto{\pgfqpoint{9.389283in}{1.022390in}}%
\pgfusepath{stroke}%
\end{pgfscope}%
\begin{pgfscope}%
\pgfpathrectangle{\pgfqpoint{8.282041in}{0.790446in}}{\pgfqpoint{1.897959in}{1.372727in}}%
\pgfusepath{clip}%
\pgfsetrectcap%
\pgfsetroundjoin%
\pgfsetlinewidth{1.505625pt}%
\definecolor{currentstroke}{rgb}{0.000000,0.750000,0.750000}%
\pgfsetstrokecolor{currentstroke}%
\pgfsetstrokeopacity{0.750000}%
\pgfsetdash{}{0pt}%
\pgfpathmoveto{\pgfqpoint{8.391651in}{1.840019in}}%
\pgfpathlineto{\pgfqpoint{8.406727in}{1.806492in}}%
\pgfpathlineto{\pgfqpoint{8.436279in}{1.750572in}}%
\pgfpathlineto{\pgfqpoint{8.465051in}{1.702133in}}%
\pgfpathlineto{\pgfqpoint{8.493066in}{1.659994in}}%
\pgfpathlineto{\pgfqpoint{8.520347in}{1.623102in}}%
\pgfpathlineto{\pgfqpoint{8.546918in}{1.590586in}}%
\pgfpathlineto{\pgfqpoint{8.572804in}{1.561744in}}%
\pgfpathlineto{\pgfqpoint{8.610401in}{1.524166in}}%
\pgfpathlineto{\pgfqpoint{8.646587in}{1.492133in}}%
\pgfpathlineto{\pgfqpoint{8.681432in}{1.464553in}}%
\pgfpathlineto{\pgfqpoint{8.714993in}{1.440606in}}%
\pgfpathlineto{\pgfqpoint{8.757832in}{1.413256in}}%
\pgfpathlineto{\pgfqpoint{8.798583in}{1.390123in}}%
\pgfpathlineto{\pgfqpoint{8.837351in}{1.370376in}}%
\pgfpathlineto{\pgfqpoint{8.883187in}{1.349510in}}%
\pgfpathlineto{\pgfqpoint{8.934618in}{1.328895in}}%
\pgfpathlineto{\pgfqpoint{8.982419in}{1.312047in}}%
\pgfpathlineto{\pgfqpoint{9.033851in}{1.296074in}}%
\pgfpathlineto{\pgfqpoint{9.093530in}{1.279987in}}%
\pgfpathlineto{\pgfqpoint{9.152045in}{1.266494in}}%
\pgfpathlineto{\pgfqpoint{9.212438in}{1.254744in}}%
\pgfpathlineto{\pgfqpoint{9.274572in}{1.244928in}}%
\pgfpathlineto{\pgfqpoint{9.331609in}{1.238133in}}%
\pgfpathlineto{\pgfqpoint{9.375985in}{1.234946in}}%
\pgfpathlineto{\pgfqpoint{9.389283in}{1.234764in}}%
\pgfpathlineto{\pgfqpoint{9.389283in}{1.234764in}}%
\pgfusepath{stroke}%
\end{pgfscope}%
\begin{pgfscope}%
\pgfpathrectangle{\pgfqpoint{8.282041in}{0.790446in}}{\pgfqpoint{1.897959in}{1.372727in}}%
\pgfusepath{clip}%
\pgfsetrectcap%
\pgfsetroundjoin%
\pgfsetlinewidth{1.505625pt}%
\definecolor{currentstroke}{rgb}{1.000000,0.000000,0.000000}%
\pgfsetstrokecolor{currentstroke}%
\pgfsetstrokeopacity{0.750000}%
\pgfsetdash{}{0pt}%
\pgfpathmoveto{\pgfqpoint{8.504795in}{1.985954in}}%
\pgfpathlineto{\pgfqpoint{8.540288in}{2.003085in}}%
\pgfpathlineto{\pgfqpoint{8.574979in}{2.016558in}}%
\pgfpathlineto{\pgfqpoint{8.608867in}{2.026737in}}%
\pgfpathlineto{\pgfqpoint{8.641949in}{2.034628in}}%
\pgfpathlineto{\pgfqpoint{8.690054in}{2.043535in}}%
\pgfpathlineto{\pgfqpoint{8.751318in}{2.051866in}}%
\pgfpathlineto{\pgfqpoint{8.823277in}{2.058868in}}%
\pgfpathlineto{\pgfqpoint{8.916090in}{2.065132in}}%
\pgfpathlineto{\pgfqpoint{9.024088in}{2.070031in}}%
\pgfpathlineto{\pgfqpoint{9.171757in}{2.074397in}}%
\pgfpathlineto{\pgfqpoint{9.397769in}{2.078606in}}%
\pgfpathlineto{\pgfqpoint{10.010993in}{2.089039in}}%
\pgfpathlineto{\pgfqpoint{10.010993in}{2.089039in}}%
\pgfusepath{stroke}%
\end{pgfscope}%
\begin{pgfscope}%
\pgfpathrectangle{\pgfqpoint{8.282041in}{0.790446in}}{\pgfqpoint{1.897959in}{1.372727in}}%
\pgfusepath{clip}%
\pgfsetrectcap%
\pgfsetroundjoin%
\pgfsetlinewidth{1.505625pt}%
\definecolor{currentstroke}{rgb}{0.000000,0.000000,1.000000}%
\pgfsetstrokecolor{currentstroke}%
\pgfsetstrokeopacity{0.750000}%
\pgfsetdash{}{0pt}%
\pgfpathmoveto{\pgfqpoint{8.504795in}{1.242610in}}%
\pgfpathlineto{\pgfqpoint{8.557734in}{1.265543in}}%
\pgfpathlineto{\pgfqpoint{8.592024in}{1.276985in}}%
\pgfpathlineto{\pgfqpoint{8.625509in}{1.286065in}}%
\pgfpathlineto{\pgfqpoint{8.674222in}{1.296698in}}%
\pgfpathlineto{\pgfqpoint{8.736314in}{1.307284in}}%
\pgfpathlineto{\pgfqpoint{8.809285in}{1.316984in}}%
\pgfpathlineto{\pgfqpoint{8.890438in}{1.325445in}}%
\pgfpathlineto{\pgfqpoint{8.989357in}{1.333271in}}%
\pgfpathlineto{\pgfqpoint{9.090112in}{1.338791in}}%
\pgfpathlineto{\pgfqpoint{9.181499in}{1.341639in}}%
\pgfpathlineto{\pgfqpoint{9.273923in}{1.342195in}}%
\pgfpathlineto{\pgfqpoint{9.358292in}{1.340334in}}%
\pgfpathlineto{\pgfqpoint{9.435548in}{1.336269in}}%
\pgfpathlineto{\pgfqpoint{9.499568in}{1.330833in}}%
\pgfpathlineto{\pgfqpoint{9.558923in}{1.323768in}}%
\pgfpathlineto{\pgfqpoint{9.614015in}{1.315094in}}%
\pgfpathlineto{\pgfqpoint{9.665311in}{1.304800in}}%
\pgfpathlineto{\pgfqpoint{9.713338in}{1.292844in}}%
\pgfpathlineto{\pgfqpoint{9.753466in}{1.280763in}}%
\pgfpathlineto{\pgfqpoint{9.791193in}{1.267236in}}%
\pgfpathlineto{\pgfqpoint{9.826640in}{1.252162in}}%
\pgfpathlineto{\pgfqpoint{9.860040in}{1.235409in}}%
\pgfpathlineto{\pgfqpoint{9.887690in}{1.219240in}}%
\pgfpathlineto{\pgfqpoint{9.913897in}{1.201510in}}%
\pgfpathlineto{\pgfqpoint{9.938686in}{1.182035in}}%
\pgfpathlineto{\pgfqpoint{9.962112in}{1.160585in}}%
\pgfpathlineto{\pgfqpoint{9.984268in}{1.136864in}}%
\pgfpathlineto{\pgfqpoint{10.002303in}{1.114431in}}%
\pgfpathlineto{\pgfqpoint{10.010993in}{1.102387in}}%
\pgfpathlineto{\pgfqpoint{10.010993in}{1.102387in}}%
\pgfusepath{stroke}%
\end{pgfscope}%
\begin{pgfscope}%
\pgfpathrectangle{\pgfqpoint{8.282041in}{0.790446in}}{\pgfqpoint{1.897959in}{1.372727in}}%
\pgfusepath{clip}%
\pgfsetrectcap%
\pgfsetroundjoin%
\pgfsetlinewidth{1.505625pt}%
\definecolor{currentstroke}{rgb}{0.000000,0.750000,0.750000}%
\pgfsetstrokecolor{currentstroke}%
\pgfsetstrokeopacity{0.750000}%
\pgfsetdash{}{0pt}%
\pgfpathmoveto{\pgfqpoint{8.504795in}{1.878246in}}%
\pgfpathlineto{\pgfqpoint{8.522641in}{1.860009in}}%
\pgfpathlineto{\pgfqpoint{8.557734in}{1.828770in}}%
\pgfpathlineto{\pgfqpoint{8.592024in}{1.800803in}}%
\pgfpathlineto{\pgfqpoint{8.625509in}{1.775734in}}%
\pgfpathlineto{\pgfqpoint{8.674222in}{1.742800in}}%
\pgfpathlineto{\pgfqpoint{8.721102in}{1.714562in}}%
\pgfpathlineto{\pgfqpoint{8.766115in}{1.690177in}}%
\pgfpathlineto{\pgfqpoint{8.809285in}{1.668974in}}%
\pgfpathlineto{\pgfqpoint{8.864114in}{1.644742in}}%
\pgfpathlineto{\pgfqpoint{8.916090in}{1.624243in}}%
\pgfpathlineto{\pgfqpoint{8.977508in}{1.602758in}}%
\pgfpathlineto{\pgfqpoint{9.035403in}{1.584891in}}%
\pgfpathlineto{\pgfqpoint{9.100698in}{1.567146in}}%
\pgfpathlineto{\pgfqpoint{9.171757in}{1.550341in}}%
\pgfpathlineto{\pgfqpoint{9.247103in}{1.534988in}}%
\pgfpathlineto{\pgfqpoint{9.325437in}{1.521342in}}%
\pgfpathlineto{\pgfqpoint{9.413079in}{1.508439in}}%
\pgfpathlineto{\pgfqpoint{9.506384in}{1.496991in}}%
\pgfpathlineto{\pgfqpoint{9.608092in}{1.486793in}}%
\pgfpathlineto{\pgfqpoint{9.713338in}{1.478435in}}%
\pgfpathlineto{\pgfqpoint{9.826640in}{1.471676in}}%
\pgfpathlineto{\pgfqpoint{9.942114in}{1.467006in}}%
\pgfpathlineto{\pgfqpoint{10.010993in}{1.465332in}}%
\pgfpathlineto{\pgfqpoint{10.010993in}{1.465332in}}%
\pgfusepath{stroke}%
\end{pgfscope}%
\begin{pgfscope}%
\pgfpathrectangle{\pgfqpoint{8.282041in}{0.790446in}}{\pgfqpoint{1.897959in}{1.372727in}}%
\pgfusepath{clip}%
\pgfsetrectcap%
\pgfsetroundjoin%
\pgfsetlinewidth{1.505625pt}%
\definecolor{currentstroke}{rgb}{1.000000,0.000000,0.000000}%
\pgfsetstrokecolor{currentstroke}%
\pgfsetstrokeopacity{0.750000}%
\pgfsetdash{}{0pt}%
\pgfpathmoveto{\pgfqpoint{8.416676in}{1.926541in}}%
\pgfpathlineto{\pgfqpoint{8.436050in}{1.945318in}}%
\pgfpathlineto{\pgfqpoint{8.455246in}{1.963431in}}%
\pgfpathlineto{\pgfqpoint{8.474267in}{1.977943in}}%
\pgfpathlineto{\pgfqpoint{8.493116in}{1.989743in}}%
\pgfpathlineto{\pgfqpoint{8.511797in}{1.999456in}}%
\pgfpathlineto{\pgfqpoint{8.530312in}{2.007540in}}%
\pgfpathlineto{\pgfqpoint{8.548666in}{2.014335in}}%
\pgfpathlineto{\pgfqpoint{8.566862in}{2.020095in}}%
\pgfpathlineto{\pgfqpoint{8.584903in}{2.025013in}}%
\pgfpathlineto{\pgfqpoint{8.602792in}{2.029241in}}%
\pgfpathlineto{\pgfqpoint{8.620533in}{2.032896in}}%
\pgfpathlineto{\pgfqpoint{8.638130in}{2.036074in}}%
\pgfpathlineto{\pgfqpoint{8.655587in}{2.038849in}}%
\pgfpathlineto{\pgfqpoint{8.672907in}{2.041282in}}%
\pgfpathlineto{\pgfqpoint{8.690093in}{2.043421in}}%
\pgfpathlineto{\pgfqpoint{8.707149in}{2.045309in}}%
\pgfpathlineto{\pgfqpoint{8.724077in}{2.046978in}}%
\pgfpathlineto{\pgfqpoint{8.740880in}{2.048458in}}%
\pgfpathlineto{\pgfqpoint{8.757561in}{2.049772in}}%
\pgfpathlineto{\pgfqpoint{8.774122in}{2.050940in}}%
\pgfpathlineto{\pgfqpoint{8.790566in}{2.051978in}}%
\pgfpathlineto{\pgfqpoint{8.806895in}{2.052903in}}%
\pgfpathlineto{\pgfqpoint{8.823110in}{2.053726in}}%
\pgfpathlineto{\pgfqpoint{8.839216in}{2.054457in}}%
\pgfpathlineto{\pgfqpoint{8.855212in}{2.055107in}}%
\pgfpathlineto{\pgfqpoint{8.871103in}{2.055683in}}%
\pgfpathlineto{\pgfqpoint{8.886890in}{2.056192in}}%
\pgfpathlineto{\pgfqpoint{8.902577in}{2.056642in}}%
\pgfpathlineto{\pgfqpoint{8.918165in}{2.057036in}}%
\pgfpathlineto{\pgfqpoint{8.933658in}{2.057381in}}%
\pgfpathlineto{\pgfqpoint{8.949058in}{2.057681in}}%
\pgfpathlineto{\pgfqpoint{8.964368in}{2.057939in}}%
\pgfpathlineto{\pgfqpoint{8.979590in}{2.058159in}}%
\pgfpathlineto{\pgfqpoint{8.994727in}{2.058344in}}%
\pgfpathlineto{\pgfqpoint{9.009782in}{2.058497in}}%
\pgfpathlineto{\pgfqpoint{9.024757in}{2.058620in}}%
\pgfpathlineto{\pgfqpoint{9.039655in}{2.058715in}}%
\pgfpathlineto{\pgfqpoint{9.054477in}{2.058784in}}%
\pgfpathlineto{\pgfqpoint{9.069226in}{2.058830in}}%
\pgfpathlineto{\pgfqpoint{9.083903in}{2.058854in}}%
\pgfpathlineto{\pgfqpoint{9.098511in}{2.058857in}}%
\pgfpathlineto{\pgfqpoint{9.113049in}{2.058841in}}%
\pgfpathlineto{\pgfqpoint{9.127521in}{2.058807in}}%
\pgfpathlineto{\pgfqpoint{9.141927in}{2.058756in}}%
\pgfpathlineto{\pgfqpoint{9.156268in}{2.058690in}}%
\pgfpathlineto{\pgfqpoint{9.170545in}{2.058609in}}%
\pgfpathlineto{\pgfqpoint{9.184759in}{2.058514in}}%
\pgfpathlineto{\pgfqpoint{9.198911in}{2.058406in}}%
\pgfpathlineto{\pgfqpoint{9.213003in}{2.058286in}}%
\pgfpathlineto{\pgfqpoint{9.227035in}{2.058155in}}%
\pgfpathlineto{\pgfqpoint{9.241008in}{2.058013in}}%
\pgfpathlineto{\pgfqpoint{9.254923in}{2.057860in}}%
\pgfpathlineto{\pgfqpoint{9.268781in}{2.057698in}}%
\pgfpathlineto{\pgfqpoint{9.282584in}{2.057526in}}%
\pgfpathlineto{\pgfqpoint{9.296333in}{2.057346in}}%
\pgfpathlineto{\pgfqpoint{9.310029in}{2.057159in}}%
\pgfpathlineto{\pgfqpoint{9.323673in}{2.056963in}}%
\pgfpathlineto{\pgfqpoint{9.337267in}{2.056759in}}%
\pgfpathlineto{\pgfqpoint{9.350812in}{2.056549in}}%
\pgfpathlineto{\pgfqpoint{9.364310in}{2.056331in}}%
\pgfpathlineto{\pgfqpoint{9.377762in}{2.056108in}}%
\pgfpathlineto{\pgfqpoint{9.391168in}{2.055879in}}%
\pgfpathlineto{\pgfqpoint{9.404532in}{2.055644in}}%
\pgfpathlineto{\pgfqpoint{9.417853in}{2.055403in}}%
\pgfpathlineto{\pgfqpoint{9.431133in}{2.055157in}}%
\pgfpathlineto{\pgfqpoint{9.444373in}{2.054907in}}%
\pgfpathlineto{\pgfqpoint{9.457574in}{2.054651in}}%
\pgfpathlineto{\pgfqpoint{9.470737in}{2.054391in}}%
\pgfpathlineto{\pgfqpoint{9.483862in}{2.054127in}}%
\pgfpathlineto{\pgfqpoint{9.496952in}{2.053858in}}%
\pgfpathlineto{\pgfqpoint{9.510006in}{2.053586in}}%
\pgfpathlineto{\pgfqpoint{9.523026in}{2.053310in}}%
\pgfpathlineto{\pgfqpoint{9.536011in}{2.053031in}}%
\pgfpathlineto{\pgfqpoint{9.548963in}{2.052748in}}%
\pgfpathlineto{\pgfqpoint{9.561882in}{2.052461in}}%
\pgfpathlineto{\pgfqpoint{9.574769in}{2.052172in}}%
\pgfpathlineto{\pgfqpoint{9.587624in}{2.051880in}}%
\pgfpathlineto{\pgfqpoint{9.600448in}{2.051584in}}%
\pgfpathlineto{\pgfqpoint{9.613242in}{2.051286in}}%
\pgfpathlineto{\pgfqpoint{9.626006in}{2.050985in}}%
\pgfpathlineto{\pgfqpoint{9.638740in}{2.050682in}}%
\pgfpathlineto{\pgfqpoint{9.651445in}{2.050377in}}%
\pgfpathlineto{\pgfqpoint{9.664121in}{2.050068in}}%
\pgfpathlineto{\pgfqpoint{9.676770in}{2.049758in}}%
\pgfpathlineto{\pgfqpoint{9.689391in}{2.049446in}}%
\pgfpathlineto{\pgfqpoint{9.701986in}{2.049132in}}%
\pgfpathlineto{\pgfqpoint{9.714553in}{2.048816in}}%
\pgfpathlineto{\pgfqpoint{9.727095in}{2.048498in}}%
\pgfpathlineto{\pgfqpoint{9.739611in}{2.048178in}}%
\pgfpathlineto{\pgfqpoint{9.752102in}{2.047855in}}%
\pgfpathlineto{\pgfqpoint{9.764568in}{2.047532in}}%
\pgfpathlineto{\pgfqpoint{9.777009in}{2.047207in}}%
\pgfpathlineto{\pgfqpoint{9.789426in}{2.046881in}}%
\pgfpathlineto{\pgfqpoint{9.801818in}{2.046553in}}%
\pgfpathlineto{\pgfqpoint{9.814187in}{2.046224in}}%
\pgfpathlineto{\pgfqpoint{9.826533in}{2.045893in}}%
\pgfpathlineto{\pgfqpoint{9.838855in}{2.045560in}}%
\pgfpathlineto{\pgfqpoint{9.851154in}{2.045227in}}%
\pgfpathlineto{\pgfqpoint{9.863430in}{2.044893in}}%
\pgfpathlineto{\pgfqpoint{9.875684in}{2.044558in}}%
\pgfpathlineto{\pgfqpoint{9.887916in}{2.044221in}}%
\pgfpathlineto{\pgfqpoint{9.900125in}{2.043883in}}%
\pgfpathlineto{\pgfqpoint{9.912313in}{2.043544in}}%
\pgfpathlineto{\pgfqpoint{9.924479in}{2.043205in}}%
\pgfpathlineto{\pgfqpoint{9.936625in}{2.042864in}}%
\pgfpathlineto{\pgfqpoint{9.948749in}{2.042523in}}%
\pgfpathlineto{\pgfqpoint{9.960853in}{2.042181in}}%
\pgfpathlineto{\pgfqpoint{9.972938in}{2.041838in}}%
\pgfpathlineto{\pgfqpoint{9.985002in}{2.041494in}}%
\pgfpathlineto{\pgfqpoint{9.997047in}{2.041150in}}%
\pgfpathlineto{\pgfqpoint{10.009074in}{2.040804in}}%
\pgfpathlineto{\pgfqpoint{10.021082in}{2.040459in}}%
\pgfpathlineto{\pgfqpoint{10.033071in}{2.040113in}}%
\pgfpathlineto{\pgfqpoint{10.045043in}{2.039766in}}%
\pgfpathlineto{\pgfqpoint{10.056997in}{2.039418in}}%
\pgfpathlineto{\pgfqpoint{10.068934in}{2.039070in}}%
\pgfpathlineto{\pgfqpoint{10.080854in}{2.038721in}}%
\pgfpathlineto{\pgfqpoint{10.092757in}{2.038372in}}%
\pgfusepath{stroke}%
\end{pgfscope}%
\begin{pgfscope}%
\pgfpathrectangle{\pgfqpoint{8.282041in}{0.790446in}}{\pgfqpoint{1.897959in}{1.372727in}}%
\pgfusepath{clip}%
\pgfsetrectcap%
\pgfsetroundjoin%
\pgfsetlinewidth{1.505625pt}%
\definecolor{currentstroke}{rgb}{0.000000,0.000000,1.000000}%
\pgfsetstrokecolor{currentstroke}%
\pgfsetstrokeopacity{0.750000}%
\pgfsetdash{}{0pt}%
\pgfpathmoveto{\pgfqpoint{8.416676in}{1.349579in}}%
\pgfpathlineto{\pgfqpoint{8.436050in}{1.369152in}}%
\pgfpathlineto{\pgfqpoint{8.455246in}{1.388455in}}%
\pgfpathlineto{\pgfqpoint{8.474267in}{1.404458in}}%
\pgfpathlineto{\pgfqpoint{8.493116in}{1.417985in}}%
\pgfpathlineto{\pgfqpoint{8.511797in}{1.429615in}}%
\pgfpathlineto{\pgfqpoint{8.530312in}{1.439770in}}%
\pgfpathlineto{\pgfqpoint{8.548666in}{1.448766in}}%
\pgfpathlineto{\pgfqpoint{8.566862in}{1.456835in}}%
\pgfpathlineto{\pgfqpoint{8.584903in}{1.464155in}}%
\pgfpathlineto{\pgfqpoint{8.602792in}{1.470864in}}%
\pgfpathlineto{\pgfqpoint{8.620533in}{1.477070in}}%
\pgfpathlineto{\pgfqpoint{8.638130in}{1.482856in}}%
\pgfpathlineto{\pgfqpoint{8.655587in}{1.488291in}}%
\pgfpathlineto{\pgfqpoint{8.672907in}{1.493429in}}%
\pgfpathlineto{\pgfqpoint{8.690093in}{1.498312in}}%
\pgfpathlineto{\pgfqpoint{8.707149in}{1.502977in}}%
\pgfpathlineto{\pgfqpoint{8.724077in}{1.507452in}}%
\pgfpathlineto{\pgfqpoint{8.740880in}{1.511763in}}%
\pgfpathlineto{\pgfqpoint{8.757561in}{1.515928in}}%
\pgfpathlineto{\pgfqpoint{8.774122in}{1.519965in}}%
\pgfpathlineto{\pgfqpoint{8.790566in}{1.523887in}}%
\pgfpathlineto{\pgfqpoint{8.806895in}{1.527707in}}%
\pgfpathlineto{\pgfqpoint{8.823110in}{1.531436in}}%
\pgfpathlineto{\pgfqpoint{8.839216in}{1.535079in}}%
\pgfpathlineto{\pgfqpoint{8.855212in}{1.538646in}}%
\pgfpathlineto{\pgfqpoint{8.871103in}{1.542143in}}%
\pgfpathlineto{\pgfqpoint{8.886890in}{1.545574in}}%
\pgfpathlineto{\pgfqpoint{8.902577in}{1.548945in}}%
\pgfpathlineto{\pgfqpoint{8.918165in}{1.552260in}}%
\pgfpathlineto{\pgfqpoint{8.933658in}{1.555523in}}%
\pgfpathlineto{\pgfqpoint{8.949058in}{1.558735in}}%
\pgfpathlineto{\pgfqpoint{8.964368in}{1.561901in}}%
\pgfpathlineto{\pgfqpoint{8.979590in}{1.565022in}}%
\pgfpathlineto{\pgfqpoint{8.994727in}{1.568101in}}%
\pgfpathlineto{\pgfqpoint{9.009782in}{1.571139in}}%
\pgfpathlineto{\pgfqpoint{9.024757in}{1.574139in}}%
\pgfpathlineto{\pgfqpoint{9.039655in}{1.577101in}}%
\pgfpathlineto{\pgfqpoint{9.054477in}{1.580026in}}%
\pgfpathlineto{\pgfqpoint{9.069226in}{1.582917in}}%
\pgfpathlineto{\pgfqpoint{9.083903in}{1.585775in}}%
\pgfpathlineto{\pgfqpoint{9.098511in}{1.588599in}}%
\pgfpathlineto{\pgfqpoint{9.113049in}{1.591391in}}%
\pgfpathlineto{\pgfqpoint{9.127521in}{1.594153in}}%
\pgfpathlineto{\pgfqpoint{9.141927in}{1.596885in}}%
\pgfpathlineto{\pgfqpoint{9.156268in}{1.599587in}}%
\pgfpathlineto{\pgfqpoint{9.170545in}{1.602260in}}%
\pgfpathlineto{\pgfqpoint{9.184759in}{1.604905in}}%
\pgfpathlineto{\pgfqpoint{9.198911in}{1.607522in}}%
\pgfpathlineto{\pgfqpoint{9.213003in}{1.610112in}}%
\pgfpathlineto{\pgfqpoint{9.227035in}{1.612676in}}%
\pgfpathlineto{\pgfqpoint{9.241008in}{1.615214in}}%
\pgfpathlineto{\pgfqpoint{9.254923in}{1.617725in}}%
\pgfpathlineto{\pgfqpoint{9.268781in}{1.620212in}}%
\pgfpathlineto{\pgfqpoint{9.282584in}{1.622674in}}%
\pgfpathlineto{\pgfqpoint{9.296333in}{1.625112in}}%
\pgfpathlineto{\pgfqpoint{9.310029in}{1.627526in}}%
\pgfpathlineto{\pgfqpoint{9.323673in}{1.629916in}}%
\pgfpathlineto{\pgfqpoint{9.337267in}{1.632282in}}%
\pgfpathlineto{\pgfqpoint{9.350812in}{1.634625in}}%
\pgfpathlineto{\pgfqpoint{9.364310in}{1.636946in}}%
\pgfpathlineto{\pgfqpoint{9.377762in}{1.639245in}}%
\pgfpathlineto{\pgfqpoint{9.391168in}{1.641522in}}%
\pgfpathlineto{\pgfqpoint{9.404532in}{1.643777in}}%
\pgfpathlineto{\pgfqpoint{9.417853in}{1.646010in}}%
\pgfpathlineto{\pgfqpoint{9.431133in}{1.648223in}}%
\pgfpathlineto{\pgfqpoint{9.444373in}{1.650415in}}%
\pgfpathlineto{\pgfqpoint{9.457574in}{1.652586in}}%
\pgfpathlineto{\pgfqpoint{9.470737in}{1.654737in}}%
\pgfpathlineto{\pgfqpoint{9.483862in}{1.656867in}}%
\pgfpathlineto{\pgfqpoint{9.496952in}{1.658978in}}%
\pgfpathlineto{\pgfqpoint{9.510006in}{1.661070in}}%
\pgfpathlineto{\pgfqpoint{9.523026in}{1.663143in}}%
\pgfpathlineto{\pgfqpoint{9.536011in}{1.665196in}}%
\pgfpathlineto{\pgfqpoint{9.548963in}{1.667230in}}%
\pgfpathlineto{\pgfqpoint{9.561882in}{1.669246in}}%
\pgfpathlineto{\pgfqpoint{9.574769in}{1.671244in}}%
\pgfpathlineto{\pgfqpoint{9.587624in}{1.673224in}}%
\pgfpathlineto{\pgfqpoint{9.600448in}{1.675186in}}%
\pgfpathlineto{\pgfqpoint{9.613242in}{1.677130in}}%
\pgfpathlineto{\pgfqpoint{9.626006in}{1.679057in}}%
\pgfpathlineto{\pgfqpoint{9.638740in}{1.680966in}}%
\pgfpathlineto{\pgfqpoint{9.651445in}{1.682859in}}%
\pgfpathlineto{\pgfqpoint{9.664121in}{1.684735in}}%
\pgfpathlineto{\pgfqpoint{9.676770in}{1.686594in}}%
\pgfpathlineto{\pgfqpoint{9.689391in}{1.688437in}}%
\pgfpathlineto{\pgfqpoint{9.701986in}{1.690265in}}%
\pgfpathlineto{\pgfqpoint{9.714553in}{1.692076in}}%
\pgfpathlineto{\pgfqpoint{9.727095in}{1.693871in}}%
\pgfpathlineto{\pgfqpoint{9.739611in}{1.695650in}}%
\pgfpathlineto{\pgfqpoint{9.752102in}{1.697413in}}%
\pgfpathlineto{\pgfqpoint{9.764568in}{1.699161in}}%
\pgfpathlineto{\pgfqpoint{9.777009in}{1.700895in}}%
\pgfpathlineto{\pgfqpoint{9.789426in}{1.702614in}}%
\pgfpathlineto{\pgfqpoint{9.801818in}{1.704318in}}%
\pgfpathlineto{\pgfqpoint{9.814187in}{1.706008in}}%
\pgfpathlineto{\pgfqpoint{9.826533in}{1.707683in}}%
\pgfpathlineto{\pgfqpoint{9.838855in}{1.709344in}}%
\pgfpathlineto{\pgfqpoint{9.851154in}{1.710991in}}%
\pgfpathlineto{\pgfqpoint{9.863430in}{1.712624in}}%
\pgfpathlineto{\pgfqpoint{9.875684in}{1.714244in}}%
\pgfpathlineto{\pgfqpoint{9.887916in}{1.715850in}}%
\pgfpathlineto{\pgfqpoint{9.900125in}{1.717442in}}%
\pgfpathlineto{\pgfqpoint{9.912313in}{1.719021in}}%
\pgfpathlineto{\pgfqpoint{9.924479in}{1.720587in}}%
\pgfpathlineto{\pgfqpoint{9.936625in}{1.722141in}}%
\pgfpathlineto{\pgfqpoint{9.948749in}{1.723681in}}%
\pgfpathlineto{\pgfqpoint{9.960853in}{1.725210in}}%
\pgfpathlineto{\pgfqpoint{9.972938in}{1.726725in}}%
\pgfpathlineto{\pgfqpoint{9.985002in}{1.728228in}}%
\pgfpathlineto{\pgfqpoint{9.997047in}{1.729719in}}%
\pgfpathlineto{\pgfqpoint{10.009074in}{1.731198in}}%
\pgfpathlineto{\pgfqpoint{10.021082in}{1.732665in}}%
\pgfpathlineto{\pgfqpoint{10.033071in}{1.734120in}}%
\pgfpathlineto{\pgfqpoint{10.045043in}{1.735564in}}%
\pgfpathlineto{\pgfqpoint{10.056997in}{1.736996in}}%
\pgfpathlineto{\pgfqpoint{10.068934in}{1.738417in}}%
\pgfpathlineto{\pgfqpoint{10.080854in}{1.739826in}}%
\pgfpathlineto{\pgfqpoint{10.092757in}{1.741224in}}%
\pgfusepath{stroke}%
\end{pgfscope}%
\begin{pgfscope}%
\pgfpathrectangle{\pgfqpoint{8.282041in}{0.790446in}}{\pgfqpoint{1.897959in}{1.372727in}}%
\pgfusepath{clip}%
\pgfsetrectcap%
\pgfsetroundjoin%
\pgfsetlinewidth{1.505625pt}%
\definecolor{currentstroke}{rgb}{0.000000,0.750000,0.750000}%
\pgfsetstrokecolor{currentstroke}%
\pgfsetstrokeopacity{0.750000}%
\pgfsetdash{}{0pt}%
\pgfpathmoveto{\pgfqpoint{8.416676in}{1.945381in}}%
\pgfpathlineto{\pgfqpoint{8.436050in}{1.916587in}}%
\pgfpathlineto{\pgfqpoint{8.455246in}{1.892942in}}%
\pgfpathlineto{\pgfqpoint{8.474267in}{1.870367in}}%
\pgfpathlineto{\pgfqpoint{8.493116in}{1.848915in}}%
\pgfpathlineto{\pgfqpoint{8.511797in}{1.828583in}}%
\pgfpathlineto{\pgfqpoint{8.530312in}{1.809341in}}%
\pgfpathlineto{\pgfqpoint{8.548666in}{1.791145in}}%
\pgfpathlineto{\pgfqpoint{8.566862in}{1.773941in}}%
\pgfpathlineto{\pgfqpoint{8.584903in}{1.757670in}}%
\pgfpathlineto{\pgfqpoint{8.602792in}{1.742275in}}%
\pgfpathlineto{\pgfqpoint{8.620533in}{1.727699in}}%
\pgfpathlineto{\pgfqpoint{8.638130in}{1.713888in}}%
\pgfpathlineto{\pgfqpoint{8.655587in}{1.700793in}}%
\pgfpathlineto{\pgfqpoint{8.672907in}{1.688364in}}%
\pgfpathlineto{\pgfqpoint{8.690093in}{1.676558in}}%
\pgfpathlineto{\pgfqpoint{8.707149in}{1.665333in}}%
\pgfpathlineto{\pgfqpoint{8.724077in}{1.654651in}}%
\pgfpathlineto{\pgfqpoint{8.740880in}{1.644478in}}%
\pgfpathlineto{\pgfqpoint{8.757561in}{1.634779in}}%
\pgfpathlineto{\pgfqpoint{8.774122in}{1.625526in}}%
\pgfpathlineto{\pgfqpoint{8.790566in}{1.616690in}}%
\pgfpathlineto{\pgfqpoint{8.806895in}{1.608246in}}%
\pgfpathlineto{\pgfqpoint{8.823110in}{1.600169in}}%
\pgfpathlineto{\pgfqpoint{8.839216in}{1.592438in}}%
\pgfpathlineto{\pgfqpoint{8.855212in}{1.585031in}}%
\pgfpathlineto{\pgfqpoint{8.871103in}{1.577930in}}%
\pgfpathlineto{\pgfqpoint{8.886890in}{1.571118in}}%
\pgfpathlineto{\pgfqpoint{8.902577in}{1.564578in}}%
\pgfpathlineto{\pgfqpoint{8.918165in}{1.558294in}}%
\pgfpathlineto{\pgfqpoint{8.933658in}{1.552253in}}%
\pgfpathlineto{\pgfqpoint{8.949058in}{1.546441in}}%
\pgfpathlineto{\pgfqpoint{8.964368in}{1.540847in}}%
\pgfpathlineto{\pgfqpoint{8.979590in}{1.535457in}}%
\pgfpathlineto{\pgfqpoint{8.994727in}{1.530262in}}%
\pgfpathlineto{\pgfqpoint{9.009782in}{1.525252in}}%
\pgfpathlineto{\pgfqpoint{9.024757in}{1.520418in}}%
\pgfpathlineto{\pgfqpoint{9.039655in}{1.515750in}}%
\pgfpathlineto{\pgfqpoint{9.054477in}{1.511239in}}%
\pgfpathlineto{\pgfqpoint{9.069226in}{1.506878in}}%
\pgfpathlineto{\pgfqpoint{9.083903in}{1.502662in}}%
\pgfpathlineto{\pgfqpoint{9.098511in}{1.498580in}}%
\pgfpathlineto{\pgfqpoint{9.113049in}{1.494629in}}%
\pgfpathlineto{\pgfqpoint{9.127521in}{1.490801in}}%
\pgfpathlineto{\pgfqpoint{9.141927in}{1.487091in}}%
\pgfpathlineto{\pgfqpoint{9.156268in}{1.483494in}}%
\pgfpathlineto{\pgfqpoint{9.170545in}{1.480004in}}%
\pgfpathlineto{\pgfqpoint{9.184759in}{1.476617in}}%
\pgfpathlineto{\pgfqpoint{9.198911in}{1.473327in}}%
\pgfpathlineto{\pgfqpoint{9.213003in}{1.470132in}}%
\pgfpathlineto{\pgfqpoint{9.227035in}{1.467026in}}%
\pgfpathlineto{\pgfqpoint{9.241008in}{1.464006in}}%
\pgfpathlineto{\pgfqpoint{9.254923in}{1.461069in}}%
\pgfpathlineto{\pgfqpoint{9.268781in}{1.458210in}}%
\pgfpathlineto{\pgfqpoint{9.282584in}{1.455427in}}%
\pgfpathlineto{\pgfqpoint{9.296333in}{1.452716in}}%
\pgfpathlineto{\pgfqpoint{9.310029in}{1.450075in}}%
\pgfpathlineto{\pgfqpoint{9.323673in}{1.447501in}}%
\pgfpathlineto{\pgfqpoint{9.337267in}{1.444991in}}%
\pgfpathlineto{\pgfqpoint{9.350812in}{1.442541in}}%
\pgfpathlineto{\pgfqpoint{9.364310in}{1.440152in}}%
\pgfpathlineto{\pgfqpoint{9.377762in}{1.437819in}}%
\pgfpathlineto{\pgfqpoint{9.391168in}{1.435541in}}%
\pgfpathlineto{\pgfqpoint{9.404532in}{1.433316in}}%
\pgfpathlineto{\pgfqpoint{9.417853in}{1.431140in}}%
\pgfpathlineto{\pgfqpoint{9.431133in}{1.429014in}}%
\pgfpathlineto{\pgfqpoint{9.444373in}{1.426936in}}%
\pgfpathlineto{\pgfqpoint{9.457574in}{1.424902in}}%
\pgfpathlineto{\pgfqpoint{9.470737in}{1.422912in}}%
\pgfpathlineto{\pgfqpoint{9.483862in}{1.420964in}}%
\pgfpathlineto{\pgfqpoint{9.496952in}{1.419057in}}%
\pgfpathlineto{\pgfqpoint{9.510006in}{1.417189in}}%
\pgfpathlineto{\pgfqpoint{9.523026in}{1.415359in}}%
\pgfpathlineto{\pgfqpoint{9.536011in}{1.413565in}}%
\pgfpathlineto{\pgfqpoint{9.548963in}{1.411806in}}%
\pgfpathlineto{\pgfqpoint{9.561882in}{1.410082in}}%
\pgfpathlineto{\pgfqpoint{9.574769in}{1.408391in}}%
\pgfpathlineto{\pgfqpoint{9.587624in}{1.406732in}}%
\pgfpathlineto{\pgfqpoint{9.600448in}{1.405103in}}%
\pgfpathlineto{\pgfqpoint{9.613242in}{1.403504in}}%
\pgfpathlineto{\pgfqpoint{9.626006in}{1.401934in}}%
\pgfpathlineto{\pgfqpoint{9.638740in}{1.400392in}}%
\pgfpathlineto{\pgfqpoint{9.651445in}{1.398878in}}%
\pgfpathlineto{\pgfqpoint{9.664121in}{1.397389in}}%
\pgfpathlineto{\pgfqpoint{9.676770in}{1.395926in}}%
\pgfpathlineto{\pgfqpoint{9.689391in}{1.394488in}}%
\pgfpathlineto{\pgfqpoint{9.701986in}{1.393073in}}%
\pgfpathlineto{\pgfqpoint{9.714553in}{1.391682in}}%
\pgfpathlineto{\pgfqpoint{9.727095in}{1.390313in}}%
\pgfpathlineto{\pgfqpoint{9.739611in}{1.388965in}}%
\pgfpathlineto{\pgfqpoint{9.752102in}{1.387638in}}%
\pgfpathlineto{\pgfqpoint{9.764568in}{1.386332in}}%
\pgfpathlineto{\pgfqpoint{9.777009in}{1.385047in}}%
\pgfpathlineto{\pgfqpoint{9.789426in}{1.383781in}}%
\pgfpathlineto{\pgfqpoint{9.801818in}{1.382533in}}%
\pgfpathlineto{\pgfqpoint{9.814187in}{1.381304in}}%
\pgfpathlineto{\pgfqpoint{9.826533in}{1.380092in}}%
\pgfpathlineto{\pgfqpoint{9.838855in}{1.378897in}}%
\pgfpathlineto{\pgfqpoint{9.851154in}{1.377720in}}%
\pgfpathlineto{\pgfqpoint{9.863430in}{1.376559in}}%
\pgfpathlineto{\pgfqpoint{9.875684in}{1.375413in}}%
\pgfpathlineto{\pgfqpoint{9.887916in}{1.374284in}}%
\pgfpathlineto{\pgfqpoint{9.900125in}{1.373169in}}%
\pgfpathlineto{\pgfqpoint{9.912313in}{1.372068in}}%
\pgfpathlineto{\pgfqpoint{9.924479in}{1.370982in}}%
\pgfpathlineto{\pgfqpoint{9.936625in}{1.369910in}}%
\pgfpathlineto{\pgfqpoint{9.948749in}{1.368852in}}%
\pgfpathlineto{\pgfqpoint{9.960853in}{1.367806in}}%
\pgfpathlineto{\pgfqpoint{9.972938in}{1.366774in}}%
\pgfpathlineto{\pgfqpoint{9.985002in}{1.365754in}}%
\pgfpathlineto{\pgfqpoint{9.997047in}{1.364746in}}%
\pgfpathlineto{\pgfqpoint{10.009074in}{1.363750in}}%
\pgfpathlineto{\pgfqpoint{10.021082in}{1.362766in}}%
\pgfpathlineto{\pgfqpoint{10.033071in}{1.361793in}}%
\pgfpathlineto{\pgfqpoint{10.045043in}{1.360831in}}%
\pgfpathlineto{\pgfqpoint{10.056997in}{1.359880in}}%
\pgfpathlineto{\pgfqpoint{10.068934in}{1.358939in}}%
\pgfpathlineto{\pgfqpoint{10.080854in}{1.358009in}}%
\pgfpathlineto{\pgfqpoint{10.092757in}{1.357089in}}%
\pgfusepath{stroke}%
\end{pgfscope}%
\begin{pgfscope}%
\pgfpathrectangle{\pgfqpoint{8.282041in}{0.790446in}}{\pgfqpoint{1.897959in}{1.372727in}}%
\pgfusepath{clip}%
\pgfsetrectcap%
\pgfsetroundjoin%
\pgfsetlinewidth{1.505625pt}%
\definecolor{currentstroke}{rgb}{1.000000,0.000000,0.000000}%
\pgfsetstrokecolor{currentstroke}%
\pgfsetstrokeopacity{0.750000}%
\pgfsetdash{}{0pt}%
\pgfpathmoveto{\pgfqpoint{8.523435in}{1.990240in}}%
\pgfpathlineto{\pgfqpoint{8.543513in}{1.997571in}}%
\pgfpathlineto{\pgfqpoint{8.563540in}{2.004647in}}%
\pgfpathlineto{\pgfqpoint{8.583512in}{2.010561in}}%
\pgfpathlineto{\pgfqpoint{8.603426in}{2.015544in}}%
\pgfpathlineto{\pgfqpoint{8.623279in}{2.019765in}}%
\pgfpathlineto{\pgfqpoint{8.643070in}{2.023358in}}%
\pgfpathlineto{\pgfqpoint{8.662795in}{2.026426in}}%
\pgfpathlineto{\pgfqpoint{8.682451in}{2.029056in}}%
\pgfpathlineto{\pgfqpoint{8.702036in}{2.031312in}}%
\pgfpathlineto{\pgfqpoint{8.721547in}{2.033250in}}%
\pgfpathlineto{\pgfqpoint{8.740981in}{2.034915in}}%
\pgfpathlineto{\pgfqpoint{8.760335in}{2.036345in}}%
\pgfpathlineto{\pgfqpoint{8.779604in}{2.037569in}}%
\pgfpathlineto{\pgfqpoint{8.798784in}{2.038614in}}%
\pgfpathlineto{\pgfqpoint{8.817875in}{2.039500in}}%
\pgfpathlineto{\pgfqpoint{8.836873in}{2.040247in}}%
\pgfpathlineto{\pgfqpoint{8.855777in}{2.040869in}}%
\pgfpathlineto{\pgfqpoint{8.874586in}{2.041381in}}%
\pgfpathlineto{\pgfqpoint{8.893298in}{2.041793in}}%
\pgfpathlineto{\pgfqpoint{8.911913in}{2.042116in}}%
\pgfpathlineto{\pgfqpoint{8.930432in}{2.042359in}}%
\pgfpathlineto{\pgfqpoint{8.948855in}{2.042529in}}%
\pgfpathlineto{\pgfqpoint{8.967183in}{2.042632in}}%
\pgfpathlineto{\pgfqpoint{8.985417in}{2.042675in}}%
\pgfpathlineto{\pgfqpoint{9.003560in}{2.042662in}}%
\pgfpathlineto{\pgfqpoint{9.021614in}{2.042599in}}%
\pgfpathlineto{\pgfqpoint{9.039581in}{2.042489in}}%
\pgfpathlineto{\pgfqpoint{9.057464in}{2.042335in}}%
\pgfpathlineto{\pgfqpoint{9.075266in}{2.042142in}}%
\pgfpathlineto{\pgfqpoint{9.092989in}{2.041911in}}%
\pgfpathlineto{\pgfqpoint{9.110637in}{2.041647in}}%
\pgfpathlineto{\pgfqpoint{9.128213in}{2.041350in}}%
\pgfpathlineto{\pgfqpoint{9.145719in}{2.041024in}}%
\pgfpathlineto{\pgfqpoint{9.163158in}{2.040670in}}%
\pgfpathlineto{\pgfqpoint{9.180533in}{2.040289in}}%
\pgfpathlineto{\pgfqpoint{9.197848in}{2.039885in}}%
\pgfpathlineto{\pgfqpoint{9.215105in}{2.039457in}}%
\pgfpathlineto{\pgfqpoint{9.232306in}{2.039008in}}%
\pgfpathlineto{\pgfqpoint{9.249454in}{2.038538in}}%
\pgfpathlineto{\pgfqpoint{9.266551in}{2.038049in}}%
\pgfpathlineto{\pgfqpoint{9.283598in}{2.037542in}}%
\pgfpathlineto{\pgfqpoint{9.300599in}{2.037019in}}%
\pgfpathlineto{\pgfqpoint{9.317555in}{2.036479in}}%
\pgfpathlineto{\pgfqpoint{9.334468in}{2.035923in}}%
\pgfpathlineto{\pgfqpoint{9.351339in}{2.035353in}}%
\pgfpathlineto{\pgfqpoint{9.368170in}{2.034768in}}%
\pgfpathlineto{\pgfqpoint{9.384961in}{2.034171in}}%
\pgfpathlineto{\pgfqpoint{9.401715in}{2.033561in}}%
\pgfpathlineto{\pgfqpoint{9.418432in}{2.032939in}}%
\pgfpathlineto{\pgfqpoint{9.435113in}{2.032305in}}%
\pgfpathlineto{\pgfqpoint{9.451761in}{2.031661in}}%
\pgfpathlineto{\pgfqpoint{9.468375in}{2.031006in}}%
\pgfpathlineto{\pgfqpoint{9.484958in}{2.030342in}}%
\pgfpathlineto{\pgfqpoint{9.501509in}{2.029667in}}%
\pgfpathlineto{\pgfqpoint{9.518029in}{2.028984in}}%
\pgfpathlineto{\pgfqpoint{9.534518in}{2.028292in}}%
\pgfpathlineto{\pgfqpoint{9.550978in}{2.027592in}}%
\pgfpathlineto{\pgfqpoint{9.567408in}{2.026884in}}%
\pgfpathlineto{\pgfqpoint{9.583810in}{2.026168in}}%
\pgfpathlineto{\pgfqpoint{9.600182in}{2.025445in}}%
\pgfpathlineto{\pgfqpoint{9.616527in}{2.024715in}}%
\pgfpathlineto{\pgfqpoint{9.632842in}{2.023978in}}%
\pgfpathlineto{\pgfqpoint{9.649129in}{2.023235in}}%
\pgfpathlineto{\pgfqpoint{9.665386in}{2.022486in}}%
\pgfpathlineto{\pgfqpoint{9.681613in}{2.021731in}}%
\pgfpathlineto{\pgfqpoint{9.697812in}{2.020969in}}%
\pgfpathlineto{\pgfqpoint{9.713981in}{2.020203in}}%
\pgfpathlineto{\pgfqpoint{9.730120in}{2.019431in}}%
\pgfpathlineto{\pgfqpoint{9.746231in}{2.018655in}}%
\pgfpathlineto{\pgfqpoint{9.762311in}{2.017874in}}%
\pgfpathlineto{\pgfqpoint{9.778361in}{2.017088in}}%
\pgfpathlineto{\pgfqpoint{9.794382in}{2.016296in}}%
\pgfpathlineto{\pgfqpoint{9.810374in}{2.015500in}}%
\pgfpathlineto{\pgfqpoint{9.826337in}{2.014701in}}%
\pgfpathlineto{\pgfqpoint{9.842273in}{2.013898in}}%
\pgfpathlineto{\pgfqpoint{9.858181in}{2.013092in}}%
\pgfpathlineto{\pgfqpoint{9.874063in}{2.012281in}}%
\pgfpathlineto{\pgfqpoint{9.889917in}{2.011466in}}%
\pgfpathlineto{\pgfqpoint{9.905745in}{2.010648in}}%
\pgfpathlineto{\pgfqpoint{9.921548in}{2.009827in}}%
\pgfpathlineto{\pgfqpoint{9.937324in}{2.009004in}}%
\pgfpathlineto{\pgfqpoint{9.953077in}{2.008177in}}%
\pgfpathlineto{\pgfqpoint{9.968804in}{2.007348in}}%
\pgfpathlineto{\pgfqpoint{9.984508in}{2.006515in}}%
\pgfpathlineto{\pgfqpoint{10.000186in}{2.005679in}}%
\pgfpathlineto{\pgfqpoint{10.015840in}{2.004842in}}%
\pgfpathlineto{\pgfqpoint{10.031468in}{2.004002in}}%
\pgfpathlineto{\pgfqpoint{10.047071in}{2.003160in}}%
\pgfpathlineto{\pgfqpoint{10.062648in}{2.002316in}}%
\pgfpathlineto{\pgfqpoint{10.078201in}{2.001469in}}%
\pgfpathlineto{\pgfqpoint{10.093729in}{2.000620in}}%
\pgfusepath{stroke}%
\end{pgfscope}%
\begin{pgfscope}%
\pgfpathrectangle{\pgfqpoint{8.282041in}{0.790446in}}{\pgfqpoint{1.897959in}{1.372727in}}%
\pgfusepath{clip}%
\pgfsetrectcap%
\pgfsetroundjoin%
\pgfsetlinewidth{1.505625pt}%
\definecolor{currentstroke}{rgb}{0.000000,0.000000,1.000000}%
\pgfsetstrokecolor{currentstroke}%
\pgfsetstrokeopacity{0.750000}%
\pgfsetdash{}{0pt}%
\pgfpathmoveto{\pgfqpoint{8.523435in}{1.502193in}}%
\pgfpathlineto{\pgfqpoint{8.543513in}{1.512919in}}%
\pgfpathlineto{\pgfqpoint{8.563540in}{1.523516in}}%
\pgfpathlineto{\pgfqpoint{8.583512in}{1.533059in}}%
\pgfpathlineto{\pgfqpoint{8.603426in}{1.541762in}}%
\pgfpathlineto{\pgfqpoint{8.623279in}{1.549783in}}%
\pgfpathlineto{\pgfqpoint{8.643070in}{1.557241in}}%
\pgfpathlineto{\pgfqpoint{8.662795in}{1.564234in}}%
\pgfpathlineto{\pgfqpoint{8.682451in}{1.570837in}}%
\pgfpathlineto{\pgfqpoint{8.702036in}{1.577109in}}%
\pgfpathlineto{\pgfqpoint{8.721547in}{1.583100in}}%
\pgfpathlineto{\pgfqpoint{8.740981in}{1.588849in}}%
\pgfpathlineto{\pgfqpoint{8.760335in}{1.594388in}}%
\pgfpathlineto{\pgfqpoint{8.779604in}{1.599743in}}%
\pgfpathlineto{\pgfqpoint{8.798784in}{1.604936in}}%
\pgfpathlineto{\pgfqpoint{8.817875in}{1.609984in}}%
\pgfpathlineto{\pgfqpoint{8.836873in}{1.614902in}}%
\pgfpathlineto{\pgfqpoint{8.855777in}{1.619702in}}%
\pgfpathlineto{\pgfqpoint{8.874586in}{1.624396in}}%
\pgfpathlineto{\pgfqpoint{8.893298in}{1.628992in}}%
\pgfpathlineto{\pgfqpoint{8.911913in}{1.633498in}}%
\pgfpathlineto{\pgfqpoint{8.930432in}{1.637920in}}%
\pgfpathlineto{\pgfqpoint{8.948855in}{1.642264in}}%
\pgfpathlineto{\pgfqpoint{8.967183in}{1.646535in}}%
\pgfpathlineto{\pgfqpoint{8.985417in}{1.650736in}}%
\pgfpathlineto{\pgfqpoint{9.003560in}{1.654871in}}%
\pgfpathlineto{\pgfqpoint{9.021614in}{1.658944in}}%
\pgfpathlineto{\pgfqpoint{9.039581in}{1.662956in}}%
\pgfpathlineto{\pgfqpoint{9.057464in}{1.666911in}}%
\pgfpathlineto{\pgfqpoint{9.075266in}{1.670810in}}%
\pgfpathlineto{\pgfqpoint{9.092989in}{1.674656in}}%
\pgfpathlineto{\pgfqpoint{9.110637in}{1.678450in}}%
\pgfpathlineto{\pgfqpoint{9.128213in}{1.682194in}}%
\pgfpathlineto{\pgfqpoint{9.145719in}{1.685888in}}%
\pgfpathlineto{\pgfqpoint{9.163158in}{1.689536in}}%
\pgfpathlineto{\pgfqpoint{9.180533in}{1.693136in}}%
\pgfpathlineto{\pgfqpoint{9.197848in}{1.696691in}}%
\pgfpathlineto{\pgfqpoint{9.215105in}{1.700202in}}%
\pgfpathlineto{\pgfqpoint{9.232306in}{1.703669in}}%
\pgfpathlineto{\pgfqpoint{9.249454in}{1.707094in}}%
\pgfpathlineto{\pgfqpoint{9.266551in}{1.710477in}}%
\pgfpathlineto{\pgfqpoint{9.283598in}{1.713818in}}%
\pgfpathlineto{\pgfqpoint{9.300599in}{1.717119in}}%
\pgfpathlineto{\pgfqpoint{9.317555in}{1.720381in}}%
\pgfpathlineto{\pgfqpoint{9.334468in}{1.723603in}}%
\pgfpathlineto{\pgfqpoint{9.351339in}{1.726786in}}%
\pgfpathlineto{\pgfqpoint{9.368170in}{1.729932in}}%
\pgfpathlineto{\pgfqpoint{9.384961in}{1.733040in}}%
\pgfpathlineto{\pgfqpoint{9.401715in}{1.736112in}}%
\pgfpathlineto{\pgfqpoint{9.418432in}{1.739147in}}%
\pgfpathlineto{\pgfqpoint{9.435113in}{1.742146in}}%
\pgfpathlineto{\pgfqpoint{9.451761in}{1.745110in}}%
\pgfpathlineto{\pgfqpoint{9.468375in}{1.748040in}}%
\pgfpathlineto{\pgfqpoint{9.484958in}{1.750935in}}%
\pgfpathlineto{\pgfqpoint{9.501509in}{1.753796in}}%
\pgfpathlineto{\pgfqpoint{9.518029in}{1.756624in}}%
\pgfpathlineto{\pgfqpoint{9.534518in}{1.759420in}}%
\pgfpathlineto{\pgfqpoint{9.550978in}{1.762183in}}%
\pgfpathlineto{\pgfqpoint{9.567408in}{1.764913in}}%
\pgfpathlineto{\pgfqpoint{9.583810in}{1.767613in}}%
\pgfpathlineto{\pgfqpoint{9.600182in}{1.770281in}}%
\pgfpathlineto{\pgfqpoint{9.616527in}{1.772919in}}%
\pgfpathlineto{\pgfqpoint{9.632842in}{1.775526in}}%
\pgfpathlineto{\pgfqpoint{9.649129in}{1.778103in}}%
\pgfpathlineto{\pgfqpoint{9.665386in}{1.780651in}}%
\pgfpathlineto{\pgfqpoint{9.681613in}{1.783170in}}%
\pgfpathlineto{\pgfqpoint{9.697812in}{1.785660in}}%
\pgfpathlineto{\pgfqpoint{9.713981in}{1.788122in}}%
\pgfpathlineto{\pgfqpoint{9.730120in}{1.790556in}}%
\pgfpathlineto{\pgfqpoint{9.746231in}{1.792963in}}%
\pgfpathlineto{\pgfqpoint{9.762311in}{1.795342in}}%
\pgfpathlineto{\pgfqpoint{9.778361in}{1.797694in}}%
\pgfpathlineto{\pgfqpoint{9.794382in}{1.800020in}}%
\pgfpathlineto{\pgfqpoint{9.810374in}{1.802319in}}%
\pgfpathlineto{\pgfqpoint{9.826337in}{1.804592in}}%
\pgfpathlineto{\pgfqpoint{9.842273in}{1.806841in}}%
\pgfpathlineto{\pgfqpoint{9.858181in}{1.809065in}}%
\pgfpathlineto{\pgfqpoint{9.874063in}{1.811263in}}%
\pgfpathlineto{\pgfqpoint{9.889917in}{1.813437in}}%
\pgfpathlineto{\pgfqpoint{9.905745in}{1.815587in}}%
\pgfpathlineto{\pgfqpoint{9.921548in}{1.817713in}}%
\pgfpathlineto{\pgfqpoint{9.937324in}{1.819816in}}%
\pgfpathlineto{\pgfqpoint{9.953077in}{1.821896in}}%
\pgfpathlineto{\pgfqpoint{9.968804in}{1.823953in}}%
\pgfpathlineto{\pgfqpoint{9.984508in}{1.825987in}}%
\pgfpathlineto{\pgfqpoint{10.000186in}{1.827998in}}%
\pgfpathlineto{\pgfqpoint{10.015840in}{1.829988in}}%
\pgfpathlineto{\pgfqpoint{10.031468in}{1.831957in}}%
\pgfpathlineto{\pgfqpoint{10.047071in}{1.833904in}}%
\pgfpathlineto{\pgfqpoint{10.062648in}{1.835830in}}%
\pgfpathlineto{\pgfqpoint{10.078201in}{1.837735in}}%
\pgfpathlineto{\pgfqpoint{10.093729in}{1.839619in}}%
\pgfusepath{stroke}%
\end{pgfscope}%
\begin{pgfscope}%
\pgfpathrectangle{\pgfqpoint{8.282041in}{0.790446in}}{\pgfqpoint{1.897959in}{1.372727in}}%
\pgfusepath{clip}%
\pgfsetrectcap%
\pgfsetroundjoin%
\pgfsetlinewidth{1.505625pt}%
\definecolor{currentstroke}{rgb}{0.000000,0.750000,0.750000}%
\pgfsetstrokecolor{currentstroke}%
\pgfsetstrokeopacity{0.750000}%
\pgfsetdash{}{0pt}%
\pgfpathmoveto{\pgfqpoint{8.523435in}{1.834646in}}%
\pgfpathlineto{\pgfqpoint{8.543513in}{1.813246in}}%
\pgfpathlineto{\pgfqpoint{8.563540in}{1.794148in}}%
\pgfpathlineto{\pgfqpoint{8.583512in}{1.776098in}}%
\pgfpathlineto{\pgfqpoint{8.603426in}{1.759044in}}%
\pgfpathlineto{\pgfqpoint{8.623279in}{1.742924in}}%
\pgfpathlineto{\pgfqpoint{8.643070in}{1.727676in}}%
\pgfpathlineto{\pgfqpoint{8.662795in}{1.713242in}}%
\pgfpathlineto{\pgfqpoint{8.682451in}{1.699568in}}%
\pgfpathlineto{\pgfqpoint{8.702036in}{1.686602in}}%
\pgfpathlineto{\pgfqpoint{8.721547in}{1.674294in}}%
\pgfpathlineto{\pgfqpoint{8.740981in}{1.662603in}}%
\pgfpathlineto{\pgfqpoint{8.760335in}{1.651485in}}%
\pgfpathlineto{\pgfqpoint{8.779604in}{1.640902in}}%
\pgfpathlineto{\pgfqpoint{8.798784in}{1.630820in}}%
\pgfpathlineto{\pgfqpoint{8.817875in}{1.621204in}}%
\pgfpathlineto{\pgfqpoint{8.836873in}{1.612024in}}%
\pgfpathlineto{\pgfqpoint{8.855777in}{1.603254in}}%
\pgfpathlineto{\pgfqpoint{8.874586in}{1.594866in}}%
\pgfpathlineto{\pgfqpoint{8.893298in}{1.586838in}}%
\pgfpathlineto{\pgfqpoint{8.911913in}{1.579146in}}%
\pgfpathlineto{\pgfqpoint{8.930432in}{1.571770in}}%
\pgfpathlineto{\pgfqpoint{8.948855in}{1.564692in}}%
\pgfpathlineto{\pgfqpoint{8.967183in}{1.557894in}}%
\pgfpathlineto{\pgfqpoint{8.985417in}{1.551358in}}%
\pgfpathlineto{\pgfqpoint{9.003560in}{1.545072in}}%
\pgfpathlineto{\pgfqpoint{9.021614in}{1.539019in}}%
\pgfpathlineto{\pgfqpoint{9.039581in}{1.533186in}}%
\pgfpathlineto{\pgfqpoint{9.057464in}{1.527561in}}%
\pgfpathlineto{\pgfqpoint{9.075266in}{1.522134in}}%
\pgfpathlineto{\pgfqpoint{9.092989in}{1.516893in}}%
\pgfpathlineto{\pgfqpoint{9.110637in}{1.511827in}}%
\pgfpathlineto{\pgfqpoint{9.128213in}{1.506929in}}%
\pgfpathlineto{\pgfqpoint{9.145719in}{1.502189in}}%
\pgfpathlineto{\pgfqpoint{9.163158in}{1.497598in}}%
\pgfpathlineto{\pgfqpoint{9.180533in}{1.493150in}}%
\pgfpathlineto{\pgfqpoint{9.197848in}{1.488836in}}%
\pgfpathlineto{\pgfqpoint{9.215105in}{1.484651in}}%
\pgfpathlineto{\pgfqpoint{9.232306in}{1.480588in}}%
\pgfpathlineto{\pgfqpoint{9.249454in}{1.476640in}}%
\pgfpathlineto{\pgfqpoint{9.266551in}{1.472802in}}%
\pgfpathlineto{\pgfqpoint{9.283598in}{1.469070in}}%
\pgfpathlineto{\pgfqpoint{9.300599in}{1.465438in}}%
\pgfpathlineto{\pgfqpoint{9.317555in}{1.461901in}}%
\pgfpathlineto{\pgfqpoint{9.334468in}{1.458454in}}%
\pgfpathlineto{\pgfqpoint{9.351339in}{1.455094in}}%
\pgfpathlineto{\pgfqpoint{9.368170in}{1.451816in}}%
\pgfpathlineto{\pgfqpoint{9.384961in}{1.448618in}}%
\pgfpathlineto{\pgfqpoint{9.401715in}{1.445495in}}%
\pgfpathlineto{\pgfqpoint{9.418432in}{1.442444in}}%
\pgfpathlineto{\pgfqpoint{9.435113in}{1.439462in}}%
\pgfpathlineto{\pgfqpoint{9.451761in}{1.436546in}}%
\pgfpathlineto{\pgfqpoint{9.468375in}{1.433693in}}%
\pgfpathlineto{\pgfqpoint{9.484958in}{1.430901in}}%
\pgfpathlineto{\pgfqpoint{9.501509in}{1.428166in}}%
\pgfpathlineto{\pgfqpoint{9.518029in}{1.425487in}}%
\pgfpathlineto{\pgfqpoint{9.534518in}{1.422862in}}%
\pgfpathlineto{\pgfqpoint{9.550978in}{1.420287in}}%
\pgfpathlineto{\pgfqpoint{9.567408in}{1.417761in}}%
\pgfpathlineto{\pgfqpoint{9.583810in}{1.415283in}}%
\pgfpathlineto{\pgfqpoint{9.600182in}{1.412850in}}%
\pgfpathlineto{\pgfqpoint{9.616527in}{1.410460in}}%
\pgfpathlineto{\pgfqpoint{9.632842in}{1.408112in}}%
\pgfpathlineto{\pgfqpoint{9.649129in}{1.405804in}}%
\pgfpathlineto{\pgfqpoint{9.665386in}{1.403535in}}%
\pgfpathlineto{\pgfqpoint{9.681613in}{1.401303in}}%
\pgfpathlineto{\pgfqpoint{9.697812in}{1.399106in}}%
\pgfpathlineto{\pgfqpoint{9.713981in}{1.396945in}}%
\pgfpathlineto{\pgfqpoint{9.730120in}{1.394817in}}%
\pgfpathlineto{\pgfqpoint{9.746231in}{1.392721in}}%
\pgfpathlineto{\pgfqpoint{9.762311in}{1.390656in}}%
\pgfpathlineto{\pgfqpoint{9.778361in}{1.388620in}}%
\pgfpathlineto{\pgfqpoint{9.794382in}{1.386613in}}%
\pgfpathlineto{\pgfqpoint{9.810374in}{1.384633in}}%
\pgfpathlineto{\pgfqpoint{9.826337in}{1.382680in}}%
\pgfpathlineto{\pgfqpoint{9.842273in}{1.380754in}}%
\pgfpathlineto{\pgfqpoint{9.858181in}{1.378853in}}%
\pgfpathlineto{\pgfqpoint{9.874063in}{1.376976in}}%
\pgfpathlineto{\pgfqpoint{9.889917in}{1.375122in}}%
\pgfpathlineto{\pgfqpoint{9.905745in}{1.373290in}}%
\pgfpathlineto{\pgfqpoint{9.921548in}{1.371481in}}%
\pgfpathlineto{\pgfqpoint{9.937324in}{1.369694in}}%
\pgfpathlineto{\pgfqpoint{9.953077in}{1.367926in}}%
\pgfpathlineto{\pgfqpoint{9.968804in}{1.366179in}}%
\pgfpathlineto{\pgfqpoint{9.984508in}{1.364451in}}%
\pgfpathlineto{\pgfqpoint{10.000186in}{1.362741in}}%
\pgfpathlineto{\pgfqpoint{10.015840in}{1.361050in}}%
\pgfpathlineto{\pgfqpoint{10.031468in}{1.359376in}}%
\pgfpathlineto{\pgfqpoint{10.047071in}{1.357720in}}%
\pgfpathlineto{\pgfqpoint{10.062648in}{1.356080in}}%
\pgfpathlineto{\pgfqpoint{10.078201in}{1.354456in}}%
\pgfpathlineto{\pgfqpoint{10.093729in}{1.352848in}}%
\pgfusepath{stroke}%
\end{pgfscope}%
\begin{pgfscope}%
\pgfsetrectcap%
\pgfsetmiterjoin%
\pgfsetlinewidth{0.803000pt}%
\definecolor{currentstroke}{rgb}{0.501961,0.501961,0.501961}%
\pgfsetstrokecolor{currentstroke}%
\pgfsetdash{}{0pt}%
\pgfpathmoveto{\pgfqpoint{8.282041in}{0.790446in}}%
\pgfpathlineto{\pgfqpoint{8.282041in}{2.163173in}}%
\pgfusepath{stroke}%
\end{pgfscope}%
\begin{pgfscope}%
\pgfsetrectcap%
\pgfsetmiterjoin%
\pgfsetlinewidth{0.803000pt}%
\definecolor{currentstroke}{rgb}{0.501961,0.501961,0.501961}%
\pgfsetstrokecolor{currentstroke}%
\pgfsetdash{}{0pt}%
\pgfpathmoveto{\pgfqpoint{10.180000in}{0.790446in}}%
\pgfpathlineto{\pgfqpoint{10.180000in}{2.163173in}}%
\pgfusepath{stroke}%
\end{pgfscope}%
\begin{pgfscope}%
\pgfsetrectcap%
\pgfsetmiterjoin%
\pgfsetlinewidth{0.803000pt}%
\definecolor{currentstroke}{rgb}{0.501961,0.501961,0.501961}%
\pgfsetstrokecolor{currentstroke}%
\pgfsetdash{}{0pt}%
\pgfpathmoveto{\pgfqpoint{8.282041in}{0.790446in}}%
\pgfpathlineto{\pgfqpoint{10.180000in}{0.790446in}}%
\pgfusepath{stroke}%
\end{pgfscope}%
\begin{pgfscope}%
\pgfsetrectcap%
\pgfsetmiterjoin%
\pgfsetlinewidth{0.803000pt}%
\definecolor{currentstroke}{rgb}{0.501961,0.501961,0.501961}%
\pgfsetstrokecolor{currentstroke}%
\pgfsetdash{}{0pt}%
\pgfpathmoveto{\pgfqpoint{8.282041in}{2.163173in}}%
\pgfpathlineto{\pgfqpoint{10.180000in}{2.163173in}}%
\pgfusepath{stroke}%
\end{pgfscope}%
\begin{pgfscope}%
\pgfsetrectcap%
\pgfsetroundjoin%
\pgfsetlinewidth{1.505625pt}%
\definecolor{currentstroke}{rgb}{1.000000,0.000000,0.000000}%
\pgfsetstrokecolor{currentstroke}%
\pgfsetstrokeopacity{0.750000}%
\pgfsetdash{}{0pt}%
\pgfpathmoveto{\pgfqpoint{10.637370in}{17.841827in}}%
\pgfpathlineto{\pgfqpoint{11.026259in}{17.841827in}}%
\pgfusepath{stroke}%
\end{pgfscope}%
\begin{pgfscope}%
\definecolor{textcolor}{rgb}{0.501961,0.501961,0.501961}%
\pgfsetstrokecolor{textcolor}%
\pgfsetfillcolor{textcolor}%
\pgftext[x=11.181814in,y=17.773771in,left,base]{\color{textcolor}\rmfamily\fontsize{14.000000}{16.800000}\selectfont Coulomb force}%
\end{pgfscope}%
\begin{pgfscope}%
\pgfsetrectcap%
\pgfsetroundjoin%
\pgfsetlinewidth{1.505625pt}%
\definecolor{currentstroke}{rgb}{0.000000,0.000000,1.000000}%
\pgfsetstrokecolor{currentstroke}%
\pgfsetstrokeopacity{0.750000}%
\pgfsetdash{}{0pt}%
\pgfpathmoveto{\pgfqpoint{10.637370in}{17.556426in}}%
\pgfpathlineto{\pgfqpoint{11.026259in}{17.556426in}}%
\pgfusepath{stroke}%
\end{pgfscope}%
\begin{pgfscope}%
\definecolor{textcolor}{rgb}{0.501961,0.501961,0.501961}%
\pgfsetstrokecolor{textcolor}%
\pgfsetfillcolor{textcolor}%
\pgftext[x=11.181814in,y=17.488371in,left,base]{\color{textcolor}\rmfamily\fontsize{14.000000}{16.800000}\selectfont Drag force}%
\end{pgfscope}%
\begin{pgfscope}%
\pgfsetrectcap%
\pgfsetroundjoin%
\pgfsetlinewidth{1.505625pt}%
\definecolor{currentstroke}{rgb}{0.000000,0.750000,0.750000}%
\pgfsetstrokecolor{currentstroke}%
\pgfsetstrokeopacity{0.750000}%
\pgfsetdash{}{0pt}%
\pgfpathmoveto{\pgfqpoint{10.637370in}{17.268273in}}%
\pgfpathlineto{\pgfqpoint{11.026259in}{17.268273in}}%
\pgfusepath{stroke}%
\end{pgfscope}%
\begin{pgfscope}%
\definecolor{textcolor}{rgb}{0.501961,0.501961,0.501961}%
\pgfsetstrokecolor{textcolor}%
\pgfsetfillcolor{textcolor}%
\pgftext[x=11.181814in,y=17.200217in,left,base]{\color{textcolor}\rmfamily\fontsize{14.000000}{16.800000}\selectfont Image force}%
\end{pgfscope}%
\begin{pgfscope}%
\pgfsetrectcap%
\pgfsetroundjoin%
\pgfsetlinewidth{1.505625pt}%
\definecolor{currentstroke}{rgb}{1.000000,0.000000,0.000000}%
\pgfsetstrokecolor{currentstroke}%
\pgfsetstrokeopacity{0.750000}%
\pgfsetdash{}{0pt}%
\pgfpathmoveto{\pgfqpoint{10.637370in}{16.980119in}}%
\pgfpathlineto{\pgfqpoint{11.026259in}{16.980119in}}%
\pgfusepath{stroke}%
\end{pgfscope}%
\begin{pgfscope}%
\definecolor{textcolor}{rgb}{0.501961,0.501961,0.501961}%
\pgfsetstrokecolor{textcolor}%
\pgfsetfillcolor{textcolor}%
\pgftext[x=11.181814in,y=16.912063in,left,base]{\color{textcolor}\rmfamily\fontsize{14.000000}{16.800000}\selectfont Coulomb force}%
\end{pgfscope}%
\begin{pgfscope}%
\pgfsetrectcap%
\pgfsetroundjoin%
\pgfsetlinewidth{1.505625pt}%
\definecolor{currentstroke}{rgb}{0.000000,0.000000,1.000000}%
\pgfsetstrokecolor{currentstroke}%
\pgfsetstrokeopacity{0.750000}%
\pgfsetdash{}{0pt}%
\pgfpathmoveto{\pgfqpoint{10.637370in}{16.694719in}}%
\pgfpathlineto{\pgfqpoint{11.026259in}{16.694719in}}%
\pgfusepath{stroke}%
\end{pgfscope}%
\begin{pgfscope}%
\definecolor{textcolor}{rgb}{0.501961,0.501961,0.501961}%
\pgfsetstrokecolor{textcolor}%
\pgfsetfillcolor{textcolor}%
\pgftext[x=11.181814in,y=16.626663in,left,base]{\color{textcolor}\rmfamily\fontsize{14.000000}{16.800000}\selectfont Drag force}%
\end{pgfscope}%
\begin{pgfscope}%
\pgfsetrectcap%
\pgfsetroundjoin%
\pgfsetlinewidth{1.505625pt}%
\definecolor{currentstroke}{rgb}{0.000000,0.750000,0.750000}%
\pgfsetstrokecolor{currentstroke}%
\pgfsetstrokeopacity{0.750000}%
\pgfsetdash{}{0pt}%
\pgfpathmoveto{\pgfqpoint{10.637370in}{16.406565in}}%
\pgfpathlineto{\pgfqpoint{11.026259in}{16.406565in}}%
\pgfusepath{stroke}%
\end{pgfscope}%
\begin{pgfscope}%
\definecolor{textcolor}{rgb}{0.501961,0.501961,0.501961}%
\pgfsetstrokecolor{textcolor}%
\pgfsetfillcolor{textcolor}%
\pgftext[x=11.181814in,y=16.338510in,left,base]{\color{textcolor}\rmfamily\fontsize{14.000000}{16.800000}\selectfont Image force}%
\end{pgfscope}%
\begin{pgfscope}%
\pgfsetrectcap%
\pgfsetroundjoin%
\pgfsetlinewidth{1.505625pt}%
\definecolor{currentstroke}{rgb}{1.000000,0.000000,0.000000}%
\pgfsetstrokecolor{currentstroke}%
\pgfsetstrokeopacity{0.750000}%
\pgfsetdash{}{0pt}%
\pgfpathmoveto{\pgfqpoint{10.637370in}{16.118412in}}%
\pgfpathlineto{\pgfqpoint{11.026259in}{16.118412in}}%
\pgfusepath{stroke}%
\end{pgfscope}%
\begin{pgfscope}%
\definecolor{textcolor}{rgb}{0.501961,0.501961,0.501961}%
\pgfsetstrokecolor{textcolor}%
\pgfsetfillcolor{textcolor}%
\pgftext[x=11.181814in,y=16.050356in,left,base]{\color{textcolor}\rmfamily\fontsize{14.000000}{16.800000}\selectfont Coulomb force}%
\end{pgfscope}%
\begin{pgfscope}%
\pgfsetrectcap%
\pgfsetroundjoin%
\pgfsetlinewidth{1.505625pt}%
\definecolor{currentstroke}{rgb}{0.000000,0.000000,1.000000}%
\pgfsetstrokecolor{currentstroke}%
\pgfsetstrokeopacity{0.750000}%
\pgfsetdash{}{0pt}%
\pgfpathmoveto{\pgfqpoint{10.637370in}{15.833011in}}%
\pgfpathlineto{\pgfqpoint{11.026259in}{15.833011in}}%
\pgfusepath{stroke}%
\end{pgfscope}%
\begin{pgfscope}%
\definecolor{textcolor}{rgb}{0.501961,0.501961,0.501961}%
\pgfsetstrokecolor{textcolor}%
\pgfsetfillcolor{textcolor}%
\pgftext[x=11.181814in,y=15.764956in,left,base]{\color{textcolor}\rmfamily\fontsize{14.000000}{16.800000}\selectfont Drag force}%
\end{pgfscope}%
\begin{pgfscope}%
\pgfsetrectcap%
\pgfsetroundjoin%
\pgfsetlinewidth{1.505625pt}%
\definecolor{currentstroke}{rgb}{0.000000,0.750000,0.750000}%
\pgfsetstrokecolor{currentstroke}%
\pgfsetstrokeopacity{0.750000}%
\pgfsetdash{}{0pt}%
\pgfpathmoveto{\pgfqpoint{10.637370in}{15.544858in}}%
\pgfpathlineto{\pgfqpoint{11.026259in}{15.544858in}}%
\pgfusepath{stroke}%
\end{pgfscope}%
\begin{pgfscope}%
\definecolor{textcolor}{rgb}{0.501961,0.501961,0.501961}%
\pgfsetstrokecolor{textcolor}%
\pgfsetfillcolor{textcolor}%
\pgftext[x=11.181814in,y=15.476802in,left,base]{\color{textcolor}\rmfamily\fontsize{14.000000}{16.800000}\selectfont Image force}%
\end{pgfscope}%
\begin{pgfscope}%
\pgfsetrectcap%
\pgfsetroundjoin%
\pgfsetlinewidth{1.505625pt}%
\definecolor{currentstroke}{rgb}{1.000000,0.000000,0.000000}%
\pgfsetstrokecolor{currentstroke}%
\pgfsetstrokeopacity{0.750000}%
\pgfsetdash{}{0pt}%
\pgfpathmoveto{\pgfqpoint{10.637370in}{15.256704in}}%
\pgfpathlineto{\pgfqpoint{11.026259in}{15.256704in}}%
\pgfusepath{stroke}%
\end{pgfscope}%
\begin{pgfscope}%
\definecolor{textcolor}{rgb}{0.501961,0.501961,0.501961}%
\pgfsetstrokecolor{textcolor}%
\pgfsetfillcolor{textcolor}%
\pgftext[x=11.181814in,y=15.188648in,left,base]{\color{textcolor}\rmfamily\fontsize{14.000000}{16.800000}\selectfont Coulomb force}%
\end{pgfscope}%
\begin{pgfscope}%
\pgfsetrectcap%
\pgfsetroundjoin%
\pgfsetlinewidth{1.505625pt}%
\definecolor{currentstroke}{rgb}{0.000000,0.000000,1.000000}%
\pgfsetstrokecolor{currentstroke}%
\pgfsetstrokeopacity{0.750000}%
\pgfsetdash{}{0pt}%
\pgfpathmoveto{\pgfqpoint{10.637370in}{14.971304in}}%
\pgfpathlineto{\pgfqpoint{11.026259in}{14.971304in}}%
\pgfusepath{stroke}%
\end{pgfscope}%
\begin{pgfscope}%
\definecolor{textcolor}{rgb}{0.501961,0.501961,0.501961}%
\pgfsetstrokecolor{textcolor}%
\pgfsetfillcolor{textcolor}%
\pgftext[x=11.181814in,y=14.903248in,left,base]{\color{textcolor}\rmfamily\fontsize{14.000000}{16.800000}\selectfont Drag force}%
\end{pgfscope}%
\begin{pgfscope}%
\pgfsetrectcap%
\pgfsetroundjoin%
\pgfsetlinewidth{1.505625pt}%
\definecolor{currentstroke}{rgb}{0.000000,0.750000,0.750000}%
\pgfsetstrokecolor{currentstroke}%
\pgfsetstrokeopacity{0.750000}%
\pgfsetdash{}{0pt}%
\pgfpathmoveto{\pgfqpoint{10.637370in}{14.683150in}}%
\pgfpathlineto{\pgfqpoint{11.026259in}{14.683150in}}%
\pgfusepath{stroke}%
\end{pgfscope}%
\begin{pgfscope}%
\definecolor{textcolor}{rgb}{0.501961,0.501961,0.501961}%
\pgfsetstrokecolor{textcolor}%
\pgfsetfillcolor{textcolor}%
\pgftext[x=11.181814in,y=14.615095in,left,base]{\color{textcolor}\rmfamily\fontsize{14.000000}{16.800000}\selectfont Image force}%
\end{pgfscope}%
\begin{pgfscope}%
\pgfsetrectcap%
\pgfsetroundjoin%
\pgfsetlinewidth{1.505625pt}%
\definecolor{currentstroke}{rgb}{1.000000,0.000000,0.000000}%
\pgfsetstrokecolor{currentstroke}%
\pgfsetstrokeopacity{0.750000}%
\pgfsetdash{}{0pt}%
\pgfpathmoveto{\pgfqpoint{10.637370in}{14.394997in}}%
\pgfpathlineto{\pgfqpoint{11.026259in}{14.394997in}}%
\pgfusepath{stroke}%
\end{pgfscope}%
\begin{pgfscope}%
\definecolor{textcolor}{rgb}{0.501961,0.501961,0.501961}%
\pgfsetstrokecolor{textcolor}%
\pgfsetfillcolor{textcolor}%
\pgftext[x=11.181814in,y=14.326941in,left,base]{\color{textcolor}\rmfamily\fontsize{14.000000}{16.800000}\selectfont Coulomb force}%
\end{pgfscope}%
\begin{pgfscope}%
\pgfsetrectcap%
\pgfsetroundjoin%
\pgfsetlinewidth{1.505625pt}%
\definecolor{currentstroke}{rgb}{0.000000,0.000000,1.000000}%
\pgfsetstrokecolor{currentstroke}%
\pgfsetstrokeopacity{0.750000}%
\pgfsetdash{}{0pt}%
\pgfpathmoveto{\pgfqpoint{10.637370in}{14.109596in}}%
\pgfpathlineto{\pgfqpoint{11.026259in}{14.109596in}}%
\pgfusepath{stroke}%
\end{pgfscope}%
\begin{pgfscope}%
\definecolor{textcolor}{rgb}{0.501961,0.501961,0.501961}%
\pgfsetstrokecolor{textcolor}%
\pgfsetfillcolor{textcolor}%
\pgftext[x=11.181814in,y=14.041541in,left,base]{\color{textcolor}\rmfamily\fontsize{14.000000}{16.800000}\selectfont Drag force}%
\end{pgfscope}%
\begin{pgfscope}%
\pgfsetrectcap%
\pgfsetroundjoin%
\pgfsetlinewidth{1.505625pt}%
\definecolor{currentstroke}{rgb}{0.000000,0.750000,0.750000}%
\pgfsetstrokecolor{currentstroke}%
\pgfsetstrokeopacity{0.750000}%
\pgfsetdash{}{0pt}%
\pgfpathmoveto{\pgfqpoint{10.637370in}{13.821443in}}%
\pgfpathlineto{\pgfqpoint{11.026259in}{13.821443in}}%
\pgfusepath{stroke}%
\end{pgfscope}%
\begin{pgfscope}%
\definecolor{textcolor}{rgb}{0.501961,0.501961,0.501961}%
\pgfsetstrokecolor{textcolor}%
\pgfsetfillcolor{textcolor}%
\pgftext[x=11.181814in,y=13.753387in,left,base]{\color{textcolor}\rmfamily\fontsize{14.000000}{16.800000}\selectfont Image force}%
\end{pgfscope}%
\begin{pgfscope}%
\pgfsetrectcap%
\pgfsetroundjoin%
\pgfsetlinewidth{1.505625pt}%
\definecolor{currentstroke}{rgb}{1.000000,0.000000,0.000000}%
\pgfsetstrokecolor{currentstroke}%
\pgfsetstrokeopacity{0.750000}%
\pgfsetdash{}{0pt}%
\pgfpathmoveto{\pgfqpoint{10.637370in}{13.533289in}}%
\pgfpathlineto{\pgfqpoint{11.026259in}{13.533289in}}%
\pgfusepath{stroke}%
\end{pgfscope}%
\begin{pgfscope}%
\definecolor{textcolor}{rgb}{0.501961,0.501961,0.501961}%
\pgfsetstrokecolor{textcolor}%
\pgfsetfillcolor{textcolor}%
\pgftext[x=11.181814in,y=13.465233in,left,base]{\color{textcolor}\rmfamily\fontsize{14.000000}{16.800000}\selectfont Coulomb force}%
\end{pgfscope}%
\begin{pgfscope}%
\pgfsetrectcap%
\pgfsetroundjoin%
\pgfsetlinewidth{1.505625pt}%
\definecolor{currentstroke}{rgb}{0.000000,0.000000,1.000000}%
\pgfsetstrokecolor{currentstroke}%
\pgfsetstrokeopacity{0.750000}%
\pgfsetdash{}{0pt}%
\pgfpathmoveto{\pgfqpoint{10.637370in}{13.247889in}}%
\pgfpathlineto{\pgfqpoint{11.026259in}{13.247889in}}%
\pgfusepath{stroke}%
\end{pgfscope}%
\begin{pgfscope}%
\definecolor{textcolor}{rgb}{0.501961,0.501961,0.501961}%
\pgfsetstrokecolor{textcolor}%
\pgfsetfillcolor{textcolor}%
\pgftext[x=11.181814in,y=13.179833in,left,base]{\color{textcolor}\rmfamily\fontsize{14.000000}{16.800000}\selectfont Drag force}%
\end{pgfscope}%
\begin{pgfscope}%
\pgfsetrectcap%
\pgfsetroundjoin%
\pgfsetlinewidth{1.505625pt}%
\definecolor{currentstroke}{rgb}{0.000000,0.750000,0.750000}%
\pgfsetstrokecolor{currentstroke}%
\pgfsetstrokeopacity{0.750000}%
\pgfsetdash{}{0pt}%
\pgfpathmoveto{\pgfqpoint{10.637370in}{12.959735in}}%
\pgfpathlineto{\pgfqpoint{11.026259in}{12.959735in}}%
\pgfusepath{stroke}%
\end{pgfscope}%
\begin{pgfscope}%
\definecolor{textcolor}{rgb}{0.501961,0.501961,0.501961}%
\pgfsetstrokecolor{textcolor}%
\pgfsetfillcolor{textcolor}%
\pgftext[x=11.181814in,y=12.891680in,left,base]{\color{textcolor}\rmfamily\fontsize{14.000000}{16.800000}\selectfont Image force}%
\end{pgfscope}%
\begin{pgfscope}%
\pgfsetrectcap%
\pgfsetroundjoin%
\pgfsetlinewidth{1.505625pt}%
\definecolor{currentstroke}{rgb}{1.000000,0.000000,0.000000}%
\pgfsetstrokecolor{currentstroke}%
\pgfsetstrokeopacity{0.750000}%
\pgfsetdash{}{0pt}%
\pgfpathmoveto{\pgfqpoint{10.637370in}{12.671582in}}%
\pgfpathlineto{\pgfqpoint{11.026259in}{12.671582in}}%
\pgfusepath{stroke}%
\end{pgfscope}%
\begin{pgfscope}%
\definecolor{textcolor}{rgb}{0.501961,0.501961,0.501961}%
\pgfsetstrokecolor{textcolor}%
\pgfsetfillcolor{textcolor}%
\pgftext[x=11.181814in,y=12.603526in,left,base]{\color{textcolor}\rmfamily\fontsize{14.000000}{16.800000}\selectfont Coulomb force}%
\end{pgfscope}%
\begin{pgfscope}%
\pgfsetrectcap%
\pgfsetroundjoin%
\pgfsetlinewidth{1.505625pt}%
\definecolor{currentstroke}{rgb}{0.000000,0.000000,1.000000}%
\pgfsetstrokecolor{currentstroke}%
\pgfsetstrokeopacity{0.750000}%
\pgfsetdash{}{0pt}%
\pgfpathmoveto{\pgfqpoint{10.637370in}{12.386181in}}%
\pgfpathlineto{\pgfqpoint{11.026259in}{12.386181in}}%
\pgfusepath{stroke}%
\end{pgfscope}%
\begin{pgfscope}%
\definecolor{textcolor}{rgb}{0.501961,0.501961,0.501961}%
\pgfsetstrokecolor{textcolor}%
\pgfsetfillcolor{textcolor}%
\pgftext[x=11.181814in,y=12.318126in,left,base]{\color{textcolor}\rmfamily\fontsize{14.000000}{16.800000}\selectfont Drag force}%
\end{pgfscope}%
\begin{pgfscope}%
\pgfsetrectcap%
\pgfsetroundjoin%
\pgfsetlinewidth{1.505625pt}%
\definecolor{currentstroke}{rgb}{0.000000,0.750000,0.750000}%
\pgfsetstrokecolor{currentstroke}%
\pgfsetstrokeopacity{0.750000}%
\pgfsetdash{}{0pt}%
\pgfpathmoveto{\pgfqpoint{10.637370in}{12.098028in}}%
\pgfpathlineto{\pgfqpoint{11.026259in}{12.098028in}}%
\pgfusepath{stroke}%
\end{pgfscope}%
\begin{pgfscope}%
\definecolor{textcolor}{rgb}{0.501961,0.501961,0.501961}%
\pgfsetstrokecolor{textcolor}%
\pgfsetfillcolor{textcolor}%
\pgftext[x=11.181814in,y=12.029972in,left,base]{\color{textcolor}\rmfamily\fontsize{14.000000}{16.800000}\selectfont Image force}%
\end{pgfscope}%
\begin{pgfscope}%
\pgfsetrectcap%
\pgfsetroundjoin%
\pgfsetlinewidth{1.505625pt}%
\definecolor{currentstroke}{rgb}{1.000000,0.000000,0.000000}%
\pgfsetstrokecolor{currentstroke}%
\pgfsetstrokeopacity{0.750000}%
\pgfsetdash{}{0pt}%
\pgfpathmoveto{\pgfqpoint{10.637370in}{11.809874in}}%
\pgfpathlineto{\pgfqpoint{11.026259in}{11.809874in}}%
\pgfusepath{stroke}%
\end{pgfscope}%
\begin{pgfscope}%
\definecolor{textcolor}{rgb}{0.501961,0.501961,0.501961}%
\pgfsetstrokecolor{textcolor}%
\pgfsetfillcolor{textcolor}%
\pgftext[x=11.181814in,y=11.741818in,left,base]{\color{textcolor}\rmfamily\fontsize{14.000000}{16.800000}\selectfont Coulomb force}%
\end{pgfscope}%
\begin{pgfscope}%
\pgfsetrectcap%
\pgfsetroundjoin%
\pgfsetlinewidth{1.505625pt}%
\definecolor{currentstroke}{rgb}{0.000000,0.000000,1.000000}%
\pgfsetstrokecolor{currentstroke}%
\pgfsetstrokeopacity{0.750000}%
\pgfsetdash{}{0pt}%
\pgfpathmoveto{\pgfqpoint{10.637370in}{11.524474in}}%
\pgfpathlineto{\pgfqpoint{11.026259in}{11.524474in}}%
\pgfusepath{stroke}%
\end{pgfscope}%
\begin{pgfscope}%
\definecolor{textcolor}{rgb}{0.501961,0.501961,0.501961}%
\pgfsetstrokecolor{textcolor}%
\pgfsetfillcolor{textcolor}%
\pgftext[x=11.181814in,y=11.456418in,left,base]{\color{textcolor}\rmfamily\fontsize{14.000000}{16.800000}\selectfont Drag force}%
\end{pgfscope}%
\begin{pgfscope}%
\pgfsetrectcap%
\pgfsetroundjoin%
\pgfsetlinewidth{1.505625pt}%
\definecolor{currentstroke}{rgb}{0.000000,0.750000,0.750000}%
\pgfsetstrokecolor{currentstroke}%
\pgfsetstrokeopacity{0.750000}%
\pgfsetdash{}{0pt}%
\pgfpathmoveto{\pgfqpoint{10.637370in}{11.236320in}}%
\pgfpathlineto{\pgfqpoint{11.026259in}{11.236320in}}%
\pgfusepath{stroke}%
\end{pgfscope}%
\begin{pgfscope}%
\definecolor{textcolor}{rgb}{0.501961,0.501961,0.501961}%
\pgfsetstrokecolor{textcolor}%
\pgfsetfillcolor{textcolor}%
\pgftext[x=11.181814in,y=11.168265in,left,base]{\color{textcolor}\rmfamily\fontsize{14.000000}{16.800000}\selectfont Image force}%
\end{pgfscope}%
\begin{pgfscope}%
\pgfsetrectcap%
\pgfsetroundjoin%
\pgfsetlinewidth{1.505625pt}%
\definecolor{currentstroke}{rgb}{1.000000,0.000000,0.000000}%
\pgfsetstrokecolor{currentstroke}%
\pgfsetstrokeopacity{0.750000}%
\pgfsetdash{}{0pt}%
\pgfpathmoveto{\pgfqpoint{10.637370in}{10.948167in}}%
\pgfpathlineto{\pgfqpoint{11.026259in}{10.948167in}}%
\pgfusepath{stroke}%
\end{pgfscope}%
\begin{pgfscope}%
\definecolor{textcolor}{rgb}{0.501961,0.501961,0.501961}%
\pgfsetstrokecolor{textcolor}%
\pgfsetfillcolor{textcolor}%
\pgftext[x=11.181814in,y=10.880111in,left,base]{\color{textcolor}\rmfamily\fontsize{14.000000}{16.800000}\selectfont Coulomb force}%
\end{pgfscope}%
\begin{pgfscope}%
\pgfsetrectcap%
\pgfsetroundjoin%
\pgfsetlinewidth{1.505625pt}%
\definecolor{currentstroke}{rgb}{0.000000,0.000000,1.000000}%
\pgfsetstrokecolor{currentstroke}%
\pgfsetstrokeopacity{0.750000}%
\pgfsetdash{}{0pt}%
\pgfpathmoveto{\pgfqpoint{10.637370in}{10.662766in}}%
\pgfpathlineto{\pgfqpoint{11.026259in}{10.662766in}}%
\pgfusepath{stroke}%
\end{pgfscope}%
\begin{pgfscope}%
\definecolor{textcolor}{rgb}{0.501961,0.501961,0.501961}%
\pgfsetstrokecolor{textcolor}%
\pgfsetfillcolor{textcolor}%
\pgftext[x=11.181814in,y=10.594711in,left,base]{\color{textcolor}\rmfamily\fontsize{14.000000}{16.800000}\selectfont Drag force}%
\end{pgfscope}%
\begin{pgfscope}%
\pgfsetrectcap%
\pgfsetroundjoin%
\pgfsetlinewidth{1.505625pt}%
\definecolor{currentstroke}{rgb}{0.000000,0.750000,0.750000}%
\pgfsetstrokecolor{currentstroke}%
\pgfsetstrokeopacity{0.750000}%
\pgfsetdash{}{0pt}%
\pgfpathmoveto{\pgfqpoint{10.637370in}{10.374613in}}%
\pgfpathlineto{\pgfqpoint{11.026259in}{10.374613in}}%
\pgfusepath{stroke}%
\end{pgfscope}%
\begin{pgfscope}%
\definecolor{textcolor}{rgb}{0.501961,0.501961,0.501961}%
\pgfsetstrokecolor{textcolor}%
\pgfsetfillcolor{textcolor}%
\pgftext[x=11.181814in,y=10.306557in,left,base]{\color{textcolor}\rmfamily\fontsize{14.000000}{16.800000}\selectfont Image force}%
\end{pgfscope}%
\begin{pgfscope}%
\pgfsetrectcap%
\pgfsetroundjoin%
\pgfsetlinewidth{1.505625pt}%
\definecolor{currentstroke}{rgb}{1.000000,0.000000,0.000000}%
\pgfsetstrokecolor{currentstroke}%
\pgfsetstrokeopacity{0.750000}%
\pgfsetdash{}{0pt}%
\pgfpathmoveto{\pgfqpoint{10.637370in}{10.086459in}}%
\pgfpathlineto{\pgfqpoint{11.026259in}{10.086459in}}%
\pgfusepath{stroke}%
\end{pgfscope}%
\begin{pgfscope}%
\definecolor{textcolor}{rgb}{0.501961,0.501961,0.501961}%
\pgfsetstrokecolor{textcolor}%
\pgfsetfillcolor{textcolor}%
\pgftext[x=11.181814in,y=10.018403in,left,base]{\color{textcolor}\rmfamily\fontsize{14.000000}{16.800000}\selectfont Coulomb force}%
\end{pgfscope}%
\begin{pgfscope}%
\pgfsetrectcap%
\pgfsetroundjoin%
\pgfsetlinewidth{1.505625pt}%
\definecolor{currentstroke}{rgb}{0.000000,0.000000,1.000000}%
\pgfsetstrokecolor{currentstroke}%
\pgfsetstrokeopacity{0.750000}%
\pgfsetdash{}{0pt}%
\pgfpathmoveto{\pgfqpoint{10.637370in}{9.801059in}}%
\pgfpathlineto{\pgfqpoint{11.026259in}{9.801059in}}%
\pgfusepath{stroke}%
\end{pgfscope}%
\begin{pgfscope}%
\definecolor{textcolor}{rgb}{0.501961,0.501961,0.501961}%
\pgfsetstrokecolor{textcolor}%
\pgfsetfillcolor{textcolor}%
\pgftext[x=11.181814in,y=9.733003in,left,base]{\color{textcolor}\rmfamily\fontsize{14.000000}{16.800000}\selectfont Drag force}%
\end{pgfscope}%
\begin{pgfscope}%
\pgfsetrectcap%
\pgfsetroundjoin%
\pgfsetlinewidth{1.505625pt}%
\definecolor{currentstroke}{rgb}{0.000000,0.750000,0.750000}%
\pgfsetstrokecolor{currentstroke}%
\pgfsetstrokeopacity{0.750000}%
\pgfsetdash{}{0pt}%
\pgfpathmoveto{\pgfqpoint{10.637370in}{9.512905in}}%
\pgfpathlineto{\pgfqpoint{11.026259in}{9.512905in}}%
\pgfusepath{stroke}%
\end{pgfscope}%
\begin{pgfscope}%
\definecolor{textcolor}{rgb}{0.501961,0.501961,0.501961}%
\pgfsetstrokecolor{textcolor}%
\pgfsetfillcolor{textcolor}%
\pgftext[x=11.181814in,y=9.444850in,left,base]{\color{textcolor}\rmfamily\fontsize{14.000000}{16.800000}\selectfont Image force}%
\end{pgfscope}%
\begin{pgfscope}%
\pgfsetrectcap%
\pgfsetroundjoin%
\pgfsetlinewidth{1.505625pt}%
\definecolor{currentstroke}{rgb}{1.000000,0.000000,0.000000}%
\pgfsetstrokecolor{currentstroke}%
\pgfsetstrokeopacity{0.750000}%
\pgfsetdash{}{0pt}%
\pgfpathmoveto{\pgfqpoint{10.637370in}{9.224752in}}%
\pgfpathlineto{\pgfqpoint{11.026259in}{9.224752in}}%
\pgfusepath{stroke}%
\end{pgfscope}%
\begin{pgfscope}%
\definecolor{textcolor}{rgb}{0.501961,0.501961,0.501961}%
\pgfsetstrokecolor{textcolor}%
\pgfsetfillcolor{textcolor}%
\pgftext[x=11.181814in,y=9.156696in,left,base]{\color{textcolor}\rmfamily\fontsize{14.000000}{16.800000}\selectfont Coulomb force}%
\end{pgfscope}%
\begin{pgfscope}%
\pgfsetrectcap%
\pgfsetroundjoin%
\pgfsetlinewidth{1.505625pt}%
\definecolor{currentstroke}{rgb}{0.000000,0.000000,1.000000}%
\pgfsetstrokecolor{currentstroke}%
\pgfsetstrokeopacity{0.750000}%
\pgfsetdash{}{0pt}%
\pgfpathmoveto{\pgfqpoint{10.637370in}{8.939351in}}%
\pgfpathlineto{\pgfqpoint{11.026259in}{8.939351in}}%
\pgfusepath{stroke}%
\end{pgfscope}%
\begin{pgfscope}%
\definecolor{textcolor}{rgb}{0.501961,0.501961,0.501961}%
\pgfsetstrokecolor{textcolor}%
\pgfsetfillcolor{textcolor}%
\pgftext[x=11.181814in,y=8.871296in,left,base]{\color{textcolor}\rmfamily\fontsize{14.000000}{16.800000}\selectfont Drag force}%
\end{pgfscope}%
\begin{pgfscope}%
\pgfsetrectcap%
\pgfsetroundjoin%
\pgfsetlinewidth{1.505625pt}%
\definecolor{currentstroke}{rgb}{0.000000,0.750000,0.750000}%
\pgfsetstrokecolor{currentstroke}%
\pgfsetstrokeopacity{0.750000}%
\pgfsetdash{}{0pt}%
\pgfpathmoveto{\pgfqpoint{10.637370in}{8.651198in}}%
\pgfpathlineto{\pgfqpoint{11.026259in}{8.651198in}}%
\pgfusepath{stroke}%
\end{pgfscope}%
\begin{pgfscope}%
\definecolor{textcolor}{rgb}{0.501961,0.501961,0.501961}%
\pgfsetstrokecolor{textcolor}%
\pgfsetfillcolor{textcolor}%
\pgftext[x=11.181814in,y=8.583142in,left,base]{\color{textcolor}\rmfamily\fontsize{14.000000}{16.800000}\selectfont Image force}%
\end{pgfscope}%
\begin{pgfscope}%
\pgfsetrectcap%
\pgfsetroundjoin%
\pgfsetlinewidth{1.505625pt}%
\definecolor{currentstroke}{rgb}{1.000000,0.000000,0.000000}%
\pgfsetstrokecolor{currentstroke}%
\pgfsetstrokeopacity{0.750000}%
\pgfsetdash{}{0pt}%
\pgfpathmoveto{\pgfqpoint{10.637370in}{8.363044in}}%
\pgfpathlineto{\pgfqpoint{11.026259in}{8.363044in}}%
\pgfusepath{stroke}%
\end{pgfscope}%
\begin{pgfscope}%
\definecolor{textcolor}{rgb}{0.501961,0.501961,0.501961}%
\pgfsetstrokecolor{textcolor}%
\pgfsetfillcolor{textcolor}%
\pgftext[x=11.181814in,y=8.294988in,left,base]{\color{textcolor}\rmfamily\fontsize{14.000000}{16.800000}\selectfont Coulomb force}%
\end{pgfscope}%
\begin{pgfscope}%
\pgfsetrectcap%
\pgfsetroundjoin%
\pgfsetlinewidth{1.505625pt}%
\definecolor{currentstroke}{rgb}{0.000000,0.000000,1.000000}%
\pgfsetstrokecolor{currentstroke}%
\pgfsetstrokeopacity{0.750000}%
\pgfsetdash{}{0pt}%
\pgfpathmoveto{\pgfqpoint{10.637370in}{8.077644in}}%
\pgfpathlineto{\pgfqpoint{11.026259in}{8.077644in}}%
\pgfusepath{stroke}%
\end{pgfscope}%
\begin{pgfscope}%
\definecolor{textcolor}{rgb}{0.501961,0.501961,0.501961}%
\pgfsetstrokecolor{textcolor}%
\pgfsetfillcolor{textcolor}%
\pgftext[x=11.181814in,y=8.009588in,left,base]{\color{textcolor}\rmfamily\fontsize{14.000000}{16.800000}\selectfont Drag force}%
\end{pgfscope}%
\begin{pgfscope}%
\pgfsetrectcap%
\pgfsetroundjoin%
\pgfsetlinewidth{1.505625pt}%
\definecolor{currentstroke}{rgb}{0.000000,0.750000,0.750000}%
\pgfsetstrokecolor{currentstroke}%
\pgfsetstrokeopacity{0.750000}%
\pgfsetdash{}{0pt}%
\pgfpathmoveto{\pgfqpoint{10.637370in}{7.789490in}}%
\pgfpathlineto{\pgfqpoint{11.026259in}{7.789490in}}%
\pgfusepath{stroke}%
\end{pgfscope}%
\begin{pgfscope}%
\definecolor{textcolor}{rgb}{0.501961,0.501961,0.501961}%
\pgfsetstrokecolor{textcolor}%
\pgfsetfillcolor{textcolor}%
\pgftext[x=11.181814in,y=7.721435in,left,base]{\color{textcolor}\rmfamily\fontsize{14.000000}{16.800000}\selectfont Image force}%
\end{pgfscope}%
\begin{pgfscope}%
\pgfsetrectcap%
\pgfsetroundjoin%
\pgfsetlinewidth{1.505625pt}%
\definecolor{currentstroke}{rgb}{1.000000,0.000000,0.000000}%
\pgfsetstrokecolor{currentstroke}%
\pgfsetstrokeopacity{0.750000}%
\pgfsetdash{}{0pt}%
\pgfpathmoveto{\pgfqpoint{10.637370in}{7.501337in}}%
\pgfpathlineto{\pgfqpoint{11.026259in}{7.501337in}}%
\pgfusepath{stroke}%
\end{pgfscope}%
\begin{pgfscope}%
\definecolor{textcolor}{rgb}{0.501961,0.501961,0.501961}%
\pgfsetstrokecolor{textcolor}%
\pgfsetfillcolor{textcolor}%
\pgftext[x=11.181814in,y=7.433281in,left,base]{\color{textcolor}\rmfamily\fontsize{14.000000}{16.800000}\selectfont Coulomb force}%
\end{pgfscope}%
\begin{pgfscope}%
\pgfsetrectcap%
\pgfsetroundjoin%
\pgfsetlinewidth{1.505625pt}%
\definecolor{currentstroke}{rgb}{0.000000,0.000000,1.000000}%
\pgfsetstrokecolor{currentstroke}%
\pgfsetstrokeopacity{0.750000}%
\pgfsetdash{}{0pt}%
\pgfpathmoveto{\pgfqpoint{10.637370in}{7.215936in}}%
\pgfpathlineto{\pgfqpoint{11.026259in}{7.215936in}}%
\pgfusepath{stroke}%
\end{pgfscope}%
\begin{pgfscope}%
\definecolor{textcolor}{rgb}{0.501961,0.501961,0.501961}%
\pgfsetstrokecolor{textcolor}%
\pgfsetfillcolor{textcolor}%
\pgftext[x=11.181814in,y=7.147881in,left,base]{\color{textcolor}\rmfamily\fontsize{14.000000}{16.800000}\selectfont Drag force}%
\end{pgfscope}%
\begin{pgfscope}%
\pgfsetrectcap%
\pgfsetroundjoin%
\pgfsetlinewidth{1.505625pt}%
\definecolor{currentstroke}{rgb}{0.000000,0.750000,0.750000}%
\pgfsetstrokecolor{currentstroke}%
\pgfsetstrokeopacity{0.750000}%
\pgfsetdash{}{0pt}%
\pgfpathmoveto{\pgfqpoint{10.637370in}{6.927783in}}%
\pgfpathlineto{\pgfqpoint{11.026259in}{6.927783in}}%
\pgfusepath{stroke}%
\end{pgfscope}%
\begin{pgfscope}%
\definecolor{textcolor}{rgb}{0.501961,0.501961,0.501961}%
\pgfsetstrokecolor{textcolor}%
\pgfsetfillcolor{textcolor}%
\pgftext[x=11.181814in,y=6.859727in,left,base]{\color{textcolor}\rmfamily\fontsize{14.000000}{16.800000}\selectfont Image force}%
\end{pgfscope}%
\begin{pgfscope}%
\pgfsetrectcap%
\pgfsetroundjoin%
\pgfsetlinewidth{1.505625pt}%
\definecolor{currentstroke}{rgb}{1.000000,0.000000,0.000000}%
\pgfsetstrokecolor{currentstroke}%
\pgfsetstrokeopacity{0.750000}%
\pgfsetdash{}{0pt}%
\pgfpathmoveto{\pgfqpoint{10.637370in}{6.639629in}}%
\pgfpathlineto{\pgfqpoint{11.026259in}{6.639629in}}%
\pgfusepath{stroke}%
\end{pgfscope}%
\begin{pgfscope}%
\definecolor{textcolor}{rgb}{0.501961,0.501961,0.501961}%
\pgfsetstrokecolor{textcolor}%
\pgfsetfillcolor{textcolor}%
\pgftext[x=11.181814in,y=6.571573in,left,base]{\color{textcolor}\rmfamily\fontsize{14.000000}{16.800000}\selectfont Coulomb force}%
\end{pgfscope}%
\begin{pgfscope}%
\pgfsetrectcap%
\pgfsetroundjoin%
\pgfsetlinewidth{1.505625pt}%
\definecolor{currentstroke}{rgb}{0.000000,0.000000,1.000000}%
\pgfsetstrokecolor{currentstroke}%
\pgfsetstrokeopacity{0.750000}%
\pgfsetdash{}{0pt}%
\pgfpathmoveto{\pgfqpoint{10.637370in}{6.354229in}}%
\pgfpathlineto{\pgfqpoint{11.026259in}{6.354229in}}%
\pgfusepath{stroke}%
\end{pgfscope}%
\begin{pgfscope}%
\definecolor{textcolor}{rgb}{0.501961,0.501961,0.501961}%
\pgfsetstrokecolor{textcolor}%
\pgfsetfillcolor{textcolor}%
\pgftext[x=11.181814in,y=6.286173in,left,base]{\color{textcolor}\rmfamily\fontsize{14.000000}{16.800000}\selectfont Drag force}%
\end{pgfscope}%
\begin{pgfscope}%
\pgfsetrectcap%
\pgfsetroundjoin%
\pgfsetlinewidth{1.505625pt}%
\definecolor{currentstroke}{rgb}{0.000000,0.750000,0.750000}%
\pgfsetstrokecolor{currentstroke}%
\pgfsetstrokeopacity{0.750000}%
\pgfsetdash{}{0pt}%
\pgfpathmoveto{\pgfqpoint{10.637370in}{6.066075in}}%
\pgfpathlineto{\pgfqpoint{11.026259in}{6.066075in}}%
\pgfusepath{stroke}%
\end{pgfscope}%
\begin{pgfscope}%
\definecolor{textcolor}{rgb}{0.501961,0.501961,0.501961}%
\pgfsetstrokecolor{textcolor}%
\pgfsetfillcolor{textcolor}%
\pgftext[x=11.181814in,y=5.998020in,left,base]{\color{textcolor}\rmfamily\fontsize{14.000000}{16.800000}\selectfont Image force}%
\end{pgfscope}%
\begin{pgfscope}%
\pgfsetrectcap%
\pgfsetroundjoin%
\pgfsetlinewidth{1.505625pt}%
\definecolor{currentstroke}{rgb}{1.000000,0.000000,0.000000}%
\pgfsetstrokecolor{currentstroke}%
\pgfsetstrokeopacity{0.750000}%
\pgfsetdash{}{0pt}%
\pgfpathmoveto{\pgfqpoint{10.637370in}{5.777922in}}%
\pgfpathlineto{\pgfqpoint{11.026259in}{5.777922in}}%
\pgfusepath{stroke}%
\end{pgfscope}%
\begin{pgfscope}%
\definecolor{textcolor}{rgb}{0.501961,0.501961,0.501961}%
\pgfsetstrokecolor{textcolor}%
\pgfsetfillcolor{textcolor}%
\pgftext[x=11.181814in,y=5.709866in,left,base]{\color{textcolor}\rmfamily\fontsize{14.000000}{16.800000}\selectfont Coulomb force}%
\end{pgfscope}%
\begin{pgfscope}%
\pgfsetrectcap%
\pgfsetroundjoin%
\pgfsetlinewidth{1.505625pt}%
\definecolor{currentstroke}{rgb}{0.000000,0.000000,1.000000}%
\pgfsetstrokecolor{currentstroke}%
\pgfsetstrokeopacity{0.750000}%
\pgfsetdash{}{0pt}%
\pgfpathmoveto{\pgfqpoint{10.637370in}{5.492521in}}%
\pgfpathlineto{\pgfqpoint{11.026259in}{5.492521in}}%
\pgfusepath{stroke}%
\end{pgfscope}%
\begin{pgfscope}%
\definecolor{textcolor}{rgb}{0.501961,0.501961,0.501961}%
\pgfsetstrokecolor{textcolor}%
\pgfsetfillcolor{textcolor}%
\pgftext[x=11.181814in,y=5.424466in,left,base]{\color{textcolor}\rmfamily\fontsize{14.000000}{16.800000}\selectfont Drag force}%
\end{pgfscope}%
\begin{pgfscope}%
\pgfsetrectcap%
\pgfsetroundjoin%
\pgfsetlinewidth{1.505625pt}%
\definecolor{currentstroke}{rgb}{0.000000,0.750000,0.750000}%
\pgfsetstrokecolor{currentstroke}%
\pgfsetstrokeopacity{0.750000}%
\pgfsetdash{}{0pt}%
\pgfpathmoveto{\pgfqpoint{10.637370in}{5.204368in}}%
\pgfpathlineto{\pgfqpoint{11.026259in}{5.204368in}}%
\pgfusepath{stroke}%
\end{pgfscope}%
\begin{pgfscope}%
\definecolor{textcolor}{rgb}{0.501961,0.501961,0.501961}%
\pgfsetstrokecolor{textcolor}%
\pgfsetfillcolor{textcolor}%
\pgftext[x=11.181814in,y=5.136312in,left,base]{\color{textcolor}\rmfamily\fontsize{14.000000}{16.800000}\selectfont Image force}%
\end{pgfscope}%
\begin{pgfscope}%
\pgfsetrectcap%
\pgfsetroundjoin%
\pgfsetlinewidth{1.505625pt}%
\definecolor{currentstroke}{rgb}{1.000000,0.000000,0.000000}%
\pgfsetstrokecolor{currentstroke}%
\pgfsetstrokeopacity{0.750000}%
\pgfsetdash{}{0pt}%
\pgfpathmoveto{\pgfqpoint{10.637370in}{4.916214in}}%
\pgfpathlineto{\pgfqpoint{11.026259in}{4.916214in}}%
\pgfusepath{stroke}%
\end{pgfscope}%
\begin{pgfscope}%
\definecolor{textcolor}{rgb}{0.501961,0.501961,0.501961}%
\pgfsetstrokecolor{textcolor}%
\pgfsetfillcolor{textcolor}%
\pgftext[x=11.181814in,y=4.848158in,left,base]{\color{textcolor}\rmfamily\fontsize{14.000000}{16.800000}\selectfont Coulomb force}%
\end{pgfscope}%
\begin{pgfscope}%
\pgfsetrectcap%
\pgfsetroundjoin%
\pgfsetlinewidth{1.505625pt}%
\definecolor{currentstroke}{rgb}{0.000000,0.000000,1.000000}%
\pgfsetstrokecolor{currentstroke}%
\pgfsetstrokeopacity{0.750000}%
\pgfsetdash{}{0pt}%
\pgfpathmoveto{\pgfqpoint{10.637370in}{4.630814in}}%
\pgfpathlineto{\pgfqpoint{11.026259in}{4.630814in}}%
\pgfusepath{stroke}%
\end{pgfscope}%
\begin{pgfscope}%
\definecolor{textcolor}{rgb}{0.501961,0.501961,0.501961}%
\pgfsetstrokecolor{textcolor}%
\pgfsetfillcolor{textcolor}%
\pgftext[x=11.181814in,y=4.562758in,left,base]{\color{textcolor}\rmfamily\fontsize{14.000000}{16.800000}\selectfont Drag force}%
\end{pgfscope}%
\begin{pgfscope}%
\pgfsetrectcap%
\pgfsetroundjoin%
\pgfsetlinewidth{1.505625pt}%
\definecolor{currentstroke}{rgb}{0.000000,0.750000,0.750000}%
\pgfsetstrokecolor{currentstroke}%
\pgfsetstrokeopacity{0.750000}%
\pgfsetdash{}{0pt}%
\pgfpathmoveto{\pgfqpoint{10.637370in}{4.342660in}}%
\pgfpathlineto{\pgfqpoint{11.026259in}{4.342660in}}%
\pgfusepath{stroke}%
\end{pgfscope}%
\begin{pgfscope}%
\definecolor{textcolor}{rgb}{0.501961,0.501961,0.501961}%
\pgfsetstrokecolor{textcolor}%
\pgfsetfillcolor{textcolor}%
\pgftext[x=11.181814in,y=4.274605in,left,base]{\color{textcolor}\rmfamily\fontsize{14.000000}{16.800000}\selectfont Image force}%
\end{pgfscope}%
\begin{pgfscope}%
\definecolor{textcolor}{rgb}{0.501961,0.501961,0.501961}%
\pgfsetstrokecolor{textcolor}%
\pgfsetfillcolor{textcolor}%
\pgftext[x=5.380000in,y=0.140446in,,base]{\color{textcolor}\rmfamily\fontsize{14.000000}{16.800000}\selectfont \(\displaystyle t\) (s)}%
\end{pgfscope}%
\begin{pgfscope}%
\definecolor{textcolor}{rgb}{0.501961,0.501961,0.501961}%
\pgfsetstrokecolor{textcolor}%
\pgfsetfillcolor{textcolor}%
\pgftext[x=0.247732in,y=4.069478in,left,base,rotate=90.000000]{\color{textcolor}\rmfamily\fontsize{14.000000}{16.800000}\selectfont Force (N)}%
\end{pgfscope}%
\end{pgfpicture}%
\makeatother%
\endgroup%
}
    \caption{Simulated forces acting on the droplet. Experiments are shown by order of increasing apoapse.\label{fig:forces}}
\end{figure}

In the non-dimensional trajectories with short-time scaling shown in Figure \ref{fig:series_s_ds}, we see that the trajectory apoapses are consistently $\mathcal{O}(1)$, but most trajectories overshoot their characteristic time scale (which predicts returns at $\bar{t}  =2$ to the first order). We also observe that that $\mathbb{E}\mbox{u}$ is not typically a small number in this regime, imperiling our use of asymptotic estimates in this regime. We can perhaps gain some insight by comparing the asymptotic estimate for return times to the scaled experiemental return times. We see in Figure \ref{fig:times} that the long-time scaled non-dimensional time of first bounce in the experiment $t_b / t_c$, compares poorly to the asymptotic estimate for returns $t_f$ as $\mathbb{E}\mbox{u}_+$ becomes large. This is very much as we'd expect, but it is also unfortunate; for small $\mathbb{E}\mbox{u}$ the asymptotic length scale could be used to improve the characteristic length scale by $t_a = t_c t_f$. We also notice a two-tailed effect in the data; for the small $\mathbb{E}\mbox{u}$ droplets, the long-time scaling distorts the return times considerably.   
\begin{figure}[htb]
    \centering
    %% Creator: Matplotlib, PGF backend
%%
%% To include the figure in your LaTeX document, write
%%   \input{<filename>.pgf}
%%
%% Make sure the required packages are loaded in your preamble
%%   \usepackage{pgf}
%%
%% Figures using additional raster images can only be included by \input if
%% they are in the same directory as the main LaTeX file. For loading figures
%% from other directories you can use the `import` package
%%   \usepackage{import}
%% and then include the figures with
%%   \import{<path to file>}{<filename>.pgf}
%%
%% Matplotlib used the following preamble
%%   \usepackage{fontspec}
%%   \setmainfont{DejaVu Serif}
%%   \setsansfont{DejaVu Sans}
%%   \setmonofont{DejaVu Sans Mono}
%%
\begingroup%
\makeatletter%
\begin{pgfpicture}%
\pgfpathrectangle{\pgfpointorigin}{\pgfqpoint{5.606387in}{3.837899in}}%
\pgfusepath{use as bounding box, clip}%
\begin{pgfscope}%
\pgfsetbuttcap%
\pgfsetmiterjoin%
\definecolor{currentfill}{rgb}{1.000000,1.000000,1.000000}%
\pgfsetfillcolor{currentfill}%
\pgfsetlinewidth{0.000000pt}%
\definecolor{currentstroke}{rgb}{1.000000,1.000000,1.000000}%
\pgfsetstrokecolor{currentstroke}%
\pgfsetdash{}{0pt}%
\pgfpathmoveto{\pgfqpoint{0.000000in}{0.000000in}}%
\pgfpathlineto{\pgfqpoint{5.606387in}{0.000000in}}%
\pgfpathlineto{\pgfqpoint{5.606387in}{3.837899in}}%
\pgfpathlineto{\pgfqpoint{0.000000in}{3.837899in}}%
\pgfpathclose%
\pgfusepath{fill}%
\end{pgfscope}%
\begin{pgfscope}%
\pgfsetbuttcap%
\pgfsetmiterjoin%
\definecolor{currentfill}{rgb}{1.000000,1.000000,1.000000}%
\pgfsetfillcolor{currentfill}%
\pgfsetlinewidth{0.000000pt}%
\definecolor{currentstroke}{rgb}{0.000000,0.000000,0.000000}%
\pgfsetstrokecolor{currentstroke}%
\pgfsetstrokeopacity{0.000000}%
\pgfsetdash{}{0pt}%
\pgfpathmoveto{\pgfqpoint{0.754713in}{0.682899in}}%
\pgfpathlineto{\pgfqpoint{4.474713in}{0.682899in}}%
\pgfpathlineto{\pgfqpoint{4.474713in}{3.702899in}}%
\pgfpathlineto{\pgfqpoint{0.754713in}{3.702899in}}%
\pgfpathclose%
\pgfusepath{fill}%
\end{pgfscope}%
\begin{pgfscope}%
\pgfsetbuttcap%
\pgfsetroundjoin%
\definecolor{currentfill}{rgb}{0.000000,0.000000,0.000000}%
\pgfsetfillcolor{currentfill}%
\pgfsetlinewidth{0.803000pt}%
\definecolor{currentstroke}{rgb}{0.000000,0.000000,0.000000}%
\pgfsetstrokecolor{currentstroke}%
\pgfsetdash{}{0pt}%
\pgfsys@defobject{currentmarker}{\pgfqpoint{0.000000in}{-0.048611in}}{\pgfqpoint{0.000000in}{0.000000in}}{%
\pgfpathmoveto{\pgfqpoint{0.000000in}{0.000000in}}%
\pgfpathlineto{\pgfqpoint{0.000000in}{-0.048611in}}%
\pgfusepath{stroke,fill}%
}%
\begin{pgfscope}%
\pgfsys@transformshift{0.754713in}{0.682899in}%
\pgfsys@useobject{currentmarker}{}%
\end{pgfscope}%
\end{pgfscope}%
\begin{pgfscope}%
\pgftext[x=0.754713in,y=0.585677in,,top]{\rmfamily\fontsize{16.000000}{19.200000}\selectfont \(\displaystyle 0\)}%
\end{pgfscope}%
\begin{pgfscope}%
\pgfsetbuttcap%
\pgfsetroundjoin%
\definecolor{currentfill}{rgb}{0.000000,0.000000,0.000000}%
\pgfsetfillcolor{currentfill}%
\pgfsetlinewidth{0.803000pt}%
\definecolor{currentstroke}{rgb}{0.000000,0.000000,0.000000}%
\pgfsetstrokecolor{currentstroke}%
\pgfsetdash{}{0pt}%
\pgfsys@defobject{currentmarker}{\pgfqpoint{0.000000in}{-0.048611in}}{\pgfqpoint{0.000000in}{0.000000in}}{%
\pgfpathmoveto{\pgfqpoint{0.000000in}{0.000000in}}%
\pgfpathlineto{\pgfqpoint{0.000000in}{-0.048611in}}%
\pgfusepath{stroke,fill}%
}%
\begin{pgfscope}%
\pgfsys@transformshift{1.817570in}{0.682899in}%
\pgfsys@useobject{currentmarker}{}%
\end{pgfscope}%
\end{pgfscope}%
\begin{pgfscope}%
\pgftext[x=1.817570in,y=0.585677in,,top]{\rmfamily\fontsize{16.000000}{19.200000}\selectfont \(\displaystyle 1\)}%
\end{pgfscope}%
\begin{pgfscope}%
\pgfsetbuttcap%
\pgfsetroundjoin%
\definecolor{currentfill}{rgb}{0.000000,0.000000,0.000000}%
\pgfsetfillcolor{currentfill}%
\pgfsetlinewidth{0.803000pt}%
\definecolor{currentstroke}{rgb}{0.000000,0.000000,0.000000}%
\pgfsetstrokecolor{currentstroke}%
\pgfsetdash{}{0pt}%
\pgfsys@defobject{currentmarker}{\pgfqpoint{0.000000in}{-0.048611in}}{\pgfqpoint{0.000000in}{0.000000in}}{%
\pgfpathmoveto{\pgfqpoint{0.000000in}{0.000000in}}%
\pgfpathlineto{\pgfqpoint{0.000000in}{-0.048611in}}%
\pgfusepath{stroke,fill}%
}%
\begin{pgfscope}%
\pgfsys@transformshift{2.880427in}{0.682899in}%
\pgfsys@useobject{currentmarker}{}%
\end{pgfscope}%
\end{pgfscope}%
\begin{pgfscope}%
\pgftext[x=2.880427in,y=0.585677in,,top]{\rmfamily\fontsize{16.000000}{19.200000}\selectfont \(\displaystyle 2\)}%
\end{pgfscope}%
\begin{pgfscope}%
\pgfsetbuttcap%
\pgfsetroundjoin%
\definecolor{currentfill}{rgb}{0.000000,0.000000,0.000000}%
\pgfsetfillcolor{currentfill}%
\pgfsetlinewidth{0.803000pt}%
\definecolor{currentstroke}{rgb}{0.000000,0.000000,0.000000}%
\pgfsetstrokecolor{currentstroke}%
\pgfsetdash{}{0pt}%
\pgfsys@defobject{currentmarker}{\pgfqpoint{0.000000in}{-0.048611in}}{\pgfqpoint{0.000000in}{0.000000in}}{%
\pgfpathmoveto{\pgfqpoint{0.000000in}{0.000000in}}%
\pgfpathlineto{\pgfqpoint{0.000000in}{-0.048611in}}%
\pgfusepath{stroke,fill}%
}%
\begin{pgfscope}%
\pgfsys@transformshift{3.943284in}{0.682899in}%
\pgfsys@useobject{currentmarker}{}%
\end{pgfscope}%
\end{pgfscope}%
\begin{pgfscope}%
\pgftext[x=3.943284in,y=0.585677in,,top]{\rmfamily\fontsize{16.000000}{19.200000}\selectfont \(\displaystyle 3\)}%
\end{pgfscope}%
\begin{pgfscope}%
\pgftext[x=2.614713in,y=0.315061in,,top]{\rmfamily\fontsize{16.000000}{19.200000}\selectfont \(\displaystyle t^*\)}%
\end{pgfscope}%
\begin{pgfscope}%
\pgfsetbuttcap%
\pgfsetroundjoin%
\definecolor{currentfill}{rgb}{0.000000,0.000000,0.000000}%
\pgfsetfillcolor{currentfill}%
\pgfsetlinewidth{0.803000pt}%
\definecolor{currentstroke}{rgb}{0.000000,0.000000,0.000000}%
\pgfsetstrokecolor{currentstroke}%
\pgfsetdash{}{0pt}%
\pgfsys@defobject{currentmarker}{\pgfqpoint{-0.048611in}{0.000000in}}{\pgfqpoint{0.000000in}{0.000000in}}{%
\pgfpathmoveto{\pgfqpoint{0.000000in}{0.000000in}}%
\pgfpathlineto{\pgfqpoint{-0.048611in}{0.000000in}}%
\pgfusepath{stroke,fill}%
}%
\begin{pgfscope}%
\pgfsys@transformshift{0.754713in}{0.933165in}%
\pgfsys@useobject{currentmarker}{}%
\end{pgfscope}%
\end{pgfscope}%
\begin{pgfscope}%
\pgftext[x=0.372077in,y=0.848746in,left,base]{\rmfamily\fontsize{16.000000}{19.200000}\selectfont \(\displaystyle 0.0\)}%
\end{pgfscope}%
\begin{pgfscope}%
\pgfsetbuttcap%
\pgfsetroundjoin%
\definecolor{currentfill}{rgb}{0.000000,0.000000,0.000000}%
\pgfsetfillcolor{currentfill}%
\pgfsetlinewidth{0.803000pt}%
\definecolor{currentstroke}{rgb}{0.000000,0.000000,0.000000}%
\pgfsetstrokecolor{currentstroke}%
\pgfsetdash{}{0pt}%
\pgfsys@defobject{currentmarker}{\pgfqpoint{-0.048611in}{0.000000in}}{\pgfqpoint{0.000000in}{0.000000in}}{%
\pgfpathmoveto{\pgfqpoint{0.000000in}{0.000000in}}%
\pgfpathlineto{\pgfqpoint{-0.048611in}{0.000000in}}%
\pgfusepath{stroke,fill}%
}%
\begin{pgfscope}%
\pgfsys@transformshift{0.754713in}{1.830828in}%
\pgfsys@useobject{currentmarker}{}%
\end{pgfscope}%
\end{pgfscope}%
\begin{pgfscope}%
\pgftext[x=0.372077in,y=1.746410in,left,base]{\rmfamily\fontsize{16.000000}{19.200000}\selectfont \(\displaystyle 0.5\)}%
\end{pgfscope}%
\begin{pgfscope}%
\pgfsetbuttcap%
\pgfsetroundjoin%
\definecolor{currentfill}{rgb}{0.000000,0.000000,0.000000}%
\pgfsetfillcolor{currentfill}%
\pgfsetlinewidth{0.803000pt}%
\definecolor{currentstroke}{rgb}{0.000000,0.000000,0.000000}%
\pgfsetstrokecolor{currentstroke}%
\pgfsetdash{}{0pt}%
\pgfsys@defobject{currentmarker}{\pgfqpoint{-0.048611in}{0.000000in}}{\pgfqpoint{0.000000in}{0.000000in}}{%
\pgfpathmoveto{\pgfqpoint{0.000000in}{0.000000in}}%
\pgfpathlineto{\pgfqpoint{-0.048611in}{0.000000in}}%
\pgfusepath{stroke,fill}%
}%
\begin{pgfscope}%
\pgfsys@transformshift{0.754713in}{2.728492in}%
\pgfsys@useobject{currentmarker}{}%
\end{pgfscope}%
\end{pgfscope}%
\begin{pgfscope}%
\pgftext[x=0.372077in,y=2.644073in,left,base]{\rmfamily\fontsize{16.000000}{19.200000}\selectfont \(\displaystyle 1.0\)}%
\end{pgfscope}%
\begin{pgfscope}%
\pgfsetbuttcap%
\pgfsetroundjoin%
\definecolor{currentfill}{rgb}{0.000000,0.000000,0.000000}%
\pgfsetfillcolor{currentfill}%
\pgfsetlinewidth{0.803000pt}%
\definecolor{currentstroke}{rgb}{0.000000,0.000000,0.000000}%
\pgfsetstrokecolor{currentstroke}%
\pgfsetdash{}{0pt}%
\pgfsys@defobject{currentmarker}{\pgfqpoint{-0.048611in}{0.000000in}}{\pgfqpoint{0.000000in}{0.000000in}}{%
\pgfpathmoveto{\pgfqpoint{0.000000in}{0.000000in}}%
\pgfpathlineto{\pgfqpoint{-0.048611in}{0.000000in}}%
\pgfusepath{stroke,fill}%
}%
\begin{pgfscope}%
\pgfsys@transformshift{0.754713in}{3.626155in}%
\pgfsys@useobject{currentmarker}{}%
\end{pgfscope}%
\end{pgfscope}%
\begin{pgfscope}%
\pgftext[x=0.372077in,y=3.541737in,left,base]{\rmfamily\fontsize{16.000000}{19.200000}\selectfont \(\displaystyle 1.5\)}%
\end{pgfscope}%
\begin{pgfscope}%
\pgftext[x=0.316521in,y=2.192899in,,bottom,rotate=90.000000]{\rmfamily\fontsize{16.000000}{19.200000}\selectfont \(\displaystyle y^*\)}%
\end{pgfscope}%
\begin{pgfscope}%
\pgfpathrectangle{\pgfqpoint{0.754713in}{0.682899in}}{\pgfqpoint{3.720000in}{3.020000in}}%
\pgfusepath{clip}%
\pgfsetrectcap%
\pgfsetroundjoin%
\pgfsetlinewidth{1.505625pt}%
\definecolor{currentstroke}{rgb}{0.993248,0.906157,0.143936}%
\pgfsetstrokecolor{currentstroke}%
\pgfsetdash{}{0pt}%
\pgfpathmoveto{\pgfqpoint{1.526605in}{2.540456in}}%
\pgfpathlineto{\pgfqpoint{1.596777in}{2.665308in}}%
\pgfpathlineto{\pgfqpoint{1.666949in}{2.817882in}}%
\pgfpathlineto{\pgfqpoint{1.737121in}{2.841919in}}%
\pgfpathlineto{\pgfqpoint{1.807293in}{2.914832in}}%
\pgfpathlineto{\pgfqpoint{1.877465in}{3.038647in}}%
\pgfpathlineto{\pgfqpoint{1.947637in}{3.082049in}}%
\pgfpathlineto{\pgfqpoint{2.017809in}{3.138944in}}%
\pgfpathlineto{\pgfqpoint{2.087981in}{3.186031in}}%
\pgfpathlineto{\pgfqpoint{2.158153in}{3.225182in}}%
\pgfpathlineto{\pgfqpoint{2.228325in}{3.238547in}}%
\pgfpathlineto{\pgfqpoint{2.298497in}{3.244670in}}%
\pgfpathlineto{\pgfqpoint{2.368669in}{3.251294in}}%
\pgfpathlineto{\pgfqpoint{2.438841in}{3.230319in}}%
\pgfpathlineto{\pgfqpoint{2.509013in}{3.220226in}}%
\pgfpathlineto{\pgfqpoint{2.579185in}{3.182532in}}%
\pgfpathlineto{\pgfqpoint{2.649357in}{3.145323in}}%
\pgfpathlineto{\pgfqpoint{2.719529in}{3.086988in}}%
\pgfpathlineto{\pgfqpoint{2.789702in}{3.029451in}}%
\pgfpathlineto{\pgfqpoint{2.859874in}{2.949412in}}%
\pgfpathlineto{\pgfqpoint{2.930046in}{2.865500in}}%
\pgfpathlineto{\pgfqpoint{3.000218in}{2.770906in}}%
\pgfpathlineto{\pgfqpoint{3.070390in}{2.658771in}}%
\pgfpathlineto{\pgfqpoint{3.140562in}{2.541389in}}%
\pgfpathlineto{\pgfqpoint{3.210734in}{2.401807in}}%
\pgfpathlineto{\pgfqpoint{3.280906in}{2.264913in}}%
\pgfpathlineto{\pgfqpoint{3.351078in}{2.098262in}}%
\pgfpathlineto{\pgfqpoint{3.421250in}{1.920170in}}%
\pgfpathlineto{\pgfqpoint{3.491422in}{1.707511in}}%
\pgfpathlineto{\pgfqpoint{3.561594in}{1.493756in}}%
\pgfpathlineto{\pgfqpoint{3.631766in}{1.222808in}}%
\pgfpathlineto{\pgfqpoint{3.701938in}{1.018307in}}%
\pgfpathlineto{\pgfqpoint{3.772110in}{0.853977in}}%
\pgfpathlineto{\pgfqpoint{3.842282in}{0.869925in}}%
\pgfpathlineto{\pgfqpoint{3.912454in}{1.015648in}}%
\pgfpathlineto{\pgfqpoint{3.982626in}{1.219280in}}%
\pgfpathlineto{\pgfqpoint{4.052798in}{1.416638in}}%
\pgfpathlineto{\pgfqpoint{4.122970in}{1.565620in}}%
\pgfpathlineto{\pgfqpoint{4.193142in}{1.681545in}}%
\pgfpathlineto{\pgfqpoint{4.263314in}{1.773229in}}%
\pgfpathlineto{\pgfqpoint{4.333486in}{1.827833in}}%
\pgfpathlineto{\pgfqpoint{4.403658in}{1.874022in}}%
\pgfpathlineto{\pgfqpoint{4.473830in}{1.873142in}}%
\pgfpathlineto{\pgfqpoint{4.484713in}{1.870017in}}%
\pgfusepath{stroke}%
\end{pgfscope}%
\begin{pgfscope}%
\pgfpathrectangle{\pgfqpoint{0.754713in}{0.682899in}}{\pgfqpoint{3.720000in}{3.020000in}}%
\pgfusepath{clip}%
\pgfsetrectcap%
\pgfsetroundjoin%
\pgfsetlinewidth{1.505625pt}%
\definecolor{currentstroke}{rgb}{0.699415,0.867117,0.175971}%
\pgfsetstrokecolor{currentstroke}%
\pgfsetdash{}{0pt}%
\pgfpathmoveto{\pgfqpoint{1.005507in}{1.122611in}}%
\pgfpathlineto{\pgfqpoint{1.068206in}{1.371153in}}%
\pgfpathlineto{\pgfqpoint{1.130904in}{1.583661in}}%
\pgfpathlineto{\pgfqpoint{1.193603in}{1.736381in}}%
\pgfpathlineto{\pgfqpoint{1.256301in}{1.864552in}}%
\pgfpathlineto{\pgfqpoint{1.319000in}{2.008757in}}%
\pgfpathlineto{\pgfqpoint{1.381699in}{2.143646in}}%
\pgfpathlineto{\pgfqpoint{1.444397in}{2.340164in}}%
\pgfpathlineto{\pgfqpoint{1.507096in}{2.473534in}}%
\pgfpathlineto{\pgfqpoint{1.569795in}{2.551234in}}%
\pgfpathlineto{\pgfqpoint{1.632493in}{2.622196in}}%
\pgfpathlineto{\pgfqpoint{1.695192in}{2.674623in}}%
\pgfpathlineto{\pgfqpoint{1.757890in}{2.782897in}}%
\pgfpathlineto{\pgfqpoint{1.820589in}{2.860214in}}%
\pgfpathlineto{\pgfqpoint{1.883288in}{2.930209in}}%
\pgfpathlineto{\pgfqpoint{1.945986in}{2.922035in}}%
\pgfpathlineto{\pgfqpoint{2.008685in}{2.933715in}}%
\pgfpathlineto{\pgfqpoint{2.071384in}{2.988627in}}%
\pgfpathlineto{\pgfqpoint{2.134082in}{3.034831in}}%
\pgfpathlineto{\pgfqpoint{2.196781in}{3.093865in}}%
\pgfpathlineto{\pgfqpoint{2.259479in}{3.080572in}}%
\pgfpathlineto{\pgfqpoint{2.322178in}{3.040494in}}%
\pgfpathlineto{\pgfqpoint{2.384877in}{3.031214in}}%
\pgfpathlineto{\pgfqpoint{2.447575in}{3.054173in}}%
\pgfpathlineto{\pgfqpoint{2.510274in}{3.067809in}}%
\pgfpathlineto{\pgfqpoint{2.572972in}{2.999990in}}%
\pgfpathlineto{\pgfqpoint{2.635671in}{2.937788in}}%
\pgfpathlineto{\pgfqpoint{2.698370in}{2.873438in}}%
\pgfpathlineto{\pgfqpoint{2.761068in}{2.828260in}}%
\pgfpathlineto{\pgfqpoint{2.823767in}{2.827357in}}%
\pgfpathlineto{\pgfqpoint{2.886466in}{2.759907in}}%
\pgfpathlineto{\pgfqpoint{2.949164in}{2.654501in}}%
\pgfpathlineto{\pgfqpoint{3.011863in}{2.527271in}}%
\pgfpathlineto{\pgfqpoint{3.074561in}{2.465584in}}%
\pgfpathlineto{\pgfqpoint{3.137260in}{2.389752in}}%
\pgfpathlineto{\pgfqpoint{3.199959in}{2.281062in}}%
\pgfpathlineto{\pgfqpoint{3.262657in}{2.135655in}}%
\pgfpathlineto{\pgfqpoint{3.325356in}{1.955252in}}%
\pgfpathlineto{\pgfqpoint{3.388055in}{1.781112in}}%
\pgfpathlineto{\pgfqpoint{3.450753in}{1.632073in}}%
\pgfpathlineto{\pgfqpoint{3.513452in}{1.442164in}}%
\pgfpathlineto{\pgfqpoint{3.576150in}{1.237340in}}%
\pgfpathlineto{\pgfqpoint{3.638849in}{1.063732in}}%
\pgfpathlineto{\pgfqpoint{3.701548in}{0.930760in}}%
\pgfpathlineto{\pgfqpoint{3.764246in}{0.842899in}}%
\pgfpathlineto{\pgfqpoint{3.826945in}{0.820172in}}%
\pgfpathlineto{\pgfqpoint{3.889643in}{0.899559in}}%
\pgfpathlineto{\pgfqpoint{3.952342in}{1.028284in}}%
\pgfpathlineto{\pgfqpoint{4.077739in}{1.358084in}}%
\pgfpathlineto{\pgfqpoint{4.140438in}{1.539273in}}%
\pgfpathlineto{\pgfqpoint{4.203137in}{1.677699in}}%
\pgfpathlineto{\pgfqpoint{4.265835in}{1.825798in}}%
\pgfpathlineto{\pgfqpoint{4.328534in}{1.978689in}}%
\pgfpathlineto{\pgfqpoint{4.391232in}{2.116514in}}%
\pgfpathlineto{\pgfqpoint{4.453931in}{2.243527in}}%
\pgfpathlineto{\pgfqpoint{4.484713in}{2.280774in}}%
\pgfpathlineto{\pgfqpoint{4.484713in}{2.280774in}}%
\pgfusepath{stroke}%
\end{pgfscope}%
\begin{pgfscope}%
\pgfpathrectangle{\pgfqpoint{0.754713in}{0.682899in}}{\pgfqpoint{3.720000in}{3.020000in}}%
\pgfusepath{clip}%
\pgfsetrectcap%
\pgfsetroundjoin%
\pgfsetlinewidth{1.505625pt}%
\definecolor{currentstroke}{rgb}{0.636902,0.856542,0.216620}%
\pgfsetstrokecolor{currentstroke}%
\pgfsetdash{}{0pt}%
\pgfpathmoveto{\pgfqpoint{1.412821in}{2.742976in}}%
\pgfpathlineto{\pgfqpoint{1.485944in}{2.862986in}}%
\pgfpathlineto{\pgfqpoint{1.559068in}{2.963475in}}%
\pgfpathlineto{\pgfqpoint{1.632191in}{3.046579in}}%
\pgfpathlineto{\pgfqpoint{1.705314in}{3.165706in}}%
\pgfpathlineto{\pgfqpoint{1.778437in}{3.256081in}}%
\pgfpathlineto{\pgfqpoint{1.851560in}{3.301449in}}%
\pgfpathlineto{\pgfqpoint{1.924684in}{3.377384in}}%
\pgfpathlineto{\pgfqpoint{1.997807in}{3.477058in}}%
\pgfpathlineto{\pgfqpoint{2.070930in}{3.467665in}}%
\pgfpathlineto{\pgfqpoint{2.144053in}{3.505926in}}%
\pgfpathlineto{\pgfqpoint{2.217176in}{3.555161in}}%
\pgfpathlineto{\pgfqpoint{2.290299in}{3.545901in}}%
\pgfpathlineto{\pgfqpoint{2.363423in}{3.565626in}}%
\pgfpathlineto{\pgfqpoint{2.436546in}{3.565125in}}%
\pgfpathlineto{\pgfqpoint{2.509669in}{3.538410in}}%
\pgfpathlineto{\pgfqpoint{2.582792in}{3.521777in}}%
\pgfpathlineto{\pgfqpoint{2.655915in}{3.484502in}}%
\pgfpathlineto{\pgfqpoint{2.729039in}{3.436700in}}%
\pgfpathlineto{\pgfqpoint{2.802162in}{3.373611in}}%
\pgfpathlineto{\pgfqpoint{2.875285in}{3.298757in}}%
\pgfpathlineto{\pgfqpoint{2.948408in}{3.231401in}}%
\pgfpathlineto{\pgfqpoint{3.021531in}{3.133660in}}%
\pgfpathlineto{\pgfqpoint{3.094655in}{3.032529in}}%
\pgfpathlineto{\pgfqpoint{3.167778in}{2.926671in}}%
\pgfpathlineto{\pgfqpoint{3.240901in}{2.793674in}}%
\pgfpathlineto{\pgfqpoint{3.314024in}{2.653696in}}%
\pgfpathlineto{\pgfqpoint{3.387147in}{2.518312in}}%
\pgfpathlineto{\pgfqpoint{3.460271in}{2.346945in}}%
\pgfpathlineto{\pgfqpoint{3.606517in}{1.945130in}}%
\pgfpathlineto{\pgfqpoint{3.679640in}{1.669486in}}%
\pgfpathlineto{\pgfqpoint{3.752763in}{1.381765in}}%
\pgfpathlineto{\pgfqpoint{3.825886in}{1.106136in}}%
\pgfpathlineto{\pgfqpoint{3.899010in}{0.955622in}}%
\pgfpathlineto{\pgfqpoint{3.972133in}{1.013819in}}%
\pgfpathlineto{\pgfqpoint{4.045256in}{1.233426in}}%
\pgfpathlineto{\pgfqpoint{4.118379in}{1.485455in}}%
\pgfpathlineto{\pgfqpoint{4.191502in}{1.713727in}}%
\pgfpathlineto{\pgfqpoint{4.264626in}{1.869282in}}%
\pgfpathlineto{\pgfqpoint{4.337749in}{1.982193in}}%
\pgfpathlineto{\pgfqpoint{4.410872in}{2.113632in}}%
\pgfpathlineto{\pgfqpoint{4.484713in}{2.177766in}}%
\pgfpathlineto{\pgfqpoint{4.484713in}{2.177766in}}%
\pgfusepath{stroke}%
\end{pgfscope}%
\begin{pgfscope}%
\pgfpathrectangle{\pgfqpoint{0.754713in}{0.682899in}}{\pgfqpoint{3.720000in}{3.020000in}}%
\pgfusepath{clip}%
\pgfsetrectcap%
\pgfsetroundjoin%
\pgfsetlinewidth{1.505625pt}%
\definecolor{currentstroke}{rgb}{0.335885,0.777018,0.402049}%
\pgfsetstrokecolor{currentstroke}%
\pgfsetdash{}{0pt}%
\pgfpathmoveto{\pgfqpoint{1.289140in}{1.788140in}}%
\pgfpathlineto{\pgfqpoint{1.355944in}{1.917285in}}%
\pgfpathlineto{\pgfqpoint{1.422747in}{2.019686in}}%
\pgfpathlineto{\pgfqpoint{1.489550in}{2.086154in}}%
\pgfpathlineto{\pgfqpoint{1.556354in}{2.179740in}}%
\pgfpathlineto{\pgfqpoint{1.623157in}{2.270406in}}%
\pgfpathlineto{\pgfqpoint{1.689961in}{2.313024in}}%
\pgfpathlineto{\pgfqpoint{1.756764in}{2.385985in}}%
\pgfpathlineto{\pgfqpoint{1.823568in}{2.436603in}}%
\pgfpathlineto{\pgfqpoint{1.890371in}{2.459970in}}%
\pgfpathlineto{\pgfqpoint{1.957175in}{2.514232in}}%
\pgfpathlineto{\pgfqpoint{2.023978in}{2.531798in}}%
\pgfpathlineto{\pgfqpoint{2.090782in}{2.527677in}}%
\pgfpathlineto{\pgfqpoint{2.157585in}{2.566682in}}%
\pgfpathlineto{\pgfqpoint{2.224388in}{2.560349in}}%
\pgfpathlineto{\pgfqpoint{2.291192in}{2.537716in}}%
\pgfpathlineto{\pgfqpoint{2.357995in}{2.548779in}}%
\pgfpathlineto{\pgfqpoint{2.491602in}{2.478044in}}%
\pgfpathlineto{\pgfqpoint{2.558406in}{2.459545in}}%
\pgfpathlineto{\pgfqpoint{2.625209in}{2.401991in}}%
\pgfpathlineto{\pgfqpoint{2.692013in}{2.339925in}}%
\pgfpathlineto{\pgfqpoint{2.758816in}{2.298901in}}%
\pgfpathlineto{\pgfqpoint{2.825619in}{2.211102in}}%
\pgfpathlineto{\pgfqpoint{2.892423in}{2.137781in}}%
\pgfpathlineto{\pgfqpoint{2.959226in}{2.067113in}}%
\pgfpathlineto{\pgfqpoint{3.092833in}{1.846593in}}%
\pgfpathlineto{\pgfqpoint{3.159637in}{1.748945in}}%
\pgfpathlineto{\pgfqpoint{3.226440in}{1.604854in}}%
\pgfpathlineto{\pgfqpoint{3.293244in}{1.472373in}}%
\pgfpathlineto{\pgfqpoint{3.360047in}{1.314418in}}%
\pgfpathlineto{\pgfqpoint{3.426851in}{1.145090in}}%
\pgfpathlineto{\pgfqpoint{3.493654in}{1.028822in}}%
\pgfpathlineto{\pgfqpoint{3.560457in}{0.890692in}}%
\pgfpathlineto{\pgfqpoint{3.627261in}{0.891872in}}%
\pgfpathlineto{\pgfqpoint{3.694064in}{0.942776in}}%
\pgfpathlineto{\pgfqpoint{3.760868in}{1.070935in}}%
\pgfpathlineto{\pgfqpoint{3.827671in}{1.220553in}}%
\pgfpathlineto{\pgfqpoint{3.894475in}{1.350079in}}%
\pgfpathlineto{\pgfqpoint{3.961278in}{1.443507in}}%
\pgfpathlineto{\pgfqpoint{4.028082in}{1.547237in}}%
\pgfpathlineto{\pgfqpoint{4.094885in}{1.627096in}}%
\pgfpathlineto{\pgfqpoint{4.161688in}{1.645603in}}%
\pgfpathlineto{\pgfqpoint{4.228492in}{1.727829in}}%
\pgfpathlineto{\pgfqpoint{4.295295in}{1.766065in}}%
\pgfpathlineto{\pgfqpoint{4.362099in}{1.753077in}}%
\pgfpathlineto{\pgfqpoint{4.428902in}{1.789050in}}%
\pgfpathlineto{\pgfqpoint{4.484713in}{1.802050in}}%
\pgfpathlineto{\pgfqpoint{4.484713in}{1.802050in}}%
\pgfusepath{stroke}%
\end{pgfscope}%
\begin{pgfscope}%
\pgfpathrectangle{\pgfqpoint{0.754713in}{0.682899in}}{\pgfqpoint{3.720000in}{3.020000in}}%
\pgfusepath{clip}%
\pgfsetrectcap%
\pgfsetroundjoin%
\pgfsetlinewidth{1.505625pt}%
\definecolor{currentstroke}{rgb}{0.122606,0.585371,0.546557}%
\pgfsetstrokecolor{currentstroke}%
\pgfsetdash{}{0pt}%
\pgfpathmoveto{\pgfqpoint{1.210064in}{2.041114in}}%
\pgfpathlineto{\pgfqpoint{1.255599in}{2.126493in}}%
\pgfpathlineto{\pgfqpoint{1.301134in}{2.186878in}}%
\pgfpathlineto{\pgfqpoint{1.346669in}{2.247201in}}%
\pgfpathlineto{\pgfqpoint{1.392204in}{2.326849in}}%
\pgfpathlineto{\pgfqpoint{1.437739in}{2.380003in}}%
\pgfpathlineto{\pgfqpoint{1.483274in}{2.431756in}}%
\pgfpathlineto{\pgfqpoint{1.528810in}{2.500981in}}%
\pgfpathlineto{\pgfqpoint{1.574345in}{2.543991in}}%
\pgfpathlineto{\pgfqpoint{1.619880in}{2.592931in}}%
\pgfpathlineto{\pgfqpoint{1.665415in}{2.644533in}}%
\pgfpathlineto{\pgfqpoint{1.710950in}{2.676291in}}%
\pgfpathlineto{\pgfqpoint{1.756485in}{2.724918in}}%
\pgfpathlineto{\pgfqpoint{1.802020in}{2.757566in}}%
\pgfpathlineto{\pgfqpoint{1.847555in}{2.789444in}}%
\pgfpathlineto{\pgfqpoint{1.893090in}{2.825149in}}%
\pgfpathlineto{\pgfqpoint{1.938626in}{2.843228in}}%
\pgfpathlineto{\pgfqpoint{1.984161in}{2.872786in}}%
\pgfpathlineto{\pgfqpoint{2.029696in}{2.890440in}}%
\pgfpathlineto{\pgfqpoint{2.075231in}{2.904761in}}%
\pgfpathlineto{\pgfqpoint{2.120766in}{2.928425in}}%
\pgfpathlineto{\pgfqpoint{2.166301in}{2.931459in}}%
\pgfpathlineto{\pgfqpoint{2.211836in}{2.943441in}}%
\pgfpathlineto{\pgfqpoint{2.257371in}{2.949398in}}%
\pgfpathlineto{\pgfqpoint{2.302907in}{2.946453in}}%
\pgfpathlineto{\pgfqpoint{2.348442in}{2.954793in}}%
\pgfpathlineto{\pgfqpoint{2.393977in}{2.945397in}}%
\pgfpathlineto{\pgfqpoint{2.439512in}{2.940366in}}%
\pgfpathlineto{\pgfqpoint{2.485047in}{2.932110in}}%
\pgfpathlineto{\pgfqpoint{2.530582in}{2.914650in}}%
\pgfpathlineto{\pgfqpoint{2.576117in}{2.907126in}}%
\pgfpathlineto{\pgfqpoint{2.621652in}{2.883169in}}%
\pgfpathlineto{\pgfqpoint{2.667187in}{2.860990in}}%
\pgfpathlineto{\pgfqpoint{2.712723in}{2.840137in}}%
\pgfpathlineto{\pgfqpoint{2.758258in}{2.806159in}}%
\pgfpathlineto{\pgfqpoint{2.803793in}{2.780714in}}%
\pgfpathlineto{\pgfqpoint{2.849328in}{2.743544in}}%
\pgfpathlineto{\pgfqpoint{2.894863in}{2.703260in}}%
\pgfpathlineto{\pgfqpoint{2.940398in}{2.667485in}}%
\pgfpathlineto{\pgfqpoint{2.985933in}{2.616876in}}%
\pgfpathlineto{\pgfqpoint{3.031468in}{2.575302in}}%
\pgfpathlineto{\pgfqpoint{3.077004in}{2.521629in}}%
\pgfpathlineto{\pgfqpoint{3.122539in}{2.464028in}}%
\pgfpathlineto{\pgfqpoint{3.168074in}{2.410573in}}%
\pgfpathlineto{\pgfqpoint{3.213609in}{2.344589in}}%
\pgfpathlineto{\pgfqpoint{3.259144in}{2.284588in}}%
\pgfpathlineto{\pgfqpoint{3.304679in}{2.214609in}}%
\pgfpathlineto{\pgfqpoint{3.350214in}{2.137290in}}%
\pgfpathlineto{\pgfqpoint{3.395749in}{2.064979in}}%
\pgfpathlineto{\pgfqpoint{3.441285in}{1.978347in}}%
\pgfpathlineto{\pgfqpoint{3.486820in}{1.895585in}}%
\pgfpathlineto{\pgfqpoint{3.532355in}{1.802502in}}%
\pgfpathlineto{\pgfqpoint{3.577890in}{1.694714in}}%
\pgfpathlineto{\pgfqpoint{3.623425in}{1.586464in}}%
\pgfpathlineto{\pgfqpoint{3.668960in}{1.471758in}}%
\pgfpathlineto{\pgfqpoint{3.714495in}{1.403724in}}%
\pgfpathlineto{\pgfqpoint{3.760030in}{1.333833in}}%
\pgfpathlineto{\pgfqpoint{3.805565in}{1.362562in}}%
\pgfpathlineto{\pgfqpoint{3.851101in}{1.415907in}}%
\pgfpathlineto{\pgfqpoint{3.896636in}{1.497923in}}%
\pgfpathlineto{\pgfqpoint{3.942171in}{1.596677in}}%
\pgfpathlineto{\pgfqpoint{3.987706in}{1.677273in}}%
\pgfpathlineto{\pgfqpoint{4.033241in}{1.736359in}}%
\pgfpathlineto{\pgfqpoint{4.078776in}{1.792979in}}%
\pgfpathlineto{\pgfqpoint{4.124311in}{1.828230in}}%
\pgfpathlineto{\pgfqpoint{4.169846in}{1.861568in}}%
\pgfpathlineto{\pgfqpoint{4.215382in}{1.897358in}}%
\pgfpathlineto{\pgfqpoint{4.260917in}{1.912466in}}%
\pgfpathlineto{\pgfqpoint{4.306452in}{1.929575in}}%
\pgfpathlineto{\pgfqpoint{4.351987in}{1.932783in}}%
\pgfpathlineto{\pgfqpoint{4.397522in}{1.926322in}}%
\pgfpathlineto{\pgfqpoint{4.443057in}{1.921874in}}%
\pgfpathlineto{\pgfqpoint{4.484713in}{1.902623in}}%
\pgfusepath{stroke}%
\end{pgfscope}%
\begin{pgfscope}%
\pgfpathrectangle{\pgfqpoint{0.754713in}{0.682899in}}{\pgfqpoint{3.720000in}{3.020000in}}%
\pgfusepath{clip}%
\pgfsetrectcap%
\pgfsetroundjoin%
\pgfsetlinewidth{1.505625pt}%
\definecolor{currentstroke}{rgb}{0.129933,0.559582,0.551864}%
\pgfsetstrokecolor{currentstroke}%
\pgfsetdash{}{0pt}%
\pgfpathmoveto{\pgfqpoint{1.156632in}{1.493148in}}%
\pgfpathlineto{\pgfqpoint{1.201289in}{1.539133in}}%
\pgfpathlineto{\pgfqpoint{1.245947in}{1.617447in}}%
\pgfpathlineto{\pgfqpoint{1.290605in}{1.698465in}}%
\pgfpathlineto{\pgfqpoint{1.335262in}{1.760049in}}%
\pgfpathlineto{\pgfqpoint{1.379920in}{1.802384in}}%
\pgfpathlineto{\pgfqpoint{1.424578in}{1.866192in}}%
\pgfpathlineto{\pgfqpoint{1.469235in}{1.935707in}}%
\pgfpathlineto{\pgfqpoint{1.513893in}{1.977048in}}%
\pgfpathlineto{\pgfqpoint{1.558551in}{2.015396in}}%
\pgfpathlineto{\pgfqpoint{1.603208in}{2.073756in}}%
\pgfpathlineto{\pgfqpoint{1.647866in}{2.122927in}}%
\pgfpathlineto{\pgfqpoint{1.692524in}{2.148056in}}%
\pgfpathlineto{\pgfqpoint{1.737181in}{2.178228in}}%
\pgfpathlineto{\pgfqpoint{1.781839in}{2.225857in}}%
\pgfpathlineto{\pgfqpoint{1.826497in}{2.258068in}}%
\pgfpathlineto{\pgfqpoint{1.871154in}{2.272246in}}%
\pgfpathlineto{\pgfqpoint{1.915812in}{2.303287in}}%
\pgfpathlineto{\pgfqpoint{1.960470in}{2.338091in}}%
\pgfpathlineto{\pgfqpoint{2.005127in}{2.352945in}}%
\pgfpathlineto{\pgfqpoint{2.049785in}{2.357238in}}%
\pgfpathlineto{\pgfqpoint{2.139100in}{2.403310in}}%
\pgfpathlineto{\pgfqpoint{2.183758in}{2.402491in}}%
\pgfpathlineto{\pgfqpoint{2.228416in}{2.406252in}}%
\pgfpathlineto{\pgfqpoint{2.273073in}{2.423190in}}%
\pgfpathlineto{\pgfqpoint{2.317731in}{2.429892in}}%
\pgfpathlineto{\pgfqpoint{2.362389in}{2.417316in}}%
\pgfpathlineto{\pgfqpoint{2.407047in}{2.413576in}}%
\pgfpathlineto{\pgfqpoint{2.451704in}{2.420560in}}%
\pgfpathlineto{\pgfqpoint{2.496362in}{2.410050in}}%
\pgfpathlineto{\pgfqpoint{2.541020in}{2.390118in}}%
\pgfpathlineto{\pgfqpoint{2.585677in}{2.383808in}}%
\pgfpathlineto{\pgfqpoint{2.630335in}{2.378628in}}%
\pgfpathlineto{\pgfqpoint{2.674993in}{2.351402in}}%
\pgfpathlineto{\pgfqpoint{2.719650in}{2.322495in}}%
\pgfpathlineto{\pgfqpoint{2.764308in}{2.307989in}}%
\pgfpathlineto{\pgfqpoint{2.808966in}{2.287267in}}%
\pgfpathlineto{\pgfqpoint{2.853623in}{2.247226in}}%
\pgfpathlineto{\pgfqpoint{2.898281in}{2.215885in}}%
\pgfpathlineto{\pgfqpoint{2.942939in}{2.192971in}}%
\pgfpathlineto{\pgfqpoint{2.987596in}{2.154496in}}%
\pgfpathlineto{\pgfqpoint{3.032254in}{2.102566in}}%
\pgfpathlineto{\pgfqpoint{3.076912in}{2.063105in}}%
\pgfpathlineto{\pgfqpoint{3.121569in}{2.025159in}}%
\pgfpathlineto{\pgfqpoint{3.210885in}{1.910846in}}%
\pgfpathlineto{\pgfqpoint{3.255542in}{1.863306in}}%
\pgfpathlineto{\pgfqpoint{3.300200in}{1.808722in}}%
\pgfpathlineto{\pgfqpoint{3.344858in}{1.735306in}}%
\pgfpathlineto{\pgfqpoint{3.389515in}{1.665257in}}%
\pgfpathlineto{\pgfqpoint{3.434173in}{1.602569in}}%
\pgfpathlineto{\pgfqpoint{3.478831in}{1.527547in}}%
\pgfpathlineto{\pgfqpoint{3.612804in}{1.264238in}}%
\pgfpathlineto{\pgfqpoint{3.657461in}{1.147990in}}%
\pgfpathlineto{\pgfqpoint{3.702119in}{1.048913in}}%
\pgfpathlineto{\pgfqpoint{3.746777in}{0.967262in}}%
\pgfpathlineto{\pgfqpoint{3.791434in}{0.894358in}}%
\pgfpathlineto{\pgfqpoint{3.836092in}{0.894884in}}%
\pgfpathlineto{\pgfqpoint{3.880750in}{0.921169in}}%
\pgfpathlineto{\pgfqpoint{3.925407in}{1.006837in}}%
\pgfpathlineto{\pgfqpoint{3.970065in}{1.081196in}}%
\pgfpathlineto{\pgfqpoint{4.014723in}{1.164549in}}%
\pgfpathlineto{\pgfqpoint{4.059381in}{1.265393in}}%
\pgfpathlineto{\pgfqpoint{4.104038in}{1.360280in}}%
\pgfpathlineto{\pgfqpoint{4.148696in}{1.436234in}}%
\pgfpathlineto{\pgfqpoint{4.193354in}{1.497367in}}%
\pgfpathlineto{\pgfqpoint{4.282669in}{1.604668in}}%
\pgfpathlineto{\pgfqpoint{4.327327in}{1.654161in}}%
\pgfpathlineto{\pgfqpoint{4.416642in}{1.732159in}}%
\pgfpathlineto{\pgfqpoint{4.461300in}{1.764655in}}%
\pgfpathlineto{\pgfqpoint{4.484713in}{1.780032in}}%
\pgfpathlineto{\pgfqpoint{4.484713in}{1.780032in}}%
\pgfusepath{stroke}%
\end{pgfscope}%
\begin{pgfscope}%
\pgfpathrectangle{\pgfqpoint{0.754713in}{0.682899in}}{\pgfqpoint{3.720000in}{3.020000in}}%
\pgfusepath{clip}%
\pgfsetrectcap%
\pgfsetroundjoin%
\pgfsetlinewidth{1.505625pt}%
\definecolor{currentstroke}{rgb}{0.192357,0.403199,0.555836}%
\pgfsetstrokecolor{currentstroke}%
\pgfsetdash{}{0pt}%
\pgfpathmoveto{\pgfqpoint{1.107813in}{1.491525in}}%
\pgfpathlineto{\pgfqpoint{1.151950in}{1.564134in}}%
\pgfpathlineto{\pgfqpoint{1.196088in}{1.620209in}}%
\pgfpathlineto{\pgfqpoint{1.240225in}{1.699304in}}%
\pgfpathlineto{\pgfqpoint{1.328500in}{1.821472in}}%
\pgfpathlineto{\pgfqpoint{1.372638in}{1.875380in}}%
\pgfpathlineto{\pgfqpoint{1.416775in}{1.938685in}}%
\pgfpathlineto{\pgfqpoint{1.460913in}{1.985363in}}%
\pgfpathlineto{\pgfqpoint{1.505050in}{2.041424in}}%
\pgfpathlineto{\pgfqpoint{1.549188in}{2.082394in}}%
\pgfpathlineto{\pgfqpoint{1.593325in}{2.131038in}}%
\pgfpathlineto{\pgfqpoint{1.637463in}{2.169230in}}%
\pgfpathlineto{\pgfqpoint{1.681600in}{2.210368in}}%
\pgfpathlineto{\pgfqpoint{1.725738in}{2.242567in}}%
\pgfpathlineto{\pgfqpoint{1.769875in}{2.280484in}}%
\pgfpathlineto{\pgfqpoint{1.814013in}{2.306282in}}%
\pgfpathlineto{\pgfqpoint{1.858150in}{2.339004in}}%
\pgfpathlineto{\pgfqpoint{1.902288in}{2.360698in}}%
\pgfpathlineto{\pgfqpoint{1.946425in}{2.388125in}}%
\pgfpathlineto{\pgfqpoint{1.990563in}{2.406323in}}%
\pgfpathlineto{\pgfqpoint{2.034700in}{2.428472in}}%
\pgfpathlineto{\pgfqpoint{2.078838in}{2.441495in}}%
\pgfpathlineto{\pgfqpoint{2.122975in}{2.458371in}}%
\pgfpathlineto{\pgfqpoint{2.167113in}{2.466987in}}%
\pgfpathlineto{\pgfqpoint{2.211251in}{2.479218in}}%
\pgfpathlineto{\pgfqpoint{2.255388in}{2.483119in}}%
\pgfpathlineto{\pgfqpoint{2.299526in}{2.490789in}}%
\pgfpathlineto{\pgfqpoint{2.343663in}{2.490365in}}%
\pgfpathlineto{\pgfqpoint{2.387801in}{2.491859in}}%
\pgfpathlineto{\pgfqpoint{2.431938in}{2.486728in}}%
\pgfpathlineto{\pgfqpoint{2.476076in}{2.484550in}}%
\pgfpathlineto{\pgfqpoint{2.520213in}{2.474764in}}%
\pgfpathlineto{\pgfqpoint{2.564351in}{2.467465in}}%
\pgfpathlineto{\pgfqpoint{2.608488in}{2.453227in}}%
\pgfpathlineto{\pgfqpoint{2.652626in}{2.441374in}}%
\pgfpathlineto{\pgfqpoint{2.696763in}{2.422071in}}%
\pgfpathlineto{\pgfqpoint{2.740901in}{2.404905in}}%
\pgfpathlineto{\pgfqpoint{2.785038in}{2.381198in}}%
\pgfpathlineto{\pgfqpoint{2.829176in}{2.359499in}}%
\pgfpathlineto{\pgfqpoint{2.873313in}{2.330524in}}%
\pgfpathlineto{\pgfqpoint{2.917451in}{2.303663in}}%
\pgfpathlineto{\pgfqpoint{3.005726in}{2.238625in}}%
\pgfpathlineto{\pgfqpoint{3.094001in}{2.162599in}}%
\pgfpathlineto{\pgfqpoint{3.138138in}{2.120650in}}%
\pgfpathlineto{\pgfqpoint{3.182276in}{2.076737in}}%
\pgfpathlineto{\pgfqpoint{3.226413in}{2.029277in}}%
\pgfpathlineto{\pgfqpoint{3.270551in}{1.980048in}}%
\pgfpathlineto{\pgfqpoint{3.314688in}{1.927228in}}%
\pgfpathlineto{\pgfqpoint{3.358826in}{1.871820in}}%
\pgfpathlineto{\pgfqpoint{3.447101in}{1.750963in}}%
\pgfpathlineto{\pgfqpoint{3.491238in}{1.686212in}}%
\pgfpathlineto{\pgfqpoint{3.535376in}{1.618502in}}%
\pgfpathlineto{\pgfqpoint{3.579513in}{1.546449in}}%
\pgfpathlineto{\pgfqpoint{3.623651in}{1.471732in}}%
\pgfpathlineto{\pgfqpoint{3.667789in}{1.392300in}}%
\pgfpathlineto{\pgfqpoint{3.711926in}{1.306109in}}%
\pgfpathlineto{\pgfqpoint{3.756064in}{1.209419in}}%
\pgfpathlineto{\pgfqpoint{3.844339in}{1.007562in}}%
\pgfpathlineto{\pgfqpoint{3.888476in}{0.935297in}}%
\pgfpathlineto{\pgfqpoint{3.932614in}{0.912801in}}%
\pgfpathlineto{\pgfqpoint{3.976751in}{0.958649in}}%
\pgfpathlineto{\pgfqpoint{4.020889in}{1.044428in}}%
\pgfpathlineto{\pgfqpoint{4.065026in}{1.140789in}}%
\pgfpathlineto{\pgfqpoint{4.153301in}{1.303261in}}%
\pgfpathlineto{\pgfqpoint{4.241576in}{1.433018in}}%
\pgfpathlineto{\pgfqpoint{4.329851in}{1.543138in}}%
\pgfpathlineto{\pgfqpoint{4.418126in}{1.639662in}}%
\pgfpathlineto{\pgfqpoint{4.484713in}{1.701058in}}%
\pgfpathlineto{\pgfqpoint{4.484713in}{1.701058in}}%
\pgfusepath{stroke}%
\end{pgfscope}%
\begin{pgfscope}%
\pgfpathrectangle{\pgfqpoint{0.754713in}{0.682899in}}{\pgfqpoint{3.720000in}{3.020000in}}%
\pgfusepath{clip}%
\pgfsetrectcap%
\pgfsetroundjoin%
\pgfsetlinewidth{1.505625pt}%
\definecolor{currentstroke}{rgb}{0.201239,0.383670,0.554294}%
\pgfsetstrokecolor{currentstroke}%
\pgfsetdash{}{0pt}%
\pgfpathmoveto{\pgfqpoint{0.844456in}{0.972025in}}%
\pgfpathlineto{\pgfqpoint{0.874371in}{1.067048in}}%
\pgfpathlineto{\pgfqpoint{0.904286in}{1.128192in}}%
\pgfpathlineto{\pgfqpoint{0.934200in}{1.168233in}}%
\pgfpathlineto{\pgfqpoint{0.964115in}{1.216930in}}%
\pgfpathlineto{\pgfqpoint{0.994030in}{1.290947in}}%
\pgfpathlineto{\pgfqpoint{1.023944in}{1.353557in}}%
\pgfpathlineto{\pgfqpoint{1.083774in}{1.434567in}}%
\pgfpathlineto{\pgfqpoint{1.113688in}{1.490544in}}%
\pgfpathlineto{\pgfqpoint{1.143603in}{1.528376in}}%
\pgfpathlineto{\pgfqpoint{1.173518in}{1.578386in}}%
\pgfpathlineto{\pgfqpoint{1.203432in}{1.613820in}}%
\pgfpathlineto{\pgfqpoint{1.233347in}{1.642253in}}%
\pgfpathlineto{\pgfqpoint{1.263262in}{1.681827in}}%
\pgfpathlineto{\pgfqpoint{1.293176in}{1.732063in}}%
\pgfpathlineto{\pgfqpoint{1.323091in}{1.772794in}}%
\pgfpathlineto{\pgfqpoint{1.353006in}{1.799500in}}%
\pgfpathlineto{\pgfqpoint{1.382920in}{1.828350in}}%
\pgfpathlineto{\pgfqpoint{1.412835in}{1.867490in}}%
\pgfpathlineto{\pgfqpoint{1.442750in}{1.909786in}}%
\pgfpathlineto{\pgfqpoint{1.472664in}{1.936075in}}%
\pgfpathlineto{\pgfqpoint{1.502579in}{1.956450in}}%
\pgfpathlineto{\pgfqpoint{1.532494in}{1.979926in}}%
\pgfpathlineto{\pgfqpoint{1.562408in}{2.014665in}}%
\pgfpathlineto{\pgfqpoint{1.592323in}{2.051555in}}%
\pgfpathlineto{\pgfqpoint{1.622237in}{2.072315in}}%
\pgfpathlineto{\pgfqpoint{1.652152in}{2.087043in}}%
\pgfpathlineto{\pgfqpoint{1.682067in}{2.112664in}}%
\pgfpathlineto{\pgfqpoint{1.711981in}{2.144611in}}%
\pgfpathlineto{\pgfqpoint{1.741896in}{2.168314in}}%
\pgfpathlineto{\pgfqpoint{1.771811in}{2.182412in}}%
\pgfpathlineto{\pgfqpoint{1.801725in}{2.194101in}}%
\pgfpathlineto{\pgfqpoint{1.831640in}{2.216523in}}%
\pgfpathlineto{\pgfqpoint{1.861555in}{2.243618in}}%
\pgfpathlineto{\pgfqpoint{1.891469in}{2.262391in}}%
\pgfpathlineto{\pgfqpoint{1.921384in}{2.268354in}}%
\pgfpathlineto{\pgfqpoint{1.951299in}{2.282678in}}%
\pgfpathlineto{\pgfqpoint{1.981213in}{2.305140in}}%
\pgfpathlineto{\pgfqpoint{2.011128in}{2.323131in}}%
\pgfpathlineto{\pgfqpoint{2.041043in}{2.332299in}}%
\pgfpathlineto{\pgfqpoint{2.070957in}{2.336567in}}%
\pgfpathlineto{\pgfqpoint{2.100872in}{2.347173in}}%
\pgfpathlineto{\pgfqpoint{2.130787in}{2.363800in}}%
\pgfpathlineto{\pgfqpoint{2.160701in}{2.378097in}}%
\pgfpathlineto{\pgfqpoint{2.190616in}{2.379066in}}%
\pgfpathlineto{\pgfqpoint{2.220531in}{2.382763in}}%
\pgfpathlineto{\pgfqpoint{2.280360in}{2.406846in}}%
\pgfpathlineto{\pgfqpoint{2.310275in}{2.411492in}}%
\pgfpathlineto{\pgfqpoint{2.340189in}{2.409220in}}%
\pgfpathlineto{\pgfqpoint{2.370104in}{2.409825in}}%
\pgfpathlineto{\pgfqpoint{2.429933in}{2.425728in}}%
\pgfpathlineto{\pgfqpoint{2.489762in}{2.417767in}}%
\pgfpathlineto{\pgfqpoint{2.519677in}{2.419663in}}%
\pgfpathlineto{\pgfqpoint{2.549592in}{2.424505in}}%
\pgfpathlineto{\pgfqpoint{2.579506in}{2.424256in}}%
\pgfpathlineto{\pgfqpoint{2.639336in}{2.408932in}}%
\pgfpathlineto{\pgfqpoint{2.669250in}{2.406726in}}%
\pgfpathlineto{\pgfqpoint{2.699165in}{2.408183in}}%
\pgfpathlineto{\pgfqpoint{2.729080in}{2.400477in}}%
\pgfpathlineto{\pgfqpoint{2.758994in}{2.387800in}}%
\pgfpathlineto{\pgfqpoint{2.788909in}{2.380394in}}%
\pgfpathlineto{\pgfqpoint{2.818824in}{2.377489in}}%
\pgfpathlineto{\pgfqpoint{2.848738in}{2.371214in}}%
\pgfpathlineto{\pgfqpoint{2.878653in}{2.358256in}}%
\pgfpathlineto{\pgfqpoint{2.908568in}{2.342972in}}%
\pgfpathlineto{\pgfqpoint{2.938482in}{2.331003in}}%
\pgfpathlineto{\pgfqpoint{2.968397in}{2.324253in}}%
\pgfpathlineto{\pgfqpoint{2.998312in}{2.311406in}}%
\pgfpathlineto{\pgfqpoint{3.028226in}{2.292175in}}%
\pgfpathlineto{\pgfqpoint{3.058141in}{2.275924in}}%
\pgfpathlineto{\pgfqpoint{3.117970in}{2.250717in}}%
\pgfpathlineto{\pgfqpoint{3.147885in}{2.231697in}}%
\pgfpathlineto{\pgfqpoint{3.177799in}{2.208913in}}%
\pgfpathlineto{\pgfqpoint{3.207714in}{2.187991in}}%
\pgfpathlineto{\pgfqpoint{3.237629in}{2.172739in}}%
\pgfpathlineto{\pgfqpoint{3.267543in}{2.153423in}}%
\pgfpathlineto{\pgfqpoint{3.297458in}{2.127367in}}%
\pgfpathlineto{\pgfqpoint{3.327373in}{2.102813in}}%
\pgfpathlineto{\pgfqpoint{3.387202in}{2.059566in}}%
\pgfpathlineto{\pgfqpoint{3.417117in}{2.033817in}}%
\pgfpathlineto{\pgfqpoint{3.476946in}{1.972612in}}%
\pgfpathlineto{\pgfqpoint{3.506861in}{1.947077in}}%
\pgfpathlineto{\pgfqpoint{3.536775in}{1.919287in}}%
\pgfpathlineto{\pgfqpoint{3.626519in}{1.818747in}}%
\pgfpathlineto{\pgfqpoint{3.656434in}{1.787868in}}%
\pgfpathlineto{\pgfqpoint{3.686349in}{1.753210in}}%
\pgfpathlineto{\pgfqpoint{3.776093in}{1.635609in}}%
\pgfpathlineto{\pgfqpoint{3.806007in}{1.597715in}}%
\pgfpathlineto{\pgfqpoint{3.895751in}{1.465065in}}%
\pgfpathlineto{\pgfqpoint{3.925666in}{1.420774in}}%
\pgfpathlineto{\pgfqpoint{3.955580in}{1.372808in}}%
\pgfpathlineto{\pgfqpoint{4.015410in}{1.268647in}}%
\pgfpathlineto{\pgfqpoint{4.075239in}{1.163642in}}%
\pgfpathlineto{\pgfqpoint{4.135068in}{1.044426in}}%
\pgfpathlineto{\pgfqpoint{4.164983in}{1.004844in}}%
\pgfpathlineto{\pgfqpoint{4.194898in}{0.971173in}}%
\pgfpathlineto{\pgfqpoint{4.224812in}{0.935253in}}%
\pgfpathlineto{\pgfqpoint{4.254727in}{0.922199in}}%
\pgfpathlineto{\pgfqpoint{4.284642in}{0.931622in}}%
\pgfpathlineto{\pgfqpoint{4.314556in}{0.950627in}}%
\pgfpathlineto{\pgfqpoint{4.344471in}{0.984305in}}%
\pgfpathlineto{\pgfqpoint{4.374386in}{1.030579in}}%
\pgfpathlineto{\pgfqpoint{4.404300in}{1.069521in}}%
\pgfpathlineto{\pgfqpoint{4.434215in}{1.131253in}}%
\pgfpathlineto{\pgfqpoint{4.484713in}{1.209165in}}%
\pgfpathlineto{\pgfqpoint{4.484713in}{1.209165in}}%
\pgfusepath{stroke}%
\end{pgfscope}%
\begin{pgfscope}%
\pgfpathrectangle{\pgfqpoint{0.754713in}{0.682899in}}{\pgfqpoint{3.720000in}{3.020000in}}%
\pgfusepath{clip}%
\pgfsetrectcap%
\pgfsetroundjoin%
\pgfsetlinewidth{1.505625pt}%
\definecolor{currentstroke}{rgb}{0.283197,0.115680,0.436115}%
\pgfsetstrokecolor{currentstroke}%
\pgfsetdash{}{0pt}%
\pgfpathmoveto{\pgfqpoint{0.913460in}{1.266667in}}%
\pgfpathlineto{\pgfqpoint{0.933304in}{1.305795in}}%
\pgfpathlineto{\pgfqpoint{0.953147in}{1.337200in}}%
\pgfpathlineto{\pgfqpoint{0.972991in}{1.365133in}}%
\pgfpathlineto{\pgfqpoint{1.012677in}{1.426777in}}%
\pgfpathlineto{\pgfqpoint{1.032521in}{1.453695in}}%
\pgfpathlineto{\pgfqpoint{1.052364in}{1.484859in}}%
\pgfpathlineto{\pgfqpoint{1.171425in}{1.646308in}}%
\pgfpathlineto{\pgfqpoint{1.211112in}{1.693424in}}%
\pgfpathlineto{\pgfqpoint{1.230956in}{1.719020in}}%
\pgfpathlineto{\pgfqpoint{1.250799in}{1.740528in}}%
\pgfpathlineto{\pgfqpoint{1.290486in}{1.788559in}}%
\pgfpathlineto{\pgfqpoint{1.330173in}{1.830148in}}%
\pgfpathlineto{\pgfqpoint{1.350016in}{1.852619in}}%
\pgfpathlineto{\pgfqpoint{1.369860in}{1.871162in}}%
\pgfpathlineto{\pgfqpoint{1.389703in}{1.892235in}}%
\pgfpathlineto{\pgfqpoint{1.449234in}{1.948412in}}%
\pgfpathlineto{\pgfqpoint{1.508764in}{2.002449in}}%
\pgfpathlineto{\pgfqpoint{1.548451in}{2.032969in}}%
\pgfpathlineto{\pgfqpoint{1.568294in}{2.050861in}}%
\pgfpathlineto{\pgfqpoint{1.588138in}{2.064129in}}%
\pgfpathlineto{\pgfqpoint{1.607981in}{2.080662in}}%
\pgfpathlineto{\pgfqpoint{1.627825in}{2.095365in}}%
\pgfpathlineto{\pgfqpoint{1.647668in}{2.108057in}}%
\pgfpathlineto{\pgfqpoint{1.687355in}{2.136979in}}%
\pgfpathlineto{\pgfqpoint{1.707199in}{2.148404in}}%
\pgfpathlineto{\pgfqpoint{1.727042in}{2.161483in}}%
\pgfpathlineto{\pgfqpoint{1.826259in}{2.218520in}}%
\pgfpathlineto{\pgfqpoint{1.865946in}{2.237153in}}%
\pgfpathlineto{\pgfqpoint{1.885790in}{2.245724in}}%
\pgfpathlineto{\pgfqpoint{1.905633in}{2.255906in}}%
\pgfpathlineto{\pgfqpoint{1.985007in}{2.287646in}}%
\pgfpathlineto{\pgfqpoint{2.064381in}{2.314493in}}%
\pgfpathlineto{\pgfqpoint{2.163598in}{2.338359in}}%
\pgfpathlineto{\pgfqpoint{2.203285in}{2.345159in}}%
\pgfpathlineto{\pgfqpoint{2.282659in}{2.356536in}}%
\pgfpathlineto{\pgfqpoint{2.362033in}{2.360399in}}%
\pgfpathlineto{\pgfqpoint{2.421563in}{2.360495in}}%
\pgfpathlineto{\pgfqpoint{2.461250in}{2.358375in}}%
\pgfpathlineto{\pgfqpoint{2.520780in}{2.353657in}}%
\pgfpathlineto{\pgfqpoint{2.540624in}{2.350216in}}%
\pgfpathlineto{\pgfqpoint{2.560467in}{2.348075in}}%
\pgfpathlineto{\pgfqpoint{2.600154in}{2.340426in}}%
\pgfpathlineto{\pgfqpoint{2.619998in}{2.337736in}}%
\pgfpathlineto{\pgfqpoint{2.679528in}{2.323278in}}%
\pgfpathlineto{\pgfqpoint{2.778745in}{2.292490in}}%
\pgfpathlineto{\pgfqpoint{2.877963in}{2.252488in}}%
\pgfpathlineto{\pgfqpoint{2.897806in}{2.243874in}}%
\pgfpathlineto{\pgfqpoint{2.937493in}{2.223731in}}%
\pgfpathlineto{\pgfqpoint{2.997023in}{2.191897in}}%
\pgfpathlineto{\pgfqpoint{3.116084in}{2.116549in}}%
\pgfpathlineto{\pgfqpoint{3.155771in}{2.087757in}}%
\pgfpathlineto{\pgfqpoint{3.175615in}{2.073911in}}%
\pgfpathlineto{\pgfqpoint{3.274832in}{1.992583in}}%
\pgfpathlineto{\pgfqpoint{3.314519in}{1.957115in}}%
\pgfpathlineto{\pgfqpoint{3.354206in}{1.920238in}}%
\pgfpathlineto{\pgfqpoint{3.413736in}{1.861849in}}%
\pgfpathlineto{\pgfqpoint{3.473267in}{1.797496in}}%
\pgfpathlineto{\pgfqpoint{3.512953in}{1.752941in}}%
\pgfpathlineto{\pgfqpoint{3.592327in}{1.657123in}}%
\pgfpathlineto{\pgfqpoint{3.632014in}{1.605852in}}%
\pgfpathlineto{\pgfqpoint{3.711388in}{1.496544in}}%
\pgfpathlineto{\pgfqpoint{3.751075in}{1.438797in}}%
\pgfpathlineto{\pgfqpoint{3.810605in}{1.346958in}}%
\pgfpathlineto{\pgfqpoint{3.889979in}{1.214432in}}%
\pgfpathlineto{\pgfqpoint{3.909823in}{1.178908in}}%
\pgfpathlineto{\pgfqpoint{3.949510in}{1.101522in}}%
\pgfpathlineto{\pgfqpoint{3.989196in}{1.046803in}}%
\pgfpathlineto{\pgfqpoint{4.009040in}{1.046427in}}%
\pgfpathlineto{\pgfqpoint{4.028883in}{1.063931in}}%
\pgfpathlineto{\pgfqpoint{4.048727in}{1.089470in}}%
\pgfpathlineto{\pgfqpoint{4.068570in}{1.119620in}}%
\pgfpathlineto{\pgfqpoint{4.088414in}{1.152975in}}%
\pgfpathlineto{\pgfqpoint{4.128101in}{1.203496in}}%
\pgfpathlineto{\pgfqpoint{4.187631in}{1.271802in}}%
\pgfpathlineto{\pgfqpoint{4.227318in}{1.308723in}}%
\pgfpathlineto{\pgfqpoint{4.247161in}{1.330310in}}%
\pgfpathlineto{\pgfqpoint{4.286848in}{1.365899in}}%
\pgfpathlineto{\pgfqpoint{4.306692in}{1.382980in}}%
\pgfpathlineto{\pgfqpoint{4.326535in}{1.397554in}}%
\pgfpathlineto{\pgfqpoint{4.366222in}{1.428849in}}%
\pgfpathlineto{\pgfqpoint{4.386066in}{1.440781in}}%
\pgfpathlineto{\pgfqpoint{4.425753in}{1.467700in}}%
\pgfpathlineto{\pgfqpoint{4.445596in}{1.477794in}}%
\pgfpathlineto{\pgfqpoint{4.484713in}{1.500267in}}%
\pgfpathlineto{\pgfqpoint{4.484713in}{1.500267in}}%
\pgfusepath{stroke}%
\end{pgfscope}%
\begin{pgfscope}%
\pgfpathrectangle{\pgfqpoint{0.754713in}{0.682899in}}{\pgfqpoint{3.720000in}{3.020000in}}%
\pgfusepath{clip}%
\pgfsetrectcap%
\pgfsetroundjoin%
\pgfsetlinewidth{1.505625pt}%
\definecolor{currentstroke}{rgb}{0.276022,0.044167,0.370164}%
\pgfsetstrokecolor{currentstroke}%
\pgfsetdash{}{0pt}%
\pgfpathmoveto{\pgfqpoint{0.877572in}{1.218494in}}%
\pgfpathlineto{\pgfqpoint{0.892929in}{1.232550in}}%
\pgfpathlineto{\pgfqpoint{0.908287in}{1.253752in}}%
\pgfpathlineto{\pgfqpoint{0.923644in}{1.293437in}}%
\pgfpathlineto{\pgfqpoint{0.939001in}{1.316000in}}%
\pgfpathlineto{\pgfqpoint{0.954359in}{1.325059in}}%
\pgfpathlineto{\pgfqpoint{0.985074in}{1.386803in}}%
\pgfpathlineto{\pgfqpoint{1.000431in}{1.405426in}}%
\pgfpathlineto{\pgfqpoint{1.015788in}{1.421348in}}%
\pgfpathlineto{\pgfqpoint{1.046503in}{1.471379in}}%
\pgfpathlineto{\pgfqpoint{1.092575in}{1.527004in}}%
\pgfpathlineto{\pgfqpoint{1.107933in}{1.550805in}}%
\pgfpathlineto{\pgfqpoint{1.123290in}{1.568180in}}%
\pgfpathlineto{\pgfqpoint{1.138648in}{1.583538in}}%
\pgfpathlineto{\pgfqpoint{1.169362in}{1.625003in}}%
\pgfpathlineto{\pgfqpoint{1.200077in}{1.655672in}}%
\pgfpathlineto{\pgfqpoint{1.230792in}{1.694370in}}%
\pgfpathlineto{\pgfqpoint{1.246149in}{1.708065in}}%
\pgfpathlineto{\pgfqpoint{1.261507in}{1.723331in}}%
\pgfpathlineto{\pgfqpoint{1.276864in}{1.742117in}}%
\pgfpathlineto{\pgfqpoint{1.292222in}{1.758918in}}%
\pgfpathlineto{\pgfqpoint{1.322936in}{1.785837in}}%
\pgfpathlineto{\pgfqpoint{1.338294in}{1.803432in}}%
\pgfpathlineto{\pgfqpoint{1.353651in}{1.819225in}}%
\pgfpathlineto{\pgfqpoint{1.369009in}{1.830374in}}%
\pgfpathlineto{\pgfqpoint{1.384366in}{1.845173in}}%
\pgfpathlineto{\pgfqpoint{1.399723in}{1.861661in}}%
\pgfpathlineto{\pgfqpoint{1.415081in}{1.875228in}}%
\pgfpathlineto{\pgfqpoint{1.430438in}{1.886468in}}%
\pgfpathlineto{\pgfqpoint{1.445796in}{1.899728in}}%
\pgfpathlineto{\pgfqpoint{1.461153in}{1.915591in}}%
\pgfpathlineto{\pgfqpoint{1.491868in}{1.937904in}}%
\pgfpathlineto{\pgfqpoint{1.522583in}{1.965579in}}%
\pgfpathlineto{\pgfqpoint{1.553297in}{1.986579in}}%
\pgfpathlineto{\pgfqpoint{1.584012in}{2.012418in}}%
\pgfpathlineto{\pgfqpoint{1.599370in}{2.021444in}}%
\pgfpathlineto{\pgfqpoint{1.614727in}{2.032099in}}%
\pgfpathlineto{\pgfqpoint{1.645442in}{2.055631in}}%
\pgfpathlineto{\pgfqpoint{1.676157in}{2.074037in}}%
\pgfpathlineto{\pgfqpoint{1.691514in}{2.086049in}}%
\pgfpathlineto{\pgfqpoint{1.706871in}{2.096051in}}%
\pgfpathlineto{\pgfqpoint{1.722229in}{2.103888in}}%
\pgfpathlineto{\pgfqpoint{1.768301in}{2.133633in}}%
\pgfpathlineto{\pgfqpoint{1.783658in}{2.141097in}}%
\pgfpathlineto{\pgfqpoint{1.829731in}{2.168024in}}%
\pgfpathlineto{\pgfqpoint{1.845088in}{2.175379in}}%
\pgfpathlineto{\pgfqpoint{1.875803in}{2.193368in}}%
\pgfpathlineto{\pgfqpoint{1.906518in}{2.207028in}}%
\pgfpathlineto{\pgfqpoint{1.937232in}{2.223725in}}%
\pgfpathlineto{\pgfqpoint{1.967947in}{2.236325in}}%
\pgfpathlineto{\pgfqpoint{1.998662in}{2.251400in}}%
\pgfpathlineto{\pgfqpoint{2.029377in}{2.263156in}}%
\pgfpathlineto{\pgfqpoint{2.044734in}{2.270610in}}%
\pgfpathlineto{\pgfqpoint{2.121521in}{2.299704in}}%
\pgfpathlineto{\pgfqpoint{2.136879in}{2.304181in}}%
\pgfpathlineto{\pgfqpoint{2.182951in}{2.320048in}}%
\pgfpathlineto{\pgfqpoint{2.213666in}{2.329643in}}%
\pgfpathlineto{\pgfqpoint{2.229023in}{2.334848in}}%
\pgfpathlineto{\pgfqpoint{2.351882in}{2.366294in}}%
\pgfpathlineto{\pgfqpoint{2.382597in}{2.371963in}}%
\pgfpathlineto{\pgfqpoint{2.413312in}{2.378622in}}%
\pgfpathlineto{\pgfqpoint{2.444027in}{2.383693in}}%
\pgfpathlineto{\pgfqpoint{2.474742in}{2.389051in}}%
\pgfpathlineto{\pgfqpoint{2.505456in}{2.393190in}}%
\pgfpathlineto{\pgfqpoint{2.536171in}{2.397093in}}%
\pgfpathlineto{\pgfqpoint{2.643673in}{2.407047in}}%
\pgfpathlineto{\pgfqpoint{2.781890in}{2.409628in}}%
\pgfpathlineto{\pgfqpoint{2.827962in}{2.408390in}}%
\pgfpathlineto{\pgfqpoint{2.935464in}{2.401162in}}%
\pgfpathlineto{\pgfqpoint{3.042965in}{2.387017in}}%
\pgfpathlineto{\pgfqpoint{3.089038in}{2.378855in}}%
\pgfpathlineto{\pgfqpoint{3.119752in}{2.373002in}}%
\pgfpathlineto{\pgfqpoint{3.242612in}{2.343035in}}%
\pgfpathlineto{\pgfqpoint{3.334756in}{2.314524in}}%
\pgfpathlineto{\pgfqpoint{3.396186in}{2.292178in}}%
\pgfpathlineto{\pgfqpoint{3.472973in}{2.260275in}}%
\pgfpathlineto{\pgfqpoint{3.549760in}{2.224028in}}%
\pgfpathlineto{\pgfqpoint{3.595832in}{2.200086in}}%
\pgfpathlineto{\pgfqpoint{3.657262in}{2.165524in}}%
\pgfpathlineto{\pgfqpoint{3.703334in}{2.137618in}}%
\pgfpathlineto{\pgfqpoint{3.749406in}{2.108167in}}%
\pgfpathlineto{\pgfqpoint{3.810836in}{2.065600in}}%
\pgfpathlineto{\pgfqpoint{3.872265in}{2.019850in}}%
\pgfpathlineto{\pgfqpoint{3.933695in}{1.970964in}}%
\pgfpathlineto{\pgfqpoint{3.995124in}{1.918436in}}%
\pgfpathlineto{\pgfqpoint{4.041197in}{1.876548in}}%
\pgfpathlineto{\pgfqpoint{4.102626in}{1.817817in}}%
\pgfpathlineto{\pgfqpoint{4.164056in}{1.754767in}}%
\pgfpathlineto{\pgfqpoint{4.225485in}{1.687653in}}%
\pgfpathlineto{\pgfqpoint{4.286915in}{1.615865in}}%
\pgfpathlineto{\pgfqpoint{4.332987in}{1.558790in}}%
\pgfpathlineto{\pgfqpoint{4.332987in}{1.558790in}}%
\pgfusepath{stroke}%
\end{pgfscope}%
\begin{pgfscope}%
\pgfpathrectangle{\pgfqpoint{0.754713in}{0.682899in}}{\pgfqpoint{3.720000in}{3.020000in}}%
\pgfusepath{clip}%
\pgfsetrectcap%
\pgfsetroundjoin%
\pgfsetlinewidth{1.505625pt}%
\definecolor{currentstroke}{rgb}{0.267004,0.004874,0.329415}%
\pgfsetstrokecolor{currentstroke}%
\pgfsetdash{}{0pt}%
\pgfpathmoveto{\pgfqpoint{0.892413in}{1.242420in}}%
\pgfpathlineto{\pgfqpoint{0.904931in}{1.248931in}}%
\pgfpathlineto{\pgfqpoint{0.917449in}{1.278821in}}%
\pgfpathlineto{\pgfqpoint{0.929967in}{1.303812in}}%
\pgfpathlineto{\pgfqpoint{0.942486in}{1.318266in}}%
\pgfpathlineto{\pgfqpoint{0.955004in}{1.336597in}}%
\pgfpathlineto{\pgfqpoint{0.967522in}{1.359057in}}%
\pgfpathlineto{\pgfqpoint{0.980040in}{1.384915in}}%
\pgfpathlineto{\pgfqpoint{0.992559in}{1.402692in}}%
\pgfpathlineto{\pgfqpoint{1.005077in}{1.417661in}}%
\pgfpathlineto{\pgfqpoint{1.017595in}{1.437862in}}%
\pgfpathlineto{\pgfqpoint{1.030113in}{1.461634in}}%
\pgfpathlineto{\pgfqpoint{1.042631in}{1.479633in}}%
\pgfpathlineto{\pgfqpoint{1.067668in}{1.510554in}}%
\pgfpathlineto{\pgfqpoint{1.080186in}{1.530047in}}%
\pgfpathlineto{\pgfqpoint{1.092704in}{1.552489in}}%
\pgfpathlineto{\pgfqpoint{1.105222in}{1.568958in}}%
\pgfpathlineto{\pgfqpoint{1.117741in}{1.580912in}}%
\pgfpathlineto{\pgfqpoint{1.130259in}{1.599190in}}%
\pgfpathlineto{\pgfqpoint{1.142777in}{1.620063in}}%
\pgfpathlineto{\pgfqpoint{1.155295in}{1.636508in}}%
\pgfpathlineto{\pgfqpoint{1.180332in}{1.663273in}}%
\pgfpathlineto{\pgfqpoint{1.205368in}{1.700357in}}%
\pgfpathlineto{\pgfqpoint{1.217886in}{1.715084in}}%
\pgfpathlineto{\pgfqpoint{1.230405in}{1.725821in}}%
\pgfpathlineto{\pgfqpoint{1.267959in}{1.775899in}}%
\pgfpathlineto{\pgfqpoint{1.292996in}{1.799738in}}%
\pgfpathlineto{\pgfqpoint{1.305514in}{1.815394in}}%
\pgfpathlineto{\pgfqpoint{1.318032in}{1.833370in}}%
\pgfpathlineto{\pgfqpoint{1.355587in}{1.870332in}}%
\pgfpathlineto{\pgfqpoint{1.368105in}{1.886978in}}%
\pgfpathlineto{\pgfqpoint{1.380623in}{1.900467in}}%
\pgfpathlineto{\pgfqpoint{1.405660in}{1.921849in}}%
\pgfpathlineto{\pgfqpoint{1.430696in}{1.952398in}}%
\pgfpathlineto{\pgfqpoint{1.443214in}{1.963237in}}%
\pgfpathlineto{\pgfqpoint{1.455732in}{1.972204in}}%
\pgfpathlineto{\pgfqpoint{1.493287in}{2.013387in}}%
\pgfpathlineto{\pgfqpoint{1.518324in}{2.032789in}}%
\pgfpathlineto{\pgfqpoint{1.530842in}{2.045862in}}%
\pgfpathlineto{\pgfqpoint{1.543360in}{2.060594in}}%
\pgfpathlineto{\pgfqpoint{1.568396in}{2.078920in}}%
\pgfpathlineto{\pgfqpoint{1.605951in}{2.116273in}}%
\pgfpathlineto{\pgfqpoint{1.630987in}{2.134066in}}%
\pgfpathlineto{\pgfqpoint{1.656024in}{2.159650in}}%
\pgfpathlineto{\pgfqpoint{1.681060in}{2.176476in}}%
\pgfpathlineto{\pgfqpoint{1.718615in}{2.210662in}}%
\pgfpathlineto{\pgfqpoint{1.743651in}{2.227126in}}%
\pgfpathlineto{\pgfqpoint{1.768688in}{2.250671in}}%
\pgfpathlineto{\pgfqpoint{1.793724in}{2.266360in}}%
\pgfpathlineto{\pgfqpoint{1.831279in}{2.297969in}}%
\pgfpathlineto{\pgfqpoint{1.856315in}{2.313145in}}%
\pgfpathlineto{\pgfqpoint{1.881352in}{2.334875in}}%
\pgfpathlineto{\pgfqpoint{1.906388in}{2.349410in}}%
\pgfpathlineto{\pgfqpoint{1.943943in}{2.378551in}}%
\pgfpathlineto{\pgfqpoint{1.968979in}{2.392584in}}%
\pgfpathlineto{\pgfqpoint{1.994016in}{2.412523in}}%
\pgfpathlineto{\pgfqpoint{2.019052in}{2.426206in}}%
\pgfpathlineto{\pgfqpoint{2.056607in}{2.453155in}}%
\pgfpathlineto{\pgfqpoint{2.069125in}{2.459093in}}%
\pgfpathlineto{\pgfqpoint{2.081643in}{2.466453in}}%
\pgfpathlineto{\pgfqpoint{2.106680in}{2.484476in}}%
\pgfpathlineto{\pgfqpoint{2.131716in}{2.497541in}}%
\pgfpathlineto{\pgfqpoint{2.169271in}{2.522208in}}%
\pgfpathlineto{\pgfqpoint{2.181789in}{2.527826in}}%
\pgfpathlineto{\pgfqpoint{2.194307in}{2.535071in}}%
\pgfpathlineto{\pgfqpoint{2.219343in}{2.551394in}}%
\pgfpathlineto{\pgfqpoint{2.244380in}{2.563724in}}%
\pgfpathlineto{\pgfqpoint{2.281935in}{2.586349in}}%
\pgfpathlineto{\pgfqpoint{2.294453in}{2.591697in}}%
\pgfpathlineto{\pgfqpoint{2.306971in}{2.598412in}}%
\pgfpathlineto{\pgfqpoint{2.332007in}{2.613755in}}%
\pgfpathlineto{\pgfqpoint{2.357044in}{2.625177in}}%
\pgfpathlineto{\pgfqpoint{2.394598in}{2.646314in}}%
\pgfpathlineto{\pgfqpoint{2.407117in}{2.651352in}}%
\pgfpathlineto{\pgfqpoint{2.419635in}{2.657610in}}%
\pgfpathlineto{\pgfqpoint{2.444671in}{2.672023in}}%
\pgfpathlineto{\pgfqpoint{2.469708in}{2.682861in}}%
\pgfpathlineto{\pgfqpoint{2.507262in}{2.702361in}}%
\pgfpathlineto{\pgfqpoint{2.532299in}{2.713231in}}%
\pgfpathlineto{\pgfqpoint{2.544817in}{2.720655in}}%
\pgfpathlineto{\pgfqpoint{2.569853in}{2.731583in}}%
\pgfpathlineto{\pgfqpoint{2.582372in}{2.736773in}}%
\pgfpathlineto{\pgfqpoint{2.607408in}{2.749928in}}%
\pgfpathlineto{\pgfqpoint{2.644963in}{2.765285in}}%
\pgfpathlineto{\pgfqpoint{2.657481in}{2.772257in}}%
\pgfpathlineto{\pgfqpoint{2.682517in}{2.782188in}}%
\pgfpathlineto{\pgfqpoint{2.695036in}{2.787086in}}%
\pgfpathlineto{\pgfqpoint{2.720072in}{2.799302in}}%
\pgfpathlineto{\pgfqpoint{2.745108in}{2.807976in}}%
\pgfpathlineto{\pgfqpoint{2.782663in}{2.825132in}}%
\pgfpathlineto{\pgfqpoint{2.807700in}{2.833983in}}%
\pgfpathlineto{\pgfqpoint{2.832736in}{2.845328in}}%
\pgfpathlineto{\pgfqpoint{2.857772in}{2.853451in}}%
\pgfpathlineto{\pgfqpoint{2.882809in}{2.865203in}}%
\pgfpathlineto{\pgfqpoint{2.970436in}{2.896250in}}%
\pgfpathlineto{\pgfqpoint{2.995473in}{2.907031in}}%
\pgfpathlineto{\pgfqpoint{3.083100in}{2.936125in}}%
\pgfpathlineto{\pgfqpoint{3.108137in}{2.946178in}}%
\pgfpathlineto{\pgfqpoint{3.158210in}{2.961949in}}%
\pgfpathlineto{\pgfqpoint{3.233319in}{2.985930in}}%
\pgfpathlineto{\pgfqpoint{3.258355in}{2.992700in}}%
\pgfpathlineto{\pgfqpoint{3.283392in}{3.000733in}}%
\pgfpathlineto{\pgfqpoint{3.308428in}{3.007833in}}%
\pgfpathlineto{\pgfqpoint{3.333464in}{3.016110in}}%
\pgfpathlineto{\pgfqpoint{3.408574in}{3.036362in}}%
\pgfpathlineto{\pgfqpoint{3.458647in}{3.050294in}}%
\pgfpathlineto{\pgfqpoint{3.483683in}{3.056923in}}%
\pgfpathlineto{\pgfqpoint{3.508719in}{3.063399in}}%
\pgfpathlineto{\pgfqpoint{3.533756in}{3.069714in}}%
\pgfpathlineto{\pgfqpoint{3.558792in}{3.076648in}}%
\pgfpathlineto{\pgfqpoint{3.596347in}{3.085337in}}%
\pgfpathlineto{\pgfqpoint{3.671456in}{3.103605in}}%
\pgfpathlineto{\pgfqpoint{3.683974in}{3.105651in}}%
\pgfpathlineto{\pgfqpoint{3.683974in}{3.105651in}}%
\pgfusepath{stroke}%
\end{pgfscope}%
\begin{pgfscope}%
\pgfsetrectcap%
\pgfsetmiterjoin%
\pgfsetlinewidth{0.803000pt}%
\definecolor{currentstroke}{rgb}{0.501961,0.501961,0.501961}%
\pgfsetstrokecolor{currentstroke}%
\pgfsetdash{}{0pt}%
\pgfpathmoveto{\pgfqpoint{0.754713in}{0.682899in}}%
\pgfpathlineto{\pgfqpoint{0.754713in}{3.702899in}}%
\pgfusepath{stroke}%
\end{pgfscope}%
\begin{pgfscope}%
\pgfsetrectcap%
\pgfsetmiterjoin%
\pgfsetlinewidth{0.803000pt}%
\definecolor{currentstroke}{rgb}{0.501961,0.501961,0.501961}%
\pgfsetstrokecolor{currentstroke}%
\pgfsetdash{}{0pt}%
\pgfpathmoveto{\pgfqpoint{4.474713in}{0.682899in}}%
\pgfpathlineto{\pgfqpoint{4.474713in}{3.702899in}}%
\pgfusepath{stroke}%
\end{pgfscope}%
\begin{pgfscope}%
\pgfsetrectcap%
\pgfsetmiterjoin%
\pgfsetlinewidth{0.803000pt}%
\definecolor{currentstroke}{rgb}{0.501961,0.501961,0.501961}%
\pgfsetstrokecolor{currentstroke}%
\pgfsetdash{}{0pt}%
\pgfpathmoveto{\pgfqpoint{0.754713in}{0.682899in}}%
\pgfpathlineto{\pgfqpoint{4.474712in}{0.682899in}}%
\pgfusepath{stroke}%
\end{pgfscope}%
\begin{pgfscope}%
\pgfsetrectcap%
\pgfsetmiterjoin%
\pgfsetlinewidth{0.803000pt}%
\definecolor{currentstroke}{rgb}{0.501961,0.501961,0.501961}%
\pgfsetstrokecolor{currentstroke}%
\pgfsetdash{}{0pt}%
\pgfpathmoveto{\pgfqpoint{0.754713in}{3.702899in}}%
\pgfpathlineto{\pgfqpoint{4.474712in}{3.702899in}}%
\pgfusepath{stroke}%
\end{pgfscope}%
\begin{pgfscope}%
\pgfpathrectangle{\pgfqpoint{4.707213in}{0.682899in}}{\pgfqpoint{0.151000in}{3.020000in}}%
\pgfusepath{clip}%
\pgfsetbuttcap%
\pgfsetmiterjoin%
\definecolor{currentfill}{rgb}{1.000000,1.000000,1.000000}%
\pgfsetfillcolor{currentfill}%
\pgfsetlinewidth{0.010037pt}%
\definecolor{currentstroke}{rgb}{1.000000,1.000000,1.000000}%
\pgfsetstrokecolor{currentstroke}%
\pgfsetdash{}{0pt}%
\pgfpathmoveto{\pgfqpoint{4.707213in}{0.682899in}}%
\pgfpathlineto{\pgfqpoint{4.707213in}{0.694696in}}%
\pgfpathlineto{\pgfqpoint{4.707213in}{3.691102in}}%
\pgfpathlineto{\pgfqpoint{4.707213in}{3.702899in}}%
\pgfpathlineto{\pgfqpoint{4.858213in}{3.702899in}}%
\pgfpathlineto{\pgfqpoint{4.858213in}{3.691102in}}%
\pgfpathlineto{\pgfqpoint{4.858213in}{0.694696in}}%
\pgfpathlineto{\pgfqpoint{4.858213in}{0.682899in}}%
\pgfpathclose%
\pgfusepath{stroke,fill}%
\end{pgfscope}%
\begin{pgfscope}%
\pgfsys@transformshift{4.710000in}{0.687899in}%
\pgftext[left,bottom]{\pgfimage[interpolate=true,width=0.150000in,height=3.020000in]{series_s_ds-img0.png}}%
\end{pgfscope}%
\begin{pgfscope}%
\pgfsetbuttcap%
\pgfsetroundjoin%
\definecolor{currentfill}{rgb}{0.000000,0.000000,0.000000}%
\pgfsetfillcolor{currentfill}%
\pgfsetlinewidth{0.803000pt}%
\definecolor{currentstroke}{rgb}{0.000000,0.000000,0.000000}%
\pgfsetstrokecolor{currentstroke}%
\pgfsetdash{}{0pt}%
\pgfsys@defobject{currentmarker}{\pgfqpoint{0.000000in}{0.000000in}}{\pgfqpoint{0.048611in}{0.000000in}}{%
\pgfpathmoveto{\pgfqpoint{0.000000in}{0.000000in}}%
\pgfpathlineto{\pgfqpoint{0.048611in}{0.000000in}}%
\pgfusepath{stroke,fill}%
}%
\begin{pgfscope}%
\pgfsys@transformshift{4.858213in}{1.468806in}%
\pgfsys@useobject{currentmarker}{}%
\end{pgfscope}%
\end{pgfscope}%
\begin{pgfscope}%
\pgfsetbuttcap%
\pgfsetroundjoin%
\definecolor{currentfill}{rgb}{0.000000,0.000000,0.000000}%
\pgfsetfillcolor{currentfill}%
\pgfsetlinewidth{0.803000pt}%
\definecolor{currentstroke}{rgb}{0.000000,0.000000,0.000000}%
\pgfsetstrokecolor{currentstroke}%
\pgfsetdash{}{0pt}%
\pgfsys@defobject{currentmarker}{\pgfqpoint{0.000000in}{0.000000in}}{\pgfqpoint{0.048611in}{0.000000in}}{%
\pgfpathmoveto{\pgfqpoint{0.000000in}{0.000000in}}%
\pgfpathlineto{\pgfqpoint{0.048611in}{0.000000in}}%
\pgfusepath{stroke,fill}%
}%
\begin{pgfscope}%
\pgfsys@transformshift{4.858213in}{1.963925in}%
\pgfsys@useobject{currentmarker}{}%
\end{pgfscope}%
\end{pgfscope}%
\begin{pgfscope}%
\pgfsetbuttcap%
\pgfsetroundjoin%
\definecolor{currentfill}{rgb}{0.000000,0.000000,0.000000}%
\pgfsetfillcolor{currentfill}%
\pgfsetlinewidth{0.803000pt}%
\definecolor{currentstroke}{rgb}{0.000000,0.000000,0.000000}%
\pgfsetstrokecolor{currentstroke}%
\pgfsetdash{}{0pt}%
\pgfsys@defobject{currentmarker}{\pgfqpoint{0.000000in}{0.000000in}}{\pgfqpoint{0.048611in}{0.000000in}}{%
\pgfpathmoveto{\pgfqpoint{0.000000in}{0.000000in}}%
\pgfpathlineto{\pgfqpoint{0.048611in}{0.000000in}}%
\pgfusepath{stroke,fill}%
}%
\begin{pgfscope}%
\pgfsys@transformshift{4.858213in}{2.315218in}%
\pgfsys@useobject{currentmarker}{}%
\end{pgfscope}%
\end{pgfscope}%
\begin{pgfscope}%
\pgfsetbuttcap%
\pgfsetroundjoin%
\definecolor{currentfill}{rgb}{0.000000,0.000000,0.000000}%
\pgfsetfillcolor{currentfill}%
\pgfsetlinewidth{0.803000pt}%
\definecolor{currentstroke}{rgb}{0.000000,0.000000,0.000000}%
\pgfsetstrokecolor{currentstroke}%
\pgfsetdash{}{0pt}%
\pgfsys@defobject{currentmarker}{\pgfqpoint{0.000000in}{0.000000in}}{\pgfqpoint{0.048611in}{0.000000in}}{%
\pgfpathmoveto{\pgfqpoint{0.000000in}{0.000000in}}%
\pgfpathlineto{\pgfqpoint{0.048611in}{0.000000in}}%
\pgfusepath{stroke,fill}%
}%
\begin{pgfscope}%
\pgfsys@transformshift{4.858213in}{2.587702in}%
\pgfsys@useobject{currentmarker}{}%
\end{pgfscope}%
\end{pgfscope}%
\begin{pgfscope}%
\pgfsetbuttcap%
\pgfsetroundjoin%
\definecolor{currentfill}{rgb}{0.000000,0.000000,0.000000}%
\pgfsetfillcolor{currentfill}%
\pgfsetlinewidth{0.803000pt}%
\definecolor{currentstroke}{rgb}{0.000000,0.000000,0.000000}%
\pgfsetstrokecolor{currentstroke}%
\pgfsetdash{}{0pt}%
\pgfsys@defobject{currentmarker}{\pgfqpoint{0.000000in}{0.000000in}}{\pgfqpoint{0.048611in}{0.000000in}}{%
\pgfpathmoveto{\pgfqpoint{0.000000in}{0.000000in}}%
\pgfpathlineto{\pgfqpoint{0.048611in}{0.000000in}}%
\pgfusepath{stroke,fill}%
}%
\begin{pgfscope}%
\pgfsys@transformshift{4.858213in}{2.810337in}%
\pgfsys@useobject{currentmarker}{}%
\end{pgfscope}%
\end{pgfscope}%
\begin{pgfscope}%
\pgfsetbuttcap%
\pgfsetroundjoin%
\definecolor{currentfill}{rgb}{0.000000,0.000000,0.000000}%
\pgfsetfillcolor{currentfill}%
\pgfsetlinewidth{0.803000pt}%
\definecolor{currentstroke}{rgb}{0.000000,0.000000,0.000000}%
\pgfsetstrokecolor{currentstroke}%
\pgfsetdash{}{0pt}%
\pgfsys@defobject{currentmarker}{\pgfqpoint{0.000000in}{0.000000in}}{\pgfqpoint{0.048611in}{0.000000in}}{%
\pgfpathmoveto{\pgfqpoint{0.000000in}{0.000000in}}%
\pgfpathlineto{\pgfqpoint{0.048611in}{0.000000in}}%
\pgfusepath{stroke,fill}%
}%
\begin{pgfscope}%
\pgfsys@transformshift{4.858213in}{2.998573in}%
\pgfsys@useobject{currentmarker}{}%
\end{pgfscope}%
\end{pgfscope}%
\begin{pgfscope}%
\pgfsetbuttcap%
\pgfsetroundjoin%
\definecolor{currentfill}{rgb}{0.000000,0.000000,0.000000}%
\pgfsetfillcolor{currentfill}%
\pgfsetlinewidth{0.803000pt}%
\definecolor{currentstroke}{rgb}{0.000000,0.000000,0.000000}%
\pgfsetstrokecolor{currentstroke}%
\pgfsetdash{}{0pt}%
\pgfsys@defobject{currentmarker}{\pgfqpoint{0.000000in}{0.000000in}}{\pgfqpoint{0.048611in}{0.000000in}}{%
\pgfpathmoveto{\pgfqpoint{0.000000in}{0.000000in}}%
\pgfpathlineto{\pgfqpoint{0.048611in}{0.000000in}}%
\pgfusepath{stroke,fill}%
}%
\begin{pgfscope}%
\pgfsys@transformshift{4.858213in}{3.161630in}%
\pgfsys@useobject{currentmarker}{}%
\end{pgfscope}%
\end{pgfscope}%
\begin{pgfscope}%
\pgfsetbuttcap%
\pgfsetroundjoin%
\definecolor{currentfill}{rgb}{0.000000,0.000000,0.000000}%
\pgfsetfillcolor{currentfill}%
\pgfsetlinewidth{0.803000pt}%
\definecolor{currentstroke}{rgb}{0.000000,0.000000,0.000000}%
\pgfsetstrokecolor{currentstroke}%
\pgfsetdash{}{0pt}%
\pgfsys@defobject{currentmarker}{\pgfqpoint{0.000000in}{0.000000in}}{\pgfqpoint{0.048611in}{0.000000in}}{%
\pgfpathmoveto{\pgfqpoint{0.000000in}{0.000000in}}%
\pgfpathlineto{\pgfqpoint{0.048611in}{0.000000in}}%
\pgfusepath{stroke,fill}%
}%
\begin{pgfscope}%
\pgfsys@transformshift{4.858213in}{3.305456in}%
\pgfsys@useobject{currentmarker}{}%
\end{pgfscope}%
\end{pgfscope}%
\begin{pgfscope}%
\pgfsetbuttcap%
\pgfsetroundjoin%
\definecolor{currentfill}{rgb}{0.000000,0.000000,0.000000}%
\pgfsetfillcolor{currentfill}%
\pgfsetlinewidth{0.803000pt}%
\definecolor{currentstroke}{rgb}{0.000000,0.000000,0.000000}%
\pgfsetstrokecolor{currentstroke}%
\pgfsetdash{}{0pt}%
\pgfsys@defobject{currentmarker}{\pgfqpoint{0.000000in}{0.000000in}}{\pgfqpoint{0.048611in}{0.000000in}}{%
\pgfpathmoveto{\pgfqpoint{0.000000in}{0.000000in}}%
\pgfpathlineto{\pgfqpoint{0.048611in}{0.000000in}}%
\pgfusepath{stroke,fill}%
}%
\begin{pgfscope}%
\pgfsys@transformshift{4.858213in}{3.434114in}%
\pgfsys@useobject{currentmarker}{}%
\end{pgfscope}%
\end{pgfscope}%
\begin{pgfscope}%
\pgftext[x=4.955435in,y=3.349695in,left,base]{\rmfamily\fontsize{16.000000}{19.200000}\selectfont \(\displaystyle 10^{1}\)}%
\end{pgfscope}%
\begin{pgfscope}%
\pgftext[x=5.369668in,y=2.192899in,,top]{\rmfamily\fontsize{14.000000}{16.800000}\selectfont \(\displaystyle {\mathbf{E} \mbox{u}}\)}%
\end{pgfscope}%
\begin{pgfscope}%
\pgfsetbuttcap%
\pgfsetmiterjoin%
\pgfsetlinewidth{0.803000pt}%
\definecolor{currentstroke}{rgb}{0.501961,0.501961,0.501961}%
\pgfsetstrokecolor{currentstroke}%
\pgfsetdash{}{0pt}%
\pgfpathmoveto{\pgfqpoint{4.707213in}{0.682899in}}%
\pgfpathlineto{\pgfqpoint{4.707213in}{0.694696in}}%
\pgfpathlineto{\pgfqpoint{4.707213in}{3.691102in}}%
\pgfpathlineto{\pgfqpoint{4.707213in}{3.702899in}}%
\pgfpathlineto{\pgfqpoint{4.858213in}{3.702899in}}%
\pgfpathlineto{\pgfqpoint{4.858213in}{3.691102in}}%
\pgfpathlineto{\pgfqpoint{4.858213in}{0.694696in}}%
\pgfpathlineto{\pgfqpoint{4.858213in}{0.682899in}}%
\pgfpathclose%
\pgfusepath{stroke}%
\end{pgfscope}%
\end{pgfpicture}%
\makeatother%
\endgroup%

    \caption{Non-dimensional trajectories with the short-time scaling.\label{fig:series_s_ds}}
\end{figure}

\begin{figure}[htb]
    \centering
    %% Creator: Matplotlib, PGF backend
%%
%% To include the figure in your LaTeX document, write
%%   \input{<filename>.pgf}
%%
%% Make sure the required packages are loaded in your preamble
%%   \usepackage{pgf}
%%
%% Figures using additional raster images can only be included by \input if
%% they are in the same directory as the main LaTeX file. For loading figures
%% from other directories you can use the `import` package
%%   \usepackage{import}
%% and then include the figures with
%%   \import{<path to file>}{<filename>.pgf}
%%
%% Matplotlib used the following preamble
%%   \usepackage{fontspec}
%%   \setmainfont{DejaVu Serif}
%%   \setsansfont{DejaVu Sans}
%%   \setmonofont{DejaVu Sans Mono}
%%
\begingroup%
\makeatletter%
\begin{pgfpicture}%
\pgfpathrectangle{\pgfpointorigin}{\pgfqpoint{5.441154in}{3.837682in}}%
\pgfusepath{use as bounding box, clip}%
\begin{pgfscope}%
\pgfsetbuttcap%
\pgfsetmiterjoin%
\definecolor{currentfill}{rgb}{1.000000,1.000000,1.000000}%
\pgfsetfillcolor{currentfill}%
\pgfsetlinewidth{0.000000pt}%
\definecolor{currentstroke}{rgb}{1.000000,1.000000,1.000000}%
\pgfsetstrokecolor{currentstroke}%
\pgfsetdash{}{0pt}%
\pgfpathmoveto{\pgfqpoint{0.000000in}{0.000000in}}%
\pgfpathlineto{\pgfqpoint{5.441154in}{0.000000in}}%
\pgfpathlineto{\pgfqpoint{5.441154in}{3.837682in}}%
\pgfpathlineto{\pgfqpoint{0.000000in}{3.837682in}}%
\pgfpathclose%
\pgfusepath{fill}%
\end{pgfscope}%
\begin{pgfscope}%
\pgfsetbuttcap%
\pgfsetmiterjoin%
\definecolor{currentfill}{rgb}{1.000000,1.000000,1.000000}%
\pgfsetfillcolor{currentfill}%
\pgfsetlinewidth{0.000000pt}%
\definecolor{currentstroke}{rgb}{0.000000,0.000000,0.000000}%
\pgfsetstrokecolor{currentstroke}%
\pgfsetstrokeopacity{0.000000}%
\pgfsetdash{}{0pt}%
\pgfpathmoveto{\pgfqpoint{0.703550in}{0.682682in}}%
\pgfpathlineto{\pgfqpoint{4.423550in}{0.682682in}}%
\pgfpathlineto{\pgfqpoint{4.423550in}{3.702682in}}%
\pgfpathlineto{\pgfqpoint{0.703550in}{3.702682in}}%
\pgfpathclose%
\pgfusepath{fill}%
\end{pgfscope}%
\begin{pgfscope}%
\pgfpathrectangle{\pgfqpoint{0.703550in}{0.682682in}}{\pgfqpoint{3.720000in}{3.020000in}} %
\pgfusepath{clip}%
\pgfsetbuttcap%
\pgfsetroundjoin%
\definecolor{currentfill}{rgb}{0.500000,0.000000,1.000000}%
\pgfsetfillcolor{currentfill}%
\pgfsetlinewidth{1.003750pt}%
\definecolor{currentstroke}{rgb}{0.500000,0.000000,1.000000}%
\pgfsetstrokecolor{currentstroke}%
\pgfsetdash{}{0pt}%
\pgfpathmoveto{\pgfqpoint{0.879821in}{0.917497in}}%
\pgfpathcurveto{\pgfqpoint{0.890871in}{0.917497in}}{\pgfqpoint{0.901470in}{0.921887in}}{\pgfqpoint{0.909283in}{0.929701in}}%
\pgfpathcurveto{\pgfqpoint{0.917097in}{0.937514in}}{\pgfqpoint{0.921487in}{0.948113in}}{\pgfqpoint{0.921487in}{0.959163in}}%
\pgfpathcurveto{\pgfqpoint{0.921487in}{0.970214in}}{\pgfqpoint{0.917097in}{0.980813in}}{\pgfqpoint{0.909283in}{0.988626in}}%
\pgfpathcurveto{\pgfqpoint{0.901470in}{0.996440in}}{\pgfqpoint{0.890871in}{1.000830in}}{\pgfqpoint{0.879821in}{1.000830in}}%
\pgfpathcurveto{\pgfqpoint{0.868770in}{1.000830in}}{\pgfqpoint{0.858171in}{0.996440in}}{\pgfqpoint{0.850358in}{0.988626in}}%
\pgfpathcurveto{\pgfqpoint{0.842544in}{0.980813in}}{\pgfqpoint{0.838154in}{0.970214in}}{\pgfqpoint{0.838154in}{0.959163in}}%
\pgfpathcurveto{\pgfqpoint{0.838154in}{0.948113in}}{\pgfqpoint{0.842544in}{0.937514in}}{\pgfqpoint{0.850358in}{0.929701in}}%
\pgfpathcurveto{\pgfqpoint{0.858171in}{0.921887in}}{\pgfqpoint{0.868770in}{0.917497in}}{\pgfqpoint{0.879821in}{0.917497in}}%
\pgfpathclose%
\pgfusepath{stroke,fill}%
\end{pgfscope}%
\begin{pgfscope}%
\pgfpathrectangle{\pgfqpoint{0.703550in}{0.682682in}}{\pgfqpoint{3.720000in}{3.020000in}} %
\pgfusepath{clip}%
\pgfsetbuttcap%
\pgfsetroundjoin%
\definecolor{currentfill}{rgb}{0.382353,0.183750,0.995734}%
\pgfsetfillcolor{currentfill}%
\pgfsetlinewidth{1.003750pt}%
\definecolor{currentstroke}{rgb}{0.382353,0.183750,0.995734}%
\pgfsetstrokecolor{currentstroke}%
\pgfsetdash{}{0pt}%
\pgfpathmoveto{\pgfqpoint{0.895094in}{1.056268in}}%
\pgfpathcurveto{\pgfqpoint{0.906144in}{1.056268in}}{\pgfqpoint{0.916743in}{1.060659in}}{\pgfqpoint{0.924557in}{1.068472in}}%
\pgfpathcurveto{\pgfqpoint{0.932371in}{1.076286in}}{\pgfqpoint{0.936761in}{1.086885in}}{\pgfqpoint{0.936761in}{1.097935in}}%
\pgfpathcurveto{\pgfqpoint{0.936761in}{1.108985in}}{\pgfqpoint{0.932371in}{1.119584in}}{\pgfqpoint{0.924557in}{1.127398in}}%
\pgfpathcurveto{\pgfqpoint{0.916743in}{1.135211in}}{\pgfqpoint{0.906144in}{1.139602in}}{\pgfqpoint{0.895094in}{1.139602in}}%
\pgfpathcurveto{\pgfqpoint{0.884044in}{1.139602in}}{\pgfqpoint{0.873445in}{1.135211in}}{\pgfqpoint{0.865632in}{1.127398in}}%
\pgfpathcurveto{\pgfqpoint{0.857818in}{1.119584in}}{\pgfqpoint{0.853428in}{1.108985in}}{\pgfqpoint{0.853428in}{1.097935in}}%
\pgfpathcurveto{\pgfqpoint{0.853428in}{1.086885in}}{\pgfqpoint{0.857818in}{1.076286in}}{\pgfqpoint{0.865632in}{1.068472in}}%
\pgfpathcurveto{\pgfqpoint{0.873445in}{1.060659in}}{\pgfqpoint{0.884044in}{1.056268in}}{\pgfqpoint{0.895094in}{1.056268in}}%
\pgfpathclose%
\pgfusepath{stroke,fill}%
\end{pgfscope}%
\begin{pgfscope}%
\pgfpathrectangle{\pgfqpoint{0.703550in}{0.682682in}}{\pgfqpoint{3.720000in}{3.020000in}} %
\pgfusepath{clip}%
\pgfsetbuttcap%
\pgfsetroundjoin%
\definecolor{currentfill}{rgb}{0.139216,0.536867,0.960122}%
\pgfsetfillcolor{currentfill}%
\pgfsetlinewidth{1.003750pt}%
\definecolor{currentstroke}{rgb}{0.139216,0.536867,0.960122}%
\pgfsetstrokecolor{currentstroke}%
\pgfsetdash{}{0pt}%
\pgfpathmoveto{\pgfqpoint{1.002684in}{1.338527in}}%
\pgfpathcurveto{\pgfqpoint{1.013735in}{1.338527in}}{\pgfqpoint{1.024334in}{1.342917in}}{\pgfqpoint{1.032147in}{1.350731in}}%
\pgfpathcurveto{\pgfqpoint{1.039961in}{1.358544in}}{\pgfqpoint{1.044351in}{1.369143in}}{\pgfqpoint{1.044351in}{1.380194in}}%
\pgfpathcurveto{\pgfqpoint{1.044351in}{1.391244in}}{\pgfqpoint{1.039961in}{1.401843in}}{\pgfqpoint{1.032147in}{1.409656in}}%
\pgfpathcurveto{\pgfqpoint{1.024334in}{1.417470in}}{\pgfqpoint{1.013735in}{1.421860in}}{\pgfqpoint{1.002684in}{1.421860in}}%
\pgfpathcurveto{\pgfqpoint{0.991634in}{1.421860in}}{\pgfqpoint{0.981035in}{1.417470in}}{\pgfqpoint{0.973222in}{1.409656in}}%
\pgfpathcurveto{\pgfqpoint{0.965408in}{1.401843in}}{\pgfqpoint{0.961018in}{1.391244in}}{\pgfqpoint{0.961018in}{1.380194in}}%
\pgfpathcurveto{\pgfqpoint{0.961018in}{1.369143in}}{\pgfqpoint{0.965408in}{1.358544in}}{\pgfqpoint{0.973222in}{1.350731in}}%
\pgfpathcurveto{\pgfqpoint{0.981035in}{1.342917in}}{\pgfqpoint{0.991634in}{1.338527in}}{\pgfqpoint{1.002684in}{1.338527in}}%
\pgfpathclose%
\pgfusepath{stroke,fill}%
\end{pgfscope}%
\begin{pgfscope}%
\pgfpathrectangle{\pgfqpoint{0.703550in}{0.682682in}}{\pgfqpoint{3.720000in}{3.020000in}} %
\pgfusepath{clip}%
\pgfsetbuttcap%
\pgfsetroundjoin%
\definecolor{currentfill}{rgb}{0.017647,0.726434,0.918487}%
\pgfsetfillcolor{currentfill}%
\pgfsetlinewidth{1.003750pt}%
\definecolor{currentstroke}{rgb}{0.017647,0.726434,0.918487}%
\pgfsetstrokecolor{currentstroke}%
\pgfsetdash{}{0pt}%
\pgfpathmoveto{\pgfqpoint{0.992774in}{1.529372in}}%
\pgfpathcurveto{\pgfqpoint{1.003824in}{1.529372in}}{\pgfqpoint{1.014423in}{1.533762in}}{\pgfqpoint{1.022236in}{1.541576in}}%
\pgfpathcurveto{\pgfqpoint{1.030050in}{1.549389in}}{\pgfqpoint{1.034440in}{1.559988in}}{\pgfqpoint{1.034440in}{1.571038in}}%
\pgfpathcurveto{\pgfqpoint{1.034440in}{1.582089in}}{\pgfqpoint{1.030050in}{1.592688in}}{\pgfqpoint{1.022236in}{1.600501in}}%
\pgfpathcurveto{\pgfqpoint{1.014423in}{1.608315in}}{\pgfqpoint{1.003824in}{1.612705in}}{\pgfqpoint{0.992774in}{1.612705in}}%
\pgfpathcurveto{\pgfqpoint{0.981723in}{1.612705in}}{\pgfqpoint{0.971124in}{1.608315in}}{\pgfqpoint{0.963311in}{1.600501in}}%
\pgfpathcurveto{\pgfqpoint{0.955497in}{1.592688in}}{\pgfqpoint{0.951107in}{1.582089in}}{\pgfqpoint{0.951107in}{1.571038in}}%
\pgfpathcurveto{\pgfqpoint{0.951107in}{1.559988in}}{\pgfqpoint{0.955497in}{1.549389in}}{\pgfqpoint{0.963311in}{1.541576in}}%
\pgfpathcurveto{\pgfqpoint{0.971124in}{1.533762in}}{\pgfqpoint{0.981723in}{1.529372in}}{\pgfqpoint{0.992774in}{1.529372in}}%
\pgfpathclose%
\pgfusepath{stroke,fill}%
\end{pgfscope}%
\begin{pgfscope}%
\pgfpathrectangle{\pgfqpoint{0.703550in}{0.682682in}}{\pgfqpoint{3.720000in}{3.020000in}} %
\pgfusepath{clip}%
\pgfsetbuttcap%
\pgfsetroundjoin%
\definecolor{currentfill}{rgb}{0.029412,0.673696,0.932472}%
\pgfsetfillcolor{currentfill}%
\pgfsetlinewidth{1.003750pt}%
\definecolor{currentstroke}{rgb}{0.029412,0.673696,0.932472}%
\pgfsetstrokecolor{currentstroke}%
\pgfsetdash{}{0pt}%
\pgfpathmoveto{\pgfqpoint{1.198812in}{1.466812in}}%
\pgfpathcurveto{\pgfqpoint{1.209862in}{1.466812in}}{\pgfqpoint{1.220461in}{1.471202in}}{\pgfqpoint{1.228275in}{1.479015in}}%
\pgfpathcurveto{\pgfqpoint{1.236088in}{1.486829in}}{\pgfqpoint{1.240479in}{1.497428in}}{\pgfqpoint{1.240479in}{1.508478in}}%
\pgfpathcurveto{\pgfqpoint{1.240479in}{1.519528in}}{\pgfqpoint{1.236088in}{1.530127in}}{\pgfqpoint{1.228275in}{1.537941in}}%
\pgfpathcurveto{\pgfqpoint{1.220461in}{1.545755in}}{\pgfqpoint{1.209862in}{1.550145in}}{\pgfqpoint{1.198812in}{1.550145in}}%
\pgfpathcurveto{\pgfqpoint{1.187762in}{1.550145in}}{\pgfqpoint{1.177163in}{1.545755in}}{\pgfqpoint{1.169349in}{1.537941in}}%
\pgfpathcurveto{\pgfqpoint{1.161536in}{1.530127in}}{\pgfqpoint{1.157145in}{1.519528in}}{\pgfqpoint{1.157145in}{1.508478in}}%
\pgfpathcurveto{\pgfqpoint{1.157145in}{1.497428in}}{\pgfqpoint{1.161536in}{1.486829in}}{\pgfqpoint{1.169349in}{1.479015in}}%
\pgfpathcurveto{\pgfqpoint{1.177163in}{1.471202in}}{\pgfqpoint{1.187762in}{1.466812in}}{\pgfqpoint{1.198812in}{1.466812in}}%
\pgfpathclose%
\pgfusepath{stroke,fill}%
\end{pgfscope}%
\begin{pgfscope}%
\pgfpathrectangle{\pgfqpoint{0.703550in}{0.682682in}}{\pgfqpoint{3.720000in}{3.020000in}} %
\pgfusepath{clip}%
\pgfsetbuttcap%
\pgfsetroundjoin%
\definecolor{currentfill}{rgb}{0.115686,0.567675,0.954791}%
\pgfsetfillcolor{currentfill}%
\pgfsetlinewidth{1.003750pt}%
\definecolor{currentstroke}{rgb}{0.115686,0.567675,0.954791}%
\pgfsetstrokecolor{currentstroke}%
\pgfsetdash{}{0pt}%
\pgfpathmoveto{\pgfqpoint{1.354878in}{1.368286in}}%
\pgfpathcurveto{\pgfqpoint{1.365928in}{1.368286in}}{\pgfqpoint{1.376527in}{1.372676in}}{\pgfqpoint{1.384341in}{1.380490in}}%
\pgfpathcurveto{\pgfqpoint{1.392154in}{1.388303in}}{\pgfqpoint{1.396545in}{1.398902in}}{\pgfqpoint{1.396545in}{1.409952in}}%
\pgfpathcurveto{\pgfqpoint{1.396545in}{1.421003in}}{\pgfqpoint{1.392154in}{1.431602in}}{\pgfqpoint{1.384341in}{1.439415in}}%
\pgfpathcurveto{\pgfqpoint{1.376527in}{1.447229in}}{\pgfqpoint{1.365928in}{1.451619in}}{\pgfqpoint{1.354878in}{1.451619in}}%
\pgfpathcurveto{\pgfqpoint{1.343828in}{1.451619in}}{\pgfqpoint{1.333229in}{1.447229in}}{\pgfqpoint{1.325415in}{1.439415in}}%
\pgfpathcurveto{\pgfqpoint{1.317601in}{1.431602in}}{\pgfqpoint{1.313211in}{1.421003in}}{\pgfqpoint{1.313211in}{1.409952in}}%
\pgfpathcurveto{\pgfqpoint{1.313211in}{1.398902in}}{\pgfqpoint{1.317601in}{1.388303in}}{\pgfqpoint{1.325415in}{1.380490in}}%
\pgfpathcurveto{\pgfqpoint{1.333229in}{1.372676in}}{\pgfqpoint{1.343828in}{1.368286in}}{\pgfqpoint{1.354878in}{1.368286in}}%
\pgfpathclose%
\pgfusepath{stroke,fill}%
\end{pgfscope}%
\begin{pgfscope}%
\pgfpathrectangle{\pgfqpoint{0.703550in}{0.682682in}}{\pgfqpoint{3.720000in}{3.020000in}} %
\pgfusepath{clip}%
\pgfsetbuttcap%
\pgfsetroundjoin%
\definecolor{currentfill}{rgb}{0.401961,0.988165,0.759405}%
\pgfsetfillcolor{currentfill}%
\pgfsetlinewidth{1.003750pt}%
\definecolor{currentstroke}{rgb}{0.401961,0.988165,0.759405}%
\pgfsetstrokecolor{currentstroke}%
\pgfsetdash{}{0pt}%
\pgfpathmoveto{\pgfqpoint{1.510907in}{1.991302in}}%
\pgfpathcurveto{\pgfqpoint{1.521957in}{1.991302in}}{\pgfqpoint{1.532556in}{1.995692in}}{\pgfqpoint{1.540370in}{2.003506in}}%
\pgfpathcurveto{\pgfqpoint{1.548183in}{2.011319in}}{\pgfqpoint{1.552574in}{2.021918in}}{\pgfqpoint{1.552574in}{2.032969in}}%
\pgfpathcurveto{\pgfqpoint{1.552574in}{2.044019in}}{\pgfqpoint{1.548183in}{2.054618in}}{\pgfqpoint{1.540370in}{2.062431in}}%
\pgfpathcurveto{\pgfqpoint{1.532556in}{2.070245in}}{\pgfqpoint{1.521957in}{2.074635in}}{\pgfqpoint{1.510907in}{2.074635in}}%
\pgfpathcurveto{\pgfqpoint{1.499857in}{2.074635in}}{\pgfqpoint{1.489258in}{2.070245in}}{\pgfqpoint{1.481444in}{2.062431in}}%
\pgfpathcurveto{\pgfqpoint{1.473631in}{2.054618in}}{\pgfqpoint{1.469240in}{2.044019in}}{\pgfqpoint{1.469240in}{2.032969in}}%
\pgfpathcurveto{\pgfqpoint{1.469240in}{2.021918in}}{\pgfqpoint{1.473631in}{2.011319in}}{\pgfqpoint{1.481444in}{2.003506in}}%
\pgfpathcurveto{\pgfqpoint{1.489258in}{1.995692in}}{\pgfqpoint{1.499857in}{1.991302in}}{\pgfqpoint{1.510907in}{1.991302in}}%
\pgfpathclose%
\pgfusepath{stroke,fill}%
\end{pgfscope}%
\begin{pgfscope}%
\pgfpathrectangle{\pgfqpoint{0.703550in}{0.682682in}}{\pgfqpoint{3.720000in}{3.020000in}} %
\pgfusepath{clip}%
\pgfsetbuttcap%
\pgfsetroundjoin%
\definecolor{currentfill}{rgb}{0.598039,0.988165,0.650618}%
\pgfsetfillcolor{currentfill}%
\pgfsetlinewidth{1.003750pt}%
\definecolor{currentstroke}{rgb}{0.598039,0.988165,0.650618}%
\pgfsetstrokecolor{currentstroke}%
\pgfsetdash{}{0pt}%
\pgfpathmoveto{\pgfqpoint{4.078197in}{2.233534in}}%
\pgfpathcurveto{\pgfqpoint{4.089247in}{2.233534in}}{\pgfqpoint{4.099846in}{2.237924in}}{\pgfqpoint{4.107660in}{2.245737in}}%
\pgfpathcurveto{\pgfqpoint{4.115474in}{2.253551in}}{\pgfqpoint{4.119864in}{2.264150in}}{\pgfqpoint{4.119864in}{2.275200in}}%
\pgfpathcurveto{\pgfqpoint{4.119864in}{2.286250in}}{\pgfqpoint{4.115474in}{2.296849in}}{\pgfqpoint{4.107660in}{2.304663in}}%
\pgfpathcurveto{\pgfqpoint{4.099846in}{2.312477in}}{\pgfqpoint{4.089247in}{2.316867in}}{\pgfqpoint{4.078197in}{2.316867in}}%
\pgfpathcurveto{\pgfqpoint{4.067147in}{2.316867in}}{\pgfqpoint{4.056548in}{2.312477in}}{\pgfqpoint{4.048734in}{2.304663in}}%
\pgfpathcurveto{\pgfqpoint{4.040921in}{2.296849in}}{\pgfqpoint{4.036530in}{2.286250in}}{\pgfqpoint{4.036530in}{2.275200in}}%
\pgfpathcurveto{\pgfqpoint{4.036530in}{2.264150in}}{\pgfqpoint{4.040921in}{2.253551in}}{\pgfqpoint{4.048734in}{2.245737in}}%
\pgfpathcurveto{\pgfqpoint{4.056548in}{2.237924in}}{\pgfqpoint{4.067147in}{2.233534in}}{\pgfqpoint{4.078197in}{2.233534in}}%
\pgfpathclose%
\pgfusepath{stroke,fill}%
\end{pgfscope}%
\begin{pgfscope}%
\pgfpathrectangle{\pgfqpoint{0.703550in}{0.682682in}}{\pgfqpoint{3.720000in}{3.020000in}} %
\pgfusepath{clip}%
\pgfsetbuttcap%
\pgfsetroundjoin%
\definecolor{currentfill}{rgb}{1.000000,0.636474,0.338158}%
\pgfsetfillcolor{currentfill}%
\pgfsetlinewidth{1.003750pt}%
\definecolor{currentstroke}{rgb}{1.000000,0.636474,0.338158}%
\pgfsetstrokecolor{currentstroke}%
\pgfsetdash{}{0pt}%
\pgfpathmoveto{\pgfqpoint{4.247279in}{2.814463in}}%
\pgfpathcurveto{\pgfqpoint{4.258329in}{2.814463in}}{\pgfqpoint{4.268928in}{2.818853in}}{\pgfqpoint{4.276741in}{2.826667in}}%
\pgfpathcurveto{\pgfqpoint{4.284555in}{2.834481in}}{\pgfqpoint{4.288945in}{2.845080in}}{\pgfqpoint{4.288945in}{2.856130in}}%
\pgfpathcurveto{\pgfqpoint{4.288945in}{2.867180in}}{\pgfqpoint{4.284555in}{2.877779in}}{\pgfqpoint{4.276741in}{2.885593in}}%
\pgfpathcurveto{\pgfqpoint{4.268928in}{2.893406in}}{\pgfqpoint{4.258329in}{2.897796in}}{\pgfqpoint{4.247279in}{2.897796in}}%
\pgfpathcurveto{\pgfqpoint{4.236228in}{2.897796in}}{\pgfqpoint{4.225629in}{2.893406in}}{\pgfqpoint{4.217816in}{2.885593in}}%
\pgfpathcurveto{\pgfqpoint{4.210002in}{2.877779in}}{\pgfqpoint{4.205612in}{2.867180in}}{\pgfqpoint{4.205612in}{2.856130in}}%
\pgfpathcurveto{\pgfqpoint{4.205612in}{2.845080in}}{\pgfqpoint{4.210002in}{2.834481in}}{\pgfqpoint{4.217816in}{2.826667in}}%
\pgfpathcurveto{\pgfqpoint{4.225629in}{2.818853in}}{\pgfqpoint{4.236228in}{2.814463in}}{\pgfqpoint{4.247279in}{2.814463in}}%
\pgfpathclose%
\pgfusepath{stroke,fill}%
\end{pgfscope}%
\begin{pgfscope}%
\pgfpathrectangle{\pgfqpoint{0.703550in}{0.682682in}}{\pgfqpoint{3.720000in}{3.020000in}} %
\pgfusepath{clip}%
\pgfsetbuttcap%
\pgfsetroundjoin%
\definecolor{currentfill}{rgb}{1.000000,0.000000,0.000000}%
\pgfsetfillcolor{currentfill}%
\pgfsetlinewidth{1.003750pt}%
\definecolor{currentstroke}{rgb}{1.000000,0.000000,0.000000}%
\pgfsetstrokecolor{currentstroke}%
\pgfsetdash{}{0pt}%
\pgfpathmoveto{\pgfqpoint{3.440031in}{3.384533in}}%
\pgfpathcurveto{\pgfqpoint{3.451081in}{3.384533in}}{\pgfqpoint{3.461680in}{3.388923in}}{\pgfqpoint{3.469494in}{3.396737in}}%
\pgfpathcurveto{\pgfqpoint{3.477307in}{3.404551in}}{\pgfqpoint{3.481697in}{3.415150in}}{\pgfqpoint{3.481697in}{3.426200in}}%
\pgfpathcurveto{\pgfqpoint{3.481697in}{3.437250in}}{\pgfqpoint{3.477307in}{3.447849in}}{\pgfqpoint{3.469494in}{3.455663in}}%
\pgfpathcurveto{\pgfqpoint{3.461680in}{3.463476in}}{\pgfqpoint{3.451081in}{3.467867in}}{\pgfqpoint{3.440031in}{3.467867in}}%
\pgfpathcurveto{\pgfqpoint{3.428981in}{3.467867in}}{\pgfqpoint{3.418382in}{3.463476in}}{\pgfqpoint{3.410568in}{3.455663in}}%
\pgfpathcurveto{\pgfqpoint{3.402754in}{3.447849in}}{\pgfqpoint{3.398364in}{3.437250in}}{\pgfqpoint{3.398364in}{3.426200in}}%
\pgfpathcurveto{\pgfqpoint{3.398364in}{3.415150in}}{\pgfqpoint{3.402754in}{3.404551in}}{\pgfqpoint{3.410568in}{3.396737in}}%
\pgfpathcurveto{\pgfqpoint{3.418382in}{3.388923in}}{\pgfqpoint{3.428981in}{3.384533in}}{\pgfqpoint{3.440031in}{3.384533in}}%
\pgfpathclose%
\pgfusepath{stroke,fill}%
\end{pgfscope}%
\begin{pgfscope}%
\pgfsetbuttcap%
\pgfsetroundjoin%
\definecolor{currentfill}{rgb}{0.000000,0.000000,0.000000}%
\pgfsetfillcolor{currentfill}%
\pgfsetlinewidth{0.803000pt}%
\definecolor{currentstroke}{rgb}{0.000000,0.000000,0.000000}%
\pgfsetstrokecolor{currentstroke}%
\pgfsetdash{}{0pt}%
\pgfsys@defobject{currentmarker}{\pgfqpoint{0.000000in}{-0.048611in}}{\pgfqpoint{0.000000in}{0.000000in}}{%
\pgfpathmoveto{\pgfqpoint{0.000000in}{0.000000in}}%
\pgfpathlineto{\pgfqpoint{0.000000in}{-0.048611in}}%
\pgfusepath{stroke,fill}%
}%
\begin{pgfscope}%
\pgfsys@transformshift{0.856675in}{0.682682in}%
\pgfsys@useobject{currentmarker}{}%
\end{pgfscope}%
\end{pgfscope}%
\begin{pgfscope}%
\pgftext[x=0.856675in,y=0.585459in,,top]{\rmfamily\fontsize{10.000000}{12.000000}\selectfont \(\displaystyle 0.0\)}%
\end{pgfscope}%
\begin{pgfscope}%
\pgfsetbuttcap%
\pgfsetroundjoin%
\definecolor{currentfill}{rgb}{0.000000,0.000000,0.000000}%
\pgfsetfillcolor{currentfill}%
\pgfsetlinewidth{0.803000pt}%
\definecolor{currentstroke}{rgb}{0.000000,0.000000,0.000000}%
\pgfsetstrokecolor{currentstroke}%
\pgfsetdash{}{0pt}%
\pgfsys@defobject{currentmarker}{\pgfqpoint{0.000000in}{-0.048611in}}{\pgfqpoint{0.000000in}{0.000000in}}{%
\pgfpathmoveto{\pgfqpoint{0.000000in}{0.000000in}}%
\pgfpathlineto{\pgfqpoint{0.000000in}{-0.048611in}}%
\pgfusepath{stroke,fill}%
}%
\begin{pgfscope}%
\pgfsys@transformshift{1.413131in}{0.682682in}%
\pgfsys@useobject{currentmarker}{}%
\end{pgfscope}%
\end{pgfscope}%
\begin{pgfscope}%
\pgftext[x=1.413131in,y=0.585459in,,top]{\rmfamily\fontsize{10.000000}{12.000000}\selectfont \(\displaystyle 0.5\)}%
\end{pgfscope}%
\begin{pgfscope}%
\pgfsetbuttcap%
\pgfsetroundjoin%
\definecolor{currentfill}{rgb}{0.000000,0.000000,0.000000}%
\pgfsetfillcolor{currentfill}%
\pgfsetlinewidth{0.803000pt}%
\definecolor{currentstroke}{rgb}{0.000000,0.000000,0.000000}%
\pgfsetstrokecolor{currentstroke}%
\pgfsetdash{}{0pt}%
\pgfsys@defobject{currentmarker}{\pgfqpoint{0.000000in}{-0.048611in}}{\pgfqpoint{0.000000in}{0.000000in}}{%
\pgfpathmoveto{\pgfqpoint{0.000000in}{0.000000in}}%
\pgfpathlineto{\pgfqpoint{0.000000in}{-0.048611in}}%
\pgfusepath{stroke,fill}%
}%
\begin{pgfscope}%
\pgfsys@transformshift{1.969587in}{0.682682in}%
\pgfsys@useobject{currentmarker}{}%
\end{pgfscope}%
\end{pgfscope}%
\begin{pgfscope}%
\pgftext[x=1.969587in,y=0.585459in,,top]{\rmfamily\fontsize{10.000000}{12.000000}\selectfont \(\displaystyle 1.0\)}%
\end{pgfscope}%
\begin{pgfscope}%
\pgfsetbuttcap%
\pgfsetroundjoin%
\definecolor{currentfill}{rgb}{0.000000,0.000000,0.000000}%
\pgfsetfillcolor{currentfill}%
\pgfsetlinewidth{0.803000pt}%
\definecolor{currentstroke}{rgb}{0.000000,0.000000,0.000000}%
\pgfsetstrokecolor{currentstroke}%
\pgfsetdash{}{0pt}%
\pgfsys@defobject{currentmarker}{\pgfqpoint{0.000000in}{-0.048611in}}{\pgfqpoint{0.000000in}{0.000000in}}{%
\pgfpathmoveto{\pgfqpoint{0.000000in}{0.000000in}}%
\pgfpathlineto{\pgfqpoint{0.000000in}{-0.048611in}}%
\pgfusepath{stroke,fill}%
}%
\begin{pgfscope}%
\pgfsys@transformshift{2.526043in}{0.682682in}%
\pgfsys@useobject{currentmarker}{}%
\end{pgfscope}%
\end{pgfscope}%
\begin{pgfscope}%
\pgftext[x=2.526043in,y=0.585459in,,top]{\rmfamily\fontsize{10.000000}{12.000000}\selectfont \(\displaystyle 1.5\)}%
\end{pgfscope}%
\begin{pgfscope}%
\pgfsetbuttcap%
\pgfsetroundjoin%
\definecolor{currentfill}{rgb}{0.000000,0.000000,0.000000}%
\pgfsetfillcolor{currentfill}%
\pgfsetlinewidth{0.803000pt}%
\definecolor{currentstroke}{rgb}{0.000000,0.000000,0.000000}%
\pgfsetstrokecolor{currentstroke}%
\pgfsetdash{}{0pt}%
\pgfsys@defobject{currentmarker}{\pgfqpoint{0.000000in}{-0.048611in}}{\pgfqpoint{0.000000in}{0.000000in}}{%
\pgfpathmoveto{\pgfqpoint{0.000000in}{0.000000in}}%
\pgfpathlineto{\pgfqpoint{0.000000in}{-0.048611in}}%
\pgfusepath{stroke,fill}%
}%
\begin{pgfscope}%
\pgfsys@transformshift{3.082499in}{0.682682in}%
\pgfsys@useobject{currentmarker}{}%
\end{pgfscope}%
\end{pgfscope}%
\begin{pgfscope}%
\pgftext[x=3.082499in,y=0.585459in,,top]{\rmfamily\fontsize{10.000000}{12.000000}\selectfont \(\displaystyle 2.0\)}%
\end{pgfscope}%
\begin{pgfscope}%
\pgfsetbuttcap%
\pgfsetroundjoin%
\definecolor{currentfill}{rgb}{0.000000,0.000000,0.000000}%
\pgfsetfillcolor{currentfill}%
\pgfsetlinewidth{0.803000pt}%
\definecolor{currentstroke}{rgb}{0.000000,0.000000,0.000000}%
\pgfsetstrokecolor{currentstroke}%
\pgfsetdash{}{0pt}%
\pgfsys@defobject{currentmarker}{\pgfqpoint{0.000000in}{-0.048611in}}{\pgfqpoint{0.000000in}{0.000000in}}{%
\pgfpathmoveto{\pgfqpoint{0.000000in}{0.000000in}}%
\pgfpathlineto{\pgfqpoint{0.000000in}{-0.048611in}}%
\pgfusepath{stroke,fill}%
}%
\begin{pgfscope}%
\pgfsys@transformshift{3.638955in}{0.682682in}%
\pgfsys@useobject{currentmarker}{}%
\end{pgfscope}%
\end{pgfscope}%
\begin{pgfscope}%
\pgftext[x=3.638955in,y=0.585459in,,top]{\rmfamily\fontsize{10.000000}{12.000000}\selectfont \(\displaystyle 2.5\)}%
\end{pgfscope}%
\begin{pgfscope}%
\pgfsetbuttcap%
\pgfsetroundjoin%
\definecolor{currentfill}{rgb}{0.000000,0.000000,0.000000}%
\pgfsetfillcolor{currentfill}%
\pgfsetlinewidth{0.803000pt}%
\definecolor{currentstroke}{rgb}{0.000000,0.000000,0.000000}%
\pgfsetstrokecolor{currentstroke}%
\pgfsetdash{}{0pt}%
\pgfsys@defobject{currentmarker}{\pgfqpoint{0.000000in}{-0.048611in}}{\pgfqpoint{0.000000in}{0.000000in}}{%
\pgfpathmoveto{\pgfqpoint{0.000000in}{0.000000in}}%
\pgfpathlineto{\pgfqpoint{0.000000in}{-0.048611in}}%
\pgfusepath{stroke,fill}%
}%
\begin{pgfscope}%
\pgfsys@transformshift{4.195411in}{0.682682in}%
\pgfsys@useobject{currentmarker}{}%
\end{pgfscope}%
\end{pgfscope}%
\begin{pgfscope}%
\pgftext[x=4.195411in,y=0.585459in,,top]{\rmfamily\fontsize{10.000000}{12.000000}\selectfont \(\displaystyle 3.0\)}%
\end{pgfscope}%
\begin{pgfscope}%
\pgftext[x=2.563550in,y=0.395491in,,top]{\rmfamily\fontsize{10.000000}{12.000000}\selectfont \(\displaystyle \frac{t_b}{t_c}\)}%
\end{pgfscope}%
\begin{pgfscope}%
\pgfsetbuttcap%
\pgfsetroundjoin%
\definecolor{currentfill}{rgb}{0.000000,0.000000,0.000000}%
\pgfsetfillcolor{currentfill}%
\pgfsetlinewidth{0.803000pt}%
\definecolor{currentstroke}{rgb}{0.000000,0.000000,0.000000}%
\pgfsetstrokecolor{currentstroke}%
\pgfsetdash{}{0pt}%
\pgfsys@defobject{currentmarker}{\pgfqpoint{-0.048611in}{0.000000in}}{\pgfqpoint{0.000000in}{0.000000in}}{%
\pgfpathmoveto{\pgfqpoint{0.000000in}{0.000000in}}%
\pgfpathlineto{\pgfqpoint{-0.048611in}{0.000000in}}%
\pgfusepath{stroke,fill}%
}%
\begin{pgfscope}%
\pgfsys@transformshift{0.703550in}{0.863819in}%
\pgfsys@useobject{currentmarker}{}%
\end{pgfscope}%
\end{pgfscope}%
\begin{pgfscope}%
\pgftext[x=0.289968in,y=0.811057in,left,base]{\rmfamily\fontsize{10.000000}{12.000000}\selectfont \(\displaystyle 2.000\)}%
\end{pgfscope}%
\begin{pgfscope}%
\pgfsetbuttcap%
\pgfsetroundjoin%
\definecolor{currentfill}{rgb}{0.000000,0.000000,0.000000}%
\pgfsetfillcolor{currentfill}%
\pgfsetlinewidth{0.803000pt}%
\definecolor{currentstroke}{rgb}{0.000000,0.000000,0.000000}%
\pgfsetstrokecolor{currentstroke}%
\pgfsetdash{}{0pt}%
\pgfsys@defobject{currentmarker}{\pgfqpoint{-0.048611in}{0.000000in}}{\pgfqpoint{0.000000in}{0.000000in}}{%
\pgfpathmoveto{\pgfqpoint{0.000000in}{0.000000in}}%
\pgfpathlineto{\pgfqpoint{-0.048611in}{0.000000in}}%
\pgfusepath{stroke,fill}%
}%
\begin{pgfscope}%
\pgfsys@transformshift{0.703550in}{1.214162in}%
\pgfsys@useobject{currentmarker}{}%
\end{pgfscope}%
\end{pgfscope}%
\begin{pgfscope}%
\pgftext[x=0.289968in,y=1.161400in,left,base]{\rmfamily\fontsize{10.000000}{12.000000}\selectfont \(\displaystyle 2.025\)}%
\end{pgfscope}%
\begin{pgfscope}%
\pgfsetbuttcap%
\pgfsetroundjoin%
\definecolor{currentfill}{rgb}{0.000000,0.000000,0.000000}%
\pgfsetfillcolor{currentfill}%
\pgfsetlinewidth{0.803000pt}%
\definecolor{currentstroke}{rgb}{0.000000,0.000000,0.000000}%
\pgfsetstrokecolor{currentstroke}%
\pgfsetdash{}{0pt}%
\pgfsys@defobject{currentmarker}{\pgfqpoint{-0.048611in}{0.000000in}}{\pgfqpoint{0.000000in}{0.000000in}}{%
\pgfpathmoveto{\pgfqpoint{0.000000in}{0.000000in}}%
\pgfpathlineto{\pgfqpoint{-0.048611in}{0.000000in}}%
\pgfusepath{stroke,fill}%
}%
\begin{pgfscope}%
\pgfsys@transformshift{0.703550in}{1.564505in}%
\pgfsys@useobject{currentmarker}{}%
\end{pgfscope}%
\end{pgfscope}%
\begin{pgfscope}%
\pgftext[x=0.289968in,y=1.511743in,left,base]{\rmfamily\fontsize{10.000000}{12.000000}\selectfont \(\displaystyle 2.050\)}%
\end{pgfscope}%
\begin{pgfscope}%
\pgfsetbuttcap%
\pgfsetroundjoin%
\definecolor{currentfill}{rgb}{0.000000,0.000000,0.000000}%
\pgfsetfillcolor{currentfill}%
\pgfsetlinewidth{0.803000pt}%
\definecolor{currentstroke}{rgb}{0.000000,0.000000,0.000000}%
\pgfsetstrokecolor{currentstroke}%
\pgfsetdash{}{0pt}%
\pgfsys@defobject{currentmarker}{\pgfqpoint{-0.048611in}{0.000000in}}{\pgfqpoint{0.000000in}{0.000000in}}{%
\pgfpathmoveto{\pgfqpoint{0.000000in}{0.000000in}}%
\pgfpathlineto{\pgfqpoint{-0.048611in}{0.000000in}}%
\pgfusepath{stroke,fill}%
}%
\begin{pgfscope}%
\pgfsys@transformshift{0.703550in}{1.914848in}%
\pgfsys@useobject{currentmarker}{}%
\end{pgfscope}%
\end{pgfscope}%
\begin{pgfscope}%
\pgftext[x=0.289968in,y=1.862086in,left,base]{\rmfamily\fontsize{10.000000}{12.000000}\selectfont \(\displaystyle 2.075\)}%
\end{pgfscope}%
\begin{pgfscope}%
\pgfsetbuttcap%
\pgfsetroundjoin%
\definecolor{currentfill}{rgb}{0.000000,0.000000,0.000000}%
\pgfsetfillcolor{currentfill}%
\pgfsetlinewidth{0.803000pt}%
\definecolor{currentstroke}{rgb}{0.000000,0.000000,0.000000}%
\pgfsetstrokecolor{currentstroke}%
\pgfsetdash{}{0pt}%
\pgfsys@defobject{currentmarker}{\pgfqpoint{-0.048611in}{0.000000in}}{\pgfqpoint{0.000000in}{0.000000in}}{%
\pgfpathmoveto{\pgfqpoint{0.000000in}{0.000000in}}%
\pgfpathlineto{\pgfqpoint{-0.048611in}{0.000000in}}%
\pgfusepath{stroke,fill}%
}%
\begin{pgfscope}%
\pgfsys@transformshift{0.703550in}{2.265191in}%
\pgfsys@useobject{currentmarker}{}%
\end{pgfscope}%
\end{pgfscope}%
\begin{pgfscope}%
\pgftext[x=0.289968in,y=2.212429in,left,base]{\rmfamily\fontsize{10.000000}{12.000000}\selectfont \(\displaystyle 2.100\)}%
\end{pgfscope}%
\begin{pgfscope}%
\pgfsetbuttcap%
\pgfsetroundjoin%
\definecolor{currentfill}{rgb}{0.000000,0.000000,0.000000}%
\pgfsetfillcolor{currentfill}%
\pgfsetlinewidth{0.803000pt}%
\definecolor{currentstroke}{rgb}{0.000000,0.000000,0.000000}%
\pgfsetstrokecolor{currentstroke}%
\pgfsetdash{}{0pt}%
\pgfsys@defobject{currentmarker}{\pgfqpoint{-0.048611in}{0.000000in}}{\pgfqpoint{0.000000in}{0.000000in}}{%
\pgfpathmoveto{\pgfqpoint{0.000000in}{0.000000in}}%
\pgfpathlineto{\pgfqpoint{-0.048611in}{0.000000in}}%
\pgfusepath{stroke,fill}%
}%
\begin{pgfscope}%
\pgfsys@transformshift{0.703550in}{2.615533in}%
\pgfsys@useobject{currentmarker}{}%
\end{pgfscope}%
\end{pgfscope}%
\begin{pgfscope}%
\pgftext[x=0.289968in,y=2.562772in,left,base]{\rmfamily\fontsize{10.000000}{12.000000}\selectfont \(\displaystyle 2.125\)}%
\end{pgfscope}%
\begin{pgfscope}%
\pgfsetbuttcap%
\pgfsetroundjoin%
\definecolor{currentfill}{rgb}{0.000000,0.000000,0.000000}%
\pgfsetfillcolor{currentfill}%
\pgfsetlinewidth{0.803000pt}%
\definecolor{currentstroke}{rgb}{0.000000,0.000000,0.000000}%
\pgfsetstrokecolor{currentstroke}%
\pgfsetdash{}{0pt}%
\pgfsys@defobject{currentmarker}{\pgfqpoint{-0.048611in}{0.000000in}}{\pgfqpoint{0.000000in}{0.000000in}}{%
\pgfpathmoveto{\pgfqpoint{0.000000in}{0.000000in}}%
\pgfpathlineto{\pgfqpoint{-0.048611in}{0.000000in}}%
\pgfusepath{stroke,fill}%
}%
\begin{pgfscope}%
\pgfsys@transformshift{0.703550in}{2.965876in}%
\pgfsys@useobject{currentmarker}{}%
\end{pgfscope}%
\end{pgfscope}%
\begin{pgfscope}%
\pgftext[x=0.289968in,y=2.913115in,left,base]{\rmfamily\fontsize{10.000000}{12.000000}\selectfont \(\displaystyle 2.150\)}%
\end{pgfscope}%
\begin{pgfscope}%
\pgfsetbuttcap%
\pgfsetroundjoin%
\definecolor{currentfill}{rgb}{0.000000,0.000000,0.000000}%
\pgfsetfillcolor{currentfill}%
\pgfsetlinewidth{0.803000pt}%
\definecolor{currentstroke}{rgb}{0.000000,0.000000,0.000000}%
\pgfsetstrokecolor{currentstroke}%
\pgfsetdash{}{0pt}%
\pgfsys@defobject{currentmarker}{\pgfqpoint{-0.048611in}{0.000000in}}{\pgfqpoint{0.000000in}{0.000000in}}{%
\pgfpathmoveto{\pgfqpoint{0.000000in}{0.000000in}}%
\pgfpathlineto{\pgfqpoint{-0.048611in}{0.000000in}}%
\pgfusepath{stroke,fill}%
}%
\begin{pgfscope}%
\pgfsys@transformshift{0.703550in}{3.316219in}%
\pgfsys@useobject{currentmarker}{}%
\end{pgfscope}%
\end{pgfscope}%
\begin{pgfscope}%
\pgftext[x=0.289968in,y=3.263458in,left,base]{\rmfamily\fontsize{10.000000}{12.000000}\selectfont \(\displaystyle 2.175\)}%
\end{pgfscope}%
\begin{pgfscope}%
\pgfsetbuttcap%
\pgfsetroundjoin%
\definecolor{currentfill}{rgb}{0.000000,0.000000,0.000000}%
\pgfsetfillcolor{currentfill}%
\pgfsetlinewidth{0.803000pt}%
\definecolor{currentstroke}{rgb}{0.000000,0.000000,0.000000}%
\pgfsetstrokecolor{currentstroke}%
\pgfsetdash{}{0pt}%
\pgfsys@defobject{currentmarker}{\pgfqpoint{-0.048611in}{0.000000in}}{\pgfqpoint{0.000000in}{0.000000in}}{%
\pgfpathmoveto{\pgfqpoint{0.000000in}{0.000000in}}%
\pgfpathlineto{\pgfqpoint{-0.048611in}{0.000000in}}%
\pgfusepath{stroke,fill}%
}%
\begin{pgfscope}%
\pgfsys@transformshift{0.703550in}{3.666562in}%
\pgfsys@useobject{currentmarker}{}%
\end{pgfscope}%
\end{pgfscope}%
\begin{pgfscope}%
\pgftext[x=0.289968in,y=3.613801in,left,base]{\rmfamily\fontsize{10.000000}{12.000000}\selectfont \(\displaystyle 2.200\)}%
\end{pgfscope}%
\begin{pgfscope}%
\pgftext[x=0.234413in,y=2.192682in,,bottom,rotate=90.000000]{\rmfamily\fontsize{10.000000}{12.000000}\selectfont \(\displaystyle t_f\)}%
\end{pgfscope}%
\begin{pgfscope}%
\pgfsetrectcap%
\pgfsetmiterjoin%
\pgfsetlinewidth{0.803000pt}%
\definecolor{currentstroke}{rgb}{0.000000,0.000000,0.000000}%
\pgfsetstrokecolor{currentstroke}%
\pgfsetdash{}{0pt}%
\pgfpathmoveto{\pgfqpoint{0.703550in}{0.682682in}}%
\pgfpathlineto{\pgfqpoint{0.703550in}{3.702682in}}%
\pgfusepath{stroke}%
\end{pgfscope}%
\begin{pgfscope}%
\pgfsetrectcap%
\pgfsetmiterjoin%
\pgfsetlinewidth{0.803000pt}%
\definecolor{currentstroke}{rgb}{0.000000,0.000000,0.000000}%
\pgfsetstrokecolor{currentstroke}%
\pgfsetdash{}{0pt}%
\pgfpathmoveto{\pgfqpoint{4.423550in}{0.682682in}}%
\pgfpathlineto{\pgfqpoint{4.423550in}{3.702682in}}%
\pgfusepath{stroke}%
\end{pgfscope}%
\begin{pgfscope}%
\pgfsetrectcap%
\pgfsetmiterjoin%
\pgfsetlinewidth{0.803000pt}%
\definecolor{currentstroke}{rgb}{0.000000,0.000000,0.000000}%
\pgfsetstrokecolor{currentstroke}%
\pgfsetdash{}{0pt}%
\pgfpathmoveto{\pgfqpoint{0.703550in}{0.682682in}}%
\pgfpathlineto{\pgfqpoint{4.423550in}{0.682682in}}%
\pgfusepath{stroke}%
\end{pgfscope}%
\begin{pgfscope}%
\pgfsetrectcap%
\pgfsetmiterjoin%
\pgfsetlinewidth{0.803000pt}%
\definecolor{currentstroke}{rgb}{0.000000,0.000000,0.000000}%
\pgfsetstrokecolor{currentstroke}%
\pgfsetdash{}{0pt}%
\pgfpathmoveto{\pgfqpoint{0.703550in}{3.702682in}}%
\pgfpathlineto{\pgfqpoint{4.423550in}{3.702682in}}%
\pgfusepath{stroke}%
\end{pgfscope}%
\begin{pgfscope}%
\pgfpathrectangle{\pgfqpoint{4.656050in}{0.682682in}}{\pgfqpoint{0.151000in}{3.020000in}} %
\pgfusepath{clip}%
\pgfsetbuttcap%
\pgfsetmiterjoin%
\definecolor{currentfill}{rgb}{1.000000,1.000000,1.000000}%
\pgfsetfillcolor{currentfill}%
\pgfsetlinewidth{0.010037pt}%
\definecolor{currentstroke}{rgb}{1.000000,1.000000,1.000000}%
\pgfsetstrokecolor{currentstroke}%
\pgfsetdash{}{0pt}%
\pgfpathmoveto{\pgfqpoint{4.656050in}{0.682682in}}%
\pgfpathlineto{\pgfqpoint{4.656050in}{0.694479in}}%
\pgfpathlineto{\pgfqpoint{4.656050in}{3.690885in}}%
\pgfpathlineto{\pgfqpoint{4.656050in}{3.702682in}}%
\pgfpathlineto{\pgfqpoint{4.807050in}{3.702682in}}%
\pgfpathlineto{\pgfqpoint{4.807050in}{3.690885in}}%
\pgfpathlineto{\pgfqpoint{4.807050in}{0.694479in}}%
\pgfpathlineto{\pgfqpoint{4.807050in}{0.682682in}}%
\pgfpathclose%
\pgfusepath{stroke,fill}%
\end{pgfscope}%
\begin{pgfscope}%
\pgfsys@transformshift{4.660000in}{0.687682in}%
\pgftext[left,bottom]{\pgfimage[interpolate=true,width=0.150000in,height=3.020000in]{times-img0.png}}%
\end{pgfscope}%
\begin{pgfscope}%
\pgfsetbuttcap%
\pgfsetroundjoin%
\definecolor{currentfill}{rgb}{0.000000,0.000000,0.000000}%
\pgfsetfillcolor{currentfill}%
\pgfsetlinewidth{0.803000pt}%
\definecolor{currentstroke}{rgb}{0.000000,0.000000,0.000000}%
\pgfsetstrokecolor{currentstroke}%
\pgfsetdash{}{0pt}%
\pgfsys@defobject{currentmarker}{\pgfqpoint{0.000000in}{0.000000in}}{\pgfqpoint{0.048611in}{0.000000in}}{%
\pgfpathmoveto{\pgfqpoint{0.000000in}{0.000000in}}%
\pgfpathlineto{\pgfqpoint{0.048611in}{0.000000in}}%
\pgfusepath{stroke,fill}%
}%
\begin{pgfscope}%
\pgfsys@transformshift{4.807050in}{1.050754in}%
\pgfsys@useobject{currentmarker}{}%
\end{pgfscope}%
\end{pgfscope}%
\begin{pgfscope}%
\pgftext[x=4.904272in,y=0.997992in,left,base]{\rmfamily\fontsize{10.000000}{12.000000}\selectfont \(\displaystyle 0.02\)}%
\end{pgfscope}%
\begin{pgfscope}%
\pgfsetbuttcap%
\pgfsetroundjoin%
\definecolor{currentfill}{rgb}{0.000000,0.000000,0.000000}%
\pgfsetfillcolor{currentfill}%
\pgfsetlinewidth{0.803000pt}%
\definecolor{currentstroke}{rgb}{0.000000,0.000000,0.000000}%
\pgfsetstrokecolor{currentstroke}%
\pgfsetdash{}{0pt}%
\pgfsys@defobject{currentmarker}{\pgfqpoint{0.000000in}{0.000000in}}{\pgfqpoint{0.048611in}{0.000000in}}{%
\pgfpathmoveto{\pgfqpoint{0.000000in}{0.000000in}}%
\pgfpathlineto{\pgfqpoint{0.048611in}{0.000000in}}%
\pgfusepath{stroke,fill}%
}%
\begin{pgfscope}%
\pgfsys@transformshift{4.807050in}{1.544429in}%
\pgfsys@useobject{currentmarker}{}%
\end{pgfscope}%
\end{pgfscope}%
\begin{pgfscope}%
\pgftext[x=4.904272in,y=1.491668in,left,base]{\rmfamily\fontsize{10.000000}{12.000000}\selectfont \(\displaystyle 0.04\)}%
\end{pgfscope}%
\begin{pgfscope}%
\pgfsetbuttcap%
\pgfsetroundjoin%
\definecolor{currentfill}{rgb}{0.000000,0.000000,0.000000}%
\pgfsetfillcolor{currentfill}%
\pgfsetlinewidth{0.803000pt}%
\definecolor{currentstroke}{rgb}{0.000000,0.000000,0.000000}%
\pgfsetstrokecolor{currentstroke}%
\pgfsetdash{}{0pt}%
\pgfsys@defobject{currentmarker}{\pgfqpoint{0.000000in}{0.000000in}}{\pgfqpoint{0.048611in}{0.000000in}}{%
\pgfpathmoveto{\pgfqpoint{0.000000in}{0.000000in}}%
\pgfpathlineto{\pgfqpoint{0.048611in}{0.000000in}}%
\pgfusepath{stroke,fill}%
}%
\begin{pgfscope}%
\pgfsys@transformshift{4.807050in}{2.038104in}%
\pgfsys@useobject{currentmarker}{}%
\end{pgfscope}%
\end{pgfscope}%
\begin{pgfscope}%
\pgftext[x=4.904272in,y=1.985343in,left,base]{\rmfamily\fontsize{10.000000}{12.000000}\selectfont \(\displaystyle 0.06\)}%
\end{pgfscope}%
\begin{pgfscope}%
\pgfsetbuttcap%
\pgfsetroundjoin%
\definecolor{currentfill}{rgb}{0.000000,0.000000,0.000000}%
\pgfsetfillcolor{currentfill}%
\pgfsetlinewidth{0.803000pt}%
\definecolor{currentstroke}{rgb}{0.000000,0.000000,0.000000}%
\pgfsetstrokecolor{currentstroke}%
\pgfsetdash{}{0pt}%
\pgfsys@defobject{currentmarker}{\pgfqpoint{0.000000in}{0.000000in}}{\pgfqpoint{0.048611in}{0.000000in}}{%
\pgfpathmoveto{\pgfqpoint{0.000000in}{0.000000in}}%
\pgfpathlineto{\pgfqpoint{0.048611in}{0.000000in}}%
\pgfusepath{stroke,fill}%
}%
\begin{pgfscope}%
\pgfsys@transformshift{4.807050in}{2.531780in}%
\pgfsys@useobject{currentmarker}{}%
\end{pgfscope}%
\end{pgfscope}%
\begin{pgfscope}%
\pgftext[x=4.904272in,y=2.479018in,left,base]{\rmfamily\fontsize{10.000000}{12.000000}\selectfont \(\displaystyle 0.08\)}%
\end{pgfscope}%
\begin{pgfscope}%
\pgfsetbuttcap%
\pgfsetroundjoin%
\definecolor{currentfill}{rgb}{0.000000,0.000000,0.000000}%
\pgfsetfillcolor{currentfill}%
\pgfsetlinewidth{0.803000pt}%
\definecolor{currentstroke}{rgb}{0.000000,0.000000,0.000000}%
\pgfsetstrokecolor{currentstroke}%
\pgfsetdash{}{0pt}%
\pgfsys@defobject{currentmarker}{\pgfqpoint{0.000000in}{0.000000in}}{\pgfqpoint{0.048611in}{0.000000in}}{%
\pgfpathmoveto{\pgfqpoint{0.000000in}{0.000000in}}%
\pgfpathlineto{\pgfqpoint{0.048611in}{0.000000in}}%
\pgfusepath{stroke,fill}%
}%
\begin{pgfscope}%
\pgfsys@transformshift{4.807050in}{3.025455in}%
\pgfsys@useobject{currentmarker}{}%
\end{pgfscope}%
\end{pgfscope}%
\begin{pgfscope}%
\pgftext[x=4.904272in,y=2.972694in,left,base]{\rmfamily\fontsize{10.000000}{12.000000}\selectfont \(\displaystyle 0.10\)}%
\end{pgfscope}%
\begin{pgfscope}%
\pgfsetbuttcap%
\pgfsetroundjoin%
\definecolor{currentfill}{rgb}{0.000000,0.000000,0.000000}%
\pgfsetfillcolor{currentfill}%
\pgfsetlinewidth{0.803000pt}%
\definecolor{currentstroke}{rgb}{0.000000,0.000000,0.000000}%
\pgfsetstrokecolor{currentstroke}%
\pgfsetdash{}{0pt}%
\pgfsys@defobject{currentmarker}{\pgfqpoint{0.000000in}{0.000000in}}{\pgfqpoint{0.048611in}{0.000000in}}{%
\pgfpathmoveto{\pgfqpoint{0.000000in}{0.000000in}}%
\pgfpathlineto{\pgfqpoint{0.048611in}{0.000000in}}%
\pgfusepath{stroke,fill}%
}%
\begin{pgfscope}%
\pgfsys@transformshift{4.807050in}{3.519130in}%
\pgfsys@useobject{currentmarker}{}%
\end{pgfscope}%
\end{pgfscope}%
\begin{pgfscope}%
\pgftext[x=4.904272in,y=3.466369in,left,base]{\rmfamily\fontsize{10.000000}{12.000000}\selectfont \(\displaystyle 0.12\)}%
\end{pgfscope}%
\begin{pgfscope}%
\pgftext[x=5.206742in,y=2.192682in,,top,rotate=90.000000]{\rmfamily\fontsize{10.000000}{12.000000}\selectfont \(\displaystyle \mathbf{E}\mbox{u}\)}%
\end{pgfscope}%
\begin{pgfscope}%
\pgfsetbuttcap%
\pgfsetmiterjoin%
\pgfsetlinewidth{0.803000pt}%
\definecolor{currentstroke}{rgb}{0.000000,0.000000,0.000000}%
\pgfsetstrokecolor{currentstroke}%
\pgfsetdash{}{0pt}%
\pgfpathmoveto{\pgfqpoint{4.656050in}{0.682682in}}%
\pgfpathlineto{\pgfqpoint{4.656050in}{0.694479in}}%
\pgfpathlineto{\pgfqpoint{4.656050in}{3.690885in}}%
\pgfpathlineto{\pgfqpoint{4.656050in}{3.702682in}}%
\pgfpathlineto{\pgfqpoint{4.807050in}{3.702682in}}%
\pgfpathlineto{\pgfqpoint{4.807050in}{3.690885in}}%
\pgfpathlineto{\pgfqpoint{4.807050in}{0.694479in}}%
\pgfpathlineto{\pgfqpoint{4.807050in}{0.682682in}}%
\pgfpathclose%
\pgfusepath{stroke}%
\end{pgfscope}%
\end{pgfpicture}%
\makeatother%
\endgroup%

    \caption{.\label{fig:times}}
\end{figure}

The covariance of $\mathbb{I}\mbox{m}$ with $\mathbb{E}\mbox{u}$ is shown in Figure \ref{fig:dnumbs}. Predictably, there is quite strong correlation between the dimensionless groups. We also see that $\mathbb{I}\mbox{m} < 1$ for all drops. Using an OLS regression, we find the model $\mathbb{I}\mbox{m} \sim (0.012 \pm 0.003) \mathbb{E}\mbox{u} + (0.212 \pm 0.036) $ with $R^2 =0.59$.
\begin{figure}[htb]
    \centering
    %% Creator: Matplotlib, PGF backend
%%
%% To include the figure in your LaTeX document, write
%%   \input{<filename>.pgf}
%%
%% Make sure the required packages are loaded in your preamble
%%   \usepackage{pgf}
%%
%% Figures using additional raster images can only be included by \input if
%% they are in the same directory as the main LaTeX file. For loading figures
%% from other directories you can use the `import` package
%%   \usepackage{import}
%% and then include the figures with
%%   \import{<path to file>}{<filename>.pgf}
%%
%% Matplotlib used the following preamble
%%   \usepackage{fontspec}
%%   \setmainfont{DejaVuSerif.ttf}[Path=/home/erin/anaconda3/lib/python3.6/site-packages/matplotlib/mpl-data/fonts/ttf/]
%%   \setsansfont{DejaVuSans.ttf}[Path=/home/erin/anaconda3/lib/python3.6/site-packages/matplotlib/mpl-data/fonts/ttf/]
%%   \setmonofont{DejaVuSansMono.ttf}[Path=/home/erin/anaconda3/lib/python3.6/site-packages/matplotlib/mpl-data/fonts/ttf/]
%%
\begingroup%
\makeatletter%
\begin{pgfpicture}%
\pgfpathrectangle{\pgfpointorigin}{\pgfqpoint{5.314660in}{3.641603in}}%
\pgfusepath{use as bounding box, clip}%
\begin{pgfscope}%
\pgfsetbuttcap%
\pgfsetmiterjoin%
\definecolor{currentfill}{rgb}{1.000000,1.000000,1.000000}%
\pgfsetfillcolor{currentfill}%
\pgfsetlinewidth{0.000000pt}%
\definecolor{currentstroke}{rgb}{1.000000,1.000000,1.000000}%
\pgfsetstrokecolor{currentstroke}%
\pgfsetdash{}{0pt}%
\pgfpathmoveto{\pgfqpoint{0.000000in}{0.000000in}}%
\pgfpathlineto{\pgfqpoint{5.314660in}{0.000000in}}%
\pgfpathlineto{\pgfqpoint{5.314660in}{3.641603in}}%
\pgfpathlineto{\pgfqpoint{0.000000in}{3.641603in}}%
\pgfpathclose%
\pgfusepath{fill}%
\end{pgfscope}%
\begin{pgfscope}%
\pgfsetbuttcap%
\pgfsetmiterjoin%
\definecolor{currentfill}{rgb}{1.000000,1.000000,1.000000}%
\pgfsetfillcolor{currentfill}%
\pgfsetlinewidth{0.000000pt}%
\definecolor{currentstroke}{rgb}{0.000000,0.000000,0.000000}%
\pgfsetstrokecolor{currentstroke}%
\pgfsetstrokeopacity{0.000000}%
\pgfsetdash{}{0pt}%
\pgfpathmoveto{\pgfqpoint{0.564660in}{0.521603in}}%
\pgfpathlineto{\pgfqpoint{5.214660in}{0.521603in}}%
\pgfpathlineto{\pgfqpoint{5.214660in}{3.541603in}}%
\pgfpathlineto{\pgfqpoint{0.564660in}{3.541603in}}%
\pgfpathclose%
\pgfusepath{fill}%
\end{pgfscope}%
\begin{pgfscope}%
\pgfpathrectangle{\pgfqpoint{0.564660in}{0.521603in}}{\pgfqpoint{4.650000in}{3.020000in}}%
\pgfusepath{clip}%
\pgfsetbuttcap%
\pgfsetroundjoin%
\definecolor{currentfill}{rgb}{1.000000,1.000000,1.000000}%
\pgfsetfillcolor{currentfill}%
\pgfsetlinewidth{1.003750pt}%
\definecolor{currentstroke}{rgb}{0.000000,0.000000,0.000000}%
\pgfsetstrokecolor{currentstroke}%
\pgfsetdash{}{0pt}%
\pgfpathmoveto{\pgfqpoint{0.816771in}{2.246882in}}%
\pgfpathcurveto{\pgfqpoint{0.827822in}{2.246882in}}{\pgfqpoint{0.838421in}{2.251272in}}{\pgfqpoint{0.846234in}{2.259086in}}%
\pgfpathcurveto{\pgfqpoint{0.854048in}{2.266900in}}{\pgfqpoint{0.858438in}{2.277499in}}{\pgfqpoint{0.858438in}{2.288549in}}%
\pgfpathcurveto{\pgfqpoint{0.858438in}{2.299599in}}{\pgfqpoint{0.854048in}{2.310198in}}{\pgfqpoint{0.846234in}{2.318012in}}%
\pgfpathcurveto{\pgfqpoint{0.838421in}{2.325825in}}{\pgfqpoint{0.827822in}{2.330215in}}{\pgfqpoint{0.816771in}{2.330215in}}%
\pgfpathcurveto{\pgfqpoint{0.805721in}{2.330215in}}{\pgfqpoint{0.795122in}{2.325825in}}{\pgfqpoint{0.787309in}{2.318012in}}%
\pgfpathcurveto{\pgfqpoint{0.779495in}{2.310198in}}{\pgfqpoint{0.775105in}{2.299599in}}{\pgfqpoint{0.775105in}{2.288549in}}%
\pgfpathcurveto{\pgfqpoint{0.775105in}{2.277499in}}{\pgfqpoint{0.779495in}{2.266900in}}{\pgfqpoint{0.787309in}{2.259086in}}%
\pgfpathcurveto{\pgfqpoint{0.795122in}{2.251272in}}{\pgfqpoint{0.805721in}{2.246882in}}{\pgfqpoint{0.816771in}{2.246882in}}%
\pgfpathclose%
\pgfusepath{stroke,fill}%
\end{pgfscope}%
\begin{pgfscope}%
\pgfpathrectangle{\pgfqpoint{0.564660in}{0.521603in}}{\pgfqpoint{4.650000in}{3.020000in}}%
\pgfusepath{clip}%
\pgfsetbuttcap%
\pgfsetroundjoin%
\definecolor{currentfill}{rgb}{1.000000,1.000000,1.000000}%
\pgfsetfillcolor{currentfill}%
\pgfsetlinewidth{1.003750pt}%
\definecolor{currentstroke}{rgb}{0.000000,0.000000,0.000000}%
\pgfsetstrokecolor{currentstroke}%
\pgfsetdash{}{0pt}%
\pgfpathmoveto{\pgfqpoint{3.858085in}{1.799540in}}%
\pgfpathcurveto{\pgfqpoint{3.869135in}{1.799540in}}{\pgfqpoint{3.879734in}{1.803931in}}{\pgfqpoint{3.887547in}{1.811744in}}%
\pgfpathcurveto{\pgfqpoint{3.895361in}{1.819558in}}{\pgfqpoint{3.899751in}{1.830157in}}{\pgfqpoint{3.899751in}{1.841207in}}%
\pgfpathcurveto{\pgfqpoint{3.899751in}{1.852257in}}{\pgfqpoint{3.895361in}{1.862856in}}{\pgfqpoint{3.887547in}{1.870670in}}%
\pgfpathcurveto{\pgfqpoint{3.879734in}{1.878484in}}{\pgfqpoint{3.869135in}{1.882874in}}{\pgfqpoint{3.858085in}{1.882874in}}%
\pgfpathcurveto{\pgfqpoint{3.847034in}{1.882874in}}{\pgfqpoint{3.836435in}{1.878484in}}{\pgfqpoint{3.828622in}{1.870670in}}%
\pgfpathcurveto{\pgfqpoint{3.820808in}{1.862856in}}{\pgfqpoint{3.816418in}{1.852257in}}{\pgfqpoint{3.816418in}{1.841207in}}%
\pgfpathcurveto{\pgfqpoint{3.816418in}{1.830157in}}{\pgfqpoint{3.820808in}{1.819558in}}{\pgfqpoint{3.828622in}{1.811744in}}%
\pgfpathcurveto{\pgfqpoint{3.836435in}{1.803931in}}{\pgfqpoint{3.847034in}{1.799540in}}{\pgfqpoint{3.858085in}{1.799540in}}%
\pgfpathclose%
\pgfusepath{stroke,fill}%
\end{pgfscope}%
\begin{pgfscope}%
\pgfpathrectangle{\pgfqpoint{0.564660in}{0.521603in}}{\pgfqpoint{4.650000in}{3.020000in}}%
\pgfusepath{clip}%
\pgfsetbuttcap%
\pgfsetroundjoin%
\definecolor{currentfill}{rgb}{1.000000,1.000000,1.000000}%
\pgfsetfillcolor{currentfill}%
\pgfsetlinewidth{1.003750pt}%
\definecolor{currentstroke}{rgb}{0.000000,0.000000,0.000000}%
\pgfsetstrokecolor{currentstroke}%
\pgfsetdash{}{0pt}%
\pgfpathmoveto{\pgfqpoint{4.993344in}{2.156424in}}%
\pgfpathcurveto{\pgfqpoint{5.004394in}{2.156424in}}{\pgfqpoint{5.014993in}{2.160815in}}{\pgfqpoint{5.022807in}{2.168628in}}%
\pgfpathcurveto{\pgfqpoint{5.030620in}{2.176442in}}{\pgfqpoint{5.035010in}{2.187041in}}{\pgfqpoint{5.035010in}{2.198091in}}%
\pgfpathcurveto{\pgfqpoint{5.035010in}{2.209141in}}{\pgfqpoint{5.030620in}{2.219740in}}{\pgfqpoint{5.022807in}{2.227554in}}%
\pgfpathcurveto{\pgfqpoint{5.014993in}{2.235367in}}{\pgfqpoint{5.004394in}{2.239758in}}{\pgfqpoint{4.993344in}{2.239758in}}%
\pgfpathcurveto{\pgfqpoint{4.982294in}{2.239758in}}{\pgfqpoint{4.971695in}{2.235367in}}{\pgfqpoint{4.963881in}{2.227554in}}%
\pgfpathcurveto{\pgfqpoint{4.956067in}{2.219740in}}{\pgfqpoint{4.951677in}{2.209141in}}{\pgfqpoint{4.951677in}{2.198091in}}%
\pgfpathcurveto{\pgfqpoint{4.951677in}{2.187041in}}{\pgfqpoint{4.956067in}{2.176442in}}{\pgfqpoint{4.963881in}{2.168628in}}%
\pgfpathcurveto{\pgfqpoint{4.971695in}{2.160815in}}{\pgfqpoint{4.982294in}{2.156424in}}{\pgfqpoint{4.993344in}{2.156424in}}%
\pgfpathclose%
\pgfusepath{stroke,fill}%
\end{pgfscope}%
\begin{pgfscope}%
\pgfpathrectangle{\pgfqpoint{0.564660in}{0.521603in}}{\pgfqpoint{4.650000in}{3.020000in}}%
\pgfusepath{clip}%
\pgfsetbuttcap%
\pgfsetroundjoin%
\definecolor{currentfill}{rgb}{1.000000,1.000000,1.000000}%
\pgfsetfillcolor{currentfill}%
\pgfsetlinewidth{1.003750pt}%
\definecolor{currentstroke}{rgb}{0.000000,0.000000,0.000000}%
\pgfsetstrokecolor{currentstroke}%
\pgfsetdash{}{0pt}%
\pgfpathmoveto{\pgfqpoint{1.914828in}{1.195437in}}%
\pgfpathcurveto{\pgfqpoint{1.925878in}{1.195437in}}{\pgfqpoint{1.936477in}{1.199828in}}{\pgfqpoint{1.944290in}{1.207641in}}%
\pgfpathcurveto{\pgfqpoint{1.952104in}{1.215455in}}{\pgfqpoint{1.956494in}{1.226054in}}{\pgfqpoint{1.956494in}{1.237104in}}%
\pgfpathcurveto{\pgfqpoint{1.956494in}{1.248154in}}{\pgfqpoint{1.952104in}{1.258753in}}{\pgfqpoint{1.944290in}{1.266567in}}%
\pgfpathcurveto{\pgfqpoint{1.936477in}{1.274380in}}{\pgfqpoint{1.925878in}{1.278771in}}{\pgfqpoint{1.914828in}{1.278771in}}%
\pgfpathcurveto{\pgfqpoint{1.903777in}{1.278771in}}{\pgfqpoint{1.893178in}{1.274380in}}{\pgfqpoint{1.885365in}{1.266567in}}%
\pgfpathcurveto{\pgfqpoint{1.877551in}{1.258753in}}{\pgfqpoint{1.873161in}{1.248154in}}{\pgfqpoint{1.873161in}{1.237104in}}%
\pgfpathcurveto{\pgfqpoint{1.873161in}{1.226054in}}{\pgfqpoint{1.877551in}{1.215455in}}{\pgfqpoint{1.885365in}{1.207641in}}%
\pgfpathcurveto{\pgfqpoint{1.893178in}{1.199828in}}{\pgfqpoint{1.903777in}{1.195437in}}{\pgfqpoint{1.914828in}{1.195437in}}%
\pgfpathclose%
\pgfusepath{stroke,fill}%
\end{pgfscope}%
\begin{pgfscope}%
\pgfpathrectangle{\pgfqpoint{0.564660in}{0.521603in}}{\pgfqpoint{4.650000in}{3.020000in}}%
\pgfusepath{clip}%
\pgfsetbuttcap%
\pgfsetroundjoin%
\definecolor{currentfill}{rgb}{1.000000,1.000000,1.000000}%
\pgfsetfillcolor{currentfill}%
\pgfsetlinewidth{1.003750pt}%
\definecolor{currentstroke}{rgb}{0.000000,0.000000,0.000000}%
\pgfsetstrokecolor{currentstroke}%
\pgfsetdash{}{0pt}%
\pgfpathmoveto{\pgfqpoint{2.773519in}{3.358147in}}%
\pgfpathcurveto{\pgfqpoint{2.784569in}{3.358147in}}{\pgfqpoint{2.795168in}{3.362537in}}{\pgfqpoint{2.802981in}{3.370350in}}%
\pgfpathcurveto{\pgfqpoint{2.810795in}{3.378164in}}{\pgfqpoint{2.815185in}{3.388763in}}{\pgfqpoint{2.815185in}{3.399813in}}%
\pgfpathcurveto{\pgfqpoint{2.815185in}{3.410863in}}{\pgfqpoint{2.810795in}{3.421462in}}{\pgfqpoint{2.802981in}{3.429276in}}%
\pgfpathcurveto{\pgfqpoint{2.795168in}{3.437090in}}{\pgfqpoint{2.784569in}{3.441480in}}{\pgfqpoint{2.773519in}{3.441480in}}%
\pgfpathcurveto{\pgfqpoint{2.762469in}{3.441480in}}{\pgfqpoint{2.751870in}{3.437090in}}{\pgfqpoint{2.744056in}{3.429276in}}%
\pgfpathcurveto{\pgfqpoint{2.736242in}{3.421462in}}{\pgfqpoint{2.731852in}{3.410863in}}{\pgfqpoint{2.731852in}{3.399813in}}%
\pgfpathcurveto{\pgfqpoint{2.731852in}{3.388763in}}{\pgfqpoint{2.736242in}{3.378164in}}{\pgfqpoint{2.744056in}{3.370350in}}%
\pgfpathcurveto{\pgfqpoint{2.751870in}{3.362537in}}{\pgfqpoint{2.762469in}{3.358147in}}{\pgfqpoint{2.773519in}{3.358147in}}%
\pgfpathclose%
\pgfusepath{stroke,fill}%
\end{pgfscope}%
\begin{pgfscope}%
\pgfpathrectangle{\pgfqpoint{0.564660in}{0.521603in}}{\pgfqpoint{4.650000in}{3.020000in}}%
\pgfusepath{clip}%
\pgfsetbuttcap%
\pgfsetroundjoin%
\definecolor{currentfill}{rgb}{1.000000,1.000000,1.000000}%
\pgfsetfillcolor{currentfill}%
\pgfsetlinewidth{1.003750pt}%
\definecolor{currentstroke}{rgb}{0.000000,0.000000,0.000000}%
\pgfsetstrokecolor{currentstroke}%
\pgfsetdash{}{0pt}%
\pgfpathmoveto{\pgfqpoint{2.234616in}{0.983815in}}%
\pgfpathcurveto{\pgfqpoint{2.245666in}{0.983815in}}{\pgfqpoint{2.256265in}{0.988205in}}{\pgfqpoint{2.264079in}{0.996018in}}%
\pgfpathcurveto{\pgfqpoint{2.271892in}{1.003832in}}{\pgfqpoint{2.276283in}{1.014431in}}{\pgfqpoint{2.276283in}{1.025481in}}%
\pgfpathcurveto{\pgfqpoint{2.276283in}{1.036531in}}{\pgfqpoint{2.271892in}{1.047130in}}{\pgfqpoint{2.264079in}{1.054944in}}%
\pgfpathcurveto{\pgfqpoint{2.256265in}{1.062758in}}{\pgfqpoint{2.245666in}{1.067148in}}{\pgfqpoint{2.234616in}{1.067148in}}%
\pgfpathcurveto{\pgfqpoint{2.223566in}{1.067148in}}{\pgfqpoint{2.212967in}{1.062758in}}{\pgfqpoint{2.205153in}{1.054944in}}%
\pgfpathcurveto{\pgfqpoint{2.197340in}{1.047130in}}{\pgfqpoint{2.192949in}{1.036531in}}{\pgfqpoint{2.192949in}{1.025481in}}%
\pgfpathcurveto{\pgfqpoint{2.192949in}{1.014431in}}{\pgfqpoint{2.197340in}{1.003832in}}{\pgfqpoint{2.205153in}{0.996018in}}%
\pgfpathcurveto{\pgfqpoint{2.212967in}{0.988205in}}{\pgfqpoint{2.223566in}{0.983815in}}{\pgfqpoint{2.234616in}{0.983815in}}%
\pgfpathclose%
\pgfusepath{stroke,fill}%
\end{pgfscope}%
\begin{pgfscope}%
\pgfpathrectangle{\pgfqpoint{0.564660in}{0.521603in}}{\pgfqpoint{4.650000in}{3.020000in}}%
\pgfusepath{clip}%
\pgfsetbuttcap%
\pgfsetroundjoin%
\definecolor{currentfill}{rgb}{1.000000,1.000000,1.000000}%
\pgfsetfillcolor{currentfill}%
\pgfsetlinewidth{1.003750pt}%
\definecolor{currentstroke}{rgb}{0.000000,0.000000,0.000000}%
\pgfsetstrokecolor{currentstroke}%
\pgfsetdash{}{0pt}%
\pgfpathmoveto{\pgfqpoint{0.819079in}{1.480427in}}%
\pgfpathcurveto{\pgfqpoint{0.830129in}{1.480427in}}{\pgfqpoint{0.840728in}{1.484817in}}{\pgfqpoint{0.848542in}{1.492631in}}%
\pgfpathcurveto{\pgfqpoint{0.856355in}{1.500444in}}{\pgfqpoint{0.860746in}{1.511043in}}{\pgfqpoint{0.860746in}{1.522093in}}%
\pgfpathcurveto{\pgfqpoint{0.860746in}{1.533144in}}{\pgfqpoint{0.856355in}{1.543743in}}{\pgfqpoint{0.848542in}{1.551556in}}%
\pgfpathcurveto{\pgfqpoint{0.840728in}{1.559370in}}{\pgfqpoint{0.830129in}{1.563760in}}{\pgfqpoint{0.819079in}{1.563760in}}%
\pgfpathcurveto{\pgfqpoint{0.808029in}{1.563760in}}{\pgfqpoint{0.797430in}{1.559370in}}{\pgfqpoint{0.789616in}{1.551556in}}%
\pgfpathcurveto{\pgfqpoint{0.781803in}{1.543743in}}{\pgfqpoint{0.777412in}{1.533144in}}{\pgfqpoint{0.777412in}{1.522093in}}%
\pgfpathcurveto{\pgfqpoint{0.777412in}{1.511043in}}{\pgfqpoint{0.781803in}{1.500444in}}{\pgfqpoint{0.789616in}{1.492631in}}%
\pgfpathcurveto{\pgfqpoint{0.797430in}{1.484817in}}{\pgfqpoint{0.808029in}{1.480427in}}{\pgfqpoint{0.819079in}{1.480427in}}%
\pgfpathclose%
\pgfusepath{stroke,fill}%
\end{pgfscope}%
\begin{pgfscope}%
\pgfpathrectangle{\pgfqpoint{0.564660in}{0.521603in}}{\pgfqpoint{4.650000in}{3.020000in}}%
\pgfusepath{clip}%
\pgfsetbuttcap%
\pgfsetroundjoin%
\definecolor{currentfill}{rgb}{1.000000,1.000000,1.000000}%
\pgfsetfillcolor{currentfill}%
\pgfsetlinewidth{1.003750pt}%
\definecolor{currentstroke}{rgb}{0.000000,0.000000,0.000000}%
\pgfsetstrokecolor{currentstroke}%
\pgfsetdash{}{0pt}%
\pgfpathmoveto{\pgfqpoint{1.363796in}{0.692093in}}%
\pgfpathcurveto{\pgfqpoint{1.374846in}{0.692093in}}{\pgfqpoint{1.385445in}{0.696483in}}{\pgfqpoint{1.393259in}{0.704297in}}%
\pgfpathcurveto{\pgfqpoint{1.401072in}{0.712110in}}{\pgfqpoint{1.405462in}{0.722709in}}{\pgfqpoint{1.405462in}{0.733759in}}%
\pgfpathcurveto{\pgfqpoint{1.405462in}{0.744810in}}{\pgfqpoint{1.401072in}{0.755409in}}{\pgfqpoint{1.393259in}{0.763222in}}%
\pgfpathcurveto{\pgfqpoint{1.385445in}{0.771036in}}{\pgfqpoint{1.374846in}{0.775426in}}{\pgfqpoint{1.363796in}{0.775426in}}%
\pgfpathcurveto{\pgfqpoint{1.352746in}{0.775426in}}{\pgfqpoint{1.342147in}{0.771036in}}{\pgfqpoint{1.334333in}{0.763222in}}%
\pgfpathcurveto{\pgfqpoint{1.326519in}{0.755409in}}{\pgfqpoint{1.322129in}{0.744810in}}{\pgfqpoint{1.322129in}{0.733759in}}%
\pgfpathcurveto{\pgfqpoint{1.322129in}{0.722709in}}{\pgfqpoint{1.326519in}{0.712110in}}{\pgfqpoint{1.334333in}{0.704297in}}%
\pgfpathcurveto{\pgfqpoint{1.342147in}{0.696483in}}{\pgfqpoint{1.352746in}{0.692093in}}{\pgfqpoint{1.363796in}{0.692093in}}%
\pgfpathclose%
\pgfusepath{stroke,fill}%
\end{pgfscope}%
\begin{pgfscope}%
\pgfpathrectangle{\pgfqpoint{0.564660in}{0.521603in}}{\pgfqpoint{4.650000in}{3.020000in}}%
\pgfusepath{clip}%
\pgfsetbuttcap%
\pgfsetroundjoin%
\definecolor{currentfill}{rgb}{1.000000,1.000000,1.000000}%
\pgfsetfillcolor{currentfill}%
\pgfsetlinewidth{1.003750pt}%
\definecolor{currentstroke}{rgb}{0.000000,0.000000,0.000000}%
\pgfsetstrokecolor{currentstroke}%
\pgfsetdash{}{0pt}%
\pgfpathmoveto{\pgfqpoint{1.096147in}{0.621727in}}%
\pgfpathcurveto{\pgfqpoint{1.107197in}{0.621727in}}{\pgfqpoint{1.117796in}{0.626117in}}{\pgfqpoint{1.125609in}{0.633931in}}%
\pgfpathcurveto{\pgfqpoint{1.133423in}{0.641744in}}{\pgfqpoint{1.137813in}{0.652343in}}{\pgfqpoint{1.137813in}{0.663393in}}%
\pgfpathcurveto{\pgfqpoint{1.137813in}{0.674444in}}{\pgfqpoint{1.133423in}{0.685043in}}{\pgfqpoint{1.125609in}{0.692856in}}%
\pgfpathcurveto{\pgfqpoint{1.117796in}{0.700670in}}{\pgfqpoint{1.107197in}{0.705060in}}{\pgfqpoint{1.096147in}{0.705060in}}%
\pgfpathcurveto{\pgfqpoint{1.085096in}{0.705060in}}{\pgfqpoint{1.074497in}{0.700670in}}{\pgfqpoint{1.066684in}{0.692856in}}%
\pgfpathcurveto{\pgfqpoint{1.058870in}{0.685043in}}{\pgfqpoint{1.054480in}{0.674444in}}{\pgfqpoint{1.054480in}{0.663393in}}%
\pgfpathcurveto{\pgfqpoint{1.054480in}{0.652343in}}{\pgfqpoint{1.058870in}{0.641744in}}{\pgfqpoint{1.066684in}{0.633931in}}%
\pgfpathcurveto{\pgfqpoint{1.074497in}{0.626117in}}{\pgfqpoint{1.085096in}{0.621727in}}{\pgfqpoint{1.096147in}{0.621727in}}%
\pgfpathclose%
\pgfusepath{stroke,fill}%
\end{pgfscope}%
\begin{pgfscope}%
\pgfpathrectangle{\pgfqpoint{0.564660in}{0.521603in}}{\pgfqpoint{4.650000in}{3.020000in}}%
\pgfusepath{clip}%
\pgfsetbuttcap%
\pgfsetroundjoin%
\definecolor{currentfill}{rgb}{1.000000,1.000000,1.000000}%
\pgfsetfillcolor{currentfill}%
\pgfsetlinewidth{1.003750pt}%
\definecolor{currentstroke}{rgb}{0.000000,0.000000,0.000000}%
\pgfsetstrokecolor{currentstroke}%
\pgfsetdash{}{0pt}%
\pgfpathmoveto{\pgfqpoint{0.887904in}{1.305315in}}%
\pgfpathcurveto{\pgfqpoint{0.898954in}{1.305315in}}{\pgfqpoint{0.909553in}{1.309705in}}{\pgfqpoint{0.917367in}{1.317519in}}%
\pgfpathcurveto{\pgfqpoint{0.925180in}{1.325332in}}{\pgfqpoint{0.929570in}{1.335932in}}{\pgfqpoint{0.929570in}{1.346982in}}%
\pgfpathcurveto{\pgfqpoint{0.929570in}{1.358032in}}{\pgfqpoint{0.925180in}{1.368631in}}{\pgfqpoint{0.917367in}{1.376444in}}%
\pgfpathcurveto{\pgfqpoint{0.909553in}{1.384258in}}{\pgfqpoint{0.898954in}{1.388648in}}{\pgfqpoint{0.887904in}{1.388648in}}%
\pgfpathcurveto{\pgfqpoint{0.876854in}{1.388648in}}{\pgfqpoint{0.866255in}{1.384258in}}{\pgfqpoint{0.858441in}{1.376444in}}%
\pgfpathcurveto{\pgfqpoint{0.850627in}{1.368631in}}{\pgfqpoint{0.846237in}{1.358032in}}{\pgfqpoint{0.846237in}{1.346982in}}%
\pgfpathcurveto{\pgfqpoint{0.846237in}{1.335932in}}{\pgfqpoint{0.850627in}{1.325332in}}{\pgfqpoint{0.858441in}{1.317519in}}%
\pgfpathcurveto{\pgfqpoint{0.866255in}{1.309705in}}{\pgfqpoint{0.876854in}{1.305315in}}{\pgfqpoint{0.887904in}{1.305315in}}%
\pgfpathclose%
\pgfusepath{stroke,fill}%
\end{pgfscope}%
\begin{pgfscope}%
\pgfpathrectangle{\pgfqpoint{0.564660in}{0.521603in}}{\pgfqpoint{4.650000in}{3.020000in}}%
\pgfusepath{clip}%
\pgfsetbuttcap%
\pgfsetroundjoin%
\definecolor{currentfill}{rgb}{1.000000,1.000000,1.000000}%
\pgfsetfillcolor{currentfill}%
\pgfsetlinewidth{1.003750pt}%
\definecolor{currentstroke}{rgb}{0.000000,0.000000,0.000000}%
\pgfsetstrokecolor{currentstroke}%
\pgfsetdash{}{0pt}%
\pgfpathmoveto{\pgfqpoint{2.143771in}{2.158900in}}%
\pgfpathcurveto{\pgfqpoint{2.154821in}{2.158900in}}{\pgfqpoint{2.165420in}{2.163290in}}{\pgfqpoint{2.173234in}{2.171104in}}%
\pgfpathcurveto{\pgfqpoint{2.181047in}{2.178918in}}{\pgfqpoint{2.185438in}{2.189517in}}{\pgfqpoint{2.185438in}{2.200567in}}%
\pgfpathcurveto{\pgfqpoint{2.185438in}{2.211617in}}{\pgfqpoint{2.181047in}{2.222216in}}{\pgfqpoint{2.173234in}{2.230029in}}%
\pgfpathcurveto{\pgfqpoint{2.165420in}{2.237843in}}{\pgfqpoint{2.154821in}{2.242233in}}{\pgfqpoint{2.143771in}{2.242233in}}%
\pgfpathcurveto{\pgfqpoint{2.132721in}{2.242233in}}{\pgfqpoint{2.122122in}{2.237843in}}{\pgfqpoint{2.114308in}{2.230029in}}%
\pgfpathcurveto{\pgfqpoint{2.106494in}{2.222216in}}{\pgfqpoint{2.102104in}{2.211617in}}{\pgfqpoint{2.102104in}{2.200567in}}%
\pgfpathcurveto{\pgfqpoint{2.102104in}{2.189517in}}{\pgfqpoint{2.106494in}{2.178918in}}{\pgfqpoint{2.114308in}{2.171104in}}%
\pgfpathcurveto{\pgfqpoint{2.122122in}{2.163290in}}{\pgfqpoint{2.132721in}{2.158900in}}{\pgfqpoint{2.143771in}{2.158900in}}%
\pgfpathclose%
\pgfusepath{stroke,fill}%
\end{pgfscope}%
\begin{pgfscope}%
\pgfpathrectangle{\pgfqpoint{0.564660in}{0.521603in}}{\pgfqpoint{4.650000in}{3.020000in}}%
\pgfusepath{clip}%
\pgfsetbuttcap%
\pgfsetroundjoin%
\definecolor{currentfill}{rgb}{1.000000,1.000000,1.000000}%
\pgfsetfillcolor{currentfill}%
\pgfsetlinewidth{1.003750pt}%
\definecolor{currentstroke}{rgb}{0.000000,0.000000,0.000000}%
\pgfsetstrokecolor{currentstroke}%
\pgfsetdash{}{0pt}%
\pgfpathmoveto{\pgfqpoint{1.086864in}{0.741510in}}%
\pgfpathcurveto{\pgfqpoint{1.097914in}{0.741510in}}{\pgfqpoint{1.108513in}{0.745901in}}{\pgfqpoint{1.116327in}{0.753714in}}%
\pgfpathcurveto{\pgfqpoint{1.124141in}{0.761528in}}{\pgfqpoint{1.128531in}{0.772127in}}{\pgfqpoint{1.128531in}{0.783177in}}%
\pgfpathcurveto{\pgfqpoint{1.128531in}{0.794227in}}{\pgfqpoint{1.124141in}{0.804826in}}{\pgfqpoint{1.116327in}{0.812640in}}%
\pgfpathcurveto{\pgfqpoint{1.108513in}{0.820453in}}{\pgfqpoint{1.097914in}{0.824844in}}{\pgfqpoint{1.086864in}{0.824844in}}%
\pgfpathcurveto{\pgfqpoint{1.075814in}{0.824844in}}{\pgfqpoint{1.065215in}{0.820453in}}{\pgfqpoint{1.057401in}{0.812640in}}%
\pgfpathcurveto{\pgfqpoint{1.049588in}{0.804826in}}{\pgfqpoint{1.045197in}{0.794227in}}{\pgfqpoint{1.045197in}{0.783177in}}%
\pgfpathcurveto{\pgfqpoint{1.045197in}{0.772127in}}{\pgfqpoint{1.049588in}{0.761528in}}{\pgfqpoint{1.057401in}{0.753714in}}%
\pgfpathcurveto{\pgfqpoint{1.065215in}{0.745901in}}{\pgfqpoint{1.075814in}{0.741510in}}{\pgfqpoint{1.086864in}{0.741510in}}%
\pgfpathclose%
\pgfusepath{stroke,fill}%
\end{pgfscope}%
\begin{pgfscope}%
\pgfpathrectangle{\pgfqpoint{0.564660in}{0.521603in}}{\pgfqpoint{4.650000in}{3.020000in}}%
\pgfusepath{clip}%
\pgfsetbuttcap%
\pgfsetroundjoin%
\definecolor{currentfill}{rgb}{1.000000,1.000000,1.000000}%
\pgfsetfillcolor{currentfill}%
\pgfsetlinewidth{1.003750pt}%
\definecolor{currentstroke}{rgb}{0.000000,0.000000,0.000000}%
\pgfsetstrokecolor{currentstroke}%
\pgfsetdash{}{0pt}%
\pgfpathmoveto{\pgfqpoint{0.785977in}{1.118987in}}%
\pgfpathcurveto{\pgfqpoint{0.797027in}{1.118987in}}{\pgfqpoint{0.807626in}{1.123377in}}{\pgfqpoint{0.815440in}{1.131190in}}%
\pgfpathcurveto{\pgfqpoint{0.823253in}{1.139004in}}{\pgfqpoint{0.827643in}{1.149603in}}{\pgfqpoint{0.827643in}{1.160653in}}%
\pgfpathcurveto{\pgfqpoint{0.827643in}{1.171703in}}{\pgfqpoint{0.823253in}{1.182302in}}{\pgfqpoint{0.815440in}{1.190116in}}%
\pgfpathcurveto{\pgfqpoint{0.807626in}{1.197930in}}{\pgfqpoint{0.797027in}{1.202320in}}{\pgfqpoint{0.785977in}{1.202320in}}%
\pgfpathcurveto{\pgfqpoint{0.774927in}{1.202320in}}{\pgfqpoint{0.764328in}{1.197930in}}{\pgfqpoint{0.756514in}{1.190116in}}%
\pgfpathcurveto{\pgfqpoint{0.748700in}{1.182302in}}{\pgfqpoint{0.744310in}{1.171703in}}{\pgfqpoint{0.744310in}{1.160653in}}%
\pgfpathcurveto{\pgfqpoint{0.744310in}{1.149603in}}{\pgfqpoint{0.748700in}{1.139004in}}{\pgfqpoint{0.756514in}{1.131190in}}%
\pgfpathcurveto{\pgfqpoint{0.764328in}{1.123377in}}{\pgfqpoint{0.774927in}{1.118987in}}{\pgfqpoint{0.785977in}{1.118987in}}%
\pgfpathclose%
\pgfusepath{stroke,fill}%
\end{pgfscope}%
\begin{pgfscope}%
\pgfpathrectangle{\pgfqpoint{0.564660in}{0.521603in}}{\pgfqpoint{4.650000in}{3.020000in}}%
\pgfusepath{clip}%
\pgfsetbuttcap%
\pgfsetroundjoin%
\definecolor{currentfill}{rgb}{1.000000,1.000000,1.000000}%
\pgfsetfillcolor{currentfill}%
\pgfsetlinewidth{1.003750pt}%
\definecolor{currentstroke}{rgb}{0.000000,0.000000,0.000000}%
\pgfsetstrokecolor{currentstroke}%
\pgfsetdash{}{0pt}%
\pgfpathmoveto{\pgfqpoint{2.036724in}{2.941966in}}%
\pgfpathcurveto{\pgfqpoint{2.047774in}{2.941966in}}{\pgfqpoint{2.058373in}{2.946356in}}{\pgfqpoint{2.066187in}{2.954170in}}%
\pgfpathcurveto{\pgfqpoint{2.074000in}{2.961983in}}{\pgfqpoint{2.078390in}{2.972582in}}{\pgfqpoint{2.078390in}{2.983632in}}%
\pgfpathcurveto{\pgfqpoint{2.078390in}{2.994682in}}{\pgfqpoint{2.074000in}{3.005282in}}{\pgfqpoint{2.066187in}{3.013095in}}%
\pgfpathcurveto{\pgfqpoint{2.058373in}{3.020909in}}{\pgfqpoint{2.047774in}{3.025299in}}{\pgfqpoint{2.036724in}{3.025299in}}%
\pgfpathcurveto{\pgfqpoint{2.025674in}{3.025299in}}{\pgfqpoint{2.015075in}{3.020909in}}{\pgfqpoint{2.007261in}{3.013095in}}%
\pgfpathcurveto{\pgfqpoint{1.999447in}{3.005282in}}{\pgfqpoint{1.995057in}{2.994682in}}{\pgfqpoint{1.995057in}{2.983632in}}%
\pgfpathcurveto{\pgfqpoint{1.995057in}{2.972582in}}{\pgfqpoint{1.999447in}{2.961983in}}{\pgfqpoint{2.007261in}{2.954170in}}%
\pgfpathcurveto{\pgfqpoint{2.015075in}{2.946356in}}{\pgfqpoint{2.025674in}{2.941966in}}{\pgfqpoint{2.036724in}{2.941966in}}%
\pgfpathclose%
\pgfusepath{stroke,fill}%
\end{pgfscope}%
\begin{pgfscope}%
\pgfpathrectangle{\pgfqpoint{0.564660in}{0.521603in}}{\pgfqpoint{4.650000in}{3.020000in}}%
\pgfusepath{clip}%
\pgfsetbuttcap%
\pgfsetroundjoin%
\definecolor{currentfill}{rgb}{1.000000,1.000000,1.000000}%
\pgfsetfillcolor{currentfill}%
\pgfsetlinewidth{1.003750pt}%
\definecolor{currentstroke}{rgb}{0.000000,0.000000,0.000000}%
\pgfsetstrokecolor{currentstroke}%
\pgfsetdash{}{0pt}%
\pgfpathmoveto{\pgfqpoint{1.309364in}{1.315672in}}%
\pgfpathcurveto{\pgfqpoint{1.320414in}{1.315672in}}{\pgfqpoint{1.331013in}{1.320062in}}{\pgfqpoint{1.338826in}{1.327876in}}%
\pgfpathcurveto{\pgfqpoint{1.346640in}{1.335690in}}{\pgfqpoint{1.351030in}{1.346289in}}{\pgfqpoint{1.351030in}{1.357339in}}%
\pgfpathcurveto{\pgfqpoint{1.351030in}{1.368389in}}{\pgfqpoint{1.346640in}{1.378988in}}{\pgfqpoint{1.338826in}{1.386802in}}%
\pgfpathcurveto{\pgfqpoint{1.331013in}{1.394615in}}{\pgfqpoint{1.320414in}{1.399006in}}{\pgfqpoint{1.309364in}{1.399006in}}%
\pgfpathcurveto{\pgfqpoint{1.298314in}{1.399006in}}{\pgfqpoint{1.287715in}{1.394615in}}{\pgfqpoint{1.279901in}{1.386802in}}%
\pgfpathcurveto{\pgfqpoint{1.272087in}{1.378988in}}{\pgfqpoint{1.267697in}{1.368389in}}{\pgfqpoint{1.267697in}{1.357339in}}%
\pgfpathcurveto{\pgfqpoint{1.267697in}{1.346289in}}{\pgfqpoint{1.272087in}{1.335690in}}{\pgfqpoint{1.279901in}{1.327876in}}%
\pgfpathcurveto{\pgfqpoint{1.287715in}{1.320062in}}{\pgfqpoint{1.298314in}{1.315672in}}{\pgfqpoint{1.309364in}{1.315672in}}%
\pgfpathclose%
\pgfusepath{stroke,fill}%
\end{pgfscope}%
\begin{pgfscope}%
\pgfsetbuttcap%
\pgfsetroundjoin%
\definecolor{currentfill}{rgb}{0.000000,0.000000,0.000000}%
\pgfsetfillcolor{currentfill}%
\pgfsetlinewidth{0.803000pt}%
\definecolor{currentstroke}{rgb}{0.000000,0.000000,0.000000}%
\pgfsetstrokecolor{currentstroke}%
\pgfsetdash{}{0pt}%
\pgfsys@defobject{currentmarker}{\pgfqpoint{0.000000in}{-0.048611in}}{\pgfqpoint{0.000000in}{0.000000in}}{%
\pgfpathmoveto{\pgfqpoint{0.000000in}{0.000000in}}%
\pgfpathlineto{\pgfqpoint{0.000000in}{-0.048611in}}%
\pgfusepath{stroke,fill}%
}%
\begin{pgfscope}%
\pgfsys@transformshift{0.667447in}{0.521603in}%
\pgfsys@useobject{currentmarker}{}%
\end{pgfscope}%
\end{pgfscope}%
\begin{pgfscope}%
\definecolor{textcolor}{rgb}{0.000000,0.000000,0.000000}%
\pgfsetstrokecolor{textcolor}%
\pgfsetfillcolor{textcolor}%
\pgftext[x=0.667447in,y=0.424381in,,top]{\color{textcolor}\rmfamily\fontsize{10.000000}{12.000000}\selectfont \(\displaystyle 0\)}%
\end{pgfscope}%
\begin{pgfscope}%
\pgfsetbuttcap%
\pgfsetroundjoin%
\definecolor{currentfill}{rgb}{0.000000,0.000000,0.000000}%
\pgfsetfillcolor{currentfill}%
\pgfsetlinewidth{0.803000pt}%
\definecolor{currentstroke}{rgb}{0.000000,0.000000,0.000000}%
\pgfsetstrokecolor{currentstroke}%
\pgfsetdash{}{0pt}%
\pgfsys@defobject{currentmarker}{\pgfqpoint{0.000000in}{-0.048611in}}{\pgfqpoint{0.000000in}{0.000000in}}{%
\pgfpathmoveto{\pgfqpoint{0.000000in}{0.000000in}}%
\pgfpathlineto{\pgfqpoint{0.000000in}{-0.048611in}}%
\pgfusepath{stroke,fill}%
}%
\begin{pgfscope}%
\pgfsys@transformshift{1.348050in}{0.521603in}%
\pgfsys@useobject{currentmarker}{}%
\end{pgfscope}%
\end{pgfscope}%
\begin{pgfscope}%
\definecolor{textcolor}{rgb}{0.000000,0.000000,0.000000}%
\pgfsetstrokecolor{textcolor}%
\pgfsetfillcolor{textcolor}%
\pgftext[x=1.348050in,y=0.424381in,,top]{\color{textcolor}\rmfamily\fontsize{10.000000}{12.000000}\selectfont \(\displaystyle 5\)}%
\end{pgfscope}%
\begin{pgfscope}%
\pgfsetbuttcap%
\pgfsetroundjoin%
\definecolor{currentfill}{rgb}{0.000000,0.000000,0.000000}%
\pgfsetfillcolor{currentfill}%
\pgfsetlinewidth{0.803000pt}%
\definecolor{currentstroke}{rgb}{0.000000,0.000000,0.000000}%
\pgfsetstrokecolor{currentstroke}%
\pgfsetdash{}{0pt}%
\pgfsys@defobject{currentmarker}{\pgfqpoint{0.000000in}{-0.048611in}}{\pgfqpoint{0.000000in}{0.000000in}}{%
\pgfpathmoveto{\pgfqpoint{0.000000in}{0.000000in}}%
\pgfpathlineto{\pgfqpoint{0.000000in}{-0.048611in}}%
\pgfusepath{stroke,fill}%
}%
\begin{pgfscope}%
\pgfsys@transformshift{2.028653in}{0.521603in}%
\pgfsys@useobject{currentmarker}{}%
\end{pgfscope}%
\end{pgfscope}%
\begin{pgfscope}%
\definecolor{textcolor}{rgb}{0.000000,0.000000,0.000000}%
\pgfsetstrokecolor{textcolor}%
\pgfsetfillcolor{textcolor}%
\pgftext[x=2.028653in,y=0.424381in,,top]{\color{textcolor}\rmfamily\fontsize{10.000000}{12.000000}\selectfont \(\displaystyle 10\)}%
\end{pgfscope}%
\begin{pgfscope}%
\pgfsetbuttcap%
\pgfsetroundjoin%
\definecolor{currentfill}{rgb}{0.000000,0.000000,0.000000}%
\pgfsetfillcolor{currentfill}%
\pgfsetlinewidth{0.803000pt}%
\definecolor{currentstroke}{rgb}{0.000000,0.000000,0.000000}%
\pgfsetstrokecolor{currentstroke}%
\pgfsetdash{}{0pt}%
\pgfsys@defobject{currentmarker}{\pgfqpoint{0.000000in}{-0.048611in}}{\pgfqpoint{0.000000in}{0.000000in}}{%
\pgfpathmoveto{\pgfqpoint{0.000000in}{0.000000in}}%
\pgfpathlineto{\pgfqpoint{0.000000in}{-0.048611in}}%
\pgfusepath{stroke,fill}%
}%
\begin{pgfscope}%
\pgfsys@transformshift{2.709256in}{0.521603in}%
\pgfsys@useobject{currentmarker}{}%
\end{pgfscope}%
\end{pgfscope}%
\begin{pgfscope}%
\definecolor{textcolor}{rgb}{0.000000,0.000000,0.000000}%
\pgfsetstrokecolor{textcolor}%
\pgfsetfillcolor{textcolor}%
\pgftext[x=2.709256in,y=0.424381in,,top]{\color{textcolor}\rmfamily\fontsize{10.000000}{12.000000}\selectfont \(\displaystyle 15\)}%
\end{pgfscope}%
\begin{pgfscope}%
\pgfsetbuttcap%
\pgfsetroundjoin%
\definecolor{currentfill}{rgb}{0.000000,0.000000,0.000000}%
\pgfsetfillcolor{currentfill}%
\pgfsetlinewidth{0.803000pt}%
\definecolor{currentstroke}{rgb}{0.000000,0.000000,0.000000}%
\pgfsetstrokecolor{currentstroke}%
\pgfsetdash{}{0pt}%
\pgfsys@defobject{currentmarker}{\pgfqpoint{0.000000in}{-0.048611in}}{\pgfqpoint{0.000000in}{0.000000in}}{%
\pgfpathmoveto{\pgfqpoint{0.000000in}{0.000000in}}%
\pgfpathlineto{\pgfqpoint{0.000000in}{-0.048611in}}%
\pgfusepath{stroke,fill}%
}%
\begin{pgfscope}%
\pgfsys@transformshift{3.389859in}{0.521603in}%
\pgfsys@useobject{currentmarker}{}%
\end{pgfscope}%
\end{pgfscope}%
\begin{pgfscope}%
\definecolor{textcolor}{rgb}{0.000000,0.000000,0.000000}%
\pgfsetstrokecolor{textcolor}%
\pgfsetfillcolor{textcolor}%
\pgftext[x=3.389859in,y=0.424381in,,top]{\color{textcolor}\rmfamily\fontsize{10.000000}{12.000000}\selectfont \(\displaystyle 20\)}%
\end{pgfscope}%
\begin{pgfscope}%
\pgfsetbuttcap%
\pgfsetroundjoin%
\definecolor{currentfill}{rgb}{0.000000,0.000000,0.000000}%
\pgfsetfillcolor{currentfill}%
\pgfsetlinewidth{0.803000pt}%
\definecolor{currentstroke}{rgb}{0.000000,0.000000,0.000000}%
\pgfsetstrokecolor{currentstroke}%
\pgfsetdash{}{0pt}%
\pgfsys@defobject{currentmarker}{\pgfqpoint{0.000000in}{-0.048611in}}{\pgfqpoint{0.000000in}{0.000000in}}{%
\pgfpathmoveto{\pgfqpoint{0.000000in}{0.000000in}}%
\pgfpathlineto{\pgfqpoint{0.000000in}{-0.048611in}}%
\pgfusepath{stroke,fill}%
}%
\begin{pgfscope}%
\pgfsys@transformshift{4.070462in}{0.521603in}%
\pgfsys@useobject{currentmarker}{}%
\end{pgfscope}%
\end{pgfscope}%
\begin{pgfscope}%
\definecolor{textcolor}{rgb}{0.000000,0.000000,0.000000}%
\pgfsetstrokecolor{textcolor}%
\pgfsetfillcolor{textcolor}%
\pgftext[x=4.070462in,y=0.424381in,,top]{\color{textcolor}\rmfamily\fontsize{10.000000}{12.000000}\selectfont \(\displaystyle 25\)}%
\end{pgfscope}%
\begin{pgfscope}%
\pgfsetbuttcap%
\pgfsetroundjoin%
\definecolor{currentfill}{rgb}{0.000000,0.000000,0.000000}%
\pgfsetfillcolor{currentfill}%
\pgfsetlinewidth{0.803000pt}%
\definecolor{currentstroke}{rgb}{0.000000,0.000000,0.000000}%
\pgfsetstrokecolor{currentstroke}%
\pgfsetdash{}{0pt}%
\pgfsys@defobject{currentmarker}{\pgfqpoint{0.000000in}{-0.048611in}}{\pgfqpoint{0.000000in}{0.000000in}}{%
\pgfpathmoveto{\pgfqpoint{0.000000in}{0.000000in}}%
\pgfpathlineto{\pgfqpoint{0.000000in}{-0.048611in}}%
\pgfusepath{stroke,fill}%
}%
\begin{pgfscope}%
\pgfsys@transformshift{4.751065in}{0.521603in}%
\pgfsys@useobject{currentmarker}{}%
\end{pgfscope}%
\end{pgfscope}%
\begin{pgfscope}%
\definecolor{textcolor}{rgb}{0.000000,0.000000,0.000000}%
\pgfsetstrokecolor{textcolor}%
\pgfsetfillcolor{textcolor}%
\pgftext[x=4.751065in,y=0.424381in,,top]{\color{textcolor}\rmfamily\fontsize{10.000000}{12.000000}\selectfont \(\displaystyle 30\)}%
\end{pgfscope}%
\begin{pgfscope}%
\definecolor{textcolor}{rgb}{0.000000,0.000000,0.000000}%
\pgfsetstrokecolor{textcolor}%
\pgfsetfillcolor{textcolor}%
\pgftext[x=2.889660in,y=0.234413in,,top]{\color{textcolor}\rmfamily\fontsize{10.000000}{12.000000}\selectfont \(\displaystyle \mathbf{E}\mbox{u}\)}%
\end{pgfscope}%
\begin{pgfscope}%
\pgfsetbuttcap%
\pgfsetroundjoin%
\definecolor{currentfill}{rgb}{0.000000,0.000000,0.000000}%
\pgfsetfillcolor{currentfill}%
\pgfsetlinewidth{0.803000pt}%
\definecolor{currentstroke}{rgb}{0.000000,0.000000,0.000000}%
\pgfsetstrokecolor{currentstroke}%
\pgfsetdash{}{0pt}%
\pgfsys@defobject{currentmarker}{\pgfqpoint{-0.048611in}{0.000000in}}{\pgfqpoint{0.000000in}{0.000000in}}{%
\pgfpathmoveto{\pgfqpoint{0.000000in}{0.000000in}}%
\pgfpathlineto{\pgfqpoint{-0.048611in}{0.000000in}}%
\pgfusepath{stroke,fill}%
}%
\begin{pgfscope}%
\pgfsys@transformshift{0.564660in}{1.017529in}%
\pgfsys@useobject{currentmarker}{}%
\end{pgfscope}%
\end{pgfscope}%
\begin{pgfscope}%
\definecolor{textcolor}{rgb}{0.000000,0.000000,0.000000}%
\pgfsetstrokecolor{textcolor}%
\pgfsetfillcolor{textcolor}%
\pgftext[x=0.289968in,y=0.964768in,left,base]{\color{textcolor}\rmfamily\fontsize{10.000000}{12.000000}\selectfont \(\displaystyle 0.3\)}%
\end{pgfscope}%
\begin{pgfscope}%
\pgfsetbuttcap%
\pgfsetroundjoin%
\definecolor{currentfill}{rgb}{0.000000,0.000000,0.000000}%
\pgfsetfillcolor{currentfill}%
\pgfsetlinewidth{0.803000pt}%
\definecolor{currentstroke}{rgb}{0.000000,0.000000,0.000000}%
\pgfsetstrokecolor{currentstroke}%
\pgfsetdash{}{0pt}%
\pgfsys@defobject{currentmarker}{\pgfqpoint{-0.048611in}{0.000000in}}{\pgfqpoint{0.000000in}{0.000000in}}{%
\pgfpathmoveto{\pgfqpoint{0.000000in}{0.000000in}}%
\pgfpathlineto{\pgfqpoint{-0.048611in}{0.000000in}}%
\pgfusepath{stroke,fill}%
}%
\begin{pgfscope}%
\pgfsys@transformshift{0.564660in}{1.618847in}%
\pgfsys@useobject{currentmarker}{}%
\end{pgfscope}%
\end{pgfscope}%
\begin{pgfscope}%
\definecolor{textcolor}{rgb}{0.000000,0.000000,0.000000}%
\pgfsetstrokecolor{textcolor}%
\pgfsetfillcolor{textcolor}%
\pgftext[x=0.289968in,y=1.566085in,left,base]{\color{textcolor}\rmfamily\fontsize{10.000000}{12.000000}\selectfont \(\displaystyle 0.4\)}%
\end{pgfscope}%
\begin{pgfscope}%
\pgfsetbuttcap%
\pgfsetroundjoin%
\definecolor{currentfill}{rgb}{0.000000,0.000000,0.000000}%
\pgfsetfillcolor{currentfill}%
\pgfsetlinewidth{0.803000pt}%
\definecolor{currentstroke}{rgb}{0.000000,0.000000,0.000000}%
\pgfsetstrokecolor{currentstroke}%
\pgfsetdash{}{0pt}%
\pgfsys@defobject{currentmarker}{\pgfqpoint{-0.048611in}{0.000000in}}{\pgfqpoint{0.000000in}{0.000000in}}{%
\pgfpathmoveto{\pgfqpoint{0.000000in}{0.000000in}}%
\pgfpathlineto{\pgfqpoint{-0.048611in}{0.000000in}}%
\pgfusepath{stroke,fill}%
}%
\begin{pgfscope}%
\pgfsys@transformshift{0.564660in}{2.220164in}%
\pgfsys@useobject{currentmarker}{}%
\end{pgfscope}%
\end{pgfscope}%
\begin{pgfscope}%
\definecolor{textcolor}{rgb}{0.000000,0.000000,0.000000}%
\pgfsetstrokecolor{textcolor}%
\pgfsetfillcolor{textcolor}%
\pgftext[x=0.289968in,y=2.167403in,left,base]{\color{textcolor}\rmfamily\fontsize{10.000000}{12.000000}\selectfont \(\displaystyle 0.5\)}%
\end{pgfscope}%
\begin{pgfscope}%
\pgfsetbuttcap%
\pgfsetroundjoin%
\definecolor{currentfill}{rgb}{0.000000,0.000000,0.000000}%
\pgfsetfillcolor{currentfill}%
\pgfsetlinewidth{0.803000pt}%
\definecolor{currentstroke}{rgb}{0.000000,0.000000,0.000000}%
\pgfsetstrokecolor{currentstroke}%
\pgfsetdash{}{0pt}%
\pgfsys@defobject{currentmarker}{\pgfqpoint{-0.048611in}{0.000000in}}{\pgfqpoint{0.000000in}{0.000000in}}{%
\pgfpathmoveto{\pgfqpoint{0.000000in}{0.000000in}}%
\pgfpathlineto{\pgfqpoint{-0.048611in}{0.000000in}}%
\pgfusepath{stroke,fill}%
}%
\begin{pgfscope}%
\pgfsys@transformshift{0.564660in}{2.821481in}%
\pgfsys@useobject{currentmarker}{}%
\end{pgfscope}%
\end{pgfscope}%
\begin{pgfscope}%
\definecolor{textcolor}{rgb}{0.000000,0.000000,0.000000}%
\pgfsetstrokecolor{textcolor}%
\pgfsetfillcolor{textcolor}%
\pgftext[x=0.289968in,y=2.768720in,left,base]{\color{textcolor}\rmfamily\fontsize{10.000000}{12.000000}\selectfont \(\displaystyle 0.6\)}%
\end{pgfscope}%
\begin{pgfscope}%
\pgfsetbuttcap%
\pgfsetroundjoin%
\definecolor{currentfill}{rgb}{0.000000,0.000000,0.000000}%
\pgfsetfillcolor{currentfill}%
\pgfsetlinewidth{0.803000pt}%
\definecolor{currentstroke}{rgb}{0.000000,0.000000,0.000000}%
\pgfsetstrokecolor{currentstroke}%
\pgfsetdash{}{0pt}%
\pgfsys@defobject{currentmarker}{\pgfqpoint{-0.048611in}{0.000000in}}{\pgfqpoint{0.000000in}{0.000000in}}{%
\pgfpathmoveto{\pgfqpoint{0.000000in}{0.000000in}}%
\pgfpathlineto{\pgfqpoint{-0.048611in}{0.000000in}}%
\pgfusepath{stroke,fill}%
}%
\begin{pgfscope}%
\pgfsys@transformshift{0.564660in}{3.422799in}%
\pgfsys@useobject{currentmarker}{}%
\end{pgfscope}%
\end{pgfscope}%
\begin{pgfscope}%
\definecolor{textcolor}{rgb}{0.000000,0.000000,0.000000}%
\pgfsetstrokecolor{textcolor}%
\pgfsetfillcolor{textcolor}%
\pgftext[x=0.289968in,y=3.370037in,left,base]{\color{textcolor}\rmfamily\fontsize{10.000000}{12.000000}\selectfont \(\displaystyle 0.7\)}%
\end{pgfscope}%
\begin{pgfscope}%
\definecolor{textcolor}{rgb}{0.000000,0.000000,0.000000}%
\pgfsetstrokecolor{textcolor}%
\pgfsetfillcolor{textcolor}%
\pgftext[x=0.234413in,y=2.031603in,,bottom,rotate=90.000000]{\color{textcolor}\rmfamily\fontsize{10.000000}{12.000000}\selectfont \(\displaystyle \mathbf{I}\mbox{g}\)}%
\end{pgfscope}%
\begin{pgfscope}%
\pgfpathrectangle{\pgfqpoint{0.564660in}{0.521603in}}{\pgfqpoint{4.650000in}{3.020000in}}%
\pgfusepath{clip}%
\pgfsetrectcap%
\pgfsetroundjoin%
\pgfsetlinewidth{1.505625pt}%
\definecolor{currentstroke}{rgb}{0.000000,0.000000,0.000000}%
\pgfsetstrokecolor{currentstroke}%
\pgfsetstrokeopacity{0.300000}%
\pgfsetdash{}{0pt}%
\pgfpathmoveto{\pgfqpoint{0.816771in}{1.337601in}}%
\pgfpathlineto{\pgfqpoint{3.858085in}{2.234101in}}%
\pgfpathlineto{\pgfqpoint{4.993344in}{2.568746in}}%
\pgfpathlineto{\pgfqpoint{1.914828in}{1.661279in}}%
\pgfpathlineto{\pgfqpoint{2.773519in}{1.914399in}}%
\pgfpathlineto{\pgfqpoint{2.234616in}{1.755544in}}%
\pgfpathlineto{\pgfqpoint{0.819079in}{1.338281in}}%
\pgfpathlineto{\pgfqpoint{1.363796in}{1.498849in}}%
\pgfpathlineto{\pgfqpoint{1.096147in}{1.419953in}}%
\pgfpathlineto{\pgfqpoint{0.887904in}{1.358569in}}%
\pgfpathlineto{\pgfqpoint{2.143771in}{1.728766in}}%
\pgfpathlineto{\pgfqpoint{1.086864in}{1.417217in}}%
\pgfpathlineto{\pgfqpoint{0.785977in}{1.328523in}}%
\pgfpathlineto{\pgfqpoint{2.036724in}{1.697211in}}%
\pgfpathlineto{\pgfqpoint{1.309364in}{1.482804in}}%
\pgfusepath{stroke}%
\end{pgfscope}%
\begin{pgfscope}%
\pgfsetrectcap%
\pgfsetmiterjoin%
\pgfsetlinewidth{0.803000pt}%
\definecolor{currentstroke}{rgb}{0.501961,0.501961,0.501961}%
\pgfsetstrokecolor{currentstroke}%
\pgfsetdash{}{0pt}%
\pgfpathmoveto{\pgfqpoint{0.564660in}{0.521603in}}%
\pgfpathlineto{\pgfqpoint{0.564660in}{3.541603in}}%
\pgfusepath{stroke}%
\end{pgfscope}%
\begin{pgfscope}%
\pgfsetrectcap%
\pgfsetmiterjoin%
\pgfsetlinewidth{0.803000pt}%
\definecolor{currentstroke}{rgb}{0.501961,0.501961,0.501961}%
\pgfsetstrokecolor{currentstroke}%
\pgfsetdash{}{0pt}%
\pgfpathmoveto{\pgfqpoint{5.214660in}{0.521603in}}%
\pgfpathlineto{\pgfqpoint{5.214660in}{3.541603in}}%
\pgfusepath{stroke}%
\end{pgfscope}%
\begin{pgfscope}%
\pgfsetrectcap%
\pgfsetmiterjoin%
\pgfsetlinewidth{0.803000pt}%
\definecolor{currentstroke}{rgb}{0.501961,0.501961,0.501961}%
\pgfsetstrokecolor{currentstroke}%
\pgfsetdash{}{0pt}%
\pgfpathmoveto{\pgfqpoint{0.564660in}{0.521603in}}%
\pgfpathlineto{\pgfqpoint{5.214660in}{0.521603in}}%
\pgfusepath{stroke}%
\end{pgfscope}%
\begin{pgfscope}%
\pgfsetrectcap%
\pgfsetmiterjoin%
\pgfsetlinewidth{0.803000pt}%
\definecolor{currentstroke}{rgb}{0.501961,0.501961,0.501961}%
\pgfsetstrokecolor{currentstroke}%
\pgfsetdash{}{0pt}%
\pgfpathmoveto{\pgfqpoint{0.564660in}{3.541603in}}%
\pgfpathlineto{\pgfqpoint{5.214660in}{3.541603in}}%
\pgfusepath{stroke}%
\end{pgfscope}%
\begin{pgfscope}%
\pgfsetbuttcap%
\pgfsetmiterjoin%
\definecolor{currentfill}{rgb}{1.000000,1.000000,1.000000}%
\pgfsetfillcolor{currentfill}%
\pgfsetfillopacity{0.800000}%
\pgfsetlinewidth{1.003750pt}%
\definecolor{currentstroke}{rgb}{0.800000,0.800000,0.800000}%
\pgfsetstrokecolor{currentstroke}%
\pgfsetstrokeopacity{0.800000}%
\pgfsetdash{}{0pt}%
\pgfpathmoveto{\pgfqpoint{2.661499in}{0.591048in}}%
\pgfpathlineto{\pgfqpoint{5.117438in}{0.591048in}}%
\pgfpathquadraticcurveto{\pgfqpoint{5.145216in}{0.591048in}}{\pgfqpoint{5.145216in}{0.618826in}}%
\pgfpathlineto{\pgfqpoint{5.145216in}{0.825238in}}%
\pgfpathquadraticcurveto{\pgfqpoint{5.145216in}{0.853016in}}{\pgfqpoint{5.117438in}{0.853016in}}%
\pgfpathlineto{\pgfqpoint{2.661499in}{0.853016in}}%
\pgfpathquadraticcurveto{\pgfqpoint{2.633721in}{0.853016in}}{\pgfqpoint{2.633721in}{0.825238in}}%
\pgfpathlineto{\pgfqpoint{2.633721in}{0.618826in}}%
\pgfpathquadraticcurveto{\pgfqpoint{2.633721in}{0.591048in}}{\pgfqpoint{2.661499in}{0.591048in}}%
\pgfpathclose%
\pgfusepath{stroke,fill}%
\end{pgfscope}%
\begin{pgfscope}%
\pgfsetrectcap%
\pgfsetroundjoin%
\pgfsetlinewidth{1.505625pt}%
\definecolor{currentstroke}{rgb}{0.000000,0.000000,0.000000}%
\pgfsetstrokecolor{currentstroke}%
\pgfsetstrokeopacity{0.300000}%
\pgfsetdash{}{0pt}%
\pgfpathmoveto{\pgfqpoint{2.689277in}{0.726071in}}%
\pgfpathlineto{\pgfqpoint{2.967055in}{0.726071in}}%
\pgfusepath{stroke}%
\end{pgfscope}%
\begin{pgfscope}%
\definecolor{textcolor}{rgb}{0.501961,0.501961,0.501961}%
\pgfsetstrokecolor{textcolor}%
\pgfsetfillcolor{textcolor}%
\pgftext[x=3.078166in,y=0.677460in,left,base]{\color{textcolor}\rmfamily\fontsize{10.000000}{12.000000}\selectfont \(\displaystyle \mathbf{I}\mbox{g} \approx 0.013 \mathbf{E}\mbox{u} + 0.230\), \(\displaystyle R^2=0.57\)}%
\end{pgfscope}%
\end{pgfpicture}%
\makeatother%
\endgroup%

    \caption{.\label{fig:dnumbs}}
\end{figure}

Finally we show several trajectories in the long-time scaled regime in Figure \ref{fig:series_l_ds}.
\begin{figure}[htb]
    \centering
    %% Creator: Matplotlib, PGF backend
%%
%% To include the figure in your LaTeX document, write
%%   \input{<filename>.pgf}
%%
%% Make sure the required packages are loaded in your preamble
%%   \usepackage{pgf}
%%
%% Figures using additional raster images can only be included by \input if
%% they are in the same directory as the main LaTeX file. For loading figures
%% from other directories you can use the `import` package
%%   \usepackage{import}
%% and then include the figures with
%%   \import{<path to file>}{<filename>.pgf}
%%
%% Matplotlib used the following preamble
%%   \usepackage{fontspec}
%%   \setmainfont{DejaVu Serif}
%%   \setsansfont{DejaVu Sans}
%%   \setmonofont{DejaVu Sans Mono}
%%
\begingroup%
\makeatletter%
\begin{pgfpicture}%
\pgfpathrectangle{\pgfpointorigin}{\pgfqpoint{5.665633in}{3.676603in}}%
\pgfusepath{use as bounding box, clip}%
\begin{pgfscope}%
\pgfsetbuttcap%
\pgfsetmiterjoin%
\definecolor{currentfill}{rgb}{1.000000,1.000000,1.000000}%
\pgfsetfillcolor{currentfill}%
\pgfsetlinewidth{0.000000pt}%
\definecolor{currentstroke}{rgb}{1.000000,1.000000,1.000000}%
\pgfsetstrokecolor{currentstroke}%
\pgfsetdash{}{0pt}%
\pgfpathmoveto{\pgfqpoint{0.000000in}{0.000000in}}%
\pgfpathlineto{\pgfqpoint{5.665633in}{0.000000in}}%
\pgfpathlineto{\pgfqpoint{5.665633in}{3.676603in}}%
\pgfpathlineto{\pgfqpoint{0.000000in}{3.676603in}}%
\pgfpathclose%
\pgfusepath{fill}%
\end{pgfscope}%
\begin{pgfscope}%
\pgfsetbuttcap%
\pgfsetmiterjoin%
\definecolor{currentfill}{rgb}{1.000000,1.000000,1.000000}%
\pgfsetfillcolor{currentfill}%
\pgfsetlinewidth{0.000000pt}%
\definecolor{currentstroke}{rgb}{0.000000,0.000000,0.000000}%
\pgfsetstrokecolor{currentstroke}%
\pgfsetstrokeopacity{0.000000}%
\pgfsetdash{}{0pt}%
\pgfpathmoveto{\pgfqpoint{0.526080in}{0.521603in}}%
\pgfpathlineto{\pgfqpoint{4.246080in}{0.521603in}}%
\pgfpathlineto{\pgfqpoint{4.246080in}{3.541603in}}%
\pgfpathlineto{\pgfqpoint{0.526080in}{3.541603in}}%
\pgfpathclose%
\pgfusepath{fill}%
\end{pgfscope}%
\begin{pgfscope}%
\pgfsetbuttcap%
\pgfsetroundjoin%
\definecolor{currentfill}{rgb}{0.000000,0.000000,0.000000}%
\pgfsetfillcolor{currentfill}%
\pgfsetlinewidth{0.803000pt}%
\definecolor{currentstroke}{rgb}{0.000000,0.000000,0.000000}%
\pgfsetstrokecolor{currentstroke}%
\pgfsetdash{}{0pt}%
\pgfsys@defobject{currentmarker}{\pgfqpoint{0.000000in}{-0.048611in}}{\pgfqpoint{0.000000in}{0.000000in}}{%
\pgfpathmoveto{\pgfqpoint{0.000000in}{0.000000in}}%
\pgfpathlineto{\pgfqpoint{0.000000in}{-0.048611in}}%
\pgfusepath{stroke,fill}%
}%
\begin{pgfscope}%
\pgfsys@transformshift{0.642375in}{0.521603in}%
\pgfsys@useobject{currentmarker}{}%
\end{pgfscope}%
\end{pgfscope}%
\begin{pgfscope}%
\pgftext[x=0.642375in,y=0.424381in,,top]{\rmfamily\fontsize{10.000000}{12.000000}\selectfont \(\displaystyle 0.0\)}%
\end{pgfscope}%
\begin{pgfscope}%
\pgfsetbuttcap%
\pgfsetroundjoin%
\definecolor{currentfill}{rgb}{0.000000,0.000000,0.000000}%
\pgfsetfillcolor{currentfill}%
\pgfsetlinewidth{0.803000pt}%
\definecolor{currentstroke}{rgb}{0.000000,0.000000,0.000000}%
\pgfsetstrokecolor{currentstroke}%
\pgfsetdash{}{0pt}%
\pgfsys@defobject{currentmarker}{\pgfqpoint{0.000000in}{-0.048611in}}{\pgfqpoint{0.000000in}{0.000000in}}{%
\pgfpathmoveto{\pgfqpoint{0.000000in}{0.000000in}}%
\pgfpathlineto{\pgfqpoint{0.000000in}{-0.048611in}}%
\pgfusepath{stroke,fill}%
}%
\begin{pgfscope}%
\pgfsys@transformshift{1.189375in}{0.521603in}%
\pgfsys@useobject{currentmarker}{}%
\end{pgfscope}%
\end{pgfscope}%
\begin{pgfscope}%
\pgftext[x=1.189375in,y=0.424381in,,top]{\rmfamily\fontsize{10.000000}{12.000000}\selectfont \(\displaystyle 2.5\)}%
\end{pgfscope}%
\begin{pgfscope}%
\pgfsetbuttcap%
\pgfsetroundjoin%
\definecolor{currentfill}{rgb}{0.000000,0.000000,0.000000}%
\pgfsetfillcolor{currentfill}%
\pgfsetlinewidth{0.803000pt}%
\definecolor{currentstroke}{rgb}{0.000000,0.000000,0.000000}%
\pgfsetstrokecolor{currentstroke}%
\pgfsetdash{}{0pt}%
\pgfsys@defobject{currentmarker}{\pgfqpoint{0.000000in}{-0.048611in}}{\pgfqpoint{0.000000in}{0.000000in}}{%
\pgfpathmoveto{\pgfqpoint{0.000000in}{0.000000in}}%
\pgfpathlineto{\pgfqpoint{0.000000in}{-0.048611in}}%
\pgfusepath{stroke,fill}%
}%
\begin{pgfscope}%
\pgfsys@transformshift{1.736375in}{0.521603in}%
\pgfsys@useobject{currentmarker}{}%
\end{pgfscope}%
\end{pgfscope}%
\begin{pgfscope}%
\pgftext[x=1.736375in,y=0.424381in,,top]{\rmfamily\fontsize{10.000000}{12.000000}\selectfont \(\displaystyle 5.0\)}%
\end{pgfscope}%
\begin{pgfscope}%
\pgfsetbuttcap%
\pgfsetroundjoin%
\definecolor{currentfill}{rgb}{0.000000,0.000000,0.000000}%
\pgfsetfillcolor{currentfill}%
\pgfsetlinewidth{0.803000pt}%
\definecolor{currentstroke}{rgb}{0.000000,0.000000,0.000000}%
\pgfsetstrokecolor{currentstroke}%
\pgfsetdash{}{0pt}%
\pgfsys@defobject{currentmarker}{\pgfqpoint{0.000000in}{-0.048611in}}{\pgfqpoint{0.000000in}{0.000000in}}{%
\pgfpathmoveto{\pgfqpoint{0.000000in}{0.000000in}}%
\pgfpathlineto{\pgfqpoint{0.000000in}{-0.048611in}}%
\pgfusepath{stroke,fill}%
}%
\begin{pgfscope}%
\pgfsys@transformshift{2.283375in}{0.521603in}%
\pgfsys@useobject{currentmarker}{}%
\end{pgfscope}%
\end{pgfscope}%
\begin{pgfscope}%
\pgftext[x=2.283375in,y=0.424381in,,top]{\rmfamily\fontsize{10.000000}{12.000000}\selectfont \(\displaystyle 7.5\)}%
\end{pgfscope}%
\begin{pgfscope}%
\pgfsetbuttcap%
\pgfsetroundjoin%
\definecolor{currentfill}{rgb}{0.000000,0.000000,0.000000}%
\pgfsetfillcolor{currentfill}%
\pgfsetlinewidth{0.803000pt}%
\definecolor{currentstroke}{rgb}{0.000000,0.000000,0.000000}%
\pgfsetstrokecolor{currentstroke}%
\pgfsetdash{}{0pt}%
\pgfsys@defobject{currentmarker}{\pgfqpoint{0.000000in}{-0.048611in}}{\pgfqpoint{0.000000in}{0.000000in}}{%
\pgfpathmoveto{\pgfqpoint{0.000000in}{0.000000in}}%
\pgfpathlineto{\pgfqpoint{0.000000in}{-0.048611in}}%
\pgfusepath{stroke,fill}%
}%
\begin{pgfscope}%
\pgfsys@transformshift{2.830375in}{0.521603in}%
\pgfsys@useobject{currentmarker}{}%
\end{pgfscope}%
\end{pgfscope}%
\begin{pgfscope}%
\pgftext[x=2.830375in,y=0.424381in,,top]{\rmfamily\fontsize{10.000000}{12.000000}\selectfont \(\displaystyle 10.0\)}%
\end{pgfscope}%
\begin{pgfscope}%
\pgfsetbuttcap%
\pgfsetroundjoin%
\definecolor{currentfill}{rgb}{0.000000,0.000000,0.000000}%
\pgfsetfillcolor{currentfill}%
\pgfsetlinewidth{0.803000pt}%
\definecolor{currentstroke}{rgb}{0.000000,0.000000,0.000000}%
\pgfsetstrokecolor{currentstroke}%
\pgfsetdash{}{0pt}%
\pgfsys@defobject{currentmarker}{\pgfqpoint{0.000000in}{-0.048611in}}{\pgfqpoint{0.000000in}{0.000000in}}{%
\pgfpathmoveto{\pgfqpoint{0.000000in}{0.000000in}}%
\pgfpathlineto{\pgfqpoint{0.000000in}{-0.048611in}}%
\pgfusepath{stroke,fill}%
}%
\begin{pgfscope}%
\pgfsys@transformshift{3.377375in}{0.521603in}%
\pgfsys@useobject{currentmarker}{}%
\end{pgfscope}%
\end{pgfscope}%
\begin{pgfscope}%
\pgftext[x=3.377375in,y=0.424381in,,top]{\rmfamily\fontsize{10.000000}{12.000000}\selectfont \(\displaystyle 12.5\)}%
\end{pgfscope}%
\begin{pgfscope}%
\pgfsetbuttcap%
\pgfsetroundjoin%
\definecolor{currentfill}{rgb}{0.000000,0.000000,0.000000}%
\pgfsetfillcolor{currentfill}%
\pgfsetlinewidth{0.803000pt}%
\definecolor{currentstroke}{rgb}{0.000000,0.000000,0.000000}%
\pgfsetstrokecolor{currentstroke}%
\pgfsetdash{}{0pt}%
\pgfsys@defobject{currentmarker}{\pgfqpoint{0.000000in}{-0.048611in}}{\pgfqpoint{0.000000in}{0.000000in}}{%
\pgfpathmoveto{\pgfqpoint{0.000000in}{0.000000in}}%
\pgfpathlineto{\pgfqpoint{0.000000in}{-0.048611in}}%
\pgfusepath{stroke,fill}%
}%
\begin{pgfscope}%
\pgfsys@transformshift{3.924375in}{0.521603in}%
\pgfsys@useobject{currentmarker}{}%
\end{pgfscope}%
\end{pgfscope}%
\begin{pgfscope}%
\pgftext[x=3.924375in,y=0.424381in,,top]{\rmfamily\fontsize{10.000000}{12.000000}\selectfont \(\displaystyle 15.0\)}%
\end{pgfscope}%
\begin{pgfscope}%
\pgftext[x=2.386080in,y=0.234413in,,top]{\rmfamily\fontsize{10.000000}{12.000000}\selectfont \(\displaystyle \bar{t}\)}%
\end{pgfscope}%
\begin{pgfscope}%
\pgfsetbuttcap%
\pgfsetroundjoin%
\definecolor{currentfill}{rgb}{0.000000,0.000000,0.000000}%
\pgfsetfillcolor{currentfill}%
\pgfsetlinewidth{0.803000pt}%
\definecolor{currentstroke}{rgb}{0.000000,0.000000,0.000000}%
\pgfsetstrokecolor{currentstroke}%
\pgfsetdash{}{0pt}%
\pgfsys@defobject{currentmarker}{\pgfqpoint{-0.048611in}{0.000000in}}{\pgfqpoint{0.000000in}{0.000000in}}{%
\pgfpathmoveto{\pgfqpoint{0.000000in}{0.000000in}}%
\pgfpathlineto{\pgfqpoint{-0.048611in}{0.000000in}}%
\pgfusepath{stroke,fill}%
}%
\begin{pgfscope}%
\pgfsys@transformshift{0.526080in}{0.624160in}%
\pgfsys@useobject{currentmarker}{}%
\end{pgfscope}%
\end{pgfscope}%
\begin{pgfscope}%
\pgftext[x=0.359413in,y=0.571398in,left,base]{\rmfamily\fontsize{10.000000}{12.000000}\selectfont \(\displaystyle 0\)}%
\end{pgfscope}%
\begin{pgfscope}%
\pgfsetbuttcap%
\pgfsetroundjoin%
\definecolor{currentfill}{rgb}{0.000000,0.000000,0.000000}%
\pgfsetfillcolor{currentfill}%
\pgfsetlinewidth{0.803000pt}%
\definecolor{currentstroke}{rgb}{0.000000,0.000000,0.000000}%
\pgfsetstrokecolor{currentstroke}%
\pgfsetdash{}{0pt}%
\pgfsys@defobject{currentmarker}{\pgfqpoint{-0.048611in}{0.000000in}}{\pgfqpoint{0.000000in}{0.000000in}}{%
\pgfpathmoveto{\pgfqpoint{0.000000in}{0.000000in}}%
\pgfpathlineto{\pgfqpoint{-0.048611in}{0.000000in}}%
\pgfusepath{stroke,fill}%
}%
\begin{pgfscope}%
\pgfsys@transformshift{0.526080in}{1.050912in}%
\pgfsys@useobject{currentmarker}{}%
\end{pgfscope}%
\end{pgfscope}%
\begin{pgfscope}%
\pgftext[x=0.359413in,y=0.998151in,left,base]{\rmfamily\fontsize{10.000000}{12.000000}\selectfont \(\displaystyle 2\)}%
\end{pgfscope}%
\begin{pgfscope}%
\pgfsetbuttcap%
\pgfsetroundjoin%
\definecolor{currentfill}{rgb}{0.000000,0.000000,0.000000}%
\pgfsetfillcolor{currentfill}%
\pgfsetlinewidth{0.803000pt}%
\definecolor{currentstroke}{rgb}{0.000000,0.000000,0.000000}%
\pgfsetstrokecolor{currentstroke}%
\pgfsetdash{}{0pt}%
\pgfsys@defobject{currentmarker}{\pgfqpoint{-0.048611in}{0.000000in}}{\pgfqpoint{0.000000in}{0.000000in}}{%
\pgfpathmoveto{\pgfqpoint{0.000000in}{0.000000in}}%
\pgfpathlineto{\pgfqpoint{-0.048611in}{0.000000in}}%
\pgfusepath{stroke,fill}%
}%
\begin{pgfscope}%
\pgfsys@transformshift{0.526080in}{1.477665in}%
\pgfsys@useobject{currentmarker}{}%
\end{pgfscope}%
\end{pgfscope}%
\begin{pgfscope}%
\pgftext[x=0.359413in,y=1.424903in,left,base]{\rmfamily\fontsize{10.000000}{12.000000}\selectfont \(\displaystyle 4\)}%
\end{pgfscope}%
\begin{pgfscope}%
\pgfsetbuttcap%
\pgfsetroundjoin%
\definecolor{currentfill}{rgb}{0.000000,0.000000,0.000000}%
\pgfsetfillcolor{currentfill}%
\pgfsetlinewidth{0.803000pt}%
\definecolor{currentstroke}{rgb}{0.000000,0.000000,0.000000}%
\pgfsetstrokecolor{currentstroke}%
\pgfsetdash{}{0pt}%
\pgfsys@defobject{currentmarker}{\pgfqpoint{-0.048611in}{0.000000in}}{\pgfqpoint{0.000000in}{0.000000in}}{%
\pgfpathmoveto{\pgfqpoint{0.000000in}{0.000000in}}%
\pgfpathlineto{\pgfqpoint{-0.048611in}{0.000000in}}%
\pgfusepath{stroke,fill}%
}%
\begin{pgfscope}%
\pgfsys@transformshift{0.526080in}{1.904417in}%
\pgfsys@useobject{currentmarker}{}%
\end{pgfscope}%
\end{pgfscope}%
\begin{pgfscope}%
\pgftext[x=0.359413in,y=1.851656in,left,base]{\rmfamily\fontsize{10.000000}{12.000000}\selectfont \(\displaystyle 6\)}%
\end{pgfscope}%
\begin{pgfscope}%
\pgfsetbuttcap%
\pgfsetroundjoin%
\definecolor{currentfill}{rgb}{0.000000,0.000000,0.000000}%
\pgfsetfillcolor{currentfill}%
\pgfsetlinewidth{0.803000pt}%
\definecolor{currentstroke}{rgb}{0.000000,0.000000,0.000000}%
\pgfsetstrokecolor{currentstroke}%
\pgfsetdash{}{0pt}%
\pgfsys@defobject{currentmarker}{\pgfqpoint{-0.048611in}{0.000000in}}{\pgfqpoint{0.000000in}{0.000000in}}{%
\pgfpathmoveto{\pgfqpoint{0.000000in}{0.000000in}}%
\pgfpathlineto{\pgfqpoint{-0.048611in}{0.000000in}}%
\pgfusepath{stroke,fill}%
}%
\begin{pgfscope}%
\pgfsys@transformshift{0.526080in}{2.331170in}%
\pgfsys@useobject{currentmarker}{}%
\end{pgfscope}%
\end{pgfscope}%
\begin{pgfscope}%
\pgftext[x=0.359413in,y=2.278408in,left,base]{\rmfamily\fontsize{10.000000}{12.000000}\selectfont \(\displaystyle 8\)}%
\end{pgfscope}%
\begin{pgfscope}%
\pgfsetbuttcap%
\pgfsetroundjoin%
\definecolor{currentfill}{rgb}{0.000000,0.000000,0.000000}%
\pgfsetfillcolor{currentfill}%
\pgfsetlinewidth{0.803000pt}%
\definecolor{currentstroke}{rgb}{0.000000,0.000000,0.000000}%
\pgfsetstrokecolor{currentstroke}%
\pgfsetdash{}{0pt}%
\pgfsys@defobject{currentmarker}{\pgfqpoint{-0.048611in}{0.000000in}}{\pgfqpoint{0.000000in}{0.000000in}}{%
\pgfpathmoveto{\pgfqpoint{0.000000in}{0.000000in}}%
\pgfpathlineto{\pgfqpoint{-0.048611in}{0.000000in}}%
\pgfusepath{stroke,fill}%
}%
\begin{pgfscope}%
\pgfsys@transformshift{0.526080in}{2.757923in}%
\pgfsys@useobject{currentmarker}{}%
\end{pgfscope}%
\end{pgfscope}%
\begin{pgfscope}%
\pgftext[x=0.289968in,y=2.705161in,left,base]{\rmfamily\fontsize{10.000000}{12.000000}\selectfont \(\displaystyle 10\)}%
\end{pgfscope}%
\begin{pgfscope}%
\pgfsetbuttcap%
\pgfsetroundjoin%
\definecolor{currentfill}{rgb}{0.000000,0.000000,0.000000}%
\pgfsetfillcolor{currentfill}%
\pgfsetlinewidth{0.803000pt}%
\definecolor{currentstroke}{rgb}{0.000000,0.000000,0.000000}%
\pgfsetstrokecolor{currentstroke}%
\pgfsetdash{}{0pt}%
\pgfsys@defobject{currentmarker}{\pgfqpoint{-0.048611in}{0.000000in}}{\pgfqpoint{0.000000in}{0.000000in}}{%
\pgfpathmoveto{\pgfqpoint{0.000000in}{0.000000in}}%
\pgfpathlineto{\pgfqpoint{-0.048611in}{0.000000in}}%
\pgfusepath{stroke,fill}%
}%
\begin{pgfscope}%
\pgfsys@transformshift{0.526080in}{3.184675in}%
\pgfsys@useobject{currentmarker}{}%
\end{pgfscope}%
\end{pgfscope}%
\begin{pgfscope}%
\pgftext[x=0.289968in,y=3.131914in,left,base]{\rmfamily\fontsize{10.000000}{12.000000}\selectfont \(\displaystyle 12\)}%
\end{pgfscope}%
\begin{pgfscope}%
\pgftext[x=0.234413in,y=2.031603in,,bottom,rotate=90.000000]{\rmfamily\fontsize{10.000000}{12.000000}\selectfont \(\displaystyle \bar{y}\)}%
\end{pgfscope}%
\begin{pgfscope}%
\pgfpathrectangle{\pgfqpoint{0.526080in}{0.521603in}}{\pgfqpoint{3.720000in}{3.020000in}} %
\pgfusepath{clip}%
\pgfsetrectcap%
\pgfsetroundjoin%
\pgfsetlinewidth{1.505625pt}%
\definecolor{currentstroke}{rgb}{1.000000,0.682749,0.366979}%
\pgfsetstrokecolor{currentstroke}%
\pgfsetdash{}{0pt}%
\pgfpathmoveto{\pgfqpoint{0.706228in}{0.663592in}}%
\pgfpathlineto{\pgfqpoint{0.714209in}{0.667809in}}%
\pgfpathlineto{\pgfqpoint{0.722191in}{0.674171in}}%
\pgfpathlineto{\pgfqpoint{0.730173in}{0.686079in}}%
\pgfpathlineto{\pgfqpoint{0.738154in}{0.692849in}}%
\pgfpathlineto{\pgfqpoint{0.746136in}{0.695567in}}%
\pgfpathlineto{\pgfqpoint{0.762099in}{0.714094in}}%
\pgfpathlineto{\pgfqpoint{0.786044in}{0.731841in}}%
\pgfpathlineto{\pgfqpoint{0.794025in}{0.739472in}}%
\pgfpathlineto{\pgfqpoint{0.817970in}{0.756162in}}%
\pgfpathlineto{\pgfqpoint{0.825952in}{0.763304in}}%
\pgfpathlineto{\pgfqpoint{0.849896in}{0.779623in}}%
\pgfpathlineto{\pgfqpoint{0.865860in}{0.790296in}}%
\pgfpathlineto{\pgfqpoint{0.873841in}{0.794770in}}%
\pgfpathlineto{\pgfqpoint{0.889804in}{0.806381in}}%
\pgfpathlineto{\pgfqpoint{0.905768in}{0.815071in}}%
\pgfpathlineto{\pgfqpoint{0.921731in}{0.825749in}}%
\pgfpathlineto{\pgfqpoint{0.937694in}{0.833826in}}%
\pgfpathlineto{\pgfqpoint{0.953657in}{0.843844in}}%
\pgfpathlineto{\pgfqpoint{0.969620in}{0.851630in}}%
\pgfpathlineto{\pgfqpoint{0.985583in}{0.860649in}}%
\pgfpathlineto{\pgfqpoint{1.001547in}{0.868000in}}%
\pgfpathlineto{\pgfqpoint{1.009528in}{0.872760in}}%
\pgfpathlineto{\pgfqpoint{1.065399in}{0.897941in}}%
\pgfpathlineto{\pgfqpoint{1.081362in}{0.904522in}}%
\pgfpathlineto{\pgfqpoint{1.161178in}{0.935486in}}%
\pgfpathlineto{\pgfqpoint{1.225031in}{0.956108in}}%
\pgfpathlineto{\pgfqpoint{1.336773in}{0.984407in}}%
\pgfpathlineto{\pgfqpoint{1.360718in}{0.989358in}}%
\pgfpathlineto{\pgfqpoint{1.408608in}{0.998560in}}%
\pgfpathlineto{\pgfqpoint{1.504387in}{1.011695in}}%
\pgfpathlineto{\pgfqpoint{1.608147in}{1.019363in}}%
\pgfpathlineto{\pgfqpoint{1.640074in}{1.020569in}}%
\pgfpathlineto{\pgfqpoint{1.687963in}{1.021125in}}%
\pgfpathlineto{\pgfqpoint{1.767779in}{1.018826in}}%
\pgfpathlineto{\pgfqpoint{1.815669in}{1.015558in}}%
\pgfpathlineto{\pgfqpoint{1.879521in}{1.008911in}}%
\pgfpathlineto{\pgfqpoint{1.935392in}{1.001017in}}%
\pgfpathlineto{\pgfqpoint{1.999245in}{0.989079in}}%
\pgfpathlineto{\pgfqpoint{2.055116in}{0.976184in}}%
\pgfpathlineto{\pgfqpoint{2.118969in}{0.958124in}}%
\pgfpathlineto{\pgfqpoint{2.174840in}{0.939380in}}%
\pgfpathlineto{\pgfqpoint{2.222730in}{0.921008in}}%
\pgfpathlineto{\pgfqpoint{2.270619in}{0.900458in}}%
\pgfpathlineto{\pgfqpoint{2.318509in}{0.877668in}}%
\pgfpathlineto{\pgfqpoint{2.366398in}{0.852369in}}%
\pgfpathlineto{\pgfqpoint{2.414288in}{0.824504in}}%
\pgfpathlineto{\pgfqpoint{2.462177in}{0.793768in}}%
\pgfpathlineto{\pgfqpoint{2.502085in}{0.765700in}}%
\pgfpathlineto{\pgfqpoint{2.502085in}{0.765700in}}%
\pgfusepath{stroke}%
\end{pgfscope}%
\begin{pgfscope}%
\pgfpathrectangle{\pgfqpoint{0.526080in}{0.521603in}}{\pgfqpoint{3.720000in}{3.020000in}} %
\pgfusepath{clip}%
\pgfsetrectcap%
\pgfsetroundjoin%
\pgfsetlinewidth{1.505625pt}%
\definecolor{currentstroke}{rgb}{1.000000,0.000000,0.000000}%
\pgfsetstrokecolor{currentstroke}%
\pgfsetdash{}{0pt}%
\pgfpathmoveto{\pgfqpoint{0.695171in}{0.658876in}}%
\pgfpathlineto{\pgfqpoint{0.770593in}{0.726383in}}%
\pgfpathlineto{\pgfqpoint{0.793220in}{0.745428in}}%
\pgfpathlineto{\pgfqpoint{0.868642in}{0.802821in}}%
\pgfpathlineto{\pgfqpoint{0.981776in}{0.876306in}}%
\pgfpathlineto{\pgfqpoint{1.072282in}{0.926083in}}%
\pgfpathlineto{\pgfqpoint{1.162789in}{0.969019in}}%
\pgfpathlineto{\pgfqpoint{1.230669in}{0.997081in}}%
\pgfpathlineto{\pgfqpoint{1.275923in}{1.014143in}}%
\pgfpathlineto{\pgfqpoint{1.351345in}{1.039585in}}%
\pgfpathlineto{\pgfqpoint{1.434309in}{1.063616in}}%
\pgfpathlineto{\pgfqpoint{1.487105in}{1.077118in}}%
\pgfpathlineto{\pgfqpoint{1.562527in}{1.093884in}}%
\pgfpathlineto{\pgfqpoint{1.622865in}{1.105308in}}%
\pgfpathlineto{\pgfqpoint{1.683203in}{1.115109in}}%
\pgfpathlineto{\pgfqpoint{1.758625in}{1.125078in}}%
\pgfpathlineto{\pgfqpoint{1.841590in}{1.133269in}}%
\pgfpathlineto{\pgfqpoint{1.917012in}{1.138211in}}%
\pgfpathlineto{\pgfqpoint{2.022604in}{1.140972in}}%
\pgfpathlineto{\pgfqpoint{2.098026in}{1.140013in}}%
\pgfpathlineto{\pgfqpoint{2.158364in}{1.137537in}}%
\pgfpathlineto{\pgfqpoint{2.241328in}{1.131323in}}%
\pgfpathlineto{\pgfqpoint{2.309208in}{1.123881in}}%
\pgfpathlineto{\pgfqpoint{2.369546in}{1.115189in}}%
\pgfpathlineto{\pgfqpoint{2.407257in}{1.108724in}}%
\pgfpathlineto{\pgfqpoint{2.407257in}{1.108724in}}%
\pgfusepath{stroke}%
\end{pgfscope}%
\begin{pgfscope}%
\pgfpathrectangle{\pgfqpoint{0.526080in}{0.521603in}}{\pgfqpoint{3.720000in}{3.020000in}} %
\pgfusepath{clip}%
\pgfsetrectcap%
\pgfsetroundjoin%
\pgfsetlinewidth{1.505625pt}%
\definecolor{currentstroke}{rgb}{0.605882,0.986201,0.645928}%
\pgfsetstrokecolor{currentstroke}%
\pgfsetdash{}{0pt}%
\pgfpathmoveto{\pgfqpoint{0.747633in}{0.697697in}}%
\pgfpathlineto{\pgfqpoint{0.757202in}{0.700571in}}%
\pgfpathlineto{\pgfqpoint{0.766770in}{0.713762in}}%
\pgfpathlineto{\pgfqpoint{0.776339in}{0.724791in}}%
\pgfpathlineto{\pgfqpoint{0.785908in}{0.731170in}}%
\pgfpathlineto{\pgfqpoint{0.795477in}{0.739259in}}%
\pgfpathlineto{\pgfqpoint{0.824184in}{0.768428in}}%
\pgfpathlineto{\pgfqpoint{0.833753in}{0.775034in}}%
\pgfpathlineto{\pgfqpoint{0.843321in}{0.783949in}}%
\pgfpathlineto{\pgfqpoint{0.852890in}{0.794440in}}%
\pgfpathlineto{\pgfqpoint{0.872028in}{0.809080in}}%
\pgfpathlineto{\pgfqpoint{0.881597in}{0.816029in}}%
\pgfpathlineto{\pgfqpoint{0.910303in}{0.841804in}}%
\pgfpathlineto{\pgfqpoint{0.919872in}{0.847079in}}%
\pgfpathlineto{\pgfqpoint{0.948579in}{0.871614in}}%
\pgfpathlineto{\pgfqpoint{0.967717in}{0.883426in}}%
\pgfpathlineto{\pgfqpoint{0.996423in}{0.906291in}}%
\pgfpathlineto{\pgfqpoint{1.005992in}{0.911030in}}%
\pgfpathlineto{\pgfqpoint{1.034699in}{0.933130in}}%
\pgfpathlineto{\pgfqpoint{1.053836in}{0.943650in}}%
\pgfpathlineto{\pgfqpoint{1.082543in}{0.963712in}}%
\pgfpathlineto{\pgfqpoint{1.092112in}{0.968323in}}%
\pgfpathlineto{\pgfqpoint{1.120819in}{0.988103in}}%
\pgfpathlineto{\pgfqpoint{1.139956in}{0.997540in}}%
\pgfpathlineto{\pgfqpoint{1.168663in}{1.015805in}}%
\pgfpathlineto{\pgfqpoint{1.178232in}{1.019762in}}%
\pgfpathlineto{\pgfqpoint{1.206938in}{1.037936in}}%
\pgfpathlineto{\pgfqpoint{1.226076in}{1.046499in}}%
\pgfpathlineto{\pgfqpoint{1.254783in}{1.062878in}}%
\pgfpathlineto{\pgfqpoint{1.264352in}{1.066857in}}%
\pgfpathlineto{\pgfqpoint{1.293058in}{1.083342in}}%
\pgfpathlineto{\pgfqpoint{1.312196in}{1.091194in}}%
\pgfpathlineto{\pgfqpoint{1.340903in}{1.106045in}}%
\pgfpathlineto{\pgfqpoint{1.360040in}{1.114829in}}%
\pgfpathlineto{\pgfqpoint{1.379178in}{1.124997in}}%
\pgfpathlineto{\pgfqpoint{1.398316in}{1.132263in}}%
\pgfpathlineto{\pgfqpoint{1.417453in}{1.142654in}}%
\pgfpathlineto{\pgfqpoint{1.436591in}{1.149577in}}%
\pgfpathlineto{\pgfqpoint{1.465298in}{1.163527in}}%
\pgfpathlineto{\pgfqpoint{1.484436in}{1.170224in}}%
\pgfpathlineto{\pgfqpoint{1.503573in}{1.179814in}}%
\pgfpathlineto{\pgfqpoint{1.522711in}{1.186228in}}%
\pgfpathlineto{\pgfqpoint{1.551418in}{1.199089in}}%
\pgfpathlineto{\pgfqpoint{1.570555in}{1.205282in}}%
\pgfpathlineto{\pgfqpoint{1.589693in}{1.214081in}}%
\pgfpathlineto{\pgfqpoint{1.608831in}{1.220120in}}%
\pgfpathlineto{\pgfqpoint{1.637537in}{1.232013in}}%
\pgfpathlineto{\pgfqpoint{1.656675in}{1.237882in}}%
\pgfpathlineto{\pgfqpoint{1.675813in}{1.245835in}}%
\pgfpathlineto{\pgfqpoint{1.694951in}{1.251601in}}%
\pgfpathlineto{\pgfqpoint{1.723657in}{1.262487in}}%
\pgfpathlineto{\pgfqpoint{1.742795in}{1.268164in}}%
\pgfpathlineto{\pgfqpoint{1.761933in}{1.275367in}}%
\pgfpathlineto{\pgfqpoint{1.790639in}{1.284273in}}%
\pgfpathlineto{\pgfqpoint{1.809777in}{1.290793in}}%
\pgfpathlineto{\pgfqpoint{1.828915in}{1.296117in}}%
\pgfpathlineto{\pgfqpoint{1.848053in}{1.302888in}}%
\pgfpathlineto{\pgfqpoint{1.876759in}{1.311158in}}%
\pgfpathlineto{\pgfqpoint{1.895897in}{1.317257in}}%
\pgfpathlineto{\pgfqpoint{1.915035in}{1.322242in}}%
\pgfpathlineto{\pgfqpoint{1.934172in}{1.328602in}}%
\pgfpathlineto{\pgfqpoint{1.962879in}{1.336425in}}%
\pgfpathlineto{\pgfqpoint{1.982017in}{1.341991in}}%
\pgfpathlineto{\pgfqpoint{2.001154in}{1.346788in}}%
\pgfpathlineto{\pgfqpoint{2.020292in}{1.352746in}}%
\pgfpathlineto{\pgfqpoint{2.048999in}{1.360126in}}%
\pgfpathlineto{\pgfqpoint{2.068137in}{1.365082in}}%
\pgfpathlineto{\pgfqpoint{2.087274in}{1.369761in}}%
\pgfpathlineto{\pgfqpoint{2.106412in}{1.375228in}}%
\pgfpathlineto{\pgfqpoint{2.135119in}{1.382148in}}%
\pgfpathlineto{\pgfqpoint{2.154256in}{1.386705in}}%
\pgfpathlineto{\pgfqpoint{2.173394in}{1.391207in}}%
\pgfpathlineto{\pgfqpoint{2.192532in}{1.396172in}}%
\pgfpathlineto{\pgfqpoint{2.211670in}{1.400078in}}%
\pgfpathlineto{\pgfqpoint{2.240376in}{1.406880in}}%
\pgfpathlineto{\pgfqpoint{2.259514in}{1.411237in}}%
\pgfpathlineto{\pgfqpoint{2.278652in}{1.415806in}}%
\pgfpathlineto{\pgfqpoint{2.307358in}{1.421830in}}%
\pgfpathlineto{\pgfqpoint{2.355203in}{1.432315in}}%
\pgfpathlineto{\pgfqpoint{2.623131in}{1.481874in}}%
\pgfpathlineto{\pgfqpoint{2.670975in}{1.489390in}}%
\pgfpathlineto{\pgfqpoint{2.737957in}{1.500038in}}%
\pgfpathlineto{\pgfqpoint{2.881490in}{1.519969in}}%
\pgfpathlineto{\pgfqpoint{2.881490in}{1.519969in}}%
\pgfusepath{stroke}%
\end{pgfscope}%
\begin{pgfscope}%
\pgfpathrectangle{\pgfqpoint{0.526080in}{0.521603in}}{\pgfqpoint{3.720000in}{3.020000in}} %
\pgfusepath{clip}%
\pgfsetrectcap%
\pgfsetroundjoin%
\pgfsetlinewidth{1.505625pt}%
\definecolor{currentstroke}{rgb}{0.131373,0.547220,0.958381}%
\pgfsetstrokecolor{currentstroke}%
\pgfsetdash{}{0pt}%
\pgfpathmoveto{\pgfqpoint{0.765106in}{0.683992in}}%
\pgfpathlineto{\pgfqpoint{0.806017in}{0.726957in}}%
\pgfpathlineto{\pgfqpoint{0.846927in}{0.760770in}}%
\pgfpathlineto{\pgfqpoint{0.887837in}{0.804022in}}%
\pgfpathlineto{\pgfqpoint{0.949203in}{0.854810in}}%
\pgfpathlineto{\pgfqpoint{0.990113in}{0.896510in}}%
\pgfpathlineto{\pgfqpoint{1.010568in}{0.910665in}}%
\pgfpathlineto{\pgfqpoint{1.031024in}{0.927219in}}%
\pgfpathlineto{\pgfqpoint{1.071934in}{0.967690in}}%
\pgfpathlineto{\pgfqpoint{1.112844in}{0.997591in}}%
\pgfpathlineto{\pgfqpoint{1.133300in}{1.014972in}}%
\pgfpathlineto{\pgfqpoint{1.153755in}{1.035542in}}%
\pgfpathlineto{\pgfqpoint{1.174210in}{1.052607in}}%
\pgfpathlineto{\pgfqpoint{1.194665in}{1.066297in}}%
\pgfpathlineto{\pgfqpoint{1.215120in}{1.082500in}}%
\pgfpathlineto{\pgfqpoint{1.235576in}{1.102136in}}%
\pgfpathlineto{\pgfqpoint{1.256031in}{1.119579in}}%
\pgfpathlineto{\pgfqpoint{1.296941in}{1.148045in}}%
\pgfpathlineto{\pgfqpoint{1.358307in}{1.199268in}}%
\pgfpathlineto{\pgfqpoint{1.378762in}{1.212600in}}%
\pgfpathlineto{\pgfqpoint{1.399217in}{1.228878in}}%
\pgfpathlineto{\pgfqpoint{1.419672in}{1.246810in}}%
\pgfpathlineto{\pgfqpoint{1.440127in}{1.262625in}}%
\pgfpathlineto{\pgfqpoint{1.481038in}{1.290327in}}%
\pgfpathlineto{\pgfqpoint{1.521948in}{1.324108in}}%
\pgfpathlineto{\pgfqpoint{1.562858in}{1.351149in}}%
\pgfpathlineto{\pgfqpoint{1.624224in}{1.398735in}}%
\pgfpathlineto{\pgfqpoint{1.665134in}{1.426044in}}%
\pgfpathlineto{\pgfqpoint{1.706045in}{1.457655in}}%
\pgfpathlineto{\pgfqpoint{1.746955in}{1.484071in}}%
\pgfpathlineto{\pgfqpoint{1.787866in}{1.515653in}}%
\pgfpathlineto{\pgfqpoint{1.849231in}{1.556147in}}%
\pgfpathlineto{\pgfqpoint{1.869686in}{1.571973in}}%
\pgfpathlineto{\pgfqpoint{1.992417in}{1.654784in}}%
\pgfpathlineto{\pgfqpoint{2.012873in}{1.667715in}}%
\pgfpathlineto{\pgfqpoint{2.074238in}{1.709613in}}%
\pgfpathlineto{\pgfqpoint{2.094693in}{1.722158in}}%
\pgfpathlineto{\pgfqpoint{2.176514in}{1.776832in}}%
\pgfpathlineto{\pgfqpoint{2.217424in}{1.803976in}}%
\pgfpathlineto{\pgfqpoint{2.237880in}{1.817858in}}%
\pgfpathlineto{\pgfqpoint{2.278790in}{1.842834in}}%
\pgfpathlineto{\pgfqpoint{2.340156in}{1.883672in}}%
\pgfpathlineto{\pgfqpoint{2.381066in}{1.909010in}}%
\pgfpathlineto{\pgfqpoint{2.421976in}{1.935929in}}%
\pgfpathlineto{\pgfqpoint{2.462887in}{1.960817in}}%
\pgfpathlineto{\pgfqpoint{2.503797in}{1.988032in}}%
\pgfpathlineto{\pgfqpoint{2.565163in}{2.025834in}}%
\pgfpathlineto{\pgfqpoint{2.606073in}{2.051615in}}%
\pgfpathlineto{\pgfqpoint{2.646983in}{2.076576in}}%
\pgfpathlineto{\pgfqpoint{2.687894in}{2.102878in}}%
\pgfpathlineto{\pgfqpoint{2.728804in}{2.127389in}}%
\pgfpathlineto{\pgfqpoint{2.790170in}{2.165352in}}%
\pgfpathlineto{\pgfqpoint{2.831080in}{2.190259in}}%
\pgfpathlineto{\pgfqpoint{2.871990in}{2.215649in}}%
\pgfpathlineto{\pgfqpoint{2.994721in}{2.289342in}}%
\pgfpathlineto{\pgfqpoint{3.056087in}{2.326774in}}%
\pgfpathlineto{\pgfqpoint{3.587922in}{2.640101in}}%
\pgfpathlineto{\pgfqpoint{3.608377in}{2.651714in}}%
\pgfpathlineto{\pgfqpoint{3.608377in}{2.651714in}}%
\pgfusepath{stroke}%
\end{pgfscope}%
\begin{pgfscope}%
\pgfpathrectangle{\pgfqpoint{0.526080in}{0.521603in}}{\pgfqpoint{3.720000in}{3.020000in}} %
\pgfusepath{clip}%
\pgfsetrectcap%
\pgfsetroundjoin%
\pgfsetlinewidth{1.505625pt}%
\definecolor{currentstroke}{rgb}{0.500000,0.000000,1.000000}%
\pgfsetstrokecolor{currentstroke}%
\pgfsetdash{}{0pt}%
\pgfpathmoveto{\pgfqpoint{0.897842in}{0.851531in}}%
\pgfpathlineto{\pgfqpoint{0.926228in}{0.874721in}}%
\pgfpathlineto{\pgfqpoint{0.954613in}{0.897808in}}%
\pgfpathlineto{\pgfqpoint{0.982998in}{0.930043in}}%
\pgfpathlineto{\pgfqpoint{1.011383in}{0.959535in}}%
\pgfpathlineto{\pgfqpoint{1.039768in}{0.984673in}}%
\pgfpathlineto{\pgfqpoint{1.068154in}{1.010878in}}%
\pgfpathlineto{\pgfqpoint{1.096539in}{1.034443in}}%
\pgfpathlineto{\pgfqpoint{1.124924in}{1.062436in}}%
\pgfpathlineto{\pgfqpoint{1.153309in}{1.092331in}}%
\pgfpathlineto{\pgfqpoint{1.181695in}{1.115067in}}%
\pgfpathlineto{\pgfqpoint{1.210080in}{1.140018in}}%
\pgfpathlineto{\pgfqpoint{1.238465in}{1.170494in}}%
\pgfpathlineto{\pgfqpoint{1.266850in}{1.197819in}}%
\pgfpathlineto{\pgfqpoint{1.295236in}{1.221825in}}%
\pgfpathlineto{\pgfqpoint{1.323621in}{1.242591in}}%
\pgfpathlineto{\pgfqpoint{1.352006in}{1.269830in}}%
\pgfpathlineto{\pgfqpoint{1.380391in}{1.298104in}}%
\pgfpathlineto{\pgfqpoint{1.408777in}{1.324582in}}%
\pgfpathlineto{\pgfqpoint{1.437162in}{1.347233in}}%
\pgfpathlineto{\pgfqpoint{1.465547in}{1.372548in}}%
\pgfpathlineto{\pgfqpoint{1.493932in}{1.399924in}}%
\pgfpathlineto{\pgfqpoint{1.522318in}{1.424375in}}%
\pgfpathlineto{\pgfqpoint{1.550703in}{1.446256in}}%
\pgfpathlineto{\pgfqpoint{1.579088in}{1.470940in}}%
\pgfpathlineto{\pgfqpoint{1.607473in}{1.496776in}}%
\pgfpathlineto{\pgfqpoint{1.635858in}{1.522881in}}%
\pgfpathlineto{\pgfqpoint{1.664244in}{1.543645in}}%
\pgfpathlineto{\pgfqpoint{1.692629in}{1.568456in}}%
\pgfpathlineto{\pgfqpoint{1.721014in}{1.594362in}}%
\pgfpathlineto{\pgfqpoint{1.749399in}{1.618943in}}%
\pgfpathlineto{\pgfqpoint{1.777785in}{1.641122in}}%
\pgfpathlineto{\pgfqpoint{1.806170in}{1.663719in}}%
\pgfpathlineto{\pgfqpoint{1.834555in}{1.688229in}}%
\pgfpathlineto{\pgfqpoint{1.862940in}{1.713460in}}%
\pgfpathlineto{\pgfqpoint{1.891326in}{1.733690in}}%
\pgfpathlineto{\pgfqpoint{1.919711in}{1.758754in}}%
\pgfpathlineto{\pgfqpoint{1.948096in}{1.782359in}}%
\pgfpathlineto{\pgfqpoint{1.976481in}{1.806856in}}%
\pgfpathlineto{\pgfqpoint{2.004867in}{1.829283in}}%
\pgfpathlineto{\pgfqpoint{2.033252in}{1.851166in}}%
\pgfpathlineto{\pgfqpoint{2.061637in}{1.872011in}}%
\pgfpathlineto{\pgfqpoint{2.090022in}{1.897080in}}%
\pgfpathlineto{\pgfqpoint{2.118408in}{1.918923in}}%
\pgfpathlineto{\pgfqpoint{2.146793in}{1.943433in}}%
\pgfpathlineto{\pgfqpoint{2.175178in}{1.966741in}}%
\pgfpathlineto{\pgfqpoint{2.203563in}{1.990779in}}%
\pgfpathlineto{\pgfqpoint{2.231948in}{2.013570in}}%
\pgfpathlineto{\pgfqpoint{2.260334in}{2.033832in}}%
\pgfpathlineto{\pgfqpoint{2.288719in}{2.053813in}}%
\pgfpathlineto{\pgfqpoint{2.317104in}{2.077674in}}%
\pgfpathlineto{\pgfqpoint{2.345489in}{2.100351in}}%
\pgfpathlineto{\pgfqpoint{2.373875in}{2.123695in}}%
\pgfpathlineto{\pgfqpoint{2.402260in}{2.147246in}}%
\pgfpathlineto{\pgfqpoint{2.430645in}{2.170564in}}%
\pgfpathlineto{\pgfqpoint{2.459030in}{2.193500in}}%
\pgfpathlineto{\pgfqpoint{2.487416in}{2.213515in}}%
\pgfpathlineto{\pgfqpoint{2.515801in}{2.232974in}}%
\pgfpathlineto{\pgfqpoint{2.544186in}{2.254460in}}%
\pgfpathlineto{\pgfqpoint{2.572571in}{2.278011in}}%
\pgfpathlineto{\pgfqpoint{2.600957in}{2.301097in}}%
\pgfpathlineto{\pgfqpoint{2.629342in}{2.325568in}}%
\pgfpathlineto{\pgfqpoint{2.657727in}{2.348175in}}%
\pgfpathlineto{\pgfqpoint{2.686112in}{2.370803in}}%
\pgfpathlineto{\pgfqpoint{2.714498in}{2.390295in}}%
\pgfpathlineto{\pgfqpoint{2.742883in}{2.409761in}}%
\pgfpathlineto{\pgfqpoint{2.771268in}{2.430227in}}%
\pgfpathlineto{\pgfqpoint{2.799653in}{2.452762in}}%
\pgfpathlineto{\pgfqpoint{2.828038in}{2.476669in}}%
\pgfpathlineto{\pgfqpoint{2.856424in}{2.501393in}}%
\pgfpathlineto{\pgfqpoint{2.884809in}{2.523432in}}%
\pgfpathlineto{\pgfqpoint{2.913194in}{2.545381in}}%
\pgfpathlineto{\pgfqpoint{2.941579in}{2.564949in}}%
\pgfpathlineto{\pgfqpoint{2.969965in}{2.583321in}}%
\pgfpathlineto{\pgfqpoint{2.998350in}{2.603499in}}%
\pgfpathlineto{\pgfqpoint{3.026735in}{2.624811in}}%
\pgfpathlineto{\pgfqpoint{3.055120in}{2.649022in}}%
\pgfpathlineto{\pgfqpoint{3.083506in}{2.673144in}}%
\pgfpathlineto{\pgfqpoint{3.111891in}{2.695830in}}%
\pgfpathlineto{\pgfqpoint{3.140276in}{2.717813in}}%
\pgfpathlineto{\pgfqpoint{3.168661in}{2.736500in}}%
\pgfpathlineto{\pgfqpoint{3.197047in}{2.754509in}}%
\pgfpathlineto{\pgfqpoint{3.225432in}{2.774623in}}%
\pgfpathlineto{\pgfqpoint{3.253817in}{2.796039in}}%
\pgfpathlineto{\pgfqpoint{3.282202in}{2.819343in}}%
\pgfpathlineto{\pgfqpoint{3.310588in}{2.843921in}}%
\pgfpathlineto{\pgfqpoint{3.338973in}{2.865888in}}%
\pgfpathlineto{\pgfqpoint{3.367358in}{2.887151in}}%
\pgfpathlineto{\pgfqpoint{3.395743in}{2.906588in}}%
\pgfpathlineto{\pgfqpoint{3.424128in}{2.923607in}}%
\pgfpathlineto{\pgfqpoint{3.452514in}{2.943185in}}%
\pgfpathlineto{\pgfqpoint{3.480899in}{2.964021in}}%
\pgfpathlineto{\pgfqpoint{3.509284in}{2.987010in}}%
\pgfpathlineto{\pgfqpoint{3.537669in}{3.011940in}}%
\pgfpathlineto{\pgfqpoint{3.566055in}{3.033738in}}%
\pgfpathlineto{\pgfqpoint{3.594440in}{3.054424in}}%
\pgfpathlineto{\pgfqpoint{3.622825in}{3.074806in}}%
\pgfpathlineto{\pgfqpoint{3.651210in}{3.091367in}}%
\pgfpathlineto{\pgfqpoint{3.679596in}{3.110316in}}%
\pgfpathlineto{\pgfqpoint{3.707981in}{3.131102in}}%
\pgfpathlineto{\pgfqpoint{3.736366in}{3.154478in}}%
\pgfpathlineto{\pgfqpoint{3.764751in}{3.178561in}}%
\pgfpathlineto{\pgfqpoint{3.793137in}{3.199724in}}%
\pgfpathlineto{\pgfqpoint{3.821522in}{3.219976in}}%
\pgfpathlineto{\pgfqpoint{3.849907in}{3.240613in}}%
\pgfpathlineto{\pgfqpoint{3.878292in}{3.257596in}}%
\pgfpathlineto{\pgfqpoint{3.906678in}{3.275667in}}%
\pgfpathlineto{\pgfqpoint{3.935063in}{3.296679in}}%
\pgfpathlineto{\pgfqpoint{3.963448in}{3.320218in}}%
\pgfpathlineto{\pgfqpoint{3.991833in}{3.344027in}}%
\pgfpathlineto{\pgfqpoint{4.020218in}{3.363704in}}%
\pgfpathlineto{\pgfqpoint{4.048604in}{3.385783in}}%
\pgfpathlineto{\pgfqpoint{4.076989in}{3.404331in}}%
\pgfusepath{stroke}%
\end{pgfscope}%
\begin{pgfscope}%
\pgfsetrectcap%
\pgfsetmiterjoin%
\pgfsetlinewidth{0.803000pt}%
\definecolor{currentstroke}{rgb}{0.000000,0.000000,0.000000}%
\pgfsetstrokecolor{currentstroke}%
\pgfsetdash{}{0pt}%
\pgfpathmoveto{\pgfqpoint{0.526080in}{0.521603in}}%
\pgfpathlineto{\pgfqpoint{0.526080in}{3.541603in}}%
\pgfusepath{stroke}%
\end{pgfscope}%
\begin{pgfscope}%
\pgfsetrectcap%
\pgfsetmiterjoin%
\pgfsetlinewidth{0.803000pt}%
\definecolor{currentstroke}{rgb}{0.000000,0.000000,0.000000}%
\pgfsetstrokecolor{currentstroke}%
\pgfsetdash{}{0pt}%
\pgfpathmoveto{\pgfqpoint{4.246080in}{0.521603in}}%
\pgfpathlineto{\pgfqpoint{4.246080in}{3.541603in}}%
\pgfusepath{stroke}%
\end{pgfscope}%
\begin{pgfscope}%
\pgfsetrectcap%
\pgfsetmiterjoin%
\pgfsetlinewidth{0.803000pt}%
\definecolor{currentstroke}{rgb}{0.000000,0.000000,0.000000}%
\pgfsetstrokecolor{currentstroke}%
\pgfsetdash{}{0pt}%
\pgfpathmoveto{\pgfqpoint{0.526080in}{0.521603in}}%
\pgfpathlineto{\pgfqpoint{4.246080in}{0.521603in}}%
\pgfusepath{stroke}%
\end{pgfscope}%
\begin{pgfscope}%
\pgfsetrectcap%
\pgfsetmiterjoin%
\pgfsetlinewidth{0.803000pt}%
\definecolor{currentstroke}{rgb}{0.000000,0.000000,0.000000}%
\pgfsetstrokecolor{currentstroke}%
\pgfsetdash{}{0pt}%
\pgfpathmoveto{\pgfqpoint{0.526080in}{3.541603in}}%
\pgfpathlineto{\pgfqpoint{4.246080in}{3.541603in}}%
\pgfusepath{stroke}%
\end{pgfscope}%
\begin{pgfscope}%
\pgfpathrectangle{\pgfqpoint{4.478580in}{0.521603in}}{\pgfqpoint{0.151000in}{3.020000in}} %
\pgfusepath{clip}%
\pgfsetbuttcap%
\pgfsetmiterjoin%
\definecolor{currentfill}{rgb}{1.000000,1.000000,1.000000}%
\pgfsetfillcolor{currentfill}%
\pgfsetlinewidth{0.010037pt}%
\definecolor{currentstroke}{rgb}{1.000000,1.000000,1.000000}%
\pgfsetstrokecolor{currentstroke}%
\pgfsetdash{}{0pt}%
\pgfpathmoveto{\pgfqpoint{4.478580in}{0.521603in}}%
\pgfpathlineto{\pgfqpoint{4.478580in}{0.533400in}}%
\pgfpathlineto{\pgfqpoint{4.478580in}{3.529806in}}%
\pgfpathlineto{\pgfqpoint{4.478580in}{3.541603in}}%
\pgfpathlineto{\pgfqpoint{4.629580in}{3.541603in}}%
\pgfpathlineto{\pgfqpoint{4.629580in}{3.529806in}}%
\pgfpathlineto{\pgfqpoint{4.629580in}{0.533400in}}%
\pgfpathlineto{\pgfqpoint{4.629580in}{0.521603in}}%
\pgfpathclose%
\pgfusepath{stroke,fill}%
\end{pgfscope}%
\begin{pgfscope}%
\pgfsys@transformshift{4.480000in}{0.526603in}%
\pgftext[left,bottom]{\pgfimage[interpolate=true,width=0.150000in,height=3.020000in]{series_l_ds-img0.png}}%
\end{pgfscope}%
\begin{pgfscope}%
\pgfsetbuttcap%
\pgfsetroundjoin%
\definecolor{currentfill}{rgb}{0.000000,0.000000,0.000000}%
\pgfsetfillcolor{currentfill}%
\pgfsetlinewidth{0.803000pt}%
\definecolor{currentstroke}{rgb}{0.000000,0.000000,0.000000}%
\pgfsetstrokecolor{currentstroke}%
\pgfsetdash{}{0pt}%
\pgfsys@defobject{currentmarker}{\pgfqpoint{0.000000in}{0.000000in}}{\pgfqpoint{0.048611in}{0.000000in}}{%
\pgfpathmoveto{\pgfqpoint{0.000000in}{0.000000in}}%
\pgfpathlineto{\pgfqpoint{0.048611in}{0.000000in}}%
\pgfusepath{stroke,fill}%
}%
\begin{pgfscope}%
\pgfsys@transformshift{4.629580in}{1.060049in}%
\pgfsys@useobject{currentmarker}{}%
\end{pgfscope}%
\end{pgfscope}%
\begin{pgfscope}%
\pgftext[x=4.726802in,y=1.007287in,left,base]{\rmfamily\fontsize{10.000000}{12.000000}\selectfont \(\displaystyle 2\times10^{-1}\)}%
\end{pgfscope}%
\begin{pgfscope}%
\pgfsetbuttcap%
\pgfsetroundjoin%
\definecolor{currentfill}{rgb}{0.000000,0.000000,0.000000}%
\pgfsetfillcolor{currentfill}%
\pgfsetlinewidth{0.803000pt}%
\definecolor{currentstroke}{rgb}{0.000000,0.000000,0.000000}%
\pgfsetstrokecolor{currentstroke}%
\pgfsetdash{}{0pt}%
\pgfsys@defobject{currentmarker}{\pgfqpoint{0.000000in}{0.000000in}}{\pgfqpoint{0.048611in}{0.000000in}}{%
\pgfpathmoveto{\pgfqpoint{0.000000in}{0.000000in}}%
\pgfpathlineto{\pgfqpoint{0.048611in}{0.000000in}}%
\pgfusepath{stroke,fill}%
}%
\begin{pgfscope}%
\pgfsys@transformshift{4.629580in}{1.889642in}%
\pgfsys@useobject{currentmarker}{}%
\end{pgfscope}%
\end{pgfscope}%
\begin{pgfscope}%
\pgftext[x=4.726802in,y=1.836881in,left,base]{\rmfamily\fontsize{10.000000}{12.000000}\selectfont \(\displaystyle 3\times10^{-1}\)}%
\end{pgfscope}%
\begin{pgfscope}%
\pgfsetbuttcap%
\pgfsetroundjoin%
\definecolor{currentfill}{rgb}{0.000000,0.000000,0.000000}%
\pgfsetfillcolor{currentfill}%
\pgfsetlinewidth{0.803000pt}%
\definecolor{currentstroke}{rgb}{0.000000,0.000000,0.000000}%
\pgfsetstrokecolor{currentstroke}%
\pgfsetdash{}{0pt}%
\pgfsys@defobject{currentmarker}{\pgfqpoint{0.000000in}{0.000000in}}{\pgfqpoint{0.048611in}{0.000000in}}{%
\pgfpathmoveto{\pgfqpoint{0.000000in}{0.000000in}}%
\pgfpathlineto{\pgfqpoint{0.048611in}{0.000000in}}%
\pgfusepath{stroke,fill}%
}%
\begin{pgfscope}%
\pgfsys@transformshift{4.629580in}{2.478248in}%
\pgfsys@useobject{currentmarker}{}%
\end{pgfscope}%
\end{pgfscope}%
\begin{pgfscope}%
\pgftext[x=4.726802in,y=2.425487in,left,base]{\rmfamily\fontsize{10.000000}{12.000000}\selectfont \(\displaystyle 4\times10^{-1}\)}%
\end{pgfscope}%
\begin{pgfscope}%
\pgfsetbuttcap%
\pgfsetroundjoin%
\definecolor{currentfill}{rgb}{0.000000,0.000000,0.000000}%
\pgfsetfillcolor{currentfill}%
\pgfsetlinewidth{0.803000pt}%
\definecolor{currentstroke}{rgb}{0.000000,0.000000,0.000000}%
\pgfsetstrokecolor{currentstroke}%
\pgfsetdash{}{0pt}%
\pgfsys@defobject{currentmarker}{\pgfqpoint{0.000000in}{0.000000in}}{\pgfqpoint{0.048611in}{0.000000in}}{%
\pgfpathmoveto{\pgfqpoint{0.000000in}{0.000000in}}%
\pgfpathlineto{\pgfqpoint{0.048611in}{0.000000in}}%
\pgfusepath{stroke,fill}%
}%
\begin{pgfscope}%
\pgfsys@transformshift{4.629580in}{2.934806in}%
\pgfsys@useobject{currentmarker}{}%
\end{pgfscope}%
\end{pgfscope}%
\begin{pgfscope}%
\pgfsetbuttcap%
\pgfsetroundjoin%
\definecolor{currentfill}{rgb}{0.000000,0.000000,0.000000}%
\pgfsetfillcolor{currentfill}%
\pgfsetlinewidth{0.803000pt}%
\definecolor{currentstroke}{rgb}{0.000000,0.000000,0.000000}%
\pgfsetstrokecolor{currentstroke}%
\pgfsetdash{}{0pt}%
\pgfsys@defobject{currentmarker}{\pgfqpoint{0.000000in}{0.000000in}}{\pgfqpoint{0.048611in}{0.000000in}}{%
\pgfpathmoveto{\pgfqpoint{0.000000in}{0.000000in}}%
\pgfpathlineto{\pgfqpoint{0.048611in}{0.000000in}}%
\pgfusepath{stroke,fill}%
}%
\begin{pgfscope}%
\pgfsys@transformshift{4.629580in}{3.307841in}%
\pgfsys@useobject{currentmarker}{}%
\end{pgfscope}%
\end{pgfscope}%
\begin{pgfscope}%
\pgftext[x=4.726802in,y=3.255080in,left,base]{\rmfamily\fontsize{10.000000}{12.000000}\selectfont \(\displaystyle 6\times10^{-1}\)}%
\end{pgfscope}%
\begin{pgfscope}%
\pgftext[x=5.448446in,y=2.031603in,,top]{\rmfamily\fontsize{12.000000}{14.400000}\selectfont \(\displaystyle {\mathbf{E} \mbox{u}}\)}%
\end{pgfscope}%
\begin{pgfscope}%
\pgfsetbuttcap%
\pgfsetmiterjoin%
\pgfsetlinewidth{0.803000pt}%
\definecolor{currentstroke}{rgb}{0.000000,0.000000,0.000000}%
\pgfsetstrokecolor{currentstroke}%
\pgfsetdash{}{0pt}%
\pgfpathmoveto{\pgfqpoint{4.478580in}{0.521603in}}%
\pgfpathlineto{\pgfqpoint{4.478580in}{0.533400in}}%
\pgfpathlineto{\pgfqpoint{4.478580in}{3.529806in}}%
\pgfpathlineto{\pgfqpoint{4.478580in}{3.541603in}}%
\pgfpathlineto{\pgfqpoint{4.629580in}{3.541603in}}%
\pgfpathlineto{\pgfqpoint{4.629580in}{3.529806in}}%
\pgfpathlineto{\pgfqpoint{4.629580in}{0.533400in}}%
\pgfpathlineto{\pgfqpoint{4.629580in}{0.521603in}}%
\pgfpathclose%
\pgfusepath{stroke}%
\end{pgfscope}%
\end{pgfpicture}%
\makeatother%
\endgroup%

    \caption{Non-dimensional trajectories with the long-time scaling as a function of ${\mathbb{E}\mbox{u}}_+$.\label{fig:series_l_ds}}
\end{figure}

\begin{figure}[htb]
    \centering
    %% Creator: Matplotlib, PGF backend
%%
%% To include the figure in your LaTeX document, write
%%   \input{<filename>.pgf}
%%
%% Make sure the required packages are loaded in your preamble
%%   \usepackage{pgf}
%%
%% Figures using additional raster images can only be included by \input if
%% they are in the same directory as the main LaTeX file. For loading figures
%% from other directories you can use the `import` package
%%   \usepackage{import}
%% and then include the figures with
%%   \import{<path to file>}{<filename>.pgf}
%%
%% Matplotlib used the following preamble
%%   \usepackage{fontspec}
%%   \setmainfont{DejaVu Serif}
%%   \setsansfont{DejaVu Sans}
%%   \setmonofont{DejaVu Sans Mono}
%%
\begingroup%
\makeatletter%
\begin{pgfpicture}%
\pgfpathrectangle{\pgfpointorigin}{\pgfqpoint{5.534462in}{3.694365in}}%
\pgfusepath{use as bounding box, clip}%
\begin{pgfscope}%
\pgfsetbuttcap%
\pgfsetmiterjoin%
\definecolor{currentfill}{rgb}{1.000000,1.000000,1.000000}%
\pgfsetfillcolor{currentfill}%
\pgfsetlinewidth{0.000000pt}%
\definecolor{currentstroke}{rgb}{1.000000,1.000000,1.000000}%
\pgfsetstrokecolor{currentstroke}%
\pgfsetdash{}{0pt}%
\pgfpathmoveto{\pgfqpoint{0.000000in}{0.000000in}}%
\pgfpathlineto{\pgfqpoint{5.534462in}{0.000000in}}%
\pgfpathlineto{\pgfqpoint{5.534462in}{3.694365in}}%
\pgfpathlineto{\pgfqpoint{0.000000in}{3.694365in}}%
\pgfpathclose%
\pgfusepath{fill}%
\end{pgfscope}%
\begin{pgfscope}%
\pgfsetbuttcap%
\pgfsetmiterjoin%
\definecolor{currentfill}{rgb}{1.000000,1.000000,1.000000}%
\pgfsetfillcolor{currentfill}%
\pgfsetlinewidth{0.000000pt}%
\definecolor{currentstroke}{rgb}{0.000000,0.000000,0.000000}%
\pgfsetstrokecolor{currentstroke}%
\pgfsetstrokeopacity{0.000000}%
\pgfsetdash{}{0pt}%
\pgfpathmoveto{\pgfqpoint{0.634105in}{0.521603in}}%
\pgfpathlineto{\pgfqpoint{4.354105in}{0.521603in}}%
\pgfpathlineto{\pgfqpoint{4.354105in}{3.541603in}}%
\pgfpathlineto{\pgfqpoint{0.634105in}{3.541603in}}%
\pgfpathclose%
\pgfusepath{fill}%
\end{pgfscope}%
\begin{pgfscope}%
\pgfsetbuttcap%
\pgfsetroundjoin%
\definecolor{currentfill}{rgb}{0.000000,0.000000,0.000000}%
\pgfsetfillcolor{currentfill}%
\pgfsetlinewidth{0.803000pt}%
\definecolor{currentstroke}{rgb}{0.000000,0.000000,0.000000}%
\pgfsetstrokecolor{currentstroke}%
\pgfsetdash{}{0pt}%
\pgfsys@defobject{currentmarker}{\pgfqpoint{0.000000in}{-0.048611in}}{\pgfqpoint{0.000000in}{0.000000in}}{%
\pgfpathmoveto{\pgfqpoint{0.000000in}{0.000000in}}%
\pgfpathlineto{\pgfqpoint{0.000000in}{-0.048611in}}%
\pgfusepath{stroke,fill}%
}%
\begin{pgfscope}%
\pgfsys@transformshift{0.634105in}{0.521603in}%
\pgfsys@useobject{currentmarker}{}%
\end{pgfscope}%
\end{pgfscope}%
\begin{pgfscope}%
\pgftext[x=0.634105in,y=0.424381in,,top]{\rmfamily\fontsize{10.000000}{12.000000}\selectfont \(\displaystyle 0.0\)}%
\end{pgfscope}%
\begin{pgfscope}%
\pgfsetbuttcap%
\pgfsetroundjoin%
\definecolor{currentfill}{rgb}{0.000000,0.000000,0.000000}%
\pgfsetfillcolor{currentfill}%
\pgfsetlinewidth{0.803000pt}%
\definecolor{currentstroke}{rgb}{0.000000,0.000000,0.000000}%
\pgfsetstrokecolor{currentstroke}%
\pgfsetdash{}{0pt}%
\pgfsys@defobject{currentmarker}{\pgfqpoint{0.000000in}{-0.048611in}}{\pgfqpoint{0.000000in}{0.000000in}}{%
\pgfpathmoveto{\pgfqpoint{0.000000in}{0.000000in}}%
\pgfpathlineto{\pgfqpoint{0.000000in}{-0.048611in}}%
\pgfusepath{stroke,fill}%
}%
\begin{pgfscope}%
\pgfsys@transformshift{1.254105in}{0.521603in}%
\pgfsys@useobject{currentmarker}{}%
\end{pgfscope}%
\end{pgfscope}%
\begin{pgfscope}%
\pgftext[x=1.254105in,y=0.424381in,,top]{\rmfamily\fontsize{10.000000}{12.000000}\selectfont \(\displaystyle 0.5\)}%
\end{pgfscope}%
\begin{pgfscope}%
\pgfsetbuttcap%
\pgfsetroundjoin%
\definecolor{currentfill}{rgb}{0.000000,0.000000,0.000000}%
\pgfsetfillcolor{currentfill}%
\pgfsetlinewidth{0.803000pt}%
\definecolor{currentstroke}{rgb}{0.000000,0.000000,0.000000}%
\pgfsetstrokecolor{currentstroke}%
\pgfsetdash{}{0pt}%
\pgfsys@defobject{currentmarker}{\pgfqpoint{0.000000in}{-0.048611in}}{\pgfqpoint{0.000000in}{0.000000in}}{%
\pgfpathmoveto{\pgfqpoint{0.000000in}{0.000000in}}%
\pgfpathlineto{\pgfqpoint{0.000000in}{-0.048611in}}%
\pgfusepath{stroke,fill}%
}%
\begin{pgfscope}%
\pgfsys@transformshift{1.874105in}{0.521603in}%
\pgfsys@useobject{currentmarker}{}%
\end{pgfscope}%
\end{pgfscope}%
\begin{pgfscope}%
\pgftext[x=1.874105in,y=0.424381in,,top]{\rmfamily\fontsize{10.000000}{12.000000}\selectfont \(\displaystyle 1.0\)}%
\end{pgfscope}%
\begin{pgfscope}%
\pgfsetbuttcap%
\pgfsetroundjoin%
\definecolor{currentfill}{rgb}{0.000000,0.000000,0.000000}%
\pgfsetfillcolor{currentfill}%
\pgfsetlinewidth{0.803000pt}%
\definecolor{currentstroke}{rgb}{0.000000,0.000000,0.000000}%
\pgfsetstrokecolor{currentstroke}%
\pgfsetdash{}{0pt}%
\pgfsys@defobject{currentmarker}{\pgfqpoint{0.000000in}{-0.048611in}}{\pgfqpoint{0.000000in}{0.000000in}}{%
\pgfpathmoveto{\pgfqpoint{0.000000in}{0.000000in}}%
\pgfpathlineto{\pgfqpoint{0.000000in}{-0.048611in}}%
\pgfusepath{stroke,fill}%
}%
\begin{pgfscope}%
\pgfsys@transformshift{2.494105in}{0.521603in}%
\pgfsys@useobject{currentmarker}{}%
\end{pgfscope}%
\end{pgfscope}%
\begin{pgfscope}%
\pgftext[x=2.494105in,y=0.424381in,,top]{\rmfamily\fontsize{10.000000}{12.000000}\selectfont \(\displaystyle 1.5\)}%
\end{pgfscope}%
\begin{pgfscope}%
\pgfsetbuttcap%
\pgfsetroundjoin%
\definecolor{currentfill}{rgb}{0.000000,0.000000,0.000000}%
\pgfsetfillcolor{currentfill}%
\pgfsetlinewidth{0.803000pt}%
\definecolor{currentstroke}{rgb}{0.000000,0.000000,0.000000}%
\pgfsetstrokecolor{currentstroke}%
\pgfsetdash{}{0pt}%
\pgfsys@defobject{currentmarker}{\pgfqpoint{0.000000in}{-0.048611in}}{\pgfqpoint{0.000000in}{0.000000in}}{%
\pgfpathmoveto{\pgfqpoint{0.000000in}{0.000000in}}%
\pgfpathlineto{\pgfqpoint{0.000000in}{-0.048611in}}%
\pgfusepath{stroke,fill}%
}%
\begin{pgfscope}%
\pgfsys@transformshift{3.114105in}{0.521603in}%
\pgfsys@useobject{currentmarker}{}%
\end{pgfscope}%
\end{pgfscope}%
\begin{pgfscope}%
\pgftext[x=3.114105in,y=0.424381in,,top]{\rmfamily\fontsize{10.000000}{12.000000}\selectfont \(\displaystyle 2.0\)}%
\end{pgfscope}%
\begin{pgfscope}%
\pgfsetbuttcap%
\pgfsetroundjoin%
\definecolor{currentfill}{rgb}{0.000000,0.000000,0.000000}%
\pgfsetfillcolor{currentfill}%
\pgfsetlinewidth{0.803000pt}%
\definecolor{currentstroke}{rgb}{0.000000,0.000000,0.000000}%
\pgfsetstrokecolor{currentstroke}%
\pgfsetdash{}{0pt}%
\pgfsys@defobject{currentmarker}{\pgfqpoint{0.000000in}{-0.048611in}}{\pgfqpoint{0.000000in}{0.000000in}}{%
\pgfpathmoveto{\pgfqpoint{0.000000in}{0.000000in}}%
\pgfpathlineto{\pgfqpoint{0.000000in}{-0.048611in}}%
\pgfusepath{stroke,fill}%
}%
\begin{pgfscope}%
\pgfsys@transformshift{3.734105in}{0.521603in}%
\pgfsys@useobject{currentmarker}{}%
\end{pgfscope}%
\end{pgfscope}%
\begin{pgfscope}%
\pgftext[x=3.734105in,y=0.424381in,,top]{\rmfamily\fontsize{10.000000}{12.000000}\selectfont \(\displaystyle 2.5\)}%
\end{pgfscope}%
\begin{pgfscope}%
\pgfsetbuttcap%
\pgfsetroundjoin%
\definecolor{currentfill}{rgb}{0.000000,0.000000,0.000000}%
\pgfsetfillcolor{currentfill}%
\pgfsetlinewidth{0.803000pt}%
\definecolor{currentstroke}{rgb}{0.000000,0.000000,0.000000}%
\pgfsetstrokecolor{currentstroke}%
\pgfsetdash{}{0pt}%
\pgfsys@defobject{currentmarker}{\pgfqpoint{0.000000in}{-0.048611in}}{\pgfqpoint{0.000000in}{0.000000in}}{%
\pgfpathmoveto{\pgfqpoint{0.000000in}{0.000000in}}%
\pgfpathlineto{\pgfqpoint{0.000000in}{-0.048611in}}%
\pgfusepath{stroke,fill}%
}%
\begin{pgfscope}%
\pgfsys@transformshift{4.354105in}{0.521603in}%
\pgfsys@useobject{currentmarker}{}%
\end{pgfscope}%
\end{pgfscope}%
\begin{pgfscope}%
\pgftext[x=4.354105in,y=0.424381in,,top]{\rmfamily\fontsize{10.000000}{12.000000}\selectfont \(\displaystyle 3.0\)}%
\end{pgfscope}%
\begin{pgfscope}%
\pgftext[x=2.494105in,y=0.234413in,,top]{\rmfamily\fontsize{10.000000}{12.000000}\selectfont \(\displaystyle \bar{t}\)}%
\end{pgfscope}%
\begin{pgfscope}%
\pgfsetbuttcap%
\pgfsetroundjoin%
\definecolor{currentfill}{rgb}{0.000000,0.000000,0.000000}%
\pgfsetfillcolor{currentfill}%
\pgfsetlinewidth{0.803000pt}%
\definecolor{currentstroke}{rgb}{0.000000,0.000000,0.000000}%
\pgfsetstrokecolor{currentstroke}%
\pgfsetdash{}{0pt}%
\pgfsys@defobject{currentmarker}{\pgfqpoint{-0.048611in}{0.000000in}}{\pgfqpoint{0.000000in}{0.000000in}}{%
\pgfpathmoveto{\pgfqpoint{0.000000in}{0.000000in}}%
\pgfpathlineto{\pgfqpoint{-0.048611in}{0.000000in}}%
\pgfusepath{stroke,fill}%
}%
\begin{pgfscope}%
\pgfsys@transformshift{0.634105in}{0.521603in}%
\pgfsys@useobject{currentmarker}{}%
\end{pgfscope}%
\end{pgfscope}%
\begin{pgfscope}%
\pgftext[x=0.289968in,y=0.468842in,left,base]{\rmfamily\fontsize{10.000000}{12.000000}\selectfont \(\displaystyle 0.00\)}%
\end{pgfscope}%
\begin{pgfscope}%
\pgfsetbuttcap%
\pgfsetroundjoin%
\definecolor{currentfill}{rgb}{0.000000,0.000000,0.000000}%
\pgfsetfillcolor{currentfill}%
\pgfsetlinewidth{0.803000pt}%
\definecolor{currentstroke}{rgb}{0.000000,0.000000,0.000000}%
\pgfsetstrokecolor{currentstroke}%
\pgfsetdash{}{0pt}%
\pgfsys@defobject{currentmarker}{\pgfqpoint{-0.048611in}{0.000000in}}{\pgfqpoint{0.000000in}{0.000000in}}{%
\pgfpathmoveto{\pgfqpoint{0.000000in}{0.000000in}}%
\pgfpathlineto{\pgfqpoint{-0.048611in}{0.000000in}}%
\pgfusepath{stroke,fill}%
}%
\begin{pgfscope}%
\pgfsys@transformshift{0.634105in}{0.899103in}%
\pgfsys@useobject{currentmarker}{}%
\end{pgfscope}%
\end{pgfscope}%
\begin{pgfscope}%
\pgftext[x=0.289968in,y=0.846342in,left,base]{\rmfamily\fontsize{10.000000}{12.000000}\selectfont \(\displaystyle 0.25\)}%
\end{pgfscope}%
\begin{pgfscope}%
\pgfsetbuttcap%
\pgfsetroundjoin%
\definecolor{currentfill}{rgb}{0.000000,0.000000,0.000000}%
\pgfsetfillcolor{currentfill}%
\pgfsetlinewidth{0.803000pt}%
\definecolor{currentstroke}{rgb}{0.000000,0.000000,0.000000}%
\pgfsetstrokecolor{currentstroke}%
\pgfsetdash{}{0pt}%
\pgfsys@defobject{currentmarker}{\pgfqpoint{-0.048611in}{0.000000in}}{\pgfqpoint{0.000000in}{0.000000in}}{%
\pgfpathmoveto{\pgfqpoint{0.000000in}{0.000000in}}%
\pgfpathlineto{\pgfqpoint{-0.048611in}{0.000000in}}%
\pgfusepath{stroke,fill}%
}%
\begin{pgfscope}%
\pgfsys@transformshift{0.634105in}{1.276603in}%
\pgfsys@useobject{currentmarker}{}%
\end{pgfscope}%
\end{pgfscope}%
\begin{pgfscope}%
\pgftext[x=0.289968in,y=1.223842in,left,base]{\rmfamily\fontsize{10.000000}{12.000000}\selectfont \(\displaystyle 0.50\)}%
\end{pgfscope}%
\begin{pgfscope}%
\pgfsetbuttcap%
\pgfsetroundjoin%
\definecolor{currentfill}{rgb}{0.000000,0.000000,0.000000}%
\pgfsetfillcolor{currentfill}%
\pgfsetlinewidth{0.803000pt}%
\definecolor{currentstroke}{rgb}{0.000000,0.000000,0.000000}%
\pgfsetstrokecolor{currentstroke}%
\pgfsetdash{}{0pt}%
\pgfsys@defobject{currentmarker}{\pgfqpoint{-0.048611in}{0.000000in}}{\pgfqpoint{0.000000in}{0.000000in}}{%
\pgfpathmoveto{\pgfqpoint{0.000000in}{0.000000in}}%
\pgfpathlineto{\pgfqpoint{-0.048611in}{0.000000in}}%
\pgfusepath{stroke,fill}%
}%
\begin{pgfscope}%
\pgfsys@transformshift{0.634105in}{1.654103in}%
\pgfsys@useobject{currentmarker}{}%
\end{pgfscope}%
\end{pgfscope}%
\begin{pgfscope}%
\pgftext[x=0.289968in,y=1.601342in,left,base]{\rmfamily\fontsize{10.000000}{12.000000}\selectfont \(\displaystyle 0.75\)}%
\end{pgfscope}%
\begin{pgfscope}%
\pgfsetbuttcap%
\pgfsetroundjoin%
\definecolor{currentfill}{rgb}{0.000000,0.000000,0.000000}%
\pgfsetfillcolor{currentfill}%
\pgfsetlinewidth{0.803000pt}%
\definecolor{currentstroke}{rgb}{0.000000,0.000000,0.000000}%
\pgfsetstrokecolor{currentstroke}%
\pgfsetdash{}{0pt}%
\pgfsys@defobject{currentmarker}{\pgfqpoint{-0.048611in}{0.000000in}}{\pgfqpoint{0.000000in}{0.000000in}}{%
\pgfpathmoveto{\pgfqpoint{0.000000in}{0.000000in}}%
\pgfpathlineto{\pgfqpoint{-0.048611in}{0.000000in}}%
\pgfusepath{stroke,fill}%
}%
\begin{pgfscope}%
\pgfsys@transformshift{0.634105in}{2.031603in}%
\pgfsys@useobject{currentmarker}{}%
\end{pgfscope}%
\end{pgfscope}%
\begin{pgfscope}%
\pgftext[x=0.289968in,y=1.978842in,left,base]{\rmfamily\fontsize{10.000000}{12.000000}\selectfont \(\displaystyle 1.00\)}%
\end{pgfscope}%
\begin{pgfscope}%
\pgfsetbuttcap%
\pgfsetroundjoin%
\definecolor{currentfill}{rgb}{0.000000,0.000000,0.000000}%
\pgfsetfillcolor{currentfill}%
\pgfsetlinewidth{0.803000pt}%
\definecolor{currentstroke}{rgb}{0.000000,0.000000,0.000000}%
\pgfsetstrokecolor{currentstroke}%
\pgfsetdash{}{0pt}%
\pgfsys@defobject{currentmarker}{\pgfqpoint{-0.048611in}{0.000000in}}{\pgfqpoint{0.000000in}{0.000000in}}{%
\pgfpathmoveto{\pgfqpoint{0.000000in}{0.000000in}}%
\pgfpathlineto{\pgfqpoint{-0.048611in}{0.000000in}}%
\pgfusepath{stroke,fill}%
}%
\begin{pgfscope}%
\pgfsys@transformshift{0.634105in}{2.409103in}%
\pgfsys@useobject{currentmarker}{}%
\end{pgfscope}%
\end{pgfscope}%
\begin{pgfscope}%
\pgftext[x=0.289968in,y=2.356342in,left,base]{\rmfamily\fontsize{10.000000}{12.000000}\selectfont \(\displaystyle 1.25\)}%
\end{pgfscope}%
\begin{pgfscope}%
\pgfsetbuttcap%
\pgfsetroundjoin%
\definecolor{currentfill}{rgb}{0.000000,0.000000,0.000000}%
\pgfsetfillcolor{currentfill}%
\pgfsetlinewidth{0.803000pt}%
\definecolor{currentstroke}{rgb}{0.000000,0.000000,0.000000}%
\pgfsetstrokecolor{currentstroke}%
\pgfsetdash{}{0pt}%
\pgfsys@defobject{currentmarker}{\pgfqpoint{-0.048611in}{0.000000in}}{\pgfqpoint{0.000000in}{0.000000in}}{%
\pgfpathmoveto{\pgfqpoint{0.000000in}{0.000000in}}%
\pgfpathlineto{\pgfqpoint{-0.048611in}{0.000000in}}%
\pgfusepath{stroke,fill}%
}%
\begin{pgfscope}%
\pgfsys@transformshift{0.634105in}{2.786603in}%
\pgfsys@useobject{currentmarker}{}%
\end{pgfscope}%
\end{pgfscope}%
\begin{pgfscope}%
\pgftext[x=0.289968in,y=2.733842in,left,base]{\rmfamily\fontsize{10.000000}{12.000000}\selectfont \(\displaystyle 1.50\)}%
\end{pgfscope}%
\begin{pgfscope}%
\pgfsetbuttcap%
\pgfsetroundjoin%
\definecolor{currentfill}{rgb}{0.000000,0.000000,0.000000}%
\pgfsetfillcolor{currentfill}%
\pgfsetlinewidth{0.803000pt}%
\definecolor{currentstroke}{rgb}{0.000000,0.000000,0.000000}%
\pgfsetstrokecolor{currentstroke}%
\pgfsetdash{}{0pt}%
\pgfsys@defobject{currentmarker}{\pgfqpoint{-0.048611in}{0.000000in}}{\pgfqpoint{0.000000in}{0.000000in}}{%
\pgfpathmoveto{\pgfqpoint{0.000000in}{0.000000in}}%
\pgfpathlineto{\pgfqpoint{-0.048611in}{0.000000in}}%
\pgfusepath{stroke,fill}%
}%
\begin{pgfscope}%
\pgfsys@transformshift{0.634105in}{3.164103in}%
\pgfsys@useobject{currentmarker}{}%
\end{pgfscope}%
\end{pgfscope}%
\begin{pgfscope}%
\pgftext[x=0.289968in,y=3.111342in,left,base]{\rmfamily\fontsize{10.000000}{12.000000}\selectfont \(\displaystyle 1.75\)}%
\end{pgfscope}%
\begin{pgfscope}%
\pgfsetbuttcap%
\pgfsetroundjoin%
\definecolor{currentfill}{rgb}{0.000000,0.000000,0.000000}%
\pgfsetfillcolor{currentfill}%
\pgfsetlinewidth{0.803000pt}%
\definecolor{currentstroke}{rgb}{0.000000,0.000000,0.000000}%
\pgfsetstrokecolor{currentstroke}%
\pgfsetdash{}{0pt}%
\pgfsys@defobject{currentmarker}{\pgfqpoint{-0.048611in}{0.000000in}}{\pgfqpoint{0.000000in}{0.000000in}}{%
\pgfpathmoveto{\pgfqpoint{0.000000in}{0.000000in}}%
\pgfpathlineto{\pgfqpoint{-0.048611in}{0.000000in}}%
\pgfusepath{stroke,fill}%
}%
\begin{pgfscope}%
\pgfsys@transformshift{0.634105in}{3.541603in}%
\pgfsys@useobject{currentmarker}{}%
\end{pgfscope}%
\end{pgfscope}%
\begin{pgfscope}%
\pgftext[x=0.289968in,y=3.488842in,left,base]{\rmfamily\fontsize{10.000000}{12.000000}\selectfont \(\displaystyle 2.00\)}%
\end{pgfscope}%
\begin{pgfscope}%
\pgftext[x=0.234413in,y=2.031603in,,bottom,rotate=90.000000]{\rmfamily\fontsize{10.000000}{12.000000}\selectfont \(\displaystyle \bar{y}\)}%
\end{pgfscope}%
\begin{pgfscope}%
\pgfpathrectangle{\pgfqpoint{0.634105in}{0.521603in}}{\pgfqpoint{3.720000in}{3.020000in}} %
\pgfusepath{clip}%
\pgfsetrectcap%
\pgfsetroundjoin%
\pgfsetlinewidth{1.505625pt}%
\definecolor{currentstroke}{rgb}{1.000000,0.231948,0.116773}%
\pgfsetstrokecolor{currentstroke}%
\pgfsetdash{}{0pt}%
\pgfpathmoveto{\pgfqpoint{0.679135in}{0.567197in}}%
\pgfpathlineto{\pgfqpoint{0.683228in}{0.572448in}}%
\pgfpathlineto{\pgfqpoint{0.687322in}{0.578864in}}%
\pgfpathlineto{\pgfqpoint{0.691416in}{0.579875in}}%
\pgfpathlineto{\pgfqpoint{0.695509in}{0.582942in}}%
\pgfpathlineto{\pgfqpoint{0.699603in}{0.588149in}}%
\pgfpathlineto{\pgfqpoint{0.703697in}{0.589974in}}%
\pgfpathlineto{\pgfqpoint{0.707790in}{0.592367in}}%
\pgfpathlineto{\pgfqpoint{0.711884in}{0.594347in}}%
\pgfpathlineto{\pgfqpoint{0.715978in}{0.595994in}}%
\pgfpathlineto{\pgfqpoint{0.720071in}{0.596556in}}%
\pgfpathlineto{\pgfqpoint{0.724165in}{0.596814in}}%
\pgfpathlineto{\pgfqpoint{0.728258in}{0.597092in}}%
\pgfpathlineto{\pgfqpoint{0.732352in}{0.596210in}}%
\pgfpathlineto{\pgfqpoint{0.736446in}{0.595786in}}%
\pgfpathlineto{\pgfqpoint{0.740539in}{0.594200in}}%
\pgfpathlineto{\pgfqpoint{0.744633in}{0.592635in}}%
\pgfpathlineto{\pgfqpoint{0.748727in}{0.590182in}}%
\pgfpathlineto{\pgfqpoint{0.752820in}{0.587762in}}%
\pgfpathlineto{\pgfqpoint{0.756914in}{0.584396in}}%
\pgfpathlineto{\pgfqpoint{0.761007in}{0.580867in}}%
\pgfpathlineto{\pgfqpoint{0.765101in}{0.576889in}}%
\pgfpathlineto{\pgfqpoint{0.769195in}{0.572173in}}%
\pgfpathlineto{\pgfqpoint{0.773288in}{0.567236in}}%
\pgfpathlineto{\pgfqpoint{0.777382in}{0.561366in}}%
\pgfpathlineto{\pgfqpoint{0.781476in}{0.555609in}}%
\pgfpathlineto{\pgfqpoint{0.785569in}{0.548600in}}%
\pgfpathlineto{\pgfqpoint{0.789663in}{0.541110in}}%
\pgfpathlineto{\pgfqpoint{0.793757in}{0.532166in}}%
\pgfpathlineto{\pgfqpoint{0.797850in}{0.523177in}}%
\pgfpathlineto{\pgfqpoint{0.801944in}{0.511781in}}%
\pgfpathlineto{\pgfqpoint{0.802029in}{0.511603in}}%
\pgfpathmoveto{\pgfqpoint{0.822398in}{0.511603in}}%
\pgfpathlineto{\pgfqpoint{0.822412in}{0.511633in}}%
\pgfpathlineto{\pgfqpoint{0.826506in}{0.519933in}}%
\pgfpathlineto{\pgfqpoint{0.830599in}{0.526199in}}%
\pgfpathlineto{\pgfqpoint{0.834693in}{0.531074in}}%
\pgfpathlineto{\pgfqpoint{0.838786in}{0.534930in}}%
\pgfpathlineto{\pgfqpoint{0.842880in}{0.537227in}}%
\pgfpathlineto{\pgfqpoint{0.846974in}{0.539169in}}%
\pgfpathlineto{\pgfqpoint{0.851067in}{0.539132in}}%
\pgfpathlineto{\pgfqpoint{0.855161in}{0.538285in}}%
\pgfpathlineto{\pgfqpoint{0.859255in}{0.536016in}}%
\pgfpathlineto{\pgfqpoint{0.863348in}{0.532797in}}%
\pgfpathlineto{\pgfqpoint{0.867442in}{0.528126in}}%
\pgfpathlineto{\pgfqpoint{0.871536in}{0.522435in}}%
\pgfpathlineto{\pgfqpoint{0.875629in}{0.513845in}}%
\pgfpathlineto{\pgfqpoint{0.876641in}{0.511603in}}%
\pgfpathmoveto{\pgfqpoint{0.894648in}{0.511603in}}%
\pgfpathlineto{\pgfqpoint{0.896097in}{0.514528in}}%
\pgfpathlineto{\pgfqpoint{0.900191in}{0.522408in}}%
\pgfpathlineto{\pgfqpoint{0.904285in}{0.527497in}}%
\pgfpathlineto{\pgfqpoint{0.908378in}{0.531007in}}%
\pgfpathlineto{\pgfqpoint{0.912472in}{0.533081in}}%
\pgfpathlineto{\pgfqpoint{0.916565in}{0.534014in}}%
\pgfpathlineto{\pgfqpoint{0.920659in}{0.533780in}}%
\pgfpathlineto{\pgfqpoint{0.924753in}{0.532208in}}%
\pgfpathlineto{\pgfqpoint{0.928846in}{0.529697in}}%
\pgfpathlineto{\pgfqpoint{0.932940in}{0.526041in}}%
\pgfpathlineto{\pgfqpoint{0.937034in}{0.519917in}}%
\pgfpathlineto{\pgfqpoint{0.941127in}{0.511912in}}%
\pgfpathlineto{\pgfqpoint{0.941391in}{0.511603in}}%
\pgfusepath{stroke}%
\end{pgfscope}%
\begin{pgfscope}%
\pgfpathrectangle{\pgfqpoint{0.634105in}{0.521603in}}{\pgfqpoint{3.720000in}{3.020000in}} %
\pgfusepath{clip}%
\pgfsetrectcap%
\pgfsetroundjoin%
\pgfsetlinewidth{1.505625pt}%
\definecolor{currentstroke}{rgb}{0.966667,0.743145,0.406737}%
\pgfsetstrokecolor{currentstroke}%
\pgfsetdash{}{0pt}%
\pgfpathmoveto{\pgfqpoint{0.684711in}{0.511603in}}%
\pgfpathlineto{\pgfqpoint{0.693694in}{0.533551in}}%
\pgfpathlineto{\pgfqpoint{0.703625in}{0.550991in}}%
\pgfpathlineto{\pgfqpoint{0.723488in}{0.582094in}}%
\pgfpathlineto{\pgfqpoint{0.733420in}{0.597498in}}%
\pgfpathlineto{\pgfqpoint{0.743351in}{0.619939in}}%
\pgfpathlineto{\pgfqpoint{0.753282in}{0.635169in}}%
\pgfpathlineto{\pgfqpoint{0.773145in}{0.652145in}}%
\pgfpathlineto{\pgfqpoint{0.783077in}{0.658132in}}%
\pgfpathlineto{\pgfqpoint{0.793008in}{0.670496in}}%
\pgfpathlineto{\pgfqpoint{0.812871in}{0.687318in}}%
\pgfpathlineto{\pgfqpoint{0.822803in}{0.686385in}}%
\pgfpathlineto{\pgfqpoint{0.832734in}{0.687719in}}%
\pgfpathlineto{\pgfqpoint{0.862529in}{0.706007in}}%
\pgfpathlineto{\pgfqpoint{0.872460in}{0.704489in}}%
\pgfpathlineto{\pgfqpoint{0.882391in}{0.699912in}}%
\pgfpathlineto{\pgfqpoint{0.892323in}{0.698852in}}%
\pgfpathlineto{\pgfqpoint{0.912186in}{0.703031in}}%
\pgfpathlineto{\pgfqpoint{0.941980in}{0.680835in}}%
\pgfpathlineto{\pgfqpoint{0.951912in}{0.675676in}}%
\pgfpathlineto{\pgfqpoint{0.961843in}{0.675573in}}%
\pgfpathlineto{\pgfqpoint{0.971775in}{0.667871in}}%
\pgfpathlineto{\pgfqpoint{0.981706in}{0.655834in}}%
\pgfpathlineto{\pgfqpoint{0.991638in}{0.641305in}}%
\pgfpathlineto{\pgfqpoint{1.001569in}{0.634261in}}%
\pgfpathlineto{\pgfqpoint{1.011501in}{0.625601in}}%
\pgfpathlineto{\pgfqpoint{1.021432in}{0.613190in}}%
\pgfpathlineto{\pgfqpoint{1.031363in}{0.596585in}}%
\pgfpathlineto{\pgfqpoint{1.051226in}{0.556099in}}%
\pgfpathlineto{\pgfqpoint{1.061158in}{0.539079in}}%
\pgfpathlineto{\pgfqpoint{1.073548in}{0.511603in}}%
\pgfpathmoveto{\pgfqpoint{1.162302in}{0.511603in}}%
\pgfpathlineto{\pgfqpoint{1.170404in}{0.528482in}}%
\pgfpathlineto{\pgfqpoint{1.190267in}{0.561201in}}%
\pgfpathlineto{\pgfqpoint{1.210130in}{0.594399in}}%
\pgfpathlineto{\pgfqpoint{1.220061in}{0.608903in}}%
\pgfpathlineto{\pgfqpoint{1.229993in}{0.617567in}}%
\pgfpathlineto{\pgfqpoint{1.249856in}{0.630064in}}%
\pgfpathlineto{\pgfqpoint{1.259787in}{0.640353in}}%
\pgfpathlineto{\pgfqpoint{1.269719in}{0.647353in}}%
\pgfpathlineto{\pgfqpoint{1.279650in}{0.650631in}}%
\pgfpathlineto{\pgfqpoint{1.299513in}{0.651555in}}%
\pgfpathlineto{\pgfqpoint{1.319376in}{0.657982in}}%
\pgfpathlineto{\pgfqpoint{1.329307in}{0.658316in}}%
\pgfpathlineto{\pgfqpoint{1.339239in}{0.655264in}}%
\pgfpathlineto{\pgfqpoint{1.349170in}{0.648368in}}%
\pgfpathlineto{\pgfqpoint{1.359102in}{0.646835in}}%
\pgfpathlineto{\pgfqpoint{1.369033in}{0.643831in}}%
\pgfpathlineto{\pgfqpoint{1.378965in}{0.639500in}}%
\pgfpathlineto{\pgfqpoint{1.408759in}{0.609630in}}%
\pgfpathlineto{\pgfqpoint{1.428622in}{0.592100in}}%
\pgfpathlineto{\pgfqpoint{1.438553in}{0.577611in}}%
\pgfpathlineto{\pgfqpoint{1.458416in}{0.541999in}}%
\pgfpathlineto{\pgfqpoint{1.473646in}{0.511603in}}%
\pgfpathmoveto{\pgfqpoint{1.571674in}{0.511603in}}%
\pgfpathlineto{\pgfqpoint{1.577594in}{0.519187in}}%
\pgfpathlineto{\pgfqpoint{1.597457in}{0.550353in}}%
\pgfpathlineto{\pgfqpoint{1.617320in}{0.565899in}}%
\pgfpathlineto{\pgfqpoint{1.637183in}{0.569407in}}%
\pgfpathlineto{\pgfqpoint{1.647114in}{0.568388in}}%
\pgfpathlineto{\pgfqpoint{1.666977in}{0.571352in}}%
\pgfpathlineto{\pgfqpoint{1.686840in}{0.555446in}}%
\pgfpathlineto{\pgfqpoint{1.696772in}{0.541893in}}%
\pgfpathlineto{\pgfqpoint{1.706703in}{0.532829in}}%
\pgfpathlineto{\pgfqpoint{1.725427in}{0.511603in}}%
\pgfpathmoveto{\pgfqpoint{1.831808in}{0.511603in}}%
\pgfpathlineto{\pgfqpoint{1.835812in}{0.516648in}}%
\pgfpathlineto{\pgfqpoint{1.845744in}{0.525459in}}%
\pgfpathlineto{\pgfqpoint{1.865606in}{0.538455in}}%
\pgfpathlineto{\pgfqpoint{1.885469in}{0.545000in}}%
\pgfpathlineto{\pgfqpoint{1.905332in}{0.549671in}}%
\pgfpathlineto{\pgfqpoint{1.915264in}{0.546432in}}%
\pgfpathlineto{\pgfqpoint{1.925195in}{0.541396in}}%
\pgfpathlineto{\pgfqpoint{1.935127in}{0.528701in}}%
\pgfpathlineto{\pgfqpoint{1.945058in}{0.519179in}}%
\pgfpathlineto{\pgfqpoint{1.958048in}{0.511603in}}%
\pgfpathmoveto{\pgfqpoint{2.078334in}{0.511603in}}%
\pgfpathlineto{\pgfqpoint{2.103962in}{0.532231in}}%
\pgfpathlineto{\pgfqpoint{2.113893in}{0.532527in}}%
\pgfpathlineto{\pgfqpoint{2.123825in}{0.531260in}}%
\pgfpathlineto{\pgfqpoint{2.133756in}{0.533424in}}%
\pgfpathlineto{\pgfqpoint{2.143687in}{0.528623in}}%
\pgfpathlineto{\pgfqpoint{2.153619in}{0.522181in}}%
\pgfpathlineto{\pgfqpoint{2.166155in}{0.511603in}}%
\pgfpathmoveto{\pgfqpoint{2.295413in}{0.511603in}}%
\pgfpathlineto{\pgfqpoint{2.312522in}{0.524666in}}%
\pgfpathlineto{\pgfqpoint{2.322454in}{0.528499in}}%
\pgfpathlineto{\pgfqpoint{2.342317in}{0.528867in}}%
\pgfpathlineto{\pgfqpoint{2.352248in}{0.525934in}}%
\pgfpathlineto{\pgfqpoint{2.372111in}{0.512800in}}%
\pgfpathlineto{\pgfqpoint{2.373618in}{0.511603in}}%
\pgfpathmoveto{\pgfqpoint{2.509184in}{0.511603in}}%
\pgfpathlineto{\pgfqpoint{2.511152in}{0.513326in}}%
\pgfpathlineto{\pgfqpoint{2.521083in}{0.520296in}}%
\pgfpathlineto{\pgfqpoint{2.531015in}{0.524421in}}%
\pgfpathlineto{\pgfqpoint{2.540946in}{0.526322in}}%
\pgfpathlineto{\pgfqpoint{2.550877in}{0.525648in}}%
\pgfpathlineto{\pgfqpoint{2.560809in}{0.521972in}}%
\pgfpathlineto{\pgfqpoint{2.577772in}{0.511603in}}%
\pgfpathmoveto{\pgfqpoint{2.721172in}{0.511603in}}%
\pgfpathlineto{\pgfqpoint{2.729644in}{0.517486in}}%
\pgfpathlineto{\pgfqpoint{2.739575in}{0.522447in}}%
\pgfpathlineto{\pgfqpoint{2.759438in}{0.522684in}}%
\pgfpathlineto{\pgfqpoint{2.769370in}{0.518631in}}%
\pgfpathlineto{\pgfqpoint{2.778545in}{0.511603in}}%
\pgfpathlineto{\pgfqpoint{2.778545in}{0.511603in}}%
\pgfusepath{stroke}%
\end{pgfscope}%
\begin{pgfscope}%
\pgfpathrectangle{\pgfqpoint{0.634105in}{0.521603in}}{\pgfqpoint{3.720000in}{3.020000in}} %
\pgfusepath{clip}%
\pgfsetrectcap%
\pgfsetroundjoin%
\pgfsetlinewidth{1.505625pt}%
\definecolor{currentstroke}{rgb}{1.000000,0.000000,0.000000}%
\pgfsetstrokecolor{currentstroke}%
\pgfsetdash{}{0pt}%
\pgfpathmoveto{\pgfqpoint{0.674529in}{0.606569in}}%
\pgfpathlineto{\pgfqpoint{0.683512in}{0.616333in}}%
\pgfpathlineto{\pgfqpoint{0.688004in}{0.620013in}}%
\pgfpathlineto{\pgfqpoint{0.696987in}{0.629290in}}%
\pgfpathlineto{\pgfqpoint{0.701479in}{0.631299in}}%
\pgfpathlineto{\pgfqpoint{0.710462in}{0.639076in}}%
\pgfpathlineto{\pgfqpoint{0.714953in}{0.638660in}}%
\pgfpathlineto{\pgfqpoint{0.723936in}{0.642534in}}%
\pgfpathlineto{\pgfqpoint{0.728428in}{0.642124in}}%
\pgfpathlineto{\pgfqpoint{0.737411in}{0.642975in}}%
\pgfpathlineto{\pgfqpoint{0.755377in}{0.637288in}}%
\pgfpathlineto{\pgfqpoint{0.768852in}{0.628197in}}%
\pgfpathlineto{\pgfqpoint{0.782327in}{0.614703in}}%
\pgfpathlineto{\pgfqpoint{0.795802in}{0.596620in}}%
\pgfpathlineto{\pgfqpoint{0.804785in}{0.580174in}}%
\pgfpathlineto{\pgfqpoint{0.809276in}{0.571238in}}%
\pgfpathlineto{\pgfqpoint{0.822751in}{0.534086in}}%
\pgfpathlineto{\pgfqpoint{0.827243in}{0.527421in}}%
\pgfpathlineto{\pgfqpoint{0.831734in}{0.529998in}}%
\pgfpathlineto{\pgfqpoint{0.849701in}{0.567880in}}%
\pgfpathlineto{\pgfqpoint{0.863175in}{0.581526in}}%
\pgfpathlineto{\pgfqpoint{0.872159in}{0.584644in}}%
\pgfpathlineto{\pgfqpoint{0.876650in}{0.584366in}}%
\pgfpathlineto{\pgfqpoint{0.885633in}{0.580293in}}%
\pgfpathlineto{\pgfqpoint{0.894616in}{0.572453in}}%
\pgfpathlineto{\pgfqpoint{0.903600in}{0.558238in}}%
\pgfpathlineto{\pgfqpoint{0.917074in}{0.529006in}}%
\pgfpathlineto{\pgfqpoint{0.921566in}{0.526426in}}%
\pgfpathlineto{\pgfqpoint{0.926057in}{0.530750in}}%
\pgfpathlineto{\pgfqpoint{0.939532in}{0.555518in}}%
\pgfpathlineto{\pgfqpoint{0.948515in}{0.563181in}}%
\pgfpathlineto{\pgfqpoint{0.953007in}{0.566396in}}%
\pgfpathlineto{\pgfqpoint{0.961990in}{0.565590in}}%
\pgfpathlineto{\pgfqpoint{0.966482in}{0.564378in}}%
\pgfpathlineto{\pgfqpoint{0.975465in}{0.556141in}}%
\pgfpathlineto{\pgfqpoint{0.979956in}{0.549731in}}%
\pgfpathlineto{\pgfqpoint{0.988940in}{0.533023in}}%
\pgfpathlineto{\pgfqpoint{0.993431in}{0.526776in}}%
\pgfpathlineto{\pgfqpoint{0.997923in}{0.527013in}}%
\pgfpathlineto{\pgfqpoint{1.002414in}{0.531980in}}%
\pgfpathlineto{\pgfqpoint{1.006906in}{0.540400in}}%
\pgfpathlineto{\pgfqpoint{1.015889in}{0.550956in}}%
\pgfpathlineto{\pgfqpoint{1.020381in}{0.552893in}}%
\pgfpathlineto{\pgfqpoint{1.024872in}{0.552804in}}%
\pgfpathlineto{\pgfqpoint{1.029364in}{0.551129in}}%
\pgfpathlineto{\pgfqpoint{1.033855in}{0.547597in}}%
\pgfpathlineto{\pgfqpoint{1.042838in}{0.534568in}}%
\pgfpathlineto{\pgfqpoint{1.047330in}{0.527910in}}%
\pgfpathlineto{\pgfqpoint{1.051822in}{0.525628in}}%
\pgfpathlineto{\pgfqpoint{1.056313in}{0.528987in}}%
\pgfpathlineto{\pgfqpoint{1.065296in}{0.541915in}}%
\pgfpathlineto{\pgfqpoint{1.069788in}{0.545347in}}%
\pgfpathlineto{\pgfqpoint{1.074280in}{0.545576in}}%
\pgfpathlineto{\pgfqpoint{1.078771in}{0.543098in}}%
\pgfpathlineto{\pgfqpoint{1.092246in}{0.529687in}}%
\pgfpathlineto{\pgfqpoint{1.096737in}{0.528651in}}%
\pgfpathlineto{\pgfqpoint{1.101229in}{0.530392in}}%
\pgfpathlineto{\pgfqpoint{1.110212in}{0.539194in}}%
\pgfpathlineto{\pgfqpoint{1.114704in}{0.541671in}}%
\pgfpathlineto{\pgfqpoint{1.119195in}{0.541046in}}%
\pgfpathlineto{\pgfqpoint{1.132670in}{0.528635in}}%
\pgfpathlineto{\pgfqpoint{1.137162in}{0.528416in}}%
\pgfpathlineto{\pgfqpoint{1.141653in}{0.531284in}}%
\pgfpathlineto{\pgfqpoint{1.150636in}{0.538934in}}%
\pgfpathlineto{\pgfqpoint{1.155128in}{0.540202in}}%
\pgfpathlineto{\pgfqpoint{1.159620in}{0.538609in}}%
\pgfpathlineto{\pgfqpoint{1.173094in}{0.530291in}}%
\pgfpathlineto{\pgfqpoint{1.177586in}{0.530619in}}%
\pgfpathlineto{\pgfqpoint{1.195552in}{0.538402in}}%
\pgfpathlineto{\pgfqpoint{1.200044in}{0.534851in}}%
\pgfpathlineto{\pgfqpoint{1.213518in}{0.531767in}}%
\pgfpathlineto{\pgfqpoint{1.231485in}{0.537803in}}%
\pgfpathlineto{\pgfqpoint{1.240468in}{0.535595in}}%
\pgfpathlineto{\pgfqpoint{1.249451in}{0.532448in}}%
\pgfpathlineto{\pgfqpoint{1.258434in}{0.533323in}}%
\pgfpathlineto{\pgfqpoint{1.271909in}{0.537449in}}%
\pgfpathlineto{\pgfqpoint{1.276401in}{0.537362in}}%
\pgfpathlineto{\pgfqpoint{1.294367in}{0.533152in}}%
\pgfpathlineto{\pgfqpoint{1.316825in}{0.535892in}}%
\pgfpathlineto{\pgfqpoint{1.325808in}{0.533478in}}%
\pgfpathlineto{\pgfqpoint{1.334791in}{0.533423in}}%
\pgfpathlineto{\pgfqpoint{1.348266in}{0.536236in}}%
\pgfpathlineto{\pgfqpoint{1.357249in}{0.534576in}}%
\pgfpathlineto{\pgfqpoint{1.366232in}{0.532832in}}%
\pgfpathlineto{\pgfqpoint{1.375215in}{0.534040in}}%
\pgfpathlineto{\pgfqpoint{1.388690in}{0.536062in}}%
\pgfpathlineto{\pgfqpoint{1.411148in}{0.533477in}}%
\pgfpathlineto{\pgfqpoint{1.429114in}{0.535737in}}%
\pgfpathlineto{\pgfqpoint{1.447081in}{0.533045in}}%
\pgfpathlineto{\pgfqpoint{1.469538in}{0.534705in}}%
\pgfpathlineto{\pgfqpoint{1.483013in}{0.533351in}}%
\pgfpathlineto{\pgfqpoint{1.505471in}{0.533820in}}%
\pgfpathlineto{\pgfqpoint{1.518946in}{0.532966in}}%
\pgfpathlineto{\pgfqpoint{1.541404in}{0.533713in}}%
\pgfpathlineto{\pgfqpoint{1.559370in}{0.533396in}}%
\pgfpathlineto{\pgfqpoint{1.577336in}{0.533893in}}%
\pgfpathlineto{\pgfqpoint{1.595303in}{0.533399in}}%
\pgfpathlineto{\pgfqpoint{1.608777in}{0.534036in}}%
\pgfpathlineto{\pgfqpoint{1.626744in}{0.533194in}}%
\pgfpathlineto{\pgfqpoint{1.649202in}{0.534036in}}%
\pgfpathlineto{\pgfqpoint{1.667168in}{0.533542in}}%
\pgfpathlineto{\pgfqpoint{1.676151in}{0.534240in}}%
\pgfpathlineto{\pgfqpoint{1.676151in}{0.534240in}}%
\pgfusepath{stroke}%
\end{pgfscope}%
\begin{pgfscope}%
\pgfpathrectangle{\pgfqpoint{0.634105in}{0.521603in}}{\pgfqpoint{3.720000in}{3.020000in}} %
\pgfusepath{clip}%
\pgfsetrectcap%
\pgfsetroundjoin%
\pgfsetlinewidth{1.505625pt}%
\definecolor{currentstroke}{rgb}{0.966667,0.743145,0.406737}%
\pgfsetstrokecolor{currentstroke}%
\pgfsetdash{}{0pt}%
\pgfpathmoveto{\pgfqpoint{0.757314in}{0.625750in}}%
\pgfpathlineto{\pgfqpoint{0.772715in}{0.647214in}}%
\pgfpathlineto{\pgfqpoint{0.788116in}{0.664234in}}%
\pgfpathlineto{\pgfqpoint{0.803517in}{0.675281in}}%
\pgfpathlineto{\pgfqpoint{0.834319in}{0.705904in}}%
\pgfpathlineto{\pgfqpoint{0.849720in}{0.712987in}}%
\pgfpathlineto{\pgfqpoint{0.865122in}{0.725114in}}%
\pgfpathlineto{\pgfqpoint{0.880523in}{0.733527in}}%
\pgfpathlineto{\pgfqpoint{0.895924in}{0.737410in}}%
\pgfpathlineto{\pgfqpoint{0.911325in}{0.746429in}}%
\pgfpathlineto{\pgfqpoint{0.926726in}{0.749348in}}%
\pgfpathlineto{\pgfqpoint{0.942127in}{0.748664in}}%
\pgfpathlineto{\pgfqpoint{0.957528in}{0.755146in}}%
\pgfpathlineto{\pgfqpoint{0.972929in}{0.754094in}}%
\pgfpathlineto{\pgfqpoint{0.988330in}{0.750332in}}%
\pgfpathlineto{\pgfqpoint{1.003732in}{0.752171in}}%
\pgfpathlineto{\pgfqpoint{1.034534in}{0.740414in}}%
\pgfpathlineto{\pgfqpoint{1.049935in}{0.737340in}}%
\pgfpathlineto{\pgfqpoint{1.080737in}{0.717459in}}%
\pgfpathlineto{\pgfqpoint{1.096138in}{0.710640in}}%
\pgfpathlineto{\pgfqpoint{1.111539in}{0.696048in}}%
\pgfpathlineto{\pgfqpoint{1.142342in}{0.672116in}}%
\pgfpathlineto{\pgfqpoint{1.173144in}{0.635465in}}%
\pgfpathlineto{\pgfqpoint{1.188545in}{0.619236in}}%
\pgfpathlineto{\pgfqpoint{1.234748in}{0.547016in}}%
\pgfpathlineto{\pgfqpoint{1.250149in}{0.518873in}}%
\pgfpathlineto{\pgfqpoint{1.255943in}{0.511603in}}%
\pgfpathmoveto{\pgfqpoint{1.330286in}{0.511603in}}%
\pgfpathlineto{\pgfqpoint{1.342556in}{0.531415in}}%
\pgfpathlineto{\pgfqpoint{1.357957in}{0.552943in}}%
\pgfpathlineto{\pgfqpoint{1.373358in}{0.568471in}}%
\pgfpathlineto{\pgfqpoint{1.388759in}{0.585711in}}%
\pgfpathlineto{\pgfqpoint{1.404160in}{0.598984in}}%
\pgfpathlineto{\pgfqpoint{1.419561in}{0.602060in}}%
\pgfpathlineto{\pgfqpoint{1.434963in}{0.615726in}}%
\pgfpathlineto{\pgfqpoint{1.450364in}{0.622081in}}%
\pgfpathlineto{\pgfqpoint{1.465765in}{0.619922in}}%
\pgfpathlineto{\pgfqpoint{1.481166in}{0.625901in}}%
\pgfpathlineto{\pgfqpoint{1.496567in}{0.628487in}}%
\pgfpathlineto{\pgfqpoint{1.511968in}{0.621430in}}%
\pgfpathlineto{\pgfqpoint{1.527369in}{0.623101in}}%
\pgfpathlineto{\pgfqpoint{1.542770in}{0.616497in}}%
\pgfpathlineto{\pgfqpoint{1.558171in}{0.606210in}}%
\pgfpathlineto{\pgfqpoint{1.573573in}{0.601948in}}%
\pgfpathlineto{\pgfqpoint{1.588974in}{0.589059in}}%
\pgfpathlineto{\pgfqpoint{1.604375in}{0.572453in}}%
\pgfpathlineto{\pgfqpoint{1.619776in}{0.560373in}}%
\pgfpathlineto{\pgfqpoint{1.656677in}{0.511603in}}%
\pgfpathmoveto{\pgfqpoint{1.733800in}{0.511603in}}%
\pgfpathlineto{\pgfqpoint{1.742985in}{0.521711in}}%
\pgfpathlineto{\pgfqpoint{1.758386in}{0.541868in}}%
\pgfpathlineto{\pgfqpoint{1.773787in}{0.553572in}}%
\pgfpathlineto{\pgfqpoint{1.789188in}{0.568459in}}%
\pgfpathlineto{\pgfqpoint{1.804589in}{0.573963in}}%
\pgfpathlineto{\pgfqpoint{1.819990in}{0.574364in}}%
\pgfpathlineto{\pgfqpoint{1.835391in}{0.577997in}}%
\pgfpathlineto{\pgfqpoint{1.850793in}{0.575150in}}%
\pgfpathlineto{\pgfqpoint{1.881595in}{0.557096in}}%
\pgfpathlineto{\pgfqpoint{1.896996in}{0.543161in}}%
\pgfpathlineto{\pgfqpoint{1.912397in}{0.522643in}}%
\pgfpathlineto{\pgfqpoint{1.926008in}{0.511603in}}%
\pgfpathmoveto{\pgfqpoint{2.000997in}{0.511603in}}%
\pgfpathlineto{\pgfqpoint{2.004804in}{0.514640in}}%
\pgfpathlineto{\pgfqpoint{2.020205in}{0.534842in}}%
\pgfpathlineto{\pgfqpoint{2.035606in}{0.548195in}}%
\pgfpathlineto{\pgfqpoint{2.051007in}{0.556620in}}%
\pgfpathlineto{\pgfqpoint{2.066408in}{0.563050in}}%
\pgfpathlineto{\pgfqpoint{2.081809in}{0.564707in}}%
\pgfpathlineto{\pgfqpoint{2.112611in}{0.557227in}}%
\pgfpathlineto{\pgfqpoint{2.128013in}{0.549363in}}%
\pgfpathlineto{\pgfqpoint{2.143414in}{0.535244in}}%
\pgfpathlineto{\pgfqpoint{2.158815in}{0.525198in}}%
\pgfpathlineto{\pgfqpoint{2.170680in}{0.511603in}}%
\pgfpathmoveto{\pgfqpoint{2.241182in}{0.511603in}}%
\pgfpathlineto{\pgfqpoint{2.251221in}{0.521918in}}%
\pgfpathlineto{\pgfqpoint{2.282024in}{0.544526in}}%
\pgfpathlineto{\pgfqpoint{2.297425in}{0.551882in}}%
\pgfpathlineto{\pgfqpoint{2.312826in}{0.555132in}}%
\pgfpathlineto{\pgfqpoint{2.328227in}{0.553199in}}%
\pgfpathlineto{\pgfqpoint{2.343628in}{0.545932in}}%
\pgfpathlineto{\pgfqpoint{2.374430in}{0.522114in}}%
\pgfpathlineto{\pgfqpoint{2.386821in}{0.511603in}}%
\pgfpathmoveto{\pgfqpoint{2.451362in}{0.511603in}}%
\pgfpathlineto{\pgfqpoint{2.497639in}{0.546702in}}%
\pgfpathlineto{\pgfqpoint{2.513040in}{0.552660in}}%
\pgfpathlineto{\pgfqpoint{2.528441in}{0.554987in}}%
\pgfpathlineto{\pgfqpoint{2.543843in}{0.551510in}}%
\pgfpathlineto{\pgfqpoint{2.559244in}{0.543980in}}%
\pgfpathlineto{\pgfqpoint{2.590046in}{0.520550in}}%
\pgfpathlineto{\pgfqpoint{2.602114in}{0.511603in}}%
\pgfpathmoveto{\pgfqpoint{2.655833in}{0.511603in}}%
\pgfpathlineto{\pgfqpoint{2.697854in}{0.544167in}}%
\pgfpathlineto{\pgfqpoint{2.713255in}{0.551956in}}%
\pgfpathlineto{\pgfqpoint{2.728656in}{0.555492in}}%
\pgfpathlineto{\pgfqpoint{2.744057in}{0.554124in}}%
\pgfpathlineto{\pgfqpoint{2.759458in}{0.548100in}}%
\pgfpathlineto{\pgfqpoint{2.790260in}{0.527853in}}%
\pgfpathlineto{\pgfqpoint{2.814690in}{0.511603in}}%
\pgfpathmoveto{\pgfqpoint{2.853794in}{0.511603in}}%
\pgfpathlineto{\pgfqpoint{2.867266in}{0.519769in}}%
\pgfpathlineto{\pgfqpoint{2.898068in}{0.542514in}}%
\pgfpathlineto{\pgfqpoint{2.913469in}{0.551721in}}%
\pgfpathlineto{\pgfqpoint{2.928870in}{0.556047in}}%
\pgfpathlineto{\pgfqpoint{2.944271in}{0.557474in}}%
\pgfpathlineto{\pgfqpoint{2.959673in}{0.554045in}}%
\pgfpathlineto{\pgfqpoint{2.990475in}{0.537387in}}%
\pgfpathlineto{\pgfqpoint{3.021277in}{0.517253in}}%
\pgfpathlineto{\pgfqpoint{3.036124in}{0.511603in}}%
\pgfpathmoveto{\pgfqpoint{3.053396in}{0.511603in}}%
\pgfpathlineto{\pgfqpoint{3.067480in}{0.516610in}}%
\pgfpathlineto{\pgfqpoint{3.113684in}{0.546533in}}%
\pgfpathlineto{\pgfqpoint{3.129085in}{0.554096in}}%
\pgfpathlineto{\pgfqpoint{3.144486in}{0.557355in}}%
\pgfpathlineto{\pgfqpoint{3.159887in}{0.556835in}}%
\pgfpathlineto{\pgfqpoint{3.175288in}{0.552420in}}%
\pgfpathlineto{\pgfqpoint{3.206090in}{0.534916in}}%
\pgfpathlineto{\pgfqpoint{3.221491in}{0.524501in}}%
\pgfpathlineto{\pgfqpoint{3.236893in}{0.517757in}}%
\pgfpathlineto{\pgfqpoint{3.252294in}{0.515346in}}%
\pgfpathlineto{\pgfqpoint{3.267695in}{0.516993in}}%
\pgfpathlineto{\pgfqpoint{3.283096in}{0.524640in}}%
\pgfpathlineto{\pgfqpoint{3.313898in}{0.543088in}}%
\pgfpathlineto{\pgfqpoint{3.329299in}{0.552142in}}%
\pgfpathlineto{\pgfqpoint{3.344700in}{0.558552in}}%
\pgfpathlineto{\pgfqpoint{3.360101in}{0.559246in}}%
\pgfpathlineto{\pgfqpoint{3.375503in}{0.557739in}}%
\pgfpathlineto{\pgfqpoint{3.390904in}{0.551775in}}%
\pgfpathlineto{\pgfqpoint{3.452508in}{0.519923in}}%
\pgfpathlineto{\pgfqpoint{3.467909in}{0.518426in}}%
\pgfpathlineto{\pgfqpoint{3.483310in}{0.521241in}}%
\pgfpathlineto{\pgfqpoint{3.498711in}{0.527261in}}%
\pgfpathlineto{\pgfqpoint{3.529514in}{0.543756in}}%
\pgfpathlineto{\pgfqpoint{3.544915in}{0.550012in}}%
\pgfpathlineto{\pgfqpoint{3.560316in}{0.554648in}}%
\pgfpathlineto{\pgfqpoint{3.575717in}{0.555702in}}%
\pgfpathlineto{\pgfqpoint{3.591118in}{0.552240in}}%
\pgfpathlineto{\pgfqpoint{3.606519in}{0.546987in}}%
\pgfpathlineto{\pgfqpoint{3.652723in}{0.525740in}}%
\pgfpathlineto{\pgfqpoint{3.668124in}{0.521945in}}%
\pgfpathlineto{\pgfqpoint{3.683525in}{0.521625in}}%
\pgfpathlineto{\pgfqpoint{3.698926in}{0.524999in}}%
\pgfpathlineto{\pgfqpoint{3.714327in}{0.530328in}}%
\pgfpathlineto{\pgfqpoint{3.745129in}{0.544184in}}%
\pgfpathlineto{\pgfqpoint{3.760530in}{0.548434in}}%
\pgfpathlineto{\pgfqpoint{3.775931in}{0.550501in}}%
\pgfpathlineto{\pgfqpoint{3.791333in}{0.549883in}}%
\pgfpathlineto{\pgfqpoint{3.806734in}{0.546075in}}%
\pgfpathlineto{\pgfqpoint{3.852937in}{0.527944in}}%
\pgfpathlineto{\pgfqpoint{3.868338in}{0.524324in}}%
\pgfpathlineto{\pgfqpoint{3.883739in}{0.523413in}}%
\pgfpathlineto{\pgfqpoint{3.899140in}{0.525199in}}%
\pgfpathlineto{\pgfqpoint{3.914541in}{0.529873in}}%
\pgfpathlineto{\pgfqpoint{3.929943in}{0.536038in}}%
\pgfpathlineto{\pgfqpoint{3.960745in}{0.545210in}}%
\pgfpathlineto{\pgfqpoint{3.976146in}{0.547218in}}%
\pgfpathlineto{\pgfqpoint{3.991547in}{0.546335in}}%
\pgfpathlineto{\pgfqpoint{4.006948in}{0.543521in}}%
\pgfpathlineto{\pgfqpoint{4.037750in}{0.533744in}}%
\pgfpathlineto{\pgfqpoint{4.053151in}{0.528955in}}%
\pgfpathlineto{\pgfqpoint{4.068553in}{0.525890in}}%
\pgfpathlineto{\pgfqpoint{4.083954in}{0.524932in}}%
\pgfpathlineto{\pgfqpoint{4.099355in}{0.527068in}}%
\pgfpathlineto{\pgfqpoint{4.130157in}{0.534764in}}%
\pgfpathlineto{\pgfqpoint{4.145558in}{0.539952in}}%
\pgfpathlineto{\pgfqpoint{4.160959in}{0.542876in}}%
\pgfpathlineto{\pgfqpoint{4.176360in}{0.544092in}}%
\pgfpathlineto{\pgfqpoint{4.191761in}{0.543474in}}%
\pgfpathlineto{\pgfqpoint{4.237965in}{0.532836in}}%
\pgfpathlineto{\pgfqpoint{4.237965in}{0.532836in}}%
\pgfusepath{stroke}%
\end{pgfscope}%
\begin{pgfscope}%
\pgfpathrectangle{\pgfqpoint{0.634105in}{0.521603in}}{\pgfqpoint{3.720000in}{3.020000in}} %
\pgfusepath{clip}%
\pgfsetrectcap%
\pgfsetroundjoin%
\pgfsetlinewidth{1.505625pt}%
\definecolor{currentstroke}{rgb}{0.770588,0.911023,0.542053}%
\pgfsetstrokecolor{currentstroke}%
\pgfsetdash{}{0pt}%
\pgfpathmoveto{\pgfqpoint{0.890481in}{0.760480in}}%
\pgfpathlineto{\pgfqpoint{0.916119in}{0.795136in}}%
\pgfpathlineto{\pgfqpoint{0.941757in}{0.819646in}}%
\pgfpathlineto{\pgfqpoint{0.967394in}{0.844131in}}%
\pgfpathlineto{\pgfqpoint{0.993032in}{0.876460in}}%
\pgfpathlineto{\pgfqpoint{1.018670in}{0.898035in}}%
\pgfpathlineto{\pgfqpoint{1.044307in}{0.919042in}}%
\pgfpathlineto{\pgfqpoint{1.069945in}{0.947140in}}%
\pgfpathlineto{\pgfqpoint{1.095583in}{0.964598in}}%
\pgfpathlineto{\pgfqpoint{1.121220in}{0.984462in}}%
\pgfpathlineto{\pgfqpoint{1.146858in}{1.005407in}}%
\pgfpathlineto{\pgfqpoint{1.172496in}{1.018298in}}%
\pgfpathlineto{\pgfqpoint{1.198133in}{1.038036in}}%
\pgfpathlineto{\pgfqpoint{1.223771in}{1.051288in}}%
\pgfpathlineto{\pgfqpoint{1.249409in}{1.064227in}}%
\pgfpathlineto{\pgfqpoint{1.275046in}{1.078720in}}%
\pgfpathlineto{\pgfqpoint{1.300684in}{1.086058in}}%
\pgfpathlineto{\pgfqpoint{1.326322in}{1.098056in}}%
\pgfpathlineto{\pgfqpoint{1.351959in}{1.105221in}}%
\pgfpathlineto{\pgfqpoint{1.377597in}{1.111034in}}%
\pgfpathlineto{\pgfqpoint{1.403235in}{1.120639in}}%
\pgfpathlineto{\pgfqpoint{1.428872in}{1.121871in}}%
\pgfpathlineto{\pgfqpoint{1.454510in}{1.126734in}}%
\pgfpathlineto{\pgfqpoint{1.480148in}{1.129152in}}%
\pgfpathlineto{\pgfqpoint{1.505785in}{1.127957in}}%
\pgfpathlineto{\pgfqpoint{1.531423in}{1.131342in}}%
\pgfpathlineto{\pgfqpoint{1.557060in}{1.127528in}}%
\pgfpathlineto{\pgfqpoint{1.582698in}{1.125486in}}%
\pgfpathlineto{\pgfqpoint{1.608336in}{1.122135in}}%
\pgfpathlineto{\pgfqpoint{1.633973in}{1.115048in}}%
\pgfpathlineto{\pgfqpoint{1.659611in}{1.111994in}}%
\pgfpathlineto{\pgfqpoint{1.685249in}{1.102270in}}%
\pgfpathlineto{\pgfqpoint{1.710886in}{1.093267in}}%
\pgfpathlineto{\pgfqpoint{1.736524in}{1.084803in}}%
\pgfpathlineto{\pgfqpoint{1.762162in}{1.071012in}}%
\pgfpathlineto{\pgfqpoint{1.787799in}{1.060684in}}%
\pgfpathlineto{\pgfqpoint{1.813437in}{1.045596in}}%
\pgfpathlineto{\pgfqpoint{1.839075in}{1.029245in}}%
\pgfpathlineto{\pgfqpoint{1.864712in}{1.014724in}}%
\pgfpathlineto{\pgfqpoint{1.890350in}{0.994182in}}%
\pgfpathlineto{\pgfqpoint{1.915988in}{0.977307in}}%
\pgfpathlineto{\pgfqpoint{1.941625in}{0.955521in}}%
\pgfpathlineto{\pgfqpoint{1.967263in}{0.932141in}}%
\pgfpathlineto{\pgfqpoint{1.992901in}{0.910443in}}%
\pgfpathlineto{\pgfqpoint{2.018538in}{0.883661in}}%
\pgfpathlineto{\pgfqpoint{2.044176in}{0.859306in}}%
\pgfpathlineto{\pgfqpoint{2.069814in}{0.830902in}}%
\pgfpathlineto{\pgfqpoint{2.095451in}{0.799518in}}%
\pgfpathlineto{\pgfqpoint{2.121089in}{0.770167in}}%
\pgfpathlineto{\pgfqpoint{2.146727in}{0.735003in}}%
\pgfpathlineto{\pgfqpoint{2.172364in}{0.701410in}}%
\pgfpathlineto{\pgfqpoint{2.198002in}{0.663628in}}%
\pgfpathlineto{\pgfqpoint{2.223640in}{0.619877in}}%
\pgfpathlineto{\pgfqpoint{2.249277in}{0.575938in}}%
\pgfpathlineto{\pgfqpoint{2.274915in}{0.529379in}}%
\pgfpathlineto{\pgfqpoint{2.291418in}{0.511603in}}%
\pgfpathmoveto{\pgfqpoint{2.381235in}{0.511603in}}%
\pgfpathlineto{\pgfqpoint{2.403103in}{0.539999in}}%
\pgfpathlineto{\pgfqpoint{2.428741in}{0.580083in}}%
\pgfpathlineto{\pgfqpoint{2.454378in}{0.612797in}}%
\pgfpathlineto{\pgfqpoint{2.480016in}{0.636780in}}%
\pgfpathlineto{\pgfqpoint{2.505654in}{0.659763in}}%
\pgfpathlineto{\pgfqpoint{2.531291in}{0.674071in}}%
\pgfpathlineto{\pgfqpoint{2.556929in}{0.687603in}}%
\pgfpathlineto{\pgfqpoint{2.582567in}{0.702130in}}%
\pgfpathlineto{\pgfqpoint{2.608204in}{0.708262in}}%
\pgfpathlineto{\pgfqpoint{2.633842in}{0.715206in}}%
\pgfpathlineto{\pgfqpoint{2.659480in}{0.716509in}}%
\pgfpathlineto{\pgfqpoint{2.685117in}{0.713886in}}%
\pgfpathlineto{\pgfqpoint{2.710755in}{0.712081in}}%
\pgfpathlineto{\pgfqpoint{2.736393in}{0.703539in}}%
\pgfpathlineto{\pgfqpoint{2.762030in}{0.695553in}}%
\pgfpathlineto{\pgfqpoint{2.787668in}{0.684121in}}%
\pgfpathlineto{\pgfqpoint{2.813306in}{0.665633in}}%
\pgfpathlineto{\pgfqpoint{2.838943in}{0.646212in}}%
\pgfpathlineto{\pgfqpoint{2.864581in}{0.620268in}}%
\pgfpathlineto{\pgfqpoint{2.890219in}{0.589130in}}%
\pgfpathlineto{\pgfqpoint{2.915856in}{0.554304in}}%
\pgfpathlineto{\pgfqpoint{2.941494in}{0.523101in}}%
\pgfpathlineto{\pgfqpoint{2.957360in}{0.511603in}}%
\pgfusepath{stroke}%
\end{pgfscope}%
\begin{pgfscope}%
\pgfpathrectangle{\pgfqpoint{0.634105in}{0.521603in}}{\pgfqpoint{3.720000in}{3.020000in}} %
\pgfusepath{clip}%
\pgfsetrectcap%
\pgfsetroundjoin%
\pgfsetlinewidth{1.505625pt}%
\definecolor{currentstroke}{rgb}{0.590196,0.989980,0.655284}%
\pgfsetstrokecolor{currentstroke}%
\pgfsetdash{}{0pt}%
\pgfpathmoveto{\pgfqpoint{0.975079in}{0.778661in}}%
\pgfpathlineto{\pgfqpoint{1.012966in}{0.806786in}}%
\pgfpathlineto{\pgfqpoint{1.088738in}{0.904234in}}%
\pgfpathlineto{\pgfqpoint{1.126624in}{0.941899in}}%
\pgfpathlineto{\pgfqpoint{1.164510in}{0.967791in}}%
\pgfpathlineto{\pgfqpoint{1.202396in}{1.006816in}}%
\pgfpathlineto{\pgfqpoint{1.240282in}{1.049332in}}%
\pgfpathlineto{\pgfqpoint{1.278168in}{1.074616in}}%
\pgfpathlineto{\pgfqpoint{1.316054in}{1.098069in}}%
\pgfpathlineto{\pgfqpoint{1.353940in}{1.133763in}}%
\pgfpathlineto{\pgfqpoint{1.391826in}{1.163836in}}%
\pgfpathlineto{\pgfqpoint{1.429712in}{1.179205in}}%
\pgfpathlineto{\pgfqpoint{1.467598in}{1.197659in}}%
\pgfpathlineto{\pgfqpoint{1.505484in}{1.226789in}}%
\pgfpathlineto{\pgfqpoint{1.543371in}{1.246489in}}%
\pgfpathlineto{\pgfqpoint{1.581257in}{1.255160in}}%
\pgfpathlineto{\pgfqpoint{1.619143in}{1.274145in}}%
\pgfpathlineto{\pgfqpoint{1.657029in}{1.295431in}}%
\pgfpathlineto{\pgfqpoint{1.694915in}{1.304516in}}%
\pgfpathlineto{\pgfqpoint{1.732801in}{1.307142in}}%
\pgfpathlineto{\pgfqpoint{1.808573in}{1.335319in}}%
\pgfpathlineto{\pgfqpoint{1.846459in}{1.334818in}}%
\pgfpathlineto{\pgfqpoint{1.884345in}{1.337119in}}%
\pgfpathlineto{\pgfqpoint{1.922231in}{1.347478in}}%
\pgfpathlineto{\pgfqpoint{1.960117in}{1.351577in}}%
\pgfpathlineto{\pgfqpoint{1.998003in}{1.343886in}}%
\pgfpathlineto{\pgfqpoint{2.035889in}{1.341598in}}%
\pgfpathlineto{\pgfqpoint{2.073776in}{1.345870in}}%
\pgfpathlineto{\pgfqpoint{2.111662in}{1.339442in}}%
\pgfpathlineto{\pgfqpoint{2.149548in}{1.327251in}}%
\pgfpathlineto{\pgfqpoint{2.225320in}{1.320223in}}%
\pgfpathlineto{\pgfqpoint{2.301092in}{1.285892in}}%
\pgfpathlineto{\pgfqpoint{2.338978in}{1.277020in}}%
\pgfpathlineto{\pgfqpoint{2.376864in}{1.264347in}}%
\pgfpathlineto{\pgfqpoint{2.414750in}{1.239858in}}%
\pgfpathlineto{\pgfqpoint{2.452636in}{1.220689in}}%
\pgfpathlineto{\pgfqpoint{2.490522in}{1.206676in}}%
\pgfpathlineto{\pgfqpoint{2.528408in}{1.183144in}}%
\pgfpathlineto{\pgfqpoint{2.566294in}{1.151384in}}%
\pgfpathlineto{\pgfqpoint{2.642067in}{1.104041in}}%
\pgfpathlineto{\pgfqpoint{2.717839in}{1.034127in}}%
\pgfpathlineto{\pgfqpoint{2.755725in}{1.005051in}}%
\pgfpathlineto{\pgfqpoint{2.793611in}{0.971667in}}%
\pgfpathlineto{\pgfqpoint{2.831497in}{0.926766in}}%
\pgfpathlineto{\pgfqpoint{2.869383in}{0.883923in}}%
\pgfpathlineto{\pgfqpoint{2.907269in}{0.845583in}}%
\pgfpathlineto{\pgfqpoint{2.945155in}{0.799699in}}%
\pgfpathlineto{\pgfqpoint{3.058813in}{0.638659in}}%
\pgfpathlineto{\pgfqpoint{3.096699in}{0.567561in}}%
\pgfpathlineto{\pgfqpoint{3.131686in}{0.511603in}}%
\pgfpathmoveto{\pgfqpoint{3.349318in}{0.511603in}}%
\pgfpathlineto{\pgfqpoint{3.361902in}{0.526709in}}%
\pgfpathlineto{\pgfqpoint{3.399788in}{0.577689in}}%
\pgfpathlineto{\pgfqpoint{3.437674in}{0.639365in}}%
\pgfpathlineto{\pgfqpoint{3.475560in}{0.697399in}}%
\pgfpathlineto{\pgfqpoint{3.513446in}{0.743852in}}%
\pgfpathlineto{\pgfqpoint{3.551332in}{0.781241in}}%
\pgfpathlineto{\pgfqpoint{3.627104in}{0.846867in}}%
\pgfpathlineto{\pgfqpoint{3.664990in}{0.877137in}}%
\pgfpathlineto{\pgfqpoint{3.740763in}{0.924841in}}%
\pgfpathlineto{\pgfqpoint{3.778649in}{0.944716in}}%
\pgfpathlineto{\pgfqpoint{3.816535in}{0.962654in}}%
\pgfpathlineto{\pgfqpoint{3.854421in}{0.976250in}}%
\pgfpathlineto{\pgfqpoint{3.930193in}{1.000212in}}%
\pgfpathlineto{\pgfqpoint{3.968079in}{1.006295in}}%
\pgfpathlineto{\pgfqpoint{4.005965in}{1.010996in}}%
\pgfpathlineto{\pgfqpoint{4.081737in}{1.014252in}}%
\pgfpathlineto{\pgfqpoint{4.119623in}{1.011378in}}%
\pgfpathlineto{\pgfqpoint{4.157509in}{1.004805in}}%
\pgfpathlineto{\pgfqpoint{4.195395in}{0.999771in}}%
\pgfpathlineto{\pgfqpoint{4.233282in}{0.990097in}}%
\pgfpathlineto{\pgfqpoint{4.271168in}{0.977067in}}%
\pgfpathlineto{\pgfqpoint{4.309054in}{0.962290in}}%
\pgfpathlineto{\pgfqpoint{4.346940in}{0.945186in}}%
\pgfpathlineto{\pgfqpoint{4.364105in}{0.936063in}}%
\pgfpathlineto{\pgfqpoint{4.364105in}{0.936063in}}%
\pgfusepath{stroke}%
\end{pgfscope}%
\begin{pgfscope}%
\pgfpathrectangle{\pgfqpoint{0.634105in}{0.521603in}}{\pgfqpoint{3.720000in}{3.020000in}} %
\pgfusepath{clip}%
\pgfsetrectcap%
\pgfsetroundjoin%
\pgfsetlinewidth{1.505625pt}%
\definecolor{currentstroke}{rgb}{0.692157,0.954791,0.592758}%
\pgfsetstrokecolor{currentstroke}%
\pgfsetdash{}{0pt}%
\pgfpathmoveto{\pgfqpoint{1.005025in}{0.959802in}}%
\pgfpathlineto{\pgfqpoint{1.051390in}{1.014789in}}%
\pgfpathlineto{\pgfqpoint{1.097755in}{1.057254in}}%
\pgfpathlineto{\pgfqpoint{1.144120in}{1.117153in}}%
\pgfpathlineto{\pgfqpoint{1.236850in}{1.209672in}}%
\pgfpathlineto{\pgfqpoint{1.283215in}{1.250496in}}%
\pgfpathlineto{\pgfqpoint{1.329580in}{1.298437in}}%
\pgfpathlineto{\pgfqpoint{1.375945in}{1.333786in}}%
\pgfpathlineto{\pgfqpoint{1.422310in}{1.376241in}}%
\pgfpathlineto{\pgfqpoint{1.468675in}{1.407268in}}%
\pgfpathlineto{\pgfqpoint{1.515040in}{1.444106in}}%
\pgfpathlineto{\pgfqpoint{1.561405in}{1.473029in}}%
\pgfpathlineto{\pgfqpoint{1.607770in}{1.504183in}}%
\pgfpathlineto{\pgfqpoint{1.654135in}{1.528567in}}%
\pgfpathlineto{\pgfqpoint{1.700500in}{1.557282in}}%
\pgfpathlineto{\pgfqpoint{1.746865in}{1.576819in}}%
\pgfpathlineto{\pgfqpoint{1.793230in}{1.601599in}}%
\pgfpathlineto{\pgfqpoint{1.839595in}{1.618028in}}%
\pgfpathlineto{\pgfqpoint{1.885960in}{1.638798in}}%
\pgfpathlineto{\pgfqpoint{1.932325in}{1.652580in}}%
\pgfpathlineto{\pgfqpoint{1.978690in}{1.669354in}}%
\pgfpathlineto{\pgfqpoint{2.025055in}{1.679216in}}%
\pgfpathlineto{\pgfqpoint{2.071420in}{1.691996in}}%
\pgfpathlineto{\pgfqpoint{2.117785in}{1.698521in}}%
\pgfpathlineto{\pgfqpoint{2.164150in}{1.707783in}}%
\pgfpathlineto{\pgfqpoint{2.210515in}{1.710738in}}%
\pgfpathlineto{\pgfqpoint{2.256880in}{1.716546in}}%
\pgfpathlineto{\pgfqpoint{2.303245in}{1.716226in}}%
\pgfpathlineto{\pgfqpoint{2.349610in}{1.717357in}}%
\pgfpathlineto{\pgfqpoint{2.395975in}{1.713471in}}%
\pgfpathlineto{\pgfqpoint{2.442340in}{1.711822in}}%
\pgfpathlineto{\pgfqpoint{2.488705in}{1.704411in}}%
\pgfpathlineto{\pgfqpoint{2.535070in}{1.698883in}}%
\pgfpathlineto{\pgfqpoint{2.581436in}{1.688101in}}%
\pgfpathlineto{\pgfqpoint{2.627801in}{1.679124in}}%
\pgfpathlineto{\pgfqpoint{2.674166in}{1.664506in}}%
\pgfpathlineto{\pgfqpoint{2.720531in}{1.651507in}}%
\pgfpathlineto{\pgfqpoint{2.766896in}{1.633553in}}%
\pgfpathlineto{\pgfqpoint{2.813261in}{1.617120in}}%
\pgfpathlineto{\pgfqpoint{2.859626in}{1.595177in}}%
\pgfpathlineto{\pgfqpoint{2.905991in}{1.574835in}}%
\pgfpathlineto{\pgfqpoint{2.998721in}{1.525582in}}%
\pgfpathlineto{\pgfqpoint{3.091451in}{1.468008in}}%
\pgfpathlineto{\pgfqpoint{3.137816in}{1.436240in}}%
\pgfpathlineto{\pgfqpoint{3.184181in}{1.402984in}}%
\pgfpathlineto{\pgfqpoint{3.276911in}{1.329761in}}%
\pgfpathlineto{\pgfqpoint{3.323276in}{1.289761in}}%
\pgfpathlineto{\pgfqpoint{3.369641in}{1.247800in}}%
\pgfpathlineto{\pgfqpoint{3.462371in}{1.156274in}}%
\pgfpathlineto{\pgfqpoint{3.508736in}{1.107239in}}%
\pgfpathlineto{\pgfqpoint{3.555101in}{1.055962in}}%
\pgfpathlineto{\pgfqpoint{3.601466in}{1.001396in}}%
\pgfpathlineto{\pgfqpoint{3.647831in}{0.944813in}}%
\pgfpathlineto{\pgfqpoint{3.694196in}{0.884659in}}%
\pgfpathlineto{\pgfqpoint{3.740561in}{0.819386in}}%
\pgfpathlineto{\pgfqpoint{3.786926in}{0.746162in}}%
\pgfpathlineto{\pgfqpoint{3.879656in}{0.593296in}}%
\pgfpathlineto{\pgfqpoint{3.926021in}{0.538569in}}%
\pgfpathlineto{\pgfqpoint{3.972386in}{0.521533in}}%
\pgfpathlineto{\pgfqpoint{4.018751in}{0.556254in}}%
\pgfpathlineto{\pgfqpoint{4.065116in}{0.621215in}}%
\pgfpathlineto{\pgfqpoint{4.111481in}{0.694189in}}%
\pgfpathlineto{\pgfqpoint{4.204211in}{0.817229in}}%
\pgfpathlineto{\pgfqpoint{4.296941in}{0.915495in}}%
\pgfpathlineto{\pgfqpoint{4.364105in}{0.975922in}}%
\pgfpathlineto{\pgfqpoint{4.364105in}{0.975922in}}%
\pgfusepath{stroke}%
\end{pgfscope}%
\begin{pgfscope}%
\pgfpathrectangle{\pgfqpoint{0.634105in}{0.521603in}}{\pgfqpoint{3.720000in}{3.020000in}} %
\pgfusepath{clip}%
\pgfsetrectcap%
\pgfsetroundjoin%
\pgfsetlinewidth{1.505625pt}%
\definecolor{currentstroke}{rgb}{0.268627,0.934680,0.823253}%
\pgfsetstrokecolor{currentstroke}%
\pgfsetdash{}{0pt}%
\pgfpathmoveto{\pgfqpoint{0.886853in}{0.511603in}}%
\pgfpathlineto{\pgfqpoint{0.927236in}{0.604079in}}%
\pgfpathlineto{\pgfqpoint{1.000519in}{0.712064in}}%
\pgfpathlineto{\pgfqpoint{1.073802in}{0.782778in}}%
\pgfpathlineto{\pgfqpoint{1.147085in}{0.868779in}}%
\pgfpathlineto{\pgfqpoint{1.220367in}{0.999497in}}%
\pgfpathlineto{\pgfqpoint{1.293650in}{1.110070in}}%
\pgfpathlineto{\pgfqpoint{1.366933in}{1.182386in}}%
\pgfpathlineto{\pgfqpoint{1.440216in}{1.253138in}}%
\pgfpathlineto{\pgfqpoint{1.513499in}{1.351997in}}%
\pgfpathlineto{\pgfqpoint{1.586781in}{1.418811in}}%
\pgfpathlineto{\pgfqpoint{1.660064in}{1.507132in}}%
\pgfpathlineto{\pgfqpoint{1.733347in}{1.569710in}}%
\pgfpathlineto{\pgfqpoint{1.806630in}{1.619925in}}%
\pgfpathlineto{\pgfqpoint{1.879913in}{1.689815in}}%
\pgfpathlineto{\pgfqpoint{1.953195in}{1.778534in}}%
\pgfpathlineto{\pgfqpoint{2.026478in}{1.850467in}}%
\pgfpathlineto{\pgfqpoint{2.099761in}{1.897632in}}%
\pgfpathlineto{\pgfqpoint{2.173044in}{1.948583in}}%
\pgfpathlineto{\pgfqpoint{2.246327in}{2.017705in}}%
\pgfpathlineto{\pgfqpoint{2.319610in}{2.092402in}}%
\pgfpathlineto{\pgfqpoint{2.392892in}{2.138831in}}%
\pgfpathlineto{\pgfqpoint{2.466175in}{2.174814in}}%
\pgfpathlineto{\pgfqpoint{2.539458in}{2.216274in}}%
\pgfpathlineto{\pgfqpoint{2.612741in}{2.277625in}}%
\pgfpathlineto{\pgfqpoint{2.686024in}{2.342775in}}%
\pgfpathlineto{\pgfqpoint{2.759306in}{2.379438in}}%
\pgfpathlineto{\pgfqpoint{2.832589in}{2.405449in}}%
\pgfpathlineto{\pgfqpoint{2.905872in}{2.450696in}}%
\pgfpathlineto{\pgfqpoint{2.979155in}{2.507117in}}%
\pgfpathlineto{\pgfqpoint{3.052438in}{2.548978in}}%
\pgfpathlineto{\pgfqpoint{3.125720in}{2.573876in}}%
\pgfpathlineto{\pgfqpoint{3.199003in}{2.594518in}}%
\pgfpathlineto{\pgfqpoint{3.272286in}{2.634118in}}%
\pgfpathlineto{\pgfqpoint{3.345569in}{2.681968in}}%
\pgfpathlineto{\pgfqpoint{3.418852in}{2.715123in}}%
\pgfpathlineto{\pgfqpoint{3.492135in}{2.725654in}}%
\pgfpathlineto{\pgfqpoint{3.565417in}{2.750951in}}%
\pgfpathlineto{\pgfqpoint{3.638700in}{2.790621in}}%
\pgfpathlineto{\pgfqpoint{3.711983in}{2.822393in}}%
\pgfpathlineto{\pgfqpoint{3.785266in}{2.838584in}}%
\pgfpathlineto{\pgfqpoint{3.858549in}{2.846121in}}%
\pgfpathlineto{\pgfqpoint{3.931831in}{2.864853in}}%
\pgfpathlineto{\pgfqpoint{4.005114in}{2.894217in}}%
\pgfpathlineto{\pgfqpoint{4.078397in}{2.919467in}}%
\pgfpathlineto{\pgfqpoint{4.151680in}{2.921178in}}%
\pgfpathlineto{\pgfqpoint{4.224963in}{2.927706in}}%
\pgfpathlineto{\pgfqpoint{4.364105in}{2.968102in}}%
\pgfpathlineto{\pgfqpoint{4.364105in}{2.968102in}}%
\pgfusepath{stroke}%
\end{pgfscope}%
\begin{pgfscope}%
\pgfpathrectangle{\pgfqpoint{0.634105in}{0.521603in}}{\pgfqpoint{3.720000in}{3.020000in}} %
\pgfusepath{clip}%
\pgfsetrectcap%
\pgfsetroundjoin%
\pgfsetlinewidth{1.505625pt}%
\definecolor{currentstroke}{rgb}{0.190196,0.883910,0.856638}%
\pgfsetstrokecolor{currentstroke}%
\pgfsetdash{}{0pt}%
\pgfpathmoveto{\pgfqpoint{1.427698in}{1.095716in}}%
\pgfpathlineto{\pgfqpoint{1.526897in}{1.236730in}}%
\pgfpathlineto{\pgfqpoint{1.626097in}{1.349912in}}%
\pgfpathlineto{\pgfqpoint{1.725296in}{1.450581in}}%
\pgfpathlineto{\pgfqpoint{1.824495in}{1.562787in}}%
\pgfpathlineto{\pgfqpoint{1.923694in}{1.672742in}}%
\pgfpathlineto{\pgfqpoint{2.022893in}{1.769751in}}%
\pgfpathlineto{\pgfqpoint{2.122092in}{1.882064in}}%
\pgfpathlineto{\pgfqpoint{2.320491in}{2.076316in}}%
\pgfpathlineto{\pgfqpoint{2.419690in}{2.178363in}}%
\pgfpathlineto{\pgfqpoint{2.518889in}{2.269153in}}%
\pgfpathlineto{\pgfqpoint{2.618088in}{2.367452in}}%
\pgfpathlineto{\pgfqpoint{2.717287in}{2.463918in}}%
\pgfpathlineto{\pgfqpoint{2.816487in}{2.546404in}}%
\pgfpathlineto{\pgfqpoint{2.915686in}{2.633721in}}%
\pgfpathlineto{\pgfqpoint{3.014885in}{2.725968in}}%
\pgfpathlineto{\pgfqpoint{3.114084in}{2.803482in}}%
\pgfpathlineto{\pgfqpoint{3.213283in}{2.888482in}}%
\pgfpathlineto{\pgfqpoint{3.312482in}{2.976580in}}%
\pgfpathlineto{\pgfqpoint{3.411682in}{3.049839in}}%
\pgfpathlineto{\pgfqpoint{3.510881in}{3.126467in}}%
\pgfpathlineto{\pgfqpoint{3.610080in}{3.207450in}}%
\pgfpathlineto{\pgfqpoint{3.709279in}{3.274279in}}%
\pgfpathlineto{\pgfqpoint{3.808478in}{3.350224in}}%
\pgfpathlineto{\pgfqpoint{3.907678in}{3.419275in}}%
\pgfpathlineto{\pgfqpoint{4.006877in}{3.483556in}}%
\pgfpathlineto{\pgfqpoint{4.104526in}{3.551603in}}%
\pgfpathlineto{\pgfqpoint{4.104526in}{3.551603in}}%
\pgfusepath{stroke}%
\end{pgfscope}%
\begin{pgfscope}%
\pgfpathrectangle{\pgfqpoint{0.634105in}{0.521603in}}{\pgfqpoint{3.720000in}{3.020000in}} %
\pgfusepath{clip}%
\pgfsetrectcap%
\pgfsetroundjoin%
\pgfsetlinewidth{1.505625pt}%
\definecolor{currentstroke}{rgb}{0.052941,0.645928,0.938988}%
\pgfsetstrokecolor{currentstroke}%
\pgfsetdash{}{0pt}%
\pgfpathmoveto{\pgfqpoint{1.742430in}{1.376272in}}%
\pgfpathlineto{\pgfqpoint{1.880970in}{1.467682in}}%
\pgfpathlineto{\pgfqpoint{2.019511in}{1.605567in}}%
\pgfpathlineto{\pgfqpoint{2.158051in}{1.863657in}}%
\pgfpathlineto{\pgfqpoint{2.296592in}{2.010400in}}%
\pgfpathlineto{\pgfqpoint{2.435133in}{2.069316in}}%
\pgfpathlineto{\pgfqpoint{2.573673in}{2.274797in}}%
\pgfpathlineto{\pgfqpoint{2.712214in}{2.470861in}}%
\pgfpathlineto{\pgfqpoint{2.850754in}{2.591980in}}%
\pgfpathlineto{\pgfqpoint{2.989295in}{2.695523in}}%
\pgfpathlineto{\pgfqpoint{3.127836in}{2.855530in}}%
\pgfpathlineto{\pgfqpoint{3.266376in}{3.020905in}}%
\pgfpathlineto{\pgfqpoint{3.404917in}{3.138215in}}%
\pgfpathlineto{\pgfqpoint{3.543457in}{3.249992in}}%
\pgfpathlineto{\pgfqpoint{3.681998in}{3.382659in}}%
\pgfpathlineto{\pgfqpoint{3.820539in}{3.537448in}}%
\pgfpathlineto{\pgfqpoint{3.837893in}{3.551603in}}%
\pgfpathlineto{\pgfqpoint{3.837893in}{3.551603in}}%
\pgfusepath{stroke}%
\end{pgfscope}%
\begin{pgfscope}%
\pgfpathrectangle{\pgfqpoint{0.634105in}{0.521603in}}{\pgfqpoint{3.720000in}{3.020000in}} %
\pgfusepath{clip}%
\pgfsetrectcap%
\pgfsetroundjoin%
\pgfsetlinewidth{1.505625pt}%
\definecolor{currentstroke}{rgb}{0.162745,0.505325,0.965124}%
\pgfsetstrokecolor{currentstroke}%
\pgfsetdash{}{0pt}%
\pgfpathmoveto{\pgfqpoint{2.272311in}{1.950771in}}%
\pgfpathlineto{\pgfqpoint{2.421239in}{2.006614in}}%
\pgfpathlineto{\pgfqpoint{2.570167in}{2.262977in}}%
\pgfpathlineto{\pgfqpoint{2.719095in}{2.477309in}}%
\pgfpathlineto{\pgfqpoint{2.868023in}{2.601283in}}%
\pgfpathlineto{\pgfqpoint{3.016950in}{2.758502in}}%
\pgfpathlineto{\pgfqpoint{3.165878in}{2.951134in}}%
\pgfpathlineto{\pgfqpoint{3.314806in}{3.172912in}}%
\pgfpathlineto{\pgfqpoint{3.463734in}{3.325380in}}%
\pgfpathlineto{\pgfqpoint{3.612662in}{3.453764in}}%
\pgfpathlineto{\pgfqpoint{3.696764in}{3.551603in}}%
\pgfpathlineto{\pgfqpoint{3.696764in}{3.551603in}}%
\pgfusepath{stroke}%
\end{pgfscope}%
\begin{pgfscope}%
\pgfpathrectangle{\pgfqpoint{0.634105in}{0.521603in}}{\pgfqpoint{3.720000in}{3.020000in}} %
\pgfusepath{clip}%
\pgfsetrectcap%
\pgfsetroundjoin%
\pgfsetlinewidth{1.505625pt}%
\definecolor{currentstroke}{rgb}{0.205882,0.895163,0.850217}%
\pgfsetstrokecolor{currentstroke}%
\pgfsetdash{}{0pt}%
\pgfpathmoveto{\pgfqpoint{1.483883in}{1.248124in}}%
\pgfpathlineto{\pgfqpoint{1.605280in}{1.371664in}}%
\pgfpathlineto{\pgfqpoint{1.726677in}{1.510548in}}%
\pgfpathlineto{\pgfqpoint{1.848074in}{1.640228in}}%
\pgfpathlineto{\pgfqpoint{1.969471in}{1.800315in}}%
\pgfpathlineto{\pgfqpoint{2.090868in}{1.921024in}}%
\pgfpathlineto{\pgfqpoint{2.333662in}{2.172306in}}%
\pgfpathlineto{\pgfqpoint{2.455059in}{2.305984in}}%
\pgfpathlineto{\pgfqpoint{2.576456in}{2.433721in}}%
\pgfpathlineto{\pgfqpoint{2.697853in}{2.557386in}}%
\pgfpathlineto{\pgfqpoint{2.819250in}{2.671996in}}%
\pgfpathlineto{\pgfqpoint{2.940647in}{2.798322in}}%
\pgfpathlineto{\pgfqpoint{3.062044in}{2.910781in}}%
\pgfpathlineto{\pgfqpoint{3.183440in}{3.020746in}}%
\pgfpathlineto{\pgfqpoint{3.304837in}{3.139999in}}%
\pgfpathlineto{\pgfqpoint{3.426234in}{3.242866in}}%
\pgfpathlineto{\pgfqpoint{3.547631in}{3.354918in}}%
\pgfpathlineto{\pgfqpoint{3.669028in}{3.452036in}}%
\pgfpathlineto{\pgfqpoint{3.778976in}{3.551603in}}%
\pgfpathlineto{\pgfqpoint{3.778976in}{3.551603in}}%
\pgfusepath{stroke}%
\end{pgfscope}%
\begin{pgfscope}%
\pgfpathrectangle{\pgfqpoint{0.634105in}{0.521603in}}{\pgfqpoint{3.720000in}{3.020000in}} %
\pgfusepath{clip}%
\pgfsetrectcap%
\pgfsetroundjoin%
\pgfsetlinewidth{1.505625pt}%
\definecolor{currentstroke}{rgb}{0.076471,0.617278,0.945184}%
\pgfsetstrokecolor{currentstroke}%
\pgfsetdash{}{0pt}%
\pgfpathmoveto{\pgfqpoint{1.996700in}{1.640427in}}%
\pgfpathlineto{\pgfqpoint{2.386013in}{2.084118in}}%
\pgfpathlineto{\pgfqpoint{2.580669in}{2.331415in}}%
\pgfpathlineto{\pgfqpoint{2.775326in}{2.553187in}}%
\pgfpathlineto{\pgfqpoint{2.969982in}{2.763132in}}%
\pgfpathlineto{\pgfqpoint{3.164639in}{3.002485in}}%
\pgfpathlineto{\pgfqpoint{3.359295in}{3.187254in}}%
\pgfpathlineto{\pgfqpoint{3.553952in}{3.411432in}}%
\pgfpathlineto{\pgfqpoint{3.687544in}{3.551603in}}%
\pgfpathlineto{\pgfqpoint{3.687544in}{3.551603in}}%
\pgfusepath{stroke}%
\end{pgfscope}%
\begin{pgfscope}%
\pgfpathrectangle{\pgfqpoint{0.634105in}{0.521603in}}{\pgfqpoint{3.720000in}{3.020000in}} %
\pgfusepath{clip}%
\pgfsetrectcap%
\pgfsetroundjoin%
\pgfsetlinewidth{1.505625pt}%
\definecolor{currentstroke}{rgb}{0.500000,0.000000,1.000000}%
\pgfsetstrokecolor{currentstroke}%
\pgfsetdash{}{0pt}%
\pgfpathmoveto{\pgfqpoint{2.761471in}{1.816645in}}%
\pgfpathlineto{\pgfqpoint{3.470593in}{2.746592in}}%
\pgfpathlineto{\pgfqpoint{3.825154in}{3.120487in}}%
\pgfpathlineto{\pgfqpoint{4.179716in}{3.478450in}}%
\pgfpathlineto{\pgfqpoint{4.234934in}{3.551603in}}%
\pgfpathlineto{\pgfqpoint{4.234934in}{3.551603in}}%
\pgfusepath{stroke}%
\end{pgfscope}%
\begin{pgfscope}%
\pgfsetrectcap%
\pgfsetmiterjoin%
\pgfsetlinewidth{0.803000pt}%
\definecolor{currentstroke}{rgb}{0.000000,0.000000,0.000000}%
\pgfsetstrokecolor{currentstroke}%
\pgfsetdash{}{0pt}%
\pgfpathmoveto{\pgfqpoint{0.634105in}{0.521603in}}%
\pgfpathlineto{\pgfqpoint{0.634105in}{3.541603in}}%
\pgfusepath{stroke}%
\end{pgfscope}%
\begin{pgfscope}%
\pgfsetrectcap%
\pgfsetmiterjoin%
\pgfsetlinewidth{0.803000pt}%
\definecolor{currentstroke}{rgb}{0.000000,0.000000,0.000000}%
\pgfsetstrokecolor{currentstroke}%
\pgfsetdash{}{0pt}%
\pgfpathmoveto{\pgfqpoint{4.354105in}{0.521603in}}%
\pgfpathlineto{\pgfqpoint{4.354105in}{3.541603in}}%
\pgfusepath{stroke}%
\end{pgfscope}%
\begin{pgfscope}%
\pgfsetrectcap%
\pgfsetmiterjoin%
\pgfsetlinewidth{0.803000pt}%
\definecolor{currentstroke}{rgb}{0.000000,0.000000,0.000000}%
\pgfsetstrokecolor{currentstroke}%
\pgfsetdash{}{0pt}%
\pgfpathmoveto{\pgfqpoint{0.634105in}{0.521603in}}%
\pgfpathlineto{\pgfqpoint{4.354105in}{0.521603in}}%
\pgfusepath{stroke}%
\end{pgfscope}%
\begin{pgfscope}%
\pgfsetrectcap%
\pgfsetmiterjoin%
\pgfsetlinewidth{0.803000pt}%
\definecolor{currentstroke}{rgb}{0.000000,0.000000,0.000000}%
\pgfsetstrokecolor{currentstroke}%
\pgfsetdash{}{0pt}%
\pgfpathmoveto{\pgfqpoint{0.634105in}{3.541603in}}%
\pgfpathlineto{\pgfqpoint{4.354105in}{3.541603in}}%
\pgfusepath{stroke}%
\end{pgfscope}%
\begin{pgfscope}%
\pgfpathrectangle{\pgfqpoint{4.586605in}{0.521603in}}{\pgfqpoint{0.151000in}{3.020000in}} %
\pgfusepath{clip}%
\pgfsetbuttcap%
\pgfsetmiterjoin%
\definecolor{currentfill}{rgb}{1.000000,1.000000,1.000000}%
\pgfsetfillcolor{currentfill}%
\pgfsetlinewidth{0.010037pt}%
\definecolor{currentstroke}{rgb}{1.000000,1.000000,1.000000}%
\pgfsetstrokecolor{currentstroke}%
\pgfsetdash{}{0pt}%
\pgfpathmoveto{\pgfqpoint{4.586605in}{0.521603in}}%
\pgfpathlineto{\pgfqpoint{4.586605in}{0.533400in}}%
\pgfpathlineto{\pgfqpoint{4.586605in}{3.529806in}}%
\pgfpathlineto{\pgfqpoint{4.586605in}{3.541603in}}%
\pgfpathlineto{\pgfqpoint{4.737605in}{3.541603in}}%
\pgfpathlineto{\pgfqpoint{4.737605in}{3.529806in}}%
\pgfpathlineto{\pgfqpoint{4.737605in}{0.533400in}}%
\pgfpathlineto{\pgfqpoint{4.737605in}{0.521603in}}%
\pgfpathclose%
\pgfusepath{stroke,fill}%
\end{pgfscope}%
\begin{pgfscope}%
\pgfsys@transformshift{4.590000in}{0.524365in}%
\pgftext[left,bottom]{\pgfimage[interpolate=true,width=0.150000in,height=3.020000in]{series_m_ds-img0.png}}%
\end{pgfscope}%
\begin{pgfscope}%
\pgfsetbuttcap%
\pgfsetroundjoin%
\definecolor{currentfill}{rgb}{0.000000,0.000000,0.000000}%
\pgfsetfillcolor{currentfill}%
\pgfsetlinewidth{0.803000pt}%
\definecolor{currentstroke}{rgb}{0.000000,0.000000,0.000000}%
\pgfsetstrokecolor{currentstroke}%
\pgfsetdash{}{0pt}%
\pgfsys@defobject{currentmarker}{\pgfqpoint{0.000000in}{0.000000in}}{\pgfqpoint{0.048611in}{0.000000in}}{%
\pgfpathmoveto{\pgfqpoint{0.000000in}{0.000000in}}%
\pgfpathlineto{\pgfqpoint{0.048611in}{0.000000in}}%
\pgfusepath{stroke,fill}%
}%
\begin{pgfscope}%
\pgfsys@transformshift{4.737605in}{0.617964in}%
\pgfsys@useobject{currentmarker}{}%
\end{pgfscope}%
\end{pgfscope}%
\begin{pgfscope}%
\pgfsetbuttcap%
\pgfsetroundjoin%
\definecolor{currentfill}{rgb}{0.000000,0.000000,0.000000}%
\pgfsetfillcolor{currentfill}%
\pgfsetlinewidth{0.803000pt}%
\definecolor{currentstroke}{rgb}{0.000000,0.000000,0.000000}%
\pgfsetstrokecolor{currentstroke}%
\pgfsetdash{}{0pt}%
\pgfsys@defobject{currentmarker}{\pgfqpoint{0.000000in}{0.000000in}}{\pgfqpoint{0.048611in}{0.000000in}}{%
\pgfpathmoveto{\pgfqpoint{0.000000in}{0.000000in}}%
\pgfpathlineto{\pgfqpoint{0.048611in}{0.000000in}}%
\pgfusepath{stroke,fill}%
}%
\begin{pgfscope}%
\pgfsys@transformshift{4.737605in}{0.796036in}%
\pgfsys@useobject{currentmarker}{}%
\end{pgfscope}%
\end{pgfscope}%
\begin{pgfscope}%
\pgfsetbuttcap%
\pgfsetroundjoin%
\definecolor{currentfill}{rgb}{0.000000,0.000000,0.000000}%
\pgfsetfillcolor{currentfill}%
\pgfsetlinewidth{0.803000pt}%
\definecolor{currentstroke}{rgb}{0.000000,0.000000,0.000000}%
\pgfsetstrokecolor{currentstroke}%
\pgfsetdash{}{0pt}%
\pgfsys@defobject{currentmarker}{\pgfqpoint{0.000000in}{0.000000in}}{\pgfqpoint{0.048611in}{0.000000in}}{%
\pgfpathmoveto{\pgfqpoint{0.000000in}{0.000000in}}%
\pgfpathlineto{\pgfqpoint{0.048611in}{0.000000in}}%
\pgfusepath{stroke,fill}%
}%
\begin{pgfscope}%
\pgfsys@transformshift{4.737605in}{0.941531in}%
\pgfsys@useobject{currentmarker}{}%
\end{pgfscope}%
\end{pgfscope}%
\begin{pgfscope}%
\pgfsetbuttcap%
\pgfsetroundjoin%
\definecolor{currentfill}{rgb}{0.000000,0.000000,0.000000}%
\pgfsetfillcolor{currentfill}%
\pgfsetlinewidth{0.803000pt}%
\definecolor{currentstroke}{rgb}{0.000000,0.000000,0.000000}%
\pgfsetstrokecolor{currentstroke}%
\pgfsetdash{}{0pt}%
\pgfsys@defobject{currentmarker}{\pgfqpoint{0.000000in}{0.000000in}}{\pgfqpoint{0.048611in}{0.000000in}}{%
\pgfpathmoveto{\pgfqpoint{0.000000in}{0.000000in}}%
\pgfpathlineto{\pgfqpoint{0.048611in}{0.000000in}}%
\pgfusepath{stroke,fill}%
}%
\begin{pgfscope}%
\pgfsys@transformshift{4.737605in}{1.064545in}%
\pgfsys@useobject{currentmarker}{}%
\end{pgfscope}%
\end{pgfscope}%
\begin{pgfscope}%
\pgfsetbuttcap%
\pgfsetroundjoin%
\definecolor{currentfill}{rgb}{0.000000,0.000000,0.000000}%
\pgfsetfillcolor{currentfill}%
\pgfsetlinewidth{0.803000pt}%
\definecolor{currentstroke}{rgb}{0.000000,0.000000,0.000000}%
\pgfsetstrokecolor{currentstroke}%
\pgfsetdash{}{0pt}%
\pgfsys@defobject{currentmarker}{\pgfqpoint{0.000000in}{0.000000in}}{\pgfqpoint{0.048611in}{0.000000in}}{%
\pgfpathmoveto{\pgfqpoint{0.000000in}{0.000000in}}%
\pgfpathlineto{\pgfqpoint{0.048611in}{0.000000in}}%
\pgfusepath{stroke,fill}%
}%
\begin{pgfscope}%
\pgfsys@transformshift{4.737605in}{1.171104in}%
\pgfsys@useobject{currentmarker}{}%
\end{pgfscope}%
\end{pgfscope}%
\begin{pgfscope}%
\pgfsetbuttcap%
\pgfsetroundjoin%
\definecolor{currentfill}{rgb}{0.000000,0.000000,0.000000}%
\pgfsetfillcolor{currentfill}%
\pgfsetlinewidth{0.803000pt}%
\definecolor{currentstroke}{rgb}{0.000000,0.000000,0.000000}%
\pgfsetstrokecolor{currentstroke}%
\pgfsetdash{}{0pt}%
\pgfsys@defobject{currentmarker}{\pgfqpoint{0.000000in}{0.000000in}}{\pgfqpoint{0.048611in}{0.000000in}}{%
\pgfpathmoveto{\pgfqpoint{0.000000in}{0.000000in}}%
\pgfpathlineto{\pgfqpoint{0.048611in}{0.000000in}}%
\pgfusepath{stroke,fill}%
}%
\begin{pgfscope}%
\pgfsys@transformshift{4.737605in}{1.265097in}%
\pgfsys@useobject{currentmarker}{}%
\end{pgfscope}%
\end{pgfscope}%
\begin{pgfscope}%
\pgfsetbuttcap%
\pgfsetroundjoin%
\definecolor{currentfill}{rgb}{0.000000,0.000000,0.000000}%
\pgfsetfillcolor{currentfill}%
\pgfsetlinewidth{0.803000pt}%
\definecolor{currentstroke}{rgb}{0.000000,0.000000,0.000000}%
\pgfsetstrokecolor{currentstroke}%
\pgfsetdash{}{0pt}%
\pgfsys@defobject{currentmarker}{\pgfqpoint{0.000000in}{0.000000in}}{\pgfqpoint{0.048611in}{0.000000in}}{%
\pgfpathmoveto{\pgfqpoint{0.000000in}{0.000000in}}%
\pgfpathlineto{\pgfqpoint{0.048611in}{0.000000in}}%
\pgfusepath{stroke,fill}%
}%
\begin{pgfscope}%
\pgfsys@transformshift{4.737605in}{1.349176in}%
\pgfsys@useobject{currentmarker}{}%
\end{pgfscope}%
\end{pgfscope}%
\begin{pgfscope}%
\pgftext[x=4.834827in,y=1.296414in,left,base]{\rmfamily\fontsize{10.000000}{12.000000}\selectfont \(\displaystyle 10^{-1}\)}%
\end{pgfscope}%
\begin{pgfscope}%
\pgfsetbuttcap%
\pgfsetroundjoin%
\definecolor{currentfill}{rgb}{0.000000,0.000000,0.000000}%
\pgfsetfillcolor{currentfill}%
\pgfsetlinewidth{0.803000pt}%
\definecolor{currentstroke}{rgb}{0.000000,0.000000,0.000000}%
\pgfsetstrokecolor{currentstroke}%
\pgfsetdash{}{0pt}%
\pgfsys@defobject{currentmarker}{\pgfqpoint{0.000000in}{0.000000in}}{\pgfqpoint{0.048611in}{0.000000in}}{%
\pgfpathmoveto{\pgfqpoint{0.000000in}{0.000000in}}%
\pgfpathlineto{\pgfqpoint{0.048611in}{0.000000in}}%
\pgfusepath{stroke,fill}%
}%
\begin{pgfscope}%
\pgfsys@transformshift{4.737605in}{1.902316in}%
\pgfsys@useobject{currentmarker}{}%
\end{pgfscope}%
\end{pgfscope}%
\begin{pgfscope}%
\pgfsetbuttcap%
\pgfsetroundjoin%
\definecolor{currentfill}{rgb}{0.000000,0.000000,0.000000}%
\pgfsetfillcolor{currentfill}%
\pgfsetlinewidth{0.803000pt}%
\definecolor{currentstroke}{rgb}{0.000000,0.000000,0.000000}%
\pgfsetstrokecolor{currentstroke}%
\pgfsetdash{}{0pt}%
\pgfsys@defobject{currentmarker}{\pgfqpoint{0.000000in}{0.000000in}}{\pgfqpoint{0.048611in}{0.000000in}}{%
\pgfpathmoveto{\pgfqpoint{0.000000in}{0.000000in}}%
\pgfpathlineto{\pgfqpoint{0.048611in}{0.000000in}}%
\pgfusepath{stroke,fill}%
}%
\begin{pgfscope}%
\pgfsys@transformshift{4.737605in}{2.225882in}%
\pgfsys@useobject{currentmarker}{}%
\end{pgfscope}%
\end{pgfscope}%
\begin{pgfscope}%
\pgfsetbuttcap%
\pgfsetroundjoin%
\definecolor{currentfill}{rgb}{0.000000,0.000000,0.000000}%
\pgfsetfillcolor{currentfill}%
\pgfsetlinewidth{0.803000pt}%
\definecolor{currentstroke}{rgb}{0.000000,0.000000,0.000000}%
\pgfsetstrokecolor{currentstroke}%
\pgfsetdash{}{0pt}%
\pgfsys@defobject{currentmarker}{\pgfqpoint{0.000000in}{0.000000in}}{\pgfqpoint{0.048611in}{0.000000in}}{%
\pgfpathmoveto{\pgfqpoint{0.000000in}{0.000000in}}%
\pgfpathlineto{\pgfqpoint{0.048611in}{0.000000in}}%
\pgfusepath{stroke,fill}%
}%
\begin{pgfscope}%
\pgfsys@transformshift{4.737605in}{2.455456in}%
\pgfsys@useobject{currentmarker}{}%
\end{pgfscope}%
\end{pgfscope}%
\begin{pgfscope}%
\pgfsetbuttcap%
\pgfsetroundjoin%
\definecolor{currentfill}{rgb}{0.000000,0.000000,0.000000}%
\pgfsetfillcolor{currentfill}%
\pgfsetlinewidth{0.803000pt}%
\definecolor{currentstroke}{rgb}{0.000000,0.000000,0.000000}%
\pgfsetstrokecolor{currentstroke}%
\pgfsetdash{}{0pt}%
\pgfsys@defobject{currentmarker}{\pgfqpoint{0.000000in}{0.000000in}}{\pgfqpoint{0.048611in}{0.000000in}}{%
\pgfpathmoveto{\pgfqpoint{0.000000in}{0.000000in}}%
\pgfpathlineto{\pgfqpoint{0.048611in}{0.000000in}}%
\pgfusepath{stroke,fill}%
}%
\begin{pgfscope}%
\pgfsys@transformshift{4.737605in}{2.633527in}%
\pgfsys@useobject{currentmarker}{}%
\end{pgfscope}%
\end{pgfscope}%
\begin{pgfscope}%
\pgfsetbuttcap%
\pgfsetroundjoin%
\definecolor{currentfill}{rgb}{0.000000,0.000000,0.000000}%
\pgfsetfillcolor{currentfill}%
\pgfsetlinewidth{0.803000pt}%
\definecolor{currentstroke}{rgb}{0.000000,0.000000,0.000000}%
\pgfsetstrokecolor{currentstroke}%
\pgfsetdash{}{0pt}%
\pgfsys@defobject{currentmarker}{\pgfqpoint{0.000000in}{0.000000in}}{\pgfqpoint{0.048611in}{0.000000in}}{%
\pgfpathmoveto{\pgfqpoint{0.000000in}{0.000000in}}%
\pgfpathlineto{\pgfqpoint{0.048611in}{0.000000in}}%
\pgfusepath{stroke,fill}%
}%
\begin{pgfscope}%
\pgfsys@transformshift{4.737605in}{2.779022in}%
\pgfsys@useobject{currentmarker}{}%
\end{pgfscope}%
\end{pgfscope}%
\begin{pgfscope}%
\pgfsetbuttcap%
\pgfsetroundjoin%
\definecolor{currentfill}{rgb}{0.000000,0.000000,0.000000}%
\pgfsetfillcolor{currentfill}%
\pgfsetlinewidth{0.803000pt}%
\definecolor{currentstroke}{rgb}{0.000000,0.000000,0.000000}%
\pgfsetstrokecolor{currentstroke}%
\pgfsetdash{}{0pt}%
\pgfsys@defobject{currentmarker}{\pgfqpoint{0.000000in}{0.000000in}}{\pgfqpoint{0.048611in}{0.000000in}}{%
\pgfpathmoveto{\pgfqpoint{0.000000in}{0.000000in}}%
\pgfpathlineto{\pgfqpoint{0.048611in}{0.000000in}}%
\pgfusepath{stroke,fill}%
}%
\begin{pgfscope}%
\pgfsys@transformshift{4.737605in}{2.902036in}%
\pgfsys@useobject{currentmarker}{}%
\end{pgfscope}%
\end{pgfscope}%
\begin{pgfscope}%
\pgfsetbuttcap%
\pgfsetroundjoin%
\definecolor{currentfill}{rgb}{0.000000,0.000000,0.000000}%
\pgfsetfillcolor{currentfill}%
\pgfsetlinewidth{0.803000pt}%
\definecolor{currentstroke}{rgb}{0.000000,0.000000,0.000000}%
\pgfsetstrokecolor{currentstroke}%
\pgfsetdash{}{0pt}%
\pgfsys@defobject{currentmarker}{\pgfqpoint{0.000000in}{0.000000in}}{\pgfqpoint{0.048611in}{0.000000in}}{%
\pgfpathmoveto{\pgfqpoint{0.000000in}{0.000000in}}%
\pgfpathlineto{\pgfqpoint{0.048611in}{0.000000in}}%
\pgfusepath{stroke,fill}%
}%
\begin{pgfscope}%
\pgfsys@transformshift{4.737605in}{3.008596in}%
\pgfsys@useobject{currentmarker}{}%
\end{pgfscope}%
\end{pgfscope}%
\begin{pgfscope}%
\pgfsetbuttcap%
\pgfsetroundjoin%
\definecolor{currentfill}{rgb}{0.000000,0.000000,0.000000}%
\pgfsetfillcolor{currentfill}%
\pgfsetlinewidth{0.803000pt}%
\definecolor{currentstroke}{rgb}{0.000000,0.000000,0.000000}%
\pgfsetstrokecolor{currentstroke}%
\pgfsetdash{}{0pt}%
\pgfsys@defobject{currentmarker}{\pgfqpoint{0.000000in}{0.000000in}}{\pgfqpoint{0.048611in}{0.000000in}}{%
\pgfpathmoveto{\pgfqpoint{0.000000in}{0.000000in}}%
\pgfpathlineto{\pgfqpoint{0.048611in}{0.000000in}}%
\pgfusepath{stroke,fill}%
}%
\begin{pgfscope}%
\pgfsys@transformshift{4.737605in}{3.102588in}%
\pgfsys@useobject{currentmarker}{}%
\end{pgfscope}%
\end{pgfscope}%
\begin{pgfscope}%
\pgfsetbuttcap%
\pgfsetroundjoin%
\definecolor{currentfill}{rgb}{0.000000,0.000000,0.000000}%
\pgfsetfillcolor{currentfill}%
\pgfsetlinewidth{0.803000pt}%
\definecolor{currentstroke}{rgb}{0.000000,0.000000,0.000000}%
\pgfsetstrokecolor{currentstroke}%
\pgfsetdash{}{0pt}%
\pgfsys@defobject{currentmarker}{\pgfqpoint{0.000000in}{0.000000in}}{\pgfqpoint{0.048611in}{0.000000in}}{%
\pgfpathmoveto{\pgfqpoint{0.000000in}{0.000000in}}%
\pgfpathlineto{\pgfqpoint{0.048611in}{0.000000in}}%
\pgfusepath{stroke,fill}%
}%
\begin{pgfscope}%
\pgfsys@transformshift{4.737605in}{3.186667in}%
\pgfsys@useobject{currentmarker}{}%
\end{pgfscope}%
\end{pgfscope}%
\begin{pgfscope}%
\pgftext[x=4.834827in,y=3.133906in,left,base]{\rmfamily\fontsize{10.000000}{12.000000}\selectfont \(\displaystyle 10^{0}\)}%
\end{pgfscope}%
\begin{pgfscope}%
\pgftext[x=5.317274in,y=2.031603in,,top]{\rmfamily\fontsize{12.000000}{14.400000}\selectfont \(\displaystyle {\mathbf{E} \mbox{u}}\)}%
\end{pgfscope}%
\begin{pgfscope}%
\pgfsetbuttcap%
\pgfsetmiterjoin%
\pgfsetlinewidth{0.803000pt}%
\definecolor{currentstroke}{rgb}{0.000000,0.000000,0.000000}%
\pgfsetstrokecolor{currentstroke}%
\pgfsetdash{}{0pt}%
\pgfpathmoveto{\pgfqpoint{4.586605in}{0.521603in}}%
\pgfpathlineto{\pgfqpoint{4.586605in}{0.533400in}}%
\pgfpathlineto{\pgfqpoint{4.586605in}{3.529806in}}%
\pgfpathlineto{\pgfqpoint{4.586605in}{3.541603in}}%
\pgfpathlineto{\pgfqpoint{4.737605in}{3.541603in}}%
\pgfpathlineto{\pgfqpoint{4.737605in}{3.529806in}}%
\pgfpathlineto{\pgfqpoint{4.737605in}{0.533400in}}%
\pgfpathlineto{\pgfqpoint{4.737605in}{0.521603in}}%
\pgfpathclose%
\pgfusepath{stroke}%
\end{pgfscope}%
\end{pgfpicture}%
\makeatother%
\endgroup%

    \caption{.\label{fig:dnumbs}}
\end{figure}
\begin{figure}[htb]
    \centering
    %% Creator: Matplotlib, PGF backend
%%
%% To include the figure in your LaTeX document, write
%%   \input{<filename>.pgf}
%%
%% Make sure the required packages are loaded in your preamble
%%   \usepackage{pgf}
%%
%% Figures using additional raster images can only be included by \input if
%% they are in the same directory as the main LaTeX file. For loading figures
%% from other directories you can use the `import` package
%%   \usepackage{import}
%% and then include the figures with
%%   \import{<path to file>}{<filename>.pgf}
%%
%% Matplotlib used the following preamble
%%   \usepackage{fontspec}
%%   \setmainfont{DejaVu Serif}
%%   \setsansfont{DejaVu Sans}
%%   \setmonofont{DejaVu Sans Mono}
%%
\begingroup%
\makeatletter%
\begin{pgfpicture}%
\pgfpathrectangle{\pgfpointorigin}{\pgfqpoint{5.426437in}{3.676603in}}%
\pgfusepath{use as bounding box, clip}%
\begin{pgfscope}%
\pgfsetbuttcap%
\pgfsetmiterjoin%
\definecolor{currentfill}{rgb}{1.000000,1.000000,1.000000}%
\pgfsetfillcolor{currentfill}%
\pgfsetlinewidth{0.000000pt}%
\definecolor{currentstroke}{rgb}{1.000000,1.000000,1.000000}%
\pgfsetstrokecolor{currentstroke}%
\pgfsetdash{}{0pt}%
\pgfpathmoveto{\pgfqpoint{0.000000in}{0.000000in}}%
\pgfpathlineto{\pgfqpoint{5.426437in}{0.000000in}}%
\pgfpathlineto{\pgfqpoint{5.426437in}{3.676603in}}%
\pgfpathlineto{\pgfqpoint{0.000000in}{3.676603in}}%
\pgfpathclose%
\pgfusepath{fill}%
\end{pgfscope}%
\begin{pgfscope}%
\pgfsetbuttcap%
\pgfsetmiterjoin%
\definecolor{currentfill}{rgb}{1.000000,1.000000,1.000000}%
\pgfsetfillcolor{currentfill}%
\pgfsetlinewidth{0.000000pt}%
\definecolor{currentstroke}{rgb}{0.000000,0.000000,0.000000}%
\pgfsetstrokecolor{currentstroke}%
\pgfsetstrokeopacity{0.000000}%
\pgfsetdash{}{0pt}%
\pgfpathmoveto{\pgfqpoint{0.526080in}{0.521603in}}%
\pgfpathlineto{\pgfqpoint{4.246080in}{0.521603in}}%
\pgfpathlineto{\pgfqpoint{4.246080in}{3.541603in}}%
\pgfpathlineto{\pgfqpoint{0.526080in}{3.541603in}}%
\pgfpathclose%
\pgfusepath{fill}%
\end{pgfscope}%
\begin{pgfscope}%
\pgfsetbuttcap%
\pgfsetroundjoin%
\definecolor{currentfill}{rgb}{0.000000,0.000000,0.000000}%
\pgfsetfillcolor{currentfill}%
\pgfsetlinewidth{0.803000pt}%
\definecolor{currentstroke}{rgb}{0.000000,0.000000,0.000000}%
\pgfsetstrokecolor{currentstroke}%
\pgfsetdash{}{0pt}%
\pgfsys@defobject{currentmarker}{\pgfqpoint{0.000000in}{-0.048611in}}{\pgfqpoint{0.000000in}{0.000000in}}{%
\pgfpathmoveto{\pgfqpoint{0.000000in}{0.000000in}}%
\pgfpathlineto{\pgfqpoint{0.000000in}{-0.048611in}}%
\pgfusepath{stroke,fill}%
}%
\begin{pgfscope}%
\pgfsys@transformshift{0.692556in}{0.521603in}%
\pgfsys@useobject{currentmarker}{}%
\end{pgfscope}%
\end{pgfscope}%
\begin{pgfscope}%
\pgftext[x=0.692556in,y=0.424381in,,top]{\rmfamily\fontsize{10.000000}{12.000000}\selectfont \(\displaystyle 0\)}%
\end{pgfscope}%
\begin{pgfscope}%
\pgfsetbuttcap%
\pgfsetroundjoin%
\definecolor{currentfill}{rgb}{0.000000,0.000000,0.000000}%
\pgfsetfillcolor{currentfill}%
\pgfsetlinewidth{0.803000pt}%
\definecolor{currentstroke}{rgb}{0.000000,0.000000,0.000000}%
\pgfsetstrokecolor{currentstroke}%
\pgfsetdash{}{0pt}%
\pgfsys@defobject{currentmarker}{\pgfqpoint{0.000000in}{-0.048611in}}{\pgfqpoint{0.000000in}{0.000000in}}{%
\pgfpathmoveto{\pgfqpoint{0.000000in}{0.000000in}}%
\pgfpathlineto{\pgfqpoint{0.000000in}{-0.048611in}}%
\pgfusepath{stroke,fill}%
}%
\begin{pgfscope}%
\pgfsys@transformshift{1.508853in}{0.521603in}%
\pgfsys@useobject{currentmarker}{}%
\end{pgfscope}%
\end{pgfscope}%
\begin{pgfscope}%
\pgftext[x=1.508853in,y=0.424381in,,top]{\rmfamily\fontsize{10.000000}{12.000000}\selectfont \(\displaystyle 10\)}%
\end{pgfscope}%
\begin{pgfscope}%
\pgfsetbuttcap%
\pgfsetroundjoin%
\definecolor{currentfill}{rgb}{0.000000,0.000000,0.000000}%
\pgfsetfillcolor{currentfill}%
\pgfsetlinewidth{0.803000pt}%
\definecolor{currentstroke}{rgb}{0.000000,0.000000,0.000000}%
\pgfsetstrokecolor{currentstroke}%
\pgfsetdash{}{0pt}%
\pgfsys@defobject{currentmarker}{\pgfqpoint{0.000000in}{-0.048611in}}{\pgfqpoint{0.000000in}{0.000000in}}{%
\pgfpathmoveto{\pgfqpoint{0.000000in}{0.000000in}}%
\pgfpathlineto{\pgfqpoint{0.000000in}{-0.048611in}}%
\pgfusepath{stroke,fill}%
}%
\begin{pgfscope}%
\pgfsys@transformshift{2.325151in}{0.521603in}%
\pgfsys@useobject{currentmarker}{}%
\end{pgfscope}%
\end{pgfscope}%
\begin{pgfscope}%
\pgftext[x=2.325151in,y=0.424381in,,top]{\rmfamily\fontsize{10.000000}{12.000000}\selectfont \(\displaystyle 20\)}%
\end{pgfscope}%
\begin{pgfscope}%
\pgfsetbuttcap%
\pgfsetroundjoin%
\definecolor{currentfill}{rgb}{0.000000,0.000000,0.000000}%
\pgfsetfillcolor{currentfill}%
\pgfsetlinewidth{0.803000pt}%
\definecolor{currentstroke}{rgb}{0.000000,0.000000,0.000000}%
\pgfsetstrokecolor{currentstroke}%
\pgfsetdash{}{0pt}%
\pgfsys@defobject{currentmarker}{\pgfqpoint{0.000000in}{-0.048611in}}{\pgfqpoint{0.000000in}{0.000000in}}{%
\pgfpathmoveto{\pgfqpoint{0.000000in}{0.000000in}}%
\pgfpathlineto{\pgfqpoint{0.000000in}{-0.048611in}}%
\pgfusepath{stroke,fill}%
}%
\begin{pgfscope}%
\pgfsys@transformshift{3.141449in}{0.521603in}%
\pgfsys@useobject{currentmarker}{}%
\end{pgfscope}%
\end{pgfscope}%
\begin{pgfscope}%
\pgftext[x=3.141449in,y=0.424381in,,top]{\rmfamily\fontsize{10.000000}{12.000000}\selectfont \(\displaystyle 30\)}%
\end{pgfscope}%
\begin{pgfscope}%
\pgfsetbuttcap%
\pgfsetroundjoin%
\definecolor{currentfill}{rgb}{0.000000,0.000000,0.000000}%
\pgfsetfillcolor{currentfill}%
\pgfsetlinewidth{0.803000pt}%
\definecolor{currentstroke}{rgb}{0.000000,0.000000,0.000000}%
\pgfsetstrokecolor{currentstroke}%
\pgfsetdash{}{0pt}%
\pgfsys@defobject{currentmarker}{\pgfqpoint{0.000000in}{-0.048611in}}{\pgfqpoint{0.000000in}{0.000000in}}{%
\pgfpathmoveto{\pgfqpoint{0.000000in}{0.000000in}}%
\pgfpathlineto{\pgfqpoint{0.000000in}{-0.048611in}}%
\pgfusepath{stroke,fill}%
}%
\begin{pgfscope}%
\pgfsys@transformshift{3.957747in}{0.521603in}%
\pgfsys@useobject{currentmarker}{}%
\end{pgfscope}%
\end{pgfscope}%
\begin{pgfscope}%
\pgftext[x=3.957747in,y=0.424381in,,top]{\rmfamily\fontsize{10.000000}{12.000000}\selectfont \(\displaystyle 40\)}%
\end{pgfscope}%
\begin{pgfscope}%
\pgftext[x=2.386080in,y=0.234413in,,top]{\rmfamily\fontsize{10.000000}{12.000000}\selectfont \(\displaystyle \bar{t}\)}%
\end{pgfscope}%
\begin{pgfscope}%
\pgfsetbuttcap%
\pgfsetroundjoin%
\definecolor{currentfill}{rgb}{0.000000,0.000000,0.000000}%
\pgfsetfillcolor{currentfill}%
\pgfsetlinewidth{0.803000pt}%
\definecolor{currentstroke}{rgb}{0.000000,0.000000,0.000000}%
\pgfsetstrokecolor{currentstroke}%
\pgfsetdash{}{0pt}%
\pgfsys@defobject{currentmarker}{\pgfqpoint{-0.048611in}{0.000000in}}{\pgfqpoint{0.000000in}{0.000000in}}{%
\pgfpathmoveto{\pgfqpoint{0.000000in}{0.000000in}}%
\pgfpathlineto{\pgfqpoint{-0.048611in}{0.000000in}}%
\pgfusepath{stroke,fill}%
}%
\begin{pgfscope}%
\pgfsys@transformshift{0.526080in}{0.672547in}%
\pgfsys@useobject{currentmarker}{}%
\end{pgfscope}%
\end{pgfscope}%
\begin{pgfscope}%
\pgftext[x=0.359413in,y=0.619786in,left,base]{\rmfamily\fontsize{10.000000}{12.000000}\selectfont \(\displaystyle 0\)}%
\end{pgfscope}%
\begin{pgfscope}%
\pgfsetbuttcap%
\pgfsetroundjoin%
\definecolor{currentfill}{rgb}{0.000000,0.000000,0.000000}%
\pgfsetfillcolor{currentfill}%
\pgfsetlinewidth{0.803000pt}%
\definecolor{currentstroke}{rgb}{0.000000,0.000000,0.000000}%
\pgfsetstrokecolor{currentstroke}%
\pgfsetdash{}{0pt}%
\pgfsys@defobject{currentmarker}{\pgfqpoint{-0.048611in}{0.000000in}}{\pgfqpoint{0.000000in}{0.000000in}}{%
\pgfpathmoveto{\pgfqpoint{0.000000in}{0.000000in}}%
\pgfpathlineto{\pgfqpoint{-0.048611in}{0.000000in}}%
\pgfusepath{stroke,fill}%
}%
\begin{pgfscope}%
\pgfsys@transformshift{0.526080in}{1.142524in}%
\pgfsys@useobject{currentmarker}{}%
\end{pgfscope}%
\end{pgfscope}%
\begin{pgfscope}%
\pgftext[x=0.359413in,y=1.089763in,left,base]{\rmfamily\fontsize{10.000000}{12.000000}\selectfont \(\displaystyle 5\)}%
\end{pgfscope}%
\begin{pgfscope}%
\pgfsetbuttcap%
\pgfsetroundjoin%
\definecolor{currentfill}{rgb}{0.000000,0.000000,0.000000}%
\pgfsetfillcolor{currentfill}%
\pgfsetlinewidth{0.803000pt}%
\definecolor{currentstroke}{rgb}{0.000000,0.000000,0.000000}%
\pgfsetstrokecolor{currentstroke}%
\pgfsetdash{}{0pt}%
\pgfsys@defobject{currentmarker}{\pgfqpoint{-0.048611in}{0.000000in}}{\pgfqpoint{0.000000in}{0.000000in}}{%
\pgfpathmoveto{\pgfqpoint{0.000000in}{0.000000in}}%
\pgfpathlineto{\pgfqpoint{-0.048611in}{0.000000in}}%
\pgfusepath{stroke,fill}%
}%
\begin{pgfscope}%
\pgfsys@transformshift{0.526080in}{1.612501in}%
\pgfsys@useobject{currentmarker}{}%
\end{pgfscope}%
\end{pgfscope}%
\begin{pgfscope}%
\pgftext[x=0.289968in,y=1.559740in,left,base]{\rmfamily\fontsize{10.000000}{12.000000}\selectfont \(\displaystyle 10\)}%
\end{pgfscope}%
\begin{pgfscope}%
\pgfsetbuttcap%
\pgfsetroundjoin%
\definecolor{currentfill}{rgb}{0.000000,0.000000,0.000000}%
\pgfsetfillcolor{currentfill}%
\pgfsetlinewidth{0.803000pt}%
\definecolor{currentstroke}{rgb}{0.000000,0.000000,0.000000}%
\pgfsetstrokecolor{currentstroke}%
\pgfsetdash{}{0pt}%
\pgfsys@defobject{currentmarker}{\pgfqpoint{-0.048611in}{0.000000in}}{\pgfqpoint{0.000000in}{0.000000in}}{%
\pgfpathmoveto{\pgfqpoint{0.000000in}{0.000000in}}%
\pgfpathlineto{\pgfqpoint{-0.048611in}{0.000000in}}%
\pgfusepath{stroke,fill}%
}%
\begin{pgfscope}%
\pgfsys@transformshift{0.526080in}{2.082478in}%
\pgfsys@useobject{currentmarker}{}%
\end{pgfscope}%
\end{pgfscope}%
\begin{pgfscope}%
\pgftext[x=0.289968in,y=2.029717in,left,base]{\rmfamily\fontsize{10.000000}{12.000000}\selectfont \(\displaystyle 15\)}%
\end{pgfscope}%
\begin{pgfscope}%
\pgfsetbuttcap%
\pgfsetroundjoin%
\definecolor{currentfill}{rgb}{0.000000,0.000000,0.000000}%
\pgfsetfillcolor{currentfill}%
\pgfsetlinewidth{0.803000pt}%
\definecolor{currentstroke}{rgb}{0.000000,0.000000,0.000000}%
\pgfsetstrokecolor{currentstroke}%
\pgfsetdash{}{0pt}%
\pgfsys@defobject{currentmarker}{\pgfqpoint{-0.048611in}{0.000000in}}{\pgfqpoint{0.000000in}{0.000000in}}{%
\pgfpathmoveto{\pgfqpoint{0.000000in}{0.000000in}}%
\pgfpathlineto{\pgfqpoint{-0.048611in}{0.000000in}}%
\pgfusepath{stroke,fill}%
}%
\begin{pgfscope}%
\pgfsys@transformshift{0.526080in}{2.552455in}%
\pgfsys@useobject{currentmarker}{}%
\end{pgfscope}%
\end{pgfscope}%
\begin{pgfscope}%
\pgftext[x=0.289968in,y=2.499694in,left,base]{\rmfamily\fontsize{10.000000}{12.000000}\selectfont \(\displaystyle 20\)}%
\end{pgfscope}%
\begin{pgfscope}%
\pgfsetbuttcap%
\pgfsetroundjoin%
\definecolor{currentfill}{rgb}{0.000000,0.000000,0.000000}%
\pgfsetfillcolor{currentfill}%
\pgfsetlinewidth{0.803000pt}%
\definecolor{currentstroke}{rgb}{0.000000,0.000000,0.000000}%
\pgfsetstrokecolor{currentstroke}%
\pgfsetdash{}{0pt}%
\pgfsys@defobject{currentmarker}{\pgfqpoint{-0.048611in}{0.000000in}}{\pgfqpoint{0.000000in}{0.000000in}}{%
\pgfpathmoveto{\pgfqpoint{0.000000in}{0.000000in}}%
\pgfpathlineto{\pgfqpoint{-0.048611in}{0.000000in}}%
\pgfusepath{stroke,fill}%
}%
\begin{pgfscope}%
\pgfsys@transformshift{0.526080in}{3.022432in}%
\pgfsys@useobject{currentmarker}{}%
\end{pgfscope}%
\end{pgfscope}%
\begin{pgfscope}%
\pgftext[x=0.289968in,y=2.969671in,left,base]{\rmfamily\fontsize{10.000000}{12.000000}\selectfont \(\displaystyle 25\)}%
\end{pgfscope}%
\begin{pgfscope}%
\pgfsetbuttcap%
\pgfsetroundjoin%
\definecolor{currentfill}{rgb}{0.000000,0.000000,0.000000}%
\pgfsetfillcolor{currentfill}%
\pgfsetlinewidth{0.803000pt}%
\definecolor{currentstroke}{rgb}{0.000000,0.000000,0.000000}%
\pgfsetstrokecolor{currentstroke}%
\pgfsetdash{}{0pt}%
\pgfsys@defobject{currentmarker}{\pgfqpoint{-0.048611in}{0.000000in}}{\pgfqpoint{0.000000in}{0.000000in}}{%
\pgfpathmoveto{\pgfqpoint{0.000000in}{0.000000in}}%
\pgfpathlineto{\pgfqpoint{-0.048611in}{0.000000in}}%
\pgfusepath{stroke,fill}%
}%
\begin{pgfscope}%
\pgfsys@transformshift{0.526080in}{3.492409in}%
\pgfsys@useobject{currentmarker}{}%
\end{pgfscope}%
\end{pgfscope}%
\begin{pgfscope}%
\pgftext[x=0.289968in,y=3.439648in,left,base]{\rmfamily\fontsize{10.000000}{12.000000}\selectfont \(\displaystyle 30\)}%
\end{pgfscope}%
\begin{pgfscope}%
\pgftext[x=0.234413in,y=2.031603in,,bottom,rotate=90.000000]{\rmfamily\fontsize{10.000000}{12.000000}\selectfont \(\displaystyle \bar{y}\)}%
\end{pgfscope}%
\begin{pgfscope}%
\pgfpathrectangle{\pgfqpoint{0.526080in}{0.521603in}}{\pgfqpoint{3.720000in}{3.020000in}} %
\pgfusepath{clip}%
\pgfsetrectcap%
\pgfsetroundjoin%
\pgfsetlinewidth{1.505625pt}%
\definecolor{currentstroke}{rgb}{1.000000,0.231948,0.116773}%
\pgfsetstrokecolor{currentstroke}%
\pgfsetdash{}{0pt}%
\pgfpathmoveto{\pgfqpoint{0.695520in}{0.675385in}}%
\pgfpathlineto{\pgfqpoint{0.695789in}{0.675712in}}%
\pgfpathlineto{\pgfqpoint{0.696059in}{0.676112in}}%
\pgfpathlineto{\pgfqpoint{0.696328in}{0.676175in}}%
\pgfpathlineto{\pgfqpoint{0.696598in}{0.676365in}}%
\pgfpathlineto{\pgfqpoint{0.696867in}{0.676690in}}%
\pgfpathlineto{\pgfqpoint{0.697137in}{0.676803in}}%
\pgfpathlineto{\pgfqpoint{0.697406in}{0.676952in}}%
\pgfpathlineto{\pgfqpoint{0.697676in}{0.677075in}}%
\pgfpathlineto{\pgfqpoint{0.697945in}{0.677178in}}%
\pgfpathlineto{\pgfqpoint{0.698215in}{0.677213in}}%
\pgfpathlineto{\pgfqpoint{0.698484in}{0.677229in}}%
\pgfpathlineto{\pgfqpoint{0.698754in}{0.677246in}}%
\pgfpathlineto{\pgfqpoint{0.699023in}{0.677191in}}%
\pgfpathlineto{\pgfqpoint{0.699293in}{0.677165in}}%
\pgfpathlineto{\pgfqpoint{0.699562in}{0.677066in}}%
\pgfpathlineto{\pgfqpoint{0.699832in}{0.676969in}}%
\pgfpathlineto{\pgfqpoint{0.700101in}{0.676816in}}%
\pgfpathlineto{\pgfqpoint{0.700371in}{0.676665in}}%
\pgfpathlineto{\pgfqpoint{0.700640in}{0.676456in}}%
\pgfpathlineto{\pgfqpoint{0.700910in}{0.676236in}}%
\pgfpathlineto{\pgfqpoint{0.701179in}{0.675989in}}%
\pgfpathlineto{\pgfqpoint{0.701449in}{0.675695in}}%
\pgfpathlineto{\pgfqpoint{0.701718in}{0.675388in}}%
\pgfpathlineto{\pgfqpoint{0.701988in}{0.675022in}}%
\pgfpathlineto{\pgfqpoint{0.702257in}{0.674664in}}%
\pgfpathlineto{\pgfqpoint{0.702527in}{0.674228in}}%
\pgfpathlineto{\pgfqpoint{0.702796in}{0.673761in}}%
\pgfpathlineto{\pgfqpoint{0.703066in}{0.673205in}}%
\pgfpathlineto{\pgfqpoint{0.703335in}{0.672645in}}%
\pgfpathlineto{\pgfqpoint{0.703605in}{0.671936in}}%
\pgfpathlineto{\pgfqpoint{0.703874in}{0.671400in}}%
\pgfpathlineto{\pgfqpoint{0.704143in}{0.670970in}}%
\pgfpathlineto{\pgfqpoint{0.704413in}{0.671012in}}%
\pgfpathlineto{\pgfqpoint{0.704682in}{0.671393in}}%
\pgfpathlineto{\pgfqpoint{0.704952in}{0.671927in}}%
\pgfpathlineto{\pgfqpoint{0.705221in}{0.672443in}}%
\pgfpathlineto{\pgfqpoint{0.705491in}{0.672833in}}%
\pgfpathlineto{\pgfqpoint{0.705760in}{0.673137in}}%
\pgfpathlineto{\pgfqpoint{0.706030in}{0.673377in}}%
\pgfpathlineto{\pgfqpoint{0.706299in}{0.673520in}}%
\pgfpathlineto{\pgfqpoint{0.706569in}{0.673641in}}%
\pgfpathlineto{\pgfqpoint{0.706838in}{0.673638in}}%
\pgfpathlineto{\pgfqpoint{0.707108in}{0.673586in}}%
\pgfpathlineto{\pgfqpoint{0.707377in}{0.673444in}}%
\pgfpathlineto{\pgfqpoint{0.707647in}{0.673244in}}%
\pgfpathlineto{\pgfqpoint{0.707916in}{0.672953in}}%
\pgfpathlineto{\pgfqpoint{0.708186in}{0.672599in}}%
\pgfpathlineto{\pgfqpoint{0.708455in}{0.672064in}}%
\pgfpathlineto{\pgfqpoint{0.708725in}{0.671500in}}%
\pgfpathlineto{\pgfqpoint{0.708994in}{0.671177in}}%
\pgfpathlineto{\pgfqpoint{0.709264in}{0.671217in}}%
\pgfpathlineto{\pgfqpoint{0.709533in}{0.671592in}}%
\pgfpathlineto{\pgfqpoint{0.709803in}{0.672107in}}%
\pgfpathlineto{\pgfqpoint{0.710072in}{0.672597in}}%
\pgfpathlineto{\pgfqpoint{0.710342in}{0.672914in}}%
\pgfpathlineto{\pgfqpoint{0.710611in}{0.673133in}}%
\pgfpathlineto{\pgfqpoint{0.710881in}{0.673262in}}%
\pgfpathlineto{\pgfqpoint{0.711150in}{0.673320in}}%
\pgfpathlineto{\pgfqpoint{0.711420in}{0.673305in}}%
\pgfpathlineto{\pgfqpoint{0.711689in}{0.673207in}}%
\pgfpathlineto{\pgfqpoint{0.711959in}{0.673051in}}%
\pgfpathlineto{\pgfqpoint{0.712228in}{0.672823in}}%
\pgfpathlineto{\pgfqpoint{0.712498in}{0.672442in}}%
\pgfpathlineto{\pgfqpoint{0.712767in}{0.671944in}}%
\pgfpathlineto{\pgfqpoint{0.713037in}{0.671646in}}%
\pgfusepath{stroke}%
\end{pgfscope}%
\begin{pgfscope}%
\pgfpathrectangle{\pgfqpoint{0.526080in}{0.521603in}}{\pgfqpoint{3.720000in}{3.020000in}} %
\pgfusepath{clip}%
\pgfsetrectcap%
\pgfsetroundjoin%
\pgfsetlinewidth{1.505625pt}%
\definecolor{currentstroke}{rgb}{0.966667,0.743145,0.406737}%
\pgfsetstrokecolor{currentstroke}%
\pgfsetdash{}{0pt}%
\pgfpathmoveto{\pgfqpoint{0.695171in}{0.670014in}}%
\pgfpathlineto{\pgfqpoint{0.698440in}{0.676313in}}%
\pgfpathlineto{\pgfqpoint{0.703670in}{0.682365in}}%
\pgfpathlineto{\pgfqpoint{0.707593in}{0.684026in}}%
\pgfpathlineto{\pgfqpoint{0.714131in}{0.682132in}}%
\pgfpathlineto{\pgfqpoint{0.718054in}{0.678248in}}%
\pgfpathlineto{\pgfqpoint{0.724591in}{0.667864in}}%
\pgfpathlineto{\pgfqpoint{0.725245in}{0.668428in}}%
\pgfpathlineto{\pgfqpoint{0.735052in}{0.680579in}}%
\pgfpathlineto{\pgfqpoint{0.741590in}{0.679886in}}%
\pgfpathlineto{\pgfqpoint{0.747474in}{0.672565in}}%
\pgfpathlineto{\pgfqpoint{0.750743in}{0.668688in}}%
\pgfpathlineto{\pgfqpoint{0.752705in}{0.669644in}}%
\pgfpathlineto{\pgfqpoint{0.759242in}{0.675459in}}%
\pgfpathlineto{\pgfqpoint{0.761204in}{0.675175in}}%
\pgfpathlineto{\pgfqpoint{0.769049in}{0.669136in}}%
\pgfpathlineto{\pgfqpoint{0.774280in}{0.673821in}}%
\pgfpathlineto{\pgfqpoint{0.777549in}{0.673779in}}%
\pgfpathlineto{\pgfqpoint{0.782779in}{0.669093in}}%
\pgfpathlineto{\pgfqpoint{0.784740in}{0.669416in}}%
\pgfpathlineto{\pgfqpoint{0.786702in}{0.671170in}}%
\pgfpathlineto{\pgfqpoint{0.790625in}{0.673148in}}%
\pgfpathlineto{\pgfqpoint{0.792586in}{0.672583in}}%
\pgfpathlineto{\pgfqpoint{0.798470in}{0.668984in}}%
\pgfpathlineto{\pgfqpoint{0.805662in}{0.672817in}}%
\pgfpathlineto{\pgfqpoint{0.812853in}{0.669156in}}%
\pgfpathlineto{\pgfqpoint{0.818738in}{0.672799in}}%
\pgfpathlineto{\pgfqpoint{0.822007in}{0.670618in}}%
\pgfpathlineto{\pgfqpoint{0.824622in}{0.668621in}}%
\pgfpathlineto{\pgfqpoint{0.826583in}{0.669054in}}%
\pgfpathlineto{\pgfqpoint{0.831813in}{0.672616in}}%
\pgfpathlineto{\pgfqpoint{0.833121in}{0.672362in}}%
\pgfpathlineto{\pgfqpoint{0.839659in}{0.668575in}}%
\pgfpathlineto{\pgfqpoint{0.840313in}{0.668971in}}%
\pgfpathlineto{\pgfqpoint{0.840313in}{0.668971in}}%
\pgfusepath{stroke}%
\end{pgfscope}%
\begin{pgfscope}%
\pgfpathrectangle{\pgfqpoint{0.526080in}{0.521603in}}{\pgfqpoint{3.720000in}{3.020000in}} %
\pgfusepath{clip}%
\pgfsetrectcap%
\pgfsetroundjoin%
\pgfsetlinewidth{1.505625pt}%
\definecolor{currentstroke}{rgb}{1.000000,0.000000,0.000000}%
\pgfsetstrokecolor{currentstroke}%
\pgfsetdash{}{0pt}%
\pgfpathmoveto{\pgfqpoint{0.695217in}{0.677836in}}%
\pgfpathlineto{\pgfqpoint{0.698469in}{0.680075in}}%
\pgfpathlineto{\pgfqpoint{0.701722in}{0.678913in}}%
\pgfpathlineto{\pgfqpoint{0.704087in}{0.675637in}}%
\pgfpathlineto{\pgfqpoint{0.705270in}{0.672909in}}%
\pgfpathlineto{\pgfqpoint{0.705861in}{0.673675in}}%
\pgfpathlineto{\pgfqpoint{0.708227in}{0.676471in}}%
\pgfpathlineto{\pgfqpoint{0.710001in}{0.675278in}}%
\pgfpathlineto{\pgfqpoint{0.711479in}{0.672847in}}%
\pgfpathlineto{\pgfqpoint{0.712071in}{0.673682in}}%
\pgfpathlineto{\pgfqpoint{0.714140in}{0.675285in}}%
\pgfpathlineto{\pgfqpoint{0.715323in}{0.674298in}}%
\pgfpathlineto{\pgfqpoint{0.716506in}{0.672884in}}%
\pgfpathlineto{\pgfqpoint{0.716802in}{0.673193in}}%
\pgfpathlineto{\pgfqpoint{0.718576in}{0.674385in}}%
\pgfpathlineto{\pgfqpoint{0.720350in}{0.673007in}}%
\pgfpathlineto{\pgfqpoint{0.720941in}{0.673812in}}%
\pgfpathlineto{\pgfqpoint{0.722124in}{0.673602in}}%
\pgfpathlineto{\pgfqpoint{0.723307in}{0.673094in}}%
\pgfpathlineto{\pgfqpoint{0.723602in}{0.673408in}}%
\pgfpathlineto{\pgfqpoint{0.725081in}{0.673218in}}%
\pgfpathlineto{\pgfqpoint{0.726263in}{0.673412in}}%
\pgfpathlineto{\pgfqpoint{0.727742in}{0.673221in}}%
\pgfpathlineto{\pgfqpoint{0.730403in}{0.673154in}}%
\pgfpathlineto{\pgfqpoint{0.745187in}{0.673385in}}%
\pgfpathlineto{\pgfqpoint{0.751101in}{0.673275in}}%
\pgfpathlineto{\pgfqpoint{0.761154in}{0.673334in}}%
\pgfpathlineto{\pgfqpoint{0.761154in}{0.673334in}}%
\pgfusepath{stroke}%
\end{pgfscope}%
\begin{pgfscope}%
\pgfpathrectangle{\pgfqpoint{0.526080in}{0.521603in}}{\pgfqpoint{3.720000in}{3.020000in}} %
\pgfusepath{clip}%
\pgfsetrectcap%
\pgfsetroundjoin%
\pgfsetlinewidth{1.505625pt}%
\definecolor{currentstroke}{rgb}{0.966667,0.743145,0.406737}%
\pgfsetstrokecolor{currentstroke}%
\pgfsetdash{}{0pt}%
\pgfpathmoveto{\pgfqpoint{0.700666in}{0.679030in}}%
\pgfpathlineto{\pgfqpoint{0.705736in}{0.684020in}}%
\pgfpathlineto{\pgfqpoint{0.714861in}{0.687019in}}%
\pgfpathlineto{\pgfqpoint{0.722971in}{0.684314in}}%
\pgfpathlineto{\pgfqpoint{0.731082in}{0.675763in}}%
\pgfpathlineto{\pgfqpoint{0.735138in}{0.669745in}}%
\pgfpathlineto{\pgfqpoint{0.737166in}{0.670284in}}%
\pgfpathlineto{\pgfqpoint{0.743249in}{0.677364in}}%
\pgfpathlineto{\pgfqpoint{0.752373in}{0.678454in}}%
\pgfpathlineto{\pgfqpoint{0.758457in}{0.673667in}}%
\pgfpathlineto{\pgfqpoint{0.762512in}{0.670185in}}%
\pgfpathlineto{\pgfqpoint{0.764540in}{0.671499in}}%
\pgfpathlineto{\pgfqpoint{0.769609in}{0.675806in}}%
\pgfpathlineto{\pgfqpoint{0.773665in}{0.675281in}}%
\pgfpathlineto{\pgfqpoint{0.776706in}{0.672612in}}%
\pgfpathlineto{\pgfqpoint{0.780762in}{0.670544in}}%
\pgfpathlineto{\pgfqpoint{0.787859in}{0.675230in}}%
\pgfpathlineto{\pgfqpoint{0.791914in}{0.673396in}}%
\pgfpathlineto{\pgfqpoint{0.796983in}{0.671081in}}%
\pgfpathlineto{\pgfqpoint{0.805094in}{0.674062in}}%
\pgfpathlineto{\pgfqpoint{0.810164in}{0.670949in}}%
\pgfpathlineto{\pgfqpoint{0.813205in}{0.672689in}}%
\pgfpathlineto{\pgfqpoint{0.817261in}{0.674625in}}%
\pgfpathlineto{\pgfqpoint{0.820302in}{0.673247in}}%
\pgfpathlineto{\pgfqpoint{0.824358in}{0.671287in}}%
\pgfpathlineto{\pgfqpoint{0.832469in}{0.674197in}}%
\pgfpathlineto{\pgfqpoint{0.838552in}{0.671852in}}%
\pgfpathlineto{\pgfqpoint{0.844635in}{0.674780in}}%
\pgfpathlineto{\pgfqpoint{0.848690in}{0.672868in}}%
\pgfpathlineto{\pgfqpoint{0.851732in}{0.671896in}}%
\pgfpathlineto{\pgfqpoint{0.860857in}{0.673915in}}%
\pgfpathlineto{\pgfqpoint{0.865926in}{0.672260in}}%
\pgfpathlineto{\pgfqpoint{0.873023in}{0.674797in}}%
\pgfpathlineto{\pgfqpoint{0.881134in}{0.672899in}}%
\pgfpathlineto{\pgfqpoint{0.886203in}{0.674670in}}%
\pgfpathlineto{\pgfqpoint{0.896342in}{0.673557in}}%
\pgfpathlineto{\pgfqpoint{0.900397in}{0.674308in}}%
\pgfpathlineto{\pgfqpoint{0.908508in}{0.673062in}}%
\pgfpathlineto{\pgfqpoint{0.913578in}{0.674087in}}%
\pgfpathlineto{\pgfqpoint{0.922702in}{0.673366in}}%
\pgfpathlineto{\pgfqpoint{0.926758in}{0.673909in}}%
\pgfpathlineto{\pgfqpoint{0.929799in}{0.673246in}}%
\pgfpathlineto{\pgfqpoint{0.929799in}{0.673246in}}%
\pgfusepath{stroke}%
\end{pgfscope}%
\begin{pgfscope}%
\pgfpathrectangle{\pgfqpoint{0.526080in}{0.521603in}}{\pgfqpoint{3.720000in}{3.020000in}} %
\pgfusepath{clip}%
\pgfsetrectcap%
\pgfsetroundjoin%
\pgfsetlinewidth{1.505625pt}%
\definecolor{currentstroke}{rgb}{0.770588,0.911023,0.542053}%
\pgfsetstrokecolor{currentstroke}%
\pgfsetdash{}{0pt}%
\pgfpathmoveto{\pgfqpoint{0.709433in}{0.687417in}}%
\pgfpathlineto{\pgfqpoint{0.711121in}{0.689574in}}%
\pgfpathlineto{\pgfqpoint{0.712808in}{0.691100in}}%
\pgfpathlineto{\pgfqpoint{0.714496in}{0.692624in}}%
\pgfpathlineto{\pgfqpoint{0.716184in}{0.694636in}}%
\pgfpathlineto{\pgfqpoint{0.717872in}{0.695980in}}%
\pgfpathlineto{\pgfqpoint{0.719559in}{0.697287in}}%
\pgfpathlineto{\pgfqpoint{0.721247in}{0.699036in}}%
\pgfpathlineto{\pgfqpoint{0.722935in}{0.700123in}}%
\pgfpathlineto{\pgfqpoint{0.724623in}{0.701359in}}%
\pgfpathlineto{\pgfqpoint{0.726310in}{0.702663in}}%
\pgfpathlineto{\pgfqpoint{0.727998in}{0.703466in}}%
\pgfpathlineto{\pgfqpoint{0.729686in}{0.704694in}}%
\pgfpathlineto{\pgfqpoint{0.731374in}{0.705519in}}%
\pgfpathlineto{\pgfqpoint{0.733061in}{0.706325in}}%
\pgfpathlineto{\pgfqpoint{0.734749in}{0.707227in}}%
\pgfpathlineto{\pgfqpoint{0.736437in}{0.707684in}}%
\pgfpathlineto{\pgfqpoint{0.738125in}{0.708431in}}%
\pgfpathlineto{\pgfqpoint{0.739812in}{0.708877in}}%
\pgfpathlineto{\pgfqpoint{0.741500in}{0.709238in}}%
\pgfpathlineto{\pgfqpoint{0.743188in}{0.709836in}}%
\pgfpathlineto{\pgfqpoint{0.744876in}{0.709913in}}%
\pgfpathlineto{\pgfqpoint{0.746563in}{0.710216in}}%
\pgfpathlineto{\pgfqpoint{0.748251in}{0.710366in}}%
\pgfpathlineto{\pgfqpoint{0.749939in}{0.710292in}}%
\pgfpathlineto{\pgfqpoint{0.751626in}{0.710503in}}%
\pgfpathlineto{\pgfqpoint{0.753314in}{0.710265in}}%
\pgfpathlineto{\pgfqpoint{0.755002in}{0.710138in}}%
\pgfpathlineto{\pgfqpoint{0.756690in}{0.709929in}}%
\pgfpathlineto{\pgfqpoint{0.758377in}{0.709488in}}%
\pgfpathlineto{\pgfqpoint{0.760065in}{0.709298in}}%
\pgfpathlineto{\pgfqpoint{0.761753in}{0.708693in}}%
\pgfpathlineto{\pgfqpoint{0.763441in}{0.708132in}}%
\pgfpathlineto{\pgfqpoint{0.765128in}{0.707606in}}%
\pgfpathlineto{\pgfqpoint{0.766816in}{0.706747in}}%
\pgfpathlineto{\pgfqpoint{0.768504in}{0.706104in}}%
\pgfpathlineto{\pgfqpoint{0.770192in}{0.705165in}}%
\pgfpathlineto{\pgfqpoint{0.771879in}{0.704147in}}%
\pgfpathlineto{\pgfqpoint{0.773567in}{0.703243in}}%
\pgfpathlineto{\pgfqpoint{0.775255in}{0.701965in}}%
\pgfpathlineto{\pgfqpoint{0.776943in}{0.700914in}}%
\pgfpathlineto{\pgfqpoint{0.778630in}{0.699558in}}%
\pgfpathlineto{\pgfqpoint{0.780318in}{0.698103in}}%
\pgfpathlineto{\pgfqpoint{0.782006in}{0.696752in}}%
\pgfpathlineto{\pgfqpoint{0.783693in}{0.695085in}}%
\pgfpathlineto{\pgfqpoint{0.785381in}{0.693569in}}%
\pgfpathlineto{\pgfqpoint{0.787069in}{0.691801in}}%
\pgfpathlineto{\pgfqpoint{0.788757in}{0.689847in}}%
\pgfpathlineto{\pgfqpoint{0.790444in}{0.688020in}}%
\pgfpathlineto{\pgfqpoint{0.792132in}{0.685831in}}%
\pgfpathlineto{\pgfqpoint{0.793820in}{0.683740in}}%
\pgfpathlineto{\pgfqpoint{0.795508in}{0.681388in}}%
\pgfpathlineto{\pgfqpoint{0.797195in}{0.678665in}}%
\pgfpathlineto{\pgfqpoint{0.798883in}{0.675929in}}%
\pgfpathlineto{\pgfqpoint{0.800571in}{0.673031in}}%
\pgfpathlineto{\pgfqpoint{0.802259in}{0.671312in}}%
\pgfpathlineto{\pgfqpoint{0.803946in}{0.669546in}}%
\pgfpathlineto{\pgfqpoint{0.805634in}{0.670272in}}%
\pgfpathlineto{\pgfqpoint{0.807322in}{0.671620in}}%
\pgfpathlineto{\pgfqpoint{0.809010in}{0.673692in}}%
\pgfpathlineto{\pgfqpoint{0.810697in}{0.676187in}}%
\pgfpathlineto{\pgfqpoint{0.812385in}{0.678224in}}%
\pgfpathlineto{\pgfqpoint{0.814073in}{0.679717in}}%
\pgfpathlineto{\pgfqpoint{0.815761in}{0.681147in}}%
\pgfpathlineto{\pgfqpoint{0.817448in}{0.682038in}}%
\pgfpathlineto{\pgfqpoint{0.819136in}{0.682880in}}%
\pgfpathlineto{\pgfqpoint{0.820824in}{0.683785in}}%
\pgfpathlineto{\pgfqpoint{0.822511in}{0.684166in}}%
\pgfpathlineto{\pgfqpoint{0.824199in}{0.684599in}}%
\pgfpathlineto{\pgfqpoint{0.825887in}{0.684680in}}%
\pgfpathlineto{\pgfqpoint{0.827575in}{0.684517in}}%
\pgfpathlineto{\pgfqpoint{0.829262in}{0.684404in}}%
\pgfpathlineto{\pgfqpoint{0.830950in}{0.683872in}}%
\pgfpathlineto{\pgfqpoint{0.832638in}{0.683375in}}%
\pgfpathlineto{\pgfqpoint{0.834326in}{0.682664in}}%
\pgfpathlineto{\pgfqpoint{0.836013in}{0.681513in}}%
\pgfpathlineto{\pgfqpoint{0.837701in}{0.680304in}}%
\pgfpathlineto{\pgfqpoint{0.839389in}{0.678689in}}%
\pgfpathlineto{\pgfqpoint{0.841077in}{0.676751in}}%
\pgfpathlineto{\pgfqpoint{0.842764in}{0.674583in}}%
\pgfpathlineto{\pgfqpoint{0.844452in}{0.672640in}}%
\pgfpathlineto{\pgfqpoint{0.846140in}{0.671484in}}%
\pgfpathlineto{\pgfqpoint{0.847828in}{0.670600in}}%
\pgfusepath{stroke}%
\end{pgfscope}%
\begin{pgfscope}%
\pgfpathrectangle{\pgfqpoint{0.526080in}{0.521603in}}{\pgfqpoint{3.720000in}{3.020000in}} %
\pgfusepath{clip}%
\pgfsetrectcap%
\pgfsetroundjoin%
\pgfsetlinewidth{1.505625pt}%
\definecolor{currentstroke}{rgb}{0.590196,0.989980,0.655284}%
\pgfsetstrokecolor{currentstroke}%
\pgfsetdash{}{0pt}%
\pgfpathmoveto{\pgfqpoint{0.715002in}{0.688549in}}%
\pgfpathlineto{\pgfqpoint{0.719990in}{0.693281in}}%
\pgfpathlineto{\pgfqpoint{0.724978in}{0.698710in}}%
\pgfpathlineto{\pgfqpoint{0.729966in}{0.702751in}}%
\pgfpathlineto{\pgfqpoint{0.734955in}{0.706971in}}%
\pgfpathlineto{\pgfqpoint{0.759895in}{0.720717in}}%
\pgfpathlineto{\pgfqpoint{0.779848in}{0.724212in}}%
\pgfpathlineto{\pgfqpoint{0.787330in}{0.723857in}}%
\pgfpathlineto{\pgfqpoint{0.799800in}{0.721224in}}%
\pgfpathlineto{\pgfqpoint{0.822247in}{0.710248in}}%
\pgfpathlineto{\pgfqpoint{0.827235in}{0.706633in}}%
\pgfpathlineto{\pgfqpoint{0.842199in}{0.692714in}}%
\pgfpathlineto{\pgfqpoint{0.852175in}{0.679834in}}%
\pgfpathlineto{\pgfqpoint{0.859657in}{0.668527in}}%
\pgfpathlineto{\pgfqpoint{0.862152in}{0.665752in}}%
\pgfpathlineto{\pgfqpoint{0.867140in}{0.666773in}}%
\pgfpathlineto{\pgfqpoint{0.884598in}{0.688709in}}%
\pgfpathlineto{\pgfqpoint{0.897068in}{0.697648in}}%
\pgfpathlineto{\pgfqpoint{0.907045in}{0.701601in}}%
\pgfpathlineto{\pgfqpoint{0.914527in}{0.703011in}}%
\pgfpathlineto{\pgfqpoint{0.924503in}{0.702626in}}%
\pgfpathlineto{\pgfqpoint{0.931985in}{0.700899in}}%
\pgfpathlineto{\pgfqpoint{0.941961in}{0.696166in}}%
\pgfpathlineto{\pgfqpoint{0.949444in}{0.690939in}}%
\pgfpathlineto{\pgfqpoint{0.959420in}{0.680822in}}%
\pgfpathlineto{\pgfqpoint{0.966902in}{0.671127in}}%
\pgfpathlineto{\pgfqpoint{0.971890in}{0.667229in}}%
\pgfpathlineto{\pgfqpoint{0.976878in}{0.668553in}}%
\pgfpathlineto{\pgfqpoint{0.981866in}{0.673130in}}%
\pgfpathlineto{\pgfqpoint{0.986854in}{0.679104in}}%
\pgfpathlineto{\pgfqpoint{0.994337in}{0.685087in}}%
\pgfpathlineto{\pgfqpoint{0.999325in}{0.687313in}}%
\pgfpathlineto{\pgfqpoint{1.009301in}{0.689168in}}%
\pgfpathlineto{\pgfqpoint{1.019277in}{0.687731in}}%
\pgfpathlineto{\pgfqpoint{1.026759in}{0.684752in}}%
\pgfpathlineto{\pgfqpoint{1.041724in}{0.671314in}}%
\pgfpathlineto{\pgfqpoint{1.049206in}{0.668604in}}%
\pgfpathlineto{\pgfqpoint{1.054194in}{0.671225in}}%
\pgfpathlineto{\pgfqpoint{1.069158in}{0.681092in}}%
\pgfpathlineto{\pgfqpoint{1.074146in}{0.682909in}}%
\pgfpathlineto{\pgfqpoint{1.086617in}{0.679618in}}%
\pgfpathlineto{\pgfqpoint{1.094099in}{0.673731in}}%
\pgfpathlineto{\pgfqpoint{1.099087in}{0.670279in}}%
\pgfpathlineto{\pgfqpoint{1.101581in}{0.668768in}}%
\pgfpathlineto{\pgfqpoint{1.106569in}{0.669309in}}%
\pgfpathlineto{\pgfqpoint{1.114051in}{0.674443in}}%
\pgfpathlineto{\pgfqpoint{1.119039in}{0.679039in}}%
\pgfpathlineto{\pgfqpoint{1.129016in}{0.682741in}}%
\pgfpathlineto{\pgfqpoint{1.138992in}{0.681762in}}%
\pgfpathlineto{\pgfqpoint{1.146474in}{0.678799in}}%
\pgfpathlineto{\pgfqpoint{1.151462in}{0.674464in}}%
\pgfpathlineto{\pgfqpoint{1.161438in}{0.669136in}}%
\pgfpathlineto{\pgfqpoint{1.166427in}{0.670542in}}%
\pgfpathlineto{\pgfqpoint{1.183885in}{0.680085in}}%
\pgfpathlineto{\pgfqpoint{1.193861in}{0.679403in}}%
\pgfpathlineto{\pgfqpoint{1.211320in}{0.670036in}}%
\pgfpathlineto{\pgfqpoint{1.216308in}{0.670177in}}%
\pgfpathlineto{\pgfqpoint{1.238754in}{0.680295in}}%
\pgfpathlineto{\pgfqpoint{1.246236in}{0.679855in}}%
\pgfpathlineto{\pgfqpoint{1.251224in}{0.677655in}}%
\pgfpathlineto{\pgfqpoint{1.261201in}{0.672716in}}%
\pgfpathlineto{\pgfqpoint{1.266189in}{0.670818in}}%
\pgfpathlineto{\pgfqpoint{1.271177in}{0.671000in}}%
\pgfpathlineto{\pgfqpoint{1.273671in}{0.671925in}}%
\pgfpathlineto{\pgfqpoint{1.273671in}{0.671925in}}%
\pgfusepath{stroke}%
\end{pgfscope}%
\begin{pgfscope}%
\pgfpathrectangle{\pgfqpoint{0.526080in}{0.521603in}}{\pgfqpoint{3.720000in}{3.020000in}} %
\pgfusepath{clip}%
\pgfsetrectcap%
\pgfsetroundjoin%
\pgfsetlinewidth{1.505625pt}%
\definecolor{currentstroke}{rgb}{0.692157,0.954791,0.592758}%
\pgfsetstrokecolor{currentstroke}%
\pgfsetdash{}{0pt}%
\pgfpathmoveto{\pgfqpoint{0.716973in}{0.699824in}}%
\pgfpathlineto{\pgfqpoint{0.732235in}{0.715378in}}%
\pgfpathlineto{\pgfqpoint{0.753600in}{0.731772in}}%
\pgfpathlineto{\pgfqpoint{0.765809in}{0.738233in}}%
\pgfpathlineto{\pgfqpoint{0.774966in}{0.742091in}}%
\pgfpathlineto{\pgfqpoint{0.793279in}{0.746385in}}%
\pgfpathlineto{\pgfqpoint{0.811593in}{0.746637in}}%
\pgfpathlineto{\pgfqpoint{0.829906in}{0.742882in}}%
\pgfpathlineto{\pgfqpoint{0.848219in}{0.735043in}}%
\pgfpathlineto{\pgfqpoint{0.863481in}{0.725175in}}%
\pgfpathlineto{\pgfqpoint{0.875689in}{0.714932in}}%
\pgfpathlineto{\pgfqpoint{0.887898in}{0.702414in}}%
\pgfpathlineto{\pgfqpoint{0.897055in}{0.691084in}}%
\pgfpathlineto{\pgfqpoint{0.909264in}{0.673603in}}%
\pgfpathlineto{\pgfqpoint{0.912316in}{0.672543in}}%
\pgfpathlineto{\pgfqpoint{0.915368in}{0.674704in}}%
\pgfpathlineto{\pgfqpoint{0.930630in}{0.693995in}}%
\pgfpathlineto{\pgfqpoint{0.939786in}{0.702257in}}%
\pgfpathlineto{\pgfqpoint{0.955047in}{0.712310in}}%
\pgfpathlineto{\pgfqpoint{0.967256in}{0.717762in}}%
\pgfpathlineto{\pgfqpoint{0.979465in}{0.721061in}}%
\pgfpathlineto{\pgfqpoint{0.991674in}{0.722303in}}%
\pgfpathlineto{\pgfqpoint{1.003883in}{0.721588in}}%
\pgfpathlineto{\pgfqpoint{1.016092in}{0.718815in}}%
\pgfpathlineto{\pgfqpoint{1.028301in}{0.714087in}}%
\pgfpathlineto{\pgfqpoint{1.040510in}{0.707307in}}%
\pgfpathlineto{\pgfqpoint{1.052719in}{0.698323in}}%
\pgfpathlineto{\pgfqpoint{1.064928in}{0.686328in}}%
\pgfpathlineto{\pgfqpoint{1.074084in}{0.675189in}}%
\pgfpathlineto{\pgfqpoint{1.077137in}{0.673503in}}%
\pgfpathlineto{\pgfqpoint{1.080189in}{0.674905in}}%
\pgfpathlineto{\pgfqpoint{1.104607in}{0.696838in}}%
\pgfpathlineto{\pgfqpoint{1.113763in}{0.701555in}}%
\pgfpathlineto{\pgfqpoint{1.129025in}{0.705452in}}%
\pgfpathlineto{\pgfqpoint{1.138181in}{0.706064in}}%
\pgfpathlineto{\pgfqpoint{1.150390in}{0.704323in}}%
\pgfpathlineto{\pgfqpoint{1.162599in}{0.699719in}}%
\pgfpathlineto{\pgfqpoint{1.174808in}{0.691845in}}%
\pgfpathlineto{\pgfqpoint{1.193121in}{0.674720in}}%
\pgfpathlineto{\pgfqpoint{1.196174in}{0.675869in}}%
\pgfpathlineto{\pgfqpoint{1.214487in}{0.691912in}}%
\pgfpathlineto{\pgfqpoint{1.223644in}{0.695769in}}%
\pgfpathlineto{\pgfqpoint{1.232800in}{0.697581in}}%
\pgfpathlineto{\pgfqpoint{1.241957in}{0.697444in}}%
\pgfpathlineto{\pgfqpoint{1.251114in}{0.695393in}}%
\pgfpathlineto{\pgfqpoint{1.260270in}{0.691283in}}%
\pgfpathlineto{\pgfqpoint{1.269427in}{0.684504in}}%
\pgfpathlineto{\pgfqpoint{1.278584in}{0.676487in}}%
\pgfpathlineto{\pgfqpoint{1.281636in}{0.676075in}}%
\pgfpathlineto{\pgfqpoint{1.284688in}{0.677194in}}%
\pgfpathlineto{\pgfqpoint{1.296897in}{0.685819in}}%
\pgfpathlineto{\pgfqpoint{1.306054in}{0.688432in}}%
\pgfpathlineto{\pgfqpoint{1.312158in}{0.688789in}}%
\pgfpathlineto{\pgfqpoint{1.321315in}{0.687175in}}%
\pgfpathlineto{\pgfqpoint{1.330472in}{0.682254in}}%
\pgfpathlineto{\pgfqpoint{1.336576in}{0.677887in}}%
\pgfpathlineto{\pgfqpoint{1.339628in}{0.676797in}}%
\pgfpathlineto{\pgfqpoint{1.342681in}{0.677257in}}%
\pgfpathlineto{\pgfqpoint{1.360994in}{0.686617in}}%
\pgfpathlineto{\pgfqpoint{1.370151in}{0.687233in}}%
\pgfpathlineto{\pgfqpoint{1.379307in}{0.685045in}}%
\pgfpathlineto{\pgfqpoint{1.394569in}{0.677364in}}%
\pgfpathlineto{\pgfqpoint{1.400673in}{0.679038in}}%
\pgfpathlineto{\pgfqpoint{1.403725in}{0.680931in}}%
\pgfpathlineto{\pgfqpoint{1.403725in}{0.680931in}}%
\pgfusepath{stroke}%
\end{pgfscope}%
\begin{pgfscope}%
\pgfpathrectangle{\pgfqpoint{0.526080in}{0.521603in}}{\pgfqpoint{3.720000in}{3.020000in}} %
\pgfusepath{clip}%
\pgfsetrectcap%
\pgfsetroundjoin%
\pgfsetlinewidth{1.505625pt}%
\definecolor{currentstroke}{rgb}{0.268627,0.934680,0.823253}%
\pgfsetstrokecolor{currentstroke}%
\pgfsetdash{}{0pt}%
\pgfpathmoveto{\pgfqpoint{0.707028in}{0.667235in}}%
\pgfpathlineto{\pgfqpoint{0.711853in}{0.677681in}}%
\pgfpathlineto{\pgfqpoint{0.716677in}{0.684403in}}%
\pgfpathlineto{\pgfqpoint{0.726325in}{0.694158in}}%
\pgfpathlineto{\pgfqpoint{0.735974in}{0.709178in}}%
\pgfpathlineto{\pgfqpoint{0.745622in}{0.718084in}}%
\pgfpathlineto{\pgfqpoint{0.750446in}{0.724238in}}%
\pgfpathlineto{\pgfqpoint{0.774568in}{0.745267in}}%
\pgfpathlineto{\pgfqpoint{0.784216in}{0.755267in}}%
\pgfpathlineto{\pgfqpoint{0.793865in}{0.761375in}}%
\pgfpathlineto{\pgfqpoint{0.808337in}{0.773217in}}%
\pgfpathlineto{\pgfqpoint{0.817986in}{0.778038in}}%
\pgfpathlineto{\pgfqpoint{0.832459in}{0.788195in}}%
\pgfpathlineto{\pgfqpoint{0.837283in}{0.789814in}}%
\pgfpathlineto{\pgfqpoint{0.856580in}{0.800298in}}%
\pgfpathlineto{\pgfqpoint{0.861404in}{0.801583in}}%
\pgfpathlineto{\pgfqpoint{0.875877in}{0.809091in}}%
\pgfpathlineto{\pgfqpoint{0.885525in}{0.811321in}}%
\pgfpathlineto{\pgfqpoint{0.895174in}{0.815768in}}%
\pgfpathlineto{\pgfqpoint{0.943416in}{0.825482in}}%
\pgfpathlineto{\pgfqpoint{0.953065in}{0.825299in}}%
\pgfpathlineto{\pgfqpoint{0.967537in}{0.826670in}}%
\pgfpathlineto{\pgfqpoint{0.977186in}{0.826380in}}%
\pgfpathlineto{\pgfqpoint{0.986834in}{0.826885in}}%
\pgfpathlineto{\pgfqpoint{1.001307in}{0.824958in}}%
\pgfpathlineto{\pgfqpoint{1.010956in}{0.824271in}}%
\pgfpathlineto{\pgfqpoint{1.020604in}{0.822063in}}%
\pgfpathlineto{\pgfqpoint{1.030253in}{0.821054in}}%
\pgfpathlineto{\pgfqpoint{1.078495in}{0.805716in}}%
\pgfpathlineto{\pgfqpoint{1.092968in}{0.799235in}}%
\pgfpathlineto{\pgfqpoint{1.102616in}{0.794247in}}%
\pgfpathlineto{\pgfqpoint{1.136386in}{0.774427in}}%
\pgfpathlineto{\pgfqpoint{1.146034in}{0.767681in}}%
\pgfpathlineto{\pgfqpoint{1.179804in}{0.740186in}}%
\pgfpathlineto{\pgfqpoint{1.194277in}{0.726323in}}%
\pgfpathlineto{\pgfqpoint{1.213574in}{0.705652in}}%
\pgfpathlineto{\pgfqpoint{1.242519in}{0.670843in}}%
\pgfpathlineto{\pgfqpoint{1.252168in}{0.663192in}}%
\pgfpathlineto{\pgfqpoint{1.256992in}{0.661757in}}%
\pgfpathlineto{\pgfqpoint{1.266640in}{0.664882in}}%
\pgfpathlineto{\pgfqpoint{1.276289in}{0.673672in}}%
\pgfpathlineto{\pgfqpoint{1.281113in}{0.677953in}}%
\pgfpathlineto{\pgfqpoint{1.290762in}{0.689792in}}%
\pgfpathlineto{\pgfqpoint{1.300410in}{0.698817in}}%
\pgfpathlineto{\pgfqpoint{1.339004in}{0.725812in}}%
\pgfpathlineto{\pgfqpoint{1.363125in}{0.736575in}}%
\pgfpathlineto{\pgfqpoint{1.392071in}{0.744513in}}%
\pgfpathlineto{\pgfqpoint{1.421016in}{0.747401in}}%
\pgfpathlineto{\pgfqpoint{1.430665in}{0.747558in}}%
\pgfpathlineto{\pgfqpoint{1.464434in}{0.741917in}}%
\pgfpathlineto{\pgfqpoint{1.478907in}{0.737528in}}%
\pgfpathlineto{\pgfqpoint{1.503028in}{0.726753in}}%
\pgfpathlineto{\pgfqpoint{1.522325in}{0.715232in}}%
\pgfpathlineto{\pgfqpoint{1.541622in}{0.700214in}}%
\pgfpathlineto{\pgfqpoint{1.556095in}{0.686035in}}%
\pgfpathlineto{\pgfqpoint{1.570568in}{0.670264in}}%
\pgfpathlineto{\pgfqpoint{1.580216in}{0.663145in}}%
\pgfpathlineto{\pgfqpoint{1.585040in}{0.662963in}}%
\pgfpathlineto{\pgfqpoint{1.589865in}{0.664798in}}%
\pgfpathlineto{\pgfqpoint{1.599513in}{0.673300in}}%
\pgfpathlineto{\pgfqpoint{1.628458in}{0.700665in}}%
\pgfpathlineto{\pgfqpoint{1.642931in}{0.709609in}}%
\pgfpathlineto{\pgfqpoint{1.657404in}{0.716853in}}%
\pgfpathlineto{\pgfqpoint{1.676701in}{0.723891in}}%
\pgfpathlineto{\pgfqpoint{1.705646in}{0.728305in}}%
\pgfpathlineto{\pgfqpoint{1.720119in}{0.728722in}}%
\pgfpathlineto{\pgfqpoint{1.749065in}{0.723976in}}%
\pgfpathlineto{\pgfqpoint{1.768361in}{0.717538in}}%
\pgfpathlineto{\pgfqpoint{1.792483in}{0.704929in}}%
\pgfpathlineto{\pgfqpoint{1.806955in}{0.694803in}}%
\pgfpathlineto{\pgfqpoint{1.806955in}{0.694803in}}%
\pgfusepath{stroke}%
\end{pgfscope}%
\begin{pgfscope}%
\pgfpathrectangle{\pgfqpoint{0.526080in}{0.521603in}}{\pgfqpoint{3.720000in}{3.020000in}} %
\pgfusepath{clip}%
\pgfsetrectcap%
\pgfsetroundjoin%
\pgfsetlinewidth{1.505625pt}%
\definecolor{currentstroke}{rgb}{0.190196,0.883910,0.856638}%
\pgfsetstrokecolor{currentstroke}%
\pgfsetdash{}{0pt}%
\pgfpathmoveto{\pgfqpoint{0.744798in}{0.708285in}}%
\pgfpathlineto{\pgfqpoint{0.757859in}{0.724108in}}%
\pgfpathlineto{\pgfqpoint{0.783980in}{0.750243in}}%
\pgfpathlineto{\pgfqpoint{0.797041in}{0.763250in}}%
\pgfpathlineto{\pgfqpoint{0.836223in}{0.798588in}}%
\pgfpathlineto{\pgfqpoint{0.875405in}{0.829926in}}%
\pgfpathlineto{\pgfqpoint{0.908056in}{0.852923in}}%
\pgfpathlineto{\pgfqpoint{0.960299in}{0.884211in}}%
\pgfpathlineto{\pgfqpoint{0.992951in}{0.900192in}}%
\pgfpathlineto{\pgfqpoint{1.012542in}{0.909028in}}%
\pgfpathlineto{\pgfqpoint{1.058254in}{0.926004in}}%
\pgfpathlineto{\pgfqpoint{1.110497in}{0.940668in}}%
\pgfpathlineto{\pgfqpoint{1.175800in}{0.950883in}}%
\pgfpathlineto{\pgfqpoint{1.201921in}{0.953040in}}%
\pgfpathlineto{\pgfqpoint{1.234573in}{0.953763in}}%
\pgfpathlineto{\pgfqpoint{1.273755in}{0.952140in}}%
\pgfpathlineto{\pgfqpoint{1.325997in}{0.945325in}}%
\pgfpathlineto{\pgfqpoint{1.378240in}{0.933396in}}%
\pgfpathlineto{\pgfqpoint{1.410892in}{0.922993in}}%
\pgfpathlineto{\pgfqpoint{1.443543in}{0.910259in}}%
\pgfpathlineto{\pgfqpoint{1.469665in}{0.898947in}}%
\pgfpathlineto{\pgfqpoint{1.521907in}{0.871137in}}%
\pgfpathlineto{\pgfqpoint{1.548028in}{0.854907in}}%
\pgfpathlineto{\pgfqpoint{1.580680in}{0.832365in}}%
\pgfpathlineto{\pgfqpoint{1.619862in}{0.801398in}}%
\pgfpathlineto{\pgfqpoint{1.652514in}{0.772383in}}%
\pgfpathlineto{\pgfqpoint{1.685165in}{0.740136in}}%
\pgfpathlineto{\pgfqpoint{1.711287in}{0.711653in}}%
\pgfpathlineto{\pgfqpoint{1.730878in}{0.688597in}}%
\pgfpathlineto{\pgfqpoint{1.743938in}{0.671236in}}%
\pgfpathlineto{\pgfqpoint{1.756999in}{0.658961in}}%
\pgfpathlineto{\pgfqpoint{1.763529in}{0.658876in}}%
\pgfpathlineto{\pgfqpoint{1.770060in}{0.662803in}}%
\pgfpathlineto{\pgfqpoint{1.783120in}{0.675296in}}%
\pgfpathlineto{\pgfqpoint{1.789651in}{0.682779in}}%
\pgfpathlineto{\pgfqpoint{1.815772in}{0.704433in}}%
\pgfpathlineto{\pgfqpoint{1.828833in}{0.713466in}}%
\pgfpathlineto{\pgfqpoint{1.861484in}{0.734378in}}%
\pgfpathlineto{\pgfqpoint{1.900666in}{0.753385in}}%
\pgfpathlineto{\pgfqpoint{1.952909in}{0.770493in}}%
\pgfpathlineto{\pgfqpoint{1.998621in}{0.777497in}}%
\pgfpathlineto{\pgfqpoint{2.018212in}{0.778813in}}%
\pgfpathlineto{\pgfqpoint{2.044333in}{0.778277in}}%
\pgfpathlineto{\pgfqpoint{2.070455in}{0.775378in}}%
\pgfpathlineto{\pgfqpoint{2.103106in}{0.768883in}}%
\pgfpathlineto{\pgfqpoint{2.142288in}{0.755460in}}%
\pgfpathlineto{\pgfqpoint{2.174940in}{0.739864in}}%
\pgfpathlineto{\pgfqpoint{2.207592in}{0.719312in}}%
\pgfpathlineto{\pgfqpoint{2.214122in}{0.714661in}}%
\pgfpathlineto{\pgfqpoint{2.214122in}{0.714661in}}%
\pgfusepath{stroke}%
\end{pgfscope}%
\begin{pgfscope}%
\pgfpathrectangle{\pgfqpoint{0.526080in}{0.521603in}}{\pgfqpoint{3.720000in}{3.020000in}} %
\pgfusepath{clip}%
\pgfsetrectcap%
\pgfsetroundjoin%
\pgfsetlinewidth{1.505625pt}%
\definecolor{currentstroke}{rgb}{0.052941,0.645928,0.938988}%
\pgfsetstrokecolor{currentstroke}%
\pgfsetdash{}{0pt}%
\pgfpathmoveto{\pgfqpoint{0.765517in}{0.725749in}}%
\pgfpathlineto{\pgfqpoint{0.774637in}{0.731439in}}%
\pgfpathlineto{\pgfqpoint{0.783758in}{0.740022in}}%
\pgfpathlineto{\pgfqpoint{0.792878in}{0.756088in}}%
\pgfpathlineto{\pgfqpoint{0.801998in}{0.765223in}}%
\pgfpathlineto{\pgfqpoint{0.811118in}{0.768890in}}%
\pgfpathlineto{\pgfqpoint{0.829358in}{0.793886in}}%
\pgfpathlineto{\pgfqpoint{0.856719in}{0.817831in}}%
\pgfpathlineto{\pgfqpoint{0.865839in}{0.828125in}}%
\pgfpathlineto{\pgfqpoint{0.893200in}{0.850644in}}%
\pgfpathlineto{\pgfqpoint{0.902320in}{0.860279in}}%
\pgfpathlineto{\pgfqpoint{0.929681in}{0.882298in}}%
\pgfpathlineto{\pgfqpoint{0.938801in}{0.890317in}}%
\pgfpathlineto{\pgfqpoint{0.966161in}{0.910498in}}%
\pgfpathlineto{\pgfqpoint{0.975282in}{0.918399in}}%
\pgfpathlineto{\pgfqpoint{0.993522in}{0.930123in}}%
\pgfpathlineto{\pgfqpoint{1.011762in}{0.944530in}}%
\pgfpathlineto{\pgfqpoint{1.030003in}{0.955428in}}%
\pgfpathlineto{\pgfqpoint{1.048243in}{0.968944in}}%
\pgfpathlineto{\pgfqpoint{1.057363in}{0.973458in}}%
\pgfpathlineto{\pgfqpoint{1.093844in}{0.996167in}}%
\pgfpathlineto{\pgfqpoint{1.157685in}{1.032582in}}%
\pgfpathlineto{\pgfqpoint{1.175926in}{1.041931in}}%
\pgfpathlineto{\pgfqpoint{1.185046in}{1.047155in}}%
\pgfpathlineto{\pgfqpoint{1.203286in}{1.055123in}}%
\pgfpathlineto{\pgfqpoint{1.221527in}{1.064649in}}%
\pgfpathlineto{\pgfqpoint{1.239767in}{1.072101in}}%
\pgfpathlineto{\pgfqpoint{1.258007in}{1.081013in}}%
\pgfpathlineto{\pgfqpoint{1.276248in}{1.088185in}}%
\pgfpathlineto{\pgfqpoint{1.294488in}{1.096227in}}%
\pgfpathlineto{\pgfqpoint{1.312729in}{1.102985in}}%
\pgfpathlineto{\pgfqpoint{1.330969in}{1.110150in}}%
\pgfpathlineto{\pgfqpoint{1.349209in}{1.116795in}}%
\pgfpathlineto{\pgfqpoint{1.367450in}{1.123244in}}%
\pgfpathlineto{\pgfqpoint{1.385690in}{1.129485in}}%
\pgfpathlineto{\pgfqpoint{1.403931in}{1.135139in}}%
\pgfpathlineto{\pgfqpoint{1.504253in}{1.163458in}}%
\pgfpathlineto{\pgfqpoint{1.549854in}{1.173445in}}%
\pgfpathlineto{\pgfqpoint{1.577214in}{1.179267in}}%
\pgfpathlineto{\pgfqpoint{1.704897in}{1.198651in}}%
\pgfpathlineto{\pgfqpoint{1.759618in}{1.203483in}}%
\pgfpathlineto{\pgfqpoint{1.805219in}{1.206492in}}%
\pgfpathlineto{\pgfqpoint{1.850820in}{1.207907in}}%
\pgfpathlineto{\pgfqpoint{1.960262in}{1.206069in}}%
\pgfpathlineto{\pgfqpoint{2.042344in}{1.199675in}}%
\pgfpathlineto{\pgfqpoint{2.087945in}{1.194436in}}%
\pgfpathlineto{\pgfqpoint{2.151786in}{1.184417in}}%
\pgfpathlineto{\pgfqpoint{2.197387in}{1.175560in}}%
\pgfpathlineto{\pgfqpoint{2.270349in}{1.157915in}}%
\pgfpathlineto{\pgfqpoint{2.334190in}{1.138968in}}%
\pgfpathlineto{\pgfqpoint{2.398031in}{1.116352in}}%
\pgfpathlineto{\pgfqpoint{2.452753in}{1.093932in}}%
\pgfpathlineto{\pgfqpoint{2.498354in}{1.073052in}}%
\pgfpathlineto{\pgfqpoint{2.543954in}{1.050164in}}%
\pgfpathlineto{\pgfqpoint{2.598676in}{1.019913in}}%
\pgfpathlineto{\pgfqpoint{2.644277in}{0.992151in}}%
\pgfpathlineto{\pgfqpoint{2.689878in}{0.962114in}}%
\pgfpathlineto{\pgfqpoint{2.735479in}{0.929467in}}%
\pgfpathlineto{\pgfqpoint{2.781079in}{0.894083in}}%
\pgfpathlineto{\pgfqpoint{2.817560in}{0.863512in}}%
\pgfpathlineto{\pgfqpoint{2.817560in}{0.863512in}}%
\pgfusepath{stroke}%
\end{pgfscope}%
\begin{pgfscope}%
\pgfpathrectangle{\pgfqpoint{0.526080in}{0.521603in}}{\pgfqpoint{3.720000in}{3.020000in}} %
\pgfusepath{clip}%
\pgfsetrectcap%
\pgfsetroundjoin%
\pgfsetlinewidth{1.505625pt}%
\definecolor{currentstroke}{rgb}{0.162745,0.505325,0.965124}%
\pgfsetstrokecolor{currentstroke}%
\pgfsetdash{}{0pt}%
\pgfpathmoveto{\pgfqpoint{0.800399in}{0.761511in}}%
\pgfpathlineto{\pgfqpoint{0.810203in}{0.764987in}}%
\pgfpathlineto{\pgfqpoint{0.820007in}{0.780945in}}%
\pgfpathlineto{\pgfqpoint{0.829811in}{0.794287in}}%
\pgfpathlineto{\pgfqpoint{0.839615in}{0.802004in}}%
\pgfpathlineto{\pgfqpoint{0.849419in}{0.811791in}}%
\pgfpathlineto{\pgfqpoint{0.859223in}{0.823782in}}%
\pgfpathlineto{\pgfqpoint{0.869027in}{0.837587in}}%
\pgfpathlineto{\pgfqpoint{0.898439in}{0.865855in}}%
\pgfpathlineto{\pgfqpoint{0.908243in}{0.878546in}}%
\pgfpathlineto{\pgfqpoint{0.927851in}{0.896257in}}%
\pgfpathlineto{\pgfqpoint{0.937655in}{0.904665in}}%
\pgfpathlineto{\pgfqpoint{0.967067in}{0.935846in}}%
\pgfpathlineto{\pgfqpoint{0.976871in}{0.942228in}}%
\pgfpathlineto{\pgfqpoint{1.006283in}{0.971910in}}%
\pgfpathlineto{\pgfqpoint{1.025891in}{0.986199in}}%
\pgfpathlineto{\pgfqpoint{1.055303in}{1.013860in}}%
\pgfpathlineto{\pgfqpoint{1.065107in}{1.019593in}}%
\pgfpathlineto{\pgfqpoint{1.094519in}{1.046329in}}%
\pgfpathlineto{\pgfqpoint{1.114127in}{1.059056in}}%
\pgfpathlineto{\pgfqpoint{1.143539in}{1.083326in}}%
\pgfpathlineto{\pgfqpoint{1.153343in}{1.088905in}}%
\pgfpathlineto{\pgfqpoint{1.182755in}{1.112834in}}%
\pgfpathlineto{\pgfqpoint{1.202363in}{1.124250in}}%
\pgfpathlineto{\pgfqpoint{1.231775in}{1.146346in}}%
\pgfpathlineto{\pgfqpoint{1.241579in}{1.151134in}}%
\pgfpathlineto{\pgfqpoint{1.270991in}{1.173120in}}%
\pgfpathlineto{\pgfqpoint{1.290599in}{1.183479in}}%
\pgfpathlineto{\pgfqpoint{1.320011in}{1.203294in}}%
\pgfpathlineto{\pgfqpoint{1.329815in}{1.208108in}}%
\pgfpathlineto{\pgfqpoint{1.359227in}{1.228050in}}%
\pgfpathlineto{\pgfqpoint{1.378835in}{1.237550in}}%
\pgfpathlineto{\pgfqpoint{1.408247in}{1.255515in}}%
\pgfpathlineto{\pgfqpoint{1.418051in}{1.260192in}}%
\pgfpathlineto{\pgfqpoint{1.457267in}{1.282755in}}%
\pgfpathlineto{\pgfqpoint{1.467071in}{1.287233in}}%
\pgfpathlineto{\pgfqpoint{1.486679in}{1.299804in}}%
\pgfpathlineto{\pgfqpoint{1.506287in}{1.308180in}}%
\pgfpathlineto{\pgfqpoint{1.535699in}{1.325055in}}%
\pgfpathlineto{\pgfqpoint{1.555307in}{1.333158in}}%
\pgfpathlineto{\pgfqpoint{1.574915in}{1.344759in}}%
\pgfpathlineto{\pgfqpoint{1.594523in}{1.352519in}}%
\pgfpathlineto{\pgfqpoint{1.623935in}{1.368077in}}%
\pgfpathlineto{\pgfqpoint{1.643543in}{1.375569in}}%
\pgfpathlineto{\pgfqpoint{1.663151in}{1.386214in}}%
\pgfpathlineto{\pgfqpoint{1.682759in}{1.393519in}}%
\pgfpathlineto{\pgfqpoint{1.712170in}{1.407908in}}%
\pgfpathlineto{\pgfqpoint{1.731778in}{1.415007in}}%
\pgfpathlineto{\pgfqpoint{1.751386in}{1.424629in}}%
\pgfpathlineto{\pgfqpoint{1.770994in}{1.431604in}}%
\pgfpathlineto{\pgfqpoint{1.800406in}{1.444774in}}%
\pgfpathlineto{\pgfqpoint{1.820014in}{1.451641in}}%
\pgfpathlineto{\pgfqpoint{1.839622in}{1.460356in}}%
\pgfpathlineto{\pgfqpoint{1.859230in}{1.466939in}}%
\pgfpathlineto{\pgfqpoint{1.888642in}{1.479018in}}%
\pgfpathlineto{\pgfqpoint{1.908250in}{1.485458in}}%
\pgfpathlineto{\pgfqpoint{1.927858in}{1.493649in}}%
\pgfpathlineto{\pgfqpoint{1.957270in}{1.503654in}}%
\pgfpathlineto{\pgfqpoint{1.976878in}{1.511033in}}%
\pgfpathlineto{\pgfqpoint{1.996486in}{1.517063in}}%
\pgfpathlineto{\pgfqpoint{2.016094in}{1.524758in}}%
\pgfpathlineto{\pgfqpoint{2.045506in}{1.534222in}}%
\pgfpathlineto{\pgfqpoint{2.065114in}{1.540955in}}%
\pgfpathlineto{\pgfqpoint{2.084722in}{1.546758in}}%
\pgfpathlineto{\pgfqpoint{2.104330in}{1.553966in}}%
\pgfpathlineto{\pgfqpoint{2.123938in}{1.559327in}}%
\pgfpathlineto{\pgfqpoint{2.153350in}{1.568890in}}%
\pgfpathlineto{\pgfqpoint{2.172958in}{1.574550in}}%
\pgfpathlineto{\pgfqpoint{2.192566in}{1.581164in}}%
\pgfpathlineto{\pgfqpoint{2.221978in}{1.589536in}}%
\pgfpathlineto{\pgfqpoint{2.241586in}{1.595048in}}%
\pgfpathlineto{\pgfqpoint{2.261194in}{1.600495in}}%
\pgfpathlineto{\pgfqpoint{2.280802in}{1.606501in}}%
\pgfpathlineto{\pgfqpoint{2.300410in}{1.611227in}}%
\pgfpathlineto{\pgfqpoint{2.329822in}{1.619455in}}%
\pgfpathlineto{\pgfqpoint{2.349430in}{1.624726in}}%
\pgfpathlineto{\pgfqpoint{2.369038in}{1.630255in}}%
\pgfpathlineto{\pgfqpoint{2.388646in}{1.634755in}}%
\pgfpathlineto{\pgfqpoint{2.418058in}{1.642236in}}%
\pgfpathlineto{\pgfqpoint{2.437666in}{1.647482in}}%
\pgfpathlineto{\pgfqpoint{2.457274in}{1.652403in}}%
\pgfpathlineto{\pgfqpoint{2.486686in}{1.659319in}}%
\pgfpathlineto{\pgfqpoint{2.545510in}{1.673088in}}%
\pgfpathlineto{\pgfqpoint{2.574922in}{1.679546in}}%
\pgfpathlineto{\pgfqpoint{2.672961in}{1.700252in}}%
\pgfpathlineto{\pgfqpoint{2.692569in}{1.704043in}}%
\pgfpathlineto{\pgfqpoint{2.721981in}{1.710181in}}%
\pgfpathlineto{\pgfqpoint{2.751393in}{1.715924in}}%
\pgfpathlineto{\pgfqpoint{2.986689in}{1.756267in}}%
\pgfpathlineto{\pgfqpoint{2.986689in}{1.756267in}}%
\pgfusepath{stroke}%
\end{pgfscope}%
\begin{pgfscope}%
\pgfpathrectangle{\pgfqpoint{0.526080in}{0.521603in}}{\pgfqpoint{3.720000in}{3.020000in}} %
\pgfusepath{clip}%
\pgfsetrectcap%
\pgfsetroundjoin%
\pgfsetlinewidth{1.505625pt}%
\definecolor{currentstroke}{rgb}{0.205882,0.895163,0.850217}%
\pgfsetstrokecolor{currentstroke}%
\pgfsetdash{}{0pt}%
\pgfpathmoveto{\pgfqpoint{0.748497in}{0.717772in}}%
\pgfpathlineto{\pgfqpoint{0.772472in}{0.742180in}}%
\pgfpathlineto{\pgfqpoint{0.780463in}{0.752145in}}%
\pgfpathlineto{\pgfqpoint{0.812430in}{0.783622in}}%
\pgfpathlineto{\pgfqpoint{0.844396in}{0.814270in}}%
\pgfpathlineto{\pgfqpoint{0.900338in}{0.861806in}}%
\pgfpathlineto{\pgfqpoint{0.924312in}{0.879682in}}%
\pgfpathlineto{\pgfqpoint{0.948287in}{0.897131in}}%
\pgfpathlineto{\pgfqpoint{0.996237in}{0.928027in}}%
\pgfpathlineto{\pgfqpoint{1.020212in}{0.942210in}}%
\pgfpathlineto{\pgfqpoint{1.092136in}{0.977893in}}%
\pgfpathlineto{\pgfqpoint{1.124103in}{0.991031in}}%
\pgfpathlineto{\pgfqpoint{1.172053in}{1.007547in}}%
\pgfpathlineto{\pgfqpoint{1.212011in}{1.018585in}}%
\pgfpathlineto{\pgfqpoint{1.251969in}{1.027127in}}%
\pgfpathlineto{\pgfqpoint{1.283935in}{1.032191in}}%
\pgfpathlineto{\pgfqpoint{1.307910in}{1.035134in}}%
\pgfpathlineto{\pgfqpoint{1.355860in}{1.038303in}}%
\pgfpathlineto{\pgfqpoint{1.395818in}{1.038353in}}%
\pgfpathlineto{\pgfqpoint{1.451759in}{1.034412in}}%
\pgfpathlineto{\pgfqpoint{1.483726in}{1.030096in}}%
\pgfpathlineto{\pgfqpoint{1.523684in}{1.022490in}}%
\pgfpathlineto{\pgfqpoint{1.563642in}{1.012400in}}%
\pgfpathlineto{\pgfqpoint{1.595608in}{1.002565in}}%
\pgfpathlineto{\pgfqpoint{1.635566in}{0.987971in}}%
\pgfpathlineto{\pgfqpoint{1.675524in}{0.970793in}}%
\pgfpathlineto{\pgfqpoint{1.715483in}{0.950854in}}%
\pgfpathlineto{\pgfqpoint{1.747449in}{0.932888in}}%
\pgfpathlineto{\pgfqpoint{1.787407in}{0.907952in}}%
\pgfpathlineto{\pgfqpoint{1.835357in}{0.873977in}}%
\pgfpathlineto{\pgfqpoint{1.875315in}{0.842408in}}%
\pgfpathlineto{\pgfqpoint{1.915273in}{0.807654in}}%
\pgfpathlineto{\pgfqpoint{1.955231in}{0.769834in}}%
\pgfpathlineto{\pgfqpoint{1.995189in}{0.728908in}}%
\pgfpathlineto{\pgfqpoint{2.027156in}{0.693205in}}%
\pgfpathlineto{\pgfqpoint{2.043139in}{0.672822in}}%
\pgfpathlineto{\pgfqpoint{2.051130in}{0.666053in}}%
\pgfpathlineto{\pgfqpoint{2.059122in}{0.662302in}}%
\pgfpathlineto{\pgfqpoint{2.067114in}{0.665870in}}%
\pgfpathlineto{\pgfqpoint{2.083097in}{0.680636in}}%
\pgfpathlineto{\pgfqpoint{2.099080in}{0.696301in}}%
\pgfpathlineto{\pgfqpoint{2.163013in}{0.742319in}}%
\pgfpathlineto{\pgfqpoint{2.234938in}{0.784960in}}%
\pgfpathlineto{\pgfqpoint{2.282887in}{0.808107in}}%
\pgfpathlineto{\pgfqpoint{2.330837in}{0.826743in}}%
\pgfpathlineto{\pgfqpoint{2.370795in}{0.838878in}}%
\pgfpathlineto{\pgfqpoint{2.402762in}{0.846395in}}%
\pgfpathlineto{\pgfqpoint{2.442720in}{0.852612in}}%
\pgfpathlineto{\pgfqpoint{2.474686in}{0.855459in}}%
\pgfpathlineto{\pgfqpoint{2.506653in}{0.856081in}}%
\pgfpathlineto{\pgfqpoint{2.538619in}{0.854610in}}%
\pgfpathlineto{\pgfqpoint{2.546611in}{0.853769in}}%
\pgfpathlineto{\pgfqpoint{2.546611in}{0.853769in}}%
\pgfusepath{stroke}%
\end{pgfscope}%
\begin{pgfscope}%
\pgfpathrectangle{\pgfqpoint{0.526080in}{0.521603in}}{\pgfqpoint{3.720000in}{3.020000in}} %
\pgfusepath{clip}%
\pgfsetrectcap%
\pgfsetroundjoin%
\pgfsetlinewidth{1.505625pt}%
\definecolor{currentstroke}{rgb}{0.076471,0.617278,0.945184}%
\pgfsetstrokecolor{currentstroke}%
\pgfsetdash{}{0pt}%
\pgfpathmoveto{\pgfqpoint{0.782256in}{0.742192in}}%
\pgfpathlineto{\pgfqpoint{0.846327in}{0.812079in}}%
\pgfpathlineto{\pgfqpoint{0.859142in}{0.826979in}}%
\pgfpathlineto{\pgfqpoint{0.871956in}{0.838480in}}%
\pgfpathlineto{\pgfqpoint{0.897585in}{0.865149in}}%
\pgfpathlineto{\pgfqpoint{0.987285in}{0.951969in}}%
\pgfpathlineto{\pgfqpoint{1.000099in}{0.962915in}}%
\pgfpathlineto{\pgfqpoint{1.012914in}{0.975372in}}%
\pgfpathlineto{\pgfqpoint{1.038542in}{0.997564in}}%
\pgfpathlineto{\pgfqpoint{1.076985in}{1.030963in}}%
\pgfpathlineto{\pgfqpoint{1.102614in}{1.051987in}}%
\pgfpathlineto{\pgfqpoint{1.166686in}{1.103037in}}%
\pgfpathlineto{\pgfqpoint{1.217943in}{1.141306in}}%
\pgfpathlineto{\pgfqpoint{1.243572in}{1.160136in}}%
\pgfpathlineto{\pgfqpoint{1.333272in}{1.221676in}}%
\pgfpathlineto{\pgfqpoint{1.448601in}{1.293491in}}%
\pgfpathlineto{\pgfqpoint{1.525487in}{1.336940in}}%
\pgfpathlineto{\pgfqpoint{1.602373in}{1.377355in}}%
\pgfpathlineto{\pgfqpoint{1.653630in}{1.402617in}}%
\pgfpathlineto{\pgfqpoint{1.717702in}{1.432398in}}%
\pgfpathlineto{\pgfqpoint{1.781773in}{1.460222in}}%
\pgfpathlineto{\pgfqpoint{1.858659in}{1.491338in}}%
\pgfpathlineto{\pgfqpoint{1.922731in}{1.515266in}}%
\pgfpathlineto{\pgfqpoint{2.025245in}{1.550177in}}%
\pgfpathlineto{\pgfqpoint{2.140574in}{1.584798in}}%
\pgfpathlineto{\pgfqpoint{2.243089in}{1.611784in}}%
\pgfpathlineto{\pgfqpoint{2.332789in}{1.632405in}}%
\pgfpathlineto{\pgfqpoint{2.435304in}{1.652880in}}%
\pgfpathlineto{\pgfqpoint{2.499375in}{1.664048in}}%
\pgfpathlineto{\pgfqpoint{2.614704in}{1.680916in}}%
\pgfpathlineto{\pgfqpoint{2.742848in}{1.695089in}}%
\pgfpathlineto{\pgfqpoint{2.845362in}{1.702935in}}%
\pgfpathlineto{\pgfqpoint{2.909434in}{1.706336in}}%
\pgfpathlineto{\pgfqpoint{3.011948in}{1.709115in}}%
\pgfpathlineto{\pgfqpoint{3.101649in}{1.709061in}}%
\pgfpathlineto{\pgfqpoint{3.178535in}{1.707003in}}%
\pgfpathlineto{\pgfqpoint{3.293864in}{1.700678in}}%
\pgfpathlineto{\pgfqpoint{3.422007in}{1.688538in}}%
\pgfpathlineto{\pgfqpoint{3.511707in}{1.676963in}}%
\pgfpathlineto{\pgfqpoint{3.588593in}{1.664725in}}%
\pgfpathlineto{\pgfqpoint{3.678293in}{1.647368in}}%
\pgfpathlineto{\pgfqpoint{3.691108in}{1.644637in}}%
\pgfpathlineto{\pgfqpoint{3.691108in}{1.644637in}}%
\pgfusepath{stroke}%
\end{pgfscope}%
\begin{pgfscope}%
\pgfpathrectangle{\pgfqpoint{0.526080in}{0.521603in}}{\pgfqpoint{3.720000in}{3.020000in}} %
\pgfusepath{clip}%
\pgfsetrectcap%
\pgfsetroundjoin%
\pgfsetlinewidth{1.505625pt}%
\definecolor{currentstroke}{rgb}{0.500000,0.000000,1.000000}%
\pgfsetstrokecolor{currentstroke}%
\pgfsetdash{}{0pt}%
\pgfpathmoveto{\pgfqpoint{0.832601in}{0.753162in}}%
\pgfpathlineto{\pgfqpoint{0.879283in}{0.811050in}}%
\pgfpathlineto{\pgfqpoint{0.925965in}{0.856607in}}%
\pgfpathlineto{\pgfqpoint{0.972647in}{0.914881in}}%
\pgfpathlineto{\pgfqpoint{0.995988in}{0.937524in}}%
\pgfpathlineto{\pgfqpoint{1.019328in}{0.958567in}}%
\pgfpathlineto{\pgfqpoint{1.042669in}{0.983309in}}%
\pgfpathlineto{\pgfqpoint{1.066010in}{1.012450in}}%
\pgfpathlineto{\pgfqpoint{1.089351in}{1.039492in}}%
\pgfpathlineto{\pgfqpoint{1.112692in}{1.058564in}}%
\pgfpathlineto{\pgfqpoint{1.136033in}{1.080868in}}%
\pgfpathlineto{\pgfqpoint{1.159374in}{1.109064in}}%
\pgfpathlineto{\pgfqpoint{1.182715in}{1.135396in}}%
\pgfpathlineto{\pgfqpoint{1.229397in}{1.175682in}}%
\pgfpathlineto{\pgfqpoint{1.252738in}{1.199100in}}%
\pgfpathlineto{\pgfqpoint{1.276079in}{1.226815in}}%
\pgfpathlineto{\pgfqpoint{1.299420in}{1.249807in}}%
\pgfpathlineto{\pgfqpoint{1.322760in}{1.268251in}}%
\pgfpathlineto{\pgfqpoint{1.346101in}{1.290082in}}%
\pgfpathlineto{\pgfqpoint{1.369442in}{1.316539in}}%
\pgfpathlineto{\pgfqpoint{1.392783in}{1.340041in}}%
\pgfpathlineto{\pgfqpoint{1.439465in}{1.378393in}}%
\pgfpathlineto{\pgfqpoint{1.462806in}{1.401823in}}%
\pgfpathlineto{\pgfqpoint{1.486147in}{1.426943in}}%
\pgfpathlineto{\pgfqpoint{1.509488in}{1.447407in}}%
\pgfpathlineto{\pgfqpoint{1.532829in}{1.465370in}}%
\pgfpathlineto{\pgfqpoint{1.556170in}{1.487302in}}%
\pgfpathlineto{\pgfqpoint{1.579511in}{1.511462in}}%
\pgfpathlineto{\pgfqpoint{1.602851in}{1.532770in}}%
\pgfpathlineto{\pgfqpoint{1.649533in}{1.570094in}}%
\pgfpathlineto{\pgfqpoint{1.696215in}{1.615608in}}%
\pgfpathlineto{\pgfqpoint{1.742897in}{1.652041in}}%
\pgfpathlineto{\pgfqpoint{1.812920in}{1.716155in}}%
\pgfpathlineto{\pgfqpoint{1.836261in}{1.733677in}}%
\pgfpathlineto{\pgfqpoint{1.859602in}{1.752950in}}%
\pgfpathlineto{\pgfqpoint{1.906283in}{1.795540in}}%
\pgfpathlineto{\pgfqpoint{1.952965in}{1.831130in}}%
\pgfpathlineto{\pgfqpoint{1.999647in}{1.873682in}}%
\pgfpathlineto{\pgfqpoint{2.069670in}{1.928240in}}%
\pgfpathlineto{\pgfqpoint{2.093011in}{1.949564in}}%
\pgfpathlineto{\pgfqpoint{2.116352in}{1.968253in}}%
\pgfpathlineto{\pgfqpoint{2.139693in}{1.985243in}}%
\pgfpathlineto{\pgfqpoint{2.163034in}{2.003919in}}%
\pgfpathlineto{\pgfqpoint{2.209715in}{2.043878in}}%
\pgfpathlineto{\pgfqpoint{2.256397in}{2.078560in}}%
\pgfpathlineto{\pgfqpoint{2.326420in}{2.135011in}}%
\pgfpathlineto{\pgfqpoint{2.349761in}{2.151913in}}%
\pgfpathlineto{\pgfqpoint{2.419784in}{2.208791in}}%
\pgfpathlineto{\pgfqpoint{2.466465in}{2.242998in}}%
\pgfpathlineto{\pgfqpoint{2.513147in}{2.280852in}}%
\pgfpathlineto{\pgfqpoint{2.559829in}{2.314503in}}%
\pgfpathlineto{\pgfqpoint{2.629852in}{2.369525in}}%
\pgfpathlineto{\pgfqpoint{2.676534in}{2.403663in}}%
\pgfpathlineto{\pgfqpoint{2.723216in}{2.439932in}}%
\pgfpathlineto{\pgfqpoint{2.769897in}{2.473465in}}%
\pgfpathlineto{\pgfqpoint{2.816579in}{2.510131in}}%
\pgfpathlineto{\pgfqpoint{2.886602in}{2.561064in}}%
\pgfpathlineto{\pgfqpoint{2.933284in}{2.595799in}}%
\pgfpathlineto{\pgfqpoint{2.979966in}{2.629430in}}%
\pgfpathlineto{\pgfqpoint{3.026648in}{2.664868in}}%
\pgfpathlineto{\pgfqpoint{3.073329in}{2.697892in}}%
\pgfpathlineto{\pgfqpoint{3.143352in}{2.749040in}}%
\pgfpathlineto{\pgfqpoint{3.190034in}{2.782598in}}%
\pgfpathlineto{\pgfqpoint{3.236716in}{2.816807in}}%
\pgfpathlineto{\pgfqpoint{3.306739in}{2.866975in}}%
\pgfpathlineto{\pgfqpoint{3.353420in}{2.899876in}}%
\pgfpathlineto{\pgfqpoint{3.400102in}{2.933290in}}%
\pgfpathlineto{\pgfqpoint{3.470125in}{2.982817in}}%
\pgfpathlineto{\pgfqpoint{3.750216in}{3.179367in}}%
\pgfpathlineto{\pgfqpoint{3.866921in}{3.260256in}}%
\pgfpathlineto{\pgfqpoint{4.076989in}{3.404331in}}%
\pgfpathlineto{\pgfqpoint{4.076989in}{3.404331in}}%
\pgfusepath{stroke}%
\end{pgfscope}%
\begin{pgfscope}%
\pgfsetrectcap%
\pgfsetmiterjoin%
\pgfsetlinewidth{0.803000pt}%
\definecolor{currentstroke}{rgb}{0.000000,0.000000,0.000000}%
\pgfsetstrokecolor{currentstroke}%
\pgfsetdash{}{0pt}%
\pgfpathmoveto{\pgfqpoint{0.526080in}{0.521603in}}%
\pgfpathlineto{\pgfqpoint{0.526080in}{3.541603in}}%
\pgfusepath{stroke}%
\end{pgfscope}%
\begin{pgfscope}%
\pgfsetrectcap%
\pgfsetmiterjoin%
\pgfsetlinewidth{0.803000pt}%
\definecolor{currentstroke}{rgb}{0.000000,0.000000,0.000000}%
\pgfsetstrokecolor{currentstroke}%
\pgfsetdash{}{0pt}%
\pgfpathmoveto{\pgfqpoint{4.246080in}{0.521603in}}%
\pgfpathlineto{\pgfqpoint{4.246080in}{3.541603in}}%
\pgfusepath{stroke}%
\end{pgfscope}%
\begin{pgfscope}%
\pgfsetrectcap%
\pgfsetmiterjoin%
\pgfsetlinewidth{0.803000pt}%
\definecolor{currentstroke}{rgb}{0.000000,0.000000,0.000000}%
\pgfsetstrokecolor{currentstroke}%
\pgfsetdash{}{0pt}%
\pgfpathmoveto{\pgfqpoint{0.526080in}{0.521603in}}%
\pgfpathlineto{\pgfqpoint{4.246080in}{0.521603in}}%
\pgfusepath{stroke}%
\end{pgfscope}%
\begin{pgfscope}%
\pgfsetrectcap%
\pgfsetmiterjoin%
\pgfsetlinewidth{0.803000pt}%
\definecolor{currentstroke}{rgb}{0.000000,0.000000,0.000000}%
\pgfsetstrokecolor{currentstroke}%
\pgfsetdash{}{0pt}%
\pgfpathmoveto{\pgfqpoint{0.526080in}{3.541603in}}%
\pgfpathlineto{\pgfqpoint{4.246080in}{3.541603in}}%
\pgfusepath{stroke}%
\end{pgfscope}%
\begin{pgfscope}%
\pgfpathrectangle{\pgfqpoint{4.478580in}{0.521603in}}{\pgfqpoint{0.151000in}{3.020000in}} %
\pgfusepath{clip}%
\pgfsetbuttcap%
\pgfsetmiterjoin%
\definecolor{currentfill}{rgb}{1.000000,1.000000,1.000000}%
\pgfsetfillcolor{currentfill}%
\pgfsetlinewidth{0.010037pt}%
\definecolor{currentstroke}{rgb}{1.000000,1.000000,1.000000}%
\pgfsetstrokecolor{currentstroke}%
\pgfsetdash{}{0pt}%
\pgfpathmoveto{\pgfqpoint{4.478580in}{0.521603in}}%
\pgfpathlineto{\pgfqpoint{4.478580in}{0.533400in}}%
\pgfpathlineto{\pgfqpoint{4.478580in}{3.529806in}}%
\pgfpathlineto{\pgfqpoint{4.478580in}{3.541603in}}%
\pgfpathlineto{\pgfqpoint{4.629580in}{3.541603in}}%
\pgfpathlineto{\pgfqpoint{4.629580in}{3.529806in}}%
\pgfpathlineto{\pgfqpoint{4.629580in}{0.533400in}}%
\pgfpathlineto{\pgfqpoint{4.629580in}{0.521603in}}%
\pgfpathclose%
\pgfusepath{stroke,fill}%
\end{pgfscope}%
\begin{pgfscope}%
\pgfsys@transformshift{4.480000in}{0.526603in}%
\pgftext[left,bottom]{\pgfimage[interpolate=true,width=0.150000in,height=3.020000in]{series_m2_ds-img0.png}}%
\end{pgfscope}%
\begin{pgfscope}%
\pgfsetbuttcap%
\pgfsetroundjoin%
\definecolor{currentfill}{rgb}{0.000000,0.000000,0.000000}%
\pgfsetfillcolor{currentfill}%
\pgfsetlinewidth{0.803000pt}%
\definecolor{currentstroke}{rgb}{0.000000,0.000000,0.000000}%
\pgfsetstrokecolor{currentstroke}%
\pgfsetdash{}{0pt}%
\pgfsys@defobject{currentmarker}{\pgfqpoint{0.000000in}{0.000000in}}{\pgfqpoint{0.048611in}{0.000000in}}{%
\pgfpathmoveto{\pgfqpoint{0.000000in}{0.000000in}}%
\pgfpathlineto{\pgfqpoint{0.048611in}{0.000000in}}%
\pgfusepath{stroke,fill}%
}%
\begin{pgfscope}%
\pgfsys@transformshift{4.629580in}{0.617964in}%
\pgfsys@useobject{currentmarker}{}%
\end{pgfscope}%
\end{pgfscope}%
\begin{pgfscope}%
\pgfsetbuttcap%
\pgfsetroundjoin%
\definecolor{currentfill}{rgb}{0.000000,0.000000,0.000000}%
\pgfsetfillcolor{currentfill}%
\pgfsetlinewidth{0.803000pt}%
\definecolor{currentstroke}{rgb}{0.000000,0.000000,0.000000}%
\pgfsetstrokecolor{currentstroke}%
\pgfsetdash{}{0pt}%
\pgfsys@defobject{currentmarker}{\pgfqpoint{0.000000in}{0.000000in}}{\pgfqpoint{0.048611in}{0.000000in}}{%
\pgfpathmoveto{\pgfqpoint{0.000000in}{0.000000in}}%
\pgfpathlineto{\pgfqpoint{0.048611in}{0.000000in}}%
\pgfusepath{stroke,fill}%
}%
\begin{pgfscope}%
\pgfsys@transformshift{4.629580in}{0.796036in}%
\pgfsys@useobject{currentmarker}{}%
\end{pgfscope}%
\end{pgfscope}%
\begin{pgfscope}%
\pgfsetbuttcap%
\pgfsetroundjoin%
\definecolor{currentfill}{rgb}{0.000000,0.000000,0.000000}%
\pgfsetfillcolor{currentfill}%
\pgfsetlinewidth{0.803000pt}%
\definecolor{currentstroke}{rgb}{0.000000,0.000000,0.000000}%
\pgfsetstrokecolor{currentstroke}%
\pgfsetdash{}{0pt}%
\pgfsys@defobject{currentmarker}{\pgfqpoint{0.000000in}{0.000000in}}{\pgfqpoint{0.048611in}{0.000000in}}{%
\pgfpathmoveto{\pgfqpoint{0.000000in}{0.000000in}}%
\pgfpathlineto{\pgfqpoint{0.048611in}{0.000000in}}%
\pgfusepath{stroke,fill}%
}%
\begin{pgfscope}%
\pgfsys@transformshift{4.629580in}{0.941531in}%
\pgfsys@useobject{currentmarker}{}%
\end{pgfscope}%
\end{pgfscope}%
\begin{pgfscope}%
\pgfsetbuttcap%
\pgfsetroundjoin%
\definecolor{currentfill}{rgb}{0.000000,0.000000,0.000000}%
\pgfsetfillcolor{currentfill}%
\pgfsetlinewidth{0.803000pt}%
\definecolor{currentstroke}{rgb}{0.000000,0.000000,0.000000}%
\pgfsetstrokecolor{currentstroke}%
\pgfsetdash{}{0pt}%
\pgfsys@defobject{currentmarker}{\pgfqpoint{0.000000in}{0.000000in}}{\pgfqpoint{0.048611in}{0.000000in}}{%
\pgfpathmoveto{\pgfqpoint{0.000000in}{0.000000in}}%
\pgfpathlineto{\pgfqpoint{0.048611in}{0.000000in}}%
\pgfusepath{stroke,fill}%
}%
\begin{pgfscope}%
\pgfsys@transformshift{4.629580in}{1.064545in}%
\pgfsys@useobject{currentmarker}{}%
\end{pgfscope}%
\end{pgfscope}%
\begin{pgfscope}%
\pgfsetbuttcap%
\pgfsetroundjoin%
\definecolor{currentfill}{rgb}{0.000000,0.000000,0.000000}%
\pgfsetfillcolor{currentfill}%
\pgfsetlinewidth{0.803000pt}%
\definecolor{currentstroke}{rgb}{0.000000,0.000000,0.000000}%
\pgfsetstrokecolor{currentstroke}%
\pgfsetdash{}{0pt}%
\pgfsys@defobject{currentmarker}{\pgfqpoint{0.000000in}{0.000000in}}{\pgfqpoint{0.048611in}{0.000000in}}{%
\pgfpathmoveto{\pgfqpoint{0.000000in}{0.000000in}}%
\pgfpathlineto{\pgfqpoint{0.048611in}{0.000000in}}%
\pgfusepath{stroke,fill}%
}%
\begin{pgfscope}%
\pgfsys@transformshift{4.629580in}{1.171104in}%
\pgfsys@useobject{currentmarker}{}%
\end{pgfscope}%
\end{pgfscope}%
\begin{pgfscope}%
\pgfsetbuttcap%
\pgfsetroundjoin%
\definecolor{currentfill}{rgb}{0.000000,0.000000,0.000000}%
\pgfsetfillcolor{currentfill}%
\pgfsetlinewidth{0.803000pt}%
\definecolor{currentstroke}{rgb}{0.000000,0.000000,0.000000}%
\pgfsetstrokecolor{currentstroke}%
\pgfsetdash{}{0pt}%
\pgfsys@defobject{currentmarker}{\pgfqpoint{0.000000in}{0.000000in}}{\pgfqpoint{0.048611in}{0.000000in}}{%
\pgfpathmoveto{\pgfqpoint{0.000000in}{0.000000in}}%
\pgfpathlineto{\pgfqpoint{0.048611in}{0.000000in}}%
\pgfusepath{stroke,fill}%
}%
\begin{pgfscope}%
\pgfsys@transformshift{4.629580in}{1.265097in}%
\pgfsys@useobject{currentmarker}{}%
\end{pgfscope}%
\end{pgfscope}%
\begin{pgfscope}%
\pgfsetbuttcap%
\pgfsetroundjoin%
\definecolor{currentfill}{rgb}{0.000000,0.000000,0.000000}%
\pgfsetfillcolor{currentfill}%
\pgfsetlinewidth{0.803000pt}%
\definecolor{currentstroke}{rgb}{0.000000,0.000000,0.000000}%
\pgfsetstrokecolor{currentstroke}%
\pgfsetdash{}{0pt}%
\pgfsys@defobject{currentmarker}{\pgfqpoint{0.000000in}{0.000000in}}{\pgfqpoint{0.048611in}{0.000000in}}{%
\pgfpathmoveto{\pgfqpoint{0.000000in}{0.000000in}}%
\pgfpathlineto{\pgfqpoint{0.048611in}{0.000000in}}%
\pgfusepath{stroke,fill}%
}%
\begin{pgfscope}%
\pgfsys@transformshift{4.629580in}{1.349176in}%
\pgfsys@useobject{currentmarker}{}%
\end{pgfscope}%
\end{pgfscope}%
\begin{pgfscope}%
\pgftext[x=4.726802in,y=1.296414in,left,base]{\rmfamily\fontsize{10.000000}{12.000000}\selectfont \(\displaystyle 10^{-1}\)}%
\end{pgfscope}%
\begin{pgfscope}%
\pgfsetbuttcap%
\pgfsetroundjoin%
\definecolor{currentfill}{rgb}{0.000000,0.000000,0.000000}%
\pgfsetfillcolor{currentfill}%
\pgfsetlinewidth{0.803000pt}%
\definecolor{currentstroke}{rgb}{0.000000,0.000000,0.000000}%
\pgfsetstrokecolor{currentstroke}%
\pgfsetdash{}{0pt}%
\pgfsys@defobject{currentmarker}{\pgfqpoint{0.000000in}{0.000000in}}{\pgfqpoint{0.048611in}{0.000000in}}{%
\pgfpathmoveto{\pgfqpoint{0.000000in}{0.000000in}}%
\pgfpathlineto{\pgfqpoint{0.048611in}{0.000000in}}%
\pgfusepath{stroke,fill}%
}%
\begin{pgfscope}%
\pgfsys@transformshift{4.629580in}{1.902316in}%
\pgfsys@useobject{currentmarker}{}%
\end{pgfscope}%
\end{pgfscope}%
\begin{pgfscope}%
\pgfsetbuttcap%
\pgfsetroundjoin%
\definecolor{currentfill}{rgb}{0.000000,0.000000,0.000000}%
\pgfsetfillcolor{currentfill}%
\pgfsetlinewidth{0.803000pt}%
\definecolor{currentstroke}{rgb}{0.000000,0.000000,0.000000}%
\pgfsetstrokecolor{currentstroke}%
\pgfsetdash{}{0pt}%
\pgfsys@defobject{currentmarker}{\pgfqpoint{0.000000in}{0.000000in}}{\pgfqpoint{0.048611in}{0.000000in}}{%
\pgfpathmoveto{\pgfqpoint{0.000000in}{0.000000in}}%
\pgfpathlineto{\pgfqpoint{0.048611in}{0.000000in}}%
\pgfusepath{stroke,fill}%
}%
\begin{pgfscope}%
\pgfsys@transformshift{4.629580in}{2.225882in}%
\pgfsys@useobject{currentmarker}{}%
\end{pgfscope}%
\end{pgfscope}%
\begin{pgfscope}%
\pgfsetbuttcap%
\pgfsetroundjoin%
\definecolor{currentfill}{rgb}{0.000000,0.000000,0.000000}%
\pgfsetfillcolor{currentfill}%
\pgfsetlinewidth{0.803000pt}%
\definecolor{currentstroke}{rgb}{0.000000,0.000000,0.000000}%
\pgfsetstrokecolor{currentstroke}%
\pgfsetdash{}{0pt}%
\pgfsys@defobject{currentmarker}{\pgfqpoint{0.000000in}{0.000000in}}{\pgfqpoint{0.048611in}{0.000000in}}{%
\pgfpathmoveto{\pgfqpoint{0.000000in}{0.000000in}}%
\pgfpathlineto{\pgfqpoint{0.048611in}{0.000000in}}%
\pgfusepath{stroke,fill}%
}%
\begin{pgfscope}%
\pgfsys@transformshift{4.629580in}{2.455456in}%
\pgfsys@useobject{currentmarker}{}%
\end{pgfscope}%
\end{pgfscope}%
\begin{pgfscope}%
\pgfsetbuttcap%
\pgfsetroundjoin%
\definecolor{currentfill}{rgb}{0.000000,0.000000,0.000000}%
\pgfsetfillcolor{currentfill}%
\pgfsetlinewidth{0.803000pt}%
\definecolor{currentstroke}{rgb}{0.000000,0.000000,0.000000}%
\pgfsetstrokecolor{currentstroke}%
\pgfsetdash{}{0pt}%
\pgfsys@defobject{currentmarker}{\pgfqpoint{0.000000in}{0.000000in}}{\pgfqpoint{0.048611in}{0.000000in}}{%
\pgfpathmoveto{\pgfqpoint{0.000000in}{0.000000in}}%
\pgfpathlineto{\pgfqpoint{0.048611in}{0.000000in}}%
\pgfusepath{stroke,fill}%
}%
\begin{pgfscope}%
\pgfsys@transformshift{4.629580in}{2.633527in}%
\pgfsys@useobject{currentmarker}{}%
\end{pgfscope}%
\end{pgfscope}%
\begin{pgfscope}%
\pgfsetbuttcap%
\pgfsetroundjoin%
\definecolor{currentfill}{rgb}{0.000000,0.000000,0.000000}%
\pgfsetfillcolor{currentfill}%
\pgfsetlinewidth{0.803000pt}%
\definecolor{currentstroke}{rgb}{0.000000,0.000000,0.000000}%
\pgfsetstrokecolor{currentstroke}%
\pgfsetdash{}{0pt}%
\pgfsys@defobject{currentmarker}{\pgfqpoint{0.000000in}{0.000000in}}{\pgfqpoint{0.048611in}{0.000000in}}{%
\pgfpathmoveto{\pgfqpoint{0.000000in}{0.000000in}}%
\pgfpathlineto{\pgfqpoint{0.048611in}{0.000000in}}%
\pgfusepath{stroke,fill}%
}%
\begin{pgfscope}%
\pgfsys@transformshift{4.629580in}{2.779022in}%
\pgfsys@useobject{currentmarker}{}%
\end{pgfscope}%
\end{pgfscope}%
\begin{pgfscope}%
\pgfsetbuttcap%
\pgfsetroundjoin%
\definecolor{currentfill}{rgb}{0.000000,0.000000,0.000000}%
\pgfsetfillcolor{currentfill}%
\pgfsetlinewidth{0.803000pt}%
\definecolor{currentstroke}{rgb}{0.000000,0.000000,0.000000}%
\pgfsetstrokecolor{currentstroke}%
\pgfsetdash{}{0pt}%
\pgfsys@defobject{currentmarker}{\pgfqpoint{0.000000in}{0.000000in}}{\pgfqpoint{0.048611in}{0.000000in}}{%
\pgfpathmoveto{\pgfqpoint{0.000000in}{0.000000in}}%
\pgfpathlineto{\pgfqpoint{0.048611in}{0.000000in}}%
\pgfusepath{stroke,fill}%
}%
\begin{pgfscope}%
\pgfsys@transformshift{4.629580in}{2.902036in}%
\pgfsys@useobject{currentmarker}{}%
\end{pgfscope}%
\end{pgfscope}%
\begin{pgfscope}%
\pgfsetbuttcap%
\pgfsetroundjoin%
\definecolor{currentfill}{rgb}{0.000000,0.000000,0.000000}%
\pgfsetfillcolor{currentfill}%
\pgfsetlinewidth{0.803000pt}%
\definecolor{currentstroke}{rgb}{0.000000,0.000000,0.000000}%
\pgfsetstrokecolor{currentstroke}%
\pgfsetdash{}{0pt}%
\pgfsys@defobject{currentmarker}{\pgfqpoint{0.000000in}{0.000000in}}{\pgfqpoint{0.048611in}{0.000000in}}{%
\pgfpathmoveto{\pgfqpoint{0.000000in}{0.000000in}}%
\pgfpathlineto{\pgfqpoint{0.048611in}{0.000000in}}%
\pgfusepath{stroke,fill}%
}%
\begin{pgfscope}%
\pgfsys@transformshift{4.629580in}{3.008596in}%
\pgfsys@useobject{currentmarker}{}%
\end{pgfscope}%
\end{pgfscope}%
\begin{pgfscope}%
\pgfsetbuttcap%
\pgfsetroundjoin%
\definecolor{currentfill}{rgb}{0.000000,0.000000,0.000000}%
\pgfsetfillcolor{currentfill}%
\pgfsetlinewidth{0.803000pt}%
\definecolor{currentstroke}{rgb}{0.000000,0.000000,0.000000}%
\pgfsetstrokecolor{currentstroke}%
\pgfsetdash{}{0pt}%
\pgfsys@defobject{currentmarker}{\pgfqpoint{0.000000in}{0.000000in}}{\pgfqpoint{0.048611in}{0.000000in}}{%
\pgfpathmoveto{\pgfqpoint{0.000000in}{0.000000in}}%
\pgfpathlineto{\pgfqpoint{0.048611in}{0.000000in}}%
\pgfusepath{stroke,fill}%
}%
\begin{pgfscope}%
\pgfsys@transformshift{4.629580in}{3.102588in}%
\pgfsys@useobject{currentmarker}{}%
\end{pgfscope}%
\end{pgfscope}%
\begin{pgfscope}%
\pgfsetbuttcap%
\pgfsetroundjoin%
\definecolor{currentfill}{rgb}{0.000000,0.000000,0.000000}%
\pgfsetfillcolor{currentfill}%
\pgfsetlinewidth{0.803000pt}%
\definecolor{currentstroke}{rgb}{0.000000,0.000000,0.000000}%
\pgfsetstrokecolor{currentstroke}%
\pgfsetdash{}{0pt}%
\pgfsys@defobject{currentmarker}{\pgfqpoint{0.000000in}{0.000000in}}{\pgfqpoint{0.048611in}{0.000000in}}{%
\pgfpathmoveto{\pgfqpoint{0.000000in}{0.000000in}}%
\pgfpathlineto{\pgfqpoint{0.048611in}{0.000000in}}%
\pgfusepath{stroke,fill}%
}%
\begin{pgfscope}%
\pgfsys@transformshift{4.629580in}{3.186667in}%
\pgfsys@useobject{currentmarker}{}%
\end{pgfscope}%
\end{pgfscope}%
\begin{pgfscope}%
\pgftext[x=4.726802in,y=3.133906in,left,base]{\rmfamily\fontsize{10.000000}{12.000000}\selectfont \(\displaystyle 10^{0}\)}%
\end{pgfscope}%
\begin{pgfscope}%
\pgftext[x=5.209249in,y=2.031603in,,top]{\rmfamily\fontsize{12.000000}{14.400000}\selectfont \(\displaystyle {\mathbf{E} \mbox{u}}\)}%
\end{pgfscope}%
\begin{pgfscope}%
\pgfsetbuttcap%
\pgfsetmiterjoin%
\pgfsetlinewidth{0.803000pt}%
\definecolor{currentstroke}{rgb}{0.000000,0.000000,0.000000}%
\pgfsetstrokecolor{currentstroke}%
\pgfsetdash{}{0pt}%
\pgfpathmoveto{\pgfqpoint{4.478580in}{0.521603in}}%
\pgfpathlineto{\pgfqpoint{4.478580in}{0.533400in}}%
\pgfpathlineto{\pgfqpoint{4.478580in}{3.529806in}}%
\pgfpathlineto{\pgfqpoint{4.478580in}{3.541603in}}%
\pgfpathlineto{\pgfqpoint{4.629580in}{3.541603in}}%
\pgfpathlineto{\pgfqpoint{4.629580in}{3.529806in}}%
\pgfpathlineto{\pgfqpoint{4.629580in}{0.533400in}}%
\pgfpathlineto{\pgfqpoint{4.629580in}{0.521603in}}%
\pgfpathclose%
\pgfusepath{stroke}%
\end{pgfscope}%
\end{pgfpicture}%
\makeatother%
\endgroup%

    \caption{.\label{fig:dnumbs}}
\end{figure}
We should note that there are several kinds of systematic error that influence our data. We assume that droplet translate purely along the central axis of the electric field, but in practice, despite the improvement in surface charge density uniformity produced by corona charging, there are still local areas of especially high charge density. In principle, this kind of error should become small for droplets which are far enough away from the charge distribution, that the geometry of the charge distribution disappears, and the electric field looks like that due to a point charge. Another form of error is in the initial velocity as it appears in $\mathbb{E}\mbox{u}$. Because we usually lose the first few frames of video due to camera shake transients at the start of the low-gravity experiment, we will consistently underestimate $U_0$ because the droplet will already have decelerated significantly during that period of time. The primary sources of random error are the effect of contact line hysteresis on the droplet initial velocity, and of the variance in the MLE parameter estimates.
\newpage

\section{Impact Dynamics \hl{[placeholder section]}}
Using the unique capabilities of the low-gravity environment we obtained data on dimensionless contact time and coefficients of restitution at very low Ohnesorge numbers for a range of electric Bond numbers. Despite strong electric fields (20-30 $kV/cm$) we found little evidence for wetting transitions due to excession of a critical pressure (the ``Fakir impalement''). There is no obvious trend in dimensionless contact time or coefficient of restitution with electric Bond number.

Jump velocities are more strongly damped for relatively small droplet volumes in the presence of the electric fields than was shown by Attari \emph{et. al.}. This may be evidence for electrowetting paradoxically enhancing the effect of contact angle hysteresis pinning on sharp corners. (How does this tie into the coefficients of restitution problem?)

\begin{figure}[htb]
    \centering
    %% Creator: Matplotlib, PGF backend
%%
%% To include the figure in your LaTeX document, write
%%   \input{<filename>.pgf}
%%
%% Make sure the required packages are loaded in your preamble
%%   \usepackage{pgf}
%%
%% Figures using additional raster images can only be included by \input if
%% they are in the same directory as the main LaTeX file. For loading figures
%% from other directories you can use the `import` package
%%   \usepackage{import}
%% and then include the figures with
%%   \import{<path to file>}{<filename>.pgf}
%%
%% Matplotlib used the following preamble
%%   \usepackage{fontspec}
%%   \setmainfont{DejaVu Serif}
%%   \setsansfont{DejaVu Sans}
%%   \setmonofont{DejaVu Sans Mono}
%%
\begingroup%
\makeatletter%
\begin{pgfpicture}%
\pgfpathrectangle{\pgfpointorigin}{\pgfqpoint{5.437191in}{3.676603in}}%
\pgfusepath{use as bounding box, clip}%
\begin{pgfscope}%
\pgfsetbuttcap%
\pgfsetmiterjoin%
\definecolor{currentfill}{rgb}{1.000000,1.000000,1.000000}%
\pgfsetfillcolor{currentfill}%
\pgfsetlinewidth{0.000000pt}%
\definecolor{currentstroke}{rgb}{1.000000,1.000000,1.000000}%
\pgfsetstrokecolor{currentstroke}%
\pgfsetdash{}{0pt}%
\pgfpathmoveto{\pgfqpoint{0.000000in}{0.000000in}}%
\pgfpathlineto{\pgfqpoint{5.437191in}{0.000000in}}%
\pgfpathlineto{\pgfqpoint{5.437191in}{3.676603in}}%
\pgfpathlineto{\pgfqpoint{0.000000in}{3.676603in}}%
\pgfpathclose%
\pgfusepath{fill}%
\end{pgfscope}%
\begin{pgfscope}%
\pgfsetbuttcap%
\pgfsetmiterjoin%
\definecolor{currentfill}{rgb}{1.000000,1.000000,1.000000}%
\pgfsetfillcolor{currentfill}%
\pgfsetlinewidth{0.000000pt}%
\definecolor{currentstroke}{rgb}{0.000000,0.000000,0.000000}%
\pgfsetstrokecolor{currentstroke}%
\pgfsetstrokeopacity{0.000000}%
\pgfsetdash{}{0pt}%
\pgfpathmoveto{\pgfqpoint{0.574151in}{0.521603in}}%
\pgfpathlineto{\pgfqpoint{4.294151in}{0.521603in}}%
\pgfpathlineto{\pgfqpoint{4.294151in}{3.541603in}}%
\pgfpathlineto{\pgfqpoint{0.574151in}{3.541603in}}%
\pgfpathclose%
\pgfusepath{fill}%
\end{pgfscope}%
\begin{pgfscope}%
\pgfpathrectangle{\pgfqpoint{0.574151in}{0.521603in}}{\pgfqpoint{3.720000in}{3.020000in}} %
\pgfusepath{clip}%
\pgfsetbuttcap%
\pgfsetroundjoin%
\definecolor{currentfill}{rgb}{0.441176,0.995734,0.739009}%
\pgfsetfillcolor{currentfill}%
\pgfsetlinewidth{1.003750pt}%
\definecolor{currentstroke}{rgb}{0.441176,0.995734,0.739009}%
\pgfsetstrokecolor{currentstroke}%
\pgfsetdash{}{0pt}%
\pgfpathmoveto{\pgfqpoint{3.151854in}{1.797941in}}%
\pgfpathcurveto{\pgfqpoint{3.162905in}{1.797941in}}{\pgfqpoint{3.173504in}{1.802331in}}{\pgfqpoint{3.181317in}{1.810145in}}%
\pgfpathcurveto{\pgfqpoint{3.189131in}{1.817958in}}{\pgfqpoint{3.193521in}{1.828557in}}{\pgfqpoint{3.193521in}{1.839607in}}%
\pgfpathcurveto{\pgfqpoint{3.193521in}{1.850657in}}{\pgfqpoint{3.189131in}{1.861257in}}{\pgfqpoint{3.181317in}{1.869070in}}%
\pgfpathcurveto{\pgfqpoint{3.173504in}{1.876884in}}{\pgfqpoint{3.162905in}{1.881274in}}{\pgfqpoint{3.151854in}{1.881274in}}%
\pgfpathcurveto{\pgfqpoint{3.140804in}{1.881274in}}{\pgfqpoint{3.130205in}{1.876884in}}{\pgfqpoint{3.122392in}{1.869070in}}%
\pgfpathcurveto{\pgfqpoint{3.114578in}{1.861257in}}{\pgfqpoint{3.110188in}{1.850657in}}{\pgfqpoint{3.110188in}{1.839607in}}%
\pgfpathcurveto{\pgfqpoint{3.110188in}{1.828557in}}{\pgfqpoint{3.114578in}{1.817958in}}{\pgfqpoint{3.122392in}{1.810145in}}%
\pgfpathcurveto{\pgfqpoint{3.130205in}{1.802331in}}{\pgfqpoint{3.140804in}{1.797941in}}{\pgfqpoint{3.151854in}{1.797941in}}%
\pgfpathclose%
\pgfusepath{stroke,fill}%
\end{pgfscope}%
\begin{pgfscope}%
\pgfpathrectangle{\pgfqpoint{0.574151in}{0.521603in}}{\pgfqpoint{3.720000in}{3.020000in}} %
\pgfusepath{clip}%
\pgfsetbuttcap%
\pgfsetroundjoin%
\definecolor{currentfill}{rgb}{0.209804,0.440216,0.974139}%
\pgfsetfillcolor{currentfill}%
\pgfsetlinewidth{1.003750pt}%
\definecolor{currentstroke}{rgb}{0.209804,0.440216,0.974139}%
\pgfsetstrokecolor{currentstroke}%
\pgfsetdash{}{0pt}%
\pgfpathmoveto{\pgfqpoint{3.179149in}{2.208860in}}%
\pgfpathcurveto{\pgfqpoint{3.190199in}{2.208860in}}{\pgfqpoint{3.200798in}{2.213251in}}{\pgfqpoint{3.208612in}{2.221064in}}%
\pgfpathcurveto{\pgfqpoint{3.216426in}{2.228878in}}{\pgfqpoint{3.220816in}{2.239477in}}{\pgfqpoint{3.220816in}{2.250527in}}%
\pgfpathcurveto{\pgfqpoint{3.220816in}{2.261577in}}{\pgfqpoint{3.216426in}{2.272176in}}{\pgfqpoint{3.208612in}{2.279990in}}%
\pgfpathcurveto{\pgfqpoint{3.200798in}{2.287803in}}{\pgfqpoint{3.190199in}{2.292194in}}{\pgfqpoint{3.179149in}{2.292194in}}%
\pgfpathcurveto{\pgfqpoint{3.168099in}{2.292194in}}{\pgfqpoint{3.157500in}{2.287803in}}{\pgfqpoint{3.149686in}{2.279990in}}%
\pgfpathcurveto{\pgfqpoint{3.141873in}{2.272176in}}{\pgfqpoint{3.137483in}{2.261577in}}{\pgfqpoint{3.137483in}{2.250527in}}%
\pgfpathcurveto{\pgfqpoint{3.137483in}{2.239477in}}{\pgfqpoint{3.141873in}{2.228878in}}{\pgfqpoint{3.149686in}{2.221064in}}%
\pgfpathcurveto{\pgfqpoint{3.157500in}{2.213251in}}{\pgfqpoint{3.168099in}{2.208860in}}{\pgfqpoint{3.179149in}{2.208860in}}%
\pgfpathclose%
\pgfusepath{stroke,fill}%
\end{pgfscope}%
\begin{pgfscope}%
\pgfpathrectangle{\pgfqpoint{0.574151in}{0.521603in}}{\pgfqpoint{3.720000in}{3.020000in}} %
\pgfusepath{clip}%
\pgfsetbuttcap%
\pgfsetroundjoin%
\definecolor{currentfill}{rgb}{0.178431,0.483911,0.968276}%
\pgfsetfillcolor{currentfill}%
\pgfsetlinewidth{1.003750pt}%
\definecolor{currentstroke}{rgb}{0.178431,0.483911,0.968276}%
\pgfsetstrokecolor{currentstroke}%
\pgfsetdash{}{0pt}%
\pgfpathmoveto{\pgfqpoint{2.797680in}{3.329817in}}%
\pgfpathcurveto{\pgfqpoint{2.808730in}{3.329817in}}{\pgfqpoint{2.819329in}{3.334207in}}{\pgfqpoint{2.827142in}{3.342021in}}%
\pgfpathcurveto{\pgfqpoint{2.834956in}{3.349835in}}{\pgfqpoint{2.839346in}{3.360434in}}{\pgfqpoint{2.839346in}{3.371484in}}%
\pgfpathcurveto{\pgfqpoint{2.839346in}{3.382534in}}{\pgfqpoint{2.834956in}{3.393133in}}{\pgfqpoint{2.827142in}{3.400947in}}%
\pgfpathcurveto{\pgfqpoint{2.819329in}{3.408760in}}{\pgfqpoint{2.808730in}{3.413151in}}{\pgfqpoint{2.797680in}{3.413151in}}%
\pgfpathcurveto{\pgfqpoint{2.786629in}{3.413151in}}{\pgfqpoint{2.776030in}{3.408760in}}{\pgfqpoint{2.768217in}{3.400947in}}%
\pgfpathcurveto{\pgfqpoint{2.760403in}{3.393133in}}{\pgfqpoint{2.756013in}{3.382534in}}{\pgfqpoint{2.756013in}{3.371484in}}%
\pgfpathcurveto{\pgfqpoint{2.756013in}{3.360434in}}{\pgfqpoint{2.760403in}{3.349835in}}{\pgfqpoint{2.768217in}{3.342021in}}%
\pgfpathcurveto{\pgfqpoint{2.776030in}{3.334207in}}{\pgfqpoint{2.786629in}{3.329817in}}{\pgfqpoint{2.797680in}{3.329817in}}%
\pgfpathclose%
\pgfusepath{stroke,fill}%
\end{pgfscope}%
\begin{pgfscope}%
\pgfpathrectangle{\pgfqpoint{0.574151in}{0.521603in}}{\pgfqpoint{3.720000in}{3.020000in}} %
\pgfusepath{clip}%
\pgfsetbuttcap%
\pgfsetroundjoin%
\definecolor{currentfill}{rgb}{0.468627,0.049260,0.999696}%
\pgfsetfillcolor{currentfill}%
\pgfsetlinewidth{1.003750pt}%
\definecolor{currentstroke}{rgb}{0.468627,0.049260,0.999696}%
\pgfsetstrokecolor{currentstroke}%
\pgfsetdash{}{0pt}%
\pgfpathmoveto{\pgfqpoint{2.011631in}{3.351296in}}%
\pgfpathcurveto{\pgfqpoint{2.022681in}{3.351296in}}{\pgfqpoint{2.033280in}{3.355687in}}{\pgfqpoint{2.041094in}{3.363500in}}%
\pgfpathcurveto{\pgfqpoint{2.048907in}{3.371314in}}{\pgfqpoint{2.053297in}{3.381913in}}{\pgfqpoint{2.053297in}{3.392963in}}%
\pgfpathcurveto{\pgfqpoint{2.053297in}{3.404013in}}{\pgfqpoint{2.048907in}{3.414612in}}{\pgfqpoint{2.041094in}{3.422426in}}%
\pgfpathcurveto{\pgfqpoint{2.033280in}{3.430239in}}{\pgfqpoint{2.022681in}{3.434630in}}{\pgfqpoint{2.011631in}{3.434630in}}%
\pgfpathcurveto{\pgfqpoint{2.000581in}{3.434630in}}{\pgfqpoint{1.989982in}{3.430239in}}{\pgfqpoint{1.982168in}{3.422426in}}%
\pgfpathcurveto{\pgfqpoint{1.974354in}{3.414612in}}{\pgfqpoint{1.969964in}{3.404013in}}{\pgfqpoint{1.969964in}{3.392963in}}%
\pgfpathcurveto{\pgfqpoint{1.969964in}{3.381913in}}{\pgfqpoint{1.974354in}{3.371314in}}{\pgfqpoint{1.982168in}{3.363500in}}%
\pgfpathcurveto{\pgfqpoint{1.989982in}{3.355687in}}{\pgfqpoint{2.000581in}{3.351296in}}{\pgfqpoint{2.011631in}{3.351296in}}%
\pgfpathclose%
\pgfusepath{stroke,fill}%
\end{pgfscope}%
\begin{pgfscope}%
\pgfpathrectangle{\pgfqpoint{0.574151in}{0.521603in}}{\pgfqpoint{3.720000in}{3.020000in}} %
\pgfusepath{clip}%
\pgfsetbuttcap%
\pgfsetroundjoin%
\definecolor{currentfill}{rgb}{1.000000,0.505325,0.261793}%
\pgfsetfillcolor{currentfill}%
\pgfsetlinewidth{1.003750pt}%
\definecolor{currentstroke}{rgb}{1.000000,0.505325,0.261793}%
\pgfsetstrokecolor{currentstroke}%
\pgfsetdash{}{0pt}%
\pgfpathmoveto{\pgfqpoint{3.917459in}{0.946599in}}%
\pgfpathcurveto{\pgfqpoint{3.928509in}{0.946599in}}{\pgfqpoint{3.939108in}{0.950989in}}{\pgfqpoint{3.946922in}{0.958803in}}%
\pgfpathcurveto{\pgfqpoint{3.954735in}{0.966616in}}{\pgfqpoint{3.959125in}{0.977215in}}{\pgfqpoint{3.959125in}{0.988265in}}%
\pgfpathcurveto{\pgfqpoint{3.959125in}{0.999315in}}{\pgfqpoint{3.954735in}{1.009914in}}{\pgfqpoint{3.946922in}{1.017728in}}%
\pgfpathcurveto{\pgfqpoint{3.939108in}{1.025542in}}{\pgfqpoint{3.928509in}{1.029932in}}{\pgfqpoint{3.917459in}{1.029932in}}%
\pgfpathcurveto{\pgfqpoint{3.906409in}{1.029932in}}{\pgfqpoint{3.895810in}{1.025542in}}{\pgfqpoint{3.887996in}{1.017728in}}%
\pgfpathcurveto{\pgfqpoint{3.880182in}{1.009914in}}{\pgfqpoint{3.875792in}{0.999315in}}{\pgfqpoint{3.875792in}{0.988265in}}%
\pgfpathcurveto{\pgfqpoint{3.875792in}{0.977215in}}{\pgfqpoint{3.880182in}{0.966616in}}{\pgfqpoint{3.887996in}{0.958803in}}%
\pgfpathcurveto{\pgfqpoint{3.895810in}{0.950989in}}{\pgfqpoint{3.906409in}{0.946599in}}{\pgfqpoint{3.917459in}{0.946599in}}%
\pgfpathclose%
\pgfusepath{stroke,fill}%
\end{pgfscope}%
\begin{pgfscope}%
\pgfpathrectangle{\pgfqpoint{0.574151in}{0.521603in}}{\pgfqpoint{3.720000in}{3.020000in}} %
\pgfusepath{clip}%
\pgfsetbuttcap%
\pgfsetroundjoin%
\definecolor{currentfill}{rgb}{1.000000,0.505325,0.261793}%
\pgfsetfillcolor{currentfill}%
\pgfsetlinewidth{1.003750pt}%
\definecolor{currentstroke}{rgb}{1.000000,0.505325,0.261793}%
\pgfsetstrokecolor{currentstroke}%
\pgfsetdash{}{0pt}%
\pgfpathmoveto{\pgfqpoint{3.571021in}{0.946599in}}%
\pgfpathcurveto{\pgfqpoint{3.582071in}{0.946599in}}{\pgfqpoint{3.592670in}{0.950989in}}{\pgfqpoint{3.600484in}{0.958803in}}%
\pgfpathcurveto{\pgfqpoint{3.608298in}{0.966616in}}{\pgfqpoint{3.612688in}{0.977215in}}{\pgfqpoint{3.612688in}{0.988265in}}%
\pgfpathcurveto{\pgfqpoint{3.612688in}{0.999315in}}{\pgfqpoint{3.608298in}{1.009914in}}{\pgfqpoint{3.600484in}{1.017728in}}%
\pgfpathcurveto{\pgfqpoint{3.592670in}{1.025542in}}{\pgfqpoint{3.582071in}{1.029932in}}{\pgfqpoint{3.571021in}{1.029932in}}%
\pgfpathcurveto{\pgfqpoint{3.559971in}{1.029932in}}{\pgfqpoint{3.549372in}{1.025542in}}{\pgfqpoint{3.541558in}{1.017728in}}%
\pgfpathcurveto{\pgfqpoint{3.533745in}{1.009914in}}{\pgfqpoint{3.529355in}{0.999315in}}{\pgfqpoint{3.529355in}{0.988265in}}%
\pgfpathcurveto{\pgfqpoint{3.529355in}{0.977215in}}{\pgfqpoint{3.533745in}{0.966616in}}{\pgfqpoint{3.541558in}{0.958803in}}%
\pgfpathcurveto{\pgfqpoint{3.549372in}{0.950989in}}{\pgfqpoint{3.559971in}{0.946599in}}{\pgfqpoint{3.571021in}{0.946599in}}%
\pgfpathclose%
\pgfusepath{stroke,fill}%
\end{pgfscope}%
\begin{pgfscope}%
\pgfpathrectangle{\pgfqpoint{0.574151in}{0.521603in}}{\pgfqpoint{3.720000in}{3.020000in}} %
\pgfusepath{clip}%
\pgfsetbuttcap%
\pgfsetroundjoin%
\definecolor{currentfill}{rgb}{1.000000,0.645928,0.343949}%
\pgfsetfillcolor{currentfill}%
\pgfsetlinewidth{1.003750pt}%
\definecolor{currentstroke}{rgb}{1.000000,0.645928,0.343949}%
\pgfsetstrokecolor{currentstroke}%
\pgfsetdash{}{0pt}%
\pgfpathmoveto{\pgfqpoint{3.526117in}{1.586433in}}%
\pgfpathcurveto{\pgfqpoint{3.537167in}{1.586433in}}{\pgfqpoint{3.547766in}{1.590823in}}{\pgfqpoint{3.555580in}{1.598637in}}%
\pgfpathcurveto{\pgfqpoint{3.563393in}{1.606450in}}{\pgfqpoint{3.567784in}{1.617049in}}{\pgfqpoint{3.567784in}{1.628099in}}%
\pgfpathcurveto{\pgfqpoint{3.567784in}{1.639149in}}{\pgfqpoint{3.563393in}{1.649749in}}{\pgfqpoint{3.555580in}{1.657562in}}%
\pgfpathcurveto{\pgfqpoint{3.547766in}{1.665376in}}{\pgfqpoint{3.537167in}{1.669766in}}{\pgfqpoint{3.526117in}{1.669766in}}%
\pgfpathcurveto{\pgfqpoint{3.515067in}{1.669766in}}{\pgfqpoint{3.504468in}{1.665376in}}{\pgfqpoint{3.496654in}{1.657562in}}%
\pgfpathcurveto{\pgfqpoint{3.488841in}{1.649749in}}{\pgfqpoint{3.484450in}{1.639149in}}{\pgfqpoint{3.484450in}{1.628099in}}%
\pgfpathcurveto{\pgfqpoint{3.484450in}{1.617049in}}{\pgfqpoint{3.488841in}{1.606450in}}{\pgfqpoint{3.496654in}{1.598637in}}%
\pgfpathcurveto{\pgfqpoint{3.504468in}{1.590823in}}{\pgfqpoint{3.515067in}{1.586433in}}{\pgfqpoint{3.526117in}{1.586433in}}%
\pgfpathclose%
\pgfusepath{stroke,fill}%
\end{pgfscope}%
\begin{pgfscope}%
\pgfpathrectangle{\pgfqpoint{0.574151in}{0.521603in}}{\pgfqpoint{3.720000in}{3.020000in}} %
\pgfusepath{clip}%
\pgfsetbuttcap%
\pgfsetroundjoin%
\definecolor{currentfill}{rgb}{1.000000,0.645928,0.343949}%
\pgfsetfillcolor{currentfill}%
\pgfsetlinewidth{1.003750pt}%
\definecolor{currentstroke}{rgb}{1.000000,0.645928,0.343949}%
\pgfsetstrokecolor{currentstroke}%
\pgfsetdash{}{0pt}%
\pgfpathmoveto{\pgfqpoint{3.302686in}{2.167561in}}%
\pgfpathcurveto{\pgfqpoint{3.313736in}{2.167561in}}{\pgfqpoint{3.324335in}{2.171951in}}{\pgfqpoint{3.332149in}{2.179765in}}%
\pgfpathcurveto{\pgfqpoint{3.339962in}{2.187578in}}{\pgfqpoint{3.344352in}{2.198177in}}{\pgfqpoint{3.344352in}{2.209228in}}%
\pgfpathcurveto{\pgfqpoint{3.344352in}{2.220278in}}{\pgfqpoint{3.339962in}{2.230877in}}{\pgfqpoint{3.332149in}{2.238690in}}%
\pgfpathcurveto{\pgfqpoint{3.324335in}{2.246504in}}{\pgfqpoint{3.313736in}{2.250894in}}{\pgfqpoint{3.302686in}{2.250894in}}%
\pgfpathcurveto{\pgfqpoint{3.291636in}{2.250894in}}{\pgfqpoint{3.281037in}{2.246504in}}{\pgfqpoint{3.273223in}{2.238690in}}%
\pgfpathcurveto{\pgfqpoint{3.265409in}{2.230877in}}{\pgfqpoint{3.261019in}{2.220278in}}{\pgfqpoint{3.261019in}{2.209228in}}%
\pgfpathcurveto{\pgfqpoint{3.261019in}{2.198177in}}{\pgfqpoint{3.265409in}{2.187578in}}{\pgfqpoint{3.273223in}{2.179765in}}%
\pgfpathcurveto{\pgfqpoint{3.281037in}{2.171951in}}{\pgfqpoint{3.291636in}{2.167561in}}{\pgfqpoint{3.302686in}{2.167561in}}%
\pgfpathclose%
\pgfusepath{stroke,fill}%
\end{pgfscope}%
\begin{pgfscope}%
\pgfpathrectangle{\pgfqpoint{0.574151in}{0.521603in}}{\pgfqpoint{3.720000in}{3.020000in}} %
\pgfusepath{clip}%
\pgfsetbuttcap%
\pgfsetroundjoin%
\definecolor{currentfill}{rgb}{0.739216,0.930229,0.562593}%
\pgfsetfillcolor{currentfill}%
\pgfsetlinewidth{1.003750pt}%
\definecolor{currentstroke}{rgb}{0.739216,0.930229,0.562593}%
\pgfsetstrokecolor{currentstroke}%
\pgfsetdash{}{0pt}%
\pgfpathmoveto{\pgfqpoint{3.115516in}{1.790323in}}%
\pgfpathcurveto{\pgfqpoint{3.126566in}{1.790323in}}{\pgfqpoint{3.137165in}{1.794713in}}{\pgfqpoint{3.144979in}{1.802527in}}%
\pgfpathcurveto{\pgfqpoint{3.152793in}{1.810341in}}{\pgfqpoint{3.157183in}{1.820940in}}{\pgfqpoint{3.157183in}{1.831990in}}%
\pgfpathcurveto{\pgfqpoint{3.157183in}{1.843040in}}{\pgfqpoint{3.152793in}{1.853639in}}{\pgfqpoint{3.144979in}{1.861453in}}%
\pgfpathcurveto{\pgfqpoint{3.137165in}{1.869266in}}{\pgfqpoint{3.126566in}{1.873657in}}{\pgfqpoint{3.115516in}{1.873657in}}%
\pgfpathcurveto{\pgfqpoint{3.104466in}{1.873657in}}{\pgfqpoint{3.093867in}{1.869266in}}{\pgfqpoint{3.086053in}{1.861453in}}%
\pgfpathcurveto{\pgfqpoint{3.078240in}{1.853639in}}{\pgfqpoint{3.073850in}{1.843040in}}{\pgfqpoint{3.073850in}{1.831990in}}%
\pgfpathcurveto{\pgfqpoint{3.073850in}{1.820940in}}{\pgfqpoint{3.078240in}{1.810341in}}{\pgfqpoint{3.086053in}{1.802527in}}%
\pgfpathcurveto{\pgfqpoint{3.093867in}{1.794713in}}{\pgfqpoint{3.104466in}{1.790323in}}{\pgfqpoint{3.115516in}{1.790323in}}%
\pgfpathclose%
\pgfusepath{stroke,fill}%
\end{pgfscope}%
\begin{pgfscope}%
\pgfpathrectangle{\pgfqpoint{0.574151in}{0.521603in}}{\pgfqpoint{3.720000in}{3.020000in}} %
\pgfusepath{clip}%
\pgfsetbuttcap%
\pgfsetroundjoin%
\definecolor{currentfill}{rgb}{0.500000,0.000000,1.000000}%
\pgfsetfillcolor{currentfill}%
\pgfsetlinewidth{1.003750pt}%
\definecolor{currentstroke}{rgb}{0.500000,0.000000,1.000000}%
\pgfsetstrokecolor{currentstroke}%
\pgfsetdash{}{0pt}%
\pgfpathmoveto{\pgfqpoint{3.074734in}{2.208860in}}%
\pgfpathcurveto{\pgfqpoint{3.085784in}{2.208860in}}{\pgfqpoint{3.096383in}{2.213251in}}{\pgfqpoint{3.104197in}{2.221064in}}%
\pgfpathcurveto{\pgfqpoint{3.112011in}{2.228878in}}{\pgfqpoint{3.116401in}{2.239477in}}{\pgfqpoint{3.116401in}{2.250527in}}%
\pgfpathcurveto{\pgfqpoint{3.116401in}{2.261577in}}{\pgfqpoint{3.112011in}{2.272176in}}{\pgfqpoint{3.104197in}{2.279990in}}%
\pgfpathcurveto{\pgfqpoint{3.096383in}{2.287803in}}{\pgfqpoint{3.085784in}{2.292194in}}{\pgfqpoint{3.074734in}{2.292194in}}%
\pgfpathcurveto{\pgfqpoint{3.063684in}{2.292194in}}{\pgfqpoint{3.053085in}{2.287803in}}{\pgfqpoint{3.045271in}{2.279990in}}%
\pgfpathcurveto{\pgfqpoint{3.037458in}{2.272176in}}{\pgfqpoint{3.033068in}{2.261577in}}{\pgfqpoint{3.033068in}{2.250527in}}%
\pgfpathcurveto{\pgfqpoint{3.033068in}{2.239477in}}{\pgfqpoint{3.037458in}{2.228878in}}{\pgfqpoint{3.045271in}{2.221064in}}%
\pgfpathcurveto{\pgfqpoint{3.053085in}{2.213251in}}{\pgfqpoint{3.063684in}{2.208860in}}{\pgfqpoint{3.074734in}{2.208860in}}%
\pgfpathclose%
\pgfusepath{stroke,fill}%
\end{pgfscope}%
\begin{pgfscope}%
\pgfpathrectangle{\pgfqpoint{0.574151in}{0.521603in}}{\pgfqpoint{3.720000in}{3.020000in}} %
\pgfusepath{clip}%
\pgfsetbuttcap%
\pgfsetroundjoin%
\definecolor{currentfill}{rgb}{0.319608,0.279583,0.989980}%
\pgfsetfillcolor{currentfill}%
\pgfsetlinewidth{1.003750pt}%
\definecolor{currentstroke}{rgb}{0.319608,0.279583,0.989980}%
\pgfsetstrokecolor{currentstroke}%
\pgfsetdash{}{0pt}%
\pgfpathmoveto{\pgfqpoint{2.630750in}{1.387021in}}%
\pgfpathcurveto{\pgfqpoint{2.641800in}{1.387021in}}{\pgfqpoint{2.652399in}{1.391411in}}{\pgfqpoint{2.660213in}{1.399225in}}%
\pgfpathcurveto{\pgfqpoint{2.668027in}{1.407039in}}{\pgfqpoint{2.672417in}{1.417638in}}{\pgfqpoint{2.672417in}{1.428688in}}%
\pgfpathcurveto{\pgfqpoint{2.672417in}{1.439738in}}{\pgfqpoint{2.668027in}{1.450337in}}{\pgfqpoint{2.660213in}{1.458150in}}%
\pgfpathcurveto{\pgfqpoint{2.652399in}{1.465964in}}{\pgfqpoint{2.641800in}{1.470354in}}{\pgfqpoint{2.630750in}{1.470354in}}%
\pgfpathcurveto{\pgfqpoint{2.619700in}{1.470354in}}{\pgfqpoint{2.609101in}{1.465964in}}{\pgfqpoint{2.601288in}{1.458150in}}%
\pgfpathcurveto{\pgfqpoint{2.593474in}{1.450337in}}{\pgfqpoint{2.589084in}{1.439738in}}{\pgfqpoint{2.589084in}{1.428688in}}%
\pgfpathcurveto{\pgfqpoint{2.589084in}{1.417638in}}{\pgfqpoint{2.593474in}{1.407039in}}{\pgfqpoint{2.601288in}{1.399225in}}%
\pgfpathcurveto{\pgfqpoint{2.609101in}{1.391411in}}{\pgfqpoint{2.619700in}{1.387021in}}{\pgfqpoint{2.630750in}{1.387021in}}%
\pgfpathclose%
\pgfusepath{stroke,fill}%
\end{pgfscope}%
\begin{pgfscope}%
\pgfpathrectangle{\pgfqpoint{0.574151in}{0.521603in}}{\pgfqpoint{3.720000in}{3.020000in}} %
\pgfusepath{clip}%
\pgfsetbuttcap%
\pgfsetroundjoin%
\definecolor{currentfill}{rgb}{0.319608,0.279583,0.989980}%
\pgfsetfillcolor{currentfill}%
\pgfsetlinewidth{1.003750pt}%
\definecolor{currentstroke}{rgb}{0.319608,0.279583,0.989980}%
\pgfsetstrokecolor{currentstroke}%
\pgfsetdash{}{0pt}%
\pgfpathmoveto{\pgfqpoint{2.181758in}{0.976101in}}%
\pgfpathcurveto{\pgfqpoint{2.192808in}{0.976101in}}{\pgfqpoint{2.203407in}{0.980492in}}{\pgfqpoint{2.211220in}{0.988305in}}%
\pgfpathcurveto{\pgfqpoint{2.219034in}{0.996119in}}{\pgfqpoint{2.223424in}{1.006718in}}{\pgfqpoint{2.223424in}{1.017768in}}%
\pgfpathcurveto{\pgfqpoint{2.223424in}{1.028818in}}{\pgfqpoint{2.219034in}{1.039417in}}{\pgfqpoint{2.211220in}{1.047231in}}%
\pgfpathcurveto{\pgfqpoint{2.203407in}{1.055044in}}{\pgfqpoint{2.192808in}{1.059435in}}{\pgfqpoint{2.181758in}{1.059435in}}%
\pgfpathcurveto{\pgfqpoint{2.170707in}{1.059435in}}{\pgfqpoint{2.160108in}{1.055044in}}{\pgfqpoint{2.152295in}{1.047231in}}%
\pgfpathcurveto{\pgfqpoint{2.144481in}{1.039417in}}{\pgfqpoint{2.140091in}{1.028818in}}{\pgfqpoint{2.140091in}{1.017768in}}%
\pgfpathcurveto{\pgfqpoint{2.140091in}{1.006718in}}{\pgfqpoint{2.144481in}{0.996119in}}{\pgfqpoint{2.152295in}{0.988305in}}%
\pgfpathcurveto{\pgfqpoint{2.160108in}{0.980492in}}{\pgfqpoint{2.170707in}{0.976101in}}{\pgfqpoint{2.181758in}{0.976101in}}%
\pgfpathclose%
\pgfusepath{stroke,fill}%
\end{pgfscope}%
\begin{pgfscope}%
\pgfpathrectangle{\pgfqpoint{0.574151in}{0.521603in}}{\pgfqpoint{3.720000in}{3.020000in}} %
\pgfusepath{clip}%
\pgfsetbuttcap%
\pgfsetroundjoin%
\definecolor{currentfill}{rgb}{1.000000,0.000000,0.000000}%
\pgfsetfillcolor{currentfill}%
\pgfsetlinewidth{1.003750pt}%
\definecolor{currentstroke}{rgb}{1.000000,0.000000,0.000000}%
\pgfsetstrokecolor{currentstroke}%
\pgfsetdash{}{0pt}%
\pgfpathmoveto{\pgfqpoint{4.121408in}{1.223734in}}%
\pgfpathcurveto{\pgfqpoint{4.132458in}{1.223734in}}{\pgfqpoint{4.143057in}{1.228125in}}{\pgfqpoint{4.150871in}{1.235938in}}%
\pgfpathcurveto{\pgfqpoint{4.158684in}{1.243752in}}{\pgfqpoint{4.163075in}{1.254351in}}{\pgfqpoint{4.163075in}{1.265401in}}%
\pgfpathcurveto{\pgfqpoint{4.163075in}{1.276451in}}{\pgfqpoint{4.158684in}{1.287050in}}{\pgfqpoint{4.150871in}{1.294864in}}%
\pgfpathcurveto{\pgfqpoint{4.143057in}{1.302677in}}{\pgfqpoint{4.132458in}{1.307068in}}{\pgfqpoint{4.121408in}{1.307068in}}%
\pgfpathcurveto{\pgfqpoint{4.110358in}{1.307068in}}{\pgfqpoint{4.099759in}{1.302677in}}{\pgfqpoint{4.091945in}{1.294864in}}%
\pgfpathcurveto{\pgfqpoint{4.084132in}{1.287050in}}{\pgfqpoint{4.079741in}{1.276451in}}{\pgfqpoint{4.079741in}{1.265401in}}%
\pgfpathcurveto{\pgfqpoint{4.079741in}{1.254351in}}{\pgfqpoint{4.084132in}{1.243752in}}{\pgfqpoint{4.091945in}{1.235938in}}%
\pgfpathcurveto{\pgfqpoint{4.099759in}{1.228125in}}{\pgfqpoint{4.110358in}{1.223734in}}{\pgfqpoint{4.121408in}{1.223734in}}%
\pgfpathclose%
\pgfusepath{stroke,fill}%
\end{pgfscope}%
\begin{pgfscope}%
\pgfpathrectangle{\pgfqpoint{0.574151in}{0.521603in}}{\pgfqpoint{3.720000in}{3.020000in}} %
\pgfusepath{clip}%
\pgfsetbuttcap%
\pgfsetroundjoin%
\definecolor{currentfill}{rgb}{1.000000,0.000000,0.000000}%
\pgfsetfillcolor{currentfill}%
\pgfsetlinewidth{1.003750pt}%
\definecolor{currentstroke}{rgb}{1.000000,0.000000,0.000000}%
\pgfsetstrokecolor{currentstroke}%
\pgfsetdash{}{0pt}%
\pgfpathmoveto{\pgfqpoint{2.883396in}{2.326348in}}%
\pgfpathcurveto{\pgfqpoint{2.894447in}{2.326348in}}{\pgfqpoint{2.905046in}{2.330738in}}{\pgfqpoint{2.912859in}{2.338551in}}%
\pgfpathcurveto{\pgfqpoint{2.920673in}{2.346365in}}{\pgfqpoint{2.925063in}{2.356964in}}{\pgfqpoint{2.925063in}{2.368014in}}%
\pgfpathcurveto{\pgfqpoint{2.925063in}{2.379064in}}{\pgfqpoint{2.920673in}{2.389663in}}{\pgfqpoint{2.912859in}{2.397477in}}%
\pgfpathcurveto{\pgfqpoint{2.905046in}{2.405291in}}{\pgfqpoint{2.894447in}{2.409681in}}{\pgfqpoint{2.883396in}{2.409681in}}%
\pgfpathcurveto{\pgfqpoint{2.872346in}{2.409681in}}{\pgfqpoint{2.861747in}{2.405291in}}{\pgfqpoint{2.853934in}{2.397477in}}%
\pgfpathcurveto{\pgfqpoint{2.846120in}{2.389663in}}{\pgfqpoint{2.841730in}{2.379064in}}{\pgfqpoint{2.841730in}{2.368014in}}%
\pgfpathcurveto{\pgfqpoint{2.841730in}{2.356964in}}{\pgfqpoint{2.846120in}{2.346365in}}{\pgfqpoint{2.853934in}{2.338551in}}%
\pgfpathcurveto{\pgfqpoint{2.861747in}{2.330738in}}{\pgfqpoint{2.872346in}{2.326348in}}{\pgfqpoint{2.883396in}{2.326348in}}%
\pgfpathclose%
\pgfusepath{stroke,fill}%
\end{pgfscope}%
\begin{pgfscope}%
\pgfpathrectangle{\pgfqpoint{0.574151in}{0.521603in}}{\pgfqpoint{3.720000in}{3.020000in}} %
\pgfusepath{clip}%
\pgfsetbuttcap%
\pgfsetroundjoin%
\definecolor{currentfill}{rgb}{0.484314,0.024637,0.999924}%
\pgfsetfillcolor{currentfill}%
\pgfsetlinewidth{1.003750pt}%
\definecolor{currentstroke}{rgb}{0.484314,0.024637,0.999924}%
\pgfsetstrokecolor{currentstroke}%
\pgfsetdash{}{0pt}%
\pgfpathmoveto{\pgfqpoint{2.989684in}{2.167561in}}%
\pgfpathcurveto{\pgfqpoint{3.000734in}{2.167561in}}{\pgfqpoint{3.011333in}{2.171951in}}{\pgfqpoint{3.019147in}{2.179765in}}%
\pgfpathcurveto{\pgfqpoint{3.026961in}{2.187578in}}{\pgfqpoint{3.031351in}{2.198177in}}{\pgfqpoint{3.031351in}{2.209228in}}%
\pgfpathcurveto{\pgfqpoint{3.031351in}{2.220278in}}{\pgfqpoint{3.026961in}{2.230877in}}{\pgfqpoint{3.019147in}{2.238690in}}%
\pgfpathcurveto{\pgfqpoint{3.011333in}{2.246504in}}{\pgfqpoint{3.000734in}{2.250894in}}{\pgfqpoint{2.989684in}{2.250894in}}%
\pgfpathcurveto{\pgfqpoint{2.978634in}{2.250894in}}{\pgfqpoint{2.968035in}{2.246504in}}{\pgfqpoint{2.960221in}{2.238690in}}%
\pgfpathcurveto{\pgfqpoint{2.952408in}{2.230877in}}{\pgfqpoint{2.948017in}{2.220278in}}{\pgfqpoint{2.948017in}{2.209228in}}%
\pgfpathcurveto{\pgfqpoint{2.948017in}{2.198177in}}{\pgfqpoint{2.952408in}{2.187578in}}{\pgfqpoint{2.960221in}{2.179765in}}%
\pgfpathcurveto{\pgfqpoint{2.968035in}{2.171951in}}{\pgfqpoint{2.978634in}{2.167561in}}{\pgfqpoint{2.989684in}{2.167561in}}%
\pgfpathclose%
\pgfusepath{stroke,fill}%
\end{pgfscope}%
\begin{pgfscope}%
\pgfpathrectangle{\pgfqpoint{0.574151in}{0.521603in}}{\pgfqpoint{3.720000in}{3.020000in}} %
\pgfusepath{clip}%
\pgfsetbuttcap%
\pgfsetroundjoin%
\definecolor{currentfill}{rgb}{0.484314,0.024637,0.999924}%
\pgfsetfillcolor{currentfill}%
\pgfsetlinewidth{1.003750pt}%
\definecolor{currentstroke}{rgb}{0.484314,0.024637,0.999924}%
\pgfsetstrokecolor{currentstroke}%
\pgfsetdash{}{0pt}%
\pgfpathmoveto{\pgfqpoint{2.724122in}{2.167561in}}%
\pgfpathcurveto{\pgfqpoint{2.735172in}{2.167561in}}{\pgfqpoint{2.745771in}{2.171951in}}{\pgfqpoint{2.753585in}{2.179765in}}%
\pgfpathcurveto{\pgfqpoint{2.761398in}{2.187578in}}{\pgfqpoint{2.765788in}{2.198177in}}{\pgfqpoint{2.765788in}{2.209228in}}%
\pgfpathcurveto{\pgfqpoint{2.765788in}{2.220278in}}{\pgfqpoint{2.761398in}{2.230877in}}{\pgfqpoint{2.753585in}{2.238690in}}%
\pgfpathcurveto{\pgfqpoint{2.745771in}{2.246504in}}{\pgfqpoint{2.735172in}{2.250894in}}{\pgfqpoint{2.724122in}{2.250894in}}%
\pgfpathcurveto{\pgfqpoint{2.713072in}{2.250894in}}{\pgfqpoint{2.702473in}{2.246504in}}{\pgfqpoint{2.694659in}{2.238690in}}%
\pgfpathcurveto{\pgfqpoint{2.686845in}{2.230877in}}{\pgfqpoint{2.682455in}{2.220278in}}{\pgfqpoint{2.682455in}{2.209228in}}%
\pgfpathcurveto{\pgfqpoint{2.682455in}{2.198177in}}{\pgfqpoint{2.686845in}{2.187578in}}{\pgfqpoint{2.694659in}{2.179765in}}%
\pgfpathcurveto{\pgfqpoint{2.702473in}{2.171951in}}{\pgfqpoint{2.713072in}{2.167561in}}{\pgfqpoint{2.724122in}{2.167561in}}%
\pgfpathclose%
\pgfusepath{stroke,fill}%
\end{pgfscope}%
\begin{pgfscope}%
\pgfpathrectangle{\pgfqpoint{0.574151in}{0.521603in}}{\pgfqpoint{3.720000in}{3.020000in}} %
\pgfusepath{clip}%
\pgfsetbuttcap%
\pgfsetroundjoin%
\definecolor{currentfill}{rgb}{0.484314,0.024637,0.999924}%
\pgfsetfillcolor{currentfill}%
\pgfsetlinewidth{1.003750pt}%
\definecolor{currentstroke}{rgb}{0.484314,0.024637,0.999924}%
\pgfsetstrokecolor{currentstroke}%
\pgfsetdash{}{0pt}%
\pgfpathmoveto{\pgfqpoint{2.567617in}{2.167561in}}%
\pgfpathcurveto{\pgfqpoint{2.578667in}{2.167561in}}{\pgfqpoint{2.589266in}{2.171951in}}{\pgfqpoint{2.597080in}{2.179765in}}%
\pgfpathcurveto{\pgfqpoint{2.604893in}{2.187578in}}{\pgfqpoint{2.609284in}{2.198177in}}{\pgfqpoint{2.609284in}{2.209228in}}%
\pgfpathcurveto{\pgfqpoint{2.609284in}{2.220278in}}{\pgfqpoint{2.604893in}{2.230877in}}{\pgfqpoint{2.597080in}{2.238690in}}%
\pgfpathcurveto{\pgfqpoint{2.589266in}{2.246504in}}{\pgfqpoint{2.578667in}{2.250894in}}{\pgfqpoint{2.567617in}{2.250894in}}%
\pgfpathcurveto{\pgfqpoint{2.556567in}{2.250894in}}{\pgfqpoint{2.545968in}{2.246504in}}{\pgfqpoint{2.538154in}{2.238690in}}%
\pgfpathcurveto{\pgfqpoint{2.530341in}{2.230877in}}{\pgfqpoint{2.525950in}{2.220278in}}{\pgfqpoint{2.525950in}{2.209228in}}%
\pgfpathcurveto{\pgfqpoint{2.525950in}{2.198177in}}{\pgfqpoint{2.530341in}{2.187578in}}{\pgfqpoint{2.538154in}{2.179765in}}%
\pgfpathcurveto{\pgfqpoint{2.545968in}{2.171951in}}{\pgfqpoint{2.556567in}{2.167561in}}{\pgfqpoint{2.567617in}{2.167561in}}%
\pgfpathclose%
\pgfusepath{stroke,fill}%
\end{pgfscope}%
\begin{pgfscope}%
\pgfpathrectangle{\pgfqpoint{0.574151in}{0.521603in}}{\pgfqpoint{3.720000in}{3.020000in}} %
\pgfusepath{clip}%
\pgfsetbuttcap%
\pgfsetroundjoin%
\definecolor{currentfill}{rgb}{0.484314,0.024637,0.999924}%
\pgfsetfillcolor{currentfill}%
\pgfsetlinewidth{1.003750pt}%
\definecolor{currentstroke}{rgb}{0.484314,0.024637,0.999924}%
\pgfsetstrokecolor{currentstroke}%
\pgfsetdash{}{0pt}%
\pgfpathmoveto{\pgfqpoint{2.421665in}{2.167561in}}%
\pgfpathcurveto{\pgfqpoint{2.432715in}{2.167561in}}{\pgfqpoint{2.443314in}{2.171951in}}{\pgfqpoint{2.451128in}{2.179765in}}%
\pgfpathcurveto{\pgfqpoint{2.458941in}{2.187578in}}{\pgfqpoint{2.463331in}{2.198177in}}{\pgfqpoint{2.463331in}{2.209228in}}%
\pgfpathcurveto{\pgfqpoint{2.463331in}{2.220278in}}{\pgfqpoint{2.458941in}{2.230877in}}{\pgfqpoint{2.451128in}{2.238690in}}%
\pgfpathcurveto{\pgfqpoint{2.443314in}{2.246504in}}{\pgfqpoint{2.432715in}{2.250894in}}{\pgfqpoint{2.421665in}{2.250894in}}%
\pgfpathcurveto{\pgfqpoint{2.410615in}{2.250894in}}{\pgfqpoint{2.400016in}{2.246504in}}{\pgfqpoint{2.392202in}{2.238690in}}%
\pgfpathcurveto{\pgfqpoint{2.384388in}{2.230877in}}{\pgfqpoint{2.379998in}{2.220278in}}{\pgfqpoint{2.379998in}{2.209228in}}%
\pgfpathcurveto{\pgfqpoint{2.379998in}{2.198177in}}{\pgfqpoint{2.384388in}{2.187578in}}{\pgfqpoint{2.392202in}{2.179765in}}%
\pgfpathcurveto{\pgfqpoint{2.400016in}{2.171951in}}{\pgfqpoint{2.410615in}{2.167561in}}{\pgfqpoint{2.421665in}{2.167561in}}%
\pgfpathclose%
\pgfusepath{stroke,fill}%
\end{pgfscope}%
\begin{pgfscope}%
\pgfpathrectangle{\pgfqpoint{0.574151in}{0.521603in}}{\pgfqpoint{3.720000in}{3.020000in}} %
\pgfusepath{clip}%
\pgfsetbuttcap%
\pgfsetroundjoin%
\definecolor{currentfill}{rgb}{0.484314,0.024637,0.999924}%
\pgfsetfillcolor{currentfill}%
\pgfsetlinewidth{1.003750pt}%
\definecolor{currentstroke}{rgb}{0.484314,0.024637,0.999924}%
\pgfsetstrokecolor{currentstroke}%
\pgfsetdash{}{0pt}%
\pgfpathmoveto{\pgfqpoint{2.018634in}{1.586433in}}%
\pgfpathcurveto{\pgfqpoint{2.029684in}{1.586433in}}{\pgfqpoint{2.040283in}{1.590823in}}{\pgfqpoint{2.048097in}{1.598637in}}%
\pgfpathcurveto{\pgfqpoint{2.055911in}{1.606450in}}{\pgfqpoint{2.060301in}{1.617049in}}{\pgfqpoint{2.060301in}{1.628099in}}%
\pgfpathcurveto{\pgfqpoint{2.060301in}{1.639149in}}{\pgfqpoint{2.055911in}{1.649749in}}{\pgfqpoint{2.048097in}{1.657562in}}%
\pgfpathcurveto{\pgfqpoint{2.040283in}{1.665376in}}{\pgfqpoint{2.029684in}{1.669766in}}{\pgfqpoint{2.018634in}{1.669766in}}%
\pgfpathcurveto{\pgfqpoint{2.007584in}{1.669766in}}{\pgfqpoint{1.996985in}{1.665376in}}{\pgfqpoint{1.989171in}{1.657562in}}%
\pgfpathcurveto{\pgfqpoint{1.981358in}{1.649749in}}{\pgfqpoint{1.976967in}{1.639149in}}{\pgfqpoint{1.976967in}{1.628099in}}%
\pgfpathcurveto{\pgfqpoint{1.976967in}{1.617049in}}{\pgfqpoint{1.981358in}{1.606450in}}{\pgfqpoint{1.989171in}{1.598637in}}%
\pgfpathcurveto{\pgfqpoint{1.996985in}{1.590823in}}{\pgfqpoint{2.007584in}{1.586433in}}{\pgfqpoint{2.018634in}{1.586433in}}%
\pgfpathclose%
\pgfusepath{stroke,fill}%
\end{pgfscope}%
\begin{pgfscope}%
\pgfsetbuttcap%
\pgfsetroundjoin%
\definecolor{currentfill}{rgb}{0.000000,0.000000,0.000000}%
\pgfsetfillcolor{currentfill}%
\pgfsetlinewidth{0.803000pt}%
\definecolor{currentstroke}{rgb}{0.000000,0.000000,0.000000}%
\pgfsetstrokecolor{currentstroke}%
\pgfsetdash{}{0pt}%
\pgfsys@defobject{currentmarker}{\pgfqpoint{0.000000in}{-0.048611in}}{\pgfqpoint{0.000000in}{0.000000in}}{%
\pgfpathmoveto{\pgfqpoint{0.000000in}{0.000000in}}%
\pgfpathlineto{\pgfqpoint{0.000000in}{-0.048611in}}%
\pgfusepath{stroke,fill}%
}%
\begin{pgfscope}%
\pgfsys@transformshift{0.743242in}{0.521603in}%
\pgfsys@useobject{currentmarker}{}%
\end{pgfscope}%
\end{pgfscope}%
\begin{pgfscope}%
\pgftext[x=0.743242in,y=0.424381in,,top]{\rmfamily\fontsize{10.000000}{12.000000}\selectfont \(\displaystyle 10^{-2}\)}%
\end{pgfscope}%
\begin{pgfscope}%
\pgfsetbuttcap%
\pgfsetroundjoin%
\definecolor{currentfill}{rgb}{0.000000,0.000000,0.000000}%
\pgfsetfillcolor{currentfill}%
\pgfsetlinewidth{0.803000pt}%
\definecolor{currentstroke}{rgb}{0.000000,0.000000,0.000000}%
\pgfsetstrokecolor{currentstroke}%
\pgfsetdash{}{0pt}%
\pgfsys@defobject{currentmarker}{\pgfqpoint{0.000000in}{-0.048611in}}{\pgfqpoint{0.000000in}{0.000000in}}{%
\pgfpathmoveto{\pgfqpoint{0.000000in}{0.000000in}}%
\pgfpathlineto{\pgfqpoint{0.000000in}{-0.048611in}}%
\pgfusepath{stroke,fill}%
}%
\begin{pgfscope}%
\pgfsys@transformshift{2.358141in}{0.521603in}%
\pgfsys@useobject{currentmarker}{}%
\end{pgfscope}%
\end{pgfscope}%
\begin{pgfscope}%
\pgftext[x=2.358141in,y=0.424381in,,top]{\rmfamily\fontsize{10.000000}{12.000000}\selectfont \(\displaystyle 10^{-1}\)}%
\end{pgfscope}%
\begin{pgfscope}%
\pgfsetbuttcap%
\pgfsetroundjoin%
\definecolor{currentfill}{rgb}{0.000000,0.000000,0.000000}%
\pgfsetfillcolor{currentfill}%
\pgfsetlinewidth{0.803000pt}%
\definecolor{currentstroke}{rgb}{0.000000,0.000000,0.000000}%
\pgfsetstrokecolor{currentstroke}%
\pgfsetdash{}{0pt}%
\pgfsys@defobject{currentmarker}{\pgfqpoint{0.000000in}{-0.048611in}}{\pgfqpoint{0.000000in}{0.000000in}}{%
\pgfpathmoveto{\pgfqpoint{0.000000in}{0.000000in}}%
\pgfpathlineto{\pgfqpoint{0.000000in}{-0.048611in}}%
\pgfusepath{stroke,fill}%
}%
\begin{pgfscope}%
\pgfsys@transformshift{3.973040in}{0.521603in}%
\pgfsys@useobject{currentmarker}{}%
\end{pgfscope}%
\end{pgfscope}%
\begin{pgfscope}%
\pgftext[x=3.973040in,y=0.424381in,,top]{\rmfamily\fontsize{10.000000}{12.000000}\selectfont \(\displaystyle 10^{0}\)}%
\end{pgfscope}%
\begin{pgfscope}%
\pgfsetbuttcap%
\pgfsetroundjoin%
\definecolor{currentfill}{rgb}{0.000000,0.000000,0.000000}%
\pgfsetfillcolor{currentfill}%
\pgfsetlinewidth{0.602250pt}%
\definecolor{currentstroke}{rgb}{0.000000,0.000000,0.000000}%
\pgfsetstrokecolor{currentstroke}%
\pgfsetdash{}{0pt}%
\pgfsys@defobject{currentmarker}{\pgfqpoint{0.000000in}{-0.027778in}}{\pgfqpoint{0.000000in}{0.000000in}}{%
\pgfpathmoveto{\pgfqpoint{0.000000in}{0.000000in}}%
\pgfpathlineto{\pgfqpoint{0.000000in}{-0.027778in}}%
\pgfusepath{stroke,fill}%
}%
\begin{pgfscope}%
\pgfsys@transformshift{0.586742in}{0.521603in}%
\pgfsys@useobject{currentmarker}{}%
\end{pgfscope}%
\end{pgfscope}%
\begin{pgfscope}%
\pgfsetbuttcap%
\pgfsetroundjoin%
\definecolor{currentfill}{rgb}{0.000000,0.000000,0.000000}%
\pgfsetfillcolor{currentfill}%
\pgfsetlinewidth{0.602250pt}%
\definecolor{currentstroke}{rgb}{0.000000,0.000000,0.000000}%
\pgfsetstrokecolor{currentstroke}%
\pgfsetdash{}{0pt}%
\pgfsys@defobject{currentmarker}{\pgfqpoint{0.000000in}{-0.027778in}}{\pgfqpoint{0.000000in}{0.000000in}}{%
\pgfpathmoveto{\pgfqpoint{0.000000in}{0.000000in}}%
\pgfpathlineto{\pgfqpoint{0.000000in}{-0.027778in}}%
\pgfusepath{stroke,fill}%
}%
\begin{pgfscope}%
\pgfsys@transformshift{0.669349in}{0.521603in}%
\pgfsys@useobject{currentmarker}{}%
\end{pgfscope}%
\end{pgfscope}%
\begin{pgfscope}%
\pgfsetbuttcap%
\pgfsetroundjoin%
\definecolor{currentfill}{rgb}{0.000000,0.000000,0.000000}%
\pgfsetfillcolor{currentfill}%
\pgfsetlinewidth{0.602250pt}%
\definecolor{currentstroke}{rgb}{0.000000,0.000000,0.000000}%
\pgfsetstrokecolor{currentstroke}%
\pgfsetdash{}{0pt}%
\pgfsys@defobject{currentmarker}{\pgfqpoint{0.000000in}{-0.027778in}}{\pgfqpoint{0.000000in}{0.000000in}}{%
\pgfpathmoveto{\pgfqpoint{0.000000in}{0.000000in}}%
\pgfpathlineto{\pgfqpoint{0.000000in}{-0.027778in}}%
\pgfusepath{stroke,fill}%
}%
\begin{pgfscope}%
\pgfsys@transformshift{1.229375in}{0.521603in}%
\pgfsys@useobject{currentmarker}{}%
\end{pgfscope}%
\end{pgfscope}%
\begin{pgfscope}%
\pgfsetbuttcap%
\pgfsetroundjoin%
\definecolor{currentfill}{rgb}{0.000000,0.000000,0.000000}%
\pgfsetfillcolor{currentfill}%
\pgfsetlinewidth{0.602250pt}%
\definecolor{currentstroke}{rgb}{0.000000,0.000000,0.000000}%
\pgfsetstrokecolor{currentstroke}%
\pgfsetdash{}{0pt}%
\pgfsys@defobject{currentmarker}{\pgfqpoint{0.000000in}{-0.027778in}}{\pgfqpoint{0.000000in}{0.000000in}}{%
\pgfpathmoveto{\pgfqpoint{0.000000in}{0.000000in}}%
\pgfpathlineto{\pgfqpoint{0.000000in}{-0.027778in}}%
\pgfusepath{stroke,fill}%
}%
\begin{pgfscope}%
\pgfsys@transformshift{1.513745in}{0.521603in}%
\pgfsys@useobject{currentmarker}{}%
\end{pgfscope}%
\end{pgfscope}%
\begin{pgfscope}%
\pgfsetbuttcap%
\pgfsetroundjoin%
\definecolor{currentfill}{rgb}{0.000000,0.000000,0.000000}%
\pgfsetfillcolor{currentfill}%
\pgfsetlinewidth{0.602250pt}%
\definecolor{currentstroke}{rgb}{0.000000,0.000000,0.000000}%
\pgfsetstrokecolor{currentstroke}%
\pgfsetdash{}{0pt}%
\pgfsys@defobject{currentmarker}{\pgfqpoint{0.000000in}{-0.027778in}}{\pgfqpoint{0.000000in}{0.000000in}}{%
\pgfpathmoveto{\pgfqpoint{0.000000in}{0.000000in}}%
\pgfpathlineto{\pgfqpoint{0.000000in}{-0.027778in}}%
\pgfusepath{stroke,fill}%
}%
\begin{pgfscope}%
\pgfsys@transformshift{1.715508in}{0.521603in}%
\pgfsys@useobject{currentmarker}{}%
\end{pgfscope}%
\end{pgfscope}%
\begin{pgfscope}%
\pgfsetbuttcap%
\pgfsetroundjoin%
\definecolor{currentfill}{rgb}{0.000000,0.000000,0.000000}%
\pgfsetfillcolor{currentfill}%
\pgfsetlinewidth{0.602250pt}%
\definecolor{currentstroke}{rgb}{0.000000,0.000000,0.000000}%
\pgfsetstrokecolor{currentstroke}%
\pgfsetdash{}{0pt}%
\pgfsys@defobject{currentmarker}{\pgfqpoint{0.000000in}{-0.027778in}}{\pgfqpoint{0.000000in}{0.000000in}}{%
\pgfpathmoveto{\pgfqpoint{0.000000in}{0.000000in}}%
\pgfpathlineto{\pgfqpoint{0.000000in}{-0.027778in}}%
\pgfusepath{stroke,fill}%
}%
\begin{pgfscope}%
\pgfsys@transformshift{1.872008in}{0.521603in}%
\pgfsys@useobject{currentmarker}{}%
\end{pgfscope}%
\end{pgfscope}%
\begin{pgfscope}%
\pgfsetbuttcap%
\pgfsetroundjoin%
\definecolor{currentfill}{rgb}{0.000000,0.000000,0.000000}%
\pgfsetfillcolor{currentfill}%
\pgfsetlinewidth{0.602250pt}%
\definecolor{currentstroke}{rgb}{0.000000,0.000000,0.000000}%
\pgfsetstrokecolor{currentstroke}%
\pgfsetdash{}{0pt}%
\pgfsys@defobject{currentmarker}{\pgfqpoint{0.000000in}{-0.027778in}}{\pgfqpoint{0.000000in}{0.000000in}}{%
\pgfpathmoveto{\pgfqpoint{0.000000in}{0.000000in}}%
\pgfpathlineto{\pgfqpoint{0.000000in}{-0.027778in}}%
\pgfusepath{stroke,fill}%
}%
\begin{pgfscope}%
\pgfsys@transformshift{1.999878in}{0.521603in}%
\pgfsys@useobject{currentmarker}{}%
\end{pgfscope}%
\end{pgfscope}%
\begin{pgfscope}%
\pgfsetbuttcap%
\pgfsetroundjoin%
\definecolor{currentfill}{rgb}{0.000000,0.000000,0.000000}%
\pgfsetfillcolor{currentfill}%
\pgfsetlinewidth{0.602250pt}%
\definecolor{currentstroke}{rgb}{0.000000,0.000000,0.000000}%
\pgfsetstrokecolor{currentstroke}%
\pgfsetdash{}{0pt}%
\pgfsys@defobject{currentmarker}{\pgfqpoint{0.000000in}{-0.027778in}}{\pgfqpoint{0.000000in}{0.000000in}}{%
\pgfpathmoveto{\pgfqpoint{0.000000in}{0.000000in}}%
\pgfpathlineto{\pgfqpoint{0.000000in}{-0.027778in}}%
\pgfusepath{stroke,fill}%
}%
\begin{pgfscope}%
\pgfsys@transformshift{2.107990in}{0.521603in}%
\pgfsys@useobject{currentmarker}{}%
\end{pgfscope}%
\end{pgfscope}%
\begin{pgfscope}%
\pgfsetbuttcap%
\pgfsetroundjoin%
\definecolor{currentfill}{rgb}{0.000000,0.000000,0.000000}%
\pgfsetfillcolor{currentfill}%
\pgfsetlinewidth{0.602250pt}%
\definecolor{currentstroke}{rgb}{0.000000,0.000000,0.000000}%
\pgfsetstrokecolor{currentstroke}%
\pgfsetdash{}{0pt}%
\pgfsys@defobject{currentmarker}{\pgfqpoint{0.000000in}{-0.027778in}}{\pgfqpoint{0.000000in}{0.000000in}}{%
\pgfpathmoveto{\pgfqpoint{0.000000in}{0.000000in}}%
\pgfpathlineto{\pgfqpoint{0.000000in}{-0.027778in}}%
\pgfusepath{stroke,fill}%
}%
\begin{pgfscope}%
\pgfsys@transformshift{2.201641in}{0.521603in}%
\pgfsys@useobject{currentmarker}{}%
\end{pgfscope}%
\end{pgfscope}%
\begin{pgfscope}%
\pgfsetbuttcap%
\pgfsetroundjoin%
\definecolor{currentfill}{rgb}{0.000000,0.000000,0.000000}%
\pgfsetfillcolor{currentfill}%
\pgfsetlinewidth{0.602250pt}%
\definecolor{currentstroke}{rgb}{0.000000,0.000000,0.000000}%
\pgfsetstrokecolor{currentstroke}%
\pgfsetdash{}{0pt}%
\pgfsys@defobject{currentmarker}{\pgfqpoint{0.000000in}{-0.027778in}}{\pgfqpoint{0.000000in}{0.000000in}}{%
\pgfpathmoveto{\pgfqpoint{0.000000in}{0.000000in}}%
\pgfpathlineto{\pgfqpoint{0.000000in}{-0.027778in}}%
\pgfusepath{stroke,fill}%
}%
\begin{pgfscope}%
\pgfsys@transformshift{2.284247in}{0.521603in}%
\pgfsys@useobject{currentmarker}{}%
\end{pgfscope}%
\end{pgfscope}%
\begin{pgfscope}%
\pgfsetbuttcap%
\pgfsetroundjoin%
\definecolor{currentfill}{rgb}{0.000000,0.000000,0.000000}%
\pgfsetfillcolor{currentfill}%
\pgfsetlinewidth{0.602250pt}%
\definecolor{currentstroke}{rgb}{0.000000,0.000000,0.000000}%
\pgfsetstrokecolor{currentstroke}%
\pgfsetdash{}{0pt}%
\pgfsys@defobject{currentmarker}{\pgfqpoint{0.000000in}{-0.027778in}}{\pgfqpoint{0.000000in}{0.000000in}}{%
\pgfpathmoveto{\pgfqpoint{0.000000in}{0.000000in}}%
\pgfpathlineto{\pgfqpoint{0.000000in}{-0.027778in}}%
\pgfusepath{stroke,fill}%
}%
\begin{pgfscope}%
\pgfsys@transformshift{2.844274in}{0.521603in}%
\pgfsys@useobject{currentmarker}{}%
\end{pgfscope}%
\end{pgfscope}%
\begin{pgfscope}%
\pgfsetbuttcap%
\pgfsetroundjoin%
\definecolor{currentfill}{rgb}{0.000000,0.000000,0.000000}%
\pgfsetfillcolor{currentfill}%
\pgfsetlinewidth{0.602250pt}%
\definecolor{currentstroke}{rgb}{0.000000,0.000000,0.000000}%
\pgfsetstrokecolor{currentstroke}%
\pgfsetdash{}{0pt}%
\pgfsys@defobject{currentmarker}{\pgfqpoint{0.000000in}{-0.027778in}}{\pgfqpoint{0.000000in}{0.000000in}}{%
\pgfpathmoveto{\pgfqpoint{0.000000in}{0.000000in}}%
\pgfpathlineto{\pgfqpoint{0.000000in}{-0.027778in}}%
\pgfusepath{stroke,fill}%
}%
\begin{pgfscope}%
\pgfsys@transformshift{3.128644in}{0.521603in}%
\pgfsys@useobject{currentmarker}{}%
\end{pgfscope}%
\end{pgfscope}%
\begin{pgfscope}%
\pgfsetbuttcap%
\pgfsetroundjoin%
\definecolor{currentfill}{rgb}{0.000000,0.000000,0.000000}%
\pgfsetfillcolor{currentfill}%
\pgfsetlinewidth{0.602250pt}%
\definecolor{currentstroke}{rgb}{0.000000,0.000000,0.000000}%
\pgfsetstrokecolor{currentstroke}%
\pgfsetdash{}{0pt}%
\pgfsys@defobject{currentmarker}{\pgfqpoint{0.000000in}{-0.027778in}}{\pgfqpoint{0.000000in}{0.000000in}}{%
\pgfpathmoveto{\pgfqpoint{0.000000in}{0.000000in}}%
\pgfpathlineto{\pgfqpoint{0.000000in}{-0.027778in}}%
\pgfusepath{stroke,fill}%
}%
\begin{pgfscope}%
\pgfsys@transformshift{3.330407in}{0.521603in}%
\pgfsys@useobject{currentmarker}{}%
\end{pgfscope}%
\end{pgfscope}%
\begin{pgfscope}%
\pgfsetbuttcap%
\pgfsetroundjoin%
\definecolor{currentfill}{rgb}{0.000000,0.000000,0.000000}%
\pgfsetfillcolor{currentfill}%
\pgfsetlinewidth{0.602250pt}%
\definecolor{currentstroke}{rgb}{0.000000,0.000000,0.000000}%
\pgfsetstrokecolor{currentstroke}%
\pgfsetdash{}{0pt}%
\pgfsys@defobject{currentmarker}{\pgfqpoint{0.000000in}{-0.027778in}}{\pgfqpoint{0.000000in}{0.000000in}}{%
\pgfpathmoveto{\pgfqpoint{0.000000in}{0.000000in}}%
\pgfpathlineto{\pgfqpoint{0.000000in}{-0.027778in}}%
\pgfusepath{stroke,fill}%
}%
\begin{pgfscope}%
\pgfsys@transformshift{3.486907in}{0.521603in}%
\pgfsys@useobject{currentmarker}{}%
\end{pgfscope}%
\end{pgfscope}%
\begin{pgfscope}%
\pgfsetbuttcap%
\pgfsetroundjoin%
\definecolor{currentfill}{rgb}{0.000000,0.000000,0.000000}%
\pgfsetfillcolor{currentfill}%
\pgfsetlinewidth{0.602250pt}%
\definecolor{currentstroke}{rgb}{0.000000,0.000000,0.000000}%
\pgfsetstrokecolor{currentstroke}%
\pgfsetdash{}{0pt}%
\pgfsys@defobject{currentmarker}{\pgfqpoint{0.000000in}{-0.027778in}}{\pgfqpoint{0.000000in}{0.000000in}}{%
\pgfpathmoveto{\pgfqpoint{0.000000in}{0.000000in}}%
\pgfpathlineto{\pgfqpoint{0.000000in}{-0.027778in}}%
\pgfusepath{stroke,fill}%
}%
\begin{pgfscope}%
\pgfsys@transformshift{3.614777in}{0.521603in}%
\pgfsys@useobject{currentmarker}{}%
\end{pgfscope}%
\end{pgfscope}%
\begin{pgfscope}%
\pgfsetbuttcap%
\pgfsetroundjoin%
\definecolor{currentfill}{rgb}{0.000000,0.000000,0.000000}%
\pgfsetfillcolor{currentfill}%
\pgfsetlinewidth{0.602250pt}%
\definecolor{currentstroke}{rgb}{0.000000,0.000000,0.000000}%
\pgfsetstrokecolor{currentstroke}%
\pgfsetdash{}{0pt}%
\pgfsys@defobject{currentmarker}{\pgfqpoint{0.000000in}{-0.027778in}}{\pgfqpoint{0.000000in}{0.000000in}}{%
\pgfpathmoveto{\pgfqpoint{0.000000in}{0.000000in}}%
\pgfpathlineto{\pgfqpoint{0.000000in}{-0.027778in}}%
\pgfusepath{stroke,fill}%
}%
\begin{pgfscope}%
\pgfsys@transformshift{3.722889in}{0.521603in}%
\pgfsys@useobject{currentmarker}{}%
\end{pgfscope}%
\end{pgfscope}%
\begin{pgfscope}%
\pgfsetbuttcap%
\pgfsetroundjoin%
\definecolor{currentfill}{rgb}{0.000000,0.000000,0.000000}%
\pgfsetfillcolor{currentfill}%
\pgfsetlinewidth{0.602250pt}%
\definecolor{currentstroke}{rgb}{0.000000,0.000000,0.000000}%
\pgfsetstrokecolor{currentstroke}%
\pgfsetdash{}{0pt}%
\pgfsys@defobject{currentmarker}{\pgfqpoint{0.000000in}{-0.027778in}}{\pgfqpoint{0.000000in}{0.000000in}}{%
\pgfpathmoveto{\pgfqpoint{0.000000in}{0.000000in}}%
\pgfpathlineto{\pgfqpoint{0.000000in}{-0.027778in}}%
\pgfusepath{stroke,fill}%
}%
\begin{pgfscope}%
\pgfsys@transformshift{3.816540in}{0.521603in}%
\pgfsys@useobject{currentmarker}{}%
\end{pgfscope}%
\end{pgfscope}%
\begin{pgfscope}%
\pgfsetbuttcap%
\pgfsetroundjoin%
\definecolor{currentfill}{rgb}{0.000000,0.000000,0.000000}%
\pgfsetfillcolor{currentfill}%
\pgfsetlinewidth{0.602250pt}%
\definecolor{currentstroke}{rgb}{0.000000,0.000000,0.000000}%
\pgfsetstrokecolor{currentstroke}%
\pgfsetdash{}{0pt}%
\pgfsys@defobject{currentmarker}{\pgfqpoint{0.000000in}{-0.027778in}}{\pgfqpoint{0.000000in}{0.000000in}}{%
\pgfpathmoveto{\pgfqpoint{0.000000in}{0.000000in}}%
\pgfpathlineto{\pgfqpoint{0.000000in}{-0.027778in}}%
\pgfusepath{stroke,fill}%
}%
\begin{pgfscope}%
\pgfsys@transformshift{3.899146in}{0.521603in}%
\pgfsys@useobject{currentmarker}{}%
\end{pgfscope}%
\end{pgfscope}%
\begin{pgfscope}%
\pgftext[x=2.434151in,y=0.234413in,,top]{\rmfamily\fontsize{10.000000}{12.000000}\selectfont \(\displaystyle We\)}%
\end{pgfscope}%
\begin{pgfscope}%
\pgfsetbuttcap%
\pgfsetroundjoin%
\definecolor{currentfill}{rgb}{0.000000,0.000000,0.000000}%
\pgfsetfillcolor{currentfill}%
\pgfsetlinewidth{0.803000pt}%
\definecolor{currentstroke}{rgb}{0.000000,0.000000,0.000000}%
\pgfsetstrokecolor{currentstroke}%
\pgfsetdash{}{0pt}%
\pgfsys@defobject{currentmarker}{\pgfqpoint{-0.048611in}{0.000000in}}{\pgfqpoint{0.000000in}{0.000000in}}{%
\pgfpathmoveto{\pgfqpoint{0.000000in}{0.000000in}}%
\pgfpathlineto{\pgfqpoint{-0.048611in}{0.000000in}}%
\pgfusepath{stroke,fill}%
}%
\begin{pgfscope}%
\pgfsys@transformshift{0.574151in}{1.002178in}%
\pgfsys@useobject{currentmarker}{}%
\end{pgfscope}%
\end{pgfscope}%
\begin{pgfscope}%
\pgftext[x=0.299459in,y=0.949416in,left,base]{\rmfamily\fontsize{10.000000}{12.000000}\selectfont \(\displaystyle 2.5\)}%
\end{pgfscope}%
\begin{pgfscope}%
\pgfsetbuttcap%
\pgfsetroundjoin%
\definecolor{currentfill}{rgb}{0.000000,0.000000,0.000000}%
\pgfsetfillcolor{currentfill}%
\pgfsetlinewidth{0.803000pt}%
\definecolor{currentstroke}{rgb}{0.000000,0.000000,0.000000}%
\pgfsetstrokecolor{currentstroke}%
\pgfsetdash{}{0pt}%
\pgfsys@defobject{currentmarker}{\pgfqpoint{-0.048611in}{0.000000in}}{\pgfqpoint{0.000000in}{0.000000in}}{%
\pgfpathmoveto{\pgfqpoint{0.000000in}{0.000000in}}%
\pgfpathlineto{\pgfqpoint{-0.048611in}{0.000000in}}%
\pgfusepath{stroke,fill}%
}%
\begin{pgfscope}%
\pgfsys@transformshift{0.574151in}{1.574347in}%
\pgfsys@useobject{currentmarker}{}%
\end{pgfscope}%
\end{pgfscope}%
\begin{pgfscope}%
\pgftext[x=0.299459in,y=1.521586in,left,base]{\rmfamily\fontsize{10.000000}{12.000000}\selectfont \(\displaystyle 3.0\)}%
\end{pgfscope}%
\begin{pgfscope}%
\pgfsetbuttcap%
\pgfsetroundjoin%
\definecolor{currentfill}{rgb}{0.000000,0.000000,0.000000}%
\pgfsetfillcolor{currentfill}%
\pgfsetlinewidth{0.803000pt}%
\definecolor{currentstroke}{rgb}{0.000000,0.000000,0.000000}%
\pgfsetstrokecolor{currentstroke}%
\pgfsetdash{}{0pt}%
\pgfsys@defobject{currentmarker}{\pgfqpoint{-0.048611in}{0.000000in}}{\pgfqpoint{0.000000in}{0.000000in}}{%
\pgfpathmoveto{\pgfqpoint{0.000000in}{0.000000in}}%
\pgfpathlineto{\pgfqpoint{-0.048611in}{0.000000in}}%
\pgfusepath{stroke,fill}%
}%
\begin{pgfscope}%
\pgfsys@transformshift{0.574151in}{2.146517in}%
\pgfsys@useobject{currentmarker}{}%
\end{pgfscope}%
\end{pgfscope}%
\begin{pgfscope}%
\pgftext[x=0.299459in,y=2.093755in,left,base]{\rmfamily\fontsize{10.000000}{12.000000}\selectfont \(\displaystyle 3.5\)}%
\end{pgfscope}%
\begin{pgfscope}%
\pgfsetbuttcap%
\pgfsetroundjoin%
\definecolor{currentfill}{rgb}{0.000000,0.000000,0.000000}%
\pgfsetfillcolor{currentfill}%
\pgfsetlinewidth{0.803000pt}%
\definecolor{currentstroke}{rgb}{0.000000,0.000000,0.000000}%
\pgfsetstrokecolor{currentstroke}%
\pgfsetdash{}{0pt}%
\pgfsys@defobject{currentmarker}{\pgfqpoint{-0.048611in}{0.000000in}}{\pgfqpoint{0.000000in}{0.000000in}}{%
\pgfpathmoveto{\pgfqpoint{0.000000in}{0.000000in}}%
\pgfpathlineto{\pgfqpoint{-0.048611in}{0.000000in}}%
\pgfusepath{stroke,fill}%
}%
\begin{pgfscope}%
\pgfsys@transformshift{0.574151in}{2.718686in}%
\pgfsys@useobject{currentmarker}{}%
\end{pgfscope}%
\end{pgfscope}%
\begin{pgfscope}%
\pgftext[x=0.299459in,y=2.665925in,left,base]{\rmfamily\fontsize{10.000000}{12.000000}\selectfont \(\displaystyle 4.0\)}%
\end{pgfscope}%
\begin{pgfscope}%
\pgfsetbuttcap%
\pgfsetroundjoin%
\definecolor{currentfill}{rgb}{0.000000,0.000000,0.000000}%
\pgfsetfillcolor{currentfill}%
\pgfsetlinewidth{0.803000pt}%
\definecolor{currentstroke}{rgb}{0.000000,0.000000,0.000000}%
\pgfsetstrokecolor{currentstroke}%
\pgfsetdash{}{0pt}%
\pgfsys@defobject{currentmarker}{\pgfqpoint{-0.048611in}{0.000000in}}{\pgfqpoint{0.000000in}{0.000000in}}{%
\pgfpathmoveto{\pgfqpoint{0.000000in}{0.000000in}}%
\pgfpathlineto{\pgfqpoint{-0.048611in}{0.000000in}}%
\pgfusepath{stroke,fill}%
}%
\begin{pgfscope}%
\pgfsys@transformshift{0.574151in}{3.290856in}%
\pgfsys@useobject{currentmarker}{}%
\end{pgfscope}%
\end{pgfscope}%
\begin{pgfscope}%
\pgftext[x=0.299459in,y=3.238094in,left,base]{\rmfamily\fontsize{10.000000}{12.000000}\selectfont \(\displaystyle 4.5\)}%
\end{pgfscope}%
\begin{pgfscope}%
\pgftext[x=0.243904in,y=2.031603in,,bottom,rotate=90.000000]{\rmfamily\fontsize{10.000000}{12.000000}\selectfont \(\displaystyle t_j/ \tau\)}%
\end{pgfscope}%
\begin{pgfscope}%
\pgfpathrectangle{\pgfqpoint{0.574151in}{0.521603in}}{\pgfqpoint{3.720000in}{3.020000in}} %
\pgfusepath{clip}%
\pgfsetrectcap%
\pgfsetroundjoin%
\pgfsetlinewidth{1.505625pt}%
\definecolor{currentstroke}{rgb}{0.121569,0.466667,0.705882}%
\pgfsetstrokecolor{currentstroke}%
\pgfsetdash{}{0pt}%
\pgfpathmoveto{\pgfqpoint{0.743242in}{0.658876in}}%
\pgfpathlineto{\pgfqpoint{1.229375in}{0.658876in}}%
\pgfpathlineto{\pgfqpoint{1.513745in}{0.658876in}}%
\pgfpathlineto{\pgfqpoint{1.715508in}{0.658876in}}%
\pgfpathlineto{\pgfqpoint{1.872008in}{0.658876in}}%
\pgfpathlineto{\pgfqpoint{1.999878in}{0.658876in}}%
\pgfpathlineto{\pgfqpoint{2.107990in}{0.658876in}}%
\pgfpathlineto{\pgfqpoint{2.201641in}{0.658876in}}%
\pgfpathlineto{\pgfqpoint{2.284247in}{0.658876in}}%
\pgfpathlineto{\pgfqpoint{2.358141in}{0.658876in}}%
\pgfpathlineto{\pgfqpoint{2.424986in}{0.658876in}}%
\pgfpathlineto{\pgfqpoint{2.486011in}{0.658876in}}%
\pgfpathlineto{\pgfqpoint{2.542148in}{0.658876in}}%
\pgfpathlineto{\pgfqpoint{2.594123in}{0.658876in}}%
\pgfpathlineto{\pgfqpoint{2.642511in}{0.658876in}}%
\pgfpathlineto{\pgfqpoint{2.687774in}{0.658876in}}%
\pgfpathlineto{\pgfqpoint{2.730293in}{0.658876in}}%
\pgfpathlineto{\pgfqpoint{2.770380in}{0.658876in}}%
\pgfpathlineto{\pgfqpoint{2.808300in}{0.658876in}}%
\pgfpathlineto{\pgfqpoint{2.844274in}{0.658876in}}%
\pgfpathlineto{\pgfqpoint{2.878493in}{0.658876in}}%
\pgfpathlineto{\pgfqpoint{2.911119in}{0.658876in}}%
\pgfpathlineto{\pgfqpoint{2.942295in}{0.658876in}}%
\pgfpathlineto{\pgfqpoint{2.972144in}{0.658876in}}%
\pgfpathlineto{\pgfqpoint{3.000774in}{0.658876in}}%
\pgfpathlineto{\pgfqpoint{3.028281in}{0.658876in}}%
\pgfpathlineto{\pgfqpoint{3.054750in}{0.658876in}}%
\pgfpathlineto{\pgfqpoint{3.080256in}{0.658876in}}%
\pgfpathlineto{\pgfqpoint{3.104867in}{0.658876in}}%
\pgfpathlineto{\pgfqpoint{3.128644in}{0.658876in}}%
\pgfpathlineto{\pgfqpoint{3.151641in}{0.658876in}}%
\pgfpathlineto{\pgfqpoint{3.173907in}{0.658876in}}%
\pgfpathlineto{\pgfqpoint{3.195489in}{0.658876in}}%
\pgfpathlineto{\pgfqpoint{3.216426in}{0.658876in}}%
\pgfpathlineto{\pgfqpoint{3.236756in}{0.658876in}}%
\pgfpathlineto{\pgfqpoint{3.256513in}{0.658876in}}%
\pgfpathlineto{\pgfqpoint{3.275729in}{0.658876in}}%
\pgfpathlineto{\pgfqpoint{3.294433in}{0.658876in}}%
\pgfpathlineto{\pgfqpoint{3.312651in}{0.658876in}}%
\pgfpathlineto{\pgfqpoint{3.330407in}{0.658876in}}%
\pgfpathlineto{\pgfqpoint{3.347725in}{0.658876in}}%
\pgfpathlineto{\pgfqpoint{3.364626in}{0.658876in}}%
\pgfpathlineto{\pgfqpoint{3.381129in}{0.658876in}}%
\pgfpathlineto{\pgfqpoint{3.397252in}{0.658876in}}%
\pgfpathlineto{\pgfqpoint{3.413013in}{0.658876in}}%
\pgfpathlineto{\pgfqpoint{3.428428in}{0.658876in}}%
\pgfpathlineto{\pgfqpoint{3.443511in}{0.658876in}}%
\pgfpathlineto{\pgfqpoint{3.458277in}{0.658876in}}%
\pgfpathlineto{\pgfqpoint{3.472738in}{0.658876in}}%
\pgfpathlineto{\pgfqpoint{3.486907in}{0.658876in}}%
\pgfpathlineto{\pgfqpoint{3.500795in}{0.658876in}}%
\pgfpathlineto{\pgfqpoint{3.514414in}{0.658876in}}%
\pgfpathlineto{\pgfqpoint{3.527773in}{0.658876in}}%
\pgfpathlineto{\pgfqpoint{3.540883in}{0.658876in}}%
\pgfpathlineto{\pgfqpoint{3.553752in}{0.658876in}}%
\pgfpathlineto{\pgfqpoint{3.566389in}{0.658876in}}%
\pgfpathlineto{\pgfqpoint{3.578803in}{0.658876in}}%
\pgfpathlineto{\pgfqpoint{3.591000in}{0.658876in}}%
\pgfpathlineto{\pgfqpoint{3.602989in}{0.658876in}}%
\pgfpathlineto{\pgfqpoint{3.614777in}{0.658876in}}%
\pgfpathlineto{\pgfqpoint{3.626369in}{0.658876in}}%
\pgfpathlineto{\pgfqpoint{3.637774in}{0.658876in}}%
\pgfpathlineto{\pgfqpoint{3.648995in}{0.658876in}}%
\pgfpathlineto{\pgfqpoint{3.660040in}{0.658876in}}%
\pgfpathlineto{\pgfqpoint{3.670914in}{0.658876in}}%
\pgfpathlineto{\pgfqpoint{3.681622in}{0.658876in}}%
\pgfpathlineto{\pgfqpoint{3.692168in}{0.658876in}}%
\pgfpathlineto{\pgfqpoint{3.702559in}{0.658876in}}%
\pgfpathlineto{\pgfqpoint{3.712798in}{0.658876in}}%
\pgfpathlineto{\pgfqpoint{3.722889in}{0.658876in}}%
\pgfpathlineto{\pgfqpoint{3.732837in}{0.658876in}}%
\pgfpathlineto{\pgfqpoint{3.742646in}{0.658876in}}%
\pgfpathlineto{\pgfqpoint{3.752320in}{0.658876in}}%
\pgfpathlineto{\pgfqpoint{3.761862in}{0.658876in}}%
\pgfpathlineto{\pgfqpoint{3.771277in}{0.658876in}}%
\pgfpathlineto{\pgfqpoint{3.780566in}{0.658876in}}%
\pgfpathlineto{\pgfqpoint{3.789734in}{0.658876in}}%
\pgfpathlineto{\pgfqpoint{3.798784in}{0.658876in}}%
\pgfpathlineto{\pgfqpoint{3.807718in}{0.658876in}}%
\pgfpathlineto{\pgfqpoint{3.816540in}{0.658876in}}%
\pgfpathlineto{\pgfqpoint{3.825253in}{0.658876in}}%
\pgfpathlineto{\pgfqpoint{3.833858in}{0.658876in}}%
\pgfpathlineto{\pgfqpoint{3.842359in}{0.658876in}}%
\pgfpathlineto{\pgfqpoint{3.850759in}{0.658876in}}%
\pgfpathlineto{\pgfqpoint{3.859059in}{0.658876in}}%
\pgfpathlineto{\pgfqpoint{3.867262in}{0.658876in}}%
\pgfpathlineto{\pgfqpoint{3.875370in}{0.658876in}}%
\pgfpathlineto{\pgfqpoint{3.883385in}{0.658876in}}%
\pgfpathlineto{\pgfqpoint{3.891310in}{0.658876in}}%
\pgfpathlineto{\pgfqpoint{3.899146in}{0.658876in}}%
\pgfpathlineto{\pgfqpoint{3.906896in}{0.658876in}}%
\pgfpathlineto{\pgfqpoint{3.914561in}{0.658876in}}%
\pgfpathlineto{\pgfqpoint{3.922143in}{0.658876in}}%
\pgfpathlineto{\pgfqpoint{3.929644in}{0.658876in}}%
\pgfpathlineto{\pgfqpoint{3.937066in}{0.658876in}}%
\pgfpathlineto{\pgfqpoint{3.944410in}{0.658876in}}%
\pgfpathlineto{\pgfqpoint{3.951678in}{0.658876in}}%
\pgfpathlineto{\pgfqpoint{3.958871in}{0.658876in}}%
\pgfpathlineto{\pgfqpoint{3.965991in}{0.658876in}}%
\pgfpathlineto{\pgfqpoint{3.973040in}{0.658876in}}%
\pgfusepath{stroke}%
\end{pgfscope}%
\begin{pgfscope}%
\pgfsetrectcap%
\pgfsetmiterjoin%
\pgfsetlinewidth{0.803000pt}%
\definecolor{currentstroke}{rgb}{0.000000,0.000000,0.000000}%
\pgfsetstrokecolor{currentstroke}%
\pgfsetdash{}{0pt}%
\pgfpathmoveto{\pgfqpoint{0.574151in}{0.521603in}}%
\pgfpathlineto{\pgfqpoint{0.574151in}{3.541603in}}%
\pgfusepath{stroke}%
\end{pgfscope}%
\begin{pgfscope}%
\pgfsetrectcap%
\pgfsetmiterjoin%
\pgfsetlinewidth{0.803000pt}%
\definecolor{currentstroke}{rgb}{0.000000,0.000000,0.000000}%
\pgfsetstrokecolor{currentstroke}%
\pgfsetdash{}{0pt}%
\pgfpathmoveto{\pgfqpoint{4.294151in}{0.521603in}}%
\pgfpathlineto{\pgfqpoint{4.294151in}{3.541603in}}%
\pgfusepath{stroke}%
\end{pgfscope}%
\begin{pgfscope}%
\pgfsetrectcap%
\pgfsetmiterjoin%
\pgfsetlinewidth{0.803000pt}%
\definecolor{currentstroke}{rgb}{0.000000,0.000000,0.000000}%
\pgfsetstrokecolor{currentstroke}%
\pgfsetdash{}{0pt}%
\pgfpathmoveto{\pgfqpoint{0.574151in}{0.521603in}}%
\pgfpathlineto{\pgfqpoint{4.294151in}{0.521603in}}%
\pgfusepath{stroke}%
\end{pgfscope}%
\begin{pgfscope}%
\pgfsetrectcap%
\pgfsetmiterjoin%
\pgfsetlinewidth{0.803000pt}%
\definecolor{currentstroke}{rgb}{0.000000,0.000000,0.000000}%
\pgfsetstrokecolor{currentstroke}%
\pgfsetdash{}{0pt}%
\pgfpathmoveto{\pgfqpoint{0.574151in}{3.541603in}}%
\pgfpathlineto{\pgfqpoint{4.294151in}{3.541603in}}%
\pgfusepath{stroke}%
\end{pgfscope}%
\begin{pgfscope}%
\pgfsetbuttcap%
\pgfsetmiterjoin%
\definecolor{currentfill}{rgb}{1.000000,1.000000,1.000000}%
\pgfsetfillcolor{currentfill}%
\pgfsetfillopacity{0.800000}%
\pgfsetlinewidth{1.003750pt}%
\definecolor{currentstroke}{rgb}{0.800000,0.800000,0.800000}%
\pgfsetstrokecolor{currentstroke}%
\pgfsetstrokeopacity{0.800000}%
\pgfsetdash{}{0pt}%
\pgfpathmoveto{\pgfqpoint{2.729264in}{1.908841in}}%
\pgfpathlineto{\pgfqpoint{4.196929in}{1.908841in}}%
\pgfpathquadraticcurveto{\pgfqpoint{4.224707in}{1.908841in}}{\pgfqpoint{4.224707in}{1.936619in}}%
\pgfpathlineto{\pgfqpoint{4.224707in}{2.126587in}}%
\pgfpathquadraticcurveto{\pgfqpoint{4.224707in}{2.154365in}}{\pgfqpoint{4.196929in}{2.154365in}}%
\pgfpathlineto{\pgfqpoint{2.729264in}{2.154365in}}%
\pgfpathquadraticcurveto{\pgfqpoint{2.701486in}{2.154365in}}{\pgfqpoint{2.701486in}{2.126587in}}%
\pgfpathlineto{\pgfqpoint{2.701486in}{1.936619in}}%
\pgfpathquadraticcurveto{\pgfqpoint{2.701486in}{1.908841in}}{\pgfqpoint{2.729264in}{1.908841in}}%
\pgfpathclose%
\pgfusepath{stroke,fill}%
\end{pgfscope}%
\begin{pgfscope}%
\pgfsetrectcap%
\pgfsetroundjoin%
\pgfsetlinewidth{1.505625pt}%
\definecolor{currentstroke}{rgb}{0.121569,0.466667,0.705882}%
\pgfsetstrokecolor{currentstroke}%
\pgfsetdash{}{0pt}%
\pgfpathmoveto{\pgfqpoint{2.757042in}{2.041898in}}%
\pgfpathlineto{\pgfqpoint{3.034820in}{2.041898in}}%
\pgfusepath{stroke}%
\end{pgfscope}%
\begin{pgfscope}%
\pgftext[x=3.145931in,y=1.993287in,left,base]{\rmfamily\fontsize{10.000000}{12.000000}\selectfont Richards 2001}%
\end{pgfscope}%
\begin{pgfscope}%
\pgfpathrectangle{\pgfqpoint{4.526651in}{0.521603in}}{\pgfqpoint{0.151000in}{3.020000in}} %
\pgfusepath{clip}%
\pgfsetbuttcap%
\pgfsetmiterjoin%
\definecolor{currentfill}{rgb}{1.000000,1.000000,1.000000}%
\pgfsetfillcolor{currentfill}%
\pgfsetlinewidth{0.010037pt}%
\definecolor{currentstroke}{rgb}{1.000000,1.000000,1.000000}%
\pgfsetstrokecolor{currentstroke}%
\pgfsetdash{}{0pt}%
\pgfpathmoveto{\pgfqpoint{4.526651in}{0.521603in}}%
\pgfpathlineto{\pgfqpoint{4.526651in}{0.533400in}}%
\pgfpathlineto{\pgfqpoint{4.526651in}{3.529806in}}%
\pgfpathlineto{\pgfqpoint{4.526651in}{3.541603in}}%
\pgfpathlineto{\pgfqpoint{4.677651in}{3.541603in}}%
\pgfpathlineto{\pgfqpoint{4.677651in}{3.529806in}}%
\pgfpathlineto{\pgfqpoint{4.677651in}{0.533400in}}%
\pgfpathlineto{\pgfqpoint{4.677651in}{0.521603in}}%
\pgfpathclose%
\pgfusepath{stroke,fill}%
\end{pgfscope}%
\begin{pgfscope}%
\pgfsys@transformshift{4.530000in}{0.526603in}%
\pgftext[left,bottom]{\pgfimage[interpolate=true,width=0.150000in,height=3.020000in]{contact-img0.png}}%
\end{pgfscope}%
\begin{pgfscope}%
\pgfsetbuttcap%
\pgfsetroundjoin%
\definecolor{currentfill}{rgb}{0.000000,0.000000,0.000000}%
\pgfsetfillcolor{currentfill}%
\pgfsetlinewidth{0.803000pt}%
\definecolor{currentstroke}{rgb}{0.000000,0.000000,0.000000}%
\pgfsetstrokecolor{currentstroke}%
\pgfsetdash{}{0pt}%
\pgfsys@defobject{currentmarker}{\pgfqpoint{0.000000in}{0.000000in}}{\pgfqpoint{0.048611in}{0.000000in}}{%
\pgfpathmoveto{\pgfqpoint{0.000000in}{0.000000in}}%
\pgfpathlineto{\pgfqpoint{0.048611in}{0.000000in}}%
\pgfusepath{stroke,fill}%
}%
\begin{pgfscope}%
\pgfsys@transformshift{4.677651in}{0.874491in}%
\pgfsys@useobject{currentmarker}{}%
\end{pgfscope}%
\end{pgfscope}%
\begin{pgfscope}%
\pgftext[x=4.774874in,y=0.821729in,left,base]{\rmfamily\fontsize{10.000000}{12.000000}\selectfont \(\displaystyle 0.1\)}%
\end{pgfscope}%
\begin{pgfscope}%
\pgfsetbuttcap%
\pgfsetroundjoin%
\definecolor{currentfill}{rgb}{0.000000,0.000000,0.000000}%
\pgfsetfillcolor{currentfill}%
\pgfsetlinewidth{0.803000pt}%
\definecolor{currentstroke}{rgb}{0.000000,0.000000,0.000000}%
\pgfsetstrokecolor{currentstroke}%
\pgfsetdash{}{0pt}%
\pgfsys@defobject{currentmarker}{\pgfqpoint{0.000000in}{0.000000in}}{\pgfqpoint{0.048611in}{0.000000in}}{%
\pgfpathmoveto{\pgfqpoint{0.000000in}{0.000000in}}%
\pgfpathlineto{\pgfqpoint{0.048611in}{0.000000in}}%
\pgfusepath{stroke,fill}%
}%
\begin{pgfscope}%
\pgfsys@transformshift{4.677651in}{1.358178in}%
\pgfsys@useobject{currentmarker}{}%
\end{pgfscope}%
\end{pgfscope}%
\begin{pgfscope}%
\pgftext[x=4.774874in,y=1.305417in,left,base]{\rmfamily\fontsize{10.000000}{12.000000}\selectfont \(\displaystyle 0.2\)}%
\end{pgfscope}%
\begin{pgfscope}%
\pgfsetbuttcap%
\pgfsetroundjoin%
\definecolor{currentfill}{rgb}{0.000000,0.000000,0.000000}%
\pgfsetfillcolor{currentfill}%
\pgfsetlinewidth{0.803000pt}%
\definecolor{currentstroke}{rgb}{0.000000,0.000000,0.000000}%
\pgfsetstrokecolor{currentstroke}%
\pgfsetdash{}{0pt}%
\pgfsys@defobject{currentmarker}{\pgfqpoint{0.000000in}{0.000000in}}{\pgfqpoint{0.048611in}{0.000000in}}{%
\pgfpathmoveto{\pgfqpoint{0.000000in}{0.000000in}}%
\pgfpathlineto{\pgfqpoint{0.048611in}{0.000000in}}%
\pgfusepath{stroke,fill}%
}%
\begin{pgfscope}%
\pgfsys@transformshift{4.677651in}{1.841866in}%
\pgfsys@useobject{currentmarker}{}%
\end{pgfscope}%
\end{pgfscope}%
\begin{pgfscope}%
\pgftext[x=4.774874in,y=1.789104in,left,base]{\rmfamily\fontsize{10.000000}{12.000000}\selectfont \(\displaystyle 0.3\)}%
\end{pgfscope}%
\begin{pgfscope}%
\pgfsetbuttcap%
\pgfsetroundjoin%
\definecolor{currentfill}{rgb}{0.000000,0.000000,0.000000}%
\pgfsetfillcolor{currentfill}%
\pgfsetlinewidth{0.803000pt}%
\definecolor{currentstroke}{rgb}{0.000000,0.000000,0.000000}%
\pgfsetstrokecolor{currentstroke}%
\pgfsetdash{}{0pt}%
\pgfsys@defobject{currentmarker}{\pgfqpoint{0.000000in}{0.000000in}}{\pgfqpoint{0.048611in}{0.000000in}}{%
\pgfpathmoveto{\pgfqpoint{0.000000in}{0.000000in}}%
\pgfpathlineto{\pgfqpoint{0.048611in}{0.000000in}}%
\pgfusepath{stroke,fill}%
}%
\begin{pgfscope}%
\pgfsys@transformshift{4.677651in}{2.325553in}%
\pgfsys@useobject{currentmarker}{}%
\end{pgfscope}%
\end{pgfscope}%
\begin{pgfscope}%
\pgftext[x=4.774874in,y=2.272791in,left,base]{\rmfamily\fontsize{10.000000}{12.000000}\selectfont \(\displaystyle 0.4\)}%
\end{pgfscope}%
\begin{pgfscope}%
\pgfsetbuttcap%
\pgfsetroundjoin%
\definecolor{currentfill}{rgb}{0.000000,0.000000,0.000000}%
\pgfsetfillcolor{currentfill}%
\pgfsetlinewidth{0.803000pt}%
\definecolor{currentstroke}{rgb}{0.000000,0.000000,0.000000}%
\pgfsetstrokecolor{currentstroke}%
\pgfsetdash{}{0pt}%
\pgfsys@defobject{currentmarker}{\pgfqpoint{0.000000in}{0.000000in}}{\pgfqpoint{0.048611in}{0.000000in}}{%
\pgfpathmoveto{\pgfqpoint{0.000000in}{0.000000in}}%
\pgfpathlineto{\pgfqpoint{0.048611in}{0.000000in}}%
\pgfusepath{stroke,fill}%
}%
\begin{pgfscope}%
\pgfsys@transformshift{4.677651in}{2.809240in}%
\pgfsys@useobject{currentmarker}{}%
\end{pgfscope}%
\end{pgfscope}%
\begin{pgfscope}%
\pgftext[x=4.774874in,y=2.756479in,left,base]{\rmfamily\fontsize{10.000000}{12.000000}\selectfont \(\displaystyle 0.5\)}%
\end{pgfscope}%
\begin{pgfscope}%
\pgfsetbuttcap%
\pgfsetroundjoin%
\definecolor{currentfill}{rgb}{0.000000,0.000000,0.000000}%
\pgfsetfillcolor{currentfill}%
\pgfsetlinewidth{0.803000pt}%
\definecolor{currentstroke}{rgb}{0.000000,0.000000,0.000000}%
\pgfsetstrokecolor{currentstroke}%
\pgfsetdash{}{0pt}%
\pgfsys@defobject{currentmarker}{\pgfqpoint{0.000000in}{0.000000in}}{\pgfqpoint{0.048611in}{0.000000in}}{%
\pgfpathmoveto{\pgfqpoint{0.000000in}{0.000000in}}%
\pgfpathlineto{\pgfqpoint{0.048611in}{0.000000in}}%
\pgfusepath{stroke,fill}%
}%
\begin{pgfscope}%
\pgfsys@transformshift{4.677651in}{3.292928in}%
\pgfsys@useobject{currentmarker}{}%
\end{pgfscope}%
\end{pgfscope}%
\begin{pgfscope}%
\pgftext[x=4.774874in,y=3.240166in,left,base]{\rmfamily\fontsize{10.000000}{12.000000}\selectfont \(\displaystyle 0.6\)}%
\end{pgfscope}%
\begin{pgfscope}%
\pgftext[x=5.007899in,y=2.031603in,,top,rotate=90.000000]{\rmfamily\fontsize{10.000000}{12.000000}\selectfont \(\displaystyle \mathrm{\mathit{Bo_e}} \equiv \frac{\epsilon E_0^2 R_0}{\gamma}\)}%
\end{pgfscope}%
\begin{pgfscope}%
\pgfsetbuttcap%
\pgfsetmiterjoin%
\pgfsetlinewidth{0.803000pt}%
\definecolor{currentstroke}{rgb}{0.000000,0.000000,0.000000}%
\pgfsetstrokecolor{currentstroke}%
\pgfsetdash{}{0pt}%
\pgfpathmoveto{\pgfqpoint{4.526651in}{0.521603in}}%
\pgfpathlineto{\pgfqpoint{4.526651in}{0.533400in}}%
\pgfpathlineto{\pgfqpoint{4.526651in}{3.529806in}}%
\pgfpathlineto{\pgfqpoint{4.526651in}{3.541603in}}%
\pgfpathlineto{\pgfqpoint{4.677651in}{3.541603in}}%
\pgfpathlineto{\pgfqpoint{4.677651in}{3.529806in}}%
\pgfpathlineto{\pgfqpoint{4.677651in}{0.533400in}}%
\pgfpathlineto{\pgfqpoint{4.677651in}{0.521603in}}%
\pgfpathclose%
\pgfusepath{stroke}%
\end{pgfscope}%
\end{pgfpicture}%
\makeatother%
\endgroup%

    \caption{.\label{fig:contact}}
\end{figure}

\begin{figure}[htb]
    \centering
    %% Creator: Matplotlib, PGF backend
%%
%% To include the figure in your LaTeX document, write
%%   \input{<filename>.pgf}
%%
%% Make sure the required packages are loaded in your preamble
%%   \usepackage{pgf}
%%
%% Figures using additional raster images can only be included by \input if
%% they are in the same directory as the main LaTeX file. For loading figures
%% from other directories you can use the `import` package
%%   \usepackage{import}
%% and then include the figures with
%%   \import{<path to file>}{<filename>.pgf}
%%
%% Matplotlib used the following preamble
%%   \usepackage{fontspec}
%%   \setmainfont{DejaVu Serif}
%%   \setsansfont{DejaVu Sans}
%%   \setmonofont{DejaVu Sans Mono}
%%
\begingroup%
\makeatletter%
\begin{pgfpicture}%
\pgfpathrectangle{\pgfpointorigin}{\pgfqpoint{5.427700in}{3.676603in}}%
\pgfusepath{use as bounding box, clip}%
\begin{pgfscope}%
\pgfsetbuttcap%
\pgfsetmiterjoin%
\definecolor{currentfill}{rgb}{1.000000,1.000000,1.000000}%
\pgfsetfillcolor{currentfill}%
\pgfsetlinewidth{0.000000pt}%
\definecolor{currentstroke}{rgb}{1.000000,1.000000,1.000000}%
\pgfsetstrokecolor{currentstroke}%
\pgfsetdash{}{0pt}%
\pgfpathmoveto{\pgfqpoint{0.000000in}{0.000000in}}%
\pgfpathlineto{\pgfqpoint{5.427700in}{0.000000in}}%
\pgfpathlineto{\pgfqpoint{5.427700in}{3.676603in}}%
\pgfpathlineto{\pgfqpoint{0.000000in}{3.676603in}}%
\pgfpathclose%
\pgfusepath{fill}%
\end{pgfscope}%
\begin{pgfscope}%
\pgfsetbuttcap%
\pgfsetmiterjoin%
\definecolor{currentfill}{rgb}{1.000000,1.000000,1.000000}%
\pgfsetfillcolor{currentfill}%
\pgfsetlinewidth{0.000000pt}%
\definecolor{currentstroke}{rgb}{0.000000,0.000000,0.000000}%
\pgfsetstrokecolor{currentstroke}%
\pgfsetstrokeopacity{0.000000}%
\pgfsetdash{}{0pt}%
\pgfpathmoveto{\pgfqpoint{0.564660in}{0.521603in}}%
\pgfpathlineto{\pgfqpoint{4.284660in}{0.521603in}}%
\pgfpathlineto{\pgfqpoint{4.284660in}{3.541603in}}%
\pgfpathlineto{\pgfqpoint{0.564660in}{3.541603in}}%
\pgfpathclose%
\pgfusepath{fill}%
\end{pgfscope}%
\begin{pgfscope}%
\pgfpathrectangle{\pgfqpoint{0.564660in}{0.521603in}}{\pgfqpoint{3.720000in}{3.020000in}} %
\pgfusepath{clip}%
\pgfsetbuttcap%
\pgfsetroundjoin%
\definecolor{currentfill}{rgb}{0.264706,0.361242,0.982973}%
\pgfsetfillcolor{currentfill}%
\pgfsetlinewidth{1.003750pt}%
\definecolor{currentstroke}{rgb}{0.264706,0.361242,0.982973}%
\pgfsetstrokecolor{currentstroke}%
\pgfsetdash{}{0pt}%
\pgfpathmoveto{\pgfqpoint{3.328555in}{1.872073in}}%
\pgfpathcurveto{\pgfqpoint{3.339606in}{1.872073in}}{\pgfqpoint{3.350205in}{1.876464in}}{\pgfqpoint{3.358018in}{1.884277in}}%
\pgfpathcurveto{\pgfqpoint{3.365832in}{1.892091in}}{\pgfqpoint{3.370222in}{1.902690in}}{\pgfqpoint{3.370222in}{1.913740in}}%
\pgfpathcurveto{\pgfqpoint{3.370222in}{1.924790in}}{\pgfqpoint{3.365832in}{1.935389in}}{\pgfqpoint{3.358018in}{1.943203in}}%
\pgfpathcurveto{\pgfqpoint{3.350205in}{1.951017in}}{\pgfqpoint{3.339606in}{1.955407in}}{\pgfqpoint{3.328555in}{1.955407in}}%
\pgfpathcurveto{\pgfqpoint{3.317505in}{1.955407in}}{\pgfqpoint{3.306906in}{1.951017in}}{\pgfqpoint{3.299093in}{1.943203in}}%
\pgfpathcurveto{\pgfqpoint{3.291279in}{1.935389in}}{\pgfqpoint{3.286889in}{1.924790in}}{\pgfqpoint{3.286889in}{1.913740in}}%
\pgfpathcurveto{\pgfqpoint{3.286889in}{1.902690in}}{\pgfqpoint{3.291279in}{1.892091in}}{\pgfqpoint{3.299093in}{1.884277in}}%
\pgfpathcurveto{\pgfqpoint{3.306906in}{1.876464in}}{\pgfqpoint{3.317505in}{1.872073in}}{\pgfqpoint{3.328555in}{1.872073in}}%
\pgfpathclose%
\pgfusepath{stroke,fill}%
\end{pgfscope}%
\begin{pgfscope}%
\pgfpathrectangle{\pgfqpoint{0.564660in}{0.521603in}}{\pgfqpoint{3.720000in}{3.020000in}} %
\pgfusepath{clip}%
\pgfsetbuttcap%
\pgfsetroundjoin%
\definecolor{currentfill}{rgb}{0.413725,0.135105,0.997705}%
\pgfsetfillcolor{currentfill}%
\pgfsetlinewidth{1.003750pt}%
\definecolor{currentstroke}{rgb}{0.413725,0.135105,0.997705}%
\pgfsetstrokecolor{currentstroke}%
\pgfsetdash{}{0pt}%
\pgfpathmoveto{\pgfqpoint{2.895722in}{1.797307in}}%
\pgfpathcurveto{\pgfqpoint{2.906772in}{1.797307in}}{\pgfqpoint{2.917371in}{1.801697in}}{\pgfqpoint{2.925185in}{1.809511in}}%
\pgfpathcurveto{\pgfqpoint{2.932998in}{1.817324in}}{\pgfqpoint{2.937389in}{1.827923in}}{\pgfqpoint{2.937389in}{1.838973in}}%
\pgfpathcurveto{\pgfqpoint{2.937389in}{1.850023in}}{\pgfqpoint{2.932998in}{1.860622in}}{\pgfqpoint{2.925185in}{1.868436in}}%
\pgfpathcurveto{\pgfqpoint{2.917371in}{1.876250in}}{\pgfqpoint{2.906772in}{1.880640in}}{\pgfqpoint{2.895722in}{1.880640in}}%
\pgfpathcurveto{\pgfqpoint{2.884672in}{1.880640in}}{\pgfqpoint{2.874073in}{1.876250in}}{\pgfqpoint{2.866259in}{1.868436in}}%
\pgfpathcurveto{\pgfqpoint{2.858445in}{1.860622in}}{\pgfqpoint{2.854055in}{1.850023in}}{\pgfqpoint{2.854055in}{1.838973in}}%
\pgfpathcurveto{\pgfqpoint{2.854055in}{1.827923in}}{\pgfqpoint{2.858445in}{1.817324in}}{\pgfqpoint{2.866259in}{1.809511in}}%
\pgfpathcurveto{\pgfqpoint{2.874073in}{1.801697in}}{\pgfqpoint{2.884672in}{1.797307in}}{\pgfqpoint{2.895722in}{1.797307in}}%
\pgfpathclose%
\pgfusepath{stroke,fill}%
\end{pgfscope}%
\begin{pgfscope}%
\pgfpathrectangle{\pgfqpoint{0.564660in}{0.521603in}}{\pgfqpoint{3.720000in}{3.020000in}} %
\pgfusepath{clip}%
\pgfsetbuttcap%
\pgfsetroundjoin%
\definecolor{currentfill}{rgb}{0.500000,0.000000,1.000000}%
\pgfsetfillcolor{currentfill}%
\pgfsetlinewidth{1.003750pt}%
\definecolor{currentstroke}{rgb}{0.500000,0.000000,1.000000}%
\pgfsetstrokecolor{currentstroke}%
\pgfsetdash{}{0pt}%
\pgfpathmoveto{\pgfqpoint{2.003834in}{2.171024in}}%
\pgfpathcurveto{\pgfqpoint{2.014884in}{2.171024in}}{\pgfqpoint{2.025483in}{2.175415in}}{\pgfqpoint{2.033297in}{2.183228in}}%
\pgfpathcurveto{\pgfqpoint{2.041110in}{2.191042in}}{\pgfqpoint{2.045500in}{2.201641in}}{\pgfqpoint{2.045500in}{2.212691in}}%
\pgfpathcurveto{\pgfqpoint{2.045500in}{2.223741in}}{\pgfqpoint{2.041110in}{2.234340in}}{\pgfqpoint{2.033297in}{2.242154in}}%
\pgfpathcurveto{\pgfqpoint{2.025483in}{2.249968in}}{\pgfqpoint{2.014884in}{2.254358in}}{\pgfqpoint{2.003834in}{2.254358in}}%
\pgfpathcurveto{\pgfqpoint{1.992784in}{2.254358in}}{\pgfqpoint{1.982185in}{2.249968in}}{\pgfqpoint{1.974371in}{2.242154in}}%
\pgfpathcurveto{\pgfqpoint{1.966557in}{2.234340in}}{\pgfqpoint{1.962167in}{2.223741in}}{\pgfqpoint{1.962167in}{2.212691in}}%
\pgfpathcurveto{\pgfqpoint{1.962167in}{2.201641in}}{\pgfqpoint{1.966557in}{2.191042in}}{\pgfqpoint{1.974371in}{2.183228in}}%
\pgfpathcurveto{\pgfqpoint{1.982185in}{2.175415in}}{\pgfqpoint{1.992784in}{2.171024in}}{\pgfqpoint{2.003834in}{2.171024in}}%
\pgfpathclose%
\pgfusepath{stroke,fill}%
\end{pgfscope}%
\begin{pgfscope}%
\pgfpathrectangle{\pgfqpoint{0.564660in}{0.521603in}}{\pgfqpoint{3.720000in}{3.020000in}} %
\pgfusepath{clip}%
\pgfsetbuttcap%
\pgfsetroundjoin%
\definecolor{currentfill}{rgb}{1.000000,0.000000,0.000000}%
\pgfsetfillcolor{currentfill}%
\pgfsetlinewidth{1.003750pt}%
\definecolor{currentstroke}{rgb}{1.000000,0.000000,0.000000}%
\pgfsetstrokecolor{currentstroke}%
\pgfsetdash{}{0pt}%
\pgfpathmoveto{\pgfqpoint{4.166276in}{1.931929in}}%
\pgfpathcurveto{\pgfqpoint{4.177326in}{1.931929in}}{\pgfqpoint{4.187926in}{1.936319in}}{\pgfqpoint{4.195739in}{1.944133in}}%
\pgfpathcurveto{\pgfqpoint{4.203553in}{1.951947in}}{\pgfqpoint{4.207943in}{1.962546in}}{\pgfqpoint{4.207943in}{1.973596in}}%
\pgfpathcurveto{\pgfqpoint{4.207943in}{1.984646in}}{\pgfqpoint{4.203553in}{1.995245in}}{\pgfqpoint{4.195739in}{2.003059in}}%
\pgfpathcurveto{\pgfqpoint{4.187926in}{2.010872in}}{\pgfqpoint{4.177326in}{2.015263in}}{\pgfqpoint{4.166276in}{2.015263in}}%
\pgfpathcurveto{\pgfqpoint{4.155226in}{2.015263in}}{\pgfqpoint{4.144627in}{2.010872in}}{\pgfqpoint{4.136814in}{2.003059in}}%
\pgfpathcurveto{\pgfqpoint{4.129000in}{1.995245in}}{\pgfqpoint{4.124610in}{1.984646in}}{\pgfqpoint{4.124610in}{1.973596in}}%
\pgfpathcurveto{\pgfqpoint{4.124610in}{1.962546in}}{\pgfqpoint{4.129000in}{1.951947in}}{\pgfqpoint{4.136814in}{1.944133in}}%
\pgfpathcurveto{\pgfqpoint{4.144627in}{1.936319in}}{\pgfqpoint{4.155226in}{1.931929in}}{\pgfqpoint{4.166276in}{1.931929in}}%
\pgfpathclose%
\pgfusepath{stroke,fill}%
\end{pgfscope}%
\begin{pgfscope}%
\pgfpathrectangle{\pgfqpoint{0.564660in}{0.521603in}}{\pgfqpoint{3.720000in}{3.020000in}} %
\pgfusepath{clip}%
\pgfsetbuttcap%
\pgfsetroundjoin%
\definecolor{currentfill}{rgb}{1.000000,0.000000,0.000000}%
\pgfsetfillcolor{currentfill}%
\pgfsetlinewidth{1.003750pt}%
\definecolor{currentstroke}{rgb}{1.000000,0.000000,0.000000}%
\pgfsetstrokecolor{currentstroke}%
\pgfsetdash{}{0pt}%
\pgfpathmoveto{\pgfqpoint{3.773192in}{2.699740in}}%
\pgfpathcurveto{\pgfqpoint{3.784242in}{2.699740in}}{\pgfqpoint{3.794841in}{2.704131in}}{\pgfqpoint{3.802655in}{2.711944in}}%
\pgfpathcurveto{\pgfqpoint{3.810468in}{2.719758in}}{\pgfqpoint{3.814859in}{2.730357in}}{\pgfqpoint{3.814859in}{2.741407in}}%
\pgfpathcurveto{\pgfqpoint{3.814859in}{2.752457in}}{\pgfqpoint{3.810468in}{2.763056in}}{\pgfqpoint{3.802655in}{2.770870in}}%
\pgfpathcurveto{\pgfqpoint{3.794841in}{2.778684in}}{\pgfqpoint{3.784242in}{2.783074in}}{\pgfqpoint{3.773192in}{2.783074in}}%
\pgfpathcurveto{\pgfqpoint{3.762142in}{2.783074in}}{\pgfqpoint{3.751543in}{2.778684in}}{\pgfqpoint{3.743729in}{2.770870in}}%
\pgfpathcurveto{\pgfqpoint{3.735915in}{2.763056in}}{\pgfqpoint{3.731525in}{2.752457in}}{\pgfqpoint{3.731525in}{2.741407in}}%
\pgfpathcurveto{\pgfqpoint{3.731525in}{2.730357in}}{\pgfqpoint{3.735915in}{2.719758in}}{\pgfqpoint{3.743729in}{2.711944in}}%
\pgfpathcurveto{\pgfqpoint{3.751543in}{2.704131in}}{\pgfqpoint{3.762142in}{2.699740in}}{\pgfqpoint{3.773192in}{2.699740in}}%
\pgfpathclose%
\pgfusepath{stroke,fill}%
\end{pgfscope}%
\begin{pgfscope}%
\pgfpathrectangle{\pgfqpoint{0.564660in}{0.521603in}}{\pgfqpoint{3.720000in}{3.020000in}} %
\pgfusepath{clip}%
\pgfsetbuttcap%
\pgfsetroundjoin%
\definecolor{currentfill}{rgb}{0.778431,0.905873,0.536867}%
\pgfsetfillcolor{currentfill}%
\pgfsetlinewidth{1.003750pt}%
\definecolor{currentstroke}{rgb}{0.778431,0.905873,0.536867}%
\pgfsetstrokecolor{currentstroke}%
\pgfsetdash{}{0pt}%
\pgfpathmoveto{\pgfqpoint{3.722241in}{2.305818in}}%
\pgfpathcurveto{\pgfqpoint{3.733292in}{2.305818in}}{\pgfqpoint{3.743891in}{2.310209in}}{\pgfqpoint{3.751704in}{2.318022in}}%
\pgfpathcurveto{\pgfqpoint{3.759518in}{2.325836in}}{\pgfqpoint{3.763908in}{2.336435in}}{\pgfqpoint{3.763908in}{2.347485in}}%
\pgfpathcurveto{\pgfqpoint{3.763908in}{2.358535in}}{\pgfqpoint{3.759518in}{2.369134in}}{\pgfqpoint{3.751704in}{2.376948in}}%
\pgfpathcurveto{\pgfqpoint{3.743891in}{2.384761in}}{\pgfqpoint{3.733292in}{2.389152in}}{\pgfqpoint{3.722241in}{2.389152in}}%
\pgfpathcurveto{\pgfqpoint{3.711191in}{2.389152in}}{\pgfqpoint{3.700592in}{2.384761in}}{\pgfqpoint{3.692779in}{2.376948in}}%
\pgfpathcurveto{\pgfqpoint{3.684965in}{2.369134in}}{\pgfqpoint{3.680575in}{2.358535in}}{\pgfqpoint{3.680575in}{2.347485in}}%
\pgfpathcurveto{\pgfqpoint{3.680575in}{2.336435in}}{\pgfqpoint{3.684965in}{2.325836in}}{\pgfqpoint{3.692779in}{2.318022in}}%
\pgfpathcurveto{\pgfqpoint{3.700592in}{2.310209in}}{\pgfqpoint{3.711191in}{2.305818in}}{\pgfqpoint{3.722241in}{2.305818in}}%
\pgfpathclose%
\pgfusepath{stroke,fill}%
\end{pgfscope}%
\begin{pgfscope}%
\pgfpathrectangle{\pgfqpoint{0.564660in}{0.521603in}}{\pgfqpoint{3.720000in}{3.020000in}} %
\pgfusepath{clip}%
\pgfsetbuttcap%
\pgfsetroundjoin%
\definecolor{currentfill}{rgb}{0.778431,0.905873,0.536867}%
\pgfsetfillcolor{currentfill}%
\pgfsetlinewidth{1.003750pt}%
\definecolor{currentstroke}{rgb}{0.778431,0.905873,0.536867}%
\pgfsetstrokecolor{currentstroke}%
\pgfsetdash{}{0pt}%
\pgfpathmoveto{\pgfqpoint{3.468726in}{0.665598in}}%
\pgfpathcurveto{\pgfqpoint{3.479776in}{0.665598in}}{\pgfqpoint{3.490375in}{0.669988in}}{\pgfqpoint{3.498189in}{0.677802in}}%
\pgfpathcurveto{\pgfqpoint{3.506002in}{0.685615in}}{\pgfqpoint{3.510392in}{0.696214in}}{\pgfqpoint{3.510392in}{0.707265in}}%
\pgfpathcurveto{\pgfqpoint{3.510392in}{0.718315in}}{\pgfqpoint{3.506002in}{0.728914in}}{\pgfqpoint{3.498189in}{0.736727in}}%
\pgfpathcurveto{\pgfqpoint{3.490375in}{0.744541in}}{\pgfqpoint{3.479776in}{0.748931in}}{\pgfqpoint{3.468726in}{0.748931in}}%
\pgfpathcurveto{\pgfqpoint{3.457676in}{0.748931in}}{\pgfqpoint{3.447077in}{0.744541in}}{\pgfqpoint{3.439263in}{0.736727in}}%
\pgfpathcurveto{\pgfqpoint{3.431449in}{0.728914in}}{\pgfqpoint{3.427059in}{0.718315in}}{\pgfqpoint{3.427059in}{0.707265in}}%
\pgfpathcurveto{\pgfqpoint{3.427059in}{0.696214in}}{\pgfqpoint{3.431449in}{0.685615in}}{\pgfqpoint{3.439263in}{0.677802in}}%
\pgfpathcurveto{\pgfqpoint{3.447077in}{0.669988in}}{\pgfqpoint{3.457676in}{0.665598in}}{\pgfqpoint{3.468726in}{0.665598in}}%
\pgfpathclose%
\pgfusepath{stroke,fill}%
\end{pgfscope}%
\begin{pgfscope}%
\pgfpathrectangle{\pgfqpoint{0.564660in}{0.521603in}}{\pgfqpoint{3.720000in}{3.020000in}} %
\pgfusepath{clip}%
\pgfsetbuttcap%
\pgfsetroundjoin%
\definecolor{currentfill}{rgb}{0.292157,0.947177,0.812622}%
\pgfsetfillcolor{currentfill}%
\pgfsetlinewidth{1.003750pt}%
\definecolor{currentstroke}{rgb}{0.292157,0.947177,0.812622}%
\pgfsetstrokecolor{currentstroke}%
\pgfsetdash{}{0pt}%
\pgfpathmoveto{\pgfqpoint{3.256354in}{0.933468in}}%
\pgfpathcurveto{\pgfqpoint{3.267405in}{0.933468in}}{\pgfqpoint{3.278004in}{0.937859in}}{\pgfqpoint{3.285817in}{0.945672in}}%
\pgfpathcurveto{\pgfqpoint{3.293631in}{0.953486in}}{\pgfqpoint{3.298021in}{0.964085in}}{\pgfqpoint{3.298021in}{0.975135in}}%
\pgfpathcurveto{\pgfqpoint{3.298021in}{0.986185in}}{\pgfqpoint{3.293631in}{0.996784in}}{\pgfqpoint{3.285817in}{1.004598in}}%
\pgfpathcurveto{\pgfqpoint{3.278004in}{1.012411in}}{\pgfqpoint{3.267405in}{1.016802in}}{\pgfqpoint{3.256354in}{1.016802in}}%
\pgfpathcurveto{\pgfqpoint{3.245304in}{1.016802in}}{\pgfqpoint{3.234705in}{1.012411in}}{\pgfqpoint{3.226892in}{1.004598in}}%
\pgfpathcurveto{\pgfqpoint{3.219078in}{0.996784in}}{\pgfqpoint{3.214688in}{0.986185in}}{\pgfqpoint{3.214688in}{0.975135in}}%
\pgfpathcurveto{\pgfqpoint{3.214688in}{0.964085in}}{\pgfqpoint{3.219078in}{0.953486in}}{\pgfqpoint{3.226892in}{0.945672in}}%
\pgfpathcurveto{\pgfqpoint{3.234705in}{0.937859in}}{\pgfqpoint{3.245304in}{0.933468in}}{\pgfqpoint{3.256354in}{0.933468in}}%
\pgfpathclose%
\pgfusepath{stroke,fill}%
\end{pgfscope}%
\begin{pgfscope}%
\pgfpathrectangle{\pgfqpoint{0.564660in}{0.521603in}}{\pgfqpoint{3.720000in}{3.020000in}} %
\pgfusepath{clip}%
\pgfsetbuttcap%
\pgfsetroundjoin%
\definecolor{currentfill}{rgb}{0.025490,0.734845,0.916034}%
\pgfsetfillcolor{currentfill}%
\pgfsetlinewidth{1.003750pt}%
\definecolor{currentstroke}{rgb}{0.025490,0.734845,0.916034}%
\pgfsetstrokecolor{currentstroke}%
\pgfsetdash{}{0pt}%
\pgfpathmoveto{\pgfqpoint{3.210081in}{0.992260in}}%
\pgfpathcurveto{\pgfqpoint{3.221131in}{0.992260in}}{\pgfqpoint{3.231730in}{0.996651in}}{\pgfqpoint{3.239544in}{1.004464in}}%
\pgfpathcurveto{\pgfqpoint{3.247358in}{1.012278in}}{\pgfqpoint{3.251748in}{1.022877in}}{\pgfqpoint{3.251748in}{1.033927in}}%
\pgfpathcurveto{\pgfqpoint{3.251748in}{1.044977in}}{\pgfqpoint{3.247358in}{1.055576in}}{\pgfqpoint{3.239544in}{1.063390in}}%
\pgfpathcurveto{\pgfqpoint{3.231730in}{1.071203in}}{\pgfqpoint{3.221131in}{1.075594in}}{\pgfqpoint{3.210081in}{1.075594in}}%
\pgfpathcurveto{\pgfqpoint{3.199031in}{1.075594in}}{\pgfqpoint{3.188432in}{1.071203in}}{\pgfqpoint{3.180618in}{1.063390in}}%
\pgfpathcurveto{\pgfqpoint{3.172805in}{1.055576in}}{\pgfqpoint{3.168415in}{1.044977in}}{\pgfqpoint{3.168415in}{1.033927in}}%
\pgfpathcurveto{\pgfqpoint{3.168415in}{1.022877in}}{\pgfqpoint{3.172805in}{1.012278in}}{\pgfqpoint{3.180618in}{1.004464in}}%
\pgfpathcurveto{\pgfqpoint{3.188432in}{0.996651in}}{\pgfqpoint{3.199031in}{0.992260in}}{\pgfqpoint{3.210081in}{0.992260in}}%
\pgfpathclose%
\pgfusepath{stroke,fill}%
\end{pgfscope}%
\begin{pgfscope}%
\pgfpathrectangle{\pgfqpoint{0.564660in}{0.521603in}}{\pgfqpoint{3.720000in}{3.020000in}} %
\pgfusepath{clip}%
\pgfsetbuttcap%
\pgfsetroundjoin%
\definecolor{currentfill}{rgb}{0.178431,0.483911,0.968276}%
\pgfsetfillcolor{currentfill}%
\pgfsetlinewidth{1.003750pt}%
\definecolor{currentstroke}{rgb}{0.178431,0.483911,0.968276}%
\pgfsetstrokecolor{currentstroke}%
\pgfsetdash{}{0pt}%
\pgfpathmoveto{\pgfqpoint{2.706316in}{1.508649in}}%
\pgfpathcurveto{\pgfqpoint{2.717366in}{1.508649in}}{\pgfqpoint{2.727965in}{1.513039in}}{\pgfqpoint{2.735779in}{1.520853in}}%
\pgfpathcurveto{\pgfqpoint{2.743593in}{1.528667in}}{\pgfqpoint{2.747983in}{1.539266in}}{\pgfqpoint{2.747983in}{1.550316in}}%
\pgfpathcurveto{\pgfqpoint{2.747983in}{1.561366in}}{\pgfqpoint{2.743593in}{1.571965in}}{\pgfqpoint{2.735779in}{1.579779in}}%
\pgfpathcurveto{\pgfqpoint{2.727965in}{1.587592in}}{\pgfqpoint{2.717366in}{1.591982in}}{\pgfqpoint{2.706316in}{1.591982in}}%
\pgfpathcurveto{\pgfqpoint{2.695266in}{1.591982in}}{\pgfqpoint{2.684667in}{1.587592in}}{\pgfqpoint{2.676853in}{1.579779in}}%
\pgfpathcurveto{\pgfqpoint{2.669040in}{1.571965in}}{\pgfqpoint{2.664649in}{1.561366in}}{\pgfqpoint{2.664649in}{1.550316in}}%
\pgfpathcurveto{\pgfqpoint{2.664649in}{1.539266in}}{\pgfqpoint{2.669040in}{1.528667in}}{\pgfqpoint{2.676853in}{1.520853in}}%
\pgfpathcurveto{\pgfqpoint{2.684667in}{1.513039in}}{\pgfqpoint{2.695266in}{1.508649in}}{\pgfqpoint{2.706316in}{1.508649in}}%
\pgfpathclose%
\pgfusepath{stroke,fill}%
\end{pgfscope}%
\begin{pgfscope}%
\pgfpathrectangle{\pgfqpoint{0.564660in}{0.521603in}}{\pgfqpoint{3.720000in}{3.020000in}} %
\pgfusepath{clip}%
\pgfsetbuttcap%
\pgfsetroundjoin%
\definecolor{currentfill}{rgb}{0.178431,0.483911,0.968276}%
\pgfsetfillcolor{currentfill}%
\pgfsetlinewidth{1.003750pt}%
\definecolor{currentstroke}{rgb}{0.178431,0.483911,0.968276}%
\pgfsetstrokecolor{currentstroke}%
\pgfsetdash{}{0pt}%
\pgfpathmoveto{\pgfqpoint{2.196868in}{2.811735in}}%
\pgfpathcurveto{\pgfqpoint{2.207918in}{2.811735in}}{\pgfqpoint{2.218517in}{2.816125in}}{\pgfqpoint{2.226330in}{2.823939in}}%
\pgfpathcurveto{\pgfqpoint{2.234144in}{2.831753in}}{\pgfqpoint{2.238534in}{2.842352in}}{\pgfqpoint{2.238534in}{2.853402in}}%
\pgfpathcurveto{\pgfqpoint{2.238534in}{2.864452in}}{\pgfqpoint{2.234144in}{2.875051in}}{\pgfqpoint{2.226330in}{2.882865in}}%
\pgfpathcurveto{\pgfqpoint{2.218517in}{2.890678in}}{\pgfqpoint{2.207918in}{2.895069in}}{\pgfqpoint{2.196868in}{2.895069in}}%
\pgfpathcurveto{\pgfqpoint{2.185818in}{2.895069in}}{\pgfqpoint{2.175219in}{2.890678in}}{\pgfqpoint{2.167405in}{2.882865in}}%
\pgfpathcurveto{\pgfqpoint{2.159591in}{2.875051in}}{\pgfqpoint{2.155201in}{2.864452in}}{\pgfqpoint{2.155201in}{2.853402in}}%
\pgfpathcurveto{\pgfqpoint{2.155201in}{2.842352in}}{\pgfqpoint{2.159591in}{2.831753in}}{\pgfqpoint{2.167405in}{2.823939in}}%
\pgfpathcurveto{\pgfqpoint{2.175219in}{2.816125in}}{\pgfqpoint{2.185818in}{2.811735in}}{\pgfqpoint{2.196868in}{2.811735in}}%
\pgfpathclose%
\pgfusepath{stroke,fill}%
\end{pgfscope}%
\begin{pgfscope}%
\pgfpathrectangle{\pgfqpoint{0.564660in}{0.521603in}}{\pgfqpoint{3.720000in}{3.020000in}} %
\pgfusepath{clip}%
\pgfsetbuttcap%
\pgfsetroundjoin%
\definecolor{currentfill}{rgb}{0.264706,0.361242,0.982973}%
\pgfsetfillcolor{currentfill}%
\pgfsetlinewidth{1.003750pt}%
\definecolor{currentstroke}{rgb}{0.264706,0.361242,0.982973}%
\pgfsetstrokecolor{currentstroke}%
\pgfsetdash{}{0pt}%
\pgfpathmoveto{\pgfqpoint{3.113579in}{2.730222in}}%
\pgfpathcurveto{\pgfqpoint{3.124629in}{2.730222in}}{\pgfqpoint{3.135229in}{2.734613in}}{\pgfqpoint{3.143042in}{2.742426in}}%
\pgfpathcurveto{\pgfqpoint{3.150856in}{2.750240in}}{\pgfqpoint{3.155246in}{2.760839in}}{\pgfqpoint{3.155246in}{2.771889in}}%
\pgfpathcurveto{\pgfqpoint{3.155246in}{2.782939in}}{\pgfqpoint{3.150856in}{2.793538in}}{\pgfqpoint{3.143042in}{2.801352in}}%
\pgfpathcurveto{\pgfqpoint{3.135229in}{2.809165in}}{\pgfqpoint{3.124629in}{2.813556in}}{\pgfqpoint{3.113579in}{2.813556in}}%
\pgfpathcurveto{\pgfqpoint{3.102529in}{2.813556in}}{\pgfqpoint{3.091930in}{2.809165in}}{\pgfqpoint{3.084117in}{2.801352in}}%
\pgfpathcurveto{\pgfqpoint{3.076303in}{2.793538in}}{\pgfqpoint{3.071913in}{2.782939in}}{\pgfqpoint{3.071913in}{2.771889in}}%
\pgfpathcurveto{\pgfqpoint{3.071913in}{2.760839in}}{\pgfqpoint{3.076303in}{2.750240in}}{\pgfqpoint{3.084117in}{2.742426in}}%
\pgfpathcurveto{\pgfqpoint{3.091930in}{2.734613in}}{\pgfqpoint{3.102529in}{2.730222in}}{\pgfqpoint{3.113579in}{2.730222in}}%
\pgfpathclose%
\pgfusepath{stroke,fill}%
\end{pgfscope}%
\begin{pgfscope}%
\pgfpathrectangle{\pgfqpoint{0.564660in}{0.521603in}}{\pgfqpoint{3.720000in}{3.020000in}} %
\pgfusepath{clip}%
\pgfsetbuttcap%
\pgfsetroundjoin%
\definecolor{currentfill}{rgb}{0.264706,0.361242,0.982973}%
\pgfsetfillcolor{currentfill}%
\pgfsetlinewidth{1.003750pt}%
\definecolor{currentstroke}{rgb}{0.264706,0.361242,0.982973}%
\pgfsetstrokecolor{currentstroke}%
\pgfsetdash{}{0pt}%
\pgfpathmoveto{\pgfqpoint{2.812260in}{2.292217in}}%
\pgfpathcurveto{\pgfqpoint{2.823310in}{2.292217in}}{\pgfqpoint{2.833909in}{2.296608in}}{\pgfqpoint{2.841723in}{2.304421in}}%
\pgfpathcurveto{\pgfqpoint{2.849536in}{2.312235in}}{\pgfqpoint{2.853926in}{2.322834in}}{\pgfqpoint{2.853926in}{2.333884in}}%
\pgfpathcurveto{\pgfqpoint{2.853926in}{2.344934in}}{\pgfqpoint{2.849536in}{2.355533in}}{\pgfqpoint{2.841723in}{2.363347in}}%
\pgfpathcurveto{\pgfqpoint{2.833909in}{2.371160in}}{\pgfqpoint{2.823310in}{2.375551in}}{\pgfqpoint{2.812260in}{2.375551in}}%
\pgfpathcurveto{\pgfqpoint{2.801210in}{2.375551in}}{\pgfqpoint{2.790611in}{2.371160in}}{\pgfqpoint{2.782797in}{2.363347in}}%
\pgfpathcurveto{\pgfqpoint{2.774983in}{2.355533in}}{\pgfqpoint{2.770593in}{2.344934in}}{\pgfqpoint{2.770593in}{2.333884in}}%
\pgfpathcurveto{\pgfqpoint{2.770593in}{2.322834in}}{\pgfqpoint{2.774983in}{2.312235in}}{\pgfqpoint{2.782797in}{2.304421in}}%
\pgfpathcurveto{\pgfqpoint{2.790611in}{2.296608in}}{\pgfqpoint{2.801210in}{2.292217in}}{\pgfqpoint{2.812260in}{2.292217in}}%
\pgfpathclose%
\pgfusepath{stroke,fill}%
\end{pgfscope}%
\begin{pgfscope}%
\pgfpathrectangle{\pgfqpoint{0.564660in}{0.521603in}}{\pgfqpoint{3.720000in}{3.020000in}} %
\pgfusepath{clip}%
\pgfsetbuttcap%
\pgfsetroundjoin%
\definecolor{currentfill}{rgb}{0.264706,0.361242,0.982973}%
\pgfsetfillcolor{currentfill}%
\pgfsetlinewidth{1.003750pt}%
\definecolor{currentstroke}{rgb}{0.264706,0.361242,0.982973}%
\pgfsetstrokecolor{currentstroke}%
\pgfsetdash{}{0pt}%
\pgfpathmoveto{\pgfqpoint{2.634682in}{2.192622in}}%
\pgfpathcurveto{\pgfqpoint{2.645732in}{2.192622in}}{\pgfqpoint{2.656331in}{2.197012in}}{\pgfqpoint{2.664145in}{2.204826in}}%
\pgfpathcurveto{\pgfqpoint{2.671958in}{2.212640in}}{\pgfqpoint{2.676349in}{2.223239in}}{\pgfqpoint{2.676349in}{2.234289in}}%
\pgfpathcurveto{\pgfqpoint{2.676349in}{2.245339in}}{\pgfqpoint{2.671958in}{2.255938in}}{\pgfqpoint{2.664145in}{2.263752in}}%
\pgfpathcurveto{\pgfqpoint{2.656331in}{2.271565in}}{\pgfqpoint{2.645732in}{2.275956in}}{\pgfqpoint{2.634682in}{2.275956in}}%
\pgfpathcurveto{\pgfqpoint{2.623632in}{2.275956in}}{\pgfqpoint{2.613033in}{2.271565in}}{\pgfqpoint{2.605219in}{2.263752in}}%
\pgfpathcurveto{\pgfqpoint{2.597406in}{2.255938in}}{\pgfqpoint{2.593015in}{2.245339in}}{\pgfqpoint{2.593015in}{2.234289in}}%
\pgfpathcurveto{\pgfqpoint{2.593015in}{2.223239in}}{\pgfqpoint{2.597406in}{2.212640in}}{\pgfqpoint{2.605219in}{2.204826in}}%
\pgfpathcurveto{\pgfqpoint{2.613033in}{2.197012in}}{\pgfqpoint{2.623632in}{2.192622in}}{\pgfqpoint{2.634682in}{2.192622in}}%
\pgfpathclose%
\pgfusepath{stroke,fill}%
\end{pgfscope}%
\begin{pgfscope}%
\pgfpathrectangle{\pgfqpoint{0.564660in}{0.521603in}}{\pgfqpoint{3.720000in}{3.020000in}} %
\pgfusepath{clip}%
\pgfsetbuttcap%
\pgfsetroundjoin%
\definecolor{currentfill}{rgb}{0.264706,0.361242,0.982973}%
\pgfsetfillcolor{currentfill}%
\pgfsetlinewidth{1.003750pt}%
\definecolor{currentstroke}{rgb}{0.264706,0.361242,0.982973}%
\pgfsetstrokecolor{currentstroke}%
\pgfsetdash{}{0pt}%
\pgfpathmoveto{\pgfqpoint{2.469078in}{1.376730in}}%
\pgfpathcurveto{\pgfqpoint{2.480128in}{1.376730in}}{\pgfqpoint{2.490727in}{1.381120in}}{\pgfqpoint{2.498541in}{1.388934in}}%
\pgfpathcurveto{\pgfqpoint{2.506354in}{1.396747in}}{\pgfqpoint{2.510744in}{1.407346in}}{\pgfqpoint{2.510744in}{1.418396in}}%
\pgfpathcurveto{\pgfqpoint{2.510744in}{1.429447in}}{\pgfqpoint{2.506354in}{1.440046in}}{\pgfqpoint{2.498541in}{1.447859in}}%
\pgfpathcurveto{\pgfqpoint{2.490727in}{1.455673in}}{\pgfqpoint{2.480128in}{1.460063in}}{\pgfqpoint{2.469078in}{1.460063in}}%
\pgfpathcurveto{\pgfqpoint{2.458028in}{1.460063in}}{\pgfqpoint{2.447429in}{1.455673in}}{\pgfqpoint{2.439615in}{1.447859in}}%
\pgfpathcurveto{\pgfqpoint{2.431801in}{1.440046in}}{\pgfqpoint{2.427411in}{1.429447in}}{\pgfqpoint{2.427411in}{1.418396in}}%
\pgfpathcurveto{\pgfqpoint{2.427411in}{1.407346in}}{\pgfqpoint{2.431801in}{1.396747in}}{\pgfqpoint{2.439615in}{1.388934in}}%
\pgfpathcurveto{\pgfqpoint{2.447429in}{1.381120in}}{\pgfqpoint{2.458028in}{1.376730in}}{\pgfqpoint{2.469078in}{1.376730in}}%
\pgfpathclose%
\pgfusepath{stroke,fill}%
\end{pgfscope}%
\begin{pgfscope}%
\pgfpathrectangle{\pgfqpoint{0.564660in}{0.521603in}}{\pgfqpoint{3.720000in}{3.020000in}} %
\pgfusepath{clip}%
\pgfsetbuttcap%
\pgfsetroundjoin%
\definecolor{currentfill}{rgb}{0.264706,0.361242,0.982973}%
\pgfsetfillcolor{currentfill}%
\pgfsetlinewidth{1.003750pt}%
\definecolor{currentstroke}{rgb}{0.264706,0.361242,0.982973}%
\pgfsetstrokecolor{currentstroke}%
\pgfsetdash{}{0pt}%
\pgfpathmoveto{\pgfqpoint{2.011780in}{3.314275in}}%
\pgfpathcurveto{\pgfqpoint{2.022830in}{3.314275in}}{\pgfqpoint{2.033429in}{3.318666in}}{\pgfqpoint{2.041243in}{3.326479in}}%
\pgfpathcurveto{\pgfqpoint{2.049057in}{3.334293in}}{\pgfqpoint{2.053447in}{3.344892in}}{\pgfqpoint{2.053447in}{3.355942in}}%
\pgfpathcurveto{\pgfqpoint{2.053447in}{3.366992in}}{\pgfqpoint{2.049057in}{3.377591in}}{\pgfqpoint{2.041243in}{3.385405in}}%
\pgfpathcurveto{\pgfqpoint{2.033429in}{3.393219in}}{\pgfqpoint{2.022830in}{3.397609in}}{\pgfqpoint{2.011780in}{3.397609in}}%
\pgfpathcurveto{\pgfqpoint{2.000730in}{3.397609in}}{\pgfqpoint{1.990131in}{3.393219in}}{\pgfqpoint{1.982317in}{3.385405in}}%
\pgfpathcurveto{\pgfqpoint{1.974504in}{3.377591in}}{\pgfqpoint{1.970113in}{3.366992in}}{\pgfqpoint{1.970113in}{3.355942in}}%
\pgfpathcurveto{\pgfqpoint{1.970113in}{3.344892in}}{\pgfqpoint{1.974504in}{3.334293in}}{\pgfqpoint{1.982317in}{3.326479in}}%
\pgfpathcurveto{\pgfqpoint{1.990131in}{3.318666in}}{\pgfqpoint{2.000730in}{3.314275in}}{\pgfqpoint{2.011780in}{3.314275in}}%
\pgfpathclose%
\pgfusepath{stroke,fill}%
\end{pgfscope}%
\begin{pgfscope}%
\pgfpathrectangle{\pgfqpoint{0.564660in}{0.521603in}}{\pgfqpoint{3.720000in}{3.020000in}} %
\pgfusepath{clip}%
\pgfsetbuttcap%
\pgfsetroundjoin%
\definecolor{currentfill}{rgb}{0.645098,0.974139,0.622113}%
\pgfsetfillcolor{currentfill}%
\pgfsetlinewidth{1.003750pt}%
\definecolor{currentstroke}{rgb}{0.645098,0.974139,0.622113}%
\pgfsetstrokecolor{currentstroke}%
\pgfsetdash{}{0pt}%
\pgfpathmoveto{\pgfqpoint{3.297585in}{2.473743in}}%
\pgfpathcurveto{\pgfqpoint{3.308636in}{2.473743in}}{\pgfqpoint{3.319235in}{2.478133in}}{\pgfqpoint{3.327048in}{2.485947in}}%
\pgfpathcurveto{\pgfqpoint{3.334862in}{2.493761in}}{\pgfqpoint{3.339252in}{2.504360in}}{\pgfqpoint{3.339252in}{2.515410in}}%
\pgfpathcurveto{\pgfqpoint{3.339252in}{2.526460in}}{\pgfqpoint{3.334862in}{2.537059in}}{\pgfqpoint{3.327048in}{2.544872in}}%
\pgfpathcurveto{\pgfqpoint{3.319235in}{2.552686in}}{\pgfqpoint{3.308636in}{2.557076in}}{\pgfqpoint{3.297585in}{2.557076in}}%
\pgfpathcurveto{\pgfqpoint{3.286535in}{2.557076in}}{\pgfqpoint{3.275936in}{2.552686in}}{\pgfqpoint{3.268123in}{2.544872in}}%
\pgfpathcurveto{\pgfqpoint{3.260309in}{2.537059in}}{\pgfqpoint{3.255919in}{2.526460in}}{\pgfqpoint{3.255919in}{2.515410in}}%
\pgfpathcurveto{\pgfqpoint{3.255919in}{2.504360in}}{\pgfqpoint{3.260309in}{2.493761in}}{\pgfqpoint{3.268123in}{2.485947in}}%
\pgfpathcurveto{\pgfqpoint{3.275936in}{2.478133in}}{\pgfqpoint{3.286535in}{2.473743in}}{\pgfqpoint{3.297585in}{2.473743in}}%
\pgfpathclose%
\pgfusepath{stroke,fill}%
\end{pgfscope}%
\begin{pgfscope}%
\pgfsetbuttcap%
\pgfsetroundjoin%
\definecolor{currentfill}{rgb}{0.000000,0.000000,0.000000}%
\pgfsetfillcolor{currentfill}%
\pgfsetlinewidth{0.803000pt}%
\definecolor{currentstroke}{rgb}{0.000000,0.000000,0.000000}%
\pgfsetstrokecolor{currentstroke}%
\pgfsetdash{}{0pt}%
\pgfsys@defobject{currentmarker}{\pgfqpoint{0.000000in}{-0.048611in}}{\pgfqpoint{0.000000in}{0.000000in}}{%
\pgfpathmoveto{\pgfqpoint{0.000000in}{0.000000in}}%
\pgfpathlineto{\pgfqpoint{0.000000in}{-0.048611in}}%
\pgfusepath{stroke,fill}%
}%
\begin{pgfscope}%
\pgfsys@transformshift{0.564660in}{0.521603in}%
\pgfsys@useobject{currentmarker}{}%
\end{pgfscope}%
\end{pgfscope}%
\begin{pgfscope}%
\pgftext[x=0.564660in,y=0.424381in,,top]{\rmfamily\fontsize{10.000000}{12.000000}\selectfont \(\displaystyle 10^{-2}\)}%
\end{pgfscope}%
\begin{pgfscope}%
\pgfsetbuttcap%
\pgfsetroundjoin%
\definecolor{currentfill}{rgb}{0.000000,0.000000,0.000000}%
\pgfsetfillcolor{currentfill}%
\pgfsetlinewidth{0.803000pt}%
\definecolor{currentstroke}{rgb}{0.000000,0.000000,0.000000}%
\pgfsetstrokecolor{currentstroke}%
\pgfsetdash{}{0pt}%
\pgfsys@defobject{currentmarker}{\pgfqpoint{0.000000in}{-0.048611in}}{\pgfqpoint{0.000000in}{0.000000in}}{%
\pgfpathmoveto{\pgfqpoint{0.000000in}{0.000000in}}%
\pgfpathlineto{\pgfqpoint{0.000000in}{-0.048611in}}%
\pgfusepath{stroke,fill}%
}%
\begin{pgfscope}%
\pgfsys@transformshift{2.397001in}{0.521603in}%
\pgfsys@useobject{currentmarker}{}%
\end{pgfscope}%
\end{pgfscope}%
\begin{pgfscope}%
\pgftext[x=2.397001in,y=0.424381in,,top]{\rmfamily\fontsize{10.000000}{12.000000}\selectfont \(\displaystyle 10^{-1}\)}%
\end{pgfscope}%
\begin{pgfscope}%
\pgfsetbuttcap%
\pgfsetroundjoin%
\definecolor{currentfill}{rgb}{0.000000,0.000000,0.000000}%
\pgfsetfillcolor{currentfill}%
\pgfsetlinewidth{0.803000pt}%
\definecolor{currentstroke}{rgb}{0.000000,0.000000,0.000000}%
\pgfsetstrokecolor{currentstroke}%
\pgfsetdash{}{0pt}%
\pgfsys@defobject{currentmarker}{\pgfqpoint{0.000000in}{-0.048611in}}{\pgfqpoint{0.000000in}{0.000000in}}{%
\pgfpathmoveto{\pgfqpoint{0.000000in}{0.000000in}}%
\pgfpathlineto{\pgfqpoint{0.000000in}{-0.048611in}}%
\pgfusepath{stroke,fill}%
}%
\begin{pgfscope}%
\pgfsys@transformshift{4.229341in}{0.521603in}%
\pgfsys@useobject{currentmarker}{}%
\end{pgfscope}%
\end{pgfscope}%
\begin{pgfscope}%
\pgftext[x=4.229341in,y=0.424381in,,top]{\rmfamily\fontsize{10.000000}{12.000000}\selectfont \(\displaystyle 10^{0}\)}%
\end{pgfscope}%
\begin{pgfscope}%
\pgfsetbuttcap%
\pgfsetroundjoin%
\definecolor{currentfill}{rgb}{0.000000,0.000000,0.000000}%
\pgfsetfillcolor{currentfill}%
\pgfsetlinewidth{0.602250pt}%
\definecolor{currentstroke}{rgb}{0.000000,0.000000,0.000000}%
\pgfsetstrokecolor{currentstroke}%
\pgfsetdash{}{0pt}%
\pgfsys@defobject{currentmarker}{\pgfqpoint{0.000000in}{-0.027778in}}{\pgfqpoint{0.000000in}{0.000000in}}{%
\pgfpathmoveto{\pgfqpoint{0.000000in}{0.000000in}}%
\pgfpathlineto{\pgfqpoint{0.000000in}{-0.027778in}}%
\pgfusepath{stroke,fill}%
}%
\begin{pgfscope}%
\pgfsys@transformshift{1.116250in}{0.521603in}%
\pgfsys@useobject{currentmarker}{}%
\end{pgfscope}%
\end{pgfscope}%
\begin{pgfscope}%
\pgfsetbuttcap%
\pgfsetroundjoin%
\definecolor{currentfill}{rgb}{0.000000,0.000000,0.000000}%
\pgfsetfillcolor{currentfill}%
\pgfsetlinewidth{0.602250pt}%
\definecolor{currentstroke}{rgb}{0.000000,0.000000,0.000000}%
\pgfsetstrokecolor{currentstroke}%
\pgfsetdash{}{0pt}%
\pgfsys@defobject{currentmarker}{\pgfqpoint{0.000000in}{-0.027778in}}{\pgfqpoint{0.000000in}{0.000000in}}{%
\pgfpathmoveto{\pgfqpoint{0.000000in}{0.000000in}}%
\pgfpathlineto{\pgfqpoint{0.000000in}{-0.027778in}}%
\pgfusepath{stroke,fill}%
}%
\begin{pgfscope}%
\pgfsys@transformshift{1.438909in}{0.521603in}%
\pgfsys@useobject{currentmarker}{}%
\end{pgfscope}%
\end{pgfscope}%
\begin{pgfscope}%
\pgfsetbuttcap%
\pgfsetroundjoin%
\definecolor{currentfill}{rgb}{0.000000,0.000000,0.000000}%
\pgfsetfillcolor{currentfill}%
\pgfsetlinewidth{0.602250pt}%
\definecolor{currentstroke}{rgb}{0.000000,0.000000,0.000000}%
\pgfsetstrokecolor{currentstroke}%
\pgfsetdash{}{0pt}%
\pgfsys@defobject{currentmarker}{\pgfqpoint{0.000000in}{-0.027778in}}{\pgfqpoint{0.000000in}{0.000000in}}{%
\pgfpathmoveto{\pgfqpoint{0.000000in}{0.000000in}}%
\pgfpathlineto{\pgfqpoint{0.000000in}{-0.027778in}}%
\pgfusepath{stroke,fill}%
}%
\begin{pgfscope}%
\pgfsys@transformshift{1.667839in}{0.521603in}%
\pgfsys@useobject{currentmarker}{}%
\end{pgfscope}%
\end{pgfscope}%
\begin{pgfscope}%
\pgfsetbuttcap%
\pgfsetroundjoin%
\definecolor{currentfill}{rgb}{0.000000,0.000000,0.000000}%
\pgfsetfillcolor{currentfill}%
\pgfsetlinewidth{0.602250pt}%
\definecolor{currentstroke}{rgb}{0.000000,0.000000,0.000000}%
\pgfsetstrokecolor{currentstroke}%
\pgfsetdash{}{0pt}%
\pgfsys@defobject{currentmarker}{\pgfqpoint{0.000000in}{-0.027778in}}{\pgfqpoint{0.000000in}{0.000000in}}{%
\pgfpathmoveto{\pgfqpoint{0.000000in}{0.000000in}}%
\pgfpathlineto{\pgfqpoint{0.000000in}{-0.027778in}}%
\pgfusepath{stroke,fill}%
}%
\begin{pgfscope}%
\pgfsys@transformshift{1.845411in}{0.521603in}%
\pgfsys@useobject{currentmarker}{}%
\end{pgfscope}%
\end{pgfscope}%
\begin{pgfscope}%
\pgfsetbuttcap%
\pgfsetroundjoin%
\definecolor{currentfill}{rgb}{0.000000,0.000000,0.000000}%
\pgfsetfillcolor{currentfill}%
\pgfsetlinewidth{0.602250pt}%
\definecolor{currentstroke}{rgb}{0.000000,0.000000,0.000000}%
\pgfsetstrokecolor{currentstroke}%
\pgfsetdash{}{0pt}%
\pgfsys@defobject{currentmarker}{\pgfqpoint{0.000000in}{-0.027778in}}{\pgfqpoint{0.000000in}{0.000000in}}{%
\pgfpathmoveto{\pgfqpoint{0.000000in}{0.000000in}}%
\pgfpathlineto{\pgfqpoint{0.000000in}{-0.027778in}}%
\pgfusepath{stroke,fill}%
}%
\begin{pgfscope}%
\pgfsys@transformshift{1.990498in}{0.521603in}%
\pgfsys@useobject{currentmarker}{}%
\end{pgfscope}%
\end{pgfscope}%
\begin{pgfscope}%
\pgfsetbuttcap%
\pgfsetroundjoin%
\definecolor{currentfill}{rgb}{0.000000,0.000000,0.000000}%
\pgfsetfillcolor{currentfill}%
\pgfsetlinewidth{0.602250pt}%
\definecolor{currentstroke}{rgb}{0.000000,0.000000,0.000000}%
\pgfsetstrokecolor{currentstroke}%
\pgfsetdash{}{0pt}%
\pgfsys@defobject{currentmarker}{\pgfqpoint{0.000000in}{-0.027778in}}{\pgfqpoint{0.000000in}{0.000000in}}{%
\pgfpathmoveto{\pgfqpoint{0.000000in}{0.000000in}}%
\pgfpathlineto{\pgfqpoint{0.000000in}{-0.027778in}}%
\pgfusepath{stroke,fill}%
}%
\begin{pgfscope}%
\pgfsys@transformshift{2.113168in}{0.521603in}%
\pgfsys@useobject{currentmarker}{}%
\end{pgfscope}%
\end{pgfscope}%
\begin{pgfscope}%
\pgfsetbuttcap%
\pgfsetroundjoin%
\definecolor{currentfill}{rgb}{0.000000,0.000000,0.000000}%
\pgfsetfillcolor{currentfill}%
\pgfsetlinewidth{0.602250pt}%
\definecolor{currentstroke}{rgb}{0.000000,0.000000,0.000000}%
\pgfsetstrokecolor{currentstroke}%
\pgfsetdash{}{0pt}%
\pgfsys@defobject{currentmarker}{\pgfqpoint{0.000000in}{-0.027778in}}{\pgfqpoint{0.000000in}{0.000000in}}{%
\pgfpathmoveto{\pgfqpoint{0.000000in}{0.000000in}}%
\pgfpathlineto{\pgfqpoint{0.000000in}{-0.027778in}}%
\pgfusepath{stroke,fill}%
}%
\begin{pgfscope}%
\pgfsys@transformshift{2.219429in}{0.521603in}%
\pgfsys@useobject{currentmarker}{}%
\end{pgfscope}%
\end{pgfscope}%
\begin{pgfscope}%
\pgfsetbuttcap%
\pgfsetroundjoin%
\definecolor{currentfill}{rgb}{0.000000,0.000000,0.000000}%
\pgfsetfillcolor{currentfill}%
\pgfsetlinewidth{0.602250pt}%
\definecolor{currentstroke}{rgb}{0.000000,0.000000,0.000000}%
\pgfsetstrokecolor{currentstroke}%
\pgfsetdash{}{0pt}%
\pgfsys@defobject{currentmarker}{\pgfqpoint{0.000000in}{-0.027778in}}{\pgfqpoint{0.000000in}{0.000000in}}{%
\pgfpathmoveto{\pgfqpoint{0.000000in}{0.000000in}}%
\pgfpathlineto{\pgfqpoint{0.000000in}{-0.027778in}}%
\pgfusepath{stroke,fill}%
}%
\begin{pgfscope}%
\pgfsys@transformshift{2.313157in}{0.521603in}%
\pgfsys@useobject{currentmarker}{}%
\end{pgfscope}%
\end{pgfscope}%
\begin{pgfscope}%
\pgfsetbuttcap%
\pgfsetroundjoin%
\definecolor{currentfill}{rgb}{0.000000,0.000000,0.000000}%
\pgfsetfillcolor{currentfill}%
\pgfsetlinewidth{0.602250pt}%
\definecolor{currentstroke}{rgb}{0.000000,0.000000,0.000000}%
\pgfsetstrokecolor{currentstroke}%
\pgfsetdash{}{0pt}%
\pgfsys@defobject{currentmarker}{\pgfqpoint{0.000000in}{-0.027778in}}{\pgfqpoint{0.000000in}{0.000000in}}{%
\pgfpathmoveto{\pgfqpoint{0.000000in}{0.000000in}}%
\pgfpathlineto{\pgfqpoint{0.000000in}{-0.027778in}}%
\pgfusepath{stroke,fill}%
}%
\begin{pgfscope}%
\pgfsys@transformshift{2.948590in}{0.521603in}%
\pgfsys@useobject{currentmarker}{}%
\end{pgfscope}%
\end{pgfscope}%
\begin{pgfscope}%
\pgfsetbuttcap%
\pgfsetroundjoin%
\definecolor{currentfill}{rgb}{0.000000,0.000000,0.000000}%
\pgfsetfillcolor{currentfill}%
\pgfsetlinewidth{0.602250pt}%
\definecolor{currentstroke}{rgb}{0.000000,0.000000,0.000000}%
\pgfsetstrokecolor{currentstroke}%
\pgfsetdash{}{0pt}%
\pgfsys@defobject{currentmarker}{\pgfqpoint{0.000000in}{-0.027778in}}{\pgfqpoint{0.000000in}{0.000000in}}{%
\pgfpathmoveto{\pgfqpoint{0.000000in}{0.000000in}}%
\pgfpathlineto{\pgfqpoint{0.000000in}{-0.027778in}}%
\pgfusepath{stroke,fill}%
}%
\begin{pgfscope}%
\pgfsys@transformshift{3.271249in}{0.521603in}%
\pgfsys@useobject{currentmarker}{}%
\end{pgfscope}%
\end{pgfscope}%
\begin{pgfscope}%
\pgfsetbuttcap%
\pgfsetroundjoin%
\definecolor{currentfill}{rgb}{0.000000,0.000000,0.000000}%
\pgfsetfillcolor{currentfill}%
\pgfsetlinewidth{0.602250pt}%
\definecolor{currentstroke}{rgb}{0.000000,0.000000,0.000000}%
\pgfsetstrokecolor{currentstroke}%
\pgfsetdash{}{0pt}%
\pgfsys@defobject{currentmarker}{\pgfqpoint{0.000000in}{-0.027778in}}{\pgfqpoint{0.000000in}{0.000000in}}{%
\pgfpathmoveto{\pgfqpoint{0.000000in}{0.000000in}}%
\pgfpathlineto{\pgfqpoint{0.000000in}{-0.027778in}}%
\pgfusepath{stroke,fill}%
}%
\begin{pgfscope}%
\pgfsys@transformshift{3.500180in}{0.521603in}%
\pgfsys@useobject{currentmarker}{}%
\end{pgfscope}%
\end{pgfscope}%
\begin{pgfscope}%
\pgfsetbuttcap%
\pgfsetroundjoin%
\definecolor{currentfill}{rgb}{0.000000,0.000000,0.000000}%
\pgfsetfillcolor{currentfill}%
\pgfsetlinewidth{0.602250pt}%
\definecolor{currentstroke}{rgb}{0.000000,0.000000,0.000000}%
\pgfsetstrokecolor{currentstroke}%
\pgfsetdash{}{0pt}%
\pgfsys@defobject{currentmarker}{\pgfqpoint{0.000000in}{-0.027778in}}{\pgfqpoint{0.000000in}{0.000000in}}{%
\pgfpathmoveto{\pgfqpoint{0.000000in}{0.000000in}}%
\pgfpathlineto{\pgfqpoint{0.000000in}{-0.027778in}}%
\pgfusepath{stroke,fill}%
}%
\begin{pgfscope}%
\pgfsys@transformshift{3.677752in}{0.521603in}%
\pgfsys@useobject{currentmarker}{}%
\end{pgfscope}%
\end{pgfscope}%
\begin{pgfscope}%
\pgfsetbuttcap%
\pgfsetroundjoin%
\definecolor{currentfill}{rgb}{0.000000,0.000000,0.000000}%
\pgfsetfillcolor{currentfill}%
\pgfsetlinewidth{0.602250pt}%
\definecolor{currentstroke}{rgb}{0.000000,0.000000,0.000000}%
\pgfsetstrokecolor{currentstroke}%
\pgfsetdash{}{0pt}%
\pgfsys@defobject{currentmarker}{\pgfqpoint{0.000000in}{-0.027778in}}{\pgfqpoint{0.000000in}{0.000000in}}{%
\pgfpathmoveto{\pgfqpoint{0.000000in}{0.000000in}}%
\pgfpathlineto{\pgfqpoint{0.000000in}{-0.027778in}}%
\pgfusepath{stroke,fill}%
}%
\begin{pgfscope}%
\pgfsys@transformshift{3.822839in}{0.521603in}%
\pgfsys@useobject{currentmarker}{}%
\end{pgfscope}%
\end{pgfscope}%
\begin{pgfscope}%
\pgfsetbuttcap%
\pgfsetroundjoin%
\definecolor{currentfill}{rgb}{0.000000,0.000000,0.000000}%
\pgfsetfillcolor{currentfill}%
\pgfsetlinewidth{0.602250pt}%
\definecolor{currentstroke}{rgb}{0.000000,0.000000,0.000000}%
\pgfsetstrokecolor{currentstroke}%
\pgfsetdash{}{0pt}%
\pgfsys@defobject{currentmarker}{\pgfqpoint{0.000000in}{-0.027778in}}{\pgfqpoint{0.000000in}{0.000000in}}{%
\pgfpathmoveto{\pgfqpoint{0.000000in}{0.000000in}}%
\pgfpathlineto{\pgfqpoint{0.000000in}{-0.027778in}}%
\pgfusepath{stroke,fill}%
}%
\begin{pgfscope}%
\pgfsys@transformshift{3.945508in}{0.521603in}%
\pgfsys@useobject{currentmarker}{}%
\end{pgfscope}%
\end{pgfscope}%
\begin{pgfscope}%
\pgfsetbuttcap%
\pgfsetroundjoin%
\definecolor{currentfill}{rgb}{0.000000,0.000000,0.000000}%
\pgfsetfillcolor{currentfill}%
\pgfsetlinewidth{0.602250pt}%
\definecolor{currentstroke}{rgb}{0.000000,0.000000,0.000000}%
\pgfsetstrokecolor{currentstroke}%
\pgfsetdash{}{0pt}%
\pgfsys@defobject{currentmarker}{\pgfqpoint{0.000000in}{-0.027778in}}{\pgfqpoint{0.000000in}{0.000000in}}{%
\pgfpathmoveto{\pgfqpoint{0.000000in}{0.000000in}}%
\pgfpathlineto{\pgfqpoint{0.000000in}{-0.027778in}}%
\pgfusepath{stroke,fill}%
}%
\begin{pgfscope}%
\pgfsys@transformshift{4.051769in}{0.521603in}%
\pgfsys@useobject{currentmarker}{}%
\end{pgfscope}%
\end{pgfscope}%
\begin{pgfscope}%
\pgfsetbuttcap%
\pgfsetroundjoin%
\definecolor{currentfill}{rgb}{0.000000,0.000000,0.000000}%
\pgfsetfillcolor{currentfill}%
\pgfsetlinewidth{0.602250pt}%
\definecolor{currentstroke}{rgb}{0.000000,0.000000,0.000000}%
\pgfsetstrokecolor{currentstroke}%
\pgfsetdash{}{0pt}%
\pgfsys@defobject{currentmarker}{\pgfqpoint{0.000000in}{-0.027778in}}{\pgfqpoint{0.000000in}{0.000000in}}{%
\pgfpathmoveto{\pgfqpoint{0.000000in}{0.000000in}}%
\pgfpathlineto{\pgfqpoint{0.000000in}{-0.027778in}}%
\pgfusepath{stroke,fill}%
}%
\begin{pgfscope}%
\pgfsys@transformshift{4.145498in}{0.521603in}%
\pgfsys@useobject{currentmarker}{}%
\end{pgfscope}%
\end{pgfscope}%
\begin{pgfscope}%
\pgftext[x=2.424660in,y=0.234413in,,top]{\rmfamily\fontsize{10.000000}{12.000000}\selectfont \(\displaystyle We\)}%
\end{pgfscope}%
\begin{pgfscope}%
\pgfsetbuttcap%
\pgfsetroundjoin%
\definecolor{currentfill}{rgb}{0.000000,0.000000,0.000000}%
\pgfsetfillcolor{currentfill}%
\pgfsetlinewidth{0.803000pt}%
\definecolor{currentstroke}{rgb}{0.000000,0.000000,0.000000}%
\pgfsetstrokecolor{currentstroke}%
\pgfsetdash{}{0pt}%
\pgfsys@defobject{currentmarker}{\pgfqpoint{-0.048611in}{0.000000in}}{\pgfqpoint{0.000000in}{0.000000in}}{%
\pgfpathmoveto{\pgfqpoint{0.000000in}{0.000000in}}%
\pgfpathlineto{\pgfqpoint{-0.048611in}{0.000000in}}%
\pgfusepath{stroke,fill}%
}%
\begin{pgfscope}%
\pgfsys@transformshift{0.564660in}{0.952279in}%
\pgfsys@useobject{currentmarker}{}%
\end{pgfscope}%
\end{pgfscope}%
\begin{pgfscope}%
\pgftext[x=0.289968in,y=0.899518in,left,base]{\rmfamily\fontsize{10.000000}{12.000000}\selectfont \(\displaystyle 0.5\)}%
\end{pgfscope}%
\begin{pgfscope}%
\pgfsetbuttcap%
\pgfsetroundjoin%
\definecolor{currentfill}{rgb}{0.000000,0.000000,0.000000}%
\pgfsetfillcolor{currentfill}%
\pgfsetlinewidth{0.803000pt}%
\definecolor{currentstroke}{rgb}{0.000000,0.000000,0.000000}%
\pgfsetstrokecolor{currentstroke}%
\pgfsetdash{}{0pt}%
\pgfsys@defobject{currentmarker}{\pgfqpoint{-0.048611in}{0.000000in}}{\pgfqpoint{0.000000in}{0.000000in}}{%
\pgfpathmoveto{\pgfqpoint{0.000000in}{0.000000in}}%
\pgfpathlineto{\pgfqpoint{-0.048611in}{0.000000in}}%
\pgfusepath{stroke,fill}%
}%
\begin{pgfscope}%
\pgfsys@transformshift{0.564660in}{1.439390in}%
\pgfsys@useobject{currentmarker}{}%
\end{pgfscope}%
\end{pgfscope}%
\begin{pgfscope}%
\pgftext[x=0.289968in,y=1.386628in,left,base]{\rmfamily\fontsize{10.000000}{12.000000}\selectfont \(\displaystyle 0.6\)}%
\end{pgfscope}%
\begin{pgfscope}%
\pgfsetbuttcap%
\pgfsetroundjoin%
\definecolor{currentfill}{rgb}{0.000000,0.000000,0.000000}%
\pgfsetfillcolor{currentfill}%
\pgfsetlinewidth{0.803000pt}%
\definecolor{currentstroke}{rgb}{0.000000,0.000000,0.000000}%
\pgfsetstrokecolor{currentstroke}%
\pgfsetdash{}{0pt}%
\pgfsys@defobject{currentmarker}{\pgfqpoint{-0.048611in}{0.000000in}}{\pgfqpoint{0.000000in}{0.000000in}}{%
\pgfpathmoveto{\pgfqpoint{0.000000in}{0.000000in}}%
\pgfpathlineto{\pgfqpoint{-0.048611in}{0.000000in}}%
\pgfusepath{stroke,fill}%
}%
\begin{pgfscope}%
\pgfsys@transformshift{0.564660in}{1.926500in}%
\pgfsys@useobject{currentmarker}{}%
\end{pgfscope}%
\end{pgfscope}%
\begin{pgfscope}%
\pgftext[x=0.289968in,y=1.873739in,left,base]{\rmfamily\fontsize{10.000000}{12.000000}\selectfont \(\displaystyle 0.7\)}%
\end{pgfscope}%
\begin{pgfscope}%
\pgfsetbuttcap%
\pgfsetroundjoin%
\definecolor{currentfill}{rgb}{0.000000,0.000000,0.000000}%
\pgfsetfillcolor{currentfill}%
\pgfsetlinewidth{0.803000pt}%
\definecolor{currentstroke}{rgb}{0.000000,0.000000,0.000000}%
\pgfsetstrokecolor{currentstroke}%
\pgfsetdash{}{0pt}%
\pgfsys@defobject{currentmarker}{\pgfqpoint{-0.048611in}{0.000000in}}{\pgfqpoint{0.000000in}{0.000000in}}{%
\pgfpathmoveto{\pgfqpoint{0.000000in}{0.000000in}}%
\pgfpathlineto{\pgfqpoint{-0.048611in}{0.000000in}}%
\pgfusepath{stroke,fill}%
}%
\begin{pgfscope}%
\pgfsys@transformshift{0.564660in}{2.413610in}%
\pgfsys@useobject{currentmarker}{}%
\end{pgfscope}%
\end{pgfscope}%
\begin{pgfscope}%
\pgftext[x=0.289968in,y=2.360849in,left,base]{\rmfamily\fontsize{10.000000}{12.000000}\selectfont \(\displaystyle 0.8\)}%
\end{pgfscope}%
\begin{pgfscope}%
\pgfsetbuttcap%
\pgfsetroundjoin%
\definecolor{currentfill}{rgb}{0.000000,0.000000,0.000000}%
\pgfsetfillcolor{currentfill}%
\pgfsetlinewidth{0.803000pt}%
\definecolor{currentstroke}{rgb}{0.000000,0.000000,0.000000}%
\pgfsetstrokecolor{currentstroke}%
\pgfsetdash{}{0pt}%
\pgfsys@defobject{currentmarker}{\pgfqpoint{-0.048611in}{0.000000in}}{\pgfqpoint{0.000000in}{0.000000in}}{%
\pgfpathmoveto{\pgfqpoint{0.000000in}{0.000000in}}%
\pgfpathlineto{\pgfqpoint{-0.048611in}{0.000000in}}%
\pgfusepath{stroke,fill}%
}%
\begin{pgfscope}%
\pgfsys@transformshift{0.564660in}{2.900721in}%
\pgfsys@useobject{currentmarker}{}%
\end{pgfscope}%
\end{pgfscope}%
\begin{pgfscope}%
\pgftext[x=0.289968in,y=2.847959in,left,base]{\rmfamily\fontsize{10.000000}{12.000000}\selectfont \(\displaystyle 0.9\)}%
\end{pgfscope}%
\begin{pgfscope}%
\pgfsetbuttcap%
\pgfsetroundjoin%
\definecolor{currentfill}{rgb}{0.000000,0.000000,0.000000}%
\pgfsetfillcolor{currentfill}%
\pgfsetlinewidth{0.803000pt}%
\definecolor{currentstroke}{rgb}{0.000000,0.000000,0.000000}%
\pgfsetstrokecolor{currentstroke}%
\pgfsetdash{}{0pt}%
\pgfsys@defobject{currentmarker}{\pgfqpoint{-0.048611in}{0.000000in}}{\pgfqpoint{0.000000in}{0.000000in}}{%
\pgfpathmoveto{\pgfqpoint{0.000000in}{0.000000in}}%
\pgfpathlineto{\pgfqpoint{-0.048611in}{0.000000in}}%
\pgfusepath{stroke,fill}%
}%
\begin{pgfscope}%
\pgfsys@transformshift{0.564660in}{3.387831in}%
\pgfsys@useobject{currentmarker}{}%
\end{pgfscope}%
\end{pgfscope}%
\begin{pgfscope}%
\pgftext[x=0.289968in,y=3.335069in,left,base]{\rmfamily\fontsize{10.000000}{12.000000}\selectfont \(\displaystyle 1.0\)}%
\end{pgfscope}%
\begin{pgfscope}%
\pgftext[x=0.234413in,y=2.031603in,,bottom,rotate=90.000000]{\rmfamily\fontsize{10.000000}{12.000000}\selectfont \(\displaystyle C_r\)}%
\end{pgfscope}%
\begin{pgfscope}%
\pgfsetrectcap%
\pgfsetmiterjoin%
\pgfsetlinewidth{0.803000pt}%
\definecolor{currentstroke}{rgb}{0.000000,0.000000,0.000000}%
\pgfsetstrokecolor{currentstroke}%
\pgfsetdash{}{0pt}%
\pgfpathmoveto{\pgfqpoint{0.564660in}{0.521603in}}%
\pgfpathlineto{\pgfqpoint{0.564660in}{3.541603in}}%
\pgfusepath{stroke}%
\end{pgfscope}%
\begin{pgfscope}%
\pgfsetrectcap%
\pgfsetmiterjoin%
\pgfsetlinewidth{0.803000pt}%
\definecolor{currentstroke}{rgb}{0.000000,0.000000,0.000000}%
\pgfsetstrokecolor{currentstroke}%
\pgfsetdash{}{0pt}%
\pgfpathmoveto{\pgfqpoint{4.284660in}{0.521603in}}%
\pgfpathlineto{\pgfqpoint{4.284660in}{3.541603in}}%
\pgfusepath{stroke}%
\end{pgfscope}%
\begin{pgfscope}%
\pgfsetrectcap%
\pgfsetmiterjoin%
\pgfsetlinewidth{0.803000pt}%
\definecolor{currentstroke}{rgb}{0.000000,0.000000,0.000000}%
\pgfsetstrokecolor{currentstroke}%
\pgfsetdash{}{0pt}%
\pgfpathmoveto{\pgfqpoint{0.564660in}{0.521603in}}%
\pgfpathlineto{\pgfqpoint{4.284660in}{0.521603in}}%
\pgfusepath{stroke}%
\end{pgfscope}%
\begin{pgfscope}%
\pgfsetrectcap%
\pgfsetmiterjoin%
\pgfsetlinewidth{0.803000pt}%
\definecolor{currentstroke}{rgb}{0.000000,0.000000,0.000000}%
\pgfsetstrokecolor{currentstroke}%
\pgfsetdash{}{0pt}%
\pgfpathmoveto{\pgfqpoint{0.564660in}{3.541603in}}%
\pgfpathlineto{\pgfqpoint{4.284660in}{3.541603in}}%
\pgfusepath{stroke}%
\end{pgfscope}%
\begin{pgfscope}%
\pgfpathrectangle{\pgfqpoint{4.517160in}{0.521603in}}{\pgfqpoint{0.151000in}{3.020000in}} %
\pgfusepath{clip}%
\pgfsetbuttcap%
\pgfsetmiterjoin%
\definecolor{currentfill}{rgb}{1.000000,1.000000,1.000000}%
\pgfsetfillcolor{currentfill}%
\pgfsetlinewidth{0.010037pt}%
\definecolor{currentstroke}{rgb}{1.000000,1.000000,1.000000}%
\pgfsetstrokecolor{currentstroke}%
\pgfsetdash{}{0pt}%
\pgfpathmoveto{\pgfqpoint{4.517160in}{0.521603in}}%
\pgfpathlineto{\pgfqpoint{4.517160in}{0.533400in}}%
\pgfpathlineto{\pgfqpoint{4.517160in}{3.529806in}}%
\pgfpathlineto{\pgfqpoint{4.517160in}{3.541603in}}%
\pgfpathlineto{\pgfqpoint{4.668160in}{3.541603in}}%
\pgfpathlineto{\pgfqpoint{4.668160in}{3.529806in}}%
\pgfpathlineto{\pgfqpoint{4.668160in}{0.533400in}}%
\pgfpathlineto{\pgfqpoint{4.668160in}{0.521603in}}%
\pgfpathclose%
\pgfusepath{stroke,fill}%
\end{pgfscope}%
\begin{pgfscope}%
\pgfsys@transformshift{4.520000in}{0.526603in}%
\pgftext[left,bottom]{\pgfimage[interpolate=true,width=0.150000in,height=3.020000in]{restitution-img0.png}}%
\end{pgfscope}%
\begin{pgfscope}%
\pgfsetbuttcap%
\pgfsetroundjoin%
\definecolor{currentfill}{rgb}{0.000000,0.000000,0.000000}%
\pgfsetfillcolor{currentfill}%
\pgfsetlinewidth{0.803000pt}%
\definecolor{currentstroke}{rgb}{0.000000,0.000000,0.000000}%
\pgfsetstrokecolor{currentstroke}%
\pgfsetdash{}{0pt}%
\pgfsys@defobject{currentmarker}{\pgfqpoint{0.000000in}{0.000000in}}{\pgfqpoint{0.048611in}{0.000000in}}{%
\pgfpathmoveto{\pgfqpoint{0.000000in}{0.000000in}}%
\pgfpathlineto{\pgfqpoint{0.048611in}{0.000000in}}%
\pgfusepath{stroke,fill}%
}%
\begin{pgfscope}%
\pgfsys@transformshift{4.668160in}{0.991682in}%
\pgfsys@useobject{currentmarker}{}%
\end{pgfscope}%
\end{pgfscope}%
\begin{pgfscope}%
\pgftext[x=4.765383in,y=0.938921in,left,base]{\rmfamily\fontsize{10.000000}{12.000000}\selectfont \(\displaystyle 0.2\)}%
\end{pgfscope}%
\begin{pgfscope}%
\pgfsetbuttcap%
\pgfsetroundjoin%
\definecolor{currentfill}{rgb}{0.000000,0.000000,0.000000}%
\pgfsetfillcolor{currentfill}%
\pgfsetlinewidth{0.803000pt}%
\definecolor{currentstroke}{rgb}{0.000000,0.000000,0.000000}%
\pgfsetstrokecolor{currentstroke}%
\pgfsetdash{}{0pt}%
\pgfsys@defobject{currentmarker}{\pgfqpoint{0.000000in}{0.000000in}}{\pgfqpoint{0.048611in}{0.000000in}}{%
\pgfpathmoveto{\pgfqpoint{0.000000in}{0.000000in}}%
\pgfpathlineto{\pgfqpoint{0.048611in}{0.000000in}}%
\pgfusepath{stroke,fill}%
}%
\begin{pgfscope}%
\pgfsys@transformshift{4.668160in}{1.583298in}%
\pgfsys@useobject{currentmarker}{}%
\end{pgfscope}%
\end{pgfscope}%
\begin{pgfscope}%
\pgftext[x=4.765383in,y=1.530536in,left,base]{\rmfamily\fontsize{10.000000}{12.000000}\selectfont \(\displaystyle 0.4\)}%
\end{pgfscope}%
\begin{pgfscope}%
\pgfsetbuttcap%
\pgfsetroundjoin%
\definecolor{currentfill}{rgb}{0.000000,0.000000,0.000000}%
\pgfsetfillcolor{currentfill}%
\pgfsetlinewidth{0.803000pt}%
\definecolor{currentstroke}{rgb}{0.000000,0.000000,0.000000}%
\pgfsetstrokecolor{currentstroke}%
\pgfsetdash{}{0pt}%
\pgfsys@defobject{currentmarker}{\pgfqpoint{0.000000in}{0.000000in}}{\pgfqpoint{0.048611in}{0.000000in}}{%
\pgfpathmoveto{\pgfqpoint{0.000000in}{0.000000in}}%
\pgfpathlineto{\pgfqpoint{0.048611in}{0.000000in}}%
\pgfusepath{stroke,fill}%
}%
\begin{pgfscope}%
\pgfsys@transformshift{4.668160in}{2.174913in}%
\pgfsys@useobject{currentmarker}{}%
\end{pgfscope}%
\end{pgfscope}%
\begin{pgfscope}%
\pgftext[x=4.765383in,y=2.122152in,left,base]{\rmfamily\fontsize{10.000000}{12.000000}\selectfont \(\displaystyle 0.6\)}%
\end{pgfscope}%
\begin{pgfscope}%
\pgfsetbuttcap%
\pgfsetroundjoin%
\definecolor{currentfill}{rgb}{0.000000,0.000000,0.000000}%
\pgfsetfillcolor{currentfill}%
\pgfsetlinewidth{0.803000pt}%
\definecolor{currentstroke}{rgb}{0.000000,0.000000,0.000000}%
\pgfsetstrokecolor{currentstroke}%
\pgfsetdash{}{0pt}%
\pgfsys@defobject{currentmarker}{\pgfqpoint{0.000000in}{0.000000in}}{\pgfqpoint{0.048611in}{0.000000in}}{%
\pgfpathmoveto{\pgfqpoint{0.000000in}{0.000000in}}%
\pgfpathlineto{\pgfqpoint{0.048611in}{0.000000in}}%
\pgfusepath{stroke,fill}%
}%
\begin{pgfscope}%
\pgfsys@transformshift{4.668160in}{2.766529in}%
\pgfsys@useobject{currentmarker}{}%
\end{pgfscope}%
\end{pgfscope}%
\begin{pgfscope}%
\pgftext[x=4.765383in,y=2.713767in,left,base]{\rmfamily\fontsize{10.000000}{12.000000}\selectfont \(\displaystyle 0.8\)}%
\end{pgfscope}%
\begin{pgfscope}%
\pgfsetbuttcap%
\pgfsetroundjoin%
\definecolor{currentfill}{rgb}{0.000000,0.000000,0.000000}%
\pgfsetfillcolor{currentfill}%
\pgfsetlinewidth{0.803000pt}%
\definecolor{currentstroke}{rgb}{0.000000,0.000000,0.000000}%
\pgfsetstrokecolor{currentstroke}%
\pgfsetdash{}{0pt}%
\pgfsys@defobject{currentmarker}{\pgfqpoint{0.000000in}{0.000000in}}{\pgfqpoint{0.048611in}{0.000000in}}{%
\pgfpathmoveto{\pgfqpoint{0.000000in}{0.000000in}}%
\pgfpathlineto{\pgfqpoint{0.048611in}{0.000000in}}%
\pgfusepath{stroke,fill}%
}%
\begin{pgfscope}%
\pgfsys@transformshift{4.668160in}{3.358144in}%
\pgfsys@useobject{currentmarker}{}%
\end{pgfscope}%
\end{pgfscope}%
\begin{pgfscope}%
\pgftext[x=4.765383in,y=3.305382in,left,base]{\rmfamily\fontsize{10.000000}{12.000000}\selectfont \(\displaystyle 1.0\)}%
\end{pgfscope}%
\begin{pgfscope}%
\pgftext[x=4.998408in,y=2.031603in,,top,rotate=90.000000]{\rmfamily\fontsize{10.000000}{12.000000}\selectfont \(\displaystyle \mathrm{\mathit{Bo_e}} \equiv \frac{\epsilon E_0^2 R_0}{\gamma}\)}%
\end{pgfscope}%
\begin{pgfscope}%
\pgfsetbuttcap%
\pgfsetmiterjoin%
\pgfsetlinewidth{0.803000pt}%
\definecolor{currentstroke}{rgb}{0.000000,0.000000,0.000000}%
\pgfsetstrokecolor{currentstroke}%
\pgfsetdash{}{0pt}%
\pgfpathmoveto{\pgfqpoint{4.517160in}{0.521603in}}%
\pgfpathlineto{\pgfqpoint{4.517160in}{0.533400in}}%
\pgfpathlineto{\pgfqpoint{4.517160in}{3.529806in}}%
\pgfpathlineto{\pgfqpoint{4.517160in}{3.541603in}}%
\pgfpathlineto{\pgfqpoint{4.668160in}{3.541603in}}%
\pgfpathlineto{\pgfqpoint{4.668160in}{3.529806in}}%
\pgfpathlineto{\pgfqpoint{4.668160in}{0.533400in}}%
\pgfpathlineto{\pgfqpoint{4.668160in}{0.521603in}}%
\pgfpathclose%
\pgfusepath{stroke}%
\end{pgfscope}%
\end{pgfpicture}%
\makeatother%
\endgroup%

    \caption{.\label{fig:restitution}}
\end{figure}

\newpage
\appendix
\section{\\Droplet Charge} \label{sec.drop_charge}
\subsection*{Parallel Plate Method}
Since, by the earlier scaling, we presuppose the source of the droplet bouncing behavior to be primarily Coulombic in origin (as opposed to dielectrophoretic), the droplet must have some free charge in addition to the charge induced by the electric field. Whether this free charge arises due to contact charge or field induction To measure this charge concurrent methodologies were used. We determined the droplet free charge by observation of the deflection of the droplets in the region of a known uniform field in a fashion inspired by Millikan's famous experiment to determine the fundamental charge of the electron.

Droplets were jumped in free-fall from a superhydrophobic surface placed between the plates of a parallel plate capacitor of known uniform electric field. The surface was charge neutralized. Since the droplet initial velocity $U_0$ is parallel to the electric field, the droplets are inertial in the direction of the electric force, and neglecting the effect of image charges mirrored across the conductors, we can determine the magnitude of the droplet charge by a balance of Coulombic force and inertia given by the equation of motion

\[ y'(t) = q\mathbf{E}. \]

Since the drag is negligible in the inertial limit we can find the charge $q$ by fitting a second-order least squares polynomial to the measured droplet positions, equating the $t^2$ term to the constant acceleration, and dividing by the known, constant magnitude of the electric field.  

A 200-880 VAC source with a full wave bridge rectifier circuit was prototyped on perf-board for initial experiments to measure droplet charge. The circuit was analyzed on an laboratory oscilloscope to verify that the AC component of the signal was appropriately small (13 mV at 35 kHz). Current was determined to be a relatively low 80 $\mu$A. The high-voltage source terminals were led to two parallel polished 150x150 mm aluminum plate electrodes. The electrodes were mounted on an insulated 80/20 extruded aluminum rail for ease of adjustment. All droplet charge experiments were conducted with an electrode spacing of 28.30 mm. With this spacing the calibrated electric field between the plates was $\mathbf{E} \approx 35$kV/m. The electrodes were electrically isolated from the drop rig by two alternating layers of 4 mm thick PMMA sheet and Kapton tape. Potential across the plates was measured periodically with a load-impedance corrected multimeter to account for battery depletion. The typical capacitor rise time of the plates was measured to be 1.4 s, thus to make the most economical use of the brief window a low-gravity a weighted switch was set by hand prior to the drop to close the high-voltage circuit, but which passively safed the system at the resumption of 1-g conditions in the tower. From a survey of literature we suppose the droplet charge, if they are indeed charged by contact with PTFE, to be some function of the droplet volume and the residence time on the superhydrophobic surface. However, sweeping though droplet volumes over a series of drop tower experiments we find little correlation between droplet volume  and free droplet charge.

A brief screening experiment was conducted which alternated the polarity of the field by switching the positive and negative terminal leads between plates. Qualitative observations of droplet electrode preference seem to indicate that the assumption of small polarization stress was well founded. Following this a orthogonal array $3^2$ factorial design experiment with two replicates was conducted to test the effect of varying droplet volume and surface stay time on free charge at the time of jumping. It was hypothesized in accordance with previous studies [ref], that free charge would increase for levels of both factors. ANOVA analysis in \emph{R} of the linear multiple regression model for the data set indicates that neither droplet volume ($p=0.105$), nor surface stay time ($p=0.358$) is significant at the 95\% confidence level. The overall model F-statistics (2.177 in 2 and 13 degrees of freedom), and coefficients of determination ($r^2 = 0.2509$) indicate that the linear model neither fits the data particularly well, nor does it offer an improvement over the mean model. The mean charge was determined to be positive $2.3 \cdot 10^{-11}$ C, with a standard deviation of $1.8 \cdot 10^{-11}$ C.

\printbibliography
\end{document}

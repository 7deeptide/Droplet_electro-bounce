 \documentclass[10pt,a4paper]{article}
\usepackage[utf8]{inputenc}
%\usepackage{fontspec} % This line only for XeLaTeX and LuaLaTeX
\usepackage{pgfplots}
\usepackage{pgf}
\usepackage[english]{babel}
\usepackage{amsmath}
\usepackage{amsfonts}
\usepackage{amssymb}
\usepackage{graphicx}
\usepackage{color,soul}
\usepackage{listings}
\usepackage{setspace}
\author{Erin Schmidt}

\newlength\figureheight
\newlength\figurewidth
\setlength\figureheight{7cm}
\setlength\figurewidth{10cm}

\begin{document}
\onehalfspacing
\section{Parameter Estimation}
A non-dimensional number: $ {\mathbb{E}\mbox{u}}_e$.

We find the parameters $\mathbf{x}$ that solve the inverse problem $G(\mathbf{x}) = \mathbf{d}$, using a direct search method (\emph{Nelder-Mead}). 
\[
\mbox{min} \hspace{2 mm} \chi^2 = \mbox{min} \hspace{2 mm} \sum^n_{i=1} \frac{\left({y_d(\mathbf{x})}_i - y_G(\mathbf{x})_i \right)^2}{y_G(\mathbf{x})_i}
\]
\begin{eqnarray*} \mbox{} \hspace{2 mm} \begin{split} \mathbf{x} = \left\{ \begin{array}{ll}      & q\\
		  &	V_d\\
          & \sigma 
          \end{array} \right. 
          \end{split} \hspace{2 mm} \mbox{subject to constraints} \hspace{2 mm} \begin{split}
          g = \left\{ \begin{array}{ll}
           V_d &\pm \hspace{2 mm} u_{exp}\\
      	   \sigma &\pm  \hspace{2 mm} u_{exp}\\
      	   y_0 &\pm \hspace{2 mm} u_{exp}\\
      	   t_0 &\pm \hspace{2 mm} u_{exp}\\
          \end{array} \right. 
          \end{split}
\end{eqnarray*}
where $y_G(\mathbf{x})$ is a numerical solution of the equation of motion
\[
m y'' = \frac{1}{2} \rho C_D A_d {y'}^2 + q E(y) + \frac{1}{4} \frac{K q^2}{y^2} + \frac{1}{2} {E(y)}^2 \nabla \epsilon\]

In particular, it is impractical to directly measure the droplet net charge $q$, during a bounce experiement. Our workflow to identify these parameters is as follows:
\begin{enumerate}
\item Experimentally vary $V_d$, $\sigma$ and capture droplet trajectories using a high-speed camera.
\item Digitize droplet trajectories by using automatic tracking of ellipse-fitted centroids on the thresholded video.
\item Slice droplet trajectories by their bounce minima, and apply a smoothing filter.
\item Maximize the goodness of fit between the experimental trajectories and solutions of the equation of motion, by varying the design vector (which encodes the unkown parameters in the equation of motion) using a direct search optimizer. 
\end{enumerate}
\subsection*{Inverse Problems}
\begin{itemize}
\item We have a dynamical model describing the droplet electro-bounce; we wish to find typical values of the key parameter, droplet net charge $q$. In general it is not always possible nor practicle to measure model parameters experimentally. They may be hard, expensive, time consuming or perhaps even impossible to measure. 
\item A mathematical model designed to fit experiemental data so as to explicitly quantify physical parameters of interest.
\item Values of model parameters are obtained using parameter estimation techniques aimed at providing a ``best fit'' to the data.
\item Generally involves an interative process to minimize the average difference between the model and the data.
\item Evaluating the quality of an inverse model involves a combination of established mathematical techniques as well as intuition and creative insight.
\item A good inverse model has: good fit, model parameters are unique, and parameter values consistent with physical inuition or prior knowledge.
\item The steps in the process are: 1. Select an appropriate mathematical model from theory. 2. Define a ``figure of merit'' function which measures agreement between the data and model for a given set of parameters. 3. Adjust model parameters to get a ``best fit''. 4. Evaluate ``goodness of fit'' to data, which tends never to be perfect due to experimental noise. 5. Estimate accuracy fo the best-fit parameter values (provide confidence intervals and determine uniqueness). 6. Determine whether a much better fit is possible (which is difficult of ruggest respose surfaces; F-test can be used for comparing models of different complexity).

\item Maximum likelihood estimation: not ``what is the probability that my set of model parameters is correct?'' but rather ``given my set of model parameters, what is the probability that this data set occurred (what is the likelihood of the parameters given the data)?''

\item \hl{[notes on $\chi^2$ goodness-of-fit]} 

\item Model parameter estimation is the process of indirect determination of unknown model parameters  from measurements of experiemental data.
\item This is mathemetically challenging \hl{[ref]}. There are practical challenges as well, as experimental data tends to be noisey and sparse.
\item In the most familliar sense determining model parameters is done using derivative based methods, essentially by estimating derivatives by finite differences (Euler's method) and fitting paramters using linear regression. However there are approaches to fitting any arbitrary function.
\item Suppose we have a model $G(\mathbf{m})$, with a vector of parameters $\mathbf{m}$, and set of perfect noiseless observations $\mathbf{d}$, we expect there to exist a relationship 
\[G(\mathbf{m}) = \mathbf{d} \]
where the operator $G$ might be an ODE.
\item Suppose we have a model given by the ODE
\[ \frac{dy}{dt} = f(t, \mathbf{y}; \mathbf{p}), \mathbf{y} \in \mathbb{R}^n, \mathbf{f} \in \mathbb{R}^n ,\]
where $\mathbf{p} \in \mathbb{R}^m$ is the vector of parameters, and a collection of measurements of experiemental data
\[ \left( t_1, \mathbf{y_1} \right), 
\left( t_2, \mathbf{y_2} \right), ... ,
\left( t_k, \mathbf{y_k} \right).\]
We wish to minimize an objection function which is the mean square error (or any goodness of fit figure of merit in general) between the model output and experimental data:
\[ \mbox{obj}(\mathbf{p}) = \sum_{j=1}^k \left|\mathbf{y}(t_j; \mathbf{p}) - \mathbf{y_j} \right|^2,\]
where $| . |$ is the Euclidian norm. Therefore the parameter estimation problem is a variety of optimization problem.
\item The process of fitting a function, defined by a collection of parameters, to a data set is called the discrete inverse, or parameter estimation problem (as opposed to the \emph{forward problem} to find $\mathbf{d}$ given $\mathbf{m}$ and $G(\mathbf{m})$).
\end{itemize}

\subsection*{Optimization}
Most generally the we state the constrained problem as 
\[
\begin{array}{lll}
\mbox{minimize:} & \hspace{2 mm} F\left(\mathbf{x}\right) & \mbox{objective function}\\
\mbox{subject to:} & & \\
& \centering g_j \left( \mathbf{x} \right) \leq 0 & \mbox{inequality constraints}\\
& \centering h_k \left(\mathbf{x} \right) = 0 & \mbox{equality constraints}
\end{array}
 \]

\[ 
\begin{array}{ll}
\mbox{where} \hspace{2 mm} \mathbf{x} = \left\{ \begin{array}{ll}

x_1 & \\
x_2 & \\
\vdots &\\
x_n &
\end{array} \right. & \mbox{design variables}
\end{array}
\]

Our specific optimization problem is non-convex, mixed discrete-continuously black-box (noisy), which is essentially the worst the worst case scenario. While in principle a gradient-based optimizer (such as ...) could be used by using finite-differences to obtain approximate gradients of the $\chi^2$ objective funtion, in practice doing so is problematic becasue the signal-to-noise ratio of the objective function scales like $\mathcal{O}(f)$ for $\frac{df}{dt}$ and $\mathcal{O}(f^2)$ for $\frac{d^2f}{ft^2}$ which will cause problems with our Hessians and Javobians respectively \hl{[Refs 5, 14, 49, 95.]}. As a further practical matter, given the relatively expensive function-calls (which requires solving a stiff, non-linear ODE) gradient-free approaches tend to offer better performence regardless. 

\hl{[notes on gradient free optimization, Nelder-mead, sci-py ref]}
\begin{itemize}
\item We use the \emph{Nelder-Mead} algorithem to minimize the $\chi^2$ goodness-of-fit. \emph{Nelder-Mead}, sometimes called simplex-search or downhill-simplex, is a derivative-free direct search, LP method.
\item \emph{Nelder-Mead} is well-established.
\item A hueristic search method, with no guarantee of optimal solutions.
\item Is based on the concept of a simplex, which is a special polytrop of $N + 1$ vertices in $N$ diemsions.
\item Is derivative-free: does not use numerical or analytic gradients.
\item Is implimented in the \emph{Scipy} package for Scientific computing in Python.
\item \emph{Nelder-Mead} is very fast, but generally suitable for low-dimensional problems. Additionally convergence of the alogrithem is highly sensitive to the initial guess design vector $x_0$.
\end{itemize}

We do not have \emph{a priori} information regarding existence of true globality for our objective function. Perhaps more importantly, given the degrees-of-freedom in the parameter space there may be a multiplicity of local extrema of the $\chi^2$ hyper-response surface. \emph{Nelder-Mead} is not a global optimizer, though there are varients which use sequential local searches with probablistic restarts to achieve globality. However global optimization usually comes at a tremendous computational cost.

That we are capabible of fitting any arbitrary model to a dataset given sufficient degrees of freedom in our parameters is admittedly a disconcerting issue, and is manifested in the locality of the extrema of the $\chi^2$ response surface. However, some of the inverse model parameters are constrained by our experimental observations of them and their associated measurement uncertainties. This, we hope, makes the spectre of an overfitted model less frightening, but does convert our uncontrained optimization problem to an constrained one which raises special difficulties of its own. 

The \emph{Nelder-Mead} direct search method cannot be used with explicity constrained problems. However, there are various implicit approaches to (approximately) solving general constrained problems using uncontrained algorithems. Generally, this is achieved by domain transformations or the use of penalty functions.  By the addition of a penalty function which depends in some way on the values of the constraints to the objective function, we minimize a pseudo-objective function where the infeasibility of the constraints is minimized simultaneously to the objective function. 

There are various penalty function schemes. We use an Exterior Penalty Function as a simple way of converting converting the constrained problem into an unconstrained one. These are are especially useful in cases where the constraints are not ``hard'' in the sense that they need to be satisfied precisely. General penalty functions, which are sequential unconstrained minimization techniques, reformulate the general constrained problem as the pseudo-objective function given by

\[ \phi(\mathbf{x}, r_p ) = F(\mathbf{x}) + r_p P(\mathbf{x}) \]
where $P \left( \mathbf{x} \right)$ the penalty function, is given by
\begin{equation} \label{exterior}
 P( \mathbf{x} ) = \sum_{j = 1}^m \left\lbrace \mbox{max} \left[ 0, g_j(\mathbf{x} ) \right] \right\rbrace^2 + 
\sum_{k = 1}^l \left[ h_k( \mathbf{x}) \right]^2 .
\end{equation}
We see from Equation \ref{exterior} that there is no penalty if the constraints $g_j(\mathbf{x})$, and $h_k(\mathbf{x})$ are satisfied.
 
The Exterior Penalty Function specifically (and all Penalty Function approaches in general) do have several drawbacks. Namely these include the possibility of the objective function being undefined outside of the set of feasible solutions. Additionaly, by naively ``encouraging'' feasibility of the solution using large values of the penalty parameter, $r_p$, we will tend to ill-condition the unconstrained formulation of the problem.
\hl{[notes on choice of optimizer parameters, initial guesses, convergence, error estimates]}. Using penalty function methods tend to exascerbate the sensitivity of \emph{Nelder-Mead} solutions to the choice of intial guess vector \hl{[ref]}. 

\subsection*{Smoothing}
In trying to recover the kinematic variables from the droplet trajectories we encounter a difficulty in the noise of the position data:

Error sources include misalignment of the camera, perspective due to objects (subject or reference scale) being out of the photographic plane, precision limits in digitization. Some of these errors are systematic in origin and introduce consistent biases into the data (e.g. coherent spectral sources, rather than truly stochastic noise). These systematic sources of error include inaccurate scales among others. Data smoothing tends not to help with the systematic errors in that they are usually of lower frequency than the signal (and here we are trying specifically to filter high frequency noise). Random errors, by contrast, are assumed to have a Gaussian distribution (by the central limit theorem), and are independent of the signal (which inherently results from a deterministic process).

We experimented with a variety of filters implimented in the \verb|scipy.signal| \emph{SciPy}[ref] module using a represenative set of trajectory data; these methods include 1D Gaussian convolution, Wiener, Butterworth, and Savitsky-Golay filters. Qualitatively comparing these smoothing methods (by handtuning filter orders and window sizes) we find that we loose too many data points in the smoothing process, large amplitudes are overly smoothed by repeated filtering passes, or there are significant end effects for most of these methods. A comparison of these smoothing approaches on a representative trajectory data set are shown in Figure \ref{fig:filters}. The power spectra for the same data are compared for these methods in Figure \ref{fig:power_spectra}.

%\begin{figure}[htb]
%\centering
%\resizebox{10cm}{!}{%% Creator: Matplotlib, PGF backend
%%
%% To include the figure in your LaTeX document, write
%%   \input{<filename>.pgf}
%%
%% Make sure the required packages are loaded in your preamble
%%   \usepackage{pgf}
%%
%% Figures using additional raster images can only be included by \input if
%% they are in the same directory as the main LaTeX file. For loading figures
%% from other directories you can use the `import` package
%%   \usepackage{import}
%% and then include the figures with
%%   \import{<path to file>}{<filename>.pgf}
%%
%% Matplotlib used the following preamble
%%   \usepackage{fontspec}
%%   \setmainfont{DejaVu Serif}
%%   \setsansfont{DejaVu Sans}
%%   \setmonofont{DejaVu Sans Mono}
%%
\begingroup%
\makeatletter%
\begin{pgfpicture}%
\pgfpathrectangle{\pgfpointorigin}{\pgfqpoint{5.066664in}{3.471851in}}%
\pgfusepath{use as bounding box, clip}%
\begin{pgfscope}%
\pgfsetbuttcap%
\pgfsetmiterjoin%
\definecolor{currentfill}{rgb}{1.000000,1.000000,1.000000}%
\pgfsetfillcolor{currentfill}%
\pgfsetlinewidth{0.000000pt}%
\definecolor{currentstroke}{rgb}{1.000000,1.000000,1.000000}%
\pgfsetstrokecolor{currentstroke}%
\pgfsetdash{}{0pt}%
\pgfpathmoveto{\pgfqpoint{0.000000in}{0.000000in}}%
\pgfpathlineto{\pgfqpoint{5.066664in}{0.000000in}}%
\pgfpathlineto{\pgfqpoint{5.066664in}{3.471851in}}%
\pgfpathlineto{\pgfqpoint{0.000000in}{3.471851in}}%
\pgfpathclose%
\pgfusepath{fill}%
\end{pgfscope}%
\begin{pgfscope}%
\pgfsetbuttcap%
\pgfsetmiterjoin%
\definecolor{currentfill}{rgb}{1.000000,1.000000,1.000000}%
\pgfsetfillcolor{currentfill}%
\pgfsetlinewidth{0.000000pt}%
\definecolor{currentstroke}{rgb}{0.000000,0.000000,0.000000}%
\pgfsetstrokecolor{currentstroke}%
\pgfsetstrokeopacity{0.000000}%
\pgfsetdash{}{0pt}%
\pgfpathmoveto{\pgfqpoint{0.281664in}{0.316851in}}%
\pgfpathlineto{\pgfqpoint{4.931664in}{0.316851in}}%
\pgfpathlineto{\pgfqpoint{4.931664in}{3.336851in}}%
\pgfpathlineto{\pgfqpoint{0.281664in}{3.336851in}}%
\pgfpathclose%
\pgfusepath{fill}%
\end{pgfscope}%
\begin{pgfscope}%
\pgftext[x=2.606664in,y=0.261295in,,top]{\rmfamily\fontsize{12.000000}{14.400000}\selectfont t}%
\end{pgfscope}%
\begin{pgfscope}%
\pgftext[x=0.142775in,y=1.826851in,,bottom]{\rmfamily\fontsize{12.000000}{14.400000}\selectfont \(\displaystyle y\)}%
\end{pgfscope}%
\begin{pgfscope}%
\pgfpathrectangle{\pgfqpoint{0.281664in}{0.316851in}}{\pgfqpoint{4.650000in}{3.020000in}} %
\pgfusepath{clip}%
\pgfsetbuttcap%
\pgfsetroundjoin%
\definecolor{currentfill}{rgb}{0.000000,0.000000,0.000000}%
\pgfsetfillcolor{currentfill}%
\pgfsetlinewidth{1.003750pt}%
\definecolor{currentstroke}{rgb}{0.000000,0.000000,0.000000}%
\pgfsetstrokecolor{currentstroke}%
\pgfsetdash{}{0pt}%
\pgfsys@defobject{currentmarker}{\pgfqpoint{-0.020833in}{-0.020833in}}{\pgfqpoint{0.020833in}{0.020833in}}{%
\pgfpathmoveto{\pgfqpoint{0.000000in}{-0.020833in}}%
\pgfpathcurveto{\pgfqpoint{0.005525in}{-0.020833in}}{\pgfqpoint{0.010825in}{-0.018638in}}{\pgfqpoint{0.014731in}{-0.014731in}}%
\pgfpathcurveto{\pgfqpoint{0.018638in}{-0.010825in}}{\pgfqpoint{0.020833in}{-0.005525in}}{\pgfqpoint{0.020833in}{0.000000in}}%
\pgfpathcurveto{\pgfqpoint{0.020833in}{0.005525in}}{\pgfqpoint{0.018638in}{0.010825in}}{\pgfqpoint{0.014731in}{0.014731in}}%
\pgfpathcurveto{\pgfqpoint{0.010825in}{0.018638in}}{\pgfqpoint{0.005525in}{0.020833in}}{\pgfqpoint{0.000000in}{0.020833in}}%
\pgfpathcurveto{\pgfqpoint{-0.005525in}{0.020833in}}{\pgfqpoint{-0.010825in}{0.018638in}}{\pgfqpoint{-0.014731in}{0.014731in}}%
\pgfpathcurveto{\pgfqpoint{-0.018638in}{0.010825in}}{\pgfqpoint{-0.020833in}{0.005525in}}{\pgfqpoint{-0.020833in}{0.000000in}}%
\pgfpathcurveto{\pgfqpoint{-0.020833in}{-0.005525in}}{\pgfqpoint{-0.018638in}{-0.010825in}}{\pgfqpoint{-0.014731in}{-0.014731in}}%
\pgfpathcurveto{\pgfqpoint{-0.010825in}{-0.018638in}}{\pgfqpoint{-0.005525in}{-0.020833in}}{\pgfqpoint{0.000000in}{-0.020833in}}%
\pgfpathclose%
\pgfusepath{stroke,fill}%
}%
\begin{pgfscope}%
\pgfsys@transformshift{0.493028in}{0.928679in}%
\pgfsys@useobject{currentmarker}{}%
\end{pgfscope}%
\begin{pgfscope}%
\pgfsys@transformshift{0.528852in}{1.000537in}%
\pgfsys@useobject{currentmarker}{}%
\end{pgfscope}%
\begin{pgfscope}%
\pgfsys@transformshift{0.564677in}{1.052538in}%
\pgfsys@useobject{currentmarker}{}%
\end{pgfscope}%
\begin{pgfscope}%
\pgfsys@transformshift{0.600501in}{1.137040in}%
\pgfsys@useobject{currentmarker}{}%
\end{pgfscope}%
\begin{pgfscope}%
\pgfsys@transformshift{0.636325in}{1.250622in}%
\pgfsys@useobject{currentmarker}{}%
\end{pgfscope}%
\begin{pgfscope}%
\pgfsys@transformshift{0.672150in}{1.335634in}%
\pgfsys@useobject{currentmarker}{}%
\end{pgfscope}%
\begin{pgfscope}%
\pgfsys@transformshift{0.707974in}{1.420754in}%
\pgfsys@useobject{currentmarker}{}%
\end{pgfscope}%
\begin{pgfscope}%
\pgfsys@transformshift{0.743798in}{1.470554in}%
\pgfsys@useobject{currentmarker}{}%
\end{pgfscope}%
\begin{pgfscope}%
\pgfsys@transformshift{0.779623in}{1.514713in}%
\pgfsys@useobject{currentmarker}{}%
\end{pgfscope}%
\begin{pgfscope}%
\pgfsys@transformshift{0.815447in}{1.568129in}%
\pgfsys@useobject{currentmarker}{}%
\end{pgfscope}%
\begin{pgfscope}%
\pgfsys@transformshift{0.851271in}{1.649470in}%
\pgfsys@useobject{currentmarker}{}%
\end{pgfscope}%
\begin{pgfscope}%
\pgfsys@transformshift{0.887096in}{1.737655in}%
\pgfsys@useobject{currentmarker}{}%
\end{pgfscope}%
\begin{pgfscope}%
\pgfsys@transformshift{0.922920in}{1.829330in}%
\pgfsys@useobject{currentmarker}{}%
\end{pgfscope}%
\begin{pgfscope}%
\pgfsys@transformshift{0.958744in}{1.883428in}%
\pgfsys@useobject{currentmarker}{}%
\end{pgfscope}%
\begin{pgfscope}%
\pgfsys@transformshift{0.994569in}{1.910614in}%
\pgfsys@useobject{currentmarker}{}%
\end{pgfscope}%
\begin{pgfscope}%
\pgfsys@transformshift{1.030393in}{1.962106in}%
\pgfsys@useobject{currentmarker}{}%
\end{pgfscope}%
\begin{pgfscope}%
\pgfsys@transformshift{1.066217in}{2.030846in}%
\pgfsys@useobject{currentmarker}{}%
\end{pgfscope}%
\begin{pgfscope}%
\pgfsys@transformshift{1.102042in}{2.111746in}%
\pgfsys@useobject{currentmarker}{}%
\end{pgfscope}%
\begin{pgfscope}%
\pgfsys@transformshift{1.137866in}{2.172149in}%
\pgfsys@useobject{currentmarker}{}%
\end{pgfscope}%
\begin{pgfscope}%
\pgfsys@transformshift{1.173690in}{2.201081in}%
\pgfsys@useobject{currentmarker}{}%
\end{pgfscope}%
\begin{pgfscope}%
\pgfsys@transformshift{1.209515in}{2.225526in}%
\pgfsys@useobject{currentmarker}{}%
\end{pgfscope}%
\begin{pgfscope}%
\pgfsys@transformshift{1.245339in}{2.259815in}%
\pgfsys@useobject{currentmarker}{}%
\end{pgfscope}%
\begin{pgfscope}%
\pgfsys@transformshift{1.281163in}{2.309263in}%
\pgfsys@useobject{currentmarker}{}%
\end{pgfscope}%
\begin{pgfscope}%
\pgfsys@transformshift{1.316988in}{2.385577in}%
\pgfsys@useobject{currentmarker}{}%
\end{pgfscope}%
\begin{pgfscope}%
\pgfsys@transformshift{1.352812in}{2.452705in}%
\pgfsys@useobject{currentmarker}{}%
\end{pgfscope}%
\begin{pgfscope}%
\pgfsys@transformshift{1.388637in}{2.481667in}%
\pgfsys@useobject{currentmarker}{}%
\end{pgfscope}%
\begin{pgfscope}%
\pgfsys@transformshift{1.424461in}{2.494290in}%
\pgfsys@useobject{currentmarker}{}%
\end{pgfscope}%
\begin{pgfscope}%
\pgfsys@transformshift{1.460285in}{2.526482in}%
\pgfsys@useobject{currentmarker}{}%
\end{pgfscope}%
\begin{pgfscope}%
\pgfsys@transformshift{1.496110in}{2.581446in}%
\pgfsys@useobject{currentmarker}{}%
\end{pgfscope}%
\begin{pgfscope}%
\pgfsys@transformshift{1.531934in}{2.642921in}%
\pgfsys@useobject{currentmarker}{}%
\end{pgfscope}%
\begin{pgfscope}%
\pgfsys@transformshift{1.567758in}{2.680748in}%
\pgfsys@useobject{currentmarker}{}%
\end{pgfscope}%
\begin{pgfscope}%
\pgfsys@transformshift{1.603583in}{2.703974in}%
\pgfsys@useobject{currentmarker}{}%
\end{pgfscope}%
\begin{pgfscope}%
\pgfsys@transformshift{1.639407in}{2.708222in}%
\pgfsys@useobject{currentmarker}{}%
\end{pgfscope}%
\begin{pgfscope}%
\pgfsys@transformshift{1.675231in}{2.722180in}%
\pgfsys@useobject{currentmarker}{}%
\end{pgfscope}%
\begin{pgfscope}%
\pgfsys@transformshift{1.711056in}{2.766227in}%
\pgfsys@useobject{currentmarker}{}%
\end{pgfscope}%
\begin{pgfscope}%
\pgfsys@transformshift{1.746880in}{2.820876in}%
\pgfsys@useobject{currentmarker}{}%
\end{pgfscope}%
\begin{pgfscope}%
\pgfsys@transformshift{1.782704in}{2.865133in}%
\pgfsys@useobject{currentmarker}{}%
\end{pgfscope}%
\begin{pgfscope}%
\pgfsys@transformshift{1.818529in}{2.879349in}%
\pgfsys@useobject{currentmarker}{}%
\end{pgfscope}%
\begin{pgfscope}%
\pgfsys@transformshift{1.854353in}{2.879949in}%
\pgfsys@useobject{currentmarker}{}%
\end{pgfscope}%
\begin{pgfscope}%
\pgfsys@transformshift{1.890177in}{2.898561in}%
\pgfsys@useobject{currentmarker}{}%
\end{pgfscope}%
\begin{pgfscope}%
\pgfsys@transformshift{1.926002in}{2.941603in}%
\pgfsys@useobject{currentmarker}{}%
\end{pgfscope}%
\begin{pgfscope}%
\pgfsys@transformshift{1.961826in}{2.983568in}%
\pgfsys@useobject{currentmarker}{}%
\end{pgfscope}%
\begin{pgfscope}%
\pgfsys@transformshift{1.997650in}{3.012917in}%
\pgfsys@useobject{currentmarker}{}%
\end{pgfscope}%
\begin{pgfscope}%
\pgfsys@transformshift{2.033475in}{3.015178in}%
\pgfsys@useobject{currentmarker}{}%
\end{pgfscope}%
\begin{pgfscope}%
\pgfsys@transformshift{2.069299in}{3.008037in}%
\pgfsys@useobject{currentmarker}{}%
\end{pgfscope}%
\begin{pgfscope}%
\pgfsys@transformshift{2.105123in}{3.014275in}%
\pgfsys@useobject{currentmarker}{}%
\end{pgfscope}%
\begin{pgfscope}%
\pgfsys@transformshift{2.140948in}{3.043740in}%
\pgfsys@useobject{currentmarker}{}%
\end{pgfscope}%
\begin{pgfscope}%
\pgfsys@transformshift{2.176772in}{3.084525in}%
\pgfsys@useobject{currentmarker}{}%
\end{pgfscope}%
\begin{pgfscope}%
\pgfsys@transformshift{2.212596in}{3.105812in}%
\pgfsys@useobject{currentmarker}{}%
\end{pgfscope}%
\begin{pgfscope}%
\pgfsys@transformshift{2.248421in}{3.105416in}%
\pgfsys@useobject{currentmarker}{}%
\end{pgfscope}%
\begin{pgfscope}%
\pgfsys@transformshift{2.284245in}{3.096852in}%
\pgfsys@useobject{currentmarker}{}%
\end{pgfscope}%
\begin{pgfscope}%
\pgfsys@transformshift{2.320069in}{3.106378in}%
\pgfsys@useobject{currentmarker}{}%
\end{pgfscope}%
\begin{pgfscope}%
\pgfsys@transformshift{2.355894in}{3.137411in}%
\pgfsys@useobject{currentmarker}{}%
\end{pgfscope}%
\begin{pgfscope}%
\pgfsys@transformshift{2.391718in}{3.169125in}%
\pgfsys@useobject{currentmarker}{}%
\end{pgfscope}%
\begin{pgfscope}%
\pgfsys@transformshift{2.427543in}{3.176838in}%
\pgfsys@useobject{currentmarker}{}%
\end{pgfscope}%
\begin{pgfscope}%
\pgfsys@transformshift{2.463367in}{3.166474in}%
\pgfsys@useobject{currentmarker}{}%
\end{pgfscope}%
\begin{pgfscope}%
\pgfsys@transformshift{2.499191in}{3.151557in}%
\pgfsys@useobject{currentmarker}{}%
\end{pgfscope}%
\begin{pgfscope}%
\pgfsys@transformshift{2.535016in}{3.147031in}%
\pgfsys@useobject{currentmarker}{}%
\end{pgfscope}%
\begin{pgfscope}%
\pgfsys@transformshift{2.570840in}{3.165847in}%
\pgfsys@useobject{currentmarker}{}%
\end{pgfscope}%
\begin{pgfscope}%
\pgfsys@transformshift{2.606664in}{3.195505in}%
\pgfsys@useobject{currentmarker}{}%
\end{pgfscope}%
\begin{pgfscope}%
\pgfsys@transformshift{2.642489in}{3.196283in}%
\pgfsys@useobject{currentmarker}{}%
\end{pgfscope}%
\begin{pgfscope}%
\pgfsys@transformshift{2.678313in}{3.181265in}%
\pgfsys@useobject{currentmarker}{}%
\end{pgfscope}%
\begin{pgfscope}%
\pgfsys@transformshift{2.714137in}{3.162399in}%
\pgfsys@useobject{currentmarker}{}%
\end{pgfscope}%
\begin{pgfscope}%
\pgfsys@transformshift{2.749962in}{3.166078in}%
\pgfsys@useobject{currentmarker}{}%
\end{pgfscope}%
\begin{pgfscope}%
\pgfsys@transformshift{2.785786in}{3.186534in}%
\pgfsys@useobject{currentmarker}{}%
\end{pgfscope}%
\begin{pgfscope}%
\pgfsys@transformshift{2.821610in}{3.199578in}%
\pgfsys@useobject{currentmarker}{}%
\end{pgfscope}%
\begin{pgfscope}%
\pgfsys@transformshift{2.857435in}{3.196022in}%
\pgfsys@useobject{currentmarker}{}%
\end{pgfscope}%
\begin{pgfscope}%
\pgfsys@transformshift{2.893259in}{3.172912in}%
\pgfsys@useobject{currentmarker}{}%
\end{pgfscope}%
\begin{pgfscope}%
\pgfsys@transformshift{2.929083in}{3.144159in}%
\pgfsys@useobject{currentmarker}{}%
\end{pgfscope}%
\begin{pgfscope}%
\pgfsys@transformshift{2.964908in}{3.134666in}%
\pgfsys@useobject{currentmarker}{}%
\end{pgfscope}%
\begin{pgfscope}%
\pgfsys@transformshift{3.000732in}{3.144276in}%
\pgfsys@useobject{currentmarker}{}%
\end{pgfscope}%
\begin{pgfscope}%
\pgfsys@transformshift{3.036556in}{3.155500in}%
\pgfsys@useobject{currentmarker}{}%
\end{pgfscope}%
\begin{pgfscope}%
\pgfsys@transformshift{3.072381in}{3.144574in}%
\pgfsys@useobject{currentmarker}{}%
\end{pgfscope}%
\begin{pgfscope}%
\pgfsys@transformshift{3.108205in}{3.110535in}%
\pgfsys@useobject{currentmarker}{}%
\end{pgfscope}%
\begin{pgfscope}%
\pgfsys@transformshift{3.144029in}{3.088562in}%
\pgfsys@useobject{currentmarker}{}%
\end{pgfscope}%
\begin{pgfscope}%
\pgfsys@transformshift{3.179854in}{3.082270in}%
\pgfsys@useobject{currentmarker}{}%
\end{pgfscope}%
\begin{pgfscope}%
\pgfsys@transformshift{3.215678in}{3.088172in}%
\pgfsys@useobject{currentmarker}{}%
\end{pgfscope}%
\begin{pgfscope}%
\pgfsys@transformshift{3.251502in}{3.088947in}%
\pgfsys@useobject{currentmarker}{}%
\end{pgfscope}%
\begin{pgfscope}%
\pgfsys@transformshift{3.287327in}{3.068084in}%
\pgfsys@useobject{currentmarker}{}%
\end{pgfscope}%
\begin{pgfscope}%
\pgfsys@transformshift{3.323151in}{3.030370in}%
\pgfsys@useobject{currentmarker}{}%
\end{pgfscope}%
\begin{pgfscope}%
\pgfsys@transformshift{3.358975in}{2.995229in}%
\pgfsys@useobject{currentmarker}{}%
\end{pgfscope}%
\begin{pgfscope}%
\pgfsys@transformshift{3.394800in}{2.977707in}%
\pgfsys@useobject{currentmarker}{}%
\end{pgfscope}%
\begin{pgfscope}%
\pgfsys@transformshift{3.430624in}{2.977950in}%
\pgfsys@useobject{currentmarker}{}%
\end{pgfscope}%
\begin{pgfscope}%
\pgfsys@transformshift{3.466449in}{2.967165in}%
\pgfsys@useobject{currentmarker}{}%
\end{pgfscope}%
\begin{pgfscope}%
\pgfsys@transformshift{3.502273in}{2.939776in}%
\pgfsys@useobject{currentmarker}{}%
\end{pgfscope}%
\begin{pgfscope}%
\pgfsys@transformshift{3.538097in}{2.896261in}%
\pgfsys@useobject{currentmarker}{}%
\end{pgfscope}%
\begin{pgfscope}%
\pgfsys@transformshift{3.573922in}{2.862819in}%
\pgfsys@useobject{currentmarker}{}%
\end{pgfscope}%
\begin{pgfscope}%
\pgfsys@transformshift{3.609746in}{2.847574in}%
\pgfsys@useobject{currentmarker}{}%
\end{pgfscope}%
\begin{pgfscope}%
\pgfsys@transformshift{3.645570in}{2.842148in}%
\pgfsys@useobject{currentmarker}{}%
\end{pgfscope}%
\begin{pgfscope}%
\pgfsys@transformshift{3.681395in}{2.822805in}%
\pgfsys@useobject{currentmarker}{}%
\end{pgfscope}%
\begin{pgfscope}%
\pgfsys@transformshift{3.717219in}{2.784820in}%
\pgfsys@useobject{currentmarker}{}%
\end{pgfscope}%
\begin{pgfscope}%
\pgfsys@transformshift{3.753043in}{2.735235in}%
\pgfsys@useobject{currentmarker}{}%
\end{pgfscope}%
\begin{pgfscope}%
\pgfsys@transformshift{3.788868in}{2.688874in}%
\pgfsys@useobject{currentmarker}{}%
\end{pgfscope}%
\begin{pgfscope}%
\pgfsys@transformshift{3.824692in}{2.665390in}%
\pgfsys@useobject{currentmarker}{}%
\end{pgfscope}%
\begin{pgfscope}%
\pgfsys@transformshift{3.860516in}{2.651989in}%
\pgfsys@useobject{currentmarker}{}%
\end{pgfscope}%
\begin{pgfscope}%
\pgfsys@transformshift{3.896341in}{2.624050in}%
\pgfsys@useobject{currentmarker}{}%
\end{pgfscope}%
\begin{pgfscope}%
\pgfsys@transformshift{3.932165in}{2.573173in}%
\pgfsys@useobject{currentmarker}{}%
\end{pgfscope}%
\begin{pgfscope}%
\pgfsys@transformshift{3.967989in}{2.516050in}%
\pgfsys@useobject{currentmarker}{}%
\end{pgfscope}%
\begin{pgfscope}%
\pgfsys@transformshift{4.003814in}{2.471636in}%
\pgfsys@useobject{currentmarker}{}%
\end{pgfscope}%
\begin{pgfscope}%
\pgfsys@transformshift{4.039638in}{2.446527in}%
\pgfsys@useobject{currentmarker}{}%
\end{pgfscope}%
\begin{pgfscope}%
\pgfsys@transformshift{4.075462in}{2.420950in}%
\pgfsys@useobject{currentmarker}{}%
\end{pgfscope}%
\begin{pgfscope}%
\pgfsys@transformshift{4.111287in}{2.383679in}%
\pgfsys@useobject{currentmarker}{}%
\end{pgfscope}%
\begin{pgfscope}%
\pgfsys@transformshift{4.147111in}{2.326908in}%
\pgfsys@useobject{currentmarker}{}%
\end{pgfscope}%
\begin{pgfscope}%
\pgfsys@transformshift{4.182935in}{2.258851in}%
\pgfsys@useobject{currentmarker}{}%
\end{pgfscope}%
\begin{pgfscope}%
\pgfsys@transformshift{4.218760in}{2.205708in}%
\pgfsys@useobject{currentmarker}{}%
\end{pgfscope}%
\begin{pgfscope}%
\pgfsys@transformshift{4.254584in}{2.170752in}%
\pgfsys@useobject{currentmarker}{}%
\end{pgfscope}%
\begin{pgfscope}%
\pgfsys@transformshift{4.290408in}{2.137112in}%
\pgfsys@useobject{currentmarker}{}%
\end{pgfscope}%
\begin{pgfscope}%
\pgfsys@transformshift{4.326233in}{2.085140in}%
\pgfsys@useobject{currentmarker}{}%
\end{pgfscope}%
\begin{pgfscope}%
\pgfsys@transformshift{4.362057in}{2.010225in}%
\pgfsys@useobject{currentmarker}{}%
\end{pgfscope}%
\begin{pgfscope}%
\pgfsys@transformshift{4.397881in}{1.938990in}%
\pgfsys@useobject{currentmarker}{}%
\end{pgfscope}%
\begin{pgfscope}%
\pgfsys@transformshift{4.433706in}{1.881144in}%
\pgfsys@useobject{currentmarker}{}%
\end{pgfscope}%
\begin{pgfscope}%
\pgfsys@transformshift{4.469530in}{1.836555in}%
\pgfsys@useobject{currentmarker}{}%
\end{pgfscope}%
\begin{pgfscope}%
\pgfsys@transformshift{4.505355in}{1.791362in}%
\pgfsys@useobject{currentmarker}{}%
\end{pgfscope}%
\begin{pgfscope}%
\pgfsys@transformshift{4.541179in}{1.729299in}%
\pgfsys@useobject{currentmarker}{}%
\end{pgfscope}%
\begin{pgfscope}%
\pgfsys@transformshift{4.577003in}{1.647511in}%
\pgfsys@useobject{currentmarker}{}%
\end{pgfscope}%
\begin{pgfscope}%
\pgfsys@transformshift{4.612828in}{1.561739in}%
\pgfsys@useobject{currentmarker}{}%
\end{pgfscope}%
\begin{pgfscope}%
\pgfsys@transformshift{4.648652in}{1.492396in}%
\pgfsys@useobject{currentmarker}{}%
\end{pgfscope}%
\begin{pgfscope}%
\pgfsys@transformshift{4.684476in}{1.436963in}%
\pgfsys@useobject{currentmarker}{}%
\end{pgfscope}%
\begin{pgfscope}%
\pgfsys@transformshift{4.720301in}{1.377489in}%
\pgfsys@useobject{currentmarker}{}%
\end{pgfscope}%
\end{pgfscope}%
\begin{pgfscope}%
\pgfpathrectangle{\pgfqpoint{0.281664in}{0.316851in}}{\pgfqpoint{4.650000in}{3.020000in}} %
\pgfusepath{clip}%
\pgfsetrectcap%
\pgfsetroundjoin%
\pgfsetlinewidth{1.505625pt}%
\definecolor{currentstroke}{rgb}{0.750000,0.000000,0.750000}%
\pgfsetstrokecolor{currentstroke}%
\pgfsetdash{}{0pt}%
\pgfpathmoveto{\pgfqpoint{0.493028in}{1.416680in}}%
\pgfpathlineto{\pgfqpoint{0.528852in}{1.421773in}}%
\pgfpathlineto{\pgfqpoint{0.564677in}{1.432125in}}%
\pgfpathlineto{\pgfqpoint{0.600501in}{1.447784in}}%
\pgfpathlineto{\pgfqpoint{0.636325in}{1.468633in}}%
\pgfpathlineto{\pgfqpoint{0.672150in}{1.494275in}}%
\pgfpathlineto{\pgfqpoint{0.707974in}{1.524039in}}%
\pgfpathlineto{\pgfqpoint{0.743798in}{1.557636in}}%
\pgfpathlineto{\pgfqpoint{0.779623in}{1.594996in}}%
\pgfpathlineto{\pgfqpoint{0.815447in}{1.635800in}}%
\pgfpathlineto{\pgfqpoint{0.851271in}{1.679629in}}%
\pgfpathlineto{\pgfqpoint{0.887096in}{1.725759in}}%
\pgfpathlineto{\pgfqpoint{0.922920in}{1.773646in}}%
\pgfpathlineto{\pgfqpoint{0.958744in}{1.822804in}}%
\pgfpathlineto{\pgfqpoint{0.994569in}{1.873101in}}%
\pgfpathlineto{\pgfqpoint{1.030393in}{1.924337in}}%
\pgfpathlineto{\pgfqpoint{1.066217in}{1.976417in}}%
\pgfpathlineto{\pgfqpoint{1.102042in}{2.028795in}}%
\pgfpathlineto{\pgfqpoint{1.137866in}{2.080832in}}%
\pgfpathlineto{\pgfqpoint{1.173690in}{2.132334in}}%
\pgfpathlineto{\pgfqpoint{1.209515in}{2.182792in}}%
\pgfpathlineto{\pgfqpoint{1.245339in}{2.231869in}}%
\pgfpathlineto{\pgfqpoint{1.281163in}{2.279704in}}%
\pgfpathlineto{\pgfqpoint{1.316988in}{2.326041in}}%
\pgfpathlineto{\pgfqpoint{1.352812in}{2.370577in}}%
\pgfpathlineto{\pgfqpoint{1.388637in}{2.413453in}}%
\pgfpathlineto{\pgfqpoint{1.424461in}{2.454756in}}%
\pgfpathlineto{\pgfqpoint{1.460285in}{2.494857in}}%
\pgfpathlineto{\pgfqpoint{1.496110in}{2.533873in}}%
\pgfpathlineto{\pgfqpoint{1.531934in}{2.571640in}}%
\pgfpathlineto{\pgfqpoint{1.567758in}{2.607869in}}%
\pgfpathlineto{\pgfqpoint{1.603583in}{2.642485in}}%
\pgfpathlineto{\pgfqpoint{1.639407in}{2.675613in}}%
\pgfpathlineto{\pgfqpoint{1.675231in}{2.707655in}}%
\pgfpathlineto{\pgfqpoint{1.711056in}{2.738796in}}%
\pgfpathlineto{\pgfqpoint{1.746880in}{2.768741in}}%
\pgfpathlineto{\pgfqpoint{1.782704in}{2.797281in}}%
\pgfpathlineto{\pgfqpoint{1.818529in}{2.824325in}}%
\pgfpathlineto{\pgfqpoint{1.854353in}{2.850073in}}%
\pgfpathlineto{\pgfqpoint{1.890177in}{2.874866in}}%
\pgfpathlineto{\pgfqpoint{1.926002in}{2.898805in}}%
\pgfpathlineto{\pgfqpoint{1.961826in}{2.921653in}}%
\pgfpathlineto{\pgfqpoint{1.997650in}{2.943214in}}%
\pgfpathlineto{\pgfqpoint{2.033475in}{2.963307in}}%
\pgfpathlineto{\pgfqpoint{2.069299in}{2.982161in}}%
\pgfpathlineto{\pgfqpoint{2.105123in}{3.000144in}}%
\pgfpathlineto{\pgfqpoint{2.140948in}{3.017316in}}%
\pgfpathlineto{\pgfqpoint{2.176772in}{3.033450in}}%
\pgfpathlineto{\pgfqpoint{2.212596in}{3.048289in}}%
\pgfpathlineto{\pgfqpoint{2.248421in}{3.061762in}}%
\pgfpathlineto{\pgfqpoint{2.284245in}{3.074142in}}%
\pgfpathlineto{\pgfqpoint{2.320069in}{3.085640in}}%
\pgfpathlineto{\pgfqpoint{2.355894in}{3.096387in}}%
\pgfpathlineto{\pgfqpoint{2.391718in}{3.106176in}}%
\pgfpathlineto{\pgfqpoint{2.427543in}{3.114666in}}%
\pgfpathlineto{\pgfqpoint{2.463367in}{3.121877in}}%
\pgfpathlineto{\pgfqpoint{2.499191in}{3.127995in}}%
\pgfpathlineto{\pgfqpoint{2.535016in}{3.133300in}}%
\pgfpathlineto{\pgfqpoint{2.570840in}{3.137847in}}%
\pgfpathlineto{\pgfqpoint{2.606664in}{3.141382in}}%
\pgfpathlineto{\pgfqpoint{2.642489in}{3.143647in}}%
\pgfpathlineto{\pgfqpoint{2.678313in}{3.144668in}}%
\pgfpathlineto{\pgfqpoint{2.714137in}{3.144636in}}%
\pgfpathlineto{\pgfqpoint{2.749962in}{3.143836in}}%
\pgfpathlineto{\pgfqpoint{2.785786in}{3.142248in}}%
\pgfpathlineto{\pgfqpoint{2.821610in}{3.139669in}}%
\pgfpathlineto{\pgfqpoint{2.857435in}{3.135844in}}%
\pgfpathlineto{\pgfqpoint{2.893259in}{3.130769in}}%
\pgfpathlineto{\pgfqpoint{2.929083in}{3.124696in}}%
\pgfpathlineto{\pgfqpoint{2.964908in}{3.117832in}}%
\pgfpathlineto{\pgfqpoint{3.000732in}{3.110137in}}%
\pgfpathlineto{\pgfqpoint{3.036556in}{3.101371in}}%
\pgfpathlineto{\pgfqpoint{3.072381in}{3.091321in}}%
\pgfpathlineto{\pgfqpoint{3.108205in}{3.080006in}}%
\pgfpathlineto{\pgfqpoint{3.144029in}{3.067719in}}%
\pgfpathlineto{\pgfqpoint{3.179854in}{3.054633in}}%
\pgfpathlineto{\pgfqpoint{3.215678in}{3.040678in}}%
\pgfpathlineto{\pgfqpoint{3.251502in}{3.025619in}}%
\pgfpathlineto{\pgfqpoint{3.287327in}{3.009254in}}%
\pgfpathlineto{\pgfqpoint{3.323151in}{2.991598in}}%
\pgfpathlineto{\pgfqpoint{3.358975in}{2.972963in}}%
\pgfpathlineto{\pgfqpoint{3.394800in}{2.953444in}}%
\pgfpathlineto{\pgfqpoint{3.430624in}{2.932980in}}%
\pgfpathlineto{\pgfqpoint{3.466449in}{2.911291in}}%
\pgfpathlineto{\pgfqpoint{3.502273in}{2.888193in}}%
\pgfpathlineto{\pgfqpoint{3.538097in}{2.863821in}}%
\pgfpathlineto{\pgfqpoint{3.573922in}{2.838392in}}%
\pgfpathlineto{\pgfqpoint{3.609746in}{2.812031in}}%
\pgfpathlineto{\pgfqpoint{3.645570in}{2.784637in}}%
\pgfpathlineto{\pgfqpoint{3.681395in}{2.755908in}}%
\pgfpathlineto{\pgfqpoint{3.717219in}{2.725728in}}%
\pgfpathlineto{\pgfqpoint{3.753043in}{2.694167in}}%
\pgfpathlineto{\pgfqpoint{3.788868in}{2.661450in}}%
\pgfpathlineto{\pgfqpoint{3.824692in}{2.627705in}}%
\pgfpathlineto{\pgfqpoint{3.860516in}{2.592728in}}%
\pgfpathlineto{\pgfqpoint{3.896341in}{2.556276in}}%
\pgfpathlineto{\pgfqpoint{3.932165in}{2.518226in}}%
\pgfpathlineto{\pgfqpoint{3.967989in}{2.478701in}}%
\pgfpathlineto{\pgfqpoint{4.003814in}{2.437917in}}%
\pgfpathlineto{\pgfqpoint{4.039638in}{2.395934in}}%
\pgfpathlineto{\pgfqpoint{4.075462in}{2.353126in}}%
\pgfpathlineto{\pgfqpoint{4.111287in}{2.309876in}}%
\pgfpathlineto{\pgfqpoint{4.147111in}{2.266250in}}%
\pgfpathlineto{\pgfqpoint{4.182935in}{2.222645in}}%
\pgfpathlineto{\pgfqpoint{4.218760in}{2.179539in}}%
\pgfpathlineto{\pgfqpoint{4.254584in}{2.137301in}}%
\pgfpathlineto{\pgfqpoint{4.290408in}{2.096052in}}%
\pgfpathlineto{\pgfqpoint{4.326233in}{2.055900in}}%
\pgfpathlineto{\pgfqpoint{4.362057in}{2.017132in}}%
\pgfpathlineto{\pgfqpoint{4.397881in}{1.980283in}}%
\pgfpathlineto{\pgfqpoint{4.433706in}{1.945923in}}%
\pgfpathlineto{\pgfqpoint{4.469530in}{1.914513in}}%
\pgfpathlineto{\pgfqpoint{4.505355in}{1.886235in}}%
\pgfpathlineto{\pgfqpoint{4.541179in}{1.861154in}}%
\pgfpathlineto{\pgfqpoint{4.577003in}{1.839519in}}%
\pgfpathlineto{\pgfqpoint{4.612828in}{1.821811in}}%
\pgfpathlineto{\pgfqpoint{4.648652in}{1.808501in}}%
\pgfpathlineto{\pgfqpoint{4.684476in}{1.799733in}}%
\pgfpathlineto{\pgfqpoint{4.720301in}{1.795416in}}%
\pgfusepath{stroke}%
\end{pgfscope}%
\begin{pgfscope}%
\pgfpathrectangle{\pgfqpoint{0.281664in}{0.316851in}}{\pgfqpoint{4.650000in}{3.020000in}} %
\pgfusepath{clip}%
\pgfsetrectcap%
\pgfsetroundjoin%
\pgfsetlinewidth{1.505625pt}%
\definecolor{currentstroke}{rgb}{0.000000,0.750000,0.750000}%
\pgfsetstrokecolor{currentstroke}%
\pgfsetdash{}{0pt}%
\pgfpathmoveto{\pgfqpoint{0.493028in}{0.454124in}}%
\pgfpathlineto{\pgfqpoint{0.528852in}{0.555674in}}%
\pgfpathlineto{\pgfqpoint{0.564677in}{0.653179in}}%
\pgfpathlineto{\pgfqpoint{0.600501in}{0.751316in}}%
\pgfpathlineto{\pgfqpoint{0.636325in}{0.849490in}}%
\pgfpathlineto{\pgfqpoint{0.672150in}{0.941040in}}%
\pgfpathlineto{\pgfqpoint{0.707974in}{1.027816in}}%
\pgfpathlineto{\pgfqpoint{0.743798in}{1.111564in}}%
\pgfpathlineto{\pgfqpoint{0.779623in}{1.212955in}}%
\pgfpathlineto{\pgfqpoint{0.815447in}{1.316978in}}%
\pgfpathlineto{\pgfqpoint{0.851271in}{1.423316in}}%
\pgfpathlineto{\pgfqpoint{0.887096in}{1.530652in}}%
\pgfpathlineto{\pgfqpoint{0.922920in}{1.638424in}}%
\pgfpathlineto{\pgfqpoint{0.958744in}{1.747306in}}%
\pgfpathlineto{\pgfqpoint{0.994569in}{1.858083in}}%
\pgfpathlineto{\pgfqpoint{1.030393in}{1.917194in}}%
\pgfpathlineto{\pgfqpoint{1.066217in}{1.975133in}}%
\pgfpathlineto{\pgfqpoint{1.102042in}{2.032079in}}%
\pgfpathlineto{\pgfqpoint{1.137866in}{2.086257in}}%
\pgfpathlineto{\pgfqpoint{1.173690in}{2.137001in}}%
\pgfpathlineto{\pgfqpoint{1.209515in}{2.186332in}}%
\pgfpathlineto{\pgfqpoint{1.245339in}{2.234612in}}%
\pgfpathlineto{\pgfqpoint{1.281163in}{2.282701in}}%
\pgfpathlineto{\pgfqpoint{1.316988in}{2.329757in}}%
\pgfpathlineto{\pgfqpoint{1.352812in}{2.374992in}}%
\pgfpathlineto{\pgfqpoint{1.388637in}{2.418064in}}%
\pgfpathlineto{\pgfqpoint{1.424461in}{2.459580in}}%
\pgfpathlineto{\pgfqpoint{1.460285in}{2.499381in}}%
\pgfpathlineto{\pgfqpoint{1.496110in}{2.538329in}}%
\pgfpathlineto{\pgfqpoint{1.531934in}{2.576417in}}%
\pgfpathlineto{\pgfqpoint{1.567758in}{2.612484in}}%
\pgfpathlineto{\pgfqpoint{1.603583in}{2.646395in}}%
\pgfpathlineto{\pgfqpoint{1.639407in}{2.678533in}}%
\pgfpathlineto{\pgfqpoint{1.675231in}{2.709994in}}%
\pgfpathlineto{\pgfqpoint{1.711056in}{2.741192in}}%
\pgfpathlineto{\pgfqpoint{1.746880in}{2.771533in}}%
\pgfpathlineto{\pgfqpoint{1.782704in}{2.800396in}}%
\pgfpathlineto{\pgfqpoint{1.818529in}{2.827883in}}%
\pgfpathlineto{\pgfqpoint{1.854353in}{2.853808in}}%
\pgfpathlineto{\pgfqpoint{1.890177in}{2.878512in}}%
\pgfpathlineto{\pgfqpoint{1.926002in}{2.902484in}}%
\pgfpathlineto{\pgfqpoint{1.961826in}{2.925663in}}%
\pgfpathlineto{\pgfqpoint{1.997650in}{2.947217in}}%
\pgfpathlineto{\pgfqpoint{2.033475in}{2.966720in}}%
\pgfpathlineto{\pgfqpoint{2.069299in}{2.984752in}}%
\pgfpathlineto{\pgfqpoint{2.105123in}{3.002502in}}%
\pgfpathlineto{\pgfqpoint{2.140948in}{3.019478in}}%
\pgfpathlineto{\pgfqpoint{2.176772in}{3.035790in}}%
\pgfpathlineto{\pgfqpoint{2.212596in}{3.050970in}}%
\pgfpathlineto{\pgfqpoint{2.248421in}{3.064758in}}%
\pgfpathlineto{\pgfqpoint{2.284245in}{3.077367in}}%
\pgfpathlineto{\pgfqpoint{2.320069in}{3.088899in}}%
\pgfpathlineto{\pgfqpoint{2.355894in}{3.099819in}}%
\pgfpathlineto{\pgfqpoint{2.391718in}{3.109921in}}%
\pgfpathlineto{\pgfqpoint{2.427543in}{3.118390in}}%
\pgfpathlineto{\pgfqpoint{2.463367in}{3.125048in}}%
\pgfpathlineto{\pgfqpoint{2.499191in}{3.130589in}}%
\pgfpathlineto{\pgfqpoint{2.535016in}{3.135506in}}%
\pgfpathlineto{\pgfqpoint{2.570840in}{3.139968in}}%
\pgfpathlineto{\pgfqpoint{2.606664in}{3.143502in}}%
\pgfpathlineto{\pgfqpoint{2.642489in}{3.146064in}}%
\pgfpathlineto{\pgfqpoint{2.678313in}{3.147392in}}%
\pgfpathlineto{\pgfqpoint{2.714137in}{3.147518in}}%
\pgfpathlineto{\pgfqpoint{2.749962in}{3.146937in}}%
\pgfpathlineto{\pgfqpoint{2.785786in}{3.145649in}}%
\pgfpathlineto{\pgfqpoint{2.821610in}{3.143357in}}%
\pgfpathlineto{\pgfqpoint{2.857435in}{3.139524in}}%
\pgfpathlineto{\pgfqpoint{2.893259in}{3.134017in}}%
\pgfpathlineto{\pgfqpoint{2.929083in}{3.127425in}}%
\pgfpathlineto{\pgfqpoint{2.964908in}{3.120195in}}%
\pgfpathlineto{\pgfqpoint{3.000732in}{3.112378in}}%
\pgfpathlineto{\pgfqpoint{3.036556in}{3.103574in}}%
\pgfpathlineto{\pgfqpoint{3.072381in}{3.093774in}}%
\pgfpathlineto{\pgfqpoint{3.108205in}{3.082799in}}%
\pgfpathlineto{\pgfqpoint{3.144029in}{3.070614in}}%
\pgfpathlineto{\pgfqpoint{3.179854in}{3.057736in}}%
\pgfpathlineto{\pgfqpoint{3.215678in}{3.044065in}}%
\pgfpathlineto{\pgfqpoint{3.251502in}{3.029335in}}%
\pgfpathlineto{\pgfqpoint{3.287327in}{3.012880in}}%
\pgfpathlineto{\pgfqpoint{3.323151in}{2.994910in}}%
\pgfpathlineto{\pgfqpoint{3.358975in}{2.976027in}}%
\pgfpathlineto{\pgfqpoint{3.394800in}{2.956304in}}%
\pgfpathlineto{\pgfqpoint{3.430624in}{2.935623in}}%
\pgfpathlineto{\pgfqpoint{3.466449in}{2.913964in}}%
\pgfpathlineto{\pgfqpoint{3.502273in}{2.891101in}}%
\pgfpathlineto{\pgfqpoint{3.538097in}{2.867041in}}%
\pgfpathlineto{\pgfqpoint{3.573922in}{2.841712in}}%
\pgfpathlineto{\pgfqpoint{3.609746in}{2.815474in}}%
\pgfpathlineto{\pgfqpoint{3.645570in}{2.788452in}}%
\pgfpathlineto{\pgfqpoint{3.681395in}{2.759842in}}%
\pgfpathlineto{\pgfqpoint{3.717219in}{2.729615in}}%
\pgfpathlineto{\pgfqpoint{3.753043in}{2.697980in}}%
\pgfpathlineto{\pgfqpoint{3.788868in}{2.665158in}}%
\pgfpathlineto{\pgfqpoint{3.824692in}{2.631264in}}%
\pgfpathlineto{\pgfqpoint{3.860516in}{2.596086in}}%
\pgfpathlineto{\pgfqpoint{3.896341in}{2.559664in}}%
\pgfpathlineto{\pgfqpoint{3.932165in}{2.521852in}}%
\pgfpathlineto{\pgfqpoint{3.967989in}{2.482493in}}%
\pgfpathlineto{\pgfqpoint{4.003814in}{2.441948in}}%
\pgfpathlineto{\pgfqpoint{4.039638in}{2.400208in}}%
\pgfpathlineto{\pgfqpoint{4.075462in}{2.357147in}}%
\pgfpathlineto{\pgfqpoint{4.111287in}{2.312283in}}%
\pgfpathlineto{\pgfqpoint{4.147111in}{2.265552in}}%
\pgfpathlineto{\pgfqpoint{4.182935in}{2.217098in}}%
\pgfpathlineto{\pgfqpoint{4.218760in}{2.167259in}}%
\pgfpathlineto{\pgfqpoint{4.254584in}{2.049469in}}%
\pgfpathlineto{\pgfqpoint{4.290408in}{1.933389in}}%
\pgfpathlineto{\pgfqpoint{4.326233in}{1.818908in}}%
\pgfpathlineto{\pgfqpoint{4.362057in}{1.705236in}}%
\pgfpathlineto{\pgfqpoint{4.397881in}{1.605093in}}%
\pgfpathlineto{\pgfqpoint{4.433706in}{1.531815in}}%
\pgfpathlineto{\pgfqpoint{4.469530in}{1.459679in}}%
\pgfpathlineto{\pgfqpoint{4.505355in}{1.387306in}}%
\pgfpathlineto{\pgfqpoint{4.541179in}{1.308862in}}%
\pgfpathlineto{\pgfqpoint{4.577003in}{1.220874in}}%
\pgfpathlineto{\pgfqpoint{4.612828in}{1.127160in}}%
\pgfpathlineto{\pgfqpoint{4.648652in}{1.034385in}}%
\pgfpathlineto{\pgfqpoint{4.684476in}{0.942463in}}%
\pgfpathlineto{\pgfqpoint{4.720301in}{0.845996in}}%
\pgfusepath{stroke}%
\end{pgfscope}%
\begin{pgfscope}%
\pgfpathrectangle{\pgfqpoint{0.281664in}{0.316851in}}{\pgfqpoint{4.650000in}{3.020000in}} %
\pgfusepath{clip}%
\pgfsetrectcap%
\pgfsetroundjoin%
\pgfsetlinewidth{1.505625pt}%
\definecolor{currentstroke}{rgb}{0.000000,0.000000,1.000000}%
\pgfsetstrokecolor{currentstroke}%
\pgfsetdash{}{0pt}%
\pgfpathmoveto{\pgfqpoint{0.493028in}{0.927729in}}%
\pgfpathlineto{\pgfqpoint{0.528852in}{1.008111in}}%
\pgfpathlineto{\pgfqpoint{0.564677in}{1.087668in}}%
\pgfpathlineto{\pgfqpoint{0.600501in}{1.166130in}}%
\pgfpathlineto{\pgfqpoint{0.636325in}{1.243281in}}%
\pgfpathlineto{\pgfqpoint{0.672150in}{1.318952in}}%
\pgfpathlineto{\pgfqpoint{0.707974in}{1.393017in}}%
\pgfpathlineto{\pgfqpoint{0.743798in}{1.465383in}}%
\pgfpathlineto{\pgfqpoint{0.779623in}{1.535986in}}%
\pgfpathlineto{\pgfqpoint{0.815447in}{1.604786in}}%
\pgfpathlineto{\pgfqpoint{0.851271in}{1.671758in}}%
\pgfpathlineto{\pgfqpoint{0.887096in}{1.736889in}}%
\pgfpathlineto{\pgfqpoint{0.922920in}{1.800178in}}%
\pgfpathlineto{\pgfqpoint{0.958744in}{1.861631in}}%
\pgfpathlineto{\pgfqpoint{0.994569in}{1.921261in}}%
\pgfpathlineto{\pgfqpoint{1.030393in}{1.979090in}}%
\pgfpathlineto{\pgfqpoint{1.066217in}{2.035147in}}%
\pgfpathlineto{\pgfqpoint{1.102042in}{2.089466in}}%
\pgfpathlineto{\pgfqpoint{1.137866in}{2.142087in}}%
\pgfpathlineto{\pgfqpoint{1.173690in}{2.193055in}}%
\pgfpathlineto{\pgfqpoint{1.209515in}{2.242417in}}%
\pgfpathlineto{\pgfqpoint{1.245339in}{2.290219in}}%
\pgfpathlineto{\pgfqpoint{1.281163in}{2.336506in}}%
\pgfpathlineto{\pgfqpoint{1.316988in}{2.381316in}}%
\pgfpathlineto{\pgfqpoint{1.352812in}{2.424687in}}%
\pgfpathlineto{\pgfqpoint{1.388637in}{2.466648in}}%
\pgfpathlineto{\pgfqpoint{1.424461in}{2.507228in}}%
\pgfpathlineto{\pgfqpoint{1.460285in}{2.546450in}}%
\pgfpathlineto{\pgfqpoint{1.496110in}{2.584338in}}%
\pgfpathlineto{\pgfqpoint{1.531934in}{2.620913in}}%
\pgfpathlineto{\pgfqpoint{1.567758in}{2.656196in}}%
\pgfpathlineto{\pgfqpoint{1.603583in}{2.690209in}}%
\pgfpathlineto{\pgfqpoint{1.639407in}{2.722973in}}%
\pgfpathlineto{\pgfqpoint{1.675231in}{2.754507in}}%
\pgfpathlineto{\pgfqpoint{1.711056in}{2.784830in}}%
\pgfpathlineto{\pgfqpoint{1.746880in}{2.813955in}}%
\pgfpathlineto{\pgfqpoint{1.782704in}{2.841895in}}%
\pgfpathlineto{\pgfqpoint{1.818529in}{2.868659in}}%
\pgfpathlineto{\pgfqpoint{1.854353in}{2.894256in}}%
\pgfpathlineto{\pgfqpoint{1.890177in}{2.918693in}}%
\pgfpathlineto{\pgfqpoint{1.926002in}{2.941979in}}%
\pgfpathlineto{\pgfqpoint{1.961826in}{2.964121in}}%
\pgfpathlineto{\pgfqpoint{1.997650in}{2.985132in}}%
\pgfpathlineto{\pgfqpoint{2.033475in}{3.005024in}}%
\pgfpathlineto{\pgfqpoint{2.069299in}{3.023811in}}%
\pgfpathlineto{\pgfqpoint{2.105123in}{3.041507in}}%
\pgfpathlineto{\pgfqpoint{2.140948in}{3.058126in}}%
\pgfpathlineto{\pgfqpoint{2.176772in}{3.073679in}}%
\pgfpathlineto{\pgfqpoint{2.212596in}{3.088177in}}%
\pgfpathlineto{\pgfqpoint{2.248421in}{3.101624in}}%
\pgfpathlineto{\pgfqpoint{2.284245in}{3.114026in}}%
\pgfpathlineto{\pgfqpoint{2.320069in}{3.125386in}}%
\pgfpathlineto{\pgfqpoint{2.355894in}{3.135704in}}%
\pgfpathlineto{\pgfqpoint{2.391718in}{3.144984in}}%
\pgfpathlineto{\pgfqpoint{2.427543in}{3.153229in}}%
\pgfpathlineto{\pgfqpoint{2.463367in}{3.160442in}}%
\pgfpathlineto{\pgfqpoint{2.499191in}{3.166631in}}%
\pgfpathlineto{\pgfqpoint{2.535016in}{3.171801in}}%
\pgfpathlineto{\pgfqpoint{2.570840in}{3.175961in}}%
\pgfpathlineto{\pgfqpoint{2.606664in}{3.179117in}}%
\pgfpathlineto{\pgfqpoint{2.642489in}{3.181274in}}%
\pgfpathlineto{\pgfqpoint{2.678313in}{3.182437in}}%
\pgfpathlineto{\pgfqpoint{2.714137in}{3.182609in}}%
\pgfpathlineto{\pgfqpoint{2.749962in}{3.181793in}}%
\pgfpathlineto{\pgfqpoint{2.785786in}{3.179991in}}%
\pgfpathlineto{\pgfqpoint{2.821610in}{3.177205in}}%
\pgfpathlineto{\pgfqpoint{2.857435in}{3.173436in}}%
\pgfpathlineto{\pgfqpoint{2.893259in}{3.168688in}}%
\pgfpathlineto{\pgfqpoint{2.929083in}{3.162961in}}%
\pgfpathlineto{\pgfqpoint{2.964908in}{3.156256in}}%
\pgfpathlineto{\pgfqpoint{3.000732in}{3.148571in}}%
\pgfpathlineto{\pgfqpoint{3.036556in}{3.139901in}}%
\pgfpathlineto{\pgfqpoint{3.072381in}{3.130236in}}%
\pgfpathlineto{\pgfqpoint{3.108205in}{3.119561in}}%
\pgfpathlineto{\pgfqpoint{3.144029in}{3.107860in}}%
\pgfpathlineto{\pgfqpoint{3.179854in}{3.095111in}}%
\pgfpathlineto{\pgfqpoint{3.215678in}{3.081292in}}%
\pgfpathlineto{\pgfqpoint{3.251502in}{3.066377in}}%
\pgfpathlineto{\pgfqpoint{3.287327in}{3.050345in}}%
\pgfpathlineto{\pgfqpoint{3.323151in}{3.033175in}}%
\pgfpathlineto{\pgfqpoint{3.358975in}{3.014849in}}%
\pgfpathlineto{\pgfqpoint{3.394800in}{2.995355in}}%
\pgfpathlineto{\pgfqpoint{3.430624in}{2.974684in}}%
\pgfpathlineto{\pgfqpoint{3.466449in}{2.952835in}}%
\pgfpathlineto{\pgfqpoint{3.502273in}{2.929812in}}%
\pgfpathlineto{\pgfqpoint{3.538097in}{2.905625in}}%
\pgfpathlineto{\pgfqpoint{3.573922in}{2.880292in}}%
\pgfpathlineto{\pgfqpoint{3.609746in}{2.853840in}}%
\pgfpathlineto{\pgfqpoint{3.645570in}{2.826297in}}%
\pgfpathlineto{\pgfqpoint{3.681395in}{2.797701in}}%
\pgfpathlineto{\pgfqpoint{3.717219in}{2.768088in}}%
\pgfpathlineto{\pgfqpoint{3.753043in}{2.737497in}}%
\pgfpathlineto{\pgfqpoint{3.788868in}{2.705957in}}%
\pgfpathlineto{\pgfqpoint{3.824692in}{2.673491in}}%
\pgfpathlineto{\pgfqpoint{3.860516in}{2.640100in}}%
\pgfpathlineto{\pgfqpoint{3.896341in}{2.605764in}}%
\pgfpathlineto{\pgfqpoint{3.932165in}{2.570437in}}%
\pgfpathlineto{\pgfqpoint{3.967989in}{2.534037in}}%
\pgfpathlineto{\pgfqpoint{4.003814in}{2.496449in}}%
\pgfpathlineto{\pgfqpoint{4.039638in}{2.457524in}}%
\pgfpathlineto{\pgfqpoint{4.075462in}{2.417077in}}%
\pgfpathlineto{\pgfqpoint{4.111287in}{2.374896in}}%
\pgfpathlineto{\pgfqpoint{4.147111in}{2.330754in}}%
\pgfpathlineto{\pgfqpoint{4.182935in}{2.284413in}}%
\pgfpathlineto{\pgfqpoint{4.218760in}{2.235646in}}%
\pgfpathlineto{\pgfqpoint{4.254584in}{2.184250in}}%
\pgfpathlineto{\pgfqpoint{4.290408in}{2.130071in}}%
\pgfpathlineto{\pgfqpoint{4.326233in}{2.073020in}}%
\pgfpathlineto{\pgfqpoint{4.362057in}{2.013097in}}%
\pgfpathlineto{\pgfqpoint{4.397881in}{1.950410in}}%
\pgfpathlineto{\pgfqpoint{4.433706in}{1.885193in}}%
\pgfpathlineto{\pgfqpoint{4.469530in}{1.817816in}}%
\pgfpathlineto{\pgfqpoint{4.505355in}{1.748793in}}%
\pgfpathlineto{\pgfqpoint{4.541179in}{1.678780in}}%
\pgfpathlineto{\pgfqpoint{4.577003in}{1.608564in}}%
\pgfpathlineto{\pgfqpoint{4.612828in}{1.539041in}}%
\pgfpathlineto{\pgfqpoint{4.648652in}{1.471186in}}%
\pgfpathlineto{\pgfqpoint{4.684476in}{1.406014in}}%
\pgfpathlineto{\pgfqpoint{4.720301in}{1.344527in}}%
\pgfusepath{stroke}%
\end{pgfscope}%
\begin{pgfscope}%
\pgfpathrectangle{\pgfqpoint{0.281664in}{0.316851in}}{\pgfqpoint{4.650000in}{3.020000in}} %
\pgfusepath{clip}%
\pgfsetrectcap%
\pgfsetroundjoin%
\pgfsetlinewidth{1.505625pt}%
\definecolor{currentstroke}{rgb}{1.000000,0.000000,0.000000}%
\pgfsetstrokecolor{currentstroke}%
\pgfsetdash{}{0pt}%
\pgfpathmoveto{\pgfqpoint{0.493028in}{0.912686in}}%
\pgfpathlineto{\pgfqpoint{0.528852in}{0.997437in}}%
\pgfpathlineto{\pgfqpoint{0.564677in}{1.080130in}}%
\pgfpathlineto{\pgfqpoint{0.600501in}{1.160788in}}%
\pgfpathlineto{\pgfqpoint{0.636325in}{1.239433in}}%
\pgfpathlineto{\pgfqpoint{0.672150in}{1.316089in}}%
\pgfpathlineto{\pgfqpoint{0.707974in}{1.390779in}}%
\pgfpathlineto{\pgfqpoint{0.743798in}{1.463525in}}%
\pgfpathlineto{\pgfqpoint{0.779623in}{1.534351in}}%
\pgfpathlineto{\pgfqpoint{0.815447in}{1.603279in}}%
\pgfpathlineto{\pgfqpoint{0.851271in}{1.670332in}}%
\pgfpathlineto{\pgfqpoint{0.887096in}{1.735533in}}%
\pgfpathlineto{\pgfqpoint{0.922920in}{1.798906in}}%
\pgfpathlineto{\pgfqpoint{0.958744in}{1.860460in}}%
\pgfpathlineto{\pgfqpoint{0.994569in}{1.920219in}}%
\pgfpathlineto{\pgfqpoint{1.030393in}{1.978208in}}%
\pgfpathlineto{\pgfqpoint{1.066217in}{2.034454in}}%
\pgfpathlineto{\pgfqpoint{1.102042in}{2.088986in}}%
\pgfpathlineto{\pgfqpoint{1.137866in}{2.141832in}}%
\pgfpathlineto{\pgfqpoint{1.173690in}{2.193022in}}%
\pgfpathlineto{\pgfqpoint{1.209515in}{2.242589in}}%
\pgfpathlineto{\pgfqpoint{1.245339in}{2.290569in}}%
\pgfpathlineto{\pgfqpoint{1.281163in}{2.337001in}}%
\pgfpathlineto{\pgfqpoint{1.316988in}{2.381921in}}%
\pgfpathlineto{\pgfqpoint{1.352812in}{2.425365in}}%
\pgfpathlineto{\pgfqpoint{1.388637in}{2.467368in}}%
\pgfpathlineto{\pgfqpoint{1.424461in}{2.507961in}}%
\pgfpathlineto{\pgfqpoint{1.460285in}{2.547175in}}%
\pgfpathlineto{\pgfqpoint{1.496110in}{2.585041in}}%
\pgfpathlineto{\pgfqpoint{1.531934in}{2.621584in}}%
\pgfpathlineto{\pgfqpoint{1.567758in}{2.656828in}}%
\pgfpathlineto{\pgfqpoint{1.603583in}{2.690795in}}%
\pgfpathlineto{\pgfqpoint{1.639407in}{2.723508in}}%
\pgfpathlineto{\pgfqpoint{1.675231in}{2.754989in}}%
\pgfpathlineto{\pgfqpoint{1.711056in}{2.785258in}}%
\pgfpathlineto{\pgfqpoint{1.746880in}{2.814331in}}%
\pgfpathlineto{\pgfqpoint{1.782704in}{2.842218in}}%
\pgfpathlineto{\pgfqpoint{1.818529in}{2.868930in}}%
\pgfpathlineto{\pgfqpoint{1.854353in}{2.894478in}}%
\pgfpathlineto{\pgfqpoint{1.890177in}{2.918875in}}%
\pgfpathlineto{\pgfqpoint{1.926002in}{2.942133in}}%
\pgfpathlineto{\pgfqpoint{1.961826in}{2.964259in}}%
\pgfpathlineto{\pgfqpoint{1.997650in}{2.985260in}}%
\pgfpathlineto{\pgfqpoint{2.033475in}{3.005147in}}%
\pgfpathlineto{\pgfqpoint{2.069299in}{3.023932in}}%
\pgfpathlineto{\pgfqpoint{2.105123in}{3.041630in}}%
\pgfpathlineto{\pgfqpoint{2.140948in}{3.058252in}}%
\pgfpathlineto{\pgfqpoint{2.176772in}{3.073807in}}%
\pgfpathlineto{\pgfqpoint{2.212596in}{3.088299in}}%
\pgfpathlineto{\pgfqpoint{2.248421in}{3.101734in}}%
\pgfpathlineto{\pgfqpoint{2.284245in}{3.114122in}}%
\pgfpathlineto{\pgfqpoint{2.320069in}{3.125470in}}%
\pgfpathlineto{\pgfqpoint{2.355894in}{3.135784in}}%
\pgfpathlineto{\pgfqpoint{2.391718in}{3.145068in}}%
\pgfpathlineto{\pgfqpoint{2.427543in}{3.153323in}}%
\pgfpathlineto{\pgfqpoint{2.463367in}{3.160556in}}%
\pgfpathlineto{\pgfqpoint{2.499191in}{3.166774in}}%
\pgfpathlineto{\pgfqpoint{2.535016in}{3.171986in}}%
\pgfpathlineto{\pgfqpoint{2.570840in}{3.176196in}}%
\pgfpathlineto{\pgfqpoint{2.606664in}{3.179404in}}%
\pgfpathlineto{\pgfqpoint{2.642489in}{3.181610in}}%
\pgfpathlineto{\pgfqpoint{2.678313in}{3.182815in}}%
\pgfpathlineto{\pgfqpoint{2.714137in}{3.183023in}}%
\pgfpathlineto{\pgfqpoint{2.749962in}{3.182235in}}%
\pgfpathlineto{\pgfqpoint{2.785786in}{3.180450in}}%
\pgfpathlineto{\pgfqpoint{2.821610in}{3.177664in}}%
\pgfpathlineto{\pgfqpoint{2.857435in}{3.173874in}}%
\pgfpathlineto{\pgfqpoint{2.893259in}{3.169080in}}%
\pgfpathlineto{\pgfqpoint{2.929083in}{3.163285in}}%
\pgfpathlineto{\pgfqpoint{2.964908in}{3.156489in}}%
\pgfpathlineto{\pgfqpoint{3.000732in}{3.148691in}}%
\pgfpathlineto{\pgfqpoint{3.036556in}{3.139884in}}%
\pgfpathlineto{\pgfqpoint{3.072381in}{3.130063in}}%
\pgfpathlineto{\pgfqpoint{3.108205in}{3.119222in}}%
\pgfpathlineto{\pgfqpoint{3.144029in}{3.107359in}}%
\pgfpathlineto{\pgfqpoint{3.179854in}{3.094468in}}%
\pgfpathlineto{\pgfqpoint{3.215678in}{3.080540in}}%
\pgfpathlineto{\pgfqpoint{3.251502in}{3.065567in}}%
\pgfpathlineto{\pgfqpoint{3.287327in}{3.049541in}}%
\pgfpathlineto{\pgfqpoint{3.323151in}{3.032456in}}%
\pgfpathlineto{\pgfqpoint{3.358975in}{3.014309in}}%
\pgfpathlineto{\pgfqpoint{3.394800in}{2.995095in}}%
\pgfpathlineto{\pgfqpoint{3.430624in}{2.974805in}}%
\pgfpathlineto{\pgfqpoint{3.466449in}{2.953425in}}%
\pgfpathlineto{\pgfqpoint{3.502273in}{2.930943in}}%
\pgfpathlineto{\pgfqpoint{3.538097in}{2.907348in}}%
\pgfpathlineto{\pgfqpoint{3.573922in}{2.882629in}}%
\pgfpathlineto{\pgfqpoint{3.609746in}{2.856770in}}%
\pgfpathlineto{\pgfqpoint{3.645570in}{2.829754in}}%
\pgfpathlineto{\pgfqpoint{3.681395in}{2.801562in}}%
\pgfpathlineto{\pgfqpoint{3.717219in}{2.772175in}}%
\pgfpathlineto{\pgfqpoint{3.753043in}{2.741578in}}%
\pgfpathlineto{\pgfqpoint{3.788868in}{2.709755in}}%
\pgfpathlineto{\pgfqpoint{3.824692in}{2.676684in}}%
\pgfpathlineto{\pgfqpoint{3.860516in}{2.642342in}}%
\pgfpathlineto{\pgfqpoint{3.896341in}{2.606701in}}%
\pgfpathlineto{\pgfqpoint{3.932165in}{2.569736in}}%
\pgfpathlineto{\pgfqpoint{3.967989in}{2.531421in}}%
\pgfpathlineto{\pgfqpoint{4.003814in}{2.491729in}}%
\pgfpathlineto{\pgfqpoint{4.039638in}{2.450632in}}%
\pgfpathlineto{\pgfqpoint{4.075462in}{2.408099in}}%
\pgfpathlineto{\pgfqpoint{4.111287in}{2.364101in}}%
\pgfpathlineto{\pgfqpoint{4.147111in}{2.318611in}}%
\pgfpathlineto{\pgfqpoint{4.182935in}{2.271601in}}%
\pgfpathlineto{\pgfqpoint{4.218760in}{2.223044in}}%
\pgfpathlineto{\pgfqpoint{4.254584in}{2.172910in}}%
\pgfpathlineto{\pgfqpoint{4.290408in}{2.121167in}}%
\pgfpathlineto{\pgfqpoint{4.326233in}{2.067786in}}%
\pgfpathlineto{\pgfqpoint{4.362057in}{2.012737in}}%
\pgfpathlineto{\pgfqpoint{4.397881in}{1.955991in}}%
\pgfpathlineto{\pgfqpoint{4.433706in}{1.897519in}}%
\pgfpathlineto{\pgfqpoint{4.469530in}{1.837292in}}%
\pgfpathlineto{\pgfqpoint{4.505355in}{1.775282in}}%
\pgfpathlineto{\pgfqpoint{4.541179in}{1.711458in}}%
\pgfpathlineto{\pgfqpoint{4.577003in}{1.645791in}}%
\pgfpathlineto{\pgfqpoint{4.612828in}{1.578253in}}%
\pgfpathlineto{\pgfqpoint{4.648652in}{1.508814in}}%
\pgfpathlineto{\pgfqpoint{4.684476in}{1.437446in}}%
\pgfpathlineto{\pgfqpoint{4.720301in}{1.364119in}}%
\pgfusepath{stroke}%
\end{pgfscope}%
\begin{pgfscope}%
\pgfsetrectcap%
\pgfsetmiterjoin%
\pgfsetlinewidth{0.803000pt}%
\definecolor{currentstroke}{rgb}{0.000000,0.000000,0.000000}%
\pgfsetstrokecolor{currentstroke}%
\pgfsetdash{}{0pt}%
\pgfpathmoveto{\pgfqpoint{0.281664in}{0.316851in}}%
\pgfpathlineto{\pgfqpoint{0.281664in}{3.336851in}}%
\pgfusepath{stroke}%
\end{pgfscope}%
\begin{pgfscope}%
\pgfsetrectcap%
\pgfsetmiterjoin%
\pgfsetlinewidth{0.803000pt}%
\definecolor{currentstroke}{rgb}{0.000000,0.000000,0.000000}%
\pgfsetstrokecolor{currentstroke}%
\pgfsetdash{}{0pt}%
\pgfpathmoveto{\pgfqpoint{4.931664in}{0.316851in}}%
\pgfpathlineto{\pgfqpoint{4.931664in}{3.336851in}}%
\pgfusepath{stroke}%
\end{pgfscope}%
\begin{pgfscope}%
\pgfsetrectcap%
\pgfsetmiterjoin%
\pgfsetlinewidth{0.803000pt}%
\definecolor{currentstroke}{rgb}{0.000000,0.000000,0.000000}%
\pgfsetstrokecolor{currentstroke}%
\pgfsetdash{}{0pt}%
\pgfpathmoveto{\pgfqpoint{0.281664in}{0.316851in}}%
\pgfpathlineto{\pgfqpoint{4.931664in}{0.316851in}}%
\pgfusepath{stroke}%
\end{pgfscope}%
\begin{pgfscope}%
\pgfsetrectcap%
\pgfsetmiterjoin%
\pgfsetlinewidth{0.803000pt}%
\definecolor{currentstroke}{rgb}{0.000000,0.000000,0.000000}%
\pgfsetstrokecolor{currentstroke}%
\pgfsetdash{}{0pt}%
\pgfpathmoveto{\pgfqpoint{0.281664in}{3.336851in}}%
\pgfpathlineto{\pgfqpoint{4.931664in}{3.336851in}}%
\pgfusepath{stroke}%
\end{pgfscope}%
\begin{pgfscope}%
\pgfsetbuttcap%
\pgfsetmiterjoin%
\definecolor{currentfill}{rgb}{1.000000,1.000000,1.000000}%
\pgfsetfillcolor{currentfill}%
\pgfsetfillopacity{0.800000}%
\pgfsetlinewidth{1.003750pt}%
\definecolor{currentstroke}{rgb}{0.800000,0.800000,0.800000}%
\pgfsetstrokecolor{currentstroke}%
\pgfsetstrokeopacity{0.800000}%
\pgfsetdash{}{0pt}%
\pgfpathmoveto{\pgfqpoint{1.509164in}{0.386295in}}%
\pgfpathlineto{\pgfqpoint{3.704165in}{0.386295in}}%
\pgfpathquadraticcurveto{\pgfqpoint{3.731942in}{0.386295in}}{\pgfqpoint{3.731942in}{0.414073in}}%
\pgfpathlineto{\pgfqpoint{3.731942in}{1.423404in}}%
\pgfpathquadraticcurveto{\pgfqpoint{3.731942in}{1.451182in}}{\pgfqpoint{3.704165in}{1.451182in}}%
\pgfpathlineto{\pgfqpoint{1.509164in}{1.451182in}}%
\pgfpathquadraticcurveto{\pgfqpoint{1.481386in}{1.451182in}}{\pgfqpoint{1.481386in}{1.423404in}}%
\pgfpathlineto{\pgfqpoint{1.481386in}{0.414073in}}%
\pgfpathquadraticcurveto{\pgfqpoint{1.481386in}{0.386295in}}{\pgfqpoint{1.509164in}{0.386295in}}%
\pgfpathclose%
\pgfusepath{stroke,fill}%
\end{pgfscope}%
\begin{pgfscope}%
\pgfsetbuttcap%
\pgfsetroundjoin%
\definecolor{currentfill}{rgb}{0.000000,0.000000,0.000000}%
\pgfsetfillcolor{currentfill}%
\pgfsetlinewidth{1.003750pt}%
\definecolor{currentstroke}{rgb}{0.000000,0.000000,0.000000}%
\pgfsetstrokecolor{currentstroke}%
\pgfsetdash{}{0pt}%
\pgfsys@defobject{currentmarker}{\pgfqpoint{-0.020833in}{-0.020833in}}{\pgfqpoint{0.020833in}{0.020833in}}{%
\pgfpathmoveto{\pgfqpoint{0.000000in}{-0.020833in}}%
\pgfpathcurveto{\pgfqpoint{0.005525in}{-0.020833in}}{\pgfqpoint{0.010825in}{-0.018638in}}{\pgfqpoint{0.014731in}{-0.014731in}}%
\pgfpathcurveto{\pgfqpoint{0.018638in}{-0.010825in}}{\pgfqpoint{0.020833in}{-0.005525in}}{\pgfqpoint{0.020833in}{0.000000in}}%
\pgfpathcurveto{\pgfqpoint{0.020833in}{0.005525in}}{\pgfqpoint{0.018638in}{0.010825in}}{\pgfqpoint{0.014731in}{0.014731in}}%
\pgfpathcurveto{\pgfqpoint{0.010825in}{0.018638in}}{\pgfqpoint{0.005525in}{0.020833in}}{\pgfqpoint{0.000000in}{0.020833in}}%
\pgfpathcurveto{\pgfqpoint{-0.005525in}{0.020833in}}{\pgfqpoint{-0.010825in}{0.018638in}}{\pgfqpoint{-0.014731in}{0.014731in}}%
\pgfpathcurveto{\pgfqpoint{-0.018638in}{0.010825in}}{\pgfqpoint{-0.020833in}{0.005525in}}{\pgfqpoint{-0.020833in}{0.000000in}}%
\pgfpathcurveto{\pgfqpoint{-0.020833in}{-0.005525in}}{\pgfqpoint{-0.018638in}{-0.010825in}}{\pgfqpoint{-0.014731in}{-0.014731in}}%
\pgfpathcurveto{\pgfqpoint{-0.010825in}{-0.018638in}}{\pgfqpoint{-0.005525in}{-0.020833in}}{\pgfqpoint{0.000000in}{-0.020833in}}%
\pgfpathclose%
\pgfusepath{stroke,fill}%
}%
\begin{pgfscope}%
\pgfsys@transformshift{1.675831in}{1.338714in}%
\pgfsys@useobject{currentmarker}{}%
\end{pgfscope}%
\end{pgfscope}%
\begin{pgfscope}%
\pgftext[x=1.925831in,y=1.290103in,left,base]{\rmfamily\fontsize{10.000000}{12.000000}\selectfont Raw signal}%
\end{pgfscope}%
\begin{pgfscope}%
\pgfsetrectcap%
\pgfsetroundjoin%
\pgfsetlinewidth{1.505625pt}%
\definecolor{currentstroke}{rgb}{0.750000,0.000000,0.750000}%
\pgfsetstrokecolor{currentstroke}%
\pgfsetdash{}{0pt}%
\pgfpathmoveto{\pgfqpoint{1.536942in}{1.132890in}}%
\pgfpathlineto{\pgfqpoint{1.814720in}{1.132890in}}%
\pgfusepath{stroke}%
\end{pgfscope}%
\begin{pgfscope}%
\pgftext[x=1.925831in,y=1.084279in,left,base]{\rmfamily\fontsize{10.000000}{12.000000}\selectfont 1D Gaussian convolution}%
\end{pgfscope}%
\begin{pgfscope}%
\pgfsetrectcap%
\pgfsetroundjoin%
\pgfsetlinewidth{1.505625pt}%
\definecolor{currentstroke}{rgb}{0.000000,0.750000,0.750000}%
\pgfsetstrokecolor{currentstroke}%
\pgfsetdash{}{0pt}%
\pgfpathmoveto{\pgfqpoint{1.536942in}{0.929033in}}%
\pgfpathlineto{\pgfqpoint{1.814720in}{0.929033in}}%
\pgfusepath{stroke}%
\end{pgfscope}%
\begin{pgfscope}%
\pgftext[x=1.925831in,y=0.880422in,left,base]{\rmfamily\fontsize{10.000000}{12.000000}\selectfont Wiener}%
\end{pgfscope}%
\begin{pgfscope}%
\pgfsetrectcap%
\pgfsetroundjoin%
\pgfsetlinewidth{1.505625pt}%
\definecolor{currentstroke}{rgb}{0.000000,0.000000,1.000000}%
\pgfsetstrokecolor{currentstroke}%
\pgfsetdash{}{0pt}%
\pgfpathmoveto{\pgfqpoint{1.536942in}{0.725176in}}%
\pgfpathlineto{\pgfqpoint{1.814720in}{0.725176in}}%
\pgfusepath{stroke}%
\end{pgfscope}%
\begin{pgfscope}%
\pgftext[x=1.925831in,y=0.676565in,left,base]{\rmfamily\fontsize{10.000000}{12.000000}\selectfont Butterworth}%
\end{pgfscope}%
\begin{pgfscope}%
\pgfsetrectcap%
\pgfsetroundjoin%
\pgfsetlinewidth{1.505625pt}%
\definecolor{currentstroke}{rgb}{1.000000,0.000000,0.000000}%
\pgfsetstrokecolor{currentstroke}%
\pgfsetdash{}{0pt}%
\pgfpathmoveto{\pgfqpoint{1.536942in}{0.521318in}}%
\pgfpathlineto{\pgfqpoint{1.814720in}{0.521318in}}%
\pgfusepath{stroke}%
\end{pgfscope}%
\begin{pgfscope}%
\pgftext[x=1.925831in,y=0.472707in,left,base]{\rmfamily\fontsize{10.000000}{12.000000}\selectfont Savitsky-Golay}%
\end{pgfscope}%
\end{pgfpicture}%
\makeatother%
\endgroup%
}
%\caption{Power spectra of the filter methods compared.}\label{fig:myfigure}
%\end{figure}

%\begin{figure}[htb]
%\centering
%\resizebox{10cm}{!}{%% Creator: Matplotlib, PGF backend
%%
%% To include the figure in your LaTeX document, write
%%   \input{<filename>.pgf}
%%
%% Make sure the required packages are loaded in your preamble
%%   \usepackage{pgf}
%%
%% Figures using additional raster images can only be included by \input if
%% they are in the same directory as the main LaTeX file. For loading figures
%% from other directories you can use the `import` package
%%   \usepackage{import}
%% and then include the figures with
%%   \import{<path to file>}{<filename>.pgf}
%%
%% Matplotlib used the following preamble
%%   \usepackage{fontspec}
%%   \setmainfont{DejaVu Serif}
%%   \setsansfont{DejaVu Sans}
%%   \setmonofont{DejaVu Sans Mono}
%%
\begingroup%
\makeatletter%
\begin{pgfpicture}%
\pgfpathrectangle{\pgfpointorigin}{\pgfqpoint{5.186840in}{3.471851in}}%
\pgfusepath{use as bounding box, clip}%
\begin{pgfscope}%
\pgfsetbuttcap%
\pgfsetmiterjoin%
\definecolor{currentfill}{rgb}{1.000000,1.000000,1.000000}%
\pgfsetfillcolor{currentfill}%
\pgfsetlinewidth{0.000000pt}%
\definecolor{currentstroke}{rgb}{1.000000,1.000000,1.000000}%
\pgfsetstrokecolor{currentstroke}%
\pgfsetdash{}{0pt}%
\pgfpathmoveto{\pgfqpoint{0.000000in}{0.000000in}}%
\pgfpathlineto{\pgfqpoint{5.186840in}{0.000000in}}%
\pgfpathlineto{\pgfqpoint{5.186840in}{3.471851in}}%
\pgfpathlineto{\pgfqpoint{0.000000in}{3.471851in}}%
\pgfpathclose%
\pgfusepath{fill}%
\end{pgfscope}%
\begin{pgfscope}%
\pgfsetbuttcap%
\pgfsetmiterjoin%
\definecolor{currentfill}{rgb}{1.000000,1.000000,1.000000}%
\pgfsetfillcolor{currentfill}%
\pgfsetlinewidth{0.000000pt}%
\definecolor{currentstroke}{rgb}{0.000000,0.000000,0.000000}%
\pgfsetstrokecolor{currentstroke}%
\pgfsetstrokeopacity{0.000000}%
\pgfsetdash{}{0pt}%
\pgfpathmoveto{\pgfqpoint{0.401840in}{0.316851in}}%
\pgfpathlineto{\pgfqpoint{5.051840in}{0.316851in}}%
\pgfpathlineto{\pgfqpoint{5.051840in}{3.336851in}}%
\pgfpathlineto{\pgfqpoint{0.401840in}{3.336851in}}%
\pgfpathclose%
\pgfusepath{fill}%
\end{pgfscope}%
\begin{pgfscope}%
\pgftext[x=2.726840in,y=0.261295in,,top]{\rmfamily\fontsize{12.000000}{14.400000}\selectfont \(\displaystyle t\)}%
\end{pgfscope}%
\begin{pgfscope}%
\pgftext[x=0.262951in,y=1.826851in,,bottom,rotate=90.000000]{\rmfamily\fontsize{12.000000}{14.400000}\selectfont \(\displaystyle y'\)}%
\end{pgfscope}%
\begin{pgfscope}%
\pgfpathrectangle{\pgfqpoint{0.401840in}{0.316851in}}{\pgfqpoint{4.650000in}{3.020000in}} %
\pgfusepath{clip}%
\pgfsetbuttcap%
\pgfsetroundjoin%
\pgfsetlinewidth{1.505625pt}%
\definecolor{currentstroke}{rgb}{1.000000,0.000000,0.000000}%
\pgfsetstrokecolor{currentstroke}%
\pgfsetstrokeopacity{0.700000}%
\pgfsetdash{{5.550000pt}{2.400000pt}}{0.000000pt}%
\pgfpathmoveto{\pgfqpoint{0.613203in}{3.199578in}}%
\pgfpathlineto{\pgfqpoint{0.649028in}{3.163846in}}%
\pgfpathlineto{\pgfqpoint{0.684852in}{3.128531in}}%
\pgfpathlineto{\pgfqpoint{0.720676in}{3.093634in}}%
\pgfpathlineto{\pgfqpoint{0.756501in}{3.059154in}}%
\pgfpathlineto{\pgfqpoint{0.792325in}{3.025091in}}%
\pgfpathlineto{\pgfqpoint{0.828149in}{2.991446in}}%
\pgfpathlineto{\pgfqpoint{0.863974in}{2.958217in}}%
\pgfpathlineto{\pgfqpoint{0.899798in}{2.925406in}}%
\pgfpathlineto{\pgfqpoint{0.935622in}{2.893013in}}%
\pgfpathlineto{\pgfqpoint{0.971447in}{2.861036in}}%
\pgfpathlineto{\pgfqpoint{1.007271in}{2.829478in}}%
\pgfpathlineto{\pgfqpoint{1.043096in}{2.798336in}}%
\pgfpathlineto{\pgfqpoint{1.078920in}{2.767454in}}%
\pgfpathlineto{\pgfqpoint{1.114744in}{2.736988in}}%
\pgfpathlineto{\pgfqpoint{1.150569in}{2.706959in}}%
\pgfpathlineto{\pgfqpoint{1.186393in}{2.677394in}}%
\pgfpathlineto{\pgfqpoint{1.222217in}{2.648322in}}%
\pgfpathlineto{\pgfqpoint{1.258042in}{2.619771in}}%
\pgfpathlineto{\pgfqpoint{1.293866in}{2.591769in}}%
\pgfpathlineto{\pgfqpoint{1.329690in}{2.564337in}}%
\pgfpathlineto{\pgfqpoint{1.365515in}{2.537490in}}%
\pgfpathlineto{\pgfqpoint{1.401339in}{2.511236in}}%
\pgfpathlineto{\pgfqpoint{1.437163in}{2.485576in}}%
\pgfpathlineto{\pgfqpoint{1.472988in}{2.460503in}}%
\pgfpathlineto{\pgfqpoint{1.508812in}{2.436000in}}%
\pgfpathlineto{\pgfqpoint{1.544636in}{2.412047in}}%
\pgfpathlineto{\pgfqpoint{1.580461in}{2.388617in}}%
\pgfpathlineto{\pgfqpoint{1.616285in}{2.365679in}}%
\pgfpathlineto{\pgfqpoint{1.652109in}{2.343198in}}%
\pgfpathlineto{\pgfqpoint{1.687934in}{2.321139in}}%
\pgfpathlineto{\pgfqpoint{1.723758in}{2.299465in}}%
\pgfpathlineto{\pgfqpoint{1.759582in}{2.278141in}}%
\pgfpathlineto{\pgfqpoint{1.795407in}{2.257130in}}%
\pgfpathlineto{\pgfqpoint{1.831231in}{2.236404in}}%
\pgfpathlineto{\pgfqpoint{1.867055in}{2.215935in}}%
\pgfpathlineto{\pgfqpoint{1.902880in}{2.195703in}}%
\pgfpathlineto{\pgfqpoint{1.938704in}{2.175689in}}%
\pgfpathlineto{\pgfqpoint{1.974528in}{2.155880in}}%
\pgfpathlineto{\pgfqpoint{2.010353in}{2.136263in}}%
\pgfpathlineto{\pgfqpoint{2.046177in}{2.116828in}}%
\pgfpathlineto{\pgfqpoint{2.082002in}{2.097570in}}%
\pgfpathlineto{\pgfqpoint{2.117826in}{2.078482in}}%
\pgfpathlineto{\pgfqpoint{2.153650in}{2.059562in}}%
\pgfpathlineto{\pgfqpoint{2.189475in}{2.040805in}}%
\pgfpathlineto{\pgfqpoint{2.225299in}{2.022207in}}%
\pgfpathlineto{\pgfqpoint{2.261123in}{2.003761in}}%
\pgfpathlineto{\pgfqpoint{2.296948in}{1.985462in}}%
\pgfpathlineto{\pgfqpoint{2.332772in}{1.967303in}}%
\pgfpathlineto{\pgfqpoint{2.368596in}{1.949275in}}%
\pgfpathlineto{\pgfqpoint{2.404421in}{1.931372in}}%
\pgfpathlineto{\pgfqpoint{2.440245in}{1.913584in}}%
\pgfpathlineto{\pgfqpoint{2.476069in}{1.895901in}}%
\pgfpathlineto{\pgfqpoint{2.511894in}{1.878315in}}%
\pgfpathlineto{\pgfqpoint{2.547718in}{1.860818in}}%
\pgfpathlineto{\pgfqpoint{2.583542in}{1.843400in}}%
\pgfpathlineto{\pgfqpoint{2.619367in}{1.826054in}}%
\pgfpathlineto{\pgfqpoint{2.655191in}{1.808771in}}%
\pgfpathlineto{\pgfqpoint{2.691015in}{1.791541in}}%
\pgfpathlineto{\pgfqpoint{2.726840in}{1.774354in}}%
\pgfpathlineto{\pgfqpoint{2.762664in}{1.757203in}}%
\pgfpathlineto{\pgfqpoint{2.798488in}{1.740078in}}%
\pgfpathlineto{\pgfqpoint{2.834313in}{1.722970in}}%
\pgfpathlineto{\pgfqpoint{2.870137in}{1.705870in}}%
\pgfpathlineto{\pgfqpoint{2.905961in}{1.688769in}}%
\pgfpathlineto{\pgfqpoint{2.941786in}{1.671659in}}%
\pgfpathlineto{\pgfqpoint{2.977610in}{1.654530in}}%
\pgfpathlineto{\pgfqpoint{3.013434in}{1.637374in}}%
\pgfpathlineto{\pgfqpoint{3.049259in}{1.620183in}}%
\pgfpathlineto{\pgfqpoint{3.085083in}{1.602947in}}%
\pgfpathlineto{\pgfqpoint{3.120908in}{1.585657in}}%
\pgfpathlineto{\pgfqpoint{3.156732in}{1.568303in}}%
\pgfpathlineto{\pgfqpoint{3.192556in}{1.550877in}}%
\pgfpathlineto{\pgfqpoint{3.228381in}{1.533373in}}%
\pgfpathlineto{\pgfqpoint{3.264205in}{1.515782in}}%
\pgfpathlineto{\pgfqpoint{3.300029in}{1.498097in}}%
\pgfpathlineto{\pgfqpoint{3.335854in}{1.480309in}}%
\pgfpathlineto{\pgfqpoint{3.371678in}{1.462409in}}%
\pgfpathlineto{\pgfqpoint{3.407502in}{1.444389in}}%
\pgfpathlineto{\pgfqpoint{3.443327in}{1.426240in}}%
\pgfpathlineto{\pgfqpoint{3.479151in}{1.407951in}}%
\pgfpathlineto{\pgfqpoint{3.514975in}{1.389510in}}%
\pgfpathlineto{\pgfqpoint{3.550800in}{1.370905in}}%
\pgfpathlineto{\pgfqpoint{3.586624in}{1.352119in}}%
\pgfpathlineto{\pgfqpoint{3.622448in}{1.333137in}}%
\pgfpathlineto{\pgfqpoint{3.658273in}{1.313941in}}%
\pgfpathlineto{\pgfqpoint{3.694097in}{1.294514in}}%
\pgfpathlineto{\pgfqpoint{3.729921in}{1.274834in}}%
\pgfpathlineto{\pgfqpoint{3.765746in}{1.254878in}}%
\pgfpathlineto{\pgfqpoint{3.801570in}{1.234624in}}%
\pgfpathlineto{\pgfqpoint{3.837394in}{1.214051in}}%
\pgfpathlineto{\pgfqpoint{3.873219in}{1.193136in}}%
\pgfpathlineto{\pgfqpoint{3.909043in}{1.171859in}}%
\pgfpathlineto{\pgfqpoint{3.944867in}{1.150201in}}%
\pgfpathlineto{\pgfqpoint{3.980692in}{1.128140in}}%
\pgfpathlineto{\pgfqpoint{4.016516in}{1.105659in}}%
\pgfpathlineto{\pgfqpoint{4.052340in}{1.082738in}}%
\pgfpathlineto{\pgfqpoint{4.088165in}{1.059365in}}%
\pgfpathlineto{\pgfqpoint{4.123989in}{1.035525in}}%
\pgfpathlineto{\pgfqpoint{4.159814in}{1.011206in}}%
\pgfpathlineto{\pgfqpoint{4.195638in}{0.986398in}}%
\pgfpathlineto{\pgfqpoint{4.231462in}{0.961093in}}%
\pgfpathlineto{\pgfqpoint{4.267287in}{0.935288in}}%
\pgfpathlineto{\pgfqpoint{4.303111in}{0.908980in}}%
\pgfpathlineto{\pgfqpoint{4.338935in}{0.882171in}}%
\pgfpathlineto{\pgfqpoint{4.374760in}{0.854859in}}%
\pgfpathlineto{\pgfqpoint{4.410584in}{0.827045in}}%
\pgfpathlineto{\pgfqpoint{4.446408in}{0.798718in}}%
\pgfpathlineto{\pgfqpoint{4.482233in}{0.769891in}}%
\pgfpathlineto{\pgfqpoint{4.518057in}{0.740564in}}%
\pgfpathlineto{\pgfqpoint{4.553881in}{0.710737in}}%
\pgfpathlineto{\pgfqpoint{4.589706in}{0.680410in}}%
\pgfpathlineto{\pgfqpoint{4.625530in}{0.649583in}}%
\pgfpathlineto{\pgfqpoint{4.661354in}{0.618257in}}%
\pgfpathlineto{\pgfqpoint{4.697179in}{0.586430in}}%
\pgfpathlineto{\pgfqpoint{4.733003in}{0.554103in}}%
\pgfpathlineto{\pgfqpoint{4.768827in}{0.521277in}}%
\pgfpathlineto{\pgfqpoint{4.804652in}{0.487950in}}%
\pgfpathlineto{\pgfqpoint{4.840476in}{0.454124in}}%
\pgfusepath{stroke}%
\end{pgfscope}%
\begin{pgfscope}%
\pgfpathrectangle{\pgfqpoint{0.401840in}{0.316851in}}{\pgfqpoint{4.650000in}{3.020000in}} %
\pgfusepath{clip}%
\pgfsetbuttcap%
\pgfsetmiterjoin%
\definecolor{currentfill}{rgb}{1.000000,0.000000,0.000000}%
\pgfsetfillcolor{currentfill}%
\pgfsetfillopacity{0.700000}%
\pgfsetlinewidth{1.003750pt}%
\definecolor{currentstroke}{rgb}{1.000000,0.000000,0.000000}%
\pgfsetstrokecolor{currentstroke}%
\pgfsetstrokeopacity{0.700000}%
\pgfsetdash{}{0pt}%
\pgfsys@defobject{currentmarker}{\pgfqpoint{-0.041667in}{-0.041667in}}{\pgfqpoint{0.041667in}{0.041667in}}{%
\pgfpathmoveto{\pgfqpoint{-0.041667in}{-0.041667in}}%
\pgfpathlineto{\pgfqpoint{0.041667in}{-0.041667in}}%
\pgfpathlineto{\pgfqpoint{0.041667in}{0.041667in}}%
\pgfpathlineto{\pgfqpoint{-0.041667in}{0.041667in}}%
\pgfpathclose%
\pgfusepath{stroke,fill}%
}%
\begin{pgfscope}%
\pgfsys@transformshift{0.613203in}{3.199578in}%
\pgfsys@useobject{currentmarker}{}%
\end{pgfscope}%
\begin{pgfscope}%
\pgfsys@transformshift{0.899798in}{2.925406in}%
\pgfsys@useobject{currentmarker}{}%
\end{pgfscope}%
\begin{pgfscope}%
\pgfsys@transformshift{1.186393in}{2.677394in}%
\pgfsys@useobject{currentmarker}{}%
\end{pgfscope}%
\begin{pgfscope}%
\pgfsys@transformshift{1.472988in}{2.460503in}%
\pgfsys@useobject{currentmarker}{}%
\end{pgfscope}%
\begin{pgfscope}%
\pgfsys@transformshift{1.759582in}{2.278141in}%
\pgfsys@useobject{currentmarker}{}%
\end{pgfscope}%
\begin{pgfscope}%
\pgfsys@transformshift{2.046177in}{2.116828in}%
\pgfsys@useobject{currentmarker}{}%
\end{pgfscope}%
\begin{pgfscope}%
\pgfsys@transformshift{2.332772in}{1.967303in}%
\pgfsys@useobject{currentmarker}{}%
\end{pgfscope}%
\begin{pgfscope}%
\pgfsys@transformshift{2.619367in}{1.826054in}%
\pgfsys@useobject{currentmarker}{}%
\end{pgfscope}%
\begin{pgfscope}%
\pgfsys@transformshift{2.905961in}{1.688769in}%
\pgfsys@useobject{currentmarker}{}%
\end{pgfscope}%
\begin{pgfscope}%
\pgfsys@transformshift{3.192556in}{1.550877in}%
\pgfsys@useobject{currentmarker}{}%
\end{pgfscope}%
\begin{pgfscope}%
\pgfsys@transformshift{3.479151in}{1.407951in}%
\pgfsys@useobject{currentmarker}{}%
\end{pgfscope}%
\begin{pgfscope}%
\pgfsys@transformshift{3.765746in}{1.254878in}%
\pgfsys@useobject{currentmarker}{}%
\end{pgfscope}%
\begin{pgfscope}%
\pgfsys@transformshift{4.052340in}{1.082738in}%
\pgfsys@useobject{currentmarker}{}%
\end{pgfscope}%
\begin{pgfscope}%
\pgfsys@transformshift{4.338935in}{0.882171in}%
\pgfsys@useobject{currentmarker}{}%
\end{pgfscope}%
\begin{pgfscope}%
\pgfsys@transformshift{4.625530in}{0.649583in}%
\pgfsys@useobject{currentmarker}{}%
\end{pgfscope}%
\end{pgfscope}%
\begin{pgfscope}%
\pgfpathrectangle{\pgfqpoint{0.401840in}{0.316851in}}{\pgfqpoint{4.650000in}{3.020000in}} %
\pgfusepath{clip}%
\pgfsetrectcap%
\pgfsetroundjoin%
\pgfsetlinewidth{1.505625pt}%
\definecolor{currentstroke}{rgb}{0.000000,0.000000,1.000000}%
\pgfsetstrokecolor{currentstroke}%
\pgfsetstrokeopacity{0.700000}%
\pgfsetdash{}{0pt}%
\pgfpathmoveto{\pgfqpoint{0.649028in}{3.093309in}}%
\pgfpathlineto{\pgfqpoint{0.684852in}{3.077671in}}%
\pgfpathlineto{\pgfqpoint{0.720676in}{3.058612in}}%
\pgfpathlineto{\pgfqpoint{0.756501in}{3.036930in}}%
\pgfpathlineto{\pgfqpoint{0.792325in}{3.013250in}}%
\pgfpathlineto{\pgfqpoint{0.828149in}{2.988048in}}%
\pgfpathlineto{\pgfqpoint{0.863974in}{2.961671in}}%
\pgfpathlineto{\pgfqpoint{0.899798in}{2.934354in}}%
\pgfpathlineto{\pgfqpoint{0.935622in}{2.906235in}}%
\pgfpathlineto{\pgfqpoint{0.971447in}{2.877387in}}%
\pgfpathlineto{\pgfqpoint{1.007271in}{2.847853in}}%
\pgfpathlineto{\pgfqpoint{1.043096in}{2.817690in}}%
\pgfpathlineto{\pgfqpoint{1.078920in}{2.786999in}}%
\pgfpathlineto{\pgfqpoint{1.114744in}{2.755941in}}%
\pgfpathlineto{\pgfqpoint{1.150569in}{2.724725in}}%
\pgfpathlineto{\pgfqpoint{1.186393in}{2.693597in}}%
\pgfpathlineto{\pgfqpoint{1.222217in}{2.662812in}}%
\pgfpathlineto{\pgfqpoint{1.258042in}{2.632617in}}%
\pgfpathlineto{\pgfqpoint{1.293866in}{2.603218in}}%
\pgfpathlineto{\pgfqpoint{1.329690in}{2.574756in}}%
\pgfpathlineto{\pgfqpoint{1.365515in}{2.547293in}}%
\pgfpathlineto{\pgfqpoint{1.401339in}{2.520803in}}%
\pgfpathlineto{\pgfqpoint{1.437163in}{2.495199in}}%
\pgfpathlineto{\pgfqpoint{1.472988in}{2.470358in}}%
\pgfpathlineto{\pgfqpoint{1.508812in}{2.446158in}}%
\pgfpathlineto{\pgfqpoint{1.544636in}{2.422495in}}%
\pgfpathlineto{\pgfqpoint{1.580461in}{2.399293in}}%
\pgfpathlineto{\pgfqpoint{1.616285in}{2.376504in}}%
\pgfpathlineto{\pgfqpoint{1.652109in}{2.354108in}}%
\pgfpathlineto{\pgfqpoint{1.687934in}{2.332106in}}%
\pgfpathlineto{\pgfqpoint{1.723758in}{2.310508in}}%
\pgfpathlineto{\pgfqpoint{1.759582in}{2.289310in}}%
\pgfpathlineto{\pgfqpoint{1.795407in}{2.268483in}}%
\pgfpathlineto{\pgfqpoint{1.831231in}{2.247966in}}%
\pgfpathlineto{\pgfqpoint{1.867055in}{2.227674in}}%
\pgfpathlineto{\pgfqpoint{1.902880in}{2.207523in}}%
\pgfpathlineto{\pgfqpoint{1.938704in}{2.187445in}}%
\pgfpathlineto{\pgfqpoint{1.974528in}{2.167403in}}%
\pgfpathlineto{\pgfqpoint{2.010353in}{2.147392in}}%
\pgfpathlineto{\pgfqpoint{2.046177in}{2.127439in}}%
\pgfpathlineto{\pgfqpoint{2.082002in}{2.107595in}}%
\pgfpathlineto{\pgfqpoint{2.117826in}{2.087934in}}%
\pgfpathlineto{\pgfqpoint{2.153650in}{2.068534in}}%
\pgfpathlineto{\pgfqpoint{2.189475in}{2.049454in}}%
\pgfpathlineto{\pgfqpoint{2.225299in}{2.030722in}}%
\pgfpathlineto{\pgfqpoint{2.261123in}{2.012318in}}%
\pgfpathlineto{\pgfqpoint{2.296948in}{1.994188in}}%
\pgfpathlineto{\pgfqpoint{2.332772in}{1.976259in}}%
\pgfpathlineto{\pgfqpoint{2.368596in}{1.958453in}}%
\pgfpathlineto{\pgfqpoint{2.404421in}{1.940704in}}%
\pgfpathlineto{\pgfqpoint{2.440245in}{1.922965in}}%
\pgfpathlineto{\pgfqpoint{2.476069in}{1.905213in}}%
\pgfpathlineto{\pgfqpoint{2.511894in}{1.887456in}}%
\pgfpathlineto{\pgfqpoint{2.547718in}{1.869725in}}%
\pgfpathlineto{\pgfqpoint{2.583542in}{1.852071in}}%
\pgfpathlineto{\pgfqpoint{2.619367in}{1.834536in}}%
\pgfpathlineto{\pgfqpoint{2.655191in}{1.817145in}}%
\pgfpathlineto{\pgfqpoint{2.691015in}{1.799890in}}%
\pgfpathlineto{\pgfqpoint{2.726840in}{1.782738in}}%
\pgfpathlineto{\pgfqpoint{2.762664in}{1.765642in}}%
\pgfpathlineto{\pgfqpoint{2.798488in}{1.748554in}}%
\pgfpathlineto{\pgfqpoint{2.834313in}{1.731432in}}%
\pgfpathlineto{\pgfqpoint{2.870137in}{1.714251in}}%
\pgfpathlineto{\pgfqpoint{2.905961in}{1.697001in}}%
\pgfpathlineto{\pgfqpoint{2.941786in}{1.679703in}}%
\pgfpathlineto{\pgfqpoint{2.977610in}{1.662396in}}%
\pgfpathlineto{\pgfqpoint{3.013434in}{1.645135in}}%
\pgfpathlineto{\pgfqpoint{3.049259in}{1.627965in}}%
\pgfpathlineto{\pgfqpoint{3.085083in}{1.610910in}}%
\pgfpathlineto{\pgfqpoint{3.120908in}{1.593960in}}%
\pgfpathlineto{\pgfqpoint{3.156732in}{1.577078in}}%
\pgfpathlineto{\pgfqpoint{3.192556in}{1.560204in}}%
\pgfpathlineto{\pgfqpoint{3.228381in}{1.543274in}}%
\pgfpathlineto{\pgfqpoint{3.264205in}{1.526215in}}%
\pgfpathlineto{\pgfqpoint{3.300029in}{1.508963in}}%
\pgfpathlineto{\pgfqpoint{3.335854in}{1.491462in}}%
\pgfpathlineto{\pgfqpoint{3.371678in}{1.473675in}}%
\pgfpathlineto{\pgfqpoint{3.407502in}{1.455585in}}%
\pgfpathlineto{\pgfqpoint{3.443327in}{1.437188in}}%
\pgfpathlineto{\pgfqpoint{3.479151in}{1.418477in}}%
\pgfpathlineto{\pgfqpoint{3.514975in}{1.399439in}}%
\pgfpathlineto{\pgfqpoint{3.550800in}{1.380052in}}%
\pgfpathlineto{\pgfqpoint{3.586624in}{1.360293in}}%
\pgfpathlineto{\pgfqpoint{3.622448in}{1.340156in}}%
\pgfpathlineto{\pgfqpoint{3.658273in}{1.319658in}}%
\pgfpathlineto{\pgfqpoint{3.694097in}{1.298848in}}%
\pgfpathlineto{\pgfqpoint{3.729921in}{1.277818in}}%
\pgfpathlineto{\pgfqpoint{3.765746in}{1.256698in}}%
\pgfpathlineto{\pgfqpoint{3.801570in}{1.235659in}}%
\pgfpathlineto{\pgfqpoint{3.837394in}{1.214900in}}%
\pgfpathlineto{\pgfqpoint{3.873219in}{1.194617in}}%
\pgfpathlineto{\pgfqpoint{3.909043in}{1.174975in}}%
\pgfpathlineto{\pgfqpoint{3.944867in}{1.156072in}}%
\pgfpathlineto{\pgfqpoint{3.980692in}{1.137913in}}%
\pgfpathlineto{\pgfqpoint{4.016516in}{1.120396in}}%
\pgfpathlineto{\pgfqpoint{4.052340in}{1.103296in}}%
\pgfpathlineto{\pgfqpoint{4.088165in}{1.086262in}}%
\pgfpathlineto{\pgfqpoint{4.123989in}{1.068817in}}%
\pgfpathlineto{\pgfqpoint{4.159814in}{1.050374in}}%
\pgfpathlineto{\pgfqpoint{4.195638in}{1.030272in}}%
\pgfpathlineto{\pgfqpoint{4.231462in}{1.007819in}}%
\pgfpathlineto{\pgfqpoint{4.267287in}{0.982348in}}%
\pgfpathlineto{\pgfqpoint{4.303111in}{0.953272in}}%
\pgfpathlineto{\pgfqpoint{4.338935in}{0.920152in}}%
\pgfpathlineto{\pgfqpoint{4.374760in}{0.882769in}}%
\pgfpathlineto{\pgfqpoint{4.410584in}{0.841200in}}%
\pgfpathlineto{\pgfqpoint{4.446408in}{0.795893in}}%
\pgfpathlineto{\pgfqpoint{4.482233in}{0.747730in}}%
\pgfpathlineto{\pgfqpoint{4.518057in}{0.698056in}}%
\pgfpathlineto{\pgfqpoint{4.553881in}{0.648687in}}%
\pgfpathlineto{\pgfqpoint{4.589706in}{0.601885in}}%
\pgfpathlineto{\pgfqpoint{4.625530in}{0.560290in}}%
\pgfpathlineto{\pgfqpoint{4.661354in}{0.526811in}}%
\pgfpathlineto{\pgfqpoint{4.697179in}{0.504464in}}%
\pgfpathlineto{\pgfqpoint{4.733003in}{0.496161in}}%
\pgfpathlineto{\pgfqpoint{4.768827in}{0.504479in}}%
\pgfpathlineto{\pgfqpoint{4.804652in}{0.531410in}}%
\pgfpathlineto{\pgfqpoint{4.840476in}{0.578120in}}%
\pgfusepath{stroke}%
\end{pgfscope}%
\begin{pgfscope}%
\pgfpathrectangle{\pgfqpoint{0.401840in}{0.316851in}}{\pgfqpoint{4.650000in}{3.020000in}} %
\pgfusepath{clip}%
\pgfsetbuttcap%
\pgfsetroundjoin%
\definecolor{currentfill}{rgb}{0.000000,0.000000,1.000000}%
\pgfsetfillcolor{currentfill}%
\pgfsetfillopacity{0.700000}%
\pgfsetlinewidth{1.003750pt}%
\definecolor{currentstroke}{rgb}{0.000000,0.000000,1.000000}%
\pgfsetstrokecolor{currentstroke}%
\pgfsetstrokeopacity{0.700000}%
\pgfsetdash{}{0pt}%
\pgfsys@defobject{currentmarker}{\pgfqpoint{-0.041667in}{-0.041667in}}{\pgfqpoint{0.041667in}{0.041667in}}{%
\pgfpathmoveto{\pgfqpoint{0.000000in}{-0.041667in}}%
\pgfpathcurveto{\pgfqpoint{0.011050in}{-0.041667in}}{\pgfqpoint{0.021649in}{-0.037276in}}{\pgfqpoint{0.029463in}{-0.029463in}}%
\pgfpathcurveto{\pgfqpoint{0.037276in}{-0.021649in}}{\pgfqpoint{0.041667in}{-0.011050in}}{\pgfqpoint{0.041667in}{0.000000in}}%
\pgfpathcurveto{\pgfqpoint{0.041667in}{0.011050in}}{\pgfqpoint{0.037276in}{0.021649in}}{\pgfqpoint{0.029463in}{0.029463in}}%
\pgfpathcurveto{\pgfqpoint{0.021649in}{0.037276in}}{\pgfqpoint{0.011050in}{0.041667in}}{\pgfqpoint{0.000000in}{0.041667in}}%
\pgfpathcurveto{\pgfqpoint{-0.011050in}{0.041667in}}{\pgfqpoint{-0.021649in}{0.037276in}}{\pgfqpoint{-0.029463in}{0.029463in}}%
\pgfpathcurveto{\pgfqpoint{-0.037276in}{0.021649in}}{\pgfqpoint{-0.041667in}{0.011050in}}{\pgfqpoint{-0.041667in}{0.000000in}}%
\pgfpathcurveto{\pgfqpoint{-0.041667in}{-0.011050in}}{\pgfqpoint{-0.037276in}{-0.021649in}}{\pgfqpoint{-0.029463in}{-0.029463in}}%
\pgfpathcurveto{\pgfqpoint{-0.021649in}{-0.037276in}}{\pgfqpoint{-0.011050in}{-0.041667in}}{\pgfqpoint{0.000000in}{-0.041667in}}%
\pgfpathclose%
\pgfusepath{stroke,fill}%
}%
\begin{pgfscope}%
\pgfsys@transformshift{0.649028in}{3.093309in}%
\pgfsys@useobject{currentmarker}{}%
\end{pgfscope}%
\begin{pgfscope}%
\pgfsys@transformshift{0.935622in}{2.906235in}%
\pgfsys@useobject{currentmarker}{}%
\end{pgfscope}%
\begin{pgfscope}%
\pgfsys@transformshift{1.222217in}{2.662812in}%
\pgfsys@useobject{currentmarker}{}%
\end{pgfscope}%
\begin{pgfscope}%
\pgfsys@transformshift{1.508812in}{2.446158in}%
\pgfsys@useobject{currentmarker}{}%
\end{pgfscope}%
\begin{pgfscope}%
\pgfsys@transformshift{1.795407in}{2.268483in}%
\pgfsys@useobject{currentmarker}{}%
\end{pgfscope}%
\begin{pgfscope}%
\pgfsys@transformshift{2.082002in}{2.107595in}%
\pgfsys@useobject{currentmarker}{}%
\end{pgfscope}%
\begin{pgfscope}%
\pgfsys@transformshift{2.368596in}{1.958453in}%
\pgfsys@useobject{currentmarker}{}%
\end{pgfscope}%
\begin{pgfscope}%
\pgfsys@transformshift{2.655191in}{1.817145in}%
\pgfsys@useobject{currentmarker}{}%
\end{pgfscope}%
\begin{pgfscope}%
\pgfsys@transformshift{2.941786in}{1.679703in}%
\pgfsys@useobject{currentmarker}{}%
\end{pgfscope}%
\begin{pgfscope}%
\pgfsys@transformshift{3.228381in}{1.543274in}%
\pgfsys@useobject{currentmarker}{}%
\end{pgfscope}%
\begin{pgfscope}%
\pgfsys@transformshift{3.514975in}{1.399439in}%
\pgfsys@useobject{currentmarker}{}%
\end{pgfscope}%
\begin{pgfscope}%
\pgfsys@transformshift{3.801570in}{1.235659in}%
\pgfsys@useobject{currentmarker}{}%
\end{pgfscope}%
\begin{pgfscope}%
\pgfsys@transformshift{4.088165in}{1.086262in}%
\pgfsys@useobject{currentmarker}{}%
\end{pgfscope}%
\begin{pgfscope}%
\pgfsys@transformshift{4.374760in}{0.882769in}%
\pgfsys@useobject{currentmarker}{}%
\end{pgfscope}%
\begin{pgfscope}%
\pgfsys@transformshift{4.661354in}{0.526811in}%
\pgfsys@useobject{currentmarker}{}%
\end{pgfscope}%
\end{pgfscope}%
\begin{pgfscope}%
\pgfsetrectcap%
\pgfsetmiterjoin%
\pgfsetlinewidth{0.803000pt}%
\definecolor{currentstroke}{rgb}{0.000000,0.000000,0.000000}%
\pgfsetstrokecolor{currentstroke}%
\pgfsetdash{}{0pt}%
\pgfpathmoveto{\pgfqpoint{0.401840in}{0.316851in}}%
\pgfpathlineto{\pgfqpoint{0.401840in}{3.336851in}}%
\pgfusepath{stroke}%
\end{pgfscope}%
\begin{pgfscope}%
\pgfsetrectcap%
\pgfsetmiterjoin%
\pgfsetlinewidth{0.803000pt}%
\definecolor{currentstroke}{rgb}{0.000000,0.000000,0.000000}%
\pgfsetstrokecolor{currentstroke}%
\pgfsetdash{}{0pt}%
\pgfpathmoveto{\pgfqpoint{5.051840in}{0.316851in}}%
\pgfpathlineto{\pgfqpoint{5.051840in}{3.336851in}}%
\pgfusepath{stroke}%
\end{pgfscope}%
\begin{pgfscope}%
\pgfsetrectcap%
\pgfsetmiterjoin%
\pgfsetlinewidth{0.803000pt}%
\definecolor{currentstroke}{rgb}{0.000000,0.000000,0.000000}%
\pgfsetstrokecolor{currentstroke}%
\pgfsetdash{}{0pt}%
\pgfpathmoveto{\pgfqpoint{0.401840in}{0.316851in}}%
\pgfpathlineto{\pgfqpoint{5.051840in}{0.316851in}}%
\pgfusepath{stroke}%
\end{pgfscope}%
\begin{pgfscope}%
\pgfsetrectcap%
\pgfsetmiterjoin%
\pgfsetlinewidth{0.803000pt}%
\definecolor{currentstroke}{rgb}{0.000000,0.000000,0.000000}%
\pgfsetstrokecolor{currentstroke}%
\pgfsetdash{}{0pt}%
\pgfpathmoveto{\pgfqpoint{0.401840in}{3.336851in}}%
\pgfpathlineto{\pgfqpoint{5.051840in}{3.336851in}}%
\pgfusepath{stroke}%
\end{pgfscope}%
\begin{pgfscope}%
\pgfsetbuttcap%
\pgfsetmiterjoin%
\definecolor{currentfill}{rgb}{1.000000,1.000000,1.000000}%
\pgfsetfillcolor{currentfill}%
\pgfsetfillopacity{0.800000}%
\pgfsetlinewidth{1.003750pt}%
\definecolor{currentstroke}{rgb}{0.800000,0.800000,0.800000}%
\pgfsetstrokecolor{currentstroke}%
\pgfsetstrokeopacity{0.800000}%
\pgfsetdash{}{0pt}%
\pgfpathmoveto{\pgfqpoint{1.985751in}{2.816059in}}%
\pgfpathlineto{\pgfqpoint{3.467929in}{2.816059in}}%
\pgfpathquadraticcurveto{\pgfqpoint{3.495706in}{2.816059in}}{\pgfqpoint{3.495706in}{2.843837in}}%
\pgfpathlineto{\pgfqpoint{3.495706in}{3.239629in}}%
\pgfpathquadraticcurveto{\pgfqpoint{3.495706in}{3.267407in}}{\pgfqpoint{3.467929in}{3.267407in}}%
\pgfpathlineto{\pgfqpoint{1.985751in}{3.267407in}}%
\pgfpathquadraticcurveto{\pgfqpoint{1.957973in}{3.267407in}}{\pgfqpoint{1.957973in}{3.239629in}}%
\pgfpathlineto{\pgfqpoint{1.957973in}{2.843837in}}%
\pgfpathquadraticcurveto{\pgfqpoint{1.957973in}{2.816059in}}{\pgfqpoint{1.985751in}{2.816059in}}%
\pgfpathclose%
\pgfusepath{stroke,fill}%
\end{pgfscope}%
\begin{pgfscope}%
\pgfsetbuttcap%
\pgfsetroundjoin%
\pgfsetlinewidth{1.505625pt}%
\definecolor{currentstroke}{rgb}{1.000000,0.000000,0.000000}%
\pgfsetstrokecolor{currentstroke}%
\pgfsetstrokeopacity{0.700000}%
\pgfsetdash{{5.550000pt}{2.400000pt}}{0.000000pt}%
\pgfpathmoveto{\pgfqpoint{2.013529in}{3.154939in}}%
\pgfpathlineto{\pgfqpoint{2.291306in}{3.154939in}}%
\pgfusepath{stroke}%
\end{pgfscope}%
\begin{pgfscope}%
\pgfsetbuttcap%
\pgfsetmiterjoin%
\definecolor{currentfill}{rgb}{1.000000,0.000000,0.000000}%
\pgfsetfillcolor{currentfill}%
\pgfsetfillopacity{0.700000}%
\pgfsetlinewidth{1.003750pt}%
\definecolor{currentstroke}{rgb}{1.000000,0.000000,0.000000}%
\pgfsetstrokecolor{currentstroke}%
\pgfsetstrokeopacity{0.700000}%
\pgfsetdash{}{0pt}%
\pgfsys@defobject{currentmarker}{\pgfqpoint{-0.041667in}{-0.041667in}}{\pgfqpoint{0.041667in}{0.041667in}}{%
\pgfpathmoveto{\pgfqpoint{-0.041667in}{-0.041667in}}%
\pgfpathlineto{\pgfqpoint{0.041667in}{-0.041667in}}%
\pgfpathlineto{\pgfqpoint{0.041667in}{0.041667in}}%
\pgfpathlineto{\pgfqpoint{-0.041667in}{0.041667in}}%
\pgfpathclose%
\pgfusepath{stroke,fill}%
}%
\begin{pgfscope}%
\pgfsys@transformshift{2.152417in}{3.154939in}%
\pgfsys@useobject{currentmarker}{}%
\end{pgfscope}%
\end{pgfscope}%
\begin{pgfscope}%
\pgftext[x=2.402417in,y=3.106328in,left,base]{\rmfamily\fontsize{10.000000}{12.000000}\selectfont Savitsky-Golay}%
\end{pgfscope}%
\begin{pgfscope}%
\pgfsetrectcap%
\pgfsetroundjoin%
\pgfsetlinewidth{1.505625pt}%
\definecolor{currentstroke}{rgb}{0.000000,0.000000,1.000000}%
\pgfsetstrokecolor{currentstroke}%
\pgfsetstrokeopacity{0.700000}%
\pgfsetdash{}{0pt}%
\pgfpathmoveto{\pgfqpoint{2.013529in}{2.949115in}}%
\pgfpathlineto{\pgfqpoint{2.291306in}{2.949115in}}%
\pgfusepath{stroke}%
\end{pgfscope}%
\begin{pgfscope}%
\pgfsetbuttcap%
\pgfsetroundjoin%
\definecolor{currentfill}{rgb}{0.000000,0.000000,1.000000}%
\pgfsetfillcolor{currentfill}%
\pgfsetfillopacity{0.700000}%
\pgfsetlinewidth{1.003750pt}%
\definecolor{currentstroke}{rgb}{0.000000,0.000000,1.000000}%
\pgfsetstrokecolor{currentstroke}%
\pgfsetstrokeopacity{0.700000}%
\pgfsetdash{}{0pt}%
\pgfsys@defobject{currentmarker}{\pgfqpoint{-0.041667in}{-0.041667in}}{\pgfqpoint{0.041667in}{0.041667in}}{%
\pgfpathmoveto{\pgfqpoint{0.000000in}{-0.041667in}}%
\pgfpathcurveto{\pgfqpoint{0.011050in}{-0.041667in}}{\pgfqpoint{0.021649in}{-0.037276in}}{\pgfqpoint{0.029463in}{-0.029463in}}%
\pgfpathcurveto{\pgfqpoint{0.037276in}{-0.021649in}}{\pgfqpoint{0.041667in}{-0.011050in}}{\pgfqpoint{0.041667in}{0.000000in}}%
\pgfpathcurveto{\pgfqpoint{0.041667in}{0.011050in}}{\pgfqpoint{0.037276in}{0.021649in}}{\pgfqpoint{0.029463in}{0.029463in}}%
\pgfpathcurveto{\pgfqpoint{0.021649in}{0.037276in}}{\pgfqpoint{0.011050in}{0.041667in}}{\pgfqpoint{0.000000in}{0.041667in}}%
\pgfpathcurveto{\pgfqpoint{-0.011050in}{0.041667in}}{\pgfqpoint{-0.021649in}{0.037276in}}{\pgfqpoint{-0.029463in}{0.029463in}}%
\pgfpathcurveto{\pgfqpoint{-0.037276in}{0.021649in}}{\pgfqpoint{-0.041667in}{0.011050in}}{\pgfqpoint{-0.041667in}{0.000000in}}%
\pgfpathcurveto{\pgfqpoint{-0.041667in}{-0.011050in}}{\pgfqpoint{-0.037276in}{-0.021649in}}{\pgfqpoint{-0.029463in}{-0.029463in}}%
\pgfpathcurveto{\pgfqpoint{-0.021649in}{-0.037276in}}{\pgfqpoint{-0.011050in}{-0.041667in}}{\pgfqpoint{0.000000in}{-0.041667in}}%
\pgfpathclose%
\pgfusepath{stroke,fill}%
}%
\begin{pgfscope}%
\pgfsys@transformshift{2.152417in}{2.949115in}%
\pgfsys@useobject{currentmarker}{}%
\end{pgfscope}%
\end{pgfscope}%
\begin{pgfscope}%
\pgftext[x=2.402417in,y=2.900504in,left,base]{\rmfamily\fontsize{10.000000}{12.000000}\selectfont Butterworth}%
\end{pgfscope}%
\end{pgfpicture}%
\makeatother%
\endgroup%
}
%\caption{Power spectra of the filter methods compared.}\label{fig:myfigure}
%\end{figure}

%\begin{figure}[htb]
%\centering
%\resizebox{10cm}{!}{%% Creator: Matplotlib, PGF backend
%%
%% To include the figure in your LaTeX document, write
%%   \input{<filename>.pgf}
%%
%% Make sure the required packages are loaded in your preamble
%%   \usepackage{pgf}
%%
%% Figures using additional raster images can only be included by \input if
%% they are in the same directory as the main LaTeX file. For loading figures
%% from other directories you can use the `import` package
%%   \usepackage{import}
%% and then include the figures with
%%   \import{<path to file>}{<filename>.pgf}
%%
%% Matplotlib used the following preamble
%%   \usepackage{fontspec}
%%   \setmainfont{DejaVu Serif}
%%   \setsansfont{DejaVu Sans}
%%   \setmonofont{DejaVu Sans Mono}
%%
\begingroup%
\makeatletter%
\begin{pgfpicture}%
\pgfpathrectangle{\pgfpointorigin}{\pgfqpoint{5.490516in}{3.678570in}}%
\pgfusepath{use as bounding box, clip}%
\begin{pgfscope}%
\pgfsetbuttcap%
\pgfsetmiterjoin%
\definecolor{currentfill}{rgb}{1.000000,1.000000,1.000000}%
\pgfsetfillcolor{currentfill}%
\pgfsetlinewidth{0.000000pt}%
\definecolor{currentstroke}{rgb}{1.000000,1.000000,1.000000}%
\pgfsetstrokecolor{currentstroke}%
\pgfsetdash{}{0pt}%
\pgfpathmoveto{\pgfqpoint{0.000000in}{0.000000in}}%
\pgfpathlineto{\pgfqpoint{5.490516in}{0.000000in}}%
\pgfpathlineto{\pgfqpoint{5.490516in}{3.678570in}}%
\pgfpathlineto{\pgfqpoint{0.000000in}{3.678570in}}%
\pgfpathclose%
\pgfusepath{fill}%
\end{pgfscope}%
\begin{pgfscope}%
\pgfsetbuttcap%
\pgfsetmiterjoin%
\definecolor{currentfill}{rgb}{1.000000,1.000000,1.000000}%
\pgfsetfillcolor{currentfill}%
\pgfsetlinewidth{0.000000pt}%
\definecolor{currentstroke}{rgb}{0.000000,0.000000,0.000000}%
\pgfsetstrokecolor{currentstroke}%
\pgfsetstrokeopacity{0.000000}%
\pgfsetdash{}{0pt}%
\pgfpathmoveto{\pgfqpoint{0.705516in}{0.523570in}}%
\pgfpathlineto{\pgfqpoint{5.355516in}{0.523570in}}%
\pgfpathlineto{\pgfqpoint{5.355516in}{3.543570in}}%
\pgfpathlineto{\pgfqpoint{0.705516in}{3.543570in}}%
\pgfpathclose%
\pgfusepath{fill}%
\end{pgfscope}%
\begin{pgfscope}%
\pgfsetbuttcap%
\pgfsetroundjoin%
\definecolor{currentfill}{rgb}{0.000000,0.000000,0.000000}%
\pgfsetfillcolor{currentfill}%
\pgfsetlinewidth{0.803000pt}%
\definecolor{currentstroke}{rgb}{0.000000,0.000000,0.000000}%
\pgfsetstrokecolor{currentstroke}%
\pgfsetdash{}{0pt}%
\pgfsys@defobject{currentmarker}{\pgfqpoint{0.000000in}{-0.048611in}}{\pgfqpoint{0.000000in}{0.000000in}}{%
\pgfpathmoveto{\pgfqpoint{0.000000in}{0.000000in}}%
\pgfpathlineto{\pgfqpoint{0.000000in}{-0.048611in}}%
\pgfusepath{stroke,fill}%
}%
\begin{pgfscope}%
\pgfsys@transformshift{0.916880in}{0.523570in}%
\pgfsys@useobject{currentmarker}{}%
\end{pgfscope}%
\end{pgfscope}%
\begin{pgfscope}%
\pgftext[x=0.916880in,y=0.426348in,,top]{\rmfamily\fontsize{10.000000}{12.000000}\selectfont \(\displaystyle 0\)}%
\end{pgfscope}%
\begin{pgfscope}%
\pgfsetbuttcap%
\pgfsetroundjoin%
\definecolor{currentfill}{rgb}{0.000000,0.000000,0.000000}%
\pgfsetfillcolor{currentfill}%
\pgfsetlinewidth{0.803000pt}%
\definecolor{currentstroke}{rgb}{0.000000,0.000000,0.000000}%
\pgfsetstrokecolor{currentstroke}%
\pgfsetdash{}{0pt}%
\pgfsys@defobject{currentmarker}{\pgfqpoint{0.000000in}{-0.048611in}}{\pgfqpoint{0.000000in}{0.000000in}}{%
\pgfpathmoveto{\pgfqpoint{0.000000in}{0.000000in}}%
\pgfpathlineto{\pgfqpoint{0.000000in}{-0.048611in}}%
\pgfusepath{stroke,fill}%
}%
\begin{pgfscope}%
\pgfsys@transformshift{1.634412in}{0.523570in}%
\pgfsys@useobject{currentmarker}{}%
\end{pgfscope}%
\end{pgfscope}%
\begin{pgfscope}%
\pgftext[x=1.634412in,y=0.426348in,,top]{\rmfamily\fontsize{10.000000}{12.000000}\selectfont \(\displaystyle 1000\)}%
\end{pgfscope}%
\begin{pgfscope}%
\pgfsetbuttcap%
\pgfsetroundjoin%
\definecolor{currentfill}{rgb}{0.000000,0.000000,0.000000}%
\pgfsetfillcolor{currentfill}%
\pgfsetlinewidth{0.803000pt}%
\definecolor{currentstroke}{rgb}{0.000000,0.000000,0.000000}%
\pgfsetstrokecolor{currentstroke}%
\pgfsetdash{}{0pt}%
\pgfsys@defobject{currentmarker}{\pgfqpoint{0.000000in}{-0.048611in}}{\pgfqpoint{0.000000in}{0.000000in}}{%
\pgfpathmoveto{\pgfqpoint{0.000000in}{0.000000in}}%
\pgfpathlineto{\pgfqpoint{0.000000in}{-0.048611in}}%
\pgfusepath{stroke,fill}%
}%
\begin{pgfscope}%
\pgfsys@transformshift{2.351944in}{0.523570in}%
\pgfsys@useobject{currentmarker}{}%
\end{pgfscope}%
\end{pgfscope}%
\begin{pgfscope}%
\pgftext[x=2.351944in,y=0.426348in,,top]{\rmfamily\fontsize{10.000000}{12.000000}\selectfont \(\displaystyle 2000\)}%
\end{pgfscope}%
\begin{pgfscope}%
\pgfsetbuttcap%
\pgfsetroundjoin%
\definecolor{currentfill}{rgb}{0.000000,0.000000,0.000000}%
\pgfsetfillcolor{currentfill}%
\pgfsetlinewidth{0.803000pt}%
\definecolor{currentstroke}{rgb}{0.000000,0.000000,0.000000}%
\pgfsetstrokecolor{currentstroke}%
\pgfsetdash{}{0pt}%
\pgfsys@defobject{currentmarker}{\pgfqpoint{0.000000in}{-0.048611in}}{\pgfqpoint{0.000000in}{0.000000in}}{%
\pgfpathmoveto{\pgfqpoint{0.000000in}{0.000000in}}%
\pgfpathlineto{\pgfqpoint{0.000000in}{-0.048611in}}%
\pgfusepath{stroke,fill}%
}%
\begin{pgfscope}%
\pgfsys@transformshift{3.069476in}{0.523570in}%
\pgfsys@useobject{currentmarker}{}%
\end{pgfscope}%
\end{pgfscope}%
\begin{pgfscope}%
\pgftext[x=3.069476in,y=0.426348in,,top]{\rmfamily\fontsize{10.000000}{12.000000}\selectfont \(\displaystyle 3000\)}%
\end{pgfscope}%
\begin{pgfscope}%
\pgfsetbuttcap%
\pgfsetroundjoin%
\definecolor{currentfill}{rgb}{0.000000,0.000000,0.000000}%
\pgfsetfillcolor{currentfill}%
\pgfsetlinewidth{0.803000pt}%
\definecolor{currentstroke}{rgb}{0.000000,0.000000,0.000000}%
\pgfsetstrokecolor{currentstroke}%
\pgfsetdash{}{0pt}%
\pgfsys@defobject{currentmarker}{\pgfqpoint{0.000000in}{-0.048611in}}{\pgfqpoint{0.000000in}{0.000000in}}{%
\pgfpathmoveto{\pgfqpoint{0.000000in}{0.000000in}}%
\pgfpathlineto{\pgfqpoint{0.000000in}{-0.048611in}}%
\pgfusepath{stroke,fill}%
}%
\begin{pgfscope}%
\pgfsys@transformshift{3.787009in}{0.523570in}%
\pgfsys@useobject{currentmarker}{}%
\end{pgfscope}%
\end{pgfscope}%
\begin{pgfscope}%
\pgftext[x=3.787009in,y=0.426348in,,top]{\rmfamily\fontsize{10.000000}{12.000000}\selectfont \(\displaystyle 4000\)}%
\end{pgfscope}%
\begin{pgfscope}%
\pgfsetbuttcap%
\pgfsetroundjoin%
\definecolor{currentfill}{rgb}{0.000000,0.000000,0.000000}%
\pgfsetfillcolor{currentfill}%
\pgfsetlinewidth{0.803000pt}%
\definecolor{currentstroke}{rgb}{0.000000,0.000000,0.000000}%
\pgfsetstrokecolor{currentstroke}%
\pgfsetdash{}{0pt}%
\pgfsys@defobject{currentmarker}{\pgfqpoint{0.000000in}{-0.048611in}}{\pgfqpoint{0.000000in}{0.000000in}}{%
\pgfpathmoveto{\pgfqpoint{0.000000in}{0.000000in}}%
\pgfpathlineto{\pgfqpoint{0.000000in}{-0.048611in}}%
\pgfusepath{stroke,fill}%
}%
\begin{pgfscope}%
\pgfsys@transformshift{4.504541in}{0.523570in}%
\pgfsys@useobject{currentmarker}{}%
\end{pgfscope}%
\end{pgfscope}%
\begin{pgfscope}%
\pgftext[x=4.504541in,y=0.426348in,,top]{\rmfamily\fontsize{10.000000}{12.000000}\selectfont \(\displaystyle 5000\)}%
\end{pgfscope}%
\begin{pgfscope}%
\pgfsetbuttcap%
\pgfsetroundjoin%
\definecolor{currentfill}{rgb}{0.000000,0.000000,0.000000}%
\pgfsetfillcolor{currentfill}%
\pgfsetlinewidth{0.803000pt}%
\definecolor{currentstroke}{rgb}{0.000000,0.000000,0.000000}%
\pgfsetstrokecolor{currentstroke}%
\pgfsetdash{}{0pt}%
\pgfsys@defobject{currentmarker}{\pgfqpoint{0.000000in}{-0.048611in}}{\pgfqpoint{0.000000in}{0.000000in}}{%
\pgfpathmoveto{\pgfqpoint{0.000000in}{0.000000in}}%
\pgfpathlineto{\pgfqpoint{0.000000in}{-0.048611in}}%
\pgfusepath{stroke,fill}%
}%
\begin{pgfscope}%
\pgfsys@transformshift{5.222073in}{0.523570in}%
\pgfsys@useobject{currentmarker}{}%
\end{pgfscope}%
\end{pgfscope}%
\begin{pgfscope}%
\pgftext[x=5.222073in,y=0.426348in,,top]{\rmfamily\fontsize{10.000000}{12.000000}\selectfont \(\displaystyle 6000\)}%
\end{pgfscope}%
\begin{pgfscope}%
\pgftext[x=3.030516in,y=0.236379in,,top]{\rmfamily\fontsize{10.000000}{12.000000}\selectfont Frequency (Hz)}%
\end{pgfscope}%
\begin{pgfscope}%
\pgfsetbuttcap%
\pgfsetroundjoin%
\definecolor{currentfill}{rgb}{0.000000,0.000000,0.000000}%
\pgfsetfillcolor{currentfill}%
\pgfsetlinewidth{0.803000pt}%
\definecolor{currentstroke}{rgb}{0.000000,0.000000,0.000000}%
\pgfsetstrokecolor{currentstroke}%
\pgfsetdash{}{0pt}%
\pgfsys@defobject{currentmarker}{\pgfqpoint{-0.048611in}{0.000000in}}{\pgfqpoint{0.000000in}{0.000000in}}{%
\pgfpathmoveto{\pgfqpoint{0.000000in}{0.000000in}}%
\pgfpathlineto{\pgfqpoint{-0.048611in}{0.000000in}}%
\pgfusepath{stroke,fill}%
}%
\begin{pgfscope}%
\pgfsys@transformshift{0.705516in}{0.908792in}%
\pgfsys@useobject{currentmarker}{}%
\end{pgfscope}%
\end{pgfscope}%
\begin{pgfscope}%
\pgftext[x=0.291935in,y=0.856030in,left,base]{\rmfamily\fontsize{10.000000}{12.000000}\selectfont \(\displaystyle -120\)}%
\end{pgfscope}%
\begin{pgfscope}%
\pgfsetbuttcap%
\pgfsetroundjoin%
\definecolor{currentfill}{rgb}{0.000000,0.000000,0.000000}%
\pgfsetfillcolor{currentfill}%
\pgfsetlinewidth{0.803000pt}%
\definecolor{currentstroke}{rgb}{0.000000,0.000000,0.000000}%
\pgfsetstrokecolor{currentstroke}%
\pgfsetdash{}{0pt}%
\pgfsys@defobject{currentmarker}{\pgfqpoint{-0.048611in}{0.000000in}}{\pgfqpoint{0.000000in}{0.000000in}}{%
\pgfpathmoveto{\pgfqpoint{0.000000in}{0.000000in}}%
\pgfpathlineto{\pgfqpoint{-0.048611in}{0.000000in}}%
\pgfusepath{stroke,fill}%
}%
\begin{pgfscope}%
\pgfsys@transformshift{0.705516in}{1.325043in}%
\pgfsys@useobject{currentmarker}{}%
\end{pgfscope}%
\end{pgfscope}%
\begin{pgfscope}%
\pgftext[x=0.291935in,y=1.272281in,left,base]{\rmfamily\fontsize{10.000000}{12.000000}\selectfont \(\displaystyle -100\)}%
\end{pgfscope}%
\begin{pgfscope}%
\pgfsetbuttcap%
\pgfsetroundjoin%
\definecolor{currentfill}{rgb}{0.000000,0.000000,0.000000}%
\pgfsetfillcolor{currentfill}%
\pgfsetlinewidth{0.803000pt}%
\definecolor{currentstroke}{rgb}{0.000000,0.000000,0.000000}%
\pgfsetstrokecolor{currentstroke}%
\pgfsetdash{}{0pt}%
\pgfsys@defobject{currentmarker}{\pgfqpoint{-0.048611in}{0.000000in}}{\pgfqpoint{0.000000in}{0.000000in}}{%
\pgfpathmoveto{\pgfqpoint{0.000000in}{0.000000in}}%
\pgfpathlineto{\pgfqpoint{-0.048611in}{0.000000in}}%
\pgfusepath{stroke,fill}%
}%
\begin{pgfscope}%
\pgfsys@transformshift{0.705516in}{1.741294in}%
\pgfsys@useobject{currentmarker}{}%
\end{pgfscope}%
\end{pgfscope}%
\begin{pgfscope}%
\pgftext[x=0.361380in,y=1.688532in,left,base]{\rmfamily\fontsize{10.000000}{12.000000}\selectfont \(\displaystyle -80\)}%
\end{pgfscope}%
\begin{pgfscope}%
\pgfsetbuttcap%
\pgfsetroundjoin%
\definecolor{currentfill}{rgb}{0.000000,0.000000,0.000000}%
\pgfsetfillcolor{currentfill}%
\pgfsetlinewidth{0.803000pt}%
\definecolor{currentstroke}{rgb}{0.000000,0.000000,0.000000}%
\pgfsetstrokecolor{currentstroke}%
\pgfsetdash{}{0pt}%
\pgfsys@defobject{currentmarker}{\pgfqpoint{-0.048611in}{0.000000in}}{\pgfqpoint{0.000000in}{0.000000in}}{%
\pgfpathmoveto{\pgfqpoint{0.000000in}{0.000000in}}%
\pgfpathlineto{\pgfqpoint{-0.048611in}{0.000000in}}%
\pgfusepath{stroke,fill}%
}%
\begin{pgfscope}%
\pgfsys@transformshift{0.705516in}{2.157545in}%
\pgfsys@useobject{currentmarker}{}%
\end{pgfscope}%
\end{pgfscope}%
\begin{pgfscope}%
\pgftext[x=0.361380in,y=2.104783in,left,base]{\rmfamily\fontsize{10.000000}{12.000000}\selectfont \(\displaystyle -60\)}%
\end{pgfscope}%
\begin{pgfscope}%
\pgfsetbuttcap%
\pgfsetroundjoin%
\definecolor{currentfill}{rgb}{0.000000,0.000000,0.000000}%
\pgfsetfillcolor{currentfill}%
\pgfsetlinewidth{0.803000pt}%
\definecolor{currentstroke}{rgb}{0.000000,0.000000,0.000000}%
\pgfsetstrokecolor{currentstroke}%
\pgfsetdash{}{0pt}%
\pgfsys@defobject{currentmarker}{\pgfqpoint{-0.048611in}{0.000000in}}{\pgfqpoint{0.000000in}{0.000000in}}{%
\pgfpathmoveto{\pgfqpoint{0.000000in}{0.000000in}}%
\pgfpathlineto{\pgfqpoint{-0.048611in}{0.000000in}}%
\pgfusepath{stroke,fill}%
}%
\begin{pgfscope}%
\pgfsys@transformshift{0.705516in}{2.573795in}%
\pgfsys@useobject{currentmarker}{}%
\end{pgfscope}%
\end{pgfscope}%
\begin{pgfscope}%
\pgftext[x=0.361380in,y=2.521034in,left,base]{\rmfamily\fontsize{10.000000}{12.000000}\selectfont \(\displaystyle -40\)}%
\end{pgfscope}%
\begin{pgfscope}%
\pgfsetbuttcap%
\pgfsetroundjoin%
\definecolor{currentfill}{rgb}{0.000000,0.000000,0.000000}%
\pgfsetfillcolor{currentfill}%
\pgfsetlinewidth{0.803000pt}%
\definecolor{currentstroke}{rgb}{0.000000,0.000000,0.000000}%
\pgfsetstrokecolor{currentstroke}%
\pgfsetdash{}{0pt}%
\pgfsys@defobject{currentmarker}{\pgfqpoint{-0.048611in}{0.000000in}}{\pgfqpoint{0.000000in}{0.000000in}}{%
\pgfpathmoveto{\pgfqpoint{0.000000in}{0.000000in}}%
\pgfpathlineto{\pgfqpoint{-0.048611in}{0.000000in}}%
\pgfusepath{stroke,fill}%
}%
\begin{pgfscope}%
\pgfsys@transformshift{0.705516in}{2.990046in}%
\pgfsys@useobject{currentmarker}{}%
\end{pgfscope}%
\end{pgfscope}%
\begin{pgfscope}%
\pgftext[x=0.361380in,y=2.937285in,left,base]{\rmfamily\fontsize{10.000000}{12.000000}\selectfont \(\displaystyle -20\)}%
\end{pgfscope}%
\begin{pgfscope}%
\pgfsetbuttcap%
\pgfsetroundjoin%
\definecolor{currentfill}{rgb}{0.000000,0.000000,0.000000}%
\pgfsetfillcolor{currentfill}%
\pgfsetlinewidth{0.803000pt}%
\definecolor{currentstroke}{rgb}{0.000000,0.000000,0.000000}%
\pgfsetstrokecolor{currentstroke}%
\pgfsetdash{}{0pt}%
\pgfsys@defobject{currentmarker}{\pgfqpoint{-0.048611in}{0.000000in}}{\pgfqpoint{0.000000in}{0.000000in}}{%
\pgfpathmoveto{\pgfqpoint{0.000000in}{0.000000in}}%
\pgfpathlineto{\pgfqpoint{-0.048611in}{0.000000in}}%
\pgfusepath{stroke,fill}%
}%
\begin{pgfscope}%
\pgfsys@transformshift{0.705516in}{3.406297in}%
\pgfsys@useobject{currentmarker}{}%
\end{pgfscope}%
\end{pgfscope}%
\begin{pgfscope}%
\pgftext[x=0.538849in,y=3.353536in,left,base]{\rmfamily\fontsize{10.000000}{12.000000}\selectfont \(\displaystyle 0\)}%
\end{pgfscope}%
\begin{pgfscope}%
\pgftext[x=0.236379in,y=2.033570in,,bottom,rotate=90.000000]{\rmfamily\fontsize{10.000000}{12.000000}\selectfont Magnitude (dB)}%
\end{pgfscope}%
\begin{pgfscope}%
\pgfpathrectangle{\pgfqpoint{0.705516in}{0.523570in}}{\pgfqpoint{4.650000in}{3.020000in}} %
\pgfusepath{clip}%
\pgfsetrectcap%
\pgfsetroundjoin%
\pgfsetlinewidth{1.505625pt}%
\definecolor{currentstroke}{rgb}{0.000000,0.000000,0.000000}%
\pgfsetstrokecolor{currentstroke}%
\pgfsetstrokeopacity{0.700000}%
\pgfsetdash{}{0pt}%
\pgfpathmoveto{\pgfqpoint{0.916880in}{3.406297in}}%
\pgfpathlineto{\pgfqpoint{0.988529in}{2.668798in}}%
\pgfpathlineto{\pgfqpoint{1.060177in}{2.228790in}}%
\pgfpathlineto{\pgfqpoint{1.131826in}{1.966121in}}%
\pgfpathlineto{\pgfqpoint{1.203475in}{1.784756in}}%
\pgfpathlineto{\pgfqpoint{1.275123in}{1.649075in}}%
\pgfpathlineto{\pgfqpoint{1.346772in}{1.542565in}}%
\pgfpathlineto{\pgfqpoint{1.418421in}{1.454360in}}%
\pgfpathlineto{\pgfqpoint{1.490069in}{1.375845in}}%
\pgfpathlineto{\pgfqpoint{1.561718in}{1.283599in}}%
\pgfpathlineto{\pgfqpoint{1.633367in}{1.313644in}}%
\pgfpathlineto{\pgfqpoint{1.705015in}{1.254328in}}%
\pgfpathlineto{\pgfqpoint{1.776664in}{1.230409in}}%
\pgfpathlineto{\pgfqpoint{1.848313in}{1.191472in}}%
\pgfpathlineto{\pgfqpoint{1.919962in}{1.156676in}}%
\pgfpathlineto{\pgfqpoint{1.991610in}{1.129872in}}%
\pgfpathlineto{\pgfqpoint{2.063259in}{1.108026in}}%
\pgfpathlineto{\pgfqpoint{2.134908in}{1.088093in}}%
\pgfpathlineto{\pgfqpoint{2.206556in}{1.082532in}}%
\pgfpathlineto{\pgfqpoint{2.278205in}{1.097681in}}%
\pgfpathlineto{\pgfqpoint{2.349854in}{1.272536in}}%
\pgfpathlineto{\pgfqpoint{2.421502in}{1.086494in}}%
\pgfpathlineto{\pgfqpoint{2.493151in}{1.004172in}}%
\pgfpathlineto{\pgfqpoint{2.564800in}{0.965381in}}%
\pgfpathlineto{\pgfqpoint{2.636448in}{0.939640in}}%
\pgfpathlineto{\pgfqpoint{2.708097in}{0.916632in}}%
\pgfpathlineto{\pgfqpoint{2.779746in}{0.901896in}}%
\pgfpathlineto{\pgfqpoint{2.851395in}{0.882169in}}%
\pgfpathlineto{\pgfqpoint{2.923043in}{0.863225in}}%
\pgfpathlineto{\pgfqpoint{2.994692in}{0.829628in}}%
\pgfpathlineto{\pgfqpoint{3.066341in}{0.736568in}}%
\pgfpathlineto{\pgfqpoint{3.137989in}{0.876987in}}%
\pgfpathlineto{\pgfqpoint{3.209638in}{0.847826in}}%
\pgfpathlineto{\pgfqpoint{3.281287in}{0.845238in}}%
\pgfpathlineto{\pgfqpoint{3.352935in}{0.769978in}}%
\pgfpathlineto{\pgfqpoint{3.424584in}{0.780304in}}%
\pgfpathlineto{\pgfqpoint{3.496233in}{0.784584in}}%
\pgfpathlineto{\pgfqpoint{3.567881in}{0.791042in}}%
\pgfpathlineto{\pgfqpoint{3.639530in}{0.763833in}}%
\pgfpathlineto{\pgfqpoint{3.711179in}{0.746188in}}%
\pgfpathlineto{\pgfqpoint{3.782827in}{0.757916in}}%
\pgfpathlineto{\pgfqpoint{3.854476in}{0.757447in}}%
\pgfpathlineto{\pgfqpoint{3.926125in}{0.775771in}}%
\pgfpathlineto{\pgfqpoint{3.997774in}{0.795559in}}%
\pgfpathlineto{\pgfqpoint{4.069422in}{0.833367in}}%
\pgfpathlineto{\pgfqpoint{4.141071in}{0.758128in}}%
\pgfpathlineto{\pgfqpoint{4.212720in}{0.762727in}}%
\pgfpathlineto{\pgfqpoint{4.284368in}{0.760962in}}%
\pgfpathlineto{\pgfqpoint{4.356017in}{0.745143in}}%
\pgfpathlineto{\pgfqpoint{4.427666in}{0.758238in}}%
\pgfpathlineto{\pgfqpoint{4.499314in}{0.759372in}}%
\pgfpathlineto{\pgfqpoint{4.570963in}{0.764988in}}%
\pgfpathlineto{\pgfqpoint{4.642612in}{0.785393in}}%
\pgfpathlineto{\pgfqpoint{4.714260in}{0.754508in}}%
\pgfpathlineto{\pgfqpoint{4.785909in}{0.740646in}}%
\pgfpathlineto{\pgfqpoint{4.857558in}{0.744052in}}%
\pgfpathlineto{\pgfqpoint{4.929207in}{0.728294in}}%
\pgfpathlineto{\pgfqpoint{5.000855in}{0.711100in}}%
\pgfpathlineto{\pgfqpoint{5.072504in}{0.693256in}}%
\pgfpathlineto{\pgfqpoint{5.144153in}{0.714114in}}%
\pgfusepath{stroke}%
\end{pgfscope}%
\begin{pgfscope}%
\pgfpathrectangle{\pgfqpoint{0.705516in}{0.523570in}}{\pgfqpoint{4.650000in}{3.020000in}} %
\pgfusepath{clip}%
\pgfsetrectcap%
\pgfsetroundjoin%
\pgfsetlinewidth{1.505625pt}%
\definecolor{currentstroke}{rgb}{0.750000,0.000000,0.750000}%
\pgfsetstrokecolor{currentstroke}%
\pgfsetstrokeopacity{0.700000}%
\pgfsetdash{}{0pt}%
\pgfpathmoveto{\pgfqpoint{0.916880in}{3.406297in}}%
\pgfpathlineto{\pgfqpoint{0.988529in}{2.631281in}}%
\pgfpathlineto{\pgfqpoint{1.060177in}{2.087149in}}%
\pgfpathlineto{\pgfqpoint{1.131826in}{1.709057in}}%
\pgfpathlineto{\pgfqpoint{1.203475in}{1.507406in}}%
\pgfpathlineto{\pgfqpoint{1.275123in}{1.409081in}}%
\pgfpathlineto{\pgfqpoint{1.346772in}{1.337310in}}%
\pgfpathlineto{\pgfqpoint{1.418421in}{1.279499in}}%
\pgfpathlineto{\pgfqpoint{1.490069in}{1.230979in}}%
\pgfpathlineto{\pgfqpoint{1.561718in}{1.187992in}}%
\pgfpathlineto{\pgfqpoint{1.633367in}{1.150673in}}%
\pgfpathlineto{\pgfqpoint{1.705015in}{1.115178in}}%
\pgfpathlineto{\pgfqpoint{1.776664in}{1.083447in}}%
\pgfpathlineto{\pgfqpoint{1.848313in}{1.055686in}}%
\pgfpathlineto{\pgfqpoint{1.919962in}{1.030274in}}%
\pgfpathlineto{\pgfqpoint{1.991610in}{1.006603in}}%
\pgfpathlineto{\pgfqpoint{2.063259in}{0.984307in}}%
\pgfpathlineto{\pgfqpoint{2.134908in}{0.963775in}}%
\pgfpathlineto{\pgfqpoint{2.206556in}{0.945069in}}%
\pgfpathlineto{\pgfqpoint{2.278205in}{0.928634in}}%
\pgfpathlineto{\pgfqpoint{2.349854in}{0.903630in}}%
\pgfpathlineto{\pgfqpoint{2.421502in}{0.895061in}}%
\pgfpathlineto{\pgfqpoint{2.493151in}{0.878691in}}%
\pgfpathlineto{\pgfqpoint{2.564800in}{0.864213in}}%
\pgfpathlineto{\pgfqpoint{2.636448in}{0.850827in}}%
\pgfpathlineto{\pgfqpoint{2.708097in}{0.838175in}}%
\pgfpathlineto{\pgfqpoint{2.779746in}{0.826095in}}%
\pgfpathlineto{\pgfqpoint{2.851395in}{0.814492in}}%
\pgfpathlineto{\pgfqpoint{2.923043in}{0.803623in}}%
\pgfpathlineto{\pgfqpoint{2.994692in}{0.793641in}}%
\pgfpathlineto{\pgfqpoint{3.066341in}{0.784348in}}%
\pgfpathlineto{\pgfqpoint{3.137989in}{0.774817in}}%
\pgfpathlineto{\pgfqpoint{3.209638in}{0.766220in}}%
\pgfpathlineto{\pgfqpoint{3.281287in}{0.758010in}}%
\pgfpathlineto{\pgfqpoint{3.352935in}{0.750098in}}%
\pgfpathlineto{\pgfqpoint{3.424584in}{0.742680in}}%
\pgfpathlineto{\pgfqpoint{3.496233in}{0.735512in}}%
\pgfpathlineto{\pgfqpoint{3.567881in}{0.728879in}}%
\pgfpathlineto{\pgfqpoint{3.639530in}{0.722662in}}%
\pgfpathlineto{\pgfqpoint{3.711179in}{0.716830in}}%
\pgfpathlineto{\pgfqpoint{3.782827in}{0.711459in}}%
\pgfpathlineto{\pgfqpoint{3.854476in}{0.706093in}}%
\pgfpathlineto{\pgfqpoint{3.926125in}{0.700673in}}%
\pgfpathlineto{\pgfqpoint{3.997774in}{0.695985in}}%
\pgfpathlineto{\pgfqpoint{4.069422in}{0.692351in}}%
\pgfpathlineto{\pgfqpoint{4.141071in}{0.687918in}}%
\pgfpathlineto{\pgfqpoint{4.212720in}{0.684348in}}%
\pgfpathlineto{\pgfqpoint{4.284368in}{0.680921in}}%
\pgfpathlineto{\pgfqpoint{4.356017in}{0.677755in}}%
\pgfpathlineto{\pgfqpoint{4.427666in}{0.674800in}}%
\pgfpathlineto{\pgfqpoint{4.499314in}{0.672294in}}%
\pgfpathlineto{\pgfqpoint{4.570963in}{0.670082in}}%
\pgfpathlineto{\pgfqpoint{4.642612in}{0.667967in}}%
\pgfpathlineto{\pgfqpoint{4.714260in}{0.666236in}}%
\pgfpathlineto{\pgfqpoint{4.785909in}{0.664899in}}%
\pgfpathlineto{\pgfqpoint{4.857558in}{0.663688in}}%
\pgfpathlineto{\pgfqpoint{4.929207in}{0.662553in}}%
\pgfpathlineto{\pgfqpoint{5.000855in}{0.661585in}}%
\pgfpathlineto{\pgfqpoint{5.072504in}{0.661080in}}%
\pgfpathlineto{\pgfqpoint{5.144153in}{0.660843in}}%
\pgfusepath{stroke}%
\end{pgfscope}%
\begin{pgfscope}%
\pgfpathrectangle{\pgfqpoint{0.705516in}{0.523570in}}{\pgfqpoint{4.650000in}{3.020000in}} %
\pgfusepath{clip}%
\pgfsetrectcap%
\pgfsetroundjoin%
\pgfsetlinewidth{1.505625pt}%
\definecolor{currentstroke}{rgb}{0.000000,0.000000,1.000000}%
\pgfsetstrokecolor{currentstroke}%
\pgfsetstrokeopacity{0.700000}%
\pgfsetdash{}{0pt}%
\pgfpathmoveto{\pgfqpoint{0.916880in}{3.406297in}}%
\pgfpathlineto{\pgfqpoint{0.988529in}{2.669290in}}%
\pgfpathlineto{\pgfqpoint{1.060177in}{2.230600in}}%
\pgfpathlineto{\pgfqpoint{1.131826in}{1.970562in}}%
\pgfpathlineto{\pgfqpoint{1.203475in}{1.791137in}}%
\pgfpathlineto{\pgfqpoint{1.275123in}{1.647572in}}%
\pgfpathlineto{\pgfqpoint{1.346772in}{1.522234in}}%
\pgfpathlineto{\pgfqpoint{1.418421in}{1.422726in}}%
\pgfpathlineto{\pgfqpoint{1.490069in}{1.346855in}}%
\pgfpathlineto{\pgfqpoint{1.561718in}{1.286286in}}%
\pgfpathlineto{\pgfqpoint{1.633367in}{1.237044in}}%
\pgfpathlineto{\pgfqpoint{1.705015in}{1.193754in}}%
\pgfpathlineto{\pgfqpoint{1.776664in}{1.155940in}}%
\pgfpathlineto{\pgfqpoint{1.848313in}{1.122285in}}%
\pgfpathlineto{\pgfqpoint{1.919962in}{1.092028in}}%
\pgfpathlineto{\pgfqpoint{1.991610in}{1.064573in}}%
\pgfpathlineto{\pgfqpoint{2.063259in}{1.039473in}}%
\pgfpathlineto{\pgfqpoint{2.134908in}{1.016387in}}%
\pgfpathlineto{\pgfqpoint{2.206556in}{0.995047in}}%
\pgfpathlineto{\pgfqpoint{2.278205in}{0.975237in}}%
\pgfpathlineto{\pgfqpoint{2.349854in}{0.956770in}}%
\pgfpathlineto{\pgfqpoint{2.421502in}{0.939537in}}%
\pgfpathlineto{\pgfqpoint{2.493151in}{0.923380in}}%
\pgfpathlineto{\pgfqpoint{2.564800in}{0.908210in}}%
\pgfpathlineto{\pgfqpoint{2.636448in}{0.893941in}}%
\pgfpathlineto{\pgfqpoint{2.708097in}{0.880497in}}%
\pgfpathlineto{\pgfqpoint{2.779746in}{0.867815in}}%
\pgfpathlineto{\pgfqpoint{2.851395in}{0.855839in}}%
\pgfpathlineto{\pgfqpoint{2.923043in}{0.844520in}}%
\pgfpathlineto{\pgfqpoint{2.994692in}{0.833816in}}%
\pgfpathlineto{\pgfqpoint{3.066341in}{0.823687in}}%
\pgfpathlineto{\pgfqpoint{3.137989in}{0.814100in}}%
\pgfpathlineto{\pgfqpoint{3.209638in}{0.805027in}}%
\pgfpathlineto{\pgfqpoint{3.281287in}{0.796439in}}%
\pgfpathlineto{\pgfqpoint{3.352935in}{0.788312in}}%
\pgfpathlineto{\pgfqpoint{3.424584in}{0.780626in}}%
\pgfpathlineto{\pgfqpoint{3.496233in}{0.773361in}}%
\pgfpathlineto{\pgfqpoint{3.567881in}{0.766498in}}%
\pgfpathlineto{\pgfqpoint{3.639530in}{0.760023in}}%
\pgfpathlineto{\pgfqpoint{3.711179in}{0.753921in}}%
\pgfpathlineto{\pgfqpoint{3.782827in}{0.748178in}}%
\pgfpathlineto{\pgfqpoint{3.854476in}{0.742784in}}%
\pgfpathlineto{\pgfqpoint{3.926125in}{0.737726in}}%
\pgfpathlineto{\pgfqpoint{3.997774in}{0.732996in}}%
\pgfpathlineto{\pgfqpoint{4.069422in}{0.728585in}}%
\pgfpathlineto{\pgfqpoint{4.141071in}{0.724484in}}%
\pgfpathlineto{\pgfqpoint{4.212720in}{0.720685in}}%
\pgfpathlineto{\pgfqpoint{4.284368in}{0.717184in}}%
\pgfpathlineto{\pgfqpoint{4.356017in}{0.713973in}}%
\pgfpathlineto{\pgfqpoint{4.427666in}{0.711047in}}%
\pgfpathlineto{\pgfqpoint{4.499314in}{0.708401in}}%
\pgfpathlineto{\pgfqpoint{4.570963in}{0.706032in}}%
\pgfpathlineto{\pgfqpoint{4.642612in}{0.703935in}}%
\pgfpathlineto{\pgfqpoint{4.714260in}{0.702107in}}%
\pgfpathlineto{\pgfqpoint{4.785909in}{0.700546in}}%
\pgfpathlineto{\pgfqpoint{4.857558in}{0.699248in}}%
\pgfpathlineto{\pgfqpoint{4.929207in}{0.698212in}}%
\pgfpathlineto{\pgfqpoint{5.000855in}{0.697436in}}%
\pgfpathlineto{\pgfqpoint{5.072504in}{0.696920in}}%
\pgfpathlineto{\pgfqpoint{5.144153in}{0.696662in}}%
\pgfusepath{stroke}%
\end{pgfscope}%
\begin{pgfscope}%
\pgfpathrectangle{\pgfqpoint{0.705516in}{0.523570in}}{\pgfqpoint{4.650000in}{3.020000in}} %
\pgfusepath{clip}%
\pgfsetrectcap%
\pgfsetroundjoin%
\pgfsetlinewidth{1.505625pt}%
\definecolor{currentstroke}{rgb}{0.000000,0.750000,0.750000}%
\pgfsetstrokecolor{currentstroke}%
\pgfsetstrokeopacity{0.700000}%
\pgfsetdash{}{0pt}%
\pgfpathmoveto{\pgfqpoint{0.916880in}{3.406297in}}%
\pgfpathlineto{\pgfqpoint{0.988529in}{2.738258in}}%
\pgfpathlineto{\pgfqpoint{1.060177in}{2.368762in}}%
\pgfpathlineto{\pgfqpoint{1.131826in}{2.142808in}}%
\pgfpathlineto{\pgfqpoint{1.203475in}{1.939810in}}%
\pgfpathlineto{\pgfqpoint{1.275123in}{1.720237in}}%
\pgfpathlineto{\pgfqpoint{1.346772in}{1.498168in}}%
\pgfpathlineto{\pgfqpoint{1.418421in}{1.366898in}}%
\pgfpathlineto{\pgfqpoint{1.490069in}{1.327581in}}%
\pgfpathlineto{\pgfqpoint{1.561718in}{1.313894in}}%
\pgfpathlineto{\pgfqpoint{1.633367in}{1.337306in}}%
\pgfpathlineto{\pgfqpoint{1.705015in}{1.345551in}}%
\pgfpathlineto{\pgfqpoint{1.776664in}{1.317370in}}%
\pgfpathlineto{\pgfqpoint{1.848313in}{1.238848in}}%
\pgfpathlineto{\pgfqpoint{1.919962in}{1.119125in}}%
\pgfpathlineto{\pgfqpoint{1.991610in}{1.006635in}}%
\pgfpathlineto{\pgfqpoint{2.063259in}{0.972229in}}%
\pgfpathlineto{\pgfqpoint{2.134908in}{0.988688in}}%
\pgfpathlineto{\pgfqpoint{2.206556in}{1.012790in}}%
\pgfpathlineto{\pgfqpoint{2.278205in}{1.029647in}}%
\pgfpathlineto{\pgfqpoint{2.349854in}{1.009553in}}%
\pgfpathlineto{\pgfqpoint{2.421502in}{1.002992in}}%
\pgfpathlineto{\pgfqpoint{2.493151in}{0.958155in}}%
\pgfpathlineto{\pgfqpoint{2.564800in}{0.913618in}}%
\pgfpathlineto{\pgfqpoint{2.636448in}{0.882517in}}%
\pgfpathlineto{\pgfqpoint{2.708097in}{0.861802in}}%
\pgfpathlineto{\pgfqpoint{2.779746in}{0.849567in}}%
\pgfpathlineto{\pgfqpoint{2.851395in}{0.848267in}}%
\pgfpathlineto{\pgfqpoint{2.923043in}{0.848973in}}%
\pgfpathlineto{\pgfqpoint{2.994692in}{0.839250in}}%
\pgfpathlineto{\pgfqpoint{3.066341in}{0.822268in}}%
\pgfpathlineto{\pgfqpoint{3.137989in}{0.800486in}}%
\pgfpathlineto{\pgfqpoint{3.209638in}{0.800974in}}%
\pgfpathlineto{\pgfqpoint{3.281287in}{0.805518in}}%
\pgfpathlineto{\pgfqpoint{3.352935in}{0.805495in}}%
\pgfpathlineto{\pgfqpoint{3.424584in}{0.805810in}}%
\pgfpathlineto{\pgfqpoint{3.496233in}{0.801884in}}%
\pgfpathlineto{\pgfqpoint{3.567881in}{0.785392in}}%
\pgfpathlineto{\pgfqpoint{3.639530in}{0.757062in}}%
\pgfpathlineto{\pgfqpoint{3.711179in}{0.730151in}}%
\pgfpathlineto{\pgfqpoint{3.782827in}{0.718249in}}%
\pgfpathlineto{\pgfqpoint{3.854476in}{0.718947in}}%
\pgfpathlineto{\pgfqpoint{3.926125in}{0.721998in}}%
\pgfpathlineto{\pgfqpoint{3.997774in}{0.727167in}}%
\pgfpathlineto{\pgfqpoint{4.069422in}{0.733702in}}%
\pgfpathlineto{\pgfqpoint{4.141071in}{0.731916in}}%
\pgfpathlineto{\pgfqpoint{4.212720in}{0.723515in}}%
\pgfpathlineto{\pgfqpoint{4.284368in}{0.711688in}}%
\pgfpathlineto{\pgfqpoint{4.356017in}{0.706642in}}%
\pgfpathlineto{\pgfqpoint{4.427666in}{0.707298in}}%
\pgfpathlineto{\pgfqpoint{4.499314in}{0.708779in}}%
\pgfpathlineto{\pgfqpoint{4.570963in}{0.708528in}}%
\pgfpathlineto{\pgfqpoint{4.642612in}{0.707277in}}%
\pgfpathlineto{\pgfqpoint{4.714260in}{0.700862in}}%
\pgfpathlineto{\pgfqpoint{4.785909in}{0.690873in}}%
\pgfpathlineto{\pgfqpoint{4.857558in}{0.681881in}}%
\pgfpathlineto{\pgfqpoint{4.929207in}{0.682714in}}%
\pgfpathlineto{\pgfqpoint{5.000855in}{0.689629in}}%
\pgfpathlineto{\pgfqpoint{5.072504in}{0.692982in}}%
\pgfpathlineto{\pgfqpoint{5.144153in}{0.692852in}}%
\pgfusepath{stroke}%
\end{pgfscope}%
\begin{pgfscope}%
\pgfpathrectangle{\pgfqpoint{0.705516in}{0.523570in}}{\pgfqpoint{4.650000in}{3.020000in}} %
\pgfusepath{clip}%
\pgfsetrectcap%
\pgfsetroundjoin%
\pgfsetlinewidth{1.505625pt}%
\definecolor{currentstroke}{rgb}{1.000000,0.000000,0.000000}%
\pgfsetstrokecolor{currentstroke}%
\pgfsetstrokeopacity{0.700000}%
\pgfsetdash{}{0pt}%
\pgfpathmoveto{\pgfqpoint{0.916880in}{3.406297in}}%
\pgfpathlineto{\pgfqpoint{0.988529in}{2.668833in}}%
\pgfpathlineto{\pgfqpoint{1.060177in}{2.228948in}}%
\pgfpathlineto{\pgfqpoint{1.131826in}{1.965694in}}%
\pgfpathlineto{\pgfqpoint{1.203475in}{1.782735in}}%
\pgfpathlineto{\pgfqpoint{1.275123in}{1.646049in}}%
\pgfpathlineto{\pgfqpoint{1.346772in}{1.540534in}}%
\pgfpathlineto{\pgfqpoint{1.418421in}{1.456474in}}%
\pgfpathlineto{\pgfqpoint{1.490069in}{1.387279in}}%
\pgfpathlineto{\pgfqpoint{1.561718in}{1.328930in}}%
\pgfpathlineto{\pgfqpoint{1.633367in}{1.278882in}}%
\pgfpathlineto{\pgfqpoint{1.705015in}{1.235176in}}%
\pgfpathlineto{\pgfqpoint{1.776664in}{1.196611in}}%
\pgfpathlineto{\pgfqpoint{1.848313in}{1.162175in}}%
\pgfpathlineto{\pgfqpoint{1.919962in}{1.131146in}}%
\pgfpathlineto{\pgfqpoint{1.991610in}{1.102974in}}%
\pgfpathlineto{\pgfqpoint{2.063259in}{1.077210in}}%
\pgfpathlineto{\pgfqpoint{2.134908in}{1.053525in}}%
\pgfpathlineto{\pgfqpoint{2.206556in}{1.031663in}}%
\pgfpathlineto{\pgfqpoint{2.278205in}{1.011394in}}%
\pgfpathlineto{\pgfqpoint{2.349854in}{0.992471in}}%
\pgfpathlineto{\pgfqpoint{2.421502in}{0.974964in}}%
\pgfpathlineto{\pgfqpoint{2.493151in}{0.958446in}}%
\pgfpathlineto{\pgfqpoint{2.564800in}{0.942982in}}%
\pgfpathlineto{\pgfqpoint{2.636448in}{0.928458in}}%
\pgfpathlineto{\pgfqpoint{2.708097in}{0.914784in}}%
\pgfpathlineto{\pgfqpoint{2.779746in}{0.901892in}}%
\pgfpathlineto{\pgfqpoint{2.851395in}{0.889723in}}%
\pgfpathlineto{\pgfqpoint{2.923043in}{0.878231in}}%
\pgfpathlineto{\pgfqpoint{2.994692in}{0.867371in}}%
\pgfpathlineto{\pgfqpoint{3.066341in}{0.857102in}}%
\pgfpathlineto{\pgfqpoint{3.137989in}{0.847396in}}%
\pgfpathlineto{\pgfqpoint{3.209638in}{0.838205in}}%
\pgfpathlineto{\pgfqpoint{3.281287in}{0.829513in}}%
\pgfpathlineto{\pgfqpoint{3.352935in}{0.821293in}}%
\pgfpathlineto{\pgfqpoint{3.424584in}{0.813520in}}%
\pgfpathlineto{\pgfqpoint{3.496233in}{0.806174in}}%
\pgfpathlineto{\pgfqpoint{3.567881in}{0.799235in}}%
\pgfpathlineto{\pgfqpoint{3.639530in}{0.792690in}}%
\pgfpathlineto{\pgfqpoint{3.711179in}{0.786526in}}%
\pgfpathlineto{\pgfqpoint{3.782827in}{0.780728in}}%
\pgfpathlineto{\pgfqpoint{3.854476in}{0.775283in}}%
\pgfpathlineto{\pgfqpoint{3.926125in}{0.770179in}}%
\pgfpathlineto{\pgfqpoint{3.997774in}{0.765408in}}%
\pgfpathlineto{\pgfqpoint{4.069422in}{0.760959in}}%
\pgfpathlineto{\pgfqpoint{4.141071in}{0.756822in}}%
\pgfpathlineto{\pgfqpoint{4.212720in}{0.752990in}}%
\pgfpathlineto{\pgfqpoint{4.284368in}{0.749457in}}%
\pgfpathlineto{\pgfqpoint{4.356017in}{0.746217in}}%
\pgfpathlineto{\pgfqpoint{4.427666in}{0.743267in}}%
\pgfpathlineto{\pgfqpoint{4.499314in}{0.740600in}}%
\pgfpathlineto{\pgfqpoint{4.570963in}{0.738212in}}%
\pgfpathlineto{\pgfqpoint{4.642612in}{0.736100in}}%
\pgfpathlineto{\pgfqpoint{4.714260in}{0.734260in}}%
\pgfpathlineto{\pgfqpoint{4.785909in}{0.732688in}}%
\pgfpathlineto{\pgfqpoint{4.857558in}{0.731380in}}%
\pgfpathlineto{\pgfqpoint{4.929207in}{0.730335in}}%
\pgfpathlineto{\pgfqpoint{5.000855in}{0.729551in}}%
\pgfpathlineto{\pgfqpoint{5.072504in}{0.729029in}}%
\pgfpathlineto{\pgfqpoint{5.144153in}{0.728769in}}%
\pgfusepath{stroke}%
\end{pgfscope}%
\begin{pgfscope}%
\pgfsetrectcap%
\pgfsetmiterjoin%
\pgfsetlinewidth{0.803000pt}%
\definecolor{currentstroke}{rgb}{0.000000,0.000000,0.000000}%
\pgfsetstrokecolor{currentstroke}%
\pgfsetdash{}{0pt}%
\pgfpathmoveto{\pgfqpoint{0.705516in}{0.523570in}}%
\pgfpathlineto{\pgfqpoint{0.705516in}{3.543570in}}%
\pgfusepath{stroke}%
\end{pgfscope}%
\begin{pgfscope}%
\pgfsetrectcap%
\pgfsetmiterjoin%
\pgfsetlinewidth{0.803000pt}%
\definecolor{currentstroke}{rgb}{0.000000,0.000000,0.000000}%
\pgfsetstrokecolor{currentstroke}%
\pgfsetdash{}{0pt}%
\pgfpathmoveto{\pgfqpoint{5.355516in}{0.523570in}}%
\pgfpathlineto{\pgfqpoint{5.355516in}{3.543570in}}%
\pgfusepath{stroke}%
\end{pgfscope}%
\begin{pgfscope}%
\pgfsetrectcap%
\pgfsetmiterjoin%
\pgfsetlinewidth{0.803000pt}%
\definecolor{currentstroke}{rgb}{0.000000,0.000000,0.000000}%
\pgfsetstrokecolor{currentstroke}%
\pgfsetdash{}{0pt}%
\pgfpathmoveto{\pgfqpoint{0.705516in}{0.523570in}}%
\pgfpathlineto{\pgfqpoint{5.355516in}{0.523570in}}%
\pgfusepath{stroke}%
\end{pgfscope}%
\begin{pgfscope}%
\pgfsetrectcap%
\pgfsetmiterjoin%
\pgfsetlinewidth{0.803000pt}%
\definecolor{currentstroke}{rgb}{0.000000,0.000000,0.000000}%
\pgfsetstrokecolor{currentstroke}%
\pgfsetdash{}{0pt}%
\pgfpathmoveto{\pgfqpoint{0.705516in}{3.543570in}}%
\pgfpathlineto{\pgfqpoint{5.355516in}{3.543570in}}%
\pgfusepath{stroke}%
\end{pgfscope}%
\begin{pgfscope}%
\pgfsetbuttcap%
\pgfsetmiterjoin%
\definecolor{currentfill}{rgb}{1.000000,1.000000,1.000000}%
\pgfsetfillcolor{currentfill}%
\pgfsetfillopacity{0.800000}%
\pgfsetlinewidth{1.003750pt}%
\definecolor{currentstroke}{rgb}{0.800000,0.800000,0.800000}%
\pgfsetstrokecolor{currentstroke}%
\pgfsetstrokeopacity{0.800000}%
\pgfsetdash{}{0pt}%
\pgfpathmoveto{\pgfqpoint{3.063293in}{2.409239in}}%
\pgfpathlineto{\pgfqpoint{5.258294in}{2.409239in}}%
\pgfpathquadraticcurveto{\pgfqpoint{5.286072in}{2.409239in}}{\pgfqpoint{5.286072in}{2.437017in}}%
\pgfpathlineto{\pgfqpoint{5.286072in}{3.446348in}}%
\pgfpathquadraticcurveto{\pgfqpoint{5.286072in}{3.474126in}}{\pgfqpoint{5.258294in}{3.474126in}}%
\pgfpathlineto{\pgfqpoint{3.063293in}{3.474126in}}%
\pgfpathquadraticcurveto{\pgfqpoint{3.035516in}{3.474126in}}{\pgfqpoint{3.035516in}{3.446348in}}%
\pgfpathlineto{\pgfqpoint{3.035516in}{2.437017in}}%
\pgfpathquadraticcurveto{\pgfqpoint{3.035516in}{2.409239in}}{\pgfqpoint{3.063293in}{2.409239in}}%
\pgfpathclose%
\pgfusepath{stroke,fill}%
\end{pgfscope}%
\begin{pgfscope}%
\pgfsetrectcap%
\pgfsetroundjoin%
\pgfsetlinewidth{1.505625pt}%
\definecolor{currentstroke}{rgb}{0.000000,0.000000,0.000000}%
\pgfsetstrokecolor{currentstroke}%
\pgfsetstrokeopacity{0.700000}%
\pgfsetdash{}{0pt}%
\pgfpathmoveto{\pgfqpoint{3.091071in}{3.361658in}}%
\pgfpathlineto{\pgfqpoint{3.368849in}{3.361658in}}%
\pgfusepath{stroke}%
\end{pgfscope}%
\begin{pgfscope}%
\pgftext[x=3.479960in,y=3.313047in,left,base]{\rmfamily\fontsize{10.000000}{12.000000}\selectfont Raw signal}%
\end{pgfscope}%
\begin{pgfscope}%
\pgfsetrectcap%
\pgfsetroundjoin%
\pgfsetlinewidth{1.505625pt}%
\definecolor{currentstroke}{rgb}{0.750000,0.000000,0.750000}%
\pgfsetstrokecolor{currentstroke}%
\pgfsetstrokeopacity{0.700000}%
\pgfsetdash{}{0pt}%
\pgfpathmoveto{\pgfqpoint{3.091071in}{3.155834in}}%
\pgfpathlineto{\pgfqpoint{3.368849in}{3.155834in}}%
\pgfusepath{stroke}%
\end{pgfscope}%
\begin{pgfscope}%
\pgftext[x=3.479960in,y=3.107223in,left,base]{\rmfamily\fontsize{10.000000}{12.000000}\selectfont 1D Gaussian convolution}%
\end{pgfscope}%
\begin{pgfscope}%
\pgfsetrectcap%
\pgfsetroundjoin%
\pgfsetlinewidth{1.505625pt}%
\definecolor{currentstroke}{rgb}{0.000000,0.000000,1.000000}%
\pgfsetstrokecolor{currentstroke}%
\pgfsetstrokeopacity{0.700000}%
\pgfsetdash{}{0pt}%
\pgfpathmoveto{\pgfqpoint{3.091071in}{2.951977in}}%
\pgfpathlineto{\pgfqpoint{3.368849in}{2.951977in}}%
\pgfusepath{stroke}%
\end{pgfscope}%
\begin{pgfscope}%
\pgftext[x=3.479960in,y=2.903366in,left,base]{\rmfamily\fontsize{10.000000}{12.000000}\selectfont Butterworth}%
\end{pgfscope}%
\begin{pgfscope}%
\pgfsetrectcap%
\pgfsetroundjoin%
\pgfsetlinewidth{1.505625pt}%
\definecolor{currentstroke}{rgb}{0.000000,0.750000,0.750000}%
\pgfsetstrokecolor{currentstroke}%
\pgfsetstrokeopacity{0.700000}%
\pgfsetdash{}{0pt}%
\pgfpathmoveto{\pgfqpoint{3.091071in}{2.748120in}}%
\pgfpathlineto{\pgfqpoint{3.368849in}{2.748120in}}%
\pgfusepath{stroke}%
\end{pgfscope}%
\begin{pgfscope}%
\pgftext[x=3.479960in,y=2.699509in,left,base]{\rmfamily\fontsize{10.000000}{12.000000}\selectfont Wiener}%
\end{pgfscope}%
\begin{pgfscope}%
\pgfsetrectcap%
\pgfsetroundjoin%
\pgfsetlinewidth{1.505625pt}%
\definecolor{currentstroke}{rgb}{1.000000,0.000000,0.000000}%
\pgfsetstrokecolor{currentstroke}%
\pgfsetstrokeopacity{0.700000}%
\pgfsetdash{}{0pt}%
\pgfpathmoveto{\pgfqpoint{3.091071in}{2.544263in}}%
\pgfpathlineto{\pgfqpoint{3.368849in}{2.544263in}}%
\pgfusepath{stroke}%
\end{pgfscope}%
\begin{pgfscope}%
\pgftext[x=3.479960in,y=2.495651in,left,base]{\rmfamily\fontsize{10.000000}{12.000000}\selectfont Savitsky-Golay}%
\end{pgfscope}%
\end{pgfpicture}%
\makeatother%
\endgroup%
}
%\caption{Power spectra of the filter methods compared.}\label{fig:myfigure}
%\end{figure}

\begin{figure}
    \centering
    %% Creator: Matplotlib, PGF backend
%%
%% To include the figure in your LaTeX document, write
%%   \input{<filename>.pgf}
%%
%% Make sure the required packages are loaded in your preamble
%%   \usepackage{pgf}
%%
%% Figures using additional raster images can only be included by \input if
%% they are in the same directory as the main LaTeX file. For loading figures
%% from other directories you can use the `import` package
%%   \usepackage{import}
%% and then include the figures with
%%   \import{<path to file>}{<filename>.pgf}
%%
%% Matplotlib used the following preamble
%%   \usepackage{fontspec}
%%   \setmainfont{DejaVu Serif}
%%   \setsansfont{DejaVu Sans}
%%   \setmonofont{DejaVu Sans Mono}
%%
\begingroup%
\makeatletter%
\begin{pgfpicture}%
\pgfpathrectangle{\pgfpointorigin}{\pgfqpoint{5.490516in}{3.678570in}}%
\pgfusepath{use as bounding box, clip}%
\begin{pgfscope}%
\pgfsetbuttcap%
\pgfsetmiterjoin%
\definecolor{currentfill}{rgb}{1.000000,1.000000,1.000000}%
\pgfsetfillcolor{currentfill}%
\pgfsetlinewidth{0.000000pt}%
\definecolor{currentstroke}{rgb}{1.000000,1.000000,1.000000}%
\pgfsetstrokecolor{currentstroke}%
\pgfsetdash{}{0pt}%
\pgfpathmoveto{\pgfqpoint{0.000000in}{0.000000in}}%
\pgfpathlineto{\pgfqpoint{5.490516in}{0.000000in}}%
\pgfpathlineto{\pgfqpoint{5.490516in}{3.678570in}}%
\pgfpathlineto{\pgfqpoint{0.000000in}{3.678570in}}%
\pgfpathclose%
\pgfusepath{fill}%
\end{pgfscope}%
\begin{pgfscope}%
\pgfsetbuttcap%
\pgfsetmiterjoin%
\definecolor{currentfill}{rgb}{1.000000,1.000000,1.000000}%
\pgfsetfillcolor{currentfill}%
\pgfsetlinewidth{0.000000pt}%
\definecolor{currentstroke}{rgb}{0.000000,0.000000,0.000000}%
\pgfsetstrokecolor{currentstroke}%
\pgfsetstrokeopacity{0.000000}%
\pgfsetdash{}{0pt}%
\pgfpathmoveto{\pgfqpoint{0.705516in}{0.523570in}}%
\pgfpathlineto{\pgfqpoint{5.355516in}{0.523570in}}%
\pgfpathlineto{\pgfqpoint{5.355516in}{3.543570in}}%
\pgfpathlineto{\pgfqpoint{0.705516in}{3.543570in}}%
\pgfpathclose%
\pgfusepath{fill}%
\end{pgfscope}%
\begin{pgfscope}%
\pgfsetbuttcap%
\pgfsetroundjoin%
\definecolor{currentfill}{rgb}{0.000000,0.000000,0.000000}%
\pgfsetfillcolor{currentfill}%
\pgfsetlinewidth{0.803000pt}%
\definecolor{currentstroke}{rgb}{0.000000,0.000000,0.000000}%
\pgfsetstrokecolor{currentstroke}%
\pgfsetdash{}{0pt}%
\pgfsys@defobject{currentmarker}{\pgfqpoint{0.000000in}{-0.048611in}}{\pgfqpoint{0.000000in}{0.000000in}}{%
\pgfpathmoveto{\pgfqpoint{0.000000in}{0.000000in}}%
\pgfpathlineto{\pgfqpoint{0.000000in}{-0.048611in}}%
\pgfusepath{stroke,fill}%
}%
\begin{pgfscope}%
\pgfsys@transformshift{0.916880in}{0.523570in}%
\pgfsys@useobject{currentmarker}{}%
\end{pgfscope}%
\end{pgfscope}%
\begin{pgfscope}%
\pgftext[x=0.916880in,y=0.426348in,,top]{\rmfamily\fontsize{10.000000}{12.000000}\selectfont \(\displaystyle 0\)}%
\end{pgfscope}%
\begin{pgfscope}%
\pgfsetbuttcap%
\pgfsetroundjoin%
\definecolor{currentfill}{rgb}{0.000000,0.000000,0.000000}%
\pgfsetfillcolor{currentfill}%
\pgfsetlinewidth{0.803000pt}%
\definecolor{currentstroke}{rgb}{0.000000,0.000000,0.000000}%
\pgfsetstrokecolor{currentstroke}%
\pgfsetdash{}{0pt}%
\pgfsys@defobject{currentmarker}{\pgfqpoint{0.000000in}{-0.048611in}}{\pgfqpoint{0.000000in}{0.000000in}}{%
\pgfpathmoveto{\pgfqpoint{0.000000in}{0.000000in}}%
\pgfpathlineto{\pgfqpoint{0.000000in}{-0.048611in}}%
\pgfusepath{stroke,fill}%
}%
\begin{pgfscope}%
\pgfsys@transformshift{1.634412in}{0.523570in}%
\pgfsys@useobject{currentmarker}{}%
\end{pgfscope}%
\end{pgfscope}%
\begin{pgfscope}%
\pgftext[x=1.634412in,y=0.426348in,,top]{\rmfamily\fontsize{10.000000}{12.000000}\selectfont \(\displaystyle 1000\)}%
\end{pgfscope}%
\begin{pgfscope}%
\pgfsetbuttcap%
\pgfsetroundjoin%
\definecolor{currentfill}{rgb}{0.000000,0.000000,0.000000}%
\pgfsetfillcolor{currentfill}%
\pgfsetlinewidth{0.803000pt}%
\definecolor{currentstroke}{rgb}{0.000000,0.000000,0.000000}%
\pgfsetstrokecolor{currentstroke}%
\pgfsetdash{}{0pt}%
\pgfsys@defobject{currentmarker}{\pgfqpoint{0.000000in}{-0.048611in}}{\pgfqpoint{0.000000in}{0.000000in}}{%
\pgfpathmoveto{\pgfqpoint{0.000000in}{0.000000in}}%
\pgfpathlineto{\pgfqpoint{0.000000in}{-0.048611in}}%
\pgfusepath{stroke,fill}%
}%
\begin{pgfscope}%
\pgfsys@transformshift{2.351944in}{0.523570in}%
\pgfsys@useobject{currentmarker}{}%
\end{pgfscope}%
\end{pgfscope}%
\begin{pgfscope}%
\pgftext[x=2.351944in,y=0.426348in,,top]{\rmfamily\fontsize{10.000000}{12.000000}\selectfont \(\displaystyle 2000\)}%
\end{pgfscope}%
\begin{pgfscope}%
\pgfsetbuttcap%
\pgfsetroundjoin%
\definecolor{currentfill}{rgb}{0.000000,0.000000,0.000000}%
\pgfsetfillcolor{currentfill}%
\pgfsetlinewidth{0.803000pt}%
\definecolor{currentstroke}{rgb}{0.000000,0.000000,0.000000}%
\pgfsetstrokecolor{currentstroke}%
\pgfsetdash{}{0pt}%
\pgfsys@defobject{currentmarker}{\pgfqpoint{0.000000in}{-0.048611in}}{\pgfqpoint{0.000000in}{0.000000in}}{%
\pgfpathmoveto{\pgfqpoint{0.000000in}{0.000000in}}%
\pgfpathlineto{\pgfqpoint{0.000000in}{-0.048611in}}%
\pgfusepath{stroke,fill}%
}%
\begin{pgfscope}%
\pgfsys@transformshift{3.069476in}{0.523570in}%
\pgfsys@useobject{currentmarker}{}%
\end{pgfscope}%
\end{pgfscope}%
\begin{pgfscope}%
\pgftext[x=3.069476in,y=0.426348in,,top]{\rmfamily\fontsize{10.000000}{12.000000}\selectfont \(\displaystyle 3000\)}%
\end{pgfscope}%
\begin{pgfscope}%
\pgfsetbuttcap%
\pgfsetroundjoin%
\definecolor{currentfill}{rgb}{0.000000,0.000000,0.000000}%
\pgfsetfillcolor{currentfill}%
\pgfsetlinewidth{0.803000pt}%
\definecolor{currentstroke}{rgb}{0.000000,0.000000,0.000000}%
\pgfsetstrokecolor{currentstroke}%
\pgfsetdash{}{0pt}%
\pgfsys@defobject{currentmarker}{\pgfqpoint{0.000000in}{-0.048611in}}{\pgfqpoint{0.000000in}{0.000000in}}{%
\pgfpathmoveto{\pgfqpoint{0.000000in}{0.000000in}}%
\pgfpathlineto{\pgfqpoint{0.000000in}{-0.048611in}}%
\pgfusepath{stroke,fill}%
}%
\begin{pgfscope}%
\pgfsys@transformshift{3.787009in}{0.523570in}%
\pgfsys@useobject{currentmarker}{}%
\end{pgfscope}%
\end{pgfscope}%
\begin{pgfscope}%
\pgftext[x=3.787009in,y=0.426348in,,top]{\rmfamily\fontsize{10.000000}{12.000000}\selectfont \(\displaystyle 4000\)}%
\end{pgfscope}%
\begin{pgfscope}%
\pgfsetbuttcap%
\pgfsetroundjoin%
\definecolor{currentfill}{rgb}{0.000000,0.000000,0.000000}%
\pgfsetfillcolor{currentfill}%
\pgfsetlinewidth{0.803000pt}%
\definecolor{currentstroke}{rgb}{0.000000,0.000000,0.000000}%
\pgfsetstrokecolor{currentstroke}%
\pgfsetdash{}{0pt}%
\pgfsys@defobject{currentmarker}{\pgfqpoint{0.000000in}{-0.048611in}}{\pgfqpoint{0.000000in}{0.000000in}}{%
\pgfpathmoveto{\pgfqpoint{0.000000in}{0.000000in}}%
\pgfpathlineto{\pgfqpoint{0.000000in}{-0.048611in}}%
\pgfusepath{stroke,fill}%
}%
\begin{pgfscope}%
\pgfsys@transformshift{4.504541in}{0.523570in}%
\pgfsys@useobject{currentmarker}{}%
\end{pgfscope}%
\end{pgfscope}%
\begin{pgfscope}%
\pgftext[x=4.504541in,y=0.426348in,,top]{\rmfamily\fontsize{10.000000}{12.000000}\selectfont \(\displaystyle 5000\)}%
\end{pgfscope}%
\begin{pgfscope}%
\pgfsetbuttcap%
\pgfsetroundjoin%
\definecolor{currentfill}{rgb}{0.000000,0.000000,0.000000}%
\pgfsetfillcolor{currentfill}%
\pgfsetlinewidth{0.803000pt}%
\definecolor{currentstroke}{rgb}{0.000000,0.000000,0.000000}%
\pgfsetstrokecolor{currentstroke}%
\pgfsetdash{}{0pt}%
\pgfsys@defobject{currentmarker}{\pgfqpoint{0.000000in}{-0.048611in}}{\pgfqpoint{0.000000in}{0.000000in}}{%
\pgfpathmoveto{\pgfqpoint{0.000000in}{0.000000in}}%
\pgfpathlineto{\pgfqpoint{0.000000in}{-0.048611in}}%
\pgfusepath{stroke,fill}%
}%
\begin{pgfscope}%
\pgfsys@transformshift{5.222073in}{0.523570in}%
\pgfsys@useobject{currentmarker}{}%
\end{pgfscope}%
\end{pgfscope}%
\begin{pgfscope}%
\pgftext[x=5.222073in,y=0.426348in,,top]{\rmfamily\fontsize{10.000000}{12.000000}\selectfont \(\displaystyle 6000\)}%
\end{pgfscope}%
\begin{pgfscope}%
\pgftext[x=3.030516in,y=0.236379in,,top]{\rmfamily\fontsize{10.000000}{12.000000}\selectfont Frequency (Hz)}%
\end{pgfscope}%
\begin{pgfscope}%
\pgfsetbuttcap%
\pgfsetroundjoin%
\definecolor{currentfill}{rgb}{0.000000,0.000000,0.000000}%
\pgfsetfillcolor{currentfill}%
\pgfsetlinewidth{0.803000pt}%
\definecolor{currentstroke}{rgb}{0.000000,0.000000,0.000000}%
\pgfsetstrokecolor{currentstroke}%
\pgfsetdash{}{0pt}%
\pgfsys@defobject{currentmarker}{\pgfqpoint{-0.048611in}{0.000000in}}{\pgfqpoint{0.000000in}{0.000000in}}{%
\pgfpathmoveto{\pgfqpoint{0.000000in}{0.000000in}}%
\pgfpathlineto{\pgfqpoint{-0.048611in}{0.000000in}}%
\pgfusepath{stroke,fill}%
}%
\begin{pgfscope}%
\pgfsys@transformshift{0.705516in}{0.908792in}%
\pgfsys@useobject{currentmarker}{}%
\end{pgfscope}%
\end{pgfscope}%
\begin{pgfscope}%
\pgftext[x=0.291935in,y=0.856030in,left,base]{\rmfamily\fontsize{10.000000}{12.000000}\selectfont \(\displaystyle -120\)}%
\end{pgfscope}%
\begin{pgfscope}%
\pgfsetbuttcap%
\pgfsetroundjoin%
\definecolor{currentfill}{rgb}{0.000000,0.000000,0.000000}%
\pgfsetfillcolor{currentfill}%
\pgfsetlinewidth{0.803000pt}%
\definecolor{currentstroke}{rgb}{0.000000,0.000000,0.000000}%
\pgfsetstrokecolor{currentstroke}%
\pgfsetdash{}{0pt}%
\pgfsys@defobject{currentmarker}{\pgfqpoint{-0.048611in}{0.000000in}}{\pgfqpoint{0.000000in}{0.000000in}}{%
\pgfpathmoveto{\pgfqpoint{0.000000in}{0.000000in}}%
\pgfpathlineto{\pgfqpoint{-0.048611in}{0.000000in}}%
\pgfusepath{stroke,fill}%
}%
\begin{pgfscope}%
\pgfsys@transformshift{0.705516in}{1.325043in}%
\pgfsys@useobject{currentmarker}{}%
\end{pgfscope}%
\end{pgfscope}%
\begin{pgfscope}%
\pgftext[x=0.291935in,y=1.272281in,left,base]{\rmfamily\fontsize{10.000000}{12.000000}\selectfont \(\displaystyle -100\)}%
\end{pgfscope}%
\begin{pgfscope}%
\pgfsetbuttcap%
\pgfsetroundjoin%
\definecolor{currentfill}{rgb}{0.000000,0.000000,0.000000}%
\pgfsetfillcolor{currentfill}%
\pgfsetlinewidth{0.803000pt}%
\definecolor{currentstroke}{rgb}{0.000000,0.000000,0.000000}%
\pgfsetstrokecolor{currentstroke}%
\pgfsetdash{}{0pt}%
\pgfsys@defobject{currentmarker}{\pgfqpoint{-0.048611in}{0.000000in}}{\pgfqpoint{0.000000in}{0.000000in}}{%
\pgfpathmoveto{\pgfqpoint{0.000000in}{0.000000in}}%
\pgfpathlineto{\pgfqpoint{-0.048611in}{0.000000in}}%
\pgfusepath{stroke,fill}%
}%
\begin{pgfscope}%
\pgfsys@transformshift{0.705516in}{1.741294in}%
\pgfsys@useobject{currentmarker}{}%
\end{pgfscope}%
\end{pgfscope}%
\begin{pgfscope}%
\pgftext[x=0.361380in,y=1.688532in,left,base]{\rmfamily\fontsize{10.000000}{12.000000}\selectfont \(\displaystyle -80\)}%
\end{pgfscope}%
\begin{pgfscope}%
\pgfsetbuttcap%
\pgfsetroundjoin%
\definecolor{currentfill}{rgb}{0.000000,0.000000,0.000000}%
\pgfsetfillcolor{currentfill}%
\pgfsetlinewidth{0.803000pt}%
\definecolor{currentstroke}{rgb}{0.000000,0.000000,0.000000}%
\pgfsetstrokecolor{currentstroke}%
\pgfsetdash{}{0pt}%
\pgfsys@defobject{currentmarker}{\pgfqpoint{-0.048611in}{0.000000in}}{\pgfqpoint{0.000000in}{0.000000in}}{%
\pgfpathmoveto{\pgfqpoint{0.000000in}{0.000000in}}%
\pgfpathlineto{\pgfqpoint{-0.048611in}{0.000000in}}%
\pgfusepath{stroke,fill}%
}%
\begin{pgfscope}%
\pgfsys@transformshift{0.705516in}{2.157545in}%
\pgfsys@useobject{currentmarker}{}%
\end{pgfscope}%
\end{pgfscope}%
\begin{pgfscope}%
\pgftext[x=0.361380in,y=2.104783in,left,base]{\rmfamily\fontsize{10.000000}{12.000000}\selectfont \(\displaystyle -60\)}%
\end{pgfscope}%
\begin{pgfscope}%
\pgfsetbuttcap%
\pgfsetroundjoin%
\definecolor{currentfill}{rgb}{0.000000,0.000000,0.000000}%
\pgfsetfillcolor{currentfill}%
\pgfsetlinewidth{0.803000pt}%
\definecolor{currentstroke}{rgb}{0.000000,0.000000,0.000000}%
\pgfsetstrokecolor{currentstroke}%
\pgfsetdash{}{0pt}%
\pgfsys@defobject{currentmarker}{\pgfqpoint{-0.048611in}{0.000000in}}{\pgfqpoint{0.000000in}{0.000000in}}{%
\pgfpathmoveto{\pgfqpoint{0.000000in}{0.000000in}}%
\pgfpathlineto{\pgfqpoint{-0.048611in}{0.000000in}}%
\pgfusepath{stroke,fill}%
}%
\begin{pgfscope}%
\pgfsys@transformshift{0.705516in}{2.573795in}%
\pgfsys@useobject{currentmarker}{}%
\end{pgfscope}%
\end{pgfscope}%
\begin{pgfscope}%
\pgftext[x=0.361380in,y=2.521034in,left,base]{\rmfamily\fontsize{10.000000}{12.000000}\selectfont \(\displaystyle -40\)}%
\end{pgfscope}%
\begin{pgfscope}%
\pgfsetbuttcap%
\pgfsetroundjoin%
\definecolor{currentfill}{rgb}{0.000000,0.000000,0.000000}%
\pgfsetfillcolor{currentfill}%
\pgfsetlinewidth{0.803000pt}%
\definecolor{currentstroke}{rgb}{0.000000,0.000000,0.000000}%
\pgfsetstrokecolor{currentstroke}%
\pgfsetdash{}{0pt}%
\pgfsys@defobject{currentmarker}{\pgfqpoint{-0.048611in}{0.000000in}}{\pgfqpoint{0.000000in}{0.000000in}}{%
\pgfpathmoveto{\pgfqpoint{0.000000in}{0.000000in}}%
\pgfpathlineto{\pgfqpoint{-0.048611in}{0.000000in}}%
\pgfusepath{stroke,fill}%
}%
\begin{pgfscope}%
\pgfsys@transformshift{0.705516in}{2.990046in}%
\pgfsys@useobject{currentmarker}{}%
\end{pgfscope}%
\end{pgfscope}%
\begin{pgfscope}%
\pgftext[x=0.361380in,y=2.937285in,left,base]{\rmfamily\fontsize{10.000000}{12.000000}\selectfont \(\displaystyle -20\)}%
\end{pgfscope}%
\begin{pgfscope}%
\pgfsetbuttcap%
\pgfsetroundjoin%
\definecolor{currentfill}{rgb}{0.000000,0.000000,0.000000}%
\pgfsetfillcolor{currentfill}%
\pgfsetlinewidth{0.803000pt}%
\definecolor{currentstroke}{rgb}{0.000000,0.000000,0.000000}%
\pgfsetstrokecolor{currentstroke}%
\pgfsetdash{}{0pt}%
\pgfsys@defobject{currentmarker}{\pgfqpoint{-0.048611in}{0.000000in}}{\pgfqpoint{0.000000in}{0.000000in}}{%
\pgfpathmoveto{\pgfqpoint{0.000000in}{0.000000in}}%
\pgfpathlineto{\pgfqpoint{-0.048611in}{0.000000in}}%
\pgfusepath{stroke,fill}%
}%
\begin{pgfscope}%
\pgfsys@transformshift{0.705516in}{3.406297in}%
\pgfsys@useobject{currentmarker}{}%
\end{pgfscope}%
\end{pgfscope}%
\begin{pgfscope}%
\pgftext[x=0.538849in,y=3.353536in,left,base]{\rmfamily\fontsize{10.000000}{12.000000}\selectfont \(\displaystyle 0\)}%
\end{pgfscope}%
\begin{pgfscope}%
\pgftext[x=0.236379in,y=2.033570in,,bottom,rotate=90.000000]{\rmfamily\fontsize{10.000000}{12.000000}\selectfont Magnitude (dB)}%
\end{pgfscope}%
\begin{pgfscope}%
\pgfpathrectangle{\pgfqpoint{0.705516in}{0.523570in}}{\pgfqpoint{4.650000in}{3.020000in}} %
\pgfusepath{clip}%
\pgfsetrectcap%
\pgfsetroundjoin%
\pgfsetlinewidth{1.505625pt}%
\definecolor{currentstroke}{rgb}{0.000000,0.000000,0.000000}%
\pgfsetstrokecolor{currentstroke}%
\pgfsetstrokeopacity{0.700000}%
\pgfsetdash{}{0pt}%
\pgfpathmoveto{\pgfqpoint{0.916880in}{3.406297in}}%
\pgfpathlineto{\pgfqpoint{0.988529in}{2.668798in}}%
\pgfpathlineto{\pgfqpoint{1.060177in}{2.228790in}}%
\pgfpathlineto{\pgfqpoint{1.131826in}{1.966121in}}%
\pgfpathlineto{\pgfqpoint{1.203475in}{1.784756in}}%
\pgfpathlineto{\pgfqpoint{1.275123in}{1.649075in}}%
\pgfpathlineto{\pgfqpoint{1.346772in}{1.542565in}}%
\pgfpathlineto{\pgfqpoint{1.418421in}{1.454360in}}%
\pgfpathlineto{\pgfqpoint{1.490069in}{1.375845in}}%
\pgfpathlineto{\pgfqpoint{1.561718in}{1.283599in}}%
\pgfpathlineto{\pgfqpoint{1.633367in}{1.313644in}}%
\pgfpathlineto{\pgfqpoint{1.705015in}{1.254328in}}%
\pgfpathlineto{\pgfqpoint{1.776664in}{1.230409in}}%
\pgfpathlineto{\pgfqpoint{1.848313in}{1.191472in}}%
\pgfpathlineto{\pgfqpoint{1.919962in}{1.156676in}}%
\pgfpathlineto{\pgfqpoint{1.991610in}{1.129872in}}%
\pgfpathlineto{\pgfqpoint{2.063259in}{1.108026in}}%
\pgfpathlineto{\pgfqpoint{2.134908in}{1.088093in}}%
\pgfpathlineto{\pgfqpoint{2.206556in}{1.082532in}}%
\pgfpathlineto{\pgfqpoint{2.278205in}{1.097681in}}%
\pgfpathlineto{\pgfqpoint{2.349854in}{1.272536in}}%
\pgfpathlineto{\pgfqpoint{2.421502in}{1.086494in}}%
\pgfpathlineto{\pgfqpoint{2.493151in}{1.004172in}}%
\pgfpathlineto{\pgfqpoint{2.564800in}{0.965381in}}%
\pgfpathlineto{\pgfqpoint{2.636448in}{0.939640in}}%
\pgfpathlineto{\pgfqpoint{2.708097in}{0.916632in}}%
\pgfpathlineto{\pgfqpoint{2.779746in}{0.901896in}}%
\pgfpathlineto{\pgfqpoint{2.851395in}{0.882169in}}%
\pgfpathlineto{\pgfqpoint{2.923043in}{0.863225in}}%
\pgfpathlineto{\pgfqpoint{2.994692in}{0.829628in}}%
\pgfpathlineto{\pgfqpoint{3.066341in}{0.736568in}}%
\pgfpathlineto{\pgfqpoint{3.137989in}{0.876987in}}%
\pgfpathlineto{\pgfqpoint{3.209638in}{0.847826in}}%
\pgfpathlineto{\pgfqpoint{3.281287in}{0.845238in}}%
\pgfpathlineto{\pgfqpoint{3.352935in}{0.769978in}}%
\pgfpathlineto{\pgfqpoint{3.424584in}{0.780304in}}%
\pgfpathlineto{\pgfqpoint{3.496233in}{0.784584in}}%
\pgfpathlineto{\pgfqpoint{3.567881in}{0.791042in}}%
\pgfpathlineto{\pgfqpoint{3.639530in}{0.763833in}}%
\pgfpathlineto{\pgfqpoint{3.711179in}{0.746188in}}%
\pgfpathlineto{\pgfqpoint{3.782827in}{0.757916in}}%
\pgfpathlineto{\pgfqpoint{3.854476in}{0.757447in}}%
\pgfpathlineto{\pgfqpoint{3.926125in}{0.775771in}}%
\pgfpathlineto{\pgfqpoint{3.997774in}{0.795559in}}%
\pgfpathlineto{\pgfqpoint{4.069422in}{0.833367in}}%
\pgfpathlineto{\pgfqpoint{4.141071in}{0.758128in}}%
\pgfpathlineto{\pgfqpoint{4.212720in}{0.762727in}}%
\pgfpathlineto{\pgfqpoint{4.284368in}{0.760962in}}%
\pgfpathlineto{\pgfqpoint{4.356017in}{0.745143in}}%
\pgfpathlineto{\pgfqpoint{4.427666in}{0.758238in}}%
\pgfpathlineto{\pgfqpoint{4.499314in}{0.759372in}}%
\pgfpathlineto{\pgfqpoint{4.570963in}{0.764988in}}%
\pgfpathlineto{\pgfqpoint{4.642612in}{0.785393in}}%
\pgfpathlineto{\pgfqpoint{4.714260in}{0.754508in}}%
\pgfpathlineto{\pgfqpoint{4.785909in}{0.740646in}}%
\pgfpathlineto{\pgfqpoint{4.857558in}{0.744052in}}%
\pgfpathlineto{\pgfqpoint{4.929207in}{0.728294in}}%
\pgfpathlineto{\pgfqpoint{5.000855in}{0.711100in}}%
\pgfpathlineto{\pgfqpoint{5.072504in}{0.693256in}}%
\pgfpathlineto{\pgfqpoint{5.144153in}{0.714114in}}%
\pgfusepath{stroke}%
\end{pgfscope}%
\begin{pgfscope}%
\pgfpathrectangle{\pgfqpoint{0.705516in}{0.523570in}}{\pgfqpoint{4.650000in}{3.020000in}} %
\pgfusepath{clip}%
\pgfsetrectcap%
\pgfsetroundjoin%
\pgfsetlinewidth{1.505625pt}%
\definecolor{currentstroke}{rgb}{0.750000,0.000000,0.750000}%
\pgfsetstrokecolor{currentstroke}%
\pgfsetstrokeopacity{0.700000}%
\pgfsetdash{}{0pt}%
\pgfpathmoveto{\pgfqpoint{0.916880in}{3.406297in}}%
\pgfpathlineto{\pgfqpoint{0.988529in}{2.631281in}}%
\pgfpathlineto{\pgfqpoint{1.060177in}{2.087149in}}%
\pgfpathlineto{\pgfqpoint{1.131826in}{1.709057in}}%
\pgfpathlineto{\pgfqpoint{1.203475in}{1.507406in}}%
\pgfpathlineto{\pgfqpoint{1.275123in}{1.409081in}}%
\pgfpathlineto{\pgfqpoint{1.346772in}{1.337310in}}%
\pgfpathlineto{\pgfqpoint{1.418421in}{1.279499in}}%
\pgfpathlineto{\pgfqpoint{1.490069in}{1.230979in}}%
\pgfpathlineto{\pgfqpoint{1.561718in}{1.187992in}}%
\pgfpathlineto{\pgfqpoint{1.633367in}{1.150673in}}%
\pgfpathlineto{\pgfqpoint{1.705015in}{1.115178in}}%
\pgfpathlineto{\pgfqpoint{1.776664in}{1.083447in}}%
\pgfpathlineto{\pgfqpoint{1.848313in}{1.055686in}}%
\pgfpathlineto{\pgfqpoint{1.919962in}{1.030274in}}%
\pgfpathlineto{\pgfqpoint{1.991610in}{1.006603in}}%
\pgfpathlineto{\pgfqpoint{2.063259in}{0.984307in}}%
\pgfpathlineto{\pgfqpoint{2.134908in}{0.963775in}}%
\pgfpathlineto{\pgfqpoint{2.206556in}{0.945069in}}%
\pgfpathlineto{\pgfqpoint{2.278205in}{0.928634in}}%
\pgfpathlineto{\pgfqpoint{2.349854in}{0.903630in}}%
\pgfpathlineto{\pgfqpoint{2.421502in}{0.895061in}}%
\pgfpathlineto{\pgfqpoint{2.493151in}{0.878691in}}%
\pgfpathlineto{\pgfqpoint{2.564800in}{0.864213in}}%
\pgfpathlineto{\pgfqpoint{2.636448in}{0.850827in}}%
\pgfpathlineto{\pgfqpoint{2.708097in}{0.838175in}}%
\pgfpathlineto{\pgfqpoint{2.779746in}{0.826095in}}%
\pgfpathlineto{\pgfqpoint{2.851395in}{0.814492in}}%
\pgfpathlineto{\pgfqpoint{2.923043in}{0.803623in}}%
\pgfpathlineto{\pgfqpoint{2.994692in}{0.793641in}}%
\pgfpathlineto{\pgfqpoint{3.066341in}{0.784348in}}%
\pgfpathlineto{\pgfqpoint{3.137989in}{0.774817in}}%
\pgfpathlineto{\pgfqpoint{3.209638in}{0.766220in}}%
\pgfpathlineto{\pgfqpoint{3.281287in}{0.758010in}}%
\pgfpathlineto{\pgfqpoint{3.352935in}{0.750098in}}%
\pgfpathlineto{\pgfqpoint{3.424584in}{0.742680in}}%
\pgfpathlineto{\pgfqpoint{3.496233in}{0.735512in}}%
\pgfpathlineto{\pgfqpoint{3.567881in}{0.728879in}}%
\pgfpathlineto{\pgfqpoint{3.639530in}{0.722662in}}%
\pgfpathlineto{\pgfqpoint{3.711179in}{0.716830in}}%
\pgfpathlineto{\pgfqpoint{3.782827in}{0.711459in}}%
\pgfpathlineto{\pgfqpoint{3.854476in}{0.706093in}}%
\pgfpathlineto{\pgfqpoint{3.926125in}{0.700673in}}%
\pgfpathlineto{\pgfqpoint{3.997774in}{0.695985in}}%
\pgfpathlineto{\pgfqpoint{4.069422in}{0.692351in}}%
\pgfpathlineto{\pgfqpoint{4.141071in}{0.687918in}}%
\pgfpathlineto{\pgfqpoint{4.212720in}{0.684348in}}%
\pgfpathlineto{\pgfqpoint{4.284368in}{0.680921in}}%
\pgfpathlineto{\pgfqpoint{4.356017in}{0.677755in}}%
\pgfpathlineto{\pgfqpoint{4.427666in}{0.674800in}}%
\pgfpathlineto{\pgfqpoint{4.499314in}{0.672294in}}%
\pgfpathlineto{\pgfqpoint{4.570963in}{0.670082in}}%
\pgfpathlineto{\pgfqpoint{4.642612in}{0.667967in}}%
\pgfpathlineto{\pgfqpoint{4.714260in}{0.666236in}}%
\pgfpathlineto{\pgfqpoint{4.785909in}{0.664899in}}%
\pgfpathlineto{\pgfqpoint{4.857558in}{0.663688in}}%
\pgfpathlineto{\pgfqpoint{4.929207in}{0.662553in}}%
\pgfpathlineto{\pgfqpoint{5.000855in}{0.661585in}}%
\pgfpathlineto{\pgfqpoint{5.072504in}{0.661080in}}%
\pgfpathlineto{\pgfqpoint{5.144153in}{0.660843in}}%
\pgfusepath{stroke}%
\end{pgfscope}%
\begin{pgfscope}%
\pgfpathrectangle{\pgfqpoint{0.705516in}{0.523570in}}{\pgfqpoint{4.650000in}{3.020000in}} %
\pgfusepath{clip}%
\pgfsetrectcap%
\pgfsetroundjoin%
\pgfsetlinewidth{1.505625pt}%
\definecolor{currentstroke}{rgb}{0.000000,0.000000,1.000000}%
\pgfsetstrokecolor{currentstroke}%
\pgfsetstrokeopacity{0.700000}%
\pgfsetdash{}{0pt}%
\pgfpathmoveto{\pgfqpoint{0.916880in}{3.406297in}}%
\pgfpathlineto{\pgfqpoint{0.988529in}{2.669290in}}%
\pgfpathlineto{\pgfqpoint{1.060177in}{2.230600in}}%
\pgfpathlineto{\pgfqpoint{1.131826in}{1.970562in}}%
\pgfpathlineto{\pgfqpoint{1.203475in}{1.791137in}}%
\pgfpathlineto{\pgfqpoint{1.275123in}{1.647572in}}%
\pgfpathlineto{\pgfqpoint{1.346772in}{1.522234in}}%
\pgfpathlineto{\pgfqpoint{1.418421in}{1.422726in}}%
\pgfpathlineto{\pgfqpoint{1.490069in}{1.346855in}}%
\pgfpathlineto{\pgfqpoint{1.561718in}{1.286286in}}%
\pgfpathlineto{\pgfqpoint{1.633367in}{1.237044in}}%
\pgfpathlineto{\pgfqpoint{1.705015in}{1.193754in}}%
\pgfpathlineto{\pgfqpoint{1.776664in}{1.155940in}}%
\pgfpathlineto{\pgfqpoint{1.848313in}{1.122285in}}%
\pgfpathlineto{\pgfqpoint{1.919962in}{1.092028in}}%
\pgfpathlineto{\pgfqpoint{1.991610in}{1.064573in}}%
\pgfpathlineto{\pgfqpoint{2.063259in}{1.039473in}}%
\pgfpathlineto{\pgfqpoint{2.134908in}{1.016387in}}%
\pgfpathlineto{\pgfqpoint{2.206556in}{0.995047in}}%
\pgfpathlineto{\pgfqpoint{2.278205in}{0.975237in}}%
\pgfpathlineto{\pgfqpoint{2.349854in}{0.956770in}}%
\pgfpathlineto{\pgfqpoint{2.421502in}{0.939537in}}%
\pgfpathlineto{\pgfqpoint{2.493151in}{0.923380in}}%
\pgfpathlineto{\pgfqpoint{2.564800in}{0.908210in}}%
\pgfpathlineto{\pgfqpoint{2.636448in}{0.893941in}}%
\pgfpathlineto{\pgfqpoint{2.708097in}{0.880497in}}%
\pgfpathlineto{\pgfqpoint{2.779746in}{0.867815in}}%
\pgfpathlineto{\pgfqpoint{2.851395in}{0.855839in}}%
\pgfpathlineto{\pgfqpoint{2.923043in}{0.844520in}}%
\pgfpathlineto{\pgfqpoint{2.994692in}{0.833816in}}%
\pgfpathlineto{\pgfqpoint{3.066341in}{0.823687in}}%
\pgfpathlineto{\pgfqpoint{3.137989in}{0.814100in}}%
\pgfpathlineto{\pgfqpoint{3.209638in}{0.805027in}}%
\pgfpathlineto{\pgfqpoint{3.281287in}{0.796439in}}%
\pgfpathlineto{\pgfqpoint{3.352935in}{0.788312in}}%
\pgfpathlineto{\pgfqpoint{3.424584in}{0.780626in}}%
\pgfpathlineto{\pgfqpoint{3.496233in}{0.773361in}}%
\pgfpathlineto{\pgfqpoint{3.567881in}{0.766498in}}%
\pgfpathlineto{\pgfqpoint{3.639530in}{0.760023in}}%
\pgfpathlineto{\pgfqpoint{3.711179in}{0.753921in}}%
\pgfpathlineto{\pgfqpoint{3.782827in}{0.748178in}}%
\pgfpathlineto{\pgfqpoint{3.854476in}{0.742784in}}%
\pgfpathlineto{\pgfqpoint{3.926125in}{0.737726in}}%
\pgfpathlineto{\pgfqpoint{3.997774in}{0.732996in}}%
\pgfpathlineto{\pgfqpoint{4.069422in}{0.728585in}}%
\pgfpathlineto{\pgfqpoint{4.141071in}{0.724484in}}%
\pgfpathlineto{\pgfqpoint{4.212720in}{0.720685in}}%
\pgfpathlineto{\pgfqpoint{4.284368in}{0.717184in}}%
\pgfpathlineto{\pgfqpoint{4.356017in}{0.713973in}}%
\pgfpathlineto{\pgfqpoint{4.427666in}{0.711047in}}%
\pgfpathlineto{\pgfqpoint{4.499314in}{0.708401in}}%
\pgfpathlineto{\pgfqpoint{4.570963in}{0.706032in}}%
\pgfpathlineto{\pgfqpoint{4.642612in}{0.703935in}}%
\pgfpathlineto{\pgfqpoint{4.714260in}{0.702107in}}%
\pgfpathlineto{\pgfqpoint{4.785909in}{0.700546in}}%
\pgfpathlineto{\pgfqpoint{4.857558in}{0.699248in}}%
\pgfpathlineto{\pgfqpoint{4.929207in}{0.698212in}}%
\pgfpathlineto{\pgfqpoint{5.000855in}{0.697436in}}%
\pgfpathlineto{\pgfqpoint{5.072504in}{0.696920in}}%
\pgfpathlineto{\pgfqpoint{5.144153in}{0.696662in}}%
\pgfusepath{stroke}%
\end{pgfscope}%
\begin{pgfscope}%
\pgfpathrectangle{\pgfqpoint{0.705516in}{0.523570in}}{\pgfqpoint{4.650000in}{3.020000in}} %
\pgfusepath{clip}%
\pgfsetrectcap%
\pgfsetroundjoin%
\pgfsetlinewidth{1.505625pt}%
\definecolor{currentstroke}{rgb}{0.000000,0.750000,0.750000}%
\pgfsetstrokecolor{currentstroke}%
\pgfsetstrokeopacity{0.700000}%
\pgfsetdash{}{0pt}%
\pgfpathmoveto{\pgfqpoint{0.916880in}{3.406297in}}%
\pgfpathlineto{\pgfqpoint{0.988529in}{2.738258in}}%
\pgfpathlineto{\pgfqpoint{1.060177in}{2.368762in}}%
\pgfpathlineto{\pgfqpoint{1.131826in}{2.142808in}}%
\pgfpathlineto{\pgfqpoint{1.203475in}{1.939810in}}%
\pgfpathlineto{\pgfqpoint{1.275123in}{1.720237in}}%
\pgfpathlineto{\pgfqpoint{1.346772in}{1.498168in}}%
\pgfpathlineto{\pgfqpoint{1.418421in}{1.366898in}}%
\pgfpathlineto{\pgfqpoint{1.490069in}{1.327581in}}%
\pgfpathlineto{\pgfqpoint{1.561718in}{1.313894in}}%
\pgfpathlineto{\pgfqpoint{1.633367in}{1.337306in}}%
\pgfpathlineto{\pgfqpoint{1.705015in}{1.345551in}}%
\pgfpathlineto{\pgfqpoint{1.776664in}{1.317370in}}%
\pgfpathlineto{\pgfqpoint{1.848313in}{1.238848in}}%
\pgfpathlineto{\pgfqpoint{1.919962in}{1.119125in}}%
\pgfpathlineto{\pgfqpoint{1.991610in}{1.006635in}}%
\pgfpathlineto{\pgfqpoint{2.063259in}{0.972229in}}%
\pgfpathlineto{\pgfqpoint{2.134908in}{0.988688in}}%
\pgfpathlineto{\pgfqpoint{2.206556in}{1.012790in}}%
\pgfpathlineto{\pgfqpoint{2.278205in}{1.029647in}}%
\pgfpathlineto{\pgfqpoint{2.349854in}{1.009553in}}%
\pgfpathlineto{\pgfqpoint{2.421502in}{1.002992in}}%
\pgfpathlineto{\pgfqpoint{2.493151in}{0.958155in}}%
\pgfpathlineto{\pgfqpoint{2.564800in}{0.913618in}}%
\pgfpathlineto{\pgfqpoint{2.636448in}{0.882517in}}%
\pgfpathlineto{\pgfqpoint{2.708097in}{0.861802in}}%
\pgfpathlineto{\pgfqpoint{2.779746in}{0.849567in}}%
\pgfpathlineto{\pgfqpoint{2.851395in}{0.848267in}}%
\pgfpathlineto{\pgfqpoint{2.923043in}{0.848973in}}%
\pgfpathlineto{\pgfqpoint{2.994692in}{0.839250in}}%
\pgfpathlineto{\pgfqpoint{3.066341in}{0.822268in}}%
\pgfpathlineto{\pgfqpoint{3.137989in}{0.800486in}}%
\pgfpathlineto{\pgfqpoint{3.209638in}{0.800974in}}%
\pgfpathlineto{\pgfqpoint{3.281287in}{0.805518in}}%
\pgfpathlineto{\pgfqpoint{3.352935in}{0.805495in}}%
\pgfpathlineto{\pgfqpoint{3.424584in}{0.805810in}}%
\pgfpathlineto{\pgfqpoint{3.496233in}{0.801884in}}%
\pgfpathlineto{\pgfqpoint{3.567881in}{0.785392in}}%
\pgfpathlineto{\pgfqpoint{3.639530in}{0.757062in}}%
\pgfpathlineto{\pgfqpoint{3.711179in}{0.730151in}}%
\pgfpathlineto{\pgfqpoint{3.782827in}{0.718249in}}%
\pgfpathlineto{\pgfqpoint{3.854476in}{0.718947in}}%
\pgfpathlineto{\pgfqpoint{3.926125in}{0.721998in}}%
\pgfpathlineto{\pgfqpoint{3.997774in}{0.727167in}}%
\pgfpathlineto{\pgfqpoint{4.069422in}{0.733702in}}%
\pgfpathlineto{\pgfqpoint{4.141071in}{0.731916in}}%
\pgfpathlineto{\pgfqpoint{4.212720in}{0.723515in}}%
\pgfpathlineto{\pgfqpoint{4.284368in}{0.711688in}}%
\pgfpathlineto{\pgfqpoint{4.356017in}{0.706642in}}%
\pgfpathlineto{\pgfqpoint{4.427666in}{0.707298in}}%
\pgfpathlineto{\pgfqpoint{4.499314in}{0.708779in}}%
\pgfpathlineto{\pgfqpoint{4.570963in}{0.708528in}}%
\pgfpathlineto{\pgfqpoint{4.642612in}{0.707277in}}%
\pgfpathlineto{\pgfqpoint{4.714260in}{0.700862in}}%
\pgfpathlineto{\pgfqpoint{4.785909in}{0.690873in}}%
\pgfpathlineto{\pgfqpoint{4.857558in}{0.681881in}}%
\pgfpathlineto{\pgfqpoint{4.929207in}{0.682714in}}%
\pgfpathlineto{\pgfqpoint{5.000855in}{0.689629in}}%
\pgfpathlineto{\pgfqpoint{5.072504in}{0.692982in}}%
\pgfpathlineto{\pgfqpoint{5.144153in}{0.692852in}}%
\pgfusepath{stroke}%
\end{pgfscope}%
\begin{pgfscope}%
\pgfpathrectangle{\pgfqpoint{0.705516in}{0.523570in}}{\pgfqpoint{4.650000in}{3.020000in}} %
\pgfusepath{clip}%
\pgfsetrectcap%
\pgfsetroundjoin%
\pgfsetlinewidth{1.505625pt}%
\definecolor{currentstroke}{rgb}{1.000000,0.000000,0.000000}%
\pgfsetstrokecolor{currentstroke}%
\pgfsetstrokeopacity{0.700000}%
\pgfsetdash{}{0pt}%
\pgfpathmoveto{\pgfqpoint{0.916880in}{3.406297in}}%
\pgfpathlineto{\pgfqpoint{0.988529in}{2.668833in}}%
\pgfpathlineto{\pgfqpoint{1.060177in}{2.228948in}}%
\pgfpathlineto{\pgfqpoint{1.131826in}{1.965694in}}%
\pgfpathlineto{\pgfqpoint{1.203475in}{1.782735in}}%
\pgfpathlineto{\pgfqpoint{1.275123in}{1.646049in}}%
\pgfpathlineto{\pgfqpoint{1.346772in}{1.540534in}}%
\pgfpathlineto{\pgfqpoint{1.418421in}{1.456474in}}%
\pgfpathlineto{\pgfqpoint{1.490069in}{1.387279in}}%
\pgfpathlineto{\pgfqpoint{1.561718in}{1.328930in}}%
\pgfpathlineto{\pgfqpoint{1.633367in}{1.278882in}}%
\pgfpathlineto{\pgfqpoint{1.705015in}{1.235176in}}%
\pgfpathlineto{\pgfqpoint{1.776664in}{1.196611in}}%
\pgfpathlineto{\pgfqpoint{1.848313in}{1.162175in}}%
\pgfpathlineto{\pgfqpoint{1.919962in}{1.131146in}}%
\pgfpathlineto{\pgfqpoint{1.991610in}{1.102974in}}%
\pgfpathlineto{\pgfqpoint{2.063259in}{1.077210in}}%
\pgfpathlineto{\pgfqpoint{2.134908in}{1.053525in}}%
\pgfpathlineto{\pgfqpoint{2.206556in}{1.031663in}}%
\pgfpathlineto{\pgfqpoint{2.278205in}{1.011394in}}%
\pgfpathlineto{\pgfqpoint{2.349854in}{0.992471in}}%
\pgfpathlineto{\pgfqpoint{2.421502in}{0.974964in}}%
\pgfpathlineto{\pgfqpoint{2.493151in}{0.958446in}}%
\pgfpathlineto{\pgfqpoint{2.564800in}{0.942982in}}%
\pgfpathlineto{\pgfqpoint{2.636448in}{0.928458in}}%
\pgfpathlineto{\pgfqpoint{2.708097in}{0.914784in}}%
\pgfpathlineto{\pgfqpoint{2.779746in}{0.901892in}}%
\pgfpathlineto{\pgfqpoint{2.851395in}{0.889723in}}%
\pgfpathlineto{\pgfqpoint{2.923043in}{0.878231in}}%
\pgfpathlineto{\pgfqpoint{2.994692in}{0.867371in}}%
\pgfpathlineto{\pgfqpoint{3.066341in}{0.857102in}}%
\pgfpathlineto{\pgfqpoint{3.137989in}{0.847396in}}%
\pgfpathlineto{\pgfqpoint{3.209638in}{0.838205in}}%
\pgfpathlineto{\pgfqpoint{3.281287in}{0.829513in}}%
\pgfpathlineto{\pgfqpoint{3.352935in}{0.821293in}}%
\pgfpathlineto{\pgfqpoint{3.424584in}{0.813520in}}%
\pgfpathlineto{\pgfqpoint{3.496233in}{0.806174in}}%
\pgfpathlineto{\pgfqpoint{3.567881in}{0.799235in}}%
\pgfpathlineto{\pgfqpoint{3.639530in}{0.792690in}}%
\pgfpathlineto{\pgfqpoint{3.711179in}{0.786526in}}%
\pgfpathlineto{\pgfqpoint{3.782827in}{0.780728in}}%
\pgfpathlineto{\pgfqpoint{3.854476in}{0.775283in}}%
\pgfpathlineto{\pgfqpoint{3.926125in}{0.770179in}}%
\pgfpathlineto{\pgfqpoint{3.997774in}{0.765408in}}%
\pgfpathlineto{\pgfqpoint{4.069422in}{0.760959in}}%
\pgfpathlineto{\pgfqpoint{4.141071in}{0.756822in}}%
\pgfpathlineto{\pgfqpoint{4.212720in}{0.752990in}}%
\pgfpathlineto{\pgfqpoint{4.284368in}{0.749457in}}%
\pgfpathlineto{\pgfqpoint{4.356017in}{0.746217in}}%
\pgfpathlineto{\pgfqpoint{4.427666in}{0.743267in}}%
\pgfpathlineto{\pgfqpoint{4.499314in}{0.740600in}}%
\pgfpathlineto{\pgfqpoint{4.570963in}{0.738212in}}%
\pgfpathlineto{\pgfqpoint{4.642612in}{0.736100in}}%
\pgfpathlineto{\pgfqpoint{4.714260in}{0.734260in}}%
\pgfpathlineto{\pgfqpoint{4.785909in}{0.732688in}}%
\pgfpathlineto{\pgfqpoint{4.857558in}{0.731380in}}%
\pgfpathlineto{\pgfqpoint{4.929207in}{0.730335in}}%
\pgfpathlineto{\pgfqpoint{5.000855in}{0.729551in}}%
\pgfpathlineto{\pgfqpoint{5.072504in}{0.729029in}}%
\pgfpathlineto{\pgfqpoint{5.144153in}{0.728769in}}%
\pgfusepath{stroke}%
\end{pgfscope}%
\begin{pgfscope}%
\pgfsetrectcap%
\pgfsetmiterjoin%
\pgfsetlinewidth{0.803000pt}%
\definecolor{currentstroke}{rgb}{0.000000,0.000000,0.000000}%
\pgfsetstrokecolor{currentstroke}%
\pgfsetdash{}{0pt}%
\pgfpathmoveto{\pgfqpoint{0.705516in}{0.523570in}}%
\pgfpathlineto{\pgfqpoint{0.705516in}{3.543570in}}%
\pgfusepath{stroke}%
\end{pgfscope}%
\begin{pgfscope}%
\pgfsetrectcap%
\pgfsetmiterjoin%
\pgfsetlinewidth{0.803000pt}%
\definecolor{currentstroke}{rgb}{0.000000,0.000000,0.000000}%
\pgfsetstrokecolor{currentstroke}%
\pgfsetdash{}{0pt}%
\pgfpathmoveto{\pgfqpoint{5.355516in}{0.523570in}}%
\pgfpathlineto{\pgfqpoint{5.355516in}{3.543570in}}%
\pgfusepath{stroke}%
\end{pgfscope}%
\begin{pgfscope}%
\pgfsetrectcap%
\pgfsetmiterjoin%
\pgfsetlinewidth{0.803000pt}%
\definecolor{currentstroke}{rgb}{0.000000,0.000000,0.000000}%
\pgfsetstrokecolor{currentstroke}%
\pgfsetdash{}{0pt}%
\pgfpathmoveto{\pgfqpoint{0.705516in}{0.523570in}}%
\pgfpathlineto{\pgfqpoint{5.355516in}{0.523570in}}%
\pgfusepath{stroke}%
\end{pgfscope}%
\begin{pgfscope}%
\pgfsetrectcap%
\pgfsetmiterjoin%
\pgfsetlinewidth{0.803000pt}%
\definecolor{currentstroke}{rgb}{0.000000,0.000000,0.000000}%
\pgfsetstrokecolor{currentstroke}%
\pgfsetdash{}{0pt}%
\pgfpathmoveto{\pgfqpoint{0.705516in}{3.543570in}}%
\pgfpathlineto{\pgfqpoint{5.355516in}{3.543570in}}%
\pgfusepath{stroke}%
\end{pgfscope}%
\begin{pgfscope}%
\pgfsetbuttcap%
\pgfsetmiterjoin%
\definecolor{currentfill}{rgb}{1.000000,1.000000,1.000000}%
\pgfsetfillcolor{currentfill}%
\pgfsetfillopacity{0.800000}%
\pgfsetlinewidth{1.003750pt}%
\definecolor{currentstroke}{rgb}{0.800000,0.800000,0.800000}%
\pgfsetstrokecolor{currentstroke}%
\pgfsetstrokeopacity{0.800000}%
\pgfsetdash{}{0pt}%
\pgfpathmoveto{\pgfqpoint{3.063293in}{2.409239in}}%
\pgfpathlineto{\pgfqpoint{5.258294in}{2.409239in}}%
\pgfpathquadraticcurveto{\pgfqpoint{5.286072in}{2.409239in}}{\pgfqpoint{5.286072in}{2.437017in}}%
\pgfpathlineto{\pgfqpoint{5.286072in}{3.446348in}}%
\pgfpathquadraticcurveto{\pgfqpoint{5.286072in}{3.474126in}}{\pgfqpoint{5.258294in}{3.474126in}}%
\pgfpathlineto{\pgfqpoint{3.063293in}{3.474126in}}%
\pgfpathquadraticcurveto{\pgfqpoint{3.035516in}{3.474126in}}{\pgfqpoint{3.035516in}{3.446348in}}%
\pgfpathlineto{\pgfqpoint{3.035516in}{2.437017in}}%
\pgfpathquadraticcurveto{\pgfqpoint{3.035516in}{2.409239in}}{\pgfqpoint{3.063293in}{2.409239in}}%
\pgfpathclose%
\pgfusepath{stroke,fill}%
\end{pgfscope}%
\begin{pgfscope}%
\pgfsetrectcap%
\pgfsetroundjoin%
\pgfsetlinewidth{1.505625pt}%
\definecolor{currentstroke}{rgb}{0.000000,0.000000,0.000000}%
\pgfsetstrokecolor{currentstroke}%
\pgfsetstrokeopacity{0.700000}%
\pgfsetdash{}{0pt}%
\pgfpathmoveto{\pgfqpoint{3.091071in}{3.361658in}}%
\pgfpathlineto{\pgfqpoint{3.368849in}{3.361658in}}%
\pgfusepath{stroke}%
\end{pgfscope}%
\begin{pgfscope}%
\pgftext[x=3.479960in,y=3.313047in,left,base]{\rmfamily\fontsize{10.000000}{12.000000}\selectfont Raw signal}%
\end{pgfscope}%
\begin{pgfscope}%
\pgfsetrectcap%
\pgfsetroundjoin%
\pgfsetlinewidth{1.505625pt}%
\definecolor{currentstroke}{rgb}{0.750000,0.000000,0.750000}%
\pgfsetstrokecolor{currentstroke}%
\pgfsetstrokeopacity{0.700000}%
\pgfsetdash{}{0pt}%
\pgfpathmoveto{\pgfqpoint{3.091071in}{3.155834in}}%
\pgfpathlineto{\pgfqpoint{3.368849in}{3.155834in}}%
\pgfusepath{stroke}%
\end{pgfscope}%
\begin{pgfscope}%
\pgftext[x=3.479960in,y=3.107223in,left,base]{\rmfamily\fontsize{10.000000}{12.000000}\selectfont 1D Gaussian convolution}%
\end{pgfscope}%
\begin{pgfscope}%
\pgfsetrectcap%
\pgfsetroundjoin%
\pgfsetlinewidth{1.505625pt}%
\definecolor{currentstroke}{rgb}{0.000000,0.000000,1.000000}%
\pgfsetstrokecolor{currentstroke}%
\pgfsetstrokeopacity{0.700000}%
\pgfsetdash{}{0pt}%
\pgfpathmoveto{\pgfqpoint{3.091071in}{2.951977in}}%
\pgfpathlineto{\pgfqpoint{3.368849in}{2.951977in}}%
\pgfusepath{stroke}%
\end{pgfscope}%
\begin{pgfscope}%
\pgftext[x=3.479960in,y=2.903366in,left,base]{\rmfamily\fontsize{10.000000}{12.000000}\selectfont Butterworth}%
\end{pgfscope}%
\begin{pgfscope}%
\pgfsetrectcap%
\pgfsetroundjoin%
\pgfsetlinewidth{1.505625pt}%
\definecolor{currentstroke}{rgb}{0.000000,0.750000,0.750000}%
\pgfsetstrokecolor{currentstroke}%
\pgfsetstrokeopacity{0.700000}%
\pgfsetdash{}{0pt}%
\pgfpathmoveto{\pgfqpoint{3.091071in}{2.748120in}}%
\pgfpathlineto{\pgfqpoint{3.368849in}{2.748120in}}%
\pgfusepath{stroke}%
\end{pgfscope}%
\begin{pgfscope}%
\pgftext[x=3.479960in,y=2.699509in,left,base]{\rmfamily\fontsize{10.000000}{12.000000}\selectfont Wiener}%
\end{pgfscope}%
\begin{pgfscope}%
\pgfsetrectcap%
\pgfsetroundjoin%
\pgfsetlinewidth{1.505625pt}%
\definecolor{currentstroke}{rgb}{1.000000,0.000000,0.000000}%
\pgfsetstrokecolor{currentstroke}%
\pgfsetstrokeopacity{0.700000}%
\pgfsetdash{}{0pt}%
\pgfpathmoveto{\pgfqpoint{3.091071in}{2.544263in}}%
\pgfpathlineto{\pgfqpoint{3.368849in}{2.544263in}}%
\pgfusepath{stroke}%
\end{pgfscope}%
\begin{pgfscope}%
\pgftext[x=3.479960in,y=2.495651in,left,base]{\rmfamily\fontsize{10.000000}{12.000000}\selectfont Savitsky-Golay}%
\end{pgfscope}%
\end{pgfpicture}%
\makeatother%
\endgroup%

    \caption{A simple EMA plot.\label{fig:ema1}}
\end{figure}

\begin{figure}
    \centering
    %% Creator: Matplotlib, PGF backend
%%
%% To include the figure in your LaTeX document, write
%%   \input{<filename>.pgf}
%%
%% Make sure the required packages are loaded in your preamble
%%   \usepackage{pgf}
%%
%% Figures using additional raster images can only be included by \input if
%% they are in the same directory as the main LaTeX file. For loading figures
%% from other directories you can use the `import` package
%%   \usepackage{import}
%% and then include the figures with
%%   \import{<path to file>}{<filename>.pgf}
%%
%% Matplotlib used the following preamble
%%   \usepackage{fontspec}
%%   \setmainfont{DejaVu Serif}
%%   \setsansfont{DejaVu Sans}
%%   \setmonofont{DejaVu Sans Mono}
%%
\begingroup%
\makeatletter%
\begin{pgfpicture}%
\pgfpathrectangle{\pgfpointorigin}{\pgfqpoint{5.066664in}{3.471851in}}%
\pgfusepath{use as bounding box, clip}%
\begin{pgfscope}%
\pgfsetbuttcap%
\pgfsetmiterjoin%
\definecolor{currentfill}{rgb}{1.000000,1.000000,1.000000}%
\pgfsetfillcolor{currentfill}%
\pgfsetlinewidth{0.000000pt}%
\definecolor{currentstroke}{rgb}{1.000000,1.000000,1.000000}%
\pgfsetstrokecolor{currentstroke}%
\pgfsetdash{}{0pt}%
\pgfpathmoveto{\pgfqpoint{0.000000in}{0.000000in}}%
\pgfpathlineto{\pgfqpoint{5.066664in}{0.000000in}}%
\pgfpathlineto{\pgfqpoint{5.066664in}{3.471851in}}%
\pgfpathlineto{\pgfqpoint{0.000000in}{3.471851in}}%
\pgfpathclose%
\pgfusepath{fill}%
\end{pgfscope}%
\begin{pgfscope}%
\pgfsetbuttcap%
\pgfsetmiterjoin%
\definecolor{currentfill}{rgb}{1.000000,1.000000,1.000000}%
\pgfsetfillcolor{currentfill}%
\pgfsetlinewidth{0.000000pt}%
\definecolor{currentstroke}{rgb}{0.000000,0.000000,0.000000}%
\pgfsetstrokecolor{currentstroke}%
\pgfsetstrokeopacity{0.000000}%
\pgfsetdash{}{0pt}%
\pgfpathmoveto{\pgfqpoint{0.281664in}{0.316851in}}%
\pgfpathlineto{\pgfqpoint{4.931664in}{0.316851in}}%
\pgfpathlineto{\pgfqpoint{4.931664in}{3.336851in}}%
\pgfpathlineto{\pgfqpoint{0.281664in}{3.336851in}}%
\pgfpathclose%
\pgfusepath{fill}%
\end{pgfscope}%
\begin{pgfscope}%
\pgftext[x=2.606664in,y=0.261295in,,top]{\rmfamily\fontsize{12.000000}{14.400000}\selectfont t}%
\end{pgfscope}%
\begin{pgfscope}%
\pgftext[x=0.142775in,y=1.826851in,,bottom]{\rmfamily\fontsize{12.000000}{14.400000}\selectfont \(\displaystyle y\)}%
\end{pgfscope}%
\begin{pgfscope}%
\pgfpathrectangle{\pgfqpoint{0.281664in}{0.316851in}}{\pgfqpoint{4.650000in}{3.020000in}} %
\pgfusepath{clip}%
\pgfsetbuttcap%
\pgfsetroundjoin%
\definecolor{currentfill}{rgb}{0.000000,0.000000,0.000000}%
\pgfsetfillcolor{currentfill}%
\pgfsetlinewidth{1.003750pt}%
\definecolor{currentstroke}{rgb}{0.000000,0.000000,0.000000}%
\pgfsetstrokecolor{currentstroke}%
\pgfsetdash{}{0pt}%
\pgfsys@defobject{currentmarker}{\pgfqpoint{-0.020833in}{-0.020833in}}{\pgfqpoint{0.020833in}{0.020833in}}{%
\pgfpathmoveto{\pgfqpoint{0.000000in}{-0.020833in}}%
\pgfpathcurveto{\pgfqpoint{0.005525in}{-0.020833in}}{\pgfqpoint{0.010825in}{-0.018638in}}{\pgfqpoint{0.014731in}{-0.014731in}}%
\pgfpathcurveto{\pgfqpoint{0.018638in}{-0.010825in}}{\pgfqpoint{0.020833in}{-0.005525in}}{\pgfqpoint{0.020833in}{0.000000in}}%
\pgfpathcurveto{\pgfqpoint{0.020833in}{0.005525in}}{\pgfqpoint{0.018638in}{0.010825in}}{\pgfqpoint{0.014731in}{0.014731in}}%
\pgfpathcurveto{\pgfqpoint{0.010825in}{0.018638in}}{\pgfqpoint{0.005525in}{0.020833in}}{\pgfqpoint{0.000000in}{0.020833in}}%
\pgfpathcurveto{\pgfqpoint{-0.005525in}{0.020833in}}{\pgfqpoint{-0.010825in}{0.018638in}}{\pgfqpoint{-0.014731in}{0.014731in}}%
\pgfpathcurveto{\pgfqpoint{-0.018638in}{0.010825in}}{\pgfqpoint{-0.020833in}{0.005525in}}{\pgfqpoint{-0.020833in}{0.000000in}}%
\pgfpathcurveto{\pgfqpoint{-0.020833in}{-0.005525in}}{\pgfqpoint{-0.018638in}{-0.010825in}}{\pgfqpoint{-0.014731in}{-0.014731in}}%
\pgfpathcurveto{\pgfqpoint{-0.010825in}{-0.018638in}}{\pgfqpoint{-0.005525in}{-0.020833in}}{\pgfqpoint{0.000000in}{-0.020833in}}%
\pgfpathclose%
\pgfusepath{stroke,fill}%
}%
\begin{pgfscope}%
\pgfsys@transformshift{0.493028in}{0.928679in}%
\pgfsys@useobject{currentmarker}{}%
\end{pgfscope}%
\begin{pgfscope}%
\pgfsys@transformshift{0.528852in}{1.000537in}%
\pgfsys@useobject{currentmarker}{}%
\end{pgfscope}%
\begin{pgfscope}%
\pgfsys@transformshift{0.564677in}{1.052538in}%
\pgfsys@useobject{currentmarker}{}%
\end{pgfscope}%
\begin{pgfscope}%
\pgfsys@transformshift{0.600501in}{1.137040in}%
\pgfsys@useobject{currentmarker}{}%
\end{pgfscope}%
\begin{pgfscope}%
\pgfsys@transformshift{0.636325in}{1.250622in}%
\pgfsys@useobject{currentmarker}{}%
\end{pgfscope}%
\begin{pgfscope}%
\pgfsys@transformshift{0.672150in}{1.335634in}%
\pgfsys@useobject{currentmarker}{}%
\end{pgfscope}%
\begin{pgfscope}%
\pgfsys@transformshift{0.707974in}{1.420754in}%
\pgfsys@useobject{currentmarker}{}%
\end{pgfscope}%
\begin{pgfscope}%
\pgfsys@transformshift{0.743798in}{1.470554in}%
\pgfsys@useobject{currentmarker}{}%
\end{pgfscope}%
\begin{pgfscope}%
\pgfsys@transformshift{0.779623in}{1.514713in}%
\pgfsys@useobject{currentmarker}{}%
\end{pgfscope}%
\begin{pgfscope}%
\pgfsys@transformshift{0.815447in}{1.568129in}%
\pgfsys@useobject{currentmarker}{}%
\end{pgfscope}%
\begin{pgfscope}%
\pgfsys@transformshift{0.851271in}{1.649470in}%
\pgfsys@useobject{currentmarker}{}%
\end{pgfscope}%
\begin{pgfscope}%
\pgfsys@transformshift{0.887096in}{1.737655in}%
\pgfsys@useobject{currentmarker}{}%
\end{pgfscope}%
\begin{pgfscope}%
\pgfsys@transformshift{0.922920in}{1.829330in}%
\pgfsys@useobject{currentmarker}{}%
\end{pgfscope}%
\begin{pgfscope}%
\pgfsys@transformshift{0.958744in}{1.883428in}%
\pgfsys@useobject{currentmarker}{}%
\end{pgfscope}%
\begin{pgfscope}%
\pgfsys@transformshift{0.994569in}{1.910614in}%
\pgfsys@useobject{currentmarker}{}%
\end{pgfscope}%
\begin{pgfscope}%
\pgfsys@transformshift{1.030393in}{1.962106in}%
\pgfsys@useobject{currentmarker}{}%
\end{pgfscope}%
\begin{pgfscope}%
\pgfsys@transformshift{1.066217in}{2.030846in}%
\pgfsys@useobject{currentmarker}{}%
\end{pgfscope}%
\begin{pgfscope}%
\pgfsys@transformshift{1.102042in}{2.111746in}%
\pgfsys@useobject{currentmarker}{}%
\end{pgfscope}%
\begin{pgfscope}%
\pgfsys@transformshift{1.137866in}{2.172149in}%
\pgfsys@useobject{currentmarker}{}%
\end{pgfscope}%
\begin{pgfscope}%
\pgfsys@transformshift{1.173690in}{2.201081in}%
\pgfsys@useobject{currentmarker}{}%
\end{pgfscope}%
\begin{pgfscope}%
\pgfsys@transformshift{1.209515in}{2.225526in}%
\pgfsys@useobject{currentmarker}{}%
\end{pgfscope}%
\begin{pgfscope}%
\pgfsys@transformshift{1.245339in}{2.259815in}%
\pgfsys@useobject{currentmarker}{}%
\end{pgfscope}%
\begin{pgfscope}%
\pgfsys@transformshift{1.281163in}{2.309263in}%
\pgfsys@useobject{currentmarker}{}%
\end{pgfscope}%
\begin{pgfscope}%
\pgfsys@transformshift{1.316988in}{2.385577in}%
\pgfsys@useobject{currentmarker}{}%
\end{pgfscope}%
\begin{pgfscope}%
\pgfsys@transformshift{1.352812in}{2.452705in}%
\pgfsys@useobject{currentmarker}{}%
\end{pgfscope}%
\begin{pgfscope}%
\pgfsys@transformshift{1.388637in}{2.481667in}%
\pgfsys@useobject{currentmarker}{}%
\end{pgfscope}%
\begin{pgfscope}%
\pgfsys@transformshift{1.424461in}{2.494290in}%
\pgfsys@useobject{currentmarker}{}%
\end{pgfscope}%
\begin{pgfscope}%
\pgfsys@transformshift{1.460285in}{2.526482in}%
\pgfsys@useobject{currentmarker}{}%
\end{pgfscope}%
\begin{pgfscope}%
\pgfsys@transformshift{1.496110in}{2.581446in}%
\pgfsys@useobject{currentmarker}{}%
\end{pgfscope}%
\begin{pgfscope}%
\pgfsys@transformshift{1.531934in}{2.642921in}%
\pgfsys@useobject{currentmarker}{}%
\end{pgfscope}%
\begin{pgfscope}%
\pgfsys@transformshift{1.567758in}{2.680748in}%
\pgfsys@useobject{currentmarker}{}%
\end{pgfscope}%
\begin{pgfscope}%
\pgfsys@transformshift{1.603583in}{2.703974in}%
\pgfsys@useobject{currentmarker}{}%
\end{pgfscope}%
\begin{pgfscope}%
\pgfsys@transformshift{1.639407in}{2.708222in}%
\pgfsys@useobject{currentmarker}{}%
\end{pgfscope}%
\begin{pgfscope}%
\pgfsys@transformshift{1.675231in}{2.722180in}%
\pgfsys@useobject{currentmarker}{}%
\end{pgfscope}%
\begin{pgfscope}%
\pgfsys@transformshift{1.711056in}{2.766227in}%
\pgfsys@useobject{currentmarker}{}%
\end{pgfscope}%
\begin{pgfscope}%
\pgfsys@transformshift{1.746880in}{2.820876in}%
\pgfsys@useobject{currentmarker}{}%
\end{pgfscope}%
\begin{pgfscope}%
\pgfsys@transformshift{1.782704in}{2.865133in}%
\pgfsys@useobject{currentmarker}{}%
\end{pgfscope}%
\begin{pgfscope}%
\pgfsys@transformshift{1.818529in}{2.879349in}%
\pgfsys@useobject{currentmarker}{}%
\end{pgfscope}%
\begin{pgfscope}%
\pgfsys@transformshift{1.854353in}{2.879949in}%
\pgfsys@useobject{currentmarker}{}%
\end{pgfscope}%
\begin{pgfscope}%
\pgfsys@transformshift{1.890177in}{2.898561in}%
\pgfsys@useobject{currentmarker}{}%
\end{pgfscope}%
\begin{pgfscope}%
\pgfsys@transformshift{1.926002in}{2.941603in}%
\pgfsys@useobject{currentmarker}{}%
\end{pgfscope}%
\begin{pgfscope}%
\pgfsys@transformshift{1.961826in}{2.983568in}%
\pgfsys@useobject{currentmarker}{}%
\end{pgfscope}%
\begin{pgfscope}%
\pgfsys@transformshift{1.997650in}{3.012917in}%
\pgfsys@useobject{currentmarker}{}%
\end{pgfscope}%
\begin{pgfscope}%
\pgfsys@transformshift{2.033475in}{3.015178in}%
\pgfsys@useobject{currentmarker}{}%
\end{pgfscope}%
\begin{pgfscope}%
\pgfsys@transformshift{2.069299in}{3.008037in}%
\pgfsys@useobject{currentmarker}{}%
\end{pgfscope}%
\begin{pgfscope}%
\pgfsys@transformshift{2.105123in}{3.014275in}%
\pgfsys@useobject{currentmarker}{}%
\end{pgfscope}%
\begin{pgfscope}%
\pgfsys@transformshift{2.140948in}{3.043740in}%
\pgfsys@useobject{currentmarker}{}%
\end{pgfscope}%
\begin{pgfscope}%
\pgfsys@transformshift{2.176772in}{3.084525in}%
\pgfsys@useobject{currentmarker}{}%
\end{pgfscope}%
\begin{pgfscope}%
\pgfsys@transformshift{2.212596in}{3.105812in}%
\pgfsys@useobject{currentmarker}{}%
\end{pgfscope}%
\begin{pgfscope}%
\pgfsys@transformshift{2.248421in}{3.105416in}%
\pgfsys@useobject{currentmarker}{}%
\end{pgfscope}%
\begin{pgfscope}%
\pgfsys@transformshift{2.284245in}{3.096852in}%
\pgfsys@useobject{currentmarker}{}%
\end{pgfscope}%
\begin{pgfscope}%
\pgfsys@transformshift{2.320069in}{3.106378in}%
\pgfsys@useobject{currentmarker}{}%
\end{pgfscope}%
\begin{pgfscope}%
\pgfsys@transformshift{2.355894in}{3.137411in}%
\pgfsys@useobject{currentmarker}{}%
\end{pgfscope}%
\begin{pgfscope}%
\pgfsys@transformshift{2.391718in}{3.169125in}%
\pgfsys@useobject{currentmarker}{}%
\end{pgfscope}%
\begin{pgfscope}%
\pgfsys@transformshift{2.427543in}{3.176838in}%
\pgfsys@useobject{currentmarker}{}%
\end{pgfscope}%
\begin{pgfscope}%
\pgfsys@transformshift{2.463367in}{3.166474in}%
\pgfsys@useobject{currentmarker}{}%
\end{pgfscope}%
\begin{pgfscope}%
\pgfsys@transformshift{2.499191in}{3.151557in}%
\pgfsys@useobject{currentmarker}{}%
\end{pgfscope}%
\begin{pgfscope}%
\pgfsys@transformshift{2.535016in}{3.147031in}%
\pgfsys@useobject{currentmarker}{}%
\end{pgfscope}%
\begin{pgfscope}%
\pgfsys@transformshift{2.570840in}{3.165847in}%
\pgfsys@useobject{currentmarker}{}%
\end{pgfscope}%
\begin{pgfscope}%
\pgfsys@transformshift{2.606664in}{3.195505in}%
\pgfsys@useobject{currentmarker}{}%
\end{pgfscope}%
\begin{pgfscope}%
\pgfsys@transformshift{2.642489in}{3.196283in}%
\pgfsys@useobject{currentmarker}{}%
\end{pgfscope}%
\begin{pgfscope}%
\pgfsys@transformshift{2.678313in}{3.181265in}%
\pgfsys@useobject{currentmarker}{}%
\end{pgfscope}%
\begin{pgfscope}%
\pgfsys@transformshift{2.714137in}{3.162399in}%
\pgfsys@useobject{currentmarker}{}%
\end{pgfscope}%
\begin{pgfscope}%
\pgfsys@transformshift{2.749962in}{3.166078in}%
\pgfsys@useobject{currentmarker}{}%
\end{pgfscope}%
\begin{pgfscope}%
\pgfsys@transformshift{2.785786in}{3.186534in}%
\pgfsys@useobject{currentmarker}{}%
\end{pgfscope}%
\begin{pgfscope}%
\pgfsys@transformshift{2.821610in}{3.199578in}%
\pgfsys@useobject{currentmarker}{}%
\end{pgfscope}%
\begin{pgfscope}%
\pgfsys@transformshift{2.857435in}{3.196022in}%
\pgfsys@useobject{currentmarker}{}%
\end{pgfscope}%
\begin{pgfscope}%
\pgfsys@transformshift{2.893259in}{3.172912in}%
\pgfsys@useobject{currentmarker}{}%
\end{pgfscope}%
\begin{pgfscope}%
\pgfsys@transformshift{2.929083in}{3.144159in}%
\pgfsys@useobject{currentmarker}{}%
\end{pgfscope}%
\begin{pgfscope}%
\pgfsys@transformshift{2.964908in}{3.134666in}%
\pgfsys@useobject{currentmarker}{}%
\end{pgfscope}%
\begin{pgfscope}%
\pgfsys@transformshift{3.000732in}{3.144276in}%
\pgfsys@useobject{currentmarker}{}%
\end{pgfscope}%
\begin{pgfscope}%
\pgfsys@transformshift{3.036556in}{3.155500in}%
\pgfsys@useobject{currentmarker}{}%
\end{pgfscope}%
\begin{pgfscope}%
\pgfsys@transformshift{3.072381in}{3.144574in}%
\pgfsys@useobject{currentmarker}{}%
\end{pgfscope}%
\begin{pgfscope}%
\pgfsys@transformshift{3.108205in}{3.110535in}%
\pgfsys@useobject{currentmarker}{}%
\end{pgfscope}%
\begin{pgfscope}%
\pgfsys@transformshift{3.144029in}{3.088562in}%
\pgfsys@useobject{currentmarker}{}%
\end{pgfscope}%
\begin{pgfscope}%
\pgfsys@transformshift{3.179854in}{3.082270in}%
\pgfsys@useobject{currentmarker}{}%
\end{pgfscope}%
\begin{pgfscope}%
\pgfsys@transformshift{3.215678in}{3.088172in}%
\pgfsys@useobject{currentmarker}{}%
\end{pgfscope}%
\begin{pgfscope}%
\pgfsys@transformshift{3.251502in}{3.088947in}%
\pgfsys@useobject{currentmarker}{}%
\end{pgfscope}%
\begin{pgfscope}%
\pgfsys@transformshift{3.287327in}{3.068084in}%
\pgfsys@useobject{currentmarker}{}%
\end{pgfscope}%
\begin{pgfscope}%
\pgfsys@transformshift{3.323151in}{3.030370in}%
\pgfsys@useobject{currentmarker}{}%
\end{pgfscope}%
\begin{pgfscope}%
\pgfsys@transformshift{3.358975in}{2.995229in}%
\pgfsys@useobject{currentmarker}{}%
\end{pgfscope}%
\begin{pgfscope}%
\pgfsys@transformshift{3.394800in}{2.977707in}%
\pgfsys@useobject{currentmarker}{}%
\end{pgfscope}%
\begin{pgfscope}%
\pgfsys@transformshift{3.430624in}{2.977950in}%
\pgfsys@useobject{currentmarker}{}%
\end{pgfscope}%
\begin{pgfscope}%
\pgfsys@transformshift{3.466449in}{2.967165in}%
\pgfsys@useobject{currentmarker}{}%
\end{pgfscope}%
\begin{pgfscope}%
\pgfsys@transformshift{3.502273in}{2.939776in}%
\pgfsys@useobject{currentmarker}{}%
\end{pgfscope}%
\begin{pgfscope}%
\pgfsys@transformshift{3.538097in}{2.896261in}%
\pgfsys@useobject{currentmarker}{}%
\end{pgfscope}%
\begin{pgfscope}%
\pgfsys@transformshift{3.573922in}{2.862819in}%
\pgfsys@useobject{currentmarker}{}%
\end{pgfscope}%
\begin{pgfscope}%
\pgfsys@transformshift{3.609746in}{2.847574in}%
\pgfsys@useobject{currentmarker}{}%
\end{pgfscope}%
\begin{pgfscope}%
\pgfsys@transformshift{3.645570in}{2.842148in}%
\pgfsys@useobject{currentmarker}{}%
\end{pgfscope}%
\begin{pgfscope}%
\pgfsys@transformshift{3.681395in}{2.822805in}%
\pgfsys@useobject{currentmarker}{}%
\end{pgfscope}%
\begin{pgfscope}%
\pgfsys@transformshift{3.717219in}{2.784820in}%
\pgfsys@useobject{currentmarker}{}%
\end{pgfscope}%
\begin{pgfscope}%
\pgfsys@transformshift{3.753043in}{2.735235in}%
\pgfsys@useobject{currentmarker}{}%
\end{pgfscope}%
\begin{pgfscope}%
\pgfsys@transformshift{3.788868in}{2.688874in}%
\pgfsys@useobject{currentmarker}{}%
\end{pgfscope}%
\begin{pgfscope}%
\pgfsys@transformshift{3.824692in}{2.665390in}%
\pgfsys@useobject{currentmarker}{}%
\end{pgfscope}%
\begin{pgfscope}%
\pgfsys@transformshift{3.860516in}{2.651989in}%
\pgfsys@useobject{currentmarker}{}%
\end{pgfscope}%
\begin{pgfscope}%
\pgfsys@transformshift{3.896341in}{2.624050in}%
\pgfsys@useobject{currentmarker}{}%
\end{pgfscope}%
\begin{pgfscope}%
\pgfsys@transformshift{3.932165in}{2.573173in}%
\pgfsys@useobject{currentmarker}{}%
\end{pgfscope}%
\begin{pgfscope}%
\pgfsys@transformshift{3.967989in}{2.516050in}%
\pgfsys@useobject{currentmarker}{}%
\end{pgfscope}%
\begin{pgfscope}%
\pgfsys@transformshift{4.003814in}{2.471636in}%
\pgfsys@useobject{currentmarker}{}%
\end{pgfscope}%
\begin{pgfscope}%
\pgfsys@transformshift{4.039638in}{2.446527in}%
\pgfsys@useobject{currentmarker}{}%
\end{pgfscope}%
\begin{pgfscope}%
\pgfsys@transformshift{4.075462in}{2.420950in}%
\pgfsys@useobject{currentmarker}{}%
\end{pgfscope}%
\begin{pgfscope}%
\pgfsys@transformshift{4.111287in}{2.383679in}%
\pgfsys@useobject{currentmarker}{}%
\end{pgfscope}%
\begin{pgfscope}%
\pgfsys@transformshift{4.147111in}{2.326908in}%
\pgfsys@useobject{currentmarker}{}%
\end{pgfscope}%
\begin{pgfscope}%
\pgfsys@transformshift{4.182935in}{2.258851in}%
\pgfsys@useobject{currentmarker}{}%
\end{pgfscope}%
\begin{pgfscope}%
\pgfsys@transformshift{4.218760in}{2.205708in}%
\pgfsys@useobject{currentmarker}{}%
\end{pgfscope}%
\begin{pgfscope}%
\pgfsys@transformshift{4.254584in}{2.170752in}%
\pgfsys@useobject{currentmarker}{}%
\end{pgfscope}%
\begin{pgfscope}%
\pgfsys@transformshift{4.290408in}{2.137112in}%
\pgfsys@useobject{currentmarker}{}%
\end{pgfscope}%
\begin{pgfscope}%
\pgfsys@transformshift{4.326233in}{2.085140in}%
\pgfsys@useobject{currentmarker}{}%
\end{pgfscope}%
\begin{pgfscope}%
\pgfsys@transformshift{4.362057in}{2.010225in}%
\pgfsys@useobject{currentmarker}{}%
\end{pgfscope}%
\begin{pgfscope}%
\pgfsys@transformshift{4.397881in}{1.938990in}%
\pgfsys@useobject{currentmarker}{}%
\end{pgfscope}%
\begin{pgfscope}%
\pgfsys@transformshift{4.433706in}{1.881144in}%
\pgfsys@useobject{currentmarker}{}%
\end{pgfscope}%
\begin{pgfscope}%
\pgfsys@transformshift{4.469530in}{1.836555in}%
\pgfsys@useobject{currentmarker}{}%
\end{pgfscope}%
\begin{pgfscope}%
\pgfsys@transformshift{4.505355in}{1.791362in}%
\pgfsys@useobject{currentmarker}{}%
\end{pgfscope}%
\begin{pgfscope}%
\pgfsys@transformshift{4.541179in}{1.729299in}%
\pgfsys@useobject{currentmarker}{}%
\end{pgfscope}%
\begin{pgfscope}%
\pgfsys@transformshift{4.577003in}{1.647511in}%
\pgfsys@useobject{currentmarker}{}%
\end{pgfscope}%
\begin{pgfscope}%
\pgfsys@transformshift{4.612828in}{1.561739in}%
\pgfsys@useobject{currentmarker}{}%
\end{pgfscope}%
\begin{pgfscope}%
\pgfsys@transformshift{4.648652in}{1.492396in}%
\pgfsys@useobject{currentmarker}{}%
\end{pgfscope}%
\begin{pgfscope}%
\pgfsys@transformshift{4.684476in}{1.436963in}%
\pgfsys@useobject{currentmarker}{}%
\end{pgfscope}%
\begin{pgfscope}%
\pgfsys@transformshift{4.720301in}{1.377489in}%
\pgfsys@useobject{currentmarker}{}%
\end{pgfscope}%
\end{pgfscope}%
\begin{pgfscope}%
\pgfpathrectangle{\pgfqpoint{0.281664in}{0.316851in}}{\pgfqpoint{4.650000in}{3.020000in}} %
\pgfusepath{clip}%
\pgfsetrectcap%
\pgfsetroundjoin%
\pgfsetlinewidth{1.505625pt}%
\definecolor{currentstroke}{rgb}{0.750000,0.000000,0.750000}%
\pgfsetstrokecolor{currentstroke}%
\pgfsetdash{}{0pt}%
\pgfpathmoveto{\pgfqpoint{0.493028in}{1.416680in}}%
\pgfpathlineto{\pgfqpoint{0.528852in}{1.421773in}}%
\pgfpathlineto{\pgfqpoint{0.564677in}{1.432125in}}%
\pgfpathlineto{\pgfqpoint{0.600501in}{1.447784in}}%
\pgfpathlineto{\pgfqpoint{0.636325in}{1.468633in}}%
\pgfpathlineto{\pgfqpoint{0.672150in}{1.494275in}}%
\pgfpathlineto{\pgfqpoint{0.707974in}{1.524039in}}%
\pgfpathlineto{\pgfqpoint{0.743798in}{1.557636in}}%
\pgfpathlineto{\pgfqpoint{0.779623in}{1.594996in}}%
\pgfpathlineto{\pgfqpoint{0.815447in}{1.635800in}}%
\pgfpathlineto{\pgfqpoint{0.851271in}{1.679629in}}%
\pgfpathlineto{\pgfqpoint{0.887096in}{1.725759in}}%
\pgfpathlineto{\pgfqpoint{0.922920in}{1.773646in}}%
\pgfpathlineto{\pgfqpoint{0.958744in}{1.822804in}}%
\pgfpathlineto{\pgfqpoint{0.994569in}{1.873101in}}%
\pgfpathlineto{\pgfqpoint{1.030393in}{1.924337in}}%
\pgfpathlineto{\pgfqpoint{1.066217in}{1.976417in}}%
\pgfpathlineto{\pgfqpoint{1.102042in}{2.028795in}}%
\pgfpathlineto{\pgfqpoint{1.137866in}{2.080832in}}%
\pgfpathlineto{\pgfqpoint{1.173690in}{2.132334in}}%
\pgfpathlineto{\pgfqpoint{1.209515in}{2.182792in}}%
\pgfpathlineto{\pgfqpoint{1.245339in}{2.231869in}}%
\pgfpathlineto{\pgfqpoint{1.281163in}{2.279704in}}%
\pgfpathlineto{\pgfqpoint{1.316988in}{2.326041in}}%
\pgfpathlineto{\pgfqpoint{1.352812in}{2.370577in}}%
\pgfpathlineto{\pgfqpoint{1.388637in}{2.413453in}}%
\pgfpathlineto{\pgfqpoint{1.424461in}{2.454756in}}%
\pgfpathlineto{\pgfqpoint{1.460285in}{2.494857in}}%
\pgfpathlineto{\pgfqpoint{1.496110in}{2.533873in}}%
\pgfpathlineto{\pgfqpoint{1.531934in}{2.571640in}}%
\pgfpathlineto{\pgfqpoint{1.567758in}{2.607869in}}%
\pgfpathlineto{\pgfqpoint{1.603583in}{2.642485in}}%
\pgfpathlineto{\pgfqpoint{1.639407in}{2.675613in}}%
\pgfpathlineto{\pgfqpoint{1.675231in}{2.707655in}}%
\pgfpathlineto{\pgfqpoint{1.711056in}{2.738796in}}%
\pgfpathlineto{\pgfqpoint{1.746880in}{2.768741in}}%
\pgfpathlineto{\pgfqpoint{1.782704in}{2.797281in}}%
\pgfpathlineto{\pgfqpoint{1.818529in}{2.824325in}}%
\pgfpathlineto{\pgfqpoint{1.854353in}{2.850073in}}%
\pgfpathlineto{\pgfqpoint{1.890177in}{2.874866in}}%
\pgfpathlineto{\pgfqpoint{1.926002in}{2.898805in}}%
\pgfpathlineto{\pgfqpoint{1.961826in}{2.921653in}}%
\pgfpathlineto{\pgfqpoint{1.997650in}{2.943214in}}%
\pgfpathlineto{\pgfqpoint{2.033475in}{2.963307in}}%
\pgfpathlineto{\pgfqpoint{2.069299in}{2.982161in}}%
\pgfpathlineto{\pgfqpoint{2.105123in}{3.000144in}}%
\pgfpathlineto{\pgfqpoint{2.140948in}{3.017316in}}%
\pgfpathlineto{\pgfqpoint{2.176772in}{3.033450in}}%
\pgfpathlineto{\pgfqpoint{2.212596in}{3.048289in}}%
\pgfpathlineto{\pgfqpoint{2.248421in}{3.061762in}}%
\pgfpathlineto{\pgfqpoint{2.284245in}{3.074142in}}%
\pgfpathlineto{\pgfqpoint{2.320069in}{3.085640in}}%
\pgfpathlineto{\pgfqpoint{2.355894in}{3.096387in}}%
\pgfpathlineto{\pgfqpoint{2.391718in}{3.106176in}}%
\pgfpathlineto{\pgfqpoint{2.427543in}{3.114666in}}%
\pgfpathlineto{\pgfqpoint{2.463367in}{3.121877in}}%
\pgfpathlineto{\pgfqpoint{2.499191in}{3.127995in}}%
\pgfpathlineto{\pgfqpoint{2.535016in}{3.133300in}}%
\pgfpathlineto{\pgfqpoint{2.570840in}{3.137847in}}%
\pgfpathlineto{\pgfqpoint{2.606664in}{3.141382in}}%
\pgfpathlineto{\pgfqpoint{2.642489in}{3.143647in}}%
\pgfpathlineto{\pgfqpoint{2.678313in}{3.144668in}}%
\pgfpathlineto{\pgfqpoint{2.714137in}{3.144636in}}%
\pgfpathlineto{\pgfqpoint{2.749962in}{3.143836in}}%
\pgfpathlineto{\pgfqpoint{2.785786in}{3.142248in}}%
\pgfpathlineto{\pgfqpoint{2.821610in}{3.139669in}}%
\pgfpathlineto{\pgfqpoint{2.857435in}{3.135844in}}%
\pgfpathlineto{\pgfqpoint{2.893259in}{3.130769in}}%
\pgfpathlineto{\pgfqpoint{2.929083in}{3.124696in}}%
\pgfpathlineto{\pgfqpoint{2.964908in}{3.117832in}}%
\pgfpathlineto{\pgfqpoint{3.000732in}{3.110137in}}%
\pgfpathlineto{\pgfqpoint{3.036556in}{3.101371in}}%
\pgfpathlineto{\pgfqpoint{3.072381in}{3.091321in}}%
\pgfpathlineto{\pgfqpoint{3.108205in}{3.080006in}}%
\pgfpathlineto{\pgfqpoint{3.144029in}{3.067719in}}%
\pgfpathlineto{\pgfqpoint{3.179854in}{3.054633in}}%
\pgfpathlineto{\pgfqpoint{3.215678in}{3.040678in}}%
\pgfpathlineto{\pgfqpoint{3.251502in}{3.025619in}}%
\pgfpathlineto{\pgfqpoint{3.287327in}{3.009254in}}%
\pgfpathlineto{\pgfqpoint{3.323151in}{2.991598in}}%
\pgfpathlineto{\pgfqpoint{3.358975in}{2.972963in}}%
\pgfpathlineto{\pgfqpoint{3.394800in}{2.953444in}}%
\pgfpathlineto{\pgfqpoint{3.430624in}{2.932980in}}%
\pgfpathlineto{\pgfqpoint{3.466449in}{2.911291in}}%
\pgfpathlineto{\pgfqpoint{3.502273in}{2.888193in}}%
\pgfpathlineto{\pgfqpoint{3.538097in}{2.863821in}}%
\pgfpathlineto{\pgfqpoint{3.573922in}{2.838392in}}%
\pgfpathlineto{\pgfqpoint{3.609746in}{2.812031in}}%
\pgfpathlineto{\pgfqpoint{3.645570in}{2.784637in}}%
\pgfpathlineto{\pgfqpoint{3.681395in}{2.755908in}}%
\pgfpathlineto{\pgfqpoint{3.717219in}{2.725728in}}%
\pgfpathlineto{\pgfqpoint{3.753043in}{2.694167in}}%
\pgfpathlineto{\pgfqpoint{3.788868in}{2.661450in}}%
\pgfpathlineto{\pgfqpoint{3.824692in}{2.627705in}}%
\pgfpathlineto{\pgfqpoint{3.860516in}{2.592728in}}%
\pgfpathlineto{\pgfqpoint{3.896341in}{2.556276in}}%
\pgfpathlineto{\pgfqpoint{3.932165in}{2.518226in}}%
\pgfpathlineto{\pgfqpoint{3.967989in}{2.478701in}}%
\pgfpathlineto{\pgfqpoint{4.003814in}{2.437917in}}%
\pgfpathlineto{\pgfqpoint{4.039638in}{2.395934in}}%
\pgfpathlineto{\pgfqpoint{4.075462in}{2.353126in}}%
\pgfpathlineto{\pgfqpoint{4.111287in}{2.309876in}}%
\pgfpathlineto{\pgfqpoint{4.147111in}{2.266250in}}%
\pgfpathlineto{\pgfqpoint{4.182935in}{2.222645in}}%
\pgfpathlineto{\pgfqpoint{4.218760in}{2.179539in}}%
\pgfpathlineto{\pgfqpoint{4.254584in}{2.137301in}}%
\pgfpathlineto{\pgfqpoint{4.290408in}{2.096052in}}%
\pgfpathlineto{\pgfqpoint{4.326233in}{2.055900in}}%
\pgfpathlineto{\pgfqpoint{4.362057in}{2.017132in}}%
\pgfpathlineto{\pgfqpoint{4.397881in}{1.980283in}}%
\pgfpathlineto{\pgfqpoint{4.433706in}{1.945923in}}%
\pgfpathlineto{\pgfqpoint{4.469530in}{1.914513in}}%
\pgfpathlineto{\pgfqpoint{4.505355in}{1.886235in}}%
\pgfpathlineto{\pgfqpoint{4.541179in}{1.861154in}}%
\pgfpathlineto{\pgfqpoint{4.577003in}{1.839519in}}%
\pgfpathlineto{\pgfqpoint{4.612828in}{1.821811in}}%
\pgfpathlineto{\pgfqpoint{4.648652in}{1.808501in}}%
\pgfpathlineto{\pgfqpoint{4.684476in}{1.799733in}}%
\pgfpathlineto{\pgfqpoint{4.720301in}{1.795416in}}%
\pgfusepath{stroke}%
\end{pgfscope}%
\begin{pgfscope}%
\pgfpathrectangle{\pgfqpoint{0.281664in}{0.316851in}}{\pgfqpoint{4.650000in}{3.020000in}} %
\pgfusepath{clip}%
\pgfsetrectcap%
\pgfsetroundjoin%
\pgfsetlinewidth{1.505625pt}%
\definecolor{currentstroke}{rgb}{0.000000,0.750000,0.750000}%
\pgfsetstrokecolor{currentstroke}%
\pgfsetdash{}{0pt}%
\pgfpathmoveto{\pgfqpoint{0.493028in}{0.454124in}}%
\pgfpathlineto{\pgfqpoint{0.528852in}{0.555674in}}%
\pgfpathlineto{\pgfqpoint{0.564677in}{0.653179in}}%
\pgfpathlineto{\pgfqpoint{0.600501in}{0.751316in}}%
\pgfpathlineto{\pgfqpoint{0.636325in}{0.849490in}}%
\pgfpathlineto{\pgfqpoint{0.672150in}{0.941040in}}%
\pgfpathlineto{\pgfqpoint{0.707974in}{1.027816in}}%
\pgfpathlineto{\pgfqpoint{0.743798in}{1.111564in}}%
\pgfpathlineto{\pgfqpoint{0.779623in}{1.212955in}}%
\pgfpathlineto{\pgfqpoint{0.815447in}{1.316978in}}%
\pgfpathlineto{\pgfqpoint{0.851271in}{1.423316in}}%
\pgfpathlineto{\pgfqpoint{0.887096in}{1.530652in}}%
\pgfpathlineto{\pgfqpoint{0.922920in}{1.638424in}}%
\pgfpathlineto{\pgfqpoint{0.958744in}{1.747306in}}%
\pgfpathlineto{\pgfqpoint{0.994569in}{1.858083in}}%
\pgfpathlineto{\pgfqpoint{1.030393in}{1.917194in}}%
\pgfpathlineto{\pgfqpoint{1.066217in}{1.975133in}}%
\pgfpathlineto{\pgfqpoint{1.102042in}{2.032079in}}%
\pgfpathlineto{\pgfqpoint{1.137866in}{2.086257in}}%
\pgfpathlineto{\pgfqpoint{1.173690in}{2.137001in}}%
\pgfpathlineto{\pgfqpoint{1.209515in}{2.186332in}}%
\pgfpathlineto{\pgfqpoint{1.245339in}{2.234612in}}%
\pgfpathlineto{\pgfqpoint{1.281163in}{2.282701in}}%
\pgfpathlineto{\pgfqpoint{1.316988in}{2.329757in}}%
\pgfpathlineto{\pgfqpoint{1.352812in}{2.374992in}}%
\pgfpathlineto{\pgfqpoint{1.388637in}{2.418064in}}%
\pgfpathlineto{\pgfqpoint{1.424461in}{2.459580in}}%
\pgfpathlineto{\pgfqpoint{1.460285in}{2.499381in}}%
\pgfpathlineto{\pgfqpoint{1.496110in}{2.538329in}}%
\pgfpathlineto{\pgfqpoint{1.531934in}{2.576417in}}%
\pgfpathlineto{\pgfqpoint{1.567758in}{2.612484in}}%
\pgfpathlineto{\pgfqpoint{1.603583in}{2.646395in}}%
\pgfpathlineto{\pgfqpoint{1.639407in}{2.678533in}}%
\pgfpathlineto{\pgfqpoint{1.675231in}{2.709994in}}%
\pgfpathlineto{\pgfqpoint{1.711056in}{2.741192in}}%
\pgfpathlineto{\pgfqpoint{1.746880in}{2.771533in}}%
\pgfpathlineto{\pgfqpoint{1.782704in}{2.800396in}}%
\pgfpathlineto{\pgfqpoint{1.818529in}{2.827883in}}%
\pgfpathlineto{\pgfqpoint{1.854353in}{2.853808in}}%
\pgfpathlineto{\pgfqpoint{1.890177in}{2.878512in}}%
\pgfpathlineto{\pgfqpoint{1.926002in}{2.902484in}}%
\pgfpathlineto{\pgfqpoint{1.961826in}{2.925663in}}%
\pgfpathlineto{\pgfqpoint{1.997650in}{2.947217in}}%
\pgfpathlineto{\pgfqpoint{2.033475in}{2.966720in}}%
\pgfpathlineto{\pgfqpoint{2.069299in}{2.984752in}}%
\pgfpathlineto{\pgfqpoint{2.105123in}{3.002502in}}%
\pgfpathlineto{\pgfqpoint{2.140948in}{3.019478in}}%
\pgfpathlineto{\pgfqpoint{2.176772in}{3.035790in}}%
\pgfpathlineto{\pgfqpoint{2.212596in}{3.050970in}}%
\pgfpathlineto{\pgfqpoint{2.248421in}{3.064758in}}%
\pgfpathlineto{\pgfqpoint{2.284245in}{3.077367in}}%
\pgfpathlineto{\pgfqpoint{2.320069in}{3.088899in}}%
\pgfpathlineto{\pgfqpoint{2.355894in}{3.099819in}}%
\pgfpathlineto{\pgfqpoint{2.391718in}{3.109921in}}%
\pgfpathlineto{\pgfqpoint{2.427543in}{3.118390in}}%
\pgfpathlineto{\pgfqpoint{2.463367in}{3.125048in}}%
\pgfpathlineto{\pgfqpoint{2.499191in}{3.130589in}}%
\pgfpathlineto{\pgfqpoint{2.535016in}{3.135506in}}%
\pgfpathlineto{\pgfqpoint{2.570840in}{3.139968in}}%
\pgfpathlineto{\pgfqpoint{2.606664in}{3.143502in}}%
\pgfpathlineto{\pgfqpoint{2.642489in}{3.146064in}}%
\pgfpathlineto{\pgfqpoint{2.678313in}{3.147392in}}%
\pgfpathlineto{\pgfqpoint{2.714137in}{3.147518in}}%
\pgfpathlineto{\pgfqpoint{2.749962in}{3.146937in}}%
\pgfpathlineto{\pgfqpoint{2.785786in}{3.145649in}}%
\pgfpathlineto{\pgfqpoint{2.821610in}{3.143357in}}%
\pgfpathlineto{\pgfqpoint{2.857435in}{3.139524in}}%
\pgfpathlineto{\pgfqpoint{2.893259in}{3.134017in}}%
\pgfpathlineto{\pgfqpoint{2.929083in}{3.127425in}}%
\pgfpathlineto{\pgfqpoint{2.964908in}{3.120195in}}%
\pgfpathlineto{\pgfqpoint{3.000732in}{3.112378in}}%
\pgfpathlineto{\pgfqpoint{3.036556in}{3.103574in}}%
\pgfpathlineto{\pgfqpoint{3.072381in}{3.093774in}}%
\pgfpathlineto{\pgfqpoint{3.108205in}{3.082799in}}%
\pgfpathlineto{\pgfqpoint{3.144029in}{3.070614in}}%
\pgfpathlineto{\pgfqpoint{3.179854in}{3.057736in}}%
\pgfpathlineto{\pgfqpoint{3.215678in}{3.044065in}}%
\pgfpathlineto{\pgfqpoint{3.251502in}{3.029335in}}%
\pgfpathlineto{\pgfqpoint{3.287327in}{3.012880in}}%
\pgfpathlineto{\pgfqpoint{3.323151in}{2.994910in}}%
\pgfpathlineto{\pgfqpoint{3.358975in}{2.976027in}}%
\pgfpathlineto{\pgfqpoint{3.394800in}{2.956304in}}%
\pgfpathlineto{\pgfqpoint{3.430624in}{2.935623in}}%
\pgfpathlineto{\pgfqpoint{3.466449in}{2.913964in}}%
\pgfpathlineto{\pgfqpoint{3.502273in}{2.891101in}}%
\pgfpathlineto{\pgfqpoint{3.538097in}{2.867041in}}%
\pgfpathlineto{\pgfqpoint{3.573922in}{2.841712in}}%
\pgfpathlineto{\pgfqpoint{3.609746in}{2.815474in}}%
\pgfpathlineto{\pgfqpoint{3.645570in}{2.788452in}}%
\pgfpathlineto{\pgfqpoint{3.681395in}{2.759842in}}%
\pgfpathlineto{\pgfqpoint{3.717219in}{2.729615in}}%
\pgfpathlineto{\pgfqpoint{3.753043in}{2.697980in}}%
\pgfpathlineto{\pgfqpoint{3.788868in}{2.665158in}}%
\pgfpathlineto{\pgfqpoint{3.824692in}{2.631264in}}%
\pgfpathlineto{\pgfqpoint{3.860516in}{2.596086in}}%
\pgfpathlineto{\pgfqpoint{3.896341in}{2.559664in}}%
\pgfpathlineto{\pgfqpoint{3.932165in}{2.521852in}}%
\pgfpathlineto{\pgfqpoint{3.967989in}{2.482493in}}%
\pgfpathlineto{\pgfqpoint{4.003814in}{2.441948in}}%
\pgfpathlineto{\pgfqpoint{4.039638in}{2.400208in}}%
\pgfpathlineto{\pgfqpoint{4.075462in}{2.357147in}}%
\pgfpathlineto{\pgfqpoint{4.111287in}{2.312283in}}%
\pgfpathlineto{\pgfqpoint{4.147111in}{2.265552in}}%
\pgfpathlineto{\pgfqpoint{4.182935in}{2.217098in}}%
\pgfpathlineto{\pgfqpoint{4.218760in}{2.167259in}}%
\pgfpathlineto{\pgfqpoint{4.254584in}{2.049469in}}%
\pgfpathlineto{\pgfqpoint{4.290408in}{1.933389in}}%
\pgfpathlineto{\pgfqpoint{4.326233in}{1.818908in}}%
\pgfpathlineto{\pgfqpoint{4.362057in}{1.705236in}}%
\pgfpathlineto{\pgfqpoint{4.397881in}{1.605093in}}%
\pgfpathlineto{\pgfqpoint{4.433706in}{1.531815in}}%
\pgfpathlineto{\pgfqpoint{4.469530in}{1.459679in}}%
\pgfpathlineto{\pgfqpoint{4.505355in}{1.387306in}}%
\pgfpathlineto{\pgfqpoint{4.541179in}{1.308862in}}%
\pgfpathlineto{\pgfqpoint{4.577003in}{1.220874in}}%
\pgfpathlineto{\pgfqpoint{4.612828in}{1.127160in}}%
\pgfpathlineto{\pgfqpoint{4.648652in}{1.034385in}}%
\pgfpathlineto{\pgfqpoint{4.684476in}{0.942463in}}%
\pgfpathlineto{\pgfqpoint{4.720301in}{0.845996in}}%
\pgfusepath{stroke}%
\end{pgfscope}%
\begin{pgfscope}%
\pgfpathrectangle{\pgfqpoint{0.281664in}{0.316851in}}{\pgfqpoint{4.650000in}{3.020000in}} %
\pgfusepath{clip}%
\pgfsetrectcap%
\pgfsetroundjoin%
\pgfsetlinewidth{1.505625pt}%
\definecolor{currentstroke}{rgb}{0.000000,0.000000,1.000000}%
\pgfsetstrokecolor{currentstroke}%
\pgfsetdash{}{0pt}%
\pgfpathmoveto{\pgfqpoint{0.493028in}{0.927729in}}%
\pgfpathlineto{\pgfqpoint{0.528852in}{1.008111in}}%
\pgfpathlineto{\pgfqpoint{0.564677in}{1.087668in}}%
\pgfpathlineto{\pgfqpoint{0.600501in}{1.166130in}}%
\pgfpathlineto{\pgfqpoint{0.636325in}{1.243281in}}%
\pgfpathlineto{\pgfqpoint{0.672150in}{1.318952in}}%
\pgfpathlineto{\pgfqpoint{0.707974in}{1.393017in}}%
\pgfpathlineto{\pgfqpoint{0.743798in}{1.465383in}}%
\pgfpathlineto{\pgfqpoint{0.779623in}{1.535986in}}%
\pgfpathlineto{\pgfqpoint{0.815447in}{1.604786in}}%
\pgfpathlineto{\pgfqpoint{0.851271in}{1.671758in}}%
\pgfpathlineto{\pgfqpoint{0.887096in}{1.736889in}}%
\pgfpathlineto{\pgfqpoint{0.922920in}{1.800178in}}%
\pgfpathlineto{\pgfqpoint{0.958744in}{1.861631in}}%
\pgfpathlineto{\pgfqpoint{0.994569in}{1.921261in}}%
\pgfpathlineto{\pgfqpoint{1.030393in}{1.979090in}}%
\pgfpathlineto{\pgfqpoint{1.066217in}{2.035147in}}%
\pgfpathlineto{\pgfqpoint{1.102042in}{2.089466in}}%
\pgfpathlineto{\pgfqpoint{1.137866in}{2.142087in}}%
\pgfpathlineto{\pgfqpoint{1.173690in}{2.193055in}}%
\pgfpathlineto{\pgfqpoint{1.209515in}{2.242417in}}%
\pgfpathlineto{\pgfqpoint{1.245339in}{2.290219in}}%
\pgfpathlineto{\pgfqpoint{1.281163in}{2.336506in}}%
\pgfpathlineto{\pgfqpoint{1.316988in}{2.381316in}}%
\pgfpathlineto{\pgfqpoint{1.352812in}{2.424687in}}%
\pgfpathlineto{\pgfqpoint{1.388637in}{2.466648in}}%
\pgfpathlineto{\pgfqpoint{1.424461in}{2.507228in}}%
\pgfpathlineto{\pgfqpoint{1.460285in}{2.546450in}}%
\pgfpathlineto{\pgfqpoint{1.496110in}{2.584338in}}%
\pgfpathlineto{\pgfqpoint{1.531934in}{2.620913in}}%
\pgfpathlineto{\pgfqpoint{1.567758in}{2.656196in}}%
\pgfpathlineto{\pgfqpoint{1.603583in}{2.690209in}}%
\pgfpathlineto{\pgfqpoint{1.639407in}{2.722973in}}%
\pgfpathlineto{\pgfqpoint{1.675231in}{2.754507in}}%
\pgfpathlineto{\pgfqpoint{1.711056in}{2.784830in}}%
\pgfpathlineto{\pgfqpoint{1.746880in}{2.813955in}}%
\pgfpathlineto{\pgfqpoint{1.782704in}{2.841895in}}%
\pgfpathlineto{\pgfqpoint{1.818529in}{2.868659in}}%
\pgfpathlineto{\pgfqpoint{1.854353in}{2.894256in}}%
\pgfpathlineto{\pgfqpoint{1.890177in}{2.918693in}}%
\pgfpathlineto{\pgfqpoint{1.926002in}{2.941979in}}%
\pgfpathlineto{\pgfqpoint{1.961826in}{2.964121in}}%
\pgfpathlineto{\pgfqpoint{1.997650in}{2.985132in}}%
\pgfpathlineto{\pgfqpoint{2.033475in}{3.005024in}}%
\pgfpathlineto{\pgfqpoint{2.069299in}{3.023811in}}%
\pgfpathlineto{\pgfqpoint{2.105123in}{3.041507in}}%
\pgfpathlineto{\pgfqpoint{2.140948in}{3.058126in}}%
\pgfpathlineto{\pgfqpoint{2.176772in}{3.073679in}}%
\pgfpathlineto{\pgfqpoint{2.212596in}{3.088177in}}%
\pgfpathlineto{\pgfqpoint{2.248421in}{3.101624in}}%
\pgfpathlineto{\pgfqpoint{2.284245in}{3.114026in}}%
\pgfpathlineto{\pgfqpoint{2.320069in}{3.125386in}}%
\pgfpathlineto{\pgfqpoint{2.355894in}{3.135704in}}%
\pgfpathlineto{\pgfqpoint{2.391718in}{3.144984in}}%
\pgfpathlineto{\pgfqpoint{2.427543in}{3.153229in}}%
\pgfpathlineto{\pgfqpoint{2.463367in}{3.160442in}}%
\pgfpathlineto{\pgfqpoint{2.499191in}{3.166631in}}%
\pgfpathlineto{\pgfqpoint{2.535016in}{3.171801in}}%
\pgfpathlineto{\pgfqpoint{2.570840in}{3.175961in}}%
\pgfpathlineto{\pgfqpoint{2.606664in}{3.179117in}}%
\pgfpathlineto{\pgfqpoint{2.642489in}{3.181274in}}%
\pgfpathlineto{\pgfqpoint{2.678313in}{3.182437in}}%
\pgfpathlineto{\pgfqpoint{2.714137in}{3.182609in}}%
\pgfpathlineto{\pgfqpoint{2.749962in}{3.181793in}}%
\pgfpathlineto{\pgfqpoint{2.785786in}{3.179991in}}%
\pgfpathlineto{\pgfqpoint{2.821610in}{3.177205in}}%
\pgfpathlineto{\pgfqpoint{2.857435in}{3.173436in}}%
\pgfpathlineto{\pgfqpoint{2.893259in}{3.168688in}}%
\pgfpathlineto{\pgfqpoint{2.929083in}{3.162961in}}%
\pgfpathlineto{\pgfqpoint{2.964908in}{3.156256in}}%
\pgfpathlineto{\pgfqpoint{3.000732in}{3.148571in}}%
\pgfpathlineto{\pgfqpoint{3.036556in}{3.139901in}}%
\pgfpathlineto{\pgfqpoint{3.072381in}{3.130236in}}%
\pgfpathlineto{\pgfqpoint{3.108205in}{3.119561in}}%
\pgfpathlineto{\pgfqpoint{3.144029in}{3.107860in}}%
\pgfpathlineto{\pgfqpoint{3.179854in}{3.095111in}}%
\pgfpathlineto{\pgfqpoint{3.215678in}{3.081292in}}%
\pgfpathlineto{\pgfqpoint{3.251502in}{3.066377in}}%
\pgfpathlineto{\pgfqpoint{3.287327in}{3.050345in}}%
\pgfpathlineto{\pgfqpoint{3.323151in}{3.033175in}}%
\pgfpathlineto{\pgfqpoint{3.358975in}{3.014849in}}%
\pgfpathlineto{\pgfqpoint{3.394800in}{2.995355in}}%
\pgfpathlineto{\pgfqpoint{3.430624in}{2.974684in}}%
\pgfpathlineto{\pgfqpoint{3.466449in}{2.952835in}}%
\pgfpathlineto{\pgfqpoint{3.502273in}{2.929812in}}%
\pgfpathlineto{\pgfqpoint{3.538097in}{2.905625in}}%
\pgfpathlineto{\pgfqpoint{3.573922in}{2.880292in}}%
\pgfpathlineto{\pgfqpoint{3.609746in}{2.853840in}}%
\pgfpathlineto{\pgfqpoint{3.645570in}{2.826297in}}%
\pgfpathlineto{\pgfqpoint{3.681395in}{2.797701in}}%
\pgfpathlineto{\pgfqpoint{3.717219in}{2.768088in}}%
\pgfpathlineto{\pgfqpoint{3.753043in}{2.737497in}}%
\pgfpathlineto{\pgfqpoint{3.788868in}{2.705957in}}%
\pgfpathlineto{\pgfqpoint{3.824692in}{2.673491in}}%
\pgfpathlineto{\pgfqpoint{3.860516in}{2.640100in}}%
\pgfpathlineto{\pgfqpoint{3.896341in}{2.605764in}}%
\pgfpathlineto{\pgfqpoint{3.932165in}{2.570437in}}%
\pgfpathlineto{\pgfqpoint{3.967989in}{2.534037in}}%
\pgfpathlineto{\pgfqpoint{4.003814in}{2.496449in}}%
\pgfpathlineto{\pgfqpoint{4.039638in}{2.457524in}}%
\pgfpathlineto{\pgfqpoint{4.075462in}{2.417077in}}%
\pgfpathlineto{\pgfqpoint{4.111287in}{2.374896in}}%
\pgfpathlineto{\pgfqpoint{4.147111in}{2.330754in}}%
\pgfpathlineto{\pgfqpoint{4.182935in}{2.284413in}}%
\pgfpathlineto{\pgfqpoint{4.218760in}{2.235646in}}%
\pgfpathlineto{\pgfqpoint{4.254584in}{2.184250in}}%
\pgfpathlineto{\pgfqpoint{4.290408in}{2.130071in}}%
\pgfpathlineto{\pgfqpoint{4.326233in}{2.073020in}}%
\pgfpathlineto{\pgfqpoint{4.362057in}{2.013097in}}%
\pgfpathlineto{\pgfqpoint{4.397881in}{1.950410in}}%
\pgfpathlineto{\pgfqpoint{4.433706in}{1.885193in}}%
\pgfpathlineto{\pgfqpoint{4.469530in}{1.817816in}}%
\pgfpathlineto{\pgfqpoint{4.505355in}{1.748793in}}%
\pgfpathlineto{\pgfqpoint{4.541179in}{1.678780in}}%
\pgfpathlineto{\pgfqpoint{4.577003in}{1.608564in}}%
\pgfpathlineto{\pgfqpoint{4.612828in}{1.539041in}}%
\pgfpathlineto{\pgfqpoint{4.648652in}{1.471186in}}%
\pgfpathlineto{\pgfqpoint{4.684476in}{1.406014in}}%
\pgfpathlineto{\pgfqpoint{4.720301in}{1.344527in}}%
\pgfusepath{stroke}%
\end{pgfscope}%
\begin{pgfscope}%
\pgfpathrectangle{\pgfqpoint{0.281664in}{0.316851in}}{\pgfqpoint{4.650000in}{3.020000in}} %
\pgfusepath{clip}%
\pgfsetrectcap%
\pgfsetroundjoin%
\pgfsetlinewidth{1.505625pt}%
\definecolor{currentstroke}{rgb}{1.000000,0.000000,0.000000}%
\pgfsetstrokecolor{currentstroke}%
\pgfsetdash{}{0pt}%
\pgfpathmoveto{\pgfqpoint{0.493028in}{0.912686in}}%
\pgfpathlineto{\pgfqpoint{0.528852in}{0.997437in}}%
\pgfpathlineto{\pgfqpoint{0.564677in}{1.080130in}}%
\pgfpathlineto{\pgfqpoint{0.600501in}{1.160788in}}%
\pgfpathlineto{\pgfqpoint{0.636325in}{1.239433in}}%
\pgfpathlineto{\pgfqpoint{0.672150in}{1.316089in}}%
\pgfpathlineto{\pgfqpoint{0.707974in}{1.390779in}}%
\pgfpathlineto{\pgfqpoint{0.743798in}{1.463525in}}%
\pgfpathlineto{\pgfqpoint{0.779623in}{1.534351in}}%
\pgfpathlineto{\pgfqpoint{0.815447in}{1.603279in}}%
\pgfpathlineto{\pgfqpoint{0.851271in}{1.670332in}}%
\pgfpathlineto{\pgfqpoint{0.887096in}{1.735533in}}%
\pgfpathlineto{\pgfqpoint{0.922920in}{1.798906in}}%
\pgfpathlineto{\pgfqpoint{0.958744in}{1.860460in}}%
\pgfpathlineto{\pgfqpoint{0.994569in}{1.920219in}}%
\pgfpathlineto{\pgfqpoint{1.030393in}{1.978208in}}%
\pgfpathlineto{\pgfqpoint{1.066217in}{2.034454in}}%
\pgfpathlineto{\pgfqpoint{1.102042in}{2.088986in}}%
\pgfpathlineto{\pgfqpoint{1.137866in}{2.141832in}}%
\pgfpathlineto{\pgfqpoint{1.173690in}{2.193022in}}%
\pgfpathlineto{\pgfqpoint{1.209515in}{2.242589in}}%
\pgfpathlineto{\pgfqpoint{1.245339in}{2.290569in}}%
\pgfpathlineto{\pgfqpoint{1.281163in}{2.337001in}}%
\pgfpathlineto{\pgfqpoint{1.316988in}{2.381921in}}%
\pgfpathlineto{\pgfqpoint{1.352812in}{2.425365in}}%
\pgfpathlineto{\pgfqpoint{1.388637in}{2.467368in}}%
\pgfpathlineto{\pgfqpoint{1.424461in}{2.507961in}}%
\pgfpathlineto{\pgfqpoint{1.460285in}{2.547175in}}%
\pgfpathlineto{\pgfqpoint{1.496110in}{2.585041in}}%
\pgfpathlineto{\pgfqpoint{1.531934in}{2.621584in}}%
\pgfpathlineto{\pgfqpoint{1.567758in}{2.656828in}}%
\pgfpathlineto{\pgfqpoint{1.603583in}{2.690795in}}%
\pgfpathlineto{\pgfqpoint{1.639407in}{2.723508in}}%
\pgfpathlineto{\pgfqpoint{1.675231in}{2.754989in}}%
\pgfpathlineto{\pgfqpoint{1.711056in}{2.785258in}}%
\pgfpathlineto{\pgfqpoint{1.746880in}{2.814331in}}%
\pgfpathlineto{\pgfqpoint{1.782704in}{2.842218in}}%
\pgfpathlineto{\pgfqpoint{1.818529in}{2.868930in}}%
\pgfpathlineto{\pgfqpoint{1.854353in}{2.894478in}}%
\pgfpathlineto{\pgfqpoint{1.890177in}{2.918875in}}%
\pgfpathlineto{\pgfqpoint{1.926002in}{2.942133in}}%
\pgfpathlineto{\pgfqpoint{1.961826in}{2.964259in}}%
\pgfpathlineto{\pgfqpoint{1.997650in}{2.985260in}}%
\pgfpathlineto{\pgfqpoint{2.033475in}{3.005147in}}%
\pgfpathlineto{\pgfqpoint{2.069299in}{3.023932in}}%
\pgfpathlineto{\pgfqpoint{2.105123in}{3.041630in}}%
\pgfpathlineto{\pgfqpoint{2.140948in}{3.058252in}}%
\pgfpathlineto{\pgfqpoint{2.176772in}{3.073807in}}%
\pgfpathlineto{\pgfqpoint{2.212596in}{3.088299in}}%
\pgfpathlineto{\pgfqpoint{2.248421in}{3.101734in}}%
\pgfpathlineto{\pgfqpoint{2.284245in}{3.114122in}}%
\pgfpathlineto{\pgfqpoint{2.320069in}{3.125470in}}%
\pgfpathlineto{\pgfqpoint{2.355894in}{3.135784in}}%
\pgfpathlineto{\pgfqpoint{2.391718in}{3.145068in}}%
\pgfpathlineto{\pgfqpoint{2.427543in}{3.153323in}}%
\pgfpathlineto{\pgfqpoint{2.463367in}{3.160556in}}%
\pgfpathlineto{\pgfqpoint{2.499191in}{3.166774in}}%
\pgfpathlineto{\pgfqpoint{2.535016in}{3.171986in}}%
\pgfpathlineto{\pgfqpoint{2.570840in}{3.176196in}}%
\pgfpathlineto{\pgfqpoint{2.606664in}{3.179404in}}%
\pgfpathlineto{\pgfqpoint{2.642489in}{3.181610in}}%
\pgfpathlineto{\pgfqpoint{2.678313in}{3.182815in}}%
\pgfpathlineto{\pgfqpoint{2.714137in}{3.183023in}}%
\pgfpathlineto{\pgfqpoint{2.749962in}{3.182235in}}%
\pgfpathlineto{\pgfqpoint{2.785786in}{3.180450in}}%
\pgfpathlineto{\pgfqpoint{2.821610in}{3.177664in}}%
\pgfpathlineto{\pgfqpoint{2.857435in}{3.173874in}}%
\pgfpathlineto{\pgfqpoint{2.893259in}{3.169080in}}%
\pgfpathlineto{\pgfqpoint{2.929083in}{3.163285in}}%
\pgfpathlineto{\pgfqpoint{2.964908in}{3.156489in}}%
\pgfpathlineto{\pgfqpoint{3.000732in}{3.148691in}}%
\pgfpathlineto{\pgfqpoint{3.036556in}{3.139884in}}%
\pgfpathlineto{\pgfqpoint{3.072381in}{3.130063in}}%
\pgfpathlineto{\pgfqpoint{3.108205in}{3.119222in}}%
\pgfpathlineto{\pgfqpoint{3.144029in}{3.107359in}}%
\pgfpathlineto{\pgfqpoint{3.179854in}{3.094468in}}%
\pgfpathlineto{\pgfqpoint{3.215678in}{3.080540in}}%
\pgfpathlineto{\pgfqpoint{3.251502in}{3.065567in}}%
\pgfpathlineto{\pgfqpoint{3.287327in}{3.049541in}}%
\pgfpathlineto{\pgfqpoint{3.323151in}{3.032456in}}%
\pgfpathlineto{\pgfqpoint{3.358975in}{3.014309in}}%
\pgfpathlineto{\pgfqpoint{3.394800in}{2.995095in}}%
\pgfpathlineto{\pgfqpoint{3.430624in}{2.974805in}}%
\pgfpathlineto{\pgfqpoint{3.466449in}{2.953425in}}%
\pgfpathlineto{\pgfqpoint{3.502273in}{2.930943in}}%
\pgfpathlineto{\pgfqpoint{3.538097in}{2.907348in}}%
\pgfpathlineto{\pgfqpoint{3.573922in}{2.882629in}}%
\pgfpathlineto{\pgfqpoint{3.609746in}{2.856770in}}%
\pgfpathlineto{\pgfqpoint{3.645570in}{2.829754in}}%
\pgfpathlineto{\pgfqpoint{3.681395in}{2.801562in}}%
\pgfpathlineto{\pgfqpoint{3.717219in}{2.772175in}}%
\pgfpathlineto{\pgfqpoint{3.753043in}{2.741578in}}%
\pgfpathlineto{\pgfqpoint{3.788868in}{2.709755in}}%
\pgfpathlineto{\pgfqpoint{3.824692in}{2.676684in}}%
\pgfpathlineto{\pgfqpoint{3.860516in}{2.642342in}}%
\pgfpathlineto{\pgfqpoint{3.896341in}{2.606701in}}%
\pgfpathlineto{\pgfqpoint{3.932165in}{2.569736in}}%
\pgfpathlineto{\pgfqpoint{3.967989in}{2.531421in}}%
\pgfpathlineto{\pgfqpoint{4.003814in}{2.491729in}}%
\pgfpathlineto{\pgfqpoint{4.039638in}{2.450632in}}%
\pgfpathlineto{\pgfqpoint{4.075462in}{2.408099in}}%
\pgfpathlineto{\pgfqpoint{4.111287in}{2.364101in}}%
\pgfpathlineto{\pgfqpoint{4.147111in}{2.318611in}}%
\pgfpathlineto{\pgfqpoint{4.182935in}{2.271601in}}%
\pgfpathlineto{\pgfqpoint{4.218760in}{2.223044in}}%
\pgfpathlineto{\pgfqpoint{4.254584in}{2.172910in}}%
\pgfpathlineto{\pgfqpoint{4.290408in}{2.121167in}}%
\pgfpathlineto{\pgfqpoint{4.326233in}{2.067786in}}%
\pgfpathlineto{\pgfqpoint{4.362057in}{2.012737in}}%
\pgfpathlineto{\pgfqpoint{4.397881in}{1.955991in}}%
\pgfpathlineto{\pgfqpoint{4.433706in}{1.897519in}}%
\pgfpathlineto{\pgfqpoint{4.469530in}{1.837292in}}%
\pgfpathlineto{\pgfqpoint{4.505355in}{1.775282in}}%
\pgfpathlineto{\pgfqpoint{4.541179in}{1.711458in}}%
\pgfpathlineto{\pgfqpoint{4.577003in}{1.645791in}}%
\pgfpathlineto{\pgfqpoint{4.612828in}{1.578253in}}%
\pgfpathlineto{\pgfqpoint{4.648652in}{1.508814in}}%
\pgfpathlineto{\pgfqpoint{4.684476in}{1.437446in}}%
\pgfpathlineto{\pgfqpoint{4.720301in}{1.364119in}}%
\pgfusepath{stroke}%
\end{pgfscope}%
\begin{pgfscope}%
\pgfsetrectcap%
\pgfsetmiterjoin%
\pgfsetlinewidth{0.803000pt}%
\definecolor{currentstroke}{rgb}{0.000000,0.000000,0.000000}%
\pgfsetstrokecolor{currentstroke}%
\pgfsetdash{}{0pt}%
\pgfpathmoveto{\pgfqpoint{0.281664in}{0.316851in}}%
\pgfpathlineto{\pgfqpoint{0.281664in}{3.336851in}}%
\pgfusepath{stroke}%
\end{pgfscope}%
\begin{pgfscope}%
\pgfsetrectcap%
\pgfsetmiterjoin%
\pgfsetlinewidth{0.803000pt}%
\definecolor{currentstroke}{rgb}{0.000000,0.000000,0.000000}%
\pgfsetstrokecolor{currentstroke}%
\pgfsetdash{}{0pt}%
\pgfpathmoveto{\pgfqpoint{4.931664in}{0.316851in}}%
\pgfpathlineto{\pgfqpoint{4.931664in}{3.336851in}}%
\pgfusepath{stroke}%
\end{pgfscope}%
\begin{pgfscope}%
\pgfsetrectcap%
\pgfsetmiterjoin%
\pgfsetlinewidth{0.803000pt}%
\definecolor{currentstroke}{rgb}{0.000000,0.000000,0.000000}%
\pgfsetstrokecolor{currentstroke}%
\pgfsetdash{}{0pt}%
\pgfpathmoveto{\pgfqpoint{0.281664in}{0.316851in}}%
\pgfpathlineto{\pgfqpoint{4.931664in}{0.316851in}}%
\pgfusepath{stroke}%
\end{pgfscope}%
\begin{pgfscope}%
\pgfsetrectcap%
\pgfsetmiterjoin%
\pgfsetlinewidth{0.803000pt}%
\definecolor{currentstroke}{rgb}{0.000000,0.000000,0.000000}%
\pgfsetstrokecolor{currentstroke}%
\pgfsetdash{}{0pt}%
\pgfpathmoveto{\pgfqpoint{0.281664in}{3.336851in}}%
\pgfpathlineto{\pgfqpoint{4.931664in}{3.336851in}}%
\pgfusepath{stroke}%
\end{pgfscope}%
\begin{pgfscope}%
\pgfsetbuttcap%
\pgfsetmiterjoin%
\definecolor{currentfill}{rgb}{1.000000,1.000000,1.000000}%
\pgfsetfillcolor{currentfill}%
\pgfsetfillopacity{0.800000}%
\pgfsetlinewidth{1.003750pt}%
\definecolor{currentstroke}{rgb}{0.800000,0.800000,0.800000}%
\pgfsetstrokecolor{currentstroke}%
\pgfsetstrokeopacity{0.800000}%
\pgfsetdash{}{0pt}%
\pgfpathmoveto{\pgfqpoint{1.509164in}{0.386295in}}%
\pgfpathlineto{\pgfqpoint{3.704165in}{0.386295in}}%
\pgfpathquadraticcurveto{\pgfqpoint{3.731942in}{0.386295in}}{\pgfqpoint{3.731942in}{0.414073in}}%
\pgfpathlineto{\pgfqpoint{3.731942in}{1.423404in}}%
\pgfpathquadraticcurveto{\pgfqpoint{3.731942in}{1.451182in}}{\pgfqpoint{3.704165in}{1.451182in}}%
\pgfpathlineto{\pgfqpoint{1.509164in}{1.451182in}}%
\pgfpathquadraticcurveto{\pgfqpoint{1.481386in}{1.451182in}}{\pgfqpoint{1.481386in}{1.423404in}}%
\pgfpathlineto{\pgfqpoint{1.481386in}{0.414073in}}%
\pgfpathquadraticcurveto{\pgfqpoint{1.481386in}{0.386295in}}{\pgfqpoint{1.509164in}{0.386295in}}%
\pgfpathclose%
\pgfusepath{stroke,fill}%
\end{pgfscope}%
\begin{pgfscope}%
\pgfsetbuttcap%
\pgfsetroundjoin%
\definecolor{currentfill}{rgb}{0.000000,0.000000,0.000000}%
\pgfsetfillcolor{currentfill}%
\pgfsetlinewidth{1.003750pt}%
\definecolor{currentstroke}{rgb}{0.000000,0.000000,0.000000}%
\pgfsetstrokecolor{currentstroke}%
\pgfsetdash{}{0pt}%
\pgfsys@defobject{currentmarker}{\pgfqpoint{-0.020833in}{-0.020833in}}{\pgfqpoint{0.020833in}{0.020833in}}{%
\pgfpathmoveto{\pgfqpoint{0.000000in}{-0.020833in}}%
\pgfpathcurveto{\pgfqpoint{0.005525in}{-0.020833in}}{\pgfqpoint{0.010825in}{-0.018638in}}{\pgfqpoint{0.014731in}{-0.014731in}}%
\pgfpathcurveto{\pgfqpoint{0.018638in}{-0.010825in}}{\pgfqpoint{0.020833in}{-0.005525in}}{\pgfqpoint{0.020833in}{0.000000in}}%
\pgfpathcurveto{\pgfqpoint{0.020833in}{0.005525in}}{\pgfqpoint{0.018638in}{0.010825in}}{\pgfqpoint{0.014731in}{0.014731in}}%
\pgfpathcurveto{\pgfqpoint{0.010825in}{0.018638in}}{\pgfqpoint{0.005525in}{0.020833in}}{\pgfqpoint{0.000000in}{0.020833in}}%
\pgfpathcurveto{\pgfqpoint{-0.005525in}{0.020833in}}{\pgfqpoint{-0.010825in}{0.018638in}}{\pgfqpoint{-0.014731in}{0.014731in}}%
\pgfpathcurveto{\pgfqpoint{-0.018638in}{0.010825in}}{\pgfqpoint{-0.020833in}{0.005525in}}{\pgfqpoint{-0.020833in}{0.000000in}}%
\pgfpathcurveto{\pgfqpoint{-0.020833in}{-0.005525in}}{\pgfqpoint{-0.018638in}{-0.010825in}}{\pgfqpoint{-0.014731in}{-0.014731in}}%
\pgfpathcurveto{\pgfqpoint{-0.010825in}{-0.018638in}}{\pgfqpoint{-0.005525in}{-0.020833in}}{\pgfqpoint{0.000000in}{-0.020833in}}%
\pgfpathclose%
\pgfusepath{stroke,fill}%
}%
\begin{pgfscope}%
\pgfsys@transformshift{1.675831in}{1.338714in}%
\pgfsys@useobject{currentmarker}{}%
\end{pgfscope}%
\end{pgfscope}%
\begin{pgfscope}%
\pgftext[x=1.925831in,y=1.290103in,left,base]{\rmfamily\fontsize{10.000000}{12.000000}\selectfont Raw signal}%
\end{pgfscope}%
\begin{pgfscope}%
\pgfsetrectcap%
\pgfsetroundjoin%
\pgfsetlinewidth{1.505625pt}%
\definecolor{currentstroke}{rgb}{0.750000,0.000000,0.750000}%
\pgfsetstrokecolor{currentstroke}%
\pgfsetdash{}{0pt}%
\pgfpathmoveto{\pgfqpoint{1.536942in}{1.132890in}}%
\pgfpathlineto{\pgfqpoint{1.814720in}{1.132890in}}%
\pgfusepath{stroke}%
\end{pgfscope}%
\begin{pgfscope}%
\pgftext[x=1.925831in,y=1.084279in,left,base]{\rmfamily\fontsize{10.000000}{12.000000}\selectfont 1D Gaussian convolution}%
\end{pgfscope}%
\begin{pgfscope}%
\pgfsetrectcap%
\pgfsetroundjoin%
\pgfsetlinewidth{1.505625pt}%
\definecolor{currentstroke}{rgb}{0.000000,0.750000,0.750000}%
\pgfsetstrokecolor{currentstroke}%
\pgfsetdash{}{0pt}%
\pgfpathmoveto{\pgfqpoint{1.536942in}{0.929033in}}%
\pgfpathlineto{\pgfqpoint{1.814720in}{0.929033in}}%
\pgfusepath{stroke}%
\end{pgfscope}%
\begin{pgfscope}%
\pgftext[x=1.925831in,y=0.880422in,left,base]{\rmfamily\fontsize{10.000000}{12.000000}\selectfont Wiener}%
\end{pgfscope}%
\begin{pgfscope}%
\pgfsetrectcap%
\pgfsetroundjoin%
\pgfsetlinewidth{1.505625pt}%
\definecolor{currentstroke}{rgb}{0.000000,0.000000,1.000000}%
\pgfsetstrokecolor{currentstroke}%
\pgfsetdash{}{0pt}%
\pgfpathmoveto{\pgfqpoint{1.536942in}{0.725176in}}%
\pgfpathlineto{\pgfqpoint{1.814720in}{0.725176in}}%
\pgfusepath{stroke}%
\end{pgfscope}%
\begin{pgfscope}%
\pgftext[x=1.925831in,y=0.676565in,left,base]{\rmfamily\fontsize{10.000000}{12.000000}\selectfont Butterworth}%
\end{pgfscope}%
\begin{pgfscope}%
\pgfsetrectcap%
\pgfsetroundjoin%
\pgfsetlinewidth{1.505625pt}%
\definecolor{currentstroke}{rgb}{1.000000,0.000000,0.000000}%
\pgfsetstrokecolor{currentstroke}%
\pgfsetdash{}{0pt}%
\pgfpathmoveto{\pgfqpoint{1.536942in}{0.521318in}}%
\pgfpathlineto{\pgfqpoint{1.814720in}{0.521318in}}%
\pgfusepath{stroke}%
\end{pgfscope}%
\begin{pgfscope}%
\pgftext[x=1.925831in,y=0.472707in,left,base]{\rmfamily\fontsize{10.000000}{12.000000}\selectfont Savitsky-Golay}%
\end{pgfscope}%
\end{pgfpicture}%
\makeatother%
\endgroup%

       \caption{A simple EMA plot.\label{fig:ema2}}
\end{figure}

\begin{figure}
    \centering
    %% Creator: Matplotlib, PGF backend
%%
%% To include the figure in your LaTeX document, write
%%   \input{<filename>.pgf}
%%
%% Make sure the required packages are loaded in your preamble
%%   \usepackage{pgf}
%%
%% Figures using additional raster images can only be included by \input if
%% they are in the same directory as the main LaTeX file. For loading figures
%% from other directories you can use the `import` package
%%   \usepackage{import}
%% and then include the figures with
%%   \import{<path to file>}{<filename>.pgf}
%%
%% Matplotlib used the following preamble
%%   \usepackage{fontspec}
%%   \setmainfont{DejaVu Serif}
%%   \setsansfont{DejaVu Sans}
%%   \setmonofont{DejaVu Sans Mono}
%%
\begingroup%
\makeatletter%
\begin{pgfpicture}%
\pgfpathrectangle{\pgfpointorigin}{\pgfqpoint{5.186840in}{3.471851in}}%
\pgfusepath{use as bounding box, clip}%
\begin{pgfscope}%
\pgfsetbuttcap%
\pgfsetmiterjoin%
\definecolor{currentfill}{rgb}{1.000000,1.000000,1.000000}%
\pgfsetfillcolor{currentfill}%
\pgfsetlinewidth{0.000000pt}%
\definecolor{currentstroke}{rgb}{1.000000,1.000000,1.000000}%
\pgfsetstrokecolor{currentstroke}%
\pgfsetdash{}{0pt}%
\pgfpathmoveto{\pgfqpoint{0.000000in}{0.000000in}}%
\pgfpathlineto{\pgfqpoint{5.186840in}{0.000000in}}%
\pgfpathlineto{\pgfqpoint{5.186840in}{3.471851in}}%
\pgfpathlineto{\pgfqpoint{0.000000in}{3.471851in}}%
\pgfpathclose%
\pgfusepath{fill}%
\end{pgfscope}%
\begin{pgfscope}%
\pgfsetbuttcap%
\pgfsetmiterjoin%
\definecolor{currentfill}{rgb}{1.000000,1.000000,1.000000}%
\pgfsetfillcolor{currentfill}%
\pgfsetlinewidth{0.000000pt}%
\definecolor{currentstroke}{rgb}{0.000000,0.000000,0.000000}%
\pgfsetstrokecolor{currentstroke}%
\pgfsetstrokeopacity{0.000000}%
\pgfsetdash{}{0pt}%
\pgfpathmoveto{\pgfqpoint{0.401840in}{0.316851in}}%
\pgfpathlineto{\pgfqpoint{5.051840in}{0.316851in}}%
\pgfpathlineto{\pgfqpoint{5.051840in}{3.336851in}}%
\pgfpathlineto{\pgfqpoint{0.401840in}{3.336851in}}%
\pgfpathclose%
\pgfusepath{fill}%
\end{pgfscope}%
\begin{pgfscope}%
\pgftext[x=2.726840in,y=0.261295in,,top]{\rmfamily\fontsize{12.000000}{14.400000}\selectfont \(\displaystyle t\)}%
\end{pgfscope}%
\begin{pgfscope}%
\pgftext[x=0.262951in,y=1.826851in,,bottom,rotate=90.000000]{\rmfamily\fontsize{12.000000}{14.400000}\selectfont \(\displaystyle y'\)}%
\end{pgfscope}%
\begin{pgfscope}%
\pgfpathrectangle{\pgfqpoint{0.401840in}{0.316851in}}{\pgfqpoint{4.650000in}{3.020000in}} %
\pgfusepath{clip}%
\pgfsetbuttcap%
\pgfsetroundjoin%
\pgfsetlinewidth{1.505625pt}%
\definecolor{currentstroke}{rgb}{1.000000,0.000000,0.000000}%
\pgfsetstrokecolor{currentstroke}%
\pgfsetstrokeopacity{0.700000}%
\pgfsetdash{{5.550000pt}{2.400000pt}}{0.000000pt}%
\pgfpathmoveto{\pgfqpoint{0.613203in}{3.199578in}}%
\pgfpathlineto{\pgfqpoint{0.649028in}{3.163846in}}%
\pgfpathlineto{\pgfqpoint{0.684852in}{3.128531in}}%
\pgfpathlineto{\pgfqpoint{0.720676in}{3.093634in}}%
\pgfpathlineto{\pgfqpoint{0.756501in}{3.059154in}}%
\pgfpathlineto{\pgfqpoint{0.792325in}{3.025091in}}%
\pgfpathlineto{\pgfqpoint{0.828149in}{2.991446in}}%
\pgfpathlineto{\pgfqpoint{0.863974in}{2.958217in}}%
\pgfpathlineto{\pgfqpoint{0.899798in}{2.925406in}}%
\pgfpathlineto{\pgfqpoint{0.935622in}{2.893013in}}%
\pgfpathlineto{\pgfqpoint{0.971447in}{2.861036in}}%
\pgfpathlineto{\pgfqpoint{1.007271in}{2.829478in}}%
\pgfpathlineto{\pgfqpoint{1.043096in}{2.798336in}}%
\pgfpathlineto{\pgfqpoint{1.078920in}{2.767454in}}%
\pgfpathlineto{\pgfqpoint{1.114744in}{2.736988in}}%
\pgfpathlineto{\pgfqpoint{1.150569in}{2.706959in}}%
\pgfpathlineto{\pgfqpoint{1.186393in}{2.677394in}}%
\pgfpathlineto{\pgfqpoint{1.222217in}{2.648322in}}%
\pgfpathlineto{\pgfqpoint{1.258042in}{2.619771in}}%
\pgfpathlineto{\pgfqpoint{1.293866in}{2.591769in}}%
\pgfpathlineto{\pgfqpoint{1.329690in}{2.564337in}}%
\pgfpathlineto{\pgfqpoint{1.365515in}{2.537490in}}%
\pgfpathlineto{\pgfqpoint{1.401339in}{2.511236in}}%
\pgfpathlineto{\pgfqpoint{1.437163in}{2.485576in}}%
\pgfpathlineto{\pgfqpoint{1.472988in}{2.460503in}}%
\pgfpathlineto{\pgfqpoint{1.508812in}{2.436000in}}%
\pgfpathlineto{\pgfqpoint{1.544636in}{2.412047in}}%
\pgfpathlineto{\pgfqpoint{1.580461in}{2.388617in}}%
\pgfpathlineto{\pgfqpoint{1.616285in}{2.365679in}}%
\pgfpathlineto{\pgfqpoint{1.652109in}{2.343198in}}%
\pgfpathlineto{\pgfqpoint{1.687934in}{2.321139in}}%
\pgfpathlineto{\pgfqpoint{1.723758in}{2.299465in}}%
\pgfpathlineto{\pgfqpoint{1.759582in}{2.278141in}}%
\pgfpathlineto{\pgfqpoint{1.795407in}{2.257130in}}%
\pgfpathlineto{\pgfqpoint{1.831231in}{2.236404in}}%
\pgfpathlineto{\pgfqpoint{1.867055in}{2.215935in}}%
\pgfpathlineto{\pgfqpoint{1.902880in}{2.195703in}}%
\pgfpathlineto{\pgfqpoint{1.938704in}{2.175689in}}%
\pgfpathlineto{\pgfqpoint{1.974528in}{2.155880in}}%
\pgfpathlineto{\pgfqpoint{2.010353in}{2.136263in}}%
\pgfpathlineto{\pgfqpoint{2.046177in}{2.116828in}}%
\pgfpathlineto{\pgfqpoint{2.082002in}{2.097570in}}%
\pgfpathlineto{\pgfqpoint{2.117826in}{2.078482in}}%
\pgfpathlineto{\pgfqpoint{2.153650in}{2.059562in}}%
\pgfpathlineto{\pgfqpoint{2.189475in}{2.040805in}}%
\pgfpathlineto{\pgfqpoint{2.225299in}{2.022207in}}%
\pgfpathlineto{\pgfqpoint{2.261123in}{2.003761in}}%
\pgfpathlineto{\pgfqpoint{2.296948in}{1.985462in}}%
\pgfpathlineto{\pgfqpoint{2.332772in}{1.967303in}}%
\pgfpathlineto{\pgfqpoint{2.368596in}{1.949275in}}%
\pgfpathlineto{\pgfqpoint{2.404421in}{1.931372in}}%
\pgfpathlineto{\pgfqpoint{2.440245in}{1.913584in}}%
\pgfpathlineto{\pgfqpoint{2.476069in}{1.895901in}}%
\pgfpathlineto{\pgfqpoint{2.511894in}{1.878315in}}%
\pgfpathlineto{\pgfqpoint{2.547718in}{1.860818in}}%
\pgfpathlineto{\pgfqpoint{2.583542in}{1.843400in}}%
\pgfpathlineto{\pgfqpoint{2.619367in}{1.826054in}}%
\pgfpathlineto{\pgfqpoint{2.655191in}{1.808771in}}%
\pgfpathlineto{\pgfqpoint{2.691015in}{1.791541in}}%
\pgfpathlineto{\pgfqpoint{2.726840in}{1.774354in}}%
\pgfpathlineto{\pgfqpoint{2.762664in}{1.757203in}}%
\pgfpathlineto{\pgfqpoint{2.798488in}{1.740078in}}%
\pgfpathlineto{\pgfqpoint{2.834313in}{1.722970in}}%
\pgfpathlineto{\pgfqpoint{2.870137in}{1.705870in}}%
\pgfpathlineto{\pgfqpoint{2.905961in}{1.688769in}}%
\pgfpathlineto{\pgfqpoint{2.941786in}{1.671659in}}%
\pgfpathlineto{\pgfqpoint{2.977610in}{1.654530in}}%
\pgfpathlineto{\pgfqpoint{3.013434in}{1.637374in}}%
\pgfpathlineto{\pgfqpoint{3.049259in}{1.620183in}}%
\pgfpathlineto{\pgfqpoint{3.085083in}{1.602947in}}%
\pgfpathlineto{\pgfqpoint{3.120908in}{1.585657in}}%
\pgfpathlineto{\pgfqpoint{3.156732in}{1.568303in}}%
\pgfpathlineto{\pgfqpoint{3.192556in}{1.550877in}}%
\pgfpathlineto{\pgfqpoint{3.228381in}{1.533373in}}%
\pgfpathlineto{\pgfqpoint{3.264205in}{1.515782in}}%
\pgfpathlineto{\pgfqpoint{3.300029in}{1.498097in}}%
\pgfpathlineto{\pgfqpoint{3.335854in}{1.480309in}}%
\pgfpathlineto{\pgfqpoint{3.371678in}{1.462409in}}%
\pgfpathlineto{\pgfqpoint{3.407502in}{1.444389in}}%
\pgfpathlineto{\pgfqpoint{3.443327in}{1.426240in}}%
\pgfpathlineto{\pgfqpoint{3.479151in}{1.407951in}}%
\pgfpathlineto{\pgfqpoint{3.514975in}{1.389510in}}%
\pgfpathlineto{\pgfqpoint{3.550800in}{1.370905in}}%
\pgfpathlineto{\pgfqpoint{3.586624in}{1.352119in}}%
\pgfpathlineto{\pgfqpoint{3.622448in}{1.333137in}}%
\pgfpathlineto{\pgfqpoint{3.658273in}{1.313941in}}%
\pgfpathlineto{\pgfqpoint{3.694097in}{1.294514in}}%
\pgfpathlineto{\pgfqpoint{3.729921in}{1.274834in}}%
\pgfpathlineto{\pgfqpoint{3.765746in}{1.254878in}}%
\pgfpathlineto{\pgfqpoint{3.801570in}{1.234624in}}%
\pgfpathlineto{\pgfqpoint{3.837394in}{1.214051in}}%
\pgfpathlineto{\pgfqpoint{3.873219in}{1.193136in}}%
\pgfpathlineto{\pgfqpoint{3.909043in}{1.171859in}}%
\pgfpathlineto{\pgfqpoint{3.944867in}{1.150201in}}%
\pgfpathlineto{\pgfqpoint{3.980692in}{1.128140in}}%
\pgfpathlineto{\pgfqpoint{4.016516in}{1.105659in}}%
\pgfpathlineto{\pgfqpoint{4.052340in}{1.082738in}}%
\pgfpathlineto{\pgfqpoint{4.088165in}{1.059365in}}%
\pgfpathlineto{\pgfqpoint{4.123989in}{1.035525in}}%
\pgfpathlineto{\pgfqpoint{4.159814in}{1.011206in}}%
\pgfpathlineto{\pgfqpoint{4.195638in}{0.986398in}}%
\pgfpathlineto{\pgfqpoint{4.231462in}{0.961093in}}%
\pgfpathlineto{\pgfqpoint{4.267287in}{0.935288in}}%
\pgfpathlineto{\pgfqpoint{4.303111in}{0.908980in}}%
\pgfpathlineto{\pgfqpoint{4.338935in}{0.882171in}}%
\pgfpathlineto{\pgfqpoint{4.374760in}{0.854859in}}%
\pgfpathlineto{\pgfqpoint{4.410584in}{0.827045in}}%
\pgfpathlineto{\pgfqpoint{4.446408in}{0.798718in}}%
\pgfpathlineto{\pgfqpoint{4.482233in}{0.769891in}}%
\pgfpathlineto{\pgfqpoint{4.518057in}{0.740564in}}%
\pgfpathlineto{\pgfqpoint{4.553881in}{0.710737in}}%
\pgfpathlineto{\pgfqpoint{4.589706in}{0.680410in}}%
\pgfpathlineto{\pgfqpoint{4.625530in}{0.649583in}}%
\pgfpathlineto{\pgfqpoint{4.661354in}{0.618257in}}%
\pgfpathlineto{\pgfqpoint{4.697179in}{0.586430in}}%
\pgfpathlineto{\pgfqpoint{4.733003in}{0.554103in}}%
\pgfpathlineto{\pgfqpoint{4.768827in}{0.521277in}}%
\pgfpathlineto{\pgfqpoint{4.804652in}{0.487950in}}%
\pgfpathlineto{\pgfqpoint{4.840476in}{0.454124in}}%
\pgfusepath{stroke}%
\end{pgfscope}%
\begin{pgfscope}%
\pgfpathrectangle{\pgfqpoint{0.401840in}{0.316851in}}{\pgfqpoint{4.650000in}{3.020000in}} %
\pgfusepath{clip}%
\pgfsetbuttcap%
\pgfsetmiterjoin%
\definecolor{currentfill}{rgb}{1.000000,0.000000,0.000000}%
\pgfsetfillcolor{currentfill}%
\pgfsetfillopacity{0.700000}%
\pgfsetlinewidth{1.003750pt}%
\definecolor{currentstroke}{rgb}{1.000000,0.000000,0.000000}%
\pgfsetstrokecolor{currentstroke}%
\pgfsetstrokeopacity{0.700000}%
\pgfsetdash{}{0pt}%
\pgfsys@defobject{currentmarker}{\pgfqpoint{-0.041667in}{-0.041667in}}{\pgfqpoint{0.041667in}{0.041667in}}{%
\pgfpathmoveto{\pgfqpoint{-0.041667in}{-0.041667in}}%
\pgfpathlineto{\pgfqpoint{0.041667in}{-0.041667in}}%
\pgfpathlineto{\pgfqpoint{0.041667in}{0.041667in}}%
\pgfpathlineto{\pgfqpoint{-0.041667in}{0.041667in}}%
\pgfpathclose%
\pgfusepath{stroke,fill}%
}%
\begin{pgfscope}%
\pgfsys@transformshift{0.613203in}{3.199578in}%
\pgfsys@useobject{currentmarker}{}%
\end{pgfscope}%
\begin{pgfscope}%
\pgfsys@transformshift{0.899798in}{2.925406in}%
\pgfsys@useobject{currentmarker}{}%
\end{pgfscope}%
\begin{pgfscope}%
\pgfsys@transformshift{1.186393in}{2.677394in}%
\pgfsys@useobject{currentmarker}{}%
\end{pgfscope}%
\begin{pgfscope}%
\pgfsys@transformshift{1.472988in}{2.460503in}%
\pgfsys@useobject{currentmarker}{}%
\end{pgfscope}%
\begin{pgfscope}%
\pgfsys@transformshift{1.759582in}{2.278141in}%
\pgfsys@useobject{currentmarker}{}%
\end{pgfscope}%
\begin{pgfscope}%
\pgfsys@transformshift{2.046177in}{2.116828in}%
\pgfsys@useobject{currentmarker}{}%
\end{pgfscope}%
\begin{pgfscope}%
\pgfsys@transformshift{2.332772in}{1.967303in}%
\pgfsys@useobject{currentmarker}{}%
\end{pgfscope}%
\begin{pgfscope}%
\pgfsys@transformshift{2.619367in}{1.826054in}%
\pgfsys@useobject{currentmarker}{}%
\end{pgfscope}%
\begin{pgfscope}%
\pgfsys@transformshift{2.905961in}{1.688769in}%
\pgfsys@useobject{currentmarker}{}%
\end{pgfscope}%
\begin{pgfscope}%
\pgfsys@transformshift{3.192556in}{1.550877in}%
\pgfsys@useobject{currentmarker}{}%
\end{pgfscope}%
\begin{pgfscope}%
\pgfsys@transformshift{3.479151in}{1.407951in}%
\pgfsys@useobject{currentmarker}{}%
\end{pgfscope}%
\begin{pgfscope}%
\pgfsys@transformshift{3.765746in}{1.254878in}%
\pgfsys@useobject{currentmarker}{}%
\end{pgfscope}%
\begin{pgfscope}%
\pgfsys@transformshift{4.052340in}{1.082738in}%
\pgfsys@useobject{currentmarker}{}%
\end{pgfscope}%
\begin{pgfscope}%
\pgfsys@transformshift{4.338935in}{0.882171in}%
\pgfsys@useobject{currentmarker}{}%
\end{pgfscope}%
\begin{pgfscope}%
\pgfsys@transformshift{4.625530in}{0.649583in}%
\pgfsys@useobject{currentmarker}{}%
\end{pgfscope}%
\end{pgfscope}%
\begin{pgfscope}%
\pgfpathrectangle{\pgfqpoint{0.401840in}{0.316851in}}{\pgfqpoint{4.650000in}{3.020000in}} %
\pgfusepath{clip}%
\pgfsetrectcap%
\pgfsetroundjoin%
\pgfsetlinewidth{1.505625pt}%
\definecolor{currentstroke}{rgb}{0.000000,0.000000,1.000000}%
\pgfsetstrokecolor{currentstroke}%
\pgfsetstrokeopacity{0.700000}%
\pgfsetdash{}{0pt}%
\pgfpathmoveto{\pgfqpoint{0.649028in}{3.093309in}}%
\pgfpathlineto{\pgfqpoint{0.684852in}{3.077671in}}%
\pgfpathlineto{\pgfqpoint{0.720676in}{3.058612in}}%
\pgfpathlineto{\pgfqpoint{0.756501in}{3.036930in}}%
\pgfpathlineto{\pgfqpoint{0.792325in}{3.013250in}}%
\pgfpathlineto{\pgfqpoint{0.828149in}{2.988048in}}%
\pgfpathlineto{\pgfqpoint{0.863974in}{2.961671in}}%
\pgfpathlineto{\pgfqpoint{0.899798in}{2.934354in}}%
\pgfpathlineto{\pgfqpoint{0.935622in}{2.906235in}}%
\pgfpathlineto{\pgfqpoint{0.971447in}{2.877387in}}%
\pgfpathlineto{\pgfqpoint{1.007271in}{2.847853in}}%
\pgfpathlineto{\pgfqpoint{1.043096in}{2.817690in}}%
\pgfpathlineto{\pgfqpoint{1.078920in}{2.786999in}}%
\pgfpathlineto{\pgfqpoint{1.114744in}{2.755941in}}%
\pgfpathlineto{\pgfqpoint{1.150569in}{2.724725in}}%
\pgfpathlineto{\pgfqpoint{1.186393in}{2.693597in}}%
\pgfpathlineto{\pgfqpoint{1.222217in}{2.662812in}}%
\pgfpathlineto{\pgfqpoint{1.258042in}{2.632617in}}%
\pgfpathlineto{\pgfqpoint{1.293866in}{2.603218in}}%
\pgfpathlineto{\pgfqpoint{1.329690in}{2.574756in}}%
\pgfpathlineto{\pgfqpoint{1.365515in}{2.547293in}}%
\pgfpathlineto{\pgfqpoint{1.401339in}{2.520803in}}%
\pgfpathlineto{\pgfqpoint{1.437163in}{2.495199in}}%
\pgfpathlineto{\pgfqpoint{1.472988in}{2.470358in}}%
\pgfpathlineto{\pgfqpoint{1.508812in}{2.446158in}}%
\pgfpathlineto{\pgfqpoint{1.544636in}{2.422495in}}%
\pgfpathlineto{\pgfqpoint{1.580461in}{2.399293in}}%
\pgfpathlineto{\pgfqpoint{1.616285in}{2.376504in}}%
\pgfpathlineto{\pgfqpoint{1.652109in}{2.354108in}}%
\pgfpathlineto{\pgfqpoint{1.687934in}{2.332106in}}%
\pgfpathlineto{\pgfqpoint{1.723758in}{2.310508in}}%
\pgfpathlineto{\pgfqpoint{1.759582in}{2.289310in}}%
\pgfpathlineto{\pgfqpoint{1.795407in}{2.268483in}}%
\pgfpathlineto{\pgfqpoint{1.831231in}{2.247966in}}%
\pgfpathlineto{\pgfqpoint{1.867055in}{2.227674in}}%
\pgfpathlineto{\pgfqpoint{1.902880in}{2.207523in}}%
\pgfpathlineto{\pgfqpoint{1.938704in}{2.187445in}}%
\pgfpathlineto{\pgfqpoint{1.974528in}{2.167403in}}%
\pgfpathlineto{\pgfqpoint{2.010353in}{2.147392in}}%
\pgfpathlineto{\pgfqpoint{2.046177in}{2.127439in}}%
\pgfpathlineto{\pgfqpoint{2.082002in}{2.107595in}}%
\pgfpathlineto{\pgfqpoint{2.117826in}{2.087934in}}%
\pgfpathlineto{\pgfqpoint{2.153650in}{2.068534in}}%
\pgfpathlineto{\pgfqpoint{2.189475in}{2.049454in}}%
\pgfpathlineto{\pgfqpoint{2.225299in}{2.030722in}}%
\pgfpathlineto{\pgfqpoint{2.261123in}{2.012318in}}%
\pgfpathlineto{\pgfqpoint{2.296948in}{1.994188in}}%
\pgfpathlineto{\pgfqpoint{2.332772in}{1.976259in}}%
\pgfpathlineto{\pgfqpoint{2.368596in}{1.958453in}}%
\pgfpathlineto{\pgfqpoint{2.404421in}{1.940704in}}%
\pgfpathlineto{\pgfqpoint{2.440245in}{1.922965in}}%
\pgfpathlineto{\pgfqpoint{2.476069in}{1.905213in}}%
\pgfpathlineto{\pgfqpoint{2.511894in}{1.887456in}}%
\pgfpathlineto{\pgfqpoint{2.547718in}{1.869725in}}%
\pgfpathlineto{\pgfqpoint{2.583542in}{1.852071in}}%
\pgfpathlineto{\pgfqpoint{2.619367in}{1.834536in}}%
\pgfpathlineto{\pgfqpoint{2.655191in}{1.817145in}}%
\pgfpathlineto{\pgfqpoint{2.691015in}{1.799890in}}%
\pgfpathlineto{\pgfqpoint{2.726840in}{1.782738in}}%
\pgfpathlineto{\pgfqpoint{2.762664in}{1.765642in}}%
\pgfpathlineto{\pgfqpoint{2.798488in}{1.748554in}}%
\pgfpathlineto{\pgfqpoint{2.834313in}{1.731432in}}%
\pgfpathlineto{\pgfqpoint{2.870137in}{1.714251in}}%
\pgfpathlineto{\pgfqpoint{2.905961in}{1.697001in}}%
\pgfpathlineto{\pgfqpoint{2.941786in}{1.679703in}}%
\pgfpathlineto{\pgfqpoint{2.977610in}{1.662396in}}%
\pgfpathlineto{\pgfqpoint{3.013434in}{1.645135in}}%
\pgfpathlineto{\pgfqpoint{3.049259in}{1.627965in}}%
\pgfpathlineto{\pgfqpoint{3.085083in}{1.610910in}}%
\pgfpathlineto{\pgfqpoint{3.120908in}{1.593960in}}%
\pgfpathlineto{\pgfqpoint{3.156732in}{1.577078in}}%
\pgfpathlineto{\pgfqpoint{3.192556in}{1.560204in}}%
\pgfpathlineto{\pgfqpoint{3.228381in}{1.543274in}}%
\pgfpathlineto{\pgfqpoint{3.264205in}{1.526215in}}%
\pgfpathlineto{\pgfqpoint{3.300029in}{1.508963in}}%
\pgfpathlineto{\pgfqpoint{3.335854in}{1.491462in}}%
\pgfpathlineto{\pgfqpoint{3.371678in}{1.473675in}}%
\pgfpathlineto{\pgfqpoint{3.407502in}{1.455585in}}%
\pgfpathlineto{\pgfqpoint{3.443327in}{1.437188in}}%
\pgfpathlineto{\pgfqpoint{3.479151in}{1.418477in}}%
\pgfpathlineto{\pgfqpoint{3.514975in}{1.399439in}}%
\pgfpathlineto{\pgfqpoint{3.550800in}{1.380052in}}%
\pgfpathlineto{\pgfqpoint{3.586624in}{1.360293in}}%
\pgfpathlineto{\pgfqpoint{3.622448in}{1.340156in}}%
\pgfpathlineto{\pgfqpoint{3.658273in}{1.319658in}}%
\pgfpathlineto{\pgfqpoint{3.694097in}{1.298848in}}%
\pgfpathlineto{\pgfqpoint{3.729921in}{1.277818in}}%
\pgfpathlineto{\pgfqpoint{3.765746in}{1.256698in}}%
\pgfpathlineto{\pgfqpoint{3.801570in}{1.235659in}}%
\pgfpathlineto{\pgfqpoint{3.837394in}{1.214900in}}%
\pgfpathlineto{\pgfqpoint{3.873219in}{1.194617in}}%
\pgfpathlineto{\pgfqpoint{3.909043in}{1.174975in}}%
\pgfpathlineto{\pgfqpoint{3.944867in}{1.156072in}}%
\pgfpathlineto{\pgfqpoint{3.980692in}{1.137913in}}%
\pgfpathlineto{\pgfqpoint{4.016516in}{1.120396in}}%
\pgfpathlineto{\pgfqpoint{4.052340in}{1.103296in}}%
\pgfpathlineto{\pgfqpoint{4.088165in}{1.086262in}}%
\pgfpathlineto{\pgfqpoint{4.123989in}{1.068817in}}%
\pgfpathlineto{\pgfqpoint{4.159814in}{1.050374in}}%
\pgfpathlineto{\pgfqpoint{4.195638in}{1.030272in}}%
\pgfpathlineto{\pgfqpoint{4.231462in}{1.007819in}}%
\pgfpathlineto{\pgfqpoint{4.267287in}{0.982348in}}%
\pgfpathlineto{\pgfqpoint{4.303111in}{0.953272in}}%
\pgfpathlineto{\pgfqpoint{4.338935in}{0.920152in}}%
\pgfpathlineto{\pgfqpoint{4.374760in}{0.882769in}}%
\pgfpathlineto{\pgfqpoint{4.410584in}{0.841200in}}%
\pgfpathlineto{\pgfqpoint{4.446408in}{0.795893in}}%
\pgfpathlineto{\pgfqpoint{4.482233in}{0.747730in}}%
\pgfpathlineto{\pgfqpoint{4.518057in}{0.698056in}}%
\pgfpathlineto{\pgfqpoint{4.553881in}{0.648687in}}%
\pgfpathlineto{\pgfqpoint{4.589706in}{0.601885in}}%
\pgfpathlineto{\pgfqpoint{4.625530in}{0.560290in}}%
\pgfpathlineto{\pgfqpoint{4.661354in}{0.526811in}}%
\pgfpathlineto{\pgfqpoint{4.697179in}{0.504464in}}%
\pgfpathlineto{\pgfqpoint{4.733003in}{0.496161in}}%
\pgfpathlineto{\pgfqpoint{4.768827in}{0.504479in}}%
\pgfpathlineto{\pgfqpoint{4.804652in}{0.531410in}}%
\pgfpathlineto{\pgfqpoint{4.840476in}{0.578120in}}%
\pgfusepath{stroke}%
\end{pgfscope}%
\begin{pgfscope}%
\pgfpathrectangle{\pgfqpoint{0.401840in}{0.316851in}}{\pgfqpoint{4.650000in}{3.020000in}} %
\pgfusepath{clip}%
\pgfsetbuttcap%
\pgfsetroundjoin%
\definecolor{currentfill}{rgb}{0.000000,0.000000,1.000000}%
\pgfsetfillcolor{currentfill}%
\pgfsetfillopacity{0.700000}%
\pgfsetlinewidth{1.003750pt}%
\definecolor{currentstroke}{rgb}{0.000000,0.000000,1.000000}%
\pgfsetstrokecolor{currentstroke}%
\pgfsetstrokeopacity{0.700000}%
\pgfsetdash{}{0pt}%
\pgfsys@defobject{currentmarker}{\pgfqpoint{-0.041667in}{-0.041667in}}{\pgfqpoint{0.041667in}{0.041667in}}{%
\pgfpathmoveto{\pgfqpoint{0.000000in}{-0.041667in}}%
\pgfpathcurveto{\pgfqpoint{0.011050in}{-0.041667in}}{\pgfqpoint{0.021649in}{-0.037276in}}{\pgfqpoint{0.029463in}{-0.029463in}}%
\pgfpathcurveto{\pgfqpoint{0.037276in}{-0.021649in}}{\pgfqpoint{0.041667in}{-0.011050in}}{\pgfqpoint{0.041667in}{0.000000in}}%
\pgfpathcurveto{\pgfqpoint{0.041667in}{0.011050in}}{\pgfqpoint{0.037276in}{0.021649in}}{\pgfqpoint{0.029463in}{0.029463in}}%
\pgfpathcurveto{\pgfqpoint{0.021649in}{0.037276in}}{\pgfqpoint{0.011050in}{0.041667in}}{\pgfqpoint{0.000000in}{0.041667in}}%
\pgfpathcurveto{\pgfqpoint{-0.011050in}{0.041667in}}{\pgfqpoint{-0.021649in}{0.037276in}}{\pgfqpoint{-0.029463in}{0.029463in}}%
\pgfpathcurveto{\pgfqpoint{-0.037276in}{0.021649in}}{\pgfqpoint{-0.041667in}{0.011050in}}{\pgfqpoint{-0.041667in}{0.000000in}}%
\pgfpathcurveto{\pgfqpoint{-0.041667in}{-0.011050in}}{\pgfqpoint{-0.037276in}{-0.021649in}}{\pgfqpoint{-0.029463in}{-0.029463in}}%
\pgfpathcurveto{\pgfqpoint{-0.021649in}{-0.037276in}}{\pgfqpoint{-0.011050in}{-0.041667in}}{\pgfqpoint{0.000000in}{-0.041667in}}%
\pgfpathclose%
\pgfusepath{stroke,fill}%
}%
\begin{pgfscope}%
\pgfsys@transformshift{0.649028in}{3.093309in}%
\pgfsys@useobject{currentmarker}{}%
\end{pgfscope}%
\begin{pgfscope}%
\pgfsys@transformshift{0.935622in}{2.906235in}%
\pgfsys@useobject{currentmarker}{}%
\end{pgfscope}%
\begin{pgfscope}%
\pgfsys@transformshift{1.222217in}{2.662812in}%
\pgfsys@useobject{currentmarker}{}%
\end{pgfscope}%
\begin{pgfscope}%
\pgfsys@transformshift{1.508812in}{2.446158in}%
\pgfsys@useobject{currentmarker}{}%
\end{pgfscope}%
\begin{pgfscope}%
\pgfsys@transformshift{1.795407in}{2.268483in}%
\pgfsys@useobject{currentmarker}{}%
\end{pgfscope}%
\begin{pgfscope}%
\pgfsys@transformshift{2.082002in}{2.107595in}%
\pgfsys@useobject{currentmarker}{}%
\end{pgfscope}%
\begin{pgfscope}%
\pgfsys@transformshift{2.368596in}{1.958453in}%
\pgfsys@useobject{currentmarker}{}%
\end{pgfscope}%
\begin{pgfscope}%
\pgfsys@transformshift{2.655191in}{1.817145in}%
\pgfsys@useobject{currentmarker}{}%
\end{pgfscope}%
\begin{pgfscope}%
\pgfsys@transformshift{2.941786in}{1.679703in}%
\pgfsys@useobject{currentmarker}{}%
\end{pgfscope}%
\begin{pgfscope}%
\pgfsys@transformshift{3.228381in}{1.543274in}%
\pgfsys@useobject{currentmarker}{}%
\end{pgfscope}%
\begin{pgfscope}%
\pgfsys@transformshift{3.514975in}{1.399439in}%
\pgfsys@useobject{currentmarker}{}%
\end{pgfscope}%
\begin{pgfscope}%
\pgfsys@transformshift{3.801570in}{1.235659in}%
\pgfsys@useobject{currentmarker}{}%
\end{pgfscope}%
\begin{pgfscope}%
\pgfsys@transformshift{4.088165in}{1.086262in}%
\pgfsys@useobject{currentmarker}{}%
\end{pgfscope}%
\begin{pgfscope}%
\pgfsys@transformshift{4.374760in}{0.882769in}%
\pgfsys@useobject{currentmarker}{}%
\end{pgfscope}%
\begin{pgfscope}%
\pgfsys@transformshift{4.661354in}{0.526811in}%
\pgfsys@useobject{currentmarker}{}%
\end{pgfscope}%
\end{pgfscope}%
\begin{pgfscope}%
\pgfsetrectcap%
\pgfsetmiterjoin%
\pgfsetlinewidth{0.803000pt}%
\definecolor{currentstroke}{rgb}{0.000000,0.000000,0.000000}%
\pgfsetstrokecolor{currentstroke}%
\pgfsetdash{}{0pt}%
\pgfpathmoveto{\pgfqpoint{0.401840in}{0.316851in}}%
\pgfpathlineto{\pgfqpoint{0.401840in}{3.336851in}}%
\pgfusepath{stroke}%
\end{pgfscope}%
\begin{pgfscope}%
\pgfsetrectcap%
\pgfsetmiterjoin%
\pgfsetlinewidth{0.803000pt}%
\definecolor{currentstroke}{rgb}{0.000000,0.000000,0.000000}%
\pgfsetstrokecolor{currentstroke}%
\pgfsetdash{}{0pt}%
\pgfpathmoveto{\pgfqpoint{5.051840in}{0.316851in}}%
\pgfpathlineto{\pgfqpoint{5.051840in}{3.336851in}}%
\pgfusepath{stroke}%
\end{pgfscope}%
\begin{pgfscope}%
\pgfsetrectcap%
\pgfsetmiterjoin%
\pgfsetlinewidth{0.803000pt}%
\definecolor{currentstroke}{rgb}{0.000000,0.000000,0.000000}%
\pgfsetstrokecolor{currentstroke}%
\pgfsetdash{}{0pt}%
\pgfpathmoveto{\pgfqpoint{0.401840in}{0.316851in}}%
\pgfpathlineto{\pgfqpoint{5.051840in}{0.316851in}}%
\pgfusepath{stroke}%
\end{pgfscope}%
\begin{pgfscope}%
\pgfsetrectcap%
\pgfsetmiterjoin%
\pgfsetlinewidth{0.803000pt}%
\definecolor{currentstroke}{rgb}{0.000000,0.000000,0.000000}%
\pgfsetstrokecolor{currentstroke}%
\pgfsetdash{}{0pt}%
\pgfpathmoveto{\pgfqpoint{0.401840in}{3.336851in}}%
\pgfpathlineto{\pgfqpoint{5.051840in}{3.336851in}}%
\pgfusepath{stroke}%
\end{pgfscope}%
\begin{pgfscope}%
\pgfsetbuttcap%
\pgfsetmiterjoin%
\definecolor{currentfill}{rgb}{1.000000,1.000000,1.000000}%
\pgfsetfillcolor{currentfill}%
\pgfsetfillopacity{0.800000}%
\pgfsetlinewidth{1.003750pt}%
\definecolor{currentstroke}{rgb}{0.800000,0.800000,0.800000}%
\pgfsetstrokecolor{currentstroke}%
\pgfsetstrokeopacity{0.800000}%
\pgfsetdash{}{0pt}%
\pgfpathmoveto{\pgfqpoint{1.985751in}{2.816059in}}%
\pgfpathlineto{\pgfqpoint{3.467929in}{2.816059in}}%
\pgfpathquadraticcurveto{\pgfqpoint{3.495706in}{2.816059in}}{\pgfqpoint{3.495706in}{2.843837in}}%
\pgfpathlineto{\pgfqpoint{3.495706in}{3.239629in}}%
\pgfpathquadraticcurveto{\pgfqpoint{3.495706in}{3.267407in}}{\pgfqpoint{3.467929in}{3.267407in}}%
\pgfpathlineto{\pgfqpoint{1.985751in}{3.267407in}}%
\pgfpathquadraticcurveto{\pgfqpoint{1.957973in}{3.267407in}}{\pgfqpoint{1.957973in}{3.239629in}}%
\pgfpathlineto{\pgfqpoint{1.957973in}{2.843837in}}%
\pgfpathquadraticcurveto{\pgfqpoint{1.957973in}{2.816059in}}{\pgfqpoint{1.985751in}{2.816059in}}%
\pgfpathclose%
\pgfusepath{stroke,fill}%
\end{pgfscope}%
\begin{pgfscope}%
\pgfsetbuttcap%
\pgfsetroundjoin%
\pgfsetlinewidth{1.505625pt}%
\definecolor{currentstroke}{rgb}{1.000000,0.000000,0.000000}%
\pgfsetstrokecolor{currentstroke}%
\pgfsetstrokeopacity{0.700000}%
\pgfsetdash{{5.550000pt}{2.400000pt}}{0.000000pt}%
\pgfpathmoveto{\pgfqpoint{2.013529in}{3.154939in}}%
\pgfpathlineto{\pgfqpoint{2.291306in}{3.154939in}}%
\pgfusepath{stroke}%
\end{pgfscope}%
\begin{pgfscope}%
\pgfsetbuttcap%
\pgfsetmiterjoin%
\definecolor{currentfill}{rgb}{1.000000,0.000000,0.000000}%
\pgfsetfillcolor{currentfill}%
\pgfsetfillopacity{0.700000}%
\pgfsetlinewidth{1.003750pt}%
\definecolor{currentstroke}{rgb}{1.000000,0.000000,0.000000}%
\pgfsetstrokecolor{currentstroke}%
\pgfsetstrokeopacity{0.700000}%
\pgfsetdash{}{0pt}%
\pgfsys@defobject{currentmarker}{\pgfqpoint{-0.041667in}{-0.041667in}}{\pgfqpoint{0.041667in}{0.041667in}}{%
\pgfpathmoveto{\pgfqpoint{-0.041667in}{-0.041667in}}%
\pgfpathlineto{\pgfqpoint{0.041667in}{-0.041667in}}%
\pgfpathlineto{\pgfqpoint{0.041667in}{0.041667in}}%
\pgfpathlineto{\pgfqpoint{-0.041667in}{0.041667in}}%
\pgfpathclose%
\pgfusepath{stroke,fill}%
}%
\begin{pgfscope}%
\pgfsys@transformshift{2.152417in}{3.154939in}%
\pgfsys@useobject{currentmarker}{}%
\end{pgfscope}%
\end{pgfscope}%
\begin{pgfscope}%
\pgftext[x=2.402417in,y=3.106328in,left,base]{\rmfamily\fontsize{10.000000}{12.000000}\selectfont Savitsky-Golay}%
\end{pgfscope}%
\begin{pgfscope}%
\pgfsetrectcap%
\pgfsetroundjoin%
\pgfsetlinewidth{1.505625pt}%
\definecolor{currentstroke}{rgb}{0.000000,0.000000,1.000000}%
\pgfsetstrokecolor{currentstroke}%
\pgfsetstrokeopacity{0.700000}%
\pgfsetdash{}{0pt}%
\pgfpathmoveto{\pgfqpoint{2.013529in}{2.949115in}}%
\pgfpathlineto{\pgfqpoint{2.291306in}{2.949115in}}%
\pgfusepath{stroke}%
\end{pgfscope}%
\begin{pgfscope}%
\pgfsetbuttcap%
\pgfsetroundjoin%
\definecolor{currentfill}{rgb}{0.000000,0.000000,1.000000}%
\pgfsetfillcolor{currentfill}%
\pgfsetfillopacity{0.700000}%
\pgfsetlinewidth{1.003750pt}%
\definecolor{currentstroke}{rgb}{0.000000,0.000000,1.000000}%
\pgfsetstrokecolor{currentstroke}%
\pgfsetstrokeopacity{0.700000}%
\pgfsetdash{}{0pt}%
\pgfsys@defobject{currentmarker}{\pgfqpoint{-0.041667in}{-0.041667in}}{\pgfqpoint{0.041667in}{0.041667in}}{%
\pgfpathmoveto{\pgfqpoint{0.000000in}{-0.041667in}}%
\pgfpathcurveto{\pgfqpoint{0.011050in}{-0.041667in}}{\pgfqpoint{0.021649in}{-0.037276in}}{\pgfqpoint{0.029463in}{-0.029463in}}%
\pgfpathcurveto{\pgfqpoint{0.037276in}{-0.021649in}}{\pgfqpoint{0.041667in}{-0.011050in}}{\pgfqpoint{0.041667in}{0.000000in}}%
\pgfpathcurveto{\pgfqpoint{0.041667in}{0.011050in}}{\pgfqpoint{0.037276in}{0.021649in}}{\pgfqpoint{0.029463in}{0.029463in}}%
\pgfpathcurveto{\pgfqpoint{0.021649in}{0.037276in}}{\pgfqpoint{0.011050in}{0.041667in}}{\pgfqpoint{0.000000in}{0.041667in}}%
\pgfpathcurveto{\pgfqpoint{-0.011050in}{0.041667in}}{\pgfqpoint{-0.021649in}{0.037276in}}{\pgfqpoint{-0.029463in}{0.029463in}}%
\pgfpathcurveto{\pgfqpoint{-0.037276in}{0.021649in}}{\pgfqpoint{-0.041667in}{0.011050in}}{\pgfqpoint{-0.041667in}{0.000000in}}%
\pgfpathcurveto{\pgfqpoint{-0.041667in}{-0.011050in}}{\pgfqpoint{-0.037276in}{-0.021649in}}{\pgfqpoint{-0.029463in}{-0.029463in}}%
\pgfpathcurveto{\pgfqpoint{-0.021649in}{-0.037276in}}{\pgfqpoint{-0.011050in}{-0.041667in}}{\pgfqpoint{0.000000in}{-0.041667in}}%
\pgfpathclose%
\pgfusepath{stroke,fill}%
}%
\begin{pgfscope}%
\pgfsys@transformshift{2.152417in}{2.949115in}%
\pgfsys@useobject{currentmarker}{}%
\end{pgfscope}%
\end{pgfscope}%
\begin{pgfscope}%
\pgftext[x=2.402417in,y=2.900504in,left,base]{\rmfamily\fontsize{10.000000}{12.000000}\selectfont Butterworth}%
\end{pgfscope}%
\end{pgfpicture}%
\makeatother%
\endgroup%

       \caption{A simple EMA plot.\label{fig:ema3}}
\end{figure}


Qualitively speaking, the 4th-order Savitsky-Golay, and Butterworth filters both produce the smooth derivatives we desire for for the optimization algorithem; but the small window-size needed for Butterworth filter tends to also prduce a noticible end effect. The Savitsky-Golay filter essentially uses a moving-window based on local least-squares polynomial approximations. It was shown that fitting a polynomial to a set of input samples and then evaluating the resulting polynomial at a single point within the approximation interval is equivalent to discrete convolution with a fixed impulse response \hl{[Savitsky, 1964]}. One property of this kind of low-pass filter is their tendency to respect waveform aplitudes, and so they are attractive in applications with having noisey signals with shaply pointed waveforms such as ultrsound or synthetic aperature radar \hl{[Schafer, 2011]}, however these filters tend to suffer from end-effects. Because Savitsky-Golay is a Finite Impulse Response (FIR) filter it requires datapoints to be equally spaced; to accomedate this we interpolate points between the small gaps which sometime occur in the tracking results from image analysis \hl{[ref]}. \hl{[Notes on window length, polynomial order]}.

\subsection*{Numerical Solution to the Equation of Motion}
The equation of motion behaves stiffly due to the large disparity in Coulombic, image charge, and dielectrophoretic lenghtscales. We integrate it numerically using the \verb|odeint| \emph{Scipy} module. This is a shake-and-bake Python wrapper for the vernerable 1982 \emph{netlib ODEPACK} library double-precision \verb|lsoda| (Livermore Solver for Ordinary Differential equations with Automatic method switching for stiff and nonstiff problems) integrator \hl{[ref]}. The function switches between Adams (nonstiff) and Backwards Differentiation Formulas (BDF, stiff) according to the dyanmic value of a set of stiffness eigenvalues.

\end{document}
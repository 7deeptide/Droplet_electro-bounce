\documentclass[10pt,a4paper]{article}
\usepackage[utf8]{inputenc}
%\usepackage{fontspec} % This line only for XeLaTeX and LuaLaTeX
\usepackage{pgfplots}
\usepackage{pgf}
\usepackage[english]{babel}
\usepackage{amsmath}
\usepackage{amsfonts}
\usepackage{amssymb}
\usepackage{graphicx}
\graphicspath{ {../figures/} }
%\usepackage{svg}
\usepackage{verbatim}
\usepackage{color,soul}
\usepackage{listings}
\usepackage{setspace}
\usepackage{float}
\author{Erin Schmidt}

\newlength\figureheight
\newlength\figurewidth
\setlength\figureheight{7cm}
\setlength\figurewidth{10cm}

\let\pgfimageWithoutPath\pgfimage 
\renewcommand{\pgfimage}[2][]{\pgfimageWithoutPath[#1]{../figures/#2}}

\usepackage[
backend=biber,
style=phys,
sorting=none
]{biblatex}
\addbibresource{thesis.bib}

\begin{document}

\doublespacing
\section{Charge Estimates}
We found the distribution of mostly likely experimental net charges for a population of the drops jumped in low-gravity. A covariance plot of the model variables is shown in Figure \ref{fig:scatter}. The multicollinear dependence of charge on drop surface area, $A$, and the characteristic electric field, $E_0$, is evident. Assuming the main effect is the interaction between charge and electric field, a Robust Least Squares model fit $q \sim kAE_0$ (using the Python \verb|statsmodels.formula.api.RLM| function), with the non-linear transformation $A = V_d^{2/3}$, found that $k=5.01 \times 10^{-11} \pm  2.85 \times 10^{-11}$ with $R^2 = 0.946$. This model uses Huber's T norm, median absolute scaling, and H1 covariance estimation. A contour plot showing the estimated drop free charge as a function of $V_d$ and $\varphi_s$ is shown in Figure \ref{fig:charge}.
\begin{figure}[H]
    \centering
    \resizebox{12cm}{!}{%% Creator: Matplotlib, PGF backend
%%
%% To include the figure in your LaTeX document, write
%%   \input{<filename>.pgf}
%%
%% Make sure the required packages are loaded in your preamble
%%   \usepackage{pgf}
%%
%% Figures using additional raster images can only be included by \input if
%% they are in the same directory as the main LaTeX file. For loading figures
%% from other directories you can use the `import` package
%%   \usepackage{import}
%% and then include the figures with
%%   \import{<path to file>}{<filename>.pgf}
%%
%% Matplotlib used the following preamble
%%   \usepackage{fontspec}
%%   \setmainfont{DejaVu Serif}
%%   \setsansfont{DejaVu Sans}
%%   \setmonofont{DejaVu Sans Mono}
%%
\begingroup%
\makeatletter%
\begin{pgfpicture}%
\pgfpathrectangle{\pgfpointorigin}{\pgfqpoint{5.674225in}{4.068832in}}%
\pgfusepath{use as bounding box, clip}%
\begin{pgfscope}%
\pgfsetbuttcap%
\pgfsetmiterjoin%
\definecolor{currentfill}{rgb}{1.000000,1.000000,1.000000}%
\pgfsetfillcolor{currentfill}%
\pgfsetlinewidth{0.000000pt}%
\definecolor{currentstroke}{rgb}{1.000000,1.000000,1.000000}%
\pgfsetstrokecolor{currentstroke}%
\pgfsetdash{}{0pt}%
\pgfpathmoveto{\pgfqpoint{0.000000in}{0.000000in}}%
\pgfpathlineto{\pgfqpoint{5.674225in}{0.000000in}}%
\pgfpathlineto{\pgfqpoint{5.674225in}{4.068832in}}%
\pgfpathlineto{\pgfqpoint{0.000000in}{4.068832in}}%
\pgfpathclose%
\pgfusepath{fill}%
\end{pgfscope}%
\begin{pgfscope}%
\pgfsetbuttcap%
\pgfsetmiterjoin%
\definecolor{currentfill}{rgb}{1.000000,1.000000,1.000000}%
\pgfsetfillcolor{currentfill}%
\pgfsetlinewidth{0.000000pt}%
\definecolor{currentstroke}{rgb}{0.000000,0.000000,0.000000}%
\pgfsetstrokecolor{currentstroke}%
\pgfsetstrokeopacity{0.000000}%
\pgfsetdash{}{0pt}%
\pgfpathmoveto{\pgfqpoint{0.889225in}{3.178832in}}%
\pgfpathlineto{\pgfqpoint{2.051725in}{3.178832in}}%
\pgfpathlineto{\pgfqpoint{2.051725in}{3.933832in}}%
\pgfpathlineto{\pgfqpoint{0.889225in}{3.933832in}}%
\pgfpathclose%
\pgfusepath{fill}%
\end{pgfscope}%
\begin{pgfscope}%
\pgfsetbuttcap%
\pgfsetroundjoin%
\definecolor{currentfill}{rgb}{0.000000,0.000000,0.000000}%
\pgfsetfillcolor{currentfill}%
\pgfsetlinewidth{0.803000pt}%
\definecolor{currentstroke}{rgb}{0.000000,0.000000,0.000000}%
\pgfsetstrokecolor{currentstroke}%
\pgfsetdash{}{0pt}%
\pgfsys@defobject{currentmarker}{\pgfqpoint{-0.048611in}{0.000000in}}{\pgfqpoint{0.000000in}{0.000000in}}{%
\pgfpathmoveto{\pgfqpoint{0.000000in}{0.000000in}}%
\pgfpathlineto{\pgfqpoint{-0.048611in}{0.000000in}}%
\pgfusepath{stroke,fill}%
}%
\begin{pgfscope}%
\pgfsys@transformshift{0.889225in}{3.404583in}%
\pgfsys@useobject{currentmarker}{}%
\end{pgfscope}%
\end{pgfscope}%
\begin{pgfscope}%
\pgftext[x=0.370616in,y=3.362373in,left,base]{\rmfamily\fontsize{8.000000}{9.600000}\selectfont 2.5e-05}%
\end{pgfscope}%
\begin{pgfscope}%
\pgfsetbuttcap%
\pgfsetroundjoin%
\definecolor{currentfill}{rgb}{0.000000,0.000000,0.000000}%
\pgfsetfillcolor{currentfill}%
\pgfsetlinewidth{0.803000pt}%
\definecolor{currentstroke}{rgb}{0.000000,0.000000,0.000000}%
\pgfsetstrokecolor{currentstroke}%
\pgfsetdash{}{0pt}%
\pgfsys@defobject{currentmarker}{\pgfqpoint{-0.048611in}{0.000000in}}{\pgfqpoint{0.000000in}{0.000000in}}{%
\pgfpathmoveto{\pgfqpoint{0.000000in}{0.000000in}}%
\pgfpathlineto{\pgfqpoint{-0.048611in}{0.000000in}}%
\pgfusepath{stroke,fill}%
}%
\begin{pgfscope}%
\pgfsys@transformshift{0.889225in}{3.743297in}%
\pgfsys@useobject{currentmarker}{}%
\end{pgfscope}%
\end{pgfscope}%
\begin{pgfscope}%
\pgftext[x=0.476627in,y=3.701087in,left,base]{\rmfamily\fontsize{8.000000}{9.600000}\selectfont 5e-05}%
\end{pgfscope}%
\begin{pgfscope}%
\pgftext[x=0.315061in,y=3.556332in,,bottom,rotate=90.000000]{\rmfamily\fontsize{16.000000}{19.200000}\selectfont area}%
\end{pgfscope}%
\begin{pgfscope}%
\pgfpathrectangle{\pgfqpoint{0.889225in}{3.178832in}}{\pgfqpoint{1.162500in}{0.755000in}}%
\pgfusepath{clip}%
\pgfsetrectcap%
\pgfsetroundjoin%
\pgfsetlinewidth{1.505625pt}%
\definecolor{currentstroke}{rgb}{0.121569,0.466667,0.705882}%
\pgfsetstrokecolor{currentstroke}%
\pgfsetdash{}{0pt}%
\pgfpathmoveto{\pgfqpoint{0.916904in}{3.605762in}}%
\pgfpathlineto{\pgfqpoint{0.947935in}{3.674214in}}%
\pgfpathlineto{\pgfqpoint{0.973424in}{3.725304in}}%
\pgfpathlineto{\pgfqpoint{0.995589in}{3.765054in}}%
\pgfpathlineto{\pgfqpoint{1.016646in}{3.798265in}}%
\pgfpathlineto{\pgfqpoint{1.035486in}{3.823954in}}%
\pgfpathlineto{\pgfqpoint{1.053218in}{3.844547in}}%
\pgfpathlineto{\pgfqpoint{1.069842in}{3.860681in}}%
\pgfpathlineto{\pgfqpoint{1.086466in}{3.873795in}}%
\pgfpathlineto{\pgfqpoint{1.103090in}{3.883997in}}%
\pgfpathlineto{\pgfqpoint{1.119714in}{3.891448in}}%
\pgfpathlineto{\pgfqpoint{1.136337in}{3.896349in}}%
\pgfpathlineto{\pgfqpoint{1.152961in}{3.898936in}}%
\pgfpathlineto{\pgfqpoint{1.170693in}{3.899439in}}%
\pgfpathlineto{\pgfqpoint{1.189533in}{3.897784in}}%
\pgfpathlineto{\pgfqpoint{1.211698in}{3.893479in}}%
\pgfpathlineto{\pgfqpoint{1.237188in}{3.886125in}}%
\pgfpathlineto{\pgfqpoint{1.269327in}{3.874363in}}%
\pgfpathlineto{\pgfqpoint{1.312549in}{3.855975in}}%
\pgfpathlineto{\pgfqpoint{1.362421in}{3.832368in}}%
\pgfpathlineto{\pgfqpoint{1.402318in}{3.811182in}}%
\pgfpathlineto{\pgfqpoint{1.435565in}{3.791115in}}%
\pgfpathlineto{\pgfqpoint{1.466596in}{3.769769in}}%
\pgfpathlineto{\pgfqpoint{1.496519in}{3.746420in}}%
\pgfpathlineto{\pgfqpoint{1.526442in}{3.720247in}}%
\pgfpathlineto{\pgfqpoint{1.558581in}{3.689172in}}%
\pgfpathlineto{\pgfqpoint{1.596262in}{3.649514in}}%
\pgfpathlineto{\pgfqpoint{1.649458in}{3.589904in}}%
\pgfpathlineto{\pgfqpoint{1.740334in}{3.487986in}}%
\pgfpathlineto{\pgfqpoint{1.790205in}{3.435782in}}%
\pgfpathlineto{\pgfqpoint{1.842293in}{3.384616in}}%
\pgfpathlineto{\pgfqpoint{1.920979in}{3.310990in}}%
\pgfpathlineto{\pgfqpoint{2.010747in}{3.226056in}}%
\pgfpathlineto{\pgfqpoint{2.024046in}{3.213151in}}%
\pgfpathlineto{\pgfqpoint{2.024046in}{3.213151in}}%
\pgfusepath{stroke}%
\end{pgfscope}%
\begin{pgfscope}%
\pgfsetrectcap%
\pgfsetmiterjoin%
\pgfsetlinewidth{0.803000pt}%
\definecolor{currentstroke}{rgb}{0.501961,0.501961,0.501961}%
\pgfsetstrokecolor{currentstroke}%
\pgfsetdash{}{0pt}%
\pgfpathmoveto{\pgfqpoint{0.889225in}{3.178832in}}%
\pgfpathlineto{\pgfqpoint{0.889225in}{3.933832in}}%
\pgfusepath{stroke}%
\end{pgfscope}%
\begin{pgfscope}%
\pgfsetrectcap%
\pgfsetmiterjoin%
\pgfsetlinewidth{0.803000pt}%
\definecolor{currentstroke}{rgb}{0.501961,0.501961,0.501961}%
\pgfsetstrokecolor{currentstroke}%
\pgfsetdash{}{0pt}%
\pgfpathmoveto{\pgfqpoint{2.051725in}{3.178832in}}%
\pgfpathlineto{\pgfqpoint{2.051725in}{3.933832in}}%
\pgfusepath{stroke}%
\end{pgfscope}%
\begin{pgfscope}%
\pgfsetrectcap%
\pgfsetmiterjoin%
\pgfsetlinewidth{0.803000pt}%
\definecolor{currentstroke}{rgb}{0.501961,0.501961,0.501961}%
\pgfsetstrokecolor{currentstroke}%
\pgfsetdash{}{0pt}%
\pgfpathmoveto{\pgfqpoint{0.889225in}{3.178832in}}%
\pgfpathlineto{\pgfqpoint{2.051725in}{3.178832in}}%
\pgfusepath{stroke}%
\end{pgfscope}%
\begin{pgfscope}%
\pgfsetrectcap%
\pgfsetmiterjoin%
\pgfsetlinewidth{0.803000pt}%
\definecolor{currentstroke}{rgb}{0.501961,0.501961,0.501961}%
\pgfsetstrokecolor{currentstroke}%
\pgfsetdash{}{0pt}%
\pgfpathmoveto{\pgfqpoint{0.889225in}{3.933832in}}%
\pgfpathlineto{\pgfqpoint{2.051725in}{3.933832in}}%
\pgfusepath{stroke}%
\end{pgfscope}%
\begin{pgfscope}%
\pgfsetbuttcap%
\pgfsetmiterjoin%
\definecolor{currentfill}{rgb}{1.000000,1.000000,1.000000}%
\pgfsetfillcolor{currentfill}%
\pgfsetlinewidth{0.000000pt}%
\definecolor{currentstroke}{rgb}{0.000000,0.000000,0.000000}%
\pgfsetstrokecolor{currentstroke}%
\pgfsetstrokeopacity{0.000000}%
\pgfsetdash{}{0pt}%
\pgfpathmoveto{\pgfqpoint{2.051725in}{3.178832in}}%
\pgfpathlineto{\pgfqpoint{3.214225in}{3.178832in}}%
\pgfpathlineto{\pgfqpoint{3.214225in}{3.933832in}}%
\pgfpathlineto{\pgfqpoint{2.051725in}{3.933832in}}%
\pgfpathclose%
\pgfusepath{fill}%
\end{pgfscope}%
\begin{pgfscope}%
\pgfpathrectangle{\pgfqpoint{2.051725in}{3.178832in}}{\pgfqpoint{1.162500in}{0.755000in}}%
\pgfusepath{clip}%
\pgfsetbuttcap%
\pgfsetroundjoin%
\definecolor{currentfill}{rgb}{0.000000,0.000000,0.000000}%
\pgfsetfillcolor{currentfill}%
\pgfsetfillopacity{0.500000}%
\pgfsetlinewidth{0.000000pt}%
\definecolor{currentstroke}{rgb}{0.000000,0.000000,0.000000}%
\pgfsetstrokecolor{currentstroke}%
\pgfsetdash{}{0pt}%
\pgfpathmoveto{\pgfqpoint{3.107569in}{3.810398in}}%
\pgfpathcurveto{\pgfqpoint{3.113094in}{3.810398in}}{\pgfqpoint{3.118394in}{3.812593in}}{\pgfqpoint{3.122301in}{3.816499in}}%
\pgfpathcurveto{\pgfqpoint{3.126207in}{3.820406in}}{\pgfqpoint{3.128403in}{3.825706in}}{\pgfqpoint{3.128403in}{3.831231in}}%
\pgfpathcurveto{\pgfqpoint{3.128403in}{3.836756in}}{\pgfqpoint{3.126207in}{3.842055in}}{\pgfqpoint{3.122301in}{3.845962in}}%
\pgfpathcurveto{\pgfqpoint{3.118394in}{3.849869in}}{\pgfqpoint{3.113094in}{3.852064in}}{\pgfqpoint{3.107569in}{3.852064in}}%
\pgfpathcurveto{\pgfqpoint{3.102044in}{3.852064in}}{\pgfqpoint{3.096745in}{3.849869in}}{\pgfqpoint{3.092838in}{3.845962in}}%
\pgfpathcurveto{\pgfqpoint{3.088931in}{3.842055in}}{\pgfqpoint{3.086736in}{3.836756in}}{\pgfqpoint{3.086736in}{3.831231in}}%
\pgfpathcurveto{\pgfqpoint{3.086736in}{3.825706in}}{\pgfqpoint{3.088931in}{3.820406in}}{\pgfqpoint{3.092838in}{3.816499in}}%
\pgfpathcurveto{\pgfqpoint{3.096745in}{3.812593in}}{\pgfqpoint{3.102044in}{3.810398in}}{\pgfqpoint{3.107569in}{3.810398in}}%
\pgfpathclose%
\pgfusepath{fill}%
\end{pgfscope}%
\begin{pgfscope}%
\pgfpathrectangle{\pgfqpoint{2.051725in}{3.178832in}}{\pgfqpoint{1.162500in}{0.755000in}}%
\pgfusepath{clip}%
\pgfsetbuttcap%
\pgfsetroundjoin%
\definecolor{currentfill}{rgb}{0.000000,0.000000,0.000000}%
\pgfsetfillcolor{currentfill}%
\pgfsetfillopacity{0.500000}%
\pgfsetlinewidth{0.000000pt}%
\definecolor{currentstroke}{rgb}{0.000000,0.000000,0.000000}%
\pgfsetstrokecolor{currentstroke}%
\pgfsetdash{}{0pt}%
\pgfpathmoveto{\pgfqpoint{2.270173in}{3.539667in}}%
\pgfpathcurveto{\pgfqpoint{2.275698in}{3.539667in}}{\pgfqpoint{2.280997in}{3.541862in}}{\pgfqpoint{2.284904in}{3.545769in}}%
\pgfpathcurveto{\pgfqpoint{2.288811in}{3.549675in}}{\pgfqpoint{2.291006in}{3.554975in}}{\pgfqpoint{2.291006in}{3.560500in}}%
\pgfpathcurveto{\pgfqpoint{2.291006in}{3.566025in}}{\pgfqpoint{2.288811in}{3.571325in}}{\pgfqpoint{2.284904in}{3.575231in}}%
\pgfpathcurveto{\pgfqpoint{2.280997in}{3.579138in}}{\pgfqpoint{2.275698in}{3.581333in}}{\pgfqpoint{2.270173in}{3.581333in}}%
\pgfpathcurveto{\pgfqpoint{2.264648in}{3.581333in}}{\pgfqpoint{2.259348in}{3.579138in}}{\pgfqpoint{2.255441in}{3.575231in}}%
\pgfpathcurveto{\pgfqpoint{2.251535in}{3.571325in}}{\pgfqpoint{2.249339in}{3.566025in}}{\pgfqpoint{2.249339in}{3.560500in}}%
\pgfpathcurveto{\pgfqpoint{2.249339in}{3.554975in}}{\pgfqpoint{2.251535in}{3.549675in}}{\pgfqpoint{2.255441in}{3.545769in}}%
\pgfpathcurveto{\pgfqpoint{2.259348in}{3.541862in}}{\pgfqpoint{2.264648in}{3.539667in}}{\pgfqpoint{2.270173in}{3.539667in}}%
\pgfpathclose%
\pgfusepath{fill}%
\end{pgfscope}%
\begin{pgfscope}%
\pgfpathrectangle{\pgfqpoint{2.051725in}{3.178832in}}{\pgfqpoint{1.162500in}{0.755000in}}%
\pgfusepath{clip}%
\pgfsetbuttcap%
\pgfsetroundjoin%
\definecolor{currentfill}{rgb}{0.000000,0.000000,0.000000}%
\pgfsetfillcolor{currentfill}%
\pgfsetfillopacity{0.500000}%
\pgfsetlinewidth{0.000000pt}%
\definecolor{currentstroke}{rgb}{0.000000,0.000000,0.000000}%
\pgfsetstrokecolor{currentstroke}%
\pgfsetdash{}{0pt}%
\pgfpathmoveto{\pgfqpoint{2.264867in}{3.541019in}}%
\pgfpathcurveto{\pgfqpoint{2.270392in}{3.541019in}}{\pgfqpoint{2.275692in}{3.543214in}}{\pgfqpoint{2.279599in}{3.547121in}}%
\pgfpathcurveto{\pgfqpoint{2.283506in}{3.551028in}}{\pgfqpoint{2.285701in}{3.556327in}}{\pgfqpoint{2.285701in}{3.561852in}}%
\pgfpathcurveto{\pgfqpoint{2.285701in}{3.567377in}}{\pgfqpoint{2.283506in}{3.572677in}}{\pgfqpoint{2.279599in}{3.576584in}}%
\pgfpathcurveto{\pgfqpoint{2.275692in}{3.580490in}}{\pgfqpoint{2.270392in}{3.582685in}}{\pgfqpoint{2.264867in}{3.582685in}}%
\pgfpathcurveto{\pgfqpoint{2.259342in}{3.582685in}}{\pgfqpoint{2.254043in}{3.580490in}}{\pgfqpoint{2.250136in}{3.576584in}}%
\pgfpathcurveto{\pgfqpoint{2.246229in}{3.572677in}}{\pgfqpoint{2.244034in}{3.567377in}}{\pgfqpoint{2.244034in}{3.561852in}}%
\pgfpathcurveto{\pgfqpoint{2.244034in}{3.556327in}}{\pgfqpoint{2.246229in}{3.551028in}}{\pgfqpoint{2.250136in}{3.547121in}}%
\pgfpathcurveto{\pgfqpoint{2.254043in}{3.543214in}}{\pgfqpoint{2.259342in}{3.541019in}}{\pgfqpoint{2.264867in}{3.541019in}}%
\pgfpathclose%
\pgfusepath{fill}%
\end{pgfscope}%
\begin{pgfscope}%
\pgfpathrectangle{\pgfqpoint{2.051725in}{3.178832in}}{\pgfqpoint{1.162500in}{0.755000in}}%
\pgfusepath{clip}%
\pgfsetbuttcap%
\pgfsetroundjoin%
\definecolor{currentfill}{rgb}{0.000000,0.000000,0.000000}%
\pgfsetfillcolor{currentfill}%
\pgfsetfillopacity{0.500000}%
\pgfsetlinewidth{0.000000pt}%
\definecolor{currentstroke}{rgb}{0.000000,0.000000,0.000000}%
\pgfsetstrokecolor{currentstroke}%
\pgfsetdash{}{0pt}%
\pgfpathmoveto{\pgfqpoint{2.154193in}{3.308342in}}%
\pgfpathcurveto{\pgfqpoint{2.159718in}{3.308342in}}{\pgfqpoint{2.165018in}{3.310537in}}{\pgfqpoint{2.168924in}{3.314444in}}%
\pgfpathcurveto{\pgfqpoint{2.172831in}{3.318351in}}{\pgfqpoint{2.175026in}{3.323651in}}{\pgfqpoint{2.175026in}{3.329176in}}%
\pgfpathcurveto{\pgfqpoint{2.175026in}{3.334701in}}{\pgfqpoint{2.172831in}{3.340000in}}{\pgfqpoint{2.168924in}{3.343907in}}%
\pgfpathcurveto{\pgfqpoint{2.165018in}{3.347814in}}{\pgfqpoint{2.159718in}{3.350009in}}{\pgfqpoint{2.154193in}{3.350009in}}%
\pgfpathcurveto{\pgfqpoint{2.148668in}{3.350009in}}{\pgfqpoint{2.143368in}{3.347814in}}{\pgfqpoint{2.139462in}{3.343907in}}%
\pgfpathcurveto{\pgfqpoint{2.135555in}{3.340000in}}{\pgfqpoint{2.133360in}{3.334701in}}{\pgfqpoint{2.133360in}{3.329176in}}%
\pgfpathcurveto{\pgfqpoint{2.133360in}{3.323651in}}{\pgfqpoint{2.135555in}{3.318351in}}{\pgfqpoint{2.139462in}{3.314444in}}%
\pgfpathcurveto{\pgfqpoint{2.143368in}{3.310537in}}{\pgfqpoint{2.148668in}{3.308342in}}{\pgfqpoint{2.154193in}{3.308342in}}%
\pgfpathclose%
\pgfusepath{fill}%
\end{pgfscope}%
\begin{pgfscope}%
\pgfpathrectangle{\pgfqpoint{2.051725in}{3.178832in}}{\pgfqpoint{1.162500in}{0.755000in}}%
\pgfusepath{clip}%
\pgfsetbuttcap%
\pgfsetroundjoin%
\definecolor{currentfill}{rgb}{0.000000,0.000000,0.000000}%
\pgfsetfillcolor{currentfill}%
\pgfsetfillopacity{0.500000}%
\pgfsetlinewidth{0.000000pt}%
\definecolor{currentstroke}{rgb}{0.000000,0.000000,0.000000}%
\pgfsetstrokecolor{currentstroke}%
\pgfsetdash{}{0pt}%
\pgfpathmoveto{\pgfqpoint{2.183344in}{3.311937in}}%
\pgfpathcurveto{\pgfqpoint{2.188869in}{3.311937in}}{\pgfqpoint{2.194169in}{3.314132in}}{\pgfqpoint{2.198076in}{3.318039in}}%
\pgfpathcurveto{\pgfqpoint{2.201982in}{3.321945in}}{\pgfqpoint{2.204178in}{3.327245in}}{\pgfqpoint{2.204178in}{3.332770in}}%
\pgfpathcurveto{\pgfqpoint{2.204178in}{3.338295in}}{\pgfqpoint{2.201982in}{3.343595in}}{\pgfqpoint{2.198076in}{3.347501in}}%
\pgfpathcurveto{\pgfqpoint{2.194169in}{3.351408in}}{\pgfqpoint{2.188869in}{3.353603in}}{\pgfqpoint{2.183344in}{3.353603in}}%
\pgfpathcurveto{\pgfqpoint{2.177819in}{3.353603in}}{\pgfqpoint{2.172520in}{3.351408in}}{\pgfqpoint{2.168613in}{3.347501in}}%
\pgfpathcurveto{\pgfqpoint{2.164706in}{3.343595in}}{\pgfqpoint{2.162511in}{3.338295in}}{\pgfqpoint{2.162511in}{3.332770in}}%
\pgfpathcurveto{\pgfqpoint{2.162511in}{3.327245in}}{\pgfqpoint{2.164706in}{3.321945in}}{\pgfqpoint{2.168613in}{3.318039in}}%
\pgfpathcurveto{\pgfqpoint{2.172520in}{3.314132in}}{\pgfqpoint{2.177819in}{3.311937in}}{\pgfqpoint{2.183344in}{3.311937in}}%
\pgfpathclose%
\pgfusepath{fill}%
\end{pgfscope}%
\begin{pgfscope}%
\pgfpathrectangle{\pgfqpoint{2.051725in}{3.178832in}}{\pgfqpoint{1.162500in}{0.755000in}}%
\pgfusepath{clip}%
\pgfsetbuttcap%
\pgfsetroundjoin%
\definecolor{currentfill}{rgb}{0.000000,0.000000,0.000000}%
\pgfsetfillcolor{currentfill}%
\pgfsetfillopacity{0.500000}%
\pgfsetlinewidth{0.000000pt}%
\definecolor{currentstroke}{rgb}{0.000000,0.000000,0.000000}%
\pgfsetstrokecolor{currentstroke}%
\pgfsetdash{}{0pt}%
\pgfpathmoveto{\pgfqpoint{2.082232in}{3.215463in}}%
\pgfpathcurveto{\pgfqpoint{2.087757in}{3.215463in}}{\pgfqpoint{2.093056in}{3.217658in}}{\pgfqpoint{2.096963in}{3.221565in}}%
\pgfpathcurveto{\pgfqpoint{2.100870in}{3.225471in}}{\pgfqpoint{2.103065in}{3.230771in}}{\pgfqpoint{2.103065in}{3.236296in}}%
\pgfpathcurveto{\pgfqpoint{2.103065in}{3.241821in}}{\pgfqpoint{2.100870in}{3.247121in}}{\pgfqpoint{2.096963in}{3.251027in}}%
\pgfpathcurveto{\pgfqpoint{2.093056in}{3.254934in}}{\pgfqpoint{2.087757in}{3.257129in}}{\pgfqpoint{2.082232in}{3.257129in}}%
\pgfpathcurveto{\pgfqpoint{2.076707in}{3.257129in}}{\pgfqpoint{2.071407in}{3.254934in}}{\pgfqpoint{2.067500in}{3.251027in}}%
\pgfpathcurveto{\pgfqpoint{2.063594in}{3.247121in}}{\pgfqpoint{2.061398in}{3.241821in}}{\pgfqpoint{2.061398in}{3.236296in}}%
\pgfpathcurveto{\pgfqpoint{2.061398in}{3.230771in}}{\pgfqpoint{2.063594in}{3.225471in}}{\pgfqpoint{2.067500in}{3.221565in}}%
\pgfpathcurveto{\pgfqpoint{2.071407in}{3.217658in}}{\pgfqpoint{2.076707in}{3.215463in}}{\pgfqpoint{2.082232in}{3.215463in}}%
\pgfpathclose%
\pgfusepath{fill}%
\end{pgfscope}%
\begin{pgfscope}%
\pgfpathrectangle{\pgfqpoint{2.051725in}{3.178832in}}{\pgfqpoint{1.162500in}{0.755000in}}%
\pgfusepath{clip}%
\pgfsetbuttcap%
\pgfsetroundjoin%
\definecolor{currentfill}{rgb}{0.000000,0.000000,0.000000}%
\pgfsetfillcolor{currentfill}%
\pgfsetfillopacity{0.500000}%
\pgfsetlinewidth{0.000000pt}%
\definecolor{currentstroke}{rgb}{0.000000,0.000000,0.000000}%
\pgfsetstrokecolor{currentstroke}%
\pgfsetdash{}{0pt}%
\pgfpathmoveto{\pgfqpoint{2.079404in}{3.175975in}}%
\pgfpathcurveto{\pgfqpoint{2.084929in}{3.175975in}}{\pgfqpoint{2.090228in}{3.178170in}}{\pgfqpoint{2.094135in}{3.182077in}}%
\pgfpathcurveto{\pgfqpoint{2.098042in}{3.185984in}}{\pgfqpoint{2.100237in}{3.191283in}}{\pgfqpoint{2.100237in}{3.196809in}}%
\pgfpathcurveto{\pgfqpoint{2.100237in}{3.202334in}}{\pgfqpoint{2.098042in}{3.207633in}}{\pgfqpoint{2.094135in}{3.211540in}}%
\pgfpathcurveto{\pgfqpoint{2.090228in}{3.215447in}}{\pgfqpoint{2.084929in}{3.217642in}}{\pgfqpoint{2.079404in}{3.217642in}}%
\pgfpathcurveto{\pgfqpoint{2.073879in}{3.217642in}}{\pgfqpoint{2.068579in}{3.215447in}}{\pgfqpoint{2.064672in}{3.211540in}}%
\pgfpathcurveto{\pgfqpoint{2.060765in}{3.207633in}}{\pgfqpoint{2.058570in}{3.202334in}}{\pgfqpoint{2.058570in}{3.196809in}}%
\pgfpathcurveto{\pgfqpoint{2.058570in}{3.191283in}}{\pgfqpoint{2.060765in}{3.185984in}}{\pgfqpoint{2.064672in}{3.182077in}}%
\pgfpathcurveto{\pgfqpoint{2.068579in}{3.178170in}}{\pgfqpoint{2.073879in}{3.175975in}}{\pgfqpoint{2.079404in}{3.175975in}}%
\pgfpathclose%
\pgfusepath{fill}%
\end{pgfscope}%
\begin{pgfscope}%
\pgfpathrectangle{\pgfqpoint{2.051725in}{3.178832in}}{\pgfqpoint{1.162500in}{0.755000in}}%
\pgfusepath{clip}%
\pgfsetbuttcap%
\pgfsetroundjoin%
\definecolor{currentfill}{rgb}{0.000000,0.000000,0.000000}%
\pgfsetfillcolor{currentfill}%
\pgfsetfillopacity{0.500000}%
\pgfsetlinewidth{0.000000pt}%
\definecolor{currentstroke}{rgb}{0.000000,0.000000,0.000000}%
\pgfsetstrokecolor{currentstroke}%
\pgfsetdash{}{0pt}%
\pgfpathmoveto{\pgfqpoint{2.536683in}{3.678982in}}%
\pgfpathcurveto{\pgfqpoint{2.542208in}{3.678982in}}{\pgfqpoint{2.547508in}{3.681177in}}{\pgfqpoint{2.551415in}{3.685084in}}%
\pgfpathcurveto{\pgfqpoint{2.555322in}{3.688990in}}{\pgfqpoint{2.557517in}{3.694290in}}{\pgfqpoint{2.557517in}{3.699815in}}%
\pgfpathcurveto{\pgfqpoint{2.557517in}{3.705340in}}{\pgfqpoint{2.555322in}{3.710640in}}{\pgfqpoint{2.551415in}{3.714546in}}%
\pgfpathcurveto{\pgfqpoint{2.547508in}{3.718453in}}{\pgfqpoint{2.542208in}{3.720648in}}{\pgfqpoint{2.536683in}{3.720648in}}%
\pgfpathcurveto{\pgfqpoint{2.531158in}{3.720648in}}{\pgfqpoint{2.525859in}{3.718453in}}{\pgfqpoint{2.521952in}{3.714546in}}%
\pgfpathcurveto{\pgfqpoint{2.518045in}{3.710640in}}{\pgfqpoint{2.515850in}{3.705340in}}{\pgfqpoint{2.515850in}{3.699815in}}%
\pgfpathcurveto{\pgfqpoint{2.515850in}{3.694290in}}{\pgfqpoint{2.518045in}{3.688990in}}{\pgfqpoint{2.521952in}{3.685084in}}%
\pgfpathcurveto{\pgfqpoint{2.525859in}{3.681177in}}{\pgfqpoint{2.531158in}{3.678982in}}{\pgfqpoint{2.536683in}{3.678982in}}%
\pgfpathclose%
\pgfusepath{fill}%
\end{pgfscope}%
\begin{pgfscope}%
\pgfpathrectangle{\pgfqpoint{2.051725in}{3.178832in}}{\pgfqpoint{1.162500in}{0.755000in}}%
\pgfusepath{clip}%
\pgfsetbuttcap%
\pgfsetroundjoin%
\definecolor{currentfill}{rgb}{0.000000,0.000000,0.000000}%
\pgfsetfillcolor{currentfill}%
\pgfsetfillopacity{0.500000}%
\pgfsetlinewidth{0.000000pt}%
\definecolor{currentstroke}{rgb}{0.000000,0.000000,0.000000}%
\pgfsetstrokecolor{currentstroke}%
\pgfsetdash{}{0pt}%
\pgfpathmoveto{\pgfqpoint{2.417381in}{3.528997in}}%
\pgfpathcurveto{\pgfqpoint{2.422906in}{3.528997in}}{\pgfqpoint{2.428205in}{3.531192in}}{\pgfqpoint{2.432112in}{3.535099in}}%
\pgfpathcurveto{\pgfqpoint{2.436019in}{3.539005in}}{\pgfqpoint{2.438214in}{3.544305in}}{\pgfqpoint{2.438214in}{3.549830in}}%
\pgfpathcurveto{\pgfqpoint{2.438214in}{3.555355in}}{\pgfqpoint{2.436019in}{3.560655in}}{\pgfqpoint{2.432112in}{3.564561in}}%
\pgfpathcurveto{\pgfqpoint{2.428205in}{3.568468in}}{\pgfqpoint{2.422906in}{3.570663in}}{\pgfqpoint{2.417381in}{3.570663in}}%
\pgfpathcurveto{\pgfqpoint{2.411856in}{3.570663in}}{\pgfqpoint{2.406556in}{3.568468in}}{\pgfqpoint{2.402649in}{3.564561in}}%
\pgfpathcurveto{\pgfqpoint{2.398743in}{3.560655in}}{\pgfqpoint{2.396547in}{3.555355in}}{\pgfqpoint{2.396547in}{3.549830in}}%
\pgfpathcurveto{\pgfqpoint{2.396547in}{3.544305in}}{\pgfqpoint{2.398743in}{3.539005in}}{\pgfqpoint{2.402649in}{3.535099in}}%
\pgfpathcurveto{\pgfqpoint{2.406556in}{3.531192in}}{\pgfqpoint{2.411856in}{3.528997in}}{\pgfqpoint{2.417381in}{3.528997in}}%
\pgfpathclose%
\pgfusepath{fill}%
\end{pgfscope}%
\begin{pgfscope}%
\pgfpathrectangle{\pgfqpoint{2.051725in}{3.178832in}}{\pgfqpoint{1.162500in}{0.755000in}}%
\pgfusepath{clip}%
\pgfsetbuttcap%
\pgfsetroundjoin%
\definecolor{currentfill}{rgb}{0.000000,0.000000,0.000000}%
\pgfsetfillcolor{currentfill}%
\pgfsetfillopacity{0.500000}%
\pgfsetlinewidth{0.000000pt}%
\definecolor{currentstroke}{rgb}{0.000000,0.000000,0.000000}%
\pgfsetstrokecolor{currentstroke}%
\pgfsetdash{}{0pt}%
\pgfpathmoveto{\pgfqpoint{2.412630in}{3.426855in}}%
\pgfpathcurveto{\pgfqpoint{2.418155in}{3.426855in}}{\pgfqpoint{2.423454in}{3.429051in}}{\pgfqpoint{2.427361in}{3.432957in}}%
\pgfpathcurveto{\pgfqpoint{2.431268in}{3.436864in}}{\pgfqpoint{2.433463in}{3.442164in}}{\pgfqpoint{2.433463in}{3.447689in}}%
\pgfpathcurveto{\pgfqpoint{2.433463in}{3.453214in}}{\pgfqpoint{2.431268in}{3.458513in}}{\pgfqpoint{2.427361in}{3.462420in}}%
\pgfpathcurveto{\pgfqpoint{2.423454in}{3.466327in}}{\pgfqpoint{2.418155in}{3.468522in}}{\pgfqpoint{2.412630in}{3.468522in}}%
\pgfpathcurveto{\pgfqpoint{2.407105in}{3.468522in}}{\pgfqpoint{2.401805in}{3.466327in}}{\pgfqpoint{2.397898in}{3.462420in}}%
\pgfpathcurveto{\pgfqpoint{2.393991in}{3.458513in}}{\pgfqpoint{2.391796in}{3.453214in}}{\pgfqpoint{2.391796in}{3.447689in}}%
\pgfpathcurveto{\pgfqpoint{2.391796in}{3.442164in}}{\pgfqpoint{2.393991in}{3.436864in}}{\pgfqpoint{2.397898in}{3.432957in}}%
\pgfpathcurveto{\pgfqpoint{2.401805in}{3.429051in}}{\pgfqpoint{2.407105in}{3.426855in}}{\pgfqpoint{2.412630in}{3.426855in}}%
\pgfpathclose%
\pgfusepath{fill}%
\end{pgfscope}%
\begin{pgfscope}%
\pgfpathrectangle{\pgfqpoint{2.051725in}{3.178832in}}{\pgfqpoint{1.162500in}{0.755000in}}%
\pgfusepath{clip}%
\pgfsetbuttcap%
\pgfsetroundjoin%
\definecolor{currentfill}{rgb}{0.000000,0.000000,0.000000}%
\pgfsetfillcolor{currentfill}%
\pgfsetfillopacity{0.500000}%
\pgfsetlinewidth{0.000000pt}%
\definecolor{currentstroke}{rgb}{0.000000,0.000000,0.000000}%
\pgfsetstrokecolor{currentstroke}%
\pgfsetdash{}{0pt}%
\pgfpathmoveto{\pgfqpoint{2.478188in}{3.677650in}}%
\pgfpathcurveto{\pgfqpoint{2.483713in}{3.677650in}}{\pgfqpoint{2.489012in}{3.679845in}}{\pgfqpoint{2.492919in}{3.683752in}}%
\pgfpathcurveto{\pgfqpoint{2.496826in}{3.687659in}}{\pgfqpoint{2.499021in}{3.692959in}}{\pgfqpoint{2.499021in}{3.698484in}}%
\pgfpathcurveto{\pgfqpoint{2.499021in}{3.704009in}}{\pgfqpoint{2.496826in}{3.709308in}}{\pgfqpoint{2.492919in}{3.713215in}}%
\pgfpathcurveto{\pgfqpoint{2.489012in}{3.717122in}}{\pgfqpoint{2.483713in}{3.719317in}}{\pgfqpoint{2.478188in}{3.719317in}}%
\pgfpathcurveto{\pgfqpoint{2.472663in}{3.719317in}}{\pgfqpoint{2.467363in}{3.717122in}}{\pgfqpoint{2.463456in}{3.713215in}}%
\pgfpathcurveto{\pgfqpoint{2.459549in}{3.709308in}}{\pgfqpoint{2.457354in}{3.704009in}}{\pgfqpoint{2.457354in}{3.698484in}}%
\pgfpathcurveto{\pgfqpoint{2.457354in}{3.692959in}}{\pgfqpoint{2.459549in}{3.687659in}}{\pgfqpoint{2.463456in}{3.683752in}}%
\pgfpathcurveto{\pgfqpoint{2.467363in}{3.679845in}}{\pgfqpoint{2.472663in}{3.677650in}}{\pgfqpoint{2.478188in}{3.677650in}}%
\pgfpathclose%
\pgfusepath{fill}%
\end{pgfscope}%
\begin{pgfscope}%
\pgfpathrectangle{\pgfqpoint{2.051725in}{3.178832in}}{\pgfqpoint{1.162500in}{0.755000in}}%
\pgfusepath{clip}%
\pgfsetbuttcap%
\pgfsetroundjoin%
\definecolor{currentfill}{rgb}{0.000000,0.000000,0.000000}%
\pgfsetfillcolor{currentfill}%
\pgfsetfillopacity{0.500000}%
\pgfsetlinewidth{0.000000pt}%
\definecolor{currentstroke}{rgb}{0.000000,0.000000,0.000000}%
\pgfsetstrokecolor{currentstroke}%
\pgfsetdash{}{0pt}%
\pgfpathmoveto{\pgfqpoint{2.172291in}{3.296446in}}%
\pgfpathcurveto{\pgfqpoint{2.177816in}{3.296446in}}{\pgfqpoint{2.183116in}{3.298641in}}{\pgfqpoint{2.187022in}{3.302548in}}%
\pgfpathcurveto{\pgfqpoint{2.190929in}{3.306455in}}{\pgfqpoint{2.193124in}{3.311754in}}{\pgfqpoint{2.193124in}{3.317279in}}%
\pgfpathcurveto{\pgfqpoint{2.193124in}{3.322804in}}{\pgfqpoint{2.190929in}{3.328104in}}{\pgfqpoint{2.187022in}{3.332011in}}%
\pgfpathcurveto{\pgfqpoint{2.183116in}{3.335917in}}{\pgfqpoint{2.177816in}{3.338112in}}{\pgfqpoint{2.172291in}{3.338112in}}%
\pgfpathcurveto{\pgfqpoint{2.166766in}{3.338112in}}{\pgfqpoint{2.161467in}{3.335917in}}{\pgfqpoint{2.157560in}{3.332011in}}%
\pgfpathcurveto{\pgfqpoint{2.153653in}{3.328104in}}{\pgfqpoint{2.151458in}{3.322804in}}{\pgfqpoint{2.151458in}{3.317279in}}%
\pgfpathcurveto{\pgfqpoint{2.151458in}{3.311754in}}{\pgfqpoint{2.153653in}{3.306455in}}{\pgfqpoint{2.157560in}{3.302548in}}%
\pgfpathcurveto{\pgfqpoint{2.161467in}{3.298641in}}{\pgfqpoint{2.166766in}{3.296446in}}{\pgfqpoint{2.172291in}{3.296446in}}%
\pgfpathclose%
\pgfusepath{fill}%
\end{pgfscope}%
\begin{pgfscope}%
\pgfpathrectangle{\pgfqpoint{2.051725in}{3.178832in}}{\pgfqpoint{1.162500in}{0.755000in}}%
\pgfusepath{clip}%
\pgfsetbuttcap%
\pgfsetroundjoin%
\definecolor{currentfill}{rgb}{0.000000,0.000000,0.000000}%
\pgfsetfillcolor{currentfill}%
\pgfsetfillopacity{0.500000}%
\pgfsetlinewidth{0.000000pt}%
\definecolor{currentstroke}{rgb}{0.000000,0.000000,0.000000}%
\pgfsetstrokecolor{currentstroke}%
\pgfsetdash{}{0pt}%
\pgfpathmoveto{\pgfqpoint{2.205718in}{3.322937in}}%
\pgfpathcurveto{\pgfqpoint{2.211243in}{3.322937in}}{\pgfqpoint{2.216543in}{3.325133in}}{\pgfqpoint{2.220450in}{3.329039in}}%
\pgfpathcurveto{\pgfqpoint{2.224357in}{3.332946in}}{\pgfqpoint{2.226552in}{3.338246in}}{\pgfqpoint{2.226552in}{3.343771in}}%
\pgfpathcurveto{\pgfqpoint{2.226552in}{3.349296in}}{\pgfqpoint{2.224357in}{3.354595in}}{\pgfqpoint{2.220450in}{3.358502in}}%
\pgfpathcurveto{\pgfqpoint{2.216543in}{3.362409in}}{\pgfqpoint{2.211243in}{3.364604in}}{\pgfqpoint{2.205718in}{3.364604in}}%
\pgfpathcurveto{\pgfqpoint{2.200193in}{3.364604in}}{\pgfqpoint{2.194894in}{3.362409in}}{\pgfqpoint{2.190987in}{3.358502in}}%
\pgfpathcurveto{\pgfqpoint{2.187080in}{3.354595in}}{\pgfqpoint{2.184885in}{3.349296in}}{\pgfqpoint{2.184885in}{3.343771in}}%
\pgfpathcurveto{\pgfqpoint{2.184885in}{3.338246in}}{\pgfqpoint{2.187080in}{3.332946in}}{\pgfqpoint{2.190987in}{3.329039in}}%
\pgfpathcurveto{\pgfqpoint{2.194894in}{3.325133in}}{\pgfqpoint{2.200193in}{3.322937in}}{\pgfqpoint{2.205718in}{3.322937in}}%
\pgfpathclose%
\pgfusepath{fill}%
\end{pgfscope}%
\begin{pgfscope}%
\pgfpathrectangle{\pgfqpoint{2.051725in}{3.178832in}}{\pgfqpoint{1.162500in}{0.755000in}}%
\pgfusepath{clip}%
\pgfsetbuttcap%
\pgfsetroundjoin%
\definecolor{currentfill}{rgb}{0.000000,0.000000,0.000000}%
\pgfsetfillcolor{currentfill}%
\pgfsetfillopacity{0.500000}%
\pgfsetlinewidth{0.000000pt}%
\definecolor{currentstroke}{rgb}{0.000000,0.000000,0.000000}%
\pgfsetstrokecolor{currentstroke}%
\pgfsetdash{}{0pt}%
\pgfpathmoveto{\pgfqpoint{3.186546in}{3.895023in}}%
\pgfpathcurveto{\pgfqpoint{3.192071in}{3.895023in}}{\pgfqpoint{3.197371in}{3.897218in}}{\pgfqpoint{3.201278in}{3.901125in}}%
\pgfpathcurveto{\pgfqpoint{3.205185in}{3.905032in}}{\pgfqpoint{3.207380in}{3.910331in}}{\pgfqpoint{3.207380in}{3.915856in}}%
\pgfpathcurveto{\pgfqpoint{3.207380in}{3.921381in}}{\pgfqpoint{3.205185in}{3.926681in}}{\pgfqpoint{3.201278in}{3.930588in}}%
\pgfpathcurveto{\pgfqpoint{3.197371in}{3.934494in}}{\pgfqpoint{3.192071in}{3.936690in}}{\pgfqpoint{3.186546in}{3.936690in}}%
\pgfpathcurveto{\pgfqpoint{3.181021in}{3.936690in}}{\pgfqpoint{3.175722in}{3.934494in}}{\pgfqpoint{3.171815in}{3.930588in}}%
\pgfpathcurveto{\pgfqpoint{3.167908in}{3.926681in}}{\pgfqpoint{3.165713in}{3.921381in}}{\pgfqpoint{3.165713in}{3.915856in}}%
\pgfpathcurveto{\pgfqpoint{3.165713in}{3.910331in}}{\pgfqpoint{3.167908in}{3.905032in}}{\pgfqpoint{3.171815in}{3.901125in}}%
\pgfpathcurveto{\pgfqpoint{3.175722in}{3.897218in}}{\pgfqpoint{3.181021in}{3.895023in}}{\pgfqpoint{3.186546in}{3.895023in}}%
\pgfpathclose%
\pgfusepath{fill}%
\end{pgfscope}%
\begin{pgfscope}%
\pgfpathrectangle{\pgfqpoint{2.051725in}{3.178832in}}{\pgfqpoint{1.162500in}{0.755000in}}%
\pgfusepath{clip}%
\pgfsetbuttcap%
\pgfsetroundjoin%
\definecolor{currentfill}{rgb}{0.000000,0.000000,0.000000}%
\pgfsetfillcolor{currentfill}%
\pgfsetfillopacity{0.500000}%
\pgfsetlinewidth{0.000000pt}%
\definecolor{currentstroke}{rgb}{0.000000,0.000000,0.000000}%
\pgfsetstrokecolor{currentstroke}%
\pgfsetdash{}{0pt}%
\pgfpathmoveto{\pgfqpoint{2.394110in}{3.491212in}}%
\pgfpathcurveto{\pgfqpoint{2.399635in}{3.491212in}}{\pgfqpoint{2.404935in}{3.493407in}}{\pgfqpoint{2.408842in}{3.497314in}}%
\pgfpathcurveto{\pgfqpoint{2.412749in}{3.501221in}}{\pgfqpoint{2.414944in}{3.506520in}}{\pgfqpoint{2.414944in}{3.512045in}}%
\pgfpathcurveto{\pgfqpoint{2.414944in}{3.517571in}}{\pgfqpoint{2.412749in}{3.522870in}}{\pgfqpoint{2.408842in}{3.526777in}}%
\pgfpathcurveto{\pgfqpoint{2.404935in}{3.530684in}}{\pgfqpoint{2.399635in}{3.532879in}}{\pgfqpoint{2.394110in}{3.532879in}}%
\pgfpathcurveto{\pgfqpoint{2.388585in}{3.532879in}}{\pgfqpoint{2.383286in}{3.530684in}}{\pgfqpoint{2.379379in}{3.526777in}}%
\pgfpathcurveto{\pgfqpoint{2.375472in}{3.522870in}}{\pgfqpoint{2.373277in}{3.517571in}}{\pgfqpoint{2.373277in}{3.512045in}}%
\pgfpathcurveto{\pgfqpoint{2.373277in}{3.506520in}}{\pgfqpoint{2.375472in}{3.501221in}}{\pgfqpoint{2.379379in}{3.497314in}}%
\pgfpathcurveto{\pgfqpoint{2.383286in}{3.493407in}}{\pgfqpoint{2.388585in}{3.491212in}}{\pgfqpoint{2.394110in}{3.491212in}}%
\pgfpathclose%
\pgfusepath{fill}%
\end{pgfscope}%
\begin{pgfscope}%
\pgfpathrectangle{\pgfqpoint{2.051725in}{3.178832in}}{\pgfqpoint{1.162500in}{0.755000in}}%
\pgfusepath{clip}%
\pgfsetbuttcap%
\pgfsetroundjoin%
\definecolor{currentfill}{rgb}{0.000000,0.000000,0.000000}%
\pgfsetfillcolor{currentfill}%
\pgfsetfillopacity{0.500000}%
\pgfsetlinewidth{0.000000pt}%
\definecolor{currentstroke}{rgb}{0.000000,0.000000,0.000000}%
\pgfsetstrokecolor{currentstroke}%
\pgfsetdash{}{0pt}%
\pgfpathmoveto{\pgfqpoint{2.125339in}{3.212721in}}%
\pgfpathcurveto{\pgfqpoint{2.130865in}{3.212721in}}{\pgfqpoint{2.136164in}{3.214917in}}{\pgfqpoint{2.140071in}{3.218823in}}%
\pgfpathcurveto{\pgfqpoint{2.143978in}{3.222730in}}{\pgfqpoint{2.146173in}{3.228030in}}{\pgfqpoint{2.146173in}{3.233555in}}%
\pgfpathcurveto{\pgfqpoint{2.146173in}{3.239080in}}{\pgfqpoint{2.143978in}{3.244379in}}{\pgfqpoint{2.140071in}{3.248286in}}%
\pgfpathcurveto{\pgfqpoint{2.136164in}{3.252193in}}{\pgfqpoint{2.130865in}{3.254388in}}{\pgfqpoint{2.125339in}{3.254388in}}%
\pgfpathcurveto{\pgfqpoint{2.119814in}{3.254388in}}{\pgfqpoint{2.114515in}{3.252193in}}{\pgfqpoint{2.110608in}{3.248286in}}%
\pgfpathcurveto{\pgfqpoint{2.106701in}{3.244379in}}{\pgfqpoint{2.104506in}{3.239080in}}{\pgfqpoint{2.104506in}{3.233555in}}%
\pgfpathcurveto{\pgfqpoint{2.104506in}{3.228030in}}{\pgfqpoint{2.106701in}{3.222730in}}{\pgfqpoint{2.110608in}{3.218823in}}%
\pgfpathcurveto{\pgfqpoint{2.114515in}{3.214917in}}{\pgfqpoint{2.119814in}{3.212721in}}{\pgfqpoint{2.125339in}{3.212721in}}%
\pgfpathclose%
\pgfusepath{fill}%
\end{pgfscope}%
\begin{pgfscope}%
\pgfsetrectcap%
\pgfsetmiterjoin%
\pgfsetlinewidth{0.803000pt}%
\definecolor{currentstroke}{rgb}{0.501961,0.501961,0.501961}%
\pgfsetstrokecolor{currentstroke}%
\pgfsetdash{}{0pt}%
\pgfpathmoveto{\pgfqpoint{2.051725in}{3.178832in}}%
\pgfpathlineto{\pgfqpoint{2.051725in}{3.933832in}}%
\pgfusepath{stroke}%
\end{pgfscope}%
\begin{pgfscope}%
\pgfsetrectcap%
\pgfsetmiterjoin%
\pgfsetlinewidth{0.803000pt}%
\definecolor{currentstroke}{rgb}{0.501961,0.501961,0.501961}%
\pgfsetstrokecolor{currentstroke}%
\pgfsetdash{}{0pt}%
\pgfpathmoveto{\pgfqpoint{3.214225in}{3.178832in}}%
\pgfpathlineto{\pgfqpoint{3.214225in}{3.933832in}}%
\pgfusepath{stroke}%
\end{pgfscope}%
\begin{pgfscope}%
\pgfsetrectcap%
\pgfsetmiterjoin%
\pgfsetlinewidth{0.803000pt}%
\definecolor{currentstroke}{rgb}{0.501961,0.501961,0.501961}%
\pgfsetstrokecolor{currentstroke}%
\pgfsetdash{}{0pt}%
\pgfpathmoveto{\pgfqpoint{2.051725in}{3.178832in}}%
\pgfpathlineto{\pgfqpoint{3.214225in}{3.178832in}}%
\pgfusepath{stroke}%
\end{pgfscope}%
\begin{pgfscope}%
\pgfsetrectcap%
\pgfsetmiterjoin%
\pgfsetlinewidth{0.803000pt}%
\definecolor{currentstroke}{rgb}{0.501961,0.501961,0.501961}%
\pgfsetstrokecolor{currentstroke}%
\pgfsetdash{}{0pt}%
\pgfpathmoveto{\pgfqpoint{2.051725in}{3.933832in}}%
\pgfpathlineto{\pgfqpoint{3.214225in}{3.933832in}}%
\pgfusepath{stroke}%
\end{pgfscope}%
\begin{pgfscope}%
\pgfsetbuttcap%
\pgfsetmiterjoin%
\definecolor{currentfill}{rgb}{1.000000,1.000000,1.000000}%
\pgfsetfillcolor{currentfill}%
\pgfsetlinewidth{0.000000pt}%
\definecolor{currentstroke}{rgb}{0.000000,0.000000,0.000000}%
\pgfsetstrokecolor{currentstroke}%
\pgfsetstrokeopacity{0.000000}%
\pgfsetdash{}{0pt}%
\pgfpathmoveto{\pgfqpoint{3.214225in}{3.178832in}}%
\pgfpathlineto{\pgfqpoint{4.376725in}{3.178832in}}%
\pgfpathlineto{\pgfqpoint{4.376725in}{3.933832in}}%
\pgfpathlineto{\pgfqpoint{3.214225in}{3.933832in}}%
\pgfpathclose%
\pgfusepath{fill}%
\end{pgfscope}%
\begin{pgfscope}%
\pgfpathrectangle{\pgfqpoint{3.214225in}{3.178832in}}{\pgfqpoint{1.162500in}{0.755000in}}%
\pgfusepath{clip}%
\pgfsetbuttcap%
\pgfsetroundjoin%
\definecolor{currentfill}{rgb}{0.000000,0.000000,0.000000}%
\pgfsetfillcolor{currentfill}%
\pgfsetfillopacity{0.500000}%
\pgfsetlinewidth{0.000000pt}%
\definecolor{currentstroke}{rgb}{0.000000,0.000000,0.000000}%
\pgfsetstrokecolor{currentstroke}%
\pgfsetdash{}{0pt}%
\pgfpathmoveto{\pgfqpoint{3.656364in}{3.810398in}}%
\pgfpathcurveto{\pgfqpoint{3.661889in}{3.810398in}}{\pgfqpoint{3.667189in}{3.812593in}}{\pgfqpoint{3.671096in}{3.816499in}}%
\pgfpathcurveto{\pgfqpoint{3.675002in}{3.820406in}}{\pgfqpoint{3.677198in}{3.825706in}}{\pgfqpoint{3.677198in}{3.831231in}}%
\pgfpathcurveto{\pgfqpoint{3.677198in}{3.836756in}}{\pgfqpoint{3.675002in}{3.842055in}}{\pgfqpoint{3.671096in}{3.845962in}}%
\pgfpathcurveto{\pgfqpoint{3.667189in}{3.849869in}}{\pgfqpoint{3.661889in}{3.852064in}}{\pgfqpoint{3.656364in}{3.852064in}}%
\pgfpathcurveto{\pgfqpoint{3.650839in}{3.852064in}}{\pgfqpoint{3.645540in}{3.849869in}}{\pgfqpoint{3.641633in}{3.845962in}}%
\pgfpathcurveto{\pgfqpoint{3.637726in}{3.842055in}}{\pgfqpoint{3.635531in}{3.836756in}}{\pgfqpoint{3.635531in}{3.831231in}}%
\pgfpathcurveto{\pgfqpoint{3.635531in}{3.825706in}}{\pgfqpoint{3.637726in}{3.820406in}}{\pgfqpoint{3.641633in}{3.816499in}}%
\pgfpathcurveto{\pgfqpoint{3.645540in}{3.812593in}}{\pgfqpoint{3.650839in}{3.810398in}}{\pgfqpoint{3.656364in}{3.810398in}}%
\pgfpathclose%
\pgfusepath{fill}%
\end{pgfscope}%
\begin{pgfscope}%
\pgfpathrectangle{\pgfqpoint{3.214225in}{3.178832in}}{\pgfqpoint{1.162500in}{0.755000in}}%
\pgfusepath{clip}%
\pgfsetbuttcap%
\pgfsetroundjoin%
\definecolor{currentfill}{rgb}{0.000000,0.000000,0.000000}%
\pgfsetfillcolor{currentfill}%
\pgfsetfillopacity{0.500000}%
\pgfsetlinewidth{0.000000pt}%
\definecolor{currentstroke}{rgb}{0.000000,0.000000,0.000000}%
\pgfsetstrokecolor{currentstroke}%
\pgfsetdash{}{0pt}%
\pgfpathmoveto{\pgfqpoint{4.259629in}{3.539667in}}%
\pgfpathcurveto{\pgfqpoint{4.265154in}{3.539667in}}{\pgfqpoint{4.270454in}{3.541862in}}{\pgfqpoint{4.274361in}{3.545769in}}%
\pgfpathcurveto{\pgfqpoint{4.278268in}{3.549675in}}{\pgfqpoint{4.280463in}{3.554975in}}{\pgfqpoint{4.280463in}{3.560500in}}%
\pgfpathcurveto{\pgfqpoint{4.280463in}{3.566025in}}{\pgfqpoint{4.278268in}{3.571325in}}{\pgfqpoint{4.274361in}{3.575231in}}%
\pgfpathcurveto{\pgfqpoint{4.270454in}{3.579138in}}{\pgfqpoint{4.265154in}{3.581333in}}{\pgfqpoint{4.259629in}{3.581333in}}%
\pgfpathcurveto{\pgfqpoint{4.254104in}{3.581333in}}{\pgfqpoint{4.248805in}{3.579138in}}{\pgfqpoint{4.244898in}{3.575231in}}%
\pgfpathcurveto{\pgfqpoint{4.240991in}{3.571325in}}{\pgfqpoint{4.238796in}{3.566025in}}{\pgfqpoint{4.238796in}{3.560500in}}%
\pgfpathcurveto{\pgfqpoint{4.238796in}{3.554975in}}{\pgfqpoint{4.240991in}{3.549675in}}{\pgfqpoint{4.244898in}{3.545769in}}%
\pgfpathcurveto{\pgfqpoint{4.248805in}{3.541862in}}{\pgfqpoint{4.254104in}{3.539667in}}{\pgfqpoint{4.259629in}{3.539667in}}%
\pgfpathclose%
\pgfusepath{fill}%
\end{pgfscope}%
\begin{pgfscope}%
\pgfpathrectangle{\pgfqpoint{3.214225in}{3.178832in}}{\pgfqpoint{1.162500in}{0.755000in}}%
\pgfusepath{clip}%
\pgfsetbuttcap%
\pgfsetroundjoin%
\definecolor{currentfill}{rgb}{0.000000,0.000000,0.000000}%
\pgfsetfillcolor{currentfill}%
\pgfsetfillopacity{0.500000}%
\pgfsetlinewidth{0.000000pt}%
\definecolor{currentstroke}{rgb}{0.000000,0.000000,0.000000}%
\pgfsetstrokecolor{currentstroke}%
\pgfsetdash{}{0pt}%
\pgfpathmoveto{\pgfqpoint{4.349046in}{3.541019in}}%
\pgfpathcurveto{\pgfqpoint{4.354571in}{3.541019in}}{\pgfqpoint{4.359871in}{3.543214in}}{\pgfqpoint{4.363778in}{3.547121in}}%
\pgfpathcurveto{\pgfqpoint{4.367685in}{3.551028in}}{\pgfqpoint{4.369880in}{3.556327in}}{\pgfqpoint{4.369880in}{3.561852in}}%
\pgfpathcurveto{\pgfqpoint{4.369880in}{3.567377in}}{\pgfqpoint{4.367685in}{3.572677in}}{\pgfqpoint{4.363778in}{3.576584in}}%
\pgfpathcurveto{\pgfqpoint{4.359871in}{3.580490in}}{\pgfqpoint{4.354571in}{3.582685in}}{\pgfqpoint{4.349046in}{3.582685in}}%
\pgfpathcurveto{\pgfqpoint{4.343521in}{3.582685in}}{\pgfqpoint{4.338222in}{3.580490in}}{\pgfqpoint{4.334315in}{3.576584in}}%
\pgfpathcurveto{\pgfqpoint{4.330408in}{3.572677in}}{\pgfqpoint{4.328213in}{3.567377in}}{\pgfqpoint{4.328213in}{3.561852in}}%
\pgfpathcurveto{\pgfqpoint{4.328213in}{3.556327in}}{\pgfqpoint{4.330408in}{3.551028in}}{\pgfqpoint{4.334315in}{3.547121in}}%
\pgfpathcurveto{\pgfqpoint{4.338222in}{3.543214in}}{\pgfqpoint{4.343521in}{3.541019in}}{\pgfqpoint{4.349046in}{3.541019in}}%
\pgfpathclose%
\pgfusepath{fill}%
\end{pgfscope}%
\begin{pgfscope}%
\pgfpathrectangle{\pgfqpoint{3.214225in}{3.178832in}}{\pgfqpoint{1.162500in}{0.755000in}}%
\pgfusepath{clip}%
\pgfsetbuttcap%
\pgfsetroundjoin%
\definecolor{currentfill}{rgb}{0.000000,0.000000,0.000000}%
\pgfsetfillcolor{currentfill}%
\pgfsetfillopacity{0.500000}%
\pgfsetlinewidth{0.000000pt}%
\definecolor{currentstroke}{rgb}{0.000000,0.000000,0.000000}%
\pgfsetstrokecolor{currentstroke}%
\pgfsetdash{}{0pt}%
\pgfpathmoveto{\pgfqpoint{3.915823in}{3.308342in}}%
\pgfpathcurveto{\pgfqpoint{3.921348in}{3.308342in}}{\pgfqpoint{3.926648in}{3.310537in}}{\pgfqpoint{3.930554in}{3.314444in}}%
\pgfpathcurveto{\pgfqpoint{3.934461in}{3.318351in}}{\pgfqpoint{3.936656in}{3.323651in}}{\pgfqpoint{3.936656in}{3.329176in}}%
\pgfpathcurveto{\pgfqpoint{3.936656in}{3.334701in}}{\pgfqpoint{3.934461in}{3.340000in}}{\pgfqpoint{3.930554in}{3.343907in}}%
\pgfpathcurveto{\pgfqpoint{3.926648in}{3.347814in}}{\pgfqpoint{3.921348in}{3.350009in}}{\pgfqpoint{3.915823in}{3.350009in}}%
\pgfpathcurveto{\pgfqpoint{3.910298in}{3.350009in}}{\pgfqpoint{3.904998in}{3.347814in}}{\pgfqpoint{3.901092in}{3.343907in}}%
\pgfpathcurveto{\pgfqpoint{3.897185in}{3.340000in}}{\pgfqpoint{3.894990in}{3.334701in}}{\pgfqpoint{3.894990in}{3.329176in}}%
\pgfpathcurveto{\pgfqpoint{3.894990in}{3.323651in}}{\pgfqpoint{3.897185in}{3.318351in}}{\pgfqpoint{3.901092in}{3.314444in}}%
\pgfpathcurveto{\pgfqpoint{3.904998in}{3.310537in}}{\pgfqpoint{3.910298in}{3.308342in}}{\pgfqpoint{3.915823in}{3.308342in}}%
\pgfpathclose%
\pgfusepath{fill}%
\end{pgfscope}%
\begin{pgfscope}%
\pgfpathrectangle{\pgfqpoint{3.214225in}{3.178832in}}{\pgfqpoint{1.162500in}{0.755000in}}%
\pgfusepath{clip}%
\pgfsetbuttcap%
\pgfsetroundjoin%
\definecolor{currentfill}{rgb}{0.000000,0.000000,0.000000}%
\pgfsetfillcolor{currentfill}%
\pgfsetfillopacity{0.500000}%
\pgfsetlinewidth{0.000000pt}%
\definecolor{currentstroke}{rgb}{0.000000,0.000000,0.000000}%
\pgfsetstrokecolor{currentstroke}%
\pgfsetdash{}{0pt}%
\pgfpathmoveto{\pgfqpoint{3.986481in}{3.311937in}}%
\pgfpathcurveto{\pgfqpoint{3.992006in}{3.311937in}}{\pgfqpoint{3.997306in}{3.314132in}}{\pgfqpoint{4.001213in}{3.318039in}}%
\pgfpathcurveto{\pgfqpoint{4.005120in}{3.321945in}}{\pgfqpoint{4.007315in}{3.327245in}}{\pgfqpoint{4.007315in}{3.332770in}}%
\pgfpathcurveto{\pgfqpoint{4.007315in}{3.338295in}}{\pgfqpoint{4.005120in}{3.343595in}}{\pgfqpoint{4.001213in}{3.347501in}}%
\pgfpathcurveto{\pgfqpoint{3.997306in}{3.351408in}}{\pgfqpoint{3.992006in}{3.353603in}}{\pgfqpoint{3.986481in}{3.353603in}}%
\pgfpathcurveto{\pgfqpoint{3.980956in}{3.353603in}}{\pgfqpoint{3.975657in}{3.351408in}}{\pgfqpoint{3.971750in}{3.347501in}}%
\pgfpathcurveto{\pgfqpoint{3.967843in}{3.343595in}}{\pgfqpoint{3.965648in}{3.338295in}}{\pgfqpoint{3.965648in}{3.332770in}}%
\pgfpathcurveto{\pgfqpoint{3.965648in}{3.327245in}}{\pgfqpoint{3.967843in}{3.321945in}}{\pgfqpoint{3.971750in}{3.318039in}}%
\pgfpathcurveto{\pgfqpoint{3.975657in}{3.314132in}}{\pgfqpoint{3.980956in}{3.311937in}}{\pgfqpoint{3.986481in}{3.311937in}}%
\pgfpathclose%
\pgfusepath{fill}%
\end{pgfscope}%
\begin{pgfscope}%
\pgfpathrectangle{\pgfqpoint{3.214225in}{3.178832in}}{\pgfqpoint{1.162500in}{0.755000in}}%
\pgfusepath{clip}%
\pgfsetbuttcap%
\pgfsetroundjoin%
\definecolor{currentfill}{rgb}{0.000000,0.000000,0.000000}%
\pgfsetfillcolor{currentfill}%
\pgfsetfillopacity{0.500000}%
\pgfsetlinewidth{0.000000pt}%
\definecolor{currentstroke}{rgb}{0.000000,0.000000,0.000000}%
\pgfsetstrokecolor{currentstroke}%
\pgfsetdash{}{0pt}%
\pgfpathmoveto{\pgfqpoint{3.831974in}{3.215463in}}%
\pgfpathcurveto{\pgfqpoint{3.837499in}{3.215463in}}{\pgfqpoint{3.842798in}{3.217658in}}{\pgfqpoint{3.846705in}{3.221565in}}%
\pgfpathcurveto{\pgfqpoint{3.850612in}{3.225471in}}{\pgfqpoint{3.852807in}{3.230771in}}{\pgfqpoint{3.852807in}{3.236296in}}%
\pgfpathcurveto{\pgfqpoint{3.852807in}{3.241821in}}{\pgfqpoint{3.850612in}{3.247121in}}{\pgfqpoint{3.846705in}{3.251027in}}%
\pgfpathcurveto{\pgfqpoint{3.842798in}{3.254934in}}{\pgfqpoint{3.837499in}{3.257129in}}{\pgfqpoint{3.831974in}{3.257129in}}%
\pgfpathcurveto{\pgfqpoint{3.826449in}{3.257129in}}{\pgfqpoint{3.821149in}{3.254934in}}{\pgfqpoint{3.817242in}{3.251027in}}%
\pgfpathcurveto{\pgfqpoint{3.813336in}{3.247121in}}{\pgfqpoint{3.811140in}{3.241821in}}{\pgfqpoint{3.811140in}{3.236296in}}%
\pgfpathcurveto{\pgfqpoint{3.811140in}{3.230771in}}{\pgfqpoint{3.813336in}{3.225471in}}{\pgfqpoint{3.817242in}{3.221565in}}%
\pgfpathcurveto{\pgfqpoint{3.821149in}{3.217658in}}{\pgfqpoint{3.826449in}{3.215463in}}{\pgfqpoint{3.831974in}{3.215463in}}%
\pgfpathclose%
\pgfusepath{fill}%
\end{pgfscope}%
\begin{pgfscope}%
\pgfpathrectangle{\pgfqpoint{3.214225in}{3.178832in}}{\pgfqpoint{1.162500in}{0.755000in}}%
\pgfusepath{clip}%
\pgfsetbuttcap%
\pgfsetroundjoin%
\definecolor{currentfill}{rgb}{0.000000,0.000000,0.000000}%
\pgfsetfillcolor{currentfill}%
\pgfsetfillopacity{0.500000}%
\pgfsetlinewidth{0.000000pt}%
\definecolor{currentstroke}{rgb}{0.000000,0.000000,0.000000}%
\pgfsetstrokecolor{currentstroke}%
\pgfsetdash{}{0pt}%
\pgfpathmoveto{\pgfqpoint{3.241904in}{3.175975in}}%
\pgfpathcurveto{\pgfqpoint{3.247429in}{3.175975in}}{\pgfqpoint{3.252728in}{3.178170in}}{\pgfqpoint{3.256635in}{3.182077in}}%
\pgfpathcurveto{\pgfqpoint{3.260542in}{3.185984in}}{\pgfqpoint{3.262737in}{3.191283in}}{\pgfqpoint{3.262737in}{3.196809in}}%
\pgfpathcurveto{\pgfqpoint{3.262737in}{3.202334in}}{\pgfqpoint{3.260542in}{3.207633in}}{\pgfqpoint{3.256635in}{3.211540in}}%
\pgfpathcurveto{\pgfqpoint{3.252728in}{3.215447in}}{\pgfqpoint{3.247429in}{3.217642in}}{\pgfqpoint{3.241904in}{3.217642in}}%
\pgfpathcurveto{\pgfqpoint{3.236379in}{3.217642in}}{\pgfqpoint{3.231079in}{3.215447in}}{\pgfqpoint{3.227172in}{3.211540in}}%
\pgfpathcurveto{\pgfqpoint{3.223265in}{3.207633in}}{\pgfqpoint{3.221070in}{3.202334in}}{\pgfqpoint{3.221070in}{3.196809in}}%
\pgfpathcurveto{\pgfqpoint{3.221070in}{3.191283in}}{\pgfqpoint{3.223265in}{3.185984in}}{\pgfqpoint{3.227172in}{3.182077in}}%
\pgfpathcurveto{\pgfqpoint{3.231079in}{3.178170in}}{\pgfqpoint{3.236379in}{3.175975in}}{\pgfqpoint{3.241904in}{3.175975in}}%
\pgfpathclose%
\pgfusepath{fill}%
\end{pgfscope}%
\begin{pgfscope}%
\pgfpathrectangle{\pgfqpoint{3.214225in}{3.178832in}}{\pgfqpoint{1.162500in}{0.755000in}}%
\pgfusepath{clip}%
\pgfsetbuttcap%
\pgfsetroundjoin%
\definecolor{currentfill}{rgb}{0.000000,0.000000,0.000000}%
\pgfsetfillcolor{currentfill}%
\pgfsetfillopacity{0.500000}%
\pgfsetlinewidth{0.000000pt}%
\definecolor{currentstroke}{rgb}{0.000000,0.000000,0.000000}%
\pgfsetstrokecolor{currentstroke}%
\pgfsetdash{}{0pt}%
\pgfpathmoveto{\pgfqpoint{4.220724in}{3.678982in}}%
\pgfpathcurveto{\pgfqpoint{4.226249in}{3.678982in}}{\pgfqpoint{4.231548in}{3.681177in}}{\pgfqpoint{4.235455in}{3.685084in}}%
\pgfpathcurveto{\pgfqpoint{4.239362in}{3.688990in}}{\pgfqpoint{4.241557in}{3.694290in}}{\pgfqpoint{4.241557in}{3.699815in}}%
\pgfpathcurveto{\pgfqpoint{4.241557in}{3.705340in}}{\pgfqpoint{4.239362in}{3.710640in}}{\pgfqpoint{4.235455in}{3.714546in}}%
\pgfpathcurveto{\pgfqpoint{4.231548in}{3.718453in}}{\pgfqpoint{4.226249in}{3.720648in}}{\pgfqpoint{4.220724in}{3.720648in}}%
\pgfpathcurveto{\pgfqpoint{4.215199in}{3.720648in}}{\pgfqpoint{4.209899in}{3.718453in}}{\pgfqpoint{4.205992in}{3.714546in}}%
\pgfpathcurveto{\pgfqpoint{4.202086in}{3.710640in}}{\pgfqpoint{4.199890in}{3.705340in}}{\pgfqpoint{4.199890in}{3.699815in}}%
\pgfpathcurveto{\pgfqpoint{4.199890in}{3.694290in}}{\pgfqpoint{4.202086in}{3.688990in}}{\pgfqpoint{4.205992in}{3.685084in}}%
\pgfpathcurveto{\pgfqpoint{4.209899in}{3.681177in}}{\pgfqpoint{4.215199in}{3.678982in}}{\pgfqpoint{4.220724in}{3.678982in}}%
\pgfpathclose%
\pgfusepath{fill}%
\end{pgfscope}%
\begin{pgfscope}%
\pgfpathrectangle{\pgfqpoint{3.214225in}{3.178832in}}{\pgfqpoint{1.162500in}{0.755000in}}%
\pgfusepath{clip}%
\pgfsetbuttcap%
\pgfsetroundjoin%
\definecolor{currentfill}{rgb}{0.000000,0.000000,0.000000}%
\pgfsetfillcolor{currentfill}%
\pgfsetfillopacity{0.500000}%
\pgfsetlinewidth{0.000000pt}%
\definecolor{currentstroke}{rgb}{0.000000,0.000000,0.000000}%
\pgfsetstrokecolor{currentstroke}%
\pgfsetdash{}{0pt}%
\pgfpathmoveto{\pgfqpoint{3.980838in}{3.528997in}}%
\pgfpathcurveto{\pgfqpoint{3.986364in}{3.528997in}}{\pgfqpoint{3.991663in}{3.531192in}}{\pgfqpoint{3.995570in}{3.535099in}}%
\pgfpathcurveto{\pgfqpoint{3.999477in}{3.539005in}}{\pgfqpoint{4.001672in}{3.544305in}}{\pgfqpoint{4.001672in}{3.549830in}}%
\pgfpathcurveto{\pgfqpoint{4.001672in}{3.555355in}}{\pgfqpoint{3.999477in}{3.560655in}}{\pgfqpoint{3.995570in}{3.564561in}}%
\pgfpathcurveto{\pgfqpoint{3.991663in}{3.568468in}}{\pgfqpoint{3.986364in}{3.570663in}}{\pgfqpoint{3.980838in}{3.570663in}}%
\pgfpathcurveto{\pgfqpoint{3.975313in}{3.570663in}}{\pgfqpoint{3.970014in}{3.568468in}}{\pgfqpoint{3.966107in}{3.564561in}}%
\pgfpathcurveto{\pgfqpoint{3.962200in}{3.560655in}}{\pgfqpoint{3.960005in}{3.555355in}}{\pgfqpoint{3.960005in}{3.549830in}}%
\pgfpathcurveto{\pgfqpoint{3.960005in}{3.544305in}}{\pgfqpoint{3.962200in}{3.539005in}}{\pgfqpoint{3.966107in}{3.535099in}}%
\pgfpathcurveto{\pgfqpoint{3.970014in}{3.531192in}}{\pgfqpoint{3.975313in}{3.528997in}}{\pgfqpoint{3.980838in}{3.528997in}}%
\pgfpathclose%
\pgfusepath{fill}%
\end{pgfscope}%
\begin{pgfscope}%
\pgfpathrectangle{\pgfqpoint{3.214225in}{3.178832in}}{\pgfqpoint{1.162500in}{0.755000in}}%
\pgfusepath{clip}%
\pgfsetbuttcap%
\pgfsetroundjoin%
\definecolor{currentfill}{rgb}{0.000000,0.000000,0.000000}%
\pgfsetfillcolor{currentfill}%
\pgfsetfillopacity{0.500000}%
\pgfsetlinewidth{0.000000pt}%
\definecolor{currentstroke}{rgb}{0.000000,0.000000,0.000000}%
\pgfsetstrokecolor{currentstroke}%
\pgfsetdash{}{0pt}%
\pgfpathmoveto{\pgfqpoint{3.708643in}{3.426855in}}%
\pgfpathcurveto{\pgfqpoint{3.714168in}{3.426855in}}{\pgfqpoint{3.719468in}{3.429051in}}{\pgfqpoint{3.723375in}{3.432957in}}%
\pgfpathcurveto{\pgfqpoint{3.727282in}{3.436864in}}{\pgfqpoint{3.729477in}{3.442164in}}{\pgfqpoint{3.729477in}{3.447689in}}%
\pgfpathcurveto{\pgfqpoint{3.729477in}{3.453214in}}{\pgfqpoint{3.727282in}{3.458513in}}{\pgfqpoint{3.723375in}{3.462420in}}%
\pgfpathcurveto{\pgfqpoint{3.719468in}{3.466327in}}{\pgfqpoint{3.714168in}{3.468522in}}{\pgfqpoint{3.708643in}{3.468522in}}%
\pgfpathcurveto{\pgfqpoint{3.703118in}{3.468522in}}{\pgfqpoint{3.697819in}{3.466327in}}{\pgfqpoint{3.693912in}{3.462420in}}%
\pgfpathcurveto{\pgfqpoint{3.690005in}{3.458513in}}{\pgfqpoint{3.687810in}{3.453214in}}{\pgfqpoint{3.687810in}{3.447689in}}%
\pgfpathcurveto{\pgfqpoint{3.687810in}{3.442164in}}{\pgfqpoint{3.690005in}{3.436864in}}{\pgfqpoint{3.693912in}{3.432957in}}%
\pgfpathcurveto{\pgfqpoint{3.697819in}{3.429051in}}{\pgfqpoint{3.703118in}{3.426855in}}{\pgfqpoint{3.708643in}{3.426855in}}%
\pgfpathclose%
\pgfusepath{fill}%
\end{pgfscope}%
\begin{pgfscope}%
\pgfpathrectangle{\pgfqpoint{3.214225in}{3.178832in}}{\pgfqpoint{1.162500in}{0.755000in}}%
\pgfusepath{clip}%
\pgfsetbuttcap%
\pgfsetroundjoin%
\definecolor{currentfill}{rgb}{0.000000,0.000000,0.000000}%
\pgfsetfillcolor{currentfill}%
\pgfsetfillopacity{0.500000}%
\pgfsetlinewidth{0.000000pt}%
\definecolor{currentstroke}{rgb}{0.000000,0.000000,0.000000}%
\pgfsetstrokecolor{currentstroke}%
\pgfsetdash{}{0pt}%
\pgfpathmoveto{\pgfqpoint{4.124371in}{3.677650in}}%
\pgfpathcurveto{\pgfqpoint{4.129896in}{3.677650in}}{\pgfqpoint{4.135195in}{3.679845in}}{\pgfqpoint{4.139102in}{3.683752in}}%
\pgfpathcurveto{\pgfqpoint{4.143009in}{3.687659in}}{\pgfqpoint{4.145204in}{3.692959in}}{\pgfqpoint{4.145204in}{3.698484in}}%
\pgfpathcurveto{\pgfqpoint{4.145204in}{3.704009in}}{\pgfqpoint{4.143009in}{3.709308in}}{\pgfqpoint{4.139102in}{3.713215in}}%
\pgfpathcurveto{\pgfqpoint{4.135195in}{3.717122in}}{\pgfqpoint{4.129896in}{3.719317in}}{\pgfqpoint{4.124371in}{3.719317in}}%
\pgfpathcurveto{\pgfqpoint{4.118846in}{3.719317in}}{\pgfqpoint{4.113546in}{3.717122in}}{\pgfqpoint{4.109639in}{3.713215in}}%
\pgfpathcurveto{\pgfqpoint{4.105732in}{3.709308in}}{\pgfqpoint{4.103537in}{3.704009in}}{\pgfqpoint{4.103537in}{3.698484in}}%
\pgfpathcurveto{\pgfqpoint{4.103537in}{3.692959in}}{\pgfqpoint{4.105732in}{3.687659in}}{\pgfqpoint{4.109639in}{3.683752in}}%
\pgfpathcurveto{\pgfqpoint{4.113546in}{3.679845in}}{\pgfqpoint{4.118846in}{3.677650in}}{\pgfqpoint{4.124371in}{3.677650in}}%
\pgfpathclose%
\pgfusepath{fill}%
\end{pgfscope}%
\begin{pgfscope}%
\pgfpathrectangle{\pgfqpoint{3.214225in}{3.178832in}}{\pgfqpoint{1.162500in}{0.755000in}}%
\pgfusepath{clip}%
\pgfsetbuttcap%
\pgfsetroundjoin%
\definecolor{currentfill}{rgb}{0.000000,0.000000,0.000000}%
\pgfsetfillcolor{currentfill}%
\pgfsetfillopacity{0.500000}%
\pgfsetlinewidth{0.000000pt}%
\definecolor{currentstroke}{rgb}{0.000000,0.000000,0.000000}%
\pgfsetstrokecolor{currentstroke}%
\pgfsetdash{}{0pt}%
\pgfpathmoveto{\pgfqpoint{3.695649in}{3.296446in}}%
\pgfpathcurveto{\pgfqpoint{3.701174in}{3.296446in}}{\pgfqpoint{3.706474in}{3.298641in}}{\pgfqpoint{3.710381in}{3.302548in}}%
\pgfpathcurveto{\pgfqpoint{3.714288in}{3.306455in}}{\pgfqpoint{3.716483in}{3.311754in}}{\pgfqpoint{3.716483in}{3.317279in}}%
\pgfpathcurveto{\pgfqpoint{3.716483in}{3.322804in}}{\pgfqpoint{3.714288in}{3.328104in}}{\pgfqpoint{3.710381in}{3.332011in}}%
\pgfpathcurveto{\pgfqpoint{3.706474in}{3.335917in}}{\pgfqpoint{3.701174in}{3.338112in}}{\pgfqpoint{3.695649in}{3.338112in}}%
\pgfpathcurveto{\pgfqpoint{3.690124in}{3.338112in}}{\pgfqpoint{3.684825in}{3.335917in}}{\pgfqpoint{3.680918in}{3.332011in}}%
\pgfpathcurveto{\pgfqpoint{3.677011in}{3.328104in}}{\pgfqpoint{3.674816in}{3.322804in}}{\pgfqpoint{3.674816in}{3.317279in}}%
\pgfpathcurveto{\pgfqpoint{3.674816in}{3.311754in}}{\pgfqpoint{3.677011in}{3.306455in}}{\pgfqpoint{3.680918in}{3.302548in}}%
\pgfpathcurveto{\pgfqpoint{3.684825in}{3.298641in}}{\pgfqpoint{3.690124in}{3.296446in}}{\pgfqpoint{3.695649in}{3.296446in}}%
\pgfpathclose%
\pgfusepath{fill}%
\end{pgfscope}%
\begin{pgfscope}%
\pgfpathrectangle{\pgfqpoint{3.214225in}{3.178832in}}{\pgfqpoint{1.162500in}{0.755000in}}%
\pgfusepath{clip}%
\pgfsetbuttcap%
\pgfsetroundjoin%
\definecolor{currentfill}{rgb}{0.000000,0.000000,0.000000}%
\pgfsetfillcolor{currentfill}%
\pgfsetfillopacity{0.500000}%
\pgfsetlinewidth{0.000000pt}%
\definecolor{currentstroke}{rgb}{0.000000,0.000000,0.000000}%
\pgfsetstrokecolor{currentstroke}%
\pgfsetdash{}{0pt}%
\pgfpathmoveto{\pgfqpoint{3.329643in}{3.322937in}}%
\pgfpathcurveto{\pgfqpoint{3.335168in}{3.322937in}}{\pgfqpoint{3.340468in}{3.325133in}}{\pgfqpoint{3.344375in}{3.329039in}}%
\pgfpathcurveto{\pgfqpoint{3.348281in}{3.332946in}}{\pgfqpoint{3.350476in}{3.338246in}}{\pgfqpoint{3.350476in}{3.343771in}}%
\pgfpathcurveto{\pgfqpoint{3.350476in}{3.349296in}}{\pgfqpoint{3.348281in}{3.354595in}}{\pgfqpoint{3.344375in}{3.358502in}}%
\pgfpathcurveto{\pgfqpoint{3.340468in}{3.362409in}}{\pgfqpoint{3.335168in}{3.364604in}}{\pgfqpoint{3.329643in}{3.364604in}}%
\pgfpathcurveto{\pgfqpoint{3.324118in}{3.364604in}}{\pgfqpoint{3.318819in}{3.362409in}}{\pgfqpoint{3.314912in}{3.358502in}}%
\pgfpathcurveto{\pgfqpoint{3.311005in}{3.354595in}}{\pgfqpoint{3.308810in}{3.349296in}}{\pgfqpoint{3.308810in}{3.343771in}}%
\pgfpathcurveto{\pgfqpoint{3.308810in}{3.338246in}}{\pgfqpoint{3.311005in}{3.332946in}}{\pgfqpoint{3.314912in}{3.329039in}}%
\pgfpathcurveto{\pgfqpoint{3.318819in}{3.325133in}}{\pgfqpoint{3.324118in}{3.322937in}}{\pgfqpoint{3.329643in}{3.322937in}}%
\pgfpathclose%
\pgfusepath{fill}%
\end{pgfscope}%
\begin{pgfscope}%
\pgfpathrectangle{\pgfqpoint{3.214225in}{3.178832in}}{\pgfqpoint{1.162500in}{0.755000in}}%
\pgfusepath{clip}%
\pgfsetbuttcap%
\pgfsetroundjoin%
\definecolor{currentfill}{rgb}{0.000000,0.000000,0.000000}%
\pgfsetfillcolor{currentfill}%
\pgfsetfillopacity{0.500000}%
\pgfsetlinewidth{0.000000pt}%
\definecolor{currentstroke}{rgb}{0.000000,0.000000,0.000000}%
\pgfsetstrokecolor{currentstroke}%
\pgfsetdash{}{0pt}%
\pgfpathmoveto{\pgfqpoint{4.167862in}{3.895023in}}%
\pgfpathcurveto{\pgfqpoint{4.173387in}{3.895023in}}{\pgfqpoint{4.178687in}{3.897218in}}{\pgfqpoint{4.182593in}{3.901125in}}%
\pgfpathcurveto{\pgfqpoint{4.186500in}{3.905032in}}{\pgfqpoint{4.188695in}{3.910331in}}{\pgfqpoint{4.188695in}{3.915856in}}%
\pgfpathcurveto{\pgfqpoint{4.188695in}{3.921381in}}{\pgfqpoint{4.186500in}{3.926681in}}{\pgfqpoint{4.182593in}{3.930588in}}%
\pgfpathcurveto{\pgfqpoint{4.178687in}{3.934494in}}{\pgfqpoint{4.173387in}{3.936690in}}{\pgfqpoint{4.167862in}{3.936690in}}%
\pgfpathcurveto{\pgfqpoint{4.162337in}{3.936690in}}{\pgfqpoint{4.157037in}{3.934494in}}{\pgfqpoint{4.153131in}{3.930588in}}%
\pgfpathcurveto{\pgfqpoint{4.149224in}{3.926681in}}{\pgfqpoint{4.147029in}{3.921381in}}{\pgfqpoint{4.147029in}{3.915856in}}%
\pgfpathcurveto{\pgfqpoint{4.147029in}{3.910331in}}{\pgfqpoint{4.149224in}{3.905032in}}{\pgfqpoint{4.153131in}{3.901125in}}%
\pgfpathcurveto{\pgfqpoint{4.157037in}{3.897218in}}{\pgfqpoint{4.162337in}{3.895023in}}{\pgfqpoint{4.167862in}{3.895023in}}%
\pgfpathclose%
\pgfusepath{fill}%
\end{pgfscope}%
\begin{pgfscope}%
\pgfpathrectangle{\pgfqpoint{3.214225in}{3.178832in}}{\pgfqpoint{1.162500in}{0.755000in}}%
\pgfusepath{clip}%
\pgfsetbuttcap%
\pgfsetroundjoin%
\definecolor{currentfill}{rgb}{0.000000,0.000000,0.000000}%
\pgfsetfillcolor{currentfill}%
\pgfsetfillopacity{0.500000}%
\pgfsetlinewidth{0.000000pt}%
\definecolor{currentstroke}{rgb}{0.000000,0.000000,0.000000}%
\pgfsetstrokecolor{currentstroke}%
\pgfsetdash{}{0pt}%
\pgfpathmoveto{\pgfqpoint{4.120816in}{3.491212in}}%
\pgfpathcurveto{\pgfqpoint{4.126341in}{3.491212in}}{\pgfqpoint{4.131641in}{3.493407in}}{\pgfqpoint{4.135548in}{3.497314in}}%
\pgfpathcurveto{\pgfqpoint{4.139455in}{3.501221in}}{\pgfqpoint{4.141650in}{3.506520in}}{\pgfqpoint{4.141650in}{3.512045in}}%
\pgfpathcurveto{\pgfqpoint{4.141650in}{3.517571in}}{\pgfqpoint{4.139455in}{3.522870in}}{\pgfqpoint{4.135548in}{3.526777in}}%
\pgfpathcurveto{\pgfqpoint{4.131641in}{3.530684in}}{\pgfqpoint{4.126341in}{3.532879in}}{\pgfqpoint{4.120816in}{3.532879in}}%
\pgfpathcurveto{\pgfqpoint{4.115291in}{3.532879in}}{\pgfqpoint{4.109992in}{3.530684in}}{\pgfqpoint{4.106085in}{3.526777in}}%
\pgfpathcurveto{\pgfqpoint{4.102178in}{3.522870in}}{\pgfqpoint{4.099983in}{3.517571in}}{\pgfqpoint{4.099983in}{3.512045in}}%
\pgfpathcurveto{\pgfqpoint{4.099983in}{3.506520in}}{\pgfqpoint{4.102178in}{3.501221in}}{\pgfqpoint{4.106085in}{3.497314in}}%
\pgfpathcurveto{\pgfqpoint{4.109992in}{3.493407in}}{\pgfqpoint{4.115291in}{3.491212in}}{\pgfqpoint{4.120816in}{3.491212in}}%
\pgfpathclose%
\pgfusepath{fill}%
\end{pgfscope}%
\begin{pgfscope}%
\pgfpathrectangle{\pgfqpoint{3.214225in}{3.178832in}}{\pgfqpoint{1.162500in}{0.755000in}}%
\pgfusepath{clip}%
\pgfsetbuttcap%
\pgfsetroundjoin%
\definecolor{currentfill}{rgb}{0.000000,0.000000,0.000000}%
\pgfsetfillcolor{currentfill}%
\pgfsetfillopacity{0.500000}%
\pgfsetlinewidth{0.000000pt}%
\definecolor{currentstroke}{rgb}{0.000000,0.000000,0.000000}%
\pgfsetstrokecolor{currentstroke}%
\pgfsetdash{}{0pt}%
\pgfpathmoveto{\pgfqpoint{3.792736in}{3.212721in}}%
\pgfpathcurveto{\pgfqpoint{3.798261in}{3.212721in}}{\pgfqpoint{3.803560in}{3.214917in}}{\pgfqpoint{3.807467in}{3.218823in}}%
\pgfpathcurveto{\pgfqpoint{3.811374in}{3.222730in}}{\pgfqpoint{3.813569in}{3.228030in}}{\pgfqpoint{3.813569in}{3.233555in}}%
\pgfpathcurveto{\pgfqpoint{3.813569in}{3.239080in}}{\pgfqpoint{3.811374in}{3.244379in}}{\pgfqpoint{3.807467in}{3.248286in}}%
\pgfpathcurveto{\pgfqpoint{3.803560in}{3.252193in}}{\pgfqpoint{3.798261in}{3.254388in}}{\pgfqpoint{3.792736in}{3.254388in}}%
\pgfpathcurveto{\pgfqpoint{3.787211in}{3.254388in}}{\pgfqpoint{3.781911in}{3.252193in}}{\pgfqpoint{3.778004in}{3.248286in}}%
\pgfpathcurveto{\pgfqpoint{3.774097in}{3.244379in}}{\pgfqpoint{3.771902in}{3.239080in}}{\pgfqpoint{3.771902in}{3.233555in}}%
\pgfpathcurveto{\pgfqpoint{3.771902in}{3.228030in}}{\pgfqpoint{3.774097in}{3.222730in}}{\pgfqpoint{3.778004in}{3.218823in}}%
\pgfpathcurveto{\pgfqpoint{3.781911in}{3.214917in}}{\pgfqpoint{3.787211in}{3.212721in}}{\pgfqpoint{3.792736in}{3.212721in}}%
\pgfpathclose%
\pgfusepath{fill}%
\end{pgfscope}%
\begin{pgfscope}%
\pgfsetrectcap%
\pgfsetmiterjoin%
\pgfsetlinewidth{0.803000pt}%
\definecolor{currentstroke}{rgb}{0.501961,0.501961,0.501961}%
\pgfsetstrokecolor{currentstroke}%
\pgfsetdash{}{0pt}%
\pgfpathmoveto{\pgfqpoint{3.214225in}{3.178832in}}%
\pgfpathlineto{\pgfqpoint{3.214225in}{3.933832in}}%
\pgfusepath{stroke}%
\end{pgfscope}%
\begin{pgfscope}%
\pgfsetrectcap%
\pgfsetmiterjoin%
\pgfsetlinewidth{0.803000pt}%
\definecolor{currentstroke}{rgb}{0.501961,0.501961,0.501961}%
\pgfsetstrokecolor{currentstroke}%
\pgfsetdash{}{0pt}%
\pgfpathmoveto{\pgfqpoint{4.376725in}{3.178832in}}%
\pgfpathlineto{\pgfqpoint{4.376725in}{3.933832in}}%
\pgfusepath{stroke}%
\end{pgfscope}%
\begin{pgfscope}%
\pgfsetrectcap%
\pgfsetmiterjoin%
\pgfsetlinewidth{0.803000pt}%
\definecolor{currentstroke}{rgb}{0.501961,0.501961,0.501961}%
\pgfsetstrokecolor{currentstroke}%
\pgfsetdash{}{0pt}%
\pgfpathmoveto{\pgfqpoint{3.214225in}{3.178832in}}%
\pgfpathlineto{\pgfqpoint{4.376725in}{3.178832in}}%
\pgfusepath{stroke}%
\end{pgfscope}%
\begin{pgfscope}%
\pgfsetrectcap%
\pgfsetmiterjoin%
\pgfsetlinewidth{0.803000pt}%
\definecolor{currentstroke}{rgb}{0.501961,0.501961,0.501961}%
\pgfsetstrokecolor{currentstroke}%
\pgfsetdash{}{0pt}%
\pgfpathmoveto{\pgfqpoint{3.214225in}{3.933832in}}%
\pgfpathlineto{\pgfqpoint{4.376725in}{3.933832in}}%
\pgfusepath{stroke}%
\end{pgfscope}%
\begin{pgfscope}%
\pgfsetbuttcap%
\pgfsetmiterjoin%
\definecolor{currentfill}{rgb}{1.000000,1.000000,1.000000}%
\pgfsetfillcolor{currentfill}%
\pgfsetlinewidth{0.000000pt}%
\definecolor{currentstroke}{rgb}{0.000000,0.000000,0.000000}%
\pgfsetstrokecolor{currentstroke}%
\pgfsetstrokeopacity{0.000000}%
\pgfsetdash{}{0pt}%
\pgfpathmoveto{\pgfqpoint{4.376725in}{3.178832in}}%
\pgfpathlineto{\pgfqpoint{5.539225in}{3.178832in}}%
\pgfpathlineto{\pgfqpoint{5.539225in}{3.933832in}}%
\pgfpathlineto{\pgfqpoint{4.376725in}{3.933832in}}%
\pgfpathclose%
\pgfusepath{fill}%
\end{pgfscope}%
\begin{pgfscope}%
\pgfpathrectangle{\pgfqpoint{4.376725in}{3.178832in}}{\pgfqpoint{1.162500in}{0.755000in}}%
\pgfusepath{clip}%
\pgfsetbuttcap%
\pgfsetroundjoin%
\definecolor{currentfill}{rgb}{0.000000,0.000000,0.000000}%
\pgfsetfillcolor{currentfill}%
\pgfsetfillopacity{0.500000}%
\pgfsetlinewidth{0.000000pt}%
\definecolor{currentstroke}{rgb}{0.000000,0.000000,0.000000}%
\pgfsetstrokecolor{currentstroke}%
\pgfsetdash{}{0pt}%
\pgfpathmoveto{\pgfqpoint{5.214960in}{3.810398in}}%
\pgfpathcurveto{\pgfqpoint{5.220485in}{3.810398in}}{\pgfqpoint{5.225785in}{3.812593in}}{\pgfqpoint{5.229692in}{3.816499in}}%
\pgfpathcurveto{\pgfqpoint{5.233598in}{3.820406in}}{\pgfqpoint{5.235794in}{3.825706in}}{\pgfqpoint{5.235794in}{3.831231in}}%
\pgfpathcurveto{\pgfqpoint{5.235794in}{3.836756in}}{\pgfqpoint{5.233598in}{3.842055in}}{\pgfqpoint{5.229692in}{3.845962in}}%
\pgfpathcurveto{\pgfqpoint{5.225785in}{3.849869in}}{\pgfqpoint{5.220485in}{3.852064in}}{\pgfqpoint{5.214960in}{3.852064in}}%
\pgfpathcurveto{\pgfqpoint{5.209435in}{3.852064in}}{\pgfqpoint{5.204136in}{3.849869in}}{\pgfqpoint{5.200229in}{3.845962in}}%
\pgfpathcurveto{\pgfqpoint{5.196322in}{3.842055in}}{\pgfqpoint{5.194127in}{3.836756in}}{\pgfqpoint{5.194127in}{3.831231in}}%
\pgfpathcurveto{\pgfqpoint{5.194127in}{3.825706in}}{\pgfqpoint{5.196322in}{3.820406in}}{\pgfqpoint{5.200229in}{3.816499in}}%
\pgfpathcurveto{\pgfqpoint{5.204136in}{3.812593in}}{\pgfqpoint{5.209435in}{3.810398in}}{\pgfqpoint{5.214960in}{3.810398in}}%
\pgfpathclose%
\pgfusepath{fill}%
\end{pgfscope}%
\begin{pgfscope}%
\pgfpathrectangle{\pgfqpoint{4.376725in}{3.178832in}}{\pgfqpoint{1.162500in}{0.755000in}}%
\pgfusepath{clip}%
\pgfsetbuttcap%
\pgfsetroundjoin%
\definecolor{currentfill}{rgb}{0.000000,0.000000,0.000000}%
\pgfsetfillcolor{currentfill}%
\pgfsetfillopacity{0.500000}%
\pgfsetlinewidth{0.000000pt}%
\definecolor{currentstroke}{rgb}{0.000000,0.000000,0.000000}%
\pgfsetstrokecolor{currentstroke}%
\pgfsetdash{}{0pt}%
\pgfpathmoveto{\pgfqpoint{4.521739in}{3.539667in}}%
\pgfpathcurveto{\pgfqpoint{4.527264in}{3.539667in}}{\pgfqpoint{4.532563in}{3.541862in}}{\pgfqpoint{4.536470in}{3.545769in}}%
\pgfpathcurveto{\pgfqpoint{4.540377in}{3.549675in}}{\pgfqpoint{4.542572in}{3.554975in}}{\pgfqpoint{4.542572in}{3.560500in}}%
\pgfpathcurveto{\pgfqpoint{4.542572in}{3.566025in}}{\pgfqpoint{4.540377in}{3.571325in}}{\pgfqpoint{4.536470in}{3.575231in}}%
\pgfpathcurveto{\pgfqpoint{4.532563in}{3.579138in}}{\pgfqpoint{4.527264in}{3.581333in}}{\pgfqpoint{4.521739in}{3.581333in}}%
\pgfpathcurveto{\pgfqpoint{4.516214in}{3.581333in}}{\pgfqpoint{4.510914in}{3.579138in}}{\pgfqpoint{4.507007in}{3.575231in}}%
\pgfpathcurveto{\pgfqpoint{4.503101in}{3.571325in}}{\pgfqpoint{4.500905in}{3.566025in}}{\pgfqpoint{4.500905in}{3.560500in}}%
\pgfpathcurveto{\pgfqpoint{4.500905in}{3.554975in}}{\pgfqpoint{4.503101in}{3.549675in}}{\pgfqpoint{4.507007in}{3.545769in}}%
\pgfpathcurveto{\pgfqpoint{4.510914in}{3.541862in}}{\pgfqpoint{4.516214in}{3.539667in}}{\pgfqpoint{4.521739in}{3.539667in}}%
\pgfpathclose%
\pgfusepath{fill}%
\end{pgfscope}%
\begin{pgfscope}%
\pgfpathrectangle{\pgfqpoint{4.376725in}{3.178832in}}{\pgfqpoint{1.162500in}{0.755000in}}%
\pgfusepath{clip}%
\pgfsetbuttcap%
\pgfsetroundjoin%
\definecolor{currentfill}{rgb}{0.000000,0.000000,0.000000}%
\pgfsetfillcolor{currentfill}%
\pgfsetfillopacity{0.500000}%
\pgfsetlinewidth{0.000000pt}%
\definecolor{currentstroke}{rgb}{0.000000,0.000000,0.000000}%
\pgfsetstrokecolor{currentstroke}%
\pgfsetdash{}{0pt}%
\pgfpathmoveto{\pgfqpoint{4.448919in}{3.541019in}}%
\pgfpathcurveto{\pgfqpoint{4.454444in}{3.541019in}}{\pgfqpoint{4.459744in}{3.543214in}}{\pgfqpoint{4.463650in}{3.547121in}}%
\pgfpathcurveto{\pgfqpoint{4.467557in}{3.551028in}}{\pgfqpoint{4.469752in}{3.556327in}}{\pgfqpoint{4.469752in}{3.561852in}}%
\pgfpathcurveto{\pgfqpoint{4.469752in}{3.567377in}}{\pgfqpoint{4.467557in}{3.572677in}}{\pgfqpoint{4.463650in}{3.576584in}}%
\pgfpathcurveto{\pgfqpoint{4.459744in}{3.580490in}}{\pgfqpoint{4.454444in}{3.582685in}}{\pgfqpoint{4.448919in}{3.582685in}}%
\pgfpathcurveto{\pgfqpoint{4.443394in}{3.582685in}}{\pgfqpoint{4.438094in}{3.580490in}}{\pgfqpoint{4.434188in}{3.576584in}}%
\pgfpathcurveto{\pgfqpoint{4.430281in}{3.572677in}}{\pgfqpoint{4.428086in}{3.567377in}}{\pgfqpoint{4.428086in}{3.561852in}}%
\pgfpathcurveto{\pgfqpoint{4.428086in}{3.556327in}}{\pgfqpoint{4.430281in}{3.551028in}}{\pgfqpoint{4.434188in}{3.547121in}}%
\pgfpathcurveto{\pgfqpoint{4.438094in}{3.543214in}}{\pgfqpoint{4.443394in}{3.541019in}}{\pgfqpoint{4.448919in}{3.541019in}}%
\pgfpathclose%
\pgfusepath{fill}%
\end{pgfscope}%
\begin{pgfscope}%
\pgfpathrectangle{\pgfqpoint{4.376725in}{3.178832in}}{\pgfqpoint{1.162500in}{0.755000in}}%
\pgfusepath{clip}%
\pgfsetbuttcap%
\pgfsetroundjoin%
\definecolor{currentfill}{rgb}{0.000000,0.000000,0.000000}%
\pgfsetfillcolor{currentfill}%
\pgfsetfillopacity{0.500000}%
\pgfsetlinewidth{0.000000pt}%
\definecolor{currentstroke}{rgb}{0.000000,0.000000,0.000000}%
\pgfsetstrokecolor{currentstroke}%
\pgfsetdash{}{0pt}%
\pgfpathmoveto{\pgfqpoint{4.704251in}{3.308342in}}%
\pgfpathcurveto{\pgfqpoint{4.709776in}{3.308342in}}{\pgfqpoint{4.715075in}{3.310537in}}{\pgfqpoint{4.718982in}{3.314444in}}%
\pgfpathcurveto{\pgfqpoint{4.722889in}{3.318351in}}{\pgfqpoint{4.725084in}{3.323651in}}{\pgfqpoint{4.725084in}{3.329176in}}%
\pgfpathcurveto{\pgfqpoint{4.725084in}{3.334701in}}{\pgfqpoint{4.722889in}{3.340000in}}{\pgfqpoint{4.718982in}{3.343907in}}%
\pgfpathcurveto{\pgfqpoint{4.715075in}{3.347814in}}{\pgfqpoint{4.709776in}{3.350009in}}{\pgfqpoint{4.704251in}{3.350009in}}%
\pgfpathcurveto{\pgfqpoint{4.698726in}{3.350009in}}{\pgfqpoint{4.693426in}{3.347814in}}{\pgfqpoint{4.689519in}{3.343907in}}%
\pgfpathcurveto{\pgfqpoint{4.685612in}{3.340000in}}{\pgfqpoint{4.683417in}{3.334701in}}{\pgfqpoint{4.683417in}{3.329176in}}%
\pgfpathcurveto{\pgfqpoint{4.683417in}{3.323651in}}{\pgfqpoint{4.685612in}{3.318351in}}{\pgfqpoint{4.689519in}{3.314444in}}%
\pgfpathcurveto{\pgfqpoint{4.693426in}{3.310537in}}{\pgfqpoint{4.698726in}{3.308342in}}{\pgfqpoint{4.704251in}{3.308342in}}%
\pgfpathclose%
\pgfusepath{fill}%
\end{pgfscope}%
\begin{pgfscope}%
\pgfpathrectangle{\pgfqpoint{4.376725in}{3.178832in}}{\pgfqpoint{1.162500in}{0.755000in}}%
\pgfusepath{clip}%
\pgfsetbuttcap%
\pgfsetroundjoin%
\definecolor{currentfill}{rgb}{0.000000,0.000000,0.000000}%
\pgfsetfillcolor{currentfill}%
\pgfsetfillopacity{0.500000}%
\pgfsetlinewidth{0.000000pt}%
\definecolor{currentstroke}{rgb}{0.000000,0.000000,0.000000}%
\pgfsetstrokecolor{currentstroke}%
\pgfsetdash{}{0pt}%
\pgfpathmoveto{\pgfqpoint{4.404404in}{3.311937in}}%
\pgfpathcurveto{\pgfqpoint{4.409929in}{3.311937in}}{\pgfqpoint{4.415228in}{3.314132in}}{\pgfqpoint{4.419135in}{3.318039in}}%
\pgfpathcurveto{\pgfqpoint{4.423042in}{3.321945in}}{\pgfqpoint{4.425237in}{3.327245in}}{\pgfqpoint{4.425237in}{3.332770in}}%
\pgfpathcurveto{\pgfqpoint{4.425237in}{3.338295in}}{\pgfqpoint{4.423042in}{3.343595in}}{\pgfqpoint{4.419135in}{3.347501in}}%
\pgfpathcurveto{\pgfqpoint{4.415228in}{3.351408in}}{\pgfqpoint{4.409929in}{3.353603in}}{\pgfqpoint{4.404404in}{3.353603in}}%
\pgfpathcurveto{\pgfqpoint{4.398879in}{3.353603in}}{\pgfqpoint{4.393579in}{3.351408in}}{\pgfqpoint{4.389672in}{3.347501in}}%
\pgfpathcurveto{\pgfqpoint{4.385765in}{3.343595in}}{\pgfqpoint{4.383570in}{3.338295in}}{\pgfqpoint{4.383570in}{3.332770in}}%
\pgfpathcurveto{\pgfqpoint{4.383570in}{3.327245in}}{\pgfqpoint{4.385765in}{3.321945in}}{\pgfqpoint{4.389672in}{3.318039in}}%
\pgfpathcurveto{\pgfqpoint{4.393579in}{3.314132in}}{\pgfqpoint{4.398879in}{3.311937in}}{\pgfqpoint{4.404404in}{3.311937in}}%
\pgfpathclose%
\pgfusepath{fill}%
\end{pgfscope}%
\begin{pgfscope}%
\pgfpathrectangle{\pgfqpoint{4.376725in}{3.178832in}}{\pgfqpoint{1.162500in}{0.755000in}}%
\pgfusepath{clip}%
\pgfsetbuttcap%
\pgfsetroundjoin%
\definecolor{currentfill}{rgb}{0.000000,0.000000,0.000000}%
\pgfsetfillcolor{currentfill}%
\pgfsetfillopacity{0.500000}%
\pgfsetlinewidth{0.000000pt}%
\definecolor{currentstroke}{rgb}{0.000000,0.000000,0.000000}%
\pgfsetstrokecolor{currentstroke}%
\pgfsetdash{}{0pt}%
\pgfpathmoveto{\pgfqpoint{4.607580in}{3.215463in}}%
\pgfpathcurveto{\pgfqpoint{4.613105in}{3.215463in}}{\pgfqpoint{4.618405in}{3.217658in}}{\pgfqpoint{4.622312in}{3.221565in}}%
\pgfpathcurveto{\pgfqpoint{4.626218in}{3.225471in}}{\pgfqpoint{4.628414in}{3.230771in}}{\pgfqpoint{4.628414in}{3.236296in}}%
\pgfpathcurveto{\pgfqpoint{4.628414in}{3.241821in}}{\pgfqpoint{4.626218in}{3.247121in}}{\pgfqpoint{4.622312in}{3.251027in}}%
\pgfpathcurveto{\pgfqpoint{4.618405in}{3.254934in}}{\pgfqpoint{4.613105in}{3.257129in}}{\pgfqpoint{4.607580in}{3.257129in}}%
\pgfpathcurveto{\pgfqpoint{4.602055in}{3.257129in}}{\pgfqpoint{4.596756in}{3.254934in}}{\pgfqpoint{4.592849in}{3.251027in}}%
\pgfpathcurveto{\pgfqpoint{4.588942in}{3.247121in}}{\pgfqpoint{4.586747in}{3.241821in}}{\pgfqpoint{4.586747in}{3.236296in}}%
\pgfpathcurveto{\pgfqpoint{4.586747in}{3.230771in}}{\pgfqpoint{4.588942in}{3.225471in}}{\pgfqpoint{4.592849in}{3.221565in}}%
\pgfpathcurveto{\pgfqpoint{4.596756in}{3.217658in}}{\pgfqpoint{4.602055in}{3.215463in}}{\pgfqpoint{4.607580in}{3.215463in}}%
\pgfpathclose%
\pgfusepath{fill}%
\end{pgfscope}%
\begin{pgfscope}%
\pgfpathrectangle{\pgfqpoint{4.376725in}{3.178832in}}{\pgfqpoint{1.162500in}{0.755000in}}%
\pgfusepath{clip}%
\pgfsetbuttcap%
\pgfsetroundjoin%
\definecolor{currentfill}{rgb}{0.000000,0.000000,0.000000}%
\pgfsetfillcolor{currentfill}%
\pgfsetfillopacity{0.500000}%
\pgfsetlinewidth{0.000000pt}%
\definecolor{currentstroke}{rgb}{0.000000,0.000000,0.000000}%
\pgfsetstrokecolor{currentstroke}%
\pgfsetdash{}{0pt}%
\pgfpathmoveto{\pgfqpoint{4.596304in}{3.175975in}}%
\pgfpathcurveto{\pgfqpoint{4.601829in}{3.175975in}}{\pgfqpoint{4.607129in}{3.178170in}}{\pgfqpoint{4.611036in}{3.182077in}}%
\pgfpathcurveto{\pgfqpoint{4.614942in}{3.185984in}}{\pgfqpoint{4.617137in}{3.191283in}}{\pgfqpoint{4.617137in}{3.196809in}}%
\pgfpathcurveto{\pgfqpoint{4.617137in}{3.202334in}}{\pgfqpoint{4.614942in}{3.207633in}}{\pgfqpoint{4.611036in}{3.211540in}}%
\pgfpathcurveto{\pgfqpoint{4.607129in}{3.215447in}}{\pgfqpoint{4.601829in}{3.217642in}}{\pgfqpoint{4.596304in}{3.217642in}}%
\pgfpathcurveto{\pgfqpoint{4.590779in}{3.217642in}}{\pgfqpoint{4.585480in}{3.215447in}}{\pgfqpoint{4.581573in}{3.211540in}}%
\pgfpathcurveto{\pgfqpoint{4.577666in}{3.207633in}}{\pgfqpoint{4.575471in}{3.202334in}}{\pgfqpoint{4.575471in}{3.196809in}}%
\pgfpathcurveto{\pgfqpoint{4.575471in}{3.191283in}}{\pgfqpoint{4.577666in}{3.185984in}}{\pgfqpoint{4.581573in}{3.182077in}}%
\pgfpathcurveto{\pgfqpoint{4.585480in}{3.178170in}}{\pgfqpoint{4.590779in}{3.175975in}}{\pgfqpoint{4.596304in}{3.175975in}}%
\pgfpathclose%
\pgfusepath{fill}%
\end{pgfscope}%
\begin{pgfscope}%
\pgfpathrectangle{\pgfqpoint{4.376725in}{3.178832in}}{\pgfqpoint{1.162500in}{0.755000in}}%
\pgfusepath{clip}%
\pgfsetbuttcap%
\pgfsetroundjoin%
\definecolor{currentfill}{rgb}{0.000000,0.000000,0.000000}%
\pgfsetfillcolor{currentfill}%
\pgfsetfillopacity{0.500000}%
\pgfsetlinewidth{0.000000pt}%
\definecolor{currentstroke}{rgb}{0.000000,0.000000,0.000000}%
\pgfsetstrokecolor{currentstroke}%
\pgfsetdash{}{0pt}%
\pgfpathmoveto{\pgfqpoint{5.423123in}{3.678982in}}%
\pgfpathcurveto{\pgfqpoint{5.428649in}{3.678982in}}{\pgfqpoint{5.433948in}{3.681177in}}{\pgfqpoint{5.437855in}{3.685084in}}%
\pgfpathcurveto{\pgfqpoint{5.441762in}{3.688990in}}{\pgfqpoint{5.443957in}{3.694290in}}{\pgfqpoint{5.443957in}{3.699815in}}%
\pgfpathcurveto{\pgfqpoint{5.443957in}{3.705340in}}{\pgfqpoint{5.441762in}{3.710640in}}{\pgfqpoint{5.437855in}{3.714546in}}%
\pgfpathcurveto{\pgfqpoint{5.433948in}{3.718453in}}{\pgfqpoint{5.428649in}{3.720648in}}{\pgfqpoint{5.423123in}{3.720648in}}%
\pgfpathcurveto{\pgfqpoint{5.417598in}{3.720648in}}{\pgfqpoint{5.412299in}{3.718453in}}{\pgfqpoint{5.408392in}{3.714546in}}%
\pgfpathcurveto{\pgfqpoint{5.404485in}{3.710640in}}{\pgfqpoint{5.402290in}{3.705340in}}{\pgfqpoint{5.402290in}{3.699815in}}%
\pgfpathcurveto{\pgfqpoint{5.402290in}{3.694290in}}{\pgfqpoint{5.404485in}{3.688990in}}{\pgfqpoint{5.408392in}{3.685084in}}%
\pgfpathcurveto{\pgfqpoint{5.412299in}{3.681177in}}{\pgfqpoint{5.417598in}{3.678982in}}{\pgfqpoint{5.423123in}{3.678982in}}%
\pgfpathclose%
\pgfusepath{fill}%
\end{pgfscope}%
\begin{pgfscope}%
\pgfpathrectangle{\pgfqpoint{4.376725in}{3.178832in}}{\pgfqpoint{1.162500in}{0.755000in}}%
\pgfusepath{clip}%
\pgfsetbuttcap%
\pgfsetroundjoin%
\definecolor{currentfill}{rgb}{0.000000,0.000000,0.000000}%
\pgfsetfillcolor{currentfill}%
\pgfsetfillopacity{0.500000}%
\pgfsetlinewidth{0.000000pt}%
\definecolor{currentstroke}{rgb}{0.000000,0.000000,0.000000}%
\pgfsetstrokecolor{currentstroke}%
\pgfsetdash{}{0pt}%
\pgfpathmoveto{\pgfqpoint{5.511546in}{3.528997in}}%
\pgfpathcurveto{\pgfqpoint{5.517071in}{3.528997in}}{\pgfqpoint{5.522371in}{3.531192in}}{\pgfqpoint{5.526278in}{3.535099in}}%
\pgfpathcurveto{\pgfqpoint{5.530185in}{3.539005in}}{\pgfqpoint{5.532380in}{3.544305in}}{\pgfqpoint{5.532380in}{3.549830in}}%
\pgfpathcurveto{\pgfqpoint{5.532380in}{3.555355in}}{\pgfqpoint{5.530185in}{3.560655in}}{\pgfqpoint{5.526278in}{3.564561in}}%
\pgfpathcurveto{\pgfqpoint{5.522371in}{3.568468in}}{\pgfqpoint{5.517071in}{3.570663in}}{\pgfqpoint{5.511546in}{3.570663in}}%
\pgfpathcurveto{\pgfqpoint{5.506021in}{3.570663in}}{\pgfqpoint{5.500722in}{3.568468in}}{\pgfqpoint{5.496815in}{3.564561in}}%
\pgfpathcurveto{\pgfqpoint{5.492908in}{3.560655in}}{\pgfqpoint{5.490713in}{3.555355in}}{\pgfqpoint{5.490713in}{3.549830in}}%
\pgfpathcurveto{\pgfqpoint{5.490713in}{3.544305in}}{\pgfqpoint{5.492908in}{3.539005in}}{\pgfqpoint{5.496815in}{3.535099in}}%
\pgfpathcurveto{\pgfqpoint{5.500722in}{3.531192in}}{\pgfqpoint{5.506021in}{3.528997in}}{\pgfqpoint{5.511546in}{3.528997in}}%
\pgfpathclose%
\pgfusepath{fill}%
\end{pgfscope}%
\begin{pgfscope}%
\pgfpathrectangle{\pgfqpoint{4.376725in}{3.178832in}}{\pgfqpoint{1.162500in}{0.755000in}}%
\pgfusepath{clip}%
\pgfsetbuttcap%
\pgfsetroundjoin%
\definecolor{currentfill}{rgb}{0.000000,0.000000,0.000000}%
\pgfsetfillcolor{currentfill}%
\pgfsetfillopacity{0.500000}%
\pgfsetlinewidth{0.000000pt}%
\definecolor{currentstroke}{rgb}{0.000000,0.000000,0.000000}%
\pgfsetstrokecolor{currentstroke}%
\pgfsetdash{}{0pt}%
\pgfpathmoveto{\pgfqpoint{5.286995in}{3.426855in}}%
\pgfpathcurveto{\pgfqpoint{5.292520in}{3.426855in}}{\pgfqpoint{5.297820in}{3.429051in}}{\pgfqpoint{5.301726in}{3.432957in}}%
\pgfpathcurveto{\pgfqpoint{5.305633in}{3.436864in}}{\pgfqpoint{5.307828in}{3.442164in}}{\pgfqpoint{5.307828in}{3.447689in}}%
\pgfpathcurveto{\pgfqpoint{5.307828in}{3.453214in}}{\pgfqpoint{5.305633in}{3.458513in}}{\pgfqpoint{5.301726in}{3.462420in}}%
\pgfpathcurveto{\pgfqpoint{5.297820in}{3.466327in}}{\pgfqpoint{5.292520in}{3.468522in}}{\pgfqpoint{5.286995in}{3.468522in}}%
\pgfpathcurveto{\pgfqpoint{5.281470in}{3.468522in}}{\pgfqpoint{5.276170in}{3.466327in}}{\pgfqpoint{5.272264in}{3.462420in}}%
\pgfpathcurveto{\pgfqpoint{5.268357in}{3.458513in}}{\pgfqpoint{5.266162in}{3.453214in}}{\pgfqpoint{5.266162in}{3.447689in}}%
\pgfpathcurveto{\pgfqpoint{5.266162in}{3.442164in}}{\pgfqpoint{5.268357in}{3.436864in}}{\pgfqpoint{5.272264in}{3.432957in}}%
\pgfpathcurveto{\pgfqpoint{5.276170in}{3.429051in}}{\pgfqpoint{5.281470in}{3.426855in}}{\pgfqpoint{5.286995in}{3.426855in}}%
\pgfpathclose%
\pgfusepath{fill}%
\end{pgfscope}%
\begin{pgfscope}%
\pgfpathrectangle{\pgfqpoint{4.376725in}{3.178832in}}{\pgfqpoint{1.162500in}{0.755000in}}%
\pgfusepath{clip}%
\pgfsetbuttcap%
\pgfsetroundjoin%
\definecolor{currentfill}{rgb}{0.000000,0.000000,0.000000}%
\pgfsetfillcolor{currentfill}%
\pgfsetfillopacity{0.500000}%
\pgfsetlinewidth{0.000000pt}%
\definecolor{currentstroke}{rgb}{0.000000,0.000000,0.000000}%
\pgfsetstrokecolor{currentstroke}%
\pgfsetdash{}{0pt}%
\pgfpathmoveto{\pgfqpoint{4.645822in}{3.677650in}}%
\pgfpathcurveto{\pgfqpoint{4.651348in}{3.677650in}}{\pgfqpoint{4.656647in}{3.679845in}}{\pgfqpoint{4.660554in}{3.683752in}}%
\pgfpathcurveto{\pgfqpoint{4.664461in}{3.687659in}}{\pgfqpoint{4.666656in}{3.692959in}}{\pgfqpoint{4.666656in}{3.698484in}}%
\pgfpathcurveto{\pgfqpoint{4.666656in}{3.704009in}}{\pgfqpoint{4.664461in}{3.709308in}}{\pgfqpoint{4.660554in}{3.713215in}}%
\pgfpathcurveto{\pgfqpoint{4.656647in}{3.717122in}}{\pgfqpoint{4.651348in}{3.719317in}}{\pgfqpoint{4.645822in}{3.719317in}}%
\pgfpathcurveto{\pgfqpoint{4.640297in}{3.719317in}}{\pgfqpoint{4.634998in}{3.717122in}}{\pgfqpoint{4.631091in}{3.713215in}}%
\pgfpathcurveto{\pgfqpoint{4.627184in}{3.709308in}}{\pgfqpoint{4.624989in}{3.704009in}}{\pgfqpoint{4.624989in}{3.698484in}}%
\pgfpathcurveto{\pgfqpoint{4.624989in}{3.692959in}}{\pgfqpoint{4.627184in}{3.687659in}}{\pgfqpoint{4.631091in}{3.683752in}}%
\pgfpathcurveto{\pgfqpoint{4.634998in}{3.679845in}}{\pgfqpoint{4.640297in}{3.677650in}}{\pgfqpoint{4.645822in}{3.677650in}}%
\pgfpathclose%
\pgfusepath{fill}%
\end{pgfscope}%
\begin{pgfscope}%
\pgfpathrectangle{\pgfqpoint{4.376725in}{3.178832in}}{\pgfqpoint{1.162500in}{0.755000in}}%
\pgfusepath{clip}%
\pgfsetbuttcap%
\pgfsetroundjoin%
\definecolor{currentfill}{rgb}{0.000000,0.000000,0.000000}%
\pgfsetfillcolor{currentfill}%
\pgfsetfillopacity{0.500000}%
\pgfsetlinewidth{0.000000pt}%
\definecolor{currentstroke}{rgb}{0.000000,0.000000,0.000000}%
\pgfsetstrokecolor{currentstroke}%
\pgfsetdash{}{0pt}%
\pgfpathmoveto{\pgfqpoint{5.069839in}{3.296446in}}%
\pgfpathcurveto{\pgfqpoint{5.075364in}{3.296446in}}{\pgfqpoint{5.080663in}{3.298641in}}{\pgfqpoint{5.084570in}{3.302548in}}%
\pgfpathcurveto{\pgfqpoint{5.088477in}{3.306455in}}{\pgfqpoint{5.090672in}{3.311754in}}{\pgfqpoint{5.090672in}{3.317279in}}%
\pgfpathcurveto{\pgfqpoint{5.090672in}{3.322804in}}{\pgfqpoint{5.088477in}{3.328104in}}{\pgfqpoint{5.084570in}{3.332011in}}%
\pgfpathcurveto{\pgfqpoint{5.080663in}{3.335917in}}{\pgfqpoint{5.075364in}{3.338112in}}{\pgfqpoint{5.069839in}{3.338112in}}%
\pgfpathcurveto{\pgfqpoint{5.064313in}{3.338112in}}{\pgfqpoint{5.059014in}{3.335917in}}{\pgfqpoint{5.055107in}{3.332011in}}%
\pgfpathcurveto{\pgfqpoint{5.051200in}{3.328104in}}{\pgfqpoint{5.049005in}{3.322804in}}{\pgfqpoint{5.049005in}{3.317279in}}%
\pgfpathcurveto{\pgfqpoint{5.049005in}{3.311754in}}{\pgfqpoint{5.051200in}{3.306455in}}{\pgfqpoint{5.055107in}{3.302548in}}%
\pgfpathcurveto{\pgfqpoint{5.059014in}{3.298641in}}{\pgfqpoint{5.064313in}{3.296446in}}{\pgfqpoint{5.069839in}{3.296446in}}%
\pgfpathclose%
\pgfusepath{fill}%
\end{pgfscope}%
\begin{pgfscope}%
\pgfpathrectangle{\pgfqpoint{4.376725in}{3.178832in}}{\pgfqpoint{1.162500in}{0.755000in}}%
\pgfusepath{clip}%
\pgfsetbuttcap%
\pgfsetroundjoin%
\definecolor{currentfill}{rgb}{0.000000,0.000000,0.000000}%
\pgfsetfillcolor{currentfill}%
\pgfsetfillopacity{0.500000}%
\pgfsetlinewidth{0.000000pt}%
\definecolor{currentstroke}{rgb}{0.000000,0.000000,0.000000}%
\pgfsetstrokecolor{currentstroke}%
\pgfsetdash{}{0pt}%
\pgfpathmoveto{\pgfqpoint{4.941375in}{3.322937in}}%
\pgfpathcurveto{\pgfqpoint{4.946900in}{3.322937in}}{\pgfqpoint{4.952199in}{3.325133in}}{\pgfqpoint{4.956106in}{3.329039in}}%
\pgfpathcurveto{\pgfqpoint{4.960013in}{3.332946in}}{\pgfqpoint{4.962208in}{3.338246in}}{\pgfqpoint{4.962208in}{3.343771in}}%
\pgfpathcurveto{\pgfqpoint{4.962208in}{3.349296in}}{\pgfqpoint{4.960013in}{3.354595in}}{\pgfqpoint{4.956106in}{3.358502in}}%
\pgfpathcurveto{\pgfqpoint{4.952199in}{3.362409in}}{\pgfqpoint{4.946900in}{3.364604in}}{\pgfqpoint{4.941375in}{3.364604in}}%
\pgfpathcurveto{\pgfqpoint{4.935850in}{3.364604in}}{\pgfqpoint{4.930550in}{3.362409in}}{\pgfqpoint{4.926643in}{3.358502in}}%
\pgfpathcurveto{\pgfqpoint{4.922736in}{3.354595in}}{\pgfqpoint{4.920541in}{3.349296in}}{\pgfqpoint{4.920541in}{3.343771in}}%
\pgfpathcurveto{\pgfqpoint{4.920541in}{3.338246in}}{\pgfqpoint{4.922736in}{3.332946in}}{\pgfqpoint{4.926643in}{3.329039in}}%
\pgfpathcurveto{\pgfqpoint{4.930550in}{3.325133in}}{\pgfqpoint{4.935850in}{3.322937in}}{\pgfqpoint{4.941375in}{3.322937in}}%
\pgfpathclose%
\pgfusepath{fill}%
\end{pgfscope}%
\begin{pgfscope}%
\pgfpathrectangle{\pgfqpoint{4.376725in}{3.178832in}}{\pgfqpoint{1.162500in}{0.755000in}}%
\pgfusepath{clip}%
\pgfsetbuttcap%
\pgfsetroundjoin%
\definecolor{currentfill}{rgb}{0.000000,0.000000,0.000000}%
\pgfsetfillcolor{currentfill}%
\pgfsetfillopacity{0.500000}%
\pgfsetlinewidth{0.000000pt}%
\definecolor{currentstroke}{rgb}{0.000000,0.000000,0.000000}%
\pgfsetstrokecolor{currentstroke}%
\pgfsetdash{}{0pt}%
\pgfpathmoveto{\pgfqpoint{4.799153in}{3.895023in}}%
\pgfpathcurveto{\pgfqpoint{4.804678in}{3.895023in}}{\pgfqpoint{4.809977in}{3.897218in}}{\pgfqpoint{4.813884in}{3.901125in}}%
\pgfpathcurveto{\pgfqpoint{4.817791in}{3.905032in}}{\pgfqpoint{4.819986in}{3.910331in}}{\pgfqpoint{4.819986in}{3.915856in}}%
\pgfpathcurveto{\pgfqpoint{4.819986in}{3.921381in}}{\pgfqpoint{4.817791in}{3.926681in}}{\pgfqpoint{4.813884in}{3.930588in}}%
\pgfpathcurveto{\pgfqpoint{4.809977in}{3.934494in}}{\pgfqpoint{4.804678in}{3.936690in}}{\pgfqpoint{4.799153in}{3.936690in}}%
\pgfpathcurveto{\pgfqpoint{4.793628in}{3.936690in}}{\pgfqpoint{4.788328in}{3.934494in}}{\pgfqpoint{4.784421in}{3.930588in}}%
\pgfpathcurveto{\pgfqpoint{4.780515in}{3.926681in}}{\pgfqpoint{4.778320in}{3.921381in}}{\pgfqpoint{4.778320in}{3.915856in}}%
\pgfpathcurveto{\pgfqpoint{4.778320in}{3.910331in}}{\pgfqpoint{4.780515in}{3.905032in}}{\pgfqpoint{4.784421in}{3.901125in}}%
\pgfpathcurveto{\pgfqpoint{4.788328in}{3.897218in}}{\pgfqpoint{4.793628in}{3.895023in}}{\pgfqpoint{4.799153in}{3.895023in}}%
\pgfpathclose%
\pgfusepath{fill}%
\end{pgfscope}%
\begin{pgfscope}%
\pgfpathrectangle{\pgfqpoint{4.376725in}{3.178832in}}{\pgfqpoint{1.162500in}{0.755000in}}%
\pgfusepath{clip}%
\pgfsetbuttcap%
\pgfsetroundjoin%
\definecolor{currentfill}{rgb}{0.000000,0.000000,0.000000}%
\pgfsetfillcolor{currentfill}%
\pgfsetfillopacity{0.500000}%
\pgfsetlinewidth{0.000000pt}%
\definecolor{currentstroke}{rgb}{0.000000,0.000000,0.000000}%
\pgfsetstrokecolor{currentstroke}%
\pgfsetdash{}{0pt}%
\pgfpathmoveto{\pgfqpoint{4.595218in}{3.491212in}}%
\pgfpathcurveto{\pgfqpoint{4.600743in}{3.491212in}}{\pgfqpoint{4.606043in}{3.493407in}}{\pgfqpoint{4.609950in}{3.497314in}}%
\pgfpathcurveto{\pgfqpoint{4.613857in}{3.501221in}}{\pgfqpoint{4.616052in}{3.506520in}}{\pgfqpoint{4.616052in}{3.512045in}}%
\pgfpathcurveto{\pgfqpoint{4.616052in}{3.517571in}}{\pgfqpoint{4.613857in}{3.522870in}}{\pgfqpoint{4.609950in}{3.526777in}}%
\pgfpathcurveto{\pgfqpoint{4.606043in}{3.530684in}}{\pgfqpoint{4.600743in}{3.532879in}}{\pgfqpoint{4.595218in}{3.532879in}}%
\pgfpathcurveto{\pgfqpoint{4.589693in}{3.532879in}}{\pgfqpoint{4.584394in}{3.530684in}}{\pgfqpoint{4.580487in}{3.526777in}}%
\pgfpathcurveto{\pgfqpoint{4.576580in}{3.522870in}}{\pgfqpoint{4.574385in}{3.517571in}}{\pgfqpoint{4.574385in}{3.512045in}}%
\pgfpathcurveto{\pgfqpoint{4.574385in}{3.506520in}}{\pgfqpoint{4.576580in}{3.501221in}}{\pgfqpoint{4.580487in}{3.497314in}}%
\pgfpathcurveto{\pgfqpoint{4.584394in}{3.493407in}}{\pgfqpoint{4.589693in}{3.491212in}}{\pgfqpoint{4.595218in}{3.491212in}}%
\pgfpathclose%
\pgfusepath{fill}%
\end{pgfscope}%
\begin{pgfscope}%
\pgfpathrectangle{\pgfqpoint{4.376725in}{3.178832in}}{\pgfqpoint{1.162500in}{0.755000in}}%
\pgfusepath{clip}%
\pgfsetbuttcap%
\pgfsetroundjoin%
\definecolor{currentfill}{rgb}{0.000000,0.000000,0.000000}%
\pgfsetfillcolor{currentfill}%
\pgfsetfillopacity{0.500000}%
\pgfsetlinewidth{0.000000pt}%
\definecolor{currentstroke}{rgb}{0.000000,0.000000,0.000000}%
\pgfsetstrokecolor{currentstroke}%
\pgfsetdash{}{0pt}%
\pgfpathmoveto{\pgfqpoint{4.834270in}{3.212721in}}%
\pgfpathcurveto{\pgfqpoint{4.839795in}{3.212721in}}{\pgfqpoint{4.845094in}{3.214917in}}{\pgfqpoint{4.849001in}{3.218823in}}%
\pgfpathcurveto{\pgfqpoint{4.852908in}{3.222730in}}{\pgfqpoint{4.855103in}{3.228030in}}{\pgfqpoint{4.855103in}{3.233555in}}%
\pgfpathcurveto{\pgfqpoint{4.855103in}{3.239080in}}{\pgfqpoint{4.852908in}{3.244379in}}{\pgfqpoint{4.849001in}{3.248286in}}%
\pgfpathcurveto{\pgfqpoint{4.845094in}{3.252193in}}{\pgfqpoint{4.839795in}{3.254388in}}{\pgfqpoint{4.834270in}{3.254388in}}%
\pgfpathcurveto{\pgfqpoint{4.828745in}{3.254388in}}{\pgfqpoint{4.823445in}{3.252193in}}{\pgfqpoint{4.819538in}{3.248286in}}%
\pgfpathcurveto{\pgfqpoint{4.815632in}{3.244379in}}{\pgfqpoint{4.813437in}{3.239080in}}{\pgfqpoint{4.813437in}{3.233555in}}%
\pgfpathcurveto{\pgfqpoint{4.813437in}{3.228030in}}{\pgfqpoint{4.815632in}{3.222730in}}{\pgfqpoint{4.819538in}{3.218823in}}%
\pgfpathcurveto{\pgfqpoint{4.823445in}{3.214917in}}{\pgfqpoint{4.828745in}{3.212721in}}{\pgfqpoint{4.834270in}{3.212721in}}%
\pgfpathclose%
\pgfusepath{fill}%
\end{pgfscope}%
\begin{pgfscope}%
\pgfsetrectcap%
\pgfsetmiterjoin%
\pgfsetlinewidth{0.803000pt}%
\definecolor{currentstroke}{rgb}{0.501961,0.501961,0.501961}%
\pgfsetstrokecolor{currentstroke}%
\pgfsetdash{}{0pt}%
\pgfpathmoveto{\pgfqpoint{4.376725in}{3.178832in}}%
\pgfpathlineto{\pgfqpoint{4.376725in}{3.933832in}}%
\pgfusepath{stroke}%
\end{pgfscope}%
\begin{pgfscope}%
\pgfsetrectcap%
\pgfsetmiterjoin%
\pgfsetlinewidth{0.803000pt}%
\definecolor{currentstroke}{rgb}{0.501961,0.501961,0.501961}%
\pgfsetstrokecolor{currentstroke}%
\pgfsetdash{}{0pt}%
\pgfpathmoveto{\pgfqpoint{5.539225in}{3.178832in}}%
\pgfpathlineto{\pgfqpoint{5.539225in}{3.933832in}}%
\pgfusepath{stroke}%
\end{pgfscope}%
\begin{pgfscope}%
\pgfsetrectcap%
\pgfsetmiterjoin%
\pgfsetlinewidth{0.803000pt}%
\definecolor{currentstroke}{rgb}{0.501961,0.501961,0.501961}%
\pgfsetstrokecolor{currentstroke}%
\pgfsetdash{}{0pt}%
\pgfpathmoveto{\pgfqpoint{4.376725in}{3.178832in}}%
\pgfpathlineto{\pgfqpoint{5.539225in}{3.178832in}}%
\pgfusepath{stroke}%
\end{pgfscope}%
\begin{pgfscope}%
\pgfsetrectcap%
\pgfsetmiterjoin%
\pgfsetlinewidth{0.803000pt}%
\definecolor{currentstroke}{rgb}{0.501961,0.501961,0.501961}%
\pgfsetstrokecolor{currentstroke}%
\pgfsetdash{}{0pt}%
\pgfpathmoveto{\pgfqpoint{4.376725in}{3.933832in}}%
\pgfpathlineto{\pgfqpoint{5.539225in}{3.933832in}}%
\pgfusepath{stroke}%
\end{pgfscope}%
\begin{pgfscope}%
\pgfsetbuttcap%
\pgfsetmiterjoin%
\definecolor{currentfill}{rgb}{1.000000,1.000000,1.000000}%
\pgfsetfillcolor{currentfill}%
\pgfsetlinewidth{0.000000pt}%
\definecolor{currentstroke}{rgb}{0.000000,0.000000,0.000000}%
\pgfsetstrokecolor{currentstroke}%
\pgfsetstrokeopacity{0.000000}%
\pgfsetdash{}{0pt}%
\pgfpathmoveto{\pgfqpoint{0.889225in}{2.423832in}}%
\pgfpathlineto{\pgfqpoint{2.051725in}{2.423832in}}%
\pgfpathlineto{\pgfqpoint{2.051725in}{3.178832in}}%
\pgfpathlineto{\pgfqpoint{0.889225in}{3.178832in}}%
\pgfpathclose%
\pgfusepath{fill}%
\end{pgfscope}%
\begin{pgfscope}%
\pgfpathrectangle{\pgfqpoint{0.889225in}{2.423832in}}{\pgfqpoint{1.162500in}{0.755000in}}%
\pgfusepath{clip}%
\pgfsetbuttcap%
\pgfsetroundjoin%
\definecolor{currentfill}{rgb}{0.000000,0.000000,0.000000}%
\pgfsetfillcolor{currentfill}%
\pgfsetfillopacity{0.500000}%
\pgfsetlinewidth{0.000000pt}%
\definecolor{currentstroke}{rgb}{0.000000,0.000000,0.000000}%
\pgfsetstrokecolor{currentstroke}%
\pgfsetdash{}{0pt}%
\pgfpathmoveto{\pgfqpoint{1.893746in}{3.088730in}}%
\pgfpathcurveto{\pgfqpoint{1.899271in}{3.088730in}}{\pgfqpoint{1.904570in}{3.090925in}}{\pgfqpoint{1.908477in}{3.094832in}}%
\pgfpathcurveto{\pgfqpoint{1.912384in}{3.098739in}}{\pgfqpoint{1.914579in}{3.104038in}}{\pgfqpoint{1.914579in}{3.109563in}}%
\pgfpathcurveto{\pgfqpoint{1.914579in}{3.115089in}}{\pgfqpoint{1.912384in}{3.120388in}}{\pgfqpoint{1.908477in}{3.124295in}}%
\pgfpathcurveto{\pgfqpoint{1.904570in}{3.128202in}}{\pgfqpoint{1.899271in}{3.130397in}}{\pgfqpoint{1.893746in}{3.130397in}}%
\pgfpathcurveto{\pgfqpoint{1.888221in}{3.130397in}}{\pgfqpoint{1.882921in}{3.128202in}}{\pgfqpoint{1.879015in}{3.124295in}}%
\pgfpathcurveto{\pgfqpoint{1.875108in}{3.120388in}}{\pgfqpoint{1.872913in}{3.115089in}}{\pgfqpoint{1.872913in}{3.109563in}}%
\pgfpathcurveto{\pgfqpoint{1.872913in}{3.104038in}}{\pgfqpoint{1.875108in}{3.098739in}}{\pgfqpoint{1.879015in}{3.094832in}}%
\pgfpathcurveto{\pgfqpoint{1.882921in}{3.090925in}}{\pgfqpoint{1.888221in}{3.088730in}}{\pgfqpoint{1.893746in}{3.088730in}}%
\pgfpathclose%
\pgfusepath{fill}%
\end{pgfscope}%
\begin{pgfscope}%
\pgfpathrectangle{\pgfqpoint{0.889225in}{2.423832in}}{\pgfqpoint{1.162500in}{0.755000in}}%
\pgfusepath{clip}%
\pgfsetbuttcap%
\pgfsetroundjoin%
\definecolor{currentfill}{rgb}{0.000000,0.000000,0.000000}%
\pgfsetfillcolor{currentfill}%
\pgfsetfillopacity{0.500000}%
\pgfsetlinewidth{0.000000pt}%
\definecolor{currentstroke}{rgb}{0.000000,0.000000,0.000000}%
\pgfsetstrokecolor{currentstroke}%
\pgfsetdash{}{0pt}%
\pgfpathmoveto{\pgfqpoint{1.476892in}{2.544873in}}%
\pgfpathcurveto{\pgfqpoint{1.482417in}{2.544873in}}{\pgfqpoint{1.487717in}{2.547068in}}{\pgfqpoint{1.491623in}{2.550975in}}%
\pgfpathcurveto{\pgfqpoint{1.495530in}{2.554881in}}{\pgfqpoint{1.497725in}{2.560181in}}{\pgfqpoint{1.497725in}{2.565706in}}%
\pgfpathcurveto{\pgfqpoint{1.497725in}{2.571231in}}{\pgfqpoint{1.495530in}{2.576531in}}{\pgfqpoint{1.491623in}{2.580437in}}%
\pgfpathcurveto{\pgfqpoint{1.487717in}{2.584344in}}{\pgfqpoint{1.482417in}{2.586539in}}{\pgfqpoint{1.476892in}{2.586539in}}%
\pgfpathcurveto{\pgfqpoint{1.471367in}{2.586539in}}{\pgfqpoint{1.466067in}{2.584344in}}{\pgfqpoint{1.462161in}{2.580437in}}%
\pgfpathcurveto{\pgfqpoint{1.458254in}{2.576531in}}{\pgfqpoint{1.456059in}{2.571231in}}{\pgfqpoint{1.456059in}{2.565706in}}%
\pgfpathcurveto{\pgfqpoint{1.456059in}{2.560181in}}{\pgfqpoint{1.458254in}{2.554881in}}{\pgfqpoint{1.462161in}{2.550975in}}%
\pgfpathcurveto{\pgfqpoint{1.466067in}{2.547068in}}{\pgfqpoint{1.471367in}{2.544873in}}{\pgfqpoint{1.476892in}{2.544873in}}%
\pgfpathclose%
\pgfusepath{fill}%
\end{pgfscope}%
\begin{pgfscope}%
\pgfpathrectangle{\pgfqpoint{0.889225in}{2.423832in}}{\pgfqpoint{1.162500in}{0.755000in}}%
\pgfusepath{clip}%
\pgfsetbuttcap%
\pgfsetroundjoin%
\definecolor{currentfill}{rgb}{0.000000,0.000000,0.000000}%
\pgfsetfillcolor{currentfill}%
\pgfsetfillopacity{0.500000}%
\pgfsetlinewidth{0.000000pt}%
\definecolor{currentstroke}{rgb}{0.000000,0.000000,0.000000}%
\pgfsetstrokecolor{currentstroke}%
\pgfsetdash{}{0pt}%
\pgfpathmoveto{\pgfqpoint{1.478974in}{2.541427in}}%
\pgfpathcurveto{\pgfqpoint{1.484499in}{2.541427in}}{\pgfqpoint{1.489799in}{2.543622in}}{\pgfqpoint{1.493705in}{2.547529in}}%
\pgfpathcurveto{\pgfqpoint{1.497612in}{2.551436in}}{\pgfqpoint{1.499807in}{2.556735in}}{\pgfqpoint{1.499807in}{2.562260in}}%
\pgfpathcurveto{\pgfqpoint{1.499807in}{2.567785in}}{\pgfqpoint{1.497612in}{2.573085in}}{\pgfqpoint{1.493705in}{2.576992in}}%
\pgfpathcurveto{\pgfqpoint{1.489799in}{2.580899in}}{\pgfqpoint{1.484499in}{2.583094in}}{\pgfqpoint{1.478974in}{2.583094in}}%
\pgfpathcurveto{\pgfqpoint{1.473449in}{2.583094in}}{\pgfqpoint{1.468149in}{2.580899in}}{\pgfqpoint{1.464243in}{2.576992in}}%
\pgfpathcurveto{\pgfqpoint{1.460336in}{2.573085in}}{\pgfqpoint{1.458141in}{2.567785in}}{\pgfqpoint{1.458141in}{2.562260in}}%
\pgfpathcurveto{\pgfqpoint{1.458141in}{2.556735in}}{\pgfqpoint{1.460336in}{2.551436in}}{\pgfqpoint{1.464243in}{2.547529in}}%
\pgfpathcurveto{\pgfqpoint{1.468149in}{2.543622in}}{\pgfqpoint{1.473449in}{2.541427in}}{\pgfqpoint{1.478974in}{2.541427in}}%
\pgfpathclose%
\pgfusepath{fill}%
\end{pgfscope}%
\begin{pgfscope}%
\pgfpathrectangle{\pgfqpoint{0.889225in}{2.423832in}}{\pgfqpoint{1.162500in}{0.755000in}}%
\pgfusepath{clip}%
\pgfsetbuttcap%
\pgfsetroundjoin%
\definecolor{currentfill}{rgb}{0.000000,0.000000,0.000000}%
\pgfsetfillcolor{currentfill}%
\pgfsetfillopacity{0.500000}%
\pgfsetlinewidth{0.000000pt}%
\definecolor{currentstroke}{rgb}{0.000000,0.000000,0.000000}%
\pgfsetstrokecolor{currentstroke}%
\pgfsetdash{}{0pt}%
\pgfpathmoveto{\pgfqpoint{1.120714in}{2.469548in}}%
\pgfpathcurveto{\pgfqpoint{1.126239in}{2.469548in}}{\pgfqpoint{1.131538in}{2.471743in}}{\pgfqpoint{1.135445in}{2.475650in}}%
\pgfpathcurveto{\pgfqpoint{1.139352in}{2.479557in}}{\pgfqpoint{1.141547in}{2.484856in}}{\pgfqpoint{1.141547in}{2.490381in}}%
\pgfpathcurveto{\pgfqpoint{1.141547in}{2.495907in}}{\pgfqpoint{1.139352in}{2.501206in}}{\pgfqpoint{1.135445in}{2.505113in}}%
\pgfpathcurveto{\pgfqpoint{1.131538in}{2.509020in}}{\pgfqpoint{1.126239in}{2.511215in}}{\pgfqpoint{1.120714in}{2.511215in}}%
\pgfpathcurveto{\pgfqpoint{1.115189in}{2.511215in}}{\pgfqpoint{1.109889in}{2.509020in}}{\pgfqpoint{1.105982in}{2.505113in}}%
\pgfpathcurveto{\pgfqpoint{1.102076in}{2.501206in}}{\pgfqpoint{1.099881in}{2.495907in}}{\pgfqpoint{1.099881in}{2.490381in}}%
\pgfpathcurveto{\pgfqpoint{1.099881in}{2.484856in}}{\pgfqpoint{1.102076in}{2.479557in}}{\pgfqpoint{1.105982in}{2.475650in}}%
\pgfpathcurveto{\pgfqpoint{1.109889in}{2.471743in}}{\pgfqpoint{1.115189in}{2.469548in}}{\pgfqpoint{1.120714in}{2.469548in}}%
\pgfpathclose%
\pgfusepath{fill}%
\end{pgfscope}%
\begin{pgfscope}%
\pgfpathrectangle{\pgfqpoint{0.889225in}{2.423832in}}{\pgfqpoint{1.162500in}{0.755000in}}%
\pgfusepath{clip}%
\pgfsetbuttcap%
\pgfsetroundjoin%
\definecolor{currentfill}{rgb}{0.000000,0.000000,0.000000}%
\pgfsetfillcolor{currentfill}%
\pgfsetfillopacity{0.500000}%
\pgfsetlinewidth{0.000000pt}%
\definecolor{currentstroke}{rgb}{0.000000,0.000000,0.000000}%
\pgfsetstrokecolor{currentstroke}%
\pgfsetdash{}{0pt}%
\pgfpathmoveto{\pgfqpoint{1.126248in}{2.488481in}}%
\pgfpathcurveto{\pgfqpoint{1.131773in}{2.488481in}}{\pgfqpoint{1.137073in}{2.490676in}}{\pgfqpoint{1.140980in}{2.494583in}}%
\pgfpathcurveto{\pgfqpoint{1.144886in}{2.498489in}}{\pgfqpoint{1.147082in}{2.503789in}}{\pgfqpoint{1.147082in}{2.509314in}}%
\pgfpathcurveto{\pgfqpoint{1.147082in}{2.514839in}}{\pgfqpoint{1.144886in}{2.520139in}}{\pgfqpoint{1.140980in}{2.524045in}}%
\pgfpathcurveto{\pgfqpoint{1.137073in}{2.527952in}}{\pgfqpoint{1.131773in}{2.530147in}}{\pgfqpoint{1.126248in}{2.530147in}}%
\pgfpathcurveto{\pgfqpoint{1.120723in}{2.530147in}}{\pgfqpoint{1.115424in}{2.527952in}}{\pgfqpoint{1.111517in}{2.524045in}}%
\pgfpathcurveto{\pgfqpoint{1.107610in}{2.520139in}}{\pgfqpoint{1.105415in}{2.514839in}}{\pgfqpoint{1.105415in}{2.509314in}}%
\pgfpathcurveto{\pgfqpoint{1.105415in}{2.503789in}}{\pgfqpoint{1.107610in}{2.498489in}}{\pgfqpoint{1.111517in}{2.494583in}}%
\pgfpathcurveto{\pgfqpoint{1.115424in}{2.490676in}}{\pgfqpoint{1.120723in}{2.488481in}}{\pgfqpoint{1.126248in}{2.488481in}}%
\pgfpathclose%
\pgfusepath{fill}%
\end{pgfscope}%
\begin{pgfscope}%
\pgfpathrectangle{\pgfqpoint{0.889225in}{2.423832in}}{\pgfqpoint{1.162500in}{0.755000in}}%
\pgfusepath{clip}%
\pgfsetbuttcap%
\pgfsetroundjoin%
\definecolor{currentfill}{rgb}{0.000000,0.000000,0.000000}%
\pgfsetfillcolor{currentfill}%
\pgfsetfillopacity{0.500000}%
\pgfsetlinewidth{0.000000pt}%
\definecolor{currentstroke}{rgb}{0.000000,0.000000,0.000000}%
\pgfsetstrokecolor{currentstroke}%
\pgfsetdash{}{0pt}%
\pgfpathmoveto{\pgfqpoint{0.977704in}{2.422812in}}%
\pgfpathcurveto{\pgfqpoint{0.983229in}{2.422812in}}{\pgfqpoint{0.988528in}{2.425007in}}{\pgfqpoint{0.992435in}{2.428914in}}%
\pgfpathcurveto{\pgfqpoint{0.996342in}{2.432821in}}{\pgfqpoint{0.998537in}{2.438120in}}{\pgfqpoint{0.998537in}{2.443645in}}%
\pgfpathcurveto{\pgfqpoint{0.998537in}{2.449170in}}{\pgfqpoint{0.996342in}{2.454470in}}{\pgfqpoint{0.992435in}{2.458377in}}%
\pgfpathcurveto{\pgfqpoint{0.988528in}{2.462284in}}{\pgfqpoint{0.983229in}{2.464479in}}{\pgfqpoint{0.977704in}{2.464479in}}%
\pgfpathcurveto{\pgfqpoint{0.972179in}{2.464479in}}{\pgfqpoint{0.966879in}{2.462284in}}{\pgfqpoint{0.962972in}{2.458377in}}%
\pgfpathcurveto{\pgfqpoint{0.959066in}{2.454470in}}{\pgfqpoint{0.956870in}{2.449170in}}{\pgfqpoint{0.956870in}{2.443645in}}%
\pgfpathcurveto{\pgfqpoint{0.956870in}{2.438120in}}{\pgfqpoint{0.959066in}{2.432821in}}{\pgfqpoint{0.962972in}{2.428914in}}%
\pgfpathcurveto{\pgfqpoint{0.966879in}{2.425007in}}{\pgfqpoint{0.972179in}{2.422812in}}{\pgfqpoint{0.977704in}{2.422812in}}%
\pgfpathclose%
\pgfusepath{fill}%
\end{pgfscope}%
\begin{pgfscope}%
\pgfpathrectangle{\pgfqpoint{0.889225in}{2.423832in}}{\pgfqpoint{1.162500in}{0.755000in}}%
\pgfusepath{clip}%
\pgfsetbuttcap%
\pgfsetroundjoin%
\definecolor{currentfill}{rgb}{0.000000,0.000000,0.000000}%
\pgfsetfillcolor{currentfill}%
\pgfsetfillopacity{0.500000}%
\pgfsetlinewidth{0.000000pt}%
\definecolor{currentstroke}{rgb}{0.000000,0.000000,0.000000}%
\pgfsetstrokecolor{currentstroke}%
\pgfsetdash{}{0pt}%
\pgfpathmoveto{\pgfqpoint{0.916904in}{2.420975in}}%
\pgfpathcurveto{\pgfqpoint{0.922429in}{2.420975in}}{\pgfqpoint{0.927728in}{2.423170in}}{\pgfqpoint{0.931635in}{2.427077in}}%
\pgfpathcurveto{\pgfqpoint{0.935542in}{2.430984in}}{\pgfqpoint{0.937737in}{2.436283in}}{\pgfqpoint{0.937737in}{2.441809in}}%
\pgfpathcurveto{\pgfqpoint{0.937737in}{2.447334in}}{\pgfqpoint{0.935542in}{2.452633in}}{\pgfqpoint{0.931635in}{2.456540in}}%
\pgfpathcurveto{\pgfqpoint{0.927728in}{2.460447in}}{\pgfqpoint{0.922429in}{2.462642in}}{\pgfqpoint{0.916904in}{2.462642in}}%
\pgfpathcurveto{\pgfqpoint{0.911379in}{2.462642in}}{\pgfqpoint{0.906079in}{2.460447in}}{\pgfqpoint{0.902172in}{2.456540in}}%
\pgfpathcurveto{\pgfqpoint{0.898265in}{2.452633in}}{\pgfqpoint{0.896070in}{2.447334in}}{\pgfqpoint{0.896070in}{2.441809in}}%
\pgfpathcurveto{\pgfqpoint{0.896070in}{2.436283in}}{\pgfqpoint{0.898265in}{2.430984in}}{\pgfqpoint{0.902172in}{2.427077in}}%
\pgfpathcurveto{\pgfqpoint{0.906079in}{2.423170in}}{\pgfqpoint{0.911379in}{2.420975in}}{\pgfqpoint{0.916904in}{2.420975in}}%
\pgfpathclose%
\pgfusepath{fill}%
\end{pgfscope}%
\begin{pgfscope}%
\pgfpathrectangle{\pgfqpoint{0.889225in}{2.423832in}}{\pgfqpoint{1.162500in}{0.755000in}}%
\pgfusepath{clip}%
\pgfsetbuttcap%
\pgfsetroundjoin%
\definecolor{currentfill}{rgb}{0.000000,0.000000,0.000000}%
\pgfsetfillcolor{currentfill}%
\pgfsetfillopacity{0.500000}%
\pgfsetlinewidth{0.000000pt}%
\definecolor{currentstroke}{rgb}{0.000000,0.000000,0.000000}%
\pgfsetstrokecolor{currentstroke}%
\pgfsetdash{}{0pt}%
\pgfpathmoveto{\pgfqpoint{1.691400in}{2.717961in}}%
\pgfpathcurveto{\pgfqpoint{1.696925in}{2.717961in}}{\pgfqpoint{1.702225in}{2.720156in}}{\pgfqpoint{1.706132in}{2.724063in}}%
\pgfpathcurveto{\pgfqpoint{1.710038in}{2.727970in}}{\pgfqpoint{1.712234in}{2.733269in}}{\pgfqpoint{1.712234in}{2.738795in}}%
\pgfpathcurveto{\pgfqpoint{1.712234in}{2.744320in}}{\pgfqpoint{1.710038in}{2.749619in}}{\pgfqpoint{1.706132in}{2.753526in}}%
\pgfpathcurveto{\pgfqpoint{1.702225in}{2.757433in}}{\pgfqpoint{1.696925in}{2.759628in}}{\pgfqpoint{1.691400in}{2.759628in}}%
\pgfpathcurveto{\pgfqpoint{1.685875in}{2.759628in}}{\pgfqpoint{1.680576in}{2.757433in}}{\pgfqpoint{1.676669in}{2.753526in}}%
\pgfpathcurveto{\pgfqpoint{1.672762in}{2.749619in}}{\pgfqpoint{1.670567in}{2.744320in}}{\pgfqpoint{1.670567in}{2.738795in}}%
\pgfpathcurveto{\pgfqpoint{1.670567in}{2.733269in}}{\pgfqpoint{1.672762in}{2.727970in}}{\pgfqpoint{1.676669in}{2.724063in}}%
\pgfpathcurveto{\pgfqpoint{1.680576in}{2.720156in}}{\pgfqpoint{1.685875in}{2.717961in}}{\pgfqpoint{1.691400in}{2.717961in}}%
\pgfpathclose%
\pgfusepath{fill}%
\end{pgfscope}%
\begin{pgfscope}%
\pgfpathrectangle{\pgfqpoint{0.889225in}{2.423832in}}{\pgfqpoint{1.162500in}{0.755000in}}%
\pgfusepath{clip}%
\pgfsetbuttcap%
\pgfsetroundjoin%
\definecolor{currentfill}{rgb}{0.000000,0.000000,0.000000}%
\pgfsetfillcolor{currentfill}%
\pgfsetfillopacity{0.500000}%
\pgfsetlinewidth{0.000000pt}%
\definecolor{currentstroke}{rgb}{0.000000,0.000000,0.000000}%
\pgfsetstrokecolor{currentstroke}%
\pgfsetdash{}{0pt}%
\pgfpathmoveto{\pgfqpoint{1.460463in}{2.640479in}}%
\pgfpathcurveto{\pgfqpoint{1.465988in}{2.640479in}}{\pgfqpoint{1.471288in}{2.642674in}}{\pgfqpoint{1.475194in}{2.646581in}}%
\pgfpathcurveto{\pgfqpoint{1.479101in}{2.650487in}}{\pgfqpoint{1.481296in}{2.655787in}}{\pgfqpoint{1.481296in}{2.661312in}}%
\pgfpathcurveto{\pgfqpoint{1.481296in}{2.666837in}}{\pgfqpoint{1.479101in}{2.672137in}}{\pgfqpoint{1.475194in}{2.676043in}}%
\pgfpathcurveto{\pgfqpoint{1.471288in}{2.679950in}}{\pgfqpoint{1.465988in}{2.682145in}}{\pgfqpoint{1.460463in}{2.682145in}}%
\pgfpathcurveto{\pgfqpoint{1.454938in}{2.682145in}}{\pgfqpoint{1.449638in}{2.679950in}}{\pgfqpoint{1.445732in}{2.676043in}}%
\pgfpathcurveto{\pgfqpoint{1.441825in}{2.672137in}}{\pgfqpoint{1.439630in}{2.666837in}}{\pgfqpoint{1.439630in}{2.661312in}}%
\pgfpathcurveto{\pgfqpoint{1.439630in}{2.655787in}}{\pgfqpoint{1.441825in}{2.650487in}}{\pgfqpoint{1.445732in}{2.646581in}}%
\pgfpathcurveto{\pgfqpoint{1.449638in}{2.642674in}}{\pgfqpoint{1.454938in}{2.640479in}}{\pgfqpoint{1.460463in}{2.640479in}}%
\pgfpathclose%
\pgfusepath{fill}%
\end{pgfscope}%
\begin{pgfscope}%
\pgfpathrectangle{\pgfqpoint{0.889225in}{2.423832in}}{\pgfqpoint{1.162500in}{0.755000in}}%
\pgfusepath{clip}%
\pgfsetbuttcap%
\pgfsetroundjoin%
\definecolor{currentfill}{rgb}{0.000000,0.000000,0.000000}%
\pgfsetfillcolor{currentfill}%
\pgfsetfillopacity{0.500000}%
\pgfsetlinewidth{0.000000pt}%
\definecolor{currentstroke}{rgb}{0.000000,0.000000,0.000000}%
\pgfsetstrokecolor{currentstroke}%
\pgfsetdash{}{0pt}%
\pgfpathmoveto{\pgfqpoint{1.303193in}{2.637393in}}%
\pgfpathcurveto{\pgfqpoint{1.308718in}{2.637393in}}{\pgfqpoint{1.314017in}{2.639588in}}{\pgfqpoint{1.317924in}{2.643495in}}%
\pgfpathcurveto{\pgfqpoint{1.321831in}{2.647402in}}{\pgfqpoint{1.324026in}{2.652701in}}{\pgfqpoint{1.324026in}{2.658226in}}%
\pgfpathcurveto{\pgfqpoint{1.324026in}{2.663751in}}{\pgfqpoint{1.321831in}{2.669051in}}{\pgfqpoint{1.317924in}{2.672958in}}%
\pgfpathcurveto{\pgfqpoint{1.314017in}{2.676865in}}{\pgfqpoint{1.308718in}{2.679060in}}{\pgfqpoint{1.303193in}{2.679060in}}%
\pgfpathcurveto{\pgfqpoint{1.297667in}{2.679060in}}{\pgfqpoint{1.292368in}{2.676865in}}{\pgfqpoint{1.288461in}{2.672958in}}%
\pgfpathcurveto{\pgfqpoint{1.284554in}{2.669051in}}{\pgfqpoint{1.282359in}{2.663751in}}{\pgfqpoint{1.282359in}{2.658226in}}%
\pgfpathcurveto{\pgfqpoint{1.282359in}{2.652701in}}{\pgfqpoint{1.284554in}{2.647402in}}{\pgfqpoint{1.288461in}{2.643495in}}%
\pgfpathcurveto{\pgfqpoint{1.292368in}{2.639588in}}{\pgfqpoint{1.297667in}{2.637393in}}{\pgfqpoint{1.303193in}{2.637393in}}%
\pgfpathclose%
\pgfusepath{fill}%
\end{pgfscope}%
\begin{pgfscope}%
\pgfpathrectangle{\pgfqpoint{0.889225in}{2.423832in}}{\pgfqpoint{1.162500in}{0.755000in}}%
\pgfusepath{clip}%
\pgfsetbuttcap%
\pgfsetroundjoin%
\definecolor{currentfill}{rgb}{0.000000,0.000000,0.000000}%
\pgfsetfillcolor{currentfill}%
\pgfsetfillopacity{0.500000}%
\pgfsetlinewidth{0.000000pt}%
\definecolor{currentstroke}{rgb}{0.000000,0.000000,0.000000}%
\pgfsetstrokecolor{currentstroke}%
\pgfsetdash{}{0pt}%
\pgfpathmoveto{\pgfqpoint{1.689350in}{2.679970in}}%
\pgfpathcurveto{\pgfqpoint{1.694875in}{2.679970in}}{\pgfqpoint{1.700175in}{2.682166in}}{\pgfqpoint{1.704082in}{2.686072in}}%
\pgfpathcurveto{\pgfqpoint{1.707989in}{2.689979in}}{\pgfqpoint{1.710184in}{2.695279in}}{\pgfqpoint{1.710184in}{2.700804in}}%
\pgfpathcurveto{\pgfqpoint{1.710184in}{2.706329in}}{\pgfqpoint{1.707989in}{2.711628in}}{\pgfqpoint{1.704082in}{2.715535in}}%
\pgfpathcurveto{\pgfqpoint{1.700175in}{2.719442in}}{\pgfqpoint{1.694875in}{2.721637in}}{\pgfqpoint{1.689350in}{2.721637in}}%
\pgfpathcurveto{\pgfqpoint{1.683825in}{2.721637in}}{\pgfqpoint{1.678526in}{2.719442in}}{\pgfqpoint{1.674619in}{2.715535in}}%
\pgfpathcurveto{\pgfqpoint{1.670712in}{2.711628in}}{\pgfqpoint{1.668517in}{2.706329in}}{\pgfqpoint{1.668517in}{2.700804in}}%
\pgfpathcurveto{\pgfqpoint{1.668517in}{2.695279in}}{\pgfqpoint{1.670712in}{2.689979in}}{\pgfqpoint{1.674619in}{2.686072in}}%
\pgfpathcurveto{\pgfqpoint{1.678526in}{2.682166in}}{\pgfqpoint{1.683825in}{2.679970in}}{\pgfqpoint{1.689350in}{2.679970in}}%
\pgfpathclose%
\pgfusepath{fill}%
\end{pgfscope}%
\begin{pgfscope}%
\pgfpathrectangle{\pgfqpoint{0.889225in}{2.423832in}}{\pgfqpoint{1.162500in}{0.755000in}}%
\pgfusepath{clip}%
\pgfsetbuttcap%
\pgfsetroundjoin%
\definecolor{currentfill}{rgb}{0.000000,0.000000,0.000000}%
\pgfsetfillcolor{currentfill}%
\pgfsetfillopacity{0.500000}%
\pgfsetlinewidth{0.000000pt}%
\definecolor{currentstroke}{rgb}{0.000000,0.000000,0.000000}%
\pgfsetstrokecolor{currentstroke}%
\pgfsetdash{}{0pt}%
\pgfpathmoveto{\pgfqpoint{1.102396in}{2.481302in}}%
\pgfpathcurveto{\pgfqpoint{1.107921in}{2.481302in}}{\pgfqpoint{1.113221in}{2.483497in}}{\pgfqpoint{1.117128in}{2.487404in}}%
\pgfpathcurveto{\pgfqpoint{1.121035in}{2.491311in}}{\pgfqpoint{1.123230in}{2.496610in}}{\pgfqpoint{1.123230in}{2.502135in}}%
\pgfpathcurveto{\pgfqpoint{1.123230in}{2.507661in}}{\pgfqpoint{1.121035in}{2.512960in}}{\pgfqpoint{1.117128in}{2.516867in}}%
\pgfpathcurveto{\pgfqpoint{1.113221in}{2.520774in}}{\pgfqpoint{1.107921in}{2.522969in}}{\pgfqpoint{1.102396in}{2.522969in}}%
\pgfpathcurveto{\pgfqpoint{1.096871in}{2.522969in}}{\pgfqpoint{1.091572in}{2.520774in}}{\pgfqpoint{1.087665in}{2.516867in}}%
\pgfpathcurveto{\pgfqpoint{1.083758in}{2.512960in}}{\pgfqpoint{1.081563in}{2.507661in}}{\pgfqpoint{1.081563in}{2.502135in}}%
\pgfpathcurveto{\pgfqpoint{1.081563in}{2.496610in}}{\pgfqpoint{1.083758in}{2.491311in}}{\pgfqpoint{1.087665in}{2.487404in}}%
\pgfpathcurveto{\pgfqpoint{1.091572in}{2.483497in}}{\pgfqpoint{1.096871in}{2.481302in}}{\pgfqpoint{1.102396in}{2.481302in}}%
\pgfpathclose%
\pgfusepath{fill}%
\end{pgfscope}%
\begin{pgfscope}%
\pgfpathrectangle{\pgfqpoint{0.889225in}{2.423832in}}{\pgfqpoint{1.162500in}{0.755000in}}%
\pgfusepath{clip}%
\pgfsetbuttcap%
\pgfsetroundjoin%
\definecolor{currentfill}{rgb}{0.000000,0.000000,0.000000}%
\pgfsetfillcolor{currentfill}%
\pgfsetfillopacity{0.500000}%
\pgfsetlinewidth{0.000000pt}%
\definecolor{currentstroke}{rgb}{0.000000,0.000000,0.000000}%
\pgfsetstrokecolor{currentstroke}%
\pgfsetdash{}{0pt}%
\pgfpathmoveto{\pgfqpoint{1.143186in}{2.503012in}}%
\pgfpathcurveto{\pgfqpoint{1.148711in}{2.503012in}}{\pgfqpoint{1.154011in}{2.505207in}}{\pgfqpoint{1.157918in}{2.509114in}}%
\pgfpathcurveto{\pgfqpoint{1.161825in}{2.513021in}}{\pgfqpoint{1.164020in}{2.518320in}}{\pgfqpoint{1.164020in}{2.523845in}}%
\pgfpathcurveto{\pgfqpoint{1.164020in}{2.529370in}}{\pgfqpoint{1.161825in}{2.534670in}}{\pgfqpoint{1.157918in}{2.538577in}}%
\pgfpathcurveto{\pgfqpoint{1.154011in}{2.542483in}}{\pgfqpoint{1.148711in}{2.544679in}}{\pgfqpoint{1.143186in}{2.544679in}}%
\pgfpathcurveto{\pgfqpoint{1.137661in}{2.544679in}}{\pgfqpoint{1.132362in}{2.542483in}}{\pgfqpoint{1.128455in}{2.538577in}}%
\pgfpathcurveto{\pgfqpoint{1.124548in}{2.534670in}}{\pgfqpoint{1.122353in}{2.529370in}}{\pgfqpoint{1.122353in}{2.523845in}}%
\pgfpathcurveto{\pgfqpoint{1.122353in}{2.518320in}}{\pgfqpoint{1.124548in}{2.513021in}}{\pgfqpoint{1.128455in}{2.509114in}}%
\pgfpathcurveto{\pgfqpoint{1.132362in}{2.505207in}}{\pgfqpoint{1.137661in}{2.503012in}}{\pgfqpoint{1.143186in}{2.503012in}}%
\pgfpathclose%
\pgfusepath{fill}%
\end{pgfscope}%
\begin{pgfscope}%
\pgfpathrectangle{\pgfqpoint{0.889225in}{2.423832in}}{\pgfqpoint{1.162500in}{0.755000in}}%
\pgfusepath{clip}%
\pgfsetbuttcap%
\pgfsetroundjoin%
\definecolor{currentfill}{rgb}{0.000000,0.000000,0.000000}%
\pgfsetfillcolor{currentfill}%
\pgfsetfillopacity{0.500000}%
\pgfsetlinewidth{0.000000pt}%
\definecolor{currentstroke}{rgb}{0.000000,0.000000,0.000000}%
\pgfsetstrokecolor{currentstroke}%
\pgfsetdash{}{0pt}%
\pgfpathmoveto{\pgfqpoint{2.024046in}{3.140023in}}%
\pgfpathcurveto{\pgfqpoint{2.029571in}{3.140023in}}{\pgfqpoint{2.034871in}{3.142218in}}{\pgfqpoint{2.038778in}{3.146125in}}%
\pgfpathcurveto{\pgfqpoint{2.042685in}{3.150032in}}{\pgfqpoint{2.044880in}{3.155331in}}{\pgfqpoint{2.044880in}{3.160856in}}%
\pgfpathcurveto{\pgfqpoint{2.044880in}{3.166381in}}{\pgfqpoint{2.042685in}{3.171681in}}{\pgfqpoint{2.038778in}{3.175588in}}%
\pgfpathcurveto{\pgfqpoint{2.034871in}{3.179494in}}{\pgfqpoint{2.029571in}{3.181690in}}{\pgfqpoint{2.024046in}{3.181690in}}%
\pgfpathcurveto{\pgfqpoint{2.018521in}{3.181690in}}{\pgfqpoint{2.013222in}{3.179494in}}{\pgfqpoint{2.009315in}{3.175588in}}%
\pgfpathcurveto{\pgfqpoint{2.005408in}{3.171681in}}{\pgfqpoint{2.003213in}{3.166381in}}{\pgfqpoint{2.003213in}{3.160856in}}%
\pgfpathcurveto{\pgfqpoint{2.003213in}{3.155331in}}{\pgfqpoint{2.005408in}{3.150032in}}{\pgfqpoint{2.009315in}{3.146125in}}%
\pgfpathcurveto{\pgfqpoint{2.013222in}{3.142218in}}{\pgfqpoint{2.018521in}{3.140023in}}{\pgfqpoint{2.024046in}{3.140023in}}%
\pgfpathclose%
\pgfusepath{fill}%
\end{pgfscope}%
\begin{pgfscope}%
\pgfpathrectangle{\pgfqpoint{0.889225in}{2.423832in}}{\pgfqpoint{1.162500in}{0.755000in}}%
\pgfusepath{clip}%
\pgfsetbuttcap%
\pgfsetroundjoin%
\definecolor{currentfill}{rgb}{0.000000,0.000000,0.000000}%
\pgfsetfillcolor{currentfill}%
\pgfsetfillopacity{0.500000}%
\pgfsetlinewidth{0.000000pt}%
\definecolor{currentstroke}{rgb}{0.000000,0.000000,0.000000}%
\pgfsetstrokecolor{currentstroke}%
\pgfsetdash{}{0pt}%
\pgfpathmoveto{\pgfqpoint{1.402285in}{2.625365in}}%
\pgfpathcurveto{\pgfqpoint{1.407810in}{2.625365in}}{\pgfqpoint{1.413110in}{2.627561in}}{\pgfqpoint{1.417016in}{2.631467in}}%
\pgfpathcurveto{\pgfqpoint{1.420923in}{2.635374in}}{\pgfqpoint{1.423118in}{2.640674in}}{\pgfqpoint{1.423118in}{2.646199in}}%
\pgfpathcurveto{\pgfqpoint{1.423118in}{2.651724in}}{\pgfqpoint{1.420923in}{2.657023in}}{\pgfqpoint{1.417016in}{2.660930in}}%
\pgfpathcurveto{\pgfqpoint{1.413110in}{2.664837in}}{\pgfqpoint{1.407810in}{2.667032in}}{\pgfqpoint{1.402285in}{2.667032in}}%
\pgfpathcurveto{\pgfqpoint{1.396760in}{2.667032in}}{\pgfqpoint{1.391460in}{2.664837in}}{\pgfqpoint{1.387554in}{2.660930in}}%
\pgfpathcurveto{\pgfqpoint{1.383647in}{2.657023in}}{\pgfqpoint{1.381452in}{2.651724in}}{\pgfqpoint{1.381452in}{2.646199in}}%
\pgfpathcurveto{\pgfqpoint{1.381452in}{2.640674in}}{\pgfqpoint{1.383647in}{2.635374in}}{\pgfqpoint{1.387554in}{2.631467in}}%
\pgfpathcurveto{\pgfqpoint{1.391460in}{2.627561in}}{\pgfqpoint{1.396760in}{2.625365in}}{\pgfqpoint{1.402285in}{2.625365in}}%
\pgfpathclose%
\pgfusepath{fill}%
\end{pgfscope}%
\begin{pgfscope}%
\pgfpathrectangle{\pgfqpoint{0.889225in}{2.423832in}}{\pgfqpoint{1.162500in}{0.755000in}}%
\pgfusepath{clip}%
\pgfsetbuttcap%
\pgfsetroundjoin%
\definecolor{currentfill}{rgb}{0.000000,0.000000,0.000000}%
\pgfsetfillcolor{currentfill}%
\pgfsetfillopacity{0.500000}%
\pgfsetlinewidth{0.000000pt}%
\definecolor{currentstroke}{rgb}{0.000000,0.000000,0.000000}%
\pgfsetstrokecolor{currentstroke}%
\pgfsetdash{}{0pt}%
\pgfpathmoveto{\pgfqpoint{0.973483in}{2.450809in}}%
\pgfpathcurveto{\pgfqpoint{0.979008in}{2.450809in}}{\pgfqpoint{0.984308in}{2.453004in}}{\pgfqpoint{0.988214in}{2.456911in}}%
\pgfpathcurveto{\pgfqpoint{0.992121in}{2.460818in}}{\pgfqpoint{0.994316in}{2.466117in}}{\pgfqpoint{0.994316in}{2.471642in}}%
\pgfpathcurveto{\pgfqpoint{0.994316in}{2.477167in}}{\pgfqpoint{0.992121in}{2.482467in}}{\pgfqpoint{0.988214in}{2.486374in}}%
\pgfpathcurveto{\pgfqpoint{0.984308in}{2.490280in}}{\pgfqpoint{0.979008in}{2.492476in}}{\pgfqpoint{0.973483in}{2.492476in}}%
\pgfpathcurveto{\pgfqpoint{0.967958in}{2.492476in}}{\pgfqpoint{0.962658in}{2.490280in}}{\pgfqpoint{0.958752in}{2.486374in}}%
\pgfpathcurveto{\pgfqpoint{0.954845in}{2.482467in}}{\pgfqpoint{0.952650in}{2.477167in}}{\pgfqpoint{0.952650in}{2.471642in}}%
\pgfpathcurveto{\pgfqpoint{0.952650in}{2.466117in}}{\pgfqpoint{0.954845in}{2.460818in}}{\pgfqpoint{0.958752in}{2.456911in}}%
\pgfpathcurveto{\pgfqpoint{0.962658in}{2.453004in}}{\pgfqpoint{0.967958in}{2.450809in}}{\pgfqpoint{0.973483in}{2.450809in}}%
\pgfpathclose%
\pgfusepath{fill}%
\end{pgfscope}%
\begin{pgfscope}%
\pgfsetbuttcap%
\pgfsetroundjoin%
\definecolor{currentfill}{rgb}{0.000000,0.000000,0.000000}%
\pgfsetfillcolor{currentfill}%
\pgfsetlinewidth{0.803000pt}%
\definecolor{currentstroke}{rgb}{0.000000,0.000000,0.000000}%
\pgfsetstrokecolor{currentstroke}%
\pgfsetdash{}{0pt}%
\pgfsys@defobject{currentmarker}{\pgfqpoint{-0.048611in}{0.000000in}}{\pgfqpoint{0.000000in}{0.000000in}}{%
\pgfpathmoveto{\pgfqpoint{0.000000in}{0.000000in}}%
\pgfpathlineto{\pgfqpoint{-0.048611in}{0.000000in}}%
\pgfusepath{stroke,fill}%
}%
\begin{pgfscope}%
\pgfsys@transformshift{0.889225in}{2.691122in}%
\pgfsys@useobject{currentmarker}{}%
\end{pgfscope}%
\end{pgfscope}%
\begin{pgfscope}%
\pgftext[x=0.641152in,y=2.648913in,left,base]{\rmfamily\fontsize{8.000000}{9.600000}\selectfont \(\displaystyle 0.5\)}%
\end{pgfscope}%
\begin{pgfscope}%
\pgfsetbuttcap%
\pgfsetroundjoin%
\definecolor{currentfill}{rgb}{0.000000,0.000000,0.000000}%
\pgfsetfillcolor{currentfill}%
\pgfsetlinewidth{0.803000pt}%
\definecolor{currentstroke}{rgb}{0.000000,0.000000,0.000000}%
\pgfsetstrokecolor{currentstroke}%
\pgfsetdash{}{0pt}%
\pgfsys@defobject{currentmarker}{\pgfqpoint{-0.048611in}{0.000000in}}{\pgfqpoint{0.000000in}{0.000000in}}{%
\pgfpathmoveto{\pgfqpoint{0.000000in}{0.000000in}}%
\pgfpathlineto{\pgfqpoint{-0.048611in}{0.000000in}}%
\pgfusepath{stroke,fill}%
}%
\begin{pgfscope}%
\pgfsys@transformshift{0.889225in}{2.985310in}%
\pgfsys@useobject{currentmarker}{}%
\end{pgfscope}%
\end{pgfscope}%
\begin{pgfscope}%
\pgftext[x=0.641152in,y=2.943101in,left,base]{\rmfamily\fontsize{8.000000}{9.600000}\selectfont \(\displaystyle 1.0\)}%
\end{pgfscope}%
\begin{pgfscope}%
\pgftext[x=0.585596in,y=2.801332in,,bottom,rotate=90.000000]{\rmfamily\fontsize{16.000000}{19.200000}\selectfont charge}%
\end{pgfscope}%
\begin{pgfscope}%
\pgftext[x=0.889225in,y=3.220499in,left,base]{\rmfamily\fontsize{16.000000}{19.200000}\selectfont \(\displaystyle \times10^{-9}\)}%
\end{pgfscope}%
\begin{pgfscope}%
\pgfsetrectcap%
\pgfsetmiterjoin%
\pgfsetlinewidth{0.803000pt}%
\definecolor{currentstroke}{rgb}{0.501961,0.501961,0.501961}%
\pgfsetstrokecolor{currentstroke}%
\pgfsetdash{}{0pt}%
\pgfpathmoveto{\pgfqpoint{0.889225in}{2.423832in}}%
\pgfpathlineto{\pgfqpoint{0.889225in}{3.178832in}}%
\pgfusepath{stroke}%
\end{pgfscope}%
\begin{pgfscope}%
\pgfsetrectcap%
\pgfsetmiterjoin%
\pgfsetlinewidth{0.803000pt}%
\definecolor{currentstroke}{rgb}{0.501961,0.501961,0.501961}%
\pgfsetstrokecolor{currentstroke}%
\pgfsetdash{}{0pt}%
\pgfpathmoveto{\pgfqpoint{2.051725in}{2.423832in}}%
\pgfpathlineto{\pgfqpoint{2.051725in}{3.178832in}}%
\pgfusepath{stroke}%
\end{pgfscope}%
\begin{pgfscope}%
\pgfsetrectcap%
\pgfsetmiterjoin%
\pgfsetlinewidth{0.803000pt}%
\definecolor{currentstroke}{rgb}{0.501961,0.501961,0.501961}%
\pgfsetstrokecolor{currentstroke}%
\pgfsetdash{}{0pt}%
\pgfpathmoveto{\pgfqpoint{0.889225in}{2.423832in}}%
\pgfpathlineto{\pgfqpoint{2.051725in}{2.423832in}}%
\pgfusepath{stroke}%
\end{pgfscope}%
\begin{pgfscope}%
\pgfsetrectcap%
\pgfsetmiterjoin%
\pgfsetlinewidth{0.803000pt}%
\definecolor{currentstroke}{rgb}{0.501961,0.501961,0.501961}%
\pgfsetstrokecolor{currentstroke}%
\pgfsetdash{}{0pt}%
\pgfpathmoveto{\pgfqpoint{0.889225in}{3.178832in}}%
\pgfpathlineto{\pgfqpoint{2.051725in}{3.178832in}}%
\pgfusepath{stroke}%
\end{pgfscope}%
\begin{pgfscope}%
\pgfsetbuttcap%
\pgfsetmiterjoin%
\definecolor{currentfill}{rgb}{1.000000,1.000000,1.000000}%
\pgfsetfillcolor{currentfill}%
\pgfsetlinewidth{0.000000pt}%
\definecolor{currentstroke}{rgb}{0.000000,0.000000,0.000000}%
\pgfsetstrokecolor{currentstroke}%
\pgfsetstrokeopacity{0.000000}%
\pgfsetdash{}{0pt}%
\pgfpathmoveto{\pgfqpoint{2.051725in}{2.423832in}}%
\pgfpathlineto{\pgfqpoint{3.214225in}{2.423832in}}%
\pgfpathlineto{\pgfqpoint{3.214225in}{3.178832in}}%
\pgfpathlineto{\pgfqpoint{2.051725in}{3.178832in}}%
\pgfpathclose%
\pgfusepath{fill}%
\end{pgfscope}%
\begin{pgfscope}%
\pgfpathrectangle{\pgfqpoint{2.051725in}{2.423832in}}{\pgfqpoint{1.162500in}{0.755000in}}%
\pgfusepath{clip}%
\pgfsetrectcap%
\pgfsetroundjoin%
\pgfsetlinewidth{1.505625pt}%
\definecolor{currentstroke}{rgb}{0.121569,0.466667,0.705882}%
\pgfsetstrokecolor{currentstroke}%
\pgfsetdash{}{0pt}%
\pgfpathmoveto{\pgfqpoint{2.079404in}{2.981458in}}%
\pgfpathlineto{\pgfqpoint{2.103785in}{3.023270in}}%
\pgfpathlineto{\pgfqpoint{2.124842in}{3.055299in}}%
\pgfpathlineto{\pgfqpoint{2.143682in}{3.080259in}}%
\pgfpathlineto{\pgfqpoint{2.161414in}{3.100257in}}%
\pgfpathlineto{\pgfqpoint{2.178038in}{3.115741in}}%
\pgfpathlineto{\pgfqpoint{2.193553in}{3.127237in}}%
\pgfpathlineto{\pgfqpoint{2.207961in}{3.135306in}}%
\pgfpathlineto{\pgfqpoint{2.222368in}{3.140849in}}%
\pgfpathlineto{\pgfqpoint{2.235667in}{3.143732in}}%
\pgfpathlineto{\pgfqpoint{2.248966in}{3.144496in}}%
\pgfpathlineto{\pgfqpoint{2.262265in}{3.143183in}}%
\pgfpathlineto{\pgfqpoint{2.276672in}{3.139480in}}%
\pgfpathlineto{\pgfqpoint{2.291080in}{3.133489in}}%
\pgfpathlineto{\pgfqpoint{2.306595in}{3.124594in}}%
\pgfpathlineto{\pgfqpoint{2.323219in}{3.112404in}}%
\pgfpathlineto{\pgfqpoint{2.340951in}{3.096574in}}%
\pgfpathlineto{\pgfqpoint{2.360899in}{3.075567in}}%
\pgfpathlineto{\pgfqpoint{2.381956in}{3.050096in}}%
\pgfpathlineto{\pgfqpoint{2.406338in}{3.016922in}}%
\pgfpathlineto{\pgfqpoint{2.434044in}{2.975246in}}%
\pgfpathlineto{\pgfqpoint{2.467291in}{2.920995in}}%
\pgfpathlineto{\pgfqpoint{2.514946in}{2.838468in}}%
\pgfpathlineto{\pgfqpoint{2.590307in}{2.707949in}}%
\pgfpathlineto{\pgfqpoint{2.625771in}{2.651279in}}%
\pgfpathlineto{\pgfqpoint{2.654586in}{2.609241in}}%
\pgfpathlineto{\pgfqpoint{2.680076in}{2.575738in}}%
\pgfpathlineto{\pgfqpoint{2.704457in}{2.547348in}}%
\pgfpathlineto{\pgfqpoint{2.726622in}{2.524875in}}%
\pgfpathlineto{\pgfqpoint{2.747679in}{2.506592in}}%
\pgfpathlineto{\pgfqpoint{2.767628in}{2.492072in}}%
\pgfpathlineto{\pgfqpoint{2.787576in}{2.480266in}}%
\pgfpathlineto{\pgfqpoint{2.807525in}{2.471117in}}%
\pgfpathlineto{\pgfqpoint{2.827473in}{2.464520in}}%
\pgfpathlineto{\pgfqpoint{2.847422in}{2.460331in}}%
\pgfpathlineto{\pgfqpoint{2.867370in}{2.458366in}}%
\pgfpathlineto{\pgfqpoint{2.888427in}{2.458462in}}%
\pgfpathlineto{\pgfqpoint{2.911700in}{2.460809in}}%
\pgfpathlineto{\pgfqpoint{2.938298in}{2.465819in}}%
\pgfpathlineto{\pgfqpoint{2.971546in}{2.474517in}}%
\pgfpathlineto{\pgfqpoint{3.079046in}{2.504560in}}%
\pgfpathlineto{\pgfqpoint{3.107861in}{2.509223in}}%
\pgfpathlineto{\pgfqpoint{3.134459in}{2.511245in}}%
\pgfpathlineto{\pgfqpoint{3.159948in}{2.510915in}}%
\pgfpathlineto{\pgfqpoint{3.185438in}{2.508314in}}%
\pgfpathlineto{\pgfqpoint{3.186546in}{2.508151in}}%
\pgfpathlineto{\pgfqpoint{3.186546in}{2.508151in}}%
\pgfusepath{stroke}%
\end{pgfscope}%
\begin{pgfscope}%
\pgfsetrectcap%
\pgfsetmiterjoin%
\pgfsetlinewidth{0.803000pt}%
\definecolor{currentstroke}{rgb}{0.501961,0.501961,0.501961}%
\pgfsetstrokecolor{currentstroke}%
\pgfsetdash{}{0pt}%
\pgfpathmoveto{\pgfqpoint{2.051725in}{2.423832in}}%
\pgfpathlineto{\pgfqpoint{2.051725in}{3.178832in}}%
\pgfusepath{stroke}%
\end{pgfscope}%
\begin{pgfscope}%
\pgfsetrectcap%
\pgfsetmiterjoin%
\pgfsetlinewidth{0.803000pt}%
\definecolor{currentstroke}{rgb}{0.501961,0.501961,0.501961}%
\pgfsetstrokecolor{currentstroke}%
\pgfsetdash{}{0pt}%
\pgfpathmoveto{\pgfqpoint{3.214225in}{2.423832in}}%
\pgfpathlineto{\pgfqpoint{3.214225in}{3.178832in}}%
\pgfusepath{stroke}%
\end{pgfscope}%
\begin{pgfscope}%
\pgfsetrectcap%
\pgfsetmiterjoin%
\pgfsetlinewidth{0.803000pt}%
\definecolor{currentstroke}{rgb}{0.501961,0.501961,0.501961}%
\pgfsetstrokecolor{currentstroke}%
\pgfsetdash{}{0pt}%
\pgfpathmoveto{\pgfqpoint{2.051725in}{2.423832in}}%
\pgfpathlineto{\pgfqpoint{3.214225in}{2.423832in}}%
\pgfusepath{stroke}%
\end{pgfscope}%
\begin{pgfscope}%
\pgfsetrectcap%
\pgfsetmiterjoin%
\pgfsetlinewidth{0.803000pt}%
\definecolor{currentstroke}{rgb}{0.501961,0.501961,0.501961}%
\pgfsetstrokecolor{currentstroke}%
\pgfsetdash{}{0pt}%
\pgfpathmoveto{\pgfqpoint{2.051725in}{3.178832in}}%
\pgfpathlineto{\pgfqpoint{3.214225in}{3.178832in}}%
\pgfusepath{stroke}%
\end{pgfscope}%
\begin{pgfscope}%
\pgfsetbuttcap%
\pgfsetmiterjoin%
\definecolor{currentfill}{rgb}{1.000000,1.000000,1.000000}%
\pgfsetfillcolor{currentfill}%
\pgfsetlinewidth{0.000000pt}%
\definecolor{currentstroke}{rgb}{0.000000,0.000000,0.000000}%
\pgfsetstrokecolor{currentstroke}%
\pgfsetstrokeopacity{0.000000}%
\pgfsetdash{}{0pt}%
\pgfpathmoveto{\pgfqpoint{3.214225in}{2.423832in}}%
\pgfpathlineto{\pgfqpoint{4.376725in}{2.423832in}}%
\pgfpathlineto{\pgfqpoint{4.376725in}{3.178832in}}%
\pgfpathlineto{\pgfqpoint{3.214225in}{3.178832in}}%
\pgfpathclose%
\pgfusepath{fill}%
\end{pgfscope}%
\begin{pgfscope}%
\pgfpathrectangle{\pgfqpoint{3.214225in}{2.423832in}}{\pgfqpoint{1.162500in}{0.755000in}}%
\pgfusepath{clip}%
\pgfsetbuttcap%
\pgfsetroundjoin%
\definecolor{currentfill}{rgb}{0.000000,0.000000,0.000000}%
\pgfsetfillcolor{currentfill}%
\pgfsetfillopacity{0.500000}%
\pgfsetlinewidth{0.000000pt}%
\definecolor{currentstroke}{rgb}{0.000000,0.000000,0.000000}%
\pgfsetstrokecolor{currentstroke}%
\pgfsetdash{}{0pt}%
\pgfpathmoveto{\pgfqpoint{3.656364in}{3.088730in}}%
\pgfpathcurveto{\pgfqpoint{3.661889in}{3.088730in}}{\pgfqpoint{3.667189in}{3.090925in}}{\pgfqpoint{3.671096in}{3.094832in}}%
\pgfpathcurveto{\pgfqpoint{3.675002in}{3.098739in}}{\pgfqpoint{3.677198in}{3.104038in}}{\pgfqpoint{3.677198in}{3.109563in}}%
\pgfpathcurveto{\pgfqpoint{3.677198in}{3.115089in}}{\pgfqpoint{3.675002in}{3.120388in}}{\pgfqpoint{3.671096in}{3.124295in}}%
\pgfpathcurveto{\pgfqpoint{3.667189in}{3.128202in}}{\pgfqpoint{3.661889in}{3.130397in}}{\pgfqpoint{3.656364in}{3.130397in}}%
\pgfpathcurveto{\pgfqpoint{3.650839in}{3.130397in}}{\pgfqpoint{3.645540in}{3.128202in}}{\pgfqpoint{3.641633in}{3.124295in}}%
\pgfpathcurveto{\pgfqpoint{3.637726in}{3.120388in}}{\pgfqpoint{3.635531in}{3.115089in}}{\pgfqpoint{3.635531in}{3.109563in}}%
\pgfpathcurveto{\pgfqpoint{3.635531in}{3.104038in}}{\pgfqpoint{3.637726in}{3.098739in}}{\pgfqpoint{3.641633in}{3.094832in}}%
\pgfpathcurveto{\pgfqpoint{3.645540in}{3.090925in}}{\pgfqpoint{3.650839in}{3.088730in}}{\pgfqpoint{3.656364in}{3.088730in}}%
\pgfpathclose%
\pgfusepath{fill}%
\end{pgfscope}%
\begin{pgfscope}%
\pgfpathrectangle{\pgfqpoint{3.214225in}{2.423832in}}{\pgfqpoint{1.162500in}{0.755000in}}%
\pgfusepath{clip}%
\pgfsetbuttcap%
\pgfsetroundjoin%
\definecolor{currentfill}{rgb}{0.000000,0.000000,0.000000}%
\pgfsetfillcolor{currentfill}%
\pgfsetfillopacity{0.500000}%
\pgfsetlinewidth{0.000000pt}%
\definecolor{currentstroke}{rgb}{0.000000,0.000000,0.000000}%
\pgfsetstrokecolor{currentstroke}%
\pgfsetdash{}{0pt}%
\pgfpathmoveto{\pgfqpoint{4.259629in}{2.544873in}}%
\pgfpathcurveto{\pgfqpoint{4.265154in}{2.544873in}}{\pgfqpoint{4.270454in}{2.547068in}}{\pgfqpoint{4.274361in}{2.550975in}}%
\pgfpathcurveto{\pgfqpoint{4.278268in}{2.554881in}}{\pgfqpoint{4.280463in}{2.560181in}}{\pgfqpoint{4.280463in}{2.565706in}}%
\pgfpathcurveto{\pgfqpoint{4.280463in}{2.571231in}}{\pgfqpoint{4.278268in}{2.576531in}}{\pgfqpoint{4.274361in}{2.580437in}}%
\pgfpathcurveto{\pgfqpoint{4.270454in}{2.584344in}}{\pgfqpoint{4.265154in}{2.586539in}}{\pgfqpoint{4.259629in}{2.586539in}}%
\pgfpathcurveto{\pgfqpoint{4.254104in}{2.586539in}}{\pgfqpoint{4.248805in}{2.584344in}}{\pgfqpoint{4.244898in}{2.580437in}}%
\pgfpathcurveto{\pgfqpoint{4.240991in}{2.576531in}}{\pgfqpoint{4.238796in}{2.571231in}}{\pgfqpoint{4.238796in}{2.565706in}}%
\pgfpathcurveto{\pgfqpoint{4.238796in}{2.560181in}}{\pgfqpoint{4.240991in}{2.554881in}}{\pgfqpoint{4.244898in}{2.550975in}}%
\pgfpathcurveto{\pgfqpoint{4.248805in}{2.547068in}}{\pgfqpoint{4.254104in}{2.544873in}}{\pgfqpoint{4.259629in}{2.544873in}}%
\pgfpathclose%
\pgfusepath{fill}%
\end{pgfscope}%
\begin{pgfscope}%
\pgfpathrectangle{\pgfqpoint{3.214225in}{2.423832in}}{\pgfqpoint{1.162500in}{0.755000in}}%
\pgfusepath{clip}%
\pgfsetbuttcap%
\pgfsetroundjoin%
\definecolor{currentfill}{rgb}{0.000000,0.000000,0.000000}%
\pgfsetfillcolor{currentfill}%
\pgfsetfillopacity{0.500000}%
\pgfsetlinewidth{0.000000pt}%
\definecolor{currentstroke}{rgb}{0.000000,0.000000,0.000000}%
\pgfsetstrokecolor{currentstroke}%
\pgfsetdash{}{0pt}%
\pgfpathmoveto{\pgfqpoint{4.349046in}{2.541427in}}%
\pgfpathcurveto{\pgfqpoint{4.354571in}{2.541427in}}{\pgfqpoint{4.359871in}{2.543622in}}{\pgfqpoint{4.363778in}{2.547529in}}%
\pgfpathcurveto{\pgfqpoint{4.367685in}{2.551436in}}{\pgfqpoint{4.369880in}{2.556735in}}{\pgfqpoint{4.369880in}{2.562260in}}%
\pgfpathcurveto{\pgfqpoint{4.369880in}{2.567785in}}{\pgfqpoint{4.367685in}{2.573085in}}{\pgfqpoint{4.363778in}{2.576992in}}%
\pgfpathcurveto{\pgfqpoint{4.359871in}{2.580899in}}{\pgfqpoint{4.354571in}{2.583094in}}{\pgfqpoint{4.349046in}{2.583094in}}%
\pgfpathcurveto{\pgfqpoint{4.343521in}{2.583094in}}{\pgfqpoint{4.338222in}{2.580899in}}{\pgfqpoint{4.334315in}{2.576992in}}%
\pgfpathcurveto{\pgfqpoint{4.330408in}{2.573085in}}{\pgfqpoint{4.328213in}{2.567785in}}{\pgfqpoint{4.328213in}{2.562260in}}%
\pgfpathcurveto{\pgfqpoint{4.328213in}{2.556735in}}{\pgfqpoint{4.330408in}{2.551436in}}{\pgfqpoint{4.334315in}{2.547529in}}%
\pgfpathcurveto{\pgfqpoint{4.338222in}{2.543622in}}{\pgfqpoint{4.343521in}{2.541427in}}{\pgfqpoint{4.349046in}{2.541427in}}%
\pgfpathclose%
\pgfusepath{fill}%
\end{pgfscope}%
\begin{pgfscope}%
\pgfpathrectangle{\pgfqpoint{3.214225in}{2.423832in}}{\pgfqpoint{1.162500in}{0.755000in}}%
\pgfusepath{clip}%
\pgfsetbuttcap%
\pgfsetroundjoin%
\definecolor{currentfill}{rgb}{0.000000,0.000000,0.000000}%
\pgfsetfillcolor{currentfill}%
\pgfsetfillopacity{0.500000}%
\pgfsetlinewidth{0.000000pt}%
\definecolor{currentstroke}{rgb}{0.000000,0.000000,0.000000}%
\pgfsetstrokecolor{currentstroke}%
\pgfsetdash{}{0pt}%
\pgfpathmoveto{\pgfqpoint{3.915823in}{2.469548in}}%
\pgfpathcurveto{\pgfqpoint{3.921348in}{2.469548in}}{\pgfqpoint{3.926648in}{2.471743in}}{\pgfqpoint{3.930554in}{2.475650in}}%
\pgfpathcurveto{\pgfqpoint{3.934461in}{2.479557in}}{\pgfqpoint{3.936656in}{2.484856in}}{\pgfqpoint{3.936656in}{2.490381in}}%
\pgfpathcurveto{\pgfqpoint{3.936656in}{2.495907in}}{\pgfqpoint{3.934461in}{2.501206in}}{\pgfqpoint{3.930554in}{2.505113in}}%
\pgfpathcurveto{\pgfqpoint{3.926648in}{2.509020in}}{\pgfqpoint{3.921348in}{2.511215in}}{\pgfqpoint{3.915823in}{2.511215in}}%
\pgfpathcurveto{\pgfqpoint{3.910298in}{2.511215in}}{\pgfqpoint{3.904998in}{2.509020in}}{\pgfqpoint{3.901092in}{2.505113in}}%
\pgfpathcurveto{\pgfqpoint{3.897185in}{2.501206in}}{\pgfqpoint{3.894990in}{2.495907in}}{\pgfqpoint{3.894990in}{2.490381in}}%
\pgfpathcurveto{\pgfqpoint{3.894990in}{2.484856in}}{\pgfqpoint{3.897185in}{2.479557in}}{\pgfqpoint{3.901092in}{2.475650in}}%
\pgfpathcurveto{\pgfqpoint{3.904998in}{2.471743in}}{\pgfqpoint{3.910298in}{2.469548in}}{\pgfqpoint{3.915823in}{2.469548in}}%
\pgfpathclose%
\pgfusepath{fill}%
\end{pgfscope}%
\begin{pgfscope}%
\pgfpathrectangle{\pgfqpoint{3.214225in}{2.423832in}}{\pgfqpoint{1.162500in}{0.755000in}}%
\pgfusepath{clip}%
\pgfsetbuttcap%
\pgfsetroundjoin%
\definecolor{currentfill}{rgb}{0.000000,0.000000,0.000000}%
\pgfsetfillcolor{currentfill}%
\pgfsetfillopacity{0.500000}%
\pgfsetlinewidth{0.000000pt}%
\definecolor{currentstroke}{rgb}{0.000000,0.000000,0.000000}%
\pgfsetstrokecolor{currentstroke}%
\pgfsetdash{}{0pt}%
\pgfpathmoveto{\pgfqpoint{3.986481in}{2.488481in}}%
\pgfpathcurveto{\pgfqpoint{3.992006in}{2.488481in}}{\pgfqpoint{3.997306in}{2.490676in}}{\pgfqpoint{4.001213in}{2.494583in}}%
\pgfpathcurveto{\pgfqpoint{4.005120in}{2.498489in}}{\pgfqpoint{4.007315in}{2.503789in}}{\pgfqpoint{4.007315in}{2.509314in}}%
\pgfpathcurveto{\pgfqpoint{4.007315in}{2.514839in}}{\pgfqpoint{4.005120in}{2.520139in}}{\pgfqpoint{4.001213in}{2.524045in}}%
\pgfpathcurveto{\pgfqpoint{3.997306in}{2.527952in}}{\pgfqpoint{3.992006in}{2.530147in}}{\pgfqpoint{3.986481in}{2.530147in}}%
\pgfpathcurveto{\pgfqpoint{3.980956in}{2.530147in}}{\pgfqpoint{3.975657in}{2.527952in}}{\pgfqpoint{3.971750in}{2.524045in}}%
\pgfpathcurveto{\pgfqpoint{3.967843in}{2.520139in}}{\pgfqpoint{3.965648in}{2.514839in}}{\pgfqpoint{3.965648in}{2.509314in}}%
\pgfpathcurveto{\pgfqpoint{3.965648in}{2.503789in}}{\pgfqpoint{3.967843in}{2.498489in}}{\pgfqpoint{3.971750in}{2.494583in}}%
\pgfpathcurveto{\pgfqpoint{3.975657in}{2.490676in}}{\pgfqpoint{3.980956in}{2.488481in}}{\pgfqpoint{3.986481in}{2.488481in}}%
\pgfpathclose%
\pgfusepath{fill}%
\end{pgfscope}%
\begin{pgfscope}%
\pgfpathrectangle{\pgfqpoint{3.214225in}{2.423832in}}{\pgfqpoint{1.162500in}{0.755000in}}%
\pgfusepath{clip}%
\pgfsetbuttcap%
\pgfsetroundjoin%
\definecolor{currentfill}{rgb}{0.000000,0.000000,0.000000}%
\pgfsetfillcolor{currentfill}%
\pgfsetfillopacity{0.500000}%
\pgfsetlinewidth{0.000000pt}%
\definecolor{currentstroke}{rgb}{0.000000,0.000000,0.000000}%
\pgfsetstrokecolor{currentstroke}%
\pgfsetdash{}{0pt}%
\pgfpathmoveto{\pgfqpoint{3.831974in}{2.422812in}}%
\pgfpathcurveto{\pgfqpoint{3.837499in}{2.422812in}}{\pgfqpoint{3.842798in}{2.425007in}}{\pgfqpoint{3.846705in}{2.428914in}}%
\pgfpathcurveto{\pgfqpoint{3.850612in}{2.432821in}}{\pgfqpoint{3.852807in}{2.438120in}}{\pgfqpoint{3.852807in}{2.443645in}}%
\pgfpathcurveto{\pgfqpoint{3.852807in}{2.449170in}}{\pgfqpoint{3.850612in}{2.454470in}}{\pgfqpoint{3.846705in}{2.458377in}}%
\pgfpathcurveto{\pgfqpoint{3.842798in}{2.462284in}}{\pgfqpoint{3.837499in}{2.464479in}}{\pgfqpoint{3.831974in}{2.464479in}}%
\pgfpathcurveto{\pgfqpoint{3.826449in}{2.464479in}}{\pgfqpoint{3.821149in}{2.462284in}}{\pgfqpoint{3.817242in}{2.458377in}}%
\pgfpathcurveto{\pgfqpoint{3.813336in}{2.454470in}}{\pgfqpoint{3.811140in}{2.449170in}}{\pgfqpoint{3.811140in}{2.443645in}}%
\pgfpathcurveto{\pgfqpoint{3.811140in}{2.438120in}}{\pgfqpoint{3.813336in}{2.432821in}}{\pgfqpoint{3.817242in}{2.428914in}}%
\pgfpathcurveto{\pgfqpoint{3.821149in}{2.425007in}}{\pgfqpoint{3.826449in}{2.422812in}}{\pgfqpoint{3.831974in}{2.422812in}}%
\pgfpathclose%
\pgfusepath{fill}%
\end{pgfscope}%
\begin{pgfscope}%
\pgfpathrectangle{\pgfqpoint{3.214225in}{2.423832in}}{\pgfqpoint{1.162500in}{0.755000in}}%
\pgfusepath{clip}%
\pgfsetbuttcap%
\pgfsetroundjoin%
\definecolor{currentfill}{rgb}{0.000000,0.000000,0.000000}%
\pgfsetfillcolor{currentfill}%
\pgfsetfillopacity{0.500000}%
\pgfsetlinewidth{0.000000pt}%
\definecolor{currentstroke}{rgb}{0.000000,0.000000,0.000000}%
\pgfsetstrokecolor{currentstroke}%
\pgfsetdash{}{0pt}%
\pgfpathmoveto{\pgfqpoint{3.241904in}{2.420975in}}%
\pgfpathcurveto{\pgfqpoint{3.247429in}{2.420975in}}{\pgfqpoint{3.252728in}{2.423170in}}{\pgfqpoint{3.256635in}{2.427077in}}%
\pgfpathcurveto{\pgfqpoint{3.260542in}{2.430984in}}{\pgfqpoint{3.262737in}{2.436283in}}{\pgfqpoint{3.262737in}{2.441809in}}%
\pgfpathcurveto{\pgfqpoint{3.262737in}{2.447334in}}{\pgfqpoint{3.260542in}{2.452633in}}{\pgfqpoint{3.256635in}{2.456540in}}%
\pgfpathcurveto{\pgfqpoint{3.252728in}{2.460447in}}{\pgfqpoint{3.247429in}{2.462642in}}{\pgfqpoint{3.241904in}{2.462642in}}%
\pgfpathcurveto{\pgfqpoint{3.236379in}{2.462642in}}{\pgfqpoint{3.231079in}{2.460447in}}{\pgfqpoint{3.227172in}{2.456540in}}%
\pgfpathcurveto{\pgfqpoint{3.223265in}{2.452633in}}{\pgfqpoint{3.221070in}{2.447334in}}{\pgfqpoint{3.221070in}{2.441809in}}%
\pgfpathcurveto{\pgfqpoint{3.221070in}{2.436283in}}{\pgfqpoint{3.223265in}{2.430984in}}{\pgfqpoint{3.227172in}{2.427077in}}%
\pgfpathcurveto{\pgfqpoint{3.231079in}{2.423170in}}{\pgfqpoint{3.236379in}{2.420975in}}{\pgfqpoint{3.241904in}{2.420975in}}%
\pgfpathclose%
\pgfusepath{fill}%
\end{pgfscope}%
\begin{pgfscope}%
\pgfpathrectangle{\pgfqpoint{3.214225in}{2.423832in}}{\pgfqpoint{1.162500in}{0.755000in}}%
\pgfusepath{clip}%
\pgfsetbuttcap%
\pgfsetroundjoin%
\definecolor{currentfill}{rgb}{0.000000,0.000000,0.000000}%
\pgfsetfillcolor{currentfill}%
\pgfsetfillopacity{0.500000}%
\pgfsetlinewidth{0.000000pt}%
\definecolor{currentstroke}{rgb}{0.000000,0.000000,0.000000}%
\pgfsetstrokecolor{currentstroke}%
\pgfsetdash{}{0pt}%
\pgfpathmoveto{\pgfqpoint{4.220724in}{2.717961in}}%
\pgfpathcurveto{\pgfqpoint{4.226249in}{2.717961in}}{\pgfqpoint{4.231548in}{2.720156in}}{\pgfqpoint{4.235455in}{2.724063in}}%
\pgfpathcurveto{\pgfqpoint{4.239362in}{2.727970in}}{\pgfqpoint{4.241557in}{2.733269in}}{\pgfqpoint{4.241557in}{2.738795in}}%
\pgfpathcurveto{\pgfqpoint{4.241557in}{2.744320in}}{\pgfqpoint{4.239362in}{2.749619in}}{\pgfqpoint{4.235455in}{2.753526in}}%
\pgfpathcurveto{\pgfqpoint{4.231548in}{2.757433in}}{\pgfqpoint{4.226249in}{2.759628in}}{\pgfqpoint{4.220724in}{2.759628in}}%
\pgfpathcurveto{\pgfqpoint{4.215199in}{2.759628in}}{\pgfqpoint{4.209899in}{2.757433in}}{\pgfqpoint{4.205992in}{2.753526in}}%
\pgfpathcurveto{\pgfqpoint{4.202086in}{2.749619in}}{\pgfqpoint{4.199890in}{2.744320in}}{\pgfqpoint{4.199890in}{2.738795in}}%
\pgfpathcurveto{\pgfqpoint{4.199890in}{2.733269in}}{\pgfqpoint{4.202086in}{2.727970in}}{\pgfqpoint{4.205992in}{2.724063in}}%
\pgfpathcurveto{\pgfqpoint{4.209899in}{2.720156in}}{\pgfqpoint{4.215199in}{2.717961in}}{\pgfqpoint{4.220724in}{2.717961in}}%
\pgfpathclose%
\pgfusepath{fill}%
\end{pgfscope}%
\begin{pgfscope}%
\pgfpathrectangle{\pgfqpoint{3.214225in}{2.423832in}}{\pgfqpoint{1.162500in}{0.755000in}}%
\pgfusepath{clip}%
\pgfsetbuttcap%
\pgfsetroundjoin%
\definecolor{currentfill}{rgb}{0.000000,0.000000,0.000000}%
\pgfsetfillcolor{currentfill}%
\pgfsetfillopacity{0.500000}%
\pgfsetlinewidth{0.000000pt}%
\definecolor{currentstroke}{rgb}{0.000000,0.000000,0.000000}%
\pgfsetstrokecolor{currentstroke}%
\pgfsetdash{}{0pt}%
\pgfpathmoveto{\pgfqpoint{3.980838in}{2.640479in}}%
\pgfpathcurveto{\pgfqpoint{3.986364in}{2.640479in}}{\pgfqpoint{3.991663in}{2.642674in}}{\pgfqpoint{3.995570in}{2.646581in}}%
\pgfpathcurveto{\pgfqpoint{3.999477in}{2.650487in}}{\pgfqpoint{4.001672in}{2.655787in}}{\pgfqpoint{4.001672in}{2.661312in}}%
\pgfpathcurveto{\pgfqpoint{4.001672in}{2.666837in}}{\pgfqpoint{3.999477in}{2.672137in}}{\pgfqpoint{3.995570in}{2.676043in}}%
\pgfpathcurveto{\pgfqpoint{3.991663in}{2.679950in}}{\pgfqpoint{3.986364in}{2.682145in}}{\pgfqpoint{3.980838in}{2.682145in}}%
\pgfpathcurveto{\pgfqpoint{3.975313in}{2.682145in}}{\pgfqpoint{3.970014in}{2.679950in}}{\pgfqpoint{3.966107in}{2.676043in}}%
\pgfpathcurveto{\pgfqpoint{3.962200in}{2.672137in}}{\pgfqpoint{3.960005in}{2.666837in}}{\pgfqpoint{3.960005in}{2.661312in}}%
\pgfpathcurveto{\pgfqpoint{3.960005in}{2.655787in}}{\pgfqpoint{3.962200in}{2.650487in}}{\pgfqpoint{3.966107in}{2.646581in}}%
\pgfpathcurveto{\pgfqpoint{3.970014in}{2.642674in}}{\pgfqpoint{3.975313in}{2.640479in}}{\pgfqpoint{3.980838in}{2.640479in}}%
\pgfpathclose%
\pgfusepath{fill}%
\end{pgfscope}%
\begin{pgfscope}%
\pgfpathrectangle{\pgfqpoint{3.214225in}{2.423832in}}{\pgfqpoint{1.162500in}{0.755000in}}%
\pgfusepath{clip}%
\pgfsetbuttcap%
\pgfsetroundjoin%
\definecolor{currentfill}{rgb}{0.000000,0.000000,0.000000}%
\pgfsetfillcolor{currentfill}%
\pgfsetfillopacity{0.500000}%
\pgfsetlinewidth{0.000000pt}%
\definecolor{currentstroke}{rgb}{0.000000,0.000000,0.000000}%
\pgfsetstrokecolor{currentstroke}%
\pgfsetdash{}{0pt}%
\pgfpathmoveto{\pgfqpoint{3.708643in}{2.637393in}}%
\pgfpathcurveto{\pgfqpoint{3.714168in}{2.637393in}}{\pgfqpoint{3.719468in}{2.639588in}}{\pgfqpoint{3.723375in}{2.643495in}}%
\pgfpathcurveto{\pgfqpoint{3.727282in}{2.647402in}}{\pgfqpoint{3.729477in}{2.652701in}}{\pgfqpoint{3.729477in}{2.658226in}}%
\pgfpathcurveto{\pgfqpoint{3.729477in}{2.663751in}}{\pgfqpoint{3.727282in}{2.669051in}}{\pgfqpoint{3.723375in}{2.672958in}}%
\pgfpathcurveto{\pgfqpoint{3.719468in}{2.676865in}}{\pgfqpoint{3.714168in}{2.679060in}}{\pgfqpoint{3.708643in}{2.679060in}}%
\pgfpathcurveto{\pgfqpoint{3.703118in}{2.679060in}}{\pgfqpoint{3.697819in}{2.676865in}}{\pgfqpoint{3.693912in}{2.672958in}}%
\pgfpathcurveto{\pgfqpoint{3.690005in}{2.669051in}}{\pgfqpoint{3.687810in}{2.663751in}}{\pgfqpoint{3.687810in}{2.658226in}}%
\pgfpathcurveto{\pgfqpoint{3.687810in}{2.652701in}}{\pgfqpoint{3.690005in}{2.647402in}}{\pgfqpoint{3.693912in}{2.643495in}}%
\pgfpathcurveto{\pgfqpoint{3.697819in}{2.639588in}}{\pgfqpoint{3.703118in}{2.637393in}}{\pgfqpoint{3.708643in}{2.637393in}}%
\pgfpathclose%
\pgfusepath{fill}%
\end{pgfscope}%
\begin{pgfscope}%
\pgfpathrectangle{\pgfqpoint{3.214225in}{2.423832in}}{\pgfqpoint{1.162500in}{0.755000in}}%
\pgfusepath{clip}%
\pgfsetbuttcap%
\pgfsetroundjoin%
\definecolor{currentfill}{rgb}{0.000000,0.000000,0.000000}%
\pgfsetfillcolor{currentfill}%
\pgfsetfillopacity{0.500000}%
\pgfsetlinewidth{0.000000pt}%
\definecolor{currentstroke}{rgb}{0.000000,0.000000,0.000000}%
\pgfsetstrokecolor{currentstroke}%
\pgfsetdash{}{0pt}%
\pgfpathmoveto{\pgfqpoint{4.124371in}{2.679970in}}%
\pgfpathcurveto{\pgfqpoint{4.129896in}{2.679970in}}{\pgfqpoint{4.135195in}{2.682166in}}{\pgfqpoint{4.139102in}{2.686072in}}%
\pgfpathcurveto{\pgfqpoint{4.143009in}{2.689979in}}{\pgfqpoint{4.145204in}{2.695279in}}{\pgfqpoint{4.145204in}{2.700804in}}%
\pgfpathcurveto{\pgfqpoint{4.145204in}{2.706329in}}{\pgfqpoint{4.143009in}{2.711628in}}{\pgfqpoint{4.139102in}{2.715535in}}%
\pgfpathcurveto{\pgfqpoint{4.135195in}{2.719442in}}{\pgfqpoint{4.129896in}{2.721637in}}{\pgfqpoint{4.124371in}{2.721637in}}%
\pgfpathcurveto{\pgfqpoint{4.118846in}{2.721637in}}{\pgfqpoint{4.113546in}{2.719442in}}{\pgfqpoint{4.109639in}{2.715535in}}%
\pgfpathcurveto{\pgfqpoint{4.105732in}{2.711628in}}{\pgfqpoint{4.103537in}{2.706329in}}{\pgfqpoint{4.103537in}{2.700804in}}%
\pgfpathcurveto{\pgfqpoint{4.103537in}{2.695279in}}{\pgfqpoint{4.105732in}{2.689979in}}{\pgfqpoint{4.109639in}{2.686072in}}%
\pgfpathcurveto{\pgfqpoint{4.113546in}{2.682166in}}{\pgfqpoint{4.118846in}{2.679970in}}{\pgfqpoint{4.124371in}{2.679970in}}%
\pgfpathclose%
\pgfusepath{fill}%
\end{pgfscope}%
\begin{pgfscope}%
\pgfpathrectangle{\pgfqpoint{3.214225in}{2.423832in}}{\pgfqpoint{1.162500in}{0.755000in}}%
\pgfusepath{clip}%
\pgfsetbuttcap%
\pgfsetroundjoin%
\definecolor{currentfill}{rgb}{0.000000,0.000000,0.000000}%
\pgfsetfillcolor{currentfill}%
\pgfsetfillopacity{0.500000}%
\pgfsetlinewidth{0.000000pt}%
\definecolor{currentstroke}{rgb}{0.000000,0.000000,0.000000}%
\pgfsetstrokecolor{currentstroke}%
\pgfsetdash{}{0pt}%
\pgfpathmoveto{\pgfqpoint{3.695649in}{2.481302in}}%
\pgfpathcurveto{\pgfqpoint{3.701174in}{2.481302in}}{\pgfqpoint{3.706474in}{2.483497in}}{\pgfqpoint{3.710381in}{2.487404in}}%
\pgfpathcurveto{\pgfqpoint{3.714288in}{2.491311in}}{\pgfqpoint{3.716483in}{2.496610in}}{\pgfqpoint{3.716483in}{2.502135in}}%
\pgfpathcurveto{\pgfqpoint{3.716483in}{2.507661in}}{\pgfqpoint{3.714288in}{2.512960in}}{\pgfqpoint{3.710381in}{2.516867in}}%
\pgfpathcurveto{\pgfqpoint{3.706474in}{2.520774in}}{\pgfqpoint{3.701174in}{2.522969in}}{\pgfqpoint{3.695649in}{2.522969in}}%
\pgfpathcurveto{\pgfqpoint{3.690124in}{2.522969in}}{\pgfqpoint{3.684825in}{2.520774in}}{\pgfqpoint{3.680918in}{2.516867in}}%
\pgfpathcurveto{\pgfqpoint{3.677011in}{2.512960in}}{\pgfqpoint{3.674816in}{2.507661in}}{\pgfqpoint{3.674816in}{2.502135in}}%
\pgfpathcurveto{\pgfqpoint{3.674816in}{2.496610in}}{\pgfqpoint{3.677011in}{2.491311in}}{\pgfqpoint{3.680918in}{2.487404in}}%
\pgfpathcurveto{\pgfqpoint{3.684825in}{2.483497in}}{\pgfqpoint{3.690124in}{2.481302in}}{\pgfqpoint{3.695649in}{2.481302in}}%
\pgfpathclose%
\pgfusepath{fill}%
\end{pgfscope}%
\begin{pgfscope}%
\pgfpathrectangle{\pgfqpoint{3.214225in}{2.423832in}}{\pgfqpoint{1.162500in}{0.755000in}}%
\pgfusepath{clip}%
\pgfsetbuttcap%
\pgfsetroundjoin%
\definecolor{currentfill}{rgb}{0.000000,0.000000,0.000000}%
\pgfsetfillcolor{currentfill}%
\pgfsetfillopacity{0.500000}%
\pgfsetlinewidth{0.000000pt}%
\definecolor{currentstroke}{rgb}{0.000000,0.000000,0.000000}%
\pgfsetstrokecolor{currentstroke}%
\pgfsetdash{}{0pt}%
\pgfpathmoveto{\pgfqpoint{3.329643in}{2.503012in}}%
\pgfpathcurveto{\pgfqpoint{3.335168in}{2.503012in}}{\pgfqpoint{3.340468in}{2.505207in}}{\pgfqpoint{3.344375in}{2.509114in}}%
\pgfpathcurveto{\pgfqpoint{3.348281in}{2.513021in}}{\pgfqpoint{3.350476in}{2.518320in}}{\pgfqpoint{3.350476in}{2.523845in}}%
\pgfpathcurveto{\pgfqpoint{3.350476in}{2.529370in}}{\pgfqpoint{3.348281in}{2.534670in}}{\pgfqpoint{3.344375in}{2.538577in}}%
\pgfpathcurveto{\pgfqpoint{3.340468in}{2.542483in}}{\pgfqpoint{3.335168in}{2.544679in}}{\pgfqpoint{3.329643in}{2.544679in}}%
\pgfpathcurveto{\pgfqpoint{3.324118in}{2.544679in}}{\pgfqpoint{3.318819in}{2.542483in}}{\pgfqpoint{3.314912in}{2.538577in}}%
\pgfpathcurveto{\pgfqpoint{3.311005in}{2.534670in}}{\pgfqpoint{3.308810in}{2.529370in}}{\pgfqpoint{3.308810in}{2.523845in}}%
\pgfpathcurveto{\pgfqpoint{3.308810in}{2.518320in}}{\pgfqpoint{3.311005in}{2.513021in}}{\pgfqpoint{3.314912in}{2.509114in}}%
\pgfpathcurveto{\pgfqpoint{3.318819in}{2.505207in}}{\pgfqpoint{3.324118in}{2.503012in}}{\pgfqpoint{3.329643in}{2.503012in}}%
\pgfpathclose%
\pgfusepath{fill}%
\end{pgfscope}%
\begin{pgfscope}%
\pgfpathrectangle{\pgfqpoint{3.214225in}{2.423832in}}{\pgfqpoint{1.162500in}{0.755000in}}%
\pgfusepath{clip}%
\pgfsetbuttcap%
\pgfsetroundjoin%
\definecolor{currentfill}{rgb}{0.000000,0.000000,0.000000}%
\pgfsetfillcolor{currentfill}%
\pgfsetfillopacity{0.500000}%
\pgfsetlinewidth{0.000000pt}%
\definecolor{currentstroke}{rgb}{0.000000,0.000000,0.000000}%
\pgfsetstrokecolor{currentstroke}%
\pgfsetdash{}{0pt}%
\pgfpathmoveto{\pgfqpoint{4.167862in}{3.140023in}}%
\pgfpathcurveto{\pgfqpoint{4.173387in}{3.140023in}}{\pgfqpoint{4.178687in}{3.142218in}}{\pgfqpoint{4.182593in}{3.146125in}}%
\pgfpathcurveto{\pgfqpoint{4.186500in}{3.150032in}}{\pgfqpoint{4.188695in}{3.155331in}}{\pgfqpoint{4.188695in}{3.160856in}}%
\pgfpathcurveto{\pgfqpoint{4.188695in}{3.166381in}}{\pgfqpoint{4.186500in}{3.171681in}}{\pgfqpoint{4.182593in}{3.175588in}}%
\pgfpathcurveto{\pgfqpoint{4.178687in}{3.179494in}}{\pgfqpoint{4.173387in}{3.181690in}}{\pgfqpoint{4.167862in}{3.181690in}}%
\pgfpathcurveto{\pgfqpoint{4.162337in}{3.181690in}}{\pgfqpoint{4.157037in}{3.179494in}}{\pgfqpoint{4.153131in}{3.175588in}}%
\pgfpathcurveto{\pgfqpoint{4.149224in}{3.171681in}}{\pgfqpoint{4.147029in}{3.166381in}}{\pgfqpoint{4.147029in}{3.160856in}}%
\pgfpathcurveto{\pgfqpoint{4.147029in}{3.155331in}}{\pgfqpoint{4.149224in}{3.150032in}}{\pgfqpoint{4.153131in}{3.146125in}}%
\pgfpathcurveto{\pgfqpoint{4.157037in}{3.142218in}}{\pgfqpoint{4.162337in}{3.140023in}}{\pgfqpoint{4.167862in}{3.140023in}}%
\pgfpathclose%
\pgfusepath{fill}%
\end{pgfscope}%
\begin{pgfscope}%
\pgfpathrectangle{\pgfqpoint{3.214225in}{2.423832in}}{\pgfqpoint{1.162500in}{0.755000in}}%
\pgfusepath{clip}%
\pgfsetbuttcap%
\pgfsetroundjoin%
\definecolor{currentfill}{rgb}{0.000000,0.000000,0.000000}%
\pgfsetfillcolor{currentfill}%
\pgfsetfillopacity{0.500000}%
\pgfsetlinewidth{0.000000pt}%
\definecolor{currentstroke}{rgb}{0.000000,0.000000,0.000000}%
\pgfsetstrokecolor{currentstroke}%
\pgfsetdash{}{0pt}%
\pgfpathmoveto{\pgfqpoint{4.120816in}{2.625365in}}%
\pgfpathcurveto{\pgfqpoint{4.126341in}{2.625365in}}{\pgfqpoint{4.131641in}{2.627561in}}{\pgfqpoint{4.135548in}{2.631467in}}%
\pgfpathcurveto{\pgfqpoint{4.139455in}{2.635374in}}{\pgfqpoint{4.141650in}{2.640674in}}{\pgfqpoint{4.141650in}{2.646199in}}%
\pgfpathcurveto{\pgfqpoint{4.141650in}{2.651724in}}{\pgfqpoint{4.139455in}{2.657023in}}{\pgfqpoint{4.135548in}{2.660930in}}%
\pgfpathcurveto{\pgfqpoint{4.131641in}{2.664837in}}{\pgfqpoint{4.126341in}{2.667032in}}{\pgfqpoint{4.120816in}{2.667032in}}%
\pgfpathcurveto{\pgfqpoint{4.115291in}{2.667032in}}{\pgfqpoint{4.109992in}{2.664837in}}{\pgfqpoint{4.106085in}{2.660930in}}%
\pgfpathcurveto{\pgfqpoint{4.102178in}{2.657023in}}{\pgfqpoint{4.099983in}{2.651724in}}{\pgfqpoint{4.099983in}{2.646199in}}%
\pgfpathcurveto{\pgfqpoint{4.099983in}{2.640674in}}{\pgfqpoint{4.102178in}{2.635374in}}{\pgfqpoint{4.106085in}{2.631467in}}%
\pgfpathcurveto{\pgfqpoint{4.109992in}{2.627561in}}{\pgfqpoint{4.115291in}{2.625365in}}{\pgfqpoint{4.120816in}{2.625365in}}%
\pgfpathclose%
\pgfusepath{fill}%
\end{pgfscope}%
\begin{pgfscope}%
\pgfpathrectangle{\pgfqpoint{3.214225in}{2.423832in}}{\pgfqpoint{1.162500in}{0.755000in}}%
\pgfusepath{clip}%
\pgfsetbuttcap%
\pgfsetroundjoin%
\definecolor{currentfill}{rgb}{0.000000,0.000000,0.000000}%
\pgfsetfillcolor{currentfill}%
\pgfsetfillopacity{0.500000}%
\pgfsetlinewidth{0.000000pt}%
\definecolor{currentstroke}{rgb}{0.000000,0.000000,0.000000}%
\pgfsetstrokecolor{currentstroke}%
\pgfsetdash{}{0pt}%
\pgfpathmoveto{\pgfqpoint{3.792736in}{2.450809in}}%
\pgfpathcurveto{\pgfqpoint{3.798261in}{2.450809in}}{\pgfqpoint{3.803560in}{2.453004in}}{\pgfqpoint{3.807467in}{2.456911in}}%
\pgfpathcurveto{\pgfqpoint{3.811374in}{2.460818in}}{\pgfqpoint{3.813569in}{2.466117in}}{\pgfqpoint{3.813569in}{2.471642in}}%
\pgfpathcurveto{\pgfqpoint{3.813569in}{2.477167in}}{\pgfqpoint{3.811374in}{2.482467in}}{\pgfqpoint{3.807467in}{2.486374in}}%
\pgfpathcurveto{\pgfqpoint{3.803560in}{2.490280in}}{\pgfqpoint{3.798261in}{2.492476in}}{\pgfqpoint{3.792736in}{2.492476in}}%
\pgfpathcurveto{\pgfqpoint{3.787211in}{2.492476in}}{\pgfqpoint{3.781911in}{2.490280in}}{\pgfqpoint{3.778004in}{2.486374in}}%
\pgfpathcurveto{\pgfqpoint{3.774097in}{2.482467in}}{\pgfqpoint{3.771902in}{2.477167in}}{\pgfqpoint{3.771902in}{2.471642in}}%
\pgfpathcurveto{\pgfqpoint{3.771902in}{2.466117in}}{\pgfqpoint{3.774097in}{2.460818in}}{\pgfqpoint{3.778004in}{2.456911in}}%
\pgfpathcurveto{\pgfqpoint{3.781911in}{2.453004in}}{\pgfqpoint{3.787211in}{2.450809in}}{\pgfqpoint{3.792736in}{2.450809in}}%
\pgfpathclose%
\pgfusepath{fill}%
\end{pgfscope}%
\begin{pgfscope}%
\pgfsetrectcap%
\pgfsetmiterjoin%
\pgfsetlinewidth{0.803000pt}%
\definecolor{currentstroke}{rgb}{0.501961,0.501961,0.501961}%
\pgfsetstrokecolor{currentstroke}%
\pgfsetdash{}{0pt}%
\pgfpathmoveto{\pgfqpoint{3.214225in}{2.423832in}}%
\pgfpathlineto{\pgfqpoint{3.214225in}{3.178832in}}%
\pgfusepath{stroke}%
\end{pgfscope}%
\begin{pgfscope}%
\pgfsetrectcap%
\pgfsetmiterjoin%
\pgfsetlinewidth{0.803000pt}%
\definecolor{currentstroke}{rgb}{0.501961,0.501961,0.501961}%
\pgfsetstrokecolor{currentstroke}%
\pgfsetdash{}{0pt}%
\pgfpathmoveto{\pgfqpoint{4.376725in}{2.423832in}}%
\pgfpathlineto{\pgfqpoint{4.376725in}{3.178832in}}%
\pgfusepath{stroke}%
\end{pgfscope}%
\begin{pgfscope}%
\pgfsetrectcap%
\pgfsetmiterjoin%
\pgfsetlinewidth{0.803000pt}%
\definecolor{currentstroke}{rgb}{0.501961,0.501961,0.501961}%
\pgfsetstrokecolor{currentstroke}%
\pgfsetdash{}{0pt}%
\pgfpathmoveto{\pgfqpoint{3.214225in}{2.423832in}}%
\pgfpathlineto{\pgfqpoint{4.376725in}{2.423832in}}%
\pgfusepath{stroke}%
\end{pgfscope}%
\begin{pgfscope}%
\pgfsetrectcap%
\pgfsetmiterjoin%
\pgfsetlinewidth{0.803000pt}%
\definecolor{currentstroke}{rgb}{0.501961,0.501961,0.501961}%
\pgfsetstrokecolor{currentstroke}%
\pgfsetdash{}{0pt}%
\pgfpathmoveto{\pgfqpoint{3.214225in}{3.178832in}}%
\pgfpathlineto{\pgfqpoint{4.376725in}{3.178832in}}%
\pgfusepath{stroke}%
\end{pgfscope}%
\begin{pgfscope}%
\pgfsetbuttcap%
\pgfsetmiterjoin%
\definecolor{currentfill}{rgb}{1.000000,1.000000,1.000000}%
\pgfsetfillcolor{currentfill}%
\pgfsetlinewidth{0.000000pt}%
\definecolor{currentstroke}{rgb}{0.000000,0.000000,0.000000}%
\pgfsetstrokecolor{currentstroke}%
\pgfsetstrokeopacity{0.000000}%
\pgfsetdash{}{0pt}%
\pgfpathmoveto{\pgfqpoint{4.376725in}{2.423832in}}%
\pgfpathlineto{\pgfqpoint{5.539225in}{2.423832in}}%
\pgfpathlineto{\pgfqpoint{5.539225in}{3.178832in}}%
\pgfpathlineto{\pgfqpoint{4.376725in}{3.178832in}}%
\pgfpathclose%
\pgfusepath{fill}%
\end{pgfscope}%
\begin{pgfscope}%
\pgfpathrectangle{\pgfqpoint{4.376725in}{2.423832in}}{\pgfqpoint{1.162500in}{0.755000in}}%
\pgfusepath{clip}%
\pgfsetbuttcap%
\pgfsetroundjoin%
\definecolor{currentfill}{rgb}{0.000000,0.000000,0.000000}%
\pgfsetfillcolor{currentfill}%
\pgfsetfillopacity{0.500000}%
\pgfsetlinewidth{0.000000pt}%
\definecolor{currentstroke}{rgb}{0.000000,0.000000,0.000000}%
\pgfsetstrokecolor{currentstroke}%
\pgfsetdash{}{0pt}%
\pgfpathmoveto{\pgfqpoint{5.214960in}{3.088730in}}%
\pgfpathcurveto{\pgfqpoint{5.220485in}{3.088730in}}{\pgfqpoint{5.225785in}{3.090925in}}{\pgfqpoint{5.229692in}{3.094832in}}%
\pgfpathcurveto{\pgfqpoint{5.233598in}{3.098739in}}{\pgfqpoint{5.235794in}{3.104038in}}{\pgfqpoint{5.235794in}{3.109563in}}%
\pgfpathcurveto{\pgfqpoint{5.235794in}{3.115089in}}{\pgfqpoint{5.233598in}{3.120388in}}{\pgfqpoint{5.229692in}{3.124295in}}%
\pgfpathcurveto{\pgfqpoint{5.225785in}{3.128202in}}{\pgfqpoint{5.220485in}{3.130397in}}{\pgfqpoint{5.214960in}{3.130397in}}%
\pgfpathcurveto{\pgfqpoint{5.209435in}{3.130397in}}{\pgfqpoint{5.204136in}{3.128202in}}{\pgfqpoint{5.200229in}{3.124295in}}%
\pgfpathcurveto{\pgfqpoint{5.196322in}{3.120388in}}{\pgfqpoint{5.194127in}{3.115089in}}{\pgfqpoint{5.194127in}{3.109563in}}%
\pgfpathcurveto{\pgfqpoint{5.194127in}{3.104038in}}{\pgfqpoint{5.196322in}{3.098739in}}{\pgfqpoint{5.200229in}{3.094832in}}%
\pgfpathcurveto{\pgfqpoint{5.204136in}{3.090925in}}{\pgfqpoint{5.209435in}{3.088730in}}{\pgfqpoint{5.214960in}{3.088730in}}%
\pgfpathclose%
\pgfusepath{fill}%
\end{pgfscope}%
\begin{pgfscope}%
\pgfpathrectangle{\pgfqpoint{4.376725in}{2.423832in}}{\pgfqpoint{1.162500in}{0.755000in}}%
\pgfusepath{clip}%
\pgfsetbuttcap%
\pgfsetroundjoin%
\definecolor{currentfill}{rgb}{0.000000,0.000000,0.000000}%
\pgfsetfillcolor{currentfill}%
\pgfsetfillopacity{0.500000}%
\pgfsetlinewidth{0.000000pt}%
\definecolor{currentstroke}{rgb}{0.000000,0.000000,0.000000}%
\pgfsetstrokecolor{currentstroke}%
\pgfsetdash{}{0pt}%
\pgfpathmoveto{\pgfqpoint{4.521739in}{2.544873in}}%
\pgfpathcurveto{\pgfqpoint{4.527264in}{2.544873in}}{\pgfqpoint{4.532563in}{2.547068in}}{\pgfqpoint{4.536470in}{2.550975in}}%
\pgfpathcurveto{\pgfqpoint{4.540377in}{2.554881in}}{\pgfqpoint{4.542572in}{2.560181in}}{\pgfqpoint{4.542572in}{2.565706in}}%
\pgfpathcurveto{\pgfqpoint{4.542572in}{2.571231in}}{\pgfqpoint{4.540377in}{2.576531in}}{\pgfqpoint{4.536470in}{2.580437in}}%
\pgfpathcurveto{\pgfqpoint{4.532563in}{2.584344in}}{\pgfqpoint{4.527264in}{2.586539in}}{\pgfqpoint{4.521739in}{2.586539in}}%
\pgfpathcurveto{\pgfqpoint{4.516214in}{2.586539in}}{\pgfqpoint{4.510914in}{2.584344in}}{\pgfqpoint{4.507007in}{2.580437in}}%
\pgfpathcurveto{\pgfqpoint{4.503101in}{2.576531in}}{\pgfqpoint{4.500905in}{2.571231in}}{\pgfqpoint{4.500905in}{2.565706in}}%
\pgfpathcurveto{\pgfqpoint{4.500905in}{2.560181in}}{\pgfqpoint{4.503101in}{2.554881in}}{\pgfqpoint{4.507007in}{2.550975in}}%
\pgfpathcurveto{\pgfqpoint{4.510914in}{2.547068in}}{\pgfqpoint{4.516214in}{2.544873in}}{\pgfqpoint{4.521739in}{2.544873in}}%
\pgfpathclose%
\pgfusepath{fill}%
\end{pgfscope}%
\begin{pgfscope}%
\pgfpathrectangle{\pgfqpoint{4.376725in}{2.423832in}}{\pgfqpoint{1.162500in}{0.755000in}}%
\pgfusepath{clip}%
\pgfsetbuttcap%
\pgfsetroundjoin%
\definecolor{currentfill}{rgb}{0.000000,0.000000,0.000000}%
\pgfsetfillcolor{currentfill}%
\pgfsetfillopacity{0.500000}%
\pgfsetlinewidth{0.000000pt}%
\definecolor{currentstroke}{rgb}{0.000000,0.000000,0.000000}%
\pgfsetstrokecolor{currentstroke}%
\pgfsetdash{}{0pt}%
\pgfpathmoveto{\pgfqpoint{4.448919in}{2.541427in}}%
\pgfpathcurveto{\pgfqpoint{4.454444in}{2.541427in}}{\pgfqpoint{4.459744in}{2.543622in}}{\pgfqpoint{4.463650in}{2.547529in}}%
\pgfpathcurveto{\pgfqpoint{4.467557in}{2.551436in}}{\pgfqpoint{4.469752in}{2.556735in}}{\pgfqpoint{4.469752in}{2.562260in}}%
\pgfpathcurveto{\pgfqpoint{4.469752in}{2.567785in}}{\pgfqpoint{4.467557in}{2.573085in}}{\pgfqpoint{4.463650in}{2.576992in}}%
\pgfpathcurveto{\pgfqpoint{4.459744in}{2.580899in}}{\pgfqpoint{4.454444in}{2.583094in}}{\pgfqpoint{4.448919in}{2.583094in}}%
\pgfpathcurveto{\pgfqpoint{4.443394in}{2.583094in}}{\pgfqpoint{4.438094in}{2.580899in}}{\pgfqpoint{4.434188in}{2.576992in}}%
\pgfpathcurveto{\pgfqpoint{4.430281in}{2.573085in}}{\pgfqpoint{4.428086in}{2.567785in}}{\pgfqpoint{4.428086in}{2.562260in}}%
\pgfpathcurveto{\pgfqpoint{4.428086in}{2.556735in}}{\pgfqpoint{4.430281in}{2.551436in}}{\pgfqpoint{4.434188in}{2.547529in}}%
\pgfpathcurveto{\pgfqpoint{4.438094in}{2.543622in}}{\pgfqpoint{4.443394in}{2.541427in}}{\pgfqpoint{4.448919in}{2.541427in}}%
\pgfpathclose%
\pgfusepath{fill}%
\end{pgfscope}%
\begin{pgfscope}%
\pgfpathrectangle{\pgfqpoint{4.376725in}{2.423832in}}{\pgfqpoint{1.162500in}{0.755000in}}%
\pgfusepath{clip}%
\pgfsetbuttcap%
\pgfsetroundjoin%
\definecolor{currentfill}{rgb}{0.000000,0.000000,0.000000}%
\pgfsetfillcolor{currentfill}%
\pgfsetfillopacity{0.500000}%
\pgfsetlinewidth{0.000000pt}%
\definecolor{currentstroke}{rgb}{0.000000,0.000000,0.000000}%
\pgfsetstrokecolor{currentstroke}%
\pgfsetdash{}{0pt}%
\pgfpathmoveto{\pgfqpoint{4.704251in}{2.469548in}}%
\pgfpathcurveto{\pgfqpoint{4.709776in}{2.469548in}}{\pgfqpoint{4.715075in}{2.471743in}}{\pgfqpoint{4.718982in}{2.475650in}}%
\pgfpathcurveto{\pgfqpoint{4.722889in}{2.479557in}}{\pgfqpoint{4.725084in}{2.484856in}}{\pgfqpoint{4.725084in}{2.490381in}}%
\pgfpathcurveto{\pgfqpoint{4.725084in}{2.495907in}}{\pgfqpoint{4.722889in}{2.501206in}}{\pgfqpoint{4.718982in}{2.505113in}}%
\pgfpathcurveto{\pgfqpoint{4.715075in}{2.509020in}}{\pgfqpoint{4.709776in}{2.511215in}}{\pgfqpoint{4.704251in}{2.511215in}}%
\pgfpathcurveto{\pgfqpoint{4.698726in}{2.511215in}}{\pgfqpoint{4.693426in}{2.509020in}}{\pgfqpoint{4.689519in}{2.505113in}}%
\pgfpathcurveto{\pgfqpoint{4.685612in}{2.501206in}}{\pgfqpoint{4.683417in}{2.495907in}}{\pgfqpoint{4.683417in}{2.490381in}}%
\pgfpathcurveto{\pgfqpoint{4.683417in}{2.484856in}}{\pgfqpoint{4.685612in}{2.479557in}}{\pgfqpoint{4.689519in}{2.475650in}}%
\pgfpathcurveto{\pgfqpoint{4.693426in}{2.471743in}}{\pgfqpoint{4.698726in}{2.469548in}}{\pgfqpoint{4.704251in}{2.469548in}}%
\pgfpathclose%
\pgfusepath{fill}%
\end{pgfscope}%
\begin{pgfscope}%
\pgfpathrectangle{\pgfqpoint{4.376725in}{2.423832in}}{\pgfqpoint{1.162500in}{0.755000in}}%
\pgfusepath{clip}%
\pgfsetbuttcap%
\pgfsetroundjoin%
\definecolor{currentfill}{rgb}{0.000000,0.000000,0.000000}%
\pgfsetfillcolor{currentfill}%
\pgfsetfillopacity{0.500000}%
\pgfsetlinewidth{0.000000pt}%
\definecolor{currentstroke}{rgb}{0.000000,0.000000,0.000000}%
\pgfsetstrokecolor{currentstroke}%
\pgfsetdash{}{0pt}%
\pgfpathmoveto{\pgfqpoint{4.404404in}{2.488481in}}%
\pgfpathcurveto{\pgfqpoint{4.409929in}{2.488481in}}{\pgfqpoint{4.415228in}{2.490676in}}{\pgfqpoint{4.419135in}{2.494583in}}%
\pgfpathcurveto{\pgfqpoint{4.423042in}{2.498489in}}{\pgfqpoint{4.425237in}{2.503789in}}{\pgfqpoint{4.425237in}{2.509314in}}%
\pgfpathcurveto{\pgfqpoint{4.425237in}{2.514839in}}{\pgfqpoint{4.423042in}{2.520139in}}{\pgfqpoint{4.419135in}{2.524045in}}%
\pgfpathcurveto{\pgfqpoint{4.415228in}{2.527952in}}{\pgfqpoint{4.409929in}{2.530147in}}{\pgfqpoint{4.404404in}{2.530147in}}%
\pgfpathcurveto{\pgfqpoint{4.398879in}{2.530147in}}{\pgfqpoint{4.393579in}{2.527952in}}{\pgfqpoint{4.389672in}{2.524045in}}%
\pgfpathcurveto{\pgfqpoint{4.385765in}{2.520139in}}{\pgfqpoint{4.383570in}{2.514839in}}{\pgfqpoint{4.383570in}{2.509314in}}%
\pgfpathcurveto{\pgfqpoint{4.383570in}{2.503789in}}{\pgfqpoint{4.385765in}{2.498489in}}{\pgfqpoint{4.389672in}{2.494583in}}%
\pgfpathcurveto{\pgfqpoint{4.393579in}{2.490676in}}{\pgfqpoint{4.398879in}{2.488481in}}{\pgfqpoint{4.404404in}{2.488481in}}%
\pgfpathclose%
\pgfusepath{fill}%
\end{pgfscope}%
\begin{pgfscope}%
\pgfpathrectangle{\pgfqpoint{4.376725in}{2.423832in}}{\pgfqpoint{1.162500in}{0.755000in}}%
\pgfusepath{clip}%
\pgfsetbuttcap%
\pgfsetroundjoin%
\definecolor{currentfill}{rgb}{0.000000,0.000000,0.000000}%
\pgfsetfillcolor{currentfill}%
\pgfsetfillopacity{0.500000}%
\pgfsetlinewidth{0.000000pt}%
\definecolor{currentstroke}{rgb}{0.000000,0.000000,0.000000}%
\pgfsetstrokecolor{currentstroke}%
\pgfsetdash{}{0pt}%
\pgfpathmoveto{\pgfqpoint{4.607580in}{2.422812in}}%
\pgfpathcurveto{\pgfqpoint{4.613105in}{2.422812in}}{\pgfqpoint{4.618405in}{2.425007in}}{\pgfqpoint{4.622312in}{2.428914in}}%
\pgfpathcurveto{\pgfqpoint{4.626218in}{2.432821in}}{\pgfqpoint{4.628414in}{2.438120in}}{\pgfqpoint{4.628414in}{2.443645in}}%
\pgfpathcurveto{\pgfqpoint{4.628414in}{2.449170in}}{\pgfqpoint{4.626218in}{2.454470in}}{\pgfqpoint{4.622312in}{2.458377in}}%
\pgfpathcurveto{\pgfqpoint{4.618405in}{2.462284in}}{\pgfqpoint{4.613105in}{2.464479in}}{\pgfqpoint{4.607580in}{2.464479in}}%
\pgfpathcurveto{\pgfqpoint{4.602055in}{2.464479in}}{\pgfqpoint{4.596756in}{2.462284in}}{\pgfqpoint{4.592849in}{2.458377in}}%
\pgfpathcurveto{\pgfqpoint{4.588942in}{2.454470in}}{\pgfqpoint{4.586747in}{2.449170in}}{\pgfqpoint{4.586747in}{2.443645in}}%
\pgfpathcurveto{\pgfqpoint{4.586747in}{2.438120in}}{\pgfqpoint{4.588942in}{2.432821in}}{\pgfqpoint{4.592849in}{2.428914in}}%
\pgfpathcurveto{\pgfqpoint{4.596756in}{2.425007in}}{\pgfqpoint{4.602055in}{2.422812in}}{\pgfqpoint{4.607580in}{2.422812in}}%
\pgfpathclose%
\pgfusepath{fill}%
\end{pgfscope}%
\begin{pgfscope}%
\pgfpathrectangle{\pgfqpoint{4.376725in}{2.423832in}}{\pgfqpoint{1.162500in}{0.755000in}}%
\pgfusepath{clip}%
\pgfsetbuttcap%
\pgfsetroundjoin%
\definecolor{currentfill}{rgb}{0.000000,0.000000,0.000000}%
\pgfsetfillcolor{currentfill}%
\pgfsetfillopacity{0.500000}%
\pgfsetlinewidth{0.000000pt}%
\definecolor{currentstroke}{rgb}{0.000000,0.000000,0.000000}%
\pgfsetstrokecolor{currentstroke}%
\pgfsetdash{}{0pt}%
\pgfpathmoveto{\pgfqpoint{4.596304in}{2.420975in}}%
\pgfpathcurveto{\pgfqpoint{4.601829in}{2.420975in}}{\pgfqpoint{4.607129in}{2.423170in}}{\pgfqpoint{4.611036in}{2.427077in}}%
\pgfpathcurveto{\pgfqpoint{4.614942in}{2.430984in}}{\pgfqpoint{4.617137in}{2.436283in}}{\pgfqpoint{4.617137in}{2.441809in}}%
\pgfpathcurveto{\pgfqpoint{4.617137in}{2.447334in}}{\pgfqpoint{4.614942in}{2.452633in}}{\pgfqpoint{4.611036in}{2.456540in}}%
\pgfpathcurveto{\pgfqpoint{4.607129in}{2.460447in}}{\pgfqpoint{4.601829in}{2.462642in}}{\pgfqpoint{4.596304in}{2.462642in}}%
\pgfpathcurveto{\pgfqpoint{4.590779in}{2.462642in}}{\pgfqpoint{4.585480in}{2.460447in}}{\pgfqpoint{4.581573in}{2.456540in}}%
\pgfpathcurveto{\pgfqpoint{4.577666in}{2.452633in}}{\pgfqpoint{4.575471in}{2.447334in}}{\pgfqpoint{4.575471in}{2.441809in}}%
\pgfpathcurveto{\pgfqpoint{4.575471in}{2.436283in}}{\pgfqpoint{4.577666in}{2.430984in}}{\pgfqpoint{4.581573in}{2.427077in}}%
\pgfpathcurveto{\pgfqpoint{4.585480in}{2.423170in}}{\pgfqpoint{4.590779in}{2.420975in}}{\pgfqpoint{4.596304in}{2.420975in}}%
\pgfpathclose%
\pgfusepath{fill}%
\end{pgfscope}%
\begin{pgfscope}%
\pgfpathrectangle{\pgfqpoint{4.376725in}{2.423832in}}{\pgfqpoint{1.162500in}{0.755000in}}%
\pgfusepath{clip}%
\pgfsetbuttcap%
\pgfsetroundjoin%
\definecolor{currentfill}{rgb}{0.000000,0.000000,0.000000}%
\pgfsetfillcolor{currentfill}%
\pgfsetfillopacity{0.500000}%
\pgfsetlinewidth{0.000000pt}%
\definecolor{currentstroke}{rgb}{0.000000,0.000000,0.000000}%
\pgfsetstrokecolor{currentstroke}%
\pgfsetdash{}{0pt}%
\pgfpathmoveto{\pgfqpoint{5.423123in}{2.717961in}}%
\pgfpathcurveto{\pgfqpoint{5.428649in}{2.717961in}}{\pgfqpoint{5.433948in}{2.720156in}}{\pgfqpoint{5.437855in}{2.724063in}}%
\pgfpathcurveto{\pgfqpoint{5.441762in}{2.727970in}}{\pgfqpoint{5.443957in}{2.733269in}}{\pgfqpoint{5.443957in}{2.738795in}}%
\pgfpathcurveto{\pgfqpoint{5.443957in}{2.744320in}}{\pgfqpoint{5.441762in}{2.749619in}}{\pgfqpoint{5.437855in}{2.753526in}}%
\pgfpathcurveto{\pgfqpoint{5.433948in}{2.757433in}}{\pgfqpoint{5.428649in}{2.759628in}}{\pgfqpoint{5.423123in}{2.759628in}}%
\pgfpathcurveto{\pgfqpoint{5.417598in}{2.759628in}}{\pgfqpoint{5.412299in}{2.757433in}}{\pgfqpoint{5.408392in}{2.753526in}}%
\pgfpathcurveto{\pgfqpoint{5.404485in}{2.749619in}}{\pgfqpoint{5.402290in}{2.744320in}}{\pgfqpoint{5.402290in}{2.738795in}}%
\pgfpathcurveto{\pgfqpoint{5.402290in}{2.733269in}}{\pgfqpoint{5.404485in}{2.727970in}}{\pgfqpoint{5.408392in}{2.724063in}}%
\pgfpathcurveto{\pgfqpoint{5.412299in}{2.720156in}}{\pgfqpoint{5.417598in}{2.717961in}}{\pgfqpoint{5.423123in}{2.717961in}}%
\pgfpathclose%
\pgfusepath{fill}%
\end{pgfscope}%
\begin{pgfscope}%
\pgfpathrectangle{\pgfqpoint{4.376725in}{2.423832in}}{\pgfqpoint{1.162500in}{0.755000in}}%
\pgfusepath{clip}%
\pgfsetbuttcap%
\pgfsetroundjoin%
\definecolor{currentfill}{rgb}{0.000000,0.000000,0.000000}%
\pgfsetfillcolor{currentfill}%
\pgfsetfillopacity{0.500000}%
\pgfsetlinewidth{0.000000pt}%
\definecolor{currentstroke}{rgb}{0.000000,0.000000,0.000000}%
\pgfsetstrokecolor{currentstroke}%
\pgfsetdash{}{0pt}%
\pgfpathmoveto{\pgfqpoint{5.511546in}{2.640479in}}%
\pgfpathcurveto{\pgfqpoint{5.517071in}{2.640479in}}{\pgfqpoint{5.522371in}{2.642674in}}{\pgfqpoint{5.526278in}{2.646581in}}%
\pgfpathcurveto{\pgfqpoint{5.530185in}{2.650487in}}{\pgfqpoint{5.532380in}{2.655787in}}{\pgfqpoint{5.532380in}{2.661312in}}%
\pgfpathcurveto{\pgfqpoint{5.532380in}{2.666837in}}{\pgfqpoint{5.530185in}{2.672137in}}{\pgfqpoint{5.526278in}{2.676043in}}%
\pgfpathcurveto{\pgfqpoint{5.522371in}{2.679950in}}{\pgfqpoint{5.517071in}{2.682145in}}{\pgfqpoint{5.511546in}{2.682145in}}%
\pgfpathcurveto{\pgfqpoint{5.506021in}{2.682145in}}{\pgfqpoint{5.500722in}{2.679950in}}{\pgfqpoint{5.496815in}{2.676043in}}%
\pgfpathcurveto{\pgfqpoint{5.492908in}{2.672137in}}{\pgfqpoint{5.490713in}{2.666837in}}{\pgfqpoint{5.490713in}{2.661312in}}%
\pgfpathcurveto{\pgfqpoint{5.490713in}{2.655787in}}{\pgfqpoint{5.492908in}{2.650487in}}{\pgfqpoint{5.496815in}{2.646581in}}%
\pgfpathcurveto{\pgfqpoint{5.500722in}{2.642674in}}{\pgfqpoint{5.506021in}{2.640479in}}{\pgfqpoint{5.511546in}{2.640479in}}%
\pgfpathclose%
\pgfusepath{fill}%
\end{pgfscope}%
\begin{pgfscope}%
\pgfpathrectangle{\pgfqpoint{4.376725in}{2.423832in}}{\pgfqpoint{1.162500in}{0.755000in}}%
\pgfusepath{clip}%
\pgfsetbuttcap%
\pgfsetroundjoin%
\definecolor{currentfill}{rgb}{0.000000,0.000000,0.000000}%
\pgfsetfillcolor{currentfill}%
\pgfsetfillopacity{0.500000}%
\pgfsetlinewidth{0.000000pt}%
\definecolor{currentstroke}{rgb}{0.000000,0.000000,0.000000}%
\pgfsetstrokecolor{currentstroke}%
\pgfsetdash{}{0pt}%
\pgfpathmoveto{\pgfqpoint{5.286995in}{2.637393in}}%
\pgfpathcurveto{\pgfqpoint{5.292520in}{2.637393in}}{\pgfqpoint{5.297820in}{2.639588in}}{\pgfqpoint{5.301726in}{2.643495in}}%
\pgfpathcurveto{\pgfqpoint{5.305633in}{2.647402in}}{\pgfqpoint{5.307828in}{2.652701in}}{\pgfqpoint{5.307828in}{2.658226in}}%
\pgfpathcurveto{\pgfqpoint{5.307828in}{2.663751in}}{\pgfqpoint{5.305633in}{2.669051in}}{\pgfqpoint{5.301726in}{2.672958in}}%
\pgfpathcurveto{\pgfqpoint{5.297820in}{2.676865in}}{\pgfqpoint{5.292520in}{2.679060in}}{\pgfqpoint{5.286995in}{2.679060in}}%
\pgfpathcurveto{\pgfqpoint{5.281470in}{2.679060in}}{\pgfqpoint{5.276170in}{2.676865in}}{\pgfqpoint{5.272264in}{2.672958in}}%
\pgfpathcurveto{\pgfqpoint{5.268357in}{2.669051in}}{\pgfqpoint{5.266162in}{2.663751in}}{\pgfqpoint{5.266162in}{2.658226in}}%
\pgfpathcurveto{\pgfqpoint{5.266162in}{2.652701in}}{\pgfqpoint{5.268357in}{2.647402in}}{\pgfqpoint{5.272264in}{2.643495in}}%
\pgfpathcurveto{\pgfqpoint{5.276170in}{2.639588in}}{\pgfqpoint{5.281470in}{2.637393in}}{\pgfqpoint{5.286995in}{2.637393in}}%
\pgfpathclose%
\pgfusepath{fill}%
\end{pgfscope}%
\begin{pgfscope}%
\pgfpathrectangle{\pgfqpoint{4.376725in}{2.423832in}}{\pgfqpoint{1.162500in}{0.755000in}}%
\pgfusepath{clip}%
\pgfsetbuttcap%
\pgfsetroundjoin%
\definecolor{currentfill}{rgb}{0.000000,0.000000,0.000000}%
\pgfsetfillcolor{currentfill}%
\pgfsetfillopacity{0.500000}%
\pgfsetlinewidth{0.000000pt}%
\definecolor{currentstroke}{rgb}{0.000000,0.000000,0.000000}%
\pgfsetstrokecolor{currentstroke}%
\pgfsetdash{}{0pt}%
\pgfpathmoveto{\pgfqpoint{4.645822in}{2.679970in}}%
\pgfpathcurveto{\pgfqpoint{4.651348in}{2.679970in}}{\pgfqpoint{4.656647in}{2.682166in}}{\pgfqpoint{4.660554in}{2.686072in}}%
\pgfpathcurveto{\pgfqpoint{4.664461in}{2.689979in}}{\pgfqpoint{4.666656in}{2.695279in}}{\pgfqpoint{4.666656in}{2.700804in}}%
\pgfpathcurveto{\pgfqpoint{4.666656in}{2.706329in}}{\pgfqpoint{4.664461in}{2.711628in}}{\pgfqpoint{4.660554in}{2.715535in}}%
\pgfpathcurveto{\pgfqpoint{4.656647in}{2.719442in}}{\pgfqpoint{4.651348in}{2.721637in}}{\pgfqpoint{4.645822in}{2.721637in}}%
\pgfpathcurveto{\pgfqpoint{4.640297in}{2.721637in}}{\pgfqpoint{4.634998in}{2.719442in}}{\pgfqpoint{4.631091in}{2.715535in}}%
\pgfpathcurveto{\pgfqpoint{4.627184in}{2.711628in}}{\pgfqpoint{4.624989in}{2.706329in}}{\pgfqpoint{4.624989in}{2.700804in}}%
\pgfpathcurveto{\pgfqpoint{4.624989in}{2.695279in}}{\pgfqpoint{4.627184in}{2.689979in}}{\pgfqpoint{4.631091in}{2.686072in}}%
\pgfpathcurveto{\pgfqpoint{4.634998in}{2.682166in}}{\pgfqpoint{4.640297in}{2.679970in}}{\pgfqpoint{4.645822in}{2.679970in}}%
\pgfpathclose%
\pgfusepath{fill}%
\end{pgfscope}%
\begin{pgfscope}%
\pgfpathrectangle{\pgfqpoint{4.376725in}{2.423832in}}{\pgfqpoint{1.162500in}{0.755000in}}%
\pgfusepath{clip}%
\pgfsetbuttcap%
\pgfsetroundjoin%
\definecolor{currentfill}{rgb}{0.000000,0.000000,0.000000}%
\pgfsetfillcolor{currentfill}%
\pgfsetfillopacity{0.500000}%
\pgfsetlinewidth{0.000000pt}%
\definecolor{currentstroke}{rgb}{0.000000,0.000000,0.000000}%
\pgfsetstrokecolor{currentstroke}%
\pgfsetdash{}{0pt}%
\pgfpathmoveto{\pgfqpoint{5.069839in}{2.481302in}}%
\pgfpathcurveto{\pgfqpoint{5.075364in}{2.481302in}}{\pgfqpoint{5.080663in}{2.483497in}}{\pgfqpoint{5.084570in}{2.487404in}}%
\pgfpathcurveto{\pgfqpoint{5.088477in}{2.491311in}}{\pgfqpoint{5.090672in}{2.496610in}}{\pgfqpoint{5.090672in}{2.502135in}}%
\pgfpathcurveto{\pgfqpoint{5.090672in}{2.507661in}}{\pgfqpoint{5.088477in}{2.512960in}}{\pgfqpoint{5.084570in}{2.516867in}}%
\pgfpathcurveto{\pgfqpoint{5.080663in}{2.520774in}}{\pgfqpoint{5.075364in}{2.522969in}}{\pgfqpoint{5.069839in}{2.522969in}}%
\pgfpathcurveto{\pgfqpoint{5.064313in}{2.522969in}}{\pgfqpoint{5.059014in}{2.520774in}}{\pgfqpoint{5.055107in}{2.516867in}}%
\pgfpathcurveto{\pgfqpoint{5.051200in}{2.512960in}}{\pgfqpoint{5.049005in}{2.507661in}}{\pgfqpoint{5.049005in}{2.502135in}}%
\pgfpathcurveto{\pgfqpoint{5.049005in}{2.496610in}}{\pgfqpoint{5.051200in}{2.491311in}}{\pgfqpoint{5.055107in}{2.487404in}}%
\pgfpathcurveto{\pgfqpoint{5.059014in}{2.483497in}}{\pgfqpoint{5.064313in}{2.481302in}}{\pgfqpoint{5.069839in}{2.481302in}}%
\pgfpathclose%
\pgfusepath{fill}%
\end{pgfscope}%
\begin{pgfscope}%
\pgfpathrectangle{\pgfqpoint{4.376725in}{2.423832in}}{\pgfqpoint{1.162500in}{0.755000in}}%
\pgfusepath{clip}%
\pgfsetbuttcap%
\pgfsetroundjoin%
\definecolor{currentfill}{rgb}{0.000000,0.000000,0.000000}%
\pgfsetfillcolor{currentfill}%
\pgfsetfillopacity{0.500000}%
\pgfsetlinewidth{0.000000pt}%
\definecolor{currentstroke}{rgb}{0.000000,0.000000,0.000000}%
\pgfsetstrokecolor{currentstroke}%
\pgfsetdash{}{0pt}%
\pgfpathmoveto{\pgfqpoint{4.941375in}{2.503012in}}%
\pgfpathcurveto{\pgfqpoint{4.946900in}{2.503012in}}{\pgfqpoint{4.952199in}{2.505207in}}{\pgfqpoint{4.956106in}{2.509114in}}%
\pgfpathcurveto{\pgfqpoint{4.960013in}{2.513021in}}{\pgfqpoint{4.962208in}{2.518320in}}{\pgfqpoint{4.962208in}{2.523845in}}%
\pgfpathcurveto{\pgfqpoint{4.962208in}{2.529370in}}{\pgfqpoint{4.960013in}{2.534670in}}{\pgfqpoint{4.956106in}{2.538577in}}%
\pgfpathcurveto{\pgfqpoint{4.952199in}{2.542483in}}{\pgfqpoint{4.946900in}{2.544679in}}{\pgfqpoint{4.941375in}{2.544679in}}%
\pgfpathcurveto{\pgfqpoint{4.935850in}{2.544679in}}{\pgfqpoint{4.930550in}{2.542483in}}{\pgfqpoint{4.926643in}{2.538577in}}%
\pgfpathcurveto{\pgfqpoint{4.922736in}{2.534670in}}{\pgfqpoint{4.920541in}{2.529370in}}{\pgfqpoint{4.920541in}{2.523845in}}%
\pgfpathcurveto{\pgfqpoint{4.920541in}{2.518320in}}{\pgfqpoint{4.922736in}{2.513021in}}{\pgfqpoint{4.926643in}{2.509114in}}%
\pgfpathcurveto{\pgfqpoint{4.930550in}{2.505207in}}{\pgfqpoint{4.935850in}{2.503012in}}{\pgfqpoint{4.941375in}{2.503012in}}%
\pgfpathclose%
\pgfusepath{fill}%
\end{pgfscope}%
\begin{pgfscope}%
\pgfpathrectangle{\pgfqpoint{4.376725in}{2.423832in}}{\pgfqpoint{1.162500in}{0.755000in}}%
\pgfusepath{clip}%
\pgfsetbuttcap%
\pgfsetroundjoin%
\definecolor{currentfill}{rgb}{0.000000,0.000000,0.000000}%
\pgfsetfillcolor{currentfill}%
\pgfsetfillopacity{0.500000}%
\pgfsetlinewidth{0.000000pt}%
\definecolor{currentstroke}{rgb}{0.000000,0.000000,0.000000}%
\pgfsetstrokecolor{currentstroke}%
\pgfsetdash{}{0pt}%
\pgfpathmoveto{\pgfqpoint{4.799153in}{3.140023in}}%
\pgfpathcurveto{\pgfqpoint{4.804678in}{3.140023in}}{\pgfqpoint{4.809977in}{3.142218in}}{\pgfqpoint{4.813884in}{3.146125in}}%
\pgfpathcurveto{\pgfqpoint{4.817791in}{3.150032in}}{\pgfqpoint{4.819986in}{3.155331in}}{\pgfqpoint{4.819986in}{3.160856in}}%
\pgfpathcurveto{\pgfqpoint{4.819986in}{3.166381in}}{\pgfqpoint{4.817791in}{3.171681in}}{\pgfqpoint{4.813884in}{3.175588in}}%
\pgfpathcurveto{\pgfqpoint{4.809977in}{3.179494in}}{\pgfqpoint{4.804678in}{3.181690in}}{\pgfqpoint{4.799153in}{3.181690in}}%
\pgfpathcurveto{\pgfqpoint{4.793628in}{3.181690in}}{\pgfqpoint{4.788328in}{3.179494in}}{\pgfqpoint{4.784421in}{3.175588in}}%
\pgfpathcurveto{\pgfqpoint{4.780515in}{3.171681in}}{\pgfqpoint{4.778320in}{3.166381in}}{\pgfqpoint{4.778320in}{3.160856in}}%
\pgfpathcurveto{\pgfqpoint{4.778320in}{3.155331in}}{\pgfqpoint{4.780515in}{3.150032in}}{\pgfqpoint{4.784421in}{3.146125in}}%
\pgfpathcurveto{\pgfqpoint{4.788328in}{3.142218in}}{\pgfqpoint{4.793628in}{3.140023in}}{\pgfqpoint{4.799153in}{3.140023in}}%
\pgfpathclose%
\pgfusepath{fill}%
\end{pgfscope}%
\begin{pgfscope}%
\pgfpathrectangle{\pgfqpoint{4.376725in}{2.423832in}}{\pgfqpoint{1.162500in}{0.755000in}}%
\pgfusepath{clip}%
\pgfsetbuttcap%
\pgfsetroundjoin%
\definecolor{currentfill}{rgb}{0.000000,0.000000,0.000000}%
\pgfsetfillcolor{currentfill}%
\pgfsetfillopacity{0.500000}%
\pgfsetlinewidth{0.000000pt}%
\definecolor{currentstroke}{rgb}{0.000000,0.000000,0.000000}%
\pgfsetstrokecolor{currentstroke}%
\pgfsetdash{}{0pt}%
\pgfpathmoveto{\pgfqpoint{4.595218in}{2.625365in}}%
\pgfpathcurveto{\pgfqpoint{4.600743in}{2.625365in}}{\pgfqpoint{4.606043in}{2.627561in}}{\pgfqpoint{4.609950in}{2.631467in}}%
\pgfpathcurveto{\pgfqpoint{4.613857in}{2.635374in}}{\pgfqpoint{4.616052in}{2.640674in}}{\pgfqpoint{4.616052in}{2.646199in}}%
\pgfpathcurveto{\pgfqpoint{4.616052in}{2.651724in}}{\pgfqpoint{4.613857in}{2.657023in}}{\pgfqpoint{4.609950in}{2.660930in}}%
\pgfpathcurveto{\pgfqpoint{4.606043in}{2.664837in}}{\pgfqpoint{4.600743in}{2.667032in}}{\pgfqpoint{4.595218in}{2.667032in}}%
\pgfpathcurveto{\pgfqpoint{4.589693in}{2.667032in}}{\pgfqpoint{4.584394in}{2.664837in}}{\pgfqpoint{4.580487in}{2.660930in}}%
\pgfpathcurveto{\pgfqpoint{4.576580in}{2.657023in}}{\pgfqpoint{4.574385in}{2.651724in}}{\pgfqpoint{4.574385in}{2.646199in}}%
\pgfpathcurveto{\pgfqpoint{4.574385in}{2.640674in}}{\pgfqpoint{4.576580in}{2.635374in}}{\pgfqpoint{4.580487in}{2.631467in}}%
\pgfpathcurveto{\pgfqpoint{4.584394in}{2.627561in}}{\pgfqpoint{4.589693in}{2.625365in}}{\pgfqpoint{4.595218in}{2.625365in}}%
\pgfpathclose%
\pgfusepath{fill}%
\end{pgfscope}%
\begin{pgfscope}%
\pgfpathrectangle{\pgfqpoint{4.376725in}{2.423832in}}{\pgfqpoint{1.162500in}{0.755000in}}%
\pgfusepath{clip}%
\pgfsetbuttcap%
\pgfsetroundjoin%
\definecolor{currentfill}{rgb}{0.000000,0.000000,0.000000}%
\pgfsetfillcolor{currentfill}%
\pgfsetfillopacity{0.500000}%
\pgfsetlinewidth{0.000000pt}%
\definecolor{currentstroke}{rgb}{0.000000,0.000000,0.000000}%
\pgfsetstrokecolor{currentstroke}%
\pgfsetdash{}{0pt}%
\pgfpathmoveto{\pgfqpoint{4.834270in}{2.450809in}}%
\pgfpathcurveto{\pgfqpoint{4.839795in}{2.450809in}}{\pgfqpoint{4.845094in}{2.453004in}}{\pgfqpoint{4.849001in}{2.456911in}}%
\pgfpathcurveto{\pgfqpoint{4.852908in}{2.460818in}}{\pgfqpoint{4.855103in}{2.466117in}}{\pgfqpoint{4.855103in}{2.471642in}}%
\pgfpathcurveto{\pgfqpoint{4.855103in}{2.477167in}}{\pgfqpoint{4.852908in}{2.482467in}}{\pgfqpoint{4.849001in}{2.486374in}}%
\pgfpathcurveto{\pgfqpoint{4.845094in}{2.490280in}}{\pgfqpoint{4.839795in}{2.492476in}}{\pgfqpoint{4.834270in}{2.492476in}}%
\pgfpathcurveto{\pgfqpoint{4.828745in}{2.492476in}}{\pgfqpoint{4.823445in}{2.490280in}}{\pgfqpoint{4.819538in}{2.486374in}}%
\pgfpathcurveto{\pgfqpoint{4.815632in}{2.482467in}}{\pgfqpoint{4.813437in}{2.477167in}}{\pgfqpoint{4.813437in}{2.471642in}}%
\pgfpathcurveto{\pgfqpoint{4.813437in}{2.466117in}}{\pgfqpoint{4.815632in}{2.460818in}}{\pgfqpoint{4.819538in}{2.456911in}}%
\pgfpathcurveto{\pgfqpoint{4.823445in}{2.453004in}}{\pgfqpoint{4.828745in}{2.450809in}}{\pgfqpoint{4.834270in}{2.450809in}}%
\pgfpathclose%
\pgfusepath{fill}%
\end{pgfscope}%
\begin{pgfscope}%
\pgfsetrectcap%
\pgfsetmiterjoin%
\pgfsetlinewidth{0.803000pt}%
\definecolor{currentstroke}{rgb}{0.501961,0.501961,0.501961}%
\pgfsetstrokecolor{currentstroke}%
\pgfsetdash{}{0pt}%
\pgfpathmoveto{\pgfqpoint{4.376725in}{2.423832in}}%
\pgfpathlineto{\pgfqpoint{4.376725in}{3.178832in}}%
\pgfusepath{stroke}%
\end{pgfscope}%
\begin{pgfscope}%
\pgfsetrectcap%
\pgfsetmiterjoin%
\pgfsetlinewidth{0.803000pt}%
\definecolor{currentstroke}{rgb}{0.501961,0.501961,0.501961}%
\pgfsetstrokecolor{currentstroke}%
\pgfsetdash{}{0pt}%
\pgfpathmoveto{\pgfqpoint{5.539225in}{2.423832in}}%
\pgfpathlineto{\pgfqpoint{5.539225in}{3.178832in}}%
\pgfusepath{stroke}%
\end{pgfscope}%
\begin{pgfscope}%
\pgfsetrectcap%
\pgfsetmiterjoin%
\pgfsetlinewidth{0.803000pt}%
\definecolor{currentstroke}{rgb}{0.501961,0.501961,0.501961}%
\pgfsetstrokecolor{currentstroke}%
\pgfsetdash{}{0pt}%
\pgfpathmoveto{\pgfqpoint{4.376725in}{2.423832in}}%
\pgfpathlineto{\pgfqpoint{5.539225in}{2.423832in}}%
\pgfusepath{stroke}%
\end{pgfscope}%
\begin{pgfscope}%
\pgfsetrectcap%
\pgfsetmiterjoin%
\pgfsetlinewidth{0.803000pt}%
\definecolor{currentstroke}{rgb}{0.501961,0.501961,0.501961}%
\pgfsetstrokecolor{currentstroke}%
\pgfsetdash{}{0pt}%
\pgfpathmoveto{\pgfqpoint{4.376725in}{3.178832in}}%
\pgfpathlineto{\pgfqpoint{5.539225in}{3.178832in}}%
\pgfusepath{stroke}%
\end{pgfscope}%
\begin{pgfscope}%
\pgfsetbuttcap%
\pgfsetmiterjoin%
\definecolor{currentfill}{rgb}{1.000000,1.000000,1.000000}%
\pgfsetfillcolor{currentfill}%
\pgfsetlinewidth{0.000000pt}%
\definecolor{currentstroke}{rgb}{0.000000,0.000000,0.000000}%
\pgfsetstrokecolor{currentstroke}%
\pgfsetstrokeopacity{0.000000}%
\pgfsetdash{}{0pt}%
\pgfpathmoveto{\pgfqpoint{0.889225in}{1.668832in}}%
\pgfpathlineto{\pgfqpoint{2.051725in}{1.668832in}}%
\pgfpathlineto{\pgfqpoint{2.051725in}{2.423832in}}%
\pgfpathlineto{\pgfqpoint{0.889225in}{2.423832in}}%
\pgfpathclose%
\pgfusepath{fill}%
\end{pgfscope}%
\begin{pgfscope}%
\pgfpathrectangle{\pgfqpoint{0.889225in}{1.668832in}}{\pgfqpoint{1.162500in}{0.755000in}}%
\pgfusepath{clip}%
\pgfsetbuttcap%
\pgfsetroundjoin%
\definecolor{currentfill}{rgb}{0.000000,0.000000,0.000000}%
\pgfsetfillcolor{currentfill}%
\pgfsetfillopacity{0.500000}%
\pgfsetlinewidth{0.000000pt}%
\definecolor{currentstroke}{rgb}{0.000000,0.000000,0.000000}%
\pgfsetstrokecolor{currentstroke}%
\pgfsetdash{}{0pt}%
\pgfpathmoveto{\pgfqpoint{1.893746in}{1.935152in}}%
\pgfpathcurveto{\pgfqpoint{1.899271in}{1.935152in}}{\pgfqpoint{1.904570in}{1.937347in}}{\pgfqpoint{1.908477in}{1.941254in}}%
\pgfpathcurveto{\pgfqpoint{1.912384in}{1.945161in}}{\pgfqpoint{1.914579in}{1.950460in}}{\pgfqpoint{1.914579in}{1.955985in}}%
\pgfpathcurveto{\pgfqpoint{1.914579in}{1.961510in}}{\pgfqpoint{1.912384in}{1.966810in}}{\pgfqpoint{1.908477in}{1.970717in}}%
\pgfpathcurveto{\pgfqpoint{1.904570in}{1.974623in}}{\pgfqpoint{1.899271in}{1.976819in}}{\pgfqpoint{1.893746in}{1.976819in}}%
\pgfpathcurveto{\pgfqpoint{1.888221in}{1.976819in}}{\pgfqpoint{1.882921in}{1.974623in}}{\pgfqpoint{1.879015in}{1.970717in}}%
\pgfpathcurveto{\pgfqpoint{1.875108in}{1.966810in}}{\pgfqpoint{1.872913in}{1.961510in}}{\pgfqpoint{1.872913in}{1.955985in}}%
\pgfpathcurveto{\pgfqpoint{1.872913in}{1.950460in}}{\pgfqpoint{1.875108in}{1.945161in}}{\pgfqpoint{1.879015in}{1.941254in}}%
\pgfpathcurveto{\pgfqpoint{1.882921in}{1.937347in}}{\pgfqpoint{1.888221in}{1.935152in}}{\pgfqpoint{1.893746in}{1.935152in}}%
\pgfpathclose%
\pgfusepath{fill}%
\end{pgfscope}%
\begin{pgfscope}%
\pgfpathrectangle{\pgfqpoint{0.889225in}{1.668832in}}{\pgfqpoint{1.162500in}{0.755000in}}%
\pgfusepath{clip}%
\pgfsetbuttcap%
\pgfsetroundjoin%
\definecolor{currentfill}{rgb}{0.000000,0.000000,0.000000}%
\pgfsetfillcolor{currentfill}%
\pgfsetfillopacity{0.500000}%
\pgfsetlinewidth{0.000000pt}%
\definecolor{currentstroke}{rgb}{0.000000,0.000000,0.000000}%
\pgfsetstrokecolor{currentstroke}%
\pgfsetdash{}{0pt}%
\pgfpathmoveto{\pgfqpoint{1.476892in}{2.326950in}}%
\pgfpathcurveto{\pgfqpoint{1.482417in}{2.326950in}}{\pgfqpoint{1.487717in}{2.329145in}}{\pgfqpoint{1.491623in}{2.333052in}}%
\pgfpathcurveto{\pgfqpoint{1.495530in}{2.336959in}}{\pgfqpoint{1.497725in}{2.342258in}}{\pgfqpoint{1.497725in}{2.347783in}}%
\pgfpathcurveto{\pgfqpoint{1.497725in}{2.353308in}}{\pgfqpoint{1.495530in}{2.358608in}}{\pgfqpoint{1.491623in}{2.362515in}}%
\pgfpathcurveto{\pgfqpoint{1.487717in}{2.366421in}}{\pgfqpoint{1.482417in}{2.368616in}}{\pgfqpoint{1.476892in}{2.368616in}}%
\pgfpathcurveto{\pgfqpoint{1.471367in}{2.368616in}}{\pgfqpoint{1.466067in}{2.366421in}}{\pgfqpoint{1.462161in}{2.362515in}}%
\pgfpathcurveto{\pgfqpoint{1.458254in}{2.358608in}}{\pgfqpoint{1.456059in}{2.353308in}}{\pgfqpoint{1.456059in}{2.347783in}}%
\pgfpathcurveto{\pgfqpoint{1.456059in}{2.342258in}}{\pgfqpoint{1.458254in}{2.336959in}}{\pgfqpoint{1.462161in}{2.333052in}}%
\pgfpathcurveto{\pgfqpoint{1.466067in}{2.329145in}}{\pgfqpoint{1.471367in}{2.326950in}}{\pgfqpoint{1.476892in}{2.326950in}}%
\pgfpathclose%
\pgfusepath{fill}%
\end{pgfscope}%
\begin{pgfscope}%
\pgfpathrectangle{\pgfqpoint{0.889225in}{1.668832in}}{\pgfqpoint{1.162500in}{0.755000in}}%
\pgfusepath{clip}%
\pgfsetbuttcap%
\pgfsetroundjoin%
\definecolor{currentfill}{rgb}{0.000000,0.000000,0.000000}%
\pgfsetfillcolor{currentfill}%
\pgfsetfillopacity{0.500000}%
\pgfsetlinewidth{0.000000pt}%
\definecolor{currentstroke}{rgb}{0.000000,0.000000,0.000000}%
\pgfsetstrokecolor{currentstroke}%
\pgfsetdash{}{0pt}%
\pgfpathmoveto{\pgfqpoint{1.478974in}{2.385023in}}%
\pgfpathcurveto{\pgfqpoint{1.484499in}{2.385023in}}{\pgfqpoint{1.489799in}{2.387218in}}{\pgfqpoint{1.493705in}{2.391125in}}%
\pgfpathcurveto{\pgfqpoint{1.497612in}{2.395032in}}{\pgfqpoint{1.499807in}{2.400331in}}{\pgfqpoint{1.499807in}{2.405856in}}%
\pgfpathcurveto{\pgfqpoint{1.499807in}{2.411381in}}{\pgfqpoint{1.497612in}{2.416681in}}{\pgfqpoint{1.493705in}{2.420588in}}%
\pgfpathcurveto{\pgfqpoint{1.489799in}{2.424494in}}{\pgfqpoint{1.484499in}{2.426690in}}{\pgfqpoint{1.478974in}{2.426690in}}%
\pgfpathcurveto{\pgfqpoint{1.473449in}{2.426690in}}{\pgfqpoint{1.468149in}{2.424494in}}{\pgfqpoint{1.464243in}{2.420588in}}%
\pgfpathcurveto{\pgfqpoint{1.460336in}{2.416681in}}{\pgfqpoint{1.458141in}{2.411381in}}{\pgfqpoint{1.458141in}{2.405856in}}%
\pgfpathcurveto{\pgfqpoint{1.458141in}{2.400331in}}{\pgfqpoint{1.460336in}{2.395032in}}{\pgfqpoint{1.464243in}{2.391125in}}%
\pgfpathcurveto{\pgfqpoint{1.468149in}{2.387218in}}{\pgfqpoint{1.473449in}{2.385023in}}{\pgfqpoint{1.478974in}{2.385023in}}%
\pgfpathclose%
\pgfusepath{fill}%
\end{pgfscope}%
\begin{pgfscope}%
\pgfpathrectangle{\pgfqpoint{0.889225in}{1.668832in}}{\pgfqpoint{1.162500in}{0.755000in}}%
\pgfusepath{clip}%
\pgfsetbuttcap%
\pgfsetroundjoin%
\definecolor{currentfill}{rgb}{0.000000,0.000000,0.000000}%
\pgfsetfillcolor{currentfill}%
\pgfsetfillopacity{0.500000}%
\pgfsetlinewidth{0.000000pt}%
\definecolor{currentstroke}{rgb}{0.000000,0.000000,0.000000}%
\pgfsetstrokecolor{currentstroke}%
\pgfsetdash{}{0pt}%
\pgfpathmoveto{\pgfqpoint{1.120714in}{2.103661in}}%
\pgfpathcurveto{\pgfqpoint{1.126239in}{2.103661in}}{\pgfqpoint{1.131538in}{2.105856in}}{\pgfqpoint{1.135445in}{2.109762in}}%
\pgfpathcurveto{\pgfqpoint{1.139352in}{2.113669in}}{\pgfqpoint{1.141547in}{2.118969in}}{\pgfqpoint{1.141547in}{2.124494in}}%
\pgfpathcurveto{\pgfqpoint{1.141547in}{2.130019in}}{\pgfqpoint{1.139352in}{2.135318in}}{\pgfqpoint{1.135445in}{2.139225in}}%
\pgfpathcurveto{\pgfqpoint{1.131538in}{2.143132in}}{\pgfqpoint{1.126239in}{2.145327in}}{\pgfqpoint{1.120714in}{2.145327in}}%
\pgfpathcurveto{\pgfqpoint{1.115189in}{2.145327in}}{\pgfqpoint{1.109889in}{2.143132in}}{\pgfqpoint{1.105982in}{2.139225in}}%
\pgfpathcurveto{\pgfqpoint{1.102076in}{2.135318in}}{\pgfqpoint{1.099881in}{2.130019in}}{\pgfqpoint{1.099881in}{2.124494in}}%
\pgfpathcurveto{\pgfqpoint{1.099881in}{2.118969in}}{\pgfqpoint{1.102076in}{2.113669in}}{\pgfqpoint{1.105982in}{2.109762in}}%
\pgfpathcurveto{\pgfqpoint{1.109889in}{2.105856in}}{\pgfqpoint{1.115189in}{2.103661in}}{\pgfqpoint{1.120714in}{2.103661in}}%
\pgfpathclose%
\pgfusepath{fill}%
\end{pgfscope}%
\begin{pgfscope}%
\pgfpathrectangle{\pgfqpoint{0.889225in}{1.668832in}}{\pgfqpoint{1.162500in}{0.755000in}}%
\pgfusepath{clip}%
\pgfsetbuttcap%
\pgfsetroundjoin%
\definecolor{currentfill}{rgb}{0.000000,0.000000,0.000000}%
\pgfsetfillcolor{currentfill}%
\pgfsetfillopacity{0.500000}%
\pgfsetlinewidth{0.000000pt}%
\definecolor{currentstroke}{rgb}{0.000000,0.000000,0.000000}%
\pgfsetstrokecolor{currentstroke}%
\pgfsetdash{}{0pt}%
\pgfpathmoveto{\pgfqpoint{1.126248in}{2.149550in}}%
\pgfpathcurveto{\pgfqpoint{1.131773in}{2.149550in}}{\pgfqpoint{1.137073in}{2.151746in}}{\pgfqpoint{1.140980in}{2.155652in}}%
\pgfpathcurveto{\pgfqpoint{1.144886in}{2.159559in}}{\pgfqpoint{1.147082in}{2.164859in}}{\pgfqpoint{1.147082in}{2.170384in}}%
\pgfpathcurveto{\pgfqpoint{1.147082in}{2.175909in}}{\pgfqpoint{1.144886in}{2.181208in}}{\pgfqpoint{1.140980in}{2.185115in}}%
\pgfpathcurveto{\pgfqpoint{1.137073in}{2.189022in}}{\pgfqpoint{1.131773in}{2.191217in}}{\pgfqpoint{1.126248in}{2.191217in}}%
\pgfpathcurveto{\pgfqpoint{1.120723in}{2.191217in}}{\pgfqpoint{1.115424in}{2.189022in}}{\pgfqpoint{1.111517in}{2.185115in}}%
\pgfpathcurveto{\pgfqpoint{1.107610in}{2.181208in}}{\pgfqpoint{1.105415in}{2.175909in}}{\pgfqpoint{1.105415in}{2.170384in}}%
\pgfpathcurveto{\pgfqpoint{1.105415in}{2.164859in}}{\pgfqpoint{1.107610in}{2.159559in}}{\pgfqpoint{1.111517in}{2.155652in}}%
\pgfpathcurveto{\pgfqpoint{1.115424in}{2.151746in}}{\pgfqpoint{1.120723in}{2.149550in}}{\pgfqpoint{1.126248in}{2.149550in}}%
\pgfpathclose%
\pgfusepath{fill}%
\end{pgfscope}%
\begin{pgfscope}%
\pgfpathrectangle{\pgfqpoint{0.889225in}{1.668832in}}{\pgfqpoint{1.162500in}{0.755000in}}%
\pgfusepath{clip}%
\pgfsetbuttcap%
\pgfsetroundjoin%
\definecolor{currentfill}{rgb}{0.000000,0.000000,0.000000}%
\pgfsetfillcolor{currentfill}%
\pgfsetfillopacity{0.500000}%
\pgfsetlinewidth{0.000000pt}%
\definecolor{currentstroke}{rgb}{0.000000,0.000000,0.000000}%
\pgfsetstrokecolor{currentstroke}%
\pgfsetdash{}{0pt}%
\pgfpathmoveto{\pgfqpoint{0.977704in}{2.049204in}}%
\pgfpathcurveto{\pgfqpoint{0.983229in}{2.049204in}}{\pgfqpoint{0.988528in}{2.051399in}}{\pgfqpoint{0.992435in}{2.055306in}}%
\pgfpathcurveto{\pgfqpoint{0.996342in}{2.059212in}}{\pgfqpoint{0.998537in}{2.064512in}}{\pgfqpoint{0.998537in}{2.070037in}}%
\pgfpathcurveto{\pgfqpoint{0.998537in}{2.075562in}}{\pgfqpoint{0.996342in}{2.080862in}}{\pgfqpoint{0.992435in}{2.084768in}}%
\pgfpathcurveto{\pgfqpoint{0.988528in}{2.088675in}}{\pgfqpoint{0.983229in}{2.090870in}}{\pgfqpoint{0.977704in}{2.090870in}}%
\pgfpathcurveto{\pgfqpoint{0.972179in}{2.090870in}}{\pgfqpoint{0.966879in}{2.088675in}}{\pgfqpoint{0.962972in}{2.084768in}}%
\pgfpathcurveto{\pgfqpoint{0.959066in}{2.080862in}}{\pgfqpoint{0.956870in}{2.075562in}}{\pgfqpoint{0.956870in}{2.070037in}}%
\pgfpathcurveto{\pgfqpoint{0.956870in}{2.064512in}}{\pgfqpoint{0.959066in}{2.059212in}}{\pgfqpoint{0.962972in}{2.055306in}}%
\pgfpathcurveto{\pgfqpoint{0.966879in}{2.051399in}}{\pgfqpoint{0.972179in}{2.049204in}}{\pgfqpoint{0.977704in}{2.049204in}}%
\pgfpathclose%
\pgfusepath{fill}%
\end{pgfscope}%
\begin{pgfscope}%
\pgfpathrectangle{\pgfqpoint{0.889225in}{1.668832in}}{\pgfqpoint{1.162500in}{0.755000in}}%
\pgfusepath{clip}%
\pgfsetbuttcap%
\pgfsetroundjoin%
\definecolor{currentfill}{rgb}{0.000000,0.000000,0.000000}%
\pgfsetfillcolor{currentfill}%
\pgfsetfillopacity{0.500000}%
\pgfsetlinewidth{0.000000pt}%
\definecolor{currentstroke}{rgb}{0.000000,0.000000,0.000000}%
\pgfsetstrokecolor{currentstroke}%
\pgfsetdash{}{0pt}%
\pgfpathmoveto{\pgfqpoint{0.916904in}{1.665975in}}%
\pgfpathcurveto{\pgfqpoint{0.922429in}{1.665975in}}{\pgfqpoint{0.927728in}{1.668170in}}{\pgfqpoint{0.931635in}{1.672077in}}%
\pgfpathcurveto{\pgfqpoint{0.935542in}{1.675984in}}{\pgfqpoint{0.937737in}{1.681283in}}{\pgfqpoint{0.937737in}{1.686809in}}%
\pgfpathcurveto{\pgfqpoint{0.937737in}{1.692334in}}{\pgfqpoint{0.935542in}{1.697633in}}{\pgfqpoint{0.931635in}{1.701540in}}%
\pgfpathcurveto{\pgfqpoint{0.927728in}{1.705447in}}{\pgfqpoint{0.922429in}{1.707642in}}{\pgfqpoint{0.916904in}{1.707642in}}%
\pgfpathcurveto{\pgfqpoint{0.911379in}{1.707642in}}{\pgfqpoint{0.906079in}{1.705447in}}{\pgfqpoint{0.902172in}{1.701540in}}%
\pgfpathcurveto{\pgfqpoint{0.898265in}{1.697633in}}{\pgfqpoint{0.896070in}{1.692334in}}{\pgfqpoint{0.896070in}{1.686809in}}%
\pgfpathcurveto{\pgfqpoint{0.896070in}{1.681283in}}{\pgfqpoint{0.898265in}{1.675984in}}{\pgfqpoint{0.902172in}{1.672077in}}%
\pgfpathcurveto{\pgfqpoint{0.906079in}{1.668170in}}{\pgfqpoint{0.911379in}{1.665975in}}{\pgfqpoint{0.916904in}{1.665975in}}%
\pgfpathclose%
\pgfusepath{fill}%
\end{pgfscope}%
\begin{pgfscope}%
\pgfpathrectangle{\pgfqpoint{0.889225in}{1.668832in}}{\pgfqpoint{1.162500in}{0.755000in}}%
\pgfusepath{clip}%
\pgfsetbuttcap%
\pgfsetroundjoin%
\definecolor{currentfill}{rgb}{0.000000,0.000000,0.000000}%
\pgfsetfillcolor{currentfill}%
\pgfsetfillopacity{0.500000}%
\pgfsetlinewidth{0.000000pt}%
\definecolor{currentstroke}{rgb}{0.000000,0.000000,0.000000}%
\pgfsetstrokecolor{currentstroke}%
\pgfsetdash{}{0pt}%
\pgfpathmoveto{\pgfqpoint{1.691400in}{2.301682in}}%
\pgfpathcurveto{\pgfqpoint{1.696925in}{2.301682in}}{\pgfqpoint{1.702225in}{2.303877in}}{\pgfqpoint{1.706132in}{2.307784in}}%
\pgfpathcurveto{\pgfqpoint{1.710038in}{2.311691in}}{\pgfqpoint{1.712234in}{2.316990in}}{\pgfqpoint{1.712234in}{2.322515in}}%
\pgfpathcurveto{\pgfqpoint{1.712234in}{2.328041in}}{\pgfqpoint{1.710038in}{2.333340in}}{\pgfqpoint{1.706132in}{2.337247in}}%
\pgfpathcurveto{\pgfqpoint{1.702225in}{2.341154in}}{\pgfqpoint{1.696925in}{2.343349in}}{\pgfqpoint{1.691400in}{2.343349in}}%
\pgfpathcurveto{\pgfqpoint{1.685875in}{2.343349in}}{\pgfqpoint{1.680576in}{2.341154in}}{\pgfqpoint{1.676669in}{2.337247in}}%
\pgfpathcurveto{\pgfqpoint{1.672762in}{2.333340in}}{\pgfqpoint{1.670567in}{2.328041in}}{\pgfqpoint{1.670567in}{2.322515in}}%
\pgfpathcurveto{\pgfqpoint{1.670567in}{2.316990in}}{\pgfqpoint{1.672762in}{2.311691in}}{\pgfqpoint{1.676669in}{2.307784in}}%
\pgfpathcurveto{\pgfqpoint{1.680576in}{2.303877in}}{\pgfqpoint{1.685875in}{2.301682in}}{\pgfqpoint{1.691400in}{2.301682in}}%
\pgfpathclose%
\pgfusepath{fill}%
\end{pgfscope}%
\begin{pgfscope}%
\pgfpathrectangle{\pgfqpoint{0.889225in}{1.668832in}}{\pgfqpoint{1.162500in}{0.755000in}}%
\pgfusepath{clip}%
\pgfsetbuttcap%
\pgfsetroundjoin%
\definecolor{currentfill}{rgb}{0.000000,0.000000,0.000000}%
\pgfsetfillcolor{currentfill}%
\pgfsetfillopacity{0.500000}%
\pgfsetlinewidth{0.000000pt}%
\definecolor{currentstroke}{rgb}{0.000000,0.000000,0.000000}%
\pgfsetstrokecolor{currentstroke}%
\pgfsetdash{}{0pt}%
\pgfpathmoveto{\pgfqpoint{1.460463in}{2.145886in}}%
\pgfpathcurveto{\pgfqpoint{1.465988in}{2.145886in}}{\pgfqpoint{1.471288in}{2.148081in}}{\pgfqpoint{1.475194in}{2.151988in}}%
\pgfpathcurveto{\pgfqpoint{1.479101in}{2.155894in}}{\pgfqpoint{1.481296in}{2.161194in}}{\pgfqpoint{1.481296in}{2.166719in}}%
\pgfpathcurveto{\pgfqpoint{1.481296in}{2.172244in}}{\pgfqpoint{1.479101in}{2.177544in}}{\pgfqpoint{1.475194in}{2.181450in}}%
\pgfpathcurveto{\pgfqpoint{1.471288in}{2.185357in}}{\pgfqpoint{1.465988in}{2.187552in}}{\pgfqpoint{1.460463in}{2.187552in}}%
\pgfpathcurveto{\pgfqpoint{1.454938in}{2.187552in}}{\pgfqpoint{1.449638in}{2.185357in}}{\pgfqpoint{1.445732in}{2.181450in}}%
\pgfpathcurveto{\pgfqpoint{1.441825in}{2.177544in}}{\pgfqpoint{1.439630in}{2.172244in}}{\pgfqpoint{1.439630in}{2.166719in}}%
\pgfpathcurveto{\pgfqpoint{1.439630in}{2.161194in}}{\pgfqpoint{1.441825in}{2.155894in}}{\pgfqpoint{1.445732in}{2.151988in}}%
\pgfpathcurveto{\pgfqpoint{1.449638in}{2.148081in}}{\pgfqpoint{1.454938in}{2.145886in}}{\pgfqpoint{1.460463in}{2.145886in}}%
\pgfpathclose%
\pgfusepath{fill}%
\end{pgfscope}%
\begin{pgfscope}%
\pgfpathrectangle{\pgfqpoint{0.889225in}{1.668832in}}{\pgfqpoint{1.162500in}{0.755000in}}%
\pgfusepath{clip}%
\pgfsetbuttcap%
\pgfsetroundjoin%
\definecolor{currentfill}{rgb}{0.000000,0.000000,0.000000}%
\pgfsetfillcolor{currentfill}%
\pgfsetfillopacity{0.500000}%
\pgfsetlinewidth{0.000000pt}%
\definecolor{currentstroke}{rgb}{0.000000,0.000000,0.000000}%
\pgfsetstrokecolor{currentstroke}%
\pgfsetdash{}{0pt}%
\pgfpathmoveto{\pgfqpoint{1.303193in}{1.969105in}}%
\pgfpathcurveto{\pgfqpoint{1.308718in}{1.969105in}}{\pgfqpoint{1.314017in}{1.971300in}}{\pgfqpoint{1.317924in}{1.975207in}}%
\pgfpathcurveto{\pgfqpoint{1.321831in}{1.979114in}}{\pgfqpoint{1.324026in}{1.984413in}}{\pgfqpoint{1.324026in}{1.989939in}}%
\pgfpathcurveto{\pgfqpoint{1.324026in}{1.995464in}}{\pgfqpoint{1.321831in}{2.000763in}}{\pgfqpoint{1.317924in}{2.004670in}}%
\pgfpathcurveto{\pgfqpoint{1.314017in}{2.008577in}}{\pgfqpoint{1.308718in}{2.010772in}}{\pgfqpoint{1.303193in}{2.010772in}}%
\pgfpathcurveto{\pgfqpoint{1.297667in}{2.010772in}}{\pgfqpoint{1.292368in}{2.008577in}}{\pgfqpoint{1.288461in}{2.004670in}}%
\pgfpathcurveto{\pgfqpoint{1.284554in}{2.000763in}}{\pgfqpoint{1.282359in}{1.995464in}}{\pgfqpoint{1.282359in}{1.989939in}}%
\pgfpathcurveto{\pgfqpoint{1.282359in}{1.984413in}}{\pgfqpoint{1.284554in}{1.979114in}}{\pgfqpoint{1.288461in}{1.975207in}}%
\pgfpathcurveto{\pgfqpoint{1.292368in}{1.971300in}}{\pgfqpoint{1.297667in}{1.969105in}}{\pgfqpoint{1.303193in}{1.969105in}}%
\pgfpathclose%
\pgfusepath{fill}%
\end{pgfscope}%
\begin{pgfscope}%
\pgfpathrectangle{\pgfqpoint{0.889225in}{1.668832in}}{\pgfqpoint{1.162500in}{0.755000in}}%
\pgfusepath{clip}%
\pgfsetbuttcap%
\pgfsetroundjoin%
\definecolor{currentfill}{rgb}{0.000000,0.000000,0.000000}%
\pgfsetfillcolor{currentfill}%
\pgfsetfillopacity{0.500000}%
\pgfsetlinewidth{0.000000pt}%
\definecolor{currentstroke}{rgb}{0.000000,0.000000,0.000000}%
\pgfsetstrokecolor{currentstroke}%
\pgfsetdash{}{0pt}%
\pgfpathmoveto{\pgfqpoint{1.689350in}{2.239104in}}%
\pgfpathcurveto{\pgfqpoint{1.694875in}{2.239104in}}{\pgfqpoint{1.700175in}{2.241300in}}{\pgfqpoint{1.704082in}{2.245206in}}%
\pgfpathcurveto{\pgfqpoint{1.707989in}{2.249113in}}{\pgfqpoint{1.710184in}{2.254413in}}{\pgfqpoint{1.710184in}{2.259938in}}%
\pgfpathcurveto{\pgfqpoint{1.710184in}{2.265463in}}{\pgfqpoint{1.707989in}{2.270762in}}{\pgfqpoint{1.704082in}{2.274669in}}%
\pgfpathcurveto{\pgfqpoint{1.700175in}{2.278576in}}{\pgfqpoint{1.694875in}{2.280771in}}{\pgfqpoint{1.689350in}{2.280771in}}%
\pgfpathcurveto{\pgfqpoint{1.683825in}{2.280771in}}{\pgfqpoint{1.678526in}{2.278576in}}{\pgfqpoint{1.674619in}{2.274669in}}%
\pgfpathcurveto{\pgfqpoint{1.670712in}{2.270762in}}{\pgfqpoint{1.668517in}{2.265463in}}{\pgfqpoint{1.668517in}{2.259938in}}%
\pgfpathcurveto{\pgfqpoint{1.668517in}{2.254413in}}{\pgfqpoint{1.670712in}{2.249113in}}{\pgfqpoint{1.674619in}{2.245206in}}%
\pgfpathcurveto{\pgfqpoint{1.678526in}{2.241300in}}{\pgfqpoint{1.683825in}{2.239104in}}{\pgfqpoint{1.689350in}{2.239104in}}%
\pgfpathclose%
\pgfusepath{fill}%
\end{pgfscope}%
\begin{pgfscope}%
\pgfpathrectangle{\pgfqpoint{0.889225in}{1.668832in}}{\pgfqpoint{1.162500in}{0.755000in}}%
\pgfusepath{clip}%
\pgfsetbuttcap%
\pgfsetroundjoin%
\definecolor{currentfill}{rgb}{0.000000,0.000000,0.000000}%
\pgfsetfillcolor{currentfill}%
\pgfsetfillopacity{0.500000}%
\pgfsetlinewidth{0.000000pt}%
\definecolor{currentstroke}{rgb}{0.000000,0.000000,0.000000}%
\pgfsetstrokecolor{currentstroke}%
\pgfsetdash{}{0pt}%
\pgfpathmoveto{\pgfqpoint{1.102396in}{1.960666in}}%
\pgfpathcurveto{\pgfqpoint{1.107921in}{1.960666in}}{\pgfqpoint{1.113221in}{1.962861in}}{\pgfqpoint{1.117128in}{1.966768in}}%
\pgfpathcurveto{\pgfqpoint{1.121035in}{1.970675in}}{\pgfqpoint{1.123230in}{1.975974in}}{\pgfqpoint{1.123230in}{1.981499in}}%
\pgfpathcurveto{\pgfqpoint{1.123230in}{1.987024in}}{\pgfqpoint{1.121035in}{1.992324in}}{\pgfqpoint{1.117128in}{1.996231in}}%
\pgfpathcurveto{\pgfqpoint{1.113221in}{2.000138in}}{\pgfqpoint{1.107921in}{2.002333in}}{\pgfqpoint{1.102396in}{2.002333in}}%
\pgfpathcurveto{\pgfqpoint{1.096871in}{2.002333in}}{\pgfqpoint{1.091572in}{2.000138in}}{\pgfqpoint{1.087665in}{1.996231in}}%
\pgfpathcurveto{\pgfqpoint{1.083758in}{1.992324in}}{\pgfqpoint{1.081563in}{1.987024in}}{\pgfqpoint{1.081563in}{1.981499in}}%
\pgfpathcurveto{\pgfqpoint{1.081563in}{1.975974in}}{\pgfqpoint{1.083758in}{1.970675in}}{\pgfqpoint{1.087665in}{1.966768in}}%
\pgfpathcurveto{\pgfqpoint{1.091572in}{1.962861in}}{\pgfqpoint{1.096871in}{1.960666in}}{\pgfqpoint{1.102396in}{1.960666in}}%
\pgfpathclose%
\pgfusepath{fill}%
\end{pgfscope}%
\begin{pgfscope}%
\pgfpathrectangle{\pgfqpoint{0.889225in}{1.668832in}}{\pgfqpoint{1.162500in}{0.755000in}}%
\pgfusepath{clip}%
\pgfsetbuttcap%
\pgfsetroundjoin%
\definecolor{currentfill}{rgb}{0.000000,0.000000,0.000000}%
\pgfsetfillcolor{currentfill}%
\pgfsetfillopacity{0.500000}%
\pgfsetlinewidth{0.000000pt}%
\definecolor{currentstroke}{rgb}{0.000000,0.000000,0.000000}%
\pgfsetstrokecolor{currentstroke}%
\pgfsetdash{}{0pt}%
\pgfpathmoveto{\pgfqpoint{1.143186in}{1.722959in}}%
\pgfpathcurveto{\pgfqpoint{1.148711in}{1.722959in}}{\pgfqpoint{1.154011in}{1.725154in}}{\pgfqpoint{1.157918in}{1.729061in}}%
\pgfpathcurveto{\pgfqpoint{1.161825in}{1.732968in}}{\pgfqpoint{1.164020in}{1.738267in}}{\pgfqpoint{1.164020in}{1.743792in}}%
\pgfpathcurveto{\pgfqpoint{1.164020in}{1.749317in}}{\pgfqpoint{1.161825in}{1.754617in}}{\pgfqpoint{1.157918in}{1.758523in}}%
\pgfpathcurveto{\pgfqpoint{1.154011in}{1.762430in}}{\pgfqpoint{1.148711in}{1.764625in}}{\pgfqpoint{1.143186in}{1.764625in}}%
\pgfpathcurveto{\pgfqpoint{1.137661in}{1.764625in}}{\pgfqpoint{1.132362in}{1.762430in}}{\pgfqpoint{1.128455in}{1.758523in}}%
\pgfpathcurveto{\pgfqpoint{1.124548in}{1.754617in}}{\pgfqpoint{1.122353in}{1.749317in}}{\pgfqpoint{1.122353in}{1.743792in}}%
\pgfpathcurveto{\pgfqpoint{1.122353in}{1.738267in}}{\pgfqpoint{1.124548in}{1.732968in}}{\pgfqpoint{1.128455in}{1.729061in}}%
\pgfpathcurveto{\pgfqpoint{1.132362in}{1.725154in}}{\pgfqpoint{1.137661in}{1.722959in}}{\pgfqpoint{1.143186in}{1.722959in}}%
\pgfpathclose%
\pgfusepath{fill}%
\end{pgfscope}%
\begin{pgfscope}%
\pgfpathrectangle{\pgfqpoint{0.889225in}{1.668832in}}{\pgfqpoint{1.162500in}{0.755000in}}%
\pgfusepath{clip}%
\pgfsetbuttcap%
\pgfsetroundjoin%
\definecolor{currentfill}{rgb}{0.000000,0.000000,0.000000}%
\pgfsetfillcolor{currentfill}%
\pgfsetfillopacity{0.500000}%
\pgfsetlinewidth{0.000000pt}%
\definecolor{currentstroke}{rgb}{0.000000,0.000000,0.000000}%
\pgfsetstrokecolor{currentstroke}%
\pgfsetdash{}{0pt}%
\pgfpathmoveto{\pgfqpoint{2.024046in}{2.267350in}}%
\pgfpathcurveto{\pgfqpoint{2.029571in}{2.267350in}}{\pgfqpoint{2.034871in}{2.269545in}}{\pgfqpoint{2.038778in}{2.273452in}}%
\pgfpathcurveto{\pgfqpoint{2.042685in}{2.277359in}}{\pgfqpoint{2.044880in}{2.282659in}}{\pgfqpoint{2.044880in}{2.288184in}}%
\pgfpathcurveto{\pgfqpoint{2.044880in}{2.293709in}}{\pgfqpoint{2.042685in}{2.299008in}}{\pgfqpoint{2.038778in}{2.302915in}}%
\pgfpathcurveto{\pgfqpoint{2.034871in}{2.306822in}}{\pgfqpoint{2.029571in}{2.309017in}}{\pgfqpoint{2.024046in}{2.309017in}}%
\pgfpathcurveto{\pgfqpoint{2.018521in}{2.309017in}}{\pgfqpoint{2.013222in}{2.306822in}}{\pgfqpoint{2.009315in}{2.302915in}}%
\pgfpathcurveto{\pgfqpoint{2.005408in}{2.299008in}}{\pgfqpoint{2.003213in}{2.293709in}}{\pgfqpoint{2.003213in}{2.288184in}}%
\pgfpathcurveto{\pgfqpoint{2.003213in}{2.282659in}}{\pgfqpoint{2.005408in}{2.277359in}}{\pgfqpoint{2.009315in}{2.273452in}}%
\pgfpathcurveto{\pgfqpoint{2.013222in}{2.269545in}}{\pgfqpoint{2.018521in}{2.267350in}}{\pgfqpoint{2.024046in}{2.267350in}}%
\pgfpathclose%
\pgfusepath{fill}%
\end{pgfscope}%
\begin{pgfscope}%
\pgfpathrectangle{\pgfqpoint{0.889225in}{1.668832in}}{\pgfqpoint{1.162500in}{0.755000in}}%
\pgfusepath{clip}%
\pgfsetbuttcap%
\pgfsetroundjoin%
\definecolor{currentfill}{rgb}{0.000000,0.000000,0.000000}%
\pgfsetfillcolor{currentfill}%
\pgfsetfillopacity{0.500000}%
\pgfsetlinewidth{0.000000pt}%
\definecolor{currentstroke}{rgb}{0.000000,0.000000,0.000000}%
\pgfsetstrokecolor{currentstroke}%
\pgfsetdash{}{0pt}%
\pgfpathmoveto{\pgfqpoint{1.402285in}{2.236796in}}%
\pgfpathcurveto{\pgfqpoint{1.407810in}{2.236796in}}{\pgfqpoint{1.413110in}{2.238991in}}{\pgfqpoint{1.417016in}{2.242898in}}%
\pgfpathcurveto{\pgfqpoint{1.420923in}{2.246805in}}{\pgfqpoint{1.423118in}{2.252104in}}{\pgfqpoint{1.423118in}{2.257629in}}%
\pgfpathcurveto{\pgfqpoint{1.423118in}{2.263154in}}{\pgfqpoint{1.420923in}{2.268454in}}{\pgfqpoint{1.417016in}{2.272361in}}%
\pgfpathcurveto{\pgfqpoint{1.413110in}{2.276267in}}{\pgfqpoint{1.407810in}{2.278463in}}{\pgfqpoint{1.402285in}{2.278463in}}%
\pgfpathcurveto{\pgfqpoint{1.396760in}{2.278463in}}{\pgfqpoint{1.391460in}{2.276267in}}{\pgfqpoint{1.387554in}{2.272361in}}%
\pgfpathcurveto{\pgfqpoint{1.383647in}{2.268454in}}{\pgfqpoint{1.381452in}{2.263154in}}{\pgfqpoint{1.381452in}{2.257629in}}%
\pgfpathcurveto{\pgfqpoint{1.381452in}{2.252104in}}{\pgfqpoint{1.383647in}{2.246805in}}{\pgfqpoint{1.387554in}{2.242898in}}%
\pgfpathcurveto{\pgfqpoint{1.391460in}{2.238991in}}{\pgfqpoint{1.396760in}{2.236796in}}{\pgfqpoint{1.402285in}{2.236796in}}%
\pgfpathclose%
\pgfusepath{fill}%
\end{pgfscope}%
\begin{pgfscope}%
\pgfpathrectangle{\pgfqpoint{0.889225in}{1.668832in}}{\pgfqpoint{1.162500in}{0.755000in}}%
\pgfusepath{clip}%
\pgfsetbuttcap%
\pgfsetroundjoin%
\definecolor{currentfill}{rgb}{0.000000,0.000000,0.000000}%
\pgfsetfillcolor{currentfill}%
\pgfsetfillopacity{0.500000}%
\pgfsetlinewidth{0.000000pt}%
\definecolor{currentstroke}{rgb}{0.000000,0.000000,0.000000}%
\pgfsetstrokecolor{currentstroke}%
\pgfsetdash{}{0pt}%
\pgfpathmoveto{\pgfqpoint{0.973483in}{2.023720in}}%
\pgfpathcurveto{\pgfqpoint{0.979008in}{2.023720in}}{\pgfqpoint{0.984308in}{2.025915in}}{\pgfqpoint{0.988214in}{2.029822in}}%
\pgfpathcurveto{\pgfqpoint{0.992121in}{2.033729in}}{\pgfqpoint{0.994316in}{2.039028in}}{\pgfqpoint{0.994316in}{2.044553in}}%
\pgfpathcurveto{\pgfqpoint{0.994316in}{2.050078in}}{\pgfqpoint{0.992121in}{2.055378in}}{\pgfqpoint{0.988214in}{2.059285in}}%
\pgfpathcurveto{\pgfqpoint{0.984308in}{2.063191in}}{\pgfqpoint{0.979008in}{2.065387in}}{\pgfqpoint{0.973483in}{2.065387in}}%
\pgfpathcurveto{\pgfqpoint{0.967958in}{2.065387in}}{\pgfqpoint{0.962658in}{2.063191in}}{\pgfqpoint{0.958752in}{2.059285in}}%
\pgfpathcurveto{\pgfqpoint{0.954845in}{2.055378in}}{\pgfqpoint{0.952650in}{2.050078in}}{\pgfqpoint{0.952650in}{2.044553in}}%
\pgfpathcurveto{\pgfqpoint{0.952650in}{2.039028in}}{\pgfqpoint{0.954845in}{2.033729in}}{\pgfqpoint{0.958752in}{2.029822in}}%
\pgfpathcurveto{\pgfqpoint{0.962658in}{2.025915in}}{\pgfqpoint{0.967958in}{2.023720in}}{\pgfqpoint{0.973483in}{2.023720in}}%
\pgfpathclose%
\pgfusepath{fill}%
\end{pgfscope}%
\begin{pgfscope}%
\pgfsetbuttcap%
\pgfsetroundjoin%
\definecolor{currentfill}{rgb}{0.000000,0.000000,0.000000}%
\pgfsetfillcolor{currentfill}%
\pgfsetlinewidth{0.803000pt}%
\definecolor{currentstroke}{rgb}{0.000000,0.000000,0.000000}%
\pgfsetstrokecolor{currentstroke}%
\pgfsetdash{}{0pt}%
\pgfsys@defobject{currentmarker}{\pgfqpoint{-0.048611in}{0.000000in}}{\pgfqpoint{0.000000in}{0.000000in}}{%
\pgfpathmoveto{\pgfqpoint{0.000000in}{0.000000in}}%
\pgfpathlineto{\pgfqpoint{-0.048611in}{0.000000in}}%
\pgfusepath{stroke,fill}%
}%
\begin{pgfscope}%
\pgfsys@transformshift{0.889225in}{1.751846in}%
\pgfsys@useobject{currentmarker}{}%
\end{pgfscope}%
\end{pgfscope}%
\begin{pgfscope}%
\pgftext[x=0.523095in,y=1.709636in,left,base]{\rmfamily\fontsize{8.000000}{9.600000}\selectfont \(\displaystyle 0.025\)}%
\end{pgfscope}%
\begin{pgfscope}%
\pgfsetbuttcap%
\pgfsetroundjoin%
\definecolor{currentfill}{rgb}{0.000000,0.000000,0.000000}%
\pgfsetfillcolor{currentfill}%
\pgfsetlinewidth{0.803000pt}%
\definecolor{currentstroke}{rgb}{0.000000,0.000000,0.000000}%
\pgfsetstrokecolor{currentstroke}%
\pgfsetdash{}{0pt}%
\pgfsys@defobject{currentmarker}{\pgfqpoint{-0.048611in}{0.000000in}}{\pgfqpoint{0.000000in}{0.000000in}}{%
\pgfpathmoveto{\pgfqpoint{0.000000in}{0.000000in}}%
\pgfpathlineto{\pgfqpoint{-0.048611in}{0.000000in}}%
\pgfusepath{stroke,fill}%
}%
\begin{pgfscope}%
\pgfsys@transformshift{0.889225in}{2.018356in}%
\pgfsys@useobject{currentmarker}{}%
\end{pgfscope}%
\end{pgfscope}%
\begin{pgfscope}%
\pgftext[x=0.523095in,y=1.976147in,left,base]{\rmfamily\fontsize{8.000000}{9.600000}\selectfont \(\displaystyle 0.050\)}%
\end{pgfscope}%
\begin{pgfscope}%
\pgfsetbuttcap%
\pgfsetroundjoin%
\definecolor{currentfill}{rgb}{0.000000,0.000000,0.000000}%
\pgfsetfillcolor{currentfill}%
\pgfsetlinewidth{0.803000pt}%
\definecolor{currentstroke}{rgb}{0.000000,0.000000,0.000000}%
\pgfsetstrokecolor{currentstroke}%
\pgfsetdash{}{0pt}%
\pgfsys@defobject{currentmarker}{\pgfqpoint{-0.048611in}{0.000000in}}{\pgfqpoint{0.000000in}{0.000000in}}{%
\pgfpathmoveto{\pgfqpoint{0.000000in}{0.000000in}}%
\pgfpathlineto{\pgfqpoint{-0.048611in}{0.000000in}}%
\pgfusepath{stroke,fill}%
}%
\begin{pgfscope}%
\pgfsys@transformshift{0.889225in}{2.284867in}%
\pgfsys@useobject{currentmarker}{}%
\end{pgfscope}%
\end{pgfscope}%
\begin{pgfscope}%
\pgftext[x=0.523095in,y=2.242658in,left,base]{\rmfamily\fontsize{8.000000}{9.600000}\selectfont \(\displaystyle 0.075\)}%
\end{pgfscope}%
\begin{pgfscope}%
\pgftext[x=0.467539in,y=2.046332in,,bottom,rotate=90.000000]{\rmfamily\fontsize{16.000000}{19.200000}\selectfont u0}%
\end{pgfscope}%
\begin{pgfscope}%
\pgfsetrectcap%
\pgfsetmiterjoin%
\pgfsetlinewidth{0.803000pt}%
\definecolor{currentstroke}{rgb}{0.501961,0.501961,0.501961}%
\pgfsetstrokecolor{currentstroke}%
\pgfsetdash{}{0pt}%
\pgfpathmoveto{\pgfqpoint{0.889225in}{1.668832in}}%
\pgfpathlineto{\pgfqpoint{0.889225in}{2.423832in}}%
\pgfusepath{stroke}%
\end{pgfscope}%
\begin{pgfscope}%
\pgfsetrectcap%
\pgfsetmiterjoin%
\pgfsetlinewidth{0.803000pt}%
\definecolor{currentstroke}{rgb}{0.501961,0.501961,0.501961}%
\pgfsetstrokecolor{currentstroke}%
\pgfsetdash{}{0pt}%
\pgfpathmoveto{\pgfqpoint{2.051725in}{1.668832in}}%
\pgfpathlineto{\pgfqpoint{2.051725in}{2.423832in}}%
\pgfusepath{stroke}%
\end{pgfscope}%
\begin{pgfscope}%
\pgfsetrectcap%
\pgfsetmiterjoin%
\pgfsetlinewidth{0.803000pt}%
\definecolor{currentstroke}{rgb}{0.501961,0.501961,0.501961}%
\pgfsetstrokecolor{currentstroke}%
\pgfsetdash{}{0pt}%
\pgfpathmoveto{\pgfqpoint{0.889225in}{1.668832in}}%
\pgfpathlineto{\pgfqpoint{2.051725in}{1.668832in}}%
\pgfusepath{stroke}%
\end{pgfscope}%
\begin{pgfscope}%
\pgfsetrectcap%
\pgfsetmiterjoin%
\pgfsetlinewidth{0.803000pt}%
\definecolor{currentstroke}{rgb}{0.501961,0.501961,0.501961}%
\pgfsetstrokecolor{currentstroke}%
\pgfsetdash{}{0pt}%
\pgfpathmoveto{\pgfqpoint{0.889225in}{2.423832in}}%
\pgfpathlineto{\pgfqpoint{2.051725in}{2.423832in}}%
\pgfusepath{stroke}%
\end{pgfscope}%
\begin{pgfscope}%
\pgfsetbuttcap%
\pgfsetmiterjoin%
\definecolor{currentfill}{rgb}{1.000000,1.000000,1.000000}%
\pgfsetfillcolor{currentfill}%
\pgfsetlinewidth{0.000000pt}%
\definecolor{currentstroke}{rgb}{0.000000,0.000000,0.000000}%
\pgfsetstrokecolor{currentstroke}%
\pgfsetstrokeopacity{0.000000}%
\pgfsetdash{}{0pt}%
\pgfpathmoveto{\pgfqpoint{2.051725in}{1.668832in}}%
\pgfpathlineto{\pgfqpoint{3.214225in}{1.668832in}}%
\pgfpathlineto{\pgfqpoint{3.214225in}{2.423832in}}%
\pgfpathlineto{\pgfqpoint{2.051725in}{2.423832in}}%
\pgfpathclose%
\pgfusepath{fill}%
\end{pgfscope}%
\begin{pgfscope}%
\pgfpathrectangle{\pgfqpoint{2.051725in}{1.668832in}}{\pgfqpoint{1.162500in}{0.755000in}}%
\pgfusepath{clip}%
\pgfsetbuttcap%
\pgfsetroundjoin%
\definecolor{currentfill}{rgb}{0.000000,0.000000,0.000000}%
\pgfsetfillcolor{currentfill}%
\pgfsetfillopacity{0.500000}%
\pgfsetlinewidth{0.000000pt}%
\definecolor{currentstroke}{rgb}{0.000000,0.000000,0.000000}%
\pgfsetstrokecolor{currentstroke}%
\pgfsetdash{}{0pt}%
\pgfpathmoveto{\pgfqpoint{3.107569in}{1.935152in}}%
\pgfpathcurveto{\pgfqpoint{3.113094in}{1.935152in}}{\pgfqpoint{3.118394in}{1.937347in}}{\pgfqpoint{3.122301in}{1.941254in}}%
\pgfpathcurveto{\pgfqpoint{3.126207in}{1.945161in}}{\pgfqpoint{3.128403in}{1.950460in}}{\pgfqpoint{3.128403in}{1.955985in}}%
\pgfpathcurveto{\pgfqpoint{3.128403in}{1.961510in}}{\pgfqpoint{3.126207in}{1.966810in}}{\pgfqpoint{3.122301in}{1.970717in}}%
\pgfpathcurveto{\pgfqpoint{3.118394in}{1.974623in}}{\pgfqpoint{3.113094in}{1.976819in}}{\pgfqpoint{3.107569in}{1.976819in}}%
\pgfpathcurveto{\pgfqpoint{3.102044in}{1.976819in}}{\pgfqpoint{3.096745in}{1.974623in}}{\pgfqpoint{3.092838in}{1.970717in}}%
\pgfpathcurveto{\pgfqpoint{3.088931in}{1.966810in}}{\pgfqpoint{3.086736in}{1.961510in}}{\pgfqpoint{3.086736in}{1.955985in}}%
\pgfpathcurveto{\pgfqpoint{3.086736in}{1.950460in}}{\pgfqpoint{3.088931in}{1.945161in}}{\pgfqpoint{3.092838in}{1.941254in}}%
\pgfpathcurveto{\pgfqpoint{3.096745in}{1.937347in}}{\pgfqpoint{3.102044in}{1.935152in}}{\pgfqpoint{3.107569in}{1.935152in}}%
\pgfpathclose%
\pgfusepath{fill}%
\end{pgfscope}%
\begin{pgfscope}%
\pgfpathrectangle{\pgfqpoint{2.051725in}{1.668832in}}{\pgfqpoint{1.162500in}{0.755000in}}%
\pgfusepath{clip}%
\pgfsetbuttcap%
\pgfsetroundjoin%
\definecolor{currentfill}{rgb}{0.000000,0.000000,0.000000}%
\pgfsetfillcolor{currentfill}%
\pgfsetfillopacity{0.500000}%
\pgfsetlinewidth{0.000000pt}%
\definecolor{currentstroke}{rgb}{0.000000,0.000000,0.000000}%
\pgfsetstrokecolor{currentstroke}%
\pgfsetdash{}{0pt}%
\pgfpathmoveto{\pgfqpoint{2.270173in}{2.326950in}}%
\pgfpathcurveto{\pgfqpoint{2.275698in}{2.326950in}}{\pgfqpoint{2.280997in}{2.329145in}}{\pgfqpoint{2.284904in}{2.333052in}}%
\pgfpathcurveto{\pgfqpoint{2.288811in}{2.336959in}}{\pgfqpoint{2.291006in}{2.342258in}}{\pgfqpoint{2.291006in}{2.347783in}}%
\pgfpathcurveto{\pgfqpoint{2.291006in}{2.353308in}}{\pgfqpoint{2.288811in}{2.358608in}}{\pgfqpoint{2.284904in}{2.362515in}}%
\pgfpathcurveto{\pgfqpoint{2.280997in}{2.366421in}}{\pgfqpoint{2.275698in}{2.368616in}}{\pgfqpoint{2.270173in}{2.368616in}}%
\pgfpathcurveto{\pgfqpoint{2.264648in}{2.368616in}}{\pgfqpoint{2.259348in}{2.366421in}}{\pgfqpoint{2.255441in}{2.362515in}}%
\pgfpathcurveto{\pgfqpoint{2.251535in}{2.358608in}}{\pgfqpoint{2.249339in}{2.353308in}}{\pgfqpoint{2.249339in}{2.347783in}}%
\pgfpathcurveto{\pgfqpoint{2.249339in}{2.342258in}}{\pgfqpoint{2.251535in}{2.336959in}}{\pgfqpoint{2.255441in}{2.333052in}}%
\pgfpathcurveto{\pgfqpoint{2.259348in}{2.329145in}}{\pgfqpoint{2.264648in}{2.326950in}}{\pgfqpoint{2.270173in}{2.326950in}}%
\pgfpathclose%
\pgfusepath{fill}%
\end{pgfscope}%
\begin{pgfscope}%
\pgfpathrectangle{\pgfqpoint{2.051725in}{1.668832in}}{\pgfqpoint{1.162500in}{0.755000in}}%
\pgfusepath{clip}%
\pgfsetbuttcap%
\pgfsetroundjoin%
\definecolor{currentfill}{rgb}{0.000000,0.000000,0.000000}%
\pgfsetfillcolor{currentfill}%
\pgfsetfillopacity{0.500000}%
\pgfsetlinewidth{0.000000pt}%
\definecolor{currentstroke}{rgb}{0.000000,0.000000,0.000000}%
\pgfsetstrokecolor{currentstroke}%
\pgfsetdash{}{0pt}%
\pgfpathmoveto{\pgfqpoint{2.264867in}{2.385023in}}%
\pgfpathcurveto{\pgfqpoint{2.270392in}{2.385023in}}{\pgfqpoint{2.275692in}{2.387218in}}{\pgfqpoint{2.279599in}{2.391125in}}%
\pgfpathcurveto{\pgfqpoint{2.283506in}{2.395032in}}{\pgfqpoint{2.285701in}{2.400331in}}{\pgfqpoint{2.285701in}{2.405856in}}%
\pgfpathcurveto{\pgfqpoint{2.285701in}{2.411381in}}{\pgfqpoint{2.283506in}{2.416681in}}{\pgfqpoint{2.279599in}{2.420588in}}%
\pgfpathcurveto{\pgfqpoint{2.275692in}{2.424494in}}{\pgfqpoint{2.270392in}{2.426690in}}{\pgfqpoint{2.264867in}{2.426690in}}%
\pgfpathcurveto{\pgfqpoint{2.259342in}{2.426690in}}{\pgfqpoint{2.254043in}{2.424494in}}{\pgfqpoint{2.250136in}{2.420588in}}%
\pgfpathcurveto{\pgfqpoint{2.246229in}{2.416681in}}{\pgfqpoint{2.244034in}{2.411381in}}{\pgfqpoint{2.244034in}{2.405856in}}%
\pgfpathcurveto{\pgfqpoint{2.244034in}{2.400331in}}{\pgfqpoint{2.246229in}{2.395032in}}{\pgfqpoint{2.250136in}{2.391125in}}%
\pgfpathcurveto{\pgfqpoint{2.254043in}{2.387218in}}{\pgfqpoint{2.259342in}{2.385023in}}{\pgfqpoint{2.264867in}{2.385023in}}%
\pgfpathclose%
\pgfusepath{fill}%
\end{pgfscope}%
\begin{pgfscope}%
\pgfpathrectangle{\pgfqpoint{2.051725in}{1.668832in}}{\pgfqpoint{1.162500in}{0.755000in}}%
\pgfusepath{clip}%
\pgfsetbuttcap%
\pgfsetroundjoin%
\definecolor{currentfill}{rgb}{0.000000,0.000000,0.000000}%
\pgfsetfillcolor{currentfill}%
\pgfsetfillopacity{0.500000}%
\pgfsetlinewidth{0.000000pt}%
\definecolor{currentstroke}{rgb}{0.000000,0.000000,0.000000}%
\pgfsetstrokecolor{currentstroke}%
\pgfsetdash{}{0pt}%
\pgfpathmoveto{\pgfqpoint{2.154193in}{2.103661in}}%
\pgfpathcurveto{\pgfqpoint{2.159718in}{2.103661in}}{\pgfqpoint{2.165018in}{2.105856in}}{\pgfqpoint{2.168924in}{2.109762in}}%
\pgfpathcurveto{\pgfqpoint{2.172831in}{2.113669in}}{\pgfqpoint{2.175026in}{2.118969in}}{\pgfqpoint{2.175026in}{2.124494in}}%
\pgfpathcurveto{\pgfqpoint{2.175026in}{2.130019in}}{\pgfqpoint{2.172831in}{2.135318in}}{\pgfqpoint{2.168924in}{2.139225in}}%
\pgfpathcurveto{\pgfqpoint{2.165018in}{2.143132in}}{\pgfqpoint{2.159718in}{2.145327in}}{\pgfqpoint{2.154193in}{2.145327in}}%
\pgfpathcurveto{\pgfqpoint{2.148668in}{2.145327in}}{\pgfqpoint{2.143368in}{2.143132in}}{\pgfqpoint{2.139462in}{2.139225in}}%
\pgfpathcurveto{\pgfqpoint{2.135555in}{2.135318in}}{\pgfqpoint{2.133360in}{2.130019in}}{\pgfqpoint{2.133360in}{2.124494in}}%
\pgfpathcurveto{\pgfqpoint{2.133360in}{2.118969in}}{\pgfqpoint{2.135555in}{2.113669in}}{\pgfqpoint{2.139462in}{2.109762in}}%
\pgfpathcurveto{\pgfqpoint{2.143368in}{2.105856in}}{\pgfqpoint{2.148668in}{2.103661in}}{\pgfqpoint{2.154193in}{2.103661in}}%
\pgfpathclose%
\pgfusepath{fill}%
\end{pgfscope}%
\begin{pgfscope}%
\pgfpathrectangle{\pgfqpoint{2.051725in}{1.668832in}}{\pgfqpoint{1.162500in}{0.755000in}}%
\pgfusepath{clip}%
\pgfsetbuttcap%
\pgfsetroundjoin%
\definecolor{currentfill}{rgb}{0.000000,0.000000,0.000000}%
\pgfsetfillcolor{currentfill}%
\pgfsetfillopacity{0.500000}%
\pgfsetlinewidth{0.000000pt}%
\definecolor{currentstroke}{rgb}{0.000000,0.000000,0.000000}%
\pgfsetstrokecolor{currentstroke}%
\pgfsetdash{}{0pt}%
\pgfpathmoveto{\pgfqpoint{2.183344in}{2.149550in}}%
\pgfpathcurveto{\pgfqpoint{2.188869in}{2.149550in}}{\pgfqpoint{2.194169in}{2.151746in}}{\pgfqpoint{2.198076in}{2.155652in}}%
\pgfpathcurveto{\pgfqpoint{2.201982in}{2.159559in}}{\pgfqpoint{2.204178in}{2.164859in}}{\pgfqpoint{2.204178in}{2.170384in}}%
\pgfpathcurveto{\pgfqpoint{2.204178in}{2.175909in}}{\pgfqpoint{2.201982in}{2.181208in}}{\pgfqpoint{2.198076in}{2.185115in}}%
\pgfpathcurveto{\pgfqpoint{2.194169in}{2.189022in}}{\pgfqpoint{2.188869in}{2.191217in}}{\pgfqpoint{2.183344in}{2.191217in}}%
\pgfpathcurveto{\pgfqpoint{2.177819in}{2.191217in}}{\pgfqpoint{2.172520in}{2.189022in}}{\pgfqpoint{2.168613in}{2.185115in}}%
\pgfpathcurveto{\pgfqpoint{2.164706in}{2.181208in}}{\pgfqpoint{2.162511in}{2.175909in}}{\pgfqpoint{2.162511in}{2.170384in}}%
\pgfpathcurveto{\pgfqpoint{2.162511in}{2.164859in}}{\pgfqpoint{2.164706in}{2.159559in}}{\pgfqpoint{2.168613in}{2.155652in}}%
\pgfpathcurveto{\pgfqpoint{2.172520in}{2.151746in}}{\pgfqpoint{2.177819in}{2.149550in}}{\pgfqpoint{2.183344in}{2.149550in}}%
\pgfpathclose%
\pgfusepath{fill}%
\end{pgfscope}%
\begin{pgfscope}%
\pgfpathrectangle{\pgfqpoint{2.051725in}{1.668832in}}{\pgfqpoint{1.162500in}{0.755000in}}%
\pgfusepath{clip}%
\pgfsetbuttcap%
\pgfsetroundjoin%
\definecolor{currentfill}{rgb}{0.000000,0.000000,0.000000}%
\pgfsetfillcolor{currentfill}%
\pgfsetfillopacity{0.500000}%
\pgfsetlinewidth{0.000000pt}%
\definecolor{currentstroke}{rgb}{0.000000,0.000000,0.000000}%
\pgfsetstrokecolor{currentstroke}%
\pgfsetdash{}{0pt}%
\pgfpathmoveto{\pgfqpoint{2.082232in}{2.049204in}}%
\pgfpathcurveto{\pgfqpoint{2.087757in}{2.049204in}}{\pgfqpoint{2.093056in}{2.051399in}}{\pgfqpoint{2.096963in}{2.055306in}}%
\pgfpathcurveto{\pgfqpoint{2.100870in}{2.059212in}}{\pgfqpoint{2.103065in}{2.064512in}}{\pgfqpoint{2.103065in}{2.070037in}}%
\pgfpathcurveto{\pgfqpoint{2.103065in}{2.075562in}}{\pgfqpoint{2.100870in}{2.080862in}}{\pgfqpoint{2.096963in}{2.084768in}}%
\pgfpathcurveto{\pgfqpoint{2.093056in}{2.088675in}}{\pgfqpoint{2.087757in}{2.090870in}}{\pgfqpoint{2.082232in}{2.090870in}}%
\pgfpathcurveto{\pgfqpoint{2.076707in}{2.090870in}}{\pgfqpoint{2.071407in}{2.088675in}}{\pgfqpoint{2.067500in}{2.084768in}}%
\pgfpathcurveto{\pgfqpoint{2.063594in}{2.080862in}}{\pgfqpoint{2.061398in}{2.075562in}}{\pgfqpoint{2.061398in}{2.070037in}}%
\pgfpathcurveto{\pgfqpoint{2.061398in}{2.064512in}}{\pgfqpoint{2.063594in}{2.059212in}}{\pgfqpoint{2.067500in}{2.055306in}}%
\pgfpathcurveto{\pgfqpoint{2.071407in}{2.051399in}}{\pgfqpoint{2.076707in}{2.049204in}}{\pgfqpoint{2.082232in}{2.049204in}}%
\pgfpathclose%
\pgfusepath{fill}%
\end{pgfscope}%
\begin{pgfscope}%
\pgfpathrectangle{\pgfqpoint{2.051725in}{1.668832in}}{\pgfqpoint{1.162500in}{0.755000in}}%
\pgfusepath{clip}%
\pgfsetbuttcap%
\pgfsetroundjoin%
\definecolor{currentfill}{rgb}{0.000000,0.000000,0.000000}%
\pgfsetfillcolor{currentfill}%
\pgfsetfillopacity{0.500000}%
\pgfsetlinewidth{0.000000pt}%
\definecolor{currentstroke}{rgb}{0.000000,0.000000,0.000000}%
\pgfsetstrokecolor{currentstroke}%
\pgfsetdash{}{0pt}%
\pgfpathmoveto{\pgfqpoint{2.079404in}{1.665975in}}%
\pgfpathcurveto{\pgfqpoint{2.084929in}{1.665975in}}{\pgfqpoint{2.090228in}{1.668170in}}{\pgfqpoint{2.094135in}{1.672077in}}%
\pgfpathcurveto{\pgfqpoint{2.098042in}{1.675984in}}{\pgfqpoint{2.100237in}{1.681283in}}{\pgfqpoint{2.100237in}{1.686809in}}%
\pgfpathcurveto{\pgfqpoint{2.100237in}{1.692334in}}{\pgfqpoint{2.098042in}{1.697633in}}{\pgfqpoint{2.094135in}{1.701540in}}%
\pgfpathcurveto{\pgfqpoint{2.090228in}{1.705447in}}{\pgfqpoint{2.084929in}{1.707642in}}{\pgfqpoint{2.079404in}{1.707642in}}%
\pgfpathcurveto{\pgfqpoint{2.073879in}{1.707642in}}{\pgfqpoint{2.068579in}{1.705447in}}{\pgfqpoint{2.064672in}{1.701540in}}%
\pgfpathcurveto{\pgfqpoint{2.060765in}{1.697633in}}{\pgfqpoint{2.058570in}{1.692334in}}{\pgfqpoint{2.058570in}{1.686809in}}%
\pgfpathcurveto{\pgfqpoint{2.058570in}{1.681283in}}{\pgfqpoint{2.060765in}{1.675984in}}{\pgfqpoint{2.064672in}{1.672077in}}%
\pgfpathcurveto{\pgfqpoint{2.068579in}{1.668170in}}{\pgfqpoint{2.073879in}{1.665975in}}{\pgfqpoint{2.079404in}{1.665975in}}%
\pgfpathclose%
\pgfusepath{fill}%
\end{pgfscope}%
\begin{pgfscope}%
\pgfpathrectangle{\pgfqpoint{2.051725in}{1.668832in}}{\pgfqpoint{1.162500in}{0.755000in}}%
\pgfusepath{clip}%
\pgfsetbuttcap%
\pgfsetroundjoin%
\definecolor{currentfill}{rgb}{0.000000,0.000000,0.000000}%
\pgfsetfillcolor{currentfill}%
\pgfsetfillopacity{0.500000}%
\pgfsetlinewidth{0.000000pt}%
\definecolor{currentstroke}{rgb}{0.000000,0.000000,0.000000}%
\pgfsetstrokecolor{currentstroke}%
\pgfsetdash{}{0pt}%
\pgfpathmoveto{\pgfqpoint{2.536683in}{2.301682in}}%
\pgfpathcurveto{\pgfqpoint{2.542208in}{2.301682in}}{\pgfqpoint{2.547508in}{2.303877in}}{\pgfqpoint{2.551415in}{2.307784in}}%
\pgfpathcurveto{\pgfqpoint{2.555322in}{2.311691in}}{\pgfqpoint{2.557517in}{2.316990in}}{\pgfqpoint{2.557517in}{2.322515in}}%
\pgfpathcurveto{\pgfqpoint{2.557517in}{2.328041in}}{\pgfqpoint{2.555322in}{2.333340in}}{\pgfqpoint{2.551415in}{2.337247in}}%
\pgfpathcurveto{\pgfqpoint{2.547508in}{2.341154in}}{\pgfqpoint{2.542208in}{2.343349in}}{\pgfqpoint{2.536683in}{2.343349in}}%
\pgfpathcurveto{\pgfqpoint{2.531158in}{2.343349in}}{\pgfqpoint{2.525859in}{2.341154in}}{\pgfqpoint{2.521952in}{2.337247in}}%
\pgfpathcurveto{\pgfqpoint{2.518045in}{2.333340in}}{\pgfqpoint{2.515850in}{2.328041in}}{\pgfqpoint{2.515850in}{2.322515in}}%
\pgfpathcurveto{\pgfqpoint{2.515850in}{2.316990in}}{\pgfqpoint{2.518045in}{2.311691in}}{\pgfqpoint{2.521952in}{2.307784in}}%
\pgfpathcurveto{\pgfqpoint{2.525859in}{2.303877in}}{\pgfqpoint{2.531158in}{2.301682in}}{\pgfqpoint{2.536683in}{2.301682in}}%
\pgfpathclose%
\pgfusepath{fill}%
\end{pgfscope}%
\begin{pgfscope}%
\pgfpathrectangle{\pgfqpoint{2.051725in}{1.668832in}}{\pgfqpoint{1.162500in}{0.755000in}}%
\pgfusepath{clip}%
\pgfsetbuttcap%
\pgfsetroundjoin%
\definecolor{currentfill}{rgb}{0.000000,0.000000,0.000000}%
\pgfsetfillcolor{currentfill}%
\pgfsetfillopacity{0.500000}%
\pgfsetlinewidth{0.000000pt}%
\definecolor{currentstroke}{rgb}{0.000000,0.000000,0.000000}%
\pgfsetstrokecolor{currentstroke}%
\pgfsetdash{}{0pt}%
\pgfpathmoveto{\pgfqpoint{2.417381in}{2.145886in}}%
\pgfpathcurveto{\pgfqpoint{2.422906in}{2.145886in}}{\pgfqpoint{2.428205in}{2.148081in}}{\pgfqpoint{2.432112in}{2.151988in}}%
\pgfpathcurveto{\pgfqpoint{2.436019in}{2.155894in}}{\pgfqpoint{2.438214in}{2.161194in}}{\pgfqpoint{2.438214in}{2.166719in}}%
\pgfpathcurveto{\pgfqpoint{2.438214in}{2.172244in}}{\pgfqpoint{2.436019in}{2.177544in}}{\pgfqpoint{2.432112in}{2.181450in}}%
\pgfpathcurveto{\pgfqpoint{2.428205in}{2.185357in}}{\pgfqpoint{2.422906in}{2.187552in}}{\pgfqpoint{2.417381in}{2.187552in}}%
\pgfpathcurveto{\pgfqpoint{2.411856in}{2.187552in}}{\pgfqpoint{2.406556in}{2.185357in}}{\pgfqpoint{2.402649in}{2.181450in}}%
\pgfpathcurveto{\pgfqpoint{2.398743in}{2.177544in}}{\pgfqpoint{2.396547in}{2.172244in}}{\pgfqpoint{2.396547in}{2.166719in}}%
\pgfpathcurveto{\pgfqpoint{2.396547in}{2.161194in}}{\pgfqpoint{2.398743in}{2.155894in}}{\pgfqpoint{2.402649in}{2.151988in}}%
\pgfpathcurveto{\pgfqpoint{2.406556in}{2.148081in}}{\pgfqpoint{2.411856in}{2.145886in}}{\pgfqpoint{2.417381in}{2.145886in}}%
\pgfpathclose%
\pgfusepath{fill}%
\end{pgfscope}%
\begin{pgfscope}%
\pgfpathrectangle{\pgfqpoint{2.051725in}{1.668832in}}{\pgfqpoint{1.162500in}{0.755000in}}%
\pgfusepath{clip}%
\pgfsetbuttcap%
\pgfsetroundjoin%
\definecolor{currentfill}{rgb}{0.000000,0.000000,0.000000}%
\pgfsetfillcolor{currentfill}%
\pgfsetfillopacity{0.500000}%
\pgfsetlinewidth{0.000000pt}%
\definecolor{currentstroke}{rgb}{0.000000,0.000000,0.000000}%
\pgfsetstrokecolor{currentstroke}%
\pgfsetdash{}{0pt}%
\pgfpathmoveto{\pgfqpoint{2.412630in}{1.969105in}}%
\pgfpathcurveto{\pgfqpoint{2.418155in}{1.969105in}}{\pgfqpoint{2.423454in}{1.971300in}}{\pgfqpoint{2.427361in}{1.975207in}}%
\pgfpathcurveto{\pgfqpoint{2.431268in}{1.979114in}}{\pgfqpoint{2.433463in}{1.984413in}}{\pgfqpoint{2.433463in}{1.989939in}}%
\pgfpathcurveto{\pgfqpoint{2.433463in}{1.995464in}}{\pgfqpoint{2.431268in}{2.000763in}}{\pgfqpoint{2.427361in}{2.004670in}}%
\pgfpathcurveto{\pgfqpoint{2.423454in}{2.008577in}}{\pgfqpoint{2.418155in}{2.010772in}}{\pgfqpoint{2.412630in}{2.010772in}}%
\pgfpathcurveto{\pgfqpoint{2.407105in}{2.010772in}}{\pgfqpoint{2.401805in}{2.008577in}}{\pgfqpoint{2.397898in}{2.004670in}}%
\pgfpathcurveto{\pgfqpoint{2.393991in}{2.000763in}}{\pgfqpoint{2.391796in}{1.995464in}}{\pgfqpoint{2.391796in}{1.989939in}}%
\pgfpathcurveto{\pgfqpoint{2.391796in}{1.984413in}}{\pgfqpoint{2.393991in}{1.979114in}}{\pgfqpoint{2.397898in}{1.975207in}}%
\pgfpathcurveto{\pgfqpoint{2.401805in}{1.971300in}}{\pgfqpoint{2.407105in}{1.969105in}}{\pgfqpoint{2.412630in}{1.969105in}}%
\pgfpathclose%
\pgfusepath{fill}%
\end{pgfscope}%
\begin{pgfscope}%
\pgfpathrectangle{\pgfqpoint{2.051725in}{1.668832in}}{\pgfqpoint{1.162500in}{0.755000in}}%
\pgfusepath{clip}%
\pgfsetbuttcap%
\pgfsetroundjoin%
\definecolor{currentfill}{rgb}{0.000000,0.000000,0.000000}%
\pgfsetfillcolor{currentfill}%
\pgfsetfillopacity{0.500000}%
\pgfsetlinewidth{0.000000pt}%
\definecolor{currentstroke}{rgb}{0.000000,0.000000,0.000000}%
\pgfsetstrokecolor{currentstroke}%
\pgfsetdash{}{0pt}%
\pgfpathmoveto{\pgfqpoint{2.478188in}{2.239104in}}%
\pgfpathcurveto{\pgfqpoint{2.483713in}{2.239104in}}{\pgfqpoint{2.489012in}{2.241300in}}{\pgfqpoint{2.492919in}{2.245206in}}%
\pgfpathcurveto{\pgfqpoint{2.496826in}{2.249113in}}{\pgfqpoint{2.499021in}{2.254413in}}{\pgfqpoint{2.499021in}{2.259938in}}%
\pgfpathcurveto{\pgfqpoint{2.499021in}{2.265463in}}{\pgfqpoint{2.496826in}{2.270762in}}{\pgfqpoint{2.492919in}{2.274669in}}%
\pgfpathcurveto{\pgfqpoint{2.489012in}{2.278576in}}{\pgfqpoint{2.483713in}{2.280771in}}{\pgfqpoint{2.478188in}{2.280771in}}%
\pgfpathcurveto{\pgfqpoint{2.472663in}{2.280771in}}{\pgfqpoint{2.467363in}{2.278576in}}{\pgfqpoint{2.463456in}{2.274669in}}%
\pgfpathcurveto{\pgfqpoint{2.459549in}{2.270762in}}{\pgfqpoint{2.457354in}{2.265463in}}{\pgfqpoint{2.457354in}{2.259938in}}%
\pgfpathcurveto{\pgfqpoint{2.457354in}{2.254413in}}{\pgfqpoint{2.459549in}{2.249113in}}{\pgfqpoint{2.463456in}{2.245206in}}%
\pgfpathcurveto{\pgfqpoint{2.467363in}{2.241300in}}{\pgfqpoint{2.472663in}{2.239104in}}{\pgfqpoint{2.478188in}{2.239104in}}%
\pgfpathclose%
\pgfusepath{fill}%
\end{pgfscope}%
\begin{pgfscope}%
\pgfpathrectangle{\pgfqpoint{2.051725in}{1.668832in}}{\pgfqpoint{1.162500in}{0.755000in}}%
\pgfusepath{clip}%
\pgfsetbuttcap%
\pgfsetroundjoin%
\definecolor{currentfill}{rgb}{0.000000,0.000000,0.000000}%
\pgfsetfillcolor{currentfill}%
\pgfsetfillopacity{0.500000}%
\pgfsetlinewidth{0.000000pt}%
\definecolor{currentstroke}{rgb}{0.000000,0.000000,0.000000}%
\pgfsetstrokecolor{currentstroke}%
\pgfsetdash{}{0pt}%
\pgfpathmoveto{\pgfqpoint{2.172291in}{1.960666in}}%
\pgfpathcurveto{\pgfqpoint{2.177816in}{1.960666in}}{\pgfqpoint{2.183116in}{1.962861in}}{\pgfqpoint{2.187022in}{1.966768in}}%
\pgfpathcurveto{\pgfqpoint{2.190929in}{1.970675in}}{\pgfqpoint{2.193124in}{1.975974in}}{\pgfqpoint{2.193124in}{1.981499in}}%
\pgfpathcurveto{\pgfqpoint{2.193124in}{1.987024in}}{\pgfqpoint{2.190929in}{1.992324in}}{\pgfqpoint{2.187022in}{1.996231in}}%
\pgfpathcurveto{\pgfqpoint{2.183116in}{2.000138in}}{\pgfqpoint{2.177816in}{2.002333in}}{\pgfqpoint{2.172291in}{2.002333in}}%
\pgfpathcurveto{\pgfqpoint{2.166766in}{2.002333in}}{\pgfqpoint{2.161467in}{2.000138in}}{\pgfqpoint{2.157560in}{1.996231in}}%
\pgfpathcurveto{\pgfqpoint{2.153653in}{1.992324in}}{\pgfqpoint{2.151458in}{1.987024in}}{\pgfqpoint{2.151458in}{1.981499in}}%
\pgfpathcurveto{\pgfqpoint{2.151458in}{1.975974in}}{\pgfqpoint{2.153653in}{1.970675in}}{\pgfqpoint{2.157560in}{1.966768in}}%
\pgfpathcurveto{\pgfqpoint{2.161467in}{1.962861in}}{\pgfqpoint{2.166766in}{1.960666in}}{\pgfqpoint{2.172291in}{1.960666in}}%
\pgfpathclose%
\pgfusepath{fill}%
\end{pgfscope}%
\begin{pgfscope}%
\pgfpathrectangle{\pgfqpoint{2.051725in}{1.668832in}}{\pgfqpoint{1.162500in}{0.755000in}}%
\pgfusepath{clip}%
\pgfsetbuttcap%
\pgfsetroundjoin%
\definecolor{currentfill}{rgb}{0.000000,0.000000,0.000000}%
\pgfsetfillcolor{currentfill}%
\pgfsetfillopacity{0.500000}%
\pgfsetlinewidth{0.000000pt}%
\definecolor{currentstroke}{rgb}{0.000000,0.000000,0.000000}%
\pgfsetstrokecolor{currentstroke}%
\pgfsetdash{}{0pt}%
\pgfpathmoveto{\pgfqpoint{2.205718in}{1.722959in}}%
\pgfpathcurveto{\pgfqpoint{2.211243in}{1.722959in}}{\pgfqpoint{2.216543in}{1.725154in}}{\pgfqpoint{2.220450in}{1.729061in}}%
\pgfpathcurveto{\pgfqpoint{2.224357in}{1.732968in}}{\pgfqpoint{2.226552in}{1.738267in}}{\pgfqpoint{2.226552in}{1.743792in}}%
\pgfpathcurveto{\pgfqpoint{2.226552in}{1.749317in}}{\pgfqpoint{2.224357in}{1.754617in}}{\pgfqpoint{2.220450in}{1.758523in}}%
\pgfpathcurveto{\pgfqpoint{2.216543in}{1.762430in}}{\pgfqpoint{2.211243in}{1.764625in}}{\pgfqpoint{2.205718in}{1.764625in}}%
\pgfpathcurveto{\pgfqpoint{2.200193in}{1.764625in}}{\pgfqpoint{2.194894in}{1.762430in}}{\pgfqpoint{2.190987in}{1.758523in}}%
\pgfpathcurveto{\pgfqpoint{2.187080in}{1.754617in}}{\pgfqpoint{2.184885in}{1.749317in}}{\pgfqpoint{2.184885in}{1.743792in}}%
\pgfpathcurveto{\pgfqpoint{2.184885in}{1.738267in}}{\pgfqpoint{2.187080in}{1.732968in}}{\pgfqpoint{2.190987in}{1.729061in}}%
\pgfpathcurveto{\pgfqpoint{2.194894in}{1.725154in}}{\pgfqpoint{2.200193in}{1.722959in}}{\pgfqpoint{2.205718in}{1.722959in}}%
\pgfpathclose%
\pgfusepath{fill}%
\end{pgfscope}%
\begin{pgfscope}%
\pgfpathrectangle{\pgfqpoint{2.051725in}{1.668832in}}{\pgfqpoint{1.162500in}{0.755000in}}%
\pgfusepath{clip}%
\pgfsetbuttcap%
\pgfsetroundjoin%
\definecolor{currentfill}{rgb}{0.000000,0.000000,0.000000}%
\pgfsetfillcolor{currentfill}%
\pgfsetfillopacity{0.500000}%
\pgfsetlinewidth{0.000000pt}%
\definecolor{currentstroke}{rgb}{0.000000,0.000000,0.000000}%
\pgfsetstrokecolor{currentstroke}%
\pgfsetdash{}{0pt}%
\pgfpathmoveto{\pgfqpoint{3.186546in}{2.267350in}}%
\pgfpathcurveto{\pgfqpoint{3.192071in}{2.267350in}}{\pgfqpoint{3.197371in}{2.269545in}}{\pgfqpoint{3.201278in}{2.273452in}}%
\pgfpathcurveto{\pgfqpoint{3.205185in}{2.277359in}}{\pgfqpoint{3.207380in}{2.282659in}}{\pgfqpoint{3.207380in}{2.288184in}}%
\pgfpathcurveto{\pgfqpoint{3.207380in}{2.293709in}}{\pgfqpoint{3.205185in}{2.299008in}}{\pgfqpoint{3.201278in}{2.302915in}}%
\pgfpathcurveto{\pgfqpoint{3.197371in}{2.306822in}}{\pgfqpoint{3.192071in}{2.309017in}}{\pgfqpoint{3.186546in}{2.309017in}}%
\pgfpathcurveto{\pgfqpoint{3.181021in}{2.309017in}}{\pgfqpoint{3.175722in}{2.306822in}}{\pgfqpoint{3.171815in}{2.302915in}}%
\pgfpathcurveto{\pgfqpoint{3.167908in}{2.299008in}}{\pgfqpoint{3.165713in}{2.293709in}}{\pgfqpoint{3.165713in}{2.288184in}}%
\pgfpathcurveto{\pgfqpoint{3.165713in}{2.282659in}}{\pgfqpoint{3.167908in}{2.277359in}}{\pgfqpoint{3.171815in}{2.273452in}}%
\pgfpathcurveto{\pgfqpoint{3.175722in}{2.269545in}}{\pgfqpoint{3.181021in}{2.267350in}}{\pgfqpoint{3.186546in}{2.267350in}}%
\pgfpathclose%
\pgfusepath{fill}%
\end{pgfscope}%
\begin{pgfscope}%
\pgfpathrectangle{\pgfqpoint{2.051725in}{1.668832in}}{\pgfqpoint{1.162500in}{0.755000in}}%
\pgfusepath{clip}%
\pgfsetbuttcap%
\pgfsetroundjoin%
\definecolor{currentfill}{rgb}{0.000000,0.000000,0.000000}%
\pgfsetfillcolor{currentfill}%
\pgfsetfillopacity{0.500000}%
\pgfsetlinewidth{0.000000pt}%
\definecolor{currentstroke}{rgb}{0.000000,0.000000,0.000000}%
\pgfsetstrokecolor{currentstroke}%
\pgfsetdash{}{0pt}%
\pgfpathmoveto{\pgfqpoint{2.394110in}{2.236796in}}%
\pgfpathcurveto{\pgfqpoint{2.399635in}{2.236796in}}{\pgfqpoint{2.404935in}{2.238991in}}{\pgfqpoint{2.408842in}{2.242898in}}%
\pgfpathcurveto{\pgfqpoint{2.412749in}{2.246805in}}{\pgfqpoint{2.414944in}{2.252104in}}{\pgfqpoint{2.414944in}{2.257629in}}%
\pgfpathcurveto{\pgfqpoint{2.414944in}{2.263154in}}{\pgfqpoint{2.412749in}{2.268454in}}{\pgfqpoint{2.408842in}{2.272361in}}%
\pgfpathcurveto{\pgfqpoint{2.404935in}{2.276267in}}{\pgfqpoint{2.399635in}{2.278463in}}{\pgfqpoint{2.394110in}{2.278463in}}%
\pgfpathcurveto{\pgfqpoint{2.388585in}{2.278463in}}{\pgfqpoint{2.383286in}{2.276267in}}{\pgfqpoint{2.379379in}{2.272361in}}%
\pgfpathcurveto{\pgfqpoint{2.375472in}{2.268454in}}{\pgfqpoint{2.373277in}{2.263154in}}{\pgfqpoint{2.373277in}{2.257629in}}%
\pgfpathcurveto{\pgfqpoint{2.373277in}{2.252104in}}{\pgfqpoint{2.375472in}{2.246805in}}{\pgfqpoint{2.379379in}{2.242898in}}%
\pgfpathcurveto{\pgfqpoint{2.383286in}{2.238991in}}{\pgfqpoint{2.388585in}{2.236796in}}{\pgfqpoint{2.394110in}{2.236796in}}%
\pgfpathclose%
\pgfusepath{fill}%
\end{pgfscope}%
\begin{pgfscope}%
\pgfpathrectangle{\pgfqpoint{2.051725in}{1.668832in}}{\pgfqpoint{1.162500in}{0.755000in}}%
\pgfusepath{clip}%
\pgfsetbuttcap%
\pgfsetroundjoin%
\definecolor{currentfill}{rgb}{0.000000,0.000000,0.000000}%
\pgfsetfillcolor{currentfill}%
\pgfsetfillopacity{0.500000}%
\pgfsetlinewidth{0.000000pt}%
\definecolor{currentstroke}{rgb}{0.000000,0.000000,0.000000}%
\pgfsetstrokecolor{currentstroke}%
\pgfsetdash{}{0pt}%
\pgfpathmoveto{\pgfqpoint{2.125339in}{2.023720in}}%
\pgfpathcurveto{\pgfqpoint{2.130865in}{2.023720in}}{\pgfqpoint{2.136164in}{2.025915in}}{\pgfqpoint{2.140071in}{2.029822in}}%
\pgfpathcurveto{\pgfqpoint{2.143978in}{2.033729in}}{\pgfqpoint{2.146173in}{2.039028in}}{\pgfqpoint{2.146173in}{2.044553in}}%
\pgfpathcurveto{\pgfqpoint{2.146173in}{2.050078in}}{\pgfqpoint{2.143978in}{2.055378in}}{\pgfqpoint{2.140071in}{2.059285in}}%
\pgfpathcurveto{\pgfqpoint{2.136164in}{2.063191in}}{\pgfqpoint{2.130865in}{2.065387in}}{\pgfqpoint{2.125339in}{2.065387in}}%
\pgfpathcurveto{\pgfqpoint{2.119814in}{2.065387in}}{\pgfqpoint{2.114515in}{2.063191in}}{\pgfqpoint{2.110608in}{2.059285in}}%
\pgfpathcurveto{\pgfqpoint{2.106701in}{2.055378in}}{\pgfqpoint{2.104506in}{2.050078in}}{\pgfqpoint{2.104506in}{2.044553in}}%
\pgfpathcurveto{\pgfqpoint{2.104506in}{2.039028in}}{\pgfqpoint{2.106701in}{2.033729in}}{\pgfqpoint{2.110608in}{2.029822in}}%
\pgfpathcurveto{\pgfqpoint{2.114515in}{2.025915in}}{\pgfqpoint{2.119814in}{2.023720in}}{\pgfqpoint{2.125339in}{2.023720in}}%
\pgfpathclose%
\pgfusepath{fill}%
\end{pgfscope}%
\begin{pgfscope}%
\pgfsetrectcap%
\pgfsetmiterjoin%
\pgfsetlinewidth{0.803000pt}%
\definecolor{currentstroke}{rgb}{0.501961,0.501961,0.501961}%
\pgfsetstrokecolor{currentstroke}%
\pgfsetdash{}{0pt}%
\pgfpathmoveto{\pgfqpoint{2.051725in}{1.668832in}}%
\pgfpathlineto{\pgfqpoint{2.051725in}{2.423832in}}%
\pgfusepath{stroke}%
\end{pgfscope}%
\begin{pgfscope}%
\pgfsetrectcap%
\pgfsetmiterjoin%
\pgfsetlinewidth{0.803000pt}%
\definecolor{currentstroke}{rgb}{0.501961,0.501961,0.501961}%
\pgfsetstrokecolor{currentstroke}%
\pgfsetdash{}{0pt}%
\pgfpathmoveto{\pgfqpoint{3.214225in}{1.668832in}}%
\pgfpathlineto{\pgfqpoint{3.214225in}{2.423832in}}%
\pgfusepath{stroke}%
\end{pgfscope}%
\begin{pgfscope}%
\pgfsetrectcap%
\pgfsetmiterjoin%
\pgfsetlinewidth{0.803000pt}%
\definecolor{currentstroke}{rgb}{0.501961,0.501961,0.501961}%
\pgfsetstrokecolor{currentstroke}%
\pgfsetdash{}{0pt}%
\pgfpathmoveto{\pgfqpoint{2.051725in}{1.668832in}}%
\pgfpathlineto{\pgfqpoint{3.214225in}{1.668832in}}%
\pgfusepath{stroke}%
\end{pgfscope}%
\begin{pgfscope}%
\pgfsetrectcap%
\pgfsetmiterjoin%
\pgfsetlinewidth{0.803000pt}%
\definecolor{currentstroke}{rgb}{0.501961,0.501961,0.501961}%
\pgfsetstrokecolor{currentstroke}%
\pgfsetdash{}{0pt}%
\pgfpathmoveto{\pgfqpoint{2.051725in}{2.423832in}}%
\pgfpathlineto{\pgfqpoint{3.214225in}{2.423832in}}%
\pgfusepath{stroke}%
\end{pgfscope}%
\begin{pgfscope}%
\pgfsetbuttcap%
\pgfsetmiterjoin%
\definecolor{currentfill}{rgb}{1.000000,1.000000,1.000000}%
\pgfsetfillcolor{currentfill}%
\pgfsetlinewidth{0.000000pt}%
\definecolor{currentstroke}{rgb}{0.000000,0.000000,0.000000}%
\pgfsetstrokecolor{currentstroke}%
\pgfsetstrokeopacity{0.000000}%
\pgfsetdash{}{0pt}%
\pgfpathmoveto{\pgfqpoint{3.214225in}{1.668832in}}%
\pgfpathlineto{\pgfqpoint{4.376725in}{1.668832in}}%
\pgfpathlineto{\pgfqpoint{4.376725in}{2.423832in}}%
\pgfpathlineto{\pgfqpoint{3.214225in}{2.423832in}}%
\pgfpathclose%
\pgfusepath{fill}%
\end{pgfscope}%
\begin{pgfscope}%
\pgfpathrectangle{\pgfqpoint{3.214225in}{1.668832in}}{\pgfqpoint{1.162500in}{0.755000in}}%
\pgfusepath{clip}%
\pgfsetrectcap%
\pgfsetroundjoin%
\pgfsetlinewidth{1.505625pt}%
\definecolor{currentstroke}{rgb}{0.121569,0.466667,0.705882}%
\pgfsetstrokecolor{currentstroke}%
\pgfsetdash{}{0pt}%
\pgfpathmoveto{\pgfqpoint{3.241904in}{1.703151in}}%
\pgfpathlineto{\pgfqpoint{3.274043in}{1.719497in}}%
\pgfpathlineto{\pgfqpoint{3.312832in}{1.736554in}}%
\pgfpathlineto{\pgfqpoint{3.382651in}{1.766730in}}%
\pgfpathlineto{\pgfqpoint{3.409249in}{1.781089in}}%
\pgfpathlineto{\pgfqpoint{3.432523in}{1.796220in}}%
\pgfpathlineto{\pgfqpoint{3.454688in}{1.813410in}}%
\pgfpathlineto{\pgfqpoint{3.475745in}{1.832588in}}%
\pgfpathlineto{\pgfqpoint{3.496801in}{1.854715in}}%
\pgfpathlineto{\pgfqpoint{3.518966in}{1.881209in}}%
\pgfpathlineto{\pgfqpoint{3.543348in}{1.913943in}}%
\pgfpathlineto{\pgfqpoint{3.571054in}{1.955094in}}%
\pgfpathlineto{\pgfqpoint{3.605410in}{2.010375in}}%
\pgfpathlineto{\pgfqpoint{3.705153in}{2.173572in}}%
\pgfpathlineto{\pgfqpoint{3.733967in}{2.215073in}}%
\pgfpathlineto{\pgfqpoint{3.759457in}{2.248062in}}%
\pgfpathlineto{\pgfqpoint{3.783838in}{2.276027in}}%
\pgfpathlineto{\pgfqpoint{3.807112in}{2.299374in}}%
\pgfpathlineto{\pgfqpoint{3.830385in}{2.319532in}}%
\pgfpathlineto{\pgfqpoint{3.853658in}{2.336684in}}%
\pgfpathlineto{\pgfqpoint{3.876931in}{2.351073in}}%
\pgfpathlineto{\pgfqpoint{3.901313in}{2.363461in}}%
\pgfpathlineto{\pgfqpoint{3.925695in}{2.373347in}}%
\pgfpathlineto{\pgfqpoint{3.950076in}{2.380899in}}%
\pgfpathlineto{\pgfqpoint{3.974458in}{2.386158in}}%
\pgfpathlineto{\pgfqpoint{3.997731in}{2.388954in}}%
\pgfpathlineto{\pgfqpoint{4.019896in}{2.389411in}}%
\pgfpathlineto{\pgfqpoint{4.040953in}{2.387609in}}%
\pgfpathlineto{\pgfqpoint{4.060901in}{2.383634in}}%
\pgfpathlineto{\pgfqpoint{4.079741in}{2.377611in}}%
\pgfpathlineto{\pgfqpoint{4.098582in}{2.369147in}}%
\pgfpathlineto{\pgfqpoint{4.116314in}{2.358752in}}%
\pgfpathlineto{\pgfqpoint{4.134046in}{2.345834in}}%
\pgfpathlineto{\pgfqpoint{4.151778in}{2.330258in}}%
\pgfpathlineto{\pgfqpoint{4.170618in}{2.310693in}}%
\pgfpathlineto{\pgfqpoint{4.190567in}{2.286536in}}%
\pgfpathlineto{\pgfqpoint{4.210515in}{2.258869in}}%
\pgfpathlineto{\pgfqpoint{4.232680in}{2.224164in}}%
\pgfpathlineto{\pgfqpoint{4.257062in}{2.181540in}}%
\pgfpathlineto{\pgfqpoint{4.283660in}{2.130435in}}%
\pgfpathlineto{\pgfqpoint{4.316907in}{2.061369in}}%
\pgfpathlineto{\pgfqpoint{4.349046in}{1.991285in}}%
\pgfpathlineto{\pgfqpoint{4.349046in}{1.991285in}}%
\pgfusepath{stroke}%
\end{pgfscope}%
\begin{pgfscope}%
\pgfsetrectcap%
\pgfsetmiterjoin%
\pgfsetlinewidth{0.803000pt}%
\definecolor{currentstroke}{rgb}{0.501961,0.501961,0.501961}%
\pgfsetstrokecolor{currentstroke}%
\pgfsetdash{}{0pt}%
\pgfpathmoveto{\pgfqpoint{3.214225in}{1.668832in}}%
\pgfpathlineto{\pgfqpoint{3.214225in}{2.423832in}}%
\pgfusepath{stroke}%
\end{pgfscope}%
\begin{pgfscope}%
\pgfsetrectcap%
\pgfsetmiterjoin%
\pgfsetlinewidth{0.803000pt}%
\definecolor{currentstroke}{rgb}{0.501961,0.501961,0.501961}%
\pgfsetstrokecolor{currentstroke}%
\pgfsetdash{}{0pt}%
\pgfpathmoveto{\pgfqpoint{4.376725in}{1.668832in}}%
\pgfpathlineto{\pgfqpoint{4.376725in}{2.423832in}}%
\pgfusepath{stroke}%
\end{pgfscope}%
\begin{pgfscope}%
\pgfsetrectcap%
\pgfsetmiterjoin%
\pgfsetlinewidth{0.803000pt}%
\definecolor{currentstroke}{rgb}{0.501961,0.501961,0.501961}%
\pgfsetstrokecolor{currentstroke}%
\pgfsetdash{}{0pt}%
\pgfpathmoveto{\pgfqpoint{3.214225in}{1.668832in}}%
\pgfpathlineto{\pgfqpoint{4.376725in}{1.668832in}}%
\pgfusepath{stroke}%
\end{pgfscope}%
\begin{pgfscope}%
\pgfsetrectcap%
\pgfsetmiterjoin%
\pgfsetlinewidth{0.803000pt}%
\definecolor{currentstroke}{rgb}{0.501961,0.501961,0.501961}%
\pgfsetstrokecolor{currentstroke}%
\pgfsetdash{}{0pt}%
\pgfpathmoveto{\pgfqpoint{3.214225in}{2.423832in}}%
\pgfpathlineto{\pgfqpoint{4.376725in}{2.423832in}}%
\pgfusepath{stroke}%
\end{pgfscope}%
\begin{pgfscope}%
\pgfsetbuttcap%
\pgfsetmiterjoin%
\definecolor{currentfill}{rgb}{1.000000,1.000000,1.000000}%
\pgfsetfillcolor{currentfill}%
\pgfsetlinewidth{0.000000pt}%
\definecolor{currentstroke}{rgb}{0.000000,0.000000,0.000000}%
\pgfsetstrokecolor{currentstroke}%
\pgfsetstrokeopacity{0.000000}%
\pgfsetdash{}{0pt}%
\pgfpathmoveto{\pgfqpoint{4.376725in}{1.668832in}}%
\pgfpathlineto{\pgfqpoint{5.539225in}{1.668832in}}%
\pgfpathlineto{\pgfqpoint{5.539225in}{2.423832in}}%
\pgfpathlineto{\pgfqpoint{4.376725in}{2.423832in}}%
\pgfpathclose%
\pgfusepath{fill}%
\end{pgfscope}%
\begin{pgfscope}%
\pgfpathrectangle{\pgfqpoint{4.376725in}{1.668832in}}{\pgfqpoint{1.162500in}{0.755000in}}%
\pgfusepath{clip}%
\pgfsetbuttcap%
\pgfsetroundjoin%
\definecolor{currentfill}{rgb}{0.000000,0.000000,0.000000}%
\pgfsetfillcolor{currentfill}%
\pgfsetfillopacity{0.500000}%
\pgfsetlinewidth{0.000000pt}%
\definecolor{currentstroke}{rgb}{0.000000,0.000000,0.000000}%
\pgfsetstrokecolor{currentstroke}%
\pgfsetdash{}{0pt}%
\pgfpathmoveto{\pgfqpoint{5.214960in}{1.935152in}}%
\pgfpathcurveto{\pgfqpoint{5.220485in}{1.935152in}}{\pgfqpoint{5.225785in}{1.937347in}}{\pgfqpoint{5.229692in}{1.941254in}}%
\pgfpathcurveto{\pgfqpoint{5.233598in}{1.945161in}}{\pgfqpoint{5.235794in}{1.950460in}}{\pgfqpoint{5.235794in}{1.955985in}}%
\pgfpathcurveto{\pgfqpoint{5.235794in}{1.961510in}}{\pgfqpoint{5.233598in}{1.966810in}}{\pgfqpoint{5.229692in}{1.970717in}}%
\pgfpathcurveto{\pgfqpoint{5.225785in}{1.974623in}}{\pgfqpoint{5.220485in}{1.976819in}}{\pgfqpoint{5.214960in}{1.976819in}}%
\pgfpathcurveto{\pgfqpoint{5.209435in}{1.976819in}}{\pgfqpoint{5.204136in}{1.974623in}}{\pgfqpoint{5.200229in}{1.970717in}}%
\pgfpathcurveto{\pgfqpoint{5.196322in}{1.966810in}}{\pgfqpoint{5.194127in}{1.961510in}}{\pgfqpoint{5.194127in}{1.955985in}}%
\pgfpathcurveto{\pgfqpoint{5.194127in}{1.950460in}}{\pgfqpoint{5.196322in}{1.945161in}}{\pgfqpoint{5.200229in}{1.941254in}}%
\pgfpathcurveto{\pgfqpoint{5.204136in}{1.937347in}}{\pgfqpoint{5.209435in}{1.935152in}}{\pgfqpoint{5.214960in}{1.935152in}}%
\pgfpathclose%
\pgfusepath{fill}%
\end{pgfscope}%
\begin{pgfscope}%
\pgfpathrectangle{\pgfqpoint{4.376725in}{1.668832in}}{\pgfqpoint{1.162500in}{0.755000in}}%
\pgfusepath{clip}%
\pgfsetbuttcap%
\pgfsetroundjoin%
\definecolor{currentfill}{rgb}{0.000000,0.000000,0.000000}%
\pgfsetfillcolor{currentfill}%
\pgfsetfillopacity{0.500000}%
\pgfsetlinewidth{0.000000pt}%
\definecolor{currentstroke}{rgb}{0.000000,0.000000,0.000000}%
\pgfsetstrokecolor{currentstroke}%
\pgfsetdash{}{0pt}%
\pgfpathmoveto{\pgfqpoint{4.521739in}{2.326950in}}%
\pgfpathcurveto{\pgfqpoint{4.527264in}{2.326950in}}{\pgfqpoint{4.532563in}{2.329145in}}{\pgfqpoint{4.536470in}{2.333052in}}%
\pgfpathcurveto{\pgfqpoint{4.540377in}{2.336959in}}{\pgfqpoint{4.542572in}{2.342258in}}{\pgfqpoint{4.542572in}{2.347783in}}%
\pgfpathcurveto{\pgfqpoint{4.542572in}{2.353308in}}{\pgfqpoint{4.540377in}{2.358608in}}{\pgfqpoint{4.536470in}{2.362515in}}%
\pgfpathcurveto{\pgfqpoint{4.532563in}{2.366421in}}{\pgfqpoint{4.527264in}{2.368616in}}{\pgfqpoint{4.521739in}{2.368616in}}%
\pgfpathcurveto{\pgfqpoint{4.516214in}{2.368616in}}{\pgfqpoint{4.510914in}{2.366421in}}{\pgfqpoint{4.507007in}{2.362515in}}%
\pgfpathcurveto{\pgfqpoint{4.503101in}{2.358608in}}{\pgfqpoint{4.500905in}{2.353308in}}{\pgfqpoint{4.500905in}{2.347783in}}%
\pgfpathcurveto{\pgfqpoint{4.500905in}{2.342258in}}{\pgfqpoint{4.503101in}{2.336959in}}{\pgfqpoint{4.507007in}{2.333052in}}%
\pgfpathcurveto{\pgfqpoint{4.510914in}{2.329145in}}{\pgfqpoint{4.516214in}{2.326950in}}{\pgfqpoint{4.521739in}{2.326950in}}%
\pgfpathclose%
\pgfusepath{fill}%
\end{pgfscope}%
\begin{pgfscope}%
\pgfpathrectangle{\pgfqpoint{4.376725in}{1.668832in}}{\pgfqpoint{1.162500in}{0.755000in}}%
\pgfusepath{clip}%
\pgfsetbuttcap%
\pgfsetroundjoin%
\definecolor{currentfill}{rgb}{0.000000,0.000000,0.000000}%
\pgfsetfillcolor{currentfill}%
\pgfsetfillopacity{0.500000}%
\pgfsetlinewidth{0.000000pt}%
\definecolor{currentstroke}{rgb}{0.000000,0.000000,0.000000}%
\pgfsetstrokecolor{currentstroke}%
\pgfsetdash{}{0pt}%
\pgfpathmoveto{\pgfqpoint{4.448919in}{2.385023in}}%
\pgfpathcurveto{\pgfqpoint{4.454444in}{2.385023in}}{\pgfqpoint{4.459744in}{2.387218in}}{\pgfqpoint{4.463650in}{2.391125in}}%
\pgfpathcurveto{\pgfqpoint{4.467557in}{2.395032in}}{\pgfqpoint{4.469752in}{2.400331in}}{\pgfqpoint{4.469752in}{2.405856in}}%
\pgfpathcurveto{\pgfqpoint{4.469752in}{2.411381in}}{\pgfqpoint{4.467557in}{2.416681in}}{\pgfqpoint{4.463650in}{2.420588in}}%
\pgfpathcurveto{\pgfqpoint{4.459744in}{2.424494in}}{\pgfqpoint{4.454444in}{2.426690in}}{\pgfqpoint{4.448919in}{2.426690in}}%
\pgfpathcurveto{\pgfqpoint{4.443394in}{2.426690in}}{\pgfqpoint{4.438094in}{2.424494in}}{\pgfqpoint{4.434188in}{2.420588in}}%
\pgfpathcurveto{\pgfqpoint{4.430281in}{2.416681in}}{\pgfqpoint{4.428086in}{2.411381in}}{\pgfqpoint{4.428086in}{2.405856in}}%
\pgfpathcurveto{\pgfqpoint{4.428086in}{2.400331in}}{\pgfqpoint{4.430281in}{2.395032in}}{\pgfqpoint{4.434188in}{2.391125in}}%
\pgfpathcurveto{\pgfqpoint{4.438094in}{2.387218in}}{\pgfqpoint{4.443394in}{2.385023in}}{\pgfqpoint{4.448919in}{2.385023in}}%
\pgfpathclose%
\pgfusepath{fill}%
\end{pgfscope}%
\begin{pgfscope}%
\pgfpathrectangle{\pgfqpoint{4.376725in}{1.668832in}}{\pgfqpoint{1.162500in}{0.755000in}}%
\pgfusepath{clip}%
\pgfsetbuttcap%
\pgfsetroundjoin%
\definecolor{currentfill}{rgb}{0.000000,0.000000,0.000000}%
\pgfsetfillcolor{currentfill}%
\pgfsetfillopacity{0.500000}%
\pgfsetlinewidth{0.000000pt}%
\definecolor{currentstroke}{rgb}{0.000000,0.000000,0.000000}%
\pgfsetstrokecolor{currentstroke}%
\pgfsetdash{}{0pt}%
\pgfpathmoveto{\pgfqpoint{4.704251in}{2.103661in}}%
\pgfpathcurveto{\pgfqpoint{4.709776in}{2.103661in}}{\pgfqpoint{4.715075in}{2.105856in}}{\pgfqpoint{4.718982in}{2.109762in}}%
\pgfpathcurveto{\pgfqpoint{4.722889in}{2.113669in}}{\pgfqpoint{4.725084in}{2.118969in}}{\pgfqpoint{4.725084in}{2.124494in}}%
\pgfpathcurveto{\pgfqpoint{4.725084in}{2.130019in}}{\pgfqpoint{4.722889in}{2.135318in}}{\pgfqpoint{4.718982in}{2.139225in}}%
\pgfpathcurveto{\pgfqpoint{4.715075in}{2.143132in}}{\pgfqpoint{4.709776in}{2.145327in}}{\pgfqpoint{4.704251in}{2.145327in}}%
\pgfpathcurveto{\pgfqpoint{4.698726in}{2.145327in}}{\pgfqpoint{4.693426in}{2.143132in}}{\pgfqpoint{4.689519in}{2.139225in}}%
\pgfpathcurveto{\pgfqpoint{4.685612in}{2.135318in}}{\pgfqpoint{4.683417in}{2.130019in}}{\pgfqpoint{4.683417in}{2.124494in}}%
\pgfpathcurveto{\pgfqpoint{4.683417in}{2.118969in}}{\pgfqpoint{4.685612in}{2.113669in}}{\pgfqpoint{4.689519in}{2.109762in}}%
\pgfpathcurveto{\pgfqpoint{4.693426in}{2.105856in}}{\pgfqpoint{4.698726in}{2.103661in}}{\pgfqpoint{4.704251in}{2.103661in}}%
\pgfpathclose%
\pgfusepath{fill}%
\end{pgfscope}%
\begin{pgfscope}%
\pgfpathrectangle{\pgfqpoint{4.376725in}{1.668832in}}{\pgfqpoint{1.162500in}{0.755000in}}%
\pgfusepath{clip}%
\pgfsetbuttcap%
\pgfsetroundjoin%
\definecolor{currentfill}{rgb}{0.000000,0.000000,0.000000}%
\pgfsetfillcolor{currentfill}%
\pgfsetfillopacity{0.500000}%
\pgfsetlinewidth{0.000000pt}%
\definecolor{currentstroke}{rgb}{0.000000,0.000000,0.000000}%
\pgfsetstrokecolor{currentstroke}%
\pgfsetdash{}{0pt}%
\pgfpathmoveto{\pgfqpoint{4.404404in}{2.149550in}}%
\pgfpathcurveto{\pgfqpoint{4.409929in}{2.149550in}}{\pgfqpoint{4.415228in}{2.151746in}}{\pgfqpoint{4.419135in}{2.155652in}}%
\pgfpathcurveto{\pgfqpoint{4.423042in}{2.159559in}}{\pgfqpoint{4.425237in}{2.164859in}}{\pgfqpoint{4.425237in}{2.170384in}}%
\pgfpathcurveto{\pgfqpoint{4.425237in}{2.175909in}}{\pgfqpoint{4.423042in}{2.181208in}}{\pgfqpoint{4.419135in}{2.185115in}}%
\pgfpathcurveto{\pgfqpoint{4.415228in}{2.189022in}}{\pgfqpoint{4.409929in}{2.191217in}}{\pgfqpoint{4.404404in}{2.191217in}}%
\pgfpathcurveto{\pgfqpoint{4.398879in}{2.191217in}}{\pgfqpoint{4.393579in}{2.189022in}}{\pgfqpoint{4.389672in}{2.185115in}}%
\pgfpathcurveto{\pgfqpoint{4.385765in}{2.181208in}}{\pgfqpoint{4.383570in}{2.175909in}}{\pgfqpoint{4.383570in}{2.170384in}}%
\pgfpathcurveto{\pgfqpoint{4.383570in}{2.164859in}}{\pgfqpoint{4.385765in}{2.159559in}}{\pgfqpoint{4.389672in}{2.155652in}}%
\pgfpathcurveto{\pgfqpoint{4.393579in}{2.151746in}}{\pgfqpoint{4.398879in}{2.149550in}}{\pgfqpoint{4.404404in}{2.149550in}}%
\pgfpathclose%
\pgfusepath{fill}%
\end{pgfscope}%
\begin{pgfscope}%
\pgfpathrectangle{\pgfqpoint{4.376725in}{1.668832in}}{\pgfqpoint{1.162500in}{0.755000in}}%
\pgfusepath{clip}%
\pgfsetbuttcap%
\pgfsetroundjoin%
\definecolor{currentfill}{rgb}{0.000000,0.000000,0.000000}%
\pgfsetfillcolor{currentfill}%
\pgfsetfillopacity{0.500000}%
\pgfsetlinewidth{0.000000pt}%
\definecolor{currentstroke}{rgb}{0.000000,0.000000,0.000000}%
\pgfsetstrokecolor{currentstroke}%
\pgfsetdash{}{0pt}%
\pgfpathmoveto{\pgfqpoint{4.607580in}{2.049204in}}%
\pgfpathcurveto{\pgfqpoint{4.613105in}{2.049204in}}{\pgfqpoint{4.618405in}{2.051399in}}{\pgfqpoint{4.622312in}{2.055306in}}%
\pgfpathcurveto{\pgfqpoint{4.626218in}{2.059212in}}{\pgfqpoint{4.628414in}{2.064512in}}{\pgfqpoint{4.628414in}{2.070037in}}%
\pgfpathcurveto{\pgfqpoint{4.628414in}{2.075562in}}{\pgfqpoint{4.626218in}{2.080862in}}{\pgfqpoint{4.622312in}{2.084768in}}%
\pgfpathcurveto{\pgfqpoint{4.618405in}{2.088675in}}{\pgfqpoint{4.613105in}{2.090870in}}{\pgfqpoint{4.607580in}{2.090870in}}%
\pgfpathcurveto{\pgfqpoint{4.602055in}{2.090870in}}{\pgfqpoint{4.596756in}{2.088675in}}{\pgfqpoint{4.592849in}{2.084768in}}%
\pgfpathcurveto{\pgfqpoint{4.588942in}{2.080862in}}{\pgfqpoint{4.586747in}{2.075562in}}{\pgfqpoint{4.586747in}{2.070037in}}%
\pgfpathcurveto{\pgfqpoint{4.586747in}{2.064512in}}{\pgfqpoint{4.588942in}{2.059212in}}{\pgfqpoint{4.592849in}{2.055306in}}%
\pgfpathcurveto{\pgfqpoint{4.596756in}{2.051399in}}{\pgfqpoint{4.602055in}{2.049204in}}{\pgfqpoint{4.607580in}{2.049204in}}%
\pgfpathclose%
\pgfusepath{fill}%
\end{pgfscope}%
\begin{pgfscope}%
\pgfpathrectangle{\pgfqpoint{4.376725in}{1.668832in}}{\pgfqpoint{1.162500in}{0.755000in}}%
\pgfusepath{clip}%
\pgfsetbuttcap%
\pgfsetroundjoin%
\definecolor{currentfill}{rgb}{0.000000,0.000000,0.000000}%
\pgfsetfillcolor{currentfill}%
\pgfsetfillopacity{0.500000}%
\pgfsetlinewidth{0.000000pt}%
\definecolor{currentstroke}{rgb}{0.000000,0.000000,0.000000}%
\pgfsetstrokecolor{currentstroke}%
\pgfsetdash{}{0pt}%
\pgfpathmoveto{\pgfqpoint{4.596304in}{1.665975in}}%
\pgfpathcurveto{\pgfqpoint{4.601829in}{1.665975in}}{\pgfqpoint{4.607129in}{1.668170in}}{\pgfqpoint{4.611036in}{1.672077in}}%
\pgfpathcurveto{\pgfqpoint{4.614942in}{1.675984in}}{\pgfqpoint{4.617137in}{1.681283in}}{\pgfqpoint{4.617137in}{1.686809in}}%
\pgfpathcurveto{\pgfqpoint{4.617137in}{1.692334in}}{\pgfqpoint{4.614942in}{1.697633in}}{\pgfqpoint{4.611036in}{1.701540in}}%
\pgfpathcurveto{\pgfqpoint{4.607129in}{1.705447in}}{\pgfqpoint{4.601829in}{1.707642in}}{\pgfqpoint{4.596304in}{1.707642in}}%
\pgfpathcurveto{\pgfqpoint{4.590779in}{1.707642in}}{\pgfqpoint{4.585480in}{1.705447in}}{\pgfqpoint{4.581573in}{1.701540in}}%
\pgfpathcurveto{\pgfqpoint{4.577666in}{1.697633in}}{\pgfqpoint{4.575471in}{1.692334in}}{\pgfqpoint{4.575471in}{1.686809in}}%
\pgfpathcurveto{\pgfqpoint{4.575471in}{1.681283in}}{\pgfqpoint{4.577666in}{1.675984in}}{\pgfqpoint{4.581573in}{1.672077in}}%
\pgfpathcurveto{\pgfqpoint{4.585480in}{1.668170in}}{\pgfqpoint{4.590779in}{1.665975in}}{\pgfqpoint{4.596304in}{1.665975in}}%
\pgfpathclose%
\pgfusepath{fill}%
\end{pgfscope}%
\begin{pgfscope}%
\pgfpathrectangle{\pgfqpoint{4.376725in}{1.668832in}}{\pgfqpoint{1.162500in}{0.755000in}}%
\pgfusepath{clip}%
\pgfsetbuttcap%
\pgfsetroundjoin%
\definecolor{currentfill}{rgb}{0.000000,0.000000,0.000000}%
\pgfsetfillcolor{currentfill}%
\pgfsetfillopacity{0.500000}%
\pgfsetlinewidth{0.000000pt}%
\definecolor{currentstroke}{rgb}{0.000000,0.000000,0.000000}%
\pgfsetstrokecolor{currentstroke}%
\pgfsetdash{}{0pt}%
\pgfpathmoveto{\pgfqpoint{5.423123in}{2.301682in}}%
\pgfpathcurveto{\pgfqpoint{5.428649in}{2.301682in}}{\pgfqpoint{5.433948in}{2.303877in}}{\pgfqpoint{5.437855in}{2.307784in}}%
\pgfpathcurveto{\pgfqpoint{5.441762in}{2.311691in}}{\pgfqpoint{5.443957in}{2.316990in}}{\pgfqpoint{5.443957in}{2.322515in}}%
\pgfpathcurveto{\pgfqpoint{5.443957in}{2.328041in}}{\pgfqpoint{5.441762in}{2.333340in}}{\pgfqpoint{5.437855in}{2.337247in}}%
\pgfpathcurveto{\pgfqpoint{5.433948in}{2.341154in}}{\pgfqpoint{5.428649in}{2.343349in}}{\pgfqpoint{5.423123in}{2.343349in}}%
\pgfpathcurveto{\pgfqpoint{5.417598in}{2.343349in}}{\pgfqpoint{5.412299in}{2.341154in}}{\pgfqpoint{5.408392in}{2.337247in}}%
\pgfpathcurveto{\pgfqpoint{5.404485in}{2.333340in}}{\pgfqpoint{5.402290in}{2.328041in}}{\pgfqpoint{5.402290in}{2.322515in}}%
\pgfpathcurveto{\pgfqpoint{5.402290in}{2.316990in}}{\pgfqpoint{5.404485in}{2.311691in}}{\pgfqpoint{5.408392in}{2.307784in}}%
\pgfpathcurveto{\pgfqpoint{5.412299in}{2.303877in}}{\pgfqpoint{5.417598in}{2.301682in}}{\pgfqpoint{5.423123in}{2.301682in}}%
\pgfpathclose%
\pgfusepath{fill}%
\end{pgfscope}%
\begin{pgfscope}%
\pgfpathrectangle{\pgfqpoint{4.376725in}{1.668832in}}{\pgfqpoint{1.162500in}{0.755000in}}%
\pgfusepath{clip}%
\pgfsetbuttcap%
\pgfsetroundjoin%
\definecolor{currentfill}{rgb}{0.000000,0.000000,0.000000}%
\pgfsetfillcolor{currentfill}%
\pgfsetfillopacity{0.500000}%
\pgfsetlinewidth{0.000000pt}%
\definecolor{currentstroke}{rgb}{0.000000,0.000000,0.000000}%
\pgfsetstrokecolor{currentstroke}%
\pgfsetdash{}{0pt}%
\pgfpathmoveto{\pgfqpoint{5.511546in}{2.145886in}}%
\pgfpathcurveto{\pgfqpoint{5.517071in}{2.145886in}}{\pgfqpoint{5.522371in}{2.148081in}}{\pgfqpoint{5.526278in}{2.151988in}}%
\pgfpathcurveto{\pgfqpoint{5.530185in}{2.155894in}}{\pgfqpoint{5.532380in}{2.161194in}}{\pgfqpoint{5.532380in}{2.166719in}}%
\pgfpathcurveto{\pgfqpoint{5.532380in}{2.172244in}}{\pgfqpoint{5.530185in}{2.177544in}}{\pgfqpoint{5.526278in}{2.181450in}}%
\pgfpathcurveto{\pgfqpoint{5.522371in}{2.185357in}}{\pgfqpoint{5.517071in}{2.187552in}}{\pgfqpoint{5.511546in}{2.187552in}}%
\pgfpathcurveto{\pgfqpoint{5.506021in}{2.187552in}}{\pgfqpoint{5.500722in}{2.185357in}}{\pgfqpoint{5.496815in}{2.181450in}}%
\pgfpathcurveto{\pgfqpoint{5.492908in}{2.177544in}}{\pgfqpoint{5.490713in}{2.172244in}}{\pgfqpoint{5.490713in}{2.166719in}}%
\pgfpathcurveto{\pgfqpoint{5.490713in}{2.161194in}}{\pgfqpoint{5.492908in}{2.155894in}}{\pgfqpoint{5.496815in}{2.151988in}}%
\pgfpathcurveto{\pgfqpoint{5.500722in}{2.148081in}}{\pgfqpoint{5.506021in}{2.145886in}}{\pgfqpoint{5.511546in}{2.145886in}}%
\pgfpathclose%
\pgfusepath{fill}%
\end{pgfscope}%
\begin{pgfscope}%
\pgfpathrectangle{\pgfqpoint{4.376725in}{1.668832in}}{\pgfqpoint{1.162500in}{0.755000in}}%
\pgfusepath{clip}%
\pgfsetbuttcap%
\pgfsetroundjoin%
\definecolor{currentfill}{rgb}{0.000000,0.000000,0.000000}%
\pgfsetfillcolor{currentfill}%
\pgfsetfillopacity{0.500000}%
\pgfsetlinewidth{0.000000pt}%
\definecolor{currentstroke}{rgb}{0.000000,0.000000,0.000000}%
\pgfsetstrokecolor{currentstroke}%
\pgfsetdash{}{0pt}%
\pgfpathmoveto{\pgfqpoint{5.286995in}{1.969105in}}%
\pgfpathcurveto{\pgfqpoint{5.292520in}{1.969105in}}{\pgfqpoint{5.297820in}{1.971300in}}{\pgfqpoint{5.301726in}{1.975207in}}%
\pgfpathcurveto{\pgfqpoint{5.305633in}{1.979114in}}{\pgfqpoint{5.307828in}{1.984413in}}{\pgfqpoint{5.307828in}{1.989939in}}%
\pgfpathcurveto{\pgfqpoint{5.307828in}{1.995464in}}{\pgfqpoint{5.305633in}{2.000763in}}{\pgfqpoint{5.301726in}{2.004670in}}%
\pgfpathcurveto{\pgfqpoint{5.297820in}{2.008577in}}{\pgfqpoint{5.292520in}{2.010772in}}{\pgfqpoint{5.286995in}{2.010772in}}%
\pgfpathcurveto{\pgfqpoint{5.281470in}{2.010772in}}{\pgfqpoint{5.276170in}{2.008577in}}{\pgfqpoint{5.272264in}{2.004670in}}%
\pgfpathcurveto{\pgfqpoint{5.268357in}{2.000763in}}{\pgfqpoint{5.266162in}{1.995464in}}{\pgfqpoint{5.266162in}{1.989939in}}%
\pgfpathcurveto{\pgfqpoint{5.266162in}{1.984413in}}{\pgfqpoint{5.268357in}{1.979114in}}{\pgfqpoint{5.272264in}{1.975207in}}%
\pgfpathcurveto{\pgfqpoint{5.276170in}{1.971300in}}{\pgfqpoint{5.281470in}{1.969105in}}{\pgfqpoint{5.286995in}{1.969105in}}%
\pgfpathclose%
\pgfusepath{fill}%
\end{pgfscope}%
\begin{pgfscope}%
\pgfpathrectangle{\pgfqpoint{4.376725in}{1.668832in}}{\pgfqpoint{1.162500in}{0.755000in}}%
\pgfusepath{clip}%
\pgfsetbuttcap%
\pgfsetroundjoin%
\definecolor{currentfill}{rgb}{0.000000,0.000000,0.000000}%
\pgfsetfillcolor{currentfill}%
\pgfsetfillopacity{0.500000}%
\pgfsetlinewidth{0.000000pt}%
\definecolor{currentstroke}{rgb}{0.000000,0.000000,0.000000}%
\pgfsetstrokecolor{currentstroke}%
\pgfsetdash{}{0pt}%
\pgfpathmoveto{\pgfqpoint{4.645822in}{2.239104in}}%
\pgfpathcurveto{\pgfqpoint{4.651348in}{2.239104in}}{\pgfqpoint{4.656647in}{2.241300in}}{\pgfqpoint{4.660554in}{2.245206in}}%
\pgfpathcurveto{\pgfqpoint{4.664461in}{2.249113in}}{\pgfqpoint{4.666656in}{2.254413in}}{\pgfqpoint{4.666656in}{2.259938in}}%
\pgfpathcurveto{\pgfqpoint{4.666656in}{2.265463in}}{\pgfqpoint{4.664461in}{2.270762in}}{\pgfqpoint{4.660554in}{2.274669in}}%
\pgfpathcurveto{\pgfqpoint{4.656647in}{2.278576in}}{\pgfqpoint{4.651348in}{2.280771in}}{\pgfqpoint{4.645822in}{2.280771in}}%
\pgfpathcurveto{\pgfqpoint{4.640297in}{2.280771in}}{\pgfqpoint{4.634998in}{2.278576in}}{\pgfqpoint{4.631091in}{2.274669in}}%
\pgfpathcurveto{\pgfqpoint{4.627184in}{2.270762in}}{\pgfqpoint{4.624989in}{2.265463in}}{\pgfqpoint{4.624989in}{2.259938in}}%
\pgfpathcurveto{\pgfqpoint{4.624989in}{2.254413in}}{\pgfqpoint{4.627184in}{2.249113in}}{\pgfqpoint{4.631091in}{2.245206in}}%
\pgfpathcurveto{\pgfqpoint{4.634998in}{2.241300in}}{\pgfqpoint{4.640297in}{2.239104in}}{\pgfqpoint{4.645822in}{2.239104in}}%
\pgfpathclose%
\pgfusepath{fill}%
\end{pgfscope}%
\begin{pgfscope}%
\pgfpathrectangle{\pgfqpoint{4.376725in}{1.668832in}}{\pgfqpoint{1.162500in}{0.755000in}}%
\pgfusepath{clip}%
\pgfsetbuttcap%
\pgfsetroundjoin%
\definecolor{currentfill}{rgb}{0.000000,0.000000,0.000000}%
\pgfsetfillcolor{currentfill}%
\pgfsetfillopacity{0.500000}%
\pgfsetlinewidth{0.000000pt}%
\definecolor{currentstroke}{rgb}{0.000000,0.000000,0.000000}%
\pgfsetstrokecolor{currentstroke}%
\pgfsetdash{}{0pt}%
\pgfpathmoveto{\pgfqpoint{5.069839in}{1.960666in}}%
\pgfpathcurveto{\pgfqpoint{5.075364in}{1.960666in}}{\pgfqpoint{5.080663in}{1.962861in}}{\pgfqpoint{5.084570in}{1.966768in}}%
\pgfpathcurveto{\pgfqpoint{5.088477in}{1.970675in}}{\pgfqpoint{5.090672in}{1.975974in}}{\pgfqpoint{5.090672in}{1.981499in}}%
\pgfpathcurveto{\pgfqpoint{5.090672in}{1.987024in}}{\pgfqpoint{5.088477in}{1.992324in}}{\pgfqpoint{5.084570in}{1.996231in}}%
\pgfpathcurveto{\pgfqpoint{5.080663in}{2.000138in}}{\pgfqpoint{5.075364in}{2.002333in}}{\pgfqpoint{5.069839in}{2.002333in}}%
\pgfpathcurveto{\pgfqpoint{5.064313in}{2.002333in}}{\pgfqpoint{5.059014in}{2.000138in}}{\pgfqpoint{5.055107in}{1.996231in}}%
\pgfpathcurveto{\pgfqpoint{5.051200in}{1.992324in}}{\pgfqpoint{5.049005in}{1.987024in}}{\pgfqpoint{5.049005in}{1.981499in}}%
\pgfpathcurveto{\pgfqpoint{5.049005in}{1.975974in}}{\pgfqpoint{5.051200in}{1.970675in}}{\pgfqpoint{5.055107in}{1.966768in}}%
\pgfpathcurveto{\pgfqpoint{5.059014in}{1.962861in}}{\pgfqpoint{5.064313in}{1.960666in}}{\pgfqpoint{5.069839in}{1.960666in}}%
\pgfpathclose%
\pgfusepath{fill}%
\end{pgfscope}%
\begin{pgfscope}%
\pgfpathrectangle{\pgfqpoint{4.376725in}{1.668832in}}{\pgfqpoint{1.162500in}{0.755000in}}%
\pgfusepath{clip}%
\pgfsetbuttcap%
\pgfsetroundjoin%
\definecolor{currentfill}{rgb}{0.000000,0.000000,0.000000}%
\pgfsetfillcolor{currentfill}%
\pgfsetfillopacity{0.500000}%
\pgfsetlinewidth{0.000000pt}%
\definecolor{currentstroke}{rgb}{0.000000,0.000000,0.000000}%
\pgfsetstrokecolor{currentstroke}%
\pgfsetdash{}{0pt}%
\pgfpathmoveto{\pgfqpoint{4.941375in}{1.722959in}}%
\pgfpathcurveto{\pgfqpoint{4.946900in}{1.722959in}}{\pgfqpoint{4.952199in}{1.725154in}}{\pgfqpoint{4.956106in}{1.729061in}}%
\pgfpathcurveto{\pgfqpoint{4.960013in}{1.732968in}}{\pgfqpoint{4.962208in}{1.738267in}}{\pgfqpoint{4.962208in}{1.743792in}}%
\pgfpathcurveto{\pgfqpoint{4.962208in}{1.749317in}}{\pgfqpoint{4.960013in}{1.754617in}}{\pgfqpoint{4.956106in}{1.758523in}}%
\pgfpathcurveto{\pgfqpoint{4.952199in}{1.762430in}}{\pgfqpoint{4.946900in}{1.764625in}}{\pgfqpoint{4.941375in}{1.764625in}}%
\pgfpathcurveto{\pgfqpoint{4.935850in}{1.764625in}}{\pgfqpoint{4.930550in}{1.762430in}}{\pgfqpoint{4.926643in}{1.758523in}}%
\pgfpathcurveto{\pgfqpoint{4.922736in}{1.754617in}}{\pgfqpoint{4.920541in}{1.749317in}}{\pgfqpoint{4.920541in}{1.743792in}}%
\pgfpathcurveto{\pgfqpoint{4.920541in}{1.738267in}}{\pgfqpoint{4.922736in}{1.732968in}}{\pgfqpoint{4.926643in}{1.729061in}}%
\pgfpathcurveto{\pgfqpoint{4.930550in}{1.725154in}}{\pgfqpoint{4.935850in}{1.722959in}}{\pgfqpoint{4.941375in}{1.722959in}}%
\pgfpathclose%
\pgfusepath{fill}%
\end{pgfscope}%
\begin{pgfscope}%
\pgfpathrectangle{\pgfqpoint{4.376725in}{1.668832in}}{\pgfqpoint{1.162500in}{0.755000in}}%
\pgfusepath{clip}%
\pgfsetbuttcap%
\pgfsetroundjoin%
\definecolor{currentfill}{rgb}{0.000000,0.000000,0.000000}%
\pgfsetfillcolor{currentfill}%
\pgfsetfillopacity{0.500000}%
\pgfsetlinewidth{0.000000pt}%
\definecolor{currentstroke}{rgb}{0.000000,0.000000,0.000000}%
\pgfsetstrokecolor{currentstroke}%
\pgfsetdash{}{0pt}%
\pgfpathmoveto{\pgfqpoint{4.799153in}{2.267350in}}%
\pgfpathcurveto{\pgfqpoint{4.804678in}{2.267350in}}{\pgfqpoint{4.809977in}{2.269545in}}{\pgfqpoint{4.813884in}{2.273452in}}%
\pgfpathcurveto{\pgfqpoint{4.817791in}{2.277359in}}{\pgfqpoint{4.819986in}{2.282659in}}{\pgfqpoint{4.819986in}{2.288184in}}%
\pgfpathcurveto{\pgfqpoint{4.819986in}{2.293709in}}{\pgfqpoint{4.817791in}{2.299008in}}{\pgfqpoint{4.813884in}{2.302915in}}%
\pgfpathcurveto{\pgfqpoint{4.809977in}{2.306822in}}{\pgfqpoint{4.804678in}{2.309017in}}{\pgfqpoint{4.799153in}{2.309017in}}%
\pgfpathcurveto{\pgfqpoint{4.793628in}{2.309017in}}{\pgfqpoint{4.788328in}{2.306822in}}{\pgfqpoint{4.784421in}{2.302915in}}%
\pgfpathcurveto{\pgfqpoint{4.780515in}{2.299008in}}{\pgfqpoint{4.778320in}{2.293709in}}{\pgfqpoint{4.778320in}{2.288184in}}%
\pgfpathcurveto{\pgfqpoint{4.778320in}{2.282659in}}{\pgfqpoint{4.780515in}{2.277359in}}{\pgfqpoint{4.784421in}{2.273452in}}%
\pgfpathcurveto{\pgfqpoint{4.788328in}{2.269545in}}{\pgfqpoint{4.793628in}{2.267350in}}{\pgfqpoint{4.799153in}{2.267350in}}%
\pgfpathclose%
\pgfusepath{fill}%
\end{pgfscope}%
\begin{pgfscope}%
\pgfpathrectangle{\pgfqpoint{4.376725in}{1.668832in}}{\pgfqpoint{1.162500in}{0.755000in}}%
\pgfusepath{clip}%
\pgfsetbuttcap%
\pgfsetroundjoin%
\definecolor{currentfill}{rgb}{0.000000,0.000000,0.000000}%
\pgfsetfillcolor{currentfill}%
\pgfsetfillopacity{0.500000}%
\pgfsetlinewidth{0.000000pt}%
\definecolor{currentstroke}{rgb}{0.000000,0.000000,0.000000}%
\pgfsetstrokecolor{currentstroke}%
\pgfsetdash{}{0pt}%
\pgfpathmoveto{\pgfqpoint{4.595218in}{2.236796in}}%
\pgfpathcurveto{\pgfqpoint{4.600743in}{2.236796in}}{\pgfqpoint{4.606043in}{2.238991in}}{\pgfqpoint{4.609950in}{2.242898in}}%
\pgfpathcurveto{\pgfqpoint{4.613857in}{2.246805in}}{\pgfqpoint{4.616052in}{2.252104in}}{\pgfqpoint{4.616052in}{2.257629in}}%
\pgfpathcurveto{\pgfqpoint{4.616052in}{2.263154in}}{\pgfqpoint{4.613857in}{2.268454in}}{\pgfqpoint{4.609950in}{2.272361in}}%
\pgfpathcurveto{\pgfqpoint{4.606043in}{2.276267in}}{\pgfqpoint{4.600743in}{2.278463in}}{\pgfqpoint{4.595218in}{2.278463in}}%
\pgfpathcurveto{\pgfqpoint{4.589693in}{2.278463in}}{\pgfqpoint{4.584394in}{2.276267in}}{\pgfqpoint{4.580487in}{2.272361in}}%
\pgfpathcurveto{\pgfqpoint{4.576580in}{2.268454in}}{\pgfqpoint{4.574385in}{2.263154in}}{\pgfqpoint{4.574385in}{2.257629in}}%
\pgfpathcurveto{\pgfqpoint{4.574385in}{2.252104in}}{\pgfqpoint{4.576580in}{2.246805in}}{\pgfqpoint{4.580487in}{2.242898in}}%
\pgfpathcurveto{\pgfqpoint{4.584394in}{2.238991in}}{\pgfqpoint{4.589693in}{2.236796in}}{\pgfqpoint{4.595218in}{2.236796in}}%
\pgfpathclose%
\pgfusepath{fill}%
\end{pgfscope}%
\begin{pgfscope}%
\pgfpathrectangle{\pgfqpoint{4.376725in}{1.668832in}}{\pgfqpoint{1.162500in}{0.755000in}}%
\pgfusepath{clip}%
\pgfsetbuttcap%
\pgfsetroundjoin%
\definecolor{currentfill}{rgb}{0.000000,0.000000,0.000000}%
\pgfsetfillcolor{currentfill}%
\pgfsetfillopacity{0.500000}%
\pgfsetlinewidth{0.000000pt}%
\definecolor{currentstroke}{rgb}{0.000000,0.000000,0.000000}%
\pgfsetstrokecolor{currentstroke}%
\pgfsetdash{}{0pt}%
\pgfpathmoveto{\pgfqpoint{4.834270in}{2.023720in}}%
\pgfpathcurveto{\pgfqpoint{4.839795in}{2.023720in}}{\pgfqpoint{4.845094in}{2.025915in}}{\pgfqpoint{4.849001in}{2.029822in}}%
\pgfpathcurveto{\pgfqpoint{4.852908in}{2.033729in}}{\pgfqpoint{4.855103in}{2.039028in}}{\pgfqpoint{4.855103in}{2.044553in}}%
\pgfpathcurveto{\pgfqpoint{4.855103in}{2.050078in}}{\pgfqpoint{4.852908in}{2.055378in}}{\pgfqpoint{4.849001in}{2.059285in}}%
\pgfpathcurveto{\pgfqpoint{4.845094in}{2.063191in}}{\pgfqpoint{4.839795in}{2.065387in}}{\pgfqpoint{4.834270in}{2.065387in}}%
\pgfpathcurveto{\pgfqpoint{4.828745in}{2.065387in}}{\pgfqpoint{4.823445in}{2.063191in}}{\pgfqpoint{4.819538in}{2.059285in}}%
\pgfpathcurveto{\pgfqpoint{4.815632in}{2.055378in}}{\pgfqpoint{4.813437in}{2.050078in}}{\pgfqpoint{4.813437in}{2.044553in}}%
\pgfpathcurveto{\pgfqpoint{4.813437in}{2.039028in}}{\pgfqpoint{4.815632in}{2.033729in}}{\pgfqpoint{4.819538in}{2.029822in}}%
\pgfpathcurveto{\pgfqpoint{4.823445in}{2.025915in}}{\pgfqpoint{4.828745in}{2.023720in}}{\pgfqpoint{4.834270in}{2.023720in}}%
\pgfpathclose%
\pgfusepath{fill}%
\end{pgfscope}%
\begin{pgfscope}%
\pgfsetrectcap%
\pgfsetmiterjoin%
\pgfsetlinewidth{0.803000pt}%
\definecolor{currentstroke}{rgb}{0.501961,0.501961,0.501961}%
\pgfsetstrokecolor{currentstroke}%
\pgfsetdash{}{0pt}%
\pgfpathmoveto{\pgfqpoint{4.376725in}{1.668832in}}%
\pgfpathlineto{\pgfqpoint{4.376725in}{2.423832in}}%
\pgfusepath{stroke}%
\end{pgfscope}%
\begin{pgfscope}%
\pgfsetrectcap%
\pgfsetmiterjoin%
\pgfsetlinewidth{0.803000pt}%
\definecolor{currentstroke}{rgb}{0.501961,0.501961,0.501961}%
\pgfsetstrokecolor{currentstroke}%
\pgfsetdash{}{0pt}%
\pgfpathmoveto{\pgfqpoint{5.539225in}{1.668832in}}%
\pgfpathlineto{\pgfqpoint{5.539225in}{2.423832in}}%
\pgfusepath{stroke}%
\end{pgfscope}%
\begin{pgfscope}%
\pgfsetrectcap%
\pgfsetmiterjoin%
\pgfsetlinewidth{0.803000pt}%
\definecolor{currentstroke}{rgb}{0.501961,0.501961,0.501961}%
\pgfsetstrokecolor{currentstroke}%
\pgfsetdash{}{0pt}%
\pgfpathmoveto{\pgfqpoint{4.376725in}{1.668832in}}%
\pgfpathlineto{\pgfqpoint{5.539225in}{1.668832in}}%
\pgfusepath{stroke}%
\end{pgfscope}%
\begin{pgfscope}%
\pgfsetrectcap%
\pgfsetmiterjoin%
\pgfsetlinewidth{0.803000pt}%
\definecolor{currentstroke}{rgb}{0.501961,0.501961,0.501961}%
\pgfsetstrokecolor{currentstroke}%
\pgfsetdash{}{0pt}%
\pgfpathmoveto{\pgfqpoint{4.376725in}{2.423832in}}%
\pgfpathlineto{\pgfqpoint{5.539225in}{2.423832in}}%
\pgfusepath{stroke}%
\end{pgfscope}%
\begin{pgfscope}%
\pgfsetbuttcap%
\pgfsetmiterjoin%
\definecolor{currentfill}{rgb}{1.000000,1.000000,1.000000}%
\pgfsetfillcolor{currentfill}%
\pgfsetlinewidth{0.000000pt}%
\definecolor{currentstroke}{rgb}{0.000000,0.000000,0.000000}%
\pgfsetstrokecolor{currentstroke}%
\pgfsetstrokeopacity{0.000000}%
\pgfsetdash{}{0pt}%
\pgfpathmoveto{\pgfqpoint{0.889225in}{0.913832in}}%
\pgfpathlineto{\pgfqpoint{2.051725in}{0.913832in}}%
\pgfpathlineto{\pgfqpoint{2.051725in}{1.668832in}}%
\pgfpathlineto{\pgfqpoint{0.889225in}{1.668832in}}%
\pgfpathclose%
\pgfusepath{fill}%
\end{pgfscope}%
\begin{pgfscope}%
\pgfpathrectangle{\pgfqpoint{0.889225in}{0.913832in}}{\pgfqpoint{1.162500in}{0.755000in}}%
\pgfusepath{clip}%
\pgfsetbuttcap%
\pgfsetroundjoin%
\definecolor{currentfill}{rgb}{0.000000,0.000000,0.000000}%
\pgfsetfillcolor{currentfill}%
\pgfsetfillopacity{0.500000}%
\pgfsetlinewidth{0.000000pt}%
\definecolor{currentstroke}{rgb}{0.000000,0.000000,0.000000}%
\pgfsetstrokecolor{currentstroke}%
\pgfsetdash{}{0pt}%
\pgfpathmoveto{\pgfqpoint{1.893746in}{1.437401in}}%
\pgfpathcurveto{\pgfqpoint{1.899271in}{1.437401in}}{\pgfqpoint{1.904570in}{1.439596in}}{\pgfqpoint{1.908477in}{1.443503in}}%
\pgfpathcurveto{\pgfqpoint{1.912384in}{1.447410in}}{\pgfqpoint{1.914579in}{1.452710in}}{\pgfqpoint{1.914579in}{1.458235in}}%
\pgfpathcurveto{\pgfqpoint{1.914579in}{1.463760in}}{\pgfqpoint{1.912384in}{1.469059in}}{\pgfqpoint{1.908477in}{1.472966in}}%
\pgfpathcurveto{\pgfqpoint{1.904570in}{1.476873in}}{\pgfqpoint{1.899271in}{1.479068in}}{\pgfqpoint{1.893746in}{1.479068in}}%
\pgfpathcurveto{\pgfqpoint{1.888221in}{1.479068in}}{\pgfqpoint{1.882921in}{1.476873in}}{\pgfqpoint{1.879015in}{1.472966in}}%
\pgfpathcurveto{\pgfqpoint{1.875108in}{1.469059in}}{\pgfqpoint{1.872913in}{1.463760in}}{\pgfqpoint{1.872913in}{1.458235in}}%
\pgfpathcurveto{\pgfqpoint{1.872913in}{1.452710in}}{\pgfqpoint{1.875108in}{1.447410in}}{\pgfqpoint{1.879015in}{1.443503in}}%
\pgfpathcurveto{\pgfqpoint{1.882921in}{1.439596in}}{\pgfqpoint{1.888221in}{1.437401in}}{\pgfqpoint{1.893746in}{1.437401in}}%
\pgfpathclose%
\pgfusepath{fill}%
\end{pgfscope}%
\begin{pgfscope}%
\pgfpathrectangle{\pgfqpoint{0.889225in}{0.913832in}}{\pgfqpoint{1.162500in}{0.755000in}}%
\pgfusepath{clip}%
\pgfsetbuttcap%
\pgfsetroundjoin%
\definecolor{currentfill}{rgb}{0.000000,0.000000,0.000000}%
\pgfsetfillcolor{currentfill}%
\pgfsetfillopacity{0.500000}%
\pgfsetlinewidth{0.000000pt}%
\definecolor{currentstroke}{rgb}{0.000000,0.000000,0.000000}%
\pgfsetstrokecolor{currentstroke}%
\pgfsetdash{}{0pt}%
\pgfpathmoveto{\pgfqpoint{1.476892in}{0.987180in}}%
\pgfpathcurveto{\pgfqpoint{1.482417in}{0.987180in}}{\pgfqpoint{1.487717in}{0.989375in}}{\pgfqpoint{1.491623in}{0.993282in}}%
\pgfpathcurveto{\pgfqpoint{1.495530in}{0.997189in}}{\pgfqpoint{1.497725in}{1.002488in}}{\pgfqpoint{1.497725in}{1.008013in}}%
\pgfpathcurveto{\pgfqpoint{1.497725in}{1.013538in}}{\pgfqpoint{1.495530in}{1.018838in}}{\pgfqpoint{1.491623in}{1.022745in}}%
\pgfpathcurveto{\pgfqpoint{1.487717in}{1.026652in}}{\pgfqpoint{1.482417in}{1.028847in}}{\pgfqpoint{1.476892in}{1.028847in}}%
\pgfpathcurveto{\pgfqpoint{1.471367in}{1.028847in}}{\pgfqpoint{1.466067in}{1.026652in}}{\pgfqpoint{1.462161in}{1.022745in}}%
\pgfpathcurveto{\pgfqpoint{1.458254in}{1.018838in}}{\pgfqpoint{1.456059in}{1.013538in}}{\pgfqpoint{1.456059in}{1.008013in}}%
\pgfpathcurveto{\pgfqpoint{1.456059in}{1.002488in}}{\pgfqpoint{1.458254in}{0.997189in}}{\pgfqpoint{1.462161in}{0.993282in}}%
\pgfpathcurveto{\pgfqpoint{1.466067in}{0.989375in}}{\pgfqpoint{1.471367in}{0.987180in}}{\pgfqpoint{1.476892in}{0.987180in}}%
\pgfpathclose%
\pgfusepath{fill}%
\end{pgfscope}%
\begin{pgfscope}%
\pgfpathrectangle{\pgfqpoint{0.889225in}{0.913832in}}{\pgfqpoint{1.162500in}{0.755000in}}%
\pgfusepath{clip}%
\pgfsetbuttcap%
\pgfsetroundjoin%
\definecolor{currentfill}{rgb}{0.000000,0.000000,0.000000}%
\pgfsetfillcolor{currentfill}%
\pgfsetfillopacity{0.500000}%
\pgfsetlinewidth{0.000000pt}%
\definecolor{currentstroke}{rgb}{0.000000,0.000000,0.000000}%
\pgfsetstrokecolor{currentstroke}%
\pgfsetdash{}{0pt}%
\pgfpathmoveto{\pgfqpoint{1.478974in}{0.939886in}}%
\pgfpathcurveto{\pgfqpoint{1.484499in}{0.939886in}}{\pgfqpoint{1.489799in}{0.942081in}}{\pgfqpoint{1.493705in}{0.945988in}}%
\pgfpathcurveto{\pgfqpoint{1.497612in}{0.949895in}}{\pgfqpoint{1.499807in}{0.955195in}}{\pgfqpoint{1.499807in}{0.960720in}}%
\pgfpathcurveto{\pgfqpoint{1.499807in}{0.966245in}}{\pgfqpoint{1.497612in}{0.971544in}}{\pgfqpoint{1.493705in}{0.975451in}}%
\pgfpathcurveto{\pgfqpoint{1.489799in}{0.979358in}}{\pgfqpoint{1.484499in}{0.981553in}}{\pgfqpoint{1.478974in}{0.981553in}}%
\pgfpathcurveto{\pgfqpoint{1.473449in}{0.981553in}}{\pgfqpoint{1.468149in}{0.979358in}}{\pgfqpoint{1.464243in}{0.975451in}}%
\pgfpathcurveto{\pgfqpoint{1.460336in}{0.971544in}}{\pgfqpoint{1.458141in}{0.966245in}}{\pgfqpoint{1.458141in}{0.960720in}}%
\pgfpathcurveto{\pgfqpoint{1.458141in}{0.955195in}}{\pgfqpoint{1.460336in}{0.949895in}}{\pgfqpoint{1.464243in}{0.945988in}}%
\pgfpathcurveto{\pgfqpoint{1.468149in}{0.942081in}}{\pgfqpoint{1.473449in}{0.939886in}}{\pgfqpoint{1.478974in}{0.939886in}}%
\pgfpathclose%
\pgfusepath{fill}%
\end{pgfscope}%
\begin{pgfscope}%
\pgfpathrectangle{\pgfqpoint{0.889225in}{0.913832in}}{\pgfqpoint{1.162500in}{0.755000in}}%
\pgfusepath{clip}%
\pgfsetbuttcap%
\pgfsetroundjoin%
\definecolor{currentfill}{rgb}{0.000000,0.000000,0.000000}%
\pgfsetfillcolor{currentfill}%
\pgfsetfillopacity{0.500000}%
\pgfsetlinewidth{0.000000pt}%
\definecolor{currentstroke}{rgb}{0.000000,0.000000,0.000000}%
\pgfsetstrokecolor{currentstroke}%
\pgfsetdash{}{0pt}%
\pgfpathmoveto{\pgfqpoint{1.120714in}{1.105715in}}%
\pgfpathcurveto{\pgfqpoint{1.126239in}{1.105715in}}{\pgfqpoint{1.131538in}{1.107910in}}{\pgfqpoint{1.135445in}{1.111817in}}%
\pgfpathcurveto{\pgfqpoint{1.139352in}{1.115723in}}{\pgfqpoint{1.141547in}{1.121023in}}{\pgfqpoint{1.141547in}{1.126548in}}%
\pgfpathcurveto{\pgfqpoint{1.141547in}{1.132073in}}{\pgfqpoint{1.139352in}{1.137372in}}{\pgfqpoint{1.135445in}{1.141279in}}%
\pgfpathcurveto{\pgfqpoint{1.131538in}{1.145186in}}{\pgfqpoint{1.126239in}{1.147381in}}{\pgfqpoint{1.120714in}{1.147381in}}%
\pgfpathcurveto{\pgfqpoint{1.115189in}{1.147381in}}{\pgfqpoint{1.109889in}{1.145186in}}{\pgfqpoint{1.105982in}{1.141279in}}%
\pgfpathcurveto{\pgfqpoint{1.102076in}{1.137372in}}{\pgfqpoint{1.099881in}{1.132073in}}{\pgfqpoint{1.099881in}{1.126548in}}%
\pgfpathcurveto{\pgfqpoint{1.099881in}{1.121023in}}{\pgfqpoint{1.102076in}{1.115723in}}{\pgfqpoint{1.105982in}{1.111817in}}%
\pgfpathcurveto{\pgfqpoint{1.109889in}{1.107910in}}{\pgfqpoint{1.115189in}{1.105715in}}{\pgfqpoint{1.120714in}{1.105715in}}%
\pgfpathclose%
\pgfusepath{fill}%
\end{pgfscope}%
\begin{pgfscope}%
\pgfpathrectangle{\pgfqpoint{0.889225in}{0.913832in}}{\pgfqpoint{1.162500in}{0.755000in}}%
\pgfusepath{clip}%
\pgfsetbuttcap%
\pgfsetroundjoin%
\definecolor{currentfill}{rgb}{0.000000,0.000000,0.000000}%
\pgfsetfillcolor{currentfill}%
\pgfsetfillopacity{0.500000}%
\pgfsetlinewidth{0.000000pt}%
\definecolor{currentstroke}{rgb}{0.000000,0.000000,0.000000}%
\pgfsetstrokecolor{currentstroke}%
\pgfsetdash{}{0pt}%
\pgfpathmoveto{\pgfqpoint{1.126248in}{0.910975in}}%
\pgfpathcurveto{\pgfqpoint{1.131773in}{0.910975in}}{\pgfqpoint{1.137073in}{0.913170in}}{\pgfqpoint{1.140980in}{0.917077in}}%
\pgfpathcurveto{\pgfqpoint{1.144886in}{0.920984in}}{\pgfqpoint{1.147082in}{0.926283in}}{\pgfqpoint{1.147082in}{0.931809in}}%
\pgfpathcurveto{\pgfqpoint{1.147082in}{0.937334in}}{\pgfqpoint{1.144886in}{0.942633in}}{\pgfqpoint{1.140980in}{0.946540in}}%
\pgfpathcurveto{\pgfqpoint{1.137073in}{0.950447in}}{\pgfqpoint{1.131773in}{0.952642in}}{\pgfqpoint{1.126248in}{0.952642in}}%
\pgfpathcurveto{\pgfqpoint{1.120723in}{0.952642in}}{\pgfqpoint{1.115424in}{0.950447in}}{\pgfqpoint{1.111517in}{0.946540in}}%
\pgfpathcurveto{\pgfqpoint{1.107610in}{0.942633in}}{\pgfqpoint{1.105415in}{0.937334in}}{\pgfqpoint{1.105415in}{0.931809in}}%
\pgfpathcurveto{\pgfqpoint{1.105415in}{0.926283in}}{\pgfqpoint{1.107610in}{0.920984in}}{\pgfqpoint{1.111517in}{0.917077in}}%
\pgfpathcurveto{\pgfqpoint{1.115424in}{0.913170in}}{\pgfqpoint{1.120723in}{0.910975in}}{\pgfqpoint{1.126248in}{0.910975in}}%
\pgfpathclose%
\pgfusepath{fill}%
\end{pgfscope}%
\begin{pgfscope}%
\pgfpathrectangle{\pgfqpoint{0.889225in}{0.913832in}}{\pgfqpoint{1.162500in}{0.755000in}}%
\pgfusepath{clip}%
\pgfsetbuttcap%
\pgfsetroundjoin%
\definecolor{currentfill}{rgb}{0.000000,0.000000,0.000000}%
\pgfsetfillcolor{currentfill}%
\pgfsetfillopacity{0.500000}%
\pgfsetlinewidth{0.000000pt}%
\definecolor{currentstroke}{rgb}{0.000000,0.000000,0.000000}%
\pgfsetstrokecolor{currentstroke}%
\pgfsetdash{}{0pt}%
\pgfpathmoveto{\pgfqpoint{0.977704in}{1.042931in}}%
\pgfpathcurveto{\pgfqpoint{0.983229in}{1.042931in}}{\pgfqpoint{0.988528in}{1.045126in}}{\pgfqpoint{0.992435in}{1.049033in}}%
\pgfpathcurveto{\pgfqpoint{0.996342in}{1.052940in}}{\pgfqpoint{0.998537in}{1.058239in}}{\pgfqpoint{0.998537in}{1.063764in}}%
\pgfpathcurveto{\pgfqpoint{0.998537in}{1.069289in}}{\pgfqpoint{0.996342in}{1.074589in}}{\pgfqpoint{0.992435in}{1.078495in}}%
\pgfpathcurveto{\pgfqpoint{0.988528in}{1.082402in}}{\pgfqpoint{0.983229in}{1.084597in}}{\pgfqpoint{0.977704in}{1.084597in}}%
\pgfpathcurveto{\pgfqpoint{0.972179in}{1.084597in}}{\pgfqpoint{0.966879in}{1.082402in}}{\pgfqpoint{0.962972in}{1.078495in}}%
\pgfpathcurveto{\pgfqpoint{0.959066in}{1.074589in}}{\pgfqpoint{0.956870in}{1.069289in}}{\pgfqpoint{0.956870in}{1.063764in}}%
\pgfpathcurveto{\pgfqpoint{0.956870in}{1.058239in}}{\pgfqpoint{0.959066in}{1.052940in}}{\pgfqpoint{0.962972in}{1.049033in}}%
\pgfpathcurveto{\pgfqpoint{0.966879in}{1.045126in}}{\pgfqpoint{0.972179in}{1.042931in}}{\pgfqpoint{0.977704in}{1.042931in}}%
\pgfpathclose%
\pgfusepath{fill}%
\end{pgfscope}%
\begin{pgfscope}%
\pgfpathrectangle{\pgfqpoint{0.889225in}{0.913832in}}{\pgfqpoint{1.162500in}{0.755000in}}%
\pgfusepath{clip}%
\pgfsetbuttcap%
\pgfsetroundjoin%
\definecolor{currentfill}{rgb}{0.000000,0.000000,0.000000}%
\pgfsetfillcolor{currentfill}%
\pgfsetfillopacity{0.500000}%
\pgfsetlinewidth{0.000000pt}%
\definecolor{currentstroke}{rgb}{0.000000,0.000000,0.000000}%
\pgfsetstrokecolor{currentstroke}%
\pgfsetdash{}{0pt}%
\pgfpathmoveto{\pgfqpoint{0.916904in}{1.035607in}}%
\pgfpathcurveto{\pgfqpoint{0.922429in}{1.035607in}}{\pgfqpoint{0.927728in}{1.037803in}}{\pgfqpoint{0.931635in}{1.041709in}}%
\pgfpathcurveto{\pgfqpoint{0.935542in}{1.045616in}}{\pgfqpoint{0.937737in}{1.050916in}}{\pgfqpoint{0.937737in}{1.056441in}}%
\pgfpathcurveto{\pgfqpoint{0.937737in}{1.061966in}}{\pgfqpoint{0.935542in}{1.067265in}}{\pgfqpoint{0.931635in}{1.071172in}}%
\pgfpathcurveto{\pgfqpoint{0.927728in}{1.075079in}}{\pgfqpoint{0.922429in}{1.077274in}}{\pgfqpoint{0.916904in}{1.077274in}}%
\pgfpathcurveto{\pgfqpoint{0.911379in}{1.077274in}}{\pgfqpoint{0.906079in}{1.075079in}}{\pgfqpoint{0.902172in}{1.071172in}}%
\pgfpathcurveto{\pgfqpoint{0.898265in}{1.067265in}}{\pgfqpoint{0.896070in}{1.061966in}}{\pgfqpoint{0.896070in}{1.056441in}}%
\pgfpathcurveto{\pgfqpoint{0.896070in}{1.050916in}}{\pgfqpoint{0.898265in}{1.045616in}}{\pgfqpoint{0.902172in}{1.041709in}}%
\pgfpathcurveto{\pgfqpoint{0.906079in}{1.037803in}}{\pgfqpoint{0.911379in}{1.035607in}}{\pgfqpoint{0.916904in}{1.035607in}}%
\pgfpathclose%
\pgfusepath{fill}%
\end{pgfscope}%
\begin{pgfscope}%
\pgfpathrectangle{\pgfqpoint{0.889225in}{0.913832in}}{\pgfqpoint{1.162500in}{0.755000in}}%
\pgfusepath{clip}%
\pgfsetbuttcap%
\pgfsetroundjoin%
\definecolor{currentfill}{rgb}{0.000000,0.000000,0.000000}%
\pgfsetfillcolor{currentfill}%
\pgfsetfillopacity{0.500000}%
\pgfsetlinewidth{0.000000pt}%
\definecolor{currentstroke}{rgb}{0.000000,0.000000,0.000000}%
\pgfsetstrokecolor{currentstroke}%
\pgfsetdash{}{0pt}%
\pgfpathmoveto{\pgfqpoint{1.691400in}{1.572595in}}%
\pgfpathcurveto{\pgfqpoint{1.696925in}{1.572595in}}{\pgfqpoint{1.702225in}{1.574791in}}{\pgfqpoint{1.706132in}{1.578697in}}%
\pgfpathcurveto{\pgfqpoint{1.710038in}{1.582604in}}{\pgfqpoint{1.712234in}{1.587904in}}{\pgfqpoint{1.712234in}{1.593429in}}%
\pgfpathcurveto{\pgfqpoint{1.712234in}{1.598954in}}{\pgfqpoint{1.710038in}{1.604253in}}{\pgfqpoint{1.706132in}{1.608160in}}%
\pgfpathcurveto{\pgfqpoint{1.702225in}{1.612067in}}{\pgfqpoint{1.696925in}{1.614262in}}{\pgfqpoint{1.691400in}{1.614262in}}%
\pgfpathcurveto{\pgfqpoint{1.685875in}{1.614262in}}{\pgfqpoint{1.680576in}{1.612067in}}{\pgfqpoint{1.676669in}{1.608160in}}%
\pgfpathcurveto{\pgfqpoint{1.672762in}{1.604253in}}{\pgfqpoint{1.670567in}{1.598954in}}{\pgfqpoint{1.670567in}{1.593429in}}%
\pgfpathcurveto{\pgfqpoint{1.670567in}{1.587904in}}{\pgfqpoint{1.672762in}{1.582604in}}{\pgfqpoint{1.676669in}{1.578697in}}%
\pgfpathcurveto{\pgfqpoint{1.680576in}{1.574791in}}{\pgfqpoint{1.685875in}{1.572595in}}{\pgfqpoint{1.691400in}{1.572595in}}%
\pgfpathclose%
\pgfusepath{fill}%
\end{pgfscope}%
\begin{pgfscope}%
\pgfpathrectangle{\pgfqpoint{0.889225in}{0.913832in}}{\pgfqpoint{1.162500in}{0.755000in}}%
\pgfusepath{clip}%
\pgfsetbuttcap%
\pgfsetroundjoin%
\definecolor{currentfill}{rgb}{0.000000,0.000000,0.000000}%
\pgfsetfillcolor{currentfill}%
\pgfsetfillopacity{0.500000}%
\pgfsetlinewidth{0.000000pt}%
\definecolor{currentstroke}{rgb}{0.000000,0.000000,0.000000}%
\pgfsetstrokecolor{currentstroke}%
\pgfsetdash{}{0pt}%
\pgfpathmoveto{\pgfqpoint{1.460463in}{1.630023in}}%
\pgfpathcurveto{\pgfqpoint{1.465988in}{1.630023in}}{\pgfqpoint{1.471288in}{1.632218in}}{\pgfqpoint{1.475194in}{1.636125in}}%
\pgfpathcurveto{\pgfqpoint{1.479101in}{1.640032in}}{\pgfqpoint{1.481296in}{1.645331in}}{\pgfqpoint{1.481296in}{1.650856in}}%
\pgfpathcurveto{\pgfqpoint{1.481296in}{1.656381in}}{\pgfqpoint{1.479101in}{1.661681in}}{\pgfqpoint{1.475194in}{1.665588in}}%
\pgfpathcurveto{\pgfqpoint{1.471288in}{1.669494in}}{\pgfqpoint{1.465988in}{1.671690in}}{\pgfqpoint{1.460463in}{1.671690in}}%
\pgfpathcurveto{\pgfqpoint{1.454938in}{1.671690in}}{\pgfqpoint{1.449638in}{1.669494in}}{\pgfqpoint{1.445732in}{1.665588in}}%
\pgfpathcurveto{\pgfqpoint{1.441825in}{1.661681in}}{\pgfqpoint{1.439630in}{1.656381in}}{\pgfqpoint{1.439630in}{1.650856in}}%
\pgfpathcurveto{\pgfqpoint{1.439630in}{1.645331in}}{\pgfqpoint{1.441825in}{1.640032in}}{\pgfqpoint{1.445732in}{1.636125in}}%
\pgfpathcurveto{\pgfqpoint{1.449638in}{1.632218in}}{\pgfqpoint{1.454938in}{1.630023in}}{\pgfqpoint{1.460463in}{1.630023in}}%
\pgfpathclose%
\pgfusepath{fill}%
\end{pgfscope}%
\begin{pgfscope}%
\pgfpathrectangle{\pgfqpoint{0.889225in}{0.913832in}}{\pgfqpoint{1.162500in}{0.755000in}}%
\pgfusepath{clip}%
\pgfsetbuttcap%
\pgfsetroundjoin%
\definecolor{currentfill}{rgb}{0.000000,0.000000,0.000000}%
\pgfsetfillcolor{currentfill}%
\pgfsetfillopacity{0.500000}%
\pgfsetlinewidth{0.000000pt}%
\definecolor{currentstroke}{rgb}{0.000000,0.000000,0.000000}%
\pgfsetstrokecolor{currentstroke}%
\pgfsetdash{}{0pt}%
\pgfpathmoveto{\pgfqpoint{1.303193in}{1.484185in}}%
\pgfpathcurveto{\pgfqpoint{1.308718in}{1.484185in}}{\pgfqpoint{1.314017in}{1.486380in}}{\pgfqpoint{1.317924in}{1.490287in}}%
\pgfpathcurveto{\pgfqpoint{1.321831in}{1.494194in}}{\pgfqpoint{1.324026in}{1.499493in}}{\pgfqpoint{1.324026in}{1.505018in}}%
\pgfpathcurveto{\pgfqpoint{1.324026in}{1.510544in}}{\pgfqpoint{1.321831in}{1.515843in}}{\pgfqpoint{1.317924in}{1.519750in}}%
\pgfpathcurveto{\pgfqpoint{1.314017in}{1.523657in}}{\pgfqpoint{1.308718in}{1.525852in}}{\pgfqpoint{1.303193in}{1.525852in}}%
\pgfpathcurveto{\pgfqpoint{1.297667in}{1.525852in}}{\pgfqpoint{1.292368in}{1.523657in}}{\pgfqpoint{1.288461in}{1.519750in}}%
\pgfpathcurveto{\pgfqpoint{1.284554in}{1.515843in}}{\pgfqpoint{1.282359in}{1.510544in}}{\pgfqpoint{1.282359in}{1.505018in}}%
\pgfpathcurveto{\pgfqpoint{1.282359in}{1.499493in}}{\pgfqpoint{1.284554in}{1.494194in}}{\pgfqpoint{1.288461in}{1.490287in}}%
\pgfpathcurveto{\pgfqpoint{1.292368in}{1.486380in}}{\pgfqpoint{1.297667in}{1.484185in}}{\pgfqpoint{1.303193in}{1.484185in}}%
\pgfpathclose%
\pgfusepath{fill}%
\end{pgfscope}%
\begin{pgfscope}%
\pgfpathrectangle{\pgfqpoint{0.889225in}{0.913832in}}{\pgfqpoint{1.162500in}{0.755000in}}%
\pgfusepath{clip}%
\pgfsetbuttcap%
\pgfsetroundjoin%
\definecolor{currentfill}{rgb}{0.000000,0.000000,0.000000}%
\pgfsetfillcolor{currentfill}%
\pgfsetfillopacity{0.500000}%
\pgfsetlinewidth{0.000000pt}%
\definecolor{currentstroke}{rgb}{0.000000,0.000000,0.000000}%
\pgfsetstrokecolor{currentstroke}%
\pgfsetdash{}{0pt}%
\pgfpathmoveto{\pgfqpoint{1.689350in}{1.067768in}}%
\pgfpathcurveto{\pgfqpoint{1.694875in}{1.067768in}}{\pgfqpoint{1.700175in}{1.069963in}}{\pgfqpoint{1.704082in}{1.073870in}}%
\pgfpathcurveto{\pgfqpoint{1.707989in}{1.077776in}}{\pgfqpoint{1.710184in}{1.083076in}}{\pgfqpoint{1.710184in}{1.088601in}}%
\pgfpathcurveto{\pgfqpoint{1.710184in}{1.094126in}}{\pgfqpoint{1.707989in}{1.099426in}}{\pgfqpoint{1.704082in}{1.103332in}}%
\pgfpathcurveto{\pgfqpoint{1.700175in}{1.107239in}}{\pgfqpoint{1.694875in}{1.109434in}}{\pgfqpoint{1.689350in}{1.109434in}}%
\pgfpathcurveto{\pgfqpoint{1.683825in}{1.109434in}}{\pgfqpoint{1.678526in}{1.107239in}}{\pgfqpoint{1.674619in}{1.103332in}}%
\pgfpathcurveto{\pgfqpoint{1.670712in}{1.099426in}}{\pgfqpoint{1.668517in}{1.094126in}}{\pgfqpoint{1.668517in}{1.088601in}}%
\pgfpathcurveto{\pgfqpoint{1.668517in}{1.083076in}}{\pgfqpoint{1.670712in}{1.077776in}}{\pgfqpoint{1.674619in}{1.073870in}}%
\pgfpathcurveto{\pgfqpoint{1.678526in}{1.069963in}}{\pgfqpoint{1.683825in}{1.067768in}}{\pgfqpoint{1.689350in}{1.067768in}}%
\pgfpathclose%
\pgfusepath{fill}%
\end{pgfscope}%
\begin{pgfscope}%
\pgfpathrectangle{\pgfqpoint{0.889225in}{0.913832in}}{\pgfqpoint{1.162500in}{0.755000in}}%
\pgfusepath{clip}%
\pgfsetbuttcap%
\pgfsetroundjoin%
\definecolor{currentfill}{rgb}{0.000000,0.000000,0.000000}%
\pgfsetfillcolor{currentfill}%
\pgfsetfillopacity{0.500000}%
\pgfsetlinewidth{0.000000pt}%
\definecolor{currentstroke}{rgb}{0.000000,0.000000,0.000000}%
\pgfsetstrokecolor{currentstroke}%
\pgfsetdash{}{0pt}%
\pgfpathmoveto{\pgfqpoint{1.102396in}{1.343150in}}%
\pgfpathcurveto{\pgfqpoint{1.107921in}{1.343150in}}{\pgfqpoint{1.113221in}{1.345345in}}{\pgfqpoint{1.117128in}{1.349252in}}%
\pgfpathcurveto{\pgfqpoint{1.121035in}{1.353159in}}{\pgfqpoint{1.123230in}{1.358458in}}{\pgfqpoint{1.123230in}{1.363984in}}%
\pgfpathcurveto{\pgfqpoint{1.123230in}{1.369509in}}{\pgfqpoint{1.121035in}{1.374808in}}{\pgfqpoint{1.117128in}{1.378715in}}%
\pgfpathcurveto{\pgfqpoint{1.113221in}{1.382622in}}{\pgfqpoint{1.107921in}{1.384817in}}{\pgfqpoint{1.102396in}{1.384817in}}%
\pgfpathcurveto{\pgfqpoint{1.096871in}{1.384817in}}{\pgfqpoint{1.091572in}{1.382622in}}{\pgfqpoint{1.087665in}{1.378715in}}%
\pgfpathcurveto{\pgfqpoint{1.083758in}{1.374808in}}{\pgfqpoint{1.081563in}{1.369509in}}{\pgfqpoint{1.081563in}{1.363984in}}%
\pgfpathcurveto{\pgfqpoint{1.081563in}{1.358458in}}{\pgfqpoint{1.083758in}{1.353159in}}{\pgfqpoint{1.087665in}{1.349252in}}%
\pgfpathcurveto{\pgfqpoint{1.091572in}{1.345345in}}{\pgfqpoint{1.096871in}{1.343150in}}{\pgfqpoint{1.102396in}{1.343150in}}%
\pgfpathclose%
\pgfusepath{fill}%
\end{pgfscope}%
\begin{pgfscope}%
\pgfpathrectangle{\pgfqpoint{0.889225in}{0.913832in}}{\pgfqpoint{1.162500in}{0.755000in}}%
\pgfusepath{clip}%
\pgfsetbuttcap%
\pgfsetroundjoin%
\definecolor{currentfill}{rgb}{0.000000,0.000000,0.000000}%
\pgfsetfillcolor{currentfill}%
\pgfsetfillopacity{0.500000}%
\pgfsetlinewidth{0.000000pt}%
\definecolor{currentstroke}{rgb}{0.000000,0.000000,0.000000}%
\pgfsetstrokecolor{currentstroke}%
\pgfsetdash{}{0pt}%
\pgfpathmoveto{\pgfqpoint{1.143186in}{1.259718in}}%
\pgfpathcurveto{\pgfqpoint{1.148711in}{1.259718in}}{\pgfqpoint{1.154011in}{1.261913in}}{\pgfqpoint{1.157918in}{1.265820in}}%
\pgfpathcurveto{\pgfqpoint{1.161825in}{1.269726in}}{\pgfqpoint{1.164020in}{1.275026in}}{\pgfqpoint{1.164020in}{1.280551in}}%
\pgfpathcurveto{\pgfqpoint{1.164020in}{1.286076in}}{\pgfqpoint{1.161825in}{1.291376in}}{\pgfqpoint{1.157918in}{1.295282in}}%
\pgfpathcurveto{\pgfqpoint{1.154011in}{1.299189in}}{\pgfqpoint{1.148711in}{1.301384in}}{\pgfqpoint{1.143186in}{1.301384in}}%
\pgfpathcurveto{\pgfqpoint{1.137661in}{1.301384in}}{\pgfqpoint{1.132362in}{1.299189in}}{\pgfqpoint{1.128455in}{1.295282in}}%
\pgfpathcurveto{\pgfqpoint{1.124548in}{1.291376in}}{\pgfqpoint{1.122353in}{1.286076in}}{\pgfqpoint{1.122353in}{1.280551in}}%
\pgfpathcurveto{\pgfqpoint{1.122353in}{1.275026in}}{\pgfqpoint{1.124548in}{1.269726in}}{\pgfqpoint{1.128455in}{1.265820in}}%
\pgfpathcurveto{\pgfqpoint{1.132362in}{1.261913in}}{\pgfqpoint{1.137661in}{1.259718in}}{\pgfqpoint{1.143186in}{1.259718in}}%
\pgfpathclose%
\pgfusepath{fill}%
\end{pgfscope}%
\begin{pgfscope}%
\pgfpathrectangle{\pgfqpoint{0.889225in}{0.913832in}}{\pgfqpoint{1.162500in}{0.755000in}}%
\pgfusepath{clip}%
\pgfsetbuttcap%
\pgfsetroundjoin%
\definecolor{currentfill}{rgb}{0.000000,0.000000,0.000000}%
\pgfsetfillcolor{currentfill}%
\pgfsetfillopacity{0.500000}%
\pgfsetlinewidth{0.000000pt}%
\definecolor{currentstroke}{rgb}{0.000000,0.000000,0.000000}%
\pgfsetstrokecolor{currentstroke}%
\pgfsetdash{}{0pt}%
\pgfpathmoveto{\pgfqpoint{2.024046in}{1.167350in}}%
\pgfpathcurveto{\pgfqpoint{2.029571in}{1.167350in}}{\pgfqpoint{2.034871in}{1.169545in}}{\pgfqpoint{2.038778in}{1.173452in}}%
\pgfpathcurveto{\pgfqpoint{2.042685in}{1.177359in}}{\pgfqpoint{2.044880in}{1.182658in}}{\pgfqpoint{2.044880in}{1.188183in}}%
\pgfpathcurveto{\pgfqpoint{2.044880in}{1.193708in}}{\pgfqpoint{2.042685in}{1.199008in}}{\pgfqpoint{2.038778in}{1.202915in}}%
\pgfpathcurveto{\pgfqpoint{2.034871in}{1.206822in}}{\pgfqpoint{2.029571in}{1.209017in}}{\pgfqpoint{2.024046in}{1.209017in}}%
\pgfpathcurveto{\pgfqpoint{2.018521in}{1.209017in}}{\pgfqpoint{2.013222in}{1.206822in}}{\pgfqpoint{2.009315in}{1.202915in}}%
\pgfpathcurveto{\pgfqpoint{2.005408in}{1.199008in}}{\pgfqpoint{2.003213in}{1.193708in}}{\pgfqpoint{2.003213in}{1.188183in}}%
\pgfpathcurveto{\pgfqpoint{2.003213in}{1.182658in}}{\pgfqpoint{2.005408in}{1.177359in}}{\pgfqpoint{2.009315in}{1.173452in}}%
\pgfpathcurveto{\pgfqpoint{2.013222in}{1.169545in}}{\pgfqpoint{2.018521in}{1.167350in}}{\pgfqpoint{2.024046in}{1.167350in}}%
\pgfpathclose%
\pgfusepath{fill}%
\end{pgfscope}%
\begin{pgfscope}%
\pgfpathrectangle{\pgfqpoint{0.889225in}{0.913832in}}{\pgfqpoint{1.162500in}{0.755000in}}%
\pgfusepath{clip}%
\pgfsetbuttcap%
\pgfsetroundjoin%
\definecolor{currentfill}{rgb}{0.000000,0.000000,0.000000}%
\pgfsetfillcolor{currentfill}%
\pgfsetfillopacity{0.500000}%
\pgfsetlinewidth{0.000000pt}%
\definecolor{currentstroke}{rgb}{0.000000,0.000000,0.000000}%
\pgfsetstrokecolor{currentstroke}%
\pgfsetdash{}{0pt}%
\pgfpathmoveto{\pgfqpoint{1.402285in}{1.034902in}}%
\pgfpathcurveto{\pgfqpoint{1.407810in}{1.034902in}}{\pgfqpoint{1.413110in}{1.037097in}}{\pgfqpoint{1.417016in}{1.041004in}}%
\pgfpathcurveto{\pgfqpoint{1.420923in}{1.044911in}}{\pgfqpoint{1.423118in}{1.050211in}}{\pgfqpoint{1.423118in}{1.055736in}}%
\pgfpathcurveto{\pgfqpoint{1.423118in}{1.061261in}}{\pgfqpoint{1.420923in}{1.066560in}}{\pgfqpoint{1.417016in}{1.070467in}}%
\pgfpathcurveto{\pgfqpoint{1.413110in}{1.074374in}}{\pgfqpoint{1.407810in}{1.076569in}}{\pgfqpoint{1.402285in}{1.076569in}}%
\pgfpathcurveto{\pgfqpoint{1.396760in}{1.076569in}}{\pgfqpoint{1.391460in}{1.074374in}}{\pgfqpoint{1.387554in}{1.070467in}}%
\pgfpathcurveto{\pgfqpoint{1.383647in}{1.066560in}}{\pgfqpoint{1.381452in}{1.061261in}}{\pgfqpoint{1.381452in}{1.055736in}}%
\pgfpathcurveto{\pgfqpoint{1.381452in}{1.050211in}}{\pgfqpoint{1.383647in}{1.044911in}}{\pgfqpoint{1.387554in}{1.041004in}}%
\pgfpathcurveto{\pgfqpoint{1.391460in}{1.037097in}}{\pgfqpoint{1.396760in}{1.034902in}}{\pgfqpoint{1.402285in}{1.034902in}}%
\pgfpathclose%
\pgfusepath{fill}%
\end{pgfscope}%
\begin{pgfscope}%
\pgfpathrectangle{\pgfqpoint{0.889225in}{0.913832in}}{\pgfqpoint{1.162500in}{0.755000in}}%
\pgfusepath{clip}%
\pgfsetbuttcap%
\pgfsetroundjoin%
\definecolor{currentfill}{rgb}{0.000000,0.000000,0.000000}%
\pgfsetfillcolor{currentfill}%
\pgfsetfillopacity{0.500000}%
\pgfsetlinewidth{0.000000pt}%
\definecolor{currentstroke}{rgb}{0.000000,0.000000,0.000000}%
\pgfsetstrokecolor{currentstroke}%
\pgfsetdash{}{0pt}%
\pgfpathmoveto{\pgfqpoint{0.973483in}{1.190157in}}%
\pgfpathcurveto{\pgfqpoint{0.979008in}{1.190157in}}{\pgfqpoint{0.984308in}{1.192352in}}{\pgfqpoint{0.988214in}{1.196259in}}%
\pgfpathcurveto{\pgfqpoint{0.992121in}{1.200166in}}{\pgfqpoint{0.994316in}{1.205465in}}{\pgfqpoint{0.994316in}{1.210991in}}%
\pgfpathcurveto{\pgfqpoint{0.994316in}{1.216516in}}{\pgfqpoint{0.992121in}{1.221815in}}{\pgfqpoint{0.988214in}{1.225722in}}%
\pgfpathcurveto{\pgfqpoint{0.984308in}{1.229629in}}{\pgfqpoint{0.979008in}{1.231824in}}{\pgfqpoint{0.973483in}{1.231824in}}%
\pgfpathcurveto{\pgfqpoint{0.967958in}{1.231824in}}{\pgfqpoint{0.962658in}{1.229629in}}{\pgfqpoint{0.958752in}{1.225722in}}%
\pgfpathcurveto{\pgfqpoint{0.954845in}{1.221815in}}{\pgfqpoint{0.952650in}{1.216516in}}{\pgfqpoint{0.952650in}{1.210991in}}%
\pgfpathcurveto{\pgfqpoint{0.952650in}{1.205465in}}{\pgfqpoint{0.954845in}{1.200166in}}{\pgfqpoint{0.958752in}{1.196259in}}%
\pgfpathcurveto{\pgfqpoint{0.962658in}{1.192352in}}{\pgfqpoint{0.967958in}{1.190157in}}{\pgfqpoint{0.973483in}{1.190157in}}%
\pgfpathclose%
\pgfusepath{fill}%
\end{pgfscope}%
\begin{pgfscope}%
\pgfsetbuttcap%
\pgfsetroundjoin%
\definecolor{currentfill}{rgb}{0.000000,0.000000,0.000000}%
\pgfsetfillcolor{currentfill}%
\pgfsetlinewidth{0.803000pt}%
\definecolor{currentstroke}{rgb}{0.000000,0.000000,0.000000}%
\pgfsetstrokecolor{currentstroke}%
\pgfsetdash{}{0pt}%
\pgfsys@defobject{currentmarker}{\pgfqpoint{0.000000in}{-0.048611in}}{\pgfqpoint{0.000000in}{0.000000in}}{%
\pgfpathmoveto{\pgfqpoint{0.000000in}{0.000000in}}%
\pgfpathlineto{\pgfqpoint{0.000000in}{-0.048611in}}%
\pgfusepath{stroke,fill}%
}%
\begin{pgfscope}%
\pgfsys@transformshift{1.236821in}{0.913832in}%
\pgfsys@useobject{currentmarker}{}%
\end{pgfscope}%
\end{pgfscope}%
\begin{pgfscope}%
\pgftext[x=1.267474in,y=0.370616in,left,base,rotate=90.000000]{\rmfamily\fontsize{8.000000}{9.600000}\selectfont \(\displaystyle 0.000025\)}%
\end{pgfscope}%
\begin{pgfscope}%
\pgfsetbuttcap%
\pgfsetroundjoin%
\definecolor{currentfill}{rgb}{0.000000,0.000000,0.000000}%
\pgfsetfillcolor{currentfill}%
\pgfsetlinewidth{0.803000pt}%
\definecolor{currentstroke}{rgb}{0.000000,0.000000,0.000000}%
\pgfsetstrokecolor{currentstroke}%
\pgfsetdash{}{0pt}%
\pgfsys@defobject{currentmarker}{\pgfqpoint{0.000000in}{-0.048611in}}{\pgfqpoint{0.000000in}{0.000000in}}{%
\pgfpathmoveto{\pgfqpoint{0.000000in}{0.000000in}}%
\pgfpathlineto{\pgfqpoint{0.000000in}{-0.048611in}}%
\pgfusepath{stroke,fill}%
}%
\begin{pgfscope}%
\pgfsys@transformshift{1.758350in}{0.913832in}%
\pgfsys@useobject{currentmarker}{}%
\end{pgfscope}%
\end{pgfscope}%
\begin{pgfscope}%
\pgftext[x=1.789004in,y=0.370616in,left,base,rotate=90.000000]{\rmfamily\fontsize{8.000000}{9.600000}\selectfont \(\displaystyle 0.000050\)}%
\end{pgfscope}%
\begin{pgfscope}%
\pgftext[x=1.470475in,y=0.315061in,,top]{\rmfamily\fontsize{16.000000}{19.200000}\selectfont area}%
\end{pgfscope}%
\begin{pgfscope}%
\pgfsetbuttcap%
\pgfsetroundjoin%
\definecolor{currentfill}{rgb}{0.000000,0.000000,0.000000}%
\pgfsetfillcolor{currentfill}%
\pgfsetlinewidth{0.803000pt}%
\definecolor{currentstroke}{rgb}{0.000000,0.000000,0.000000}%
\pgfsetstrokecolor{currentstroke}%
\pgfsetdash{}{0pt}%
\pgfsys@defobject{currentmarker}{\pgfqpoint{-0.048611in}{0.000000in}}{\pgfqpoint{0.000000in}{0.000000in}}{%
\pgfpathmoveto{\pgfqpoint{0.000000in}{0.000000in}}%
\pgfpathlineto{\pgfqpoint{-0.048611in}{0.000000in}}%
\pgfusepath{stroke,fill}%
}%
\begin{pgfscope}%
\pgfsys@transformshift{0.889225in}{0.965005in}%
\pgfsys@useobject{currentmarker}{}%
\end{pgfscope}%
\end{pgfscope}%
\begin{pgfscope}%
\pgftext[x=0.641152in,y=0.922796in,left,base]{\rmfamily\fontsize{8.000000}{9.600000}\selectfont \(\displaystyle 0.5\)}%
\end{pgfscope}%
\begin{pgfscope}%
\pgfsetbuttcap%
\pgfsetroundjoin%
\definecolor{currentfill}{rgb}{0.000000,0.000000,0.000000}%
\pgfsetfillcolor{currentfill}%
\pgfsetlinewidth{0.803000pt}%
\definecolor{currentstroke}{rgb}{0.000000,0.000000,0.000000}%
\pgfsetstrokecolor{currentstroke}%
\pgfsetdash{}{0pt}%
\pgfsys@defobject{currentmarker}{\pgfqpoint{-0.048611in}{0.000000in}}{\pgfqpoint{0.000000in}{0.000000in}}{%
\pgfpathmoveto{\pgfqpoint{0.000000in}{0.000000in}}%
\pgfpathlineto{\pgfqpoint{-0.048611in}{0.000000in}}%
\pgfusepath{stroke,fill}%
}%
\begin{pgfscope}%
\pgfsys@transformshift{0.889225in}{1.260040in}%
\pgfsys@useobject{currentmarker}{}%
\end{pgfscope}%
\end{pgfscope}%
\begin{pgfscope}%
\pgftext[x=0.641152in,y=1.217831in,left,base]{\rmfamily\fontsize{8.000000}{9.600000}\selectfont \(\displaystyle 1.0\)}%
\end{pgfscope}%
\begin{pgfscope}%
\pgfsetbuttcap%
\pgfsetroundjoin%
\definecolor{currentfill}{rgb}{0.000000,0.000000,0.000000}%
\pgfsetfillcolor{currentfill}%
\pgfsetlinewidth{0.803000pt}%
\definecolor{currentstroke}{rgb}{0.000000,0.000000,0.000000}%
\pgfsetstrokecolor{currentstroke}%
\pgfsetdash{}{0pt}%
\pgfsys@defobject{currentmarker}{\pgfqpoint{-0.048611in}{0.000000in}}{\pgfqpoint{0.000000in}{0.000000in}}{%
\pgfpathmoveto{\pgfqpoint{0.000000in}{0.000000in}}%
\pgfpathlineto{\pgfqpoint{-0.048611in}{0.000000in}}%
\pgfusepath{stroke,fill}%
}%
\begin{pgfscope}%
\pgfsys@transformshift{0.889225in}{1.555075in}%
\pgfsys@useobject{currentmarker}{}%
\end{pgfscope}%
\end{pgfscope}%
\begin{pgfscope}%
\pgftext[x=0.641152in,y=1.512866in,left,base]{\rmfamily\fontsize{8.000000}{9.600000}\selectfont \(\displaystyle 1.5\)}%
\end{pgfscope}%
\begin{pgfscope}%
\pgftext[x=0.585596in,y=1.291332in,,bottom,rotate=90.000000]{\rmfamily\fontsize{16.000000}{19.200000}\selectfont Ef0}%
\end{pgfscope}%
\begin{pgfscope}%
\pgfsetrectcap%
\pgfsetmiterjoin%
\pgfsetlinewidth{0.803000pt}%
\definecolor{currentstroke}{rgb}{0.501961,0.501961,0.501961}%
\pgfsetstrokecolor{currentstroke}%
\pgfsetdash{}{0pt}%
\pgfpathmoveto{\pgfqpoint{0.889225in}{0.913832in}}%
\pgfpathlineto{\pgfqpoint{0.889225in}{1.668832in}}%
\pgfusepath{stroke}%
\end{pgfscope}%
\begin{pgfscope}%
\pgfsetrectcap%
\pgfsetmiterjoin%
\pgfsetlinewidth{0.803000pt}%
\definecolor{currentstroke}{rgb}{0.501961,0.501961,0.501961}%
\pgfsetstrokecolor{currentstroke}%
\pgfsetdash{}{0pt}%
\pgfpathmoveto{\pgfqpoint{2.051725in}{0.913832in}}%
\pgfpathlineto{\pgfqpoint{2.051725in}{1.668832in}}%
\pgfusepath{stroke}%
\end{pgfscope}%
\begin{pgfscope}%
\pgfsetrectcap%
\pgfsetmiterjoin%
\pgfsetlinewidth{0.803000pt}%
\definecolor{currentstroke}{rgb}{0.501961,0.501961,0.501961}%
\pgfsetstrokecolor{currentstroke}%
\pgfsetdash{}{0pt}%
\pgfpathmoveto{\pgfqpoint{0.889225in}{0.913832in}}%
\pgfpathlineto{\pgfqpoint{2.051725in}{0.913832in}}%
\pgfusepath{stroke}%
\end{pgfscope}%
\begin{pgfscope}%
\pgfsetrectcap%
\pgfsetmiterjoin%
\pgfsetlinewidth{0.803000pt}%
\definecolor{currentstroke}{rgb}{0.501961,0.501961,0.501961}%
\pgfsetstrokecolor{currentstroke}%
\pgfsetdash{}{0pt}%
\pgfpathmoveto{\pgfqpoint{0.889225in}{1.668832in}}%
\pgfpathlineto{\pgfqpoint{2.051725in}{1.668832in}}%
\pgfusepath{stroke}%
\end{pgfscope}%
\begin{pgfscope}%
\pgfsetbuttcap%
\pgfsetmiterjoin%
\definecolor{currentfill}{rgb}{1.000000,1.000000,1.000000}%
\pgfsetfillcolor{currentfill}%
\pgfsetlinewidth{0.000000pt}%
\definecolor{currentstroke}{rgb}{0.000000,0.000000,0.000000}%
\pgfsetstrokecolor{currentstroke}%
\pgfsetstrokeopacity{0.000000}%
\pgfsetdash{}{0pt}%
\pgfpathmoveto{\pgfqpoint{2.051725in}{0.913832in}}%
\pgfpathlineto{\pgfqpoint{3.214225in}{0.913832in}}%
\pgfpathlineto{\pgfqpoint{3.214225in}{1.668832in}}%
\pgfpathlineto{\pgfqpoint{2.051725in}{1.668832in}}%
\pgfpathclose%
\pgfusepath{fill}%
\end{pgfscope}%
\begin{pgfscope}%
\pgfpathrectangle{\pgfqpoint{2.051725in}{0.913832in}}{\pgfqpoint{1.162500in}{0.755000in}}%
\pgfusepath{clip}%
\pgfsetbuttcap%
\pgfsetroundjoin%
\definecolor{currentfill}{rgb}{0.000000,0.000000,0.000000}%
\pgfsetfillcolor{currentfill}%
\pgfsetfillopacity{0.500000}%
\pgfsetlinewidth{0.000000pt}%
\definecolor{currentstroke}{rgb}{0.000000,0.000000,0.000000}%
\pgfsetstrokecolor{currentstroke}%
\pgfsetdash{}{0pt}%
\pgfpathmoveto{\pgfqpoint{3.107569in}{1.437401in}}%
\pgfpathcurveto{\pgfqpoint{3.113094in}{1.437401in}}{\pgfqpoint{3.118394in}{1.439596in}}{\pgfqpoint{3.122301in}{1.443503in}}%
\pgfpathcurveto{\pgfqpoint{3.126207in}{1.447410in}}{\pgfqpoint{3.128403in}{1.452710in}}{\pgfqpoint{3.128403in}{1.458235in}}%
\pgfpathcurveto{\pgfqpoint{3.128403in}{1.463760in}}{\pgfqpoint{3.126207in}{1.469059in}}{\pgfqpoint{3.122301in}{1.472966in}}%
\pgfpathcurveto{\pgfqpoint{3.118394in}{1.476873in}}{\pgfqpoint{3.113094in}{1.479068in}}{\pgfqpoint{3.107569in}{1.479068in}}%
\pgfpathcurveto{\pgfqpoint{3.102044in}{1.479068in}}{\pgfqpoint{3.096745in}{1.476873in}}{\pgfqpoint{3.092838in}{1.472966in}}%
\pgfpathcurveto{\pgfqpoint{3.088931in}{1.469059in}}{\pgfqpoint{3.086736in}{1.463760in}}{\pgfqpoint{3.086736in}{1.458235in}}%
\pgfpathcurveto{\pgfqpoint{3.086736in}{1.452710in}}{\pgfqpoint{3.088931in}{1.447410in}}{\pgfqpoint{3.092838in}{1.443503in}}%
\pgfpathcurveto{\pgfqpoint{3.096745in}{1.439596in}}{\pgfqpoint{3.102044in}{1.437401in}}{\pgfqpoint{3.107569in}{1.437401in}}%
\pgfpathclose%
\pgfusepath{fill}%
\end{pgfscope}%
\begin{pgfscope}%
\pgfpathrectangle{\pgfqpoint{2.051725in}{0.913832in}}{\pgfqpoint{1.162500in}{0.755000in}}%
\pgfusepath{clip}%
\pgfsetbuttcap%
\pgfsetroundjoin%
\definecolor{currentfill}{rgb}{0.000000,0.000000,0.000000}%
\pgfsetfillcolor{currentfill}%
\pgfsetfillopacity{0.500000}%
\pgfsetlinewidth{0.000000pt}%
\definecolor{currentstroke}{rgb}{0.000000,0.000000,0.000000}%
\pgfsetstrokecolor{currentstroke}%
\pgfsetdash{}{0pt}%
\pgfpathmoveto{\pgfqpoint{2.270173in}{0.987180in}}%
\pgfpathcurveto{\pgfqpoint{2.275698in}{0.987180in}}{\pgfqpoint{2.280997in}{0.989375in}}{\pgfqpoint{2.284904in}{0.993282in}}%
\pgfpathcurveto{\pgfqpoint{2.288811in}{0.997189in}}{\pgfqpoint{2.291006in}{1.002488in}}{\pgfqpoint{2.291006in}{1.008013in}}%
\pgfpathcurveto{\pgfqpoint{2.291006in}{1.013538in}}{\pgfqpoint{2.288811in}{1.018838in}}{\pgfqpoint{2.284904in}{1.022745in}}%
\pgfpathcurveto{\pgfqpoint{2.280997in}{1.026652in}}{\pgfqpoint{2.275698in}{1.028847in}}{\pgfqpoint{2.270173in}{1.028847in}}%
\pgfpathcurveto{\pgfqpoint{2.264648in}{1.028847in}}{\pgfqpoint{2.259348in}{1.026652in}}{\pgfqpoint{2.255441in}{1.022745in}}%
\pgfpathcurveto{\pgfqpoint{2.251535in}{1.018838in}}{\pgfqpoint{2.249339in}{1.013538in}}{\pgfqpoint{2.249339in}{1.008013in}}%
\pgfpathcurveto{\pgfqpoint{2.249339in}{1.002488in}}{\pgfqpoint{2.251535in}{0.997189in}}{\pgfqpoint{2.255441in}{0.993282in}}%
\pgfpathcurveto{\pgfqpoint{2.259348in}{0.989375in}}{\pgfqpoint{2.264648in}{0.987180in}}{\pgfqpoint{2.270173in}{0.987180in}}%
\pgfpathclose%
\pgfusepath{fill}%
\end{pgfscope}%
\begin{pgfscope}%
\pgfpathrectangle{\pgfqpoint{2.051725in}{0.913832in}}{\pgfqpoint{1.162500in}{0.755000in}}%
\pgfusepath{clip}%
\pgfsetbuttcap%
\pgfsetroundjoin%
\definecolor{currentfill}{rgb}{0.000000,0.000000,0.000000}%
\pgfsetfillcolor{currentfill}%
\pgfsetfillopacity{0.500000}%
\pgfsetlinewidth{0.000000pt}%
\definecolor{currentstroke}{rgb}{0.000000,0.000000,0.000000}%
\pgfsetstrokecolor{currentstroke}%
\pgfsetdash{}{0pt}%
\pgfpathmoveto{\pgfqpoint{2.264867in}{0.939886in}}%
\pgfpathcurveto{\pgfqpoint{2.270392in}{0.939886in}}{\pgfqpoint{2.275692in}{0.942081in}}{\pgfqpoint{2.279599in}{0.945988in}}%
\pgfpathcurveto{\pgfqpoint{2.283506in}{0.949895in}}{\pgfqpoint{2.285701in}{0.955195in}}{\pgfqpoint{2.285701in}{0.960720in}}%
\pgfpathcurveto{\pgfqpoint{2.285701in}{0.966245in}}{\pgfqpoint{2.283506in}{0.971544in}}{\pgfqpoint{2.279599in}{0.975451in}}%
\pgfpathcurveto{\pgfqpoint{2.275692in}{0.979358in}}{\pgfqpoint{2.270392in}{0.981553in}}{\pgfqpoint{2.264867in}{0.981553in}}%
\pgfpathcurveto{\pgfqpoint{2.259342in}{0.981553in}}{\pgfqpoint{2.254043in}{0.979358in}}{\pgfqpoint{2.250136in}{0.975451in}}%
\pgfpathcurveto{\pgfqpoint{2.246229in}{0.971544in}}{\pgfqpoint{2.244034in}{0.966245in}}{\pgfqpoint{2.244034in}{0.960720in}}%
\pgfpathcurveto{\pgfqpoint{2.244034in}{0.955195in}}{\pgfqpoint{2.246229in}{0.949895in}}{\pgfqpoint{2.250136in}{0.945988in}}%
\pgfpathcurveto{\pgfqpoint{2.254043in}{0.942081in}}{\pgfqpoint{2.259342in}{0.939886in}}{\pgfqpoint{2.264867in}{0.939886in}}%
\pgfpathclose%
\pgfusepath{fill}%
\end{pgfscope}%
\begin{pgfscope}%
\pgfpathrectangle{\pgfqpoint{2.051725in}{0.913832in}}{\pgfqpoint{1.162500in}{0.755000in}}%
\pgfusepath{clip}%
\pgfsetbuttcap%
\pgfsetroundjoin%
\definecolor{currentfill}{rgb}{0.000000,0.000000,0.000000}%
\pgfsetfillcolor{currentfill}%
\pgfsetfillopacity{0.500000}%
\pgfsetlinewidth{0.000000pt}%
\definecolor{currentstroke}{rgb}{0.000000,0.000000,0.000000}%
\pgfsetstrokecolor{currentstroke}%
\pgfsetdash{}{0pt}%
\pgfpathmoveto{\pgfqpoint{2.154193in}{1.105715in}}%
\pgfpathcurveto{\pgfqpoint{2.159718in}{1.105715in}}{\pgfqpoint{2.165018in}{1.107910in}}{\pgfqpoint{2.168924in}{1.111817in}}%
\pgfpathcurveto{\pgfqpoint{2.172831in}{1.115723in}}{\pgfqpoint{2.175026in}{1.121023in}}{\pgfqpoint{2.175026in}{1.126548in}}%
\pgfpathcurveto{\pgfqpoint{2.175026in}{1.132073in}}{\pgfqpoint{2.172831in}{1.137372in}}{\pgfqpoint{2.168924in}{1.141279in}}%
\pgfpathcurveto{\pgfqpoint{2.165018in}{1.145186in}}{\pgfqpoint{2.159718in}{1.147381in}}{\pgfqpoint{2.154193in}{1.147381in}}%
\pgfpathcurveto{\pgfqpoint{2.148668in}{1.147381in}}{\pgfqpoint{2.143368in}{1.145186in}}{\pgfqpoint{2.139462in}{1.141279in}}%
\pgfpathcurveto{\pgfqpoint{2.135555in}{1.137372in}}{\pgfqpoint{2.133360in}{1.132073in}}{\pgfqpoint{2.133360in}{1.126548in}}%
\pgfpathcurveto{\pgfqpoint{2.133360in}{1.121023in}}{\pgfqpoint{2.135555in}{1.115723in}}{\pgfqpoint{2.139462in}{1.111817in}}%
\pgfpathcurveto{\pgfqpoint{2.143368in}{1.107910in}}{\pgfqpoint{2.148668in}{1.105715in}}{\pgfqpoint{2.154193in}{1.105715in}}%
\pgfpathclose%
\pgfusepath{fill}%
\end{pgfscope}%
\begin{pgfscope}%
\pgfpathrectangle{\pgfqpoint{2.051725in}{0.913832in}}{\pgfqpoint{1.162500in}{0.755000in}}%
\pgfusepath{clip}%
\pgfsetbuttcap%
\pgfsetroundjoin%
\definecolor{currentfill}{rgb}{0.000000,0.000000,0.000000}%
\pgfsetfillcolor{currentfill}%
\pgfsetfillopacity{0.500000}%
\pgfsetlinewidth{0.000000pt}%
\definecolor{currentstroke}{rgb}{0.000000,0.000000,0.000000}%
\pgfsetstrokecolor{currentstroke}%
\pgfsetdash{}{0pt}%
\pgfpathmoveto{\pgfqpoint{2.183344in}{0.910975in}}%
\pgfpathcurveto{\pgfqpoint{2.188869in}{0.910975in}}{\pgfqpoint{2.194169in}{0.913170in}}{\pgfqpoint{2.198076in}{0.917077in}}%
\pgfpathcurveto{\pgfqpoint{2.201982in}{0.920984in}}{\pgfqpoint{2.204178in}{0.926283in}}{\pgfqpoint{2.204178in}{0.931809in}}%
\pgfpathcurveto{\pgfqpoint{2.204178in}{0.937334in}}{\pgfqpoint{2.201982in}{0.942633in}}{\pgfqpoint{2.198076in}{0.946540in}}%
\pgfpathcurveto{\pgfqpoint{2.194169in}{0.950447in}}{\pgfqpoint{2.188869in}{0.952642in}}{\pgfqpoint{2.183344in}{0.952642in}}%
\pgfpathcurveto{\pgfqpoint{2.177819in}{0.952642in}}{\pgfqpoint{2.172520in}{0.950447in}}{\pgfqpoint{2.168613in}{0.946540in}}%
\pgfpathcurveto{\pgfqpoint{2.164706in}{0.942633in}}{\pgfqpoint{2.162511in}{0.937334in}}{\pgfqpoint{2.162511in}{0.931809in}}%
\pgfpathcurveto{\pgfqpoint{2.162511in}{0.926283in}}{\pgfqpoint{2.164706in}{0.920984in}}{\pgfqpoint{2.168613in}{0.917077in}}%
\pgfpathcurveto{\pgfqpoint{2.172520in}{0.913170in}}{\pgfqpoint{2.177819in}{0.910975in}}{\pgfqpoint{2.183344in}{0.910975in}}%
\pgfpathclose%
\pgfusepath{fill}%
\end{pgfscope}%
\begin{pgfscope}%
\pgfpathrectangle{\pgfqpoint{2.051725in}{0.913832in}}{\pgfqpoint{1.162500in}{0.755000in}}%
\pgfusepath{clip}%
\pgfsetbuttcap%
\pgfsetroundjoin%
\definecolor{currentfill}{rgb}{0.000000,0.000000,0.000000}%
\pgfsetfillcolor{currentfill}%
\pgfsetfillopacity{0.500000}%
\pgfsetlinewidth{0.000000pt}%
\definecolor{currentstroke}{rgb}{0.000000,0.000000,0.000000}%
\pgfsetstrokecolor{currentstroke}%
\pgfsetdash{}{0pt}%
\pgfpathmoveto{\pgfqpoint{2.082232in}{1.042931in}}%
\pgfpathcurveto{\pgfqpoint{2.087757in}{1.042931in}}{\pgfqpoint{2.093056in}{1.045126in}}{\pgfqpoint{2.096963in}{1.049033in}}%
\pgfpathcurveto{\pgfqpoint{2.100870in}{1.052940in}}{\pgfqpoint{2.103065in}{1.058239in}}{\pgfqpoint{2.103065in}{1.063764in}}%
\pgfpathcurveto{\pgfqpoint{2.103065in}{1.069289in}}{\pgfqpoint{2.100870in}{1.074589in}}{\pgfqpoint{2.096963in}{1.078495in}}%
\pgfpathcurveto{\pgfqpoint{2.093056in}{1.082402in}}{\pgfqpoint{2.087757in}{1.084597in}}{\pgfqpoint{2.082232in}{1.084597in}}%
\pgfpathcurveto{\pgfqpoint{2.076707in}{1.084597in}}{\pgfqpoint{2.071407in}{1.082402in}}{\pgfqpoint{2.067500in}{1.078495in}}%
\pgfpathcurveto{\pgfqpoint{2.063594in}{1.074589in}}{\pgfqpoint{2.061398in}{1.069289in}}{\pgfqpoint{2.061398in}{1.063764in}}%
\pgfpathcurveto{\pgfqpoint{2.061398in}{1.058239in}}{\pgfqpoint{2.063594in}{1.052940in}}{\pgfqpoint{2.067500in}{1.049033in}}%
\pgfpathcurveto{\pgfqpoint{2.071407in}{1.045126in}}{\pgfqpoint{2.076707in}{1.042931in}}{\pgfqpoint{2.082232in}{1.042931in}}%
\pgfpathclose%
\pgfusepath{fill}%
\end{pgfscope}%
\begin{pgfscope}%
\pgfpathrectangle{\pgfqpoint{2.051725in}{0.913832in}}{\pgfqpoint{1.162500in}{0.755000in}}%
\pgfusepath{clip}%
\pgfsetbuttcap%
\pgfsetroundjoin%
\definecolor{currentfill}{rgb}{0.000000,0.000000,0.000000}%
\pgfsetfillcolor{currentfill}%
\pgfsetfillopacity{0.500000}%
\pgfsetlinewidth{0.000000pt}%
\definecolor{currentstroke}{rgb}{0.000000,0.000000,0.000000}%
\pgfsetstrokecolor{currentstroke}%
\pgfsetdash{}{0pt}%
\pgfpathmoveto{\pgfqpoint{2.079404in}{1.035607in}}%
\pgfpathcurveto{\pgfqpoint{2.084929in}{1.035607in}}{\pgfqpoint{2.090228in}{1.037803in}}{\pgfqpoint{2.094135in}{1.041709in}}%
\pgfpathcurveto{\pgfqpoint{2.098042in}{1.045616in}}{\pgfqpoint{2.100237in}{1.050916in}}{\pgfqpoint{2.100237in}{1.056441in}}%
\pgfpathcurveto{\pgfqpoint{2.100237in}{1.061966in}}{\pgfqpoint{2.098042in}{1.067265in}}{\pgfqpoint{2.094135in}{1.071172in}}%
\pgfpathcurveto{\pgfqpoint{2.090228in}{1.075079in}}{\pgfqpoint{2.084929in}{1.077274in}}{\pgfqpoint{2.079404in}{1.077274in}}%
\pgfpathcurveto{\pgfqpoint{2.073879in}{1.077274in}}{\pgfqpoint{2.068579in}{1.075079in}}{\pgfqpoint{2.064672in}{1.071172in}}%
\pgfpathcurveto{\pgfqpoint{2.060765in}{1.067265in}}{\pgfqpoint{2.058570in}{1.061966in}}{\pgfqpoint{2.058570in}{1.056441in}}%
\pgfpathcurveto{\pgfqpoint{2.058570in}{1.050916in}}{\pgfqpoint{2.060765in}{1.045616in}}{\pgfqpoint{2.064672in}{1.041709in}}%
\pgfpathcurveto{\pgfqpoint{2.068579in}{1.037803in}}{\pgfqpoint{2.073879in}{1.035607in}}{\pgfqpoint{2.079404in}{1.035607in}}%
\pgfpathclose%
\pgfusepath{fill}%
\end{pgfscope}%
\begin{pgfscope}%
\pgfpathrectangle{\pgfqpoint{2.051725in}{0.913832in}}{\pgfqpoint{1.162500in}{0.755000in}}%
\pgfusepath{clip}%
\pgfsetbuttcap%
\pgfsetroundjoin%
\definecolor{currentfill}{rgb}{0.000000,0.000000,0.000000}%
\pgfsetfillcolor{currentfill}%
\pgfsetfillopacity{0.500000}%
\pgfsetlinewidth{0.000000pt}%
\definecolor{currentstroke}{rgb}{0.000000,0.000000,0.000000}%
\pgfsetstrokecolor{currentstroke}%
\pgfsetdash{}{0pt}%
\pgfpathmoveto{\pgfqpoint{2.536683in}{1.572595in}}%
\pgfpathcurveto{\pgfqpoint{2.542208in}{1.572595in}}{\pgfqpoint{2.547508in}{1.574791in}}{\pgfqpoint{2.551415in}{1.578697in}}%
\pgfpathcurveto{\pgfqpoint{2.555322in}{1.582604in}}{\pgfqpoint{2.557517in}{1.587904in}}{\pgfqpoint{2.557517in}{1.593429in}}%
\pgfpathcurveto{\pgfqpoint{2.557517in}{1.598954in}}{\pgfqpoint{2.555322in}{1.604253in}}{\pgfqpoint{2.551415in}{1.608160in}}%
\pgfpathcurveto{\pgfqpoint{2.547508in}{1.612067in}}{\pgfqpoint{2.542208in}{1.614262in}}{\pgfqpoint{2.536683in}{1.614262in}}%
\pgfpathcurveto{\pgfqpoint{2.531158in}{1.614262in}}{\pgfqpoint{2.525859in}{1.612067in}}{\pgfqpoint{2.521952in}{1.608160in}}%
\pgfpathcurveto{\pgfqpoint{2.518045in}{1.604253in}}{\pgfqpoint{2.515850in}{1.598954in}}{\pgfqpoint{2.515850in}{1.593429in}}%
\pgfpathcurveto{\pgfqpoint{2.515850in}{1.587904in}}{\pgfqpoint{2.518045in}{1.582604in}}{\pgfqpoint{2.521952in}{1.578697in}}%
\pgfpathcurveto{\pgfqpoint{2.525859in}{1.574791in}}{\pgfqpoint{2.531158in}{1.572595in}}{\pgfqpoint{2.536683in}{1.572595in}}%
\pgfpathclose%
\pgfusepath{fill}%
\end{pgfscope}%
\begin{pgfscope}%
\pgfpathrectangle{\pgfqpoint{2.051725in}{0.913832in}}{\pgfqpoint{1.162500in}{0.755000in}}%
\pgfusepath{clip}%
\pgfsetbuttcap%
\pgfsetroundjoin%
\definecolor{currentfill}{rgb}{0.000000,0.000000,0.000000}%
\pgfsetfillcolor{currentfill}%
\pgfsetfillopacity{0.500000}%
\pgfsetlinewidth{0.000000pt}%
\definecolor{currentstroke}{rgb}{0.000000,0.000000,0.000000}%
\pgfsetstrokecolor{currentstroke}%
\pgfsetdash{}{0pt}%
\pgfpathmoveto{\pgfqpoint{2.417381in}{1.630023in}}%
\pgfpathcurveto{\pgfqpoint{2.422906in}{1.630023in}}{\pgfqpoint{2.428205in}{1.632218in}}{\pgfqpoint{2.432112in}{1.636125in}}%
\pgfpathcurveto{\pgfqpoint{2.436019in}{1.640032in}}{\pgfqpoint{2.438214in}{1.645331in}}{\pgfqpoint{2.438214in}{1.650856in}}%
\pgfpathcurveto{\pgfqpoint{2.438214in}{1.656381in}}{\pgfqpoint{2.436019in}{1.661681in}}{\pgfqpoint{2.432112in}{1.665588in}}%
\pgfpathcurveto{\pgfqpoint{2.428205in}{1.669494in}}{\pgfqpoint{2.422906in}{1.671690in}}{\pgfqpoint{2.417381in}{1.671690in}}%
\pgfpathcurveto{\pgfqpoint{2.411856in}{1.671690in}}{\pgfqpoint{2.406556in}{1.669494in}}{\pgfqpoint{2.402649in}{1.665588in}}%
\pgfpathcurveto{\pgfqpoint{2.398743in}{1.661681in}}{\pgfqpoint{2.396547in}{1.656381in}}{\pgfqpoint{2.396547in}{1.650856in}}%
\pgfpathcurveto{\pgfqpoint{2.396547in}{1.645331in}}{\pgfqpoint{2.398743in}{1.640032in}}{\pgfqpoint{2.402649in}{1.636125in}}%
\pgfpathcurveto{\pgfqpoint{2.406556in}{1.632218in}}{\pgfqpoint{2.411856in}{1.630023in}}{\pgfqpoint{2.417381in}{1.630023in}}%
\pgfpathclose%
\pgfusepath{fill}%
\end{pgfscope}%
\begin{pgfscope}%
\pgfpathrectangle{\pgfqpoint{2.051725in}{0.913832in}}{\pgfqpoint{1.162500in}{0.755000in}}%
\pgfusepath{clip}%
\pgfsetbuttcap%
\pgfsetroundjoin%
\definecolor{currentfill}{rgb}{0.000000,0.000000,0.000000}%
\pgfsetfillcolor{currentfill}%
\pgfsetfillopacity{0.500000}%
\pgfsetlinewidth{0.000000pt}%
\definecolor{currentstroke}{rgb}{0.000000,0.000000,0.000000}%
\pgfsetstrokecolor{currentstroke}%
\pgfsetdash{}{0pt}%
\pgfpathmoveto{\pgfqpoint{2.412630in}{1.484185in}}%
\pgfpathcurveto{\pgfqpoint{2.418155in}{1.484185in}}{\pgfqpoint{2.423454in}{1.486380in}}{\pgfqpoint{2.427361in}{1.490287in}}%
\pgfpathcurveto{\pgfqpoint{2.431268in}{1.494194in}}{\pgfqpoint{2.433463in}{1.499493in}}{\pgfqpoint{2.433463in}{1.505018in}}%
\pgfpathcurveto{\pgfqpoint{2.433463in}{1.510544in}}{\pgfqpoint{2.431268in}{1.515843in}}{\pgfqpoint{2.427361in}{1.519750in}}%
\pgfpathcurveto{\pgfqpoint{2.423454in}{1.523657in}}{\pgfqpoint{2.418155in}{1.525852in}}{\pgfqpoint{2.412630in}{1.525852in}}%
\pgfpathcurveto{\pgfqpoint{2.407105in}{1.525852in}}{\pgfqpoint{2.401805in}{1.523657in}}{\pgfqpoint{2.397898in}{1.519750in}}%
\pgfpathcurveto{\pgfqpoint{2.393991in}{1.515843in}}{\pgfqpoint{2.391796in}{1.510544in}}{\pgfqpoint{2.391796in}{1.505018in}}%
\pgfpathcurveto{\pgfqpoint{2.391796in}{1.499493in}}{\pgfqpoint{2.393991in}{1.494194in}}{\pgfqpoint{2.397898in}{1.490287in}}%
\pgfpathcurveto{\pgfqpoint{2.401805in}{1.486380in}}{\pgfqpoint{2.407105in}{1.484185in}}{\pgfqpoint{2.412630in}{1.484185in}}%
\pgfpathclose%
\pgfusepath{fill}%
\end{pgfscope}%
\begin{pgfscope}%
\pgfpathrectangle{\pgfqpoint{2.051725in}{0.913832in}}{\pgfqpoint{1.162500in}{0.755000in}}%
\pgfusepath{clip}%
\pgfsetbuttcap%
\pgfsetroundjoin%
\definecolor{currentfill}{rgb}{0.000000,0.000000,0.000000}%
\pgfsetfillcolor{currentfill}%
\pgfsetfillopacity{0.500000}%
\pgfsetlinewidth{0.000000pt}%
\definecolor{currentstroke}{rgb}{0.000000,0.000000,0.000000}%
\pgfsetstrokecolor{currentstroke}%
\pgfsetdash{}{0pt}%
\pgfpathmoveto{\pgfqpoint{2.478188in}{1.067768in}}%
\pgfpathcurveto{\pgfqpoint{2.483713in}{1.067768in}}{\pgfqpoint{2.489012in}{1.069963in}}{\pgfqpoint{2.492919in}{1.073870in}}%
\pgfpathcurveto{\pgfqpoint{2.496826in}{1.077776in}}{\pgfqpoint{2.499021in}{1.083076in}}{\pgfqpoint{2.499021in}{1.088601in}}%
\pgfpathcurveto{\pgfqpoint{2.499021in}{1.094126in}}{\pgfqpoint{2.496826in}{1.099426in}}{\pgfqpoint{2.492919in}{1.103332in}}%
\pgfpathcurveto{\pgfqpoint{2.489012in}{1.107239in}}{\pgfqpoint{2.483713in}{1.109434in}}{\pgfqpoint{2.478188in}{1.109434in}}%
\pgfpathcurveto{\pgfqpoint{2.472663in}{1.109434in}}{\pgfqpoint{2.467363in}{1.107239in}}{\pgfqpoint{2.463456in}{1.103332in}}%
\pgfpathcurveto{\pgfqpoint{2.459549in}{1.099426in}}{\pgfqpoint{2.457354in}{1.094126in}}{\pgfqpoint{2.457354in}{1.088601in}}%
\pgfpathcurveto{\pgfqpoint{2.457354in}{1.083076in}}{\pgfqpoint{2.459549in}{1.077776in}}{\pgfqpoint{2.463456in}{1.073870in}}%
\pgfpathcurveto{\pgfqpoint{2.467363in}{1.069963in}}{\pgfqpoint{2.472663in}{1.067768in}}{\pgfqpoint{2.478188in}{1.067768in}}%
\pgfpathclose%
\pgfusepath{fill}%
\end{pgfscope}%
\begin{pgfscope}%
\pgfpathrectangle{\pgfqpoint{2.051725in}{0.913832in}}{\pgfqpoint{1.162500in}{0.755000in}}%
\pgfusepath{clip}%
\pgfsetbuttcap%
\pgfsetroundjoin%
\definecolor{currentfill}{rgb}{0.000000,0.000000,0.000000}%
\pgfsetfillcolor{currentfill}%
\pgfsetfillopacity{0.500000}%
\pgfsetlinewidth{0.000000pt}%
\definecolor{currentstroke}{rgb}{0.000000,0.000000,0.000000}%
\pgfsetstrokecolor{currentstroke}%
\pgfsetdash{}{0pt}%
\pgfpathmoveto{\pgfqpoint{2.172291in}{1.343150in}}%
\pgfpathcurveto{\pgfqpoint{2.177816in}{1.343150in}}{\pgfqpoint{2.183116in}{1.345345in}}{\pgfqpoint{2.187022in}{1.349252in}}%
\pgfpathcurveto{\pgfqpoint{2.190929in}{1.353159in}}{\pgfqpoint{2.193124in}{1.358458in}}{\pgfqpoint{2.193124in}{1.363984in}}%
\pgfpathcurveto{\pgfqpoint{2.193124in}{1.369509in}}{\pgfqpoint{2.190929in}{1.374808in}}{\pgfqpoint{2.187022in}{1.378715in}}%
\pgfpathcurveto{\pgfqpoint{2.183116in}{1.382622in}}{\pgfqpoint{2.177816in}{1.384817in}}{\pgfqpoint{2.172291in}{1.384817in}}%
\pgfpathcurveto{\pgfqpoint{2.166766in}{1.384817in}}{\pgfqpoint{2.161467in}{1.382622in}}{\pgfqpoint{2.157560in}{1.378715in}}%
\pgfpathcurveto{\pgfqpoint{2.153653in}{1.374808in}}{\pgfqpoint{2.151458in}{1.369509in}}{\pgfqpoint{2.151458in}{1.363984in}}%
\pgfpathcurveto{\pgfqpoint{2.151458in}{1.358458in}}{\pgfqpoint{2.153653in}{1.353159in}}{\pgfqpoint{2.157560in}{1.349252in}}%
\pgfpathcurveto{\pgfqpoint{2.161467in}{1.345345in}}{\pgfqpoint{2.166766in}{1.343150in}}{\pgfqpoint{2.172291in}{1.343150in}}%
\pgfpathclose%
\pgfusepath{fill}%
\end{pgfscope}%
\begin{pgfscope}%
\pgfpathrectangle{\pgfqpoint{2.051725in}{0.913832in}}{\pgfqpoint{1.162500in}{0.755000in}}%
\pgfusepath{clip}%
\pgfsetbuttcap%
\pgfsetroundjoin%
\definecolor{currentfill}{rgb}{0.000000,0.000000,0.000000}%
\pgfsetfillcolor{currentfill}%
\pgfsetfillopacity{0.500000}%
\pgfsetlinewidth{0.000000pt}%
\definecolor{currentstroke}{rgb}{0.000000,0.000000,0.000000}%
\pgfsetstrokecolor{currentstroke}%
\pgfsetdash{}{0pt}%
\pgfpathmoveto{\pgfqpoint{2.205718in}{1.259718in}}%
\pgfpathcurveto{\pgfqpoint{2.211243in}{1.259718in}}{\pgfqpoint{2.216543in}{1.261913in}}{\pgfqpoint{2.220450in}{1.265820in}}%
\pgfpathcurveto{\pgfqpoint{2.224357in}{1.269726in}}{\pgfqpoint{2.226552in}{1.275026in}}{\pgfqpoint{2.226552in}{1.280551in}}%
\pgfpathcurveto{\pgfqpoint{2.226552in}{1.286076in}}{\pgfqpoint{2.224357in}{1.291376in}}{\pgfqpoint{2.220450in}{1.295282in}}%
\pgfpathcurveto{\pgfqpoint{2.216543in}{1.299189in}}{\pgfqpoint{2.211243in}{1.301384in}}{\pgfqpoint{2.205718in}{1.301384in}}%
\pgfpathcurveto{\pgfqpoint{2.200193in}{1.301384in}}{\pgfqpoint{2.194894in}{1.299189in}}{\pgfqpoint{2.190987in}{1.295282in}}%
\pgfpathcurveto{\pgfqpoint{2.187080in}{1.291376in}}{\pgfqpoint{2.184885in}{1.286076in}}{\pgfqpoint{2.184885in}{1.280551in}}%
\pgfpathcurveto{\pgfqpoint{2.184885in}{1.275026in}}{\pgfqpoint{2.187080in}{1.269726in}}{\pgfqpoint{2.190987in}{1.265820in}}%
\pgfpathcurveto{\pgfqpoint{2.194894in}{1.261913in}}{\pgfqpoint{2.200193in}{1.259718in}}{\pgfqpoint{2.205718in}{1.259718in}}%
\pgfpathclose%
\pgfusepath{fill}%
\end{pgfscope}%
\begin{pgfscope}%
\pgfpathrectangle{\pgfqpoint{2.051725in}{0.913832in}}{\pgfqpoint{1.162500in}{0.755000in}}%
\pgfusepath{clip}%
\pgfsetbuttcap%
\pgfsetroundjoin%
\definecolor{currentfill}{rgb}{0.000000,0.000000,0.000000}%
\pgfsetfillcolor{currentfill}%
\pgfsetfillopacity{0.500000}%
\pgfsetlinewidth{0.000000pt}%
\definecolor{currentstroke}{rgb}{0.000000,0.000000,0.000000}%
\pgfsetstrokecolor{currentstroke}%
\pgfsetdash{}{0pt}%
\pgfpathmoveto{\pgfqpoint{3.186546in}{1.167350in}}%
\pgfpathcurveto{\pgfqpoint{3.192071in}{1.167350in}}{\pgfqpoint{3.197371in}{1.169545in}}{\pgfqpoint{3.201278in}{1.173452in}}%
\pgfpathcurveto{\pgfqpoint{3.205185in}{1.177359in}}{\pgfqpoint{3.207380in}{1.182658in}}{\pgfqpoint{3.207380in}{1.188183in}}%
\pgfpathcurveto{\pgfqpoint{3.207380in}{1.193708in}}{\pgfqpoint{3.205185in}{1.199008in}}{\pgfqpoint{3.201278in}{1.202915in}}%
\pgfpathcurveto{\pgfqpoint{3.197371in}{1.206822in}}{\pgfqpoint{3.192071in}{1.209017in}}{\pgfqpoint{3.186546in}{1.209017in}}%
\pgfpathcurveto{\pgfqpoint{3.181021in}{1.209017in}}{\pgfqpoint{3.175722in}{1.206822in}}{\pgfqpoint{3.171815in}{1.202915in}}%
\pgfpathcurveto{\pgfqpoint{3.167908in}{1.199008in}}{\pgfqpoint{3.165713in}{1.193708in}}{\pgfqpoint{3.165713in}{1.188183in}}%
\pgfpathcurveto{\pgfqpoint{3.165713in}{1.182658in}}{\pgfqpoint{3.167908in}{1.177359in}}{\pgfqpoint{3.171815in}{1.173452in}}%
\pgfpathcurveto{\pgfqpoint{3.175722in}{1.169545in}}{\pgfqpoint{3.181021in}{1.167350in}}{\pgfqpoint{3.186546in}{1.167350in}}%
\pgfpathclose%
\pgfusepath{fill}%
\end{pgfscope}%
\begin{pgfscope}%
\pgfpathrectangle{\pgfqpoint{2.051725in}{0.913832in}}{\pgfqpoint{1.162500in}{0.755000in}}%
\pgfusepath{clip}%
\pgfsetbuttcap%
\pgfsetroundjoin%
\definecolor{currentfill}{rgb}{0.000000,0.000000,0.000000}%
\pgfsetfillcolor{currentfill}%
\pgfsetfillopacity{0.500000}%
\pgfsetlinewidth{0.000000pt}%
\definecolor{currentstroke}{rgb}{0.000000,0.000000,0.000000}%
\pgfsetstrokecolor{currentstroke}%
\pgfsetdash{}{0pt}%
\pgfpathmoveto{\pgfqpoint{2.394110in}{1.034902in}}%
\pgfpathcurveto{\pgfqpoint{2.399635in}{1.034902in}}{\pgfqpoint{2.404935in}{1.037097in}}{\pgfqpoint{2.408842in}{1.041004in}}%
\pgfpathcurveto{\pgfqpoint{2.412749in}{1.044911in}}{\pgfqpoint{2.414944in}{1.050211in}}{\pgfqpoint{2.414944in}{1.055736in}}%
\pgfpathcurveto{\pgfqpoint{2.414944in}{1.061261in}}{\pgfqpoint{2.412749in}{1.066560in}}{\pgfqpoint{2.408842in}{1.070467in}}%
\pgfpathcurveto{\pgfqpoint{2.404935in}{1.074374in}}{\pgfqpoint{2.399635in}{1.076569in}}{\pgfqpoint{2.394110in}{1.076569in}}%
\pgfpathcurveto{\pgfqpoint{2.388585in}{1.076569in}}{\pgfqpoint{2.383286in}{1.074374in}}{\pgfqpoint{2.379379in}{1.070467in}}%
\pgfpathcurveto{\pgfqpoint{2.375472in}{1.066560in}}{\pgfqpoint{2.373277in}{1.061261in}}{\pgfqpoint{2.373277in}{1.055736in}}%
\pgfpathcurveto{\pgfqpoint{2.373277in}{1.050211in}}{\pgfqpoint{2.375472in}{1.044911in}}{\pgfqpoint{2.379379in}{1.041004in}}%
\pgfpathcurveto{\pgfqpoint{2.383286in}{1.037097in}}{\pgfqpoint{2.388585in}{1.034902in}}{\pgfqpoint{2.394110in}{1.034902in}}%
\pgfpathclose%
\pgfusepath{fill}%
\end{pgfscope}%
\begin{pgfscope}%
\pgfpathrectangle{\pgfqpoint{2.051725in}{0.913832in}}{\pgfqpoint{1.162500in}{0.755000in}}%
\pgfusepath{clip}%
\pgfsetbuttcap%
\pgfsetroundjoin%
\definecolor{currentfill}{rgb}{0.000000,0.000000,0.000000}%
\pgfsetfillcolor{currentfill}%
\pgfsetfillopacity{0.500000}%
\pgfsetlinewidth{0.000000pt}%
\definecolor{currentstroke}{rgb}{0.000000,0.000000,0.000000}%
\pgfsetstrokecolor{currentstroke}%
\pgfsetdash{}{0pt}%
\pgfpathmoveto{\pgfqpoint{2.125339in}{1.190157in}}%
\pgfpathcurveto{\pgfqpoint{2.130865in}{1.190157in}}{\pgfqpoint{2.136164in}{1.192352in}}{\pgfqpoint{2.140071in}{1.196259in}}%
\pgfpathcurveto{\pgfqpoint{2.143978in}{1.200166in}}{\pgfqpoint{2.146173in}{1.205465in}}{\pgfqpoint{2.146173in}{1.210991in}}%
\pgfpathcurveto{\pgfqpoint{2.146173in}{1.216516in}}{\pgfqpoint{2.143978in}{1.221815in}}{\pgfqpoint{2.140071in}{1.225722in}}%
\pgfpathcurveto{\pgfqpoint{2.136164in}{1.229629in}}{\pgfqpoint{2.130865in}{1.231824in}}{\pgfqpoint{2.125339in}{1.231824in}}%
\pgfpathcurveto{\pgfqpoint{2.119814in}{1.231824in}}{\pgfqpoint{2.114515in}{1.229629in}}{\pgfqpoint{2.110608in}{1.225722in}}%
\pgfpathcurveto{\pgfqpoint{2.106701in}{1.221815in}}{\pgfqpoint{2.104506in}{1.216516in}}{\pgfqpoint{2.104506in}{1.210991in}}%
\pgfpathcurveto{\pgfqpoint{2.104506in}{1.205465in}}{\pgfqpoint{2.106701in}{1.200166in}}{\pgfqpoint{2.110608in}{1.196259in}}%
\pgfpathcurveto{\pgfqpoint{2.114515in}{1.192352in}}{\pgfqpoint{2.119814in}{1.190157in}}{\pgfqpoint{2.125339in}{1.190157in}}%
\pgfpathclose%
\pgfusepath{fill}%
\end{pgfscope}%
\begin{pgfscope}%
\pgfsetbuttcap%
\pgfsetroundjoin%
\definecolor{currentfill}{rgb}{0.000000,0.000000,0.000000}%
\pgfsetfillcolor{currentfill}%
\pgfsetlinewidth{0.803000pt}%
\definecolor{currentstroke}{rgb}{0.000000,0.000000,0.000000}%
\pgfsetstrokecolor{currentstroke}%
\pgfsetdash{}{0pt}%
\pgfsys@defobject{currentmarker}{\pgfqpoint{0.000000in}{-0.048611in}}{\pgfqpoint{0.000000in}{0.000000in}}{%
\pgfpathmoveto{\pgfqpoint{0.000000in}{0.000000in}}%
\pgfpathlineto{\pgfqpoint{0.000000in}{-0.048611in}}%
\pgfusepath{stroke,fill}%
}%
\begin{pgfscope}%
\pgfsys@transformshift{2.463280in}{0.913832in}%
\pgfsys@useobject{currentmarker}{}%
\end{pgfscope}%
\end{pgfscope}%
\begin{pgfscope}%
\pgftext[x=2.493933in,y=0.665759in,left,base,rotate=90.000000]{\rmfamily\fontsize{8.000000}{9.600000}\selectfont \(\displaystyle 0.5\)}%
\end{pgfscope}%
\begin{pgfscope}%
\pgfsetbuttcap%
\pgfsetroundjoin%
\definecolor{currentfill}{rgb}{0.000000,0.000000,0.000000}%
\pgfsetfillcolor{currentfill}%
\pgfsetlinewidth{0.803000pt}%
\definecolor{currentstroke}{rgb}{0.000000,0.000000,0.000000}%
\pgfsetstrokecolor{currentstroke}%
\pgfsetdash{}{0pt}%
\pgfsys@defobject{currentmarker}{\pgfqpoint{0.000000in}{-0.048611in}}{\pgfqpoint{0.000000in}{0.000000in}}{%
\pgfpathmoveto{\pgfqpoint{0.000000in}{0.000000in}}%
\pgfpathlineto{\pgfqpoint{0.000000in}{-0.048611in}}%
\pgfusepath{stroke,fill}%
}%
\begin{pgfscope}%
\pgfsys@transformshift{2.916252in}{0.913832in}%
\pgfsys@useobject{currentmarker}{}%
\end{pgfscope}%
\end{pgfscope}%
\begin{pgfscope}%
\pgftext[x=2.946906in,y=0.665759in,left,base,rotate=90.000000]{\rmfamily\fontsize{8.000000}{9.600000}\selectfont \(\displaystyle 1.0\)}%
\end{pgfscope}%
\begin{pgfscope}%
\pgftext[x=2.632975in,y=0.610204in,,top]{\rmfamily\fontsize{16.000000}{19.200000}\selectfont charge}%
\end{pgfscope}%
\begin{pgfscope}%
\pgftext[x=3.214225in,y=0.624092in,right,top]{\rmfamily\fontsize{16.000000}{19.200000}\selectfont \(\displaystyle \times10^{-9}\)}%
\end{pgfscope}%
\begin{pgfscope}%
\pgfsetrectcap%
\pgfsetmiterjoin%
\pgfsetlinewidth{0.803000pt}%
\definecolor{currentstroke}{rgb}{0.501961,0.501961,0.501961}%
\pgfsetstrokecolor{currentstroke}%
\pgfsetdash{}{0pt}%
\pgfpathmoveto{\pgfqpoint{2.051725in}{0.913832in}}%
\pgfpathlineto{\pgfqpoint{2.051725in}{1.668832in}}%
\pgfusepath{stroke}%
\end{pgfscope}%
\begin{pgfscope}%
\pgfsetrectcap%
\pgfsetmiterjoin%
\pgfsetlinewidth{0.803000pt}%
\definecolor{currentstroke}{rgb}{0.501961,0.501961,0.501961}%
\pgfsetstrokecolor{currentstroke}%
\pgfsetdash{}{0pt}%
\pgfpathmoveto{\pgfqpoint{3.214225in}{0.913832in}}%
\pgfpathlineto{\pgfqpoint{3.214225in}{1.668832in}}%
\pgfusepath{stroke}%
\end{pgfscope}%
\begin{pgfscope}%
\pgfsetrectcap%
\pgfsetmiterjoin%
\pgfsetlinewidth{0.803000pt}%
\definecolor{currentstroke}{rgb}{0.501961,0.501961,0.501961}%
\pgfsetstrokecolor{currentstroke}%
\pgfsetdash{}{0pt}%
\pgfpathmoveto{\pgfqpoint{2.051725in}{0.913832in}}%
\pgfpathlineto{\pgfqpoint{3.214225in}{0.913832in}}%
\pgfusepath{stroke}%
\end{pgfscope}%
\begin{pgfscope}%
\pgfsetrectcap%
\pgfsetmiterjoin%
\pgfsetlinewidth{0.803000pt}%
\definecolor{currentstroke}{rgb}{0.501961,0.501961,0.501961}%
\pgfsetstrokecolor{currentstroke}%
\pgfsetdash{}{0pt}%
\pgfpathmoveto{\pgfqpoint{2.051725in}{1.668832in}}%
\pgfpathlineto{\pgfqpoint{3.214225in}{1.668832in}}%
\pgfusepath{stroke}%
\end{pgfscope}%
\begin{pgfscope}%
\pgfsetbuttcap%
\pgfsetmiterjoin%
\definecolor{currentfill}{rgb}{1.000000,1.000000,1.000000}%
\pgfsetfillcolor{currentfill}%
\pgfsetlinewidth{0.000000pt}%
\definecolor{currentstroke}{rgb}{0.000000,0.000000,0.000000}%
\pgfsetstrokecolor{currentstroke}%
\pgfsetstrokeopacity{0.000000}%
\pgfsetdash{}{0pt}%
\pgfpathmoveto{\pgfqpoint{3.214225in}{0.913832in}}%
\pgfpathlineto{\pgfqpoint{4.376725in}{0.913832in}}%
\pgfpathlineto{\pgfqpoint{4.376725in}{1.668832in}}%
\pgfpathlineto{\pgfqpoint{3.214225in}{1.668832in}}%
\pgfpathclose%
\pgfusepath{fill}%
\end{pgfscope}%
\begin{pgfscope}%
\pgfpathrectangle{\pgfqpoint{3.214225in}{0.913832in}}{\pgfqpoint{1.162500in}{0.755000in}}%
\pgfusepath{clip}%
\pgfsetbuttcap%
\pgfsetroundjoin%
\definecolor{currentfill}{rgb}{0.000000,0.000000,0.000000}%
\pgfsetfillcolor{currentfill}%
\pgfsetfillopacity{0.500000}%
\pgfsetlinewidth{0.000000pt}%
\definecolor{currentstroke}{rgb}{0.000000,0.000000,0.000000}%
\pgfsetstrokecolor{currentstroke}%
\pgfsetdash{}{0pt}%
\pgfpathmoveto{\pgfqpoint{3.656364in}{1.437401in}}%
\pgfpathcurveto{\pgfqpoint{3.661889in}{1.437401in}}{\pgfqpoint{3.667189in}{1.439596in}}{\pgfqpoint{3.671096in}{1.443503in}}%
\pgfpathcurveto{\pgfqpoint{3.675002in}{1.447410in}}{\pgfqpoint{3.677198in}{1.452710in}}{\pgfqpoint{3.677198in}{1.458235in}}%
\pgfpathcurveto{\pgfqpoint{3.677198in}{1.463760in}}{\pgfqpoint{3.675002in}{1.469059in}}{\pgfqpoint{3.671096in}{1.472966in}}%
\pgfpathcurveto{\pgfqpoint{3.667189in}{1.476873in}}{\pgfqpoint{3.661889in}{1.479068in}}{\pgfqpoint{3.656364in}{1.479068in}}%
\pgfpathcurveto{\pgfqpoint{3.650839in}{1.479068in}}{\pgfqpoint{3.645540in}{1.476873in}}{\pgfqpoint{3.641633in}{1.472966in}}%
\pgfpathcurveto{\pgfqpoint{3.637726in}{1.469059in}}{\pgfqpoint{3.635531in}{1.463760in}}{\pgfqpoint{3.635531in}{1.458235in}}%
\pgfpathcurveto{\pgfqpoint{3.635531in}{1.452710in}}{\pgfqpoint{3.637726in}{1.447410in}}{\pgfqpoint{3.641633in}{1.443503in}}%
\pgfpathcurveto{\pgfqpoint{3.645540in}{1.439596in}}{\pgfqpoint{3.650839in}{1.437401in}}{\pgfqpoint{3.656364in}{1.437401in}}%
\pgfpathclose%
\pgfusepath{fill}%
\end{pgfscope}%
\begin{pgfscope}%
\pgfpathrectangle{\pgfqpoint{3.214225in}{0.913832in}}{\pgfqpoint{1.162500in}{0.755000in}}%
\pgfusepath{clip}%
\pgfsetbuttcap%
\pgfsetroundjoin%
\definecolor{currentfill}{rgb}{0.000000,0.000000,0.000000}%
\pgfsetfillcolor{currentfill}%
\pgfsetfillopacity{0.500000}%
\pgfsetlinewidth{0.000000pt}%
\definecolor{currentstroke}{rgb}{0.000000,0.000000,0.000000}%
\pgfsetstrokecolor{currentstroke}%
\pgfsetdash{}{0pt}%
\pgfpathmoveto{\pgfqpoint{4.259629in}{0.987180in}}%
\pgfpathcurveto{\pgfqpoint{4.265154in}{0.987180in}}{\pgfqpoint{4.270454in}{0.989375in}}{\pgfqpoint{4.274361in}{0.993282in}}%
\pgfpathcurveto{\pgfqpoint{4.278268in}{0.997189in}}{\pgfqpoint{4.280463in}{1.002488in}}{\pgfqpoint{4.280463in}{1.008013in}}%
\pgfpathcurveto{\pgfqpoint{4.280463in}{1.013538in}}{\pgfqpoint{4.278268in}{1.018838in}}{\pgfqpoint{4.274361in}{1.022745in}}%
\pgfpathcurveto{\pgfqpoint{4.270454in}{1.026652in}}{\pgfqpoint{4.265154in}{1.028847in}}{\pgfqpoint{4.259629in}{1.028847in}}%
\pgfpathcurveto{\pgfqpoint{4.254104in}{1.028847in}}{\pgfqpoint{4.248805in}{1.026652in}}{\pgfqpoint{4.244898in}{1.022745in}}%
\pgfpathcurveto{\pgfqpoint{4.240991in}{1.018838in}}{\pgfqpoint{4.238796in}{1.013538in}}{\pgfqpoint{4.238796in}{1.008013in}}%
\pgfpathcurveto{\pgfqpoint{4.238796in}{1.002488in}}{\pgfqpoint{4.240991in}{0.997189in}}{\pgfqpoint{4.244898in}{0.993282in}}%
\pgfpathcurveto{\pgfqpoint{4.248805in}{0.989375in}}{\pgfqpoint{4.254104in}{0.987180in}}{\pgfqpoint{4.259629in}{0.987180in}}%
\pgfpathclose%
\pgfusepath{fill}%
\end{pgfscope}%
\begin{pgfscope}%
\pgfpathrectangle{\pgfqpoint{3.214225in}{0.913832in}}{\pgfqpoint{1.162500in}{0.755000in}}%
\pgfusepath{clip}%
\pgfsetbuttcap%
\pgfsetroundjoin%
\definecolor{currentfill}{rgb}{0.000000,0.000000,0.000000}%
\pgfsetfillcolor{currentfill}%
\pgfsetfillopacity{0.500000}%
\pgfsetlinewidth{0.000000pt}%
\definecolor{currentstroke}{rgb}{0.000000,0.000000,0.000000}%
\pgfsetstrokecolor{currentstroke}%
\pgfsetdash{}{0pt}%
\pgfpathmoveto{\pgfqpoint{4.349046in}{0.939886in}}%
\pgfpathcurveto{\pgfqpoint{4.354571in}{0.939886in}}{\pgfqpoint{4.359871in}{0.942081in}}{\pgfqpoint{4.363778in}{0.945988in}}%
\pgfpathcurveto{\pgfqpoint{4.367685in}{0.949895in}}{\pgfqpoint{4.369880in}{0.955195in}}{\pgfqpoint{4.369880in}{0.960720in}}%
\pgfpathcurveto{\pgfqpoint{4.369880in}{0.966245in}}{\pgfqpoint{4.367685in}{0.971544in}}{\pgfqpoint{4.363778in}{0.975451in}}%
\pgfpathcurveto{\pgfqpoint{4.359871in}{0.979358in}}{\pgfqpoint{4.354571in}{0.981553in}}{\pgfqpoint{4.349046in}{0.981553in}}%
\pgfpathcurveto{\pgfqpoint{4.343521in}{0.981553in}}{\pgfqpoint{4.338222in}{0.979358in}}{\pgfqpoint{4.334315in}{0.975451in}}%
\pgfpathcurveto{\pgfqpoint{4.330408in}{0.971544in}}{\pgfqpoint{4.328213in}{0.966245in}}{\pgfqpoint{4.328213in}{0.960720in}}%
\pgfpathcurveto{\pgfqpoint{4.328213in}{0.955195in}}{\pgfqpoint{4.330408in}{0.949895in}}{\pgfqpoint{4.334315in}{0.945988in}}%
\pgfpathcurveto{\pgfqpoint{4.338222in}{0.942081in}}{\pgfqpoint{4.343521in}{0.939886in}}{\pgfqpoint{4.349046in}{0.939886in}}%
\pgfpathclose%
\pgfusepath{fill}%
\end{pgfscope}%
\begin{pgfscope}%
\pgfpathrectangle{\pgfqpoint{3.214225in}{0.913832in}}{\pgfqpoint{1.162500in}{0.755000in}}%
\pgfusepath{clip}%
\pgfsetbuttcap%
\pgfsetroundjoin%
\definecolor{currentfill}{rgb}{0.000000,0.000000,0.000000}%
\pgfsetfillcolor{currentfill}%
\pgfsetfillopacity{0.500000}%
\pgfsetlinewidth{0.000000pt}%
\definecolor{currentstroke}{rgb}{0.000000,0.000000,0.000000}%
\pgfsetstrokecolor{currentstroke}%
\pgfsetdash{}{0pt}%
\pgfpathmoveto{\pgfqpoint{3.915823in}{1.105715in}}%
\pgfpathcurveto{\pgfqpoint{3.921348in}{1.105715in}}{\pgfqpoint{3.926648in}{1.107910in}}{\pgfqpoint{3.930554in}{1.111817in}}%
\pgfpathcurveto{\pgfqpoint{3.934461in}{1.115723in}}{\pgfqpoint{3.936656in}{1.121023in}}{\pgfqpoint{3.936656in}{1.126548in}}%
\pgfpathcurveto{\pgfqpoint{3.936656in}{1.132073in}}{\pgfqpoint{3.934461in}{1.137372in}}{\pgfqpoint{3.930554in}{1.141279in}}%
\pgfpathcurveto{\pgfqpoint{3.926648in}{1.145186in}}{\pgfqpoint{3.921348in}{1.147381in}}{\pgfqpoint{3.915823in}{1.147381in}}%
\pgfpathcurveto{\pgfqpoint{3.910298in}{1.147381in}}{\pgfqpoint{3.904998in}{1.145186in}}{\pgfqpoint{3.901092in}{1.141279in}}%
\pgfpathcurveto{\pgfqpoint{3.897185in}{1.137372in}}{\pgfqpoint{3.894990in}{1.132073in}}{\pgfqpoint{3.894990in}{1.126548in}}%
\pgfpathcurveto{\pgfqpoint{3.894990in}{1.121023in}}{\pgfqpoint{3.897185in}{1.115723in}}{\pgfqpoint{3.901092in}{1.111817in}}%
\pgfpathcurveto{\pgfqpoint{3.904998in}{1.107910in}}{\pgfqpoint{3.910298in}{1.105715in}}{\pgfqpoint{3.915823in}{1.105715in}}%
\pgfpathclose%
\pgfusepath{fill}%
\end{pgfscope}%
\begin{pgfscope}%
\pgfpathrectangle{\pgfqpoint{3.214225in}{0.913832in}}{\pgfqpoint{1.162500in}{0.755000in}}%
\pgfusepath{clip}%
\pgfsetbuttcap%
\pgfsetroundjoin%
\definecolor{currentfill}{rgb}{0.000000,0.000000,0.000000}%
\pgfsetfillcolor{currentfill}%
\pgfsetfillopacity{0.500000}%
\pgfsetlinewidth{0.000000pt}%
\definecolor{currentstroke}{rgb}{0.000000,0.000000,0.000000}%
\pgfsetstrokecolor{currentstroke}%
\pgfsetdash{}{0pt}%
\pgfpathmoveto{\pgfqpoint{3.986481in}{0.910975in}}%
\pgfpathcurveto{\pgfqpoint{3.992006in}{0.910975in}}{\pgfqpoint{3.997306in}{0.913170in}}{\pgfqpoint{4.001213in}{0.917077in}}%
\pgfpathcurveto{\pgfqpoint{4.005120in}{0.920984in}}{\pgfqpoint{4.007315in}{0.926283in}}{\pgfqpoint{4.007315in}{0.931809in}}%
\pgfpathcurveto{\pgfqpoint{4.007315in}{0.937334in}}{\pgfqpoint{4.005120in}{0.942633in}}{\pgfqpoint{4.001213in}{0.946540in}}%
\pgfpathcurveto{\pgfqpoint{3.997306in}{0.950447in}}{\pgfqpoint{3.992006in}{0.952642in}}{\pgfqpoint{3.986481in}{0.952642in}}%
\pgfpathcurveto{\pgfqpoint{3.980956in}{0.952642in}}{\pgfqpoint{3.975657in}{0.950447in}}{\pgfqpoint{3.971750in}{0.946540in}}%
\pgfpathcurveto{\pgfqpoint{3.967843in}{0.942633in}}{\pgfqpoint{3.965648in}{0.937334in}}{\pgfqpoint{3.965648in}{0.931809in}}%
\pgfpathcurveto{\pgfqpoint{3.965648in}{0.926283in}}{\pgfqpoint{3.967843in}{0.920984in}}{\pgfqpoint{3.971750in}{0.917077in}}%
\pgfpathcurveto{\pgfqpoint{3.975657in}{0.913170in}}{\pgfqpoint{3.980956in}{0.910975in}}{\pgfqpoint{3.986481in}{0.910975in}}%
\pgfpathclose%
\pgfusepath{fill}%
\end{pgfscope}%
\begin{pgfscope}%
\pgfpathrectangle{\pgfqpoint{3.214225in}{0.913832in}}{\pgfqpoint{1.162500in}{0.755000in}}%
\pgfusepath{clip}%
\pgfsetbuttcap%
\pgfsetroundjoin%
\definecolor{currentfill}{rgb}{0.000000,0.000000,0.000000}%
\pgfsetfillcolor{currentfill}%
\pgfsetfillopacity{0.500000}%
\pgfsetlinewidth{0.000000pt}%
\definecolor{currentstroke}{rgb}{0.000000,0.000000,0.000000}%
\pgfsetstrokecolor{currentstroke}%
\pgfsetdash{}{0pt}%
\pgfpathmoveto{\pgfqpoint{3.831974in}{1.042931in}}%
\pgfpathcurveto{\pgfqpoint{3.837499in}{1.042931in}}{\pgfqpoint{3.842798in}{1.045126in}}{\pgfqpoint{3.846705in}{1.049033in}}%
\pgfpathcurveto{\pgfqpoint{3.850612in}{1.052940in}}{\pgfqpoint{3.852807in}{1.058239in}}{\pgfqpoint{3.852807in}{1.063764in}}%
\pgfpathcurveto{\pgfqpoint{3.852807in}{1.069289in}}{\pgfqpoint{3.850612in}{1.074589in}}{\pgfqpoint{3.846705in}{1.078495in}}%
\pgfpathcurveto{\pgfqpoint{3.842798in}{1.082402in}}{\pgfqpoint{3.837499in}{1.084597in}}{\pgfqpoint{3.831974in}{1.084597in}}%
\pgfpathcurveto{\pgfqpoint{3.826449in}{1.084597in}}{\pgfqpoint{3.821149in}{1.082402in}}{\pgfqpoint{3.817242in}{1.078495in}}%
\pgfpathcurveto{\pgfqpoint{3.813336in}{1.074589in}}{\pgfqpoint{3.811140in}{1.069289in}}{\pgfqpoint{3.811140in}{1.063764in}}%
\pgfpathcurveto{\pgfqpoint{3.811140in}{1.058239in}}{\pgfqpoint{3.813336in}{1.052940in}}{\pgfqpoint{3.817242in}{1.049033in}}%
\pgfpathcurveto{\pgfqpoint{3.821149in}{1.045126in}}{\pgfqpoint{3.826449in}{1.042931in}}{\pgfqpoint{3.831974in}{1.042931in}}%
\pgfpathclose%
\pgfusepath{fill}%
\end{pgfscope}%
\begin{pgfscope}%
\pgfpathrectangle{\pgfqpoint{3.214225in}{0.913832in}}{\pgfqpoint{1.162500in}{0.755000in}}%
\pgfusepath{clip}%
\pgfsetbuttcap%
\pgfsetroundjoin%
\definecolor{currentfill}{rgb}{0.000000,0.000000,0.000000}%
\pgfsetfillcolor{currentfill}%
\pgfsetfillopacity{0.500000}%
\pgfsetlinewidth{0.000000pt}%
\definecolor{currentstroke}{rgb}{0.000000,0.000000,0.000000}%
\pgfsetstrokecolor{currentstroke}%
\pgfsetdash{}{0pt}%
\pgfpathmoveto{\pgfqpoint{3.241904in}{1.035607in}}%
\pgfpathcurveto{\pgfqpoint{3.247429in}{1.035607in}}{\pgfqpoint{3.252728in}{1.037803in}}{\pgfqpoint{3.256635in}{1.041709in}}%
\pgfpathcurveto{\pgfqpoint{3.260542in}{1.045616in}}{\pgfqpoint{3.262737in}{1.050916in}}{\pgfqpoint{3.262737in}{1.056441in}}%
\pgfpathcurveto{\pgfqpoint{3.262737in}{1.061966in}}{\pgfqpoint{3.260542in}{1.067265in}}{\pgfqpoint{3.256635in}{1.071172in}}%
\pgfpathcurveto{\pgfqpoint{3.252728in}{1.075079in}}{\pgfqpoint{3.247429in}{1.077274in}}{\pgfqpoint{3.241904in}{1.077274in}}%
\pgfpathcurveto{\pgfqpoint{3.236379in}{1.077274in}}{\pgfqpoint{3.231079in}{1.075079in}}{\pgfqpoint{3.227172in}{1.071172in}}%
\pgfpathcurveto{\pgfqpoint{3.223265in}{1.067265in}}{\pgfqpoint{3.221070in}{1.061966in}}{\pgfqpoint{3.221070in}{1.056441in}}%
\pgfpathcurveto{\pgfqpoint{3.221070in}{1.050916in}}{\pgfqpoint{3.223265in}{1.045616in}}{\pgfqpoint{3.227172in}{1.041709in}}%
\pgfpathcurveto{\pgfqpoint{3.231079in}{1.037803in}}{\pgfqpoint{3.236379in}{1.035607in}}{\pgfqpoint{3.241904in}{1.035607in}}%
\pgfpathclose%
\pgfusepath{fill}%
\end{pgfscope}%
\begin{pgfscope}%
\pgfpathrectangle{\pgfqpoint{3.214225in}{0.913832in}}{\pgfqpoint{1.162500in}{0.755000in}}%
\pgfusepath{clip}%
\pgfsetbuttcap%
\pgfsetroundjoin%
\definecolor{currentfill}{rgb}{0.000000,0.000000,0.000000}%
\pgfsetfillcolor{currentfill}%
\pgfsetfillopacity{0.500000}%
\pgfsetlinewidth{0.000000pt}%
\definecolor{currentstroke}{rgb}{0.000000,0.000000,0.000000}%
\pgfsetstrokecolor{currentstroke}%
\pgfsetdash{}{0pt}%
\pgfpathmoveto{\pgfqpoint{4.220724in}{1.572595in}}%
\pgfpathcurveto{\pgfqpoint{4.226249in}{1.572595in}}{\pgfqpoint{4.231548in}{1.574791in}}{\pgfqpoint{4.235455in}{1.578697in}}%
\pgfpathcurveto{\pgfqpoint{4.239362in}{1.582604in}}{\pgfqpoint{4.241557in}{1.587904in}}{\pgfqpoint{4.241557in}{1.593429in}}%
\pgfpathcurveto{\pgfqpoint{4.241557in}{1.598954in}}{\pgfqpoint{4.239362in}{1.604253in}}{\pgfqpoint{4.235455in}{1.608160in}}%
\pgfpathcurveto{\pgfqpoint{4.231548in}{1.612067in}}{\pgfqpoint{4.226249in}{1.614262in}}{\pgfqpoint{4.220724in}{1.614262in}}%
\pgfpathcurveto{\pgfqpoint{4.215199in}{1.614262in}}{\pgfqpoint{4.209899in}{1.612067in}}{\pgfqpoint{4.205992in}{1.608160in}}%
\pgfpathcurveto{\pgfqpoint{4.202086in}{1.604253in}}{\pgfqpoint{4.199890in}{1.598954in}}{\pgfqpoint{4.199890in}{1.593429in}}%
\pgfpathcurveto{\pgfqpoint{4.199890in}{1.587904in}}{\pgfqpoint{4.202086in}{1.582604in}}{\pgfqpoint{4.205992in}{1.578697in}}%
\pgfpathcurveto{\pgfqpoint{4.209899in}{1.574791in}}{\pgfqpoint{4.215199in}{1.572595in}}{\pgfqpoint{4.220724in}{1.572595in}}%
\pgfpathclose%
\pgfusepath{fill}%
\end{pgfscope}%
\begin{pgfscope}%
\pgfpathrectangle{\pgfqpoint{3.214225in}{0.913832in}}{\pgfqpoint{1.162500in}{0.755000in}}%
\pgfusepath{clip}%
\pgfsetbuttcap%
\pgfsetroundjoin%
\definecolor{currentfill}{rgb}{0.000000,0.000000,0.000000}%
\pgfsetfillcolor{currentfill}%
\pgfsetfillopacity{0.500000}%
\pgfsetlinewidth{0.000000pt}%
\definecolor{currentstroke}{rgb}{0.000000,0.000000,0.000000}%
\pgfsetstrokecolor{currentstroke}%
\pgfsetdash{}{0pt}%
\pgfpathmoveto{\pgfqpoint{3.980838in}{1.630023in}}%
\pgfpathcurveto{\pgfqpoint{3.986364in}{1.630023in}}{\pgfqpoint{3.991663in}{1.632218in}}{\pgfqpoint{3.995570in}{1.636125in}}%
\pgfpathcurveto{\pgfqpoint{3.999477in}{1.640032in}}{\pgfqpoint{4.001672in}{1.645331in}}{\pgfqpoint{4.001672in}{1.650856in}}%
\pgfpathcurveto{\pgfqpoint{4.001672in}{1.656381in}}{\pgfqpoint{3.999477in}{1.661681in}}{\pgfqpoint{3.995570in}{1.665588in}}%
\pgfpathcurveto{\pgfqpoint{3.991663in}{1.669494in}}{\pgfqpoint{3.986364in}{1.671690in}}{\pgfqpoint{3.980838in}{1.671690in}}%
\pgfpathcurveto{\pgfqpoint{3.975313in}{1.671690in}}{\pgfqpoint{3.970014in}{1.669494in}}{\pgfqpoint{3.966107in}{1.665588in}}%
\pgfpathcurveto{\pgfqpoint{3.962200in}{1.661681in}}{\pgfqpoint{3.960005in}{1.656381in}}{\pgfqpoint{3.960005in}{1.650856in}}%
\pgfpathcurveto{\pgfqpoint{3.960005in}{1.645331in}}{\pgfqpoint{3.962200in}{1.640032in}}{\pgfqpoint{3.966107in}{1.636125in}}%
\pgfpathcurveto{\pgfqpoint{3.970014in}{1.632218in}}{\pgfqpoint{3.975313in}{1.630023in}}{\pgfqpoint{3.980838in}{1.630023in}}%
\pgfpathclose%
\pgfusepath{fill}%
\end{pgfscope}%
\begin{pgfscope}%
\pgfpathrectangle{\pgfqpoint{3.214225in}{0.913832in}}{\pgfqpoint{1.162500in}{0.755000in}}%
\pgfusepath{clip}%
\pgfsetbuttcap%
\pgfsetroundjoin%
\definecolor{currentfill}{rgb}{0.000000,0.000000,0.000000}%
\pgfsetfillcolor{currentfill}%
\pgfsetfillopacity{0.500000}%
\pgfsetlinewidth{0.000000pt}%
\definecolor{currentstroke}{rgb}{0.000000,0.000000,0.000000}%
\pgfsetstrokecolor{currentstroke}%
\pgfsetdash{}{0pt}%
\pgfpathmoveto{\pgfqpoint{3.708643in}{1.484185in}}%
\pgfpathcurveto{\pgfqpoint{3.714168in}{1.484185in}}{\pgfqpoint{3.719468in}{1.486380in}}{\pgfqpoint{3.723375in}{1.490287in}}%
\pgfpathcurveto{\pgfqpoint{3.727282in}{1.494194in}}{\pgfqpoint{3.729477in}{1.499493in}}{\pgfqpoint{3.729477in}{1.505018in}}%
\pgfpathcurveto{\pgfqpoint{3.729477in}{1.510544in}}{\pgfqpoint{3.727282in}{1.515843in}}{\pgfqpoint{3.723375in}{1.519750in}}%
\pgfpathcurveto{\pgfqpoint{3.719468in}{1.523657in}}{\pgfqpoint{3.714168in}{1.525852in}}{\pgfqpoint{3.708643in}{1.525852in}}%
\pgfpathcurveto{\pgfqpoint{3.703118in}{1.525852in}}{\pgfqpoint{3.697819in}{1.523657in}}{\pgfqpoint{3.693912in}{1.519750in}}%
\pgfpathcurveto{\pgfqpoint{3.690005in}{1.515843in}}{\pgfqpoint{3.687810in}{1.510544in}}{\pgfqpoint{3.687810in}{1.505018in}}%
\pgfpathcurveto{\pgfqpoint{3.687810in}{1.499493in}}{\pgfqpoint{3.690005in}{1.494194in}}{\pgfqpoint{3.693912in}{1.490287in}}%
\pgfpathcurveto{\pgfqpoint{3.697819in}{1.486380in}}{\pgfqpoint{3.703118in}{1.484185in}}{\pgfqpoint{3.708643in}{1.484185in}}%
\pgfpathclose%
\pgfusepath{fill}%
\end{pgfscope}%
\begin{pgfscope}%
\pgfpathrectangle{\pgfqpoint{3.214225in}{0.913832in}}{\pgfqpoint{1.162500in}{0.755000in}}%
\pgfusepath{clip}%
\pgfsetbuttcap%
\pgfsetroundjoin%
\definecolor{currentfill}{rgb}{0.000000,0.000000,0.000000}%
\pgfsetfillcolor{currentfill}%
\pgfsetfillopacity{0.500000}%
\pgfsetlinewidth{0.000000pt}%
\definecolor{currentstroke}{rgb}{0.000000,0.000000,0.000000}%
\pgfsetstrokecolor{currentstroke}%
\pgfsetdash{}{0pt}%
\pgfpathmoveto{\pgfqpoint{4.124371in}{1.067768in}}%
\pgfpathcurveto{\pgfqpoint{4.129896in}{1.067768in}}{\pgfqpoint{4.135195in}{1.069963in}}{\pgfqpoint{4.139102in}{1.073870in}}%
\pgfpathcurveto{\pgfqpoint{4.143009in}{1.077776in}}{\pgfqpoint{4.145204in}{1.083076in}}{\pgfqpoint{4.145204in}{1.088601in}}%
\pgfpathcurveto{\pgfqpoint{4.145204in}{1.094126in}}{\pgfqpoint{4.143009in}{1.099426in}}{\pgfqpoint{4.139102in}{1.103332in}}%
\pgfpathcurveto{\pgfqpoint{4.135195in}{1.107239in}}{\pgfqpoint{4.129896in}{1.109434in}}{\pgfqpoint{4.124371in}{1.109434in}}%
\pgfpathcurveto{\pgfqpoint{4.118846in}{1.109434in}}{\pgfqpoint{4.113546in}{1.107239in}}{\pgfqpoint{4.109639in}{1.103332in}}%
\pgfpathcurveto{\pgfqpoint{4.105732in}{1.099426in}}{\pgfqpoint{4.103537in}{1.094126in}}{\pgfqpoint{4.103537in}{1.088601in}}%
\pgfpathcurveto{\pgfqpoint{4.103537in}{1.083076in}}{\pgfqpoint{4.105732in}{1.077776in}}{\pgfqpoint{4.109639in}{1.073870in}}%
\pgfpathcurveto{\pgfqpoint{4.113546in}{1.069963in}}{\pgfqpoint{4.118846in}{1.067768in}}{\pgfqpoint{4.124371in}{1.067768in}}%
\pgfpathclose%
\pgfusepath{fill}%
\end{pgfscope}%
\begin{pgfscope}%
\pgfpathrectangle{\pgfqpoint{3.214225in}{0.913832in}}{\pgfqpoint{1.162500in}{0.755000in}}%
\pgfusepath{clip}%
\pgfsetbuttcap%
\pgfsetroundjoin%
\definecolor{currentfill}{rgb}{0.000000,0.000000,0.000000}%
\pgfsetfillcolor{currentfill}%
\pgfsetfillopacity{0.500000}%
\pgfsetlinewidth{0.000000pt}%
\definecolor{currentstroke}{rgb}{0.000000,0.000000,0.000000}%
\pgfsetstrokecolor{currentstroke}%
\pgfsetdash{}{0pt}%
\pgfpathmoveto{\pgfqpoint{3.695649in}{1.343150in}}%
\pgfpathcurveto{\pgfqpoint{3.701174in}{1.343150in}}{\pgfqpoint{3.706474in}{1.345345in}}{\pgfqpoint{3.710381in}{1.349252in}}%
\pgfpathcurveto{\pgfqpoint{3.714288in}{1.353159in}}{\pgfqpoint{3.716483in}{1.358458in}}{\pgfqpoint{3.716483in}{1.363984in}}%
\pgfpathcurveto{\pgfqpoint{3.716483in}{1.369509in}}{\pgfqpoint{3.714288in}{1.374808in}}{\pgfqpoint{3.710381in}{1.378715in}}%
\pgfpathcurveto{\pgfqpoint{3.706474in}{1.382622in}}{\pgfqpoint{3.701174in}{1.384817in}}{\pgfqpoint{3.695649in}{1.384817in}}%
\pgfpathcurveto{\pgfqpoint{3.690124in}{1.384817in}}{\pgfqpoint{3.684825in}{1.382622in}}{\pgfqpoint{3.680918in}{1.378715in}}%
\pgfpathcurveto{\pgfqpoint{3.677011in}{1.374808in}}{\pgfqpoint{3.674816in}{1.369509in}}{\pgfqpoint{3.674816in}{1.363984in}}%
\pgfpathcurveto{\pgfqpoint{3.674816in}{1.358458in}}{\pgfqpoint{3.677011in}{1.353159in}}{\pgfqpoint{3.680918in}{1.349252in}}%
\pgfpathcurveto{\pgfqpoint{3.684825in}{1.345345in}}{\pgfqpoint{3.690124in}{1.343150in}}{\pgfqpoint{3.695649in}{1.343150in}}%
\pgfpathclose%
\pgfusepath{fill}%
\end{pgfscope}%
\begin{pgfscope}%
\pgfpathrectangle{\pgfqpoint{3.214225in}{0.913832in}}{\pgfqpoint{1.162500in}{0.755000in}}%
\pgfusepath{clip}%
\pgfsetbuttcap%
\pgfsetroundjoin%
\definecolor{currentfill}{rgb}{0.000000,0.000000,0.000000}%
\pgfsetfillcolor{currentfill}%
\pgfsetfillopacity{0.500000}%
\pgfsetlinewidth{0.000000pt}%
\definecolor{currentstroke}{rgb}{0.000000,0.000000,0.000000}%
\pgfsetstrokecolor{currentstroke}%
\pgfsetdash{}{0pt}%
\pgfpathmoveto{\pgfqpoint{3.329643in}{1.259718in}}%
\pgfpathcurveto{\pgfqpoint{3.335168in}{1.259718in}}{\pgfqpoint{3.340468in}{1.261913in}}{\pgfqpoint{3.344375in}{1.265820in}}%
\pgfpathcurveto{\pgfqpoint{3.348281in}{1.269726in}}{\pgfqpoint{3.350476in}{1.275026in}}{\pgfqpoint{3.350476in}{1.280551in}}%
\pgfpathcurveto{\pgfqpoint{3.350476in}{1.286076in}}{\pgfqpoint{3.348281in}{1.291376in}}{\pgfqpoint{3.344375in}{1.295282in}}%
\pgfpathcurveto{\pgfqpoint{3.340468in}{1.299189in}}{\pgfqpoint{3.335168in}{1.301384in}}{\pgfqpoint{3.329643in}{1.301384in}}%
\pgfpathcurveto{\pgfqpoint{3.324118in}{1.301384in}}{\pgfqpoint{3.318819in}{1.299189in}}{\pgfqpoint{3.314912in}{1.295282in}}%
\pgfpathcurveto{\pgfqpoint{3.311005in}{1.291376in}}{\pgfqpoint{3.308810in}{1.286076in}}{\pgfqpoint{3.308810in}{1.280551in}}%
\pgfpathcurveto{\pgfqpoint{3.308810in}{1.275026in}}{\pgfqpoint{3.311005in}{1.269726in}}{\pgfqpoint{3.314912in}{1.265820in}}%
\pgfpathcurveto{\pgfqpoint{3.318819in}{1.261913in}}{\pgfqpoint{3.324118in}{1.259718in}}{\pgfqpoint{3.329643in}{1.259718in}}%
\pgfpathclose%
\pgfusepath{fill}%
\end{pgfscope}%
\begin{pgfscope}%
\pgfpathrectangle{\pgfqpoint{3.214225in}{0.913832in}}{\pgfqpoint{1.162500in}{0.755000in}}%
\pgfusepath{clip}%
\pgfsetbuttcap%
\pgfsetroundjoin%
\definecolor{currentfill}{rgb}{0.000000,0.000000,0.000000}%
\pgfsetfillcolor{currentfill}%
\pgfsetfillopacity{0.500000}%
\pgfsetlinewidth{0.000000pt}%
\definecolor{currentstroke}{rgb}{0.000000,0.000000,0.000000}%
\pgfsetstrokecolor{currentstroke}%
\pgfsetdash{}{0pt}%
\pgfpathmoveto{\pgfqpoint{4.167862in}{1.167350in}}%
\pgfpathcurveto{\pgfqpoint{4.173387in}{1.167350in}}{\pgfqpoint{4.178687in}{1.169545in}}{\pgfqpoint{4.182593in}{1.173452in}}%
\pgfpathcurveto{\pgfqpoint{4.186500in}{1.177359in}}{\pgfqpoint{4.188695in}{1.182658in}}{\pgfqpoint{4.188695in}{1.188183in}}%
\pgfpathcurveto{\pgfqpoint{4.188695in}{1.193708in}}{\pgfqpoint{4.186500in}{1.199008in}}{\pgfqpoint{4.182593in}{1.202915in}}%
\pgfpathcurveto{\pgfqpoint{4.178687in}{1.206822in}}{\pgfqpoint{4.173387in}{1.209017in}}{\pgfqpoint{4.167862in}{1.209017in}}%
\pgfpathcurveto{\pgfqpoint{4.162337in}{1.209017in}}{\pgfqpoint{4.157037in}{1.206822in}}{\pgfqpoint{4.153131in}{1.202915in}}%
\pgfpathcurveto{\pgfqpoint{4.149224in}{1.199008in}}{\pgfqpoint{4.147029in}{1.193708in}}{\pgfqpoint{4.147029in}{1.188183in}}%
\pgfpathcurveto{\pgfqpoint{4.147029in}{1.182658in}}{\pgfqpoint{4.149224in}{1.177359in}}{\pgfqpoint{4.153131in}{1.173452in}}%
\pgfpathcurveto{\pgfqpoint{4.157037in}{1.169545in}}{\pgfqpoint{4.162337in}{1.167350in}}{\pgfqpoint{4.167862in}{1.167350in}}%
\pgfpathclose%
\pgfusepath{fill}%
\end{pgfscope}%
\begin{pgfscope}%
\pgfpathrectangle{\pgfqpoint{3.214225in}{0.913832in}}{\pgfqpoint{1.162500in}{0.755000in}}%
\pgfusepath{clip}%
\pgfsetbuttcap%
\pgfsetroundjoin%
\definecolor{currentfill}{rgb}{0.000000,0.000000,0.000000}%
\pgfsetfillcolor{currentfill}%
\pgfsetfillopacity{0.500000}%
\pgfsetlinewidth{0.000000pt}%
\definecolor{currentstroke}{rgb}{0.000000,0.000000,0.000000}%
\pgfsetstrokecolor{currentstroke}%
\pgfsetdash{}{0pt}%
\pgfpathmoveto{\pgfqpoint{4.120816in}{1.034902in}}%
\pgfpathcurveto{\pgfqpoint{4.126341in}{1.034902in}}{\pgfqpoint{4.131641in}{1.037097in}}{\pgfqpoint{4.135548in}{1.041004in}}%
\pgfpathcurveto{\pgfqpoint{4.139455in}{1.044911in}}{\pgfqpoint{4.141650in}{1.050211in}}{\pgfqpoint{4.141650in}{1.055736in}}%
\pgfpathcurveto{\pgfqpoint{4.141650in}{1.061261in}}{\pgfqpoint{4.139455in}{1.066560in}}{\pgfqpoint{4.135548in}{1.070467in}}%
\pgfpathcurveto{\pgfqpoint{4.131641in}{1.074374in}}{\pgfqpoint{4.126341in}{1.076569in}}{\pgfqpoint{4.120816in}{1.076569in}}%
\pgfpathcurveto{\pgfqpoint{4.115291in}{1.076569in}}{\pgfqpoint{4.109992in}{1.074374in}}{\pgfqpoint{4.106085in}{1.070467in}}%
\pgfpathcurveto{\pgfqpoint{4.102178in}{1.066560in}}{\pgfqpoint{4.099983in}{1.061261in}}{\pgfqpoint{4.099983in}{1.055736in}}%
\pgfpathcurveto{\pgfqpoint{4.099983in}{1.050211in}}{\pgfqpoint{4.102178in}{1.044911in}}{\pgfqpoint{4.106085in}{1.041004in}}%
\pgfpathcurveto{\pgfqpoint{4.109992in}{1.037097in}}{\pgfqpoint{4.115291in}{1.034902in}}{\pgfqpoint{4.120816in}{1.034902in}}%
\pgfpathclose%
\pgfusepath{fill}%
\end{pgfscope}%
\begin{pgfscope}%
\pgfpathrectangle{\pgfqpoint{3.214225in}{0.913832in}}{\pgfqpoint{1.162500in}{0.755000in}}%
\pgfusepath{clip}%
\pgfsetbuttcap%
\pgfsetroundjoin%
\definecolor{currentfill}{rgb}{0.000000,0.000000,0.000000}%
\pgfsetfillcolor{currentfill}%
\pgfsetfillopacity{0.500000}%
\pgfsetlinewidth{0.000000pt}%
\definecolor{currentstroke}{rgb}{0.000000,0.000000,0.000000}%
\pgfsetstrokecolor{currentstroke}%
\pgfsetdash{}{0pt}%
\pgfpathmoveto{\pgfqpoint{3.792736in}{1.190157in}}%
\pgfpathcurveto{\pgfqpoint{3.798261in}{1.190157in}}{\pgfqpoint{3.803560in}{1.192352in}}{\pgfqpoint{3.807467in}{1.196259in}}%
\pgfpathcurveto{\pgfqpoint{3.811374in}{1.200166in}}{\pgfqpoint{3.813569in}{1.205465in}}{\pgfqpoint{3.813569in}{1.210991in}}%
\pgfpathcurveto{\pgfqpoint{3.813569in}{1.216516in}}{\pgfqpoint{3.811374in}{1.221815in}}{\pgfqpoint{3.807467in}{1.225722in}}%
\pgfpathcurveto{\pgfqpoint{3.803560in}{1.229629in}}{\pgfqpoint{3.798261in}{1.231824in}}{\pgfqpoint{3.792736in}{1.231824in}}%
\pgfpathcurveto{\pgfqpoint{3.787211in}{1.231824in}}{\pgfqpoint{3.781911in}{1.229629in}}{\pgfqpoint{3.778004in}{1.225722in}}%
\pgfpathcurveto{\pgfqpoint{3.774097in}{1.221815in}}{\pgfqpoint{3.771902in}{1.216516in}}{\pgfqpoint{3.771902in}{1.210991in}}%
\pgfpathcurveto{\pgfqpoint{3.771902in}{1.205465in}}{\pgfqpoint{3.774097in}{1.200166in}}{\pgfqpoint{3.778004in}{1.196259in}}%
\pgfpathcurveto{\pgfqpoint{3.781911in}{1.192352in}}{\pgfqpoint{3.787211in}{1.190157in}}{\pgfqpoint{3.792736in}{1.190157in}}%
\pgfpathclose%
\pgfusepath{fill}%
\end{pgfscope}%
\begin{pgfscope}%
\pgfsetbuttcap%
\pgfsetroundjoin%
\definecolor{currentfill}{rgb}{0.000000,0.000000,0.000000}%
\pgfsetfillcolor{currentfill}%
\pgfsetlinewidth{0.803000pt}%
\definecolor{currentstroke}{rgb}{0.000000,0.000000,0.000000}%
\pgfsetstrokecolor{currentstroke}%
\pgfsetdash{}{0pt}%
\pgfsys@defobject{currentmarker}{\pgfqpoint{0.000000in}{-0.048611in}}{\pgfqpoint{0.000000in}{0.000000in}}{%
\pgfpathmoveto{\pgfqpoint{0.000000in}{0.000000in}}%
\pgfpathlineto{\pgfqpoint{0.000000in}{-0.048611in}}%
\pgfusepath{stroke,fill}%
}%
\begin{pgfscope}%
\pgfsys@transformshift{3.342043in}{0.913832in}%
\pgfsys@useobject{currentmarker}{}%
\end{pgfscope}%
\end{pgfscope}%
\begin{pgfscope}%
\pgftext[x=3.372697in,y=0.547702in,left,base,rotate=90.000000]{\rmfamily\fontsize{8.000000}{9.600000}\selectfont \(\displaystyle 0.025\)}%
\end{pgfscope}%
\begin{pgfscope}%
\pgfsetbuttcap%
\pgfsetroundjoin%
\definecolor{currentfill}{rgb}{0.000000,0.000000,0.000000}%
\pgfsetfillcolor{currentfill}%
\pgfsetlinewidth{0.803000pt}%
\definecolor{currentstroke}{rgb}{0.000000,0.000000,0.000000}%
\pgfsetstrokecolor{currentstroke}%
\pgfsetdash{}{0pt}%
\pgfsys@defobject{currentmarker}{\pgfqpoint{0.000000in}{-0.048611in}}{\pgfqpoint{0.000000in}{0.000000in}}{%
\pgfpathmoveto{\pgfqpoint{0.000000in}{0.000000in}}%
\pgfpathlineto{\pgfqpoint{0.000000in}{-0.048611in}}%
\pgfusepath{stroke,fill}%
}%
\begin{pgfscope}%
\pgfsys@transformshift{3.752399in}{0.913832in}%
\pgfsys@useobject{currentmarker}{}%
\end{pgfscope}%
\end{pgfscope}%
\begin{pgfscope}%
\pgftext[x=3.783053in,y=0.547702in,left,base,rotate=90.000000]{\rmfamily\fontsize{8.000000}{9.600000}\selectfont \(\displaystyle 0.050\)}%
\end{pgfscope}%
\begin{pgfscope}%
\pgfsetbuttcap%
\pgfsetroundjoin%
\definecolor{currentfill}{rgb}{0.000000,0.000000,0.000000}%
\pgfsetfillcolor{currentfill}%
\pgfsetlinewidth{0.803000pt}%
\definecolor{currentstroke}{rgb}{0.000000,0.000000,0.000000}%
\pgfsetstrokecolor{currentstroke}%
\pgfsetdash{}{0pt}%
\pgfsys@defobject{currentmarker}{\pgfqpoint{0.000000in}{-0.048611in}}{\pgfqpoint{0.000000in}{0.000000in}}{%
\pgfpathmoveto{\pgfqpoint{0.000000in}{0.000000in}}%
\pgfpathlineto{\pgfqpoint{0.000000in}{-0.048611in}}%
\pgfusepath{stroke,fill}%
}%
\begin{pgfscope}%
\pgfsys@transformshift{4.162756in}{0.913832in}%
\pgfsys@useobject{currentmarker}{}%
\end{pgfscope}%
\end{pgfscope}%
\begin{pgfscope}%
\pgftext[x=4.193409in,y=0.547702in,left,base,rotate=90.000000]{\rmfamily\fontsize{8.000000}{9.600000}\selectfont \(\displaystyle 0.075\)}%
\end{pgfscope}%
\begin{pgfscope}%
\pgftext[x=3.795475in,y=0.492146in,,top]{\rmfamily\fontsize{16.000000}{19.200000}\selectfont u0}%
\end{pgfscope}%
\begin{pgfscope}%
\pgfsetrectcap%
\pgfsetmiterjoin%
\pgfsetlinewidth{0.803000pt}%
\definecolor{currentstroke}{rgb}{0.501961,0.501961,0.501961}%
\pgfsetstrokecolor{currentstroke}%
\pgfsetdash{}{0pt}%
\pgfpathmoveto{\pgfqpoint{3.214225in}{0.913832in}}%
\pgfpathlineto{\pgfqpoint{3.214225in}{1.668832in}}%
\pgfusepath{stroke}%
\end{pgfscope}%
\begin{pgfscope}%
\pgfsetrectcap%
\pgfsetmiterjoin%
\pgfsetlinewidth{0.803000pt}%
\definecolor{currentstroke}{rgb}{0.501961,0.501961,0.501961}%
\pgfsetstrokecolor{currentstroke}%
\pgfsetdash{}{0pt}%
\pgfpathmoveto{\pgfqpoint{4.376725in}{0.913832in}}%
\pgfpathlineto{\pgfqpoint{4.376725in}{1.668832in}}%
\pgfusepath{stroke}%
\end{pgfscope}%
\begin{pgfscope}%
\pgfsetrectcap%
\pgfsetmiterjoin%
\pgfsetlinewidth{0.803000pt}%
\definecolor{currentstroke}{rgb}{0.501961,0.501961,0.501961}%
\pgfsetstrokecolor{currentstroke}%
\pgfsetdash{}{0pt}%
\pgfpathmoveto{\pgfqpoint{3.214225in}{0.913832in}}%
\pgfpathlineto{\pgfqpoint{4.376725in}{0.913832in}}%
\pgfusepath{stroke}%
\end{pgfscope}%
\begin{pgfscope}%
\pgfsetrectcap%
\pgfsetmiterjoin%
\pgfsetlinewidth{0.803000pt}%
\definecolor{currentstroke}{rgb}{0.501961,0.501961,0.501961}%
\pgfsetstrokecolor{currentstroke}%
\pgfsetdash{}{0pt}%
\pgfpathmoveto{\pgfqpoint{3.214225in}{1.668832in}}%
\pgfpathlineto{\pgfqpoint{4.376725in}{1.668832in}}%
\pgfusepath{stroke}%
\end{pgfscope}%
\begin{pgfscope}%
\pgfsetbuttcap%
\pgfsetmiterjoin%
\definecolor{currentfill}{rgb}{1.000000,1.000000,1.000000}%
\pgfsetfillcolor{currentfill}%
\pgfsetlinewidth{0.000000pt}%
\definecolor{currentstroke}{rgb}{0.000000,0.000000,0.000000}%
\pgfsetstrokecolor{currentstroke}%
\pgfsetstrokeopacity{0.000000}%
\pgfsetdash{}{0pt}%
\pgfpathmoveto{\pgfqpoint{4.376725in}{0.913832in}}%
\pgfpathlineto{\pgfqpoint{5.539225in}{0.913832in}}%
\pgfpathlineto{\pgfqpoint{5.539225in}{1.668832in}}%
\pgfpathlineto{\pgfqpoint{4.376725in}{1.668832in}}%
\pgfpathclose%
\pgfusepath{fill}%
\end{pgfscope}%
\begin{pgfscope}%
\pgfsetbuttcap%
\pgfsetroundjoin%
\definecolor{currentfill}{rgb}{0.000000,0.000000,0.000000}%
\pgfsetfillcolor{currentfill}%
\pgfsetlinewidth{0.803000pt}%
\definecolor{currentstroke}{rgb}{0.000000,0.000000,0.000000}%
\pgfsetstrokecolor{currentstroke}%
\pgfsetdash{}{0pt}%
\pgfsys@defobject{currentmarker}{\pgfqpoint{0.000000in}{-0.048611in}}{\pgfqpoint{0.000000in}{0.000000in}}{%
\pgfpathmoveto{\pgfqpoint{0.000000in}{0.000000in}}%
\pgfpathlineto{\pgfqpoint{0.000000in}{-0.048611in}}%
\pgfusepath{stroke,fill}%
}%
\begin{pgfscope}%
\pgfsys@transformshift{4.455518in}{0.913832in}%
\pgfsys@useobject{currentmarker}{}%
\end{pgfscope}%
\end{pgfscope}%
\begin{pgfscope}%
\pgftext[x=4.486171in,y=0.665759in,left,base,rotate=90.000000]{\rmfamily\fontsize{8.000000}{9.600000}\selectfont \(\displaystyle 0.5\)}%
\end{pgfscope}%
\begin{pgfscope}%
\pgfsetbuttcap%
\pgfsetroundjoin%
\definecolor{currentfill}{rgb}{0.000000,0.000000,0.000000}%
\pgfsetfillcolor{currentfill}%
\pgfsetlinewidth{0.803000pt}%
\definecolor{currentstroke}{rgb}{0.000000,0.000000,0.000000}%
\pgfsetstrokecolor{currentstroke}%
\pgfsetdash{}{0pt}%
\pgfsys@defobject{currentmarker}{\pgfqpoint{0.000000in}{-0.048611in}}{\pgfqpoint{0.000000in}{0.000000in}}{%
\pgfpathmoveto{\pgfqpoint{0.000000in}{0.000000in}}%
\pgfpathlineto{\pgfqpoint{0.000000in}{-0.048611in}}%
\pgfusepath{stroke,fill}%
}%
\begin{pgfscope}%
\pgfsys@transformshift{4.909793in}{0.913832in}%
\pgfsys@useobject{currentmarker}{}%
\end{pgfscope}%
\end{pgfscope}%
\begin{pgfscope}%
\pgftext[x=4.940447in,y=0.665759in,left,base,rotate=90.000000]{\rmfamily\fontsize{8.000000}{9.600000}\selectfont \(\displaystyle 1.0\)}%
\end{pgfscope}%
\begin{pgfscope}%
\pgfsetbuttcap%
\pgfsetroundjoin%
\definecolor{currentfill}{rgb}{0.000000,0.000000,0.000000}%
\pgfsetfillcolor{currentfill}%
\pgfsetlinewidth{0.803000pt}%
\definecolor{currentstroke}{rgb}{0.000000,0.000000,0.000000}%
\pgfsetstrokecolor{currentstroke}%
\pgfsetdash{}{0pt}%
\pgfsys@defobject{currentmarker}{\pgfqpoint{0.000000in}{-0.048611in}}{\pgfqpoint{0.000000in}{0.000000in}}{%
\pgfpathmoveto{\pgfqpoint{0.000000in}{0.000000in}}%
\pgfpathlineto{\pgfqpoint{0.000000in}{-0.048611in}}%
\pgfusepath{stroke,fill}%
}%
\begin{pgfscope}%
\pgfsys@transformshift{5.364069in}{0.913832in}%
\pgfsys@useobject{currentmarker}{}%
\end{pgfscope}%
\end{pgfscope}%
\begin{pgfscope}%
\pgftext[x=5.394722in,y=0.665759in,left,base,rotate=90.000000]{\rmfamily\fontsize{8.000000}{9.600000}\selectfont \(\displaystyle 1.5\)}%
\end{pgfscope}%
\begin{pgfscope}%
\pgftext[x=4.957975in,y=0.610204in,,top]{\rmfamily\fontsize{16.000000}{19.200000}\selectfont Ef0}%
\end{pgfscope}%
\begin{pgfscope}%
\pgfpathrectangle{\pgfqpoint{4.376725in}{0.913832in}}{\pgfqpoint{1.162500in}{0.755000in}}%
\pgfusepath{clip}%
\pgfsetrectcap%
\pgfsetroundjoin%
\pgfsetlinewidth{1.505625pt}%
\definecolor{currentstroke}{rgb}{0.121569,0.466667,0.705882}%
\pgfsetstrokecolor{currentstroke}%
\pgfsetdash{}{0pt}%
\pgfpathmoveto{\pgfqpoint{4.404404in}{1.292854in}}%
\pgfpathlineto{\pgfqpoint{4.443192in}{1.385258in}}%
\pgfpathlineto{\pgfqpoint{4.470899in}{1.445863in}}%
\pgfpathlineto{\pgfqpoint{4.495280in}{1.493940in}}%
\pgfpathlineto{\pgfqpoint{4.516337in}{1.530674in}}%
\pgfpathlineto{\pgfqpoint{4.535177in}{1.559293in}}%
\pgfpathlineto{\pgfqpoint{4.552909in}{1.582261in}}%
\pgfpathlineto{\pgfqpoint{4.569533in}{1.600110in}}%
\pgfpathlineto{\pgfqpoint{4.585049in}{1.613447in}}%
\pgfpathlineto{\pgfqpoint{4.599456in}{1.622913in}}%
\pgfpathlineto{\pgfqpoint{4.612755in}{1.629148in}}%
\pgfpathlineto{\pgfqpoint{4.626054in}{1.632994in}}%
\pgfpathlineto{\pgfqpoint{4.639353in}{1.634490in}}%
\pgfpathlineto{\pgfqpoint{4.652652in}{1.633693in}}%
\pgfpathlineto{\pgfqpoint{4.665951in}{1.630682in}}%
\pgfpathlineto{\pgfqpoint{4.680358in}{1.625032in}}%
\pgfpathlineto{\pgfqpoint{4.695874in}{1.616332in}}%
\pgfpathlineto{\pgfqpoint{4.712497in}{1.604229in}}%
\pgfpathlineto{\pgfqpoint{4.730229in}{1.588458in}}%
\pgfpathlineto{\pgfqpoint{4.750178in}{1.567637in}}%
\pgfpathlineto{\pgfqpoint{4.773451in}{1.539943in}}%
\pgfpathlineto{\pgfqpoint{4.802266in}{1.501880in}}%
\pgfpathlineto{\pgfqpoint{4.844379in}{1.442067in}}%
\pgfpathlineto{\pgfqpoint{4.908658in}{1.351028in}}%
\pgfpathlineto{\pgfqpoint{4.941905in}{1.307935in}}%
\pgfpathlineto{\pgfqpoint{4.970720in}{1.274145in}}%
\pgfpathlineto{\pgfqpoint{4.997318in}{1.246340in}}%
\pgfpathlineto{\pgfqpoint{5.022808in}{1.222922in}}%
\pgfpathlineto{\pgfqpoint{5.047189in}{1.203484in}}%
\pgfpathlineto{\pgfqpoint{5.071571in}{1.186825in}}%
\pgfpathlineto{\pgfqpoint{5.097061in}{1.172140in}}%
\pgfpathlineto{\pgfqpoint{5.123659in}{1.159413in}}%
\pgfpathlineto{\pgfqpoint{5.153581in}{1.147648in}}%
\pgfpathlineto{\pgfqpoint{5.190154in}{1.135777in}}%
\pgfpathlineto{\pgfqpoint{5.293221in}{1.103752in}}%
\pgfpathlineto{\pgfqpoint{5.324252in}{1.090901in}}%
\pgfpathlineto{\pgfqpoint{5.353067in}{1.076511in}}%
\pgfpathlineto{\pgfqpoint{5.380773in}{1.060111in}}%
\pgfpathlineto{\pgfqpoint{5.408479in}{1.041034in}}%
\pgfpathlineto{\pgfqpoint{5.437294in}{1.018351in}}%
\pgfpathlineto{\pgfqpoint{5.467216in}{0.991904in}}%
\pgfpathlineto{\pgfqpoint{5.500464in}{0.959524in}}%
\pgfpathlineto{\pgfqpoint{5.511546in}{0.948151in}}%
\pgfpathlineto{\pgfqpoint{5.511546in}{0.948151in}}%
\pgfusepath{stroke}%
\end{pgfscope}%
\begin{pgfscope}%
\pgfsetrectcap%
\pgfsetmiterjoin%
\pgfsetlinewidth{0.803000pt}%
\definecolor{currentstroke}{rgb}{0.501961,0.501961,0.501961}%
\pgfsetstrokecolor{currentstroke}%
\pgfsetdash{}{0pt}%
\pgfpathmoveto{\pgfqpoint{4.376725in}{0.913832in}}%
\pgfpathlineto{\pgfqpoint{4.376725in}{1.668832in}}%
\pgfusepath{stroke}%
\end{pgfscope}%
\begin{pgfscope}%
\pgfsetrectcap%
\pgfsetmiterjoin%
\pgfsetlinewidth{0.803000pt}%
\definecolor{currentstroke}{rgb}{0.501961,0.501961,0.501961}%
\pgfsetstrokecolor{currentstroke}%
\pgfsetdash{}{0pt}%
\pgfpathmoveto{\pgfqpoint{5.539225in}{0.913832in}}%
\pgfpathlineto{\pgfqpoint{5.539225in}{1.668832in}}%
\pgfusepath{stroke}%
\end{pgfscope}%
\begin{pgfscope}%
\pgfsetrectcap%
\pgfsetmiterjoin%
\pgfsetlinewidth{0.803000pt}%
\definecolor{currentstroke}{rgb}{0.501961,0.501961,0.501961}%
\pgfsetstrokecolor{currentstroke}%
\pgfsetdash{}{0pt}%
\pgfpathmoveto{\pgfqpoint{4.376725in}{0.913832in}}%
\pgfpathlineto{\pgfqpoint{5.539225in}{0.913832in}}%
\pgfusepath{stroke}%
\end{pgfscope}%
\begin{pgfscope}%
\pgfsetrectcap%
\pgfsetmiterjoin%
\pgfsetlinewidth{0.803000pt}%
\definecolor{currentstroke}{rgb}{0.501961,0.501961,0.501961}%
\pgfsetstrokecolor{currentstroke}%
\pgfsetdash{}{0pt}%
\pgfpathmoveto{\pgfqpoint{4.376725in}{1.668832in}}%
\pgfpathlineto{\pgfqpoint{5.539225in}{1.668832in}}%
\pgfusepath{stroke}%
\end{pgfscope}%
\end{pgfpicture}%
\makeatother%
\endgroup%
}
    \caption{Covariance plot of main parameters $E_0$, $U_0$, droplet area ($V_d^{2/3})$, and $q$.\label{fig:scatter}}
\end{figure}
\begin{figure}[H]
    \centering
    %% Creator: Matplotlib, PGF backend
%%
%% To include the figure in your LaTeX document, write
%%   \input{<filename>.pgf}
%%
%% Make sure the required packages are loaded in your preamble
%%   \usepackage{pgf}
%%
%% Figures using additional raster images can only be included by \input if
%% they are in the same directory as the main LaTeX file. For loading figures
%% from other directories you can use the `import` package
%%   \usepackage{import}
%% and then include the figures with
%%   \import{<path to file>}{<filename>.pgf}
%%
%% Matplotlib used the following preamble
%%   \usepackage{fontspec}
%%   \setmainfont{DejaVuSerif.ttf}[Path=/home/erin/anaconda3/lib/python3.6/site-packages/matplotlib/mpl-data/fonts/ttf/]
%%   \setsansfont{DejaVuSans.ttf}[Path=/home/erin/anaconda3/lib/python3.6/site-packages/matplotlib/mpl-data/fonts/ttf/]
%%   \setmonofont{DejaVuSansMono.ttf}[Path=/home/erin/anaconda3/lib/python3.6/site-packages/matplotlib/mpl-data/fonts/ttf/]
%%
\begingroup%
\makeatletter%
\begin{pgfpicture}%
\pgfpathrectangle{\pgfpointorigin}{\pgfqpoint{5.232821in}{3.788793in}}%
\pgfusepath{use as bounding box, clip}%
\begin{pgfscope}%
\pgfsetbuttcap%
\pgfsetmiterjoin%
\definecolor{currentfill}{rgb}{1.000000,1.000000,1.000000}%
\pgfsetfillcolor{currentfill}%
\pgfsetlinewidth{0.000000pt}%
\definecolor{currentstroke}{rgb}{1.000000,1.000000,1.000000}%
\pgfsetstrokecolor{currentstroke}%
\pgfsetdash{}{0pt}%
\pgfpathmoveto{\pgfqpoint{0.000000in}{0.000000in}}%
\pgfpathlineto{\pgfqpoint{5.232821in}{0.000000in}}%
\pgfpathlineto{\pgfqpoint{5.232821in}{3.788793in}}%
\pgfpathlineto{\pgfqpoint{0.000000in}{3.788793in}}%
\pgfpathclose%
\pgfusepath{fill}%
\end{pgfscope}%
\begin{pgfscope}%
\pgfsetbuttcap%
\pgfsetmiterjoin%
\definecolor{currentfill}{rgb}{1.000000,1.000000,1.000000}%
\pgfsetfillcolor{currentfill}%
\pgfsetlinewidth{0.000000pt}%
\definecolor{currentstroke}{rgb}{0.000000,0.000000,0.000000}%
\pgfsetstrokecolor{currentstroke}%
\pgfsetstrokeopacity{0.000000}%
\pgfsetdash{}{0pt}%
\pgfpathmoveto{\pgfqpoint{0.564660in}{0.521603in}}%
\pgfpathlineto{\pgfqpoint{4.284660in}{0.521603in}}%
\pgfpathlineto{\pgfqpoint{4.284660in}{3.541603in}}%
\pgfpathlineto{\pgfqpoint{0.564660in}{3.541603in}}%
\pgfpathclose%
\pgfusepath{fill}%
\end{pgfscope}%
\begin{pgfscope}%
\pgfpathrectangle{\pgfqpoint{0.564660in}{0.521603in}}{\pgfqpoint{3.720000in}{3.020000in}}%
\pgfusepath{clip}%
\pgfsetbuttcap%
\pgfsetroundjoin%
\definecolor{currentfill}{rgb}{0.061765,0.061765,0.085934}%
\pgfsetfillcolor{currentfill}%
\pgfsetlinewidth{0.000000pt}%
\definecolor{currentstroke}{rgb}{0.000000,0.000000,0.000000}%
\pgfsetstrokecolor{currentstroke}%
\pgfsetdash{}{0pt}%
\pgfpathmoveto{\pgfqpoint{1.238132in}{0.750263in}}%
\pgfpathlineto{\pgfqpoint{1.305934in}{0.750263in}}%
\pgfpathlineto{\pgfqpoint{1.373735in}{0.750263in}}%
\pgfpathlineto{\pgfqpoint{1.441537in}{0.750263in}}%
\pgfpathlineto{\pgfqpoint{1.509338in}{0.750263in}}%
\pgfpathlineto{\pgfqpoint{1.577140in}{0.804788in}}%
\pgfpathlineto{\pgfqpoint{1.644942in}{0.804788in}}%
\pgfpathlineto{\pgfqpoint{1.712743in}{0.804788in}}%
\pgfpathlineto{\pgfqpoint{1.780545in}{0.859313in}}%
\pgfpathlineto{\pgfqpoint{1.848347in}{0.859313in}}%
\pgfpathlineto{\pgfqpoint{1.916148in}{0.913838in}}%
\pgfpathlineto{\pgfqpoint{1.983950in}{0.913838in}}%
\pgfpathlineto{\pgfqpoint{2.051751in}{0.913838in}}%
\pgfpathlineto{\pgfqpoint{2.119553in}{0.968363in}}%
\pgfpathlineto{\pgfqpoint{2.187355in}{0.968363in}}%
\pgfpathlineto{\pgfqpoint{2.255156in}{1.022888in}}%
\pgfpathlineto{\pgfqpoint{2.264819in}{1.022888in}}%
\pgfpathlineto{\pgfqpoint{2.255156in}{1.025047in}}%
\pgfpathlineto{\pgfqpoint{2.187355in}{1.025510in}}%
\pgfpathlineto{\pgfqpoint{2.119553in}{1.024502in}}%
\pgfpathlineto{\pgfqpoint{2.078575in}{1.022888in}}%
\pgfpathlineto{\pgfqpoint{2.051751in}{1.021485in}}%
\pgfpathlineto{\pgfqpoint{1.983950in}{1.014951in}}%
\pgfpathlineto{\pgfqpoint{1.916148in}{1.008416in}}%
\pgfpathlineto{\pgfqpoint{1.885240in}{1.022888in}}%
\pgfpathlineto{\pgfqpoint{1.848347in}{1.047771in}}%
\pgfpathlineto{\pgfqpoint{1.805159in}{1.077414in}}%
\pgfpathlineto{\pgfqpoint{1.780545in}{1.090266in}}%
\pgfpathlineto{\pgfqpoint{1.712743in}{1.125668in}}%
\pgfpathlineto{\pgfqpoint{1.700735in}{1.131939in}}%
\pgfpathlineto{\pgfqpoint{1.644942in}{1.161071in}}%
\pgfpathlineto{\pgfqpoint{1.613824in}{1.186464in}}%
\pgfpathlineto{\pgfqpoint{1.577140in}{1.212962in}}%
\pgfpathlineto{\pgfqpoint{1.532843in}{1.240989in}}%
\pgfpathlineto{\pgfqpoint{1.509338in}{1.260474in}}%
\pgfpathlineto{\pgfqpoint{1.441537in}{1.291166in}}%
\pgfpathlineto{\pgfqpoint{1.430114in}{1.295514in}}%
\pgfpathlineto{\pgfqpoint{1.373735in}{1.316975in}}%
\pgfpathlineto{\pgfqpoint{1.305934in}{1.342785in}}%
\pgfpathlineto{\pgfqpoint{1.282780in}{1.295514in}}%
\pgfpathlineto{\pgfqpoint{1.272931in}{1.240989in}}%
\pgfpathlineto{\pgfqpoint{1.268059in}{1.186464in}}%
\pgfpathlineto{\pgfqpoint{1.263374in}{1.131939in}}%
\pgfpathlineto{\pgfqpoint{1.249947in}{1.077414in}}%
\pgfpathlineto{\pgfqpoint{1.238132in}{1.033194in}}%
\pgfpathlineto{\pgfqpoint{1.170330in}{1.023200in}}%
\pgfpathlineto{\pgfqpoint{1.102529in}{1.027231in}}%
\pgfpathlineto{\pgfqpoint{1.034727in}{1.031263in}}%
\pgfpathlineto{\pgfqpoint{0.966926in}{1.035294in}}%
\pgfpathlineto{\pgfqpoint{0.899124in}{1.039326in}}%
\pgfpathlineto{\pgfqpoint{0.831322in}{1.043357in}}%
\pgfpathlineto{\pgfqpoint{0.831322in}{1.022888in}}%
\pgfpathlineto{\pgfqpoint{0.899124in}{0.968363in}}%
\pgfpathlineto{\pgfqpoint{0.966926in}{0.913838in}}%
\pgfpathlineto{\pgfqpoint{1.034727in}{0.913838in}}%
\pgfpathlineto{\pgfqpoint{1.102529in}{0.859313in}}%
\pgfpathlineto{\pgfqpoint{1.170330in}{0.804788in}}%
\pgfpathclose%
\pgfusepath{fill}%
\end{pgfscope}%
\begin{pgfscope}%
\pgfpathrectangle{\pgfqpoint{0.564660in}{0.521603in}}{\pgfqpoint{3.720000in}{3.020000in}}%
\pgfusepath{clip}%
\pgfsetbuttcap%
\pgfsetroundjoin%
\definecolor{currentfill}{rgb}{0.185294,0.185294,0.257801}%
\pgfsetfillcolor{currentfill}%
\pgfsetlinewidth{0.000000pt}%
\definecolor{currentstroke}{rgb}{0.000000,0.000000,0.000000}%
\pgfsetstrokecolor{currentstroke}%
\pgfsetdash{}{0pt}%
\pgfpathmoveto{\pgfqpoint{1.916148in}{1.008416in}}%
\pgfpathlineto{\pgfqpoint{1.983950in}{1.014951in}}%
\pgfpathlineto{\pgfqpoint{2.051751in}{1.021485in}}%
\pgfpathlineto{\pgfqpoint{2.078575in}{1.022888in}}%
\pgfpathlineto{\pgfqpoint{2.119553in}{1.024502in}}%
\pgfpathlineto{\pgfqpoint{2.187355in}{1.025510in}}%
\pgfpathlineto{\pgfqpoint{2.255156in}{1.025047in}}%
\pgfpathlineto{\pgfqpoint{2.264819in}{1.022888in}}%
\pgfpathlineto{\pgfqpoint{2.322958in}{1.022888in}}%
\pgfpathlineto{\pgfqpoint{2.390759in}{1.077414in}}%
\pgfpathlineto{\pgfqpoint{2.458561in}{1.131939in}}%
\pgfpathlineto{\pgfqpoint{2.526363in}{1.186464in}}%
\pgfpathlineto{\pgfqpoint{2.594164in}{1.186464in}}%
\pgfpathlineto{\pgfqpoint{2.661966in}{1.240989in}}%
\pgfpathlineto{\pgfqpoint{2.729768in}{1.295514in}}%
\pgfpathlineto{\pgfqpoint{2.756979in}{1.317397in}}%
\pgfpathlineto{\pgfqpoint{2.729768in}{1.323475in}}%
\pgfpathlineto{\pgfqpoint{2.661966in}{1.338621in}}%
\pgfpathlineto{\pgfqpoint{2.610848in}{1.350039in}}%
\pgfpathlineto{\pgfqpoint{2.594164in}{1.353766in}}%
\pgfpathlineto{\pgfqpoint{2.526363in}{1.368911in}}%
\pgfpathlineto{\pgfqpoint{2.458561in}{1.384057in}}%
\pgfpathlineto{\pgfqpoint{2.390759in}{1.399202in}}%
\pgfpathlineto{\pgfqpoint{2.366754in}{1.404564in}}%
\pgfpathlineto{\pgfqpoint{2.322958in}{1.414347in}}%
\pgfpathlineto{\pgfqpoint{2.255156in}{1.429493in}}%
\pgfpathlineto{\pgfqpoint{2.187355in}{1.444638in}}%
\pgfpathlineto{\pgfqpoint{2.119553in}{1.452237in}}%
\pgfpathlineto{\pgfqpoint{2.051751in}{1.451228in}}%
\pgfpathlineto{\pgfqpoint{1.983950in}{1.450219in}}%
\pgfpathlineto{\pgfqpoint{1.916148in}{1.449211in}}%
\pgfpathlineto{\pgfqpoint{1.905252in}{1.459089in}}%
\pgfpathlineto{\pgfqpoint{1.848347in}{1.505259in}}%
\pgfpathlineto{\pgfqpoint{1.840382in}{1.513615in}}%
\pgfpathlineto{\pgfqpoint{1.788412in}{1.568140in}}%
\pgfpathlineto{\pgfqpoint{1.780545in}{1.576394in}}%
\pgfpathlineto{\pgfqpoint{1.736442in}{1.622665in}}%
\pgfpathlineto{\pgfqpoint{1.712743in}{1.647529in}}%
\pgfpathlineto{\pgfqpoint{1.684473in}{1.677190in}}%
\pgfpathlineto{\pgfqpoint{1.644942in}{1.718664in}}%
\pgfpathlineto{\pgfqpoint{1.631711in}{1.731715in}}%
\pgfpathlineto{\pgfqpoint{1.580533in}{1.786240in}}%
\pgfpathlineto{\pgfqpoint{1.577140in}{1.789800in}}%
\pgfpathlineto{\pgfqpoint{1.519093in}{1.840765in}}%
\pgfpathlineto{\pgfqpoint{1.509338in}{1.851646in}}%
\pgfpathlineto{\pgfqpoint{1.441537in}{1.890104in}}%
\pgfpathlineto{\pgfqpoint{1.433944in}{1.895291in}}%
\pgfpathlineto{\pgfqpoint{1.373735in}{1.931697in}}%
\pgfpathlineto{\pgfqpoint{1.351722in}{1.949816in}}%
\pgfpathlineto{\pgfqpoint{1.305934in}{1.987504in}}%
\pgfpathlineto{\pgfqpoint{1.285478in}{2.004341in}}%
\pgfpathlineto{\pgfqpoint{1.238132in}{2.043312in}}%
\pgfpathlineto{\pgfqpoint{1.196249in}{2.058866in}}%
\pgfpathlineto{\pgfqpoint{1.170330in}{2.067546in}}%
\pgfpathlineto{\pgfqpoint{1.102529in}{2.062611in}}%
\pgfpathlineto{\pgfqpoint{1.102529in}{2.058866in}}%
\pgfpathlineto{\pgfqpoint{1.102529in}{2.004341in}}%
\pgfpathlineto{\pgfqpoint{1.034727in}{1.949816in}}%
\pgfpathlineto{\pgfqpoint{1.034727in}{1.895291in}}%
\pgfpathlineto{\pgfqpoint{0.966926in}{1.840765in}}%
\pgfpathlineto{\pgfqpoint{0.966926in}{1.786240in}}%
\pgfpathlineto{\pgfqpoint{0.966926in}{1.731715in}}%
\pgfpathlineto{\pgfqpoint{0.966926in}{1.677190in}}%
\pgfpathlineto{\pgfqpoint{0.966926in}{1.622665in}}%
\pgfpathlineto{\pgfqpoint{0.899124in}{1.568140in}}%
\pgfpathlineto{\pgfqpoint{0.899124in}{1.513615in}}%
\pgfpathlineto{\pgfqpoint{0.899124in}{1.459089in}}%
\pgfpathlineto{\pgfqpoint{0.899124in}{1.404564in}}%
\pgfpathlineto{\pgfqpoint{0.899124in}{1.350039in}}%
\pgfpathlineto{\pgfqpoint{0.831322in}{1.295514in}}%
\pgfpathlineto{\pgfqpoint{0.831322in}{1.240989in}}%
\pgfpathlineto{\pgfqpoint{0.831322in}{1.186464in}}%
\pgfpathlineto{\pgfqpoint{0.831322in}{1.131939in}}%
\pgfpathlineto{\pgfqpoint{0.831322in}{1.077414in}}%
\pgfpathlineto{\pgfqpoint{0.831322in}{1.043357in}}%
\pgfpathlineto{\pgfqpoint{0.899124in}{1.039326in}}%
\pgfpathlineto{\pgfqpoint{0.966926in}{1.035294in}}%
\pgfpathlineto{\pgfqpoint{1.034727in}{1.031263in}}%
\pgfpathlineto{\pgfqpoint{1.102529in}{1.027231in}}%
\pgfpathlineto{\pgfqpoint{1.170330in}{1.023200in}}%
\pgfpathlineto{\pgfqpoint{1.238132in}{1.033194in}}%
\pgfpathlineto{\pgfqpoint{1.249947in}{1.077414in}}%
\pgfpathlineto{\pgfqpoint{1.263374in}{1.131939in}}%
\pgfpathlineto{\pgfqpoint{1.268059in}{1.186464in}}%
\pgfpathlineto{\pgfqpoint{1.272931in}{1.240989in}}%
\pgfpathlineto{\pgfqpoint{1.282780in}{1.295514in}}%
\pgfpathlineto{\pgfqpoint{1.305934in}{1.342785in}}%
\pgfpathlineto{\pgfqpoint{1.373735in}{1.316975in}}%
\pgfpathlineto{\pgfqpoint{1.430114in}{1.295514in}}%
\pgfpathlineto{\pgfqpoint{1.441537in}{1.291166in}}%
\pgfpathlineto{\pgfqpoint{1.509338in}{1.260474in}}%
\pgfpathlineto{\pgfqpoint{1.532843in}{1.240989in}}%
\pgfpathlineto{\pgfqpoint{1.577140in}{1.212962in}}%
\pgfpathlineto{\pgfqpoint{1.613824in}{1.186464in}}%
\pgfpathlineto{\pgfqpoint{1.644942in}{1.161071in}}%
\pgfpathlineto{\pgfqpoint{1.700735in}{1.131939in}}%
\pgfpathlineto{\pgfqpoint{1.712743in}{1.125668in}}%
\pgfpathlineto{\pgfqpoint{1.780545in}{1.090266in}}%
\pgfpathlineto{\pgfqpoint{1.805159in}{1.077414in}}%
\pgfpathlineto{\pgfqpoint{1.848347in}{1.047771in}}%
\pgfpathlineto{\pgfqpoint{1.885240in}{1.022888in}}%
\pgfpathclose%
\pgfpathmoveto{\pgfqpoint{1.091247in}{1.568140in}}%
\pgfpathlineto{\pgfqpoint{1.101834in}{1.622665in}}%
\pgfpathlineto{\pgfqpoint{1.102529in}{1.623097in}}%
\pgfpathlineto{\pgfqpoint{1.102695in}{1.622665in}}%
\pgfpathlineto{\pgfqpoint{1.105105in}{1.568140in}}%
\pgfpathlineto{\pgfqpoint{1.102529in}{1.554543in}}%
\pgfpathclose%
\pgfusepath{fill}%
\end{pgfscope}%
\begin{pgfscope}%
\pgfpathrectangle{\pgfqpoint{0.564660in}{0.521603in}}{\pgfqpoint{3.720000in}{3.020000in}}%
\pgfusepath{clip}%
\pgfsetbuttcap%
\pgfsetroundjoin%
\definecolor{currentfill}{rgb}{0.185294,0.185294,0.257801}%
\pgfsetfillcolor{currentfill}%
\pgfsetlinewidth{0.000000pt}%
\definecolor{currentstroke}{rgb}{0.000000,0.000000,0.000000}%
\pgfsetstrokecolor{currentstroke}%
\pgfsetdash{}{0pt}%
\pgfpathmoveto{\pgfqpoint{2.933172in}{1.404564in}}%
\pgfpathlineto{\pgfqpoint{2.939480in}{1.409637in}}%
\pgfpathlineto{\pgfqpoint{2.933172in}{1.408107in}}%
\pgfpathlineto{\pgfqpoint{2.927424in}{1.404564in}}%
\pgfpathclose%
\pgfusepath{fill}%
\end{pgfscope}%
\begin{pgfscope}%
\pgfpathrectangle{\pgfqpoint{0.564660in}{0.521603in}}{\pgfqpoint{3.720000in}{3.020000in}}%
\pgfusepath{clip}%
\pgfsetbuttcap%
\pgfsetroundjoin%
\definecolor{currentfill}{rgb}{0.312255,0.312255,0.434442}%
\pgfsetfillcolor{currentfill}%
\pgfsetlinewidth{0.000000pt}%
\definecolor{currentstroke}{rgb}{0.000000,0.000000,0.000000}%
\pgfsetstrokecolor{currentstroke}%
\pgfsetdash{}{0pt}%
\pgfpathmoveto{\pgfqpoint{2.661966in}{1.338621in}}%
\pgfpathlineto{\pgfqpoint{2.729768in}{1.323475in}}%
\pgfpathlineto{\pgfqpoint{2.756979in}{1.317397in}}%
\pgfpathlineto{\pgfqpoint{2.797569in}{1.350039in}}%
\pgfpathlineto{\pgfqpoint{2.865371in}{1.404564in}}%
\pgfpathlineto{\pgfqpoint{2.927424in}{1.404564in}}%
\pgfpathlineto{\pgfqpoint{2.933172in}{1.408107in}}%
\pgfpathlineto{\pgfqpoint{2.939480in}{1.409637in}}%
\pgfpathlineto{\pgfqpoint{3.000974in}{1.459089in}}%
\pgfpathlineto{\pgfqpoint{3.068776in}{1.513615in}}%
\pgfpathlineto{\pgfqpoint{3.136577in}{1.568140in}}%
\pgfpathlineto{\pgfqpoint{3.204379in}{1.568140in}}%
\pgfpathlineto{\pgfqpoint{3.272180in}{1.622665in}}%
\pgfpathlineto{\pgfqpoint{3.339982in}{1.677190in}}%
\pgfpathlineto{\pgfqpoint{3.407784in}{1.731715in}}%
\pgfpathlineto{\pgfqpoint{3.475585in}{1.731715in}}%
\pgfpathlineto{\pgfqpoint{3.543387in}{1.786240in}}%
\pgfpathlineto{\pgfqpoint{3.543387in}{1.840765in}}%
\pgfpathlineto{\pgfqpoint{3.611189in}{1.895291in}}%
\pgfpathlineto{\pgfqpoint{3.678990in}{1.949816in}}%
\pgfpathlineto{\pgfqpoint{3.746792in}{2.004341in}}%
\pgfpathlineto{\pgfqpoint{3.746792in}{2.058866in}}%
\pgfpathlineto{\pgfqpoint{3.814593in}{2.113391in}}%
\pgfpathlineto{\pgfqpoint{3.882395in}{2.167916in}}%
\pgfpathlineto{\pgfqpoint{3.950197in}{2.222441in}}%
\pgfpathlineto{\pgfqpoint{3.950197in}{2.276966in}}%
\pgfpathlineto{\pgfqpoint{4.017998in}{2.331492in}}%
\pgfpathlineto{\pgfqpoint{4.017998in}{2.339530in}}%
\pgfpathlineto{\pgfqpoint{4.006327in}{2.331492in}}%
\pgfpathlineto{\pgfqpoint{3.950197in}{2.292831in}}%
\pgfpathlineto{\pgfqpoint{3.927163in}{2.276966in}}%
\pgfpathlineto{\pgfqpoint{3.882395in}{2.246132in}}%
\pgfpathlineto{\pgfqpoint{3.847998in}{2.222441in}}%
\pgfpathlineto{\pgfqpoint{3.814593in}{2.199433in}}%
\pgfpathlineto{\pgfqpoint{3.768834in}{2.167916in}}%
\pgfpathlineto{\pgfqpoint{3.746792in}{2.152734in}}%
\pgfpathlineto{\pgfqpoint{3.689670in}{2.113391in}}%
\pgfpathlineto{\pgfqpoint{3.678990in}{2.106035in}}%
\pgfpathlineto{\pgfqpoint{3.611189in}{2.059336in}}%
\pgfpathlineto{\pgfqpoint{3.610505in}{2.058866in}}%
\pgfpathlineto{\pgfqpoint{3.543387in}{2.012638in}}%
\pgfpathlineto{\pgfqpoint{3.530445in}{2.004341in}}%
\pgfpathlineto{\pgfqpoint{3.475585in}{1.969319in}}%
\pgfpathlineto{\pgfqpoint{3.440935in}{1.949816in}}%
\pgfpathlineto{\pgfqpoint{3.407784in}{1.931156in}}%
\pgfpathlineto{\pgfqpoint{3.344064in}{1.895291in}}%
\pgfpathlineto{\pgfqpoint{3.339982in}{1.892993in}}%
\pgfpathlineto{\pgfqpoint{3.272180in}{1.854830in}}%
\pgfpathlineto{\pgfqpoint{3.247194in}{1.840765in}}%
\pgfpathlineto{\pgfqpoint{3.204379in}{1.816666in}}%
\pgfpathlineto{\pgfqpoint{3.150323in}{1.786240in}}%
\pgfpathlineto{\pgfqpoint{3.136577in}{1.778503in}}%
\pgfpathlineto{\pgfqpoint{3.068776in}{1.740340in}}%
\pgfpathlineto{\pgfqpoint{3.051071in}{1.731715in}}%
\pgfpathlineto{\pgfqpoint{3.000974in}{1.704773in}}%
\pgfpathlineto{\pgfqpoint{2.933172in}{1.707964in}}%
\pgfpathlineto{\pgfqpoint{2.865371in}{1.716699in}}%
\pgfpathlineto{\pgfqpoint{2.797569in}{1.725434in}}%
\pgfpathlineto{\pgfqpoint{2.748816in}{1.731715in}}%
\pgfpathlineto{\pgfqpoint{2.729768in}{1.734169in}}%
\pgfpathlineto{\pgfqpoint{2.661966in}{1.743597in}}%
\pgfpathlineto{\pgfqpoint{2.594164in}{1.758212in}}%
\pgfpathlineto{\pgfqpoint{2.526363in}{1.773357in}}%
\pgfpathlineto{\pgfqpoint{2.468688in}{1.786240in}}%
\pgfpathlineto{\pgfqpoint{2.458561in}{1.788502in}}%
\pgfpathlineto{\pgfqpoint{2.390759in}{1.803648in}}%
\pgfpathlineto{\pgfqpoint{2.322958in}{1.818793in}}%
\pgfpathlineto{\pgfqpoint{2.255156in}{1.833938in}}%
\pgfpathlineto{\pgfqpoint{2.224593in}{1.840765in}}%
\pgfpathlineto{\pgfqpoint{2.187355in}{1.849084in}}%
\pgfpathlineto{\pgfqpoint{2.119553in}{1.864229in}}%
\pgfpathlineto{\pgfqpoint{2.051751in}{1.878612in}}%
\pgfpathlineto{\pgfqpoint{1.983950in}{1.877955in}}%
\pgfpathlineto{\pgfqpoint{1.916148in}{1.876946in}}%
\pgfpathlineto{\pgfqpoint{1.848347in}{1.888029in}}%
\pgfpathlineto{\pgfqpoint{1.841425in}{1.895291in}}%
\pgfpathlineto{\pgfqpoint{1.789455in}{1.949816in}}%
\pgfpathlineto{\pgfqpoint{1.780545in}{1.959164in}}%
\pgfpathlineto{\pgfqpoint{1.737486in}{2.004341in}}%
\pgfpathlineto{\pgfqpoint{1.712743in}{2.030299in}}%
\pgfpathlineto{\pgfqpoint{1.685516in}{2.058866in}}%
\pgfpathlineto{\pgfqpoint{1.644942in}{2.101435in}}%
\pgfpathlineto{\pgfqpoint{1.633546in}{2.113391in}}%
\pgfpathlineto{\pgfqpoint{1.581576in}{2.167916in}}%
\pgfpathlineto{\pgfqpoint{1.577140in}{2.172570in}}%
\pgfpathlineto{\pgfqpoint{1.529606in}{2.222441in}}%
\pgfpathlineto{\pgfqpoint{1.509338in}{2.243705in}}%
\pgfpathlineto{\pgfqpoint{1.477636in}{2.276966in}}%
\pgfpathlineto{\pgfqpoint{1.441537in}{2.314840in}}%
\pgfpathlineto{\pgfqpoint{1.425666in}{2.331492in}}%
\pgfpathlineto{\pgfqpoint{1.373735in}{2.385975in}}%
\pgfpathlineto{\pgfqpoint{1.373696in}{2.386017in}}%
\pgfpathlineto{\pgfqpoint{1.321726in}{2.440542in}}%
\pgfpathlineto{\pgfqpoint{1.305934in}{2.446237in}}%
\pgfpathlineto{\pgfqpoint{1.305934in}{2.440542in}}%
\pgfpathlineto{\pgfqpoint{1.295552in}{2.432193in}}%
\pgfpathlineto{\pgfqpoint{1.276312in}{2.386017in}}%
\pgfpathlineto{\pgfqpoint{1.238132in}{2.363523in}}%
\pgfpathlineto{\pgfqpoint{1.238132in}{2.331492in}}%
\pgfpathlineto{\pgfqpoint{1.238132in}{2.276966in}}%
\pgfpathlineto{\pgfqpoint{1.170330in}{2.222441in}}%
\pgfpathlineto{\pgfqpoint{1.170330in}{2.167916in}}%
\pgfpathlineto{\pgfqpoint{1.102529in}{2.113391in}}%
\pgfpathlineto{\pgfqpoint{1.102529in}{2.062611in}}%
\pgfpathlineto{\pgfqpoint{1.170330in}{2.067546in}}%
\pgfpathlineto{\pgfqpoint{1.196249in}{2.058866in}}%
\pgfpathlineto{\pgfqpoint{1.238132in}{2.043312in}}%
\pgfpathlineto{\pgfqpoint{1.285478in}{2.004341in}}%
\pgfpathlineto{\pgfqpoint{1.305934in}{1.987504in}}%
\pgfpathlineto{\pgfqpoint{1.351722in}{1.949816in}}%
\pgfpathlineto{\pgfqpoint{1.373735in}{1.931697in}}%
\pgfpathlineto{\pgfqpoint{1.433944in}{1.895291in}}%
\pgfpathlineto{\pgfqpoint{1.441537in}{1.890104in}}%
\pgfpathlineto{\pgfqpoint{1.509338in}{1.851646in}}%
\pgfpathlineto{\pgfqpoint{1.519093in}{1.840765in}}%
\pgfpathlineto{\pgfqpoint{1.577140in}{1.789800in}}%
\pgfpathlineto{\pgfqpoint{1.580533in}{1.786240in}}%
\pgfpathlineto{\pgfqpoint{1.631711in}{1.731715in}}%
\pgfpathlineto{\pgfqpoint{1.644942in}{1.718664in}}%
\pgfpathlineto{\pgfqpoint{1.684473in}{1.677190in}}%
\pgfpathlineto{\pgfqpoint{1.712743in}{1.647529in}}%
\pgfpathlineto{\pgfqpoint{1.736442in}{1.622665in}}%
\pgfpathlineto{\pgfqpoint{1.780545in}{1.576394in}}%
\pgfpathlineto{\pgfqpoint{1.788412in}{1.568140in}}%
\pgfpathlineto{\pgfqpoint{1.840382in}{1.513615in}}%
\pgfpathlineto{\pgfqpoint{1.848347in}{1.505259in}}%
\pgfpathlineto{\pgfqpoint{1.905252in}{1.459089in}}%
\pgfpathlineto{\pgfqpoint{1.916148in}{1.449211in}}%
\pgfpathlineto{\pgfqpoint{1.983950in}{1.450219in}}%
\pgfpathlineto{\pgfqpoint{2.051751in}{1.451228in}}%
\pgfpathlineto{\pgfqpoint{2.119553in}{1.452237in}}%
\pgfpathlineto{\pgfqpoint{2.187355in}{1.444638in}}%
\pgfpathlineto{\pgfqpoint{2.255156in}{1.429493in}}%
\pgfpathlineto{\pgfqpoint{2.322958in}{1.414347in}}%
\pgfpathlineto{\pgfqpoint{2.366754in}{1.404564in}}%
\pgfpathlineto{\pgfqpoint{2.390759in}{1.399202in}}%
\pgfpathlineto{\pgfqpoint{2.458561in}{1.384057in}}%
\pgfpathlineto{\pgfqpoint{2.526363in}{1.368911in}}%
\pgfpathlineto{\pgfqpoint{2.594164in}{1.353766in}}%
\pgfpathlineto{\pgfqpoint{2.610848in}{1.350039in}}%
\pgfpathclose%
\pgfusepath{fill}%
\end{pgfscope}%
\begin{pgfscope}%
\pgfpathrectangle{\pgfqpoint{0.564660in}{0.521603in}}{\pgfqpoint{3.720000in}{3.020000in}}%
\pgfusepath{clip}%
\pgfsetbuttcap%
\pgfsetroundjoin%
\definecolor{currentfill}{rgb}{0.312255,0.312255,0.434442}%
\pgfsetfillcolor{currentfill}%
\pgfsetlinewidth{0.000000pt}%
\definecolor{currentstroke}{rgb}{0.000000,0.000000,0.000000}%
\pgfsetstrokecolor{currentstroke}%
\pgfsetdash{}{0pt}%
\pgfpathmoveto{\pgfqpoint{1.102529in}{1.554543in}}%
\pgfpathlineto{\pgfqpoint{1.105105in}{1.568140in}}%
\pgfpathlineto{\pgfqpoint{1.102695in}{1.622665in}}%
\pgfpathlineto{\pgfqpoint{1.102529in}{1.623097in}}%
\pgfpathlineto{\pgfqpoint{1.101834in}{1.622665in}}%
\pgfpathlineto{\pgfqpoint{1.091247in}{1.568140in}}%
\pgfpathclose%
\pgfusepath{fill}%
\end{pgfscope}%
\begin{pgfscope}%
\pgfpathrectangle{\pgfqpoint{0.564660in}{0.521603in}}{\pgfqpoint{3.720000in}{3.020000in}}%
\pgfusepath{clip}%
\pgfsetbuttcap%
\pgfsetroundjoin%
\definecolor{currentfill}{rgb}{0.439216,0.484130,0.564216}%
\pgfsetfillcolor{currentfill}%
\pgfsetlinewidth{0.000000pt}%
\definecolor{currentstroke}{rgb}{0.000000,0.000000,0.000000}%
\pgfsetstrokecolor{currentstroke}%
\pgfsetdash{}{0pt}%
\pgfpathmoveto{\pgfqpoint{2.797569in}{1.725434in}}%
\pgfpathlineto{\pgfqpoint{2.865371in}{1.716699in}}%
\pgfpathlineto{\pgfqpoint{2.933172in}{1.707964in}}%
\pgfpathlineto{\pgfqpoint{3.000974in}{1.704773in}}%
\pgfpathlineto{\pgfqpoint{3.051071in}{1.731715in}}%
\pgfpathlineto{\pgfqpoint{3.068776in}{1.740340in}}%
\pgfpathlineto{\pgfqpoint{3.136577in}{1.778503in}}%
\pgfpathlineto{\pgfqpoint{3.150323in}{1.786240in}}%
\pgfpathlineto{\pgfqpoint{3.204379in}{1.816666in}}%
\pgfpathlineto{\pgfqpoint{3.247194in}{1.840765in}}%
\pgfpathlineto{\pgfqpoint{3.272180in}{1.854830in}}%
\pgfpathlineto{\pgfqpoint{3.339982in}{1.892993in}}%
\pgfpathlineto{\pgfqpoint{3.344064in}{1.895291in}}%
\pgfpathlineto{\pgfqpoint{3.407784in}{1.931156in}}%
\pgfpathlineto{\pgfqpoint{3.440935in}{1.949816in}}%
\pgfpathlineto{\pgfqpoint{3.475585in}{1.969319in}}%
\pgfpathlineto{\pgfqpoint{3.530445in}{2.004341in}}%
\pgfpathlineto{\pgfqpoint{3.543387in}{2.012638in}}%
\pgfpathlineto{\pgfqpoint{3.610505in}{2.058866in}}%
\pgfpathlineto{\pgfqpoint{3.611189in}{2.059336in}}%
\pgfpathlineto{\pgfqpoint{3.678990in}{2.106035in}}%
\pgfpathlineto{\pgfqpoint{3.689670in}{2.113391in}}%
\pgfpathlineto{\pgfqpoint{3.746792in}{2.152734in}}%
\pgfpathlineto{\pgfqpoint{3.768834in}{2.167916in}}%
\pgfpathlineto{\pgfqpoint{3.814593in}{2.199433in}}%
\pgfpathlineto{\pgfqpoint{3.847998in}{2.222441in}}%
\pgfpathlineto{\pgfqpoint{3.882395in}{2.246132in}}%
\pgfpathlineto{\pgfqpoint{3.927163in}{2.276966in}}%
\pgfpathlineto{\pgfqpoint{3.950197in}{2.292831in}}%
\pgfpathlineto{\pgfqpoint{4.006327in}{2.331492in}}%
\pgfpathlineto{\pgfqpoint{4.017998in}{2.339530in}}%
\pgfpathlineto{\pgfqpoint{4.017998in}{2.386017in}}%
\pgfpathlineto{\pgfqpoint{3.950197in}{2.440542in}}%
\pgfpathlineto{\pgfqpoint{3.882395in}{2.495067in}}%
\pgfpathlineto{\pgfqpoint{3.882395in}{2.549592in}}%
\pgfpathlineto{\pgfqpoint{3.868264in}{2.560956in}}%
\pgfpathlineto{\pgfqpoint{3.851765in}{2.549592in}}%
\pgfpathlineto{\pgfqpoint{3.814593in}{2.523990in}}%
\pgfpathlineto{\pgfqpoint{3.772600in}{2.495067in}}%
\pgfpathlineto{\pgfqpoint{3.746792in}{2.477291in}}%
\pgfpathlineto{\pgfqpoint{3.693436in}{2.440542in}}%
\pgfpathlineto{\pgfqpoint{3.678990in}{2.430592in}}%
\pgfpathlineto{\pgfqpoint{3.614272in}{2.386017in}}%
\pgfpathlineto{\pgfqpoint{3.611189in}{2.383893in}}%
\pgfpathlineto{\pgfqpoint{3.543387in}{2.337194in}}%
\pgfpathlineto{\pgfqpoint{3.534764in}{2.331492in}}%
\pgfpathlineto{\pgfqpoint{3.475585in}{2.292518in}}%
\pgfpathlineto{\pgfqpoint{3.447956in}{2.276966in}}%
\pgfpathlineto{\pgfqpoint{3.407784in}{2.254355in}}%
\pgfpathlineto{\pgfqpoint{3.351085in}{2.222441in}}%
\pgfpathlineto{\pgfqpoint{3.339982in}{2.216192in}}%
\pgfpathlineto{\pgfqpoint{3.272180in}{2.178028in}}%
\pgfpathlineto{\pgfqpoint{3.254215in}{2.167916in}}%
\pgfpathlineto{\pgfqpoint{3.204379in}{2.139865in}}%
\pgfpathlineto{\pgfqpoint{3.139209in}{2.113391in}}%
\pgfpathlineto{\pgfqpoint{3.136577in}{2.112063in}}%
\pgfpathlineto{\pgfqpoint{3.125530in}{2.113391in}}%
\pgfpathlineto{\pgfqpoint{3.068776in}{2.120703in}}%
\pgfpathlineto{\pgfqpoint{3.000974in}{2.129438in}}%
\pgfpathlineto{\pgfqpoint{2.933172in}{2.138173in}}%
\pgfpathlineto{\pgfqpoint{2.865371in}{2.146908in}}%
\pgfpathlineto{\pgfqpoint{2.797569in}{2.155643in}}%
\pgfpathlineto{\pgfqpoint{2.729768in}{2.164378in}}%
\pgfpathlineto{\pgfqpoint{2.702308in}{2.167916in}}%
\pgfpathlineto{\pgfqpoint{2.661966in}{2.173114in}}%
\pgfpathlineto{\pgfqpoint{2.594164in}{2.181849in}}%
\pgfpathlineto{\pgfqpoint{2.526363in}{2.190584in}}%
\pgfpathlineto{\pgfqpoint{2.458561in}{2.199319in}}%
\pgfpathlineto{\pgfqpoint{2.390759in}{2.208709in}}%
\pgfpathlineto{\pgfqpoint{2.326733in}{2.222441in}}%
\pgfpathlineto{\pgfqpoint{2.322958in}{2.223239in}}%
\pgfpathlineto{\pgfqpoint{2.255156in}{2.238384in}}%
\pgfpathlineto{\pgfqpoint{2.187355in}{2.253529in}}%
\pgfpathlineto{\pgfqpoint{2.119553in}{2.268675in}}%
\pgfpathlineto{\pgfqpoint{2.082433in}{2.276966in}}%
\pgfpathlineto{\pgfqpoint{2.051751in}{2.283820in}}%
\pgfpathlineto{\pgfqpoint{1.983950in}{2.298965in}}%
\pgfpathlineto{\pgfqpoint{1.916148in}{2.304681in}}%
\pgfpathlineto{\pgfqpoint{1.848347in}{2.303672in}}%
\pgfpathlineto{\pgfqpoint{1.800582in}{2.331492in}}%
\pgfpathlineto{\pgfqpoint{1.780545in}{2.341934in}}%
\pgfpathlineto{\pgfqpoint{1.738529in}{2.386017in}}%
\pgfpathlineto{\pgfqpoint{1.712743in}{2.413070in}}%
\pgfpathlineto{\pgfqpoint{1.686559in}{2.440542in}}%
\pgfpathlineto{\pgfqpoint{1.644942in}{2.484205in}}%
\pgfpathlineto{\pgfqpoint{1.634589in}{2.495067in}}%
\pgfpathlineto{\pgfqpoint{1.582619in}{2.549592in}}%
\pgfpathlineto{\pgfqpoint{1.577140in}{2.555340in}}%
\pgfpathlineto{\pgfqpoint{1.530649in}{2.604117in}}%
\pgfpathlineto{\pgfqpoint{1.509338in}{2.626475in}}%
\pgfpathlineto{\pgfqpoint{1.478679in}{2.658642in}}%
\pgfpathlineto{\pgfqpoint{1.441537in}{2.697610in}}%
\pgfpathlineto{\pgfqpoint{1.404225in}{2.658642in}}%
\pgfpathlineto{\pgfqpoint{1.373735in}{2.649805in}}%
\pgfpathlineto{\pgfqpoint{1.373735in}{2.604117in}}%
\pgfpathlineto{\pgfqpoint{1.373735in}{2.549592in}}%
\pgfpathlineto{\pgfqpoint{1.305934in}{2.495067in}}%
\pgfpathlineto{\pgfqpoint{1.305934in}{2.446237in}}%
\pgfpathlineto{\pgfqpoint{1.321726in}{2.440542in}}%
\pgfpathlineto{\pgfqpoint{1.373696in}{2.386017in}}%
\pgfpathlineto{\pgfqpoint{1.373735in}{2.385975in}}%
\pgfpathlineto{\pgfqpoint{1.425666in}{2.331492in}}%
\pgfpathlineto{\pgfqpoint{1.441537in}{2.314840in}}%
\pgfpathlineto{\pgfqpoint{1.477636in}{2.276966in}}%
\pgfpathlineto{\pgfqpoint{1.509338in}{2.243705in}}%
\pgfpathlineto{\pgfqpoint{1.529606in}{2.222441in}}%
\pgfpathlineto{\pgfqpoint{1.577140in}{2.172570in}}%
\pgfpathlineto{\pgfqpoint{1.581576in}{2.167916in}}%
\pgfpathlineto{\pgfqpoint{1.633546in}{2.113391in}}%
\pgfpathlineto{\pgfqpoint{1.644942in}{2.101435in}}%
\pgfpathlineto{\pgfqpoint{1.685516in}{2.058866in}}%
\pgfpathlineto{\pgfqpoint{1.712743in}{2.030299in}}%
\pgfpathlineto{\pgfqpoint{1.737486in}{2.004341in}}%
\pgfpathlineto{\pgfqpoint{1.780545in}{1.959164in}}%
\pgfpathlineto{\pgfqpoint{1.789455in}{1.949816in}}%
\pgfpathlineto{\pgfqpoint{1.841425in}{1.895291in}}%
\pgfpathlineto{\pgfqpoint{1.848347in}{1.888029in}}%
\pgfpathlineto{\pgfqpoint{1.916148in}{1.876946in}}%
\pgfpathlineto{\pgfqpoint{1.983950in}{1.877955in}}%
\pgfpathlineto{\pgfqpoint{2.051751in}{1.878612in}}%
\pgfpathlineto{\pgfqpoint{2.119553in}{1.864229in}}%
\pgfpathlineto{\pgfqpoint{2.187355in}{1.849084in}}%
\pgfpathlineto{\pgfqpoint{2.224593in}{1.840765in}}%
\pgfpathlineto{\pgfqpoint{2.255156in}{1.833938in}}%
\pgfpathlineto{\pgfqpoint{2.322958in}{1.818793in}}%
\pgfpathlineto{\pgfqpoint{2.390759in}{1.803648in}}%
\pgfpathlineto{\pgfqpoint{2.458561in}{1.788502in}}%
\pgfpathlineto{\pgfqpoint{2.468688in}{1.786240in}}%
\pgfpathlineto{\pgfqpoint{2.526363in}{1.773357in}}%
\pgfpathlineto{\pgfqpoint{2.594164in}{1.758212in}}%
\pgfpathlineto{\pgfqpoint{2.661966in}{1.743597in}}%
\pgfpathlineto{\pgfqpoint{2.729768in}{1.734169in}}%
\pgfpathlineto{\pgfqpoint{2.748816in}{1.731715in}}%
\pgfpathclose%
\pgfusepath{fill}%
\end{pgfscope}%
\begin{pgfscope}%
\pgfpathrectangle{\pgfqpoint{0.564660in}{0.521603in}}{\pgfqpoint{3.720000in}{3.020000in}}%
\pgfusepath{clip}%
\pgfsetbuttcap%
\pgfsetroundjoin%
\definecolor{currentfill}{rgb}{0.439216,0.484130,0.564216}%
\pgfsetfillcolor{currentfill}%
\pgfsetlinewidth{0.000000pt}%
\definecolor{currentstroke}{rgb}{0.000000,0.000000,0.000000}%
\pgfsetstrokecolor{currentstroke}%
\pgfsetdash{}{0pt}%
\pgfpathmoveto{\pgfqpoint{1.276312in}{2.386017in}}%
\pgfpathlineto{\pgfqpoint{1.295552in}{2.432193in}}%
\pgfpathlineto{\pgfqpoint{1.238132in}{2.386017in}}%
\pgfpathlineto{\pgfqpoint{1.238132in}{2.363523in}}%
\pgfpathclose%
\pgfusepath{fill}%
\end{pgfscope}%
\begin{pgfscope}%
\pgfpathrectangle{\pgfqpoint{0.564660in}{0.521603in}}{\pgfqpoint{3.720000in}{3.020000in}}%
\pgfusepath{clip}%
\pgfsetbuttcap%
\pgfsetroundjoin%
\definecolor{currentfill}{rgb}{0.562745,0.653983,0.687745}%
\pgfsetfillcolor{currentfill}%
\pgfsetlinewidth{0.000000pt}%
\definecolor{currentstroke}{rgb}{0.000000,0.000000,0.000000}%
\pgfsetstrokecolor{currentstroke}%
\pgfsetdash{}{0pt}%
\pgfpathmoveto{\pgfqpoint{3.136577in}{2.112063in}}%
\pgfpathlineto{\pgfqpoint{3.139209in}{2.113391in}}%
\pgfpathlineto{\pgfqpoint{3.204379in}{2.139865in}}%
\pgfpathlineto{\pgfqpoint{3.254215in}{2.167916in}}%
\pgfpathlineto{\pgfqpoint{3.272180in}{2.178028in}}%
\pgfpathlineto{\pgfqpoint{3.339982in}{2.216192in}}%
\pgfpathlineto{\pgfqpoint{3.351085in}{2.222441in}}%
\pgfpathlineto{\pgfqpoint{3.407784in}{2.254355in}}%
\pgfpathlineto{\pgfqpoint{3.447956in}{2.276966in}}%
\pgfpathlineto{\pgfqpoint{3.475585in}{2.292518in}}%
\pgfpathlineto{\pgfqpoint{3.534764in}{2.331492in}}%
\pgfpathlineto{\pgfqpoint{3.543387in}{2.337194in}}%
\pgfpathlineto{\pgfqpoint{3.611189in}{2.383893in}}%
\pgfpathlineto{\pgfqpoint{3.614272in}{2.386017in}}%
\pgfpathlineto{\pgfqpoint{3.678990in}{2.430592in}}%
\pgfpathlineto{\pgfqpoint{3.693436in}{2.440542in}}%
\pgfpathlineto{\pgfqpoint{3.746792in}{2.477291in}}%
\pgfpathlineto{\pgfqpoint{3.772600in}{2.495067in}}%
\pgfpathlineto{\pgfqpoint{3.814593in}{2.523990in}}%
\pgfpathlineto{\pgfqpoint{3.851765in}{2.549592in}}%
\pgfpathlineto{\pgfqpoint{3.868264in}{2.560956in}}%
\pgfpathlineto{\pgfqpoint{3.814593in}{2.604117in}}%
\pgfpathlineto{\pgfqpoint{3.746792in}{2.658642in}}%
\pgfpathlineto{\pgfqpoint{3.678990in}{2.713167in}}%
\pgfpathlineto{\pgfqpoint{3.650870in}{2.735781in}}%
\pgfpathlineto{\pgfqpoint{3.618038in}{2.713167in}}%
\pgfpathlineto{\pgfqpoint{3.611189in}{2.708450in}}%
\pgfpathlineto{\pgfqpoint{3.543387in}{2.661751in}}%
\pgfpathlineto{\pgfqpoint{3.538827in}{2.658642in}}%
\pgfpathlineto{\pgfqpoint{3.475585in}{2.615717in}}%
\pgfpathlineto{\pgfqpoint{3.454977in}{2.604117in}}%
\pgfpathlineto{\pgfqpoint{3.407784in}{2.577554in}}%
\pgfpathlineto{\pgfqpoint{3.358106in}{2.549592in}}%
\pgfpathlineto{\pgfqpoint{3.339982in}{2.539390in}}%
\pgfpathlineto{\pgfqpoint{3.272180in}{2.524707in}}%
\pgfpathlineto{\pgfqpoint{3.204379in}{2.533442in}}%
\pgfpathlineto{\pgfqpoint{3.136577in}{2.542177in}}%
\pgfpathlineto{\pgfqpoint{3.079023in}{2.549592in}}%
\pgfpathlineto{\pgfqpoint{3.068776in}{2.550912in}}%
\pgfpathlineto{\pgfqpoint{3.000974in}{2.559647in}}%
\pgfpathlineto{\pgfqpoint{2.933172in}{2.568382in}}%
\pgfpathlineto{\pgfqpoint{2.865371in}{2.577118in}}%
\pgfpathlineto{\pgfqpoint{2.797569in}{2.585853in}}%
\pgfpathlineto{\pgfqpoint{2.729768in}{2.594588in}}%
\pgfpathlineto{\pgfqpoint{2.661966in}{2.603323in}}%
\pgfpathlineto{\pgfqpoint{2.655801in}{2.604117in}}%
\pgfpathlineto{\pgfqpoint{2.594164in}{2.612058in}}%
\pgfpathlineto{\pgfqpoint{2.526363in}{2.620793in}}%
\pgfpathlineto{\pgfqpoint{2.458561in}{2.629528in}}%
\pgfpathlineto{\pgfqpoint{2.390759in}{2.638263in}}%
\pgfpathlineto{\pgfqpoint{2.322958in}{2.646998in}}%
\pgfpathlineto{\pgfqpoint{2.255156in}{2.655734in}}%
\pgfpathlineto{\pgfqpoint{2.232579in}{2.658642in}}%
\pgfpathlineto{\pgfqpoint{2.187355in}{2.664469in}}%
\pgfpathlineto{\pgfqpoint{2.119553in}{2.673809in}}%
\pgfpathlineto{\pgfqpoint{2.051751in}{2.688266in}}%
\pgfpathlineto{\pgfqpoint{1.983950in}{2.703411in}}%
\pgfpathlineto{\pgfqpoint{1.940273in}{2.713167in}}%
\pgfpathlineto{\pgfqpoint{1.916148in}{2.718556in}}%
\pgfpathlineto{\pgfqpoint{1.848347in}{2.731408in}}%
\pgfpathlineto{\pgfqpoint{1.780545in}{2.730399in}}%
\pgfpathlineto{\pgfqpoint{1.743764in}{2.767693in}}%
\pgfpathlineto{\pgfqpoint{1.712743in}{2.795840in}}%
\pgfpathlineto{\pgfqpoint{1.687602in}{2.822218in}}%
\pgfpathlineto{\pgfqpoint{1.644942in}{2.866975in}}%
\pgfpathlineto{\pgfqpoint{1.635632in}{2.876743in}}%
\pgfpathlineto{\pgfqpoint{1.583662in}{2.931268in}}%
\pgfpathlineto{\pgfqpoint{1.577140in}{2.938110in}}%
\pgfpathlineto{\pgfqpoint{1.534395in}{2.951418in}}%
\pgfpathlineto{\pgfqpoint{1.509338in}{2.931268in}}%
\pgfpathlineto{\pgfqpoint{1.509338in}{2.876743in}}%
\pgfpathlineto{\pgfqpoint{1.509338in}{2.822218in}}%
\pgfpathlineto{\pgfqpoint{1.441537in}{2.767693in}}%
\pgfpathlineto{\pgfqpoint{1.441537in}{2.713167in}}%
\pgfpathlineto{\pgfqpoint{1.373735in}{2.658642in}}%
\pgfpathlineto{\pgfqpoint{1.373735in}{2.649805in}}%
\pgfpathlineto{\pgfqpoint{1.404225in}{2.658642in}}%
\pgfpathlineto{\pgfqpoint{1.441537in}{2.697610in}}%
\pgfpathlineto{\pgfqpoint{1.478679in}{2.658642in}}%
\pgfpathlineto{\pgfqpoint{1.509338in}{2.626475in}}%
\pgfpathlineto{\pgfqpoint{1.530649in}{2.604117in}}%
\pgfpathlineto{\pgfqpoint{1.577140in}{2.555340in}}%
\pgfpathlineto{\pgfqpoint{1.582619in}{2.549592in}}%
\pgfpathlineto{\pgfqpoint{1.634589in}{2.495067in}}%
\pgfpathlineto{\pgfqpoint{1.644942in}{2.484205in}}%
\pgfpathlineto{\pgfqpoint{1.686559in}{2.440542in}}%
\pgfpathlineto{\pgfqpoint{1.712743in}{2.413070in}}%
\pgfpathlineto{\pgfqpoint{1.738529in}{2.386017in}}%
\pgfpathlineto{\pgfqpoint{1.780545in}{2.341934in}}%
\pgfpathlineto{\pgfqpoint{1.800582in}{2.331492in}}%
\pgfpathlineto{\pgfqpoint{1.848347in}{2.303672in}}%
\pgfpathlineto{\pgfqpoint{1.916148in}{2.304681in}}%
\pgfpathlineto{\pgfqpoint{1.983950in}{2.298965in}}%
\pgfpathlineto{\pgfqpoint{2.051751in}{2.283820in}}%
\pgfpathlineto{\pgfqpoint{2.082433in}{2.276966in}}%
\pgfpathlineto{\pgfqpoint{2.119553in}{2.268675in}}%
\pgfpathlineto{\pgfqpoint{2.187355in}{2.253529in}}%
\pgfpathlineto{\pgfqpoint{2.255156in}{2.238384in}}%
\pgfpathlineto{\pgfqpoint{2.322958in}{2.223239in}}%
\pgfpathlineto{\pgfqpoint{2.326733in}{2.222441in}}%
\pgfpathlineto{\pgfqpoint{2.390759in}{2.208709in}}%
\pgfpathlineto{\pgfqpoint{2.458561in}{2.199319in}}%
\pgfpathlineto{\pgfqpoint{2.526363in}{2.190584in}}%
\pgfpathlineto{\pgfqpoint{2.594164in}{2.181849in}}%
\pgfpathlineto{\pgfqpoint{2.661966in}{2.173114in}}%
\pgfpathlineto{\pgfqpoint{2.702308in}{2.167916in}}%
\pgfpathlineto{\pgfqpoint{2.729768in}{2.164378in}}%
\pgfpathlineto{\pgfqpoint{2.797569in}{2.155643in}}%
\pgfpathlineto{\pgfqpoint{2.865371in}{2.146908in}}%
\pgfpathlineto{\pgfqpoint{2.933172in}{2.138173in}}%
\pgfpathlineto{\pgfqpoint{3.000974in}{2.129438in}}%
\pgfpathlineto{\pgfqpoint{3.068776in}{2.120703in}}%
\pgfpathlineto{\pgfqpoint{3.125530in}{2.113391in}}%
\pgfpathclose%
\pgfusepath{fill}%
\end{pgfscope}%
\begin{pgfscope}%
\pgfpathrectangle{\pgfqpoint{0.564660in}{0.521603in}}{\pgfqpoint{3.720000in}{3.020000in}}%
\pgfusepath{clip}%
\pgfsetbuttcap%
\pgfsetroundjoin%
\definecolor{currentfill}{rgb}{0.710478,0.814706,0.814706}%
\pgfsetfillcolor{currentfill}%
\pgfsetlinewidth{0.000000pt}%
\definecolor{currentstroke}{rgb}{0.000000,0.000000,0.000000}%
\pgfsetstrokecolor{currentstroke}%
\pgfsetdash{}{0pt}%
\pgfpathmoveto{\pgfqpoint{3.136577in}{2.542177in}}%
\pgfpathlineto{\pgfqpoint{3.204379in}{2.533442in}}%
\pgfpathlineto{\pgfqpoint{3.272180in}{2.524707in}}%
\pgfpathlineto{\pgfqpoint{3.339982in}{2.539390in}}%
\pgfpathlineto{\pgfqpoint{3.358106in}{2.549592in}}%
\pgfpathlineto{\pgfqpoint{3.407784in}{2.577554in}}%
\pgfpathlineto{\pgfqpoint{3.454977in}{2.604117in}}%
\pgfpathlineto{\pgfqpoint{3.475585in}{2.615717in}}%
\pgfpathlineto{\pgfqpoint{3.538827in}{2.658642in}}%
\pgfpathlineto{\pgfqpoint{3.543387in}{2.661751in}}%
\pgfpathlineto{\pgfqpoint{3.611189in}{2.708450in}}%
\pgfpathlineto{\pgfqpoint{3.618038in}{2.713167in}}%
\pgfpathlineto{\pgfqpoint{3.650870in}{2.735781in}}%
\pgfpathlineto{\pgfqpoint{3.611189in}{2.767693in}}%
\pgfpathlineto{\pgfqpoint{3.543387in}{2.822218in}}%
\pgfpathlineto{\pgfqpoint{3.543387in}{2.876743in}}%
\pgfpathlineto{\pgfqpoint{3.475585in}{2.931268in}}%
\pgfpathlineto{\pgfqpoint{3.407784in}{2.931268in}}%
\pgfpathlineto{\pgfqpoint{3.339982in}{2.931268in}}%
\pgfpathlineto{\pgfqpoint{3.317900in}{2.949026in}}%
\pgfpathlineto{\pgfqpoint{3.272180in}{2.954916in}}%
\pgfpathlineto{\pgfqpoint{3.204379in}{2.963651in}}%
\pgfpathlineto{\pgfqpoint{3.136577in}{2.972386in}}%
\pgfpathlineto{\pgfqpoint{3.068776in}{2.981122in}}%
\pgfpathlineto{\pgfqpoint{3.032515in}{2.985793in}}%
\pgfpathlineto{\pgfqpoint{3.000974in}{2.989857in}}%
\pgfpathlineto{\pgfqpoint{2.933172in}{2.998592in}}%
\pgfpathlineto{\pgfqpoint{2.865371in}{3.007327in}}%
\pgfpathlineto{\pgfqpoint{2.797569in}{3.016062in}}%
\pgfpathlineto{\pgfqpoint{2.729768in}{3.024797in}}%
\pgfpathlineto{\pgfqpoint{2.661966in}{3.033532in}}%
\pgfpathlineto{\pgfqpoint{2.609293in}{3.040318in}}%
\pgfpathlineto{\pgfqpoint{2.594164in}{3.042267in}}%
\pgfpathlineto{\pgfqpoint{2.526363in}{3.051002in}}%
\pgfpathlineto{\pgfqpoint{2.458561in}{3.059738in}}%
\pgfpathlineto{\pgfqpoint{2.390759in}{3.068473in}}%
\pgfpathlineto{\pgfqpoint{2.322958in}{3.077208in}}%
\pgfpathlineto{\pgfqpoint{2.255156in}{3.085943in}}%
\pgfpathlineto{\pgfqpoint{2.187355in}{3.094678in}}%
\pgfpathlineto{\pgfqpoint{2.186072in}{3.094843in}}%
\pgfpathlineto{\pgfqpoint{2.119553in}{3.103413in}}%
\pgfpathlineto{\pgfqpoint{2.051751in}{3.112148in}}%
\pgfpathlineto{\pgfqpoint{1.983950in}{3.120883in}}%
\pgfpathlineto{\pgfqpoint{1.916148in}{3.129618in}}%
\pgfpathlineto{\pgfqpoint{1.848347in}{3.138846in}}%
\pgfpathlineto{\pgfqpoint{1.799178in}{3.149368in}}%
\pgfpathlineto{\pgfqpoint{1.780545in}{3.153293in}}%
\pgfpathlineto{\pgfqpoint{1.712743in}{3.178610in}}%
\pgfpathlineto{\pgfqpoint{1.688645in}{3.203894in}}%
\pgfpathlineto{\pgfqpoint{1.669682in}{3.223789in}}%
\pgfpathlineto{\pgfqpoint{1.644942in}{3.203894in}}%
\pgfpathlineto{\pgfqpoint{1.644942in}{3.149368in}}%
\pgfpathlineto{\pgfqpoint{1.644942in}{3.094843in}}%
\pgfpathlineto{\pgfqpoint{1.577140in}{3.040318in}}%
\pgfpathlineto{\pgfqpoint{1.577140in}{2.985793in}}%
\pgfpathlineto{\pgfqpoint{1.534395in}{2.951418in}}%
\pgfpathlineto{\pgfqpoint{1.577140in}{2.938110in}}%
\pgfpathlineto{\pgfqpoint{1.583662in}{2.931268in}}%
\pgfpathlineto{\pgfqpoint{1.635632in}{2.876743in}}%
\pgfpathlineto{\pgfqpoint{1.644942in}{2.866975in}}%
\pgfpathlineto{\pgfqpoint{1.687602in}{2.822218in}}%
\pgfpathlineto{\pgfqpoint{1.712743in}{2.795840in}}%
\pgfpathlineto{\pgfqpoint{1.743764in}{2.767693in}}%
\pgfpathlineto{\pgfqpoint{1.780545in}{2.730399in}}%
\pgfpathlineto{\pgfqpoint{1.848347in}{2.731408in}}%
\pgfpathlineto{\pgfqpoint{1.916148in}{2.718556in}}%
\pgfpathlineto{\pgfqpoint{1.940273in}{2.713167in}}%
\pgfpathlineto{\pgfqpoint{1.983950in}{2.703411in}}%
\pgfpathlineto{\pgfqpoint{2.051751in}{2.688266in}}%
\pgfpathlineto{\pgfqpoint{2.119553in}{2.673809in}}%
\pgfpathlineto{\pgfqpoint{2.187355in}{2.664469in}}%
\pgfpathlineto{\pgfqpoint{2.232579in}{2.658642in}}%
\pgfpathlineto{\pgfqpoint{2.255156in}{2.655734in}}%
\pgfpathlineto{\pgfqpoint{2.322958in}{2.646998in}}%
\pgfpathlineto{\pgfqpoint{2.390759in}{2.638263in}}%
\pgfpathlineto{\pgfqpoint{2.458561in}{2.629528in}}%
\pgfpathlineto{\pgfqpoint{2.526363in}{2.620793in}}%
\pgfpathlineto{\pgfqpoint{2.594164in}{2.612058in}}%
\pgfpathlineto{\pgfqpoint{2.655801in}{2.604117in}}%
\pgfpathlineto{\pgfqpoint{2.661966in}{2.603323in}}%
\pgfpathlineto{\pgfqpoint{2.729768in}{2.594588in}}%
\pgfpathlineto{\pgfqpoint{2.797569in}{2.585853in}}%
\pgfpathlineto{\pgfqpoint{2.865371in}{2.577118in}}%
\pgfpathlineto{\pgfqpoint{2.933172in}{2.568382in}}%
\pgfpathlineto{\pgfqpoint{3.000974in}{2.559647in}}%
\pgfpathlineto{\pgfqpoint{3.068776in}{2.550912in}}%
\pgfpathlineto{\pgfqpoint{3.079023in}{2.549592in}}%
\pgfpathclose%
\pgfusepath{fill}%
\end{pgfscope}%
\begin{pgfscope}%
\pgfpathrectangle{\pgfqpoint{0.564660in}{0.521603in}}{\pgfqpoint{3.720000in}{3.020000in}}%
\pgfusepath{clip}%
\pgfsetbuttcap%
\pgfsetroundjoin%
\definecolor{currentfill}{rgb}{0.903493,0.938235,0.938235}%
\pgfsetfillcolor{currentfill}%
\pgfsetlinewidth{0.000000pt}%
\definecolor{currentstroke}{rgb}{0.000000,0.000000,0.000000}%
\pgfsetstrokecolor{currentstroke}%
\pgfsetdash{}{0pt}%
\pgfpathmoveto{\pgfqpoint{3.068776in}{2.981122in}}%
\pgfpathlineto{\pgfqpoint{3.136577in}{2.972386in}}%
\pgfpathlineto{\pgfqpoint{3.204379in}{2.963651in}}%
\pgfpathlineto{\pgfqpoint{3.272180in}{2.954916in}}%
\pgfpathlineto{\pgfqpoint{3.317900in}{2.949026in}}%
\pgfpathlineto{\pgfqpoint{3.272180in}{2.985793in}}%
\pgfpathlineto{\pgfqpoint{3.204379in}{2.985793in}}%
\pgfpathlineto{\pgfqpoint{3.136577in}{2.985793in}}%
\pgfpathlineto{\pgfqpoint{3.068776in}{3.040318in}}%
\pgfpathlineto{\pgfqpoint{3.000974in}{3.040318in}}%
\pgfpathlineto{\pgfqpoint{2.933172in}{3.040318in}}%
\pgfpathlineto{\pgfqpoint{2.865371in}{3.094843in}}%
\pgfpathlineto{\pgfqpoint{2.797569in}{3.094843in}}%
\pgfpathlineto{\pgfqpoint{2.729768in}{3.094843in}}%
\pgfpathlineto{\pgfqpoint{2.661966in}{3.094843in}}%
\pgfpathlineto{\pgfqpoint{2.594164in}{3.149368in}}%
\pgfpathlineto{\pgfqpoint{2.526363in}{3.149368in}}%
\pgfpathlineto{\pgfqpoint{2.458561in}{3.149368in}}%
\pgfpathlineto{\pgfqpoint{2.390759in}{3.203894in}}%
\pgfpathlineto{\pgfqpoint{2.322958in}{3.203894in}}%
\pgfpathlineto{\pgfqpoint{2.255156in}{3.203894in}}%
\pgfpathlineto{\pgfqpoint{2.187355in}{3.203894in}}%
\pgfpathlineto{\pgfqpoint{2.119553in}{3.258419in}}%
\pgfpathlineto{\pgfqpoint{2.051751in}{3.258419in}}%
\pgfpathlineto{\pgfqpoint{1.983950in}{3.258419in}}%
\pgfpathlineto{\pgfqpoint{1.916148in}{3.312944in}}%
\pgfpathlineto{\pgfqpoint{1.848347in}{3.312944in}}%
\pgfpathlineto{\pgfqpoint{1.780545in}{3.312944in}}%
\pgfpathlineto{\pgfqpoint{1.712743in}{3.312944in}}%
\pgfpathlineto{\pgfqpoint{1.712743in}{3.258419in}}%
\pgfpathlineto{\pgfqpoint{1.669682in}{3.223789in}}%
\pgfpathlineto{\pgfqpoint{1.688645in}{3.203894in}}%
\pgfpathlineto{\pgfqpoint{1.712743in}{3.178610in}}%
\pgfpathlineto{\pgfqpoint{1.780545in}{3.153293in}}%
\pgfpathlineto{\pgfqpoint{1.799178in}{3.149368in}}%
\pgfpathlineto{\pgfqpoint{1.848347in}{3.138846in}}%
\pgfpathlineto{\pgfqpoint{1.916148in}{3.129618in}}%
\pgfpathlineto{\pgfqpoint{1.983950in}{3.120883in}}%
\pgfpathlineto{\pgfqpoint{2.051751in}{3.112148in}}%
\pgfpathlineto{\pgfqpoint{2.119553in}{3.103413in}}%
\pgfpathlineto{\pgfqpoint{2.186072in}{3.094843in}}%
\pgfpathlineto{\pgfqpoint{2.187355in}{3.094678in}}%
\pgfpathlineto{\pgfqpoint{2.255156in}{3.085943in}}%
\pgfpathlineto{\pgfqpoint{2.322958in}{3.077208in}}%
\pgfpathlineto{\pgfqpoint{2.390759in}{3.068473in}}%
\pgfpathlineto{\pgfqpoint{2.458561in}{3.059738in}}%
\pgfpathlineto{\pgfqpoint{2.526363in}{3.051002in}}%
\pgfpathlineto{\pgfqpoint{2.594164in}{3.042267in}}%
\pgfpathlineto{\pgfqpoint{2.609293in}{3.040318in}}%
\pgfpathlineto{\pgfqpoint{2.661966in}{3.033532in}}%
\pgfpathlineto{\pgfqpoint{2.729768in}{3.024797in}}%
\pgfpathlineto{\pgfqpoint{2.797569in}{3.016062in}}%
\pgfpathlineto{\pgfqpoint{2.865371in}{3.007327in}}%
\pgfpathlineto{\pgfqpoint{2.933172in}{2.998592in}}%
\pgfpathlineto{\pgfqpoint{3.000974in}{2.989857in}}%
\pgfpathlineto{\pgfqpoint{3.032515in}{2.985793in}}%
\pgfpathclose%
\pgfusepath{fill}%
\end{pgfscope}%
\begin{pgfscope}%
\pgfpathrectangle{\pgfqpoint{0.564660in}{0.521603in}}{\pgfqpoint{3.720000in}{3.020000in}}%
\pgfusepath{clip}%
\pgfsetbuttcap%
\pgfsetroundjoin%
\definecolor{currentfill}{rgb}{1.000000,1.000000,1.000000}%
\pgfsetfillcolor{currentfill}%
\pgfsetlinewidth{1.003750pt}%
\definecolor{currentstroke}{rgb}{0.000000,0.000000,0.000000}%
\pgfsetstrokecolor{currentstroke}%
\pgfsetdash{}{0pt}%
\pgfpathmoveto{\pgfqpoint{3.469480in}{2.911879in}}%
\pgfpathcurveto{\pgfqpoint{3.480530in}{2.911879in}}{\pgfqpoint{3.491129in}{2.916269in}}{\pgfqpoint{3.498943in}{2.924083in}}%
\pgfpathcurveto{\pgfqpoint{3.506756in}{2.931896in}}{\pgfqpoint{3.511147in}{2.942495in}}{\pgfqpoint{3.511147in}{2.953545in}}%
\pgfpathcurveto{\pgfqpoint{3.511147in}{2.964596in}}{\pgfqpoint{3.506756in}{2.975195in}}{\pgfqpoint{3.498943in}{2.983008in}}%
\pgfpathcurveto{\pgfqpoint{3.491129in}{2.990822in}}{\pgfqpoint{3.480530in}{2.995212in}}{\pgfqpoint{3.469480in}{2.995212in}}%
\pgfpathcurveto{\pgfqpoint{3.458430in}{2.995212in}}{\pgfqpoint{3.447831in}{2.990822in}}{\pgfqpoint{3.440017in}{2.983008in}}%
\pgfpathcurveto{\pgfqpoint{3.432204in}{2.975195in}}{\pgfqpoint{3.427813in}{2.964596in}}{\pgfqpoint{3.427813in}{2.953545in}}%
\pgfpathcurveto{\pgfqpoint{3.427813in}{2.942495in}}{\pgfqpoint{3.432204in}{2.931896in}}{\pgfqpoint{3.440017in}{2.924083in}}%
\pgfpathcurveto{\pgfqpoint{3.447831in}{2.916269in}}{\pgfqpoint{3.458430in}{2.911879in}}{\pgfqpoint{3.469480in}{2.911879in}}%
\pgfpathclose%
\pgfusepath{stroke,fill}%
\end{pgfscope}%
\begin{pgfscope}%
\pgfpathrectangle{\pgfqpoint{0.564660in}{0.521603in}}{\pgfqpoint{3.720000in}{3.020000in}}%
\pgfusepath{clip}%
\pgfsetbuttcap%
\pgfsetroundjoin%
\definecolor{currentfill}{rgb}{1.000000,1.000000,1.000000}%
\pgfsetfillcolor{currentfill}%
\pgfsetlinewidth{1.003750pt}%
\definecolor{currentstroke}{rgb}{0.000000,0.000000,0.000000}%
\pgfsetstrokecolor{currentstroke}%
\pgfsetdash{}{0pt}%
\pgfpathmoveto{\pgfqpoint{1.171593in}{1.744492in}}%
\pgfpathcurveto{\pgfqpoint{1.182643in}{1.744492in}}{\pgfqpoint{1.193242in}{1.748882in}}{\pgfqpoint{1.201056in}{1.756696in}}%
\pgfpathcurveto{\pgfqpoint{1.208869in}{1.764510in}}{\pgfqpoint{1.213260in}{1.775109in}}{\pgfqpoint{1.213260in}{1.786159in}}%
\pgfpathcurveto{\pgfqpoint{1.213260in}{1.797209in}}{\pgfqpoint{1.208869in}{1.807808in}}{\pgfqpoint{1.201056in}{1.815622in}}%
\pgfpathcurveto{\pgfqpoint{1.193242in}{1.823435in}}{\pgfqpoint{1.182643in}{1.827826in}}{\pgfqpoint{1.171593in}{1.827826in}}%
\pgfpathcurveto{\pgfqpoint{1.160543in}{1.827826in}}{\pgfqpoint{1.149944in}{1.823435in}}{\pgfqpoint{1.142130in}{1.815622in}}%
\pgfpathcurveto{\pgfqpoint{1.134317in}{1.807808in}}{\pgfqpoint{1.129926in}{1.797209in}}{\pgfqpoint{1.129926in}{1.786159in}}%
\pgfpathcurveto{\pgfqpoint{1.129926in}{1.775109in}}{\pgfqpoint{1.134317in}{1.764510in}}{\pgfqpoint{1.142130in}{1.756696in}}%
\pgfpathcurveto{\pgfqpoint{1.149944in}{1.748882in}}{\pgfqpoint{1.160543in}{1.744492in}}{\pgfqpoint{1.171593in}{1.744492in}}%
\pgfpathclose%
\pgfusepath{stroke,fill}%
\end{pgfscope}%
\begin{pgfscope}%
\pgfpathrectangle{\pgfqpoint{0.564660in}{0.521603in}}{\pgfqpoint{3.720000in}{3.020000in}}%
\pgfusepath{clip}%
\pgfsetbuttcap%
\pgfsetroundjoin%
\definecolor{currentfill}{rgb}{1.000000,1.000000,1.000000}%
\pgfsetfillcolor{currentfill}%
\pgfsetlinewidth{1.003750pt}%
\definecolor{currentstroke}{rgb}{0.000000,0.000000,0.000000}%
\pgfsetstrokecolor{currentstroke}%
\pgfsetdash{}{0pt}%
\pgfpathmoveto{\pgfqpoint{0.941940in}{1.749672in}}%
\pgfpathcurveto{\pgfqpoint{0.952990in}{1.749672in}}{\pgfqpoint{0.963589in}{1.754062in}}{\pgfqpoint{0.971403in}{1.761876in}}%
\pgfpathcurveto{\pgfqpoint{0.979216in}{1.769689in}}{\pgfqpoint{0.983606in}{1.780289in}}{\pgfqpoint{0.983606in}{1.791339in}}%
\pgfpathcurveto{\pgfqpoint{0.983606in}{1.802389in}}{\pgfqpoint{0.979216in}{1.812988in}}{\pgfqpoint{0.971403in}{1.820801in}}%
\pgfpathcurveto{\pgfqpoint{0.963589in}{1.828615in}}{\pgfqpoint{0.952990in}{1.833005in}}{\pgfqpoint{0.941940in}{1.833005in}}%
\pgfpathcurveto{\pgfqpoint{0.930890in}{1.833005in}}{\pgfqpoint{0.920291in}{1.828615in}}{\pgfqpoint{0.912477in}{1.820801in}}%
\pgfpathcurveto{\pgfqpoint{0.904663in}{1.812988in}}{\pgfqpoint{0.900273in}{1.802389in}}{\pgfqpoint{0.900273in}{1.791339in}}%
\pgfpathcurveto{\pgfqpoint{0.900273in}{1.780289in}}{\pgfqpoint{0.904663in}{1.769689in}}{\pgfqpoint{0.912477in}{1.761876in}}%
\pgfpathcurveto{\pgfqpoint{0.920291in}{1.754062in}}{\pgfqpoint{0.930890in}{1.749672in}}{\pgfqpoint{0.941940in}{1.749672in}}%
\pgfpathclose%
\pgfusepath{stroke,fill}%
\end{pgfscope}%
\begin{pgfscope}%
\pgfpathrectangle{\pgfqpoint{0.564660in}{0.521603in}}{\pgfqpoint{3.720000in}{3.020000in}}%
\pgfusepath{clip}%
\pgfsetbuttcap%
\pgfsetroundjoin%
\definecolor{currentfill}{rgb}{1.000000,1.000000,1.000000}%
\pgfsetfillcolor{currentfill}%
\pgfsetlinewidth{1.003750pt}%
\definecolor{currentstroke}{rgb}{0.000000,0.000000,0.000000}%
\pgfsetstrokecolor{currentstroke}%
\pgfsetdash{}{0pt}%
\pgfpathmoveto{\pgfqpoint{1.680850in}{0.972425in}}%
\pgfpathcurveto{\pgfqpoint{1.691900in}{0.972425in}}{\pgfqpoint{1.702499in}{0.976815in}}{\pgfqpoint{1.710313in}{0.984629in}}%
\pgfpathcurveto{\pgfqpoint{1.718127in}{0.992443in}}{\pgfqpoint{1.722517in}{1.003042in}}{\pgfqpoint{1.722517in}{1.014092in}}%
\pgfpathcurveto{\pgfqpoint{1.722517in}{1.025142in}}{\pgfqpoint{1.718127in}{1.035741in}}{\pgfqpoint{1.710313in}{1.043555in}}%
\pgfpathcurveto{\pgfqpoint{1.702499in}{1.051368in}}{\pgfqpoint{1.691900in}{1.055759in}}{\pgfqpoint{1.680850in}{1.055759in}}%
\pgfpathcurveto{\pgfqpoint{1.669800in}{1.055759in}}{\pgfqpoint{1.659201in}{1.051368in}}{\pgfqpoint{1.651387in}{1.043555in}}%
\pgfpathcurveto{\pgfqpoint{1.643574in}{1.035741in}}{\pgfqpoint{1.639183in}{1.025142in}}{\pgfqpoint{1.639183in}{1.014092in}}%
\pgfpathcurveto{\pgfqpoint{1.639183in}{1.003042in}}{\pgfqpoint{1.643574in}{0.992443in}}{\pgfqpoint{1.651387in}{0.984629in}}%
\pgfpathcurveto{\pgfqpoint{1.659201in}{0.976815in}}{\pgfqpoint{1.669800in}{0.972425in}}{\pgfqpoint{1.680850in}{0.972425in}}%
\pgfpathclose%
\pgfusepath{stroke,fill}%
\end{pgfscope}%
\begin{pgfscope}%
\pgfpathrectangle{\pgfqpoint{0.564660in}{0.521603in}}{\pgfqpoint{3.720000in}{3.020000in}}%
\pgfusepath{clip}%
\pgfsetbuttcap%
\pgfsetroundjoin%
\definecolor{currentfill}{rgb}{1.000000,1.000000,1.000000}%
\pgfsetfillcolor{currentfill}%
\pgfsetlinewidth{1.003750pt}%
\definecolor{currentstroke}{rgb}{0.000000,0.000000,0.000000}%
\pgfsetstrokecolor{currentstroke}%
\pgfsetdash{}{0pt}%
\pgfpathmoveto{\pgfqpoint{0.763521in}{0.982499in}}%
\pgfpathcurveto{\pgfqpoint{0.774571in}{0.982499in}}{\pgfqpoint{0.785170in}{0.986889in}}{\pgfqpoint{0.792983in}{0.994702in}}%
\pgfpathcurveto{\pgfqpoint{0.800797in}{1.002516in}}{\pgfqpoint{0.805187in}{1.013115in}}{\pgfqpoint{0.805187in}{1.024165in}}%
\pgfpathcurveto{\pgfqpoint{0.805187in}{1.035215in}}{\pgfqpoint{0.800797in}{1.045814in}}{\pgfqpoint{0.792983in}{1.053628in}}%
\pgfpathcurveto{\pgfqpoint{0.785170in}{1.061442in}}{\pgfqpoint{0.774571in}{1.065832in}}{\pgfqpoint{0.763521in}{1.065832in}}%
\pgfpathcurveto{\pgfqpoint{0.752471in}{1.065832in}}{\pgfqpoint{0.741872in}{1.061442in}}{\pgfqpoint{0.734058in}{1.053628in}}%
\pgfpathcurveto{\pgfqpoint{0.726244in}{1.045814in}}{\pgfqpoint{0.721854in}{1.035215in}}{\pgfqpoint{0.721854in}{1.024165in}}%
\pgfpathcurveto{\pgfqpoint{0.721854in}{1.013115in}}{\pgfqpoint{0.726244in}{1.002516in}}{\pgfqpoint{0.734058in}{0.994702in}}%
\pgfpathcurveto{\pgfqpoint{0.741872in}{0.986889in}}{\pgfqpoint{0.752471in}{0.982499in}}{\pgfqpoint{0.763521in}{0.982499in}}%
\pgfpathclose%
\pgfusepath{stroke,fill}%
\end{pgfscope}%
\begin{pgfscope}%
\pgfpathrectangle{\pgfqpoint{0.564660in}{0.521603in}}{\pgfqpoint{3.720000in}{3.020000in}}%
\pgfusepath{clip}%
\pgfsetbuttcap%
\pgfsetroundjoin%
\definecolor{currentfill}{rgb}{1.000000,1.000000,1.000000}%
\pgfsetfillcolor{currentfill}%
\pgfsetlinewidth{1.003750pt}%
\definecolor{currentstroke}{rgb}{0.000000,0.000000,0.000000}%
\pgfsetstrokecolor{currentstroke}%
\pgfsetdash{}{0pt}%
\pgfpathmoveto{\pgfqpoint{1.355680in}{0.737445in}}%
\pgfpathcurveto{\pgfqpoint{1.366730in}{0.737445in}}{\pgfqpoint{1.377329in}{0.741835in}}{\pgfqpoint{1.385143in}{0.749649in}}%
\pgfpathcurveto{\pgfqpoint{1.392956in}{0.757463in}}{\pgfqpoint{1.397346in}{0.768062in}}{\pgfqpoint{1.397346in}{0.779112in}}%
\pgfpathcurveto{\pgfqpoint{1.397346in}{0.790162in}}{\pgfqpoint{1.392956in}{0.800761in}}{\pgfqpoint{1.385143in}{0.808575in}}%
\pgfpathcurveto{\pgfqpoint{1.377329in}{0.816388in}}{\pgfqpoint{1.366730in}{0.820779in}}{\pgfqpoint{1.355680in}{0.820779in}}%
\pgfpathcurveto{\pgfqpoint{1.344630in}{0.820779in}}{\pgfqpoint{1.334031in}{0.816388in}}{\pgfqpoint{1.326217in}{0.808575in}}%
\pgfpathcurveto{\pgfqpoint{1.318403in}{0.800761in}}{\pgfqpoint{1.314013in}{0.790162in}}{\pgfqpoint{1.314013in}{0.779112in}}%
\pgfpathcurveto{\pgfqpoint{1.314013in}{0.768062in}}{\pgfqpoint{1.318403in}{0.757463in}}{\pgfqpoint{1.326217in}{0.749649in}}%
\pgfpathcurveto{\pgfqpoint{1.334031in}{0.741835in}}{\pgfqpoint{1.344630in}{0.737445in}}{\pgfqpoint{1.355680in}{0.737445in}}%
\pgfpathclose%
\pgfusepath{stroke,fill}%
\end{pgfscope}%
\begin{pgfscope}%
\pgfpathrectangle{\pgfqpoint{0.564660in}{0.521603in}}{\pgfqpoint{3.720000in}{3.020000in}}%
\pgfusepath{clip}%
\pgfsetbuttcap%
\pgfsetroundjoin%
\definecolor{currentfill}{rgb}{1.000000,1.000000,1.000000}%
\pgfsetfillcolor{currentfill}%
\pgfsetlinewidth{1.003750pt}%
\definecolor{currentstroke}{rgb}{0.000000,0.000000,0.000000}%
\pgfsetstrokecolor{currentstroke}%
\pgfsetdash{}{0pt}%
\pgfpathmoveto{\pgfqpoint{1.307448in}{0.654071in}}%
\pgfpathcurveto{\pgfqpoint{1.318498in}{0.654071in}}{\pgfqpoint{1.329097in}{0.658461in}}{\pgfqpoint{1.336910in}{0.666275in}}%
\pgfpathcurveto{\pgfqpoint{1.344724in}{0.674089in}}{\pgfqpoint{1.349114in}{0.684688in}}{\pgfqpoint{1.349114in}{0.695738in}}%
\pgfpathcurveto{\pgfqpoint{1.349114in}{0.706788in}}{\pgfqpoint{1.344724in}{0.717387in}}{\pgfqpoint{1.336910in}{0.725200in}}%
\pgfpathcurveto{\pgfqpoint{1.329097in}{0.733014in}}{\pgfqpoint{1.318498in}{0.737404in}}{\pgfqpoint{1.307448in}{0.737404in}}%
\pgfpathcurveto{\pgfqpoint{1.296398in}{0.737404in}}{\pgfqpoint{1.285799in}{0.733014in}}{\pgfqpoint{1.277985in}{0.725200in}}%
\pgfpathcurveto{\pgfqpoint{1.270171in}{0.717387in}}{\pgfqpoint{1.265781in}{0.706788in}}{\pgfqpoint{1.265781in}{0.695738in}}%
\pgfpathcurveto{\pgfqpoint{1.265781in}{0.684688in}}{\pgfqpoint{1.270171in}{0.674089in}}{\pgfqpoint{1.277985in}{0.666275in}}%
\pgfpathcurveto{\pgfqpoint{1.285799in}{0.658461in}}{\pgfqpoint{1.296398in}{0.654071in}}{\pgfqpoint{1.307448in}{0.654071in}}%
\pgfpathclose%
\pgfusepath{stroke,fill}%
\end{pgfscope}%
\begin{pgfscope}%
\pgfpathrectangle{\pgfqpoint{0.564660in}{0.521603in}}{\pgfqpoint{3.720000in}{3.020000in}}%
\pgfusepath{clip}%
\pgfsetbuttcap%
\pgfsetroundjoin%
\definecolor{currentfill}{rgb}{1.000000,1.000000,1.000000}%
\pgfsetfillcolor{currentfill}%
\pgfsetlinewidth{1.003750pt}%
\definecolor{currentstroke}{rgb}{0.000000,0.000000,0.000000}%
\pgfsetstrokecolor{currentstroke}%
\pgfsetdash{}{0pt}%
\pgfpathmoveto{\pgfqpoint{4.085800in}{2.313769in}}%
\pgfpathcurveto{\pgfqpoint{4.096850in}{2.313769in}}{\pgfqpoint{4.107449in}{2.318159in}}{\pgfqpoint{4.115263in}{2.325973in}}%
\pgfpathcurveto{\pgfqpoint{4.123076in}{2.333786in}}{\pgfqpoint{4.127467in}{2.344385in}}{\pgfqpoint{4.127467in}{2.355435in}}%
\pgfpathcurveto{\pgfqpoint{4.127467in}{2.366486in}}{\pgfqpoint{4.123076in}{2.377085in}}{\pgfqpoint{4.115263in}{2.384898in}}%
\pgfpathcurveto{\pgfqpoint{4.107449in}{2.392712in}}{\pgfqpoint{4.096850in}{2.397102in}}{\pgfqpoint{4.085800in}{2.397102in}}%
\pgfpathcurveto{\pgfqpoint{4.074750in}{2.397102in}}{\pgfqpoint{4.064151in}{2.392712in}}{\pgfqpoint{4.056337in}{2.384898in}}%
\pgfpathcurveto{\pgfqpoint{4.048523in}{2.377085in}}{\pgfqpoint{4.044133in}{2.366486in}}{\pgfqpoint{4.044133in}{2.355435in}}%
\pgfpathcurveto{\pgfqpoint{4.044133in}{2.344385in}}{\pgfqpoint{4.048523in}{2.333786in}}{\pgfqpoint{4.056337in}{2.325973in}}%
\pgfpathcurveto{\pgfqpoint{4.064151in}{2.318159in}}{\pgfqpoint{4.074750in}{2.313769in}}{\pgfqpoint{4.085800in}{2.313769in}}%
\pgfpathclose%
\pgfusepath{stroke,fill}%
\end{pgfscope}%
\begin{pgfscope}%
\pgfpathrectangle{\pgfqpoint{0.564660in}{0.521603in}}{\pgfqpoint{3.720000in}{3.020000in}}%
\pgfusepath{clip}%
\pgfsetbuttcap%
\pgfsetroundjoin%
\definecolor{currentfill}{rgb}{1.000000,1.000000,1.000000}%
\pgfsetfillcolor{currentfill}%
\pgfsetlinewidth{1.003750pt}%
\definecolor{currentstroke}{rgb}{0.000000,0.000000,0.000000}%
\pgfsetstrokecolor{currentstroke}%
\pgfsetdash{}{0pt}%
\pgfpathmoveto{\pgfqpoint{3.510562in}{1.703866in}}%
\pgfpathcurveto{\pgfqpoint{3.521612in}{1.703866in}}{\pgfqpoint{3.532211in}{1.708257in}}{\pgfqpoint{3.540025in}{1.716070in}}%
\pgfpathcurveto{\pgfqpoint{3.547839in}{1.723884in}}{\pgfqpoint{3.552229in}{1.734483in}}{\pgfqpoint{3.552229in}{1.745533in}}%
\pgfpathcurveto{\pgfqpoint{3.552229in}{1.756583in}}{\pgfqpoint{3.547839in}{1.767182in}}{\pgfqpoint{3.540025in}{1.774996in}}%
\pgfpathcurveto{\pgfqpoint{3.532211in}{1.782810in}}{\pgfqpoint{3.521612in}{1.787200in}}{\pgfqpoint{3.510562in}{1.787200in}}%
\pgfpathcurveto{\pgfqpoint{3.499512in}{1.787200in}}{\pgfqpoint{3.488913in}{1.782810in}}{\pgfqpoint{3.481099in}{1.774996in}}%
\pgfpathcurveto{\pgfqpoint{3.473286in}{1.767182in}}{\pgfqpoint{3.468895in}{1.756583in}}{\pgfqpoint{3.468895in}{1.745533in}}%
\pgfpathcurveto{\pgfqpoint{3.468895in}{1.734483in}}{\pgfqpoint{3.473286in}{1.723884in}}{\pgfqpoint{3.481099in}{1.716070in}}%
\pgfpathcurveto{\pgfqpoint{3.488913in}{1.708257in}}{\pgfqpoint{3.499512in}{1.703866in}}{\pgfqpoint{3.510562in}{1.703866in}}%
\pgfpathclose%
\pgfusepath{stroke,fill}%
\end{pgfscope}%
\begin{pgfscope}%
\pgfpathrectangle{\pgfqpoint{0.564660in}{0.521603in}}{\pgfqpoint{3.720000in}{3.020000in}}%
\pgfusepath{clip}%
\pgfsetbuttcap%
\pgfsetroundjoin%
\definecolor{currentfill}{rgb}{1.000000,1.000000,1.000000}%
\pgfsetfillcolor{currentfill}%
\pgfsetlinewidth{1.003750pt}%
\definecolor{currentstroke}{rgb}{0.000000,0.000000,0.000000}%
\pgfsetstrokecolor{currentstroke}%
\pgfsetdash{}{0pt}%
\pgfpathmoveto{\pgfqpoint{2.876111in}{1.338281in}}%
\pgfpathcurveto{\pgfqpoint{2.887161in}{1.338281in}}{\pgfqpoint{2.897760in}{1.342671in}}{\pgfqpoint{2.905574in}{1.350485in}}%
\pgfpathcurveto{\pgfqpoint{2.913387in}{1.358299in}}{\pgfqpoint{2.917777in}{1.368898in}}{\pgfqpoint{2.917777in}{1.379948in}}%
\pgfpathcurveto{\pgfqpoint{2.917777in}{1.390998in}}{\pgfqpoint{2.913387in}{1.401597in}}{\pgfqpoint{2.905574in}{1.409410in}}%
\pgfpathcurveto{\pgfqpoint{2.897760in}{1.417224in}}{\pgfqpoint{2.887161in}{1.421614in}}{\pgfqpoint{2.876111in}{1.421614in}}%
\pgfpathcurveto{\pgfqpoint{2.865061in}{1.421614in}}{\pgfqpoint{2.854462in}{1.417224in}}{\pgfqpoint{2.846648in}{1.409410in}}%
\pgfpathcurveto{\pgfqpoint{2.838834in}{1.401597in}}{\pgfqpoint{2.834444in}{1.390998in}}{\pgfqpoint{2.834444in}{1.379948in}}%
\pgfpathcurveto{\pgfqpoint{2.834444in}{1.368898in}}{\pgfqpoint{2.838834in}{1.358299in}}{\pgfqpoint{2.846648in}{1.350485in}}%
\pgfpathcurveto{\pgfqpoint{2.854462in}{1.342671in}}{\pgfqpoint{2.865061in}{1.338281in}}{\pgfqpoint{2.876111in}{1.338281in}}%
\pgfpathclose%
\pgfusepath{stroke,fill}%
\end{pgfscope}%
\begin{pgfscope}%
\pgfpathrectangle{\pgfqpoint{0.564660in}{0.521603in}}{\pgfqpoint{3.720000in}{3.020000in}}%
\pgfusepath{clip}%
\pgfsetbuttcap%
\pgfsetroundjoin%
\definecolor{currentfill}{rgb}{1.000000,1.000000,1.000000}%
\pgfsetfillcolor{currentfill}%
\pgfsetlinewidth{1.003750pt}%
\definecolor{currentstroke}{rgb}{0.000000,0.000000,0.000000}%
\pgfsetstrokecolor{currentstroke}%
\pgfsetdash{}{0pt}%
\pgfpathmoveto{\pgfqpoint{1.255941in}{2.308002in}}%
\pgfpathcurveto{\pgfqpoint{1.266991in}{2.308002in}}{\pgfqpoint{1.277590in}{2.312392in}}{\pgfqpoint{1.285404in}{2.320206in}}%
\pgfpathcurveto{\pgfqpoint{1.293218in}{2.328020in}}{\pgfqpoint{1.297608in}{2.338619in}}{\pgfqpoint{1.297608in}{2.349669in}}%
\pgfpathcurveto{\pgfqpoint{1.297608in}{2.360719in}}{\pgfqpoint{1.293218in}{2.371318in}}{\pgfqpoint{1.285404in}{2.379132in}}%
\pgfpathcurveto{\pgfqpoint{1.277590in}{2.386945in}}{\pgfqpoint{1.266991in}{2.391336in}}{\pgfqpoint{1.255941in}{2.391336in}}%
\pgfpathcurveto{\pgfqpoint{1.244891in}{2.391336in}}{\pgfqpoint{1.234292in}{2.386945in}}{\pgfqpoint{1.226478in}{2.379132in}}%
\pgfpathcurveto{\pgfqpoint{1.218665in}{2.371318in}}{\pgfqpoint{1.214275in}{2.360719in}}{\pgfqpoint{1.214275in}{2.349669in}}%
\pgfpathcurveto{\pgfqpoint{1.214275in}{2.338619in}}{\pgfqpoint{1.218665in}{2.328020in}}{\pgfqpoint{1.226478in}{2.320206in}}%
\pgfpathcurveto{\pgfqpoint{1.234292in}{2.312392in}}{\pgfqpoint{1.244891in}{2.308002in}}{\pgfqpoint{1.255941in}{2.308002in}}%
\pgfpathclose%
\pgfusepath{stroke,fill}%
\end{pgfscope}%
\begin{pgfscope}%
\pgfpathrectangle{\pgfqpoint{0.564660in}{0.521603in}}{\pgfqpoint{3.720000in}{3.020000in}}%
\pgfusepath{clip}%
\pgfsetbuttcap%
\pgfsetroundjoin%
\definecolor{currentfill}{rgb}{1.000000,1.000000,1.000000}%
\pgfsetfillcolor{currentfill}%
\pgfsetlinewidth{1.003750pt}%
\definecolor{currentstroke}{rgb}{0.000000,0.000000,0.000000}%
\pgfsetstrokecolor{currentstroke}%
\pgfsetdash{}{0pt}%
\pgfpathmoveto{\pgfqpoint{2.259339in}{0.939575in}}%
\pgfpathcurveto{\pgfqpoint{2.270389in}{0.939575in}}{\pgfqpoint{2.280988in}{0.943966in}}{\pgfqpoint{2.288801in}{0.951779in}}%
\pgfpathcurveto{\pgfqpoint{2.296615in}{0.959593in}}{\pgfqpoint{2.301005in}{0.970192in}}{\pgfqpoint{2.301005in}{0.981242in}}%
\pgfpathcurveto{\pgfqpoint{2.301005in}{0.992292in}}{\pgfqpoint{2.296615in}{1.002891in}}{\pgfqpoint{2.288801in}{1.010705in}}%
\pgfpathcurveto{\pgfqpoint{2.280988in}{1.018518in}}{\pgfqpoint{2.270389in}{1.022909in}}{\pgfqpoint{2.259339in}{1.022909in}}%
\pgfpathcurveto{\pgfqpoint{2.248289in}{1.022909in}}{\pgfqpoint{2.237690in}{1.018518in}}{\pgfqpoint{2.229876in}{1.010705in}}%
\pgfpathcurveto{\pgfqpoint{2.222062in}{1.002891in}}{\pgfqpoint{2.217672in}{0.992292in}}{\pgfqpoint{2.217672in}{0.981242in}}%
\pgfpathcurveto{\pgfqpoint{2.217672in}{0.970192in}}{\pgfqpoint{2.222062in}{0.959593in}}{\pgfqpoint{2.229876in}{0.951779in}}%
\pgfpathcurveto{\pgfqpoint{2.237690in}{0.943966in}}{\pgfqpoint{2.248289in}{0.939575in}}{\pgfqpoint{2.259339in}{0.939575in}}%
\pgfpathclose%
\pgfusepath{stroke,fill}%
\end{pgfscope}%
\begin{pgfscope}%
\pgfpathrectangle{\pgfqpoint{0.564660in}{0.521603in}}{\pgfqpoint{3.720000in}{3.020000in}}%
\pgfusepath{clip}%
\pgfsetbuttcap%
\pgfsetroundjoin%
\definecolor{currentfill}{rgb}{1.000000,1.000000,1.000000}%
\pgfsetfillcolor{currentfill}%
\pgfsetlinewidth{1.003750pt}%
\definecolor{currentstroke}{rgb}{0.000000,0.000000,0.000000}%
\pgfsetstrokecolor{currentstroke}%
\pgfsetdash{}{0pt}%
\pgfpathmoveto{\pgfqpoint{1.940911in}{1.013750in}}%
\pgfpathcurveto{\pgfqpoint{1.951961in}{1.013750in}}{\pgfqpoint{1.962560in}{1.018140in}}{\pgfqpoint{1.970374in}{1.025954in}}%
\pgfpathcurveto{\pgfqpoint{1.978187in}{1.033768in}}{\pgfqpoint{1.982578in}{1.044367in}}{\pgfqpoint{1.982578in}{1.055417in}}%
\pgfpathcurveto{\pgfqpoint{1.982578in}{1.066467in}}{\pgfqpoint{1.978187in}{1.077066in}}{\pgfqpoint{1.970374in}{1.084880in}}%
\pgfpathcurveto{\pgfqpoint{1.962560in}{1.092693in}}{\pgfqpoint{1.951961in}{1.097083in}}{\pgfqpoint{1.940911in}{1.097083in}}%
\pgfpathcurveto{\pgfqpoint{1.929861in}{1.097083in}}{\pgfqpoint{1.919262in}{1.092693in}}{\pgfqpoint{1.911448in}{1.084880in}}%
\pgfpathcurveto{\pgfqpoint{1.903635in}{1.077066in}}{\pgfqpoint{1.899244in}{1.066467in}}{\pgfqpoint{1.899244in}{1.055417in}}%
\pgfpathcurveto{\pgfqpoint{1.899244in}{1.044367in}}{\pgfqpoint{1.903635in}{1.033768in}}{\pgfqpoint{1.911448in}{1.025954in}}%
\pgfpathcurveto{\pgfqpoint{1.919262in}{1.018140in}}{\pgfqpoint{1.929861in}{1.013750in}}{\pgfqpoint{1.940911in}{1.013750in}}%
\pgfpathclose%
\pgfusepath{stroke,fill}%
\end{pgfscope}%
\begin{pgfscope}%
\pgfpathrectangle{\pgfqpoint{0.564660in}{0.521603in}}{\pgfqpoint{3.720000in}{3.020000in}}%
\pgfusepath{clip}%
\pgfsetbuttcap%
\pgfsetroundjoin%
\definecolor{currentfill}{rgb}{1.000000,1.000000,1.000000}%
\pgfsetfillcolor{currentfill}%
\pgfsetlinewidth{1.003750pt}%
\definecolor{currentstroke}{rgb}{0.000000,0.000000,0.000000}%
\pgfsetstrokecolor{currentstroke}%
\pgfsetdash{}{0pt}%
\pgfpathmoveto{\pgfqpoint{1.712153in}{3.325802in}}%
\pgfpathcurveto{\pgfqpoint{1.723203in}{3.325802in}}{\pgfqpoint{1.733802in}{3.330193in}}{\pgfqpoint{1.741616in}{3.338006in}}%
\pgfpathcurveto{\pgfqpoint{1.749429in}{3.345820in}}{\pgfqpoint{1.753820in}{3.356419in}}{\pgfqpoint{1.753820in}{3.367469in}}%
\pgfpathcurveto{\pgfqpoint{1.753820in}{3.378519in}}{\pgfqpoint{1.749429in}{3.389118in}}{\pgfqpoint{1.741616in}{3.396932in}}%
\pgfpathcurveto{\pgfqpoint{1.733802in}{3.404745in}}{\pgfqpoint{1.723203in}{3.409136in}}{\pgfqpoint{1.712153in}{3.409136in}}%
\pgfpathcurveto{\pgfqpoint{1.701103in}{3.409136in}}{\pgfqpoint{1.690504in}{3.404745in}}{\pgfqpoint{1.682690in}{3.396932in}}%
\pgfpathcurveto{\pgfqpoint{1.674876in}{3.389118in}}{\pgfqpoint{1.670486in}{3.378519in}}{\pgfqpoint{1.670486in}{3.367469in}}%
\pgfpathcurveto{\pgfqpoint{1.670486in}{3.356419in}}{\pgfqpoint{1.674876in}{3.345820in}}{\pgfqpoint{1.682690in}{3.338006in}}%
\pgfpathcurveto{\pgfqpoint{1.690504in}{3.330193in}}{\pgfqpoint{1.701103in}{3.325802in}}{\pgfqpoint{1.712153in}{3.325802in}}%
\pgfpathclose%
\pgfusepath{stroke,fill}%
\end{pgfscope}%
\begin{pgfscope}%
\pgfpathrectangle{\pgfqpoint{0.564660in}{0.521603in}}{\pgfqpoint{3.720000in}{3.020000in}}%
\pgfusepath{clip}%
\pgfsetbuttcap%
\pgfsetroundjoin%
\definecolor{currentfill}{rgb}{1.000000,1.000000,1.000000}%
\pgfsetfillcolor{currentfill}%
\pgfsetlinewidth{1.003750pt}%
\definecolor{currentstroke}{rgb}{0.000000,0.000000,0.000000}%
\pgfsetstrokecolor{currentstroke}%
\pgfsetdash{}{0pt}%
\pgfpathmoveto{\pgfqpoint{1.087553in}{1.563620in}}%
\pgfpathcurveto{\pgfqpoint{1.098603in}{1.563620in}}{\pgfqpoint{1.109202in}{1.568010in}}{\pgfqpoint{1.117016in}{1.575823in}}%
\pgfpathcurveto{\pgfqpoint{1.124830in}{1.583637in}}{\pgfqpoint{1.129220in}{1.594236in}}{\pgfqpoint{1.129220in}{1.605286in}}%
\pgfpathcurveto{\pgfqpoint{1.129220in}{1.616336in}}{\pgfqpoint{1.124830in}{1.626935in}}{\pgfqpoint{1.117016in}{1.634749in}}%
\pgfpathcurveto{\pgfqpoint{1.109202in}{1.642563in}}{\pgfqpoint{1.098603in}{1.646953in}}{\pgfqpoint{1.087553in}{1.646953in}}%
\pgfpathcurveto{\pgfqpoint{1.076503in}{1.646953in}}{\pgfqpoint{1.065904in}{1.642563in}}{\pgfqpoint{1.058090in}{1.634749in}}%
\pgfpathcurveto{\pgfqpoint{1.050277in}{1.626935in}}{\pgfqpoint{1.045886in}{1.616336in}}{\pgfqpoint{1.045886in}{1.605286in}}%
\pgfpathcurveto{\pgfqpoint{1.045886in}{1.594236in}}{\pgfqpoint{1.050277in}{1.583637in}}{\pgfqpoint{1.058090in}{1.575823in}}%
\pgfpathcurveto{\pgfqpoint{1.065904in}{1.568010in}}{\pgfqpoint{1.076503in}{1.563620in}}{\pgfqpoint{1.087553in}{1.563620in}}%
\pgfpathclose%
\pgfusepath{stroke,fill}%
\end{pgfscope}%
\begin{pgfscope}%
\pgfpathrectangle{\pgfqpoint{0.564660in}{0.521603in}}{\pgfqpoint{3.720000in}{3.020000in}}%
\pgfusepath{clip}%
\pgfsetbuttcap%
\pgfsetroundjoin%
\definecolor{currentfill}{rgb}{1.000000,1.000000,1.000000}%
\pgfsetfillcolor{currentfill}%
\pgfsetlinewidth{1.003750pt}%
\definecolor{currentstroke}{rgb}{0.000000,0.000000,0.000000}%
\pgfsetstrokecolor{currentstroke}%
\pgfsetdash{}{0pt}%
\pgfpathmoveto{\pgfqpoint{1.626658in}{0.731310in}}%
\pgfpathcurveto{\pgfqpoint{1.637708in}{0.731310in}}{\pgfqpoint{1.648307in}{0.735701in}}{\pgfqpoint{1.656121in}{0.743514in}}%
\pgfpathcurveto{\pgfqpoint{1.663934in}{0.751328in}}{\pgfqpoint{1.668325in}{0.761927in}}{\pgfqpoint{1.668325in}{0.772977in}}%
\pgfpathcurveto{\pgfqpoint{1.668325in}{0.784027in}}{\pgfqpoint{1.663934in}{0.794626in}}{\pgfqpoint{1.656121in}{0.802440in}}%
\pgfpathcurveto{\pgfqpoint{1.648307in}{0.810253in}}{\pgfqpoint{1.637708in}{0.814644in}}{\pgfqpoint{1.626658in}{0.814644in}}%
\pgfpathcurveto{\pgfqpoint{1.615608in}{0.814644in}}{\pgfqpoint{1.605009in}{0.810253in}}{\pgfqpoint{1.597195in}{0.802440in}}%
\pgfpathcurveto{\pgfqpoint{1.589382in}{0.794626in}}{\pgfqpoint{1.584991in}{0.784027in}}{\pgfqpoint{1.584991in}{0.772977in}}%
\pgfpathcurveto{\pgfqpoint{1.584991in}{0.761927in}}{\pgfqpoint{1.589382in}{0.751328in}}{\pgfqpoint{1.597195in}{0.743514in}}%
\pgfpathcurveto{\pgfqpoint{1.605009in}{0.735701in}}{\pgfqpoint{1.615608in}{0.731310in}}{\pgfqpoint{1.626658in}{0.731310in}}%
\pgfpathclose%
\pgfusepath{stroke,fill}%
\end{pgfscope}%
\begin{pgfscope}%
\pgfsetbuttcap%
\pgfsetroundjoin%
\definecolor{currentfill}{rgb}{0.000000,0.000000,0.000000}%
\pgfsetfillcolor{currentfill}%
\pgfsetlinewidth{0.803000pt}%
\definecolor{currentstroke}{rgb}{0.000000,0.000000,0.000000}%
\pgfsetstrokecolor{currentstroke}%
\pgfsetdash{}{0pt}%
\pgfsys@defobject{currentmarker}{\pgfqpoint{0.000000in}{-0.048611in}}{\pgfqpoint{0.000000in}{0.000000in}}{%
\pgfpathmoveto{\pgfqpoint{0.000000in}{0.000000in}}%
\pgfpathlineto{\pgfqpoint{0.000000in}{-0.048611in}}%
\pgfusepath{stroke,fill}%
}%
\begin{pgfscope}%
\pgfsys@transformshift{1.061822in}{0.521603in}%
\pgfsys@useobject{currentmarker}{}%
\end{pgfscope}%
\end{pgfscope}%
\begin{pgfscope}%
\definecolor{textcolor}{rgb}{0.000000,0.000000,0.000000}%
\pgfsetstrokecolor{textcolor}%
\pgfsetfillcolor{textcolor}%
\pgftext[x=1.061822in,y=0.424381in,,top]{\color{textcolor}\rmfamily\fontsize{10.000000}{12.000000}\selectfont \(\displaystyle 0.6\)}%
\end{pgfscope}%
\begin{pgfscope}%
\pgfsetbuttcap%
\pgfsetroundjoin%
\definecolor{currentfill}{rgb}{0.000000,0.000000,0.000000}%
\pgfsetfillcolor{currentfill}%
\pgfsetlinewidth{0.803000pt}%
\definecolor{currentstroke}{rgb}{0.000000,0.000000,0.000000}%
\pgfsetstrokecolor{currentstroke}%
\pgfsetdash{}{0pt}%
\pgfsys@defobject{currentmarker}{\pgfqpoint{0.000000in}{-0.048611in}}{\pgfqpoint{0.000000in}{0.000000in}}{%
\pgfpathmoveto{\pgfqpoint{0.000000in}{0.000000in}}%
\pgfpathlineto{\pgfqpoint{0.000000in}{-0.048611in}}%
\pgfusepath{stroke,fill}%
}%
\begin{pgfscope}%
\pgfsys@transformshift{1.572925in}{0.521603in}%
\pgfsys@useobject{currentmarker}{}%
\end{pgfscope}%
\end{pgfscope}%
\begin{pgfscope}%
\definecolor{textcolor}{rgb}{0.000000,0.000000,0.000000}%
\pgfsetstrokecolor{textcolor}%
\pgfsetfillcolor{textcolor}%
\pgftext[x=1.572925in,y=0.424381in,,top]{\color{textcolor}\rmfamily\fontsize{10.000000}{12.000000}\selectfont \(\displaystyle 0.8\)}%
\end{pgfscope}%
\begin{pgfscope}%
\pgfsetbuttcap%
\pgfsetroundjoin%
\definecolor{currentfill}{rgb}{0.000000,0.000000,0.000000}%
\pgfsetfillcolor{currentfill}%
\pgfsetlinewidth{0.803000pt}%
\definecolor{currentstroke}{rgb}{0.000000,0.000000,0.000000}%
\pgfsetstrokecolor{currentstroke}%
\pgfsetdash{}{0pt}%
\pgfsys@defobject{currentmarker}{\pgfqpoint{0.000000in}{-0.048611in}}{\pgfqpoint{0.000000in}{0.000000in}}{%
\pgfpathmoveto{\pgfqpoint{0.000000in}{0.000000in}}%
\pgfpathlineto{\pgfqpoint{0.000000in}{-0.048611in}}%
\pgfusepath{stroke,fill}%
}%
\begin{pgfscope}%
\pgfsys@transformshift{2.084027in}{0.521603in}%
\pgfsys@useobject{currentmarker}{}%
\end{pgfscope}%
\end{pgfscope}%
\begin{pgfscope}%
\definecolor{textcolor}{rgb}{0.000000,0.000000,0.000000}%
\pgfsetstrokecolor{textcolor}%
\pgfsetfillcolor{textcolor}%
\pgftext[x=2.084027in,y=0.424381in,,top]{\color{textcolor}\rmfamily\fontsize{10.000000}{12.000000}\selectfont \(\displaystyle 1.0\)}%
\end{pgfscope}%
\begin{pgfscope}%
\pgfsetbuttcap%
\pgfsetroundjoin%
\definecolor{currentfill}{rgb}{0.000000,0.000000,0.000000}%
\pgfsetfillcolor{currentfill}%
\pgfsetlinewidth{0.803000pt}%
\definecolor{currentstroke}{rgb}{0.000000,0.000000,0.000000}%
\pgfsetstrokecolor{currentstroke}%
\pgfsetdash{}{0pt}%
\pgfsys@defobject{currentmarker}{\pgfqpoint{0.000000in}{-0.048611in}}{\pgfqpoint{0.000000in}{0.000000in}}{%
\pgfpathmoveto{\pgfqpoint{0.000000in}{0.000000in}}%
\pgfpathlineto{\pgfqpoint{0.000000in}{-0.048611in}}%
\pgfusepath{stroke,fill}%
}%
\begin{pgfscope}%
\pgfsys@transformshift{2.595130in}{0.521603in}%
\pgfsys@useobject{currentmarker}{}%
\end{pgfscope}%
\end{pgfscope}%
\begin{pgfscope}%
\definecolor{textcolor}{rgb}{0.000000,0.000000,0.000000}%
\pgfsetstrokecolor{textcolor}%
\pgfsetfillcolor{textcolor}%
\pgftext[x=2.595130in,y=0.424381in,,top]{\color{textcolor}\rmfamily\fontsize{10.000000}{12.000000}\selectfont \(\displaystyle 1.2\)}%
\end{pgfscope}%
\begin{pgfscope}%
\pgfsetbuttcap%
\pgfsetroundjoin%
\definecolor{currentfill}{rgb}{0.000000,0.000000,0.000000}%
\pgfsetfillcolor{currentfill}%
\pgfsetlinewidth{0.803000pt}%
\definecolor{currentstroke}{rgb}{0.000000,0.000000,0.000000}%
\pgfsetstrokecolor{currentstroke}%
\pgfsetdash{}{0pt}%
\pgfsys@defobject{currentmarker}{\pgfqpoint{0.000000in}{-0.048611in}}{\pgfqpoint{0.000000in}{0.000000in}}{%
\pgfpathmoveto{\pgfqpoint{0.000000in}{0.000000in}}%
\pgfpathlineto{\pgfqpoint{0.000000in}{-0.048611in}}%
\pgfusepath{stroke,fill}%
}%
\begin{pgfscope}%
\pgfsys@transformshift{3.106233in}{0.521603in}%
\pgfsys@useobject{currentmarker}{}%
\end{pgfscope}%
\end{pgfscope}%
\begin{pgfscope}%
\definecolor{textcolor}{rgb}{0.000000,0.000000,0.000000}%
\pgfsetstrokecolor{textcolor}%
\pgfsetfillcolor{textcolor}%
\pgftext[x=3.106233in,y=0.424381in,,top]{\color{textcolor}\rmfamily\fontsize{10.000000}{12.000000}\selectfont \(\displaystyle 1.4\)}%
\end{pgfscope}%
\begin{pgfscope}%
\pgfsetbuttcap%
\pgfsetroundjoin%
\definecolor{currentfill}{rgb}{0.000000,0.000000,0.000000}%
\pgfsetfillcolor{currentfill}%
\pgfsetlinewidth{0.803000pt}%
\definecolor{currentstroke}{rgb}{0.000000,0.000000,0.000000}%
\pgfsetstrokecolor{currentstroke}%
\pgfsetdash{}{0pt}%
\pgfsys@defobject{currentmarker}{\pgfqpoint{0.000000in}{-0.048611in}}{\pgfqpoint{0.000000in}{0.000000in}}{%
\pgfpathmoveto{\pgfqpoint{0.000000in}{0.000000in}}%
\pgfpathlineto{\pgfqpoint{0.000000in}{-0.048611in}}%
\pgfusepath{stroke,fill}%
}%
\begin{pgfscope}%
\pgfsys@transformshift{3.617336in}{0.521603in}%
\pgfsys@useobject{currentmarker}{}%
\end{pgfscope}%
\end{pgfscope}%
\begin{pgfscope}%
\definecolor{textcolor}{rgb}{0.000000,0.000000,0.000000}%
\pgfsetstrokecolor{textcolor}%
\pgfsetfillcolor{textcolor}%
\pgftext[x=3.617336in,y=0.424381in,,top]{\color{textcolor}\rmfamily\fontsize{10.000000}{12.000000}\selectfont \(\displaystyle 1.6\)}%
\end{pgfscope}%
\begin{pgfscope}%
\pgfsetbuttcap%
\pgfsetroundjoin%
\definecolor{currentfill}{rgb}{0.000000,0.000000,0.000000}%
\pgfsetfillcolor{currentfill}%
\pgfsetlinewidth{0.803000pt}%
\definecolor{currentstroke}{rgb}{0.000000,0.000000,0.000000}%
\pgfsetstrokecolor{currentstroke}%
\pgfsetdash{}{0pt}%
\pgfsys@defobject{currentmarker}{\pgfqpoint{0.000000in}{-0.048611in}}{\pgfqpoint{0.000000in}{0.000000in}}{%
\pgfpathmoveto{\pgfqpoint{0.000000in}{0.000000in}}%
\pgfpathlineto{\pgfqpoint{0.000000in}{-0.048611in}}%
\pgfusepath{stroke,fill}%
}%
\begin{pgfscope}%
\pgfsys@transformshift{4.128439in}{0.521603in}%
\pgfsys@useobject{currentmarker}{}%
\end{pgfscope}%
\end{pgfscope}%
\begin{pgfscope}%
\definecolor{textcolor}{rgb}{0.000000,0.000000,0.000000}%
\pgfsetstrokecolor{textcolor}%
\pgfsetfillcolor{textcolor}%
\pgftext[x=4.128439in,y=0.424381in,,top]{\color{textcolor}\rmfamily\fontsize{10.000000}{12.000000}\selectfont \(\displaystyle 1.8\)}%
\end{pgfscope}%
\begin{pgfscope}%
\definecolor{textcolor}{rgb}{0.000000,0.000000,0.000000}%
\pgfsetstrokecolor{textcolor}%
\pgfsetfillcolor{textcolor}%
\pgftext[x=2.424660in,y=0.234413in,,top]{\color{textcolor}\rmfamily\fontsize{10.000000}{12.000000}\selectfont \(\displaystyle \varphi_s\) (kV)}%
\end{pgfscope}%
\begin{pgfscope}%
\pgfsetbuttcap%
\pgfsetroundjoin%
\definecolor{currentfill}{rgb}{0.000000,0.000000,0.000000}%
\pgfsetfillcolor{currentfill}%
\pgfsetlinewidth{0.803000pt}%
\definecolor{currentstroke}{rgb}{0.000000,0.000000,0.000000}%
\pgfsetstrokecolor{currentstroke}%
\pgfsetdash{}{0pt}%
\pgfsys@defobject{currentmarker}{\pgfqpoint{-0.048611in}{0.000000in}}{\pgfqpoint{0.000000in}{0.000000in}}{%
\pgfpathmoveto{\pgfqpoint{0.000000in}{0.000000in}}%
\pgfpathlineto{\pgfqpoint{-0.048611in}{0.000000in}}%
\pgfusepath{stroke,fill}%
}%
\begin{pgfscope}%
\pgfsys@transformshift{0.564660in}{0.523802in}%
\pgfsys@useobject{currentmarker}{}%
\end{pgfscope}%
\end{pgfscope}%
\begin{pgfscope}%
\definecolor{textcolor}{rgb}{0.000000,0.000000,0.000000}%
\pgfsetstrokecolor{textcolor}%
\pgfsetfillcolor{textcolor}%
\pgftext[x=0.289968in,y=0.471040in,left,base]{\color{textcolor}\rmfamily\fontsize{10.000000}{12.000000}\selectfont \(\displaystyle 0.0\)}%
\end{pgfscope}%
\begin{pgfscope}%
\pgfsetbuttcap%
\pgfsetroundjoin%
\definecolor{currentfill}{rgb}{0.000000,0.000000,0.000000}%
\pgfsetfillcolor{currentfill}%
\pgfsetlinewidth{0.803000pt}%
\definecolor{currentstroke}{rgb}{0.000000,0.000000,0.000000}%
\pgfsetstrokecolor{currentstroke}%
\pgfsetdash{}{0pt}%
\pgfsys@defobject{currentmarker}{\pgfqpoint{-0.048611in}{0.000000in}}{\pgfqpoint{0.000000in}{0.000000in}}{%
\pgfpathmoveto{\pgfqpoint{0.000000in}{0.000000in}}%
\pgfpathlineto{\pgfqpoint{-0.048611in}{0.000000in}}%
\pgfusepath{stroke,fill}%
}%
\begin{pgfscope}%
\pgfsys@transformshift{0.564660in}{1.096070in}%
\pgfsys@useobject{currentmarker}{}%
\end{pgfscope}%
\end{pgfscope}%
\begin{pgfscope}%
\definecolor{textcolor}{rgb}{0.000000,0.000000,0.000000}%
\pgfsetstrokecolor{textcolor}%
\pgfsetfillcolor{textcolor}%
\pgftext[x=0.289968in,y=1.043308in,left,base]{\color{textcolor}\rmfamily\fontsize{10.000000}{12.000000}\selectfont \(\displaystyle 0.1\)}%
\end{pgfscope}%
\begin{pgfscope}%
\pgfsetbuttcap%
\pgfsetroundjoin%
\definecolor{currentfill}{rgb}{0.000000,0.000000,0.000000}%
\pgfsetfillcolor{currentfill}%
\pgfsetlinewidth{0.803000pt}%
\definecolor{currentstroke}{rgb}{0.000000,0.000000,0.000000}%
\pgfsetstrokecolor{currentstroke}%
\pgfsetdash{}{0pt}%
\pgfsys@defobject{currentmarker}{\pgfqpoint{-0.048611in}{0.000000in}}{\pgfqpoint{0.000000in}{0.000000in}}{%
\pgfpathmoveto{\pgfqpoint{0.000000in}{0.000000in}}%
\pgfpathlineto{\pgfqpoint{-0.048611in}{0.000000in}}%
\pgfusepath{stroke,fill}%
}%
\begin{pgfscope}%
\pgfsys@transformshift{0.564660in}{1.668338in}%
\pgfsys@useobject{currentmarker}{}%
\end{pgfscope}%
\end{pgfscope}%
\begin{pgfscope}%
\definecolor{textcolor}{rgb}{0.000000,0.000000,0.000000}%
\pgfsetstrokecolor{textcolor}%
\pgfsetfillcolor{textcolor}%
\pgftext[x=0.289968in,y=1.615576in,left,base]{\color{textcolor}\rmfamily\fontsize{10.000000}{12.000000}\selectfont \(\displaystyle 0.2\)}%
\end{pgfscope}%
\begin{pgfscope}%
\pgfsetbuttcap%
\pgfsetroundjoin%
\definecolor{currentfill}{rgb}{0.000000,0.000000,0.000000}%
\pgfsetfillcolor{currentfill}%
\pgfsetlinewidth{0.803000pt}%
\definecolor{currentstroke}{rgb}{0.000000,0.000000,0.000000}%
\pgfsetstrokecolor{currentstroke}%
\pgfsetdash{}{0pt}%
\pgfsys@defobject{currentmarker}{\pgfqpoint{-0.048611in}{0.000000in}}{\pgfqpoint{0.000000in}{0.000000in}}{%
\pgfpathmoveto{\pgfqpoint{0.000000in}{0.000000in}}%
\pgfpathlineto{\pgfqpoint{-0.048611in}{0.000000in}}%
\pgfusepath{stroke,fill}%
}%
\begin{pgfscope}%
\pgfsys@transformshift{0.564660in}{2.240606in}%
\pgfsys@useobject{currentmarker}{}%
\end{pgfscope}%
\end{pgfscope}%
\begin{pgfscope}%
\definecolor{textcolor}{rgb}{0.000000,0.000000,0.000000}%
\pgfsetstrokecolor{textcolor}%
\pgfsetfillcolor{textcolor}%
\pgftext[x=0.289968in,y=2.187844in,left,base]{\color{textcolor}\rmfamily\fontsize{10.000000}{12.000000}\selectfont \(\displaystyle 0.3\)}%
\end{pgfscope}%
\begin{pgfscope}%
\pgfsetbuttcap%
\pgfsetroundjoin%
\definecolor{currentfill}{rgb}{0.000000,0.000000,0.000000}%
\pgfsetfillcolor{currentfill}%
\pgfsetlinewidth{0.803000pt}%
\definecolor{currentstroke}{rgb}{0.000000,0.000000,0.000000}%
\pgfsetstrokecolor{currentstroke}%
\pgfsetdash{}{0pt}%
\pgfsys@defobject{currentmarker}{\pgfqpoint{-0.048611in}{0.000000in}}{\pgfqpoint{0.000000in}{0.000000in}}{%
\pgfpathmoveto{\pgfqpoint{0.000000in}{0.000000in}}%
\pgfpathlineto{\pgfqpoint{-0.048611in}{0.000000in}}%
\pgfusepath{stroke,fill}%
}%
\begin{pgfscope}%
\pgfsys@transformshift{0.564660in}{2.812874in}%
\pgfsys@useobject{currentmarker}{}%
\end{pgfscope}%
\end{pgfscope}%
\begin{pgfscope}%
\definecolor{textcolor}{rgb}{0.000000,0.000000,0.000000}%
\pgfsetstrokecolor{textcolor}%
\pgfsetfillcolor{textcolor}%
\pgftext[x=0.289968in,y=2.760112in,left,base]{\color{textcolor}\rmfamily\fontsize{10.000000}{12.000000}\selectfont \(\displaystyle 0.4\)}%
\end{pgfscope}%
\begin{pgfscope}%
\pgfsetbuttcap%
\pgfsetroundjoin%
\definecolor{currentfill}{rgb}{0.000000,0.000000,0.000000}%
\pgfsetfillcolor{currentfill}%
\pgfsetlinewidth{0.803000pt}%
\definecolor{currentstroke}{rgb}{0.000000,0.000000,0.000000}%
\pgfsetstrokecolor{currentstroke}%
\pgfsetdash{}{0pt}%
\pgfsys@defobject{currentmarker}{\pgfqpoint{-0.048611in}{0.000000in}}{\pgfqpoint{0.000000in}{0.000000in}}{%
\pgfpathmoveto{\pgfqpoint{0.000000in}{0.000000in}}%
\pgfpathlineto{\pgfqpoint{-0.048611in}{0.000000in}}%
\pgfusepath{stroke,fill}%
}%
\begin{pgfscope}%
\pgfsys@transformshift{0.564660in}{3.385142in}%
\pgfsys@useobject{currentmarker}{}%
\end{pgfscope}%
\end{pgfscope}%
\begin{pgfscope}%
\definecolor{textcolor}{rgb}{0.000000,0.000000,0.000000}%
\pgfsetstrokecolor{textcolor}%
\pgfsetfillcolor{textcolor}%
\pgftext[x=0.289968in,y=3.332380in,left,base]{\color{textcolor}\rmfamily\fontsize{10.000000}{12.000000}\selectfont \(\displaystyle 0.5\)}%
\end{pgfscope}%
\begin{pgfscope}%
\definecolor{textcolor}{rgb}{0.000000,0.000000,0.000000}%
\pgfsetstrokecolor{textcolor}%
\pgfsetfillcolor{textcolor}%
\pgftext[x=0.234413in,y=2.031603in,,bottom,rotate=90.000000]{\color{textcolor}\rmfamily\fontsize{10.000000}{12.000000}\selectfont \(\displaystyle V_d\) (mL)}%
\end{pgfscope}%
\begin{pgfscope}%
\pgfsetrectcap%
\pgfsetmiterjoin%
\pgfsetlinewidth{0.803000pt}%
\definecolor{currentstroke}{rgb}{0.501961,0.501961,0.501961}%
\pgfsetstrokecolor{currentstroke}%
\pgfsetdash{}{0pt}%
\pgfpathmoveto{\pgfqpoint{0.564660in}{0.521603in}}%
\pgfpathlineto{\pgfqpoint{0.564660in}{3.541603in}}%
\pgfusepath{stroke}%
\end{pgfscope}%
\begin{pgfscope}%
\pgfsetrectcap%
\pgfsetmiterjoin%
\pgfsetlinewidth{0.803000pt}%
\definecolor{currentstroke}{rgb}{0.501961,0.501961,0.501961}%
\pgfsetstrokecolor{currentstroke}%
\pgfsetdash{}{0pt}%
\pgfpathmoveto{\pgfqpoint{4.284660in}{0.521603in}}%
\pgfpathlineto{\pgfqpoint{4.284660in}{3.541603in}}%
\pgfusepath{stroke}%
\end{pgfscope}%
\begin{pgfscope}%
\pgfsetrectcap%
\pgfsetmiterjoin%
\pgfsetlinewidth{0.803000pt}%
\definecolor{currentstroke}{rgb}{0.501961,0.501961,0.501961}%
\pgfsetstrokecolor{currentstroke}%
\pgfsetdash{}{0pt}%
\pgfpathmoveto{\pgfqpoint{0.564660in}{0.521603in}}%
\pgfpathlineto{\pgfqpoint{4.284660in}{0.521603in}}%
\pgfusepath{stroke}%
\end{pgfscope}%
\begin{pgfscope}%
\pgfsetrectcap%
\pgfsetmiterjoin%
\pgfsetlinewidth{0.803000pt}%
\definecolor{currentstroke}{rgb}{0.501961,0.501961,0.501961}%
\pgfsetstrokecolor{currentstroke}%
\pgfsetdash{}{0pt}%
\pgfpathmoveto{\pgfqpoint{0.564660in}{3.541603in}}%
\pgfpathlineto{\pgfqpoint{4.284660in}{3.541603in}}%
\pgfusepath{stroke}%
\end{pgfscope}%
\begin{pgfscope}%
\pgfpathrectangle{\pgfqpoint{4.517160in}{0.521603in}}{\pgfqpoint{0.151000in}{3.020000in}}%
\pgfusepath{clip}%
\pgfsetbuttcap%
\pgfsetmiterjoin%
\definecolor{currentfill}{rgb}{1.000000,1.000000,1.000000}%
\pgfsetfillcolor{currentfill}%
\pgfsetlinewidth{0.010037pt}%
\definecolor{currentstroke}{rgb}{1.000000,1.000000,1.000000}%
\pgfsetstrokecolor{currentstroke}%
\pgfsetdash{}{0pt}%
\pgfpathmoveto{\pgfqpoint{4.517160in}{0.521603in}}%
\pgfpathlineto{\pgfqpoint{4.517160in}{0.953032in}}%
\pgfpathlineto{\pgfqpoint{4.517160in}{3.110175in}}%
\pgfpathlineto{\pgfqpoint{4.517160in}{3.541603in}}%
\pgfpathlineto{\pgfqpoint{4.668160in}{3.541603in}}%
\pgfpathlineto{\pgfqpoint{4.668160in}{3.110175in}}%
\pgfpathlineto{\pgfqpoint{4.668160in}{0.953032in}}%
\pgfpathlineto{\pgfqpoint{4.668160in}{0.521603in}}%
\pgfpathclose%
\pgfusepath{stroke,fill}%
\end{pgfscope}%
\begin{pgfscope}%
\pgfpathrectangle{\pgfqpoint{4.517160in}{0.521603in}}{\pgfqpoint{0.151000in}{3.020000in}}%
\pgfusepath{clip}%
\pgfsetbuttcap%
\pgfsetroundjoin%
\definecolor{currentfill}{rgb}{0.061765,0.061765,0.085934}%
\pgfsetfillcolor{currentfill}%
\pgfsetlinewidth{0.000000pt}%
\definecolor{currentstroke}{rgb}{0.000000,0.000000,0.000000}%
\pgfsetstrokecolor{currentstroke}%
\pgfsetdash{}{0pt}%
\pgfpathmoveto{\pgfqpoint{4.517160in}{0.521603in}}%
\pgfpathlineto{\pgfqpoint{4.668160in}{0.521603in}}%
\pgfpathlineto{\pgfqpoint{4.668160in}{0.953032in}}%
\pgfpathlineto{\pgfqpoint{4.517160in}{0.953032in}}%
\pgfpathlineto{\pgfqpoint{4.517160in}{0.521603in}}%
\pgfusepath{fill}%
\end{pgfscope}%
\begin{pgfscope}%
\pgfpathrectangle{\pgfqpoint{4.517160in}{0.521603in}}{\pgfqpoint{0.151000in}{3.020000in}}%
\pgfusepath{clip}%
\pgfsetbuttcap%
\pgfsetroundjoin%
\definecolor{currentfill}{rgb}{0.185294,0.185294,0.257801}%
\pgfsetfillcolor{currentfill}%
\pgfsetlinewidth{0.000000pt}%
\definecolor{currentstroke}{rgb}{0.000000,0.000000,0.000000}%
\pgfsetstrokecolor{currentstroke}%
\pgfsetdash{}{0pt}%
\pgfpathmoveto{\pgfqpoint{4.517160in}{0.953032in}}%
\pgfpathlineto{\pgfqpoint{4.668160in}{0.953032in}}%
\pgfpathlineto{\pgfqpoint{4.668160in}{1.384460in}}%
\pgfpathlineto{\pgfqpoint{4.517160in}{1.384460in}}%
\pgfpathlineto{\pgfqpoint{4.517160in}{0.953032in}}%
\pgfusepath{fill}%
\end{pgfscope}%
\begin{pgfscope}%
\pgfpathrectangle{\pgfqpoint{4.517160in}{0.521603in}}{\pgfqpoint{0.151000in}{3.020000in}}%
\pgfusepath{clip}%
\pgfsetbuttcap%
\pgfsetroundjoin%
\definecolor{currentfill}{rgb}{0.312255,0.312255,0.434442}%
\pgfsetfillcolor{currentfill}%
\pgfsetlinewidth{0.000000pt}%
\definecolor{currentstroke}{rgb}{0.000000,0.000000,0.000000}%
\pgfsetstrokecolor{currentstroke}%
\pgfsetdash{}{0pt}%
\pgfpathmoveto{\pgfqpoint{4.517160in}{1.384460in}}%
\pgfpathlineto{\pgfqpoint{4.668160in}{1.384460in}}%
\pgfpathlineto{\pgfqpoint{4.668160in}{1.815889in}}%
\pgfpathlineto{\pgfqpoint{4.517160in}{1.815889in}}%
\pgfpathlineto{\pgfqpoint{4.517160in}{1.384460in}}%
\pgfusepath{fill}%
\end{pgfscope}%
\begin{pgfscope}%
\pgfpathrectangle{\pgfqpoint{4.517160in}{0.521603in}}{\pgfqpoint{0.151000in}{3.020000in}}%
\pgfusepath{clip}%
\pgfsetbuttcap%
\pgfsetroundjoin%
\definecolor{currentfill}{rgb}{0.439216,0.484130,0.564216}%
\pgfsetfillcolor{currentfill}%
\pgfsetlinewidth{0.000000pt}%
\definecolor{currentstroke}{rgb}{0.000000,0.000000,0.000000}%
\pgfsetstrokecolor{currentstroke}%
\pgfsetdash{}{0pt}%
\pgfpathmoveto{\pgfqpoint{4.517160in}{1.815889in}}%
\pgfpathlineto{\pgfqpoint{4.668160in}{1.815889in}}%
\pgfpathlineto{\pgfqpoint{4.668160in}{2.247318in}}%
\pgfpathlineto{\pgfqpoint{4.517160in}{2.247318in}}%
\pgfpathlineto{\pgfqpoint{4.517160in}{1.815889in}}%
\pgfusepath{fill}%
\end{pgfscope}%
\begin{pgfscope}%
\pgfpathrectangle{\pgfqpoint{4.517160in}{0.521603in}}{\pgfqpoint{0.151000in}{3.020000in}}%
\pgfusepath{clip}%
\pgfsetbuttcap%
\pgfsetroundjoin%
\definecolor{currentfill}{rgb}{0.562745,0.653983,0.687745}%
\pgfsetfillcolor{currentfill}%
\pgfsetlinewidth{0.000000pt}%
\definecolor{currentstroke}{rgb}{0.000000,0.000000,0.000000}%
\pgfsetstrokecolor{currentstroke}%
\pgfsetdash{}{0pt}%
\pgfpathmoveto{\pgfqpoint{4.517160in}{2.247318in}}%
\pgfpathlineto{\pgfqpoint{4.668160in}{2.247318in}}%
\pgfpathlineto{\pgfqpoint{4.668160in}{2.678746in}}%
\pgfpathlineto{\pgfqpoint{4.517160in}{2.678746in}}%
\pgfpathlineto{\pgfqpoint{4.517160in}{2.247318in}}%
\pgfusepath{fill}%
\end{pgfscope}%
\begin{pgfscope}%
\pgfpathrectangle{\pgfqpoint{4.517160in}{0.521603in}}{\pgfqpoint{0.151000in}{3.020000in}}%
\pgfusepath{clip}%
\pgfsetbuttcap%
\pgfsetroundjoin%
\definecolor{currentfill}{rgb}{0.710478,0.814706,0.814706}%
\pgfsetfillcolor{currentfill}%
\pgfsetlinewidth{0.000000pt}%
\definecolor{currentstroke}{rgb}{0.000000,0.000000,0.000000}%
\pgfsetstrokecolor{currentstroke}%
\pgfsetdash{}{0pt}%
\pgfpathmoveto{\pgfqpoint{4.517160in}{2.678746in}}%
\pgfpathlineto{\pgfqpoint{4.668160in}{2.678746in}}%
\pgfpathlineto{\pgfqpoint{4.668160in}{3.110175in}}%
\pgfpathlineto{\pgfqpoint{4.517160in}{3.110175in}}%
\pgfpathlineto{\pgfqpoint{4.517160in}{2.678746in}}%
\pgfusepath{fill}%
\end{pgfscope}%
\begin{pgfscope}%
\pgfpathrectangle{\pgfqpoint{4.517160in}{0.521603in}}{\pgfqpoint{0.151000in}{3.020000in}}%
\pgfusepath{clip}%
\pgfsetbuttcap%
\pgfsetroundjoin%
\definecolor{currentfill}{rgb}{0.903493,0.938235,0.938235}%
\pgfsetfillcolor{currentfill}%
\pgfsetlinewidth{0.000000pt}%
\definecolor{currentstroke}{rgb}{0.000000,0.000000,0.000000}%
\pgfsetstrokecolor{currentstroke}%
\pgfsetdash{}{0pt}%
\pgfpathmoveto{\pgfqpoint{4.517160in}{3.110175in}}%
\pgfpathlineto{\pgfqpoint{4.668160in}{3.110175in}}%
\pgfpathlineto{\pgfqpoint{4.668160in}{3.541603in}}%
\pgfpathlineto{\pgfqpoint{4.517160in}{3.541603in}}%
\pgfpathlineto{\pgfqpoint{4.517160in}{3.110175in}}%
\pgfusepath{fill}%
\end{pgfscope}%
\begin{pgfscope}%
\pgfsetbuttcap%
\pgfsetroundjoin%
\definecolor{currentfill}{rgb}{0.000000,0.000000,0.000000}%
\pgfsetfillcolor{currentfill}%
\pgfsetlinewidth{0.803000pt}%
\definecolor{currentstroke}{rgb}{0.000000,0.000000,0.000000}%
\pgfsetstrokecolor{currentstroke}%
\pgfsetdash{}{0pt}%
\pgfsys@defobject{currentmarker}{\pgfqpoint{0.000000in}{0.000000in}}{\pgfqpoint{0.048611in}{0.000000in}}{%
\pgfpathmoveto{\pgfqpoint{0.000000in}{0.000000in}}%
\pgfpathlineto{\pgfqpoint{0.048611in}{0.000000in}}%
\pgfusepath{stroke,fill}%
}%
\begin{pgfscope}%
\pgfsys@transformshift{4.668160in}{0.521603in}%
\pgfsys@useobject{currentmarker}{}%
\end{pgfscope}%
\end{pgfscope}%
\begin{pgfscope}%
\definecolor{textcolor}{rgb}{0.000000,0.000000,0.000000}%
\pgfsetstrokecolor{textcolor}%
\pgfsetfillcolor{textcolor}%
\pgftext[x=4.765383in,y=0.468842in,left,base]{\color{textcolor}\rmfamily\fontsize{10.000000}{12.000000}\selectfont \(\displaystyle 0.0\)}%
\end{pgfscope}%
\begin{pgfscope}%
\pgfsetbuttcap%
\pgfsetroundjoin%
\definecolor{currentfill}{rgb}{0.000000,0.000000,0.000000}%
\pgfsetfillcolor{currentfill}%
\pgfsetlinewidth{0.803000pt}%
\definecolor{currentstroke}{rgb}{0.000000,0.000000,0.000000}%
\pgfsetstrokecolor{currentstroke}%
\pgfsetdash{}{0pt}%
\pgfsys@defobject{currentmarker}{\pgfqpoint{0.000000in}{0.000000in}}{\pgfqpoint{0.048611in}{0.000000in}}{%
\pgfpathmoveto{\pgfqpoint{0.000000in}{0.000000in}}%
\pgfpathlineto{\pgfqpoint{0.048611in}{0.000000in}}%
\pgfusepath{stroke,fill}%
}%
\begin{pgfscope}%
\pgfsys@transformshift{4.668160in}{0.953032in}%
\pgfsys@useobject{currentmarker}{}%
\end{pgfscope}%
\end{pgfscope}%
\begin{pgfscope}%
\definecolor{textcolor}{rgb}{0.000000,0.000000,0.000000}%
\pgfsetstrokecolor{textcolor}%
\pgfsetfillcolor{textcolor}%
\pgftext[x=4.765383in,y=0.900270in,left,base]{\color{textcolor}\rmfamily\fontsize{10.000000}{12.000000}\selectfont \(\displaystyle 0.2\)}%
\end{pgfscope}%
\begin{pgfscope}%
\pgfsetbuttcap%
\pgfsetroundjoin%
\definecolor{currentfill}{rgb}{0.000000,0.000000,0.000000}%
\pgfsetfillcolor{currentfill}%
\pgfsetlinewidth{0.803000pt}%
\definecolor{currentstroke}{rgb}{0.000000,0.000000,0.000000}%
\pgfsetstrokecolor{currentstroke}%
\pgfsetdash{}{0pt}%
\pgfsys@defobject{currentmarker}{\pgfqpoint{0.000000in}{0.000000in}}{\pgfqpoint{0.048611in}{0.000000in}}{%
\pgfpathmoveto{\pgfqpoint{0.000000in}{0.000000in}}%
\pgfpathlineto{\pgfqpoint{0.048611in}{0.000000in}}%
\pgfusepath{stroke,fill}%
}%
\begin{pgfscope}%
\pgfsys@transformshift{4.668160in}{1.384460in}%
\pgfsys@useobject{currentmarker}{}%
\end{pgfscope}%
\end{pgfscope}%
\begin{pgfscope}%
\definecolor{textcolor}{rgb}{0.000000,0.000000,0.000000}%
\pgfsetstrokecolor{textcolor}%
\pgfsetfillcolor{textcolor}%
\pgftext[x=4.765383in,y=1.331699in,left,base]{\color{textcolor}\rmfamily\fontsize{10.000000}{12.000000}\selectfont \(\displaystyle 0.4\)}%
\end{pgfscope}%
\begin{pgfscope}%
\pgfsetbuttcap%
\pgfsetroundjoin%
\definecolor{currentfill}{rgb}{0.000000,0.000000,0.000000}%
\pgfsetfillcolor{currentfill}%
\pgfsetlinewidth{0.803000pt}%
\definecolor{currentstroke}{rgb}{0.000000,0.000000,0.000000}%
\pgfsetstrokecolor{currentstroke}%
\pgfsetdash{}{0pt}%
\pgfsys@defobject{currentmarker}{\pgfqpoint{0.000000in}{0.000000in}}{\pgfqpoint{0.048611in}{0.000000in}}{%
\pgfpathmoveto{\pgfqpoint{0.000000in}{0.000000in}}%
\pgfpathlineto{\pgfqpoint{0.048611in}{0.000000in}}%
\pgfusepath{stroke,fill}%
}%
\begin{pgfscope}%
\pgfsys@transformshift{4.668160in}{1.815889in}%
\pgfsys@useobject{currentmarker}{}%
\end{pgfscope}%
\end{pgfscope}%
\begin{pgfscope}%
\definecolor{textcolor}{rgb}{0.000000,0.000000,0.000000}%
\pgfsetstrokecolor{textcolor}%
\pgfsetfillcolor{textcolor}%
\pgftext[x=4.765383in,y=1.763128in,left,base]{\color{textcolor}\rmfamily\fontsize{10.000000}{12.000000}\selectfont \(\displaystyle 0.6\)}%
\end{pgfscope}%
\begin{pgfscope}%
\pgfsetbuttcap%
\pgfsetroundjoin%
\definecolor{currentfill}{rgb}{0.000000,0.000000,0.000000}%
\pgfsetfillcolor{currentfill}%
\pgfsetlinewidth{0.803000pt}%
\definecolor{currentstroke}{rgb}{0.000000,0.000000,0.000000}%
\pgfsetstrokecolor{currentstroke}%
\pgfsetdash{}{0pt}%
\pgfsys@defobject{currentmarker}{\pgfqpoint{0.000000in}{0.000000in}}{\pgfqpoint{0.048611in}{0.000000in}}{%
\pgfpathmoveto{\pgfqpoint{0.000000in}{0.000000in}}%
\pgfpathlineto{\pgfqpoint{0.048611in}{0.000000in}}%
\pgfusepath{stroke,fill}%
}%
\begin{pgfscope}%
\pgfsys@transformshift{4.668160in}{2.247318in}%
\pgfsys@useobject{currentmarker}{}%
\end{pgfscope}%
\end{pgfscope}%
\begin{pgfscope}%
\definecolor{textcolor}{rgb}{0.000000,0.000000,0.000000}%
\pgfsetstrokecolor{textcolor}%
\pgfsetfillcolor{textcolor}%
\pgftext[x=4.765383in,y=2.194556in,left,base]{\color{textcolor}\rmfamily\fontsize{10.000000}{12.000000}\selectfont \(\displaystyle 0.8\)}%
\end{pgfscope}%
\begin{pgfscope}%
\pgfsetbuttcap%
\pgfsetroundjoin%
\definecolor{currentfill}{rgb}{0.000000,0.000000,0.000000}%
\pgfsetfillcolor{currentfill}%
\pgfsetlinewidth{0.803000pt}%
\definecolor{currentstroke}{rgb}{0.000000,0.000000,0.000000}%
\pgfsetstrokecolor{currentstroke}%
\pgfsetdash{}{0pt}%
\pgfsys@defobject{currentmarker}{\pgfqpoint{0.000000in}{0.000000in}}{\pgfqpoint{0.048611in}{0.000000in}}{%
\pgfpathmoveto{\pgfqpoint{0.000000in}{0.000000in}}%
\pgfpathlineto{\pgfqpoint{0.048611in}{0.000000in}}%
\pgfusepath{stroke,fill}%
}%
\begin{pgfscope}%
\pgfsys@transformshift{4.668160in}{2.678746in}%
\pgfsys@useobject{currentmarker}{}%
\end{pgfscope}%
\end{pgfscope}%
\begin{pgfscope}%
\definecolor{textcolor}{rgb}{0.000000,0.000000,0.000000}%
\pgfsetstrokecolor{textcolor}%
\pgfsetfillcolor{textcolor}%
\pgftext[x=4.765383in,y=2.625985in,left,base]{\color{textcolor}\rmfamily\fontsize{10.000000}{12.000000}\selectfont \(\displaystyle 1.0\)}%
\end{pgfscope}%
\begin{pgfscope}%
\pgfsetbuttcap%
\pgfsetroundjoin%
\definecolor{currentfill}{rgb}{0.000000,0.000000,0.000000}%
\pgfsetfillcolor{currentfill}%
\pgfsetlinewidth{0.803000pt}%
\definecolor{currentstroke}{rgb}{0.000000,0.000000,0.000000}%
\pgfsetstrokecolor{currentstroke}%
\pgfsetdash{}{0pt}%
\pgfsys@defobject{currentmarker}{\pgfqpoint{0.000000in}{0.000000in}}{\pgfqpoint{0.048611in}{0.000000in}}{%
\pgfpathmoveto{\pgfqpoint{0.000000in}{0.000000in}}%
\pgfpathlineto{\pgfqpoint{0.048611in}{0.000000in}}%
\pgfusepath{stroke,fill}%
}%
\begin{pgfscope}%
\pgfsys@transformshift{4.668160in}{3.110175in}%
\pgfsys@useobject{currentmarker}{}%
\end{pgfscope}%
\end{pgfscope}%
\begin{pgfscope}%
\definecolor{textcolor}{rgb}{0.000000,0.000000,0.000000}%
\pgfsetstrokecolor{textcolor}%
\pgfsetfillcolor{textcolor}%
\pgftext[x=4.765383in,y=3.057413in,left,base]{\color{textcolor}\rmfamily\fontsize{10.000000}{12.000000}\selectfont \(\displaystyle 1.2\)}%
\end{pgfscope}%
\begin{pgfscope}%
\pgfsetbuttcap%
\pgfsetroundjoin%
\definecolor{currentfill}{rgb}{0.000000,0.000000,0.000000}%
\pgfsetfillcolor{currentfill}%
\pgfsetlinewidth{0.803000pt}%
\definecolor{currentstroke}{rgb}{0.000000,0.000000,0.000000}%
\pgfsetstrokecolor{currentstroke}%
\pgfsetdash{}{0pt}%
\pgfsys@defobject{currentmarker}{\pgfqpoint{0.000000in}{0.000000in}}{\pgfqpoint{0.048611in}{0.000000in}}{%
\pgfpathmoveto{\pgfqpoint{0.000000in}{0.000000in}}%
\pgfpathlineto{\pgfqpoint{0.048611in}{0.000000in}}%
\pgfusepath{stroke,fill}%
}%
\begin{pgfscope}%
\pgfsys@transformshift{4.668160in}{3.541603in}%
\pgfsys@useobject{currentmarker}{}%
\end{pgfscope}%
\end{pgfscope}%
\begin{pgfscope}%
\definecolor{textcolor}{rgb}{0.000000,0.000000,0.000000}%
\pgfsetstrokecolor{textcolor}%
\pgfsetfillcolor{textcolor}%
\pgftext[x=4.765383in,y=3.488842in,left,base]{\color{textcolor}\rmfamily\fontsize{10.000000}{12.000000}\selectfont \(\displaystyle 1.4\)}%
\end{pgfscope}%
\begin{pgfscope}%
\definecolor{textcolor}{rgb}{0.000000,0.000000,0.000000}%
\pgfsetstrokecolor{textcolor}%
\pgfsetfillcolor{textcolor}%
\pgftext[x=4.998408in,y=2.031603in,,top,rotate=90.000000]{\color{textcolor}\rmfamily\fontsize{10.000000}{12.000000}\selectfont \(\displaystyle q\) (C)}%
\end{pgfscope}%
\begin{pgfscope}%
\definecolor{textcolor}{rgb}{0.000000,0.000000,0.000000}%
\pgfsetstrokecolor{textcolor}%
\pgfsetfillcolor{textcolor}%
\pgftext[x=4.668160in,y=3.583270in,right,base]{\color{textcolor}\rmfamily\fontsize{10.000000}{12.000000}\selectfont \(\displaystyle \times10^{-9}\)}%
\end{pgfscope}%
\begin{pgfscope}%
\pgfsetbuttcap%
\pgfsetmiterjoin%
\pgfsetlinewidth{0.803000pt}%
\definecolor{currentstroke}{rgb}{0.501961,0.501961,0.501961}%
\pgfsetstrokecolor{currentstroke}%
\pgfsetdash{}{0pt}%
\pgfpathmoveto{\pgfqpoint{4.517160in}{0.521603in}}%
\pgfpathlineto{\pgfqpoint{4.517160in}{0.953032in}}%
\pgfpathlineto{\pgfqpoint{4.517160in}{3.110175in}}%
\pgfpathlineto{\pgfqpoint{4.517160in}{3.541603in}}%
\pgfpathlineto{\pgfqpoint{4.668160in}{3.541603in}}%
\pgfpathlineto{\pgfqpoint{4.668160in}{3.110175in}}%
\pgfpathlineto{\pgfqpoint{4.668160in}{0.953032in}}%
\pgfpathlineto{\pgfqpoint{4.668160in}{0.521603in}}%
\pgfpathclose%
\pgfusepath{stroke}%
\end{pgfscope}%
\end{pgfpicture}%
\makeatother%
\endgroup%

    \caption{Charge $q$, as a function of $V_d$, $\varphi_s$.\label{fig:charge}}
\end{figure}

A two-ways T-test comparison of charge distributions between the drop bounce experiment and a corollary experiment with zero electric field at the time of drop deposition on the superhydrophobic surface suggests that the drop charge is induced by the electric field, rather than through contact charging on the PTFE layer ($t = 5.11, p = 0.0002$). The T-test informs us that the charge distribution  are about 5 times more different from each other as they are within each other, and there is a 0.02$\%$ probability that this result happened by chance. This corollary experiment is documented in Appendix \ref{sec.drop_charge}.

The model $q \sim kAE_0$ is incidentally very similar to the classical solution for the surface charge density of a half-spherical conductor with a field-induced dipole \cite{david_j._griffiths_introduction_1999}
\begin{eqnarray*}
q &=& 3 \epsilon_0 E_0 \int_A \cos \theta dA \\
&=& 3 \pi^{1/3} 6 \left(6 V_d \right)^{2/3} \epsilon_0 E_0 \int^{4 \pi/2}_{\pi / 2} \cos \theta d\theta \\
&=& k E_0 V_d^{2/3}
\end{eqnarray*}
with $k \approx 1.3 \times 10^{-10}$. This is also of a similar form to the charge found by Takamatsu and coauthors for drops falling from a grounded nozzle in an external electric field \cite{takamatsu_theoretical_1981}
\[q = 4 \pi \epsilon_0 \beta E_0 R_d^2 \]
with $\beta \approx 2.63$.

\section{Scale Quantities}
The dielectrophoretic force plays a very small role when drops have net charge in a DC field; the condition to neglect the DEP force was satisfied for all experiments in the dataset. Dimensional drop apoapses scale closely with $\mathbb{E}\mbox{u}$ as seen in Figure \ref{fig:series_s_eu}. The relative magnitudes of the simulated forces felt by the drops is shown in Figure \ref{fig:forces}. Forces acting on the drops vary in magnitude between $\mathcal{O}(10^{-6})$-$\mathcal{O}(10^{-4})$ N. We see that, of the drops in the experimental dataset only the two with the largest $\mathbb{E}\mbox{u}$, $\mathbb{E}\mbox{u} \sim \mathcal{O}(1)$ could appropriately be said to be in the inertial electro-viscous regime. In all other cases image forces are much stronger than drag. For these drops $\mathbb{E}\mbox{u} \gg 1/8 \pi$, and are likely on escape trajectories. The image forces themselves rapidly become small compared to Coulomb forces for drops with apoapses $\mbox{max}\left( y\right) \gtrapprox L$, thus it is reasonable to claim that for intermediate drops Coulomb force scales as inertia, and we can neglect the effects of drag and image forces.

\begin{figure}[!htb]
    \centering
    \input{../figures/series_s_eu.pgf}
    \caption{Drop trajectories as a function of $\mathbb{E}\mbox{u}$.\label{fig:series_s_eu}}
\end{figure}
\begin{figure}[!htb]
    \centering
    \resizebox{14cm}{!}{%% Creator: Matplotlib, PGF backend
%%
%% To include the figure in your LaTeX document, write
%%   \input{<filename>.pgf}
%%
%% Make sure the required packages are loaded in your preamble
%%   \usepackage{pgf}
%%
%% Figures using additional raster images can only be included by \input if
%% they are in the same directory as the main LaTeX file. For loading figures
%% from other directories you can use the `import` package
%%   \usepackage{import}
%% and then include the figures with
%%   \import{<path to file>}{<filename>.pgf}
%%
%% Matplotlib used the following preamble
%%   \usepackage{fontspec}
%%   \setmainfont{DejaVuSerif.ttf}[Path=/home/erin/anaconda3/lib/python3.6/site-packages/matplotlib/mpl-data/fonts/ttf/]
%%   \setsansfont{DejaVuSans.ttf}[Path=/home/erin/anaconda3/lib/python3.6/site-packages/matplotlib/mpl-data/fonts/ttf/]
%%   \setmonofont{DejaVuSansMono.ttf}[Path=/home/erin/anaconda3/lib/python3.6/site-packages/matplotlib/mpl-data/fonts/ttf/]
%%
\begingroup%
\makeatletter%
\begin{pgfpicture}%
\pgfpathrectangle{\pgfpointorigin}{\pgfqpoint{5.473804in}{3.694691in}}%
\pgfusepath{use as bounding box, clip}%
\begin{pgfscope}%
\pgfsetbuttcap%
\pgfsetmiterjoin%
\definecolor{currentfill}{rgb}{1.000000,1.000000,1.000000}%
\pgfsetfillcolor{currentfill}%
\pgfsetlinewidth{0.000000pt}%
\definecolor{currentstroke}{rgb}{1.000000,1.000000,1.000000}%
\pgfsetstrokecolor{currentstroke}%
\pgfsetdash{}{0pt}%
\pgfpathmoveto{\pgfqpoint{0.000000in}{0.000000in}}%
\pgfpathlineto{\pgfqpoint{5.473804in}{0.000000in}}%
\pgfpathlineto{\pgfqpoint{5.473804in}{3.694691in}}%
\pgfpathlineto{\pgfqpoint{0.000000in}{3.694691in}}%
\pgfpathclose%
\pgfusepath{fill}%
\end{pgfscope}%
\begin{pgfscope}%
\pgfsetbuttcap%
\pgfsetmiterjoin%
\definecolor{currentfill}{rgb}{1.000000,1.000000,1.000000}%
\pgfsetfillcolor{currentfill}%
\pgfsetlinewidth{0.000000pt}%
\definecolor{currentstroke}{rgb}{0.000000,0.000000,0.000000}%
\pgfsetstrokecolor{currentstroke}%
\pgfsetstrokeopacity{0.000000}%
\pgfsetdash{}{0pt}%
\pgfpathmoveto{\pgfqpoint{0.675193in}{0.526079in}}%
\pgfpathlineto{\pgfqpoint{5.325193in}{0.526079in}}%
\pgfpathlineto{\pgfqpoint{5.325193in}{3.546079in}}%
\pgfpathlineto{\pgfqpoint{0.675193in}{3.546079in}}%
\pgfpathclose%
\pgfusepath{fill}%
\end{pgfscope}%
\begin{pgfscope}%
\pgfsetbuttcap%
\pgfsetroundjoin%
\definecolor{currentfill}{rgb}{0.000000,0.000000,0.000000}%
\pgfsetfillcolor{currentfill}%
\pgfsetlinewidth{0.803000pt}%
\definecolor{currentstroke}{rgb}{0.000000,0.000000,0.000000}%
\pgfsetstrokecolor{currentstroke}%
\pgfsetdash{}{0pt}%
\pgfsys@defobject{currentmarker}{\pgfqpoint{0.000000in}{-0.048611in}}{\pgfqpoint{0.000000in}{0.000000in}}{%
\pgfpathmoveto{\pgfqpoint{0.000000in}{0.000000in}}%
\pgfpathlineto{\pgfqpoint{0.000000in}{-0.048611in}}%
\pgfusepath{stroke,fill}%
}%
\begin{pgfscope}%
\pgfsys@transformshift{1.495889in}{0.526079in}%
\pgfsys@useobject{currentmarker}{}%
\end{pgfscope}%
\end{pgfscope}%
\begin{pgfscope}%
\definecolor{textcolor}{rgb}{0.000000,0.000000,0.000000}%
\pgfsetstrokecolor{textcolor}%
\pgfsetfillcolor{textcolor}%
\pgftext[x=1.495889in,y=0.428857in,,top]{\color{textcolor}\rmfamily\fontsize{10.000000}{12.000000}\selectfont \(\displaystyle 0.2\)}%
\end{pgfscope}%
\begin{pgfscope}%
\pgfsetbuttcap%
\pgfsetroundjoin%
\definecolor{currentfill}{rgb}{0.000000,0.000000,0.000000}%
\pgfsetfillcolor{currentfill}%
\pgfsetlinewidth{0.803000pt}%
\definecolor{currentstroke}{rgb}{0.000000,0.000000,0.000000}%
\pgfsetstrokecolor{currentstroke}%
\pgfsetdash{}{0pt}%
\pgfsys@defobject{currentmarker}{\pgfqpoint{0.000000in}{-0.048611in}}{\pgfqpoint{0.000000in}{0.000000in}}{%
\pgfpathmoveto{\pgfqpoint{0.000000in}{0.000000in}}%
\pgfpathlineto{\pgfqpoint{0.000000in}{-0.048611in}}%
\pgfusepath{stroke,fill}%
}%
\begin{pgfscope}%
\pgfsys@transformshift{2.393477in}{0.526079in}%
\pgfsys@useobject{currentmarker}{}%
\end{pgfscope}%
\end{pgfscope}%
\begin{pgfscope}%
\definecolor{textcolor}{rgb}{0.000000,0.000000,0.000000}%
\pgfsetstrokecolor{textcolor}%
\pgfsetfillcolor{textcolor}%
\pgftext[x=2.393477in,y=0.428857in,,top]{\color{textcolor}\rmfamily\fontsize{10.000000}{12.000000}\selectfont \(\displaystyle 0.4\)}%
\end{pgfscope}%
\begin{pgfscope}%
\pgfsetbuttcap%
\pgfsetroundjoin%
\definecolor{currentfill}{rgb}{0.000000,0.000000,0.000000}%
\pgfsetfillcolor{currentfill}%
\pgfsetlinewidth{0.803000pt}%
\definecolor{currentstroke}{rgb}{0.000000,0.000000,0.000000}%
\pgfsetstrokecolor{currentstroke}%
\pgfsetdash{}{0pt}%
\pgfsys@defobject{currentmarker}{\pgfqpoint{0.000000in}{-0.048611in}}{\pgfqpoint{0.000000in}{0.000000in}}{%
\pgfpathmoveto{\pgfqpoint{0.000000in}{0.000000in}}%
\pgfpathlineto{\pgfqpoint{0.000000in}{-0.048611in}}%
\pgfusepath{stroke,fill}%
}%
\begin{pgfscope}%
\pgfsys@transformshift{3.291064in}{0.526079in}%
\pgfsys@useobject{currentmarker}{}%
\end{pgfscope}%
\end{pgfscope}%
\begin{pgfscope}%
\definecolor{textcolor}{rgb}{0.000000,0.000000,0.000000}%
\pgfsetstrokecolor{textcolor}%
\pgfsetfillcolor{textcolor}%
\pgftext[x=3.291064in,y=0.428857in,,top]{\color{textcolor}\rmfamily\fontsize{10.000000}{12.000000}\selectfont \(\displaystyle 0.6\)}%
\end{pgfscope}%
\begin{pgfscope}%
\pgfsetbuttcap%
\pgfsetroundjoin%
\definecolor{currentfill}{rgb}{0.000000,0.000000,0.000000}%
\pgfsetfillcolor{currentfill}%
\pgfsetlinewidth{0.803000pt}%
\definecolor{currentstroke}{rgb}{0.000000,0.000000,0.000000}%
\pgfsetstrokecolor{currentstroke}%
\pgfsetdash{}{0pt}%
\pgfsys@defobject{currentmarker}{\pgfqpoint{0.000000in}{-0.048611in}}{\pgfqpoint{0.000000in}{0.000000in}}{%
\pgfpathmoveto{\pgfqpoint{0.000000in}{0.000000in}}%
\pgfpathlineto{\pgfqpoint{0.000000in}{-0.048611in}}%
\pgfusepath{stroke,fill}%
}%
\begin{pgfscope}%
\pgfsys@transformshift{4.188652in}{0.526079in}%
\pgfsys@useobject{currentmarker}{}%
\end{pgfscope}%
\end{pgfscope}%
\begin{pgfscope}%
\definecolor{textcolor}{rgb}{0.000000,0.000000,0.000000}%
\pgfsetstrokecolor{textcolor}%
\pgfsetfillcolor{textcolor}%
\pgftext[x=4.188652in,y=0.428857in,,top]{\color{textcolor}\rmfamily\fontsize{10.000000}{12.000000}\selectfont \(\displaystyle 0.8\)}%
\end{pgfscope}%
\begin{pgfscope}%
\pgfsetbuttcap%
\pgfsetroundjoin%
\definecolor{currentfill}{rgb}{0.000000,0.000000,0.000000}%
\pgfsetfillcolor{currentfill}%
\pgfsetlinewidth{0.803000pt}%
\definecolor{currentstroke}{rgb}{0.000000,0.000000,0.000000}%
\pgfsetstrokecolor{currentstroke}%
\pgfsetdash{}{0pt}%
\pgfsys@defobject{currentmarker}{\pgfqpoint{0.000000in}{-0.048611in}}{\pgfqpoint{0.000000in}{0.000000in}}{%
\pgfpathmoveto{\pgfqpoint{0.000000in}{0.000000in}}%
\pgfpathlineto{\pgfqpoint{0.000000in}{-0.048611in}}%
\pgfusepath{stroke,fill}%
}%
\begin{pgfscope}%
\pgfsys@transformshift{5.086239in}{0.526079in}%
\pgfsys@useobject{currentmarker}{}%
\end{pgfscope}%
\end{pgfscope}%
\begin{pgfscope}%
\definecolor{textcolor}{rgb}{0.000000,0.000000,0.000000}%
\pgfsetstrokecolor{textcolor}%
\pgfsetfillcolor{textcolor}%
\pgftext[x=5.086239in,y=0.428857in,,top]{\color{textcolor}\rmfamily\fontsize{10.000000}{12.000000}\selectfont \(\displaystyle 1.0\)}%
\end{pgfscope}%
\begin{pgfscope}%
\definecolor{textcolor}{rgb}{0.000000,0.000000,0.000000}%
\pgfsetstrokecolor{textcolor}%
\pgfsetfillcolor{textcolor}%
\pgftext[x=3.000193in,y=0.238889in,,top]{\color{textcolor}\rmfamily\fontsize{10.000000}{12.000000}\selectfont \(\displaystyle y/L\)}%
\end{pgfscope}%
\begin{pgfscope}%
\pgfsetbuttcap%
\pgfsetroundjoin%
\definecolor{currentfill}{rgb}{0.000000,0.000000,0.000000}%
\pgfsetfillcolor{currentfill}%
\pgfsetlinewidth{0.803000pt}%
\definecolor{currentstroke}{rgb}{0.000000,0.000000,0.000000}%
\pgfsetstrokecolor{currentstroke}%
\pgfsetdash{}{0pt}%
\pgfsys@defobject{currentmarker}{\pgfqpoint{-0.048611in}{0.000000in}}{\pgfqpoint{0.000000in}{0.000000in}}{%
\pgfpathmoveto{\pgfqpoint{0.000000in}{0.000000in}}%
\pgfpathlineto{\pgfqpoint{-0.048611in}{0.000000in}}%
\pgfusepath{stroke,fill}%
}%
\begin{pgfscope}%
\pgfsys@transformshift{0.675193in}{0.903536in}%
\pgfsys@useobject{currentmarker}{}%
\end{pgfscope}%
\end{pgfscope}%
\begin{pgfscope}%
\definecolor{textcolor}{rgb}{0.000000,0.000000,0.000000}%
\pgfsetstrokecolor{textcolor}%
\pgfsetfillcolor{textcolor}%
\pgftext[x=0.289968in,y=0.850774in,left,base]{\color{textcolor}\rmfamily\fontsize{10.000000}{12.000000}\selectfont \(\displaystyle 10^{-2}\)}%
\end{pgfscope}%
\begin{pgfscope}%
\pgfsetbuttcap%
\pgfsetroundjoin%
\definecolor{currentfill}{rgb}{0.000000,0.000000,0.000000}%
\pgfsetfillcolor{currentfill}%
\pgfsetlinewidth{0.803000pt}%
\definecolor{currentstroke}{rgb}{0.000000,0.000000,0.000000}%
\pgfsetstrokecolor{currentstroke}%
\pgfsetdash{}{0pt}%
\pgfsys@defobject{currentmarker}{\pgfqpoint{-0.048611in}{0.000000in}}{\pgfqpoint{0.000000in}{0.000000in}}{%
\pgfpathmoveto{\pgfqpoint{0.000000in}{0.000000in}}%
\pgfpathlineto{\pgfqpoint{-0.048611in}{0.000000in}}%
\pgfusepath{stroke,fill}%
}%
\begin{pgfscope}%
\pgfsys@transformshift{0.675193in}{2.157420in}%
\pgfsys@useobject{currentmarker}{}%
\end{pgfscope}%
\end{pgfscope}%
\begin{pgfscope}%
\definecolor{textcolor}{rgb}{0.000000,0.000000,0.000000}%
\pgfsetstrokecolor{textcolor}%
\pgfsetfillcolor{textcolor}%
\pgftext[x=0.289968in,y=2.104658in,left,base]{\color{textcolor}\rmfamily\fontsize{10.000000}{12.000000}\selectfont \(\displaystyle 10^{-1}\)}%
\end{pgfscope}%
\begin{pgfscope}%
\pgfsetbuttcap%
\pgfsetroundjoin%
\definecolor{currentfill}{rgb}{0.000000,0.000000,0.000000}%
\pgfsetfillcolor{currentfill}%
\pgfsetlinewidth{0.803000pt}%
\definecolor{currentstroke}{rgb}{0.000000,0.000000,0.000000}%
\pgfsetstrokecolor{currentstroke}%
\pgfsetdash{}{0pt}%
\pgfsys@defobject{currentmarker}{\pgfqpoint{-0.048611in}{0.000000in}}{\pgfqpoint{0.000000in}{0.000000in}}{%
\pgfpathmoveto{\pgfqpoint{0.000000in}{0.000000in}}%
\pgfpathlineto{\pgfqpoint{-0.048611in}{0.000000in}}%
\pgfusepath{stroke,fill}%
}%
\begin{pgfscope}%
\pgfsys@transformshift{0.675193in}{3.411303in}%
\pgfsys@useobject{currentmarker}{}%
\end{pgfscope}%
\end{pgfscope}%
\begin{pgfscope}%
\definecolor{textcolor}{rgb}{0.000000,0.000000,0.000000}%
\pgfsetstrokecolor{textcolor}%
\pgfsetfillcolor{textcolor}%
\pgftext[x=0.376774in,y=3.358542in,left,base]{\color{textcolor}\rmfamily\fontsize{10.000000}{12.000000}\selectfont \(\displaystyle 10^{0}\)}%
\end{pgfscope}%
\begin{pgfscope}%
\pgfsetbuttcap%
\pgfsetroundjoin%
\definecolor{currentfill}{rgb}{0.000000,0.000000,0.000000}%
\pgfsetfillcolor{currentfill}%
\pgfsetlinewidth{0.602250pt}%
\definecolor{currentstroke}{rgb}{0.000000,0.000000,0.000000}%
\pgfsetstrokecolor{currentstroke}%
\pgfsetdash{}{0pt}%
\pgfsys@defobject{currentmarker}{\pgfqpoint{-0.027778in}{0.000000in}}{\pgfqpoint{0.000000in}{0.000000in}}{%
\pgfpathmoveto{\pgfqpoint{0.000000in}{0.000000in}}%
\pgfpathlineto{\pgfqpoint{-0.027778in}{0.000000in}}%
\pgfusepath{stroke,fill}%
}%
\begin{pgfscope}%
\pgfsys@transformshift{0.675193in}{0.526079in}%
\pgfsys@useobject{currentmarker}{}%
\end{pgfscope}%
\end{pgfscope}%
\begin{pgfscope}%
\pgfsetbuttcap%
\pgfsetroundjoin%
\definecolor{currentfill}{rgb}{0.000000,0.000000,0.000000}%
\pgfsetfillcolor{currentfill}%
\pgfsetlinewidth{0.602250pt}%
\definecolor{currentstroke}{rgb}{0.000000,0.000000,0.000000}%
\pgfsetstrokecolor{currentstroke}%
\pgfsetdash{}{0pt}%
\pgfsys@defobject{currentmarker}{\pgfqpoint{-0.027778in}{0.000000in}}{\pgfqpoint{0.000000in}{0.000000in}}{%
\pgfpathmoveto{\pgfqpoint{0.000000in}{0.000000in}}%
\pgfpathlineto{\pgfqpoint{-0.027778in}{0.000000in}}%
\pgfusepath{stroke,fill}%
}%
\begin{pgfscope}%
\pgfsys@transformshift{0.675193in}{0.625364in}%
\pgfsys@useobject{currentmarker}{}%
\end{pgfscope}%
\end{pgfscope}%
\begin{pgfscope}%
\pgfsetbuttcap%
\pgfsetroundjoin%
\definecolor{currentfill}{rgb}{0.000000,0.000000,0.000000}%
\pgfsetfillcolor{currentfill}%
\pgfsetlinewidth{0.602250pt}%
\definecolor{currentstroke}{rgb}{0.000000,0.000000,0.000000}%
\pgfsetstrokecolor{currentstroke}%
\pgfsetdash{}{0pt}%
\pgfsys@defobject{currentmarker}{\pgfqpoint{-0.027778in}{0.000000in}}{\pgfqpoint{0.000000in}{0.000000in}}{%
\pgfpathmoveto{\pgfqpoint{0.000000in}{0.000000in}}%
\pgfpathlineto{\pgfqpoint{-0.027778in}{0.000000in}}%
\pgfusepath{stroke,fill}%
}%
\begin{pgfscope}%
\pgfsys@transformshift{0.675193in}{0.709307in}%
\pgfsys@useobject{currentmarker}{}%
\end{pgfscope}%
\end{pgfscope}%
\begin{pgfscope}%
\pgfsetbuttcap%
\pgfsetroundjoin%
\definecolor{currentfill}{rgb}{0.000000,0.000000,0.000000}%
\pgfsetfillcolor{currentfill}%
\pgfsetlinewidth{0.602250pt}%
\definecolor{currentstroke}{rgb}{0.000000,0.000000,0.000000}%
\pgfsetstrokecolor{currentstroke}%
\pgfsetdash{}{0pt}%
\pgfsys@defobject{currentmarker}{\pgfqpoint{-0.027778in}{0.000000in}}{\pgfqpoint{0.000000in}{0.000000in}}{%
\pgfpathmoveto{\pgfqpoint{0.000000in}{0.000000in}}%
\pgfpathlineto{\pgfqpoint{-0.027778in}{0.000000in}}%
\pgfusepath{stroke,fill}%
}%
\begin{pgfscope}%
\pgfsys@transformshift{0.675193in}{0.782022in}%
\pgfsys@useobject{currentmarker}{}%
\end{pgfscope}%
\end{pgfscope}%
\begin{pgfscope}%
\pgfsetbuttcap%
\pgfsetroundjoin%
\definecolor{currentfill}{rgb}{0.000000,0.000000,0.000000}%
\pgfsetfillcolor{currentfill}%
\pgfsetlinewidth{0.602250pt}%
\definecolor{currentstroke}{rgb}{0.000000,0.000000,0.000000}%
\pgfsetstrokecolor{currentstroke}%
\pgfsetdash{}{0pt}%
\pgfsys@defobject{currentmarker}{\pgfqpoint{-0.027778in}{0.000000in}}{\pgfqpoint{0.000000in}{0.000000in}}{%
\pgfpathmoveto{\pgfqpoint{0.000000in}{0.000000in}}%
\pgfpathlineto{\pgfqpoint{-0.027778in}{0.000000in}}%
\pgfusepath{stroke,fill}%
}%
\begin{pgfscope}%
\pgfsys@transformshift{0.675193in}{0.846161in}%
\pgfsys@useobject{currentmarker}{}%
\end{pgfscope}%
\end{pgfscope}%
\begin{pgfscope}%
\pgfsetbuttcap%
\pgfsetroundjoin%
\definecolor{currentfill}{rgb}{0.000000,0.000000,0.000000}%
\pgfsetfillcolor{currentfill}%
\pgfsetlinewidth{0.602250pt}%
\definecolor{currentstroke}{rgb}{0.000000,0.000000,0.000000}%
\pgfsetstrokecolor{currentstroke}%
\pgfsetdash{}{0pt}%
\pgfsys@defobject{currentmarker}{\pgfqpoint{-0.027778in}{0.000000in}}{\pgfqpoint{0.000000in}{0.000000in}}{%
\pgfpathmoveto{\pgfqpoint{0.000000in}{0.000000in}}%
\pgfpathlineto{\pgfqpoint{-0.027778in}{0.000000in}}%
\pgfusepath{stroke,fill}%
}%
\begin{pgfscope}%
\pgfsys@transformshift{0.675193in}{1.280993in}%
\pgfsys@useobject{currentmarker}{}%
\end{pgfscope}%
\end{pgfscope}%
\begin{pgfscope}%
\pgfsetbuttcap%
\pgfsetroundjoin%
\definecolor{currentfill}{rgb}{0.000000,0.000000,0.000000}%
\pgfsetfillcolor{currentfill}%
\pgfsetlinewidth{0.602250pt}%
\definecolor{currentstroke}{rgb}{0.000000,0.000000,0.000000}%
\pgfsetstrokecolor{currentstroke}%
\pgfsetdash{}{0pt}%
\pgfsys@defobject{currentmarker}{\pgfqpoint{-0.027778in}{0.000000in}}{\pgfqpoint{0.000000in}{0.000000in}}{%
\pgfpathmoveto{\pgfqpoint{0.000000in}{0.000000in}}%
\pgfpathlineto{\pgfqpoint{-0.027778in}{0.000000in}}%
\pgfusepath{stroke,fill}%
}%
\begin{pgfscope}%
\pgfsys@transformshift{0.675193in}{1.501791in}%
\pgfsys@useobject{currentmarker}{}%
\end{pgfscope}%
\end{pgfscope}%
\begin{pgfscope}%
\pgfsetbuttcap%
\pgfsetroundjoin%
\definecolor{currentfill}{rgb}{0.000000,0.000000,0.000000}%
\pgfsetfillcolor{currentfill}%
\pgfsetlinewidth{0.602250pt}%
\definecolor{currentstroke}{rgb}{0.000000,0.000000,0.000000}%
\pgfsetstrokecolor{currentstroke}%
\pgfsetdash{}{0pt}%
\pgfsys@defobject{currentmarker}{\pgfqpoint{-0.027778in}{0.000000in}}{\pgfqpoint{0.000000in}{0.000000in}}{%
\pgfpathmoveto{\pgfqpoint{0.000000in}{0.000000in}}%
\pgfpathlineto{\pgfqpoint{-0.027778in}{0.000000in}}%
\pgfusepath{stroke,fill}%
}%
\begin{pgfscope}%
\pgfsys@transformshift{0.675193in}{1.658449in}%
\pgfsys@useobject{currentmarker}{}%
\end{pgfscope}%
\end{pgfscope}%
\begin{pgfscope}%
\pgfsetbuttcap%
\pgfsetroundjoin%
\definecolor{currentfill}{rgb}{0.000000,0.000000,0.000000}%
\pgfsetfillcolor{currentfill}%
\pgfsetlinewidth{0.602250pt}%
\definecolor{currentstroke}{rgb}{0.000000,0.000000,0.000000}%
\pgfsetstrokecolor{currentstroke}%
\pgfsetdash{}{0pt}%
\pgfsys@defobject{currentmarker}{\pgfqpoint{-0.027778in}{0.000000in}}{\pgfqpoint{0.000000in}{0.000000in}}{%
\pgfpathmoveto{\pgfqpoint{0.000000in}{0.000000in}}%
\pgfpathlineto{\pgfqpoint{-0.027778in}{0.000000in}}%
\pgfusepath{stroke,fill}%
}%
\begin{pgfscope}%
\pgfsys@transformshift{0.675193in}{1.779963in}%
\pgfsys@useobject{currentmarker}{}%
\end{pgfscope}%
\end{pgfscope}%
\begin{pgfscope}%
\pgfsetbuttcap%
\pgfsetroundjoin%
\definecolor{currentfill}{rgb}{0.000000,0.000000,0.000000}%
\pgfsetfillcolor{currentfill}%
\pgfsetlinewidth{0.602250pt}%
\definecolor{currentstroke}{rgb}{0.000000,0.000000,0.000000}%
\pgfsetstrokecolor{currentstroke}%
\pgfsetdash{}{0pt}%
\pgfsys@defobject{currentmarker}{\pgfqpoint{-0.027778in}{0.000000in}}{\pgfqpoint{0.000000in}{0.000000in}}{%
\pgfpathmoveto{\pgfqpoint{0.000000in}{0.000000in}}%
\pgfpathlineto{\pgfqpoint{-0.027778in}{0.000000in}}%
\pgfusepath{stroke,fill}%
}%
\begin{pgfscope}%
\pgfsys@transformshift{0.675193in}{1.879247in}%
\pgfsys@useobject{currentmarker}{}%
\end{pgfscope}%
\end{pgfscope}%
\begin{pgfscope}%
\pgfsetbuttcap%
\pgfsetroundjoin%
\definecolor{currentfill}{rgb}{0.000000,0.000000,0.000000}%
\pgfsetfillcolor{currentfill}%
\pgfsetlinewidth{0.602250pt}%
\definecolor{currentstroke}{rgb}{0.000000,0.000000,0.000000}%
\pgfsetstrokecolor{currentstroke}%
\pgfsetdash{}{0pt}%
\pgfsys@defobject{currentmarker}{\pgfqpoint{-0.027778in}{0.000000in}}{\pgfqpoint{0.000000in}{0.000000in}}{%
\pgfpathmoveto{\pgfqpoint{0.000000in}{0.000000in}}%
\pgfpathlineto{\pgfqpoint{-0.027778in}{0.000000in}}%
\pgfusepath{stroke,fill}%
}%
\begin{pgfscope}%
\pgfsys@transformshift{0.675193in}{1.963191in}%
\pgfsys@useobject{currentmarker}{}%
\end{pgfscope}%
\end{pgfscope}%
\begin{pgfscope}%
\pgfsetbuttcap%
\pgfsetroundjoin%
\definecolor{currentfill}{rgb}{0.000000,0.000000,0.000000}%
\pgfsetfillcolor{currentfill}%
\pgfsetlinewidth{0.602250pt}%
\definecolor{currentstroke}{rgb}{0.000000,0.000000,0.000000}%
\pgfsetstrokecolor{currentstroke}%
\pgfsetdash{}{0pt}%
\pgfsys@defobject{currentmarker}{\pgfqpoint{-0.027778in}{0.000000in}}{\pgfqpoint{0.000000in}{0.000000in}}{%
\pgfpathmoveto{\pgfqpoint{0.000000in}{0.000000in}}%
\pgfpathlineto{\pgfqpoint{-0.027778in}{0.000000in}}%
\pgfusepath{stroke,fill}%
}%
\begin{pgfscope}%
\pgfsys@transformshift{0.675193in}{2.035906in}%
\pgfsys@useobject{currentmarker}{}%
\end{pgfscope}%
\end{pgfscope}%
\begin{pgfscope}%
\pgfsetbuttcap%
\pgfsetroundjoin%
\definecolor{currentfill}{rgb}{0.000000,0.000000,0.000000}%
\pgfsetfillcolor{currentfill}%
\pgfsetlinewidth{0.602250pt}%
\definecolor{currentstroke}{rgb}{0.000000,0.000000,0.000000}%
\pgfsetstrokecolor{currentstroke}%
\pgfsetdash{}{0pt}%
\pgfsys@defobject{currentmarker}{\pgfqpoint{-0.027778in}{0.000000in}}{\pgfqpoint{0.000000in}{0.000000in}}{%
\pgfpathmoveto{\pgfqpoint{0.000000in}{0.000000in}}%
\pgfpathlineto{\pgfqpoint{-0.027778in}{0.000000in}}%
\pgfusepath{stroke,fill}%
}%
\begin{pgfscope}%
\pgfsys@transformshift{0.675193in}{2.100045in}%
\pgfsys@useobject{currentmarker}{}%
\end{pgfscope}%
\end{pgfscope}%
\begin{pgfscope}%
\pgfsetbuttcap%
\pgfsetroundjoin%
\definecolor{currentfill}{rgb}{0.000000,0.000000,0.000000}%
\pgfsetfillcolor{currentfill}%
\pgfsetlinewidth{0.602250pt}%
\definecolor{currentstroke}{rgb}{0.000000,0.000000,0.000000}%
\pgfsetstrokecolor{currentstroke}%
\pgfsetdash{}{0pt}%
\pgfsys@defobject{currentmarker}{\pgfqpoint{-0.027778in}{0.000000in}}{\pgfqpoint{0.000000in}{0.000000in}}{%
\pgfpathmoveto{\pgfqpoint{0.000000in}{0.000000in}}%
\pgfpathlineto{\pgfqpoint{-0.027778in}{0.000000in}}%
\pgfusepath{stroke,fill}%
}%
\begin{pgfscope}%
\pgfsys@transformshift{0.675193in}{2.534876in}%
\pgfsys@useobject{currentmarker}{}%
\end{pgfscope}%
\end{pgfscope}%
\begin{pgfscope}%
\pgfsetbuttcap%
\pgfsetroundjoin%
\definecolor{currentfill}{rgb}{0.000000,0.000000,0.000000}%
\pgfsetfillcolor{currentfill}%
\pgfsetlinewidth{0.602250pt}%
\definecolor{currentstroke}{rgb}{0.000000,0.000000,0.000000}%
\pgfsetstrokecolor{currentstroke}%
\pgfsetdash{}{0pt}%
\pgfsys@defobject{currentmarker}{\pgfqpoint{-0.027778in}{0.000000in}}{\pgfqpoint{0.000000in}{0.000000in}}{%
\pgfpathmoveto{\pgfqpoint{0.000000in}{0.000000in}}%
\pgfpathlineto{\pgfqpoint{-0.027778in}{0.000000in}}%
\pgfusepath{stroke,fill}%
}%
\begin{pgfscope}%
\pgfsys@transformshift{0.675193in}{2.755674in}%
\pgfsys@useobject{currentmarker}{}%
\end{pgfscope}%
\end{pgfscope}%
\begin{pgfscope}%
\pgfsetbuttcap%
\pgfsetroundjoin%
\definecolor{currentfill}{rgb}{0.000000,0.000000,0.000000}%
\pgfsetfillcolor{currentfill}%
\pgfsetlinewidth{0.602250pt}%
\definecolor{currentstroke}{rgb}{0.000000,0.000000,0.000000}%
\pgfsetstrokecolor{currentstroke}%
\pgfsetdash{}{0pt}%
\pgfsys@defobject{currentmarker}{\pgfqpoint{-0.027778in}{0.000000in}}{\pgfqpoint{0.000000in}{0.000000in}}{%
\pgfpathmoveto{\pgfqpoint{0.000000in}{0.000000in}}%
\pgfpathlineto{\pgfqpoint{-0.027778in}{0.000000in}}%
\pgfusepath{stroke,fill}%
}%
\begin{pgfscope}%
\pgfsys@transformshift{0.675193in}{2.912333in}%
\pgfsys@useobject{currentmarker}{}%
\end{pgfscope}%
\end{pgfscope}%
\begin{pgfscope}%
\pgfsetbuttcap%
\pgfsetroundjoin%
\definecolor{currentfill}{rgb}{0.000000,0.000000,0.000000}%
\pgfsetfillcolor{currentfill}%
\pgfsetlinewidth{0.602250pt}%
\definecolor{currentstroke}{rgb}{0.000000,0.000000,0.000000}%
\pgfsetstrokecolor{currentstroke}%
\pgfsetdash{}{0pt}%
\pgfsys@defobject{currentmarker}{\pgfqpoint{-0.027778in}{0.000000in}}{\pgfqpoint{0.000000in}{0.000000in}}{%
\pgfpathmoveto{\pgfqpoint{0.000000in}{0.000000in}}%
\pgfpathlineto{\pgfqpoint{-0.027778in}{0.000000in}}%
\pgfusepath{stroke,fill}%
}%
\begin{pgfscope}%
\pgfsys@transformshift{0.675193in}{3.033847in}%
\pgfsys@useobject{currentmarker}{}%
\end{pgfscope}%
\end{pgfscope}%
\begin{pgfscope}%
\pgfsetbuttcap%
\pgfsetroundjoin%
\definecolor{currentfill}{rgb}{0.000000,0.000000,0.000000}%
\pgfsetfillcolor{currentfill}%
\pgfsetlinewidth{0.602250pt}%
\definecolor{currentstroke}{rgb}{0.000000,0.000000,0.000000}%
\pgfsetstrokecolor{currentstroke}%
\pgfsetdash{}{0pt}%
\pgfsys@defobject{currentmarker}{\pgfqpoint{-0.027778in}{0.000000in}}{\pgfqpoint{0.000000in}{0.000000in}}{%
\pgfpathmoveto{\pgfqpoint{0.000000in}{0.000000in}}%
\pgfpathlineto{\pgfqpoint{-0.027778in}{0.000000in}}%
\pgfusepath{stroke,fill}%
}%
\begin{pgfscope}%
\pgfsys@transformshift{0.675193in}{3.133131in}%
\pgfsys@useobject{currentmarker}{}%
\end{pgfscope}%
\end{pgfscope}%
\begin{pgfscope}%
\pgfsetbuttcap%
\pgfsetroundjoin%
\definecolor{currentfill}{rgb}{0.000000,0.000000,0.000000}%
\pgfsetfillcolor{currentfill}%
\pgfsetlinewidth{0.602250pt}%
\definecolor{currentstroke}{rgb}{0.000000,0.000000,0.000000}%
\pgfsetstrokecolor{currentstroke}%
\pgfsetdash{}{0pt}%
\pgfsys@defobject{currentmarker}{\pgfqpoint{-0.027778in}{0.000000in}}{\pgfqpoint{0.000000in}{0.000000in}}{%
\pgfpathmoveto{\pgfqpoint{0.000000in}{0.000000in}}%
\pgfpathlineto{\pgfqpoint{-0.027778in}{0.000000in}}%
\pgfusepath{stroke,fill}%
}%
\begin{pgfscope}%
\pgfsys@transformshift{0.675193in}{3.217074in}%
\pgfsys@useobject{currentmarker}{}%
\end{pgfscope}%
\end{pgfscope}%
\begin{pgfscope}%
\pgfsetbuttcap%
\pgfsetroundjoin%
\definecolor{currentfill}{rgb}{0.000000,0.000000,0.000000}%
\pgfsetfillcolor{currentfill}%
\pgfsetlinewidth{0.602250pt}%
\definecolor{currentstroke}{rgb}{0.000000,0.000000,0.000000}%
\pgfsetstrokecolor{currentstroke}%
\pgfsetdash{}{0pt}%
\pgfsys@defobject{currentmarker}{\pgfqpoint{-0.027778in}{0.000000in}}{\pgfqpoint{0.000000in}{0.000000in}}{%
\pgfpathmoveto{\pgfqpoint{0.000000in}{0.000000in}}%
\pgfpathlineto{\pgfqpoint{-0.027778in}{0.000000in}}%
\pgfusepath{stroke,fill}%
}%
\begin{pgfscope}%
\pgfsys@transformshift{0.675193in}{3.289789in}%
\pgfsys@useobject{currentmarker}{}%
\end{pgfscope}%
\end{pgfscope}%
\begin{pgfscope}%
\pgfsetbuttcap%
\pgfsetroundjoin%
\definecolor{currentfill}{rgb}{0.000000,0.000000,0.000000}%
\pgfsetfillcolor{currentfill}%
\pgfsetlinewidth{0.602250pt}%
\definecolor{currentstroke}{rgb}{0.000000,0.000000,0.000000}%
\pgfsetstrokecolor{currentstroke}%
\pgfsetdash{}{0pt}%
\pgfsys@defobject{currentmarker}{\pgfqpoint{-0.027778in}{0.000000in}}{\pgfqpoint{0.000000in}{0.000000in}}{%
\pgfpathmoveto{\pgfqpoint{0.000000in}{0.000000in}}%
\pgfpathlineto{\pgfqpoint{-0.027778in}{0.000000in}}%
\pgfusepath{stroke,fill}%
}%
\begin{pgfscope}%
\pgfsys@transformshift{0.675193in}{3.353929in}%
\pgfsys@useobject{currentmarker}{}%
\end{pgfscope}%
\end{pgfscope}%
\begin{pgfscope}%
\definecolor{textcolor}{rgb}{0.000000,0.000000,0.000000}%
\pgfsetstrokecolor{textcolor}%
\pgfsetfillcolor{textcolor}%
\pgftext[x=0.234413in,y=2.036079in,,bottom,rotate=90.000000]{\color{textcolor}\rmfamily\fontsize{10.000000}{12.000000}\selectfont Dimensionless force}%
\end{pgfscope}%
\begin{pgfscope}%
\pgfpathrectangle{\pgfqpoint{0.675193in}{0.526079in}}{\pgfqpoint{4.650000in}{3.020000in}}%
\pgfusepath{clip}%
\pgfsetrectcap%
\pgfsetroundjoin%
\pgfsetlinewidth{1.505625pt}%
\definecolor{currentstroke}{rgb}{1.000000,0.000000,0.000000}%
\pgfsetstrokecolor{currentstroke}%
\pgfsetstrokeopacity{0.750000}%
\pgfsetdash{}{0pt}%
\pgfpathmoveto{\pgfqpoint{0.991862in}{3.368536in}}%
\pgfpathlineto{\pgfqpoint{1.017859in}{3.371958in}}%
\pgfpathlineto{\pgfqpoint{1.042736in}{3.375299in}}%
\pgfpathlineto{\pgfqpoint{1.066497in}{3.378100in}}%
\pgfusepath{stroke}%
\end{pgfscope}%
\begin{pgfscope}%
\pgfpathrectangle{\pgfqpoint{0.675193in}{0.526079in}}{\pgfqpoint{4.650000in}{3.020000in}}%
\pgfusepath{clip}%
\pgfsetrectcap%
\pgfsetroundjoin%
\pgfsetlinewidth{1.505625pt}%
\definecolor{currentstroke}{rgb}{0.000000,0.000000,1.000000}%
\pgfsetstrokecolor{currentstroke}%
\pgfsetstrokeopacity{0.750000}%
\pgfsetdash{}{0pt}%
\pgfpathmoveto{\pgfqpoint{0.991862in}{1.209667in}}%
\pgfpathlineto{\pgfqpoint{1.017859in}{1.186125in}}%
\pgfpathlineto{\pgfqpoint{1.042736in}{1.161377in}}%
\pgfpathlineto{\pgfqpoint{1.066497in}{1.134828in}}%
\pgfusepath{stroke}%
\end{pgfscope}%
\begin{pgfscope}%
\pgfpathrectangle{\pgfqpoint{0.675193in}{0.526079in}}{\pgfqpoint{4.650000in}{3.020000in}}%
\pgfusepath{clip}%
\pgfsetrectcap%
\pgfsetroundjoin%
\pgfsetlinewidth{1.505625pt}%
\definecolor{currentstroke}{rgb}{0.000000,0.750000,0.750000}%
\pgfsetstrokecolor{currentstroke}%
\pgfsetstrokeopacity{0.750000}%
\pgfsetdash{}{0pt}%
\pgfpathmoveto{\pgfqpoint{0.991862in}{1.867466in}}%
\pgfpathlineto{\pgfqpoint{1.017859in}{1.807053in}}%
\pgfpathlineto{\pgfqpoint{1.042736in}{1.753572in}}%
\pgfpathlineto{\pgfqpoint{1.066497in}{1.705539in}}%
\pgfusepath{stroke}%
\end{pgfscope}%
\begin{pgfscope}%
\pgfpathrectangle{\pgfqpoint{0.675193in}{0.526079in}}{\pgfqpoint{4.650000in}{3.020000in}}%
\pgfusepath{clip}%
\pgfsetrectcap%
\pgfsetroundjoin%
\pgfsetlinewidth{1.505625pt}%
\definecolor{currentstroke}{rgb}{1.000000,0.000000,0.000000}%
\pgfsetstrokecolor{currentstroke}%
\pgfsetstrokeopacity{0.750000}%
\pgfsetdash{}{0pt}%
\pgfpathmoveto{\pgfqpoint{0.965930in}{3.361237in}}%
\pgfpathlineto{\pgfqpoint{0.995533in}{3.365490in}}%
\pgfpathlineto{\pgfqpoint{1.023927in}{3.369587in}}%
\pgfpathlineto{\pgfqpoint{1.051125in}{3.372966in}}%
\pgfpathlineto{\pgfqpoint{1.077137in}{3.375820in}}%
\pgfpathlineto{\pgfqpoint{1.101975in}{3.378274in}}%
\pgfpathlineto{\pgfqpoint{1.125650in}{3.380419in}}%
\pgfpathlineto{\pgfqpoint{1.148175in}{3.382321in}}%
\pgfusepath{stroke}%
\end{pgfscope}%
\begin{pgfscope}%
\pgfpathrectangle{\pgfqpoint{0.675193in}{0.526079in}}{\pgfqpoint{4.650000in}{3.020000in}}%
\pgfusepath{clip}%
\pgfsetrectcap%
\pgfsetroundjoin%
\pgfsetlinewidth{1.505625pt}%
\definecolor{currentstroke}{rgb}{0.000000,0.000000,1.000000}%
\pgfsetstrokecolor{currentstroke}%
\pgfsetstrokeopacity{0.750000}%
\pgfsetdash{}{0pt}%
\pgfpathmoveto{\pgfqpoint{0.965930in}{1.466956in}}%
\pgfpathlineto{\pgfqpoint{0.995533in}{1.448329in}}%
\pgfpathlineto{\pgfqpoint{1.023927in}{1.428733in}}%
\pgfpathlineto{\pgfqpoint{1.051125in}{1.407502in}}%
\pgfpathlineto{\pgfqpoint{1.077137in}{1.384725in}}%
\pgfpathlineto{\pgfqpoint{1.101975in}{1.360424in}}%
\pgfpathlineto{\pgfqpoint{1.125650in}{1.334579in}}%
\pgfpathlineto{\pgfqpoint{1.148175in}{1.307137in}}%
\pgfusepath{stroke}%
\end{pgfscope}%
\begin{pgfscope}%
\pgfpathrectangle{\pgfqpoint{0.675193in}{0.526079in}}{\pgfqpoint{4.650000in}{3.020000in}}%
\pgfusepath{clip}%
\pgfsetrectcap%
\pgfsetroundjoin%
\pgfsetlinewidth{1.505625pt}%
\definecolor{currentstroke}{rgb}{0.000000,0.750000,0.750000}%
\pgfsetstrokecolor{currentstroke}%
\pgfsetstrokeopacity{0.750000}%
\pgfsetdash{}{0pt}%
\pgfpathmoveto{\pgfqpoint{0.965930in}{1.885088in}}%
\pgfpathlineto{\pgfqpoint{0.995533in}{1.812143in}}%
\pgfpathlineto{\pgfqpoint{1.023927in}{1.748092in}}%
\pgfpathlineto{\pgfqpoint{1.051125in}{1.690887in}}%
\pgfpathlineto{\pgfqpoint{1.077137in}{1.639571in}}%
\pgfpathlineto{\pgfqpoint{1.101975in}{1.593370in}}%
\pgfpathlineto{\pgfqpoint{1.125650in}{1.551648in}}%
\pgfpathlineto{\pgfqpoint{1.148175in}{1.513884in}}%
\pgfusepath{stroke}%
\end{pgfscope}%
\begin{pgfscope}%
\pgfpathrectangle{\pgfqpoint{0.675193in}{0.526079in}}{\pgfqpoint{4.650000in}{3.020000in}}%
\pgfusepath{clip}%
\pgfsetrectcap%
\pgfsetroundjoin%
\pgfsetlinewidth{1.505625pt}%
\definecolor{currentstroke}{rgb}{1.000000,0.000000,0.000000}%
\pgfsetstrokecolor{currentstroke}%
\pgfsetstrokeopacity{0.750000}%
\pgfsetdash{}{0pt}%
\pgfpathmoveto{\pgfqpoint{1.041681in}{3.359676in}}%
\pgfpathlineto{\pgfqpoint{1.077630in}{3.364661in}}%
\pgfpathlineto{\pgfqpoint{1.112004in}{3.369467in}}%
\pgfpathlineto{\pgfqpoint{1.144820in}{3.373331in}}%
\pgfpathlineto{\pgfqpoint{1.176094in}{3.376501in}}%
\pgfpathlineto{\pgfqpoint{1.205844in}{3.379147in}}%
\pgfpathlineto{\pgfqpoint{1.234084in}{3.381388in}}%
\pgfusepath{stroke}%
\end{pgfscope}%
\begin{pgfscope}%
\pgfpathrectangle{\pgfqpoint{0.675193in}{0.526079in}}{\pgfqpoint{4.650000in}{3.020000in}}%
\pgfusepath{clip}%
\pgfsetrectcap%
\pgfsetroundjoin%
\pgfsetlinewidth{1.505625pt}%
\definecolor{currentstroke}{rgb}{0.000000,0.000000,1.000000}%
\pgfsetstrokecolor{currentstroke}%
\pgfsetstrokeopacity{0.750000}%
\pgfsetdash{}{0pt}%
\pgfpathmoveto{\pgfqpoint{1.041681in}{0.978295in}}%
\pgfpathlineto{\pgfqpoint{1.077630in}{0.957448in}}%
\pgfpathlineto{\pgfqpoint{1.112004in}{0.935486in}}%
\pgfpathlineto{\pgfqpoint{1.144820in}{0.911499in}}%
\pgfpathlineto{\pgfqpoint{1.176094in}{0.885598in}}%
\pgfpathlineto{\pgfqpoint{1.205844in}{0.857811in}}%
\pgfpathlineto{\pgfqpoint{1.234084in}{0.828109in}}%
\pgfusepath{stroke}%
\end{pgfscope}%
\begin{pgfscope}%
\pgfpathrectangle{\pgfqpoint{0.675193in}{0.526079in}}{\pgfqpoint{4.650000in}{3.020000in}}%
\pgfusepath{clip}%
\pgfsetrectcap%
\pgfsetroundjoin%
\pgfsetlinewidth{1.505625pt}%
\definecolor{currentstroke}{rgb}{0.000000,0.750000,0.750000}%
\pgfsetstrokecolor{currentstroke}%
\pgfsetstrokeopacity{0.750000}%
\pgfsetdash{}{0pt}%
\pgfpathmoveto{\pgfqpoint{1.041681in}{2.037695in}}%
\pgfpathlineto{\pgfqpoint{1.077630in}{1.966579in}}%
\pgfpathlineto{\pgfqpoint{1.112004in}{1.904641in}}%
\pgfpathlineto{\pgfqpoint{1.144820in}{1.849532in}}%
\pgfpathlineto{\pgfqpoint{1.176094in}{1.800289in}}%
\pgfpathlineto{\pgfqpoint{1.205844in}{1.756128in}}%
\pgfpathlineto{\pgfqpoint{1.234084in}{1.716412in}}%
\pgfusepath{stroke}%
\end{pgfscope}%
\begin{pgfscope}%
\pgfpathrectangle{\pgfqpoint{0.675193in}{0.526079in}}{\pgfqpoint{4.650000in}{3.020000in}}%
\pgfusepath{clip}%
\pgfsetrectcap%
\pgfsetroundjoin%
\pgfsetlinewidth{1.505625pt}%
\definecolor{currentstroke}{rgb}{1.000000,0.000000,0.000000}%
\pgfsetstrokecolor{currentstroke}%
\pgfsetstrokeopacity{0.750000}%
\pgfsetdash{}{0pt}%
\pgfpathmoveto{\pgfqpoint{0.886557in}{3.331498in}}%
\pgfpathlineto{\pgfqpoint{0.917018in}{3.338080in}}%
\pgfpathlineto{\pgfqpoint{0.947047in}{3.344108in}}%
\pgfpathlineto{\pgfqpoint{0.976642in}{3.348797in}}%
\pgfpathlineto{\pgfqpoint{1.005800in}{3.352551in}}%
\pgfpathlineto{\pgfqpoint{1.034518in}{3.355628in}}%
\pgfpathlineto{\pgfqpoint{1.062793in}{3.358199in}}%
\pgfpathlineto{\pgfqpoint{1.090622in}{3.360390in}}%
\pgfpathlineto{\pgfqpoint{1.118001in}{3.362289in}}%
\pgfpathlineto{\pgfqpoint{1.144929in}{3.363957in}}%
\pgfpathlineto{\pgfqpoint{1.171401in}{3.365442in}}%
\pgfpathlineto{\pgfqpoint{1.197416in}{3.366780in}}%
\pgfpathlineto{\pgfqpoint{1.222969in}{3.367998in}}%
\pgfpathlineto{\pgfqpoint{1.248048in}{3.369118in}}%
\pgfpathlineto{\pgfqpoint{1.272654in}{3.370159in}}%
\pgfpathlineto{\pgfqpoint{1.296784in}{3.371131in}}%
\pgfpathlineto{\pgfqpoint{1.320436in}{3.372047in}}%
\pgfpathlineto{\pgfqpoint{1.343612in}{3.372915in}}%
\pgfpathlineto{\pgfqpoint{1.366312in}{3.373743in}}%
\pgfpathlineto{\pgfqpoint{1.388539in}{3.374537in}}%
\pgfpathlineto{\pgfqpoint{1.410295in}{3.375300in}}%
\pgfpathlineto{\pgfqpoint{1.431585in}{3.376039in}}%
\pgfpathlineto{\pgfqpoint{1.452415in}{3.376757in}}%
\pgfpathlineto{\pgfqpoint{1.472790in}{3.377455in}}%
\pgfpathlineto{\pgfqpoint{1.492716in}{3.378138in}}%
\pgfpathlineto{\pgfqpoint{1.512201in}{3.378806in}}%
\pgfpathlineto{\pgfqpoint{1.531251in}{3.379462in}}%
\pgfpathlineto{\pgfqpoint{1.549873in}{3.380107in}}%
\pgfpathlineto{\pgfqpoint{1.568071in}{3.380743in}}%
\pgfpathlineto{\pgfqpoint{1.585852in}{3.381371in}}%
\pgfpathlineto{\pgfqpoint{1.603221in}{3.381991in}}%
\pgfpathlineto{\pgfqpoint{1.620183in}{3.382605in}}%
\pgfpathlineto{\pgfqpoint{1.636740in}{3.383213in}}%
\pgfpathlineto{\pgfqpoint{1.652898in}{3.383815in}}%
\pgfpathlineto{\pgfqpoint{1.668658in}{3.384413in}}%
\pgfpathlineto{\pgfqpoint{1.684024in}{3.385008in}}%
\pgfpathlineto{\pgfqpoint{1.698999in}{3.385598in}}%
\pgfpathlineto{\pgfqpoint{1.713585in}{3.386185in}}%
\pgfpathlineto{\pgfqpoint{1.727784in}{3.386769in}}%
\pgfpathlineto{\pgfqpoint{1.741600in}{3.387350in}}%
\pgfpathlineto{\pgfqpoint{1.755033in}{3.387927in}}%
\pgfpathlineto{\pgfqpoint{1.768088in}{3.388503in}}%
\pgfpathlineto{\pgfqpoint{1.780767in}{3.389076in}}%
\pgfpathlineto{\pgfqpoint{1.793072in}{3.389645in}}%
\pgfpathlineto{\pgfqpoint{1.805007in}{3.390213in}}%
\pgfpathlineto{\pgfqpoint{1.816573in}{3.390778in}}%
\pgfpathlineto{\pgfqpoint{1.827775in}{3.391340in}}%
\pgfpathlineto{\pgfqpoint{1.838615in}{3.391900in}}%
\pgfpathlineto{\pgfqpoint{1.849095in}{3.392458in}}%
\pgfpathlineto{\pgfqpoint{1.859219in}{3.393013in}}%
\pgfpathlineto{\pgfqpoint{1.868987in}{3.393564in}}%
\pgfpathlineto{\pgfqpoint{1.878404in}{3.394113in}}%
\pgfpathlineto{\pgfqpoint{1.887470in}{3.394660in}}%
\pgfpathlineto{\pgfqpoint{1.896187in}{3.395203in}}%
\pgfpathlineto{\pgfqpoint{1.904557in}{3.395744in}}%
\pgfpathlineto{\pgfqpoint{1.912581in}{3.396281in}}%
\pgfpathlineto{\pgfqpoint{1.920262in}{3.396815in}}%
\pgfpathlineto{\pgfqpoint{1.927599in}{3.397344in}}%
\pgfpathlineto{\pgfqpoint{1.934594in}{3.397868in}}%
\pgfusepath{stroke}%
\end{pgfscope}%
\begin{pgfscope}%
\pgfpathrectangle{\pgfqpoint{0.675193in}{0.526079in}}{\pgfqpoint{4.650000in}{3.020000in}}%
\pgfusepath{clip}%
\pgfsetrectcap%
\pgfsetroundjoin%
\pgfsetlinewidth{1.505625pt}%
\definecolor{currentstroke}{rgb}{0.000000,0.000000,1.000000}%
\pgfsetstrokecolor{currentstroke}%
\pgfsetstrokeopacity{0.750000}%
\pgfsetdash{}{0pt}%
\pgfpathmoveto{\pgfqpoint{0.886557in}{1.883580in}}%
\pgfpathlineto{\pgfqpoint{0.917018in}{1.883838in}}%
\pgfpathlineto{\pgfqpoint{0.947047in}{1.883579in}}%
\pgfpathlineto{\pgfqpoint{0.976642in}{1.881975in}}%
\pgfpathlineto{\pgfqpoint{1.005800in}{1.879400in}}%
\pgfpathlineto{\pgfqpoint{1.034518in}{1.876086in}}%
\pgfpathlineto{\pgfqpoint{1.062793in}{1.872188in}}%
\pgfpathlineto{\pgfqpoint{1.090622in}{1.867814in}}%
\pgfpathlineto{\pgfqpoint{1.118001in}{1.863040in}}%
\pgfpathlineto{\pgfqpoint{1.144929in}{1.857914in}}%
\pgfpathlineto{\pgfqpoint{1.171401in}{1.852476in}}%
\pgfpathlineto{\pgfqpoint{1.197416in}{1.846752in}}%
\pgfpathlineto{\pgfqpoint{1.222969in}{1.840759in}}%
\pgfpathlineto{\pgfqpoint{1.248048in}{1.834514in}}%
\pgfpathlineto{\pgfqpoint{1.272654in}{1.828024in}}%
\pgfpathlineto{\pgfqpoint{1.296784in}{1.821296in}}%
\pgfpathlineto{\pgfqpoint{1.320436in}{1.814333in}}%
\pgfpathlineto{\pgfqpoint{1.343612in}{1.807136in}}%
\pgfpathlineto{\pgfqpoint{1.366312in}{1.799706in}}%
\pgfpathlineto{\pgfqpoint{1.388539in}{1.792043in}}%
\pgfpathlineto{\pgfqpoint{1.410295in}{1.784142in}}%
\pgfpathlineto{\pgfqpoint{1.431585in}{1.776003in}}%
\pgfpathlineto{\pgfqpoint{1.452415in}{1.767621in}}%
\pgfpathlineto{\pgfqpoint{1.472790in}{1.758991in}}%
\pgfpathlineto{\pgfqpoint{1.492716in}{1.750109in}}%
\pgfpathlineto{\pgfqpoint{1.512201in}{1.740968in}}%
\pgfpathlineto{\pgfqpoint{1.531251in}{1.731562in}}%
\pgfpathlineto{\pgfqpoint{1.549873in}{1.721885in}}%
\pgfpathlineto{\pgfqpoint{1.568071in}{1.711928in}}%
\pgfpathlineto{\pgfqpoint{1.585852in}{1.701684in}}%
\pgfpathlineto{\pgfqpoint{1.603221in}{1.691144in}}%
\pgfpathlineto{\pgfqpoint{1.620183in}{1.680299in}}%
\pgfpathlineto{\pgfqpoint{1.636740in}{1.669139in}}%
\pgfpathlineto{\pgfqpoint{1.652898in}{1.657653in}}%
\pgfpathlineto{\pgfqpoint{1.668658in}{1.645830in}}%
\pgfpathlineto{\pgfqpoint{1.684024in}{1.633660in}}%
\pgfpathlineto{\pgfqpoint{1.698999in}{1.621126in}}%
\pgfpathlineto{\pgfqpoint{1.713585in}{1.608217in}}%
\pgfpathlineto{\pgfqpoint{1.727784in}{1.594918in}}%
\pgfpathlineto{\pgfqpoint{1.741600in}{1.581213in}}%
\pgfpathlineto{\pgfqpoint{1.755033in}{1.567083in}}%
\pgfpathlineto{\pgfqpoint{1.768088in}{1.552512in}}%
\pgfpathlineto{\pgfqpoint{1.780767in}{1.537479in}}%
\pgfpathlineto{\pgfqpoint{1.793072in}{1.521963in}}%
\pgfpathlineto{\pgfqpoint{1.805007in}{1.505941in}}%
\pgfpathlineto{\pgfqpoint{1.816573in}{1.489387in}}%
\pgfpathlineto{\pgfqpoint{1.827775in}{1.472274in}}%
\pgfpathlineto{\pgfqpoint{1.838615in}{1.454573in}}%
\pgfpathlineto{\pgfqpoint{1.849095in}{1.436251in}}%
\pgfpathlineto{\pgfqpoint{1.859219in}{1.417273in}}%
\pgfpathlineto{\pgfqpoint{1.868987in}{1.397599in}}%
\pgfpathlineto{\pgfqpoint{1.878404in}{1.377187in}}%
\pgfpathlineto{\pgfqpoint{1.887470in}{1.355989in}}%
\pgfpathlineto{\pgfqpoint{1.896187in}{1.333952in}}%
\pgfpathlineto{\pgfqpoint{1.904557in}{1.311019in}}%
\pgfpathlineto{\pgfqpoint{1.912581in}{1.287123in}}%
\pgfpathlineto{\pgfqpoint{1.920262in}{1.262190in}}%
\pgfpathlineto{\pgfqpoint{1.927599in}{1.236138in}}%
\pgfpathlineto{\pgfqpoint{1.934594in}{1.208873in}}%
\pgfusepath{stroke}%
\end{pgfscope}%
\begin{pgfscope}%
\pgfpathrectangle{\pgfqpoint{0.675193in}{0.526079in}}{\pgfqpoint{4.650000in}{3.020000in}}%
\pgfusepath{clip}%
\pgfsetrectcap%
\pgfsetroundjoin%
\pgfsetlinewidth{1.505625pt}%
\definecolor{currentstroke}{rgb}{0.000000,0.750000,0.750000}%
\pgfsetstrokecolor{currentstroke}%
\pgfsetstrokeopacity{0.750000}%
\pgfsetdash{}{0pt}%
\pgfpathmoveto{\pgfqpoint{0.886557in}{2.017788in}}%
\pgfpathlineto{\pgfqpoint{0.917018in}{1.919406in}}%
\pgfpathlineto{\pgfqpoint{0.947047in}{1.832835in}}%
\pgfpathlineto{\pgfqpoint{0.976642in}{1.754977in}}%
\pgfpathlineto{\pgfqpoint{1.005800in}{1.684516in}}%
\pgfpathlineto{\pgfqpoint{1.034518in}{1.620389in}}%
\pgfpathlineto{\pgfqpoint{1.062793in}{1.561739in}}%
\pgfpathlineto{\pgfqpoint{1.090622in}{1.507868in}}%
\pgfpathlineto{\pgfqpoint{1.118001in}{1.458199in}}%
\pgfpathlineto{\pgfqpoint{1.144929in}{1.412245in}}%
\pgfpathlineto{\pgfqpoint{1.171401in}{1.369602in}}%
\pgfpathlineto{\pgfqpoint{1.197416in}{1.329922in}}%
\pgfpathlineto{\pgfqpoint{1.222969in}{1.292908in}}%
\pgfpathlineto{\pgfqpoint{1.248048in}{1.258306in}}%
\pgfpathlineto{\pgfqpoint{1.272654in}{1.225894in}}%
\pgfpathlineto{\pgfqpoint{1.296784in}{1.195476in}}%
\pgfpathlineto{\pgfqpoint{1.320436in}{1.166883in}}%
\pgfpathlineto{\pgfqpoint{1.343612in}{1.139964in}}%
\pgfpathlineto{\pgfqpoint{1.366312in}{1.114586in}}%
\pgfpathlineto{\pgfqpoint{1.388539in}{1.090630in}}%
\pgfpathlineto{\pgfqpoint{1.410295in}{1.067989in}}%
\pgfpathlineto{\pgfqpoint{1.431585in}{1.046569in}}%
\pgfpathlineto{\pgfqpoint{1.452415in}{1.026285in}}%
\pgfpathlineto{\pgfqpoint{1.472790in}{1.007057in}}%
\pgfpathlineto{\pgfqpoint{1.492716in}{0.988817in}}%
\pgfpathlineto{\pgfqpoint{1.512201in}{0.971500in}}%
\pgfpathlineto{\pgfqpoint{1.531251in}{0.955048in}}%
\pgfpathlineto{\pgfqpoint{1.549873in}{0.939407in}}%
\pgfpathlineto{\pgfqpoint{1.568071in}{0.924531in}}%
\pgfpathlineto{\pgfqpoint{1.585852in}{0.910373in}}%
\pgfpathlineto{\pgfqpoint{1.603221in}{0.896893in}}%
\pgfpathlineto{\pgfqpoint{1.620183in}{0.884053in}}%
\pgfpathlineto{\pgfqpoint{1.636740in}{0.871818in}}%
\pgfpathlineto{\pgfqpoint{1.652898in}{0.860157in}}%
\pgfpathlineto{\pgfqpoint{1.668658in}{0.849040in}}%
\pgfpathlineto{\pgfqpoint{1.684024in}{0.838439in}}%
\pgfpathlineto{\pgfqpoint{1.698999in}{0.828328in}}%
\pgfpathlineto{\pgfqpoint{1.713585in}{0.818683in}}%
\pgfpathlineto{\pgfqpoint{1.727784in}{0.809484in}}%
\pgfpathlineto{\pgfqpoint{1.741600in}{0.800708in}}%
\pgfpathlineto{\pgfqpoint{1.755033in}{0.792337in}}%
\pgfpathlineto{\pgfqpoint{1.768088in}{0.784353in}}%
\pgfpathlineto{\pgfqpoint{1.780767in}{0.776739in}}%
\pgfpathlineto{\pgfqpoint{1.793072in}{0.769478in}}%
\pgfpathlineto{\pgfqpoint{1.805007in}{0.762558in}}%
\pgfpathlineto{\pgfqpoint{1.816573in}{0.755964in}}%
\pgfpathlineto{\pgfqpoint{1.827775in}{0.749683in}}%
\pgfpathlineto{\pgfqpoint{1.838615in}{0.743702in}}%
\pgfpathlineto{\pgfqpoint{1.849095in}{0.738012in}}%
\pgfpathlineto{\pgfqpoint{1.859219in}{0.732600in}}%
\pgfpathlineto{\pgfqpoint{1.868987in}{0.727457in}}%
\pgfpathlineto{\pgfqpoint{1.878404in}{0.722574in}}%
\pgfpathlineto{\pgfqpoint{1.887470in}{0.717942in}}%
\pgfpathlineto{\pgfqpoint{1.896187in}{0.713552in}}%
\pgfpathlineto{\pgfqpoint{1.904557in}{0.709398in}}%
\pgfpathlineto{\pgfqpoint{1.912581in}{0.705470in}}%
\pgfpathlineto{\pgfqpoint{1.920262in}{0.701763in}}%
\pgfpathlineto{\pgfqpoint{1.927599in}{0.698270in}}%
\pgfpathlineto{\pgfqpoint{1.934594in}{0.694983in}}%
\pgfusepath{stroke}%
\end{pgfscope}%
\begin{pgfscope}%
\pgfpathrectangle{\pgfqpoint{0.675193in}{0.526079in}}{\pgfqpoint{4.650000in}{3.020000in}}%
\pgfusepath{clip}%
\pgfsetrectcap%
\pgfsetroundjoin%
\pgfsetlinewidth{1.505625pt}%
\definecolor{currentstroke}{rgb}{1.000000,0.000000,0.000000}%
\pgfsetstrokecolor{currentstroke}%
\pgfsetstrokeopacity{0.750000}%
\pgfsetdash{}{0pt}%
\pgfpathmoveto{\pgfqpoint{0.954489in}{3.336626in}}%
\pgfpathlineto{\pgfqpoint{0.988306in}{3.343309in}}%
\pgfpathlineto{\pgfqpoint{1.021505in}{3.349518in}}%
\pgfpathlineto{\pgfqpoint{1.054091in}{3.354442in}}%
\pgfpathlineto{\pgfqpoint{1.086070in}{3.358437in}}%
\pgfpathlineto{\pgfqpoint{1.117448in}{3.361742in}}%
\pgfpathlineto{\pgfqpoint{1.148230in}{3.364519in}}%
\pgfpathlineto{\pgfqpoint{1.178420in}{3.366889in}}%
\pgfpathlineto{\pgfqpoint{1.208025in}{3.368941in}}%
\pgfpathlineto{\pgfqpoint{1.237049in}{3.370738in}}%
\pgfpathlineto{\pgfqpoint{1.265499in}{3.372329in}}%
\pgfpathlineto{\pgfqpoint{1.293379in}{3.373752in}}%
\pgfpathlineto{\pgfqpoint{1.320695in}{3.375036in}}%
\pgfpathlineto{\pgfqpoint{1.347455in}{3.376206in}}%
\pgfpathlineto{\pgfqpoint{1.373663in}{3.377279in}}%
\pgfpathlineto{\pgfqpoint{1.399324in}{3.378272in}}%
\pgfpathlineto{\pgfqpoint{1.424442in}{3.379195in}}%
\pgfpathlineto{\pgfqpoint{1.449023in}{3.380059in}}%
\pgfpathlineto{\pgfqpoint{1.473071in}{3.380873in}}%
\pgfpathlineto{\pgfqpoint{1.496589in}{3.381643in}}%
\pgfpathlineto{\pgfqpoint{1.519582in}{3.382374in}}%
\pgfpathlineto{\pgfqpoint{1.542053in}{3.383073in}}%
\pgfpathlineto{\pgfqpoint{1.564006in}{3.383743in}}%
\pgfpathlineto{\pgfqpoint{1.585446in}{3.384387in}}%
\pgfpathlineto{\pgfqpoint{1.606377in}{3.385010in}}%
\pgfpathlineto{\pgfqpoint{1.626802in}{3.385612in}}%
\pgfpathlineto{\pgfqpoint{1.646726in}{3.386196in}}%
\pgfpathlineto{\pgfqpoint{1.666156in}{3.386765in}}%
\pgfpathlineto{\pgfqpoint{1.685096in}{3.387319in}}%
\pgfpathlineto{\pgfqpoint{1.703553in}{3.387861in}}%
\pgfpathlineto{\pgfqpoint{1.721531in}{3.388391in}}%
\pgfpathlineto{\pgfqpoint{1.739037in}{3.388911in}}%
\pgfpathlineto{\pgfqpoint{1.756077in}{3.389421in}}%
\pgfpathlineto{\pgfqpoint{1.772657in}{3.389923in}}%
\pgfpathlineto{\pgfqpoint{1.788784in}{3.390417in}}%
\pgfpathlineto{\pgfqpoint{1.804462in}{3.390903in}}%
\pgfpathlineto{\pgfqpoint{1.819697in}{3.391383in}}%
\pgfpathlineto{\pgfqpoint{1.834494in}{3.391856in}}%
\pgfpathlineto{\pgfqpoint{1.848858in}{3.392323in}}%
\pgfpathlineto{\pgfqpoint{1.862793in}{3.392785in}}%
\pgfpathlineto{\pgfqpoint{1.876303in}{3.393241in}}%
\pgfpathlineto{\pgfqpoint{1.889391in}{3.393692in}}%
\pgfpathlineto{\pgfqpoint{1.902061in}{3.394138in}}%
\pgfpathlineto{\pgfqpoint{1.914315in}{3.394579in}}%
\pgfpathlineto{\pgfqpoint{1.926156in}{3.395015in}}%
\pgfpathlineto{\pgfqpoint{1.937586in}{3.395446in}}%
\pgfpathlineto{\pgfqpoint{1.948607in}{3.395874in}}%
\pgfpathlineto{\pgfqpoint{1.959221in}{3.396296in}}%
\pgfpathlineto{\pgfqpoint{1.969429in}{3.396714in}}%
\pgfpathlineto{\pgfqpoint{1.979233in}{3.397128in}}%
\pgfpathlineto{\pgfqpoint{1.988636in}{3.397537in}}%
\pgfpathlineto{\pgfqpoint{1.997640in}{3.397943in}}%
\pgfpathlineto{\pgfqpoint{2.006245in}{3.398343in}}%
\pgfpathlineto{\pgfqpoint{2.014455in}{3.398738in}}%
\pgfusepath{stroke}%
\end{pgfscope}%
\begin{pgfscope}%
\pgfpathrectangle{\pgfqpoint{0.675193in}{0.526079in}}{\pgfqpoint{4.650000in}{3.020000in}}%
\pgfusepath{clip}%
\pgfsetrectcap%
\pgfsetroundjoin%
\pgfsetlinewidth{1.505625pt}%
\definecolor{currentstroke}{rgb}{0.000000,0.000000,1.000000}%
\pgfsetstrokecolor{currentstroke}%
\pgfsetstrokeopacity{0.750000}%
\pgfsetdash{}{0pt}%
\pgfpathmoveto{\pgfqpoint{0.954489in}{1.659320in}}%
\pgfpathlineto{\pgfqpoint{0.988306in}{1.659068in}}%
\pgfpathlineto{\pgfqpoint{1.021505in}{1.658378in}}%
\pgfpathlineto{\pgfqpoint{1.054091in}{1.656393in}}%
\pgfpathlineto{\pgfqpoint{1.086070in}{1.653435in}}%
\pgfpathlineto{\pgfqpoint{1.117448in}{1.649713in}}%
\pgfpathlineto{\pgfqpoint{1.148230in}{1.645369in}}%
\pgfpathlineto{\pgfqpoint{1.178420in}{1.640506in}}%
\pgfpathlineto{\pgfqpoint{1.208025in}{1.635196in}}%
\pgfpathlineto{\pgfqpoint{1.237049in}{1.629489in}}%
\pgfpathlineto{\pgfqpoint{1.265499in}{1.623421in}}%
\pgfpathlineto{\pgfqpoint{1.293379in}{1.617019in}}%
\pgfpathlineto{\pgfqpoint{1.320695in}{1.610302in}}%
\pgfpathlineto{\pgfqpoint{1.347455in}{1.603283in}}%
\pgfpathlineto{\pgfqpoint{1.373663in}{1.595972in}}%
\pgfpathlineto{\pgfqpoint{1.399324in}{1.588374in}}%
\pgfpathlineto{\pgfqpoint{1.424442in}{1.580491in}}%
\pgfpathlineto{\pgfqpoint{1.449023in}{1.572326in}}%
\pgfpathlineto{\pgfqpoint{1.473071in}{1.563877in}}%
\pgfpathlineto{\pgfqpoint{1.496589in}{1.555143in}}%
\pgfpathlineto{\pgfqpoint{1.519582in}{1.546118in}}%
\pgfpathlineto{\pgfqpoint{1.542053in}{1.536802in}}%
\pgfpathlineto{\pgfqpoint{1.564006in}{1.527186in}}%
\pgfpathlineto{\pgfqpoint{1.585446in}{1.517265in}}%
\pgfpathlineto{\pgfqpoint{1.606377in}{1.507033in}}%
\pgfpathlineto{\pgfqpoint{1.626802in}{1.496480in}}%
\pgfpathlineto{\pgfqpoint{1.646726in}{1.485599in}}%
\pgfpathlineto{\pgfqpoint{1.666156in}{1.474379in}}%
\pgfpathlineto{\pgfqpoint{1.685096in}{1.462812in}}%
\pgfpathlineto{\pgfqpoint{1.703553in}{1.450885in}}%
\pgfpathlineto{\pgfqpoint{1.721531in}{1.438586in}}%
\pgfpathlineto{\pgfqpoint{1.739037in}{1.425903in}}%
\pgfpathlineto{\pgfqpoint{1.756077in}{1.412822in}}%
\pgfpathlineto{\pgfqpoint{1.772657in}{1.399329in}}%
\pgfpathlineto{\pgfqpoint{1.788784in}{1.385406in}}%
\pgfpathlineto{\pgfqpoint{1.804462in}{1.371037in}}%
\pgfpathlineto{\pgfqpoint{1.819697in}{1.356202in}}%
\pgfpathlineto{\pgfqpoint{1.834494in}{1.340883in}}%
\pgfpathlineto{\pgfqpoint{1.848858in}{1.325055in}}%
\pgfpathlineto{\pgfqpoint{1.862793in}{1.308697in}}%
\pgfpathlineto{\pgfqpoint{1.876303in}{1.291781in}}%
\pgfpathlineto{\pgfqpoint{1.889391in}{1.274281in}}%
\pgfpathlineto{\pgfqpoint{1.902061in}{1.256165in}}%
\pgfpathlineto{\pgfqpoint{1.914315in}{1.237401in}}%
\pgfpathlineto{\pgfqpoint{1.926156in}{1.217950in}}%
\pgfpathlineto{\pgfqpoint{1.937586in}{1.197775in}}%
\pgfpathlineto{\pgfqpoint{1.948607in}{1.176829in}}%
\pgfpathlineto{\pgfqpoint{1.959221in}{1.155064in}}%
\pgfpathlineto{\pgfqpoint{1.969429in}{1.132426in}}%
\pgfpathlineto{\pgfqpoint{1.979233in}{1.108854in}}%
\pgfpathlineto{\pgfqpoint{1.988636in}{1.084280in}}%
\pgfpathlineto{\pgfqpoint{1.997640in}{1.058628in}}%
\pgfpathlineto{\pgfqpoint{2.006245in}{1.031812in}}%
\pgfpathlineto{\pgfqpoint{2.014455in}{1.003735in}}%
\pgfusepath{stroke}%
\end{pgfscope}%
\begin{pgfscope}%
\pgfpathrectangle{\pgfqpoint{0.675193in}{0.526079in}}{\pgfqpoint{4.650000in}{3.020000in}}%
\pgfusepath{clip}%
\pgfsetrectcap%
\pgfsetroundjoin%
\pgfsetlinewidth{1.505625pt}%
\definecolor{currentstroke}{rgb}{0.000000,0.750000,0.750000}%
\pgfsetstrokecolor{currentstroke}%
\pgfsetstrokeopacity{0.750000}%
\pgfsetdash{}{0pt}%
\pgfpathmoveto{\pgfqpoint{0.954489in}{2.097449in}}%
\pgfpathlineto{\pgfqpoint{0.988306in}{2.012521in}}%
\pgfpathlineto{\pgfqpoint{1.021505in}{1.937287in}}%
\pgfpathlineto{\pgfqpoint{1.054091in}{1.869173in}}%
\pgfpathlineto{\pgfqpoint{1.086070in}{1.807193in}}%
\pgfpathlineto{\pgfqpoint{1.117448in}{1.750533in}}%
\pgfpathlineto{\pgfqpoint{1.148230in}{1.698523in}}%
\pgfpathlineto{\pgfqpoint{1.178420in}{1.650608in}}%
\pgfpathlineto{\pgfqpoint{1.208025in}{1.606322in}}%
\pgfpathlineto{\pgfqpoint{1.237049in}{1.565270in}}%
\pgfpathlineto{\pgfqpoint{1.265499in}{1.527114in}}%
\pgfpathlineto{\pgfqpoint{1.293379in}{1.491564in}}%
\pgfpathlineto{\pgfqpoint{1.320695in}{1.458371in}}%
\pgfpathlineto{\pgfqpoint{1.347455in}{1.427319in}}%
\pgfpathlineto{\pgfqpoint{1.373663in}{1.398215in}}%
\pgfpathlineto{\pgfqpoint{1.399324in}{1.370894in}}%
\pgfpathlineto{\pgfqpoint{1.424442in}{1.345207in}}%
\pgfpathlineto{\pgfqpoint{1.449023in}{1.321022in}}%
\pgfpathlineto{\pgfqpoint{1.473071in}{1.298224in}}%
\pgfpathlineto{\pgfqpoint{1.496589in}{1.276708in}}%
\pgfpathlineto{\pgfqpoint{1.519582in}{1.256379in}}%
\pgfpathlineto{\pgfqpoint{1.542053in}{1.237155in}}%
\pgfpathlineto{\pgfqpoint{1.564006in}{1.218958in}}%
\pgfpathlineto{\pgfqpoint{1.585446in}{1.201719in}}%
\pgfpathlineto{\pgfqpoint{1.606377in}{1.185376in}}%
\pgfpathlineto{\pgfqpoint{1.626802in}{1.169871in}}%
\pgfpathlineto{\pgfqpoint{1.646726in}{1.155153in}}%
\pgfpathlineto{\pgfqpoint{1.666156in}{1.141173in}}%
\pgfpathlineto{\pgfqpoint{1.685096in}{1.127889in}}%
\pgfpathlineto{\pgfqpoint{1.703553in}{1.115259in}}%
\pgfpathlineto{\pgfqpoint{1.721531in}{1.103247in}}%
\pgfpathlineto{\pgfqpoint{1.739037in}{1.091819in}}%
\pgfpathlineto{\pgfqpoint{1.756077in}{1.080943in}}%
\pgfpathlineto{\pgfqpoint{1.772657in}{1.070592in}}%
\pgfpathlineto{\pgfqpoint{1.788784in}{1.060737in}}%
\pgfpathlineto{\pgfqpoint{1.804462in}{1.051354in}}%
\pgfpathlineto{\pgfqpoint{1.819697in}{1.042420in}}%
\pgfpathlineto{\pgfqpoint{1.834494in}{1.033913in}}%
\pgfpathlineto{\pgfqpoint{1.848858in}{1.025813in}}%
\pgfpathlineto{\pgfqpoint{1.862793in}{1.018103in}}%
\pgfpathlineto{\pgfqpoint{1.876303in}{1.010763in}}%
\pgfpathlineto{\pgfqpoint{1.889391in}{1.003779in}}%
\pgfpathlineto{\pgfqpoint{1.902061in}{0.997135in}}%
\pgfpathlineto{\pgfqpoint{1.914315in}{0.990817in}}%
\pgfpathlineto{\pgfqpoint{1.926156in}{0.984812in}}%
\pgfpathlineto{\pgfqpoint{1.937586in}{0.979108in}}%
\pgfpathlineto{\pgfqpoint{1.948607in}{0.973693in}}%
\pgfpathlineto{\pgfqpoint{1.959221in}{0.968557in}}%
\pgfpathlineto{\pgfqpoint{1.969429in}{0.963688in}}%
\pgfpathlineto{\pgfqpoint{1.979233in}{0.959079in}}%
\pgfpathlineto{\pgfqpoint{1.988636in}{0.954720in}}%
\pgfpathlineto{\pgfqpoint{1.997640in}{0.950604in}}%
\pgfpathlineto{\pgfqpoint{2.006245in}{0.946721in}}%
\pgfpathlineto{\pgfqpoint{2.014455in}{0.943065in}}%
\pgfusepath{stroke}%
\end{pgfscope}%
\begin{pgfscope}%
\pgfpathrectangle{\pgfqpoint{0.675193in}{0.526079in}}{\pgfqpoint{4.650000in}{3.020000in}}%
\pgfusepath{clip}%
\pgfsetrectcap%
\pgfsetroundjoin%
\pgfsetlinewidth{1.505625pt}%
\definecolor{currentstroke}{rgb}{1.000000,0.000000,0.000000}%
\pgfsetstrokecolor{currentstroke}%
\pgfsetstrokeopacity{0.750000}%
\pgfsetdash{}{0pt}%
\pgfpathmoveto{\pgfqpoint{1.132009in}{3.358593in}}%
\pgfpathlineto{\pgfqpoint{1.176445in}{3.363332in}}%
\pgfpathlineto{\pgfqpoint{1.219682in}{3.367933in}}%
\pgfpathlineto{\pgfqpoint{1.261735in}{3.371645in}}%
\pgfpathlineto{\pgfqpoint{1.302618in}{3.374699in}}%
\pgfpathlineto{\pgfqpoint{1.342347in}{3.377254in}}%
\pgfpathlineto{\pgfqpoint{1.380936in}{3.379421in}}%
\pgfpathlineto{\pgfqpoint{1.418401in}{3.381284in}}%
\pgfpathlineto{\pgfqpoint{1.454756in}{3.382905in}}%
\pgfpathlineto{\pgfqpoint{1.490016in}{3.384329in}}%
\pgfpathlineto{\pgfqpoint{1.524195in}{3.385593in}}%
\pgfpathlineto{\pgfqpoint{1.557309in}{3.386724in}}%
\pgfpathlineto{\pgfqpoint{1.589373in}{3.387744in}}%
\pgfpathlineto{\pgfqpoint{1.620401in}{3.388671in}}%
\pgfpathlineto{\pgfqpoint{1.650410in}{3.389519in}}%
\pgfpathlineto{\pgfqpoint{1.679414in}{3.390298in}}%
\pgfpathlineto{\pgfqpoint{1.707427in}{3.391020in}}%
\pgfpathlineto{\pgfqpoint{1.734466in}{3.391690in}}%
\pgfpathlineto{\pgfqpoint{1.760544in}{3.392316in}}%
\pgfpathlineto{\pgfqpoint{1.785678in}{3.392903in}}%
\pgfpathlineto{\pgfqpoint{1.809881in}{3.393456in}}%
\pgfpathlineto{\pgfqpoint{1.833167in}{3.393978in}}%
\pgfpathlineto{\pgfqpoint{1.855549in}{3.394472in}}%
\pgfpathlineto{\pgfqpoint{1.877041in}{3.394942in}}%
\pgfpathlineto{\pgfqpoint{1.897656in}{3.395388in}}%
\pgfpathlineto{\pgfqpoint{1.917406in}{3.395815in}}%
\pgfpathlineto{\pgfqpoint{1.936302in}{3.396222in}}%
\pgfpathlineto{\pgfqpoint{1.954357in}{3.396613in}}%
\pgfpathlineto{\pgfqpoint{1.971581in}{3.396987in}}%
\pgfpathlineto{\pgfqpoint{1.987984in}{3.397346in}}%
\pgfusepath{stroke}%
\end{pgfscope}%
\begin{pgfscope}%
\pgfpathrectangle{\pgfqpoint{0.675193in}{0.526079in}}{\pgfqpoint{4.650000in}{3.020000in}}%
\pgfusepath{clip}%
\pgfsetrectcap%
\pgfsetroundjoin%
\pgfsetlinewidth{1.505625pt}%
\definecolor{currentstroke}{rgb}{0.000000,0.000000,1.000000}%
\pgfsetstrokecolor{currentstroke}%
\pgfsetstrokeopacity{0.750000}%
\pgfsetdash{}{0pt}%
\pgfpathmoveto{\pgfqpoint{1.132009in}{1.144534in}}%
\pgfpathlineto{\pgfqpoint{1.176445in}{1.137841in}}%
\pgfpathlineto{\pgfqpoint{1.219682in}{1.130848in}}%
\pgfpathlineto{\pgfqpoint{1.261735in}{1.122747in}}%
\pgfpathlineto{\pgfqpoint{1.302618in}{1.113720in}}%
\pgfpathlineto{\pgfqpoint{1.342347in}{1.103886in}}%
\pgfpathlineto{\pgfqpoint{1.380936in}{1.093320in}}%
\pgfpathlineto{\pgfqpoint{1.418401in}{1.082073in}}%
\pgfpathlineto{\pgfqpoint{1.454756in}{1.070177in}}%
\pgfpathlineto{\pgfqpoint{1.490016in}{1.057648in}}%
\pgfpathlineto{\pgfqpoint{1.524195in}{1.044494in}}%
\pgfpathlineto{\pgfqpoint{1.557309in}{1.030714in}}%
\pgfpathlineto{\pgfqpoint{1.589373in}{1.016301in}}%
\pgfpathlineto{\pgfqpoint{1.620401in}{1.001245in}}%
\pgfpathlineto{\pgfqpoint{1.650410in}{0.985529in}}%
\pgfpathlineto{\pgfqpoint{1.679414in}{0.969133in}}%
\pgfpathlineto{\pgfqpoint{1.707427in}{0.952034in}}%
\pgfpathlineto{\pgfqpoint{1.734466in}{0.934204in}}%
\pgfpathlineto{\pgfqpoint{1.760544in}{0.915613in}}%
\pgfpathlineto{\pgfqpoint{1.785678in}{0.896225in}}%
\pgfpathlineto{\pgfqpoint{1.809881in}{0.876001in}}%
\pgfpathlineto{\pgfqpoint{1.833167in}{0.854899in}}%
\pgfpathlineto{\pgfqpoint{1.855549in}{0.832867in}}%
\pgfpathlineto{\pgfqpoint{1.877041in}{0.809853in}}%
\pgfpathlineto{\pgfqpoint{1.897656in}{0.785796in}}%
\pgfpathlineto{\pgfqpoint{1.917406in}{0.760627in}}%
\pgfpathlineto{\pgfqpoint{1.936302in}{0.734269in}}%
\pgfpathlineto{\pgfqpoint{1.954357in}{0.706637in}}%
\pgfpathlineto{\pgfqpoint{1.971581in}{0.677631in}}%
\pgfpathlineto{\pgfqpoint{1.987984in}{0.647140in}}%
\pgfusepath{stroke}%
\end{pgfscope}%
\begin{pgfscope}%
\pgfpathrectangle{\pgfqpoint{0.675193in}{0.526079in}}{\pgfqpoint{4.650000in}{3.020000in}}%
\pgfusepath{clip}%
\pgfsetrectcap%
\pgfsetroundjoin%
\pgfsetlinewidth{1.505625pt}%
\definecolor{currentstroke}{rgb}{0.000000,0.750000,0.750000}%
\pgfsetstrokecolor{currentstroke}%
\pgfsetstrokeopacity{0.750000}%
\pgfsetdash{}{0pt}%
\pgfpathmoveto{\pgfqpoint{1.132009in}{2.021584in}}%
\pgfpathlineto{\pgfqpoint{1.176445in}{1.950266in}}%
\pgfpathlineto{\pgfqpoint{1.219682in}{1.887398in}}%
\pgfpathlineto{\pgfqpoint{1.261735in}{1.830806in}}%
\pgfpathlineto{\pgfqpoint{1.302618in}{1.779619in}}%
\pgfpathlineto{\pgfqpoint{1.342347in}{1.733121in}}%
\pgfpathlineto{\pgfqpoint{1.380936in}{1.690722in}}%
\pgfpathlineto{\pgfqpoint{1.418401in}{1.651930in}}%
\pgfpathlineto{\pgfqpoint{1.454756in}{1.616332in}}%
\pgfpathlineto{\pgfqpoint{1.490016in}{1.583577in}}%
\pgfpathlineto{\pgfqpoint{1.524195in}{1.553366in}}%
\pgfpathlineto{\pgfqpoint{1.557309in}{1.525441in}}%
\pgfpathlineto{\pgfqpoint{1.589373in}{1.499579in}}%
\pgfpathlineto{\pgfqpoint{1.620401in}{1.475589in}}%
\pgfpathlineto{\pgfqpoint{1.650410in}{1.453300in}}%
\pgfpathlineto{\pgfqpoint{1.679414in}{1.432563in}}%
\pgfpathlineto{\pgfqpoint{1.707427in}{1.413248in}}%
\pgfpathlineto{\pgfqpoint{1.734466in}{1.395238in}}%
\pgfpathlineto{\pgfqpoint{1.760544in}{1.378430in}}%
\pgfpathlineto{\pgfqpoint{1.785678in}{1.362733in}}%
\pgfpathlineto{\pgfqpoint{1.809881in}{1.348065in}}%
\pgfpathlineto{\pgfqpoint{1.833167in}{1.334352in}}%
\pgfpathlineto{\pgfqpoint{1.855549in}{1.321526in}}%
\pgfpathlineto{\pgfqpoint{1.877041in}{1.309529in}}%
\pgfpathlineto{\pgfqpoint{1.897656in}{1.298307in}}%
\pgfpathlineto{\pgfqpoint{1.917406in}{1.287811in}}%
\pgfpathlineto{\pgfqpoint{1.936302in}{1.277997in}}%
\pgfpathlineto{\pgfqpoint{1.954357in}{1.268826in}}%
\pgfpathlineto{\pgfqpoint{1.971581in}{1.260260in}}%
\pgfpathlineto{\pgfqpoint{1.987984in}{1.252266in}}%
\pgfusepath{stroke}%
\end{pgfscope}%
\begin{pgfscope}%
\pgfpathrectangle{\pgfqpoint{0.675193in}{0.526079in}}{\pgfqpoint{4.650000in}{3.020000in}}%
\pgfusepath{clip}%
\pgfsetrectcap%
\pgfsetroundjoin%
\pgfsetlinewidth{1.505625pt}%
\definecolor{currentstroke}{rgb}{1.000000,0.000000,0.000000}%
\pgfsetstrokecolor{currentstroke}%
\pgfsetstrokeopacity{0.750000}%
\pgfsetdash{}{0pt}%
\pgfpathmoveto{\pgfqpoint{1.223532in}{3.281749in}}%
\pgfpathlineto{\pgfqpoint{1.267962in}{3.291974in}}%
\pgfpathlineto{\pgfqpoint{1.311046in}{3.301864in}}%
\pgfpathlineto{\pgfqpoint{1.352819in}{3.310144in}}%
\pgfpathlineto{\pgfqpoint{1.393311in}{3.317161in}}%
\pgfpathlineto{\pgfqpoint{1.432557in}{3.323168in}}%
\pgfpathlineto{\pgfqpoint{1.470589in}{3.328361in}}%
\pgfpathlineto{\pgfqpoint{1.507439in}{3.332887in}}%
\pgfpathlineto{\pgfqpoint{1.543140in}{3.336861in}}%
\pgfpathlineto{\pgfqpoint{1.577725in}{3.340376in}}%
\pgfpathlineto{\pgfqpoint{1.611227in}{3.343504in}}%
\pgfpathlineto{\pgfqpoint{1.643678in}{3.346304in}}%
\pgfpathlineto{\pgfqpoint{1.675112in}{3.348825in}}%
\pgfpathlineto{\pgfqpoint{1.705577in}{3.351104in}}%
\pgfpathlineto{\pgfqpoint{1.735101in}{3.353174in}}%
\pgfpathlineto{\pgfqpoint{1.763713in}{3.355063in}}%
\pgfpathlineto{\pgfqpoint{1.791444in}{3.356793in}}%
\pgfpathlineto{\pgfqpoint{1.818323in}{3.358383in}}%
\pgfpathlineto{\pgfqpoint{1.844379in}{3.359849in}}%
\pgfpathlineto{\pgfqpoint{1.869636in}{3.361206in}}%
\pgfpathlineto{\pgfqpoint{1.894120in}{3.362465in}}%
\pgfpathlineto{\pgfqpoint{1.917850in}{3.363637in}}%
\pgfpathlineto{\pgfqpoint{1.940850in}{3.364730in}}%
\pgfpathlineto{\pgfqpoint{1.963137in}{3.365751in}}%
\pgfpathlineto{\pgfqpoint{1.984729in}{3.366708in}}%
\pgfpathlineto{\pgfqpoint{2.005641in}{3.367607in}}%
\pgfpathlineto{\pgfqpoint{2.025885in}{3.368452in}}%
\pgfpathlineto{\pgfqpoint{2.045473in}{3.369249in}}%
\pgfpathlineto{\pgfqpoint{2.064418in}{3.370001in}}%
\pgfpathlineto{\pgfqpoint{2.082730in}{3.370711in}}%
\pgfpathlineto{\pgfqpoint{2.100420in}{3.371384in}}%
\pgfpathlineto{\pgfqpoint{2.117499in}{3.372021in}}%
\pgfpathlineto{\pgfqpoint{2.133975in}{3.372625in}}%
\pgfpathlineto{\pgfqpoint{2.149860in}{3.373199in}}%
\pgfpathlineto{\pgfqpoint{2.165164in}{3.373744in}}%
\pgfpathlineto{\pgfqpoint{2.179897in}{3.374263in}}%
\pgfpathlineto{\pgfqpoint{2.194068in}{3.374755in}}%
\pgfpathlineto{\pgfqpoint{2.207685in}{3.375224in}}%
\pgfpathlineto{\pgfqpoint{2.220755in}{3.375670in}}%
\pgfusepath{stroke}%
\end{pgfscope}%
\begin{pgfscope}%
\pgfpathrectangle{\pgfqpoint{0.675193in}{0.526079in}}{\pgfqpoint{4.650000in}{3.020000in}}%
\pgfusepath{clip}%
\pgfsetrectcap%
\pgfsetroundjoin%
\pgfsetlinewidth{1.505625pt}%
\definecolor{currentstroke}{rgb}{0.000000,0.000000,1.000000}%
\pgfsetstrokecolor{currentstroke}%
\pgfsetstrokeopacity{0.750000}%
\pgfsetdash{}{0pt}%
\pgfpathmoveto{\pgfqpoint{1.223532in}{1.038478in}}%
\pgfpathlineto{\pgfqpoint{1.267962in}{1.038131in}}%
\pgfpathlineto{\pgfqpoint{1.311046in}{1.037547in}}%
\pgfpathlineto{\pgfqpoint{1.352819in}{1.035369in}}%
\pgfpathlineto{\pgfqpoint{1.393311in}{1.031885in}}%
\pgfpathlineto{\pgfqpoint{1.432557in}{1.027297in}}%
\pgfpathlineto{\pgfqpoint{1.470589in}{1.021756in}}%
\pgfpathlineto{\pgfqpoint{1.507439in}{1.015374in}}%
\pgfpathlineto{\pgfqpoint{1.543140in}{1.008235in}}%
\pgfpathlineto{\pgfqpoint{1.577725in}{1.000402in}}%
\pgfpathlineto{\pgfqpoint{1.611227in}{0.991924in}}%
\pgfpathlineto{\pgfqpoint{1.643678in}{0.982834in}}%
\pgfpathlineto{\pgfqpoint{1.675112in}{0.973157in}}%
\pgfpathlineto{\pgfqpoint{1.705577in}{0.962912in}}%
\pgfpathlineto{\pgfqpoint{1.735101in}{0.952111in}}%
\pgfpathlineto{\pgfqpoint{1.763713in}{0.940759in}}%
\pgfpathlineto{\pgfqpoint{1.791444in}{0.928859in}}%
\pgfpathlineto{\pgfqpoint{1.818323in}{0.916411in}}%
\pgfpathlineto{\pgfqpoint{1.844379in}{0.903408in}}%
\pgfpathlineto{\pgfqpoint{1.869636in}{0.889845in}}%
\pgfpathlineto{\pgfqpoint{1.894120in}{0.875710in}}%
\pgfpathlineto{\pgfqpoint{1.917850in}{0.860991in}}%
\pgfpathlineto{\pgfqpoint{1.940850in}{0.845673in}}%
\pgfpathlineto{\pgfqpoint{1.963137in}{0.829736in}}%
\pgfpathlineto{\pgfqpoint{1.984729in}{0.813161in}}%
\pgfpathlineto{\pgfqpoint{2.005641in}{0.795924in}}%
\pgfpathlineto{\pgfqpoint{2.025885in}{0.778000in}}%
\pgfpathlineto{\pgfqpoint{2.045473in}{0.759358in}}%
\pgfpathlineto{\pgfqpoint{2.064418in}{0.739967in}}%
\pgfpathlineto{\pgfqpoint{2.082730in}{0.719790in}}%
\pgfpathlineto{\pgfqpoint{2.100420in}{0.698786in}}%
\pgfpathlineto{\pgfqpoint{2.117499in}{0.676910in}}%
\pgfpathlineto{\pgfqpoint{2.133975in}{0.654113in}}%
\pgfpathlineto{\pgfqpoint{2.149860in}{0.630338in}}%
\pgfpathlineto{\pgfqpoint{2.165164in}{0.605522in}}%
\pgfpathlineto{\pgfqpoint{2.179897in}{0.579596in}}%
\pgfpathlineto{\pgfqpoint{2.194068in}{0.552479in}}%
\pgfpathlineto{\pgfqpoint{2.207685in}{0.524082in}}%
\pgfpathlineto{\pgfqpoint{2.212904in}{0.512191in}}%
\pgfusepath{stroke}%
\end{pgfscope}%
\begin{pgfscope}%
\pgfpathrectangle{\pgfqpoint{0.675193in}{0.526079in}}{\pgfqpoint{4.650000in}{3.020000in}}%
\pgfusepath{clip}%
\pgfsetrectcap%
\pgfsetroundjoin%
\pgfsetlinewidth{1.505625pt}%
\definecolor{currentstroke}{rgb}{0.000000,0.750000,0.750000}%
\pgfsetstrokecolor{currentstroke}%
\pgfsetstrokeopacity{0.750000}%
\pgfsetdash{}{0pt}%
\pgfpathmoveto{\pgfqpoint{1.223532in}{2.537324in}}%
\pgfpathlineto{\pgfqpoint{1.267962in}{2.486556in}}%
\pgfpathlineto{\pgfqpoint{1.311046in}{2.441538in}}%
\pgfpathlineto{\pgfqpoint{1.352819in}{2.400101in}}%
\pgfpathlineto{\pgfqpoint{1.393311in}{2.361877in}}%
\pgfpathlineto{\pgfqpoint{1.432557in}{2.326538in}}%
\pgfpathlineto{\pgfqpoint{1.470589in}{2.293799in}}%
\pgfpathlineto{\pgfqpoint{1.507439in}{2.263409in}}%
\pgfpathlineto{\pgfqpoint{1.543140in}{2.235148in}}%
\pgfpathlineto{\pgfqpoint{1.577725in}{2.208822in}}%
\pgfpathlineto{\pgfqpoint{1.611227in}{2.184259in}}%
\pgfpathlineto{\pgfqpoint{1.643678in}{2.161308in}}%
\pgfpathlineto{\pgfqpoint{1.675112in}{2.139834in}}%
\pgfpathlineto{\pgfqpoint{1.705577in}{2.119715in}}%
\pgfpathlineto{\pgfqpoint{1.735101in}{2.100844in}}%
\pgfpathlineto{\pgfqpoint{1.763713in}{2.083125in}}%
\pgfpathlineto{\pgfqpoint{1.791444in}{2.066471in}}%
\pgfpathlineto{\pgfqpoint{1.818323in}{2.050804in}}%
\pgfpathlineto{\pgfqpoint{1.844379in}{2.036053in}}%
\pgfpathlineto{\pgfqpoint{1.869636in}{2.022154in}}%
\pgfpathlineto{\pgfqpoint{1.894120in}{2.009050in}}%
\pgfpathlineto{\pgfqpoint{1.917850in}{1.996687in}}%
\pgfpathlineto{\pgfqpoint{1.940850in}{1.985019in}}%
\pgfpathlineto{\pgfqpoint{1.963137in}{1.974000in}}%
\pgfpathlineto{\pgfqpoint{1.984729in}{1.963591in}}%
\pgfpathlineto{\pgfqpoint{2.005641in}{1.953756in}}%
\pgfpathlineto{\pgfqpoint{2.025885in}{1.944462in}}%
\pgfpathlineto{\pgfqpoint{2.045473in}{1.935676in}}%
\pgfpathlineto{\pgfqpoint{2.064418in}{1.927372in}}%
\pgfpathlineto{\pgfqpoint{2.082730in}{1.919523in}}%
\pgfpathlineto{\pgfqpoint{2.100420in}{1.912105in}}%
\pgfpathlineto{\pgfqpoint{2.117499in}{1.905097in}}%
\pgfpathlineto{\pgfqpoint{2.133975in}{1.898477in}}%
\pgfpathlineto{\pgfqpoint{2.149860in}{1.892228in}}%
\pgfpathlineto{\pgfqpoint{2.165164in}{1.886331in}}%
\pgfpathlineto{\pgfqpoint{2.179897in}{1.880772in}}%
\pgfpathlineto{\pgfqpoint{2.194068in}{1.875533in}}%
\pgfpathlineto{\pgfqpoint{2.207685in}{1.870603in}}%
\pgfpathlineto{\pgfqpoint{2.220755in}{1.865969in}}%
\pgfusepath{stroke}%
\end{pgfscope}%
\begin{pgfscope}%
\pgfpathrectangle{\pgfqpoint{0.675193in}{0.526079in}}{\pgfqpoint{4.650000in}{3.020000in}}%
\pgfusepath{clip}%
\pgfsetrectcap%
\pgfsetroundjoin%
\pgfsetlinewidth{1.505625pt}%
\definecolor{currentstroke}{rgb}{1.000000,0.000000,0.000000}%
\pgfsetstrokecolor{currentstroke}%
\pgfsetstrokeopacity{0.750000}%
\pgfsetdash{}{0pt}%
\pgfpathmoveto{\pgfqpoint{1.067481in}{3.299343in}}%
\pgfpathlineto{\pgfqpoint{1.110135in}{3.309660in}}%
\pgfpathlineto{\pgfqpoint{1.152001in}{3.319374in}}%
\pgfpathlineto{\pgfqpoint{1.193092in}{3.327203in}}%
\pgfpathlineto{\pgfqpoint{1.233421in}{3.333621in}}%
\pgfpathlineto{\pgfqpoint{1.273001in}{3.338956in}}%
\pgfpathlineto{\pgfqpoint{1.311843in}{3.343447in}}%
\pgfpathlineto{\pgfqpoint{1.349961in}{3.347272in}}%
\pgfpathlineto{\pgfqpoint{1.387368in}{3.350562in}}%
\pgfpathlineto{\pgfqpoint{1.424075in}{3.353418in}}%
\pgfpathlineto{\pgfqpoint{1.460095in}{3.355917in}}%
\pgfpathlineto{\pgfqpoint{1.495442in}{3.358121in}}%
\pgfpathlineto{\pgfqpoint{1.530127in}{3.360078in}}%
\pgfpathlineto{\pgfqpoint{1.564159in}{3.361828in}}%
\pgfpathlineto{\pgfqpoint{1.597553in}{3.363402in}}%
\pgfpathlineto{\pgfqpoint{1.630322in}{3.364825in}}%
\pgfpathlineto{\pgfqpoint{1.662479in}{3.366119in}}%
\pgfpathlineto{\pgfqpoint{1.694039in}{3.367300in}}%
\pgfpathlineto{\pgfqpoint{1.725017in}{3.368385in}}%
\pgfpathlineto{\pgfqpoint{1.755427in}{3.369386in}}%
\pgfpathlineto{\pgfqpoint{1.785284in}{3.370313in}}%
\pgfpathlineto{\pgfqpoint{1.814604in}{3.371174in}}%
\pgfpathlineto{\pgfqpoint{1.843400in}{3.371978in}}%
\pgfpathlineto{\pgfqpoint{1.871689in}{3.372732in}}%
\pgfpathlineto{\pgfqpoint{1.899483in}{3.373440in}}%
\pgfpathlineto{\pgfqpoint{1.926797in}{3.374108in}}%
\pgfpathlineto{\pgfqpoint{1.953643in}{3.374740in}}%
\pgfpathlineto{\pgfqpoint{1.980032in}{3.375339in}}%
\pgfpathlineto{\pgfqpoint{2.005975in}{3.375910in}}%
\pgfpathlineto{\pgfqpoint{2.031482in}{3.376456in}}%
\pgfpathlineto{\pgfqpoint{2.056561in}{3.376977in}}%
\pgfpathlineto{\pgfqpoint{2.081220in}{3.377478in}}%
\pgfpathlineto{\pgfqpoint{2.105467in}{3.377960in}}%
\pgfpathlineto{\pgfqpoint{2.129309in}{3.378424in}}%
\pgfpathlineto{\pgfqpoint{2.152752in}{3.378872in}}%
\pgfpathlineto{\pgfqpoint{2.175801in}{3.379307in}}%
\pgfpathlineto{\pgfqpoint{2.198463in}{3.379728in}}%
\pgfpathlineto{\pgfqpoint{2.220743in}{3.380137in}}%
\pgfpathlineto{\pgfqpoint{2.242646in}{3.380536in}}%
\pgfpathlineto{\pgfqpoint{2.264177in}{3.380925in}}%
\pgfpathlineto{\pgfqpoint{2.285342in}{3.381305in}}%
\pgfpathlineto{\pgfqpoint{2.306146in}{3.381676in}}%
\pgfpathlineto{\pgfqpoint{2.326594in}{3.382040in}}%
\pgfpathlineto{\pgfqpoint{2.346691in}{3.382398in}}%
\pgfpathlineto{\pgfqpoint{2.366442in}{3.382749in}}%
\pgfpathlineto{\pgfqpoint{2.385853in}{3.383092in}}%
\pgfpathlineto{\pgfqpoint{2.404929in}{3.383430in}}%
\pgfpathlineto{\pgfqpoint{2.423673in}{3.383764in}}%
\pgfpathlineto{\pgfqpoint{2.442092in}{3.384093in}}%
\pgfpathlineto{\pgfqpoint{2.460188in}{3.384418in}}%
\pgfpathlineto{\pgfqpoint{2.477967in}{3.384738in}}%
\pgfpathlineto{\pgfqpoint{2.495432in}{3.385055in}}%
\pgfpathlineto{\pgfqpoint{2.512587in}{3.385369in}}%
\pgfpathlineto{\pgfqpoint{2.529435in}{3.385679in}}%
\pgfpathlineto{\pgfqpoint{2.545981in}{3.385986in}}%
\pgfpathlineto{\pgfqpoint{2.562226in}{3.386290in}}%
\pgfpathlineto{\pgfqpoint{2.578173in}{3.386591in}}%
\pgfpathlineto{\pgfqpoint{2.593827in}{3.386890in}}%
\pgfpathlineto{\pgfqpoint{2.609189in}{3.387186in}}%
\pgfpathlineto{\pgfqpoint{2.624262in}{3.387480in}}%
\pgfpathlineto{\pgfqpoint{2.639048in}{3.387772in}}%
\pgfpathlineto{\pgfqpoint{2.653551in}{3.388062in}}%
\pgfpathlineto{\pgfqpoint{2.667773in}{3.388350in}}%
\pgfpathlineto{\pgfqpoint{2.681715in}{3.388636in}}%
\pgfpathlineto{\pgfqpoint{2.695381in}{3.388919in}}%
\pgfpathlineto{\pgfqpoint{2.708772in}{3.389201in}}%
\pgfpathlineto{\pgfqpoint{2.721892in}{3.389481in}}%
\pgfpathlineto{\pgfqpoint{2.734743in}{3.389760in}}%
\pgfpathlineto{\pgfqpoint{2.747326in}{3.390036in}}%
\pgfpathlineto{\pgfqpoint{2.759646in}{3.390311in}}%
\pgfpathlineto{\pgfqpoint{2.771703in}{3.390584in}}%
\pgfpathlineto{\pgfqpoint{2.783500in}{3.390856in}}%
\pgfpathlineto{\pgfqpoint{2.795040in}{3.391126in}}%
\pgfpathlineto{\pgfqpoint{2.806324in}{3.391394in}}%
\pgfpathlineto{\pgfqpoint{2.817355in}{3.391661in}}%
\pgfpathlineto{\pgfqpoint{2.828134in}{3.391926in}}%
\pgfpathlineto{\pgfqpoint{2.838665in}{3.392189in}}%
\pgfpathlineto{\pgfqpoint{2.848947in}{3.392451in}}%
\pgfpathlineto{\pgfqpoint{2.858984in}{3.392711in}}%
\pgfpathlineto{\pgfqpoint{2.868775in}{3.392970in}}%
\pgfpathlineto{\pgfqpoint{2.878324in}{3.393226in}}%
\pgfpathlineto{\pgfqpoint{2.887631in}{3.393481in}}%
\pgfpathlineto{\pgfqpoint{2.896696in}{3.393734in}}%
\pgfpathlineto{\pgfqpoint{2.905523in}{3.393986in}}%
\pgfpathlineto{\pgfqpoint{2.914111in}{3.394235in}}%
\pgfpathlineto{\pgfqpoint{2.922462in}{3.394483in}}%
\pgfpathlineto{\pgfqpoint{2.930576in}{3.394728in}}%
\pgfpathlineto{\pgfqpoint{2.938455in}{3.394972in}}%
\pgfpathlineto{\pgfqpoint{2.946100in}{3.395213in}}%
\pgfpathlineto{\pgfqpoint{2.953512in}{3.395453in}}%
\pgfpathlineto{\pgfqpoint{2.960692in}{3.395691in}}%
\pgfpathlineto{\pgfqpoint{2.967640in}{3.395926in}}%
\pgfpathlineto{\pgfqpoint{2.974359in}{3.396159in}}%
\pgfpathlineto{\pgfqpoint{2.980849in}{3.396390in}}%
\pgfpathlineto{\pgfqpoint{2.987110in}{3.396619in}}%
\pgfpathlineto{\pgfqpoint{2.993145in}{3.396845in}}%
\pgfpathlineto{\pgfqpoint{2.998953in}{3.397069in}}%
\pgfpathlineto{\pgfqpoint{3.004537in}{3.397290in}}%
\pgfpathlineto{\pgfqpoint{3.009897in}{3.397508in}}%
\pgfpathlineto{\pgfqpoint{3.015034in}{3.397723in}}%
\pgfpathlineto{\pgfqpoint{3.019948in}{3.397936in}}%
\pgfpathlineto{\pgfqpoint{3.024641in}{3.398146in}}%
\pgfusepath{stroke}%
\end{pgfscope}%
\begin{pgfscope}%
\pgfpathrectangle{\pgfqpoint{0.675193in}{0.526079in}}{\pgfqpoint{4.650000in}{3.020000in}}%
\pgfusepath{clip}%
\pgfsetrectcap%
\pgfsetroundjoin%
\pgfsetlinewidth{1.505625pt}%
\definecolor{currentstroke}{rgb}{0.000000,0.000000,1.000000}%
\pgfsetstrokecolor{currentstroke}%
\pgfsetstrokeopacity{0.750000}%
\pgfsetdash{}{0pt}%
\pgfpathmoveto{\pgfqpoint{1.067481in}{1.531089in}}%
\pgfpathlineto{\pgfqpoint{1.110135in}{1.537955in}}%
\pgfpathlineto{\pgfqpoint{1.152001in}{1.544434in}}%
\pgfpathlineto{\pgfqpoint{1.193092in}{1.549183in}}%
\pgfpathlineto{\pgfqpoint{1.233421in}{1.552631in}}%
\pgfpathlineto{\pgfqpoint{1.273001in}{1.555071in}}%
\pgfpathlineto{\pgfqpoint{1.311843in}{1.556713in}}%
\pgfpathlineto{\pgfqpoint{1.349961in}{1.557713in}}%
\pgfpathlineto{\pgfqpoint{1.387368in}{1.558183in}}%
\pgfpathlineto{\pgfqpoint{1.424075in}{1.558209in}}%
\pgfpathlineto{\pgfqpoint{1.460095in}{1.557855in}}%
\pgfpathlineto{\pgfqpoint{1.495442in}{1.557170in}}%
\pgfpathlineto{\pgfqpoint{1.530127in}{1.556195in}}%
\pgfpathlineto{\pgfqpoint{1.564159in}{1.554958in}}%
\pgfpathlineto{\pgfqpoint{1.597553in}{1.553483in}}%
\pgfpathlineto{\pgfqpoint{1.630322in}{1.551790in}}%
\pgfpathlineto{\pgfqpoint{1.662479in}{1.549892in}}%
\pgfpathlineto{\pgfqpoint{1.694039in}{1.547801in}}%
\pgfpathlineto{\pgfqpoint{1.725017in}{1.545528in}}%
\pgfpathlineto{\pgfqpoint{1.755427in}{1.543081in}}%
\pgfpathlineto{\pgfqpoint{1.785284in}{1.540465in}}%
\pgfpathlineto{\pgfqpoint{1.814604in}{1.537684in}}%
\pgfpathlineto{\pgfqpoint{1.843400in}{1.534743in}}%
\pgfpathlineto{\pgfqpoint{1.871689in}{1.531645in}}%
\pgfpathlineto{\pgfqpoint{1.899483in}{1.528392in}}%
\pgfpathlineto{\pgfqpoint{1.926797in}{1.524986in}}%
\pgfpathlineto{\pgfqpoint{1.953643in}{1.521427in}}%
\pgfpathlineto{\pgfqpoint{1.980032in}{1.517717in}}%
\pgfpathlineto{\pgfqpoint{2.005975in}{1.513857in}}%
\pgfpathlineto{\pgfqpoint{2.031482in}{1.509846in}}%
\pgfpathlineto{\pgfqpoint{2.056561in}{1.505685in}}%
\pgfpathlineto{\pgfqpoint{2.081220in}{1.501373in}}%
\pgfpathlineto{\pgfqpoint{2.105467in}{1.496910in}}%
\pgfpathlineto{\pgfqpoint{2.129309in}{1.492294in}}%
\pgfpathlineto{\pgfqpoint{2.152752in}{1.487526in}}%
\pgfpathlineto{\pgfqpoint{2.175801in}{1.482603in}}%
\pgfpathlineto{\pgfqpoint{2.198463in}{1.477525in}}%
\pgfpathlineto{\pgfqpoint{2.220743in}{1.472291in}}%
\pgfpathlineto{\pgfqpoint{2.242646in}{1.466898in}}%
\pgfpathlineto{\pgfqpoint{2.264177in}{1.461346in}}%
\pgfpathlineto{\pgfqpoint{2.285342in}{1.455632in}}%
\pgfpathlineto{\pgfqpoint{2.306146in}{1.449755in}}%
\pgfpathlineto{\pgfqpoint{2.326594in}{1.443712in}}%
\pgfpathlineto{\pgfqpoint{2.346691in}{1.437502in}}%
\pgfpathlineto{\pgfqpoint{2.366442in}{1.431123in}}%
\pgfpathlineto{\pgfqpoint{2.385853in}{1.424569in}}%
\pgfpathlineto{\pgfqpoint{2.404929in}{1.417840in}}%
\pgfpathlineto{\pgfqpoint{2.423673in}{1.410934in}}%
\pgfpathlineto{\pgfqpoint{2.442092in}{1.403847in}}%
\pgfpathlineto{\pgfqpoint{2.460188in}{1.396577in}}%
\pgfpathlineto{\pgfqpoint{2.477967in}{1.389119in}}%
\pgfpathlineto{\pgfqpoint{2.495432in}{1.381471in}}%
\pgfpathlineto{\pgfqpoint{2.512587in}{1.373628in}}%
\pgfpathlineto{\pgfqpoint{2.529435in}{1.365587in}}%
\pgfpathlineto{\pgfqpoint{2.545981in}{1.357345in}}%
\pgfpathlineto{\pgfqpoint{2.562226in}{1.348895in}}%
\pgfpathlineto{\pgfqpoint{2.578173in}{1.340235in}}%
\pgfpathlineto{\pgfqpoint{2.593827in}{1.331359in}}%
\pgfpathlineto{\pgfqpoint{2.609189in}{1.322262in}}%
\pgfpathlineto{\pgfqpoint{2.624262in}{1.312940in}}%
\pgfpathlineto{\pgfqpoint{2.639048in}{1.303385in}}%
\pgfpathlineto{\pgfqpoint{2.653551in}{1.293594in}}%
\pgfpathlineto{\pgfqpoint{2.667773in}{1.283558in}}%
\pgfpathlineto{\pgfqpoint{2.681715in}{1.273271in}}%
\pgfpathlineto{\pgfqpoint{2.695381in}{1.262726in}}%
\pgfpathlineto{\pgfqpoint{2.708772in}{1.251916in}}%
\pgfpathlineto{\pgfqpoint{2.721892in}{1.240834in}}%
\pgfpathlineto{\pgfqpoint{2.734743in}{1.229470in}}%
\pgfpathlineto{\pgfqpoint{2.747326in}{1.217815in}}%
\pgfpathlineto{\pgfqpoint{2.759646in}{1.205861in}}%
\pgfpathlineto{\pgfqpoint{2.771703in}{1.193596in}}%
\pgfpathlineto{\pgfqpoint{2.783500in}{1.181010in}}%
\pgfpathlineto{\pgfqpoint{2.795040in}{1.168092in}}%
\pgfpathlineto{\pgfqpoint{2.806324in}{1.154830in}}%
\pgfpathlineto{\pgfqpoint{2.817355in}{1.141210in}}%
\pgfpathlineto{\pgfqpoint{2.828134in}{1.127218in}}%
\pgfpathlineto{\pgfqpoint{2.838665in}{1.112839in}}%
\pgfpathlineto{\pgfqpoint{2.848947in}{1.098057in}}%
\pgfpathlineto{\pgfqpoint{2.858984in}{1.082854in}}%
\pgfpathlineto{\pgfqpoint{2.868775in}{1.067213in}}%
\pgfpathlineto{\pgfqpoint{2.878324in}{1.051111in}}%
\pgfpathlineto{\pgfqpoint{2.887631in}{1.034529in}}%
\pgfpathlineto{\pgfqpoint{2.896696in}{1.017441in}}%
\pgfpathlineto{\pgfqpoint{2.905523in}{0.999823in}}%
\pgfpathlineto{\pgfqpoint{2.914111in}{0.981646in}}%
\pgfpathlineto{\pgfqpoint{2.922462in}{0.962880in}}%
\pgfpathlineto{\pgfqpoint{2.930576in}{0.943492in}}%
\pgfpathlineto{\pgfqpoint{2.938455in}{0.923444in}}%
\pgfpathlineto{\pgfqpoint{2.946100in}{0.902698in}}%
\pgfpathlineto{\pgfqpoint{2.953512in}{0.881208in}}%
\pgfpathlineto{\pgfqpoint{2.960692in}{0.858926in}}%
\pgfpathlineto{\pgfqpoint{2.967640in}{0.835797in}}%
\pgfpathlineto{\pgfqpoint{2.974359in}{0.811761in}}%
\pgfpathlineto{\pgfqpoint{2.980849in}{0.786751in}}%
\pgfpathlineto{\pgfqpoint{2.987110in}{0.760689in}}%
\pgfpathlineto{\pgfqpoint{2.993145in}{0.733491in}}%
\pgfpathlineto{\pgfqpoint{2.998953in}{0.705059in}}%
\pgfpathlineto{\pgfqpoint{3.004537in}{0.675281in}}%
\pgfpathlineto{\pgfqpoint{3.009897in}{0.644030in}}%
\pgfpathlineto{\pgfqpoint{3.015034in}{0.611158in}}%
\pgfpathlineto{\pgfqpoint{3.019948in}{0.576494in}}%
\pgfpathlineto{\pgfqpoint{3.024641in}{0.539836in}}%
\pgfusepath{stroke}%
\end{pgfscope}%
\begin{pgfscope}%
\pgfpathrectangle{\pgfqpoint{0.675193in}{0.526079in}}{\pgfqpoint{4.650000in}{3.020000in}}%
\pgfusepath{clip}%
\pgfsetrectcap%
\pgfsetroundjoin%
\pgfsetlinewidth{1.505625pt}%
\definecolor{currentstroke}{rgb}{0.000000,0.750000,0.750000}%
\pgfsetstrokecolor{currentstroke}%
\pgfsetstrokeopacity{0.750000}%
\pgfsetdash{}{0pt}%
\pgfpathmoveto{\pgfqpoint{1.067481in}{2.400885in}}%
\pgfpathlineto{\pgfqpoint{1.110135in}{2.329509in}}%
\pgfpathlineto{\pgfqpoint{1.152001in}{2.266137in}}%
\pgfpathlineto{\pgfqpoint{1.193092in}{2.208080in}}%
\pgfpathlineto{\pgfqpoint{1.233421in}{2.154721in}}%
\pgfpathlineto{\pgfqpoint{1.273001in}{2.105523in}}%
\pgfpathlineto{\pgfqpoint{1.311843in}{2.060026in}}%
\pgfpathlineto{\pgfqpoint{1.349961in}{2.017839in}}%
\pgfpathlineto{\pgfqpoint{1.387368in}{1.978622in}}%
\pgfpathlineto{\pgfqpoint{1.424075in}{1.942077in}}%
\pgfpathlineto{\pgfqpoint{1.460095in}{1.907950in}}%
\pgfpathlineto{\pgfqpoint{1.495442in}{1.876015in}}%
\pgfpathlineto{\pgfqpoint{1.530127in}{1.846076in}}%
\pgfpathlineto{\pgfqpoint{1.564159in}{1.817958in}}%
\pgfpathlineto{\pgfqpoint{1.597553in}{1.791509in}}%
\pgfpathlineto{\pgfqpoint{1.630322in}{1.766590in}}%
\pgfpathlineto{\pgfqpoint{1.662479in}{1.743079in}}%
\pgfpathlineto{\pgfqpoint{1.694039in}{1.720867in}}%
\pgfpathlineto{\pgfqpoint{1.725017in}{1.699856in}}%
\pgfpathlineto{\pgfqpoint{1.755427in}{1.679957in}}%
\pgfpathlineto{\pgfqpoint{1.785284in}{1.661092in}}%
\pgfpathlineto{\pgfqpoint{1.814604in}{1.643186in}}%
\pgfpathlineto{\pgfqpoint{1.843400in}{1.626175in}}%
\pgfpathlineto{\pgfqpoint{1.871689in}{1.609998in}}%
\pgfpathlineto{\pgfqpoint{1.899483in}{1.594601in}}%
\pgfpathlineto{\pgfqpoint{1.926797in}{1.579934in}}%
\pgfpathlineto{\pgfqpoint{1.953643in}{1.565950in}}%
\pgfpathlineto{\pgfqpoint{1.980032in}{1.552608in}}%
\pgfpathlineto{\pgfqpoint{2.005975in}{1.539870in}}%
\pgfpathlineto{\pgfqpoint{2.031482in}{1.527700in}}%
\pgfpathlineto{\pgfqpoint{2.056561in}{1.516064in}}%
\pgfpathlineto{\pgfqpoint{2.081220in}{1.504932in}}%
\pgfpathlineto{\pgfqpoint{2.105467in}{1.494277in}}%
\pgfpathlineto{\pgfqpoint{2.129309in}{1.484071in}}%
\pgfpathlineto{\pgfqpoint{2.152752in}{1.474292in}}%
\pgfpathlineto{\pgfqpoint{2.175801in}{1.464915in}}%
\pgfpathlineto{\pgfqpoint{2.198463in}{1.455921in}}%
\pgfpathlineto{\pgfqpoint{2.220743in}{1.447289in}}%
\pgfpathlineto{\pgfqpoint{2.242646in}{1.439002in}}%
\pgfpathlineto{\pgfqpoint{2.264177in}{1.431042in}}%
\pgfpathlineto{\pgfqpoint{2.285342in}{1.423394in}}%
\pgfpathlineto{\pgfqpoint{2.306146in}{1.416042in}}%
\pgfpathlineto{\pgfqpoint{2.326594in}{1.408973in}}%
\pgfpathlineto{\pgfqpoint{2.346691in}{1.402174in}}%
\pgfpathlineto{\pgfqpoint{2.366442in}{1.395631in}}%
\pgfpathlineto{\pgfqpoint{2.385853in}{1.389333in}}%
\pgfpathlineto{\pgfqpoint{2.404929in}{1.383269in}}%
\pgfpathlineto{\pgfqpoint{2.423673in}{1.377430in}}%
\pgfpathlineto{\pgfqpoint{2.442092in}{1.371806in}}%
\pgfpathlineto{\pgfqpoint{2.460188in}{1.366387in}}%
\pgfpathlineto{\pgfqpoint{2.477967in}{1.361165in}}%
\pgfpathlineto{\pgfqpoint{2.495432in}{1.356132in}}%
\pgfpathlineto{\pgfqpoint{2.512587in}{1.351280in}}%
\pgfpathlineto{\pgfqpoint{2.529435in}{1.346602in}}%
\pgfpathlineto{\pgfqpoint{2.545981in}{1.342092in}}%
\pgfpathlineto{\pgfqpoint{2.562226in}{1.337741in}}%
\pgfpathlineto{\pgfqpoint{2.578173in}{1.333545in}}%
\pgfpathlineto{\pgfqpoint{2.593827in}{1.329498in}}%
\pgfpathlineto{\pgfqpoint{2.609189in}{1.325593in}}%
\pgfpathlineto{\pgfqpoint{2.624262in}{1.321827in}}%
\pgfpathlineto{\pgfqpoint{2.639048in}{1.318193in}}%
\pgfpathlineto{\pgfqpoint{2.653551in}{1.314688in}}%
\pgfpathlineto{\pgfqpoint{2.667773in}{1.311306in}}%
\pgfpathlineto{\pgfqpoint{2.681715in}{1.308043in}}%
\pgfpathlineto{\pgfqpoint{2.695381in}{1.304895in}}%
\pgfpathlineto{\pgfqpoint{2.708772in}{1.301858in}}%
\pgfpathlineto{\pgfqpoint{2.721892in}{1.298929in}}%
\pgfpathlineto{\pgfqpoint{2.734743in}{1.296105in}}%
\pgfpathlineto{\pgfqpoint{2.747326in}{1.293382in}}%
\pgfpathlineto{\pgfqpoint{2.759646in}{1.290755in}}%
\pgfpathlineto{\pgfqpoint{2.771703in}{1.288223in}}%
\pgfpathlineto{\pgfqpoint{2.783500in}{1.285783in}}%
\pgfpathlineto{\pgfqpoint{2.795040in}{1.283433in}}%
\pgfpathlineto{\pgfqpoint{2.806324in}{1.281168in}}%
\pgfpathlineto{\pgfqpoint{2.817355in}{1.278987in}}%
\pgfpathlineto{\pgfqpoint{2.828134in}{1.276887in}}%
\pgfpathlineto{\pgfqpoint{2.838665in}{1.274867in}}%
\pgfpathlineto{\pgfqpoint{2.848947in}{1.272923in}}%
\pgfpathlineto{\pgfqpoint{2.858984in}{1.271054in}}%
\pgfpathlineto{\pgfqpoint{2.868775in}{1.269258in}}%
\pgfpathlineto{\pgfqpoint{2.878324in}{1.267533in}}%
\pgfpathlineto{\pgfqpoint{2.887631in}{1.265877in}}%
\pgfpathlineto{\pgfqpoint{2.896696in}{1.264288in}}%
\pgfpathlineto{\pgfqpoint{2.905523in}{1.262765in}}%
\pgfpathlineto{\pgfqpoint{2.914111in}{1.261306in}}%
\pgfpathlineto{\pgfqpoint{2.922462in}{1.259909in}}%
\pgfpathlineto{\pgfqpoint{2.930576in}{1.258573in}}%
\pgfpathlineto{\pgfqpoint{2.938455in}{1.257297in}}%
\pgfpathlineto{\pgfqpoint{2.946100in}{1.256079in}}%
\pgfpathlineto{\pgfqpoint{2.953512in}{1.254919in}}%
\pgfpathlineto{\pgfqpoint{2.960692in}{1.253814in}}%
\pgfpathlineto{\pgfqpoint{2.967640in}{1.252764in}}%
\pgfpathlineto{\pgfqpoint{2.974359in}{1.251768in}}%
\pgfpathlineto{\pgfqpoint{2.980849in}{1.250824in}}%
\pgfpathlineto{\pgfqpoint{2.987110in}{1.249932in}}%
\pgfpathlineto{\pgfqpoint{2.993145in}{1.249091in}}%
\pgfpathlineto{\pgfqpoint{2.998953in}{1.248299in}}%
\pgfpathlineto{\pgfqpoint{3.004537in}{1.247555in}}%
\pgfpathlineto{\pgfqpoint{3.009897in}{1.246859in}}%
\pgfpathlineto{\pgfqpoint{3.015034in}{1.246211in}}%
\pgfpathlineto{\pgfqpoint{3.019948in}{1.245609in}}%
\pgfpathlineto{\pgfqpoint{3.024641in}{1.245054in}}%
\pgfusepath{stroke}%
\end{pgfscope}%
\begin{pgfscope}%
\pgfpathrectangle{\pgfqpoint{0.675193in}{0.526079in}}{\pgfqpoint{4.650000in}{3.020000in}}%
\pgfusepath{clip}%
\pgfsetrectcap%
\pgfsetroundjoin%
\pgfsetlinewidth{1.505625pt}%
\definecolor{currentstroke}{rgb}{1.000000,0.000000,0.000000}%
\pgfsetstrokecolor{currentstroke}%
\pgfsetstrokeopacity{0.750000}%
\pgfsetdash{}{0pt}%
\pgfpathmoveto{\pgfqpoint{0.943739in}{3.269665in}}%
\pgfpathlineto{\pgfqpoint{1.017119in}{3.296514in}}%
\pgfpathlineto{\pgfqpoint{1.053078in}{3.306553in}}%
\pgfpathlineto{\pgfqpoint{1.123568in}{3.321021in}}%
\pgfpathlineto{\pgfqpoint{1.192204in}{3.330787in}}%
\pgfpathlineto{\pgfqpoint{1.291806in}{3.340473in}}%
\pgfpathlineto{\pgfqpoint{1.418661in}{3.348417in}}%
\pgfpathlineto{\pgfqpoint{1.568331in}{3.354407in}}%
\pgfpathlineto{\pgfqpoint{1.789059in}{3.359981in}}%
\pgfpathlineto{\pgfqpoint{2.212153in}{3.366789in}}%
\pgfpathlineto{\pgfqpoint{2.793221in}{3.376590in}}%
\pgfpathlineto{\pgfqpoint{3.079157in}{3.384386in}}%
\pgfpathlineto{\pgfqpoint{3.271738in}{3.392491in}}%
\pgfpathlineto{\pgfqpoint{3.376355in}{3.399608in}}%
\pgfpathlineto{\pgfqpoint{3.387937in}{3.400757in}}%
\pgfpathlineto{\pgfqpoint{3.387937in}{3.400757in}}%
\pgfusepath{stroke}%
\end{pgfscope}%
\begin{pgfscope}%
\pgfpathrectangle{\pgfqpoint{0.675193in}{0.526079in}}{\pgfqpoint{4.650000in}{3.020000in}}%
\pgfusepath{clip}%
\pgfsetrectcap%
\pgfsetroundjoin%
\pgfsetlinewidth{1.505625pt}%
\definecolor{currentstroke}{rgb}{0.000000,0.000000,1.000000}%
\pgfsetstrokecolor{currentstroke}%
\pgfsetstrokeopacity{0.750000}%
\pgfsetdash{}{0pt}%
\pgfpathmoveto{\pgfqpoint{0.943739in}{1.847438in}}%
\pgfpathlineto{\pgfqpoint{1.017119in}{1.870742in}}%
\pgfpathlineto{\pgfqpoint{1.053078in}{1.879355in}}%
\pgfpathlineto{\pgfqpoint{1.123568in}{1.891406in}}%
\pgfpathlineto{\pgfqpoint{1.192204in}{1.899127in}}%
\pgfpathlineto{\pgfqpoint{1.291806in}{1.906111in}}%
\pgfpathlineto{\pgfqpoint{1.418661in}{1.910655in}}%
\pgfpathlineto{\pgfqpoint{1.568331in}{1.912107in}}%
\pgfpathlineto{\pgfqpoint{1.735927in}{1.910011in}}%
\pgfpathlineto{\pgfqpoint{1.891425in}{1.904755in}}%
\pgfpathlineto{\pgfqpoint{2.035702in}{1.896701in}}%
\pgfpathlineto{\pgfqpoint{2.169646in}{1.885950in}}%
\pgfpathlineto{\pgfqpoint{2.294129in}{1.872502in}}%
\pgfpathlineto{\pgfqpoint{2.391121in}{1.859198in}}%
\pgfpathlineto{\pgfqpoint{2.482260in}{1.843940in}}%
\pgfpathlineto{\pgfqpoint{2.567776in}{1.826667in}}%
\pgfpathlineto{\pgfqpoint{2.647929in}{1.807295in}}%
\pgfpathlineto{\pgfqpoint{2.722995in}{1.785733in}}%
\pgfpathlineto{\pgfqpoint{2.793221in}{1.761861in}}%
\pgfpathlineto{\pgfqpoint{2.846058in}{1.741004in}}%
\pgfpathlineto{\pgfqpoint{2.896038in}{1.718490in}}%
\pgfpathlineto{\pgfqpoint{2.943273in}{1.694213in}}%
\pgfpathlineto{\pgfqpoint{2.987880in}{1.668053in}}%
\pgfpathlineto{\pgfqpoint{3.029967in}{1.639867in}}%
\pgfpathlineto{\pgfqpoint{3.069618in}{1.609487in}}%
\pgfpathlineto{\pgfqpoint{3.106895in}{1.576710in}}%
\pgfpathlineto{\pgfqpoint{3.141850in}{1.541295in}}%
\pgfpathlineto{\pgfqpoint{3.174533in}{1.502951in}}%
\pgfpathlineto{\pgfqpoint{3.204996in}{1.461321in}}%
\pgfpathlineto{\pgfqpoint{3.233290in}{1.415966in}}%
\pgfpathlineto{\pgfqpoint{3.259450in}{1.366334in}}%
\pgfpathlineto{\pgfqpoint{3.277684in}{1.325887in}}%
\pgfpathlineto{\pgfqpoint{3.294737in}{1.282270in}}%
\pgfpathlineto{\pgfqpoint{3.310620in}{1.235024in}}%
\pgfpathlineto{\pgfqpoint{3.325349in}{1.183572in}}%
\pgfpathlineto{\pgfqpoint{3.338946in}{1.127178in}}%
\pgfpathlineto{\pgfqpoint{3.351435in}{1.064877in}}%
\pgfpathlineto{\pgfqpoint{3.362836in}{0.995375in}}%
\pgfpathlineto{\pgfqpoint{3.373156in}{0.916874in}}%
\pgfpathlineto{\pgfqpoint{3.382390in}{0.826773in}}%
\pgfpathlineto{\pgfqpoint{3.387937in}{0.758388in}}%
\pgfpathlineto{\pgfqpoint{3.387937in}{0.758388in}}%
\pgfusepath{stroke}%
\end{pgfscope}%
\begin{pgfscope}%
\pgfpathrectangle{\pgfqpoint{0.675193in}{0.526079in}}{\pgfqpoint{4.650000in}{3.020000in}}%
\pgfusepath{clip}%
\pgfsetrectcap%
\pgfsetroundjoin%
\pgfsetlinewidth{1.505625pt}%
\definecolor{currentstroke}{rgb}{0.000000,0.750000,0.750000}%
\pgfsetstrokecolor{currentstroke}%
\pgfsetstrokeopacity{0.750000}%
\pgfsetdash{}{0pt}%
\pgfpathmoveto{\pgfqpoint{0.943739in}{2.464167in}}%
\pgfpathlineto{\pgfqpoint{0.980675in}{2.376812in}}%
\pgfpathlineto{\pgfqpoint{1.017119in}{2.299901in}}%
\pgfpathlineto{\pgfqpoint{1.053078in}{2.229584in}}%
\pgfpathlineto{\pgfqpoint{1.088558in}{2.165074in}}%
\pgfpathlineto{\pgfqpoint{1.123568in}{2.105678in}}%
\pgfpathlineto{\pgfqpoint{1.158114in}{2.050812in}}%
\pgfpathlineto{\pgfqpoint{1.192204in}{1.999973in}}%
\pgfpathlineto{\pgfqpoint{1.259043in}{1.908706in}}%
\pgfpathlineto{\pgfqpoint{1.324142in}{1.829086in}}%
\pgfpathlineto{\pgfqpoint{1.387562in}{1.759017in}}%
\pgfpathlineto{\pgfqpoint{1.449363in}{1.696883in}}%
\pgfpathlineto{\pgfqpoint{1.509602in}{1.641425in}}%
\pgfpathlineto{\pgfqpoint{1.568331in}{1.591646in}}%
\pgfpathlineto{\pgfqpoint{1.625600in}{1.546740in}}%
\pgfpathlineto{\pgfqpoint{1.681452in}{1.506050in}}%
\pgfpathlineto{\pgfqpoint{1.762658in}{1.451760in}}%
\pgfpathlineto{\pgfqpoint{1.840881in}{1.404297in}}%
\pgfpathlineto{\pgfqpoint{1.916227in}{1.362523in}}%
\pgfpathlineto{\pgfqpoint{1.988803in}{1.325542in}}%
\pgfpathlineto{\pgfqpoint{2.081453in}{1.282471in}}%
\pgfpathlineto{\pgfqpoint{2.169646in}{1.245329in}}%
\pgfpathlineto{\pgfqpoint{2.253639in}{1.213086in}}%
\pgfpathlineto{\pgfqpoint{2.353039in}{1.178461in}}%
\pgfpathlineto{\pgfqpoint{2.446492in}{1.149026in}}%
\pgfpathlineto{\pgfqpoint{2.551111in}{1.119265in}}%
\pgfpathlineto{\pgfqpoint{2.663341in}{1.090694in}}%
\pgfpathlineto{\pgfqpoint{2.779552in}{1.064448in}}%
\pgfpathlineto{\pgfqpoint{2.896038in}{1.041278in}}%
\pgfpathlineto{\pgfqpoint{3.019676in}{1.019959in}}%
\pgfpathlineto{\pgfqpoint{3.141850in}{1.002245in}}%
\pgfpathlineto{\pgfqpoint{3.253109in}{0.989271in}}%
\pgfpathlineto{\pgfqpoint{3.343231in}{0.981744in}}%
\pgfpathlineto{\pgfqpoint{3.387937in}{0.979978in}}%
\pgfpathlineto{\pgfqpoint{3.387937in}{0.979978in}}%
\pgfusepath{stroke}%
\end{pgfscope}%
\begin{pgfscope}%
\pgfpathrectangle{\pgfqpoint{0.675193in}{0.526079in}}{\pgfqpoint{4.650000in}{3.020000in}}%
\pgfusepath{clip}%
\pgfsetrectcap%
\pgfsetroundjoin%
\pgfsetlinewidth{1.505625pt}%
\definecolor{currentstroke}{rgb}{1.000000,0.000000,0.000000}%
\pgfsetstrokecolor{currentstroke}%
\pgfsetstrokeopacity{0.750000}%
\pgfsetdash{}{0pt}%
\pgfpathmoveto{\pgfqpoint{1.220940in}{3.326916in}}%
\pgfpathlineto{\pgfqpoint{1.350641in}{3.341971in}}%
\pgfpathlineto{\pgfqpoint{1.475918in}{3.351628in}}%
\pgfpathlineto{\pgfqpoint{1.636036in}{3.359560in}}%
\pgfpathlineto{\pgfqpoint{1.824922in}{3.365419in}}%
\pgfpathlineto{\pgfqpoint{2.101271in}{3.370481in}}%
\pgfpathlineto{\pgfqpoint{2.520930in}{3.374478in}}%
\pgfpathlineto{\pgfqpoint{3.572486in}{3.379817in}}%
\pgfpathlineto{\pgfqpoint{4.244009in}{3.385221in}}%
\pgfpathlineto{\pgfqpoint{4.742371in}{3.392243in}}%
\pgfpathlineto{\pgfqpoint{4.911127in}{3.395497in}}%
\pgfpathlineto{\pgfqpoint{4.911127in}{3.395497in}}%
\pgfusepath{stroke}%
\end{pgfscope}%
\begin{pgfscope}%
\pgfpathrectangle{\pgfqpoint{0.675193in}{0.526079in}}{\pgfqpoint{4.650000in}{3.020000in}}%
\pgfusepath{clip}%
\pgfsetrectcap%
\pgfsetroundjoin%
\pgfsetlinewidth{1.505625pt}%
\definecolor{currentstroke}{rgb}{0.000000,0.000000,1.000000}%
\pgfsetstrokecolor{currentstroke}%
\pgfsetstrokeopacity{0.750000}%
\pgfsetdash{}{0pt}%
\pgfpathmoveto{\pgfqpoint{1.220940in}{1.570241in}}%
\pgfpathlineto{\pgfqpoint{1.350641in}{1.588461in}}%
\pgfpathlineto{\pgfqpoint{1.475918in}{1.602061in}}%
\pgfpathlineto{\pgfqpoint{1.636036in}{1.615832in}}%
\pgfpathlineto{\pgfqpoint{1.861176in}{1.631534in}}%
\pgfpathlineto{\pgfqpoint{2.133731in}{1.647119in}}%
\pgfpathlineto{\pgfqpoint{2.379087in}{1.658289in}}%
\pgfpathlineto{\pgfqpoint{2.602242in}{1.665647in}}%
\pgfpathlineto{\pgfqpoint{2.806534in}{1.669463in}}%
\pgfpathlineto{\pgfqpoint{2.994763in}{1.669925in}}%
\pgfpathlineto{\pgfqpoint{3.169252in}{1.667185in}}%
\pgfpathlineto{\pgfqpoint{3.331694in}{1.661369in}}%
\pgfpathlineto{\pgfqpoint{3.483097in}{1.652575in}}%
\pgfpathlineto{\pgfqpoint{3.607180in}{1.642492in}}%
\pgfpathlineto{\pgfqpoint{3.724111in}{1.630207in}}%
\pgfpathlineto{\pgfqpoint{3.834420in}{1.615728in}}%
\pgfpathlineto{\pgfqpoint{3.938531in}{1.599038in}}%
\pgfpathlineto{\pgfqpoint{4.037009in}{1.580103in}}%
\pgfpathlineto{\pgfqpoint{4.130507in}{1.558864in}}%
\pgfpathlineto{\pgfqpoint{4.219430in}{1.535237in}}%
\pgfpathlineto{\pgfqpoint{4.292060in}{1.513000in}}%
\pgfpathlineto{\pgfqpoint{4.361377in}{1.488835in}}%
\pgfpathlineto{\pgfqpoint{4.427512in}{1.462634in}}%
\pgfpathlineto{\pgfqpoint{4.490710in}{1.434260in}}%
\pgfpathlineto{\pgfqpoint{4.551186in}{1.403554in}}%
\pgfpathlineto{\pgfqpoint{4.609035in}{1.370316in}}%
\pgfpathlineto{\pgfqpoint{4.655255in}{1.340518in}}%
\pgfpathlineto{\pgfqpoint{4.699688in}{1.308633in}}%
\pgfpathlineto{\pgfqpoint{4.742371in}{1.274468in}}%
\pgfpathlineto{\pgfqpoint{4.783363in}{1.237788in}}%
\pgfpathlineto{\pgfqpoint{4.822752in}{1.198315in}}%
\pgfpathlineto{\pgfqpoint{4.860616in}{1.155705in}}%
\pgfpathlineto{\pgfqpoint{4.896993in}{1.109541in}}%
\pgfpathlineto{\pgfqpoint{4.911127in}{1.089966in}}%
\pgfpathlineto{\pgfqpoint{4.911127in}{1.089966in}}%
\pgfusepath{stroke}%
\end{pgfscope}%
\begin{pgfscope}%
\pgfpathrectangle{\pgfqpoint{0.675193in}{0.526079in}}{\pgfqpoint{4.650000in}{3.020000in}}%
\pgfusepath{clip}%
\pgfsetrectcap%
\pgfsetroundjoin%
\pgfsetlinewidth{1.505625pt}%
\definecolor{currentstroke}{rgb}{0.000000,0.750000,0.750000}%
\pgfsetstrokecolor{currentstroke}%
\pgfsetstrokeopacity{0.750000}%
\pgfsetdash{}{0pt}%
\pgfpathmoveto{\pgfqpoint{1.220940in}{2.211949in}}%
\pgfpathlineto{\pgfqpoint{1.264664in}{2.157737in}}%
\pgfpathlineto{\pgfqpoint{1.307898in}{2.108407in}}%
\pgfpathlineto{\pgfqpoint{1.350641in}{2.062642in}}%
\pgfpathlineto{\pgfqpoint{1.392893in}{2.020085in}}%
\pgfpathlineto{\pgfqpoint{1.434652in}{1.980418in}}%
\pgfpathlineto{\pgfqpoint{1.516690in}{1.908663in}}%
\pgfpathlineto{\pgfqpoint{1.596750in}{1.845527in}}%
\pgfpathlineto{\pgfqpoint{1.674825in}{1.789587in}}%
\pgfpathlineto{\pgfqpoint{1.750894in}{1.739711in}}%
\pgfpathlineto{\pgfqpoint{1.824922in}{1.694997in}}%
\pgfpathlineto{\pgfqpoint{1.896926in}{1.654713in}}%
\pgfpathlineto{\pgfqpoint{1.966942in}{1.618263in}}%
\pgfpathlineto{\pgfqpoint{2.068375in}{1.569711in}}%
\pgfpathlineto{\pgfqpoint{2.165767in}{1.527337in}}%
\pgfpathlineto{\pgfqpoint{2.259447in}{1.490096in}}%
\pgfpathlineto{\pgfqpoint{2.349713in}{1.457162in}}%
\pgfpathlineto{\pgfqpoint{2.465171in}{1.418828in}}%
\pgfpathlineto{\pgfqpoint{2.575441in}{1.385762in}}%
\pgfpathlineto{\pgfqpoint{2.680903in}{1.357021in}}%
\pgfpathlineto{\pgfqpoint{2.806534in}{1.326069in}}%
\pgfpathlineto{\pgfqpoint{2.925900in}{1.299629in}}%
\pgfpathlineto{\pgfqpoint{3.061699in}{1.272686in}}%
\pgfpathlineto{\pgfqpoint{3.210943in}{1.246500in}}%
\pgfpathlineto{\pgfqpoint{3.370549in}{1.221962in}}%
\pgfpathlineto{\pgfqpoint{3.537193in}{1.199639in}}%
\pgfpathlineto{\pgfqpoint{3.707822in}{1.179820in}}%
\pgfpathlineto{\pgfqpoint{3.894639in}{1.161238in}}%
\pgfpathlineto{\pgfqpoint{4.091011in}{1.144845in}}%
\pgfpathlineto{\pgfqpoint{4.292060in}{1.131169in}}%
\pgfpathlineto{\pgfqpoint{4.511164in}{1.119461in}}%
\pgfpathlineto{\pgfqpoint{4.725505in}{1.111165in}}%
\pgfpathlineto{\pgfqpoint{4.911127in}{1.106607in}}%
\pgfpathlineto{\pgfqpoint{4.911127in}{1.106607in}}%
\pgfusepath{stroke}%
\end{pgfscope}%
\begin{pgfscope}%
\pgfpathrectangle{\pgfqpoint{0.675193in}{0.526079in}}{\pgfqpoint{4.650000in}{3.020000in}}%
\pgfusepath{clip}%
\pgfsetrectcap%
\pgfsetroundjoin%
\pgfsetlinewidth{1.505625pt}%
\definecolor{currentstroke}{rgb}{1.000000,0.000000,0.000000}%
\pgfsetstrokecolor{currentstroke}%
\pgfsetstrokeopacity{0.750000}%
\pgfsetdash{}{0pt}%
\pgfpathmoveto{\pgfqpoint{1.005049in}{3.282734in}}%
\pgfpathlineto{\pgfqpoint{1.052517in}{3.294674in}}%
\pgfpathlineto{\pgfqpoint{1.099547in}{3.305606in}}%
\pgfpathlineto{\pgfqpoint{1.146148in}{3.313898in}}%
\pgfpathlineto{\pgfqpoint{1.192328in}{3.320315in}}%
\pgfpathlineto{\pgfqpoint{1.238095in}{3.325357in}}%
\pgfpathlineto{\pgfqpoint{1.283458in}{3.329364in}}%
\pgfpathlineto{\pgfqpoint{1.328425in}{3.332583in}}%
\pgfpathlineto{\pgfqpoint{1.373005in}{3.335186in}}%
\pgfpathlineto{\pgfqpoint{1.417205in}{3.337300in}}%
\pgfpathlineto{\pgfqpoint{1.461033in}{3.339021in}}%
\pgfpathlineto{\pgfqpoint{1.504500in}{3.340421in}}%
\pgfpathlineto{\pgfqpoint{1.547611in}{3.341558in}}%
\pgfpathlineto{\pgfqpoint{1.590380in}{3.342476in}}%
\pgfpathlineto{\pgfqpoint{1.632814in}{3.343209in}}%
\pgfpathlineto{\pgfqpoint{1.674921in}{3.343785in}}%
\pgfpathlineto{\pgfqpoint{1.716707in}{3.344227in}}%
\pgfpathlineto{\pgfqpoint{1.758181in}{3.344553in}}%
\pgfpathlineto{\pgfqpoint{1.799349in}{3.344777in}}%
\pgfpathlineto{\pgfqpoint{1.840218in}{3.344914in}}%
\pgfpathlineto{\pgfqpoint{1.880793in}{3.344975in}}%
\pgfpathlineto{\pgfqpoint{1.921080in}{3.344968in}}%
\pgfpathlineto{\pgfqpoint{1.961085in}{3.344900in}}%
\pgfpathlineto{\pgfqpoint{2.000814in}{3.344778in}}%
\pgfpathlineto{\pgfqpoint{2.040271in}{3.344607in}}%
\pgfpathlineto{\pgfqpoint{2.079463in}{3.344393in}}%
\pgfpathlineto{\pgfqpoint{2.118395in}{3.344139in}}%
\pgfpathlineto{\pgfqpoint{2.157075in}{3.343850in}}%
\pgfpathlineto{\pgfqpoint{2.195507in}{3.343527in}}%
\pgfpathlineto{\pgfqpoint{2.233698in}{3.343175in}}%
\pgfpathlineto{\pgfqpoint{2.271655in}{3.342794in}}%
\pgfpathlineto{\pgfqpoint{2.309385in}{3.342391in}}%
\pgfpathlineto{\pgfqpoint{2.346894in}{3.341964in}}%
\pgfpathlineto{\pgfqpoint{2.384188in}{3.341514in}}%
\pgfpathlineto{\pgfqpoint{2.421275in}{3.341046in}}%
\pgfpathlineto{\pgfqpoint{2.458159in}{3.340560in}}%
\pgfpathlineto{\pgfqpoint{2.494848in}{3.340055in}}%
\pgfpathlineto{\pgfqpoint{2.531347in}{3.339535in}}%
\pgfpathlineto{\pgfqpoint{2.567662in}{3.339001in}}%
\pgfpathlineto{\pgfqpoint{2.603796in}{3.338452in}}%
\pgfpathlineto{\pgfqpoint{2.639756in}{3.337890in}}%
\pgfpathlineto{\pgfqpoint{2.675544in}{3.337317in}}%
\pgfpathlineto{\pgfqpoint{2.711164in}{3.336731in}}%
\pgfpathlineto{\pgfqpoint{2.746620in}{3.336135in}}%
\pgfpathlineto{\pgfqpoint{2.781914in}{3.335529in}}%
\pgfpathlineto{\pgfqpoint{2.817049in}{3.334913in}}%
\pgfpathlineto{\pgfqpoint{2.852028in}{3.334288in}}%
\pgfpathlineto{\pgfqpoint{2.886853in}{3.333655in}}%
\pgfpathlineto{\pgfqpoint{2.921526in}{3.333014in}}%
\pgfpathlineto{\pgfqpoint{2.956050in}{3.332366in}}%
\pgfpathlineto{\pgfqpoint{2.990428in}{3.331711in}}%
\pgfpathlineto{\pgfqpoint{3.024662in}{3.331049in}}%
\pgfpathlineto{\pgfqpoint{3.058754in}{3.330378in}}%
\pgfpathlineto{\pgfqpoint{3.092707in}{3.329700in}}%
\pgfpathlineto{\pgfqpoint{3.126524in}{3.329018in}}%
\pgfpathlineto{\pgfqpoint{3.160209in}{3.328331in}}%
\pgfpathlineto{\pgfqpoint{3.193764in}{3.327638in}}%
\pgfpathlineto{\pgfqpoint{3.227192in}{3.326940in}}%
\pgfpathlineto{\pgfqpoint{3.260498in}{3.326237in}}%
\pgfpathlineto{\pgfqpoint{3.293683in}{3.325530in}}%
\pgfpathlineto{\pgfqpoint{3.326753in}{3.324819in}}%
\pgfpathlineto{\pgfqpoint{3.359709in}{3.324103in}}%
\pgfpathlineto{\pgfqpoint{3.392556in}{3.323383in}}%
\pgfpathlineto{\pgfqpoint{3.425296in}{3.322660in}}%
\pgfpathlineto{\pgfqpoint{3.457932in}{3.321933in}}%
\pgfpathlineto{\pgfqpoint{3.490468in}{3.321203in}}%
\pgfpathlineto{\pgfqpoint{3.522906in}{3.320469in}}%
\pgfpathlineto{\pgfqpoint{3.555248in}{3.319732in}}%
\pgfpathlineto{\pgfqpoint{3.587498in}{3.318993in}}%
\pgfpathlineto{\pgfqpoint{3.619656in}{3.318250in}}%
\pgfpathlineto{\pgfqpoint{3.651726in}{3.317505in}}%
\pgfpathlineto{\pgfqpoint{3.683709in}{3.316757in}}%
\pgfpathlineto{\pgfqpoint{3.715606in}{3.316007in}}%
\pgfpathlineto{\pgfqpoint{3.747421in}{3.315255in}}%
\pgfpathlineto{\pgfqpoint{3.779153in}{3.314501in}}%
\pgfpathlineto{\pgfqpoint{3.810805in}{3.313744in}}%
\pgfpathlineto{\pgfqpoint{3.842378in}{3.312985in}}%
\pgfpathlineto{\pgfqpoint{3.873873in}{3.312223in}}%
\pgfpathlineto{\pgfqpoint{3.905292in}{3.311460in}}%
\pgfpathlineto{\pgfqpoint{3.936636in}{3.310695in}}%
\pgfpathlineto{\pgfqpoint{3.967907in}{3.309928in}}%
\pgfpathlineto{\pgfqpoint{3.999105in}{3.309160in}}%
\pgfpathlineto{\pgfqpoint{4.030233in}{3.308390in}}%
\pgfpathlineto{\pgfqpoint{4.061290in}{3.307620in}}%
\pgfpathlineto{\pgfqpoint{4.092280in}{3.306849in}}%
\pgfpathlineto{\pgfqpoint{4.123202in}{3.306077in}}%
\pgfpathlineto{\pgfqpoint{4.154058in}{3.305303in}}%
\pgfpathlineto{\pgfqpoint{4.184849in}{3.304527in}}%
\pgfpathlineto{\pgfqpoint{4.215576in}{3.303750in}}%
\pgfpathlineto{\pgfqpoint{4.246240in}{3.302973in}}%
\pgfpathlineto{\pgfqpoint{4.276843in}{3.302194in}}%
\pgfpathlineto{\pgfqpoint{4.307384in}{3.301416in}}%
\pgfpathlineto{\pgfqpoint{4.337864in}{3.300637in}}%
\pgfpathlineto{\pgfqpoint{4.368286in}{3.299856in}}%
\pgfpathlineto{\pgfqpoint{4.398648in}{3.299074in}}%
\pgfpathlineto{\pgfqpoint{4.428952in}{3.298292in}}%
\pgfpathlineto{\pgfqpoint{4.459198in}{3.297509in}}%
\pgfpathlineto{\pgfqpoint{4.489388in}{3.296726in}}%
\pgfpathlineto{\pgfqpoint{4.519521in}{3.295943in}}%
\pgfpathlineto{\pgfqpoint{4.549598in}{3.295159in}}%
\pgfpathlineto{\pgfqpoint{4.579619in}{3.294374in}}%
\pgfpathlineto{\pgfqpoint{4.609587in}{3.293589in}}%
\pgfpathlineto{\pgfqpoint{4.639500in}{3.292803in}}%
\pgfpathlineto{\pgfqpoint{4.669360in}{3.292018in}}%
\pgfpathlineto{\pgfqpoint{4.699168in}{3.291232in}}%
\pgfpathlineto{\pgfqpoint{4.728924in}{3.290446in}}%
\pgfpathlineto{\pgfqpoint{4.758629in}{3.289660in}}%
\pgfpathlineto{\pgfqpoint{4.788284in}{3.288873in}}%
\pgfpathlineto{\pgfqpoint{4.817890in}{3.288087in}}%
\pgfpathlineto{\pgfqpoint{4.847448in}{3.287300in}}%
\pgfpathlineto{\pgfqpoint{4.876959in}{3.286513in}}%
\pgfpathlineto{\pgfqpoint{4.906424in}{3.285727in}}%
\pgfpathlineto{\pgfqpoint{4.935843in}{3.284940in}}%
\pgfpathlineto{\pgfqpoint{4.965218in}{3.284154in}}%
\pgfpathlineto{\pgfqpoint{4.994548in}{3.283367in}}%
\pgfpathlineto{\pgfqpoint{5.023836in}{3.282581in}}%
\pgfpathlineto{\pgfqpoint{5.053081in}{3.281795in}}%
\pgfpathlineto{\pgfqpoint{5.082285in}{3.281008in}}%
\pgfpathlineto{\pgfqpoint{5.111447in}{3.280222in}}%
\pgfusepath{stroke}%
\end{pgfscope}%
\begin{pgfscope}%
\pgfpathrectangle{\pgfqpoint{0.675193in}{0.526079in}}{\pgfqpoint{4.650000in}{3.020000in}}%
\pgfusepath{clip}%
\pgfsetrectcap%
\pgfsetroundjoin%
\pgfsetlinewidth{1.505625pt}%
\definecolor{currentstroke}{rgb}{0.000000,0.000000,1.000000}%
\pgfsetstrokecolor{currentstroke}%
\pgfsetstrokeopacity{0.750000}%
\pgfsetdash{}{0pt}%
\pgfpathmoveto{\pgfqpoint{1.005049in}{1.903514in}}%
\pgfpathlineto{\pgfqpoint{1.052517in}{1.920295in}}%
\pgfpathlineto{\pgfqpoint{1.099547in}{1.936347in}}%
\pgfpathlineto{\pgfqpoint{1.146148in}{1.949985in}}%
\pgfpathlineto{\pgfqpoint{1.192328in}{1.961941in}}%
\pgfpathlineto{\pgfqpoint{1.238095in}{1.972687in}}%
\pgfpathlineto{\pgfqpoint{1.283458in}{1.982544in}}%
\pgfpathlineto{\pgfqpoint{1.328425in}{1.991737in}}%
\pgfpathlineto{\pgfqpoint{1.373005in}{2.000426in}}%
\pgfpathlineto{\pgfqpoint{1.417205in}{2.008724in}}%
\pgfpathlineto{\pgfqpoint{1.461033in}{2.016716in}}%
\pgfpathlineto{\pgfqpoint{1.504500in}{2.024464in}}%
\pgfpathlineto{\pgfqpoint{1.547611in}{2.032017in}}%
\pgfpathlineto{\pgfqpoint{1.590380in}{2.039412in}}%
\pgfpathlineto{\pgfqpoint{1.632814in}{2.046672in}}%
\pgfpathlineto{\pgfqpoint{1.674921in}{2.053821in}}%
\pgfpathlineto{\pgfqpoint{1.716707in}{2.060874in}}%
\pgfpathlineto{\pgfqpoint{1.758181in}{2.067845in}}%
\pgfpathlineto{\pgfqpoint{1.799349in}{2.074741in}}%
\pgfpathlineto{\pgfqpoint{1.840218in}{2.081571in}}%
\pgfpathlineto{\pgfqpoint{1.880793in}{2.088343in}}%
\pgfpathlineto{\pgfqpoint{1.921080in}{2.095060in}}%
\pgfpathlineto{\pgfqpoint{1.961085in}{2.101724in}}%
\pgfpathlineto{\pgfqpoint{2.000814in}{2.108339in}}%
\pgfpathlineto{\pgfqpoint{2.040271in}{2.114907in}}%
\pgfpathlineto{\pgfqpoint{2.079463in}{2.121429in}}%
\pgfpathlineto{\pgfqpoint{2.118395in}{2.127907in}}%
\pgfpathlineto{\pgfqpoint{2.157075in}{2.134340in}}%
\pgfpathlineto{\pgfqpoint{2.195507in}{2.140729in}}%
\pgfpathlineto{\pgfqpoint{2.233698in}{2.147075in}}%
\pgfpathlineto{\pgfqpoint{2.271655in}{2.153378in}}%
\pgfpathlineto{\pgfqpoint{2.309385in}{2.159639in}}%
\pgfpathlineto{\pgfqpoint{2.346894in}{2.165858in}}%
\pgfpathlineto{\pgfqpoint{2.384188in}{2.172032in}}%
\pgfpathlineto{\pgfqpoint{2.421275in}{2.178164in}}%
\pgfpathlineto{\pgfqpoint{2.458159in}{2.184253in}}%
\pgfpathlineto{\pgfqpoint{2.494848in}{2.190297in}}%
\pgfpathlineto{\pgfqpoint{2.531347in}{2.196299in}}%
\pgfpathlineto{\pgfqpoint{2.567662in}{2.202257in}}%
\pgfpathlineto{\pgfqpoint{2.603796in}{2.208171in}}%
\pgfpathlineto{\pgfqpoint{2.639756in}{2.214041in}}%
\pgfpathlineto{\pgfqpoint{2.675544in}{2.219867in}}%
\pgfpathlineto{\pgfqpoint{2.711164in}{2.225648in}}%
\pgfpathlineto{\pgfqpoint{2.746620in}{2.231386in}}%
\pgfpathlineto{\pgfqpoint{2.781914in}{2.237079in}}%
\pgfpathlineto{\pgfqpoint{2.817049in}{2.242728in}}%
\pgfpathlineto{\pgfqpoint{2.852028in}{2.248333in}}%
\pgfpathlineto{\pgfqpoint{2.886853in}{2.253893in}}%
\pgfpathlineto{\pgfqpoint{2.921526in}{2.259410in}}%
\pgfpathlineto{\pgfqpoint{2.956050in}{2.264883in}}%
\pgfpathlineto{\pgfqpoint{2.990428in}{2.270312in}}%
\pgfpathlineto{\pgfqpoint{3.024662in}{2.275697in}}%
\pgfpathlineto{\pgfqpoint{3.058754in}{2.281036in}}%
\pgfpathlineto{\pgfqpoint{3.092707in}{2.286331in}}%
\pgfpathlineto{\pgfqpoint{3.126524in}{2.291583in}}%
\pgfpathlineto{\pgfqpoint{3.160209in}{2.296793in}}%
\pgfpathlineto{\pgfqpoint{3.193764in}{2.301959in}}%
\pgfpathlineto{\pgfqpoint{3.227192in}{2.307082in}}%
\pgfpathlineto{\pgfqpoint{3.260498in}{2.312162in}}%
\pgfpathlineto{\pgfqpoint{3.293683in}{2.317200in}}%
\pgfpathlineto{\pgfqpoint{3.326753in}{2.322196in}}%
\pgfpathlineto{\pgfqpoint{3.359709in}{2.327150in}}%
\pgfpathlineto{\pgfqpoint{3.392556in}{2.332062in}}%
\pgfpathlineto{\pgfqpoint{3.425296in}{2.336933in}}%
\pgfpathlineto{\pgfqpoint{3.457932in}{2.341763in}}%
\pgfpathlineto{\pgfqpoint{3.490468in}{2.346552in}}%
\pgfpathlineto{\pgfqpoint{3.522906in}{2.351300in}}%
\pgfpathlineto{\pgfqpoint{3.555248in}{2.356008in}}%
\pgfpathlineto{\pgfqpoint{3.587498in}{2.360676in}}%
\pgfpathlineto{\pgfqpoint{3.619656in}{2.365305in}}%
\pgfpathlineto{\pgfqpoint{3.651726in}{2.369894in}}%
\pgfpathlineto{\pgfqpoint{3.683709in}{2.374444in}}%
\pgfpathlineto{\pgfqpoint{3.715606in}{2.378957in}}%
\pgfpathlineto{\pgfqpoint{3.747421in}{2.383431in}}%
\pgfpathlineto{\pgfqpoint{3.779153in}{2.387867in}}%
\pgfpathlineto{\pgfqpoint{3.810805in}{2.392265in}}%
\pgfpathlineto{\pgfqpoint{3.842378in}{2.396626in}}%
\pgfpathlineto{\pgfqpoint{3.873873in}{2.400949in}}%
\pgfpathlineto{\pgfqpoint{3.905292in}{2.405235in}}%
\pgfpathlineto{\pgfqpoint{3.936636in}{2.409486in}}%
\pgfpathlineto{\pgfqpoint{3.967907in}{2.413700in}}%
\pgfpathlineto{\pgfqpoint{3.999105in}{2.417880in}}%
\pgfpathlineto{\pgfqpoint{4.030233in}{2.422024in}}%
\pgfpathlineto{\pgfqpoint{4.061290in}{2.426134in}}%
\pgfpathlineto{\pgfqpoint{4.092280in}{2.430211in}}%
\pgfpathlineto{\pgfqpoint{4.123202in}{2.434253in}}%
\pgfpathlineto{\pgfqpoint{4.154058in}{2.438260in}}%
\pgfpathlineto{\pgfqpoint{4.184849in}{2.442234in}}%
\pgfpathlineto{\pgfqpoint{4.215576in}{2.446175in}}%
\pgfpathlineto{\pgfqpoint{4.246240in}{2.450083in}}%
\pgfpathlineto{\pgfqpoint{4.276843in}{2.453959in}}%
\pgfpathlineto{\pgfqpoint{4.307384in}{2.457803in}}%
\pgfpathlineto{\pgfqpoint{4.337864in}{2.461616in}}%
\pgfpathlineto{\pgfqpoint{4.368286in}{2.465396in}}%
\pgfpathlineto{\pgfqpoint{4.398648in}{2.469146in}}%
\pgfpathlineto{\pgfqpoint{4.428952in}{2.472864in}}%
\pgfpathlineto{\pgfqpoint{4.459198in}{2.476553in}}%
\pgfpathlineto{\pgfqpoint{4.489388in}{2.480211in}}%
\pgfpathlineto{\pgfqpoint{4.519521in}{2.483839in}}%
\pgfpathlineto{\pgfqpoint{4.549598in}{2.487438in}}%
\pgfpathlineto{\pgfqpoint{4.579619in}{2.491008in}}%
\pgfpathlineto{\pgfqpoint{4.609587in}{2.494548in}}%
\pgfpathlineto{\pgfqpoint{4.639500in}{2.498061in}}%
\pgfpathlineto{\pgfqpoint{4.669360in}{2.501544in}}%
\pgfpathlineto{\pgfqpoint{4.699168in}{2.505000in}}%
\pgfpathlineto{\pgfqpoint{4.728924in}{2.508429in}}%
\pgfpathlineto{\pgfqpoint{4.758629in}{2.511830in}}%
\pgfpathlineto{\pgfqpoint{4.788284in}{2.515204in}}%
\pgfpathlineto{\pgfqpoint{4.817890in}{2.518550in}}%
\pgfpathlineto{\pgfqpoint{4.847448in}{2.521871in}}%
\pgfpathlineto{\pgfqpoint{4.876959in}{2.525165in}}%
\pgfpathlineto{\pgfqpoint{4.906424in}{2.528433in}}%
\pgfpathlineto{\pgfqpoint{4.935843in}{2.531676in}}%
\pgfpathlineto{\pgfqpoint{4.965218in}{2.534893in}}%
\pgfpathlineto{\pgfqpoint{4.994548in}{2.538086in}}%
\pgfpathlineto{\pgfqpoint{5.023836in}{2.541253in}}%
\pgfpathlineto{\pgfqpoint{5.053081in}{2.544395in}}%
\pgfpathlineto{\pgfqpoint{5.082285in}{2.547513in}}%
\pgfpathlineto{\pgfqpoint{5.111447in}{2.550608in}}%
\pgfusepath{stroke}%
\end{pgfscope}%
\begin{pgfscope}%
\pgfpathrectangle{\pgfqpoint{0.675193in}{0.526079in}}{\pgfqpoint{4.650000in}{3.020000in}}%
\pgfusepath{clip}%
\pgfsetrectcap%
\pgfsetroundjoin%
\pgfsetlinewidth{1.505625pt}%
\definecolor{currentstroke}{rgb}{0.000000,0.750000,0.750000}%
\pgfsetstrokecolor{currentstroke}%
\pgfsetstrokeopacity{0.750000}%
\pgfsetdash{}{0pt}%
\pgfpathmoveto{\pgfqpoint{1.005049in}{2.380485in}}%
\pgfpathlineto{\pgfqpoint{1.052517in}{2.286113in}}%
\pgfpathlineto{\pgfqpoint{1.099547in}{2.203202in}}%
\pgfpathlineto{\pgfqpoint{1.146148in}{2.127793in}}%
\pgfpathlineto{\pgfqpoint{1.192328in}{2.058923in}}%
\pgfpathlineto{\pgfqpoint{1.238095in}{1.995763in}}%
\pgfpathlineto{\pgfqpoint{1.283458in}{1.937618in}}%
\pgfpathlineto{\pgfqpoint{1.328425in}{1.883906in}}%
\pgfpathlineto{\pgfqpoint{1.373005in}{1.834129in}}%
\pgfpathlineto{\pgfqpoint{1.417205in}{1.787862in}}%
\pgfpathlineto{\pgfqpoint{1.461033in}{1.744740in}}%
\pgfpathlineto{\pgfqpoint{1.504500in}{1.704450in}}%
\pgfpathlineto{\pgfqpoint{1.547611in}{1.666721in}}%
\pgfpathlineto{\pgfqpoint{1.590380in}{1.631316in}}%
\pgfpathlineto{\pgfqpoint{1.632814in}{1.598025in}}%
\pgfpathlineto{\pgfqpoint{1.674921in}{1.566665in}}%
\pgfpathlineto{\pgfqpoint{1.716707in}{1.537074in}}%
\pgfpathlineto{\pgfqpoint{1.758181in}{1.509108in}}%
\pgfpathlineto{\pgfqpoint{1.799349in}{1.482637in}}%
\pgfpathlineto{\pgfqpoint{1.840218in}{1.457547in}}%
\pgfpathlineto{\pgfqpoint{1.880793in}{1.433736in}}%
\pgfpathlineto{\pgfqpoint{1.921080in}{1.411108in}}%
\pgfpathlineto{\pgfqpoint{1.961085in}{1.389581in}}%
\pgfpathlineto{\pgfqpoint{2.000814in}{1.369076in}}%
\pgfpathlineto{\pgfqpoint{2.040271in}{1.349524in}}%
\pgfpathlineto{\pgfqpoint{2.079463in}{1.330863in}}%
\pgfpathlineto{\pgfqpoint{2.118395in}{1.313033in}}%
\pgfpathlineto{\pgfqpoint{2.157075in}{1.295983in}}%
\pgfpathlineto{\pgfqpoint{2.195507in}{1.279662in}}%
\pgfpathlineto{\pgfqpoint{2.233698in}{1.264026in}}%
\pgfpathlineto{\pgfqpoint{2.271655in}{1.249033in}}%
\pgfpathlineto{\pgfqpoint{2.309385in}{1.234649in}}%
\pgfpathlineto{\pgfqpoint{2.346894in}{1.220835in}}%
\pgfpathlineto{\pgfqpoint{2.384188in}{1.207558in}}%
\pgfpathlineto{\pgfqpoint{2.421275in}{1.194790in}}%
\pgfpathlineto{\pgfqpoint{2.458159in}{1.182501in}}%
\pgfpathlineto{\pgfqpoint{2.494848in}{1.170666in}}%
\pgfpathlineto{\pgfqpoint{2.531347in}{1.159260in}}%
\pgfpathlineto{\pgfqpoint{2.567662in}{1.148262in}}%
\pgfpathlineto{\pgfqpoint{2.603796in}{1.137649in}}%
\pgfpathlineto{\pgfqpoint{2.639756in}{1.127402in}}%
\pgfpathlineto{\pgfqpoint{2.675544in}{1.117502in}}%
\pgfpathlineto{\pgfqpoint{2.711164in}{1.107932in}}%
\pgfpathlineto{\pgfqpoint{2.746620in}{1.098676in}}%
\pgfpathlineto{\pgfqpoint{2.781914in}{1.089720in}}%
\pgfpathlineto{\pgfqpoint{2.817049in}{1.081047in}}%
\pgfpathlineto{\pgfqpoint{2.852028in}{1.072645in}}%
\pgfpathlineto{\pgfqpoint{2.886853in}{1.064503in}}%
\pgfpathlineto{\pgfqpoint{2.921526in}{1.056606in}}%
\pgfpathlineto{\pgfqpoint{2.956050in}{1.048946in}}%
\pgfpathlineto{\pgfqpoint{2.990428in}{1.041511in}}%
\pgfpathlineto{\pgfqpoint{3.024662in}{1.034291in}}%
\pgfpathlineto{\pgfqpoint{3.058754in}{1.027273in}}%
\pgfpathlineto{\pgfqpoint{3.092707in}{1.020451in}}%
\pgfpathlineto{\pgfqpoint{3.126524in}{1.013819in}}%
\pgfpathlineto{\pgfqpoint{3.160209in}{1.007368in}}%
\pgfpathlineto{\pgfqpoint{3.193764in}{1.001089in}}%
\pgfpathlineto{\pgfqpoint{3.227192in}{0.994975in}}%
\pgfpathlineto{\pgfqpoint{3.260498in}{0.989019in}}%
\pgfpathlineto{\pgfqpoint{3.293683in}{0.983217in}}%
\pgfpathlineto{\pgfqpoint{3.326753in}{0.977560in}}%
\pgfpathlineto{\pgfqpoint{3.359709in}{0.972044in}}%
\pgfpathlineto{\pgfqpoint{3.392556in}{0.966663in}}%
\pgfpathlineto{\pgfqpoint{3.425296in}{0.961411in}}%
\pgfpathlineto{\pgfqpoint{3.457932in}{0.956284in}}%
\pgfpathlineto{\pgfqpoint{3.490468in}{0.951277in}}%
\pgfpathlineto{\pgfqpoint{3.522906in}{0.946386in}}%
\pgfpathlineto{\pgfqpoint{3.555248in}{0.941605in}}%
\pgfpathlineto{\pgfqpoint{3.587498in}{0.936932in}}%
\pgfpathlineto{\pgfqpoint{3.619656in}{0.932362in}}%
\pgfpathlineto{\pgfqpoint{3.651726in}{0.927891in}}%
\pgfpathlineto{\pgfqpoint{3.683709in}{0.923516in}}%
\pgfpathlineto{\pgfqpoint{3.715606in}{0.919234in}}%
\pgfpathlineto{\pgfqpoint{3.747421in}{0.915042in}}%
\pgfpathlineto{\pgfqpoint{3.779153in}{0.910935in}}%
\pgfpathlineto{\pgfqpoint{3.810805in}{0.906912in}}%
\pgfpathlineto{\pgfqpoint{3.842378in}{0.902968in}}%
\pgfpathlineto{\pgfqpoint{3.873873in}{0.899102in}}%
\pgfpathlineto{\pgfqpoint{3.905292in}{0.895310in}}%
\pgfpathlineto{\pgfqpoint{3.936636in}{0.891591in}}%
\pgfpathlineto{\pgfqpoint{3.967907in}{0.887942in}}%
\pgfpathlineto{\pgfqpoint{3.999105in}{0.884362in}}%
\pgfpathlineto{\pgfqpoint{4.030233in}{0.880847in}}%
\pgfpathlineto{\pgfqpoint{4.061290in}{0.877396in}}%
\pgfpathlineto{\pgfqpoint{4.092280in}{0.874008in}}%
\pgfpathlineto{\pgfqpoint{4.123202in}{0.870680in}}%
\pgfpathlineto{\pgfqpoint{4.154058in}{0.867408in}}%
\pgfpathlineto{\pgfqpoint{4.184849in}{0.864192in}}%
\pgfpathlineto{\pgfqpoint{4.215576in}{0.861031in}}%
\pgfpathlineto{\pgfqpoint{4.246240in}{0.857922in}}%
\pgfpathlineto{\pgfqpoint{4.276843in}{0.854865in}}%
\pgfpathlineto{\pgfqpoint{4.307384in}{0.851858in}}%
\pgfpathlineto{\pgfqpoint{4.337864in}{0.848899in}}%
\pgfpathlineto{\pgfqpoint{4.368286in}{0.845986in}}%
\pgfpathlineto{\pgfqpoint{4.398648in}{0.843118in}}%
\pgfpathlineto{\pgfqpoint{4.428952in}{0.840294in}}%
\pgfpathlineto{\pgfqpoint{4.459198in}{0.837514in}}%
\pgfpathlineto{\pgfqpoint{4.489388in}{0.834775in}}%
\pgfpathlineto{\pgfqpoint{4.519521in}{0.832076in}}%
\pgfpathlineto{\pgfqpoint{4.549598in}{0.829417in}}%
\pgfpathlineto{\pgfqpoint{4.579619in}{0.826796in}}%
\pgfpathlineto{\pgfqpoint{4.609587in}{0.824213in}}%
\pgfpathlineto{\pgfqpoint{4.639500in}{0.821665in}}%
\pgfpathlineto{\pgfqpoint{4.669360in}{0.819153in}}%
\pgfpathlineto{\pgfqpoint{4.699168in}{0.816675in}}%
\pgfpathlineto{\pgfqpoint{4.728924in}{0.814230in}}%
\pgfpathlineto{\pgfqpoint{4.758629in}{0.811818in}}%
\pgfpathlineto{\pgfqpoint{4.788284in}{0.809438in}}%
\pgfpathlineto{\pgfqpoint{4.817890in}{0.807088in}}%
\pgfpathlineto{\pgfqpoint{4.847448in}{0.804768in}}%
\pgfpathlineto{\pgfqpoint{4.876959in}{0.802478in}}%
\pgfpathlineto{\pgfqpoint{4.906424in}{0.800217in}}%
\pgfpathlineto{\pgfqpoint{4.935843in}{0.797983in}}%
\pgfpathlineto{\pgfqpoint{4.965218in}{0.795777in}}%
\pgfpathlineto{\pgfqpoint{4.994548in}{0.793597in}}%
\pgfpathlineto{\pgfqpoint{5.023836in}{0.791442in}}%
\pgfpathlineto{\pgfqpoint{5.053081in}{0.789314in}}%
\pgfpathlineto{\pgfqpoint{5.082285in}{0.787209in}}%
\pgfpathlineto{\pgfqpoint{5.111447in}{0.785129in}}%
\pgfusepath{stroke}%
\end{pgfscope}%
\begin{pgfscope}%
\pgfpathrectangle{\pgfqpoint{0.675193in}{0.526079in}}{\pgfqpoint{4.650000in}{3.020000in}}%
\pgfusepath{clip}%
\pgfsetrectcap%
\pgfsetroundjoin%
\pgfsetlinewidth{1.505625pt}%
\definecolor{currentstroke}{rgb}{1.000000,0.000000,0.000000}%
\pgfsetstrokecolor{currentstroke}%
\pgfsetstrokeopacity{0.750000}%
\pgfsetdash{}{0pt}%
\pgfpathmoveto{\pgfqpoint{1.266608in}{3.308691in}}%
\pgfpathlineto{\pgfqpoint{1.315801in}{3.311820in}}%
\pgfpathlineto{\pgfqpoint{1.364866in}{3.314782in}}%
\pgfpathlineto{\pgfqpoint{1.413797in}{3.317039in}}%
\pgfpathlineto{\pgfqpoint{1.462586in}{3.318755in}}%
\pgfpathlineto{\pgfqpoint{1.511228in}{3.320036in}}%
\pgfpathlineto{\pgfqpoint{1.559715in}{3.320965in}}%
\pgfpathlineto{\pgfqpoint{1.608040in}{3.321602in}}%
\pgfpathlineto{\pgfqpoint{1.656198in}{3.321994in}}%
\pgfpathlineto{\pgfqpoint{1.704181in}{3.322179in}}%
\pgfpathlineto{\pgfqpoint{1.751983in}{3.322185in}}%
\pgfpathlineto{\pgfqpoint{1.799596in}{3.322040in}}%
\pgfpathlineto{\pgfqpoint{1.847015in}{3.321764in}}%
\pgfpathlineto{\pgfqpoint{1.894222in}{3.321372in}}%
\pgfpathlineto{\pgfqpoint{1.941215in}{3.320877in}}%
\pgfpathlineto{\pgfqpoint{1.987986in}{3.320291in}}%
\pgfpathlineto{\pgfqpoint{2.034532in}{3.319624in}}%
\pgfpathlineto{\pgfqpoint{2.080847in}{3.318883in}}%
\pgfpathlineto{\pgfqpoint{2.126928in}{3.318075in}}%
\pgfpathlineto{\pgfqpoint{2.172772in}{3.317206in}}%
\pgfpathlineto{\pgfqpoint{2.218380in}{3.316281in}}%
\pgfpathlineto{\pgfqpoint{2.263751in}{3.315305in}}%
\pgfpathlineto{\pgfqpoint{2.308887in}{3.314284in}}%
\pgfpathlineto{\pgfqpoint{2.353790in}{3.313218in}}%
\pgfpathlineto{\pgfqpoint{2.398465in}{3.312111in}}%
\pgfpathlineto{\pgfqpoint{2.442915in}{3.310967in}}%
\pgfpathlineto{\pgfqpoint{2.487147in}{3.309787in}}%
\pgfpathlineto{\pgfqpoint{2.531166in}{3.308573in}}%
\pgfpathlineto{\pgfqpoint{2.574980in}{3.307328in}}%
\pgfpathlineto{\pgfqpoint{2.618594in}{3.306053in}}%
\pgfpathlineto{\pgfqpoint{2.662017in}{3.304751in}}%
\pgfpathlineto{\pgfqpoint{2.705255in}{3.303422in}}%
\pgfpathlineto{\pgfqpoint{2.748315in}{3.302066in}}%
\pgfpathlineto{\pgfqpoint{2.791204in}{3.300688in}}%
\pgfpathlineto{\pgfqpoint{2.833930in}{3.299287in}}%
\pgfpathlineto{\pgfqpoint{2.876500in}{3.297864in}}%
\pgfpathlineto{\pgfqpoint{2.918921in}{3.296420in}}%
\pgfpathlineto{\pgfqpoint{2.961199in}{3.294956in}}%
\pgfpathlineto{\pgfqpoint{3.003342in}{3.293474in}}%
\pgfpathlineto{\pgfqpoint{3.045355in}{3.291975in}}%
\pgfpathlineto{\pgfqpoint{3.087242in}{3.290457in}}%
\pgfpathlineto{\pgfqpoint{3.129009in}{3.288920in}}%
\pgfpathlineto{\pgfqpoint{3.170661in}{3.287366in}}%
\pgfpathlineto{\pgfqpoint{3.212203in}{3.285798in}}%
\pgfpathlineto{\pgfqpoint{3.253639in}{3.284216in}}%
\pgfpathlineto{\pgfqpoint{3.294973in}{3.282618in}}%
\pgfpathlineto{\pgfqpoint{3.336209in}{3.281007in}}%
\pgfpathlineto{\pgfqpoint{3.377348in}{3.279383in}}%
\pgfpathlineto{\pgfqpoint{3.418395in}{3.277746in}}%
\pgfpathlineto{\pgfqpoint{3.459351in}{3.276096in}}%
\pgfpathlineto{\pgfqpoint{3.500221in}{3.274434in}}%
\pgfpathlineto{\pgfqpoint{3.541007in}{3.272760in}}%
\pgfpathlineto{\pgfqpoint{3.581713in}{3.271076in}}%
\pgfpathlineto{\pgfqpoint{3.622340in}{3.269380in}}%
\pgfpathlineto{\pgfqpoint{3.662890in}{3.267674in}}%
\pgfpathlineto{\pgfqpoint{3.703364in}{3.265958in}}%
\pgfpathlineto{\pgfqpoint{3.743763in}{3.264232in}}%
\pgfpathlineto{\pgfqpoint{3.784089in}{3.262497in}}%
\pgfpathlineto{\pgfqpoint{3.824343in}{3.260753in}}%
\pgfpathlineto{\pgfqpoint{3.864526in}{3.259000in}}%
\pgfpathlineto{\pgfqpoint{3.904640in}{3.257239in}}%
\pgfpathlineto{\pgfqpoint{3.944684in}{3.255470in}}%
\pgfpathlineto{\pgfqpoint{3.984657in}{3.253692in}}%
\pgfpathlineto{\pgfqpoint{4.024559in}{3.251906in}}%
\pgfpathlineto{\pgfqpoint{4.064388in}{3.250113in}}%
\pgfpathlineto{\pgfqpoint{4.104146in}{3.248311in}}%
\pgfpathlineto{\pgfqpoint{4.143832in}{3.246503in}}%
\pgfpathlineto{\pgfqpoint{4.183446in}{3.244688in}}%
\pgfpathlineto{\pgfqpoint{4.222988in}{3.242867in}}%
\pgfpathlineto{\pgfqpoint{4.262458in}{3.241041in}}%
\pgfpathlineto{\pgfqpoint{4.301855in}{3.239209in}}%
\pgfpathlineto{\pgfqpoint{4.341179in}{3.237372in}}%
\pgfpathlineto{\pgfqpoint{4.380429in}{3.235529in}}%
\pgfpathlineto{\pgfqpoint{4.419609in}{3.233680in}}%
\pgfpathlineto{\pgfqpoint{4.458719in}{3.231825in}}%
\pgfpathlineto{\pgfqpoint{4.497761in}{3.229966in}}%
\pgfpathlineto{\pgfqpoint{4.536737in}{3.228102in}}%
\pgfpathlineto{\pgfqpoint{4.575646in}{3.226233in}}%
\pgfpathlineto{\pgfqpoint{4.614490in}{3.224362in}}%
\pgfpathlineto{\pgfqpoint{4.653269in}{3.222485in}}%
\pgfpathlineto{\pgfqpoint{4.691985in}{3.220604in}}%
\pgfpathlineto{\pgfqpoint{4.730638in}{3.218718in}}%
\pgfpathlineto{\pgfqpoint{4.769231in}{3.216829in}}%
\pgfpathlineto{\pgfqpoint{4.807764in}{3.214937in}}%
\pgfpathlineto{\pgfqpoint{4.846237in}{3.213041in}}%
\pgfpathlineto{\pgfqpoint{4.884649in}{3.211142in}}%
\pgfpathlineto{\pgfqpoint{4.923000in}{3.209240in}}%
\pgfpathlineto{\pgfqpoint{4.961289in}{3.207335in}}%
\pgfpathlineto{\pgfqpoint{4.999516in}{3.205426in}}%
\pgfpathlineto{\pgfqpoint{5.037682in}{3.203515in}}%
\pgfpathlineto{\pgfqpoint{5.075786in}{3.201602in}}%
\pgfpathlineto{\pgfqpoint{5.113829in}{3.199686in}}%
\pgfusepath{stroke}%
\end{pgfscope}%
\begin{pgfscope}%
\pgfpathrectangle{\pgfqpoint{0.675193in}{0.526079in}}{\pgfqpoint{4.650000in}{3.020000in}}%
\pgfusepath{clip}%
\pgfsetrectcap%
\pgfsetroundjoin%
\pgfsetlinewidth{1.505625pt}%
\definecolor{currentstroke}{rgb}{0.000000,0.000000,1.000000}%
\pgfsetstrokecolor{currentstroke}%
\pgfsetstrokeopacity{0.750000}%
\pgfsetdash{}{0pt}%
\pgfpathmoveto{\pgfqpoint{1.266608in}{2.141431in}}%
\pgfpathlineto{\pgfqpoint{1.315801in}{2.153201in}}%
\pgfpathlineto{\pgfqpoint{1.364866in}{2.164948in}}%
\pgfpathlineto{\pgfqpoint{1.413797in}{2.176117in}}%
\pgfpathlineto{\pgfqpoint{1.462586in}{2.186858in}}%
\pgfpathlineto{\pgfqpoint{1.511228in}{2.197264in}}%
\pgfpathlineto{\pgfqpoint{1.559715in}{2.207406in}}%
\pgfpathlineto{\pgfqpoint{1.608040in}{2.217333in}}%
\pgfpathlineto{\pgfqpoint{1.656198in}{2.227082in}}%
\pgfpathlineto{\pgfqpoint{1.704181in}{2.236681in}}%
\pgfpathlineto{\pgfqpoint{1.751983in}{2.246150in}}%
\pgfpathlineto{\pgfqpoint{1.799596in}{2.255509in}}%
\pgfpathlineto{\pgfqpoint{1.847015in}{2.264769in}}%
\pgfpathlineto{\pgfqpoint{1.894222in}{2.273940in}}%
\pgfpathlineto{\pgfqpoint{1.941215in}{2.283027in}}%
\pgfpathlineto{\pgfqpoint{1.987986in}{2.292036in}}%
\pgfpathlineto{\pgfqpoint{2.034532in}{2.300971in}}%
\pgfpathlineto{\pgfqpoint{2.080847in}{2.309834in}}%
\pgfpathlineto{\pgfqpoint{2.126928in}{2.318627in}}%
\pgfpathlineto{\pgfqpoint{2.172772in}{2.327352in}}%
\pgfpathlineto{\pgfqpoint{2.218380in}{2.336007in}}%
\pgfpathlineto{\pgfqpoint{2.263751in}{2.344595in}}%
\pgfpathlineto{\pgfqpoint{2.308887in}{2.353118in}}%
\pgfpathlineto{\pgfqpoint{2.353790in}{2.361573in}}%
\pgfpathlineto{\pgfqpoint{2.398465in}{2.369959in}}%
\pgfpathlineto{\pgfqpoint{2.442915in}{2.378279in}}%
\pgfpathlineto{\pgfqpoint{2.487147in}{2.386530in}}%
\pgfpathlineto{\pgfqpoint{2.531166in}{2.394711in}}%
\pgfpathlineto{\pgfqpoint{2.574980in}{2.402825in}}%
\pgfpathlineto{\pgfqpoint{2.618594in}{2.410868in}}%
\pgfpathlineto{\pgfqpoint{2.662017in}{2.418842in}}%
\pgfpathlineto{\pgfqpoint{2.705255in}{2.426746in}}%
\pgfpathlineto{\pgfqpoint{2.748315in}{2.434579in}}%
\pgfpathlineto{\pgfqpoint{2.791204in}{2.442341in}}%
\pgfpathlineto{\pgfqpoint{2.833930in}{2.450033in}}%
\pgfpathlineto{\pgfqpoint{2.876500in}{2.457654in}}%
\pgfpathlineto{\pgfqpoint{2.918921in}{2.465203in}}%
\pgfpathlineto{\pgfqpoint{2.961199in}{2.472682in}}%
\pgfpathlineto{\pgfqpoint{3.003342in}{2.480090in}}%
\pgfpathlineto{\pgfqpoint{3.045355in}{2.487427in}}%
\pgfpathlineto{\pgfqpoint{3.087242in}{2.494692in}}%
\pgfpathlineto{\pgfqpoint{3.129009in}{2.501885in}}%
\pgfpathlineto{\pgfqpoint{3.170661in}{2.509005in}}%
\pgfpathlineto{\pgfqpoint{3.212203in}{2.516057in}}%
\pgfpathlineto{\pgfqpoint{3.253639in}{2.523038in}}%
\pgfpathlineto{\pgfqpoint{3.294973in}{2.529949in}}%
\pgfpathlineto{\pgfqpoint{3.336209in}{2.536790in}}%
\pgfpathlineto{\pgfqpoint{3.377348in}{2.543561in}}%
\pgfpathlineto{\pgfqpoint{3.418395in}{2.550264in}}%
\pgfpathlineto{\pgfqpoint{3.459351in}{2.556897in}}%
\pgfpathlineto{\pgfqpoint{3.500221in}{2.563462in}}%
\pgfpathlineto{\pgfqpoint{3.541007in}{2.569959in}}%
\pgfpathlineto{\pgfqpoint{3.581713in}{2.576389in}}%
\pgfpathlineto{\pgfqpoint{3.622340in}{2.582752in}}%
\pgfpathlineto{\pgfqpoint{3.662890in}{2.589048in}}%
\pgfpathlineto{\pgfqpoint{3.703364in}{2.595278in}}%
\pgfpathlineto{\pgfqpoint{3.743763in}{2.601443in}}%
\pgfpathlineto{\pgfqpoint{3.784089in}{2.607543in}}%
\pgfpathlineto{\pgfqpoint{3.824343in}{2.613578in}}%
\pgfpathlineto{\pgfqpoint{3.864526in}{2.619551in}}%
\pgfpathlineto{\pgfqpoint{3.904640in}{2.625460in}}%
\pgfpathlineto{\pgfqpoint{3.944684in}{2.631307in}}%
\pgfpathlineto{\pgfqpoint{3.984657in}{2.637091in}}%
\pgfpathlineto{\pgfqpoint{4.024559in}{2.642814in}}%
\pgfpathlineto{\pgfqpoint{4.064388in}{2.648475in}}%
\pgfpathlineto{\pgfqpoint{4.104146in}{2.654075in}}%
\pgfpathlineto{\pgfqpoint{4.143832in}{2.659616in}}%
\pgfpathlineto{\pgfqpoint{4.183446in}{2.665098in}}%
\pgfpathlineto{\pgfqpoint{4.222988in}{2.670522in}}%
\pgfpathlineto{\pgfqpoint{4.262458in}{2.675889in}}%
\pgfpathlineto{\pgfqpoint{4.301855in}{2.681200in}}%
\pgfpathlineto{\pgfqpoint{4.341179in}{2.686455in}}%
\pgfpathlineto{\pgfqpoint{4.380429in}{2.691654in}}%
\pgfpathlineto{\pgfqpoint{4.419609in}{2.696797in}}%
\pgfpathlineto{\pgfqpoint{4.458719in}{2.701884in}}%
\pgfpathlineto{\pgfqpoint{4.497761in}{2.706918in}}%
\pgfpathlineto{\pgfqpoint{4.536737in}{2.711899in}}%
\pgfpathlineto{\pgfqpoint{4.575646in}{2.716827in}}%
\pgfpathlineto{\pgfqpoint{4.614490in}{2.721705in}}%
\pgfpathlineto{\pgfqpoint{4.653269in}{2.726530in}}%
\pgfpathlineto{\pgfqpoint{4.691985in}{2.731304in}}%
\pgfpathlineto{\pgfqpoint{4.730638in}{2.736027in}}%
\pgfpathlineto{\pgfqpoint{4.769231in}{2.740701in}}%
\pgfpathlineto{\pgfqpoint{4.807764in}{2.745325in}}%
\pgfpathlineto{\pgfqpoint{4.846237in}{2.749902in}}%
\pgfpathlineto{\pgfqpoint{4.884649in}{2.754430in}}%
\pgfpathlineto{\pgfqpoint{4.923000in}{2.758911in}}%
\pgfpathlineto{\pgfqpoint{4.961289in}{2.763345in}}%
\pgfpathlineto{\pgfqpoint{4.999516in}{2.767733in}}%
\pgfpathlineto{\pgfqpoint{5.037682in}{2.772075in}}%
\pgfpathlineto{\pgfqpoint{5.075786in}{2.776372in}}%
\pgfpathlineto{\pgfqpoint{5.113829in}{2.780625in}}%
\pgfusepath{stroke}%
\end{pgfscope}%
\begin{pgfscope}%
\pgfpathrectangle{\pgfqpoint{0.675193in}{0.526079in}}{\pgfqpoint{4.650000in}{3.020000in}}%
\pgfusepath{clip}%
\pgfsetrectcap%
\pgfsetroundjoin%
\pgfsetlinewidth{1.505625pt}%
\definecolor{currentstroke}{rgb}{0.000000,0.750000,0.750000}%
\pgfsetstrokecolor{currentstroke}%
\pgfsetstrokeopacity{0.750000}%
\pgfsetdash{}{0pt}%
\pgfpathmoveto{\pgfqpoint{1.266608in}{2.003850in}}%
\pgfpathlineto{\pgfqpoint{1.315801in}{1.941998in}}%
\pgfpathlineto{\pgfqpoint{1.364866in}{1.885667in}}%
\pgfpathlineto{\pgfqpoint{1.413797in}{1.833564in}}%
\pgfpathlineto{\pgfqpoint{1.462586in}{1.785238in}}%
\pgfpathlineto{\pgfqpoint{1.511228in}{1.740285in}}%
\pgfpathlineto{\pgfqpoint{1.559715in}{1.698362in}}%
\pgfpathlineto{\pgfqpoint{1.608040in}{1.659167in}}%
\pgfpathlineto{\pgfqpoint{1.656198in}{1.622438in}}%
\pgfpathlineto{\pgfqpoint{1.704181in}{1.587948in}}%
\pgfpathlineto{\pgfqpoint{1.751983in}{1.555497in}}%
\pgfpathlineto{\pgfqpoint{1.799596in}{1.524910in}}%
\pgfpathlineto{\pgfqpoint{1.847015in}{1.496031in}}%
\pgfpathlineto{\pgfqpoint{1.894222in}{1.468721in}}%
\pgfpathlineto{\pgfqpoint{1.941215in}{1.442854in}}%
\pgfpathlineto{\pgfqpoint{1.987986in}{1.418319in}}%
\pgfpathlineto{\pgfqpoint{2.034532in}{1.395016in}}%
\pgfpathlineto{\pgfqpoint{2.080847in}{1.372854in}}%
\pgfpathlineto{\pgfqpoint{2.126928in}{1.351751in}}%
\pgfpathlineto{\pgfqpoint{2.172772in}{1.331631in}}%
\pgfpathlineto{\pgfqpoint{2.218380in}{1.312428in}}%
\pgfpathlineto{\pgfqpoint{2.263751in}{1.294078in}}%
\pgfpathlineto{\pgfqpoint{2.308887in}{1.276530in}}%
\pgfpathlineto{\pgfqpoint{2.353790in}{1.259726in}}%
\pgfpathlineto{\pgfqpoint{2.398465in}{1.243621in}}%
\pgfpathlineto{\pgfqpoint{2.442915in}{1.228171in}}%
\pgfpathlineto{\pgfqpoint{2.487147in}{1.213336in}}%
\pgfpathlineto{\pgfqpoint{2.531166in}{1.199077in}}%
\pgfpathlineto{\pgfqpoint{2.574980in}{1.185362in}}%
\pgfpathlineto{\pgfqpoint{2.618594in}{1.172157in}}%
\pgfpathlineto{\pgfqpoint{2.662017in}{1.159434in}}%
\pgfpathlineto{\pgfqpoint{2.705255in}{1.147165in}}%
\pgfpathlineto{\pgfqpoint{2.748315in}{1.135324in}}%
\pgfpathlineto{\pgfqpoint{2.791204in}{1.123889in}}%
\pgfpathlineto{\pgfqpoint{2.833930in}{1.112837in}}%
\pgfpathlineto{\pgfqpoint{2.876500in}{1.102146in}}%
\pgfpathlineto{\pgfqpoint{2.918921in}{1.091799in}}%
\pgfpathlineto{\pgfqpoint{2.961199in}{1.081777in}}%
\pgfpathlineto{\pgfqpoint{3.003342in}{1.072064in}}%
\pgfpathlineto{\pgfqpoint{3.045355in}{1.062645in}}%
\pgfpathlineto{\pgfqpoint{3.087242in}{1.053502in}}%
\pgfpathlineto{\pgfqpoint{3.129009in}{1.044621in}}%
\pgfpathlineto{\pgfqpoint{3.170661in}{1.035990in}}%
\pgfpathlineto{\pgfqpoint{3.212203in}{1.027600in}}%
\pgfpathlineto{\pgfqpoint{3.253639in}{1.019437in}}%
\pgfpathlineto{\pgfqpoint{3.294973in}{1.011490in}}%
\pgfpathlineto{\pgfqpoint{3.336209in}{1.003750in}}%
\pgfpathlineto{\pgfqpoint{3.377348in}{0.996207in}}%
\pgfpathlineto{\pgfqpoint{3.418395in}{0.988852in}}%
\pgfpathlineto{\pgfqpoint{3.459351in}{0.981675in}}%
\pgfpathlineto{\pgfqpoint{3.500221in}{0.974670in}}%
\pgfpathlineto{\pgfqpoint{3.541007in}{0.967829in}}%
\pgfpathlineto{\pgfqpoint{3.581713in}{0.961144in}}%
\pgfpathlineto{\pgfqpoint{3.622340in}{0.954609in}}%
\pgfpathlineto{\pgfqpoint{3.662890in}{0.948216in}}%
\pgfpathlineto{\pgfqpoint{3.703364in}{0.941961in}}%
\pgfpathlineto{\pgfqpoint{3.743763in}{0.935838in}}%
\pgfpathlineto{\pgfqpoint{3.784089in}{0.929840in}}%
\pgfpathlineto{\pgfqpoint{3.824343in}{0.923963in}}%
\pgfpathlineto{\pgfqpoint{3.864526in}{0.918202in}}%
\pgfpathlineto{\pgfqpoint{3.904640in}{0.912553in}}%
\pgfpathlineto{\pgfqpoint{3.944684in}{0.907010in}}%
\pgfpathlineto{\pgfqpoint{3.984657in}{0.901569in}}%
\pgfpathlineto{\pgfqpoint{4.024559in}{0.896227in}}%
\pgfpathlineto{\pgfqpoint{4.064388in}{0.890978in}}%
\pgfpathlineto{\pgfqpoint{4.104146in}{0.885820in}}%
\pgfpathlineto{\pgfqpoint{4.143832in}{0.880750in}}%
\pgfpathlineto{\pgfqpoint{4.183446in}{0.875764in}}%
\pgfpathlineto{\pgfqpoint{4.222988in}{0.870860in}}%
\pgfpathlineto{\pgfqpoint{4.262458in}{0.866034in}}%
\pgfpathlineto{\pgfqpoint{4.301855in}{0.861285in}}%
\pgfpathlineto{\pgfqpoint{4.341179in}{0.856609in}}%
\pgfpathlineto{\pgfqpoint{4.380429in}{0.852003in}}%
\pgfpathlineto{\pgfqpoint{4.419609in}{0.847463in}}%
\pgfpathlineto{\pgfqpoint{4.458719in}{0.842989in}}%
\pgfpathlineto{\pgfqpoint{4.497761in}{0.838577in}}%
\pgfpathlineto{\pgfqpoint{4.536737in}{0.834226in}}%
\pgfpathlineto{\pgfqpoint{4.575646in}{0.829935in}}%
\pgfpathlineto{\pgfqpoint{4.614490in}{0.825701in}}%
\pgfpathlineto{\pgfqpoint{4.653269in}{0.821522in}}%
\pgfpathlineto{\pgfqpoint{4.691985in}{0.817396in}}%
\pgfpathlineto{\pgfqpoint{4.730638in}{0.813320in}}%
\pgfpathlineto{\pgfqpoint{4.769231in}{0.809295in}}%
\pgfpathlineto{\pgfqpoint{4.807764in}{0.805318in}}%
\pgfpathlineto{\pgfqpoint{4.846237in}{0.801387in}}%
\pgfpathlineto{\pgfqpoint{4.884649in}{0.797503in}}%
\pgfpathlineto{\pgfqpoint{4.923000in}{0.793661in}}%
\pgfpathlineto{\pgfqpoint{4.961289in}{0.789862in}}%
\pgfpathlineto{\pgfqpoint{4.999516in}{0.786105in}}%
\pgfpathlineto{\pgfqpoint{5.037682in}{0.782387in}}%
\pgfpathlineto{\pgfqpoint{5.075786in}{0.778708in}}%
\pgfpathlineto{\pgfqpoint{5.113829in}{0.775067in}}%
\pgfusepath{stroke}%
\end{pgfscope}%
\begin{pgfscope}%
\pgfsetrectcap%
\pgfsetmiterjoin%
\pgfsetlinewidth{0.803000pt}%
\definecolor{currentstroke}{rgb}{0.501961,0.501961,0.501961}%
\pgfsetstrokecolor{currentstroke}%
\pgfsetdash{}{0pt}%
\pgfpathmoveto{\pgfqpoint{0.675193in}{0.526079in}}%
\pgfpathlineto{\pgfqpoint{0.675193in}{3.546079in}}%
\pgfusepath{stroke}%
\end{pgfscope}%
\begin{pgfscope}%
\pgfsetrectcap%
\pgfsetmiterjoin%
\pgfsetlinewidth{0.803000pt}%
\definecolor{currentstroke}{rgb}{0.501961,0.501961,0.501961}%
\pgfsetstrokecolor{currentstroke}%
\pgfsetdash{}{0pt}%
\pgfpathmoveto{\pgfqpoint{5.325193in}{0.526079in}}%
\pgfpathlineto{\pgfqpoint{5.325193in}{3.546079in}}%
\pgfusepath{stroke}%
\end{pgfscope}%
\begin{pgfscope}%
\pgfsetrectcap%
\pgfsetmiterjoin%
\pgfsetlinewidth{0.803000pt}%
\definecolor{currentstroke}{rgb}{0.501961,0.501961,0.501961}%
\pgfsetstrokecolor{currentstroke}%
\pgfsetdash{}{0pt}%
\pgfpathmoveto{\pgfqpoint{0.675193in}{0.526079in}}%
\pgfpathlineto{\pgfqpoint{5.325193in}{0.526079in}}%
\pgfusepath{stroke}%
\end{pgfscope}%
\begin{pgfscope}%
\pgfsetrectcap%
\pgfsetmiterjoin%
\pgfsetlinewidth{0.803000pt}%
\definecolor{currentstroke}{rgb}{0.501961,0.501961,0.501961}%
\pgfsetstrokecolor{currentstroke}%
\pgfsetdash{}{0pt}%
\pgfpathmoveto{\pgfqpoint{0.675193in}{3.546079in}}%
\pgfpathlineto{\pgfqpoint{5.325193in}{3.546079in}}%
\pgfusepath{stroke}%
\end{pgfscope}%
\end{pgfpicture}%
\makeatother%
\endgroup%
}
    \caption{Simulated forces acting on the drop. Experiments are shown by order of increasing apoapse.\label{fig:forces}}
\end{figure}

In the non-dimensional trajectories with short-time scaling shown in Figure \ref{fig:series_s_ds}, we see that the trajectory apoapses are consistently $\mathcal{O}(1)$, but most trajectories overshoot their characteristic time scale (which predicts returns at $\bar{t}  =2$ to the first order). We also observe that that $\mathbb{E}\mbox{u}$ is not typically a small number in this regime, imperiling our use of asymptotic estimates in this regime. We can perhaps gain some insight by comparing the asymptotic estimate for return times to the scaled experimental return times. We see in Figure \ref{fig:times} that the long-time scaled non-dimensional time of first bounce in the experiment $t_b / t_c$, compares poorly to the asymptotic estimate for returns $t_f$ in the limit of small $\mathbb{E}\mbox{u}_+$, this is due to the use on long-time scaling for drops with $y/L \ll 1$.   
\begin{figure}[H]
    \centering
    %% Creator: Matplotlib, PGF backend
%%
%% To include the figure in your LaTeX document, write
%%   \input{<filename>.pgf}
%%
%% Make sure the required packages are loaded in your preamble
%%   \usepackage{pgf}
%%
%% Figures using additional raster images can only be included by \input if
%% they are in the same directory as the main LaTeX file. For loading figures
%% from other directories you can use the `import` package
%%   \usepackage{import}
%% and then include the figures with
%%   \import{<path to file>}{<filename>.pgf}
%%
%% Matplotlib used the following preamble
%%   \usepackage{fontspec}
%%   \setmainfont{DejaVuSerif.ttf}[Path=/home/erin/anaconda3/lib/python3.6/site-packages/matplotlib/mpl-data/fonts/ttf/]
%%   \setsansfont{DejaVuSans.ttf}[Path=/home/erin/anaconda3/lib/python3.6/site-packages/matplotlib/mpl-data/fonts/ttf/]
%%   \setmonofont{DejaVuSansMono.ttf}[Path=/home/erin/anaconda3/lib/python3.6/site-packages/matplotlib/mpl-data/fonts/ttf/]
%%
\begingroup%
\makeatletter%
\begin{pgfpicture}%
\pgfpathrectangle{\pgfpointorigin}{\pgfqpoint{5.311276in}{3.690214in}}%
\pgfusepath{use as bounding box, clip}%
\begin{pgfscope}%
\pgfsetbuttcap%
\pgfsetmiterjoin%
\definecolor{currentfill}{rgb}{1.000000,1.000000,1.000000}%
\pgfsetfillcolor{currentfill}%
\pgfsetlinewidth{0.000000pt}%
\definecolor{currentstroke}{rgb}{1.000000,1.000000,1.000000}%
\pgfsetstrokecolor{currentstroke}%
\pgfsetdash{}{0pt}%
\pgfpathmoveto{\pgfqpoint{0.000000in}{0.000000in}}%
\pgfpathlineto{\pgfqpoint{5.311276in}{0.000000in}}%
\pgfpathlineto{\pgfqpoint{5.311276in}{3.690214in}}%
\pgfpathlineto{\pgfqpoint{0.000000in}{3.690214in}}%
\pgfpathclose%
\pgfusepath{fill}%
\end{pgfscope}%
\begin{pgfscope}%
\pgfsetbuttcap%
\pgfsetmiterjoin%
\definecolor{currentfill}{rgb}{1.000000,1.000000,1.000000}%
\pgfsetfillcolor{currentfill}%
\pgfsetlinewidth{0.000000pt}%
\definecolor{currentstroke}{rgb}{0.000000,0.000000,0.000000}%
\pgfsetstrokecolor{currentstroke}%
\pgfsetstrokeopacity{0.000000}%
\pgfsetdash{}{0pt}%
\pgfpathmoveto{\pgfqpoint{0.564660in}{0.521603in}}%
\pgfpathlineto{\pgfqpoint{4.284660in}{0.521603in}}%
\pgfpathlineto{\pgfqpoint{4.284660in}{3.541603in}}%
\pgfpathlineto{\pgfqpoint{0.564660in}{3.541603in}}%
\pgfpathclose%
\pgfusepath{fill}%
\end{pgfscope}%
\begin{pgfscope}%
\pgfsetbuttcap%
\pgfsetroundjoin%
\definecolor{currentfill}{rgb}{0.000000,0.000000,0.000000}%
\pgfsetfillcolor{currentfill}%
\pgfsetlinewidth{0.803000pt}%
\definecolor{currentstroke}{rgb}{0.000000,0.000000,0.000000}%
\pgfsetstrokecolor{currentstroke}%
\pgfsetdash{}{0pt}%
\pgfsys@defobject{currentmarker}{\pgfqpoint{0.000000in}{-0.048611in}}{\pgfqpoint{0.000000in}{0.000000in}}{%
\pgfpathmoveto{\pgfqpoint{0.000000in}{0.000000in}}%
\pgfpathlineto{\pgfqpoint{0.000000in}{-0.048611in}}%
\pgfusepath{stroke,fill}%
}%
\begin{pgfscope}%
\pgfsys@transformshift{0.564660in}{0.521603in}%
\pgfsys@useobject{currentmarker}{}%
\end{pgfscope}%
\end{pgfscope}%
\begin{pgfscope}%
\definecolor{textcolor}{rgb}{0.000000,0.000000,0.000000}%
\pgfsetstrokecolor{textcolor}%
\pgfsetfillcolor{textcolor}%
\pgftext[x=0.564660in,y=0.424381in,,top]{\color{textcolor}\rmfamily\fontsize{10.000000}{12.000000}\selectfont \(\displaystyle 0.0\)}%
\end{pgfscope}%
\begin{pgfscope}%
\pgfsetbuttcap%
\pgfsetroundjoin%
\definecolor{currentfill}{rgb}{0.000000,0.000000,0.000000}%
\pgfsetfillcolor{currentfill}%
\pgfsetlinewidth{0.803000pt}%
\definecolor{currentstroke}{rgb}{0.000000,0.000000,0.000000}%
\pgfsetstrokecolor{currentstroke}%
\pgfsetdash{}{0pt}%
\pgfsys@defobject{currentmarker}{\pgfqpoint{0.000000in}{-0.048611in}}{\pgfqpoint{0.000000in}{0.000000in}}{%
\pgfpathmoveto{\pgfqpoint{0.000000in}{0.000000in}}%
\pgfpathlineto{\pgfqpoint{0.000000in}{-0.048611in}}%
\pgfusepath{stroke,fill}%
}%
\begin{pgfscope}%
\pgfsys@transformshift{1.029660in}{0.521603in}%
\pgfsys@useobject{currentmarker}{}%
\end{pgfscope}%
\end{pgfscope}%
\begin{pgfscope}%
\definecolor{textcolor}{rgb}{0.000000,0.000000,0.000000}%
\pgfsetstrokecolor{textcolor}%
\pgfsetfillcolor{textcolor}%
\pgftext[x=1.029660in,y=0.424381in,,top]{\color{textcolor}\rmfamily\fontsize{10.000000}{12.000000}\selectfont \(\displaystyle 0.5\)}%
\end{pgfscope}%
\begin{pgfscope}%
\pgfsetbuttcap%
\pgfsetroundjoin%
\definecolor{currentfill}{rgb}{0.000000,0.000000,0.000000}%
\pgfsetfillcolor{currentfill}%
\pgfsetlinewidth{0.803000pt}%
\definecolor{currentstroke}{rgb}{0.000000,0.000000,0.000000}%
\pgfsetstrokecolor{currentstroke}%
\pgfsetdash{}{0pt}%
\pgfsys@defobject{currentmarker}{\pgfqpoint{0.000000in}{-0.048611in}}{\pgfqpoint{0.000000in}{0.000000in}}{%
\pgfpathmoveto{\pgfqpoint{0.000000in}{0.000000in}}%
\pgfpathlineto{\pgfqpoint{0.000000in}{-0.048611in}}%
\pgfusepath{stroke,fill}%
}%
\begin{pgfscope}%
\pgfsys@transformshift{1.494660in}{0.521603in}%
\pgfsys@useobject{currentmarker}{}%
\end{pgfscope}%
\end{pgfscope}%
\begin{pgfscope}%
\definecolor{textcolor}{rgb}{0.000000,0.000000,0.000000}%
\pgfsetstrokecolor{textcolor}%
\pgfsetfillcolor{textcolor}%
\pgftext[x=1.494660in,y=0.424381in,,top]{\color{textcolor}\rmfamily\fontsize{10.000000}{12.000000}\selectfont \(\displaystyle 1.0\)}%
\end{pgfscope}%
\begin{pgfscope}%
\pgfsetbuttcap%
\pgfsetroundjoin%
\definecolor{currentfill}{rgb}{0.000000,0.000000,0.000000}%
\pgfsetfillcolor{currentfill}%
\pgfsetlinewidth{0.803000pt}%
\definecolor{currentstroke}{rgb}{0.000000,0.000000,0.000000}%
\pgfsetstrokecolor{currentstroke}%
\pgfsetdash{}{0pt}%
\pgfsys@defobject{currentmarker}{\pgfqpoint{0.000000in}{-0.048611in}}{\pgfqpoint{0.000000in}{0.000000in}}{%
\pgfpathmoveto{\pgfqpoint{0.000000in}{0.000000in}}%
\pgfpathlineto{\pgfqpoint{0.000000in}{-0.048611in}}%
\pgfusepath{stroke,fill}%
}%
\begin{pgfscope}%
\pgfsys@transformshift{1.959660in}{0.521603in}%
\pgfsys@useobject{currentmarker}{}%
\end{pgfscope}%
\end{pgfscope}%
\begin{pgfscope}%
\definecolor{textcolor}{rgb}{0.000000,0.000000,0.000000}%
\pgfsetstrokecolor{textcolor}%
\pgfsetfillcolor{textcolor}%
\pgftext[x=1.959660in,y=0.424381in,,top]{\color{textcolor}\rmfamily\fontsize{10.000000}{12.000000}\selectfont \(\displaystyle 1.5\)}%
\end{pgfscope}%
\begin{pgfscope}%
\pgfsetbuttcap%
\pgfsetroundjoin%
\definecolor{currentfill}{rgb}{0.000000,0.000000,0.000000}%
\pgfsetfillcolor{currentfill}%
\pgfsetlinewidth{0.803000pt}%
\definecolor{currentstroke}{rgb}{0.000000,0.000000,0.000000}%
\pgfsetstrokecolor{currentstroke}%
\pgfsetdash{}{0pt}%
\pgfsys@defobject{currentmarker}{\pgfqpoint{0.000000in}{-0.048611in}}{\pgfqpoint{0.000000in}{0.000000in}}{%
\pgfpathmoveto{\pgfqpoint{0.000000in}{0.000000in}}%
\pgfpathlineto{\pgfqpoint{0.000000in}{-0.048611in}}%
\pgfusepath{stroke,fill}%
}%
\begin{pgfscope}%
\pgfsys@transformshift{2.424660in}{0.521603in}%
\pgfsys@useobject{currentmarker}{}%
\end{pgfscope}%
\end{pgfscope}%
\begin{pgfscope}%
\definecolor{textcolor}{rgb}{0.000000,0.000000,0.000000}%
\pgfsetstrokecolor{textcolor}%
\pgfsetfillcolor{textcolor}%
\pgftext[x=2.424660in,y=0.424381in,,top]{\color{textcolor}\rmfamily\fontsize{10.000000}{12.000000}\selectfont \(\displaystyle 2.0\)}%
\end{pgfscope}%
\begin{pgfscope}%
\pgfsetbuttcap%
\pgfsetroundjoin%
\definecolor{currentfill}{rgb}{0.000000,0.000000,0.000000}%
\pgfsetfillcolor{currentfill}%
\pgfsetlinewidth{0.803000pt}%
\definecolor{currentstroke}{rgb}{0.000000,0.000000,0.000000}%
\pgfsetstrokecolor{currentstroke}%
\pgfsetdash{}{0pt}%
\pgfsys@defobject{currentmarker}{\pgfqpoint{0.000000in}{-0.048611in}}{\pgfqpoint{0.000000in}{0.000000in}}{%
\pgfpathmoveto{\pgfqpoint{0.000000in}{0.000000in}}%
\pgfpathlineto{\pgfqpoint{0.000000in}{-0.048611in}}%
\pgfusepath{stroke,fill}%
}%
\begin{pgfscope}%
\pgfsys@transformshift{2.889660in}{0.521603in}%
\pgfsys@useobject{currentmarker}{}%
\end{pgfscope}%
\end{pgfscope}%
\begin{pgfscope}%
\definecolor{textcolor}{rgb}{0.000000,0.000000,0.000000}%
\pgfsetstrokecolor{textcolor}%
\pgfsetfillcolor{textcolor}%
\pgftext[x=2.889660in,y=0.424381in,,top]{\color{textcolor}\rmfamily\fontsize{10.000000}{12.000000}\selectfont \(\displaystyle 2.5\)}%
\end{pgfscope}%
\begin{pgfscope}%
\pgfsetbuttcap%
\pgfsetroundjoin%
\definecolor{currentfill}{rgb}{0.000000,0.000000,0.000000}%
\pgfsetfillcolor{currentfill}%
\pgfsetlinewidth{0.803000pt}%
\definecolor{currentstroke}{rgb}{0.000000,0.000000,0.000000}%
\pgfsetstrokecolor{currentstroke}%
\pgfsetdash{}{0pt}%
\pgfsys@defobject{currentmarker}{\pgfqpoint{0.000000in}{-0.048611in}}{\pgfqpoint{0.000000in}{0.000000in}}{%
\pgfpathmoveto{\pgfqpoint{0.000000in}{0.000000in}}%
\pgfpathlineto{\pgfqpoint{0.000000in}{-0.048611in}}%
\pgfusepath{stroke,fill}%
}%
\begin{pgfscope}%
\pgfsys@transformshift{3.354660in}{0.521603in}%
\pgfsys@useobject{currentmarker}{}%
\end{pgfscope}%
\end{pgfscope}%
\begin{pgfscope}%
\definecolor{textcolor}{rgb}{0.000000,0.000000,0.000000}%
\pgfsetstrokecolor{textcolor}%
\pgfsetfillcolor{textcolor}%
\pgftext[x=3.354660in,y=0.424381in,,top]{\color{textcolor}\rmfamily\fontsize{10.000000}{12.000000}\selectfont \(\displaystyle 3.0\)}%
\end{pgfscope}%
\begin{pgfscope}%
\pgfsetbuttcap%
\pgfsetroundjoin%
\definecolor{currentfill}{rgb}{0.000000,0.000000,0.000000}%
\pgfsetfillcolor{currentfill}%
\pgfsetlinewidth{0.803000pt}%
\definecolor{currentstroke}{rgb}{0.000000,0.000000,0.000000}%
\pgfsetstrokecolor{currentstroke}%
\pgfsetdash{}{0pt}%
\pgfsys@defobject{currentmarker}{\pgfqpoint{0.000000in}{-0.048611in}}{\pgfqpoint{0.000000in}{0.000000in}}{%
\pgfpathmoveto{\pgfqpoint{0.000000in}{0.000000in}}%
\pgfpathlineto{\pgfqpoint{0.000000in}{-0.048611in}}%
\pgfusepath{stroke,fill}%
}%
\begin{pgfscope}%
\pgfsys@transformshift{3.819660in}{0.521603in}%
\pgfsys@useobject{currentmarker}{}%
\end{pgfscope}%
\end{pgfscope}%
\begin{pgfscope}%
\definecolor{textcolor}{rgb}{0.000000,0.000000,0.000000}%
\pgfsetstrokecolor{textcolor}%
\pgfsetfillcolor{textcolor}%
\pgftext[x=3.819660in,y=0.424381in,,top]{\color{textcolor}\rmfamily\fontsize{10.000000}{12.000000}\selectfont \(\displaystyle 3.5\)}%
\end{pgfscope}%
\begin{pgfscope}%
\pgfsetbuttcap%
\pgfsetroundjoin%
\definecolor{currentfill}{rgb}{0.000000,0.000000,0.000000}%
\pgfsetfillcolor{currentfill}%
\pgfsetlinewidth{0.803000pt}%
\definecolor{currentstroke}{rgb}{0.000000,0.000000,0.000000}%
\pgfsetstrokecolor{currentstroke}%
\pgfsetdash{}{0pt}%
\pgfsys@defobject{currentmarker}{\pgfqpoint{0.000000in}{-0.048611in}}{\pgfqpoint{0.000000in}{0.000000in}}{%
\pgfpathmoveto{\pgfqpoint{0.000000in}{0.000000in}}%
\pgfpathlineto{\pgfqpoint{0.000000in}{-0.048611in}}%
\pgfusepath{stroke,fill}%
}%
\begin{pgfscope}%
\pgfsys@transformshift{4.284660in}{0.521603in}%
\pgfsys@useobject{currentmarker}{}%
\end{pgfscope}%
\end{pgfscope}%
\begin{pgfscope}%
\definecolor{textcolor}{rgb}{0.000000,0.000000,0.000000}%
\pgfsetstrokecolor{textcolor}%
\pgfsetfillcolor{textcolor}%
\pgftext[x=4.284660in,y=0.424381in,,top]{\color{textcolor}\rmfamily\fontsize{10.000000}{12.000000}\selectfont \(\displaystyle 4.0\)}%
\end{pgfscope}%
\begin{pgfscope}%
\definecolor{textcolor}{rgb}{0.000000,0.000000,0.000000}%
\pgfsetstrokecolor{textcolor}%
\pgfsetfillcolor{textcolor}%
\pgftext[x=2.424660in,y=0.234413in,,top]{\color{textcolor}\rmfamily\fontsize{10.000000}{12.000000}\selectfont \(\displaystyle t^*\)}%
\end{pgfscope}%
\begin{pgfscope}%
\pgfsetbuttcap%
\pgfsetroundjoin%
\definecolor{currentfill}{rgb}{0.000000,0.000000,0.000000}%
\pgfsetfillcolor{currentfill}%
\pgfsetlinewidth{0.803000pt}%
\definecolor{currentstroke}{rgb}{0.000000,0.000000,0.000000}%
\pgfsetstrokecolor{currentstroke}%
\pgfsetdash{}{0pt}%
\pgfsys@defobject{currentmarker}{\pgfqpoint{-0.048611in}{0.000000in}}{\pgfqpoint{0.000000in}{0.000000in}}{%
\pgfpathmoveto{\pgfqpoint{0.000000in}{0.000000in}}%
\pgfpathlineto{\pgfqpoint{-0.048611in}{0.000000in}}%
\pgfusepath{stroke,fill}%
}%
\begin{pgfscope}%
\pgfsys@transformshift{0.564660in}{0.671953in}%
\pgfsys@useobject{currentmarker}{}%
\end{pgfscope}%
\end{pgfscope}%
\begin{pgfscope}%
\definecolor{textcolor}{rgb}{0.000000,0.000000,0.000000}%
\pgfsetstrokecolor{textcolor}%
\pgfsetfillcolor{textcolor}%
\pgftext[x=0.289968in,y=0.619192in,left,base]{\color{textcolor}\rmfamily\fontsize{10.000000}{12.000000}\selectfont \(\displaystyle 0.0\)}%
\end{pgfscope}%
\begin{pgfscope}%
\pgfsetbuttcap%
\pgfsetroundjoin%
\definecolor{currentfill}{rgb}{0.000000,0.000000,0.000000}%
\pgfsetfillcolor{currentfill}%
\pgfsetlinewidth{0.803000pt}%
\definecolor{currentstroke}{rgb}{0.000000,0.000000,0.000000}%
\pgfsetstrokecolor{currentstroke}%
\pgfsetdash{}{0pt}%
\pgfsys@defobject{currentmarker}{\pgfqpoint{-0.048611in}{0.000000in}}{\pgfqpoint{0.000000in}{0.000000in}}{%
\pgfpathmoveto{\pgfqpoint{0.000000in}{0.000000in}}%
\pgfpathlineto{\pgfqpoint{-0.048611in}{0.000000in}}%
\pgfusepath{stroke,fill}%
}%
\begin{pgfscope}%
\pgfsys@transformshift{0.564660in}{1.124453in}%
\pgfsys@useobject{currentmarker}{}%
\end{pgfscope}%
\end{pgfscope}%
\begin{pgfscope}%
\definecolor{textcolor}{rgb}{0.000000,0.000000,0.000000}%
\pgfsetstrokecolor{textcolor}%
\pgfsetfillcolor{textcolor}%
\pgftext[x=0.289968in,y=1.071692in,left,base]{\color{textcolor}\rmfamily\fontsize{10.000000}{12.000000}\selectfont \(\displaystyle 0.2\)}%
\end{pgfscope}%
\begin{pgfscope}%
\pgfsetbuttcap%
\pgfsetroundjoin%
\definecolor{currentfill}{rgb}{0.000000,0.000000,0.000000}%
\pgfsetfillcolor{currentfill}%
\pgfsetlinewidth{0.803000pt}%
\definecolor{currentstroke}{rgb}{0.000000,0.000000,0.000000}%
\pgfsetstrokecolor{currentstroke}%
\pgfsetdash{}{0pt}%
\pgfsys@defobject{currentmarker}{\pgfqpoint{-0.048611in}{0.000000in}}{\pgfqpoint{0.000000in}{0.000000in}}{%
\pgfpathmoveto{\pgfqpoint{0.000000in}{0.000000in}}%
\pgfpathlineto{\pgfqpoint{-0.048611in}{0.000000in}}%
\pgfusepath{stroke,fill}%
}%
\begin{pgfscope}%
\pgfsys@transformshift{0.564660in}{1.576953in}%
\pgfsys@useobject{currentmarker}{}%
\end{pgfscope}%
\end{pgfscope}%
\begin{pgfscope}%
\definecolor{textcolor}{rgb}{0.000000,0.000000,0.000000}%
\pgfsetstrokecolor{textcolor}%
\pgfsetfillcolor{textcolor}%
\pgftext[x=0.289968in,y=1.524192in,left,base]{\color{textcolor}\rmfamily\fontsize{10.000000}{12.000000}\selectfont \(\displaystyle 0.4\)}%
\end{pgfscope}%
\begin{pgfscope}%
\pgfsetbuttcap%
\pgfsetroundjoin%
\definecolor{currentfill}{rgb}{0.000000,0.000000,0.000000}%
\pgfsetfillcolor{currentfill}%
\pgfsetlinewidth{0.803000pt}%
\definecolor{currentstroke}{rgb}{0.000000,0.000000,0.000000}%
\pgfsetstrokecolor{currentstroke}%
\pgfsetdash{}{0pt}%
\pgfsys@defobject{currentmarker}{\pgfqpoint{-0.048611in}{0.000000in}}{\pgfqpoint{0.000000in}{0.000000in}}{%
\pgfpathmoveto{\pgfqpoint{0.000000in}{0.000000in}}%
\pgfpathlineto{\pgfqpoint{-0.048611in}{0.000000in}}%
\pgfusepath{stroke,fill}%
}%
\begin{pgfscope}%
\pgfsys@transformshift{0.564660in}{2.029453in}%
\pgfsys@useobject{currentmarker}{}%
\end{pgfscope}%
\end{pgfscope}%
\begin{pgfscope}%
\definecolor{textcolor}{rgb}{0.000000,0.000000,0.000000}%
\pgfsetstrokecolor{textcolor}%
\pgfsetfillcolor{textcolor}%
\pgftext[x=0.289968in,y=1.976692in,left,base]{\color{textcolor}\rmfamily\fontsize{10.000000}{12.000000}\selectfont \(\displaystyle 0.6\)}%
\end{pgfscope}%
\begin{pgfscope}%
\pgfsetbuttcap%
\pgfsetroundjoin%
\definecolor{currentfill}{rgb}{0.000000,0.000000,0.000000}%
\pgfsetfillcolor{currentfill}%
\pgfsetlinewidth{0.803000pt}%
\definecolor{currentstroke}{rgb}{0.000000,0.000000,0.000000}%
\pgfsetstrokecolor{currentstroke}%
\pgfsetdash{}{0pt}%
\pgfsys@defobject{currentmarker}{\pgfqpoint{-0.048611in}{0.000000in}}{\pgfqpoint{0.000000in}{0.000000in}}{%
\pgfpathmoveto{\pgfqpoint{0.000000in}{0.000000in}}%
\pgfpathlineto{\pgfqpoint{-0.048611in}{0.000000in}}%
\pgfusepath{stroke,fill}%
}%
\begin{pgfscope}%
\pgfsys@transformshift{0.564660in}{2.481953in}%
\pgfsys@useobject{currentmarker}{}%
\end{pgfscope}%
\end{pgfscope}%
\begin{pgfscope}%
\definecolor{textcolor}{rgb}{0.000000,0.000000,0.000000}%
\pgfsetstrokecolor{textcolor}%
\pgfsetfillcolor{textcolor}%
\pgftext[x=0.289968in,y=2.429192in,left,base]{\color{textcolor}\rmfamily\fontsize{10.000000}{12.000000}\selectfont \(\displaystyle 0.8\)}%
\end{pgfscope}%
\begin{pgfscope}%
\pgfsetbuttcap%
\pgfsetroundjoin%
\definecolor{currentfill}{rgb}{0.000000,0.000000,0.000000}%
\pgfsetfillcolor{currentfill}%
\pgfsetlinewidth{0.803000pt}%
\definecolor{currentstroke}{rgb}{0.000000,0.000000,0.000000}%
\pgfsetstrokecolor{currentstroke}%
\pgfsetdash{}{0pt}%
\pgfsys@defobject{currentmarker}{\pgfqpoint{-0.048611in}{0.000000in}}{\pgfqpoint{0.000000in}{0.000000in}}{%
\pgfpathmoveto{\pgfqpoint{0.000000in}{0.000000in}}%
\pgfpathlineto{\pgfqpoint{-0.048611in}{0.000000in}}%
\pgfusepath{stroke,fill}%
}%
\begin{pgfscope}%
\pgfsys@transformshift{0.564660in}{2.934453in}%
\pgfsys@useobject{currentmarker}{}%
\end{pgfscope}%
\end{pgfscope}%
\begin{pgfscope}%
\definecolor{textcolor}{rgb}{0.000000,0.000000,0.000000}%
\pgfsetstrokecolor{textcolor}%
\pgfsetfillcolor{textcolor}%
\pgftext[x=0.289968in,y=2.881692in,left,base]{\color{textcolor}\rmfamily\fontsize{10.000000}{12.000000}\selectfont \(\displaystyle 1.0\)}%
\end{pgfscope}%
\begin{pgfscope}%
\pgfsetbuttcap%
\pgfsetroundjoin%
\definecolor{currentfill}{rgb}{0.000000,0.000000,0.000000}%
\pgfsetfillcolor{currentfill}%
\pgfsetlinewidth{0.803000pt}%
\definecolor{currentstroke}{rgb}{0.000000,0.000000,0.000000}%
\pgfsetstrokecolor{currentstroke}%
\pgfsetdash{}{0pt}%
\pgfsys@defobject{currentmarker}{\pgfqpoint{-0.048611in}{0.000000in}}{\pgfqpoint{0.000000in}{0.000000in}}{%
\pgfpathmoveto{\pgfqpoint{0.000000in}{0.000000in}}%
\pgfpathlineto{\pgfqpoint{-0.048611in}{0.000000in}}%
\pgfusepath{stroke,fill}%
}%
\begin{pgfscope}%
\pgfsys@transformshift{0.564660in}{3.386954in}%
\pgfsys@useobject{currentmarker}{}%
\end{pgfscope}%
\end{pgfscope}%
\begin{pgfscope}%
\definecolor{textcolor}{rgb}{0.000000,0.000000,0.000000}%
\pgfsetstrokecolor{textcolor}%
\pgfsetfillcolor{textcolor}%
\pgftext[x=0.289968in,y=3.334192in,left,base]{\color{textcolor}\rmfamily\fontsize{10.000000}{12.000000}\selectfont \(\displaystyle 1.2\)}%
\end{pgfscope}%
\begin{pgfscope}%
\definecolor{textcolor}{rgb}{0.000000,0.000000,0.000000}%
\pgfsetstrokecolor{textcolor}%
\pgfsetfillcolor{textcolor}%
\pgftext[x=0.234413in,y=2.031603in,,bottom,rotate=90.000000]{\color{textcolor}\rmfamily\fontsize{10.000000}{12.000000}\selectfont \(\displaystyle y^*\)}%
\end{pgfscope}%
\begin{pgfscope}%
\pgfpathrectangle{\pgfqpoint{0.564660in}{0.521603in}}{\pgfqpoint{3.720000in}{3.020000in}}%
\pgfusepath{clip}%
\pgfsetrectcap%
\pgfsetroundjoin%
\pgfsetlinewidth{1.505625pt}%
\definecolor{currentstroke}{rgb}{0.993248,0.906157,0.143936}%
\pgfsetstrokecolor{currentstroke}%
\pgfsetdash{}{0pt}%
\pgfpathmoveto{\pgfqpoint{1.409488in}{2.299358in}}%
\pgfpathlineto{\pgfqpoint{1.486291in}{2.475880in}}%
\pgfpathlineto{\pgfqpoint{1.563093in}{2.637799in}}%
\pgfpathlineto{\pgfqpoint{1.639896in}{2.784889in}}%
\pgfpathlineto{\pgfqpoint{1.716698in}{2.916927in}}%
\pgfpathlineto{\pgfqpoint{1.793501in}{3.033686in}}%
\pgfpathlineto{\pgfqpoint{1.870303in}{3.134943in}}%
\pgfpathlineto{\pgfqpoint{1.947106in}{3.220472in}}%
\pgfpathlineto{\pgfqpoint{2.023908in}{3.290048in}}%
\pgfpathlineto{\pgfqpoint{2.100711in}{3.343447in}}%
\pgfpathlineto{\pgfqpoint{2.177513in}{3.380444in}}%
\pgfpathlineto{\pgfqpoint{2.254316in}{3.400813in}}%
\pgfpathlineto{\pgfqpoint{2.331118in}{3.404331in}}%
\pgfpathlineto{\pgfqpoint{2.407921in}{3.390904in}}%
\pgfpathlineto{\pgfqpoint{2.484723in}{3.360261in}}%
\pgfpathlineto{\pgfqpoint{2.561526in}{3.312160in}}%
\pgfpathlineto{\pgfqpoint{2.638328in}{3.246345in}}%
\pgfpathlineto{\pgfqpoint{2.715131in}{3.162550in}}%
\pgfpathlineto{\pgfqpoint{2.791933in}{3.060539in}}%
\pgfpathlineto{\pgfqpoint{2.868736in}{2.939892in}}%
\pgfpathlineto{\pgfqpoint{2.945539in}{2.800327in}}%
\pgfpathlineto{\pgfqpoint{3.022341in}{2.641559in}}%
\pgfpathlineto{\pgfqpoint{3.099144in}{2.463303in}}%
\pgfpathlineto{\pgfqpoint{3.175946in}{2.265275in}}%
\pgfpathlineto{\pgfqpoint{3.252749in}{2.047191in}}%
\pgfpathlineto{\pgfqpoint{3.329551in}{1.808766in}}%
\pgfpathlineto{\pgfqpoint{3.406354in}{1.549716in}}%
\pgfpathlineto{\pgfqpoint{3.483156in}{1.269756in}}%
\pgfpathlineto{\pgfqpoint{3.559959in}{0.968602in}}%
\pgfusepath{stroke}%
\end{pgfscope}%
\begin{pgfscope}%
\pgfpathrectangle{\pgfqpoint{0.564660in}{0.521603in}}{\pgfqpoint{3.720000in}{3.020000in}}%
\pgfusepath{clip}%
\pgfsetrectcap%
\pgfsetroundjoin%
\pgfsetlinewidth{1.505625pt}%
\definecolor{currentstroke}{rgb}{0.762373,0.876424,0.137064}%
\pgfsetstrokecolor{currentstroke}%
\pgfsetdash{}{0pt}%
\pgfpathmoveto{\pgfqpoint{1.233611in}{0.756271in}}%
\pgfpathlineto{\pgfqpoint{1.300506in}{0.938723in}}%
\pgfpathlineto{\pgfqpoint{1.367401in}{1.107373in}}%
\pgfpathlineto{\pgfqpoint{1.434296in}{1.262342in}}%
\pgfpathlineto{\pgfqpoint{1.501191in}{1.403753in}}%
\pgfpathlineto{\pgfqpoint{1.568086in}{1.531729in}}%
\pgfpathlineto{\pgfqpoint{1.634981in}{1.646390in}}%
\pgfpathlineto{\pgfqpoint{1.701876in}{1.747859in}}%
\pgfpathlineto{\pgfqpoint{1.768772in}{1.836259in}}%
\pgfpathlineto{\pgfqpoint{1.835667in}{1.911710in}}%
\pgfpathlineto{\pgfqpoint{1.902562in}{1.974336in}}%
\pgfpathlineto{\pgfqpoint{1.969457in}{2.024259in}}%
\pgfpathlineto{\pgfqpoint{2.036352in}{2.061600in}}%
\pgfpathlineto{\pgfqpoint{2.103247in}{2.086685in}}%
\pgfpathlineto{\pgfqpoint{2.170142in}{2.099560in}}%
\pgfpathlineto{\pgfqpoint{2.237037in}{2.100321in}}%
\pgfpathlineto{\pgfqpoint{2.303932in}{2.089048in}}%
\pgfpathlineto{\pgfqpoint{2.370827in}{2.065802in}}%
\pgfpathlineto{\pgfqpoint{2.437722in}{2.030627in}}%
\pgfpathlineto{\pgfqpoint{2.504617in}{1.983622in}}%
\pgfpathlineto{\pgfqpoint{2.571512in}{1.924599in}}%
\pgfpathlineto{\pgfqpoint{2.638407in}{1.853579in}}%
\pgfpathlineto{\pgfqpoint{2.705302in}{1.770581in}}%
\pgfpathlineto{\pgfqpoint{2.772198in}{1.675623in}}%
\pgfpathlineto{\pgfqpoint{2.839093in}{1.568725in}}%
\pgfpathlineto{\pgfqpoint{2.905988in}{1.449905in}}%
\pgfpathlineto{\pgfqpoint{2.972883in}{1.319184in}}%
\pgfpathlineto{\pgfqpoint{3.039778in}{1.176579in}}%
\pgfpathlineto{\pgfqpoint{3.106673in}{1.022111in}}%
\pgfpathlineto{\pgfqpoint{3.173568in}{0.855798in}}%
\pgfpathlineto{\pgfqpoint{3.240463in}{0.677659in}}%
\pgfusepath{stroke}%
\end{pgfscope}%
\begin{pgfscope}%
\pgfpathrectangle{\pgfqpoint{0.564660in}{0.521603in}}{\pgfqpoint{3.720000in}{3.020000in}}%
\pgfusepath{clip}%
\pgfsetrectcap%
\pgfsetroundjoin%
\pgfsetlinewidth{1.505625pt}%
\definecolor{currentstroke}{rgb}{0.751884,0.874951,0.143228}%
\pgfsetstrokecolor{currentstroke}%
\pgfsetdash{}{0pt}%
\pgfpathmoveto{\pgfqpoint{1.178273in}{2.036187in}}%
\pgfpathlineto{\pgfqpoint{1.246452in}{2.193330in}}%
\pgfpathlineto{\pgfqpoint{1.314631in}{2.339382in}}%
\pgfpathlineto{\pgfqpoint{1.382811in}{2.474114in}}%
\pgfpathlineto{\pgfqpoint{1.450990in}{2.597300in}}%
\pgfpathlineto{\pgfqpoint{1.519169in}{2.708710in}}%
\pgfpathlineto{\pgfqpoint{1.587348in}{2.808117in}}%
\pgfpathlineto{\pgfqpoint{1.655527in}{2.895293in}}%
\pgfpathlineto{\pgfqpoint{1.723707in}{2.970010in}}%
\pgfpathlineto{\pgfqpoint{1.791886in}{3.032040in}}%
\pgfpathlineto{\pgfqpoint{1.860065in}{3.081155in}}%
\pgfpathlineto{\pgfqpoint{1.928244in}{3.117128in}}%
\pgfpathlineto{\pgfqpoint{1.996423in}{3.139729in}}%
\pgfpathlineto{\pgfqpoint{2.064603in}{3.148971in}}%
\pgfpathlineto{\pgfqpoint{2.132782in}{3.144533in}}%
\pgfpathlineto{\pgfqpoint{2.200961in}{3.126161in}}%
\pgfpathlineto{\pgfqpoint{2.269140in}{3.093578in}}%
\pgfpathlineto{\pgfqpoint{2.337319in}{3.046489in}}%
\pgfpathlineto{\pgfqpoint{2.405499in}{2.984576in}}%
\pgfpathlineto{\pgfqpoint{2.473678in}{2.907503in}}%
\pgfpathlineto{\pgfqpoint{2.541857in}{2.814997in}}%
\pgfpathlineto{\pgfqpoint{2.610036in}{2.706455in}}%
\pgfpathlineto{\pgfqpoint{2.678215in}{2.581512in}}%
\pgfpathlineto{\pgfqpoint{2.746395in}{2.439805in}}%
\pgfpathlineto{\pgfqpoint{2.814574in}{2.280968in}}%
\pgfpathlineto{\pgfqpoint{2.882753in}{2.104637in}}%
\pgfpathlineto{\pgfqpoint{2.950932in}{1.910447in}}%
\pgfpathlineto{\pgfqpoint{3.019111in}{1.698036in}}%
\pgfpathlineto{\pgfqpoint{3.087291in}{1.467037in}}%
\pgfpathlineto{\pgfqpoint{3.155470in}{1.217087in}}%
\pgfpathlineto{\pgfqpoint{3.223649in}{0.947822in}}%
\pgfpathlineto{\pgfqpoint{3.291828in}{0.658876in}}%
\pgfusepath{stroke}%
\end{pgfscope}%
\begin{pgfscope}%
\pgfpathrectangle{\pgfqpoint{0.564660in}{0.521603in}}{\pgfqpoint{3.720000in}{3.020000in}}%
\pgfusepath{clip}%
\pgfsetrectcap%
\pgfsetroundjoin%
\pgfsetlinewidth{1.505625pt}%
\definecolor{currentstroke}{rgb}{0.377779,0.791781,0.377939}%
\pgfsetstrokecolor{currentstroke}%
\pgfsetdash{}{0pt}%
\pgfpathmoveto{\pgfqpoint{1.114895in}{1.137753in}}%
\pgfpathlineto{\pgfqpoint{1.183674in}{1.296900in}}%
\pgfpathlineto{\pgfqpoint{1.252453in}{1.444075in}}%
\pgfpathlineto{\pgfqpoint{1.321233in}{1.579273in}}%
\pgfpathlineto{\pgfqpoint{1.390012in}{1.702487in}}%
\pgfpathlineto{\pgfqpoint{1.458791in}{1.813712in}}%
\pgfpathlineto{\pgfqpoint{1.527570in}{1.912942in}}%
\pgfpathlineto{\pgfqpoint{1.596350in}{2.000173in}}%
\pgfpathlineto{\pgfqpoint{1.665129in}{2.075397in}}%
\pgfpathlineto{\pgfqpoint{1.733908in}{2.138609in}}%
\pgfpathlineto{\pgfqpoint{1.802688in}{2.189804in}}%
\pgfpathlineto{\pgfqpoint{1.871467in}{2.228976in}}%
\pgfpathlineto{\pgfqpoint{1.940246in}{2.256119in}}%
\pgfpathlineto{\pgfqpoint{2.009025in}{2.271336in}}%
\pgfpathlineto{\pgfqpoint{2.077805in}{2.274580in}}%
\pgfpathlineto{\pgfqpoint{2.146584in}{2.265833in}}%
\pgfpathlineto{\pgfqpoint{2.215363in}{2.245065in}}%
\pgfpathlineto{\pgfqpoint{2.284143in}{2.212239in}}%
\pgfpathlineto{\pgfqpoint{2.352922in}{2.167308in}}%
\pgfpathlineto{\pgfqpoint{2.421701in}{2.110253in}}%
\pgfpathlineto{\pgfqpoint{2.490481in}{2.040905in}}%
\pgfpathlineto{\pgfqpoint{2.559260in}{1.959203in}}%
\pgfpathlineto{\pgfqpoint{2.628039in}{1.865086in}}%
\pgfpathlineto{\pgfqpoint{2.696818in}{1.758495in}}%
\pgfpathlineto{\pgfqpoint{2.765598in}{1.639368in}}%
\pgfpathlineto{\pgfqpoint{2.834377in}{1.507646in}}%
\pgfpathlineto{\pgfqpoint{2.903156in}{1.363268in}}%
\pgfpathlineto{\pgfqpoint{2.971936in}{1.206173in}}%
\pgfpathlineto{\pgfqpoint{3.040715in}{1.036301in}}%
\pgfpathlineto{\pgfqpoint{3.109494in}{0.853591in}}%
\pgfusepath{stroke}%
\end{pgfscope}%
\begin{pgfscope}%
\pgfpathrectangle{\pgfqpoint{0.564660in}{0.521603in}}{\pgfqpoint{3.720000in}{3.020000in}}%
\pgfusepath{clip}%
\pgfsetrectcap%
\pgfsetroundjoin%
\pgfsetlinewidth{1.505625pt}%
\definecolor{currentstroke}{rgb}{0.128729,0.563265,0.551229}%
\pgfsetstrokecolor{currentstroke}%
\pgfsetdash{}{0pt}%
\pgfpathmoveto{\pgfqpoint{1.002735in}{1.139100in}}%
\pgfpathlineto{\pgfqpoint{1.046542in}{1.242081in}}%
\pgfpathlineto{\pgfqpoint{1.090350in}{1.340628in}}%
\pgfpathlineto{\pgfqpoint{1.134157in}{1.434751in}}%
\pgfpathlineto{\pgfqpoint{1.177965in}{1.524460in}}%
\pgfpathlineto{\pgfqpoint{1.221772in}{1.609765in}}%
\pgfpathlineto{\pgfqpoint{1.265580in}{1.690674in}}%
\pgfpathlineto{\pgfqpoint{1.309387in}{1.767198in}}%
\pgfpathlineto{\pgfqpoint{1.353195in}{1.839346in}}%
\pgfpathlineto{\pgfqpoint{1.397002in}{1.907128in}}%
\pgfpathlineto{\pgfqpoint{1.440810in}{1.970554in}}%
\pgfpathlineto{\pgfqpoint{1.484617in}{2.029633in}}%
\pgfpathlineto{\pgfqpoint{1.528425in}{2.084375in}}%
\pgfpathlineto{\pgfqpoint{1.572232in}{2.134787in}}%
\pgfpathlineto{\pgfqpoint{1.616040in}{2.180881in}}%
\pgfpathlineto{\pgfqpoint{1.659847in}{2.222665in}}%
\pgfpathlineto{\pgfqpoint{1.703655in}{2.260153in}}%
\pgfpathlineto{\pgfqpoint{1.747462in}{2.293356in}}%
\pgfpathlineto{\pgfqpoint{1.791269in}{2.322287in}}%
\pgfpathlineto{\pgfqpoint{1.835077in}{2.346957in}}%
\pgfpathlineto{\pgfqpoint{1.878884in}{2.367380in}}%
\pgfpathlineto{\pgfqpoint{1.922692in}{2.383567in}}%
\pgfpathlineto{\pgfqpoint{1.966499in}{2.395529in}}%
\pgfpathlineto{\pgfqpoint{2.010307in}{2.403277in}}%
\pgfpathlineto{\pgfqpoint{2.054114in}{2.406820in}}%
\pgfpathlineto{\pgfqpoint{2.097922in}{2.406163in}}%
\pgfpathlineto{\pgfqpoint{2.141729in}{2.401310in}}%
\pgfpathlineto{\pgfqpoint{2.185537in}{2.392259in}}%
\pgfpathlineto{\pgfqpoint{2.229344in}{2.379005in}}%
\pgfpathlineto{\pgfqpoint{2.273152in}{2.361536in}}%
\pgfpathlineto{\pgfqpoint{2.316959in}{2.339837in}}%
\pgfpathlineto{\pgfqpoint{2.360767in}{2.313886in}}%
\pgfpathlineto{\pgfqpoint{2.404574in}{2.283652in}}%
\pgfpathlineto{\pgfqpoint{2.448382in}{2.249093in}}%
\pgfpathlineto{\pgfqpoint{2.492189in}{2.210162in}}%
\pgfpathlineto{\pgfqpoint{2.535997in}{2.166798in}}%
\pgfpathlineto{\pgfqpoint{2.579804in}{2.118935in}}%
\pgfpathlineto{\pgfqpoint{2.623611in}{2.066498in}}%
\pgfpathlineto{\pgfqpoint{2.667419in}{2.009403in}}%
\pgfpathlineto{\pgfqpoint{2.711226in}{1.947561in}}%
\pgfpathlineto{\pgfqpoint{2.755034in}{1.880877in}}%
\pgfpathlineto{\pgfqpoint{2.798841in}{1.809275in}}%
\pgfpathlineto{\pgfqpoint{2.842649in}{1.732590in}}%
\pgfpathlineto{\pgfqpoint{2.886456in}{1.650724in}}%
\pgfpathlineto{\pgfqpoint{2.930264in}{1.563579in}}%
\pgfpathlineto{\pgfqpoint{2.974071in}{1.471057in}}%
\pgfpathlineto{\pgfqpoint{3.017879in}{1.373060in}}%
\pgfpathlineto{\pgfqpoint{3.061686in}{1.269489in}}%
\pgfpathlineto{\pgfqpoint{3.105494in}{1.160248in}}%
\pgfpathlineto{\pgfqpoint{3.149301in}{1.045237in}}%
\pgfpathlineto{\pgfqpoint{3.193109in}{0.924360in}}%
\pgfpathlineto{\pgfqpoint{3.236916in}{0.797517in}}%
\pgfusepath{stroke}%
\end{pgfscope}%
\begin{pgfscope}%
\pgfpathrectangle{\pgfqpoint{0.564660in}{0.521603in}}{\pgfqpoint{3.720000in}{3.020000in}}%
\pgfusepath{clip}%
\pgfsetrectcap%
\pgfsetroundjoin%
\pgfsetlinewidth{1.505625pt}%
\definecolor{currentstroke}{rgb}{0.131172,0.555899,0.552459}%
\pgfsetstrokecolor{currentstroke}%
\pgfsetdash{}{0pt}%
\pgfpathmoveto{\pgfqpoint{0.946461in}{0.834509in}}%
\pgfpathlineto{\pgfqpoint{0.988883in}{0.934924in}}%
\pgfpathlineto{\pgfqpoint{1.031305in}{1.030942in}}%
\pgfpathlineto{\pgfqpoint{1.073728in}{1.122608in}}%
\pgfpathlineto{\pgfqpoint{1.116150in}{1.209968in}}%
\pgfpathlineto{\pgfqpoint{1.158572in}{1.293067in}}%
\pgfpathlineto{\pgfqpoint{1.200994in}{1.371953in}}%
\pgfpathlineto{\pgfqpoint{1.243417in}{1.446670in}}%
\pgfpathlineto{\pgfqpoint{1.285839in}{1.517264in}}%
\pgfpathlineto{\pgfqpoint{1.328261in}{1.583780in}}%
\pgfpathlineto{\pgfqpoint{1.370683in}{1.646266in}}%
\pgfpathlineto{\pgfqpoint{1.413106in}{1.704766in}}%
\pgfpathlineto{\pgfqpoint{1.455528in}{1.759326in}}%
\pgfpathlineto{\pgfqpoint{1.497950in}{1.810015in}}%
\pgfpathlineto{\pgfqpoint{1.540372in}{1.856870in}}%
\pgfpathlineto{\pgfqpoint{1.582795in}{1.899937in}}%
\pgfpathlineto{\pgfqpoint{1.625217in}{1.939262in}}%
\pgfpathlineto{\pgfqpoint{1.667639in}{1.974887in}}%
\pgfpathlineto{\pgfqpoint{1.710062in}{2.006852in}}%
\pgfpathlineto{\pgfqpoint{1.752484in}{2.035195in}}%
\pgfpathlineto{\pgfqpoint{1.794906in}{2.059948in}}%
\pgfpathlineto{\pgfqpoint{1.837328in}{2.081141in}}%
\pgfpathlineto{\pgfqpoint{1.879751in}{2.098802in}}%
\pgfpathlineto{\pgfqpoint{1.922173in}{2.112950in}}%
\pgfpathlineto{\pgfqpoint{1.964595in}{2.123602in}}%
\pgfpathlineto{\pgfqpoint{2.007017in}{2.130770in}}%
\pgfpathlineto{\pgfqpoint{2.049440in}{2.134462in}}%
\pgfpathlineto{\pgfqpoint{2.091862in}{2.134679in}}%
\pgfpathlineto{\pgfqpoint{2.134284in}{2.131418in}}%
\pgfpathlineto{\pgfqpoint{2.176706in}{2.124673in}}%
\pgfpathlineto{\pgfqpoint{2.219129in}{2.114430in}}%
\pgfpathlineto{\pgfqpoint{2.261551in}{2.100668in}}%
\pgfpathlineto{\pgfqpoint{2.303973in}{2.083364in}}%
\pgfpathlineto{\pgfqpoint{2.346396in}{2.062487in}}%
\pgfpathlineto{\pgfqpoint{2.388818in}{2.037999in}}%
\pgfpathlineto{\pgfqpoint{2.431240in}{2.009850in}}%
\pgfpathlineto{\pgfqpoint{2.473662in}{1.977985in}}%
\pgfpathlineto{\pgfqpoint{2.516085in}{1.942336in}}%
\pgfpathlineto{\pgfqpoint{2.558507in}{1.902831in}}%
\pgfpathlineto{\pgfqpoint{2.600929in}{1.859388in}}%
\pgfpathlineto{\pgfqpoint{2.643351in}{1.811918in}}%
\pgfpathlineto{\pgfqpoint{2.685774in}{1.760326in}}%
\pgfpathlineto{\pgfqpoint{2.728196in}{1.704511in}}%
\pgfpathlineto{\pgfqpoint{2.770618in}{1.644390in}}%
\pgfpathlineto{\pgfqpoint{2.813041in}{1.579793in}}%
\pgfpathlineto{\pgfqpoint{2.855463in}{1.510617in}}%
\pgfpathlineto{\pgfqpoint{2.897885in}{1.436756in}}%
\pgfpathlineto{\pgfqpoint{2.940307in}{1.358108in}}%
\pgfpathlineto{\pgfqpoint{2.982730in}{1.274568in}}%
\pgfpathlineto{\pgfqpoint{3.025152in}{1.186032in}}%
\pgfpathlineto{\pgfqpoint{3.067574in}{1.092395in}}%
\pgfpathlineto{\pgfqpoint{3.109996in}{0.993555in}}%
\pgfpathlineto{\pgfqpoint{3.152419in}{0.889407in}}%
\pgfpathlineto{\pgfqpoint{3.194841in}{0.779846in}}%
\pgfusepath{stroke}%
\end{pgfscope}%
\begin{pgfscope}%
\pgfpathrectangle{\pgfqpoint{0.564660in}{0.521603in}}{\pgfqpoint{3.720000in}{3.020000in}}%
\pgfusepath{clip}%
\pgfsetrectcap%
\pgfsetroundjoin%
\pgfsetlinewidth{1.505625pt}%
\definecolor{currentstroke}{rgb}{0.192357,0.403199,0.555836}%
\pgfsetstrokecolor{currentstroke}%
\pgfsetdash{}{0pt}%
\pgfpathmoveto{\pgfqpoint{0.880673in}{1.105364in}}%
\pgfpathlineto{\pgfqpoint{0.920175in}{1.199183in}}%
\pgfpathlineto{\pgfqpoint{0.959676in}{1.289171in}}%
\pgfpathlineto{\pgfqpoint{0.999178in}{1.375366in}}%
\pgfpathlineto{\pgfqpoint{1.038679in}{1.457803in}}%
\pgfpathlineto{\pgfqpoint{1.078181in}{1.536521in}}%
\pgfpathlineto{\pgfqpoint{1.117683in}{1.611554in}}%
\pgfpathlineto{\pgfqpoint{1.157184in}{1.682940in}}%
\pgfpathlineto{\pgfqpoint{1.196686in}{1.750715in}}%
\pgfpathlineto{\pgfqpoint{1.236187in}{1.814915in}}%
\pgfpathlineto{\pgfqpoint{1.275689in}{1.875578in}}%
\pgfpathlineto{\pgfqpoint{1.315191in}{1.932740in}}%
\pgfpathlineto{\pgfqpoint{1.354692in}{1.986437in}}%
\pgfpathlineto{\pgfqpoint{1.394194in}{2.036717in}}%
\pgfpathlineto{\pgfqpoint{1.433695in}{2.083612in}}%
\pgfpathlineto{\pgfqpoint{1.473197in}{2.127160in}}%
\pgfpathlineto{\pgfqpoint{1.512698in}{2.167399in}}%
\pgfpathlineto{\pgfqpoint{1.552200in}{2.204364in}}%
\pgfpathlineto{\pgfqpoint{1.591702in}{2.238089in}}%
\pgfpathlineto{\pgfqpoint{1.631203in}{2.268605in}}%
\pgfpathlineto{\pgfqpoint{1.670705in}{2.295944in}}%
\pgfpathlineto{\pgfqpoint{1.710206in}{2.320132in}}%
\pgfpathlineto{\pgfqpoint{1.749708in}{2.341194in}}%
\pgfpathlineto{\pgfqpoint{1.789210in}{2.359153in}}%
\pgfpathlineto{\pgfqpoint{1.828711in}{2.374028in}}%
\pgfpathlineto{\pgfqpoint{1.868213in}{2.385835in}}%
\pgfpathlineto{\pgfqpoint{1.907714in}{2.394588in}}%
\pgfpathlineto{\pgfqpoint{1.947216in}{2.400298in}}%
\pgfpathlineto{\pgfqpoint{1.986718in}{2.402974in}}%
\pgfpathlineto{\pgfqpoint{2.026219in}{2.402622in}}%
\pgfpathlineto{\pgfqpoint{2.065721in}{2.399247in}}%
\pgfpathlineto{\pgfqpoint{2.105222in}{2.392852in}}%
\pgfpathlineto{\pgfqpoint{2.144724in}{2.383435in}}%
\pgfpathlineto{\pgfqpoint{2.184226in}{2.370997in}}%
\pgfpathlineto{\pgfqpoint{2.223727in}{2.355533in}}%
\pgfpathlineto{\pgfqpoint{2.263229in}{2.337035in}}%
\pgfpathlineto{\pgfqpoint{2.302730in}{2.315494in}}%
\pgfpathlineto{\pgfqpoint{2.342232in}{2.290894in}}%
\pgfpathlineto{\pgfqpoint{2.381734in}{2.263219in}}%
\pgfpathlineto{\pgfqpoint{2.421235in}{2.232447in}}%
\pgfpathlineto{\pgfqpoint{2.460737in}{2.198548in}}%
\pgfpathlineto{\pgfqpoint{2.500238in}{2.161482in}}%
\pgfpathlineto{\pgfqpoint{2.539740in}{2.121202in}}%
\pgfpathlineto{\pgfqpoint{2.579241in}{2.077655in}}%
\pgfpathlineto{\pgfqpoint{2.618743in}{2.030778in}}%
\pgfpathlineto{\pgfqpoint{2.658245in}{1.980499in}}%
\pgfpathlineto{\pgfqpoint{2.697746in}{1.926742in}}%
\pgfpathlineto{\pgfqpoint{2.737248in}{1.869423in}}%
\pgfpathlineto{\pgfqpoint{2.776749in}{1.808450in}}%
\pgfpathlineto{\pgfqpoint{2.816251in}{1.743754in}}%
\pgfpathlineto{\pgfqpoint{2.855753in}{1.675176in}}%
\pgfpathlineto{\pgfqpoint{2.895254in}{1.602623in}}%
\pgfpathlineto{\pgfqpoint{2.934756in}{1.526002in}}%
\pgfpathlineto{\pgfqpoint{2.974257in}{1.445220in}}%
\pgfpathlineto{\pgfqpoint{3.013759in}{1.360184in}}%
\pgfpathlineto{\pgfqpoint{3.053261in}{1.270802in}}%
\pgfpathlineto{\pgfqpoint{3.092762in}{1.176979in}}%
\pgfpathlineto{\pgfqpoint{3.132264in}{1.078623in}}%
\pgfpathlineto{\pgfqpoint{3.171765in}{0.975641in}}%
\pgfpathlineto{\pgfqpoint{3.211267in}{0.867940in}}%
\pgfpathlineto{\pgfqpoint{3.250769in}{0.755427in}}%
\pgfusepath{stroke}%
\end{pgfscope}%
\begin{pgfscope}%
\pgfpathrectangle{\pgfqpoint{0.564660in}{0.521603in}}{\pgfqpoint{3.720000in}{3.020000in}}%
\pgfusepath{clip}%
\pgfsetrectcap%
\pgfsetroundjoin%
\pgfsetlinewidth{1.505625pt}%
\definecolor{currentstroke}{rgb}{0.204903,0.375746,0.553533}%
\pgfsetstrokecolor{currentstroke}%
\pgfsetdash{}{0pt}%
\pgfpathmoveto{\pgfqpoint{0.777183in}{0.903667in}}%
\pgfpathlineto{\pgfqpoint{0.824410in}{1.028543in}}%
\pgfpathlineto{\pgfqpoint{0.871637in}{1.144664in}}%
\pgfpathlineto{\pgfqpoint{0.918864in}{1.252614in}}%
\pgfpathlineto{\pgfqpoint{0.966091in}{1.352977in}}%
\pgfpathlineto{\pgfqpoint{1.013319in}{1.446337in}}%
\pgfpathlineto{\pgfqpoint{1.060546in}{1.533277in}}%
\pgfpathlineto{\pgfqpoint{1.107773in}{1.614526in}}%
\pgfpathlineto{\pgfqpoint{1.155000in}{1.690645in}}%
\pgfpathlineto{\pgfqpoint{1.202227in}{1.762123in}}%
\pgfpathlineto{\pgfqpoint{1.249455in}{1.829368in}}%
\pgfpathlineto{\pgfqpoint{1.296682in}{1.892688in}}%
\pgfpathlineto{\pgfqpoint{1.343909in}{1.952300in}}%
\pgfpathlineto{\pgfqpoint{1.391136in}{2.008353in}}%
\pgfpathlineto{\pgfqpoint{1.438363in}{2.060952in}}%
\pgfpathlineto{\pgfqpoint{1.485591in}{2.110180in}}%
\pgfpathlineto{\pgfqpoint{1.532818in}{2.156125in}}%
\pgfpathlineto{\pgfqpoint{1.580045in}{2.198865in}}%
\pgfpathlineto{\pgfqpoint{1.627272in}{2.238491in}}%
\pgfpathlineto{\pgfqpoint{1.674499in}{2.275099in}}%
\pgfpathlineto{\pgfqpoint{1.721727in}{2.308759in}}%
\pgfpathlineto{\pgfqpoint{1.768954in}{2.339537in}}%
\pgfpathlineto{\pgfqpoint{1.816181in}{2.367501in}}%
\pgfpathlineto{\pgfqpoint{1.863408in}{2.392699in}}%
\pgfpathlineto{\pgfqpoint{1.910635in}{2.415187in}}%
\pgfpathlineto{\pgfqpoint{1.957863in}{2.435022in}}%
\pgfpathlineto{\pgfqpoint{2.005090in}{2.452236in}}%
\pgfpathlineto{\pgfqpoint{2.052317in}{2.466867in}}%
\pgfpathlineto{\pgfqpoint{2.099544in}{2.478946in}}%
\pgfpathlineto{\pgfqpoint{2.146771in}{2.488493in}}%
\pgfpathlineto{\pgfqpoint{2.193998in}{2.495539in}}%
\pgfpathlineto{\pgfqpoint{2.241226in}{2.500111in}}%
\pgfpathlineto{\pgfqpoint{2.288453in}{2.502214in}}%
\pgfpathlineto{\pgfqpoint{2.335680in}{2.501856in}}%
\pgfpathlineto{\pgfqpoint{2.382907in}{2.499038in}}%
\pgfpathlineto{\pgfqpoint{2.430134in}{2.493748in}}%
\pgfpathlineto{\pgfqpoint{2.477362in}{2.485987in}}%
\pgfpathlineto{\pgfqpoint{2.524589in}{2.475751in}}%
\pgfpathlineto{\pgfqpoint{2.571816in}{2.463015in}}%
\pgfpathlineto{\pgfqpoint{2.619043in}{2.447756in}}%
\pgfpathlineto{\pgfqpoint{2.666270in}{2.429945in}}%
\pgfpathlineto{\pgfqpoint{2.713498in}{2.409538in}}%
\pgfpathlineto{\pgfqpoint{2.760725in}{2.386510in}}%
\pgfpathlineto{\pgfqpoint{2.807952in}{2.360822in}}%
\pgfpathlineto{\pgfqpoint{2.855179in}{2.332412in}}%
\pgfpathlineto{\pgfqpoint{2.902406in}{2.301223in}}%
\pgfpathlineto{\pgfqpoint{2.949634in}{2.267176in}}%
\pgfpathlineto{\pgfqpoint{2.996861in}{2.230185in}}%
\pgfpathlineto{\pgfqpoint{3.044088in}{2.190174in}}%
\pgfpathlineto{\pgfqpoint{3.091315in}{2.147045in}}%
\pgfpathlineto{\pgfqpoint{3.138542in}{2.100688in}}%
\pgfpathlineto{\pgfqpoint{3.185770in}{2.050993in}}%
\pgfpathlineto{\pgfqpoint{3.232997in}{1.997837in}}%
\pgfpathlineto{\pgfqpoint{3.280224in}{1.941090in}}%
\pgfpathlineto{\pgfqpoint{3.327451in}{1.880616in}}%
\pgfpathlineto{\pgfqpoint{3.374678in}{1.816238in}}%
\pgfpathlineto{\pgfqpoint{3.421906in}{1.747737in}}%
\pgfpathlineto{\pgfqpoint{3.469133in}{1.674840in}}%
\pgfpathlineto{\pgfqpoint{3.516360in}{1.597184in}}%
\pgfpathlineto{\pgfqpoint{3.563587in}{1.514330in}}%
\pgfpathlineto{\pgfqpoint{3.610814in}{1.425765in}}%
\pgfpathlineto{\pgfqpoint{3.658041in}{1.330913in}}%
\pgfpathlineto{\pgfqpoint{3.705269in}{1.229080in}}%
\pgfpathlineto{\pgfqpoint{3.752496in}{1.119610in}}%
\pgfpathlineto{\pgfqpoint{3.799723in}{1.001890in}}%
\pgfpathlineto{\pgfqpoint{3.846950in}{0.875303in}}%
\pgfpathlineto{\pgfqpoint{3.894177in}{0.739236in}}%
\pgfpathlineto{\pgfqpoint{3.894177in}{0.739236in}}%
\pgfusepath{stroke}%
\end{pgfscope}%
\begin{pgfscope}%
\pgfpathrectangle{\pgfqpoint{0.564660in}{0.521603in}}{\pgfqpoint{3.720000in}{3.020000in}}%
\pgfusepath{clip}%
\pgfsetrectcap%
\pgfsetroundjoin%
\pgfsetlinewidth{1.505625pt}%
\definecolor{currentstroke}{rgb}{0.206756,0.371758,0.553117}%
\pgfsetstrokecolor{currentstroke}%
\pgfsetdash{}{0pt}%
\pgfpathmoveto{\pgfqpoint{0.760917in}{0.842885in}}%
\pgfpathlineto{\pgfqpoint{0.788954in}{0.915650in}}%
\pgfpathlineto{\pgfqpoint{0.816990in}{0.986020in}}%
\pgfpathlineto{\pgfqpoint{0.845027in}{1.054055in}}%
\pgfpathlineto{\pgfqpoint{0.873064in}{1.119814in}}%
\pgfpathlineto{\pgfqpoint{0.901100in}{1.183358in}}%
\pgfpathlineto{\pgfqpoint{0.929137in}{1.244745in}}%
\pgfpathlineto{\pgfqpoint{0.957174in}{1.304035in}}%
\pgfpathlineto{\pgfqpoint{0.985210in}{1.361287in}}%
\pgfpathlineto{\pgfqpoint{1.013247in}{1.416561in}}%
\pgfpathlineto{\pgfqpoint{1.041283in}{1.469916in}}%
\pgfpathlineto{\pgfqpoint{1.069320in}{1.521412in}}%
\pgfpathlineto{\pgfqpoint{1.097357in}{1.571108in}}%
\pgfpathlineto{\pgfqpoint{1.125393in}{1.619107in}}%
\pgfpathlineto{\pgfqpoint{1.153430in}{1.665454in}}%
\pgfpathlineto{\pgfqpoint{1.181467in}{1.710206in}}%
\pgfpathlineto{\pgfqpoint{1.209503in}{1.753415in}}%
\pgfpathlineto{\pgfqpoint{1.237540in}{1.795130in}}%
\pgfpathlineto{\pgfqpoint{1.265577in}{1.835393in}}%
\pgfpathlineto{\pgfqpoint{1.293613in}{1.874245in}}%
\pgfpathlineto{\pgfqpoint{1.321650in}{1.911718in}}%
\pgfpathlineto{\pgfqpoint{1.349687in}{1.947842in}}%
\pgfpathlineto{\pgfqpoint{1.377723in}{1.982638in}}%
\pgfpathlineto{\pgfqpoint{1.405760in}{2.016128in}}%
\pgfpathlineto{\pgfqpoint{1.433797in}{2.048325in}}%
\pgfpathlineto{\pgfqpoint{1.461833in}{2.079241in}}%
\pgfpathlineto{\pgfqpoint{1.489870in}{2.108889in}}%
\pgfpathlineto{\pgfqpoint{1.517907in}{2.137277in}}%
\pgfpathlineto{\pgfqpoint{1.545943in}{2.164419in}}%
\pgfpathlineto{\pgfqpoint{1.573980in}{2.190323in}}%
\pgfpathlineto{\pgfqpoint{1.602017in}{2.215000in}}%
\pgfpathlineto{\pgfqpoint{1.630053in}{2.238457in}}%
\pgfpathlineto{\pgfqpoint{1.658090in}{2.260704in}}%
\pgfpathlineto{\pgfqpoint{1.686127in}{2.281750in}}%
\pgfpathlineto{\pgfqpoint{1.714163in}{2.301607in}}%
\pgfpathlineto{\pgfqpoint{1.742200in}{2.320281in}}%
\pgfpathlineto{\pgfqpoint{1.770237in}{2.337783in}}%
\pgfpathlineto{\pgfqpoint{1.798273in}{2.354122in}}%
\pgfpathlineto{\pgfqpoint{1.826310in}{2.369308in}}%
\pgfpathlineto{\pgfqpoint{1.854347in}{2.383351in}}%
\pgfpathlineto{\pgfqpoint{1.882383in}{2.396257in}}%
\pgfpathlineto{\pgfqpoint{1.910420in}{2.408035in}}%
\pgfpathlineto{\pgfqpoint{1.938457in}{2.418693in}}%
\pgfpathlineto{\pgfqpoint{1.966493in}{2.428238in}}%
\pgfpathlineto{\pgfqpoint{1.994530in}{2.436674in}}%
\pgfpathlineto{\pgfqpoint{2.022567in}{2.444005in}}%
\pgfpathlineto{\pgfqpoint{2.050603in}{2.450235in}}%
\pgfpathlineto{\pgfqpoint{2.078640in}{2.455370in}}%
\pgfpathlineto{\pgfqpoint{2.106677in}{2.459411in}}%
\pgfpathlineto{\pgfqpoint{2.134713in}{2.462360in}}%
\pgfpathlineto{\pgfqpoint{2.162750in}{2.464218in}}%
\pgfpathlineto{\pgfqpoint{2.190787in}{2.464987in}}%
\pgfpathlineto{\pgfqpoint{2.218823in}{2.464668in}}%
\pgfpathlineto{\pgfqpoint{2.246860in}{2.463261in}}%
\pgfpathlineto{\pgfqpoint{2.274897in}{2.460761in}}%
\pgfpathlineto{\pgfqpoint{2.302933in}{2.457169in}}%
\pgfpathlineto{\pgfqpoint{2.330970in}{2.452482in}}%
\pgfpathlineto{\pgfqpoint{2.359006in}{2.446698in}}%
\pgfpathlineto{\pgfqpoint{2.387043in}{2.439814in}}%
\pgfpathlineto{\pgfqpoint{2.415080in}{2.431826in}}%
\pgfpathlineto{\pgfqpoint{2.443116in}{2.422731in}}%
\pgfpathlineto{\pgfqpoint{2.471153in}{2.412527in}}%
\pgfpathlineto{\pgfqpoint{2.499190in}{2.401211in}}%
\pgfpathlineto{\pgfqpoint{2.527226in}{2.388777in}}%
\pgfpathlineto{\pgfqpoint{2.555263in}{2.375221in}}%
\pgfpathlineto{\pgfqpoint{2.583300in}{2.360538in}}%
\pgfpathlineto{\pgfqpoint{2.611336in}{2.344725in}}%
\pgfpathlineto{\pgfqpoint{2.639373in}{2.327775in}}%
\pgfpathlineto{\pgfqpoint{2.667410in}{2.309681in}}%
\pgfpathlineto{\pgfqpoint{2.695446in}{2.290436in}}%
\pgfpathlineto{\pgfqpoint{2.723483in}{2.270033in}}%
\pgfpathlineto{\pgfqpoint{2.751520in}{2.248463in}}%
\pgfpathlineto{\pgfqpoint{2.779556in}{2.225716in}}%
\pgfpathlineto{\pgfqpoint{2.807593in}{2.201781in}}%
\pgfpathlineto{\pgfqpoint{2.835630in}{2.176646in}}%
\pgfpathlineto{\pgfqpoint{2.863666in}{2.150300in}}%
\pgfpathlineto{\pgfqpoint{2.891703in}{2.122729in}}%
\pgfpathlineto{\pgfqpoint{2.919740in}{2.093919in}}%
\pgfpathlineto{\pgfqpoint{2.947776in}{2.063854in}}%
\pgfpathlineto{\pgfqpoint{2.975813in}{2.032517in}}%
\pgfpathlineto{\pgfqpoint{3.003850in}{1.999893in}}%
\pgfpathlineto{\pgfqpoint{3.031886in}{1.965961in}}%
\pgfpathlineto{\pgfqpoint{3.059923in}{1.930700in}}%
\pgfpathlineto{\pgfqpoint{3.087960in}{1.894087in}}%
\pgfpathlineto{\pgfqpoint{3.115996in}{1.856099in}}%
\pgfpathlineto{\pgfqpoint{3.144033in}{1.816708in}}%
\pgfpathlineto{\pgfqpoint{3.172070in}{1.775884in}}%
\pgfpathlineto{\pgfqpoint{3.200106in}{1.733591in}}%
\pgfpathlineto{\pgfqpoint{3.228143in}{1.689794in}}%
\pgfpathlineto{\pgfqpoint{3.256180in}{1.644451in}}%
\pgfpathlineto{\pgfqpoint{3.284216in}{1.597521in}}%
\pgfpathlineto{\pgfqpoint{3.312253in}{1.548956in}}%
\pgfpathlineto{\pgfqpoint{3.340290in}{1.498708in}}%
\pgfpathlineto{\pgfqpoint{3.368326in}{1.446725in}}%
\pgfpathlineto{\pgfqpoint{3.396363in}{1.392964in}}%
\pgfpathlineto{\pgfqpoint{3.424400in}{1.337346in}}%
\pgfpathlineto{\pgfqpoint{3.452436in}{1.279820in}}%
\pgfpathlineto{\pgfqpoint{3.480473in}{1.220332in}}%
\pgfpathlineto{\pgfqpoint{3.508510in}{1.158830in}}%
\pgfpathlineto{\pgfqpoint{3.536546in}{1.095262in}}%
\pgfpathlineto{\pgfqpoint{3.564583in}{1.029575in}}%
\pgfpathlineto{\pgfqpoint{3.592620in}{0.961718in}}%
\pgfpathlineto{\pgfqpoint{3.620656in}{0.891637in}}%
\pgfpathlineto{\pgfqpoint{3.648693in}{0.819280in}}%
\pgfpathlineto{\pgfqpoint{3.676729in}{0.744596in}}%
\pgfusepath{stroke}%
\end{pgfscope}%
\begin{pgfscope}%
\pgfpathrectangle{\pgfqpoint{0.564660in}{0.521603in}}{\pgfqpoint{3.720000in}{3.020000in}}%
\pgfusepath{clip}%
\pgfsetrectcap%
\pgfsetroundjoin%
\pgfsetlinewidth{1.505625pt}%
\definecolor{currentstroke}{rgb}{0.282656,0.100196,0.422160}%
\pgfsetstrokecolor{currentstroke}%
\pgfsetdash{}{0pt}%
\pgfpathmoveto{\pgfqpoint{0.712990in}{0.794843in}}%
\pgfpathlineto{\pgfqpoint{0.768614in}{0.924476in}}%
\pgfpathlineto{\pgfqpoint{0.824238in}{1.046999in}}%
\pgfpathlineto{\pgfqpoint{0.879862in}{1.162599in}}%
\pgfpathlineto{\pgfqpoint{0.935485in}{1.271466in}}%
\pgfpathlineto{\pgfqpoint{0.991109in}{1.373802in}}%
\pgfpathlineto{\pgfqpoint{1.046733in}{1.469785in}}%
\pgfpathlineto{\pgfqpoint{1.102357in}{1.559566in}}%
\pgfpathlineto{\pgfqpoint{1.157980in}{1.643285in}}%
\pgfpathlineto{\pgfqpoint{1.213604in}{1.721089in}}%
\pgfpathlineto{\pgfqpoint{1.269228in}{1.793159in}}%
\pgfpathlineto{\pgfqpoint{1.324852in}{1.859702in}}%
\pgfpathlineto{\pgfqpoint{1.380475in}{1.920924in}}%
\pgfpathlineto{\pgfqpoint{1.436099in}{1.977014in}}%
\pgfpathlineto{\pgfqpoint{1.491723in}{2.028122in}}%
\pgfpathlineto{\pgfqpoint{1.547347in}{2.074359in}}%
\pgfpathlineto{\pgfqpoint{1.602970in}{2.115802in}}%
\pgfpathlineto{\pgfqpoint{1.658594in}{2.152519in}}%
\pgfpathlineto{\pgfqpoint{1.714218in}{2.184578in}}%
\pgfpathlineto{\pgfqpoint{1.769842in}{2.212052in}}%
\pgfpathlineto{\pgfqpoint{1.825465in}{2.235017in}}%
\pgfpathlineto{\pgfqpoint{1.881089in}{2.253552in}}%
\pgfpathlineto{\pgfqpoint{1.936713in}{2.267724in}}%
\pgfpathlineto{\pgfqpoint{1.992337in}{2.277577in}}%
\pgfpathlineto{\pgfqpoint{2.047960in}{2.283134in}}%
\pgfpathlineto{\pgfqpoint{2.103584in}{2.284398in}}%
\pgfpathlineto{\pgfqpoint{2.159208in}{2.281353in}}%
\pgfpathlineto{\pgfqpoint{2.214832in}{2.273974in}}%
\pgfpathlineto{\pgfqpoint{2.270455in}{2.262231in}}%
\pgfpathlineto{\pgfqpoint{2.326079in}{2.246084in}}%
\pgfpathlineto{\pgfqpoint{2.381703in}{2.225476in}}%
\pgfpathlineto{\pgfqpoint{2.437327in}{2.200336in}}%
\pgfpathlineto{\pgfqpoint{2.492950in}{2.170586in}}%
\pgfpathlineto{\pgfqpoint{2.548574in}{2.136148in}}%
\pgfpathlineto{\pgfqpoint{2.604198in}{2.096952in}}%
\pgfpathlineto{\pgfqpoint{2.659821in}{2.052936in}}%
\pgfpathlineto{\pgfqpoint{2.715445in}{2.004040in}}%
\pgfpathlineto{\pgfqpoint{2.771069in}{1.950187in}}%
\pgfpathlineto{\pgfqpoint{2.826693in}{1.891268in}}%
\pgfpathlineto{\pgfqpoint{2.882316in}{1.827141in}}%
\pgfpathlineto{\pgfqpoint{2.937940in}{1.757638in}}%
\pgfpathlineto{\pgfqpoint{2.993564in}{1.682578in}}%
\pgfpathlineto{\pgfqpoint{3.049188in}{1.601776in}}%
\pgfpathlineto{\pgfqpoint{3.104811in}{1.515043in}}%
\pgfpathlineto{\pgfqpoint{3.160435in}{1.422172in}}%
\pgfpathlineto{\pgfqpoint{3.216059in}{1.322907in}}%
\pgfpathlineto{\pgfqpoint{3.271683in}{1.216926in}}%
\pgfpathlineto{\pgfqpoint{3.327306in}{1.103849in}}%
\pgfpathlineto{\pgfqpoint{3.382930in}{0.983232in}}%
\pgfpathlineto{\pgfqpoint{3.438554in}{0.854656in}}%
\pgfpathlineto{\pgfqpoint{3.494178in}{0.717710in}}%
\pgfpathlineto{\pgfqpoint{3.512719in}{0.670130in}}%
\pgfpathlineto{\pgfqpoint{3.512719in}{0.670130in}}%
\pgfusepath{stroke}%
\end{pgfscope}%
\begin{pgfscope}%
\pgfsetrectcap%
\pgfsetmiterjoin%
\pgfsetlinewidth{0.803000pt}%
\definecolor{currentstroke}{rgb}{0.501961,0.501961,0.501961}%
\pgfsetstrokecolor{currentstroke}%
\pgfsetdash{}{0pt}%
\pgfpathmoveto{\pgfqpoint{0.564660in}{0.521603in}}%
\pgfpathlineto{\pgfqpoint{0.564660in}{3.541603in}}%
\pgfusepath{stroke}%
\end{pgfscope}%
\begin{pgfscope}%
\pgfsetrectcap%
\pgfsetmiterjoin%
\pgfsetlinewidth{0.803000pt}%
\definecolor{currentstroke}{rgb}{0.501961,0.501961,0.501961}%
\pgfsetstrokecolor{currentstroke}%
\pgfsetdash{}{0pt}%
\pgfpathmoveto{\pgfqpoint{4.284660in}{0.521603in}}%
\pgfpathlineto{\pgfqpoint{4.284660in}{3.541603in}}%
\pgfusepath{stroke}%
\end{pgfscope}%
\begin{pgfscope}%
\pgfsetrectcap%
\pgfsetmiterjoin%
\pgfsetlinewidth{0.803000pt}%
\definecolor{currentstroke}{rgb}{0.501961,0.501961,0.501961}%
\pgfsetstrokecolor{currentstroke}%
\pgfsetdash{}{0pt}%
\pgfpathmoveto{\pgfqpoint{0.564660in}{0.521603in}}%
\pgfpathlineto{\pgfqpoint{4.284660in}{0.521603in}}%
\pgfusepath{stroke}%
\end{pgfscope}%
\begin{pgfscope}%
\pgfsetrectcap%
\pgfsetmiterjoin%
\pgfsetlinewidth{0.803000pt}%
\definecolor{currentstroke}{rgb}{0.501961,0.501961,0.501961}%
\pgfsetstrokecolor{currentstroke}%
\pgfsetdash{}{0pt}%
\pgfpathmoveto{\pgfqpoint{0.564660in}{3.541603in}}%
\pgfpathlineto{\pgfqpoint{4.284660in}{3.541603in}}%
\pgfusepath{stroke}%
\end{pgfscope}%
\begin{pgfscope}%
\pgfpathrectangle{\pgfqpoint{4.517160in}{0.521603in}}{\pgfqpoint{0.151000in}{3.020000in}}%
\pgfusepath{clip}%
\pgfsetbuttcap%
\pgfsetmiterjoin%
\definecolor{currentfill}{rgb}{1.000000,1.000000,1.000000}%
\pgfsetfillcolor{currentfill}%
\pgfsetlinewidth{0.010037pt}%
\definecolor{currentstroke}{rgb}{1.000000,1.000000,1.000000}%
\pgfsetstrokecolor{currentstroke}%
\pgfsetdash{}{0pt}%
\pgfpathmoveto{\pgfqpoint{4.517160in}{0.521603in}}%
\pgfpathlineto{\pgfqpoint{4.517160in}{0.533400in}}%
\pgfpathlineto{\pgfqpoint{4.517160in}{3.529806in}}%
\pgfpathlineto{\pgfqpoint{4.517160in}{3.541603in}}%
\pgfpathlineto{\pgfqpoint{4.668160in}{3.541603in}}%
\pgfpathlineto{\pgfqpoint{4.668160in}{3.529806in}}%
\pgfpathlineto{\pgfqpoint{4.668160in}{0.533400in}}%
\pgfpathlineto{\pgfqpoint{4.668160in}{0.521603in}}%
\pgfpathclose%
\pgfusepath{stroke,fill}%
\end{pgfscope}%
\begin{pgfscope}%
\pgfsys@transformshift{4.513889in}{0.537437in}%
\pgftext[left,bottom]{\pgfimage[interpolate=true,width=0.152778in,height=3.013889in]{series_s_ds-img0.png}}%
\end{pgfscope}%
\begin{pgfscope}%
\pgfsetbuttcap%
\pgfsetroundjoin%
\definecolor{currentfill}{rgb}{0.000000,0.000000,0.000000}%
\pgfsetfillcolor{currentfill}%
\pgfsetlinewidth{0.803000pt}%
\definecolor{currentstroke}{rgb}{0.000000,0.000000,0.000000}%
\pgfsetstrokecolor{currentstroke}%
\pgfsetdash{}{0pt}%
\pgfsys@defobject{currentmarker}{\pgfqpoint{0.000000in}{0.000000in}}{\pgfqpoint{0.048611in}{0.000000in}}{%
\pgfpathmoveto{\pgfqpoint{0.000000in}{0.000000in}}%
\pgfpathlineto{\pgfqpoint{0.048611in}{0.000000in}}%
\pgfusepath{stroke,fill}%
}%
\begin{pgfscope}%
\pgfsys@transformshift{4.668160in}{0.687294in}%
\pgfsys@useobject{currentmarker}{}%
\end{pgfscope}%
\end{pgfscope}%
\begin{pgfscope}%
\definecolor{textcolor}{rgb}{0.000000,0.000000,0.000000}%
\pgfsetstrokecolor{textcolor}%
\pgfsetfillcolor{textcolor}%
\pgftext[x=4.765383in,y=0.634533in,left,base]{\color{textcolor}\rmfamily\fontsize{10.000000}{12.000000}\selectfont \(\displaystyle 10^{0}\)}%
\end{pgfscope}%
\begin{pgfscope}%
\pgfsetbuttcap%
\pgfsetroundjoin%
\definecolor{currentfill}{rgb}{0.000000,0.000000,0.000000}%
\pgfsetfillcolor{currentfill}%
\pgfsetlinewidth{0.803000pt}%
\definecolor{currentstroke}{rgb}{0.000000,0.000000,0.000000}%
\pgfsetstrokecolor{currentstroke}%
\pgfsetdash{}{0pt}%
\pgfsys@defobject{currentmarker}{\pgfqpoint{0.000000in}{0.000000in}}{\pgfqpoint{0.048611in}{0.000000in}}{%
\pgfpathmoveto{\pgfqpoint{0.000000in}{0.000000in}}%
\pgfpathlineto{\pgfqpoint{0.048611in}{0.000000in}}%
\pgfusepath{stroke,fill}%
}%
\begin{pgfscope}%
\pgfsys@transformshift{4.668160in}{3.444394in}%
\pgfsys@useobject{currentmarker}{}%
\end{pgfscope}%
\end{pgfscope}%
\begin{pgfscope}%
\definecolor{textcolor}{rgb}{0.000000,0.000000,0.000000}%
\pgfsetstrokecolor{textcolor}%
\pgfsetfillcolor{textcolor}%
\pgftext[x=4.765383in,y=3.391633in,left,base]{\color{textcolor}\rmfamily\fontsize{10.000000}{12.000000}\selectfont \(\displaystyle 10^{1}\)}%
\end{pgfscope}%
\begin{pgfscope}%
\pgfsetbuttcap%
\pgfsetroundjoin%
\definecolor{currentfill}{rgb}{0.000000,0.000000,0.000000}%
\pgfsetfillcolor{currentfill}%
\pgfsetlinewidth{0.602250pt}%
\definecolor{currentstroke}{rgb}{0.000000,0.000000,0.000000}%
\pgfsetstrokecolor{currentstroke}%
\pgfsetdash{}{0pt}%
\pgfsys@defobject{currentmarker}{\pgfqpoint{0.000000in}{0.000000in}}{\pgfqpoint{0.027778in}{0.000000in}}{%
\pgfpathmoveto{\pgfqpoint{0.000000in}{0.000000in}}%
\pgfpathlineto{\pgfqpoint{0.027778in}{0.000000in}}%
\pgfusepath{stroke,fill}%
}%
\begin{pgfscope}%
\pgfsys@transformshift{4.668160in}{0.561136in}%
\pgfsys@useobject{currentmarker}{}%
\end{pgfscope}%
\end{pgfscope}%
\begin{pgfscope}%
\pgfsetbuttcap%
\pgfsetroundjoin%
\definecolor{currentfill}{rgb}{0.000000,0.000000,0.000000}%
\pgfsetfillcolor{currentfill}%
\pgfsetlinewidth{0.602250pt}%
\definecolor{currentstroke}{rgb}{0.000000,0.000000,0.000000}%
\pgfsetstrokecolor{currentstroke}%
\pgfsetdash{}{0pt}%
\pgfsys@defobject{currentmarker}{\pgfqpoint{0.000000in}{0.000000in}}{\pgfqpoint{0.027778in}{0.000000in}}{%
\pgfpathmoveto{\pgfqpoint{0.000000in}{0.000000in}}%
\pgfpathlineto{\pgfqpoint{0.027778in}{0.000000in}}%
\pgfusepath{stroke,fill}%
}%
\begin{pgfscope}%
\pgfsys@transformshift{4.668160in}{1.517264in}%
\pgfsys@useobject{currentmarker}{}%
\end{pgfscope}%
\end{pgfscope}%
\begin{pgfscope}%
\pgfsetbuttcap%
\pgfsetroundjoin%
\definecolor{currentfill}{rgb}{0.000000,0.000000,0.000000}%
\pgfsetfillcolor{currentfill}%
\pgfsetlinewidth{0.602250pt}%
\definecolor{currentstroke}{rgb}{0.000000,0.000000,0.000000}%
\pgfsetstrokecolor{currentstroke}%
\pgfsetdash{}{0pt}%
\pgfsys@defobject{currentmarker}{\pgfqpoint{0.000000in}{0.000000in}}{\pgfqpoint{0.027778in}{0.000000in}}{%
\pgfpathmoveto{\pgfqpoint{0.000000in}{0.000000in}}%
\pgfpathlineto{\pgfqpoint{0.027778in}{0.000000in}}%
\pgfusepath{stroke,fill}%
}%
\begin{pgfscope}%
\pgfsys@transformshift{4.668160in}{2.002765in}%
\pgfsys@useobject{currentmarker}{}%
\end{pgfscope}%
\end{pgfscope}%
\begin{pgfscope}%
\pgfsetbuttcap%
\pgfsetroundjoin%
\definecolor{currentfill}{rgb}{0.000000,0.000000,0.000000}%
\pgfsetfillcolor{currentfill}%
\pgfsetlinewidth{0.602250pt}%
\definecolor{currentstroke}{rgb}{0.000000,0.000000,0.000000}%
\pgfsetstrokecolor{currentstroke}%
\pgfsetdash{}{0pt}%
\pgfsys@defobject{currentmarker}{\pgfqpoint{0.000000in}{0.000000in}}{\pgfqpoint{0.027778in}{0.000000in}}{%
\pgfpathmoveto{\pgfqpoint{0.000000in}{0.000000in}}%
\pgfpathlineto{\pgfqpoint{0.027778in}{0.000000in}}%
\pgfusepath{stroke,fill}%
}%
\begin{pgfscope}%
\pgfsys@transformshift{4.668160in}{2.347234in}%
\pgfsys@useobject{currentmarker}{}%
\end{pgfscope}%
\end{pgfscope}%
\begin{pgfscope}%
\pgfsetbuttcap%
\pgfsetroundjoin%
\definecolor{currentfill}{rgb}{0.000000,0.000000,0.000000}%
\pgfsetfillcolor{currentfill}%
\pgfsetlinewidth{0.602250pt}%
\definecolor{currentstroke}{rgb}{0.000000,0.000000,0.000000}%
\pgfsetstrokecolor{currentstroke}%
\pgfsetdash{}{0pt}%
\pgfsys@defobject{currentmarker}{\pgfqpoint{0.000000in}{0.000000in}}{\pgfqpoint{0.027778in}{0.000000in}}{%
\pgfpathmoveto{\pgfqpoint{0.000000in}{0.000000in}}%
\pgfpathlineto{\pgfqpoint{0.027778in}{0.000000in}}%
\pgfusepath{stroke,fill}%
}%
\begin{pgfscope}%
\pgfsys@transformshift{4.668160in}{2.614424in}%
\pgfsys@useobject{currentmarker}{}%
\end{pgfscope}%
\end{pgfscope}%
\begin{pgfscope}%
\pgfsetbuttcap%
\pgfsetroundjoin%
\definecolor{currentfill}{rgb}{0.000000,0.000000,0.000000}%
\pgfsetfillcolor{currentfill}%
\pgfsetlinewidth{0.602250pt}%
\definecolor{currentstroke}{rgb}{0.000000,0.000000,0.000000}%
\pgfsetstrokecolor{currentstroke}%
\pgfsetdash{}{0pt}%
\pgfsys@defobject{currentmarker}{\pgfqpoint{0.000000in}{0.000000in}}{\pgfqpoint{0.027778in}{0.000000in}}{%
\pgfpathmoveto{\pgfqpoint{0.000000in}{0.000000in}}%
\pgfpathlineto{\pgfqpoint{0.027778in}{0.000000in}}%
\pgfusepath{stroke,fill}%
}%
\begin{pgfscope}%
\pgfsys@transformshift{4.668160in}{2.832735in}%
\pgfsys@useobject{currentmarker}{}%
\end{pgfscope}%
\end{pgfscope}%
\begin{pgfscope}%
\pgfsetbuttcap%
\pgfsetroundjoin%
\definecolor{currentfill}{rgb}{0.000000,0.000000,0.000000}%
\pgfsetfillcolor{currentfill}%
\pgfsetlinewidth{0.602250pt}%
\definecolor{currentstroke}{rgb}{0.000000,0.000000,0.000000}%
\pgfsetstrokecolor{currentstroke}%
\pgfsetdash{}{0pt}%
\pgfsys@defobject{currentmarker}{\pgfqpoint{0.000000in}{0.000000in}}{\pgfqpoint{0.027778in}{0.000000in}}{%
\pgfpathmoveto{\pgfqpoint{0.000000in}{0.000000in}}%
\pgfpathlineto{\pgfqpoint{0.027778in}{0.000000in}}%
\pgfusepath{stroke,fill}%
}%
\begin{pgfscope}%
\pgfsys@transformshift{4.668160in}{3.017314in}%
\pgfsys@useobject{currentmarker}{}%
\end{pgfscope}%
\end{pgfscope}%
\begin{pgfscope}%
\pgfsetbuttcap%
\pgfsetroundjoin%
\definecolor{currentfill}{rgb}{0.000000,0.000000,0.000000}%
\pgfsetfillcolor{currentfill}%
\pgfsetlinewidth{0.602250pt}%
\definecolor{currentstroke}{rgb}{0.000000,0.000000,0.000000}%
\pgfsetstrokecolor{currentstroke}%
\pgfsetdash{}{0pt}%
\pgfsys@defobject{currentmarker}{\pgfqpoint{0.000000in}{0.000000in}}{\pgfqpoint{0.027778in}{0.000000in}}{%
\pgfpathmoveto{\pgfqpoint{0.000000in}{0.000000in}}%
\pgfpathlineto{\pgfqpoint{0.027778in}{0.000000in}}%
\pgfusepath{stroke,fill}%
}%
\begin{pgfscope}%
\pgfsys@transformshift{4.668160in}{3.177204in}%
\pgfsys@useobject{currentmarker}{}%
\end{pgfscope}%
\end{pgfscope}%
\begin{pgfscope}%
\pgfsetbuttcap%
\pgfsetroundjoin%
\definecolor{currentfill}{rgb}{0.000000,0.000000,0.000000}%
\pgfsetfillcolor{currentfill}%
\pgfsetlinewidth{0.602250pt}%
\definecolor{currentstroke}{rgb}{0.000000,0.000000,0.000000}%
\pgfsetstrokecolor{currentstroke}%
\pgfsetdash{}{0pt}%
\pgfsys@defobject{currentmarker}{\pgfqpoint{0.000000in}{0.000000in}}{\pgfqpoint{0.027778in}{0.000000in}}{%
\pgfpathmoveto{\pgfqpoint{0.000000in}{0.000000in}}%
\pgfpathlineto{\pgfqpoint{0.027778in}{0.000000in}}%
\pgfusepath{stroke,fill}%
}%
\begin{pgfscope}%
\pgfsys@transformshift{4.668160in}{3.318236in}%
\pgfsys@useobject{currentmarker}{}%
\end{pgfscope}%
\end{pgfscope}%
\begin{pgfscope}%
\definecolor{textcolor}{rgb}{0.000000,0.000000,0.000000}%
\pgfsetstrokecolor{textcolor}%
\pgfsetfillcolor{textcolor}%
\pgftext[x=5.077690in,y=2.031603in,,top]{\color{textcolor}\rmfamily\fontsize{14.000000}{16.800000}\selectfont \(\displaystyle {\mathbf{E} \mbox{u}}\)}%
\end{pgfscope}%
\begin{pgfscope}%
\pgfsetbuttcap%
\pgfsetmiterjoin%
\pgfsetlinewidth{0.803000pt}%
\definecolor{currentstroke}{rgb}{0.501961,0.501961,0.501961}%
\pgfsetstrokecolor{currentstroke}%
\pgfsetdash{}{0pt}%
\pgfpathmoveto{\pgfqpoint{4.517160in}{0.521603in}}%
\pgfpathlineto{\pgfqpoint{4.517160in}{0.533400in}}%
\pgfpathlineto{\pgfqpoint{4.517160in}{3.529806in}}%
\pgfpathlineto{\pgfqpoint{4.517160in}{3.541603in}}%
\pgfpathlineto{\pgfqpoint{4.668160in}{3.541603in}}%
\pgfpathlineto{\pgfqpoint{4.668160in}{3.529806in}}%
\pgfpathlineto{\pgfqpoint{4.668160in}{0.533400in}}%
\pgfpathlineto{\pgfqpoint{4.668160in}{0.521603in}}%
\pgfpathclose%
\pgfusepath{stroke}%
\end{pgfscope}%
\end{pgfpicture}%
\makeatother%
\endgroup%

    \caption{Non-dimensional trajectories with the short-time scaling.\label{fig:series_s_ds}}
\end{figure}

The covariance of $\mathbb{I}\mbox{m}$ with $\mathbb{E}\mbox{u}$ is shown in Figure \ref{fig:dnumbs}. Predictably, there is quite strong correlation between the dimensionless groups. We also see that $\mathbb{I}\mbox{m} < 1$ for all drops. Using an OLS regression, we find the model $\mathbb{I}\mbox{m} \sim (0.012 \pm 0.003) \mathbb{E}\mbox{u} + (0.212 \pm 0.036) $ with $R^2 =0.59$.
\begin{figure}[H]
    \centering
    %% Creator: Matplotlib, PGF backend
%%
%% To include the figure in your LaTeX document, write
%%   \input{<filename>.pgf}
%%
%% Make sure the required packages are loaded in your preamble
%%   \usepackage{pgf}
%%
%% Figures using additional raster images can only be included by \input if
%% they are in the same directory as the main LaTeX file. For loading figures
%% from other directories you can use the `import` package
%%   \usepackage{import}
%% and then include the figures with
%%   \import{<path to file>}{<filename>.pgf}
%%
%% Matplotlib used the following preamble
%%   \usepackage{fontspec}
%%   \setmainfont{DejaVuSerif.ttf}[Path=/home/erin/anaconda3/lib/python3.6/site-packages/matplotlib/mpl-data/fonts/ttf/]
%%   \setsansfont{DejaVuSans.ttf}[Path=/home/erin/anaconda3/lib/python3.6/site-packages/matplotlib/mpl-data/fonts/ttf/]
%%   \setmonofont{DejaVuSansMono.ttf}[Path=/home/erin/anaconda3/lib/python3.6/site-packages/matplotlib/mpl-data/fonts/ttf/]
%%
\begingroup%
\makeatletter%
\begin{pgfpicture}%
\pgfpathrectangle{\pgfpointorigin}{\pgfqpoint{5.314660in}{3.641603in}}%
\pgfusepath{use as bounding box, clip}%
\begin{pgfscope}%
\pgfsetbuttcap%
\pgfsetmiterjoin%
\definecolor{currentfill}{rgb}{1.000000,1.000000,1.000000}%
\pgfsetfillcolor{currentfill}%
\pgfsetlinewidth{0.000000pt}%
\definecolor{currentstroke}{rgb}{1.000000,1.000000,1.000000}%
\pgfsetstrokecolor{currentstroke}%
\pgfsetdash{}{0pt}%
\pgfpathmoveto{\pgfqpoint{0.000000in}{0.000000in}}%
\pgfpathlineto{\pgfqpoint{5.314660in}{0.000000in}}%
\pgfpathlineto{\pgfqpoint{5.314660in}{3.641603in}}%
\pgfpathlineto{\pgfqpoint{0.000000in}{3.641603in}}%
\pgfpathclose%
\pgfusepath{fill}%
\end{pgfscope}%
\begin{pgfscope}%
\pgfsetbuttcap%
\pgfsetmiterjoin%
\definecolor{currentfill}{rgb}{1.000000,1.000000,1.000000}%
\pgfsetfillcolor{currentfill}%
\pgfsetlinewidth{0.000000pt}%
\definecolor{currentstroke}{rgb}{0.000000,0.000000,0.000000}%
\pgfsetstrokecolor{currentstroke}%
\pgfsetstrokeopacity{0.000000}%
\pgfsetdash{}{0pt}%
\pgfpathmoveto{\pgfqpoint{0.564660in}{0.521603in}}%
\pgfpathlineto{\pgfqpoint{5.214660in}{0.521603in}}%
\pgfpathlineto{\pgfqpoint{5.214660in}{3.541603in}}%
\pgfpathlineto{\pgfqpoint{0.564660in}{3.541603in}}%
\pgfpathclose%
\pgfusepath{fill}%
\end{pgfscope}%
\begin{pgfscope}%
\pgfpathrectangle{\pgfqpoint{0.564660in}{0.521603in}}{\pgfqpoint{4.650000in}{3.020000in}}%
\pgfusepath{clip}%
\pgfsetbuttcap%
\pgfsetroundjoin%
\definecolor{currentfill}{rgb}{1.000000,1.000000,1.000000}%
\pgfsetfillcolor{currentfill}%
\pgfsetlinewidth{1.003750pt}%
\definecolor{currentstroke}{rgb}{0.000000,0.000000,0.000000}%
\pgfsetstrokecolor{currentstroke}%
\pgfsetdash{}{0pt}%
\pgfpathmoveto{\pgfqpoint{0.816771in}{2.246882in}}%
\pgfpathcurveto{\pgfqpoint{0.827822in}{2.246882in}}{\pgfqpoint{0.838421in}{2.251272in}}{\pgfqpoint{0.846234in}{2.259086in}}%
\pgfpathcurveto{\pgfqpoint{0.854048in}{2.266900in}}{\pgfqpoint{0.858438in}{2.277499in}}{\pgfqpoint{0.858438in}{2.288549in}}%
\pgfpathcurveto{\pgfqpoint{0.858438in}{2.299599in}}{\pgfqpoint{0.854048in}{2.310198in}}{\pgfqpoint{0.846234in}{2.318012in}}%
\pgfpathcurveto{\pgfqpoint{0.838421in}{2.325825in}}{\pgfqpoint{0.827822in}{2.330215in}}{\pgfqpoint{0.816771in}{2.330215in}}%
\pgfpathcurveto{\pgfqpoint{0.805721in}{2.330215in}}{\pgfqpoint{0.795122in}{2.325825in}}{\pgfqpoint{0.787309in}{2.318012in}}%
\pgfpathcurveto{\pgfqpoint{0.779495in}{2.310198in}}{\pgfqpoint{0.775105in}{2.299599in}}{\pgfqpoint{0.775105in}{2.288549in}}%
\pgfpathcurveto{\pgfqpoint{0.775105in}{2.277499in}}{\pgfqpoint{0.779495in}{2.266900in}}{\pgfqpoint{0.787309in}{2.259086in}}%
\pgfpathcurveto{\pgfqpoint{0.795122in}{2.251272in}}{\pgfqpoint{0.805721in}{2.246882in}}{\pgfqpoint{0.816771in}{2.246882in}}%
\pgfpathclose%
\pgfusepath{stroke,fill}%
\end{pgfscope}%
\begin{pgfscope}%
\pgfpathrectangle{\pgfqpoint{0.564660in}{0.521603in}}{\pgfqpoint{4.650000in}{3.020000in}}%
\pgfusepath{clip}%
\pgfsetbuttcap%
\pgfsetroundjoin%
\definecolor{currentfill}{rgb}{1.000000,1.000000,1.000000}%
\pgfsetfillcolor{currentfill}%
\pgfsetlinewidth{1.003750pt}%
\definecolor{currentstroke}{rgb}{0.000000,0.000000,0.000000}%
\pgfsetstrokecolor{currentstroke}%
\pgfsetdash{}{0pt}%
\pgfpathmoveto{\pgfqpoint{3.858085in}{1.799540in}}%
\pgfpathcurveto{\pgfqpoint{3.869135in}{1.799540in}}{\pgfqpoint{3.879734in}{1.803931in}}{\pgfqpoint{3.887547in}{1.811744in}}%
\pgfpathcurveto{\pgfqpoint{3.895361in}{1.819558in}}{\pgfqpoint{3.899751in}{1.830157in}}{\pgfqpoint{3.899751in}{1.841207in}}%
\pgfpathcurveto{\pgfqpoint{3.899751in}{1.852257in}}{\pgfqpoint{3.895361in}{1.862856in}}{\pgfqpoint{3.887547in}{1.870670in}}%
\pgfpathcurveto{\pgfqpoint{3.879734in}{1.878484in}}{\pgfqpoint{3.869135in}{1.882874in}}{\pgfqpoint{3.858085in}{1.882874in}}%
\pgfpathcurveto{\pgfqpoint{3.847034in}{1.882874in}}{\pgfqpoint{3.836435in}{1.878484in}}{\pgfqpoint{3.828622in}{1.870670in}}%
\pgfpathcurveto{\pgfqpoint{3.820808in}{1.862856in}}{\pgfqpoint{3.816418in}{1.852257in}}{\pgfqpoint{3.816418in}{1.841207in}}%
\pgfpathcurveto{\pgfqpoint{3.816418in}{1.830157in}}{\pgfqpoint{3.820808in}{1.819558in}}{\pgfqpoint{3.828622in}{1.811744in}}%
\pgfpathcurveto{\pgfqpoint{3.836435in}{1.803931in}}{\pgfqpoint{3.847034in}{1.799540in}}{\pgfqpoint{3.858085in}{1.799540in}}%
\pgfpathclose%
\pgfusepath{stroke,fill}%
\end{pgfscope}%
\begin{pgfscope}%
\pgfpathrectangle{\pgfqpoint{0.564660in}{0.521603in}}{\pgfqpoint{4.650000in}{3.020000in}}%
\pgfusepath{clip}%
\pgfsetbuttcap%
\pgfsetroundjoin%
\definecolor{currentfill}{rgb}{1.000000,1.000000,1.000000}%
\pgfsetfillcolor{currentfill}%
\pgfsetlinewidth{1.003750pt}%
\definecolor{currentstroke}{rgb}{0.000000,0.000000,0.000000}%
\pgfsetstrokecolor{currentstroke}%
\pgfsetdash{}{0pt}%
\pgfpathmoveto{\pgfqpoint{4.993344in}{2.156424in}}%
\pgfpathcurveto{\pgfqpoint{5.004394in}{2.156424in}}{\pgfqpoint{5.014993in}{2.160815in}}{\pgfqpoint{5.022807in}{2.168628in}}%
\pgfpathcurveto{\pgfqpoint{5.030620in}{2.176442in}}{\pgfqpoint{5.035010in}{2.187041in}}{\pgfqpoint{5.035010in}{2.198091in}}%
\pgfpathcurveto{\pgfqpoint{5.035010in}{2.209141in}}{\pgfqpoint{5.030620in}{2.219740in}}{\pgfqpoint{5.022807in}{2.227554in}}%
\pgfpathcurveto{\pgfqpoint{5.014993in}{2.235367in}}{\pgfqpoint{5.004394in}{2.239758in}}{\pgfqpoint{4.993344in}{2.239758in}}%
\pgfpathcurveto{\pgfqpoint{4.982294in}{2.239758in}}{\pgfqpoint{4.971695in}{2.235367in}}{\pgfqpoint{4.963881in}{2.227554in}}%
\pgfpathcurveto{\pgfqpoint{4.956067in}{2.219740in}}{\pgfqpoint{4.951677in}{2.209141in}}{\pgfqpoint{4.951677in}{2.198091in}}%
\pgfpathcurveto{\pgfqpoint{4.951677in}{2.187041in}}{\pgfqpoint{4.956067in}{2.176442in}}{\pgfqpoint{4.963881in}{2.168628in}}%
\pgfpathcurveto{\pgfqpoint{4.971695in}{2.160815in}}{\pgfqpoint{4.982294in}{2.156424in}}{\pgfqpoint{4.993344in}{2.156424in}}%
\pgfpathclose%
\pgfusepath{stroke,fill}%
\end{pgfscope}%
\begin{pgfscope}%
\pgfpathrectangle{\pgfqpoint{0.564660in}{0.521603in}}{\pgfqpoint{4.650000in}{3.020000in}}%
\pgfusepath{clip}%
\pgfsetbuttcap%
\pgfsetroundjoin%
\definecolor{currentfill}{rgb}{1.000000,1.000000,1.000000}%
\pgfsetfillcolor{currentfill}%
\pgfsetlinewidth{1.003750pt}%
\definecolor{currentstroke}{rgb}{0.000000,0.000000,0.000000}%
\pgfsetstrokecolor{currentstroke}%
\pgfsetdash{}{0pt}%
\pgfpathmoveto{\pgfqpoint{1.914828in}{1.195437in}}%
\pgfpathcurveto{\pgfqpoint{1.925878in}{1.195437in}}{\pgfqpoint{1.936477in}{1.199828in}}{\pgfqpoint{1.944290in}{1.207641in}}%
\pgfpathcurveto{\pgfqpoint{1.952104in}{1.215455in}}{\pgfqpoint{1.956494in}{1.226054in}}{\pgfqpoint{1.956494in}{1.237104in}}%
\pgfpathcurveto{\pgfqpoint{1.956494in}{1.248154in}}{\pgfqpoint{1.952104in}{1.258753in}}{\pgfqpoint{1.944290in}{1.266567in}}%
\pgfpathcurveto{\pgfqpoint{1.936477in}{1.274380in}}{\pgfqpoint{1.925878in}{1.278771in}}{\pgfqpoint{1.914828in}{1.278771in}}%
\pgfpathcurveto{\pgfqpoint{1.903777in}{1.278771in}}{\pgfqpoint{1.893178in}{1.274380in}}{\pgfqpoint{1.885365in}{1.266567in}}%
\pgfpathcurveto{\pgfqpoint{1.877551in}{1.258753in}}{\pgfqpoint{1.873161in}{1.248154in}}{\pgfqpoint{1.873161in}{1.237104in}}%
\pgfpathcurveto{\pgfqpoint{1.873161in}{1.226054in}}{\pgfqpoint{1.877551in}{1.215455in}}{\pgfqpoint{1.885365in}{1.207641in}}%
\pgfpathcurveto{\pgfqpoint{1.893178in}{1.199828in}}{\pgfqpoint{1.903777in}{1.195437in}}{\pgfqpoint{1.914828in}{1.195437in}}%
\pgfpathclose%
\pgfusepath{stroke,fill}%
\end{pgfscope}%
\begin{pgfscope}%
\pgfpathrectangle{\pgfqpoint{0.564660in}{0.521603in}}{\pgfqpoint{4.650000in}{3.020000in}}%
\pgfusepath{clip}%
\pgfsetbuttcap%
\pgfsetroundjoin%
\definecolor{currentfill}{rgb}{1.000000,1.000000,1.000000}%
\pgfsetfillcolor{currentfill}%
\pgfsetlinewidth{1.003750pt}%
\definecolor{currentstroke}{rgb}{0.000000,0.000000,0.000000}%
\pgfsetstrokecolor{currentstroke}%
\pgfsetdash{}{0pt}%
\pgfpathmoveto{\pgfqpoint{2.773519in}{3.358147in}}%
\pgfpathcurveto{\pgfqpoint{2.784569in}{3.358147in}}{\pgfqpoint{2.795168in}{3.362537in}}{\pgfqpoint{2.802981in}{3.370350in}}%
\pgfpathcurveto{\pgfqpoint{2.810795in}{3.378164in}}{\pgfqpoint{2.815185in}{3.388763in}}{\pgfqpoint{2.815185in}{3.399813in}}%
\pgfpathcurveto{\pgfqpoint{2.815185in}{3.410863in}}{\pgfqpoint{2.810795in}{3.421462in}}{\pgfqpoint{2.802981in}{3.429276in}}%
\pgfpathcurveto{\pgfqpoint{2.795168in}{3.437090in}}{\pgfqpoint{2.784569in}{3.441480in}}{\pgfqpoint{2.773519in}{3.441480in}}%
\pgfpathcurveto{\pgfqpoint{2.762469in}{3.441480in}}{\pgfqpoint{2.751870in}{3.437090in}}{\pgfqpoint{2.744056in}{3.429276in}}%
\pgfpathcurveto{\pgfqpoint{2.736242in}{3.421462in}}{\pgfqpoint{2.731852in}{3.410863in}}{\pgfqpoint{2.731852in}{3.399813in}}%
\pgfpathcurveto{\pgfqpoint{2.731852in}{3.388763in}}{\pgfqpoint{2.736242in}{3.378164in}}{\pgfqpoint{2.744056in}{3.370350in}}%
\pgfpathcurveto{\pgfqpoint{2.751870in}{3.362537in}}{\pgfqpoint{2.762469in}{3.358147in}}{\pgfqpoint{2.773519in}{3.358147in}}%
\pgfpathclose%
\pgfusepath{stroke,fill}%
\end{pgfscope}%
\begin{pgfscope}%
\pgfpathrectangle{\pgfqpoint{0.564660in}{0.521603in}}{\pgfqpoint{4.650000in}{3.020000in}}%
\pgfusepath{clip}%
\pgfsetbuttcap%
\pgfsetroundjoin%
\definecolor{currentfill}{rgb}{1.000000,1.000000,1.000000}%
\pgfsetfillcolor{currentfill}%
\pgfsetlinewidth{1.003750pt}%
\definecolor{currentstroke}{rgb}{0.000000,0.000000,0.000000}%
\pgfsetstrokecolor{currentstroke}%
\pgfsetdash{}{0pt}%
\pgfpathmoveto{\pgfqpoint{2.234616in}{0.983815in}}%
\pgfpathcurveto{\pgfqpoint{2.245666in}{0.983815in}}{\pgfqpoint{2.256265in}{0.988205in}}{\pgfqpoint{2.264079in}{0.996018in}}%
\pgfpathcurveto{\pgfqpoint{2.271892in}{1.003832in}}{\pgfqpoint{2.276283in}{1.014431in}}{\pgfqpoint{2.276283in}{1.025481in}}%
\pgfpathcurveto{\pgfqpoint{2.276283in}{1.036531in}}{\pgfqpoint{2.271892in}{1.047130in}}{\pgfqpoint{2.264079in}{1.054944in}}%
\pgfpathcurveto{\pgfqpoint{2.256265in}{1.062758in}}{\pgfqpoint{2.245666in}{1.067148in}}{\pgfqpoint{2.234616in}{1.067148in}}%
\pgfpathcurveto{\pgfqpoint{2.223566in}{1.067148in}}{\pgfqpoint{2.212967in}{1.062758in}}{\pgfqpoint{2.205153in}{1.054944in}}%
\pgfpathcurveto{\pgfqpoint{2.197340in}{1.047130in}}{\pgfqpoint{2.192949in}{1.036531in}}{\pgfqpoint{2.192949in}{1.025481in}}%
\pgfpathcurveto{\pgfqpoint{2.192949in}{1.014431in}}{\pgfqpoint{2.197340in}{1.003832in}}{\pgfqpoint{2.205153in}{0.996018in}}%
\pgfpathcurveto{\pgfqpoint{2.212967in}{0.988205in}}{\pgfqpoint{2.223566in}{0.983815in}}{\pgfqpoint{2.234616in}{0.983815in}}%
\pgfpathclose%
\pgfusepath{stroke,fill}%
\end{pgfscope}%
\begin{pgfscope}%
\pgfpathrectangle{\pgfqpoint{0.564660in}{0.521603in}}{\pgfqpoint{4.650000in}{3.020000in}}%
\pgfusepath{clip}%
\pgfsetbuttcap%
\pgfsetroundjoin%
\definecolor{currentfill}{rgb}{1.000000,1.000000,1.000000}%
\pgfsetfillcolor{currentfill}%
\pgfsetlinewidth{1.003750pt}%
\definecolor{currentstroke}{rgb}{0.000000,0.000000,0.000000}%
\pgfsetstrokecolor{currentstroke}%
\pgfsetdash{}{0pt}%
\pgfpathmoveto{\pgfqpoint{0.819079in}{1.480427in}}%
\pgfpathcurveto{\pgfqpoint{0.830129in}{1.480427in}}{\pgfqpoint{0.840728in}{1.484817in}}{\pgfqpoint{0.848542in}{1.492631in}}%
\pgfpathcurveto{\pgfqpoint{0.856355in}{1.500444in}}{\pgfqpoint{0.860746in}{1.511043in}}{\pgfqpoint{0.860746in}{1.522093in}}%
\pgfpathcurveto{\pgfqpoint{0.860746in}{1.533144in}}{\pgfqpoint{0.856355in}{1.543743in}}{\pgfqpoint{0.848542in}{1.551556in}}%
\pgfpathcurveto{\pgfqpoint{0.840728in}{1.559370in}}{\pgfqpoint{0.830129in}{1.563760in}}{\pgfqpoint{0.819079in}{1.563760in}}%
\pgfpathcurveto{\pgfqpoint{0.808029in}{1.563760in}}{\pgfqpoint{0.797430in}{1.559370in}}{\pgfqpoint{0.789616in}{1.551556in}}%
\pgfpathcurveto{\pgfqpoint{0.781803in}{1.543743in}}{\pgfqpoint{0.777412in}{1.533144in}}{\pgfqpoint{0.777412in}{1.522093in}}%
\pgfpathcurveto{\pgfqpoint{0.777412in}{1.511043in}}{\pgfqpoint{0.781803in}{1.500444in}}{\pgfqpoint{0.789616in}{1.492631in}}%
\pgfpathcurveto{\pgfqpoint{0.797430in}{1.484817in}}{\pgfqpoint{0.808029in}{1.480427in}}{\pgfqpoint{0.819079in}{1.480427in}}%
\pgfpathclose%
\pgfusepath{stroke,fill}%
\end{pgfscope}%
\begin{pgfscope}%
\pgfpathrectangle{\pgfqpoint{0.564660in}{0.521603in}}{\pgfqpoint{4.650000in}{3.020000in}}%
\pgfusepath{clip}%
\pgfsetbuttcap%
\pgfsetroundjoin%
\definecolor{currentfill}{rgb}{1.000000,1.000000,1.000000}%
\pgfsetfillcolor{currentfill}%
\pgfsetlinewidth{1.003750pt}%
\definecolor{currentstroke}{rgb}{0.000000,0.000000,0.000000}%
\pgfsetstrokecolor{currentstroke}%
\pgfsetdash{}{0pt}%
\pgfpathmoveto{\pgfqpoint{1.363796in}{0.692093in}}%
\pgfpathcurveto{\pgfqpoint{1.374846in}{0.692093in}}{\pgfqpoint{1.385445in}{0.696483in}}{\pgfqpoint{1.393259in}{0.704297in}}%
\pgfpathcurveto{\pgfqpoint{1.401072in}{0.712110in}}{\pgfqpoint{1.405462in}{0.722709in}}{\pgfqpoint{1.405462in}{0.733759in}}%
\pgfpathcurveto{\pgfqpoint{1.405462in}{0.744810in}}{\pgfqpoint{1.401072in}{0.755409in}}{\pgfqpoint{1.393259in}{0.763222in}}%
\pgfpathcurveto{\pgfqpoint{1.385445in}{0.771036in}}{\pgfqpoint{1.374846in}{0.775426in}}{\pgfqpoint{1.363796in}{0.775426in}}%
\pgfpathcurveto{\pgfqpoint{1.352746in}{0.775426in}}{\pgfqpoint{1.342147in}{0.771036in}}{\pgfqpoint{1.334333in}{0.763222in}}%
\pgfpathcurveto{\pgfqpoint{1.326519in}{0.755409in}}{\pgfqpoint{1.322129in}{0.744810in}}{\pgfqpoint{1.322129in}{0.733759in}}%
\pgfpathcurveto{\pgfqpoint{1.322129in}{0.722709in}}{\pgfqpoint{1.326519in}{0.712110in}}{\pgfqpoint{1.334333in}{0.704297in}}%
\pgfpathcurveto{\pgfqpoint{1.342147in}{0.696483in}}{\pgfqpoint{1.352746in}{0.692093in}}{\pgfqpoint{1.363796in}{0.692093in}}%
\pgfpathclose%
\pgfusepath{stroke,fill}%
\end{pgfscope}%
\begin{pgfscope}%
\pgfpathrectangle{\pgfqpoint{0.564660in}{0.521603in}}{\pgfqpoint{4.650000in}{3.020000in}}%
\pgfusepath{clip}%
\pgfsetbuttcap%
\pgfsetroundjoin%
\definecolor{currentfill}{rgb}{1.000000,1.000000,1.000000}%
\pgfsetfillcolor{currentfill}%
\pgfsetlinewidth{1.003750pt}%
\definecolor{currentstroke}{rgb}{0.000000,0.000000,0.000000}%
\pgfsetstrokecolor{currentstroke}%
\pgfsetdash{}{0pt}%
\pgfpathmoveto{\pgfqpoint{1.096147in}{0.621727in}}%
\pgfpathcurveto{\pgfqpoint{1.107197in}{0.621727in}}{\pgfqpoint{1.117796in}{0.626117in}}{\pgfqpoint{1.125609in}{0.633931in}}%
\pgfpathcurveto{\pgfqpoint{1.133423in}{0.641744in}}{\pgfqpoint{1.137813in}{0.652343in}}{\pgfqpoint{1.137813in}{0.663393in}}%
\pgfpathcurveto{\pgfqpoint{1.137813in}{0.674444in}}{\pgfqpoint{1.133423in}{0.685043in}}{\pgfqpoint{1.125609in}{0.692856in}}%
\pgfpathcurveto{\pgfqpoint{1.117796in}{0.700670in}}{\pgfqpoint{1.107197in}{0.705060in}}{\pgfqpoint{1.096147in}{0.705060in}}%
\pgfpathcurveto{\pgfqpoint{1.085096in}{0.705060in}}{\pgfqpoint{1.074497in}{0.700670in}}{\pgfqpoint{1.066684in}{0.692856in}}%
\pgfpathcurveto{\pgfqpoint{1.058870in}{0.685043in}}{\pgfqpoint{1.054480in}{0.674444in}}{\pgfqpoint{1.054480in}{0.663393in}}%
\pgfpathcurveto{\pgfqpoint{1.054480in}{0.652343in}}{\pgfqpoint{1.058870in}{0.641744in}}{\pgfqpoint{1.066684in}{0.633931in}}%
\pgfpathcurveto{\pgfqpoint{1.074497in}{0.626117in}}{\pgfqpoint{1.085096in}{0.621727in}}{\pgfqpoint{1.096147in}{0.621727in}}%
\pgfpathclose%
\pgfusepath{stroke,fill}%
\end{pgfscope}%
\begin{pgfscope}%
\pgfpathrectangle{\pgfqpoint{0.564660in}{0.521603in}}{\pgfqpoint{4.650000in}{3.020000in}}%
\pgfusepath{clip}%
\pgfsetbuttcap%
\pgfsetroundjoin%
\definecolor{currentfill}{rgb}{1.000000,1.000000,1.000000}%
\pgfsetfillcolor{currentfill}%
\pgfsetlinewidth{1.003750pt}%
\definecolor{currentstroke}{rgb}{0.000000,0.000000,0.000000}%
\pgfsetstrokecolor{currentstroke}%
\pgfsetdash{}{0pt}%
\pgfpathmoveto{\pgfqpoint{0.887904in}{1.305315in}}%
\pgfpathcurveto{\pgfqpoint{0.898954in}{1.305315in}}{\pgfqpoint{0.909553in}{1.309705in}}{\pgfqpoint{0.917367in}{1.317519in}}%
\pgfpathcurveto{\pgfqpoint{0.925180in}{1.325332in}}{\pgfqpoint{0.929570in}{1.335932in}}{\pgfqpoint{0.929570in}{1.346982in}}%
\pgfpathcurveto{\pgfqpoint{0.929570in}{1.358032in}}{\pgfqpoint{0.925180in}{1.368631in}}{\pgfqpoint{0.917367in}{1.376444in}}%
\pgfpathcurveto{\pgfqpoint{0.909553in}{1.384258in}}{\pgfqpoint{0.898954in}{1.388648in}}{\pgfqpoint{0.887904in}{1.388648in}}%
\pgfpathcurveto{\pgfqpoint{0.876854in}{1.388648in}}{\pgfqpoint{0.866255in}{1.384258in}}{\pgfqpoint{0.858441in}{1.376444in}}%
\pgfpathcurveto{\pgfqpoint{0.850627in}{1.368631in}}{\pgfqpoint{0.846237in}{1.358032in}}{\pgfqpoint{0.846237in}{1.346982in}}%
\pgfpathcurveto{\pgfqpoint{0.846237in}{1.335932in}}{\pgfqpoint{0.850627in}{1.325332in}}{\pgfqpoint{0.858441in}{1.317519in}}%
\pgfpathcurveto{\pgfqpoint{0.866255in}{1.309705in}}{\pgfqpoint{0.876854in}{1.305315in}}{\pgfqpoint{0.887904in}{1.305315in}}%
\pgfpathclose%
\pgfusepath{stroke,fill}%
\end{pgfscope}%
\begin{pgfscope}%
\pgfpathrectangle{\pgfqpoint{0.564660in}{0.521603in}}{\pgfqpoint{4.650000in}{3.020000in}}%
\pgfusepath{clip}%
\pgfsetbuttcap%
\pgfsetroundjoin%
\definecolor{currentfill}{rgb}{1.000000,1.000000,1.000000}%
\pgfsetfillcolor{currentfill}%
\pgfsetlinewidth{1.003750pt}%
\definecolor{currentstroke}{rgb}{0.000000,0.000000,0.000000}%
\pgfsetstrokecolor{currentstroke}%
\pgfsetdash{}{0pt}%
\pgfpathmoveto{\pgfqpoint{2.143771in}{2.158900in}}%
\pgfpathcurveto{\pgfqpoint{2.154821in}{2.158900in}}{\pgfqpoint{2.165420in}{2.163290in}}{\pgfqpoint{2.173234in}{2.171104in}}%
\pgfpathcurveto{\pgfqpoint{2.181047in}{2.178918in}}{\pgfqpoint{2.185438in}{2.189517in}}{\pgfqpoint{2.185438in}{2.200567in}}%
\pgfpathcurveto{\pgfqpoint{2.185438in}{2.211617in}}{\pgfqpoint{2.181047in}{2.222216in}}{\pgfqpoint{2.173234in}{2.230029in}}%
\pgfpathcurveto{\pgfqpoint{2.165420in}{2.237843in}}{\pgfqpoint{2.154821in}{2.242233in}}{\pgfqpoint{2.143771in}{2.242233in}}%
\pgfpathcurveto{\pgfqpoint{2.132721in}{2.242233in}}{\pgfqpoint{2.122122in}{2.237843in}}{\pgfqpoint{2.114308in}{2.230029in}}%
\pgfpathcurveto{\pgfqpoint{2.106494in}{2.222216in}}{\pgfqpoint{2.102104in}{2.211617in}}{\pgfqpoint{2.102104in}{2.200567in}}%
\pgfpathcurveto{\pgfqpoint{2.102104in}{2.189517in}}{\pgfqpoint{2.106494in}{2.178918in}}{\pgfqpoint{2.114308in}{2.171104in}}%
\pgfpathcurveto{\pgfqpoint{2.122122in}{2.163290in}}{\pgfqpoint{2.132721in}{2.158900in}}{\pgfqpoint{2.143771in}{2.158900in}}%
\pgfpathclose%
\pgfusepath{stroke,fill}%
\end{pgfscope}%
\begin{pgfscope}%
\pgfpathrectangle{\pgfqpoint{0.564660in}{0.521603in}}{\pgfqpoint{4.650000in}{3.020000in}}%
\pgfusepath{clip}%
\pgfsetbuttcap%
\pgfsetroundjoin%
\definecolor{currentfill}{rgb}{1.000000,1.000000,1.000000}%
\pgfsetfillcolor{currentfill}%
\pgfsetlinewidth{1.003750pt}%
\definecolor{currentstroke}{rgb}{0.000000,0.000000,0.000000}%
\pgfsetstrokecolor{currentstroke}%
\pgfsetdash{}{0pt}%
\pgfpathmoveto{\pgfqpoint{1.086864in}{0.741510in}}%
\pgfpathcurveto{\pgfqpoint{1.097914in}{0.741510in}}{\pgfqpoint{1.108513in}{0.745901in}}{\pgfqpoint{1.116327in}{0.753714in}}%
\pgfpathcurveto{\pgfqpoint{1.124141in}{0.761528in}}{\pgfqpoint{1.128531in}{0.772127in}}{\pgfqpoint{1.128531in}{0.783177in}}%
\pgfpathcurveto{\pgfqpoint{1.128531in}{0.794227in}}{\pgfqpoint{1.124141in}{0.804826in}}{\pgfqpoint{1.116327in}{0.812640in}}%
\pgfpathcurveto{\pgfqpoint{1.108513in}{0.820453in}}{\pgfqpoint{1.097914in}{0.824844in}}{\pgfqpoint{1.086864in}{0.824844in}}%
\pgfpathcurveto{\pgfqpoint{1.075814in}{0.824844in}}{\pgfqpoint{1.065215in}{0.820453in}}{\pgfqpoint{1.057401in}{0.812640in}}%
\pgfpathcurveto{\pgfqpoint{1.049588in}{0.804826in}}{\pgfqpoint{1.045197in}{0.794227in}}{\pgfqpoint{1.045197in}{0.783177in}}%
\pgfpathcurveto{\pgfqpoint{1.045197in}{0.772127in}}{\pgfqpoint{1.049588in}{0.761528in}}{\pgfqpoint{1.057401in}{0.753714in}}%
\pgfpathcurveto{\pgfqpoint{1.065215in}{0.745901in}}{\pgfqpoint{1.075814in}{0.741510in}}{\pgfqpoint{1.086864in}{0.741510in}}%
\pgfpathclose%
\pgfusepath{stroke,fill}%
\end{pgfscope}%
\begin{pgfscope}%
\pgfpathrectangle{\pgfqpoint{0.564660in}{0.521603in}}{\pgfqpoint{4.650000in}{3.020000in}}%
\pgfusepath{clip}%
\pgfsetbuttcap%
\pgfsetroundjoin%
\definecolor{currentfill}{rgb}{1.000000,1.000000,1.000000}%
\pgfsetfillcolor{currentfill}%
\pgfsetlinewidth{1.003750pt}%
\definecolor{currentstroke}{rgb}{0.000000,0.000000,0.000000}%
\pgfsetstrokecolor{currentstroke}%
\pgfsetdash{}{0pt}%
\pgfpathmoveto{\pgfqpoint{0.785977in}{1.118987in}}%
\pgfpathcurveto{\pgfqpoint{0.797027in}{1.118987in}}{\pgfqpoint{0.807626in}{1.123377in}}{\pgfqpoint{0.815440in}{1.131190in}}%
\pgfpathcurveto{\pgfqpoint{0.823253in}{1.139004in}}{\pgfqpoint{0.827643in}{1.149603in}}{\pgfqpoint{0.827643in}{1.160653in}}%
\pgfpathcurveto{\pgfqpoint{0.827643in}{1.171703in}}{\pgfqpoint{0.823253in}{1.182302in}}{\pgfqpoint{0.815440in}{1.190116in}}%
\pgfpathcurveto{\pgfqpoint{0.807626in}{1.197930in}}{\pgfqpoint{0.797027in}{1.202320in}}{\pgfqpoint{0.785977in}{1.202320in}}%
\pgfpathcurveto{\pgfqpoint{0.774927in}{1.202320in}}{\pgfqpoint{0.764328in}{1.197930in}}{\pgfqpoint{0.756514in}{1.190116in}}%
\pgfpathcurveto{\pgfqpoint{0.748700in}{1.182302in}}{\pgfqpoint{0.744310in}{1.171703in}}{\pgfqpoint{0.744310in}{1.160653in}}%
\pgfpathcurveto{\pgfqpoint{0.744310in}{1.149603in}}{\pgfqpoint{0.748700in}{1.139004in}}{\pgfqpoint{0.756514in}{1.131190in}}%
\pgfpathcurveto{\pgfqpoint{0.764328in}{1.123377in}}{\pgfqpoint{0.774927in}{1.118987in}}{\pgfqpoint{0.785977in}{1.118987in}}%
\pgfpathclose%
\pgfusepath{stroke,fill}%
\end{pgfscope}%
\begin{pgfscope}%
\pgfpathrectangle{\pgfqpoint{0.564660in}{0.521603in}}{\pgfqpoint{4.650000in}{3.020000in}}%
\pgfusepath{clip}%
\pgfsetbuttcap%
\pgfsetroundjoin%
\definecolor{currentfill}{rgb}{1.000000,1.000000,1.000000}%
\pgfsetfillcolor{currentfill}%
\pgfsetlinewidth{1.003750pt}%
\definecolor{currentstroke}{rgb}{0.000000,0.000000,0.000000}%
\pgfsetstrokecolor{currentstroke}%
\pgfsetdash{}{0pt}%
\pgfpathmoveto{\pgfqpoint{2.036724in}{2.941966in}}%
\pgfpathcurveto{\pgfqpoint{2.047774in}{2.941966in}}{\pgfqpoint{2.058373in}{2.946356in}}{\pgfqpoint{2.066187in}{2.954170in}}%
\pgfpathcurveto{\pgfqpoint{2.074000in}{2.961983in}}{\pgfqpoint{2.078390in}{2.972582in}}{\pgfqpoint{2.078390in}{2.983632in}}%
\pgfpathcurveto{\pgfqpoint{2.078390in}{2.994682in}}{\pgfqpoint{2.074000in}{3.005282in}}{\pgfqpoint{2.066187in}{3.013095in}}%
\pgfpathcurveto{\pgfqpoint{2.058373in}{3.020909in}}{\pgfqpoint{2.047774in}{3.025299in}}{\pgfqpoint{2.036724in}{3.025299in}}%
\pgfpathcurveto{\pgfqpoint{2.025674in}{3.025299in}}{\pgfqpoint{2.015075in}{3.020909in}}{\pgfqpoint{2.007261in}{3.013095in}}%
\pgfpathcurveto{\pgfqpoint{1.999447in}{3.005282in}}{\pgfqpoint{1.995057in}{2.994682in}}{\pgfqpoint{1.995057in}{2.983632in}}%
\pgfpathcurveto{\pgfqpoint{1.995057in}{2.972582in}}{\pgfqpoint{1.999447in}{2.961983in}}{\pgfqpoint{2.007261in}{2.954170in}}%
\pgfpathcurveto{\pgfqpoint{2.015075in}{2.946356in}}{\pgfqpoint{2.025674in}{2.941966in}}{\pgfqpoint{2.036724in}{2.941966in}}%
\pgfpathclose%
\pgfusepath{stroke,fill}%
\end{pgfscope}%
\begin{pgfscope}%
\pgfpathrectangle{\pgfqpoint{0.564660in}{0.521603in}}{\pgfqpoint{4.650000in}{3.020000in}}%
\pgfusepath{clip}%
\pgfsetbuttcap%
\pgfsetroundjoin%
\definecolor{currentfill}{rgb}{1.000000,1.000000,1.000000}%
\pgfsetfillcolor{currentfill}%
\pgfsetlinewidth{1.003750pt}%
\definecolor{currentstroke}{rgb}{0.000000,0.000000,0.000000}%
\pgfsetstrokecolor{currentstroke}%
\pgfsetdash{}{0pt}%
\pgfpathmoveto{\pgfqpoint{1.309364in}{1.315672in}}%
\pgfpathcurveto{\pgfqpoint{1.320414in}{1.315672in}}{\pgfqpoint{1.331013in}{1.320062in}}{\pgfqpoint{1.338826in}{1.327876in}}%
\pgfpathcurveto{\pgfqpoint{1.346640in}{1.335690in}}{\pgfqpoint{1.351030in}{1.346289in}}{\pgfqpoint{1.351030in}{1.357339in}}%
\pgfpathcurveto{\pgfqpoint{1.351030in}{1.368389in}}{\pgfqpoint{1.346640in}{1.378988in}}{\pgfqpoint{1.338826in}{1.386802in}}%
\pgfpathcurveto{\pgfqpoint{1.331013in}{1.394615in}}{\pgfqpoint{1.320414in}{1.399006in}}{\pgfqpoint{1.309364in}{1.399006in}}%
\pgfpathcurveto{\pgfqpoint{1.298314in}{1.399006in}}{\pgfqpoint{1.287715in}{1.394615in}}{\pgfqpoint{1.279901in}{1.386802in}}%
\pgfpathcurveto{\pgfqpoint{1.272087in}{1.378988in}}{\pgfqpoint{1.267697in}{1.368389in}}{\pgfqpoint{1.267697in}{1.357339in}}%
\pgfpathcurveto{\pgfqpoint{1.267697in}{1.346289in}}{\pgfqpoint{1.272087in}{1.335690in}}{\pgfqpoint{1.279901in}{1.327876in}}%
\pgfpathcurveto{\pgfqpoint{1.287715in}{1.320062in}}{\pgfqpoint{1.298314in}{1.315672in}}{\pgfqpoint{1.309364in}{1.315672in}}%
\pgfpathclose%
\pgfusepath{stroke,fill}%
\end{pgfscope}%
\begin{pgfscope}%
\pgfsetbuttcap%
\pgfsetroundjoin%
\definecolor{currentfill}{rgb}{0.000000,0.000000,0.000000}%
\pgfsetfillcolor{currentfill}%
\pgfsetlinewidth{0.803000pt}%
\definecolor{currentstroke}{rgb}{0.000000,0.000000,0.000000}%
\pgfsetstrokecolor{currentstroke}%
\pgfsetdash{}{0pt}%
\pgfsys@defobject{currentmarker}{\pgfqpoint{0.000000in}{-0.048611in}}{\pgfqpoint{0.000000in}{0.000000in}}{%
\pgfpathmoveto{\pgfqpoint{0.000000in}{0.000000in}}%
\pgfpathlineto{\pgfqpoint{0.000000in}{-0.048611in}}%
\pgfusepath{stroke,fill}%
}%
\begin{pgfscope}%
\pgfsys@transformshift{0.667447in}{0.521603in}%
\pgfsys@useobject{currentmarker}{}%
\end{pgfscope}%
\end{pgfscope}%
\begin{pgfscope}%
\definecolor{textcolor}{rgb}{0.000000,0.000000,0.000000}%
\pgfsetstrokecolor{textcolor}%
\pgfsetfillcolor{textcolor}%
\pgftext[x=0.667447in,y=0.424381in,,top]{\color{textcolor}\rmfamily\fontsize{10.000000}{12.000000}\selectfont \(\displaystyle 0\)}%
\end{pgfscope}%
\begin{pgfscope}%
\pgfsetbuttcap%
\pgfsetroundjoin%
\definecolor{currentfill}{rgb}{0.000000,0.000000,0.000000}%
\pgfsetfillcolor{currentfill}%
\pgfsetlinewidth{0.803000pt}%
\definecolor{currentstroke}{rgb}{0.000000,0.000000,0.000000}%
\pgfsetstrokecolor{currentstroke}%
\pgfsetdash{}{0pt}%
\pgfsys@defobject{currentmarker}{\pgfqpoint{0.000000in}{-0.048611in}}{\pgfqpoint{0.000000in}{0.000000in}}{%
\pgfpathmoveto{\pgfqpoint{0.000000in}{0.000000in}}%
\pgfpathlineto{\pgfqpoint{0.000000in}{-0.048611in}}%
\pgfusepath{stroke,fill}%
}%
\begin{pgfscope}%
\pgfsys@transformshift{1.348050in}{0.521603in}%
\pgfsys@useobject{currentmarker}{}%
\end{pgfscope}%
\end{pgfscope}%
\begin{pgfscope}%
\definecolor{textcolor}{rgb}{0.000000,0.000000,0.000000}%
\pgfsetstrokecolor{textcolor}%
\pgfsetfillcolor{textcolor}%
\pgftext[x=1.348050in,y=0.424381in,,top]{\color{textcolor}\rmfamily\fontsize{10.000000}{12.000000}\selectfont \(\displaystyle 5\)}%
\end{pgfscope}%
\begin{pgfscope}%
\pgfsetbuttcap%
\pgfsetroundjoin%
\definecolor{currentfill}{rgb}{0.000000,0.000000,0.000000}%
\pgfsetfillcolor{currentfill}%
\pgfsetlinewidth{0.803000pt}%
\definecolor{currentstroke}{rgb}{0.000000,0.000000,0.000000}%
\pgfsetstrokecolor{currentstroke}%
\pgfsetdash{}{0pt}%
\pgfsys@defobject{currentmarker}{\pgfqpoint{0.000000in}{-0.048611in}}{\pgfqpoint{0.000000in}{0.000000in}}{%
\pgfpathmoveto{\pgfqpoint{0.000000in}{0.000000in}}%
\pgfpathlineto{\pgfqpoint{0.000000in}{-0.048611in}}%
\pgfusepath{stroke,fill}%
}%
\begin{pgfscope}%
\pgfsys@transformshift{2.028653in}{0.521603in}%
\pgfsys@useobject{currentmarker}{}%
\end{pgfscope}%
\end{pgfscope}%
\begin{pgfscope}%
\definecolor{textcolor}{rgb}{0.000000,0.000000,0.000000}%
\pgfsetstrokecolor{textcolor}%
\pgfsetfillcolor{textcolor}%
\pgftext[x=2.028653in,y=0.424381in,,top]{\color{textcolor}\rmfamily\fontsize{10.000000}{12.000000}\selectfont \(\displaystyle 10\)}%
\end{pgfscope}%
\begin{pgfscope}%
\pgfsetbuttcap%
\pgfsetroundjoin%
\definecolor{currentfill}{rgb}{0.000000,0.000000,0.000000}%
\pgfsetfillcolor{currentfill}%
\pgfsetlinewidth{0.803000pt}%
\definecolor{currentstroke}{rgb}{0.000000,0.000000,0.000000}%
\pgfsetstrokecolor{currentstroke}%
\pgfsetdash{}{0pt}%
\pgfsys@defobject{currentmarker}{\pgfqpoint{0.000000in}{-0.048611in}}{\pgfqpoint{0.000000in}{0.000000in}}{%
\pgfpathmoveto{\pgfqpoint{0.000000in}{0.000000in}}%
\pgfpathlineto{\pgfqpoint{0.000000in}{-0.048611in}}%
\pgfusepath{stroke,fill}%
}%
\begin{pgfscope}%
\pgfsys@transformshift{2.709256in}{0.521603in}%
\pgfsys@useobject{currentmarker}{}%
\end{pgfscope}%
\end{pgfscope}%
\begin{pgfscope}%
\definecolor{textcolor}{rgb}{0.000000,0.000000,0.000000}%
\pgfsetstrokecolor{textcolor}%
\pgfsetfillcolor{textcolor}%
\pgftext[x=2.709256in,y=0.424381in,,top]{\color{textcolor}\rmfamily\fontsize{10.000000}{12.000000}\selectfont \(\displaystyle 15\)}%
\end{pgfscope}%
\begin{pgfscope}%
\pgfsetbuttcap%
\pgfsetroundjoin%
\definecolor{currentfill}{rgb}{0.000000,0.000000,0.000000}%
\pgfsetfillcolor{currentfill}%
\pgfsetlinewidth{0.803000pt}%
\definecolor{currentstroke}{rgb}{0.000000,0.000000,0.000000}%
\pgfsetstrokecolor{currentstroke}%
\pgfsetdash{}{0pt}%
\pgfsys@defobject{currentmarker}{\pgfqpoint{0.000000in}{-0.048611in}}{\pgfqpoint{0.000000in}{0.000000in}}{%
\pgfpathmoveto{\pgfqpoint{0.000000in}{0.000000in}}%
\pgfpathlineto{\pgfqpoint{0.000000in}{-0.048611in}}%
\pgfusepath{stroke,fill}%
}%
\begin{pgfscope}%
\pgfsys@transformshift{3.389859in}{0.521603in}%
\pgfsys@useobject{currentmarker}{}%
\end{pgfscope}%
\end{pgfscope}%
\begin{pgfscope}%
\definecolor{textcolor}{rgb}{0.000000,0.000000,0.000000}%
\pgfsetstrokecolor{textcolor}%
\pgfsetfillcolor{textcolor}%
\pgftext[x=3.389859in,y=0.424381in,,top]{\color{textcolor}\rmfamily\fontsize{10.000000}{12.000000}\selectfont \(\displaystyle 20\)}%
\end{pgfscope}%
\begin{pgfscope}%
\pgfsetbuttcap%
\pgfsetroundjoin%
\definecolor{currentfill}{rgb}{0.000000,0.000000,0.000000}%
\pgfsetfillcolor{currentfill}%
\pgfsetlinewidth{0.803000pt}%
\definecolor{currentstroke}{rgb}{0.000000,0.000000,0.000000}%
\pgfsetstrokecolor{currentstroke}%
\pgfsetdash{}{0pt}%
\pgfsys@defobject{currentmarker}{\pgfqpoint{0.000000in}{-0.048611in}}{\pgfqpoint{0.000000in}{0.000000in}}{%
\pgfpathmoveto{\pgfqpoint{0.000000in}{0.000000in}}%
\pgfpathlineto{\pgfqpoint{0.000000in}{-0.048611in}}%
\pgfusepath{stroke,fill}%
}%
\begin{pgfscope}%
\pgfsys@transformshift{4.070462in}{0.521603in}%
\pgfsys@useobject{currentmarker}{}%
\end{pgfscope}%
\end{pgfscope}%
\begin{pgfscope}%
\definecolor{textcolor}{rgb}{0.000000,0.000000,0.000000}%
\pgfsetstrokecolor{textcolor}%
\pgfsetfillcolor{textcolor}%
\pgftext[x=4.070462in,y=0.424381in,,top]{\color{textcolor}\rmfamily\fontsize{10.000000}{12.000000}\selectfont \(\displaystyle 25\)}%
\end{pgfscope}%
\begin{pgfscope}%
\pgfsetbuttcap%
\pgfsetroundjoin%
\definecolor{currentfill}{rgb}{0.000000,0.000000,0.000000}%
\pgfsetfillcolor{currentfill}%
\pgfsetlinewidth{0.803000pt}%
\definecolor{currentstroke}{rgb}{0.000000,0.000000,0.000000}%
\pgfsetstrokecolor{currentstroke}%
\pgfsetdash{}{0pt}%
\pgfsys@defobject{currentmarker}{\pgfqpoint{0.000000in}{-0.048611in}}{\pgfqpoint{0.000000in}{0.000000in}}{%
\pgfpathmoveto{\pgfqpoint{0.000000in}{0.000000in}}%
\pgfpathlineto{\pgfqpoint{0.000000in}{-0.048611in}}%
\pgfusepath{stroke,fill}%
}%
\begin{pgfscope}%
\pgfsys@transformshift{4.751065in}{0.521603in}%
\pgfsys@useobject{currentmarker}{}%
\end{pgfscope}%
\end{pgfscope}%
\begin{pgfscope}%
\definecolor{textcolor}{rgb}{0.000000,0.000000,0.000000}%
\pgfsetstrokecolor{textcolor}%
\pgfsetfillcolor{textcolor}%
\pgftext[x=4.751065in,y=0.424381in,,top]{\color{textcolor}\rmfamily\fontsize{10.000000}{12.000000}\selectfont \(\displaystyle 30\)}%
\end{pgfscope}%
\begin{pgfscope}%
\definecolor{textcolor}{rgb}{0.000000,0.000000,0.000000}%
\pgfsetstrokecolor{textcolor}%
\pgfsetfillcolor{textcolor}%
\pgftext[x=2.889660in,y=0.234413in,,top]{\color{textcolor}\rmfamily\fontsize{10.000000}{12.000000}\selectfont \(\displaystyle \mathbf{E}\mbox{u}\)}%
\end{pgfscope}%
\begin{pgfscope}%
\pgfsetbuttcap%
\pgfsetroundjoin%
\definecolor{currentfill}{rgb}{0.000000,0.000000,0.000000}%
\pgfsetfillcolor{currentfill}%
\pgfsetlinewidth{0.803000pt}%
\definecolor{currentstroke}{rgb}{0.000000,0.000000,0.000000}%
\pgfsetstrokecolor{currentstroke}%
\pgfsetdash{}{0pt}%
\pgfsys@defobject{currentmarker}{\pgfqpoint{-0.048611in}{0.000000in}}{\pgfqpoint{0.000000in}{0.000000in}}{%
\pgfpathmoveto{\pgfqpoint{0.000000in}{0.000000in}}%
\pgfpathlineto{\pgfqpoint{-0.048611in}{0.000000in}}%
\pgfusepath{stroke,fill}%
}%
\begin{pgfscope}%
\pgfsys@transformshift{0.564660in}{1.017529in}%
\pgfsys@useobject{currentmarker}{}%
\end{pgfscope}%
\end{pgfscope}%
\begin{pgfscope}%
\definecolor{textcolor}{rgb}{0.000000,0.000000,0.000000}%
\pgfsetstrokecolor{textcolor}%
\pgfsetfillcolor{textcolor}%
\pgftext[x=0.289968in,y=0.964768in,left,base]{\color{textcolor}\rmfamily\fontsize{10.000000}{12.000000}\selectfont \(\displaystyle 0.3\)}%
\end{pgfscope}%
\begin{pgfscope}%
\pgfsetbuttcap%
\pgfsetroundjoin%
\definecolor{currentfill}{rgb}{0.000000,0.000000,0.000000}%
\pgfsetfillcolor{currentfill}%
\pgfsetlinewidth{0.803000pt}%
\definecolor{currentstroke}{rgb}{0.000000,0.000000,0.000000}%
\pgfsetstrokecolor{currentstroke}%
\pgfsetdash{}{0pt}%
\pgfsys@defobject{currentmarker}{\pgfqpoint{-0.048611in}{0.000000in}}{\pgfqpoint{0.000000in}{0.000000in}}{%
\pgfpathmoveto{\pgfqpoint{0.000000in}{0.000000in}}%
\pgfpathlineto{\pgfqpoint{-0.048611in}{0.000000in}}%
\pgfusepath{stroke,fill}%
}%
\begin{pgfscope}%
\pgfsys@transformshift{0.564660in}{1.618847in}%
\pgfsys@useobject{currentmarker}{}%
\end{pgfscope}%
\end{pgfscope}%
\begin{pgfscope}%
\definecolor{textcolor}{rgb}{0.000000,0.000000,0.000000}%
\pgfsetstrokecolor{textcolor}%
\pgfsetfillcolor{textcolor}%
\pgftext[x=0.289968in,y=1.566085in,left,base]{\color{textcolor}\rmfamily\fontsize{10.000000}{12.000000}\selectfont \(\displaystyle 0.4\)}%
\end{pgfscope}%
\begin{pgfscope}%
\pgfsetbuttcap%
\pgfsetroundjoin%
\definecolor{currentfill}{rgb}{0.000000,0.000000,0.000000}%
\pgfsetfillcolor{currentfill}%
\pgfsetlinewidth{0.803000pt}%
\definecolor{currentstroke}{rgb}{0.000000,0.000000,0.000000}%
\pgfsetstrokecolor{currentstroke}%
\pgfsetdash{}{0pt}%
\pgfsys@defobject{currentmarker}{\pgfqpoint{-0.048611in}{0.000000in}}{\pgfqpoint{0.000000in}{0.000000in}}{%
\pgfpathmoveto{\pgfqpoint{0.000000in}{0.000000in}}%
\pgfpathlineto{\pgfqpoint{-0.048611in}{0.000000in}}%
\pgfusepath{stroke,fill}%
}%
\begin{pgfscope}%
\pgfsys@transformshift{0.564660in}{2.220164in}%
\pgfsys@useobject{currentmarker}{}%
\end{pgfscope}%
\end{pgfscope}%
\begin{pgfscope}%
\definecolor{textcolor}{rgb}{0.000000,0.000000,0.000000}%
\pgfsetstrokecolor{textcolor}%
\pgfsetfillcolor{textcolor}%
\pgftext[x=0.289968in,y=2.167403in,left,base]{\color{textcolor}\rmfamily\fontsize{10.000000}{12.000000}\selectfont \(\displaystyle 0.5\)}%
\end{pgfscope}%
\begin{pgfscope}%
\pgfsetbuttcap%
\pgfsetroundjoin%
\definecolor{currentfill}{rgb}{0.000000,0.000000,0.000000}%
\pgfsetfillcolor{currentfill}%
\pgfsetlinewidth{0.803000pt}%
\definecolor{currentstroke}{rgb}{0.000000,0.000000,0.000000}%
\pgfsetstrokecolor{currentstroke}%
\pgfsetdash{}{0pt}%
\pgfsys@defobject{currentmarker}{\pgfqpoint{-0.048611in}{0.000000in}}{\pgfqpoint{0.000000in}{0.000000in}}{%
\pgfpathmoveto{\pgfqpoint{0.000000in}{0.000000in}}%
\pgfpathlineto{\pgfqpoint{-0.048611in}{0.000000in}}%
\pgfusepath{stroke,fill}%
}%
\begin{pgfscope}%
\pgfsys@transformshift{0.564660in}{2.821481in}%
\pgfsys@useobject{currentmarker}{}%
\end{pgfscope}%
\end{pgfscope}%
\begin{pgfscope}%
\definecolor{textcolor}{rgb}{0.000000,0.000000,0.000000}%
\pgfsetstrokecolor{textcolor}%
\pgfsetfillcolor{textcolor}%
\pgftext[x=0.289968in,y=2.768720in,left,base]{\color{textcolor}\rmfamily\fontsize{10.000000}{12.000000}\selectfont \(\displaystyle 0.6\)}%
\end{pgfscope}%
\begin{pgfscope}%
\pgfsetbuttcap%
\pgfsetroundjoin%
\definecolor{currentfill}{rgb}{0.000000,0.000000,0.000000}%
\pgfsetfillcolor{currentfill}%
\pgfsetlinewidth{0.803000pt}%
\definecolor{currentstroke}{rgb}{0.000000,0.000000,0.000000}%
\pgfsetstrokecolor{currentstroke}%
\pgfsetdash{}{0pt}%
\pgfsys@defobject{currentmarker}{\pgfqpoint{-0.048611in}{0.000000in}}{\pgfqpoint{0.000000in}{0.000000in}}{%
\pgfpathmoveto{\pgfqpoint{0.000000in}{0.000000in}}%
\pgfpathlineto{\pgfqpoint{-0.048611in}{0.000000in}}%
\pgfusepath{stroke,fill}%
}%
\begin{pgfscope}%
\pgfsys@transformshift{0.564660in}{3.422799in}%
\pgfsys@useobject{currentmarker}{}%
\end{pgfscope}%
\end{pgfscope}%
\begin{pgfscope}%
\definecolor{textcolor}{rgb}{0.000000,0.000000,0.000000}%
\pgfsetstrokecolor{textcolor}%
\pgfsetfillcolor{textcolor}%
\pgftext[x=0.289968in,y=3.370037in,left,base]{\color{textcolor}\rmfamily\fontsize{10.000000}{12.000000}\selectfont \(\displaystyle 0.7\)}%
\end{pgfscope}%
\begin{pgfscope}%
\definecolor{textcolor}{rgb}{0.000000,0.000000,0.000000}%
\pgfsetstrokecolor{textcolor}%
\pgfsetfillcolor{textcolor}%
\pgftext[x=0.234413in,y=2.031603in,,bottom,rotate=90.000000]{\color{textcolor}\rmfamily\fontsize{10.000000}{12.000000}\selectfont \(\displaystyle \mathbf{I}\mbox{g}\)}%
\end{pgfscope}%
\begin{pgfscope}%
\pgfpathrectangle{\pgfqpoint{0.564660in}{0.521603in}}{\pgfqpoint{4.650000in}{3.020000in}}%
\pgfusepath{clip}%
\pgfsetrectcap%
\pgfsetroundjoin%
\pgfsetlinewidth{1.505625pt}%
\definecolor{currentstroke}{rgb}{0.000000,0.000000,0.000000}%
\pgfsetstrokecolor{currentstroke}%
\pgfsetstrokeopacity{0.300000}%
\pgfsetdash{}{0pt}%
\pgfpathmoveto{\pgfqpoint{0.816771in}{1.337601in}}%
\pgfpathlineto{\pgfqpoint{3.858085in}{2.234101in}}%
\pgfpathlineto{\pgfqpoint{4.993344in}{2.568746in}}%
\pgfpathlineto{\pgfqpoint{1.914828in}{1.661279in}}%
\pgfpathlineto{\pgfqpoint{2.773519in}{1.914399in}}%
\pgfpathlineto{\pgfqpoint{2.234616in}{1.755544in}}%
\pgfpathlineto{\pgfqpoint{0.819079in}{1.338281in}}%
\pgfpathlineto{\pgfqpoint{1.363796in}{1.498849in}}%
\pgfpathlineto{\pgfqpoint{1.096147in}{1.419953in}}%
\pgfpathlineto{\pgfqpoint{0.887904in}{1.358569in}}%
\pgfpathlineto{\pgfqpoint{2.143771in}{1.728766in}}%
\pgfpathlineto{\pgfqpoint{1.086864in}{1.417217in}}%
\pgfpathlineto{\pgfqpoint{0.785977in}{1.328523in}}%
\pgfpathlineto{\pgfqpoint{2.036724in}{1.697211in}}%
\pgfpathlineto{\pgfqpoint{1.309364in}{1.482804in}}%
\pgfusepath{stroke}%
\end{pgfscope}%
\begin{pgfscope}%
\pgfsetrectcap%
\pgfsetmiterjoin%
\pgfsetlinewidth{0.803000pt}%
\definecolor{currentstroke}{rgb}{0.501961,0.501961,0.501961}%
\pgfsetstrokecolor{currentstroke}%
\pgfsetdash{}{0pt}%
\pgfpathmoveto{\pgfqpoint{0.564660in}{0.521603in}}%
\pgfpathlineto{\pgfqpoint{0.564660in}{3.541603in}}%
\pgfusepath{stroke}%
\end{pgfscope}%
\begin{pgfscope}%
\pgfsetrectcap%
\pgfsetmiterjoin%
\pgfsetlinewidth{0.803000pt}%
\definecolor{currentstroke}{rgb}{0.501961,0.501961,0.501961}%
\pgfsetstrokecolor{currentstroke}%
\pgfsetdash{}{0pt}%
\pgfpathmoveto{\pgfqpoint{5.214660in}{0.521603in}}%
\pgfpathlineto{\pgfqpoint{5.214660in}{3.541603in}}%
\pgfusepath{stroke}%
\end{pgfscope}%
\begin{pgfscope}%
\pgfsetrectcap%
\pgfsetmiterjoin%
\pgfsetlinewidth{0.803000pt}%
\definecolor{currentstroke}{rgb}{0.501961,0.501961,0.501961}%
\pgfsetstrokecolor{currentstroke}%
\pgfsetdash{}{0pt}%
\pgfpathmoveto{\pgfqpoint{0.564660in}{0.521603in}}%
\pgfpathlineto{\pgfqpoint{5.214660in}{0.521603in}}%
\pgfusepath{stroke}%
\end{pgfscope}%
\begin{pgfscope}%
\pgfsetrectcap%
\pgfsetmiterjoin%
\pgfsetlinewidth{0.803000pt}%
\definecolor{currentstroke}{rgb}{0.501961,0.501961,0.501961}%
\pgfsetstrokecolor{currentstroke}%
\pgfsetdash{}{0pt}%
\pgfpathmoveto{\pgfqpoint{0.564660in}{3.541603in}}%
\pgfpathlineto{\pgfqpoint{5.214660in}{3.541603in}}%
\pgfusepath{stroke}%
\end{pgfscope}%
\begin{pgfscope}%
\pgfsetbuttcap%
\pgfsetmiterjoin%
\definecolor{currentfill}{rgb}{1.000000,1.000000,1.000000}%
\pgfsetfillcolor{currentfill}%
\pgfsetfillopacity{0.800000}%
\pgfsetlinewidth{1.003750pt}%
\definecolor{currentstroke}{rgb}{0.800000,0.800000,0.800000}%
\pgfsetstrokecolor{currentstroke}%
\pgfsetstrokeopacity{0.800000}%
\pgfsetdash{}{0pt}%
\pgfpathmoveto{\pgfqpoint{2.661499in}{0.591048in}}%
\pgfpathlineto{\pgfqpoint{5.117438in}{0.591048in}}%
\pgfpathquadraticcurveto{\pgfqpoint{5.145216in}{0.591048in}}{\pgfqpoint{5.145216in}{0.618826in}}%
\pgfpathlineto{\pgfqpoint{5.145216in}{0.825238in}}%
\pgfpathquadraticcurveto{\pgfqpoint{5.145216in}{0.853016in}}{\pgfqpoint{5.117438in}{0.853016in}}%
\pgfpathlineto{\pgfqpoint{2.661499in}{0.853016in}}%
\pgfpathquadraticcurveto{\pgfqpoint{2.633721in}{0.853016in}}{\pgfqpoint{2.633721in}{0.825238in}}%
\pgfpathlineto{\pgfqpoint{2.633721in}{0.618826in}}%
\pgfpathquadraticcurveto{\pgfqpoint{2.633721in}{0.591048in}}{\pgfqpoint{2.661499in}{0.591048in}}%
\pgfpathclose%
\pgfusepath{stroke,fill}%
\end{pgfscope}%
\begin{pgfscope}%
\pgfsetrectcap%
\pgfsetroundjoin%
\pgfsetlinewidth{1.505625pt}%
\definecolor{currentstroke}{rgb}{0.000000,0.000000,0.000000}%
\pgfsetstrokecolor{currentstroke}%
\pgfsetstrokeopacity{0.300000}%
\pgfsetdash{}{0pt}%
\pgfpathmoveto{\pgfqpoint{2.689277in}{0.726071in}}%
\pgfpathlineto{\pgfqpoint{2.967055in}{0.726071in}}%
\pgfusepath{stroke}%
\end{pgfscope}%
\begin{pgfscope}%
\definecolor{textcolor}{rgb}{0.501961,0.501961,0.501961}%
\pgfsetstrokecolor{textcolor}%
\pgfsetfillcolor{textcolor}%
\pgftext[x=3.078166in,y=0.677460in,left,base]{\color{textcolor}\rmfamily\fontsize{10.000000}{12.000000}\selectfont \(\displaystyle \mathbf{I}\mbox{g} \approx 0.013 \mathbf{E}\mbox{u} + 0.230\), \(\displaystyle R^2=0.57\)}%
\end{pgfscope}%
\end{pgfpicture}%
\makeatother%
\endgroup%

    \caption{Experimental covariance between $\mathbb{E}\mbox{u}$ and $\mathbb{I}\mbox{m}$.\label{fig:dnumbs}}
\end{figure}

We should note that there are several kinds of systematic error that influence our data. We assume that drop translate purely along the central axis of the electric field, but in practice, despite the improvement in surface charge density uniformity produced by corona charging, there are still local areas of especially high charge density. In principle, this kind of error should become small for drops which are far enough away from the charge distribution, that the geometry of the charge distribution disappears, and the electric field looks like that due to a point charge. Another form of error is in the initial velocity as it appears in $\mathbb{E}\mbox{u}$. Because we usually lose the first few frames of video due to camera shake transients at the start of the low-gravity experiment, we will consistently underestimate $U_0$ because the drop will already have decelerated significantly during that period of time. The primary sources of random error are the effect of contact line hysteresis on the drop initial velocity, and of the variance in the MLE parameter estimates.

\section{Impact Dynamics}
\hl{Check the Oh numbers}. Very little work to date on drop impacts outside of two regimes: first, very low $\mathbb{R}\mbox{e}$ viscous drop spreading driven by capillary forces at the contact line, and second, impacts at ``high'' $\mathbb{W}\mbox{e}$. There has been some computational work with impacts at $\mathbb{R}\mbox{e}=1.4$ (Fukai \emph{et al.}), but the results are suspect as the model neglects uncompensated Young's force at the contact line (which is the dominant spreading force for low $\mathbb{R}\mbox{e}$ impacts). Of course models for dynamic contact lines in general remain controversial, even for ordinary spreading of liquids, despite decades of work in the area. 

Naivly neglecting dimensionless groups governing the dynamic contact line, the pertinent dimensionless groups for isothermal droplet impacts are the Weber number $\mathbb{W}\mbox{e} = \frac{\rho U^2 R_d}{\sigma}$, which is a ratio of droplet inertia to surface tension, the Ohnesorge number, $\mathbb{O}\mbox{h} = \frac{\mu}{\sqrt{\rho \sigma R_d}}$, which is a ratio of viscous to inertial and surface tension forces, and the Bond number, $\mathbb{B}\mbox{o}$, defined previously. The dynamics of spreading are characterized primarily by $\mathbb{W}\mbox{e}$ and $\mathbb{O}\mbox{h}$. Additionally the final stages of spreading depend strongly on $\theta_e$, which is the static contact angle. The Weber number scales the \emph{driving force} of the impact. In the more familiar case of high $\mathbb{W}\mbox{e}$ the drop liquid bulk is driven radially outward by the impact induced pressure gradient, whereas in the case of small $\mathbb{W}\mbox{e}$ the liquid is pulled outwards by capillary force (e.g. uncompensated Young's force) at the contact line. The Ohnesorge number, by contrast, scales the force that \emph{resists} spreading. For large $\mathbb{O}\mbox{h}$ the resistive force is viscous, whereas for low $\mathbb{O}\mbox{h}$ the force is inertia.

We observe average drop impact $\mathbb{O}\mbox{h} \approx 2.18 \pm 0.36$, and  $\mathbb{W}\mbox{e} \approx 0.28 \pm 0.22$. Thus impact velocity plays little role in the spreading dynamics of the bounces, and viscous effects are important but do not dominate inertia. Notably we observe underdamped oscillations of drop interfaces during impact. according to (Schiaffino, 1997) viscous forces play a role in the final (which? describe) stage of spreading even for drop impact at very low $\mathbb{O}\mbox{h}$. In the regime of intermediate viscosity, following spreading the oscillations of the interface are damped with a characteristic time that is generally longer than the spreading time (again, given by $t_c \sim \left( \sigma / \rho R_d^3 \right)^{1/2}$). The celebrated Hoffman-Tanner-Voinov law, which relates the contact line velocity $U_c$ of an isothermal spreading process to the static contact angle $\theta_e$ to the dynamic advancing contact angle $\theta_a$ by
\[
\frac{\mu U_c}{\sigma} \approx \kappa \left( \theta_a^3 - \theta_e^3 \right),
\]
where $\kappa \approx 0.013$ is an empirical coefficient extracted from Hoffman's data. The Hoffman-Tanner-Voinov law implies that the transition between low $\mathbb{W}\mbox{e}$ inviscid and viscous regimes occurs at $\mathbb{O}\mbox{h} \sim \mathcal{O}(10^{-2})$ rather than $\mathbb{O}\mbox{h} \sim \mathcal{O}(1)$ given by a naive scaling.

Some notes:
\begin{itemize}
\item{\textbf{Schiaffino, 1997}} 
\end{itemize}

%\newpage
\end{document}

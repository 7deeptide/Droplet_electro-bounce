\documentclass[10pt,a4paper]{article}
\usepackage[utf8]{inputenc}
\usepackage[english]{babel}
\usepackage{amsmath}
\usepackage{amsfonts}
\usepackage{amssymb}
\usepackage{graphicx}
\author{Erin Schmidt}
\title{Abstract}
\begin{document}
Our terrestrially born intuition about how liquids flow is easily confounded in a low-gravity environment. This should come as little surprise, as we are creatures evolved at the bottom of a steep gravity well; gravity is natural to us. However when the magnitude of the gravity body-force becomes small, other forces come into play in fluid dynamics which are otherwise negligible, relatively speaking, under normal circumstances in 1-g. This 1-g cognitive bias leads to a plethora of problems which are both trivially easy to solve in 1-g, and which despite decades of study continue to elude solutions in a low-gravity context. One example of these problems is the so-called "phase separation" problem of separating a gas phase from a multi-phase flow (or the reverse case) without the aid of gravitational buoyancy. Sans buoyancy otherwise mundane activities such as venting a gas or settling a liquid in a tank become problematic. Trapped bubbles can vapor lock ECLSS (Environmental Control and Life Support), power, or propulsion systems. Issues of phase separation have so bedeviled human endeavors in space that an entire Apollo Saturn 1B mission (AS-203) was earmarked to study them. This problem has for some time motivated a quest for substitute body forces, and the present work follows in that august tradition.

We investigate the dynamics of spontaneous jumps of water droplets from electrically charged superhydrophobic dielectric surfaces during a sudden step reduction in gravity level. In the brief free-fall environment of a drop tower, with a non-homogeneous external electric field (with strengths ~0.39-2.36 kV/cm) arising due to dielectric surface charges, body forces acting on the jumped droplets are primarily supplied by polarization stress and Coulombic attraction instead of gravity. This electric body force leads to a droplet bouncing behavior similar to well-known phenomena in 1-g, though occurring for much larger droplets (~0.5 mL). We show a simple model for the phenomenon, its scaling, and asymptotic estimates for droplet time of flight. The droplet net charge, estimated to be on the order of 1E-9 C, is field induced rather than by contact charging at the (PTFE coated) hydrophobic interface. In 1-g, for Weber numbers > 0.4, impact recoil behavior on a super-hydrophobic surface is normally dominated by damping from contact line hysteresis. However, at the low Bond and Ohnesorge numbers occurring in free-fall, the droplet impact dynamics additionally include electrohydrodynamic surface wettability effects. This is qualitatively discussed in terms of trends in coefficients of restitution and dimensionless contact time. 
\end{document}
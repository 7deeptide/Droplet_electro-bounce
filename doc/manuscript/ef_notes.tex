\documentclass[a4paper, 12pt]{article}
\usepackage{changepage}
\usepackage{color,soul}
\usepackage{listings}
\usepackage{verbatim}
\usepackage{pgf}
\lstset{
basicstyle=\small\ttfamily,
columns=flexible,
breaklines=true
}
\title{\textsf{\textbf{Droplet Electro-Bouncing in $\mu$-Gravity}}}
\vspace{-25mm}
\author{Erin S. Schmidt, Mark M. Weislogel}
\date{}

\usepackage{abstract}
\renewcommand{\abstractnamefont}{\normalfont\bfseries}
\renewcommand{\abstracttextfont}{\normalfont\small\itshape}

\usepackage{setspace}
\begin{document}
\maketitle

\begin{abstract}
\noindent
Notes on electric fields `n stuff.
\end{abstract}
\doublespacing
\section{The Electric Field}
\begin{comment}
Being thus armed with the assumption of a purely electrostatic problem (the electric field is conservative, that is $\nabla \times \mathbf{E} = 0$, where $\mathbf{E}$ is the electric field) we can describe the electric field by the Laplace equation for the electric potential. In the case of the volume charge density of the bulk medium to being zero, the divergence of the electric field is zero and we have

\begin{eqnarray}
\mathbf{E} &=& -\nabla \varphi \\
\nabla \cdot \mathbf{E} &=& 0\\
\nabla \cdot \mathbf{E} &=& - \nabla \cdot \nabla \varphi = \nabla^2 \varphi = 0,
\end{eqnarray}
it being convenient to solve the field in terms to $\varphi$ due to the form of the \emph{boundary conditions} usually given by
\begin{enumerate}
\item For finite charge distributions, the potential $\varphi$ goes to zero at infinity, is constant throughout a conductors, and is continuous across physical boundaries.
\item The normal component of the displacement vector $\mathbf{D}$, differs on two sides of a boundary by the free charge density $\rho_f$ residing on the boundary.
\item The tangential component of the field intensity $\mathbf{E}$ is continuous across a boundary.
\end{enumerate}

The in lieu to an analytical solution to Laplace's equation on a dielectric half-space domain by separation of variables or other methods we might try modeling the static charge distribution by a series of Dirac delta funtions after the fasion of a Green's function. In the 2D case we have,
\begin{equation}
\varphi(x,z) = \frac{z}{\pi}\int^\infty_{-\infty} \frac{\rho(x^\prime)}{(x-x^\prime)^2 + z^2}dx^\prime
\end{equation}
\end{comment}
We are faced with the need to compute the total electric field arising from the presence of free charge on the surface of a polarizable dielectric. In the electrostatic case we have
\begin{eqnarray}
\nabla \times \mathbf{E(r)}&=&0 \\
\nabla \cdot \mathbf{D(r)}&=&\rho(\mathbf{r}).
\end{eqnarray}
If the dielectric is linear and isotropic then the displacement field is 
\begin{equation}\label{displacement}
\mathbf{D(r)} = \epsilon(\mathbf{r})\mathbf{E}(\mathbf{r}).
\end{equation}
Given that the electric field is defined as the gradient of the scalar potential, $\mathbf{E}=-\nabla \varphi $, we can write equation \ref{displacement} as 
\begin{equation}
\nabla \cdot \left[\epsilon(\mathbf{r})\nabla \varphi (\mathbf{r}) \right] = -\rho (\mathbf{r}).
\end{equation}
This is a form of \emph{Poisson's equation}. Our general method for finding the electric field, will solve $\varphi (\mathbf{r})$ on a half-space domain with permittivities $\epsilon_1$, $\epsilon_2$ for a point source using the free-space \emph{Green's function}. We can then find the total electric field by superposition of the individual Green's functions by direct integration.

The non-homogoneous part of the Poisson's equation for the electrostatic potential is a Green's function, denoted by $\mathbf{G}(\mathbf{r} | \mathbf{r^\prime})$, where $\mathbf{G}$ satisfies Poisson's equation at $\mathbf{r}$ when the point source is located at $\mathbf{r^\prime}$ such that
\begin{equation}
\nabla^2 \mathbf{G}(\mathbf{r} | \mathbf{r^\prime}) = -\delta( \mathbf{r} - \mathbf{r^\prime}).
\end{equation}
Using the identity for the position vector $\nabla$

\end{document}

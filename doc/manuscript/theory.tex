\documentclass[a4paper, 12pt]{article}
\usepackage{changepage}
\usepackage{color,soul}
\usepackage{listings}
\usepackage{verbatim}
\usepackage{pgfplots}
\usepackage{pgf}
\usepackage[english]{babel}
\usepackage{amsmath}
\usepackage{amsfonts}
\usepackage{amssymb}
\usepackage{graphicx}
\usepackage{setspace}
\lstset{
basicstyle=\small\ttfamily,
columns=flexible,
breaklines=true
}
\title{\textsf{\textbf{Droplet Electro-Bouncing in $\mu$-Gravity}}}
\vspace{-25mm}
\author{Erin S. Schmidt, Mark M. Weislogel}
\date{}

\usepackage{abstract}
\renewcommand{\abstractnamefont}{\normalfont\bfseries}
\renewcommand{\abstracttextfont}{\normalfont\small\itshape}

\usepackage{setspace}
\begin{document}
%\maketitle

\doublespacing

\section{Trajectory Model}
\subsection{Forces and the Equation of Motion}
Treating a droplet as a particle, with radius $R_d$, the  equation of motion for a charged droplet translating vertically along the central axis of a finite square charged dielectric sheet is given by
\begin{equation}
m y'' = - \mathbf{F}_D - \mathbf{F}_E, \hspace{5 mm} y(0) = R_d, \hspace{5 mm} y'(0) = U_0,
\label{gov_eqn}
\end{equation}
where $m$ is the droplet mass, $y'' = \frac{d^2 y}{d t^2}$ is the droplet acceleration, $\mathbf{F}_D$ is the drag force, and $\mathbf{F}_E$ is the electrostatic force. The initial conditions are such that the droplet leaves its resting position $R_d$ at $t=0$, with initial velocity $U_0$. The signs of the forces on the right side of Equation \ref{gov_eqn} indicate that they act in the opposite direction of the droplet motion. In this section we aim specifiy models for the various forces in this equation.

If we assume an intermediate range of Reynolds numbers $\mathbb{R}\mbox{e} \equiv \frac{2UR_d}{\nu}$, $1 <\mathbb{R}\mbox{e} < 1000 $ then the drag is quadratic,
\begin{equation*}\label{drag_force}
\mathbf{F}_D = \frac{1}{2}C_D \rho A {y'}^2,
\end{equation*}
where $C_D$ is the drag coefficient, $\rho$ is the density of the sorrounding fluid medium (air, in this case), and $A$ is the frontal area of the droplet. For this range of Reynolds numbers we may also approximate the drag coefficient by the well known Abraham correlation \hl{ref.}
\[C_D = \frac{24}{9.06^2} \left( 1 + \frac{9.06}{\sqrt{\mathbb{R}\mbox{e}}} \right)^2\]

Modeling the electrostatic force is somehwhat more involved, but we will adopt the standard electrohydrodynamic (EHD) approximation model because of the dramatic simplifications it offers \hl{ref}. We first assume a DC electric field, such that $Re \langle \epsilon \rangle \approx  \mbox{constant}$, where $\epsilon$ is the dielectric permittivity of the respective media. We also assume that currents are small such that the effects of magnetic fields can be neglected. For the validity of this assumption to hold the characteristic time scale for electrical phenomena $\tau_e = \epsilon \epsilon_0/\sigma_e \ll 1$, where $\tau$ is the ratio of absolute dielectric permittivity $\kappa = \epsilon \epsilon_0$, to conductivity $\sigma_e$, of the medium [ref]. \hl{Given the respective conductivity, and permittivity of water ($\sigma_e = 2.5 \cdot 10^{-4}$ $\Omega^{-1} \mbox{cm}^{-1}$), we estimate $\tau_e \approx 7 \times 10^{-8}$ s}. This assumption also allows us to assume that the net charge present in the medium surrounding the droplets remains approximately constant during the typical time interval of a low-gravity experiment, and no transfers of charge occur after the droplet leaves the surface.

If we suppose that electrical forces acting on free charges and dipoles in a fluid are transferred directly to the fluid itself, then this overall electrical body force will be the the divergence of the Maxwell stress tensor $\tau_m $, by

\[ \mathbf{F}_E = \nabla \cdot \tau_m = \nabla \cdot \left( \epsilon \epsilon_0 \mathbf{E} \mathbf{E} - \frac{1}{2} \epsilon \epsilon_0 \mathbf{E} \cdot \mathbf{E} \delta \right) ,\]
where $\mathbf{F}_E$ is the electric body force per unit volume, and $\delta$ is the delta function. The product of the electric field vectors is the dyadic product.  

The classical Korteweg-Helmholtz force density formulation of the Maxwell stress tensor is usually expressed as \hl{[ref]}
\begin{equation}\label{force_density}
\mathbf{F}_E = \rho_f \mathbf{E} + \frac{1}{2} \left| \mathbf{E} \right|^2 \nabla \epsilon - \nabla \left( \frac{1}{2} \rho \left( \frac{\partial \epsilon}{\partial \rho} \right)_T \left| \mathbf{E} \right|^2 \right) .
\end{equation}

The first term in this expression, equivalently written as $q\mathbf{E}$, is the well known Coulombic force or electrophoretic force, which arises from the presence of free charge in an external electric field. The second term is the force arising from polarization stresses due to a nonuniform field acting across a gradient in permittivity. This force is widely termed the dielectrophoretic force (DEP). The third term describes forces due to electrostriction. It has been noted by Melcher and Hurwitz that the electrostriction term is the gradient of a scalar and can thus be cannononically lumped togeather with the hydrostatic pressure for incompressible fluids)\hl{[ref]}; we neglect it in our analysis. 

It is common to approximate the polarization stress by idealizing the droplet as a simple dipole using the effective dipole moment method first suggested by Pohl and Jones\hl{[ref][ref]}. This approach can be related back to the force density by means of a Taylor series expansion of $\mathbf{E}$ in the limit of a small gradient \hl{[ref]}. The DEP force is distinct from the Coulombic force in that net charge is not required, and that the force vector goes in the direction of the gradient of the field, $\nabla \left| \mathbf{E} \right|^2$, rather than in the direction of $\mathbf{E}$. The DEP force is related to the dipole moment (induced or permanent) of polarizable media which has a tendency to align the dipole with the electric field. If there is a gradient in the field then for a finite separation of charge one end of the dipole will feel a stronger electric field than the other, resulting in a net force. Whether the force is positive or negative in the direction of the electric field gradient depends on the difference of dielectric permittivites between the fluids, rather than on the polarity of $\mathbf{E}$ itself. It bears repeating that droplets will polarize in a uniform field, but since there is no gradient in the field the forces felt by the dipoles are symmetric and there is no net force. The dipole moment of a spherical linear-dielectric particle immersed in a dielectric medium is given by
\begin{equation} \label{dipole_m_1}
\mu = V_d \mathbf{P} = \frac{4}{3} \pi R_d^3 \mathbf{P},
\end{equation} 
where $\mathbf{P} = \left(\kappa_1 - 1 \right) \epsilon_0 \mathbf{E}_{iz} = \chi_e \epsilon_0 \mathbf{E}_{iz}$ is the polarization moment, and $R_d$ is the particle radius, \hl{$\kappa_1 = \frac{\epsilon}{\epsilon_0}$ being the relative dielectric constant of the medium (air in this case)}, $\chi_e = \kappa_1 - 1$ being the electric susceptibility of the dielectric medium, and $\mathbf{E}_{iz}$ is the $z$-coordinate component of the electric field internal to the sphere, assuming the external electric field to be oriented parallel to the $z$-axis. The excess polarization $\mathbf{P}_e$, in the sphere is given by
\begin{equation} \label{polarization}
\mathbf{P}_e = \left( \kappa_2 - \kappa_1 \right) \epsilon_0 \mathbf{E}_{iz} = \frac{3 \kappa_1}{\kappa_2 +2\kappa_1}\mathbf{E}_{iz},
\end{equation}
where $\kappa_2$ is the relative dielectric constant of the spherical particle. Taking together equations \ref{dipole_m_1}, and \ref{polarization} we find that the effective dipole moment of the particle is given by 
\begin{equation}\label{dipole_m_2}
\mu = 4 \pi R_d^3 \left( \frac{\kappa_2 - \kappa_1}{\kappa_2 + 2 \kappa_1} \right) \kappa_1 \epsilon_0 \mathbf{E},
\end{equation}
and the force felt by the dipole is 
\begin{eqnarray} \label{dep_force}
\mathbf{F}_{DEP} &=& \left( \mathbf{P}_e \cdot \nabla \right) \mathbf{E} \nonumber \\
&=& 2 \pi R_d^3 \kappa_1 \epsilon_0 K \nabla E^2,
\end{eqnarray}
where it is an asthetically pleasing shorthand to refer to the permittivity ratio by $K = \frac{\kappa_2 - \kappa_1}{\kappa_2 + 2 \kappa_1}$, which is also known as the Clausius-Mossotti factor. In cases where $K <$ 0, or $K>$ 0 the particle will be repelled or attracted to regions of strong field respectively. In our experiment, taking the relative dielectric constants to be $\kappa_1 \approx$ 1 and $\kappa_2 \approx$ 80, we have $K \approx$ 0.96. We also note that the equivalent dipole approximation requires an assumption of small physical scale of the particle relative to the lengthscale of nonuniformity of the field, which in this case we take to be the length of the charged superhydrophobic surface ($L =$ 25 mm $\gg a \approx$ 2.5 mm).

When the droplet is close to the dielectric surface, the net charge on the droplet will tend to polarize the dielectric, perturbing the electric field. The polarization bound charge in the dielectric will be of the opposite sign of the net droplet charge and thus there will be a force of attraction. This so-called image force is a correction to the Colomb force due to the external electric field only, and can be found by the method of images \hl{ref}. The image force $\mathbf{F}_I$, is given by

\begin{equation}
\mathbf{F}_I = \frac{k q^2}{16 \pi \epsilon_0} y^{-2} \hat{\mathbf{j}},
\label{image_force}
\end{equation}
where the factor $k$ is a function of the dielectric surface susceptibility $k = \frac{\chi_e}{\chi_e + 2}$, and $\hat{\mathbf{j}}$ is a unit vector normal to the dielectric surface.

By substituting Equations \ref{dep_force}, \ref{image_force} into Equation \ref{force_density} we have
\begin{eqnarray*}
 \mathbf{F}_E &=& q \mathbf{E} + \mathbf{F}_{DEP} + \mathbf{F}_I \\
 &=& q \mathbf{E} + \frac{k q^2}{16 \pi \epsilon_0 } y^{-2} \hat{\mathbf{j}} + 2 \pi R_d^3 \kappa_1 \epsilon_0 K \nabla E^2, 
\end{eqnarray*}

and the 1-D governing equation becomes
\begin{eqnarray} \label{gov_eqn_subs}
&m y'' = - \frac{1}{2} C_D \rho A {y'}^2 - q E - \frac{k q^2}{16 \pi \epsilon_0} y^{-2}- 2 \pi R_d^3 \kappa_1 \epsilon_0 K \nabla E^2,& \nonumber \\
&y(0) = R, \hspace{1 mm} y'(0) = U_0 .&
\end{eqnarray}

By comparing DEP and Coulombic terms in Equation \ref{gov_eqn_subs}, we note that a condition to neglect the DEP term is
\begin{eqnarray}
1 &\gg& \frac{R_d^2 \kappa_2 \epsilon_0 K E_0}{q} \nonumber
\end{eqnarray}
As this condition holds in all cases under study we herefore neglect the DEP force in our analysis.


\subsection{The Electric Field}
If we consider the charged dielectric surface of our experiments to be a square sheet of charge lying in the $xz$-plane with width $L$, the symmetry of the problem happily lets us obtain the $y$-component of the electric field $\mathbf{E}_y$ by direct integration \hl{ref}. In particular it is easy to constuct the electric field due to a finite plane of charge by supoerposition of the electric fields of a series of line charges. By symmetry the electric field points along the y-axis; for a point along the $y$-axis the position vector is $\mathbf{r} = \left( x^2 + y^2 + z^2 \right)^{1/2} \hat{\mathbf{r}}$. The $y$-component of $\mathbf{E}$ is given by $\mathbf{E}_y \cos \theta = \mathbf{E}_y y/ \mathbf{r}$. Then if the charge in an element of area, $dx dz$ is $\sigma dx dz$ the electric field $\mathbf{E}_y$ is
\[ \mathbf{E}_y = \frac{\sigma y}{4 \pi \epsilon_0} \int^{L/2}_{L/2} \int^{L/2}_{L/2} \left( x^2 + y^2 + z^2 \right)^{3/2} dx dz \hspace{1mm}\hat{\mathbf{r}} 
,\]
where $\sigma$ is the surface charge density. This can be easily integrated to obtain an expression for the electric field in terms of $y$, 
\begin{equation}
\mathbf{E}_y = \frac{\sigma y}{ \pi \epsilon_0} \tan^{-1} \left( \frac{L^2}{y \sqrt{2L^2 + y^2}}\right)
.\end{equation}

By taking Taylor series expansions in large and small limits we can intuit a bit about the behavior of this field. In the limit $L \rightarrow \infty, \hspace{1mm} y \ll L$ the argument of the function tends towards infinity and
\[ \lim_{x\to\infty} \tan^{-1}(x) = \frac{\pi}{2}
,\]
and thus
\begin{equation}
\mathbf{E}_y \approx \frac{\sigma}{4 \pi \epsilon_0} \hat{\mathbf{j}} \hspace{10mm} y \ll L
,\end{equation}
which is constant, and equivalent to the electric field due to an infinite plane of charge. In the limit of $y \gg L$, the argument of the arctangent function can be approximated by
\[ \frac{L^2}{2 y \left( 2 L^2 + 4 y^2\right)^{1/2}} = \frac{L^2}{4 y^2 \left( 1 + L^2/2y^2 \right)^{1/2}} \approx \frac{L^2}{4 y^2}
.\]
For small $x$, $\tan^{-1}(x) \sim x$ and we thus find the familiar electric field due to a point charge
\begin{equation}
\mathbf{E}_y \approx \frac{\sigma L^2}{4 \pi \epsilon_0} y^{-2} \hat{\mathbf{j}} \hspace{10mm} y \gg L
.\end{equation}
With the characteristic electric field given by $E_0 = \frac{\sigma}{4 \pi \epsilon_0}$, both these regiemes can be clearly seen in the plot of $\mathbf{E}_y$ shown in Figure \ref{fig:E0}.
\begin{figure}[htb]
    \centering
    %% Creator: Matplotlib, PGF backend
%%
%% To include the figure in your LaTeX document, write
%%   \input{<filename>.pgf}
%%
%% Make sure the required packages are loaded in your preamble
%%   \usepackage{pgf}
%%
%% Figures using additional raster images can only be included by \input if
%% they are in the same directory as the main LaTeX file. For loading figures
%% from other directories you can use the `import` package
%%   \usepackage{import}
%% and then include the figures with
%%   \import{<path to file>}{<filename>.pgf}
%%
%% Matplotlib used the following preamble
%%   \usepackage{fontspec}
%%   \setmainfont{DejaVu Serif}
%%   \setsansfont{DejaVu Sans}
%%   \setmonofont{DejaVu Sans Mono}
%%
\begingroup%
\makeatletter%
\begin{pgfpicture}%
\pgfpathrectangle{\pgfpointorigin}{\pgfqpoint{5.698771in}{3.862421in}}%
\pgfusepath{use as bounding box, clip}%
\begin{pgfscope}%
\pgfsetbuttcap%
\pgfsetmiterjoin%
\definecolor{currentfill}{rgb}{1.000000,1.000000,1.000000}%
\pgfsetfillcolor{currentfill}%
\pgfsetlinewidth{0.000000pt}%
\definecolor{currentstroke}{rgb}{1.000000,1.000000,1.000000}%
\pgfsetstrokecolor{currentstroke}%
\pgfsetdash{}{0pt}%
\pgfpathmoveto{\pgfqpoint{0.000000in}{0.000000in}}%
\pgfpathlineto{\pgfqpoint{5.698771in}{0.000000in}}%
\pgfpathlineto{\pgfqpoint{5.698771in}{3.862421in}}%
\pgfpathlineto{\pgfqpoint{0.000000in}{3.862421in}}%
\pgfpathclose%
\pgfusepath{fill}%
\end{pgfscope}%
\begin{pgfscope}%
\pgfsetbuttcap%
\pgfsetmiterjoin%
\definecolor{currentfill}{rgb}{1.000000,1.000000,1.000000}%
\pgfsetfillcolor{currentfill}%
\pgfsetlinewidth{0.000000pt}%
\definecolor{currentstroke}{rgb}{0.000000,0.000000,0.000000}%
\pgfsetstrokecolor{currentstroke}%
\pgfsetstrokeopacity{0.000000}%
\pgfsetdash{}{0pt}%
\pgfpathmoveto{\pgfqpoint{0.913771in}{0.707421in}}%
\pgfpathlineto{\pgfqpoint{5.563771in}{0.707421in}}%
\pgfpathlineto{\pgfqpoint{5.563771in}{3.727421in}}%
\pgfpathlineto{\pgfqpoint{0.913771in}{3.727421in}}%
\pgfpathclose%
\pgfusepath{fill}%
\end{pgfscope}%
\begin{pgfscope}%
\pgfsetbuttcap%
\pgfsetroundjoin%
\definecolor{currentfill}{rgb}{0.000000,0.000000,0.000000}%
\pgfsetfillcolor{currentfill}%
\pgfsetlinewidth{0.803000pt}%
\definecolor{currentstroke}{rgb}{0.000000,0.000000,0.000000}%
\pgfsetstrokecolor{currentstroke}%
\pgfsetdash{}{0pt}%
\pgfsys@defobject{currentmarker}{\pgfqpoint{0.000000in}{-0.048611in}}{\pgfqpoint{0.000000in}{0.000000in}}{%
\pgfpathmoveto{\pgfqpoint{0.000000in}{0.000000in}}%
\pgfpathlineto{\pgfqpoint{0.000000in}{-0.048611in}}%
\pgfusepath{stroke,fill}%
}%
\begin{pgfscope}%
\pgfsys@transformshift{2.445783in}{0.707421in}%
\pgfsys@useobject{currentmarker}{}%
\end{pgfscope}%
\end{pgfscope}%
\begin{pgfscope}%
\pgftext[x=2.445783in,y=0.610199in,,top]{\rmfamily\fontsize{16.000000}{19.200000}\selectfont \(\displaystyle 10^{-1}\)}%
\end{pgfscope}%
\begin{pgfscope}%
\pgfsetbuttcap%
\pgfsetroundjoin%
\definecolor{currentfill}{rgb}{0.000000,0.000000,0.000000}%
\pgfsetfillcolor{currentfill}%
\pgfsetlinewidth{0.803000pt}%
\definecolor{currentstroke}{rgb}{0.000000,0.000000,0.000000}%
\pgfsetstrokecolor{currentstroke}%
\pgfsetdash{}{0pt}%
\pgfsys@defobject{currentmarker}{\pgfqpoint{0.000000in}{-0.048611in}}{\pgfqpoint{0.000000in}{0.000000in}}{%
\pgfpathmoveto{\pgfqpoint{0.000000in}{0.000000in}}%
\pgfpathlineto{\pgfqpoint{0.000000in}{-0.048611in}}%
\pgfusepath{stroke,fill}%
}%
\begin{pgfscope}%
\pgfsys@transformshift{4.070370in}{0.707421in}%
\pgfsys@useobject{currentmarker}{}%
\end{pgfscope}%
\end{pgfscope}%
\begin{pgfscope}%
\pgftext[x=4.070370in,y=0.610199in,,top]{\rmfamily\fontsize{16.000000}{19.200000}\selectfont \(\displaystyle 10^{0}\)}%
\end{pgfscope}%
\begin{pgfscope}%
\pgfsetbuttcap%
\pgfsetroundjoin%
\definecolor{currentfill}{rgb}{0.000000,0.000000,0.000000}%
\pgfsetfillcolor{currentfill}%
\pgfsetlinewidth{0.602250pt}%
\definecolor{currentstroke}{rgb}{0.000000,0.000000,0.000000}%
\pgfsetstrokecolor{currentstroke}%
\pgfsetdash{}{0pt}%
\pgfsys@defobject{currentmarker}{\pgfqpoint{0.000000in}{-0.027778in}}{\pgfqpoint{0.000000in}{0.000000in}}{%
\pgfpathmoveto{\pgfqpoint{0.000000in}{0.000000in}}%
\pgfpathlineto{\pgfqpoint{0.000000in}{-0.027778in}}%
\pgfusepath{stroke,fill}%
}%
\begin{pgfscope}%
\pgfsys@transformshift{1.310245in}{0.707421in}%
\pgfsys@useobject{currentmarker}{}%
\end{pgfscope}%
\end{pgfscope}%
\begin{pgfscope}%
\pgfsetbuttcap%
\pgfsetroundjoin%
\definecolor{currentfill}{rgb}{0.000000,0.000000,0.000000}%
\pgfsetfillcolor{currentfill}%
\pgfsetlinewidth{0.602250pt}%
\definecolor{currentstroke}{rgb}{0.000000,0.000000,0.000000}%
\pgfsetstrokecolor{currentstroke}%
\pgfsetdash{}{0pt}%
\pgfsys@defobject{currentmarker}{\pgfqpoint{0.000000in}{-0.027778in}}{\pgfqpoint{0.000000in}{0.000000in}}{%
\pgfpathmoveto{\pgfqpoint{0.000000in}{0.000000in}}%
\pgfpathlineto{\pgfqpoint{0.000000in}{-0.027778in}}%
\pgfusepath{stroke,fill}%
}%
\begin{pgfscope}%
\pgfsys@transformshift{1.596321in}{0.707421in}%
\pgfsys@useobject{currentmarker}{}%
\end{pgfscope}%
\end{pgfscope}%
\begin{pgfscope}%
\pgfsetbuttcap%
\pgfsetroundjoin%
\definecolor{currentfill}{rgb}{0.000000,0.000000,0.000000}%
\pgfsetfillcolor{currentfill}%
\pgfsetlinewidth{0.602250pt}%
\definecolor{currentstroke}{rgb}{0.000000,0.000000,0.000000}%
\pgfsetstrokecolor{currentstroke}%
\pgfsetdash{}{0pt}%
\pgfsys@defobject{currentmarker}{\pgfqpoint{0.000000in}{-0.027778in}}{\pgfqpoint{0.000000in}{0.000000in}}{%
\pgfpathmoveto{\pgfqpoint{0.000000in}{0.000000in}}%
\pgfpathlineto{\pgfqpoint{0.000000in}{-0.027778in}}%
\pgfusepath{stroke,fill}%
}%
\begin{pgfscope}%
\pgfsys@transformshift{1.799295in}{0.707421in}%
\pgfsys@useobject{currentmarker}{}%
\end{pgfscope}%
\end{pgfscope}%
\begin{pgfscope}%
\pgfsetbuttcap%
\pgfsetroundjoin%
\definecolor{currentfill}{rgb}{0.000000,0.000000,0.000000}%
\pgfsetfillcolor{currentfill}%
\pgfsetlinewidth{0.602250pt}%
\definecolor{currentstroke}{rgb}{0.000000,0.000000,0.000000}%
\pgfsetstrokecolor{currentstroke}%
\pgfsetdash{}{0pt}%
\pgfsys@defobject{currentmarker}{\pgfqpoint{0.000000in}{-0.027778in}}{\pgfqpoint{0.000000in}{0.000000in}}{%
\pgfpathmoveto{\pgfqpoint{0.000000in}{0.000000in}}%
\pgfpathlineto{\pgfqpoint{0.000000in}{-0.027778in}}%
\pgfusepath{stroke,fill}%
}%
\begin{pgfscope}%
\pgfsys@transformshift{1.956733in}{0.707421in}%
\pgfsys@useobject{currentmarker}{}%
\end{pgfscope}%
\end{pgfscope}%
\begin{pgfscope}%
\pgfsetbuttcap%
\pgfsetroundjoin%
\definecolor{currentfill}{rgb}{0.000000,0.000000,0.000000}%
\pgfsetfillcolor{currentfill}%
\pgfsetlinewidth{0.602250pt}%
\definecolor{currentstroke}{rgb}{0.000000,0.000000,0.000000}%
\pgfsetstrokecolor{currentstroke}%
\pgfsetdash{}{0pt}%
\pgfsys@defobject{currentmarker}{\pgfqpoint{0.000000in}{-0.027778in}}{\pgfqpoint{0.000000in}{0.000000in}}{%
\pgfpathmoveto{\pgfqpoint{0.000000in}{0.000000in}}%
\pgfpathlineto{\pgfqpoint{0.000000in}{-0.027778in}}%
\pgfusepath{stroke,fill}%
}%
\begin{pgfscope}%
\pgfsys@transformshift{2.085370in}{0.707421in}%
\pgfsys@useobject{currentmarker}{}%
\end{pgfscope}%
\end{pgfscope}%
\begin{pgfscope}%
\pgfsetbuttcap%
\pgfsetroundjoin%
\definecolor{currentfill}{rgb}{0.000000,0.000000,0.000000}%
\pgfsetfillcolor{currentfill}%
\pgfsetlinewidth{0.602250pt}%
\definecolor{currentstroke}{rgb}{0.000000,0.000000,0.000000}%
\pgfsetstrokecolor{currentstroke}%
\pgfsetdash{}{0pt}%
\pgfsys@defobject{currentmarker}{\pgfqpoint{0.000000in}{-0.027778in}}{\pgfqpoint{0.000000in}{0.000000in}}{%
\pgfpathmoveto{\pgfqpoint{0.000000in}{0.000000in}}%
\pgfpathlineto{\pgfqpoint{0.000000in}{-0.027778in}}%
\pgfusepath{stroke,fill}%
}%
\begin{pgfscope}%
\pgfsys@transformshift{2.194131in}{0.707421in}%
\pgfsys@useobject{currentmarker}{}%
\end{pgfscope}%
\end{pgfscope}%
\begin{pgfscope}%
\pgfsetbuttcap%
\pgfsetroundjoin%
\definecolor{currentfill}{rgb}{0.000000,0.000000,0.000000}%
\pgfsetfillcolor{currentfill}%
\pgfsetlinewidth{0.602250pt}%
\definecolor{currentstroke}{rgb}{0.000000,0.000000,0.000000}%
\pgfsetstrokecolor{currentstroke}%
\pgfsetdash{}{0pt}%
\pgfsys@defobject{currentmarker}{\pgfqpoint{0.000000in}{-0.027778in}}{\pgfqpoint{0.000000in}{0.000000in}}{%
\pgfpathmoveto{\pgfqpoint{0.000000in}{0.000000in}}%
\pgfpathlineto{\pgfqpoint{0.000000in}{-0.027778in}}%
\pgfusepath{stroke,fill}%
}%
\begin{pgfscope}%
\pgfsys@transformshift{2.288344in}{0.707421in}%
\pgfsys@useobject{currentmarker}{}%
\end{pgfscope}%
\end{pgfscope}%
\begin{pgfscope}%
\pgfsetbuttcap%
\pgfsetroundjoin%
\definecolor{currentfill}{rgb}{0.000000,0.000000,0.000000}%
\pgfsetfillcolor{currentfill}%
\pgfsetlinewidth{0.602250pt}%
\definecolor{currentstroke}{rgb}{0.000000,0.000000,0.000000}%
\pgfsetstrokecolor{currentstroke}%
\pgfsetdash{}{0pt}%
\pgfsys@defobject{currentmarker}{\pgfqpoint{0.000000in}{-0.027778in}}{\pgfqpoint{0.000000in}{0.000000in}}{%
\pgfpathmoveto{\pgfqpoint{0.000000in}{0.000000in}}%
\pgfpathlineto{\pgfqpoint{0.000000in}{-0.027778in}}%
\pgfusepath{stroke,fill}%
}%
\begin{pgfscope}%
\pgfsys@transformshift{2.371446in}{0.707421in}%
\pgfsys@useobject{currentmarker}{}%
\end{pgfscope}%
\end{pgfscope}%
\begin{pgfscope}%
\pgfsetbuttcap%
\pgfsetroundjoin%
\definecolor{currentfill}{rgb}{0.000000,0.000000,0.000000}%
\pgfsetfillcolor{currentfill}%
\pgfsetlinewidth{0.602250pt}%
\definecolor{currentstroke}{rgb}{0.000000,0.000000,0.000000}%
\pgfsetstrokecolor{currentstroke}%
\pgfsetdash{}{0pt}%
\pgfsys@defobject{currentmarker}{\pgfqpoint{0.000000in}{-0.027778in}}{\pgfqpoint{0.000000in}{0.000000in}}{%
\pgfpathmoveto{\pgfqpoint{0.000000in}{0.000000in}}%
\pgfpathlineto{\pgfqpoint{0.000000in}{-0.027778in}}%
\pgfusepath{stroke,fill}%
}%
\begin{pgfscope}%
\pgfsys@transformshift{2.934832in}{0.707421in}%
\pgfsys@useobject{currentmarker}{}%
\end{pgfscope}%
\end{pgfscope}%
\begin{pgfscope}%
\pgfsetbuttcap%
\pgfsetroundjoin%
\definecolor{currentfill}{rgb}{0.000000,0.000000,0.000000}%
\pgfsetfillcolor{currentfill}%
\pgfsetlinewidth{0.602250pt}%
\definecolor{currentstroke}{rgb}{0.000000,0.000000,0.000000}%
\pgfsetstrokecolor{currentstroke}%
\pgfsetdash{}{0pt}%
\pgfsys@defobject{currentmarker}{\pgfqpoint{0.000000in}{-0.027778in}}{\pgfqpoint{0.000000in}{0.000000in}}{%
\pgfpathmoveto{\pgfqpoint{0.000000in}{0.000000in}}%
\pgfpathlineto{\pgfqpoint{0.000000in}{-0.027778in}}%
\pgfusepath{stroke,fill}%
}%
\begin{pgfscope}%
\pgfsys@transformshift{3.220908in}{0.707421in}%
\pgfsys@useobject{currentmarker}{}%
\end{pgfscope}%
\end{pgfscope}%
\begin{pgfscope}%
\pgfsetbuttcap%
\pgfsetroundjoin%
\definecolor{currentfill}{rgb}{0.000000,0.000000,0.000000}%
\pgfsetfillcolor{currentfill}%
\pgfsetlinewidth{0.602250pt}%
\definecolor{currentstroke}{rgb}{0.000000,0.000000,0.000000}%
\pgfsetstrokecolor{currentstroke}%
\pgfsetdash{}{0pt}%
\pgfsys@defobject{currentmarker}{\pgfqpoint{0.000000in}{-0.027778in}}{\pgfqpoint{0.000000in}{0.000000in}}{%
\pgfpathmoveto{\pgfqpoint{0.000000in}{0.000000in}}%
\pgfpathlineto{\pgfqpoint{0.000000in}{-0.027778in}}%
\pgfusepath{stroke,fill}%
}%
\begin{pgfscope}%
\pgfsys@transformshift{3.423882in}{0.707421in}%
\pgfsys@useobject{currentmarker}{}%
\end{pgfscope}%
\end{pgfscope}%
\begin{pgfscope}%
\pgfsetbuttcap%
\pgfsetroundjoin%
\definecolor{currentfill}{rgb}{0.000000,0.000000,0.000000}%
\pgfsetfillcolor{currentfill}%
\pgfsetlinewidth{0.602250pt}%
\definecolor{currentstroke}{rgb}{0.000000,0.000000,0.000000}%
\pgfsetstrokecolor{currentstroke}%
\pgfsetdash{}{0pt}%
\pgfsys@defobject{currentmarker}{\pgfqpoint{0.000000in}{-0.027778in}}{\pgfqpoint{0.000000in}{0.000000in}}{%
\pgfpathmoveto{\pgfqpoint{0.000000in}{0.000000in}}%
\pgfpathlineto{\pgfqpoint{0.000000in}{-0.027778in}}%
\pgfusepath{stroke,fill}%
}%
\begin{pgfscope}%
\pgfsys@transformshift{3.581320in}{0.707421in}%
\pgfsys@useobject{currentmarker}{}%
\end{pgfscope}%
\end{pgfscope}%
\begin{pgfscope}%
\pgfsetbuttcap%
\pgfsetroundjoin%
\definecolor{currentfill}{rgb}{0.000000,0.000000,0.000000}%
\pgfsetfillcolor{currentfill}%
\pgfsetlinewidth{0.602250pt}%
\definecolor{currentstroke}{rgb}{0.000000,0.000000,0.000000}%
\pgfsetstrokecolor{currentstroke}%
\pgfsetdash{}{0pt}%
\pgfsys@defobject{currentmarker}{\pgfqpoint{0.000000in}{-0.027778in}}{\pgfqpoint{0.000000in}{0.000000in}}{%
\pgfpathmoveto{\pgfqpoint{0.000000in}{0.000000in}}%
\pgfpathlineto{\pgfqpoint{0.000000in}{-0.027778in}}%
\pgfusepath{stroke,fill}%
}%
\begin{pgfscope}%
\pgfsys@transformshift{3.709957in}{0.707421in}%
\pgfsys@useobject{currentmarker}{}%
\end{pgfscope}%
\end{pgfscope}%
\begin{pgfscope}%
\pgfsetbuttcap%
\pgfsetroundjoin%
\definecolor{currentfill}{rgb}{0.000000,0.000000,0.000000}%
\pgfsetfillcolor{currentfill}%
\pgfsetlinewidth{0.602250pt}%
\definecolor{currentstroke}{rgb}{0.000000,0.000000,0.000000}%
\pgfsetstrokecolor{currentstroke}%
\pgfsetdash{}{0pt}%
\pgfsys@defobject{currentmarker}{\pgfqpoint{0.000000in}{-0.027778in}}{\pgfqpoint{0.000000in}{0.000000in}}{%
\pgfpathmoveto{\pgfqpoint{0.000000in}{0.000000in}}%
\pgfpathlineto{\pgfqpoint{0.000000in}{-0.027778in}}%
\pgfusepath{stroke,fill}%
}%
\begin{pgfscope}%
\pgfsys@transformshift{3.818718in}{0.707421in}%
\pgfsys@useobject{currentmarker}{}%
\end{pgfscope}%
\end{pgfscope}%
\begin{pgfscope}%
\pgfsetbuttcap%
\pgfsetroundjoin%
\definecolor{currentfill}{rgb}{0.000000,0.000000,0.000000}%
\pgfsetfillcolor{currentfill}%
\pgfsetlinewidth{0.602250pt}%
\definecolor{currentstroke}{rgb}{0.000000,0.000000,0.000000}%
\pgfsetstrokecolor{currentstroke}%
\pgfsetdash{}{0pt}%
\pgfsys@defobject{currentmarker}{\pgfqpoint{0.000000in}{-0.027778in}}{\pgfqpoint{0.000000in}{0.000000in}}{%
\pgfpathmoveto{\pgfqpoint{0.000000in}{0.000000in}}%
\pgfpathlineto{\pgfqpoint{0.000000in}{-0.027778in}}%
\pgfusepath{stroke,fill}%
}%
\begin{pgfscope}%
\pgfsys@transformshift{3.912931in}{0.707421in}%
\pgfsys@useobject{currentmarker}{}%
\end{pgfscope}%
\end{pgfscope}%
\begin{pgfscope}%
\pgfsetbuttcap%
\pgfsetroundjoin%
\definecolor{currentfill}{rgb}{0.000000,0.000000,0.000000}%
\pgfsetfillcolor{currentfill}%
\pgfsetlinewidth{0.602250pt}%
\definecolor{currentstroke}{rgb}{0.000000,0.000000,0.000000}%
\pgfsetstrokecolor{currentstroke}%
\pgfsetdash{}{0pt}%
\pgfsys@defobject{currentmarker}{\pgfqpoint{0.000000in}{-0.027778in}}{\pgfqpoint{0.000000in}{0.000000in}}{%
\pgfpathmoveto{\pgfqpoint{0.000000in}{0.000000in}}%
\pgfpathlineto{\pgfqpoint{0.000000in}{-0.027778in}}%
\pgfusepath{stroke,fill}%
}%
\begin{pgfscope}%
\pgfsys@transformshift{3.996033in}{0.707421in}%
\pgfsys@useobject{currentmarker}{}%
\end{pgfscope}%
\end{pgfscope}%
\begin{pgfscope}%
\pgfsetbuttcap%
\pgfsetroundjoin%
\definecolor{currentfill}{rgb}{0.000000,0.000000,0.000000}%
\pgfsetfillcolor{currentfill}%
\pgfsetlinewidth{0.602250pt}%
\definecolor{currentstroke}{rgb}{0.000000,0.000000,0.000000}%
\pgfsetstrokecolor{currentstroke}%
\pgfsetdash{}{0pt}%
\pgfsys@defobject{currentmarker}{\pgfqpoint{0.000000in}{-0.027778in}}{\pgfqpoint{0.000000in}{0.000000in}}{%
\pgfpathmoveto{\pgfqpoint{0.000000in}{0.000000in}}%
\pgfpathlineto{\pgfqpoint{0.000000in}{-0.027778in}}%
\pgfusepath{stroke,fill}%
}%
\begin{pgfscope}%
\pgfsys@transformshift{4.559419in}{0.707421in}%
\pgfsys@useobject{currentmarker}{}%
\end{pgfscope}%
\end{pgfscope}%
\begin{pgfscope}%
\pgfsetbuttcap%
\pgfsetroundjoin%
\definecolor{currentfill}{rgb}{0.000000,0.000000,0.000000}%
\pgfsetfillcolor{currentfill}%
\pgfsetlinewidth{0.602250pt}%
\definecolor{currentstroke}{rgb}{0.000000,0.000000,0.000000}%
\pgfsetstrokecolor{currentstroke}%
\pgfsetdash{}{0pt}%
\pgfsys@defobject{currentmarker}{\pgfqpoint{0.000000in}{-0.027778in}}{\pgfqpoint{0.000000in}{0.000000in}}{%
\pgfpathmoveto{\pgfqpoint{0.000000in}{0.000000in}}%
\pgfpathlineto{\pgfqpoint{0.000000in}{-0.027778in}}%
\pgfusepath{stroke,fill}%
}%
\begin{pgfscope}%
\pgfsys@transformshift{4.845495in}{0.707421in}%
\pgfsys@useobject{currentmarker}{}%
\end{pgfscope}%
\end{pgfscope}%
\begin{pgfscope}%
\pgfsetbuttcap%
\pgfsetroundjoin%
\definecolor{currentfill}{rgb}{0.000000,0.000000,0.000000}%
\pgfsetfillcolor{currentfill}%
\pgfsetlinewidth{0.602250pt}%
\definecolor{currentstroke}{rgb}{0.000000,0.000000,0.000000}%
\pgfsetstrokecolor{currentstroke}%
\pgfsetdash{}{0pt}%
\pgfsys@defobject{currentmarker}{\pgfqpoint{0.000000in}{-0.027778in}}{\pgfqpoint{0.000000in}{0.000000in}}{%
\pgfpathmoveto{\pgfqpoint{0.000000in}{0.000000in}}%
\pgfpathlineto{\pgfqpoint{0.000000in}{-0.027778in}}%
\pgfusepath{stroke,fill}%
}%
\begin{pgfscope}%
\pgfsys@transformshift{5.048469in}{0.707421in}%
\pgfsys@useobject{currentmarker}{}%
\end{pgfscope}%
\end{pgfscope}%
\begin{pgfscope}%
\pgfsetbuttcap%
\pgfsetroundjoin%
\definecolor{currentfill}{rgb}{0.000000,0.000000,0.000000}%
\pgfsetfillcolor{currentfill}%
\pgfsetlinewidth{0.602250pt}%
\definecolor{currentstroke}{rgb}{0.000000,0.000000,0.000000}%
\pgfsetstrokecolor{currentstroke}%
\pgfsetdash{}{0pt}%
\pgfsys@defobject{currentmarker}{\pgfqpoint{0.000000in}{-0.027778in}}{\pgfqpoint{0.000000in}{0.000000in}}{%
\pgfpathmoveto{\pgfqpoint{0.000000in}{0.000000in}}%
\pgfpathlineto{\pgfqpoint{0.000000in}{-0.027778in}}%
\pgfusepath{stroke,fill}%
}%
\begin{pgfscope}%
\pgfsys@transformshift{5.205907in}{0.707421in}%
\pgfsys@useobject{currentmarker}{}%
\end{pgfscope}%
\end{pgfscope}%
\begin{pgfscope}%
\pgfsetbuttcap%
\pgfsetroundjoin%
\definecolor{currentfill}{rgb}{0.000000,0.000000,0.000000}%
\pgfsetfillcolor{currentfill}%
\pgfsetlinewidth{0.602250pt}%
\definecolor{currentstroke}{rgb}{0.000000,0.000000,0.000000}%
\pgfsetstrokecolor{currentstroke}%
\pgfsetdash{}{0pt}%
\pgfsys@defobject{currentmarker}{\pgfqpoint{0.000000in}{-0.027778in}}{\pgfqpoint{0.000000in}{0.000000in}}{%
\pgfpathmoveto{\pgfqpoint{0.000000in}{0.000000in}}%
\pgfpathlineto{\pgfqpoint{0.000000in}{-0.027778in}}%
\pgfusepath{stroke,fill}%
}%
\begin{pgfscope}%
\pgfsys@transformshift{5.334544in}{0.707421in}%
\pgfsys@useobject{currentmarker}{}%
\end{pgfscope}%
\end{pgfscope}%
\begin{pgfscope}%
\pgfsetbuttcap%
\pgfsetroundjoin%
\definecolor{currentfill}{rgb}{0.000000,0.000000,0.000000}%
\pgfsetfillcolor{currentfill}%
\pgfsetlinewidth{0.602250pt}%
\definecolor{currentstroke}{rgb}{0.000000,0.000000,0.000000}%
\pgfsetstrokecolor{currentstroke}%
\pgfsetdash{}{0pt}%
\pgfsys@defobject{currentmarker}{\pgfqpoint{0.000000in}{-0.027778in}}{\pgfqpoint{0.000000in}{0.000000in}}{%
\pgfpathmoveto{\pgfqpoint{0.000000in}{0.000000in}}%
\pgfpathlineto{\pgfqpoint{0.000000in}{-0.027778in}}%
\pgfusepath{stroke,fill}%
}%
\begin{pgfscope}%
\pgfsys@transformshift{5.443305in}{0.707421in}%
\pgfsys@useobject{currentmarker}{}%
\end{pgfscope}%
\end{pgfscope}%
\begin{pgfscope}%
\pgfsetbuttcap%
\pgfsetroundjoin%
\definecolor{currentfill}{rgb}{0.000000,0.000000,0.000000}%
\pgfsetfillcolor{currentfill}%
\pgfsetlinewidth{0.602250pt}%
\definecolor{currentstroke}{rgb}{0.000000,0.000000,0.000000}%
\pgfsetstrokecolor{currentstroke}%
\pgfsetdash{}{0pt}%
\pgfsys@defobject{currentmarker}{\pgfqpoint{0.000000in}{-0.027778in}}{\pgfqpoint{0.000000in}{0.000000in}}{%
\pgfpathmoveto{\pgfqpoint{0.000000in}{0.000000in}}%
\pgfpathlineto{\pgfqpoint{0.000000in}{-0.027778in}}%
\pgfusepath{stroke,fill}%
}%
\begin{pgfscope}%
\pgfsys@transformshift{5.537518in}{0.707421in}%
\pgfsys@useobject{currentmarker}{}%
\end{pgfscope}%
\end{pgfscope}%
\begin{pgfscope}%
\pgftext[x=3.238771in,y=0.339583in,,top]{\rmfamily\fontsize{16.000000}{19.200000}\selectfont \(\displaystyle y/L\)}%
\end{pgfscope}%
\begin{pgfscope}%
\pgfsetbuttcap%
\pgfsetroundjoin%
\definecolor{currentfill}{rgb}{0.000000,0.000000,0.000000}%
\pgfsetfillcolor{currentfill}%
\pgfsetlinewidth{0.803000pt}%
\definecolor{currentstroke}{rgb}{0.000000,0.000000,0.000000}%
\pgfsetstrokecolor{currentstroke}%
\pgfsetdash{}{0pt}%
\pgfsys@defobject{currentmarker}{\pgfqpoint{-0.048611in}{0.000000in}}{\pgfqpoint{0.000000in}{0.000000in}}{%
\pgfpathmoveto{\pgfqpoint{0.000000in}{0.000000in}}%
\pgfpathlineto{\pgfqpoint{-0.048611in}{0.000000in}}%
\pgfusepath{stroke,fill}%
}%
\begin{pgfscope}%
\pgfsys@transformshift{0.913771in}{1.286260in}%
\pgfsys@useobject{currentmarker}{}%
\end{pgfscope}%
\end{pgfscope}%
\begin{pgfscope}%
\pgftext[x=0.395138in,y=1.201841in,left,base]{\rmfamily\fontsize{16.000000}{19.200000}\selectfont \(\displaystyle 10^{-2}\)}%
\end{pgfscope}%
\begin{pgfscope}%
\pgfsetbuttcap%
\pgfsetroundjoin%
\definecolor{currentfill}{rgb}{0.000000,0.000000,0.000000}%
\pgfsetfillcolor{currentfill}%
\pgfsetlinewidth{0.803000pt}%
\definecolor{currentstroke}{rgb}{0.000000,0.000000,0.000000}%
\pgfsetstrokecolor{currentstroke}%
\pgfsetdash{}{0pt}%
\pgfsys@defobject{currentmarker}{\pgfqpoint{-0.048611in}{0.000000in}}{\pgfqpoint{0.000000in}{0.000000in}}{%
\pgfpathmoveto{\pgfqpoint{0.000000in}{0.000000in}}%
\pgfpathlineto{\pgfqpoint{-0.048611in}{0.000000in}}%
\pgfusepath{stroke,fill}%
}%
\begin{pgfscope}%
\pgfsys@transformshift{0.913771in}{2.438211in}%
\pgfsys@useobject{currentmarker}{}%
\end{pgfscope}%
\end{pgfscope}%
\begin{pgfscope}%
\pgftext[x=0.395138in,y=2.353793in,left,base]{\rmfamily\fontsize{16.000000}{19.200000}\selectfont \(\displaystyle 10^{-1}\)}%
\end{pgfscope}%
\begin{pgfscope}%
\pgfsetbuttcap%
\pgfsetroundjoin%
\definecolor{currentfill}{rgb}{0.000000,0.000000,0.000000}%
\pgfsetfillcolor{currentfill}%
\pgfsetlinewidth{0.803000pt}%
\definecolor{currentstroke}{rgb}{0.000000,0.000000,0.000000}%
\pgfsetstrokecolor{currentstroke}%
\pgfsetdash{}{0pt}%
\pgfsys@defobject{currentmarker}{\pgfqpoint{-0.048611in}{0.000000in}}{\pgfqpoint{0.000000in}{0.000000in}}{%
\pgfpathmoveto{\pgfqpoint{0.000000in}{0.000000in}}%
\pgfpathlineto{\pgfqpoint{-0.048611in}{0.000000in}}%
\pgfusepath{stroke,fill}%
}%
\begin{pgfscope}%
\pgfsys@transformshift{0.913771in}{3.590162in}%
\pgfsys@useobject{currentmarker}{}%
\end{pgfscope}%
\end{pgfscope}%
\begin{pgfscope}%
\pgftext[x=0.513426in,y=3.505744in,left,base]{\rmfamily\fontsize{16.000000}{19.200000}\selectfont \(\displaystyle 10^{0}\)}%
\end{pgfscope}%
\begin{pgfscope}%
\pgfsetbuttcap%
\pgfsetroundjoin%
\definecolor{currentfill}{rgb}{0.000000,0.000000,0.000000}%
\pgfsetfillcolor{currentfill}%
\pgfsetlinewidth{0.602250pt}%
\definecolor{currentstroke}{rgb}{0.000000,0.000000,0.000000}%
\pgfsetstrokecolor{currentstroke}%
\pgfsetdash{}{0pt}%
\pgfsys@defobject{currentmarker}{\pgfqpoint{-0.027778in}{0.000000in}}{\pgfqpoint{0.000000in}{0.000000in}}{%
\pgfpathmoveto{\pgfqpoint{0.000000in}{0.000000in}}%
\pgfpathlineto{\pgfqpoint{-0.027778in}{0.000000in}}%
\pgfusepath{stroke,fill}%
}%
\begin{pgfscope}%
\pgfsys@transformshift{0.913771in}{0.827852in}%
\pgfsys@useobject{currentmarker}{}%
\end{pgfscope}%
\end{pgfscope}%
\begin{pgfscope}%
\pgfsetbuttcap%
\pgfsetroundjoin%
\definecolor{currentfill}{rgb}{0.000000,0.000000,0.000000}%
\pgfsetfillcolor{currentfill}%
\pgfsetlinewidth{0.602250pt}%
\definecolor{currentstroke}{rgb}{0.000000,0.000000,0.000000}%
\pgfsetstrokecolor{currentstroke}%
\pgfsetdash{}{0pt}%
\pgfsys@defobject{currentmarker}{\pgfqpoint{-0.027778in}{0.000000in}}{\pgfqpoint{0.000000in}{0.000000in}}{%
\pgfpathmoveto{\pgfqpoint{0.000000in}{0.000000in}}%
\pgfpathlineto{\pgfqpoint{-0.027778in}{0.000000in}}%
\pgfusepath{stroke,fill}%
}%
\begin{pgfscope}%
\pgfsys@transformshift{0.913771in}{0.939488in}%
\pgfsys@useobject{currentmarker}{}%
\end{pgfscope}%
\end{pgfscope}%
\begin{pgfscope}%
\pgfsetbuttcap%
\pgfsetroundjoin%
\definecolor{currentfill}{rgb}{0.000000,0.000000,0.000000}%
\pgfsetfillcolor{currentfill}%
\pgfsetlinewidth{0.602250pt}%
\definecolor{currentstroke}{rgb}{0.000000,0.000000,0.000000}%
\pgfsetstrokecolor{currentstroke}%
\pgfsetdash{}{0pt}%
\pgfsys@defobject{currentmarker}{\pgfqpoint{-0.027778in}{0.000000in}}{\pgfqpoint{0.000000in}{0.000000in}}{%
\pgfpathmoveto{\pgfqpoint{0.000000in}{0.000000in}}%
\pgfpathlineto{\pgfqpoint{-0.027778in}{0.000000in}}%
\pgfusepath{stroke,fill}%
}%
\begin{pgfscope}%
\pgfsys@transformshift{0.913771in}{1.030701in}%
\pgfsys@useobject{currentmarker}{}%
\end{pgfscope}%
\end{pgfscope}%
\begin{pgfscope}%
\pgfsetbuttcap%
\pgfsetroundjoin%
\definecolor{currentfill}{rgb}{0.000000,0.000000,0.000000}%
\pgfsetfillcolor{currentfill}%
\pgfsetlinewidth{0.602250pt}%
\definecolor{currentstroke}{rgb}{0.000000,0.000000,0.000000}%
\pgfsetstrokecolor{currentstroke}%
\pgfsetdash{}{0pt}%
\pgfsys@defobject{currentmarker}{\pgfqpoint{-0.027778in}{0.000000in}}{\pgfqpoint{0.000000in}{0.000000in}}{%
\pgfpathmoveto{\pgfqpoint{0.000000in}{0.000000in}}%
\pgfpathlineto{\pgfqpoint{-0.027778in}{0.000000in}}%
\pgfusepath{stroke,fill}%
}%
\begin{pgfscope}%
\pgfsys@transformshift{0.913771in}{1.107820in}%
\pgfsys@useobject{currentmarker}{}%
\end{pgfscope}%
\end{pgfscope}%
\begin{pgfscope}%
\pgfsetbuttcap%
\pgfsetroundjoin%
\definecolor{currentfill}{rgb}{0.000000,0.000000,0.000000}%
\pgfsetfillcolor{currentfill}%
\pgfsetlinewidth{0.602250pt}%
\definecolor{currentstroke}{rgb}{0.000000,0.000000,0.000000}%
\pgfsetstrokecolor{currentstroke}%
\pgfsetdash{}{0pt}%
\pgfsys@defobject{currentmarker}{\pgfqpoint{-0.027778in}{0.000000in}}{\pgfqpoint{0.000000in}{0.000000in}}{%
\pgfpathmoveto{\pgfqpoint{0.000000in}{0.000000in}}%
\pgfpathlineto{\pgfqpoint{-0.027778in}{0.000000in}}%
\pgfusepath{stroke,fill}%
}%
\begin{pgfscope}%
\pgfsys@transformshift{0.913771in}{1.174624in}%
\pgfsys@useobject{currentmarker}{}%
\end{pgfscope}%
\end{pgfscope}%
\begin{pgfscope}%
\pgfsetbuttcap%
\pgfsetroundjoin%
\definecolor{currentfill}{rgb}{0.000000,0.000000,0.000000}%
\pgfsetfillcolor{currentfill}%
\pgfsetlinewidth{0.602250pt}%
\definecolor{currentstroke}{rgb}{0.000000,0.000000,0.000000}%
\pgfsetstrokecolor{currentstroke}%
\pgfsetdash{}{0pt}%
\pgfsys@defobject{currentmarker}{\pgfqpoint{-0.027778in}{0.000000in}}{\pgfqpoint{0.000000in}{0.000000in}}{%
\pgfpathmoveto{\pgfqpoint{0.000000in}{0.000000in}}%
\pgfpathlineto{\pgfqpoint{-0.027778in}{0.000000in}}%
\pgfusepath{stroke,fill}%
}%
\begin{pgfscope}%
\pgfsys@transformshift{0.913771in}{1.233549in}%
\pgfsys@useobject{currentmarker}{}%
\end{pgfscope}%
\end{pgfscope}%
\begin{pgfscope}%
\pgfsetbuttcap%
\pgfsetroundjoin%
\definecolor{currentfill}{rgb}{0.000000,0.000000,0.000000}%
\pgfsetfillcolor{currentfill}%
\pgfsetlinewidth{0.602250pt}%
\definecolor{currentstroke}{rgb}{0.000000,0.000000,0.000000}%
\pgfsetstrokecolor{currentstroke}%
\pgfsetdash{}{0pt}%
\pgfsys@defobject{currentmarker}{\pgfqpoint{-0.027778in}{0.000000in}}{\pgfqpoint{0.000000in}{0.000000in}}{%
\pgfpathmoveto{\pgfqpoint{0.000000in}{0.000000in}}%
\pgfpathlineto{\pgfqpoint{-0.027778in}{0.000000in}}%
\pgfusepath{stroke,fill}%
}%
\begin{pgfscope}%
\pgfsys@transformshift{0.913771in}{1.633031in}%
\pgfsys@useobject{currentmarker}{}%
\end{pgfscope}%
\end{pgfscope}%
\begin{pgfscope}%
\pgfsetbuttcap%
\pgfsetroundjoin%
\definecolor{currentfill}{rgb}{0.000000,0.000000,0.000000}%
\pgfsetfillcolor{currentfill}%
\pgfsetlinewidth{0.602250pt}%
\definecolor{currentstroke}{rgb}{0.000000,0.000000,0.000000}%
\pgfsetstrokecolor{currentstroke}%
\pgfsetdash{}{0pt}%
\pgfsys@defobject{currentmarker}{\pgfqpoint{-0.027778in}{0.000000in}}{\pgfqpoint{0.000000in}{0.000000in}}{%
\pgfpathmoveto{\pgfqpoint{0.000000in}{0.000000in}}%
\pgfpathlineto{\pgfqpoint{-0.027778in}{0.000000in}}%
\pgfusepath{stroke,fill}%
}%
\begin{pgfscope}%
\pgfsys@transformshift{0.913771in}{1.835880in}%
\pgfsys@useobject{currentmarker}{}%
\end{pgfscope}%
\end{pgfscope}%
\begin{pgfscope}%
\pgfsetbuttcap%
\pgfsetroundjoin%
\definecolor{currentfill}{rgb}{0.000000,0.000000,0.000000}%
\pgfsetfillcolor{currentfill}%
\pgfsetlinewidth{0.602250pt}%
\definecolor{currentstroke}{rgb}{0.000000,0.000000,0.000000}%
\pgfsetstrokecolor{currentstroke}%
\pgfsetdash{}{0pt}%
\pgfsys@defobject{currentmarker}{\pgfqpoint{-0.027778in}{0.000000in}}{\pgfqpoint{0.000000in}{0.000000in}}{%
\pgfpathmoveto{\pgfqpoint{0.000000in}{0.000000in}}%
\pgfpathlineto{\pgfqpoint{-0.027778in}{0.000000in}}%
\pgfusepath{stroke,fill}%
}%
\begin{pgfscope}%
\pgfsys@transformshift{0.913771in}{1.979803in}%
\pgfsys@useobject{currentmarker}{}%
\end{pgfscope}%
\end{pgfscope}%
\begin{pgfscope}%
\pgfsetbuttcap%
\pgfsetroundjoin%
\definecolor{currentfill}{rgb}{0.000000,0.000000,0.000000}%
\pgfsetfillcolor{currentfill}%
\pgfsetlinewidth{0.602250pt}%
\definecolor{currentstroke}{rgb}{0.000000,0.000000,0.000000}%
\pgfsetstrokecolor{currentstroke}%
\pgfsetdash{}{0pt}%
\pgfsys@defobject{currentmarker}{\pgfqpoint{-0.027778in}{0.000000in}}{\pgfqpoint{0.000000in}{0.000000in}}{%
\pgfpathmoveto{\pgfqpoint{0.000000in}{0.000000in}}%
\pgfpathlineto{\pgfqpoint{-0.027778in}{0.000000in}}%
\pgfusepath{stroke,fill}%
}%
\begin{pgfscope}%
\pgfsys@transformshift{0.913771in}{2.091439in}%
\pgfsys@useobject{currentmarker}{}%
\end{pgfscope}%
\end{pgfscope}%
\begin{pgfscope}%
\pgfsetbuttcap%
\pgfsetroundjoin%
\definecolor{currentfill}{rgb}{0.000000,0.000000,0.000000}%
\pgfsetfillcolor{currentfill}%
\pgfsetlinewidth{0.602250pt}%
\definecolor{currentstroke}{rgb}{0.000000,0.000000,0.000000}%
\pgfsetstrokecolor{currentstroke}%
\pgfsetdash{}{0pt}%
\pgfsys@defobject{currentmarker}{\pgfqpoint{-0.027778in}{0.000000in}}{\pgfqpoint{0.000000in}{0.000000in}}{%
\pgfpathmoveto{\pgfqpoint{0.000000in}{0.000000in}}%
\pgfpathlineto{\pgfqpoint{-0.027778in}{0.000000in}}%
\pgfusepath{stroke,fill}%
}%
\begin{pgfscope}%
\pgfsys@transformshift{0.913771in}{2.182652in}%
\pgfsys@useobject{currentmarker}{}%
\end{pgfscope}%
\end{pgfscope}%
\begin{pgfscope}%
\pgfsetbuttcap%
\pgfsetroundjoin%
\definecolor{currentfill}{rgb}{0.000000,0.000000,0.000000}%
\pgfsetfillcolor{currentfill}%
\pgfsetlinewidth{0.602250pt}%
\definecolor{currentstroke}{rgb}{0.000000,0.000000,0.000000}%
\pgfsetstrokecolor{currentstroke}%
\pgfsetdash{}{0pt}%
\pgfsys@defobject{currentmarker}{\pgfqpoint{-0.027778in}{0.000000in}}{\pgfqpoint{0.000000in}{0.000000in}}{%
\pgfpathmoveto{\pgfqpoint{0.000000in}{0.000000in}}%
\pgfpathlineto{\pgfqpoint{-0.027778in}{0.000000in}}%
\pgfusepath{stroke,fill}%
}%
\begin{pgfscope}%
\pgfsys@transformshift{0.913771in}{2.259771in}%
\pgfsys@useobject{currentmarker}{}%
\end{pgfscope}%
\end{pgfscope}%
\begin{pgfscope}%
\pgfsetbuttcap%
\pgfsetroundjoin%
\definecolor{currentfill}{rgb}{0.000000,0.000000,0.000000}%
\pgfsetfillcolor{currentfill}%
\pgfsetlinewidth{0.602250pt}%
\definecolor{currentstroke}{rgb}{0.000000,0.000000,0.000000}%
\pgfsetstrokecolor{currentstroke}%
\pgfsetdash{}{0pt}%
\pgfsys@defobject{currentmarker}{\pgfqpoint{-0.027778in}{0.000000in}}{\pgfqpoint{0.000000in}{0.000000in}}{%
\pgfpathmoveto{\pgfqpoint{0.000000in}{0.000000in}}%
\pgfpathlineto{\pgfqpoint{-0.027778in}{0.000000in}}%
\pgfusepath{stroke,fill}%
}%
\begin{pgfscope}%
\pgfsys@transformshift{0.913771in}{2.326575in}%
\pgfsys@useobject{currentmarker}{}%
\end{pgfscope}%
\end{pgfscope}%
\begin{pgfscope}%
\pgfsetbuttcap%
\pgfsetroundjoin%
\definecolor{currentfill}{rgb}{0.000000,0.000000,0.000000}%
\pgfsetfillcolor{currentfill}%
\pgfsetlinewidth{0.602250pt}%
\definecolor{currentstroke}{rgb}{0.000000,0.000000,0.000000}%
\pgfsetstrokecolor{currentstroke}%
\pgfsetdash{}{0pt}%
\pgfsys@defobject{currentmarker}{\pgfqpoint{-0.027778in}{0.000000in}}{\pgfqpoint{0.000000in}{0.000000in}}{%
\pgfpathmoveto{\pgfqpoint{0.000000in}{0.000000in}}%
\pgfpathlineto{\pgfqpoint{-0.027778in}{0.000000in}}%
\pgfusepath{stroke,fill}%
}%
\begin{pgfscope}%
\pgfsys@transformshift{0.913771in}{2.385501in}%
\pgfsys@useobject{currentmarker}{}%
\end{pgfscope}%
\end{pgfscope}%
\begin{pgfscope}%
\pgfsetbuttcap%
\pgfsetroundjoin%
\definecolor{currentfill}{rgb}{0.000000,0.000000,0.000000}%
\pgfsetfillcolor{currentfill}%
\pgfsetlinewidth{0.602250pt}%
\definecolor{currentstroke}{rgb}{0.000000,0.000000,0.000000}%
\pgfsetstrokecolor{currentstroke}%
\pgfsetdash{}{0pt}%
\pgfsys@defobject{currentmarker}{\pgfqpoint{-0.027778in}{0.000000in}}{\pgfqpoint{0.000000in}{0.000000in}}{%
\pgfpathmoveto{\pgfqpoint{0.000000in}{0.000000in}}%
\pgfpathlineto{\pgfqpoint{-0.027778in}{0.000000in}}%
\pgfusepath{stroke,fill}%
}%
\begin{pgfscope}%
\pgfsys@transformshift{0.913771in}{2.784983in}%
\pgfsys@useobject{currentmarker}{}%
\end{pgfscope}%
\end{pgfscope}%
\begin{pgfscope}%
\pgfsetbuttcap%
\pgfsetroundjoin%
\definecolor{currentfill}{rgb}{0.000000,0.000000,0.000000}%
\pgfsetfillcolor{currentfill}%
\pgfsetlinewidth{0.602250pt}%
\definecolor{currentstroke}{rgb}{0.000000,0.000000,0.000000}%
\pgfsetstrokecolor{currentstroke}%
\pgfsetdash{}{0pt}%
\pgfsys@defobject{currentmarker}{\pgfqpoint{-0.027778in}{0.000000in}}{\pgfqpoint{0.000000in}{0.000000in}}{%
\pgfpathmoveto{\pgfqpoint{0.000000in}{0.000000in}}%
\pgfpathlineto{\pgfqpoint{-0.027778in}{0.000000in}}%
\pgfusepath{stroke,fill}%
}%
\begin{pgfscope}%
\pgfsys@transformshift{0.913771in}{2.987832in}%
\pgfsys@useobject{currentmarker}{}%
\end{pgfscope}%
\end{pgfscope}%
\begin{pgfscope}%
\pgfsetbuttcap%
\pgfsetroundjoin%
\definecolor{currentfill}{rgb}{0.000000,0.000000,0.000000}%
\pgfsetfillcolor{currentfill}%
\pgfsetlinewidth{0.602250pt}%
\definecolor{currentstroke}{rgb}{0.000000,0.000000,0.000000}%
\pgfsetstrokecolor{currentstroke}%
\pgfsetdash{}{0pt}%
\pgfsys@defobject{currentmarker}{\pgfqpoint{-0.027778in}{0.000000in}}{\pgfqpoint{0.000000in}{0.000000in}}{%
\pgfpathmoveto{\pgfqpoint{0.000000in}{0.000000in}}%
\pgfpathlineto{\pgfqpoint{-0.027778in}{0.000000in}}%
\pgfusepath{stroke,fill}%
}%
\begin{pgfscope}%
\pgfsys@transformshift{0.913771in}{3.131755in}%
\pgfsys@useobject{currentmarker}{}%
\end{pgfscope}%
\end{pgfscope}%
\begin{pgfscope}%
\pgfsetbuttcap%
\pgfsetroundjoin%
\definecolor{currentfill}{rgb}{0.000000,0.000000,0.000000}%
\pgfsetfillcolor{currentfill}%
\pgfsetlinewidth{0.602250pt}%
\definecolor{currentstroke}{rgb}{0.000000,0.000000,0.000000}%
\pgfsetstrokecolor{currentstroke}%
\pgfsetdash{}{0pt}%
\pgfsys@defobject{currentmarker}{\pgfqpoint{-0.027778in}{0.000000in}}{\pgfqpoint{0.000000in}{0.000000in}}{%
\pgfpathmoveto{\pgfqpoint{0.000000in}{0.000000in}}%
\pgfpathlineto{\pgfqpoint{-0.027778in}{0.000000in}}%
\pgfusepath{stroke,fill}%
}%
\begin{pgfscope}%
\pgfsys@transformshift{0.913771in}{3.243391in}%
\pgfsys@useobject{currentmarker}{}%
\end{pgfscope}%
\end{pgfscope}%
\begin{pgfscope}%
\pgfsetbuttcap%
\pgfsetroundjoin%
\definecolor{currentfill}{rgb}{0.000000,0.000000,0.000000}%
\pgfsetfillcolor{currentfill}%
\pgfsetlinewidth{0.602250pt}%
\definecolor{currentstroke}{rgb}{0.000000,0.000000,0.000000}%
\pgfsetstrokecolor{currentstroke}%
\pgfsetdash{}{0pt}%
\pgfsys@defobject{currentmarker}{\pgfqpoint{-0.027778in}{0.000000in}}{\pgfqpoint{0.000000in}{0.000000in}}{%
\pgfpathmoveto{\pgfqpoint{0.000000in}{0.000000in}}%
\pgfpathlineto{\pgfqpoint{-0.027778in}{0.000000in}}%
\pgfusepath{stroke,fill}%
}%
\begin{pgfscope}%
\pgfsys@transformshift{0.913771in}{3.334603in}%
\pgfsys@useobject{currentmarker}{}%
\end{pgfscope}%
\end{pgfscope}%
\begin{pgfscope}%
\pgfsetbuttcap%
\pgfsetroundjoin%
\definecolor{currentfill}{rgb}{0.000000,0.000000,0.000000}%
\pgfsetfillcolor{currentfill}%
\pgfsetlinewidth{0.602250pt}%
\definecolor{currentstroke}{rgb}{0.000000,0.000000,0.000000}%
\pgfsetstrokecolor{currentstroke}%
\pgfsetdash{}{0pt}%
\pgfsys@defobject{currentmarker}{\pgfqpoint{-0.027778in}{0.000000in}}{\pgfqpoint{0.000000in}{0.000000in}}{%
\pgfpathmoveto{\pgfqpoint{0.000000in}{0.000000in}}%
\pgfpathlineto{\pgfqpoint{-0.027778in}{0.000000in}}%
\pgfusepath{stroke,fill}%
}%
\begin{pgfscope}%
\pgfsys@transformshift{0.913771in}{3.411723in}%
\pgfsys@useobject{currentmarker}{}%
\end{pgfscope}%
\end{pgfscope}%
\begin{pgfscope}%
\pgfsetbuttcap%
\pgfsetroundjoin%
\definecolor{currentfill}{rgb}{0.000000,0.000000,0.000000}%
\pgfsetfillcolor{currentfill}%
\pgfsetlinewidth{0.602250pt}%
\definecolor{currentstroke}{rgb}{0.000000,0.000000,0.000000}%
\pgfsetstrokecolor{currentstroke}%
\pgfsetdash{}{0pt}%
\pgfsys@defobject{currentmarker}{\pgfqpoint{-0.027778in}{0.000000in}}{\pgfqpoint{0.000000in}{0.000000in}}{%
\pgfpathmoveto{\pgfqpoint{0.000000in}{0.000000in}}%
\pgfpathlineto{\pgfqpoint{-0.027778in}{0.000000in}}%
\pgfusepath{stroke,fill}%
}%
\begin{pgfscope}%
\pgfsys@transformshift{0.913771in}{3.478527in}%
\pgfsys@useobject{currentmarker}{}%
\end{pgfscope}%
\end{pgfscope}%
\begin{pgfscope}%
\pgfsetbuttcap%
\pgfsetroundjoin%
\definecolor{currentfill}{rgb}{0.000000,0.000000,0.000000}%
\pgfsetfillcolor{currentfill}%
\pgfsetlinewidth{0.602250pt}%
\definecolor{currentstroke}{rgb}{0.000000,0.000000,0.000000}%
\pgfsetstrokecolor{currentstroke}%
\pgfsetdash{}{0pt}%
\pgfsys@defobject{currentmarker}{\pgfqpoint{-0.027778in}{0.000000in}}{\pgfqpoint{0.000000in}{0.000000in}}{%
\pgfpathmoveto{\pgfqpoint{0.000000in}{0.000000in}}%
\pgfpathlineto{\pgfqpoint{-0.027778in}{0.000000in}}%
\pgfusepath{stroke,fill}%
}%
\begin{pgfscope}%
\pgfsys@transformshift{0.913771in}{3.537452in}%
\pgfsys@useobject{currentmarker}{}%
\end{pgfscope}%
\end{pgfscope}%
\begin{pgfscope}%
\pgftext[x=0.339583in,y=2.217421in,,bottom,rotate=90.000000]{\rmfamily\fontsize{16.000000}{19.200000}\selectfont \(\displaystyle E/E_0\)}%
\end{pgfscope}%
\begin{pgfscope}%
\pgfpathrectangle{\pgfqpoint{0.913771in}{0.707421in}}{\pgfqpoint{4.650000in}{3.020000in}}%
\pgfusepath{clip}%
\pgfsetrectcap%
\pgfsetroundjoin%
\pgfsetlinewidth{1.505625pt}%
\definecolor{currentstroke}{rgb}{0.000000,0.000000,0.000000}%
\pgfsetstrokecolor{currentstroke}%
\pgfsetdash{}{0pt}%
\pgfpathmoveto{\pgfqpoint{1.125134in}{3.590149in}}%
\pgfpathlineto{\pgfqpoint{1.381890in}{3.583984in}}%
\pgfpathlineto{\pgfqpoint{1.599100in}{3.576604in}}%
\pgfpathlineto{\pgfqpoint{1.787323in}{3.567962in}}%
\pgfpathlineto{\pgfqpoint{1.944629in}{3.558598in}}%
\pgfpathlineto{\pgfqpoint{2.087717in}{3.547890in}}%
\pgfpathlineto{\pgfqpoint{2.212674in}{3.536392in}}%
\pgfpathlineto{\pgfqpoint{2.329091in}{3.523460in}}%
\pgfpathlineto{\pgfqpoint{2.433413in}{3.509673in}}%
\pgfpathlineto{\pgfqpoint{2.531988in}{3.494368in}}%
\pgfpathlineto{\pgfqpoint{2.621846in}{3.478150in}}%
\pgfpathlineto{\pgfqpoint{2.704550in}{3.461009in}}%
\pgfpathlineto{\pgfqpoint{2.783954in}{3.442263in}}%
\pgfpathlineto{\pgfqpoint{2.857738in}{3.422568in}}%
\pgfpathlineto{\pgfqpoint{2.928913in}{3.401237in}}%
\pgfpathlineto{\pgfqpoint{2.995551in}{3.378962in}}%
\pgfpathlineto{\pgfqpoint{3.060071in}{3.355060in}}%
\pgfpathlineto{\pgfqpoint{3.122510in}{3.329545in}}%
\pgfpathlineto{\pgfqpoint{3.182923in}{3.302448in}}%
\pgfpathlineto{\pgfqpoint{3.241380in}{3.273813in}}%
\pgfpathlineto{\pgfqpoint{3.299249in}{3.242983in}}%
\pgfpathlineto{\pgfqpoint{3.356313in}{3.210036in}}%
\pgfpathlineto{\pgfqpoint{3.412415in}{3.175077in}}%
\pgfpathlineto{\pgfqpoint{3.468460in}{3.137525in}}%
\pgfpathlineto{\pgfqpoint{3.524147in}{3.097551in}}%
\pgfpathlineto{\pgfqpoint{3.580978in}{3.053981in}}%
\pgfpathlineto{\pgfqpoint{3.638381in}{3.007118in}}%
\pgfpathlineto{\pgfqpoint{3.696632in}{2.956650in}}%
\pgfpathlineto{\pgfqpoint{3.755870in}{2.902375in}}%
\pgfpathlineto{\pgfqpoint{3.817370in}{2.842973in}}%
\pgfpathlineto{\pgfqpoint{3.880770in}{2.778622in}}%
\pgfpathlineto{\pgfqpoint{3.947239in}{2.707957in}}%
\pgfpathlineto{\pgfqpoint{4.017383in}{2.630096in}}%
\pgfpathlineto{\pgfqpoint{4.092184in}{2.543682in}}%
\pgfpathlineto{\pgfqpoint{4.172665in}{2.447238in}}%
\pgfpathlineto{\pgfqpoint{4.260288in}{2.338691in}}%
\pgfpathlineto{\pgfqpoint{4.357223in}{2.214987in}}%
\pgfpathlineto{\pgfqpoint{4.466558in}{2.071758in}}%
\pgfpathlineto{\pgfqpoint{4.592883in}{1.902496in}}%
\pgfpathlineto{\pgfqpoint{4.743679in}{1.696593in}}%
\pgfpathlineto{\pgfqpoint{4.932626in}{1.434654in}}%
\pgfpathlineto{\pgfqpoint{5.188055in}{1.076499in}}%
\pgfpathlineto{\pgfqpoint{5.352407in}{0.844694in}}%
\pgfpathlineto{\pgfqpoint{5.352407in}{0.844694in}}%
\pgfusepath{stroke}%
\end{pgfscope}%
\begin{pgfscope}%
\pgfsetrectcap%
\pgfsetmiterjoin%
\pgfsetlinewidth{0.803000pt}%
\definecolor{currentstroke}{rgb}{0.501961,0.501961,0.501961}%
\pgfsetstrokecolor{currentstroke}%
\pgfsetdash{}{0pt}%
\pgfpathmoveto{\pgfqpoint{0.913771in}{0.707421in}}%
\pgfpathlineto{\pgfqpoint{0.913771in}{3.727421in}}%
\pgfusepath{stroke}%
\end{pgfscope}%
\begin{pgfscope}%
\pgfsetrectcap%
\pgfsetmiterjoin%
\pgfsetlinewidth{0.803000pt}%
\definecolor{currentstroke}{rgb}{0.501961,0.501961,0.501961}%
\pgfsetstrokecolor{currentstroke}%
\pgfsetdash{}{0pt}%
\pgfpathmoveto{\pgfqpoint{5.563771in}{0.707421in}}%
\pgfpathlineto{\pgfqpoint{5.563771in}{3.727421in}}%
\pgfusepath{stroke}%
\end{pgfscope}%
\begin{pgfscope}%
\pgfsetrectcap%
\pgfsetmiterjoin%
\pgfsetlinewidth{0.803000pt}%
\definecolor{currentstroke}{rgb}{0.501961,0.501961,0.501961}%
\pgfsetstrokecolor{currentstroke}%
\pgfsetdash{}{0pt}%
\pgfpathmoveto{\pgfqpoint{0.913771in}{0.707421in}}%
\pgfpathlineto{\pgfqpoint{5.563771in}{0.707421in}}%
\pgfusepath{stroke}%
\end{pgfscope}%
\begin{pgfscope}%
\pgfsetrectcap%
\pgfsetmiterjoin%
\pgfsetlinewidth{0.803000pt}%
\definecolor{currentstroke}{rgb}{0.501961,0.501961,0.501961}%
\pgfsetstrokecolor{currentstroke}%
\pgfsetdash{}{0pt}%
\pgfpathmoveto{\pgfqpoint{0.913771in}{3.727421in}}%
\pgfpathlineto{\pgfqpoint{5.563771in}{3.727421in}}%
\pgfusepath{stroke}%
\end{pgfscope}%
\end{pgfpicture}%
\makeatother%
\endgroup%

    \caption{A log-log plot of the magnitude of the non-dimensional electric field, $\mathbf{E}_y$.\label{fig:E0}}
\end{figure}



\subsection{Scaling}
Introducing scaled variables
\[ \bar{t} = \frac{t}{t_c}, \hspace{10 mm} \bar{y} = \frac{y}{y_c}, \]
where $y_c$ and $t_c$ are characteristic length and time scales respectively, and using the cooridnate transformation $y(0) - R = 0$, the governing equation becomes
\begin{eqnarray}
& \bar{y}'' = - \frac{1}{2} \frac{C_D \rho A y_c}{m} \bar{y}'^2
- \frac{q E_0 t_c^2}{m y_c} \bar{E} 
- \frac{k q^2 t_c^2}{16 \pi \epsilon_0 R^2 m y_c} \frac{1}{\left( \frac{y_c}{R} \bar{y} + 1 \right)^2}, & \nonumber \\
& \bar{y}(0) = 0, \hspace{1 mm} \bar{y}'(0) = \frac{U_0 t_c}{y_c} .&
\end{eqnarray}
We note several dimensionless groups
\[ \mathbf{\Pi}_1 = \frac{C_D \rho A y_c}{2 m}, \hspace{5 mm}
\mathbf{\Pi}_2 = \frac{q E_0 t_c^2}{m y_c}, \hspace{5 mm}
\mathbf{\Pi}_3 = \frac{k q^2 t_c^2}{16 \pi \epsilon_0 R^2 m y_c}, \hspace{5 mm}
\mathbf{\Pi}_4 = \frac{y_c}{R}.\]

\subsubsection{Inertial Electro-Image Limit}
In the limit of small $y, \hspace{1mm} t$ we expect intertia to scale as Coulombic and image force. With $y_c \sim U_0 t_c$ and picking $t_c$ such that Coulombic force is $\mathcal{O}(1)$
\[ t_c \sim \frac{m U_0}{q E_0}, \hspace{5 mm}
y_c \sim \frac{m U_0^2}{q E_0} .
\]
With these scales the governing equation then becomes
\begin{eqnarray}
& \bar{y}'' = -1 - \mathbb{I}\mbox{m} \frac{1}{\left( \mathbb{E}\mbox{u}\bar{y} + 1 \right)^2} ,& \nonumber \\
& \bar{y}(0) = 0, \hspace{1 mm} \bar{y}'(0) = 1 .&
\end{eqnarray} 
with 
\[ \mathbb{I}\mbox{m} \equiv \frac{k q}{16 \pi \epsilon_0 R_d^2 E_0} = \mathbf{\Pi}_3, \hspace{5 mm}
\mathbb{E}\mbox{u} \equiv \frac{m U_0^2}{q E_0} = \mathbf{\Pi}_4 ,
\]
where $\mathbb{I}\mbox{m}$ is the Image number, and denotes the ratio of image forces to the Coulombic force of the unperturbed field, and where where $\mathbb{E}\mbox{u}$ is the electrostatic Euler number, and is a ratio of inertia to electrostatic force.

\subsubsection{Inertial Electro-Viscous Limit}
In the limit of large $y, \hspace{1mm} t$ we expect droplet inertia to scale as Coulombic force and drag. In this case there are several obvious choices of scales:
\begin{enumerate}
\item $y_c \sim U_0 t_c$ and make Coulomb force $\mathcal{O}(1)$.
\item $y_c \sim R_d$ or $L$, $t_c \sim \frac{L}{U_0}$ but this makes the governing equation singular.
\item $y_c \sim R_d$ or $L$, $t_c \sim \left( \frac{L m}{q E_0} \right)^{1/2}$.
\item $y_c \sim R_d$ or $L$, and making Coulomb force $\mathcal{O}(1)$.
\end{enumerate}

In Case 3, the characterisitc time is $t_c \sim \left( \frac{4 \pi R_d^2}{q E_0 L} \right)^2 $ and the non-dimensional governing equation is given by
\begin{eqnarray}
&\bar{y}'' = - \frac{C_D \rho A L}{2 m} \bar{y}'^2 - \frac{1}{\left( \frac{L}{R} \bar{y} + 1\right)^2} ,& \nonumber \\
& \bar{y}(0) = \frac{R}{L}, \hspace{5 mm} \bar{y}'(0) = \left( \frac{4 \pi U_0^2 R_d^2}{q E_0 L^3}\right)^{1/2} = \frac{R}{L} \sqrt{\mathbb{E}\mbox{u}_+}.& \nonumber
\end{eqnarray}
where $\mathbb{E}\mbox{u}_+ = \frac{4 \pi m U_0^2}{q E_0 L}$ is a long time scaled electrostatic Euler number. We prefer the approach with the greatest physical simplicity, fewest Pi terms, and has homogoenous initial conditions.

In Case 1, the characteristic dimensions are
\[ t_c \sim \frac{R^2}{L^2} \frac{4 \pi m U_0}{q E_0}, \hspace{5mm} y_c \sim \frac{R^2}{L^2} \frac{4 \pi m U_0^2}{q E_0}.
\]
With this scaling the non-dimensional governing equation is 
\begin{eqnarray}
&\bar{y}'' = - \mathbb{D}\mbox{g} \mathbb{E}\mbox{u}_+ \bar{y}'^2 - \frac{1}{\left( \mathbb{E}\mbox{u}_+ \bar{y} + 1 \right)^2}, & \nonumber \\
& \bar{y}(0) = 0, \hspace{1 mm} \bar{y}'(0) = 1 & 
\end{eqnarray}
where $\mathbb{D}\mbox{g} = \frac{C_D \rho_a}{\rho_l}$. This is the preferred scaling.

\subsection{Asymptotic Estimates}


\subsubsection{Inertial Electro-Image Limit}
With $\epsilon = \mathbb{E}\mbox{u}$, where $\epsilon$ is a small parameter, and $\alpha = \mathbb{I}\mbox{m}$,
\begin{eqnarray*}
&\bar{y}(\bar{t}) = \bar{t} + \frac{\bar{t}^{2}}{2} \left(-1 - \alpha\right) + \epsilon \left(\frac{\alpha \bar{t}^{3}}{3} + \frac{\alpha \bar{t}^{4}}{12} \left(-1 - \alpha\right)\right)& \\
&+ \epsilon^{2} \left(- \frac{\alpha \bar{t}^{4}}{4} + \frac{\alpha \bar{t}^{5}}{60} \left(9 + 11 \alpha\right) + \frac{\alpha \bar{t}^{6}}{360} \left(-9 - 20 \alpha - 11 \alpha^{2}\right)\right) + \mathcal{O}(\epsilon^3)&
\end{eqnarray*}

\newpage
\begin{figure}[htb]
    \centering
    \resizebox{1\textwidth}{!}{%% Creator: Matplotlib, PGF backend
%%
%% To include the figure in your LaTeX document, write
%%   \input{<filename>.pgf}
%%
%% Make sure the required packages are loaded in your preamble
%%   \usepackage{pgf}
%%
%% Figures using additional raster images can only be included by \input if
%% they are in the same directory as the main LaTeX file. For loading figures
%% from other directories you can use the `import` package
%%   \usepackage{import}
%% and then include the figures with
%%   \import{<path to file>}{<filename>.pgf}
%%
%% Matplotlib used the following preamble
%%   \usepackage{fontspec}
%%   \setmainfont{DejaVu Serif}
%%   \setsansfont{DejaVu Sans}
%%   \setmonofont{DejaVu Sans Mono}
%%
\begingroup%
\makeatletter%
\begin{pgfpicture}%
\pgfpathrectangle{\pgfpointorigin}{\pgfqpoint{10.328611in}{8.489057in}}%
\pgfusepath{use as bounding box, clip}%
\begin{pgfscope}%
\pgfsetbuttcap%
\pgfsetmiterjoin%
\definecolor{currentfill}{rgb}{1.000000,1.000000,1.000000}%
\pgfsetfillcolor{currentfill}%
\pgfsetlinewidth{0.000000pt}%
\definecolor{currentstroke}{rgb}{1.000000,1.000000,1.000000}%
\pgfsetstrokecolor{currentstroke}%
\pgfsetdash{}{0pt}%
\pgfpathmoveto{\pgfqpoint{0.000000in}{0.000000in}}%
\pgfpathlineto{\pgfqpoint{10.328611in}{0.000000in}}%
\pgfpathlineto{\pgfqpoint{10.328611in}{8.489057in}}%
\pgfpathlineto{\pgfqpoint{0.000000in}{8.489057in}}%
\pgfpathclose%
\pgfusepath{fill}%
\end{pgfscope}%
\begin{pgfscope}%
\pgfsetbuttcap%
\pgfsetmiterjoin%
\definecolor{currentfill}{rgb}{1.000000,1.000000,1.000000}%
\pgfsetfillcolor{currentfill}%
\pgfsetlinewidth{0.000000pt}%
\definecolor{currentstroke}{rgb}{0.000000,0.000000,0.000000}%
\pgfsetstrokecolor{currentstroke}%
\pgfsetstrokeopacity{0.000000}%
\pgfsetdash{}{0pt}%
\pgfpathmoveto{\pgfqpoint{0.880000in}{4.908628in}}%
\pgfpathlineto{\pgfqpoint{5.107273in}{4.908628in}}%
\pgfpathlineto{\pgfqpoint{5.107273in}{8.340446in}}%
\pgfpathlineto{\pgfqpoint{0.880000in}{8.340446in}}%
\pgfpathclose%
\pgfusepath{fill}%
\end{pgfscope}%
\begin{pgfscope}%
\pgfsetbuttcap%
\pgfsetroundjoin%
\definecolor{currentfill}{rgb}{0.000000,0.000000,0.000000}%
\pgfsetfillcolor{currentfill}%
\pgfsetlinewidth{0.803000pt}%
\definecolor{currentstroke}{rgb}{0.000000,0.000000,0.000000}%
\pgfsetstrokecolor{currentstroke}%
\pgfsetdash{}{0pt}%
\pgfsys@defobject{currentmarker}{\pgfqpoint{0.000000in}{-0.048611in}}{\pgfqpoint{0.000000in}{0.000000in}}{%
\pgfpathmoveto{\pgfqpoint{0.000000in}{0.000000in}}%
\pgfpathlineto{\pgfqpoint{0.000000in}{-0.048611in}}%
\pgfusepath{stroke,fill}%
}%
\begin{pgfscope}%
\pgfsys@transformshift{1.072149in}{4.908628in}%
\pgfsys@useobject{currentmarker}{}%
\end{pgfscope}%
\end{pgfscope}%
\begin{pgfscope}%
\pgftext[x=1.072149in,y=4.811405in,,top]{\rmfamily\fontsize{10.000000}{12.000000}\selectfont \(\displaystyle 0.0\)}%
\end{pgfscope}%
\begin{pgfscope}%
\pgfsetbuttcap%
\pgfsetroundjoin%
\definecolor{currentfill}{rgb}{0.000000,0.000000,0.000000}%
\pgfsetfillcolor{currentfill}%
\pgfsetlinewidth{0.803000pt}%
\definecolor{currentstroke}{rgb}{0.000000,0.000000,0.000000}%
\pgfsetstrokecolor{currentstroke}%
\pgfsetdash{}{0pt}%
\pgfsys@defobject{currentmarker}{\pgfqpoint{0.000000in}{-0.048611in}}{\pgfqpoint{0.000000in}{0.000000in}}{%
\pgfpathmoveto{\pgfqpoint{0.000000in}{0.000000in}}%
\pgfpathlineto{\pgfqpoint{0.000000in}{-0.048611in}}%
\pgfusepath{stroke,fill}%
}%
\begin{pgfscope}%
\pgfsys@transformshift{1.841513in}{4.908628in}%
\pgfsys@useobject{currentmarker}{}%
\end{pgfscope}%
\end{pgfscope}%
\begin{pgfscope}%
\pgftext[x=1.841513in,y=4.811405in,,top]{\rmfamily\fontsize{10.000000}{12.000000}\selectfont \(\displaystyle 0.2\)}%
\end{pgfscope}%
\begin{pgfscope}%
\pgfsetbuttcap%
\pgfsetroundjoin%
\definecolor{currentfill}{rgb}{0.000000,0.000000,0.000000}%
\pgfsetfillcolor{currentfill}%
\pgfsetlinewidth{0.803000pt}%
\definecolor{currentstroke}{rgb}{0.000000,0.000000,0.000000}%
\pgfsetstrokecolor{currentstroke}%
\pgfsetdash{}{0pt}%
\pgfsys@defobject{currentmarker}{\pgfqpoint{0.000000in}{-0.048611in}}{\pgfqpoint{0.000000in}{0.000000in}}{%
\pgfpathmoveto{\pgfqpoint{0.000000in}{0.000000in}}%
\pgfpathlineto{\pgfqpoint{0.000000in}{-0.048611in}}%
\pgfusepath{stroke,fill}%
}%
\begin{pgfscope}%
\pgfsys@transformshift{2.610878in}{4.908628in}%
\pgfsys@useobject{currentmarker}{}%
\end{pgfscope}%
\end{pgfscope}%
\begin{pgfscope}%
\pgftext[x=2.610878in,y=4.811405in,,top]{\rmfamily\fontsize{10.000000}{12.000000}\selectfont \(\displaystyle 0.4\)}%
\end{pgfscope}%
\begin{pgfscope}%
\pgfsetbuttcap%
\pgfsetroundjoin%
\definecolor{currentfill}{rgb}{0.000000,0.000000,0.000000}%
\pgfsetfillcolor{currentfill}%
\pgfsetlinewidth{0.803000pt}%
\definecolor{currentstroke}{rgb}{0.000000,0.000000,0.000000}%
\pgfsetstrokecolor{currentstroke}%
\pgfsetdash{}{0pt}%
\pgfsys@defobject{currentmarker}{\pgfqpoint{0.000000in}{-0.048611in}}{\pgfqpoint{0.000000in}{0.000000in}}{%
\pgfpathmoveto{\pgfqpoint{0.000000in}{0.000000in}}%
\pgfpathlineto{\pgfqpoint{0.000000in}{-0.048611in}}%
\pgfusepath{stroke,fill}%
}%
\begin{pgfscope}%
\pgfsys@transformshift{3.380242in}{4.908628in}%
\pgfsys@useobject{currentmarker}{}%
\end{pgfscope}%
\end{pgfscope}%
\begin{pgfscope}%
\pgftext[x=3.380242in,y=4.811405in,,top]{\rmfamily\fontsize{10.000000}{12.000000}\selectfont \(\displaystyle 0.6\)}%
\end{pgfscope}%
\begin{pgfscope}%
\pgfsetbuttcap%
\pgfsetroundjoin%
\definecolor{currentfill}{rgb}{0.000000,0.000000,0.000000}%
\pgfsetfillcolor{currentfill}%
\pgfsetlinewidth{0.803000pt}%
\definecolor{currentstroke}{rgb}{0.000000,0.000000,0.000000}%
\pgfsetstrokecolor{currentstroke}%
\pgfsetdash{}{0pt}%
\pgfsys@defobject{currentmarker}{\pgfqpoint{0.000000in}{-0.048611in}}{\pgfqpoint{0.000000in}{0.000000in}}{%
\pgfpathmoveto{\pgfqpoint{0.000000in}{0.000000in}}%
\pgfpathlineto{\pgfqpoint{0.000000in}{-0.048611in}}%
\pgfusepath{stroke,fill}%
}%
\begin{pgfscope}%
\pgfsys@transformshift{4.149606in}{4.908628in}%
\pgfsys@useobject{currentmarker}{}%
\end{pgfscope}%
\end{pgfscope}%
\begin{pgfscope}%
\pgftext[x=4.149606in,y=4.811405in,,top]{\rmfamily\fontsize{10.000000}{12.000000}\selectfont \(\displaystyle 0.8\)}%
\end{pgfscope}%
\begin{pgfscope}%
\pgfsetbuttcap%
\pgfsetroundjoin%
\definecolor{currentfill}{rgb}{0.000000,0.000000,0.000000}%
\pgfsetfillcolor{currentfill}%
\pgfsetlinewidth{0.803000pt}%
\definecolor{currentstroke}{rgb}{0.000000,0.000000,0.000000}%
\pgfsetstrokecolor{currentstroke}%
\pgfsetdash{}{0pt}%
\pgfsys@defobject{currentmarker}{\pgfqpoint{0.000000in}{-0.048611in}}{\pgfqpoint{0.000000in}{0.000000in}}{%
\pgfpathmoveto{\pgfqpoint{0.000000in}{0.000000in}}%
\pgfpathlineto{\pgfqpoint{0.000000in}{-0.048611in}}%
\pgfusepath{stroke,fill}%
}%
\begin{pgfscope}%
\pgfsys@transformshift{4.918971in}{4.908628in}%
\pgfsys@useobject{currentmarker}{}%
\end{pgfscope}%
\end{pgfscope}%
\begin{pgfscope}%
\pgftext[x=4.918971in,y=4.811405in,,top]{\rmfamily\fontsize{10.000000}{12.000000}\selectfont \(\displaystyle 1.0\)}%
\end{pgfscope}%
\begin{pgfscope}%
\pgfsetbuttcap%
\pgfsetroundjoin%
\definecolor{currentfill}{rgb}{0.000000,0.000000,0.000000}%
\pgfsetfillcolor{currentfill}%
\pgfsetlinewidth{0.803000pt}%
\definecolor{currentstroke}{rgb}{0.000000,0.000000,0.000000}%
\pgfsetstrokecolor{currentstroke}%
\pgfsetdash{}{0pt}%
\pgfsys@defobject{currentmarker}{\pgfqpoint{-0.048611in}{0.000000in}}{\pgfqpoint{0.000000in}{0.000000in}}{%
\pgfpathmoveto{\pgfqpoint{0.000000in}{0.000000in}}%
\pgfpathlineto{\pgfqpoint{-0.048611in}{0.000000in}}%
\pgfusepath{stroke,fill}%
}%
\begin{pgfscope}%
\pgfsys@transformshift{0.880000in}{5.054743in}%
\pgfsys@useobject{currentmarker}{}%
\end{pgfscope}%
\end{pgfscope}%
\begin{pgfscope}%
\pgftext[x=0.497283in,y=5.001982in,left,base]{\rmfamily\fontsize{10.000000}{12.000000}\selectfont \(\displaystyle -0.5\)}%
\end{pgfscope}%
\begin{pgfscope}%
\pgfsetbuttcap%
\pgfsetroundjoin%
\definecolor{currentfill}{rgb}{0.000000,0.000000,0.000000}%
\pgfsetfillcolor{currentfill}%
\pgfsetlinewidth{0.803000pt}%
\definecolor{currentstroke}{rgb}{0.000000,0.000000,0.000000}%
\pgfsetstrokecolor{currentstroke}%
\pgfsetdash{}{0pt}%
\pgfsys@defobject{currentmarker}{\pgfqpoint{-0.048611in}{0.000000in}}{\pgfqpoint{0.000000in}{0.000000in}}{%
\pgfpathmoveto{\pgfqpoint{0.000000in}{0.000000in}}%
\pgfpathlineto{\pgfqpoint{-0.048611in}{0.000000in}}%
\pgfusepath{stroke,fill}%
}%
\begin{pgfscope}%
\pgfsys@transformshift{0.880000in}{5.510867in}%
\pgfsys@useobject{currentmarker}{}%
\end{pgfscope}%
\end{pgfscope}%
\begin{pgfscope}%
\pgftext[x=0.497283in,y=5.458106in,left,base]{\rmfamily\fontsize{10.000000}{12.000000}\selectfont \(\displaystyle -0.4\)}%
\end{pgfscope}%
\begin{pgfscope}%
\pgfsetbuttcap%
\pgfsetroundjoin%
\definecolor{currentfill}{rgb}{0.000000,0.000000,0.000000}%
\pgfsetfillcolor{currentfill}%
\pgfsetlinewidth{0.803000pt}%
\definecolor{currentstroke}{rgb}{0.000000,0.000000,0.000000}%
\pgfsetstrokecolor{currentstroke}%
\pgfsetdash{}{0pt}%
\pgfsys@defobject{currentmarker}{\pgfqpoint{-0.048611in}{0.000000in}}{\pgfqpoint{0.000000in}{0.000000in}}{%
\pgfpathmoveto{\pgfqpoint{0.000000in}{0.000000in}}%
\pgfpathlineto{\pgfqpoint{-0.048611in}{0.000000in}}%
\pgfusepath{stroke,fill}%
}%
\begin{pgfscope}%
\pgfsys@transformshift{0.880000in}{5.966991in}%
\pgfsys@useobject{currentmarker}{}%
\end{pgfscope}%
\end{pgfscope}%
\begin{pgfscope}%
\pgftext[x=0.497283in,y=5.914230in,left,base]{\rmfamily\fontsize{10.000000}{12.000000}\selectfont \(\displaystyle -0.3\)}%
\end{pgfscope}%
\begin{pgfscope}%
\pgfsetbuttcap%
\pgfsetroundjoin%
\definecolor{currentfill}{rgb}{0.000000,0.000000,0.000000}%
\pgfsetfillcolor{currentfill}%
\pgfsetlinewidth{0.803000pt}%
\definecolor{currentstroke}{rgb}{0.000000,0.000000,0.000000}%
\pgfsetstrokecolor{currentstroke}%
\pgfsetdash{}{0pt}%
\pgfsys@defobject{currentmarker}{\pgfqpoint{-0.048611in}{0.000000in}}{\pgfqpoint{0.000000in}{0.000000in}}{%
\pgfpathmoveto{\pgfqpoint{0.000000in}{0.000000in}}%
\pgfpathlineto{\pgfqpoint{-0.048611in}{0.000000in}}%
\pgfusepath{stroke,fill}%
}%
\begin{pgfscope}%
\pgfsys@transformshift{0.880000in}{6.423115in}%
\pgfsys@useobject{currentmarker}{}%
\end{pgfscope}%
\end{pgfscope}%
\begin{pgfscope}%
\pgftext[x=0.497283in,y=6.370354in,left,base]{\rmfamily\fontsize{10.000000}{12.000000}\selectfont \(\displaystyle -0.2\)}%
\end{pgfscope}%
\begin{pgfscope}%
\pgfsetbuttcap%
\pgfsetroundjoin%
\definecolor{currentfill}{rgb}{0.000000,0.000000,0.000000}%
\pgfsetfillcolor{currentfill}%
\pgfsetlinewidth{0.803000pt}%
\definecolor{currentstroke}{rgb}{0.000000,0.000000,0.000000}%
\pgfsetstrokecolor{currentstroke}%
\pgfsetdash{}{0pt}%
\pgfsys@defobject{currentmarker}{\pgfqpoint{-0.048611in}{0.000000in}}{\pgfqpoint{0.000000in}{0.000000in}}{%
\pgfpathmoveto{\pgfqpoint{0.000000in}{0.000000in}}%
\pgfpathlineto{\pgfqpoint{-0.048611in}{0.000000in}}%
\pgfusepath{stroke,fill}%
}%
\begin{pgfscope}%
\pgfsys@transformshift{0.880000in}{6.879239in}%
\pgfsys@useobject{currentmarker}{}%
\end{pgfscope}%
\end{pgfscope}%
\begin{pgfscope}%
\pgftext[x=0.497283in,y=6.826478in,left,base]{\rmfamily\fontsize{10.000000}{12.000000}\selectfont \(\displaystyle -0.1\)}%
\end{pgfscope}%
\begin{pgfscope}%
\pgfsetbuttcap%
\pgfsetroundjoin%
\definecolor{currentfill}{rgb}{0.000000,0.000000,0.000000}%
\pgfsetfillcolor{currentfill}%
\pgfsetlinewidth{0.803000pt}%
\definecolor{currentstroke}{rgb}{0.000000,0.000000,0.000000}%
\pgfsetstrokecolor{currentstroke}%
\pgfsetdash{}{0pt}%
\pgfsys@defobject{currentmarker}{\pgfqpoint{-0.048611in}{0.000000in}}{\pgfqpoint{0.000000in}{0.000000in}}{%
\pgfpathmoveto{\pgfqpoint{0.000000in}{0.000000in}}%
\pgfpathlineto{\pgfqpoint{-0.048611in}{0.000000in}}%
\pgfusepath{stroke,fill}%
}%
\begin{pgfscope}%
\pgfsys@transformshift{0.880000in}{7.335363in}%
\pgfsys@useobject{currentmarker}{}%
\end{pgfscope}%
\end{pgfscope}%
\begin{pgfscope}%
\pgftext[x=0.605308in,y=7.282601in,left,base]{\rmfamily\fontsize{10.000000}{12.000000}\selectfont \(\displaystyle 0.0\)}%
\end{pgfscope}%
\begin{pgfscope}%
\pgfsetbuttcap%
\pgfsetroundjoin%
\definecolor{currentfill}{rgb}{0.000000,0.000000,0.000000}%
\pgfsetfillcolor{currentfill}%
\pgfsetlinewidth{0.803000pt}%
\definecolor{currentstroke}{rgb}{0.000000,0.000000,0.000000}%
\pgfsetstrokecolor{currentstroke}%
\pgfsetdash{}{0pt}%
\pgfsys@defobject{currentmarker}{\pgfqpoint{-0.048611in}{0.000000in}}{\pgfqpoint{0.000000in}{0.000000in}}{%
\pgfpathmoveto{\pgfqpoint{0.000000in}{0.000000in}}%
\pgfpathlineto{\pgfqpoint{-0.048611in}{0.000000in}}%
\pgfusepath{stroke,fill}%
}%
\begin{pgfscope}%
\pgfsys@transformshift{0.880000in}{7.791487in}%
\pgfsys@useobject{currentmarker}{}%
\end{pgfscope}%
\end{pgfscope}%
\begin{pgfscope}%
\pgftext[x=0.605308in,y=7.738725in,left,base]{\rmfamily\fontsize{10.000000}{12.000000}\selectfont \(\displaystyle 0.1\)}%
\end{pgfscope}%
\begin{pgfscope}%
\pgfsetbuttcap%
\pgfsetroundjoin%
\definecolor{currentfill}{rgb}{0.000000,0.000000,0.000000}%
\pgfsetfillcolor{currentfill}%
\pgfsetlinewidth{0.803000pt}%
\definecolor{currentstroke}{rgb}{0.000000,0.000000,0.000000}%
\pgfsetstrokecolor{currentstroke}%
\pgfsetdash{}{0pt}%
\pgfsys@defobject{currentmarker}{\pgfqpoint{-0.048611in}{0.000000in}}{\pgfqpoint{0.000000in}{0.000000in}}{%
\pgfpathmoveto{\pgfqpoint{0.000000in}{0.000000in}}%
\pgfpathlineto{\pgfqpoint{-0.048611in}{0.000000in}}%
\pgfusepath{stroke,fill}%
}%
\begin{pgfscope}%
\pgfsys@transformshift{0.880000in}{8.247611in}%
\pgfsys@useobject{currentmarker}{}%
\end{pgfscope}%
\end{pgfscope}%
\begin{pgfscope}%
\pgftext[x=0.605308in,y=8.194849in,left,base]{\rmfamily\fontsize{10.000000}{12.000000}\selectfont \(\displaystyle 0.2\)}%
\end{pgfscope}%
\begin{pgfscope}%
\pgfpathrectangle{\pgfqpoint{0.880000in}{4.908628in}}{\pgfqpoint{4.227273in}{3.431818in}} %
\pgfusepath{clip}%
\pgfsetbuttcap%
\pgfsetroundjoin%
\pgfsetlinewidth{1.505625pt}%
\definecolor{currentstroke}{rgb}{1.000000,0.000000,0.000000}%
\pgfsetstrokecolor{currentstroke}%
\pgfsetdash{{5.550000pt}{2.400000pt}}{0.000000pt}%
\pgfpathmoveto{\pgfqpoint{1.072149in}{7.335363in}}%
\pgfpathlineto{\pgfqpoint{1.145238in}{7.419577in}}%
\pgfpathlineto{\pgfqpoint{1.218328in}{7.498968in}}%
\pgfpathlineto{\pgfqpoint{1.291418in}{7.573643in}}%
\pgfpathlineto{\pgfqpoint{1.360660in}{7.640123in}}%
\pgfpathlineto{\pgfqpoint{1.429903in}{7.702527in}}%
\pgfpathlineto{\pgfqpoint{1.499146in}{7.760921in}}%
\pgfpathlineto{\pgfqpoint{1.568389in}{7.815364in}}%
\pgfpathlineto{\pgfqpoint{1.637632in}{7.865909in}}%
\pgfpathlineto{\pgfqpoint{1.706874in}{7.912604in}}%
\pgfpathlineto{\pgfqpoint{1.776117in}{7.955492in}}%
\pgfpathlineto{\pgfqpoint{1.845360in}{7.994611in}}%
\pgfpathlineto{\pgfqpoint{1.910756in}{8.028127in}}%
\pgfpathlineto{\pgfqpoint{1.976152in}{8.058338in}}%
\pgfpathlineto{\pgfqpoint{2.041548in}{8.085267in}}%
\pgfpathlineto{\pgfqpoint{2.106944in}{8.108934in}}%
\pgfpathlineto{\pgfqpoint{2.172340in}{8.129356in}}%
\pgfpathlineto{\pgfqpoint{2.237736in}{8.146548in}}%
\pgfpathlineto{\pgfqpoint{2.303132in}{8.160524in}}%
\pgfpathlineto{\pgfqpoint{2.368528in}{8.171293in}}%
\pgfpathlineto{\pgfqpoint{2.433924in}{8.178864in}}%
\pgfpathlineto{\pgfqpoint{2.499320in}{8.183241in}}%
\pgfpathlineto{\pgfqpoint{2.564716in}{8.184428in}}%
\pgfpathlineto{\pgfqpoint{2.630112in}{8.182426in}}%
\pgfpathlineto{\pgfqpoint{2.695508in}{8.177234in}}%
\pgfpathlineto{\pgfqpoint{2.760904in}{8.168846in}}%
\pgfpathlineto{\pgfqpoint{2.826300in}{8.157256in}}%
\pgfpathlineto{\pgfqpoint{2.891696in}{8.142455in}}%
\pgfpathlineto{\pgfqpoint{2.957092in}{8.124430in}}%
\pgfpathlineto{\pgfqpoint{3.022488in}{8.103166in}}%
\pgfpathlineto{\pgfqpoint{3.087884in}{8.078645in}}%
\pgfpathlineto{\pgfqpoint{3.153279in}{8.050848in}}%
\pgfpathlineto{\pgfqpoint{3.218675in}{8.019749in}}%
\pgfpathlineto{\pgfqpoint{3.284071in}{7.985322in}}%
\pgfpathlineto{\pgfqpoint{3.349467in}{7.947538in}}%
\pgfpathlineto{\pgfqpoint{3.418710in}{7.903837in}}%
\pgfpathlineto{\pgfqpoint{3.487953in}{7.856293in}}%
\pgfpathlineto{\pgfqpoint{3.557196in}{7.804862in}}%
\pgfpathlineto{\pgfqpoint{3.626439in}{7.749494in}}%
\pgfpathlineto{\pgfqpoint{3.695681in}{7.690135in}}%
\pgfpathlineto{\pgfqpoint{3.764924in}{7.626730in}}%
\pgfpathlineto{\pgfqpoint{3.834167in}{7.559218in}}%
\pgfpathlineto{\pgfqpoint{3.903410in}{7.487533in}}%
\pgfpathlineto{\pgfqpoint{3.976499in}{7.407260in}}%
\pgfpathlineto{\pgfqpoint{4.049589in}{7.322168in}}%
\pgfpathlineto{\pgfqpoint{4.122679in}{7.232158in}}%
\pgfpathlineto{\pgfqpoint{4.195768in}{7.137117in}}%
\pgfpathlineto{\pgfqpoint{4.268858in}{7.036919in}}%
\pgfpathlineto{\pgfqpoint{4.341947in}{6.931417in}}%
\pgfpathlineto{\pgfqpoint{4.415037in}{6.820438in}}%
\pgfpathlineto{\pgfqpoint{4.488127in}{6.703778in}}%
\pgfpathlineto{\pgfqpoint{4.561216in}{6.581191in}}%
\pgfpathlineto{\pgfqpoint{4.634306in}{6.452378in}}%
\pgfpathlineto{\pgfqpoint{4.707396in}{6.316973in}}%
\pgfpathlineto{\pgfqpoint{4.776638in}{6.182210in}}%
\pgfpathlineto{\pgfqpoint{4.845881in}{6.040664in}}%
\pgfpathlineto{\pgfqpoint{4.915124in}{5.891769in}}%
\pgfpathlineto{\pgfqpoint{4.915124in}{5.891769in}}%
\pgfusepath{stroke}%
\end{pgfscope}%
\begin{pgfscope}%
\pgfpathrectangle{\pgfqpoint{0.880000in}{4.908628in}}{\pgfqpoint{4.227273in}{3.431818in}} %
\pgfusepath{clip}%
\pgfsetbuttcap%
\pgfsetmiterjoin%
\definecolor{currentfill}{rgb}{1.000000,0.000000,0.000000}%
\pgfsetfillcolor{currentfill}%
\pgfsetlinewidth{1.003750pt}%
\definecolor{currentstroke}{rgb}{1.000000,0.000000,0.000000}%
\pgfsetstrokecolor{currentstroke}%
\pgfsetdash{}{0pt}%
\pgfsys@defobject{currentmarker}{\pgfqpoint{-0.041667in}{-0.041667in}}{\pgfqpoint{0.041667in}{0.041667in}}{%
\pgfpathmoveto{\pgfqpoint{-0.041667in}{-0.041667in}}%
\pgfpathlineto{\pgfqpoint{0.041667in}{-0.041667in}}%
\pgfpathlineto{\pgfqpoint{0.041667in}{0.041667in}}%
\pgfpathlineto{\pgfqpoint{-0.041667in}{0.041667in}}%
\pgfpathclose%
\pgfusepath{stroke,fill}%
}%
\begin{pgfscope}%
\pgfsys@transformshift{1.072149in}{7.335363in}%
\pgfsys@useobject{currentmarker}{}%
\end{pgfscope}%
\begin{pgfscope}%
\pgfsys@transformshift{1.456831in}{7.725709in}%
\pgfsys@useobject{currentmarker}{}%
\end{pgfscope}%
\begin{pgfscope}%
\pgfsys@transformshift{1.841513in}{7.992536in}%
\pgfsys@useobject{currentmarker}{}%
\end{pgfscope}%
\begin{pgfscope}%
\pgfsys@transformshift{2.226195in}{8.143748in}%
\pgfsys@useobject{currentmarker}{}%
\end{pgfscope}%
\begin{pgfscope}%
\pgfsys@transformshift{2.610878in}{8.183346in}%
\pgfsys@useobject{currentmarker}{}%
\end{pgfscope}%
\begin{pgfscope}%
\pgfsys@transformshift{2.995560in}{8.112315in}%
\pgfsys@useobject{currentmarker}{}%
\end{pgfscope}%
\begin{pgfscope}%
\pgfsys@transformshift{3.380242in}{7.928587in}%
\pgfsys@useobject{currentmarker}{}%
\end{pgfscope}%
\begin{pgfscope}%
\pgfsys@transformshift{3.764924in}{7.626730in}%
\pgfsys@useobject{currentmarker}{}%
\end{pgfscope}%
\begin{pgfscope}%
\pgfsys@transformshift{4.149606in}{7.197735in}%
\pgfsys@useobject{currentmarker}{}%
\end{pgfscope}%
\begin{pgfscope}%
\pgfsys@transformshift{4.534289in}{6.627062in}%
\pgfsys@useobject{currentmarker}{}%
\end{pgfscope}%
\end{pgfscope}%
\begin{pgfscope}%
\pgfpathrectangle{\pgfqpoint{0.880000in}{4.908628in}}{\pgfqpoint{4.227273in}{3.431818in}} %
\pgfusepath{clip}%
\pgfsetrectcap%
\pgfsetroundjoin%
\pgfsetlinewidth{1.505625pt}%
\definecolor{currentstroke}{rgb}{0.000000,0.000000,1.000000}%
\pgfsetstrokecolor{currentstroke}%
\pgfsetdash{}{0pt}%
\pgfpathmoveto{\pgfqpoint{1.072149in}{7.335363in}}%
\pgfpathlineto{\pgfqpoint{1.145238in}{7.419559in}}%
\pgfpathlineto{\pgfqpoint{1.214481in}{7.494777in}}%
\pgfpathlineto{\pgfqpoint{1.283724in}{7.565583in}}%
\pgfpathlineto{\pgfqpoint{1.352967in}{7.631984in}}%
\pgfpathlineto{\pgfqpoint{1.422210in}{7.693991in}}%
\pgfpathlineto{\pgfqpoint{1.487606in}{7.748523in}}%
\pgfpathlineto{\pgfqpoint{1.553002in}{7.799148in}}%
\pgfpathlineto{\pgfqpoint{1.618397in}{7.845871in}}%
\pgfpathlineto{\pgfqpoint{1.683793in}{7.888697in}}%
\pgfpathlineto{\pgfqpoint{1.745343in}{7.925450in}}%
\pgfpathlineto{\pgfqpoint{1.806892in}{7.958758in}}%
\pgfpathlineto{\pgfqpoint{1.868441in}{7.988626in}}%
\pgfpathlineto{\pgfqpoint{1.929990in}{8.015056in}}%
\pgfpathlineto{\pgfqpoint{1.991539in}{8.038051in}}%
\pgfpathlineto{\pgfqpoint{2.053088in}{8.057613in}}%
\pgfpathlineto{\pgfqpoint{2.110791in}{8.072837in}}%
\pgfpathlineto{\pgfqpoint{2.168493in}{8.085047in}}%
\pgfpathlineto{\pgfqpoint{2.226195in}{8.094244in}}%
\pgfpathlineto{\pgfqpoint{2.283898in}{8.100429in}}%
\pgfpathlineto{\pgfqpoint{2.341600in}{8.103601in}}%
\pgfpathlineto{\pgfqpoint{2.399302in}{8.103763in}}%
\pgfpathlineto{\pgfqpoint{2.457005in}{8.100913in}}%
\pgfpathlineto{\pgfqpoint{2.514707in}{8.095051in}}%
\pgfpathlineto{\pgfqpoint{2.572409in}{8.086178in}}%
\pgfpathlineto{\pgfqpoint{2.630112in}{8.074291in}}%
\pgfpathlineto{\pgfqpoint{2.687814in}{8.059390in}}%
\pgfpathlineto{\pgfqpoint{2.745516in}{8.041475in}}%
\pgfpathlineto{\pgfqpoint{2.807065in}{8.019039in}}%
\pgfpathlineto{\pgfqpoint{2.868615in}{7.993169in}}%
\pgfpathlineto{\pgfqpoint{2.930164in}{7.963862in}}%
\pgfpathlineto{\pgfqpoint{2.991713in}{7.931115in}}%
\pgfpathlineto{\pgfqpoint{3.053262in}{7.894925in}}%
\pgfpathlineto{\pgfqpoint{3.114811in}{7.855287in}}%
\pgfpathlineto{\pgfqpoint{3.180207in}{7.809392in}}%
\pgfpathlineto{\pgfqpoint{3.245603in}{7.759596in}}%
\pgfpathlineto{\pgfqpoint{3.310999in}{7.705893in}}%
\pgfpathlineto{\pgfqpoint{3.376395in}{7.648278in}}%
\pgfpathlineto{\pgfqpoint{3.445638in}{7.583003in}}%
\pgfpathlineto{\pgfqpoint{3.514881in}{7.513326in}}%
\pgfpathlineto{\pgfqpoint{3.584124in}{7.439238in}}%
\pgfpathlineto{\pgfqpoint{3.653366in}{7.360730in}}%
\pgfpathlineto{\pgfqpoint{3.726456in}{7.273054in}}%
\pgfpathlineto{\pgfqpoint{3.799546in}{7.180429in}}%
\pgfpathlineto{\pgfqpoint{3.872635in}{7.082842in}}%
\pgfpathlineto{\pgfqpoint{3.945725in}{6.980278in}}%
\pgfpathlineto{\pgfqpoint{4.022661in}{6.866923in}}%
\pgfpathlineto{\pgfqpoint{4.099598in}{6.748018in}}%
\pgfpathlineto{\pgfqpoint{4.176534in}{6.623543in}}%
\pgfpathlineto{\pgfqpoint{4.257317in}{6.486829in}}%
\pgfpathlineto{\pgfqpoint{4.338101in}{6.343925in}}%
\pgfpathlineto{\pgfqpoint{4.418884in}{6.194806in}}%
\pgfpathlineto{\pgfqpoint{4.503514in}{6.031889in}}%
\pgfpathlineto{\pgfqpoint{4.588144in}{5.862086in}}%
\pgfpathlineto{\pgfqpoint{4.672774in}{5.685360in}}%
\pgfpathlineto{\pgfqpoint{4.757404in}{5.501673in}}%
\pgfpathlineto{\pgfqpoint{4.845881in}{5.302150in}}%
\pgfpathlineto{\pgfqpoint{4.915124in}{5.140634in}}%
\pgfpathlineto{\pgfqpoint{4.915124in}{5.140634in}}%
\pgfusepath{stroke}%
\end{pgfscope}%
\begin{pgfscope}%
\pgfpathrectangle{\pgfqpoint{0.880000in}{4.908628in}}{\pgfqpoint{4.227273in}{3.431818in}} %
\pgfusepath{clip}%
\pgfsetbuttcap%
\pgfsetroundjoin%
\definecolor{currentfill}{rgb}{0.000000,0.000000,1.000000}%
\pgfsetfillcolor{currentfill}%
\pgfsetlinewidth{1.003750pt}%
\definecolor{currentstroke}{rgb}{0.000000,0.000000,1.000000}%
\pgfsetstrokecolor{currentstroke}%
\pgfsetdash{}{0pt}%
\pgfsys@defobject{currentmarker}{\pgfqpoint{-0.041667in}{-0.041667in}}{\pgfqpoint{0.041667in}{0.041667in}}{%
\pgfpathmoveto{\pgfqpoint{0.000000in}{-0.041667in}}%
\pgfpathcurveto{\pgfqpoint{0.011050in}{-0.041667in}}{\pgfqpoint{0.021649in}{-0.037276in}}{\pgfqpoint{0.029463in}{-0.029463in}}%
\pgfpathcurveto{\pgfqpoint{0.037276in}{-0.021649in}}{\pgfqpoint{0.041667in}{-0.011050in}}{\pgfqpoint{0.041667in}{0.000000in}}%
\pgfpathcurveto{\pgfqpoint{0.041667in}{0.011050in}}{\pgfqpoint{0.037276in}{0.021649in}}{\pgfqpoint{0.029463in}{0.029463in}}%
\pgfpathcurveto{\pgfqpoint{0.021649in}{0.037276in}}{\pgfqpoint{0.011050in}{0.041667in}}{\pgfqpoint{0.000000in}{0.041667in}}%
\pgfpathcurveto{\pgfqpoint{-0.011050in}{0.041667in}}{\pgfqpoint{-0.021649in}{0.037276in}}{\pgfqpoint{-0.029463in}{0.029463in}}%
\pgfpathcurveto{\pgfqpoint{-0.037276in}{0.021649in}}{\pgfqpoint{-0.041667in}{0.011050in}}{\pgfqpoint{-0.041667in}{0.000000in}}%
\pgfpathcurveto{\pgfqpoint{-0.041667in}{-0.011050in}}{\pgfqpoint{-0.037276in}{-0.021649in}}{\pgfqpoint{-0.029463in}{-0.029463in}}%
\pgfpathcurveto{\pgfqpoint{-0.021649in}{-0.037276in}}{\pgfqpoint{-0.011050in}{-0.041667in}}{\pgfqpoint{0.000000in}{-0.041667in}}%
\pgfpathclose%
\pgfusepath{stroke,fill}%
}%
\begin{pgfscope}%
\pgfsys@transformshift{1.072149in}{7.335363in}%
\pgfsys@useobject{currentmarker}{}%
\end{pgfscope}%
\begin{pgfscope}%
\pgfsys@transformshift{1.456831in}{7.723348in}%
\pgfsys@useobject{currentmarker}{}%
\end{pgfscope}%
\begin{pgfscope}%
\pgfsys@transformshift{1.841513in}{7.975982in}%
\pgfsys@useobject{currentmarker}{}%
\end{pgfscope}%
\begin{pgfscope}%
\pgfsys@transformshift{2.226195in}{8.094244in}%
\pgfsys@useobject{currentmarker}{}%
\end{pgfscope}%
\begin{pgfscope}%
\pgfsys@transformshift{2.610878in}{8.078588in}%
\pgfsys@useobject{currentmarker}{}%
\end{pgfscope}%
\begin{pgfscope}%
\pgfsys@transformshift{2.995560in}{7.928954in}%
\pgfsys@useobject{currentmarker}{}%
\end{pgfscope}%
\begin{pgfscope}%
\pgfsys@transformshift{3.380242in}{7.644767in}%
\pgfsys@useobject{currentmarker}{}%
\end{pgfscope}%
\begin{pgfscope}%
\pgfsys@transformshift{3.764924in}{7.224922in}%
\pgfsys@useobject{currentmarker}{}%
\end{pgfscope}%
\begin{pgfscope}%
\pgfsys@transformshift{4.149606in}{6.667744in}%
\pgfsys@useobject{currentmarker}{}%
\end{pgfscope}%
\begin{pgfscope}%
\pgfsys@transformshift{4.534289in}{5.970941in}%
\pgfsys@useobject{currentmarker}{}%
\end{pgfscope}%
\end{pgfscope}%
\begin{pgfscope}%
\pgfpathrectangle{\pgfqpoint{0.880000in}{4.908628in}}{\pgfqpoint{4.227273in}{3.431818in}} %
\pgfusepath{clip}%
\pgfsetbuttcap%
\pgfsetroundjoin%
\pgfsetlinewidth{1.505625pt}%
\definecolor{currentstroke}{rgb}{0.000000,0.750000,0.750000}%
\pgfsetstrokecolor{currentstroke}%
\pgfsetdash{{9.600000pt}{2.400000pt}{1.500000pt}{2.400000pt}}{0.000000pt}%
\pgfpathmoveto{\pgfqpoint{1.072149in}{7.335363in}}%
\pgfpathlineto{\pgfqpoint{1.145238in}{7.419557in}}%
\pgfpathlineto{\pgfqpoint{1.214481in}{7.494764in}}%
\pgfpathlineto{\pgfqpoint{1.283724in}{7.565539in}}%
\pgfpathlineto{\pgfqpoint{1.352967in}{7.631884in}}%
\pgfpathlineto{\pgfqpoint{1.422210in}{7.693800in}}%
\pgfpathlineto{\pgfqpoint{1.487606in}{7.748209in}}%
\pgfpathlineto{\pgfqpoint{1.553002in}{7.798668in}}%
\pgfpathlineto{\pgfqpoint{1.618397in}{7.845177in}}%
\pgfpathlineto{\pgfqpoint{1.683793in}{7.887739in}}%
\pgfpathlineto{\pgfqpoint{1.745343in}{7.924189in}}%
\pgfpathlineto{\pgfqpoint{1.806892in}{7.957143in}}%
\pgfpathlineto{\pgfqpoint{1.868441in}{7.986600in}}%
\pgfpathlineto{\pgfqpoint{1.929990in}{8.012561in}}%
\pgfpathlineto{\pgfqpoint{1.991539in}{8.035026in}}%
\pgfpathlineto{\pgfqpoint{2.049242in}{8.052911in}}%
\pgfpathlineto{\pgfqpoint{2.106944in}{8.067724in}}%
\pgfpathlineto{\pgfqpoint{2.164646in}{8.079465in}}%
\pgfpathlineto{\pgfqpoint{2.222349in}{8.088134in}}%
\pgfpathlineto{\pgfqpoint{2.280051in}{8.093731in}}%
\pgfpathlineto{\pgfqpoint{2.337753in}{8.096256in}}%
\pgfpathlineto{\pgfqpoint{2.395456in}{8.095708in}}%
\pgfpathlineto{\pgfqpoint{2.453158in}{8.092089in}}%
\pgfpathlineto{\pgfqpoint{2.510860in}{8.085397in}}%
\pgfpathlineto{\pgfqpoint{2.568563in}{8.075634in}}%
\pgfpathlineto{\pgfqpoint{2.626265in}{8.062798in}}%
\pgfpathlineto{\pgfqpoint{2.683967in}{8.046890in}}%
\pgfpathlineto{\pgfqpoint{2.741670in}{8.027910in}}%
\pgfpathlineto{\pgfqpoint{2.803219in}{8.004277in}}%
\pgfpathlineto{\pgfqpoint{2.864768in}{7.977149in}}%
\pgfpathlineto{\pgfqpoint{2.926317in}{7.946523in}}%
\pgfpathlineto{\pgfqpoint{2.987866in}{7.912401in}}%
\pgfpathlineto{\pgfqpoint{3.049415in}{7.874782in}}%
\pgfpathlineto{\pgfqpoint{3.110964in}{7.833666in}}%
\pgfpathlineto{\pgfqpoint{3.176360in}{7.786147in}}%
\pgfpathlineto{\pgfqpoint{3.241756in}{7.734678in}}%
\pgfpathlineto{\pgfqpoint{3.307152in}{7.679259in}}%
\pgfpathlineto{\pgfqpoint{3.372548in}{7.619890in}}%
\pgfpathlineto{\pgfqpoint{3.441791in}{7.552722in}}%
\pgfpathlineto{\pgfqpoint{3.511034in}{7.481123in}}%
\pgfpathlineto{\pgfqpoint{3.580277in}{7.405092in}}%
\pgfpathlineto{\pgfqpoint{3.649520in}{7.324629in}}%
\pgfpathlineto{\pgfqpoint{3.722609in}{7.234886in}}%
\pgfpathlineto{\pgfqpoint{3.795699in}{7.140201in}}%
\pgfpathlineto{\pgfqpoint{3.868788in}{7.040574in}}%
\pgfpathlineto{\pgfqpoint{3.945725in}{6.930361in}}%
\pgfpathlineto{\pgfqpoint{4.022661in}{6.814669in}}%
\pgfpathlineto{\pgfqpoint{4.099598in}{6.693495in}}%
\pgfpathlineto{\pgfqpoint{4.180381in}{6.560361in}}%
\pgfpathlineto{\pgfqpoint{4.261164in}{6.421178in}}%
\pgfpathlineto{\pgfqpoint{4.341947in}{6.275944in}}%
\pgfpathlineto{\pgfqpoint{4.426578in}{6.117302in}}%
\pgfpathlineto{\pgfqpoint{4.511208in}{5.952013in}}%
\pgfpathlineto{\pgfqpoint{4.595838in}{5.780075in}}%
\pgfpathlineto{\pgfqpoint{4.684315in}{5.593207in}}%
\pgfpathlineto{\pgfqpoint{4.772792in}{5.399064in}}%
\pgfpathlineto{\pgfqpoint{4.861268in}{5.197640in}}%
\pgfpathlineto{\pgfqpoint{4.915124in}{5.071469in}}%
\pgfpathlineto{\pgfqpoint{4.915124in}{5.071469in}}%
\pgfusepath{stroke}%
\end{pgfscope}%
\begin{pgfscope}%
\pgfpathrectangle{\pgfqpoint{0.880000in}{4.908628in}}{\pgfqpoint{4.227273in}{3.431818in}} %
\pgfusepath{clip}%
\pgfsetbuttcap%
\pgfsetmiterjoin%
\definecolor{currentfill}{rgb}{0.000000,0.750000,0.750000}%
\pgfsetfillcolor{currentfill}%
\pgfsetlinewidth{1.003750pt}%
\definecolor{currentstroke}{rgb}{0.000000,0.750000,0.750000}%
\pgfsetstrokecolor{currentstroke}%
\pgfsetdash{}{0pt}%
\pgfsys@defobject{currentmarker}{\pgfqpoint{-0.041667in}{-0.041667in}}{\pgfqpoint{0.041667in}{0.041667in}}{%
\pgfpathmoveto{\pgfqpoint{-0.000000in}{-0.041667in}}%
\pgfpathlineto{\pgfqpoint{0.041667in}{0.041667in}}%
\pgfpathlineto{\pgfqpoint{-0.041667in}{0.041667in}}%
\pgfpathclose%
\pgfusepath{stroke,fill}%
}%
\begin{pgfscope}%
\pgfsys@transformshift{1.072149in}{7.335363in}%
\pgfsys@useobject{currentmarker}{}%
\end{pgfscope}%
\begin{pgfscope}%
\pgfsys@transformshift{1.456831in}{7.723096in}%
\pgfsys@useobject{currentmarker}{}%
\end{pgfscope}%
\begin{pgfscope}%
\pgfsys@transformshift{1.841513in}{7.974143in}%
\pgfsys@useobject{currentmarker}{}%
\end{pgfscope}%
\begin{pgfscope}%
\pgfsys@transformshift{2.226195in}{8.088603in}%
\pgfsys@useobject{currentmarker}{}%
\end{pgfscope}%
\begin{pgfscope}%
\pgfsys@transformshift{2.610878in}{8.066522in}%
\pgfsys@useobject{currentmarker}{}%
\end{pgfscope}%
\begin{pgfscope}%
\pgfsys@transformshift{2.995560in}{7.907890in}%
\pgfsys@useobject{currentmarker}{}%
\end{pgfscope}%
\begin{pgfscope}%
\pgfsys@transformshift{3.380242in}{7.612646in}%
\pgfsys@useobject{currentmarker}{}%
\end{pgfscope}%
\begin{pgfscope}%
\pgfsys@transformshift{3.764924in}{7.180671in}%
\pgfsys@useobject{currentmarker}{}%
\end{pgfscope}%
\begin{pgfscope}%
\pgfsys@transformshift{4.149606in}{6.611792in}%
\pgfsys@useobject{currentmarker}{}%
\end{pgfscope}%
\begin{pgfscope}%
\pgfsys@transformshift{4.534289in}{5.905781in}%
\pgfsys@useobject{currentmarker}{}%
\end{pgfscope}%
\end{pgfscope}%
\begin{pgfscope}%
\pgfpathrectangle{\pgfqpoint{0.880000in}{4.908628in}}{\pgfqpoint{4.227273in}{3.431818in}} %
\pgfusepath{clip}%
\pgfsetbuttcap%
\pgfsetroundjoin%
\pgfsetlinewidth{1.505625pt}%
\definecolor{currentstroke}{rgb}{0.000000,0.000000,0.000000}%
\pgfsetstrokecolor{currentstroke}%
\pgfsetdash{{1.500000pt}{2.475000pt}}{0.000000pt}%
\pgfpathmoveto{\pgfqpoint{1.072149in}{7.335363in}}%
\pgfpathlineto{\pgfqpoint{1.145238in}{7.419557in}}%
\pgfpathlineto{\pgfqpoint{1.214481in}{7.494762in}}%
\pgfpathlineto{\pgfqpoint{1.283724in}{7.565535in}}%
\pgfpathlineto{\pgfqpoint{1.352967in}{7.631874in}}%
\pgfpathlineto{\pgfqpoint{1.422210in}{7.693780in}}%
\pgfpathlineto{\pgfqpoint{1.487606in}{7.748177in}}%
\pgfpathlineto{\pgfqpoint{1.553002in}{7.798619in}}%
\pgfpathlineto{\pgfqpoint{1.618397in}{7.845107in}}%
\pgfpathlineto{\pgfqpoint{1.683793in}{7.887642in}}%
\pgfpathlineto{\pgfqpoint{1.745343in}{7.924062in}}%
\pgfpathlineto{\pgfqpoint{1.806892in}{7.956980in}}%
\pgfpathlineto{\pgfqpoint{1.868441in}{7.986395in}}%
\pgfpathlineto{\pgfqpoint{1.929990in}{8.012309in}}%
\pgfpathlineto{\pgfqpoint{1.991539in}{8.034719in}}%
\pgfpathlineto{\pgfqpoint{2.049242in}{8.052549in}}%
\pgfpathlineto{\pgfqpoint{2.106944in}{8.067300in}}%
\pgfpathlineto{\pgfqpoint{2.164646in}{8.078973in}}%
\pgfpathlineto{\pgfqpoint{2.222349in}{8.087568in}}%
\pgfpathlineto{\pgfqpoint{2.280051in}{8.093084in}}%
\pgfpathlineto{\pgfqpoint{2.337753in}{8.095523in}}%
\pgfpathlineto{\pgfqpoint{2.395456in}{8.094883in}}%
\pgfpathlineto{\pgfqpoint{2.453158in}{8.091165in}}%
\pgfpathlineto{\pgfqpoint{2.510860in}{8.084369in}}%
\pgfpathlineto{\pgfqpoint{2.568563in}{8.074495in}}%
\pgfpathlineto{\pgfqpoint{2.626265in}{8.061543in}}%
\pgfpathlineto{\pgfqpoint{2.683967in}{8.045512in}}%
\pgfpathlineto{\pgfqpoint{2.741670in}{8.026403in}}%
\pgfpathlineto{\pgfqpoint{2.799372in}{8.004216in}}%
\pgfpathlineto{\pgfqpoint{2.860921in}{7.977157in}}%
\pgfpathlineto{\pgfqpoint{2.922470in}{7.946596in}}%
\pgfpathlineto{\pgfqpoint{2.984019in}{7.912532in}}%
\pgfpathlineto{\pgfqpoint{3.045568in}{7.874966in}}%
\pgfpathlineto{\pgfqpoint{3.107118in}{7.833897in}}%
\pgfpathlineto{\pgfqpoint{3.172514in}{7.786424in}}%
\pgfpathlineto{\pgfqpoint{3.237910in}{7.734997in}}%
\pgfpathlineto{\pgfqpoint{3.303306in}{7.679616in}}%
\pgfpathlineto{\pgfqpoint{3.368702in}{7.620280in}}%
\pgfpathlineto{\pgfqpoint{3.437944in}{7.553144in}}%
\pgfpathlineto{\pgfqpoint{3.507187in}{7.481575in}}%
\pgfpathlineto{\pgfqpoint{3.576430in}{7.405572in}}%
\pgfpathlineto{\pgfqpoint{3.645673in}{7.325136in}}%
\pgfpathlineto{\pgfqpoint{3.718762in}{7.235422in}}%
\pgfpathlineto{\pgfqpoint{3.791852in}{7.140768in}}%
\pgfpathlineto{\pgfqpoint{3.864942in}{7.041173in}}%
\pgfpathlineto{\pgfqpoint{3.941878in}{6.931000in}}%
\pgfpathlineto{\pgfqpoint{4.018814in}{6.815352in}}%
\pgfpathlineto{\pgfqpoint{4.095751in}{6.694230in}}%
\pgfpathlineto{\pgfqpoint{4.176534in}{6.561160in}}%
\pgfpathlineto{\pgfqpoint{4.257317in}{6.422054in}}%
\pgfpathlineto{\pgfqpoint{4.338101in}{6.276912in}}%
\pgfpathlineto{\pgfqpoint{4.422731in}{6.118385in}}%
\pgfpathlineto{\pgfqpoint{4.507361in}{5.953231in}}%
\pgfpathlineto{\pgfqpoint{4.591991in}{5.781453in}}%
\pgfpathlineto{\pgfqpoint{4.680468in}{5.594781in}}%
\pgfpathlineto{\pgfqpoint{4.768945in}{5.400867in}}%
\pgfpathlineto{\pgfqpoint{4.857422in}{5.199711in}}%
\pgfpathlineto{\pgfqpoint{4.915124in}{5.064619in}}%
\pgfpathlineto{\pgfqpoint{4.915124in}{5.064619in}}%
\pgfusepath{stroke}%
\end{pgfscope}%
\begin{pgfscope}%
\pgfpathrectangle{\pgfqpoint{0.880000in}{4.908628in}}{\pgfqpoint{4.227273in}{3.431818in}} %
\pgfusepath{clip}%
\pgfsetbuttcap%
\pgfsetroundjoin%
\definecolor{currentfill}{rgb}{0.000000,0.000000,0.000000}%
\pgfsetfillcolor{currentfill}%
\pgfsetlinewidth{1.003750pt}%
\definecolor{currentstroke}{rgb}{0.000000,0.000000,0.000000}%
\pgfsetstrokecolor{currentstroke}%
\pgfsetdash{}{0pt}%
\pgfsys@defobject{currentmarker}{\pgfqpoint{-0.041667in}{-0.041667in}}{\pgfqpoint{0.041667in}{0.041667in}}{%
\pgfpathmoveto{\pgfqpoint{-0.041667in}{0.000000in}}%
\pgfpathlineto{\pgfqpoint{0.041667in}{0.000000in}}%
\pgfpathmoveto{\pgfqpoint{0.000000in}{-0.041667in}}%
\pgfpathlineto{\pgfqpoint{0.000000in}{0.041667in}}%
\pgfusepath{stroke,fill}%
}%
\begin{pgfscope}%
\pgfsys@transformshift{1.072149in}{7.335363in}%
\pgfsys@useobject{currentmarker}{}%
\end{pgfscope}%
\begin{pgfscope}%
\pgfsys@transformshift{1.456831in}{7.723071in}%
\pgfsys@useobject{currentmarker}{}%
\end{pgfscope}%
\begin{pgfscope}%
\pgfsys@transformshift{1.841513in}{7.973957in}%
\pgfsys@useobject{currentmarker}{}%
\end{pgfscope}%
\begin{pgfscope}%
\pgfsys@transformshift{2.226195in}{8.088031in}%
\pgfsys@useobject{currentmarker}{}%
\end{pgfscope}%
\begin{pgfscope}%
\pgfsys@transformshift{2.610878in}{8.065297in}%
\pgfsys@useobject{currentmarker}{}%
\end{pgfscope}%
\begin{pgfscope}%
\pgfsys@transformshift{2.995560in}{7.905755in}%
\pgfsys@useobject{currentmarker}{}%
\end{pgfscope}%
\begin{pgfscope}%
\pgfsys@transformshift{3.380242in}{7.609399in}%
\pgfsys@useobject{currentmarker}{}%
\end{pgfscope}%
\begin{pgfscope}%
\pgfsys@transformshift{3.764924in}{7.176215in}%
\pgfsys@useobject{currentmarker}{}%
\end{pgfscope}%
\begin{pgfscope}%
\pgfsys@transformshift{4.149606in}{6.606188in}%
\pgfsys@useobject{currentmarker}{}%
\end{pgfscope}%
\begin{pgfscope}%
\pgfsys@transformshift{4.534289in}{5.899293in}%
\pgfsys@useobject{currentmarker}{}%
\end{pgfscope}%
\end{pgfscope}%
\begin{pgfscope}%
\pgfsetrectcap%
\pgfsetmiterjoin%
\pgfsetlinewidth{0.803000pt}%
\definecolor{currentstroke}{rgb}{0.000000,0.000000,0.000000}%
\pgfsetstrokecolor{currentstroke}%
\pgfsetdash{}{0pt}%
\pgfpathmoveto{\pgfqpoint{0.880000in}{4.908628in}}%
\pgfpathlineto{\pgfqpoint{0.880000in}{8.340446in}}%
\pgfusepath{stroke}%
\end{pgfscope}%
\begin{pgfscope}%
\pgfsetrectcap%
\pgfsetmiterjoin%
\pgfsetlinewidth{0.803000pt}%
\definecolor{currentstroke}{rgb}{0.000000,0.000000,0.000000}%
\pgfsetstrokecolor{currentstroke}%
\pgfsetdash{}{0pt}%
\pgfpathmoveto{\pgfqpoint{5.107273in}{4.908628in}}%
\pgfpathlineto{\pgfqpoint{5.107273in}{8.340446in}}%
\pgfusepath{stroke}%
\end{pgfscope}%
\begin{pgfscope}%
\pgfsetrectcap%
\pgfsetmiterjoin%
\pgfsetlinewidth{0.803000pt}%
\definecolor{currentstroke}{rgb}{0.000000,0.000000,0.000000}%
\pgfsetstrokecolor{currentstroke}%
\pgfsetdash{}{0pt}%
\pgfpathmoveto{\pgfqpoint{0.880000in}{4.908628in}}%
\pgfpathlineto{\pgfqpoint{5.107273in}{4.908628in}}%
\pgfusepath{stroke}%
\end{pgfscope}%
\begin{pgfscope}%
\pgfsetrectcap%
\pgfsetmiterjoin%
\pgfsetlinewidth{0.803000pt}%
\definecolor{currentstroke}{rgb}{0.000000,0.000000,0.000000}%
\pgfsetstrokecolor{currentstroke}%
\pgfsetdash{}{0pt}%
\pgfpathmoveto{\pgfqpoint{0.880000in}{8.340446in}}%
\pgfpathlineto{\pgfqpoint{5.107273in}{8.340446in}}%
\pgfusepath{stroke}%
\end{pgfscope}%
\begin{pgfscope}%
\pgfsetbuttcap%
\pgfsetmiterjoin%
\definecolor{currentfill}{rgb}{1.000000,1.000000,1.000000}%
\pgfsetfillcolor{currentfill}%
\pgfsetfillopacity{0.800000}%
\pgfsetlinewidth{1.003750pt}%
\definecolor{currentstroke}{rgb}{0.800000,0.800000,0.800000}%
\pgfsetstrokecolor{currentstroke}%
\pgfsetstrokeopacity{0.800000}%
\pgfsetdash{}{0pt}%
\pgfpathmoveto{\pgfqpoint{2.442086in}{4.978072in}}%
\pgfpathlineto{\pgfqpoint{3.545187in}{4.978072in}}%
\pgfpathquadraticcurveto{\pgfqpoint{3.572965in}{4.978072in}}{\pgfqpoint{3.572965in}{5.005850in}}%
\pgfpathlineto{\pgfqpoint{3.572965in}{6.011247in}}%
\pgfpathquadraticcurveto{\pgfqpoint{3.572965in}{6.039025in}}{\pgfqpoint{3.545187in}{6.039025in}}%
\pgfpathlineto{\pgfqpoint{2.442086in}{6.039025in}}%
\pgfpathquadraticcurveto{\pgfqpoint{2.414308in}{6.039025in}}{\pgfqpoint{2.414308in}{6.011247in}}%
\pgfpathlineto{\pgfqpoint{2.414308in}{5.005850in}}%
\pgfpathquadraticcurveto{\pgfqpoint{2.414308in}{4.978072in}}{\pgfqpoint{2.442086in}{4.978072in}}%
\pgfpathclose%
\pgfusepath{stroke,fill}%
\end{pgfscope}%
\begin{pgfscope}%
\pgftext[x=2.802437in,y=5.877946in,left,base]{\rmfamily\fontsize{10.000000}{12.000000}\selectfont \(\displaystyle \alpha\) = 2}%
\end{pgfscope}%
\begin{pgfscope}%
\pgfsetbuttcap%
\pgfsetroundjoin%
\pgfsetlinewidth{1.505625pt}%
\definecolor{currentstroke}{rgb}{1.000000,0.000000,0.000000}%
\pgfsetstrokecolor{currentstroke}%
\pgfsetdash{{5.550000pt}{2.400000pt}}{0.000000pt}%
\pgfpathmoveto{\pgfqpoint{2.469863in}{5.722700in}}%
\pgfpathlineto{\pgfqpoint{2.747641in}{5.722700in}}%
\pgfusepath{stroke}%
\end{pgfscope}%
\begin{pgfscope}%
\pgfsetbuttcap%
\pgfsetmiterjoin%
\definecolor{currentfill}{rgb}{1.000000,0.000000,0.000000}%
\pgfsetfillcolor{currentfill}%
\pgfsetlinewidth{1.003750pt}%
\definecolor{currentstroke}{rgb}{1.000000,0.000000,0.000000}%
\pgfsetstrokecolor{currentstroke}%
\pgfsetdash{}{0pt}%
\pgfsys@defobject{currentmarker}{\pgfqpoint{-0.041667in}{-0.041667in}}{\pgfqpoint{0.041667in}{0.041667in}}{%
\pgfpathmoveto{\pgfqpoint{-0.041667in}{-0.041667in}}%
\pgfpathlineto{\pgfqpoint{0.041667in}{-0.041667in}}%
\pgfpathlineto{\pgfqpoint{0.041667in}{0.041667in}}%
\pgfpathlineto{\pgfqpoint{-0.041667in}{0.041667in}}%
\pgfpathclose%
\pgfusepath{stroke,fill}%
}%
\begin{pgfscope}%
\pgfsys@transformshift{2.608752in}{5.722700in}%
\pgfsys@useobject{currentmarker}{}%
\end{pgfscope}%
\end{pgfscope}%
\begin{pgfscope}%
\pgftext[x=2.858752in,y=5.674089in,left,base]{\rmfamily\fontsize{10.000000}{12.000000}\selectfont \(\displaystyle \epsilon\) = 1}%
\end{pgfscope}%
\begin{pgfscope}%
\pgfsetrectcap%
\pgfsetroundjoin%
\pgfsetlinewidth{1.505625pt}%
\definecolor{currentstroke}{rgb}{0.000000,0.000000,1.000000}%
\pgfsetstrokecolor{currentstroke}%
\pgfsetdash{}{0pt}%
\pgfpathmoveto{\pgfqpoint{2.469863in}{5.518843in}}%
\pgfpathlineto{\pgfqpoint{2.747641in}{5.518843in}}%
\pgfusepath{stroke}%
\end{pgfscope}%
\begin{pgfscope}%
\pgfsetbuttcap%
\pgfsetroundjoin%
\definecolor{currentfill}{rgb}{0.000000,0.000000,1.000000}%
\pgfsetfillcolor{currentfill}%
\pgfsetlinewidth{1.003750pt}%
\definecolor{currentstroke}{rgb}{0.000000,0.000000,1.000000}%
\pgfsetstrokecolor{currentstroke}%
\pgfsetdash{}{0pt}%
\pgfsys@defobject{currentmarker}{\pgfqpoint{-0.041667in}{-0.041667in}}{\pgfqpoint{0.041667in}{0.041667in}}{%
\pgfpathmoveto{\pgfqpoint{0.000000in}{-0.041667in}}%
\pgfpathcurveto{\pgfqpoint{0.011050in}{-0.041667in}}{\pgfqpoint{0.021649in}{-0.037276in}}{\pgfqpoint{0.029463in}{-0.029463in}}%
\pgfpathcurveto{\pgfqpoint{0.037276in}{-0.021649in}}{\pgfqpoint{0.041667in}{-0.011050in}}{\pgfqpoint{0.041667in}{0.000000in}}%
\pgfpathcurveto{\pgfqpoint{0.041667in}{0.011050in}}{\pgfqpoint{0.037276in}{0.021649in}}{\pgfqpoint{0.029463in}{0.029463in}}%
\pgfpathcurveto{\pgfqpoint{0.021649in}{0.037276in}}{\pgfqpoint{0.011050in}{0.041667in}}{\pgfqpoint{0.000000in}{0.041667in}}%
\pgfpathcurveto{\pgfqpoint{-0.011050in}{0.041667in}}{\pgfqpoint{-0.021649in}{0.037276in}}{\pgfqpoint{-0.029463in}{0.029463in}}%
\pgfpathcurveto{\pgfqpoint{-0.037276in}{0.021649in}}{\pgfqpoint{-0.041667in}{0.011050in}}{\pgfqpoint{-0.041667in}{0.000000in}}%
\pgfpathcurveto{\pgfqpoint{-0.041667in}{-0.011050in}}{\pgfqpoint{-0.037276in}{-0.021649in}}{\pgfqpoint{-0.029463in}{-0.029463in}}%
\pgfpathcurveto{\pgfqpoint{-0.021649in}{-0.037276in}}{\pgfqpoint{-0.011050in}{-0.041667in}}{\pgfqpoint{0.000000in}{-0.041667in}}%
\pgfpathclose%
\pgfusepath{stroke,fill}%
}%
\begin{pgfscope}%
\pgfsys@transformshift{2.608752in}{5.518843in}%
\pgfsys@useobject{currentmarker}{}%
\end{pgfscope}%
\end{pgfscope}%
\begin{pgfscope}%
\pgftext[x=2.858752in,y=5.470232in,left,base]{\rmfamily\fontsize{10.000000}{12.000000}\selectfont \(\displaystyle \epsilon\) = 0.1}%
\end{pgfscope}%
\begin{pgfscope}%
\pgfsetbuttcap%
\pgfsetroundjoin%
\pgfsetlinewidth{1.505625pt}%
\definecolor{currentstroke}{rgb}{0.000000,0.750000,0.750000}%
\pgfsetstrokecolor{currentstroke}%
\pgfsetdash{{9.600000pt}{2.400000pt}{1.500000pt}{2.400000pt}}{0.000000pt}%
\pgfpathmoveto{\pgfqpoint{2.469863in}{5.314986in}}%
\pgfpathlineto{\pgfqpoint{2.747641in}{5.314986in}}%
\pgfusepath{stroke}%
\end{pgfscope}%
\begin{pgfscope}%
\pgfsetbuttcap%
\pgfsetmiterjoin%
\definecolor{currentfill}{rgb}{0.000000,0.750000,0.750000}%
\pgfsetfillcolor{currentfill}%
\pgfsetlinewidth{1.003750pt}%
\definecolor{currentstroke}{rgb}{0.000000,0.750000,0.750000}%
\pgfsetstrokecolor{currentstroke}%
\pgfsetdash{}{0pt}%
\pgfsys@defobject{currentmarker}{\pgfqpoint{-0.041667in}{-0.041667in}}{\pgfqpoint{0.041667in}{0.041667in}}{%
\pgfpathmoveto{\pgfqpoint{-0.000000in}{-0.041667in}}%
\pgfpathlineto{\pgfqpoint{0.041667in}{0.041667in}}%
\pgfpathlineto{\pgfqpoint{-0.041667in}{0.041667in}}%
\pgfpathclose%
\pgfusepath{stroke,fill}%
}%
\begin{pgfscope}%
\pgfsys@transformshift{2.608752in}{5.314986in}%
\pgfsys@useobject{currentmarker}{}%
\end{pgfscope}%
\end{pgfscope}%
\begin{pgfscope}%
\pgftext[x=2.858752in,y=5.266375in,left,base]{\rmfamily\fontsize{10.000000}{12.000000}\selectfont \(\displaystyle \epsilon\) = 0.01}%
\end{pgfscope}%
\begin{pgfscope}%
\pgfsetbuttcap%
\pgfsetroundjoin%
\pgfsetlinewidth{1.505625pt}%
\definecolor{currentstroke}{rgb}{0.000000,0.000000,0.000000}%
\pgfsetstrokecolor{currentstroke}%
\pgfsetdash{{1.500000pt}{2.475000pt}}{0.000000pt}%
\pgfpathmoveto{\pgfqpoint{2.469863in}{5.111128in}}%
\pgfpathlineto{\pgfqpoint{2.747641in}{5.111128in}}%
\pgfusepath{stroke}%
\end{pgfscope}%
\begin{pgfscope}%
\pgfsetbuttcap%
\pgfsetroundjoin%
\definecolor{currentfill}{rgb}{0.000000,0.000000,0.000000}%
\pgfsetfillcolor{currentfill}%
\pgfsetlinewidth{1.003750pt}%
\definecolor{currentstroke}{rgb}{0.000000,0.000000,0.000000}%
\pgfsetstrokecolor{currentstroke}%
\pgfsetdash{}{0pt}%
\pgfsys@defobject{currentmarker}{\pgfqpoint{-0.041667in}{-0.041667in}}{\pgfqpoint{0.041667in}{0.041667in}}{%
\pgfpathmoveto{\pgfqpoint{-0.041667in}{0.000000in}}%
\pgfpathlineto{\pgfqpoint{0.041667in}{0.000000in}}%
\pgfpathmoveto{\pgfqpoint{0.000000in}{-0.041667in}}%
\pgfpathlineto{\pgfqpoint{0.000000in}{0.041667in}}%
\pgfusepath{stroke,fill}%
}%
\begin{pgfscope}%
\pgfsys@transformshift{2.608752in}{5.111128in}%
\pgfsys@useobject{currentmarker}{}%
\end{pgfscope}%
\end{pgfscope}%
\begin{pgfscope}%
\pgftext[x=2.858752in,y=5.062517in,left,base]{\rmfamily\fontsize{10.000000}{12.000000}\selectfont \(\displaystyle \epsilon\) = 0.001}%
\end{pgfscope}%
\begin{pgfscope}%
\pgfsetbuttcap%
\pgfsetmiterjoin%
\definecolor{currentfill}{rgb}{1.000000,1.000000,1.000000}%
\pgfsetfillcolor{currentfill}%
\pgfsetlinewidth{0.000000pt}%
\definecolor{currentstroke}{rgb}{0.000000,0.000000,0.000000}%
\pgfsetstrokecolor{currentstroke}%
\pgfsetstrokeopacity{0.000000}%
\pgfsetdash{}{0pt}%
\pgfpathmoveto{\pgfqpoint{5.952727in}{4.908628in}}%
\pgfpathlineto{\pgfqpoint{10.180000in}{4.908628in}}%
\pgfpathlineto{\pgfqpoint{10.180000in}{8.340446in}}%
\pgfpathlineto{\pgfqpoint{5.952727in}{8.340446in}}%
\pgfpathclose%
\pgfusepath{fill}%
\end{pgfscope}%
\begin{pgfscope}%
\pgfsetbuttcap%
\pgfsetroundjoin%
\definecolor{currentfill}{rgb}{0.000000,0.000000,0.000000}%
\pgfsetfillcolor{currentfill}%
\pgfsetlinewidth{0.803000pt}%
\definecolor{currentstroke}{rgb}{0.000000,0.000000,0.000000}%
\pgfsetstrokecolor{currentstroke}%
\pgfsetdash{}{0pt}%
\pgfsys@defobject{currentmarker}{\pgfqpoint{0.000000in}{-0.048611in}}{\pgfqpoint{0.000000in}{0.000000in}}{%
\pgfpathmoveto{\pgfqpoint{0.000000in}{0.000000in}}%
\pgfpathlineto{\pgfqpoint{0.000000in}{-0.048611in}}%
\pgfusepath{stroke,fill}%
}%
\begin{pgfscope}%
\pgfsys@transformshift{6.144876in}{4.908628in}%
\pgfsys@useobject{currentmarker}{}%
\end{pgfscope}%
\end{pgfscope}%
\begin{pgfscope}%
\pgftext[x=6.144876in,y=4.811405in,,top]{\rmfamily\fontsize{10.000000}{12.000000}\selectfont \(\displaystyle 0.0\)}%
\end{pgfscope}%
\begin{pgfscope}%
\pgfsetbuttcap%
\pgfsetroundjoin%
\definecolor{currentfill}{rgb}{0.000000,0.000000,0.000000}%
\pgfsetfillcolor{currentfill}%
\pgfsetlinewidth{0.803000pt}%
\definecolor{currentstroke}{rgb}{0.000000,0.000000,0.000000}%
\pgfsetstrokecolor{currentstroke}%
\pgfsetdash{}{0pt}%
\pgfsys@defobject{currentmarker}{\pgfqpoint{0.000000in}{-0.048611in}}{\pgfqpoint{0.000000in}{0.000000in}}{%
\pgfpathmoveto{\pgfqpoint{0.000000in}{0.000000in}}%
\pgfpathlineto{\pgfqpoint{0.000000in}{-0.048611in}}%
\pgfusepath{stroke,fill}%
}%
\begin{pgfscope}%
\pgfsys@transformshift{6.914240in}{4.908628in}%
\pgfsys@useobject{currentmarker}{}%
\end{pgfscope}%
\end{pgfscope}%
\begin{pgfscope}%
\pgftext[x=6.914240in,y=4.811405in,,top]{\rmfamily\fontsize{10.000000}{12.000000}\selectfont \(\displaystyle 0.2\)}%
\end{pgfscope}%
\begin{pgfscope}%
\pgfsetbuttcap%
\pgfsetroundjoin%
\definecolor{currentfill}{rgb}{0.000000,0.000000,0.000000}%
\pgfsetfillcolor{currentfill}%
\pgfsetlinewidth{0.803000pt}%
\definecolor{currentstroke}{rgb}{0.000000,0.000000,0.000000}%
\pgfsetstrokecolor{currentstroke}%
\pgfsetdash{}{0pt}%
\pgfsys@defobject{currentmarker}{\pgfqpoint{0.000000in}{-0.048611in}}{\pgfqpoint{0.000000in}{0.000000in}}{%
\pgfpathmoveto{\pgfqpoint{0.000000in}{0.000000in}}%
\pgfpathlineto{\pgfqpoint{0.000000in}{-0.048611in}}%
\pgfusepath{stroke,fill}%
}%
\begin{pgfscope}%
\pgfsys@transformshift{7.683605in}{4.908628in}%
\pgfsys@useobject{currentmarker}{}%
\end{pgfscope}%
\end{pgfscope}%
\begin{pgfscope}%
\pgftext[x=7.683605in,y=4.811405in,,top]{\rmfamily\fontsize{10.000000}{12.000000}\selectfont \(\displaystyle 0.4\)}%
\end{pgfscope}%
\begin{pgfscope}%
\pgfsetbuttcap%
\pgfsetroundjoin%
\definecolor{currentfill}{rgb}{0.000000,0.000000,0.000000}%
\pgfsetfillcolor{currentfill}%
\pgfsetlinewidth{0.803000pt}%
\definecolor{currentstroke}{rgb}{0.000000,0.000000,0.000000}%
\pgfsetstrokecolor{currentstroke}%
\pgfsetdash{}{0pt}%
\pgfsys@defobject{currentmarker}{\pgfqpoint{0.000000in}{-0.048611in}}{\pgfqpoint{0.000000in}{0.000000in}}{%
\pgfpathmoveto{\pgfqpoint{0.000000in}{0.000000in}}%
\pgfpathlineto{\pgfqpoint{0.000000in}{-0.048611in}}%
\pgfusepath{stroke,fill}%
}%
\begin{pgfscope}%
\pgfsys@transformshift{8.452969in}{4.908628in}%
\pgfsys@useobject{currentmarker}{}%
\end{pgfscope}%
\end{pgfscope}%
\begin{pgfscope}%
\pgftext[x=8.452969in,y=4.811405in,,top]{\rmfamily\fontsize{10.000000}{12.000000}\selectfont \(\displaystyle 0.6\)}%
\end{pgfscope}%
\begin{pgfscope}%
\pgfsetbuttcap%
\pgfsetroundjoin%
\definecolor{currentfill}{rgb}{0.000000,0.000000,0.000000}%
\pgfsetfillcolor{currentfill}%
\pgfsetlinewidth{0.803000pt}%
\definecolor{currentstroke}{rgb}{0.000000,0.000000,0.000000}%
\pgfsetstrokecolor{currentstroke}%
\pgfsetdash{}{0pt}%
\pgfsys@defobject{currentmarker}{\pgfqpoint{0.000000in}{-0.048611in}}{\pgfqpoint{0.000000in}{0.000000in}}{%
\pgfpathmoveto{\pgfqpoint{0.000000in}{0.000000in}}%
\pgfpathlineto{\pgfqpoint{0.000000in}{-0.048611in}}%
\pgfusepath{stroke,fill}%
}%
\begin{pgfscope}%
\pgfsys@transformshift{9.222334in}{4.908628in}%
\pgfsys@useobject{currentmarker}{}%
\end{pgfscope}%
\end{pgfscope}%
\begin{pgfscope}%
\pgftext[x=9.222334in,y=4.811405in,,top]{\rmfamily\fontsize{10.000000}{12.000000}\selectfont \(\displaystyle 0.8\)}%
\end{pgfscope}%
\begin{pgfscope}%
\pgfsetbuttcap%
\pgfsetroundjoin%
\definecolor{currentfill}{rgb}{0.000000,0.000000,0.000000}%
\pgfsetfillcolor{currentfill}%
\pgfsetlinewidth{0.803000pt}%
\definecolor{currentstroke}{rgb}{0.000000,0.000000,0.000000}%
\pgfsetstrokecolor{currentstroke}%
\pgfsetdash{}{0pt}%
\pgfsys@defobject{currentmarker}{\pgfqpoint{0.000000in}{-0.048611in}}{\pgfqpoint{0.000000in}{0.000000in}}{%
\pgfpathmoveto{\pgfqpoint{0.000000in}{0.000000in}}%
\pgfpathlineto{\pgfqpoint{0.000000in}{-0.048611in}}%
\pgfusepath{stroke,fill}%
}%
\begin{pgfscope}%
\pgfsys@transformshift{9.991698in}{4.908628in}%
\pgfsys@useobject{currentmarker}{}%
\end{pgfscope}%
\end{pgfscope}%
\begin{pgfscope}%
\pgftext[x=9.991698in,y=4.811405in,,top]{\rmfamily\fontsize{10.000000}{12.000000}\selectfont \(\displaystyle 1.0\)}%
\end{pgfscope}%
\begin{pgfscope}%
\pgfsetbuttcap%
\pgfsetroundjoin%
\definecolor{currentfill}{rgb}{0.000000,0.000000,0.000000}%
\pgfsetfillcolor{currentfill}%
\pgfsetlinewidth{0.803000pt}%
\definecolor{currentstroke}{rgb}{0.000000,0.000000,0.000000}%
\pgfsetstrokecolor{currentstroke}%
\pgfsetdash{}{0pt}%
\pgfsys@defobject{currentmarker}{\pgfqpoint{-0.048611in}{0.000000in}}{\pgfqpoint{0.000000in}{0.000000in}}{%
\pgfpathmoveto{\pgfqpoint{0.000000in}{0.000000in}}%
\pgfpathlineto{\pgfqpoint{-0.048611in}{0.000000in}}%
\pgfusepath{stroke,fill}%
}%
\begin{pgfscope}%
\pgfsys@transformshift{5.952727in}{5.064619in}%
\pgfsys@useobject{currentmarker}{}%
\end{pgfscope}%
\end{pgfscope}%
\begin{pgfscope}%
\pgftext[x=5.608591in,y=5.011858in,left,base]{\rmfamily\fontsize{10.000000}{12.000000}\selectfont \(\displaystyle 0.00\)}%
\end{pgfscope}%
\begin{pgfscope}%
\pgfsetbuttcap%
\pgfsetroundjoin%
\definecolor{currentfill}{rgb}{0.000000,0.000000,0.000000}%
\pgfsetfillcolor{currentfill}%
\pgfsetlinewidth{0.803000pt}%
\definecolor{currentstroke}{rgb}{0.000000,0.000000,0.000000}%
\pgfsetstrokecolor{currentstroke}%
\pgfsetdash{}{0pt}%
\pgfsys@defobject{currentmarker}{\pgfqpoint{-0.048611in}{0.000000in}}{\pgfqpoint{0.000000in}{0.000000in}}{%
\pgfpathmoveto{\pgfqpoint{0.000000in}{0.000000in}}%
\pgfpathlineto{\pgfqpoint{-0.048611in}{0.000000in}}%
\pgfusepath{stroke,fill}%
}%
\begin{pgfscope}%
\pgfsys@transformshift{5.952727in}{5.620090in}%
\pgfsys@useobject{currentmarker}{}%
\end{pgfscope}%
\end{pgfscope}%
\begin{pgfscope}%
\pgftext[x=5.608591in,y=5.567329in,left,base]{\rmfamily\fontsize{10.000000}{12.000000}\selectfont \(\displaystyle 0.05\)}%
\end{pgfscope}%
\begin{pgfscope}%
\pgfsetbuttcap%
\pgfsetroundjoin%
\definecolor{currentfill}{rgb}{0.000000,0.000000,0.000000}%
\pgfsetfillcolor{currentfill}%
\pgfsetlinewidth{0.803000pt}%
\definecolor{currentstroke}{rgb}{0.000000,0.000000,0.000000}%
\pgfsetstrokecolor{currentstroke}%
\pgfsetdash{}{0pt}%
\pgfsys@defobject{currentmarker}{\pgfqpoint{-0.048611in}{0.000000in}}{\pgfqpoint{0.000000in}{0.000000in}}{%
\pgfpathmoveto{\pgfqpoint{0.000000in}{0.000000in}}%
\pgfpathlineto{\pgfqpoint{-0.048611in}{0.000000in}}%
\pgfusepath{stroke,fill}%
}%
\begin{pgfscope}%
\pgfsys@transformshift{5.952727in}{6.175561in}%
\pgfsys@useobject{currentmarker}{}%
\end{pgfscope}%
\end{pgfscope}%
\begin{pgfscope}%
\pgftext[x=5.608591in,y=6.122799in,left,base]{\rmfamily\fontsize{10.000000}{12.000000}\selectfont \(\displaystyle 0.10\)}%
\end{pgfscope}%
\begin{pgfscope}%
\pgfsetbuttcap%
\pgfsetroundjoin%
\definecolor{currentfill}{rgb}{0.000000,0.000000,0.000000}%
\pgfsetfillcolor{currentfill}%
\pgfsetlinewidth{0.803000pt}%
\definecolor{currentstroke}{rgb}{0.000000,0.000000,0.000000}%
\pgfsetstrokecolor{currentstroke}%
\pgfsetdash{}{0pt}%
\pgfsys@defobject{currentmarker}{\pgfqpoint{-0.048611in}{0.000000in}}{\pgfqpoint{0.000000in}{0.000000in}}{%
\pgfpathmoveto{\pgfqpoint{0.000000in}{0.000000in}}%
\pgfpathlineto{\pgfqpoint{-0.048611in}{0.000000in}}%
\pgfusepath{stroke,fill}%
}%
\begin{pgfscope}%
\pgfsys@transformshift{5.952727in}{6.731031in}%
\pgfsys@useobject{currentmarker}{}%
\end{pgfscope}%
\end{pgfscope}%
\begin{pgfscope}%
\pgftext[x=5.608591in,y=6.678270in,left,base]{\rmfamily\fontsize{10.000000}{12.000000}\selectfont \(\displaystyle 0.15\)}%
\end{pgfscope}%
\begin{pgfscope}%
\pgfsetbuttcap%
\pgfsetroundjoin%
\definecolor{currentfill}{rgb}{0.000000,0.000000,0.000000}%
\pgfsetfillcolor{currentfill}%
\pgfsetlinewidth{0.803000pt}%
\definecolor{currentstroke}{rgb}{0.000000,0.000000,0.000000}%
\pgfsetstrokecolor{currentstroke}%
\pgfsetdash{}{0pt}%
\pgfsys@defobject{currentmarker}{\pgfqpoint{-0.048611in}{0.000000in}}{\pgfqpoint{0.000000in}{0.000000in}}{%
\pgfpathmoveto{\pgfqpoint{0.000000in}{0.000000in}}%
\pgfpathlineto{\pgfqpoint{-0.048611in}{0.000000in}}%
\pgfusepath{stroke,fill}%
}%
\begin{pgfscope}%
\pgfsys@transformshift{5.952727in}{7.286502in}%
\pgfsys@useobject{currentmarker}{}%
\end{pgfscope}%
\end{pgfscope}%
\begin{pgfscope}%
\pgftext[x=5.608591in,y=7.233741in,left,base]{\rmfamily\fontsize{10.000000}{12.000000}\selectfont \(\displaystyle 0.20\)}%
\end{pgfscope}%
\begin{pgfscope}%
\pgfsetbuttcap%
\pgfsetroundjoin%
\definecolor{currentfill}{rgb}{0.000000,0.000000,0.000000}%
\pgfsetfillcolor{currentfill}%
\pgfsetlinewidth{0.803000pt}%
\definecolor{currentstroke}{rgb}{0.000000,0.000000,0.000000}%
\pgfsetstrokecolor{currentstroke}%
\pgfsetdash{}{0pt}%
\pgfsys@defobject{currentmarker}{\pgfqpoint{-0.048611in}{0.000000in}}{\pgfqpoint{0.000000in}{0.000000in}}{%
\pgfpathmoveto{\pgfqpoint{0.000000in}{0.000000in}}%
\pgfpathlineto{\pgfqpoint{-0.048611in}{0.000000in}}%
\pgfusepath{stroke,fill}%
}%
\begin{pgfscope}%
\pgfsys@transformshift{5.952727in}{7.841973in}%
\pgfsys@useobject{currentmarker}{}%
\end{pgfscope}%
\end{pgfscope}%
\begin{pgfscope}%
\pgftext[x=5.608591in,y=7.789211in,left,base]{\rmfamily\fontsize{10.000000}{12.000000}\selectfont \(\displaystyle 0.25\)}%
\end{pgfscope}%
\begin{pgfscope}%
\pgfpathrectangle{\pgfqpoint{5.952727in}{4.908628in}}{\pgfqpoint{4.227273in}{3.431818in}} %
\pgfusepath{clip}%
\pgfsetbuttcap%
\pgfsetroundjoin%
\pgfsetlinewidth{1.505625pt}%
\definecolor{currentstroke}{rgb}{1.000000,0.000000,0.000000}%
\pgfsetstrokecolor{currentstroke}%
\pgfsetdash{{5.550000pt}{2.400000pt}}{0.000000pt}%
\pgfpathmoveto{\pgfqpoint{6.144876in}{5.064619in}}%
\pgfpathlineto{\pgfqpoint{6.225659in}{5.293051in}}%
\pgfpathlineto{\pgfqpoint{6.302596in}{5.501673in}}%
\pgfpathlineto{\pgfqpoint{6.379532in}{5.701737in}}%
\pgfpathlineto{\pgfqpoint{6.456469in}{5.893380in}}%
\pgfpathlineto{\pgfqpoint{6.533405in}{6.076731in}}%
\pgfpathlineto{\pgfqpoint{6.610341in}{6.251905in}}%
\pgfpathlineto{\pgfqpoint{6.683431in}{6.410841in}}%
\pgfpathlineto{\pgfqpoint{6.756521in}{6.562576in}}%
\pgfpathlineto{\pgfqpoint{6.829610in}{6.707185in}}%
\pgfpathlineto{\pgfqpoint{6.902700in}{6.844739in}}%
\pgfpathlineto{\pgfqpoint{6.971943in}{6.968603in}}%
\pgfpathlineto{\pgfqpoint{7.041186in}{7.086242in}}%
\pgfpathlineto{\pgfqpoint{7.110428in}{7.197705in}}%
\pgfpathlineto{\pgfqpoint{7.175824in}{7.297341in}}%
\pgfpathlineto{\pgfqpoint{7.241220in}{7.391539in}}%
\pgfpathlineto{\pgfqpoint{7.306616in}{7.480331in}}%
\pgfpathlineto{\pgfqpoint{7.372012in}{7.563745in}}%
\pgfpathlineto{\pgfqpoint{7.433561in}{7.637363in}}%
\pgfpathlineto{\pgfqpoint{7.495111in}{7.706263in}}%
\pgfpathlineto{\pgfqpoint{7.556660in}{7.770463in}}%
\pgfpathlineto{\pgfqpoint{7.614362in}{7.826398in}}%
\pgfpathlineto{\pgfqpoint{7.672064in}{7.878233in}}%
\pgfpathlineto{\pgfqpoint{7.729767in}{7.925980in}}%
\pgfpathlineto{\pgfqpoint{7.787469in}{7.969652in}}%
\pgfpathlineto{\pgfqpoint{7.845171in}{8.009259in}}%
\pgfpathlineto{\pgfqpoint{7.899027in}{8.042568in}}%
\pgfpathlineto{\pgfqpoint{7.952882in}{8.072352in}}%
\pgfpathlineto{\pgfqpoint{8.006738in}{8.098619in}}%
\pgfpathlineto{\pgfqpoint{8.060593in}{8.121376in}}%
\pgfpathlineto{\pgfqpoint{8.114449in}{8.140627in}}%
\pgfpathlineto{\pgfqpoint{8.164458in}{8.155370in}}%
\pgfpathlineto{\pgfqpoint{8.214466in}{8.167097in}}%
\pgfpathlineto{\pgfqpoint{8.264475in}{8.175813in}}%
\pgfpathlineto{\pgfqpoint{8.314484in}{8.181519in}}%
\pgfpathlineto{\pgfqpoint{8.364492in}{8.184217in}}%
\pgfpathlineto{\pgfqpoint{8.414501in}{8.183908in}}%
\pgfpathlineto{\pgfqpoint{8.464510in}{8.180593in}}%
\pgfpathlineto{\pgfqpoint{8.514518in}{8.174271in}}%
\pgfpathlineto{\pgfqpoint{8.564527in}{8.164941in}}%
\pgfpathlineto{\pgfqpoint{8.614536in}{8.152601in}}%
\pgfpathlineto{\pgfqpoint{8.664544in}{8.137249in}}%
\pgfpathlineto{\pgfqpoint{8.718400in}{8.117344in}}%
\pgfpathlineto{\pgfqpoint{8.772255in}{8.093936in}}%
\pgfpathlineto{\pgfqpoint{8.826111in}{8.067020in}}%
\pgfpathlineto{\pgfqpoint{8.879966in}{8.036587in}}%
\pgfpathlineto{\pgfqpoint{8.933822in}{8.002629in}}%
\pgfpathlineto{\pgfqpoint{8.987678in}{7.965136in}}%
\pgfpathlineto{\pgfqpoint{9.045380in}{7.921029in}}%
\pgfpathlineto{\pgfqpoint{9.103082in}{7.872836in}}%
\pgfpathlineto{\pgfqpoint{9.160785in}{7.820540in}}%
\pgfpathlineto{\pgfqpoint{9.218487in}{7.764122in}}%
\pgfpathlineto{\pgfqpoint{9.276189in}{7.703562in}}%
\pgfpathlineto{\pgfqpoint{9.337738in}{7.634375in}}%
\pgfpathlineto{\pgfqpoint{9.399287in}{7.560422in}}%
\pgfpathlineto{\pgfqpoint{9.460837in}{7.481673in}}%
\pgfpathlineto{\pgfqpoint{9.522386in}{7.398096in}}%
\pgfpathlineto{\pgfqpoint{9.587782in}{7.303968in}}%
\pgfpathlineto{\pgfqpoint{9.653178in}{7.204309in}}%
\pgfpathlineto{\pgfqpoint{9.718574in}{7.099076in}}%
\pgfpathlineto{\pgfqpoint{9.783970in}{6.988223in}}%
\pgfpathlineto{\pgfqpoint{9.853212in}{6.864673in}}%
\pgfpathlineto{\pgfqpoint{9.922455in}{6.734713in}}%
\pgfpathlineto{\pgfqpoint{9.987851in}{6.606037in}}%
\pgfpathlineto{\pgfqpoint{9.987851in}{6.606037in}}%
\pgfusepath{stroke}%
\end{pgfscope}%
\begin{pgfscope}%
\pgfpathrectangle{\pgfqpoint{5.952727in}{4.908628in}}{\pgfqpoint{4.227273in}{3.431818in}} %
\pgfusepath{clip}%
\pgfsetbuttcap%
\pgfsetmiterjoin%
\definecolor{currentfill}{rgb}{1.000000,0.000000,0.000000}%
\pgfsetfillcolor{currentfill}%
\pgfsetlinewidth{1.003750pt}%
\definecolor{currentstroke}{rgb}{1.000000,0.000000,0.000000}%
\pgfsetstrokecolor{currentstroke}%
\pgfsetdash{}{0pt}%
\pgfsys@defobject{currentmarker}{\pgfqpoint{-0.041667in}{-0.041667in}}{\pgfqpoint{0.041667in}{0.041667in}}{%
\pgfpathmoveto{\pgfqpoint{-0.041667in}{-0.041667in}}%
\pgfpathlineto{\pgfqpoint{0.041667in}{-0.041667in}}%
\pgfpathlineto{\pgfqpoint{0.041667in}{0.041667in}}%
\pgfpathlineto{\pgfqpoint{-0.041667in}{0.041667in}}%
\pgfpathclose%
\pgfusepath{stroke,fill}%
}%
\begin{pgfscope}%
\pgfsys@transformshift{6.144876in}{5.064619in}%
\pgfsys@useobject{currentmarker}{}%
\end{pgfscope}%
\begin{pgfscope}%
\pgfsys@transformshift{6.529558in}{6.067759in}%
\pgfsys@useobject{currentmarker}{}%
\end{pgfscope}%
\begin{pgfscope}%
\pgfsys@transformshift{6.914240in}{6.865817in}%
\pgfsys@useobject{currentmarker}{}%
\end{pgfscope}%
\begin{pgfscope}%
\pgfsys@transformshift{7.298923in}{7.470165in}%
\pgfsys@useobject{currentmarker}{}%
\end{pgfscope}%
\begin{pgfscope}%
\pgfsys@transformshift{7.683605in}{7.888109in}%
\pgfsys@useobject{currentmarker}{}%
\end{pgfscope}%
\begin{pgfscope}%
\pgfsys@transformshift{8.068287in}{8.124341in}%
\pgfsys@useobject{currentmarker}{}%
\end{pgfscope}%
\begin{pgfscope}%
\pgfsys@transformshift{8.452969in}{8.181625in}%
\pgfsys@useobject{currentmarker}{}%
\end{pgfscope}%
\begin{pgfscope}%
\pgfsys@transformshift{8.837651in}{8.060795in}%
\pgfsys@useobject{currentmarker}{}%
\end{pgfscope}%
\begin{pgfscope}%
\pgfsys@transformshift{9.222334in}{7.760214in}%
\pgfsys@useobject{currentmarker}{}%
\end{pgfscope}%
\begin{pgfscope}%
\pgfsys@transformshift{9.607016in}{7.275232in}%
\pgfsys@useobject{currentmarker}{}%
\end{pgfscope}%
\end{pgfscope}%
\begin{pgfscope}%
\pgfpathrectangle{\pgfqpoint{5.952727in}{4.908628in}}{\pgfqpoint{4.227273in}{3.431818in}} %
\pgfusepath{clip}%
\pgfsetrectcap%
\pgfsetroundjoin%
\pgfsetlinewidth{1.505625pt}%
\definecolor{currentstroke}{rgb}{0.000000,0.000000,1.000000}%
\pgfsetstrokecolor{currentstroke}%
\pgfsetdash{}{0pt}%
\pgfpathmoveto{\pgfqpoint{6.144876in}{5.064619in}}%
\pgfpathlineto{\pgfqpoint{6.225659in}{5.293021in}}%
\pgfpathlineto{\pgfqpoint{6.302596in}{5.501455in}}%
\pgfpathlineto{\pgfqpoint{6.379532in}{5.701037in}}%
\pgfpathlineto{\pgfqpoint{6.456469in}{5.891781in}}%
\pgfpathlineto{\pgfqpoint{6.529558in}{6.064816in}}%
\pgfpathlineto{\pgfqpoint{6.602648in}{6.229901in}}%
\pgfpathlineto{\pgfqpoint{6.671891in}{6.378975in}}%
\pgfpathlineto{\pgfqpoint{6.741133in}{6.520934in}}%
\pgfpathlineto{\pgfqpoint{6.810376in}{6.655787in}}%
\pgfpathlineto{\pgfqpoint{6.875772in}{6.776630in}}%
\pgfpathlineto{\pgfqpoint{6.941168in}{6.891148in}}%
\pgfpathlineto{\pgfqpoint{7.006564in}{6.999348in}}%
\pgfpathlineto{\pgfqpoint{7.068113in}{7.095417in}}%
\pgfpathlineto{\pgfqpoint{7.129662in}{7.185900in}}%
\pgfpathlineto{\pgfqpoint{7.191212in}{7.270799in}}%
\pgfpathlineto{\pgfqpoint{7.248914in}{7.345327in}}%
\pgfpathlineto{\pgfqpoint{7.306616in}{7.414954in}}%
\pgfpathlineto{\pgfqpoint{7.364319in}{7.479685in}}%
\pgfpathlineto{\pgfqpoint{7.418174in}{7.535685in}}%
\pgfpathlineto{\pgfqpoint{7.472030in}{7.587423in}}%
\pgfpathlineto{\pgfqpoint{7.525885in}{7.634903in}}%
\pgfpathlineto{\pgfqpoint{7.579741in}{7.678124in}}%
\pgfpathlineto{\pgfqpoint{7.629749in}{7.714448in}}%
\pgfpathlineto{\pgfqpoint{7.679758in}{7.747103in}}%
\pgfpathlineto{\pgfqpoint{7.729767in}{7.776091in}}%
\pgfpathlineto{\pgfqpoint{7.779775in}{7.801412in}}%
\pgfpathlineto{\pgfqpoint{7.829784in}{7.823067in}}%
\pgfpathlineto{\pgfqpoint{7.875946in}{7.839804in}}%
\pgfpathlineto{\pgfqpoint{7.922108in}{7.853417in}}%
\pgfpathlineto{\pgfqpoint{7.968270in}{7.863909in}}%
\pgfpathlineto{\pgfqpoint{8.014432in}{7.871279in}}%
\pgfpathlineto{\pgfqpoint{8.060593in}{7.875527in}}%
\pgfpathlineto{\pgfqpoint{8.106755in}{7.876654in}}%
\pgfpathlineto{\pgfqpoint{8.152917in}{7.874659in}}%
\pgfpathlineto{\pgfqpoint{8.199079in}{7.869543in}}%
\pgfpathlineto{\pgfqpoint{8.245241in}{7.861305in}}%
\pgfpathlineto{\pgfqpoint{8.291403in}{7.849945in}}%
\pgfpathlineto{\pgfqpoint{8.337565in}{7.835463in}}%
\pgfpathlineto{\pgfqpoint{8.383726in}{7.817858in}}%
\pgfpathlineto{\pgfqpoint{8.429888in}{7.797130in}}%
\pgfpathlineto{\pgfqpoint{8.479897in}{7.771150in}}%
\pgfpathlineto{\pgfqpoint{8.529906in}{7.741503in}}%
\pgfpathlineto{\pgfqpoint{8.579914in}{7.708189in}}%
\pgfpathlineto{\pgfqpoint{8.629923in}{7.671206in}}%
\pgfpathlineto{\pgfqpoint{8.679932in}{7.630552in}}%
\pgfpathlineto{\pgfqpoint{8.733787in}{7.582666in}}%
\pgfpathlineto{\pgfqpoint{8.787643in}{7.530521in}}%
\pgfpathlineto{\pgfqpoint{8.841498in}{7.474114in}}%
\pgfpathlineto{\pgfqpoint{8.895354in}{7.413444in}}%
\pgfpathlineto{\pgfqpoint{8.953056in}{7.343706in}}%
\pgfpathlineto{\pgfqpoint{9.010758in}{7.269069in}}%
\pgfpathlineto{\pgfqpoint{9.068461in}{7.189529in}}%
\pgfpathlineto{\pgfqpoint{9.130010in}{7.099278in}}%
\pgfpathlineto{\pgfqpoint{9.191559in}{7.003441in}}%
\pgfpathlineto{\pgfqpoint{9.253108in}{6.902012in}}%
\pgfpathlineto{\pgfqpoint{9.318504in}{6.788112in}}%
\pgfpathlineto{\pgfqpoint{9.383900in}{6.667888in}}%
\pgfpathlineto{\pgfqpoint{9.449296in}{6.541333in}}%
\pgfpathlineto{\pgfqpoint{9.518539in}{6.400427in}}%
\pgfpathlineto{\pgfqpoint{9.587782in}{6.252406in}}%
\pgfpathlineto{\pgfqpoint{9.657025in}{6.097262in}}%
\pgfpathlineto{\pgfqpoint{9.730114in}{5.925760in}}%
\pgfpathlineto{\pgfqpoint{9.803204in}{5.746299in}}%
\pgfpathlineto{\pgfqpoint{9.876293in}{5.558866in}}%
\pgfpathlineto{\pgfqpoint{9.953230in}{5.352940in}}%
\pgfpathlineto{\pgfqpoint{9.987851in}{5.257382in}}%
\pgfpathlineto{\pgfqpoint{9.987851in}{5.257382in}}%
\pgfusepath{stroke}%
\end{pgfscope}%
\begin{pgfscope}%
\pgfpathrectangle{\pgfqpoint{5.952727in}{4.908628in}}{\pgfqpoint{4.227273in}{3.431818in}} %
\pgfusepath{clip}%
\pgfsetbuttcap%
\pgfsetroundjoin%
\definecolor{currentfill}{rgb}{0.000000,0.000000,1.000000}%
\pgfsetfillcolor{currentfill}%
\pgfsetlinewidth{1.003750pt}%
\definecolor{currentstroke}{rgb}{0.000000,0.000000,1.000000}%
\pgfsetstrokecolor{currentstroke}%
\pgfsetdash{}{0pt}%
\pgfsys@defobject{currentmarker}{\pgfqpoint{-0.041667in}{-0.041667in}}{\pgfqpoint{0.041667in}{0.041667in}}{%
\pgfpathmoveto{\pgfqpoint{0.000000in}{-0.041667in}}%
\pgfpathcurveto{\pgfqpoint{0.011050in}{-0.041667in}}{\pgfqpoint{0.021649in}{-0.037276in}}{\pgfqpoint{0.029463in}{-0.029463in}}%
\pgfpathcurveto{\pgfqpoint{0.037276in}{-0.021649in}}{\pgfqpoint{0.041667in}{-0.011050in}}{\pgfqpoint{0.041667in}{0.000000in}}%
\pgfpathcurveto{\pgfqpoint{0.041667in}{0.011050in}}{\pgfqpoint{0.037276in}{0.021649in}}{\pgfqpoint{0.029463in}{0.029463in}}%
\pgfpathcurveto{\pgfqpoint{0.021649in}{0.037276in}}{\pgfqpoint{0.011050in}{0.041667in}}{\pgfqpoint{0.000000in}{0.041667in}}%
\pgfpathcurveto{\pgfqpoint{-0.011050in}{0.041667in}}{\pgfqpoint{-0.021649in}{0.037276in}}{\pgfqpoint{-0.029463in}{0.029463in}}%
\pgfpathcurveto{\pgfqpoint{-0.037276in}{0.021649in}}{\pgfqpoint{-0.041667in}{0.011050in}}{\pgfqpoint{-0.041667in}{0.000000in}}%
\pgfpathcurveto{\pgfqpoint{-0.041667in}{-0.011050in}}{\pgfqpoint{-0.037276in}{-0.021649in}}{\pgfqpoint{-0.029463in}{-0.029463in}}%
\pgfpathcurveto{\pgfqpoint{-0.021649in}{-0.037276in}}{\pgfqpoint{-0.011050in}{-0.041667in}}{\pgfqpoint{0.000000in}{-0.041667in}}%
\pgfpathclose%
\pgfusepath{stroke,fill}%
}%
\begin{pgfscope}%
\pgfsys@transformshift{6.144876in}{5.064619in}%
\pgfsys@useobject{currentmarker}{}%
\end{pgfscope}%
\begin{pgfscope}%
\pgfsys@transformshift{6.529558in}{6.064816in}%
\pgfsys@useobject{currentmarker}{}%
\end{pgfscope}%
\begin{pgfscope}%
\pgfsys@transformshift{6.914240in}{6.844759in}%
\pgfsys@useobject{currentmarker}{}%
\end{pgfscope}%
\begin{pgfscope}%
\pgfsys@transformshift{7.298923in}{7.405953in}%
\pgfsys@useobject{currentmarker}{}%
\end{pgfscope}%
\begin{pgfscope}%
\pgfsys@transformshift{7.683605in}{7.749463in}%
\pgfsys@useobject{currentmarker}{}%
\end{pgfscope}%
\begin{pgfscope}%
\pgfsys@transformshift{8.068287in}{7.875932in}%
\pgfsys@useobject{currentmarker}{}%
\end{pgfscope}%
\begin{pgfscope}%
\pgfsys@transformshift{8.452969in}{7.785595in}%
\pgfsys@useobject{currentmarker}{}%
\end{pgfscope}%
\begin{pgfscope}%
\pgfsys@transformshift{8.837651in}{7.478285in}%
\pgfsys@useobject{currentmarker}{}%
\end{pgfscope}%
\begin{pgfscope}%
\pgfsys@transformshift{9.222334in}{6.953426in}%
\pgfsys@useobject{currentmarker}{}%
\end{pgfscope}%
\begin{pgfscope}%
\pgfsys@transformshift{9.607016in}{6.210025in}%
\pgfsys@useobject{currentmarker}{}%
\end{pgfscope}%
\end{pgfscope}%
\begin{pgfscope}%
\pgfpathrectangle{\pgfqpoint{5.952727in}{4.908628in}}{\pgfqpoint{4.227273in}{3.431818in}} %
\pgfusepath{clip}%
\pgfsetbuttcap%
\pgfsetroundjoin%
\pgfsetlinewidth{1.505625pt}%
\definecolor{currentstroke}{rgb}{0.000000,0.750000,0.750000}%
\pgfsetstrokecolor{currentstroke}%
\pgfsetdash{{9.600000pt}{2.400000pt}{1.500000pt}{2.400000pt}}{0.000000pt}%
\pgfpathmoveto{\pgfqpoint{6.144876in}{5.064619in}}%
\pgfpathlineto{\pgfqpoint{6.225659in}{5.293018in}}%
\pgfpathlineto{\pgfqpoint{6.302596in}{5.501433in}}%
\pgfpathlineto{\pgfqpoint{6.379532in}{5.700964in}}%
\pgfpathlineto{\pgfqpoint{6.452622in}{5.882290in}}%
\pgfpathlineto{\pgfqpoint{6.525711in}{6.055602in}}%
\pgfpathlineto{\pgfqpoint{6.598801in}{6.220900in}}%
\pgfpathlineto{\pgfqpoint{6.668044in}{6.370107in}}%
\pgfpathlineto{\pgfqpoint{6.737287in}{6.512123in}}%
\pgfpathlineto{\pgfqpoint{6.806529in}{6.646950in}}%
\pgfpathlineto{\pgfqpoint{6.871925in}{6.767685in}}%
\pgfpathlineto{\pgfqpoint{6.937321in}{6.882010in}}%
\pgfpathlineto{\pgfqpoint{7.002717in}{6.989923in}}%
\pgfpathlineto{\pgfqpoint{7.064266in}{7.085633in}}%
\pgfpathlineto{\pgfqpoint{7.125816in}{7.175665in}}%
\pgfpathlineto{\pgfqpoint{7.187365in}{7.260020in}}%
\pgfpathlineto{\pgfqpoint{7.245067in}{7.333947in}}%
\pgfpathlineto{\pgfqpoint{7.302769in}{7.402885in}}%
\pgfpathlineto{\pgfqpoint{7.360472in}{7.466835in}}%
\pgfpathlineto{\pgfqpoint{7.414327in}{7.522020in}}%
\pgfpathlineto{\pgfqpoint{7.468183in}{7.572860in}}%
\pgfpathlineto{\pgfqpoint{7.522038in}{7.619355in}}%
\pgfpathlineto{\pgfqpoint{7.572047in}{7.658638in}}%
\pgfpathlineto{\pgfqpoint{7.622056in}{7.694175in}}%
\pgfpathlineto{\pgfqpoint{7.672064in}{7.725967in}}%
\pgfpathlineto{\pgfqpoint{7.722073in}{7.754012in}}%
\pgfpathlineto{\pgfqpoint{7.772082in}{7.778311in}}%
\pgfpathlineto{\pgfqpoint{7.818244in}{7.797416in}}%
\pgfpathlineto{\pgfqpoint{7.864405in}{7.813329in}}%
\pgfpathlineto{\pgfqpoint{7.910567in}{7.826051in}}%
\pgfpathlineto{\pgfqpoint{7.956729in}{7.835582in}}%
\pgfpathlineto{\pgfqpoint{8.002891in}{7.841920in}}%
\pgfpathlineto{\pgfqpoint{8.049053in}{7.845067in}}%
\pgfpathlineto{\pgfqpoint{8.095215in}{7.845023in}}%
\pgfpathlineto{\pgfqpoint{8.141377in}{7.841787in}}%
\pgfpathlineto{\pgfqpoint{8.187539in}{7.835360in}}%
\pgfpathlineto{\pgfqpoint{8.233700in}{7.825741in}}%
\pgfpathlineto{\pgfqpoint{8.279862in}{7.812930in}}%
\pgfpathlineto{\pgfqpoint{8.326024in}{7.796928in}}%
\pgfpathlineto{\pgfqpoint{8.372186in}{7.777734in}}%
\pgfpathlineto{\pgfqpoint{8.418348in}{7.755349in}}%
\pgfpathlineto{\pgfqpoint{8.468357in}{7.727496in}}%
\pgfpathlineto{\pgfqpoint{8.518365in}{7.695897in}}%
\pgfpathlineto{\pgfqpoint{8.568374in}{7.660552in}}%
\pgfpathlineto{\pgfqpoint{8.618383in}{7.621461in}}%
\pgfpathlineto{\pgfqpoint{8.668391in}{7.578623in}}%
\pgfpathlineto{\pgfqpoint{8.722247in}{7.528300in}}%
\pgfpathlineto{\pgfqpoint{8.776102in}{7.473632in}}%
\pgfpathlineto{\pgfqpoint{8.829958in}{7.414619in}}%
\pgfpathlineto{\pgfqpoint{8.887660in}{7.346568in}}%
\pgfpathlineto{\pgfqpoint{8.945362in}{7.273527in}}%
\pgfpathlineto{\pgfqpoint{9.003065in}{7.195498in}}%
\pgfpathlineto{\pgfqpoint{9.060767in}{7.112479in}}%
\pgfpathlineto{\pgfqpoint{9.122316in}{7.018425in}}%
\pgfpathlineto{\pgfqpoint{9.183865in}{6.918693in}}%
\pgfpathlineto{\pgfqpoint{9.245415in}{6.813282in}}%
\pgfpathlineto{\pgfqpoint{9.310811in}{6.695061in}}%
\pgfpathlineto{\pgfqpoint{9.376207in}{6.570428in}}%
\pgfpathlineto{\pgfqpoint{9.441603in}{6.439382in}}%
\pgfpathlineto{\pgfqpoint{9.510845in}{6.293638in}}%
\pgfpathlineto{\pgfqpoint{9.580088in}{6.140702in}}%
\pgfpathlineto{\pgfqpoint{9.653178in}{5.971468in}}%
\pgfpathlineto{\pgfqpoint{9.726267in}{5.794219in}}%
\pgfpathlineto{\pgfqpoint{9.799357in}{5.608955in}}%
\pgfpathlineto{\pgfqpoint{9.876293in}{5.405278in}}%
\pgfpathlineto{\pgfqpoint{9.953230in}{5.192717in}}%
\pgfpathlineto{\pgfqpoint{9.987851in}{5.094165in}}%
\pgfpathlineto{\pgfqpoint{9.987851in}{5.094165in}}%
\pgfusepath{stroke}%
\end{pgfscope}%
\begin{pgfscope}%
\pgfpathrectangle{\pgfqpoint{5.952727in}{4.908628in}}{\pgfqpoint{4.227273in}{3.431818in}} %
\pgfusepath{clip}%
\pgfsetbuttcap%
\pgfsetmiterjoin%
\definecolor{currentfill}{rgb}{0.000000,0.750000,0.750000}%
\pgfsetfillcolor{currentfill}%
\pgfsetlinewidth{1.003750pt}%
\definecolor{currentstroke}{rgb}{0.000000,0.750000,0.750000}%
\pgfsetstrokecolor{currentstroke}%
\pgfsetdash{}{0pt}%
\pgfsys@defobject{currentmarker}{\pgfqpoint{-0.041667in}{-0.041667in}}{\pgfqpoint{0.041667in}{0.041667in}}{%
\pgfpathmoveto{\pgfqpoint{-0.000000in}{-0.041667in}}%
\pgfpathlineto{\pgfqpoint{0.041667in}{0.041667in}}%
\pgfpathlineto{\pgfqpoint{-0.041667in}{0.041667in}}%
\pgfpathclose%
\pgfusepath{stroke,fill}%
}%
\begin{pgfscope}%
\pgfsys@transformshift{6.144876in}{5.064619in}%
\pgfsys@useobject{currentmarker}{}%
\end{pgfscope}%
\begin{pgfscope}%
\pgfsys@transformshift{6.529558in}{6.064502in}%
\pgfsys@useobject{currentmarker}{}%
\end{pgfscope}%
\begin{pgfscope}%
\pgfsys@transformshift{6.914240in}{6.842392in}%
\pgfsys@useobject{currentmarker}{}%
\end{pgfscope}%
\begin{pgfscope}%
\pgfsys@transformshift{7.298923in}{7.398445in}%
\pgfsys@useobject{currentmarker}{}%
\end{pgfscope}%
\begin{pgfscope}%
\pgfsys@transformshift{7.683605in}{7.732771in}%
\pgfsys@useobject{currentmarker}{}%
\end{pgfscope}%
\begin{pgfscope}%
\pgfsys@transformshift{8.068287in}{7.845437in}%
\pgfsys@useobject{currentmarker}{}%
\end{pgfscope}%
\begin{pgfscope}%
\pgfsys@transformshift{8.452969in}{7.736465in}%
\pgfsys@useobject{currentmarker}{}%
\end{pgfscope}%
\begin{pgfscope}%
\pgfsys@transformshift{8.837651in}{7.405834in}%
\pgfsys@useobject{currentmarker}{}%
\end{pgfscope}%
\begin{pgfscope}%
\pgfsys@transformshift{9.222334in}{6.853477in}%
\pgfsys@useobject{currentmarker}{}%
\end{pgfscope}%
\begin{pgfscope}%
\pgfsys@transformshift{9.607016in}{6.079285in}%
\pgfsys@useobject{currentmarker}{}%
\end{pgfscope}%
\end{pgfscope}%
\begin{pgfscope}%
\pgfpathrectangle{\pgfqpoint{5.952727in}{4.908628in}}{\pgfqpoint{4.227273in}{3.431818in}} %
\pgfusepath{clip}%
\pgfsetbuttcap%
\pgfsetroundjoin%
\pgfsetlinewidth{1.505625pt}%
\definecolor{currentstroke}{rgb}{0.000000,0.000000,0.000000}%
\pgfsetstrokecolor{currentstroke}%
\pgfsetdash{{1.500000pt}{2.475000pt}}{0.000000pt}%
\pgfpathmoveto{\pgfqpoint{6.144876in}{5.064619in}}%
\pgfpathlineto{\pgfqpoint{6.225659in}{5.293018in}}%
\pgfpathlineto{\pgfqpoint{6.302596in}{5.501431in}}%
\pgfpathlineto{\pgfqpoint{6.379532in}{5.700956in}}%
\pgfpathlineto{\pgfqpoint{6.452622in}{5.882274in}}%
\pgfpathlineto{\pgfqpoint{6.525711in}{6.055571in}}%
\pgfpathlineto{\pgfqpoint{6.598801in}{6.220848in}}%
\pgfpathlineto{\pgfqpoint{6.668044in}{6.370029in}}%
\pgfpathlineto{\pgfqpoint{6.737287in}{6.512011in}}%
\pgfpathlineto{\pgfqpoint{6.806529in}{6.646795in}}%
\pgfpathlineto{\pgfqpoint{6.871925in}{6.767482in}}%
\pgfpathlineto{\pgfqpoint{6.937321in}{6.881749in}}%
\pgfpathlineto{\pgfqpoint{7.002717in}{6.989595in}}%
\pgfpathlineto{\pgfqpoint{7.064266in}{7.085233in}}%
\pgfpathlineto{\pgfqpoint{7.125816in}{7.175184in}}%
\pgfpathlineto{\pgfqpoint{7.187365in}{7.259448in}}%
\pgfpathlineto{\pgfqpoint{7.245067in}{7.333280in}}%
\pgfpathlineto{\pgfqpoint{7.302769in}{7.402115in}}%
\pgfpathlineto{\pgfqpoint{7.360472in}{7.465951in}}%
\pgfpathlineto{\pgfqpoint{7.414327in}{7.521022in}}%
\pgfpathlineto{\pgfqpoint{7.468183in}{7.571739in}}%
\pgfpathlineto{\pgfqpoint{7.522038in}{7.618102in}}%
\pgfpathlineto{\pgfqpoint{7.572047in}{7.657255in}}%
\pgfpathlineto{\pgfqpoint{7.622056in}{7.692654in}}%
\pgfpathlineto{\pgfqpoint{7.672064in}{7.724299in}}%
\pgfpathlineto{\pgfqpoint{7.722073in}{7.752189in}}%
\pgfpathlineto{\pgfqpoint{7.772082in}{7.776326in}}%
\pgfpathlineto{\pgfqpoint{7.818244in}{7.795274in}}%
\pgfpathlineto{\pgfqpoint{7.864405in}{7.811023in}}%
\pgfpathlineto{\pgfqpoint{7.910567in}{7.823574in}}%
\pgfpathlineto{\pgfqpoint{7.956729in}{7.832926in}}%
\pgfpathlineto{\pgfqpoint{8.002891in}{7.839079in}}%
\pgfpathlineto{\pgfqpoint{8.049053in}{7.842033in}}%
\pgfpathlineto{\pgfqpoint{8.095215in}{7.841789in}}%
\pgfpathlineto{\pgfqpoint{8.141377in}{7.838346in}}%
\pgfpathlineto{\pgfqpoint{8.187539in}{7.831704in}}%
\pgfpathlineto{\pgfqpoint{8.233700in}{7.821863in}}%
\pgfpathlineto{\pgfqpoint{8.279862in}{7.808824in}}%
\pgfpathlineto{\pgfqpoint{8.326024in}{7.792586in}}%
\pgfpathlineto{\pgfqpoint{8.372186in}{7.773150in}}%
\pgfpathlineto{\pgfqpoint{8.418348in}{7.750514in}}%
\pgfpathlineto{\pgfqpoint{8.468357in}{7.722383in}}%
\pgfpathlineto{\pgfqpoint{8.518365in}{7.690497in}}%
\pgfpathlineto{\pgfqpoint{8.568374in}{7.654858in}}%
\pgfpathlineto{\pgfqpoint{8.618383in}{7.615464in}}%
\pgfpathlineto{\pgfqpoint{8.668391in}{7.572317in}}%
\pgfpathlineto{\pgfqpoint{8.722247in}{7.521651in}}%
\pgfpathlineto{\pgfqpoint{8.776102in}{7.466632in}}%
\pgfpathlineto{\pgfqpoint{8.829958in}{7.407259in}}%
\pgfpathlineto{\pgfqpoint{8.887660in}{7.338813in}}%
\pgfpathlineto{\pgfqpoint{8.945362in}{7.265370in}}%
\pgfpathlineto{\pgfqpoint{9.003065in}{7.186928in}}%
\pgfpathlineto{\pgfqpoint{9.060767in}{7.103487in}}%
\pgfpathlineto{\pgfqpoint{9.122316in}{7.008975in}}%
\pgfpathlineto{\pgfqpoint{9.183865in}{6.908776in}}%
\pgfpathlineto{\pgfqpoint{9.245415in}{6.802889in}}%
\pgfpathlineto{\pgfqpoint{9.310811in}{6.684153in}}%
\pgfpathlineto{\pgfqpoint{9.376207in}{6.558997in}}%
\pgfpathlineto{\pgfqpoint{9.445449in}{6.419481in}}%
\pgfpathlineto{\pgfqpoint{9.514692in}{6.272767in}}%
\pgfpathlineto{\pgfqpoint{9.583935in}{6.118855in}}%
\pgfpathlineto{\pgfqpoint{9.657025in}{5.948583in}}%
\pgfpathlineto{\pgfqpoint{9.730114in}{5.770290in}}%
\pgfpathlineto{\pgfqpoint{9.803204in}{5.583977in}}%
\pgfpathlineto{\pgfqpoint{9.880140in}{5.379193in}}%
\pgfpathlineto{\pgfqpoint{9.957077in}{5.165522in}}%
\pgfpathlineto{\pgfqpoint{9.987851in}{5.077565in}}%
\pgfpathlineto{\pgfqpoint{9.987851in}{5.077565in}}%
\pgfusepath{stroke}%
\end{pgfscope}%
\begin{pgfscope}%
\pgfpathrectangle{\pgfqpoint{5.952727in}{4.908628in}}{\pgfqpoint{4.227273in}{3.431818in}} %
\pgfusepath{clip}%
\pgfsetbuttcap%
\pgfsetroundjoin%
\definecolor{currentfill}{rgb}{0.000000,0.000000,0.000000}%
\pgfsetfillcolor{currentfill}%
\pgfsetlinewidth{1.003750pt}%
\definecolor{currentstroke}{rgb}{0.000000,0.000000,0.000000}%
\pgfsetstrokecolor{currentstroke}%
\pgfsetdash{}{0pt}%
\pgfsys@defobject{currentmarker}{\pgfqpoint{-0.041667in}{-0.041667in}}{\pgfqpoint{0.041667in}{0.041667in}}{%
\pgfpathmoveto{\pgfqpoint{-0.041667in}{0.000000in}}%
\pgfpathlineto{\pgfqpoint{0.041667in}{0.000000in}}%
\pgfpathmoveto{\pgfqpoint{0.000000in}{-0.041667in}}%
\pgfpathlineto{\pgfqpoint{0.000000in}{0.041667in}}%
\pgfusepath{stroke,fill}%
}%
\begin{pgfscope}%
\pgfsys@transformshift{6.144876in}{5.064619in}%
\pgfsys@useobject{currentmarker}{}%
\end{pgfscope}%
\begin{pgfscope}%
\pgfsys@transformshift{6.529558in}{6.064470in}%
\pgfsys@useobject{currentmarker}{}%
\end{pgfscope}%
\begin{pgfscope}%
\pgfsys@transformshift{6.914240in}{6.842152in}%
\pgfsys@useobject{currentmarker}{}%
\end{pgfscope}%
\begin{pgfscope}%
\pgfsys@transformshift{7.298923in}{7.397681in}%
\pgfsys@useobject{currentmarker}{}%
\end{pgfscope}%
\begin{pgfscope}%
\pgfsys@transformshift{7.683605in}{7.731068in}%
\pgfsys@useobject{currentmarker}{}%
\end{pgfscope}%
\begin{pgfscope}%
\pgfsys@transformshift{8.068287in}{7.842320in}%
\pgfsys@useobject{currentmarker}{}%
\end{pgfscope}%
\begin{pgfscope}%
\pgfsys@transformshift{8.452969in}{7.731439in}%
\pgfsys@useobject{currentmarker}{}%
\end{pgfscope}%
\begin{pgfscope}%
\pgfsys@transformshift{8.837651in}{7.398422in}%
\pgfsys@useobject{currentmarker}{}%
\end{pgfscope}%
\begin{pgfscope}%
\pgfsys@transformshift{9.222334in}{6.843263in}%
\pgfsys@useobject{currentmarker}{}%
\end{pgfscope}%
\begin{pgfscope}%
\pgfsys@transformshift{9.607016in}{6.065951in}%
\pgfsys@useobject{currentmarker}{}%
\end{pgfscope}%
\end{pgfscope}%
\begin{pgfscope}%
\pgfsetrectcap%
\pgfsetmiterjoin%
\pgfsetlinewidth{0.803000pt}%
\definecolor{currentstroke}{rgb}{0.000000,0.000000,0.000000}%
\pgfsetstrokecolor{currentstroke}%
\pgfsetdash{}{0pt}%
\pgfpathmoveto{\pgfqpoint{5.952727in}{4.908628in}}%
\pgfpathlineto{\pgfqpoint{5.952727in}{8.340446in}}%
\pgfusepath{stroke}%
\end{pgfscope}%
\begin{pgfscope}%
\pgfsetrectcap%
\pgfsetmiterjoin%
\pgfsetlinewidth{0.803000pt}%
\definecolor{currentstroke}{rgb}{0.000000,0.000000,0.000000}%
\pgfsetstrokecolor{currentstroke}%
\pgfsetdash{}{0pt}%
\pgfpathmoveto{\pgfqpoint{10.180000in}{4.908628in}}%
\pgfpathlineto{\pgfqpoint{10.180000in}{8.340446in}}%
\pgfusepath{stroke}%
\end{pgfscope}%
\begin{pgfscope}%
\pgfsetrectcap%
\pgfsetmiterjoin%
\pgfsetlinewidth{0.803000pt}%
\definecolor{currentstroke}{rgb}{0.000000,0.000000,0.000000}%
\pgfsetstrokecolor{currentstroke}%
\pgfsetdash{}{0pt}%
\pgfpathmoveto{\pgfqpoint{5.952727in}{4.908628in}}%
\pgfpathlineto{\pgfqpoint{10.180000in}{4.908628in}}%
\pgfusepath{stroke}%
\end{pgfscope}%
\begin{pgfscope}%
\pgfsetrectcap%
\pgfsetmiterjoin%
\pgfsetlinewidth{0.803000pt}%
\definecolor{currentstroke}{rgb}{0.000000,0.000000,0.000000}%
\pgfsetstrokecolor{currentstroke}%
\pgfsetdash{}{0pt}%
\pgfpathmoveto{\pgfqpoint{5.952727in}{8.340446in}}%
\pgfpathlineto{\pgfqpoint{10.180000in}{8.340446in}}%
\pgfusepath{stroke}%
\end{pgfscope}%
\begin{pgfscope}%
\pgfsetbuttcap%
\pgfsetmiterjoin%
\definecolor{currentfill}{rgb}{1.000000,1.000000,1.000000}%
\pgfsetfillcolor{currentfill}%
\pgfsetfillopacity{0.800000}%
\pgfsetlinewidth{1.003750pt}%
\definecolor{currentstroke}{rgb}{0.800000,0.800000,0.800000}%
\pgfsetstrokecolor{currentstroke}%
\pgfsetstrokeopacity{0.800000}%
\pgfsetdash{}{0pt}%
\pgfpathmoveto{\pgfqpoint{7.514813in}{4.978072in}}%
\pgfpathlineto{\pgfqpoint{8.617914in}{4.978072in}}%
\pgfpathquadraticcurveto{\pgfqpoint{8.645692in}{4.978072in}}{\pgfqpoint{8.645692in}{5.005850in}}%
\pgfpathlineto{\pgfqpoint{8.645692in}{6.011247in}}%
\pgfpathquadraticcurveto{\pgfqpoint{8.645692in}{6.039025in}}{\pgfqpoint{8.617914in}{6.039025in}}%
\pgfpathlineto{\pgfqpoint{7.514813in}{6.039025in}}%
\pgfpathquadraticcurveto{\pgfqpoint{7.487035in}{6.039025in}}{\pgfqpoint{7.487035in}{6.011247in}}%
\pgfpathlineto{\pgfqpoint{7.487035in}{5.005850in}}%
\pgfpathquadraticcurveto{\pgfqpoint{7.487035in}{4.978072in}}{\pgfqpoint{7.514813in}{4.978072in}}%
\pgfpathclose%
\pgfusepath{stroke,fill}%
\end{pgfscope}%
\begin{pgfscope}%
\pgftext[x=7.875165in,y=5.877946in,left,base]{\rmfamily\fontsize{10.000000}{12.000000}\selectfont \(\displaystyle \alpha\) = 1}%
\end{pgfscope}%
\begin{pgfscope}%
\pgfsetbuttcap%
\pgfsetroundjoin%
\pgfsetlinewidth{1.505625pt}%
\definecolor{currentstroke}{rgb}{1.000000,0.000000,0.000000}%
\pgfsetstrokecolor{currentstroke}%
\pgfsetdash{{5.550000pt}{2.400000pt}}{0.000000pt}%
\pgfpathmoveto{\pgfqpoint{7.542591in}{5.722700in}}%
\pgfpathlineto{\pgfqpoint{7.820368in}{5.722700in}}%
\pgfusepath{stroke}%
\end{pgfscope}%
\begin{pgfscope}%
\pgfsetbuttcap%
\pgfsetmiterjoin%
\definecolor{currentfill}{rgb}{1.000000,0.000000,0.000000}%
\pgfsetfillcolor{currentfill}%
\pgfsetlinewidth{1.003750pt}%
\definecolor{currentstroke}{rgb}{1.000000,0.000000,0.000000}%
\pgfsetstrokecolor{currentstroke}%
\pgfsetdash{}{0pt}%
\pgfsys@defobject{currentmarker}{\pgfqpoint{-0.041667in}{-0.041667in}}{\pgfqpoint{0.041667in}{0.041667in}}{%
\pgfpathmoveto{\pgfqpoint{-0.041667in}{-0.041667in}}%
\pgfpathlineto{\pgfqpoint{0.041667in}{-0.041667in}}%
\pgfpathlineto{\pgfqpoint{0.041667in}{0.041667in}}%
\pgfpathlineto{\pgfqpoint{-0.041667in}{0.041667in}}%
\pgfpathclose%
\pgfusepath{stroke,fill}%
}%
\begin{pgfscope}%
\pgfsys@transformshift{7.681480in}{5.722700in}%
\pgfsys@useobject{currentmarker}{}%
\end{pgfscope}%
\end{pgfscope}%
\begin{pgfscope}%
\pgftext[x=7.931480in,y=5.674089in,left,base]{\rmfamily\fontsize{10.000000}{12.000000}\selectfont \(\displaystyle \epsilon\) = 1}%
\end{pgfscope}%
\begin{pgfscope}%
\pgfsetrectcap%
\pgfsetroundjoin%
\pgfsetlinewidth{1.505625pt}%
\definecolor{currentstroke}{rgb}{0.000000,0.000000,1.000000}%
\pgfsetstrokecolor{currentstroke}%
\pgfsetdash{}{0pt}%
\pgfpathmoveto{\pgfqpoint{7.542591in}{5.518843in}}%
\pgfpathlineto{\pgfqpoint{7.820368in}{5.518843in}}%
\pgfusepath{stroke}%
\end{pgfscope}%
\begin{pgfscope}%
\pgfsetbuttcap%
\pgfsetroundjoin%
\definecolor{currentfill}{rgb}{0.000000,0.000000,1.000000}%
\pgfsetfillcolor{currentfill}%
\pgfsetlinewidth{1.003750pt}%
\definecolor{currentstroke}{rgb}{0.000000,0.000000,1.000000}%
\pgfsetstrokecolor{currentstroke}%
\pgfsetdash{}{0pt}%
\pgfsys@defobject{currentmarker}{\pgfqpoint{-0.041667in}{-0.041667in}}{\pgfqpoint{0.041667in}{0.041667in}}{%
\pgfpathmoveto{\pgfqpoint{0.000000in}{-0.041667in}}%
\pgfpathcurveto{\pgfqpoint{0.011050in}{-0.041667in}}{\pgfqpoint{0.021649in}{-0.037276in}}{\pgfqpoint{0.029463in}{-0.029463in}}%
\pgfpathcurveto{\pgfqpoint{0.037276in}{-0.021649in}}{\pgfqpoint{0.041667in}{-0.011050in}}{\pgfqpoint{0.041667in}{0.000000in}}%
\pgfpathcurveto{\pgfqpoint{0.041667in}{0.011050in}}{\pgfqpoint{0.037276in}{0.021649in}}{\pgfqpoint{0.029463in}{0.029463in}}%
\pgfpathcurveto{\pgfqpoint{0.021649in}{0.037276in}}{\pgfqpoint{0.011050in}{0.041667in}}{\pgfqpoint{0.000000in}{0.041667in}}%
\pgfpathcurveto{\pgfqpoint{-0.011050in}{0.041667in}}{\pgfqpoint{-0.021649in}{0.037276in}}{\pgfqpoint{-0.029463in}{0.029463in}}%
\pgfpathcurveto{\pgfqpoint{-0.037276in}{0.021649in}}{\pgfqpoint{-0.041667in}{0.011050in}}{\pgfqpoint{-0.041667in}{0.000000in}}%
\pgfpathcurveto{\pgfqpoint{-0.041667in}{-0.011050in}}{\pgfqpoint{-0.037276in}{-0.021649in}}{\pgfqpoint{-0.029463in}{-0.029463in}}%
\pgfpathcurveto{\pgfqpoint{-0.021649in}{-0.037276in}}{\pgfqpoint{-0.011050in}{-0.041667in}}{\pgfqpoint{0.000000in}{-0.041667in}}%
\pgfpathclose%
\pgfusepath{stroke,fill}%
}%
\begin{pgfscope}%
\pgfsys@transformshift{7.681480in}{5.518843in}%
\pgfsys@useobject{currentmarker}{}%
\end{pgfscope}%
\end{pgfscope}%
\begin{pgfscope}%
\pgftext[x=7.931480in,y=5.470232in,left,base]{\rmfamily\fontsize{10.000000}{12.000000}\selectfont \(\displaystyle \epsilon\) = 0.1}%
\end{pgfscope}%
\begin{pgfscope}%
\pgfsetbuttcap%
\pgfsetroundjoin%
\pgfsetlinewidth{1.505625pt}%
\definecolor{currentstroke}{rgb}{0.000000,0.750000,0.750000}%
\pgfsetstrokecolor{currentstroke}%
\pgfsetdash{{9.600000pt}{2.400000pt}{1.500000pt}{2.400000pt}}{0.000000pt}%
\pgfpathmoveto{\pgfqpoint{7.542591in}{5.314986in}}%
\pgfpathlineto{\pgfqpoint{7.820368in}{5.314986in}}%
\pgfusepath{stroke}%
\end{pgfscope}%
\begin{pgfscope}%
\pgfsetbuttcap%
\pgfsetmiterjoin%
\definecolor{currentfill}{rgb}{0.000000,0.750000,0.750000}%
\pgfsetfillcolor{currentfill}%
\pgfsetlinewidth{1.003750pt}%
\definecolor{currentstroke}{rgb}{0.000000,0.750000,0.750000}%
\pgfsetstrokecolor{currentstroke}%
\pgfsetdash{}{0pt}%
\pgfsys@defobject{currentmarker}{\pgfqpoint{-0.041667in}{-0.041667in}}{\pgfqpoint{0.041667in}{0.041667in}}{%
\pgfpathmoveto{\pgfqpoint{-0.000000in}{-0.041667in}}%
\pgfpathlineto{\pgfqpoint{0.041667in}{0.041667in}}%
\pgfpathlineto{\pgfqpoint{-0.041667in}{0.041667in}}%
\pgfpathclose%
\pgfusepath{stroke,fill}%
}%
\begin{pgfscope}%
\pgfsys@transformshift{7.681480in}{5.314986in}%
\pgfsys@useobject{currentmarker}{}%
\end{pgfscope}%
\end{pgfscope}%
\begin{pgfscope}%
\pgftext[x=7.931480in,y=5.266375in,left,base]{\rmfamily\fontsize{10.000000}{12.000000}\selectfont \(\displaystyle \epsilon\) = 0.01}%
\end{pgfscope}%
\begin{pgfscope}%
\pgfsetbuttcap%
\pgfsetroundjoin%
\pgfsetlinewidth{1.505625pt}%
\definecolor{currentstroke}{rgb}{0.000000,0.000000,0.000000}%
\pgfsetstrokecolor{currentstroke}%
\pgfsetdash{{1.500000pt}{2.475000pt}}{0.000000pt}%
\pgfpathmoveto{\pgfqpoint{7.542591in}{5.111128in}}%
\pgfpathlineto{\pgfqpoint{7.820368in}{5.111128in}}%
\pgfusepath{stroke}%
\end{pgfscope}%
\begin{pgfscope}%
\pgfsetbuttcap%
\pgfsetroundjoin%
\definecolor{currentfill}{rgb}{0.000000,0.000000,0.000000}%
\pgfsetfillcolor{currentfill}%
\pgfsetlinewidth{1.003750pt}%
\definecolor{currentstroke}{rgb}{0.000000,0.000000,0.000000}%
\pgfsetstrokecolor{currentstroke}%
\pgfsetdash{}{0pt}%
\pgfsys@defobject{currentmarker}{\pgfqpoint{-0.041667in}{-0.041667in}}{\pgfqpoint{0.041667in}{0.041667in}}{%
\pgfpathmoveto{\pgfqpoint{-0.041667in}{0.000000in}}%
\pgfpathlineto{\pgfqpoint{0.041667in}{0.000000in}}%
\pgfpathmoveto{\pgfqpoint{0.000000in}{-0.041667in}}%
\pgfpathlineto{\pgfqpoint{0.000000in}{0.041667in}}%
\pgfusepath{stroke,fill}%
}%
\begin{pgfscope}%
\pgfsys@transformshift{7.681480in}{5.111128in}%
\pgfsys@useobject{currentmarker}{}%
\end{pgfscope}%
\end{pgfscope}%
\begin{pgfscope}%
\pgftext[x=7.931480in,y=5.062517in,left,base]{\rmfamily\fontsize{10.000000}{12.000000}\selectfont \(\displaystyle \epsilon\) = 0.001}%
\end{pgfscope}%
\begin{pgfscope}%
\pgfsetbuttcap%
\pgfsetmiterjoin%
\definecolor{currentfill}{rgb}{1.000000,1.000000,1.000000}%
\pgfsetfillcolor{currentfill}%
\pgfsetlinewidth{0.000000pt}%
\definecolor{currentstroke}{rgb}{0.000000,0.000000,0.000000}%
\pgfsetstrokecolor{currentstroke}%
\pgfsetstrokeopacity{0.000000}%
\pgfsetdash{}{0pt}%
\pgfpathmoveto{\pgfqpoint{0.880000in}{0.790446in}}%
\pgfpathlineto{\pgfqpoint{5.107273in}{0.790446in}}%
\pgfpathlineto{\pgfqpoint{5.107273in}{4.222264in}}%
\pgfpathlineto{\pgfqpoint{0.880000in}{4.222264in}}%
\pgfpathclose%
\pgfusepath{fill}%
\end{pgfscope}%
\begin{pgfscope}%
\pgfsetbuttcap%
\pgfsetroundjoin%
\definecolor{currentfill}{rgb}{0.000000,0.000000,0.000000}%
\pgfsetfillcolor{currentfill}%
\pgfsetlinewidth{0.803000pt}%
\definecolor{currentstroke}{rgb}{0.000000,0.000000,0.000000}%
\pgfsetstrokecolor{currentstroke}%
\pgfsetdash{}{0pt}%
\pgfsys@defobject{currentmarker}{\pgfqpoint{0.000000in}{-0.048611in}}{\pgfqpoint{0.000000in}{0.000000in}}{%
\pgfpathmoveto{\pgfqpoint{0.000000in}{0.000000in}}%
\pgfpathlineto{\pgfqpoint{0.000000in}{-0.048611in}}%
\pgfusepath{stroke,fill}%
}%
\begin{pgfscope}%
\pgfsys@transformshift{1.072149in}{0.790446in}%
\pgfsys@useobject{currentmarker}{}%
\end{pgfscope}%
\end{pgfscope}%
\begin{pgfscope}%
\pgftext[x=1.072149in,y=0.693224in,,top]{\rmfamily\fontsize{10.000000}{12.000000}\selectfont \(\displaystyle 0.0\)}%
\end{pgfscope}%
\begin{pgfscope}%
\pgfsetbuttcap%
\pgfsetroundjoin%
\definecolor{currentfill}{rgb}{0.000000,0.000000,0.000000}%
\pgfsetfillcolor{currentfill}%
\pgfsetlinewidth{0.803000pt}%
\definecolor{currentstroke}{rgb}{0.000000,0.000000,0.000000}%
\pgfsetstrokecolor{currentstroke}%
\pgfsetdash{}{0pt}%
\pgfsys@defobject{currentmarker}{\pgfqpoint{0.000000in}{-0.048611in}}{\pgfqpoint{0.000000in}{0.000000in}}{%
\pgfpathmoveto{\pgfqpoint{0.000000in}{0.000000in}}%
\pgfpathlineto{\pgfqpoint{0.000000in}{-0.048611in}}%
\pgfusepath{stroke,fill}%
}%
\begin{pgfscope}%
\pgfsys@transformshift{1.841513in}{0.790446in}%
\pgfsys@useobject{currentmarker}{}%
\end{pgfscope}%
\end{pgfscope}%
\begin{pgfscope}%
\pgftext[x=1.841513in,y=0.693224in,,top]{\rmfamily\fontsize{10.000000}{12.000000}\selectfont \(\displaystyle 0.2\)}%
\end{pgfscope}%
\begin{pgfscope}%
\pgfsetbuttcap%
\pgfsetroundjoin%
\definecolor{currentfill}{rgb}{0.000000,0.000000,0.000000}%
\pgfsetfillcolor{currentfill}%
\pgfsetlinewidth{0.803000pt}%
\definecolor{currentstroke}{rgb}{0.000000,0.000000,0.000000}%
\pgfsetstrokecolor{currentstroke}%
\pgfsetdash{}{0pt}%
\pgfsys@defobject{currentmarker}{\pgfqpoint{0.000000in}{-0.048611in}}{\pgfqpoint{0.000000in}{0.000000in}}{%
\pgfpathmoveto{\pgfqpoint{0.000000in}{0.000000in}}%
\pgfpathlineto{\pgfqpoint{0.000000in}{-0.048611in}}%
\pgfusepath{stroke,fill}%
}%
\begin{pgfscope}%
\pgfsys@transformshift{2.610878in}{0.790446in}%
\pgfsys@useobject{currentmarker}{}%
\end{pgfscope}%
\end{pgfscope}%
\begin{pgfscope}%
\pgftext[x=2.610878in,y=0.693224in,,top]{\rmfamily\fontsize{10.000000}{12.000000}\selectfont \(\displaystyle 0.4\)}%
\end{pgfscope}%
\begin{pgfscope}%
\pgfsetbuttcap%
\pgfsetroundjoin%
\definecolor{currentfill}{rgb}{0.000000,0.000000,0.000000}%
\pgfsetfillcolor{currentfill}%
\pgfsetlinewidth{0.803000pt}%
\definecolor{currentstroke}{rgb}{0.000000,0.000000,0.000000}%
\pgfsetstrokecolor{currentstroke}%
\pgfsetdash{}{0pt}%
\pgfsys@defobject{currentmarker}{\pgfqpoint{0.000000in}{-0.048611in}}{\pgfqpoint{0.000000in}{0.000000in}}{%
\pgfpathmoveto{\pgfqpoint{0.000000in}{0.000000in}}%
\pgfpathlineto{\pgfqpoint{0.000000in}{-0.048611in}}%
\pgfusepath{stroke,fill}%
}%
\begin{pgfscope}%
\pgfsys@transformshift{3.380242in}{0.790446in}%
\pgfsys@useobject{currentmarker}{}%
\end{pgfscope}%
\end{pgfscope}%
\begin{pgfscope}%
\pgftext[x=3.380242in,y=0.693224in,,top]{\rmfamily\fontsize{10.000000}{12.000000}\selectfont \(\displaystyle 0.6\)}%
\end{pgfscope}%
\begin{pgfscope}%
\pgfsetbuttcap%
\pgfsetroundjoin%
\definecolor{currentfill}{rgb}{0.000000,0.000000,0.000000}%
\pgfsetfillcolor{currentfill}%
\pgfsetlinewidth{0.803000pt}%
\definecolor{currentstroke}{rgb}{0.000000,0.000000,0.000000}%
\pgfsetstrokecolor{currentstroke}%
\pgfsetdash{}{0pt}%
\pgfsys@defobject{currentmarker}{\pgfqpoint{0.000000in}{-0.048611in}}{\pgfqpoint{0.000000in}{0.000000in}}{%
\pgfpathmoveto{\pgfqpoint{0.000000in}{0.000000in}}%
\pgfpathlineto{\pgfqpoint{0.000000in}{-0.048611in}}%
\pgfusepath{stroke,fill}%
}%
\begin{pgfscope}%
\pgfsys@transformshift{4.149606in}{0.790446in}%
\pgfsys@useobject{currentmarker}{}%
\end{pgfscope}%
\end{pgfscope}%
\begin{pgfscope}%
\pgftext[x=4.149606in,y=0.693224in,,top]{\rmfamily\fontsize{10.000000}{12.000000}\selectfont \(\displaystyle 0.8\)}%
\end{pgfscope}%
\begin{pgfscope}%
\pgfsetbuttcap%
\pgfsetroundjoin%
\definecolor{currentfill}{rgb}{0.000000,0.000000,0.000000}%
\pgfsetfillcolor{currentfill}%
\pgfsetlinewidth{0.803000pt}%
\definecolor{currentstroke}{rgb}{0.000000,0.000000,0.000000}%
\pgfsetstrokecolor{currentstroke}%
\pgfsetdash{}{0pt}%
\pgfsys@defobject{currentmarker}{\pgfqpoint{0.000000in}{-0.048611in}}{\pgfqpoint{0.000000in}{0.000000in}}{%
\pgfpathmoveto{\pgfqpoint{0.000000in}{0.000000in}}%
\pgfpathlineto{\pgfqpoint{0.000000in}{-0.048611in}}%
\pgfusepath{stroke,fill}%
}%
\begin{pgfscope}%
\pgfsys@transformshift{4.918971in}{0.790446in}%
\pgfsys@useobject{currentmarker}{}%
\end{pgfscope}%
\end{pgfscope}%
\begin{pgfscope}%
\pgftext[x=4.918971in,y=0.693224in,,top]{\rmfamily\fontsize{10.000000}{12.000000}\selectfont \(\displaystyle 1.0\)}%
\end{pgfscope}%
\begin{pgfscope}%
\pgfsetbuttcap%
\pgfsetroundjoin%
\definecolor{currentfill}{rgb}{0.000000,0.000000,0.000000}%
\pgfsetfillcolor{currentfill}%
\pgfsetlinewidth{0.803000pt}%
\definecolor{currentstroke}{rgb}{0.000000,0.000000,0.000000}%
\pgfsetstrokecolor{currentstroke}%
\pgfsetdash{}{0pt}%
\pgfsys@defobject{currentmarker}{\pgfqpoint{-0.048611in}{0.000000in}}{\pgfqpoint{0.000000in}{0.000000in}}{%
\pgfpathmoveto{\pgfqpoint{0.000000in}{0.000000in}}%
\pgfpathlineto{\pgfqpoint{-0.048611in}{0.000000in}}%
\pgfusepath{stroke,fill}%
}%
\begin{pgfscope}%
\pgfsys@transformshift{0.880000in}{0.946438in}%
\pgfsys@useobject{currentmarker}{}%
\end{pgfscope}%
\end{pgfscope}%
\begin{pgfscope}%
\pgftext[x=0.535863in,y=0.893676in,left,base]{\rmfamily\fontsize{10.000000}{12.000000}\selectfont \(\displaystyle 0.00\)}%
\end{pgfscope}%
\begin{pgfscope}%
\pgfsetbuttcap%
\pgfsetroundjoin%
\definecolor{currentfill}{rgb}{0.000000,0.000000,0.000000}%
\pgfsetfillcolor{currentfill}%
\pgfsetlinewidth{0.803000pt}%
\definecolor{currentstroke}{rgb}{0.000000,0.000000,0.000000}%
\pgfsetstrokecolor{currentstroke}%
\pgfsetdash{}{0pt}%
\pgfsys@defobject{currentmarker}{\pgfqpoint{-0.048611in}{0.000000in}}{\pgfqpoint{0.000000in}{0.000000in}}{%
\pgfpathmoveto{\pgfqpoint{0.000000in}{0.000000in}}%
\pgfpathlineto{\pgfqpoint{-0.048611in}{0.000000in}}%
\pgfusepath{stroke,fill}%
}%
\begin{pgfscope}%
\pgfsys@transformshift{0.880000in}{1.372706in}%
\pgfsys@useobject{currentmarker}{}%
\end{pgfscope}%
\end{pgfscope}%
\begin{pgfscope}%
\pgftext[x=0.535863in,y=1.319944in,left,base]{\rmfamily\fontsize{10.000000}{12.000000}\selectfont \(\displaystyle 0.05\)}%
\end{pgfscope}%
\begin{pgfscope}%
\pgfsetbuttcap%
\pgfsetroundjoin%
\definecolor{currentfill}{rgb}{0.000000,0.000000,0.000000}%
\pgfsetfillcolor{currentfill}%
\pgfsetlinewidth{0.803000pt}%
\definecolor{currentstroke}{rgb}{0.000000,0.000000,0.000000}%
\pgfsetstrokecolor{currentstroke}%
\pgfsetdash{}{0pt}%
\pgfsys@defobject{currentmarker}{\pgfqpoint{-0.048611in}{0.000000in}}{\pgfqpoint{0.000000in}{0.000000in}}{%
\pgfpathmoveto{\pgfqpoint{0.000000in}{0.000000in}}%
\pgfpathlineto{\pgfqpoint{-0.048611in}{0.000000in}}%
\pgfusepath{stroke,fill}%
}%
\begin{pgfscope}%
\pgfsys@transformshift{0.880000in}{1.798974in}%
\pgfsys@useobject{currentmarker}{}%
\end{pgfscope}%
\end{pgfscope}%
\begin{pgfscope}%
\pgftext[x=0.535863in,y=1.746212in,left,base]{\rmfamily\fontsize{10.000000}{12.000000}\selectfont \(\displaystyle 0.10\)}%
\end{pgfscope}%
\begin{pgfscope}%
\pgfsetbuttcap%
\pgfsetroundjoin%
\definecolor{currentfill}{rgb}{0.000000,0.000000,0.000000}%
\pgfsetfillcolor{currentfill}%
\pgfsetlinewidth{0.803000pt}%
\definecolor{currentstroke}{rgb}{0.000000,0.000000,0.000000}%
\pgfsetstrokecolor{currentstroke}%
\pgfsetdash{}{0pt}%
\pgfsys@defobject{currentmarker}{\pgfqpoint{-0.048611in}{0.000000in}}{\pgfqpoint{0.000000in}{0.000000in}}{%
\pgfpathmoveto{\pgfqpoint{0.000000in}{0.000000in}}%
\pgfpathlineto{\pgfqpoint{-0.048611in}{0.000000in}}%
\pgfusepath{stroke,fill}%
}%
\begin{pgfscope}%
\pgfsys@transformshift{0.880000in}{2.225242in}%
\pgfsys@useobject{currentmarker}{}%
\end{pgfscope}%
\end{pgfscope}%
\begin{pgfscope}%
\pgftext[x=0.535863in,y=2.172481in,left,base]{\rmfamily\fontsize{10.000000}{12.000000}\selectfont \(\displaystyle 0.15\)}%
\end{pgfscope}%
\begin{pgfscope}%
\pgfsetbuttcap%
\pgfsetroundjoin%
\definecolor{currentfill}{rgb}{0.000000,0.000000,0.000000}%
\pgfsetfillcolor{currentfill}%
\pgfsetlinewidth{0.803000pt}%
\definecolor{currentstroke}{rgb}{0.000000,0.000000,0.000000}%
\pgfsetstrokecolor{currentstroke}%
\pgfsetdash{}{0pt}%
\pgfsys@defobject{currentmarker}{\pgfqpoint{-0.048611in}{0.000000in}}{\pgfqpoint{0.000000in}{0.000000in}}{%
\pgfpathmoveto{\pgfqpoint{0.000000in}{0.000000in}}%
\pgfpathlineto{\pgfqpoint{-0.048611in}{0.000000in}}%
\pgfusepath{stroke,fill}%
}%
\begin{pgfscope}%
\pgfsys@transformshift{0.880000in}{2.651510in}%
\pgfsys@useobject{currentmarker}{}%
\end{pgfscope}%
\end{pgfscope}%
\begin{pgfscope}%
\pgftext[x=0.535863in,y=2.598749in,left,base]{\rmfamily\fontsize{10.000000}{12.000000}\selectfont \(\displaystyle 0.20\)}%
\end{pgfscope}%
\begin{pgfscope}%
\pgfsetbuttcap%
\pgfsetroundjoin%
\definecolor{currentfill}{rgb}{0.000000,0.000000,0.000000}%
\pgfsetfillcolor{currentfill}%
\pgfsetlinewidth{0.803000pt}%
\definecolor{currentstroke}{rgb}{0.000000,0.000000,0.000000}%
\pgfsetstrokecolor{currentstroke}%
\pgfsetdash{}{0pt}%
\pgfsys@defobject{currentmarker}{\pgfqpoint{-0.048611in}{0.000000in}}{\pgfqpoint{0.000000in}{0.000000in}}{%
\pgfpathmoveto{\pgfqpoint{0.000000in}{0.000000in}}%
\pgfpathlineto{\pgfqpoint{-0.048611in}{0.000000in}}%
\pgfusepath{stroke,fill}%
}%
\begin{pgfscope}%
\pgfsys@transformshift{0.880000in}{3.077779in}%
\pgfsys@useobject{currentmarker}{}%
\end{pgfscope}%
\end{pgfscope}%
\begin{pgfscope}%
\pgftext[x=0.535863in,y=3.025017in,left,base]{\rmfamily\fontsize{10.000000}{12.000000}\selectfont \(\displaystyle 0.25\)}%
\end{pgfscope}%
\begin{pgfscope}%
\pgfsetbuttcap%
\pgfsetroundjoin%
\definecolor{currentfill}{rgb}{0.000000,0.000000,0.000000}%
\pgfsetfillcolor{currentfill}%
\pgfsetlinewidth{0.803000pt}%
\definecolor{currentstroke}{rgb}{0.000000,0.000000,0.000000}%
\pgfsetstrokecolor{currentstroke}%
\pgfsetdash{}{0pt}%
\pgfsys@defobject{currentmarker}{\pgfqpoint{-0.048611in}{0.000000in}}{\pgfqpoint{0.000000in}{0.000000in}}{%
\pgfpathmoveto{\pgfqpoint{0.000000in}{0.000000in}}%
\pgfpathlineto{\pgfqpoint{-0.048611in}{0.000000in}}%
\pgfusepath{stroke,fill}%
}%
\begin{pgfscope}%
\pgfsys@transformshift{0.880000in}{3.504047in}%
\pgfsys@useobject{currentmarker}{}%
\end{pgfscope}%
\end{pgfscope}%
\begin{pgfscope}%
\pgftext[x=0.535863in,y=3.451285in,left,base]{\rmfamily\fontsize{10.000000}{12.000000}\selectfont \(\displaystyle 0.30\)}%
\end{pgfscope}%
\begin{pgfscope}%
\pgfsetbuttcap%
\pgfsetroundjoin%
\definecolor{currentfill}{rgb}{0.000000,0.000000,0.000000}%
\pgfsetfillcolor{currentfill}%
\pgfsetlinewidth{0.803000pt}%
\definecolor{currentstroke}{rgb}{0.000000,0.000000,0.000000}%
\pgfsetstrokecolor{currentstroke}%
\pgfsetdash{}{0pt}%
\pgfsys@defobject{currentmarker}{\pgfqpoint{-0.048611in}{0.000000in}}{\pgfqpoint{0.000000in}{0.000000in}}{%
\pgfpathmoveto{\pgfqpoint{0.000000in}{0.000000in}}%
\pgfpathlineto{\pgfqpoint{-0.048611in}{0.000000in}}%
\pgfusepath{stroke,fill}%
}%
\begin{pgfscope}%
\pgfsys@transformshift{0.880000in}{3.930315in}%
\pgfsys@useobject{currentmarker}{}%
\end{pgfscope}%
\end{pgfscope}%
\begin{pgfscope}%
\pgftext[x=0.535863in,y=3.877553in,left,base]{\rmfamily\fontsize{10.000000}{12.000000}\selectfont \(\displaystyle 0.35\)}%
\end{pgfscope}%
\begin{pgfscope}%
\pgfpathrectangle{\pgfqpoint{0.880000in}{0.790446in}}{\pgfqpoint{4.227273in}{3.431818in}} %
\pgfusepath{clip}%
\pgfsetbuttcap%
\pgfsetroundjoin%
\pgfsetlinewidth{1.505625pt}%
\definecolor{currentstroke}{rgb}{1.000000,0.000000,0.000000}%
\pgfsetstrokecolor{currentstroke}%
\pgfsetdash{{5.550000pt}{2.400000pt}}{0.000000pt}%
\pgfpathmoveto{\pgfqpoint{1.072149in}{0.946438in}}%
\pgfpathlineto{\pgfqpoint{1.164472in}{1.147382in}}%
\pgfpathlineto{\pgfqpoint{1.256796in}{1.341072in}}%
\pgfpathlineto{\pgfqpoint{1.349120in}{1.527608in}}%
\pgfpathlineto{\pgfqpoint{1.441444in}{1.707080in}}%
\pgfpathlineto{\pgfqpoint{1.529921in}{1.872523in}}%
\pgfpathlineto{\pgfqpoint{1.618397in}{2.031620in}}%
\pgfpathlineto{\pgfqpoint{1.706874in}{2.184433in}}%
\pgfpathlineto{\pgfqpoint{1.795351in}{2.331015in}}%
\pgfpathlineto{\pgfqpoint{1.879981in}{2.465440in}}%
\pgfpathlineto{\pgfqpoint{1.964611in}{2.594251in}}%
\pgfpathlineto{\pgfqpoint{2.049242in}{2.717485in}}%
\pgfpathlineto{\pgfqpoint{2.133872in}{2.835175in}}%
\pgfpathlineto{\pgfqpoint{2.214655in}{2.942373in}}%
\pgfpathlineto{\pgfqpoint{2.295438in}{3.044573in}}%
\pgfpathlineto{\pgfqpoint{2.376221in}{3.141799in}}%
\pgfpathlineto{\pgfqpoint{2.457005in}{3.234071in}}%
\pgfpathlineto{\pgfqpoint{2.533941in}{3.317361in}}%
\pgfpathlineto{\pgfqpoint{2.610878in}{3.396193in}}%
\pgfpathlineto{\pgfqpoint{2.687814in}{3.470579in}}%
\pgfpathlineto{\pgfqpoint{2.764750in}{3.540535in}}%
\pgfpathlineto{\pgfqpoint{2.837840in}{3.602901in}}%
\pgfpathlineto{\pgfqpoint{2.910930in}{3.661290in}}%
\pgfpathlineto{\pgfqpoint{2.984019in}{3.715712in}}%
\pgfpathlineto{\pgfqpoint{3.057109in}{3.766177in}}%
\pgfpathlineto{\pgfqpoint{3.126352in}{3.810343in}}%
\pgfpathlineto{\pgfqpoint{3.195595in}{3.850973in}}%
\pgfpathlineto{\pgfqpoint{3.264837in}{3.888073in}}%
\pgfpathlineto{\pgfqpoint{3.334080in}{3.921650in}}%
\pgfpathlineto{\pgfqpoint{3.403323in}{3.951710in}}%
\pgfpathlineto{\pgfqpoint{3.472566in}{3.978260in}}%
\pgfpathlineto{\pgfqpoint{3.537962in}{4.000116in}}%
\pgfpathlineto{\pgfqpoint{3.603358in}{4.018851in}}%
\pgfpathlineto{\pgfqpoint{3.668754in}{4.034468in}}%
\pgfpathlineto{\pgfqpoint{3.734150in}{4.046972in}}%
\pgfpathlineto{\pgfqpoint{3.799546in}{4.056367in}}%
\pgfpathlineto{\pgfqpoint{3.864942in}{4.062656in}}%
\pgfpathlineto{\pgfqpoint{3.930338in}{4.065841in}}%
\pgfpathlineto{\pgfqpoint{3.995734in}{4.065926in}}%
\pgfpathlineto{\pgfqpoint{4.061129in}{4.062913in}}%
\pgfpathlineto{\pgfqpoint{4.126525in}{4.056804in}}%
\pgfpathlineto{\pgfqpoint{4.191921in}{4.047600in}}%
\pgfpathlineto{\pgfqpoint{4.257317in}{4.035300in}}%
\pgfpathlineto{\pgfqpoint{4.322713in}{4.019907in}}%
\pgfpathlineto{\pgfqpoint{4.388109in}{4.001418in}}%
\pgfpathlineto{\pgfqpoint{4.453505in}{3.979833in}}%
\pgfpathlineto{\pgfqpoint{4.518901in}{3.955148in}}%
\pgfpathlineto{\pgfqpoint{4.588144in}{3.925631in}}%
\pgfpathlineto{\pgfqpoint{4.657387in}{3.892632in}}%
\pgfpathlineto{\pgfqpoint{4.726630in}{3.856145in}}%
\pgfpathlineto{\pgfqpoint{4.795872in}{3.816164in}}%
\pgfpathlineto{\pgfqpoint{4.865115in}{3.772680in}}%
\pgfpathlineto{\pgfqpoint{4.915124in}{3.739091in}}%
\pgfpathlineto{\pgfqpoint{4.915124in}{3.739091in}}%
\pgfusepath{stroke}%
\end{pgfscope}%
\begin{pgfscope}%
\pgfpathrectangle{\pgfqpoint{0.880000in}{0.790446in}}{\pgfqpoint{4.227273in}{3.431818in}} %
\pgfusepath{clip}%
\pgfsetbuttcap%
\pgfsetmiterjoin%
\definecolor{currentfill}{rgb}{1.000000,0.000000,0.000000}%
\pgfsetfillcolor{currentfill}%
\pgfsetlinewidth{1.003750pt}%
\definecolor{currentstroke}{rgb}{1.000000,0.000000,0.000000}%
\pgfsetstrokecolor{currentstroke}%
\pgfsetdash{}{0pt}%
\pgfsys@defobject{currentmarker}{\pgfqpoint{-0.041667in}{-0.041667in}}{\pgfqpoint{0.041667in}{0.041667in}}{%
\pgfpathmoveto{\pgfqpoint{-0.041667in}{-0.041667in}}%
\pgfpathlineto{\pgfqpoint{0.041667in}{-0.041667in}}%
\pgfpathlineto{\pgfqpoint{0.041667in}{0.041667in}}%
\pgfpathlineto{\pgfqpoint{-0.041667in}{0.041667in}}%
\pgfpathclose%
\pgfusepath{stroke,fill}%
}%
\begin{pgfscope}%
\pgfsys@transformshift{1.072149in}{0.946438in}%
\pgfsys@useobject{currentmarker}{}%
\end{pgfscope}%
\begin{pgfscope}%
\pgfsys@transformshift{1.456831in}{1.736311in}%
\pgfsys@useobject{currentmarker}{}%
\end{pgfscope}%
\begin{pgfscope}%
\pgfsys@transformshift{1.841513in}{2.405036in}%
\pgfsys@useobject{currentmarker}{}%
\end{pgfscope}%
\begin{pgfscope}%
\pgfsys@transformshift{2.226195in}{2.957278in}%
\pgfsys@useobject{currentmarker}{}%
\end{pgfscope}%
\begin{pgfscope}%
\pgfsys@transformshift{2.610878in}{3.396193in}%
\pgfsys@useobject{currentmarker}{}%
\end{pgfscope}%
\begin{pgfscope}%
\pgfsys@transformshift{2.995560in}{3.723943in}%
\pgfsys@useobject{currentmarker}{}%
\end{pgfscope}%
\begin{pgfscope}%
\pgfsys@transformshift{3.380242in}{3.942080in}%
\pgfsys@useobject{currentmarker}{}%
\end{pgfscope}%
\begin{pgfscope}%
\pgfsys@transformshift{3.764924in}{4.051781in}%
\pgfsys@useobject{currentmarker}{}%
\end{pgfscope}%
\begin{pgfscope}%
\pgfsys@transformshift{4.149606in}{4.053909in}%
\pgfsys@useobject{currentmarker}{}%
\end{pgfscope}%
\begin{pgfscope}%
\pgfsys@transformshift{4.534289in}{3.948890in}%
\pgfsys@useobject{currentmarker}{}%
\end{pgfscope}%
\end{pgfscope}%
\begin{pgfscope}%
\pgfpathrectangle{\pgfqpoint{0.880000in}{0.790446in}}{\pgfqpoint{4.227273in}{3.431818in}} %
\pgfusepath{clip}%
\pgfsetrectcap%
\pgfsetroundjoin%
\pgfsetlinewidth{1.505625pt}%
\definecolor{currentstroke}{rgb}{0.000000,0.000000,1.000000}%
\pgfsetstrokecolor{currentstroke}%
\pgfsetdash{}{0pt}%
\pgfpathmoveto{\pgfqpoint{1.072149in}{0.946438in}}%
\pgfpathlineto{\pgfqpoint{1.164472in}{1.147365in}}%
\pgfpathlineto{\pgfqpoint{1.256796in}{1.340939in}}%
\pgfpathlineto{\pgfqpoint{1.349120in}{1.527169in}}%
\pgfpathlineto{\pgfqpoint{1.437597in}{1.698758in}}%
\pgfpathlineto{\pgfqpoint{1.526074in}{1.863621in}}%
\pgfpathlineto{\pgfqpoint{1.610704in}{2.015032in}}%
\pgfpathlineto{\pgfqpoint{1.695334in}{2.160303in}}%
\pgfpathlineto{\pgfqpoint{1.779964in}{2.299443in}}%
\pgfpathlineto{\pgfqpoint{1.860747in}{2.426543in}}%
\pgfpathlineto{\pgfqpoint{1.941531in}{2.548068in}}%
\pgfpathlineto{\pgfqpoint{2.022314in}{2.664022in}}%
\pgfpathlineto{\pgfqpoint{2.103097in}{2.774410in}}%
\pgfpathlineto{\pgfqpoint{2.180034in}{2.874371in}}%
\pgfpathlineto{\pgfqpoint{2.256970in}{2.969291in}}%
\pgfpathlineto{\pgfqpoint{2.333906in}{3.059174in}}%
\pgfpathlineto{\pgfqpoint{2.406996in}{3.139900in}}%
\pgfpathlineto{\pgfqpoint{2.480086in}{3.216086in}}%
\pgfpathlineto{\pgfqpoint{2.553175in}{3.287735in}}%
\pgfpathlineto{\pgfqpoint{2.622418in}{3.351428in}}%
\pgfpathlineto{\pgfqpoint{2.691661in}{3.411053in}}%
\pgfpathlineto{\pgfqpoint{2.760904in}{3.466612in}}%
\pgfpathlineto{\pgfqpoint{2.830146in}{3.518105in}}%
\pgfpathlineto{\pgfqpoint{2.895542in}{3.563006in}}%
\pgfpathlineto{\pgfqpoint{2.960938in}{3.604284in}}%
\pgfpathlineto{\pgfqpoint{3.026334in}{3.641939in}}%
\pgfpathlineto{\pgfqpoint{3.091730in}{3.675973in}}%
\pgfpathlineto{\pgfqpoint{3.157126in}{3.706386in}}%
\pgfpathlineto{\pgfqpoint{3.218675in}{3.731704in}}%
\pgfpathlineto{\pgfqpoint{3.280225in}{3.753816in}}%
\pgfpathlineto{\pgfqpoint{3.341774in}{3.772722in}}%
\pgfpathlineto{\pgfqpoint{3.403323in}{3.788424in}}%
\pgfpathlineto{\pgfqpoint{3.464872in}{3.800922in}}%
\pgfpathlineto{\pgfqpoint{3.526421in}{3.810215in}}%
\pgfpathlineto{\pgfqpoint{3.587970in}{3.816304in}}%
\pgfpathlineto{\pgfqpoint{3.649520in}{3.819189in}}%
\pgfpathlineto{\pgfqpoint{3.711069in}{3.818871in}}%
\pgfpathlineto{\pgfqpoint{3.772618in}{3.815349in}}%
\pgfpathlineto{\pgfqpoint{3.834167in}{3.808623in}}%
\pgfpathlineto{\pgfqpoint{3.895716in}{3.798693in}}%
\pgfpathlineto{\pgfqpoint{3.957265in}{3.785559in}}%
\pgfpathlineto{\pgfqpoint{4.018814in}{3.769220in}}%
\pgfpathlineto{\pgfqpoint{4.080364in}{3.749676in}}%
\pgfpathlineto{\pgfqpoint{4.141913in}{3.726927in}}%
\pgfpathlineto{\pgfqpoint{4.203462in}{3.700972in}}%
\pgfpathlineto{\pgfqpoint{4.265011in}{3.671811in}}%
\pgfpathlineto{\pgfqpoint{4.330407in}{3.637312in}}%
\pgfpathlineto{\pgfqpoint{4.395803in}{3.599193in}}%
\pgfpathlineto{\pgfqpoint{4.461199in}{3.557450in}}%
\pgfpathlineto{\pgfqpoint{4.526595in}{3.512085in}}%
\pgfpathlineto{\pgfqpoint{4.591991in}{3.463094in}}%
\pgfpathlineto{\pgfqpoint{4.661234in}{3.407269in}}%
\pgfpathlineto{\pgfqpoint{4.730477in}{3.347377in}}%
\pgfpathlineto{\pgfqpoint{4.799719in}{3.283417in}}%
\pgfpathlineto{\pgfqpoint{4.868962in}{3.215385in}}%
\pgfpathlineto{\pgfqpoint{4.915124in}{3.167769in}}%
\pgfpathlineto{\pgfqpoint{4.915124in}{3.167769in}}%
\pgfusepath{stroke}%
\end{pgfscope}%
\begin{pgfscope}%
\pgfpathrectangle{\pgfqpoint{0.880000in}{0.790446in}}{\pgfqpoint{4.227273in}{3.431818in}} %
\pgfusepath{clip}%
\pgfsetbuttcap%
\pgfsetroundjoin%
\definecolor{currentfill}{rgb}{0.000000,0.000000,1.000000}%
\pgfsetfillcolor{currentfill}%
\pgfsetlinewidth{1.003750pt}%
\definecolor{currentstroke}{rgb}{0.000000,0.000000,1.000000}%
\pgfsetstrokecolor{currentstroke}%
\pgfsetdash{}{0pt}%
\pgfsys@defobject{currentmarker}{\pgfqpoint{-0.041667in}{-0.041667in}}{\pgfqpoint{0.041667in}{0.041667in}}{%
\pgfpathmoveto{\pgfqpoint{0.000000in}{-0.041667in}}%
\pgfpathcurveto{\pgfqpoint{0.011050in}{-0.041667in}}{\pgfqpoint{0.021649in}{-0.037276in}}{\pgfqpoint{0.029463in}{-0.029463in}}%
\pgfpathcurveto{\pgfqpoint{0.037276in}{-0.021649in}}{\pgfqpoint{0.041667in}{-0.011050in}}{\pgfqpoint{0.041667in}{0.000000in}}%
\pgfpathcurveto{\pgfqpoint{0.041667in}{0.011050in}}{\pgfqpoint{0.037276in}{0.021649in}}{\pgfqpoint{0.029463in}{0.029463in}}%
\pgfpathcurveto{\pgfqpoint{0.021649in}{0.037276in}}{\pgfqpoint{0.011050in}{0.041667in}}{\pgfqpoint{0.000000in}{0.041667in}}%
\pgfpathcurveto{\pgfqpoint{-0.011050in}{0.041667in}}{\pgfqpoint{-0.021649in}{0.037276in}}{\pgfqpoint{-0.029463in}{0.029463in}}%
\pgfpathcurveto{\pgfqpoint{-0.037276in}{0.021649in}}{\pgfqpoint{-0.041667in}{0.011050in}}{\pgfqpoint{-0.041667in}{0.000000in}}%
\pgfpathcurveto{\pgfqpoint{-0.041667in}{-0.011050in}}{\pgfqpoint{-0.037276in}{-0.021649in}}{\pgfqpoint{-0.029463in}{-0.029463in}}%
\pgfpathcurveto{\pgfqpoint{-0.021649in}{-0.037276in}}{\pgfqpoint{-0.011050in}{-0.041667in}}{\pgfqpoint{0.000000in}{-0.041667in}}%
\pgfpathclose%
\pgfusepath{stroke,fill}%
}%
\begin{pgfscope}%
\pgfsys@transformshift{1.072149in}{0.946438in}%
\pgfsys@useobject{currentmarker}{}%
\end{pgfscope}%
\begin{pgfscope}%
\pgfsys@transformshift{1.456831in}{1.735170in}%
\pgfsys@useobject{currentmarker}{}%
\end{pgfscope}%
\begin{pgfscope}%
\pgfsys@transformshift{1.841513in}{2.396787in}%
\pgfsys@useobject{currentmarker}{}%
\end{pgfscope}%
\begin{pgfscope}%
\pgfsys@transformshift{2.226195in}{2.931928in}%
\pgfsys@useobject{currentmarker}{}%
\end{pgfscope}%
\begin{pgfscope}%
\pgfsys@transformshift{2.610878in}{3.341095in}%
\pgfsys@useobject{currentmarker}{}%
\end{pgfscope}%
\begin{pgfscope}%
\pgfsys@transformshift{2.995560in}{3.624670in}%
\pgfsys@useobject{currentmarker}{}%
\end{pgfscope}%
\begin{pgfscope}%
\pgfsys@transformshift{3.380242in}{3.782912in}%
\pgfsys@useobject{currentmarker}{}%
\end{pgfscope}%
\begin{pgfscope}%
\pgfsys@transformshift{3.764924in}{3.815964in}%
\pgfsys@useobject{currentmarker}{}%
\end{pgfscope}%
\begin{pgfscope}%
\pgfsys@transformshift{4.149606in}{3.723858in}%
\pgfsys@useobject{currentmarker}{}%
\end{pgfscope}%
\begin{pgfscope}%
\pgfsys@transformshift{4.534289in}{3.506509in}%
\pgfsys@useobject{currentmarker}{}%
\end{pgfscope}%
\end{pgfscope}%
\begin{pgfscope}%
\pgfpathrectangle{\pgfqpoint{0.880000in}{0.790446in}}{\pgfqpoint{4.227273in}{3.431818in}} %
\pgfusepath{clip}%
\pgfsetbuttcap%
\pgfsetroundjoin%
\pgfsetlinewidth{1.505625pt}%
\definecolor{currentstroke}{rgb}{0.000000,0.750000,0.750000}%
\pgfsetstrokecolor{currentstroke}%
\pgfsetdash{{9.600000pt}{2.400000pt}{1.500000pt}{2.400000pt}}{0.000000pt}%
\pgfpathmoveto{\pgfqpoint{1.072149in}{0.946438in}}%
\pgfpathlineto{\pgfqpoint{1.164472in}{1.147364in}}%
\pgfpathlineto{\pgfqpoint{1.256796in}{1.340925in}}%
\pgfpathlineto{\pgfqpoint{1.345273in}{1.519511in}}%
\pgfpathlineto{\pgfqpoint{1.433750in}{1.691336in}}%
\pgfpathlineto{\pgfqpoint{1.522227in}{1.856399in}}%
\pgfpathlineto{\pgfqpoint{1.606857in}{2.007960in}}%
\pgfpathlineto{\pgfqpoint{1.691487in}{2.153337in}}%
\pgfpathlineto{\pgfqpoint{1.776117in}{2.292531in}}%
\pgfpathlineto{\pgfqpoint{1.856900in}{2.419629in}}%
\pgfpathlineto{\pgfqpoint{1.937684in}{2.541095in}}%
\pgfpathlineto{\pgfqpoint{2.018467in}{2.656928in}}%
\pgfpathlineto{\pgfqpoint{2.095403in}{2.762009in}}%
\pgfpathlineto{\pgfqpoint{2.172340in}{2.861982in}}%
\pgfpathlineto{\pgfqpoint{2.249276in}{2.956848in}}%
\pgfpathlineto{\pgfqpoint{2.322366in}{3.042240in}}%
\pgfpathlineto{\pgfqpoint{2.395456in}{3.123022in}}%
\pgfpathlineto{\pgfqpoint{2.468545in}{3.199197in}}%
\pgfpathlineto{\pgfqpoint{2.541635in}{3.270762in}}%
\pgfpathlineto{\pgfqpoint{2.610878in}{3.334311in}}%
\pgfpathlineto{\pgfqpoint{2.680120in}{3.393723in}}%
\pgfpathlineto{\pgfqpoint{2.749363in}{3.449001in}}%
\pgfpathlineto{\pgfqpoint{2.814759in}{3.497410in}}%
\pgfpathlineto{\pgfqpoint{2.880155in}{3.542131in}}%
\pgfpathlineto{\pgfqpoint{2.945551in}{3.583164in}}%
\pgfpathlineto{\pgfqpoint{3.010947in}{3.620508in}}%
\pgfpathlineto{\pgfqpoint{3.076343in}{3.654165in}}%
\pgfpathlineto{\pgfqpoint{3.137892in}{3.682472in}}%
\pgfpathlineto{\pgfqpoint{3.199441in}{3.707513in}}%
\pgfpathlineto{\pgfqpoint{3.260990in}{3.729288in}}%
\pgfpathlineto{\pgfqpoint{3.322540in}{3.747795in}}%
\pgfpathlineto{\pgfqpoint{3.384089in}{3.763036in}}%
\pgfpathlineto{\pgfqpoint{3.445638in}{3.775011in}}%
\pgfpathlineto{\pgfqpoint{3.507187in}{3.783718in}}%
\pgfpathlineto{\pgfqpoint{3.568736in}{3.789160in}}%
\pgfpathlineto{\pgfqpoint{3.630285in}{3.791335in}}%
\pgfpathlineto{\pgfqpoint{3.691835in}{3.790243in}}%
\pgfpathlineto{\pgfqpoint{3.753384in}{3.785885in}}%
\pgfpathlineto{\pgfqpoint{3.814933in}{3.778260in}}%
\pgfpathlineto{\pgfqpoint{3.876482in}{3.767369in}}%
\pgfpathlineto{\pgfqpoint{3.938031in}{3.753212in}}%
\pgfpathlineto{\pgfqpoint{3.999580in}{3.735787in}}%
\pgfpathlineto{\pgfqpoint{4.061129in}{3.715096in}}%
\pgfpathlineto{\pgfqpoint{4.122679in}{3.691139in}}%
\pgfpathlineto{\pgfqpoint{4.184228in}{3.663914in}}%
\pgfpathlineto{\pgfqpoint{4.245777in}{3.633423in}}%
\pgfpathlineto{\pgfqpoint{4.311173in}{3.597447in}}%
\pgfpathlineto{\pgfqpoint{4.376569in}{3.557782in}}%
\pgfpathlineto{\pgfqpoint{4.441965in}{3.514429in}}%
\pgfpathlineto{\pgfqpoint{4.507361in}{3.467388in}}%
\pgfpathlineto{\pgfqpoint{4.572757in}{3.416659in}}%
\pgfpathlineto{\pgfqpoint{4.642000in}{3.358924in}}%
\pgfpathlineto{\pgfqpoint{4.711242in}{3.297055in}}%
\pgfpathlineto{\pgfqpoint{4.780485in}{3.231049in}}%
\pgfpathlineto{\pgfqpoint{4.849728in}{3.160908in}}%
\pgfpathlineto{\pgfqpoint{4.915124in}{3.090865in}}%
\pgfpathlineto{\pgfqpoint{4.915124in}{3.090865in}}%
\pgfusepath{stroke}%
\end{pgfscope}%
\begin{pgfscope}%
\pgfpathrectangle{\pgfqpoint{0.880000in}{0.790446in}}{\pgfqpoint{4.227273in}{3.431818in}} %
\pgfusepath{clip}%
\pgfsetbuttcap%
\pgfsetmiterjoin%
\definecolor{currentfill}{rgb}{0.000000,0.750000,0.750000}%
\pgfsetfillcolor{currentfill}%
\pgfsetlinewidth{1.003750pt}%
\definecolor{currentstroke}{rgb}{0.000000,0.750000,0.750000}%
\pgfsetstrokecolor{currentstroke}%
\pgfsetdash{}{0pt}%
\pgfsys@defobject{currentmarker}{\pgfqpoint{-0.041667in}{-0.041667in}}{\pgfqpoint{0.041667in}{0.041667in}}{%
\pgfpathmoveto{\pgfqpoint{-0.000000in}{-0.041667in}}%
\pgfpathlineto{\pgfqpoint{0.041667in}{0.041667in}}%
\pgfpathlineto{\pgfqpoint{-0.041667in}{0.041667in}}%
\pgfpathclose%
\pgfusepath{stroke,fill}%
}%
\begin{pgfscope}%
\pgfsys@transformshift{1.072149in}{0.946438in}%
\pgfsys@useobject{currentmarker}{}%
\end{pgfscope}%
\begin{pgfscope}%
\pgfsys@transformshift{1.456831in}{1.735047in}%
\pgfsys@useobject{currentmarker}{}%
\end{pgfscope}%
\begin{pgfscope}%
\pgfsys@transformshift{1.841513in}{2.395854in}%
\pgfsys@useobject{currentmarker}{}%
\end{pgfscope}%
\begin{pgfscope}%
\pgfsys@transformshift{2.226195in}{2.928925in}%
\pgfsys@useobject{currentmarker}{}%
\end{pgfscope}%
\begin{pgfscope}%
\pgfsys@transformshift{2.610878in}{3.334311in}%
\pgfsys@useobject{currentmarker}{}%
\end{pgfscope}%
\begin{pgfscope}%
\pgfsys@transformshift{2.995560in}{3.612053in}%
\pgfsys@useobject{currentmarker}{}%
\end{pgfscope}%
\begin{pgfscope}%
\pgfsys@transformshift{3.380242in}{3.762179in}%
\pgfsys@useobject{currentmarker}{}%
\end{pgfscope}%
\begin{pgfscope}%
\pgfsys@transformshift{3.764924in}{3.784704in}%
\pgfsys@useobject{currentmarker}{}%
\end{pgfscope}%
\begin{pgfscope}%
\pgfsys@transformshift{4.149606in}{3.679630in}%
\pgfsys@useobject{currentmarker}{}%
\end{pgfscope}%
\begin{pgfscope}%
\pgfsys@transformshift{4.534289in}{3.446946in}%
\pgfsys@useobject{currentmarker}{}%
\end{pgfscope}%
\end{pgfscope}%
\begin{pgfscope}%
\pgfpathrectangle{\pgfqpoint{0.880000in}{0.790446in}}{\pgfqpoint{4.227273in}{3.431818in}} %
\pgfusepath{clip}%
\pgfsetbuttcap%
\pgfsetroundjoin%
\pgfsetlinewidth{1.505625pt}%
\definecolor{currentstroke}{rgb}{0.000000,0.000000,0.000000}%
\pgfsetstrokecolor{currentstroke}%
\pgfsetdash{{1.500000pt}{2.475000pt}}{0.000000pt}%
\pgfpathmoveto{\pgfqpoint{1.072149in}{0.946438in}}%
\pgfpathlineto{\pgfqpoint{1.164472in}{1.147363in}}%
\pgfpathlineto{\pgfqpoint{1.256796in}{1.340923in}}%
\pgfpathlineto{\pgfqpoint{1.345273in}{1.519507in}}%
\pgfpathlineto{\pgfqpoint{1.433750in}{1.691325in}}%
\pgfpathlineto{\pgfqpoint{1.522227in}{1.856380in}}%
\pgfpathlineto{\pgfqpoint{1.606857in}{2.007928in}}%
\pgfpathlineto{\pgfqpoint{1.691487in}{2.153287in}}%
\pgfpathlineto{\pgfqpoint{1.776117in}{2.292458in}}%
\pgfpathlineto{\pgfqpoint{1.856900in}{2.419529in}}%
\pgfpathlineto{\pgfqpoint{1.937684in}{2.540962in}}%
\pgfpathlineto{\pgfqpoint{2.018467in}{2.656756in}}%
\pgfpathlineto{\pgfqpoint{2.095403in}{2.761793in}}%
\pgfpathlineto{\pgfqpoint{2.172340in}{2.861716in}}%
\pgfpathlineto{\pgfqpoint{2.249276in}{2.956524in}}%
\pgfpathlineto{\pgfqpoint{2.322366in}{3.041855in}}%
\pgfpathlineto{\pgfqpoint{2.395456in}{3.122570in}}%
\pgfpathlineto{\pgfqpoint{2.468545in}{3.198669in}}%
\pgfpathlineto{\pgfqpoint{2.541635in}{3.270153in}}%
\pgfpathlineto{\pgfqpoint{2.610878in}{3.333617in}}%
\pgfpathlineto{\pgfqpoint{2.680120in}{3.392938in}}%
\pgfpathlineto{\pgfqpoint{2.749363in}{3.448117in}}%
\pgfpathlineto{\pgfqpoint{2.814759in}{3.496426in}}%
\pgfpathlineto{\pgfqpoint{2.880155in}{3.541040in}}%
\pgfpathlineto{\pgfqpoint{2.945551in}{3.581960in}}%
\pgfpathlineto{\pgfqpoint{3.010947in}{3.619184in}}%
\pgfpathlineto{\pgfqpoint{3.076343in}{3.652714in}}%
\pgfpathlineto{\pgfqpoint{3.137892in}{3.680896in}}%
\pgfpathlineto{\pgfqpoint{3.199441in}{3.705805in}}%
\pgfpathlineto{\pgfqpoint{3.260990in}{3.727440in}}%
\pgfpathlineto{\pgfqpoint{3.322540in}{3.745803in}}%
\pgfpathlineto{\pgfqpoint{3.384089in}{3.760893in}}%
\pgfpathlineto{\pgfqpoint{3.445638in}{3.772709in}}%
\pgfpathlineto{\pgfqpoint{3.507187in}{3.781253in}}%
\pgfpathlineto{\pgfqpoint{3.568736in}{3.786524in}}%
\pgfpathlineto{\pgfqpoint{3.630285in}{3.788521in}}%
\pgfpathlineto{\pgfqpoint{3.691835in}{3.787246in}}%
\pgfpathlineto{\pgfqpoint{3.753384in}{3.782698in}}%
\pgfpathlineto{\pgfqpoint{3.814933in}{3.774876in}}%
\pgfpathlineto{\pgfqpoint{3.876482in}{3.763782in}}%
\pgfpathlineto{\pgfqpoint{3.938031in}{3.749414in}}%
\pgfpathlineto{\pgfqpoint{3.999580in}{3.731774in}}%
\pgfpathlineto{\pgfqpoint{4.061129in}{3.710860in}}%
\pgfpathlineto{\pgfqpoint{4.122679in}{3.686674in}}%
\pgfpathlineto{\pgfqpoint{4.184228in}{3.659214in}}%
\pgfpathlineto{\pgfqpoint{4.245777in}{3.628482in}}%
\pgfpathlineto{\pgfqpoint{4.311173in}{3.592242in}}%
\pgfpathlineto{\pgfqpoint{4.376569in}{3.552307in}}%
\pgfpathlineto{\pgfqpoint{4.441965in}{3.508678in}}%
\pgfpathlineto{\pgfqpoint{4.507361in}{3.461353in}}%
\pgfpathlineto{\pgfqpoint{4.572757in}{3.410334in}}%
\pgfpathlineto{\pgfqpoint{4.642000in}{3.352285in}}%
\pgfpathlineto{\pgfqpoint{4.711242in}{3.290094in}}%
\pgfpathlineto{\pgfqpoint{4.780485in}{3.223761in}}%
\pgfpathlineto{\pgfqpoint{4.849728in}{3.153285in}}%
\pgfpathlineto{\pgfqpoint{4.915124in}{3.082921in}}%
\pgfpathlineto{\pgfqpoint{4.915124in}{3.082921in}}%
\pgfusepath{stroke}%
\end{pgfscope}%
\begin{pgfscope}%
\pgfpathrectangle{\pgfqpoint{0.880000in}{0.790446in}}{\pgfqpoint{4.227273in}{3.431818in}} %
\pgfusepath{clip}%
\pgfsetbuttcap%
\pgfsetroundjoin%
\definecolor{currentfill}{rgb}{0.000000,0.000000,0.000000}%
\pgfsetfillcolor{currentfill}%
\pgfsetlinewidth{1.003750pt}%
\definecolor{currentstroke}{rgb}{0.000000,0.000000,0.000000}%
\pgfsetstrokecolor{currentstroke}%
\pgfsetdash{}{0pt}%
\pgfsys@defobject{currentmarker}{\pgfqpoint{-0.041667in}{-0.041667in}}{\pgfqpoint{0.041667in}{0.041667in}}{%
\pgfpathmoveto{\pgfqpoint{-0.041667in}{0.000000in}}%
\pgfpathlineto{\pgfqpoint{0.041667in}{0.000000in}}%
\pgfpathmoveto{\pgfqpoint{0.000000in}{-0.041667in}}%
\pgfpathlineto{\pgfqpoint{0.000000in}{0.041667in}}%
\pgfusepath{stroke,fill}%
}%
\begin{pgfscope}%
\pgfsys@transformshift{1.072149in}{0.946438in}%
\pgfsys@useobject{currentmarker}{}%
\end{pgfscope}%
\begin{pgfscope}%
\pgfsys@transformshift{1.456831in}{1.735035in}%
\pgfsys@useobject{currentmarker}{}%
\end{pgfscope}%
\begin{pgfscope}%
\pgfsys@transformshift{1.841513in}{2.395760in}%
\pgfsys@useobject{currentmarker}{}%
\end{pgfscope}%
\begin{pgfscope}%
\pgfsys@transformshift{2.226195in}{2.928619in}%
\pgfsys@useobject{currentmarker}{}%
\end{pgfscope}%
\begin{pgfscope}%
\pgfsys@transformshift{2.610878in}{3.333617in}%
\pgfsys@useobject{currentmarker}{}%
\end{pgfscope}%
\begin{pgfscope}%
\pgfsys@transformshift{2.995560in}{3.610758in}%
\pgfsys@useobject{currentmarker}{}%
\end{pgfscope}%
\begin{pgfscope}%
\pgfsys@transformshift{3.380242in}{3.760046in}%
\pgfsys@useobject{currentmarker}{}%
\end{pgfscope}%
\begin{pgfscope}%
\pgfsys@transformshift{3.764924in}{3.781480in}%
\pgfsys@useobject{currentmarker}{}%
\end{pgfscope}%
\begin{pgfscope}%
\pgfsys@transformshift{4.149606in}{3.675063in}%
\pgfsys@useobject{currentmarker}{}%
\end{pgfscope}%
\begin{pgfscope}%
\pgfsys@transformshift{4.534289in}{3.440793in}%
\pgfsys@useobject{currentmarker}{}%
\end{pgfscope}%
\end{pgfscope}%
\begin{pgfscope}%
\pgfsetrectcap%
\pgfsetmiterjoin%
\pgfsetlinewidth{0.803000pt}%
\definecolor{currentstroke}{rgb}{0.000000,0.000000,0.000000}%
\pgfsetstrokecolor{currentstroke}%
\pgfsetdash{}{0pt}%
\pgfpathmoveto{\pgfqpoint{0.880000in}{0.790446in}}%
\pgfpathlineto{\pgfqpoint{0.880000in}{4.222264in}}%
\pgfusepath{stroke}%
\end{pgfscope}%
\begin{pgfscope}%
\pgfsetrectcap%
\pgfsetmiterjoin%
\pgfsetlinewidth{0.803000pt}%
\definecolor{currentstroke}{rgb}{0.000000,0.000000,0.000000}%
\pgfsetstrokecolor{currentstroke}%
\pgfsetdash{}{0pt}%
\pgfpathmoveto{\pgfqpoint{5.107273in}{0.790446in}}%
\pgfpathlineto{\pgfqpoint{5.107273in}{4.222264in}}%
\pgfusepath{stroke}%
\end{pgfscope}%
\begin{pgfscope}%
\pgfsetrectcap%
\pgfsetmiterjoin%
\pgfsetlinewidth{0.803000pt}%
\definecolor{currentstroke}{rgb}{0.000000,0.000000,0.000000}%
\pgfsetstrokecolor{currentstroke}%
\pgfsetdash{}{0pt}%
\pgfpathmoveto{\pgfqpoint{0.880000in}{0.790446in}}%
\pgfpathlineto{\pgfqpoint{5.107273in}{0.790446in}}%
\pgfusepath{stroke}%
\end{pgfscope}%
\begin{pgfscope}%
\pgfsetrectcap%
\pgfsetmiterjoin%
\pgfsetlinewidth{0.803000pt}%
\definecolor{currentstroke}{rgb}{0.000000,0.000000,0.000000}%
\pgfsetstrokecolor{currentstroke}%
\pgfsetdash{}{0pt}%
\pgfpathmoveto{\pgfqpoint{0.880000in}{4.222264in}}%
\pgfpathlineto{\pgfqpoint{5.107273in}{4.222264in}}%
\pgfusepath{stroke}%
\end{pgfscope}%
\begin{pgfscope}%
\pgfsetbuttcap%
\pgfsetmiterjoin%
\definecolor{currentfill}{rgb}{1.000000,1.000000,1.000000}%
\pgfsetfillcolor{currentfill}%
\pgfsetfillopacity{0.800000}%
\pgfsetlinewidth{1.003750pt}%
\definecolor{currentstroke}{rgb}{0.800000,0.800000,0.800000}%
\pgfsetstrokecolor{currentstroke}%
\pgfsetstrokeopacity{0.800000}%
\pgfsetdash{}{0pt}%
\pgfpathmoveto{\pgfqpoint{2.442086in}{0.859890in}}%
\pgfpathlineto{\pgfqpoint{3.545187in}{0.859890in}}%
\pgfpathquadraticcurveto{\pgfqpoint{3.572965in}{0.859890in}}{\pgfqpoint{3.572965in}{0.887668in}}%
\pgfpathlineto{\pgfqpoint{3.572965in}{1.893065in}}%
\pgfpathquadraticcurveto{\pgfqpoint{3.572965in}{1.920843in}}{\pgfqpoint{3.545187in}{1.920843in}}%
\pgfpathlineto{\pgfqpoint{2.442086in}{1.920843in}}%
\pgfpathquadraticcurveto{\pgfqpoint{2.414308in}{1.920843in}}{\pgfqpoint{2.414308in}{1.893065in}}%
\pgfpathlineto{\pgfqpoint{2.414308in}{0.887668in}}%
\pgfpathquadraticcurveto{\pgfqpoint{2.414308in}{0.859890in}}{\pgfqpoint{2.442086in}{0.859890in}}%
\pgfpathclose%
\pgfusepath{stroke,fill}%
\end{pgfscope}%
\begin{pgfscope}%
\pgftext[x=2.736180in,y=1.759764in,left,base]{\rmfamily\fontsize{10.000000}{12.000000}\selectfont \(\displaystyle \alpha\) = 0.5}%
\end{pgfscope}%
\begin{pgfscope}%
\pgfsetbuttcap%
\pgfsetroundjoin%
\pgfsetlinewidth{1.505625pt}%
\definecolor{currentstroke}{rgb}{1.000000,0.000000,0.000000}%
\pgfsetstrokecolor{currentstroke}%
\pgfsetdash{{5.550000pt}{2.400000pt}}{0.000000pt}%
\pgfpathmoveto{\pgfqpoint{2.469863in}{1.604518in}}%
\pgfpathlineto{\pgfqpoint{2.747641in}{1.604518in}}%
\pgfusepath{stroke}%
\end{pgfscope}%
\begin{pgfscope}%
\pgfsetbuttcap%
\pgfsetmiterjoin%
\definecolor{currentfill}{rgb}{1.000000,0.000000,0.000000}%
\pgfsetfillcolor{currentfill}%
\pgfsetlinewidth{1.003750pt}%
\definecolor{currentstroke}{rgb}{1.000000,0.000000,0.000000}%
\pgfsetstrokecolor{currentstroke}%
\pgfsetdash{}{0pt}%
\pgfsys@defobject{currentmarker}{\pgfqpoint{-0.041667in}{-0.041667in}}{\pgfqpoint{0.041667in}{0.041667in}}{%
\pgfpathmoveto{\pgfqpoint{-0.041667in}{-0.041667in}}%
\pgfpathlineto{\pgfqpoint{0.041667in}{-0.041667in}}%
\pgfpathlineto{\pgfqpoint{0.041667in}{0.041667in}}%
\pgfpathlineto{\pgfqpoint{-0.041667in}{0.041667in}}%
\pgfpathclose%
\pgfusepath{stroke,fill}%
}%
\begin{pgfscope}%
\pgfsys@transformshift{2.608752in}{1.604518in}%
\pgfsys@useobject{currentmarker}{}%
\end{pgfscope}%
\end{pgfscope}%
\begin{pgfscope}%
\pgftext[x=2.858752in,y=1.555907in,left,base]{\rmfamily\fontsize{10.000000}{12.000000}\selectfont \(\displaystyle \epsilon\) = 1}%
\end{pgfscope}%
\begin{pgfscope}%
\pgfsetrectcap%
\pgfsetroundjoin%
\pgfsetlinewidth{1.505625pt}%
\definecolor{currentstroke}{rgb}{0.000000,0.000000,1.000000}%
\pgfsetstrokecolor{currentstroke}%
\pgfsetdash{}{0pt}%
\pgfpathmoveto{\pgfqpoint{2.469863in}{1.400661in}}%
\pgfpathlineto{\pgfqpoint{2.747641in}{1.400661in}}%
\pgfusepath{stroke}%
\end{pgfscope}%
\begin{pgfscope}%
\pgfsetbuttcap%
\pgfsetroundjoin%
\definecolor{currentfill}{rgb}{0.000000,0.000000,1.000000}%
\pgfsetfillcolor{currentfill}%
\pgfsetlinewidth{1.003750pt}%
\definecolor{currentstroke}{rgb}{0.000000,0.000000,1.000000}%
\pgfsetstrokecolor{currentstroke}%
\pgfsetdash{}{0pt}%
\pgfsys@defobject{currentmarker}{\pgfqpoint{-0.041667in}{-0.041667in}}{\pgfqpoint{0.041667in}{0.041667in}}{%
\pgfpathmoveto{\pgfqpoint{0.000000in}{-0.041667in}}%
\pgfpathcurveto{\pgfqpoint{0.011050in}{-0.041667in}}{\pgfqpoint{0.021649in}{-0.037276in}}{\pgfqpoint{0.029463in}{-0.029463in}}%
\pgfpathcurveto{\pgfqpoint{0.037276in}{-0.021649in}}{\pgfqpoint{0.041667in}{-0.011050in}}{\pgfqpoint{0.041667in}{0.000000in}}%
\pgfpathcurveto{\pgfqpoint{0.041667in}{0.011050in}}{\pgfqpoint{0.037276in}{0.021649in}}{\pgfqpoint{0.029463in}{0.029463in}}%
\pgfpathcurveto{\pgfqpoint{0.021649in}{0.037276in}}{\pgfqpoint{0.011050in}{0.041667in}}{\pgfqpoint{0.000000in}{0.041667in}}%
\pgfpathcurveto{\pgfqpoint{-0.011050in}{0.041667in}}{\pgfqpoint{-0.021649in}{0.037276in}}{\pgfqpoint{-0.029463in}{0.029463in}}%
\pgfpathcurveto{\pgfqpoint{-0.037276in}{0.021649in}}{\pgfqpoint{-0.041667in}{0.011050in}}{\pgfqpoint{-0.041667in}{0.000000in}}%
\pgfpathcurveto{\pgfqpoint{-0.041667in}{-0.011050in}}{\pgfqpoint{-0.037276in}{-0.021649in}}{\pgfqpoint{-0.029463in}{-0.029463in}}%
\pgfpathcurveto{\pgfqpoint{-0.021649in}{-0.037276in}}{\pgfqpoint{-0.011050in}{-0.041667in}}{\pgfqpoint{0.000000in}{-0.041667in}}%
\pgfpathclose%
\pgfusepath{stroke,fill}%
}%
\begin{pgfscope}%
\pgfsys@transformshift{2.608752in}{1.400661in}%
\pgfsys@useobject{currentmarker}{}%
\end{pgfscope}%
\end{pgfscope}%
\begin{pgfscope}%
\pgftext[x=2.858752in,y=1.352050in,left,base]{\rmfamily\fontsize{10.000000}{12.000000}\selectfont \(\displaystyle \epsilon\) = 0.1}%
\end{pgfscope}%
\begin{pgfscope}%
\pgfsetbuttcap%
\pgfsetroundjoin%
\pgfsetlinewidth{1.505625pt}%
\definecolor{currentstroke}{rgb}{0.000000,0.750000,0.750000}%
\pgfsetstrokecolor{currentstroke}%
\pgfsetdash{{9.600000pt}{2.400000pt}{1.500000pt}{2.400000pt}}{0.000000pt}%
\pgfpathmoveto{\pgfqpoint{2.469863in}{1.196804in}}%
\pgfpathlineto{\pgfqpoint{2.747641in}{1.196804in}}%
\pgfusepath{stroke}%
\end{pgfscope}%
\begin{pgfscope}%
\pgfsetbuttcap%
\pgfsetmiterjoin%
\definecolor{currentfill}{rgb}{0.000000,0.750000,0.750000}%
\pgfsetfillcolor{currentfill}%
\pgfsetlinewidth{1.003750pt}%
\definecolor{currentstroke}{rgb}{0.000000,0.750000,0.750000}%
\pgfsetstrokecolor{currentstroke}%
\pgfsetdash{}{0pt}%
\pgfsys@defobject{currentmarker}{\pgfqpoint{-0.041667in}{-0.041667in}}{\pgfqpoint{0.041667in}{0.041667in}}{%
\pgfpathmoveto{\pgfqpoint{-0.000000in}{-0.041667in}}%
\pgfpathlineto{\pgfqpoint{0.041667in}{0.041667in}}%
\pgfpathlineto{\pgfqpoint{-0.041667in}{0.041667in}}%
\pgfpathclose%
\pgfusepath{stroke,fill}%
}%
\begin{pgfscope}%
\pgfsys@transformshift{2.608752in}{1.196804in}%
\pgfsys@useobject{currentmarker}{}%
\end{pgfscope}%
\end{pgfscope}%
\begin{pgfscope}%
\pgftext[x=2.858752in,y=1.148193in,left,base]{\rmfamily\fontsize{10.000000}{12.000000}\selectfont \(\displaystyle \epsilon\) = 0.01}%
\end{pgfscope}%
\begin{pgfscope}%
\pgfsetbuttcap%
\pgfsetroundjoin%
\pgfsetlinewidth{1.505625pt}%
\definecolor{currentstroke}{rgb}{0.000000,0.000000,0.000000}%
\pgfsetstrokecolor{currentstroke}%
\pgfsetdash{{1.500000pt}{2.475000pt}}{0.000000pt}%
\pgfpathmoveto{\pgfqpoint{2.469863in}{0.992947in}}%
\pgfpathlineto{\pgfqpoint{2.747641in}{0.992947in}}%
\pgfusepath{stroke}%
\end{pgfscope}%
\begin{pgfscope}%
\pgfsetbuttcap%
\pgfsetroundjoin%
\definecolor{currentfill}{rgb}{0.000000,0.000000,0.000000}%
\pgfsetfillcolor{currentfill}%
\pgfsetlinewidth{1.003750pt}%
\definecolor{currentstroke}{rgb}{0.000000,0.000000,0.000000}%
\pgfsetstrokecolor{currentstroke}%
\pgfsetdash{}{0pt}%
\pgfsys@defobject{currentmarker}{\pgfqpoint{-0.041667in}{-0.041667in}}{\pgfqpoint{0.041667in}{0.041667in}}{%
\pgfpathmoveto{\pgfqpoint{-0.041667in}{0.000000in}}%
\pgfpathlineto{\pgfqpoint{0.041667in}{0.000000in}}%
\pgfpathmoveto{\pgfqpoint{0.000000in}{-0.041667in}}%
\pgfpathlineto{\pgfqpoint{0.000000in}{0.041667in}}%
\pgfusepath{stroke,fill}%
}%
\begin{pgfscope}%
\pgfsys@transformshift{2.608752in}{0.992947in}%
\pgfsys@useobject{currentmarker}{}%
\end{pgfscope}%
\end{pgfscope}%
\begin{pgfscope}%
\pgftext[x=2.858752in,y=0.944336in,left,base]{\rmfamily\fontsize{10.000000}{12.000000}\selectfont \(\displaystyle \epsilon\) = 0.001}%
\end{pgfscope}%
\begin{pgfscope}%
\pgfsetbuttcap%
\pgfsetmiterjoin%
\definecolor{currentfill}{rgb}{1.000000,1.000000,1.000000}%
\pgfsetfillcolor{currentfill}%
\pgfsetlinewidth{0.000000pt}%
\definecolor{currentstroke}{rgb}{0.000000,0.000000,0.000000}%
\pgfsetstrokecolor{currentstroke}%
\pgfsetstrokeopacity{0.000000}%
\pgfsetdash{}{0pt}%
\pgfpathmoveto{\pgfqpoint{5.952727in}{0.790446in}}%
\pgfpathlineto{\pgfqpoint{10.180000in}{0.790446in}}%
\pgfpathlineto{\pgfqpoint{10.180000in}{4.222264in}}%
\pgfpathlineto{\pgfqpoint{5.952727in}{4.222264in}}%
\pgfpathclose%
\pgfusepath{fill}%
\end{pgfscope}%
\begin{pgfscope}%
\pgfsetbuttcap%
\pgfsetroundjoin%
\definecolor{currentfill}{rgb}{0.000000,0.000000,0.000000}%
\pgfsetfillcolor{currentfill}%
\pgfsetlinewidth{0.803000pt}%
\definecolor{currentstroke}{rgb}{0.000000,0.000000,0.000000}%
\pgfsetstrokecolor{currentstroke}%
\pgfsetdash{}{0pt}%
\pgfsys@defobject{currentmarker}{\pgfqpoint{0.000000in}{-0.048611in}}{\pgfqpoint{0.000000in}{0.000000in}}{%
\pgfpathmoveto{\pgfqpoint{0.000000in}{0.000000in}}%
\pgfpathlineto{\pgfqpoint{0.000000in}{-0.048611in}}%
\pgfusepath{stroke,fill}%
}%
\begin{pgfscope}%
\pgfsys@transformshift{6.144876in}{0.790446in}%
\pgfsys@useobject{currentmarker}{}%
\end{pgfscope}%
\end{pgfscope}%
\begin{pgfscope}%
\pgftext[x=6.144876in,y=0.693224in,,top]{\rmfamily\fontsize{10.000000}{12.000000}\selectfont \(\displaystyle 0.0\)}%
\end{pgfscope}%
\begin{pgfscope}%
\pgfsetbuttcap%
\pgfsetroundjoin%
\definecolor{currentfill}{rgb}{0.000000,0.000000,0.000000}%
\pgfsetfillcolor{currentfill}%
\pgfsetlinewidth{0.803000pt}%
\definecolor{currentstroke}{rgb}{0.000000,0.000000,0.000000}%
\pgfsetstrokecolor{currentstroke}%
\pgfsetdash{}{0pt}%
\pgfsys@defobject{currentmarker}{\pgfqpoint{0.000000in}{-0.048611in}}{\pgfqpoint{0.000000in}{0.000000in}}{%
\pgfpathmoveto{\pgfqpoint{0.000000in}{0.000000in}}%
\pgfpathlineto{\pgfqpoint{0.000000in}{-0.048611in}}%
\pgfusepath{stroke,fill}%
}%
\begin{pgfscope}%
\pgfsys@transformshift{6.914240in}{0.790446in}%
\pgfsys@useobject{currentmarker}{}%
\end{pgfscope}%
\end{pgfscope}%
\begin{pgfscope}%
\pgftext[x=6.914240in,y=0.693224in,,top]{\rmfamily\fontsize{10.000000}{12.000000}\selectfont \(\displaystyle 0.2\)}%
\end{pgfscope}%
\begin{pgfscope}%
\pgfsetbuttcap%
\pgfsetroundjoin%
\definecolor{currentfill}{rgb}{0.000000,0.000000,0.000000}%
\pgfsetfillcolor{currentfill}%
\pgfsetlinewidth{0.803000pt}%
\definecolor{currentstroke}{rgb}{0.000000,0.000000,0.000000}%
\pgfsetstrokecolor{currentstroke}%
\pgfsetdash{}{0pt}%
\pgfsys@defobject{currentmarker}{\pgfqpoint{0.000000in}{-0.048611in}}{\pgfqpoint{0.000000in}{0.000000in}}{%
\pgfpathmoveto{\pgfqpoint{0.000000in}{0.000000in}}%
\pgfpathlineto{\pgfqpoint{0.000000in}{-0.048611in}}%
\pgfusepath{stroke,fill}%
}%
\begin{pgfscope}%
\pgfsys@transformshift{7.683605in}{0.790446in}%
\pgfsys@useobject{currentmarker}{}%
\end{pgfscope}%
\end{pgfscope}%
\begin{pgfscope}%
\pgftext[x=7.683605in,y=0.693224in,,top]{\rmfamily\fontsize{10.000000}{12.000000}\selectfont \(\displaystyle 0.4\)}%
\end{pgfscope}%
\begin{pgfscope}%
\pgfsetbuttcap%
\pgfsetroundjoin%
\definecolor{currentfill}{rgb}{0.000000,0.000000,0.000000}%
\pgfsetfillcolor{currentfill}%
\pgfsetlinewidth{0.803000pt}%
\definecolor{currentstroke}{rgb}{0.000000,0.000000,0.000000}%
\pgfsetstrokecolor{currentstroke}%
\pgfsetdash{}{0pt}%
\pgfsys@defobject{currentmarker}{\pgfqpoint{0.000000in}{-0.048611in}}{\pgfqpoint{0.000000in}{0.000000in}}{%
\pgfpathmoveto{\pgfqpoint{0.000000in}{0.000000in}}%
\pgfpathlineto{\pgfqpoint{0.000000in}{-0.048611in}}%
\pgfusepath{stroke,fill}%
}%
\begin{pgfscope}%
\pgfsys@transformshift{8.452969in}{0.790446in}%
\pgfsys@useobject{currentmarker}{}%
\end{pgfscope}%
\end{pgfscope}%
\begin{pgfscope}%
\pgftext[x=8.452969in,y=0.693224in,,top]{\rmfamily\fontsize{10.000000}{12.000000}\selectfont \(\displaystyle 0.6\)}%
\end{pgfscope}%
\begin{pgfscope}%
\pgfsetbuttcap%
\pgfsetroundjoin%
\definecolor{currentfill}{rgb}{0.000000,0.000000,0.000000}%
\pgfsetfillcolor{currentfill}%
\pgfsetlinewidth{0.803000pt}%
\definecolor{currentstroke}{rgb}{0.000000,0.000000,0.000000}%
\pgfsetstrokecolor{currentstroke}%
\pgfsetdash{}{0pt}%
\pgfsys@defobject{currentmarker}{\pgfqpoint{0.000000in}{-0.048611in}}{\pgfqpoint{0.000000in}{0.000000in}}{%
\pgfpathmoveto{\pgfqpoint{0.000000in}{0.000000in}}%
\pgfpathlineto{\pgfqpoint{0.000000in}{-0.048611in}}%
\pgfusepath{stroke,fill}%
}%
\begin{pgfscope}%
\pgfsys@transformshift{9.222334in}{0.790446in}%
\pgfsys@useobject{currentmarker}{}%
\end{pgfscope}%
\end{pgfscope}%
\begin{pgfscope}%
\pgftext[x=9.222334in,y=0.693224in,,top]{\rmfamily\fontsize{10.000000}{12.000000}\selectfont \(\displaystyle 0.8\)}%
\end{pgfscope}%
\begin{pgfscope}%
\pgfsetbuttcap%
\pgfsetroundjoin%
\definecolor{currentfill}{rgb}{0.000000,0.000000,0.000000}%
\pgfsetfillcolor{currentfill}%
\pgfsetlinewidth{0.803000pt}%
\definecolor{currentstroke}{rgb}{0.000000,0.000000,0.000000}%
\pgfsetstrokecolor{currentstroke}%
\pgfsetdash{}{0pt}%
\pgfsys@defobject{currentmarker}{\pgfqpoint{0.000000in}{-0.048611in}}{\pgfqpoint{0.000000in}{0.000000in}}{%
\pgfpathmoveto{\pgfqpoint{0.000000in}{0.000000in}}%
\pgfpathlineto{\pgfqpoint{0.000000in}{-0.048611in}}%
\pgfusepath{stroke,fill}%
}%
\begin{pgfscope}%
\pgfsys@transformshift{9.991698in}{0.790446in}%
\pgfsys@useobject{currentmarker}{}%
\end{pgfscope}%
\end{pgfscope}%
\begin{pgfscope}%
\pgftext[x=9.991698in,y=0.693224in,,top]{\rmfamily\fontsize{10.000000}{12.000000}\selectfont \(\displaystyle 1.0\)}%
\end{pgfscope}%
\begin{pgfscope}%
\pgfsetbuttcap%
\pgfsetroundjoin%
\definecolor{currentfill}{rgb}{0.000000,0.000000,0.000000}%
\pgfsetfillcolor{currentfill}%
\pgfsetlinewidth{0.803000pt}%
\definecolor{currentstroke}{rgb}{0.000000,0.000000,0.000000}%
\pgfsetstrokecolor{currentstroke}%
\pgfsetdash{}{0pt}%
\pgfsys@defobject{currentmarker}{\pgfqpoint{-0.048611in}{0.000000in}}{\pgfqpoint{0.000000in}{0.000000in}}{%
\pgfpathmoveto{\pgfqpoint{0.000000in}{0.000000in}}%
\pgfpathlineto{\pgfqpoint{-0.048611in}{0.000000in}}%
\pgfusepath{stroke,fill}%
}%
\begin{pgfscope}%
\pgfsys@transformshift{5.952727in}{0.946438in}%
\pgfsys@useobject{currentmarker}{}%
\end{pgfscope}%
\end{pgfscope}%
\begin{pgfscope}%
\pgftext[x=5.678035in,y=0.893676in,left,base]{\rmfamily\fontsize{10.000000}{12.000000}\selectfont \(\displaystyle 0.0\)}%
\end{pgfscope}%
\begin{pgfscope}%
\pgfsetbuttcap%
\pgfsetroundjoin%
\definecolor{currentfill}{rgb}{0.000000,0.000000,0.000000}%
\pgfsetfillcolor{currentfill}%
\pgfsetlinewidth{0.803000pt}%
\definecolor{currentstroke}{rgb}{0.000000,0.000000,0.000000}%
\pgfsetstrokecolor{currentstroke}%
\pgfsetdash{}{0pt}%
\pgfsys@defobject{currentmarker}{\pgfqpoint{-0.048611in}{0.000000in}}{\pgfqpoint{0.000000in}{0.000000in}}{%
\pgfpathmoveto{\pgfqpoint{0.000000in}{0.000000in}}%
\pgfpathlineto{\pgfqpoint{-0.048611in}{0.000000in}}%
\pgfusepath{stroke,fill}%
}%
\begin{pgfscope}%
\pgfsys@transformshift{5.952727in}{1.574696in}%
\pgfsys@useobject{currentmarker}{}%
\end{pgfscope}%
\end{pgfscope}%
\begin{pgfscope}%
\pgftext[x=5.678035in,y=1.521935in,left,base]{\rmfamily\fontsize{10.000000}{12.000000}\selectfont \(\displaystyle 0.1\)}%
\end{pgfscope}%
\begin{pgfscope}%
\pgfsetbuttcap%
\pgfsetroundjoin%
\definecolor{currentfill}{rgb}{0.000000,0.000000,0.000000}%
\pgfsetfillcolor{currentfill}%
\pgfsetlinewidth{0.803000pt}%
\definecolor{currentstroke}{rgb}{0.000000,0.000000,0.000000}%
\pgfsetstrokecolor{currentstroke}%
\pgfsetdash{}{0pt}%
\pgfsys@defobject{currentmarker}{\pgfqpoint{-0.048611in}{0.000000in}}{\pgfqpoint{0.000000in}{0.000000in}}{%
\pgfpathmoveto{\pgfqpoint{0.000000in}{0.000000in}}%
\pgfpathlineto{\pgfqpoint{-0.048611in}{0.000000in}}%
\pgfusepath{stroke,fill}%
}%
\begin{pgfscope}%
\pgfsys@transformshift{5.952727in}{2.202955in}%
\pgfsys@useobject{currentmarker}{}%
\end{pgfscope}%
\end{pgfscope}%
\begin{pgfscope}%
\pgftext[x=5.678035in,y=2.150193in,left,base]{\rmfamily\fontsize{10.000000}{12.000000}\selectfont \(\displaystyle 0.2\)}%
\end{pgfscope}%
\begin{pgfscope}%
\pgfsetbuttcap%
\pgfsetroundjoin%
\definecolor{currentfill}{rgb}{0.000000,0.000000,0.000000}%
\pgfsetfillcolor{currentfill}%
\pgfsetlinewidth{0.803000pt}%
\definecolor{currentstroke}{rgb}{0.000000,0.000000,0.000000}%
\pgfsetstrokecolor{currentstroke}%
\pgfsetdash{}{0pt}%
\pgfsys@defobject{currentmarker}{\pgfqpoint{-0.048611in}{0.000000in}}{\pgfqpoint{0.000000in}{0.000000in}}{%
\pgfpathmoveto{\pgfqpoint{0.000000in}{0.000000in}}%
\pgfpathlineto{\pgfqpoint{-0.048611in}{0.000000in}}%
\pgfusepath{stroke,fill}%
}%
\begin{pgfscope}%
\pgfsys@transformshift{5.952727in}{2.831213in}%
\pgfsys@useobject{currentmarker}{}%
\end{pgfscope}%
\end{pgfscope}%
\begin{pgfscope}%
\pgftext[x=5.678035in,y=2.778452in,left,base]{\rmfamily\fontsize{10.000000}{12.000000}\selectfont \(\displaystyle 0.3\)}%
\end{pgfscope}%
\begin{pgfscope}%
\pgfsetbuttcap%
\pgfsetroundjoin%
\definecolor{currentfill}{rgb}{0.000000,0.000000,0.000000}%
\pgfsetfillcolor{currentfill}%
\pgfsetlinewidth{0.803000pt}%
\definecolor{currentstroke}{rgb}{0.000000,0.000000,0.000000}%
\pgfsetstrokecolor{currentstroke}%
\pgfsetdash{}{0pt}%
\pgfsys@defobject{currentmarker}{\pgfqpoint{-0.048611in}{0.000000in}}{\pgfqpoint{0.000000in}{0.000000in}}{%
\pgfpathmoveto{\pgfqpoint{0.000000in}{0.000000in}}%
\pgfpathlineto{\pgfqpoint{-0.048611in}{0.000000in}}%
\pgfusepath{stroke,fill}%
}%
\begin{pgfscope}%
\pgfsys@transformshift{5.952727in}{3.459472in}%
\pgfsys@useobject{currentmarker}{}%
\end{pgfscope}%
\end{pgfscope}%
\begin{pgfscope}%
\pgftext[x=5.678035in,y=3.406710in,left,base]{\rmfamily\fontsize{10.000000}{12.000000}\selectfont \(\displaystyle 0.4\)}%
\end{pgfscope}%
\begin{pgfscope}%
\pgfsetbuttcap%
\pgfsetroundjoin%
\definecolor{currentfill}{rgb}{0.000000,0.000000,0.000000}%
\pgfsetfillcolor{currentfill}%
\pgfsetlinewidth{0.803000pt}%
\definecolor{currentstroke}{rgb}{0.000000,0.000000,0.000000}%
\pgfsetstrokecolor{currentstroke}%
\pgfsetdash{}{0pt}%
\pgfsys@defobject{currentmarker}{\pgfqpoint{-0.048611in}{0.000000in}}{\pgfqpoint{0.000000in}{0.000000in}}{%
\pgfpathmoveto{\pgfqpoint{0.000000in}{0.000000in}}%
\pgfpathlineto{\pgfqpoint{-0.048611in}{0.000000in}}%
\pgfusepath{stroke,fill}%
}%
\begin{pgfscope}%
\pgfsys@transformshift{5.952727in}{4.087730in}%
\pgfsys@useobject{currentmarker}{}%
\end{pgfscope}%
\end{pgfscope}%
\begin{pgfscope}%
\pgftext[x=5.678035in,y=4.034969in,left,base]{\rmfamily\fontsize{10.000000}{12.000000}\selectfont \(\displaystyle 0.5\)}%
\end{pgfscope}%
\begin{pgfscope}%
\pgfpathrectangle{\pgfqpoint{5.952727in}{0.790446in}}{\pgfqpoint{4.227273in}{3.431818in}} %
\pgfusepath{clip}%
\pgfsetbuttcap%
\pgfsetroundjoin%
\pgfsetlinewidth{1.505625pt}%
\definecolor{currentstroke}{rgb}{1.000000,0.000000,0.000000}%
\pgfsetstrokecolor{currentstroke}%
\pgfsetdash{{5.550000pt}{2.400000pt}}{0.000000pt}%
\pgfpathmoveto{\pgfqpoint{6.144876in}{0.946438in}}%
\pgfpathlineto{\pgfqpoint{6.264128in}{1.138149in}}%
\pgfpathlineto{\pgfqpoint{6.379532in}{1.317874in}}%
\pgfpathlineto{\pgfqpoint{6.494937in}{1.491894in}}%
\pgfpathlineto{\pgfqpoint{6.610341in}{1.660212in}}%
\pgfpathlineto{\pgfqpoint{6.721899in}{1.817501in}}%
\pgfpathlineto{\pgfqpoint{6.833457in}{1.969466in}}%
\pgfpathlineto{\pgfqpoint{6.945015in}{2.116107in}}%
\pgfpathlineto{\pgfqpoint{7.052726in}{2.252643in}}%
\pgfpathlineto{\pgfqpoint{7.160437in}{2.384220in}}%
\pgfpathlineto{\pgfqpoint{7.268148in}{2.510838in}}%
\pgfpathlineto{\pgfqpoint{7.372012in}{2.628240in}}%
\pgfpathlineto{\pgfqpoint{7.475876in}{2.741032in}}%
\pgfpathlineto{\pgfqpoint{7.579741in}{2.849216in}}%
\pgfpathlineto{\pgfqpoint{7.679758in}{2.949039in}}%
\pgfpathlineto{\pgfqpoint{7.779775in}{3.044590in}}%
\pgfpathlineto{\pgfqpoint{7.879793in}{3.135869in}}%
\pgfpathlineto{\pgfqpoint{7.979810in}{3.222877in}}%
\pgfpathlineto{\pgfqpoint{8.075981in}{3.302511in}}%
\pgfpathlineto{\pgfqpoint{8.172151in}{3.378196in}}%
\pgfpathlineto{\pgfqpoint{8.268322in}{3.449932in}}%
\pgfpathlineto{\pgfqpoint{8.364492in}{3.517721in}}%
\pgfpathlineto{\pgfqpoint{8.456816in}{3.579083in}}%
\pgfpathlineto{\pgfqpoint{8.549140in}{3.636807in}}%
\pgfpathlineto{\pgfqpoint{8.641464in}{3.690892in}}%
\pgfpathlineto{\pgfqpoint{8.733787in}{3.741339in}}%
\pgfpathlineto{\pgfqpoint{8.826111in}{3.788147in}}%
\pgfpathlineto{\pgfqpoint{8.914588in}{3.829590in}}%
\pgfpathlineto{\pgfqpoint{9.003065in}{3.867692in}}%
\pgfpathlineto{\pgfqpoint{9.091542in}{3.902453in}}%
\pgfpathlineto{\pgfqpoint{9.180019in}{3.933871in}}%
\pgfpathlineto{\pgfqpoint{9.268496in}{3.961948in}}%
\pgfpathlineto{\pgfqpoint{9.356972in}{3.986683in}}%
\pgfpathlineto{\pgfqpoint{9.441603in}{4.007215in}}%
\pgfpathlineto{\pgfqpoint{9.526233in}{4.024690in}}%
\pgfpathlineto{\pgfqpoint{9.610863in}{4.039108in}}%
\pgfpathlineto{\pgfqpoint{9.695493in}{4.050468in}}%
\pgfpathlineto{\pgfqpoint{9.780123in}{4.058770in}}%
\pgfpathlineto{\pgfqpoint{9.864753in}{4.064015in}}%
\pgfpathlineto{\pgfqpoint{9.949383in}{4.066202in}}%
\pgfpathlineto{\pgfqpoint{9.987851in}{4.066185in}}%
\pgfpathlineto{\pgfqpoint{9.987851in}{4.066185in}}%
\pgfusepath{stroke}%
\end{pgfscope}%
\begin{pgfscope}%
\pgfpathrectangle{\pgfqpoint{5.952727in}{0.790446in}}{\pgfqpoint{4.227273in}{3.431818in}} %
\pgfusepath{clip}%
\pgfsetbuttcap%
\pgfsetmiterjoin%
\definecolor{currentfill}{rgb}{1.000000,0.000000,0.000000}%
\pgfsetfillcolor{currentfill}%
\pgfsetlinewidth{1.003750pt}%
\definecolor{currentstroke}{rgb}{1.000000,0.000000,0.000000}%
\pgfsetstrokecolor{currentstroke}%
\pgfsetdash{}{0pt}%
\pgfsys@defobject{currentmarker}{\pgfqpoint{-0.041667in}{-0.041667in}}{\pgfqpoint{0.041667in}{0.041667in}}{%
\pgfpathmoveto{\pgfqpoint{-0.041667in}{-0.041667in}}%
\pgfpathlineto{\pgfqpoint{0.041667in}{-0.041667in}}%
\pgfpathlineto{\pgfqpoint{0.041667in}{0.041667in}}%
\pgfpathlineto{\pgfqpoint{-0.041667in}{0.041667in}}%
\pgfpathclose%
\pgfusepath{stroke,fill}%
}%
\begin{pgfscope}%
\pgfsys@transformshift{6.144876in}{0.946438in}%
\pgfsys@useobject{currentmarker}{}%
\end{pgfscope}%
\begin{pgfscope}%
\pgfsys@transformshift{6.529558in}{1.542988in}%
\pgfsys@useobject{currentmarker}{}%
\end{pgfscope}%
\begin{pgfscope}%
\pgfsys@transformshift{6.914240in}{2.076186in}%
\pgfsys@useobject{currentmarker}{}%
\end{pgfscope}%
\begin{pgfscope}%
\pgfsys@transformshift{7.298923in}{2.546104in}%
\pgfsys@useobject{currentmarker}{}%
\end{pgfscope}%
\begin{pgfscope}%
\pgfsys@transformshift{7.683605in}{2.952793in}%
\pgfsys@useobject{currentmarker}{}%
\end{pgfscope}%
\begin{pgfscope}%
\pgfsys@transformshift{8.068287in}{3.296285in}%
\pgfsys@useobject{currentmarker}{}%
\end{pgfscope}%
\begin{pgfscope}%
\pgfsys@transformshift{8.452969in}{3.576599in}%
\pgfsys@useobject{currentmarker}{}%
\end{pgfscope}%
\begin{pgfscope}%
\pgfsys@transformshift{8.837651in}{3.793742in}%
\pgfsys@useobject{currentmarker}{}%
\end{pgfscope}%
\begin{pgfscope}%
\pgfsys@transformshift{9.222334in}{3.947716in}%
\pgfsys@useobject{currentmarker}{}%
\end{pgfscope}%
\begin{pgfscope}%
\pgfsys@transformshift{9.607016in}{4.038519in}%
\pgfsys@useobject{currentmarker}{}%
\end{pgfscope}%
\end{pgfscope}%
\begin{pgfscope}%
\pgfpathrectangle{\pgfqpoint{5.952727in}{0.790446in}}{\pgfqpoint{4.227273in}{3.431818in}} %
\pgfusepath{clip}%
\pgfsetrectcap%
\pgfsetroundjoin%
\pgfsetlinewidth{1.505625pt}%
\definecolor{currentstroke}{rgb}{0.000000,0.000000,1.000000}%
\pgfsetstrokecolor{currentstroke}%
\pgfsetdash{}{0pt}%
\pgfpathmoveto{\pgfqpoint{6.144876in}{0.946438in}}%
\pgfpathlineto{\pgfqpoint{6.264128in}{1.138149in}}%
\pgfpathlineto{\pgfqpoint{6.379532in}{1.317870in}}%
\pgfpathlineto{\pgfqpoint{6.494937in}{1.491881in}}%
\pgfpathlineto{\pgfqpoint{6.606495in}{1.654664in}}%
\pgfpathlineto{\pgfqpoint{6.718053in}{1.812112in}}%
\pgfpathlineto{\pgfqpoint{6.829610in}{1.964225in}}%
\pgfpathlineto{\pgfqpoint{6.941168in}{2.111003in}}%
\pgfpathlineto{\pgfqpoint{7.048879in}{2.247658in}}%
\pgfpathlineto{\pgfqpoint{7.156590in}{2.379340in}}%
\pgfpathlineto{\pgfqpoint{7.264301in}{2.506049in}}%
\pgfpathlineto{\pgfqpoint{7.368165in}{2.623524in}}%
\pgfpathlineto{\pgfqpoint{7.472030in}{2.736375in}}%
\pgfpathlineto{\pgfqpoint{7.575894in}{2.844603in}}%
\pgfpathlineto{\pgfqpoint{7.675911in}{2.944453in}}%
\pgfpathlineto{\pgfqpoint{7.775929in}{3.040016in}}%
\pgfpathlineto{\pgfqpoint{7.875946in}{3.131293in}}%
\pgfpathlineto{\pgfqpoint{7.975963in}{3.218282in}}%
\pgfpathlineto{\pgfqpoint{8.072134in}{3.297884in}}%
\pgfpathlineto{\pgfqpoint{8.168304in}{3.373522in}}%
\pgfpathlineto{\pgfqpoint{8.264475in}{3.445197in}}%
\pgfpathlineto{\pgfqpoint{8.360646in}{3.512910in}}%
\pgfpathlineto{\pgfqpoint{8.452969in}{3.574185in}}%
\pgfpathlineto{\pgfqpoint{8.545293in}{3.631809in}}%
\pgfpathlineto{\pgfqpoint{8.637617in}{3.685780in}}%
\pgfpathlineto{\pgfqpoint{8.729940in}{3.736100in}}%
\pgfpathlineto{\pgfqpoint{8.818417in}{3.780896in}}%
\pgfpathlineto{\pgfqpoint{8.906894in}{3.822338in}}%
\pgfpathlineto{\pgfqpoint{8.995371in}{3.860426in}}%
\pgfpathlineto{\pgfqpoint{9.083848in}{3.895160in}}%
\pgfpathlineto{\pgfqpoint{9.172325in}{3.926541in}}%
\pgfpathlineto{\pgfqpoint{9.260802in}{3.954568in}}%
\pgfpathlineto{\pgfqpoint{9.349279in}{3.979241in}}%
\pgfpathlineto{\pgfqpoint{9.433909in}{3.999703in}}%
\pgfpathlineto{\pgfqpoint{9.518539in}{4.017096in}}%
\pgfpathlineto{\pgfqpoint{9.603169in}{4.031422in}}%
\pgfpathlineto{\pgfqpoint{9.687799in}{4.042678in}}%
\pgfpathlineto{\pgfqpoint{9.772429in}{4.050867in}}%
\pgfpathlineto{\pgfqpoint{9.857059in}{4.055987in}}%
\pgfpathlineto{\pgfqpoint{9.941689in}{4.058039in}}%
\pgfpathlineto{\pgfqpoint{9.987851in}{4.057864in}}%
\pgfpathlineto{\pgfqpoint{9.987851in}{4.057864in}}%
\pgfusepath{stroke}%
\end{pgfscope}%
\begin{pgfscope}%
\pgfpathrectangle{\pgfqpoint{5.952727in}{0.790446in}}{\pgfqpoint{4.227273in}{3.431818in}} %
\pgfusepath{clip}%
\pgfsetbuttcap%
\pgfsetroundjoin%
\definecolor{currentfill}{rgb}{0.000000,0.000000,1.000000}%
\pgfsetfillcolor{currentfill}%
\pgfsetlinewidth{1.003750pt}%
\definecolor{currentstroke}{rgb}{0.000000,0.000000,1.000000}%
\pgfsetstrokecolor{currentstroke}%
\pgfsetdash{}{0pt}%
\pgfsys@defobject{currentmarker}{\pgfqpoint{-0.041667in}{-0.041667in}}{\pgfqpoint{0.041667in}{0.041667in}}{%
\pgfpathmoveto{\pgfqpoint{0.000000in}{-0.041667in}}%
\pgfpathcurveto{\pgfqpoint{0.011050in}{-0.041667in}}{\pgfqpoint{0.021649in}{-0.037276in}}{\pgfqpoint{0.029463in}{-0.029463in}}%
\pgfpathcurveto{\pgfqpoint{0.037276in}{-0.021649in}}{\pgfqpoint{0.041667in}{-0.011050in}}{\pgfqpoint{0.041667in}{0.000000in}}%
\pgfpathcurveto{\pgfqpoint{0.041667in}{0.011050in}}{\pgfqpoint{0.037276in}{0.021649in}}{\pgfqpoint{0.029463in}{0.029463in}}%
\pgfpathcurveto{\pgfqpoint{0.021649in}{0.037276in}}{\pgfqpoint{0.011050in}{0.041667in}}{\pgfqpoint{0.000000in}{0.041667in}}%
\pgfpathcurveto{\pgfqpoint{-0.011050in}{0.041667in}}{\pgfqpoint{-0.021649in}{0.037276in}}{\pgfqpoint{-0.029463in}{0.029463in}}%
\pgfpathcurveto{\pgfqpoint{-0.037276in}{0.021649in}}{\pgfqpoint{-0.041667in}{0.011050in}}{\pgfqpoint{-0.041667in}{0.000000in}}%
\pgfpathcurveto{\pgfqpoint{-0.041667in}{-0.011050in}}{\pgfqpoint{-0.037276in}{-0.021649in}}{\pgfqpoint{-0.029463in}{-0.029463in}}%
\pgfpathcurveto{\pgfqpoint{-0.021649in}{-0.037276in}}{\pgfqpoint{-0.011050in}{-0.041667in}}{\pgfqpoint{0.000000in}{-0.041667in}}%
\pgfpathclose%
\pgfusepath{stroke,fill}%
}%
\begin{pgfscope}%
\pgfsys@transformshift{6.144876in}{0.946438in}%
\pgfsys@useobject{currentmarker}{}%
\end{pgfscope}%
\begin{pgfscope}%
\pgfsys@transformshift{6.529558in}{1.542971in}%
\pgfsys@useobject{currentmarker}{}%
\end{pgfscope}%
\begin{pgfscope}%
\pgfsys@transformshift{6.914240in}{2.076062in}%
\pgfsys@useobject{currentmarker}{}%
\end{pgfscope}%
\begin{pgfscope}%
\pgfsys@transformshift{7.298923in}{2.545721in}%
\pgfsys@useobject{currentmarker}{}%
\end{pgfscope}%
\begin{pgfscope}%
\pgfsys@transformshift{7.683605in}{2.951956in}%
\pgfsys@useobject{currentmarker}{}%
\end{pgfscope}%
\begin{pgfscope}%
\pgfsys@transformshift{8.068287in}{3.294776in}%
\pgfsys@useobject{currentmarker}{}%
\end{pgfscope}%
\begin{pgfscope}%
\pgfsys@transformshift{8.452969in}{3.574185in}%
\pgfsys@useobject{currentmarker}{}%
\end{pgfscope}%
\begin{pgfscope}%
\pgfsys@transformshift{8.837651in}{3.790190in}%
\pgfsys@useobject{currentmarker}{}%
\end{pgfscope}%
\begin{pgfscope}%
\pgfsys@transformshift{9.222334in}{3.942794in}%
\pgfsys@useobject{currentmarker}{}%
\end{pgfscope}%
\begin{pgfscope}%
\pgfsys@transformshift{9.607016in}{4.032000in}%
\pgfsys@useobject{currentmarker}{}%
\end{pgfscope}%
\end{pgfscope}%
\begin{pgfscope}%
\pgfpathrectangle{\pgfqpoint{5.952727in}{0.790446in}}{\pgfqpoint{4.227273in}{3.431818in}} %
\pgfusepath{clip}%
\pgfsetbuttcap%
\pgfsetroundjoin%
\pgfsetlinewidth{1.505625pt}%
\definecolor{currentstroke}{rgb}{0.000000,0.750000,0.750000}%
\pgfsetstrokecolor{currentstroke}%
\pgfsetdash{{9.600000pt}{2.400000pt}{1.500000pt}{2.400000pt}}{0.000000pt}%
\pgfpathmoveto{\pgfqpoint{6.144876in}{0.946438in}}%
\pgfpathlineto{\pgfqpoint{6.264128in}{1.138149in}}%
\pgfpathlineto{\pgfqpoint{6.379532in}{1.317870in}}%
\pgfpathlineto{\pgfqpoint{6.494937in}{1.491880in}}%
\pgfpathlineto{\pgfqpoint{6.606495in}{1.654661in}}%
\pgfpathlineto{\pgfqpoint{6.718053in}{1.812106in}}%
\pgfpathlineto{\pgfqpoint{6.829610in}{1.964215in}}%
\pgfpathlineto{\pgfqpoint{6.941168in}{2.110987in}}%
\pgfpathlineto{\pgfqpoint{7.048879in}{2.247635in}}%
\pgfpathlineto{\pgfqpoint{7.156590in}{2.379308in}}%
\pgfpathlineto{\pgfqpoint{7.264301in}{2.506007in}}%
\pgfpathlineto{\pgfqpoint{7.368165in}{2.623469in}}%
\pgfpathlineto{\pgfqpoint{7.472030in}{2.736306in}}%
\pgfpathlineto{\pgfqpoint{7.575894in}{2.844517in}}%
\pgfpathlineto{\pgfqpoint{7.675911in}{2.944349in}}%
\pgfpathlineto{\pgfqpoint{7.775929in}{3.039892in}}%
\pgfpathlineto{\pgfqpoint{7.875946in}{3.131145in}}%
\pgfpathlineto{\pgfqpoint{7.975963in}{3.218109in}}%
\pgfpathlineto{\pgfqpoint{8.072134in}{3.297684in}}%
\pgfpathlineto{\pgfqpoint{8.168304in}{3.373292in}}%
\pgfpathlineto{\pgfqpoint{8.264475in}{3.444936in}}%
\pgfpathlineto{\pgfqpoint{8.360646in}{3.512613in}}%
\pgfpathlineto{\pgfqpoint{8.452969in}{3.573853in}}%
\pgfpathlineto{\pgfqpoint{8.545293in}{3.631438in}}%
\pgfpathlineto{\pgfqpoint{8.637617in}{3.685369in}}%
\pgfpathlineto{\pgfqpoint{8.729940in}{3.735644in}}%
\pgfpathlineto{\pgfqpoint{8.818417in}{3.780396in}}%
\pgfpathlineto{\pgfqpoint{8.906894in}{3.821791in}}%
\pgfpathlineto{\pgfqpoint{8.995371in}{3.859830in}}%
\pgfpathlineto{\pgfqpoint{9.083848in}{3.894512in}}%
\pgfpathlineto{\pgfqpoint{9.172325in}{3.925838in}}%
\pgfpathlineto{\pgfqpoint{9.260802in}{3.953808in}}%
\pgfpathlineto{\pgfqpoint{9.349279in}{3.978421in}}%
\pgfpathlineto{\pgfqpoint{9.433909in}{3.998823in}}%
\pgfpathlineto{\pgfqpoint{9.518539in}{4.016155in}}%
\pgfpathlineto{\pgfqpoint{9.603169in}{4.030415in}}%
\pgfpathlineto{\pgfqpoint{9.687799in}{4.041605in}}%
\pgfpathlineto{\pgfqpoint{9.772429in}{4.049723in}}%
\pgfpathlineto{\pgfqpoint{9.857059in}{4.054771in}}%
\pgfpathlineto{\pgfqpoint{9.941689in}{4.056748in}}%
\pgfpathlineto{\pgfqpoint{9.987851in}{4.056532in}}%
\pgfpathlineto{\pgfqpoint{9.987851in}{4.056532in}}%
\pgfusepath{stroke}%
\end{pgfscope}%
\begin{pgfscope}%
\pgfpathrectangle{\pgfqpoint{5.952727in}{0.790446in}}{\pgfqpoint{4.227273in}{3.431818in}} %
\pgfusepath{clip}%
\pgfsetbuttcap%
\pgfsetmiterjoin%
\definecolor{currentfill}{rgb}{0.000000,0.750000,0.750000}%
\pgfsetfillcolor{currentfill}%
\pgfsetlinewidth{1.003750pt}%
\definecolor{currentstroke}{rgb}{0.000000,0.750000,0.750000}%
\pgfsetstrokecolor{currentstroke}%
\pgfsetdash{}{0pt}%
\pgfsys@defobject{currentmarker}{\pgfqpoint{-0.041667in}{-0.041667in}}{\pgfqpoint{0.041667in}{0.041667in}}{%
\pgfpathmoveto{\pgfqpoint{-0.000000in}{-0.041667in}}%
\pgfpathlineto{\pgfqpoint{0.041667in}{0.041667in}}%
\pgfpathlineto{\pgfqpoint{-0.041667in}{0.041667in}}%
\pgfpathclose%
\pgfusepath{stroke,fill}%
}%
\begin{pgfscope}%
\pgfsys@transformshift{6.144876in}{0.946438in}%
\pgfsys@useobject{currentmarker}{}%
\end{pgfscope}%
\begin{pgfscope}%
\pgfsys@transformshift{6.529558in}{1.542969in}%
\pgfsys@useobject{currentmarker}{}%
\end{pgfscope}%
\begin{pgfscope}%
\pgfsys@transformshift{6.914240in}{2.076048in}%
\pgfsys@useobject{currentmarker}{}%
\end{pgfscope}%
\begin{pgfscope}%
\pgfsys@transformshift{7.298923in}{2.545675in}%
\pgfsys@useobject{currentmarker}{}%
\end{pgfscope}%
\begin{pgfscope}%
\pgfsys@transformshift{7.683605in}{2.951851in}%
\pgfsys@useobject{currentmarker}{}%
\end{pgfscope}%
\begin{pgfscope}%
\pgfsys@transformshift{8.068287in}{3.294577in}%
\pgfsys@useobject{currentmarker}{}%
\end{pgfscope}%
\begin{pgfscope}%
\pgfsys@transformshift{8.452969in}{3.573853in}%
\pgfsys@useobject{currentmarker}{}%
\end{pgfscope}%
\begin{pgfscope}%
\pgfsys@transformshift{8.837651in}{3.789681in}%
\pgfsys@useobject{currentmarker}{}%
\end{pgfscope}%
\begin{pgfscope}%
\pgfsys@transformshift{9.222334in}{3.942059in}%
\pgfsys@useobject{currentmarker}{}%
\end{pgfscope}%
\begin{pgfscope}%
\pgfsys@transformshift{9.607016in}{4.030990in}%
\pgfsys@useobject{currentmarker}{}%
\end{pgfscope}%
\end{pgfscope}%
\begin{pgfscope}%
\pgfpathrectangle{\pgfqpoint{5.952727in}{0.790446in}}{\pgfqpoint{4.227273in}{3.431818in}} %
\pgfusepath{clip}%
\pgfsetbuttcap%
\pgfsetroundjoin%
\pgfsetlinewidth{1.505625pt}%
\definecolor{currentstroke}{rgb}{0.000000,0.000000,0.000000}%
\pgfsetstrokecolor{currentstroke}%
\pgfsetdash{{1.500000pt}{2.475000pt}}{0.000000pt}%
\pgfpathmoveto{\pgfqpoint{6.144876in}{0.946438in}}%
\pgfpathlineto{\pgfqpoint{6.264128in}{1.138149in}}%
\pgfpathlineto{\pgfqpoint{6.379532in}{1.317870in}}%
\pgfpathlineto{\pgfqpoint{6.494937in}{1.491880in}}%
\pgfpathlineto{\pgfqpoint{6.606495in}{1.654661in}}%
\pgfpathlineto{\pgfqpoint{6.718053in}{1.812106in}}%
\pgfpathlineto{\pgfqpoint{6.829610in}{1.964214in}}%
\pgfpathlineto{\pgfqpoint{6.941168in}{2.110986in}}%
\pgfpathlineto{\pgfqpoint{7.048879in}{2.247633in}}%
\pgfpathlineto{\pgfqpoint{7.156590in}{2.379305in}}%
\pgfpathlineto{\pgfqpoint{7.264301in}{2.506003in}}%
\pgfpathlineto{\pgfqpoint{7.368165in}{2.623464in}}%
\pgfpathlineto{\pgfqpoint{7.472030in}{2.736299in}}%
\pgfpathlineto{\pgfqpoint{7.575894in}{2.844509in}}%
\pgfpathlineto{\pgfqpoint{7.675911in}{2.944338in}}%
\pgfpathlineto{\pgfqpoint{7.775929in}{3.039879in}}%
\pgfpathlineto{\pgfqpoint{7.875946in}{3.131130in}}%
\pgfpathlineto{\pgfqpoint{7.975963in}{3.218091in}}%
\pgfpathlineto{\pgfqpoint{8.072134in}{3.297663in}}%
\pgfpathlineto{\pgfqpoint{8.168304in}{3.373269in}}%
\pgfpathlineto{\pgfqpoint{8.264475in}{3.444909in}}%
\pgfpathlineto{\pgfqpoint{8.360646in}{3.512583in}}%
\pgfpathlineto{\pgfqpoint{8.452969in}{3.573819in}}%
\pgfpathlineto{\pgfqpoint{8.545293in}{3.631400in}}%
\pgfpathlineto{\pgfqpoint{8.637617in}{3.685326in}}%
\pgfpathlineto{\pgfqpoint{8.729940in}{3.735597in}}%
\pgfpathlineto{\pgfqpoint{8.818417in}{3.780344in}}%
\pgfpathlineto{\pgfqpoint{8.906894in}{3.821734in}}%
\pgfpathlineto{\pgfqpoint{8.995371in}{3.859768in}}%
\pgfpathlineto{\pgfqpoint{9.083848in}{3.894445in}}%
\pgfpathlineto{\pgfqpoint{9.172325in}{3.925765in}}%
\pgfpathlineto{\pgfqpoint{9.260802in}{3.953728in}}%
\pgfpathlineto{\pgfqpoint{9.349279in}{3.978335in}}%
\pgfpathlineto{\pgfqpoint{9.433909in}{3.998731in}}%
\pgfpathlineto{\pgfqpoint{9.518539in}{4.016056in}}%
\pgfpathlineto{\pgfqpoint{9.603169in}{4.030310in}}%
\pgfpathlineto{\pgfqpoint{9.687799in}{4.041492in}}%
\pgfpathlineto{\pgfqpoint{9.772429in}{4.049604in}}%
\pgfpathlineto{\pgfqpoint{9.857059in}{4.054644in}}%
\pgfpathlineto{\pgfqpoint{9.941689in}{4.056613in}}%
\pgfpathlineto{\pgfqpoint{9.987851in}{4.056393in}}%
\pgfpathlineto{\pgfqpoint{9.987851in}{4.056393in}}%
\pgfusepath{stroke}%
\end{pgfscope}%
\begin{pgfscope}%
\pgfpathrectangle{\pgfqpoint{5.952727in}{0.790446in}}{\pgfqpoint{4.227273in}{3.431818in}} %
\pgfusepath{clip}%
\pgfsetbuttcap%
\pgfsetroundjoin%
\definecolor{currentfill}{rgb}{0.000000,0.000000,0.000000}%
\pgfsetfillcolor{currentfill}%
\pgfsetlinewidth{1.003750pt}%
\definecolor{currentstroke}{rgb}{0.000000,0.000000,0.000000}%
\pgfsetstrokecolor{currentstroke}%
\pgfsetdash{}{0pt}%
\pgfsys@defobject{currentmarker}{\pgfqpoint{-0.041667in}{-0.041667in}}{\pgfqpoint{0.041667in}{0.041667in}}{%
\pgfpathmoveto{\pgfqpoint{-0.041667in}{0.000000in}}%
\pgfpathlineto{\pgfqpoint{0.041667in}{0.000000in}}%
\pgfpathmoveto{\pgfqpoint{0.000000in}{-0.041667in}}%
\pgfpathlineto{\pgfqpoint{0.000000in}{0.041667in}}%
\pgfusepath{stroke,fill}%
}%
\begin{pgfscope}%
\pgfsys@transformshift{6.144876in}{0.946438in}%
\pgfsys@useobject{currentmarker}{}%
\end{pgfscope}%
\begin{pgfscope}%
\pgfsys@transformshift{6.529558in}{1.542969in}%
\pgfsys@useobject{currentmarker}{}%
\end{pgfscope}%
\begin{pgfscope}%
\pgfsys@transformshift{6.914240in}{2.076047in}%
\pgfsys@useobject{currentmarker}{}%
\end{pgfscope}%
\begin{pgfscope}%
\pgfsys@transformshift{7.298923in}{2.545670in}%
\pgfsys@useobject{currentmarker}{}%
\end{pgfscope}%
\begin{pgfscope}%
\pgfsys@transformshift{7.683605in}{2.951840in}%
\pgfsys@useobject{currentmarker}{}%
\end{pgfscope}%
\begin{pgfscope}%
\pgfsys@transformshift{8.068287in}{3.294556in}%
\pgfsys@useobject{currentmarker}{}%
\end{pgfscope}%
\begin{pgfscope}%
\pgfsys@transformshift{8.452969in}{3.573819in}%
\pgfsys@useobject{currentmarker}{}%
\end{pgfscope}%
\begin{pgfscope}%
\pgfsys@transformshift{8.837651in}{3.789628in}%
\pgfsys@useobject{currentmarker}{}%
\end{pgfscope}%
\begin{pgfscope}%
\pgfsys@transformshift{9.222334in}{3.941983in}%
\pgfsys@useobject{currentmarker}{}%
\end{pgfscope}%
\begin{pgfscope}%
\pgfsys@transformshift{9.607016in}{4.030885in}%
\pgfsys@useobject{currentmarker}{}%
\end{pgfscope}%
\end{pgfscope}%
\begin{pgfscope}%
\pgfsetrectcap%
\pgfsetmiterjoin%
\pgfsetlinewidth{0.803000pt}%
\definecolor{currentstroke}{rgb}{0.000000,0.000000,0.000000}%
\pgfsetstrokecolor{currentstroke}%
\pgfsetdash{}{0pt}%
\pgfpathmoveto{\pgfqpoint{5.952727in}{0.790446in}}%
\pgfpathlineto{\pgfqpoint{5.952727in}{4.222264in}}%
\pgfusepath{stroke}%
\end{pgfscope}%
\begin{pgfscope}%
\pgfsetrectcap%
\pgfsetmiterjoin%
\pgfsetlinewidth{0.803000pt}%
\definecolor{currentstroke}{rgb}{0.000000,0.000000,0.000000}%
\pgfsetstrokecolor{currentstroke}%
\pgfsetdash{}{0pt}%
\pgfpathmoveto{\pgfqpoint{10.180000in}{0.790446in}}%
\pgfpathlineto{\pgfqpoint{10.180000in}{4.222264in}}%
\pgfusepath{stroke}%
\end{pgfscope}%
\begin{pgfscope}%
\pgfsetrectcap%
\pgfsetmiterjoin%
\pgfsetlinewidth{0.803000pt}%
\definecolor{currentstroke}{rgb}{0.000000,0.000000,0.000000}%
\pgfsetstrokecolor{currentstroke}%
\pgfsetdash{}{0pt}%
\pgfpathmoveto{\pgfqpoint{5.952727in}{0.790446in}}%
\pgfpathlineto{\pgfqpoint{10.180000in}{0.790446in}}%
\pgfusepath{stroke}%
\end{pgfscope}%
\begin{pgfscope}%
\pgfsetrectcap%
\pgfsetmiterjoin%
\pgfsetlinewidth{0.803000pt}%
\definecolor{currentstroke}{rgb}{0.000000,0.000000,0.000000}%
\pgfsetstrokecolor{currentstroke}%
\pgfsetdash{}{0pt}%
\pgfpathmoveto{\pgfqpoint{5.952727in}{4.222264in}}%
\pgfpathlineto{\pgfqpoint{10.180000in}{4.222264in}}%
\pgfusepath{stroke}%
\end{pgfscope}%
\begin{pgfscope}%
\pgfsetbuttcap%
\pgfsetmiterjoin%
\definecolor{currentfill}{rgb}{1.000000,1.000000,1.000000}%
\pgfsetfillcolor{currentfill}%
\pgfsetfillopacity{0.800000}%
\pgfsetlinewidth{1.003750pt}%
\definecolor{currentstroke}{rgb}{0.800000,0.800000,0.800000}%
\pgfsetstrokecolor{currentstroke}%
\pgfsetstrokeopacity{0.800000}%
\pgfsetdash{}{0pt}%
\pgfpathmoveto{\pgfqpoint{7.514813in}{0.859890in}}%
\pgfpathlineto{\pgfqpoint{8.617914in}{0.859890in}}%
\pgfpathquadraticcurveto{\pgfqpoint{8.645692in}{0.859890in}}{\pgfqpoint{8.645692in}{0.887668in}}%
\pgfpathlineto{\pgfqpoint{8.645692in}{1.893065in}}%
\pgfpathquadraticcurveto{\pgfqpoint{8.645692in}{1.920843in}}{\pgfqpoint{8.617914in}{1.920843in}}%
\pgfpathlineto{\pgfqpoint{7.514813in}{1.920843in}}%
\pgfpathquadraticcurveto{\pgfqpoint{7.487035in}{1.920843in}}{\pgfqpoint{7.487035in}{1.893065in}}%
\pgfpathlineto{\pgfqpoint{7.487035in}{0.887668in}}%
\pgfpathquadraticcurveto{\pgfqpoint{7.487035in}{0.859890in}}{\pgfqpoint{7.514813in}{0.859890in}}%
\pgfpathclose%
\pgfusepath{stroke,fill}%
\end{pgfscope}%
\begin{pgfscope}%
\pgftext[x=7.764725in,y=1.759764in,left,base]{\rmfamily\fontsize{10.000000}{12.000000}\selectfont \(\displaystyle \alpha\) = 0.01}%
\end{pgfscope}%
\begin{pgfscope}%
\pgfsetbuttcap%
\pgfsetroundjoin%
\pgfsetlinewidth{1.505625pt}%
\definecolor{currentstroke}{rgb}{1.000000,0.000000,0.000000}%
\pgfsetstrokecolor{currentstroke}%
\pgfsetdash{{5.550000pt}{2.400000pt}}{0.000000pt}%
\pgfpathmoveto{\pgfqpoint{7.542591in}{1.604518in}}%
\pgfpathlineto{\pgfqpoint{7.820368in}{1.604518in}}%
\pgfusepath{stroke}%
\end{pgfscope}%
\begin{pgfscope}%
\pgfsetbuttcap%
\pgfsetmiterjoin%
\definecolor{currentfill}{rgb}{1.000000,0.000000,0.000000}%
\pgfsetfillcolor{currentfill}%
\pgfsetlinewidth{1.003750pt}%
\definecolor{currentstroke}{rgb}{1.000000,0.000000,0.000000}%
\pgfsetstrokecolor{currentstroke}%
\pgfsetdash{}{0pt}%
\pgfsys@defobject{currentmarker}{\pgfqpoint{-0.041667in}{-0.041667in}}{\pgfqpoint{0.041667in}{0.041667in}}{%
\pgfpathmoveto{\pgfqpoint{-0.041667in}{-0.041667in}}%
\pgfpathlineto{\pgfqpoint{0.041667in}{-0.041667in}}%
\pgfpathlineto{\pgfqpoint{0.041667in}{0.041667in}}%
\pgfpathlineto{\pgfqpoint{-0.041667in}{0.041667in}}%
\pgfpathclose%
\pgfusepath{stroke,fill}%
}%
\begin{pgfscope}%
\pgfsys@transformshift{7.681480in}{1.604518in}%
\pgfsys@useobject{currentmarker}{}%
\end{pgfscope}%
\end{pgfscope}%
\begin{pgfscope}%
\pgftext[x=7.931480in,y=1.555907in,left,base]{\rmfamily\fontsize{10.000000}{12.000000}\selectfont \(\displaystyle \epsilon\) = 1}%
\end{pgfscope}%
\begin{pgfscope}%
\pgfsetrectcap%
\pgfsetroundjoin%
\pgfsetlinewidth{1.505625pt}%
\definecolor{currentstroke}{rgb}{0.000000,0.000000,1.000000}%
\pgfsetstrokecolor{currentstroke}%
\pgfsetdash{}{0pt}%
\pgfpathmoveto{\pgfqpoint{7.542591in}{1.400661in}}%
\pgfpathlineto{\pgfqpoint{7.820368in}{1.400661in}}%
\pgfusepath{stroke}%
\end{pgfscope}%
\begin{pgfscope}%
\pgfsetbuttcap%
\pgfsetroundjoin%
\definecolor{currentfill}{rgb}{0.000000,0.000000,1.000000}%
\pgfsetfillcolor{currentfill}%
\pgfsetlinewidth{1.003750pt}%
\definecolor{currentstroke}{rgb}{0.000000,0.000000,1.000000}%
\pgfsetstrokecolor{currentstroke}%
\pgfsetdash{}{0pt}%
\pgfsys@defobject{currentmarker}{\pgfqpoint{-0.041667in}{-0.041667in}}{\pgfqpoint{0.041667in}{0.041667in}}{%
\pgfpathmoveto{\pgfqpoint{0.000000in}{-0.041667in}}%
\pgfpathcurveto{\pgfqpoint{0.011050in}{-0.041667in}}{\pgfqpoint{0.021649in}{-0.037276in}}{\pgfqpoint{0.029463in}{-0.029463in}}%
\pgfpathcurveto{\pgfqpoint{0.037276in}{-0.021649in}}{\pgfqpoint{0.041667in}{-0.011050in}}{\pgfqpoint{0.041667in}{0.000000in}}%
\pgfpathcurveto{\pgfqpoint{0.041667in}{0.011050in}}{\pgfqpoint{0.037276in}{0.021649in}}{\pgfqpoint{0.029463in}{0.029463in}}%
\pgfpathcurveto{\pgfqpoint{0.021649in}{0.037276in}}{\pgfqpoint{0.011050in}{0.041667in}}{\pgfqpoint{0.000000in}{0.041667in}}%
\pgfpathcurveto{\pgfqpoint{-0.011050in}{0.041667in}}{\pgfqpoint{-0.021649in}{0.037276in}}{\pgfqpoint{-0.029463in}{0.029463in}}%
\pgfpathcurveto{\pgfqpoint{-0.037276in}{0.021649in}}{\pgfqpoint{-0.041667in}{0.011050in}}{\pgfqpoint{-0.041667in}{0.000000in}}%
\pgfpathcurveto{\pgfqpoint{-0.041667in}{-0.011050in}}{\pgfqpoint{-0.037276in}{-0.021649in}}{\pgfqpoint{-0.029463in}{-0.029463in}}%
\pgfpathcurveto{\pgfqpoint{-0.021649in}{-0.037276in}}{\pgfqpoint{-0.011050in}{-0.041667in}}{\pgfqpoint{0.000000in}{-0.041667in}}%
\pgfpathclose%
\pgfusepath{stroke,fill}%
}%
\begin{pgfscope}%
\pgfsys@transformshift{7.681480in}{1.400661in}%
\pgfsys@useobject{currentmarker}{}%
\end{pgfscope}%
\end{pgfscope}%
\begin{pgfscope}%
\pgftext[x=7.931480in,y=1.352050in,left,base]{\rmfamily\fontsize{10.000000}{12.000000}\selectfont \(\displaystyle \epsilon\) = 0.1}%
\end{pgfscope}%
\begin{pgfscope}%
\pgfsetbuttcap%
\pgfsetroundjoin%
\pgfsetlinewidth{1.505625pt}%
\definecolor{currentstroke}{rgb}{0.000000,0.750000,0.750000}%
\pgfsetstrokecolor{currentstroke}%
\pgfsetdash{{9.600000pt}{2.400000pt}{1.500000pt}{2.400000pt}}{0.000000pt}%
\pgfpathmoveto{\pgfqpoint{7.542591in}{1.196804in}}%
\pgfpathlineto{\pgfqpoint{7.820368in}{1.196804in}}%
\pgfusepath{stroke}%
\end{pgfscope}%
\begin{pgfscope}%
\pgfsetbuttcap%
\pgfsetmiterjoin%
\definecolor{currentfill}{rgb}{0.000000,0.750000,0.750000}%
\pgfsetfillcolor{currentfill}%
\pgfsetlinewidth{1.003750pt}%
\definecolor{currentstroke}{rgb}{0.000000,0.750000,0.750000}%
\pgfsetstrokecolor{currentstroke}%
\pgfsetdash{}{0pt}%
\pgfsys@defobject{currentmarker}{\pgfqpoint{-0.041667in}{-0.041667in}}{\pgfqpoint{0.041667in}{0.041667in}}{%
\pgfpathmoveto{\pgfqpoint{-0.000000in}{-0.041667in}}%
\pgfpathlineto{\pgfqpoint{0.041667in}{0.041667in}}%
\pgfpathlineto{\pgfqpoint{-0.041667in}{0.041667in}}%
\pgfpathclose%
\pgfusepath{stroke,fill}%
}%
\begin{pgfscope}%
\pgfsys@transformshift{7.681480in}{1.196804in}%
\pgfsys@useobject{currentmarker}{}%
\end{pgfscope}%
\end{pgfscope}%
\begin{pgfscope}%
\pgftext[x=7.931480in,y=1.148193in,left,base]{\rmfamily\fontsize{10.000000}{12.000000}\selectfont \(\displaystyle \epsilon\) = 0.01}%
\end{pgfscope}%
\begin{pgfscope}%
\pgfsetbuttcap%
\pgfsetroundjoin%
\pgfsetlinewidth{1.505625pt}%
\definecolor{currentstroke}{rgb}{0.000000,0.000000,0.000000}%
\pgfsetstrokecolor{currentstroke}%
\pgfsetdash{{1.500000pt}{2.475000pt}}{0.000000pt}%
\pgfpathmoveto{\pgfqpoint{7.542591in}{0.992947in}}%
\pgfpathlineto{\pgfqpoint{7.820368in}{0.992947in}}%
\pgfusepath{stroke}%
\end{pgfscope}%
\begin{pgfscope}%
\pgfsetbuttcap%
\pgfsetroundjoin%
\definecolor{currentfill}{rgb}{0.000000,0.000000,0.000000}%
\pgfsetfillcolor{currentfill}%
\pgfsetlinewidth{1.003750pt}%
\definecolor{currentstroke}{rgb}{0.000000,0.000000,0.000000}%
\pgfsetstrokecolor{currentstroke}%
\pgfsetdash{}{0pt}%
\pgfsys@defobject{currentmarker}{\pgfqpoint{-0.041667in}{-0.041667in}}{\pgfqpoint{0.041667in}{0.041667in}}{%
\pgfpathmoveto{\pgfqpoint{-0.041667in}{0.000000in}}%
\pgfpathlineto{\pgfqpoint{0.041667in}{0.000000in}}%
\pgfpathmoveto{\pgfqpoint{0.000000in}{-0.041667in}}%
\pgfpathlineto{\pgfqpoint{0.000000in}{0.041667in}}%
\pgfusepath{stroke,fill}%
}%
\begin{pgfscope}%
\pgfsys@transformshift{7.681480in}{0.992947in}%
\pgfsys@useobject{currentmarker}{}%
\end{pgfscope}%
\end{pgfscope}%
\begin{pgfscope}%
\pgftext[x=7.931480in,y=0.944336in,left,base]{\rmfamily\fontsize{10.000000}{12.000000}\selectfont \(\displaystyle \epsilon\) = 0.001}%
\end{pgfscope}%
\begin{pgfscope}%
\pgftext[x=5.380000in,y=0.140446in,,base]{\rmfamily\fontsize{14.000000}{16.800000}\selectfont \(\displaystyle \bar{t}\)}%
\end{pgfscope}%
\begin{pgfscope}%
\pgftext[x=0.247732in,y=4.489115in,left,base,rotate=90.000000]{\rmfamily\fontsize{14.000000}{16.800000}\selectfont \(\displaystyle \bar{y}\)}%
\end{pgfscope}%
\end{pgfpicture}%
\makeatother%
\endgroup%
}
    \caption{Text.}
    \label{fig:short_times}
\end{figure}

\subsubsection{Inertial Electro-Viscous Limit}
Asymptotic estimate of the trajectory. With $\epsilon = {\mathbb{E}\mbox{u}}_+$, where $\epsilon$ is a small parameter, and $\beta = \mathbb{D}\mbox{g}$,
\begin{eqnarray*}
&\bar{y}(\bar{t}) = \bar{t} - \frac{\bar{t}^{2}}{2} + \epsilon \left(\frac{\bar{t}^{3}}{3} \left(1 + \beta\right) + \frac{\bar{t}^{4}}{12} \left(-1 - \beta\right) - \frac{\beta \bar{t}^{2}}{2}\right)&  \\
&+ \epsilon^{2} \left(\frac{\bar{t}^{4}}{12} \left(-3 - 3 \beta - 4 \beta^{2}\right) + \frac{\bar{t}^{5}}{60} \left(11 + 10 \beta + 8 \beta^{2}\right)+ \frac{\bar{t}^{6}}{360} \left(-11 - 10 \beta - 8 \beta^{2}\right) + \frac{\beta^{2} \bar{t}^{3}}{3}\right)& \\
 &+ \mathcal{O}(\epsilon^3)&
\end{eqnarray*}

\newpage
\begin{figure}[htb]
    \centering
    %% Creator: Matplotlib, PGF backend
%%
%% To include the figure in your LaTeX document, write
%%   \input{<filename>.pgf}
%%
%% Make sure the required packages are loaded in your preamble
%%   \usepackage{pgf}
%%
%% Figures using additional raster images can only be included by \input if
%% they are in the same directory as the main LaTeX file. For loading figures
%% from other directories you can use the `import` package
%%   \usepackage{import}
%% and then include the figures with
%%   \import{<path to file>}{<filename>.pgf}
%%
%% Matplotlib used the following preamble
%%   \usepackage{fontspec}
%%   \setmainfont{DejaVu Serif}
%%   \setsansfont{DejaVu Sans}
%%   \setmonofont{DejaVu Sans Mono}
%%
\begingroup%
\makeatletter%
\begin{pgfpicture}%
\pgfpathrectangle{\pgfpointorigin}{\pgfqpoint{5.566298in}{3.823000in}}%
\pgfusepath{use as bounding box, clip}%
\begin{pgfscope}%
\pgfsetbuttcap%
\pgfsetmiterjoin%
\definecolor{currentfill}{rgb}{1.000000,1.000000,1.000000}%
\pgfsetfillcolor{currentfill}%
\pgfsetlinewidth{0.000000pt}%
\definecolor{currentstroke}{rgb}{1.000000,1.000000,1.000000}%
\pgfsetstrokecolor{currentstroke}%
\pgfsetdash{}{0pt}%
\pgfpathmoveto{\pgfqpoint{0.000000in}{0.000000in}}%
\pgfpathlineto{\pgfqpoint{5.566298in}{0.000000in}}%
\pgfpathlineto{\pgfqpoint{5.566298in}{3.823000in}}%
\pgfpathlineto{\pgfqpoint{0.000000in}{3.823000in}}%
\pgfpathclose%
\pgfusepath{fill}%
\end{pgfscope}%
\begin{pgfscope}%
\pgfsetbuttcap%
\pgfsetmiterjoin%
\definecolor{currentfill}{rgb}{1.000000,1.000000,1.000000}%
\pgfsetfillcolor{currentfill}%
\pgfsetlinewidth{0.000000pt}%
\definecolor{currentstroke}{rgb}{0.000000,0.000000,0.000000}%
\pgfsetstrokecolor{currentstroke}%
\pgfsetstrokeopacity{0.000000}%
\pgfsetdash{}{0pt}%
\pgfpathmoveto{\pgfqpoint{0.691184in}{0.629134in}}%
\pgfpathlineto{\pgfqpoint{5.341184in}{0.629134in}}%
\pgfpathlineto{\pgfqpoint{5.341184in}{3.649134in}}%
\pgfpathlineto{\pgfqpoint{0.691184in}{3.649134in}}%
\pgfpathclose%
\pgfusepath{fill}%
\end{pgfscope}%
\begin{pgfscope}%
\pgfsetbuttcap%
\pgfsetroundjoin%
\definecolor{currentfill}{rgb}{0.000000,0.000000,0.000000}%
\pgfsetfillcolor{currentfill}%
\pgfsetlinewidth{0.803000pt}%
\definecolor{currentstroke}{rgb}{0.000000,0.000000,0.000000}%
\pgfsetstrokecolor{currentstroke}%
\pgfsetdash{}{0pt}%
\pgfsys@defobject{currentmarker}{\pgfqpoint{0.000000in}{-0.048611in}}{\pgfqpoint{0.000000in}{0.000000in}}{%
\pgfpathmoveto{\pgfqpoint{0.000000in}{0.000000in}}%
\pgfpathlineto{\pgfqpoint{0.000000in}{-0.048611in}}%
\pgfusepath{stroke,fill}%
}%
\begin{pgfscope}%
\pgfsys@transformshift{0.912402in}{0.629134in}%
\pgfsys@useobject{currentmarker}{}%
\end{pgfscope}%
\end{pgfscope}%
\begin{pgfscope}%
\pgftext[x=0.912402in,y=0.531912in,,top]{\rmfamily\fontsize{14.000000}{16.800000}\selectfont \(\displaystyle 0.0\)}%
\end{pgfscope}%
\begin{pgfscope}%
\pgfsetbuttcap%
\pgfsetroundjoin%
\definecolor{currentfill}{rgb}{0.000000,0.000000,0.000000}%
\pgfsetfillcolor{currentfill}%
\pgfsetlinewidth{0.803000pt}%
\definecolor{currentstroke}{rgb}{0.000000,0.000000,0.000000}%
\pgfsetstrokecolor{currentstroke}%
\pgfsetdash{}{0pt}%
\pgfsys@defobject{currentmarker}{\pgfqpoint{0.000000in}{-0.048611in}}{\pgfqpoint{0.000000in}{0.000000in}}{%
\pgfpathmoveto{\pgfqpoint{0.000000in}{0.000000in}}%
\pgfpathlineto{\pgfqpoint{0.000000in}{-0.048611in}}%
\pgfusepath{stroke,fill}%
}%
\begin{pgfscope}%
\pgfsys@transformshift{1.798158in}{0.629134in}%
\pgfsys@useobject{currentmarker}{}%
\end{pgfscope}%
\end{pgfscope}%
\begin{pgfscope}%
\pgftext[x=1.798158in,y=0.531912in,,top]{\rmfamily\fontsize{14.000000}{16.800000}\selectfont \(\displaystyle 0.2\)}%
\end{pgfscope}%
\begin{pgfscope}%
\pgfsetbuttcap%
\pgfsetroundjoin%
\definecolor{currentfill}{rgb}{0.000000,0.000000,0.000000}%
\pgfsetfillcolor{currentfill}%
\pgfsetlinewidth{0.803000pt}%
\definecolor{currentstroke}{rgb}{0.000000,0.000000,0.000000}%
\pgfsetstrokecolor{currentstroke}%
\pgfsetdash{}{0pt}%
\pgfsys@defobject{currentmarker}{\pgfqpoint{0.000000in}{-0.048611in}}{\pgfqpoint{0.000000in}{0.000000in}}{%
\pgfpathmoveto{\pgfqpoint{0.000000in}{0.000000in}}%
\pgfpathlineto{\pgfqpoint{0.000000in}{-0.048611in}}%
\pgfusepath{stroke,fill}%
}%
\begin{pgfscope}%
\pgfsys@transformshift{2.683915in}{0.629134in}%
\pgfsys@useobject{currentmarker}{}%
\end{pgfscope}%
\end{pgfscope}%
\begin{pgfscope}%
\pgftext[x=2.683915in,y=0.531912in,,top]{\rmfamily\fontsize{14.000000}{16.800000}\selectfont \(\displaystyle 0.4\)}%
\end{pgfscope}%
\begin{pgfscope}%
\pgfsetbuttcap%
\pgfsetroundjoin%
\definecolor{currentfill}{rgb}{0.000000,0.000000,0.000000}%
\pgfsetfillcolor{currentfill}%
\pgfsetlinewidth{0.803000pt}%
\definecolor{currentstroke}{rgb}{0.000000,0.000000,0.000000}%
\pgfsetstrokecolor{currentstroke}%
\pgfsetdash{}{0pt}%
\pgfsys@defobject{currentmarker}{\pgfqpoint{0.000000in}{-0.048611in}}{\pgfqpoint{0.000000in}{0.000000in}}{%
\pgfpathmoveto{\pgfqpoint{0.000000in}{0.000000in}}%
\pgfpathlineto{\pgfqpoint{0.000000in}{-0.048611in}}%
\pgfusepath{stroke,fill}%
}%
\begin{pgfscope}%
\pgfsys@transformshift{3.569671in}{0.629134in}%
\pgfsys@useobject{currentmarker}{}%
\end{pgfscope}%
\end{pgfscope}%
\begin{pgfscope}%
\pgftext[x=3.569671in,y=0.531912in,,top]{\rmfamily\fontsize{14.000000}{16.800000}\selectfont \(\displaystyle 0.6\)}%
\end{pgfscope}%
\begin{pgfscope}%
\pgfsetbuttcap%
\pgfsetroundjoin%
\definecolor{currentfill}{rgb}{0.000000,0.000000,0.000000}%
\pgfsetfillcolor{currentfill}%
\pgfsetlinewidth{0.803000pt}%
\definecolor{currentstroke}{rgb}{0.000000,0.000000,0.000000}%
\pgfsetstrokecolor{currentstroke}%
\pgfsetdash{}{0pt}%
\pgfsys@defobject{currentmarker}{\pgfqpoint{0.000000in}{-0.048611in}}{\pgfqpoint{0.000000in}{0.000000in}}{%
\pgfpathmoveto{\pgfqpoint{0.000000in}{0.000000in}}%
\pgfpathlineto{\pgfqpoint{0.000000in}{-0.048611in}}%
\pgfusepath{stroke,fill}%
}%
\begin{pgfscope}%
\pgfsys@transformshift{4.455428in}{0.629134in}%
\pgfsys@useobject{currentmarker}{}%
\end{pgfscope}%
\end{pgfscope}%
\begin{pgfscope}%
\pgftext[x=4.455428in,y=0.531912in,,top]{\rmfamily\fontsize{14.000000}{16.800000}\selectfont \(\displaystyle 0.8\)}%
\end{pgfscope}%
\begin{pgfscope}%
\pgfsetbuttcap%
\pgfsetroundjoin%
\definecolor{currentfill}{rgb}{0.000000,0.000000,0.000000}%
\pgfsetfillcolor{currentfill}%
\pgfsetlinewidth{0.803000pt}%
\definecolor{currentstroke}{rgb}{0.000000,0.000000,0.000000}%
\pgfsetstrokecolor{currentstroke}%
\pgfsetdash{}{0pt}%
\pgfsys@defobject{currentmarker}{\pgfqpoint{0.000000in}{-0.048611in}}{\pgfqpoint{0.000000in}{0.000000in}}{%
\pgfpathmoveto{\pgfqpoint{0.000000in}{0.000000in}}%
\pgfpathlineto{\pgfqpoint{0.000000in}{-0.048611in}}%
\pgfusepath{stroke,fill}%
}%
\begin{pgfscope}%
\pgfsys@transformshift{5.341184in}{0.629134in}%
\pgfsys@useobject{currentmarker}{}%
\end{pgfscope}%
\end{pgfscope}%
\begin{pgfscope}%
\pgftext[x=5.341184in,y=0.531912in,,top]{\rmfamily\fontsize{14.000000}{16.800000}\selectfont \(\displaystyle 1.0\)}%
\end{pgfscope}%
\begin{pgfscope}%
\pgftext[x=3.016184in,y=0.288178in,,top]{\rmfamily\fontsize{14.000000}{16.800000}\selectfont \(\displaystyle t^*\)}%
\end{pgfscope}%
\begin{pgfscope}%
\pgfsetbuttcap%
\pgfsetroundjoin%
\definecolor{currentfill}{rgb}{0.000000,0.000000,0.000000}%
\pgfsetfillcolor{currentfill}%
\pgfsetlinewidth{0.803000pt}%
\definecolor{currentstroke}{rgb}{0.000000,0.000000,0.000000}%
\pgfsetstrokecolor{currentstroke}%
\pgfsetdash{}{0pt}%
\pgfsys@defobject{currentmarker}{\pgfqpoint{-0.048611in}{0.000000in}}{\pgfqpoint{0.000000in}{0.000000in}}{%
\pgfpathmoveto{\pgfqpoint{0.000000in}{0.000000in}}%
\pgfpathlineto{\pgfqpoint{-0.048611in}{0.000000in}}%
\pgfusepath{stroke,fill}%
}%
\begin{pgfscope}%
\pgfsys@transformshift{0.691184in}{0.629134in}%
\pgfsys@useobject{currentmarker}{}%
\end{pgfscope}%
\end{pgfscope}%
\begin{pgfscope}%
\pgftext[x=0.343734in,y=0.555268in,left,base]{\rmfamily\fontsize{14.000000}{16.800000}\selectfont \(\displaystyle 0.0\)}%
\end{pgfscope}%
\begin{pgfscope}%
\pgfsetbuttcap%
\pgfsetroundjoin%
\definecolor{currentfill}{rgb}{0.000000,0.000000,0.000000}%
\pgfsetfillcolor{currentfill}%
\pgfsetlinewidth{0.803000pt}%
\definecolor{currentstroke}{rgb}{0.000000,0.000000,0.000000}%
\pgfsetstrokecolor{currentstroke}%
\pgfsetdash{}{0pt}%
\pgfsys@defobject{currentmarker}{\pgfqpoint{-0.048611in}{0.000000in}}{\pgfqpoint{0.000000in}{0.000000in}}{%
\pgfpathmoveto{\pgfqpoint{0.000000in}{0.000000in}}%
\pgfpathlineto{\pgfqpoint{-0.048611in}{0.000000in}}%
\pgfusepath{stroke,fill}%
}%
\begin{pgfscope}%
\pgfsys@transformshift{0.691184in}{1.060562in}%
\pgfsys@useobject{currentmarker}{}%
\end{pgfscope}%
\end{pgfscope}%
\begin{pgfscope}%
\pgftext[x=0.343734in,y=0.986696in,left,base]{\rmfamily\fontsize{14.000000}{16.800000}\selectfont \(\displaystyle 0.1\)}%
\end{pgfscope}%
\begin{pgfscope}%
\pgfsetbuttcap%
\pgfsetroundjoin%
\definecolor{currentfill}{rgb}{0.000000,0.000000,0.000000}%
\pgfsetfillcolor{currentfill}%
\pgfsetlinewidth{0.803000pt}%
\definecolor{currentstroke}{rgb}{0.000000,0.000000,0.000000}%
\pgfsetstrokecolor{currentstroke}%
\pgfsetdash{}{0pt}%
\pgfsys@defobject{currentmarker}{\pgfqpoint{-0.048611in}{0.000000in}}{\pgfqpoint{0.000000in}{0.000000in}}{%
\pgfpathmoveto{\pgfqpoint{0.000000in}{0.000000in}}%
\pgfpathlineto{\pgfqpoint{-0.048611in}{0.000000in}}%
\pgfusepath{stroke,fill}%
}%
\begin{pgfscope}%
\pgfsys@transformshift{0.691184in}{1.491991in}%
\pgfsys@useobject{currentmarker}{}%
\end{pgfscope}%
\end{pgfscope}%
\begin{pgfscope}%
\pgftext[x=0.343734in,y=1.418125in,left,base]{\rmfamily\fontsize{14.000000}{16.800000}\selectfont \(\displaystyle 0.2\)}%
\end{pgfscope}%
\begin{pgfscope}%
\pgfsetbuttcap%
\pgfsetroundjoin%
\definecolor{currentfill}{rgb}{0.000000,0.000000,0.000000}%
\pgfsetfillcolor{currentfill}%
\pgfsetlinewidth{0.803000pt}%
\definecolor{currentstroke}{rgb}{0.000000,0.000000,0.000000}%
\pgfsetstrokecolor{currentstroke}%
\pgfsetdash{}{0pt}%
\pgfsys@defobject{currentmarker}{\pgfqpoint{-0.048611in}{0.000000in}}{\pgfqpoint{0.000000in}{0.000000in}}{%
\pgfpathmoveto{\pgfqpoint{0.000000in}{0.000000in}}%
\pgfpathlineto{\pgfqpoint{-0.048611in}{0.000000in}}%
\pgfusepath{stroke,fill}%
}%
\begin{pgfscope}%
\pgfsys@transformshift{0.691184in}{1.923420in}%
\pgfsys@useobject{currentmarker}{}%
\end{pgfscope}%
\end{pgfscope}%
\begin{pgfscope}%
\pgftext[x=0.343734in,y=1.849553in,left,base]{\rmfamily\fontsize{14.000000}{16.800000}\selectfont \(\displaystyle 0.3\)}%
\end{pgfscope}%
\begin{pgfscope}%
\pgfsetbuttcap%
\pgfsetroundjoin%
\definecolor{currentfill}{rgb}{0.000000,0.000000,0.000000}%
\pgfsetfillcolor{currentfill}%
\pgfsetlinewidth{0.803000pt}%
\definecolor{currentstroke}{rgb}{0.000000,0.000000,0.000000}%
\pgfsetstrokecolor{currentstroke}%
\pgfsetdash{}{0pt}%
\pgfsys@defobject{currentmarker}{\pgfqpoint{-0.048611in}{0.000000in}}{\pgfqpoint{0.000000in}{0.000000in}}{%
\pgfpathmoveto{\pgfqpoint{0.000000in}{0.000000in}}%
\pgfpathlineto{\pgfqpoint{-0.048611in}{0.000000in}}%
\pgfusepath{stroke,fill}%
}%
\begin{pgfscope}%
\pgfsys@transformshift{0.691184in}{2.354848in}%
\pgfsys@useobject{currentmarker}{}%
\end{pgfscope}%
\end{pgfscope}%
\begin{pgfscope}%
\pgftext[x=0.343734in,y=2.280982in,left,base]{\rmfamily\fontsize{14.000000}{16.800000}\selectfont \(\displaystyle 0.4\)}%
\end{pgfscope}%
\begin{pgfscope}%
\pgfsetbuttcap%
\pgfsetroundjoin%
\definecolor{currentfill}{rgb}{0.000000,0.000000,0.000000}%
\pgfsetfillcolor{currentfill}%
\pgfsetlinewidth{0.803000pt}%
\definecolor{currentstroke}{rgb}{0.000000,0.000000,0.000000}%
\pgfsetstrokecolor{currentstroke}%
\pgfsetdash{}{0pt}%
\pgfsys@defobject{currentmarker}{\pgfqpoint{-0.048611in}{0.000000in}}{\pgfqpoint{0.000000in}{0.000000in}}{%
\pgfpathmoveto{\pgfqpoint{0.000000in}{0.000000in}}%
\pgfpathlineto{\pgfqpoint{-0.048611in}{0.000000in}}%
\pgfusepath{stroke,fill}%
}%
\begin{pgfscope}%
\pgfsys@transformshift{0.691184in}{2.786277in}%
\pgfsys@useobject{currentmarker}{}%
\end{pgfscope}%
\end{pgfscope}%
\begin{pgfscope}%
\pgftext[x=0.343734in,y=2.712411in,left,base]{\rmfamily\fontsize{14.000000}{16.800000}\selectfont \(\displaystyle 0.5\)}%
\end{pgfscope}%
\begin{pgfscope}%
\pgfsetbuttcap%
\pgfsetroundjoin%
\definecolor{currentfill}{rgb}{0.000000,0.000000,0.000000}%
\pgfsetfillcolor{currentfill}%
\pgfsetlinewidth{0.803000pt}%
\definecolor{currentstroke}{rgb}{0.000000,0.000000,0.000000}%
\pgfsetstrokecolor{currentstroke}%
\pgfsetdash{}{0pt}%
\pgfsys@defobject{currentmarker}{\pgfqpoint{-0.048611in}{0.000000in}}{\pgfqpoint{0.000000in}{0.000000in}}{%
\pgfpathmoveto{\pgfqpoint{0.000000in}{0.000000in}}%
\pgfpathlineto{\pgfqpoint{-0.048611in}{0.000000in}}%
\pgfusepath{stroke,fill}%
}%
\begin{pgfscope}%
\pgfsys@transformshift{0.691184in}{3.217705in}%
\pgfsys@useobject{currentmarker}{}%
\end{pgfscope}%
\end{pgfscope}%
\begin{pgfscope}%
\pgftext[x=0.343734in,y=3.143839in,left,base]{\rmfamily\fontsize{14.000000}{16.800000}\selectfont \(\displaystyle 0.6\)}%
\end{pgfscope}%
\begin{pgfscope}%
\pgfsetbuttcap%
\pgfsetroundjoin%
\definecolor{currentfill}{rgb}{0.000000,0.000000,0.000000}%
\pgfsetfillcolor{currentfill}%
\pgfsetlinewidth{0.803000pt}%
\definecolor{currentstroke}{rgb}{0.000000,0.000000,0.000000}%
\pgfsetstrokecolor{currentstroke}%
\pgfsetdash{}{0pt}%
\pgfsys@defobject{currentmarker}{\pgfqpoint{-0.048611in}{0.000000in}}{\pgfqpoint{0.000000in}{0.000000in}}{%
\pgfpathmoveto{\pgfqpoint{0.000000in}{0.000000in}}%
\pgfpathlineto{\pgfqpoint{-0.048611in}{0.000000in}}%
\pgfusepath{stroke,fill}%
}%
\begin{pgfscope}%
\pgfsys@transformshift{0.691184in}{3.649134in}%
\pgfsys@useobject{currentmarker}{}%
\end{pgfscope}%
\end{pgfscope}%
\begin{pgfscope}%
\pgftext[x=0.343734in,y=3.575268in,left,base]{\rmfamily\fontsize{14.000000}{16.800000}\selectfont \(\displaystyle 0.7\)}%
\end{pgfscope}%
\begin{pgfscope}%
\pgftext[x=0.288178in,y=2.139134in,,bottom,rotate=90.000000]{\rmfamily\fontsize{14.000000}{16.800000}\selectfont \(\displaystyle y^*\)}%
\end{pgfscope}%
\begin{pgfscope}%
\pgfpathrectangle{\pgfqpoint{0.691184in}{0.629134in}}{\pgfqpoint{4.650000in}{3.020000in}}%
\pgfusepath{clip}%
\pgfsetbuttcap%
\pgfsetroundjoin%
\pgfsetlinewidth{1.505625pt}%
\definecolor{currentstroke}{rgb}{1.000000,0.000000,0.000000}%
\pgfsetstrokecolor{currentstroke}%
\pgfsetdash{{5.550000pt}{2.400000pt}}{0.000000pt}%
\pgfpathmoveto{\pgfqpoint{0.912402in}{0.629134in}}%
\pgfpathlineto{\pgfqpoint{0.969976in}{0.684858in}}%
\pgfpathlineto{\pgfqpoint{1.031979in}{0.744074in}}%
\pgfpathlineto{\pgfqpoint{1.093982in}{0.802486in}}%
\pgfpathlineto{\pgfqpoint{1.155985in}{0.860117in}}%
\pgfpathlineto{\pgfqpoint{1.217988in}{0.916985in}}%
\pgfpathlineto{\pgfqpoint{1.279991in}{0.973110in}}%
\pgfpathlineto{\pgfqpoint{1.341994in}{1.028508in}}%
\pgfpathlineto{\pgfqpoint{1.403997in}{1.083198in}}%
\pgfpathlineto{\pgfqpoint{1.470428in}{1.141026in}}%
\pgfpathlineto{\pgfqpoint{1.536860in}{1.198077in}}%
\pgfpathlineto{\pgfqpoint{1.603292in}{1.254370in}}%
\pgfpathlineto{\pgfqpoint{1.669723in}{1.309922in}}%
\pgfpathlineto{\pgfqpoint{1.736155in}{1.364749in}}%
\pgfpathlineto{\pgfqpoint{1.802587in}{1.418866in}}%
\pgfpathlineto{\pgfqpoint{1.869019in}{1.472289in}}%
\pgfpathlineto{\pgfqpoint{1.935450in}{1.525031in}}%
\pgfpathlineto{\pgfqpoint{2.006311in}{1.580555in}}%
\pgfpathlineto{\pgfqpoint{2.077171in}{1.635335in}}%
\pgfpathlineto{\pgfqpoint{2.148032in}{1.689387in}}%
\pgfpathlineto{\pgfqpoint{2.218892in}{1.742725in}}%
\pgfpathlineto{\pgfqpoint{2.289753in}{1.795363in}}%
\pgfpathlineto{\pgfqpoint{2.360614in}{1.847314in}}%
\pgfpathlineto{\pgfqpoint{2.431474in}{1.898590in}}%
\pgfpathlineto{\pgfqpoint{2.502335in}{1.949204in}}%
\pgfpathlineto{\pgfqpoint{2.573195in}{1.999167in}}%
\pgfpathlineto{\pgfqpoint{2.648484in}{2.051553in}}%
\pgfpathlineto{\pgfqpoint{2.723774in}{2.103230in}}%
\pgfpathlineto{\pgfqpoint{2.799063in}{2.154212in}}%
\pgfpathlineto{\pgfqpoint{2.874352in}{2.204512in}}%
\pgfpathlineto{\pgfqpoint{2.949642in}{2.254142in}}%
\pgfpathlineto{\pgfqpoint{3.024931in}{2.303114in}}%
\pgfpathlineto{\pgfqpoint{3.100220in}{2.351441in}}%
\pgfpathlineto{\pgfqpoint{3.175509in}{2.399135in}}%
\pgfpathlineto{\pgfqpoint{3.255228in}{2.448960in}}%
\pgfpathlineto{\pgfqpoint{3.334946in}{2.498105in}}%
\pgfpathlineto{\pgfqpoint{3.414664in}{2.546585in}}%
\pgfpathlineto{\pgfqpoint{3.494382in}{2.594416in}}%
\pgfpathlineto{\pgfqpoint{3.574100in}{2.641615in}}%
\pgfpathlineto{\pgfqpoint{3.658247in}{2.690767in}}%
\pgfpathlineto{\pgfqpoint{3.742394in}{2.739254in}}%
\pgfpathlineto{\pgfqpoint{3.826540in}{2.787095in}}%
\pgfpathlineto{\pgfqpoint{3.910687in}{2.834313in}}%
\pgfpathlineto{\pgfqpoint{3.999263in}{2.883366in}}%
\pgfpathlineto{\pgfqpoint{4.087839in}{2.931781in}}%
\pgfpathlineto{\pgfqpoint{4.180843in}{2.981958in}}%
\pgfpathlineto{\pgfqpoint{4.273847in}{3.031494in}}%
\pgfpathlineto{\pgfqpoint{4.371281in}{3.082740in}}%
\pgfpathlineto{\pgfqpoint{4.468714in}{3.133360in}}%
\pgfpathlineto{\pgfqpoint{4.570576in}{3.185657in}}%
\pgfpathlineto{\pgfqpoint{4.676867in}{3.239599in}}%
\pgfpathlineto{\pgfqpoint{4.787586in}{3.295162in}}%
\pgfpathlineto{\pgfqpoint{4.902735in}{3.352336in}}%
\pgfpathlineto{\pgfqpoint{5.026740in}{3.413285in}}%
\pgfpathlineto{\pgfqpoint{5.159604in}{3.477959in}}%
\pgfpathlineto{\pgfqpoint{5.301325in}{3.546329in}}%
\pgfpathlineto{\pgfqpoint{5.336755in}{3.563333in}}%
\pgfpathlineto{\pgfqpoint{5.336755in}{3.563333in}}%
\pgfusepath{stroke}%
\end{pgfscope}%
\begin{pgfscope}%
\pgfpathrectangle{\pgfqpoint{0.691184in}{0.629134in}}{\pgfqpoint{4.650000in}{3.020000in}}%
\pgfusepath{clip}%
\pgfsetbuttcap%
\pgfsetmiterjoin%
\definecolor{currentfill}{rgb}{1.000000,0.000000,0.000000}%
\pgfsetfillcolor{currentfill}%
\pgfsetlinewidth{1.003750pt}%
\definecolor{currentstroke}{rgb}{1.000000,0.000000,0.000000}%
\pgfsetstrokecolor{currentstroke}%
\pgfsetdash{}{0pt}%
\pgfsys@defobject{currentmarker}{\pgfqpoint{-0.041667in}{-0.041667in}}{\pgfqpoint{0.041667in}{0.041667in}}{%
\pgfpathmoveto{\pgfqpoint{-0.041667in}{-0.041667in}}%
\pgfpathlineto{\pgfqpoint{0.041667in}{-0.041667in}}%
\pgfpathlineto{\pgfqpoint{0.041667in}{0.041667in}}%
\pgfpathlineto{\pgfqpoint{-0.041667in}{0.041667in}}%
\pgfpathclose%
\pgfusepath{stroke,fill}%
}%
\begin{pgfscope}%
\pgfsys@transformshift{0.912402in}{0.629134in}%
\pgfsys@useobject{currentmarker}{}%
\end{pgfscope}%
\begin{pgfscope}%
\pgfsys@transformshift{1.355280in}{1.040287in}%
\pgfsys@useobject{currentmarker}{}%
\end{pgfscope}%
\begin{pgfscope}%
\pgfsys@transformshift{1.798158in}{1.415280in}%
\pgfsys@useobject{currentmarker}{}%
\end{pgfscope}%
\begin{pgfscope}%
\pgfsys@transformshift{2.241036in}{1.759249in}%
\pgfsys@useobject{currentmarker}{}%
\end{pgfscope}%
\begin{pgfscope}%
\pgfsys@transformshift{2.683915in}{2.075959in}%
\pgfsys@useobject{currentmarker}{}%
\end{pgfscope}%
\begin{pgfscope}%
\pgfsys@transformshift{3.126793in}{2.368346in}%
\pgfsys@useobject{currentmarker}{}%
\end{pgfscope}%
\begin{pgfscope}%
\pgfsys@transformshift{3.569671in}{2.639009in}%
\pgfsys@useobject{currentmarker}{}%
\end{pgfscope}%
\begin{pgfscope}%
\pgfsys@transformshift{4.012549in}{2.890669in}%
\pgfsys@useobject{currentmarker}{}%
\end{pgfscope}%
\begin{pgfscope}%
\pgfsys@transformshift{4.455428in}{3.126492in}%
\pgfsys@useobject{currentmarker}{}%
\end{pgfscope}%
\begin{pgfscope}%
\pgfsys@transformshift{4.898306in}{3.350148in}%
\pgfsys@useobject{currentmarker}{}%
\end{pgfscope}%
\end{pgfscope}%
\begin{pgfscope}%
\pgfpathrectangle{\pgfqpoint{0.691184in}{0.629134in}}{\pgfqpoint{4.650000in}{3.020000in}}%
\pgfusepath{clip}%
\pgfsetrectcap%
\pgfsetroundjoin%
\pgfsetlinewidth{1.505625pt}%
\definecolor{currentstroke}{rgb}{0.000000,0.000000,1.000000}%
\pgfsetstrokecolor{currentstroke}%
\pgfsetdash{}{0pt}%
\pgfpathmoveto{\pgfqpoint{0.912402in}{0.629134in}}%
\pgfpathlineto{\pgfqpoint{0.969976in}{0.684856in}}%
\pgfpathlineto{\pgfqpoint{1.027550in}{0.739859in}}%
\pgfpathlineto{\pgfqpoint{1.089553in}{0.798298in}}%
\pgfpathlineto{\pgfqpoint{1.151556in}{0.855923in}}%
\pgfpathlineto{\pgfqpoint{1.213559in}{0.912744in}}%
\pgfpathlineto{\pgfqpoint{1.275562in}{0.968773in}}%
\pgfpathlineto{\pgfqpoint{1.337565in}{1.024019in}}%
\pgfpathlineto{\pgfqpoint{1.399568in}{1.078492in}}%
\pgfpathlineto{\pgfqpoint{1.461571in}{1.132200in}}%
\pgfpathlineto{\pgfqpoint{1.523574in}{1.185153in}}%
\pgfpathlineto{\pgfqpoint{1.585577in}{1.237359in}}%
\pgfpathlineto{\pgfqpoint{1.647580in}{1.288828in}}%
\pgfpathlineto{\pgfqpoint{1.709583in}{1.339566in}}%
\pgfpathlineto{\pgfqpoint{1.771585in}{1.389582in}}%
\pgfpathlineto{\pgfqpoint{1.833588in}{1.438884in}}%
\pgfpathlineto{\pgfqpoint{1.895591in}{1.487479in}}%
\pgfpathlineto{\pgfqpoint{1.957594in}{1.535375in}}%
\pgfpathlineto{\pgfqpoint{2.019597in}{1.582577in}}%
\pgfpathlineto{\pgfqpoint{2.081600in}{1.629094in}}%
\pgfpathlineto{\pgfqpoint{2.143603in}{1.674931in}}%
\pgfpathlineto{\pgfqpoint{2.205606in}{1.720095in}}%
\pgfpathlineto{\pgfqpoint{2.267609in}{1.764593in}}%
\pgfpathlineto{\pgfqpoint{2.329612in}{1.808430in}}%
\pgfpathlineto{\pgfqpoint{2.391615in}{1.851613in}}%
\pgfpathlineto{\pgfqpoint{2.453618in}{1.894146in}}%
\pgfpathlineto{\pgfqpoint{2.515621in}{1.936037in}}%
\pgfpathlineto{\pgfqpoint{2.577624in}{1.977289in}}%
\pgfpathlineto{\pgfqpoint{2.639627in}{2.017908in}}%
\pgfpathlineto{\pgfqpoint{2.701630in}{2.057900in}}%
\pgfpathlineto{\pgfqpoint{2.763633in}{2.097269in}}%
\pgfpathlineto{\pgfqpoint{2.825636in}{2.136021in}}%
\pgfpathlineto{\pgfqpoint{2.887639in}{2.174159in}}%
\pgfpathlineto{\pgfqpoint{2.954070in}{2.214347in}}%
\pgfpathlineto{\pgfqpoint{3.020502in}{2.253842in}}%
\pgfpathlineto{\pgfqpoint{3.086934in}{2.292650in}}%
\pgfpathlineto{\pgfqpoint{3.153366in}{2.330775in}}%
\pgfpathlineto{\pgfqpoint{3.219797in}{2.368223in}}%
\pgfpathlineto{\pgfqpoint{3.286229in}{2.404999in}}%
\pgfpathlineto{\pgfqpoint{3.352661in}{2.441107in}}%
\pgfpathlineto{\pgfqpoint{3.419092in}{2.476553in}}%
\pgfpathlineto{\pgfqpoint{3.485524in}{2.511341in}}%
\pgfpathlineto{\pgfqpoint{3.551956in}{2.545476in}}%
\pgfpathlineto{\pgfqpoint{3.618388in}{2.578961in}}%
\pgfpathlineto{\pgfqpoint{3.684819in}{2.611802in}}%
\pgfpathlineto{\pgfqpoint{3.751251in}{2.644002in}}%
\pgfpathlineto{\pgfqpoint{3.817683in}{2.675565in}}%
\pgfpathlineto{\pgfqpoint{3.884115in}{2.706495in}}%
\pgfpathlineto{\pgfqpoint{3.950546in}{2.736797in}}%
\pgfpathlineto{\pgfqpoint{4.016978in}{2.766474in}}%
\pgfpathlineto{\pgfqpoint{4.083410in}{2.795530in}}%
\pgfpathlineto{\pgfqpoint{4.149842in}{2.823968in}}%
\pgfpathlineto{\pgfqpoint{4.216273in}{2.851792in}}%
\pgfpathlineto{\pgfqpoint{4.282705in}{2.879006in}}%
\pgfpathlineto{\pgfqpoint{4.349137in}{2.905612in}}%
\pgfpathlineto{\pgfqpoint{4.415569in}{2.931615in}}%
\pgfpathlineto{\pgfqpoint{4.482000in}{2.957017in}}%
\pgfpathlineto{\pgfqpoint{4.548432in}{2.981821in}}%
\pgfpathlineto{\pgfqpoint{4.614864in}{3.006031in}}%
\pgfpathlineto{\pgfqpoint{4.681295in}{3.029649in}}%
\pgfpathlineto{\pgfqpoint{4.747727in}{3.052679in}}%
\pgfpathlineto{\pgfqpoint{4.814159in}{3.075123in}}%
\pgfpathlineto{\pgfqpoint{4.880591in}{3.096983in}}%
\pgfpathlineto{\pgfqpoint{4.947022in}{3.118264in}}%
\pgfpathlineto{\pgfqpoint{5.013454in}{3.138966in}}%
\pgfpathlineto{\pgfqpoint{5.079886in}{3.159093in}}%
\pgfpathlineto{\pgfqpoint{5.146318in}{3.178646in}}%
\pgfpathlineto{\pgfqpoint{5.212749in}{3.197629in}}%
\pgfpathlineto{\pgfqpoint{5.279181in}{3.216043in}}%
\pgfpathlineto{\pgfqpoint{5.336755in}{3.231543in}}%
\pgfpathlineto{\pgfqpoint{5.336755in}{3.231543in}}%
\pgfusepath{stroke}%
\end{pgfscope}%
\begin{pgfscope}%
\pgfpathrectangle{\pgfqpoint{0.691184in}{0.629134in}}{\pgfqpoint{4.650000in}{3.020000in}}%
\pgfusepath{clip}%
\pgfsetbuttcap%
\pgfsetroundjoin%
\definecolor{currentfill}{rgb}{0.000000,0.000000,1.000000}%
\pgfsetfillcolor{currentfill}%
\pgfsetlinewidth{1.003750pt}%
\definecolor{currentstroke}{rgb}{0.000000,0.000000,1.000000}%
\pgfsetstrokecolor{currentstroke}%
\pgfsetdash{}{0pt}%
\pgfsys@defobject{currentmarker}{\pgfqpoint{-0.041667in}{-0.041667in}}{\pgfqpoint{0.041667in}{0.041667in}}{%
\pgfpathmoveto{\pgfqpoint{0.000000in}{-0.041667in}}%
\pgfpathcurveto{\pgfqpoint{0.011050in}{-0.041667in}}{\pgfqpoint{0.021649in}{-0.037276in}}{\pgfqpoint{0.029463in}{-0.029463in}}%
\pgfpathcurveto{\pgfqpoint{0.037276in}{-0.021649in}}{\pgfqpoint{0.041667in}{-0.011050in}}{\pgfqpoint{0.041667in}{0.000000in}}%
\pgfpathcurveto{\pgfqpoint{0.041667in}{0.011050in}}{\pgfqpoint{0.037276in}{0.021649in}}{\pgfqpoint{0.029463in}{0.029463in}}%
\pgfpathcurveto{\pgfqpoint{0.021649in}{0.037276in}}{\pgfqpoint{0.011050in}{0.041667in}}{\pgfqpoint{0.000000in}{0.041667in}}%
\pgfpathcurveto{\pgfqpoint{-0.011050in}{0.041667in}}{\pgfqpoint{-0.021649in}{0.037276in}}{\pgfqpoint{-0.029463in}{0.029463in}}%
\pgfpathcurveto{\pgfqpoint{-0.037276in}{0.021649in}}{\pgfqpoint{-0.041667in}{0.011050in}}{\pgfqpoint{-0.041667in}{0.000000in}}%
\pgfpathcurveto{\pgfqpoint{-0.041667in}{-0.011050in}}{\pgfqpoint{-0.037276in}{-0.021649in}}{\pgfqpoint{-0.029463in}{-0.029463in}}%
\pgfpathcurveto{\pgfqpoint{-0.021649in}{-0.037276in}}{\pgfqpoint{-0.011050in}{-0.041667in}}{\pgfqpoint{0.000000in}{-0.041667in}}%
\pgfpathclose%
\pgfusepath{stroke,fill}%
}%
\begin{pgfscope}%
\pgfsys@transformshift{0.912402in}{0.629134in}%
\pgfsys@useobject{currentmarker}{}%
\end{pgfscope}%
\begin{pgfscope}%
\pgfsys@transformshift{1.355280in}{1.039661in}%
\pgfsys@useobject{currentmarker}{}%
\end{pgfscope}%
\begin{pgfscope}%
\pgfsys@transformshift{1.798158in}{1.410799in}%
\pgfsys@useobject{currentmarker}{}%
\end{pgfscope}%
\begin{pgfscope}%
\pgfsys@transformshift{2.241036in}{1.745604in}%
\pgfsys@useobject{currentmarker}{}%
\end{pgfscope}%
\begin{pgfscope}%
\pgfsys@transformshift{2.683915in}{2.046538in}%
\pgfsys@useobject{currentmarker}{}%
\end{pgfscope}%
\begin{pgfscope}%
\pgfsys@transformshift{3.126793in}{2.315606in}%
\pgfsys@useobject{currentmarker}{}%
\end{pgfscope}%
\begin{pgfscope}%
\pgfsys@transformshift{3.569671in}{2.554469in}%
\pgfsys@useobject{currentmarker}{}%
\end{pgfscope}%
\begin{pgfscope}%
\pgfsys@transformshift{4.012549in}{2.764515in}%
\pgfsys@useobject{currentmarker}{}%
\end{pgfscope}%
\begin{pgfscope}%
\pgfsys@transformshift{4.455428in}{2.946928in}%
\pgfsys@useobject{currentmarker}{}%
\end{pgfscope}%
\begin{pgfscope}%
\pgfsys@transformshift{4.898306in}{3.102715in}%
\pgfsys@useobject{currentmarker}{}%
\end{pgfscope}%
\end{pgfscope}%
\begin{pgfscope}%
\pgfpathrectangle{\pgfqpoint{0.691184in}{0.629134in}}{\pgfqpoint{4.650000in}{3.020000in}}%
\pgfusepath{clip}%
\pgfsetbuttcap%
\pgfsetroundjoin%
\pgfsetlinewidth{1.505625pt}%
\definecolor{currentstroke}{rgb}{0.000000,0.750000,0.750000}%
\pgfsetstrokecolor{currentstroke}%
\pgfsetdash{{9.600000pt}{2.400000pt}{1.500000pt}{2.400000pt}}{0.000000pt}%
\pgfpathmoveto{\pgfqpoint{0.912402in}{0.629134in}}%
\pgfpathlineto{\pgfqpoint{0.969976in}{0.684855in}}%
\pgfpathlineto{\pgfqpoint{1.027550in}{0.739850in}}%
\pgfpathlineto{\pgfqpoint{1.085124in}{0.794118in}}%
\pgfpathlineto{\pgfqpoint{1.142698in}{0.847663in}}%
\pgfpathlineto{\pgfqpoint{1.200273in}{0.900487in}}%
\pgfpathlineto{\pgfqpoint{1.257847in}{0.952590in}}%
\pgfpathlineto{\pgfqpoint{1.315421in}{1.003975in}}%
\pgfpathlineto{\pgfqpoint{1.372995in}{1.054643in}}%
\pgfpathlineto{\pgfqpoint{1.430569in}{1.104596in}}%
\pgfpathlineto{\pgfqpoint{1.488143in}{1.153836in}}%
\pgfpathlineto{\pgfqpoint{1.545718in}{1.202364in}}%
\pgfpathlineto{\pgfqpoint{1.603292in}{1.250182in}}%
\pgfpathlineto{\pgfqpoint{1.660866in}{1.297292in}}%
\pgfpathlineto{\pgfqpoint{1.718440in}{1.343694in}}%
\pgfpathlineto{\pgfqpoint{1.776014in}{1.389391in}}%
\pgfpathlineto{\pgfqpoint{1.833588in}{1.434383in}}%
\pgfpathlineto{\pgfqpoint{1.891163in}{1.478673in}}%
\pgfpathlineto{\pgfqpoint{1.948737in}{1.522262in}}%
\pgfpathlineto{\pgfqpoint{2.006311in}{1.565151in}}%
\pgfpathlineto{\pgfqpoint{2.063885in}{1.607341in}}%
\pgfpathlineto{\pgfqpoint{2.121459in}{1.648834in}}%
\pgfpathlineto{\pgfqpoint{2.179033in}{1.689631in}}%
\pgfpathlineto{\pgfqpoint{2.236608in}{1.729734in}}%
\pgfpathlineto{\pgfqpoint{2.294182in}{1.769144in}}%
\pgfpathlineto{\pgfqpoint{2.351756in}{1.807861in}}%
\pgfpathlineto{\pgfqpoint{2.409330in}{1.845887in}}%
\pgfpathlineto{\pgfqpoint{2.466904in}{1.883224in}}%
\pgfpathlineto{\pgfqpoint{2.524478in}{1.919872in}}%
\pgfpathlineto{\pgfqpoint{2.582053in}{1.955833in}}%
\pgfpathlineto{\pgfqpoint{2.639627in}{1.991108in}}%
\pgfpathlineto{\pgfqpoint{2.697201in}{2.025697in}}%
\pgfpathlineto{\pgfqpoint{2.754775in}{2.059602in}}%
\pgfpathlineto{\pgfqpoint{2.812349in}{2.092824in}}%
\pgfpathlineto{\pgfqpoint{2.869923in}{2.125364in}}%
\pgfpathlineto{\pgfqpoint{2.927498in}{2.157223in}}%
\pgfpathlineto{\pgfqpoint{2.985072in}{2.188402in}}%
\pgfpathlineto{\pgfqpoint{3.042646in}{2.218902in}}%
\pgfpathlineto{\pgfqpoint{3.100220in}{2.248724in}}%
\pgfpathlineto{\pgfqpoint{3.153366in}{2.275650in}}%
\pgfpathlineto{\pgfqpoint{3.206511in}{2.302000in}}%
\pgfpathlineto{\pgfqpoint{3.259656in}{2.327774in}}%
\pgfpathlineto{\pgfqpoint{3.312802in}{2.352973in}}%
\pgfpathlineto{\pgfqpoint{3.365947in}{2.377598in}}%
\pgfpathlineto{\pgfqpoint{3.419092in}{2.401649in}}%
\pgfpathlineto{\pgfqpoint{3.472238in}{2.425126in}}%
\pgfpathlineto{\pgfqpoint{3.525383in}{2.448031in}}%
\pgfpathlineto{\pgfqpoint{3.578529in}{2.470364in}}%
\pgfpathlineto{\pgfqpoint{3.631674in}{2.492126in}}%
\pgfpathlineto{\pgfqpoint{3.684819in}{2.513316in}}%
\pgfpathlineto{\pgfqpoint{3.737965in}{2.533937in}}%
\pgfpathlineto{\pgfqpoint{3.791110in}{2.553987in}}%
\pgfpathlineto{\pgfqpoint{3.844256in}{2.573468in}}%
\pgfpathlineto{\pgfqpoint{3.897401in}{2.592381in}}%
\pgfpathlineto{\pgfqpoint{3.950546in}{2.610725in}}%
\pgfpathlineto{\pgfqpoint{4.003692in}{2.628501in}}%
\pgfpathlineto{\pgfqpoint{4.056837in}{2.645709in}}%
\pgfpathlineto{\pgfqpoint{4.109983in}{2.662351in}}%
\pgfpathlineto{\pgfqpoint{4.163128in}{2.678426in}}%
\pgfpathlineto{\pgfqpoint{4.216273in}{2.693934in}}%
\pgfpathlineto{\pgfqpoint{4.269419in}{2.708877in}}%
\pgfpathlineto{\pgfqpoint{4.322564in}{2.723255in}}%
\pgfpathlineto{\pgfqpoint{4.375709in}{2.737067in}}%
\pgfpathlineto{\pgfqpoint{4.428855in}{2.750315in}}%
\pgfpathlineto{\pgfqpoint{4.482000in}{2.762999in}}%
\pgfpathlineto{\pgfqpoint{4.535146in}{2.775118in}}%
\pgfpathlineto{\pgfqpoint{4.588291in}{2.786673in}}%
\pgfpathlineto{\pgfqpoint{4.641436in}{2.797665in}}%
\pgfpathlineto{\pgfqpoint{4.694582in}{2.808094in}}%
\pgfpathlineto{\pgfqpoint{4.747727in}{2.817960in}}%
\pgfpathlineto{\pgfqpoint{4.800873in}{2.827263in}}%
\pgfpathlineto{\pgfqpoint{4.854018in}{2.836004in}}%
\pgfpathlineto{\pgfqpoint{4.907163in}{2.844182in}}%
\pgfpathlineto{\pgfqpoint{4.960309in}{2.851798in}}%
\pgfpathlineto{\pgfqpoint{5.013454in}{2.858853in}}%
\pgfpathlineto{\pgfqpoint{5.066600in}{2.865345in}}%
\pgfpathlineto{\pgfqpoint{5.119745in}{2.871277in}}%
\pgfpathlineto{\pgfqpoint{5.172890in}{2.876646in}}%
\pgfpathlineto{\pgfqpoint{5.226036in}{2.881455in}}%
\pgfpathlineto{\pgfqpoint{5.279181in}{2.885702in}}%
\pgfpathlineto{\pgfqpoint{5.332326in}{2.889389in}}%
\pgfpathlineto{\pgfqpoint{5.336755in}{2.889671in}}%
\pgfpathlineto{\pgfqpoint{5.336755in}{2.889671in}}%
\pgfusepath{stroke}%
\end{pgfscope}%
\begin{pgfscope}%
\pgfpathrectangle{\pgfqpoint{0.691184in}{0.629134in}}{\pgfqpoint{4.650000in}{3.020000in}}%
\pgfusepath{clip}%
\pgfsetbuttcap%
\pgfsetmiterjoin%
\definecolor{currentfill}{rgb}{0.000000,0.750000,0.750000}%
\pgfsetfillcolor{currentfill}%
\pgfsetlinewidth{1.003750pt}%
\definecolor{currentstroke}{rgb}{0.000000,0.750000,0.750000}%
\pgfsetstrokecolor{currentstroke}%
\pgfsetdash{}{0pt}%
\pgfsys@defobject{currentmarker}{\pgfqpoint{-0.041667in}{-0.041667in}}{\pgfqpoint{0.041667in}{0.041667in}}{%
\pgfpathmoveto{\pgfqpoint{-0.000000in}{-0.041667in}}%
\pgfpathlineto{\pgfqpoint{0.041667in}{0.041667in}}%
\pgfpathlineto{\pgfqpoint{-0.041667in}{0.041667in}}%
\pgfpathclose%
\pgfusepath{stroke,fill}%
}%
\begin{pgfscope}%
\pgfsys@transformshift{0.912402in}{0.629134in}%
\pgfsys@useobject{currentmarker}{}%
\end{pgfscope}%
\begin{pgfscope}%
\pgfsys@transformshift{1.355280in}{1.039129in}%
\pgfsys@useobject{currentmarker}{}%
\end{pgfscope}%
\begin{pgfscope}%
\pgfsys@transformshift{1.798158in}{1.406779in}%
\pgfsys@useobject{currentmarker}{}%
\end{pgfscope}%
\begin{pgfscope}%
\pgfsys@transformshift{2.241036in}{1.732790in}%
\pgfsys@useobject{currentmarker}{}%
\end{pgfscope}%
\begin{pgfscope}%
\pgfsys@transformshift{2.683915in}{2.017776in}%
\pgfsys@useobject{currentmarker}{}%
\end{pgfscope}%
\begin{pgfscope}%
\pgfsys@transformshift{3.126793in}{2.262259in}%
\pgfsys@useobject{currentmarker}{}%
\end{pgfscope}%
\begin{pgfscope}%
\pgfsys@transformshift{3.569671in}{2.466682in}%
\pgfsys@useobject{currentmarker}{}%
\end{pgfscope}%
\begin{pgfscope}%
\pgfsys@transformshift{4.012549in}{2.631408in}%
\pgfsys@useobject{currentmarker}{}%
\end{pgfscope}%
\begin{pgfscope}%
\pgfsys@transformshift{4.455428in}{2.756727in}%
\pgfsys@useobject{currentmarker}{}%
\end{pgfscope}%
\begin{pgfscope}%
\pgfsys@transformshift{4.898306in}{2.842858in}%
\pgfsys@useobject{currentmarker}{}%
\end{pgfscope}%
\end{pgfscope}%
\begin{pgfscope}%
\pgfsetrectcap%
\pgfsetmiterjoin%
\pgfsetlinewidth{0.803000pt}%
\definecolor{currentstroke}{rgb}{0.501961,0.501961,0.501961}%
\pgfsetstrokecolor{currentstroke}%
\pgfsetdash{}{0pt}%
\pgfpathmoveto{\pgfqpoint{0.691184in}{0.629134in}}%
\pgfpathlineto{\pgfqpoint{0.691184in}{3.649134in}}%
\pgfusepath{stroke}%
\end{pgfscope}%
\begin{pgfscope}%
\pgfsetrectcap%
\pgfsetmiterjoin%
\pgfsetlinewidth{0.803000pt}%
\definecolor{currentstroke}{rgb}{0.501961,0.501961,0.501961}%
\pgfsetstrokecolor{currentstroke}%
\pgfsetdash{}{0pt}%
\pgfpathmoveto{\pgfqpoint{5.341184in}{0.629134in}}%
\pgfpathlineto{\pgfqpoint{5.341184in}{3.649134in}}%
\pgfusepath{stroke}%
\end{pgfscope}%
\begin{pgfscope}%
\pgfsetrectcap%
\pgfsetmiterjoin%
\pgfsetlinewidth{0.803000pt}%
\definecolor{currentstroke}{rgb}{0.501961,0.501961,0.501961}%
\pgfsetstrokecolor{currentstroke}%
\pgfsetdash{}{0pt}%
\pgfpathmoveto{\pgfqpoint{0.691184in}{0.629134in}}%
\pgfpathlineto{\pgfqpoint{5.341184in}{0.629134in}}%
\pgfusepath{stroke}%
\end{pgfscope}%
\begin{pgfscope}%
\pgfsetrectcap%
\pgfsetmiterjoin%
\pgfsetlinewidth{0.803000pt}%
\definecolor{currentstroke}{rgb}{0.501961,0.501961,0.501961}%
\pgfsetstrokecolor{currentstroke}%
\pgfsetdash{}{0pt}%
\pgfpathmoveto{\pgfqpoint{0.691184in}{3.649134in}}%
\pgfpathlineto{\pgfqpoint{5.341184in}{3.649134in}}%
\pgfusepath{stroke}%
\end{pgfscope}%
\begin{pgfscope}%
\pgfsetbuttcap%
\pgfsetmiterjoin%
\definecolor{currentfill}{rgb}{1.000000,1.000000,1.000000}%
\pgfsetfillcolor{currentfill}%
\pgfsetfillopacity{0.800000}%
\pgfsetlinewidth{0.000000pt}%
\definecolor{currentstroke}{rgb}{0.800000,0.800000,0.800000}%
\pgfsetstrokecolor{currentstroke}%
\pgfsetstrokeopacity{0.800000}%
\pgfsetdash{}{0pt}%
\pgfpathmoveto{\pgfqpoint{0.827295in}{2.324872in}}%
\pgfpathlineto{\pgfqpoint{2.434627in}{2.324872in}}%
\pgfpathquadraticcurveto{\pgfqpoint{2.473516in}{2.324872in}}{\pgfqpoint{2.473516in}{2.363761in}}%
\pgfpathlineto{\pgfqpoint{2.473516in}{3.513023in}}%
\pgfpathquadraticcurveto{\pgfqpoint{2.473516in}{3.551912in}}{\pgfqpoint{2.434627in}{3.551912in}}%
\pgfpathlineto{\pgfqpoint{0.827295in}{3.551912in}}%
\pgfpathquadraticcurveto{\pgfqpoint{0.788406in}{3.551912in}}{\pgfqpoint{0.788406in}{3.513023in}}%
\pgfpathlineto{\pgfqpoint{0.788406in}{2.363761in}}%
\pgfpathquadraticcurveto{\pgfqpoint{0.788406in}{2.324872in}}{\pgfqpoint{0.827295in}{2.324872in}}%
\pgfpathclose%
\pgfusepath{fill}%
\end{pgfscope}%
\begin{pgfscope}%
\pgftext[x=1.011390in,y=3.302049in,left,base]{\rmfamily\fontsize{14.000000}{16.800000}\selectfont \(\displaystyle \mathbf{D}\mbox{g}\) = 6\(\displaystyle \times 10^{-4}\)}%
\end{pgfscope}%
\begin{pgfscope}%
\pgfsetbuttcap%
\pgfsetroundjoin%
\pgfsetlinewidth{1.505625pt}%
\definecolor{currentstroke}{rgb}{1.000000,0.000000,0.000000}%
\pgfsetstrokecolor{currentstroke}%
\pgfsetdash{{5.550000pt}{2.400000pt}}{0.000000pt}%
\pgfpathmoveto{\pgfqpoint{0.866184in}{3.081951in}}%
\pgfpathlineto{\pgfqpoint{1.255073in}{3.081951in}}%
\pgfusepath{stroke}%
\end{pgfscope}%
\begin{pgfscope}%
\pgfsetbuttcap%
\pgfsetmiterjoin%
\definecolor{currentfill}{rgb}{1.000000,0.000000,0.000000}%
\pgfsetfillcolor{currentfill}%
\pgfsetlinewidth{1.003750pt}%
\definecolor{currentstroke}{rgb}{1.000000,0.000000,0.000000}%
\pgfsetstrokecolor{currentstroke}%
\pgfsetdash{}{0pt}%
\pgfsys@defobject{currentmarker}{\pgfqpoint{-0.041667in}{-0.041667in}}{\pgfqpoint{0.041667in}{0.041667in}}{%
\pgfpathmoveto{\pgfqpoint{-0.041667in}{-0.041667in}}%
\pgfpathlineto{\pgfqpoint{0.041667in}{-0.041667in}}%
\pgfpathlineto{\pgfqpoint{0.041667in}{0.041667in}}%
\pgfpathlineto{\pgfqpoint{-0.041667in}{0.041667in}}%
\pgfpathclose%
\pgfusepath{stroke,fill}%
}%
\begin{pgfscope}%
\pgfsys@transformshift{1.060628in}{3.081951in}%
\pgfsys@useobject{currentmarker}{}%
\end{pgfscope}%
\end{pgfscope}%
\begin{pgfscope}%
\pgftext[x=1.410628in,y=3.013896in,left,base]{\rmfamily\fontsize{14.000000}{16.800000}\selectfont \(\displaystyle \phi \mathbf{E}\mbox{u}\) = 1}%
\end{pgfscope}%
\begin{pgfscope}%
\pgfsetrectcap%
\pgfsetroundjoin%
\pgfsetlinewidth{1.505625pt}%
\definecolor{currentstroke}{rgb}{0.000000,0.000000,1.000000}%
\pgfsetstrokecolor{currentstroke}%
\pgfsetdash{}{0pt}%
\pgfpathmoveto{\pgfqpoint{0.866184in}{2.796551in}}%
\pgfpathlineto{\pgfqpoint{1.255073in}{2.796551in}}%
\pgfusepath{stroke}%
\end{pgfscope}%
\begin{pgfscope}%
\pgfsetbuttcap%
\pgfsetroundjoin%
\definecolor{currentfill}{rgb}{0.000000,0.000000,1.000000}%
\pgfsetfillcolor{currentfill}%
\pgfsetlinewidth{1.003750pt}%
\definecolor{currentstroke}{rgb}{0.000000,0.000000,1.000000}%
\pgfsetstrokecolor{currentstroke}%
\pgfsetdash{}{0pt}%
\pgfsys@defobject{currentmarker}{\pgfqpoint{-0.041667in}{-0.041667in}}{\pgfqpoint{0.041667in}{0.041667in}}{%
\pgfpathmoveto{\pgfqpoint{0.000000in}{-0.041667in}}%
\pgfpathcurveto{\pgfqpoint{0.011050in}{-0.041667in}}{\pgfqpoint{0.021649in}{-0.037276in}}{\pgfqpoint{0.029463in}{-0.029463in}}%
\pgfpathcurveto{\pgfqpoint{0.037276in}{-0.021649in}}{\pgfqpoint{0.041667in}{-0.011050in}}{\pgfqpoint{0.041667in}{0.000000in}}%
\pgfpathcurveto{\pgfqpoint{0.041667in}{0.011050in}}{\pgfqpoint{0.037276in}{0.021649in}}{\pgfqpoint{0.029463in}{0.029463in}}%
\pgfpathcurveto{\pgfqpoint{0.021649in}{0.037276in}}{\pgfqpoint{0.011050in}{0.041667in}}{\pgfqpoint{0.000000in}{0.041667in}}%
\pgfpathcurveto{\pgfqpoint{-0.011050in}{0.041667in}}{\pgfqpoint{-0.021649in}{0.037276in}}{\pgfqpoint{-0.029463in}{0.029463in}}%
\pgfpathcurveto{\pgfqpoint{-0.037276in}{0.021649in}}{\pgfqpoint{-0.041667in}{0.011050in}}{\pgfqpoint{-0.041667in}{0.000000in}}%
\pgfpathcurveto{\pgfqpoint{-0.041667in}{-0.011050in}}{\pgfqpoint{-0.037276in}{-0.021649in}}{\pgfqpoint{-0.029463in}{-0.029463in}}%
\pgfpathcurveto{\pgfqpoint{-0.021649in}{-0.037276in}}{\pgfqpoint{-0.011050in}{-0.041667in}}{\pgfqpoint{0.000000in}{-0.041667in}}%
\pgfpathclose%
\pgfusepath{stroke,fill}%
}%
\begin{pgfscope}%
\pgfsys@transformshift{1.060628in}{2.796551in}%
\pgfsys@useobject{currentmarker}{}%
\end{pgfscope}%
\end{pgfscope}%
\begin{pgfscope}%
\pgftext[x=1.410628in,y=2.728496in,left,base]{\rmfamily\fontsize{14.000000}{16.800000}\selectfont \(\displaystyle \phi \mathbf{E}\mbox{u}\) = 0.5}%
\end{pgfscope}%
\begin{pgfscope}%
\pgfsetbuttcap%
\pgfsetroundjoin%
\pgfsetlinewidth{1.505625pt}%
\definecolor{currentstroke}{rgb}{0.000000,0.750000,0.750000}%
\pgfsetstrokecolor{currentstroke}%
\pgfsetdash{{9.600000pt}{2.400000pt}{1.500000pt}{2.400000pt}}{0.000000pt}%
\pgfpathmoveto{\pgfqpoint{0.866184in}{2.511151in}}%
\pgfpathlineto{\pgfqpoint{1.255073in}{2.511151in}}%
\pgfusepath{stroke}%
\end{pgfscope}%
\begin{pgfscope}%
\pgfsetbuttcap%
\pgfsetmiterjoin%
\definecolor{currentfill}{rgb}{0.000000,0.750000,0.750000}%
\pgfsetfillcolor{currentfill}%
\pgfsetlinewidth{1.003750pt}%
\definecolor{currentstroke}{rgb}{0.000000,0.750000,0.750000}%
\pgfsetstrokecolor{currentstroke}%
\pgfsetdash{}{0pt}%
\pgfsys@defobject{currentmarker}{\pgfqpoint{-0.041667in}{-0.041667in}}{\pgfqpoint{0.041667in}{0.041667in}}{%
\pgfpathmoveto{\pgfqpoint{-0.000000in}{-0.041667in}}%
\pgfpathlineto{\pgfqpoint{0.041667in}{0.041667in}}%
\pgfpathlineto{\pgfqpoint{-0.041667in}{0.041667in}}%
\pgfpathclose%
\pgfusepath{stroke,fill}%
}%
\begin{pgfscope}%
\pgfsys@transformshift{1.060628in}{2.511151in}%
\pgfsys@useobject{currentmarker}{}%
\end{pgfscope}%
\end{pgfscope}%
\begin{pgfscope}%
\pgftext[x=1.410628in,y=2.443095in,left,base]{\rmfamily\fontsize{14.000000}{16.800000}\selectfont \(\displaystyle \phi \mathbf{E}\mbox{u}\) = 0.1}%
\end{pgfscope}%
\end{pgfpicture}%
\makeatother%
\endgroup%

    \caption{Text.}
     \label{fig:long_times}
\end{figure}

Times for returns
\[t_c = 2 + \epsilon \left(\frac{4}{3} - \frac{2 \beta}{3}\right) + \epsilon^{2} \left(\frac{4}{5} - \frac{4 \beta}{3} + \frac{2 \beta^{2}}{5}\right)
\]
\begin{figure}[htb]
    \centering
    %% Creator: Matplotlib, PGF backend
%%
%% To include the figure in your LaTeX document, write
%%   \input{<filename>.pgf}
%%
%% Make sure the required packages are loaded in your preamble
%%   \usepackage{pgf}
%%
%% Figures using additional raster images can only be included by \input if
%% they are in the same directory as the main LaTeX file. For loading figures
%% from other directories you can use the `import` package
%%   \usepackage{import}
%% and then include the figures with
%%   \import{<path to file>}{<filename>.pgf}
%%
%% Matplotlib used the following preamble
%%   \usepackage{fontspec}
%%   \setmainfont{DejaVu Serif}
%%   \setsansfont{DejaVu Sans}
%%   \setmonofont{DejaVu Sans Mono}
%%
\begingroup%
\makeatletter%
\begin{pgfpicture}%
\pgfpathrectangle{\pgfpointorigin}{\pgfqpoint{5.326327in}{3.653270in}}%
\pgfusepath{use as bounding box, clip}%
\begin{pgfscope}%
\pgfsetbuttcap%
\pgfsetmiterjoin%
\definecolor{currentfill}{rgb}{1.000000,1.000000,1.000000}%
\pgfsetfillcolor{currentfill}%
\pgfsetlinewidth{0.000000pt}%
\definecolor{currentstroke}{rgb}{1.000000,1.000000,1.000000}%
\pgfsetstrokecolor{currentstroke}%
\pgfsetdash{}{0pt}%
\pgfpathmoveto{\pgfqpoint{0.000000in}{0.000000in}}%
\pgfpathlineto{\pgfqpoint{5.326327in}{0.000000in}}%
\pgfpathlineto{\pgfqpoint{5.326327in}{3.653270in}}%
\pgfpathlineto{\pgfqpoint{0.000000in}{3.653270in}}%
\pgfpathclose%
\pgfusepath{fill}%
\end{pgfscope}%
\begin{pgfscope}%
\pgfsetbuttcap%
\pgfsetmiterjoin%
\definecolor{currentfill}{rgb}{1.000000,1.000000,1.000000}%
\pgfsetfillcolor{currentfill}%
\pgfsetlinewidth{0.000000pt}%
\definecolor{currentstroke}{rgb}{0.000000,0.000000,0.000000}%
\pgfsetstrokecolor{currentstroke}%
\pgfsetstrokeopacity{0.000000}%
\pgfsetdash{}{0pt}%
\pgfpathmoveto{\pgfqpoint{0.564660in}{0.521603in}}%
\pgfpathlineto{\pgfqpoint{5.214660in}{0.521603in}}%
\pgfpathlineto{\pgfqpoint{5.214660in}{3.541603in}}%
\pgfpathlineto{\pgfqpoint{0.564660in}{3.541603in}}%
\pgfpathclose%
\pgfusepath{fill}%
\end{pgfscope}%
\begin{pgfscope}%
\pgfsetbuttcap%
\pgfsetroundjoin%
\definecolor{currentfill}{rgb}{0.000000,0.000000,0.000000}%
\pgfsetfillcolor{currentfill}%
\pgfsetlinewidth{0.803000pt}%
\definecolor{currentstroke}{rgb}{0.000000,0.000000,0.000000}%
\pgfsetstrokecolor{currentstroke}%
\pgfsetdash{}{0pt}%
\pgfsys@defobject{currentmarker}{\pgfqpoint{0.000000in}{-0.048611in}}{\pgfqpoint{0.000000in}{0.000000in}}{%
\pgfpathmoveto{\pgfqpoint{0.000000in}{0.000000in}}%
\pgfpathlineto{\pgfqpoint{0.000000in}{-0.048611in}}%
\pgfusepath{stroke,fill}%
}%
\begin{pgfscope}%
\pgfsys@transformshift{0.776024in}{0.521603in}%
\pgfsys@useobject{currentmarker}{}%
\end{pgfscope}%
\end{pgfscope}%
\begin{pgfscope}%
\pgftext[x=0.776024in,y=0.424381in,,top]{\rmfamily\fontsize{10.000000}{12.000000}\selectfont \(\displaystyle 10^{-2}\)}%
\end{pgfscope}%
\begin{pgfscope}%
\pgfsetbuttcap%
\pgfsetroundjoin%
\definecolor{currentfill}{rgb}{0.000000,0.000000,0.000000}%
\pgfsetfillcolor{currentfill}%
\pgfsetlinewidth{0.803000pt}%
\definecolor{currentstroke}{rgb}{0.000000,0.000000,0.000000}%
\pgfsetstrokecolor{currentstroke}%
\pgfsetdash{}{0pt}%
\pgfsys@defobject{currentmarker}{\pgfqpoint{0.000000in}{-0.048611in}}{\pgfqpoint{0.000000in}{0.000000in}}{%
\pgfpathmoveto{\pgfqpoint{0.000000in}{0.000000in}}%
\pgfpathlineto{\pgfqpoint{0.000000in}{-0.048611in}}%
\pgfusepath{stroke,fill}%
}%
\begin{pgfscope}%
\pgfsys@transformshift{2.890120in}{0.521603in}%
\pgfsys@useobject{currentmarker}{}%
\end{pgfscope}%
\end{pgfscope}%
\begin{pgfscope}%
\pgftext[x=2.890120in,y=0.424381in,,top]{\rmfamily\fontsize{10.000000}{12.000000}\selectfont \(\displaystyle 10^{-1}\)}%
\end{pgfscope}%
\begin{pgfscope}%
\pgfsetbuttcap%
\pgfsetroundjoin%
\definecolor{currentfill}{rgb}{0.000000,0.000000,0.000000}%
\pgfsetfillcolor{currentfill}%
\pgfsetlinewidth{0.803000pt}%
\definecolor{currentstroke}{rgb}{0.000000,0.000000,0.000000}%
\pgfsetstrokecolor{currentstroke}%
\pgfsetdash{}{0pt}%
\pgfsys@defobject{currentmarker}{\pgfqpoint{0.000000in}{-0.048611in}}{\pgfqpoint{0.000000in}{0.000000in}}{%
\pgfpathmoveto{\pgfqpoint{0.000000in}{0.000000in}}%
\pgfpathlineto{\pgfqpoint{0.000000in}{-0.048611in}}%
\pgfusepath{stroke,fill}%
}%
\begin{pgfscope}%
\pgfsys@transformshift{5.004215in}{0.521603in}%
\pgfsys@useobject{currentmarker}{}%
\end{pgfscope}%
\end{pgfscope}%
\begin{pgfscope}%
\pgftext[x=5.004215in,y=0.424381in,,top]{\rmfamily\fontsize{10.000000}{12.000000}\selectfont \(\displaystyle 10^{0}\)}%
\end{pgfscope}%
\begin{pgfscope}%
\pgfsetbuttcap%
\pgfsetroundjoin%
\definecolor{currentfill}{rgb}{0.000000,0.000000,0.000000}%
\pgfsetfillcolor{currentfill}%
\pgfsetlinewidth{0.602250pt}%
\definecolor{currentstroke}{rgb}{0.000000,0.000000,0.000000}%
\pgfsetstrokecolor{currentstroke}%
\pgfsetdash{}{0pt}%
\pgfsys@defobject{currentmarker}{\pgfqpoint{0.000000in}{-0.027778in}}{\pgfqpoint{0.000000in}{0.000000in}}{%
\pgfpathmoveto{\pgfqpoint{0.000000in}{0.000000in}}%
\pgfpathlineto{\pgfqpoint{0.000000in}{-0.027778in}}%
\pgfusepath{stroke,fill}%
}%
\begin{pgfscope}%
\pgfsys@transformshift{0.571147in}{0.521603in}%
\pgfsys@useobject{currentmarker}{}%
\end{pgfscope}%
\end{pgfscope}%
\begin{pgfscope}%
\pgfsetbuttcap%
\pgfsetroundjoin%
\definecolor{currentfill}{rgb}{0.000000,0.000000,0.000000}%
\pgfsetfillcolor{currentfill}%
\pgfsetlinewidth{0.602250pt}%
\definecolor{currentstroke}{rgb}{0.000000,0.000000,0.000000}%
\pgfsetstrokecolor{currentstroke}%
\pgfsetdash{}{0pt}%
\pgfsys@defobject{currentmarker}{\pgfqpoint{0.000000in}{-0.027778in}}{\pgfqpoint{0.000000in}{0.000000in}}{%
\pgfpathmoveto{\pgfqpoint{0.000000in}{0.000000in}}%
\pgfpathlineto{\pgfqpoint{0.000000in}{-0.027778in}}%
\pgfusepath{stroke,fill}%
}%
\begin{pgfscope}%
\pgfsys@transformshift{0.679288in}{0.521603in}%
\pgfsys@useobject{currentmarker}{}%
\end{pgfscope}%
\end{pgfscope}%
\begin{pgfscope}%
\pgfsetbuttcap%
\pgfsetroundjoin%
\definecolor{currentfill}{rgb}{0.000000,0.000000,0.000000}%
\pgfsetfillcolor{currentfill}%
\pgfsetlinewidth{0.602250pt}%
\definecolor{currentstroke}{rgb}{0.000000,0.000000,0.000000}%
\pgfsetstrokecolor{currentstroke}%
\pgfsetdash{}{0pt}%
\pgfsys@defobject{currentmarker}{\pgfqpoint{0.000000in}{-0.027778in}}{\pgfqpoint{0.000000in}{0.000000in}}{%
\pgfpathmoveto{\pgfqpoint{0.000000in}{0.000000in}}%
\pgfpathlineto{\pgfqpoint{0.000000in}{-0.027778in}}%
\pgfusepath{stroke,fill}%
}%
\begin{pgfscope}%
\pgfsys@transformshift{1.412430in}{0.521603in}%
\pgfsys@useobject{currentmarker}{}%
\end{pgfscope}%
\end{pgfscope}%
\begin{pgfscope}%
\pgfsetbuttcap%
\pgfsetroundjoin%
\definecolor{currentfill}{rgb}{0.000000,0.000000,0.000000}%
\pgfsetfillcolor{currentfill}%
\pgfsetlinewidth{0.602250pt}%
\definecolor{currentstroke}{rgb}{0.000000,0.000000,0.000000}%
\pgfsetstrokecolor{currentstroke}%
\pgfsetdash{}{0pt}%
\pgfsys@defobject{currentmarker}{\pgfqpoint{0.000000in}{-0.027778in}}{\pgfqpoint{0.000000in}{0.000000in}}{%
\pgfpathmoveto{\pgfqpoint{0.000000in}{0.000000in}}%
\pgfpathlineto{\pgfqpoint{0.000000in}{-0.027778in}}%
\pgfusepath{stroke,fill}%
}%
\begin{pgfscope}%
\pgfsys@transformshift{1.784704in}{0.521603in}%
\pgfsys@useobject{currentmarker}{}%
\end{pgfscope}%
\end{pgfscope}%
\begin{pgfscope}%
\pgfsetbuttcap%
\pgfsetroundjoin%
\definecolor{currentfill}{rgb}{0.000000,0.000000,0.000000}%
\pgfsetfillcolor{currentfill}%
\pgfsetlinewidth{0.602250pt}%
\definecolor{currentstroke}{rgb}{0.000000,0.000000,0.000000}%
\pgfsetstrokecolor{currentstroke}%
\pgfsetdash{}{0pt}%
\pgfsys@defobject{currentmarker}{\pgfqpoint{0.000000in}{-0.027778in}}{\pgfqpoint{0.000000in}{0.000000in}}{%
\pgfpathmoveto{\pgfqpoint{0.000000in}{0.000000in}}%
\pgfpathlineto{\pgfqpoint{0.000000in}{-0.027778in}}%
\pgfusepath{stroke,fill}%
}%
\begin{pgfscope}%
\pgfsys@transformshift{2.048836in}{0.521603in}%
\pgfsys@useobject{currentmarker}{}%
\end{pgfscope}%
\end{pgfscope}%
\begin{pgfscope}%
\pgfsetbuttcap%
\pgfsetroundjoin%
\definecolor{currentfill}{rgb}{0.000000,0.000000,0.000000}%
\pgfsetfillcolor{currentfill}%
\pgfsetlinewidth{0.602250pt}%
\definecolor{currentstroke}{rgb}{0.000000,0.000000,0.000000}%
\pgfsetstrokecolor{currentstroke}%
\pgfsetdash{}{0pt}%
\pgfsys@defobject{currentmarker}{\pgfqpoint{0.000000in}{-0.027778in}}{\pgfqpoint{0.000000in}{0.000000in}}{%
\pgfpathmoveto{\pgfqpoint{0.000000in}{0.000000in}}%
\pgfpathlineto{\pgfqpoint{0.000000in}{-0.027778in}}%
\pgfusepath{stroke,fill}%
}%
\begin{pgfscope}%
\pgfsys@transformshift{2.253713in}{0.521603in}%
\pgfsys@useobject{currentmarker}{}%
\end{pgfscope}%
\end{pgfscope}%
\begin{pgfscope}%
\pgfsetbuttcap%
\pgfsetroundjoin%
\definecolor{currentfill}{rgb}{0.000000,0.000000,0.000000}%
\pgfsetfillcolor{currentfill}%
\pgfsetlinewidth{0.602250pt}%
\definecolor{currentstroke}{rgb}{0.000000,0.000000,0.000000}%
\pgfsetstrokecolor{currentstroke}%
\pgfsetdash{}{0pt}%
\pgfsys@defobject{currentmarker}{\pgfqpoint{0.000000in}{-0.027778in}}{\pgfqpoint{0.000000in}{0.000000in}}{%
\pgfpathmoveto{\pgfqpoint{0.000000in}{0.000000in}}%
\pgfpathlineto{\pgfqpoint{0.000000in}{-0.027778in}}%
\pgfusepath{stroke,fill}%
}%
\begin{pgfscope}%
\pgfsys@transformshift{2.421110in}{0.521603in}%
\pgfsys@useobject{currentmarker}{}%
\end{pgfscope}%
\end{pgfscope}%
\begin{pgfscope}%
\pgfsetbuttcap%
\pgfsetroundjoin%
\definecolor{currentfill}{rgb}{0.000000,0.000000,0.000000}%
\pgfsetfillcolor{currentfill}%
\pgfsetlinewidth{0.602250pt}%
\definecolor{currentstroke}{rgb}{0.000000,0.000000,0.000000}%
\pgfsetstrokecolor{currentstroke}%
\pgfsetdash{}{0pt}%
\pgfsys@defobject{currentmarker}{\pgfqpoint{0.000000in}{-0.027778in}}{\pgfqpoint{0.000000in}{0.000000in}}{%
\pgfpathmoveto{\pgfqpoint{0.000000in}{0.000000in}}%
\pgfpathlineto{\pgfqpoint{0.000000in}{-0.027778in}}%
\pgfusepath{stroke,fill}%
}%
\begin{pgfscope}%
\pgfsys@transformshift{2.562642in}{0.521603in}%
\pgfsys@useobject{currentmarker}{}%
\end{pgfscope}%
\end{pgfscope}%
\begin{pgfscope}%
\pgfsetbuttcap%
\pgfsetroundjoin%
\definecolor{currentfill}{rgb}{0.000000,0.000000,0.000000}%
\pgfsetfillcolor{currentfill}%
\pgfsetlinewidth{0.602250pt}%
\definecolor{currentstroke}{rgb}{0.000000,0.000000,0.000000}%
\pgfsetstrokecolor{currentstroke}%
\pgfsetdash{}{0pt}%
\pgfsys@defobject{currentmarker}{\pgfqpoint{0.000000in}{-0.027778in}}{\pgfqpoint{0.000000in}{0.000000in}}{%
\pgfpathmoveto{\pgfqpoint{0.000000in}{0.000000in}}%
\pgfpathlineto{\pgfqpoint{0.000000in}{-0.027778in}}%
\pgfusepath{stroke,fill}%
}%
\begin{pgfscope}%
\pgfsys@transformshift{2.685243in}{0.521603in}%
\pgfsys@useobject{currentmarker}{}%
\end{pgfscope}%
\end{pgfscope}%
\begin{pgfscope}%
\pgfsetbuttcap%
\pgfsetroundjoin%
\definecolor{currentfill}{rgb}{0.000000,0.000000,0.000000}%
\pgfsetfillcolor{currentfill}%
\pgfsetlinewidth{0.602250pt}%
\definecolor{currentstroke}{rgb}{0.000000,0.000000,0.000000}%
\pgfsetstrokecolor{currentstroke}%
\pgfsetdash{}{0pt}%
\pgfsys@defobject{currentmarker}{\pgfqpoint{0.000000in}{-0.027778in}}{\pgfqpoint{0.000000in}{0.000000in}}{%
\pgfpathmoveto{\pgfqpoint{0.000000in}{0.000000in}}%
\pgfpathlineto{\pgfqpoint{0.000000in}{-0.027778in}}%
\pgfusepath{stroke,fill}%
}%
\begin{pgfscope}%
\pgfsys@transformshift{2.793384in}{0.521603in}%
\pgfsys@useobject{currentmarker}{}%
\end{pgfscope}%
\end{pgfscope}%
\begin{pgfscope}%
\pgfsetbuttcap%
\pgfsetroundjoin%
\definecolor{currentfill}{rgb}{0.000000,0.000000,0.000000}%
\pgfsetfillcolor{currentfill}%
\pgfsetlinewidth{0.602250pt}%
\definecolor{currentstroke}{rgb}{0.000000,0.000000,0.000000}%
\pgfsetstrokecolor{currentstroke}%
\pgfsetdash{}{0pt}%
\pgfsys@defobject{currentmarker}{\pgfqpoint{0.000000in}{-0.027778in}}{\pgfqpoint{0.000000in}{0.000000in}}{%
\pgfpathmoveto{\pgfqpoint{0.000000in}{0.000000in}}%
\pgfpathlineto{\pgfqpoint{0.000000in}{-0.027778in}}%
\pgfusepath{stroke,fill}%
}%
\begin{pgfscope}%
\pgfsys@transformshift{3.526526in}{0.521603in}%
\pgfsys@useobject{currentmarker}{}%
\end{pgfscope}%
\end{pgfscope}%
\begin{pgfscope}%
\pgfsetbuttcap%
\pgfsetroundjoin%
\definecolor{currentfill}{rgb}{0.000000,0.000000,0.000000}%
\pgfsetfillcolor{currentfill}%
\pgfsetlinewidth{0.602250pt}%
\definecolor{currentstroke}{rgb}{0.000000,0.000000,0.000000}%
\pgfsetstrokecolor{currentstroke}%
\pgfsetdash{}{0pt}%
\pgfsys@defobject{currentmarker}{\pgfqpoint{0.000000in}{-0.027778in}}{\pgfqpoint{0.000000in}{0.000000in}}{%
\pgfpathmoveto{\pgfqpoint{0.000000in}{0.000000in}}%
\pgfpathlineto{\pgfqpoint{0.000000in}{-0.027778in}}%
\pgfusepath{stroke,fill}%
}%
\begin{pgfscope}%
\pgfsys@transformshift{3.898800in}{0.521603in}%
\pgfsys@useobject{currentmarker}{}%
\end{pgfscope}%
\end{pgfscope}%
\begin{pgfscope}%
\pgfsetbuttcap%
\pgfsetroundjoin%
\definecolor{currentfill}{rgb}{0.000000,0.000000,0.000000}%
\pgfsetfillcolor{currentfill}%
\pgfsetlinewidth{0.602250pt}%
\definecolor{currentstroke}{rgb}{0.000000,0.000000,0.000000}%
\pgfsetstrokecolor{currentstroke}%
\pgfsetdash{}{0pt}%
\pgfsys@defobject{currentmarker}{\pgfqpoint{0.000000in}{-0.027778in}}{\pgfqpoint{0.000000in}{0.000000in}}{%
\pgfpathmoveto{\pgfqpoint{0.000000in}{0.000000in}}%
\pgfpathlineto{\pgfqpoint{0.000000in}{-0.027778in}}%
\pgfusepath{stroke,fill}%
}%
\begin{pgfscope}%
\pgfsys@transformshift{4.162932in}{0.521603in}%
\pgfsys@useobject{currentmarker}{}%
\end{pgfscope}%
\end{pgfscope}%
\begin{pgfscope}%
\pgfsetbuttcap%
\pgfsetroundjoin%
\definecolor{currentfill}{rgb}{0.000000,0.000000,0.000000}%
\pgfsetfillcolor{currentfill}%
\pgfsetlinewidth{0.602250pt}%
\definecolor{currentstroke}{rgb}{0.000000,0.000000,0.000000}%
\pgfsetstrokecolor{currentstroke}%
\pgfsetdash{}{0pt}%
\pgfsys@defobject{currentmarker}{\pgfqpoint{0.000000in}{-0.027778in}}{\pgfqpoint{0.000000in}{0.000000in}}{%
\pgfpathmoveto{\pgfqpoint{0.000000in}{0.000000in}}%
\pgfpathlineto{\pgfqpoint{0.000000in}{-0.027778in}}%
\pgfusepath{stroke,fill}%
}%
\begin{pgfscope}%
\pgfsys@transformshift{4.367809in}{0.521603in}%
\pgfsys@useobject{currentmarker}{}%
\end{pgfscope}%
\end{pgfscope}%
\begin{pgfscope}%
\pgfsetbuttcap%
\pgfsetroundjoin%
\definecolor{currentfill}{rgb}{0.000000,0.000000,0.000000}%
\pgfsetfillcolor{currentfill}%
\pgfsetlinewidth{0.602250pt}%
\definecolor{currentstroke}{rgb}{0.000000,0.000000,0.000000}%
\pgfsetstrokecolor{currentstroke}%
\pgfsetdash{}{0pt}%
\pgfsys@defobject{currentmarker}{\pgfqpoint{0.000000in}{-0.027778in}}{\pgfqpoint{0.000000in}{0.000000in}}{%
\pgfpathmoveto{\pgfqpoint{0.000000in}{0.000000in}}%
\pgfpathlineto{\pgfqpoint{0.000000in}{-0.027778in}}%
\pgfusepath{stroke,fill}%
}%
\begin{pgfscope}%
\pgfsys@transformshift{4.535206in}{0.521603in}%
\pgfsys@useobject{currentmarker}{}%
\end{pgfscope}%
\end{pgfscope}%
\begin{pgfscope}%
\pgfsetbuttcap%
\pgfsetroundjoin%
\definecolor{currentfill}{rgb}{0.000000,0.000000,0.000000}%
\pgfsetfillcolor{currentfill}%
\pgfsetlinewidth{0.602250pt}%
\definecolor{currentstroke}{rgb}{0.000000,0.000000,0.000000}%
\pgfsetstrokecolor{currentstroke}%
\pgfsetdash{}{0pt}%
\pgfsys@defobject{currentmarker}{\pgfqpoint{0.000000in}{-0.027778in}}{\pgfqpoint{0.000000in}{0.000000in}}{%
\pgfpathmoveto{\pgfqpoint{0.000000in}{0.000000in}}%
\pgfpathlineto{\pgfqpoint{0.000000in}{-0.027778in}}%
\pgfusepath{stroke,fill}%
}%
\begin{pgfscope}%
\pgfsys@transformshift{4.676738in}{0.521603in}%
\pgfsys@useobject{currentmarker}{}%
\end{pgfscope}%
\end{pgfscope}%
\begin{pgfscope}%
\pgfsetbuttcap%
\pgfsetroundjoin%
\definecolor{currentfill}{rgb}{0.000000,0.000000,0.000000}%
\pgfsetfillcolor{currentfill}%
\pgfsetlinewidth{0.602250pt}%
\definecolor{currentstroke}{rgb}{0.000000,0.000000,0.000000}%
\pgfsetstrokecolor{currentstroke}%
\pgfsetdash{}{0pt}%
\pgfsys@defobject{currentmarker}{\pgfqpoint{0.000000in}{-0.027778in}}{\pgfqpoint{0.000000in}{0.000000in}}{%
\pgfpathmoveto{\pgfqpoint{0.000000in}{0.000000in}}%
\pgfpathlineto{\pgfqpoint{0.000000in}{-0.027778in}}%
\pgfusepath{stroke,fill}%
}%
\begin{pgfscope}%
\pgfsys@transformshift{4.799338in}{0.521603in}%
\pgfsys@useobject{currentmarker}{}%
\end{pgfscope}%
\end{pgfscope}%
\begin{pgfscope}%
\pgfsetbuttcap%
\pgfsetroundjoin%
\definecolor{currentfill}{rgb}{0.000000,0.000000,0.000000}%
\pgfsetfillcolor{currentfill}%
\pgfsetlinewidth{0.602250pt}%
\definecolor{currentstroke}{rgb}{0.000000,0.000000,0.000000}%
\pgfsetstrokecolor{currentstroke}%
\pgfsetdash{}{0pt}%
\pgfsys@defobject{currentmarker}{\pgfqpoint{0.000000in}{-0.027778in}}{\pgfqpoint{0.000000in}{0.000000in}}{%
\pgfpathmoveto{\pgfqpoint{0.000000in}{0.000000in}}%
\pgfpathlineto{\pgfqpoint{0.000000in}{-0.027778in}}%
\pgfusepath{stroke,fill}%
}%
\begin{pgfscope}%
\pgfsys@transformshift{4.907480in}{0.521603in}%
\pgfsys@useobject{currentmarker}{}%
\end{pgfscope}%
\end{pgfscope}%
\begin{pgfscope}%
\pgftext[x=2.889660in,y=0.234413in,,top]{\rmfamily\fontsize{10.000000}{12.000000}\selectfont \(\displaystyle B \mathbf{E}\mbox{u}\)}%
\end{pgfscope}%
\begin{pgfscope}%
\pgfsetbuttcap%
\pgfsetroundjoin%
\definecolor{currentfill}{rgb}{0.000000,0.000000,0.000000}%
\pgfsetfillcolor{currentfill}%
\pgfsetlinewidth{0.803000pt}%
\definecolor{currentstroke}{rgb}{0.000000,0.000000,0.000000}%
\pgfsetstrokecolor{currentstroke}%
\pgfsetdash{}{0pt}%
\pgfsys@defobject{currentmarker}{\pgfqpoint{-0.048611in}{0.000000in}}{\pgfqpoint{0.000000in}{0.000000in}}{%
\pgfpathmoveto{\pgfqpoint{0.000000in}{0.000000in}}%
\pgfpathlineto{\pgfqpoint{-0.048611in}{0.000000in}}%
\pgfusepath{stroke,fill}%
}%
\begin{pgfscope}%
\pgfsys@transformshift{0.564660in}{0.644567in}%
\pgfsys@useobject{currentmarker}{}%
\end{pgfscope}%
\end{pgfscope}%
\begin{pgfscope}%
\pgftext[x=0.289968in,y=0.591805in,left,base]{\rmfamily\fontsize{10.000000}{12.000000}\selectfont \(\displaystyle 2.0\)}%
\end{pgfscope}%
\begin{pgfscope}%
\pgfsetbuttcap%
\pgfsetroundjoin%
\definecolor{currentfill}{rgb}{0.000000,0.000000,0.000000}%
\pgfsetfillcolor{currentfill}%
\pgfsetlinewidth{0.803000pt}%
\definecolor{currentstroke}{rgb}{0.000000,0.000000,0.000000}%
\pgfsetstrokecolor{currentstroke}%
\pgfsetdash{}{0pt}%
\pgfsys@defobject{currentmarker}{\pgfqpoint{-0.048611in}{0.000000in}}{\pgfqpoint{0.000000in}{0.000000in}}{%
\pgfpathmoveto{\pgfqpoint{0.000000in}{0.000000in}}%
\pgfpathlineto{\pgfqpoint{-0.048611in}{0.000000in}}%
\pgfusepath{stroke,fill}%
}%
\begin{pgfscope}%
\pgfsys@transformshift{0.564660in}{1.178128in}%
\pgfsys@useobject{currentmarker}{}%
\end{pgfscope}%
\end{pgfscope}%
\begin{pgfscope}%
\pgftext[x=0.289968in,y=1.125366in,left,base]{\rmfamily\fontsize{10.000000}{12.000000}\selectfont \(\displaystyle 2.5\)}%
\end{pgfscope}%
\begin{pgfscope}%
\pgfsetbuttcap%
\pgfsetroundjoin%
\definecolor{currentfill}{rgb}{0.000000,0.000000,0.000000}%
\pgfsetfillcolor{currentfill}%
\pgfsetlinewidth{0.803000pt}%
\definecolor{currentstroke}{rgb}{0.000000,0.000000,0.000000}%
\pgfsetstrokecolor{currentstroke}%
\pgfsetdash{}{0pt}%
\pgfsys@defobject{currentmarker}{\pgfqpoint{-0.048611in}{0.000000in}}{\pgfqpoint{0.000000in}{0.000000in}}{%
\pgfpathmoveto{\pgfqpoint{0.000000in}{0.000000in}}%
\pgfpathlineto{\pgfqpoint{-0.048611in}{0.000000in}}%
\pgfusepath{stroke,fill}%
}%
\begin{pgfscope}%
\pgfsys@transformshift{0.564660in}{1.711689in}%
\pgfsys@useobject{currentmarker}{}%
\end{pgfscope}%
\end{pgfscope}%
\begin{pgfscope}%
\pgftext[x=0.289968in,y=1.658928in,left,base]{\rmfamily\fontsize{10.000000}{12.000000}\selectfont \(\displaystyle 3.0\)}%
\end{pgfscope}%
\begin{pgfscope}%
\pgfsetbuttcap%
\pgfsetroundjoin%
\definecolor{currentfill}{rgb}{0.000000,0.000000,0.000000}%
\pgfsetfillcolor{currentfill}%
\pgfsetlinewidth{0.803000pt}%
\definecolor{currentstroke}{rgb}{0.000000,0.000000,0.000000}%
\pgfsetstrokecolor{currentstroke}%
\pgfsetdash{}{0pt}%
\pgfsys@defobject{currentmarker}{\pgfqpoint{-0.048611in}{0.000000in}}{\pgfqpoint{0.000000in}{0.000000in}}{%
\pgfpathmoveto{\pgfqpoint{0.000000in}{0.000000in}}%
\pgfpathlineto{\pgfqpoint{-0.048611in}{0.000000in}}%
\pgfusepath{stroke,fill}%
}%
\begin{pgfscope}%
\pgfsys@transformshift{0.564660in}{2.245250in}%
\pgfsys@useobject{currentmarker}{}%
\end{pgfscope}%
\end{pgfscope}%
\begin{pgfscope}%
\pgftext[x=0.289968in,y=2.192489in,left,base]{\rmfamily\fontsize{10.000000}{12.000000}\selectfont \(\displaystyle 3.5\)}%
\end{pgfscope}%
\begin{pgfscope}%
\pgfsetbuttcap%
\pgfsetroundjoin%
\definecolor{currentfill}{rgb}{0.000000,0.000000,0.000000}%
\pgfsetfillcolor{currentfill}%
\pgfsetlinewidth{0.803000pt}%
\definecolor{currentstroke}{rgb}{0.000000,0.000000,0.000000}%
\pgfsetstrokecolor{currentstroke}%
\pgfsetdash{}{0pt}%
\pgfsys@defobject{currentmarker}{\pgfqpoint{-0.048611in}{0.000000in}}{\pgfqpoint{0.000000in}{0.000000in}}{%
\pgfpathmoveto{\pgfqpoint{0.000000in}{0.000000in}}%
\pgfpathlineto{\pgfqpoint{-0.048611in}{0.000000in}}%
\pgfusepath{stroke,fill}%
}%
\begin{pgfscope}%
\pgfsys@transformshift{0.564660in}{2.778812in}%
\pgfsys@useobject{currentmarker}{}%
\end{pgfscope}%
\end{pgfscope}%
\begin{pgfscope}%
\pgftext[x=0.289968in,y=2.726050in,left,base]{\rmfamily\fontsize{10.000000}{12.000000}\selectfont \(\displaystyle 4.0\)}%
\end{pgfscope}%
\begin{pgfscope}%
\pgfsetbuttcap%
\pgfsetroundjoin%
\definecolor{currentfill}{rgb}{0.000000,0.000000,0.000000}%
\pgfsetfillcolor{currentfill}%
\pgfsetlinewidth{0.803000pt}%
\definecolor{currentstroke}{rgb}{0.000000,0.000000,0.000000}%
\pgfsetstrokecolor{currentstroke}%
\pgfsetdash{}{0pt}%
\pgfsys@defobject{currentmarker}{\pgfqpoint{-0.048611in}{0.000000in}}{\pgfqpoint{0.000000in}{0.000000in}}{%
\pgfpathmoveto{\pgfqpoint{0.000000in}{0.000000in}}%
\pgfpathlineto{\pgfqpoint{-0.048611in}{0.000000in}}%
\pgfusepath{stroke,fill}%
}%
\begin{pgfscope}%
\pgfsys@transformshift{0.564660in}{3.312373in}%
\pgfsys@useobject{currentmarker}{}%
\end{pgfscope}%
\end{pgfscope}%
\begin{pgfscope}%
\pgftext[x=0.289968in,y=3.259611in,left,base]{\rmfamily\fontsize{10.000000}{12.000000}\selectfont \(\displaystyle 4.5\)}%
\end{pgfscope}%
\begin{pgfscope}%
\pgftext[x=0.234413in,y=2.031603in,,bottom,rotate=90.000000]{\rmfamily\fontsize{10.000000}{12.000000}\selectfont \(\displaystyle t_f\)}%
\end{pgfscope}%
\begin{pgfscope}%
\pgfpathrectangle{\pgfqpoint{0.564660in}{0.521603in}}{\pgfqpoint{4.650000in}{3.020000in}} %
\pgfusepath{clip}%
\pgfsetrectcap%
\pgfsetroundjoin%
\pgfsetlinewidth{1.505625pt}%
\definecolor{currentstroke}{rgb}{1.000000,0.000000,0.000000}%
\pgfsetstrokecolor{currentstroke}%
\pgfsetdash{}{0pt}%
\pgfpathmoveto{\pgfqpoint{0.776024in}{0.658881in}}%
\pgfpathlineto{\pgfqpoint{1.016911in}{0.663209in}}%
\pgfpathlineto{\pgfqpoint{1.207553in}{0.667552in}}%
\pgfpathlineto{\pgfqpoint{1.365336in}{0.671912in}}%
\pgfpathlineto{\pgfqpoint{1.499938in}{0.676287in}}%
\pgfpathlineto{\pgfqpoint{1.653317in}{0.682146in}}%
\pgfpathlineto{\pgfqpoint{1.784704in}{0.688033in}}%
\pgfpathlineto{\pgfqpoint{1.899621in}{0.693949in}}%
\pgfpathlineto{\pgfqpoint{2.001742in}{0.699894in}}%
\pgfpathlineto{\pgfqpoint{2.093633in}{0.705868in}}%
\pgfpathlineto{\pgfqpoint{2.196903in}{0.713376in}}%
\pgfpathlineto{\pgfqpoint{2.289723in}{0.720931in}}%
\pgfpathlineto{\pgfqpoint{2.374016in}{0.728532in}}%
\pgfpathlineto{\pgfqpoint{2.451216in}{0.736180in}}%
\pgfpathlineto{\pgfqpoint{2.522425in}{0.743875in}}%
\pgfpathlineto{\pgfqpoint{2.601171in}{0.753172in}}%
\pgfpathlineto{\pgfqpoint{2.673693in}{0.762539in}}%
\pgfpathlineto{\pgfqpoint{2.740904in}{0.771975in}}%
\pgfpathlineto{\pgfqpoint{2.803529in}{0.781481in}}%
\pgfpathlineto{\pgfqpoint{2.862154in}{0.791059in}}%
\pgfpathlineto{\pgfqpoint{2.926130in}{0.802323in}}%
\pgfpathlineto{\pgfqpoint{2.985937in}{0.813686in}}%
\pgfpathlineto{\pgfqpoint{3.042085in}{0.825149in}}%
\pgfpathlineto{\pgfqpoint{3.094997in}{0.836712in}}%
\pgfpathlineto{\pgfqpoint{3.145024in}{0.848377in}}%
\pgfpathlineto{\pgfqpoint{3.192467in}{0.860144in}}%
\pgfpathlineto{\pgfqpoint{3.237577in}{0.872015in}}%
\pgfpathlineto{\pgfqpoint{3.286556in}{0.885710in}}%
\pgfpathlineto{\pgfqpoint{3.333054in}{0.899544in}}%
\pgfpathlineto{\pgfqpoint{3.377311in}{0.913516in}}%
\pgfpathlineto{\pgfqpoint{3.419531in}{0.927630in}}%
\pgfpathlineto{\pgfqpoint{3.459896in}{0.941887in}}%
\pgfpathlineto{\pgfqpoint{3.498560in}{0.956287in}}%
\pgfpathlineto{\pgfqpoint{3.535662in}{0.970833in}}%
\pgfpathlineto{\pgfqpoint{3.571322in}{0.985527in}}%
\pgfpathlineto{\pgfqpoint{3.609851in}{1.002234in}}%
\pgfpathlineto{\pgfqpoint{3.646828in}{1.019132in}}%
\pgfpathlineto{\pgfqpoint{3.682373in}{1.036223in}}%
\pgfpathlineto{\pgfqpoint{3.716594in}{1.053507in}}%
\pgfpathlineto{\pgfqpoint{3.749584in}{1.070989in}}%
\pgfpathlineto{\pgfqpoint{3.781431in}{1.088669in}}%
\pgfpathlineto{\pgfqpoint{3.812209in}{1.106550in}}%
\pgfpathlineto{\pgfqpoint{3.841989in}{1.124634in}}%
\pgfpathlineto{\pgfqpoint{3.873983in}{1.144968in}}%
\pgfpathlineto{\pgfqpoint{3.904900in}{1.165558in}}%
\pgfpathlineto{\pgfqpoint{3.934810in}{1.186408in}}%
\pgfpathlineto{\pgfqpoint{3.963775in}{1.207519in}}%
\pgfpathlineto{\pgfqpoint{3.991855in}{1.228896in}}%
\pgfpathlineto{\pgfqpoint{4.019102in}{1.250540in}}%
\pgfpathlineto{\pgfqpoint{4.045563in}{1.272456in}}%
\pgfpathlineto{\pgfqpoint{4.071283in}{1.294645in}}%
\pgfpathlineto{\pgfqpoint{4.098767in}{1.319372in}}%
\pgfpathlineto{\pgfqpoint{4.125452in}{1.344439in}}%
\pgfpathlineto{\pgfqpoint{4.151383in}{1.369847in}}%
\pgfpathlineto{\pgfqpoint{4.176602in}{1.395603in}}%
\pgfpathlineto{\pgfqpoint{4.201147in}{1.421709in}}%
\pgfpathlineto{\pgfqpoint{4.225052in}{1.448169in}}%
\pgfpathlineto{\pgfqpoint{4.248351in}{1.474987in}}%
\pgfpathlineto{\pgfqpoint{4.273111in}{1.504656in}}%
\pgfpathlineto{\pgfqpoint{4.297221in}{1.534762in}}%
\pgfpathlineto{\pgfqpoint{4.320715in}{1.565308in}}%
\pgfpathlineto{\pgfqpoint{4.343622in}{1.596300in}}%
\pgfpathlineto{\pgfqpoint{4.365971in}{1.627744in}}%
\pgfpathlineto{\pgfqpoint{4.387789in}{1.659643in}}%
\pgfpathlineto{\pgfqpoint{4.410855in}{1.694722in}}%
\pgfpathlineto{\pgfqpoint{4.433355in}{1.730348in}}%
\pgfpathlineto{\pgfqpoint{4.455317in}{1.766529in}}%
\pgfpathlineto{\pgfqpoint{4.476766in}{1.803270in}}%
\pgfpathlineto{\pgfqpoint{4.497725in}{1.840578in}}%
\pgfpathlineto{\pgfqpoint{4.519774in}{1.881398in}}%
\pgfpathlineto{\pgfqpoint{4.541306in}{1.922891in}}%
\pgfpathlineto{\pgfqpoint{4.562345in}{1.965065in}}%
\pgfpathlineto{\pgfqpoint{4.582912in}{2.007928in}}%
\pgfpathlineto{\pgfqpoint{4.603029in}{2.051488in}}%
\pgfpathlineto{\pgfqpoint{4.624104in}{2.098943in}}%
\pgfpathlineto{\pgfqpoint{4.644706in}{2.147217in}}%
\pgfpathlineto{\pgfqpoint{4.664856in}{2.196320in}}%
\pgfpathlineto{\pgfqpoint{4.684574in}{2.246263in}}%
\pgfpathlineto{\pgfqpoint{4.705149in}{2.300471in}}%
\pgfpathlineto{\pgfqpoint{4.725274in}{2.355658in}}%
\pgfpathlineto{\pgfqpoint{4.744967in}{2.411834in}}%
\pgfpathlineto{\pgfqpoint{4.764246in}{2.469013in}}%
\pgfpathlineto{\pgfqpoint{4.784296in}{2.530876in}}%
\pgfpathlineto{\pgfqpoint{4.803917in}{2.593899in}}%
\pgfpathlineto{\pgfqpoint{4.823128in}{2.658095in}}%
\pgfpathlineto{\pgfqpoint{4.841946in}{2.723479in}}%
\pgfpathlineto{\pgfqpoint{4.861458in}{2.794020in}}%
\pgfpathlineto{\pgfqpoint{4.880565in}{2.865926in}}%
\pgfpathlineto{\pgfqpoint{4.899282in}{2.939215in}}%
\pgfpathlineto{\pgfqpoint{4.918633in}{3.018092in}}%
\pgfpathlineto{\pgfqpoint{4.937585in}{3.098549in}}%
\pgfpathlineto{\pgfqpoint{4.956154in}{3.180605in}}%
\pgfpathlineto{\pgfqpoint{4.974354in}{3.264280in}}%
\pgfpathlineto{\pgfqpoint{4.993131in}{3.354130in}}%
\pgfpathlineto{\pgfqpoint{5.003297in}{3.404331in}}%
\pgfpathlineto{\pgfqpoint{5.003297in}{3.404331in}}%
\pgfusepath{stroke}%
\end{pgfscope}%
\begin{pgfscope}%
\pgfpathrectangle{\pgfqpoint{0.564660in}{0.521603in}}{\pgfqpoint{4.650000in}{3.020000in}} %
\pgfusepath{clip}%
\pgfsetbuttcap%
\pgfsetroundjoin%
\pgfsetlinewidth{1.505625pt}%
\definecolor{currentstroke}{rgb}{0.000000,0.000000,0.000000}%
\pgfsetstrokecolor{currentstroke}%
\pgfsetdash{{5.550000pt}{2.400000pt}}{0.000000pt}%
\pgfpathmoveto{\pgfqpoint{0.776024in}{0.658876in}}%
\pgfpathlineto{\pgfqpoint{1.016911in}{0.663203in}}%
\pgfpathlineto{\pgfqpoint{1.207553in}{0.667545in}}%
\pgfpathlineto{\pgfqpoint{1.365336in}{0.671903in}}%
\pgfpathlineto{\pgfqpoint{1.499938in}{0.676277in}}%
\pgfpathlineto{\pgfqpoint{1.653317in}{0.682133in}}%
\pgfpathlineto{\pgfqpoint{1.784704in}{0.688018in}}%
\pgfpathlineto{\pgfqpoint{1.899621in}{0.693932in}}%
\pgfpathlineto{\pgfqpoint{2.001742in}{0.699875in}}%
\pgfpathlineto{\pgfqpoint{2.093633in}{0.705846in}}%
\pgfpathlineto{\pgfqpoint{2.196903in}{0.713352in}}%
\pgfpathlineto{\pgfqpoint{2.289723in}{0.720904in}}%
\pgfpathlineto{\pgfqpoint{2.374016in}{0.728502in}}%
\pgfpathlineto{\pgfqpoint{2.451216in}{0.736147in}}%
\pgfpathlineto{\pgfqpoint{2.522425in}{0.743840in}}%
\pgfpathlineto{\pgfqpoint{2.601171in}{0.753133in}}%
\pgfpathlineto{\pgfqpoint{2.673693in}{0.762496in}}%
\pgfpathlineto{\pgfqpoint{2.740904in}{0.771928in}}%
\pgfpathlineto{\pgfqpoint{2.803529in}{0.781431in}}%
\pgfpathlineto{\pgfqpoint{2.862154in}{0.791004in}}%
\pgfpathlineto{\pgfqpoint{2.926130in}{0.802264in}}%
\pgfpathlineto{\pgfqpoint{2.985937in}{0.813622in}}%
\pgfpathlineto{\pgfqpoint{3.042085in}{0.825080in}}%
\pgfpathlineto{\pgfqpoint{3.094997in}{0.836638in}}%
\pgfpathlineto{\pgfqpoint{3.145024in}{0.848297in}}%
\pgfpathlineto{\pgfqpoint{3.192467in}{0.860059in}}%
\pgfpathlineto{\pgfqpoint{3.237577in}{0.871925in}}%
\pgfpathlineto{\pgfqpoint{3.286556in}{0.885613in}}%
\pgfpathlineto{\pgfqpoint{3.333054in}{0.899440in}}%
\pgfpathlineto{\pgfqpoint{3.377311in}{0.913406in}}%
\pgfpathlineto{\pgfqpoint{3.419531in}{0.927513in}}%
\pgfpathlineto{\pgfqpoint{3.459896in}{0.941762in}}%
\pgfpathlineto{\pgfqpoint{3.498560in}{0.956155in}}%
\pgfpathlineto{\pgfqpoint{3.535662in}{0.970694in}}%
\pgfpathlineto{\pgfqpoint{3.575684in}{0.987225in}}%
\pgfpathlineto{\pgfqpoint{3.614034in}{1.003945in}}%
\pgfpathlineto{\pgfqpoint{3.650846in}{1.020855in}}%
\pgfpathlineto{\pgfqpoint{3.686239in}{1.037957in}}%
\pgfpathlineto{\pgfqpoint{3.720319in}{1.055254in}}%
\pgfpathlineto{\pgfqpoint{3.753178in}{1.072747in}}%
\pgfpathlineto{\pgfqpoint{3.784902in}{1.090439in}}%
\pgfpathlineto{\pgfqpoint{3.815566in}{1.108331in}}%
\pgfpathlineto{\pgfqpoint{3.845239in}{1.126427in}}%
\pgfpathlineto{\pgfqpoint{3.877122in}{1.146774in}}%
\pgfpathlineto{\pgfqpoint{3.907935in}{1.167377in}}%
\pgfpathlineto{\pgfqpoint{3.937748in}{1.188240in}}%
\pgfpathlineto{\pgfqpoint{3.966622in}{1.209364in}}%
\pgfpathlineto{\pgfqpoint{3.994617in}{1.230753in}}%
\pgfpathlineto{\pgfqpoint{4.021783in}{1.252410in}}%
\pgfpathlineto{\pgfqpoint{4.048168in}{1.274337in}}%
\pgfpathlineto{\pgfqpoint{4.073816in}{1.296538in}}%
\pgfpathlineto{\pgfqpoint{4.101225in}{1.321279in}}%
\pgfpathlineto{\pgfqpoint{4.127840in}{1.346359in}}%
\pgfpathlineto{\pgfqpoint{4.153704in}{1.371781in}}%
\pgfpathlineto{\pgfqpoint{4.178860in}{1.397549in}}%
\pgfpathlineto{\pgfqpoint{4.203346in}{1.423667in}}%
\pgfpathlineto{\pgfqpoint{4.227195in}{1.450140in}}%
\pgfpathlineto{\pgfqpoint{4.250440in}{1.476970in}}%
\pgfpathlineto{\pgfqpoint{4.275145in}{1.506653in}}%
\pgfpathlineto{\pgfqpoint{4.299202in}{1.536771in}}%
\pgfpathlineto{\pgfqpoint{4.322645in}{1.567330in}}%
\pgfpathlineto{\pgfqpoint{4.345505in}{1.598335in}}%
\pgfpathlineto{\pgfqpoint{4.367809in}{1.629790in}}%
\pgfpathlineto{\pgfqpoint{4.389584in}{1.661702in}}%
\pgfpathlineto{\pgfqpoint{4.412605in}{1.696793in}}%
\pgfpathlineto{\pgfqpoint{4.435063in}{1.732431in}}%
\pgfpathlineto{\pgfqpoint{4.456985in}{1.768623in}}%
\pgfpathlineto{\pgfqpoint{4.478395in}{1.805375in}}%
\pgfpathlineto{\pgfqpoint{4.499318in}{1.842694in}}%
\pgfpathlineto{\pgfqpoint{4.521329in}{1.883525in}}%
\pgfpathlineto{\pgfqpoint{4.542825in}{1.925028in}}%
\pgfpathlineto{\pgfqpoint{4.563829in}{1.967212in}}%
\pgfpathlineto{\pgfqpoint{4.584364in}{2.010085in}}%
\pgfpathlineto{\pgfqpoint{4.604449in}{2.053654in}}%
\pgfpathlineto{\pgfqpoint{4.625492in}{2.101118in}}%
\pgfpathlineto{\pgfqpoint{4.646063in}{2.149400in}}%
\pgfpathlineto{\pgfqpoint{4.666184in}{2.198511in}}%
\pgfpathlineto{\pgfqpoint{4.685873in}{2.248460in}}%
\pgfpathlineto{\pgfqpoint{4.706420in}{2.302675in}}%
\pgfpathlineto{\pgfqpoint{4.726517in}{2.357866in}}%
\pgfpathlineto{\pgfqpoint{4.746183in}{2.414047in}}%
\pgfpathlineto{\pgfqpoint{4.765437in}{2.471229in}}%
\pgfpathlineto{\pgfqpoint{4.785462in}{2.533096in}}%
\pgfpathlineto{\pgfqpoint{4.805059in}{2.596120in}}%
\pgfpathlineto{\pgfqpoint{4.824246in}{2.660317in}}%
\pgfpathlineto{\pgfqpoint{4.843041in}{2.725700in}}%
\pgfpathlineto{\pgfqpoint{4.862530in}{2.796239in}}%
\pgfpathlineto{\pgfqpoint{4.881615in}{2.868141in}}%
\pgfpathlineto{\pgfqpoint{4.900311in}{2.941424in}}%
\pgfpathlineto{\pgfqpoint{4.919641in}{3.020295in}}%
\pgfpathlineto{\pgfqpoint{4.938572in}{3.100744in}}%
\pgfpathlineto{\pgfqpoint{4.957121in}{3.182789in}}%
\pgfpathlineto{\pgfqpoint{4.975302in}{3.266452in}}%
\pgfpathlineto{\pgfqpoint{4.994060in}{3.356287in}}%
\pgfpathlineto{\pgfqpoint{5.003297in}{3.401893in}}%
\pgfpathlineto{\pgfqpoint{5.003297in}{3.401893in}}%
\pgfusepath{stroke}%
\end{pgfscope}%
\begin{pgfscope}%
\pgfsetrectcap%
\pgfsetmiterjoin%
\pgfsetlinewidth{0.803000pt}%
\definecolor{currentstroke}{rgb}{0.000000,0.000000,0.000000}%
\pgfsetstrokecolor{currentstroke}%
\pgfsetdash{}{0pt}%
\pgfpathmoveto{\pgfqpoint{0.564660in}{0.521603in}}%
\pgfpathlineto{\pgfqpoint{0.564660in}{3.541603in}}%
\pgfusepath{stroke}%
\end{pgfscope}%
\begin{pgfscope}%
\pgfsetrectcap%
\pgfsetmiterjoin%
\pgfsetlinewidth{0.803000pt}%
\definecolor{currentstroke}{rgb}{0.000000,0.000000,0.000000}%
\pgfsetstrokecolor{currentstroke}%
\pgfsetdash{}{0pt}%
\pgfpathmoveto{\pgfqpoint{5.214660in}{0.521603in}}%
\pgfpathlineto{\pgfqpoint{5.214660in}{3.541603in}}%
\pgfusepath{stroke}%
\end{pgfscope}%
\begin{pgfscope}%
\pgfsetrectcap%
\pgfsetmiterjoin%
\pgfsetlinewidth{0.803000pt}%
\definecolor{currentstroke}{rgb}{0.000000,0.000000,0.000000}%
\pgfsetstrokecolor{currentstroke}%
\pgfsetdash{}{0pt}%
\pgfpathmoveto{\pgfqpoint{0.564660in}{0.521603in}}%
\pgfpathlineto{\pgfqpoint{5.214660in}{0.521603in}}%
\pgfusepath{stroke}%
\end{pgfscope}%
\begin{pgfscope}%
\pgfsetrectcap%
\pgfsetmiterjoin%
\pgfsetlinewidth{0.803000pt}%
\definecolor{currentstroke}{rgb}{0.000000,0.000000,0.000000}%
\pgfsetstrokecolor{currentstroke}%
\pgfsetdash{}{0pt}%
\pgfpathmoveto{\pgfqpoint{0.564660in}{3.541603in}}%
\pgfpathlineto{\pgfqpoint{5.214660in}{3.541603in}}%
\pgfusepath{stroke}%
\end{pgfscope}%
\begin{pgfscope}%
\pgfsetbuttcap%
\pgfsetmiterjoin%
\definecolor{currentfill}{rgb}{1.000000,1.000000,1.000000}%
\pgfsetfillcolor{currentfill}%
\pgfsetfillopacity{0.800000}%
\pgfsetlinewidth{1.003750pt}%
\definecolor{currentstroke}{rgb}{0.800000,0.800000,0.800000}%
\pgfsetstrokecolor{currentstroke}%
\pgfsetstrokeopacity{0.800000}%
\pgfsetdash{}{0pt}%
\pgfpathmoveto{\pgfqpoint{0.661883in}{3.004367in}}%
\pgfpathlineto{\pgfqpoint{2.004742in}{3.004367in}}%
\pgfpathquadraticcurveto{\pgfqpoint{2.032519in}{3.004367in}}{\pgfqpoint{2.032519in}{3.032145in}}%
\pgfpathlineto{\pgfqpoint{2.032519in}{3.444381in}}%
\pgfpathquadraticcurveto{\pgfqpoint{2.032519in}{3.472159in}}{\pgfqpoint{2.004742in}{3.472159in}}%
\pgfpathlineto{\pgfqpoint{0.661883in}{3.472159in}}%
\pgfpathquadraticcurveto{\pgfqpoint{0.634105in}{3.472159in}}{\pgfqpoint{0.634105in}{3.444381in}}%
\pgfpathlineto{\pgfqpoint{0.634105in}{3.032145in}}%
\pgfpathquadraticcurveto{\pgfqpoint{0.634105in}{3.004367in}}{\pgfqpoint{0.661883in}{3.004367in}}%
\pgfpathclose%
\pgfusepath{stroke,fill}%
\end{pgfscope}%
\begin{pgfscope}%
\pgfsetrectcap%
\pgfsetroundjoin%
\pgfsetlinewidth{1.505625pt}%
\definecolor{currentstroke}{rgb}{1.000000,0.000000,0.000000}%
\pgfsetstrokecolor{currentstroke}%
\pgfsetdash{}{0pt}%
\pgfpathmoveto{\pgfqpoint{0.689660in}{3.359691in}}%
\pgfpathlineto{\pgfqpoint{0.967438in}{3.359691in}}%
\pgfusepath{stroke}%
\end{pgfscope}%
\begin{pgfscope}%
\pgftext[x=1.078549in,y=3.311080in,left,base]{\rmfamily\fontsize{10.000000}{12.000000}\selectfont \(\displaystyle \mathbf{D}\mbox{g}\) = 0}%
\end{pgfscope}%
\begin{pgfscope}%
\pgfsetbuttcap%
\pgfsetroundjoin%
\pgfsetlinewidth{1.505625pt}%
\definecolor{currentstroke}{rgb}{0.000000,0.000000,0.000000}%
\pgfsetstrokecolor{currentstroke}%
\pgfsetdash{{5.550000pt}{2.400000pt}}{0.000000pt}%
\pgfpathmoveto{\pgfqpoint{0.689660in}{3.139390in}}%
\pgfpathlineto{\pgfqpoint{0.967438in}{3.139390in}}%
\pgfusepath{stroke}%
\end{pgfscope}%
\begin{pgfscope}%
\pgftext[x=1.078549in,y=3.090779in,left,base]{\rmfamily\fontsize{10.000000}{12.000000}\selectfont \(\displaystyle \mathbf{D}\mbox{g}\) = 6\(\displaystyle \times 10^{-4}\)}%
\end{pgfscope}%
\end{pgfpicture}%
\makeatother%
\endgroup%

    \caption{Text.}
    \label{fig:drag}
\end{figure}
\end{document}

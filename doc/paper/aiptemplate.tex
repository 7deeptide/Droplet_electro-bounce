%% ****** Start of file aiptemplate.tex ****** %
%%
%%   This file is part of the files in the distribution of AIP substyles for REVTeX4.
%%   Version 4.1 of 9 October 2009.
%%
%
% This is a template for producing documents for use with 
% the REVTEX 4.1 document class and the AIP substyles.
% 
% Copy this file to another name and then work on that file.
% That way, you always have this original template file to use.

%\documentclass[aip,graphicx]{revtex4-1}
\documentclass[aip,reprint, floatfix]{revtex4-1}

\draft % marks overfull lines with a black rule on the right
\usepackage{amsmath}
\usepackage{amsfonts}
\usepackage{amssymb}
\usepackage{graphicx}
\usepackage{pgfplots}
\usepackage{pgf}
\usepackage{color,soul}
\usepackage{color}
\usepackage{bm}% bold math
\usepackage{tikz}
\usepackage{import}
\usepackage[normalem]{ulem}
\let\pgfimageWithoutPath\pgfimage 
\renewcommand{\pgfimage}[2][]{\pgfimageWithoutPath[#1]{../figures/#2}}
\begin{document}

\newcommand{\redline}{\raisebox{2pt}{\tikz{\draw[-,red,solid,line width = 1.5pt](0,0) -- (5mm,0);}}}
\graphicspath{ {../figures/} }

\newcommand{\blueline}{\raisebox{2pt}{\tikz{\draw[-,blue,solid,line width = 1.5pt](0,0) -- (5mm,0);}}}
\graphicspath{ {../figures/} }

\newcommand{\cyanline}{\raisebox{2pt}{\tikz{\draw[-,cyan,solid,line width = 1.5pt](0,0) -- (5mm,0);}}}
\graphicspath{ {../figures/} }

% Use the \preprint command to place your local institutional report number 
% on the title page in preprint mode.
% Multiple \preprint commands are allowed.
%\preprint{}

\title{Electro-Drop Bouncing in Low-Gravity} %Title of paper

% repeat the \author .. \affiliation  etc. as needed
% \email, \thanks, \homepage, \altaffiliation all apply to the current author.
% Explanatory text should go in the []'s, 
% actual e-mail address or url should go in the {}'s for \email and \homepage.
% Please use the appropriate macro for the type of information

% \affiliation command applies to all authors since the last \affiliation command. 
% The \affiliation command should follow the other information.

\author{Erin S. Schmidt}
%\email[]{Your e-mail address}
%\homepage[]{Your web page}
%\thanks{}
%\altaffiliation{}
\affiliation{Portland State University}

\author{Mark M. Weislogel}
\affiliation{Portland State University}

% Collaboration name, if desired (requires use of superscriptaddress option in \documentclass). 
% \noaffiliation is required (may also be used with the \author command).
%\collaboration{}
%\noaffiliation

\date{\today}

\begin{abstract}
% insert abstract here

We investigate the dynamics of spontaneous jumps of water drops from electrically charged superhydrophobic dielectric substrates during a sudden step reduction in gravity level. In the brief free-fall environment of a drop tower, with an external electric field drop dynamics are dominated by Coulombic force instead of gravity. These forces lead to a drop bouncing behavior similar to well-known terrestrial phenomena, though occurring for much larger drops ($\sim$ 0.5 mL). We provide a 1-dimensional model for the phenomenon, its scaling, and asymptotic estimates for drop time-of-flight in two regimes: at short-times close to the substrate when drop inertia balances Coulombic force due to net free charge and image charges in the dielectric substrate and at long-times far from the substrate when drop inertia balances free charge Coulombic force and drag. In both regimes the dimensionless electrostatic Euler number $\mathbb{E}\mbox{u}$, which is a ratio of inertia to electrostatic force, appears as a key parameter.
\end{abstract}

\pacs{}% insert suggested PACS numbers in braces on next line

\maketitle %\maketitle must follow title, authors, abstract and \pacs

% Body of paper goes here. Use proper sectioning commands. 
% References should be done using the \cite, \ref, and \label commands
\section{Spontaneous Drop Jump}
When a nonwetting, gravity-dominated sessile drop (e.g. a puddle)\sout{, initially at rest on a surface in the Cassie-Baxter state} suddenly undergoes a large step reduction in Bond number it will spontaneously jump away from the surface. The Bond number is given by $\mathbb{B}\mbox{o} \equiv \frac{g R^2 \Delta \rho}{\gamma}$, where $g$ is the acceleration, $R$ is the characteristic interfacial length scale, $\Delta \rho$ is the difference in densities across the interface \sout{(which simplifies to $\rho$, for large $\Delta \rho$, as in the case of an air-water interface)}, and where $\gamma$ is the surface tension. The spontaneous drop jump was first observed experimentally in the Soviet Union by Kirko \emph{et al.} \cite{kirko_phenomenon_1970} in 1970 for drops of mercury in hydrochloric acid, and later by Wollman \emph{et al.} in 2016 for water drops in air in a set of experiments conducted using drop towers \cite{wollman_more_2016}. \sout{The kinetic energy of the jump is supplied by the defect in free surface energy as the new minimum energy surface equilibrium has approximately constant curvature.} The motive force of the jump owes to the inertia of internal flows which occur as the interface deflects under the suddenly lessened $\mathbb{B}\mbox{o}$. For drops with radial symmetry and sufficiently high initial $\mathbb{B}\mbox{o}$, inward radial capillary waves constructively interfere at the axis, leading to geysering and creation of satellite droplets by the Rayleigh-Plateau breakup of the geyser. In the case of smaller jumping drops the capillary waves are quickly damped by viscous forces.

The physics of these relatively massive drops (far beyond the 1-$g_0$ millimetric capillary length scale) at once utterly defy terrestrial expectations about the ways in which liquid `should' behave, and also are of critical practical importance to space systems design where examples of such large capillary length scale multiphase flows are commonplace.

During the `rolling-up' of drops under ideal conditions, the spontaneous jump phenomenon is governed by a balance of inertia and surface tension forces, and once aloft the drop motion is nominally in a regime of pure drag. However, other forces can come into play. 

In the course of our work with spontaneous drop jump we have observed jumped drops to occasionally decelerate and return to the superhydrophobic surface, rebounding multiple times in the fashion of rigid bodies bouncing under 1-$g_0$. The forces at work in such situations are presumably electrostatic in origin. A time-lapsed composite image showing an example of the phenomenon is shown in Figure \ref{fig:bounce} for a 0.5 mL drop, and as a time-series in Figure \ref{fig:bounce_time} for a 0.05 mL drop.
\begin{figure}[htb]
\centering
\includegraphics[width=0.5\textwidth]{bounce}
\caption{The trajectory of a 0.5 mL drop is captured in a composite image over a single bounce period ($\sim 1.25$ s) presented at $\approx 10$ Hz. The surface potential of the superhydrophobic dielectric is $\varphi_s = 1.25 \pm 0.41$ kV. \label{fig:bounce}}
\end{figure}

\begin{figure}[htb]
\centering
\resizebox{0.5\textwidth}{!}{%% Creator: Matplotlib, PGF backend
%%
%% To include the figure in your LaTeX document, write
%%   \input{<filename>.pgf}
%%
%% Make sure the required packages are loaded in your preamble
%%   \usepackage{pgf}
%%
%% Figures using additional raster images can only be included by \input if
%% they are in the same directory as the main LaTeX file. For loading figures
%% from other directories you can use the `import` package
%%   \usepackage{import}
%% and then include the figures with
%%   \import{<path to file>}{<filename>.pgf}
%%
%% Matplotlib used the following preamble
%%   \usepackage{fontspec}
%%   \setmainfont{DejaVuSerif.ttf}[Path=/home/erin/anaconda3/lib/python3.6/site-packages/matplotlib/mpl-data/fonts/ttf/]
%%   \setsansfont{DejaVuSans.ttf}[Path=/home/erin/anaconda3/lib/python3.6/site-packages/matplotlib/mpl-data/fonts/ttf/]
%%   \setmonofont{DejaVuSansMono.ttf}[Path=/home/erin/anaconda3/lib/python3.6/site-packages/matplotlib/mpl-data/fonts/ttf/]
%%
\begingroup%
\makeatletter%
\begin{pgfpicture}%
\pgfpathrectangle{\pgfpointorigin}{\pgfqpoint{5.441184in}{3.749134in}}%
\pgfusepath{use as bounding box, clip}%
\begin{pgfscope}%
\pgfsetbuttcap%
\pgfsetmiterjoin%
\definecolor{currentfill}{rgb}{1.000000,1.000000,1.000000}%
\pgfsetfillcolor{currentfill}%
\pgfsetlinewidth{0.000000pt}%
\definecolor{currentstroke}{rgb}{1.000000,1.000000,1.000000}%
\pgfsetstrokecolor{currentstroke}%
\pgfsetdash{}{0pt}%
\pgfpathmoveto{\pgfqpoint{0.000000in}{0.000000in}}%
\pgfpathlineto{\pgfqpoint{5.441184in}{0.000000in}}%
\pgfpathlineto{\pgfqpoint{5.441184in}{3.749134in}}%
\pgfpathlineto{\pgfqpoint{0.000000in}{3.749134in}}%
\pgfpathclose%
\pgfusepath{fill}%
\end{pgfscope}%
\begin{pgfscope}%
\pgfsetbuttcap%
\pgfsetmiterjoin%
\definecolor{currentfill}{rgb}{1.000000,1.000000,1.000000}%
\pgfsetfillcolor{currentfill}%
\pgfsetlinewidth{0.000000pt}%
\definecolor{currentstroke}{rgb}{0.000000,0.000000,0.000000}%
\pgfsetstrokecolor{currentstroke}%
\pgfsetstrokeopacity{0.000000}%
\pgfsetdash{}{0pt}%
\pgfpathmoveto{\pgfqpoint{0.691184in}{0.629134in}}%
\pgfpathlineto{\pgfqpoint{5.341184in}{0.629134in}}%
\pgfpathlineto{\pgfqpoint{5.341184in}{3.649134in}}%
\pgfpathlineto{\pgfqpoint{0.691184in}{3.649134in}}%
\pgfpathclose%
\pgfusepath{fill}%
\end{pgfscope}%
\begin{pgfscope}%
\pgfsetbuttcap%
\pgfsetroundjoin%
\definecolor{currentfill}{rgb}{0.000000,0.000000,0.000000}%
\pgfsetfillcolor{currentfill}%
\pgfsetlinewidth{0.803000pt}%
\definecolor{currentstroke}{rgb}{0.000000,0.000000,0.000000}%
\pgfsetstrokecolor{currentstroke}%
\pgfsetdash{}{0pt}%
\pgfsys@defobject{currentmarker}{\pgfqpoint{0.000000in}{-0.048611in}}{\pgfqpoint{0.000000in}{0.000000in}}{%
\pgfpathmoveto{\pgfqpoint{0.000000in}{0.000000in}}%
\pgfpathlineto{\pgfqpoint{0.000000in}{-0.048611in}}%
\pgfusepath{stroke,fill}%
}%
\begin{pgfscope}%
\pgfsys@transformshift{0.748127in}{0.629134in}%
\pgfsys@useobject{currentmarker}{}%
\end{pgfscope}%
\end{pgfscope}%
\begin{pgfscope}%
\definecolor{textcolor}{rgb}{0.000000,0.000000,0.000000}%
\pgfsetstrokecolor{textcolor}%
\pgfsetfillcolor{textcolor}%
\pgftext[x=0.748127in,y=0.531912in,,top]{\color{textcolor}\rmfamily\fontsize{14.000000}{16.800000}\selectfont \(\displaystyle 0.0\)}%
\end{pgfscope}%
\begin{pgfscope}%
\pgfsetbuttcap%
\pgfsetroundjoin%
\definecolor{currentfill}{rgb}{0.000000,0.000000,0.000000}%
\pgfsetfillcolor{currentfill}%
\pgfsetlinewidth{0.803000pt}%
\definecolor{currentstroke}{rgb}{0.000000,0.000000,0.000000}%
\pgfsetstrokecolor{currentstroke}%
\pgfsetdash{}{0pt}%
\pgfsys@defobject{currentmarker}{\pgfqpoint{0.000000in}{-0.048611in}}{\pgfqpoint{0.000000in}{0.000000in}}{%
\pgfpathmoveto{\pgfqpoint{0.000000in}{0.000000in}}%
\pgfpathlineto{\pgfqpoint{0.000000in}{-0.048611in}}%
\pgfusepath{stroke,fill}%
}%
\begin{pgfscope}%
\pgfsys@transformshift{1.906284in}{0.629134in}%
\pgfsys@useobject{currentmarker}{}%
\end{pgfscope}%
\end{pgfscope}%
\begin{pgfscope}%
\definecolor{textcolor}{rgb}{0.000000,0.000000,0.000000}%
\pgfsetstrokecolor{textcolor}%
\pgfsetfillcolor{textcolor}%
\pgftext[x=1.906284in,y=0.531912in,,top]{\color{textcolor}\rmfamily\fontsize{14.000000}{16.800000}\selectfont \(\displaystyle 0.5\)}%
\end{pgfscope}%
\begin{pgfscope}%
\pgfsetbuttcap%
\pgfsetroundjoin%
\definecolor{currentfill}{rgb}{0.000000,0.000000,0.000000}%
\pgfsetfillcolor{currentfill}%
\pgfsetlinewidth{0.803000pt}%
\definecolor{currentstroke}{rgb}{0.000000,0.000000,0.000000}%
\pgfsetstrokecolor{currentstroke}%
\pgfsetdash{}{0pt}%
\pgfsys@defobject{currentmarker}{\pgfqpoint{0.000000in}{-0.048611in}}{\pgfqpoint{0.000000in}{0.000000in}}{%
\pgfpathmoveto{\pgfqpoint{0.000000in}{0.000000in}}%
\pgfpathlineto{\pgfqpoint{0.000000in}{-0.048611in}}%
\pgfusepath{stroke,fill}%
}%
\begin{pgfscope}%
\pgfsys@transformshift{3.064441in}{0.629134in}%
\pgfsys@useobject{currentmarker}{}%
\end{pgfscope}%
\end{pgfscope}%
\begin{pgfscope}%
\definecolor{textcolor}{rgb}{0.000000,0.000000,0.000000}%
\pgfsetstrokecolor{textcolor}%
\pgfsetfillcolor{textcolor}%
\pgftext[x=3.064441in,y=0.531912in,,top]{\color{textcolor}\rmfamily\fontsize{14.000000}{16.800000}\selectfont \(\displaystyle 1.0\)}%
\end{pgfscope}%
\begin{pgfscope}%
\pgfsetbuttcap%
\pgfsetroundjoin%
\definecolor{currentfill}{rgb}{0.000000,0.000000,0.000000}%
\pgfsetfillcolor{currentfill}%
\pgfsetlinewidth{0.803000pt}%
\definecolor{currentstroke}{rgb}{0.000000,0.000000,0.000000}%
\pgfsetstrokecolor{currentstroke}%
\pgfsetdash{}{0pt}%
\pgfsys@defobject{currentmarker}{\pgfqpoint{0.000000in}{-0.048611in}}{\pgfqpoint{0.000000in}{0.000000in}}{%
\pgfpathmoveto{\pgfqpoint{0.000000in}{0.000000in}}%
\pgfpathlineto{\pgfqpoint{0.000000in}{-0.048611in}}%
\pgfusepath{stroke,fill}%
}%
\begin{pgfscope}%
\pgfsys@transformshift{4.222597in}{0.629134in}%
\pgfsys@useobject{currentmarker}{}%
\end{pgfscope}%
\end{pgfscope}%
\begin{pgfscope}%
\definecolor{textcolor}{rgb}{0.000000,0.000000,0.000000}%
\pgfsetstrokecolor{textcolor}%
\pgfsetfillcolor{textcolor}%
\pgftext[x=4.222597in,y=0.531912in,,top]{\color{textcolor}\rmfamily\fontsize{14.000000}{16.800000}\selectfont \(\displaystyle 1.5\)}%
\end{pgfscope}%
\begin{pgfscope}%
\definecolor{textcolor}{rgb}{0.000000,0.000000,0.000000}%
\pgfsetstrokecolor{textcolor}%
\pgfsetfillcolor{textcolor}%
\pgftext[x=3.016184in,y=0.288178in,,top]{\color{textcolor}\rmfamily\fontsize{14.000000}{16.800000}\selectfont \(\displaystyle t\) (s)}%
\end{pgfscope}%
\begin{pgfscope}%
\pgfsetbuttcap%
\pgfsetroundjoin%
\definecolor{currentfill}{rgb}{0.000000,0.000000,0.000000}%
\pgfsetfillcolor{currentfill}%
\pgfsetlinewidth{0.803000pt}%
\definecolor{currentstroke}{rgb}{0.000000,0.000000,0.000000}%
\pgfsetstrokecolor{currentstroke}%
\pgfsetdash{}{0pt}%
\pgfsys@defobject{currentmarker}{\pgfqpoint{-0.048611in}{0.000000in}}{\pgfqpoint{0.000000in}{0.000000in}}{%
\pgfpathmoveto{\pgfqpoint{0.000000in}{0.000000in}}%
\pgfpathlineto{\pgfqpoint{-0.048611in}{0.000000in}}%
\pgfusepath{stroke,fill}%
}%
\begin{pgfscope}%
\pgfsys@transformshift{0.691184in}{1.111133in}%
\pgfsys@useobject{currentmarker}{}%
\end{pgfscope}%
\end{pgfscope}%
\begin{pgfscope}%
\definecolor{textcolor}{rgb}{0.000000,0.000000,0.000000}%
\pgfsetstrokecolor{textcolor}%
\pgfsetfillcolor{textcolor}%
\pgftext[x=0.343734in,y=1.037267in,left,base]{\color{textcolor}\rmfamily\fontsize{14.000000}{16.800000}\selectfont \(\displaystyle 0.4\)}%
\end{pgfscope}%
\begin{pgfscope}%
\pgfsetbuttcap%
\pgfsetroundjoin%
\definecolor{currentfill}{rgb}{0.000000,0.000000,0.000000}%
\pgfsetfillcolor{currentfill}%
\pgfsetlinewidth{0.803000pt}%
\definecolor{currentstroke}{rgb}{0.000000,0.000000,0.000000}%
\pgfsetstrokecolor{currentstroke}%
\pgfsetdash{}{0pt}%
\pgfsys@defobject{currentmarker}{\pgfqpoint{-0.048611in}{0.000000in}}{\pgfqpoint{0.000000in}{0.000000in}}{%
\pgfpathmoveto{\pgfqpoint{0.000000in}{0.000000in}}%
\pgfpathlineto{\pgfqpoint{-0.048611in}{0.000000in}}%
\pgfusepath{stroke,fill}%
}%
\begin{pgfscope}%
\pgfsys@transformshift{0.691184in}{1.777765in}%
\pgfsys@useobject{currentmarker}{}%
\end{pgfscope}%
\end{pgfscope}%
\begin{pgfscope}%
\definecolor{textcolor}{rgb}{0.000000,0.000000,0.000000}%
\pgfsetstrokecolor{textcolor}%
\pgfsetfillcolor{textcolor}%
\pgftext[x=0.343734in,y=1.703899in,left,base]{\color{textcolor}\rmfamily\fontsize{14.000000}{16.800000}\selectfont \(\displaystyle 0.6\)}%
\end{pgfscope}%
\begin{pgfscope}%
\pgfsetbuttcap%
\pgfsetroundjoin%
\definecolor{currentfill}{rgb}{0.000000,0.000000,0.000000}%
\pgfsetfillcolor{currentfill}%
\pgfsetlinewidth{0.803000pt}%
\definecolor{currentstroke}{rgb}{0.000000,0.000000,0.000000}%
\pgfsetstrokecolor{currentstroke}%
\pgfsetdash{}{0pt}%
\pgfsys@defobject{currentmarker}{\pgfqpoint{-0.048611in}{0.000000in}}{\pgfqpoint{0.000000in}{0.000000in}}{%
\pgfpathmoveto{\pgfqpoint{0.000000in}{0.000000in}}%
\pgfpathlineto{\pgfqpoint{-0.048611in}{0.000000in}}%
\pgfusepath{stroke,fill}%
}%
\begin{pgfscope}%
\pgfsys@transformshift{0.691184in}{2.444397in}%
\pgfsys@useobject{currentmarker}{}%
\end{pgfscope}%
\end{pgfscope}%
\begin{pgfscope}%
\definecolor{textcolor}{rgb}{0.000000,0.000000,0.000000}%
\pgfsetstrokecolor{textcolor}%
\pgfsetfillcolor{textcolor}%
\pgftext[x=0.343734in,y=2.370531in,left,base]{\color{textcolor}\rmfamily\fontsize{14.000000}{16.800000}\selectfont \(\displaystyle 0.8\)}%
\end{pgfscope}%
\begin{pgfscope}%
\pgfsetbuttcap%
\pgfsetroundjoin%
\definecolor{currentfill}{rgb}{0.000000,0.000000,0.000000}%
\pgfsetfillcolor{currentfill}%
\pgfsetlinewidth{0.803000pt}%
\definecolor{currentstroke}{rgb}{0.000000,0.000000,0.000000}%
\pgfsetstrokecolor{currentstroke}%
\pgfsetdash{}{0pt}%
\pgfsys@defobject{currentmarker}{\pgfqpoint{-0.048611in}{0.000000in}}{\pgfqpoint{0.000000in}{0.000000in}}{%
\pgfpathmoveto{\pgfqpoint{0.000000in}{0.000000in}}%
\pgfpathlineto{\pgfqpoint{-0.048611in}{0.000000in}}%
\pgfusepath{stroke,fill}%
}%
\begin{pgfscope}%
\pgfsys@transformshift{0.691184in}{3.111029in}%
\pgfsys@useobject{currentmarker}{}%
\end{pgfscope}%
\end{pgfscope}%
\begin{pgfscope}%
\definecolor{textcolor}{rgb}{0.000000,0.000000,0.000000}%
\pgfsetstrokecolor{textcolor}%
\pgfsetfillcolor{textcolor}%
\pgftext[x=0.343734in,y=3.037163in,left,base]{\color{textcolor}\rmfamily\fontsize{14.000000}{16.800000}\selectfont \(\displaystyle 1.0\)}%
\end{pgfscope}%
\begin{pgfscope}%
\definecolor{textcolor}{rgb}{0.000000,0.000000,0.000000}%
\pgfsetstrokecolor{textcolor}%
\pgfsetfillcolor{textcolor}%
\pgftext[x=0.288178in,y=2.139134in,,bottom,rotate=90.000000]{\color{textcolor}\rmfamily\fontsize{14.000000}{16.800000}\selectfont \(\displaystyle y\) (cm)}%
\end{pgfscope}%
\begin{pgfscope}%
\pgfpathrectangle{\pgfqpoint{0.691184in}{0.629134in}}{\pgfqpoint{4.650000in}{3.020000in}}%
\pgfusepath{clip}%
\pgfsetbuttcap%
\pgfsetroundjoin%
\definecolor{currentfill}{rgb}{1.000000,1.000000,1.000000}%
\pgfsetfillcolor{currentfill}%
\pgfsetlinewidth{1.003750pt}%
\definecolor{currentstroke}{rgb}{0.000000,0.000000,0.000000}%
\pgfsetstrokecolor{currentstroke}%
\pgfsetdash{}{0pt}%
\pgfsys@defobject{currentmarker}{\pgfqpoint{-0.041667in}{-0.041667in}}{\pgfqpoint{0.041667in}{0.041667in}}{%
\pgfpathmoveto{\pgfqpoint{0.000000in}{-0.041667in}}%
\pgfpathcurveto{\pgfqpoint{0.011050in}{-0.041667in}}{\pgfqpoint{0.021649in}{-0.037276in}}{\pgfqpoint{0.029463in}{-0.029463in}}%
\pgfpathcurveto{\pgfqpoint{0.037276in}{-0.021649in}}{\pgfqpoint{0.041667in}{-0.011050in}}{\pgfqpoint{0.041667in}{0.000000in}}%
\pgfpathcurveto{\pgfqpoint{0.041667in}{0.011050in}}{\pgfqpoint{0.037276in}{0.021649in}}{\pgfqpoint{0.029463in}{0.029463in}}%
\pgfpathcurveto{\pgfqpoint{0.021649in}{0.037276in}}{\pgfqpoint{0.011050in}{0.041667in}}{\pgfqpoint{0.000000in}{0.041667in}}%
\pgfpathcurveto{\pgfqpoint{-0.011050in}{0.041667in}}{\pgfqpoint{-0.021649in}{0.037276in}}{\pgfqpoint{-0.029463in}{0.029463in}}%
\pgfpathcurveto{\pgfqpoint{-0.037276in}{0.021649in}}{\pgfqpoint{-0.041667in}{0.011050in}}{\pgfqpoint{-0.041667in}{0.000000in}}%
\pgfpathcurveto{\pgfqpoint{-0.041667in}{-0.011050in}}{\pgfqpoint{-0.037276in}{-0.021649in}}{\pgfqpoint{-0.029463in}{-0.029463in}}%
\pgfpathcurveto{\pgfqpoint{-0.021649in}{-0.037276in}}{\pgfqpoint{-0.011050in}{-0.041667in}}{\pgfqpoint{0.000000in}{-0.041667in}}%
\pgfpathclose%
\pgfusepath{stroke,fill}%
}%
\begin{pgfscope}%
\pgfsys@transformshift{0.902548in}{1.544403in}%
\pgfsys@useobject{currentmarker}{}%
\end{pgfscope}%
\begin{pgfscope}%
\pgfsys@transformshift{0.921850in}{1.686653in}%
\pgfsys@useobject{currentmarker}{}%
\end{pgfscope}%
\begin{pgfscope}%
\pgfsys@transformshift{0.941153in}{1.823094in}%
\pgfsys@useobject{currentmarker}{}%
\end{pgfscope}%
\begin{pgfscope}%
\pgfsys@transformshift{0.960456in}{1.953784in}%
\pgfsys@useobject{currentmarker}{}%
\end{pgfscope}%
\begin{pgfscope}%
\pgfsys@transformshift{0.979758in}{2.078778in}%
\pgfsys@useobject{currentmarker}{}%
\end{pgfscope}%
\begin{pgfscope}%
\pgfsys@transformshift{0.999061in}{2.198130in}%
\pgfsys@useobject{currentmarker}{}%
\end{pgfscope}%
\begin{pgfscope}%
\pgfsys@transformshift{1.018363in}{2.311897in}%
\pgfsys@useobject{currentmarker}{}%
\end{pgfscope}%
\begin{pgfscope}%
\pgfsys@transformshift{1.037666in}{2.420133in}%
\pgfsys@useobject{currentmarker}{}%
\end{pgfscope}%
\begin{pgfscope}%
\pgfsys@transformshift{1.056969in}{2.522895in}%
\pgfsys@useobject{currentmarker}{}%
\end{pgfscope}%
\begin{pgfscope}%
\pgfsys@transformshift{1.076271in}{2.620237in}%
\pgfsys@useobject{currentmarker}{}%
\end{pgfscope}%
\begin{pgfscope}%
\pgfsys@transformshift{1.095574in}{2.712215in}%
\pgfsys@useobject{currentmarker}{}%
\end{pgfscope}%
\begin{pgfscope}%
\pgfsys@transformshift{1.114876in}{2.798885in}%
\pgfsys@useobject{currentmarker}{}%
\end{pgfscope}%
\begin{pgfscope}%
\pgfsys@transformshift{1.134179in}{2.880301in}%
\pgfsys@useobject{currentmarker}{}%
\end{pgfscope}%
\begin{pgfscope}%
\pgfsys@transformshift{1.153482in}{2.956535in}%
\pgfsys@useobject{currentmarker}{}%
\end{pgfscope}%
\begin{pgfscope}%
\pgfsys@transformshift{1.172784in}{3.027639in}%
\pgfsys@useobject{currentmarker}{}%
\end{pgfscope}%
\begin{pgfscope}%
\pgfsys@transformshift{1.192087in}{3.093668in}%
\pgfsys@useobject{currentmarker}{}%
\end{pgfscope}%
\begin{pgfscope}%
\pgfsys@transformshift{1.211390in}{3.154679in}%
\pgfsys@useobject{currentmarker}{}%
\end{pgfscope}%
\begin{pgfscope}%
\pgfsys@transformshift{1.230692in}{3.210726in}%
\pgfsys@useobject{currentmarker}{}%
\end{pgfscope}%
\begin{pgfscope}%
\pgfsys@transformshift{1.249995in}{3.261859in}%
\pgfsys@useobject{currentmarker}{}%
\end{pgfscope}%
\begin{pgfscope}%
\pgfsys@transformshift{1.269297in}{3.308129in}%
\pgfsys@useobject{currentmarker}{}%
\end{pgfscope}%
\begin{pgfscope}%
\pgfsys@transformshift{1.288600in}{3.349580in}%
\pgfsys@useobject{currentmarker}{}%
\end{pgfscope}%
\begin{pgfscope}%
\pgfsys@transformshift{1.307903in}{3.386254in}%
\pgfsys@useobject{currentmarker}{}%
\end{pgfscope}%
\begin{pgfscope}%
\pgfsys@transformshift{1.327205in}{3.418190in}%
\pgfsys@useobject{currentmarker}{}%
\end{pgfscope}%
\begin{pgfscope}%
\pgfsys@transformshift{1.346508in}{3.445419in}%
\pgfsys@useobject{currentmarker}{}%
\end{pgfscope}%
\begin{pgfscope}%
\pgfsys@transformshift{1.365810in}{3.467972in}%
\pgfsys@useobject{currentmarker}{}%
\end{pgfscope}%
\begin{pgfscope}%
\pgfsys@transformshift{1.385113in}{3.485874in}%
\pgfsys@useobject{currentmarker}{}%
\end{pgfscope}%
\begin{pgfscope}%
\pgfsys@transformshift{1.404416in}{3.499146in}%
\pgfsys@useobject{currentmarker}{}%
\end{pgfscope}%
\begin{pgfscope}%
\pgfsys@transformshift{1.423718in}{3.507804in}%
\pgfsys@useobject{currentmarker}{}%
\end{pgfscope}%
\begin{pgfscope}%
\pgfsys@transformshift{1.443021in}{3.511861in}%
\pgfsys@useobject{currentmarker}{}%
\end{pgfscope}%
\begin{pgfscope}%
\pgfsys@transformshift{1.462324in}{3.511328in}%
\pgfsys@useobject{currentmarker}{}%
\end{pgfscope}%
\begin{pgfscope}%
\pgfsys@transformshift{1.481626in}{3.506211in}%
\pgfsys@useobject{currentmarker}{}%
\end{pgfscope}%
\begin{pgfscope}%
\pgfsys@transformshift{1.500929in}{3.496513in}%
\pgfsys@useobject{currentmarker}{}%
\end{pgfscope}%
\begin{pgfscope}%
\pgfsys@transformshift{1.520231in}{3.482236in}%
\pgfsys@useobject{currentmarker}{}%
\end{pgfscope}%
\begin{pgfscope}%
\pgfsys@transformshift{1.539534in}{3.463377in}%
\pgfsys@useobject{currentmarker}{}%
\end{pgfscope}%
\begin{pgfscope}%
\pgfsys@transformshift{1.558837in}{3.439930in}%
\pgfsys@useobject{currentmarker}{}%
\end{pgfscope}%
\begin{pgfscope}%
\pgfsys@transformshift{1.578139in}{3.411883in}%
\pgfsys@useobject{currentmarker}{}%
\end{pgfscope}%
\begin{pgfscope}%
\pgfsys@transformshift{1.597442in}{3.379222in}%
\pgfsys@useobject{currentmarker}{}%
\end{pgfscope}%
\begin{pgfscope}%
\pgfsys@transformshift{1.616744in}{3.341923in}%
\pgfsys@useobject{currentmarker}{}%
\end{pgfscope}%
\begin{pgfscope}%
\pgfsys@transformshift{1.636047in}{3.299963in}%
\pgfsys@useobject{currentmarker}{}%
\end{pgfscope}%
\begin{pgfscope}%
\pgfsys@transformshift{1.655350in}{3.253306in}%
\pgfsys@useobject{currentmarker}{}%
\end{pgfscope}%
\begin{pgfscope}%
\pgfsys@transformshift{1.674652in}{3.201907in}%
\pgfsys@useobject{currentmarker}{}%
\end{pgfscope}%
\begin{pgfscope}%
\pgfsys@transformshift{1.693955in}{3.145706in}%
\pgfsys@useobject{currentmarker}{}%
\end{pgfscope}%
\begin{pgfscope}%
\pgfsys@transformshift{1.713258in}{3.084634in}%
\pgfsys@useobject{currentmarker}{}%
\end{pgfscope}%
\begin{pgfscope}%
\pgfsys@transformshift{1.732560in}{3.018608in}%
\pgfsys@useobject{currentmarker}{}%
\end{pgfscope}%
\begin{pgfscope}%
\pgfsys@transformshift{1.751863in}{2.947531in}%
\pgfsys@useobject{currentmarker}{}%
\end{pgfscope}%
\begin{pgfscope}%
\pgfsys@transformshift{1.771165in}{2.871298in}%
\pgfsys@useobject{currentmarker}{}%
\end{pgfscope}%
\begin{pgfscope}%
\pgfsys@transformshift{1.790468in}{2.789790in}%
\pgfsys@useobject{currentmarker}{}%
\end{pgfscope}%
\begin{pgfscope}%
\pgfsys@transformshift{1.809771in}{2.702881in}%
\pgfsys@useobject{currentmarker}{}%
\end{pgfscope}%
\begin{pgfscope}%
\pgfsys@transformshift{1.829073in}{2.610435in}%
\pgfsys@useobject{currentmarker}{}%
\end{pgfscope}%
\begin{pgfscope}%
\pgfsys@transformshift{1.848376in}{2.512341in}%
\pgfsys@useobject{currentmarker}{}%
\end{pgfscope}%
\begin{pgfscope}%
\pgfsys@transformshift{1.867678in}{2.408361in}%
\pgfsys@useobject{currentmarker}{}%
\end{pgfscope}%
\begin{pgfscope}%
\pgfsys@transformshift{1.886981in}{2.298355in}%
\pgfsys@useobject{currentmarker}{}%
\end{pgfscope}%
\begin{pgfscope}%
\pgfsys@transformshift{1.906284in}{2.182181in}%
\pgfsys@useobject{currentmarker}{}%
\end{pgfscope}%
\begin{pgfscope}%
\pgfsys@transformshift{1.925586in}{2.059699in}%
\pgfsys@useobject{currentmarker}{}%
\end{pgfscope}%
\begin{pgfscope}%
\pgfsys@transformshift{1.944889in}{1.930766in}%
\pgfsys@useobject{currentmarker}{}%
\end{pgfscope}%
\begin{pgfscope}%
\pgfsys@transformshift{1.964191in}{1.795242in}%
\pgfsys@useobject{currentmarker}{}%
\end{pgfscope}%
\begin{pgfscope}%
\pgfsys@transformshift{1.983494in}{1.652987in}%
\pgfsys@useobject{currentmarker}{}%
\end{pgfscope}%
\begin{pgfscope}%
\pgfsys@transformshift{2.002797in}{1.503858in}%
\pgfsys@useobject{currentmarker}{}%
\end{pgfscope}%
\begin{pgfscope}%
\pgfsys@transformshift{2.022099in}{1.347715in}%
\pgfsys@useobject{currentmarker}{}%
\end{pgfscope}%
\begin{pgfscope}%
\pgfsys@transformshift{2.041402in}{1.184417in}%
\pgfsys@useobject{currentmarker}{}%
\end{pgfscope}%
\begin{pgfscope}%
\pgfsys@transformshift{2.060705in}{1.013823in}%
\pgfsys@useobject{currentmarker}{}%
\end{pgfscope}%
\begin{pgfscope}%
\pgfsys@transformshift{2.195823in}{0.879984in}%
\pgfsys@useobject{currentmarker}{}%
\end{pgfscope}%
\begin{pgfscope}%
\pgfsys@transformshift{2.215125in}{1.028132in}%
\pgfsys@useobject{currentmarker}{}%
\end{pgfscope}%
\begin{pgfscope}%
\pgfsys@transformshift{2.234428in}{1.167890in}%
\pgfsys@useobject{currentmarker}{}%
\end{pgfscope}%
\begin{pgfscope}%
\pgfsys@transformshift{2.253731in}{1.299440in}%
\pgfsys@useobject{currentmarker}{}%
\end{pgfscope}%
\begin{pgfscope}%
\pgfsys@transformshift{2.273033in}{1.422965in}%
\pgfsys@useobject{currentmarker}{}%
\end{pgfscope}%
\begin{pgfscope}%
\pgfsys@transformshift{2.292336in}{1.538648in}%
\pgfsys@useobject{currentmarker}{}%
\end{pgfscope}%
\begin{pgfscope}%
\pgfsys@transformshift{2.311639in}{1.646671in}%
\pgfsys@useobject{currentmarker}{}%
\end{pgfscope}%
\begin{pgfscope}%
\pgfsys@transformshift{2.330941in}{1.747215in}%
\pgfsys@useobject{currentmarker}{}%
\end{pgfscope}%
\begin{pgfscope}%
\pgfsys@transformshift{2.350244in}{1.840465in}%
\pgfsys@useobject{currentmarker}{}%
\end{pgfscope}%
\begin{pgfscope}%
\pgfsys@transformshift{2.369546in}{1.926602in}%
\pgfsys@useobject{currentmarker}{}%
\end{pgfscope}%
\begin{pgfscope}%
\pgfsys@transformshift{2.388849in}{2.005810in}%
\pgfsys@useobject{currentmarker}{}%
\end{pgfscope}%
\begin{pgfscope}%
\pgfsys@transformshift{2.408152in}{2.078269in}%
\pgfsys@useobject{currentmarker}{}%
\end{pgfscope}%
\begin{pgfscope}%
\pgfsys@transformshift{2.427454in}{2.144164in}%
\pgfsys@useobject{currentmarker}{}%
\end{pgfscope}%
\begin{pgfscope}%
\pgfsys@transformshift{2.446757in}{2.203812in}%
\pgfsys@useobject{currentmarker}{}%
\end{pgfscope}%
\begin{pgfscope}%
\pgfsys@transformshift{2.466059in}{2.257352in}%
\pgfsys@useobject{currentmarker}{}%
\end{pgfscope}%
\begin{pgfscope}%
\pgfsys@transformshift{2.485362in}{2.304961in}%
\pgfsys@useobject{currentmarker}{}%
\end{pgfscope}%
\begin{pgfscope}%
\pgfsys@transformshift{2.504665in}{2.346801in}%
\pgfsys@useobject{currentmarker}{}%
\end{pgfscope}%
\begin{pgfscope}%
\pgfsys@transformshift{2.523967in}{2.383024in}%
\pgfsys@useobject{currentmarker}{}%
\end{pgfscope}%
\begin{pgfscope}%
\pgfsys@transformshift{2.543270in}{2.413766in}%
\pgfsys@useobject{currentmarker}{}%
\end{pgfscope}%
\begin{pgfscope}%
\pgfsys@transformshift{2.562573in}{2.439148in}%
\pgfsys@useobject{currentmarker}{}%
\end{pgfscope}%
\begin{pgfscope}%
\pgfsys@transformshift{2.581875in}{2.459274in}%
\pgfsys@useobject{currentmarker}{}%
\end{pgfscope}%
\begin{pgfscope}%
\pgfsys@transformshift{2.601178in}{2.474232in}%
\pgfsys@useobject{currentmarker}{}%
\end{pgfscope}%
\begin{pgfscope}%
\pgfsys@transformshift{2.620480in}{2.484090in}%
\pgfsys@useobject{currentmarker}{}%
\end{pgfscope}%
\begin{pgfscope}%
\pgfsys@transformshift{2.639783in}{2.488900in}%
\pgfsys@useobject{currentmarker}{}%
\end{pgfscope}%
\begin{pgfscope}%
\pgfsys@transformshift{2.659086in}{2.488696in}%
\pgfsys@useobject{currentmarker}{}%
\end{pgfscope}%
\begin{pgfscope}%
\pgfsys@transformshift{2.678388in}{2.483489in}%
\pgfsys@useobject{currentmarker}{}%
\end{pgfscope}%
\begin{pgfscope}%
\pgfsys@transformshift{2.697691in}{2.473295in}%
\pgfsys@useobject{currentmarker}{}%
\end{pgfscope}%
\begin{pgfscope}%
\pgfsys@transformshift{2.716993in}{2.458112in}%
\pgfsys@useobject{currentmarker}{}%
\end{pgfscope}%
\begin{pgfscope}%
\pgfsys@transformshift{2.736296in}{2.437919in}%
\pgfsys@useobject{currentmarker}{}%
\end{pgfscope}%
\begin{pgfscope}%
\pgfsys@transformshift{2.755599in}{2.412681in}%
\pgfsys@useobject{currentmarker}{}%
\end{pgfscope}%
\begin{pgfscope}%
\pgfsys@transformshift{2.774901in}{2.382350in}%
\pgfsys@useobject{currentmarker}{}%
\end{pgfscope}%
\begin{pgfscope}%
\pgfsys@transformshift{2.794204in}{2.346863in}%
\pgfsys@useobject{currentmarker}{}%
\end{pgfscope}%
\begin{pgfscope}%
\pgfsys@transformshift{2.813507in}{2.306142in}%
\pgfsys@useobject{currentmarker}{}%
\end{pgfscope}%
\begin{pgfscope}%
\pgfsys@transformshift{2.832809in}{2.260101in}%
\pgfsys@useobject{currentmarker}{}%
\end{pgfscope}%
\begin{pgfscope}%
\pgfsys@transformshift{2.852112in}{2.208641in}%
\pgfsys@useobject{currentmarker}{}%
\end{pgfscope}%
\begin{pgfscope}%
\pgfsys@transformshift{2.871414in}{2.151654in}%
\pgfsys@useobject{currentmarker}{}%
\end{pgfscope}%
\begin{pgfscope}%
\pgfsys@transformshift{2.890717in}{2.089023in}%
\pgfsys@useobject{currentmarker}{}%
\end{pgfscope}%
\begin{pgfscope}%
\pgfsys@transformshift{2.910020in}{2.020658in}%
\pgfsys@useobject{currentmarker}{}%
\end{pgfscope}%
\begin{pgfscope}%
\pgfsys@transformshift{2.929322in}{1.946332in}%
\pgfsys@useobject{currentmarker}{}%
\end{pgfscope}%
\begin{pgfscope}%
\pgfsys@transformshift{2.948625in}{1.865924in}%
\pgfsys@useobject{currentmarker}{}%
\end{pgfscope}%
\begin{pgfscope}%
\pgfsys@transformshift{2.967927in}{1.779309in}%
\pgfsys@useobject{currentmarker}{}%
\end{pgfscope}%
\begin{pgfscope}%
\pgfsys@transformshift{2.987230in}{1.686368in}%
\pgfsys@useobject{currentmarker}{}%
\end{pgfscope}%
\begin{pgfscope}%
\pgfsys@transformshift{3.006533in}{1.586976in}%
\pgfsys@useobject{currentmarker}{}%
\end{pgfscope}%
\begin{pgfscope}%
\pgfsys@transformshift{3.025835in}{1.481013in}%
\pgfsys@useobject{currentmarker}{}%
\end{pgfscope}%
\begin{pgfscope}%
\pgfsys@transformshift{3.045138in}{1.368356in}%
\pgfsys@useobject{currentmarker}{}%
\end{pgfscope}%
\begin{pgfscope}%
\pgfsys@transformshift{3.064441in}{1.248883in}%
\pgfsys@useobject{currentmarker}{}%
\end{pgfscope}%
\begin{pgfscope}%
\pgfsys@transformshift{3.083743in}{1.122472in}%
\pgfsys@useobject{currentmarker}{}%
\end{pgfscope}%
\begin{pgfscope}%
\pgfsys@transformshift{3.103046in}{0.989000in}%
\pgfsys@useobject{currentmarker}{}%
\end{pgfscope}%
\begin{pgfscope}%
\pgfsys@transformshift{3.122348in}{0.848346in}%
\pgfsys@useobject{currentmarker}{}%
\end{pgfscope}%
\begin{pgfscope}%
\pgfsys@transformshift{3.257467in}{0.946459in}%
\pgfsys@useobject{currentmarker}{}%
\end{pgfscope}%
\begin{pgfscope}%
\pgfsys@transformshift{3.276769in}{1.056379in}%
\pgfsys@useobject{currentmarker}{}%
\end{pgfscope}%
\begin{pgfscope}%
\pgfsys@transformshift{3.296072in}{1.158914in}%
\pgfsys@useobject{currentmarker}{}%
\end{pgfscope}%
\begin{pgfscope}%
\pgfsys@transformshift{3.315375in}{1.254063in}%
\pgfsys@useobject{currentmarker}{}%
\end{pgfscope}%
\begin{pgfscope}%
\pgfsys@transformshift{3.334677in}{1.341825in}%
\pgfsys@useobject{currentmarker}{}%
\end{pgfscope}%
\begin{pgfscope}%
\pgfsys@transformshift{3.353980in}{1.422199in}%
\pgfsys@useobject{currentmarker}{}%
\end{pgfscope}%
\begin{pgfscope}%
\pgfsys@transformshift{3.373282in}{1.495184in}%
\pgfsys@useobject{currentmarker}{}%
\end{pgfscope}%
\begin{pgfscope}%
\pgfsys@transformshift{3.392585in}{1.560780in}%
\pgfsys@useobject{currentmarker}{}%
\end{pgfscope}%
\begin{pgfscope}%
\pgfsys@transformshift{3.411888in}{1.618985in}%
\pgfsys@useobject{currentmarker}{}%
\end{pgfscope}%
\begin{pgfscope}%
\pgfsys@transformshift{3.431190in}{1.669799in}%
\pgfsys@useobject{currentmarker}{}%
\end{pgfscope}%
\begin{pgfscope}%
\pgfsys@transformshift{3.450493in}{1.713220in}%
\pgfsys@useobject{currentmarker}{}%
\end{pgfscope}%
\begin{pgfscope}%
\pgfsys@transformshift{3.469795in}{1.749248in}%
\pgfsys@useobject{currentmarker}{}%
\end{pgfscope}%
\begin{pgfscope}%
\pgfsys@transformshift{3.489098in}{1.777883in}%
\pgfsys@useobject{currentmarker}{}%
\end{pgfscope}%
\begin{pgfscope}%
\pgfsys@transformshift{3.508401in}{1.799339in}%
\pgfsys@useobject{currentmarker}{}%
\end{pgfscope}%
\begin{pgfscope}%
\pgfsys@transformshift{3.527703in}{1.813533in}%
\pgfsys@useobject{currentmarker}{}%
\end{pgfscope}%
\begin{pgfscope}%
\pgfsys@transformshift{3.547006in}{1.820438in}%
\pgfsys@useobject{currentmarker}{}%
\end{pgfscope}%
\begin{pgfscope}%
\pgfsys@transformshift{3.566309in}{1.820012in}%
\pgfsys@useobject{currentmarker}{}%
\end{pgfscope}%
\begin{pgfscope}%
\pgfsys@transformshift{3.585611in}{1.812191in}%
\pgfsys@useobject{currentmarker}{}%
\end{pgfscope}%
\begin{pgfscope}%
\pgfsys@transformshift{3.604914in}{1.796894in}%
\pgfsys@useobject{currentmarker}{}%
\end{pgfscope}%
\begin{pgfscope}%
\pgfsys@transformshift{3.624216in}{1.774019in}%
\pgfsys@useobject{currentmarker}{}%
\end{pgfscope}%
\begin{pgfscope}%
\pgfsys@transformshift{3.643519in}{1.743527in}%
\pgfsys@useobject{currentmarker}{}%
\end{pgfscope}%
\begin{pgfscope}%
\pgfsys@transformshift{3.662822in}{1.705075in}%
\pgfsys@useobject{currentmarker}{}%
\end{pgfscope}%
\begin{pgfscope}%
\pgfsys@transformshift{3.682124in}{1.658538in}%
\pgfsys@useobject{currentmarker}{}%
\end{pgfscope}%
\begin{pgfscope}%
\pgfsys@transformshift{3.701427in}{1.603792in}%
\pgfsys@useobject{currentmarker}{}%
\end{pgfscope}%
\begin{pgfscope}%
\pgfsys@transformshift{3.720729in}{1.540711in}%
\pgfsys@useobject{currentmarker}{}%
\end{pgfscope}%
\begin{pgfscope}%
\pgfsys@transformshift{3.740032in}{1.469171in}%
\pgfsys@useobject{currentmarker}{}%
\end{pgfscope}%
\begin{pgfscope}%
\pgfsys@transformshift{3.759335in}{1.389048in}%
\pgfsys@useobject{currentmarker}{}%
\end{pgfscope}%
\begin{pgfscope}%
\pgfsys@transformshift{3.778637in}{1.300215in}%
\pgfsys@useobject{currentmarker}{}%
\end{pgfscope}%
\begin{pgfscope}%
\pgfsys@transformshift{3.797940in}{1.202549in}%
\pgfsys@useobject{currentmarker}{}%
\end{pgfscope}%
\begin{pgfscope}%
\pgfsys@transformshift{3.817243in}{1.095924in}%
\pgfsys@useobject{currentmarker}{}%
\end{pgfscope}%
\begin{pgfscope}%
\pgfsys@transformshift{3.836545in}{0.980216in}%
\pgfsys@useobject{currentmarker}{}%
\end{pgfscope}%
\begin{pgfscope}%
\pgfsys@transformshift{3.855848in}{0.855300in}%
\pgfsys@useobject{currentmarker}{}%
\end{pgfscope}%
\begin{pgfscope}%
\pgfsys@transformshift{3.990966in}{0.957762in}%
\pgfsys@useobject{currentmarker}{}%
\end{pgfscope}%
\begin{pgfscope}%
\pgfsys@transformshift{4.010269in}{1.051401in}%
\pgfsys@useobject{currentmarker}{}%
\end{pgfscope}%
\begin{pgfscope}%
\pgfsys@transformshift{4.029571in}{1.135752in}%
\pgfsys@useobject{currentmarker}{}%
\end{pgfscope}%
\begin{pgfscope}%
\pgfsys@transformshift{4.048874in}{1.210768in}%
\pgfsys@useobject{currentmarker}{}%
\end{pgfscope}%
\begin{pgfscope}%
\pgfsys@transformshift{4.068177in}{1.276406in}%
\pgfsys@useobject{currentmarker}{}%
\end{pgfscope}%
\begin{pgfscope}%
\pgfsys@transformshift{4.087479in}{1.332619in}%
\pgfsys@useobject{currentmarker}{}%
\end{pgfscope}%
\begin{pgfscope}%
\pgfsys@transformshift{4.106782in}{1.379362in}%
\pgfsys@useobject{currentmarker}{}%
\end{pgfscope}%
\begin{pgfscope}%
\pgfsys@transformshift{4.126084in}{1.416591in}%
\pgfsys@useobject{currentmarker}{}%
\end{pgfscope}%
\begin{pgfscope}%
\pgfsys@transformshift{4.145387in}{1.444260in}%
\pgfsys@useobject{currentmarker}{}%
\end{pgfscope}%
\begin{pgfscope}%
\pgfsys@transformshift{4.164690in}{1.462323in}%
\pgfsys@useobject{currentmarker}{}%
\end{pgfscope}%
\begin{pgfscope}%
\pgfsys@transformshift{4.183992in}{1.470736in}%
\pgfsys@useobject{currentmarker}{}%
\end{pgfscope}%
\begin{pgfscope}%
\pgfsys@transformshift{4.203295in}{1.469453in}%
\pgfsys@useobject{currentmarker}{}%
\end{pgfscope}%
\begin{pgfscope}%
\pgfsys@transformshift{4.222597in}{1.458429in}%
\pgfsys@useobject{currentmarker}{}%
\end{pgfscope}%
\begin{pgfscope}%
\pgfsys@transformshift{4.241900in}{1.437618in}%
\pgfsys@useobject{currentmarker}{}%
\end{pgfscope}%
\begin{pgfscope}%
\pgfsys@transformshift{4.261203in}{1.406976in}%
\pgfsys@useobject{currentmarker}{}%
\end{pgfscope}%
\begin{pgfscope}%
\pgfsys@transformshift{4.280505in}{1.366457in}%
\pgfsys@useobject{currentmarker}{}%
\end{pgfscope}%
\begin{pgfscope}%
\pgfsys@transformshift{4.299808in}{1.316016in}%
\pgfsys@useobject{currentmarker}{}%
\end{pgfscope}%
\begin{pgfscope}%
\pgfsys@transformshift{4.319111in}{1.255608in}%
\pgfsys@useobject{currentmarker}{}%
\end{pgfscope}%
\begin{pgfscope}%
\pgfsys@transformshift{4.338413in}{1.185187in}%
\pgfsys@useobject{currentmarker}{}%
\end{pgfscope}%
\begin{pgfscope}%
\pgfsys@transformshift{4.357716in}{1.104708in}%
\pgfsys@useobject{currentmarker}{}%
\end{pgfscope}%
\begin{pgfscope}%
\pgfsys@transformshift{4.377018in}{1.014126in}%
\pgfsys@useobject{currentmarker}{}%
\end{pgfscope}%
\begin{pgfscope}%
\pgfsys@transformshift{4.396321in}{0.913396in}%
\pgfsys@useobject{currentmarker}{}%
\end{pgfscope}%
\begin{pgfscope}%
\pgfsys@transformshift{4.415624in}{0.802472in}%
\pgfsys@useobject{currentmarker}{}%
\end{pgfscope}%
\begin{pgfscope}%
\pgfsys@transformshift{4.550742in}{0.901130in}%
\pgfsys@useobject{currentmarker}{}%
\end{pgfscope}%
\begin{pgfscope}%
\pgfsys@transformshift{4.570045in}{0.964338in}%
\pgfsys@useobject{currentmarker}{}%
\end{pgfscope}%
\begin{pgfscope}%
\pgfsys@transformshift{4.589347in}{1.015579in}%
\pgfsys@useobject{currentmarker}{}%
\end{pgfscope}%
\begin{pgfscope}%
\pgfsys@transformshift{4.608650in}{1.054643in}%
\pgfsys@useobject{currentmarker}{}%
\end{pgfscope}%
\begin{pgfscope}%
\pgfsys@transformshift{4.627952in}{1.081321in}%
\pgfsys@useobject{currentmarker}{}%
\end{pgfscope}%
\begin{pgfscope}%
\pgfsys@transformshift{4.647255in}{1.095404in}%
\pgfsys@useobject{currentmarker}{}%
\end{pgfscope}%
\begin{pgfscope}%
\pgfsys@transformshift{4.666558in}{1.096681in}%
\pgfsys@useobject{currentmarker}{}%
\end{pgfscope}%
\begin{pgfscope}%
\pgfsys@transformshift{4.685860in}{1.084943in}%
\pgfsys@useobject{currentmarker}{}%
\end{pgfscope}%
\begin{pgfscope}%
\pgfsys@transformshift{4.705163in}{1.059980in}%
\pgfsys@useobject{currentmarker}{}%
\end{pgfscope}%
\begin{pgfscope}%
\pgfsys@transformshift{4.724465in}{1.021584in}%
\pgfsys@useobject{currentmarker}{}%
\end{pgfscope}%
\begin{pgfscope}%
\pgfsys@transformshift{4.743768in}{0.969543in}%
\pgfsys@useobject{currentmarker}{}%
\end{pgfscope}%
\begin{pgfscope}%
\pgfsys@transformshift{4.763071in}{0.903650in}%
\pgfsys@useobject{currentmarker}{}%
\end{pgfscope}%
\begin{pgfscope}%
\pgfsys@transformshift{4.782373in}{0.823693in}%
\pgfsys@useobject{currentmarker}{}%
\end{pgfscope}%
\begin{pgfscope}%
\pgfsys@transformshift{4.898189in}{0.766407in}%
\pgfsys@useobject{currentmarker}{}%
\end{pgfscope}%
\begin{pgfscope}%
\pgfsys@transformshift{4.917492in}{0.846090in}%
\pgfsys@useobject{currentmarker}{}%
\end{pgfscope}%
\begin{pgfscope}%
\pgfsys@transformshift{4.936794in}{0.912072in}%
\pgfsys@useobject{currentmarker}{}%
\end{pgfscope}%
\begin{pgfscope}%
\pgfsys@transformshift{4.956097in}{0.964212in}%
\pgfsys@useobject{currentmarker}{}%
\end{pgfscope}%
\begin{pgfscope}%
\pgfsys@transformshift{4.975399in}{1.002372in}%
\pgfsys@useobject{currentmarker}{}%
\end{pgfscope}%
\begin{pgfscope}%
\pgfsys@transformshift{4.994702in}{1.026412in}%
\pgfsys@useobject{currentmarker}{}%
\end{pgfscope}%
\begin{pgfscope}%
\pgfsys@transformshift{5.014005in}{1.036192in}%
\pgfsys@useobject{currentmarker}{}%
\end{pgfscope}%
\begin{pgfscope}%
\pgfsys@transformshift{5.033307in}{1.031574in}%
\pgfsys@useobject{currentmarker}{}%
\end{pgfscope}%
\begin{pgfscope}%
\pgfsys@transformshift{5.052610in}{1.012418in}%
\pgfsys@useobject{currentmarker}{}%
\end{pgfscope}%
\begin{pgfscope}%
\pgfsys@transformshift{5.071913in}{0.978583in}%
\pgfsys@useobject{currentmarker}{}%
\end{pgfscope}%
\begin{pgfscope}%
\pgfsys@transformshift{5.091215in}{0.929932in}%
\pgfsys@useobject{currentmarker}{}%
\end{pgfscope}%
\begin{pgfscope}%
\pgfsys@transformshift{5.110518in}{0.866325in}%
\pgfsys@useobject{currentmarker}{}%
\end{pgfscope}%
\begin{pgfscope}%
\pgfsys@transformshift{5.129820in}{0.787622in}%
\pgfsys@useobject{currentmarker}{}%
\end{pgfscope}%
\end{pgfscope}%
\begin{pgfscope}%
\pgfpathrectangle{\pgfqpoint{0.691184in}{0.629134in}}{\pgfqpoint{4.650000in}{3.020000in}}%
\pgfusepath{clip}%
\pgfsetbuttcap%
\pgfsetroundjoin%
\definecolor{currentfill}{rgb}{0.121569,0.466667,0.705882}%
\pgfsetfillcolor{currentfill}%
\pgfsetlinewidth{1.003750pt}%
\definecolor{currentstroke}{rgb}{0.121569,0.466667,0.705882}%
\pgfsetstrokecolor{currentstroke}%
\pgfsetdash{}{0pt}%
\pgfsys@defobject{currentmarker}{\pgfqpoint{-0.041667in}{-0.041667in}}{\pgfqpoint{0.041667in}{0.041667in}}{%
\pgfpathmoveto{\pgfqpoint{0.000000in}{-0.041667in}}%
\pgfpathcurveto{\pgfqpoint{0.011050in}{-0.041667in}}{\pgfqpoint{0.021649in}{-0.037276in}}{\pgfqpoint{0.029463in}{-0.029463in}}%
\pgfpathcurveto{\pgfqpoint{0.037276in}{-0.021649in}}{\pgfqpoint{0.041667in}{-0.011050in}}{\pgfqpoint{0.041667in}{0.000000in}}%
\pgfpathcurveto{\pgfqpoint{0.041667in}{0.011050in}}{\pgfqpoint{0.037276in}{0.021649in}}{\pgfqpoint{0.029463in}{0.029463in}}%
\pgfpathcurveto{\pgfqpoint{0.021649in}{0.037276in}}{\pgfqpoint{0.011050in}{0.041667in}}{\pgfqpoint{0.000000in}{0.041667in}}%
\pgfpathcurveto{\pgfqpoint{-0.011050in}{0.041667in}}{\pgfqpoint{-0.021649in}{0.037276in}}{\pgfqpoint{-0.029463in}{0.029463in}}%
\pgfpathcurveto{\pgfqpoint{-0.037276in}{0.021649in}}{\pgfqpoint{-0.041667in}{0.011050in}}{\pgfqpoint{-0.041667in}{0.000000in}}%
\pgfpathcurveto{\pgfqpoint{-0.041667in}{-0.011050in}}{\pgfqpoint{-0.037276in}{-0.021649in}}{\pgfqpoint{-0.029463in}{-0.029463in}}%
\pgfpathcurveto{\pgfqpoint{-0.021649in}{-0.037276in}}{\pgfqpoint{-0.011050in}{-0.041667in}}{\pgfqpoint{0.000000in}{-0.041667in}}%
\pgfpathclose%
\pgfusepath{stroke,fill}%
}%
\begin{pgfscope}%
\pgfsys@transformshift{0.902548in}{1.544403in}%
\pgfsys@useobject{currentmarker}{}%
\end{pgfscope}%
\begin{pgfscope}%
\pgfsys@transformshift{2.060705in}{1.013823in}%
\pgfsys@useobject{currentmarker}{}%
\end{pgfscope}%
\begin{pgfscope}%
\pgfsys@transformshift{2.195823in}{0.879984in}%
\pgfsys@useobject{currentmarker}{}%
\end{pgfscope}%
\begin{pgfscope}%
\pgfsys@transformshift{3.122348in}{0.848346in}%
\pgfsys@useobject{currentmarker}{}%
\end{pgfscope}%
\begin{pgfscope}%
\pgfsys@transformshift{3.257467in}{0.946459in}%
\pgfsys@useobject{currentmarker}{}%
\end{pgfscope}%
\begin{pgfscope}%
\pgfsys@transformshift{3.855848in}{0.855300in}%
\pgfsys@useobject{currentmarker}{}%
\end{pgfscope}%
\begin{pgfscope}%
\pgfsys@transformshift{3.990966in}{0.957762in}%
\pgfsys@useobject{currentmarker}{}%
\end{pgfscope}%
\begin{pgfscope}%
\pgfsys@transformshift{4.415624in}{0.802472in}%
\pgfsys@useobject{currentmarker}{}%
\end{pgfscope}%
\begin{pgfscope}%
\pgfsys@transformshift{4.550742in}{0.901130in}%
\pgfsys@useobject{currentmarker}{}%
\end{pgfscope}%
\begin{pgfscope}%
\pgfsys@transformshift{4.782373in}{0.823693in}%
\pgfsys@useobject{currentmarker}{}%
\end{pgfscope}%
\begin{pgfscope}%
\pgfsys@transformshift{4.898189in}{0.766407in}%
\pgfsys@useobject{currentmarker}{}%
\end{pgfscope}%
\begin{pgfscope}%
\pgfsys@transformshift{5.129820in}{0.787622in}%
\pgfsys@useobject{currentmarker}{}%
\end{pgfscope}%
\end{pgfscope}%
\begin{pgfscope}%
\pgfsetrectcap%
\pgfsetmiterjoin%
\pgfsetlinewidth{0.803000pt}%
\definecolor{currentstroke}{rgb}{0.501961,0.501961,0.501961}%
\pgfsetstrokecolor{currentstroke}%
\pgfsetdash{}{0pt}%
\pgfpathmoveto{\pgfqpoint{0.691184in}{0.629134in}}%
\pgfpathlineto{\pgfqpoint{0.691184in}{3.649134in}}%
\pgfusepath{stroke}%
\end{pgfscope}%
\begin{pgfscope}%
\pgfsetrectcap%
\pgfsetmiterjoin%
\pgfsetlinewidth{0.803000pt}%
\definecolor{currentstroke}{rgb}{0.501961,0.501961,0.501961}%
\pgfsetstrokecolor{currentstroke}%
\pgfsetdash{}{0pt}%
\pgfpathmoveto{\pgfqpoint{5.341184in}{0.629134in}}%
\pgfpathlineto{\pgfqpoint{5.341184in}{3.649134in}}%
\pgfusepath{stroke}%
\end{pgfscope}%
\begin{pgfscope}%
\pgfsetrectcap%
\pgfsetmiterjoin%
\pgfsetlinewidth{0.803000pt}%
\definecolor{currentstroke}{rgb}{0.501961,0.501961,0.501961}%
\pgfsetstrokecolor{currentstroke}%
\pgfsetdash{}{0pt}%
\pgfpathmoveto{\pgfqpoint{0.691184in}{0.629134in}}%
\pgfpathlineto{\pgfqpoint{5.341184in}{0.629134in}}%
\pgfusepath{stroke}%
\end{pgfscope}%
\begin{pgfscope}%
\pgfsetrectcap%
\pgfsetmiterjoin%
\pgfsetlinewidth{0.803000pt}%
\definecolor{currentstroke}{rgb}{0.501961,0.501961,0.501961}%
\pgfsetstrokecolor{currentstroke}%
\pgfsetdash{}{0pt}%
\pgfpathmoveto{\pgfqpoint{0.691184in}{3.649134in}}%
\pgfpathlineto{\pgfqpoint{5.341184in}{3.649134in}}%
\pgfusepath{stroke}%
\end{pgfscope}%
\end{pgfpicture}%
\makeatother%
\endgroup%
}
\caption{The trajectory time-series of a 0.05 mL drop with surface potential $\varphi_s = 0.75 \pm 0.5$ kV. \label{fig:bounce_time}}
\end{figure}

We begin with some preliminary observations of the phenomenon:
\begin{itemize}
\item Observed maximum drop (de-)accelerations are on the order of $\sim$30 cm/s$^2$ for a range of drop volumes $0.03 \lessapprox V_d \lessapprox 0.5$ mL.
\item The water drops are attracted to regions of high electric field. The horizontal (surface plane parallel) components of the drop trajectory usually oscillate about some central position during the experiment (except in cases of nearly pure 1-D vertical translation). For especially small drops close to the spontaneous drop jump limit (identified by Attari \emph{et al.} \citep{attari_puddle_2016} as $V_d \sim$ 0.01 mL) the drops do not jump, but translate across the surface in a rolling regime until either they reach a local maximum of the electric field, or until their motion is sufficiently damped by contact line hysteresis that pinning arrests their motion.
\item The drops appear to have net free charge. In cases of multiple simultaneous drop jumps the drops repel each other as they bounce or roll in orbital motion around regions of high field.
\item The magnitude of the drop trajectory maxima (apoapses) are related to the drop volume (mass), and initial jump velocity (inertia), and to the electric field strength.
\end{itemize}

\sout{We expect that if the drop electrostatic potential energy is greater than the release of surface energy under the sudden change in $\mathbb{B}\mbox{o}$ in the form of the drop kinetic energy, the drop will return to an equilibrium state on the charged surface which minimizes the potential.}
What is the origin of the extemporaneous electric field at work in these observations? It is well known that water acquires positive free charge when in contact with certain polymers, especially polytetrafluoroethylene (PTFE), through a process called contact charging \cite{langmuir_surface_1938}. PTFE, on the other hand, tends to readily acquire negative charge by contact with water. The superhydrophobic surfaces used in the spontaneous drop jump experiments have thin (nanometric) PTFE coatings, and we observe that it is extremely easy to produce significant surface potentials $\varphi_s \sim$ 100-500 V by simply flowing streams of distilled water over them. A study of this water on PTFE contact charging phenomenon was conducted by Yatsuzuka \emph{et al.} \cite{yatsuzuka_electrification_1994}, who suggest that this process results from formation of an electrical double layer driven by selective adsorption of ($\mbox{OH}^-$) ions at the polymer surface; other recent work supports this hypothesis \cite{beattie_intrinsic_2006, strazdaite_water_2015}. Given the large roughness, or the ratio of projected to actual surface area, of the superhydrophobic surfaces used in the experiment, and given that the drops are initially in a Cassie-Baxter state, a somewhat electrically resistive air layer is maintained that reduces grounding of the drops despite the large potential difference between them and the surface charges.

The source of the net free charge on the drops is another issue. The drop charge could be due to the contact charging mechanism mentioned previously. For instance, in a 1996 paper, NASA flight engineer Don Pettit discusses the problem of low-gravity flow induced charging of liquids, resulting ultimately from contact charging phenomenon \cite{pettit_donald_flow_????}. However, a more likely mechanism for the drop charge is field-induced charging occuring due to breakup of a conductor having a field-induced dipole (e.g., a physical separation of charge). In our work this might occur when a drop is deposited on the charged surface by a grounded syringe. The metal syringe needle tip and the liquid in the syringe itself are essentially a ground connection which is broken when the syringe is suddenly removed in the presence of an external electric field. Field-induced charging is at work in the famous Kelvin thunderstorm and is applied in inkjet and electrospay technologies, where in each case the breakup is by the Rayleigh-Plateau instability. Notably, in Pettit's aforementioned discussion of contact charging of liquids in low-gravity, he remarks on accidental electrostatic `hula-ing' of silicone oil drops when ejected from a syringe in the vicinity of a highly-charged polymer surface during an experiment conducted aboard STS-5 by Space Shuttle mission specialist Joseph Allen \cite{pettit_donald_flow_????}. \sout{Depending on the (highly-variable) electrical conductivity of the silicone oil, and the material of the syringe used in the experiment, the charge could arise just as easily by field-induction as by contact charging.} Relatedly, in a series of informal and somewhat whimsical experiments Pettit himself electrostatically orbited small water drops around a triboelectrically charged PTFE knitting needle while aboard ISS during expedition 30/31 \cite{stevenson_electrostatic_2015}. Again, in this case, the drop charge is likely field-induced.

This electro-drop bounce phenomenon may have useful applications in a space environment, generally to tackle various manifestations of the so-called ``phase separation'' problem of separating a gas phase from a multi-phase flow (or the reverse case), in low-gravity. Removal of satellite droplets produced during pipetting in wet-lab research outside of a glovebox environment aboard ISS has been recently suggested as a application of the work [private communication, NASA JSC, E. Unger, 2017]. Drops can become spontaneously charged by contact with standard micropipette tips \citep{choi_spontaneous_2013}, and this free charge can possibly be leveraged for the purposes of phase separation in low-gravity.

\section{Theory}
\subsection{Equation of Motion}
We develop here a simple 1-dimensional model of the dynamics of drops dominated by electrostatic forces. We treat a drop as a particle with radius $R_d$, which translates vertically along the central axis of a charged dielectric square sheet substrate. The equation of motion for this system is given by,
\begin{equation}
m y'' = - F_D - F_E,
\label{gov_eqn}
\end{equation}
subject to
\begin{equation}
y(0) = R_d, \hspace{5 mm} \mbox{and} \hspace{5 mm} y'(0) = U_0,
\end{equation}
where $m$ is the drop mass, $y'' = \frac{d^2 y}{d t^2}$ is the drop acceleration, $F_D$ is the drag force which always opposes motion, and $F_E$ is the electrostatic force. The assumed initial conditions are such that, when $\mathbb{B}\mbox{o}$ is suddenly reduced at the start of the drop tower free-fall period, the drop jumps with instantaneous initial velocity $U_0$ from its 1-$g_0$ resting position of $R_d$ at $t=0$. The dynamical system is sketched schematically in Figure \ref{fig:apparatus}. We now define models for each of the forces in this equation.

\begin{figure}[ht]
\centering
\includegraphics[width=0.45\textwidth]{../figures/apparatus3.pdf}
\caption{Schematic representation of drop jump with return and rebound from an electrically charged superhydrophobic substrate. The characteristic time and length scales $t_c$ and $y_c$ describe the time of flight and apoapse associated with the drop trajectory. $L$ denotes the length scale characteristic of the charged superhydrophobic substrate.}
\label{fig:apparatus}
\end{figure}

For the intermediate range of Reynolds numbers $1 \leq \mathbb{R}\mbox{e} \equiv \frac{2UR_d}{\nu} \leq 50 $ observed in our experiments, we assume the force of drag acting on the drop to be quadratic, with $\nu$ being the kinematic viscosity.

In modeling the electrostatic force we begin with the standard electrohydrodynamic (EHD) approximation \cite{saville_electrohydrodynamics:_1997}. \sout{Under a DC electric field, we assume that the real part of the dielectric permittivity $\epsilon$, $\mbox{Re} \left( \epsilon \right) \approx  \mbox{constant}$. We also assume that electric currents are small enough that the effects of magnetic fields can be neglected. The validity of this assumption rests on the characteristic time scale $\tau_e = \epsilon /\sigma_e \ll 1$, where $\tau_e$ is the ratio of absolute dielectric permittivity $\epsilon = \kappa \epsilon_0$, to conductivity $\sigma_e$, of the medium, $\kappa$ is the relative dielectric permittivity, and $\epsilon_0$ is the vacuum permittivity. This characteristic time $\tau_e$ is also known as the relaxation time, and is a measure of how quickly the polarization of a dielectric responds to a change in electric field. Given the conductivity and permittivity in the limiting case of extremely-pure water ($ \epsilon \approx 80$, $\sigma_e = 18.2 \times 10^{6}$ $\Omega\mbox{cm}$) \cite{yatsuzuka_electrification_1994}, we estimate $\tau_e \approx 4 \times 10^{-6}$ s. The relaxation time for the common distilled water that is actually used in the drop experiments is shorter due to the presence of solvated ions which tend to increase the conductivity. Neglecting the effects of an electric double layer on hydration of ions in the water or the ambient atmosphere due to the relatively massive size of the drops studied, the assumption of small relaxation time further implies that the free charge present in the drops will remain approximately constant during the typical time interval of a low-gravity experiment.}

Supposing that electrical forces acting on free charges and dipoles in a fluid are transferred directly to the fluid itself, the overall electrical body force density will be the divergence of the Maxwell stress tensor $\tau_m $,
\begin{eqnarray} \label{e_force}
 \mathbf{F}_E &=& \nabla \cdot \tau_m \nonumber \\ 
 &=& \rho_f \mathbf{E} + \frac{1}{2} \left| E \right|^2 \nabla \epsilon - \nabla \left( \frac{1}{2} \rho \left( \frac{\partial \epsilon}{\partial \rho} \right)_T \left| E \right|^2 \right) .
\end{eqnarray}
The first term on the right hand side of this expression is the well known Coulombic force or electrophoretic force, which arises from the presence of free charge in an external electric field. We expect this term to dominate the electric force in a DC field. The second term is the force arising from polarization stresses due to a nonuniform field acting across a gradient in permittivity. This force is widely termed the dielectrophoretic (DEP) force. The third term describes forces due to electrostriction. It has been noted by Melcher and Hurwitz \cite{hurwitz_electrohydrodynamic_1966} that the electrostriction term is the gradient of a scalar and can thus be canonically lumped together with the hydrostatic pressure for incompressible fluids. We neglect it in our analysis. 

It is common to approximate the dielectrophoretic force by idealizing the drop as a simple dipole using the effective dipole moment method first described by Pohl in 1958 \cite{pohl_effects_1958}. The force felt by the dipole is 
\begin{eqnarray} \label{dep_force}
\mathbf{F}_{DEP} &=& \left( \mathbf{P}_e \cdot \nabla \right) \mathbf{E} \nonumber \\
&=& 2 \pi R_d^3 \kappa_w \epsilon_0 K \nabla E^2,
\end{eqnarray}
where $\mathbf{P}_e=(\kappa_w - \kappa_a)\epsilon_0 \mathbf{E}$ is the excess polarization, $\epsilon_0$ is the vacuum permittivity, and $\kappa_w$ and $\kappa_a$ are the relative dielectric constants of the water particle and air host fluid respectively. Here it is convenient to use the simplifying shorthand $K = \frac{\kappa_w - \kappa_a}{\kappa_w + 2 \kappa_a}$, known as the Clausius-Mossotti factor. In cases where $K <$ 0, or $K>$ 0 the particle will be repelled or attracted to regions of strong field respectively. \sout{In our experiment, choosing the relative dielectric constants $\kappa_a \approx$ 1 and $\kappa_w \approx$ 80, we estimate $K \approx$ 0.96. It is important to note that the equivalent dipole approximation critically requires an assumption of small physical scale of the particle relative to the length scale of nonuniformity of the field, which in this case we take to be the length of the charged superhydrophobic surface, $L =$ 25 mm $\gg R_d \approx$ 2 mm.} By comparing DEP and Coulombic forces we note that a condition to neglect the DEP force is
\begin{eqnarray}
\frac{ \kappa_w \epsilon_0 K R_d^2 E_0}{q} \ll 1. \nonumber
\end{eqnarray}
This conditions prevails in our experiments and we henceforth neglect the DEP force. \sout{There is some physical intuition to support this conclusion as well. The dielectric displacement $\mathbf{D} = \kappa \epsilon_0 \mathbf{E}$ of a water drop in air is high due to the large relative dielectric constant of water. This implies that the field strength within the drop is about 80 times smaller than in the surrounding medium. Thus it is not particularly inaccurate to model the dielectric volume of a drop as an equipotential conductive shell with zero field in its interior. As an aside, in treating the drop as an ideal conductor we note that in our specific case the electrostatic force is not a body force \emph{per se} as the electric field is acting on charges on the surface of the conductor.}

When the drop is close to the dielectric surface, the free charge on the drop will tend to induce polarization of the dielectric which perturbs the electric field. The polarization bound charge in the dielectric will be of the opposite sign of the free drop charge and thus there will be a force of attraction. This so-called image force is a correction to the Coulomb force due to the external electric field only, and can be found by a Green's Function solution of Laplace's equation for the electric field, the so-called the `method of images' \cite{david_j._griffiths_introduction_1999}. This resulting image force $\mathbf{F}_I$ is given by
\begin{equation}
\mathbf{F}_I = \frac{k q^2}{16 \pi \epsilon_0} y^{-2} \hat{\mathbf{j}},
\label{image_force}
\end{equation}
where the factor $k$ is a function of the dielectric surface susceptibility $k = \frac{\chi_e}{\chi_e + 2}$, $\chi_e = \kappa_d - 1$, $\kappa_d$ is the relative dielectric constant of the dielectric substrate, and $\hat{\mathbf{j}}$ is a unit vector normal to the dielectric surface.

By substituting Equation \ref{image_force} into \ref{e_force} we have for a single drop
\begin{eqnarray}
 \mathbf{F}_E &=& q \mathbf{E} + \mathbf{F}_I \nonumber \\
 &=& q \mathbf{E} + \frac{k q^2}{16 \pi \epsilon_0 } y^{-2} \hat{\mathbf{j}} \label{e_forces}
\end{eqnarray}

Thus the 1-D governing Equation \ref{e_forces}, becomes
\begin{equation}
 \label{gov_eqn_subs}
m y'' = - \frac{1}{2} C_D \rho A {y'}^2 - q E - \frac{k q^2}{16 \pi \epsilon_0} y^{-2},
\end{equation}
subject to
\begin{equation}
y(0) = R_d, \hspace{5 mm} \mbox{and} \hspace{5 mm} y'(0) = U_0 .
\end{equation}

\subsection{Electric Field}
If we consider the charged dielectric surface of our experiments to be a square sheet of charge lying in the $xz$-plane with width $L$, the symmetry of the problem happily lets us obtain the $y$-component of the electric field $\mathbf{E}$ by direct integration (as a superposition of line charges) \citep{david_j._griffiths_introduction_1999} as
\begin{equation}
\label{e_field}
E = \frac{\sigma}{ \pi \epsilon_0} \tan^{-1} \left( \frac{L^2}{y \sqrt{2L^2 + 4y^2}}\right)
,\end{equation}
where $\sigma$ is the surface charge density. We note that this 1D model of the electric field is valid when $R_d \ll L$, which will be true for cases of small drops `far' from the dielectric surface. 

By taking Taylor series expansions in large and small $y$-limits we can intuit a bit about the behavior of this field. In the limit $y/L \ll 1$ Equation \ref{e_field} reduces to
\begin{equation}
\label{near_field}
E \approx \frac{\sigma}{4 \pi \epsilon_0} = E_0,
\end{equation}
where $E_0$ is the characteristic electric field. This field is constant and equivalent to the electric field due to an infinite plane of charge. In the limit of $y/L \gg 1$, Equation \ref{e_field} reduces to the familiar electric field due to a point charge
\begin{equation}
\label{far_field}
E \approx L^2 E_0 y^{-2}.
\end{equation}
Both regimes given by Equations \ref{near_field} and \ref{far_field} are identified in Figure \ref{fig:E0}.
\begin{figure}[h]
    \centering
    \def\svgwidth{\columnwidth}
    %% Creator: Matplotlib, PGF backend
%%
%% To include the figure in your LaTeX document, write
%%   \input{<filename>.pgf}
%%
%% Make sure the required packages are loaded in your preamble
%%   \usepackage{pgf}
%%
%% Figures using additional raster images can only be included by \input if
%% they are in the same directory as the main LaTeX file. For loading figures
%% from other directories you can use the `import` package
%%   \usepackage{import}
%% and then include the figures with
%%   \import{<path to file>}{<filename>.pgf}
%%
%% Matplotlib used the following preamble
%%   \usepackage{fontspec}
%%   \setmainfont{DejaVu Serif}
%%   \setsansfont{DejaVu Sans}
%%   \setmonofont{DejaVu Sans Mono}
%%
\begingroup%
\makeatletter%
\begin{pgfpicture}%
\pgfpathrectangle{\pgfpointorigin}{\pgfqpoint{5.464669in}{3.681079in}}%
\pgfusepath{use as bounding box, clip}%
\begin{pgfscope}%
\pgfsetbuttcap%
\pgfsetmiterjoin%
\definecolor{currentfill}{rgb}{1.000000,1.000000,1.000000}%
\pgfsetfillcolor{currentfill}%
\pgfsetlinewidth{0.000000pt}%
\definecolor{currentstroke}{rgb}{1.000000,1.000000,1.000000}%
\pgfsetstrokecolor{currentstroke}%
\pgfsetdash{}{0pt}%
\pgfpathmoveto{\pgfqpoint{0.000000in}{0.000000in}}%
\pgfpathlineto{\pgfqpoint{5.464669in}{0.000000in}}%
\pgfpathlineto{\pgfqpoint{5.464669in}{3.681079in}}%
\pgfpathlineto{\pgfqpoint{0.000000in}{3.681079in}}%
\pgfpathclose%
\pgfusepath{fill}%
\end{pgfscope}%
\begin{pgfscope}%
\pgfsetbuttcap%
\pgfsetmiterjoin%
\definecolor{currentfill}{rgb}{1.000000,1.000000,1.000000}%
\pgfsetfillcolor{currentfill}%
\pgfsetlinewidth{0.000000pt}%
\definecolor{currentstroke}{rgb}{0.000000,0.000000,0.000000}%
\pgfsetstrokecolor{currentstroke}%
\pgfsetstrokeopacity{0.000000}%
\pgfsetdash{}{0pt}%
\pgfpathmoveto{\pgfqpoint{0.679669in}{0.526079in}}%
\pgfpathlineto{\pgfqpoint{5.329669in}{0.526079in}}%
\pgfpathlineto{\pgfqpoint{5.329669in}{3.546079in}}%
\pgfpathlineto{\pgfqpoint{0.679669in}{3.546079in}}%
\pgfpathclose%
\pgfusepath{fill}%
\end{pgfscope}%
\begin{pgfscope}%
\pgfsetbuttcap%
\pgfsetroundjoin%
\definecolor{currentfill}{rgb}{0.000000,0.000000,0.000000}%
\pgfsetfillcolor{currentfill}%
\pgfsetlinewidth{0.803000pt}%
\definecolor{currentstroke}{rgb}{0.000000,0.000000,0.000000}%
\pgfsetstrokecolor{currentstroke}%
\pgfsetdash{}{0pt}%
\pgfsys@defobject{currentmarker}{\pgfqpoint{0.000000in}{-0.048611in}}{\pgfqpoint{0.000000in}{0.000000in}}{%
\pgfpathmoveto{\pgfqpoint{0.000000in}{0.000000in}}%
\pgfpathlineto{\pgfqpoint{0.000000in}{-0.048611in}}%
\pgfusepath{stroke,fill}%
}%
\begin{pgfscope}%
\pgfsys@transformshift{2.211681in}{0.526079in}%
\pgfsys@useobject{currentmarker}{}%
\end{pgfscope}%
\end{pgfscope}%
\begin{pgfscope}%
\pgftext[x=2.211681in,y=0.428857in,,top]{\rmfamily\fontsize{10.000000}{12.000000}\selectfont \(\displaystyle 10^{-1}\)}%
\end{pgfscope}%
\begin{pgfscope}%
\pgfsetbuttcap%
\pgfsetroundjoin%
\definecolor{currentfill}{rgb}{0.000000,0.000000,0.000000}%
\pgfsetfillcolor{currentfill}%
\pgfsetlinewidth{0.803000pt}%
\definecolor{currentstroke}{rgb}{0.000000,0.000000,0.000000}%
\pgfsetstrokecolor{currentstroke}%
\pgfsetdash{}{0pt}%
\pgfsys@defobject{currentmarker}{\pgfqpoint{0.000000in}{-0.048611in}}{\pgfqpoint{0.000000in}{0.000000in}}{%
\pgfpathmoveto{\pgfqpoint{0.000000in}{0.000000in}}%
\pgfpathlineto{\pgfqpoint{0.000000in}{-0.048611in}}%
\pgfusepath{stroke,fill}%
}%
\begin{pgfscope}%
\pgfsys@transformshift{3.836268in}{0.526079in}%
\pgfsys@useobject{currentmarker}{}%
\end{pgfscope}%
\end{pgfscope}%
\begin{pgfscope}%
\pgftext[x=3.836268in,y=0.428857in,,top]{\rmfamily\fontsize{10.000000}{12.000000}\selectfont \(\displaystyle 10^{0}\)}%
\end{pgfscope}%
\begin{pgfscope}%
\pgfsetbuttcap%
\pgfsetroundjoin%
\definecolor{currentfill}{rgb}{0.000000,0.000000,0.000000}%
\pgfsetfillcolor{currentfill}%
\pgfsetlinewidth{0.602250pt}%
\definecolor{currentstroke}{rgb}{0.000000,0.000000,0.000000}%
\pgfsetstrokecolor{currentstroke}%
\pgfsetdash{}{0pt}%
\pgfsys@defobject{currentmarker}{\pgfqpoint{0.000000in}{-0.027778in}}{\pgfqpoint{0.000000in}{0.000000in}}{%
\pgfpathmoveto{\pgfqpoint{0.000000in}{0.000000in}}%
\pgfpathlineto{\pgfqpoint{0.000000in}{-0.027778in}}%
\pgfusepath{stroke,fill}%
}%
\begin{pgfscope}%
\pgfsys@transformshift{1.076144in}{0.526079in}%
\pgfsys@useobject{currentmarker}{}%
\end{pgfscope}%
\end{pgfscope}%
\begin{pgfscope}%
\pgfsetbuttcap%
\pgfsetroundjoin%
\definecolor{currentfill}{rgb}{0.000000,0.000000,0.000000}%
\pgfsetfillcolor{currentfill}%
\pgfsetlinewidth{0.602250pt}%
\definecolor{currentstroke}{rgb}{0.000000,0.000000,0.000000}%
\pgfsetstrokecolor{currentstroke}%
\pgfsetdash{}{0pt}%
\pgfsys@defobject{currentmarker}{\pgfqpoint{0.000000in}{-0.027778in}}{\pgfqpoint{0.000000in}{0.000000in}}{%
\pgfpathmoveto{\pgfqpoint{0.000000in}{0.000000in}}%
\pgfpathlineto{\pgfqpoint{0.000000in}{-0.027778in}}%
\pgfusepath{stroke,fill}%
}%
\begin{pgfscope}%
\pgfsys@transformshift{1.362219in}{0.526079in}%
\pgfsys@useobject{currentmarker}{}%
\end{pgfscope}%
\end{pgfscope}%
\begin{pgfscope}%
\pgfsetbuttcap%
\pgfsetroundjoin%
\definecolor{currentfill}{rgb}{0.000000,0.000000,0.000000}%
\pgfsetfillcolor{currentfill}%
\pgfsetlinewidth{0.602250pt}%
\definecolor{currentstroke}{rgb}{0.000000,0.000000,0.000000}%
\pgfsetstrokecolor{currentstroke}%
\pgfsetdash{}{0pt}%
\pgfsys@defobject{currentmarker}{\pgfqpoint{0.000000in}{-0.027778in}}{\pgfqpoint{0.000000in}{0.000000in}}{%
\pgfpathmoveto{\pgfqpoint{0.000000in}{0.000000in}}%
\pgfpathlineto{\pgfqpoint{0.000000in}{-0.027778in}}%
\pgfusepath{stroke,fill}%
}%
\begin{pgfscope}%
\pgfsys@transformshift{1.565193in}{0.526079in}%
\pgfsys@useobject{currentmarker}{}%
\end{pgfscope}%
\end{pgfscope}%
\begin{pgfscope}%
\pgfsetbuttcap%
\pgfsetroundjoin%
\definecolor{currentfill}{rgb}{0.000000,0.000000,0.000000}%
\pgfsetfillcolor{currentfill}%
\pgfsetlinewidth{0.602250pt}%
\definecolor{currentstroke}{rgb}{0.000000,0.000000,0.000000}%
\pgfsetstrokecolor{currentstroke}%
\pgfsetdash{}{0pt}%
\pgfsys@defobject{currentmarker}{\pgfqpoint{0.000000in}{-0.027778in}}{\pgfqpoint{0.000000in}{0.000000in}}{%
\pgfpathmoveto{\pgfqpoint{0.000000in}{0.000000in}}%
\pgfpathlineto{\pgfqpoint{0.000000in}{-0.027778in}}%
\pgfusepath{stroke,fill}%
}%
\begin{pgfscope}%
\pgfsys@transformshift{1.722632in}{0.526079in}%
\pgfsys@useobject{currentmarker}{}%
\end{pgfscope}%
\end{pgfscope}%
\begin{pgfscope}%
\pgfsetbuttcap%
\pgfsetroundjoin%
\definecolor{currentfill}{rgb}{0.000000,0.000000,0.000000}%
\pgfsetfillcolor{currentfill}%
\pgfsetlinewidth{0.602250pt}%
\definecolor{currentstroke}{rgb}{0.000000,0.000000,0.000000}%
\pgfsetstrokecolor{currentstroke}%
\pgfsetdash{}{0pt}%
\pgfsys@defobject{currentmarker}{\pgfqpoint{0.000000in}{-0.027778in}}{\pgfqpoint{0.000000in}{0.000000in}}{%
\pgfpathmoveto{\pgfqpoint{0.000000in}{0.000000in}}%
\pgfpathlineto{\pgfqpoint{0.000000in}{-0.027778in}}%
\pgfusepath{stroke,fill}%
}%
\begin{pgfscope}%
\pgfsys@transformshift{1.851269in}{0.526079in}%
\pgfsys@useobject{currentmarker}{}%
\end{pgfscope}%
\end{pgfscope}%
\begin{pgfscope}%
\pgfsetbuttcap%
\pgfsetroundjoin%
\definecolor{currentfill}{rgb}{0.000000,0.000000,0.000000}%
\pgfsetfillcolor{currentfill}%
\pgfsetlinewidth{0.602250pt}%
\definecolor{currentstroke}{rgb}{0.000000,0.000000,0.000000}%
\pgfsetstrokecolor{currentstroke}%
\pgfsetdash{}{0pt}%
\pgfsys@defobject{currentmarker}{\pgfqpoint{0.000000in}{-0.027778in}}{\pgfqpoint{0.000000in}{0.000000in}}{%
\pgfpathmoveto{\pgfqpoint{0.000000in}{0.000000in}}%
\pgfpathlineto{\pgfqpoint{0.000000in}{-0.027778in}}%
\pgfusepath{stroke,fill}%
}%
\begin{pgfscope}%
\pgfsys@transformshift{1.960030in}{0.526079in}%
\pgfsys@useobject{currentmarker}{}%
\end{pgfscope}%
\end{pgfscope}%
\begin{pgfscope}%
\pgfsetbuttcap%
\pgfsetroundjoin%
\definecolor{currentfill}{rgb}{0.000000,0.000000,0.000000}%
\pgfsetfillcolor{currentfill}%
\pgfsetlinewidth{0.602250pt}%
\definecolor{currentstroke}{rgb}{0.000000,0.000000,0.000000}%
\pgfsetstrokecolor{currentstroke}%
\pgfsetdash{}{0pt}%
\pgfsys@defobject{currentmarker}{\pgfqpoint{0.000000in}{-0.027778in}}{\pgfqpoint{0.000000in}{0.000000in}}{%
\pgfpathmoveto{\pgfqpoint{0.000000in}{0.000000in}}%
\pgfpathlineto{\pgfqpoint{0.000000in}{-0.027778in}}%
\pgfusepath{stroke,fill}%
}%
\begin{pgfscope}%
\pgfsys@transformshift{2.054242in}{0.526079in}%
\pgfsys@useobject{currentmarker}{}%
\end{pgfscope}%
\end{pgfscope}%
\begin{pgfscope}%
\pgfsetbuttcap%
\pgfsetroundjoin%
\definecolor{currentfill}{rgb}{0.000000,0.000000,0.000000}%
\pgfsetfillcolor{currentfill}%
\pgfsetlinewidth{0.602250pt}%
\definecolor{currentstroke}{rgb}{0.000000,0.000000,0.000000}%
\pgfsetstrokecolor{currentstroke}%
\pgfsetdash{}{0pt}%
\pgfsys@defobject{currentmarker}{\pgfqpoint{0.000000in}{-0.027778in}}{\pgfqpoint{0.000000in}{0.000000in}}{%
\pgfpathmoveto{\pgfqpoint{0.000000in}{0.000000in}}%
\pgfpathlineto{\pgfqpoint{0.000000in}{-0.027778in}}%
\pgfusepath{stroke,fill}%
}%
\begin{pgfscope}%
\pgfsys@transformshift{2.137344in}{0.526079in}%
\pgfsys@useobject{currentmarker}{}%
\end{pgfscope}%
\end{pgfscope}%
\begin{pgfscope}%
\pgfsetbuttcap%
\pgfsetroundjoin%
\definecolor{currentfill}{rgb}{0.000000,0.000000,0.000000}%
\pgfsetfillcolor{currentfill}%
\pgfsetlinewidth{0.602250pt}%
\definecolor{currentstroke}{rgb}{0.000000,0.000000,0.000000}%
\pgfsetstrokecolor{currentstroke}%
\pgfsetdash{}{0pt}%
\pgfsys@defobject{currentmarker}{\pgfqpoint{0.000000in}{-0.027778in}}{\pgfqpoint{0.000000in}{0.000000in}}{%
\pgfpathmoveto{\pgfqpoint{0.000000in}{0.000000in}}%
\pgfpathlineto{\pgfqpoint{0.000000in}{-0.027778in}}%
\pgfusepath{stroke,fill}%
}%
\begin{pgfscope}%
\pgfsys@transformshift{2.700731in}{0.526079in}%
\pgfsys@useobject{currentmarker}{}%
\end{pgfscope}%
\end{pgfscope}%
\begin{pgfscope}%
\pgfsetbuttcap%
\pgfsetroundjoin%
\definecolor{currentfill}{rgb}{0.000000,0.000000,0.000000}%
\pgfsetfillcolor{currentfill}%
\pgfsetlinewidth{0.602250pt}%
\definecolor{currentstroke}{rgb}{0.000000,0.000000,0.000000}%
\pgfsetstrokecolor{currentstroke}%
\pgfsetdash{}{0pt}%
\pgfsys@defobject{currentmarker}{\pgfqpoint{0.000000in}{-0.027778in}}{\pgfqpoint{0.000000in}{0.000000in}}{%
\pgfpathmoveto{\pgfqpoint{0.000000in}{0.000000in}}%
\pgfpathlineto{\pgfqpoint{0.000000in}{-0.027778in}}%
\pgfusepath{stroke,fill}%
}%
\begin{pgfscope}%
\pgfsys@transformshift{2.986806in}{0.526079in}%
\pgfsys@useobject{currentmarker}{}%
\end{pgfscope}%
\end{pgfscope}%
\begin{pgfscope}%
\pgfsetbuttcap%
\pgfsetroundjoin%
\definecolor{currentfill}{rgb}{0.000000,0.000000,0.000000}%
\pgfsetfillcolor{currentfill}%
\pgfsetlinewidth{0.602250pt}%
\definecolor{currentstroke}{rgb}{0.000000,0.000000,0.000000}%
\pgfsetstrokecolor{currentstroke}%
\pgfsetdash{}{0pt}%
\pgfsys@defobject{currentmarker}{\pgfqpoint{0.000000in}{-0.027778in}}{\pgfqpoint{0.000000in}{0.000000in}}{%
\pgfpathmoveto{\pgfqpoint{0.000000in}{0.000000in}}%
\pgfpathlineto{\pgfqpoint{0.000000in}{-0.027778in}}%
\pgfusepath{stroke,fill}%
}%
\begin{pgfscope}%
\pgfsys@transformshift{3.189780in}{0.526079in}%
\pgfsys@useobject{currentmarker}{}%
\end{pgfscope}%
\end{pgfscope}%
\begin{pgfscope}%
\pgfsetbuttcap%
\pgfsetroundjoin%
\definecolor{currentfill}{rgb}{0.000000,0.000000,0.000000}%
\pgfsetfillcolor{currentfill}%
\pgfsetlinewidth{0.602250pt}%
\definecolor{currentstroke}{rgb}{0.000000,0.000000,0.000000}%
\pgfsetstrokecolor{currentstroke}%
\pgfsetdash{}{0pt}%
\pgfsys@defobject{currentmarker}{\pgfqpoint{0.000000in}{-0.027778in}}{\pgfqpoint{0.000000in}{0.000000in}}{%
\pgfpathmoveto{\pgfqpoint{0.000000in}{0.000000in}}%
\pgfpathlineto{\pgfqpoint{0.000000in}{-0.027778in}}%
\pgfusepath{stroke,fill}%
}%
\begin{pgfscope}%
\pgfsys@transformshift{3.347219in}{0.526079in}%
\pgfsys@useobject{currentmarker}{}%
\end{pgfscope}%
\end{pgfscope}%
\begin{pgfscope}%
\pgfsetbuttcap%
\pgfsetroundjoin%
\definecolor{currentfill}{rgb}{0.000000,0.000000,0.000000}%
\pgfsetfillcolor{currentfill}%
\pgfsetlinewidth{0.602250pt}%
\definecolor{currentstroke}{rgb}{0.000000,0.000000,0.000000}%
\pgfsetstrokecolor{currentstroke}%
\pgfsetdash{}{0pt}%
\pgfsys@defobject{currentmarker}{\pgfqpoint{0.000000in}{-0.027778in}}{\pgfqpoint{0.000000in}{0.000000in}}{%
\pgfpathmoveto{\pgfqpoint{0.000000in}{0.000000in}}%
\pgfpathlineto{\pgfqpoint{0.000000in}{-0.027778in}}%
\pgfusepath{stroke,fill}%
}%
\begin{pgfscope}%
\pgfsys@transformshift{3.475856in}{0.526079in}%
\pgfsys@useobject{currentmarker}{}%
\end{pgfscope}%
\end{pgfscope}%
\begin{pgfscope}%
\pgfsetbuttcap%
\pgfsetroundjoin%
\definecolor{currentfill}{rgb}{0.000000,0.000000,0.000000}%
\pgfsetfillcolor{currentfill}%
\pgfsetlinewidth{0.602250pt}%
\definecolor{currentstroke}{rgb}{0.000000,0.000000,0.000000}%
\pgfsetstrokecolor{currentstroke}%
\pgfsetdash{}{0pt}%
\pgfsys@defobject{currentmarker}{\pgfqpoint{0.000000in}{-0.027778in}}{\pgfqpoint{0.000000in}{0.000000in}}{%
\pgfpathmoveto{\pgfqpoint{0.000000in}{0.000000in}}%
\pgfpathlineto{\pgfqpoint{0.000000in}{-0.027778in}}%
\pgfusepath{stroke,fill}%
}%
\begin{pgfscope}%
\pgfsys@transformshift{3.584616in}{0.526079in}%
\pgfsys@useobject{currentmarker}{}%
\end{pgfscope}%
\end{pgfscope}%
\begin{pgfscope}%
\pgfsetbuttcap%
\pgfsetroundjoin%
\definecolor{currentfill}{rgb}{0.000000,0.000000,0.000000}%
\pgfsetfillcolor{currentfill}%
\pgfsetlinewidth{0.602250pt}%
\definecolor{currentstroke}{rgb}{0.000000,0.000000,0.000000}%
\pgfsetstrokecolor{currentstroke}%
\pgfsetdash{}{0pt}%
\pgfsys@defobject{currentmarker}{\pgfqpoint{0.000000in}{-0.027778in}}{\pgfqpoint{0.000000in}{0.000000in}}{%
\pgfpathmoveto{\pgfqpoint{0.000000in}{0.000000in}}%
\pgfpathlineto{\pgfqpoint{0.000000in}{-0.027778in}}%
\pgfusepath{stroke,fill}%
}%
\begin{pgfscope}%
\pgfsys@transformshift{3.678829in}{0.526079in}%
\pgfsys@useobject{currentmarker}{}%
\end{pgfscope}%
\end{pgfscope}%
\begin{pgfscope}%
\pgfsetbuttcap%
\pgfsetroundjoin%
\definecolor{currentfill}{rgb}{0.000000,0.000000,0.000000}%
\pgfsetfillcolor{currentfill}%
\pgfsetlinewidth{0.602250pt}%
\definecolor{currentstroke}{rgb}{0.000000,0.000000,0.000000}%
\pgfsetstrokecolor{currentstroke}%
\pgfsetdash{}{0pt}%
\pgfsys@defobject{currentmarker}{\pgfqpoint{0.000000in}{-0.027778in}}{\pgfqpoint{0.000000in}{0.000000in}}{%
\pgfpathmoveto{\pgfqpoint{0.000000in}{0.000000in}}%
\pgfpathlineto{\pgfqpoint{0.000000in}{-0.027778in}}%
\pgfusepath{stroke,fill}%
}%
\begin{pgfscope}%
\pgfsys@transformshift{3.761931in}{0.526079in}%
\pgfsys@useobject{currentmarker}{}%
\end{pgfscope}%
\end{pgfscope}%
\begin{pgfscope}%
\pgfsetbuttcap%
\pgfsetroundjoin%
\definecolor{currentfill}{rgb}{0.000000,0.000000,0.000000}%
\pgfsetfillcolor{currentfill}%
\pgfsetlinewidth{0.602250pt}%
\definecolor{currentstroke}{rgb}{0.000000,0.000000,0.000000}%
\pgfsetstrokecolor{currentstroke}%
\pgfsetdash{}{0pt}%
\pgfsys@defobject{currentmarker}{\pgfqpoint{0.000000in}{-0.027778in}}{\pgfqpoint{0.000000in}{0.000000in}}{%
\pgfpathmoveto{\pgfqpoint{0.000000in}{0.000000in}}%
\pgfpathlineto{\pgfqpoint{0.000000in}{-0.027778in}}%
\pgfusepath{stroke,fill}%
}%
\begin{pgfscope}%
\pgfsys@transformshift{4.325318in}{0.526079in}%
\pgfsys@useobject{currentmarker}{}%
\end{pgfscope}%
\end{pgfscope}%
\begin{pgfscope}%
\pgfsetbuttcap%
\pgfsetroundjoin%
\definecolor{currentfill}{rgb}{0.000000,0.000000,0.000000}%
\pgfsetfillcolor{currentfill}%
\pgfsetlinewidth{0.602250pt}%
\definecolor{currentstroke}{rgb}{0.000000,0.000000,0.000000}%
\pgfsetstrokecolor{currentstroke}%
\pgfsetdash{}{0pt}%
\pgfsys@defobject{currentmarker}{\pgfqpoint{0.000000in}{-0.027778in}}{\pgfqpoint{0.000000in}{0.000000in}}{%
\pgfpathmoveto{\pgfqpoint{0.000000in}{0.000000in}}%
\pgfpathlineto{\pgfqpoint{0.000000in}{-0.027778in}}%
\pgfusepath{stroke,fill}%
}%
\begin{pgfscope}%
\pgfsys@transformshift{4.611393in}{0.526079in}%
\pgfsys@useobject{currentmarker}{}%
\end{pgfscope}%
\end{pgfscope}%
\begin{pgfscope}%
\pgfsetbuttcap%
\pgfsetroundjoin%
\definecolor{currentfill}{rgb}{0.000000,0.000000,0.000000}%
\pgfsetfillcolor{currentfill}%
\pgfsetlinewidth{0.602250pt}%
\definecolor{currentstroke}{rgb}{0.000000,0.000000,0.000000}%
\pgfsetstrokecolor{currentstroke}%
\pgfsetdash{}{0pt}%
\pgfsys@defobject{currentmarker}{\pgfqpoint{0.000000in}{-0.027778in}}{\pgfqpoint{0.000000in}{0.000000in}}{%
\pgfpathmoveto{\pgfqpoint{0.000000in}{0.000000in}}%
\pgfpathlineto{\pgfqpoint{0.000000in}{-0.027778in}}%
\pgfusepath{stroke,fill}%
}%
\begin{pgfscope}%
\pgfsys@transformshift{4.814367in}{0.526079in}%
\pgfsys@useobject{currentmarker}{}%
\end{pgfscope}%
\end{pgfscope}%
\begin{pgfscope}%
\pgfsetbuttcap%
\pgfsetroundjoin%
\definecolor{currentfill}{rgb}{0.000000,0.000000,0.000000}%
\pgfsetfillcolor{currentfill}%
\pgfsetlinewidth{0.602250pt}%
\definecolor{currentstroke}{rgb}{0.000000,0.000000,0.000000}%
\pgfsetstrokecolor{currentstroke}%
\pgfsetdash{}{0pt}%
\pgfsys@defobject{currentmarker}{\pgfqpoint{0.000000in}{-0.027778in}}{\pgfqpoint{0.000000in}{0.000000in}}{%
\pgfpathmoveto{\pgfqpoint{0.000000in}{0.000000in}}%
\pgfpathlineto{\pgfqpoint{0.000000in}{-0.027778in}}%
\pgfusepath{stroke,fill}%
}%
\begin{pgfscope}%
\pgfsys@transformshift{4.971806in}{0.526079in}%
\pgfsys@useobject{currentmarker}{}%
\end{pgfscope}%
\end{pgfscope}%
\begin{pgfscope}%
\pgfsetbuttcap%
\pgfsetroundjoin%
\definecolor{currentfill}{rgb}{0.000000,0.000000,0.000000}%
\pgfsetfillcolor{currentfill}%
\pgfsetlinewidth{0.602250pt}%
\definecolor{currentstroke}{rgb}{0.000000,0.000000,0.000000}%
\pgfsetstrokecolor{currentstroke}%
\pgfsetdash{}{0pt}%
\pgfsys@defobject{currentmarker}{\pgfqpoint{0.000000in}{-0.027778in}}{\pgfqpoint{0.000000in}{0.000000in}}{%
\pgfpathmoveto{\pgfqpoint{0.000000in}{0.000000in}}%
\pgfpathlineto{\pgfqpoint{0.000000in}{-0.027778in}}%
\pgfusepath{stroke,fill}%
}%
\begin{pgfscope}%
\pgfsys@transformshift{5.100443in}{0.526079in}%
\pgfsys@useobject{currentmarker}{}%
\end{pgfscope}%
\end{pgfscope}%
\begin{pgfscope}%
\pgfsetbuttcap%
\pgfsetroundjoin%
\definecolor{currentfill}{rgb}{0.000000,0.000000,0.000000}%
\pgfsetfillcolor{currentfill}%
\pgfsetlinewidth{0.602250pt}%
\definecolor{currentstroke}{rgb}{0.000000,0.000000,0.000000}%
\pgfsetstrokecolor{currentstroke}%
\pgfsetdash{}{0pt}%
\pgfsys@defobject{currentmarker}{\pgfqpoint{0.000000in}{-0.027778in}}{\pgfqpoint{0.000000in}{0.000000in}}{%
\pgfpathmoveto{\pgfqpoint{0.000000in}{0.000000in}}%
\pgfpathlineto{\pgfqpoint{0.000000in}{-0.027778in}}%
\pgfusepath{stroke,fill}%
}%
\begin{pgfscope}%
\pgfsys@transformshift{5.209203in}{0.526079in}%
\pgfsys@useobject{currentmarker}{}%
\end{pgfscope}%
\end{pgfscope}%
\begin{pgfscope}%
\pgfsetbuttcap%
\pgfsetroundjoin%
\definecolor{currentfill}{rgb}{0.000000,0.000000,0.000000}%
\pgfsetfillcolor{currentfill}%
\pgfsetlinewidth{0.602250pt}%
\definecolor{currentstroke}{rgb}{0.000000,0.000000,0.000000}%
\pgfsetstrokecolor{currentstroke}%
\pgfsetdash{}{0pt}%
\pgfsys@defobject{currentmarker}{\pgfqpoint{0.000000in}{-0.027778in}}{\pgfqpoint{0.000000in}{0.000000in}}{%
\pgfpathmoveto{\pgfqpoint{0.000000in}{0.000000in}}%
\pgfpathlineto{\pgfqpoint{0.000000in}{-0.027778in}}%
\pgfusepath{stroke,fill}%
}%
\begin{pgfscope}%
\pgfsys@transformshift{5.303416in}{0.526079in}%
\pgfsys@useobject{currentmarker}{}%
\end{pgfscope}%
\end{pgfscope}%
\begin{pgfscope}%
\pgftext[x=3.004669in,y=0.238889in,,top]{\rmfamily\fontsize{10.000000}{12.000000}\selectfont \(\displaystyle y/L\)}%
\end{pgfscope}%
\begin{pgfscope}%
\pgfsetbuttcap%
\pgfsetroundjoin%
\definecolor{currentfill}{rgb}{0.000000,0.000000,0.000000}%
\pgfsetfillcolor{currentfill}%
\pgfsetlinewidth{0.803000pt}%
\definecolor{currentstroke}{rgb}{0.000000,0.000000,0.000000}%
\pgfsetstrokecolor{currentstroke}%
\pgfsetdash{}{0pt}%
\pgfsys@defobject{currentmarker}{\pgfqpoint{-0.048611in}{0.000000in}}{\pgfqpoint{0.000000in}{0.000000in}}{%
\pgfpathmoveto{\pgfqpoint{0.000000in}{0.000000in}}%
\pgfpathlineto{\pgfqpoint{-0.048611in}{0.000000in}}%
\pgfusepath{stroke,fill}%
}%
\begin{pgfscope}%
\pgfsys@transformshift{0.679669in}{1.104918in}%
\pgfsys@useobject{currentmarker}{}%
\end{pgfscope}%
\end{pgfscope}%
\begin{pgfscope}%
\pgftext[x=0.294444in,y=1.052156in,left,base]{\rmfamily\fontsize{10.000000}{12.000000}\selectfont \(\displaystyle 10^{-2}\)}%
\end{pgfscope}%
\begin{pgfscope}%
\pgfsetbuttcap%
\pgfsetroundjoin%
\definecolor{currentfill}{rgb}{0.000000,0.000000,0.000000}%
\pgfsetfillcolor{currentfill}%
\pgfsetlinewidth{0.803000pt}%
\definecolor{currentstroke}{rgb}{0.000000,0.000000,0.000000}%
\pgfsetstrokecolor{currentstroke}%
\pgfsetdash{}{0pt}%
\pgfsys@defobject{currentmarker}{\pgfqpoint{-0.048611in}{0.000000in}}{\pgfqpoint{0.000000in}{0.000000in}}{%
\pgfpathmoveto{\pgfqpoint{0.000000in}{0.000000in}}%
\pgfpathlineto{\pgfqpoint{-0.048611in}{0.000000in}}%
\pgfusepath{stroke,fill}%
}%
\begin{pgfscope}%
\pgfsys@transformshift{0.679669in}{2.256869in}%
\pgfsys@useobject{currentmarker}{}%
\end{pgfscope}%
\end{pgfscope}%
\begin{pgfscope}%
\pgftext[x=0.294444in,y=2.204108in,left,base]{\rmfamily\fontsize{10.000000}{12.000000}\selectfont \(\displaystyle 10^{-1}\)}%
\end{pgfscope}%
\begin{pgfscope}%
\pgfsetbuttcap%
\pgfsetroundjoin%
\definecolor{currentfill}{rgb}{0.000000,0.000000,0.000000}%
\pgfsetfillcolor{currentfill}%
\pgfsetlinewidth{0.803000pt}%
\definecolor{currentstroke}{rgb}{0.000000,0.000000,0.000000}%
\pgfsetstrokecolor{currentstroke}%
\pgfsetdash{}{0pt}%
\pgfsys@defobject{currentmarker}{\pgfqpoint{-0.048611in}{0.000000in}}{\pgfqpoint{0.000000in}{0.000000in}}{%
\pgfpathmoveto{\pgfqpoint{0.000000in}{0.000000in}}%
\pgfpathlineto{\pgfqpoint{-0.048611in}{0.000000in}}%
\pgfusepath{stroke,fill}%
}%
\begin{pgfscope}%
\pgfsys@transformshift{0.679669in}{3.408821in}%
\pgfsys@useobject{currentmarker}{}%
\end{pgfscope}%
\end{pgfscope}%
\begin{pgfscope}%
\pgftext[x=0.381250in,y=3.356059in,left,base]{\rmfamily\fontsize{10.000000}{12.000000}\selectfont \(\displaystyle 10^{0}\)}%
\end{pgfscope}%
\begin{pgfscope}%
\pgfsetbuttcap%
\pgfsetroundjoin%
\definecolor{currentfill}{rgb}{0.000000,0.000000,0.000000}%
\pgfsetfillcolor{currentfill}%
\pgfsetlinewidth{0.602250pt}%
\definecolor{currentstroke}{rgb}{0.000000,0.000000,0.000000}%
\pgfsetstrokecolor{currentstroke}%
\pgfsetdash{}{0pt}%
\pgfsys@defobject{currentmarker}{\pgfqpoint{-0.027778in}{0.000000in}}{\pgfqpoint{0.000000in}{0.000000in}}{%
\pgfpathmoveto{\pgfqpoint{0.000000in}{0.000000in}}%
\pgfpathlineto{\pgfqpoint{-0.027778in}{0.000000in}}%
\pgfusepath{stroke,fill}%
}%
\begin{pgfscope}%
\pgfsys@transformshift{0.679669in}{0.646510in}%
\pgfsys@useobject{currentmarker}{}%
\end{pgfscope}%
\end{pgfscope}%
\begin{pgfscope}%
\pgfsetbuttcap%
\pgfsetroundjoin%
\definecolor{currentfill}{rgb}{0.000000,0.000000,0.000000}%
\pgfsetfillcolor{currentfill}%
\pgfsetlinewidth{0.602250pt}%
\definecolor{currentstroke}{rgb}{0.000000,0.000000,0.000000}%
\pgfsetstrokecolor{currentstroke}%
\pgfsetdash{}{0pt}%
\pgfsys@defobject{currentmarker}{\pgfqpoint{-0.027778in}{0.000000in}}{\pgfqpoint{0.000000in}{0.000000in}}{%
\pgfpathmoveto{\pgfqpoint{0.000000in}{0.000000in}}%
\pgfpathlineto{\pgfqpoint{-0.027778in}{0.000000in}}%
\pgfusepath{stroke,fill}%
}%
\begin{pgfscope}%
\pgfsys@transformshift{0.679669in}{0.758146in}%
\pgfsys@useobject{currentmarker}{}%
\end{pgfscope}%
\end{pgfscope}%
\begin{pgfscope}%
\pgfsetbuttcap%
\pgfsetroundjoin%
\definecolor{currentfill}{rgb}{0.000000,0.000000,0.000000}%
\pgfsetfillcolor{currentfill}%
\pgfsetlinewidth{0.602250pt}%
\definecolor{currentstroke}{rgb}{0.000000,0.000000,0.000000}%
\pgfsetstrokecolor{currentstroke}%
\pgfsetdash{}{0pt}%
\pgfsys@defobject{currentmarker}{\pgfqpoint{-0.027778in}{0.000000in}}{\pgfqpoint{0.000000in}{0.000000in}}{%
\pgfpathmoveto{\pgfqpoint{0.000000in}{0.000000in}}%
\pgfpathlineto{\pgfqpoint{-0.027778in}{0.000000in}}%
\pgfusepath{stroke,fill}%
}%
\begin{pgfscope}%
\pgfsys@transformshift{0.679669in}{0.849359in}%
\pgfsys@useobject{currentmarker}{}%
\end{pgfscope}%
\end{pgfscope}%
\begin{pgfscope}%
\pgfsetbuttcap%
\pgfsetroundjoin%
\definecolor{currentfill}{rgb}{0.000000,0.000000,0.000000}%
\pgfsetfillcolor{currentfill}%
\pgfsetlinewidth{0.602250pt}%
\definecolor{currentstroke}{rgb}{0.000000,0.000000,0.000000}%
\pgfsetstrokecolor{currentstroke}%
\pgfsetdash{}{0pt}%
\pgfsys@defobject{currentmarker}{\pgfqpoint{-0.027778in}{0.000000in}}{\pgfqpoint{0.000000in}{0.000000in}}{%
\pgfpathmoveto{\pgfqpoint{0.000000in}{0.000000in}}%
\pgfpathlineto{\pgfqpoint{-0.027778in}{0.000000in}}%
\pgfusepath{stroke,fill}%
}%
\begin{pgfscope}%
\pgfsys@transformshift{0.679669in}{0.926478in}%
\pgfsys@useobject{currentmarker}{}%
\end{pgfscope}%
\end{pgfscope}%
\begin{pgfscope}%
\pgfsetbuttcap%
\pgfsetroundjoin%
\definecolor{currentfill}{rgb}{0.000000,0.000000,0.000000}%
\pgfsetfillcolor{currentfill}%
\pgfsetlinewidth{0.602250pt}%
\definecolor{currentstroke}{rgb}{0.000000,0.000000,0.000000}%
\pgfsetstrokecolor{currentstroke}%
\pgfsetdash{}{0pt}%
\pgfsys@defobject{currentmarker}{\pgfqpoint{-0.027778in}{0.000000in}}{\pgfqpoint{0.000000in}{0.000000in}}{%
\pgfpathmoveto{\pgfqpoint{0.000000in}{0.000000in}}%
\pgfpathlineto{\pgfqpoint{-0.027778in}{0.000000in}}%
\pgfusepath{stroke,fill}%
}%
\begin{pgfscope}%
\pgfsys@transformshift{0.679669in}{0.993282in}%
\pgfsys@useobject{currentmarker}{}%
\end{pgfscope}%
\end{pgfscope}%
\begin{pgfscope}%
\pgfsetbuttcap%
\pgfsetroundjoin%
\definecolor{currentfill}{rgb}{0.000000,0.000000,0.000000}%
\pgfsetfillcolor{currentfill}%
\pgfsetlinewidth{0.602250pt}%
\definecolor{currentstroke}{rgb}{0.000000,0.000000,0.000000}%
\pgfsetstrokecolor{currentstroke}%
\pgfsetdash{}{0pt}%
\pgfsys@defobject{currentmarker}{\pgfqpoint{-0.027778in}{0.000000in}}{\pgfqpoint{0.000000in}{0.000000in}}{%
\pgfpathmoveto{\pgfqpoint{0.000000in}{0.000000in}}%
\pgfpathlineto{\pgfqpoint{-0.027778in}{0.000000in}}%
\pgfusepath{stroke,fill}%
}%
\begin{pgfscope}%
\pgfsys@transformshift{0.679669in}{1.052207in}%
\pgfsys@useobject{currentmarker}{}%
\end{pgfscope}%
\end{pgfscope}%
\begin{pgfscope}%
\pgfsetbuttcap%
\pgfsetroundjoin%
\definecolor{currentfill}{rgb}{0.000000,0.000000,0.000000}%
\pgfsetfillcolor{currentfill}%
\pgfsetlinewidth{0.602250pt}%
\definecolor{currentstroke}{rgb}{0.000000,0.000000,0.000000}%
\pgfsetstrokecolor{currentstroke}%
\pgfsetdash{}{0pt}%
\pgfsys@defobject{currentmarker}{\pgfqpoint{-0.027778in}{0.000000in}}{\pgfqpoint{0.000000in}{0.000000in}}{%
\pgfpathmoveto{\pgfqpoint{0.000000in}{0.000000in}}%
\pgfpathlineto{\pgfqpoint{-0.027778in}{0.000000in}}%
\pgfusepath{stroke,fill}%
}%
\begin{pgfscope}%
\pgfsys@transformshift{0.679669in}{1.451690in}%
\pgfsys@useobject{currentmarker}{}%
\end{pgfscope}%
\end{pgfscope}%
\begin{pgfscope}%
\pgfsetbuttcap%
\pgfsetroundjoin%
\definecolor{currentfill}{rgb}{0.000000,0.000000,0.000000}%
\pgfsetfillcolor{currentfill}%
\pgfsetlinewidth{0.602250pt}%
\definecolor{currentstroke}{rgb}{0.000000,0.000000,0.000000}%
\pgfsetstrokecolor{currentstroke}%
\pgfsetdash{}{0pt}%
\pgfsys@defobject{currentmarker}{\pgfqpoint{-0.027778in}{0.000000in}}{\pgfqpoint{0.000000in}{0.000000in}}{%
\pgfpathmoveto{\pgfqpoint{0.000000in}{0.000000in}}%
\pgfpathlineto{\pgfqpoint{-0.027778in}{0.000000in}}%
\pgfusepath{stroke,fill}%
}%
\begin{pgfscope}%
\pgfsys@transformshift{0.679669in}{1.654538in}%
\pgfsys@useobject{currentmarker}{}%
\end{pgfscope}%
\end{pgfscope}%
\begin{pgfscope}%
\pgfsetbuttcap%
\pgfsetroundjoin%
\definecolor{currentfill}{rgb}{0.000000,0.000000,0.000000}%
\pgfsetfillcolor{currentfill}%
\pgfsetlinewidth{0.602250pt}%
\definecolor{currentstroke}{rgb}{0.000000,0.000000,0.000000}%
\pgfsetstrokecolor{currentstroke}%
\pgfsetdash{}{0pt}%
\pgfsys@defobject{currentmarker}{\pgfqpoint{-0.027778in}{0.000000in}}{\pgfqpoint{0.000000in}{0.000000in}}{%
\pgfpathmoveto{\pgfqpoint{0.000000in}{0.000000in}}%
\pgfpathlineto{\pgfqpoint{-0.027778in}{0.000000in}}%
\pgfusepath{stroke,fill}%
}%
\begin{pgfscope}%
\pgfsys@transformshift{0.679669in}{1.798462in}%
\pgfsys@useobject{currentmarker}{}%
\end{pgfscope}%
\end{pgfscope}%
\begin{pgfscope}%
\pgfsetbuttcap%
\pgfsetroundjoin%
\definecolor{currentfill}{rgb}{0.000000,0.000000,0.000000}%
\pgfsetfillcolor{currentfill}%
\pgfsetlinewidth{0.602250pt}%
\definecolor{currentstroke}{rgb}{0.000000,0.000000,0.000000}%
\pgfsetstrokecolor{currentstroke}%
\pgfsetdash{}{0pt}%
\pgfsys@defobject{currentmarker}{\pgfqpoint{-0.027778in}{0.000000in}}{\pgfqpoint{0.000000in}{0.000000in}}{%
\pgfpathmoveto{\pgfqpoint{0.000000in}{0.000000in}}%
\pgfpathlineto{\pgfqpoint{-0.027778in}{0.000000in}}%
\pgfusepath{stroke,fill}%
}%
\begin{pgfscope}%
\pgfsys@transformshift{0.679669in}{1.910097in}%
\pgfsys@useobject{currentmarker}{}%
\end{pgfscope}%
\end{pgfscope}%
\begin{pgfscope}%
\pgfsetbuttcap%
\pgfsetroundjoin%
\definecolor{currentfill}{rgb}{0.000000,0.000000,0.000000}%
\pgfsetfillcolor{currentfill}%
\pgfsetlinewidth{0.602250pt}%
\definecolor{currentstroke}{rgb}{0.000000,0.000000,0.000000}%
\pgfsetstrokecolor{currentstroke}%
\pgfsetdash{}{0pt}%
\pgfsys@defobject{currentmarker}{\pgfqpoint{-0.027778in}{0.000000in}}{\pgfqpoint{0.000000in}{0.000000in}}{%
\pgfpathmoveto{\pgfqpoint{0.000000in}{0.000000in}}%
\pgfpathlineto{\pgfqpoint{-0.027778in}{0.000000in}}%
\pgfusepath{stroke,fill}%
}%
\begin{pgfscope}%
\pgfsys@transformshift{0.679669in}{2.001310in}%
\pgfsys@useobject{currentmarker}{}%
\end{pgfscope}%
\end{pgfscope}%
\begin{pgfscope}%
\pgfsetbuttcap%
\pgfsetroundjoin%
\definecolor{currentfill}{rgb}{0.000000,0.000000,0.000000}%
\pgfsetfillcolor{currentfill}%
\pgfsetlinewidth{0.602250pt}%
\definecolor{currentstroke}{rgb}{0.000000,0.000000,0.000000}%
\pgfsetstrokecolor{currentstroke}%
\pgfsetdash{}{0pt}%
\pgfsys@defobject{currentmarker}{\pgfqpoint{-0.027778in}{0.000000in}}{\pgfqpoint{0.000000in}{0.000000in}}{%
\pgfpathmoveto{\pgfqpoint{0.000000in}{0.000000in}}%
\pgfpathlineto{\pgfqpoint{-0.027778in}{0.000000in}}%
\pgfusepath{stroke,fill}%
}%
\begin{pgfscope}%
\pgfsys@transformshift{0.679669in}{2.078430in}%
\pgfsys@useobject{currentmarker}{}%
\end{pgfscope}%
\end{pgfscope}%
\begin{pgfscope}%
\pgfsetbuttcap%
\pgfsetroundjoin%
\definecolor{currentfill}{rgb}{0.000000,0.000000,0.000000}%
\pgfsetfillcolor{currentfill}%
\pgfsetlinewidth{0.602250pt}%
\definecolor{currentstroke}{rgb}{0.000000,0.000000,0.000000}%
\pgfsetstrokecolor{currentstroke}%
\pgfsetdash{}{0pt}%
\pgfsys@defobject{currentmarker}{\pgfqpoint{-0.027778in}{0.000000in}}{\pgfqpoint{0.000000in}{0.000000in}}{%
\pgfpathmoveto{\pgfqpoint{0.000000in}{0.000000in}}%
\pgfpathlineto{\pgfqpoint{-0.027778in}{0.000000in}}%
\pgfusepath{stroke,fill}%
}%
\begin{pgfscope}%
\pgfsys@transformshift{0.679669in}{2.145234in}%
\pgfsys@useobject{currentmarker}{}%
\end{pgfscope}%
\end{pgfscope}%
\begin{pgfscope}%
\pgfsetbuttcap%
\pgfsetroundjoin%
\definecolor{currentfill}{rgb}{0.000000,0.000000,0.000000}%
\pgfsetfillcolor{currentfill}%
\pgfsetlinewidth{0.602250pt}%
\definecolor{currentstroke}{rgb}{0.000000,0.000000,0.000000}%
\pgfsetstrokecolor{currentstroke}%
\pgfsetdash{}{0pt}%
\pgfsys@defobject{currentmarker}{\pgfqpoint{-0.027778in}{0.000000in}}{\pgfqpoint{0.000000in}{0.000000in}}{%
\pgfpathmoveto{\pgfqpoint{0.000000in}{0.000000in}}%
\pgfpathlineto{\pgfqpoint{-0.027778in}{0.000000in}}%
\pgfusepath{stroke,fill}%
}%
\begin{pgfscope}%
\pgfsys@transformshift{0.679669in}{2.204159in}%
\pgfsys@useobject{currentmarker}{}%
\end{pgfscope}%
\end{pgfscope}%
\begin{pgfscope}%
\pgfsetbuttcap%
\pgfsetroundjoin%
\definecolor{currentfill}{rgb}{0.000000,0.000000,0.000000}%
\pgfsetfillcolor{currentfill}%
\pgfsetlinewidth{0.602250pt}%
\definecolor{currentstroke}{rgb}{0.000000,0.000000,0.000000}%
\pgfsetstrokecolor{currentstroke}%
\pgfsetdash{}{0pt}%
\pgfsys@defobject{currentmarker}{\pgfqpoint{-0.027778in}{0.000000in}}{\pgfqpoint{0.000000in}{0.000000in}}{%
\pgfpathmoveto{\pgfqpoint{0.000000in}{0.000000in}}%
\pgfpathlineto{\pgfqpoint{-0.027778in}{0.000000in}}%
\pgfusepath{stroke,fill}%
}%
\begin{pgfscope}%
\pgfsys@transformshift{0.679669in}{2.603641in}%
\pgfsys@useobject{currentmarker}{}%
\end{pgfscope}%
\end{pgfscope}%
\begin{pgfscope}%
\pgfsetbuttcap%
\pgfsetroundjoin%
\definecolor{currentfill}{rgb}{0.000000,0.000000,0.000000}%
\pgfsetfillcolor{currentfill}%
\pgfsetlinewidth{0.602250pt}%
\definecolor{currentstroke}{rgb}{0.000000,0.000000,0.000000}%
\pgfsetstrokecolor{currentstroke}%
\pgfsetdash{}{0pt}%
\pgfsys@defobject{currentmarker}{\pgfqpoint{-0.027778in}{0.000000in}}{\pgfqpoint{0.000000in}{0.000000in}}{%
\pgfpathmoveto{\pgfqpoint{0.000000in}{0.000000in}}%
\pgfpathlineto{\pgfqpoint{-0.027778in}{0.000000in}}%
\pgfusepath{stroke,fill}%
}%
\begin{pgfscope}%
\pgfsys@transformshift{0.679669in}{2.806490in}%
\pgfsys@useobject{currentmarker}{}%
\end{pgfscope}%
\end{pgfscope}%
\begin{pgfscope}%
\pgfsetbuttcap%
\pgfsetroundjoin%
\definecolor{currentfill}{rgb}{0.000000,0.000000,0.000000}%
\pgfsetfillcolor{currentfill}%
\pgfsetlinewidth{0.602250pt}%
\definecolor{currentstroke}{rgb}{0.000000,0.000000,0.000000}%
\pgfsetstrokecolor{currentstroke}%
\pgfsetdash{}{0pt}%
\pgfsys@defobject{currentmarker}{\pgfqpoint{-0.027778in}{0.000000in}}{\pgfqpoint{0.000000in}{0.000000in}}{%
\pgfpathmoveto{\pgfqpoint{0.000000in}{0.000000in}}%
\pgfpathlineto{\pgfqpoint{-0.027778in}{0.000000in}}%
\pgfusepath{stroke,fill}%
}%
\begin{pgfscope}%
\pgfsys@transformshift{0.679669in}{2.950413in}%
\pgfsys@useobject{currentmarker}{}%
\end{pgfscope}%
\end{pgfscope}%
\begin{pgfscope}%
\pgfsetbuttcap%
\pgfsetroundjoin%
\definecolor{currentfill}{rgb}{0.000000,0.000000,0.000000}%
\pgfsetfillcolor{currentfill}%
\pgfsetlinewidth{0.602250pt}%
\definecolor{currentstroke}{rgb}{0.000000,0.000000,0.000000}%
\pgfsetstrokecolor{currentstroke}%
\pgfsetdash{}{0pt}%
\pgfsys@defobject{currentmarker}{\pgfqpoint{-0.027778in}{0.000000in}}{\pgfqpoint{0.000000in}{0.000000in}}{%
\pgfpathmoveto{\pgfqpoint{0.000000in}{0.000000in}}%
\pgfpathlineto{\pgfqpoint{-0.027778in}{0.000000in}}%
\pgfusepath{stroke,fill}%
}%
\begin{pgfscope}%
\pgfsys@transformshift{0.679669in}{3.062049in}%
\pgfsys@useobject{currentmarker}{}%
\end{pgfscope}%
\end{pgfscope}%
\begin{pgfscope}%
\pgfsetbuttcap%
\pgfsetroundjoin%
\definecolor{currentfill}{rgb}{0.000000,0.000000,0.000000}%
\pgfsetfillcolor{currentfill}%
\pgfsetlinewidth{0.602250pt}%
\definecolor{currentstroke}{rgb}{0.000000,0.000000,0.000000}%
\pgfsetstrokecolor{currentstroke}%
\pgfsetdash{}{0pt}%
\pgfsys@defobject{currentmarker}{\pgfqpoint{-0.027778in}{0.000000in}}{\pgfqpoint{0.000000in}{0.000000in}}{%
\pgfpathmoveto{\pgfqpoint{0.000000in}{0.000000in}}%
\pgfpathlineto{\pgfqpoint{-0.027778in}{0.000000in}}%
\pgfusepath{stroke,fill}%
}%
\begin{pgfscope}%
\pgfsys@transformshift{0.679669in}{3.153262in}%
\pgfsys@useobject{currentmarker}{}%
\end{pgfscope}%
\end{pgfscope}%
\begin{pgfscope}%
\pgfsetbuttcap%
\pgfsetroundjoin%
\definecolor{currentfill}{rgb}{0.000000,0.000000,0.000000}%
\pgfsetfillcolor{currentfill}%
\pgfsetlinewidth{0.602250pt}%
\definecolor{currentstroke}{rgb}{0.000000,0.000000,0.000000}%
\pgfsetstrokecolor{currentstroke}%
\pgfsetdash{}{0pt}%
\pgfsys@defobject{currentmarker}{\pgfqpoint{-0.027778in}{0.000000in}}{\pgfqpoint{0.000000in}{0.000000in}}{%
\pgfpathmoveto{\pgfqpoint{0.000000in}{0.000000in}}%
\pgfpathlineto{\pgfqpoint{-0.027778in}{0.000000in}}%
\pgfusepath{stroke,fill}%
}%
\begin{pgfscope}%
\pgfsys@transformshift{0.679669in}{3.230381in}%
\pgfsys@useobject{currentmarker}{}%
\end{pgfscope}%
\end{pgfscope}%
\begin{pgfscope}%
\pgfsetbuttcap%
\pgfsetroundjoin%
\definecolor{currentfill}{rgb}{0.000000,0.000000,0.000000}%
\pgfsetfillcolor{currentfill}%
\pgfsetlinewidth{0.602250pt}%
\definecolor{currentstroke}{rgb}{0.000000,0.000000,0.000000}%
\pgfsetstrokecolor{currentstroke}%
\pgfsetdash{}{0pt}%
\pgfsys@defobject{currentmarker}{\pgfqpoint{-0.027778in}{0.000000in}}{\pgfqpoint{0.000000in}{0.000000in}}{%
\pgfpathmoveto{\pgfqpoint{0.000000in}{0.000000in}}%
\pgfpathlineto{\pgfqpoint{-0.027778in}{0.000000in}}%
\pgfusepath{stroke,fill}%
}%
\begin{pgfscope}%
\pgfsys@transformshift{0.679669in}{3.297185in}%
\pgfsys@useobject{currentmarker}{}%
\end{pgfscope}%
\end{pgfscope}%
\begin{pgfscope}%
\pgfsetbuttcap%
\pgfsetroundjoin%
\definecolor{currentfill}{rgb}{0.000000,0.000000,0.000000}%
\pgfsetfillcolor{currentfill}%
\pgfsetlinewidth{0.602250pt}%
\definecolor{currentstroke}{rgb}{0.000000,0.000000,0.000000}%
\pgfsetstrokecolor{currentstroke}%
\pgfsetdash{}{0pt}%
\pgfsys@defobject{currentmarker}{\pgfqpoint{-0.027778in}{0.000000in}}{\pgfqpoint{0.000000in}{0.000000in}}{%
\pgfpathmoveto{\pgfqpoint{0.000000in}{0.000000in}}%
\pgfpathlineto{\pgfqpoint{-0.027778in}{0.000000in}}%
\pgfusepath{stroke,fill}%
}%
\begin{pgfscope}%
\pgfsys@transformshift{0.679669in}{3.356110in}%
\pgfsys@useobject{currentmarker}{}%
\end{pgfscope}%
\end{pgfscope}%
\begin{pgfscope}%
\pgftext[x=0.238889in,y=2.036079in,,bottom,rotate=90.000000]{\rmfamily\fontsize{10.000000}{12.000000}\selectfont \(\displaystyle E/E_0\)}%
\end{pgfscope}%
\begin{pgfscope}%
\pgfpathrectangle{\pgfqpoint{0.679669in}{0.526079in}}{\pgfqpoint{4.650000in}{3.020000in}} %
\pgfusepath{clip}%
\pgfsetrectcap%
\pgfsetroundjoin%
\pgfsetlinewidth{1.505625pt}%
\definecolor{currentstroke}{rgb}{0.000000,0.000000,0.000000}%
\pgfsetstrokecolor{currentstroke}%
\pgfsetdash{}{0pt}%
\pgfpathmoveto{\pgfqpoint{0.891033in}{3.408807in}}%
\pgfpathlineto{\pgfqpoint{1.147789in}{3.402642in}}%
\pgfpathlineto{\pgfqpoint{1.364998in}{3.395262in}}%
\pgfpathlineto{\pgfqpoint{1.553221in}{3.386620in}}%
\pgfpathlineto{\pgfqpoint{1.710528in}{3.377257in}}%
\pgfpathlineto{\pgfqpoint{1.853615in}{3.366548in}}%
\pgfpathlineto{\pgfqpoint{1.978573in}{3.355050in}}%
\pgfpathlineto{\pgfqpoint{2.094989in}{3.342118in}}%
\pgfpathlineto{\pgfqpoint{2.199312in}{3.328331in}}%
\pgfpathlineto{\pgfqpoint{2.297886in}{3.313026in}}%
\pgfpathlineto{\pgfqpoint{2.387745in}{3.296809in}}%
\pgfpathlineto{\pgfqpoint{2.470449in}{3.279667in}}%
\pgfpathlineto{\pgfqpoint{2.549852in}{3.260922in}}%
\pgfpathlineto{\pgfqpoint{2.623636in}{3.241226in}}%
\pgfpathlineto{\pgfqpoint{2.694811in}{3.219895in}}%
\pgfpathlineto{\pgfqpoint{2.761449in}{3.197620in}}%
\pgfpathlineto{\pgfqpoint{2.825970in}{3.173718in}}%
\pgfpathlineto{\pgfqpoint{2.888408in}{3.148203in}}%
\pgfpathlineto{\pgfqpoint{2.948822in}{3.121106in}}%
\pgfpathlineto{\pgfqpoint{3.007279in}{3.092471in}}%
\pgfpathlineto{\pgfqpoint{3.065147in}{3.061641in}}%
\pgfpathlineto{\pgfqpoint{3.122212in}{3.028694in}}%
\pgfpathlineto{\pgfqpoint{3.178313in}{2.993735in}}%
\pgfpathlineto{\pgfqpoint{3.234359in}{2.956183in}}%
\pgfpathlineto{\pgfqpoint{3.290046in}{2.916209in}}%
\pgfpathlineto{\pgfqpoint{3.346877in}{2.872639in}}%
\pgfpathlineto{\pgfqpoint{3.404280in}{2.825776in}}%
\pgfpathlineto{\pgfqpoint{3.462530in}{2.775309in}}%
\pgfpathlineto{\pgfqpoint{3.521768in}{2.721033in}}%
\pgfpathlineto{\pgfqpoint{3.583268in}{2.661631in}}%
\pgfpathlineto{\pgfqpoint{3.646668in}{2.597281in}}%
\pgfpathlineto{\pgfqpoint{3.713138in}{2.526615in}}%
\pgfpathlineto{\pgfqpoint{3.783282in}{2.448754in}}%
\pgfpathlineto{\pgfqpoint{3.858083in}{2.362341in}}%
\pgfpathlineto{\pgfqpoint{3.938564in}{2.265896in}}%
\pgfpathlineto{\pgfqpoint{4.026186in}{2.157350in}}%
\pgfpathlineto{\pgfqpoint{4.123122in}{2.033645in}}%
\pgfpathlineto{\pgfqpoint{4.232457in}{1.890416in}}%
\pgfpathlineto{\pgfqpoint{4.358782in}{1.721155in}}%
\pgfpathlineto{\pgfqpoint{4.509577in}{1.515251in}}%
\pgfpathlineto{\pgfqpoint{4.698525in}{1.253313in}}%
\pgfpathlineto{\pgfqpoint{4.953954in}{0.895157in}}%
\pgfpathlineto{\pgfqpoint{5.118306in}{0.663352in}}%
\pgfpathlineto{\pgfqpoint{5.118306in}{0.663352in}}%
\pgfusepath{stroke}%
\end{pgfscope}%
\begin{pgfscope}%
\pgfsetrectcap%
\pgfsetmiterjoin%
\pgfsetlinewidth{0.803000pt}%
\definecolor{currentstroke}{rgb}{0.000000,0.000000,0.000000}%
\pgfsetstrokecolor{currentstroke}%
\pgfsetdash{}{0pt}%
\pgfpathmoveto{\pgfqpoint{0.679669in}{0.526079in}}%
\pgfpathlineto{\pgfqpoint{0.679669in}{3.546079in}}%
\pgfusepath{stroke}%
\end{pgfscope}%
\begin{pgfscope}%
\pgfsetrectcap%
\pgfsetmiterjoin%
\pgfsetlinewidth{0.803000pt}%
\definecolor{currentstroke}{rgb}{0.000000,0.000000,0.000000}%
\pgfsetstrokecolor{currentstroke}%
\pgfsetdash{}{0pt}%
\pgfpathmoveto{\pgfqpoint{5.329669in}{0.526079in}}%
\pgfpathlineto{\pgfqpoint{5.329669in}{3.546079in}}%
\pgfusepath{stroke}%
\end{pgfscope}%
\begin{pgfscope}%
\pgfsetrectcap%
\pgfsetmiterjoin%
\pgfsetlinewidth{0.803000pt}%
\definecolor{currentstroke}{rgb}{0.000000,0.000000,0.000000}%
\pgfsetstrokecolor{currentstroke}%
\pgfsetdash{}{0pt}%
\pgfpathmoveto{\pgfqpoint{0.679669in}{0.526079in}}%
\pgfpathlineto{\pgfqpoint{5.329669in}{0.526079in}}%
\pgfusepath{stroke}%
\end{pgfscope}%
\begin{pgfscope}%
\pgfsetrectcap%
\pgfsetmiterjoin%
\pgfsetlinewidth{0.803000pt}%
\definecolor{currentstroke}{rgb}{0.000000,0.000000,0.000000}%
\pgfsetstrokecolor{currentstroke}%
\pgfsetdash{}{0pt}%
\pgfpathmoveto{\pgfqpoint{0.679669in}{3.546079in}}%
\pgfpathlineto{\pgfqpoint{5.329669in}{3.546079in}}%
\pgfusepath{stroke}%
\end{pgfscope}%
\end{pgfpicture}%
\makeatother%
\endgroup%

    \caption{A log-log plot of the magnitude of the dimensionless electric field $E$.\label{fig:E0}}
\end{figure}

\subsection{Scaling}
Equation \ref{gov_eqn_subs} is non-linear, non-homogeneous and must be solved numerically. Nevertheless we seek approximate solutions to the equation analytically using perturbation methods. Introducing the scaled variables
\begin{equation}
 t^* = \frac{t}{t_c}, \hspace{10 mm} y^* = \frac{y}{y_c}, \end{equation}
where $y_c$ and $t_c$ are characteristic length and time scales respectively, and using the coordinate transformation $y(0) - R = 0$, the governing equation becomes
\begin{equation}
\label{pi_terms}
 {y^*}'' = - \mathbf{\Pi}_1 {y^*}'^2
- \mathbf{\Pi}_2 E^* ( {y^*} ) 
- \mathbf{\Pi}_3 \left( \mathbf{\Pi}_4 {y^*} + 1 \right)^{-2},
\end{equation}
subject to
\begin{equation*}
{y^*}(0) = 0, \hspace{5 mm} \mbox{and} \hspace{5 mm} {y^*}'(0) = \mathbf{\Pi}_5,
\end{equation*}
where we note the existence of several dimensionless groups
\begin{gather*}
\mathbf{\Pi}_1 = \frac{C_D \rho A y_c}{2 m}, \hspace{5 mm}
\mathbf{\Pi}_2 = \frac{q E_0 t_c^2}{m y_c}, \hspace{5 mm}
\mathbf{\Pi}_3 = \frac{k q^2 t_c^2}{16 \pi \epsilon_0 R_d^2 m y_c}, \hspace{5 mm}\\
\mathbf{\Pi}_4 = \frac{y_c}{R_d}, \hspace{5 mm}
\mathbf{\Pi}_5 = \frac{U_0 t_c}{y_c}.
\end{gather*}

\subsubsection{Inertial Electro-Image Limit}
In the limit of small $y$ and $t$ we expect inertia to scale with Coulombic and image forces. In this limit we can approximate the electric field as the constant $E_0$. One possible characteristic length scale is $y_c \sim R_d$, however this scale is overly restrictive with respect to time. With $y_c \sim U_0 t_c$ and picking $t_c$ such that the Coulombic force $\mathbf{\Pi}_3 \sim \mathcal{O}(1)$, the intrinsic scales are found such that
\[ t_c \sim \frac{m U_0}{q E_0}, \hspace{5 mm} \mbox{and} \hspace{5 mm} 
y_c \sim \frac{m U_0^2}{q E_0} .
\]
With these scales Equation \ref{pi_terms} becomes
\begin{equation}
{y^*}'' = -1 - \mathbb{I}\mbox{g} \left( \mathbb{E}\mbox{u}{y^*} + 1 \right)^{-2} , \label{img_limit}
\end{equation}
subject to
\begin{equation*}
{y^*}(0) = 0, \hspace{5 mm} \mbox{and} \hspace{5 mm} {y^*}'(0) = 1 .
\end{equation*}
with 
\[ \mathbb{I}\mbox{g} \equiv \frac{k q}{16 \pi \epsilon_0 R_d^2 E_0} = \mathbf{\Pi}_3, \hspace{5 mm}
\mathbb{E}\mbox{u} \equiv \frac{m U_0^2}{q E_0 R_d} = \mathbf{\Pi}_4 ,
\]
where the Image number $\mathbb{I}\mbox{g}$ is the ratio of image forces to the Coulombic force of the unperturbed field, and the electrostatic Euler number $\mathbb{E}\mbox{u}$ is the ratio of inertia to electrostatic force, or equivalently, in a conservative system $\mathbb{E}\mbox{u}$ can be thought of as a ratio of kinetic energy $m U_0^2$ and electrostatic potential energy $q E_0 R_d$.

\subsubsection{Inertial Electro-Viscous Limit}
In the limit of large $y$ and $t$ we expect drop inertia to balance Coulombic force and drag. Here we approximate the electric field as $E \approx y_c^2 E_0 y^{-2}$. We choose the scaling $y_c \sim U_0 t_c$ and $\mathbf{\Pi}_3 \sim \mathcal{O}(1)$ for its combination of physical simplicity, few $\mathbf{\Pi}$ terms, and homogeneous initial conditions. The intrinsic scales for this case are given by
\[ t_c \sim \frac{R_d^2}{L^2} \frac{4 \pi m U_0}{q E_0} \hspace{5mm} \mbox{and} \hspace{5 mm} y_c \sim \frac{R_d^2}{L^2} \frac{4 \pi m U_0^2}{q E_0}.
\]
With these scales Equation \ref{pi_terms} becomes 
\begin{eqnarray}
&{y^*}'' = - \mathbb{D}\mbox{g} \phi \mathbb{E}\mbox{u} {y^*}'^2 - \left( \phi \mathbb{E}\mbox{u} {y^*} + 1 \right)^{-2}, & \label{drag_limit}
\end{eqnarray}
subject to 
\begin{equation*}
{y^*}(0) = 0, \hspace{5 mm} \mbox{and} \hspace{5 mm} {y^*}'(0) = 1, 
\end{equation*}
where we call $\mathbb{D}\mbox{g}$ the drag number $\mathbb{D}\mbox{g} \equiv \frac{C_D \rho_a}{\rho_l} = \mathbf{\Pi}_1 \phi^{-1}{\mathbb{E}\mbox{u}}^{-1}$, and $\phi = 4 \pi \frac{R_d^2}{L^2}$ is a drop-to-substrate aspect ratio.

\subsection{Asymptotic Estimates}
Equations \ref{img_limit} and \ref{drag_limit} are weakly non-linear differential equations in the sense that they reduce to linear equations as $\mathbb{E}\mbox{u} \rightarrow 0$. For $\mathbb{E}\mbox{u} \ll 1$ we find asymptotic solutions of the non-linear equations by means of regular perturbations. In this case we use the naive expansion
\begin{equation}
{y^*}({t^*}) \sim y^*_0({t^*}) + \mathbb{E}\mbox{u} y^*_1({t^*}) + \mathbb{E}\mbox{u}^2 y^*_2({t^*}) \ldots \mathbb{E}\mbox{u}^n y^*_n ({t^*})  . \label{regular_pert}
\end{equation}

\subsubsection{Inertial Electro-Image Limit}
Substituting Equation \ref{regular_pert} and its derivatives into \ref{pi_terms}, and equating terms by order we find the $\mathcal{O}(1)$ unperturbed solution
\[{y^*_{0}}{\left ({t^*} \right )} = {t^*} + \frac{{t^*}^{2}}{2} \left(-1 - \mathbb{I}\mbox{g}\right). \]
By inspection, it is evident that if $\mathbb{I}\mbox{g}=0$, the solution is the classical kinematic equation for projectile motion without drag under constant gravity $g_0$. Continuing on to higher order, after some tedious computations documented in the project repository for this work \cite{schmidt_droplet_electro-bounce:_2017}, we find the $\mathcal{O}(\mathbb{E}\mbox{u}^5)$ order accurate solution which is truncated to $\mathcal{O}(\mathbb{E}\mbox{u}^2)$ below:

\begin{gather}
{y^*}({t^*}) = {t^*} + \frac{{t^*}^{2}}{2} \left(-1 - \mathbb{I}\mbox{g}\right)
 + \mathbb{E}\mbox{u} \left(\frac{\mathbb{I}\mbox{g} {t^*}^{3}}{3} + \frac{\mathbb{I}\mbox{g} {t^*}^{4}}{12} \left(-1 - \mathbb{I}\mbox{g} \right)\right) \nonumber \\
 + \mathcal{O}(\mathbb{E}\mbox{u}^2). \label{perturb_image}
\end{gather}

We plot the approximate short-time solution of Equation \ref{perturb_image} with varying values of $\mathbb{I}\mbox{g}$ in Figure \ref{fig:short_times}. These plots show a trend of decreasing time-of-flight $t_f$, which is the time for the drop to return to the origin (a single `bounce'), and height at apoapse with increasing values of $\mathbb{I}\mbox{g}$. \sout{When $\mathbb{I}\mbox{g} = 1$, $t_f$ is exactly half of the characteristic time scale in this regime. In the limit of small $\mathbb{I}\mbox{m}$ the trajectories collapse to the $\mathcal{O}(1)$ solution regardless of the electrostatic Euler number.} Trajectories with $\mathbb{E}\mbox{u} \leq 0.1$ are essentially coincident given the scale of the axes used here. In principle there is some coupling between $\mathbb{E}\mbox{u}$ and $\mathbb{I}\mbox{g}$; notably this relationship does not depend on the electric field $E_0$ but on a charge to mass ratio. The effect of contact line hysteresis on the initial jump velocity $U_0$ will also tend to decohere the natural covariance between these parameters.

\begin{figure*}[htp]
    \resizebox{0.8\textwidth}{!}{%% Creator: Matplotlib, PGF backend
%%
%% To include the figure in your LaTeX document, write
%%   \input{<filename>.pgf}
%%
%% Make sure the required packages are loaded in your preamble
%%   \usepackage{pgf}
%%
%% Figures using additional raster images can only be included by \input if
%% they are in the same directory as the main LaTeX file. For loading figures
%% from other directories you can use the `import` package
%%   \usepackage{import}
%% and then include the figures with
%%   \import{<path to file>}{<filename>.pgf}
%%
%% Matplotlib used the following preamble
%%   \usepackage{fontspec}
%%   \setmainfont{DejaVu Serif}
%%   \setsansfont{DejaVu Sans}
%%   \setmonofont{DejaVu Sans Mono}
%%
\begingroup%
\makeatletter%
\begin{pgfpicture}%
\pgfpathrectangle{\pgfpointorigin}{\pgfqpoint{10.267664in}{8.135326in}}%
\pgfusepath{use as bounding box, clip}%
\begin{pgfscope}%
\pgfsetbuttcap%
\pgfsetmiterjoin%
\definecolor{currentfill}{rgb}{1.000000,1.000000,1.000000}%
\pgfsetfillcolor{currentfill}%
\pgfsetlinewidth{0.000000pt}%
\definecolor{currentstroke}{rgb}{1.000000,1.000000,1.000000}%
\pgfsetstrokecolor{currentstroke}%
\pgfsetdash{}{0pt}%
\pgfpathmoveto{\pgfqpoint{0.000000in}{0.000000in}}%
\pgfpathlineto{\pgfqpoint{10.267664in}{0.000000in}}%
\pgfpathlineto{\pgfqpoint{10.267664in}{8.135326in}}%
\pgfpathlineto{\pgfqpoint{0.000000in}{8.135326in}}%
\pgfpathclose%
\pgfusepath{fill}%
\end{pgfscope}%
\begin{pgfscope}%
\pgfsetbuttcap%
\pgfsetmiterjoin%
\definecolor{currentfill}{rgb}{1.000000,1.000000,1.000000}%
\pgfsetfillcolor{currentfill}%
\pgfsetlinewidth{0.000000pt}%
\definecolor{currentstroke}{rgb}{0.000000,0.000000,0.000000}%
\pgfsetstrokecolor{currentstroke}%
\pgfsetstrokeopacity{0.000000}%
\pgfsetdash{}{0pt}%
\pgfpathmoveto{\pgfqpoint{0.984216in}{4.549747in}}%
\pgfpathlineto{\pgfqpoint{5.442272in}{4.549747in}}%
\pgfpathlineto{\pgfqpoint{5.442272in}{7.950908in}}%
\pgfpathlineto{\pgfqpoint{0.984216in}{7.950908in}}%
\pgfpathclose%
\pgfusepath{fill}%
\end{pgfscope}%
\begin{pgfscope}%
\pgfsetbuttcap%
\pgfsetroundjoin%
\definecolor{currentfill}{rgb}{0.000000,0.000000,0.000000}%
\pgfsetfillcolor{currentfill}%
\pgfsetlinewidth{0.803000pt}%
\definecolor{currentstroke}{rgb}{0.000000,0.000000,0.000000}%
\pgfsetstrokecolor{currentstroke}%
\pgfsetdash{}{0pt}%
\pgfsys@defobject{currentmarker}{\pgfqpoint{0.000000in}{-0.048611in}}{\pgfqpoint{0.000000in}{0.000000in}}{%
\pgfpathmoveto{\pgfqpoint{0.000000in}{0.000000in}}%
\pgfpathlineto{\pgfqpoint{0.000000in}{-0.048611in}}%
\pgfusepath{stroke,fill}%
}%
\begin{pgfscope}%
\pgfsys@transformshift{1.225002in}{4.549747in}%
\pgfsys@useobject{currentmarker}{}%
\end{pgfscope}%
\end{pgfscope}%
\begin{pgfscope}%
\pgfsetbuttcap%
\pgfsetroundjoin%
\definecolor{currentfill}{rgb}{0.000000,0.000000,0.000000}%
\pgfsetfillcolor{currentfill}%
\pgfsetlinewidth{0.803000pt}%
\definecolor{currentstroke}{rgb}{0.000000,0.000000,0.000000}%
\pgfsetstrokecolor{currentstroke}%
\pgfsetdash{}{0pt}%
\pgfsys@defobject{currentmarker}{\pgfqpoint{0.000000in}{-0.048611in}}{\pgfqpoint{0.000000in}{0.000000in}}{%
\pgfpathmoveto{\pgfqpoint{0.000000in}{0.000000in}}%
\pgfpathlineto{\pgfqpoint{0.000000in}{-0.048611in}}%
\pgfusepath{stroke,fill}%
}%
\begin{pgfscope}%
\pgfsys@transformshift{2.028291in}{4.549747in}%
\pgfsys@useobject{currentmarker}{}%
\end{pgfscope}%
\end{pgfscope}%
\begin{pgfscope}%
\pgfsetbuttcap%
\pgfsetroundjoin%
\definecolor{currentfill}{rgb}{0.000000,0.000000,0.000000}%
\pgfsetfillcolor{currentfill}%
\pgfsetlinewidth{0.803000pt}%
\definecolor{currentstroke}{rgb}{0.000000,0.000000,0.000000}%
\pgfsetstrokecolor{currentstroke}%
\pgfsetdash{}{0pt}%
\pgfsys@defobject{currentmarker}{\pgfqpoint{0.000000in}{-0.048611in}}{\pgfqpoint{0.000000in}{0.000000in}}{%
\pgfpathmoveto{\pgfqpoint{0.000000in}{0.000000in}}%
\pgfpathlineto{\pgfqpoint{0.000000in}{-0.048611in}}%
\pgfusepath{stroke,fill}%
}%
\begin{pgfscope}%
\pgfsys@transformshift{2.831581in}{4.549747in}%
\pgfsys@useobject{currentmarker}{}%
\end{pgfscope}%
\end{pgfscope}%
\begin{pgfscope}%
\pgfsetbuttcap%
\pgfsetroundjoin%
\definecolor{currentfill}{rgb}{0.000000,0.000000,0.000000}%
\pgfsetfillcolor{currentfill}%
\pgfsetlinewidth{0.803000pt}%
\definecolor{currentstroke}{rgb}{0.000000,0.000000,0.000000}%
\pgfsetstrokecolor{currentstroke}%
\pgfsetdash{}{0pt}%
\pgfsys@defobject{currentmarker}{\pgfqpoint{0.000000in}{-0.048611in}}{\pgfqpoint{0.000000in}{0.000000in}}{%
\pgfpathmoveto{\pgfqpoint{0.000000in}{0.000000in}}%
\pgfpathlineto{\pgfqpoint{0.000000in}{-0.048611in}}%
\pgfusepath{stroke,fill}%
}%
\begin{pgfscope}%
\pgfsys@transformshift{3.634870in}{4.549747in}%
\pgfsys@useobject{currentmarker}{}%
\end{pgfscope}%
\end{pgfscope}%
\begin{pgfscope}%
\pgfsetbuttcap%
\pgfsetroundjoin%
\definecolor{currentfill}{rgb}{0.000000,0.000000,0.000000}%
\pgfsetfillcolor{currentfill}%
\pgfsetlinewidth{0.803000pt}%
\definecolor{currentstroke}{rgb}{0.000000,0.000000,0.000000}%
\pgfsetstrokecolor{currentstroke}%
\pgfsetdash{}{0pt}%
\pgfsys@defobject{currentmarker}{\pgfqpoint{0.000000in}{-0.048611in}}{\pgfqpoint{0.000000in}{0.000000in}}{%
\pgfpathmoveto{\pgfqpoint{0.000000in}{0.000000in}}%
\pgfpathlineto{\pgfqpoint{0.000000in}{-0.048611in}}%
\pgfusepath{stroke,fill}%
}%
\begin{pgfscope}%
\pgfsys@transformshift{4.438160in}{4.549747in}%
\pgfsys@useobject{currentmarker}{}%
\end{pgfscope}%
\end{pgfscope}%
\begin{pgfscope}%
\pgfsetbuttcap%
\pgfsetroundjoin%
\definecolor{currentfill}{rgb}{0.000000,0.000000,0.000000}%
\pgfsetfillcolor{currentfill}%
\pgfsetlinewidth{0.803000pt}%
\definecolor{currentstroke}{rgb}{0.000000,0.000000,0.000000}%
\pgfsetstrokecolor{currentstroke}%
\pgfsetdash{}{0pt}%
\pgfsys@defobject{currentmarker}{\pgfqpoint{0.000000in}{-0.048611in}}{\pgfqpoint{0.000000in}{0.000000in}}{%
\pgfpathmoveto{\pgfqpoint{0.000000in}{0.000000in}}%
\pgfpathlineto{\pgfqpoint{0.000000in}{-0.048611in}}%
\pgfusepath{stroke,fill}%
}%
\begin{pgfscope}%
\pgfsys@transformshift{5.241450in}{4.549747in}%
\pgfsys@useobject{currentmarker}{}%
\end{pgfscope}%
\end{pgfscope}%
\begin{pgfscope}%
\pgfsetbuttcap%
\pgfsetroundjoin%
\definecolor{currentfill}{rgb}{0.000000,0.000000,0.000000}%
\pgfsetfillcolor{currentfill}%
\pgfsetlinewidth{0.803000pt}%
\definecolor{currentstroke}{rgb}{0.000000,0.000000,0.000000}%
\pgfsetstrokecolor{currentstroke}%
\pgfsetdash{}{0pt}%
\pgfsys@defobject{currentmarker}{\pgfqpoint{-0.048611in}{0.000000in}}{\pgfqpoint{0.000000in}{0.000000in}}{%
\pgfpathmoveto{\pgfqpoint{0.000000in}{0.000000in}}%
\pgfpathlineto{\pgfqpoint{-0.048611in}{0.000000in}}%
\pgfusepath{stroke,fill}%
}%
\begin{pgfscope}%
\pgfsys@transformshift{0.984216in}{4.549747in}%
\pgfsys@useobject{currentmarker}{}%
\end{pgfscope}%
\end{pgfscope}%
\begin{pgfscope}%
\pgftext[x=0.601580in,y=4.465329in,left,base]{\rmfamily\fontsize{16.000000}{19.200000}\selectfont \(\displaystyle 0.0\)}%
\end{pgfscope}%
\begin{pgfscope}%
\pgfsetbuttcap%
\pgfsetroundjoin%
\definecolor{currentfill}{rgb}{0.000000,0.000000,0.000000}%
\pgfsetfillcolor{currentfill}%
\pgfsetlinewidth{0.803000pt}%
\definecolor{currentstroke}{rgb}{0.000000,0.000000,0.000000}%
\pgfsetstrokecolor{currentstroke}%
\pgfsetdash{}{0pt}%
\pgfsys@defobject{currentmarker}{\pgfqpoint{-0.048611in}{0.000000in}}{\pgfqpoint{0.000000in}{0.000000in}}{%
\pgfpathmoveto{\pgfqpoint{0.000000in}{0.000000in}}%
\pgfpathlineto{\pgfqpoint{-0.048611in}{0.000000in}}%
\pgfusepath{stroke,fill}%
}%
\begin{pgfscope}%
\pgfsys@transformshift{0.984216in}{5.229979in}%
\pgfsys@useobject{currentmarker}{}%
\end{pgfscope}%
\end{pgfscope}%
\begin{pgfscope}%
\pgftext[x=0.601580in,y=5.145561in,left,base]{\rmfamily\fontsize{16.000000}{19.200000}\selectfont \(\displaystyle 0.1\)}%
\end{pgfscope}%
\begin{pgfscope}%
\pgfsetbuttcap%
\pgfsetroundjoin%
\definecolor{currentfill}{rgb}{0.000000,0.000000,0.000000}%
\pgfsetfillcolor{currentfill}%
\pgfsetlinewidth{0.803000pt}%
\definecolor{currentstroke}{rgb}{0.000000,0.000000,0.000000}%
\pgfsetstrokecolor{currentstroke}%
\pgfsetdash{}{0pt}%
\pgfsys@defobject{currentmarker}{\pgfqpoint{-0.048611in}{0.000000in}}{\pgfqpoint{0.000000in}{0.000000in}}{%
\pgfpathmoveto{\pgfqpoint{0.000000in}{0.000000in}}%
\pgfpathlineto{\pgfqpoint{-0.048611in}{0.000000in}}%
\pgfusepath{stroke,fill}%
}%
\begin{pgfscope}%
\pgfsys@transformshift{0.984216in}{5.910211in}%
\pgfsys@useobject{currentmarker}{}%
\end{pgfscope}%
\end{pgfscope}%
\begin{pgfscope}%
\pgftext[x=0.601580in,y=5.825793in,left,base]{\rmfamily\fontsize{16.000000}{19.200000}\selectfont \(\displaystyle 0.2\)}%
\end{pgfscope}%
\begin{pgfscope}%
\pgfsetbuttcap%
\pgfsetroundjoin%
\definecolor{currentfill}{rgb}{0.000000,0.000000,0.000000}%
\pgfsetfillcolor{currentfill}%
\pgfsetlinewidth{0.803000pt}%
\definecolor{currentstroke}{rgb}{0.000000,0.000000,0.000000}%
\pgfsetstrokecolor{currentstroke}%
\pgfsetdash{}{0pt}%
\pgfsys@defobject{currentmarker}{\pgfqpoint{-0.048611in}{0.000000in}}{\pgfqpoint{0.000000in}{0.000000in}}{%
\pgfpathmoveto{\pgfqpoint{0.000000in}{0.000000in}}%
\pgfpathlineto{\pgfqpoint{-0.048611in}{0.000000in}}%
\pgfusepath{stroke,fill}%
}%
\begin{pgfscope}%
\pgfsys@transformshift{0.984216in}{6.590443in}%
\pgfsys@useobject{currentmarker}{}%
\end{pgfscope}%
\end{pgfscope}%
\begin{pgfscope}%
\pgftext[x=0.601580in,y=6.506025in,left,base]{\rmfamily\fontsize{16.000000}{19.200000}\selectfont \(\displaystyle 0.3\)}%
\end{pgfscope}%
\begin{pgfscope}%
\pgfsetbuttcap%
\pgfsetroundjoin%
\definecolor{currentfill}{rgb}{0.000000,0.000000,0.000000}%
\pgfsetfillcolor{currentfill}%
\pgfsetlinewidth{0.803000pt}%
\definecolor{currentstroke}{rgb}{0.000000,0.000000,0.000000}%
\pgfsetstrokecolor{currentstroke}%
\pgfsetdash{}{0pt}%
\pgfsys@defobject{currentmarker}{\pgfqpoint{-0.048611in}{0.000000in}}{\pgfqpoint{0.000000in}{0.000000in}}{%
\pgfpathmoveto{\pgfqpoint{0.000000in}{0.000000in}}%
\pgfpathlineto{\pgfqpoint{-0.048611in}{0.000000in}}%
\pgfusepath{stroke,fill}%
}%
\begin{pgfscope}%
\pgfsys@transformshift{0.984216in}{7.270676in}%
\pgfsys@useobject{currentmarker}{}%
\end{pgfscope}%
\end{pgfscope}%
\begin{pgfscope}%
\pgftext[x=0.601580in,y=7.186257in,left,base]{\rmfamily\fontsize{16.000000}{19.200000}\selectfont \(\displaystyle 0.4\)}%
\end{pgfscope}%
\begin{pgfscope}%
\pgfsetbuttcap%
\pgfsetroundjoin%
\definecolor{currentfill}{rgb}{0.000000,0.000000,0.000000}%
\pgfsetfillcolor{currentfill}%
\pgfsetlinewidth{0.803000pt}%
\definecolor{currentstroke}{rgb}{0.000000,0.000000,0.000000}%
\pgfsetstrokecolor{currentstroke}%
\pgfsetdash{}{0pt}%
\pgfsys@defobject{currentmarker}{\pgfqpoint{-0.048611in}{0.000000in}}{\pgfqpoint{0.000000in}{0.000000in}}{%
\pgfpathmoveto{\pgfqpoint{0.000000in}{0.000000in}}%
\pgfpathlineto{\pgfqpoint{-0.048611in}{0.000000in}}%
\pgfusepath{stroke,fill}%
}%
\begin{pgfscope}%
\pgfsys@transformshift{0.984216in}{7.950908in}%
\pgfsys@useobject{currentmarker}{}%
\end{pgfscope}%
\end{pgfscope}%
\begin{pgfscope}%
\pgftext[x=0.601580in,y=7.866489in,left,base]{\rmfamily\fontsize{16.000000}{19.200000}\selectfont \(\displaystyle 0.5\)}%
\end{pgfscope}%
\begin{pgfscope}%
\pgfpathrectangle{\pgfqpoint{0.984216in}{4.549747in}}{\pgfqpoint{4.458056in}{3.401160in}} %
\pgfusepath{clip}%
\pgfsetbuttcap%
\pgfsetroundjoin%
\pgfsetlinewidth{1.505625pt}%
\definecolor{currentstroke}{rgb}{1.000000,0.000000,0.000000}%
\pgfsetstrokecolor{currentstroke}%
\pgfsetdash{{5.550000pt}{2.400000pt}}{0.000000pt}%
\pgfpathmoveto{\pgfqpoint{1.225002in}{4.549747in}}%
\pgfpathlineto{\pgfqpoint{1.257133in}{4.603515in}}%
\pgfpathlineto{\pgfqpoint{1.289265in}{4.655991in}}%
\pgfpathlineto{\pgfqpoint{1.321396in}{4.707186in}}%
\pgfpathlineto{\pgfqpoint{1.353528in}{4.757115in}}%
\pgfpathlineto{\pgfqpoint{1.385660in}{4.805788in}}%
\pgfpathlineto{\pgfqpoint{1.417791in}{4.853218in}}%
\pgfpathlineto{\pgfqpoint{1.449923in}{4.899414in}}%
\pgfpathlineto{\pgfqpoint{1.482054in}{4.944387in}}%
\pgfpathlineto{\pgfqpoint{1.514186in}{4.988146in}}%
\pgfpathlineto{\pgfqpoint{1.546317in}{5.030702in}}%
\pgfpathlineto{\pgfqpoint{1.578449in}{5.072063in}}%
\pgfpathlineto{\pgfqpoint{1.610581in}{5.112238in}}%
\pgfpathlineto{\pgfqpoint{1.642712in}{5.151234in}}%
\pgfpathlineto{\pgfqpoint{1.674844in}{5.189060in}}%
\pgfpathlineto{\pgfqpoint{1.706975in}{5.225724in}}%
\pgfpathlineto{\pgfqpoint{1.739107in}{5.261231in}}%
\pgfpathlineto{\pgfqpoint{1.771239in}{5.295589in}}%
\pgfpathlineto{\pgfqpoint{1.803370in}{5.328805in}}%
\pgfpathlineto{\pgfqpoint{1.835502in}{5.360885in}}%
\pgfpathlineto{\pgfqpoint{1.867633in}{5.391834in}}%
\pgfpathlineto{\pgfqpoint{1.899765in}{5.421658in}}%
\pgfpathlineto{\pgfqpoint{1.931896in}{5.450363in}}%
\pgfpathlineto{\pgfqpoint{1.964028in}{5.477953in}}%
\pgfpathlineto{\pgfqpoint{1.992143in}{5.501185in}}%
\pgfpathlineto{\pgfqpoint{2.020258in}{5.523570in}}%
\pgfpathlineto{\pgfqpoint{2.048373in}{5.545111in}}%
\pgfpathlineto{\pgfqpoint{2.076489in}{5.565812in}}%
\pgfpathlineto{\pgfqpoint{2.104604in}{5.585675in}}%
\pgfpathlineto{\pgfqpoint{2.132719in}{5.604703in}}%
\pgfpathlineto{\pgfqpoint{2.160834in}{5.622897in}}%
\pgfpathlineto{\pgfqpoint{2.188949in}{5.640262in}}%
\pgfpathlineto{\pgfqpoint{2.217064in}{5.656798in}}%
\pgfpathlineto{\pgfqpoint{2.245179in}{5.672507in}}%
\pgfpathlineto{\pgfqpoint{2.273295in}{5.687393in}}%
\pgfpathlineto{\pgfqpoint{2.301410in}{5.701456in}}%
\pgfpathlineto{\pgfqpoint{2.329525in}{5.714698in}}%
\pgfpathlineto{\pgfqpoint{2.357640in}{5.727122in}}%
\pgfpathlineto{\pgfqpoint{2.385755in}{5.738728in}}%
\pgfpathlineto{\pgfqpoint{2.413870in}{5.749518in}}%
\pgfpathlineto{\pgfqpoint{2.441985in}{5.759494in}}%
\pgfpathlineto{\pgfqpoint{2.470101in}{5.768656in}}%
\pgfpathlineto{\pgfqpoint{2.498216in}{5.777006in}}%
\pgfpathlineto{\pgfqpoint{2.526331in}{5.784545in}}%
\pgfpathlineto{\pgfqpoint{2.554446in}{5.791273in}}%
\pgfpathlineto{\pgfqpoint{2.582561in}{5.797192in}}%
\pgfpathlineto{\pgfqpoint{2.610676in}{5.802303in}}%
\pgfpathlineto{\pgfqpoint{2.638791in}{5.806605in}}%
\pgfpathlineto{\pgfqpoint{2.666906in}{5.810100in}}%
\pgfpathlineto{\pgfqpoint{2.695022in}{5.812788in}}%
\pgfpathlineto{\pgfqpoint{2.723137in}{5.814670in}}%
\pgfpathlineto{\pgfqpoint{2.751252in}{5.815744in}}%
\pgfpathlineto{\pgfqpoint{2.779367in}{5.816013in}}%
\pgfpathlineto{\pgfqpoint{2.807482in}{5.815475in}}%
\pgfpathlineto{\pgfqpoint{2.835597in}{5.814130in}}%
\pgfpathlineto{\pgfqpoint{2.863712in}{5.811979in}}%
\pgfpathlineto{\pgfqpoint{2.891828in}{5.809021in}}%
\pgfpathlineto{\pgfqpoint{2.919943in}{5.805256in}}%
\pgfpathlineto{\pgfqpoint{2.948058in}{5.800683in}}%
\pgfpathlineto{\pgfqpoint{2.976173in}{5.795301in}}%
\pgfpathlineto{\pgfqpoint{3.004288in}{5.789110in}}%
\pgfpathlineto{\pgfqpoint{3.032403in}{5.782109in}}%
\pgfpathlineto{\pgfqpoint{3.060518in}{5.774297in}}%
\pgfpathlineto{\pgfqpoint{3.088633in}{5.765673in}}%
\pgfpathlineto{\pgfqpoint{3.116749in}{5.756235in}}%
\pgfpathlineto{\pgfqpoint{3.144864in}{5.745983in}}%
\pgfpathlineto{\pgfqpoint{3.172979in}{5.734914in}}%
\pgfpathlineto{\pgfqpoint{3.201094in}{5.723028in}}%
\pgfpathlineto{\pgfqpoint{3.229209in}{5.710322in}}%
\pgfpathlineto{\pgfqpoint{3.257324in}{5.696796in}}%
\pgfpathlineto{\pgfqpoint{3.285439in}{5.682446in}}%
\pgfpathlineto{\pgfqpoint{3.313555in}{5.667271in}}%
\pgfpathlineto{\pgfqpoint{3.341670in}{5.651268in}}%
\pgfpathlineto{\pgfqpoint{3.369785in}{5.634436in}}%
\pgfpathlineto{\pgfqpoint{3.397900in}{5.616772in}}%
\pgfpathlineto{\pgfqpoint{3.426015in}{5.598273in}}%
\pgfpathlineto{\pgfqpoint{3.454130in}{5.578937in}}%
\pgfpathlineto{\pgfqpoint{3.482245in}{5.558761in}}%
\pgfpathlineto{\pgfqpoint{3.510361in}{5.537742in}}%
\pgfpathlineto{\pgfqpoint{3.538476in}{5.515876in}}%
\pgfpathlineto{\pgfqpoint{3.566591in}{5.493162in}}%
\pgfpathlineto{\pgfqpoint{3.594706in}{5.469594in}}%
\pgfpathlineto{\pgfqpoint{3.622821in}{5.445171in}}%
\pgfpathlineto{\pgfqpoint{3.654953in}{5.416206in}}%
\pgfpathlineto{\pgfqpoint{3.687084in}{5.386113in}}%
\pgfpathlineto{\pgfqpoint{3.719216in}{5.354886in}}%
\pgfpathlineto{\pgfqpoint{3.751347in}{5.322519in}}%
\pgfpathlineto{\pgfqpoint{3.783479in}{5.289006in}}%
\pgfpathlineto{\pgfqpoint{3.815611in}{5.254340in}}%
\pgfpathlineto{\pgfqpoint{3.847742in}{5.218516in}}%
\pgfpathlineto{\pgfqpoint{3.879874in}{5.181527in}}%
\pgfpathlineto{\pgfqpoint{3.912005in}{5.143364in}}%
\pgfpathlineto{\pgfqpoint{3.944137in}{5.104022in}}%
\pgfpathlineto{\pgfqpoint{3.976268in}{5.063492in}}%
\pgfpathlineto{\pgfqpoint{4.008400in}{5.021768in}}%
\pgfpathlineto{\pgfqpoint{4.040532in}{4.978841in}}%
\pgfpathlineto{\pgfqpoint{4.072663in}{4.934703in}}%
\pgfpathlineto{\pgfqpoint{4.104795in}{4.889345in}}%
\pgfpathlineto{\pgfqpoint{4.136926in}{4.842760in}}%
\pgfpathlineto{\pgfqpoint{4.169058in}{4.794937in}}%
\pgfpathlineto{\pgfqpoint{4.201190in}{4.745868in}}%
\pgfpathlineto{\pgfqpoint{4.233321in}{4.695543in}}%
\pgfpathlineto{\pgfqpoint{4.265453in}{4.643952in}}%
\pgfpathlineto{\pgfqpoint{4.297584in}{4.591085in}}%
\pgfpathlineto{\pgfqpoint{4.324142in}{4.546414in}}%
\pgfpathlineto{\pgfqpoint{4.324142in}{4.546414in}}%
\pgfusepath{stroke}%
\end{pgfscope}%
\begin{pgfscope}%
\pgfpathrectangle{\pgfqpoint{0.984216in}{4.549747in}}{\pgfqpoint{4.458056in}{3.401160in}} %
\pgfusepath{clip}%
\pgfsetbuttcap%
\pgfsetmiterjoin%
\definecolor{currentfill}{rgb}{1.000000,0.000000,0.000000}%
\pgfsetfillcolor{currentfill}%
\pgfsetlinewidth{1.003750pt}%
\definecolor{currentstroke}{rgb}{1.000000,0.000000,0.000000}%
\pgfsetstrokecolor{currentstroke}%
\pgfsetdash{}{0pt}%
\pgfsys@defobject{currentmarker}{\pgfqpoint{-0.041667in}{-0.041667in}}{\pgfqpoint{0.041667in}{0.041667in}}{%
\pgfpathmoveto{\pgfqpoint{-0.041667in}{-0.041667in}}%
\pgfpathlineto{\pgfqpoint{0.041667in}{-0.041667in}}%
\pgfpathlineto{\pgfqpoint{0.041667in}{0.041667in}}%
\pgfpathlineto{\pgfqpoint{-0.041667in}{0.041667in}}%
\pgfpathclose%
\pgfusepath{stroke,fill}%
}%
\begin{pgfscope}%
\pgfsys@transformshift{1.225002in}{4.549747in}%
\pgfsys@useobject{currentmarker}{}%
\end{pgfscope}%
\begin{pgfscope}%
\pgfsys@transformshift{1.626646in}{5.131883in}%
\pgfsys@useobject{currentmarker}{}%
\end{pgfscope}%
\begin{pgfscope}%
\pgfsys@transformshift{2.028291in}{5.529810in}%
\pgfsys@useobject{currentmarker}{}%
\end{pgfscope}%
\begin{pgfscope}%
\pgfsys@transformshift{2.429936in}{5.755318in}%
\pgfsys@useobject{currentmarker}{}%
\end{pgfscope}%
\begin{pgfscope}%
\pgfsys@transformshift{2.831581in}{5.814372in}%
\pgfsys@useobject{currentmarker}{}%
\end{pgfscope}%
\begin{pgfscope}%
\pgfsys@transformshift{3.233226in}{5.708440in}%
\pgfsys@useobject{currentmarker}{}%
\end{pgfscope}%
\begin{pgfscope}%
\pgfsys@transformshift{3.634870in}{5.434441in}%
\pgfsys@useobject{currentmarker}{}%
\end{pgfscope}%
\begin{pgfscope}%
\pgfsys@transformshift{4.036515in}{4.984273in}%
\pgfsys@useobject{currentmarker}{}%
\end{pgfscope}%
\begin{pgfscope}%
\pgfsys@transformshift{4.438160in}{4.344498in}%
\pgfsys@useobject{currentmarker}{}%
\end{pgfscope}%
\begin{pgfscope}%
\pgfsys@transformshift{4.839805in}{3.493435in}%
\pgfsys@useobject{currentmarker}{}%
\end{pgfscope}%
\begin{pgfscope}%
\pgfsys@transformshift{5.241450in}{2.384192in}%
\pgfsys@useobject{currentmarker}{}%
\end{pgfscope}%
\begin{pgfscope}%
\pgfsys@transformshift{5.643094in}{0.885517in}%
\pgfsys@useobject{currentmarker}{}%
\end{pgfscope}%
\end{pgfscope}%
\begin{pgfscope}%
\pgfpathrectangle{\pgfqpoint{0.984216in}{4.549747in}}{\pgfqpoint{4.458056in}{3.401160in}} %
\pgfusepath{clip}%
\pgfsetrectcap%
\pgfsetroundjoin%
\pgfsetlinewidth{1.505625pt}%
\definecolor{currentstroke}{rgb}{0.000000,0.000000,1.000000}%
\pgfsetstrokecolor{currentstroke}%
\pgfsetdash{}{0pt}%
\pgfpathmoveto{\pgfqpoint{1.225002in}{4.549747in}}%
\pgfpathlineto{\pgfqpoint{1.257133in}{4.603513in}}%
\pgfpathlineto{\pgfqpoint{1.289265in}{4.655974in}}%
\pgfpathlineto{\pgfqpoint{1.321396in}{4.707132in}}%
\pgfpathlineto{\pgfqpoint{1.353528in}{4.756988in}}%
\pgfpathlineto{\pgfqpoint{1.385660in}{4.805543in}}%
\pgfpathlineto{\pgfqpoint{1.417791in}{4.852798in}}%
\pgfpathlineto{\pgfqpoint{1.449923in}{4.898755in}}%
\pgfpathlineto{\pgfqpoint{1.482054in}{4.943415in}}%
\pgfpathlineto{\pgfqpoint{1.514186in}{4.986779in}}%
\pgfpathlineto{\pgfqpoint{1.546317in}{5.028848in}}%
\pgfpathlineto{\pgfqpoint{1.578449in}{5.069623in}}%
\pgfpathlineto{\pgfqpoint{1.610581in}{5.109105in}}%
\pgfpathlineto{\pgfqpoint{1.642712in}{5.147295in}}%
\pgfpathlineto{\pgfqpoint{1.674844in}{5.184194in}}%
\pgfpathlineto{\pgfqpoint{1.702959in}{5.215423in}}%
\pgfpathlineto{\pgfqpoint{1.731074in}{5.245664in}}%
\pgfpathlineto{\pgfqpoint{1.759189in}{5.274919in}}%
\pgfpathlineto{\pgfqpoint{1.787304in}{5.303188in}}%
\pgfpathlineto{\pgfqpoint{1.815419in}{5.330472in}}%
\pgfpathlineto{\pgfqpoint{1.843535in}{5.356771in}}%
\pgfpathlineto{\pgfqpoint{1.871650in}{5.382085in}}%
\pgfpathlineto{\pgfqpoint{1.899765in}{5.406415in}}%
\pgfpathlineto{\pgfqpoint{1.927880in}{5.429762in}}%
\pgfpathlineto{\pgfqpoint{1.955995in}{5.452126in}}%
\pgfpathlineto{\pgfqpoint{1.984110in}{5.473507in}}%
\pgfpathlineto{\pgfqpoint{2.012225in}{5.493906in}}%
\pgfpathlineto{\pgfqpoint{2.040341in}{5.513324in}}%
\pgfpathlineto{\pgfqpoint{2.068456in}{5.531759in}}%
\pgfpathlineto{\pgfqpoint{2.096571in}{5.549214in}}%
\pgfpathlineto{\pgfqpoint{2.124686in}{5.565688in}}%
\pgfpathlineto{\pgfqpoint{2.152801in}{5.581181in}}%
\pgfpathlineto{\pgfqpoint{2.176900in}{5.593681in}}%
\pgfpathlineto{\pgfqpoint{2.200998in}{5.605461in}}%
\pgfpathlineto{\pgfqpoint{2.225097in}{5.616522in}}%
\pgfpathlineto{\pgfqpoint{2.249196in}{5.626862in}}%
\pgfpathlineto{\pgfqpoint{2.273295in}{5.636483in}}%
\pgfpathlineto{\pgfqpoint{2.297393in}{5.645385in}}%
\pgfpathlineto{\pgfqpoint{2.321492in}{5.653567in}}%
\pgfpathlineto{\pgfqpoint{2.345591in}{5.661031in}}%
\pgfpathlineto{\pgfqpoint{2.369689in}{5.667775in}}%
\pgfpathlineto{\pgfqpoint{2.393788in}{5.673800in}}%
\pgfpathlineto{\pgfqpoint{2.417887in}{5.679107in}}%
\pgfpathlineto{\pgfqpoint{2.441985in}{5.683695in}}%
\pgfpathlineto{\pgfqpoint{2.466084in}{5.687564in}}%
\pgfpathlineto{\pgfqpoint{2.490183in}{5.690714in}}%
\pgfpathlineto{\pgfqpoint{2.514281in}{5.693145in}}%
\pgfpathlineto{\pgfqpoint{2.538380in}{5.694859in}}%
\pgfpathlineto{\pgfqpoint{2.562479in}{5.695853in}}%
\pgfpathlineto{\pgfqpoint{2.586577in}{5.696129in}}%
\pgfpathlineto{\pgfqpoint{2.610676in}{5.695686in}}%
\pgfpathlineto{\pgfqpoint{2.634775in}{5.694525in}}%
\pgfpathlineto{\pgfqpoint{2.658874in}{5.692646in}}%
\pgfpathlineto{\pgfqpoint{2.682972in}{5.690047in}}%
\pgfpathlineto{\pgfqpoint{2.707071in}{5.686730in}}%
\pgfpathlineto{\pgfqpoint{2.731170in}{5.682695in}}%
\pgfpathlineto{\pgfqpoint{2.755268in}{5.677940in}}%
\pgfpathlineto{\pgfqpoint{2.779367in}{5.672467in}}%
\pgfpathlineto{\pgfqpoint{2.803466in}{5.666275in}}%
\pgfpathlineto{\pgfqpoint{2.827564in}{5.659364in}}%
\pgfpathlineto{\pgfqpoint{2.851663in}{5.651734in}}%
\pgfpathlineto{\pgfqpoint{2.875762in}{5.643385in}}%
\pgfpathlineto{\pgfqpoint{2.899860in}{5.634316in}}%
\pgfpathlineto{\pgfqpoint{2.923959in}{5.624528in}}%
\pgfpathlineto{\pgfqpoint{2.948058in}{5.614021in}}%
\pgfpathlineto{\pgfqpoint{2.972156in}{5.602794in}}%
\pgfpathlineto{\pgfqpoint{2.996255in}{5.590847in}}%
\pgfpathlineto{\pgfqpoint{3.020354in}{5.578180in}}%
\pgfpathlineto{\pgfqpoint{3.048469in}{5.562492in}}%
\pgfpathlineto{\pgfqpoint{3.076584in}{5.545823in}}%
\pgfpathlineto{\pgfqpoint{3.104699in}{5.528173in}}%
\pgfpathlineto{\pgfqpoint{3.132814in}{5.509542in}}%
\pgfpathlineto{\pgfqpoint{3.160930in}{5.489930in}}%
\pgfpathlineto{\pgfqpoint{3.189045in}{5.469336in}}%
\pgfpathlineto{\pgfqpoint{3.217160in}{5.447759in}}%
\pgfpathlineto{\pgfqpoint{3.245275in}{5.425200in}}%
\pgfpathlineto{\pgfqpoint{3.273390in}{5.401657in}}%
\pgfpathlineto{\pgfqpoint{3.301505in}{5.377131in}}%
\pgfpathlineto{\pgfqpoint{3.329620in}{5.351622in}}%
\pgfpathlineto{\pgfqpoint{3.357735in}{5.325127in}}%
\pgfpathlineto{\pgfqpoint{3.385851in}{5.297648in}}%
\pgfpathlineto{\pgfqpoint{3.413966in}{5.269183in}}%
\pgfpathlineto{\pgfqpoint{3.442081in}{5.239732in}}%
\pgfpathlineto{\pgfqpoint{3.470196in}{5.209294in}}%
\pgfpathlineto{\pgfqpoint{3.498311in}{5.177869in}}%
\pgfpathlineto{\pgfqpoint{3.526426in}{5.145457in}}%
\pgfpathlineto{\pgfqpoint{3.558558in}{5.107203in}}%
\pgfpathlineto{\pgfqpoint{3.590689in}{5.067658in}}%
\pgfpathlineto{\pgfqpoint{3.622821in}{5.026820in}}%
\pgfpathlineto{\pgfqpoint{3.654953in}{4.984688in}}%
\pgfpathlineto{\pgfqpoint{3.687084in}{4.941260in}}%
\pgfpathlineto{\pgfqpoint{3.719216in}{4.896537in}}%
\pgfpathlineto{\pgfqpoint{3.751347in}{4.850516in}}%
\pgfpathlineto{\pgfqpoint{3.783479in}{4.803197in}}%
\pgfpathlineto{\pgfqpoint{3.815611in}{4.754579in}}%
\pgfpathlineto{\pgfqpoint{3.847742in}{4.704660in}}%
\pgfpathlineto{\pgfqpoint{3.879874in}{4.653438in}}%
\pgfpathlineto{\pgfqpoint{3.912005in}{4.600913in}}%
\pgfpathlineto{\pgfqpoint{3.944531in}{4.546414in}}%
\pgfpathlineto{\pgfqpoint{3.944531in}{4.546414in}}%
\pgfusepath{stroke}%
\end{pgfscope}%
\begin{pgfscope}%
\pgfpathrectangle{\pgfqpoint{0.984216in}{4.549747in}}{\pgfqpoint{4.458056in}{3.401160in}} %
\pgfusepath{clip}%
\pgfsetbuttcap%
\pgfsetroundjoin%
\definecolor{currentfill}{rgb}{0.000000,0.000000,1.000000}%
\pgfsetfillcolor{currentfill}%
\pgfsetlinewidth{1.003750pt}%
\definecolor{currentstroke}{rgb}{0.000000,0.000000,1.000000}%
\pgfsetstrokecolor{currentstroke}%
\pgfsetdash{}{0pt}%
\pgfsys@defobject{currentmarker}{\pgfqpoint{-0.041667in}{-0.041667in}}{\pgfqpoint{0.041667in}{0.041667in}}{%
\pgfpathmoveto{\pgfqpoint{0.000000in}{-0.041667in}}%
\pgfpathcurveto{\pgfqpoint{0.011050in}{-0.041667in}}{\pgfqpoint{0.021649in}{-0.037276in}}{\pgfqpoint{0.029463in}{-0.029463in}}%
\pgfpathcurveto{\pgfqpoint{0.037276in}{-0.021649in}}{\pgfqpoint{0.041667in}{-0.011050in}}{\pgfqpoint{0.041667in}{0.000000in}}%
\pgfpathcurveto{\pgfqpoint{0.041667in}{0.011050in}}{\pgfqpoint{0.037276in}{0.021649in}}{\pgfqpoint{0.029463in}{0.029463in}}%
\pgfpathcurveto{\pgfqpoint{0.021649in}{0.037276in}}{\pgfqpoint{0.011050in}{0.041667in}}{\pgfqpoint{0.000000in}{0.041667in}}%
\pgfpathcurveto{\pgfqpoint{-0.011050in}{0.041667in}}{\pgfqpoint{-0.021649in}{0.037276in}}{\pgfqpoint{-0.029463in}{0.029463in}}%
\pgfpathcurveto{\pgfqpoint{-0.037276in}{0.021649in}}{\pgfqpoint{-0.041667in}{0.011050in}}{\pgfqpoint{-0.041667in}{0.000000in}}%
\pgfpathcurveto{\pgfqpoint{-0.041667in}{-0.011050in}}{\pgfqpoint{-0.037276in}{-0.021649in}}{\pgfqpoint{-0.029463in}{-0.029463in}}%
\pgfpathcurveto{\pgfqpoint{-0.021649in}{-0.037276in}}{\pgfqpoint{-0.011050in}{-0.041667in}}{\pgfqpoint{0.000000in}{-0.041667in}}%
\pgfpathclose%
\pgfusepath{stroke,fill}%
}%
\begin{pgfscope}%
\pgfsys@transformshift{1.225002in}{4.549747in}%
\pgfsys@useobject{currentmarker}{}%
\end{pgfscope}%
\begin{pgfscope}%
\pgfsys@transformshift{1.626646in}{5.128361in}%
\pgfsys@useobject{currentmarker}{}%
\end{pgfscope}%
\begin{pgfscope}%
\pgfsys@transformshift{2.028291in}{5.505122in}%
\pgfsys@useobject{currentmarker}{}%
\end{pgfscope}%
\begin{pgfscope}%
\pgfsys@transformshift{2.429936in}{5.681491in}%
\pgfsys@useobject{currentmarker}{}%
\end{pgfscope}%
\begin{pgfscope}%
\pgfsys@transformshift{2.831581in}{5.658142in}%
\pgfsys@useobject{currentmarker}{}%
\end{pgfscope}%
\begin{pgfscope}%
\pgfsys@transformshift{3.233226in}{5.434988in}%
\pgfsys@useobject{currentmarker}{}%
\end{pgfscope}%
\begin{pgfscope}%
\pgfsys@transformshift{3.634870in}{5.011172in}%
\pgfsys@useobject{currentmarker}{}%
\end{pgfscope}%
\begin{pgfscope}%
\pgfsys@transformshift{4.036515in}{4.385043in}%
\pgfsys@useobject{currentmarker}{}%
\end{pgfscope}%
\begin{pgfscope}%
\pgfsys@transformshift{4.438160in}{3.554106in}%
\pgfsys@useobject{currentmarker}{}%
\end{pgfscope}%
\begin{pgfscope}%
\pgfsys@transformshift{4.839805in}{2.514942in}%
\pgfsys@useobject{currentmarker}{}%
\end{pgfscope}%
\begin{pgfscope}%
\pgfsys@transformshift{5.241450in}{1.263089in}%
\pgfsys@useobject{currentmarker}{}%
\end{pgfscope}%
\begin{pgfscope}%
\pgfsys@transformshift{5.643094in}{-0.207116in}%
\pgfsys@useobject{currentmarker}{}%
\end{pgfscope}%
\end{pgfscope}%
\begin{pgfscope}%
\pgfpathrectangle{\pgfqpoint{0.984216in}{4.549747in}}{\pgfqpoint{4.458056in}{3.401160in}} %
\pgfusepath{clip}%
\pgfsetbuttcap%
\pgfsetroundjoin%
\pgfsetlinewidth{1.505625pt}%
\definecolor{currentstroke}{rgb}{0.000000,0.750000,0.750000}%
\pgfsetstrokecolor{currentstroke}%
\pgfsetdash{{9.600000pt}{2.400000pt}{1.500000pt}{2.400000pt}}{0.000000pt}%
\pgfpathmoveto{\pgfqpoint{1.225002in}{4.549747in}}%
\pgfpathlineto{\pgfqpoint{1.257133in}{4.603513in}}%
\pgfpathlineto{\pgfqpoint{1.289265in}{4.655973in}}%
\pgfpathlineto{\pgfqpoint{1.321396in}{4.707126in}}%
\pgfpathlineto{\pgfqpoint{1.353528in}{4.756975in}}%
\pgfpathlineto{\pgfqpoint{1.385660in}{4.805517in}}%
\pgfpathlineto{\pgfqpoint{1.417791in}{4.852755in}}%
\pgfpathlineto{\pgfqpoint{1.449923in}{4.898687in}}%
\pgfpathlineto{\pgfqpoint{1.482054in}{4.943314in}}%
\pgfpathlineto{\pgfqpoint{1.514186in}{4.986636in}}%
\pgfpathlineto{\pgfqpoint{1.546317in}{5.028653in}}%
\pgfpathlineto{\pgfqpoint{1.578449in}{5.069365in}}%
\pgfpathlineto{\pgfqpoint{1.610581in}{5.108772in}}%
\pgfpathlineto{\pgfqpoint{1.642712in}{5.146875in}}%
\pgfpathlineto{\pgfqpoint{1.670827in}{5.179145in}}%
\pgfpathlineto{\pgfqpoint{1.698942in}{5.210416in}}%
\pgfpathlineto{\pgfqpoint{1.727058in}{5.240688in}}%
\pgfpathlineto{\pgfqpoint{1.755173in}{5.269962in}}%
\pgfpathlineto{\pgfqpoint{1.783288in}{5.298237in}}%
\pgfpathlineto{\pgfqpoint{1.811403in}{5.325514in}}%
\pgfpathlineto{\pgfqpoint{1.839518in}{5.351793in}}%
\pgfpathlineto{\pgfqpoint{1.867633in}{5.377073in}}%
\pgfpathlineto{\pgfqpoint{1.895748in}{5.401355in}}%
\pgfpathlineto{\pgfqpoint{1.923864in}{5.424638in}}%
\pgfpathlineto{\pgfqpoint{1.951979in}{5.446923in}}%
\pgfpathlineto{\pgfqpoint{1.980094in}{5.468210in}}%
\pgfpathlineto{\pgfqpoint{2.008209in}{5.488499in}}%
\pgfpathlineto{\pgfqpoint{2.036324in}{5.507790in}}%
\pgfpathlineto{\pgfqpoint{2.064439in}{5.526083in}}%
\pgfpathlineto{\pgfqpoint{2.092554in}{5.543377in}}%
\pgfpathlineto{\pgfqpoint{2.120670in}{5.559674in}}%
\pgfpathlineto{\pgfqpoint{2.144768in}{5.572849in}}%
\pgfpathlineto{\pgfqpoint{2.168867in}{5.585290in}}%
\pgfpathlineto{\pgfqpoint{2.192966in}{5.596998in}}%
\pgfpathlineto{\pgfqpoint{2.217064in}{5.607972in}}%
\pgfpathlineto{\pgfqpoint{2.241163in}{5.618214in}}%
\pgfpathlineto{\pgfqpoint{2.265262in}{5.627723in}}%
\pgfpathlineto{\pgfqpoint{2.289360in}{5.636498in}}%
\pgfpathlineto{\pgfqpoint{2.313459in}{5.644541in}}%
\pgfpathlineto{\pgfqpoint{2.337558in}{5.651850in}}%
\pgfpathlineto{\pgfqpoint{2.361656in}{5.658426in}}%
\pgfpathlineto{\pgfqpoint{2.385755in}{5.664269in}}%
\pgfpathlineto{\pgfqpoint{2.409854in}{5.669379in}}%
\pgfpathlineto{\pgfqpoint{2.433952in}{5.673756in}}%
\pgfpathlineto{\pgfqpoint{2.458051in}{5.677400in}}%
\pgfpathlineto{\pgfqpoint{2.482150in}{5.680312in}}%
\pgfpathlineto{\pgfqpoint{2.506249in}{5.682489in}}%
\pgfpathlineto{\pgfqpoint{2.530347in}{5.683934in}}%
\pgfpathlineto{\pgfqpoint{2.554446in}{5.684646in}}%
\pgfpathlineto{\pgfqpoint{2.578545in}{5.684625in}}%
\pgfpathlineto{\pgfqpoint{2.602643in}{5.683871in}}%
\pgfpathlineto{\pgfqpoint{2.626742in}{5.682384in}}%
\pgfpathlineto{\pgfqpoint{2.650841in}{5.680164in}}%
\pgfpathlineto{\pgfqpoint{2.674939in}{5.677211in}}%
\pgfpathlineto{\pgfqpoint{2.699038in}{5.673525in}}%
\pgfpathlineto{\pgfqpoint{2.723137in}{5.669105in}}%
\pgfpathlineto{\pgfqpoint{2.747235in}{5.663953in}}%
\pgfpathlineto{\pgfqpoint{2.771334in}{5.658068in}}%
\pgfpathlineto{\pgfqpoint{2.795433in}{5.651449in}}%
\pgfpathlineto{\pgfqpoint{2.819531in}{5.644098in}}%
\pgfpathlineto{\pgfqpoint{2.843630in}{5.636013in}}%
\pgfpathlineto{\pgfqpoint{2.867729in}{5.627196in}}%
\pgfpathlineto{\pgfqpoint{2.891828in}{5.617645in}}%
\pgfpathlineto{\pgfqpoint{2.915926in}{5.607361in}}%
\pgfpathlineto{\pgfqpoint{2.940025in}{5.596344in}}%
\pgfpathlineto{\pgfqpoint{2.964124in}{5.584594in}}%
\pgfpathlineto{\pgfqpoint{2.988222in}{5.572111in}}%
\pgfpathlineto{\pgfqpoint{3.012321in}{5.558894in}}%
\pgfpathlineto{\pgfqpoint{3.036420in}{5.544945in}}%
\pgfpathlineto{\pgfqpoint{3.064535in}{5.527743in}}%
\pgfpathlineto{\pgfqpoint{3.092650in}{5.509544in}}%
\pgfpathlineto{\pgfqpoint{3.120765in}{5.490347in}}%
\pgfpathlineto{\pgfqpoint{3.148880in}{5.470151in}}%
\pgfpathlineto{\pgfqpoint{3.176995in}{5.448957in}}%
\pgfpathlineto{\pgfqpoint{3.205110in}{5.426766in}}%
\pgfpathlineto{\pgfqpoint{3.233226in}{5.403575in}}%
\pgfpathlineto{\pgfqpoint{3.261341in}{5.379387in}}%
\pgfpathlineto{\pgfqpoint{3.289456in}{5.354200in}}%
\pgfpathlineto{\pgfqpoint{3.317571in}{5.328015in}}%
\pgfpathlineto{\pgfqpoint{3.345686in}{5.300832in}}%
\pgfpathlineto{\pgfqpoint{3.373801in}{5.272650in}}%
\pgfpathlineto{\pgfqpoint{3.401916in}{5.243470in}}%
\pgfpathlineto{\pgfqpoint{3.430032in}{5.213291in}}%
\pgfpathlineto{\pgfqpoint{3.458147in}{5.182113in}}%
\pgfpathlineto{\pgfqpoint{3.486262in}{5.149937in}}%
\pgfpathlineto{\pgfqpoint{3.514377in}{5.116761in}}%
\pgfpathlineto{\pgfqpoint{3.546509in}{5.077624in}}%
\pgfpathlineto{\pgfqpoint{3.578640in}{5.037182in}}%
\pgfpathlineto{\pgfqpoint{3.610772in}{4.995435in}}%
\pgfpathlineto{\pgfqpoint{3.642903in}{4.952383in}}%
\pgfpathlineto{\pgfqpoint{3.675035in}{4.908026in}}%
\pgfpathlineto{\pgfqpoint{3.707166in}{4.862364in}}%
\pgfpathlineto{\pgfqpoint{3.739298in}{4.815397in}}%
\pgfpathlineto{\pgfqpoint{3.771430in}{4.767124in}}%
\pgfpathlineto{\pgfqpoint{3.803561in}{4.717546in}}%
\pgfpathlineto{\pgfqpoint{3.835693in}{4.666662in}}%
\pgfpathlineto{\pgfqpoint{3.867824in}{4.614472in}}%
\pgfpathlineto{\pgfqpoint{3.899956in}{4.560977in}}%
\pgfpathlineto{\pgfqpoint{3.908569in}{4.546414in}}%
\pgfpathlineto{\pgfqpoint{3.908569in}{4.546414in}}%
\pgfusepath{stroke}%
\end{pgfscope}%
\begin{pgfscope}%
\pgfpathrectangle{\pgfqpoint{0.984216in}{4.549747in}}{\pgfqpoint{4.458056in}{3.401160in}} %
\pgfusepath{clip}%
\pgfsetbuttcap%
\pgfsetmiterjoin%
\definecolor{currentfill}{rgb}{0.000000,0.750000,0.750000}%
\pgfsetfillcolor{currentfill}%
\pgfsetlinewidth{1.003750pt}%
\definecolor{currentstroke}{rgb}{0.000000,0.750000,0.750000}%
\pgfsetstrokecolor{currentstroke}%
\pgfsetdash{}{0pt}%
\pgfsys@defobject{currentmarker}{\pgfqpoint{-0.041667in}{-0.041667in}}{\pgfqpoint{0.041667in}{0.041667in}}{%
\pgfpathmoveto{\pgfqpoint{-0.000000in}{-0.041667in}}%
\pgfpathlineto{\pgfqpoint{0.041667in}{0.041667in}}%
\pgfpathlineto{\pgfqpoint{-0.041667in}{0.041667in}}%
\pgfpathclose%
\pgfusepath{stroke,fill}%
}%
\begin{pgfscope}%
\pgfsys@transformshift{1.225002in}{4.549747in}%
\pgfsys@useobject{currentmarker}{}%
\end{pgfscope}%
\begin{pgfscope}%
\pgfsys@transformshift{1.626646in}{5.127986in}%
\pgfsys@useobject{currentmarker}{}%
\end{pgfscope}%
\begin{pgfscope}%
\pgfsys@transformshift{2.028291in}{5.502380in}%
\pgfsys@useobject{currentmarker}{}%
\end{pgfscope}%
\begin{pgfscope}%
\pgfsys@transformshift{2.429936in}{5.673078in}%
\pgfsys@useobject{currentmarker}{}%
\end{pgfscope}%
\begin{pgfscope}%
\pgfsys@transformshift{2.831581in}{5.640147in}%
\pgfsys@useobject{currentmarker}{}%
\end{pgfscope}%
\begin{pgfscope}%
\pgfsys@transformshift{3.233226in}{5.403575in}%
\pgfsys@useobject{currentmarker}{}%
\end{pgfscope}%
\begin{pgfscope}%
\pgfsys@transformshift{3.634870in}{4.963268in}%
\pgfsys@useobject{currentmarker}{}%
\end{pgfscope}%
\begin{pgfscope}%
\pgfsys@transformshift{4.036515in}{4.319050in}%
\pgfsys@useobject{currentmarker}{}%
\end{pgfscope}%
\begin{pgfscope}%
\pgfsys@transformshift{4.438160in}{3.470662in}%
\pgfsys@useobject{currentmarker}{}%
\end{pgfscope}%
\begin{pgfscope}%
\pgfsys@transformshift{4.839805in}{2.417766in}%
\pgfsys@useobject{currentmarker}{}%
\end{pgfscope}%
\begin{pgfscope}%
\pgfsys@transformshift{5.241450in}{1.159936in}%
\pgfsys@useobject{currentmarker}{}%
\end{pgfscope}%
\begin{pgfscope}%
\pgfsys@transformshift{5.643094in}{-0.303337in}%
\pgfsys@useobject{currentmarker}{}%
\end{pgfscope}%
\end{pgfscope}%
\begin{pgfscope}%
\pgfsetrectcap%
\pgfsetmiterjoin%
\pgfsetlinewidth{0.803000pt}%
\definecolor{currentstroke}{rgb}{0.000000,0.000000,0.000000}%
\pgfsetstrokecolor{currentstroke}%
\pgfsetdash{}{0pt}%
\pgfpathmoveto{\pgfqpoint{0.984216in}{4.549747in}}%
\pgfpathlineto{\pgfqpoint{0.984216in}{7.950908in}}%
\pgfusepath{stroke}%
\end{pgfscope}%
\begin{pgfscope}%
\pgfsetrectcap%
\pgfsetmiterjoin%
\pgfsetlinewidth{0.803000pt}%
\definecolor{currentstroke}{rgb}{0.000000,0.000000,0.000000}%
\pgfsetstrokecolor{currentstroke}%
\pgfsetdash{}{0pt}%
\pgfpathmoveto{\pgfqpoint{5.442272in}{4.549747in}}%
\pgfpathlineto{\pgfqpoint{5.442272in}{7.950908in}}%
\pgfusepath{stroke}%
\end{pgfscope}%
\begin{pgfscope}%
\pgfsetrectcap%
\pgfsetmiterjoin%
\pgfsetlinewidth{0.803000pt}%
\definecolor{currentstroke}{rgb}{0.000000,0.000000,0.000000}%
\pgfsetstrokecolor{currentstroke}%
\pgfsetdash{}{0pt}%
\pgfpathmoveto{\pgfqpoint{0.984216in}{4.549747in}}%
\pgfpathlineto{\pgfqpoint{5.442272in}{4.549747in}}%
\pgfusepath{stroke}%
\end{pgfscope}%
\begin{pgfscope}%
\pgfsetrectcap%
\pgfsetmiterjoin%
\pgfsetlinewidth{0.803000pt}%
\definecolor{currentstroke}{rgb}{0.000000,0.000000,0.000000}%
\pgfsetstrokecolor{currentstroke}%
\pgfsetdash{}{0pt}%
\pgfpathmoveto{\pgfqpoint{0.984216in}{7.950908in}}%
\pgfpathlineto{\pgfqpoint{5.442272in}{7.950908in}}%
\pgfusepath{stroke}%
\end{pgfscope}%
\begin{pgfscope}%
\pgfsetbuttcap%
\pgfsetmiterjoin%
\definecolor{currentfill}{rgb}{1.000000,1.000000,1.000000}%
\pgfsetfillcolor{currentfill}%
\pgfsetfillopacity{0.800000}%
\pgfsetlinewidth{1.003750pt}%
\definecolor{currentstroke}{rgb}{0.800000,0.800000,0.800000}%
\pgfsetstrokecolor{currentstroke}%
\pgfsetstrokeopacity{0.800000}%
\pgfsetdash{}{0pt}%
\pgfpathmoveto{\pgfqpoint{1.120327in}{6.650998in}}%
\pgfpathlineto{\pgfqpoint{2.718605in}{6.650998in}}%
\pgfpathquadraticcurveto{\pgfqpoint{2.757494in}{6.650998in}}{\pgfqpoint{2.757494in}{6.689886in}}%
\pgfpathlineto{\pgfqpoint{2.757494in}{7.814796in}}%
\pgfpathquadraticcurveto{\pgfqpoint{2.757494in}{7.853685in}}{\pgfqpoint{2.718605in}{7.853685in}}%
\pgfpathlineto{\pgfqpoint{1.120327in}{7.853685in}}%
\pgfpathquadraticcurveto{\pgfqpoint{1.081438in}{7.853685in}}{\pgfqpoint{1.081438in}{7.814796in}}%
\pgfpathlineto{\pgfqpoint{1.081438in}{6.689886in}}%
\pgfpathquadraticcurveto{\pgfqpoint{1.081438in}{6.650998in}}{\pgfqpoint{1.120327in}{6.650998in}}%
\pgfpathclose%
\pgfusepath{stroke,fill}%
\end{pgfscope}%
\begin{pgfscope}%
\pgftext[x=1.615059in,y=7.628175in,left,base]{\rmfamily\fontsize{14.000000}{16.800000}\selectfont \(\displaystyle \mathbf{I}\mbox{g} = \) 2}%
\end{pgfscope}%
\begin{pgfscope}%
\pgfsetbuttcap%
\pgfsetroundjoin%
\pgfsetlinewidth{1.505625pt}%
\definecolor{currentstroke}{rgb}{1.000000,0.000000,0.000000}%
\pgfsetstrokecolor{currentstroke}%
\pgfsetdash{{5.550000pt}{2.400000pt}}{0.000000pt}%
\pgfpathmoveto{\pgfqpoint{1.159216in}{7.408077in}}%
\pgfpathlineto{\pgfqpoint{1.548104in}{7.408077in}}%
\pgfusepath{stroke}%
\end{pgfscope}%
\begin{pgfscope}%
\pgfsetbuttcap%
\pgfsetmiterjoin%
\definecolor{currentfill}{rgb}{1.000000,0.000000,0.000000}%
\pgfsetfillcolor{currentfill}%
\pgfsetlinewidth{1.003750pt}%
\definecolor{currentstroke}{rgb}{1.000000,0.000000,0.000000}%
\pgfsetstrokecolor{currentstroke}%
\pgfsetdash{}{0pt}%
\pgfsys@defobject{currentmarker}{\pgfqpoint{-0.041667in}{-0.041667in}}{\pgfqpoint{0.041667in}{0.041667in}}{%
\pgfpathmoveto{\pgfqpoint{-0.041667in}{-0.041667in}}%
\pgfpathlineto{\pgfqpoint{0.041667in}{-0.041667in}}%
\pgfpathlineto{\pgfqpoint{0.041667in}{0.041667in}}%
\pgfpathlineto{\pgfqpoint{-0.041667in}{0.041667in}}%
\pgfpathclose%
\pgfusepath{stroke,fill}%
}%
\begin{pgfscope}%
\pgfsys@transformshift{1.353660in}{7.408077in}%
\pgfsys@useobject{currentmarker}{}%
\end{pgfscope}%
\end{pgfscope}%
\begin{pgfscope}%
\pgftext[x=1.703660in,y=7.340022in,left,base]{\rmfamily\fontsize{14.000000}{16.800000}\selectfont \(\displaystyle \mathbf{E}\mbox{u}=\) 1}%
\end{pgfscope}%
\begin{pgfscope}%
\pgfsetrectcap%
\pgfsetroundjoin%
\pgfsetlinewidth{1.505625pt}%
\definecolor{currentstroke}{rgb}{0.000000,0.000000,1.000000}%
\pgfsetstrokecolor{currentstroke}%
\pgfsetdash{}{0pt}%
\pgfpathmoveto{\pgfqpoint{1.159216in}{7.122677in}}%
\pgfpathlineto{\pgfqpoint{1.548104in}{7.122677in}}%
\pgfusepath{stroke}%
\end{pgfscope}%
\begin{pgfscope}%
\pgfsetbuttcap%
\pgfsetroundjoin%
\definecolor{currentfill}{rgb}{0.000000,0.000000,1.000000}%
\pgfsetfillcolor{currentfill}%
\pgfsetlinewidth{1.003750pt}%
\definecolor{currentstroke}{rgb}{0.000000,0.000000,1.000000}%
\pgfsetstrokecolor{currentstroke}%
\pgfsetdash{}{0pt}%
\pgfsys@defobject{currentmarker}{\pgfqpoint{-0.041667in}{-0.041667in}}{\pgfqpoint{0.041667in}{0.041667in}}{%
\pgfpathmoveto{\pgfqpoint{0.000000in}{-0.041667in}}%
\pgfpathcurveto{\pgfqpoint{0.011050in}{-0.041667in}}{\pgfqpoint{0.021649in}{-0.037276in}}{\pgfqpoint{0.029463in}{-0.029463in}}%
\pgfpathcurveto{\pgfqpoint{0.037276in}{-0.021649in}}{\pgfqpoint{0.041667in}{-0.011050in}}{\pgfqpoint{0.041667in}{0.000000in}}%
\pgfpathcurveto{\pgfqpoint{0.041667in}{0.011050in}}{\pgfqpoint{0.037276in}{0.021649in}}{\pgfqpoint{0.029463in}{0.029463in}}%
\pgfpathcurveto{\pgfqpoint{0.021649in}{0.037276in}}{\pgfqpoint{0.011050in}{0.041667in}}{\pgfqpoint{0.000000in}{0.041667in}}%
\pgfpathcurveto{\pgfqpoint{-0.011050in}{0.041667in}}{\pgfqpoint{-0.021649in}{0.037276in}}{\pgfqpoint{-0.029463in}{0.029463in}}%
\pgfpathcurveto{\pgfqpoint{-0.037276in}{0.021649in}}{\pgfqpoint{-0.041667in}{0.011050in}}{\pgfqpoint{-0.041667in}{0.000000in}}%
\pgfpathcurveto{\pgfqpoint{-0.041667in}{-0.011050in}}{\pgfqpoint{-0.037276in}{-0.021649in}}{\pgfqpoint{-0.029463in}{-0.029463in}}%
\pgfpathcurveto{\pgfqpoint{-0.021649in}{-0.037276in}}{\pgfqpoint{-0.011050in}{-0.041667in}}{\pgfqpoint{0.000000in}{-0.041667in}}%
\pgfpathclose%
\pgfusepath{stroke,fill}%
}%
\begin{pgfscope}%
\pgfsys@transformshift{1.353660in}{7.122677in}%
\pgfsys@useobject{currentmarker}{}%
\end{pgfscope}%
\end{pgfscope}%
\begin{pgfscope}%
\pgftext[x=1.703660in,y=7.054621in,left,base]{\rmfamily\fontsize{14.000000}{16.800000}\selectfont \(\displaystyle \mathbf{E}\mbox{u}=\) 0.1}%
\end{pgfscope}%
\begin{pgfscope}%
\pgfsetbuttcap%
\pgfsetroundjoin%
\pgfsetlinewidth{1.505625pt}%
\definecolor{currentstroke}{rgb}{0.000000,0.750000,0.750000}%
\pgfsetstrokecolor{currentstroke}%
\pgfsetdash{{9.600000pt}{2.400000pt}{1.500000pt}{2.400000pt}}{0.000000pt}%
\pgfpathmoveto{\pgfqpoint{1.159216in}{6.837277in}}%
\pgfpathlineto{\pgfqpoint{1.548104in}{6.837277in}}%
\pgfusepath{stroke}%
\end{pgfscope}%
\begin{pgfscope}%
\pgfsetbuttcap%
\pgfsetmiterjoin%
\definecolor{currentfill}{rgb}{0.000000,0.750000,0.750000}%
\pgfsetfillcolor{currentfill}%
\pgfsetlinewidth{1.003750pt}%
\definecolor{currentstroke}{rgb}{0.000000,0.750000,0.750000}%
\pgfsetstrokecolor{currentstroke}%
\pgfsetdash{}{0pt}%
\pgfsys@defobject{currentmarker}{\pgfqpoint{-0.041667in}{-0.041667in}}{\pgfqpoint{0.041667in}{0.041667in}}{%
\pgfpathmoveto{\pgfqpoint{-0.000000in}{-0.041667in}}%
\pgfpathlineto{\pgfqpoint{0.041667in}{0.041667in}}%
\pgfpathlineto{\pgfqpoint{-0.041667in}{0.041667in}}%
\pgfpathclose%
\pgfusepath{stroke,fill}%
}%
\begin{pgfscope}%
\pgfsys@transformshift{1.353660in}{6.837277in}%
\pgfsys@useobject{currentmarker}{}%
\end{pgfscope}%
\end{pgfscope}%
\begin{pgfscope}%
\pgftext[x=1.703660in,y=6.769221in,left,base]{\rmfamily\fontsize{14.000000}{16.800000}\selectfont \(\displaystyle \mathbf{E}\mbox{u}=\) 0.01}%
\end{pgfscope}%
\begin{pgfscope}%
\pgfsetbuttcap%
\pgfsetmiterjoin%
\definecolor{currentfill}{rgb}{1.000000,1.000000,1.000000}%
\pgfsetfillcolor{currentfill}%
\pgfsetlinewidth{0.000000pt}%
\definecolor{currentstroke}{rgb}{0.000000,0.000000,0.000000}%
\pgfsetstrokecolor{currentstroke}%
\pgfsetstrokeopacity{0.000000}%
\pgfsetdash{}{0pt}%
\pgfpathmoveto{\pgfqpoint{5.697941in}{4.549747in}}%
\pgfpathlineto{\pgfqpoint{10.155998in}{4.549747in}}%
\pgfpathlineto{\pgfqpoint{10.155998in}{7.950908in}}%
\pgfpathlineto{\pgfqpoint{5.697941in}{7.950908in}}%
\pgfpathclose%
\pgfusepath{fill}%
\end{pgfscope}%
\begin{pgfscope}%
\pgfsetbuttcap%
\pgfsetroundjoin%
\definecolor{currentfill}{rgb}{0.000000,0.000000,0.000000}%
\pgfsetfillcolor{currentfill}%
\pgfsetlinewidth{0.803000pt}%
\definecolor{currentstroke}{rgb}{0.000000,0.000000,0.000000}%
\pgfsetstrokecolor{currentstroke}%
\pgfsetdash{}{0pt}%
\pgfsys@defobject{currentmarker}{\pgfqpoint{0.000000in}{-0.048611in}}{\pgfqpoint{0.000000in}{0.000000in}}{%
\pgfpathmoveto{\pgfqpoint{0.000000in}{0.000000in}}%
\pgfpathlineto{\pgfqpoint{0.000000in}{-0.048611in}}%
\pgfusepath{stroke,fill}%
}%
\begin{pgfscope}%
\pgfsys@transformshift{5.938727in}{4.549747in}%
\pgfsys@useobject{currentmarker}{}%
\end{pgfscope}%
\end{pgfscope}%
\begin{pgfscope}%
\pgfsetbuttcap%
\pgfsetroundjoin%
\definecolor{currentfill}{rgb}{0.000000,0.000000,0.000000}%
\pgfsetfillcolor{currentfill}%
\pgfsetlinewidth{0.803000pt}%
\definecolor{currentstroke}{rgb}{0.000000,0.000000,0.000000}%
\pgfsetstrokecolor{currentstroke}%
\pgfsetdash{}{0pt}%
\pgfsys@defobject{currentmarker}{\pgfqpoint{0.000000in}{-0.048611in}}{\pgfqpoint{0.000000in}{0.000000in}}{%
\pgfpathmoveto{\pgfqpoint{0.000000in}{0.000000in}}%
\pgfpathlineto{\pgfqpoint{0.000000in}{-0.048611in}}%
\pgfusepath{stroke,fill}%
}%
\begin{pgfscope}%
\pgfsys@transformshift{6.742017in}{4.549747in}%
\pgfsys@useobject{currentmarker}{}%
\end{pgfscope}%
\end{pgfscope}%
\begin{pgfscope}%
\pgfsetbuttcap%
\pgfsetroundjoin%
\definecolor{currentfill}{rgb}{0.000000,0.000000,0.000000}%
\pgfsetfillcolor{currentfill}%
\pgfsetlinewidth{0.803000pt}%
\definecolor{currentstroke}{rgb}{0.000000,0.000000,0.000000}%
\pgfsetstrokecolor{currentstroke}%
\pgfsetdash{}{0pt}%
\pgfsys@defobject{currentmarker}{\pgfqpoint{0.000000in}{-0.048611in}}{\pgfqpoint{0.000000in}{0.000000in}}{%
\pgfpathmoveto{\pgfqpoint{0.000000in}{0.000000in}}%
\pgfpathlineto{\pgfqpoint{0.000000in}{-0.048611in}}%
\pgfusepath{stroke,fill}%
}%
\begin{pgfscope}%
\pgfsys@transformshift{7.545306in}{4.549747in}%
\pgfsys@useobject{currentmarker}{}%
\end{pgfscope}%
\end{pgfscope}%
\begin{pgfscope}%
\pgfsetbuttcap%
\pgfsetroundjoin%
\definecolor{currentfill}{rgb}{0.000000,0.000000,0.000000}%
\pgfsetfillcolor{currentfill}%
\pgfsetlinewidth{0.803000pt}%
\definecolor{currentstroke}{rgb}{0.000000,0.000000,0.000000}%
\pgfsetstrokecolor{currentstroke}%
\pgfsetdash{}{0pt}%
\pgfsys@defobject{currentmarker}{\pgfqpoint{0.000000in}{-0.048611in}}{\pgfqpoint{0.000000in}{0.000000in}}{%
\pgfpathmoveto{\pgfqpoint{0.000000in}{0.000000in}}%
\pgfpathlineto{\pgfqpoint{0.000000in}{-0.048611in}}%
\pgfusepath{stroke,fill}%
}%
\begin{pgfscope}%
\pgfsys@transformshift{8.348596in}{4.549747in}%
\pgfsys@useobject{currentmarker}{}%
\end{pgfscope}%
\end{pgfscope}%
\begin{pgfscope}%
\pgfsetbuttcap%
\pgfsetroundjoin%
\definecolor{currentfill}{rgb}{0.000000,0.000000,0.000000}%
\pgfsetfillcolor{currentfill}%
\pgfsetlinewidth{0.803000pt}%
\definecolor{currentstroke}{rgb}{0.000000,0.000000,0.000000}%
\pgfsetstrokecolor{currentstroke}%
\pgfsetdash{}{0pt}%
\pgfsys@defobject{currentmarker}{\pgfqpoint{0.000000in}{-0.048611in}}{\pgfqpoint{0.000000in}{0.000000in}}{%
\pgfpathmoveto{\pgfqpoint{0.000000in}{0.000000in}}%
\pgfpathlineto{\pgfqpoint{0.000000in}{-0.048611in}}%
\pgfusepath{stroke,fill}%
}%
\begin{pgfscope}%
\pgfsys@transformshift{9.151886in}{4.549747in}%
\pgfsys@useobject{currentmarker}{}%
\end{pgfscope}%
\end{pgfscope}%
\begin{pgfscope}%
\pgfsetbuttcap%
\pgfsetroundjoin%
\definecolor{currentfill}{rgb}{0.000000,0.000000,0.000000}%
\pgfsetfillcolor{currentfill}%
\pgfsetlinewidth{0.803000pt}%
\definecolor{currentstroke}{rgb}{0.000000,0.000000,0.000000}%
\pgfsetstrokecolor{currentstroke}%
\pgfsetdash{}{0pt}%
\pgfsys@defobject{currentmarker}{\pgfqpoint{0.000000in}{-0.048611in}}{\pgfqpoint{0.000000in}{0.000000in}}{%
\pgfpathmoveto{\pgfqpoint{0.000000in}{0.000000in}}%
\pgfpathlineto{\pgfqpoint{0.000000in}{-0.048611in}}%
\pgfusepath{stroke,fill}%
}%
\begin{pgfscope}%
\pgfsys@transformshift{9.955175in}{4.549747in}%
\pgfsys@useobject{currentmarker}{}%
\end{pgfscope}%
\end{pgfscope}%
\begin{pgfscope}%
\pgfsetbuttcap%
\pgfsetroundjoin%
\definecolor{currentfill}{rgb}{0.000000,0.000000,0.000000}%
\pgfsetfillcolor{currentfill}%
\pgfsetlinewidth{0.803000pt}%
\definecolor{currentstroke}{rgb}{0.000000,0.000000,0.000000}%
\pgfsetstrokecolor{currentstroke}%
\pgfsetdash{}{0pt}%
\pgfsys@defobject{currentmarker}{\pgfqpoint{-0.048611in}{0.000000in}}{\pgfqpoint{0.000000in}{0.000000in}}{%
\pgfpathmoveto{\pgfqpoint{0.000000in}{0.000000in}}%
\pgfpathlineto{\pgfqpoint{-0.048611in}{0.000000in}}%
\pgfusepath{stroke,fill}%
}%
\begin{pgfscope}%
\pgfsys@transformshift{5.697941in}{4.549747in}%
\pgfsys@useobject{currentmarker}{}%
\end{pgfscope}%
\end{pgfscope}%
\begin{pgfscope}%
\pgfsetbuttcap%
\pgfsetroundjoin%
\definecolor{currentfill}{rgb}{0.000000,0.000000,0.000000}%
\pgfsetfillcolor{currentfill}%
\pgfsetlinewidth{0.803000pt}%
\definecolor{currentstroke}{rgb}{0.000000,0.000000,0.000000}%
\pgfsetstrokecolor{currentstroke}%
\pgfsetdash{}{0pt}%
\pgfsys@defobject{currentmarker}{\pgfqpoint{-0.048611in}{0.000000in}}{\pgfqpoint{0.000000in}{0.000000in}}{%
\pgfpathmoveto{\pgfqpoint{0.000000in}{0.000000in}}%
\pgfpathlineto{\pgfqpoint{-0.048611in}{0.000000in}}%
\pgfusepath{stroke,fill}%
}%
\begin{pgfscope}%
\pgfsys@transformshift{5.697941in}{5.229979in}%
\pgfsys@useobject{currentmarker}{}%
\end{pgfscope}%
\end{pgfscope}%
\begin{pgfscope}%
\pgfsetbuttcap%
\pgfsetroundjoin%
\definecolor{currentfill}{rgb}{0.000000,0.000000,0.000000}%
\pgfsetfillcolor{currentfill}%
\pgfsetlinewidth{0.803000pt}%
\definecolor{currentstroke}{rgb}{0.000000,0.000000,0.000000}%
\pgfsetstrokecolor{currentstroke}%
\pgfsetdash{}{0pt}%
\pgfsys@defobject{currentmarker}{\pgfqpoint{-0.048611in}{0.000000in}}{\pgfqpoint{0.000000in}{0.000000in}}{%
\pgfpathmoveto{\pgfqpoint{0.000000in}{0.000000in}}%
\pgfpathlineto{\pgfqpoint{-0.048611in}{0.000000in}}%
\pgfusepath{stroke,fill}%
}%
\begin{pgfscope}%
\pgfsys@transformshift{5.697941in}{5.910211in}%
\pgfsys@useobject{currentmarker}{}%
\end{pgfscope}%
\end{pgfscope}%
\begin{pgfscope}%
\pgfsetbuttcap%
\pgfsetroundjoin%
\definecolor{currentfill}{rgb}{0.000000,0.000000,0.000000}%
\pgfsetfillcolor{currentfill}%
\pgfsetlinewidth{0.803000pt}%
\definecolor{currentstroke}{rgb}{0.000000,0.000000,0.000000}%
\pgfsetstrokecolor{currentstroke}%
\pgfsetdash{}{0pt}%
\pgfsys@defobject{currentmarker}{\pgfqpoint{-0.048611in}{0.000000in}}{\pgfqpoint{0.000000in}{0.000000in}}{%
\pgfpathmoveto{\pgfqpoint{0.000000in}{0.000000in}}%
\pgfpathlineto{\pgfqpoint{-0.048611in}{0.000000in}}%
\pgfusepath{stroke,fill}%
}%
\begin{pgfscope}%
\pgfsys@transformshift{5.697941in}{6.590443in}%
\pgfsys@useobject{currentmarker}{}%
\end{pgfscope}%
\end{pgfscope}%
\begin{pgfscope}%
\pgfsetbuttcap%
\pgfsetroundjoin%
\definecolor{currentfill}{rgb}{0.000000,0.000000,0.000000}%
\pgfsetfillcolor{currentfill}%
\pgfsetlinewidth{0.803000pt}%
\definecolor{currentstroke}{rgb}{0.000000,0.000000,0.000000}%
\pgfsetstrokecolor{currentstroke}%
\pgfsetdash{}{0pt}%
\pgfsys@defobject{currentmarker}{\pgfqpoint{-0.048611in}{0.000000in}}{\pgfqpoint{0.000000in}{0.000000in}}{%
\pgfpathmoveto{\pgfqpoint{0.000000in}{0.000000in}}%
\pgfpathlineto{\pgfqpoint{-0.048611in}{0.000000in}}%
\pgfusepath{stroke,fill}%
}%
\begin{pgfscope}%
\pgfsys@transformshift{5.697941in}{7.270676in}%
\pgfsys@useobject{currentmarker}{}%
\end{pgfscope}%
\end{pgfscope}%
\begin{pgfscope}%
\pgfsetbuttcap%
\pgfsetroundjoin%
\definecolor{currentfill}{rgb}{0.000000,0.000000,0.000000}%
\pgfsetfillcolor{currentfill}%
\pgfsetlinewidth{0.803000pt}%
\definecolor{currentstroke}{rgb}{0.000000,0.000000,0.000000}%
\pgfsetstrokecolor{currentstroke}%
\pgfsetdash{}{0pt}%
\pgfsys@defobject{currentmarker}{\pgfqpoint{-0.048611in}{0.000000in}}{\pgfqpoint{0.000000in}{0.000000in}}{%
\pgfpathmoveto{\pgfqpoint{0.000000in}{0.000000in}}%
\pgfpathlineto{\pgfqpoint{-0.048611in}{0.000000in}}%
\pgfusepath{stroke,fill}%
}%
\begin{pgfscope}%
\pgfsys@transformshift{5.697941in}{7.950908in}%
\pgfsys@useobject{currentmarker}{}%
\end{pgfscope}%
\end{pgfscope}%
\begin{pgfscope}%
\pgfpathrectangle{\pgfqpoint{5.697941in}{4.549747in}}{\pgfqpoint{4.458056in}{3.401160in}} %
\pgfusepath{clip}%
\pgfsetbuttcap%
\pgfsetroundjoin%
\pgfsetlinewidth{1.505625pt}%
\definecolor{currentstroke}{rgb}{1.000000,0.000000,0.000000}%
\pgfsetstrokecolor{currentstroke}%
\pgfsetdash{{5.550000pt}{2.400000pt}}{0.000000pt}%
\pgfpathmoveto{\pgfqpoint{5.938727in}{4.549747in}}%
\pgfpathlineto{\pgfqpoint{5.978892in}{4.617093in}}%
\pgfpathlineto{\pgfqpoint{6.019056in}{4.683091in}}%
\pgfpathlineto{\pgfqpoint{6.059221in}{4.747754in}}%
\pgfpathlineto{\pgfqpoint{6.099385in}{4.811095in}}%
\pgfpathlineto{\pgfqpoint{6.139550in}{4.873124in}}%
\pgfpathlineto{\pgfqpoint{6.179714in}{4.933854in}}%
\pgfpathlineto{\pgfqpoint{6.219879in}{4.993294in}}%
\pgfpathlineto{\pgfqpoint{6.260043in}{5.051454in}}%
\pgfpathlineto{\pgfqpoint{6.300207in}{5.108343in}}%
\pgfpathlineto{\pgfqpoint{6.340372in}{5.163972in}}%
\pgfpathlineto{\pgfqpoint{6.380536in}{5.218348in}}%
\pgfpathlineto{\pgfqpoint{6.420701in}{5.271480in}}%
\pgfpathlineto{\pgfqpoint{6.460865in}{5.323375in}}%
\pgfpathlineto{\pgfqpoint{6.501030in}{5.374042in}}%
\pgfpathlineto{\pgfqpoint{6.541194in}{5.423488in}}%
\pgfpathlineto{\pgfqpoint{6.581359in}{5.471718in}}%
\pgfpathlineto{\pgfqpoint{6.621523in}{5.518741in}}%
\pgfpathlineto{\pgfqpoint{6.661688in}{5.564563in}}%
\pgfpathlineto{\pgfqpoint{6.701852in}{5.609189in}}%
\pgfpathlineto{\pgfqpoint{6.742017in}{5.652625in}}%
\pgfpathlineto{\pgfqpoint{6.782181in}{5.694877in}}%
\pgfpathlineto{\pgfqpoint{6.822346in}{5.735950in}}%
\pgfpathlineto{\pgfqpoint{6.858494in}{5.771913in}}%
\pgfpathlineto{\pgfqpoint{6.894642in}{5.806928in}}%
\pgfpathlineto{\pgfqpoint{6.930790in}{5.841000in}}%
\pgfpathlineto{\pgfqpoint{6.966938in}{5.874131in}}%
\pgfpathlineto{\pgfqpoint{7.003086in}{5.906325in}}%
\pgfpathlineto{\pgfqpoint{7.039234in}{5.937584in}}%
\pgfpathlineto{\pgfqpoint{7.075382in}{5.967913in}}%
\pgfpathlineto{\pgfqpoint{7.111530in}{5.997312in}}%
\pgfpathlineto{\pgfqpoint{7.147678in}{6.025787in}}%
\pgfpathlineto{\pgfqpoint{7.183826in}{6.053337in}}%
\pgfpathlineto{\pgfqpoint{7.219974in}{6.079968in}}%
\pgfpathlineto{\pgfqpoint{7.256122in}{6.105680in}}%
\pgfpathlineto{\pgfqpoint{7.292270in}{6.130476in}}%
\pgfpathlineto{\pgfqpoint{7.328418in}{6.154358in}}%
\pgfpathlineto{\pgfqpoint{7.364566in}{6.177329in}}%
\pgfpathlineto{\pgfqpoint{7.400714in}{6.199390in}}%
\pgfpathlineto{\pgfqpoint{7.436862in}{6.220543in}}%
\pgfpathlineto{\pgfqpoint{7.473010in}{6.240791in}}%
\pgfpathlineto{\pgfqpoint{7.509158in}{6.260135in}}%
\pgfpathlineto{\pgfqpoint{7.545306in}{6.278577in}}%
\pgfpathlineto{\pgfqpoint{7.581454in}{6.296118in}}%
\pgfpathlineto{\pgfqpoint{7.617602in}{6.312761in}}%
\pgfpathlineto{\pgfqpoint{7.653750in}{6.328506in}}%
\pgfpathlineto{\pgfqpoint{7.689898in}{6.343355in}}%
\pgfpathlineto{\pgfqpoint{7.726047in}{6.357309in}}%
\pgfpathlineto{\pgfqpoint{7.758178in}{6.368964in}}%
\pgfpathlineto{\pgfqpoint{7.790310in}{6.379913in}}%
\pgfpathlineto{\pgfqpoint{7.822441in}{6.390158in}}%
\pgfpathlineto{\pgfqpoint{7.854573in}{6.399700in}}%
\pgfpathlineto{\pgfqpoint{7.886704in}{6.408539in}}%
\pgfpathlineto{\pgfqpoint{7.918836in}{6.416677in}}%
\pgfpathlineto{\pgfqpoint{7.950968in}{6.424113in}}%
\pgfpathlineto{\pgfqpoint{7.983099in}{6.430849in}}%
\pgfpathlineto{\pgfqpoint{8.015231in}{6.436885in}}%
\pgfpathlineto{\pgfqpoint{8.047362in}{6.442221in}}%
\pgfpathlineto{\pgfqpoint{8.079494in}{6.446858in}}%
\pgfpathlineto{\pgfqpoint{8.111626in}{6.450797in}}%
\pgfpathlineto{\pgfqpoint{8.143757in}{6.454037in}}%
\pgfpathlineto{\pgfqpoint{8.175889in}{6.456580in}}%
\pgfpathlineto{\pgfqpoint{8.208020in}{6.458425in}}%
\pgfpathlineto{\pgfqpoint{8.240152in}{6.459572in}}%
\pgfpathlineto{\pgfqpoint{8.272283in}{6.460022in}}%
\pgfpathlineto{\pgfqpoint{8.304415in}{6.459775in}}%
\pgfpathlineto{\pgfqpoint{8.336547in}{6.458831in}}%
\pgfpathlineto{\pgfqpoint{8.368678in}{6.457190in}}%
\pgfpathlineto{\pgfqpoint{8.400810in}{6.454851in}}%
\pgfpathlineto{\pgfqpoint{8.432941in}{6.451815in}}%
\pgfpathlineto{\pgfqpoint{8.465073in}{6.448082in}}%
\pgfpathlineto{\pgfqpoint{8.497205in}{6.443650in}}%
\pgfpathlineto{\pgfqpoint{8.529336in}{6.438520in}}%
\pgfpathlineto{\pgfqpoint{8.561468in}{6.432692in}}%
\pgfpathlineto{\pgfqpoint{8.593599in}{6.426165in}}%
\pgfpathlineto{\pgfqpoint{8.625731in}{6.418938in}}%
\pgfpathlineto{\pgfqpoint{8.657862in}{6.411011in}}%
\pgfpathlineto{\pgfqpoint{8.689994in}{6.402382in}}%
\pgfpathlineto{\pgfqpoint{8.722126in}{6.393053in}}%
\pgfpathlineto{\pgfqpoint{8.754257in}{6.383020in}}%
\pgfpathlineto{\pgfqpoint{8.786389in}{6.372284in}}%
\pgfpathlineto{\pgfqpoint{8.818520in}{6.360844in}}%
\pgfpathlineto{\pgfqpoint{8.850652in}{6.348697in}}%
\pgfpathlineto{\pgfqpoint{8.882784in}{6.335844in}}%
\pgfpathlineto{\pgfqpoint{8.918932in}{6.320539in}}%
\pgfpathlineto{\pgfqpoint{8.955080in}{6.304335in}}%
\pgfpathlineto{\pgfqpoint{8.991228in}{6.287231in}}%
\pgfpathlineto{\pgfqpoint{9.027376in}{6.269225in}}%
\pgfpathlineto{\pgfqpoint{9.063524in}{6.250315in}}%
\pgfpathlineto{\pgfqpoint{9.099672in}{6.230497in}}%
\pgfpathlineto{\pgfqpoint{9.135820in}{6.209771in}}%
\pgfpathlineto{\pgfqpoint{9.171968in}{6.188132in}}%
\pgfpathlineto{\pgfqpoint{9.208116in}{6.165578in}}%
\pgfpathlineto{\pgfqpoint{9.244264in}{6.142107in}}%
\pgfpathlineto{\pgfqpoint{9.280412in}{6.117714in}}%
\pgfpathlineto{\pgfqpoint{9.316560in}{6.092398in}}%
\pgfpathlineto{\pgfqpoint{9.352708in}{6.066155in}}%
\pgfpathlineto{\pgfqpoint{9.388856in}{6.038980in}}%
\pgfpathlineto{\pgfqpoint{9.425004in}{6.010872in}}%
\pgfpathlineto{\pgfqpoint{9.461152in}{5.981826in}}%
\pgfpathlineto{\pgfqpoint{9.497300in}{5.951838in}}%
\pgfpathlineto{\pgfqpoint{9.533448in}{5.920906in}}%
\pgfpathlineto{\pgfqpoint{9.569596in}{5.889024in}}%
\pgfpathlineto{\pgfqpoint{9.605744in}{5.856189in}}%
\pgfpathlineto{\pgfqpoint{9.641892in}{5.822397in}}%
\pgfpathlineto{\pgfqpoint{9.678040in}{5.787644in}}%
\pgfpathlineto{\pgfqpoint{9.714188in}{5.751925in}}%
\pgfpathlineto{\pgfqpoint{9.750336in}{5.715237in}}%
\pgfpathlineto{\pgfqpoint{9.786484in}{5.677575in}}%
\pgfpathlineto{\pgfqpoint{9.822632in}{5.638935in}}%
\pgfpathlineto{\pgfqpoint{9.858780in}{5.599313in}}%
\pgfpathlineto{\pgfqpoint{9.898945in}{5.554130in}}%
\pgfpathlineto{\pgfqpoint{9.939109in}{5.507723in}}%
\pgfpathlineto{\pgfqpoint{9.979274in}{5.460085in}}%
\pgfpathlineto{\pgfqpoint{10.019438in}{5.411210in}}%
\pgfpathlineto{\pgfqpoint{10.059603in}{5.361093in}}%
\pgfpathlineto{\pgfqpoint{10.099767in}{5.309726in}}%
\pgfpathlineto{\pgfqpoint{10.139932in}{5.257104in}}%
\pgfpathlineto{\pgfqpoint{10.159331in}{5.231236in}}%
\pgfpathlineto{\pgfqpoint{10.159331in}{5.231236in}}%
\pgfusepath{stroke}%
\end{pgfscope}%
\begin{pgfscope}%
\pgfpathrectangle{\pgfqpoint{5.697941in}{4.549747in}}{\pgfqpoint{4.458056in}{3.401160in}} %
\pgfusepath{clip}%
\pgfsetbuttcap%
\pgfsetmiterjoin%
\definecolor{currentfill}{rgb}{1.000000,0.000000,0.000000}%
\pgfsetfillcolor{currentfill}%
\pgfsetlinewidth{1.003750pt}%
\definecolor{currentstroke}{rgb}{1.000000,0.000000,0.000000}%
\pgfsetstrokecolor{currentstroke}%
\pgfsetdash{}{0pt}%
\pgfsys@defobject{currentmarker}{\pgfqpoint{-0.041667in}{-0.041667in}}{\pgfqpoint{0.041667in}{0.041667in}}{%
\pgfpathmoveto{\pgfqpoint{-0.041667in}{-0.041667in}}%
\pgfpathlineto{\pgfqpoint{0.041667in}{-0.041667in}}%
\pgfpathlineto{\pgfqpoint{0.041667in}{0.041667in}}%
\pgfpathlineto{\pgfqpoint{-0.041667in}{0.041667in}}%
\pgfpathclose%
\pgfusepath{stroke,fill}%
}%
\begin{pgfscope}%
\pgfsys@transformshift{5.938727in}{4.549747in}%
\pgfsys@useobject{currentmarker}{}%
\end{pgfscope}%
\begin{pgfscope}%
\pgfsys@transformshift{6.340372in}{5.163972in}%
\pgfsys@useobject{currentmarker}{}%
\end{pgfscope}%
\begin{pgfscope}%
\pgfsys@transformshift{6.742017in}{5.652625in}%
\pgfsys@useobject{currentmarker}{}%
\end{pgfscope}%
\begin{pgfscope}%
\pgfsys@transformshift{7.143662in}{6.022668in}%
\pgfsys@useobject{currentmarker}{}%
\end{pgfscope}%
\begin{pgfscope}%
\pgfsys@transformshift{7.545306in}{6.278577in}%
\pgfsys@useobject{currentmarker}{}%
\end{pgfscope}%
\begin{pgfscope}%
\pgfsys@transformshift{7.946951in}{6.423222in}%
\pgfsys@useobject{currentmarker}{}%
\end{pgfscope}%
\begin{pgfscope}%
\pgfsys@transformshift{8.348596in}{6.458297in}%
\pgfsys@useobject{currentmarker}{}%
\end{pgfscope}%
\begin{pgfscope}%
\pgfsys@transformshift{8.750241in}{6.384313in}%
\pgfsys@useobject{currentmarker}{}%
\end{pgfscope}%
\begin{pgfscope}%
\pgfsys@transformshift{9.151886in}{6.200266in}%
\pgfsys@useobject{currentmarker}{}%
\end{pgfscope}%
\begin{pgfscope}%
\pgfsys@transformshift{9.553530in}{5.903311in}%
\pgfsys@useobject{currentmarker}{}%
\end{pgfscope}%
\begin{pgfscope}%
\pgfsys@transformshift{9.955175in}{5.488815in}%
\pgfsys@useobject{currentmarker}{}%
\end{pgfscope}%
\begin{pgfscope}%
\pgfsys@transformshift{10.356820in}{4.950985in}%
\pgfsys@useobject{currentmarker}{}%
\end{pgfscope}%
\end{pgfscope}%
\begin{pgfscope}%
\pgfpathrectangle{\pgfqpoint{5.697941in}{4.549747in}}{\pgfqpoint{4.458056in}{3.401160in}} %
\pgfusepath{clip}%
\pgfsetrectcap%
\pgfsetroundjoin%
\pgfsetlinewidth{1.505625pt}%
\definecolor{currentstroke}{rgb}{0.000000,0.000000,1.000000}%
\pgfsetstrokecolor{currentstroke}%
\pgfsetdash{}{0pt}%
\pgfpathmoveto{\pgfqpoint{5.938727in}{4.549747in}}%
\pgfpathlineto{\pgfqpoint{5.978892in}{4.617091in}}%
\pgfpathlineto{\pgfqpoint{6.019056in}{4.683075in}}%
\pgfpathlineto{\pgfqpoint{6.059221in}{4.747701in}}%
\pgfpathlineto{\pgfqpoint{6.099385in}{4.810971in}}%
\pgfpathlineto{\pgfqpoint{6.139550in}{4.872885in}}%
\pgfpathlineto{\pgfqpoint{6.179714in}{4.933446in}}%
\pgfpathlineto{\pgfqpoint{6.219879in}{4.992653in}}%
\pgfpathlineto{\pgfqpoint{6.260043in}{5.050509in}}%
\pgfpathlineto{\pgfqpoint{6.300207in}{5.107014in}}%
\pgfpathlineto{\pgfqpoint{6.340372in}{5.162170in}}%
\pgfpathlineto{\pgfqpoint{6.380536in}{5.215978in}}%
\pgfpathlineto{\pgfqpoint{6.420701in}{5.268438in}}%
\pgfpathlineto{\pgfqpoint{6.456849in}{5.314501in}}%
\pgfpathlineto{\pgfqpoint{6.492997in}{5.359474in}}%
\pgfpathlineto{\pgfqpoint{6.529145in}{5.403358in}}%
\pgfpathlineto{\pgfqpoint{6.565293in}{5.446154in}}%
\pgfpathlineto{\pgfqpoint{6.601441in}{5.487862in}}%
\pgfpathlineto{\pgfqpoint{6.637589in}{5.528483in}}%
\pgfpathlineto{\pgfqpoint{6.673737in}{5.568017in}}%
\pgfpathlineto{\pgfqpoint{6.709885in}{5.606466in}}%
\pgfpathlineto{\pgfqpoint{6.746033in}{5.643829in}}%
\pgfpathlineto{\pgfqpoint{6.782181in}{5.680108in}}%
\pgfpathlineto{\pgfqpoint{6.818329in}{5.715302in}}%
\pgfpathlineto{\pgfqpoint{6.854477in}{5.749413in}}%
\pgfpathlineto{\pgfqpoint{6.890625in}{5.782441in}}%
\pgfpathlineto{\pgfqpoint{6.926773in}{5.814386in}}%
\pgfpathlineto{\pgfqpoint{6.962921in}{5.845250in}}%
\pgfpathlineto{\pgfqpoint{6.999069in}{5.875031in}}%
\pgfpathlineto{\pgfqpoint{7.031201in}{5.900597in}}%
\pgfpathlineto{\pgfqpoint{7.063333in}{5.925308in}}%
\pgfpathlineto{\pgfqpoint{7.095464in}{5.949165in}}%
\pgfpathlineto{\pgfqpoint{7.127596in}{5.972169in}}%
\pgfpathlineto{\pgfqpoint{7.159727in}{5.994321in}}%
\pgfpathlineto{\pgfqpoint{7.191859in}{6.015619in}}%
\pgfpathlineto{\pgfqpoint{7.223991in}{6.036065in}}%
\pgfpathlineto{\pgfqpoint{7.256122in}{6.055658in}}%
\pgfpathlineto{\pgfqpoint{7.288254in}{6.074400in}}%
\pgfpathlineto{\pgfqpoint{7.320385in}{6.092290in}}%
\pgfpathlineto{\pgfqpoint{7.352517in}{6.109328in}}%
\pgfpathlineto{\pgfqpoint{7.384648in}{6.125515in}}%
\pgfpathlineto{\pgfqpoint{7.416780in}{6.140850in}}%
\pgfpathlineto{\pgfqpoint{7.448912in}{6.155335in}}%
\pgfpathlineto{\pgfqpoint{7.481043in}{6.168968in}}%
\pgfpathlineto{\pgfqpoint{7.513175in}{6.181751in}}%
\pgfpathlineto{\pgfqpoint{7.545306in}{6.193684in}}%
\pgfpathlineto{\pgfqpoint{7.577438in}{6.204766in}}%
\pgfpathlineto{\pgfqpoint{7.609570in}{6.214997in}}%
\pgfpathlineto{\pgfqpoint{7.641701in}{6.224379in}}%
\pgfpathlineto{\pgfqpoint{7.673833in}{6.232911in}}%
\pgfpathlineto{\pgfqpoint{7.705964in}{6.240592in}}%
\pgfpathlineto{\pgfqpoint{7.738096in}{6.247424in}}%
\pgfpathlineto{\pgfqpoint{7.770227in}{6.253406in}}%
\pgfpathlineto{\pgfqpoint{7.802359in}{6.258538in}}%
\pgfpathlineto{\pgfqpoint{7.834491in}{6.262821in}}%
\pgfpathlineto{\pgfqpoint{7.866622in}{6.266254in}}%
\pgfpathlineto{\pgfqpoint{7.898754in}{6.268838in}}%
\pgfpathlineto{\pgfqpoint{7.926869in}{6.270402in}}%
\pgfpathlineto{\pgfqpoint{7.954984in}{6.271315in}}%
\pgfpathlineto{\pgfqpoint{7.983099in}{6.271578in}}%
\pgfpathlineto{\pgfqpoint{8.011214in}{6.271191in}}%
\pgfpathlineto{\pgfqpoint{8.039329in}{6.270153in}}%
\pgfpathlineto{\pgfqpoint{8.067445in}{6.268465in}}%
\pgfpathlineto{\pgfqpoint{8.095560in}{6.266127in}}%
\pgfpathlineto{\pgfqpoint{8.127691in}{6.262658in}}%
\pgfpathlineto{\pgfqpoint{8.159823in}{6.258339in}}%
\pgfpathlineto{\pgfqpoint{8.191954in}{6.253171in}}%
\pgfpathlineto{\pgfqpoint{8.224086in}{6.247153in}}%
\pgfpathlineto{\pgfqpoint{8.256218in}{6.240286in}}%
\pgfpathlineto{\pgfqpoint{8.288349in}{6.232568in}}%
\pgfpathlineto{\pgfqpoint{8.320481in}{6.224001in}}%
\pgfpathlineto{\pgfqpoint{8.352612in}{6.214584in}}%
\pgfpathlineto{\pgfqpoint{8.384744in}{6.204316in}}%
\pgfpathlineto{\pgfqpoint{8.416876in}{6.193198in}}%
\pgfpathlineto{\pgfqpoint{8.449007in}{6.181230in}}%
\pgfpathlineto{\pgfqpoint{8.481139in}{6.168411in}}%
\pgfpathlineto{\pgfqpoint{8.513270in}{6.154742in}}%
\pgfpathlineto{\pgfqpoint{8.545402in}{6.140221in}}%
\pgfpathlineto{\pgfqpoint{8.577533in}{6.124850in}}%
\pgfpathlineto{\pgfqpoint{8.609665in}{6.108628in}}%
\pgfpathlineto{\pgfqpoint{8.641797in}{6.091554in}}%
\pgfpathlineto{\pgfqpoint{8.673928in}{6.073628in}}%
\pgfpathlineto{\pgfqpoint{8.706060in}{6.054851in}}%
\pgfpathlineto{\pgfqpoint{8.738191in}{6.035221in}}%
\pgfpathlineto{\pgfqpoint{8.770323in}{6.014739in}}%
\pgfpathlineto{\pgfqpoint{8.802455in}{5.993405in}}%
\pgfpathlineto{\pgfqpoint{8.834586in}{5.971218in}}%
\pgfpathlineto{\pgfqpoint{8.866718in}{5.948178in}}%
\pgfpathlineto{\pgfqpoint{8.898849in}{5.924284in}}%
\pgfpathlineto{\pgfqpoint{8.930981in}{5.899537in}}%
\pgfpathlineto{\pgfqpoint{8.963112in}{5.873936in}}%
\pgfpathlineto{\pgfqpoint{8.995244in}{5.847481in}}%
\pgfpathlineto{\pgfqpoint{9.031392in}{5.816697in}}%
\pgfpathlineto{\pgfqpoint{9.067540in}{5.784832in}}%
\pgfpathlineto{\pgfqpoint{9.103688in}{5.751883in}}%
\pgfpathlineto{\pgfqpoint{9.139836in}{5.717852in}}%
\pgfpathlineto{\pgfqpoint{9.175984in}{5.682738in}}%
\pgfpathlineto{\pgfqpoint{9.212132in}{5.646539in}}%
\pgfpathlineto{\pgfqpoint{9.248280in}{5.609255in}}%
\pgfpathlineto{\pgfqpoint{9.284428in}{5.570887in}}%
\pgfpathlineto{\pgfqpoint{9.320576in}{5.531432in}}%
\pgfpathlineto{\pgfqpoint{9.356724in}{5.490892in}}%
\pgfpathlineto{\pgfqpoint{9.392872in}{5.449264in}}%
\pgfpathlineto{\pgfqpoint{9.429020in}{5.406548in}}%
\pgfpathlineto{\pgfqpoint{9.465168in}{5.362744in}}%
\pgfpathlineto{\pgfqpoint{9.501317in}{5.317851in}}%
\pgfpathlineto{\pgfqpoint{9.537465in}{5.271869in}}%
\pgfpathlineto{\pgfqpoint{9.573613in}{5.224796in}}%
\pgfpathlineto{\pgfqpoint{9.613777in}{5.171212in}}%
\pgfpathlineto{\pgfqpoint{9.653942in}{5.116281in}}%
\pgfpathlineto{\pgfqpoint{9.694106in}{5.060000in}}%
\pgfpathlineto{\pgfqpoint{9.734270in}{5.002369in}}%
\pgfpathlineto{\pgfqpoint{9.774435in}{4.943386in}}%
\pgfpathlineto{\pgfqpoint{9.814599in}{4.883051in}}%
\pgfpathlineto{\pgfqpoint{9.854764in}{4.821362in}}%
\pgfpathlineto{\pgfqpoint{9.894928in}{4.758317in}}%
\pgfpathlineto{\pgfqpoint{9.935093in}{4.693917in}}%
\pgfpathlineto{\pgfqpoint{9.975257in}{4.628158in}}%
\pgfpathlineto{\pgfqpoint{10.015422in}{4.561041in}}%
\pgfpathlineto{\pgfqpoint{10.024068in}{4.546414in}}%
\pgfpathlineto{\pgfqpoint{10.024068in}{4.546414in}}%
\pgfusepath{stroke}%
\end{pgfscope}%
\begin{pgfscope}%
\pgfpathrectangle{\pgfqpoint{5.697941in}{4.549747in}}{\pgfqpoint{4.458056in}{3.401160in}} %
\pgfusepath{clip}%
\pgfsetbuttcap%
\pgfsetroundjoin%
\definecolor{currentfill}{rgb}{0.000000,0.000000,1.000000}%
\pgfsetfillcolor{currentfill}%
\pgfsetlinewidth{1.003750pt}%
\definecolor{currentstroke}{rgb}{0.000000,0.000000,1.000000}%
\pgfsetstrokecolor{currentstroke}%
\pgfsetdash{}{0pt}%
\pgfsys@defobject{currentmarker}{\pgfqpoint{-0.041667in}{-0.041667in}}{\pgfqpoint{0.041667in}{0.041667in}}{%
\pgfpathmoveto{\pgfqpoint{0.000000in}{-0.041667in}}%
\pgfpathcurveto{\pgfqpoint{0.011050in}{-0.041667in}}{\pgfqpoint{0.021649in}{-0.037276in}}{\pgfqpoint{0.029463in}{-0.029463in}}%
\pgfpathcurveto{\pgfqpoint{0.037276in}{-0.021649in}}{\pgfqpoint{0.041667in}{-0.011050in}}{\pgfqpoint{0.041667in}{0.000000in}}%
\pgfpathcurveto{\pgfqpoint{0.041667in}{0.011050in}}{\pgfqpoint{0.037276in}{0.021649in}}{\pgfqpoint{0.029463in}{0.029463in}}%
\pgfpathcurveto{\pgfqpoint{0.021649in}{0.037276in}}{\pgfqpoint{0.011050in}{0.041667in}}{\pgfqpoint{0.000000in}{0.041667in}}%
\pgfpathcurveto{\pgfqpoint{-0.011050in}{0.041667in}}{\pgfqpoint{-0.021649in}{0.037276in}}{\pgfqpoint{-0.029463in}{0.029463in}}%
\pgfpathcurveto{\pgfqpoint{-0.037276in}{0.021649in}}{\pgfqpoint{-0.041667in}{0.011050in}}{\pgfqpoint{-0.041667in}{0.000000in}}%
\pgfpathcurveto{\pgfqpoint{-0.041667in}{-0.011050in}}{\pgfqpoint{-0.037276in}{-0.021649in}}{\pgfqpoint{-0.029463in}{-0.029463in}}%
\pgfpathcurveto{\pgfqpoint{-0.021649in}{-0.037276in}}{\pgfqpoint{-0.011050in}{-0.041667in}}{\pgfqpoint{0.000000in}{-0.041667in}}%
\pgfpathclose%
\pgfusepath{stroke,fill}%
}%
\begin{pgfscope}%
\pgfsys@transformshift{5.938727in}{4.549747in}%
\pgfsys@useobject{currentmarker}{}%
\end{pgfscope}%
\begin{pgfscope}%
\pgfsys@transformshift{6.340372in}{5.162170in}%
\pgfsys@useobject{currentmarker}{}%
\end{pgfscope}%
\begin{pgfscope}%
\pgfsys@transformshift{6.742017in}{5.639731in}%
\pgfsys@useobject{currentmarker}{}%
\end{pgfscope}%
\begin{pgfscope}%
\pgfsys@transformshift{7.143662in}{5.983352in}%
\pgfsys@useobject{currentmarker}{}%
\end{pgfscope}%
\begin{pgfscope}%
\pgfsys@transformshift{7.545306in}{6.193684in}%
\pgfsys@useobject{currentmarker}{}%
\end{pgfscope}%
\begin{pgfscope}%
\pgfsys@transformshift{7.946951in}{6.271121in}%
\pgfsys@useobject{currentmarker}{}%
\end{pgfscope}%
\begin{pgfscope}%
\pgfsys@transformshift{8.348596in}{6.215807in}%
\pgfsys@useobject{currentmarker}{}%
\end{pgfscope}%
\begin{pgfscope}%
\pgfsys@transformshift{8.750241in}{6.027640in}%
\pgfsys@useobject{currentmarker}{}%
\end{pgfscope}%
\begin{pgfscope}%
\pgfsys@transformshift{9.151886in}{5.706268in}%
\pgfsys@useobject{currentmarker}{}%
\end{pgfscope}%
\begin{pgfscope}%
\pgfsys@transformshift{9.553530in}{5.251082in}%
\pgfsys@useobject{currentmarker}{}%
\end{pgfscope}%
\begin{pgfscope}%
\pgfsys@transformshift{9.955175in}{4.661207in}%
\pgfsys@useobject{currentmarker}{}%
\end{pgfscope}%
\begin{pgfscope}%
\pgfsys@transformshift{10.356820in}{3.935483in}%
\pgfsys@useobject{currentmarker}{}%
\end{pgfscope}%
\end{pgfscope}%
\begin{pgfscope}%
\pgfpathrectangle{\pgfqpoint{5.697941in}{4.549747in}}{\pgfqpoint{4.458056in}{3.401160in}} %
\pgfusepath{clip}%
\pgfsetbuttcap%
\pgfsetroundjoin%
\pgfsetlinewidth{1.505625pt}%
\definecolor{currentstroke}{rgb}{0.000000,0.750000,0.750000}%
\pgfsetstrokecolor{currentstroke}%
\pgfsetdash{{9.600000pt}{2.400000pt}{1.500000pt}{2.400000pt}}{0.000000pt}%
\pgfpathmoveto{\pgfqpoint{5.938727in}{4.549747in}}%
\pgfpathlineto{\pgfqpoint{5.978892in}{4.617090in}}%
\pgfpathlineto{\pgfqpoint{6.019056in}{4.683073in}}%
\pgfpathlineto{\pgfqpoint{6.059221in}{4.747695in}}%
\pgfpathlineto{\pgfqpoint{6.099385in}{4.810958in}}%
\pgfpathlineto{\pgfqpoint{6.139550in}{4.872860in}}%
\pgfpathlineto{\pgfqpoint{6.179714in}{4.933403in}}%
\pgfpathlineto{\pgfqpoint{6.219879in}{4.992586in}}%
\pgfpathlineto{\pgfqpoint{6.260043in}{5.050409in}}%
\pgfpathlineto{\pgfqpoint{6.300207in}{5.106873in}}%
\pgfpathlineto{\pgfqpoint{6.340372in}{5.161978in}}%
\pgfpathlineto{\pgfqpoint{6.380536in}{5.215723in}}%
\pgfpathlineto{\pgfqpoint{6.420701in}{5.268109in}}%
\pgfpathlineto{\pgfqpoint{6.456849in}{5.314095in}}%
\pgfpathlineto{\pgfqpoint{6.492997in}{5.358980in}}%
\pgfpathlineto{\pgfqpoint{6.529145in}{5.402764in}}%
\pgfpathlineto{\pgfqpoint{6.565293in}{5.445447in}}%
\pgfpathlineto{\pgfqpoint{6.601441in}{5.487030in}}%
\pgfpathlineto{\pgfqpoint{6.637589in}{5.527513in}}%
\pgfpathlineto{\pgfqpoint{6.673737in}{5.566895in}}%
\pgfpathlineto{\pgfqpoint{6.709885in}{5.605177in}}%
\pgfpathlineto{\pgfqpoint{6.746033in}{5.642359in}}%
\pgfpathlineto{\pgfqpoint{6.782181in}{5.678440in}}%
\pgfpathlineto{\pgfqpoint{6.818329in}{5.713421in}}%
\pgfpathlineto{\pgfqpoint{6.854477in}{5.747302in}}%
\pgfpathlineto{\pgfqpoint{6.890625in}{5.780083in}}%
\pgfpathlineto{\pgfqpoint{6.926773in}{5.811765in}}%
\pgfpathlineto{\pgfqpoint{6.958905in}{5.839002in}}%
\pgfpathlineto{\pgfqpoint{6.991037in}{5.865371in}}%
\pgfpathlineto{\pgfqpoint{7.023168in}{5.890870in}}%
\pgfpathlineto{\pgfqpoint{7.055300in}{5.915501in}}%
\pgfpathlineto{\pgfqpoint{7.087431in}{5.939262in}}%
\pgfpathlineto{\pgfqpoint{7.119563in}{5.962155in}}%
\pgfpathlineto{\pgfqpoint{7.151694in}{5.984179in}}%
\pgfpathlineto{\pgfqpoint{7.183826in}{6.005333in}}%
\pgfpathlineto{\pgfqpoint{7.215958in}{6.025620in}}%
\pgfpathlineto{\pgfqpoint{7.248089in}{6.045037in}}%
\pgfpathlineto{\pgfqpoint{7.280221in}{6.063585in}}%
\pgfpathlineto{\pgfqpoint{7.312352in}{6.081265in}}%
\pgfpathlineto{\pgfqpoint{7.344484in}{6.098076in}}%
\pgfpathlineto{\pgfqpoint{7.376616in}{6.114018in}}%
\pgfpathlineto{\pgfqpoint{7.408747in}{6.129091in}}%
\pgfpathlineto{\pgfqpoint{7.440879in}{6.143296in}}%
\pgfpathlineto{\pgfqpoint{7.473010in}{6.156632in}}%
\pgfpathlineto{\pgfqpoint{7.505142in}{6.169100in}}%
\pgfpathlineto{\pgfqpoint{7.537273in}{6.180699in}}%
\pgfpathlineto{\pgfqpoint{7.569405in}{6.191429in}}%
\pgfpathlineto{\pgfqpoint{7.601537in}{6.201291in}}%
\pgfpathlineto{\pgfqpoint{7.633668in}{6.210284in}}%
\pgfpathlineto{\pgfqpoint{7.665800in}{6.218408in}}%
\pgfpathlineto{\pgfqpoint{7.697931in}{6.225664in}}%
\pgfpathlineto{\pgfqpoint{7.726047in}{6.231301in}}%
\pgfpathlineto{\pgfqpoint{7.754162in}{6.236272in}}%
\pgfpathlineto{\pgfqpoint{7.782277in}{6.240579in}}%
\pgfpathlineto{\pgfqpoint{7.810392in}{6.244220in}}%
\pgfpathlineto{\pgfqpoint{7.838507in}{6.247197in}}%
\pgfpathlineto{\pgfqpoint{7.866622in}{6.249508in}}%
\pgfpathlineto{\pgfqpoint{7.894737in}{6.251155in}}%
\pgfpathlineto{\pgfqpoint{7.922852in}{6.252136in}}%
\pgfpathlineto{\pgfqpoint{7.950968in}{6.252453in}}%
\pgfpathlineto{\pgfqpoint{7.979083in}{6.252105in}}%
\pgfpathlineto{\pgfqpoint{8.007198in}{6.251091in}}%
\pgfpathlineto{\pgfqpoint{8.035313in}{6.249413in}}%
\pgfpathlineto{\pgfqpoint{8.063428in}{6.247070in}}%
\pgfpathlineto{\pgfqpoint{8.091543in}{6.244062in}}%
\pgfpathlineto{\pgfqpoint{8.119658in}{6.240389in}}%
\pgfpathlineto{\pgfqpoint{8.147774in}{6.236050in}}%
\pgfpathlineto{\pgfqpoint{8.175889in}{6.231047in}}%
\pgfpathlineto{\pgfqpoint{8.204004in}{6.225379in}}%
\pgfpathlineto{\pgfqpoint{8.232119in}{6.219046in}}%
\pgfpathlineto{\pgfqpoint{8.264251in}{6.210994in}}%
\pgfpathlineto{\pgfqpoint{8.296382in}{6.202073in}}%
\pgfpathlineto{\pgfqpoint{8.328514in}{6.192284in}}%
\pgfpathlineto{\pgfqpoint{8.360645in}{6.181626in}}%
\pgfpathlineto{\pgfqpoint{8.392777in}{6.170100in}}%
\pgfpathlineto{\pgfqpoint{8.424908in}{6.157705in}}%
\pgfpathlineto{\pgfqpoint{8.457040in}{6.144441in}}%
\pgfpathlineto{\pgfqpoint{8.489172in}{6.130308in}}%
\pgfpathlineto{\pgfqpoint{8.521303in}{6.115307in}}%
\pgfpathlineto{\pgfqpoint{8.553435in}{6.099437in}}%
\pgfpathlineto{\pgfqpoint{8.585566in}{6.082699in}}%
\pgfpathlineto{\pgfqpoint{8.617698in}{6.065092in}}%
\pgfpathlineto{\pgfqpoint{8.649830in}{6.046616in}}%
\pgfpathlineto{\pgfqpoint{8.681961in}{6.027271in}}%
\pgfpathlineto{\pgfqpoint{8.714093in}{6.007057in}}%
\pgfpathlineto{\pgfqpoint{8.746224in}{5.985975in}}%
\pgfpathlineto{\pgfqpoint{8.778356in}{5.964023in}}%
\pgfpathlineto{\pgfqpoint{8.810487in}{5.941203in}}%
\pgfpathlineto{\pgfqpoint{8.842619in}{5.917514in}}%
\pgfpathlineto{\pgfqpoint{8.874751in}{5.892956in}}%
\pgfpathlineto{\pgfqpoint{8.906882in}{5.867529in}}%
\pgfpathlineto{\pgfqpoint{8.939014in}{5.841233in}}%
\pgfpathlineto{\pgfqpoint{8.971145in}{5.814068in}}%
\pgfpathlineto{\pgfqpoint{9.003277in}{5.786033in}}%
\pgfpathlineto{\pgfqpoint{9.039425in}{5.753456in}}%
\pgfpathlineto{\pgfqpoint{9.075573in}{5.719778in}}%
\pgfpathlineto{\pgfqpoint{9.111721in}{5.685001in}}%
\pgfpathlineto{\pgfqpoint{9.147869in}{5.649123in}}%
\pgfpathlineto{\pgfqpoint{9.184017in}{5.612145in}}%
\pgfpathlineto{\pgfqpoint{9.220165in}{5.574067in}}%
\pgfpathlineto{\pgfqpoint{9.256313in}{5.534889in}}%
\pgfpathlineto{\pgfqpoint{9.292461in}{5.494610in}}%
\pgfpathlineto{\pgfqpoint{9.328609in}{5.453231in}}%
\pgfpathlineto{\pgfqpoint{9.364757in}{5.410751in}}%
\pgfpathlineto{\pgfqpoint{9.400905in}{5.367171in}}%
\pgfpathlineto{\pgfqpoint{9.437053in}{5.322490in}}%
\pgfpathlineto{\pgfqpoint{9.473201in}{5.276708in}}%
\pgfpathlineto{\pgfqpoint{9.509349in}{5.229826in}}%
\pgfpathlineto{\pgfqpoint{9.545497in}{5.181842in}}%
\pgfpathlineto{\pgfqpoint{9.585662in}{5.127236in}}%
\pgfpathlineto{\pgfqpoint{9.625826in}{5.071271in}}%
\pgfpathlineto{\pgfqpoint{9.665991in}{5.013946in}}%
\pgfpathlineto{\pgfqpoint{9.706155in}{4.955261in}}%
\pgfpathlineto{\pgfqpoint{9.746320in}{4.895217in}}%
\pgfpathlineto{\pgfqpoint{9.786484in}{4.833813in}}%
\pgfpathlineto{\pgfqpoint{9.826649in}{4.771050in}}%
\pgfpathlineto{\pgfqpoint{9.866813in}{4.706926in}}%
\pgfpathlineto{\pgfqpoint{9.906978in}{4.641442in}}%
\pgfpathlineto{\pgfqpoint{9.947142in}{4.574598in}}%
\pgfpathlineto{\pgfqpoint{9.963836in}{4.546414in}}%
\pgfpathlineto{\pgfqpoint{9.963836in}{4.546414in}}%
\pgfusepath{stroke}%
\end{pgfscope}%
\begin{pgfscope}%
\pgfpathrectangle{\pgfqpoint{5.697941in}{4.549747in}}{\pgfqpoint{4.458056in}{3.401160in}} %
\pgfusepath{clip}%
\pgfsetbuttcap%
\pgfsetmiterjoin%
\definecolor{currentfill}{rgb}{0.000000,0.750000,0.750000}%
\pgfsetfillcolor{currentfill}%
\pgfsetlinewidth{1.003750pt}%
\definecolor{currentstroke}{rgb}{0.000000,0.750000,0.750000}%
\pgfsetstrokecolor{currentstroke}%
\pgfsetdash{}{0pt}%
\pgfsys@defobject{currentmarker}{\pgfqpoint{-0.041667in}{-0.041667in}}{\pgfqpoint{0.041667in}{0.041667in}}{%
\pgfpathmoveto{\pgfqpoint{-0.000000in}{-0.041667in}}%
\pgfpathlineto{\pgfqpoint{0.041667in}{0.041667in}}%
\pgfpathlineto{\pgfqpoint{-0.041667in}{0.041667in}}%
\pgfpathclose%
\pgfusepath{stroke,fill}%
}%
\begin{pgfscope}%
\pgfsys@transformshift{5.938727in}{4.549747in}%
\pgfsys@useobject{currentmarker}{}%
\end{pgfscope}%
\begin{pgfscope}%
\pgfsys@transformshift{6.340372in}{5.161978in}%
\pgfsys@useobject{currentmarker}{}%
\end{pgfscope}%
\begin{pgfscope}%
\pgfsys@transformshift{6.742017in}{5.638282in}%
\pgfsys@useobject{currentmarker}{}%
\end{pgfscope}%
\begin{pgfscope}%
\pgfsys@transformshift{7.143662in}{5.978754in}%
\pgfsys@useobject{currentmarker}{}%
\end{pgfscope}%
\begin{pgfscope}%
\pgfsys@transformshift{7.545306in}{6.183463in}%
\pgfsys@useobject{currentmarker}{}%
\end{pgfscope}%
\begin{pgfscope}%
\pgfsys@transformshift{7.946951in}{6.252448in}%
\pgfsys@useobject{currentmarker}{}%
\end{pgfscope}%
\begin{pgfscope}%
\pgfsys@transformshift{8.348596in}{6.185725in}%
\pgfsys@useobject{currentmarker}{}%
\end{pgfscope}%
\begin{pgfscope}%
\pgfsys@transformshift{8.750241in}{5.983278in}%
\pgfsys@useobject{currentmarker}{}%
\end{pgfscope}%
\begin{pgfscope}%
\pgfsys@transformshift{9.151886in}{5.645069in}%
\pgfsys@useobject{currentmarker}{}%
\end{pgfscope}%
\begin{pgfscope}%
\pgfsys@transformshift{9.553530in}{5.171030in}%
\pgfsys@useobject{currentmarker}{}%
\end{pgfscope}%
\begin{pgfscope}%
\pgfsys@transformshift{9.955175in}{4.561066in}%
\pgfsys@useobject{currentmarker}{}%
\end{pgfscope}%
\begin{pgfscope}%
\pgfsys@transformshift{10.356820in}{3.815055in}%
\pgfsys@useobject{currentmarker}{}%
\end{pgfscope}%
\end{pgfscope}%
\begin{pgfscope}%
\pgfsetrectcap%
\pgfsetmiterjoin%
\pgfsetlinewidth{0.803000pt}%
\definecolor{currentstroke}{rgb}{0.000000,0.000000,0.000000}%
\pgfsetstrokecolor{currentstroke}%
\pgfsetdash{}{0pt}%
\pgfpathmoveto{\pgfqpoint{5.697941in}{4.549747in}}%
\pgfpathlineto{\pgfqpoint{5.697941in}{7.950908in}}%
\pgfusepath{stroke}%
\end{pgfscope}%
\begin{pgfscope}%
\pgfsetrectcap%
\pgfsetmiterjoin%
\pgfsetlinewidth{0.803000pt}%
\definecolor{currentstroke}{rgb}{0.000000,0.000000,0.000000}%
\pgfsetstrokecolor{currentstroke}%
\pgfsetdash{}{0pt}%
\pgfpathmoveto{\pgfqpoint{10.155998in}{4.549747in}}%
\pgfpathlineto{\pgfqpoint{10.155998in}{7.950908in}}%
\pgfusepath{stroke}%
\end{pgfscope}%
\begin{pgfscope}%
\pgfsetrectcap%
\pgfsetmiterjoin%
\pgfsetlinewidth{0.803000pt}%
\definecolor{currentstroke}{rgb}{0.000000,0.000000,0.000000}%
\pgfsetstrokecolor{currentstroke}%
\pgfsetdash{}{0pt}%
\pgfpathmoveto{\pgfqpoint{5.697941in}{4.549747in}}%
\pgfpathlineto{\pgfqpoint{10.155998in}{4.549747in}}%
\pgfusepath{stroke}%
\end{pgfscope}%
\begin{pgfscope}%
\pgfsetrectcap%
\pgfsetmiterjoin%
\pgfsetlinewidth{0.803000pt}%
\definecolor{currentstroke}{rgb}{0.000000,0.000000,0.000000}%
\pgfsetstrokecolor{currentstroke}%
\pgfsetdash{}{0pt}%
\pgfpathmoveto{\pgfqpoint{5.697941in}{7.950908in}}%
\pgfpathlineto{\pgfqpoint{10.155998in}{7.950908in}}%
\pgfusepath{stroke}%
\end{pgfscope}%
\begin{pgfscope}%
\pgfsetbuttcap%
\pgfsetmiterjoin%
\definecolor{currentfill}{rgb}{1.000000,1.000000,1.000000}%
\pgfsetfillcolor{currentfill}%
\pgfsetfillopacity{0.800000}%
\pgfsetlinewidth{1.003750pt}%
\definecolor{currentstroke}{rgb}{0.800000,0.800000,0.800000}%
\pgfsetstrokecolor{currentstroke}%
\pgfsetstrokeopacity{0.800000}%
\pgfsetdash{}{0pt}%
\pgfpathmoveto{\pgfqpoint{5.834052in}{7.409976in}}%
\pgfpathlineto{\pgfqpoint{6.520644in}{7.409976in}}%
\pgfpathquadraticcurveto{\pgfqpoint{6.559533in}{7.409976in}}{\pgfqpoint{6.559533in}{7.448865in}}%
\pgfpathlineto{\pgfqpoint{6.559533in}{7.814796in}}%
\pgfpathquadraticcurveto{\pgfqpoint{6.559533in}{7.853685in}}{\pgfqpoint{6.520644in}{7.853685in}}%
\pgfpathlineto{\pgfqpoint{5.834052in}{7.853685in}}%
\pgfpathquadraticcurveto{\pgfqpoint{5.795163in}{7.853685in}}{\pgfqpoint{5.795163in}{7.814796in}}%
\pgfpathlineto{\pgfqpoint{5.795163in}{7.448865in}}%
\pgfpathquadraticcurveto{\pgfqpoint{5.795163in}{7.409976in}}{\pgfqpoint{5.834052in}{7.409976in}}%
\pgfpathclose%
\pgfusepath{stroke,fill}%
\end{pgfscope}%
\begin{pgfscope}%
\pgftext[x=5.872941in,y=7.628175in,left,base]{\rmfamily\fontsize{14.000000}{16.800000}\selectfont \(\displaystyle \mathbf{I}\mbox{g} = \) 1}%
\end{pgfscope}%
\begin{pgfscope}%
\pgfsetbuttcap%
\pgfsetmiterjoin%
\definecolor{currentfill}{rgb}{1.000000,1.000000,1.000000}%
\pgfsetfillcolor{currentfill}%
\pgfsetlinewidth{0.000000pt}%
\definecolor{currentstroke}{rgb}{0.000000,0.000000,0.000000}%
\pgfsetstrokecolor{currentstroke}%
\pgfsetstrokeopacity{0.000000}%
\pgfsetdash{}{0pt}%
\pgfpathmoveto{\pgfqpoint{0.984216in}{0.741795in}}%
\pgfpathlineto{\pgfqpoint{5.442272in}{0.741795in}}%
\pgfpathlineto{\pgfqpoint{5.442272in}{4.142955in}}%
\pgfpathlineto{\pgfqpoint{0.984216in}{4.142955in}}%
\pgfpathclose%
\pgfusepath{fill}%
\end{pgfscope}%
\begin{pgfscope}%
\pgfsetbuttcap%
\pgfsetroundjoin%
\definecolor{currentfill}{rgb}{0.000000,0.000000,0.000000}%
\pgfsetfillcolor{currentfill}%
\pgfsetlinewidth{0.803000pt}%
\definecolor{currentstroke}{rgb}{0.000000,0.000000,0.000000}%
\pgfsetstrokecolor{currentstroke}%
\pgfsetdash{}{0pt}%
\pgfsys@defobject{currentmarker}{\pgfqpoint{0.000000in}{-0.048611in}}{\pgfqpoint{0.000000in}{0.000000in}}{%
\pgfpathmoveto{\pgfqpoint{0.000000in}{0.000000in}}%
\pgfpathlineto{\pgfqpoint{0.000000in}{-0.048611in}}%
\pgfusepath{stroke,fill}%
}%
\begin{pgfscope}%
\pgfsys@transformshift{1.225002in}{0.741795in}%
\pgfsys@useobject{currentmarker}{}%
\end{pgfscope}%
\end{pgfscope}%
\begin{pgfscope}%
\pgftext[x=1.225002in,y=0.644572in,,top]{\rmfamily\fontsize{16.000000}{19.200000}\selectfont \(\displaystyle 0.0\)}%
\end{pgfscope}%
\begin{pgfscope}%
\pgfsetbuttcap%
\pgfsetroundjoin%
\definecolor{currentfill}{rgb}{0.000000,0.000000,0.000000}%
\pgfsetfillcolor{currentfill}%
\pgfsetlinewidth{0.803000pt}%
\definecolor{currentstroke}{rgb}{0.000000,0.000000,0.000000}%
\pgfsetstrokecolor{currentstroke}%
\pgfsetdash{}{0pt}%
\pgfsys@defobject{currentmarker}{\pgfqpoint{0.000000in}{-0.048611in}}{\pgfqpoint{0.000000in}{0.000000in}}{%
\pgfpathmoveto{\pgfqpoint{0.000000in}{0.000000in}}%
\pgfpathlineto{\pgfqpoint{0.000000in}{-0.048611in}}%
\pgfusepath{stroke,fill}%
}%
\begin{pgfscope}%
\pgfsys@transformshift{2.028291in}{0.741795in}%
\pgfsys@useobject{currentmarker}{}%
\end{pgfscope}%
\end{pgfscope}%
\begin{pgfscope}%
\pgftext[x=2.028291in,y=0.644572in,,top]{\rmfamily\fontsize{16.000000}{19.200000}\selectfont \(\displaystyle 0.2\)}%
\end{pgfscope}%
\begin{pgfscope}%
\pgfsetbuttcap%
\pgfsetroundjoin%
\definecolor{currentfill}{rgb}{0.000000,0.000000,0.000000}%
\pgfsetfillcolor{currentfill}%
\pgfsetlinewidth{0.803000pt}%
\definecolor{currentstroke}{rgb}{0.000000,0.000000,0.000000}%
\pgfsetstrokecolor{currentstroke}%
\pgfsetdash{}{0pt}%
\pgfsys@defobject{currentmarker}{\pgfqpoint{0.000000in}{-0.048611in}}{\pgfqpoint{0.000000in}{0.000000in}}{%
\pgfpathmoveto{\pgfqpoint{0.000000in}{0.000000in}}%
\pgfpathlineto{\pgfqpoint{0.000000in}{-0.048611in}}%
\pgfusepath{stroke,fill}%
}%
\begin{pgfscope}%
\pgfsys@transformshift{2.831581in}{0.741795in}%
\pgfsys@useobject{currentmarker}{}%
\end{pgfscope}%
\end{pgfscope}%
\begin{pgfscope}%
\pgftext[x=2.831581in,y=0.644572in,,top]{\rmfamily\fontsize{16.000000}{19.200000}\selectfont \(\displaystyle 0.4\)}%
\end{pgfscope}%
\begin{pgfscope}%
\pgfsetbuttcap%
\pgfsetroundjoin%
\definecolor{currentfill}{rgb}{0.000000,0.000000,0.000000}%
\pgfsetfillcolor{currentfill}%
\pgfsetlinewidth{0.803000pt}%
\definecolor{currentstroke}{rgb}{0.000000,0.000000,0.000000}%
\pgfsetstrokecolor{currentstroke}%
\pgfsetdash{}{0pt}%
\pgfsys@defobject{currentmarker}{\pgfqpoint{0.000000in}{-0.048611in}}{\pgfqpoint{0.000000in}{0.000000in}}{%
\pgfpathmoveto{\pgfqpoint{0.000000in}{0.000000in}}%
\pgfpathlineto{\pgfqpoint{0.000000in}{-0.048611in}}%
\pgfusepath{stroke,fill}%
}%
\begin{pgfscope}%
\pgfsys@transformshift{3.634870in}{0.741795in}%
\pgfsys@useobject{currentmarker}{}%
\end{pgfscope}%
\end{pgfscope}%
\begin{pgfscope}%
\pgftext[x=3.634870in,y=0.644572in,,top]{\rmfamily\fontsize{16.000000}{19.200000}\selectfont \(\displaystyle 0.6\)}%
\end{pgfscope}%
\begin{pgfscope}%
\pgfsetbuttcap%
\pgfsetroundjoin%
\definecolor{currentfill}{rgb}{0.000000,0.000000,0.000000}%
\pgfsetfillcolor{currentfill}%
\pgfsetlinewidth{0.803000pt}%
\definecolor{currentstroke}{rgb}{0.000000,0.000000,0.000000}%
\pgfsetstrokecolor{currentstroke}%
\pgfsetdash{}{0pt}%
\pgfsys@defobject{currentmarker}{\pgfqpoint{0.000000in}{-0.048611in}}{\pgfqpoint{0.000000in}{0.000000in}}{%
\pgfpathmoveto{\pgfqpoint{0.000000in}{0.000000in}}%
\pgfpathlineto{\pgfqpoint{0.000000in}{-0.048611in}}%
\pgfusepath{stroke,fill}%
}%
\begin{pgfscope}%
\pgfsys@transformshift{4.438160in}{0.741795in}%
\pgfsys@useobject{currentmarker}{}%
\end{pgfscope}%
\end{pgfscope}%
\begin{pgfscope}%
\pgftext[x=4.438160in,y=0.644572in,,top]{\rmfamily\fontsize{16.000000}{19.200000}\selectfont \(\displaystyle 0.8\)}%
\end{pgfscope}%
\begin{pgfscope}%
\pgfsetbuttcap%
\pgfsetroundjoin%
\definecolor{currentfill}{rgb}{0.000000,0.000000,0.000000}%
\pgfsetfillcolor{currentfill}%
\pgfsetlinewidth{0.803000pt}%
\definecolor{currentstroke}{rgb}{0.000000,0.000000,0.000000}%
\pgfsetstrokecolor{currentstroke}%
\pgfsetdash{}{0pt}%
\pgfsys@defobject{currentmarker}{\pgfqpoint{0.000000in}{-0.048611in}}{\pgfqpoint{0.000000in}{0.000000in}}{%
\pgfpathmoveto{\pgfqpoint{0.000000in}{0.000000in}}%
\pgfpathlineto{\pgfqpoint{0.000000in}{-0.048611in}}%
\pgfusepath{stroke,fill}%
}%
\begin{pgfscope}%
\pgfsys@transformshift{5.241450in}{0.741795in}%
\pgfsys@useobject{currentmarker}{}%
\end{pgfscope}%
\end{pgfscope}%
\begin{pgfscope}%
\pgftext[x=5.241450in,y=0.644572in,,top]{\rmfamily\fontsize{16.000000}{19.200000}\selectfont \(\displaystyle 1.0\)}%
\end{pgfscope}%
\begin{pgfscope}%
\pgfsetbuttcap%
\pgfsetroundjoin%
\definecolor{currentfill}{rgb}{0.000000,0.000000,0.000000}%
\pgfsetfillcolor{currentfill}%
\pgfsetlinewidth{0.803000pt}%
\definecolor{currentstroke}{rgb}{0.000000,0.000000,0.000000}%
\pgfsetstrokecolor{currentstroke}%
\pgfsetdash{}{0pt}%
\pgfsys@defobject{currentmarker}{\pgfqpoint{-0.048611in}{0.000000in}}{\pgfqpoint{0.000000in}{0.000000in}}{%
\pgfpathmoveto{\pgfqpoint{0.000000in}{0.000000in}}%
\pgfpathlineto{\pgfqpoint{-0.048611in}{0.000000in}}%
\pgfusepath{stroke,fill}%
}%
\begin{pgfscope}%
\pgfsys@transformshift{0.984216in}{0.741795in}%
\pgfsys@useobject{currentmarker}{}%
\end{pgfscope}%
\end{pgfscope}%
\begin{pgfscope}%
\pgftext[x=0.601580in,y=0.657376in,left,base]{\rmfamily\fontsize{16.000000}{19.200000}\selectfont \(\displaystyle 0.0\)}%
\end{pgfscope}%
\begin{pgfscope}%
\pgfsetbuttcap%
\pgfsetroundjoin%
\definecolor{currentfill}{rgb}{0.000000,0.000000,0.000000}%
\pgfsetfillcolor{currentfill}%
\pgfsetlinewidth{0.803000pt}%
\definecolor{currentstroke}{rgb}{0.000000,0.000000,0.000000}%
\pgfsetstrokecolor{currentstroke}%
\pgfsetdash{}{0pt}%
\pgfsys@defobject{currentmarker}{\pgfqpoint{-0.048611in}{0.000000in}}{\pgfqpoint{0.000000in}{0.000000in}}{%
\pgfpathmoveto{\pgfqpoint{0.000000in}{0.000000in}}%
\pgfpathlineto{\pgfqpoint{-0.048611in}{0.000000in}}%
\pgfusepath{stroke,fill}%
}%
\begin{pgfscope}%
\pgfsys@transformshift{0.984216in}{1.422027in}%
\pgfsys@useobject{currentmarker}{}%
\end{pgfscope}%
\end{pgfscope}%
\begin{pgfscope}%
\pgftext[x=0.601580in,y=1.337608in,left,base]{\rmfamily\fontsize{16.000000}{19.200000}\selectfont \(\displaystyle 0.1\)}%
\end{pgfscope}%
\begin{pgfscope}%
\pgfsetbuttcap%
\pgfsetroundjoin%
\definecolor{currentfill}{rgb}{0.000000,0.000000,0.000000}%
\pgfsetfillcolor{currentfill}%
\pgfsetlinewidth{0.803000pt}%
\definecolor{currentstroke}{rgb}{0.000000,0.000000,0.000000}%
\pgfsetstrokecolor{currentstroke}%
\pgfsetdash{}{0pt}%
\pgfsys@defobject{currentmarker}{\pgfqpoint{-0.048611in}{0.000000in}}{\pgfqpoint{0.000000in}{0.000000in}}{%
\pgfpathmoveto{\pgfqpoint{0.000000in}{0.000000in}}%
\pgfpathlineto{\pgfqpoint{-0.048611in}{0.000000in}}%
\pgfusepath{stroke,fill}%
}%
\begin{pgfscope}%
\pgfsys@transformshift{0.984216in}{2.102259in}%
\pgfsys@useobject{currentmarker}{}%
\end{pgfscope}%
\end{pgfscope}%
\begin{pgfscope}%
\pgftext[x=0.601580in,y=2.017840in,left,base]{\rmfamily\fontsize{16.000000}{19.200000}\selectfont \(\displaystyle 0.2\)}%
\end{pgfscope}%
\begin{pgfscope}%
\pgfsetbuttcap%
\pgfsetroundjoin%
\definecolor{currentfill}{rgb}{0.000000,0.000000,0.000000}%
\pgfsetfillcolor{currentfill}%
\pgfsetlinewidth{0.803000pt}%
\definecolor{currentstroke}{rgb}{0.000000,0.000000,0.000000}%
\pgfsetstrokecolor{currentstroke}%
\pgfsetdash{}{0pt}%
\pgfsys@defobject{currentmarker}{\pgfqpoint{-0.048611in}{0.000000in}}{\pgfqpoint{0.000000in}{0.000000in}}{%
\pgfpathmoveto{\pgfqpoint{0.000000in}{0.000000in}}%
\pgfpathlineto{\pgfqpoint{-0.048611in}{0.000000in}}%
\pgfusepath{stroke,fill}%
}%
\begin{pgfscope}%
\pgfsys@transformshift{0.984216in}{2.782491in}%
\pgfsys@useobject{currentmarker}{}%
\end{pgfscope}%
\end{pgfscope}%
\begin{pgfscope}%
\pgftext[x=0.601580in,y=2.698072in,left,base]{\rmfamily\fontsize{16.000000}{19.200000}\selectfont \(\displaystyle 0.3\)}%
\end{pgfscope}%
\begin{pgfscope}%
\pgfsetbuttcap%
\pgfsetroundjoin%
\definecolor{currentfill}{rgb}{0.000000,0.000000,0.000000}%
\pgfsetfillcolor{currentfill}%
\pgfsetlinewidth{0.803000pt}%
\definecolor{currentstroke}{rgb}{0.000000,0.000000,0.000000}%
\pgfsetstrokecolor{currentstroke}%
\pgfsetdash{}{0pt}%
\pgfsys@defobject{currentmarker}{\pgfqpoint{-0.048611in}{0.000000in}}{\pgfqpoint{0.000000in}{0.000000in}}{%
\pgfpathmoveto{\pgfqpoint{0.000000in}{0.000000in}}%
\pgfpathlineto{\pgfqpoint{-0.048611in}{0.000000in}}%
\pgfusepath{stroke,fill}%
}%
\begin{pgfscope}%
\pgfsys@transformshift{0.984216in}{3.462723in}%
\pgfsys@useobject{currentmarker}{}%
\end{pgfscope}%
\end{pgfscope}%
\begin{pgfscope}%
\pgftext[x=0.601580in,y=3.378304in,left,base]{\rmfamily\fontsize{16.000000}{19.200000}\selectfont \(\displaystyle 0.4\)}%
\end{pgfscope}%
\begin{pgfscope}%
\pgfsetbuttcap%
\pgfsetroundjoin%
\definecolor{currentfill}{rgb}{0.000000,0.000000,0.000000}%
\pgfsetfillcolor{currentfill}%
\pgfsetlinewidth{0.803000pt}%
\definecolor{currentstroke}{rgb}{0.000000,0.000000,0.000000}%
\pgfsetstrokecolor{currentstroke}%
\pgfsetdash{}{0pt}%
\pgfsys@defobject{currentmarker}{\pgfqpoint{-0.048611in}{0.000000in}}{\pgfqpoint{0.000000in}{0.000000in}}{%
\pgfpathmoveto{\pgfqpoint{0.000000in}{0.000000in}}%
\pgfpathlineto{\pgfqpoint{-0.048611in}{0.000000in}}%
\pgfusepath{stroke,fill}%
}%
\begin{pgfscope}%
\pgfsys@transformshift{0.984216in}{4.142955in}%
\pgfsys@useobject{currentmarker}{}%
\end{pgfscope}%
\end{pgfscope}%
\begin{pgfscope}%
\pgftext[x=0.601580in,y=4.058536in,left,base]{\rmfamily\fontsize{16.000000}{19.200000}\selectfont \(\displaystyle 0.5\)}%
\end{pgfscope}%
\begin{pgfscope}%
\pgfpathrectangle{\pgfqpoint{0.984216in}{0.741795in}}{\pgfqpoint{4.458056in}{3.401160in}} %
\pgfusepath{clip}%
\pgfsetbuttcap%
\pgfsetroundjoin%
\pgfsetlinewidth{1.505625pt}%
\definecolor{currentstroke}{rgb}{1.000000,0.000000,0.000000}%
\pgfsetstrokecolor{currentstroke}%
\pgfsetdash{{5.550000pt}{2.400000pt}}{0.000000pt}%
\pgfpathmoveto{\pgfqpoint{1.225002in}{0.741795in}}%
\pgfpathlineto{\pgfqpoint{1.273199in}{0.822690in}}%
\pgfpathlineto{\pgfqpoint{1.321396in}{0.902127in}}%
\pgfpathlineto{\pgfqpoint{1.369594in}{0.980117in}}%
\pgfpathlineto{\pgfqpoint{1.417791in}{1.056670in}}%
\pgfpathlineto{\pgfqpoint{1.465989in}{1.131797in}}%
\pgfpathlineto{\pgfqpoint{1.514186in}{1.205506in}}%
\pgfpathlineto{\pgfqpoint{1.562383in}{1.277806in}}%
\pgfpathlineto{\pgfqpoint{1.610581in}{1.348705in}}%
\pgfpathlineto{\pgfqpoint{1.658778in}{1.418212in}}%
\pgfpathlineto{\pgfqpoint{1.706975in}{1.486335in}}%
\pgfpathlineto{\pgfqpoint{1.755173in}{1.553080in}}%
\pgfpathlineto{\pgfqpoint{1.803370in}{1.618454in}}%
\pgfpathlineto{\pgfqpoint{1.847551in}{1.677182in}}%
\pgfpathlineto{\pgfqpoint{1.891732in}{1.734769in}}%
\pgfpathlineto{\pgfqpoint{1.935913in}{1.791219in}}%
\pgfpathlineto{\pgfqpoint{1.980094in}{1.846538in}}%
\pgfpathlineto{\pgfqpoint{2.024275in}{1.900728in}}%
\pgfpathlineto{\pgfqpoint{2.068456in}{1.953794in}}%
\pgfpathlineto{\pgfqpoint{2.112637in}{2.005741in}}%
\pgfpathlineto{\pgfqpoint{2.156818in}{2.056572in}}%
\pgfpathlineto{\pgfqpoint{2.200998in}{2.106290in}}%
\pgfpathlineto{\pgfqpoint{2.245179in}{2.154899in}}%
\pgfpathlineto{\pgfqpoint{2.289360in}{2.202402in}}%
\pgfpathlineto{\pgfqpoint{2.333541in}{2.248803in}}%
\pgfpathlineto{\pgfqpoint{2.377722in}{2.294104in}}%
\pgfpathlineto{\pgfqpoint{2.421903in}{2.338309in}}%
\pgfpathlineto{\pgfqpoint{2.466084in}{2.381420in}}%
\pgfpathlineto{\pgfqpoint{2.510265in}{2.423439in}}%
\pgfpathlineto{\pgfqpoint{2.554446in}{2.464370in}}%
\pgfpathlineto{\pgfqpoint{2.598627in}{2.504215in}}%
\pgfpathlineto{\pgfqpoint{2.642808in}{2.542976in}}%
\pgfpathlineto{\pgfqpoint{2.682972in}{2.577275in}}%
\pgfpathlineto{\pgfqpoint{2.723137in}{2.610681in}}%
\pgfpathlineto{\pgfqpoint{2.763301in}{2.643197in}}%
\pgfpathlineto{\pgfqpoint{2.803466in}{2.674824in}}%
\pgfpathlineto{\pgfqpoint{2.843630in}{2.705563in}}%
\pgfpathlineto{\pgfqpoint{2.883795in}{2.735416in}}%
\pgfpathlineto{\pgfqpoint{2.923959in}{2.764385in}}%
\pgfpathlineto{\pgfqpoint{2.964124in}{2.792470in}}%
\pgfpathlineto{\pgfqpoint{3.004288in}{2.819673in}}%
\pgfpathlineto{\pgfqpoint{3.044453in}{2.845995in}}%
\pgfpathlineto{\pgfqpoint{3.084617in}{2.871437in}}%
\pgfpathlineto{\pgfqpoint{3.124782in}{2.896001in}}%
\pgfpathlineto{\pgfqpoint{3.164946in}{2.919687in}}%
\pgfpathlineto{\pgfqpoint{3.205110in}{2.942498in}}%
\pgfpathlineto{\pgfqpoint{3.245275in}{2.964433in}}%
\pgfpathlineto{\pgfqpoint{3.285439in}{2.985494in}}%
\pgfpathlineto{\pgfqpoint{3.325604in}{3.005682in}}%
\pgfpathlineto{\pgfqpoint{3.365768in}{3.024998in}}%
\pgfpathlineto{\pgfqpoint{3.405933in}{3.043443in}}%
\pgfpathlineto{\pgfqpoint{3.446097in}{3.061018in}}%
\pgfpathlineto{\pgfqpoint{3.486262in}{3.077724in}}%
\pgfpathlineto{\pgfqpoint{3.526426in}{3.093562in}}%
\pgfpathlineto{\pgfqpoint{3.566591in}{3.108533in}}%
\pgfpathlineto{\pgfqpoint{3.606755in}{3.122636in}}%
\pgfpathlineto{\pgfqpoint{3.646920in}{3.135875in}}%
\pgfpathlineto{\pgfqpoint{3.683068in}{3.147050in}}%
\pgfpathlineto{\pgfqpoint{3.719216in}{3.157525in}}%
\pgfpathlineto{\pgfqpoint{3.755364in}{3.167301in}}%
\pgfpathlineto{\pgfqpoint{3.791512in}{3.176378in}}%
\pgfpathlineto{\pgfqpoint{3.827660in}{3.184757in}}%
\pgfpathlineto{\pgfqpoint{3.863808in}{3.192438in}}%
\pgfpathlineto{\pgfqpoint{3.899956in}{3.199422in}}%
\pgfpathlineto{\pgfqpoint{3.936104in}{3.205710in}}%
\pgfpathlineto{\pgfqpoint{3.972252in}{3.211301in}}%
\pgfpathlineto{\pgfqpoint{4.008400in}{3.216196in}}%
\pgfpathlineto{\pgfqpoint{4.044548in}{3.220396in}}%
\pgfpathlineto{\pgfqpoint{4.080696in}{3.223901in}}%
\pgfpathlineto{\pgfqpoint{4.116844in}{3.226712in}}%
\pgfpathlineto{\pgfqpoint{4.152992in}{3.228829in}}%
\pgfpathlineto{\pgfqpoint{4.189140in}{3.230251in}}%
\pgfpathlineto{\pgfqpoint{4.225288in}{3.230980in}}%
\pgfpathlineto{\pgfqpoint{4.261436in}{3.231016in}}%
\pgfpathlineto{\pgfqpoint{4.297584in}{3.230359in}}%
\pgfpathlineto{\pgfqpoint{4.333732in}{3.229010in}}%
\pgfpathlineto{\pgfqpoint{4.369880in}{3.226967in}}%
\pgfpathlineto{\pgfqpoint{4.406028in}{3.224233in}}%
\pgfpathlineto{\pgfqpoint{4.442176in}{3.220806in}}%
\pgfpathlineto{\pgfqpoint{4.478324in}{3.216687in}}%
\pgfpathlineto{\pgfqpoint{4.514473in}{3.211877in}}%
\pgfpathlineto{\pgfqpoint{4.550621in}{3.206374in}}%
\pgfpathlineto{\pgfqpoint{4.586769in}{3.200179in}}%
\pgfpathlineto{\pgfqpoint{4.622917in}{3.193292in}}%
\pgfpathlineto{\pgfqpoint{4.659065in}{3.185713in}}%
\pgfpathlineto{\pgfqpoint{4.695213in}{3.177441in}}%
\pgfpathlineto{\pgfqpoint{4.731361in}{3.168477in}}%
\pgfpathlineto{\pgfqpoint{4.767509in}{3.158820in}}%
\pgfpathlineto{\pgfqpoint{4.803657in}{3.148471in}}%
\pgfpathlineto{\pgfqpoint{4.843821in}{3.136157in}}%
\pgfpathlineto{\pgfqpoint{4.883986in}{3.122988in}}%
\pgfpathlineto{\pgfqpoint{4.924150in}{3.108961in}}%
\pgfpathlineto{\pgfqpoint{4.964315in}{3.094076in}}%
\pgfpathlineto{\pgfqpoint{5.004479in}{3.078332in}}%
\pgfpathlineto{\pgfqpoint{5.044644in}{3.061728in}}%
\pgfpathlineto{\pgfqpoint{5.084808in}{3.044264in}}%
\pgfpathlineto{\pgfqpoint{5.124973in}{3.025938in}}%
\pgfpathlineto{\pgfqpoint{5.165137in}{3.006749in}}%
\pgfpathlineto{\pgfqpoint{5.205302in}{2.986696in}}%
\pgfpathlineto{\pgfqpoint{5.245466in}{2.965778in}}%
\pgfpathlineto{\pgfqpoint{5.285631in}{2.943992in}}%
\pgfpathlineto{\pgfqpoint{5.325795in}{2.921338in}}%
\pgfpathlineto{\pgfqpoint{5.365959in}{2.897814in}}%
\pgfpathlineto{\pgfqpoint{5.406124in}{2.873417in}}%
\pgfpathlineto{\pgfqpoint{5.445605in}{2.848584in}}%
\pgfpathlineto{\pgfqpoint{5.445605in}{2.848584in}}%
\pgfusepath{stroke}%
\end{pgfscope}%
\begin{pgfscope}%
\pgfpathrectangle{\pgfqpoint{0.984216in}{0.741795in}}{\pgfqpoint{4.458056in}{3.401160in}} %
\pgfusepath{clip}%
\pgfsetbuttcap%
\pgfsetmiterjoin%
\definecolor{currentfill}{rgb}{1.000000,0.000000,0.000000}%
\pgfsetfillcolor{currentfill}%
\pgfsetlinewidth{1.003750pt}%
\definecolor{currentstroke}{rgb}{1.000000,0.000000,0.000000}%
\pgfsetstrokecolor{currentstroke}%
\pgfsetdash{}{0pt}%
\pgfsys@defobject{currentmarker}{\pgfqpoint{-0.041667in}{-0.041667in}}{\pgfqpoint{0.041667in}{0.041667in}}{%
\pgfpathmoveto{\pgfqpoint{-0.041667in}{-0.041667in}}%
\pgfpathlineto{\pgfqpoint{0.041667in}{-0.041667in}}%
\pgfpathlineto{\pgfqpoint{0.041667in}{0.041667in}}%
\pgfpathlineto{\pgfqpoint{-0.041667in}{0.041667in}}%
\pgfpathclose%
\pgfusepath{stroke,fill}%
}%
\begin{pgfscope}%
\pgfsys@transformshift{1.225002in}{0.741795in}%
\pgfsys@useobject{currentmarker}{}%
\end{pgfscope}%
\begin{pgfscope}%
\pgfsys@transformshift{1.626646in}{1.372029in}%
\pgfsys@useobject{currentmarker}{}%
\end{pgfscope}%
\begin{pgfscope}%
\pgfsys@transformshift{2.028291in}{1.905598in}%
\pgfsys@useobject{currentmarker}{}%
\end{pgfscope}%
\begin{pgfscope}%
\pgfsys@transformshift{2.429936in}{2.346228in}%
\pgfsys@useobject{currentmarker}{}%
\end{pgfscope}%
\begin{pgfscope}%
\pgfsys@transformshift{2.831581in}{2.696435in}%
\pgfsys@useobject{currentmarker}{}%
\end{pgfscope}%
\begin{pgfscope}%
\pgfsys@transformshift{3.233226in}{2.957944in}%
\pgfsys@useobject{currentmarker}{}%
\end{pgfscope}%
\begin{pgfscope}%
\pgfsys@transformshift{3.634870in}{3.131994in}%
\pgfsys@useobject{currentmarker}{}%
\end{pgfscope}%
\begin{pgfscope}%
\pgfsys@transformshift{4.036515in}{3.219523in}%
\pgfsys@useobject{currentmarker}{}%
\end{pgfscope}%
\begin{pgfscope}%
\pgfsys@transformshift{4.438160in}{3.221221in}%
\pgfsys@useobject{currentmarker}{}%
\end{pgfscope}%
\begin{pgfscope}%
\pgfsys@transformshift{4.839805in}{3.137427in}%
\pgfsys@useobject{currentmarker}{}%
\end{pgfscope}%
\begin{pgfscope}%
\pgfsys@transformshift{5.241450in}{2.967909in}%
\pgfsys@useobject{currentmarker}{}%
\end{pgfscope}%
\begin{pgfscope}%
\pgfsys@transformshift{5.643094in}{2.711624in}%
\pgfsys@useobject{currentmarker}{}%
\end{pgfscope}%
\end{pgfscope}%
\begin{pgfscope}%
\pgfpathrectangle{\pgfqpoint{0.984216in}{0.741795in}}{\pgfqpoint{4.458056in}{3.401160in}} %
\pgfusepath{clip}%
\pgfsetrectcap%
\pgfsetroundjoin%
\pgfsetlinewidth{1.505625pt}%
\definecolor{currentstroke}{rgb}{0.000000,0.000000,1.000000}%
\pgfsetstrokecolor{currentstroke}%
\pgfsetdash{}{0pt}%
\pgfpathmoveto{\pgfqpoint{1.225002in}{0.741795in}}%
\pgfpathlineto{\pgfqpoint{1.273199in}{0.822688in}}%
\pgfpathlineto{\pgfqpoint{1.321396in}{0.902113in}}%
\pgfpathlineto{\pgfqpoint{1.369594in}{0.980071in}}%
\pgfpathlineto{\pgfqpoint{1.417791in}{1.056564in}}%
\pgfpathlineto{\pgfqpoint{1.465989in}{1.131591in}}%
\pgfpathlineto{\pgfqpoint{1.514186in}{1.205155in}}%
\pgfpathlineto{\pgfqpoint{1.562383in}{1.277256in}}%
\pgfpathlineto{\pgfqpoint{1.606564in}{1.342065in}}%
\pgfpathlineto{\pgfqpoint{1.650745in}{1.405646in}}%
\pgfpathlineto{\pgfqpoint{1.694926in}{1.468000in}}%
\pgfpathlineto{\pgfqpoint{1.739107in}{1.529129in}}%
\pgfpathlineto{\pgfqpoint{1.783288in}{1.589032in}}%
\pgfpathlineto{\pgfqpoint{1.827469in}{1.647711in}}%
\pgfpathlineto{\pgfqpoint{1.871650in}{1.705166in}}%
\pgfpathlineto{\pgfqpoint{1.915831in}{1.761397in}}%
\pgfpathlineto{\pgfqpoint{1.960012in}{1.816406in}}%
\pgfpathlineto{\pgfqpoint{2.004193in}{1.870193in}}%
\pgfpathlineto{\pgfqpoint{2.048373in}{1.922759in}}%
\pgfpathlineto{\pgfqpoint{2.092554in}{1.974104in}}%
\pgfpathlineto{\pgfqpoint{2.136735in}{2.024229in}}%
\pgfpathlineto{\pgfqpoint{2.180916in}{2.073135in}}%
\pgfpathlineto{\pgfqpoint{2.221081in}{2.116536in}}%
\pgfpathlineto{\pgfqpoint{2.261245in}{2.158931in}}%
\pgfpathlineto{\pgfqpoint{2.301410in}{2.200319in}}%
\pgfpathlineto{\pgfqpoint{2.341574in}{2.240701in}}%
\pgfpathlineto{\pgfqpoint{2.381739in}{2.280077in}}%
\pgfpathlineto{\pgfqpoint{2.421903in}{2.318448in}}%
\pgfpathlineto{\pgfqpoint{2.462068in}{2.355813in}}%
\pgfpathlineto{\pgfqpoint{2.502232in}{2.392174in}}%
\pgfpathlineto{\pgfqpoint{2.542397in}{2.427530in}}%
\pgfpathlineto{\pgfqpoint{2.582561in}{2.461882in}}%
\pgfpathlineto{\pgfqpoint{2.622726in}{2.495231in}}%
\pgfpathlineto{\pgfqpoint{2.662890in}{2.527576in}}%
\pgfpathlineto{\pgfqpoint{2.703054in}{2.558917in}}%
\pgfpathlineto{\pgfqpoint{2.743219in}{2.589256in}}%
\pgfpathlineto{\pgfqpoint{2.783383in}{2.618592in}}%
\pgfpathlineto{\pgfqpoint{2.823548in}{2.646926in}}%
\pgfpathlineto{\pgfqpoint{2.859696in}{2.671570in}}%
\pgfpathlineto{\pgfqpoint{2.895844in}{2.695402in}}%
\pgfpathlineto{\pgfqpoint{2.931992in}{2.718423in}}%
\pgfpathlineto{\pgfqpoint{2.968140in}{2.740633in}}%
\pgfpathlineto{\pgfqpoint{3.004288in}{2.762032in}}%
\pgfpathlineto{\pgfqpoint{3.040436in}{2.782620in}}%
\pgfpathlineto{\pgfqpoint{3.076584in}{2.802397in}}%
\pgfpathlineto{\pgfqpoint{3.112732in}{2.821364in}}%
\pgfpathlineto{\pgfqpoint{3.148880in}{2.839521in}}%
\pgfpathlineto{\pgfqpoint{3.185028in}{2.856867in}}%
\pgfpathlineto{\pgfqpoint{3.221176in}{2.873403in}}%
\pgfpathlineto{\pgfqpoint{3.257324in}{2.889129in}}%
\pgfpathlineto{\pgfqpoint{3.293472in}{2.904045in}}%
\pgfpathlineto{\pgfqpoint{3.329620in}{2.918152in}}%
\pgfpathlineto{\pgfqpoint{3.365768in}{2.931448in}}%
\pgfpathlineto{\pgfqpoint{3.401916in}{2.943936in}}%
\pgfpathlineto{\pgfqpoint{3.438064in}{2.955613in}}%
\pgfpathlineto{\pgfqpoint{3.474212in}{2.966482in}}%
\pgfpathlineto{\pgfqpoint{3.510361in}{2.976541in}}%
\pgfpathlineto{\pgfqpoint{3.546509in}{2.985790in}}%
\pgfpathlineto{\pgfqpoint{3.582657in}{2.994231in}}%
\pgfpathlineto{\pgfqpoint{3.618805in}{3.001863in}}%
\pgfpathlineto{\pgfqpoint{3.654953in}{3.008685in}}%
\pgfpathlineto{\pgfqpoint{3.691101in}{3.014699in}}%
\pgfpathlineto{\pgfqpoint{3.727249in}{3.019903in}}%
\pgfpathlineto{\pgfqpoint{3.763397in}{3.024299in}}%
\pgfpathlineto{\pgfqpoint{3.799545in}{3.027885in}}%
\pgfpathlineto{\pgfqpoint{3.835693in}{3.030663in}}%
\pgfpathlineto{\pgfqpoint{3.871841in}{3.032632in}}%
\pgfpathlineto{\pgfqpoint{3.907989in}{3.033792in}}%
\pgfpathlineto{\pgfqpoint{3.944137in}{3.034144in}}%
\pgfpathlineto{\pgfqpoint{3.980285in}{3.033686in}}%
\pgfpathlineto{\pgfqpoint{4.016433in}{3.032420in}}%
\pgfpathlineto{\pgfqpoint{4.052581in}{3.030345in}}%
\pgfpathlineto{\pgfqpoint{4.088729in}{3.027461in}}%
\pgfpathlineto{\pgfqpoint{4.124877in}{3.023768in}}%
\pgfpathlineto{\pgfqpoint{4.161025in}{3.019267in}}%
\pgfpathlineto{\pgfqpoint{4.197173in}{3.013956in}}%
\pgfpathlineto{\pgfqpoint{4.233321in}{3.007837in}}%
\pgfpathlineto{\pgfqpoint{4.269469in}{3.000908in}}%
\pgfpathlineto{\pgfqpoint{4.305617in}{2.993170in}}%
\pgfpathlineto{\pgfqpoint{4.341765in}{2.984624in}}%
\pgfpathlineto{\pgfqpoint{4.377913in}{2.975268in}}%
\pgfpathlineto{\pgfqpoint{4.414061in}{2.965103in}}%
\pgfpathlineto{\pgfqpoint{4.450209in}{2.954128in}}%
\pgfpathlineto{\pgfqpoint{4.486357in}{2.942344in}}%
\pgfpathlineto{\pgfqpoint{4.522505in}{2.929751in}}%
\pgfpathlineto{\pgfqpoint{4.558653in}{2.916348in}}%
\pgfpathlineto{\pgfqpoint{4.594801in}{2.902135in}}%
\pgfpathlineto{\pgfqpoint{4.630949in}{2.887113in}}%
\pgfpathlineto{\pgfqpoint{4.667098in}{2.871281in}}%
\pgfpathlineto{\pgfqpoint{4.703246in}{2.854639in}}%
\pgfpathlineto{\pgfqpoint{4.739394in}{2.837186in}}%
\pgfpathlineto{\pgfqpoint{4.775542in}{2.818923in}}%
\pgfpathlineto{\pgfqpoint{4.811690in}{2.799850in}}%
\pgfpathlineto{\pgfqpoint{4.847838in}{2.779967in}}%
\pgfpathlineto{\pgfqpoint{4.883986in}{2.759272in}}%
\pgfpathlineto{\pgfqpoint{4.920134in}{2.737767in}}%
\pgfpathlineto{\pgfqpoint{4.956282in}{2.715451in}}%
\pgfpathlineto{\pgfqpoint{4.992430in}{2.692324in}}%
\pgfpathlineto{\pgfqpoint{5.028578in}{2.668385in}}%
\pgfpathlineto{\pgfqpoint{5.064726in}{2.643635in}}%
\pgfpathlineto{\pgfqpoint{5.104890in}{2.615183in}}%
\pgfpathlineto{\pgfqpoint{5.145055in}{2.585728in}}%
\pgfpathlineto{\pgfqpoint{5.185219in}{2.555271in}}%
\pgfpathlineto{\pgfqpoint{5.225384in}{2.523811in}}%
\pgfpathlineto{\pgfqpoint{5.265548in}{2.491348in}}%
\pgfpathlineto{\pgfqpoint{5.305713in}{2.457881in}}%
\pgfpathlineto{\pgfqpoint{5.345877in}{2.423410in}}%
\pgfpathlineto{\pgfqpoint{5.386042in}{2.387935in}}%
\pgfpathlineto{\pgfqpoint{5.426206in}{2.351456in}}%
\pgfpathlineto{\pgfqpoint{5.445605in}{2.333476in}}%
\pgfpathlineto{\pgfqpoint{5.445605in}{2.333476in}}%
\pgfusepath{stroke}%
\end{pgfscope}%
\begin{pgfscope}%
\pgfpathrectangle{\pgfqpoint{0.984216in}{0.741795in}}{\pgfqpoint{4.458056in}{3.401160in}} %
\pgfusepath{clip}%
\pgfsetbuttcap%
\pgfsetroundjoin%
\definecolor{currentfill}{rgb}{0.000000,0.000000,1.000000}%
\pgfsetfillcolor{currentfill}%
\pgfsetlinewidth{1.003750pt}%
\definecolor{currentstroke}{rgb}{0.000000,0.000000,1.000000}%
\pgfsetstrokecolor{currentstroke}%
\pgfsetdash{}{0pt}%
\pgfsys@defobject{currentmarker}{\pgfqpoint{-0.041667in}{-0.041667in}}{\pgfqpoint{0.041667in}{0.041667in}}{%
\pgfpathmoveto{\pgfqpoint{0.000000in}{-0.041667in}}%
\pgfpathcurveto{\pgfqpoint{0.011050in}{-0.041667in}}{\pgfqpoint{0.021649in}{-0.037276in}}{\pgfqpoint{0.029463in}{-0.029463in}}%
\pgfpathcurveto{\pgfqpoint{0.037276in}{-0.021649in}}{\pgfqpoint{0.041667in}{-0.011050in}}{\pgfqpoint{0.041667in}{0.000000in}}%
\pgfpathcurveto{\pgfqpoint{0.041667in}{0.011050in}}{\pgfqpoint{0.037276in}{0.021649in}}{\pgfqpoint{0.029463in}{0.029463in}}%
\pgfpathcurveto{\pgfqpoint{0.021649in}{0.037276in}}{\pgfqpoint{0.011050in}{0.041667in}}{\pgfqpoint{0.000000in}{0.041667in}}%
\pgfpathcurveto{\pgfqpoint{-0.011050in}{0.041667in}}{\pgfqpoint{-0.021649in}{0.037276in}}{\pgfqpoint{-0.029463in}{0.029463in}}%
\pgfpathcurveto{\pgfqpoint{-0.037276in}{0.021649in}}{\pgfqpoint{-0.041667in}{0.011050in}}{\pgfqpoint{-0.041667in}{0.000000in}}%
\pgfpathcurveto{\pgfqpoint{-0.041667in}{-0.011050in}}{\pgfqpoint{-0.037276in}{-0.021649in}}{\pgfqpoint{-0.029463in}{-0.029463in}}%
\pgfpathcurveto{\pgfqpoint{-0.021649in}{-0.037276in}}{\pgfqpoint{-0.011050in}{-0.041667in}}{\pgfqpoint{0.000000in}{-0.041667in}}%
\pgfpathclose%
\pgfusepath{stroke,fill}%
}%
\begin{pgfscope}%
\pgfsys@transformshift{1.225002in}{0.741795in}%
\pgfsys@useobject{currentmarker}{}%
\end{pgfscope}%
\begin{pgfscope}%
\pgfsys@transformshift{1.626646in}{1.371118in}%
\pgfsys@useobject{currentmarker}{}%
\end{pgfscope}%
\begin{pgfscope}%
\pgfsys@transformshift{2.028291in}{1.899017in}%
\pgfsys@useobject{currentmarker}{}%
\end{pgfscope}%
\begin{pgfscope}%
\pgfsys@transformshift{2.429936in}{2.326001in}%
\pgfsys@useobject{currentmarker}{}%
\end{pgfscope}%
\begin{pgfscope}%
\pgfsys@transformshift{2.831581in}{2.652473in}%
\pgfsys@useobject{currentmarker}{}%
\end{pgfscope}%
\begin{pgfscope}%
\pgfsys@transformshift{3.233226in}{2.878735in}%
\pgfsys@useobject{currentmarker}{}%
\end{pgfscope}%
\begin{pgfscope}%
\pgfsys@transformshift{3.634870in}{3.004995in}%
\pgfsys@useobject{currentmarker}{}%
\end{pgfscope}%
\begin{pgfscope}%
\pgfsys@transformshift{4.036515in}{3.031367in}%
\pgfsys@useobject{currentmarker}{}%
\end{pgfscope}%
\begin{pgfscope}%
\pgfsys@transformshift{4.438160in}{2.957876in}%
\pgfsys@useobject{currentmarker}{}%
\end{pgfscope}%
\begin{pgfscope}%
\pgfsys@transformshift{4.839805in}{2.784455in}%
\pgfsys@useobject{currentmarker}{}%
\end{pgfscope}%
\begin{pgfscope}%
\pgfsys@transformshift{5.241450in}{2.510946in}%
\pgfsys@useobject{currentmarker}{}%
\end{pgfscope}%
\begin{pgfscope}%
\pgfsys@transformshift{5.643094in}{2.137096in}%
\pgfsys@useobject{currentmarker}{}%
\end{pgfscope}%
\end{pgfscope}%
\begin{pgfscope}%
\pgfpathrectangle{\pgfqpoint{0.984216in}{0.741795in}}{\pgfqpoint{4.458056in}{3.401160in}} %
\pgfusepath{clip}%
\pgfsetbuttcap%
\pgfsetroundjoin%
\pgfsetlinewidth{1.505625pt}%
\definecolor{currentstroke}{rgb}{0.000000,0.750000,0.750000}%
\pgfsetstrokecolor{currentstroke}%
\pgfsetdash{{9.600000pt}{2.400000pt}{1.500000pt}{2.400000pt}}{0.000000pt}%
\pgfpathmoveto{\pgfqpoint{1.225002in}{0.741795in}}%
\pgfpathlineto{\pgfqpoint{1.273199in}{0.822688in}}%
\pgfpathlineto{\pgfqpoint{1.321396in}{0.902112in}}%
\pgfpathlineto{\pgfqpoint{1.369594in}{0.980067in}}%
\pgfpathlineto{\pgfqpoint{1.417791in}{1.056553in}}%
\pgfpathlineto{\pgfqpoint{1.465989in}{1.131570in}}%
\pgfpathlineto{\pgfqpoint{1.514186in}{1.205118in}}%
\pgfpathlineto{\pgfqpoint{1.558367in}{1.271247in}}%
\pgfpathlineto{\pgfqpoint{1.602548in}{1.336143in}}%
\pgfpathlineto{\pgfqpoint{1.646729in}{1.399804in}}%
\pgfpathlineto{\pgfqpoint{1.690910in}{1.462232in}}%
\pgfpathlineto{\pgfqpoint{1.735091in}{1.523425in}}%
\pgfpathlineto{\pgfqpoint{1.779271in}{1.583385in}}%
\pgfpathlineto{\pgfqpoint{1.823452in}{1.642112in}}%
\pgfpathlineto{\pgfqpoint{1.867633in}{1.699605in}}%
\pgfpathlineto{\pgfqpoint{1.911814in}{1.755864in}}%
\pgfpathlineto{\pgfqpoint{1.955995in}{1.810890in}}%
\pgfpathlineto{\pgfqpoint{2.000176in}{1.864683in}}%
\pgfpathlineto{\pgfqpoint{2.044357in}{1.917243in}}%
\pgfpathlineto{\pgfqpoint{2.088538in}{1.968569in}}%
\pgfpathlineto{\pgfqpoint{2.132719in}{2.018662in}}%
\pgfpathlineto{\pgfqpoint{2.172883in}{2.063131in}}%
\pgfpathlineto{\pgfqpoint{2.213048in}{2.106581in}}%
\pgfpathlineto{\pgfqpoint{2.253212in}{2.149013in}}%
\pgfpathlineto{\pgfqpoint{2.293377in}{2.190425in}}%
\pgfpathlineto{\pgfqpoint{2.333541in}{2.230818in}}%
\pgfpathlineto{\pgfqpoint{2.373706in}{2.270193in}}%
\pgfpathlineto{\pgfqpoint{2.413870in}{2.308548in}}%
\pgfpathlineto{\pgfqpoint{2.454035in}{2.345885in}}%
\pgfpathlineto{\pgfqpoint{2.494199in}{2.382203in}}%
\pgfpathlineto{\pgfqpoint{2.534364in}{2.417503in}}%
\pgfpathlineto{\pgfqpoint{2.574528in}{2.451783in}}%
\pgfpathlineto{\pgfqpoint{2.614693in}{2.485045in}}%
\pgfpathlineto{\pgfqpoint{2.654857in}{2.517289in}}%
\pgfpathlineto{\pgfqpoint{2.695022in}{2.548514in}}%
\pgfpathlineto{\pgfqpoint{2.735186in}{2.578720in}}%
\pgfpathlineto{\pgfqpoint{2.775351in}{2.607908in}}%
\pgfpathlineto{\pgfqpoint{2.811499in}{2.633306in}}%
\pgfpathlineto{\pgfqpoint{2.847647in}{2.657879in}}%
\pgfpathlineto{\pgfqpoint{2.883795in}{2.681627in}}%
\pgfpathlineto{\pgfqpoint{2.919943in}{2.704551in}}%
\pgfpathlineto{\pgfqpoint{2.956091in}{2.726649in}}%
\pgfpathlineto{\pgfqpoint{2.992239in}{2.747923in}}%
\pgfpathlineto{\pgfqpoint{3.028387in}{2.768372in}}%
\pgfpathlineto{\pgfqpoint{3.064535in}{2.787996in}}%
\pgfpathlineto{\pgfqpoint{3.100683in}{2.806795in}}%
\pgfpathlineto{\pgfqpoint{3.136831in}{2.824769in}}%
\pgfpathlineto{\pgfqpoint{3.172979in}{2.841918in}}%
\pgfpathlineto{\pgfqpoint{3.209127in}{2.858243in}}%
\pgfpathlineto{\pgfqpoint{3.245275in}{2.873743in}}%
\pgfpathlineto{\pgfqpoint{3.281423in}{2.888418in}}%
\pgfpathlineto{\pgfqpoint{3.317571in}{2.902269in}}%
\pgfpathlineto{\pgfqpoint{3.353719in}{2.915294in}}%
\pgfpathlineto{\pgfqpoint{3.389867in}{2.927495in}}%
\pgfpathlineto{\pgfqpoint{3.426015in}{2.938871in}}%
\pgfpathlineto{\pgfqpoint{3.462163in}{2.949423in}}%
\pgfpathlineto{\pgfqpoint{3.498311in}{2.959150in}}%
\pgfpathlineto{\pgfqpoint{3.534459in}{2.968052in}}%
\pgfpathlineto{\pgfqpoint{3.570607in}{2.976129in}}%
\pgfpathlineto{\pgfqpoint{3.606755in}{2.983382in}}%
\pgfpathlineto{\pgfqpoint{3.642903in}{2.989810in}}%
\pgfpathlineto{\pgfqpoint{3.679051in}{2.995413in}}%
\pgfpathlineto{\pgfqpoint{3.715199in}{3.000192in}}%
\pgfpathlineto{\pgfqpoint{3.751347in}{3.004146in}}%
\pgfpathlineto{\pgfqpoint{3.787495in}{3.007275in}}%
\pgfpathlineto{\pgfqpoint{3.823643in}{3.009580in}}%
\pgfpathlineto{\pgfqpoint{3.859791in}{3.011060in}}%
\pgfpathlineto{\pgfqpoint{3.895940in}{3.011715in}}%
\pgfpathlineto{\pgfqpoint{3.932088in}{3.011546in}}%
\pgfpathlineto{\pgfqpoint{3.968236in}{3.010552in}}%
\pgfpathlineto{\pgfqpoint{4.004384in}{3.008734in}}%
\pgfpathlineto{\pgfqpoint{4.040532in}{3.006091in}}%
\pgfpathlineto{\pgfqpoint{4.076680in}{3.002623in}}%
\pgfpathlineto{\pgfqpoint{4.112828in}{2.998330in}}%
\pgfpathlineto{\pgfqpoint{4.148976in}{2.993213in}}%
\pgfpathlineto{\pgfqpoint{4.185124in}{2.987271in}}%
\pgfpathlineto{\pgfqpoint{4.221272in}{2.980505in}}%
\pgfpathlineto{\pgfqpoint{4.257420in}{2.972913in}}%
\pgfpathlineto{\pgfqpoint{4.293568in}{2.964498in}}%
\pgfpathlineto{\pgfqpoint{4.329716in}{2.955257in}}%
\pgfpathlineto{\pgfqpoint{4.365864in}{2.945192in}}%
\pgfpathlineto{\pgfqpoint{4.402012in}{2.934302in}}%
\pgfpathlineto{\pgfqpoint{4.438160in}{2.922587in}}%
\pgfpathlineto{\pgfqpoint{4.474308in}{2.910048in}}%
\pgfpathlineto{\pgfqpoint{4.510456in}{2.896684in}}%
\pgfpathlineto{\pgfqpoint{4.546604in}{2.882495in}}%
\pgfpathlineto{\pgfqpoint{4.582752in}{2.867481in}}%
\pgfpathlineto{\pgfqpoint{4.618900in}{2.851643in}}%
\pgfpathlineto{\pgfqpoint{4.655048in}{2.834979in}}%
\pgfpathlineto{\pgfqpoint{4.691196in}{2.817491in}}%
\pgfpathlineto{\pgfqpoint{4.727344in}{2.799179in}}%
\pgfpathlineto{\pgfqpoint{4.763492in}{2.780041in}}%
\pgfpathlineto{\pgfqpoint{4.799640in}{2.760079in}}%
\pgfpathlineto{\pgfqpoint{4.835788in}{2.739291in}}%
\pgfpathlineto{\pgfqpoint{4.871936in}{2.717679in}}%
\pgfpathlineto{\pgfqpoint{4.908084in}{2.695242in}}%
\pgfpathlineto{\pgfqpoint{4.944232in}{2.671980in}}%
\pgfpathlineto{\pgfqpoint{4.980380in}{2.647893in}}%
\pgfpathlineto{\pgfqpoint{5.016528in}{2.622981in}}%
\pgfpathlineto{\pgfqpoint{5.052677in}{2.597244in}}%
\pgfpathlineto{\pgfqpoint{5.088825in}{2.570683in}}%
\pgfpathlineto{\pgfqpoint{5.128989in}{2.540202in}}%
\pgfpathlineto{\pgfqpoint{5.169154in}{2.508703in}}%
\pgfpathlineto{\pgfqpoint{5.209318in}{2.476185in}}%
\pgfpathlineto{\pgfqpoint{5.249482in}{2.442648in}}%
\pgfpathlineto{\pgfqpoint{5.289647in}{2.408093in}}%
\pgfpathlineto{\pgfqpoint{5.329811in}{2.372519in}}%
\pgfpathlineto{\pgfqpoint{5.369976in}{2.335927in}}%
\pgfpathlineto{\pgfqpoint{5.410140in}{2.298315in}}%
\pgfpathlineto{\pgfqpoint{5.445605in}{2.264257in}}%
\pgfpathlineto{\pgfqpoint{5.445605in}{2.264257in}}%
\pgfusepath{stroke}%
\end{pgfscope}%
\begin{pgfscope}%
\pgfpathrectangle{\pgfqpoint{0.984216in}{0.741795in}}{\pgfqpoint{4.458056in}{3.401160in}} %
\pgfusepath{clip}%
\pgfsetbuttcap%
\pgfsetmiterjoin%
\definecolor{currentfill}{rgb}{0.000000,0.750000,0.750000}%
\pgfsetfillcolor{currentfill}%
\pgfsetlinewidth{1.003750pt}%
\definecolor{currentstroke}{rgb}{0.000000,0.750000,0.750000}%
\pgfsetstrokecolor{currentstroke}%
\pgfsetdash{}{0pt}%
\pgfsys@defobject{currentmarker}{\pgfqpoint{-0.041667in}{-0.041667in}}{\pgfqpoint{0.041667in}{0.041667in}}{%
\pgfpathmoveto{\pgfqpoint{-0.000000in}{-0.041667in}}%
\pgfpathlineto{\pgfqpoint{0.041667in}{0.041667in}}%
\pgfpathlineto{\pgfqpoint{-0.041667in}{0.041667in}}%
\pgfpathclose%
\pgfusepath{stroke,fill}%
}%
\begin{pgfscope}%
\pgfsys@transformshift{1.225002in}{0.741795in}%
\pgfsys@useobject{currentmarker}{}%
\end{pgfscope}%
\begin{pgfscope}%
\pgfsys@transformshift{1.626646in}{1.371020in}%
\pgfsys@useobject{currentmarker}{}%
\end{pgfscope}%
\begin{pgfscope}%
\pgfsys@transformshift{2.028291in}{1.898273in}%
\pgfsys@useobject{currentmarker}{}%
\end{pgfscope}%
\begin{pgfscope}%
\pgfsys@transformshift{2.429936in}{2.323605in}%
\pgfsys@useobject{currentmarker}{}%
\end{pgfscope}%
\begin{pgfscope}%
\pgfsys@transformshift{2.831581in}{2.647060in}%
\pgfsys@useobject{currentmarker}{}%
\end{pgfscope}%
\begin{pgfscope}%
\pgfsys@transformshift{3.233226in}{2.868668in}%
\pgfsys@useobject{currentmarker}{}%
\end{pgfscope}%
\begin{pgfscope}%
\pgfsys@transformshift{3.634870in}{2.988453in}%
\pgfsys@useobject{currentmarker}{}%
\end{pgfscope}%
\begin{pgfscope}%
\pgfsys@transformshift{4.036515in}{3.006425in}%
\pgfsys@useobject{currentmarker}{}%
\end{pgfscope}%
\begin{pgfscope}%
\pgfsys@transformshift{4.438160in}{2.922587in}%
\pgfsys@useobject{currentmarker}{}%
\end{pgfscope}%
\begin{pgfscope}%
\pgfsys@transformshift{4.839805in}{2.736931in}%
\pgfsys@useobject{currentmarker}{}%
\end{pgfscope}%
\begin{pgfscope}%
\pgfsys@transformshift{5.241450in}{2.449437in}%
\pgfsys@useobject{currentmarker}{}%
\end{pgfscope}%
\begin{pgfscope}%
\pgfsys@transformshift{5.643094in}{2.060078in}%
\pgfsys@useobject{currentmarker}{}%
\end{pgfscope}%
\end{pgfscope}%
\begin{pgfscope}%
\pgfsetrectcap%
\pgfsetmiterjoin%
\pgfsetlinewidth{0.803000pt}%
\definecolor{currentstroke}{rgb}{0.000000,0.000000,0.000000}%
\pgfsetstrokecolor{currentstroke}%
\pgfsetdash{}{0pt}%
\pgfpathmoveto{\pgfqpoint{0.984216in}{0.741795in}}%
\pgfpathlineto{\pgfqpoint{0.984216in}{4.142955in}}%
\pgfusepath{stroke}%
\end{pgfscope}%
\begin{pgfscope}%
\pgfsetrectcap%
\pgfsetmiterjoin%
\pgfsetlinewidth{0.803000pt}%
\definecolor{currentstroke}{rgb}{0.000000,0.000000,0.000000}%
\pgfsetstrokecolor{currentstroke}%
\pgfsetdash{}{0pt}%
\pgfpathmoveto{\pgfqpoint{5.442272in}{0.741795in}}%
\pgfpathlineto{\pgfqpoint{5.442272in}{4.142955in}}%
\pgfusepath{stroke}%
\end{pgfscope}%
\begin{pgfscope}%
\pgfsetrectcap%
\pgfsetmiterjoin%
\pgfsetlinewidth{0.803000pt}%
\definecolor{currentstroke}{rgb}{0.000000,0.000000,0.000000}%
\pgfsetstrokecolor{currentstroke}%
\pgfsetdash{}{0pt}%
\pgfpathmoveto{\pgfqpoint{0.984216in}{0.741795in}}%
\pgfpathlineto{\pgfqpoint{5.442272in}{0.741795in}}%
\pgfusepath{stroke}%
\end{pgfscope}%
\begin{pgfscope}%
\pgfsetrectcap%
\pgfsetmiterjoin%
\pgfsetlinewidth{0.803000pt}%
\definecolor{currentstroke}{rgb}{0.000000,0.000000,0.000000}%
\pgfsetstrokecolor{currentstroke}%
\pgfsetdash{}{0pt}%
\pgfpathmoveto{\pgfqpoint{0.984216in}{4.142955in}}%
\pgfpathlineto{\pgfqpoint{5.442272in}{4.142955in}}%
\pgfusepath{stroke}%
\end{pgfscope}%
\begin{pgfscope}%
\pgfsetbuttcap%
\pgfsetmiterjoin%
\definecolor{currentfill}{rgb}{1.000000,1.000000,1.000000}%
\pgfsetfillcolor{currentfill}%
\pgfsetfillopacity{0.800000}%
\pgfsetlinewidth{1.003750pt}%
\definecolor{currentstroke}{rgb}{0.800000,0.800000,0.800000}%
\pgfsetstrokecolor{currentstroke}%
\pgfsetstrokeopacity{0.800000}%
\pgfsetdash{}{0pt}%
\pgfpathmoveto{\pgfqpoint{1.120327in}{3.602023in}}%
\pgfpathlineto{\pgfqpoint{1.992439in}{3.602023in}}%
\pgfpathquadraticcurveto{\pgfqpoint{2.031328in}{3.602023in}}{\pgfqpoint{2.031328in}{3.640912in}}%
\pgfpathlineto{\pgfqpoint{2.031328in}{4.006844in}}%
\pgfpathquadraticcurveto{\pgfqpoint{2.031328in}{4.045733in}}{\pgfqpoint{1.992439in}{4.045733in}}%
\pgfpathlineto{\pgfqpoint{1.120327in}{4.045733in}}%
\pgfpathquadraticcurveto{\pgfqpoint{1.081438in}{4.045733in}}{\pgfqpoint{1.081438in}{4.006844in}}%
\pgfpathlineto{\pgfqpoint{1.081438in}{3.640912in}}%
\pgfpathquadraticcurveto{\pgfqpoint{1.081438in}{3.602023in}}{\pgfqpoint{1.120327in}{3.602023in}}%
\pgfpathclose%
\pgfusepath{stroke,fill}%
\end{pgfscope}%
\begin{pgfscope}%
\pgftext[x=1.159216in,y=3.820223in,left,base]{\rmfamily\fontsize{14.000000}{16.800000}\selectfont \(\displaystyle \mathbf{I}\mbox{g} = \) 0.5}%
\end{pgfscope}%
\begin{pgfscope}%
\pgfsetbuttcap%
\pgfsetmiterjoin%
\definecolor{currentfill}{rgb}{1.000000,1.000000,1.000000}%
\pgfsetfillcolor{currentfill}%
\pgfsetlinewidth{0.000000pt}%
\definecolor{currentstroke}{rgb}{0.000000,0.000000,0.000000}%
\pgfsetstrokecolor{currentstroke}%
\pgfsetstrokeopacity{0.000000}%
\pgfsetdash{}{0pt}%
\pgfpathmoveto{\pgfqpoint{5.697941in}{0.741795in}}%
\pgfpathlineto{\pgfqpoint{10.155998in}{0.741795in}}%
\pgfpathlineto{\pgfqpoint{10.155998in}{4.142955in}}%
\pgfpathlineto{\pgfqpoint{5.697941in}{4.142955in}}%
\pgfpathclose%
\pgfusepath{fill}%
\end{pgfscope}%
\begin{pgfscope}%
\pgfsetbuttcap%
\pgfsetroundjoin%
\definecolor{currentfill}{rgb}{0.000000,0.000000,0.000000}%
\pgfsetfillcolor{currentfill}%
\pgfsetlinewidth{0.803000pt}%
\definecolor{currentstroke}{rgb}{0.000000,0.000000,0.000000}%
\pgfsetstrokecolor{currentstroke}%
\pgfsetdash{}{0pt}%
\pgfsys@defobject{currentmarker}{\pgfqpoint{0.000000in}{-0.048611in}}{\pgfqpoint{0.000000in}{0.000000in}}{%
\pgfpathmoveto{\pgfqpoint{0.000000in}{0.000000in}}%
\pgfpathlineto{\pgfqpoint{0.000000in}{-0.048611in}}%
\pgfusepath{stroke,fill}%
}%
\begin{pgfscope}%
\pgfsys@transformshift{5.938727in}{0.741795in}%
\pgfsys@useobject{currentmarker}{}%
\end{pgfscope}%
\end{pgfscope}%
\begin{pgfscope}%
\pgftext[x=5.938727in,y=0.644572in,,top]{\rmfamily\fontsize{16.000000}{19.200000}\selectfont \(\displaystyle 0.0\)}%
\end{pgfscope}%
\begin{pgfscope}%
\pgfsetbuttcap%
\pgfsetroundjoin%
\definecolor{currentfill}{rgb}{0.000000,0.000000,0.000000}%
\pgfsetfillcolor{currentfill}%
\pgfsetlinewidth{0.803000pt}%
\definecolor{currentstroke}{rgb}{0.000000,0.000000,0.000000}%
\pgfsetstrokecolor{currentstroke}%
\pgfsetdash{}{0pt}%
\pgfsys@defobject{currentmarker}{\pgfqpoint{0.000000in}{-0.048611in}}{\pgfqpoint{0.000000in}{0.000000in}}{%
\pgfpathmoveto{\pgfqpoint{0.000000in}{0.000000in}}%
\pgfpathlineto{\pgfqpoint{0.000000in}{-0.048611in}}%
\pgfusepath{stroke,fill}%
}%
\begin{pgfscope}%
\pgfsys@transformshift{6.742017in}{0.741795in}%
\pgfsys@useobject{currentmarker}{}%
\end{pgfscope}%
\end{pgfscope}%
\begin{pgfscope}%
\pgftext[x=6.742017in,y=0.644572in,,top]{\rmfamily\fontsize{16.000000}{19.200000}\selectfont \(\displaystyle 0.2\)}%
\end{pgfscope}%
\begin{pgfscope}%
\pgfsetbuttcap%
\pgfsetroundjoin%
\definecolor{currentfill}{rgb}{0.000000,0.000000,0.000000}%
\pgfsetfillcolor{currentfill}%
\pgfsetlinewidth{0.803000pt}%
\definecolor{currentstroke}{rgb}{0.000000,0.000000,0.000000}%
\pgfsetstrokecolor{currentstroke}%
\pgfsetdash{}{0pt}%
\pgfsys@defobject{currentmarker}{\pgfqpoint{0.000000in}{-0.048611in}}{\pgfqpoint{0.000000in}{0.000000in}}{%
\pgfpathmoveto{\pgfqpoint{0.000000in}{0.000000in}}%
\pgfpathlineto{\pgfqpoint{0.000000in}{-0.048611in}}%
\pgfusepath{stroke,fill}%
}%
\begin{pgfscope}%
\pgfsys@transformshift{7.545306in}{0.741795in}%
\pgfsys@useobject{currentmarker}{}%
\end{pgfscope}%
\end{pgfscope}%
\begin{pgfscope}%
\pgftext[x=7.545306in,y=0.644572in,,top]{\rmfamily\fontsize{16.000000}{19.200000}\selectfont \(\displaystyle 0.4\)}%
\end{pgfscope}%
\begin{pgfscope}%
\pgfsetbuttcap%
\pgfsetroundjoin%
\definecolor{currentfill}{rgb}{0.000000,0.000000,0.000000}%
\pgfsetfillcolor{currentfill}%
\pgfsetlinewidth{0.803000pt}%
\definecolor{currentstroke}{rgb}{0.000000,0.000000,0.000000}%
\pgfsetstrokecolor{currentstroke}%
\pgfsetdash{}{0pt}%
\pgfsys@defobject{currentmarker}{\pgfqpoint{0.000000in}{-0.048611in}}{\pgfqpoint{0.000000in}{0.000000in}}{%
\pgfpathmoveto{\pgfqpoint{0.000000in}{0.000000in}}%
\pgfpathlineto{\pgfqpoint{0.000000in}{-0.048611in}}%
\pgfusepath{stroke,fill}%
}%
\begin{pgfscope}%
\pgfsys@transformshift{8.348596in}{0.741795in}%
\pgfsys@useobject{currentmarker}{}%
\end{pgfscope}%
\end{pgfscope}%
\begin{pgfscope}%
\pgftext[x=8.348596in,y=0.644572in,,top]{\rmfamily\fontsize{16.000000}{19.200000}\selectfont \(\displaystyle 0.6\)}%
\end{pgfscope}%
\begin{pgfscope}%
\pgfsetbuttcap%
\pgfsetroundjoin%
\definecolor{currentfill}{rgb}{0.000000,0.000000,0.000000}%
\pgfsetfillcolor{currentfill}%
\pgfsetlinewidth{0.803000pt}%
\definecolor{currentstroke}{rgb}{0.000000,0.000000,0.000000}%
\pgfsetstrokecolor{currentstroke}%
\pgfsetdash{}{0pt}%
\pgfsys@defobject{currentmarker}{\pgfqpoint{0.000000in}{-0.048611in}}{\pgfqpoint{0.000000in}{0.000000in}}{%
\pgfpathmoveto{\pgfqpoint{0.000000in}{0.000000in}}%
\pgfpathlineto{\pgfqpoint{0.000000in}{-0.048611in}}%
\pgfusepath{stroke,fill}%
}%
\begin{pgfscope}%
\pgfsys@transformshift{9.151886in}{0.741795in}%
\pgfsys@useobject{currentmarker}{}%
\end{pgfscope}%
\end{pgfscope}%
\begin{pgfscope}%
\pgftext[x=9.151886in,y=0.644572in,,top]{\rmfamily\fontsize{16.000000}{19.200000}\selectfont \(\displaystyle 0.8\)}%
\end{pgfscope}%
\begin{pgfscope}%
\pgfsetbuttcap%
\pgfsetroundjoin%
\definecolor{currentfill}{rgb}{0.000000,0.000000,0.000000}%
\pgfsetfillcolor{currentfill}%
\pgfsetlinewidth{0.803000pt}%
\definecolor{currentstroke}{rgb}{0.000000,0.000000,0.000000}%
\pgfsetstrokecolor{currentstroke}%
\pgfsetdash{}{0pt}%
\pgfsys@defobject{currentmarker}{\pgfqpoint{0.000000in}{-0.048611in}}{\pgfqpoint{0.000000in}{0.000000in}}{%
\pgfpathmoveto{\pgfqpoint{0.000000in}{0.000000in}}%
\pgfpathlineto{\pgfqpoint{0.000000in}{-0.048611in}}%
\pgfusepath{stroke,fill}%
}%
\begin{pgfscope}%
\pgfsys@transformshift{9.955175in}{0.741795in}%
\pgfsys@useobject{currentmarker}{}%
\end{pgfscope}%
\end{pgfscope}%
\begin{pgfscope}%
\pgftext[x=9.955175in,y=0.644572in,,top]{\rmfamily\fontsize{16.000000}{19.200000}\selectfont \(\displaystyle 1.0\)}%
\end{pgfscope}%
\begin{pgfscope}%
\pgfsetbuttcap%
\pgfsetroundjoin%
\definecolor{currentfill}{rgb}{0.000000,0.000000,0.000000}%
\pgfsetfillcolor{currentfill}%
\pgfsetlinewidth{0.803000pt}%
\definecolor{currentstroke}{rgb}{0.000000,0.000000,0.000000}%
\pgfsetstrokecolor{currentstroke}%
\pgfsetdash{}{0pt}%
\pgfsys@defobject{currentmarker}{\pgfqpoint{-0.048611in}{0.000000in}}{\pgfqpoint{0.000000in}{0.000000in}}{%
\pgfpathmoveto{\pgfqpoint{0.000000in}{0.000000in}}%
\pgfpathlineto{\pgfqpoint{-0.048611in}{0.000000in}}%
\pgfusepath{stroke,fill}%
}%
\begin{pgfscope}%
\pgfsys@transformshift{5.697941in}{0.741795in}%
\pgfsys@useobject{currentmarker}{}%
\end{pgfscope}%
\end{pgfscope}%
\begin{pgfscope}%
\pgfsetbuttcap%
\pgfsetroundjoin%
\definecolor{currentfill}{rgb}{0.000000,0.000000,0.000000}%
\pgfsetfillcolor{currentfill}%
\pgfsetlinewidth{0.803000pt}%
\definecolor{currentstroke}{rgb}{0.000000,0.000000,0.000000}%
\pgfsetstrokecolor{currentstroke}%
\pgfsetdash{}{0pt}%
\pgfsys@defobject{currentmarker}{\pgfqpoint{-0.048611in}{0.000000in}}{\pgfqpoint{0.000000in}{0.000000in}}{%
\pgfpathmoveto{\pgfqpoint{0.000000in}{0.000000in}}%
\pgfpathlineto{\pgfqpoint{-0.048611in}{0.000000in}}%
\pgfusepath{stroke,fill}%
}%
\begin{pgfscope}%
\pgfsys@transformshift{5.697941in}{1.422027in}%
\pgfsys@useobject{currentmarker}{}%
\end{pgfscope}%
\end{pgfscope}%
\begin{pgfscope}%
\pgfsetbuttcap%
\pgfsetroundjoin%
\definecolor{currentfill}{rgb}{0.000000,0.000000,0.000000}%
\pgfsetfillcolor{currentfill}%
\pgfsetlinewidth{0.803000pt}%
\definecolor{currentstroke}{rgb}{0.000000,0.000000,0.000000}%
\pgfsetstrokecolor{currentstroke}%
\pgfsetdash{}{0pt}%
\pgfsys@defobject{currentmarker}{\pgfqpoint{-0.048611in}{0.000000in}}{\pgfqpoint{0.000000in}{0.000000in}}{%
\pgfpathmoveto{\pgfqpoint{0.000000in}{0.000000in}}%
\pgfpathlineto{\pgfqpoint{-0.048611in}{0.000000in}}%
\pgfusepath{stroke,fill}%
}%
\begin{pgfscope}%
\pgfsys@transformshift{5.697941in}{2.102259in}%
\pgfsys@useobject{currentmarker}{}%
\end{pgfscope}%
\end{pgfscope}%
\begin{pgfscope}%
\pgfsetbuttcap%
\pgfsetroundjoin%
\definecolor{currentfill}{rgb}{0.000000,0.000000,0.000000}%
\pgfsetfillcolor{currentfill}%
\pgfsetlinewidth{0.803000pt}%
\definecolor{currentstroke}{rgb}{0.000000,0.000000,0.000000}%
\pgfsetstrokecolor{currentstroke}%
\pgfsetdash{}{0pt}%
\pgfsys@defobject{currentmarker}{\pgfqpoint{-0.048611in}{0.000000in}}{\pgfqpoint{0.000000in}{0.000000in}}{%
\pgfpathmoveto{\pgfqpoint{0.000000in}{0.000000in}}%
\pgfpathlineto{\pgfqpoint{-0.048611in}{0.000000in}}%
\pgfusepath{stroke,fill}%
}%
\begin{pgfscope}%
\pgfsys@transformshift{5.697941in}{2.782491in}%
\pgfsys@useobject{currentmarker}{}%
\end{pgfscope}%
\end{pgfscope}%
\begin{pgfscope}%
\pgfsetbuttcap%
\pgfsetroundjoin%
\definecolor{currentfill}{rgb}{0.000000,0.000000,0.000000}%
\pgfsetfillcolor{currentfill}%
\pgfsetlinewidth{0.803000pt}%
\definecolor{currentstroke}{rgb}{0.000000,0.000000,0.000000}%
\pgfsetstrokecolor{currentstroke}%
\pgfsetdash{}{0pt}%
\pgfsys@defobject{currentmarker}{\pgfqpoint{-0.048611in}{0.000000in}}{\pgfqpoint{0.000000in}{0.000000in}}{%
\pgfpathmoveto{\pgfqpoint{0.000000in}{0.000000in}}%
\pgfpathlineto{\pgfqpoint{-0.048611in}{0.000000in}}%
\pgfusepath{stroke,fill}%
}%
\begin{pgfscope}%
\pgfsys@transformshift{5.697941in}{3.462723in}%
\pgfsys@useobject{currentmarker}{}%
\end{pgfscope}%
\end{pgfscope}%
\begin{pgfscope}%
\pgfsetbuttcap%
\pgfsetroundjoin%
\definecolor{currentfill}{rgb}{0.000000,0.000000,0.000000}%
\pgfsetfillcolor{currentfill}%
\pgfsetlinewidth{0.803000pt}%
\definecolor{currentstroke}{rgb}{0.000000,0.000000,0.000000}%
\pgfsetstrokecolor{currentstroke}%
\pgfsetdash{}{0pt}%
\pgfsys@defobject{currentmarker}{\pgfqpoint{-0.048611in}{0.000000in}}{\pgfqpoint{0.000000in}{0.000000in}}{%
\pgfpathmoveto{\pgfqpoint{0.000000in}{0.000000in}}%
\pgfpathlineto{\pgfqpoint{-0.048611in}{0.000000in}}%
\pgfusepath{stroke,fill}%
}%
\begin{pgfscope}%
\pgfsys@transformshift{5.697941in}{4.142955in}%
\pgfsys@useobject{currentmarker}{}%
\end{pgfscope}%
\end{pgfscope}%
\begin{pgfscope}%
\pgfpathrectangle{\pgfqpoint{5.697941in}{0.741795in}}{\pgfqpoint{4.458056in}{3.401160in}} %
\pgfusepath{clip}%
\pgfsetbuttcap%
\pgfsetroundjoin%
\pgfsetlinewidth{1.505625pt}%
\definecolor{currentstroke}{rgb}{1.000000,0.000000,0.000000}%
\pgfsetstrokecolor{currentstroke}%
\pgfsetdash{{5.550000pt}{2.400000pt}}{0.000000pt}%
\pgfpathmoveto{\pgfqpoint{5.938727in}{0.741795in}}%
\pgfpathlineto{\pgfqpoint{5.998974in}{0.843056in}}%
\pgfpathlineto{\pgfqpoint{6.055204in}{0.936173in}}%
\pgfpathlineto{\pgfqpoint{6.111434in}{1.027944in}}%
\pgfpathlineto{\pgfqpoint{6.167665in}{1.118370in}}%
\pgfpathlineto{\pgfqpoint{6.223895in}{1.207450in}}%
\pgfpathlineto{\pgfqpoint{6.280125in}{1.295185in}}%
\pgfpathlineto{\pgfqpoint{6.336356in}{1.381576in}}%
\pgfpathlineto{\pgfqpoint{6.392586in}{1.466622in}}%
\pgfpathlineto{\pgfqpoint{6.448816in}{1.550325in}}%
\pgfpathlineto{\pgfqpoint{6.505046in}{1.632683in}}%
\pgfpathlineto{\pgfqpoint{6.561277in}{1.713697in}}%
\pgfpathlineto{\pgfqpoint{6.617507in}{1.793368in}}%
\pgfpathlineto{\pgfqpoint{6.673737in}{1.871696in}}%
\pgfpathlineto{\pgfqpoint{6.729967in}{1.948681in}}%
\pgfpathlineto{\pgfqpoint{6.786198in}{2.024323in}}%
\pgfpathlineto{\pgfqpoint{6.838412in}{2.093359in}}%
\pgfpathlineto{\pgfqpoint{6.890625in}{2.161238in}}%
\pgfpathlineto{\pgfqpoint{6.942839in}{2.227960in}}%
\pgfpathlineto{\pgfqpoint{6.995053in}{2.293524in}}%
\pgfpathlineto{\pgfqpoint{7.047267in}{2.357931in}}%
\pgfpathlineto{\pgfqpoint{7.099481in}{2.421181in}}%
\pgfpathlineto{\pgfqpoint{7.151694in}{2.483273in}}%
\pgfpathlineto{\pgfqpoint{7.203908in}{2.544209in}}%
\pgfpathlineto{\pgfqpoint{7.256122in}{2.603988in}}%
\pgfpathlineto{\pgfqpoint{7.308336in}{2.662610in}}%
\pgfpathlineto{\pgfqpoint{7.360550in}{2.720076in}}%
\pgfpathlineto{\pgfqpoint{7.412764in}{2.776385in}}%
\pgfpathlineto{\pgfqpoint{7.464977in}{2.831537in}}%
\pgfpathlineto{\pgfqpoint{7.517191in}{2.885533in}}%
\pgfpathlineto{\pgfqpoint{7.565389in}{2.934349in}}%
\pgfpathlineto{\pgfqpoint{7.613586in}{2.982180in}}%
\pgfpathlineto{\pgfqpoint{7.661783in}{3.029025in}}%
\pgfpathlineto{\pgfqpoint{7.709981in}{3.074885in}}%
\pgfpathlineto{\pgfqpoint{7.758178in}{3.119760in}}%
\pgfpathlineto{\pgfqpoint{7.806375in}{3.163650in}}%
\pgfpathlineto{\pgfqpoint{7.854573in}{3.206555in}}%
\pgfpathlineto{\pgfqpoint{7.902770in}{3.248475in}}%
\pgfpathlineto{\pgfqpoint{7.950968in}{3.289410in}}%
\pgfpathlineto{\pgfqpoint{7.999165in}{3.329360in}}%
\pgfpathlineto{\pgfqpoint{8.047362in}{3.368325in}}%
\pgfpathlineto{\pgfqpoint{8.095560in}{3.406304in}}%
\pgfpathlineto{\pgfqpoint{8.143757in}{3.443299in}}%
\pgfpathlineto{\pgfqpoint{8.191954in}{3.479309in}}%
\pgfpathlineto{\pgfqpoint{8.240152in}{3.514334in}}%
\pgfpathlineto{\pgfqpoint{8.288349in}{3.548374in}}%
\pgfpathlineto{\pgfqpoint{8.332530in}{3.578712in}}%
\pgfpathlineto{\pgfqpoint{8.376711in}{3.608223in}}%
\pgfpathlineto{\pgfqpoint{8.420892in}{3.636906in}}%
\pgfpathlineto{\pgfqpoint{8.465073in}{3.664761in}}%
\pgfpathlineto{\pgfqpoint{8.509254in}{3.691789in}}%
\pgfpathlineto{\pgfqpoint{8.553435in}{3.717990in}}%
\pgfpathlineto{\pgfqpoint{8.597616in}{3.743362in}}%
\pgfpathlineto{\pgfqpoint{8.641797in}{3.767907in}}%
\pgfpathlineto{\pgfqpoint{8.685978in}{3.791625in}}%
\pgfpathlineto{\pgfqpoint{8.730159in}{3.814515in}}%
\pgfpathlineto{\pgfqpoint{8.774339in}{3.836577in}}%
\pgfpathlineto{\pgfqpoint{8.818520in}{3.857812in}}%
\pgfpathlineto{\pgfqpoint{8.862701in}{3.878219in}}%
\pgfpathlineto{\pgfqpoint{8.906882in}{3.897799in}}%
\pgfpathlineto{\pgfqpoint{8.951063in}{3.916551in}}%
\pgfpathlineto{\pgfqpoint{8.995244in}{3.934476in}}%
\pgfpathlineto{\pgfqpoint{9.039425in}{3.951573in}}%
\pgfpathlineto{\pgfqpoint{9.083606in}{3.967842in}}%
\pgfpathlineto{\pgfqpoint{9.127787in}{3.983284in}}%
\pgfpathlineto{\pgfqpoint{9.171968in}{3.997898in}}%
\pgfpathlineto{\pgfqpoint{9.216149in}{4.011684in}}%
\pgfpathlineto{\pgfqpoint{9.260330in}{4.024643in}}%
\pgfpathlineto{\pgfqpoint{9.304511in}{4.036774in}}%
\pgfpathlineto{\pgfqpoint{9.348691in}{4.048078in}}%
\pgfpathlineto{\pgfqpoint{9.392872in}{4.058554in}}%
\pgfpathlineto{\pgfqpoint{9.437053in}{4.068203in}}%
\pgfpathlineto{\pgfqpoint{9.481234in}{4.077023in}}%
\pgfpathlineto{\pgfqpoint{9.525415in}{4.085017in}}%
\pgfpathlineto{\pgfqpoint{9.565580in}{4.091565in}}%
\pgfpathlineto{\pgfqpoint{9.605744in}{4.097429in}}%
\pgfpathlineto{\pgfqpoint{9.645909in}{4.102610in}}%
\pgfpathlineto{\pgfqpoint{9.686073in}{4.107106in}}%
\pgfpathlineto{\pgfqpoint{9.726238in}{4.110918in}}%
\pgfpathlineto{\pgfqpoint{9.766402in}{4.114047in}}%
\pgfpathlineto{\pgfqpoint{9.806567in}{4.116491in}}%
\pgfpathlineto{\pgfqpoint{9.846731in}{4.118251in}}%
\pgfpathlineto{\pgfqpoint{9.886896in}{4.119328in}}%
\pgfpathlineto{\pgfqpoint{9.927060in}{4.119720in}}%
\pgfpathlineto{\pgfqpoint{9.967224in}{4.119428in}}%
\pgfpathlineto{\pgfqpoint{10.007389in}{4.118453in}}%
\pgfpathlineto{\pgfqpoint{10.047553in}{4.116793in}}%
\pgfpathlineto{\pgfqpoint{10.087718in}{4.114450in}}%
\pgfpathlineto{\pgfqpoint{10.127882in}{4.111422in}}%
\pgfpathlineto{\pgfqpoint{10.159331in}{4.108573in}}%
\pgfpathlineto{\pgfqpoint{10.159331in}{4.108573in}}%
\pgfusepath{stroke}%
\end{pgfscope}%
\begin{pgfscope}%
\pgfpathrectangle{\pgfqpoint{5.697941in}{0.741795in}}{\pgfqpoint{4.458056in}{3.401160in}} %
\pgfusepath{clip}%
\pgfsetbuttcap%
\pgfsetmiterjoin%
\definecolor{currentfill}{rgb}{1.000000,0.000000,0.000000}%
\pgfsetfillcolor{currentfill}%
\pgfsetlinewidth{1.003750pt}%
\definecolor{currentstroke}{rgb}{1.000000,0.000000,0.000000}%
\pgfsetstrokecolor{currentstroke}%
\pgfsetdash{}{0pt}%
\pgfsys@defobject{currentmarker}{\pgfqpoint{-0.041667in}{-0.041667in}}{\pgfqpoint{0.041667in}{0.041667in}}{%
\pgfpathmoveto{\pgfqpoint{-0.041667in}{-0.041667in}}%
\pgfpathlineto{\pgfqpoint{0.041667in}{-0.041667in}}%
\pgfpathlineto{\pgfqpoint{0.041667in}{0.041667in}}%
\pgfpathlineto{\pgfqpoint{-0.041667in}{0.041667in}}%
\pgfpathclose%
\pgfusepath{stroke,fill}%
}%
\begin{pgfscope}%
\pgfsys@transformshift{5.938727in}{0.741795in}%
\pgfsys@useobject{currentmarker}{}%
\end{pgfscope}%
\begin{pgfscope}%
\pgfsys@transformshift{6.340372in}{1.387695in}%
\pgfsys@useobject{currentmarker}{}%
\end{pgfscope}%
\begin{pgfscope}%
\pgfsys@transformshift{6.742017in}{1.965003in}%
\pgfsys@useobject{currentmarker}{}%
\end{pgfscope}%
\begin{pgfscope}%
\pgfsys@transformshift{7.143662in}{2.473796in}%
\pgfsys@useobject{currentmarker}{}%
\end{pgfscope}%
\begin{pgfscope}%
\pgfsys@transformshift{7.545306in}{2.914129in}%
\pgfsys@useobject{currentmarker}{}%
\end{pgfscope}%
\begin{pgfscope}%
\pgfsys@transformshift{7.946951in}{3.286036in}%
\pgfsys@useobject{currentmarker}{}%
\end{pgfscope}%
\begin{pgfscope}%
\pgfsys@transformshift{8.348596in}{3.589539in}%
\pgfsys@useobject{currentmarker}{}%
\end{pgfscope}%
\begin{pgfscope}%
\pgfsys@transformshift{8.750241in}{3.824646in}%
\pgfsys@useobject{currentmarker}{}%
\end{pgfscope}%
\begin{pgfscope}%
\pgfsys@transformshift{9.151886in}{3.991358in}%
\pgfsys@useobject{currentmarker}{}%
\end{pgfscope}%
\begin{pgfscope}%
\pgfsys@transformshift{9.553530in}{4.089672in}%
\pgfsys@useobject{currentmarker}{}%
\end{pgfscope}%
\begin{pgfscope}%
\pgfsys@transformshift{9.955175in}{4.119588in}%
\pgfsys@useobject{currentmarker}{}%
\end{pgfscope}%
\begin{pgfscope}%
\pgfsys@transformshift{10.356820in}{4.081104in}%
\pgfsys@useobject{currentmarker}{}%
\end{pgfscope}%
\end{pgfscope}%
\begin{pgfscope}%
\pgfpathrectangle{\pgfqpoint{5.697941in}{0.741795in}}{\pgfqpoint{4.458056in}{3.401160in}} %
\pgfusepath{clip}%
\pgfsetrectcap%
\pgfsetroundjoin%
\pgfsetlinewidth{1.505625pt}%
\definecolor{currentstroke}{rgb}{0.000000,0.000000,1.000000}%
\pgfsetstrokecolor{currentstroke}%
\pgfsetdash{}{0pt}%
\pgfpathmoveto{\pgfqpoint{5.938727in}{0.741795in}}%
\pgfpathlineto{\pgfqpoint{5.998974in}{0.843056in}}%
\pgfpathlineto{\pgfqpoint{6.055204in}{0.936173in}}%
\pgfpathlineto{\pgfqpoint{6.111434in}{1.027943in}}%
\pgfpathlineto{\pgfqpoint{6.167665in}{1.118366in}}%
\pgfpathlineto{\pgfqpoint{6.223895in}{1.207443in}}%
\pgfpathlineto{\pgfqpoint{6.280125in}{1.295174in}}%
\pgfpathlineto{\pgfqpoint{6.336356in}{1.381558in}}%
\pgfpathlineto{\pgfqpoint{6.392586in}{1.466596in}}%
\pgfpathlineto{\pgfqpoint{6.448816in}{1.550288in}}%
\pgfpathlineto{\pgfqpoint{6.505046in}{1.632633in}}%
\pgfpathlineto{\pgfqpoint{6.561277in}{1.713632in}}%
\pgfpathlineto{\pgfqpoint{6.617507in}{1.793285in}}%
\pgfpathlineto{\pgfqpoint{6.673737in}{1.871592in}}%
\pgfpathlineto{\pgfqpoint{6.729967in}{1.948552in}}%
\pgfpathlineto{\pgfqpoint{6.782181in}{2.018810in}}%
\pgfpathlineto{\pgfqpoint{6.834395in}{2.087908in}}%
\pgfpathlineto{\pgfqpoint{6.886609in}{2.155844in}}%
\pgfpathlineto{\pgfqpoint{6.938823in}{2.222620in}}%
\pgfpathlineto{\pgfqpoint{6.991037in}{2.288236in}}%
\pgfpathlineto{\pgfqpoint{7.043250in}{2.352691in}}%
\pgfpathlineto{\pgfqpoint{7.095464in}{2.415985in}}%
\pgfpathlineto{\pgfqpoint{7.147678in}{2.478119in}}%
\pgfpathlineto{\pgfqpoint{7.199892in}{2.539092in}}%
\pgfpathlineto{\pgfqpoint{7.252106in}{2.598905in}}%
\pgfpathlineto{\pgfqpoint{7.304319in}{2.657557in}}%
\pgfpathlineto{\pgfqpoint{7.356533in}{2.715049in}}%
\pgfpathlineto{\pgfqpoint{7.408747in}{2.771380in}}%
\pgfpathlineto{\pgfqpoint{7.460961in}{2.826551in}}%
\pgfpathlineto{\pgfqpoint{7.513175in}{2.880562in}}%
\pgfpathlineto{\pgfqpoint{7.561372in}{2.929388in}}%
\pgfpathlineto{\pgfqpoint{7.609570in}{2.977225in}}%
\pgfpathlineto{\pgfqpoint{7.657767in}{3.024074in}}%
\pgfpathlineto{\pgfqpoint{7.705964in}{3.069933in}}%
\pgfpathlineto{\pgfqpoint{7.754162in}{3.114804in}}%
\pgfpathlineto{\pgfqpoint{7.802359in}{3.158687in}}%
\pgfpathlineto{\pgfqpoint{7.850556in}{3.201581in}}%
\pgfpathlineto{\pgfqpoint{7.898754in}{3.243486in}}%
\pgfpathlineto{\pgfqpoint{7.946951in}{3.284402in}}%
\pgfpathlineto{\pgfqpoint{7.995149in}{3.324330in}}%
\pgfpathlineto{\pgfqpoint{8.043346in}{3.363269in}}%
\pgfpathlineto{\pgfqpoint{8.091543in}{3.401220in}}%
\pgfpathlineto{\pgfqpoint{8.139741in}{3.438181in}}%
\pgfpathlineto{\pgfqpoint{8.187938in}{3.474155in}}%
\pgfpathlineto{\pgfqpoint{8.236135in}{3.509140in}}%
\pgfpathlineto{\pgfqpoint{8.284333in}{3.543136in}}%
\pgfpathlineto{\pgfqpoint{8.328514in}{3.573430in}}%
\pgfpathlineto{\pgfqpoint{8.372695in}{3.602895in}}%
\pgfpathlineto{\pgfqpoint{8.416876in}{3.631528in}}%
\pgfpathlineto{\pgfqpoint{8.461056in}{3.659331in}}%
\pgfpathlineto{\pgfqpoint{8.505237in}{3.686303in}}%
\pgfpathlineto{\pgfqpoint{8.549418in}{3.712444in}}%
\pgfpathlineto{\pgfqpoint{8.593599in}{3.737755in}}%
\pgfpathlineto{\pgfqpoint{8.637780in}{3.762235in}}%
\pgfpathlineto{\pgfqpoint{8.681961in}{3.785885in}}%
\pgfpathlineto{\pgfqpoint{8.726142in}{3.808704in}}%
\pgfpathlineto{\pgfqpoint{8.770323in}{3.830692in}}%
\pgfpathlineto{\pgfqpoint{8.814504in}{3.851850in}}%
\pgfpathlineto{\pgfqpoint{8.858685in}{3.872177in}}%
\pgfpathlineto{\pgfqpoint{8.902866in}{3.891673in}}%
\pgfpathlineto{\pgfqpoint{8.947047in}{3.910339in}}%
\pgfpathlineto{\pgfqpoint{8.991228in}{3.928175in}}%
\pgfpathlineto{\pgfqpoint{9.035409in}{3.945179in}}%
\pgfpathlineto{\pgfqpoint{9.079589in}{3.961354in}}%
\pgfpathlineto{\pgfqpoint{9.123770in}{3.976697in}}%
\pgfpathlineto{\pgfqpoint{9.167951in}{3.991210in}}%
\pgfpathlineto{\pgfqpoint{9.212132in}{4.004892in}}%
\pgfpathlineto{\pgfqpoint{9.256313in}{4.017744in}}%
\pgfpathlineto{\pgfqpoint{9.300494in}{4.029765in}}%
\pgfpathlineto{\pgfqpoint{9.344675in}{4.040956in}}%
\pgfpathlineto{\pgfqpoint{9.388856in}{4.051316in}}%
\pgfpathlineto{\pgfqpoint{9.433037in}{4.060846in}}%
\pgfpathlineto{\pgfqpoint{9.477218in}{4.069545in}}%
\pgfpathlineto{\pgfqpoint{9.517382in}{4.076732in}}%
\pgfpathlineto{\pgfqpoint{9.557547in}{4.083233in}}%
\pgfpathlineto{\pgfqpoint{9.597711in}{4.089048in}}%
\pgfpathlineto{\pgfqpoint{9.637876in}{4.094176in}}%
\pgfpathlineto{\pgfqpoint{9.678040in}{4.098618in}}%
\pgfpathlineto{\pgfqpoint{9.718205in}{4.102373in}}%
\pgfpathlineto{\pgfqpoint{9.758369in}{4.105442in}}%
\pgfpathlineto{\pgfqpoint{9.798534in}{4.107825in}}%
\pgfpathlineto{\pgfqpoint{9.838698in}{4.109521in}}%
\pgfpathlineto{\pgfqpoint{9.878863in}{4.110530in}}%
\pgfpathlineto{\pgfqpoint{9.919027in}{4.110854in}}%
\pgfpathlineto{\pgfqpoint{9.959192in}{4.110491in}}%
\pgfpathlineto{\pgfqpoint{9.999356in}{4.109441in}}%
\pgfpathlineto{\pgfqpoint{10.039521in}{4.107705in}}%
\pgfpathlineto{\pgfqpoint{10.079685in}{4.105283in}}%
\pgfpathlineto{\pgfqpoint{10.119849in}{4.102174in}}%
\pgfpathlineto{\pgfqpoint{10.159331in}{4.098449in}}%
\pgfpathlineto{\pgfqpoint{10.159331in}{4.098449in}}%
\pgfusepath{stroke}%
\end{pgfscope}%
\begin{pgfscope}%
\pgfpathrectangle{\pgfqpoint{5.697941in}{0.741795in}}{\pgfqpoint{4.458056in}{3.401160in}} %
\pgfusepath{clip}%
\pgfsetbuttcap%
\pgfsetroundjoin%
\definecolor{currentfill}{rgb}{0.000000,0.000000,1.000000}%
\pgfsetfillcolor{currentfill}%
\pgfsetlinewidth{1.003750pt}%
\definecolor{currentstroke}{rgb}{0.000000,0.000000,1.000000}%
\pgfsetstrokecolor{currentstroke}%
\pgfsetdash{}{0pt}%
\pgfsys@defobject{currentmarker}{\pgfqpoint{-0.041667in}{-0.041667in}}{\pgfqpoint{0.041667in}{0.041667in}}{%
\pgfpathmoveto{\pgfqpoint{0.000000in}{-0.041667in}}%
\pgfpathcurveto{\pgfqpoint{0.011050in}{-0.041667in}}{\pgfqpoint{0.021649in}{-0.037276in}}{\pgfqpoint{0.029463in}{-0.029463in}}%
\pgfpathcurveto{\pgfqpoint{0.037276in}{-0.021649in}}{\pgfqpoint{0.041667in}{-0.011050in}}{\pgfqpoint{0.041667in}{0.000000in}}%
\pgfpathcurveto{\pgfqpoint{0.041667in}{0.011050in}}{\pgfqpoint{0.037276in}{0.021649in}}{\pgfqpoint{0.029463in}{0.029463in}}%
\pgfpathcurveto{\pgfqpoint{0.021649in}{0.037276in}}{\pgfqpoint{0.011050in}{0.041667in}}{\pgfqpoint{0.000000in}{0.041667in}}%
\pgfpathcurveto{\pgfqpoint{-0.011050in}{0.041667in}}{\pgfqpoint{-0.021649in}{0.037276in}}{\pgfqpoint{-0.029463in}{0.029463in}}%
\pgfpathcurveto{\pgfqpoint{-0.037276in}{0.021649in}}{\pgfqpoint{-0.041667in}{0.011050in}}{\pgfqpoint{-0.041667in}{0.000000in}}%
\pgfpathcurveto{\pgfqpoint{-0.041667in}{-0.011050in}}{\pgfqpoint{-0.037276in}{-0.021649in}}{\pgfqpoint{-0.029463in}{-0.029463in}}%
\pgfpathcurveto{\pgfqpoint{-0.021649in}{-0.037276in}}{\pgfqpoint{-0.011050in}{-0.041667in}}{\pgfqpoint{0.000000in}{-0.041667in}}%
\pgfpathclose%
\pgfusepath{stroke,fill}%
}%
\begin{pgfscope}%
\pgfsys@transformshift{5.938727in}{0.741795in}%
\pgfsys@useobject{currentmarker}{}%
\end{pgfscope}%
\begin{pgfscope}%
\pgfsys@transformshift{6.340372in}{1.387677in}%
\pgfsys@useobject{currentmarker}{}%
\end{pgfscope}%
\begin{pgfscope}%
\pgfsys@transformshift{6.742017in}{1.964869in}%
\pgfsys@useobject{currentmarker}{}%
\end{pgfscope}%
\begin{pgfscope}%
\pgfsys@transformshift{7.143662in}{2.473381in}%
\pgfsys@useobject{currentmarker}{}%
\end{pgfscope}%
\begin{pgfscope}%
\pgfsys@transformshift{7.545306in}{2.913222in}%
\pgfsys@useobject{currentmarker}{}%
\end{pgfscope}%
\begin{pgfscope}%
\pgfsys@transformshift{7.946951in}{3.284402in}%
\pgfsys@useobject{currentmarker}{}%
\end{pgfscope}%
\begin{pgfscope}%
\pgfsys@transformshift{8.348596in}{3.586926in}%
\pgfsys@useobject{currentmarker}{}%
\end{pgfscope}%
\begin{pgfscope}%
\pgfsys@transformshift{8.750241in}{3.820800in}%
\pgfsys@useobject{currentmarker}{}%
\end{pgfscope}%
\begin{pgfscope}%
\pgfsys@transformshift{9.151886in}{3.986029in}%
\pgfsys@useobject{currentmarker}{}%
\end{pgfscope}%
\begin{pgfscope}%
\pgfsys@transformshift{9.553530in}{4.082614in}%
\pgfsys@useobject{currentmarker}{}%
\end{pgfscope}%
\begin{pgfscope}%
\pgfsys@transformshift{9.955175in}{4.110558in}%
\pgfsys@useobject{currentmarker}{}%
\end{pgfscope}%
\begin{pgfscope}%
\pgfsys@transformshift{10.356820in}{4.069861in}%
\pgfsys@useobject{currentmarker}{}%
\end{pgfscope}%
\end{pgfscope}%
\begin{pgfscope}%
\pgfpathrectangle{\pgfqpoint{5.697941in}{0.741795in}}{\pgfqpoint{4.458056in}{3.401160in}} %
\pgfusepath{clip}%
\pgfsetbuttcap%
\pgfsetroundjoin%
\pgfsetlinewidth{1.505625pt}%
\definecolor{currentstroke}{rgb}{0.000000,0.750000,0.750000}%
\pgfsetstrokecolor{currentstroke}%
\pgfsetdash{{9.600000pt}{2.400000pt}{1.500000pt}{2.400000pt}}{0.000000pt}%
\pgfpathmoveto{\pgfqpoint{5.938727in}{0.741795in}}%
\pgfpathlineto{\pgfqpoint{5.998974in}{0.843056in}}%
\pgfpathlineto{\pgfqpoint{6.055204in}{0.936173in}}%
\pgfpathlineto{\pgfqpoint{6.111434in}{1.027943in}}%
\pgfpathlineto{\pgfqpoint{6.167665in}{1.118366in}}%
\pgfpathlineto{\pgfqpoint{6.223895in}{1.207443in}}%
\pgfpathlineto{\pgfqpoint{6.280125in}{1.295173in}}%
\pgfpathlineto{\pgfqpoint{6.336356in}{1.381556in}}%
\pgfpathlineto{\pgfqpoint{6.392586in}{1.466593in}}%
\pgfpathlineto{\pgfqpoint{6.448816in}{1.550284in}}%
\pgfpathlineto{\pgfqpoint{6.505046in}{1.632628in}}%
\pgfpathlineto{\pgfqpoint{6.561277in}{1.713625in}}%
\pgfpathlineto{\pgfqpoint{6.617507in}{1.793276in}}%
\pgfpathlineto{\pgfqpoint{6.673737in}{1.871580in}}%
\pgfpathlineto{\pgfqpoint{6.729967in}{1.948538in}}%
\pgfpathlineto{\pgfqpoint{6.782181in}{2.018793in}}%
\pgfpathlineto{\pgfqpoint{6.834395in}{2.087887in}}%
\pgfpathlineto{\pgfqpoint{6.886609in}{2.155820in}}%
\pgfpathlineto{\pgfqpoint{6.938823in}{2.222591in}}%
\pgfpathlineto{\pgfqpoint{6.991037in}{2.288202in}}%
\pgfpathlineto{\pgfqpoint{7.043250in}{2.352652in}}%
\pgfpathlineto{\pgfqpoint{7.095464in}{2.415941in}}%
\pgfpathlineto{\pgfqpoint{7.147678in}{2.478069in}}%
\pgfpathlineto{\pgfqpoint{7.199892in}{2.539035in}}%
\pgfpathlineto{\pgfqpoint{7.252106in}{2.598841in}}%
\pgfpathlineto{\pgfqpoint{7.304319in}{2.657486in}}%
\pgfpathlineto{\pgfqpoint{7.356533in}{2.714969in}}%
\pgfpathlineto{\pgfqpoint{7.408747in}{2.771292in}}%
\pgfpathlineto{\pgfqpoint{7.460961in}{2.826454in}}%
\pgfpathlineto{\pgfqpoint{7.509158in}{2.876342in}}%
\pgfpathlineto{\pgfqpoint{7.557356in}{2.925240in}}%
\pgfpathlineto{\pgfqpoint{7.605553in}{2.973150in}}%
\pgfpathlineto{\pgfqpoint{7.653750in}{3.020070in}}%
\pgfpathlineto{\pgfqpoint{7.701948in}{3.066000in}}%
\pgfpathlineto{\pgfqpoint{7.750145in}{3.110942in}}%
\pgfpathlineto{\pgfqpoint{7.798343in}{3.154894in}}%
\pgfpathlineto{\pgfqpoint{7.846540in}{3.197857in}}%
\pgfpathlineto{\pgfqpoint{7.894737in}{3.239831in}}%
\pgfpathlineto{\pgfqpoint{7.942935in}{3.280816in}}%
\pgfpathlineto{\pgfqpoint{7.991132in}{3.320811in}}%
\pgfpathlineto{\pgfqpoint{8.039329in}{3.359817in}}%
\pgfpathlineto{\pgfqpoint{8.087527in}{3.397834in}}%
\pgfpathlineto{\pgfqpoint{8.135724in}{3.434861in}}%
\pgfpathlineto{\pgfqpoint{8.183922in}{3.470900in}}%
\pgfpathlineto{\pgfqpoint{8.232119in}{3.505949in}}%
\pgfpathlineto{\pgfqpoint{8.280316in}{3.540008in}}%
\pgfpathlineto{\pgfqpoint{8.324497in}{3.570361in}}%
\pgfpathlineto{\pgfqpoint{8.368678in}{3.599882in}}%
\pgfpathlineto{\pgfqpoint{8.412859in}{3.628572in}}%
\pgfpathlineto{\pgfqpoint{8.457040in}{3.656431in}}%
\pgfpathlineto{\pgfqpoint{8.501221in}{3.683458in}}%
\pgfpathlineto{\pgfqpoint{8.545402in}{3.709654in}}%
\pgfpathlineto{\pgfqpoint{8.589583in}{3.735019in}}%
\pgfpathlineto{\pgfqpoint{8.633764in}{3.759553in}}%
\pgfpathlineto{\pgfqpoint{8.677945in}{3.783256in}}%
\pgfpathlineto{\pgfqpoint{8.722126in}{3.806127in}}%
\pgfpathlineto{\pgfqpoint{8.766307in}{3.828167in}}%
\pgfpathlineto{\pgfqpoint{8.810487in}{3.849376in}}%
\pgfpathlineto{\pgfqpoint{8.854668in}{3.869753in}}%
\pgfpathlineto{\pgfqpoint{8.898849in}{3.889299in}}%
\pgfpathlineto{\pgfqpoint{8.943030in}{3.908015in}}%
\pgfpathlineto{\pgfqpoint{8.987211in}{3.925898in}}%
\pgfpathlineto{\pgfqpoint{9.031392in}{3.942951in}}%
\pgfpathlineto{\pgfqpoint{9.075573in}{3.959172in}}%
\pgfpathlineto{\pgfqpoint{9.119754in}{3.974562in}}%
\pgfpathlineto{\pgfqpoint{9.163935in}{3.989121in}}%
\pgfpathlineto{\pgfqpoint{9.208116in}{4.002849in}}%
\pgfpathlineto{\pgfqpoint{9.252297in}{4.015745in}}%
\pgfpathlineto{\pgfqpoint{9.296478in}{4.027811in}}%
\pgfpathlineto{\pgfqpoint{9.340659in}{4.039045in}}%
\pgfpathlineto{\pgfqpoint{9.384840in}{4.049447in}}%
\pgfpathlineto{\pgfqpoint{9.429020in}{4.059019in}}%
\pgfpathlineto{\pgfqpoint{9.473201in}{4.067759in}}%
\pgfpathlineto{\pgfqpoint{9.513366in}{4.074983in}}%
\pgfpathlineto{\pgfqpoint{9.553530in}{4.081521in}}%
\pgfpathlineto{\pgfqpoint{9.593695in}{4.087371in}}%
\pgfpathlineto{\pgfqpoint{9.633859in}{4.092535in}}%
\pgfpathlineto{\pgfqpoint{9.674024in}{4.097011in}}%
\pgfpathlineto{\pgfqpoint{9.714188in}{4.100801in}}%
\pgfpathlineto{\pgfqpoint{9.754353in}{4.103903in}}%
\pgfpathlineto{\pgfqpoint{9.794517in}{4.106319in}}%
\pgfpathlineto{\pgfqpoint{9.834682in}{4.108048in}}%
\pgfpathlineto{\pgfqpoint{9.874846in}{4.109089in}}%
\pgfpathlineto{\pgfqpoint{9.915011in}{4.109444in}}%
\pgfpathlineto{\pgfqpoint{9.955175in}{4.109112in}}%
\pgfpathlineto{\pgfqpoint{9.995340in}{4.108093in}}%
\pgfpathlineto{\pgfqpoint{10.035504in}{4.106386in}}%
\pgfpathlineto{\pgfqpoint{10.075669in}{4.103993in}}%
\pgfpathlineto{\pgfqpoint{10.115833in}{4.100913in}}%
\pgfpathlineto{\pgfqpoint{10.159331in}{4.096802in}}%
\pgfpathlineto{\pgfqpoint{10.159331in}{4.096802in}}%
\pgfusepath{stroke}%
\end{pgfscope}%
\begin{pgfscope}%
\pgfpathrectangle{\pgfqpoint{5.697941in}{0.741795in}}{\pgfqpoint{4.458056in}{3.401160in}} %
\pgfusepath{clip}%
\pgfsetbuttcap%
\pgfsetmiterjoin%
\definecolor{currentfill}{rgb}{0.000000,0.750000,0.750000}%
\pgfsetfillcolor{currentfill}%
\pgfsetlinewidth{1.003750pt}%
\definecolor{currentstroke}{rgb}{0.000000,0.750000,0.750000}%
\pgfsetstrokecolor{currentstroke}%
\pgfsetdash{}{0pt}%
\pgfsys@defobject{currentmarker}{\pgfqpoint{-0.041667in}{-0.041667in}}{\pgfqpoint{0.041667in}{0.041667in}}{%
\pgfpathmoveto{\pgfqpoint{-0.000000in}{-0.041667in}}%
\pgfpathlineto{\pgfqpoint{0.041667in}{0.041667in}}%
\pgfpathlineto{\pgfqpoint{-0.041667in}{0.041667in}}%
\pgfpathclose%
\pgfusepath{stroke,fill}%
}%
\begin{pgfscope}%
\pgfsys@transformshift{5.938727in}{0.741795in}%
\pgfsys@useobject{currentmarker}{}%
\end{pgfscope}%
\begin{pgfscope}%
\pgfsys@transformshift{6.340372in}{1.387675in}%
\pgfsys@useobject{currentmarker}{}%
\end{pgfscope}%
\begin{pgfscope}%
\pgfsys@transformshift{6.742017in}{1.964853in}%
\pgfsys@useobject{currentmarker}{}%
\end{pgfscope}%
\begin{pgfscope}%
\pgfsys@transformshift{7.143662in}{2.473331in}%
\pgfsys@useobject{currentmarker}{}%
\end{pgfscope}%
\begin{pgfscope}%
\pgfsys@transformshift{7.545306in}{2.913108in}%
\pgfsys@useobject{currentmarker}{}%
\end{pgfscope}%
\begin{pgfscope}%
\pgfsys@transformshift{7.946951in}{3.284186in}%
\pgfsys@useobject{currentmarker}{}%
\end{pgfscope}%
\begin{pgfscope}%
\pgfsys@transformshift{8.348596in}{3.586566in}%
\pgfsys@useobject{currentmarker}{}%
\end{pgfscope}%
\begin{pgfscope}%
\pgfsys@transformshift{8.750241in}{3.820248in}%
\pgfsys@useobject{currentmarker}{}%
\end{pgfscope}%
\begin{pgfscope}%
\pgfsys@transformshift{9.151886in}{3.985233in}%
\pgfsys@useobject{currentmarker}{}%
\end{pgfscope}%
\begin{pgfscope}%
\pgfsys@transformshift{9.553530in}{4.081521in}%
\pgfsys@useobject{currentmarker}{}%
\end{pgfscope}%
\begin{pgfscope}%
\pgfsys@transformshift{9.955175in}{4.109112in}%
\pgfsys@useobject{currentmarker}{}%
\end{pgfscope}%
\begin{pgfscope}%
\pgfsys@transformshift{10.356820in}{4.068006in}%
\pgfsys@useobject{currentmarker}{}%
\end{pgfscope}%
\end{pgfscope}%
\begin{pgfscope}%
\pgfsetrectcap%
\pgfsetmiterjoin%
\pgfsetlinewidth{0.803000pt}%
\definecolor{currentstroke}{rgb}{0.000000,0.000000,0.000000}%
\pgfsetstrokecolor{currentstroke}%
\pgfsetdash{}{0pt}%
\pgfpathmoveto{\pgfqpoint{5.697941in}{0.741795in}}%
\pgfpathlineto{\pgfqpoint{5.697941in}{4.142955in}}%
\pgfusepath{stroke}%
\end{pgfscope}%
\begin{pgfscope}%
\pgfsetrectcap%
\pgfsetmiterjoin%
\pgfsetlinewidth{0.803000pt}%
\definecolor{currentstroke}{rgb}{0.000000,0.000000,0.000000}%
\pgfsetstrokecolor{currentstroke}%
\pgfsetdash{}{0pt}%
\pgfpathmoveto{\pgfqpoint{10.155998in}{0.741795in}}%
\pgfpathlineto{\pgfqpoint{10.155998in}{4.142955in}}%
\pgfusepath{stroke}%
\end{pgfscope}%
\begin{pgfscope}%
\pgfsetrectcap%
\pgfsetmiterjoin%
\pgfsetlinewidth{0.803000pt}%
\definecolor{currentstroke}{rgb}{0.000000,0.000000,0.000000}%
\pgfsetstrokecolor{currentstroke}%
\pgfsetdash{}{0pt}%
\pgfpathmoveto{\pgfqpoint{5.697941in}{0.741795in}}%
\pgfpathlineto{\pgfqpoint{10.155998in}{0.741795in}}%
\pgfusepath{stroke}%
\end{pgfscope}%
\begin{pgfscope}%
\pgfsetrectcap%
\pgfsetmiterjoin%
\pgfsetlinewidth{0.803000pt}%
\definecolor{currentstroke}{rgb}{0.000000,0.000000,0.000000}%
\pgfsetstrokecolor{currentstroke}%
\pgfsetdash{}{0pt}%
\pgfpathmoveto{\pgfqpoint{5.697941in}{4.142955in}}%
\pgfpathlineto{\pgfqpoint{10.155998in}{4.142955in}}%
\pgfusepath{stroke}%
\end{pgfscope}%
\begin{pgfscope}%
\pgfsetbuttcap%
\pgfsetmiterjoin%
\definecolor{currentfill}{rgb}{1.000000,1.000000,1.000000}%
\pgfsetfillcolor{currentfill}%
\pgfsetfillopacity{0.800000}%
\pgfsetlinewidth{1.003750pt}%
\definecolor{currentstroke}{rgb}{0.800000,0.800000,0.800000}%
\pgfsetstrokecolor{currentstroke}%
\pgfsetstrokeopacity{0.800000}%
\pgfsetdash{}{0pt}%
\pgfpathmoveto{\pgfqpoint{5.834052in}{3.602023in}}%
\pgfpathlineto{\pgfqpoint{6.829876in}{3.602023in}}%
\pgfpathquadraticcurveto{\pgfqpoint{6.868765in}{3.602023in}}{\pgfqpoint{6.868765in}{3.640912in}}%
\pgfpathlineto{\pgfqpoint{6.868765in}{4.006844in}}%
\pgfpathquadraticcurveto{\pgfqpoint{6.868765in}{4.045733in}}{\pgfqpoint{6.829876in}{4.045733in}}%
\pgfpathlineto{\pgfqpoint{5.834052in}{4.045733in}}%
\pgfpathquadraticcurveto{\pgfqpoint{5.795163in}{4.045733in}}{\pgfqpoint{5.795163in}{4.006844in}}%
\pgfpathlineto{\pgfqpoint{5.795163in}{3.640912in}}%
\pgfpathquadraticcurveto{\pgfqpoint{5.795163in}{3.602023in}}{\pgfqpoint{5.834052in}{3.602023in}}%
\pgfpathclose%
\pgfusepath{stroke,fill}%
\end{pgfscope}%
\begin{pgfscope}%
\pgftext[x=5.872941in,y=3.820223in,left,base]{\rmfamily\fontsize{14.000000}{16.800000}\selectfont \(\displaystyle \mathbf{I}\mbox{g} = \) 0.01}%
\end{pgfscope}%
\begin{pgfscope}%
\pgftext[x=5.400000in,y=0.157780in,,base]{\rmfamily\fontsize{20.000000}{24.000000}\selectfont \(\displaystyle t^*\)}%
\end{pgfscope}%
\begin{pgfscope}%
\pgftext[x=0.311046in,y=4.110377in,left,base,rotate=90.000000]{\rmfamily\fontsize{20.000000}{24.000000}\selectfont \(\displaystyle y^*\)}%
\end{pgfscope}%
\end{pgfpicture}%
\makeatother%
\endgroup%
}
    \caption{Short-time scaled drop trajectories for various values of $\mathbb{E}\mbox{u}$, $\mathbb{I}\mbox{g}$. The trajectory reduces to the classical $\mathcal{O}(1)$ solution for small values of $\mathbb{I}\mbox{g}$. It should be noted that despite what is implied by these plots $\mathbb{I}\mbox{g}$ is not necessarily independent of $\mathbb{E}\mbox{u}$, sharing $q$ and $E_0$ as common factors, as well as the covariance relationships $U_0 \propto R_d^2$, and $m \propto R_d^3$.}
    \label{fig:short_times}
\end{figure*}

\subsubsection{Inertial Electro-Viscous Limit}
By similar arguments we find an asymptotic estimate of the trajectory in the long-time regime. If we assume a constant scale for the drag coefficient $C_D \approx 0.5$ then $\mathbb{D}\mbox{g}$ is approximately a constant $\mathbb{D}\mbox{g} \approx 6 \times 10^{-4}$ in all of our experiments. The approximate solution is 
\begin{gather}
{y^*}({t^*}) = {t^*} - \frac{{t^*}^{2}}{2}  \nonumber \\
+ \phi\mathbb{E}\mbox{u} \left(\frac{{t^*}^{3}}{3} \left(1 + \mathbb{D}\mbox{g}\right) + \frac{{t^*}^{4}}{12} \left(-1 - \mathbb{D}\mbox{g}\right) - \frac{\mathbb{D}\mbox{g} {t^*}^{2}}{2}\right) \nonumber \\
 + \mathcal{O}(\phi^2 \mathbb{E}\mbox{u}^2).
\label{perturb_viscous}
\end{gather}
Trajectories for this solution are shown in Figure \ref{fig:long_times}.
\begin{figure}[htb]
    \centering
    \resizebox{0.5\textwidth}{!}{%% Creator: Matplotlib, PGF backend
%%
%% To include the figure in your LaTeX document, write
%%   \input{<filename>.pgf}
%%
%% Make sure the required packages are loaded in your preamble
%%   \usepackage{pgf}
%%
%% Figures using additional raster images can only be included by \input if
%% they are in the same directory as the main LaTeX file. For loading figures
%% from other directories you can use the `import` package
%%   \usepackage{import}
%% and then include the figures with
%%   \import{<path to file>}{<filename>.pgf}
%%
%% Matplotlib used the following preamble
%%   \usepackage{fontspec}
%%   \setmainfont{DejaVu Serif}
%%   \setsansfont{DejaVu Sans}
%%   \setmonofont{DejaVu Sans Mono}
%%
\begingroup%
\makeatletter%
\begin{pgfpicture}%
\pgfpathrectangle{\pgfpointorigin}{\pgfqpoint{5.403395in}{3.694365in}}%
\pgfusepath{use as bounding box, clip}%
\begin{pgfscope}%
\pgfsetbuttcap%
\pgfsetmiterjoin%
\definecolor{currentfill}{rgb}{1.000000,1.000000,1.000000}%
\pgfsetfillcolor{currentfill}%
\pgfsetlinewidth{0.000000pt}%
\definecolor{currentstroke}{rgb}{1.000000,1.000000,1.000000}%
\pgfsetstrokecolor{currentstroke}%
\pgfsetdash{}{0pt}%
\pgfpathmoveto{\pgfqpoint{0.000000in}{0.000000in}}%
\pgfpathlineto{\pgfqpoint{5.403395in}{0.000000in}}%
\pgfpathlineto{\pgfqpoint{5.403395in}{3.694365in}}%
\pgfpathlineto{\pgfqpoint{0.000000in}{3.694365in}}%
\pgfpathclose%
\pgfusepath{fill}%
\end{pgfscope}%
\begin{pgfscope}%
\pgfsetbuttcap%
\pgfsetmiterjoin%
\definecolor{currentfill}{rgb}{1.000000,1.000000,1.000000}%
\pgfsetfillcolor{currentfill}%
\pgfsetlinewidth{0.000000pt}%
\definecolor{currentstroke}{rgb}{0.000000,0.000000,0.000000}%
\pgfsetstrokecolor{currentstroke}%
\pgfsetstrokeopacity{0.000000}%
\pgfsetdash{}{0pt}%
\pgfpathmoveto{\pgfqpoint{0.564660in}{0.521603in}}%
\pgfpathlineto{\pgfqpoint{5.214660in}{0.521603in}}%
\pgfpathlineto{\pgfqpoint{5.214660in}{3.541603in}}%
\pgfpathlineto{\pgfqpoint{0.564660in}{3.541603in}}%
\pgfpathclose%
\pgfusepath{fill}%
\end{pgfscope}%
\begin{pgfscope}%
\pgfsetbuttcap%
\pgfsetroundjoin%
\definecolor{currentfill}{rgb}{0.000000,0.000000,0.000000}%
\pgfsetfillcolor{currentfill}%
\pgfsetlinewidth{0.803000pt}%
\definecolor{currentstroke}{rgb}{0.000000,0.000000,0.000000}%
\pgfsetstrokecolor{currentstroke}%
\pgfsetdash{}{0pt}%
\pgfsys@defobject{currentmarker}{\pgfqpoint{0.000000in}{-0.048611in}}{\pgfqpoint{0.000000in}{0.000000in}}{%
\pgfpathmoveto{\pgfqpoint{0.000000in}{0.000000in}}%
\pgfpathlineto{\pgfqpoint{0.000000in}{-0.048611in}}%
\pgfusepath{stroke,fill}%
}%
\begin{pgfscope}%
\pgfsys@transformshift{0.785878in}{0.521603in}%
\pgfsys@useobject{currentmarker}{}%
\end{pgfscope}%
\end{pgfscope}%
\begin{pgfscope}%
\pgftext[x=0.785878in,y=0.424381in,,top]{\rmfamily\fontsize{10.000000}{12.000000}\selectfont \(\displaystyle 0.0\)}%
\end{pgfscope}%
\begin{pgfscope}%
\pgfsetbuttcap%
\pgfsetroundjoin%
\definecolor{currentfill}{rgb}{0.000000,0.000000,0.000000}%
\pgfsetfillcolor{currentfill}%
\pgfsetlinewidth{0.803000pt}%
\definecolor{currentstroke}{rgb}{0.000000,0.000000,0.000000}%
\pgfsetstrokecolor{currentstroke}%
\pgfsetdash{}{0pt}%
\pgfsys@defobject{currentmarker}{\pgfqpoint{0.000000in}{-0.048611in}}{\pgfqpoint{0.000000in}{0.000000in}}{%
\pgfpathmoveto{\pgfqpoint{0.000000in}{0.000000in}}%
\pgfpathlineto{\pgfqpoint{0.000000in}{-0.048611in}}%
\pgfusepath{stroke,fill}%
}%
\begin{pgfscope}%
\pgfsys@transformshift{1.671634in}{0.521603in}%
\pgfsys@useobject{currentmarker}{}%
\end{pgfscope}%
\end{pgfscope}%
\begin{pgfscope}%
\pgftext[x=1.671634in,y=0.424381in,,top]{\rmfamily\fontsize{10.000000}{12.000000}\selectfont \(\displaystyle 0.2\)}%
\end{pgfscope}%
\begin{pgfscope}%
\pgfsetbuttcap%
\pgfsetroundjoin%
\definecolor{currentfill}{rgb}{0.000000,0.000000,0.000000}%
\pgfsetfillcolor{currentfill}%
\pgfsetlinewidth{0.803000pt}%
\definecolor{currentstroke}{rgb}{0.000000,0.000000,0.000000}%
\pgfsetstrokecolor{currentstroke}%
\pgfsetdash{}{0pt}%
\pgfsys@defobject{currentmarker}{\pgfqpoint{0.000000in}{-0.048611in}}{\pgfqpoint{0.000000in}{0.000000in}}{%
\pgfpathmoveto{\pgfqpoint{0.000000in}{0.000000in}}%
\pgfpathlineto{\pgfqpoint{0.000000in}{-0.048611in}}%
\pgfusepath{stroke,fill}%
}%
\begin{pgfscope}%
\pgfsys@transformshift{2.557391in}{0.521603in}%
\pgfsys@useobject{currentmarker}{}%
\end{pgfscope}%
\end{pgfscope}%
\begin{pgfscope}%
\pgftext[x=2.557391in,y=0.424381in,,top]{\rmfamily\fontsize{10.000000}{12.000000}\selectfont \(\displaystyle 0.4\)}%
\end{pgfscope}%
\begin{pgfscope}%
\pgfsetbuttcap%
\pgfsetroundjoin%
\definecolor{currentfill}{rgb}{0.000000,0.000000,0.000000}%
\pgfsetfillcolor{currentfill}%
\pgfsetlinewidth{0.803000pt}%
\definecolor{currentstroke}{rgb}{0.000000,0.000000,0.000000}%
\pgfsetstrokecolor{currentstroke}%
\pgfsetdash{}{0pt}%
\pgfsys@defobject{currentmarker}{\pgfqpoint{0.000000in}{-0.048611in}}{\pgfqpoint{0.000000in}{0.000000in}}{%
\pgfpathmoveto{\pgfqpoint{0.000000in}{0.000000in}}%
\pgfpathlineto{\pgfqpoint{0.000000in}{-0.048611in}}%
\pgfusepath{stroke,fill}%
}%
\begin{pgfscope}%
\pgfsys@transformshift{3.443147in}{0.521603in}%
\pgfsys@useobject{currentmarker}{}%
\end{pgfscope}%
\end{pgfscope}%
\begin{pgfscope}%
\pgftext[x=3.443147in,y=0.424381in,,top]{\rmfamily\fontsize{10.000000}{12.000000}\selectfont \(\displaystyle 0.6\)}%
\end{pgfscope}%
\begin{pgfscope}%
\pgfsetbuttcap%
\pgfsetroundjoin%
\definecolor{currentfill}{rgb}{0.000000,0.000000,0.000000}%
\pgfsetfillcolor{currentfill}%
\pgfsetlinewidth{0.803000pt}%
\definecolor{currentstroke}{rgb}{0.000000,0.000000,0.000000}%
\pgfsetstrokecolor{currentstroke}%
\pgfsetdash{}{0pt}%
\pgfsys@defobject{currentmarker}{\pgfqpoint{0.000000in}{-0.048611in}}{\pgfqpoint{0.000000in}{0.000000in}}{%
\pgfpathmoveto{\pgfqpoint{0.000000in}{0.000000in}}%
\pgfpathlineto{\pgfqpoint{0.000000in}{-0.048611in}}%
\pgfusepath{stroke,fill}%
}%
\begin{pgfscope}%
\pgfsys@transformshift{4.328904in}{0.521603in}%
\pgfsys@useobject{currentmarker}{}%
\end{pgfscope}%
\end{pgfscope}%
\begin{pgfscope}%
\pgftext[x=4.328904in,y=0.424381in,,top]{\rmfamily\fontsize{10.000000}{12.000000}\selectfont \(\displaystyle 0.8\)}%
\end{pgfscope}%
\begin{pgfscope}%
\pgfsetbuttcap%
\pgfsetroundjoin%
\definecolor{currentfill}{rgb}{0.000000,0.000000,0.000000}%
\pgfsetfillcolor{currentfill}%
\pgfsetlinewidth{0.803000pt}%
\definecolor{currentstroke}{rgb}{0.000000,0.000000,0.000000}%
\pgfsetstrokecolor{currentstroke}%
\pgfsetdash{}{0pt}%
\pgfsys@defobject{currentmarker}{\pgfqpoint{0.000000in}{-0.048611in}}{\pgfqpoint{0.000000in}{0.000000in}}{%
\pgfpathmoveto{\pgfqpoint{0.000000in}{0.000000in}}%
\pgfpathlineto{\pgfqpoint{0.000000in}{-0.048611in}}%
\pgfusepath{stroke,fill}%
}%
\begin{pgfscope}%
\pgfsys@transformshift{5.214660in}{0.521603in}%
\pgfsys@useobject{currentmarker}{}%
\end{pgfscope}%
\end{pgfscope}%
\begin{pgfscope}%
\pgftext[x=5.214660in,y=0.424381in,,top]{\rmfamily\fontsize{10.000000}{12.000000}\selectfont \(\displaystyle 1.0\)}%
\end{pgfscope}%
\begin{pgfscope}%
\pgftext[x=2.889660in,y=0.234413in,,top]{\rmfamily\fontsize{10.000000}{12.000000}\selectfont \(\displaystyle \bar{t}\)}%
\end{pgfscope}%
\begin{pgfscope}%
\pgfsetbuttcap%
\pgfsetroundjoin%
\definecolor{currentfill}{rgb}{0.000000,0.000000,0.000000}%
\pgfsetfillcolor{currentfill}%
\pgfsetlinewidth{0.803000pt}%
\definecolor{currentstroke}{rgb}{0.000000,0.000000,0.000000}%
\pgfsetstrokecolor{currentstroke}%
\pgfsetdash{}{0pt}%
\pgfsys@defobject{currentmarker}{\pgfqpoint{-0.048611in}{0.000000in}}{\pgfqpoint{0.000000in}{0.000000in}}{%
\pgfpathmoveto{\pgfqpoint{0.000000in}{0.000000in}}%
\pgfpathlineto{\pgfqpoint{-0.048611in}{0.000000in}}%
\pgfusepath{stroke,fill}%
}%
\begin{pgfscope}%
\pgfsys@transformshift{0.564660in}{0.521603in}%
\pgfsys@useobject{currentmarker}{}%
\end{pgfscope}%
\end{pgfscope}%
\begin{pgfscope}%
\pgftext[x=0.289968in,y=0.468842in,left,base]{\rmfamily\fontsize{10.000000}{12.000000}\selectfont \(\displaystyle 0.0\)}%
\end{pgfscope}%
\begin{pgfscope}%
\pgfsetbuttcap%
\pgfsetroundjoin%
\definecolor{currentfill}{rgb}{0.000000,0.000000,0.000000}%
\pgfsetfillcolor{currentfill}%
\pgfsetlinewidth{0.803000pt}%
\definecolor{currentstroke}{rgb}{0.000000,0.000000,0.000000}%
\pgfsetstrokecolor{currentstroke}%
\pgfsetdash{}{0pt}%
\pgfsys@defobject{currentmarker}{\pgfqpoint{-0.048611in}{0.000000in}}{\pgfqpoint{0.000000in}{0.000000in}}{%
\pgfpathmoveto{\pgfqpoint{0.000000in}{0.000000in}}%
\pgfpathlineto{\pgfqpoint{-0.048611in}{0.000000in}}%
\pgfusepath{stroke,fill}%
}%
\begin{pgfscope}%
\pgfsys@transformshift{0.564660in}{0.953032in}%
\pgfsys@useobject{currentmarker}{}%
\end{pgfscope}%
\end{pgfscope}%
\begin{pgfscope}%
\pgftext[x=0.289968in,y=0.900270in,left,base]{\rmfamily\fontsize{10.000000}{12.000000}\selectfont \(\displaystyle 0.1\)}%
\end{pgfscope}%
\begin{pgfscope}%
\pgfsetbuttcap%
\pgfsetroundjoin%
\definecolor{currentfill}{rgb}{0.000000,0.000000,0.000000}%
\pgfsetfillcolor{currentfill}%
\pgfsetlinewidth{0.803000pt}%
\definecolor{currentstroke}{rgb}{0.000000,0.000000,0.000000}%
\pgfsetstrokecolor{currentstroke}%
\pgfsetdash{}{0pt}%
\pgfsys@defobject{currentmarker}{\pgfqpoint{-0.048611in}{0.000000in}}{\pgfqpoint{0.000000in}{0.000000in}}{%
\pgfpathmoveto{\pgfqpoint{0.000000in}{0.000000in}}%
\pgfpathlineto{\pgfqpoint{-0.048611in}{0.000000in}}%
\pgfusepath{stroke,fill}%
}%
\begin{pgfscope}%
\pgfsys@transformshift{0.564660in}{1.384460in}%
\pgfsys@useobject{currentmarker}{}%
\end{pgfscope}%
\end{pgfscope}%
\begin{pgfscope}%
\pgftext[x=0.289968in,y=1.331699in,left,base]{\rmfamily\fontsize{10.000000}{12.000000}\selectfont \(\displaystyle 0.2\)}%
\end{pgfscope}%
\begin{pgfscope}%
\pgfsetbuttcap%
\pgfsetroundjoin%
\definecolor{currentfill}{rgb}{0.000000,0.000000,0.000000}%
\pgfsetfillcolor{currentfill}%
\pgfsetlinewidth{0.803000pt}%
\definecolor{currentstroke}{rgb}{0.000000,0.000000,0.000000}%
\pgfsetstrokecolor{currentstroke}%
\pgfsetdash{}{0pt}%
\pgfsys@defobject{currentmarker}{\pgfqpoint{-0.048611in}{0.000000in}}{\pgfqpoint{0.000000in}{0.000000in}}{%
\pgfpathmoveto{\pgfqpoint{0.000000in}{0.000000in}}%
\pgfpathlineto{\pgfqpoint{-0.048611in}{0.000000in}}%
\pgfusepath{stroke,fill}%
}%
\begin{pgfscope}%
\pgfsys@transformshift{0.564660in}{1.815889in}%
\pgfsys@useobject{currentmarker}{}%
\end{pgfscope}%
\end{pgfscope}%
\begin{pgfscope}%
\pgftext[x=0.289968in,y=1.763128in,left,base]{\rmfamily\fontsize{10.000000}{12.000000}\selectfont \(\displaystyle 0.3\)}%
\end{pgfscope}%
\begin{pgfscope}%
\pgfsetbuttcap%
\pgfsetroundjoin%
\definecolor{currentfill}{rgb}{0.000000,0.000000,0.000000}%
\pgfsetfillcolor{currentfill}%
\pgfsetlinewidth{0.803000pt}%
\definecolor{currentstroke}{rgb}{0.000000,0.000000,0.000000}%
\pgfsetstrokecolor{currentstroke}%
\pgfsetdash{}{0pt}%
\pgfsys@defobject{currentmarker}{\pgfqpoint{-0.048611in}{0.000000in}}{\pgfqpoint{0.000000in}{0.000000in}}{%
\pgfpathmoveto{\pgfqpoint{0.000000in}{0.000000in}}%
\pgfpathlineto{\pgfqpoint{-0.048611in}{0.000000in}}%
\pgfusepath{stroke,fill}%
}%
\begin{pgfscope}%
\pgfsys@transformshift{0.564660in}{2.247318in}%
\pgfsys@useobject{currentmarker}{}%
\end{pgfscope}%
\end{pgfscope}%
\begin{pgfscope}%
\pgftext[x=0.289968in,y=2.194556in,left,base]{\rmfamily\fontsize{10.000000}{12.000000}\selectfont \(\displaystyle 0.4\)}%
\end{pgfscope}%
\begin{pgfscope}%
\pgfsetbuttcap%
\pgfsetroundjoin%
\definecolor{currentfill}{rgb}{0.000000,0.000000,0.000000}%
\pgfsetfillcolor{currentfill}%
\pgfsetlinewidth{0.803000pt}%
\definecolor{currentstroke}{rgb}{0.000000,0.000000,0.000000}%
\pgfsetstrokecolor{currentstroke}%
\pgfsetdash{}{0pt}%
\pgfsys@defobject{currentmarker}{\pgfqpoint{-0.048611in}{0.000000in}}{\pgfqpoint{0.000000in}{0.000000in}}{%
\pgfpathmoveto{\pgfqpoint{0.000000in}{0.000000in}}%
\pgfpathlineto{\pgfqpoint{-0.048611in}{0.000000in}}%
\pgfusepath{stroke,fill}%
}%
\begin{pgfscope}%
\pgfsys@transformshift{0.564660in}{2.678746in}%
\pgfsys@useobject{currentmarker}{}%
\end{pgfscope}%
\end{pgfscope}%
\begin{pgfscope}%
\pgftext[x=0.289968in,y=2.625985in,left,base]{\rmfamily\fontsize{10.000000}{12.000000}\selectfont \(\displaystyle 0.5\)}%
\end{pgfscope}%
\begin{pgfscope}%
\pgfsetbuttcap%
\pgfsetroundjoin%
\definecolor{currentfill}{rgb}{0.000000,0.000000,0.000000}%
\pgfsetfillcolor{currentfill}%
\pgfsetlinewidth{0.803000pt}%
\definecolor{currentstroke}{rgb}{0.000000,0.000000,0.000000}%
\pgfsetstrokecolor{currentstroke}%
\pgfsetdash{}{0pt}%
\pgfsys@defobject{currentmarker}{\pgfqpoint{-0.048611in}{0.000000in}}{\pgfqpoint{0.000000in}{0.000000in}}{%
\pgfpathmoveto{\pgfqpoint{0.000000in}{0.000000in}}%
\pgfpathlineto{\pgfqpoint{-0.048611in}{0.000000in}}%
\pgfusepath{stroke,fill}%
}%
\begin{pgfscope}%
\pgfsys@transformshift{0.564660in}{3.110175in}%
\pgfsys@useobject{currentmarker}{}%
\end{pgfscope}%
\end{pgfscope}%
\begin{pgfscope}%
\pgftext[x=0.289968in,y=3.057413in,left,base]{\rmfamily\fontsize{10.000000}{12.000000}\selectfont \(\displaystyle 0.6\)}%
\end{pgfscope}%
\begin{pgfscope}%
\pgfsetbuttcap%
\pgfsetroundjoin%
\definecolor{currentfill}{rgb}{0.000000,0.000000,0.000000}%
\pgfsetfillcolor{currentfill}%
\pgfsetlinewidth{0.803000pt}%
\definecolor{currentstroke}{rgb}{0.000000,0.000000,0.000000}%
\pgfsetstrokecolor{currentstroke}%
\pgfsetdash{}{0pt}%
\pgfsys@defobject{currentmarker}{\pgfqpoint{-0.048611in}{0.000000in}}{\pgfqpoint{0.000000in}{0.000000in}}{%
\pgfpathmoveto{\pgfqpoint{0.000000in}{0.000000in}}%
\pgfpathlineto{\pgfqpoint{-0.048611in}{0.000000in}}%
\pgfusepath{stroke,fill}%
}%
\begin{pgfscope}%
\pgfsys@transformshift{0.564660in}{3.541603in}%
\pgfsys@useobject{currentmarker}{}%
\end{pgfscope}%
\end{pgfscope}%
\begin{pgfscope}%
\pgftext[x=0.289968in,y=3.488842in,left,base]{\rmfamily\fontsize{10.000000}{12.000000}\selectfont \(\displaystyle 0.7\)}%
\end{pgfscope}%
\begin{pgfscope}%
\pgftext[x=0.234413in,y=2.031603in,,bottom,rotate=90.000000]{\rmfamily\fontsize{10.000000}{12.000000}\selectfont \(\displaystyle \bar{y}\)}%
\end{pgfscope}%
\begin{pgfscope}%
\pgfpathrectangle{\pgfqpoint{0.564660in}{0.521603in}}{\pgfqpoint{4.650000in}{3.020000in}} %
\pgfusepath{clip}%
\pgfsetbuttcap%
\pgfsetroundjoin%
\pgfsetlinewidth{1.505625pt}%
\definecolor{currentstroke}{rgb}{1.000000,0.000000,0.000000}%
\pgfsetstrokecolor{currentstroke}%
\pgfsetdash{{5.550000pt}{2.400000pt}}{0.000000pt}%
\pgfpathmoveto{\pgfqpoint{0.785878in}{0.521603in}}%
\pgfpathlineto{\pgfqpoint{0.905455in}{0.636543in}}%
\pgfpathlineto{\pgfqpoint{1.029461in}{0.752586in}}%
\pgfpathlineto{\pgfqpoint{1.153467in}{0.865579in}}%
\pgfpathlineto{\pgfqpoint{1.281902in}{0.979547in}}%
\pgfpathlineto{\pgfqpoint{1.410336in}{1.090547in}}%
\pgfpathlineto{\pgfqpoint{1.543200in}{1.202392in}}%
\pgfpathlineto{\pgfqpoint{1.676063in}{1.311336in}}%
\pgfpathlineto{\pgfqpoint{1.813355in}{1.420993in}}%
\pgfpathlineto{\pgfqpoint{1.950648in}{1.527805in}}%
\pgfpathlineto{\pgfqpoint{2.092369in}{1.635195in}}%
\pgfpathlineto{\pgfqpoint{2.234090in}{1.739783in}}%
\pgfpathlineto{\pgfqpoint{2.380240in}{1.844815in}}%
\pgfpathlineto{\pgfqpoint{2.526389in}{1.947082in}}%
\pgfpathlineto{\pgfqpoint{2.672539in}{2.046682in}}%
\pgfpathlineto{\pgfqpoint{2.823118in}{2.146611in}}%
\pgfpathlineto{\pgfqpoint{2.973696in}{2.243910in}}%
\pgfpathlineto{\pgfqpoint{3.128704in}{2.341429in}}%
\pgfpathlineto{\pgfqpoint{3.288140in}{2.439054in}}%
\pgfpathlineto{\pgfqpoint{3.447576in}{2.534084in}}%
\pgfpathlineto{\pgfqpoint{3.611441in}{2.629188in}}%
\pgfpathlineto{\pgfqpoint{3.779735in}{2.724312in}}%
\pgfpathlineto{\pgfqpoint{3.956886in}{2.821844in}}%
\pgfpathlineto{\pgfqpoint{4.138466in}{2.919273in}}%
\pgfpathlineto{\pgfqpoint{4.333333in}{3.021252in}}%
\pgfpathlineto{\pgfqpoint{4.541485in}{3.127596in}}%
\pgfpathlineto{\pgfqpoint{4.767353in}{3.240428in}}%
\pgfpathlineto{\pgfqpoint{5.019794in}{3.363988in}}%
\pgfpathlineto{\pgfqpoint{5.210231in}{3.455803in}}%
\pgfpathlineto{\pgfqpoint{5.210231in}{3.455803in}}%
\pgfusepath{stroke}%
\end{pgfscope}%
\begin{pgfscope}%
\pgfpathrectangle{\pgfqpoint{0.564660in}{0.521603in}}{\pgfqpoint{4.650000in}{3.020000in}} %
\pgfusepath{clip}%
\pgfsetbuttcap%
\pgfsetmiterjoin%
\definecolor{currentfill}{rgb}{1.000000,0.000000,0.000000}%
\pgfsetfillcolor{currentfill}%
\pgfsetlinewidth{1.003750pt}%
\definecolor{currentstroke}{rgb}{1.000000,0.000000,0.000000}%
\pgfsetstrokecolor{currentstroke}%
\pgfsetdash{}{0pt}%
\pgfsys@defobject{currentmarker}{\pgfqpoint{-0.041667in}{-0.041667in}}{\pgfqpoint{0.041667in}{0.041667in}}{%
\pgfpathmoveto{\pgfqpoint{-0.041667in}{-0.041667in}}%
\pgfpathlineto{\pgfqpoint{0.041667in}{-0.041667in}}%
\pgfpathlineto{\pgfqpoint{0.041667in}{0.041667in}}%
\pgfpathlineto{\pgfqpoint{-0.041667in}{0.041667in}}%
\pgfpathclose%
\pgfusepath{stroke,fill}%
}%
\begin{pgfscope}%
\pgfsys@transformshift{0.785878in}{0.521603in}%
\pgfsys@useobject{currentmarker}{}%
\end{pgfscope}%
\begin{pgfscope}%
\pgfsys@transformshift{1.228756in}{0.932756in}%
\pgfsys@useobject{currentmarker}{}%
\end{pgfscope}%
\begin{pgfscope}%
\pgfsys@transformshift{1.671634in}{1.307750in}%
\pgfsys@useobject{currentmarker}{}%
\end{pgfscope}%
\begin{pgfscope}%
\pgfsys@transformshift{2.114513in}{1.651719in}%
\pgfsys@useobject{currentmarker}{}%
\end{pgfscope}%
\begin{pgfscope}%
\pgfsys@transformshift{2.557391in}{1.968429in}%
\pgfsys@useobject{currentmarker}{}%
\end{pgfscope}%
\begin{pgfscope}%
\pgfsys@transformshift{3.000269in}{2.260815in}%
\pgfsys@useobject{currentmarker}{}%
\end{pgfscope}%
\begin{pgfscope}%
\pgfsys@transformshift{3.443147in}{2.531478in}%
\pgfsys@useobject{currentmarker}{}%
\end{pgfscope}%
\begin{pgfscope}%
\pgfsys@transformshift{3.886026in}{2.783138in}%
\pgfsys@useobject{currentmarker}{}%
\end{pgfscope}%
\begin{pgfscope}%
\pgfsys@transformshift{4.328904in}{3.018962in}%
\pgfsys@useobject{currentmarker}{}%
\end{pgfscope}%
\begin{pgfscope}%
\pgfsys@transformshift{4.771782in}{3.242617in}%
\pgfsys@useobject{currentmarker}{}%
\end{pgfscope}%
\end{pgfscope}%
\begin{pgfscope}%
\pgfpathrectangle{\pgfqpoint{0.564660in}{0.521603in}}{\pgfqpoint{4.650000in}{3.020000in}} %
\pgfusepath{clip}%
\pgfsetrectcap%
\pgfsetroundjoin%
\pgfsetlinewidth{1.505625pt}%
\definecolor{currentstroke}{rgb}{0.000000,0.000000,1.000000}%
\pgfsetstrokecolor{currentstroke}%
\pgfsetdash{}{0pt}%
\pgfpathmoveto{\pgfqpoint{0.785878in}{0.521603in}}%
\pgfpathlineto{\pgfqpoint{0.905455in}{0.636519in}}%
\pgfpathlineto{\pgfqpoint{1.025032in}{0.748306in}}%
\pgfpathlineto{\pgfqpoint{1.144609in}{0.856981in}}%
\pgfpathlineto{\pgfqpoint{1.264186in}{0.962559in}}%
\pgfpathlineto{\pgfqpoint{1.379335in}{1.061313in}}%
\pgfpathlineto{\pgfqpoint{1.494483in}{1.157222in}}%
\pgfpathlineto{\pgfqpoint{1.609631in}{1.250299in}}%
\pgfpathlineto{\pgfqpoint{1.724780in}{1.340555in}}%
\pgfpathlineto{\pgfqpoint{1.839928in}{1.428002in}}%
\pgfpathlineto{\pgfqpoint{1.955076in}{1.512652in}}%
\pgfpathlineto{\pgfqpoint{2.070225in}{1.594514in}}%
\pgfpathlineto{\pgfqpoint{2.185373in}{1.673600in}}%
\pgfpathlineto{\pgfqpoint{2.300522in}{1.749918in}}%
\pgfpathlineto{\pgfqpoint{2.415670in}{1.823480in}}%
\pgfpathlineto{\pgfqpoint{2.526389in}{1.891620in}}%
\pgfpathlineto{\pgfqpoint{2.637109in}{1.957227in}}%
\pgfpathlineto{\pgfqpoint{2.747829in}{2.020309in}}%
\pgfpathlineto{\pgfqpoint{2.858548in}{2.080872in}}%
\pgfpathlineto{\pgfqpoint{2.969268in}{2.138923in}}%
\pgfpathlineto{\pgfqpoint{3.079987in}{2.194469in}}%
\pgfpathlineto{\pgfqpoint{3.190707in}{2.247517in}}%
\pgfpathlineto{\pgfqpoint{3.301426in}{2.298071in}}%
\pgfpathlineto{\pgfqpoint{3.412146in}{2.346137in}}%
\pgfpathlineto{\pgfqpoint{3.522865in}{2.391722in}}%
\pgfpathlineto{\pgfqpoint{3.633585in}{2.434830in}}%
\pgfpathlineto{\pgfqpoint{3.744305in}{2.475465in}}%
\pgfpathlineto{\pgfqpoint{3.850595in}{2.512153in}}%
\pgfpathlineto{\pgfqpoint{3.956886in}{2.546570in}}%
\pgfpathlineto{\pgfqpoint{4.063177in}{2.578720in}}%
\pgfpathlineto{\pgfqpoint{4.169468in}{2.608606in}}%
\pgfpathlineto{\pgfqpoint{4.275758in}{2.636231in}}%
\pgfpathlineto{\pgfqpoint{4.382049in}{2.661598in}}%
\pgfpathlineto{\pgfqpoint{4.488340in}{2.684709in}}%
\pgfpathlineto{\pgfqpoint{4.594631in}{2.705567in}}%
\pgfpathlineto{\pgfqpoint{4.700922in}{2.724173in}}%
\pgfpathlineto{\pgfqpoint{4.807212in}{2.740530in}}%
\pgfpathlineto{\pgfqpoint{4.913503in}{2.754639in}}%
\pgfpathlineto{\pgfqpoint{5.019794in}{2.766501in}}%
\pgfpathlineto{\pgfqpoint{5.126085in}{2.776118in}}%
\pgfpathlineto{\pgfqpoint{5.210231in}{2.782140in}}%
\pgfpathlineto{\pgfqpoint{5.210231in}{2.782140in}}%
\pgfusepath{stroke}%
\end{pgfscope}%
\begin{pgfscope}%
\pgfpathrectangle{\pgfqpoint{0.564660in}{0.521603in}}{\pgfqpoint{4.650000in}{3.020000in}} %
\pgfusepath{clip}%
\pgfsetbuttcap%
\pgfsetroundjoin%
\definecolor{currentfill}{rgb}{0.000000,0.000000,1.000000}%
\pgfsetfillcolor{currentfill}%
\pgfsetlinewidth{1.003750pt}%
\definecolor{currentstroke}{rgb}{0.000000,0.000000,1.000000}%
\pgfsetstrokecolor{currentstroke}%
\pgfsetdash{}{0pt}%
\pgfsys@defobject{currentmarker}{\pgfqpoint{-0.041667in}{-0.041667in}}{\pgfqpoint{0.041667in}{0.041667in}}{%
\pgfpathmoveto{\pgfqpoint{0.000000in}{-0.041667in}}%
\pgfpathcurveto{\pgfqpoint{0.011050in}{-0.041667in}}{\pgfqpoint{0.021649in}{-0.037276in}}{\pgfqpoint{0.029463in}{-0.029463in}}%
\pgfpathcurveto{\pgfqpoint{0.037276in}{-0.021649in}}{\pgfqpoint{0.041667in}{-0.011050in}}{\pgfqpoint{0.041667in}{0.000000in}}%
\pgfpathcurveto{\pgfqpoint{0.041667in}{0.011050in}}{\pgfqpoint{0.037276in}{0.021649in}}{\pgfqpoint{0.029463in}{0.029463in}}%
\pgfpathcurveto{\pgfqpoint{0.021649in}{0.037276in}}{\pgfqpoint{0.011050in}{0.041667in}}{\pgfqpoint{0.000000in}{0.041667in}}%
\pgfpathcurveto{\pgfqpoint{-0.011050in}{0.041667in}}{\pgfqpoint{-0.021649in}{0.037276in}}{\pgfqpoint{-0.029463in}{0.029463in}}%
\pgfpathcurveto{\pgfqpoint{-0.037276in}{0.021649in}}{\pgfqpoint{-0.041667in}{0.011050in}}{\pgfqpoint{-0.041667in}{0.000000in}}%
\pgfpathcurveto{\pgfqpoint{-0.041667in}{-0.011050in}}{\pgfqpoint{-0.037276in}{-0.021649in}}{\pgfqpoint{-0.029463in}{-0.029463in}}%
\pgfpathcurveto{\pgfqpoint{-0.021649in}{-0.037276in}}{\pgfqpoint{-0.011050in}{-0.041667in}}{\pgfqpoint{0.000000in}{-0.041667in}}%
\pgfpathclose%
\pgfusepath{stroke,fill}%
}%
\begin{pgfscope}%
\pgfsys@transformshift{0.785878in}{0.521603in}%
\pgfsys@useobject{currentmarker}{}%
\end{pgfscope}%
\begin{pgfscope}%
\pgfsys@transformshift{1.228756in}{0.931598in}%
\pgfsys@useobject{currentmarker}{}%
\end{pgfscope}%
\begin{pgfscope}%
\pgfsys@transformshift{1.671634in}{1.299248in}%
\pgfsys@useobject{currentmarker}{}%
\end{pgfscope}%
\begin{pgfscope}%
\pgfsys@transformshift{2.114513in}{1.625260in}%
\pgfsys@useobject{currentmarker}{}%
\end{pgfscope}%
\begin{pgfscope}%
\pgfsys@transformshift{2.557391in}{1.910245in}%
\pgfsys@useobject{currentmarker}{}%
\end{pgfscope}%
\begin{pgfscope}%
\pgfsys@transformshift{3.000269in}{2.154728in}%
\pgfsys@useobject{currentmarker}{}%
\end{pgfscope}%
\begin{pgfscope}%
\pgfsys@transformshift{3.443147in}{2.359151in}%
\pgfsys@useobject{currentmarker}{}%
\end{pgfscope}%
\begin{pgfscope}%
\pgfsys@transformshift{3.886026in}{2.523878in}%
\pgfsys@useobject{currentmarker}{}%
\end{pgfscope}%
\begin{pgfscope}%
\pgfsys@transformshift{4.328904in}{2.649197in}%
\pgfsys@useobject{currentmarker}{}%
\end{pgfscope}%
\begin{pgfscope}%
\pgfsys@transformshift{4.771782in}{2.735327in}%
\pgfsys@useobject{currentmarker}{}%
\end{pgfscope}%
\end{pgfscope}%
\begin{pgfscope}%
\pgfpathrectangle{\pgfqpoint{0.564660in}{0.521603in}}{\pgfqpoint{4.650000in}{3.020000in}} %
\pgfusepath{clip}%
\pgfsetbuttcap%
\pgfsetroundjoin%
\pgfsetlinewidth{1.505625pt}%
\definecolor{currentstroke}{rgb}{0.000000,0.750000,0.750000}%
\pgfsetstrokecolor{currentstroke}%
\pgfsetdash{{9.600000pt}{2.400000pt}{1.500000pt}{2.400000pt}}{0.000000pt}%
\pgfpathmoveto{\pgfqpoint{0.785878in}{0.521603in}}%
\pgfpathlineto{\pgfqpoint{0.905455in}{0.636517in}}%
\pgfpathlineto{\pgfqpoint{1.025032in}{0.748287in}}%
\pgfpathlineto{\pgfqpoint{1.140181in}{0.852948in}}%
\pgfpathlineto{\pgfqpoint{1.255329in}{0.954696in}}%
\pgfpathlineto{\pgfqpoint{1.370477in}{1.053535in}}%
\pgfpathlineto{\pgfqpoint{1.485626in}{1.149464in}}%
\pgfpathlineto{\pgfqpoint{1.600774in}{1.242485in}}%
\pgfpathlineto{\pgfqpoint{1.715922in}{1.332599in}}%
\pgfpathlineto{\pgfqpoint{1.831071in}{1.419807in}}%
\pgfpathlineto{\pgfqpoint{1.941790in}{1.500923in}}%
\pgfpathlineto{\pgfqpoint{2.052510in}{1.579354in}}%
\pgfpathlineto{\pgfqpoint{2.163229in}{1.655102in}}%
\pgfpathlineto{\pgfqpoint{2.273949in}{1.728168in}}%
\pgfpathlineto{\pgfqpoint{2.384668in}{1.798552in}}%
\pgfpathlineto{\pgfqpoint{2.495388in}{1.866256in}}%
\pgfpathlineto{\pgfqpoint{2.606107in}{1.931280in}}%
\pgfpathlineto{\pgfqpoint{2.712398in}{1.991183in}}%
\pgfpathlineto{\pgfqpoint{2.818689in}{2.048617in}}%
\pgfpathlineto{\pgfqpoint{2.924980in}{2.103584in}}%
\pgfpathlineto{\pgfqpoint{3.031271in}{2.156084in}}%
\pgfpathlineto{\pgfqpoint{3.137561in}{2.206118in}}%
\pgfpathlineto{\pgfqpoint{3.243852in}{2.253686in}}%
\pgfpathlineto{\pgfqpoint{3.350143in}{2.298789in}}%
\pgfpathlineto{\pgfqpoint{3.456434in}{2.341427in}}%
\pgfpathlineto{\pgfqpoint{3.562724in}{2.381601in}}%
\pgfpathlineto{\pgfqpoint{3.669015in}{2.419311in}}%
\pgfpathlineto{\pgfqpoint{3.770877in}{2.453138in}}%
\pgfpathlineto{\pgfqpoint{3.872739in}{2.484704in}}%
\pgfpathlineto{\pgfqpoint{3.974601in}{2.514008in}}%
\pgfpathlineto{\pgfqpoint{4.076463in}{2.541050in}}%
\pgfpathlineto{\pgfqpoint{4.178325in}{2.565832in}}%
\pgfpathlineto{\pgfqpoint{4.280187in}{2.588352in}}%
\pgfpathlineto{\pgfqpoint{4.382049in}{2.608613in}}%
\pgfpathlineto{\pgfqpoint{4.483911in}{2.626612in}}%
\pgfpathlineto{\pgfqpoint{4.585773in}{2.642352in}}%
\pgfpathlineto{\pgfqpoint{4.687635in}{2.655832in}}%
\pgfpathlineto{\pgfqpoint{4.789497in}{2.667051in}}%
\pgfpathlineto{\pgfqpoint{4.891359in}{2.676012in}}%
\pgfpathlineto{\pgfqpoint{4.993221in}{2.682712in}}%
\pgfpathlineto{\pgfqpoint{5.095083in}{2.687153in}}%
\pgfpathlineto{\pgfqpoint{5.196945in}{2.689334in}}%
\pgfpathlineto{\pgfqpoint{5.210231in}{2.689452in}}%
\pgfpathlineto{\pgfqpoint{5.210231in}{2.689452in}}%
\pgfusepath{stroke}%
\end{pgfscope}%
\begin{pgfscope}%
\pgfpathrectangle{\pgfqpoint{0.564660in}{0.521603in}}{\pgfqpoint{4.650000in}{3.020000in}} %
\pgfusepath{clip}%
\pgfsetbuttcap%
\pgfsetmiterjoin%
\definecolor{currentfill}{rgb}{0.000000,0.750000,0.750000}%
\pgfsetfillcolor{currentfill}%
\pgfsetlinewidth{1.003750pt}%
\definecolor{currentstroke}{rgb}{0.000000,0.750000,0.750000}%
\pgfsetstrokecolor{currentstroke}%
\pgfsetdash{}{0pt}%
\pgfsys@defobject{currentmarker}{\pgfqpoint{-0.041667in}{-0.041667in}}{\pgfqpoint{0.041667in}{0.041667in}}{%
\pgfpathmoveto{\pgfqpoint{-0.000000in}{-0.041667in}}%
\pgfpathlineto{\pgfqpoint{0.041667in}{0.041667in}}%
\pgfpathlineto{\pgfqpoint{-0.041667in}{0.041667in}}%
\pgfpathclose%
\pgfusepath{stroke,fill}%
}%
\begin{pgfscope}%
\pgfsys@transformshift{0.785878in}{0.521603in}%
\pgfsys@useobject{currentmarker}{}%
\end{pgfscope}%
\begin{pgfscope}%
\pgfsys@transformshift{1.228756in}{0.931474in}%
\pgfsys@useobject{currentmarker}{}%
\end{pgfscope}%
\begin{pgfscope}%
\pgfsys@transformshift{1.671634in}{1.298283in}%
\pgfsys@useobject{currentmarker}{}%
\end{pgfscope}%
\begin{pgfscope}%
\pgfsys@transformshift{2.114513in}{1.622104in}%
\pgfsys@useobject{currentmarker}{}%
\end{pgfscope}%
\begin{pgfscope}%
\pgfsys@transformshift{2.557391in}{1.902999in}%
\pgfsys@useobject{currentmarker}{}%
\end{pgfscope}%
\begin{pgfscope}%
\pgfsys@transformshift{3.000269in}{2.141026in}%
\pgfsys@useobject{currentmarker}{}%
\end{pgfscope}%
\begin{pgfscope}%
\pgfsys@transformshift{3.443147in}{2.336232in}%
\pgfsys@useobject{currentmarker}{}%
\end{pgfscope}%
\begin{pgfscope}%
\pgfsys@transformshift{3.886026in}{2.488654in}%
\pgfsys@useobject{currentmarker}{}%
\end{pgfscope}%
\begin{pgfscope}%
\pgfsys@transformshift{4.328904in}{2.598324in}%
\pgfsys@useobject{currentmarker}{}%
\end{pgfscope}%
\begin{pgfscope}%
\pgfsys@transformshift{4.771782in}{2.665263in}%
\pgfsys@useobject{currentmarker}{}%
\end{pgfscope}%
\end{pgfscope}%
\begin{pgfscope}%
\pgfsetrectcap%
\pgfsetmiterjoin%
\pgfsetlinewidth{0.803000pt}%
\definecolor{currentstroke}{rgb}{0.000000,0.000000,0.000000}%
\pgfsetstrokecolor{currentstroke}%
\pgfsetdash{}{0pt}%
\pgfpathmoveto{\pgfqpoint{0.564660in}{0.521603in}}%
\pgfpathlineto{\pgfqpoint{0.564660in}{3.541603in}}%
\pgfusepath{stroke}%
\end{pgfscope}%
\begin{pgfscope}%
\pgfsetrectcap%
\pgfsetmiterjoin%
\pgfsetlinewidth{0.803000pt}%
\definecolor{currentstroke}{rgb}{0.000000,0.000000,0.000000}%
\pgfsetstrokecolor{currentstroke}%
\pgfsetdash{}{0pt}%
\pgfpathmoveto{\pgfqpoint{5.214660in}{0.521603in}}%
\pgfpathlineto{\pgfqpoint{5.214660in}{3.541603in}}%
\pgfusepath{stroke}%
\end{pgfscope}%
\begin{pgfscope}%
\pgfsetrectcap%
\pgfsetmiterjoin%
\pgfsetlinewidth{0.803000pt}%
\definecolor{currentstroke}{rgb}{0.000000,0.000000,0.000000}%
\pgfsetstrokecolor{currentstroke}%
\pgfsetdash{}{0pt}%
\pgfpathmoveto{\pgfqpoint{0.564660in}{0.521603in}}%
\pgfpathlineto{\pgfqpoint{5.214660in}{0.521603in}}%
\pgfusepath{stroke}%
\end{pgfscope}%
\begin{pgfscope}%
\pgfsetrectcap%
\pgfsetmiterjoin%
\pgfsetlinewidth{0.803000pt}%
\definecolor{currentstroke}{rgb}{0.000000,0.000000,0.000000}%
\pgfsetstrokecolor{currentstroke}%
\pgfsetdash{}{0pt}%
\pgfpathmoveto{\pgfqpoint{0.564660in}{3.541603in}}%
\pgfpathlineto{\pgfqpoint{5.214660in}{3.541603in}}%
\pgfusepath{stroke}%
\end{pgfscope}%
\begin{pgfscope}%
\pgfsetbuttcap%
\pgfsetmiterjoin%
\definecolor{currentfill}{rgb}{1.000000,1.000000,1.000000}%
\pgfsetfillcolor{currentfill}%
\pgfsetfillopacity{0.800000}%
\pgfsetlinewidth{1.003750pt}%
\definecolor{currentstroke}{rgb}{0.800000,0.800000,0.800000}%
\pgfsetstrokecolor{currentstroke}%
\pgfsetstrokeopacity{0.800000}%
\pgfsetdash{}{0pt}%
\pgfpathmoveto{\pgfqpoint{0.661883in}{2.602470in}}%
\pgfpathlineto{\pgfqpoint{1.676618in}{2.602470in}}%
\pgfpathquadraticcurveto{\pgfqpoint{1.704396in}{2.602470in}}{\pgfqpoint{1.704396in}{2.630247in}}%
\pgfpathlineto{\pgfqpoint{1.704396in}{3.444381in}}%
\pgfpathquadraticcurveto{\pgfqpoint{1.704396in}{3.472159in}}{\pgfqpoint{1.676618in}{3.472159in}}%
\pgfpathlineto{\pgfqpoint{0.661883in}{3.472159in}}%
\pgfpathquadraticcurveto{\pgfqpoint{0.634105in}{3.472159in}}{\pgfqpoint{0.634105in}{3.444381in}}%
\pgfpathlineto{\pgfqpoint{0.634105in}{2.630247in}}%
\pgfpathquadraticcurveto{\pgfqpoint{0.634105in}{2.602470in}}{\pgfqpoint{0.661883in}{2.602470in}}%
\pgfpathclose%
\pgfusepath{stroke,fill}%
\end{pgfscope}%
\begin{pgfscope}%
\pgftext[x=0.781774in,y=3.298487in,left,base]{\rmfamily\fontsize{10.000000}{12.000000}\selectfont \(\displaystyle \beta\) = 6\(\displaystyle \times 10^{-4}\)}%
\end{pgfscope}%
\begin{pgfscope}%
\pgfsetbuttcap%
\pgfsetroundjoin%
\pgfsetlinewidth{1.505625pt}%
\definecolor{currentstroke}{rgb}{1.000000,0.000000,0.000000}%
\pgfsetstrokecolor{currentstroke}%
\pgfsetdash{{5.550000pt}{2.400000pt}}{0.000000pt}%
\pgfpathmoveto{\pgfqpoint{0.689660in}{3.143240in}}%
\pgfpathlineto{\pgfqpoint{0.967438in}{3.143240in}}%
\pgfusepath{stroke}%
\end{pgfscope}%
\begin{pgfscope}%
\pgfsetbuttcap%
\pgfsetmiterjoin%
\definecolor{currentfill}{rgb}{1.000000,0.000000,0.000000}%
\pgfsetfillcolor{currentfill}%
\pgfsetlinewidth{1.003750pt}%
\definecolor{currentstroke}{rgb}{1.000000,0.000000,0.000000}%
\pgfsetstrokecolor{currentstroke}%
\pgfsetdash{}{0pt}%
\pgfsys@defobject{currentmarker}{\pgfqpoint{-0.041667in}{-0.041667in}}{\pgfqpoint{0.041667in}{0.041667in}}{%
\pgfpathmoveto{\pgfqpoint{-0.041667in}{-0.041667in}}%
\pgfpathlineto{\pgfqpoint{0.041667in}{-0.041667in}}%
\pgfpathlineto{\pgfqpoint{0.041667in}{0.041667in}}%
\pgfpathlineto{\pgfqpoint{-0.041667in}{0.041667in}}%
\pgfpathclose%
\pgfusepath{stroke,fill}%
}%
\begin{pgfscope}%
\pgfsys@transformshift{0.828549in}{3.143240in}%
\pgfsys@useobject{currentmarker}{}%
\end{pgfscope}%
\end{pgfscope}%
\begin{pgfscope}%
\pgftext[x=1.078549in,y=3.094629in,left,base]{\rmfamily\fontsize{10.000000}{12.000000}\selectfont \(\displaystyle \epsilon\) = 1}%
\end{pgfscope}%
\begin{pgfscope}%
\pgfsetrectcap%
\pgfsetroundjoin%
\pgfsetlinewidth{1.505625pt}%
\definecolor{currentstroke}{rgb}{0.000000,0.000000,1.000000}%
\pgfsetstrokecolor{currentstroke}%
\pgfsetdash{}{0pt}%
\pgfpathmoveto{\pgfqpoint{0.689660in}{2.939383in}}%
\pgfpathlineto{\pgfqpoint{0.967438in}{2.939383in}}%
\pgfusepath{stroke}%
\end{pgfscope}%
\begin{pgfscope}%
\pgfsetbuttcap%
\pgfsetroundjoin%
\definecolor{currentfill}{rgb}{0.000000,0.000000,1.000000}%
\pgfsetfillcolor{currentfill}%
\pgfsetlinewidth{1.003750pt}%
\definecolor{currentstroke}{rgb}{0.000000,0.000000,1.000000}%
\pgfsetstrokecolor{currentstroke}%
\pgfsetdash{}{0pt}%
\pgfsys@defobject{currentmarker}{\pgfqpoint{-0.041667in}{-0.041667in}}{\pgfqpoint{0.041667in}{0.041667in}}{%
\pgfpathmoveto{\pgfqpoint{0.000000in}{-0.041667in}}%
\pgfpathcurveto{\pgfqpoint{0.011050in}{-0.041667in}}{\pgfqpoint{0.021649in}{-0.037276in}}{\pgfqpoint{0.029463in}{-0.029463in}}%
\pgfpathcurveto{\pgfqpoint{0.037276in}{-0.021649in}}{\pgfqpoint{0.041667in}{-0.011050in}}{\pgfqpoint{0.041667in}{0.000000in}}%
\pgfpathcurveto{\pgfqpoint{0.041667in}{0.011050in}}{\pgfqpoint{0.037276in}{0.021649in}}{\pgfqpoint{0.029463in}{0.029463in}}%
\pgfpathcurveto{\pgfqpoint{0.021649in}{0.037276in}}{\pgfqpoint{0.011050in}{0.041667in}}{\pgfqpoint{0.000000in}{0.041667in}}%
\pgfpathcurveto{\pgfqpoint{-0.011050in}{0.041667in}}{\pgfqpoint{-0.021649in}{0.037276in}}{\pgfqpoint{-0.029463in}{0.029463in}}%
\pgfpathcurveto{\pgfqpoint{-0.037276in}{0.021649in}}{\pgfqpoint{-0.041667in}{0.011050in}}{\pgfqpoint{-0.041667in}{0.000000in}}%
\pgfpathcurveto{\pgfqpoint{-0.041667in}{-0.011050in}}{\pgfqpoint{-0.037276in}{-0.021649in}}{\pgfqpoint{-0.029463in}{-0.029463in}}%
\pgfpathcurveto{\pgfqpoint{-0.021649in}{-0.037276in}}{\pgfqpoint{-0.011050in}{-0.041667in}}{\pgfqpoint{0.000000in}{-0.041667in}}%
\pgfpathclose%
\pgfusepath{stroke,fill}%
}%
\begin{pgfscope}%
\pgfsys@transformshift{0.828549in}{2.939383in}%
\pgfsys@useobject{currentmarker}{}%
\end{pgfscope}%
\end{pgfscope}%
\begin{pgfscope}%
\pgftext[x=1.078549in,y=2.890772in,left,base]{\rmfamily\fontsize{10.000000}{12.000000}\selectfont \(\displaystyle \epsilon\) = 0.1}%
\end{pgfscope}%
\begin{pgfscope}%
\pgfsetbuttcap%
\pgfsetroundjoin%
\pgfsetlinewidth{1.505625pt}%
\definecolor{currentstroke}{rgb}{0.000000,0.750000,0.750000}%
\pgfsetstrokecolor{currentstroke}%
\pgfsetdash{{9.600000pt}{2.400000pt}{1.500000pt}{2.400000pt}}{0.000000pt}%
\pgfpathmoveto{\pgfqpoint{0.689660in}{2.735526in}}%
\pgfpathlineto{\pgfqpoint{0.967438in}{2.735526in}}%
\pgfusepath{stroke}%
\end{pgfscope}%
\begin{pgfscope}%
\pgfsetbuttcap%
\pgfsetmiterjoin%
\definecolor{currentfill}{rgb}{0.000000,0.750000,0.750000}%
\pgfsetfillcolor{currentfill}%
\pgfsetlinewidth{1.003750pt}%
\definecolor{currentstroke}{rgb}{0.000000,0.750000,0.750000}%
\pgfsetstrokecolor{currentstroke}%
\pgfsetdash{}{0pt}%
\pgfsys@defobject{currentmarker}{\pgfqpoint{-0.041667in}{-0.041667in}}{\pgfqpoint{0.041667in}{0.041667in}}{%
\pgfpathmoveto{\pgfqpoint{-0.000000in}{-0.041667in}}%
\pgfpathlineto{\pgfqpoint{0.041667in}{0.041667in}}%
\pgfpathlineto{\pgfqpoint{-0.041667in}{0.041667in}}%
\pgfpathclose%
\pgfusepath{stroke,fill}%
}%
\begin{pgfscope}%
\pgfsys@transformshift{0.828549in}{2.735526in}%
\pgfsys@useobject{currentmarker}{}%
\end{pgfscope}%
\end{pgfscope}%
\begin{pgfscope}%
\pgftext[x=1.078549in,y=2.686915in,left,base]{\rmfamily\fontsize{10.000000}{12.000000}\selectfont \(\displaystyle \epsilon\) = 0.01}%
\end{pgfscope}%
\end{pgfpicture}%
\makeatother%
\endgroup%
}
    \caption{Long-time scaled drop trajectories for various values of $\phi \mathbb{E}\mbox{u}$.}
     \label{fig:long_times}
\end{figure}
We note that the trajectory reduces to the classical $\mathcal{O}(1)$ solution for small values of of $\phi \mathbb{E}\mbox{u}$. We also note that with $\mathbb{D}\mbox{g} = 6 \times 10^{-4}$ the effect of drag is slight, appearing only as a slight correction to the higher order terms.
 
By again applying a regular perturbation to the asymptotic solution, Equation \ref{perturb_viscous}, with the expansion
\[ t^* \sim t^*_0 + \phi \mathbb{E}\mbox{u} t^*_1 + \phi^2 \mathbb{E}\mbox{u}^2 t^*_2 \ldots \phi^n \mathbb{E}\mbox{u}^n t^*_n  
,\]
and solving for the roots at times when $y^* = 0$, we find an asymptotic estimate for the time-of-flight. The $\mathcal{O}(\phi^2 \mathbb{E}\mbox{u}^2)$ accurate time-of-flight estimate is given by
\begin{gather*}
t_f = 2 + \phi \mathbb{E}\mbox{u} \left(\frac{4}{3} - \frac{2 \mathbb{D}\mbox{g}}{3}\right) 
+ \phi^2 \mathbb{E}\mbox{u}^{2} \left(\frac{4}{5} - \frac{4 \mathbb{D}\mbox{g}}{3} + \frac{2 \mathbb{D}\mbox{g}^{2}}{5}\right) \\
+ \mathcal{O}(\phi^3 \mathbb{E}\mbox{u}^3).
\end{gather*}
Substituting the experimental value of $\mathbb{D}\mbox{g}$ we find the time-of-flight estimate for water drops is
\begin{equation} \label{time_of_flight}
t_f = 2 + 1.333 \phi \mathbb{E}\mbox{u} + 0.799 \phi^2 \mathbb{E}\mbox{u}^{2} + \mathcal{O}(\phi^3 \mathbb{E}\mbox{u}^3). 
\end{equation} 

As $\phi \mathbb{E}\mbox{u}$ grows to be no longer small, the time-of-flight grows rapidly with an asymptote at a certain critical velocity; this is an electrostatic escape velocity $U_e$. We can find the escape velocity by solving a limiting version of the equation of motion
\[ m u' = - \frac{q E_0 y_c^2}{y^2}, \]
where $u = \frac{d y}{d t}$ is the drop velocity. This has the solution
\[ u(y) = \pm U_0 \left(1 + \frac{2q E_0 y_c^2}{m U_0^2} \left( \frac{1}{y} - \frac{1}{R_d} \right) \right)^{1/2}.
\]
This equation has an asymptotic velocity $U_{\infty}$ at $y = \infty$, which is real provided 
\[ U_0 \geq  U_e = y_c \sqrt{\frac{2 q E_0 }{m R_d}},
\]
where $U_e$ is the escape velocity and $U_{\infty} = \sqrt{U_0^2 - U_e^2}$. If $y_c=L$ the condition for the drops to escape the electric field is then given by
\begin{equation}\label{escape}
\frac{1}{8 \pi} \phi \mathbb{E}\mbox{u} > 1.
\end{equation}

\section{Methods}
\subsection{Overview}
Using various scaling arguments we have gleaned from our simple model a set of dimensionless numbers characteristic of drop bounce apoapses and times of flight. The dimensionless numbers depend on physical properties not all of which are easily measured by experiment. In particular, direct determination of net drop free electric charge $q$ is difficult as high-input resistance electrometers are not well-suited to the sudden 15-$g_0$ decelerations characteristic of drop tower experiments. To estimate the drop free charge $q$ we use parameter estimation techniques. Our work flow to identify $q$ in a drop tower experiment is as follows:
\begin{enumerate}
\item We vary the independent variables drop volume $V_d$ and dielectric surface charge density $\sigma$ in a set of single-drop spontaneous drop jump experiments under low-gravity in a $2.1$ s drop tower. 
\item We capture video and digitize the trajectories of the drops. 
\item We solve the inverse problem to find the drop free charge $q$ by maximizing the log-likelihood of the experimental trajectories given the dynamical model by varying the parameter vector $\mathbf{x} = \left\langle q, \hspace{2 mm} V_d, \hspace{2 mm} \sigma \right\rangle$ using a direct search optimization.                     
\end{enumerate}

\subsection{Experimental Methods}
The Dryden Drop Tower at Portland State University uses a dual capsule design, inspired by the 2.2 s facility at NASA Glenn Research Center, which decouples drag acceleration felt by the external drag shield from the experiment. The experiment experiences approximately $\lesssim 1 \times 10^{-4}$ $g_0$ during free-fall for 2.1 s as the experiment and drag shield togeather plummet to the bottom of drop tower 6 stories below. A drop tower rig with a mounted experiment is shown in Figure \ref{fig:rig}. Single drops of distilled water in a range of volumes ($0.01 \leq V_d \leq 0.5$ mL) are carefully deposited on rig-mounted charged superhydrophobic substrates using a grounded glass syringe with $\pm $1 $\mu$L accuracy and then dropped in the drop tower. In this work 16 such single-drop drop jump experiments were conducted in low-gravity. Drops are colored with red dye to improve thresholding accuracy in trajectory digitization. Drop trajectories are captured using a Panasonic HC-WX970 Camera recording at 120 fps. Drop trajectories are digitized from video using the particle tracking module in Fiji \cite{schindelin_fiji:_2012}.

\begin{figure}
    \centering
    \fontsize{12pt}{13pt}\selectfont
    \def\svgwidth{\columnwidth}
        \input{../figures/rig_paper.pdf_tex}%
    \caption{The electro-drop bounce experiment hardware mounted on a drop tower rig. (a) HV DC-DC converter and control electronics. (b) Camera. (c) Light panel. (d) Test cell. \label{fig:rig}}
\end{figure}

Superhydrophobic electret substrates are prepared with surface potentials $\varphi_s = 0.4$-$1.8$ kV. We use an isothermal electret formation process which is a variation of the widely applied corona-charging technique. A Ptec IN5120 balanced AC corona ion source directs a net neutral stream of ions towards a dielectric substrate, which we polarize by an embedded electrode with an EMCO P20P $2$ kV$+$ absolute reference DC-DC converter. The ion stream compensates the surface and space bound charges arising due to the polarization of the dielectric. After the corona source and DC-DC converter are powered off and the electrode is shorted across a bleed resistor the deposited negative ions remain on the substrate surface. The electret is lamina of 3 to 4 0.4 mm thick corona charged polymethyl methacrylate (PMMA) sheets. The electric field strength scales with the number of dielectric lamina as has also been shown in work on laminated electret based vibrational energy harvesters \cite{wada_stacking_2012} and in water desalinators \cite{ni_desalination_2005}. The RC time constant for decay of the surface charge is observed to be $\tau \approx 2000$ s.

The topmost face of the electret lamina is made superhydrophobic by first laser ablating a pillared topology  on the surface followed by deposition of a thin layer of PTFE. We use a pattern with pillar heights $\sim \hspace{-1mm} 775$ $\mu$m, widths $\sim \hspace{-1mm} 70$ $\mu$m, and pitch $\sim \hspace{-1mm} 100$ $\mu$m. An SEM image of the pillar geometry is shown in Figure \ref{fig:SEM}. Contact angles of distilled water on the surface, measured using the tangent method, are $\sim \hspace{-1mm} 150^{\circ}$ when the surface is uncharged. The roll-off angle of a 1 mL drop is $1^{\circ} \pm 0.5^{\circ}$. \sout{The hysteresis of the contact angle (the difference between the advancing $\theta_a$ and receding $\theta_r$ contact angles) is estimated from the roll-off angle using the model of Furmidge \cite{furmidge_studies_1962}, and is found to be approximately $25^{\circ} \pm 10^{\circ}$.}

\begin{figure}
    \centering
    \def\svgwidth{\columnwidth}
        \includegraphics[width=0.4\textwidth]{../figures/close_up.pdf}
    \caption{Close up of the experimental test cell.\label{fig:SEM}}
\end{figure}

Surface potentials $\varphi_s$ are measured on the superhydrophobic surface using a Simco-Ion FMX-004 electrostatic fieldmeter and the method for determination of surface charge density for low conductivity polymers described in Davies \cite{davies_examination_1967}. For this measurement the fieldmeter is shielded, the electret substrate rests over a conductive ground plane, and the surface charge density is determined from $\sigma = \varphi_s \kappa \epsilon_0/l,$ where $l$ is the thickness of the dielectric substrate. The relative dielectric constant of the PMMA sheet $\kappa$, is measured by using a 65$\times$65 mm polished aluminum parallel plate capacitor with $C = \kappa \epsilon_0 A/l$, where $C$ is the capacitance, and $A$ is the sheet area. Measuring the capacitance with 3 sample thicknesses using a GenRad 1657 RLC Digibridge, we find the relative permittivity to be $\kappa = 3.5 \pm 0.4$.  

\subsection{Parameter Estimation}
Generally we seek the parameters $\mathbf{x}$ that solve the inverse problem $G(\mathbf{x}) + \mathbf{u} = \mathbf{d}$, where the model $G(\mathbf{x})$ describes some relationship between the vector of parameters $\mathbf{x}$ and a set of observations $\mathbf{d}$, and where $\mathbf{u}$ is the measurement error. In particular we seek the parameter set $\mathbf{x} = \left\langle q, \hspace{2 mm} V_d, \hspace{2 mm} \sigma \right\rangle$ that has the highest probability (i.e. the maximum of the posterior Probability Density Function) of observing the data given the model, which is a numerical solution to Equation \ref{gov_eqn_subs}. This can be determined by maximizing the log-likelihood (or equivalently by minimizing $\chi^2$ goodness-of-fit) of the data $\mathbf{d}$ with respect to the model. This problem is formally stated as  
\[
\mbox{min} \hspace{2 mm} \chi^2 = \mbox{min} \hspace{2 mm} \sum^n_{i=1} \frac{\left({y_d(t)}_i - y_G(t, \mathbf{x})_i \right)^2}{{\sigma_d}_i},
\]
\begin{eqnarray*} \mbox{} \hspace{2 mm} \begin{split} \mathbf{x} = \left\{ \begin{array}{ll}      & q\\
		  &	V_d\\
          & \sigma 
          \end{array}, \right. 
          \end{split} \hspace{2 mm} \mbox{subject to constraints} \hspace{2 mm} \begin{split}
          g = \left\{ \begin{array}{ll}
           V_d &\pm \hspace{2 mm} u_{exp}\\
      	   \sigma &\pm  \hspace{2 mm} u_{exp}\\
      	   y_0 &\pm \hspace{2 mm} u_{exp}\\
      	   t_0 &\pm \hspace{2 mm} u_{exp}\\
          \end{array}, \right. 
          \end{split}
\end{eqnarray*}
where $y_G(t, \mathbf{x})$ is the $y$-coordinate position at time $t$ of the numerical solution of the equation of motion, $y_d(t)$ is the corresponding experimentally observed drop $y$-coordinate position at time $t$, $\sigma_d$ is the standard error of the observed position, and $u_{exp}$ are the measurement uncertainties. The vector $\mathbf{x}$ that minimizes $\chi^2$ is the so called Maximum Likelihood Estimate (MLE) of the experimental parameters.

We integrate Equation \ref{gov_eqn_subs} numerically using the \emph{netlib ODEPACK} \verb|lsoda| integrator implimented as \verb|scipy.integrate.odeint| in the \emph{Scipy} \cite{oliphant_python_2007} library in Python. The optimization problem in this case is non-convex, mixed discrete-continuous black-box (noisy), and highly ill-conditioned. The ill-conditioning arises due to the strong covariance between several of the model parameters, namely $q=q(V_d, E_0)$. The non-convexity of the problem implies that there are many local minima of the objective function. We use a gradient-free direct-search Nelder-Mead \cite{nelder_simplex_1965} algorithm implemented as \verb|scipy.optimize.minimize(method='Nelder-Mead')| in \emph{SciPy}. Nelder-Mead is relatively robust to noise and is thrifty with our computationally expensive function-calls. Experimental trajectory data is smoothed using a Savitsky-Golay filter \cite{savitzky_smoothing_1964} to improve the convergence characteristics of the optimizer. This filter is implemented as \verb|scipy.signal.savgol_filter| in \emph{SciPy}. We precondition the optimization problem by minimizing $\ln(\chi^2)$ and using a naive $\sim \mathcal{O}(1)$ scaling of our constraints by their initial guesses. Here the goal is to make the problem equally sensitive to steps in any direction. The so-called identifiability problem of bounding uncertainty of the parameter estimates is resolved by constraining the parameter values by our experimental observations of them and their associated measurement uncertainties. Because Nelder-Mead cannot be used for explicitly constrained problems we implement the constraints using an exterior penalty function. \sout{In the case of $q$, which we do not directly observe in experiment, we bound identifiability by using a Montecarlo bootstrap approach [WIP].}

\begin{figure}[h]
    \centering
    \resizebox{0.5\textwidth}{!}{%% Creator: Matplotlib, PGF backend
%%
%% To include the figure in your LaTeX document, write
%%   \input{<filename>.pgf}
%%
%% Make sure the required packages are loaded in your preamble
%%   \usepackage{pgf}
%%
%% Figures using additional raster images can only be included by \input if
%% they are in the same directory as the main LaTeX file. For loading figures
%% from other directories you can use the `import` package
%%   \usepackage{import}
%% and then include the figures with
%%   \import{<path to file>}{<filename>.pgf}
%%
%% Matplotlib used the following preamble
%%   \usepackage{fontspec}
%%   \setmainfont{DejaVu Serif}
%%   \setsansfont{DejaVu Sans}
%%   \setmonofont{DejaVu Sans Mono}
%%
\begingroup%
\makeatletter%
\begin{pgfpicture}%
\pgfpathrectangle{\pgfpointorigin}{\pgfqpoint{5.390049in}{3.837899in}}%
\pgfusepath{use as bounding box, clip}%
\begin{pgfscope}%
\pgfsetbuttcap%
\pgfsetmiterjoin%
\definecolor{currentfill}{rgb}{1.000000,1.000000,1.000000}%
\pgfsetfillcolor{currentfill}%
\pgfsetlinewidth{0.000000pt}%
\definecolor{currentstroke}{rgb}{1.000000,1.000000,1.000000}%
\pgfsetstrokecolor{currentstroke}%
\pgfsetdash{}{0pt}%
\pgfpathmoveto{\pgfqpoint{0.000000in}{0.000000in}}%
\pgfpathlineto{\pgfqpoint{5.390049in}{0.000000in}}%
\pgfpathlineto{\pgfqpoint{5.390049in}{3.837899in}}%
\pgfpathlineto{\pgfqpoint{0.000000in}{3.837899in}}%
\pgfpathclose%
\pgfusepath{fill}%
\end{pgfscope}%
\begin{pgfscope}%
\pgfsetbuttcap%
\pgfsetmiterjoin%
\definecolor{currentfill}{rgb}{1.000000,1.000000,1.000000}%
\pgfsetfillcolor{currentfill}%
\pgfsetlinewidth{0.000000pt}%
\definecolor{currentstroke}{rgb}{0.000000,0.000000,0.000000}%
\pgfsetstrokecolor{currentstroke}%
\pgfsetstrokeopacity{0.000000}%
\pgfsetdash{}{0pt}%
\pgfpathmoveto{\pgfqpoint{0.578349in}{0.682899in}}%
\pgfpathlineto{\pgfqpoint{5.228349in}{0.682899in}}%
\pgfpathlineto{\pgfqpoint{5.228349in}{3.702899in}}%
\pgfpathlineto{\pgfqpoint{0.578349in}{3.702899in}}%
\pgfpathclose%
\pgfusepath{fill}%
\end{pgfscope}%
\begin{pgfscope}%
\pgfsetbuttcap%
\pgfsetroundjoin%
\definecolor{currentfill}{rgb}{0.000000,0.000000,0.000000}%
\pgfsetfillcolor{currentfill}%
\pgfsetlinewidth{0.803000pt}%
\definecolor{currentstroke}{rgb}{0.000000,0.000000,0.000000}%
\pgfsetstrokecolor{currentstroke}%
\pgfsetdash{}{0pt}%
\pgfsys@defobject{currentmarker}{\pgfqpoint{0.000000in}{-0.048611in}}{\pgfqpoint{0.000000in}{0.000000in}}{%
\pgfpathmoveto{\pgfqpoint{0.000000in}{0.000000in}}%
\pgfpathlineto{\pgfqpoint{0.000000in}{-0.048611in}}%
\pgfusepath{stroke,fill}%
}%
\begin{pgfscope}%
\pgfsys@transformshift{0.677979in}{0.682899in}%
\pgfsys@useobject{currentmarker}{}%
\end{pgfscope}%
\end{pgfscope}%
\begin{pgfscope}%
\pgftext[x=0.677979in,y=0.585677in,,top]{\rmfamily\fontsize{16.000000}{19.200000}\selectfont \(\displaystyle 0.0\)}%
\end{pgfscope}%
\begin{pgfscope}%
\pgfsetbuttcap%
\pgfsetroundjoin%
\definecolor{currentfill}{rgb}{0.000000,0.000000,0.000000}%
\pgfsetfillcolor{currentfill}%
\pgfsetlinewidth{0.803000pt}%
\definecolor{currentstroke}{rgb}{0.000000,0.000000,0.000000}%
\pgfsetstrokecolor{currentstroke}%
\pgfsetdash{}{0pt}%
\pgfsys@defobject{currentmarker}{\pgfqpoint{0.000000in}{-0.048611in}}{\pgfqpoint{0.000000in}{0.000000in}}{%
\pgfpathmoveto{\pgfqpoint{0.000000in}{0.000000in}}%
\pgfpathlineto{\pgfqpoint{0.000000in}{-0.048611in}}%
\pgfusepath{stroke,fill}%
}%
\begin{pgfscope}%
\pgfsys@transformshift{1.795320in}{0.682899in}%
\pgfsys@useobject{currentmarker}{}%
\end{pgfscope}%
\end{pgfscope}%
\begin{pgfscope}%
\pgftext[x=1.795320in,y=0.585677in,,top]{\rmfamily\fontsize{16.000000}{19.200000}\selectfont \(\displaystyle 0.5\)}%
\end{pgfscope}%
\begin{pgfscope}%
\pgfsetbuttcap%
\pgfsetroundjoin%
\definecolor{currentfill}{rgb}{0.000000,0.000000,0.000000}%
\pgfsetfillcolor{currentfill}%
\pgfsetlinewidth{0.803000pt}%
\definecolor{currentstroke}{rgb}{0.000000,0.000000,0.000000}%
\pgfsetstrokecolor{currentstroke}%
\pgfsetdash{}{0pt}%
\pgfsys@defobject{currentmarker}{\pgfqpoint{0.000000in}{-0.048611in}}{\pgfqpoint{0.000000in}{0.000000in}}{%
\pgfpathmoveto{\pgfqpoint{0.000000in}{0.000000in}}%
\pgfpathlineto{\pgfqpoint{0.000000in}{-0.048611in}}%
\pgfusepath{stroke,fill}%
}%
\begin{pgfscope}%
\pgfsys@transformshift{2.912660in}{0.682899in}%
\pgfsys@useobject{currentmarker}{}%
\end{pgfscope}%
\end{pgfscope}%
\begin{pgfscope}%
\pgftext[x=2.912660in,y=0.585677in,,top]{\rmfamily\fontsize{16.000000}{19.200000}\selectfont \(\displaystyle 1.0\)}%
\end{pgfscope}%
\begin{pgfscope}%
\pgfsetbuttcap%
\pgfsetroundjoin%
\definecolor{currentfill}{rgb}{0.000000,0.000000,0.000000}%
\pgfsetfillcolor{currentfill}%
\pgfsetlinewidth{0.803000pt}%
\definecolor{currentstroke}{rgb}{0.000000,0.000000,0.000000}%
\pgfsetstrokecolor{currentstroke}%
\pgfsetdash{}{0pt}%
\pgfsys@defobject{currentmarker}{\pgfqpoint{0.000000in}{-0.048611in}}{\pgfqpoint{0.000000in}{0.000000in}}{%
\pgfpathmoveto{\pgfqpoint{0.000000in}{0.000000in}}%
\pgfpathlineto{\pgfqpoint{0.000000in}{-0.048611in}}%
\pgfusepath{stroke,fill}%
}%
\begin{pgfscope}%
\pgfsys@transformshift{4.030001in}{0.682899in}%
\pgfsys@useobject{currentmarker}{}%
\end{pgfscope}%
\end{pgfscope}%
\begin{pgfscope}%
\pgftext[x=4.030001in,y=0.585677in,,top]{\rmfamily\fontsize{16.000000}{19.200000}\selectfont \(\displaystyle 1.5\)}%
\end{pgfscope}%
\begin{pgfscope}%
\pgfsetbuttcap%
\pgfsetroundjoin%
\definecolor{currentfill}{rgb}{0.000000,0.000000,0.000000}%
\pgfsetfillcolor{currentfill}%
\pgfsetlinewidth{0.803000pt}%
\definecolor{currentstroke}{rgb}{0.000000,0.000000,0.000000}%
\pgfsetstrokecolor{currentstroke}%
\pgfsetdash{}{0pt}%
\pgfsys@defobject{currentmarker}{\pgfqpoint{0.000000in}{-0.048611in}}{\pgfqpoint{0.000000in}{0.000000in}}{%
\pgfpathmoveto{\pgfqpoint{0.000000in}{0.000000in}}%
\pgfpathlineto{\pgfqpoint{0.000000in}{-0.048611in}}%
\pgfusepath{stroke,fill}%
}%
\begin{pgfscope}%
\pgfsys@transformshift{5.147342in}{0.682899in}%
\pgfsys@useobject{currentmarker}{}%
\end{pgfscope}%
\end{pgfscope}%
\begin{pgfscope}%
\pgftext[x=5.147342in,y=0.585677in,,top]{\rmfamily\fontsize{16.000000}{19.200000}\selectfont \(\displaystyle 2.0\)}%
\end{pgfscope}%
\begin{pgfscope}%
\pgftext[x=2.903349in,y=0.315061in,,top]{\rmfamily\fontsize{16.000000}{19.200000}\selectfont \(\displaystyle t\) (s)}%
\end{pgfscope}%
\begin{pgfscope}%
\pgfsetbuttcap%
\pgfsetroundjoin%
\definecolor{currentfill}{rgb}{0.000000,0.000000,0.000000}%
\pgfsetfillcolor{currentfill}%
\pgfsetlinewidth{0.803000pt}%
\definecolor{currentstroke}{rgb}{0.000000,0.000000,0.000000}%
\pgfsetstrokecolor{currentstroke}%
\pgfsetdash{}{0pt}%
\pgfsys@defobject{currentmarker}{\pgfqpoint{-0.048611in}{0.000000in}}{\pgfqpoint{0.000000in}{0.000000in}}{%
\pgfpathmoveto{\pgfqpoint{0.000000in}{0.000000in}}%
\pgfpathlineto{\pgfqpoint{-0.048611in}{0.000000in}}%
\pgfusepath{stroke,fill}%
}%
\begin{pgfscope}%
\pgfsys@transformshift{0.578349in}{1.430522in}%
\pgfsys@useobject{currentmarker}{}%
\end{pgfscope}%
\end{pgfscope}%
\begin{pgfscope}%
\pgftext[x=0.371059in,y=1.346104in,left,base]{\rmfamily\fontsize{16.000000}{19.200000}\selectfont \(\displaystyle 2\)}%
\end{pgfscope}%
\begin{pgfscope}%
\pgfsetbuttcap%
\pgfsetroundjoin%
\definecolor{currentfill}{rgb}{0.000000,0.000000,0.000000}%
\pgfsetfillcolor{currentfill}%
\pgfsetlinewidth{0.803000pt}%
\definecolor{currentstroke}{rgb}{0.000000,0.000000,0.000000}%
\pgfsetstrokecolor{currentstroke}%
\pgfsetdash{}{0pt}%
\pgfsys@defobject{currentmarker}{\pgfqpoint{-0.048611in}{0.000000in}}{\pgfqpoint{0.000000in}{0.000000in}}{%
\pgfpathmoveto{\pgfqpoint{0.000000in}{0.000000in}}%
\pgfpathlineto{\pgfqpoint{-0.048611in}{0.000000in}}%
\pgfusepath{stroke,fill}%
}%
\begin{pgfscope}%
\pgfsys@transformshift{0.578349in}{2.193161in}%
\pgfsys@useobject{currentmarker}{}%
\end{pgfscope}%
\end{pgfscope}%
\begin{pgfscope}%
\pgftext[x=0.371059in,y=2.108742in,left,base]{\rmfamily\fontsize{16.000000}{19.200000}\selectfont \(\displaystyle 4\)}%
\end{pgfscope}%
\begin{pgfscope}%
\pgfsetbuttcap%
\pgfsetroundjoin%
\definecolor{currentfill}{rgb}{0.000000,0.000000,0.000000}%
\pgfsetfillcolor{currentfill}%
\pgfsetlinewidth{0.803000pt}%
\definecolor{currentstroke}{rgb}{0.000000,0.000000,0.000000}%
\pgfsetstrokecolor{currentstroke}%
\pgfsetdash{}{0pt}%
\pgfsys@defobject{currentmarker}{\pgfqpoint{-0.048611in}{0.000000in}}{\pgfqpoint{0.000000in}{0.000000in}}{%
\pgfpathmoveto{\pgfqpoint{0.000000in}{0.000000in}}%
\pgfpathlineto{\pgfqpoint{-0.048611in}{0.000000in}}%
\pgfusepath{stroke,fill}%
}%
\begin{pgfscope}%
\pgfsys@transformshift{0.578349in}{2.955799in}%
\pgfsys@useobject{currentmarker}{}%
\end{pgfscope}%
\end{pgfscope}%
\begin{pgfscope}%
\pgftext[x=0.371059in,y=2.871381in,left,base]{\rmfamily\fontsize{16.000000}{19.200000}\selectfont \(\displaystyle 6\)}%
\end{pgfscope}%
\begin{pgfscope}%
\pgftext[x=0.315503in,y=2.192899in,,bottom,rotate=90.000000]{\rmfamily\fontsize{16.000000}{19.200000}\selectfont \(\displaystyle y\) (cm)}%
\end{pgfscope}%
\begin{pgfscope}%
\pgfpathrectangle{\pgfqpoint{0.578349in}{0.682899in}}{\pgfqpoint{4.650000in}{3.020000in}}%
\pgfusepath{clip}%
\pgfsetbuttcap%
\pgfsetroundjoin%
\definecolor{currentfill}{rgb}{0.000000,0.000000,0.000000}%
\pgfsetfillcolor{currentfill}%
\pgfsetfillopacity{0.500000}%
\pgfsetlinewidth{1.003750pt}%
\definecolor{currentstroke}{rgb}{0.000000,0.000000,0.000000}%
\pgfsetstrokecolor{currentstroke}%
\pgfsetstrokeopacity{0.500000}%
\pgfsetdash{}{0pt}%
\pgfsys@defobject{currentmarker}{\pgfqpoint{-0.041667in}{-0.041667in}}{\pgfqpoint{0.041667in}{0.041667in}}{%
\pgfpathmoveto{\pgfqpoint{0.000000in}{-0.041667in}}%
\pgfpathcurveto{\pgfqpoint{0.011050in}{-0.041667in}}{\pgfqpoint{0.021649in}{-0.037276in}}{\pgfqpoint{0.029463in}{-0.029463in}}%
\pgfpathcurveto{\pgfqpoint{0.037276in}{-0.021649in}}{\pgfqpoint{0.041667in}{-0.011050in}}{\pgfqpoint{0.041667in}{0.000000in}}%
\pgfpathcurveto{\pgfqpoint{0.041667in}{0.011050in}}{\pgfqpoint{0.037276in}{0.021649in}}{\pgfqpoint{0.029463in}{0.029463in}}%
\pgfpathcurveto{\pgfqpoint{0.021649in}{0.037276in}}{\pgfqpoint{0.011050in}{0.041667in}}{\pgfqpoint{0.000000in}{0.041667in}}%
\pgfpathcurveto{\pgfqpoint{-0.011050in}{0.041667in}}{\pgfqpoint{-0.021649in}{0.037276in}}{\pgfqpoint{-0.029463in}{0.029463in}}%
\pgfpathcurveto{\pgfqpoint{-0.037276in}{0.021649in}}{\pgfqpoint{-0.041667in}{0.011050in}}{\pgfqpoint{-0.041667in}{0.000000in}}%
\pgfpathcurveto{\pgfqpoint{-0.041667in}{-0.011050in}}{\pgfqpoint{-0.037276in}{-0.021649in}}{\pgfqpoint{-0.029463in}{-0.029463in}}%
\pgfpathcurveto{\pgfqpoint{-0.021649in}{-0.037276in}}{\pgfqpoint{-0.011050in}{-0.041667in}}{\pgfqpoint{0.000000in}{-0.041667in}}%
\pgfpathclose%
\pgfusepath{stroke,fill}%
}%
\begin{pgfscope}%
\pgfsys@transformshift{0.845580in}{1.035269in}%
\pgfsys@useobject{currentmarker}{}%
\end{pgfscope}%
\begin{pgfscope}%
\pgfsys@transformshift{0.864202in}{1.062311in}%
\pgfsys@useobject{currentmarker}{}%
\end{pgfscope}%
\begin{pgfscope}%
\pgfsys@transformshift{0.882825in}{1.089284in}%
\pgfsys@useobject{currentmarker}{}%
\end{pgfscope}%
\begin{pgfscope}%
\pgfsys@transformshift{0.901447in}{1.116182in}%
\pgfsys@useobject{currentmarker}{}%
\end{pgfscope}%
\begin{pgfscope}%
\pgfsys@transformshift{0.920069in}{1.143003in}%
\pgfsys@useobject{currentmarker}{}%
\end{pgfscope}%
\begin{pgfscope}%
\pgfsys@transformshift{0.938692in}{1.169743in}%
\pgfsys@useobject{currentmarker}{}%
\end{pgfscope}%
\begin{pgfscope}%
\pgfsys@transformshift{0.957314in}{1.196397in}%
\pgfsys@useobject{currentmarker}{}%
\end{pgfscope}%
\begin{pgfscope}%
\pgfsys@transformshift{0.975936in}{1.222963in}%
\pgfsys@useobject{currentmarker}{}%
\end{pgfscope}%
\begin{pgfscope}%
\pgfsys@transformshift{0.994559in}{1.249437in}%
\pgfsys@useobject{currentmarker}{}%
\end{pgfscope}%
\begin{pgfscope}%
\pgfsys@transformshift{1.013181in}{1.275814in}%
\pgfsys@useobject{currentmarker}{}%
\end{pgfscope}%
\begin{pgfscope}%
\pgfsys@transformshift{1.031803in}{1.302092in}%
\pgfsys@useobject{currentmarker}{}%
\end{pgfscope}%
\begin{pgfscope}%
\pgfsys@transformshift{1.050426in}{1.328266in}%
\pgfsys@useobject{currentmarker}{}%
\end{pgfscope}%
\begin{pgfscope}%
\pgfsys@transformshift{1.069048in}{1.354334in}%
\pgfsys@useobject{currentmarker}{}%
\end{pgfscope}%
\begin{pgfscope}%
\pgfsys@transformshift{1.087670in}{1.380285in}%
\pgfsys@useobject{currentmarker}{}%
\end{pgfscope}%
\begin{pgfscope}%
\pgfsys@transformshift{1.106293in}{1.406118in}%
\pgfsys@useobject{currentmarker}{}%
\end{pgfscope}%
\begin{pgfscope}%
\pgfsys@transformshift{1.124915in}{1.431829in}%
\pgfsys@useobject{currentmarker}{}%
\end{pgfscope}%
\begin{pgfscope}%
\pgfsys@transformshift{1.143538in}{1.457417in}%
\pgfsys@useobject{currentmarker}{}%
\end{pgfscope}%
\begin{pgfscope}%
\pgfsys@transformshift{1.162160in}{1.482877in}%
\pgfsys@useobject{currentmarker}{}%
\end{pgfscope}%
\begin{pgfscope}%
\pgfsys@transformshift{1.180782in}{1.508209in}%
\pgfsys@useobject{currentmarker}{}%
\end{pgfscope}%
\begin{pgfscope}%
\pgfsys@transformshift{1.199405in}{1.533411in}%
\pgfsys@useobject{currentmarker}{}%
\end{pgfscope}%
\begin{pgfscope}%
\pgfsys@transformshift{1.218027in}{1.558483in}%
\pgfsys@useobject{currentmarker}{}%
\end{pgfscope}%
\begin{pgfscope}%
\pgfsys@transformshift{1.236649in}{1.583424in}%
\pgfsys@useobject{currentmarker}{}%
\end{pgfscope}%
\begin{pgfscope}%
\pgfsys@transformshift{1.255272in}{1.608237in}%
\pgfsys@useobject{currentmarker}{}%
\end{pgfscope}%
\begin{pgfscope}%
\pgfsys@transformshift{1.273894in}{1.632921in}%
\pgfsys@useobject{currentmarker}{}%
\end{pgfscope}%
\begin{pgfscope}%
\pgfsys@transformshift{1.292516in}{1.657480in}%
\pgfsys@useobject{currentmarker}{}%
\end{pgfscope}%
\begin{pgfscope}%
\pgfsys@transformshift{1.311139in}{1.681916in}%
\pgfsys@useobject{currentmarker}{}%
\end{pgfscope}%
\begin{pgfscope}%
\pgfsys@transformshift{1.329761in}{1.706231in}%
\pgfsys@useobject{currentmarker}{}%
\end{pgfscope}%
\begin{pgfscope}%
\pgfsys@transformshift{1.348383in}{1.730430in}%
\pgfsys@useobject{currentmarker}{}%
\end{pgfscope}%
\begin{pgfscope}%
\pgfsys@transformshift{1.367006in}{1.754515in}%
\pgfsys@useobject{currentmarker}{}%
\end{pgfscope}%
\begin{pgfscope}%
\pgfsys@transformshift{1.385628in}{1.778491in}%
\pgfsys@useobject{currentmarker}{}%
\end{pgfscope}%
\begin{pgfscope}%
\pgfsys@transformshift{1.404250in}{1.802361in}%
\pgfsys@useobject{currentmarker}{}%
\end{pgfscope}%
\begin{pgfscope}%
\pgfsys@transformshift{1.422873in}{1.826130in}%
\pgfsys@useobject{currentmarker}{}%
\end{pgfscope}%
\begin{pgfscope}%
\pgfsys@transformshift{1.441495in}{1.849802in}%
\pgfsys@useobject{currentmarker}{}%
\end{pgfscope}%
\begin{pgfscope}%
\pgfsys@transformshift{1.460117in}{1.873379in}%
\pgfsys@useobject{currentmarker}{}%
\end{pgfscope}%
\begin{pgfscope}%
\pgfsys@transformshift{1.478740in}{1.896866in}%
\pgfsys@useobject{currentmarker}{}%
\end{pgfscope}%
\begin{pgfscope}%
\pgfsys@transformshift{1.497362in}{1.920268in}%
\pgfsys@useobject{currentmarker}{}%
\end{pgfscope}%
\begin{pgfscope}%
\pgfsys@transformshift{1.515984in}{1.943588in}%
\pgfsys@useobject{currentmarker}{}%
\end{pgfscope}%
\begin{pgfscope}%
\pgfsys@transformshift{1.534607in}{1.966830in}%
\pgfsys@useobject{currentmarker}{}%
\end{pgfscope}%
\begin{pgfscope}%
\pgfsys@transformshift{1.553229in}{1.989997in}%
\pgfsys@useobject{currentmarker}{}%
\end{pgfscope}%
\begin{pgfscope}%
\pgfsys@transformshift{1.571852in}{2.013092in}%
\pgfsys@useobject{currentmarker}{}%
\end{pgfscope}%
\begin{pgfscope}%
\pgfsys@transformshift{1.590474in}{2.036119in}%
\pgfsys@useobject{currentmarker}{}%
\end{pgfscope}%
\begin{pgfscope}%
\pgfsys@transformshift{1.609096in}{2.059079in}%
\pgfsys@useobject{currentmarker}{}%
\end{pgfscope}%
\begin{pgfscope}%
\pgfsys@transformshift{1.627719in}{2.081976in}%
\pgfsys@useobject{currentmarker}{}%
\end{pgfscope}%
\begin{pgfscope}%
\pgfsys@transformshift{1.646341in}{2.104813in}%
\pgfsys@useobject{currentmarker}{}%
\end{pgfscope}%
\begin{pgfscope}%
\pgfsys@transformshift{1.664963in}{2.127591in}%
\pgfsys@useobject{currentmarker}{}%
\end{pgfscope}%
\begin{pgfscope}%
\pgfsys@transformshift{1.683586in}{2.150314in}%
\pgfsys@useobject{currentmarker}{}%
\end{pgfscope}%
\begin{pgfscope}%
\pgfsys@transformshift{1.702208in}{2.172982in}%
\pgfsys@useobject{currentmarker}{}%
\end{pgfscope}%
\begin{pgfscope}%
\pgfsys@transformshift{1.720830in}{2.195597in}%
\pgfsys@useobject{currentmarker}{}%
\end{pgfscope}%
\begin{pgfscope}%
\pgfsys@transformshift{1.739453in}{2.218162in}%
\pgfsys@useobject{currentmarker}{}%
\end{pgfscope}%
\begin{pgfscope}%
\pgfsys@transformshift{1.758075in}{2.240677in}%
\pgfsys@useobject{currentmarker}{}%
\end{pgfscope}%
\begin{pgfscope}%
\pgfsys@transformshift{1.776697in}{2.263144in}%
\pgfsys@useobject{currentmarker}{}%
\end{pgfscope}%
\begin{pgfscope}%
\pgfsys@transformshift{1.795320in}{2.285565in}%
\pgfsys@useobject{currentmarker}{}%
\end{pgfscope}%
\begin{pgfscope}%
\pgfsys@transformshift{1.813942in}{2.307942in}%
\pgfsys@useobject{currentmarker}{}%
\end{pgfscope}%
\begin{pgfscope}%
\pgfsys@transformshift{1.832564in}{2.330276in}%
\pgfsys@useobject{currentmarker}{}%
\end{pgfscope}%
\begin{pgfscope}%
\pgfsys@transformshift{1.851187in}{2.352567in}%
\pgfsys@useobject{currentmarker}{}%
\end{pgfscope}%
\begin{pgfscope}%
\pgfsys@transformshift{1.869809in}{2.374817in}%
\pgfsys@useobject{currentmarker}{}%
\end{pgfscope}%
\begin{pgfscope}%
\pgfsys@transformshift{1.888431in}{2.397025in}%
\pgfsys@useobject{currentmarker}{}%
\end{pgfscope}%
\begin{pgfscope}%
\pgfsys@transformshift{1.907054in}{2.419194in}%
\pgfsys@useobject{currentmarker}{}%
\end{pgfscope}%
\begin{pgfscope}%
\pgfsys@transformshift{1.925676in}{2.441322in}%
\pgfsys@useobject{currentmarker}{}%
\end{pgfscope}%
\begin{pgfscope}%
\pgfsys@transformshift{1.944298in}{2.463412in}%
\pgfsys@useobject{currentmarker}{}%
\end{pgfscope}%
\begin{pgfscope}%
\pgfsys@transformshift{1.962921in}{2.485463in}%
\pgfsys@useobject{currentmarker}{}%
\end{pgfscope}%
\begin{pgfscope}%
\pgfsys@transformshift{1.981543in}{2.507477in}%
\pgfsys@useobject{currentmarker}{}%
\end{pgfscope}%
\begin{pgfscope}%
\pgfsys@transformshift{2.000165in}{2.529451in}%
\pgfsys@useobject{currentmarker}{}%
\end{pgfscope}%
\begin{pgfscope}%
\pgfsys@transformshift{2.018788in}{2.551386in}%
\pgfsys@useobject{currentmarker}{}%
\end{pgfscope}%
\begin{pgfscope}%
\pgfsys@transformshift{2.037410in}{2.573281in}%
\pgfsys@useobject{currentmarker}{}%
\end{pgfscope}%
\begin{pgfscope}%
\pgfsys@transformshift{2.056033in}{2.595137in}%
\pgfsys@useobject{currentmarker}{}%
\end{pgfscope}%
\begin{pgfscope}%
\pgfsys@transformshift{2.074655in}{2.616953in}%
\pgfsys@useobject{currentmarker}{}%
\end{pgfscope}%
\begin{pgfscope}%
\pgfsys@transformshift{2.093277in}{2.638730in}%
\pgfsys@useobject{currentmarker}{}%
\end{pgfscope}%
\begin{pgfscope}%
\pgfsys@transformshift{2.111900in}{2.660468in}%
\pgfsys@useobject{currentmarker}{}%
\end{pgfscope}%
\begin{pgfscope}%
\pgfsys@transformshift{2.130522in}{2.682165in}%
\pgfsys@useobject{currentmarker}{}%
\end{pgfscope}%
\begin{pgfscope}%
\pgfsys@transformshift{2.149144in}{2.703823in}%
\pgfsys@useobject{currentmarker}{}%
\end{pgfscope}%
\begin{pgfscope}%
\pgfsys@transformshift{2.167767in}{2.725440in}%
\pgfsys@useobject{currentmarker}{}%
\end{pgfscope}%
\begin{pgfscope}%
\pgfsys@transformshift{2.186389in}{2.747017in}%
\pgfsys@useobject{currentmarker}{}%
\end{pgfscope}%
\begin{pgfscope}%
\pgfsys@transformshift{2.205011in}{2.768555in}%
\pgfsys@useobject{currentmarker}{}%
\end{pgfscope}%
\begin{pgfscope}%
\pgfsys@transformshift{2.223634in}{2.790055in}%
\pgfsys@useobject{currentmarker}{}%
\end{pgfscope}%
\begin{pgfscope}%
\pgfsys@transformshift{2.242256in}{2.811518in}%
\pgfsys@useobject{currentmarker}{}%
\end{pgfscope}%
\begin{pgfscope}%
\pgfsys@transformshift{2.260878in}{2.832944in}%
\pgfsys@useobject{currentmarker}{}%
\end{pgfscope}%
\begin{pgfscope}%
\pgfsys@transformshift{2.279501in}{2.854333in}%
\pgfsys@useobject{currentmarker}{}%
\end{pgfscope}%
\begin{pgfscope}%
\pgfsys@transformshift{2.298123in}{2.875687in}%
\pgfsys@useobject{currentmarker}{}%
\end{pgfscope}%
\begin{pgfscope}%
\pgfsys@transformshift{2.316745in}{2.897004in}%
\pgfsys@useobject{currentmarker}{}%
\end{pgfscope}%
\begin{pgfscope}%
\pgfsys@transformshift{2.335368in}{2.918287in}%
\pgfsys@useobject{currentmarker}{}%
\end{pgfscope}%
\begin{pgfscope}%
\pgfsys@transformshift{2.353990in}{2.939536in}%
\pgfsys@useobject{currentmarker}{}%
\end{pgfscope}%
\begin{pgfscope}%
\pgfsys@transformshift{2.372612in}{2.960751in}%
\pgfsys@useobject{currentmarker}{}%
\end{pgfscope}%
\begin{pgfscope}%
\pgfsys@transformshift{2.391235in}{2.981934in}%
\pgfsys@useobject{currentmarker}{}%
\end{pgfscope}%
\begin{pgfscope}%
\pgfsys@transformshift{2.409857in}{3.003084in}%
\pgfsys@useobject{currentmarker}{}%
\end{pgfscope}%
\begin{pgfscope}%
\pgfsys@transformshift{2.428479in}{3.024200in}%
\pgfsys@useobject{currentmarker}{}%
\end{pgfscope}%
\begin{pgfscope}%
\pgfsys@transformshift{2.447102in}{3.045282in}%
\pgfsys@useobject{currentmarker}{}%
\end{pgfscope}%
\begin{pgfscope}%
\pgfsys@transformshift{2.465724in}{3.066331in}%
\pgfsys@useobject{currentmarker}{}%
\end{pgfscope}%
\begin{pgfscope}%
\pgfsys@transformshift{2.484347in}{3.087345in}%
\pgfsys@useobject{currentmarker}{}%
\end{pgfscope}%
\begin{pgfscope}%
\pgfsys@transformshift{2.502969in}{3.108326in}%
\pgfsys@useobject{currentmarker}{}%
\end{pgfscope}%
\begin{pgfscope}%
\pgfsys@transformshift{2.521591in}{3.129273in}%
\pgfsys@useobject{currentmarker}{}%
\end{pgfscope}%
\begin{pgfscope}%
\pgfsys@transformshift{2.540214in}{3.150186in}%
\pgfsys@useobject{currentmarker}{}%
\end{pgfscope}%
\begin{pgfscope}%
\pgfsys@transformshift{2.558836in}{3.171066in}%
\pgfsys@useobject{currentmarker}{}%
\end{pgfscope}%
\begin{pgfscope}%
\pgfsys@transformshift{2.577458in}{3.191913in}%
\pgfsys@useobject{currentmarker}{}%
\end{pgfscope}%
\begin{pgfscope}%
\pgfsys@transformshift{2.596081in}{3.212728in}%
\pgfsys@useobject{currentmarker}{}%
\end{pgfscope}%
\begin{pgfscope}%
\pgfsys@transformshift{2.614703in}{3.233515in}%
\pgfsys@useobject{currentmarker}{}%
\end{pgfscope}%
\begin{pgfscope}%
\pgfsys@transformshift{2.633325in}{3.254274in}%
\pgfsys@useobject{currentmarker}{}%
\end{pgfscope}%
\begin{pgfscope}%
\pgfsys@transformshift{2.651948in}{3.275011in}%
\pgfsys@useobject{currentmarker}{}%
\end{pgfscope}%
\begin{pgfscope}%
\pgfsys@transformshift{2.670570in}{3.295728in}%
\pgfsys@useobject{currentmarker}{}%
\end{pgfscope}%
\begin{pgfscope}%
\pgfsys@transformshift{2.689192in}{3.316429in}%
\pgfsys@useobject{currentmarker}{}%
\end{pgfscope}%
\begin{pgfscope}%
\pgfsys@transformshift{2.707815in}{3.337120in}%
\pgfsys@useobject{currentmarker}{}%
\end{pgfscope}%
\begin{pgfscope}%
\pgfsys@transformshift{2.726437in}{3.357803in}%
\pgfsys@useobject{currentmarker}{}%
\end{pgfscope}%
\begin{pgfscope}%
\pgfsys@transformshift{2.745059in}{3.378489in}%
\pgfsys@useobject{currentmarker}{}%
\end{pgfscope}%
\begin{pgfscope}%
\pgfsys@transformshift{2.763682in}{3.399183in}%
\pgfsys@useobject{currentmarker}{}%
\end{pgfscope}%
\begin{pgfscope}%
\pgfsys@transformshift{2.782304in}{3.419890in}%
\pgfsys@useobject{currentmarker}{}%
\end{pgfscope}%
\begin{pgfscope}%
\pgfsys@transformshift{2.800926in}{3.440615in}%
\pgfsys@useobject{currentmarker}{}%
\end{pgfscope}%
\begin{pgfscope}%
\pgfsys@transformshift{2.819549in}{3.461363in}%
\pgfsys@useobject{currentmarker}{}%
\end{pgfscope}%
\begin{pgfscope}%
\pgfsys@transformshift{2.838171in}{3.482139in}%
\pgfsys@useobject{currentmarker}{}%
\end{pgfscope}%
\begin{pgfscope}%
\pgfsys@transformshift{2.856793in}{3.502948in}%
\pgfsys@useobject{currentmarker}{}%
\end{pgfscope}%
\begin{pgfscope}%
\pgfsys@transformshift{2.875416in}{3.523796in}%
\pgfsys@useobject{currentmarker}{}%
\end{pgfscope}%
\begin{pgfscope}%
\pgfsys@transformshift{2.894038in}{3.544687in}%
\pgfsys@useobject{currentmarker}{}%
\end{pgfscope}%
\begin{pgfscope}%
\pgfsys@transformshift{2.912660in}{3.565626in}%
\pgfsys@useobject{currentmarker}{}%
\end{pgfscope}%
\end{pgfscope}%
\begin{pgfscope}%
\pgfpathrectangle{\pgfqpoint{0.578349in}{0.682899in}}{\pgfqpoint{4.650000in}{3.020000in}}%
\pgfusepath{clip}%
\pgfsetrectcap%
\pgfsetroundjoin%
\pgfsetlinewidth{1.505625pt}%
\definecolor{currentstroke}{rgb}{1.000000,0.000000,0.000000}%
\pgfsetstrokecolor{currentstroke}%
\pgfsetdash{}{0pt}%
\pgfpathmoveto{\pgfqpoint{0.845580in}{1.035269in}}%
\pgfpathlineto{\pgfqpoint{0.864202in}{1.062796in}}%
\pgfpathlineto{\pgfqpoint{0.882825in}{1.090129in}}%
\pgfpathlineto{\pgfqpoint{0.901447in}{1.117279in}}%
\pgfpathlineto{\pgfqpoint{0.920069in}{1.144254in}}%
\pgfpathlineto{\pgfqpoint{0.938692in}{1.171062in}}%
\pgfpathlineto{\pgfqpoint{0.957314in}{1.197710in}}%
\pgfpathlineto{\pgfqpoint{0.975936in}{1.224203in}}%
\pgfpathlineto{\pgfqpoint{0.994559in}{1.250549in}}%
\pgfpathlineto{\pgfqpoint{1.013181in}{1.276751in}}%
\pgfpathlineto{\pgfqpoint{1.031803in}{1.302815in}}%
\pgfpathlineto{\pgfqpoint{1.050426in}{1.328745in}}%
\pgfpathlineto{\pgfqpoint{1.069048in}{1.354545in}}%
\pgfpathlineto{\pgfqpoint{1.087670in}{1.380219in}}%
\pgfpathlineto{\pgfqpoint{1.106293in}{1.405770in}}%
\pgfpathlineto{\pgfqpoint{1.124915in}{1.431203in}}%
\pgfpathlineto{\pgfqpoint{1.143538in}{1.456520in}}%
\pgfpathlineto{\pgfqpoint{1.162160in}{1.481724in}}%
\pgfpathlineto{\pgfqpoint{1.180782in}{1.506818in}}%
\pgfpathlineto{\pgfqpoint{1.199405in}{1.531805in}}%
\pgfpathlineto{\pgfqpoint{1.218027in}{1.556688in}}%
\pgfpathlineto{\pgfqpoint{1.236649in}{1.581470in}}%
\pgfpathlineto{\pgfqpoint{1.255272in}{1.606152in}}%
\pgfpathlineto{\pgfqpoint{1.273894in}{1.630736in}}%
\pgfpathlineto{\pgfqpoint{1.292516in}{1.655227in}}%
\pgfpathlineto{\pgfqpoint{1.311139in}{1.679624in}}%
\pgfpathlineto{\pgfqpoint{1.329761in}{1.703931in}}%
\pgfpathlineto{\pgfqpoint{1.348383in}{1.728149in}}%
\pgfpathlineto{\pgfqpoint{1.367006in}{1.752281in}}%
\pgfpathlineto{\pgfqpoint{1.385628in}{1.776328in}}%
\pgfpathlineto{\pgfqpoint{1.404250in}{1.800292in}}%
\pgfpathlineto{\pgfqpoint{1.422873in}{1.824174in}}%
\pgfpathlineto{\pgfqpoint{1.441495in}{1.847977in}}%
\pgfpathlineto{\pgfqpoint{1.460117in}{1.871702in}}%
\pgfpathlineto{\pgfqpoint{1.478740in}{1.895350in}}%
\pgfpathlineto{\pgfqpoint{1.497362in}{1.918923in}}%
\pgfpathlineto{\pgfqpoint{1.515984in}{1.942423in}}%
\pgfpathlineto{\pgfqpoint{1.534607in}{1.965850in}}%
\pgfpathlineto{\pgfqpoint{1.553229in}{1.989207in}}%
\pgfpathlineto{\pgfqpoint{1.571852in}{2.012494in}}%
\pgfpathlineto{\pgfqpoint{1.590474in}{2.035712in}}%
\pgfpathlineto{\pgfqpoint{1.609096in}{2.058864in}}%
\pgfpathlineto{\pgfqpoint{1.627719in}{2.081950in}}%
\pgfpathlineto{\pgfqpoint{1.646341in}{2.104971in}}%
\pgfpathlineto{\pgfqpoint{1.664963in}{2.127929in}}%
\pgfpathlineto{\pgfqpoint{1.683586in}{2.150825in}}%
\pgfpathlineto{\pgfqpoint{1.702208in}{2.173658in}}%
\pgfpathlineto{\pgfqpoint{1.720830in}{2.196432in}}%
\pgfpathlineto{\pgfqpoint{1.739453in}{2.219146in}}%
\pgfpathlineto{\pgfqpoint{1.758075in}{2.241802in}}%
\pgfpathlineto{\pgfqpoint{1.776697in}{2.264400in}}%
\pgfpathlineto{\pgfqpoint{1.795320in}{2.286942in}}%
\pgfpathlineto{\pgfqpoint{1.813942in}{2.309428in}}%
\pgfpathlineto{\pgfqpoint{1.832564in}{2.331859in}}%
\pgfpathlineto{\pgfqpoint{1.851187in}{2.354237in}}%
\pgfpathlineto{\pgfqpoint{1.869809in}{2.376561in}}%
\pgfpathlineto{\pgfqpoint{1.888431in}{2.398833in}}%
\pgfpathlineto{\pgfqpoint{1.907054in}{2.421053in}}%
\pgfpathlineto{\pgfqpoint{1.925676in}{2.443223in}}%
\pgfpathlineto{\pgfqpoint{1.944298in}{2.465342in}}%
\pgfpathlineto{\pgfqpoint{1.962921in}{2.487412in}}%
\pgfpathlineto{\pgfqpoint{1.981543in}{2.509434in}}%
\pgfpathlineto{\pgfqpoint{2.000165in}{2.531407in}}%
\pgfpathlineto{\pgfqpoint{2.018788in}{2.553334in}}%
\pgfpathlineto{\pgfqpoint{2.037410in}{2.575213in}}%
\pgfpathlineto{\pgfqpoint{2.056033in}{2.597047in}}%
\pgfpathlineto{\pgfqpoint{2.074655in}{2.618835in}}%
\pgfpathlineto{\pgfqpoint{2.093277in}{2.640578in}}%
\pgfpathlineto{\pgfqpoint{2.111900in}{2.662277in}}%
\pgfpathlineto{\pgfqpoint{2.130522in}{2.683932in}}%
\pgfpathlineto{\pgfqpoint{2.149144in}{2.705545in}}%
\pgfpathlineto{\pgfqpoint{2.167767in}{2.727114in}}%
\pgfpathlineto{\pgfqpoint{2.186389in}{2.748642in}}%
\pgfpathlineto{\pgfqpoint{2.205011in}{2.770128in}}%
\pgfpathlineto{\pgfqpoint{2.223634in}{2.791573in}}%
\pgfpathlineto{\pgfqpoint{2.242256in}{2.812978in}}%
\pgfpathlineto{\pgfqpoint{2.260878in}{2.834342in}}%
\pgfpathlineto{\pgfqpoint{2.279501in}{2.855667in}}%
\pgfpathlineto{\pgfqpoint{2.298123in}{2.876953in}}%
\pgfpathlineto{\pgfqpoint{2.316745in}{2.898200in}}%
\pgfpathlineto{\pgfqpoint{2.335368in}{2.919410in}}%
\pgfpathlineto{\pgfqpoint{2.353990in}{2.940581in}}%
\pgfpathlineto{\pgfqpoint{2.372612in}{2.961715in}}%
\pgfpathlineto{\pgfqpoint{2.391235in}{2.982812in}}%
\pgfpathlineto{\pgfqpoint{2.409857in}{3.003872in}}%
\pgfpathlineto{\pgfqpoint{2.428479in}{3.024897in}}%
\pgfpathlineto{\pgfqpoint{2.447102in}{3.045885in}}%
\pgfpathlineto{\pgfqpoint{2.465724in}{3.066838in}}%
\pgfpathlineto{\pgfqpoint{2.484347in}{3.087757in}}%
\pgfpathlineto{\pgfqpoint{2.502969in}{3.108640in}}%
\pgfpathlineto{\pgfqpoint{2.521591in}{3.129490in}}%
\pgfpathlineto{\pgfqpoint{2.540214in}{3.150305in}}%
\pgfpathlineto{\pgfqpoint{2.558836in}{3.171087in}}%
\pgfpathlineto{\pgfqpoint{2.577458in}{3.191835in}}%
\pgfpathlineto{\pgfqpoint{2.596081in}{3.212551in}}%
\pgfpathlineto{\pgfqpoint{2.614703in}{3.233234in}}%
\pgfpathlineto{\pgfqpoint{2.633325in}{3.253885in}}%
\pgfpathlineto{\pgfqpoint{2.651948in}{3.274504in}}%
\pgfpathlineto{\pgfqpoint{2.670570in}{3.295091in}}%
\pgfpathlineto{\pgfqpoint{2.689192in}{3.315647in}}%
\pgfpathlineto{\pgfqpoint{2.707815in}{3.336171in}}%
\pgfpathlineto{\pgfqpoint{2.726437in}{3.356665in}}%
\pgfpathlineto{\pgfqpoint{2.745059in}{3.377128in}}%
\pgfpathlineto{\pgfqpoint{2.763682in}{3.397561in}}%
\pgfpathlineto{\pgfqpoint{2.782304in}{3.417965in}}%
\pgfpathlineto{\pgfqpoint{2.800926in}{3.438338in}}%
\pgfpathlineto{\pgfqpoint{2.819549in}{3.458682in}}%
\pgfpathlineto{\pgfqpoint{2.838171in}{3.478997in}}%
\pgfpathlineto{\pgfqpoint{2.856793in}{3.499282in}}%
\pgfpathlineto{\pgfqpoint{2.875416in}{3.519539in}}%
\pgfpathlineto{\pgfqpoint{2.894038in}{3.539768in}}%
\pgfpathlineto{\pgfqpoint{2.912660in}{3.559944in}}%
\pgfusepath{stroke}%
\end{pgfscope}%
\begin{pgfscope}%
\pgfpathrectangle{\pgfqpoint{0.578349in}{0.682899in}}{\pgfqpoint{4.650000in}{3.020000in}}%
\pgfusepath{clip}%
\pgfsetbuttcap%
\pgfsetroundjoin%
\definecolor{currentfill}{rgb}{0.000000,0.000000,0.000000}%
\pgfsetfillcolor{currentfill}%
\pgfsetfillopacity{0.500000}%
\pgfsetlinewidth{1.003750pt}%
\definecolor{currentstroke}{rgb}{0.000000,0.000000,0.000000}%
\pgfsetstrokecolor{currentstroke}%
\pgfsetstrokeopacity{0.500000}%
\pgfsetdash{}{0pt}%
\pgfsys@defobject{currentmarker}{\pgfqpoint{-0.041667in}{-0.041667in}}{\pgfqpoint{0.041667in}{0.041667in}}{%
\pgfpathmoveto{\pgfqpoint{0.000000in}{-0.041667in}}%
\pgfpathcurveto{\pgfqpoint{0.011050in}{-0.041667in}}{\pgfqpoint{0.021649in}{-0.037276in}}{\pgfqpoint{0.029463in}{-0.029463in}}%
\pgfpathcurveto{\pgfqpoint{0.037276in}{-0.021649in}}{\pgfqpoint{0.041667in}{-0.011050in}}{\pgfqpoint{0.041667in}{0.000000in}}%
\pgfpathcurveto{\pgfqpoint{0.041667in}{0.011050in}}{\pgfqpoint{0.037276in}{0.021649in}}{\pgfqpoint{0.029463in}{0.029463in}}%
\pgfpathcurveto{\pgfqpoint{0.021649in}{0.037276in}}{\pgfqpoint{0.011050in}{0.041667in}}{\pgfqpoint{0.000000in}{0.041667in}}%
\pgfpathcurveto{\pgfqpoint{-0.011050in}{0.041667in}}{\pgfqpoint{-0.021649in}{0.037276in}}{\pgfqpoint{-0.029463in}{0.029463in}}%
\pgfpathcurveto{\pgfqpoint{-0.037276in}{0.021649in}}{\pgfqpoint{-0.041667in}{0.011050in}}{\pgfqpoint{-0.041667in}{0.000000in}}%
\pgfpathcurveto{\pgfqpoint{-0.041667in}{-0.011050in}}{\pgfqpoint{-0.037276in}{-0.021649in}}{\pgfqpoint{-0.029463in}{-0.029463in}}%
\pgfpathcurveto{\pgfqpoint{-0.021649in}{-0.037276in}}{\pgfqpoint{-0.011050in}{-0.041667in}}{\pgfqpoint{0.000000in}{-0.041667in}}%
\pgfpathclose%
\pgfusepath{stroke,fill}%
}%
\begin{pgfscope}%
\pgfsys@transformshift{0.789713in}{0.891483in}%
\pgfsys@useobject{currentmarker}{}%
\end{pgfscope}%
\begin{pgfscope}%
\pgfsys@transformshift{0.808335in}{0.917577in}%
\pgfsys@useobject{currentmarker}{}%
\end{pgfscope}%
\begin{pgfscope}%
\pgfsys@transformshift{0.826958in}{0.943431in}%
\pgfsys@useobject{currentmarker}{}%
\end{pgfscope}%
\begin{pgfscope}%
\pgfsys@transformshift{0.845580in}{0.969049in}%
\pgfsys@useobject{currentmarker}{}%
\end{pgfscope}%
\begin{pgfscope}%
\pgfsys@transformshift{0.864202in}{0.994435in}%
\pgfsys@useobject{currentmarker}{}%
\end{pgfscope}%
\begin{pgfscope}%
\pgfsys@transformshift{0.882825in}{1.019595in}%
\pgfsys@useobject{currentmarker}{}%
\end{pgfscope}%
\begin{pgfscope}%
\pgfsys@transformshift{0.901447in}{1.044532in}%
\pgfsys@useobject{currentmarker}{}%
\end{pgfscope}%
\begin{pgfscope}%
\pgfsys@transformshift{0.920069in}{1.069252in}%
\pgfsys@useobject{currentmarker}{}%
\end{pgfscope}%
\begin{pgfscope}%
\pgfsys@transformshift{0.938692in}{1.093758in}%
\pgfsys@useobject{currentmarker}{}%
\end{pgfscope}%
\begin{pgfscope}%
\pgfsys@transformshift{0.957314in}{1.118056in}%
\pgfsys@useobject{currentmarker}{}%
\end{pgfscope}%
\begin{pgfscope}%
\pgfsys@transformshift{0.975936in}{1.142150in}%
\pgfsys@useobject{currentmarker}{}%
\end{pgfscope}%
\begin{pgfscope}%
\pgfsys@transformshift{0.994559in}{1.166044in}%
\pgfsys@useobject{currentmarker}{}%
\end{pgfscope}%
\begin{pgfscope}%
\pgfsys@transformshift{1.013181in}{1.189744in}%
\pgfsys@useobject{currentmarker}{}%
\end{pgfscope}%
\begin{pgfscope}%
\pgfsys@transformshift{1.031803in}{1.213255in}%
\pgfsys@useobject{currentmarker}{}%
\end{pgfscope}%
\begin{pgfscope}%
\pgfsys@transformshift{1.050426in}{1.236582in}%
\pgfsys@useobject{currentmarker}{}%
\end{pgfscope}%
\begin{pgfscope}%
\pgfsys@transformshift{1.069048in}{1.259729in}%
\pgfsys@useobject{currentmarker}{}%
\end{pgfscope}%
\begin{pgfscope}%
\pgfsys@transformshift{1.087670in}{1.282700in}%
\pgfsys@useobject{currentmarker}{}%
\end{pgfscope}%
\begin{pgfscope}%
\pgfsys@transformshift{1.106293in}{1.305499in}%
\pgfsys@useobject{currentmarker}{}%
\end{pgfscope}%
\begin{pgfscope}%
\pgfsys@transformshift{1.124915in}{1.328131in}%
\pgfsys@useobject{currentmarker}{}%
\end{pgfscope}%
\begin{pgfscope}%
\pgfsys@transformshift{1.143538in}{1.350597in}%
\pgfsys@useobject{currentmarker}{}%
\end{pgfscope}%
\begin{pgfscope}%
\pgfsys@transformshift{1.162160in}{1.372902in}%
\pgfsys@useobject{currentmarker}{}%
\end{pgfscope}%
\begin{pgfscope}%
\pgfsys@transformshift{1.180782in}{1.395049in}%
\pgfsys@useobject{currentmarker}{}%
\end{pgfscope}%
\begin{pgfscope}%
\pgfsys@transformshift{1.199405in}{1.417041in}%
\pgfsys@useobject{currentmarker}{}%
\end{pgfscope}%
\begin{pgfscope}%
\pgfsys@transformshift{1.218027in}{1.438881in}%
\pgfsys@useobject{currentmarker}{}%
\end{pgfscope}%
\begin{pgfscope}%
\pgfsys@transformshift{1.236649in}{1.460572in}%
\pgfsys@useobject{currentmarker}{}%
\end{pgfscope}%
\begin{pgfscope}%
\pgfsys@transformshift{1.255272in}{1.482116in}%
\pgfsys@useobject{currentmarker}{}%
\end{pgfscope}%
\begin{pgfscope}%
\pgfsys@transformshift{1.273894in}{1.503519in}%
\pgfsys@useobject{currentmarker}{}%
\end{pgfscope}%
\begin{pgfscope}%
\pgfsys@transformshift{1.292516in}{1.524782in}%
\pgfsys@useobject{currentmarker}{}%
\end{pgfscope}%
\begin{pgfscope}%
\pgfsys@transformshift{1.311139in}{1.545909in}%
\pgfsys@useobject{currentmarker}{}%
\end{pgfscope}%
\begin{pgfscope}%
\pgfsys@transformshift{1.329761in}{1.566903in}%
\pgfsys@useobject{currentmarker}{}%
\end{pgfscope}%
\begin{pgfscope}%
\pgfsys@transformshift{1.348383in}{1.587769in}%
\pgfsys@useobject{currentmarker}{}%
\end{pgfscope}%
\begin{pgfscope}%
\pgfsys@transformshift{1.367006in}{1.608510in}%
\pgfsys@useobject{currentmarker}{}%
\end{pgfscope}%
\begin{pgfscope}%
\pgfsys@transformshift{1.385628in}{1.629130in}%
\pgfsys@useobject{currentmarker}{}%
\end{pgfscope}%
\begin{pgfscope}%
\pgfsys@transformshift{1.404250in}{1.649632in}%
\pgfsys@useobject{currentmarker}{}%
\end{pgfscope}%
\begin{pgfscope}%
\pgfsys@transformshift{1.422873in}{1.670019in}%
\pgfsys@useobject{currentmarker}{}%
\end{pgfscope}%
\begin{pgfscope}%
\pgfsys@transformshift{1.441495in}{1.690296in}%
\pgfsys@useobject{currentmarker}{}%
\end{pgfscope}%
\begin{pgfscope}%
\pgfsys@transformshift{1.460117in}{1.710464in}%
\pgfsys@useobject{currentmarker}{}%
\end{pgfscope}%
\begin{pgfscope}%
\pgfsys@transformshift{1.478740in}{1.730529in}%
\pgfsys@useobject{currentmarker}{}%
\end{pgfscope}%
\begin{pgfscope}%
\pgfsys@transformshift{1.497362in}{1.750492in}%
\pgfsys@useobject{currentmarker}{}%
\end{pgfscope}%
\begin{pgfscope}%
\pgfsys@transformshift{1.515984in}{1.770356in}%
\pgfsys@useobject{currentmarker}{}%
\end{pgfscope}%
\begin{pgfscope}%
\pgfsys@transformshift{1.534607in}{1.790124in}%
\pgfsys@useobject{currentmarker}{}%
\end{pgfscope}%
\begin{pgfscope}%
\pgfsys@transformshift{1.553229in}{1.809798in}%
\pgfsys@useobject{currentmarker}{}%
\end{pgfscope}%
\begin{pgfscope}%
\pgfsys@transformshift{1.571852in}{1.829379in}%
\pgfsys@useobject{currentmarker}{}%
\end{pgfscope}%
\begin{pgfscope}%
\pgfsys@transformshift{1.590474in}{1.848870in}%
\pgfsys@useobject{currentmarker}{}%
\end{pgfscope}%
\begin{pgfscope}%
\pgfsys@transformshift{1.609096in}{1.868272in}%
\pgfsys@useobject{currentmarker}{}%
\end{pgfscope}%
\begin{pgfscope}%
\pgfsys@transformshift{1.627719in}{1.887587in}%
\pgfsys@useobject{currentmarker}{}%
\end{pgfscope}%
\begin{pgfscope}%
\pgfsys@transformshift{1.646341in}{1.906815in}%
\pgfsys@useobject{currentmarker}{}%
\end{pgfscope}%
\begin{pgfscope}%
\pgfsys@transformshift{1.664963in}{1.925959in}%
\pgfsys@useobject{currentmarker}{}%
\end{pgfscope}%
\begin{pgfscope}%
\pgfsys@transformshift{1.683586in}{1.945020in}%
\pgfsys@useobject{currentmarker}{}%
\end{pgfscope}%
\begin{pgfscope}%
\pgfsys@transformshift{1.702208in}{1.963999in}%
\pgfsys@useobject{currentmarker}{}%
\end{pgfscope}%
\begin{pgfscope}%
\pgfsys@transformshift{1.720830in}{1.982897in}%
\pgfsys@useobject{currentmarker}{}%
\end{pgfscope}%
\begin{pgfscope}%
\pgfsys@transformshift{1.739453in}{2.001717in}%
\pgfsys@useobject{currentmarker}{}%
\end{pgfscope}%
\begin{pgfscope}%
\pgfsys@transformshift{1.758075in}{2.020458in}%
\pgfsys@useobject{currentmarker}{}%
\end{pgfscope}%
\begin{pgfscope}%
\pgfsys@transformshift{1.776697in}{2.039123in}%
\pgfsys@useobject{currentmarker}{}%
\end{pgfscope}%
\begin{pgfscope}%
\pgfsys@transformshift{1.795320in}{2.057713in}%
\pgfsys@useobject{currentmarker}{}%
\end{pgfscope}%
\begin{pgfscope}%
\pgfsys@transformshift{1.813942in}{2.076231in}%
\pgfsys@useobject{currentmarker}{}%
\end{pgfscope}%
\begin{pgfscope}%
\pgfsys@transformshift{1.832564in}{2.094677in}%
\pgfsys@useobject{currentmarker}{}%
\end{pgfscope}%
\begin{pgfscope}%
\pgfsys@transformshift{1.851187in}{2.113053in}%
\pgfsys@useobject{currentmarker}{}%
\end{pgfscope}%
\begin{pgfscope}%
\pgfsys@transformshift{1.869809in}{2.131362in}%
\pgfsys@useobject{currentmarker}{}%
\end{pgfscope}%
\begin{pgfscope}%
\pgfsys@transformshift{1.888431in}{2.149605in}%
\pgfsys@useobject{currentmarker}{}%
\end{pgfscope}%
\begin{pgfscope}%
\pgfsys@transformshift{1.907054in}{2.167784in}%
\pgfsys@useobject{currentmarker}{}%
\end{pgfscope}%
\begin{pgfscope}%
\pgfsys@transformshift{1.925676in}{2.185901in}%
\pgfsys@useobject{currentmarker}{}%
\end{pgfscope}%
\begin{pgfscope}%
\pgfsys@transformshift{1.944298in}{2.203958in}%
\pgfsys@useobject{currentmarker}{}%
\end{pgfscope}%
\begin{pgfscope}%
\pgfsys@transformshift{1.962921in}{2.221956in}%
\pgfsys@useobject{currentmarker}{}%
\end{pgfscope}%
\begin{pgfscope}%
\pgfsys@transformshift{1.981543in}{2.239897in}%
\pgfsys@useobject{currentmarker}{}%
\end{pgfscope}%
\begin{pgfscope}%
\pgfsys@transformshift{2.000165in}{2.257782in}%
\pgfsys@useobject{currentmarker}{}%
\end{pgfscope}%
\begin{pgfscope}%
\pgfsys@transformshift{2.018788in}{2.275614in}%
\pgfsys@useobject{currentmarker}{}%
\end{pgfscope}%
\begin{pgfscope}%
\pgfsys@transformshift{2.037410in}{2.293394in}%
\pgfsys@useobject{currentmarker}{}%
\end{pgfscope}%
\begin{pgfscope}%
\pgfsys@transformshift{2.056033in}{2.311122in}%
\pgfsys@useobject{currentmarker}{}%
\end{pgfscope}%
\begin{pgfscope}%
\pgfsys@transformshift{2.074655in}{2.328800in}%
\pgfsys@useobject{currentmarker}{}%
\end{pgfscope}%
\begin{pgfscope}%
\pgfsys@transformshift{2.093277in}{2.346430in}%
\pgfsys@useobject{currentmarker}{}%
\end{pgfscope}%
\begin{pgfscope}%
\pgfsys@transformshift{2.111900in}{2.364012in}%
\pgfsys@useobject{currentmarker}{}%
\end{pgfscope}%
\begin{pgfscope}%
\pgfsys@transformshift{2.130522in}{2.381547in}%
\pgfsys@useobject{currentmarker}{}%
\end{pgfscope}%
\begin{pgfscope}%
\pgfsys@transformshift{2.149144in}{2.399036in}%
\pgfsys@useobject{currentmarker}{}%
\end{pgfscope}%
\begin{pgfscope}%
\pgfsys@transformshift{2.167767in}{2.416480in}%
\pgfsys@useobject{currentmarker}{}%
\end{pgfscope}%
\begin{pgfscope}%
\pgfsys@transformshift{2.186389in}{2.433880in}%
\pgfsys@useobject{currentmarker}{}%
\end{pgfscope}%
\begin{pgfscope}%
\pgfsys@transformshift{2.205011in}{2.451236in}%
\pgfsys@useobject{currentmarker}{}%
\end{pgfscope}%
\begin{pgfscope}%
\pgfsys@transformshift{2.223634in}{2.468550in}%
\pgfsys@useobject{currentmarker}{}%
\end{pgfscope}%
\begin{pgfscope}%
\pgfsys@transformshift{2.242256in}{2.485822in}%
\pgfsys@useobject{currentmarker}{}%
\end{pgfscope}%
\begin{pgfscope}%
\pgfsys@transformshift{2.260878in}{2.503053in}%
\pgfsys@useobject{currentmarker}{}%
\end{pgfscope}%
\begin{pgfscope}%
\pgfsys@transformshift{2.279501in}{2.520243in}%
\pgfsys@useobject{currentmarker}{}%
\end{pgfscope}%
\begin{pgfscope}%
\pgfsys@transformshift{2.298123in}{2.537393in}%
\pgfsys@useobject{currentmarker}{}%
\end{pgfscope}%
\begin{pgfscope}%
\pgfsys@transformshift{2.316745in}{2.554505in}%
\pgfsys@useobject{currentmarker}{}%
\end{pgfscope}%
\begin{pgfscope}%
\pgfsys@transformshift{2.335368in}{2.571578in}%
\pgfsys@useobject{currentmarker}{}%
\end{pgfscope}%
\begin{pgfscope}%
\pgfsys@transformshift{2.353990in}{2.588614in}%
\pgfsys@useobject{currentmarker}{}%
\end{pgfscope}%
\begin{pgfscope}%
\pgfsys@transformshift{2.372612in}{2.605613in}%
\pgfsys@useobject{currentmarker}{}%
\end{pgfscope}%
\begin{pgfscope}%
\pgfsys@transformshift{2.391235in}{2.622575in}%
\pgfsys@useobject{currentmarker}{}%
\end{pgfscope}%
\begin{pgfscope}%
\pgfsys@transformshift{2.409857in}{2.639502in}%
\pgfsys@useobject{currentmarker}{}%
\end{pgfscope}%
\begin{pgfscope}%
\pgfsys@transformshift{2.428479in}{2.656393in}%
\pgfsys@useobject{currentmarker}{}%
\end{pgfscope}%
\begin{pgfscope}%
\pgfsys@transformshift{2.447102in}{2.673250in}%
\pgfsys@useobject{currentmarker}{}%
\end{pgfscope}%
\begin{pgfscope}%
\pgfsys@transformshift{2.465724in}{2.690073in}%
\pgfsys@useobject{currentmarker}{}%
\end{pgfscope}%
\begin{pgfscope}%
\pgfsys@transformshift{2.484347in}{2.706862in}%
\pgfsys@useobject{currentmarker}{}%
\end{pgfscope}%
\begin{pgfscope}%
\pgfsys@transformshift{2.502969in}{2.723618in}%
\pgfsys@useobject{currentmarker}{}%
\end{pgfscope}%
\begin{pgfscope}%
\pgfsys@transformshift{2.521591in}{2.740342in}%
\pgfsys@useobject{currentmarker}{}%
\end{pgfscope}%
\begin{pgfscope}%
\pgfsys@transformshift{2.540214in}{2.757033in}%
\pgfsys@useobject{currentmarker}{}%
\end{pgfscope}%
\begin{pgfscope}%
\pgfsys@transformshift{2.558836in}{2.773691in}%
\pgfsys@useobject{currentmarker}{}%
\end{pgfscope}%
\begin{pgfscope}%
\pgfsys@transformshift{2.577458in}{2.790319in}%
\pgfsys@useobject{currentmarker}{}%
\end{pgfscope}%
\begin{pgfscope}%
\pgfsys@transformshift{2.596081in}{2.806914in}%
\pgfsys@useobject{currentmarker}{}%
\end{pgfscope}%
\begin{pgfscope}%
\pgfsys@transformshift{2.614703in}{2.823479in}%
\pgfsys@useobject{currentmarker}{}%
\end{pgfscope}%
\begin{pgfscope}%
\pgfsys@transformshift{2.633325in}{2.840013in}%
\pgfsys@useobject{currentmarker}{}%
\end{pgfscope}%
\begin{pgfscope}%
\pgfsys@transformshift{2.651948in}{2.856517in}%
\pgfsys@useobject{currentmarker}{}%
\end{pgfscope}%
\begin{pgfscope}%
\pgfsys@transformshift{2.670570in}{2.872991in}%
\pgfsys@useobject{currentmarker}{}%
\end{pgfscope}%
\begin{pgfscope}%
\pgfsys@transformshift{2.689192in}{2.889435in}%
\pgfsys@useobject{currentmarker}{}%
\end{pgfscope}%
\begin{pgfscope}%
\pgfsys@transformshift{2.707815in}{2.905850in}%
\pgfsys@useobject{currentmarker}{}%
\end{pgfscope}%
\begin{pgfscope}%
\pgfsys@transformshift{2.726437in}{2.922236in}%
\pgfsys@useobject{currentmarker}{}%
\end{pgfscope}%
\begin{pgfscope}%
\pgfsys@transformshift{2.745059in}{2.938594in}%
\pgfsys@useobject{currentmarker}{}%
\end{pgfscope}%
\begin{pgfscope}%
\pgfsys@transformshift{2.763682in}{2.954923in}%
\pgfsys@useobject{currentmarker}{}%
\end{pgfscope}%
\begin{pgfscope}%
\pgfsys@transformshift{2.782304in}{2.971226in}%
\pgfsys@useobject{currentmarker}{}%
\end{pgfscope}%
\begin{pgfscope}%
\pgfsys@transformshift{2.800926in}{2.987501in}%
\pgfsys@useobject{currentmarker}{}%
\end{pgfscope}%
\begin{pgfscope}%
\pgfsys@transformshift{2.819549in}{3.003750in}%
\pgfsys@useobject{currentmarker}{}%
\end{pgfscope}%
\begin{pgfscope}%
\pgfsys@transformshift{2.838171in}{3.019973in}%
\pgfsys@useobject{currentmarker}{}%
\end{pgfscope}%
\begin{pgfscope}%
\pgfsys@transformshift{2.856793in}{3.036170in}%
\pgfsys@useobject{currentmarker}{}%
\end{pgfscope}%
\begin{pgfscope}%
\pgfsys@transformshift{2.875416in}{3.052342in}%
\pgfsys@useobject{currentmarker}{}%
\end{pgfscope}%
\begin{pgfscope}%
\pgfsys@transformshift{2.894038in}{3.068490in}%
\pgfsys@useobject{currentmarker}{}%
\end{pgfscope}%
\begin{pgfscope}%
\pgfsys@transformshift{2.912660in}{3.084614in}%
\pgfsys@useobject{currentmarker}{}%
\end{pgfscope}%
\begin{pgfscope}%
\pgfsys@transformshift{2.931283in}{3.100715in}%
\pgfsys@useobject{currentmarker}{}%
\end{pgfscope}%
\begin{pgfscope}%
\pgfsys@transformshift{2.949905in}{3.116792in}%
\pgfsys@useobject{currentmarker}{}%
\end{pgfscope}%
\begin{pgfscope}%
\pgfsys@transformshift{2.968528in}{3.132845in}%
\pgfsys@useobject{currentmarker}{}%
\end{pgfscope}%
\begin{pgfscope}%
\pgfsys@transformshift{2.987150in}{3.148877in}%
\pgfsys@useobject{currentmarker}{}%
\end{pgfscope}%
\begin{pgfscope}%
\pgfsys@transformshift{3.005772in}{3.164885in}%
\pgfsys@useobject{currentmarker}{}%
\end{pgfscope}%
\begin{pgfscope}%
\pgfsys@transformshift{3.024395in}{3.180871in}%
\pgfsys@useobject{currentmarker}{}%
\end{pgfscope}%
\begin{pgfscope}%
\pgfsys@transformshift{3.043017in}{3.196835in}%
\pgfsys@useobject{currentmarker}{}%
\end{pgfscope}%
\begin{pgfscope}%
\pgfsys@transformshift{3.061639in}{3.212777in}%
\pgfsys@useobject{currentmarker}{}%
\end{pgfscope}%
\begin{pgfscope}%
\pgfsys@transformshift{3.080262in}{3.228697in}%
\pgfsys@useobject{currentmarker}{}%
\end{pgfscope}%
\begin{pgfscope}%
\pgfsys@transformshift{3.098884in}{3.244594in}%
\pgfsys@useobject{currentmarker}{}%
\end{pgfscope}%
\begin{pgfscope}%
\pgfsys@transformshift{3.117506in}{3.260469in}%
\pgfsys@useobject{currentmarker}{}%
\end{pgfscope}%
\begin{pgfscope}%
\pgfsys@transformshift{3.136129in}{3.276322in}%
\pgfsys@useobject{currentmarker}{}%
\end{pgfscope}%
\begin{pgfscope}%
\pgfsys@transformshift{3.154751in}{3.292153in}%
\pgfsys@useobject{currentmarker}{}%
\end{pgfscope}%
\begin{pgfscope}%
\pgfsys@transformshift{3.173373in}{3.307960in}%
\pgfsys@useobject{currentmarker}{}%
\end{pgfscope}%
\begin{pgfscope}%
\pgfsys@transformshift{3.191996in}{3.323745in}%
\pgfsys@useobject{currentmarker}{}%
\end{pgfscope}%
\begin{pgfscope}%
\pgfsys@transformshift{3.210618in}{3.339507in}%
\pgfsys@useobject{currentmarker}{}%
\end{pgfscope}%
\begin{pgfscope}%
\pgfsys@transformshift{3.229240in}{3.355246in}%
\pgfsys@useobject{currentmarker}{}%
\end{pgfscope}%
\begin{pgfscope}%
\pgfsys@transformshift{3.247863in}{3.370961in}%
\pgfsys@useobject{currentmarker}{}%
\end{pgfscope}%
\begin{pgfscope}%
\pgfsys@transformshift{3.266485in}{3.386653in}%
\pgfsys@useobject{currentmarker}{}%
\end{pgfscope}%
\begin{pgfscope}%
\pgfsys@transformshift{3.285107in}{3.402322in}%
\pgfsys@useobject{currentmarker}{}%
\end{pgfscope}%
\begin{pgfscope}%
\pgfsys@transformshift{3.303730in}{3.417967in}%
\pgfsys@useobject{currentmarker}{}%
\end{pgfscope}%
\begin{pgfscope}%
\pgfsys@transformshift{3.322352in}{3.433588in}%
\pgfsys@useobject{currentmarker}{}%
\end{pgfscope}%
\begin{pgfscope}%
\pgfsys@transformshift{3.340974in}{3.449185in}%
\pgfsys@useobject{currentmarker}{}%
\end{pgfscope}%
\begin{pgfscope}%
\pgfsys@transformshift{3.359597in}{3.464759in}%
\pgfsys@useobject{currentmarker}{}%
\end{pgfscope}%
\end{pgfscope}%
\begin{pgfscope}%
\pgfpathrectangle{\pgfqpoint{0.578349in}{0.682899in}}{\pgfqpoint{4.650000in}{3.020000in}}%
\pgfusepath{clip}%
\pgfsetrectcap%
\pgfsetroundjoin%
\pgfsetlinewidth{1.505625pt}%
\definecolor{currentstroke}{rgb}{1.000000,0.000000,0.000000}%
\pgfsetstrokecolor{currentstroke}%
\pgfsetdash{}{0pt}%
\pgfpathmoveto{\pgfqpoint{0.789713in}{0.891483in}}%
\pgfpathlineto{\pgfqpoint{0.864202in}{0.993981in}}%
\pgfpathlineto{\pgfqpoint{0.938692in}{1.092355in}}%
\pgfpathlineto{\pgfqpoint{1.013181in}{1.187431in}}%
\pgfpathlineto{\pgfqpoint{1.106293in}{1.302352in}}%
\pgfpathlineto{\pgfqpoint{1.199405in}{1.413488in}}%
\pgfpathlineto{\pgfqpoint{1.292516in}{1.521302in}}%
\pgfpathlineto{\pgfqpoint{1.385628in}{1.626156in}}%
\pgfpathlineto{\pgfqpoint{1.497362in}{1.748487in}}%
\pgfpathlineto{\pgfqpoint{1.609096in}{1.867404in}}%
\pgfpathlineto{\pgfqpoint{1.720830in}{1.983256in}}%
\pgfpathlineto{\pgfqpoint{1.851187in}{2.114934in}}%
\pgfpathlineto{\pgfqpoint{1.981543in}{2.243229in}}%
\pgfpathlineto{\pgfqpoint{2.111900in}{2.368470in}}%
\pgfpathlineto{\pgfqpoint{2.260878in}{2.508215in}}%
\pgfpathlineto{\pgfqpoint{2.409857in}{2.644678in}}%
\pgfpathlineto{\pgfqpoint{2.558836in}{2.778154in}}%
\pgfpathlineto{\pgfqpoint{2.707815in}{2.908892in}}%
\pgfpathlineto{\pgfqpoint{2.875416in}{3.052967in}}%
\pgfpathlineto{\pgfqpoint{3.043017in}{3.194111in}}%
\pgfpathlineto{\pgfqpoint{3.210618in}{3.332550in}}%
\pgfpathlineto{\pgfqpoint{3.359597in}{3.453477in}}%
\pgfpathlineto{\pgfqpoint{3.359597in}{3.453477in}}%
\pgfusepath{stroke}%
\end{pgfscope}%
\begin{pgfscope}%
\pgfpathrectangle{\pgfqpoint{0.578349in}{0.682899in}}{\pgfqpoint{4.650000in}{3.020000in}}%
\pgfusepath{clip}%
\pgfsetbuttcap%
\pgfsetroundjoin%
\definecolor{currentfill}{rgb}{0.000000,0.000000,0.000000}%
\pgfsetfillcolor{currentfill}%
\pgfsetfillopacity{0.500000}%
\pgfsetlinewidth{1.003750pt}%
\definecolor{currentstroke}{rgb}{0.000000,0.000000,0.000000}%
\pgfsetstrokecolor{currentstroke}%
\pgfsetstrokeopacity{0.500000}%
\pgfsetdash{}{0pt}%
\pgfsys@defobject{currentmarker}{\pgfqpoint{-0.041667in}{-0.041667in}}{\pgfqpoint{0.041667in}{0.041667in}}{%
\pgfpathmoveto{\pgfqpoint{0.000000in}{-0.041667in}}%
\pgfpathcurveto{\pgfqpoint{0.011050in}{-0.041667in}}{\pgfqpoint{0.021649in}{-0.037276in}}{\pgfqpoint{0.029463in}{-0.029463in}}%
\pgfpathcurveto{\pgfqpoint{0.037276in}{-0.021649in}}{\pgfqpoint{0.041667in}{-0.011050in}}{\pgfqpoint{0.041667in}{0.000000in}}%
\pgfpathcurveto{\pgfqpoint{0.041667in}{0.011050in}}{\pgfqpoint{0.037276in}{0.021649in}}{\pgfqpoint{0.029463in}{0.029463in}}%
\pgfpathcurveto{\pgfqpoint{0.021649in}{0.037276in}}{\pgfqpoint{0.011050in}{0.041667in}}{\pgfqpoint{0.000000in}{0.041667in}}%
\pgfpathcurveto{\pgfqpoint{-0.011050in}{0.041667in}}{\pgfqpoint{-0.021649in}{0.037276in}}{\pgfqpoint{-0.029463in}{0.029463in}}%
\pgfpathcurveto{\pgfqpoint{-0.037276in}{0.021649in}}{\pgfqpoint{-0.041667in}{0.011050in}}{\pgfqpoint{-0.041667in}{0.000000in}}%
\pgfpathcurveto{\pgfqpoint{-0.041667in}{-0.011050in}}{\pgfqpoint{-0.037276in}{-0.021649in}}{\pgfqpoint{-0.029463in}{-0.029463in}}%
\pgfpathcurveto{\pgfqpoint{-0.021649in}{-0.037276in}}{\pgfqpoint{-0.011050in}{-0.041667in}}{\pgfqpoint{0.000000in}{-0.041667in}}%
\pgfpathclose%
\pgfusepath{stroke,fill}%
}%
\begin{pgfscope}%
\pgfsys@transformshift{0.882825in}{1.010164in}%
\pgfsys@useobject{currentmarker}{}%
\end{pgfscope}%
\begin{pgfscope}%
\pgfsys@transformshift{0.901447in}{1.034200in}%
\pgfsys@useobject{currentmarker}{}%
\end{pgfscope}%
\begin{pgfscope}%
\pgfsys@transformshift{0.920069in}{1.057967in}%
\pgfsys@useobject{currentmarker}{}%
\end{pgfscope}%
\begin{pgfscope}%
\pgfsys@transformshift{0.938692in}{1.081464in}%
\pgfsys@useobject{currentmarker}{}%
\end{pgfscope}%
\begin{pgfscope}%
\pgfsys@transformshift{0.957314in}{1.104691in}%
\pgfsys@useobject{currentmarker}{}%
\end{pgfscope}%
\begin{pgfscope}%
\pgfsys@transformshift{0.975936in}{1.127647in}%
\pgfsys@useobject{currentmarker}{}%
\end{pgfscope}%
\begin{pgfscope}%
\pgfsys@transformshift{0.994559in}{1.150332in}%
\pgfsys@useobject{currentmarker}{}%
\end{pgfscope}%
\begin{pgfscope}%
\pgfsys@transformshift{1.013181in}{1.172746in}%
\pgfsys@useobject{currentmarker}{}%
\end{pgfscope}%
\begin{pgfscope}%
\pgfsys@transformshift{1.031803in}{1.194887in}%
\pgfsys@useobject{currentmarker}{}%
\end{pgfscope}%
\begin{pgfscope}%
\pgfsys@transformshift{1.050426in}{1.216757in}%
\pgfsys@useobject{currentmarker}{}%
\end{pgfscope}%
\begin{pgfscope}%
\pgfsys@transformshift{1.069048in}{1.238353in}%
\pgfsys@useobject{currentmarker}{}%
\end{pgfscope}%
\begin{pgfscope}%
\pgfsys@transformshift{1.087670in}{1.259676in}%
\pgfsys@useobject{currentmarker}{}%
\end{pgfscope}%
\begin{pgfscope}%
\pgfsys@transformshift{1.106293in}{1.280726in}%
\pgfsys@useobject{currentmarker}{}%
\end{pgfscope}%
\begin{pgfscope}%
\pgfsys@transformshift{1.124915in}{1.301493in}%
\pgfsys@useobject{currentmarker}{}%
\end{pgfscope}%
\begin{pgfscope}%
\pgfsys@transformshift{1.143538in}{1.321981in}%
\pgfsys@useobject{currentmarker}{}%
\end{pgfscope}%
\begin{pgfscope}%
\pgfsys@transformshift{1.162160in}{1.342189in}%
\pgfsys@useobject{currentmarker}{}%
\end{pgfscope}%
\begin{pgfscope}%
\pgfsys@transformshift{1.180782in}{1.362118in}%
\pgfsys@useobject{currentmarker}{}%
\end{pgfscope}%
\begin{pgfscope}%
\pgfsys@transformshift{1.199405in}{1.381771in}%
\pgfsys@useobject{currentmarker}{}%
\end{pgfscope}%
\begin{pgfscope}%
\pgfsys@transformshift{1.218027in}{1.401151in}%
\pgfsys@useobject{currentmarker}{}%
\end{pgfscope}%
\begin{pgfscope}%
\pgfsys@transformshift{1.236649in}{1.420260in}%
\pgfsys@useobject{currentmarker}{}%
\end{pgfscope}%
\begin{pgfscope}%
\pgfsys@transformshift{1.255272in}{1.439105in}%
\pgfsys@useobject{currentmarker}{}%
\end{pgfscope}%
\begin{pgfscope}%
\pgfsys@transformshift{1.273894in}{1.457690in}%
\pgfsys@useobject{currentmarker}{}%
\end{pgfscope}%
\begin{pgfscope}%
\pgfsys@transformshift{1.292516in}{1.476021in}%
\pgfsys@useobject{currentmarker}{}%
\end{pgfscope}%
\begin{pgfscope}%
\pgfsys@transformshift{1.311139in}{1.494105in}%
\pgfsys@useobject{currentmarker}{}%
\end{pgfscope}%
\begin{pgfscope}%
\pgfsys@transformshift{1.329761in}{1.511949in}%
\pgfsys@useobject{currentmarker}{}%
\end{pgfscope}%
\begin{pgfscope}%
\pgfsys@transformshift{1.348383in}{1.529560in}%
\pgfsys@useobject{currentmarker}{}%
\end{pgfscope}%
\begin{pgfscope}%
\pgfsys@transformshift{1.367006in}{1.546944in}%
\pgfsys@useobject{currentmarker}{}%
\end{pgfscope}%
\begin{pgfscope}%
\pgfsys@transformshift{1.385628in}{1.564108in}%
\pgfsys@useobject{currentmarker}{}%
\end{pgfscope}%
\begin{pgfscope}%
\pgfsys@transformshift{1.404250in}{1.581058in}%
\pgfsys@useobject{currentmarker}{}%
\end{pgfscope}%
\begin{pgfscope}%
\pgfsys@transformshift{1.422873in}{1.597801in}%
\pgfsys@useobject{currentmarker}{}%
\end{pgfscope}%
\begin{pgfscope}%
\pgfsys@transformshift{1.441495in}{1.614339in}%
\pgfsys@useobject{currentmarker}{}%
\end{pgfscope}%
\begin{pgfscope}%
\pgfsys@transformshift{1.460117in}{1.630680in}%
\pgfsys@useobject{currentmarker}{}%
\end{pgfscope}%
\begin{pgfscope}%
\pgfsys@transformshift{1.478740in}{1.646827in}%
\pgfsys@useobject{currentmarker}{}%
\end{pgfscope}%
\begin{pgfscope}%
\pgfsys@transformshift{1.497362in}{1.662786in}%
\pgfsys@useobject{currentmarker}{}%
\end{pgfscope}%
\begin{pgfscope}%
\pgfsys@transformshift{1.515984in}{1.678559in}%
\pgfsys@useobject{currentmarker}{}%
\end{pgfscope}%
\begin{pgfscope}%
\pgfsys@transformshift{1.534607in}{1.694150in}%
\pgfsys@useobject{currentmarker}{}%
\end{pgfscope}%
\begin{pgfscope}%
\pgfsys@transformshift{1.553229in}{1.709564in}%
\pgfsys@useobject{currentmarker}{}%
\end{pgfscope}%
\begin{pgfscope}%
\pgfsys@transformshift{1.571852in}{1.724802in}%
\pgfsys@useobject{currentmarker}{}%
\end{pgfscope}%
\begin{pgfscope}%
\pgfsys@transformshift{1.590474in}{1.739870in}%
\pgfsys@useobject{currentmarker}{}%
\end{pgfscope}%
\begin{pgfscope}%
\pgfsys@transformshift{1.609096in}{1.754768in}%
\pgfsys@useobject{currentmarker}{}%
\end{pgfscope}%
\begin{pgfscope}%
\pgfsys@transformshift{1.627719in}{1.769502in}%
\pgfsys@useobject{currentmarker}{}%
\end{pgfscope}%
\begin{pgfscope}%
\pgfsys@transformshift{1.646341in}{1.784073in}%
\pgfsys@useobject{currentmarker}{}%
\end{pgfscope}%
\begin{pgfscope}%
\pgfsys@transformshift{1.664963in}{1.798486in}%
\pgfsys@useobject{currentmarker}{}%
\end{pgfscope}%
\begin{pgfscope}%
\pgfsys@transformshift{1.683586in}{1.812744in}%
\pgfsys@useobject{currentmarker}{}%
\end{pgfscope}%
\begin{pgfscope}%
\pgfsys@transformshift{1.702208in}{1.826849in}%
\pgfsys@useobject{currentmarker}{}%
\end{pgfscope}%
\begin{pgfscope}%
\pgfsys@transformshift{1.720830in}{1.840804in}%
\pgfsys@useobject{currentmarker}{}%
\end{pgfscope}%
\begin{pgfscope}%
\pgfsys@transformshift{1.739453in}{1.854614in}%
\pgfsys@useobject{currentmarker}{}%
\end{pgfscope}%
\begin{pgfscope}%
\pgfsys@transformshift{1.758075in}{1.868280in}%
\pgfsys@useobject{currentmarker}{}%
\end{pgfscope}%
\begin{pgfscope}%
\pgfsys@transformshift{1.776697in}{1.881806in}%
\pgfsys@useobject{currentmarker}{}%
\end{pgfscope}%
\begin{pgfscope}%
\pgfsys@transformshift{1.795320in}{1.895194in}%
\pgfsys@useobject{currentmarker}{}%
\end{pgfscope}%
\begin{pgfscope}%
\pgfsys@transformshift{1.813942in}{1.908447in}%
\pgfsys@useobject{currentmarker}{}%
\end{pgfscope}%
\begin{pgfscope}%
\pgfsys@transformshift{1.832564in}{1.921569in}%
\pgfsys@useobject{currentmarker}{}%
\end{pgfscope}%
\begin{pgfscope}%
\pgfsys@transformshift{1.851187in}{1.934561in}%
\pgfsys@useobject{currentmarker}{}%
\end{pgfscope}%
\begin{pgfscope}%
\pgfsys@transformshift{1.869809in}{1.947425in}%
\pgfsys@useobject{currentmarker}{}%
\end{pgfscope}%
\begin{pgfscope}%
\pgfsys@transformshift{1.888431in}{1.960165in}%
\pgfsys@useobject{currentmarker}{}%
\end{pgfscope}%
\begin{pgfscope}%
\pgfsys@transformshift{1.907054in}{1.972783in}%
\pgfsys@useobject{currentmarker}{}%
\end{pgfscope}%
\begin{pgfscope}%
\pgfsys@transformshift{1.925676in}{1.985281in}%
\pgfsys@useobject{currentmarker}{}%
\end{pgfscope}%
\begin{pgfscope}%
\pgfsys@transformshift{1.944298in}{1.997661in}%
\pgfsys@useobject{currentmarker}{}%
\end{pgfscope}%
\begin{pgfscope}%
\pgfsys@transformshift{1.962921in}{2.009926in}%
\pgfsys@useobject{currentmarker}{}%
\end{pgfscope}%
\begin{pgfscope}%
\pgfsys@transformshift{1.981543in}{2.022077in}%
\pgfsys@useobject{currentmarker}{}%
\end{pgfscope}%
\begin{pgfscope}%
\pgfsys@transformshift{2.000165in}{2.034117in}%
\pgfsys@useobject{currentmarker}{}%
\end{pgfscope}%
\begin{pgfscope}%
\pgfsys@transformshift{2.018788in}{2.046048in}%
\pgfsys@useobject{currentmarker}{}%
\end{pgfscope}%
\begin{pgfscope}%
\pgfsys@transformshift{2.037410in}{2.057871in}%
\pgfsys@useobject{currentmarker}{}%
\end{pgfscope}%
\begin{pgfscope}%
\pgfsys@transformshift{2.056033in}{2.069589in}%
\pgfsys@useobject{currentmarker}{}%
\end{pgfscope}%
\begin{pgfscope}%
\pgfsys@transformshift{2.074655in}{2.081202in}%
\pgfsys@useobject{currentmarker}{}%
\end{pgfscope}%
\begin{pgfscope}%
\pgfsys@transformshift{2.093277in}{2.092712in}%
\pgfsys@useobject{currentmarker}{}%
\end{pgfscope}%
\begin{pgfscope}%
\pgfsys@transformshift{2.111900in}{2.104120in}%
\pgfsys@useobject{currentmarker}{}%
\end{pgfscope}%
\begin{pgfscope}%
\pgfsys@transformshift{2.130522in}{2.115428in}%
\pgfsys@useobject{currentmarker}{}%
\end{pgfscope}%
\begin{pgfscope}%
\pgfsys@transformshift{2.149144in}{2.126637in}%
\pgfsys@useobject{currentmarker}{}%
\end{pgfscope}%
\begin{pgfscope}%
\pgfsys@transformshift{2.167767in}{2.137748in}%
\pgfsys@useobject{currentmarker}{}%
\end{pgfscope}%
\begin{pgfscope}%
\pgfsys@transformshift{2.186389in}{2.148762in}%
\pgfsys@useobject{currentmarker}{}%
\end{pgfscope}%
\begin{pgfscope}%
\pgfsys@transformshift{2.205011in}{2.159679in}%
\pgfsys@useobject{currentmarker}{}%
\end{pgfscope}%
\begin{pgfscope}%
\pgfsys@transformshift{2.223634in}{2.170500in}%
\pgfsys@useobject{currentmarker}{}%
\end{pgfscope}%
\begin{pgfscope}%
\pgfsys@transformshift{2.242256in}{2.181227in}%
\pgfsys@useobject{currentmarker}{}%
\end{pgfscope}%
\begin{pgfscope}%
\pgfsys@transformshift{2.260878in}{2.191860in}%
\pgfsys@useobject{currentmarker}{}%
\end{pgfscope}%
\begin{pgfscope}%
\pgfsys@transformshift{2.279501in}{2.202400in}%
\pgfsys@useobject{currentmarker}{}%
\end{pgfscope}%
\begin{pgfscope}%
\pgfsys@transformshift{2.298123in}{2.212847in}%
\pgfsys@useobject{currentmarker}{}%
\end{pgfscope}%
\begin{pgfscope}%
\pgfsys@transformshift{2.316745in}{2.223203in}%
\pgfsys@useobject{currentmarker}{}%
\end{pgfscope}%
\begin{pgfscope}%
\pgfsys@transformshift{2.335368in}{2.233468in}%
\pgfsys@useobject{currentmarker}{}%
\end{pgfscope}%
\begin{pgfscope}%
\pgfsys@transformshift{2.353990in}{2.243644in}%
\pgfsys@useobject{currentmarker}{}%
\end{pgfscope}%
\begin{pgfscope}%
\pgfsys@transformshift{2.372612in}{2.253730in}%
\pgfsys@useobject{currentmarker}{}%
\end{pgfscope}%
\begin{pgfscope}%
\pgfsys@transformshift{2.391235in}{2.263729in}%
\pgfsys@useobject{currentmarker}{}%
\end{pgfscope}%
\begin{pgfscope}%
\pgfsys@transformshift{2.409857in}{2.273641in}%
\pgfsys@useobject{currentmarker}{}%
\end{pgfscope}%
\begin{pgfscope}%
\pgfsys@transformshift{2.428479in}{2.283468in}%
\pgfsys@useobject{currentmarker}{}%
\end{pgfscope}%
\begin{pgfscope}%
\pgfsys@transformshift{2.447102in}{2.293211in}%
\pgfsys@useobject{currentmarker}{}%
\end{pgfscope}%
\begin{pgfscope}%
\pgfsys@transformshift{2.465724in}{2.302870in}%
\pgfsys@useobject{currentmarker}{}%
\end{pgfscope}%
\begin{pgfscope}%
\pgfsys@transformshift{2.484347in}{2.312446in}%
\pgfsys@useobject{currentmarker}{}%
\end{pgfscope}%
\begin{pgfscope}%
\pgfsys@transformshift{2.502969in}{2.321942in}%
\pgfsys@useobject{currentmarker}{}%
\end{pgfscope}%
\begin{pgfscope}%
\pgfsys@transformshift{2.521591in}{2.331358in}%
\pgfsys@useobject{currentmarker}{}%
\end{pgfscope}%
\begin{pgfscope}%
\pgfsys@transformshift{2.540214in}{2.340694in}%
\pgfsys@useobject{currentmarker}{}%
\end{pgfscope}%
\begin{pgfscope}%
\pgfsys@transformshift{2.558836in}{2.349952in}%
\pgfsys@useobject{currentmarker}{}%
\end{pgfscope}%
\begin{pgfscope}%
\pgfsys@transformshift{2.577458in}{2.359134in}%
\pgfsys@useobject{currentmarker}{}%
\end{pgfscope}%
\begin{pgfscope}%
\pgfsys@transformshift{2.596081in}{2.368238in}%
\pgfsys@useobject{currentmarker}{}%
\end{pgfscope}%
\begin{pgfscope}%
\pgfsys@transformshift{2.614703in}{2.377268in}%
\pgfsys@useobject{currentmarker}{}%
\end{pgfscope}%
\begin{pgfscope}%
\pgfsys@transformshift{2.633325in}{2.386222in}%
\pgfsys@useobject{currentmarker}{}%
\end{pgfscope}%
\begin{pgfscope}%
\pgfsys@transformshift{2.651948in}{2.395102in}%
\pgfsys@useobject{currentmarker}{}%
\end{pgfscope}%
\begin{pgfscope}%
\pgfsys@transformshift{2.670570in}{2.403909in}%
\pgfsys@useobject{currentmarker}{}%
\end{pgfscope}%
\begin{pgfscope}%
\pgfsys@transformshift{2.689192in}{2.412643in}%
\pgfsys@useobject{currentmarker}{}%
\end{pgfscope}%
\begin{pgfscope}%
\pgfsys@transformshift{2.707815in}{2.421304in}%
\pgfsys@useobject{currentmarker}{}%
\end{pgfscope}%
\begin{pgfscope}%
\pgfsys@transformshift{2.726437in}{2.429894in}%
\pgfsys@useobject{currentmarker}{}%
\end{pgfscope}%
\begin{pgfscope}%
\pgfsys@transformshift{2.745059in}{2.438413in}%
\pgfsys@useobject{currentmarker}{}%
\end{pgfscope}%
\begin{pgfscope}%
\pgfsys@transformshift{2.763682in}{2.446862in}%
\pgfsys@useobject{currentmarker}{}%
\end{pgfscope}%
\begin{pgfscope}%
\pgfsys@transformshift{2.782304in}{2.455240in}%
\pgfsys@useobject{currentmarker}{}%
\end{pgfscope}%
\begin{pgfscope}%
\pgfsys@transformshift{2.800926in}{2.463550in}%
\pgfsys@useobject{currentmarker}{}%
\end{pgfscope}%
\begin{pgfscope}%
\pgfsys@transformshift{2.819549in}{2.471792in}%
\pgfsys@useobject{currentmarker}{}%
\end{pgfscope}%
\begin{pgfscope}%
\pgfsys@transformshift{2.838171in}{2.479966in}%
\pgfsys@useobject{currentmarker}{}%
\end{pgfscope}%
\begin{pgfscope}%
\pgfsys@transformshift{2.856793in}{2.488074in}%
\pgfsys@useobject{currentmarker}{}%
\end{pgfscope}%
\begin{pgfscope}%
\pgfsys@transformshift{2.875416in}{2.496116in}%
\pgfsys@useobject{currentmarker}{}%
\end{pgfscope}%
\begin{pgfscope}%
\pgfsys@transformshift{2.894038in}{2.504094in}%
\pgfsys@useobject{currentmarker}{}%
\end{pgfscope}%
\begin{pgfscope}%
\pgfsys@transformshift{2.912660in}{2.512009in}%
\pgfsys@useobject{currentmarker}{}%
\end{pgfscope}%
\begin{pgfscope}%
\pgfsys@transformshift{2.931283in}{2.519861in}%
\pgfsys@useobject{currentmarker}{}%
\end{pgfscope}%
\begin{pgfscope}%
\pgfsys@transformshift{2.949905in}{2.527653in}%
\pgfsys@useobject{currentmarker}{}%
\end{pgfscope}%
\begin{pgfscope}%
\pgfsys@transformshift{2.968528in}{2.535384in}%
\pgfsys@useobject{currentmarker}{}%
\end{pgfscope}%
\begin{pgfscope}%
\pgfsys@transformshift{2.987150in}{2.543057in}%
\pgfsys@useobject{currentmarker}{}%
\end{pgfscope}%
\begin{pgfscope}%
\pgfsys@transformshift{3.005772in}{2.550672in}%
\pgfsys@useobject{currentmarker}{}%
\end{pgfscope}%
\begin{pgfscope}%
\pgfsys@transformshift{3.024395in}{2.558230in}%
\pgfsys@useobject{currentmarker}{}%
\end{pgfscope}%
\begin{pgfscope}%
\pgfsys@transformshift{3.043017in}{2.565733in}%
\pgfsys@useobject{currentmarker}{}%
\end{pgfscope}%
\begin{pgfscope}%
\pgfsys@transformshift{3.061639in}{2.573181in}%
\pgfsys@useobject{currentmarker}{}%
\end{pgfscope}%
\begin{pgfscope}%
\pgfsys@transformshift{3.080262in}{2.580575in}%
\pgfsys@useobject{currentmarker}{}%
\end{pgfscope}%
\begin{pgfscope}%
\pgfsys@transformshift{3.098884in}{2.587916in}%
\pgfsys@useobject{currentmarker}{}%
\end{pgfscope}%
\begin{pgfscope}%
\pgfsys@transformshift{3.117506in}{2.595205in}%
\pgfsys@useobject{currentmarker}{}%
\end{pgfscope}%
\begin{pgfscope}%
\pgfsys@transformshift{3.136129in}{2.602443in}%
\pgfsys@useobject{currentmarker}{}%
\end{pgfscope}%
\begin{pgfscope}%
\pgfsys@transformshift{3.154751in}{2.609629in}%
\pgfsys@useobject{currentmarker}{}%
\end{pgfscope}%
\begin{pgfscope}%
\pgfsys@transformshift{3.173373in}{2.616764in}%
\pgfsys@useobject{currentmarker}{}%
\end{pgfscope}%
\begin{pgfscope}%
\pgfsys@transformshift{3.191996in}{2.623848in}%
\pgfsys@useobject{currentmarker}{}%
\end{pgfscope}%
\begin{pgfscope}%
\pgfsys@transformshift{3.210618in}{2.630882in}%
\pgfsys@useobject{currentmarker}{}%
\end{pgfscope}%
\begin{pgfscope}%
\pgfsys@transformshift{3.229240in}{2.637865in}%
\pgfsys@useobject{currentmarker}{}%
\end{pgfscope}%
\begin{pgfscope}%
\pgfsys@transformshift{3.247863in}{2.644798in}%
\pgfsys@useobject{currentmarker}{}%
\end{pgfscope}%
\begin{pgfscope}%
\pgfsys@transformshift{3.266485in}{2.651680in}%
\pgfsys@useobject{currentmarker}{}%
\end{pgfscope}%
\begin{pgfscope}%
\pgfsys@transformshift{3.285107in}{2.658512in}%
\pgfsys@useobject{currentmarker}{}%
\end{pgfscope}%
\begin{pgfscope}%
\pgfsys@transformshift{3.303730in}{2.665293in}%
\pgfsys@useobject{currentmarker}{}%
\end{pgfscope}%
\begin{pgfscope}%
\pgfsys@transformshift{3.322352in}{2.672024in}%
\pgfsys@useobject{currentmarker}{}%
\end{pgfscope}%
\begin{pgfscope}%
\pgfsys@transformshift{3.340974in}{2.678704in}%
\pgfsys@useobject{currentmarker}{}%
\end{pgfscope}%
\begin{pgfscope}%
\pgfsys@transformshift{3.359597in}{2.685333in}%
\pgfsys@useobject{currentmarker}{}%
\end{pgfscope}%
\begin{pgfscope}%
\pgfsys@transformshift{3.378219in}{2.691911in}%
\pgfsys@useobject{currentmarker}{}%
\end{pgfscope}%
\begin{pgfscope}%
\pgfsys@transformshift{3.396841in}{2.698438in}%
\pgfsys@useobject{currentmarker}{}%
\end{pgfscope}%
\begin{pgfscope}%
\pgfsys@transformshift{3.415464in}{2.704915in}%
\pgfsys@useobject{currentmarker}{}%
\end{pgfscope}%
\begin{pgfscope}%
\pgfsys@transformshift{3.434086in}{2.711341in}%
\pgfsys@useobject{currentmarker}{}%
\end{pgfscope}%
\begin{pgfscope}%
\pgfsys@transformshift{3.452709in}{2.717716in}%
\pgfsys@useobject{currentmarker}{}%
\end{pgfscope}%
\begin{pgfscope}%
\pgfsys@transformshift{3.471331in}{2.724041in}%
\pgfsys@useobject{currentmarker}{}%
\end{pgfscope}%
\begin{pgfscope}%
\pgfsys@transformshift{3.489953in}{2.730317in}%
\pgfsys@useobject{currentmarker}{}%
\end{pgfscope}%
\begin{pgfscope}%
\pgfsys@transformshift{3.508576in}{2.736544in}%
\pgfsys@useobject{currentmarker}{}%
\end{pgfscope}%
\begin{pgfscope}%
\pgfsys@transformshift{3.527198in}{2.742721in}%
\pgfsys@useobject{currentmarker}{}%
\end{pgfscope}%
\begin{pgfscope}%
\pgfsys@transformshift{3.545820in}{2.748851in}%
\pgfsys@useobject{currentmarker}{}%
\end{pgfscope}%
\begin{pgfscope}%
\pgfsys@transformshift{3.564443in}{2.754933in}%
\pgfsys@useobject{currentmarker}{}%
\end{pgfscope}%
\begin{pgfscope}%
\pgfsys@transformshift{3.583065in}{2.760968in}%
\pgfsys@useobject{currentmarker}{}%
\end{pgfscope}%
\begin{pgfscope}%
\pgfsys@transformshift{3.601687in}{2.766957in}%
\pgfsys@useobject{currentmarker}{}%
\end{pgfscope}%
\begin{pgfscope}%
\pgfsys@transformshift{3.620310in}{2.772900in}%
\pgfsys@useobject{currentmarker}{}%
\end{pgfscope}%
\begin{pgfscope}%
\pgfsys@transformshift{3.638932in}{2.778798in}%
\pgfsys@useobject{currentmarker}{}%
\end{pgfscope}%
\begin{pgfscope}%
\pgfsys@transformshift{3.657554in}{2.784653in}%
\pgfsys@useobject{currentmarker}{}%
\end{pgfscope}%
\begin{pgfscope}%
\pgfsys@transformshift{3.676177in}{2.790463in}%
\pgfsys@useobject{currentmarker}{}%
\end{pgfscope}%
\begin{pgfscope}%
\pgfsys@transformshift{3.694799in}{2.796231in}%
\pgfsys@useobject{currentmarker}{}%
\end{pgfscope}%
\begin{pgfscope}%
\pgfsys@transformshift{3.713421in}{2.801957in}%
\pgfsys@useobject{currentmarker}{}%
\end{pgfscope}%
\begin{pgfscope}%
\pgfsys@transformshift{3.732044in}{2.807641in}%
\pgfsys@useobject{currentmarker}{}%
\end{pgfscope}%
\begin{pgfscope}%
\pgfsys@transformshift{3.750666in}{2.813284in}%
\pgfsys@useobject{currentmarker}{}%
\end{pgfscope}%
\begin{pgfscope}%
\pgfsys@transformshift{3.769288in}{2.818885in}%
\pgfsys@useobject{currentmarker}{}%
\end{pgfscope}%
\begin{pgfscope}%
\pgfsys@transformshift{3.787911in}{2.824446in}%
\pgfsys@useobject{currentmarker}{}%
\end{pgfscope}%
\begin{pgfscope}%
\pgfsys@transformshift{3.806533in}{2.829967in}%
\pgfsys@useobject{currentmarker}{}%
\end{pgfscope}%
\begin{pgfscope}%
\pgfsys@transformshift{3.825155in}{2.835447in}%
\pgfsys@useobject{currentmarker}{}%
\end{pgfscope}%
\begin{pgfscope}%
\pgfsys@transformshift{3.843778in}{2.840887in}%
\pgfsys@useobject{currentmarker}{}%
\end{pgfscope}%
\begin{pgfscope}%
\pgfsys@transformshift{3.862400in}{2.846286in}%
\pgfsys@useobject{currentmarker}{}%
\end{pgfscope}%
\begin{pgfscope}%
\pgfsys@transformshift{3.881023in}{2.851646in}%
\pgfsys@useobject{currentmarker}{}%
\end{pgfscope}%
\begin{pgfscope}%
\pgfsys@transformshift{3.899645in}{2.856966in}%
\pgfsys@useobject{currentmarker}{}%
\end{pgfscope}%
\begin{pgfscope}%
\pgfsys@transformshift{3.918267in}{2.862246in}%
\pgfsys@useobject{currentmarker}{}%
\end{pgfscope}%
\begin{pgfscope}%
\pgfsys@transformshift{3.936890in}{2.867487in}%
\pgfsys@useobject{currentmarker}{}%
\end{pgfscope}%
\begin{pgfscope}%
\pgfsys@transformshift{3.955512in}{2.872688in}%
\pgfsys@useobject{currentmarker}{}%
\end{pgfscope}%
\begin{pgfscope}%
\pgfsys@transformshift{3.974134in}{2.877848in}%
\pgfsys@useobject{currentmarker}{}%
\end{pgfscope}%
\begin{pgfscope}%
\pgfsys@transformshift{3.992757in}{2.882970in}%
\pgfsys@useobject{currentmarker}{}%
\end{pgfscope}%
\begin{pgfscope}%
\pgfsys@transformshift{4.011379in}{2.888051in}%
\pgfsys@useobject{currentmarker}{}%
\end{pgfscope}%
\begin{pgfscope}%
\pgfsys@transformshift{4.030001in}{2.893093in}%
\pgfsys@useobject{currentmarker}{}%
\end{pgfscope}%
\begin{pgfscope}%
\pgfsys@transformshift{4.048624in}{2.898096in}%
\pgfsys@useobject{currentmarker}{}%
\end{pgfscope}%
\begin{pgfscope}%
\pgfsys@transformshift{4.067246in}{2.903059in}%
\pgfsys@useobject{currentmarker}{}%
\end{pgfscope}%
\begin{pgfscope}%
\pgfsys@transformshift{4.085868in}{2.907983in}%
\pgfsys@useobject{currentmarker}{}%
\end{pgfscope}%
\begin{pgfscope}%
\pgfsys@transformshift{4.104491in}{2.912868in}%
\pgfsys@useobject{currentmarker}{}%
\end{pgfscope}%
\begin{pgfscope}%
\pgfsys@transformshift{4.123113in}{2.917715in}%
\pgfsys@useobject{currentmarker}{}%
\end{pgfscope}%
\begin{pgfscope}%
\pgfsys@transformshift{4.141735in}{2.922522in}%
\pgfsys@useobject{currentmarker}{}%
\end{pgfscope}%
\begin{pgfscope}%
\pgfsys@transformshift{4.160358in}{2.927291in}%
\pgfsys@useobject{currentmarker}{}%
\end{pgfscope}%
\begin{pgfscope}%
\pgfsys@transformshift{4.178980in}{2.932022in}%
\pgfsys@useobject{currentmarker}{}%
\end{pgfscope}%
\begin{pgfscope}%
\pgfsys@transformshift{4.197602in}{2.936715in}%
\pgfsys@useobject{currentmarker}{}%
\end{pgfscope}%
\begin{pgfscope}%
\pgfsys@transformshift{4.216225in}{2.941369in}%
\pgfsys@useobject{currentmarker}{}%
\end{pgfscope}%
\begin{pgfscope}%
\pgfsys@transformshift{4.234847in}{2.945986in}%
\pgfsys@useobject{currentmarker}{}%
\end{pgfscope}%
\begin{pgfscope}%
\pgfsys@transformshift{4.253469in}{2.950566in}%
\pgfsys@useobject{currentmarker}{}%
\end{pgfscope}%
\begin{pgfscope}%
\pgfsys@transformshift{4.272092in}{2.955109in}%
\pgfsys@useobject{currentmarker}{}%
\end{pgfscope}%
\begin{pgfscope}%
\pgfsys@transformshift{4.290714in}{2.959615in}%
\pgfsys@useobject{currentmarker}{}%
\end{pgfscope}%
\begin{pgfscope}%
\pgfsys@transformshift{4.309336in}{2.964086in}%
\pgfsys@useobject{currentmarker}{}%
\end{pgfscope}%
\begin{pgfscope}%
\pgfsys@transformshift{4.327959in}{2.968520in}%
\pgfsys@useobject{currentmarker}{}%
\end{pgfscope}%
\begin{pgfscope}%
\pgfsys@transformshift{4.346581in}{2.972920in}%
\pgfsys@useobject{currentmarker}{}%
\end{pgfscope}%
\begin{pgfscope}%
\pgfsys@transformshift{4.365204in}{2.977285in}%
\pgfsys@useobject{currentmarker}{}%
\end{pgfscope}%
\begin{pgfscope}%
\pgfsys@transformshift{4.383826in}{2.981615in}%
\pgfsys@useobject{currentmarker}{}%
\end{pgfscope}%
\begin{pgfscope}%
\pgfsys@transformshift{4.402448in}{2.985911in}%
\pgfsys@useobject{currentmarker}{}%
\end{pgfscope}%
\begin{pgfscope}%
\pgfsys@transformshift{4.421071in}{2.990173in}%
\pgfsys@useobject{currentmarker}{}%
\end{pgfscope}%
\begin{pgfscope}%
\pgfsys@transformshift{4.439693in}{2.994402in}%
\pgfsys@useobject{currentmarker}{}%
\end{pgfscope}%
\begin{pgfscope}%
\pgfsys@transformshift{4.458315in}{2.998598in}%
\pgfsys@useobject{currentmarker}{}%
\end{pgfscope}%
\begin{pgfscope}%
\pgfsys@transformshift{4.476938in}{3.002761in}%
\pgfsys@useobject{currentmarker}{}%
\end{pgfscope}%
\begin{pgfscope}%
\pgfsys@transformshift{4.495560in}{3.006891in}%
\pgfsys@useobject{currentmarker}{}%
\end{pgfscope}%
\begin{pgfscope}%
\pgfsys@transformshift{4.514182in}{3.010988in}%
\pgfsys@useobject{currentmarker}{}%
\end{pgfscope}%
\begin{pgfscope}%
\pgfsys@transformshift{4.532805in}{3.015053in}%
\pgfsys@useobject{currentmarker}{}%
\end{pgfscope}%
\begin{pgfscope}%
\pgfsys@transformshift{4.551427in}{3.019085in}%
\pgfsys@useobject{currentmarker}{}%
\end{pgfscope}%
\begin{pgfscope}%
\pgfsys@transformshift{4.570049in}{3.023084in}%
\pgfsys@useobject{currentmarker}{}%
\end{pgfscope}%
\begin{pgfscope}%
\pgfsys@transformshift{4.588672in}{3.027051in}%
\pgfsys@useobject{currentmarker}{}%
\end{pgfscope}%
\begin{pgfscope}%
\pgfsys@transformshift{4.607294in}{3.030985in}%
\pgfsys@useobject{currentmarker}{}%
\end{pgfscope}%
\begin{pgfscope}%
\pgfsys@transformshift{4.625916in}{3.034887in}%
\pgfsys@useobject{currentmarker}{}%
\end{pgfscope}%
\begin{pgfscope}%
\pgfsys@transformshift{4.644539in}{3.038755in}%
\pgfsys@useobject{currentmarker}{}%
\end{pgfscope}%
\begin{pgfscope}%
\pgfsys@transformshift{4.663161in}{3.042592in}%
\pgfsys@useobject{currentmarker}{}%
\end{pgfscope}%
\begin{pgfscope}%
\pgfsys@transformshift{4.681783in}{3.046395in}%
\pgfsys@useobject{currentmarker}{}%
\end{pgfscope}%
\begin{pgfscope}%
\pgfsys@transformshift{4.700406in}{3.050166in}%
\pgfsys@useobject{currentmarker}{}%
\end{pgfscope}%
\begin{pgfscope}%
\pgfsys@transformshift{4.719028in}{3.053905in}%
\pgfsys@useobject{currentmarker}{}%
\end{pgfscope}%
\begin{pgfscope}%
\pgfsys@transformshift{4.737650in}{3.057612in}%
\pgfsys@useobject{currentmarker}{}%
\end{pgfscope}%
\begin{pgfscope}%
\pgfsys@transformshift{4.756273in}{3.061287in}%
\pgfsys@useobject{currentmarker}{}%
\end{pgfscope}%
\begin{pgfscope}%
\pgfsys@transformshift{4.774895in}{3.064930in}%
\pgfsys@useobject{currentmarker}{}%
\end{pgfscope}%
\begin{pgfscope}%
\pgfsys@transformshift{4.793518in}{3.068542in}%
\pgfsys@useobject{currentmarker}{}%
\end{pgfscope}%
\begin{pgfscope}%
\pgfsys@transformshift{4.812140in}{3.072124in}%
\pgfsys@useobject{currentmarker}{}%
\end{pgfscope}%
\begin{pgfscope}%
\pgfsys@transformshift{4.830762in}{3.075674in}%
\pgfsys@useobject{currentmarker}{}%
\end{pgfscope}%
\begin{pgfscope}%
\pgfsys@transformshift{4.849385in}{3.079195in}%
\pgfsys@useobject{currentmarker}{}%
\end{pgfscope}%
\begin{pgfscope}%
\pgfsys@transformshift{4.868007in}{3.082686in}%
\pgfsys@useobject{currentmarker}{}%
\end{pgfscope}%
\begin{pgfscope}%
\pgfsys@transformshift{4.886629in}{3.086147in}%
\pgfsys@useobject{currentmarker}{}%
\end{pgfscope}%
\begin{pgfscope}%
\pgfsys@transformshift{4.905252in}{3.089579in}%
\pgfsys@useobject{currentmarker}{}%
\end{pgfscope}%
\begin{pgfscope}%
\pgfsys@transformshift{4.923874in}{3.092983in}%
\pgfsys@useobject{currentmarker}{}%
\end{pgfscope}%
\begin{pgfscope}%
\pgfsys@transformshift{4.942496in}{3.096358in}%
\pgfsys@useobject{currentmarker}{}%
\end{pgfscope}%
\begin{pgfscope}%
\pgfsys@transformshift{4.961119in}{3.099705in}%
\pgfsys@useobject{currentmarker}{}%
\end{pgfscope}%
\begin{pgfscope}%
\pgfsys@transformshift{4.979741in}{3.103024in}%
\pgfsys@useobject{currentmarker}{}%
\end{pgfscope}%
\begin{pgfscope}%
\pgfsys@transformshift{4.998363in}{3.106316in}%
\pgfsys@useobject{currentmarker}{}%
\end{pgfscope}%
\begin{pgfscope}%
\pgfsys@transformshift{5.016986in}{3.109581in}%
\pgfsys@useobject{currentmarker}{}%
\end{pgfscope}%
\end{pgfscope}%
\begin{pgfscope}%
\pgfpathrectangle{\pgfqpoint{0.578349in}{0.682899in}}{\pgfqpoint{4.650000in}{3.020000in}}%
\pgfusepath{clip}%
\pgfsetrectcap%
\pgfsetroundjoin%
\pgfsetlinewidth{1.505625pt}%
\definecolor{currentstroke}{rgb}{1.000000,0.000000,0.000000}%
\pgfsetstrokecolor{currentstroke}%
\pgfsetdash{}{0pt}%
\pgfpathmoveto{\pgfqpoint{0.882825in}{1.010164in}}%
\pgfpathlineto{\pgfqpoint{0.938692in}{1.079779in}}%
\pgfpathlineto{\pgfqpoint{0.994559in}{1.146301in}}%
\pgfpathlineto{\pgfqpoint{1.050426in}{1.210108in}}%
\pgfpathlineto{\pgfqpoint{1.124915in}{1.291423in}}%
\pgfpathlineto{\pgfqpoint{1.199405in}{1.368878in}}%
\pgfpathlineto{\pgfqpoint{1.273894in}{1.442848in}}%
\pgfpathlineto{\pgfqpoint{1.348383in}{1.513634in}}%
\pgfpathlineto{\pgfqpoint{1.422873in}{1.581487in}}%
\pgfpathlineto{\pgfqpoint{1.497362in}{1.646619in}}%
\pgfpathlineto{\pgfqpoint{1.571852in}{1.709216in}}%
\pgfpathlineto{\pgfqpoint{1.646341in}{1.769437in}}%
\pgfpathlineto{\pgfqpoint{1.739453in}{1.841586in}}%
\pgfpathlineto{\pgfqpoint{1.832564in}{1.910487in}}%
\pgfpathlineto{\pgfqpoint{1.925676in}{1.976349in}}%
\pgfpathlineto{\pgfqpoint{2.018788in}{2.039357in}}%
\pgfpathlineto{\pgfqpoint{2.111900in}{2.099676in}}%
\pgfpathlineto{\pgfqpoint{2.205011in}{2.157448in}}%
\pgfpathlineto{\pgfqpoint{2.298123in}{2.212804in}}%
\pgfpathlineto{\pgfqpoint{2.391235in}{2.265860in}}%
\pgfpathlineto{\pgfqpoint{2.484347in}{2.316721in}}%
\pgfpathlineto{\pgfqpoint{2.596081in}{2.374987in}}%
\pgfpathlineto{\pgfqpoint{2.707815in}{2.430373in}}%
\pgfpathlineto{\pgfqpoint{2.819549in}{2.483010in}}%
\pgfpathlineto{\pgfqpoint{2.931283in}{2.533016in}}%
\pgfpathlineto{\pgfqpoint{3.043017in}{2.580496in}}%
\pgfpathlineto{\pgfqpoint{3.154751in}{2.625546in}}%
\pgfpathlineto{\pgfqpoint{3.266485in}{2.668252in}}%
\pgfpathlineto{\pgfqpoint{3.378219in}{2.708692in}}%
\pgfpathlineto{\pgfqpoint{3.489953in}{2.746938in}}%
\pgfpathlineto{\pgfqpoint{3.601687in}{2.783055in}}%
\pgfpathlineto{\pgfqpoint{3.713421in}{2.817101in}}%
\pgfpathlineto{\pgfqpoint{3.825155in}{2.849130in}}%
\pgfpathlineto{\pgfqpoint{3.955512in}{2.884015in}}%
\pgfpathlineto{\pgfqpoint{4.085868in}{2.916294in}}%
\pgfpathlineto{\pgfqpoint{4.216225in}{2.946030in}}%
\pgfpathlineto{\pgfqpoint{4.346581in}{2.973280in}}%
\pgfpathlineto{\pgfqpoint{4.476938in}{2.998097in}}%
\pgfpathlineto{\pgfqpoint{4.607294in}{3.020526in}}%
\pgfpathlineto{\pgfqpoint{4.737650in}{3.040608in}}%
\pgfpathlineto{\pgfqpoint{4.868007in}{3.058380in}}%
\pgfpathlineto{\pgfqpoint{4.998363in}{3.073874in}}%
\pgfpathlineto{\pgfqpoint{5.016986in}{3.075900in}}%
\pgfpathlineto{\pgfqpoint{5.016986in}{3.075900in}}%
\pgfusepath{stroke}%
\end{pgfscope}%
\begin{pgfscope}%
\pgfpathrectangle{\pgfqpoint{0.578349in}{0.682899in}}{\pgfqpoint{4.650000in}{3.020000in}}%
\pgfusepath{clip}%
\pgfsetbuttcap%
\pgfsetroundjoin%
\definecolor{currentfill}{rgb}{0.000000,0.000000,0.000000}%
\pgfsetfillcolor{currentfill}%
\pgfsetfillopacity{0.500000}%
\pgfsetlinewidth{1.003750pt}%
\definecolor{currentstroke}{rgb}{0.000000,0.000000,0.000000}%
\pgfsetstrokecolor{currentstroke}%
\pgfsetstrokeopacity{0.500000}%
\pgfsetdash{}{0pt}%
\pgfsys@defobject{currentmarker}{\pgfqpoint{-0.041667in}{-0.041667in}}{\pgfqpoint{0.041667in}{0.041667in}}{%
\pgfpathmoveto{\pgfqpoint{0.000000in}{-0.041667in}}%
\pgfpathcurveto{\pgfqpoint{0.011050in}{-0.041667in}}{\pgfqpoint{0.021649in}{-0.037276in}}{\pgfqpoint{0.029463in}{-0.029463in}}%
\pgfpathcurveto{\pgfqpoint{0.037276in}{-0.021649in}}{\pgfqpoint{0.041667in}{-0.011050in}}{\pgfqpoint{0.041667in}{0.000000in}}%
\pgfpathcurveto{\pgfqpoint{0.041667in}{0.011050in}}{\pgfqpoint{0.037276in}{0.021649in}}{\pgfqpoint{0.029463in}{0.029463in}}%
\pgfpathcurveto{\pgfqpoint{0.021649in}{0.037276in}}{\pgfqpoint{0.011050in}{0.041667in}}{\pgfqpoint{0.000000in}{0.041667in}}%
\pgfpathcurveto{\pgfqpoint{-0.011050in}{0.041667in}}{\pgfqpoint{-0.021649in}{0.037276in}}{\pgfqpoint{-0.029463in}{0.029463in}}%
\pgfpathcurveto{\pgfqpoint{-0.037276in}{0.021649in}}{\pgfqpoint{-0.041667in}{0.011050in}}{\pgfqpoint{-0.041667in}{0.000000in}}%
\pgfpathcurveto{\pgfqpoint{-0.041667in}{-0.011050in}}{\pgfqpoint{-0.037276in}{-0.021649in}}{\pgfqpoint{-0.029463in}{-0.029463in}}%
\pgfpathcurveto{\pgfqpoint{-0.021649in}{-0.037276in}}{\pgfqpoint{-0.011050in}{-0.041667in}}{\pgfqpoint{0.000000in}{-0.041667in}}%
\pgfpathclose%
\pgfusepath{stroke,fill}%
}%
\begin{pgfscope}%
\pgfsys@transformshift{0.826958in}{0.925804in}%
\pgfsys@useobject{currentmarker}{}%
\end{pgfscope}%
\begin{pgfscope}%
\pgfsys@transformshift{0.845580in}{0.949252in}%
\pgfsys@useobject{currentmarker}{}%
\end{pgfscope}%
\begin{pgfscope}%
\pgfsys@transformshift{0.864202in}{0.972266in}%
\pgfsys@useobject{currentmarker}{}%
\end{pgfscope}%
\begin{pgfscope}%
\pgfsys@transformshift{0.882825in}{0.994855in}%
\pgfsys@useobject{currentmarker}{}%
\end{pgfscope}%
\begin{pgfscope}%
\pgfsys@transformshift{0.901447in}{1.017025in}%
\pgfsys@useobject{currentmarker}{}%
\end{pgfscope}%
\begin{pgfscope}%
\pgfsys@transformshift{0.920069in}{1.038783in}%
\pgfsys@useobject{currentmarker}{}%
\end{pgfscope}%
\begin{pgfscope}%
\pgfsys@transformshift{0.938692in}{1.060136in}%
\pgfsys@useobject{currentmarker}{}%
\end{pgfscope}%
\begin{pgfscope}%
\pgfsys@transformshift{0.957314in}{1.081090in}%
\pgfsys@useobject{currentmarker}{}%
\end{pgfscope}%
\begin{pgfscope}%
\pgfsys@transformshift{0.975936in}{1.101654in}%
\pgfsys@useobject{currentmarker}{}%
\end{pgfscope}%
\begin{pgfscope}%
\pgfsys@transformshift{0.994559in}{1.121832in}%
\pgfsys@useobject{currentmarker}{}%
\end{pgfscope}%
\begin{pgfscope}%
\pgfsys@transformshift{1.013181in}{1.141634in}%
\pgfsys@useobject{currentmarker}{}%
\end{pgfscope}%
\begin{pgfscope}%
\pgfsys@transformshift{1.031803in}{1.161065in}%
\pgfsys@useobject{currentmarker}{}%
\end{pgfscope}%
\begin{pgfscope}%
\pgfsys@transformshift{1.050426in}{1.180132in}%
\pgfsys@useobject{currentmarker}{}%
\end{pgfscope}%
\begin{pgfscope}%
\pgfsys@transformshift{1.069048in}{1.198841in}%
\pgfsys@useobject{currentmarker}{}%
\end{pgfscope}%
\begin{pgfscope}%
\pgfsys@transformshift{1.087670in}{1.217198in}%
\pgfsys@useobject{currentmarker}{}%
\end{pgfscope}%
\begin{pgfscope}%
\pgfsys@transformshift{1.106293in}{1.235212in}%
\pgfsys@useobject{currentmarker}{}%
\end{pgfscope}%
\begin{pgfscope}%
\pgfsys@transformshift{1.124915in}{1.252890in}%
\pgfsys@useobject{currentmarker}{}%
\end{pgfscope}%
\begin{pgfscope}%
\pgfsys@transformshift{1.143538in}{1.270239in}%
\pgfsys@useobject{currentmarker}{}%
\end{pgfscope}%
\begin{pgfscope}%
\pgfsys@transformshift{1.162160in}{1.287268in}%
\pgfsys@useobject{currentmarker}{}%
\end{pgfscope}%
\begin{pgfscope}%
\pgfsys@transformshift{1.180782in}{1.303985in}%
\pgfsys@useobject{currentmarker}{}%
\end{pgfscope}%
\begin{pgfscope}%
\pgfsys@transformshift{1.199405in}{1.320399in}%
\pgfsys@useobject{currentmarker}{}%
\end{pgfscope}%
\begin{pgfscope}%
\pgfsys@transformshift{1.218027in}{1.336516in}%
\pgfsys@useobject{currentmarker}{}%
\end{pgfscope}%
\begin{pgfscope}%
\pgfsys@transformshift{1.236649in}{1.352347in}%
\pgfsys@useobject{currentmarker}{}%
\end{pgfscope}%
\begin{pgfscope}%
\pgfsys@transformshift{1.255272in}{1.367898in}%
\pgfsys@useobject{currentmarker}{}%
\end{pgfscope}%
\begin{pgfscope}%
\pgfsys@transformshift{1.273894in}{1.383177in}%
\pgfsys@useobject{currentmarker}{}%
\end{pgfscope}%
\begin{pgfscope}%
\pgfsys@transformshift{1.292516in}{1.398192in}%
\pgfsys@useobject{currentmarker}{}%
\end{pgfscope}%
\begin{pgfscope}%
\pgfsys@transformshift{1.311139in}{1.412950in}%
\pgfsys@useobject{currentmarker}{}%
\end{pgfscope}%
\begin{pgfscope}%
\pgfsys@transformshift{1.329761in}{1.427457in}%
\pgfsys@useobject{currentmarker}{}%
\end{pgfscope}%
\begin{pgfscope}%
\pgfsys@transformshift{1.348383in}{1.441718in}%
\pgfsys@useobject{currentmarker}{}%
\end{pgfscope}%
\begin{pgfscope}%
\pgfsys@transformshift{1.367006in}{1.455740in}%
\pgfsys@useobject{currentmarker}{}%
\end{pgfscope}%
\begin{pgfscope}%
\pgfsys@transformshift{1.385628in}{1.469526in}%
\pgfsys@useobject{currentmarker}{}%
\end{pgfscope}%
\begin{pgfscope}%
\pgfsys@transformshift{1.404250in}{1.483082in}%
\pgfsys@useobject{currentmarker}{}%
\end{pgfscope}%
\begin{pgfscope}%
\pgfsys@transformshift{1.422873in}{1.496412in}%
\pgfsys@useobject{currentmarker}{}%
\end{pgfscope}%
\begin{pgfscope}%
\pgfsys@transformshift{1.441495in}{1.509518in}%
\pgfsys@useobject{currentmarker}{}%
\end{pgfscope}%
\begin{pgfscope}%
\pgfsys@transformshift{1.460117in}{1.522405in}%
\pgfsys@useobject{currentmarker}{}%
\end{pgfscope}%
\begin{pgfscope}%
\pgfsys@transformshift{1.478740in}{1.535076in}%
\pgfsys@useobject{currentmarker}{}%
\end{pgfscope}%
\begin{pgfscope}%
\pgfsys@transformshift{1.497362in}{1.547534in}%
\pgfsys@useobject{currentmarker}{}%
\end{pgfscope}%
\begin{pgfscope}%
\pgfsys@transformshift{1.515984in}{1.559782in}%
\pgfsys@useobject{currentmarker}{}%
\end{pgfscope}%
\begin{pgfscope}%
\pgfsys@transformshift{1.534607in}{1.571822in}%
\pgfsys@useobject{currentmarker}{}%
\end{pgfscope}%
\begin{pgfscope}%
\pgfsys@transformshift{1.553229in}{1.583659in}%
\pgfsys@useobject{currentmarker}{}%
\end{pgfscope}%
\begin{pgfscope}%
\pgfsys@transformshift{1.571852in}{1.595293in}%
\pgfsys@useobject{currentmarker}{}%
\end{pgfscope}%
\begin{pgfscope}%
\pgfsys@transformshift{1.590474in}{1.606730in}%
\pgfsys@useobject{currentmarker}{}%
\end{pgfscope}%
\begin{pgfscope}%
\pgfsys@transformshift{1.609096in}{1.617971in}%
\pgfsys@useobject{currentmarker}{}%
\end{pgfscope}%
\begin{pgfscope}%
\pgfsys@transformshift{1.627719in}{1.629018in}%
\pgfsys@useobject{currentmarker}{}%
\end{pgfscope}%
\begin{pgfscope}%
\pgfsys@transformshift{1.646341in}{1.639876in}%
\pgfsys@useobject{currentmarker}{}%
\end{pgfscope}%
\begin{pgfscope}%
\pgfsys@transformshift{1.664963in}{1.650547in}%
\pgfsys@useobject{currentmarker}{}%
\end{pgfscope}%
\begin{pgfscope}%
\pgfsys@transformshift{1.683586in}{1.661033in}%
\pgfsys@useobject{currentmarker}{}%
\end{pgfscope}%
\begin{pgfscope}%
\pgfsys@transformshift{1.702208in}{1.671338in}%
\pgfsys@useobject{currentmarker}{}%
\end{pgfscope}%
\begin{pgfscope}%
\pgfsys@transformshift{1.720830in}{1.681463in}%
\pgfsys@useobject{currentmarker}{}%
\end{pgfscope}%
\begin{pgfscope}%
\pgfsys@transformshift{1.739453in}{1.691411in}%
\pgfsys@useobject{currentmarker}{}%
\end{pgfscope}%
\begin{pgfscope}%
\pgfsys@transformshift{1.758075in}{1.701185in}%
\pgfsys@useobject{currentmarker}{}%
\end{pgfscope}%
\begin{pgfscope}%
\pgfsys@transformshift{1.776697in}{1.710785in}%
\pgfsys@useobject{currentmarker}{}%
\end{pgfscope}%
\begin{pgfscope}%
\pgfsys@transformshift{1.795320in}{1.720216in}%
\pgfsys@useobject{currentmarker}{}%
\end{pgfscope}%
\begin{pgfscope}%
\pgfsys@transformshift{1.813942in}{1.729478in}%
\pgfsys@useobject{currentmarker}{}%
\end{pgfscope}%
\begin{pgfscope}%
\pgfsys@transformshift{1.832564in}{1.738573in}%
\pgfsys@useobject{currentmarker}{}%
\end{pgfscope}%
\begin{pgfscope}%
\pgfsys@transformshift{1.851187in}{1.747503in}%
\pgfsys@useobject{currentmarker}{}%
\end{pgfscope}%
\begin{pgfscope}%
\pgfsys@transformshift{1.869809in}{1.756270in}%
\pgfsys@useobject{currentmarker}{}%
\end{pgfscope}%
\begin{pgfscope}%
\pgfsys@transformshift{1.888431in}{1.764876in}%
\pgfsys@useobject{currentmarker}{}%
\end{pgfscope}%
\begin{pgfscope}%
\pgfsys@transformshift{1.907054in}{1.773321in}%
\pgfsys@useobject{currentmarker}{}%
\end{pgfscope}%
\begin{pgfscope}%
\pgfsys@transformshift{1.925676in}{1.781607in}%
\pgfsys@useobject{currentmarker}{}%
\end{pgfscope}%
\begin{pgfscope}%
\pgfsys@transformshift{1.944298in}{1.789735in}%
\pgfsys@useobject{currentmarker}{}%
\end{pgfscope}%
\begin{pgfscope}%
\pgfsys@transformshift{1.962921in}{1.797708in}%
\pgfsys@useobject{currentmarker}{}%
\end{pgfscope}%
\begin{pgfscope}%
\pgfsys@transformshift{1.981543in}{1.805525in}%
\pgfsys@useobject{currentmarker}{}%
\end{pgfscope}%
\begin{pgfscope}%
\pgfsys@transformshift{2.000165in}{1.813190in}%
\pgfsys@useobject{currentmarker}{}%
\end{pgfscope}%
\begin{pgfscope}%
\pgfsys@transformshift{2.018788in}{1.820702in}%
\pgfsys@useobject{currentmarker}{}%
\end{pgfscope}%
\begin{pgfscope}%
\pgfsys@transformshift{2.037410in}{1.828064in}%
\pgfsys@useobject{currentmarker}{}%
\end{pgfscope}%
\begin{pgfscope}%
\pgfsys@transformshift{2.056033in}{1.835276in}%
\pgfsys@useobject{currentmarker}{}%
\end{pgfscope}%
\begin{pgfscope}%
\pgfsys@transformshift{2.074655in}{1.842341in}%
\pgfsys@useobject{currentmarker}{}%
\end{pgfscope}%
\begin{pgfscope}%
\pgfsys@transformshift{2.093277in}{1.849258in}%
\pgfsys@useobject{currentmarker}{}%
\end{pgfscope}%
\begin{pgfscope}%
\pgfsys@transformshift{2.111900in}{1.856030in}%
\pgfsys@useobject{currentmarker}{}%
\end{pgfscope}%
\begin{pgfscope}%
\pgfsys@transformshift{2.130522in}{1.862658in}%
\pgfsys@useobject{currentmarker}{}%
\end{pgfscope}%
\begin{pgfscope}%
\pgfsys@transformshift{2.149144in}{1.869144in}%
\pgfsys@useobject{currentmarker}{}%
\end{pgfscope}%
\begin{pgfscope}%
\pgfsys@transformshift{2.167767in}{1.875487in}%
\pgfsys@useobject{currentmarker}{}%
\end{pgfscope}%
\begin{pgfscope}%
\pgfsys@transformshift{2.186389in}{1.881691in}%
\pgfsys@useobject{currentmarker}{}%
\end{pgfscope}%
\begin{pgfscope}%
\pgfsys@transformshift{2.205011in}{1.887755in}%
\pgfsys@useobject{currentmarker}{}%
\end{pgfscope}%
\begin{pgfscope}%
\pgfsys@transformshift{2.223634in}{1.893680in}%
\pgfsys@useobject{currentmarker}{}%
\end{pgfscope}%
\begin{pgfscope}%
\pgfsys@transformshift{2.242256in}{1.899469in}%
\pgfsys@useobject{currentmarker}{}%
\end{pgfscope}%
\begin{pgfscope}%
\pgfsys@transformshift{2.260878in}{1.905122in}%
\pgfsys@useobject{currentmarker}{}%
\end{pgfscope}%
\begin{pgfscope}%
\pgfsys@transformshift{2.279501in}{1.910639in}%
\pgfsys@useobject{currentmarker}{}%
\end{pgfscope}%
\begin{pgfscope}%
\pgfsys@transformshift{2.298123in}{1.916022in}%
\pgfsys@useobject{currentmarker}{}%
\end{pgfscope}%
\begin{pgfscope}%
\pgfsys@transformshift{2.316745in}{1.921271in}%
\pgfsys@useobject{currentmarker}{}%
\end{pgfscope}%
\begin{pgfscope}%
\pgfsys@transformshift{2.335368in}{1.926387in}%
\pgfsys@useobject{currentmarker}{}%
\end{pgfscope}%
\begin{pgfscope}%
\pgfsys@transformshift{2.353990in}{1.931371in}%
\pgfsys@useobject{currentmarker}{}%
\end{pgfscope}%
\begin{pgfscope}%
\pgfsys@transformshift{2.372612in}{1.936223in}%
\pgfsys@useobject{currentmarker}{}%
\end{pgfscope}%
\begin{pgfscope}%
\pgfsys@transformshift{2.391235in}{1.940944in}%
\pgfsys@useobject{currentmarker}{}%
\end{pgfscope}%
\begin{pgfscope}%
\pgfsys@transformshift{2.409857in}{1.945535in}%
\pgfsys@useobject{currentmarker}{}%
\end{pgfscope}%
\begin{pgfscope}%
\pgfsys@transformshift{2.428479in}{1.949995in}%
\pgfsys@useobject{currentmarker}{}%
\end{pgfscope}%
\begin{pgfscope}%
\pgfsys@transformshift{2.447102in}{1.954327in}%
\pgfsys@useobject{currentmarker}{}%
\end{pgfscope}%
\begin{pgfscope}%
\pgfsys@transformshift{2.465724in}{1.958529in}%
\pgfsys@useobject{currentmarker}{}%
\end{pgfscope}%
\begin{pgfscope}%
\pgfsys@transformshift{2.484347in}{1.962604in}%
\pgfsys@useobject{currentmarker}{}%
\end{pgfscope}%
\begin{pgfscope}%
\pgfsys@transformshift{2.502969in}{1.966551in}%
\pgfsys@useobject{currentmarker}{}%
\end{pgfscope}%
\begin{pgfscope}%
\pgfsys@transformshift{2.521591in}{1.970371in}%
\pgfsys@useobject{currentmarker}{}%
\end{pgfscope}%
\begin{pgfscope}%
\pgfsys@transformshift{2.540214in}{1.974064in}%
\pgfsys@useobject{currentmarker}{}%
\end{pgfscope}%
\begin{pgfscope}%
\pgfsys@transformshift{2.558836in}{1.977631in}%
\pgfsys@useobject{currentmarker}{}%
\end{pgfscope}%
\begin{pgfscope}%
\pgfsys@transformshift{2.577458in}{1.981074in}%
\pgfsys@useobject{currentmarker}{}%
\end{pgfscope}%
\begin{pgfscope}%
\pgfsys@transformshift{2.596081in}{1.984391in}%
\pgfsys@useobject{currentmarker}{}%
\end{pgfscope}%
\begin{pgfscope}%
\pgfsys@transformshift{2.614703in}{1.987584in}%
\pgfsys@useobject{currentmarker}{}%
\end{pgfscope}%
\begin{pgfscope}%
\pgfsys@transformshift{2.633325in}{1.990654in}%
\pgfsys@useobject{currentmarker}{}%
\end{pgfscope}%
\begin{pgfscope}%
\pgfsys@transformshift{2.651948in}{1.993600in}%
\pgfsys@useobject{currentmarker}{}%
\end{pgfscope}%
\begin{pgfscope}%
\pgfsys@transformshift{2.670570in}{1.996424in}%
\pgfsys@useobject{currentmarker}{}%
\end{pgfscope}%
\begin{pgfscope}%
\pgfsys@transformshift{2.689192in}{1.999125in}%
\pgfsys@useobject{currentmarker}{}%
\end{pgfscope}%
\begin{pgfscope}%
\pgfsys@transformshift{2.707815in}{2.001705in}%
\pgfsys@useobject{currentmarker}{}%
\end{pgfscope}%
\begin{pgfscope}%
\pgfsys@transformshift{2.726437in}{2.004164in}%
\pgfsys@useobject{currentmarker}{}%
\end{pgfscope}%
\begin{pgfscope}%
\pgfsys@transformshift{2.745059in}{2.006501in}%
\pgfsys@useobject{currentmarker}{}%
\end{pgfscope}%
\begin{pgfscope}%
\pgfsys@transformshift{2.763682in}{2.008718in}%
\pgfsys@useobject{currentmarker}{}%
\end{pgfscope}%
\begin{pgfscope}%
\pgfsys@transformshift{2.782304in}{2.010814in}%
\pgfsys@useobject{currentmarker}{}%
\end{pgfscope}%
\begin{pgfscope}%
\pgfsys@transformshift{2.800926in}{2.012790in}%
\pgfsys@useobject{currentmarker}{}%
\end{pgfscope}%
\begin{pgfscope}%
\pgfsys@transformshift{2.819549in}{2.014646in}%
\pgfsys@useobject{currentmarker}{}%
\end{pgfscope}%
\begin{pgfscope}%
\pgfsys@transformshift{2.838171in}{2.016382in}%
\pgfsys@useobject{currentmarker}{}%
\end{pgfscope}%
\begin{pgfscope}%
\pgfsys@transformshift{2.856793in}{2.017999in}%
\pgfsys@useobject{currentmarker}{}%
\end{pgfscope}%
\begin{pgfscope}%
\pgfsys@transformshift{2.875416in}{2.019497in}%
\pgfsys@useobject{currentmarker}{}%
\end{pgfscope}%
\begin{pgfscope}%
\pgfsys@transformshift{2.894038in}{2.020875in}%
\pgfsys@useobject{currentmarker}{}%
\end{pgfscope}%
\begin{pgfscope}%
\pgfsys@transformshift{2.912660in}{2.022135in}%
\pgfsys@useobject{currentmarker}{}%
\end{pgfscope}%
\begin{pgfscope}%
\pgfsys@transformshift{2.931283in}{2.023276in}%
\pgfsys@useobject{currentmarker}{}%
\end{pgfscope}%
\begin{pgfscope}%
\pgfsys@transformshift{2.949905in}{2.024298in}%
\pgfsys@useobject{currentmarker}{}%
\end{pgfscope}%
\begin{pgfscope}%
\pgfsys@transformshift{2.968528in}{2.025202in}%
\pgfsys@useobject{currentmarker}{}%
\end{pgfscope}%
\begin{pgfscope}%
\pgfsys@transformshift{2.987150in}{2.025988in}%
\pgfsys@useobject{currentmarker}{}%
\end{pgfscope}%
\begin{pgfscope}%
\pgfsys@transformshift{3.005772in}{2.026655in}%
\pgfsys@useobject{currentmarker}{}%
\end{pgfscope}%
\begin{pgfscope}%
\pgfsys@transformshift{3.024395in}{2.027205in}%
\pgfsys@useobject{currentmarker}{}%
\end{pgfscope}%
\begin{pgfscope}%
\pgfsys@transformshift{3.043017in}{2.027637in}%
\pgfsys@useobject{currentmarker}{}%
\end{pgfscope}%
\begin{pgfscope}%
\pgfsys@transformshift{3.061639in}{2.027950in}%
\pgfsys@useobject{currentmarker}{}%
\end{pgfscope}%
\begin{pgfscope}%
\pgfsys@transformshift{3.080262in}{2.028146in}%
\pgfsys@useobject{currentmarker}{}%
\end{pgfscope}%
\end{pgfscope}%
\begin{pgfscope}%
\pgfpathrectangle{\pgfqpoint{0.578349in}{0.682899in}}{\pgfqpoint{4.650000in}{3.020000in}}%
\pgfusepath{clip}%
\pgfsetrectcap%
\pgfsetroundjoin%
\pgfsetlinewidth{1.505625pt}%
\definecolor{currentstroke}{rgb}{1.000000,0.000000,0.000000}%
\pgfsetstrokecolor{currentstroke}%
\pgfsetdash{}{0pt}%
\pgfpathmoveto{\pgfqpoint{0.826958in}{0.925804in}}%
\pgfpathlineto{\pgfqpoint{0.845580in}{0.949376in}}%
\pgfpathlineto{\pgfqpoint{0.864202in}{0.972385in}}%
\pgfpathlineto{\pgfqpoint{0.882825in}{0.994874in}}%
\pgfpathlineto{\pgfqpoint{0.901447in}{1.016876in}}%
\pgfpathlineto{\pgfqpoint{0.920069in}{1.038420in}}%
\pgfpathlineto{\pgfqpoint{0.938692in}{1.059528in}}%
\pgfpathlineto{\pgfqpoint{0.957314in}{1.080221in}}%
\pgfpathlineto{\pgfqpoint{0.975936in}{1.100517in}}%
\pgfpathlineto{\pgfqpoint{0.994559in}{1.120431in}}%
\pgfpathlineto{\pgfqpoint{1.013181in}{1.139977in}}%
\pgfpathlineto{\pgfqpoint{1.031803in}{1.159168in}}%
\pgfpathlineto{\pgfqpoint{1.050426in}{1.178015in}}%
\pgfpathlineto{\pgfqpoint{1.069048in}{1.196528in}}%
\pgfpathlineto{\pgfqpoint{1.087670in}{1.214717in}}%
\pgfpathlineto{\pgfqpoint{1.106293in}{1.232591in}}%
\pgfpathlineto{\pgfqpoint{1.124915in}{1.250158in}}%
\pgfpathlineto{\pgfqpoint{1.143538in}{1.267425in}}%
\pgfpathlineto{\pgfqpoint{1.162160in}{1.284399in}}%
\pgfpathlineto{\pgfqpoint{1.180782in}{1.301088in}}%
\pgfpathlineto{\pgfqpoint{1.199405in}{1.317498in}}%
\pgfpathlineto{\pgfqpoint{1.218027in}{1.333634in}}%
\pgfpathlineto{\pgfqpoint{1.236649in}{1.349502in}}%
\pgfpathlineto{\pgfqpoint{1.255272in}{1.365108in}}%
\pgfpathlineto{\pgfqpoint{1.273894in}{1.380456in}}%
\pgfpathlineto{\pgfqpoint{1.292516in}{1.395552in}}%
\pgfpathlineto{\pgfqpoint{1.311139in}{1.410401in}}%
\pgfpathlineto{\pgfqpoint{1.329761in}{1.425006in}}%
\pgfpathlineto{\pgfqpoint{1.348383in}{1.439371in}}%
\pgfpathlineto{\pgfqpoint{1.367006in}{1.453502in}}%
\pgfpathlineto{\pgfqpoint{1.385628in}{1.467401in}}%
\pgfpathlineto{\pgfqpoint{1.404250in}{1.481073in}}%
\pgfpathlineto{\pgfqpoint{1.422873in}{1.494520in}}%
\pgfpathlineto{\pgfqpoint{1.441495in}{1.507747in}}%
\pgfpathlineto{\pgfqpoint{1.460117in}{1.520757in}}%
\pgfpathlineto{\pgfqpoint{1.478740in}{1.533553in}}%
\pgfpathlineto{\pgfqpoint{1.497362in}{1.546138in}}%
\pgfpathlineto{\pgfqpoint{1.515984in}{1.558515in}}%
\pgfpathlineto{\pgfqpoint{1.534607in}{1.570687in}}%
\pgfpathlineto{\pgfqpoint{1.553229in}{1.582656in}}%
\pgfpathlineto{\pgfqpoint{1.571852in}{1.594426in}}%
\pgfpathlineto{\pgfqpoint{1.590474in}{1.605998in}}%
\pgfpathlineto{\pgfqpoint{1.609096in}{1.617376in}}%
\pgfpathlineto{\pgfqpoint{1.627719in}{1.628561in}}%
\pgfpathlineto{\pgfqpoint{1.646341in}{1.639556in}}%
\pgfpathlineto{\pgfqpoint{1.664963in}{1.650364in}}%
\pgfpathlineto{\pgfqpoint{1.683586in}{1.660986in}}%
\pgfpathlineto{\pgfqpoint{1.702208in}{1.671425in}}%
\pgfpathlineto{\pgfqpoint{1.720830in}{1.681682in}}%
\pgfpathlineto{\pgfqpoint{1.739453in}{1.691759in}}%
\pgfpathlineto{\pgfqpoint{1.758075in}{1.701660in}}%
\pgfpathlineto{\pgfqpoint{1.776697in}{1.711384in}}%
\pgfpathlineto{\pgfqpoint{1.795320in}{1.720935in}}%
\pgfpathlineto{\pgfqpoint{1.813942in}{1.730313in}}%
\pgfpathlineto{\pgfqpoint{1.832564in}{1.739521in}}%
\pgfpathlineto{\pgfqpoint{1.851187in}{1.748560in}}%
\pgfpathlineto{\pgfqpoint{1.869809in}{1.757432in}}%
\pgfpathlineto{\pgfqpoint{1.888431in}{1.766138in}}%
\pgfpathlineto{\pgfqpoint{1.907054in}{1.774680in}}%
\pgfpathlineto{\pgfqpoint{1.925676in}{1.783059in}}%
\pgfpathlineto{\pgfqpoint{1.944298in}{1.791277in}}%
\pgfpathlineto{\pgfqpoint{1.962921in}{1.799335in}}%
\pgfpathlineto{\pgfqpoint{1.981543in}{1.807234in}}%
\pgfpathlineto{\pgfqpoint{2.000165in}{1.814976in}}%
\pgfpathlineto{\pgfqpoint{2.018788in}{1.822562in}}%
\pgfpathlineto{\pgfqpoint{2.037410in}{1.829993in}}%
\pgfpathlineto{\pgfqpoint{2.056033in}{1.837270in}}%
\pgfpathlineto{\pgfqpoint{2.074655in}{1.844395in}}%
\pgfpathlineto{\pgfqpoint{2.093277in}{1.851368in}}%
\pgfpathlineto{\pgfqpoint{2.111900in}{1.858191in}}%
\pgfpathlineto{\pgfqpoint{2.130522in}{1.864865in}}%
\pgfpathlineto{\pgfqpoint{2.149144in}{1.871390in}}%
\pgfpathlineto{\pgfqpoint{2.167767in}{1.877768in}}%
\pgfpathlineto{\pgfqpoint{2.186389in}{1.884000in}}%
\pgfpathlineto{\pgfqpoint{2.205011in}{1.890086in}}%
\pgfpathlineto{\pgfqpoint{2.223634in}{1.896028in}}%
\pgfpathlineto{\pgfqpoint{2.242256in}{1.901826in}}%
\pgfpathlineto{\pgfqpoint{2.260878in}{1.907481in}}%
\pgfpathlineto{\pgfqpoint{2.279501in}{1.912995in}}%
\pgfpathlineto{\pgfqpoint{2.298123in}{1.918367in}}%
\pgfpathlineto{\pgfqpoint{2.316745in}{1.923599in}}%
\pgfpathlineto{\pgfqpoint{2.335368in}{1.928691in}}%
\pgfpathlineto{\pgfqpoint{2.353990in}{1.933644in}}%
\pgfpathlineto{\pgfqpoint{2.372612in}{1.938459in}}%
\pgfpathlineto{\pgfqpoint{2.391235in}{1.943136in}}%
\pgfpathlineto{\pgfqpoint{2.409857in}{1.947677in}}%
\pgfpathlineto{\pgfqpoint{2.428479in}{1.952081in}}%
\pgfpathlineto{\pgfqpoint{2.447102in}{1.956350in}}%
\pgfpathlineto{\pgfqpoint{2.465724in}{1.960483in}}%
\pgfpathlineto{\pgfqpoint{2.484347in}{1.964482in}}%
\pgfpathlineto{\pgfqpoint{2.502969in}{1.968347in}}%
\pgfpathlineto{\pgfqpoint{2.521591in}{1.972079in}}%
\pgfpathlineto{\pgfqpoint{2.540214in}{1.975678in}}%
\pgfpathlineto{\pgfqpoint{2.558836in}{1.979145in}}%
\pgfpathlineto{\pgfqpoint{2.577458in}{1.982479in}}%
\pgfpathlineto{\pgfqpoint{2.596081in}{1.985682in}}%
\pgfpathlineto{\pgfqpoint{2.614703in}{1.988754in}}%
\pgfpathlineto{\pgfqpoint{2.633325in}{1.991696in}}%
\pgfpathlineto{\pgfqpoint{2.651948in}{1.994507in}}%
\pgfpathlineto{\pgfqpoint{2.670570in}{1.997189in}}%
\pgfpathlineto{\pgfqpoint{2.689192in}{1.999741in}}%
\pgfpathlineto{\pgfqpoint{2.707815in}{2.002164in}}%
\pgfpathlineto{\pgfqpoint{2.726437in}{2.004458in}}%
\pgfpathlineto{\pgfqpoint{2.745059in}{2.006624in}}%
\pgfpathlineto{\pgfqpoint{2.763682in}{2.008661in}}%
\pgfpathlineto{\pgfqpoint{2.782304in}{2.010571in}}%
\pgfpathlineto{\pgfqpoint{2.800926in}{2.012353in}}%
\pgfpathlineto{\pgfqpoint{2.819549in}{2.014008in}}%
\pgfpathlineto{\pgfqpoint{2.838171in}{2.015536in}}%
\pgfpathlineto{\pgfqpoint{2.856793in}{2.016936in}}%
\pgfpathlineto{\pgfqpoint{2.875416in}{2.018211in}}%
\pgfpathlineto{\pgfqpoint{2.894038in}{2.019358in}}%
\pgfpathlineto{\pgfqpoint{2.912660in}{2.020379in}}%
\pgfpathlineto{\pgfqpoint{2.931283in}{2.021275in}}%
\pgfpathlineto{\pgfqpoint{2.949905in}{2.022044in}}%
\pgfpathlineto{\pgfqpoint{2.968528in}{2.022687in}}%
\pgfpathlineto{\pgfqpoint{2.987150in}{2.023204in}}%
\pgfpathlineto{\pgfqpoint{3.005772in}{2.023596in}}%
\pgfpathlineto{\pgfqpoint{3.024395in}{2.023862in}}%
\pgfpathlineto{\pgfqpoint{3.043017in}{2.024002in}}%
\pgfpathlineto{\pgfqpoint{3.061639in}{2.024017in}}%
\pgfpathlineto{\pgfqpoint{3.080262in}{2.023906in}}%
\pgfusepath{stroke}%
\end{pgfscope}%
\begin{pgfscope}%
\pgfpathrectangle{\pgfqpoint{0.578349in}{0.682899in}}{\pgfqpoint{4.650000in}{3.020000in}}%
\pgfusepath{clip}%
\pgfsetbuttcap%
\pgfsetroundjoin%
\definecolor{currentfill}{rgb}{0.000000,0.000000,0.000000}%
\pgfsetfillcolor{currentfill}%
\pgfsetfillopacity{0.500000}%
\pgfsetlinewidth{1.003750pt}%
\definecolor{currentstroke}{rgb}{0.000000,0.000000,0.000000}%
\pgfsetstrokecolor{currentstroke}%
\pgfsetstrokeopacity{0.500000}%
\pgfsetdash{}{0pt}%
\pgfsys@defobject{currentmarker}{\pgfqpoint{-0.041667in}{-0.041667in}}{\pgfqpoint{0.041667in}{0.041667in}}{%
\pgfpathmoveto{\pgfqpoint{0.000000in}{-0.041667in}}%
\pgfpathcurveto{\pgfqpoint{0.011050in}{-0.041667in}}{\pgfqpoint{0.021649in}{-0.037276in}}{\pgfqpoint{0.029463in}{-0.029463in}}%
\pgfpathcurveto{\pgfqpoint{0.037276in}{-0.021649in}}{\pgfqpoint{0.041667in}{-0.011050in}}{\pgfqpoint{0.041667in}{0.000000in}}%
\pgfpathcurveto{\pgfqpoint{0.041667in}{0.011050in}}{\pgfqpoint{0.037276in}{0.021649in}}{\pgfqpoint{0.029463in}{0.029463in}}%
\pgfpathcurveto{\pgfqpoint{0.021649in}{0.037276in}}{\pgfqpoint{0.011050in}{0.041667in}}{\pgfqpoint{0.000000in}{0.041667in}}%
\pgfpathcurveto{\pgfqpoint{-0.011050in}{0.041667in}}{\pgfqpoint{-0.021649in}{0.037276in}}{\pgfqpoint{-0.029463in}{0.029463in}}%
\pgfpathcurveto{\pgfqpoint{-0.037276in}{0.021649in}}{\pgfqpoint{-0.041667in}{0.011050in}}{\pgfqpoint{-0.041667in}{0.000000in}}%
\pgfpathcurveto{\pgfqpoint{-0.041667in}{-0.011050in}}{\pgfqpoint{-0.037276in}{-0.021649in}}{\pgfqpoint{-0.029463in}{-0.029463in}}%
\pgfpathcurveto{\pgfqpoint{-0.021649in}{-0.037276in}}{\pgfqpoint{-0.011050in}{-0.041667in}}{\pgfqpoint{0.000000in}{-0.041667in}}%
\pgfpathclose%
\pgfusepath{stroke,fill}%
}%
\begin{pgfscope}%
\pgfsys@transformshift{0.826958in}{0.961276in}%
\pgfsys@useobject{currentmarker}{}%
\end{pgfscope}%
\begin{pgfscope}%
\pgfsys@transformshift{0.845580in}{0.985704in}%
\pgfsys@useobject{currentmarker}{}%
\end{pgfscope}%
\begin{pgfscope}%
\pgfsys@transformshift{0.864202in}{1.009472in}%
\pgfsys@useobject{currentmarker}{}%
\end{pgfscope}%
\begin{pgfscope}%
\pgfsys@transformshift{0.882825in}{1.032590in}%
\pgfsys@useobject{currentmarker}{}%
\end{pgfscope}%
\begin{pgfscope}%
\pgfsys@transformshift{0.901447in}{1.055065in}%
\pgfsys@useobject{currentmarker}{}%
\end{pgfscope}%
\begin{pgfscope}%
\pgfsys@transformshift{0.920069in}{1.076905in}%
\pgfsys@useobject{currentmarker}{}%
\end{pgfscope}%
\begin{pgfscope}%
\pgfsys@transformshift{0.938692in}{1.098118in}%
\pgfsys@useobject{currentmarker}{}%
\end{pgfscope}%
\begin{pgfscope}%
\pgfsys@transformshift{0.957314in}{1.118714in}%
\pgfsys@useobject{currentmarker}{}%
\end{pgfscope}%
\begin{pgfscope}%
\pgfsys@transformshift{0.975936in}{1.138699in}%
\pgfsys@useobject{currentmarker}{}%
\end{pgfscope}%
\begin{pgfscope}%
\pgfsys@transformshift{0.994559in}{1.158082in}%
\pgfsys@useobject{currentmarker}{}%
\end{pgfscope}%
\begin{pgfscope}%
\pgfsys@transformshift{1.013181in}{1.176871in}%
\pgfsys@useobject{currentmarker}{}%
\end{pgfscope}%
\begin{pgfscope}%
\pgfsys@transformshift{1.031803in}{1.195075in}%
\pgfsys@useobject{currentmarker}{}%
\end{pgfscope}%
\begin{pgfscope}%
\pgfsys@transformshift{1.050426in}{1.212701in}%
\pgfsys@useobject{currentmarker}{}%
\end{pgfscope}%
\begin{pgfscope}%
\pgfsys@transformshift{1.069048in}{1.229758in}%
\pgfsys@useobject{currentmarker}{}%
\end{pgfscope}%
\begin{pgfscope}%
\pgfsys@transformshift{1.087670in}{1.246255in}%
\pgfsys@useobject{currentmarker}{}%
\end{pgfscope}%
\begin{pgfscope}%
\pgfsys@transformshift{1.106293in}{1.262199in}%
\pgfsys@useobject{currentmarker}{}%
\end{pgfscope}%
\begin{pgfscope}%
\pgfsys@transformshift{1.124915in}{1.277599in}%
\pgfsys@useobject{currentmarker}{}%
\end{pgfscope}%
\begin{pgfscope}%
\pgfsys@transformshift{1.143538in}{1.292463in}%
\pgfsys@useobject{currentmarker}{}%
\end{pgfscope}%
\begin{pgfscope}%
\pgfsys@transformshift{1.162160in}{1.306799in}%
\pgfsys@useobject{currentmarker}{}%
\end{pgfscope}%
\begin{pgfscope}%
\pgfsys@transformshift{1.180782in}{1.320615in}%
\pgfsys@useobject{currentmarker}{}%
\end{pgfscope}%
\begin{pgfscope}%
\pgfsys@transformshift{1.199405in}{1.333920in}%
\pgfsys@useobject{currentmarker}{}%
\end{pgfscope}%
\begin{pgfscope}%
\pgfsys@transformshift{1.218027in}{1.346721in}%
\pgfsys@useobject{currentmarker}{}%
\end{pgfscope}%
\begin{pgfscope}%
\pgfsys@transformshift{1.236649in}{1.359025in}%
\pgfsys@useobject{currentmarker}{}%
\end{pgfscope}%
\begin{pgfscope}%
\pgfsys@transformshift{1.255272in}{1.370840in}%
\pgfsys@useobject{currentmarker}{}%
\end{pgfscope}%
\begin{pgfscope}%
\pgfsys@transformshift{1.273894in}{1.382172in}%
\pgfsys@useobject{currentmarker}{}%
\end{pgfscope}%
\begin{pgfscope}%
\pgfsys@transformshift{1.292516in}{1.393029in}%
\pgfsys@useobject{currentmarker}{}%
\end{pgfscope}%
\begin{pgfscope}%
\pgfsys@transformshift{1.311139in}{1.403417in}%
\pgfsys@useobject{currentmarker}{}%
\end{pgfscope}%
\begin{pgfscope}%
\pgfsys@transformshift{1.329761in}{1.413342in}%
\pgfsys@useobject{currentmarker}{}%
\end{pgfscope}%
\begin{pgfscope}%
\pgfsys@transformshift{1.348383in}{1.422811in}%
\pgfsys@useobject{currentmarker}{}%
\end{pgfscope}%
\begin{pgfscope}%
\pgfsys@transformshift{1.367006in}{1.431828in}%
\pgfsys@useobject{currentmarker}{}%
\end{pgfscope}%
\begin{pgfscope}%
\pgfsys@transformshift{1.385628in}{1.440398in}%
\pgfsys@useobject{currentmarker}{}%
\end{pgfscope}%
\begin{pgfscope}%
\pgfsys@transformshift{1.404250in}{1.448527in}%
\pgfsys@useobject{currentmarker}{}%
\end{pgfscope}%
\begin{pgfscope}%
\pgfsys@transformshift{1.422873in}{1.456218in}%
\pgfsys@useobject{currentmarker}{}%
\end{pgfscope}%
\begin{pgfscope}%
\pgfsys@transformshift{1.441495in}{1.463475in}%
\pgfsys@useobject{currentmarker}{}%
\end{pgfscope}%
\begin{pgfscope}%
\pgfsys@transformshift{1.460117in}{1.470302in}%
\pgfsys@useobject{currentmarker}{}%
\end{pgfscope}%
\begin{pgfscope}%
\pgfsys@transformshift{1.478740in}{1.476702in}%
\pgfsys@useobject{currentmarker}{}%
\end{pgfscope}%
\begin{pgfscope}%
\pgfsys@transformshift{1.497362in}{1.482678in}%
\pgfsys@useobject{currentmarker}{}%
\end{pgfscope}%
\begin{pgfscope}%
\pgfsys@transformshift{1.515984in}{1.488233in}%
\pgfsys@useobject{currentmarker}{}%
\end{pgfscope}%
\begin{pgfscope}%
\pgfsys@transformshift{1.534607in}{1.493372in}%
\pgfsys@useobject{currentmarker}{}%
\end{pgfscope}%
\begin{pgfscope}%
\pgfsys@transformshift{1.553229in}{1.498095in}%
\pgfsys@useobject{currentmarker}{}%
\end{pgfscope}%
\begin{pgfscope}%
\pgfsys@transformshift{1.571852in}{1.502405in}%
\pgfsys@useobject{currentmarker}{}%
\end{pgfscope}%
\begin{pgfscope}%
\pgfsys@transformshift{1.590474in}{1.506306in}%
\pgfsys@useobject{currentmarker}{}%
\end{pgfscope}%
\begin{pgfscope}%
\pgfsys@transformshift{1.609096in}{1.509799in}%
\pgfsys@useobject{currentmarker}{}%
\end{pgfscope}%
\begin{pgfscope}%
\pgfsys@transformshift{1.627719in}{1.512886in}%
\pgfsys@useobject{currentmarker}{}%
\end{pgfscope}%
\begin{pgfscope}%
\pgfsys@transformshift{1.646341in}{1.515568in}%
\pgfsys@useobject{currentmarker}{}%
\end{pgfscope}%
\begin{pgfscope}%
\pgfsys@transformshift{1.664963in}{1.517848in}%
\pgfsys@useobject{currentmarker}{}%
\end{pgfscope}%
\begin{pgfscope}%
\pgfsys@transformshift{1.683586in}{1.519726in}%
\pgfsys@useobject{currentmarker}{}%
\end{pgfscope}%
\begin{pgfscope}%
\pgfsys@transformshift{1.702208in}{1.521204in}%
\pgfsys@useobject{currentmarker}{}%
\end{pgfscope}%
\begin{pgfscope}%
\pgfsys@transformshift{1.720830in}{1.522283in}%
\pgfsys@useobject{currentmarker}{}%
\end{pgfscope}%
\begin{pgfscope}%
\pgfsys@transformshift{1.739453in}{1.522963in}%
\pgfsys@useobject{currentmarker}{}%
\end{pgfscope}%
\end{pgfscope}%
\begin{pgfscope}%
\pgfpathrectangle{\pgfqpoint{0.578349in}{0.682899in}}{\pgfqpoint{4.650000in}{3.020000in}}%
\pgfusepath{clip}%
\pgfsetrectcap%
\pgfsetroundjoin%
\pgfsetlinewidth{1.505625pt}%
\definecolor{currentstroke}{rgb}{1.000000,0.000000,0.000000}%
\pgfsetstrokecolor{currentstroke}%
\pgfsetdash{}{0pt}%
\pgfpathmoveto{\pgfqpoint{0.826958in}{0.961276in}}%
\pgfpathlineto{\pgfqpoint{0.845580in}{0.986102in}}%
\pgfpathlineto{\pgfqpoint{0.864202in}{1.010139in}}%
\pgfpathlineto{\pgfqpoint{0.882825in}{1.033423in}}%
\pgfpathlineto{\pgfqpoint{0.901447in}{1.055982in}}%
\pgfpathlineto{\pgfqpoint{0.920069in}{1.077843in}}%
\pgfpathlineto{\pgfqpoint{0.938692in}{1.099028in}}%
\pgfpathlineto{\pgfqpoint{0.957314in}{1.119558in}}%
\pgfpathlineto{\pgfqpoint{0.975936in}{1.139451in}}%
\pgfpathlineto{\pgfqpoint{0.994559in}{1.158724in}}%
\pgfpathlineto{\pgfqpoint{1.013181in}{1.177392in}}%
\pgfpathlineto{\pgfqpoint{1.031803in}{1.195468in}}%
\pgfpathlineto{\pgfqpoint{1.050426in}{1.212967in}}%
\pgfpathlineto{\pgfqpoint{1.069048in}{1.229900in}}%
\pgfpathlineto{\pgfqpoint{1.087670in}{1.246279in}}%
\pgfpathlineto{\pgfqpoint{1.106293in}{1.262114in}}%
\pgfpathlineto{\pgfqpoint{1.124915in}{1.277415in}}%
\pgfpathlineto{\pgfqpoint{1.143538in}{1.292192in}}%
\pgfpathlineto{\pgfqpoint{1.162160in}{1.306453in}}%
\pgfpathlineto{\pgfqpoint{1.180782in}{1.320207in}}%
\pgfpathlineto{\pgfqpoint{1.199405in}{1.333461in}}%
\pgfpathlineto{\pgfqpoint{1.218027in}{1.346223in}}%
\pgfpathlineto{\pgfqpoint{1.236649in}{1.358500in}}%
\pgfpathlineto{\pgfqpoint{1.255272in}{1.370298in}}%
\pgfpathlineto{\pgfqpoint{1.273894in}{1.381624in}}%
\pgfpathlineto{\pgfqpoint{1.292516in}{1.392483in}}%
\pgfpathlineto{\pgfqpoint{1.311139in}{1.402881in}}%
\pgfpathlineto{\pgfqpoint{1.329761in}{1.412823in}}%
\pgfpathlineto{\pgfqpoint{1.348383in}{1.422314in}}%
\pgfpathlineto{\pgfqpoint{1.367006in}{1.431359in}}%
\pgfpathlineto{\pgfqpoint{1.385628in}{1.439961in}}%
\pgfpathlineto{\pgfqpoint{1.404250in}{1.448126in}}%
\pgfpathlineto{\pgfqpoint{1.422873in}{1.455857in}}%
\pgfpathlineto{\pgfqpoint{1.441495in}{1.463158in}}%
\pgfpathlineto{\pgfqpoint{1.460117in}{1.470031in}}%
\pgfpathlineto{\pgfqpoint{1.478740in}{1.476481in}}%
\pgfpathlineto{\pgfqpoint{1.497362in}{1.482510in}}%
\pgfpathlineto{\pgfqpoint{1.515984in}{1.488121in}}%
\pgfpathlineto{\pgfqpoint{1.534607in}{1.493317in}}%
\pgfpathlineto{\pgfqpoint{1.553229in}{1.498099in}}%
\pgfpathlineto{\pgfqpoint{1.571852in}{1.502470in}}%
\pgfpathlineto{\pgfqpoint{1.590474in}{1.506432in}}%
\pgfpathlineto{\pgfqpoint{1.609096in}{1.509987in}}%
\pgfpathlineto{\pgfqpoint{1.627719in}{1.513136in}}%
\pgfpathlineto{\pgfqpoint{1.646341in}{1.515881in}}%
\pgfpathlineto{\pgfqpoint{1.664963in}{1.518223in}}%
\pgfpathlineto{\pgfqpoint{1.683586in}{1.520163in}}%
\pgfpathlineto{\pgfqpoint{1.702208in}{1.521702in}}%
\pgfpathlineto{\pgfqpoint{1.720830in}{1.522840in}}%
\pgfpathlineto{\pgfqpoint{1.739453in}{1.523579in}}%
\pgfusepath{stroke}%
\end{pgfscope}%
\begin{pgfscope}%
\pgfpathrectangle{\pgfqpoint{0.578349in}{0.682899in}}{\pgfqpoint{4.650000in}{3.020000in}}%
\pgfusepath{clip}%
\pgfsetbuttcap%
\pgfsetroundjoin%
\definecolor{currentfill}{rgb}{0.000000,0.000000,0.000000}%
\pgfsetfillcolor{currentfill}%
\pgfsetfillopacity{0.500000}%
\pgfsetlinewidth{1.003750pt}%
\definecolor{currentstroke}{rgb}{0.000000,0.000000,0.000000}%
\pgfsetstrokecolor{currentstroke}%
\pgfsetstrokeopacity{0.500000}%
\pgfsetdash{}{0pt}%
\pgfsys@defobject{currentmarker}{\pgfqpoint{-0.041667in}{-0.041667in}}{\pgfqpoint{0.041667in}{0.041667in}}{%
\pgfpathmoveto{\pgfqpoint{0.000000in}{-0.041667in}}%
\pgfpathcurveto{\pgfqpoint{0.011050in}{-0.041667in}}{\pgfqpoint{0.021649in}{-0.037276in}}{\pgfqpoint{0.029463in}{-0.029463in}}%
\pgfpathcurveto{\pgfqpoint{0.037276in}{-0.021649in}}{\pgfqpoint{0.041667in}{-0.011050in}}{\pgfqpoint{0.041667in}{0.000000in}}%
\pgfpathcurveto{\pgfqpoint{0.041667in}{0.011050in}}{\pgfqpoint{0.037276in}{0.021649in}}{\pgfqpoint{0.029463in}{0.029463in}}%
\pgfpathcurveto{\pgfqpoint{0.021649in}{0.037276in}}{\pgfqpoint{0.011050in}{0.041667in}}{\pgfqpoint{0.000000in}{0.041667in}}%
\pgfpathcurveto{\pgfqpoint{-0.011050in}{0.041667in}}{\pgfqpoint{-0.021649in}{0.037276in}}{\pgfqpoint{-0.029463in}{0.029463in}}%
\pgfpathcurveto{\pgfqpoint{-0.037276in}{0.021649in}}{\pgfqpoint{-0.041667in}{0.011050in}}{\pgfqpoint{-0.041667in}{0.000000in}}%
\pgfpathcurveto{\pgfqpoint{-0.041667in}{-0.011050in}}{\pgfqpoint{-0.037276in}{-0.021649in}}{\pgfqpoint{-0.029463in}{-0.029463in}}%
\pgfpathcurveto{\pgfqpoint{-0.021649in}{-0.037276in}}{\pgfqpoint{-0.011050in}{-0.041667in}}{\pgfqpoint{0.000000in}{-0.041667in}}%
\pgfpathclose%
\pgfusepath{stroke,fill}%
}%
\begin{pgfscope}%
\pgfsys@transformshift{0.845580in}{0.911621in}%
\pgfsys@useobject{currentmarker}{}%
\end{pgfscope}%
\begin{pgfscope}%
\pgfsys@transformshift{0.864202in}{0.931383in}%
\pgfsys@useobject{currentmarker}{}%
\end{pgfscope}%
\begin{pgfscope}%
\pgfsys@transformshift{0.882825in}{0.950279in}%
\pgfsys@useobject{currentmarker}{}%
\end{pgfscope}%
\begin{pgfscope}%
\pgfsys@transformshift{0.901447in}{0.968319in}%
\pgfsys@useobject{currentmarker}{}%
\end{pgfscope}%
\begin{pgfscope}%
\pgfsys@transformshift{0.920069in}{0.985511in}%
\pgfsys@useobject{currentmarker}{}%
\end{pgfscope}%
\begin{pgfscope}%
\pgfsys@transformshift{0.938692in}{1.001865in}%
\pgfsys@useobject{currentmarker}{}%
\end{pgfscope}%
\begin{pgfscope}%
\pgfsys@transformshift{0.957314in}{1.017390in}%
\pgfsys@useobject{currentmarker}{}%
\end{pgfscope}%
\begin{pgfscope}%
\pgfsys@transformshift{0.975936in}{1.032094in}%
\pgfsys@useobject{currentmarker}{}%
\end{pgfscope}%
\begin{pgfscope}%
\pgfsys@transformshift{0.994559in}{1.045987in}%
\pgfsys@useobject{currentmarker}{}%
\end{pgfscope}%
\begin{pgfscope}%
\pgfsys@transformshift{1.013181in}{1.059078in}%
\pgfsys@useobject{currentmarker}{}%
\end{pgfscope}%
\begin{pgfscope}%
\pgfsys@transformshift{1.031803in}{1.071375in}%
\pgfsys@useobject{currentmarker}{}%
\end{pgfscope}%
\begin{pgfscope}%
\pgfsys@transformshift{1.050426in}{1.082888in}%
\pgfsys@useobject{currentmarker}{}%
\end{pgfscope}%
\begin{pgfscope}%
\pgfsys@transformshift{1.069048in}{1.093625in}%
\pgfsys@useobject{currentmarker}{}%
\end{pgfscope}%
\begin{pgfscope}%
\pgfsys@transformshift{1.087670in}{1.103600in}%
\pgfsys@useobject{currentmarker}{}%
\end{pgfscope}%
\begin{pgfscope}%
\pgfsys@transformshift{1.106293in}{1.112822in}%
\pgfsys@useobject{currentmarker}{}%
\end{pgfscope}%
\begin{pgfscope}%
\pgfsys@transformshift{1.124915in}{1.121297in}%
\pgfsys@useobject{currentmarker}{}%
\end{pgfscope}%
\begin{pgfscope}%
\pgfsys@transformshift{1.143538in}{1.129036in}%
\pgfsys@useobject{currentmarker}{}%
\end{pgfscope}%
\begin{pgfscope}%
\pgfsys@transformshift{1.162160in}{1.136047in}%
\pgfsys@useobject{currentmarker}{}%
\end{pgfscope}%
\begin{pgfscope}%
\pgfsys@transformshift{1.180782in}{1.142338in}%
\pgfsys@useobject{currentmarker}{}%
\end{pgfscope}%
\begin{pgfscope}%
\pgfsys@transformshift{1.199405in}{1.147916in}%
\pgfsys@useobject{currentmarker}{}%
\end{pgfscope}%
\begin{pgfscope}%
\pgfsys@transformshift{1.218027in}{1.152787in}%
\pgfsys@useobject{currentmarker}{}%
\end{pgfscope}%
\begin{pgfscope}%
\pgfsys@transformshift{1.236649in}{1.156958in}%
\pgfsys@useobject{currentmarker}{}%
\end{pgfscope}%
\begin{pgfscope}%
\pgfsys@transformshift{1.255272in}{1.160434in}%
\pgfsys@useobject{currentmarker}{}%
\end{pgfscope}%
\begin{pgfscope}%
\pgfsys@transformshift{1.273894in}{1.163218in}%
\pgfsys@useobject{currentmarker}{}%
\end{pgfscope}%
\begin{pgfscope}%
\pgfsys@transformshift{1.292516in}{1.165314in}%
\pgfsys@useobject{currentmarker}{}%
\end{pgfscope}%
\begin{pgfscope}%
\pgfsys@transformshift{1.311139in}{1.166725in}%
\pgfsys@useobject{currentmarker}{}%
\end{pgfscope}%
\begin{pgfscope}%
\pgfsys@transformshift{1.329761in}{1.167452in}%
\pgfsys@useobject{currentmarker}{}%
\end{pgfscope}%
\end{pgfscope}%
\begin{pgfscope}%
\pgfpathrectangle{\pgfqpoint{0.578349in}{0.682899in}}{\pgfqpoint{4.650000in}{3.020000in}}%
\pgfusepath{clip}%
\pgfsetrectcap%
\pgfsetroundjoin%
\pgfsetlinewidth{1.505625pt}%
\definecolor{currentstroke}{rgb}{1.000000,0.000000,0.000000}%
\pgfsetstrokecolor{currentstroke}%
\pgfsetdash{}{0pt}%
\pgfpathmoveto{\pgfqpoint{0.845580in}{0.911621in}}%
\pgfpathlineto{\pgfqpoint{0.864202in}{0.931620in}}%
\pgfpathlineto{\pgfqpoint{0.882825in}{0.950644in}}%
\pgfpathlineto{\pgfqpoint{0.901447in}{0.968732in}}%
\pgfpathlineto{\pgfqpoint{0.920069in}{0.985914in}}%
\pgfpathlineto{\pgfqpoint{0.938692in}{1.002218in}}%
\pgfpathlineto{\pgfqpoint{0.957314in}{1.017669in}}%
\pgfpathlineto{\pgfqpoint{0.975936in}{1.032285in}}%
\pgfpathlineto{\pgfqpoint{0.994559in}{1.046086in}}%
\pgfpathlineto{\pgfqpoint{1.013181in}{1.059088in}}%
\pgfpathlineto{\pgfqpoint{1.031803in}{1.071305in}}%
\pgfpathlineto{\pgfqpoint{1.050426in}{1.082751in}}%
\pgfpathlineto{\pgfqpoint{1.069048in}{1.093438in}}%
\pgfpathlineto{\pgfqpoint{1.087670in}{1.103376in}}%
\pgfpathlineto{\pgfqpoint{1.106293in}{1.112575in}}%
\pgfpathlineto{\pgfqpoint{1.124915in}{1.121044in}}%
\pgfpathlineto{\pgfqpoint{1.143538in}{1.128791in}}%
\pgfpathlineto{\pgfqpoint{1.162160in}{1.135822in}}%
\pgfpathlineto{\pgfqpoint{1.180782in}{1.142145in}}%
\pgfpathlineto{\pgfqpoint{1.199405in}{1.147765in}}%
\pgfpathlineto{\pgfqpoint{1.218027in}{1.152687in}}%
\pgfpathlineto{\pgfqpoint{1.236649in}{1.156915in}}%
\pgfpathlineto{\pgfqpoint{1.255272in}{1.160453in}}%
\pgfpathlineto{\pgfqpoint{1.273894in}{1.163305in}}%
\pgfpathlineto{\pgfqpoint{1.292516in}{1.165472in}}%
\pgfpathlineto{\pgfqpoint{1.311139in}{1.166956in}}%
\pgfpathlineto{\pgfqpoint{1.329761in}{1.167760in}}%
\pgfusepath{stroke}%
\end{pgfscope}%
\begin{pgfscope}%
\pgfpathrectangle{\pgfqpoint{0.578349in}{0.682899in}}{\pgfqpoint{4.650000in}{3.020000in}}%
\pgfusepath{clip}%
\pgfsetbuttcap%
\pgfsetroundjoin%
\definecolor{currentfill}{rgb}{0.000000,0.000000,0.000000}%
\pgfsetfillcolor{currentfill}%
\pgfsetfillopacity{0.500000}%
\pgfsetlinewidth{1.003750pt}%
\definecolor{currentstroke}{rgb}{0.000000,0.000000,0.000000}%
\pgfsetstrokecolor{currentstroke}%
\pgfsetstrokeopacity{0.500000}%
\pgfsetdash{}{0pt}%
\pgfsys@defobject{currentmarker}{\pgfqpoint{-0.041667in}{-0.041667in}}{\pgfqpoint{0.041667in}{0.041667in}}{%
\pgfpathmoveto{\pgfqpoint{0.000000in}{-0.041667in}}%
\pgfpathcurveto{\pgfqpoint{0.011050in}{-0.041667in}}{\pgfqpoint{0.021649in}{-0.037276in}}{\pgfqpoint{0.029463in}{-0.029463in}}%
\pgfpathcurveto{\pgfqpoint{0.037276in}{-0.021649in}}{\pgfqpoint{0.041667in}{-0.011050in}}{\pgfqpoint{0.041667in}{0.000000in}}%
\pgfpathcurveto{\pgfqpoint{0.041667in}{0.011050in}}{\pgfqpoint{0.037276in}{0.021649in}}{\pgfqpoint{0.029463in}{0.029463in}}%
\pgfpathcurveto{\pgfqpoint{0.021649in}{0.037276in}}{\pgfqpoint{0.011050in}{0.041667in}}{\pgfqpoint{0.000000in}{0.041667in}}%
\pgfpathcurveto{\pgfqpoint{-0.011050in}{0.041667in}}{\pgfqpoint{-0.021649in}{0.037276in}}{\pgfqpoint{-0.029463in}{0.029463in}}%
\pgfpathcurveto{\pgfqpoint{-0.037276in}{0.021649in}}{\pgfqpoint{-0.041667in}{0.011050in}}{\pgfqpoint{-0.041667in}{0.000000in}}%
\pgfpathcurveto{\pgfqpoint{-0.041667in}{-0.011050in}}{\pgfqpoint{-0.037276in}{-0.021649in}}{\pgfqpoint{-0.029463in}{-0.029463in}}%
\pgfpathcurveto{\pgfqpoint{-0.021649in}{-0.037276in}}{\pgfqpoint{-0.011050in}{-0.041667in}}{\pgfqpoint{0.000000in}{-0.041667in}}%
\pgfpathclose%
\pgfusepath{stroke,fill}%
}%
\begin{pgfscope}%
\pgfsys@transformshift{0.826958in}{0.873605in}%
\pgfsys@useobject{currentmarker}{}%
\end{pgfscope}%
\begin{pgfscope}%
\pgfsys@transformshift{0.845580in}{0.887911in}%
\pgfsys@useobject{currentmarker}{}%
\end{pgfscope}%
\begin{pgfscope}%
\pgfsys@transformshift{0.864202in}{0.901141in}%
\pgfsys@useobject{currentmarker}{}%
\end{pgfscope}%
\begin{pgfscope}%
\pgfsys@transformshift{0.882825in}{0.913294in}%
\pgfsys@useobject{currentmarker}{}%
\end{pgfscope}%
\begin{pgfscope}%
\pgfsys@transformshift{0.901447in}{0.924370in}%
\pgfsys@useobject{currentmarker}{}%
\end{pgfscope}%
\begin{pgfscope}%
\pgfsys@transformshift{0.920069in}{0.934368in}%
\pgfsys@useobject{currentmarker}{}%
\end{pgfscope}%
\begin{pgfscope}%
\pgfsys@transformshift{0.938692in}{0.943288in}%
\pgfsys@useobject{currentmarker}{}%
\end{pgfscope}%
\begin{pgfscope}%
\pgfsys@transformshift{0.957314in}{0.951130in}%
\pgfsys@useobject{currentmarker}{}%
\end{pgfscope}%
\begin{pgfscope}%
\pgfsys@transformshift{0.975936in}{0.957892in}%
\pgfsys@useobject{currentmarker}{}%
\end{pgfscope}%
\begin{pgfscope}%
\pgfsys@transformshift{0.994559in}{0.963574in}%
\pgfsys@useobject{currentmarker}{}%
\end{pgfscope}%
\begin{pgfscope}%
\pgfsys@transformshift{1.013181in}{0.968176in}%
\pgfsys@useobject{currentmarker}{}%
\end{pgfscope}%
\begin{pgfscope}%
\pgfsys@transformshift{1.031803in}{0.971697in}%
\pgfsys@useobject{currentmarker}{}%
\end{pgfscope}%
\begin{pgfscope}%
\pgfsys@transformshift{1.050426in}{0.974137in}%
\pgfsys@useobject{currentmarker}{}%
\end{pgfscope}%
\begin{pgfscope}%
\pgfsys@transformshift{1.069048in}{0.975505in}%
\pgfsys@useobject{currentmarker}{}%
\end{pgfscope}%
\end{pgfscope}%
\begin{pgfscope}%
\pgfpathrectangle{\pgfqpoint{0.578349in}{0.682899in}}{\pgfqpoint{4.650000in}{3.020000in}}%
\pgfusepath{clip}%
\pgfsetrectcap%
\pgfsetroundjoin%
\pgfsetlinewidth{1.505625pt}%
\definecolor{currentstroke}{rgb}{1.000000,0.000000,0.000000}%
\pgfsetstrokecolor{currentstroke}%
\pgfsetdash{}{0pt}%
\pgfpathmoveto{\pgfqpoint{0.826958in}{0.873605in}}%
\pgfpathlineto{\pgfqpoint{0.845580in}{0.888144in}}%
\pgfpathlineto{\pgfqpoint{0.864202in}{0.901453in}}%
\pgfpathlineto{\pgfqpoint{0.882825in}{0.913581in}}%
\pgfpathlineto{\pgfqpoint{0.901447in}{0.924568in}}%
\pgfpathlineto{\pgfqpoint{0.920069in}{0.934444in}}%
\pgfpathlineto{\pgfqpoint{0.938692in}{0.943237in}}%
\pgfpathlineto{\pgfqpoint{0.957314in}{0.950968in}}%
\pgfpathlineto{\pgfqpoint{0.975936in}{0.957656in}}%
\pgfpathlineto{\pgfqpoint{0.994559in}{0.963314in}}%
\pgfpathlineto{\pgfqpoint{1.013181in}{0.967956in}}%
\pgfpathlineto{\pgfqpoint{1.031803in}{0.971591in}}%
\pgfpathlineto{\pgfqpoint{1.050426in}{0.974226in}}%
\pgfpathlineto{\pgfqpoint{1.069048in}{0.975867in}}%
\pgfusepath{stroke}%
\end{pgfscope}%
\begin{pgfscope}%
\pgfpathrectangle{\pgfqpoint{0.578349in}{0.682899in}}{\pgfqpoint{4.650000in}{3.020000in}}%
\pgfusepath{clip}%
\pgfsetbuttcap%
\pgfsetroundjoin%
\definecolor{currentfill}{rgb}{0.000000,0.000000,0.000000}%
\pgfsetfillcolor{currentfill}%
\pgfsetfillopacity{0.500000}%
\pgfsetlinewidth{1.003750pt}%
\definecolor{currentstroke}{rgb}{0.000000,0.000000,0.000000}%
\pgfsetstrokecolor{currentstroke}%
\pgfsetstrokeopacity{0.500000}%
\pgfsetdash{}{0pt}%
\pgfsys@defobject{currentmarker}{\pgfqpoint{-0.041667in}{-0.041667in}}{\pgfqpoint{0.041667in}{0.041667in}}{%
\pgfpathmoveto{\pgfqpoint{0.000000in}{-0.041667in}}%
\pgfpathcurveto{\pgfqpoint{0.011050in}{-0.041667in}}{\pgfqpoint{0.021649in}{-0.037276in}}{\pgfqpoint{0.029463in}{-0.029463in}}%
\pgfpathcurveto{\pgfqpoint{0.037276in}{-0.021649in}}{\pgfqpoint{0.041667in}{-0.011050in}}{\pgfqpoint{0.041667in}{0.000000in}}%
\pgfpathcurveto{\pgfqpoint{0.041667in}{0.011050in}}{\pgfqpoint{0.037276in}{0.021649in}}{\pgfqpoint{0.029463in}{0.029463in}}%
\pgfpathcurveto{\pgfqpoint{0.021649in}{0.037276in}}{\pgfqpoint{0.011050in}{0.041667in}}{\pgfqpoint{0.000000in}{0.041667in}}%
\pgfpathcurveto{\pgfqpoint{-0.011050in}{0.041667in}}{\pgfqpoint{-0.021649in}{0.037276in}}{\pgfqpoint{-0.029463in}{0.029463in}}%
\pgfpathcurveto{\pgfqpoint{-0.037276in}{0.021649in}}{\pgfqpoint{-0.041667in}{0.011050in}}{\pgfqpoint{-0.041667in}{0.000000in}}%
\pgfpathcurveto{\pgfqpoint{-0.041667in}{-0.011050in}}{\pgfqpoint{-0.037276in}{-0.021649in}}{\pgfqpoint{-0.029463in}{-0.029463in}}%
\pgfpathcurveto{\pgfqpoint{-0.021649in}{-0.037276in}}{\pgfqpoint{-0.011050in}{-0.041667in}}{\pgfqpoint{0.000000in}{-0.041667in}}%
\pgfpathclose%
\pgfusepath{stroke,fill}%
}%
\begin{pgfscope}%
\pgfsys@transformshift{0.845580in}{0.820172in}%
\pgfsys@useobject{currentmarker}{}%
\end{pgfscope}%
\begin{pgfscope}%
\pgfsys@transformshift{0.864202in}{0.825861in}%
\pgfsys@useobject{currentmarker}{}%
\end{pgfscope}%
\begin{pgfscope}%
\pgfsys@transformshift{0.882825in}{0.831149in}%
\pgfsys@useobject{currentmarker}{}%
\end{pgfscope}%
\begin{pgfscope}%
\pgfsys@transformshift{0.901447in}{0.836028in}%
\pgfsys@useobject{currentmarker}{}%
\end{pgfscope}%
\begin{pgfscope}%
\pgfsys@transformshift{0.920069in}{0.840488in}%
\pgfsys@useobject{currentmarker}{}%
\end{pgfscope}%
\begin{pgfscope}%
\pgfsys@transformshift{0.938692in}{0.844522in}%
\pgfsys@useobject{currentmarker}{}%
\end{pgfscope}%
\begin{pgfscope}%
\pgfsys@transformshift{0.957314in}{0.848121in}%
\pgfsys@useobject{currentmarker}{}%
\end{pgfscope}%
\begin{pgfscope}%
\pgfsys@transformshift{0.975936in}{0.851277in}%
\pgfsys@useobject{currentmarker}{}%
\end{pgfscope}%
\begin{pgfscope}%
\pgfsys@transformshift{0.994559in}{0.853983in}%
\pgfsys@useobject{currentmarker}{}%
\end{pgfscope}%
\begin{pgfscope}%
\pgfsys@transformshift{1.013181in}{0.856229in}%
\pgfsys@useobject{currentmarker}{}%
\end{pgfscope}%
\begin{pgfscope}%
\pgfsys@transformshift{1.031803in}{0.858007in}%
\pgfsys@useobject{currentmarker}{}%
\end{pgfscope}%
\begin{pgfscope}%
\pgfsys@transformshift{1.050426in}{0.859309in}%
\pgfsys@useobject{currentmarker}{}%
\end{pgfscope}%
\begin{pgfscope}%
\pgfsys@transformshift{1.069048in}{0.860128in}%
\pgfsys@useobject{currentmarker}{}%
\end{pgfscope}%
\end{pgfscope}%
\begin{pgfscope}%
\pgfpathrectangle{\pgfqpoint{0.578349in}{0.682899in}}{\pgfqpoint{4.650000in}{3.020000in}}%
\pgfusepath{clip}%
\pgfsetrectcap%
\pgfsetroundjoin%
\pgfsetlinewidth{1.505625pt}%
\definecolor{currentstroke}{rgb}{1.000000,0.000000,0.000000}%
\pgfsetstrokecolor{currentstroke}%
\pgfsetdash{}{0pt}%
\pgfpathmoveto{\pgfqpoint{0.845580in}{0.820172in}}%
\pgfpathlineto{\pgfqpoint{0.864202in}{0.825956in}}%
\pgfpathlineto{\pgfqpoint{0.882825in}{0.831278in}}%
\pgfpathlineto{\pgfqpoint{0.901447in}{0.836147in}}%
\pgfpathlineto{\pgfqpoint{0.920069in}{0.840568in}}%
\pgfpathlineto{\pgfqpoint{0.938692in}{0.844545in}}%
\pgfpathlineto{\pgfqpoint{0.957314in}{0.848085in}}%
\pgfpathlineto{\pgfqpoint{0.975936in}{0.851190in}}%
\pgfpathlineto{\pgfqpoint{0.994559in}{0.853864in}}%
\pgfpathlineto{\pgfqpoint{1.013181in}{0.856111in}}%
\pgfpathlineto{\pgfqpoint{1.031803in}{0.857932in}}%
\pgfpathlineto{\pgfqpoint{1.050426in}{0.859331in}}%
\pgfpathlineto{\pgfqpoint{1.069048in}{0.860308in}}%
\pgfusepath{stroke}%
\end{pgfscope}%
\begin{pgfscope}%
\pgfsetrectcap%
\pgfsetmiterjoin%
\pgfsetlinewidth{0.803000pt}%
\definecolor{currentstroke}{rgb}{0.501961,0.501961,0.501961}%
\pgfsetstrokecolor{currentstroke}%
\pgfsetdash{}{0pt}%
\pgfpathmoveto{\pgfqpoint{0.578349in}{0.682899in}}%
\pgfpathlineto{\pgfqpoint{0.578349in}{3.702899in}}%
\pgfusepath{stroke}%
\end{pgfscope}%
\begin{pgfscope}%
\pgfsetrectcap%
\pgfsetmiterjoin%
\pgfsetlinewidth{0.803000pt}%
\definecolor{currentstroke}{rgb}{0.501961,0.501961,0.501961}%
\pgfsetstrokecolor{currentstroke}%
\pgfsetdash{}{0pt}%
\pgfpathmoveto{\pgfqpoint{5.228349in}{0.682899in}}%
\pgfpathlineto{\pgfqpoint{5.228349in}{3.702899in}}%
\pgfusepath{stroke}%
\end{pgfscope}%
\begin{pgfscope}%
\pgfsetrectcap%
\pgfsetmiterjoin%
\pgfsetlinewidth{0.803000pt}%
\definecolor{currentstroke}{rgb}{0.501961,0.501961,0.501961}%
\pgfsetstrokecolor{currentstroke}%
\pgfsetdash{}{0pt}%
\pgfpathmoveto{\pgfqpoint{0.578349in}{0.682899in}}%
\pgfpathlineto{\pgfqpoint{5.228349in}{0.682899in}}%
\pgfusepath{stroke}%
\end{pgfscope}%
\begin{pgfscope}%
\pgfsetrectcap%
\pgfsetmiterjoin%
\pgfsetlinewidth{0.803000pt}%
\definecolor{currentstroke}{rgb}{0.501961,0.501961,0.501961}%
\pgfsetstrokecolor{currentstroke}%
\pgfsetdash{}{0pt}%
\pgfpathmoveto{\pgfqpoint{0.578349in}{3.702899in}}%
\pgfpathlineto{\pgfqpoint{5.228349in}{3.702899in}}%
\pgfusepath{stroke}%
\end{pgfscope}%
\begin{pgfscope}%
\pgfsetbuttcap%
\pgfsetmiterjoin%
\definecolor{currentfill}{rgb}{1.000000,1.000000,1.000000}%
\pgfsetfillcolor{currentfill}%
\pgfsetfillopacity{0.800000}%
\pgfsetlinewidth{0.000000pt}%
\definecolor{currentstroke}{rgb}{0.800000,0.800000,0.800000}%
\pgfsetstrokecolor{currentstroke}%
\pgfsetstrokeopacity{0.800000}%
\pgfsetdash{}{0pt}%
\pgfpathmoveto{\pgfqpoint{3.079131in}{0.794010in}}%
\pgfpathlineto{\pgfqpoint{5.072794in}{0.794010in}}%
\pgfpathquadraticcurveto{\pgfqpoint{5.117238in}{0.794010in}}{\pgfqpoint{5.117238in}{0.838454in}}%
\pgfpathlineto{\pgfqpoint{5.117238in}{1.468576in}}%
\pgfpathquadraticcurveto{\pgfqpoint{5.117238in}{1.513020in}}{\pgfqpoint{5.072794in}{1.513020in}}%
\pgfpathlineto{\pgfqpoint{3.079131in}{1.513020in}}%
\pgfpathquadraticcurveto{\pgfqpoint{3.034686in}{1.513020in}}{\pgfqpoint{3.034686in}{1.468576in}}%
\pgfpathlineto{\pgfqpoint{3.034686in}{0.838454in}}%
\pgfpathquadraticcurveto{\pgfqpoint{3.034686in}{0.794010in}}{\pgfqpoint{3.079131in}{0.794010in}}%
\pgfpathclose%
\pgfusepath{fill}%
\end{pgfscope}%
\begin{pgfscope}%
\pgfsetbuttcap%
\pgfsetroundjoin%
\definecolor{currentfill}{rgb}{0.000000,0.000000,0.000000}%
\pgfsetfillcolor{currentfill}%
\pgfsetfillopacity{0.000000}%
\pgfsetlinewidth{1.003750pt}%
\definecolor{currentstroke}{rgb}{0.000000,0.000000,0.000000}%
\pgfsetstrokecolor{currentstroke}%
\pgfsetdash{}{0pt}%
\pgfsys@defobject{currentmarker}{\pgfqpoint{-0.041667in}{-0.041667in}}{\pgfqpoint{0.041667in}{0.041667in}}{%
\pgfpathmoveto{\pgfqpoint{0.000000in}{-0.041667in}}%
\pgfpathcurveto{\pgfqpoint{0.011050in}{-0.041667in}}{\pgfqpoint{0.021649in}{-0.037276in}}{\pgfqpoint{0.029463in}{-0.029463in}}%
\pgfpathcurveto{\pgfqpoint{0.037276in}{-0.021649in}}{\pgfqpoint{0.041667in}{-0.011050in}}{\pgfqpoint{0.041667in}{0.000000in}}%
\pgfpathcurveto{\pgfqpoint{0.041667in}{0.011050in}}{\pgfqpoint{0.037276in}{0.021649in}}{\pgfqpoint{0.029463in}{0.029463in}}%
\pgfpathcurveto{\pgfqpoint{0.021649in}{0.037276in}}{\pgfqpoint{0.011050in}{0.041667in}}{\pgfqpoint{0.000000in}{0.041667in}}%
\pgfpathcurveto{\pgfqpoint{-0.011050in}{0.041667in}}{\pgfqpoint{-0.021649in}{0.037276in}}{\pgfqpoint{-0.029463in}{0.029463in}}%
\pgfpathcurveto{\pgfqpoint{-0.037276in}{0.021649in}}{\pgfqpoint{-0.041667in}{0.011050in}}{\pgfqpoint{-0.041667in}{0.000000in}}%
\pgfpathcurveto{\pgfqpoint{-0.041667in}{-0.011050in}}{\pgfqpoint{-0.037276in}{-0.021649in}}{\pgfqpoint{-0.029463in}{-0.029463in}}%
\pgfpathcurveto{\pgfqpoint{-0.021649in}{-0.037276in}}{\pgfqpoint{-0.011050in}{-0.041667in}}{\pgfqpoint{0.000000in}{-0.041667in}}%
\pgfpathclose%
\pgfusepath{stroke,fill}%
}%
\begin{pgfscope}%
\pgfsys@transformshift{3.345797in}{1.333072in}%
\pgfsys@useobject{currentmarker}{}%
\end{pgfscope}%
\end{pgfscope}%
\begin{pgfscope}%
\pgftext[x=3.745797in,y=1.255294in,left,base]{\rmfamily\fontsize{16.000000}{19.200000}\selectfont experiment}%
\end{pgfscope}%
\begin{pgfscope}%
\pgfsetrectcap%
\pgfsetroundjoin%
\pgfsetlinewidth{1.505625pt}%
\definecolor{currentstroke}{rgb}{1.000000,0.000000,0.000000}%
\pgfsetstrokecolor{currentstroke}%
\pgfsetdash{}{0pt}%
\pgfpathmoveto{\pgfqpoint{3.123575in}{1.006900in}}%
\pgfpathlineto{\pgfqpoint{3.568019in}{1.006900in}}%
\pgfusepath{stroke}%
\end{pgfscope}%
\begin{pgfscope}%
\pgftext[x=3.745797in,y=0.929123in,left,base]{\rmfamily\fontsize{16.000000}{19.200000}\selectfont model}%
\end{pgfscope}%
\end{pgfpicture}%
\makeatother%
\endgroup%
}
    \caption{A set of drop trajectories showing the results of the parameter estimation. The trajectories are shown only up to the first apoapse before rebound. The (\protect\redline) \hspace{0.25 mm} lines show the numerical solution of Equation \ref{gov_eqn_subs} with the given MLE parameter vector. $\chi^2$ goodness-of-fit varies between $1 \times 10^{-5}$ and $1 \times 10^{-8}$ with the better fit occurring typically for the drops with the lowest apoapses.}
    \label{fig:inverse_problem}
\end{figure}

\section{Results}
\subsection{Parameter Estimates}
The MLE estimates of $q$, as well as the measured values of $V_d$, $E_0$, and $U_0$ for the entire population of drop tower tests are shown in the scatter plot matrix of Figure \ref{fig:scatter}. The dependence of charge on drop volume $V_d$ is immediately evident, while the effect of electric field on drop charge is less obvious. This co-linearity is the source of the ill-conditioning in the parameter estimation process. However, assuming the main effect is the interaction between charge and electric field, a forced entry Robust Least Squares model 
\begin{equation}
q \sim kAE_0,
\label{ls_model}
\end{equation} 
with the transformation $A = V_d^{2/3}$ finds that $k=7.589 \times 10^{-11} \pm  1.47 \times 10^{-11}$ F/m with $R^2 = 0.65$, $F=27$. 
\begin{figure}[h]
    \centering
    \resizebox{0.5\textwidth}{!}{%% Creator: Matplotlib, PGF backend
%%
%% To include the figure in your LaTeX document, write
%%   \input{<filename>.pgf}
%%
%% Make sure the required packages are loaded in your preamble
%%   \usepackage{pgf}
%%
%% Figures using additional raster images can only be included by \input if
%% they are in the same directory as the main LaTeX file. For loading figures
%% from other directories you can use the `import` package
%%   \usepackage{import}
%% and then include the figures with
%%   \import{<path to file>}{<filename>.pgf}
%%
%% Matplotlib used the following preamble
%%   \usepackage{fontspec}
%%   \setmainfont{DejaVu Serif}
%%   \setsansfont{DejaVu Sans}
%%   \setmonofont{DejaVu Sans Mono}
%%
\begingroup%
\makeatletter%
\begin{pgfpicture}%
\pgfpathrectangle{\pgfpointorigin}{\pgfqpoint{5.674225in}{4.068832in}}%
\pgfusepath{use as bounding box, clip}%
\begin{pgfscope}%
\pgfsetbuttcap%
\pgfsetmiterjoin%
\definecolor{currentfill}{rgb}{1.000000,1.000000,1.000000}%
\pgfsetfillcolor{currentfill}%
\pgfsetlinewidth{0.000000pt}%
\definecolor{currentstroke}{rgb}{1.000000,1.000000,1.000000}%
\pgfsetstrokecolor{currentstroke}%
\pgfsetdash{}{0pt}%
\pgfpathmoveto{\pgfqpoint{0.000000in}{0.000000in}}%
\pgfpathlineto{\pgfqpoint{5.674225in}{0.000000in}}%
\pgfpathlineto{\pgfqpoint{5.674225in}{4.068832in}}%
\pgfpathlineto{\pgfqpoint{0.000000in}{4.068832in}}%
\pgfpathclose%
\pgfusepath{fill}%
\end{pgfscope}%
\begin{pgfscope}%
\pgfsetbuttcap%
\pgfsetmiterjoin%
\definecolor{currentfill}{rgb}{1.000000,1.000000,1.000000}%
\pgfsetfillcolor{currentfill}%
\pgfsetlinewidth{0.000000pt}%
\definecolor{currentstroke}{rgb}{0.000000,0.000000,0.000000}%
\pgfsetstrokecolor{currentstroke}%
\pgfsetstrokeopacity{0.000000}%
\pgfsetdash{}{0pt}%
\pgfpathmoveto{\pgfqpoint{0.889225in}{3.178832in}}%
\pgfpathlineto{\pgfqpoint{2.051725in}{3.178832in}}%
\pgfpathlineto{\pgfqpoint{2.051725in}{3.933832in}}%
\pgfpathlineto{\pgfqpoint{0.889225in}{3.933832in}}%
\pgfpathclose%
\pgfusepath{fill}%
\end{pgfscope}%
\begin{pgfscope}%
\pgfsetbuttcap%
\pgfsetroundjoin%
\definecolor{currentfill}{rgb}{0.000000,0.000000,0.000000}%
\pgfsetfillcolor{currentfill}%
\pgfsetlinewidth{0.803000pt}%
\definecolor{currentstroke}{rgb}{0.000000,0.000000,0.000000}%
\pgfsetstrokecolor{currentstroke}%
\pgfsetdash{}{0pt}%
\pgfsys@defobject{currentmarker}{\pgfqpoint{-0.048611in}{0.000000in}}{\pgfqpoint{0.000000in}{0.000000in}}{%
\pgfpathmoveto{\pgfqpoint{0.000000in}{0.000000in}}%
\pgfpathlineto{\pgfqpoint{-0.048611in}{0.000000in}}%
\pgfusepath{stroke,fill}%
}%
\begin{pgfscope}%
\pgfsys@transformshift{0.889225in}{3.404583in}%
\pgfsys@useobject{currentmarker}{}%
\end{pgfscope}%
\end{pgfscope}%
\begin{pgfscope}%
\pgftext[x=0.370616in,y=3.362373in,left,base]{\rmfamily\fontsize{8.000000}{9.600000}\selectfont 2.5e-05}%
\end{pgfscope}%
\begin{pgfscope}%
\pgfsetbuttcap%
\pgfsetroundjoin%
\definecolor{currentfill}{rgb}{0.000000,0.000000,0.000000}%
\pgfsetfillcolor{currentfill}%
\pgfsetlinewidth{0.803000pt}%
\definecolor{currentstroke}{rgb}{0.000000,0.000000,0.000000}%
\pgfsetstrokecolor{currentstroke}%
\pgfsetdash{}{0pt}%
\pgfsys@defobject{currentmarker}{\pgfqpoint{-0.048611in}{0.000000in}}{\pgfqpoint{0.000000in}{0.000000in}}{%
\pgfpathmoveto{\pgfqpoint{0.000000in}{0.000000in}}%
\pgfpathlineto{\pgfqpoint{-0.048611in}{0.000000in}}%
\pgfusepath{stroke,fill}%
}%
\begin{pgfscope}%
\pgfsys@transformshift{0.889225in}{3.743297in}%
\pgfsys@useobject{currentmarker}{}%
\end{pgfscope}%
\end{pgfscope}%
\begin{pgfscope}%
\pgftext[x=0.476627in,y=3.701087in,left,base]{\rmfamily\fontsize{8.000000}{9.600000}\selectfont 5e-05}%
\end{pgfscope}%
\begin{pgfscope}%
\pgftext[x=0.315061in,y=3.556332in,,bottom,rotate=90.000000]{\rmfamily\fontsize{16.000000}{19.200000}\selectfont area}%
\end{pgfscope}%
\begin{pgfscope}%
\pgfpathrectangle{\pgfqpoint{0.889225in}{3.178832in}}{\pgfqpoint{1.162500in}{0.755000in}}%
\pgfusepath{clip}%
\pgfsetrectcap%
\pgfsetroundjoin%
\pgfsetlinewidth{1.505625pt}%
\definecolor{currentstroke}{rgb}{0.121569,0.466667,0.705882}%
\pgfsetstrokecolor{currentstroke}%
\pgfsetdash{}{0pt}%
\pgfpathmoveto{\pgfqpoint{0.916904in}{3.605762in}}%
\pgfpathlineto{\pgfqpoint{0.947935in}{3.674214in}}%
\pgfpathlineto{\pgfqpoint{0.973424in}{3.725304in}}%
\pgfpathlineto{\pgfqpoint{0.995589in}{3.765054in}}%
\pgfpathlineto{\pgfqpoint{1.016646in}{3.798265in}}%
\pgfpathlineto{\pgfqpoint{1.035486in}{3.823954in}}%
\pgfpathlineto{\pgfqpoint{1.053218in}{3.844547in}}%
\pgfpathlineto{\pgfqpoint{1.069842in}{3.860681in}}%
\pgfpathlineto{\pgfqpoint{1.086466in}{3.873795in}}%
\pgfpathlineto{\pgfqpoint{1.103090in}{3.883997in}}%
\pgfpathlineto{\pgfqpoint{1.119714in}{3.891448in}}%
\pgfpathlineto{\pgfqpoint{1.136337in}{3.896349in}}%
\pgfpathlineto{\pgfqpoint{1.152961in}{3.898936in}}%
\pgfpathlineto{\pgfqpoint{1.170693in}{3.899439in}}%
\pgfpathlineto{\pgfqpoint{1.189533in}{3.897784in}}%
\pgfpathlineto{\pgfqpoint{1.211698in}{3.893479in}}%
\pgfpathlineto{\pgfqpoint{1.237188in}{3.886125in}}%
\pgfpathlineto{\pgfqpoint{1.269327in}{3.874363in}}%
\pgfpathlineto{\pgfqpoint{1.312549in}{3.855975in}}%
\pgfpathlineto{\pgfqpoint{1.362421in}{3.832368in}}%
\pgfpathlineto{\pgfqpoint{1.402318in}{3.811182in}}%
\pgfpathlineto{\pgfqpoint{1.435565in}{3.791115in}}%
\pgfpathlineto{\pgfqpoint{1.466596in}{3.769769in}}%
\pgfpathlineto{\pgfqpoint{1.496519in}{3.746420in}}%
\pgfpathlineto{\pgfqpoint{1.526442in}{3.720247in}}%
\pgfpathlineto{\pgfqpoint{1.558581in}{3.689172in}}%
\pgfpathlineto{\pgfqpoint{1.596262in}{3.649514in}}%
\pgfpathlineto{\pgfqpoint{1.649458in}{3.589904in}}%
\pgfpathlineto{\pgfqpoint{1.740334in}{3.487986in}}%
\pgfpathlineto{\pgfqpoint{1.790205in}{3.435782in}}%
\pgfpathlineto{\pgfqpoint{1.842293in}{3.384616in}}%
\pgfpathlineto{\pgfqpoint{1.920979in}{3.310990in}}%
\pgfpathlineto{\pgfqpoint{2.010747in}{3.226056in}}%
\pgfpathlineto{\pgfqpoint{2.024046in}{3.213151in}}%
\pgfpathlineto{\pgfqpoint{2.024046in}{3.213151in}}%
\pgfusepath{stroke}%
\end{pgfscope}%
\begin{pgfscope}%
\pgfsetrectcap%
\pgfsetmiterjoin%
\pgfsetlinewidth{0.803000pt}%
\definecolor{currentstroke}{rgb}{0.501961,0.501961,0.501961}%
\pgfsetstrokecolor{currentstroke}%
\pgfsetdash{}{0pt}%
\pgfpathmoveto{\pgfqpoint{0.889225in}{3.178832in}}%
\pgfpathlineto{\pgfqpoint{0.889225in}{3.933832in}}%
\pgfusepath{stroke}%
\end{pgfscope}%
\begin{pgfscope}%
\pgfsetrectcap%
\pgfsetmiterjoin%
\pgfsetlinewidth{0.803000pt}%
\definecolor{currentstroke}{rgb}{0.501961,0.501961,0.501961}%
\pgfsetstrokecolor{currentstroke}%
\pgfsetdash{}{0pt}%
\pgfpathmoveto{\pgfqpoint{2.051725in}{3.178832in}}%
\pgfpathlineto{\pgfqpoint{2.051725in}{3.933832in}}%
\pgfusepath{stroke}%
\end{pgfscope}%
\begin{pgfscope}%
\pgfsetrectcap%
\pgfsetmiterjoin%
\pgfsetlinewidth{0.803000pt}%
\definecolor{currentstroke}{rgb}{0.501961,0.501961,0.501961}%
\pgfsetstrokecolor{currentstroke}%
\pgfsetdash{}{0pt}%
\pgfpathmoveto{\pgfqpoint{0.889225in}{3.178832in}}%
\pgfpathlineto{\pgfqpoint{2.051725in}{3.178832in}}%
\pgfusepath{stroke}%
\end{pgfscope}%
\begin{pgfscope}%
\pgfsetrectcap%
\pgfsetmiterjoin%
\pgfsetlinewidth{0.803000pt}%
\definecolor{currentstroke}{rgb}{0.501961,0.501961,0.501961}%
\pgfsetstrokecolor{currentstroke}%
\pgfsetdash{}{0pt}%
\pgfpathmoveto{\pgfqpoint{0.889225in}{3.933832in}}%
\pgfpathlineto{\pgfqpoint{2.051725in}{3.933832in}}%
\pgfusepath{stroke}%
\end{pgfscope}%
\begin{pgfscope}%
\pgfsetbuttcap%
\pgfsetmiterjoin%
\definecolor{currentfill}{rgb}{1.000000,1.000000,1.000000}%
\pgfsetfillcolor{currentfill}%
\pgfsetlinewidth{0.000000pt}%
\definecolor{currentstroke}{rgb}{0.000000,0.000000,0.000000}%
\pgfsetstrokecolor{currentstroke}%
\pgfsetstrokeopacity{0.000000}%
\pgfsetdash{}{0pt}%
\pgfpathmoveto{\pgfqpoint{2.051725in}{3.178832in}}%
\pgfpathlineto{\pgfqpoint{3.214225in}{3.178832in}}%
\pgfpathlineto{\pgfqpoint{3.214225in}{3.933832in}}%
\pgfpathlineto{\pgfqpoint{2.051725in}{3.933832in}}%
\pgfpathclose%
\pgfusepath{fill}%
\end{pgfscope}%
\begin{pgfscope}%
\pgfpathrectangle{\pgfqpoint{2.051725in}{3.178832in}}{\pgfqpoint{1.162500in}{0.755000in}}%
\pgfusepath{clip}%
\pgfsetbuttcap%
\pgfsetroundjoin%
\definecolor{currentfill}{rgb}{0.000000,0.000000,0.000000}%
\pgfsetfillcolor{currentfill}%
\pgfsetfillopacity{0.500000}%
\pgfsetlinewidth{0.000000pt}%
\definecolor{currentstroke}{rgb}{0.000000,0.000000,0.000000}%
\pgfsetstrokecolor{currentstroke}%
\pgfsetdash{}{0pt}%
\pgfpathmoveto{\pgfqpoint{3.107569in}{3.810398in}}%
\pgfpathcurveto{\pgfqpoint{3.113094in}{3.810398in}}{\pgfqpoint{3.118394in}{3.812593in}}{\pgfqpoint{3.122301in}{3.816499in}}%
\pgfpathcurveto{\pgfqpoint{3.126207in}{3.820406in}}{\pgfqpoint{3.128403in}{3.825706in}}{\pgfqpoint{3.128403in}{3.831231in}}%
\pgfpathcurveto{\pgfqpoint{3.128403in}{3.836756in}}{\pgfqpoint{3.126207in}{3.842055in}}{\pgfqpoint{3.122301in}{3.845962in}}%
\pgfpathcurveto{\pgfqpoint{3.118394in}{3.849869in}}{\pgfqpoint{3.113094in}{3.852064in}}{\pgfqpoint{3.107569in}{3.852064in}}%
\pgfpathcurveto{\pgfqpoint{3.102044in}{3.852064in}}{\pgfqpoint{3.096745in}{3.849869in}}{\pgfqpoint{3.092838in}{3.845962in}}%
\pgfpathcurveto{\pgfqpoint{3.088931in}{3.842055in}}{\pgfqpoint{3.086736in}{3.836756in}}{\pgfqpoint{3.086736in}{3.831231in}}%
\pgfpathcurveto{\pgfqpoint{3.086736in}{3.825706in}}{\pgfqpoint{3.088931in}{3.820406in}}{\pgfqpoint{3.092838in}{3.816499in}}%
\pgfpathcurveto{\pgfqpoint{3.096745in}{3.812593in}}{\pgfqpoint{3.102044in}{3.810398in}}{\pgfqpoint{3.107569in}{3.810398in}}%
\pgfpathclose%
\pgfusepath{fill}%
\end{pgfscope}%
\begin{pgfscope}%
\pgfpathrectangle{\pgfqpoint{2.051725in}{3.178832in}}{\pgfqpoint{1.162500in}{0.755000in}}%
\pgfusepath{clip}%
\pgfsetbuttcap%
\pgfsetroundjoin%
\definecolor{currentfill}{rgb}{0.000000,0.000000,0.000000}%
\pgfsetfillcolor{currentfill}%
\pgfsetfillopacity{0.500000}%
\pgfsetlinewidth{0.000000pt}%
\definecolor{currentstroke}{rgb}{0.000000,0.000000,0.000000}%
\pgfsetstrokecolor{currentstroke}%
\pgfsetdash{}{0pt}%
\pgfpathmoveto{\pgfqpoint{2.270173in}{3.539667in}}%
\pgfpathcurveto{\pgfqpoint{2.275698in}{3.539667in}}{\pgfqpoint{2.280997in}{3.541862in}}{\pgfqpoint{2.284904in}{3.545769in}}%
\pgfpathcurveto{\pgfqpoint{2.288811in}{3.549675in}}{\pgfqpoint{2.291006in}{3.554975in}}{\pgfqpoint{2.291006in}{3.560500in}}%
\pgfpathcurveto{\pgfqpoint{2.291006in}{3.566025in}}{\pgfqpoint{2.288811in}{3.571325in}}{\pgfqpoint{2.284904in}{3.575231in}}%
\pgfpathcurveto{\pgfqpoint{2.280997in}{3.579138in}}{\pgfqpoint{2.275698in}{3.581333in}}{\pgfqpoint{2.270173in}{3.581333in}}%
\pgfpathcurveto{\pgfqpoint{2.264648in}{3.581333in}}{\pgfqpoint{2.259348in}{3.579138in}}{\pgfqpoint{2.255441in}{3.575231in}}%
\pgfpathcurveto{\pgfqpoint{2.251535in}{3.571325in}}{\pgfqpoint{2.249339in}{3.566025in}}{\pgfqpoint{2.249339in}{3.560500in}}%
\pgfpathcurveto{\pgfqpoint{2.249339in}{3.554975in}}{\pgfqpoint{2.251535in}{3.549675in}}{\pgfqpoint{2.255441in}{3.545769in}}%
\pgfpathcurveto{\pgfqpoint{2.259348in}{3.541862in}}{\pgfqpoint{2.264648in}{3.539667in}}{\pgfqpoint{2.270173in}{3.539667in}}%
\pgfpathclose%
\pgfusepath{fill}%
\end{pgfscope}%
\begin{pgfscope}%
\pgfpathrectangle{\pgfqpoint{2.051725in}{3.178832in}}{\pgfqpoint{1.162500in}{0.755000in}}%
\pgfusepath{clip}%
\pgfsetbuttcap%
\pgfsetroundjoin%
\definecolor{currentfill}{rgb}{0.000000,0.000000,0.000000}%
\pgfsetfillcolor{currentfill}%
\pgfsetfillopacity{0.500000}%
\pgfsetlinewidth{0.000000pt}%
\definecolor{currentstroke}{rgb}{0.000000,0.000000,0.000000}%
\pgfsetstrokecolor{currentstroke}%
\pgfsetdash{}{0pt}%
\pgfpathmoveto{\pgfqpoint{2.264867in}{3.541019in}}%
\pgfpathcurveto{\pgfqpoint{2.270392in}{3.541019in}}{\pgfqpoint{2.275692in}{3.543214in}}{\pgfqpoint{2.279599in}{3.547121in}}%
\pgfpathcurveto{\pgfqpoint{2.283506in}{3.551028in}}{\pgfqpoint{2.285701in}{3.556327in}}{\pgfqpoint{2.285701in}{3.561852in}}%
\pgfpathcurveto{\pgfqpoint{2.285701in}{3.567377in}}{\pgfqpoint{2.283506in}{3.572677in}}{\pgfqpoint{2.279599in}{3.576584in}}%
\pgfpathcurveto{\pgfqpoint{2.275692in}{3.580490in}}{\pgfqpoint{2.270392in}{3.582685in}}{\pgfqpoint{2.264867in}{3.582685in}}%
\pgfpathcurveto{\pgfqpoint{2.259342in}{3.582685in}}{\pgfqpoint{2.254043in}{3.580490in}}{\pgfqpoint{2.250136in}{3.576584in}}%
\pgfpathcurveto{\pgfqpoint{2.246229in}{3.572677in}}{\pgfqpoint{2.244034in}{3.567377in}}{\pgfqpoint{2.244034in}{3.561852in}}%
\pgfpathcurveto{\pgfqpoint{2.244034in}{3.556327in}}{\pgfqpoint{2.246229in}{3.551028in}}{\pgfqpoint{2.250136in}{3.547121in}}%
\pgfpathcurveto{\pgfqpoint{2.254043in}{3.543214in}}{\pgfqpoint{2.259342in}{3.541019in}}{\pgfqpoint{2.264867in}{3.541019in}}%
\pgfpathclose%
\pgfusepath{fill}%
\end{pgfscope}%
\begin{pgfscope}%
\pgfpathrectangle{\pgfqpoint{2.051725in}{3.178832in}}{\pgfqpoint{1.162500in}{0.755000in}}%
\pgfusepath{clip}%
\pgfsetbuttcap%
\pgfsetroundjoin%
\definecolor{currentfill}{rgb}{0.000000,0.000000,0.000000}%
\pgfsetfillcolor{currentfill}%
\pgfsetfillopacity{0.500000}%
\pgfsetlinewidth{0.000000pt}%
\definecolor{currentstroke}{rgb}{0.000000,0.000000,0.000000}%
\pgfsetstrokecolor{currentstroke}%
\pgfsetdash{}{0pt}%
\pgfpathmoveto{\pgfqpoint{2.154193in}{3.308342in}}%
\pgfpathcurveto{\pgfqpoint{2.159718in}{3.308342in}}{\pgfqpoint{2.165018in}{3.310537in}}{\pgfqpoint{2.168924in}{3.314444in}}%
\pgfpathcurveto{\pgfqpoint{2.172831in}{3.318351in}}{\pgfqpoint{2.175026in}{3.323651in}}{\pgfqpoint{2.175026in}{3.329176in}}%
\pgfpathcurveto{\pgfqpoint{2.175026in}{3.334701in}}{\pgfqpoint{2.172831in}{3.340000in}}{\pgfqpoint{2.168924in}{3.343907in}}%
\pgfpathcurveto{\pgfqpoint{2.165018in}{3.347814in}}{\pgfqpoint{2.159718in}{3.350009in}}{\pgfqpoint{2.154193in}{3.350009in}}%
\pgfpathcurveto{\pgfqpoint{2.148668in}{3.350009in}}{\pgfqpoint{2.143368in}{3.347814in}}{\pgfqpoint{2.139462in}{3.343907in}}%
\pgfpathcurveto{\pgfqpoint{2.135555in}{3.340000in}}{\pgfqpoint{2.133360in}{3.334701in}}{\pgfqpoint{2.133360in}{3.329176in}}%
\pgfpathcurveto{\pgfqpoint{2.133360in}{3.323651in}}{\pgfqpoint{2.135555in}{3.318351in}}{\pgfqpoint{2.139462in}{3.314444in}}%
\pgfpathcurveto{\pgfqpoint{2.143368in}{3.310537in}}{\pgfqpoint{2.148668in}{3.308342in}}{\pgfqpoint{2.154193in}{3.308342in}}%
\pgfpathclose%
\pgfusepath{fill}%
\end{pgfscope}%
\begin{pgfscope}%
\pgfpathrectangle{\pgfqpoint{2.051725in}{3.178832in}}{\pgfqpoint{1.162500in}{0.755000in}}%
\pgfusepath{clip}%
\pgfsetbuttcap%
\pgfsetroundjoin%
\definecolor{currentfill}{rgb}{0.000000,0.000000,0.000000}%
\pgfsetfillcolor{currentfill}%
\pgfsetfillopacity{0.500000}%
\pgfsetlinewidth{0.000000pt}%
\definecolor{currentstroke}{rgb}{0.000000,0.000000,0.000000}%
\pgfsetstrokecolor{currentstroke}%
\pgfsetdash{}{0pt}%
\pgfpathmoveto{\pgfqpoint{2.183344in}{3.311937in}}%
\pgfpathcurveto{\pgfqpoint{2.188869in}{3.311937in}}{\pgfqpoint{2.194169in}{3.314132in}}{\pgfqpoint{2.198076in}{3.318039in}}%
\pgfpathcurveto{\pgfqpoint{2.201982in}{3.321945in}}{\pgfqpoint{2.204178in}{3.327245in}}{\pgfqpoint{2.204178in}{3.332770in}}%
\pgfpathcurveto{\pgfqpoint{2.204178in}{3.338295in}}{\pgfqpoint{2.201982in}{3.343595in}}{\pgfqpoint{2.198076in}{3.347501in}}%
\pgfpathcurveto{\pgfqpoint{2.194169in}{3.351408in}}{\pgfqpoint{2.188869in}{3.353603in}}{\pgfqpoint{2.183344in}{3.353603in}}%
\pgfpathcurveto{\pgfqpoint{2.177819in}{3.353603in}}{\pgfqpoint{2.172520in}{3.351408in}}{\pgfqpoint{2.168613in}{3.347501in}}%
\pgfpathcurveto{\pgfqpoint{2.164706in}{3.343595in}}{\pgfqpoint{2.162511in}{3.338295in}}{\pgfqpoint{2.162511in}{3.332770in}}%
\pgfpathcurveto{\pgfqpoint{2.162511in}{3.327245in}}{\pgfqpoint{2.164706in}{3.321945in}}{\pgfqpoint{2.168613in}{3.318039in}}%
\pgfpathcurveto{\pgfqpoint{2.172520in}{3.314132in}}{\pgfqpoint{2.177819in}{3.311937in}}{\pgfqpoint{2.183344in}{3.311937in}}%
\pgfpathclose%
\pgfusepath{fill}%
\end{pgfscope}%
\begin{pgfscope}%
\pgfpathrectangle{\pgfqpoint{2.051725in}{3.178832in}}{\pgfqpoint{1.162500in}{0.755000in}}%
\pgfusepath{clip}%
\pgfsetbuttcap%
\pgfsetroundjoin%
\definecolor{currentfill}{rgb}{0.000000,0.000000,0.000000}%
\pgfsetfillcolor{currentfill}%
\pgfsetfillopacity{0.500000}%
\pgfsetlinewidth{0.000000pt}%
\definecolor{currentstroke}{rgb}{0.000000,0.000000,0.000000}%
\pgfsetstrokecolor{currentstroke}%
\pgfsetdash{}{0pt}%
\pgfpathmoveto{\pgfqpoint{2.082232in}{3.215463in}}%
\pgfpathcurveto{\pgfqpoint{2.087757in}{3.215463in}}{\pgfqpoint{2.093056in}{3.217658in}}{\pgfqpoint{2.096963in}{3.221565in}}%
\pgfpathcurveto{\pgfqpoint{2.100870in}{3.225471in}}{\pgfqpoint{2.103065in}{3.230771in}}{\pgfqpoint{2.103065in}{3.236296in}}%
\pgfpathcurveto{\pgfqpoint{2.103065in}{3.241821in}}{\pgfqpoint{2.100870in}{3.247121in}}{\pgfqpoint{2.096963in}{3.251027in}}%
\pgfpathcurveto{\pgfqpoint{2.093056in}{3.254934in}}{\pgfqpoint{2.087757in}{3.257129in}}{\pgfqpoint{2.082232in}{3.257129in}}%
\pgfpathcurveto{\pgfqpoint{2.076707in}{3.257129in}}{\pgfqpoint{2.071407in}{3.254934in}}{\pgfqpoint{2.067500in}{3.251027in}}%
\pgfpathcurveto{\pgfqpoint{2.063594in}{3.247121in}}{\pgfqpoint{2.061398in}{3.241821in}}{\pgfqpoint{2.061398in}{3.236296in}}%
\pgfpathcurveto{\pgfqpoint{2.061398in}{3.230771in}}{\pgfqpoint{2.063594in}{3.225471in}}{\pgfqpoint{2.067500in}{3.221565in}}%
\pgfpathcurveto{\pgfqpoint{2.071407in}{3.217658in}}{\pgfqpoint{2.076707in}{3.215463in}}{\pgfqpoint{2.082232in}{3.215463in}}%
\pgfpathclose%
\pgfusepath{fill}%
\end{pgfscope}%
\begin{pgfscope}%
\pgfpathrectangle{\pgfqpoint{2.051725in}{3.178832in}}{\pgfqpoint{1.162500in}{0.755000in}}%
\pgfusepath{clip}%
\pgfsetbuttcap%
\pgfsetroundjoin%
\definecolor{currentfill}{rgb}{0.000000,0.000000,0.000000}%
\pgfsetfillcolor{currentfill}%
\pgfsetfillopacity{0.500000}%
\pgfsetlinewidth{0.000000pt}%
\definecolor{currentstroke}{rgb}{0.000000,0.000000,0.000000}%
\pgfsetstrokecolor{currentstroke}%
\pgfsetdash{}{0pt}%
\pgfpathmoveto{\pgfqpoint{2.079404in}{3.175975in}}%
\pgfpathcurveto{\pgfqpoint{2.084929in}{3.175975in}}{\pgfqpoint{2.090228in}{3.178170in}}{\pgfqpoint{2.094135in}{3.182077in}}%
\pgfpathcurveto{\pgfqpoint{2.098042in}{3.185984in}}{\pgfqpoint{2.100237in}{3.191283in}}{\pgfqpoint{2.100237in}{3.196809in}}%
\pgfpathcurveto{\pgfqpoint{2.100237in}{3.202334in}}{\pgfqpoint{2.098042in}{3.207633in}}{\pgfqpoint{2.094135in}{3.211540in}}%
\pgfpathcurveto{\pgfqpoint{2.090228in}{3.215447in}}{\pgfqpoint{2.084929in}{3.217642in}}{\pgfqpoint{2.079404in}{3.217642in}}%
\pgfpathcurveto{\pgfqpoint{2.073879in}{3.217642in}}{\pgfqpoint{2.068579in}{3.215447in}}{\pgfqpoint{2.064672in}{3.211540in}}%
\pgfpathcurveto{\pgfqpoint{2.060765in}{3.207633in}}{\pgfqpoint{2.058570in}{3.202334in}}{\pgfqpoint{2.058570in}{3.196809in}}%
\pgfpathcurveto{\pgfqpoint{2.058570in}{3.191283in}}{\pgfqpoint{2.060765in}{3.185984in}}{\pgfqpoint{2.064672in}{3.182077in}}%
\pgfpathcurveto{\pgfqpoint{2.068579in}{3.178170in}}{\pgfqpoint{2.073879in}{3.175975in}}{\pgfqpoint{2.079404in}{3.175975in}}%
\pgfpathclose%
\pgfusepath{fill}%
\end{pgfscope}%
\begin{pgfscope}%
\pgfpathrectangle{\pgfqpoint{2.051725in}{3.178832in}}{\pgfqpoint{1.162500in}{0.755000in}}%
\pgfusepath{clip}%
\pgfsetbuttcap%
\pgfsetroundjoin%
\definecolor{currentfill}{rgb}{0.000000,0.000000,0.000000}%
\pgfsetfillcolor{currentfill}%
\pgfsetfillopacity{0.500000}%
\pgfsetlinewidth{0.000000pt}%
\definecolor{currentstroke}{rgb}{0.000000,0.000000,0.000000}%
\pgfsetstrokecolor{currentstroke}%
\pgfsetdash{}{0pt}%
\pgfpathmoveto{\pgfqpoint{2.536683in}{3.678982in}}%
\pgfpathcurveto{\pgfqpoint{2.542208in}{3.678982in}}{\pgfqpoint{2.547508in}{3.681177in}}{\pgfqpoint{2.551415in}{3.685084in}}%
\pgfpathcurveto{\pgfqpoint{2.555322in}{3.688990in}}{\pgfqpoint{2.557517in}{3.694290in}}{\pgfqpoint{2.557517in}{3.699815in}}%
\pgfpathcurveto{\pgfqpoint{2.557517in}{3.705340in}}{\pgfqpoint{2.555322in}{3.710640in}}{\pgfqpoint{2.551415in}{3.714546in}}%
\pgfpathcurveto{\pgfqpoint{2.547508in}{3.718453in}}{\pgfqpoint{2.542208in}{3.720648in}}{\pgfqpoint{2.536683in}{3.720648in}}%
\pgfpathcurveto{\pgfqpoint{2.531158in}{3.720648in}}{\pgfqpoint{2.525859in}{3.718453in}}{\pgfqpoint{2.521952in}{3.714546in}}%
\pgfpathcurveto{\pgfqpoint{2.518045in}{3.710640in}}{\pgfqpoint{2.515850in}{3.705340in}}{\pgfqpoint{2.515850in}{3.699815in}}%
\pgfpathcurveto{\pgfqpoint{2.515850in}{3.694290in}}{\pgfqpoint{2.518045in}{3.688990in}}{\pgfqpoint{2.521952in}{3.685084in}}%
\pgfpathcurveto{\pgfqpoint{2.525859in}{3.681177in}}{\pgfqpoint{2.531158in}{3.678982in}}{\pgfqpoint{2.536683in}{3.678982in}}%
\pgfpathclose%
\pgfusepath{fill}%
\end{pgfscope}%
\begin{pgfscope}%
\pgfpathrectangle{\pgfqpoint{2.051725in}{3.178832in}}{\pgfqpoint{1.162500in}{0.755000in}}%
\pgfusepath{clip}%
\pgfsetbuttcap%
\pgfsetroundjoin%
\definecolor{currentfill}{rgb}{0.000000,0.000000,0.000000}%
\pgfsetfillcolor{currentfill}%
\pgfsetfillopacity{0.500000}%
\pgfsetlinewidth{0.000000pt}%
\definecolor{currentstroke}{rgb}{0.000000,0.000000,0.000000}%
\pgfsetstrokecolor{currentstroke}%
\pgfsetdash{}{0pt}%
\pgfpathmoveto{\pgfqpoint{2.417381in}{3.528997in}}%
\pgfpathcurveto{\pgfqpoint{2.422906in}{3.528997in}}{\pgfqpoint{2.428205in}{3.531192in}}{\pgfqpoint{2.432112in}{3.535099in}}%
\pgfpathcurveto{\pgfqpoint{2.436019in}{3.539005in}}{\pgfqpoint{2.438214in}{3.544305in}}{\pgfqpoint{2.438214in}{3.549830in}}%
\pgfpathcurveto{\pgfqpoint{2.438214in}{3.555355in}}{\pgfqpoint{2.436019in}{3.560655in}}{\pgfqpoint{2.432112in}{3.564561in}}%
\pgfpathcurveto{\pgfqpoint{2.428205in}{3.568468in}}{\pgfqpoint{2.422906in}{3.570663in}}{\pgfqpoint{2.417381in}{3.570663in}}%
\pgfpathcurveto{\pgfqpoint{2.411856in}{3.570663in}}{\pgfqpoint{2.406556in}{3.568468in}}{\pgfqpoint{2.402649in}{3.564561in}}%
\pgfpathcurveto{\pgfqpoint{2.398743in}{3.560655in}}{\pgfqpoint{2.396547in}{3.555355in}}{\pgfqpoint{2.396547in}{3.549830in}}%
\pgfpathcurveto{\pgfqpoint{2.396547in}{3.544305in}}{\pgfqpoint{2.398743in}{3.539005in}}{\pgfqpoint{2.402649in}{3.535099in}}%
\pgfpathcurveto{\pgfqpoint{2.406556in}{3.531192in}}{\pgfqpoint{2.411856in}{3.528997in}}{\pgfqpoint{2.417381in}{3.528997in}}%
\pgfpathclose%
\pgfusepath{fill}%
\end{pgfscope}%
\begin{pgfscope}%
\pgfpathrectangle{\pgfqpoint{2.051725in}{3.178832in}}{\pgfqpoint{1.162500in}{0.755000in}}%
\pgfusepath{clip}%
\pgfsetbuttcap%
\pgfsetroundjoin%
\definecolor{currentfill}{rgb}{0.000000,0.000000,0.000000}%
\pgfsetfillcolor{currentfill}%
\pgfsetfillopacity{0.500000}%
\pgfsetlinewidth{0.000000pt}%
\definecolor{currentstroke}{rgb}{0.000000,0.000000,0.000000}%
\pgfsetstrokecolor{currentstroke}%
\pgfsetdash{}{0pt}%
\pgfpathmoveto{\pgfqpoint{2.412630in}{3.426855in}}%
\pgfpathcurveto{\pgfqpoint{2.418155in}{3.426855in}}{\pgfqpoint{2.423454in}{3.429051in}}{\pgfqpoint{2.427361in}{3.432957in}}%
\pgfpathcurveto{\pgfqpoint{2.431268in}{3.436864in}}{\pgfqpoint{2.433463in}{3.442164in}}{\pgfqpoint{2.433463in}{3.447689in}}%
\pgfpathcurveto{\pgfqpoint{2.433463in}{3.453214in}}{\pgfqpoint{2.431268in}{3.458513in}}{\pgfqpoint{2.427361in}{3.462420in}}%
\pgfpathcurveto{\pgfqpoint{2.423454in}{3.466327in}}{\pgfqpoint{2.418155in}{3.468522in}}{\pgfqpoint{2.412630in}{3.468522in}}%
\pgfpathcurveto{\pgfqpoint{2.407105in}{3.468522in}}{\pgfqpoint{2.401805in}{3.466327in}}{\pgfqpoint{2.397898in}{3.462420in}}%
\pgfpathcurveto{\pgfqpoint{2.393991in}{3.458513in}}{\pgfqpoint{2.391796in}{3.453214in}}{\pgfqpoint{2.391796in}{3.447689in}}%
\pgfpathcurveto{\pgfqpoint{2.391796in}{3.442164in}}{\pgfqpoint{2.393991in}{3.436864in}}{\pgfqpoint{2.397898in}{3.432957in}}%
\pgfpathcurveto{\pgfqpoint{2.401805in}{3.429051in}}{\pgfqpoint{2.407105in}{3.426855in}}{\pgfqpoint{2.412630in}{3.426855in}}%
\pgfpathclose%
\pgfusepath{fill}%
\end{pgfscope}%
\begin{pgfscope}%
\pgfpathrectangle{\pgfqpoint{2.051725in}{3.178832in}}{\pgfqpoint{1.162500in}{0.755000in}}%
\pgfusepath{clip}%
\pgfsetbuttcap%
\pgfsetroundjoin%
\definecolor{currentfill}{rgb}{0.000000,0.000000,0.000000}%
\pgfsetfillcolor{currentfill}%
\pgfsetfillopacity{0.500000}%
\pgfsetlinewidth{0.000000pt}%
\definecolor{currentstroke}{rgb}{0.000000,0.000000,0.000000}%
\pgfsetstrokecolor{currentstroke}%
\pgfsetdash{}{0pt}%
\pgfpathmoveto{\pgfqpoint{2.478188in}{3.677650in}}%
\pgfpathcurveto{\pgfqpoint{2.483713in}{3.677650in}}{\pgfqpoint{2.489012in}{3.679845in}}{\pgfqpoint{2.492919in}{3.683752in}}%
\pgfpathcurveto{\pgfqpoint{2.496826in}{3.687659in}}{\pgfqpoint{2.499021in}{3.692959in}}{\pgfqpoint{2.499021in}{3.698484in}}%
\pgfpathcurveto{\pgfqpoint{2.499021in}{3.704009in}}{\pgfqpoint{2.496826in}{3.709308in}}{\pgfqpoint{2.492919in}{3.713215in}}%
\pgfpathcurveto{\pgfqpoint{2.489012in}{3.717122in}}{\pgfqpoint{2.483713in}{3.719317in}}{\pgfqpoint{2.478188in}{3.719317in}}%
\pgfpathcurveto{\pgfqpoint{2.472663in}{3.719317in}}{\pgfqpoint{2.467363in}{3.717122in}}{\pgfqpoint{2.463456in}{3.713215in}}%
\pgfpathcurveto{\pgfqpoint{2.459549in}{3.709308in}}{\pgfqpoint{2.457354in}{3.704009in}}{\pgfqpoint{2.457354in}{3.698484in}}%
\pgfpathcurveto{\pgfqpoint{2.457354in}{3.692959in}}{\pgfqpoint{2.459549in}{3.687659in}}{\pgfqpoint{2.463456in}{3.683752in}}%
\pgfpathcurveto{\pgfqpoint{2.467363in}{3.679845in}}{\pgfqpoint{2.472663in}{3.677650in}}{\pgfqpoint{2.478188in}{3.677650in}}%
\pgfpathclose%
\pgfusepath{fill}%
\end{pgfscope}%
\begin{pgfscope}%
\pgfpathrectangle{\pgfqpoint{2.051725in}{3.178832in}}{\pgfqpoint{1.162500in}{0.755000in}}%
\pgfusepath{clip}%
\pgfsetbuttcap%
\pgfsetroundjoin%
\definecolor{currentfill}{rgb}{0.000000,0.000000,0.000000}%
\pgfsetfillcolor{currentfill}%
\pgfsetfillopacity{0.500000}%
\pgfsetlinewidth{0.000000pt}%
\definecolor{currentstroke}{rgb}{0.000000,0.000000,0.000000}%
\pgfsetstrokecolor{currentstroke}%
\pgfsetdash{}{0pt}%
\pgfpathmoveto{\pgfqpoint{2.172291in}{3.296446in}}%
\pgfpathcurveto{\pgfqpoint{2.177816in}{3.296446in}}{\pgfqpoint{2.183116in}{3.298641in}}{\pgfqpoint{2.187022in}{3.302548in}}%
\pgfpathcurveto{\pgfqpoint{2.190929in}{3.306455in}}{\pgfqpoint{2.193124in}{3.311754in}}{\pgfqpoint{2.193124in}{3.317279in}}%
\pgfpathcurveto{\pgfqpoint{2.193124in}{3.322804in}}{\pgfqpoint{2.190929in}{3.328104in}}{\pgfqpoint{2.187022in}{3.332011in}}%
\pgfpathcurveto{\pgfqpoint{2.183116in}{3.335917in}}{\pgfqpoint{2.177816in}{3.338112in}}{\pgfqpoint{2.172291in}{3.338112in}}%
\pgfpathcurveto{\pgfqpoint{2.166766in}{3.338112in}}{\pgfqpoint{2.161467in}{3.335917in}}{\pgfqpoint{2.157560in}{3.332011in}}%
\pgfpathcurveto{\pgfqpoint{2.153653in}{3.328104in}}{\pgfqpoint{2.151458in}{3.322804in}}{\pgfqpoint{2.151458in}{3.317279in}}%
\pgfpathcurveto{\pgfqpoint{2.151458in}{3.311754in}}{\pgfqpoint{2.153653in}{3.306455in}}{\pgfqpoint{2.157560in}{3.302548in}}%
\pgfpathcurveto{\pgfqpoint{2.161467in}{3.298641in}}{\pgfqpoint{2.166766in}{3.296446in}}{\pgfqpoint{2.172291in}{3.296446in}}%
\pgfpathclose%
\pgfusepath{fill}%
\end{pgfscope}%
\begin{pgfscope}%
\pgfpathrectangle{\pgfqpoint{2.051725in}{3.178832in}}{\pgfqpoint{1.162500in}{0.755000in}}%
\pgfusepath{clip}%
\pgfsetbuttcap%
\pgfsetroundjoin%
\definecolor{currentfill}{rgb}{0.000000,0.000000,0.000000}%
\pgfsetfillcolor{currentfill}%
\pgfsetfillopacity{0.500000}%
\pgfsetlinewidth{0.000000pt}%
\definecolor{currentstroke}{rgb}{0.000000,0.000000,0.000000}%
\pgfsetstrokecolor{currentstroke}%
\pgfsetdash{}{0pt}%
\pgfpathmoveto{\pgfqpoint{2.205718in}{3.322937in}}%
\pgfpathcurveto{\pgfqpoint{2.211243in}{3.322937in}}{\pgfqpoint{2.216543in}{3.325133in}}{\pgfqpoint{2.220450in}{3.329039in}}%
\pgfpathcurveto{\pgfqpoint{2.224357in}{3.332946in}}{\pgfqpoint{2.226552in}{3.338246in}}{\pgfqpoint{2.226552in}{3.343771in}}%
\pgfpathcurveto{\pgfqpoint{2.226552in}{3.349296in}}{\pgfqpoint{2.224357in}{3.354595in}}{\pgfqpoint{2.220450in}{3.358502in}}%
\pgfpathcurveto{\pgfqpoint{2.216543in}{3.362409in}}{\pgfqpoint{2.211243in}{3.364604in}}{\pgfqpoint{2.205718in}{3.364604in}}%
\pgfpathcurveto{\pgfqpoint{2.200193in}{3.364604in}}{\pgfqpoint{2.194894in}{3.362409in}}{\pgfqpoint{2.190987in}{3.358502in}}%
\pgfpathcurveto{\pgfqpoint{2.187080in}{3.354595in}}{\pgfqpoint{2.184885in}{3.349296in}}{\pgfqpoint{2.184885in}{3.343771in}}%
\pgfpathcurveto{\pgfqpoint{2.184885in}{3.338246in}}{\pgfqpoint{2.187080in}{3.332946in}}{\pgfqpoint{2.190987in}{3.329039in}}%
\pgfpathcurveto{\pgfqpoint{2.194894in}{3.325133in}}{\pgfqpoint{2.200193in}{3.322937in}}{\pgfqpoint{2.205718in}{3.322937in}}%
\pgfpathclose%
\pgfusepath{fill}%
\end{pgfscope}%
\begin{pgfscope}%
\pgfpathrectangle{\pgfqpoint{2.051725in}{3.178832in}}{\pgfqpoint{1.162500in}{0.755000in}}%
\pgfusepath{clip}%
\pgfsetbuttcap%
\pgfsetroundjoin%
\definecolor{currentfill}{rgb}{0.000000,0.000000,0.000000}%
\pgfsetfillcolor{currentfill}%
\pgfsetfillopacity{0.500000}%
\pgfsetlinewidth{0.000000pt}%
\definecolor{currentstroke}{rgb}{0.000000,0.000000,0.000000}%
\pgfsetstrokecolor{currentstroke}%
\pgfsetdash{}{0pt}%
\pgfpathmoveto{\pgfqpoint{3.186546in}{3.895023in}}%
\pgfpathcurveto{\pgfqpoint{3.192071in}{3.895023in}}{\pgfqpoint{3.197371in}{3.897218in}}{\pgfqpoint{3.201278in}{3.901125in}}%
\pgfpathcurveto{\pgfqpoint{3.205185in}{3.905032in}}{\pgfqpoint{3.207380in}{3.910331in}}{\pgfqpoint{3.207380in}{3.915856in}}%
\pgfpathcurveto{\pgfqpoint{3.207380in}{3.921381in}}{\pgfqpoint{3.205185in}{3.926681in}}{\pgfqpoint{3.201278in}{3.930588in}}%
\pgfpathcurveto{\pgfqpoint{3.197371in}{3.934494in}}{\pgfqpoint{3.192071in}{3.936690in}}{\pgfqpoint{3.186546in}{3.936690in}}%
\pgfpathcurveto{\pgfqpoint{3.181021in}{3.936690in}}{\pgfqpoint{3.175722in}{3.934494in}}{\pgfqpoint{3.171815in}{3.930588in}}%
\pgfpathcurveto{\pgfqpoint{3.167908in}{3.926681in}}{\pgfqpoint{3.165713in}{3.921381in}}{\pgfqpoint{3.165713in}{3.915856in}}%
\pgfpathcurveto{\pgfqpoint{3.165713in}{3.910331in}}{\pgfqpoint{3.167908in}{3.905032in}}{\pgfqpoint{3.171815in}{3.901125in}}%
\pgfpathcurveto{\pgfqpoint{3.175722in}{3.897218in}}{\pgfqpoint{3.181021in}{3.895023in}}{\pgfqpoint{3.186546in}{3.895023in}}%
\pgfpathclose%
\pgfusepath{fill}%
\end{pgfscope}%
\begin{pgfscope}%
\pgfpathrectangle{\pgfqpoint{2.051725in}{3.178832in}}{\pgfqpoint{1.162500in}{0.755000in}}%
\pgfusepath{clip}%
\pgfsetbuttcap%
\pgfsetroundjoin%
\definecolor{currentfill}{rgb}{0.000000,0.000000,0.000000}%
\pgfsetfillcolor{currentfill}%
\pgfsetfillopacity{0.500000}%
\pgfsetlinewidth{0.000000pt}%
\definecolor{currentstroke}{rgb}{0.000000,0.000000,0.000000}%
\pgfsetstrokecolor{currentstroke}%
\pgfsetdash{}{0pt}%
\pgfpathmoveto{\pgfqpoint{2.394110in}{3.491212in}}%
\pgfpathcurveto{\pgfqpoint{2.399635in}{3.491212in}}{\pgfqpoint{2.404935in}{3.493407in}}{\pgfqpoint{2.408842in}{3.497314in}}%
\pgfpathcurveto{\pgfqpoint{2.412749in}{3.501221in}}{\pgfqpoint{2.414944in}{3.506520in}}{\pgfqpoint{2.414944in}{3.512045in}}%
\pgfpathcurveto{\pgfqpoint{2.414944in}{3.517571in}}{\pgfqpoint{2.412749in}{3.522870in}}{\pgfqpoint{2.408842in}{3.526777in}}%
\pgfpathcurveto{\pgfqpoint{2.404935in}{3.530684in}}{\pgfqpoint{2.399635in}{3.532879in}}{\pgfqpoint{2.394110in}{3.532879in}}%
\pgfpathcurveto{\pgfqpoint{2.388585in}{3.532879in}}{\pgfqpoint{2.383286in}{3.530684in}}{\pgfqpoint{2.379379in}{3.526777in}}%
\pgfpathcurveto{\pgfqpoint{2.375472in}{3.522870in}}{\pgfqpoint{2.373277in}{3.517571in}}{\pgfqpoint{2.373277in}{3.512045in}}%
\pgfpathcurveto{\pgfqpoint{2.373277in}{3.506520in}}{\pgfqpoint{2.375472in}{3.501221in}}{\pgfqpoint{2.379379in}{3.497314in}}%
\pgfpathcurveto{\pgfqpoint{2.383286in}{3.493407in}}{\pgfqpoint{2.388585in}{3.491212in}}{\pgfqpoint{2.394110in}{3.491212in}}%
\pgfpathclose%
\pgfusepath{fill}%
\end{pgfscope}%
\begin{pgfscope}%
\pgfpathrectangle{\pgfqpoint{2.051725in}{3.178832in}}{\pgfqpoint{1.162500in}{0.755000in}}%
\pgfusepath{clip}%
\pgfsetbuttcap%
\pgfsetroundjoin%
\definecolor{currentfill}{rgb}{0.000000,0.000000,0.000000}%
\pgfsetfillcolor{currentfill}%
\pgfsetfillopacity{0.500000}%
\pgfsetlinewidth{0.000000pt}%
\definecolor{currentstroke}{rgb}{0.000000,0.000000,0.000000}%
\pgfsetstrokecolor{currentstroke}%
\pgfsetdash{}{0pt}%
\pgfpathmoveto{\pgfqpoint{2.125339in}{3.212721in}}%
\pgfpathcurveto{\pgfqpoint{2.130865in}{3.212721in}}{\pgfqpoint{2.136164in}{3.214917in}}{\pgfqpoint{2.140071in}{3.218823in}}%
\pgfpathcurveto{\pgfqpoint{2.143978in}{3.222730in}}{\pgfqpoint{2.146173in}{3.228030in}}{\pgfqpoint{2.146173in}{3.233555in}}%
\pgfpathcurveto{\pgfqpoint{2.146173in}{3.239080in}}{\pgfqpoint{2.143978in}{3.244379in}}{\pgfqpoint{2.140071in}{3.248286in}}%
\pgfpathcurveto{\pgfqpoint{2.136164in}{3.252193in}}{\pgfqpoint{2.130865in}{3.254388in}}{\pgfqpoint{2.125339in}{3.254388in}}%
\pgfpathcurveto{\pgfqpoint{2.119814in}{3.254388in}}{\pgfqpoint{2.114515in}{3.252193in}}{\pgfqpoint{2.110608in}{3.248286in}}%
\pgfpathcurveto{\pgfqpoint{2.106701in}{3.244379in}}{\pgfqpoint{2.104506in}{3.239080in}}{\pgfqpoint{2.104506in}{3.233555in}}%
\pgfpathcurveto{\pgfqpoint{2.104506in}{3.228030in}}{\pgfqpoint{2.106701in}{3.222730in}}{\pgfqpoint{2.110608in}{3.218823in}}%
\pgfpathcurveto{\pgfqpoint{2.114515in}{3.214917in}}{\pgfqpoint{2.119814in}{3.212721in}}{\pgfqpoint{2.125339in}{3.212721in}}%
\pgfpathclose%
\pgfusepath{fill}%
\end{pgfscope}%
\begin{pgfscope}%
\pgfsetrectcap%
\pgfsetmiterjoin%
\pgfsetlinewidth{0.803000pt}%
\definecolor{currentstroke}{rgb}{0.501961,0.501961,0.501961}%
\pgfsetstrokecolor{currentstroke}%
\pgfsetdash{}{0pt}%
\pgfpathmoveto{\pgfqpoint{2.051725in}{3.178832in}}%
\pgfpathlineto{\pgfqpoint{2.051725in}{3.933832in}}%
\pgfusepath{stroke}%
\end{pgfscope}%
\begin{pgfscope}%
\pgfsetrectcap%
\pgfsetmiterjoin%
\pgfsetlinewidth{0.803000pt}%
\definecolor{currentstroke}{rgb}{0.501961,0.501961,0.501961}%
\pgfsetstrokecolor{currentstroke}%
\pgfsetdash{}{0pt}%
\pgfpathmoveto{\pgfqpoint{3.214225in}{3.178832in}}%
\pgfpathlineto{\pgfqpoint{3.214225in}{3.933832in}}%
\pgfusepath{stroke}%
\end{pgfscope}%
\begin{pgfscope}%
\pgfsetrectcap%
\pgfsetmiterjoin%
\pgfsetlinewidth{0.803000pt}%
\definecolor{currentstroke}{rgb}{0.501961,0.501961,0.501961}%
\pgfsetstrokecolor{currentstroke}%
\pgfsetdash{}{0pt}%
\pgfpathmoveto{\pgfqpoint{2.051725in}{3.178832in}}%
\pgfpathlineto{\pgfqpoint{3.214225in}{3.178832in}}%
\pgfusepath{stroke}%
\end{pgfscope}%
\begin{pgfscope}%
\pgfsetrectcap%
\pgfsetmiterjoin%
\pgfsetlinewidth{0.803000pt}%
\definecolor{currentstroke}{rgb}{0.501961,0.501961,0.501961}%
\pgfsetstrokecolor{currentstroke}%
\pgfsetdash{}{0pt}%
\pgfpathmoveto{\pgfqpoint{2.051725in}{3.933832in}}%
\pgfpathlineto{\pgfqpoint{3.214225in}{3.933832in}}%
\pgfusepath{stroke}%
\end{pgfscope}%
\begin{pgfscope}%
\pgfsetbuttcap%
\pgfsetmiterjoin%
\definecolor{currentfill}{rgb}{1.000000,1.000000,1.000000}%
\pgfsetfillcolor{currentfill}%
\pgfsetlinewidth{0.000000pt}%
\definecolor{currentstroke}{rgb}{0.000000,0.000000,0.000000}%
\pgfsetstrokecolor{currentstroke}%
\pgfsetstrokeopacity{0.000000}%
\pgfsetdash{}{0pt}%
\pgfpathmoveto{\pgfqpoint{3.214225in}{3.178832in}}%
\pgfpathlineto{\pgfqpoint{4.376725in}{3.178832in}}%
\pgfpathlineto{\pgfqpoint{4.376725in}{3.933832in}}%
\pgfpathlineto{\pgfqpoint{3.214225in}{3.933832in}}%
\pgfpathclose%
\pgfusepath{fill}%
\end{pgfscope}%
\begin{pgfscope}%
\pgfpathrectangle{\pgfqpoint{3.214225in}{3.178832in}}{\pgfqpoint{1.162500in}{0.755000in}}%
\pgfusepath{clip}%
\pgfsetbuttcap%
\pgfsetroundjoin%
\definecolor{currentfill}{rgb}{0.000000,0.000000,0.000000}%
\pgfsetfillcolor{currentfill}%
\pgfsetfillopacity{0.500000}%
\pgfsetlinewidth{0.000000pt}%
\definecolor{currentstroke}{rgb}{0.000000,0.000000,0.000000}%
\pgfsetstrokecolor{currentstroke}%
\pgfsetdash{}{0pt}%
\pgfpathmoveto{\pgfqpoint{3.656364in}{3.810398in}}%
\pgfpathcurveto{\pgfqpoint{3.661889in}{3.810398in}}{\pgfqpoint{3.667189in}{3.812593in}}{\pgfqpoint{3.671096in}{3.816499in}}%
\pgfpathcurveto{\pgfqpoint{3.675002in}{3.820406in}}{\pgfqpoint{3.677198in}{3.825706in}}{\pgfqpoint{3.677198in}{3.831231in}}%
\pgfpathcurveto{\pgfqpoint{3.677198in}{3.836756in}}{\pgfqpoint{3.675002in}{3.842055in}}{\pgfqpoint{3.671096in}{3.845962in}}%
\pgfpathcurveto{\pgfqpoint{3.667189in}{3.849869in}}{\pgfqpoint{3.661889in}{3.852064in}}{\pgfqpoint{3.656364in}{3.852064in}}%
\pgfpathcurveto{\pgfqpoint{3.650839in}{3.852064in}}{\pgfqpoint{3.645540in}{3.849869in}}{\pgfqpoint{3.641633in}{3.845962in}}%
\pgfpathcurveto{\pgfqpoint{3.637726in}{3.842055in}}{\pgfqpoint{3.635531in}{3.836756in}}{\pgfqpoint{3.635531in}{3.831231in}}%
\pgfpathcurveto{\pgfqpoint{3.635531in}{3.825706in}}{\pgfqpoint{3.637726in}{3.820406in}}{\pgfqpoint{3.641633in}{3.816499in}}%
\pgfpathcurveto{\pgfqpoint{3.645540in}{3.812593in}}{\pgfqpoint{3.650839in}{3.810398in}}{\pgfqpoint{3.656364in}{3.810398in}}%
\pgfpathclose%
\pgfusepath{fill}%
\end{pgfscope}%
\begin{pgfscope}%
\pgfpathrectangle{\pgfqpoint{3.214225in}{3.178832in}}{\pgfqpoint{1.162500in}{0.755000in}}%
\pgfusepath{clip}%
\pgfsetbuttcap%
\pgfsetroundjoin%
\definecolor{currentfill}{rgb}{0.000000,0.000000,0.000000}%
\pgfsetfillcolor{currentfill}%
\pgfsetfillopacity{0.500000}%
\pgfsetlinewidth{0.000000pt}%
\definecolor{currentstroke}{rgb}{0.000000,0.000000,0.000000}%
\pgfsetstrokecolor{currentstroke}%
\pgfsetdash{}{0pt}%
\pgfpathmoveto{\pgfqpoint{4.259629in}{3.539667in}}%
\pgfpathcurveto{\pgfqpoint{4.265154in}{3.539667in}}{\pgfqpoint{4.270454in}{3.541862in}}{\pgfqpoint{4.274361in}{3.545769in}}%
\pgfpathcurveto{\pgfqpoint{4.278268in}{3.549675in}}{\pgfqpoint{4.280463in}{3.554975in}}{\pgfqpoint{4.280463in}{3.560500in}}%
\pgfpathcurveto{\pgfqpoint{4.280463in}{3.566025in}}{\pgfqpoint{4.278268in}{3.571325in}}{\pgfqpoint{4.274361in}{3.575231in}}%
\pgfpathcurveto{\pgfqpoint{4.270454in}{3.579138in}}{\pgfqpoint{4.265154in}{3.581333in}}{\pgfqpoint{4.259629in}{3.581333in}}%
\pgfpathcurveto{\pgfqpoint{4.254104in}{3.581333in}}{\pgfqpoint{4.248805in}{3.579138in}}{\pgfqpoint{4.244898in}{3.575231in}}%
\pgfpathcurveto{\pgfqpoint{4.240991in}{3.571325in}}{\pgfqpoint{4.238796in}{3.566025in}}{\pgfqpoint{4.238796in}{3.560500in}}%
\pgfpathcurveto{\pgfqpoint{4.238796in}{3.554975in}}{\pgfqpoint{4.240991in}{3.549675in}}{\pgfqpoint{4.244898in}{3.545769in}}%
\pgfpathcurveto{\pgfqpoint{4.248805in}{3.541862in}}{\pgfqpoint{4.254104in}{3.539667in}}{\pgfqpoint{4.259629in}{3.539667in}}%
\pgfpathclose%
\pgfusepath{fill}%
\end{pgfscope}%
\begin{pgfscope}%
\pgfpathrectangle{\pgfqpoint{3.214225in}{3.178832in}}{\pgfqpoint{1.162500in}{0.755000in}}%
\pgfusepath{clip}%
\pgfsetbuttcap%
\pgfsetroundjoin%
\definecolor{currentfill}{rgb}{0.000000,0.000000,0.000000}%
\pgfsetfillcolor{currentfill}%
\pgfsetfillopacity{0.500000}%
\pgfsetlinewidth{0.000000pt}%
\definecolor{currentstroke}{rgb}{0.000000,0.000000,0.000000}%
\pgfsetstrokecolor{currentstroke}%
\pgfsetdash{}{0pt}%
\pgfpathmoveto{\pgfqpoint{4.349046in}{3.541019in}}%
\pgfpathcurveto{\pgfqpoint{4.354571in}{3.541019in}}{\pgfqpoint{4.359871in}{3.543214in}}{\pgfqpoint{4.363778in}{3.547121in}}%
\pgfpathcurveto{\pgfqpoint{4.367685in}{3.551028in}}{\pgfqpoint{4.369880in}{3.556327in}}{\pgfqpoint{4.369880in}{3.561852in}}%
\pgfpathcurveto{\pgfqpoint{4.369880in}{3.567377in}}{\pgfqpoint{4.367685in}{3.572677in}}{\pgfqpoint{4.363778in}{3.576584in}}%
\pgfpathcurveto{\pgfqpoint{4.359871in}{3.580490in}}{\pgfqpoint{4.354571in}{3.582685in}}{\pgfqpoint{4.349046in}{3.582685in}}%
\pgfpathcurveto{\pgfqpoint{4.343521in}{3.582685in}}{\pgfqpoint{4.338222in}{3.580490in}}{\pgfqpoint{4.334315in}{3.576584in}}%
\pgfpathcurveto{\pgfqpoint{4.330408in}{3.572677in}}{\pgfqpoint{4.328213in}{3.567377in}}{\pgfqpoint{4.328213in}{3.561852in}}%
\pgfpathcurveto{\pgfqpoint{4.328213in}{3.556327in}}{\pgfqpoint{4.330408in}{3.551028in}}{\pgfqpoint{4.334315in}{3.547121in}}%
\pgfpathcurveto{\pgfqpoint{4.338222in}{3.543214in}}{\pgfqpoint{4.343521in}{3.541019in}}{\pgfqpoint{4.349046in}{3.541019in}}%
\pgfpathclose%
\pgfusepath{fill}%
\end{pgfscope}%
\begin{pgfscope}%
\pgfpathrectangle{\pgfqpoint{3.214225in}{3.178832in}}{\pgfqpoint{1.162500in}{0.755000in}}%
\pgfusepath{clip}%
\pgfsetbuttcap%
\pgfsetroundjoin%
\definecolor{currentfill}{rgb}{0.000000,0.000000,0.000000}%
\pgfsetfillcolor{currentfill}%
\pgfsetfillopacity{0.500000}%
\pgfsetlinewidth{0.000000pt}%
\definecolor{currentstroke}{rgb}{0.000000,0.000000,0.000000}%
\pgfsetstrokecolor{currentstroke}%
\pgfsetdash{}{0pt}%
\pgfpathmoveto{\pgfqpoint{3.915823in}{3.308342in}}%
\pgfpathcurveto{\pgfqpoint{3.921348in}{3.308342in}}{\pgfqpoint{3.926648in}{3.310537in}}{\pgfqpoint{3.930554in}{3.314444in}}%
\pgfpathcurveto{\pgfqpoint{3.934461in}{3.318351in}}{\pgfqpoint{3.936656in}{3.323651in}}{\pgfqpoint{3.936656in}{3.329176in}}%
\pgfpathcurveto{\pgfqpoint{3.936656in}{3.334701in}}{\pgfqpoint{3.934461in}{3.340000in}}{\pgfqpoint{3.930554in}{3.343907in}}%
\pgfpathcurveto{\pgfqpoint{3.926648in}{3.347814in}}{\pgfqpoint{3.921348in}{3.350009in}}{\pgfqpoint{3.915823in}{3.350009in}}%
\pgfpathcurveto{\pgfqpoint{3.910298in}{3.350009in}}{\pgfqpoint{3.904998in}{3.347814in}}{\pgfqpoint{3.901092in}{3.343907in}}%
\pgfpathcurveto{\pgfqpoint{3.897185in}{3.340000in}}{\pgfqpoint{3.894990in}{3.334701in}}{\pgfqpoint{3.894990in}{3.329176in}}%
\pgfpathcurveto{\pgfqpoint{3.894990in}{3.323651in}}{\pgfqpoint{3.897185in}{3.318351in}}{\pgfqpoint{3.901092in}{3.314444in}}%
\pgfpathcurveto{\pgfqpoint{3.904998in}{3.310537in}}{\pgfqpoint{3.910298in}{3.308342in}}{\pgfqpoint{3.915823in}{3.308342in}}%
\pgfpathclose%
\pgfusepath{fill}%
\end{pgfscope}%
\begin{pgfscope}%
\pgfpathrectangle{\pgfqpoint{3.214225in}{3.178832in}}{\pgfqpoint{1.162500in}{0.755000in}}%
\pgfusepath{clip}%
\pgfsetbuttcap%
\pgfsetroundjoin%
\definecolor{currentfill}{rgb}{0.000000,0.000000,0.000000}%
\pgfsetfillcolor{currentfill}%
\pgfsetfillopacity{0.500000}%
\pgfsetlinewidth{0.000000pt}%
\definecolor{currentstroke}{rgb}{0.000000,0.000000,0.000000}%
\pgfsetstrokecolor{currentstroke}%
\pgfsetdash{}{0pt}%
\pgfpathmoveto{\pgfqpoint{3.986481in}{3.311937in}}%
\pgfpathcurveto{\pgfqpoint{3.992006in}{3.311937in}}{\pgfqpoint{3.997306in}{3.314132in}}{\pgfqpoint{4.001213in}{3.318039in}}%
\pgfpathcurveto{\pgfqpoint{4.005120in}{3.321945in}}{\pgfqpoint{4.007315in}{3.327245in}}{\pgfqpoint{4.007315in}{3.332770in}}%
\pgfpathcurveto{\pgfqpoint{4.007315in}{3.338295in}}{\pgfqpoint{4.005120in}{3.343595in}}{\pgfqpoint{4.001213in}{3.347501in}}%
\pgfpathcurveto{\pgfqpoint{3.997306in}{3.351408in}}{\pgfqpoint{3.992006in}{3.353603in}}{\pgfqpoint{3.986481in}{3.353603in}}%
\pgfpathcurveto{\pgfqpoint{3.980956in}{3.353603in}}{\pgfqpoint{3.975657in}{3.351408in}}{\pgfqpoint{3.971750in}{3.347501in}}%
\pgfpathcurveto{\pgfqpoint{3.967843in}{3.343595in}}{\pgfqpoint{3.965648in}{3.338295in}}{\pgfqpoint{3.965648in}{3.332770in}}%
\pgfpathcurveto{\pgfqpoint{3.965648in}{3.327245in}}{\pgfqpoint{3.967843in}{3.321945in}}{\pgfqpoint{3.971750in}{3.318039in}}%
\pgfpathcurveto{\pgfqpoint{3.975657in}{3.314132in}}{\pgfqpoint{3.980956in}{3.311937in}}{\pgfqpoint{3.986481in}{3.311937in}}%
\pgfpathclose%
\pgfusepath{fill}%
\end{pgfscope}%
\begin{pgfscope}%
\pgfpathrectangle{\pgfqpoint{3.214225in}{3.178832in}}{\pgfqpoint{1.162500in}{0.755000in}}%
\pgfusepath{clip}%
\pgfsetbuttcap%
\pgfsetroundjoin%
\definecolor{currentfill}{rgb}{0.000000,0.000000,0.000000}%
\pgfsetfillcolor{currentfill}%
\pgfsetfillopacity{0.500000}%
\pgfsetlinewidth{0.000000pt}%
\definecolor{currentstroke}{rgb}{0.000000,0.000000,0.000000}%
\pgfsetstrokecolor{currentstroke}%
\pgfsetdash{}{0pt}%
\pgfpathmoveto{\pgfqpoint{3.831974in}{3.215463in}}%
\pgfpathcurveto{\pgfqpoint{3.837499in}{3.215463in}}{\pgfqpoint{3.842798in}{3.217658in}}{\pgfqpoint{3.846705in}{3.221565in}}%
\pgfpathcurveto{\pgfqpoint{3.850612in}{3.225471in}}{\pgfqpoint{3.852807in}{3.230771in}}{\pgfqpoint{3.852807in}{3.236296in}}%
\pgfpathcurveto{\pgfqpoint{3.852807in}{3.241821in}}{\pgfqpoint{3.850612in}{3.247121in}}{\pgfqpoint{3.846705in}{3.251027in}}%
\pgfpathcurveto{\pgfqpoint{3.842798in}{3.254934in}}{\pgfqpoint{3.837499in}{3.257129in}}{\pgfqpoint{3.831974in}{3.257129in}}%
\pgfpathcurveto{\pgfqpoint{3.826449in}{3.257129in}}{\pgfqpoint{3.821149in}{3.254934in}}{\pgfqpoint{3.817242in}{3.251027in}}%
\pgfpathcurveto{\pgfqpoint{3.813336in}{3.247121in}}{\pgfqpoint{3.811140in}{3.241821in}}{\pgfqpoint{3.811140in}{3.236296in}}%
\pgfpathcurveto{\pgfqpoint{3.811140in}{3.230771in}}{\pgfqpoint{3.813336in}{3.225471in}}{\pgfqpoint{3.817242in}{3.221565in}}%
\pgfpathcurveto{\pgfqpoint{3.821149in}{3.217658in}}{\pgfqpoint{3.826449in}{3.215463in}}{\pgfqpoint{3.831974in}{3.215463in}}%
\pgfpathclose%
\pgfusepath{fill}%
\end{pgfscope}%
\begin{pgfscope}%
\pgfpathrectangle{\pgfqpoint{3.214225in}{3.178832in}}{\pgfqpoint{1.162500in}{0.755000in}}%
\pgfusepath{clip}%
\pgfsetbuttcap%
\pgfsetroundjoin%
\definecolor{currentfill}{rgb}{0.000000,0.000000,0.000000}%
\pgfsetfillcolor{currentfill}%
\pgfsetfillopacity{0.500000}%
\pgfsetlinewidth{0.000000pt}%
\definecolor{currentstroke}{rgb}{0.000000,0.000000,0.000000}%
\pgfsetstrokecolor{currentstroke}%
\pgfsetdash{}{0pt}%
\pgfpathmoveto{\pgfqpoint{3.241904in}{3.175975in}}%
\pgfpathcurveto{\pgfqpoint{3.247429in}{3.175975in}}{\pgfqpoint{3.252728in}{3.178170in}}{\pgfqpoint{3.256635in}{3.182077in}}%
\pgfpathcurveto{\pgfqpoint{3.260542in}{3.185984in}}{\pgfqpoint{3.262737in}{3.191283in}}{\pgfqpoint{3.262737in}{3.196809in}}%
\pgfpathcurveto{\pgfqpoint{3.262737in}{3.202334in}}{\pgfqpoint{3.260542in}{3.207633in}}{\pgfqpoint{3.256635in}{3.211540in}}%
\pgfpathcurveto{\pgfqpoint{3.252728in}{3.215447in}}{\pgfqpoint{3.247429in}{3.217642in}}{\pgfqpoint{3.241904in}{3.217642in}}%
\pgfpathcurveto{\pgfqpoint{3.236379in}{3.217642in}}{\pgfqpoint{3.231079in}{3.215447in}}{\pgfqpoint{3.227172in}{3.211540in}}%
\pgfpathcurveto{\pgfqpoint{3.223265in}{3.207633in}}{\pgfqpoint{3.221070in}{3.202334in}}{\pgfqpoint{3.221070in}{3.196809in}}%
\pgfpathcurveto{\pgfqpoint{3.221070in}{3.191283in}}{\pgfqpoint{3.223265in}{3.185984in}}{\pgfqpoint{3.227172in}{3.182077in}}%
\pgfpathcurveto{\pgfqpoint{3.231079in}{3.178170in}}{\pgfqpoint{3.236379in}{3.175975in}}{\pgfqpoint{3.241904in}{3.175975in}}%
\pgfpathclose%
\pgfusepath{fill}%
\end{pgfscope}%
\begin{pgfscope}%
\pgfpathrectangle{\pgfqpoint{3.214225in}{3.178832in}}{\pgfqpoint{1.162500in}{0.755000in}}%
\pgfusepath{clip}%
\pgfsetbuttcap%
\pgfsetroundjoin%
\definecolor{currentfill}{rgb}{0.000000,0.000000,0.000000}%
\pgfsetfillcolor{currentfill}%
\pgfsetfillopacity{0.500000}%
\pgfsetlinewidth{0.000000pt}%
\definecolor{currentstroke}{rgb}{0.000000,0.000000,0.000000}%
\pgfsetstrokecolor{currentstroke}%
\pgfsetdash{}{0pt}%
\pgfpathmoveto{\pgfqpoint{4.220724in}{3.678982in}}%
\pgfpathcurveto{\pgfqpoint{4.226249in}{3.678982in}}{\pgfqpoint{4.231548in}{3.681177in}}{\pgfqpoint{4.235455in}{3.685084in}}%
\pgfpathcurveto{\pgfqpoint{4.239362in}{3.688990in}}{\pgfqpoint{4.241557in}{3.694290in}}{\pgfqpoint{4.241557in}{3.699815in}}%
\pgfpathcurveto{\pgfqpoint{4.241557in}{3.705340in}}{\pgfqpoint{4.239362in}{3.710640in}}{\pgfqpoint{4.235455in}{3.714546in}}%
\pgfpathcurveto{\pgfqpoint{4.231548in}{3.718453in}}{\pgfqpoint{4.226249in}{3.720648in}}{\pgfqpoint{4.220724in}{3.720648in}}%
\pgfpathcurveto{\pgfqpoint{4.215199in}{3.720648in}}{\pgfqpoint{4.209899in}{3.718453in}}{\pgfqpoint{4.205992in}{3.714546in}}%
\pgfpathcurveto{\pgfqpoint{4.202086in}{3.710640in}}{\pgfqpoint{4.199890in}{3.705340in}}{\pgfqpoint{4.199890in}{3.699815in}}%
\pgfpathcurveto{\pgfqpoint{4.199890in}{3.694290in}}{\pgfqpoint{4.202086in}{3.688990in}}{\pgfqpoint{4.205992in}{3.685084in}}%
\pgfpathcurveto{\pgfqpoint{4.209899in}{3.681177in}}{\pgfqpoint{4.215199in}{3.678982in}}{\pgfqpoint{4.220724in}{3.678982in}}%
\pgfpathclose%
\pgfusepath{fill}%
\end{pgfscope}%
\begin{pgfscope}%
\pgfpathrectangle{\pgfqpoint{3.214225in}{3.178832in}}{\pgfqpoint{1.162500in}{0.755000in}}%
\pgfusepath{clip}%
\pgfsetbuttcap%
\pgfsetroundjoin%
\definecolor{currentfill}{rgb}{0.000000,0.000000,0.000000}%
\pgfsetfillcolor{currentfill}%
\pgfsetfillopacity{0.500000}%
\pgfsetlinewidth{0.000000pt}%
\definecolor{currentstroke}{rgb}{0.000000,0.000000,0.000000}%
\pgfsetstrokecolor{currentstroke}%
\pgfsetdash{}{0pt}%
\pgfpathmoveto{\pgfqpoint{3.980838in}{3.528997in}}%
\pgfpathcurveto{\pgfqpoint{3.986364in}{3.528997in}}{\pgfqpoint{3.991663in}{3.531192in}}{\pgfqpoint{3.995570in}{3.535099in}}%
\pgfpathcurveto{\pgfqpoint{3.999477in}{3.539005in}}{\pgfqpoint{4.001672in}{3.544305in}}{\pgfqpoint{4.001672in}{3.549830in}}%
\pgfpathcurveto{\pgfqpoint{4.001672in}{3.555355in}}{\pgfqpoint{3.999477in}{3.560655in}}{\pgfqpoint{3.995570in}{3.564561in}}%
\pgfpathcurveto{\pgfqpoint{3.991663in}{3.568468in}}{\pgfqpoint{3.986364in}{3.570663in}}{\pgfqpoint{3.980838in}{3.570663in}}%
\pgfpathcurveto{\pgfqpoint{3.975313in}{3.570663in}}{\pgfqpoint{3.970014in}{3.568468in}}{\pgfqpoint{3.966107in}{3.564561in}}%
\pgfpathcurveto{\pgfqpoint{3.962200in}{3.560655in}}{\pgfqpoint{3.960005in}{3.555355in}}{\pgfqpoint{3.960005in}{3.549830in}}%
\pgfpathcurveto{\pgfqpoint{3.960005in}{3.544305in}}{\pgfqpoint{3.962200in}{3.539005in}}{\pgfqpoint{3.966107in}{3.535099in}}%
\pgfpathcurveto{\pgfqpoint{3.970014in}{3.531192in}}{\pgfqpoint{3.975313in}{3.528997in}}{\pgfqpoint{3.980838in}{3.528997in}}%
\pgfpathclose%
\pgfusepath{fill}%
\end{pgfscope}%
\begin{pgfscope}%
\pgfpathrectangle{\pgfqpoint{3.214225in}{3.178832in}}{\pgfqpoint{1.162500in}{0.755000in}}%
\pgfusepath{clip}%
\pgfsetbuttcap%
\pgfsetroundjoin%
\definecolor{currentfill}{rgb}{0.000000,0.000000,0.000000}%
\pgfsetfillcolor{currentfill}%
\pgfsetfillopacity{0.500000}%
\pgfsetlinewidth{0.000000pt}%
\definecolor{currentstroke}{rgb}{0.000000,0.000000,0.000000}%
\pgfsetstrokecolor{currentstroke}%
\pgfsetdash{}{0pt}%
\pgfpathmoveto{\pgfqpoint{3.708643in}{3.426855in}}%
\pgfpathcurveto{\pgfqpoint{3.714168in}{3.426855in}}{\pgfqpoint{3.719468in}{3.429051in}}{\pgfqpoint{3.723375in}{3.432957in}}%
\pgfpathcurveto{\pgfqpoint{3.727282in}{3.436864in}}{\pgfqpoint{3.729477in}{3.442164in}}{\pgfqpoint{3.729477in}{3.447689in}}%
\pgfpathcurveto{\pgfqpoint{3.729477in}{3.453214in}}{\pgfqpoint{3.727282in}{3.458513in}}{\pgfqpoint{3.723375in}{3.462420in}}%
\pgfpathcurveto{\pgfqpoint{3.719468in}{3.466327in}}{\pgfqpoint{3.714168in}{3.468522in}}{\pgfqpoint{3.708643in}{3.468522in}}%
\pgfpathcurveto{\pgfqpoint{3.703118in}{3.468522in}}{\pgfqpoint{3.697819in}{3.466327in}}{\pgfqpoint{3.693912in}{3.462420in}}%
\pgfpathcurveto{\pgfqpoint{3.690005in}{3.458513in}}{\pgfqpoint{3.687810in}{3.453214in}}{\pgfqpoint{3.687810in}{3.447689in}}%
\pgfpathcurveto{\pgfqpoint{3.687810in}{3.442164in}}{\pgfqpoint{3.690005in}{3.436864in}}{\pgfqpoint{3.693912in}{3.432957in}}%
\pgfpathcurveto{\pgfqpoint{3.697819in}{3.429051in}}{\pgfqpoint{3.703118in}{3.426855in}}{\pgfqpoint{3.708643in}{3.426855in}}%
\pgfpathclose%
\pgfusepath{fill}%
\end{pgfscope}%
\begin{pgfscope}%
\pgfpathrectangle{\pgfqpoint{3.214225in}{3.178832in}}{\pgfqpoint{1.162500in}{0.755000in}}%
\pgfusepath{clip}%
\pgfsetbuttcap%
\pgfsetroundjoin%
\definecolor{currentfill}{rgb}{0.000000,0.000000,0.000000}%
\pgfsetfillcolor{currentfill}%
\pgfsetfillopacity{0.500000}%
\pgfsetlinewidth{0.000000pt}%
\definecolor{currentstroke}{rgb}{0.000000,0.000000,0.000000}%
\pgfsetstrokecolor{currentstroke}%
\pgfsetdash{}{0pt}%
\pgfpathmoveto{\pgfqpoint{4.124371in}{3.677650in}}%
\pgfpathcurveto{\pgfqpoint{4.129896in}{3.677650in}}{\pgfqpoint{4.135195in}{3.679845in}}{\pgfqpoint{4.139102in}{3.683752in}}%
\pgfpathcurveto{\pgfqpoint{4.143009in}{3.687659in}}{\pgfqpoint{4.145204in}{3.692959in}}{\pgfqpoint{4.145204in}{3.698484in}}%
\pgfpathcurveto{\pgfqpoint{4.145204in}{3.704009in}}{\pgfqpoint{4.143009in}{3.709308in}}{\pgfqpoint{4.139102in}{3.713215in}}%
\pgfpathcurveto{\pgfqpoint{4.135195in}{3.717122in}}{\pgfqpoint{4.129896in}{3.719317in}}{\pgfqpoint{4.124371in}{3.719317in}}%
\pgfpathcurveto{\pgfqpoint{4.118846in}{3.719317in}}{\pgfqpoint{4.113546in}{3.717122in}}{\pgfqpoint{4.109639in}{3.713215in}}%
\pgfpathcurveto{\pgfqpoint{4.105732in}{3.709308in}}{\pgfqpoint{4.103537in}{3.704009in}}{\pgfqpoint{4.103537in}{3.698484in}}%
\pgfpathcurveto{\pgfqpoint{4.103537in}{3.692959in}}{\pgfqpoint{4.105732in}{3.687659in}}{\pgfqpoint{4.109639in}{3.683752in}}%
\pgfpathcurveto{\pgfqpoint{4.113546in}{3.679845in}}{\pgfqpoint{4.118846in}{3.677650in}}{\pgfqpoint{4.124371in}{3.677650in}}%
\pgfpathclose%
\pgfusepath{fill}%
\end{pgfscope}%
\begin{pgfscope}%
\pgfpathrectangle{\pgfqpoint{3.214225in}{3.178832in}}{\pgfqpoint{1.162500in}{0.755000in}}%
\pgfusepath{clip}%
\pgfsetbuttcap%
\pgfsetroundjoin%
\definecolor{currentfill}{rgb}{0.000000,0.000000,0.000000}%
\pgfsetfillcolor{currentfill}%
\pgfsetfillopacity{0.500000}%
\pgfsetlinewidth{0.000000pt}%
\definecolor{currentstroke}{rgb}{0.000000,0.000000,0.000000}%
\pgfsetstrokecolor{currentstroke}%
\pgfsetdash{}{0pt}%
\pgfpathmoveto{\pgfqpoint{3.695649in}{3.296446in}}%
\pgfpathcurveto{\pgfqpoint{3.701174in}{3.296446in}}{\pgfqpoint{3.706474in}{3.298641in}}{\pgfqpoint{3.710381in}{3.302548in}}%
\pgfpathcurveto{\pgfqpoint{3.714288in}{3.306455in}}{\pgfqpoint{3.716483in}{3.311754in}}{\pgfqpoint{3.716483in}{3.317279in}}%
\pgfpathcurveto{\pgfqpoint{3.716483in}{3.322804in}}{\pgfqpoint{3.714288in}{3.328104in}}{\pgfqpoint{3.710381in}{3.332011in}}%
\pgfpathcurveto{\pgfqpoint{3.706474in}{3.335917in}}{\pgfqpoint{3.701174in}{3.338112in}}{\pgfqpoint{3.695649in}{3.338112in}}%
\pgfpathcurveto{\pgfqpoint{3.690124in}{3.338112in}}{\pgfqpoint{3.684825in}{3.335917in}}{\pgfqpoint{3.680918in}{3.332011in}}%
\pgfpathcurveto{\pgfqpoint{3.677011in}{3.328104in}}{\pgfqpoint{3.674816in}{3.322804in}}{\pgfqpoint{3.674816in}{3.317279in}}%
\pgfpathcurveto{\pgfqpoint{3.674816in}{3.311754in}}{\pgfqpoint{3.677011in}{3.306455in}}{\pgfqpoint{3.680918in}{3.302548in}}%
\pgfpathcurveto{\pgfqpoint{3.684825in}{3.298641in}}{\pgfqpoint{3.690124in}{3.296446in}}{\pgfqpoint{3.695649in}{3.296446in}}%
\pgfpathclose%
\pgfusepath{fill}%
\end{pgfscope}%
\begin{pgfscope}%
\pgfpathrectangle{\pgfqpoint{3.214225in}{3.178832in}}{\pgfqpoint{1.162500in}{0.755000in}}%
\pgfusepath{clip}%
\pgfsetbuttcap%
\pgfsetroundjoin%
\definecolor{currentfill}{rgb}{0.000000,0.000000,0.000000}%
\pgfsetfillcolor{currentfill}%
\pgfsetfillopacity{0.500000}%
\pgfsetlinewidth{0.000000pt}%
\definecolor{currentstroke}{rgb}{0.000000,0.000000,0.000000}%
\pgfsetstrokecolor{currentstroke}%
\pgfsetdash{}{0pt}%
\pgfpathmoveto{\pgfqpoint{3.329643in}{3.322937in}}%
\pgfpathcurveto{\pgfqpoint{3.335168in}{3.322937in}}{\pgfqpoint{3.340468in}{3.325133in}}{\pgfqpoint{3.344375in}{3.329039in}}%
\pgfpathcurveto{\pgfqpoint{3.348281in}{3.332946in}}{\pgfqpoint{3.350476in}{3.338246in}}{\pgfqpoint{3.350476in}{3.343771in}}%
\pgfpathcurveto{\pgfqpoint{3.350476in}{3.349296in}}{\pgfqpoint{3.348281in}{3.354595in}}{\pgfqpoint{3.344375in}{3.358502in}}%
\pgfpathcurveto{\pgfqpoint{3.340468in}{3.362409in}}{\pgfqpoint{3.335168in}{3.364604in}}{\pgfqpoint{3.329643in}{3.364604in}}%
\pgfpathcurveto{\pgfqpoint{3.324118in}{3.364604in}}{\pgfqpoint{3.318819in}{3.362409in}}{\pgfqpoint{3.314912in}{3.358502in}}%
\pgfpathcurveto{\pgfqpoint{3.311005in}{3.354595in}}{\pgfqpoint{3.308810in}{3.349296in}}{\pgfqpoint{3.308810in}{3.343771in}}%
\pgfpathcurveto{\pgfqpoint{3.308810in}{3.338246in}}{\pgfqpoint{3.311005in}{3.332946in}}{\pgfqpoint{3.314912in}{3.329039in}}%
\pgfpathcurveto{\pgfqpoint{3.318819in}{3.325133in}}{\pgfqpoint{3.324118in}{3.322937in}}{\pgfqpoint{3.329643in}{3.322937in}}%
\pgfpathclose%
\pgfusepath{fill}%
\end{pgfscope}%
\begin{pgfscope}%
\pgfpathrectangle{\pgfqpoint{3.214225in}{3.178832in}}{\pgfqpoint{1.162500in}{0.755000in}}%
\pgfusepath{clip}%
\pgfsetbuttcap%
\pgfsetroundjoin%
\definecolor{currentfill}{rgb}{0.000000,0.000000,0.000000}%
\pgfsetfillcolor{currentfill}%
\pgfsetfillopacity{0.500000}%
\pgfsetlinewidth{0.000000pt}%
\definecolor{currentstroke}{rgb}{0.000000,0.000000,0.000000}%
\pgfsetstrokecolor{currentstroke}%
\pgfsetdash{}{0pt}%
\pgfpathmoveto{\pgfqpoint{4.167862in}{3.895023in}}%
\pgfpathcurveto{\pgfqpoint{4.173387in}{3.895023in}}{\pgfqpoint{4.178687in}{3.897218in}}{\pgfqpoint{4.182593in}{3.901125in}}%
\pgfpathcurveto{\pgfqpoint{4.186500in}{3.905032in}}{\pgfqpoint{4.188695in}{3.910331in}}{\pgfqpoint{4.188695in}{3.915856in}}%
\pgfpathcurveto{\pgfqpoint{4.188695in}{3.921381in}}{\pgfqpoint{4.186500in}{3.926681in}}{\pgfqpoint{4.182593in}{3.930588in}}%
\pgfpathcurveto{\pgfqpoint{4.178687in}{3.934494in}}{\pgfqpoint{4.173387in}{3.936690in}}{\pgfqpoint{4.167862in}{3.936690in}}%
\pgfpathcurveto{\pgfqpoint{4.162337in}{3.936690in}}{\pgfqpoint{4.157037in}{3.934494in}}{\pgfqpoint{4.153131in}{3.930588in}}%
\pgfpathcurveto{\pgfqpoint{4.149224in}{3.926681in}}{\pgfqpoint{4.147029in}{3.921381in}}{\pgfqpoint{4.147029in}{3.915856in}}%
\pgfpathcurveto{\pgfqpoint{4.147029in}{3.910331in}}{\pgfqpoint{4.149224in}{3.905032in}}{\pgfqpoint{4.153131in}{3.901125in}}%
\pgfpathcurveto{\pgfqpoint{4.157037in}{3.897218in}}{\pgfqpoint{4.162337in}{3.895023in}}{\pgfqpoint{4.167862in}{3.895023in}}%
\pgfpathclose%
\pgfusepath{fill}%
\end{pgfscope}%
\begin{pgfscope}%
\pgfpathrectangle{\pgfqpoint{3.214225in}{3.178832in}}{\pgfqpoint{1.162500in}{0.755000in}}%
\pgfusepath{clip}%
\pgfsetbuttcap%
\pgfsetroundjoin%
\definecolor{currentfill}{rgb}{0.000000,0.000000,0.000000}%
\pgfsetfillcolor{currentfill}%
\pgfsetfillopacity{0.500000}%
\pgfsetlinewidth{0.000000pt}%
\definecolor{currentstroke}{rgb}{0.000000,0.000000,0.000000}%
\pgfsetstrokecolor{currentstroke}%
\pgfsetdash{}{0pt}%
\pgfpathmoveto{\pgfqpoint{4.120816in}{3.491212in}}%
\pgfpathcurveto{\pgfqpoint{4.126341in}{3.491212in}}{\pgfqpoint{4.131641in}{3.493407in}}{\pgfqpoint{4.135548in}{3.497314in}}%
\pgfpathcurveto{\pgfqpoint{4.139455in}{3.501221in}}{\pgfqpoint{4.141650in}{3.506520in}}{\pgfqpoint{4.141650in}{3.512045in}}%
\pgfpathcurveto{\pgfqpoint{4.141650in}{3.517571in}}{\pgfqpoint{4.139455in}{3.522870in}}{\pgfqpoint{4.135548in}{3.526777in}}%
\pgfpathcurveto{\pgfqpoint{4.131641in}{3.530684in}}{\pgfqpoint{4.126341in}{3.532879in}}{\pgfqpoint{4.120816in}{3.532879in}}%
\pgfpathcurveto{\pgfqpoint{4.115291in}{3.532879in}}{\pgfqpoint{4.109992in}{3.530684in}}{\pgfqpoint{4.106085in}{3.526777in}}%
\pgfpathcurveto{\pgfqpoint{4.102178in}{3.522870in}}{\pgfqpoint{4.099983in}{3.517571in}}{\pgfqpoint{4.099983in}{3.512045in}}%
\pgfpathcurveto{\pgfqpoint{4.099983in}{3.506520in}}{\pgfqpoint{4.102178in}{3.501221in}}{\pgfqpoint{4.106085in}{3.497314in}}%
\pgfpathcurveto{\pgfqpoint{4.109992in}{3.493407in}}{\pgfqpoint{4.115291in}{3.491212in}}{\pgfqpoint{4.120816in}{3.491212in}}%
\pgfpathclose%
\pgfusepath{fill}%
\end{pgfscope}%
\begin{pgfscope}%
\pgfpathrectangle{\pgfqpoint{3.214225in}{3.178832in}}{\pgfqpoint{1.162500in}{0.755000in}}%
\pgfusepath{clip}%
\pgfsetbuttcap%
\pgfsetroundjoin%
\definecolor{currentfill}{rgb}{0.000000,0.000000,0.000000}%
\pgfsetfillcolor{currentfill}%
\pgfsetfillopacity{0.500000}%
\pgfsetlinewidth{0.000000pt}%
\definecolor{currentstroke}{rgb}{0.000000,0.000000,0.000000}%
\pgfsetstrokecolor{currentstroke}%
\pgfsetdash{}{0pt}%
\pgfpathmoveto{\pgfqpoint{3.792736in}{3.212721in}}%
\pgfpathcurveto{\pgfqpoint{3.798261in}{3.212721in}}{\pgfqpoint{3.803560in}{3.214917in}}{\pgfqpoint{3.807467in}{3.218823in}}%
\pgfpathcurveto{\pgfqpoint{3.811374in}{3.222730in}}{\pgfqpoint{3.813569in}{3.228030in}}{\pgfqpoint{3.813569in}{3.233555in}}%
\pgfpathcurveto{\pgfqpoint{3.813569in}{3.239080in}}{\pgfqpoint{3.811374in}{3.244379in}}{\pgfqpoint{3.807467in}{3.248286in}}%
\pgfpathcurveto{\pgfqpoint{3.803560in}{3.252193in}}{\pgfqpoint{3.798261in}{3.254388in}}{\pgfqpoint{3.792736in}{3.254388in}}%
\pgfpathcurveto{\pgfqpoint{3.787211in}{3.254388in}}{\pgfqpoint{3.781911in}{3.252193in}}{\pgfqpoint{3.778004in}{3.248286in}}%
\pgfpathcurveto{\pgfqpoint{3.774097in}{3.244379in}}{\pgfqpoint{3.771902in}{3.239080in}}{\pgfqpoint{3.771902in}{3.233555in}}%
\pgfpathcurveto{\pgfqpoint{3.771902in}{3.228030in}}{\pgfqpoint{3.774097in}{3.222730in}}{\pgfqpoint{3.778004in}{3.218823in}}%
\pgfpathcurveto{\pgfqpoint{3.781911in}{3.214917in}}{\pgfqpoint{3.787211in}{3.212721in}}{\pgfqpoint{3.792736in}{3.212721in}}%
\pgfpathclose%
\pgfusepath{fill}%
\end{pgfscope}%
\begin{pgfscope}%
\pgfsetrectcap%
\pgfsetmiterjoin%
\pgfsetlinewidth{0.803000pt}%
\definecolor{currentstroke}{rgb}{0.501961,0.501961,0.501961}%
\pgfsetstrokecolor{currentstroke}%
\pgfsetdash{}{0pt}%
\pgfpathmoveto{\pgfqpoint{3.214225in}{3.178832in}}%
\pgfpathlineto{\pgfqpoint{3.214225in}{3.933832in}}%
\pgfusepath{stroke}%
\end{pgfscope}%
\begin{pgfscope}%
\pgfsetrectcap%
\pgfsetmiterjoin%
\pgfsetlinewidth{0.803000pt}%
\definecolor{currentstroke}{rgb}{0.501961,0.501961,0.501961}%
\pgfsetstrokecolor{currentstroke}%
\pgfsetdash{}{0pt}%
\pgfpathmoveto{\pgfqpoint{4.376725in}{3.178832in}}%
\pgfpathlineto{\pgfqpoint{4.376725in}{3.933832in}}%
\pgfusepath{stroke}%
\end{pgfscope}%
\begin{pgfscope}%
\pgfsetrectcap%
\pgfsetmiterjoin%
\pgfsetlinewidth{0.803000pt}%
\definecolor{currentstroke}{rgb}{0.501961,0.501961,0.501961}%
\pgfsetstrokecolor{currentstroke}%
\pgfsetdash{}{0pt}%
\pgfpathmoveto{\pgfqpoint{3.214225in}{3.178832in}}%
\pgfpathlineto{\pgfqpoint{4.376725in}{3.178832in}}%
\pgfusepath{stroke}%
\end{pgfscope}%
\begin{pgfscope}%
\pgfsetrectcap%
\pgfsetmiterjoin%
\pgfsetlinewidth{0.803000pt}%
\definecolor{currentstroke}{rgb}{0.501961,0.501961,0.501961}%
\pgfsetstrokecolor{currentstroke}%
\pgfsetdash{}{0pt}%
\pgfpathmoveto{\pgfqpoint{3.214225in}{3.933832in}}%
\pgfpathlineto{\pgfqpoint{4.376725in}{3.933832in}}%
\pgfusepath{stroke}%
\end{pgfscope}%
\begin{pgfscope}%
\pgfsetbuttcap%
\pgfsetmiterjoin%
\definecolor{currentfill}{rgb}{1.000000,1.000000,1.000000}%
\pgfsetfillcolor{currentfill}%
\pgfsetlinewidth{0.000000pt}%
\definecolor{currentstroke}{rgb}{0.000000,0.000000,0.000000}%
\pgfsetstrokecolor{currentstroke}%
\pgfsetstrokeopacity{0.000000}%
\pgfsetdash{}{0pt}%
\pgfpathmoveto{\pgfqpoint{4.376725in}{3.178832in}}%
\pgfpathlineto{\pgfqpoint{5.539225in}{3.178832in}}%
\pgfpathlineto{\pgfqpoint{5.539225in}{3.933832in}}%
\pgfpathlineto{\pgfqpoint{4.376725in}{3.933832in}}%
\pgfpathclose%
\pgfusepath{fill}%
\end{pgfscope}%
\begin{pgfscope}%
\pgfpathrectangle{\pgfqpoint{4.376725in}{3.178832in}}{\pgfqpoint{1.162500in}{0.755000in}}%
\pgfusepath{clip}%
\pgfsetbuttcap%
\pgfsetroundjoin%
\definecolor{currentfill}{rgb}{0.000000,0.000000,0.000000}%
\pgfsetfillcolor{currentfill}%
\pgfsetfillopacity{0.500000}%
\pgfsetlinewidth{0.000000pt}%
\definecolor{currentstroke}{rgb}{0.000000,0.000000,0.000000}%
\pgfsetstrokecolor{currentstroke}%
\pgfsetdash{}{0pt}%
\pgfpathmoveto{\pgfqpoint{5.214960in}{3.810398in}}%
\pgfpathcurveto{\pgfqpoint{5.220485in}{3.810398in}}{\pgfqpoint{5.225785in}{3.812593in}}{\pgfqpoint{5.229692in}{3.816499in}}%
\pgfpathcurveto{\pgfqpoint{5.233598in}{3.820406in}}{\pgfqpoint{5.235794in}{3.825706in}}{\pgfqpoint{5.235794in}{3.831231in}}%
\pgfpathcurveto{\pgfqpoint{5.235794in}{3.836756in}}{\pgfqpoint{5.233598in}{3.842055in}}{\pgfqpoint{5.229692in}{3.845962in}}%
\pgfpathcurveto{\pgfqpoint{5.225785in}{3.849869in}}{\pgfqpoint{5.220485in}{3.852064in}}{\pgfqpoint{5.214960in}{3.852064in}}%
\pgfpathcurveto{\pgfqpoint{5.209435in}{3.852064in}}{\pgfqpoint{5.204136in}{3.849869in}}{\pgfqpoint{5.200229in}{3.845962in}}%
\pgfpathcurveto{\pgfqpoint{5.196322in}{3.842055in}}{\pgfqpoint{5.194127in}{3.836756in}}{\pgfqpoint{5.194127in}{3.831231in}}%
\pgfpathcurveto{\pgfqpoint{5.194127in}{3.825706in}}{\pgfqpoint{5.196322in}{3.820406in}}{\pgfqpoint{5.200229in}{3.816499in}}%
\pgfpathcurveto{\pgfqpoint{5.204136in}{3.812593in}}{\pgfqpoint{5.209435in}{3.810398in}}{\pgfqpoint{5.214960in}{3.810398in}}%
\pgfpathclose%
\pgfusepath{fill}%
\end{pgfscope}%
\begin{pgfscope}%
\pgfpathrectangle{\pgfqpoint{4.376725in}{3.178832in}}{\pgfqpoint{1.162500in}{0.755000in}}%
\pgfusepath{clip}%
\pgfsetbuttcap%
\pgfsetroundjoin%
\definecolor{currentfill}{rgb}{0.000000,0.000000,0.000000}%
\pgfsetfillcolor{currentfill}%
\pgfsetfillopacity{0.500000}%
\pgfsetlinewidth{0.000000pt}%
\definecolor{currentstroke}{rgb}{0.000000,0.000000,0.000000}%
\pgfsetstrokecolor{currentstroke}%
\pgfsetdash{}{0pt}%
\pgfpathmoveto{\pgfqpoint{4.521739in}{3.539667in}}%
\pgfpathcurveto{\pgfqpoint{4.527264in}{3.539667in}}{\pgfqpoint{4.532563in}{3.541862in}}{\pgfqpoint{4.536470in}{3.545769in}}%
\pgfpathcurveto{\pgfqpoint{4.540377in}{3.549675in}}{\pgfqpoint{4.542572in}{3.554975in}}{\pgfqpoint{4.542572in}{3.560500in}}%
\pgfpathcurveto{\pgfqpoint{4.542572in}{3.566025in}}{\pgfqpoint{4.540377in}{3.571325in}}{\pgfqpoint{4.536470in}{3.575231in}}%
\pgfpathcurveto{\pgfqpoint{4.532563in}{3.579138in}}{\pgfqpoint{4.527264in}{3.581333in}}{\pgfqpoint{4.521739in}{3.581333in}}%
\pgfpathcurveto{\pgfqpoint{4.516214in}{3.581333in}}{\pgfqpoint{4.510914in}{3.579138in}}{\pgfqpoint{4.507007in}{3.575231in}}%
\pgfpathcurveto{\pgfqpoint{4.503101in}{3.571325in}}{\pgfqpoint{4.500905in}{3.566025in}}{\pgfqpoint{4.500905in}{3.560500in}}%
\pgfpathcurveto{\pgfqpoint{4.500905in}{3.554975in}}{\pgfqpoint{4.503101in}{3.549675in}}{\pgfqpoint{4.507007in}{3.545769in}}%
\pgfpathcurveto{\pgfqpoint{4.510914in}{3.541862in}}{\pgfqpoint{4.516214in}{3.539667in}}{\pgfqpoint{4.521739in}{3.539667in}}%
\pgfpathclose%
\pgfusepath{fill}%
\end{pgfscope}%
\begin{pgfscope}%
\pgfpathrectangle{\pgfqpoint{4.376725in}{3.178832in}}{\pgfqpoint{1.162500in}{0.755000in}}%
\pgfusepath{clip}%
\pgfsetbuttcap%
\pgfsetroundjoin%
\definecolor{currentfill}{rgb}{0.000000,0.000000,0.000000}%
\pgfsetfillcolor{currentfill}%
\pgfsetfillopacity{0.500000}%
\pgfsetlinewidth{0.000000pt}%
\definecolor{currentstroke}{rgb}{0.000000,0.000000,0.000000}%
\pgfsetstrokecolor{currentstroke}%
\pgfsetdash{}{0pt}%
\pgfpathmoveto{\pgfqpoint{4.448919in}{3.541019in}}%
\pgfpathcurveto{\pgfqpoint{4.454444in}{3.541019in}}{\pgfqpoint{4.459744in}{3.543214in}}{\pgfqpoint{4.463650in}{3.547121in}}%
\pgfpathcurveto{\pgfqpoint{4.467557in}{3.551028in}}{\pgfqpoint{4.469752in}{3.556327in}}{\pgfqpoint{4.469752in}{3.561852in}}%
\pgfpathcurveto{\pgfqpoint{4.469752in}{3.567377in}}{\pgfqpoint{4.467557in}{3.572677in}}{\pgfqpoint{4.463650in}{3.576584in}}%
\pgfpathcurveto{\pgfqpoint{4.459744in}{3.580490in}}{\pgfqpoint{4.454444in}{3.582685in}}{\pgfqpoint{4.448919in}{3.582685in}}%
\pgfpathcurveto{\pgfqpoint{4.443394in}{3.582685in}}{\pgfqpoint{4.438094in}{3.580490in}}{\pgfqpoint{4.434188in}{3.576584in}}%
\pgfpathcurveto{\pgfqpoint{4.430281in}{3.572677in}}{\pgfqpoint{4.428086in}{3.567377in}}{\pgfqpoint{4.428086in}{3.561852in}}%
\pgfpathcurveto{\pgfqpoint{4.428086in}{3.556327in}}{\pgfqpoint{4.430281in}{3.551028in}}{\pgfqpoint{4.434188in}{3.547121in}}%
\pgfpathcurveto{\pgfqpoint{4.438094in}{3.543214in}}{\pgfqpoint{4.443394in}{3.541019in}}{\pgfqpoint{4.448919in}{3.541019in}}%
\pgfpathclose%
\pgfusepath{fill}%
\end{pgfscope}%
\begin{pgfscope}%
\pgfpathrectangle{\pgfqpoint{4.376725in}{3.178832in}}{\pgfqpoint{1.162500in}{0.755000in}}%
\pgfusepath{clip}%
\pgfsetbuttcap%
\pgfsetroundjoin%
\definecolor{currentfill}{rgb}{0.000000,0.000000,0.000000}%
\pgfsetfillcolor{currentfill}%
\pgfsetfillopacity{0.500000}%
\pgfsetlinewidth{0.000000pt}%
\definecolor{currentstroke}{rgb}{0.000000,0.000000,0.000000}%
\pgfsetstrokecolor{currentstroke}%
\pgfsetdash{}{0pt}%
\pgfpathmoveto{\pgfqpoint{4.704251in}{3.308342in}}%
\pgfpathcurveto{\pgfqpoint{4.709776in}{3.308342in}}{\pgfqpoint{4.715075in}{3.310537in}}{\pgfqpoint{4.718982in}{3.314444in}}%
\pgfpathcurveto{\pgfqpoint{4.722889in}{3.318351in}}{\pgfqpoint{4.725084in}{3.323651in}}{\pgfqpoint{4.725084in}{3.329176in}}%
\pgfpathcurveto{\pgfqpoint{4.725084in}{3.334701in}}{\pgfqpoint{4.722889in}{3.340000in}}{\pgfqpoint{4.718982in}{3.343907in}}%
\pgfpathcurveto{\pgfqpoint{4.715075in}{3.347814in}}{\pgfqpoint{4.709776in}{3.350009in}}{\pgfqpoint{4.704251in}{3.350009in}}%
\pgfpathcurveto{\pgfqpoint{4.698726in}{3.350009in}}{\pgfqpoint{4.693426in}{3.347814in}}{\pgfqpoint{4.689519in}{3.343907in}}%
\pgfpathcurveto{\pgfqpoint{4.685612in}{3.340000in}}{\pgfqpoint{4.683417in}{3.334701in}}{\pgfqpoint{4.683417in}{3.329176in}}%
\pgfpathcurveto{\pgfqpoint{4.683417in}{3.323651in}}{\pgfqpoint{4.685612in}{3.318351in}}{\pgfqpoint{4.689519in}{3.314444in}}%
\pgfpathcurveto{\pgfqpoint{4.693426in}{3.310537in}}{\pgfqpoint{4.698726in}{3.308342in}}{\pgfqpoint{4.704251in}{3.308342in}}%
\pgfpathclose%
\pgfusepath{fill}%
\end{pgfscope}%
\begin{pgfscope}%
\pgfpathrectangle{\pgfqpoint{4.376725in}{3.178832in}}{\pgfqpoint{1.162500in}{0.755000in}}%
\pgfusepath{clip}%
\pgfsetbuttcap%
\pgfsetroundjoin%
\definecolor{currentfill}{rgb}{0.000000,0.000000,0.000000}%
\pgfsetfillcolor{currentfill}%
\pgfsetfillopacity{0.500000}%
\pgfsetlinewidth{0.000000pt}%
\definecolor{currentstroke}{rgb}{0.000000,0.000000,0.000000}%
\pgfsetstrokecolor{currentstroke}%
\pgfsetdash{}{0pt}%
\pgfpathmoveto{\pgfqpoint{4.404404in}{3.311937in}}%
\pgfpathcurveto{\pgfqpoint{4.409929in}{3.311937in}}{\pgfqpoint{4.415228in}{3.314132in}}{\pgfqpoint{4.419135in}{3.318039in}}%
\pgfpathcurveto{\pgfqpoint{4.423042in}{3.321945in}}{\pgfqpoint{4.425237in}{3.327245in}}{\pgfqpoint{4.425237in}{3.332770in}}%
\pgfpathcurveto{\pgfqpoint{4.425237in}{3.338295in}}{\pgfqpoint{4.423042in}{3.343595in}}{\pgfqpoint{4.419135in}{3.347501in}}%
\pgfpathcurveto{\pgfqpoint{4.415228in}{3.351408in}}{\pgfqpoint{4.409929in}{3.353603in}}{\pgfqpoint{4.404404in}{3.353603in}}%
\pgfpathcurveto{\pgfqpoint{4.398879in}{3.353603in}}{\pgfqpoint{4.393579in}{3.351408in}}{\pgfqpoint{4.389672in}{3.347501in}}%
\pgfpathcurveto{\pgfqpoint{4.385765in}{3.343595in}}{\pgfqpoint{4.383570in}{3.338295in}}{\pgfqpoint{4.383570in}{3.332770in}}%
\pgfpathcurveto{\pgfqpoint{4.383570in}{3.327245in}}{\pgfqpoint{4.385765in}{3.321945in}}{\pgfqpoint{4.389672in}{3.318039in}}%
\pgfpathcurveto{\pgfqpoint{4.393579in}{3.314132in}}{\pgfqpoint{4.398879in}{3.311937in}}{\pgfqpoint{4.404404in}{3.311937in}}%
\pgfpathclose%
\pgfusepath{fill}%
\end{pgfscope}%
\begin{pgfscope}%
\pgfpathrectangle{\pgfqpoint{4.376725in}{3.178832in}}{\pgfqpoint{1.162500in}{0.755000in}}%
\pgfusepath{clip}%
\pgfsetbuttcap%
\pgfsetroundjoin%
\definecolor{currentfill}{rgb}{0.000000,0.000000,0.000000}%
\pgfsetfillcolor{currentfill}%
\pgfsetfillopacity{0.500000}%
\pgfsetlinewidth{0.000000pt}%
\definecolor{currentstroke}{rgb}{0.000000,0.000000,0.000000}%
\pgfsetstrokecolor{currentstroke}%
\pgfsetdash{}{0pt}%
\pgfpathmoveto{\pgfqpoint{4.607580in}{3.215463in}}%
\pgfpathcurveto{\pgfqpoint{4.613105in}{3.215463in}}{\pgfqpoint{4.618405in}{3.217658in}}{\pgfqpoint{4.622312in}{3.221565in}}%
\pgfpathcurveto{\pgfqpoint{4.626218in}{3.225471in}}{\pgfqpoint{4.628414in}{3.230771in}}{\pgfqpoint{4.628414in}{3.236296in}}%
\pgfpathcurveto{\pgfqpoint{4.628414in}{3.241821in}}{\pgfqpoint{4.626218in}{3.247121in}}{\pgfqpoint{4.622312in}{3.251027in}}%
\pgfpathcurveto{\pgfqpoint{4.618405in}{3.254934in}}{\pgfqpoint{4.613105in}{3.257129in}}{\pgfqpoint{4.607580in}{3.257129in}}%
\pgfpathcurveto{\pgfqpoint{4.602055in}{3.257129in}}{\pgfqpoint{4.596756in}{3.254934in}}{\pgfqpoint{4.592849in}{3.251027in}}%
\pgfpathcurveto{\pgfqpoint{4.588942in}{3.247121in}}{\pgfqpoint{4.586747in}{3.241821in}}{\pgfqpoint{4.586747in}{3.236296in}}%
\pgfpathcurveto{\pgfqpoint{4.586747in}{3.230771in}}{\pgfqpoint{4.588942in}{3.225471in}}{\pgfqpoint{4.592849in}{3.221565in}}%
\pgfpathcurveto{\pgfqpoint{4.596756in}{3.217658in}}{\pgfqpoint{4.602055in}{3.215463in}}{\pgfqpoint{4.607580in}{3.215463in}}%
\pgfpathclose%
\pgfusepath{fill}%
\end{pgfscope}%
\begin{pgfscope}%
\pgfpathrectangle{\pgfqpoint{4.376725in}{3.178832in}}{\pgfqpoint{1.162500in}{0.755000in}}%
\pgfusepath{clip}%
\pgfsetbuttcap%
\pgfsetroundjoin%
\definecolor{currentfill}{rgb}{0.000000,0.000000,0.000000}%
\pgfsetfillcolor{currentfill}%
\pgfsetfillopacity{0.500000}%
\pgfsetlinewidth{0.000000pt}%
\definecolor{currentstroke}{rgb}{0.000000,0.000000,0.000000}%
\pgfsetstrokecolor{currentstroke}%
\pgfsetdash{}{0pt}%
\pgfpathmoveto{\pgfqpoint{4.596304in}{3.175975in}}%
\pgfpathcurveto{\pgfqpoint{4.601829in}{3.175975in}}{\pgfqpoint{4.607129in}{3.178170in}}{\pgfqpoint{4.611036in}{3.182077in}}%
\pgfpathcurveto{\pgfqpoint{4.614942in}{3.185984in}}{\pgfqpoint{4.617137in}{3.191283in}}{\pgfqpoint{4.617137in}{3.196809in}}%
\pgfpathcurveto{\pgfqpoint{4.617137in}{3.202334in}}{\pgfqpoint{4.614942in}{3.207633in}}{\pgfqpoint{4.611036in}{3.211540in}}%
\pgfpathcurveto{\pgfqpoint{4.607129in}{3.215447in}}{\pgfqpoint{4.601829in}{3.217642in}}{\pgfqpoint{4.596304in}{3.217642in}}%
\pgfpathcurveto{\pgfqpoint{4.590779in}{3.217642in}}{\pgfqpoint{4.585480in}{3.215447in}}{\pgfqpoint{4.581573in}{3.211540in}}%
\pgfpathcurveto{\pgfqpoint{4.577666in}{3.207633in}}{\pgfqpoint{4.575471in}{3.202334in}}{\pgfqpoint{4.575471in}{3.196809in}}%
\pgfpathcurveto{\pgfqpoint{4.575471in}{3.191283in}}{\pgfqpoint{4.577666in}{3.185984in}}{\pgfqpoint{4.581573in}{3.182077in}}%
\pgfpathcurveto{\pgfqpoint{4.585480in}{3.178170in}}{\pgfqpoint{4.590779in}{3.175975in}}{\pgfqpoint{4.596304in}{3.175975in}}%
\pgfpathclose%
\pgfusepath{fill}%
\end{pgfscope}%
\begin{pgfscope}%
\pgfpathrectangle{\pgfqpoint{4.376725in}{3.178832in}}{\pgfqpoint{1.162500in}{0.755000in}}%
\pgfusepath{clip}%
\pgfsetbuttcap%
\pgfsetroundjoin%
\definecolor{currentfill}{rgb}{0.000000,0.000000,0.000000}%
\pgfsetfillcolor{currentfill}%
\pgfsetfillopacity{0.500000}%
\pgfsetlinewidth{0.000000pt}%
\definecolor{currentstroke}{rgb}{0.000000,0.000000,0.000000}%
\pgfsetstrokecolor{currentstroke}%
\pgfsetdash{}{0pt}%
\pgfpathmoveto{\pgfqpoint{5.423123in}{3.678982in}}%
\pgfpathcurveto{\pgfqpoint{5.428649in}{3.678982in}}{\pgfqpoint{5.433948in}{3.681177in}}{\pgfqpoint{5.437855in}{3.685084in}}%
\pgfpathcurveto{\pgfqpoint{5.441762in}{3.688990in}}{\pgfqpoint{5.443957in}{3.694290in}}{\pgfqpoint{5.443957in}{3.699815in}}%
\pgfpathcurveto{\pgfqpoint{5.443957in}{3.705340in}}{\pgfqpoint{5.441762in}{3.710640in}}{\pgfqpoint{5.437855in}{3.714546in}}%
\pgfpathcurveto{\pgfqpoint{5.433948in}{3.718453in}}{\pgfqpoint{5.428649in}{3.720648in}}{\pgfqpoint{5.423123in}{3.720648in}}%
\pgfpathcurveto{\pgfqpoint{5.417598in}{3.720648in}}{\pgfqpoint{5.412299in}{3.718453in}}{\pgfqpoint{5.408392in}{3.714546in}}%
\pgfpathcurveto{\pgfqpoint{5.404485in}{3.710640in}}{\pgfqpoint{5.402290in}{3.705340in}}{\pgfqpoint{5.402290in}{3.699815in}}%
\pgfpathcurveto{\pgfqpoint{5.402290in}{3.694290in}}{\pgfqpoint{5.404485in}{3.688990in}}{\pgfqpoint{5.408392in}{3.685084in}}%
\pgfpathcurveto{\pgfqpoint{5.412299in}{3.681177in}}{\pgfqpoint{5.417598in}{3.678982in}}{\pgfqpoint{5.423123in}{3.678982in}}%
\pgfpathclose%
\pgfusepath{fill}%
\end{pgfscope}%
\begin{pgfscope}%
\pgfpathrectangle{\pgfqpoint{4.376725in}{3.178832in}}{\pgfqpoint{1.162500in}{0.755000in}}%
\pgfusepath{clip}%
\pgfsetbuttcap%
\pgfsetroundjoin%
\definecolor{currentfill}{rgb}{0.000000,0.000000,0.000000}%
\pgfsetfillcolor{currentfill}%
\pgfsetfillopacity{0.500000}%
\pgfsetlinewidth{0.000000pt}%
\definecolor{currentstroke}{rgb}{0.000000,0.000000,0.000000}%
\pgfsetstrokecolor{currentstroke}%
\pgfsetdash{}{0pt}%
\pgfpathmoveto{\pgfqpoint{5.511546in}{3.528997in}}%
\pgfpathcurveto{\pgfqpoint{5.517071in}{3.528997in}}{\pgfqpoint{5.522371in}{3.531192in}}{\pgfqpoint{5.526278in}{3.535099in}}%
\pgfpathcurveto{\pgfqpoint{5.530185in}{3.539005in}}{\pgfqpoint{5.532380in}{3.544305in}}{\pgfqpoint{5.532380in}{3.549830in}}%
\pgfpathcurveto{\pgfqpoint{5.532380in}{3.555355in}}{\pgfqpoint{5.530185in}{3.560655in}}{\pgfqpoint{5.526278in}{3.564561in}}%
\pgfpathcurveto{\pgfqpoint{5.522371in}{3.568468in}}{\pgfqpoint{5.517071in}{3.570663in}}{\pgfqpoint{5.511546in}{3.570663in}}%
\pgfpathcurveto{\pgfqpoint{5.506021in}{3.570663in}}{\pgfqpoint{5.500722in}{3.568468in}}{\pgfqpoint{5.496815in}{3.564561in}}%
\pgfpathcurveto{\pgfqpoint{5.492908in}{3.560655in}}{\pgfqpoint{5.490713in}{3.555355in}}{\pgfqpoint{5.490713in}{3.549830in}}%
\pgfpathcurveto{\pgfqpoint{5.490713in}{3.544305in}}{\pgfqpoint{5.492908in}{3.539005in}}{\pgfqpoint{5.496815in}{3.535099in}}%
\pgfpathcurveto{\pgfqpoint{5.500722in}{3.531192in}}{\pgfqpoint{5.506021in}{3.528997in}}{\pgfqpoint{5.511546in}{3.528997in}}%
\pgfpathclose%
\pgfusepath{fill}%
\end{pgfscope}%
\begin{pgfscope}%
\pgfpathrectangle{\pgfqpoint{4.376725in}{3.178832in}}{\pgfqpoint{1.162500in}{0.755000in}}%
\pgfusepath{clip}%
\pgfsetbuttcap%
\pgfsetroundjoin%
\definecolor{currentfill}{rgb}{0.000000,0.000000,0.000000}%
\pgfsetfillcolor{currentfill}%
\pgfsetfillopacity{0.500000}%
\pgfsetlinewidth{0.000000pt}%
\definecolor{currentstroke}{rgb}{0.000000,0.000000,0.000000}%
\pgfsetstrokecolor{currentstroke}%
\pgfsetdash{}{0pt}%
\pgfpathmoveto{\pgfqpoint{5.286995in}{3.426855in}}%
\pgfpathcurveto{\pgfqpoint{5.292520in}{3.426855in}}{\pgfqpoint{5.297820in}{3.429051in}}{\pgfqpoint{5.301726in}{3.432957in}}%
\pgfpathcurveto{\pgfqpoint{5.305633in}{3.436864in}}{\pgfqpoint{5.307828in}{3.442164in}}{\pgfqpoint{5.307828in}{3.447689in}}%
\pgfpathcurveto{\pgfqpoint{5.307828in}{3.453214in}}{\pgfqpoint{5.305633in}{3.458513in}}{\pgfqpoint{5.301726in}{3.462420in}}%
\pgfpathcurveto{\pgfqpoint{5.297820in}{3.466327in}}{\pgfqpoint{5.292520in}{3.468522in}}{\pgfqpoint{5.286995in}{3.468522in}}%
\pgfpathcurveto{\pgfqpoint{5.281470in}{3.468522in}}{\pgfqpoint{5.276170in}{3.466327in}}{\pgfqpoint{5.272264in}{3.462420in}}%
\pgfpathcurveto{\pgfqpoint{5.268357in}{3.458513in}}{\pgfqpoint{5.266162in}{3.453214in}}{\pgfqpoint{5.266162in}{3.447689in}}%
\pgfpathcurveto{\pgfqpoint{5.266162in}{3.442164in}}{\pgfqpoint{5.268357in}{3.436864in}}{\pgfqpoint{5.272264in}{3.432957in}}%
\pgfpathcurveto{\pgfqpoint{5.276170in}{3.429051in}}{\pgfqpoint{5.281470in}{3.426855in}}{\pgfqpoint{5.286995in}{3.426855in}}%
\pgfpathclose%
\pgfusepath{fill}%
\end{pgfscope}%
\begin{pgfscope}%
\pgfpathrectangle{\pgfqpoint{4.376725in}{3.178832in}}{\pgfqpoint{1.162500in}{0.755000in}}%
\pgfusepath{clip}%
\pgfsetbuttcap%
\pgfsetroundjoin%
\definecolor{currentfill}{rgb}{0.000000,0.000000,0.000000}%
\pgfsetfillcolor{currentfill}%
\pgfsetfillopacity{0.500000}%
\pgfsetlinewidth{0.000000pt}%
\definecolor{currentstroke}{rgb}{0.000000,0.000000,0.000000}%
\pgfsetstrokecolor{currentstroke}%
\pgfsetdash{}{0pt}%
\pgfpathmoveto{\pgfqpoint{4.645822in}{3.677650in}}%
\pgfpathcurveto{\pgfqpoint{4.651348in}{3.677650in}}{\pgfqpoint{4.656647in}{3.679845in}}{\pgfqpoint{4.660554in}{3.683752in}}%
\pgfpathcurveto{\pgfqpoint{4.664461in}{3.687659in}}{\pgfqpoint{4.666656in}{3.692959in}}{\pgfqpoint{4.666656in}{3.698484in}}%
\pgfpathcurveto{\pgfqpoint{4.666656in}{3.704009in}}{\pgfqpoint{4.664461in}{3.709308in}}{\pgfqpoint{4.660554in}{3.713215in}}%
\pgfpathcurveto{\pgfqpoint{4.656647in}{3.717122in}}{\pgfqpoint{4.651348in}{3.719317in}}{\pgfqpoint{4.645822in}{3.719317in}}%
\pgfpathcurveto{\pgfqpoint{4.640297in}{3.719317in}}{\pgfqpoint{4.634998in}{3.717122in}}{\pgfqpoint{4.631091in}{3.713215in}}%
\pgfpathcurveto{\pgfqpoint{4.627184in}{3.709308in}}{\pgfqpoint{4.624989in}{3.704009in}}{\pgfqpoint{4.624989in}{3.698484in}}%
\pgfpathcurveto{\pgfqpoint{4.624989in}{3.692959in}}{\pgfqpoint{4.627184in}{3.687659in}}{\pgfqpoint{4.631091in}{3.683752in}}%
\pgfpathcurveto{\pgfqpoint{4.634998in}{3.679845in}}{\pgfqpoint{4.640297in}{3.677650in}}{\pgfqpoint{4.645822in}{3.677650in}}%
\pgfpathclose%
\pgfusepath{fill}%
\end{pgfscope}%
\begin{pgfscope}%
\pgfpathrectangle{\pgfqpoint{4.376725in}{3.178832in}}{\pgfqpoint{1.162500in}{0.755000in}}%
\pgfusepath{clip}%
\pgfsetbuttcap%
\pgfsetroundjoin%
\definecolor{currentfill}{rgb}{0.000000,0.000000,0.000000}%
\pgfsetfillcolor{currentfill}%
\pgfsetfillopacity{0.500000}%
\pgfsetlinewidth{0.000000pt}%
\definecolor{currentstroke}{rgb}{0.000000,0.000000,0.000000}%
\pgfsetstrokecolor{currentstroke}%
\pgfsetdash{}{0pt}%
\pgfpathmoveto{\pgfqpoint{5.069839in}{3.296446in}}%
\pgfpathcurveto{\pgfqpoint{5.075364in}{3.296446in}}{\pgfqpoint{5.080663in}{3.298641in}}{\pgfqpoint{5.084570in}{3.302548in}}%
\pgfpathcurveto{\pgfqpoint{5.088477in}{3.306455in}}{\pgfqpoint{5.090672in}{3.311754in}}{\pgfqpoint{5.090672in}{3.317279in}}%
\pgfpathcurveto{\pgfqpoint{5.090672in}{3.322804in}}{\pgfqpoint{5.088477in}{3.328104in}}{\pgfqpoint{5.084570in}{3.332011in}}%
\pgfpathcurveto{\pgfqpoint{5.080663in}{3.335917in}}{\pgfqpoint{5.075364in}{3.338112in}}{\pgfqpoint{5.069839in}{3.338112in}}%
\pgfpathcurveto{\pgfqpoint{5.064313in}{3.338112in}}{\pgfqpoint{5.059014in}{3.335917in}}{\pgfqpoint{5.055107in}{3.332011in}}%
\pgfpathcurveto{\pgfqpoint{5.051200in}{3.328104in}}{\pgfqpoint{5.049005in}{3.322804in}}{\pgfqpoint{5.049005in}{3.317279in}}%
\pgfpathcurveto{\pgfqpoint{5.049005in}{3.311754in}}{\pgfqpoint{5.051200in}{3.306455in}}{\pgfqpoint{5.055107in}{3.302548in}}%
\pgfpathcurveto{\pgfqpoint{5.059014in}{3.298641in}}{\pgfqpoint{5.064313in}{3.296446in}}{\pgfqpoint{5.069839in}{3.296446in}}%
\pgfpathclose%
\pgfusepath{fill}%
\end{pgfscope}%
\begin{pgfscope}%
\pgfpathrectangle{\pgfqpoint{4.376725in}{3.178832in}}{\pgfqpoint{1.162500in}{0.755000in}}%
\pgfusepath{clip}%
\pgfsetbuttcap%
\pgfsetroundjoin%
\definecolor{currentfill}{rgb}{0.000000,0.000000,0.000000}%
\pgfsetfillcolor{currentfill}%
\pgfsetfillopacity{0.500000}%
\pgfsetlinewidth{0.000000pt}%
\definecolor{currentstroke}{rgb}{0.000000,0.000000,0.000000}%
\pgfsetstrokecolor{currentstroke}%
\pgfsetdash{}{0pt}%
\pgfpathmoveto{\pgfqpoint{4.941375in}{3.322937in}}%
\pgfpathcurveto{\pgfqpoint{4.946900in}{3.322937in}}{\pgfqpoint{4.952199in}{3.325133in}}{\pgfqpoint{4.956106in}{3.329039in}}%
\pgfpathcurveto{\pgfqpoint{4.960013in}{3.332946in}}{\pgfqpoint{4.962208in}{3.338246in}}{\pgfqpoint{4.962208in}{3.343771in}}%
\pgfpathcurveto{\pgfqpoint{4.962208in}{3.349296in}}{\pgfqpoint{4.960013in}{3.354595in}}{\pgfqpoint{4.956106in}{3.358502in}}%
\pgfpathcurveto{\pgfqpoint{4.952199in}{3.362409in}}{\pgfqpoint{4.946900in}{3.364604in}}{\pgfqpoint{4.941375in}{3.364604in}}%
\pgfpathcurveto{\pgfqpoint{4.935850in}{3.364604in}}{\pgfqpoint{4.930550in}{3.362409in}}{\pgfqpoint{4.926643in}{3.358502in}}%
\pgfpathcurveto{\pgfqpoint{4.922736in}{3.354595in}}{\pgfqpoint{4.920541in}{3.349296in}}{\pgfqpoint{4.920541in}{3.343771in}}%
\pgfpathcurveto{\pgfqpoint{4.920541in}{3.338246in}}{\pgfqpoint{4.922736in}{3.332946in}}{\pgfqpoint{4.926643in}{3.329039in}}%
\pgfpathcurveto{\pgfqpoint{4.930550in}{3.325133in}}{\pgfqpoint{4.935850in}{3.322937in}}{\pgfqpoint{4.941375in}{3.322937in}}%
\pgfpathclose%
\pgfusepath{fill}%
\end{pgfscope}%
\begin{pgfscope}%
\pgfpathrectangle{\pgfqpoint{4.376725in}{3.178832in}}{\pgfqpoint{1.162500in}{0.755000in}}%
\pgfusepath{clip}%
\pgfsetbuttcap%
\pgfsetroundjoin%
\definecolor{currentfill}{rgb}{0.000000,0.000000,0.000000}%
\pgfsetfillcolor{currentfill}%
\pgfsetfillopacity{0.500000}%
\pgfsetlinewidth{0.000000pt}%
\definecolor{currentstroke}{rgb}{0.000000,0.000000,0.000000}%
\pgfsetstrokecolor{currentstroke}%
\pgfsetdash{}{0pt}%
\pgfpathmoveto{\pgfqpoint{4.799153in}{3.895023in}}%
\pgfpathcurveto{\pgfqpoint{4.804678in}{3.895023in}}{\pgfqpoint{4.809977in}{3.897218in}}{\pgfqpoint{4.813884in}{3.901125in}}%
\pgfpathcurveto{\pgfqpoint{4.817791in}{3.905032in}}{\pgfqpoint{4.819986in}{3.910331in}}{\pgfqpoint{4.819986in}{3.915856in}}%
\pgfpathcurveto{\pgfqpoint{4.819986in}{3.921381in}}{\pgfqpoint{4.817791in}{3.926681in}}{\pgfqpoint{4.813884in}{3.930588in}}%
\pgfpathcurveto{\pgfqpoint{4.809977in}{3.934494in}}{\pgfqpoint{4.804678in}{3.936690in}}{\pgfqpoint{4.799153in}{3.936690in}}%
\pgfpathcurveto{\pgfqpoint{4.793628in}{3.936690in}}{\pgfqpoint{4.788328in}{3.934494in}}{\pgfqpoint{4.784421in}{3.930588in}}%
\pgfpathcurveto{\pgfqpoint{4.780515in}{3.926681in}}{\pgfqpoint{4.778320in}{3.921381in}}{\pgfqpoint{4.778320in}{3.915856in}}%
\pgfpathcurveto{\pgfqpoint{4.778320in}{3.910331in}}{\pgfqpoint{4.780515in}{3.905032in}}{\pgfqpoint{4.784421in}{3.901125in}}%
\pgfpathcurveto{\pgfqpoint{4.788328in}{3.897218in}}{\pgfqpoint{4.793628in}{3.895023in}}{\pgfqpoint{4.799153in}{3.895023in}}%
\pgfpathclose%
\pgfusepath{fill}%
\end{pgfscope}%
\begin{pgfscope}%
\pgfpathrectangle{\pgfqpoint{4.376725in}{3.178832in}}{\pgfqpoint{1.162500in}{0.755000in}}%
\pgfusepath{clip}%
\pgfsetbuttcap%
\pgfsetroundjoin%
\definecolor{currentfill}{rgb}{0.000000,0.000000,0.000000}%
\pgfsetfillcolor{currentfill}%
\pgfsetfillopacity{0.500000}%
\pgfsetlinewidth{0.000000pt}%
\definecolor{currentstroke}{rgb}{0.000000,0.000000,0.000000}%
\pgfsetstrokecolor{currentstroke}%
\pgfsetdash{}{0pt}%
\pgfpathmoveto{\pgfqpoint{4.595218in}{3.491212in}}%
\pgfpathcurveto{\pgfqpoint{4.600743in}{3.491212in}}{\pgfqpoint{4.606043in}{3.493407in}}{\pgfqpoint{4.609950in}{3.497314in}}%
\pgfpathcurveto{\pgfqpoint{4.613857in}{3.501221in}}{\pgfqpoint{4.616052in}{3.506520in}}{\pgfqpoint{4.616052in}{3.512045in}}%
\pgfpathcurveto{\pgfqpoint{4.616052in}{3.517571in}}{\pgfqpoint{4.613857in}{3.522870in}}{\pgfqpoint{4.609950in}{3.526777in}}%
\pgfpathcurveto{\pgfqpoint{4.606043in}{3.530684in}}{\pgfqpoint{4.600743in}{3.532879in}}{\pgfqpoint{4.595218in}{3.532879in}}%
\pgfpathcurveto{\pgfqpoint{4.589693in}{3.532879in}}{\pgfqpoint{4.584394in}{3.530684in}}{\pgfqpoint{4.580487in}{3.526777in}}%
\pgfpathcurveto{\pgfqpoint{4.576580in}{3.522870in}}{\pgfqpoint{4.574385in}{3.517571in}}{\pgfqpoint{4.574385in}{3.512045in}}%
\pgfpathcurveto{\pgfqpoint{4.574385in}{3.506520in}}{\pgfqpoint{4.576580in}{3.501221in}}{\pgfqpoint{4.580487in}{3.497314in}}%
\pgfpathcurveto{\pgfqpoint{4.584394in}{3.493407in}}{\pgfqpoint{4.589693in}{3.491212in}}{\pgfqpoint{4.595218in}{3.491212in}}%
\pgfpathclose%
\pgfusepath{fill}%
\end{pgfscope}%
\begin{pgfscope}%
\pgfpathrectangle{\pgfqpoint{4.376725in}{3.178832in}}{\pgfqpoint{1.162500in}{0.755000in}}%
\pgfusepath{clip}%
\pgfsetbuttcap%
\pgfsetroundjoin%
\definecolor{currentfill}{rgb}{0.000000,0.000000,0.000000}%
\pgfsetfillcolor{currentfill}%
\pgfsetfillopacity{0.500000}%
\pgfsetlinewidth{0.000000pt}%
\definecolor{currentstroke}{rgb}{0.000000,0.000000,0.000000}%
\pgfsetstrokecolor{currentstroke}%
\pgfsetdash{}{0pt}%
\pgfpathmoveto{\pgfqpoint{4.834270in}{3.212721in}}%
\pgfpathcurveto{\pgfqpoint{4.839795in}{3.212721in}}{\pgfqpoint{4.845094in}{3.214917in}}{\pgfqpoint{4.849001in}{3.218823in}}%
\pgfpathcurveto{\pgfqpoint{4.852908in}{3.222730in}}{\pgfqpoint{4.855103in}{3.228030in}}{\pgfqpoint{4.855103in}{3.233555in}}%
\pgfpathcurveto{\pgfqpoint{4.855103in}{3.239080in}}{\pgfqpoint{4.852908in}{3.244379in}}{\pgfqpoint{4.849001in}{3.248286in}}%
\pgfpathcurveto{\pgfqpoint{4.845094in}{3.252193in}}{\pgfqpoint{4.839795in}{3.254388in}}{\pgfqpoint{4.834270in}{3.254388in}}%
\pgfpathcurveto{\pgfqpoint{4.828745in}{3.254388in}}{\pgfqpoint{4.823445in}{3.252193in}}{\pgfqpoint{4.819538in}{3.248286in}}%
\pgfpathcurveto{\pgfqpoint{4.815632in}{3.244379in}}{\pgfqpoint{4.813437in}{3.239080in}}{\pgfqpoint{4.813437in}{3.233555in}}%
\pgfpathcurveto{\pgfqpoint{4.813437in}{3.228030in}}{\pgfqpoint{4.815632in}{3.222730in}}{\pgfqpoint{4.819538in}{3.218823in}}%
\pgfpathcurveto{\pgfqpoint{4.823445in}{3.214917in}}{\pgfqpoint{4.828745in}{3.212721in}}{\pgfqpoint{4.834270in}{3.212721in}}%
\pgfpathclose%
\pgfusepath{fill}%
\end{pgfscope}%
\begin{pgfscope}%
\pgfsetrectcap%
\pgfsetmiterjoin%
\pgfsetlinewidth{0.803000pt}%
\definecolor{currentstroke}{rgb}{0.501961,0.501961,0.501961}%
\pgfsetstrokecolor{currentstroke}%
\pgfsetdash{}{0pt}%
\pgfpathmoveto{\pgfqpoint{4.376725in}{3.178832in}}%
\pgfpathlineto{\pgfqpoint{4.376725in}{3.933832in}}%
\pgfusepath{stroke}%
\end{pgfscope}%
\begin{pgfscope}%
\pgfsetrectcap%
\pgfsetmiterjoin%
\pgfsetlinewidth{0.803000pt}%
\definecolor{currentstroke}{rgb}{0.501961,0.501961,0.501961}%
\pgfsetstrokecolor{currentstroke}%
\pgfsetdash{}{0pt}%
\pgfpathmoveto{\pgfqpoint{5.539225in}{3.178832in}}%
\pgfpathlineto{\pgfqpoint{5.539225in}{3.933832in}}%
\pgfusepath{stroke}%
\end{pgfscope}%
\begin{pgfscope}%
\pgfsetrectcap%
\pgfsetmiterjoin%
\pgfsetlinewidth{0.803000pt}%
\definecolor{currentstroke}{rgb}{0.501961,0.501961,0.501961}%
\pgfsetstrokecolor{currentstroke}%
\pgfsetdash{}{0pt}%
\pgfpathmoveto{\pgfqpoint{4.376725in}{3.178832in}}%
\pgfpathlineto{\pgfqpoint{5.539225in}{3.178832in}}%
\pgfusepath{stroke}%
\end{pgfscope}%
\begin{pgfscope}%
\pgfsetrectcap%
\pgfsetmiterjoin%
\pgfsetlinewidth{0.803000pt}%
\definecolor{currentstroke}{rgb}{0.501961,0.501961,0.501961}%
\pgfsetstrokecolor{currentstroke}%
\pgfsetdash{}{0pt}%
\pgfpathmoveto{\pgfqpoint{4.376725in}{3.933832in}}%
\pgfpathlineto{\pgfqpoint{5.539225in}{3.933832in}}%
\pgfusepath{stroke}%
\end{pgfscope}%
\begin{pgfscope}%
\pgfsetbuttcap%
\pgfsetmiterjoin%
\definecolor{currentfill}{rgb}{1.000000,1.000000,1.000000}%
\pgfsetfillcolor{currentfill}%
\pgfsetlinewidth{0.000000pt}%
\definecolor{currentstroke}{rgb}{0.000000,0.000000,0.000000}%
\pgfsetstrokecolor{currentstroke}%
\pgfsetstrokeopacity{0.000000}%
\pgfsetdash{}{0pt}%
\pgfpathmoveto{\pgfqpoint{0.889225in}{2.423832in}}%
\pgfpathlineto{\pgfqpoint{2.051725in}{2.423832in}}%
\pgfpathlineto{\pgfqpoint{2.051725in}{3.178832in}}%
\pgfpathlineto{\pgfqpoint{0.889225in}{3.178832in}}%
\pgfpathclose%
\pgfusepath{fill}%
\end{pgfscope}%
\begin{pgfscope}%
\pgfpathrectangle{\pgfqpoint{0.889225in}{2.423832in}}{\pgfqpoint{1.162500in}{0.755000in}}%
\pgfusepath{clip}%
\pgfsetbuttcap%
\pgfsetroundjoin%
\definecolor{currentfill}{rgb}{0.000000,0.000000,0.000000}%
\pgfsetfillcolor{currentfill}%
\pgfsetfillopacity{0.500000}%
\pgfsetlinewidth{0.000000pt}%
\definecolor{currentstroke}{rgb}{0.000000,0.000000,0.000000}%
\pgfsetstrokecolor{currentstroke}%
\pgfsetdash{}{0pt}%
\pgfpathmoveto{\pgfqpoint{1.893746in}{3.088730in}}%
\pgfpathcurveto{\pgfqpoint{1.899271in}{3.088730in}}{\pgfqpoint{1.904570in}{3.090925in}}{\pgfqpoint{1.908477in}{3.094832in}}%
\pgfpathcurveto{\pgfqpoint{1.912384in}{3.098739in}}{\pgfqpoint{1.914579in}{3.104038in}}{\pgfqpoint{1.914579in}{3.109563in}}%
\pgfpathcurveto{\pgfqpoint{1.914579in}{3.115089in}}{\pgfqpoint{1.912384in}{3.120388in}}{\pgfqpoint{1.908477in}{3.124295in}}%
\pgfpathcurveto{\pgfqpoint{1.904570in}{3.128202in}}{\pgfqpoint{1.899271in}{3.130397in}}{\pgfqpoint{1.893746in}{3.130397in}}%
\pgfpathcurveto{\pgfqpoint{1.888221in}{3.130397in}}{\pgfqpoint{1.882921in}{3.128202in}}{\pgfqpoint{1.879015in}{3.124295in}}%
\pgfpathcurveto{\pgfqpoint{1.875108in}{3.120388in}}{\pgfqpoint{1.872913in}{3.115089in}}{\pgfqpoint{1.872913in}{3.109563in}}%
\pgfpathcurveto{\pgfqpoint{1.872913in}{3.104038in}}{\pgfqpoint{1.875108in}{3.098739in}}{\pgfqpoint{1.879015in}{3.094832in}}%
\pgfpathcurveto{\pgfqpoint{1.882921in}{3.090925in}}{\pgfqpoint{1.888221in}{3.088730in}}{\pgfqpoint{1.893746in}{3.088730in}}%
\pgfpathclose%
\pgfusepath{fill}%
\end{pgfscope}%
\begin{pgfscope}%
\pgfpathrectangle{\pgfqpoint{0.889225in}{2.423832in}}{\pgfqpoint{1.162500in}{0.755000in}}%
\pgfusepath{clip}%
\pgfsetbuttcap%
\pgfsetroundjoin%
\definecolor{currentfill}{rgb}{0.000000,0.000000,0.000000}%
\pgfsetfillcolor{currentfill}%
\pgfsetfillopacity{0.500000}%
\pgfsetlinewidth{0.000000pt}%
\definecolor{currentstroke}{rgb}{0.000000,0.000000,0.000000}%
\pgfsetstrokecolor{currentstroke}%
\pgfsetdash{}{0pt}%
\pgfpathmoveto{\pgfqpoint{1.476892in}{2.544873in}}%
\pgfpathcurveto{\pgfqpoint{1.482417in}{2.544873in}}{\pgfqpoint{1.487717in}{2.547068in}}{\pgfqpoint{1.491623in}{2.550975in}}%
\pgfpathcurveto{\pgfqpoint{1.495530in}{2.554881in}}{\pgfqpoint{1.497725in}{2.560181in}}{\pgfqpoint{1.497725in}{2.565706in}}%
\pgfpathcurveto{\pgfqpoint{1.497725in}{2.571231in}}{\pgfqpoint{1.495530in}{2.576531in}}{\pgfqpoint{1.491623in}{2.580437in}}%
\pgfpathcurveto{\pgfqpoint{1.487717in}{2.584344in}}{\pgfqpoint{1.482417in}{2.586539in}}{\pgfqpoint{1.476892in}{2.586539in}}%
\pgfpathcurveto{\pgfqpoint{1.471367in}{2.586539in}}{\pgfqpoint{1.466067in}{2.584344in}}{\pgfqpoint{1.462161in}{2.580437in}}%
\pgfpathcurveto{\pgfqpoint{1.458254in}{2.576531in}}{\pgfqpoint{1.456059in}{2.571231in}}{\pgfqpoint{1.456059in}{2.565706in}}%
\pgfpathcurveto{\pgfqpoint{1.456059in}{2.560181in}}{\pgfqpoint{1.458254in}{2.554881in}}{\pgfqpoint{1.462161in}{2.550975in}}%
\pgfpathcurveto{\pgfqpoint{1.466067in}{2.547068in}}{\pgfqpoint{1.471367in}{2.544873in}}{\pgfqpoint{1.476892in}{2.544873in}}%
\pgfpathclose%
\pgfusepath{fill}%
\end{pgfscope}%
\begin{pgfscope}%
\pgfpathrectangle{\pgfqpoint{0.889225in}{2.423832in}}{\pgfqpoint{1.162500in}{0.755000in}}%
\pgfusepath{clip}%
\pgfsetbuttcap%
\pgfsetroundjoin%
\definecolor{currentfill}{rgb}{0.000000,0.000000,0.000000}%
\pgfsetfillcolor{currentfill}%
\pgfsetfillopacity{0.500000}%
\pgfsetlinewidth{0.000000pt}%
\definecolor{currentstroke}{rgb}{0.000000,0.000000,0.000000}%
\pgfsetstrokecolor{currentstroke}%
\pgfsetdash{}{0pt}%
\pgfpathmoveto{\pgfqpoint{1.478974in}{2.541427in}}%
\pgfpathcurveto{\pgfqpoint{1.484499in}{2.541427in}}{\pgfqpoint{1.489799in}{2.543622in}}{\pgfqpoint{1.493705in}{2.547529in}}%
\pgfpathcurveto{\pgfqpoint{1.497612in}{2.551436in}}{\pgfqpoint{1.499807in}{2.556735in}}{\pgfqpoint{1.499807in}{2.562260in}}%
\pgfpathcurveto{\pgfqpoint{1.499807in}{2.567785in}}{\pgfqpoint{1.497612in}{2.573085in}}{\pgfqpoint{1.493705in}{2.576992in}}%
\pgfpathcurveto{\pgfqpoint{1.489799in}{2.580899in}}{\pgfqpoint{1.484499in}{2.583094in}}{\pgfqpoint{1.478974in}{2.583094in}}%
\pgfpathcurveto{\pgfqpoint{1.473449in}{2.583094in}}{\pgfqpoint{1.468149in}{2.580899in}}{\pgfqpoint{1.464243in}{2.576992in}}%
\pgfpathcurveto{\pgfqpoint{1.460336in}{2.573085in}}{\pgfqpoint{1.458141in}{2.567785in}}{\pgfqpoint{1.458141in}{2.562260in}}%
\pgfpathcurveto{\pgfqpoint{1.458141in}{2.556735in}}{\pgfqpoint{1.460336in}{2.551436in}}{\pgfqpoint{1.464243in}{2.547529in}}%
\pgfpathcurveto{\pgfqpoint{1.468149in}{2.543622in}}{\pgfqpoint{1.473449in}{2.541427in}}{\pgfqpoint{1.478974in}{2.541427in}}%
\pgfpathclose%
\pgfusepath{fill}%
\end{pgfscope}%
\begin{pgfscope}%
\pgfpathrectangle{\pgfqpoint{0.889225in}{2.423832in}}{\pgfqpoint{1.162500in}{0.755000in}}%
\pgfusepath{clip}%
\pgfsetbuttcap%
\pgfsetroundjoin%
\definecolor{currentfill}{rgb}{0.000000,0.000000,0.000000}%
\pgfsetfillcolor{currentfill}%
\pgfsetfillopacity{0.500000}%
\pgfsetlinewidth{0.000000pt}%
\definecolor{currentstroke}{rgb}{0.000000,0.000000,0.000000}%
\pgfsetstrokecolor{currentstroke}%
\pgfsetdash{}{0pt}%
\pgfpathmoveto{\pgfqpoint{1.120714in}{2.469548in}}%
\pgfpathcurveto{\pgfqpoint{1.126239in}{2.469548in}}{\pgfqpoint{1.131538in}{2.471743in}}{\pgfqpoint{1.135445in}{2.475650in}}%
\pgfpathcurveto{\pgfqpoint{1.139352in}{2.479557in}}{\pgfqpoint{1.141547in}{2.484856in}}{\pgfqpoint{1.141547in}{2.490381in}}%
\pgfpathcurveto{\pgfqpoint{1.141547in}{2.495907in}}{\pgfqpoint{1.139352in}{2.501206in}}{\pgfqpoint{1.135445in}{2.505113in}}%
\pgfpathcurveto{\pgfqpoint{1.131538in}{2.509020in}}{\pgfqpoint{1.126239in}{2.511215in}}{\pgfqpoint{1.120714in}{2.511215in}}%
\pgfpathcurveto{\pgfqpoint{1.115189in}{2.511215in}}{\pgfqpoint{1.109889in}{2.509020in}}{\pgfqpoint{1.105982in}{2.505113in}}%
\pgfpathcurveto{\pgfqpoint{1.102076in}{2.501206in}}{\pgfqpoint{1.099881in}{2.495907in}}{\pgfqpoint{1.099881in}{2.490381in}}%
\pgfpathcurveto{\pgfqpoint{1.099881in}{2.484856in}}{\pgfqpoint{1.102076in}{2.479557in}}{\pgfqpoint{1.105982in}{2.475650in}}%
\pgfpathcurveto{\pgfqpoint{1.109889in}{2.471743in}}{\pgfqpoint{1.115189in}{2.469548in}}{\pgfqpoint{1.120714in}{2.469548in}}%
\pgfpathclose%
\pgfusepath{fill}%
\end{pgfscope}%
\begin{pgfscope}%
\pgfpathrectangle{\pgfqpoint{0.889225in}{2.423832in}}{\pgfqpoint{1.162500in}{0.755000in}}%
\pgfusepath{clip}%
\pgfsetbuttcap%
\pgfsetroundjoin%
\definecolor{currentfill}{rgb}{0.000000,0.000000,0.000000}%
\pgfsetfillcolor{currentfill}%
\pgfsetfillopacity{0.500000}%
\pgfsetlinewidth{0.000000pt}%
\definecolor{currentstroke}{rgb}{0.000000,0.000000,0.000000}%
\pgfsetstrokecolor{currentstroke}%
\pgfsetdash{}{0pt}%
\pgfpathmoveto{\pgfqpoint{1.126248in}{2.488481in}}%
\pgfpathcurveto{\pgfqpoint{1.131773in}{2.488481in}}{\pgfqpoint{1.137073in}{2.490676in}}{\pgfqpoint{1.140980in}{2.494583in}}%
\pgfpathcurveto{\pgfqpoint{1.144886in}{2.498489in}}{\pgfqpoint{1.147082in}{2.503789in}}{\pgfqpoint{1.147082in}{2.509314in}}%
\pgfpathcurveto{\pgfqpoint{1.147082in}{2.514839in}}{\pgfqpoint{1.144886in}{2.520139in}}{\pgfqpoint{1.140980in}{2.524045in}}%
\pgfpathcurveto{\pgfqpoint{1.137073in}{2.527952in}}{\pgfqpoint{1.131773in}{2.530147in}}{\pgfqpoint{1.126248in}{2.530147in}}%
\pgfpathcurveto{\pgfqpoint{1.120723in}{2.530147in}}{\pgfqpoint{1.115424in}{2.527952in}}{\pgfqpoint{1.111517in}{2.524045in}}%
\pgfpathcurveto{\pgfqpoint{1.107610in}{2.520139in}}{\pgfqpoint{1.105415in}{2.514839in}}{\pgfqpoint{1.105415in}{2.509314in}}%
\pgfpathcurveto{\pgfqpoint{1.105415in}{2.503789in}}{\pgfqpoint{1.107610in}{2.498489in}}{\pgfqpoint{1.111517in}{2.494583in}}%
\pgfpathcurveto{\pgfqpoint{1.115424in}{2.490676in}}{\pgfqpoint{1.120723in}{2.488481in}}{\pgfqpoint{1.126248in}{2.488481in}}%
\pgfpathclose%
\pgfusepath{fill}%
\end{pgfscope}%
\begin{pgfscope}%
\pgfpathrectangle{\pgfqpoint{0.889225in}{2.423832in}}{\pgfqpoint{1.162500in}{0.755000in}}%
\pgfusepath{clip}%
\pgfsetbuttcap%
\pgfsetroundjoin%
\definecolor{currentfill}{rgb}{0.000000,0.000000,0.000000}%
\pgfsetfillcolor{currentfill}%
\pgfsetfillopacity{0.500000}%
\pgfsetlinewidth{0.000000pt}%
\definecolor{currentstroke}{rgb}{0.000000,0.000000,0.000000}%
\pgfsetstrokecolor{currentstroke}%
\pgfsetdash{}{0pt}%
\pgfpathmoveto{\pgfqpoint{0.977704in}{2.422812in}}%
\pgfpathcurveto{\pgfqpoint{0.983229in}{2.422812in}}{\pgfqpoint{0.988528in}{2.425007in}}{\pgfqpoint{0.992435in}{2.428914in}}%
\pgfpathcurveto{\pgfqpoint{0.996342in}{2.432821in}}{\pgfqpoint{0.998537in}{2.438120in}}{\pgfqpoint{0.998537in}{2.443645in}}%
\pgfpathcurveto{\pgfqpoint{0.998537in}{2.449170in}}{\pgfqpoint{0.996342in}{2.454470in}}{\pgfqpoint{0.992435in}{2.458377in}}%
\pgfpathcurveto{\pgfqpoint{0.988528in}{2.462284in}}{\pgfqpoint{0.983229in}{2.464479in}}{\pgfqpoint{0.977704in}{2.464479in}}%
\pgfpathcurveto{\pgfqpoint{0.972179in}{2.464479in}}{\pgfqpoint{0.966879in}{2.462284in}}{\pgfqpoint{0.962972in}{2.458377in}}%
\pgfpathcurveto{\pgfqpoint{0.959066in}{2.454470in}}{\pgfqpoint{0.956870in}{2.449170in}}{\pgfqpoint{0.956870in}{2.443645in}}%
\pgfpathcurveto{\pgfqpoint{0.956870in}{2.438120in}}{\pgfqpoint{0.959066in}{2.432821in}}{\pgfqpoint{0.962972in}{2.428914in}}%
\pgfpathcurveto{\pgfqpoint{0.966879in}{2.425007in}}{\pgfqpoint{0.972179in}{2.422812in}}{\pgfqpoint{0.977704in}{2.422812in}}%
\pgfpathclose%
\pgfusepath{fill}%
\end{pgfscope}%
\begin{pgfscope}%
\pgfpathrectangle{\pgfqpoint{0.889225in}{2.423832in}}{\pgfqpoint{1.162500in}{0.755000in}}%
\pgfusepath{clip}%
\pgfsetbuttcap%
\pgfsetroundjoin%
\definecolor{currentfill}{rgb}{0.000000,0.000000,0.000000}%
\pgfsetfillcolor{currentfill}%
\pgfsetfillopacity{0.500000}%
\pgfsetlinewidth{0.000000pt}%
\definecolor{currentstroke}{rgb}{0.000000,0.000000,0.000000}%
\pgfsetstrokecolor{currentstroke}%
\pgfsetdash{}{0pt}%
\pgfpathmoveto{\pgfqpoint{0.916904in}{2.420975in}}%
\pgfpathcurveto{\pgfqpoint{0.922429in}{2.420975in}}{\pgfqpoint{0.927728in}{2.423170in}}{\pgfqpoint{0.931635in}{2.427077in}}%
\pgfpathcurveto{\pgfqpoint{0.935542in}{2.430984in}}{\pgfqpoint{0.937737in}{2.436283in}}{\pgfqpoint{0.937737in}{2.441809in}}%
\pgfpathcurveto{\pgfqpoint{0.937737in}{2.447334in}}{\pgfqpoint{0.935542in}{2.452633in}}{\pgfqpoint{0.931635in}{2.456540in}}%
\pgfpathcurveto{\pgfqpoint{0.927728in}{2.460447in}}{\pgfqpoint{0.922429in}{2.462642in}}{\pgfqpoint{0.916904in}{2.462642in}}%
\pgfpathcurveto{\pgfqpoint{0.911379in}{2.462642in}}{\pgfqpoint{0.906079in}{2.460447in}}{\pgfqpoint{0.902172in}{2.456540in}}%
\pgfpathcurveto{\pgfqpoint{0.898265in}{2.452633in}}{\pgfqpoint{0.896070in}{2.447334in}}{\pgfqpoint{0.896070in}{2.441809in}}%
\pgfpathcurveto{\pgfqpoint{0.896070in}{2.436283in}}{\pgfqpoint{0.898265in}{2.430984in}}{\pgfqpoint{0.902172in}{2.427077in}}%
\pgfpathcurveto{\pgfqpoint{0.906079in}{2.423170in}}{\pgfqpoint{0.911379in}{2.420975in}}{\pgfqpoint{0.916904in}{2.420975in}}%
\pgfpathclose%
\pgfusepath{fill}%
\end{pgfscope}%
\begin{pgfscope}%
\pgfpathrectangle{\pgfqpoint{0.889225in}{2.423832in}}{\pgfqpoint{1.162500in}{0.755000in}}%
\pgfusepath{clip}%
\pgfsetbuttcap%
\pgfsetroundjoin%
\definecolor{currentfill}{rgb}{0.000000,0.000000,0.000000}%
\pgfsetfillcolor{currentfill}%
\pgfsetfillopacity{0.500000}%
\pgfsetlinewidth{0.000000pt}%
\definecolor{currentstroke}{rgb}{0.000000,0.000000,0.000000}%
\pgfsetstrokecolor{currentstroke}%
\pgfsetdash{}{0pt}%
\pgfpathmoveto{\pgfqpoint{1.691400in}{2.717961in}}%
\pgfpathcurveto{\pgfqpoint{1.696925in}{2.717961in}}{\pgfqpoint{1.702225in}{2.720156in}}{\pgfqpoint{1.706132in}{2.724063in}}%
\pgfpathcurveto{\pgfqpoint{1.710038in}{2.727970in}}{\pgfqpoint{1.712234in}{2.733269in}}{\pgfqpoint{1.712234in}{2.738795in}}%
\pgfpathcurveto{\pgfqpoint{1.712234in}{2.744320in}}{\pgfqpoint{1.710038in}{2.749619in}}{\pgfqpoint{1.706132in}{2.753526in}}%
\pgfpathcurveto{\pgfqpoint{1.702225in}{2.757433in}}{\pgfqpoint{1.696925in}{2.759628in}}{\pgfqpoint{1.691400in}{2.759628in}}%
\pgfpathcurveto{\pgfqpoint{1.685875in}{2.759628in}}{\pgfqpoint{1.680576in}{2.757433in}}{\pgfqpoint{1.676669in}{2.753526in}}%
\pgfpathcurveto{\pgfqpoint{1.672762in}{2.749619in}}{\pgfqpoint{1.670567in}{2.744320in}}{\pgfqpoint{1.670567in}{2.738795in}}%
\pgfpathcurveto{\pgfqpoint{1.670567in}{2.733269in}}{\pgfqpoint{1.672762in}{2.727970in}}{\pgfqpoint{1.676669in}{2.724063in}}%
\pgfpathcurveto{\pgfqpoint{1.680576in}{2.720156in}}{\pgfqpoint{1.685875in}{2.717961in}}{\pgfqpoint{1.691400in}{2.717961in}}%
\pgfpathclose%
\pgfusepath{fill}%
\end{pgfscope}%
\begin{pgfscope}%
\pgfpathrectangle{\pgfqpoint{0.889225in}{2.423832in}}{\pgfqpoint{1.162500in}{0.755000in}}%
\pgfusepath{clip}%
\pgfsetbuttcap%
\pgfsetroundjoin%
\definecolor{currentfill}{rgb}{0.000000,0.000000,0.000000}%
\pgfsetfillcolor{currentfill}%
\pgfsetfillopacity{0.500000}%
\pgfsetlinewidth{0.000000pt}%
\definecolor{currentstroke}{rgb}{0.000000,0.000000,0.000000}%
\pgfsetstrokecolor{currentstroke}%
\pgfsetdash{}{0pt}%
\pgfpathmoveto{\pgfqpoint{1.460463in}{2.640479in}}%
\pgfpathcurveto{\pgfqpoint{1.465988in}{2.640479in}}{\pgfqpoint{1.471288in}{2.642674in}}{\pgfqpoint{1.475194in}{2.646581in}}%
\pgfpathcurveto{\pgfqpoint{1.479101in}{2.650487in}}{\pgfqpoint{1.481296in}{2.655787in}}{\pgfqpoint{1.481296in}{2.661312in}}%
\pgfpathcurveto{\pgfqpoint{1.481296in}{2.666837in}}{\pgfqpoint{1.479101in}{2.672137in}}{\pgfqpoint{1.475194in}{2.676043in}}%
\pgfpathcurveto{\pgfqpoint{1.471288in}{2.679950in}}{\pgfqpoint{1.465988in}{2.682145in}}{\pgfqpoint{1.460463in}{2.682145in}}%
\pgfpathcurveto{\pgfqpoint{1.454938in}{2.682145in}}{\pgfqpoint{1.449638in}{2.679950in}}{\pgfqpoint{1.445732in}{2.676043in}}%
\pgfpathcurveto{\pgfqpoint{1.441825in}{2.672137in}}{\pgfqpoint{1.439630in}{2.666837in}}{\pgfqpoint{1.439630in}{2.661312in}}%
\pgfpathcurveto{\pgfqpoint{1.439630in}{2.655787in}}{\pgfqpoint{1.441825in}{2.650487in}}{\pgfqpoint{1.445732in}{2.646581in}}%
\pgfpathcurveto{\pgfqpoint{1.449638in}{2.642674in}}{\pgfqpoint{1.454938in}{2.640479in}}{\pgfqpoint{1.460463in}{2.640479in}}%
\pgfpathclose%
\pgfusepath{fill}%
\end{pgfscope}%
\begin{pgfscope}%
\pgfpathrectangle{\pgfqpoint{0.889225in}{2.423832in}}{\pgfqpoint{1.162500in}{0.755000in}}%
\pgfusepath{clip}%
\pgfsetbuttcap%
\pgfsetroundjoin%
\definecolor{currentfill}{rgb}{0.000000,0.000000,0.000000}%
\pgfsetfillcolor{currentfill}%
\pgfsetfillopacity{0.500000}%
\pgfsetlinewidth{0.000000pt}%
\definecolor{currentstroke}{rgb}{0.000000,0.000000,0.000000}%
\pgfsetstrokecolor{currentstroke}%
\pgfsetdash{}{0pt}%
\pgfpathmoveto{\pgfqpoint{1.303193in}{2.637393in}}%
\pgfpathcurveto{\pgfqpoint{1.308718in}{2.637393in}}{\pgfqpoint{1.314017in}{2.639588in}}{\pgfqpoint{1.317924in}{2.643495in}}%
\pgfpathcurveto{\pgfqpoint{1.321831in}{2.647402in}}{\pgfqpoint{1.324026in}{2.652701in}}{\pgfqpoint{1.324026in}{2.658226in}}%
\pgfpathcurveto{\pgfqpoint{1.324026in}{2.663751in}}{\pgfqpoint{1.321831in}{2.669051in}}{\pgfqpoint{1.317924in}{2.672958in}}%
\pgfpathcurveto{\pgfqpoint{1.314017in}{2.676865in}}{\pgfqpoint{1.308718in}{2.679060in}}{\pgfqpoint{1.303193in}{2.679060in}}%
\pgfpathcurveto{\pgfqpoint{1.297667in}{2.679060in}}{\pgfqpoint{1.292368in}{2.676865in}}{\pgfqpoint{1.288461in}{2.672958in}}%
\pgfpathcurveto{\pgfqpoint{1.284554in}{2.669051in}}{\pgfqpoint{1.282359in}{2.663751in}}{\pgfqpoint{1.282359in}{2.658226in}}%
\pgfpathcurveto{\pgfqpoint{1.282359in}{2.652701in}}{\pgfqpoint{1.284554in}{2.647402in}}{\pgfqpoint{1.288461in}{2.643495in}}%
\pgfpathcurveto{\pgfqpoint{1.292368in}{2.639588in}}{\pgfqpoint{1.297667in}{2.637393in}}{\pgfqpoint{1.303193in}{2.637393in}}%
\pgfpathclose%
\pgfusepath{fill}%
\end{pgfscope}%
\begin{pgfscope}%
\pgfpathrectangle{\pgfqpoint{0.889225in}{2.423832in}}{\pgfqpoint{1.162500in}{0.755000in}}%
\pgfusepath{clip}%
\pgfsetbuttcap%
\pgfsetroundjoin%
\definecolor{currentfill}{rgb}{0.000000,0.000000,0.000000}%
\pgfsetfillcolor{currentfill}%
\pgfsetfillopacity{0.500000}%
\pgfsetlinewidth{0.000000pt}%
\definecolor{currentstroke}{rgb}{0.000000,0.000000,0.000000}%
\pgfsetstrokecolor{currentstroke}%
\pgfsetdash{}{0pt}%
\pgfpathmoveto{\pgfqpoint{1.689350in}{2.679970in}}%
\pgfpathcurveto{\pgfqpoint{1.694875in}{2.679970in}}{\pgfqpoint{1.700175in}{2.682166in}}{\pgfqpoint{1.704082in}{2.686072in}}%
\pgfpathcurveto{\pgfqpoint{1.707989in}{2.689979in}}{\pgfqpoint{1.710184in}{2.695279in}}{\pgfqpoint{1.710184in}{2.700804in}}%
\pgfpathcurveto{\pgfqpoint{1.710184in}{2.706329in}}{\pgfqpoint{1.707989in}{2.711628in}}{\pgfqpoint{1.704082in}{2.715535in}}%
\pgfpathcurveto{\pgfqpoint{1.700175in}{2.719442in}}{\pgfqpoint{1.694875in}{2.721637in}}{\pgfqpoint{1.689350in}{2.721637in}}%
\pgfpathcurveto{\pgfqpoint{1.683825in}{2.721637in}}{\pgfqpoint{1.678526in}{2.719442in}}{\pgfqpoint{1.674619in}{2.715535in}}%
\pgfpathcurveto{\pgfqpoint{1.670712in}{2.711628in}}{\pgfqpoint{1.668517in}{2.706329in}}{\pgfqpoint{1.668517in}{2.700804in}}%
\pgfpathcurveto{\pgfqpoint{1.668517in}{2.695279in}}{\pgfqpoint{1.670712in}{2.689979in}}{\pgfqpoint{1.674619in}{2.686072in}}%
\pgfpathcurveto{\pgfqpoint{1.678526in}{2.682166in}}{\pgfqpoint{1.683825in}{2.679970in}}{\pgfqpoint{1.689350in}{2.679970in}}%
\pgfpathclose%
\pgfusepath{fill}%
\end{pgfscope}%
\begin{pgfscope}%
\pgfpathrectangle{\pgfqpoint{0.889225in}{2.423832in}}{\pgfqpoint{1.162500in}{0.755000in}}%
\pgfusepath{clip}%
\pgfsetbuttcap%
\pgfsetroundjoin%
\definecolor{currentfill}{rgb}{0.000000,0.000000,0.000000}%
\pgfsetfillcolor{currentfill}%
\pgfsetfillopacity{0.500000}%
\pgfsetlinewidth{0.000000pt}%
\definecolor{currentstroke}{rgb}{0.000000,0.000000,0.000000}%
\pgfsetstrokecolor{currentstroke}%
\pgfsetdash{}{0pt}%
\pgfpathmoveto{\pgfqpoint{1.102396in}{2.481302in}}%
\pgfpathcurveto{\pgfqpoint{1.107921in}{2.481302in}}{\pgfqpoint{1.113221in}{2.483497in}}{\pgfqpoint{1.117128in}{2.487404in}}%
\pgfpathcurveto{\pgfqpoint{1.121035in}{2.491311in}}{\pgfqpoint{1.123230in}{2.496610in}}{\pgfqpoint{1.123230in}{2.502135in}}%
\pgfpathcurveto{\pgfqpoint{1.123230in}{2.507661in}}{\pgfqpoint{1.121035in}{2.512960in}}{\pgfqpoint{1.117128in}{2.516867in}}%
\pgfpathcurveto{\pgfqpoint{1.113221in}{2.520774in}}{\pgfqpoint{1.107921in}{2.522969in}}{\pgfqpoint{1.102396in}{2.522969in}}%
\pgfpathcurveto{\pgfqpoint{1.096871in}{2.522969in}}{\pgfqpoint{1.091572in}{2.520774in}}{\pgfqpoint{1.087665in}{2.516867in}}%
\pgfpathcurveto{\pgfqpoint{1.083758in}{2.512960in}}{\pgfqpoint{1.081563in}{2.507661in}}{\pgfqpoint{1.081563in}{2.502135in}}%
\pgfpathcurveto{\pgfqpoint{1.081563in}{2.496610in}}{\pgfqpoint{1.083758in}{2.491311in}}{\pgfqpoint{1.087665in}{2.487404in}}%
\pgfpathcurveto{\pgfqpoint{1.091572in}{2.483497in}}{\pgfqpoint{1.096871in}{2.481302in}}{\pgfqpoint{1.102396in}{2.481302in}}%
\pgfpathclose%
\pgfusepath{fill}%
\end{pgfscope}%
\begin{pgfscope}%
\pgfpathrectangle{\pgfqpoint{0.889225in}{2.423832in}}{\pgfqpoint{1.162500in}{0.755000in}}%
\pgfusepath{clip}%
\pgfsetbuttcap%
\pgfsetroundjoin%
\definecolor{currentfill}{rgb}{0.000000,0.000000,0.000000}%
\pgfsetfillcolor{currentfill}%
\pgfsetfillopacity{0.500000}%
\pgfsetlinewidth{0.000000pt}%
\definecolor{currentstroke}{rgb}{0.000000,0.000000,0.000000}%
\pgfsetstrokecolor{currentstroke}%
\pgfsetdash{}{0pt}%
\pgfpathmoveto{\pgfqpoint{1.143186in}{2.503012in}}%
\pgfpathcurveto{\pgfqpoint{1.148711in}{2.503012in}}{\pgfqpoint{1.154011in}{2.505207in}}{\pgfqpoint{1.157918in}{2.509114in}}%
\pgfpathcurveto{\pgfqpoint{1.161825in}{2.513021in}}{\pgfqpoint{1.164020in}{2.518320in}}{\pgfqpoint{1.164020in}{2.523845in}}%
\pgfpathcurveto{\pgfqpoint{1.164020in}{2.529370in}}{\pgfqpoint{1.161825in}{2.534670in}}{\pgfqpoint{1.157918in}{2.538577in}}%
\pgfpathcurveto{\pgfqpoint{1.154011in}{2.542483in}}{\pgfqpoint{1.148711in}{2.544679in}}{\pgfqpoint{1.143186in}{2.544679in}}%
\pgfpathcurveto{\pgfqpoint{1.137661in}{2.544679in}}{\pgfqpoint{1.132362in}{2.542483in}}{\pgfqpoint{1.128455in}{2.538577in}}%
\pgfpathcurveto{\pgfqpoint{1.124548in}{2.534670in}}{\pgfqpoint{1.122353in}{2.529370in}}{\pgfqpoint{1.122353in}{2.523845in}}%
\pgfpathcurveto{\pgfqpoint{1.122353in}{2.518320in}}{\pgfqpoint{1.124548in}{2.513021in}}{\pgfqpoint{1.128455in}{2.509114in}}%
\pgfpathcurveto{\pgfqpoint{1.132362in}{2.505207in}}{\pgfqpoint{1.137661in}{2.503012in}}{\pgfqpoint{1.143186in}{2.503012in}}%
\pgfpathclose%
\pgfusepath{fill}%
\end{pgfscope}%
\begin{pgfscope}%
\pgfpathrectangle{\pgfqpoint{0.889225in}{2.423832in}}{\pgfqpoint{1.162500in}{0.755000in}}%
\pgfusepath{clip}%
\pgfsetbuttcap%
\pgfsetroundjoin%
\definecolor{currentfill}{rgb}{0.000000,0.000000,0.000000}%
\pgfsetfillcolor{currentfill}%
\pgfsetfillopacity{0.500000}%
\pgfsetlinewidth{0.000000pt}%
\definecolor{currentstroke}{rgb}{0.000000,0.000000,0.000000}%
\pgfsetstrokecolor{currentstroke}%
\pgfsetdash{}{0pt}%
\pgfpathmoveto{\pgfqpoint{2.024046in}{3.140023in}}%
\pgfpathcurveto{\pgfqpoint{2.029571in}{3.140023in}}{\pgfqpoint{2.034871in}{3.142218in}}{\pgfqpoint{2.038778in}{3.146125in}}%
\pgfpathcurveto{\pgfqpoint{2.042685in}{3.150032in}}{\pgfqpoint{2.044880in}{3.155331in}}{\pgfqpoint{2.044880in}{3.160856in}}%
\pgfpathcurveto{\pgfqpoint{2.044880in}{3.166381in}}{\pgfqpoint{2.042685in}{3.171681in}}{\pgfqpoint{2.038778in}{3.175588in}}%
\pgfpathcurveto{\pgfqpoint{2.034871in}{3.179494in}}{\pgfqpoint{2.029571in}{3.181690in}}{\pgfqpoint{2.024046in}{3.181690in}}%
\pgfpathcurveto{\pgfqpoint{2.018521in}{3.181690in}}{\pgfqpoint{2.013222in}{3.179494in}}{\pgfqpoint{2.009315in}{3.175588in}}%
\pgfpathcurveto{\pgfqpoint{2.005408in}{3.171681in}}{\pgfqpoint{2.003213in}{3.166381in}}{\pgfqpoint{2.003213in}{3.160856in}}%
\pgfpathcurveto{\pgfqpoint{2.003213in}{3.155331in}}{\pgfqpoint{2.005408in}{3.150032in}}{\pgfqpoint{2.009315in}{3.146125in}}%
\pgfpathcurveto{\pgfqpoint{2.013222in}{3.142218in}}{\pgfqpoint{2.018521in}{3.140023in}}{\pgfqpoint{2.024046in}{3.140023in}}%
\pgfpathclose%
\pgfusepath{fill}%
\end{pgfscope}%
\begin{pgfscope}%
\pgfpathrectangle{\pgfqpoint{0.889225in}{2.423832in}}{\pgfqpoint{1.162500in}{0.755000in}}%
\pgfusepath{clip}%
\pgfsetbuttcap%
\pgfsetroundjoin%
\definecolor{currentfill}{rgb}{0.000000,0.000000,0.000000}%
\pgfsetfillcolor{currentfill}%
\pgfsetfillopacity{0.500000}%
\pgfsetlinewidth{0.000000pt}%
\definecolor{currentstroke}{rgb}{0.000000,0.000000,0.000000}%
\pgfsetstrokecolor{currentstroke}%
\pgfsetdash{}{0pt}%
\pgfpathmoveto{\pgfqpoint{1.402285in}{2.625365in}}%
\pgfpathcurveto{\pgfqpoint{1.407810in}{2.625365in}}{\pgfqpoint{1.413110in}{2.627561in}}{\pgfqpoint{1.417016in}{2.631467in}}%
\pgfpathcurveto{\pgfqpoint{1.420923in}{2.635374in}}{\pgfqpoint{1.423118in}{2.640674in}}{\pgfqpoint{1.423118in}{2.646199in}}%
\pgfpathcurveto{\pgfqpoint{1.423118in}{2.651724in}}{\pgfqpoint{1.420923in}{2.657023in}}{\pgfqpoint{1.417016in}{2.660930in}}%
\pgfpathcurveto{\pgfqpoint{1.413110in}{2.664837in}}{\pgfqpoint{1.407810in}{2.667032in}}{\pgfqpoint{1.402285in}{2.667032in}}%
\pgfpathcurveto{\pgfqpoint{1.396760in}{2.667032in}}{\pgfqpoint{1.391460in}{2.664837in}}{\pgfqpoint{1.387554in}{2.660930in}}%
\pgfpathcurveto{\pgfqpoint{1.383647in}{2.657023in}}{\pgfqpoint{1.381452in}{2.651724in}}{\pgfqpoint{1.381452in}{2.646199in}}%
\pgfpathcurveto{\pgfqpoint{1.381452in}{2.640674in}}{\pgfqpoint{1.383647in}{2.635374in}}{\pgfqpoint{1.387554in}{2.631467in}}%
\pgfpathcurveto{\pgfqpoint{1.391460in}{2.627561in}}{\pgfqpoint{1.396760in}{2.625365in}}{\pgfqpoint{1.402285in}{2.625365in}}%
\pgfpathclose%
\pgfusepath{fill}%
\end{pgfscope}%
\begin{pgfscope}%
\pgfpathrectangle{\pgfqpoint{0.889225in}{2.423832in}}{\pgfqpoint{1.162500in}{0.755000in}}%
\pgfusepath{clip}%
\pgfsetbuttcap%
\pgfsetroundjoin%
\definecolor{currentfill}{rgb}{0.000000,0.000000,0.000000}%
\pgfsetfillcolor{currentfill}%
\pgfsetfillopacity{0.500000}%
\pgfsetlinewidth{0.000000pt}%
\definecolor{currentstroke}{rgb}{0.000000,0.000000,0.000000}%
\pgfsetstrokecolor{currentstroke}%
\pgfsetdash{}{0pt}%
\pgfpathmoveto{\pgfqpoint{0.973483in}{2.450809in}}%
\pgfpathcurveto{\pgfqpoint{0.979008in}{2.450809in}}{\pgfqpoint{0.984308in}{2.453004in}}{\pgfqpoint{0.988214in}{2.456911in}}%
\pgfpathcurveto{\pgfqpoint{0.992121in}{2.460818in}}{\pgfqpoint{0.994316in}{2.466117in}}{\pgfqpoint{0.994316in}{2.471642in}}%
\pgfpathcurveto{\pgfqpoint{0.994316in}{2.477167in}}{\pgfqpoint{0.992121in}{2.482467in}}{\pgfqpoint{0.988214in}{2.486374in}}%
\pgfpathcurveto{\pgfqpoint{0.984308in}{2.490280in}}{\pgfqpoint{0.979008in}{2.492476in}}{\pgfqpoint{0.973483in}{2.492476in}}%
\pgfpathcurveto{\pgfqpoint{0.967958in}{2.492476in}}{\pgfqpoint{0.962658in}{2.490280in}}{\pgfqpoint{0.958752in}{2.486374in}}%
\pgfpathcurveto{\pgfqpoint{0.954845in}{2.482467in}}{\pgfqpoint{0.952650in}{2.477167in}}{\pgfqpoint{0.952650in}{2.471642in}}%
\pgfpathcurveto{\pgfqpoint{0.952650in}{2.466117in}}{\pgfqpoint{0.954845in}{2.460818in}}{\pgfqpoint{0.958752in}{2.456911in}}%
\pgfpathcurveto{\pgfqpoint{0.962658in}{2.453004in}}{\pgfqpoint{0.967958in}{2.450809in}}{\pgfqpoint{0.973483in}{2.450809in}}%
\pgfpathclose%
\pgfusepath{fill}%
\end{pgfscope}%
\begin{pgfscope}%
\pgfsetbuttcap%
\pgfsetroundjoin%
\definecolor{currentfill}{rgb}{0.000000,0.000000,0.000000}%
\pgfsetfillcolor{currentfill}%
\pgfsetlinewidth{0.803000pt}%
\definecolor{currentstroke}{rgb}{0.000000,0.000000,0.000000}%
\pgfsetstrokecolor{currentstroke}%
\pgfsetdash{}{0pt}%
\pgfsys@defobject{currentmarker}{\pgfqpoint{-0.048611in}{0.000000in}}{\pgfqpoint{0.000000in}{0.000000in}}{%
\pgfpathmoveto{\pgfqpoint{0.000000in}{0.000000in}}%
\pgfpathlineto{\pgfqpoint{-0.048611in}{0.000000in}}%
\pgfusepath{stroke,fill}%
}%
\begin{pgfscope}%
\pgfsys@transformshift{0.889225in}{2.691122in}%
\pgfsys@useobject{currentmarker}{}%
\end{pgfscope}%
\end{pgfscope}%
\begin{pgfscope}%
\pgftext[x=0.641152in,y=2.648913in,left,base]{\rmfamily\fontsize{8.000000}{9.600000}\selectfont \(\displaystyle 0.5\)}%
\end{pgfscope}%
\begin{pgfscope}%
\pgfsetbuttcap%
\pgfsetroundjoin%
\definecolor{currentfill}{rgb}{0.000000,0.000000,0.000000}%
\pgfsetfillcolor{currentfill}%
\pgfsetlinewidth{0.803000pt}%
\definecolor{currentstroke}{rgb}{0.000000,0.000000,0.000000}%
\pgfsetstrokecolor{currentstroke}%
\pgfsetdash{}{0pt}%
\pgfsys@defobject{currentmarker}{\pgfqpoint{-0.048611in}{0.000000in}}{\pgfqpoint{0.000000in}{0.000000in}}{%
\pgfpathmoveto{\pgfqpoint{0.000000in}{0.000000in}}%
\pgfpathlineto{\pgfqpoint{-0.048611in}{0.000000in}}%
\pgfusepath{stroke,fill}%
}%
\begin{pgfscope}%
\pgfsys@transformshift{0.889225in}{2.985310in}%
\pgfsys@useobject{currentmarker}{}%
\end{pgfscope}%
\end{pgfscope}%
\begin{pgfscope}%
\pgftext[x=0.641152in,y=2.943101in,left,base]{\rmfamily\fontsize{8.000000}{9.600000}\selectfont \(\displaystyle 1.0\)}%
\end{pgfscope}%
\begin{pgfscope}%
\pgftext[x=0.585596in,y=2.801332in,,bottom,rotate=90.000000]{\rmfamily\fontsize{16.000000}{19.200000}\selectfont charge}%
\end{pgfscope}%
\begin{pgfscope}%
\pgftext[x=0.889225in,y=3.220499in,left,base]{\rmfamily\fontsize{16.000000}{19.200000}\selectfont \(\displaystyle \times10^{-9}\)}%
\end{pgfscope}%
\begin{pgfscope}%
\pgfsetrectcap%
\pgfsetmiterjoin%
\pgfsetlinewidth{0.803000pt}%
\definecolor{currentstroke}{rgb}{0.501961,0.501961,0.501961}%
\pgfsetstrokecolor{currentstroke}%
\pgfsetdash{}{0pt}%
\pgfpathmoveto{\pgfqpoint{0.889225in}{2.423832in}}%
\pgfpathlineto{\pgfqpoint{0.889225in}{3.178832in}}%
\pgfusepath{stroke}%
\end{pgfscope}%
\begin{pgfscope}%
\pgfsetrectcap%
\pgfsetmiterjoin%
\pgfsetlinewidth{0.803000pt}%
\definecolor{currentstroke}{rgb}{0.501961,0.501961,0.501961}%
\pgfsetstrokecolor{currentstroke}%
\pgfsetdash{}{0pt}%
\pgfpathmoveto{\pgfqpoint{2.051725in}{2.423832in}}%
\pgfpathlineto{\pgfqpoint{2.051725in}{3.178832in}}%
\pgfusepath{stroke}%
\end{pgfscope}%
\begin{pgfscope}%
\pgfsetrectcap%
\pgfsetmiterjoin%
\pgfsetlinewidth{0.803000pt}%
\definecolor{currentstroke}{rgb}{0.501961,0.501961,0.501961}%
\pgfsetstrokecolor{currentstroke}%
\pgfsetdash{}{0pt}%
\pgfpathmoveto{\pgfqpoint{0.889225in}{2.423832in}}%
\pgfpathlineto{\pgfqpoint{2.051725in}{2.423832in}}%
\pgfusepath{stroke}%
\end{pgfscope}%
\begin{pgfscope}%
\pgfsetrectcap%
\pgfsetmiterjoin%
\pgfsetlinewidth{0.803000pt}%
\definecolor{currentstroke}{rgb}{0.501961,0.501961,0.501961}%
\pgfsetstrokecolor{currentstroke}%
\pgfsetdash{}{0pt}%
\pgfpathmoveto{\pgfqpoint{0.889225in}{3.178832in}}%
\pgfpathlineto{\pgfqpoint{2.051725in}{3.178832in}}%
\pgfusepath{stroke}%
\end{pgfscope}%
\begin{pgfscope}%
\pgfsetbuttcap%
\pgfsetmiterjoin%
\definecolor{currentfill}{rgb}{1.000000,1.000000,1.000000}%
\pgfsetfillcolor{currentfill}%
\pgfsetlinewidth{0.000000pt}%
\definecolor{currentstroke}{rgb}{0.000000,0.000000,0.000000}%
\pgfsetstrokecolor{currentstroke}%
\pgfsetstrokeopacity{0.000000}%
\pgfsetdash{}{0pt}%
\pgfpathmoveto{\pgfqpoint{2.051725in}{2.423832in}}%
\pgfpathlineto{\pgfqpoint{3.214225in}{2.423832in}}%
\pgfpathlineto{\pgfqpoint{3.214225in}{3.178832in}}%
\pgfpathlineto{\pgfqpoint{2.051725in}{3.178832in}}%
\pgfpathclose%
\pgfusepath{fill}%
\end{pgfscope}%
\begin{pgfscope}%
\pgfpathrectangle{\pgfqpoint{2.051725in}{2.423832in}}{\pgfqpoint{1.162500in}{0.755000in}}%
\pgfusepath{clip}%
\pgfsetrectcap%
\pgfsetroundjoin%
\pgfsetlinewidth{1.505625pt}%
\definecolor{currentstroke}{rgb}{0.121569,0.466667,0.705882}%
\pgfsetstrokecolor{currentstroke}%
\pgfsetdash{}{0pt}%
\pgfpathmoveto{\pgfqpoint{2.079404in}{2.981458in}}%
\pgfpathlineto{\pgfqpoint{2.103785in}{3.023270in}}%
\pgfpathlineto{\pgfqpoint{2.124842in}{3.055299in}}%
\pgfpathlineto{\pgfqpoint{2.143682in}{3.080259in}}%
\pgfpathlineto{\pgfqpoint{2.161414in}{3.100257in}}%
\pgfpathlineto{\pgfqpoint{2.178038in}{3.115741in}}%
\pgfpathlineto{\pgfqpoint{2.193553in}{3.127237in}}%
\pgfpathlineto{\pgfqpoint{2.207961in}{3.135306in}}%
\pgfpathlineto{\pgfqpoint{2.222368in}{3.140849in}}%
\pgfpathlineto{\pgfqpoint{2.235667in}{3.143732in}}%
\pgfpathlineto{\pgfqpoint{2.248966in}{3.144496in}}%
\pgfpathlineto{\pgfqpoint{2.262265in}{3.143183in}}%
\pgfpathlineto{\pgfqpoint{2.276672in}{3.139480in}}%
\pgfpathlineto{\pgfqpoint{2.291080in}{3.133489in}}%
\pgfpathlineto{\pgfqpoint{2.306595in}{3.124594in}}%
\pgfpathlineto{\pgfqpoint{2.323219in}{3.112404in}}%
\pgfpathlineto{\pgfqpoint{2.340951in}{3.096574in}}%
\pgfpathlineto{\pgfqpoint{2.360899in}{3.075567in}}%
\pgfpathlineto{\pgfqpoint{2.381956in}{3.050096in}}%
\pgfpathlineto{\pgfqpoint{2.406338in}{3.016922in}}%
\pgfpathlineto{\pgfqpoint{2.434044in}{2.975246in}}%
\pgfpathlineto{\pgfqpoint{2.467291in}{2.920995in}}%
\pgfpathlineto{\pgfqpoint{2.514946in}{2.838468in}}%
\pgfpathlineto{\pgfqpoint{2.590307in}{2.707949in}}%
\pgfpathlineto{\pgfqpoint{2.625771in}{2.651279in}}%
\pgfpathlineto{\pgfqpoint{2.654586in}{2.609241in}}%
\pgfpathlineto{\pgfqpoint{2.680076in}{2.575738in}}%
\pgfpathlineto{\pgfqpoint{2.704457in}{2.547348in}}%
\pgfpathlineto{\pgfqpoint{2.726622in}{2.524875in}}%
\pgfpathlineto{\pgfqpoint{2.747679in}{2.506592in}}%
\pgfpathlineto{\pgfqpoint{2.767628in}{2.492072in}}%
\pgfpathlineto{\pgfqpoint{2.787576in}{2.480266in}}%
\pgfpathlineto{\pgfqpoint{2.807525in}{2.471117in}}%
\pgfpathlineto{\pgfqpoint{2.827473in}{2.464520in}}%
\pgfpathlineto{\pgfqpoint{2.847422in}{2.460331in}}%
\pgfpathlineto{\pgfqpoint{2.867370in}{2.458366in}}%
\pgfpathlineto{\pgfqpoint{2.888427in}{2.458462in}}%
\pgfpathlineto{\pgfqpoint{2.911700in}{2.460809in}}%
\pgfpathlineto{\pgfqpoint{2.938298in}{2.465819in}}%
\pgfpathlineto{\pgfqpoint{2.971546in}{2.474517in}}%
\pgfpathlineto{\pgfqpoint{3.079046in}{2.504560in}}%
\pgfpathlineto{\pgfqpoint{3.107861in}{2.509223in}}%
\pgfpathlineto{\pgfqpoint{3.134459in}{2.511245in}}%
\pgfpathlineto{\pgfqpoint{3.159948in}{2.510915in}}%
\pgfpathlineto{\pgfqpoint{3.185438in}{2.508314in}}%
\pgfpathlineto{\pgfqpoint{3.186546in}{2.508151in}}%
\pgfpathlineto{\pgfqpoint{3.186546in}{2.508151in}}%
\pgfusepath{stroke}%
\end{pgfscope}%
\begin{pgfscope}%
\pgfsetrectcap%
\pgfsetmiterjoin%
\pgfsetlinewidth{0.803000pt}%
\definecolor{currentstroke}{rgb}{0.501961,0.501961,0.501961}%
\pgfsetstrokecolor{currentstroke}%
\pgfsetdash{}{0pt}%
\pgfpathmoveto{\pgfqpoint{2.051725in}{2.423832in}}%
\pgfpathlineto{\pgfqpoint{2.051725in}{3.178832in}}%
\pgfusepath{stroke}%
\end{pgfscope}%
\begin{pgfscope}%
\pgfsetrectcap%
\pgfsetmiterjoin%
\pgfsetlinewidth{0.803000pt}%
\definecolor{currentstroke}{rgb}{0.501961,0.501961,0.501961}%
\pgfsetstrokecolor{currentstroke}%
\pgfsetdash{}{0pt}%
\pgfpathmoveto{\pgfqpoint{3.214225in}{2.423832in}}%
\pgfpathlineto{\pgfqpoint{3.214225in}{3.178832in}}%
\pgfusepath{stroke}%
\end{pgfscope}%
\begin{pgfscope}%
\pgfsetrectcap%
\pgfsetmiterjoin%
\pgfsetlinewidth{0.803000pt}%
\definecolor{currentstroke}{rgb}{0.501961,0.501961,0.501961}%
\pgfsetstrokecolor{currentstroke}%
\pgfsetdash{}{0pt}%
\pgfpathmoveto{\pgfqpoint{2.051725in}{2.423832in}}%
\pgfpathlineto{\pgfqpoint{3.214225in}{2.423832in}}%
\pgfusepath{stroke}%
\end{pgfscope}%
\begin{pgfscope}%
\pgfsetrectcap%
\pgfsetmiterjoin%
\pgfsetlinewidth{0.803000pt}%
\definecolor{currentstroke}{rgb}{0.501961,0.501961,0.501961}%
\pgfsetstrokecolor{currentstroke}%
\pgfsetdash{}{0pt}%
\pgfpathmoveto{\pgfqpoint{2.051725in}{3.178832in}}%
\pgfpathlineto{\pgfqpoint{3.214225in}{3.178832in}}%
\pgfusepath{stroke}%
\end{pgfscope}%
\begin{pgfscope}%
\pgfsetbuttcap%
\pgfsetmiterjoin%
\definecolor{currentfill}{rgb}{1.000000,1.000000,1.000000}%
\pgfsetfillcolor{currentfill}%
\pgfsetlinewidth{0.000000pt}%
\definecolor{currentstroke}{rgb}{0.000000,0.000000,0.000000}%
\pgfsetstrokecolor{currentstroke}%
\pgfsetstrokeopacity{0.000000}%
\pgfsetdash{}{0pt}%
\pgfpathmoveto{\pgfqpoint{3.214225in}{2.423832in}}%
\pgfpathlineto{\pgfqpoint{4.376725in}{2.423832in}}%
\pgfpathlineto{\pgfqpoint{4.376725in}{3.178832in}}%
\pgfpathlineto{\pgfqpoint{3.214225in}{3.178832in}}%
\pgfpathclose%
\pgfusepath{fill}%
\end{pgfscope}%
\begin{pgfscope}%
\pgfpathrectangle{\pgfqpoint{3.214225in}{2.423832in}}{\pgfqpoint{1.162500in}{0.755000in}}%
\pgfusepath{clip}%
\pgfsetbuttcap%
\pgfsetroundjoin%
\definecolor{currentfill}{rgb}{0.000000,0.000000,0.000000}%
\pgfsetfillcolor{currentfill}%
\pgfsetfillopacity{0.500000}%
\pgfsetlinewidth{0.000000pt}%
\definecolor{currentstroke}{rgb}{0.000000,0.000000,0.000000}%
\pgfsetstrokecolor{currentstroke}%
\pgfsetdash{}{0pt}%
\pgfpathmoveto{\pgfqpoint{3.656364in}{3.088730in}}%
\pgfpathcurveto{\pgfqpoint{3.661889in}{3.088730in}}{\pgfqpoint{3.667189in}{3.090925in}}{\pgfqpoint{3.671096in}{3.094832in}}%
\pgfpathcurveto{\pgfqpoint{3.675002in}{3.098739in}}{\pgfqpoint{3.677198in}{3.104038in}}{\pgfqpoint{3.677198in}{3.109563in}}%
\pgfpathcurveto{\pgfqpoint{3.677198in}{3.115089in}}{\pgfqpoint{3.675002in}{3.120388in}}{\pgfqpoint{3.671096in}{3.124295in}}%
\pgfpathcurveto{\pgfqpoint{3.667189in}{3.128202in}}{\pgfqpoint{3.661889in}{3.130397in}}{\pgfqpoint{3.656364in}{3.130397in}}%
\pgfpathcurveto{\pgfqpoint{3.650839in}{3.130397in}}{\pgfqpoint{3.645540in}{3.128202in}}{\pgfqpoint{3.641633in}{3.124295in}}%
\pgfpathcurveto{\pgfqpoint{3.637726in}{3.120388in}}{\pgfqpoint{3.635531in}{3.115089in}}{\pgfqpoint{3.635531in}{3.109563in}}%
\pgfpathcurveto{\pgfqpoint{3.635531in}{3.104038in}}{\pgfqpoint{3.637726in}{3.098739in}}{\pgfqpoint{3.641633in}{3.094832in}}%
\pgfpathcurveto{\pgfqpoint{3.645540in}{3.090925in}}{\pgfqpoint{3.650839in}{3.088730in}}{\pgfqpoint{3.656364in}{3.088730in}}%
\pgfpathclose%
\pgfusepath{fill}%
\end{pgfscope}%
\begin{pgfscope}%
\pgfpathrectangle{\pgfqpoint{3.214225in}{2.423832in}}{\pgfqpoint{1.162500in}{0.755000in}}%
\pgfusepath{clip}%
\pgfsetbuttcap%
\pgfsetroundjoin%
\definecolor{currentfill}{rgb}{0.000000,0.000000,0.000000}%
\pgfsetfillcolor{currentfill}%
\pgfsetfillopacity{0.500000}%
\pgfsetlinewidth{0.000000pt}%
\definecolor{currentstroke}{rgb}{0.000000,0.000000,0.000000}%
\pgfsetstrokecolor{currentstroke}%
\pgfsetdash{}{0pt}%
\pgfpathmoveto{\pgfqpoint{4.259629in}{2.544873in}}%
\pgfpathcurveto{\pgfqpoint{4.265154in}{2.544873in}}{\pgfqpoint{4.270454in}{2.547068in}}{\pgfqpoint{4.274361in}{2.550975in}}%
\pgfpathcurveto{\pgfqpoint{4.278268in}{2.554881in}}{\pgfqpoint{4.280463in}{2.560181in}}{\pgfqpoint{4.280463in}{2.565706in}}%
\pgfpathcurveto{\pgfqpoint{4.280463in}{2.571231in}}{\pgfqpoint{4.278268in}{2.576531in}}{\pgfqpoint{4.274361in}{2.580437in}}%
\pgfpathcurveto{\pgfqpoint{4.270454in}{2.584344in}}{\pgfqpoint{4.265154in}{2.586539in}}{\pgfqpoint{4.259629in}{2.586539in}}%
\pgfpathcurveto{\pgfqpoint{4.254104in}{2.586539in}}{\pgfqpoint{4.248805in}{2.584344in}}{\pgfqpoint{4.244898in}{2.580437in}}%
\pgfpathcurveto{\pgfqpoint{4.240991in}{2.576531in}}{\pgfqpoint{4.238796in}{2.571231in}}{\pgfqpoint{4.238796in}{2.565706in}}%
\pgfpathcurveto{\pgfqpoint{4.238796in}{2.560181in}}{\pgfqpoint{4.240991in}{2.554881in}}{\pgfqpoint{4.244898in}{2.550975in}}%
\pgfpathcurveto{\pgfqpoint{4.248805in}{2.547068in}}{\pgfqpoint{4.254104in}{2.544873in}}{\pgfqpoint{4.259629in}{2.544873in}}%
\pgfpathclose%
\pgfusepath{fill}%
\end{pgfscope}%
\begin{pgfscope}%
\pgfpathrectangle{\pgfqpoint{3.214225in}{2.423832in}}{\pgfqpoint{1.162500in}{0.755000in}}%
\pgfusepath{clip}%
\pgfsetbuttcap%
\pgfsetroundjoin%
\definecolor{currentfill}{rgb}{0.000000,0.000000,0.000000}%
\pgfsetfillcolor{currentfill}%
\pgfsetfillopacity{0.500000}%
\pgfsetlinewidth{0.000000pt}%
\definecolor{currentstroke}{rgb}{0.000000,0.000000,0.000000}%
\pgfsetstrokecolor{currentstroke}%
\pgfsetdash{}{0pt}%
\pgfpathmoveto{\pgfqpoint{4.349046in}{2.541427in}}%
\pgfpathcurveto{\pgfqpoint{4.354571in}{2.541427in}}{\pgfqpoint{4.359871in}{2.543622in}}{\pgfqpoint{4.363778in}{2.547529in}}%
\pgfpathcurveto{\pgfqpoint{4.367685in}{2.551436in}}{\pgfqpoint{4.369880in}{2.556735in}}{\pgfqpoint{4.369880in}{2.562260in}}%
\pgfpathcurveto{\pgfqpoint{4.369880in}{2.567785in}}{\pgfqpoint{4.367685in}{2.573085in}}{\pgfqpoint{4.363778in}{2.576992in}}%
\pgfpathcurveto{\pgfqpoint{4.359871in}{2.580899in}}{\pgfqpoint{4.354571in}{2.583094in}}{\pgfqpoint{4.349046in}{2.583094in}}%
\pgfpathcurveto{\pgfqpoint{4.343521in}{2.583094in}}{\pgfqpoint{4.338222in}{2.580899in}}{\pgfqpoint{4.334315in}{2.576992in}}%
\pgfpathcurveto{\pgfqpoint{4.330408in}{2.573085in}}{\pgfqpoint{4.328213in}{2.567785in}}{\pgfqpoint{4.328213in}{2.562260in}}%
\pgfpathcurveto{\pgfqpoint{4.328213in}{2.556735in}}{\pgfqpoint{4.330408in}{2.551436in}}{\pgfqpoint{4.334315in}{2.547529in}}%
\pgfpathcurveto{\pgfqpoint{4.338222in}{2.543622in}}{\pgfqpoint{4.343521in}{2.541427in}}{\pgfqpoint{4.349046in}{2.541427in}}%
\pgfpathclose%
\pgfusepath{fill}%
\end{pgfscope}%
\begin{pgfscope}%
\pgfpathrectangle{\pgfqpoint{3.214225in}{2.423832in}}{\pgfqpoint{1.162500in}{0.755000in}}%
\pgfusepath{clip}%
\pgfsetbuttcap%
\pgfsetroundjoin%
\definecolor{currentfill}{rgb}{0.000000,0.000000,0.000000}%
\pgfsetfillcolor{currentfill}%
\pgfsetfillopacity{0.500000}%
\pgfsetlinewidth{0.000000pt}%
\definecolor{currentstroke}{rgb}{0.000000,0.000000,0.000000}%
\pgfsetstrokecolor{currentstroke}%
\pgfsetdash{}{0pt}%
\pgfpathmoveto{\pgfqpoint{3.915823in}{2.469548in}}%
\pgfpathcurveto{\pgfqpoint{3.921348in}{2.469548in}}{\pgfqpoint{3.926648in}{2.471743in}}{\pgfqpoint{3.930554in}{2.475650in}}%
\pgfpathcurveto{\pgfqpoint{3.934461in}{2.479557in}}{\pgfqpoint{3.936656in}{2.484856in}}{\pgfqpoint{3.936656in}{2.490381in}}%
\pgfpathcurveto{\pgfqpoint{3.936656in}{2.495907in}}{\pgfqpoint{3.934461in}{2.501206in}}{\pgfqpoint{3.930554in}{2.505113in}}%
\pgfpathcurveto{\pgfqpoint{3.926648in}{2.509020in}}{\pgfqpoint{3.921348in}{2.511215in}}{\pgfqpoint{3.915823in}{2.511215in}}%
\pgfpathcurveto{\pgfqpoint{3.910298in}{2.511215in}}{\pgfqpoint{3.904998in}{2.509020in}}{\pgfqpoint{3.901092in}{2.505113in}}%
\pgfpathcurveto{\pgfqpoint{3.897185in}{2.501206in}}{\pgfqpoint{3.894990in}{2.495907in}}{\pgfqpoint{3.894990in}{2.490381in}}%
\pgfpathcurveto{\pgfqpoint{3.894990in}{2.484856in}}{\pgfqpoint{3.897185in}{2.479557in}}{\pgfqpoint{3.901092in}{2.475650in}}%
\pgfpathcurveto{\pgfqpoint{3.904998in}{2.471743in}}{\pgfqpoint{3.910298in}{2.469548in}}{\pgfqpoint{3.915823in}{2.469548in}}%
\pgfpathclose%
\pgfusepath{fill}%
\end{pgfscope}%
\begin{pgfscope}%
\pgfpathrectangle{\pgfqpoint{3.214225in}{2.423832in}}{\pgfqpoint{1.162500in}{0.755000in}}%
\pgfusepath{clip}%
\pgfsetbuttcap%
\pgfsetroundjoin%
\definecolor{currentfill}{rgb}{0.000000,0.000000,0.000000}%
\pgfsetfillcolor{currentfill}%
\pgfsetfillopacity{0.500000}%
\pgfsetlinewidth{0.000000pt}%
\definecolor{currentstroke}{rgb}{0.000000,0.000000,0.000000}%
\pgfsetstrokecolor{currentstroke}%
\pgfsetdash{}{0pt}%
\pgfpathmoveto{\pgfqpoint{3.986481in}{2.488481in}}%
\pgfpathcurveto{\pgfqpoint{3.992006in}{2.488481in}}{\pgfqpoint{3.997306in}{2.490676in}}{\pgfqpoint{4.001213in}{2.494583in}}%
\pgfpathcurveto{\pgfqpoint{4.005120in}{2.498489in}}{\pgfqpoint{4.007315in}{2.503789in}}{\pgfqpoint{4.007315in}{2.509314in}}%
\pgfpathcurveto{\pgfqpoint{4.007315in}{2.514839in}}{\pgfqpoint{4.005120in}{2.520139in}}{\pgfqpoint{4.001213in}{2.524045in}}%
\pgfpathcurveto{\pgfqpoint{3.997306in}{2.527952in}}{\pgfqpoint{3.992006in}{2.530147in}}{\pgfqpoint{3.986481in}{2.530147in}}%
\pgfpathcurveto{\pgfqpoint{3.980956in}{2.530147in}}{\pgfqpoint{3.975657in}{2.527952in}}{\pgfqpoint{3.971750in}{2.524045in}}%
\pgfpathcurveto{\pgfqpoint{3.967843in}{2.520139in}}{\pgfqpoint{3.965648in}{2.514839in}}{\pgfqpoint{3.965648in}{2.509314in}}%
\pgfpathcurveto{\pgfqpoint{3.965648in}{2.503789in}}{\pgfqpoint{3.967843in}{2.498489in}}{\pgfqpoint{3.971750in}{2.494583in}}%
\pgfpathcurveto{\pgfqpoint{3.975657in}{2.490676in}}{\pgfqpoint{3.980956in}{2.488481in}}{\pgfqpoint{3.986481in}{2.488481in}}%
\pgfpathclose%
\pgfusepath{fill}%
\end{pgfscope}%
\begin{pgfscope}%
\pgfpathrectangle{\pgfqpoint{3.214225in}{2.423832in}}{\pgfqpoint{1.162500in}{0.755000in}}%
\pgfusepath{clip}%
\pgfsetbuttcap%
\pgfsetroundjoin%
\definecolor{currentfill}{rgb}{0.000000,0.000000,0.000000}%
\pgfsetfillcolor{currentfill}%
\pgfsetfillopacity{0.500000}%
\pgfsetlinewidth{0.000000pt}%
\definecolor{currentstroke}{rgb}{0.000000,0.000000,0.000000}%
\pgfsetstrokecolor{currentstroke}%
\pgfsetdash{}{0pt}%
\pgfpathmoveto{\pgfqpoint{3.831974in}{2.422812in}}%
\pgfpathcurveto{\pgfqpoint{3.837499in}{2.422812in}}{\pgfqpoint{3.842798in}{2.425007in}}{\pgfqpoint{3.846705in}{2.428914in}}%
\pgfpathcurveto{\pgfqpoint{3.850612in}{2.432821in}}{\pgfqpoint{3.852807in}{2.438120in}}{\pgfqpoint{3.852807in}{2.443645in}}%
\pgfpathcurveto{\pgfqpoint{3.852807in}{2.449170in}}{\pgfqpoint{3.850612in}{2.454470in}}{\pgfqpoint{3.846705in}{2.458377in}}%
\pgfpathcurveto{\pgfqpoint{3.842798in}{2.462284in}}{\pgfqpoint{3.837499in}{2.464479in}}{\pgfqpoint{3.831974in}{2.464479in}}%
\pgfpathcurveto{\pgfqpoint{3.826449in}{2.464479in}}{\pgfqpoint{3.821149in}{2.462284in}}{\pgfqpoint{3.817242in}{2.458377in}}%
\pgfpathcurveto{\pgfqpoint{3.813336in}{2.454470in}}{\pgfqpoint{3.811140in}{2.449170in}}{\pgfqpoint{3.811140in}{2.443645in}}%
\pgfpathcurveto{\pgfqpoint{3.811140in}{2.438120in}}{\pgfqpoint{3.813336in}{2.432821in}}{\pgfqpoint{3.817242in}{2.428914in}}%
\pgfpathcurveto{\pgfqpoint{3.821149in}{2.425007in}}{\pgfqpoint{3.826449in}{2.422812in}}{\pgfqpoint{3.831974in}{2.422812in}}%
\pgfpathclose%
\pgfusepath{fill}%
\end{pgfscope}%
\begin{pgfscope}%
\pgfpathrectangle{\pgfqpoint{3.214225in}{2.423832in}}{\pgfqpoint{1.162500in}{0.755000in}}%
\pgfusepath{clip}%
\pgfsetbuttcap%
\pgfsetroundjoin%
\definecolor{currentfill}{rgb}{0.000000,0.000000,0.000000}%
\pgfsetfillcolor{currentfill}%
\pgfsetfillopacity{0.500000}%
\pgfsetlinewidth{0.000000pt}%
\definecolor{currentstroke}{rgb}{0.000000,0.000000,0.000000}%
\pgfsetstrokecolor{currentstroke}%
\pgfsetdash{}{0pt}%
\pgfpathmoveto{\pgfqpoint{3.241904in}{2.420975in}}%
\pgfpathcurveto{\pgfqpoint{3.247429in}{2.420975in}}{\pgfqpoint{3.252728in}{2.423170in}}{\pgfqpoint{3.256635in}{2.427077in}}%
\pgfpathcurveto{\pgfqpoint{3.260542in}{2.430984in}}{\pgfqpoint{3.262737in}{2.436283in}}{\pgfqpoint{3.262737in}{2.441809in}}%
\pgfpathcurveto{\pgfqpoint{3.262737in}{2.447334in}}{\pgfqpoint{3.260542in}{2.452633in}}{\pgfqpoint{3.256635in}{2.456540in}}%
\pgfpathcurveto{\pgfqpoint{3.252728in}{2.460447in}}{\pgfqpoint{3.247429in}{2.462642in}}{\pgfqpoint{3.241904in}{2.462642in}}%
\pgfpathcurveto{\pgfqpoint{3.236379in}{2.462642in}}{\pgfqpoint{3.231079in}{2.460447in}}{\pgfqpoint{3.227172in}{2.456540in}}%
\pgfpathcurveto{\pgfqpoint{3.223265in}{2.452633in}}{\pgfqpoint{3.221070in}{2.447334in}}{\pgfqpoint{3.221070in}{2.441809in}}%
\pgfpathcurveto{\pgfqpoint{3.221070in}{2.436283in}}{\pgfqpoint{3.223265in}{2.430984in}}{\pgfqpoint{3.227172in}{2.427077in}}%
\pgfpathcurveto{\pgfqpoint{3.231079in}{2.423170in}}{\pgfqpoint{3.236379in}{2.420975in}}{\pgfqpoint{3.241904in}{2.420975in}}%
\pgfpathclose%
\pgfusepath{fill}%
\end{pgfscope}%
\begin{pgfscope}%
\pgfpathrectangle{\pgfqpoint{3.214225in}{2.423832in}}{\pgfqpoint{1.162500in}{0.755000in}}%
\pgfusepath{clip}%
\pgfsetbuttcap%
\pgfsetroundjoin%
\definecolor{currentfill}{rgb}{0.000000,0.000000,0.000000}%
\pgfsetfillcolor{currentfill}%
\pgfsetfillopacity{0.500000}%
\pgfsetlinewidth{0.000000pt}%
\definecolor{currentstroke}{rgb}{0.000000,0.000000,0.000000}%
\pgfsetstrokecolor{currentstroke}%
\pgfsetdash{}{0pt}%
\pgfpathmoveto{\pgfqpoint{4.220724in}{2.717961in}}%
\pgfpathcurveto{\pgfqpoint{4.226249in}{2.717961in}}{\pgfqpoint{4.231548in}{2.720156in}}{\pgfqpoint{4.235455in}{2.724063in}}%
\pgfpathcurveto{\pgfqpoint{4.239362in}{2.727970in}}{\pgfqpoint{4.241557in}{2.733269in}}{\pgfqpoint{4.241557in}{2.738795in}}%
\pgfpathcurveto{\pgfqpoint{4.241557in}{2.744320in}}{\pgfqpoint{4.239362in}{2.749619in}}{\pgfqpoint{4.235455in}{2.753526in}}%
\pgfpathcurveto{\pgfqpoint{4.231548in}{2.757433in}}{\pgfqpoint{4.226249in}{2.759628in}}{\pgfqpoint{4.220724in}{2.759628in}}%
\pgfpathcurveto{\pgfqpoint{4.215199in}{2.759628in}}{\pgfqpoint{4.209899in}{2.757433in}}{\pgfqpoint{4.205992in}{2.753526in}}%
\pgfpathcurveto{\pgfqpoint{4.202086in}{2.749619in}}{\pgfqpoint{4.199890in}{2.744320in}}{\pgfqpoint{4.199890in}{2.738795in}}%
\pgfpathcurveto{\pgfqpoint{4.199890in}{2.733269in}}{\pgfqpoint{4.202086in}{2.727970in}}{\pgfqpoint{4.205992in}{2.724063in}}%
\pgfpathcurveto{\pgfqpoint{4.209899in}{2.720156in}}{\pgfqpoint{4.215199in}{2.717961in}}{\pgfqpoint{4.220724in}{2.717961in}}%
\pgfpathclose%
\pgfusepath{fill}%
\end{pgfscope}%
\begin{pgfscope}%
\pgfpathrectangle{\pgfqpoint{3.214225in}{2.423832in}}{\pgfqpoint{1.162500in}{0.755000in}}%
\pgfusepath{clip}%
\pgfsetbuttcap%
\pgfsetroundjoin%
\definecolor{currentfill}{rgb}{0.000000,0.000000,0.000000}%
\pgfsetfillcolor{currentfill}%
\pgfsetfillopacity{0.500000}%
\pgfsetlinewidth{0.000000pt}%
\definecolor{currentstroke}{rgb}{0.000000,0.000000,0.000000}%
\pgfsetstrokecolor{currentstroke}%
\pgfsetdash{}{0pt}%
\pgfpathmoveto{\pgfqpoint{3.980838in}{2.640479in}}%
\pgfpathcurveto{\pgfqpoint{3.986364in}{2.640479in}}{\pgfqpoint{3.991663in}{2.642674in}}{\pgfqpoint{3.995570in}{2.646581in}}%
\pgfpathcurveto{\pgfqpoint{3.999477in}{2.650487in}}{\pgfqpoint{4.001672in}{2.655787in}}{\pgfqpoint{4.001672in}{2.661312in}}%
\pgfpathcurveto{\pgfqpoint{4.001672in}{2.666837in}}{\pgfqpoint{3.999477in}{2.672137in}}{\pgfqpoint{3.995570in}{2.676043in}}%
\pgfpathcurveto{\pgfqpoint{3.991663in}{2.679950in}}{\pgfqpoint{3.986364in}{2.682145in}}{\pgfqpoint{3.980838in}{2.682145in}}%
\pgfpathcurveto{\pgfqpoint{3.975313in}{2.682145in}}{\pgfqpoint{3.970014in}{2.679950in}}{\pgfqpoint{3.966107in}{2.676043in}}%
\pgfpathcurveto{\pgfqpoint{3.962200in}{2.672137in}}{\pgfqpoint{3.960005in}{2.666837in}}{\pgfqpoint{3.960005in}{2.661312in}}%
\pgfpathcurveto{\pgfqpoint{3.960005in}{2.655787in}}{\pgfqpoint{3.962200in}{2.650487in}}{\pgfqpoint{3.966107in}{2.646581in}}%
\pgfpathcurveto{\pgfqpoint{3.970014in}{2.642674in}}{\pgfqpoint{3.975313in}{2.640479in}}{\pgfqpoint{3.980838in}{2.640479in}}%
\pgfpathclose%
\pgfusepath{fill}%
\end{pgfscope}%
\begin{pgfscope}%
\pgfpathrectangle{\pgfqpoint{3.214225in}{2.423832in}}{\pgfqpoint{1.162500in}{0.755000in}}%
\pgfusepath{clip}%
\pgfsetbuttcap%
\pgfsetroundjoin%
\definecolor{currentfill}{rgb}{0.000000,0.000000,0.000000}%
\pgfsetfillcolor{currentfill}%
\pgfsetfillopacity{0.500000}%
\pgfsetlinewidth{0.000000pt}%
\definecolor{currentstroke}{rgb}{0.000000,0.000000,0.000000}%
\pgfsetstrokecolor{currentstroke}%
\pgfsetdash{}{0pt}%
\pgfpathmoveto{\pgfqpoint{3.708643in}{2.637393in}}%
\pgfpathcurveto{\pgfqpoint{3.714168in}{2.637393in}}{\pgfqpoint{3.719468in}{2.639588in}}{\pgfqpoint{3.723375in}{2.643495in}}%
\pgfpathcurveto{\pgfqpoint{3.727282in}{2.647402in}}{\pgfqpoint{3.729477in}{2.652701in}}{\pgfqpoint{3.729477in}{2.658226in}}%
\pgfpathcurveto{\pgfqpoint{3.729477in}{2.663751in}}{\pgfqpoint{3.727282in}{2.669051in}}{\pgfqpoint{3.723375in}{2.672958in}}%
\pgfpathcurveto{\pgfqpoint{3.719468in}{2.676865in}}{\pgfqpoint{3.714168in}{2.679060in}}{\pgfqpoint{3.708643in}{2.679060in}}%
\pgfpathcurveto{\pgfqpoint{3.703118in}{2.679060in}}{\pgfqpoint{3.697819in}{2.676865in}}{\pgfqpoint{3.693912in}{2.672958in}}%
\pgfpathcurveto{\pgfqpoint{3.690005in}{2.669051in}}{\pgfqpoint{3.687810in}{2.663751in}}{\pgfqpoint{3.687810in}{2.658226in}}%
\pgfpathcurveto{\pgfqpoint{3.687810in}{2.652701in}}{\pgfqpoint{3.690005in}{2.647402in}}{\pgfqpoint{3.693912in}{2.643495in}}%
\pgfpathcurveto{\pgfqpoint{3.697819in}{2.639588in}}{\pgfqpoint{3.703118in}{2.637393in}}{\pgfqpoint{3.708643in}{2.637393in}}%
\pgfpathclose%
\pgfusepath{fill}%
\end{pgfscope}%
\begin{pgfscope}%
\pgfpathrectangle{\pgfqpoint{3.214225in}{2.423832in}}{\pgfqpoint{1.162500in}{0.755000in}}%
\pgfusepath{clip}%
\pgfsetbuttcap%
\pgfsetroundjoin%
\definecolor{currentfill}{rgb}{0.000000,0.000000,0.000000}%
\pgfsetfillcolor{currentfill}%
\pgfsetfillopacity{0.500000}%
\pgfsetlinewidth{0.000000pt}%
\definecolor{currentstroke}{rgb}{0.000000,0.000000,0.000000}%
\pgfsetstrokecolor{currentstroke}%
\pgfsetdash{}{0pt}%
\pgfpathmoveto{\pgfqpoint{4.124371in}{2.679970in}}%
\pgfpathcurveto{\pgfqpoint{4.129896in}{2.679970in}}{\pgfqpoint{4.135195in}{2.682166in}}{\pgfqpoint{4.139102in}{2.686072in}}%
\pgfpathcurveto{\pgfqpoint{4.143009in}{2.689979in}}{\pgfqpoint{4.145204in}{2.695279in}}{\pgfqpoint{4.145204in}{2.700804in}}%
\pgfpathcurveto{\pgfqpoint{4.145204in}{2.706329in}}{\pgfqpoint{4.143009in}{2.711628in}}{\pgfqpoint{4.139102in}{2.715535in}}%
\pgfpathcurveto{\pgfqpoint{4.135195in}{2.719442in}}{\pgfqpoint{4.129896in}{2.721637in}}{\pgfqpoint{4.124371in}{2.721637in}}%
\pgfpathcurveto{\pgfqpoint{4.118846in}{2.721637in}}{\pgfqpoint{4.113546in}{2.719442in}}{\pgfqpoint{4.109639in}{2.715535in}}%
\pgfpathcurveto{\pgfqpoint{4.105732in}{2.711628in}}{\pgfqpoint{4.103537in}{2.706329in}}{\pgfqpoint{4.103537in}{2.700804in}}%
\pgfpathcurveto{\pgfqpoint{4.103537in}{2.695279in}}{\pgfqpoint{4.105732in}{2.689979in}}{\pgfqpoint{4.109639in}{2.686072in}}%
\pgfpathcurveto{\pgfqpoint{4.113546in}{2.682166in}}{\pgfqpoint{4.118846in}{2.679970in}}{\pgfqpoint{4.124371in}{2.679970in}}%
\pgfpathclose%
\pgfusepath{fill}%
\end{pgfscope}%
\begin{pgfscope}%
\pgfpathrectangle{\pgfqpoint{3.214225in}{2.423832in}}{\pgfqpoint{1.162500in}{0.755000in}}%
\pgfusepath{clip}%
\pgfsetbuttcap%
\pgfsetroundjoin%
\definecolor{currentfill}{rgb}{0.000000,0.000000,0.000000}%
\pgfsetfillcolor{currentfill}%
\pgfsetfillopacity{0.500000}%
\pgfsetlinewidth{0.000000pt}%
\definecolor{currentstroke}{rgb}{0.000000,0.000000,0.000000}%
\pgfsetstrokecolor{currentstroke}%
\pgfsetdash{}{0pt}%
\pgfpathmoveto{\pgfqpoint{3.695649in}{2.481302in}}%
\pgfpathcurveto{\pgfqpoint{3.701174in}{2.481302in}}{\pgfqpoint{3.706474in}{2.483497in}}{\pgfqpoint{3.710381in}{2.487404in}}%
\pgfpathcurveto{\pgfqpoint{3.714288in}{2.491311in}}{\pgfqpoint{3.716483in}{2.496610in}}{\pgfqpoint{3.716483in}{2.502135in}}%
\pgfpathcurveto{\pgfqpoint{3.716483in}{2.507661in}}{\pgfqpoint{3.714288in}{2.512960in}}{\pgfqpoint{3.710381in}{2.516867in}}%
\pgfpathcurveto{\pgfqpoint{3.706474in}{2.520774in}}{\pgfqpoint{3.701174in}{2.522969in}}{\pgfqpoint{3.695649in}{2.522969in}}%
\pgfpathcurveto{\pgfqpoint{3.690124in}{2.522969in}}{\pgfqpoint{3.684825in}{2.520774in}}{\pgfqpoint{3.680918in}{2.516867in}}%
\pgfpathcurveto{\pgfqpoint{3.677011in}{2.512960in}}{\pgfqpoint{3.674816in}{2.507661in}}{\pgfqpoint{3.674816in}{2.502135in}}%
\pgfpathcurveto{\pgfqpoint{3.674816in}{2.496610in}}{\pgfqpoint{3.677011in}{2.491311in}}{\pgfqpoint{3.680918in}{2.487404in}}%
\pgfpathcurveto{\pgfqpoint{3.684825in}{2.483497in}}{\pgfqpoint{3.690124in}{2.481302in}}{\pgfqpoint{3.695649in}{2.481302in}}%
\pgfpathclose%
\pgfusepath{fill}%
\end{pgfscope}%
\begin{pgfscope}%
\pgfpathrectangle{\pgfqpoint{3.214225in}{2.423832in}}{\pgfqpoint{1.162500in}{0.755000in}}%
\pgfusepath{clip}%
\pgfsetbuttcap%
\pgfsetroundjoin%
\definecolor{currentfill}{rgb}{0.000000,0.000000,0.000000}%
\pgfsetfillcolor{currentfill}%
\pgfsetfillopacity{0.500000}%
\pgfsetlinewidth{0.000000pt}%
\definecolor{currentstroke}{rgb}{0.000000,0.000000,0.000000}%
\pgfsetstrokecolor{currentstroke}%
\pgfsetdash{}{0pt}%
\pgfpathmoveto{\pgfqpoint{3.329643in}{2.503012in}}%
\pgfpathcurveto{\pgfqpoint{3.335168in}{2.503012in}}{\pgfqpoint{3.340468in}{2.505207in}}{\pgfqpoint{3.344375in}{2.509114in}}%
\pgfpathcurveto{\pgfqpoint{3.348281in}{2.513021in}}{\pgfqpoint{3.350476in}{2.518320in}}{\pgfqpoint{3.350476in}{2.523845in}}%
\pgfpathcurveto{\pgfqpoint{3.350476in}{2.529370in}}{\pgfqpoint{3.348281in}{2.534670in}}{\pgfqpoint{3.344375in}{2.538577in}}%
\pgfpathcurveto{\pgfqpoint{3.340468in}{2.542483in}}{\pgfqpoint{3.335168in}{2.544679in}}{\pgfqpoint{3.329643in}{2.544679in}}%
\pgfpathcurveto{\pgfqpoint{3.324118in}{2.544679in}}{\pgfqpoint{3.318819in}{2.542483in}}{\pgfqpoint{3.314912in}{2.538577in}}%
\pgfpathcurveto{\pgfqpoint{3.311005in}{2.534670in}}{\pgfqpoint{3.308810in}{2.529370in}}{\pgfqpoint{3.308810in}{2.523845in}}%
\pgfpathcurveto{\pgfqpoint{3.308810in}{2.518320in}}{\pgfqpoint{3.311005in}{2.513021in}}{\pgfqpoint{3.314912in}{2.509114in}}%
\pgfpathcurveto{\pgfqpoint{3.318819in}{2.505207in}}{\pgfqpoint{3.324118in}{2.503012in}}{\pgfqpoint{3.329643in}{2.503012in}}%
\pgfpathclose%
\pgfusepath{fill}%
\end{pgfscope}%
\begin{pgfscope}%
\pgfpathrectangle{\pgfqpoint{3.214225in}{2.423832in}}{\pgfqpoint{1.162500in}{0.755000in}}%
\pgfusepath{clip}%
\pgfsetbuttcap%
\pgfsetroundjoin%
\definecolor{currentfill}{rgb}{0.000000,0.000000,0.000000}%
\pgfsetfillcolor{currentfill}%
\pgfsetfillopacity{0.500000}%
\pgfsetlinewidth{0.000000pt}%
\definecolor{currentstroke}{rgb}{0.000000,0.000000,0.000000}%
\pgfsetstrokecolor{currentstroke}%
\pgfsetdash{}{0pt}%
\pgfpathmoveto{\pgfqpoint{4.167862in}{3.140023in}}%
\pgfpathcurveto{\pgfqpoint{4.173387in}{3.140023in}}{\pgfqpoint{4.178687in}{3.142218in}}{\pgfqpoint{4.182593in}{3.146125in}}%
\pgfpathcurveto{\pgfqpoint{4.186500in}{3.150032in}}{\pgfqpoint{4.188695in}{3.155331in}}{\pgfqpoint{4.188695in}{3.160856in}}%
\pgfpathcurveto{\pgfqpoint{4.188695in}{3.166381in}}{\pgfqpoint{4.186500in}{3.171681in}}{\pgfqpoint{4.182593in}{3.175588in}}%
\pgfpathcurveto{\pgfqpoint{4.178687in}{3.179494in}}{\pgfqpoint{4.173387in}{3.181690in}}{\pgfqpoint{4.167862in}{3.181690in}}%
\pgfpathcurveto{\pgfqpoint{4.162337in}{3.181690in}}{\pgfqpoint{4.157037in}{3.179494in}}{\pgfqpoint{4.153131in}{3.175588in}}%
\pgfpathcurveto{\pgfqpoint{4.149224in}{3.171681in}}{\pgfqpoint{4.147029in}{3.166381in}}{\pgfqpoint{4.147029in}{3.160856in}}%
\pgfpathcurveto{\pgfqpoint{4.147029in}{3.155331in}}{\pgfqpoint{4.149224in}{3.150032in}}{\pgfqpoint{4.153131in}{3.146125in}}%
\pgfpathcurveto{\pgfqpoint{4.157037in}{3.142218in}}{\pgfqpoint{4.162337in}{3.140023in}}{\pgfqpoint{4.167862in}{3.140023in}}%
\pgfpathclose%
\pgfusepath{fill}%
\end{pgfscope}%
\begin{pgfscope}%
\pgfpathrectangle{\pgfqpoint{3.214225in}{2.423832in}}{\pgfqpoint{1.162500in}{0.755000in}}%
\pgfusepath{clip}%
\pgfsetbuttcap%
\pgfsetroundjoin%
\definecolor{currentfill}{rgb}{0.000000,0.000000,0.000000}%
\pgfsetfillcolor{currentfill}%
\pgfsetfillopacity{0.500000}%
\pgfsetlinewidth{0.000000pt}%
\definecolor{currentstroke}{rgb}{0.000000,0.000000,0.000000}%
\pgfsetstrokecolor{currentstroke}%
\pgfsetdash{}{0pt}%
\pgfpathmoveto{\pgfqpoint{4.120816in}{2.625365in}}%
\pgfpathcurveto{\pgfqpoint{4.126341in}{2.625365in}}{\pgfqpoint{4.131641in}{2.627561in}}{\pgfqpoint{4.135548in}{2.631467in}}%
\pgfpathcurveto{\pgfqpoint{4.139455in}{2.635374in}}{\pgfqpoint{4.141650in}{2.640674in}}{\pgfqpoint{4.141650in}{2.646199in}}%
\pgfpathcurveto{\pgfqpoint{4.141650in}{2.651724in}}{\pgfqpoint{4.139455in}{2.657023in}}{\pgfqpoint{4.135548in}{2.660930in}}%
\pgfpathcurveto{\pgfqpoint{4.131641in}{2.664837in}}{\pgfqpoint{4.126341in}{2.667032in}}{\pgfqpoint{4.120816in}{2.667032in}}%
\pgfpathcurveto{\pgfqpoint{4.115291in}{2.667032in}}{\pgfqpoint{4.109992in}{2.664837in}}{\pgfqpoint{4.106085in}{2.660930in}}%
\pgfpathcurveto{\pgfqpoint{4.102178in}{2.657023in}}{\pgfqpoint{4.099983in}{2.651724in}}{\pgfqpoint{4.099983in}{2.646199in}}%
\pgfpathcurveto{\pgfqpoint{4.099983in}{2.640674in}}{\pgfqpoint{4.102178in}{2.635374in}}{\pgfqpoint{4.106085in}{2.631467in}}%
\pgfpathcurveto{\pgfqpoint{4.109992in}{2.627561in}}{\pgfqpoint{4.115291in}{2.625365in}}{\pgfqpoint{4.120816in}{2.625365in}}%
\pgfpathclose%
\pgfusepath{fill}%
\end{pgfscope}%
\begin{pgfscope}%
\pgfpathrectangle{\pgfqpoint{3.214225in}{2.423832in}}{\pgfqpoint{1.162500in}{0.755000in}}%
\pgfusepath{clip}%
\pgfsetbuttcap%
\pgfsetroundjoin%
\definecolor{currentfill}{rgb}{0.000000,0.000000,0.000000}%
\pgfsetfillcolor{currentfill}%
\pgfsetfillopacity{0.500000}%
\pgfsetlinewidth{0.000000pt}%
\definecolor{currentstroke}{rgb}{0.000000,0.000000,0.000000}%
\pgfsetstrokecolor{currentstroke}%
\pgfsetdash{}{0pt}%
\pgfpathmoveto{\pgfqpoint{3.792736in}{2.450809in}}%
\pgfpathcurveto{\pgfqpoint{3.798261in}{2.450809in}}{\pgfqpoint{3.803560in}{2.453004in}}{\pgfqpoint{3.807467in}{2.456911in}}%
\pgfpathcurveto{\pgfqpoint{3.811374in}{2.460818in}}{\pgfqpoint{3.813569in}{2.466117in}}{\pgfqpoint{3.813569in}{2.471642in}}%
\pgfpathcurveto{\pgfqpoint{3.813569in}{2.477167in}}{\pgfqpoint{3.811374in}{2.482467in}}{\pgfqpoint{3.807467in}{2.486374in}}%
\pgfpathcurveto{\pgfqpoint{3.803560in}{2.490280in}}{\pgfqpoint{3.798261in}{2.492476in}}{\pgfqpoint{3.792736in}{2.492476in}}%
\pgfpathcurveto{\pgfqpoint{3.787211in}{2.492476in}}{\pgfqpoint{3.781911in}{2.490280in}}{\pgfqpoint{3.778004in}{2.486374in}}%
\pgfpathcurveto{\pgfqpoint{3.774097in}{2.482467in}}{\pgfqpoint{3.771902in}{2.477167in}}{\pgfqpoint{3.771902in}{2.471642in}}%
\pgfpathcurveto{\pgfqpoint{3.771902in}{2.466117in}}{\pgfqpoint{3.774097in}{2.460818in}}{\pgfqpoint{3.778004in}{2.456911in}}%
\pgfpathcurveto{\pgfqpoint{3.781911in}{2.453004in}}{\pgfqpoint{3.787211in}{2.450809in}}{\pgfqpoint{3.792736in}{2.450809in}}%
\pgfpathclose%
\pgfusepath{fill}%
\end{pgfscope}%
\begin{pgfscope}%
\pgfsetrectcap%
\pgfsetmiterjoin%
\pgfsetlinewidth{0.803000pt}%
\definecolor{currentstroke}{rgb}{0.501961,0.501961,0.501961}%
\pgfsetstrokecolor{currentstroke}%
\pgfsetdash{}{0pt}%
\pgfpathmoveto{\pgfqpoint{3.214225in}{2.423832in}}%
\pgfpathlineto{\pgfqpoint{3.214225in}{3.178832in}}%
\pgfusepath{stroke}%
\end{pgfscope}%
\begin{pgfscope}%
\pgfsetrectcap%
\pgfsetmiterjoin%
\pgfsetlinewidth{0.803000pt}%
\definecolor{currentstroke}{rgb}{0.501961,0.501961,0.501961}%
\pgfsetstrokecolor{currentstroke}%
\pgfsetdash{}{0pt}%
\pgfpathmoveto{\pgfqpoint{4.376725in}{2.423832in}}%
\pgfpathlineto{\pgfqpoint{4.376725in}{3.178832in}}%
\pgfusepath{stroke}%
\end{pgfscope}%
\begin{pgfscope}%
\pgfsetrectcap%
\pgfsetmiterjoin%
\pgfsetlinewidth{0.803000pt}%
\definecolor{currentstroke}{rgb}{0.501961,0.501961,0.501961}%
\pgfsetstrokecolor{currentstroke}%
\pgfsetdash{}{0pt}%
\pgfpathmoveto{\pgfqpoint{3.214225in}{2.423832in}}%
\pgfpathlineto{\pgfqpoint{4.376725in}{2.423832in}}%
\pgfusepath{stroke}%
\end{pgfscope}%
\begin{pgfscope}%
\pgfsetrectcap%
\pgfsetmiterjoin%
\pgfsetlinewidth{0.803000pt}%
\definecolor{currentstroke}{rgb}{0.501961,0.501961,0.501961}%
\pgfsetstrokecolor{currentstroke}%
\pgfsetdash{}{0pt}%
\pgfpathmoveto{\pgfqpoint{3.214225in}{3.178832in}}%
\pgfpathlineto{\pgfqpoint{4.376725in}{3.178832in}}%
\pgfusepath{stroke}%
\end{pgfscope}%
\begin{pgfscope}%
\pgfsetbuttcap%
\pgfsetmiterjoin%
\definecolor{currentfill}{rgb}{1.000000,1.000000,1.000000}%
\pgfsetfillcolor{currentfill}%
\pgfsetlinewidth{0.000000pt}%
\definecolor{currentstroke}{rgb}{0.000000,0.000000,0.000000}%
\pgfsetstrokecolor{currentstroke}%
\pgfsetstrokeopacity{0.000000}%
\pgfsetdash{}{0pt}%
\pgfpathmoveto{\pgfqpoint{4.376725in}{2.423832in}}%
\pgfpathlineto{\pgfqpoint{5.539225in}{2.423832in}}%
\pgfpathlineto{\pgfqpoint{5.539225in}{3.178832in}}%
\pgfpathlineto{\pgfqpoint{4.376725in}{3.178832in}}%
\pgfpathclose%
\pgfusepath{fill}%
\end{pgfscope}%
\begin{pgfscope}%
\pgfpathrectangle{\pgfqpoint{4.376725in}{2.423832in}}{\pgfqpoint{1.162500in}{0.755000in}}%
\pgfusepath{clip}%
\pgfsetbuttcap%
\pgfsetroundjoin%
\definecolor{currentfill}{rgb}{0.000000,0.000000,0.000000}%
\pgfsetfillcolor{currentfill}%
\pgfsetfillopacity{0.500000}%
\pgfsetlinewidth{0.000000pt}%
\definecolor{currentstroke}{rgb}{0.000000,0.000000,0.000000}%
\pgfsetstrokecolor{currentstroke}%
\pgfsetdash{}{0pt}%
\pgfpathmoveto{\pgfqpoint{5.214960in}{3.088730in}}%
\pgfpathcurveto{\pgfqpoint{5.220485in}{3.088730in}}{\pgfqpoint{5.225785in}{3.090925in}}{\pgfqpoint{5.229692in}{3.094832in}}%
\pgfpathcurveto{\pgfqpoint{5.233598in}{3.098739in}}{\pgfqpoint{5.235794in}{3.104038in}}{\pgfqpoint{5.235794in}{3.109563in}}%
\pgfpathcurveto{\pgfqpoint{5.235794in}{3.115089in}}{\pgfqpoint{5.233598in}{3.120388in}}{\pgfqpoint{5.229692in}{3.124295in}}%
\pgfpathcurveto{\pgfqpoint{5.225785in}{3.128202in}}{\pgfqpoint{5.220485in}{3.130397in}}{\pgfqpoint{5.214960in}{3.130397in}}%
\pgfpathcurveto{\pgfqpoint{5.209435in}{3.130397in}}{\pgfqpoint{5.204136in}{3.128202in}}{\pgfqpoint{5.200229in}{3.124295in}}%
\pgfpathcurveto{\pgfqpoint{5.196322in}{3.120388in}}{\pgfqpoint{5.194127in}{3.115089in}}{\pgfqpoint{5.194127in}{3.109563in}}%
\pgfpathcurveto{\pgfqpoint{5.194127in}{3.104038in}}{\pgfqpoint{5.196322in}{3.098739in}}{\pgfqpoint{5.200229in}{3.094832in}}%
\pgfpathcurveto{\pgfqpoint{5.204136in}{3.090925in}}{\pgfqpoint{5.209435in}{3.088730in}}{\pgfqpoint{5.214960in}{3.088730in}}%
\pgfpathclose%
\pgfusepath{fill}%
\end{pgfscope}%
\begin{pgfscope}%
\pgfpathrectangle{\pgfqpoint{4.376725in}{2.423832in}}{\pgfqpoint{1.162500in}{0.755000in}}%
\pgfusepath{clip}%
\pgfsetbuttcap%
\pgfsetroundjoin%
\definecolor{currentfill}{rgb}{0.000000,0.000000,0.000000}%
\pgfsetfillcolor{currentfill}%
\pgfsetfillopacity{0.500000}%
\pgfsetlinewidth{0.000000pt}%
\definecolor{currentstroke}{rgb}{0.000000,0.000000,0.000000}%
\pgfsetstrokecolor{currentstroke}%
\pgfsetdash{}{0pt}%
\pgfpathmoveto{\pgfqpoint{4.521739in}{2.544873in}}%
\pgfpathcurveto{\pgfqpoint{4.527264in}{2.544873in}}{\pgfqpoint{4.532563in}{2.547068in}}{\pgfqpoint{4.536470in}{2.550975in}}%
\pgfpathcurveto{\pgfqpoint{4.540377in}{2.554881in}}{\pgfqpoint{4.542572in}{2.560181in}}{\pgfqpoint{4.542572in}{2.565706in}}%
\pgfpathcurveto{\pgfqpoint{4.542572in}{2.571231in}}{\pgfqpoint{4.540377in}{2.576531in}}{\pgfqpoint{4.536470in}{2.580437in}}%
\pgfpathcurveto{\pgfqpoint{4.532563in}{2.584344in}}{\pgfqpoint{4.527264in}{2.586539in}}{\pgfqpoint{4.521739in}{2.586539in}}%
\pgfpathcurveto{\pgfqpoint{4.516214in}{2.586539in}}{\pgfqpoint{4.510914in}{2.584344in}}{\pgfqpoint{4.507007in}{2.580437in}}%
\pgfpathcurveto{\pgfqpoint{4.503101in}{2.576531in}}{\pgfqpoint{4.500905in}{2.571231in}}{\pgfqpoint{4.500905in}{2.565706in}}%
\pgfpathcurveto{\pgfqpoint{4.500905in}{2.560181in}}{\pgfqpoint{4.503101in}{2.554881in}}{\pgfqpoint{4.507007in}{2.550975in}}%
\pgfpathcurveto{\pgfqpoint{4.510914in}{2.547068in}}{\pgfqpoint{4.516214in}{2.544873in}}{\pgfqpoint{4.521739in}{2.544873in}}%
\pgfpathclose%
\pgfusepath{fill}%
\end{pgfscope}%
\begin{pgfscope}%
\pgfpathrectangle{\pgfqpoint{4.376725in}{2.423832in}}{\pgfqpoint{1.162500in}{0.755000in}}%
\pgfusepath{clip}%
\pgfsetbuttcap%
\pgfsetroundjoin%
\definecolor{currentfill}{rgb}{0.000000,0.000000,0.000000}%
\pgfsetfillcolor{currentfill}%
\pgfsetfillopacity{0.500000}%
\pgfsetlinewidth{0.000000pt}%
\definecolor{currentstroke}{rgb}{0.000000,0.000000,0.000000}%
\pgfsetstrokecolor{currentstroke}%
\pgfsetdash{}{0pt}%
\pgfpathmoveto{\pgfqpoint{4.448919in}{2.541427in}}%
\pgfpathcurveto{\pgfqpoint{4.454444in}{2.541427in}}{\pgfqpoint{4.459744in}{2.543622in}}{\pgfqpoint{4.463650in}{2.547529in}}%
\pgfpathcurveto{\pgfqpoint{4.467557in}{2.551436in}}{\pgfqpoint{4.469752in}{2.556735in}}{\pgfqpoint{4.469752in}{2.562260in}}%
\pgfpathcurveto{\pgfqpoint{4.469752in}{2.567785in}}{\pgfqpoint{4.467557in}{2.573085in}}{\pgfqpoint{4.463650in}{2.576992in}}%
\pgfpathcurveto{\pgfqpoint{4.459744in}{2.580899in}}{\pgfqpoint{4.454444in}{2.583094in}}{\pgfqpoint{4.448919in}{2.583094in}}%
\pgfpathcurveto{\pgfqpoint{4.443394in}{2.583094in}}{\pgfqpoint{4.438094in}{2.580899in}}{\pgfqpoint{4.434188in}{2.576992in}}%
\pgfpathcurveto{\pgfqpoint{4.430281in}{2.573085in}}{\pgfqpoint{4.428086in}{2.567785in}}{\pgfqpoint{4.428086in}{2.562260in}}%
\pgfpathcurveto{\pgfqpoint{4.428086in}{2.556735in}}{\pgfqpoint{4.430281in}{2.551436in}}{\pgfqpoint{4.434188in}{2.547529in}}%
\pgfpathcurveto{\pgfqpoint{4.438094in}{2.543622in}}{\pgfqpoint{4.443394in}{2.541427in}}{\pgfqpoint{4.448919in}{2.541427in}}%
\pgfpathclose%
\pgfusepath{fill}%
\end{pgfscope}%
\begin{pgfscope}%
\pgfpathrectangle{\pgfqpoint{4.376725in}{2.423832in}}{\pgfqpoint{1.162500in}{0.755000in}}%
\pgfusepath{clip}%
\pgfsetbuttcap%
\pgfsetroundjoin%
\definecolor{currentfill}{rgb}{0.000000,0.000000,0.000000}%
\pgfsetfillcolor{currentfill}%
\pgfsetfillopacity{0.500000}%
\pgfsetlinewidth{0.000000pt}%
\definecolor{currentstroke}{rgb}{0.000000,0.000000,0.000000}%
\pgfsetstrokecolor{currentstroke}%
\pgfsetdash{}{0pt}%
\pgfpathmoveto{\pgfqpoint{4.704251in}{2.469548in}}%
\pgfpathcurveto{\pgfqpoint{4.709776in}{2.469548in}}{\pgfqpoint{4.715075in}{2.471743in}}{\pgfqpoint{4.718982in}{2.475650in}}%
\pgfpathcurveto{\pgfqpoint{4.722889in}{2.479557in}}{\pgfqpoint{4.725084in}{2.484856in}}{\pgfqpoint{4.725084in}{2.490381in}}%
\pgfpathcurveto{\pgfqpoint{4.725084in}{2.495907in}}{\pgfqpoint{4.722889in}{2.501206in}}{\pgfqpoint{4.718982in}{2.505113in}}%
\pgfpathcurveto{\pgfqpoint{4.715075in}{2.509020in}}{\pgfqpoint{4.709776in}{2.511215in}}{\pgfqpoint{4.704251in}{2.511215in}}%
\pgfpathcurveto{\pgfqpoint{4.698726in}{2.511215in}}{\pgfqpoint{4.693426in}{2.509020in}}{\pgfqpoint{4.689519in}{2.505113in}}%
\pgfpathcurveto{\pgfqpoint{4.685612in}{2.501206in}}{\pgfqpoint{4.683417in}{2.495907in}}{\pgfqpoint{4.683417in}{2.490381in}}%
\pgfpathcurveto{\pgfqpoint{4.683417in}{2.484856in}}{\pgfqpoint{4.685612in}{2.479557in}}{\pgfqpoint{4.689519in}{2.475650in}}%
\pgfpathcurveto{\pgfqpoint{4.693426in}{2.471743in}}{\pgfqpoint{4.698726in}{2.469548in}}{\pgfqpoint{4.704251in}{2.469548in}}%
\pgfpathclose%
\pgfusepath{fill}%
\end{pgfscope}%
\begin{pgfscope}%
\pgfpathrectangle{\pgfqpoint{4.376725in}{2.423832in}}{\pgfqpoint{1.162500in}{0.755000in}}%
\pgfusepath{clip}%
\pgfsetbuttcap%
\pgfsetroundjoin%
\definecolor{currentfill}{rgb}{0.000000,0.000000,0.000000}%
\pgfsetfillcolor{currentfill}%
\pgfsetfillopacity{0.500000}%
\pgfsetlinewidth{0.000000pt}%
\definecolor{currentstroke}{rgb}{0.000000,0.000000,0.000000}%
\pgfsetstrokecolor{currentstroke}%
\pgfsetdash{}{0pt}%
\pgfpathmoveto{\pgfqpoint{4.404404in}{2.488481in}}%
\pgfpathcurveto{\pgfqpoint{4.409929in}{2.488481in}}{\pgfqpoint{4.415228in}{2.490676in}}{\pgfqpoint{4.419135in}{2.494583in}}%
\pgfpathcurveto{\pgfqpoint{4.423042in}{2.498489in}}{\pgfqpoint{4.425237in}{2.503789in}}{\pgfqpoint{4.425237in}{2.509314in}}%
\pgfpathcurveto{\pgfqpoint{4.425237in}{2.514839in}}{\pgfqpoint{4.423042in}{2.520139in}}{\pgfqpoint{4.419135in}{2.524045in}}%
\pgfpathcurveto{\pgfqpoint{4.415228in}{2.527952in}}{\pgfqpoint{4.409929in}{2.530147in}}{\pgfqpoint{4.404404in}{2.530147in}}%
\pgfpathcurveto{\pgfqpoint{4.398879in}{2.530147in}}{\pgfqpoint{4.393579in}{2.527952in}}{\pgfqpoint{4.389672in}{2.524045in}}%
\pgfpathcurveto{\pgfqpoint{4.385765in}{2.520139in}}{\pgfqpoint{4.383570in}{2.514839in}}{\pgfqpoint{4.383570in}{2.509314in}}%
\pgfpathcurveto{\pgfqpoint{4.383570in}{2.503789in}}{\pgfqpoint{4.385765in}{2.498489in}}{\pgfqpoint{4.389672in}{2.494583in}}%
\pgfpathcurveto{\pgfqpoint{4.393579in}{2.490676in}}{\pgfqpoint{4.398879in}{2.488481in}}{\pgfqpoint{4.404404in}{2.488481in}}%
\pgfpathclose%
\pgfusepath{fill}%
\end{pgfscope}%
\begin{pgfscope}%
\pgfpathrectangle{\pgfqpoint{4.376725in}{2.423832in}}{\pgfqpoint{1.162500in}{0.755000in}}%
\pgfusepath{clip}%
\pgfsetbuttcap%
\pgfsetroundjoin%
\definecolor{currentfill}{rgb}{0.000000,0.000000,0.000000}%
\pgfsetfillcolor{currentfill}%
\pgfsetfillopacity{0.500000}%
\pgfsetlinewidth{0.000000pt}%
\definecolor{currentstroke}{rgb}{0.000000,0.000000,0.000000}%
\pgfsetstrokecolor{currentstroke}%
\pgfsetdash{}{0pt}%
\pgfpathmoveto{\pgfqpoint{4.607580in}{2.422812in}}%
\pgfpathcurveto{\pgfqpoint{4.613105in}{2.422812in}}{\pgfqpoint{4.618405in}{2.425007in}}{\pgfqpoint{4.622312in}{2.428914in}}%
\pgfpathcurveto{\pgfqpoint{4.626218in}{2.432821in}}{\pgfqpoint{4.628414in}{2.438120in}}{\pgfqpoint{4.628414in}{2.443645in}}%
\pgfpathcurveto{\pgfqpoint{4.628414in}{2.449170in}}{\pgfqpoint{4.626218in}{2.454470in}}{\pgfqpoint{4.622312in}{2.458377in}}%
\pgfpathcurveto{\pgfqpoint{4.618405in}{2.462284in}}{\pgfqpoint{4.613105in}{2.464479in}}{\pgfqpoint{4.607580in}{2.464479in}}%
\pgfpathcurveto{\pgfqpoint{4.602055in}{2.464479in}}{\pgfqpoint{4.596756in}{2.462284in}}{\pgfqpoint{4.592849in}{2.458377in}}%
\pgfpathcurveto{\pgfqpoint{4.588942in}{2.454470in}}{\pgfqpoint{4.586747in}{2.449170in}}{\pgfqpoint{4.586747in}{2.443645in}}%
\pgfpathcurveto{\pgfqpoint{4.586747in}{2.438120in}}{\pgfqpoint{4.588942in}{2.432821in}}{\pgfqpoint{4.592849in}{2.428914in}}%
\pgfpathcurveto{\pgfqpoint{4.596756in}{2.425007in}}{\pgfqpoint{4.602055in}{2.422812in}}{\pgfqpoint{4.607580in}{2.422812in}}%
\pgfpathclose%
\pgfusepath{fill}%
\end{pgfscope}%
\begin{pgfscope}%
\pgfpathrectangle{\pgfqpoint{4.376725in}{2.423832in}}{\pgfqpoint{1.162500in}{0.755000in}}%
\pgfusepath{clip}%
\pgfsetbuttcap%
\pgfsetroundjoin%
\definecolor{currentfill}{rgb}{0.000000,0.000000,0.000000}%
\pgfsetfillcolor{currentfill}%
\pgfsetfillopacity{0.500000}%
\pgfsetlinewidth{0.000000pt}%
\definecolor{currentstroke}{rgb}{0.000000,0.000000,0.000000}%
\pgfsetstrokecolor{currentstroke}%
\pgfsetdash{}{0pt}%
\pgfpathmoveto{\pgfqpoint{4.596304in}{2.420975in}}%
\pgfpathcurveto{\pgfqpoint{4.601829in}{2.420975in}}{\pgfqpoint{4.607129in}{2.423170in}}{\pgfqpoint{4.611036in}{2.427077in}}%
\pgfpathcurveto{\pgfqpoint{4.614942in}{2.430984in}}{\pgfqpoint{4.617137in}{2.436283in}}{\pgfqpoint{4.617137in}{2.441809in}}%
\pgfpathcurveto{\pgfqpoint{4.617137in}{2.447334in}}{\pgfqpoint{4.614942in}{2.452633in}}{\pgfqpoint{4.611036in}{2.456540in}}%
\pgfpathcurveto{\pgfqpoint{4.607129in}{2.460447in}}{\pgfqpoint{4.601829in}{2.462642in}}{\pgfqpoint{4.596304in}{2.462642in}}%
\pgfpathcurveto{\pgfqpoint{4.590779in}{2.462642in}}{\pgfqpoint{4.585480in}{2.460447in}}{\pgfqpoint{4.581573in}{2.456540in}}%
\pgfpathcurveto{\pgfqpoint{4.577666in}{2.452633in}}{\pgfqpoint{4.575471in}{2.447334in}}{\pgfqpoint{4.575471in}{2.441809in}}%
\pgfpathcurveto{\pgfqpoint{4.575471in}{2.436283in}}{\pgfqpoint{4.577666in}{2.430984in}}{\pgfqpoint{4.581573in}{2.427077in}}%
\pgfpathcurveto{\pgfqpoint{4.585480in}{2.423170in}}{\pgfqpoint{4.590779in}{2.420975in}}{\pgfqpoint{4.596304in}{2.420975in}}%
\pgfpathclose%
\pgfusepath{fill}%
\end{pgfscope}%
\begin{pgfscope}%
\pgfpathrectangle{\pgfqpoint{4.376725in}{2.423832in}}{\pgfqpoint{1.162500in}{0.755000in}}%
\pgfusepath{clip}%
\pgfsetbuttcap%
\pgfsetroundjoin%
\definecolor{currentfill}{rgb}{0.000000,0.000000,0.000000}%
\pgfsetfillcolor{currentfill}%
\pgfsetfillopacity{0.500000}%
\pgfsetlinewidth{0.000000pt}%
\definecolor{currentstroke}{rgb}{0.000000,0.000000,0.000000}%
\pgfsetstrokecolor{currentstroke}%
\pgfsetdash{}{0pt}%
\pgfpathmoveto{\pgfqpoint{5.423123in}{2.717961in}}%
\pgfpathcurveto{\pgfqpoint{5.428649in}{2.717961in}}{\pgfqpoint{5.433948in}{2.720156in}}{\pgfqpoint{5.437855in}{2.724063in}}%
\pgfpathcurveto{\pgfqpoint{5.441762in}{2.727970in}}{\pgfqpoint{5.443957in}{2.733269in}}{\pgfqpoint{5.443957in}{2.738795in}}%
\pgfpathcurveto{\pgfqpoint{5.443957in}{2.744320in}}{\pgfqpoint{5.441762in}{2.749619in}}{\pgfqpoint{5.437855in}{2.753526in}}%
\pgfpathcurveto{\pgfqpoint{5.433948in}{2.757433in}}{\pgfqpoint{5.428649in}{2.759628in}}{\pgfqpoint{5.423123in}{2.759628in}}%
\pgfpathcurveto{\pgfqpoint{5.417598in}{2.759628in}}{\pgfqpoint{5.412299in}{2.757433in}}{\pgfqpoint{5.408392in}{2.753526in}}%
\pgfpathcurveto{\pgfqpoint{5.404485in}{2.749619in}}{\pgfqpoint{5.402290in}{2.744320in}}{\pgfqpoint{5.402290in}{2.738795in}}%
\pgfpathcurveto{\pgfqpoint{5.402290in}{2.733269in}}{\pgfqpoint{5.404485in}{2.727970in}}{\pgfqpoint{5.408392in}{2.724063in}}%
\pgfpathcurveto{\pgfqpoint{5.412299in}{2.720156in}}{\pgfqpoint{5.417598in}{2.717961in}}{\pgfqpoint{5.423123in}{2.717961in}}%
\pgfpathclose%
\pgfusepath{fill}%
\end{pgfscope}%
\begin{pgfscope}%
\pgfpathrectangle{\pgfqpoint{4.376725in}{2.423832in}}{\pgfqpoint{1.162500in}{0.755000in}}%
\pgfusepath{clip}%
\pgfsetbuttcap%
\pgfsetroundjoin%
\definecolor{currentfill}{rgb}{0.000000,0.000000,0.000000}%
\pgfsetfillcolor{currentfill}%
\pgfsetfillopacity{0.500000}%
\pgfsetlinewidth{0.000000pt}%
\definecolor{currentstroke}{rgb}{0.000000,0.000000,0.000000}%
\pgfsetstrokecolor{currentstroke}%
\pgfsetdash{}{0pt}%
\pgfpathmoveto{\pgfqpoint{5.511546in}{2.640479in}}%
\pgfpathcurveto{\pgfqpoint{5.517071in}{2.640479in}}{\pgfqpoint{5.522371in}{2.642674in}}{\pgfqpoint{5.526278in}{2.646581in}}%
\pgfpathcurveto{\pgfqpoint{5.530185in}{2.650487in}}{\pgfqpoint{5.532380in}{2.655787in}}{\pgfqpoint{5.532380in}{2.661312in}}%
\pgfpathcurveto{\pgfqpoint{5.532380in}{2.666837in}}{\pgfqpoint{5.530185in}{2.672137in}}{\pgfqpoint{5.526278in}{2.676043in}}%
\pgfpathcurveto{\pgfqpoint{5.522371in}{2.679950in}}{\pgfqpoint{5.517071in}{2.682145in}}{\pgfqpoint{5.511546in}{2.682145in}}%
\pgfpathcurveto{\pgfqpoint{5.506021in}{2.682145in}}{\pgfqpoint{5.500722in}{2.679950in}}{\pgfqpoint{5.496815in}{2.676043in}}%
\pgfpathcurveto{\pgfqpoint{5.492908in}{2.672137in}}{\pgfqpoint{5.490713in}{2.666837in}}{\pgfqpoint{5.490713in}{2.661312in}}%
\pgfpathcurveto{\pgfqpoint{5.490713in}{2.655787in}}{\pgfqpoint{5.492908in}{2.650487in}}{\pgfqpoint{5.496815in}{2.646581in}}%
\pgfpathcurveto{\pgfqpoint{5.500722in}{2.642674in}}{\pgfqpoint{5.506021in}{2.640479in}}{\pgfqpoint{5.511546in}{2.640479in}}%
\pgfpathclose%
\pgfusepath{fill}%
\end{pgfscope}%
\begin{pgfscope}%
\pgfpathrectangle{\pgfqpoint{4.376725in}{2.423832in}}{\pgfqpoint{1.162500in}{0.755000in}}%
\pgfusepath{clip}%
\pgfsetbuttcap%
\pgfsetroundjoin%
\definecolor{currentfill}{rgb}{0.000000,0.000000,0.000000}%
\pgfsetfillcolor{currentfill}%
\pgfsetfillopacity{0.500000}%
\pgfsetlinewidth{0.000000pt}%
\definecolor{currentstroke}{rgb}{0.000000,0.000000,0.000000}%
\pgfsetstrokecolor{currentstroke}%
\pgfsetdash{}{0pt}%
\pgfpathmoveto{\pgfqpoint{5.286995in}{2.637393in}}%
\pgfpathcurveto{\pgfqpoint{5.292520in}{2.637393in}}{\pgfqpoint{5.297820in}{2.639588in}}{\pgfqpoint{5.301726in}{2.643495in}}%
\pgfpathcurveto{\pgfqpoint{5.305633in}{2.647402in}}{\pgfqpoint{5.307828in}{2.652701in}}{\pgfqpoint{5.307828in}{2.658226in}}%
\pgfpathcurveto{\pgfqpoint{5.307828in}{2.663751in}}{\pgfqpoint{5.305633in}{2.669051in}}{\pgfqpoint{5.301726in}{2.672958in}}%
\pgfpathcurveto{\pgfqpoint{5.297820in}{2.676865in}}{\pgfqpoint{5.292520in}{2.679060in}}{\pgfqpoint{5.286995in}{2.679060in}}%
\pgfpathcurveto{\pgfqpoint{5.281470in}{2.679060in}}{\pgfqpoint{5.276170in}{2.676865in}}{\pgfqpoint{5.272264in}{2.672958in}}%
\pgfpathcurveto{\pgfqpoint{5.268357in}{2.669051in}}{\pgfqpoint{5.266162in}{2.663751in}}{\pgfqpoint{5.266162in}{2.658226in}}%
\pgfpathcurveto{\pgfqpoint{5.266162in}{2.652701in}}{\pgfqpoint{5.268357in}{2.647402in}}{\pgfqpoint{5.272264in}{2.643495in}}%
\pgfpathcurveto{\pgfqpoint{5.276170in}{2.639588in}}{\pgfqpoint{5.281470in}{2.637393in}}{\pgfqpoint{5.286995in}{2.637393in}}%
\pgfpathclose%
\pgfusepath{fill}%
\end{pgfscope}%
\begin{pgfscope}%
\pgfpathrectangle{\pgfqpoint{4.376725in}{2.423832in}}{\pgfqpoint{1.162500in}{0.755000in}}%
\pgfusepath{clip}%
\pgfsetbuttcap%
\pgfsetroundjoin%
\definecolor{currentfill}{rgb}{0.000000,0.000000,0.000000}%
\pgfsetfillcolor{currentfill}%
\pgfsetfillopacity{0.500000}%
\pgfsetlinewidth{0.000000pt}%
\definecolor{currentstroke}{rgb}{0.000000,0.000000,0.000000}%
\pgfsetstrokecolor{currentstroke}%
\pgfsetdash{}{0pt}%
\pgfpathmoveto{\pgfqpoint{4.645822in}{2.679970in}}%
\pgfpathcurveto{\pgfqpoint{4.651348in}{2.679970in}}{\pgfqpoint{4.656647in}{2.682166in}}{\pgfqpoint{4.660554in}{2.686072in}}%
\pgfpathcurveto{\pgfqpoint{4.664461in}{2.689979in}}{\pgfqpoint{4.666656in}{2.695279in}}{\pgfqpoint{4.666656in}{2.700804in}}%
\pgfpathcurveto{\pgfqpoint{4.666656in}{2.706329in}}{\pgfqpoint{4.664461in}{2.711628in}}{\pgfqpoint{4.660554in}{2.715535in}}%
\pgfpathcurveto{\pgfqpoint{4.656647in}{2.719442in}}{\pgfqpoint{4.651348in}{2.721637in}}{\pgfqpoint{4.645822in}{2.721637in}}%
\pgfpathcurveto{\pgfqpoint{4.640297in}{2.721637in}}{\pgfqpoint{4.634998in}{2.719442in}}{\pgfqpoint{4.631091in}{2.715535in}}%
\pgfpathcurveto{\pgfqpoint{4.627184in}{2.711628in}}{\pgfqpoint{4.624989in}{2.706329in}}{\pgfqpoint{4.624989in}{2.700804in}}%
\pgfpathcurveto{\pgfqpoint{4.624989in}{2.695279in}}{\pgfqpoint{4.627184in}{2.689979in}}{\pgfqpoint{4.631091in}{2.686072in}}%
\pgfpathcurveto{\pgfqpoint{4.634998in}{2.682166in}}{\pgfqpoint{4.640297in}{2.679970in}}{\pgfqpoint{4.645822in}{2.679970in}}%
\pgfpathclose%
\pgfusepath{fill}%
\end{pgfscope}%
\begin{pgfscope}%
\pgfpathrectangle{\pgfqpoint{4.376725in}{2.423832in}}{\pgfqpoint{1.162500in}{0.755000in}}%
\pgfusepath{clip}%
\pgfsetbuttcap%
\pgfsetroundjoin%
\definecolor{currentfill}{rgb}{0.000000,0.000000,0.000000}%
\pgfsetfillcolor{currentfill}%
\pgfsetfillopacity{0.500000}%
\pgfsetlinewidth{0.000000pt}%
\definecolor{currentstroke}{rgb}{0.000000,0.000000,0.000000}%
\pgfsetstrokecolor{currentstroke}%
\pgfsetdash{}{0pt}%
\pgfpathmoveto{\pgfqpoint{5.069839in}{2.481302in}}%
\pgfpathcurveto{\pgfqpoint{5.075364in}{2.481302in}}{\pgfqpoint{5.080663in}{2.483497in}}{\pgfqpoint{5.084570in}{2.487404in}}%
\pgfpathcurveto{\pgfqpoint{5.088477in}{2.491311in}}{\pgfqpoint{5.090672in}{2.496610in}}{\pgfqpoint{5.090672in}{2.502135in}}%
\pgfpathcurveto{\pgfqpoint{5.090672in}{2.507661in}}{\pgfqpoint{5.088477in}{2.512960in}}{\pgfqpoint{5.084570in}{2.516867in}}%
\pgfpathcurveto{\pgfqpoint{5.080663in}{2.520774in}}{\pgfqpoint{5.075364in}{2.522969in}}{\pgfqpoint{5.069839in}{2.522969in}}%
\pgfpathcurveto{\pgfqpoint{5.064313in}{2.522969in}}{\pgfqpoint{5.059014in}{2.520774in}}{\pgfqpoint{5.055107in}{2.516867in}}%
\pgfpathcurveto{\pgfqpoint{5.051200in}{2.512960in}}{\pgfqpoint{5.049005in}{2.507661in}}{\pgfqpoint{5.049005in}{2.502135in}}%
\pgfpathcurveto{\pgfqpoint{5.049005in}{2.496610in}}{\pgfqpoint{5.051200in}{2.491311in}}{\pgfqpoint{5.055107in}{2.487404in}}%
\pgfpathcurveto{\pgfqpoint{5.059014in}{2.483497in}}{\pgfqpoint{5.064313in}{2.481302in}}{\pgfqpoint{5.069839in}{2.481302in}}%
\pgfpathclose%
\pgfusepath{fill}%
\end{pgfscope}%
\begin{pgfscope}%
\pgfpathrectangle{\pgfqpoint{4.376725in}{2.423832in}}{\pgfqpoint{1.162500in}{0.755000in}}%
\pgfusepath{clip}%
\pgfsetbuttcap%
\pgfsetroundjoin%
\definecolor{currentfill}{rgb}{0.000000,0.000000,0.000000}%
\pgfsetfillcolor{currentfill}%
\pgfsetfillopacity{0.500000}%
\pgfsetlinewidth{0.000000pt}%
\definecolor{currentstroke}{rgb}{0.000000,0.000000,0.000000}%
\pgfsetstrokecolor{currentstroke}%
\pgfsetdash{}{0pt}%
\pgfpathmoveto{\pgfqpoint{4.941375in}{2.503012in}}%
\pgfpathcurveto{\pgfqpoint{4.946900in}{2.503012in}}{\pgfqpoint{4.952199in}{2.505207in}}{\pgfqpoint{4.956106in}{2.509114in}}%
\pgfpathcurveto{\pgfqpoint{4.960013in}{2.513021in}}{\pgfqpoint{4.962208in}{2.518320in}}{\pgfqpoint{4.962208in}{2.523845in}}%
\pgfpathcurveto{\pgfqpoint{4.962208in}{2.529370in}}{\pgfqpoint{4.960013in}{2.534670in}}{\pgfqpoint{4.956106in}{2.538577in}}%
\pgfpathcurveto{\pgfqpoint{4.952199in}{2.542483in}}{\pgfqpoint{4.946900in}{2.544679in}}{\pgfqpoint{4.941375in}{2.544679in}}%
\pgfpathcurveto{\pgfqpoint{4.935850in}{2.544679in}}{\pgfqpoint{4.930550in}{2.542483in}}{\pgfqpoint{4.926643in}{2.538577in}}%
\pgfpathcurveto{\pgfqpoint{4.922736in}{2.534670in}}{\pgfqpoint{4.920541in}{2.529370in}}{\pgfqpoint{4.920541in}{2.523845in}}%
\pgfpathcurveto{\pgfqpoint{4.920541in}{2.518320in}}{\pgfqpoint{4.922736in}{2.513021in}}{\pgfqpoint{4.926643in}{2.509114in}}%
\pgfpathcurveto{\pgfqpoint{4.930550in}{2.505207in}}{\pgfqpoint{4.935850in}{2.503012in}}{\pgfqpoint{4.941375in}{2.503012in}}%
\pgfpathclose%
\pgfusepath{fill}%
\end{pgfscope}%
\begin{pgfscope}%
\pgfpathrectangle{\pgfqpoint{4.376725in}{2.423832in}}{\pgfqpoint{1.162500in}{0.755000in}}%
\pgfusepath{clip}%
\pgfsetbuttcap%
\pgfsetroundjoin%
\definecolor{currentfill}{rgb}{0.000000,0.000000,0.000000}%
\pgfsetfillcolor{currentfill}%
\pgfsetfillopacity{0.500000}%
\pgfsetlinewidth{0.000000pt}%
\definecolor{currentstroke}{rgb}{0.000000,0.000000,0.000000}%
\pgfsetstrokecolor{currentstroke}%
\pgfsetdash{}{0pt}%
\pgfpathmoveto{\pgfqpoint{4.799153in}{3.140023in}}%
\pgfpathcurveto{\pgfqpoint{4.804678in}{3.140023in}}{\pgfqpoint{4.809977in}{3.142218in}}{\pgfqpoint{4.813884in}{3.146125in}}%
\pgfpathcurveto{\pgfqpoint{4.817791in}{3.150032in}}{\pgfqpoint{4.819986in}{3.155331in}}{\pgfqpoint{4.819986in}{3.160856in}}%
\pgfpathcurveto{\pgfqpoint{4.819986in}{3.166381in}}{\pgfqpoint{4.817791in}{3.171681in}}{\pgfqpoint{4.813884in}{3.175588in}}%
\pgfpathcurveto{\pgfqpoint{4.809977in}{3.179494in}}{\pgfqpoint{4.804678in}{3.181690in}}{\pgfqpoint{4.799153in}{3.181690in}}%
\pgfpathcurveto{\pgfqpoint{4.793628in}{3.181690in}}{\pgfqpoint{4.788328in}{3.179494in}}{\pgfqpoint{4.784421in}{3.175588in}}%
\pgfpathcurveto{\pgfqpoint{4.780515in}{3.171681in}}{\pgfqpoint{4.778320in}{3.166381in}}{\pgfqpoint{4.778320in}{3.160856in}}%
\pgfpathcurveto{\pgfqpoint{4.778320in}{3.155331in}}{\pgfqpoint{4.780515in}{3.150032in}}{\pgfqpoint{4.784421in}{3.146125in}}%
\pgfpathcurveto{\pgfqpoint{4.788328in}{3.142218in}}{\pgfqpoint{4.793628in}{3.140023in}}{\pgfqpoint{4.799153in}{3.140023in}}%
\pgfpathclose%
\pgfusepath{fill}%
\end{pgfscope}%
\begin{pgfscope}%
\pgfpathrectangle{\pgfqpoint{4.376725in}{2.423832in}}{\pgfqpoint{1.162500in}{0.755000in}}%
\pgfusepath{clip}%
\pgfsetbuttcap%
\pgfsetroundjoin%
\definecolor{currentfill}{rgb}{0.000000,0.000000,0.000000}%
\pgfsetfillcolor{currentfill}%
\pgfsetfillopacity{0.500000}%
\pgfsetlinewidth{0.000000pt}%
\definecolor{currentstroke}{rgb}{0.000000,0.000000,0.000000}%
\pgfsetstrokecolor{currentstroke}%
\pgfsetdash{}{0pt}%
\pgfpathmoveto{\pgfqpoint{4.595218in}{2.625365in}}%
\pgfpathcurveto{\pgfqpoint{4.600743in}{2.625365in}}{\pgfqpoint{4.606043in}{2.627561in}}{\pgfqpoint{4.609950in}{2.631467in}}%
\pgfpathcurveto{\pgfqpoint{4.613857in}{2.635374in}}{\pgfqpoint{4.616052in}{2.640674in}}{\pgfqpoint{4.616052in}{2.646199in}}%
\pgfpathcurveto{\pgfqpoint{4.616052in}{2.651724in}}{\pgfqpoint{4.613857in}{2.657023in}}{\pgfqpoint{4.609950in}{2.660930in}}%
\pgfpathcurveto{\pgfqpoint{4.606043in}{2.664837in}}{\pgfqpoint{4.600743in}{2.667032in}}{\pgfqpoint{4.595218in}{2.667032in}}%
\pgfpathcurveto{\pgfqpoint{4.589693in}{2.667032in}}{\pgfqpoint{4.584394in}{2.664837in}}{\pgfqpoint{4.580487in}{2.660930in}}%
\pgfpathcurveto{\pgfqpoint{4.576580in}{2.657023in}}{\pgfqpoint{4.574385in}{2.651724in}}{\pgfqpoint{4.574385in}{2.646199in}}%
\pgfpathcurveto{\pgfqpoint{4.574385in}{2.640674in}}{\pgfqpoint{4.576580in}{2.635374in}}{\pgfqpoint{4.580487in}{2.631467in}}%
\pgfpathcurveto{\pgfqpoint{4.584394in}{2.627561in}}{\pgfqpoint{4.589693in}{2.625365in}}{\pgfqpoint{4.595218in}{2.625365in}}%
\pgfpathclose%
\pgfusepath{fill}%
\end{pgfscope}%
\begin{pgfscope}%
\pgfpathrectangle{\pgfqpoint{4.376725in}{2.423832in}}{\pgfqpoint{1.162500in}{0.755000in}}%
\pgfusepath{clip}%
\pgfsetbuttcap%
\pgfsetroundjoin%
\definecolor{currentfill}{rgb}{0.000000,0.000000,0.000000}%
\pgfsetfillcolor{currentfill}%
\pgfsetfillopacity{0.500000}%
\pgfsetlinewidth{0.000000pt}%
\definecolor{currentstroke}{rgb}{0.000000,0.000000,0.000000}%
\pgfsetstrokecolor{currentstroke}%
\pgfsetdash{}{0pt}%
\pgfpathmoveto{\pgfqpoint{4.834270in}{2.450809in}}%
\pgfpathcurveto{\pgfqpoint{4.839795in}{2.450809in}}{\pgfqpoint{4.845094in}{2.453004in}}{\pgfqpoint{4.849001in}{2.456911in}}%
\pgfpathcurveto{\pgfqpoint{4.852908in}{2.460818in}}{\pgfqpoint{4.855103in}{2.466117in}}{\pgfqpoint{4.855103in}{2.471642in}}%
\pgfpathcurveto{\pgfqpoint{4.855103in}{2.477167in}}{\pgfqpoint{4.852908in}{2.482467in}}{\pgfqpoint{4.849001in}{2.486374in}}%
\pgfpathcurveto{\pgfqpoint{4.845094in}{2.490280in}}{\pgfqpoint{4.839795in}{2.492476in}}{\pgfqpoint{4.834270in}{2.492476in}}%
\pgfpathcurveto{\pgfqpoint{4.828745in}{2.492476in}}{\pgfqpoint{4.823445in}{2.490280in}}{\pgfqpoint{4.819538in}{2.486374in}}%
\pgfpathcurveto{\pgfqpoint{4.815632in}{2.482467in}}{\pgfqpoint{4.813437in}{2.477167in}}{\pgfqpoint{4.813437in}{2.471642in}}%
\pgfpathcurveto{\pgfqpoint{4.813437in}{2.466117in}}{\pgfqpoint{4.815632in}{2.460818in}}{\pgfqpoint{4.819538in}{2.456911in}}%
\pgfpathcurveto{\pgfqpoint{4.823445in}{2.453004in}}{\pgfqpoint{4.828745in}{2.450809in}}{\pgfqpoint{4.834270in}{2.450809in}}%
\pgfpathclose%
\pgfusepath{fill}%
\end{pgfscope}%
\begin{pgfscope}%
\pgfsetrectcap%
\pgfsetmiterjoin%
\pgfsetlinewidth{0.803000pt}%
\definecolor{currentstroke}{rgb}{0.501961,0.501961,0.501961}%
\pgfsetstrokecolor{currentstroke}%
\pgfsetdash{}{0pt}%
\pgfpathmoveto{\pgfqpoint{4.376725in}{2.423832in}}%
\pgfpathlineto{\pgfqpoint{4.376725in}{3.178832in}}%
\pgfusepath{stroke}%
\end{pgfscope}%
\begin{pgfscope}%
\pgfsetrectcap%
\pgfsetmiterjoin%
\pgfsetlinewidth{0.803000pt}%
\definecolor{currentstroke}{rgb}{0.501961,0.501961,0.501961}%
\pgfsetstrokecolor{currentstroke}%
\pgfsetdash{}{0pt}%
\pgfpathmoveto{\pgfqpoint{5.539225in}{2.423832in}}%
\pgfpathlineto{\pgfqpoint{5.539225in}{3.178832in}}%
\pgfusepath{stroke}%
\end{pgfscope}%
\begin{pgfscope}%
\pgfsetrectcap%
\pgfsetmiterjoin%
\pgfsetlinewidth{0.803000pt}%
\definecolor{currentstroke}{rgb}{0.501961,0.501961,0.501961}%
\pgfsetstrokecolor{currentstroke}%
\pgfsetdash{}{0pt}%
\pgfpathmoveto{\pgfqpoint{4.376725in}{2.423832in}}%
\pgfpathlineto{\pgfqpoint{5.539225in}{2.423832in}}%
\pgfusepath{stroke}%
\end{pgfscope}%
\begin{pgfscope}%
\pgfsetrectcap%
\pgfsetmiterjoin%
\pgfsetlinewidth{0.803000pt}%
\definecolor{currentstroke}{rgb}{0.501961,0.501961,0.501961}%
\pgfsetstrokecolor{currentstroke}%
\pgfsetdash{}{0pt}%
\pgfpathmoveto{\pgfqpoint{4.376725in}{3.178832in}}%
\pgfpathlineto{\pgfqpoint{5.539225in}{3.178832in}}%
\pgfusepath{stroke}%
\end{pgfscope}%
\begin{pgfscope}%
\pgfsetbuttcap%
\pgfsetmiterjoin%
\definecolor{currentfill}{rgb}{1.000000,1.000000,1.000000}%
\pgfsetfillcolor{currentfill}%
\pgfsetlinewidth{0.000000pt}%
\definecolor{currentstroke}{rgb}{0.000000,0.000000,0.000000}%
\pgfsetstrokecolor{currentstroke}%
\pgfsetstrokeopacity{0.000000}%
\pgfsetdash{}{0pt}%
\pgfpathmoveto{\pgfqpoint{0.889225in}{1.668832in}}%
\pgfpathlineto{\pgfqpoint{2.051725in}{1.668832in}}%
\pgfpathlineto{\pgfqpoint{2.051725in}{2.423832in}}%
\pgfpathlineto{\pgfqpoint{0.889225in}{2.423832in}}%
\pgfpathclose%
\pgfusepath{fill}%
\end{pgfscope}%
\begin{pgfscope}%
\pgfpathrectangle{\pgfqpoint{0.889225in}{1.668832in}}{\pgfqpoint{1.162500in}{0.755000in}}%
\pgfusepath{clip}%
\pgfsetbuttcap%
\pgfsetroundjoin%
\definecolor{currentfill}{rgb}{0.000000,0.000000,0.000000}%
\pgfsetfillcolor{currentfill}%
\pgfsetfillopacity{0.500000}%
\pgfsetlinewidth{0.000000pt}%
\definecolor{currentstroke}{rgb}{0.000000,0.000000,0.000000}%
\pgfsetstrokecolor{currentstroke}%
\pgfsetdash{}{0pt}%
\pgfpathmoveto{\pgfqpoint{1.893746in}{1.935152in}}%
\pgfpathcurveto{\pgfqpoint{1.899271in}{1.935152in}}{\pgfqpoint{1.904570in}{1.937347in}}{\pgfqpoint{1.908477in}{1.941254in}}%
\pgfpathcurveto{\pgfqpoint{1.912384in}{1.945161in}}{\pgfqpoint{1.914579in}{1.950460in}}{\pgfqpoint{1.914579in}{1.955985in}}%
\pgfpathcurveto{\pgfqpoint{1.914579in}{1.961510in}}{\pgfqpoint{1.912384in}{1.966810in}}{\pgfqpoint{1.908477in}{1.970717in}}%
\pgfpathcurveto{\pgfqpoint{1.904570in}{1.974623in}}{\pgfqpoint{1.899271in}{1.976819in}}{\pgfqpoint{1.893746in}{1.976819in}}%
\pgfpathcurveto{\pgfqpoint{1.888221in}{1.976819in}}{\pgfqpoint{1.882921in}{1.974623in}}{\pgfqpoint{1.879015in}{1.970717in}}%
\pgfpathcurveto{\pgfqpoint{1.875108in}{1.966810in}}{\pgfqpoint{1.872913in}{1.961510in}}{\pgfqpoint{1.872913in}{1.955985in}}%
\pgfpathcurveto{\pgfqpoint{1.872913in}{1.950460in}}{\pgfqpoint{1.875108in}{1.945161in}}{\pgfqpoint{1.879015in}{1.941254in}}%
\pgfpathcurveto{\pgfqpoint{1.882921in}{1.937347in}}{\pgfqpoint{1.888221in}{1.935152in}}{\pgfqpoint{1.893746in}{1.935152in}}%
\pgfpathclose%
\pgfusepath{fill}%
\end{pgfscope}%
\begin{pgfscope}%
\pgfpathrectangle{\pgfqpoint{0.889225in}{1.668832in}}{\pgfqpoint{1.162500in}{0.755000in}}%
\pgfusepath{clip}%
\pgfsetbuttcap%
\pgfsetroundjoin%
\definecolor{currentfill}{rgb}{0.000000,0.000000,0.000000}%
\pgfsetfillcolor{currentfill}%
\pgfsetfillopacity{0.500000}%
\pgfsetlinewidth{0.000000pt}%
\definecolor{currentstroke}{rgb}{0.000000,0.000000,0.000000}%
\pgfsetstrokecolor{currentstroke}%
\pgfsetdash{}{0pt}%
\pgfpathmoveto{\pgfqpoint{1.476892in}{2.326950in}}%
\pgfpathcurveto{\pgfqpoint{1.482417in}{2.326950in}}{\pgfqpoint{1.487717in}{2.329145in}}{\pgfqpoint{1.491623in}{2.333052in}}%
\pgfpathcurveto{\pgfqpoint{1.495530in}{2.336959in}}{\pgfqpoint{1.497725in}{2.342258in}}{\pgfqpoint{1.497725in}{2.347783in}}%
\pgfpathcurveto{\pgfqpoint{1.497725in}{2.353308in}}{\pgfqpoint{1.495530in}{2.358608in}}{\pgfqpoint{1.491623in}{2.362515in}}%
\pgfpathcurveto{\pgfqpoint{1.487717in}{2.366421in}}{\pgfqpoint{1.482417in}{2.368616in}}{\pgfqpoint{1.476892in}{2.368616in}}%
\pgfpathcurveto{\pgfqpoint{1.471367in}{2.368616in}}{\pgfqpoint{1.466067in}{2.366421in}}{\pgfqpoint{1.462161in}{2.362515in}}%
\pgfpathcurveto{\pgfqpoint{1.458254in}{2.358608in}}{\pgfqpoint{1.456059in}{2.353308in}}{\pgfqpoint{1.456059in}{2.347783in}}%
\pgfpathcurveto{\pgfqpoint{1.456059in}{2.342258in}}{\pgfqpoint{1.458254in}{2.336959in}}{\pgfqpoint{1.462161in}{2.333052in}}%
\pgfpathcurveto{\pgfqpoint{1.466067in}{2.329145in}}{\pgfqpoint{1.471367in}{2.326950in}}{\pgfqpoint{1.476892in}{2.326950in}}%
\pgfpathclose%
\pgfusepath{fill}%
\end{pgfscope}%
\begin{pgfscope}%
\pgfpathrectangle{\pgfqpoint{0.889225in}{1.668832in}}{\pgfqpoint{1.162500in}{0.755000in}}%
\pgfusepath{clip}%
\pgfsetbuttcap%
\pgfsetroundjoin%
\definecolor{currentfill}{rgb}{0.000000,0.000000,0.000000}%
\pgfsetfillcolor{currentfill}%
\pgfsetfillopacity{0.500000}%
\pgfsetlinewidth{0.000000pt}%
\definecolor{currentstroke}{rgb}{0.000000,0.000000,0.000000}%
\pgfsetstrokecolor{currentstroke}%
\pgfsetdash{}{0pt}%
\pgfpathmoveto{\pgfqpoint{1.478974in}{2.385023in}}%
\pgfpathcurveto{\pgfqpoint{1.484499in}{2.385023in}}{\pgfqpoint{1.489799in}{2.387218in}}{\pgfqpoint{1.493705in}{2.391125in}}%
\pgfpathcurveto{\pgfqpoint{1.497612in}{2.395032in}}{\pgfqpoint{1.499807in}{2.400331in}}{\pgfqpoint{1.499807in}{2.405856in}}%
\pgfpathcurveto{\pgfqpoint{1.499807in}{2.411381in}}{\pgfqpoint{1.497612in}{2.416681in}}{\pgfqpoint{1.493705in}{2.420588in}}%
\pgfpathcurveto{\pgfqpoint{1.489799in}{2.424494in}}{\pgfqpoint{1.484499in}{2.426690in}}{\pgfqpoint{1.478974in}{2.426690in}}%
\pgfpathcurveto{\pgfqpoint{1.473449in}{2.426690in}}{\pgfqpoint{1.468149in}{2.424494in}}{\pgfqpoint{1.464243in}{2.420588in}}%
\pgfpathcurveto{\pgfqpoint{1.460336in}{2.416681in}}{\pgfqpoint{1.458141in}{2.411381in}}{\pgfqpoint{1.458141in}{2.405856in}}%
\pgfpathcurveto{\pgfqpoint{1.458141in}{2.400331in}}{\pgfqpoint{1.460336in}{2.395032in}}{\pgfqpoint{1.464243in}{2.391125in}}%
\pgfpathcurveto{\pgfqpoint{1.468149in}{2.387218in}}{\pgfqpoint{1.473449in}{2.385023in}}{\pgfqpoint{1.478974in}{2.385023in}}%
\pgfpathclose%
\pgfusepath{fill}%
\end{pgfscope}%
\begin{pgfscope}%
\pgfpathrectangle{\pgfqpoint{0.889225in}{1.668832in}}{\pgfqpoint{1.162500in}{0.755000in}}%
\pgfusepath{clip}%
\pgfsetbuttcap%
\pgfsetroundjoin%
\definecolor{currentfill}{rgb}{0.000000,0.000000,0.000000}%
\pgfsetfillcolor{currentfill}%
\pgfsetfillopacity{0.500000}%
\pgfsetlinewidth{0.000000pt}%
\definecolor{currentstroke}{rgb}{0.000000,0.000000,0.000000}%
\pgfsetstrokecolor{currentstroke}%
\pgfsetdash{}{0pt}%
\pgfpathmoveto{\pgfqpoint{1.120714in}{2.103661in}}%
\pgfpathcurveto{\pgfqpoint{1.126239in}{2.103661in}}{\pgfqpoint{1.131538in}{2.105856in}}{\pgfqpoint{1.135445in}{2.109762in}}%
\pgfpathcurveto{\pgfqpoint{1.139352in}{2.113669in}}{\pgfqpoint{1.141547in}{2.118969in}}{\pgfqpoint{1.141547in}{2.124494in}}%
\pgfpathcurveto{\pgfqpoint{1.141547in}{2.130019in}}{\pgfqpoint{1.139352in}{2.135318in}}{\pgfqpoint{1.135445in}{2.139225in}}%
\pgfpathcurveto{\pgfqpoint{1.131538in}{2.143132in}}{\pgfqpoint{1.126239in}{2.145327in}}{\pgfqpoint{1.120714in}{2.145327in}}%
\pgfpathcurveto{\pgfqpoint{1.115189in}{2.145327in}}{\pgfqpoint{1.109889in}{2.143132in}}{\pgfqpoint{1.105982in}{2.139225in}}%
\pgfpathcurveto{\pgfqpoint{1.102076in}{2.135318in}}{\pgfqpoint{1.099881in}{2.130019in}}{\pgfqpoint{1.099881in}{2.124494in}}%
\pgfpathcurveto{\pgfqpoint{1.099881in}{2.118969in}}{\pgfqpoint{1.102076in}{2.113669in}}{\pgfqpoint{1.105982in}{2.109762in}}%
\pgfpathcurveto{\pgfqpoint{1.109889in}{2.105856in}}{\pgfqpoint{1.115189in}{2.103661in}}{\pgfqpoint{1.120714in}{2.103661in}}%
\pgfpathclose%
\pgfusepath{fill}%
\end{pgfscope}%
\begin{pgfscope}%
\pgfpathrectangle{\pgfqpoint{0.889225in}{1.668832in}}{\pgfqpoint{1.162500in}{0.755000in}}%
\pgfusepath{clip}%
\pgfsetbuttcap%
\pgfsetroundjoin%
\definecolor{currentfill}{rgb}{0.000000,0.000000,0.000000}%
\pgfsetfillcolor{currentfill}%
\pgfsetfillopacity{0.500000}%
\pgfsetlinewidth{0.000000pt}%
\definecolor{currentstroke}{rgb}{0.000000,0.000000,0.000000}%
\pgfsetstrokecolor{currentstroke}%
\pgfsetdash{}{0pt}%
\pgfpathmoveto{\pgfqpoint{1.126248in}{2.149550in}}%
\pgfpathcurveto{\pgfqpoint{1.131773in}{2.149550in}}{\pgfqpoint{1.137073in}{2.151746in}}{\pgfqpoint{1.140980in}{2.155652in}}%
\pgfpathcurveto{\pgfqpoint{1.144886in}{2.159559in}}{\pgfqpoint{1.147082in}{2.164859in}}{\pgfqpoint{1.147082in}{2.170384in}}%
\pgfpathcurveto{\pgfqpoint{1.147082in}{2.175909in}}{\pgfqpoint{1.144886in}{2.181208in}}{\pgfqpoint{1.140980in}{2.185115in}}%
\pgfpathcurveto{\pgfqpoint{1.137073in}{2.189022in}}{\pgfqpoint{1.131773in}{2.191217in}}{\pgfqpoint{1.126248in}{2.191217in}}%
\pgfpathcurveto{\pgfqpoint{1.120723in}{2.191217in}}{\pgfqpoint{1.115424in}{2.189022in}}{\pgfqpoint{1.111517in}{2.185115in}}%
\pgfpathcurveto{\pgfqpoint{1.107610in}{2.181208in}}{\pgfqpoint{1.105415in}{2.175909in}}{\pgfqpoint{1.105415in}{2.170384in}}%
\pgfpathcurveto{\pgfqpoint{1.105415in}{2.164859in}}{\pgfqpoint{1.107610in}{2.159559in}}{\pgfqpoint{1.111517in}{2.155652in}}%
\pgfpathcurveto{\pgfqpoint{1.115424in}{2.151746in}}{\pgfqpoint{1.120723in}{2.149550in}}{\pgfqpoint{1.126248in}{2.149550in}}%
\pgfpathclose%
\pgfusepath{fill}%
\end{pgfscope}%
\begin{pgfscope}%
\pgfpathrectangle{\pgfqpoint{0.889225in}{1.668832in}}{\pgfqpoint{1.162500in}{0.755000in}}%
\pgfusepath{clip}%
\pgfsetbuttcap%
\pgfsetroundjoin%
\definecolor{currentfill}{rgb}{0.000000,0.000000,0.000000}%
\pgfsetfillcolor{currentfill}%
\pgfsetfillopacity{0.500000}%
\pgfsetlinewidth{0.000000pt}%
\definecolor{currentstroke}{rgb}{0.000000,0.000000,0.000000}%
\pgfsetstrokecolor{currentstroke}%
\pgfsetdash{}{0pt}%
\pgfpathmoveto{\pgfqpoint{0.977704in}{2.049204in}}%
\pgfpathcurveto{\pgfqpoint{0.983229in}{2.049204in}}{\pgfqpoint{0.988528in}{2.051399in}}{\pgfqpoint{0.992435in}{2.055306in}}%
\pgfpathcurveto{\pgfqpoint{0.996342in}{2.059212in}}{\pgfqpoint{0.998537in}{2.064512in}}{\pgfqpoint{0.998537in}{2.070037in}}%
\pgfpathcurveto{\pgfqpoint{0.998537in}{2.075562in}}{\pgfqpoint{0.996342in}{2.080862in}}{\pgfqpoint{0.992435in}{2.084768in}}%
\pgfpathcurveto{\pgfqpoint{0.988528in}{2.088675in}}{\pgfqpoint{0.983229in}{2.090870in}}{\pgfqpoint{0.977704in}{2.090870in}}%
\pgfpathcurveto{\pgfqpoint{0.972179in}{2.090870in}}{\pgfqpoint{0.966879in}{2.088675in}}{\pgfqpoint{0.962972in}{2.084768in}}%
\pgfpathcurveto{\pgfqpoint{0.959066in}{2.080862in}}{\pgfqpoint{0.956870in}{2.075562in}}{\pgfqpoint{0.956870in}{2.070037in}}%
\pgfpathcurveto{\pgfqpoint{0.956870in}{2.064512in}}{\pgfqpoint{0.959066in}{2.059212in}}{\pgfqpoint{0.962972in}{2.055306in}}%
\pgfpathcurveto{\pgfqpoint{0.966879in}{2.051399in}}{\pgfqpoint{0.972179in}{2.049204in}}{\pgfqpoint{0.977704in}{2.049204in}}%
\pgfpathclose%
\pgfusepath{fill}%
\end{pgfscope}%
\begin{pgfscope}%
\pgfpathrectangle{\pgfqpoint{0.889225in}{1.668832in}}{\pgfqpoint{1.162500in}{0.755000in}}%
\pgfusepath{clip}%
\pgfsetbuttcap%
\pgfsetroundjoin%
\definecolor{currentfill}{rgb}{0.000000,0.000000,0.000000}%
\pgfsetfillcolor{currentfill}%
\pgfsetfillopacity{0.500000}%
\pgfsetlinewidth{0.000000pt}%
\definecolor{currentstroke}{rgb}{0.000000,0.000000,0.000000}%
\pgfsetstrokecolor{currentstroke}%
\pgfsetdash{}{0pt}%
\pgfpathmoveto{\pgfqpoint{0.916904in}{1.665975in}}%
\pgfpathcurveto{\pgfqpoint{0.922429in}{1.665975in}}{\pgfqpoint{0.927728in}{1.668170in}}{\pgfqpoint{0.931635in}{1.672077in}}%
\pgfpathcurveto{\pgfqpoint{0.935542in}{1.675984in}}{\pgfqpoint{0.937737in}{1.681283in}}{\pgfqpoint{0.937737in}{1.686809in}}%
\pgfpathcurveto{\pgfqpoint{0.937737in}{1.692334in}}{\pgfqpoint{0.935542in}{1.697633in}}{\pgfqpoint{0.931635in}{1.701540in}}%
\pgfpathcurveto{\pgfqpoint{0.927728in}{1.705447in}}{\pgfqpoint{0.922429in}{1.707642in}}{\pgfqpoint{0.916904in}{1.707642in}}%
\pgfpathcurveto{\pgfqpoint{0.911379in}{1.707642in}}{\pgfqpoint{0.906079in}{1.705447in}}{\pgfqpoint{0.902172in}{1.701540in}}%
\pgfpathcurveto{\pgfqpoint{0.898265in}{1.697633in}}{\pgfqpoint{0.896070in}{1.692334in}}{\pgfqpoint{0.896070in}{1.686809in}}%
\pgfpathcurveto{\pgfqpoint{0.896070in}{1.681283in}}{\pgfqpoint{0.898265in}{1.675984in}}{\pgfqpoint{0.902172in}{1.672077in}}%
\pgfpathcurveto{\pgfqpoint{0.906079in}{1.668170in}}{\pgfqpoint{0.911379in}{1.665975in}}{\pgfqpoint{0.916904in}{1.665975in}}%
\pgfpathclose%
\pgfusepath{fill}%
\end{pgfscope}%
\begin{pgfscope}%
\pgfpathrectangle{\pgfqpoint{0.889225in}{1.668832in}}{\pgfqpoint{1.162500in}{0.755000in}}%
\pgfusepath{clip}%
\pgfsetbuttcap%
\pgfsetroundjoin%
\definecolor{currentfill}{rgb}{0.000000,0.000000,0.000000}%
\pgfsetfillcolor{currentfill}%
\pgfsetfillopacity{0.500000}%
\pgfsetlinewidth{0.000000pt}%
\definecolor{currentstroke}{rgb}{0.000000,0.000000,0.000000}%
\pgfsetstrokecolor{currentstroke}%
\pgfsetdash{}{0pt}%
\pgfpathmoveto{\pgfqpoint{1.691400in}{2.301682in}}%
\pgfpathcurveto{\pgfqpoint{1.696925in}{2.301682in}}{\pgfqpoint{1.702225in}{2.303877in}}{\pgfqpoint{1.706132in}{2.307784in}}%
\pgfpathcurveto{\pgfqpoint{1.710038in}{2.311691in}}{\pgfqpoint{1.712234in}{2.316990in}}{\pgfqpoint{1.712234in}{2.322515in}}%
\pgfpathcurveto{\pgfqpoint{1.712234in}{2.328041in}}{\pgfqpoint{1.710038in}{2.333340in}}{\pgfqpoint{1.706132in}{2.337247in}}%
\pgfpathcurveto{\pgfqpoint{1.702225in}{2.341154in}}{\pgfqpoint{1.696925in}{2.343349in}}{\pgfqpoint{1.691400in}{2.343349in}}%
\pgfpathcurveto{\pgfqpoint{1.685875in}{2.343349in}}{\pgfqpoint{1.680576in}{2.341154in}}{\pgfqpoint{1.676669in}{2.337247in}}%
\pgfpathcurveto{\pgfqpoint{1.672762in}{2.333340in}}{\pgfqpoint{1.670567in}{2.328041in}}{\pgfqpoint{1.670567in}{2.322515in}}%
\pgfpathcurveto{\pgfqpoint{1.670567in}{2.316990in}}{\pgfqpoint{1.672762in}{2.311691in}}{\pgfqpoint{1.676669in}{2.307784in}}%
\pgfpathcurveto{\pgfqpoint{1.680576in}{2.303877in}}{\pgfqpoint{1.685875in}{2.301682in}}{\pgfqpoint{1.691400in}{2.301682in}}%
\pgfpathclose%
\pgfusepath{fill}%
\end{pgfscope}%
\begin{pgfscope}%
\pgfpathrectangle{\pgfqpoint{0.889225in}{1.668832in}}{\pgfqpoint{1.162500in}{0.755000in}}%
\pgfusepath{clip}%
\pgfsetbuttcap%
\pgfsetroundjoin%
\definecolor{currentfill}{rgb}{0.000000,0.000000,0.000000}%
\pgfsetfillcolor{currentfill}%
\pgfsetfillopacity{0.500000}%
\pgfsetlinewidth{0.000000pt}%
\definecolor{currentstroke}{rgb}{0.000000,0.000000,0.000000}%
\pgfsetstrokecolor{currentstroke}%
\pgfsetdash{}{0pt}%
\pgfpathmoveto{\pgfqpoint{1.460463in}{2.145886in}}%
\pgfpathcurveto{\pgfqpoint{1.465988in}{2.145886in}}{\pgfqpoint{1.471288in}{2.148081in}}{\pgfqpoint{1.475194in}{2.151988in}}%
\pgfpathcurveto{\pgfqpoint{1.479101in}{2.155894in}}{\pgfqpoint{1.481296in}{2.161194in}}{\pgfqpoint{1.481296in}{2.166719in}}%
\pgfpathcurveto{\pgfqpoint{1.481296in}{2.172244in}}{\pgfqpoint{1.479101in}{2.177544in}}{\pgfqpoint{1.475194in}{2.181450in}}%
\pgfpathcurveto{\pgfqpoint{1.471288in}{2.185357in}}{\pgfqpoint{1.465988in}{2.187552in}}{\pgfqpoint{1.460463in}{2.187552in}}%
\pgfpathcurveto{\pgfqpoint{1.454938in}{2.187552in}}{\pgfqpoint{1.449638in}{2.185357in}}{\pgfqpoint{1.445732in}{2.181450in}}%
\pgfpathcurveto{\pgfqpoint{1.441825in}{2.177544in}}{\pgfqpoint{1.439630in}{2.172244in}}{\pgfqpoint{1.439630in}{2.166719in}}%
\pgfpathcurveto{\pgfqpoint{1.439630in}{2.161194in}}{\pgfqpoint{1.441825in}{2.155894in}}{\pgfqpoint{1.445732in}{2.151988in}}%
\pgfpathcurveto{\pgfqpoint{1.449638in}{2.148081in}}{\pgfqpoint{1.454938in}{2.145886in}}{\pgfqpoint{1.460463in}{2.145886in}}%
\pgfpathclose%
\pgfusepath{fill}%
\end{pgfscope}%
\begin{pgfscope}%
\pgfpathrectangle{\pgfqpoint{0.889225in}{1.668832in}}{\pgfqpoint{1.162500in}{0.755000in}}%
\pgfusepath{clip}%
\pgfsetbuttcap%
\pgfsetroundjoin%
\definecolor{currentfill}{rgb}{0.000000,0.000000,0.000000}%
\pgfsetfillcolor{currentfill}%
\pgfsetfillopacity{0.500000}%
\pgfsetlinewidth{0.000000pt}%
\definecolor{currentstroke}{rgb}{0.000000,0.000000,0.000000}%
\pgfsetstrokecolor{currentstroke}%
\pgfsetdash{}{0pt}%
\pgfpathmoveto{\pgfqpoint{1.303193in}{1.969105in}}%
\pgfpathcurveto{\pgfqpoint{1.308718in}{1.969105in}}{\pgfqpoint{1.314017in}{1.971300in}}{\pgfqpoint{1.317924in}{1.975207in}}%
\pgfpathcurveto{\pgfqpoint{1.321831in}{1.979114in}}{\pgfqpoint{1.324026in}{1.984413in}}{\pgfqpoint{1.324026in}{1.989939in}}%
\pgfpathcurveto{\pgfqpoint{1.324026in}{1.995464in}}{\pgfqpoint{1.321831in}{2.000763in}}{\pgfqpoint{1.317924in}{2.004670in}}%
\pgfpathcurveto{\pgfqpoint{1.314017in}{2.008577in}}{\pgfqpoint{1.308718in}{2.010772in}}{\pgfqpoint{1.303193in}{2.010772in}}%
\pgfpathcurveto{\pgfqpoint{1.297667in}{2.010772in}}{\pgfqpoint{1.292368in}{2.008577in}}{\pgfqpoint{1.288461in}{2.004670in}}%
\pgfpathcurveto{\pgfqpoint{1.284554in}{2.000763in}}{\pgfqpoint{1.282359in}{1.995464in}}{\pgfqpoint{1.282359in}{1.989939in}}%
\pgfpathcurveto{\pgfqpoint{1.282359in}{1.984413in}}{\pgfqpoint{1.284554in}{1.979114in}}{\pgfqpoint{1.288461in}{1.975207in}}%
\pgfpathcurveto{\pgfqpoint{1.292368in}{1.971300in}}{\pgfqpoint{1.297667in}{1.969105in}}{\pgfqpoint{1.303193in}{1.969105in}}%
\pgfpathclose%
\pgfusepath{fill}%
\end{pgfscope}%
\begin{pgfscope}%
\pgfpathrectangle{\pgfqpoint{0.889225in}{1.668832in}}{\pgfqpoint{1.162500in}{0.755000in}}%
\pgfusepath{clip}%
\pgfsetbuttcap%
\pgfsetroundjoin%
\definecolor{currentfill}{rgb}{0.000000,0.000000,0.000000}%
\pgfsetfillcolor{currentfill}%
\pgfsetfillopacity{0.500000}%
\pgfsetlinewidth{0.000000pt}%
\definecolor{currentstroke}{rgb}{0.000000,0.000000,0.000000}%
\pgfsetstrokecolor{currentstroke}%
\pgfsetdash{}{0pt}%
\pgfpathmoveto{\pgfqpoint{1.689350in}{2.239104in}}%
\pgfpathcurveto{\pgfqpoint{1.694875in}{2.239104in}}{\pgfqpoint{1.700175in}{2.241300in}}{\pgfqpoint{1.704082in}{2.245206in}}%
\pgfpathcurveto{\pgfqpoint{1.707989in}{2.249113in}}{\pgfqpoint{1.710184in}{2.254413in}}{\pgfqpoint{1.710184in}{2.259938in}}%
\pgfpathcurveto{\pgfqpoint{1.710184in}{2.265463in}}{\pgfqpoint{1.707989in}{2.270762in}}{\pgfqpoint{1.704082in}{2.274669in}}%
\pgfpathcurveto{\pgfqpoint{1.700175in}{2.278576in}}{\pgfqpoint{1.694875in}{2.280771in}}{\pgfqpoint{1.689350in}{2.280771in}}%
\pgfpathcurveto{\pgfqpoint{1.683825in}{2.280771in}}{\pgfqpoint{1.678526in}{2.278576in}}{\pgfqpoint{1.674619in}{2.274669in}}%
\pgfpathcurveto{\pgfqpoint{1.670712in}{2.270762in}}{\pgfqpoint{1.668517in}{2.265463in}}{\pgfqpoint{1.668517in}{2.259938in}}%
\pgfpathcurveto{\pgfqpoint{1.668517in}{2.254413in}}{\pgfqpoint{1.670712in}{2.249113in}}{\pgfqpoint{1.674619in}{2.245206in}}%
\pgfpathcurveto{\pgfqpoint{1.678526in}{2.241300in}}{\pgfqpoint{1.683825in}{2.239104in}}{\pgfqpoint{1.689350in}{2.239104in}}%
\pgfpathclose%
\pgfusepath{fill}%
\end{pgfscope}%
\begin{pgfscope}%
\pgfpathrectangle{\pgfqpoint{0.889225in}{1.668832in}}{\pgfqpoint{1.162500in}{0.755000in}}%
\pgfusepath{clip}%
\pgfsetbuttcap%
\pgfsetroundjoin%
\definecolor{currentfill}{rgb}{0.000000,0.000000,0.000000}%
\pgfsetfillcolor{currentfill}%
\pgfsetfillopacity{0.500000}%
\pgfsetlinewidth{0.000000pt}%
\definecolor{currentstroke}{rgb}{0.000000,0.000000,0.000000}%
\pgfsetstrokecolor{currentstroke}%
\pgfsetdash{}{0pt}%
\pgfpathmoveto{\pgfqpoint{1.102396in}{1.960666in}}%
\pgfpathcurveto{\pgfqpoint{1.107921in}{1.960666in}}{\pgfqpoint{1.113221in}{1.962861in}}{\pgfqpoint{1.117128in}{1.966768in}}%
\pgfpathcurveto{\pgfqpoint{1.121035in}{1.970675in}}{\pgfqpoint{1.123230in}{1.975974in}}{\pgfqpoint{1.123230in}{1.981499in}}%
\pgfpathcurveto{\pgfqpoint{1.123230in}{1.987024in}}{\pgfqpoint{1.121035in}{1.992324in}}{\pgfqpoint{1.117128in}{1.996231in}}%
\pgfpathcurveto{\pgfqpoint{1.113221in}{2.000138in}}{\pgfqpoint{1.107921in}{2.002333in}}{\pgfqpoint{1.102396in}{2.002333in}}%
\pgfpathcurveto{\pgfqpoint{1.096871in}{2.002333in}}{\pgfqpoint{1.091572in}{2.000138in}}{\pgfqpoint{1.087665in}{1.996231in}}%
\pgfpathcurveto{\pgfqpoint{1.083758in}{1.992324in}}{\pgfqpoint{1.081563in}{1.987024in}}{\pgfqpoint{1.081563in}{1.981499in}}%
\pgfpathcurveto{\pgfqpoint{1.081563in}{1.975974in}}{\pgfqpoint{1.083758in}{1.970675in}}{\pgfqpoint{1.087665in}{1.966768in}}%
\pgfpathcurveto{\pgfqpoint{1.091572in}{1.962861in}}{\pgfqpoint{1.096871in}{1.960666in}}{\pgfqpoint{1.102396in}{1.960666in}}%
\pgfpathclose%
\pgfusepath{fill}%
\end{pgfscope}%
\begin{pgfscope}%
\pgfpathrectangle{\pgfqpoint{0.889225in}{1.668832in}}{\pgfqpoint{1.162500in}{0.755000in}}%
\pgfusepath{clip}%
\pgfsetbuttcap%
\pgfsetroundjoin%
\definecolor{currentfill}{rgb}{0.000000,0.000000,0.000000}%
\pgfsetfillcolor{currentfill}%
\pgfsetfillopacity{0.500000}%
\pgfsetlinewidth{0.000000pt}%
\definecolor{currentstroke}{rgb}{0.000000,0.000000,0.000000}%
\pgfsetstrokecolor{currentstroke}%
\pgfsetdash{}{0pt}%
\pgfpathmoveto{\pgfqpoint{1.143186in}{1.722959in}}%
\pgfpathcurveto{\pgfqpoint{1.148711in}{1.722959in}}{\pgfqpoint{1.154011in}{1.725154in}}{\pgfqpoint{1.157918in}{1.729061in}}%
\pgfpathcurveto{\pgfqpoint{1.161825in}{1.732968in}}{\pgfqpoint{1.164020in}{1.738267in}}{\pgfqpoint{1.164020in}{1.743792in}}%
\pgfpathcurveto{\pgfqpoint{1.164020in}{1.749317in}}{\pgfqpoint{1.161825in}{1.754617in}}{\pgfqpoint{1.157918in}{1.758523in}}%
\pgfpathcurveto{\pgfqpoint{1.154011in}{1.762430in}}{\pgfqpoint{1.148711in}{1.764625in}}{\pgfqpoint{1.143186in}{1.764625in}}%
\pgfpathcurveto{\pgfqpoint{1.137661in}{1.764625in}}{\pgfqpoint{1.132362in}{1.762430in}}{\pgfqpoint{1.128455in}{1.758523in}}%
\pgfpathcurveto{\pgfqpoint{1.124548in}{1.754617in}}{\pgfqpoint{1.122353in}{1.749317in}}{\pgfqpoint{1.122353in}{1.743792in}}%
\pgfpathcurveto{\pgfqpoint{1.122353in}{1.738267in}}{\pgfqpoint{1.124548in}{1.732968in}}{\pgfqpoint{1.128455in}{1.729061in}}%
\pgfpathcurveto{\pgfqpoint{1.132362in}{1.725154in}}{\pgfqpoint{1.137661in}{1.722959in}}{\pgfqpoint{1.143186in}{1.722959in}}%
\pgfpathclose%
\pgfusepath{fill}%
\end{pgfscope}%
\begin{pgfscope}%
\pgfpathrectangle{\pgfqpoint{0.889225in}{1.668832in}}{\pgfqpoint{1.162500in}{0.755000in}}%
\pgfusepath{clip}%
\pgfsetbuttcap%
\pgfsetroundjoin%
\definecolor{currentfill}{rgb}{0.000000,0.000000,0.000000}%
\pgfsetfillcolor{currentfill}%
\pgfsetfillopacity{0.500000}%
\pgfsetlinewidth{0.000000pt}%
\definecolor{currentstroke}{rgb}{0.000000,0.000000,0.000000}%
\pgfsetstrokecolor{currentstroke}%
\pgfsetdash{}{0pt}%
\pgfpathmoveto{\pgfqpoint{2.024046in}{2.267350in}}%
\pgfpathcurveto{\pgfqpoint{2.029571in}{2.267350in}}{\pgfqpoint{2.034871in}{2.269545in}}{\pgfqpoint{2.038778in}{2.273452in}}%
\pgfpathcurveto{\pgfqpoint{2.042685in}{2.277359in}}{\pgfqpoint{2.044880in}{2.282659in}}{\pgfqpoint{2.044880in}{2.288184in}}%
\pgfpathcurveto{\pgfqpoint{2.044880in}{2.293709in}}{\pgfqpoint{2.042685in}{2.299008in}}{\pgfqpoint{2.038778in}{2.302915in}}%
\pgfpathcurveto{\pgfqpoint{2.034871in}{2.306822in}}{\pgfqpoint{2.029571in}{2.309017in}}{\pgfqpoint{2.024046in}{2.309017in}}%
\pgfpathcurveto{\pgfqpoint{2.018521in}{2.309017in}}{\pgfqpoint{2.013222in}{2.306822in}}{\pgfqpoint{2.009315in}{2.302915in}}%
\pgfpathcurveto{\pgfqpoint{2.005408in}{2.299008in}}{\pgfqpoint{2.003213in}{2.293709in}}{\pgfqpoint{2.003213in}{2.288184in}}%
\pgfpathcurveto{\pgfqpoint{2.003213in}{2.282659in}}{\pgfqpoint{2.005408in}{2.277359in}}{\pgfqpoint{2.009315in}{2.273452in}}%
\pgfpathcurveto{\pgfqpoint{2.013222in}{2.269545in}}{\pgfqpoint{2.018521in}{2.267350in}}{\pgfqpoint{2.024046in}{2.267350in}}%
\pgfpathclose%
\pgfusepath{fill}%
\end{pgfscope}%
\begin{pgfscope}%
\pgfpathrectangle{\pgfqpoint{0.889225in}{1.668832in}}{\pgfqpoint{1.162500in}{0.755000in}}%
\pgfusepath{clip}%
\pgfsetbuttcap%
\pgfsetroundjoin%
\definecolor{currentfill}{rgb}{0.000000,0.000000,0.000000}%
\pgfsetfillcolor{currentfill}%
\pgfsetfillopacity{0.500000}%
\pgfsetlinewidth{0.000000pt}%
\definecolor{currentstroke}{rgb}{0.000000,0.000000,0.000000}%
\pgfsetstrokecolor{currentstroke}%
\pgfsetdash{}{0pt}%
\pgfpathmoveto{\pgfqpoint{1.402285in}{2.236796in}}%
\pgfpathcurveto{\pgfqpoint{1.407810in}{2.236796in}}{\pgfqpoint{1.413110in}{2.238991in}}{\pgfqpoint{1.417016in}{2.242898in}}%
\pgfpathcurveto{\pgfqpoint{1.420923in}{2.246805in}}{\pgfqpoint{1.423118in}{2.252104in}}{\pgfqpoint{1.423118in}{2.257629in}}%
\pgfpathcurveto{\pgfqpoint{1.423118in}{2.263154in}}{\pgfqpoint{1.420923in}{2.268454in}}{\pgfqpoint{1.417016in}{2.272361in}}%
\pgfpathcurveto{\pgfqpoint{1.413110in}{2.276267in}}{\pgfqpoint{1.407810in}{2.278463in}}{\pgfqpoint{1.402285in}{2.278463in}}%
\pgfpathcurveto{\pgfqpoint{1.396760in}{2.278463in}}{\pgfqpoint{1.391460in}{2.276267in}}{\pgfqpoint{1.387554in}{2.272361in}}%
\pgfpathcurveto{\pgfqpoint{1.383647in}{2.268454in}}{\pgfqpoint{1.381452in}{2.263154in}}{\pgfqpoint{1.381452in}{2.257629in}}%
\pgfpathcurveto{\pgfqpoint{1.381452in}{2.252104in}}{\pgfqpoint{1.383647in}{2.246805in}}{\pgfqpoint{1.387554in}{2.242898in}}%
\pgfpathcurveto{\pgfqpoint{1.391460in}{2.238991in}}{\pgfqpoint{1.396760in}{2.236796in}}{\pgfqpoint{1.402285in}{2.236796in}}%
\pgfpathclose%
\pgfusepath{fill}%
\end{pgfscope}%
\begin{pgfscope}%
\pgfpathrectangle{\pgfqpoint{0.889225in}{1.668832in}}{\pgfqpoint{1.162500in}{0.755000in}}%
\pgfusepath{clip}%
\pgfsetbuttcap%
\pgfsetroundjoin%
\definecolor{currentfill}{rgb}{0.000000,0.000000,0.000000}%
\pgfsetfillcolor{currentfill}%
\pgfsetfillopacity{0.500000}%
\pgfsetlinewidth{0.000000pt}%
\definecolor{currentstroke}{rgb}{0.000000,0.000000,0.000000}%
\pgfsetstrokecolor{currentstroke}%
\pgfsetdash{}{0pt}%
\pgfpathmoveto{\pgfqpoint{0.973483in}{2.023720in}}%
\pgfpathcurveto{\pgfqpoint{0.979008in}{2.023720in}}{\pgfqpoint{0.984308in}{2.025915in}}{\pgfqpoint{0.988214in}{2.029822in}}%
\pgfpathcurveto{\pgfqpoint{0.992121in}{2.033729in}}{\pgfqpoint{0.994316in}{2.039028in}}{\pgfqpoint{0.994316in}{2.044553in}}%
\pgfpathcurveto{\pgfqpoint{0.994316in}{2.050078in}}{\pgfqpoint{0.992121in}{2.055378in}}{\pgfqpoint{0.988214in}{2.059285in}}%
\pgfpathcurveto{\pgfqpoint{0.984308in}{2.063191in}}{\pgfqpoint{0.979008in}{2.065387in}}{\pgfqpoint{0.973483in}{2.065387in}}%
\pgfpathcurveto{\pgfqpoint{0.967958in}{2.065387in}}{\pgfqpoint{0.962658in}{2.063191in}}{\pgfqpoint{0.958752in}{2.059285in}}%
\pgfpathcurveto{\pgfqpoint{0.954845in}{2.055378in}}{\pgfqpoint{0.952650in}{2.050078in}}{\pgfqpoint{0.952650in}{2.044553in}}%
\pgfpathcurveto{\pgfqpoint{0.952650in}{2.039028in}}{\pgfqpoint{0.954845in}{2.033729in}}{\pgfqpoint{0.958752in}{2.029822in}}%
\pgfpathcurveto{\pgfqpoint{0.962658in}{2.025915in}}{\pgfqpoint{0.967958in}{2.023720in}}{\pgfqpoint{0.973483in}{2.023720in}}%
\pgfpathclose%
\pgfusepath{fill}%
\end{pgfscope}%
\begin{pgfscope}%
\pgfsetbuttcap%
\pgfsetroundjoin%
\definecolor{currentfill}{rgb}{0.000000,0.000000,0.000000}%
\pgfsetfillcolor{currentfill}%
\pgfsetlinewidth{0.803000pt}%
\definecolor{currentstroke}{rgb}{0.000000,0.000000,0.000000}%
\pgfsetstrokecolor{currentstroke}%
\pgfsetdash{}{0pt}%
\pgfsys@defobject{currentmarker}{\pgfqpoint{-0.048611in}{0.000000in}}{\pgfqpoint{0.000000in}{0.000000in}}{%
\pgfpathmoveto{\pgfqpoint{0.000000in}{0.000000in}}%
\pgfpathlineto{\pgfqpoint{-0.048611in}{0.000000in}}%
\pgfusepath{stroke,fill}%
}%
\begin{pgfscope}%
\pgfsys@transformshift{0.889225in}{1.751846in}%
\pgfsys@useobject{currentmarker}{}%
\end{pgfscope}%
\end{pgfscope}%
\begin{pgfscope}%
\pgftext[x=0.523095in,y=1.709636in,left,base]{\rmfamily\fontsize{8.000000}{9.600000}\selectfont \(\displaystyle 0.025\)}%
\end{pgfscope}%
\begin{pgfscope}%
\pgfsetbuttcap%
\pgfsetroundjoin%
\definecolor{currentfill}{rgb}{0.000000,0.000000,0.000000}%
\pgfsetfillcolor{currentfill}%
\pgfsetlinewidth{0.803000pt}%
\definecolor{currentstroke}{rgb}{0.000000,0.000000,0.000000}%
\pgfsetstrokecolor{currentstroke}%
\pgfsetdash{}{0pt}%
\pgfsys@defobject{currentmarker}{\pgfqpoint{-0.048611in}{0.000000in}}{\pgfqpoint{0.000000in}{0.000000in}}{%
\pgfpathmoveto{\pgfqpoint{0.000000in}{0.000000in}}%
\pgfpathlineto{\pgfqpoint{-0.048611in}{0.000000in}}%
\pgfusepath{stroke,fill}%
}%
\begin{pgfscope}%
\pgfsys@transformshift{0.889225in}{2.018356in}%
\pgfsys@useobject{currentmarker}{}%
\end{pgfscope}%
\end{pgfscope}%
\begin{pgfscope}%
\pgftext[x=0.523095in,y=1.976147in,left,base]{\rmfamily\fontsize{8.000000}{9.600000}\selectfont \(\displaystyle 0.050\)}%
\end{pgfscope}%
\begin{pgfscope}%
\pgfsetbuttcap%
\pgfsetroundjoin%
\definecolor{currentfill}{rgb}{0.000000,0.000000,0.000000}%
\pgfsetfillcolor{currentfill}%
\pgfsetlinewidth{0.803000pt}%
\definecolor{currentstroke}{rgb}{0.000000,0.000000,0.000000}%
\pgfsetstrokecolor{currentstroke}%
\pgfsetdash{}{0pt}%
\pgfsys@defobject{currentmarker}{\pgfqpoint{-0.048611in}{0.000000in}}{\pgfqpoint{0.000000in}{0.000000in}}{%
\pgfpathmoveto{\pgfqpoint{0.000000in}{0.000000in}}%
\pgfpathlineto{\pgfqpoint{-0.048611in}{0.000000in}}%
\pgfusepath{stroke,fill}%
}%
\begin{pgfscope}%
\pgfsys@transformshift{0.889225in}{2.284867in}%
\pgfsys@useobject{currentmarker}{}%
\end{pgfscope}%
\end{pgfscope}%
\begin{pgfscope}%
\pgftext[x=0.523095in,y=2.242658in,left,base]{\rmfamily\fontsize{8.000000}{9.600000}\selectfont \(\displaystyle 0.075\)}%
\end{pgfscope}%
\begin{pgfscope}%
\pgftext[x=0.467539in,y=2.046332in,,bottom,rotate=90.000000]{\rmfamily\fontsize{16.000000}{19.200000}\selectfont u0}%
\end{pgfscope}%
\begin{pgfscope}%
\pgfsetrectcap%
\pgfsetmiterjoin%
\pgfsetlinewidth{0.803000pt}%
\definecolor{currentstroke}{rgb}{0.501961,0.501961,0.501961}%
\pgfsetstrokecolor{currentstroke}%
\pgfsetdash{}{0pt}%
\pgfpathmoveto{\pgfqpoint{0.889225in}{1.668832in}}%
\pgfpathlineto{\pgfqpoint{0.889225in}{2.423832in}}%
\pgfusepath{stroke}%
\end{pgfscope}%
\begin{pgfscope}%
\pgfsetrectcap%
\pgfsetmiterjoin%
\pgfsetlinewidth{0.803000pt}%
\definecolor{currentstroke}{rgb}{0.501961,0.501961,0.501961}%
\pgfsetstrokecolor{currentstroke}%
\pgfsetdash{}{0pt}%
\pgfpathmoveto{\pgfqpoint{2.051725in}{1.668832in}}%
\pgfpathlineto{\pgfqpoint{2.051725in}{2.423832in}}%
\pgfusepath{stroke}%
\end{pgfscope}%
\begin{pgfscope}%
\pgfsetrectcap%
\pgfsetmiterjoin%
\pgfsetlinewidth{0.803000pt}%
\definecolor{currentstroke}{rgb}{0.501961,0.501961,0.501961}%
\pgfsetstrokecolor{currentstroke}%
\pgfsetdash{}{0pt}%
\pgfpathmoveto{\pgfqpoint{0.889225in}{1.668832in}}%
\pgfpathlineto{\pgfqpoint{2.051725in}{1.668832in}}%
\pgfusepath{stroke}%
\end{pgfscope}%
\begin{pgfscope}%
\pgfsetrectcap%
\pgfsetmiterjoin%
\pgfsetlinewidth{0.803000pt}%
\definecolor{currentstroke}{rgb}{0.501961,0.501961,0.501961}%
\pgfsetstrokecolor{currentstroke}%
\pgfsetdash{}{0pt}%
\pgfpathmoveto{\pgfqpoint{0.889225in}{2.423832in}}%
\pgfpathlineto{\pgfqpoint{2.051725in}{2.423832in}}%
\pgfusepath{stroke}%
\end{pgfscope}%
\begin{pgfscope}%
\pgfsetbuttcap%
\pgfsetmiterjoin%
\definecolor{currentfill}{rgb}{1.000000,1.000000,1.000000}%
\pgfsetfillcolor{currentfill}%
\pgfsetlinewidth{0.000000pt}%
\definecolor{currentstroke}{rgb}{0.000000,0.000000,0.000000}%
\pgfsetstrokecolor{currentstroke}%
\pgfsetstrokeopacity{0.000000}%
\pgfsetdash{}{0pt}%
\pgfpathmoveto{\pgfqpoint{2.051725in}{1.668832in}}%
\pgfpathlineto{\pgfqpoint{3.214225in}{1.668832in}}%
\pgfpathlineto{\pgfqpoint{3.214225in}{2.423832in}}%
\pgfpathlineto{\pgfqpoint{2.051725in}{2.423832in}}%
\pgfpathclose%
\pgfusepath{fill}%
\end{pgfscope}%
\begin{pgfscope}%
\pgfpathrectangle{\pgfqpoint{2.051725in}{1.668832in}}{\pgfqpoint{1.162500in}{0.755000in}}%
\pgfusepath{clip}%
\pgfsetbuttcap%
\pgfsetroundjoin%
\definecolor{currentfill}{rgb}{0.000000,0.000000,0.000000}%
\pgfsetfillcolor{currentfill}%
\pgfsetfillopacity{0.500000}%
\pgfsetlinewidth{0.000000pt}%
\definecolor{currentstroke}{rgb}{0.000000,0.000000,0.000000}%
\pgfsetstrokecolor{currentstroke}%
\pgfsetdash{}{0pt}%
\pgfpathmoveto{\pgfqpoint{3.107569in}{1.935152in}}%
\pgfpathcurveto{\pgfqpoint{3.113094in}{1.935152in}}{\pgfqpoint{3.118394in}{1.937347in}}{\pgfqpoint{3.122301in}{1.941254in}}%
\pgfpathcurveto{\pgfqpoint{3.126207in}{1.945161in}}{\pgfqpoint{3.128403in}{1.950460in}}{\pgfqpoint{3.128403in}{1.955985in}}%
\pgfpathcurveto{\pgfqpoint{3.128403in}{1.961510in}}{\pgfqpoint{3.126207in}{1.966810in}}{\pgfqpoint{3.122301in}{1.970717in}}%
\pgfpathcurveto{\pgfqpoint{3.118394in}{1.974623in}}{\pgfqpoint{3.113094in}{1.976819in}}{\pgfqpoint{3.107569in}{1.976819in}}%
\pgfpathcurveto{\pgfqpoint{3.102044in}{1.976819in}}{\pgfqpoint{3.096745in}{1.974623in}}{\pgfqpoint{3.092838in}{1.970717in}}%
\pgfpathcurveto{\pgfqpoint{3.088931in}{1.966810in}}{\pgfqpoint{3.086736in}{1.961510in}}{\pgfqpoint{3.086736in}{1.955985in}}%
\pgfpathcurveto{\pgfqpoint{3.086736in}{1.950460in}}{\pgfqpoint{3.088931in}{1.945161in}}{\pgfqpoint{3.092838in}{1.941254in}}%
\pgfpathcurveto{\pgfqpoint{3.096745in}{1.937347in}}{\pgfqpoint{3.102044in}{1.935152in}}{\pgfqpoint{3.107569in}{1.935152in}}%
\pgfpathclose%
\pgfusepath{fill}%
\end{pgfscope}%
\begin{pgfscope}%
\pgfpathrectangle{\pgfqpoint{2.051725in}{1.668832in}}{\pgfqpoint{1.162500in}{0.755000in}}%
\pgfusepath{clip}%
\pgfsetbuttcap%
\pgfsetroundjoin%
\definecolor{currentfill}{rgb}{0.000000,0.000000,0.000000}%
\pgfsetfillcolor{currentfill}%
\pgfsetfillopacity{0.500000}%
\pgfsetlinewidth{0.000000pt}%
\definecolor{currentstroke}{rgb}{0.000000,0.000000,0.000000}%
\pgfsetstrokecolor{currentstroke}%
\pgfsetdash{}{0pt}%
\pgfpathmoveto{\pgfqpoint{2.270173in}{2.326950in}}%
\pgfpathcurveto{\pgfqpoint{2.275698in}{2.326950in}}{\pgfqpoint{2.280997in}{2.329145in}}{\pgfqpoint{2.284904in}{2.333052in}}%
\pgfpathcurveto{\pgfqpoint{2.288811in}{2.336959in}}{\pgfqpoint{2.291006in}{2.342258in}}{\pgfqpoint{2.291006in}{2.347783in}}%
\pgfpathcurveto{\pgfqpoint{2.291006in}{2.353308in}}{\pgfqpoint{2.288811in}{2.358608in}}{\pgfqpoint{2.284904in}{2.362515in}}%
\pgfpathcurveto{\pgfqpoint{2.280997in}{2.366421in}}{\pgfqpoint{2.275698in}{2.368616in}}{\pgfqpoint{2.270173in}{2.368616in}}%
\pgfpathcurveto{\pgfqpoint{2.264648in}{2.368616in}}{\pgfqpoint{2.259348in}{2.366421in}}{\pgfqpoint{2.255441in}{2.362515in}}%
\pgfpathcurveto{\pgfqpoint{2.251535in}{2.358608in}}{\pgfqpoint{2.249339in}{2.353308in}}{\pgfqpoint{2.249339in}{2.347783in}}%
\pgfpathcurveto{\pgfqpoint{2.249339in}{2.342258in}}{\pgfqpoint{2.251535in}{2.336959in}}{\pgfqpoint{2.255441in}{2.333052in}}%
\pgfpathcurveto{\pgfqpoint{2.259348in}{2.329145in}}{\pgfqpoint{2.264648in}{2.326950in}}{\pgfqpoint{2.270173in}{2.326950in}}%
\pgfpathclose%
\pgfusepath{fill}%
\end{pgfscope}%
\begin{pgfscope}%
\pgfpathrectangle{\pgfqpoint{2.051725in}{1.668832in}}{\pgfqpoint{1.162500in}{0.755000in}}%
\pgfusepath{clip}%
\pgfsetbuttcap%
\pgfsetroundjoin%
\definecolor{currentfill}{rgb}{0.000000,0.000000,0.000000}%
\pgfsetfillcolor{currentfill}%
\pgfsetfillopacity{0.500000}%
\pgfsetlinewidth{0.000000pt}%
\definecolor{currentstroke}{rgb}{0.000000,0.000000,0.000000}%
\pgfsetstrokecolor{currentstroke}%
\pgfsetdash{}{0pt}%
\pgfpathmoveto{\pgfqpoint{2.264867in}{2.385023in}}%
\pgfpathcurveto{\pgfqpoint{2.270392in}{2.385023in}}{\pgfqpoint{2.275692in}{2.387218in}}{\pgfqpoint{2.279599in}{2.391125in}}%
\pgfpathcurveto{\pgfqpoint{2.283506in}{2.395032in}}{\pgfqpoint{2.285701in}{2.400331in}}{\pgfqpoint{2.285701in}{2.405856in}}%
\pgfpathcurveto{\pgfqpoint{2.285701in}{2.411381in}}{\pgfqpoint{2.283506in}{2.416681in}}{\pgfqpoint{2.279599in}{2.420588in}}%
\pgfpathcurveto{\pgfqpoint{2.275692in}{2.424494in}}{\pgfqpoint{2.270392in}{2.426690in}}{\pgfqpoint{2.264867in}{2.426690in}}%
\pgfpathcurveto{\pgfqpoint{2.259342in}{2.426690in}}{\pgfqpoint{2.254043in}{2.424494in}}{\pgfqpoint{2.250136in}{2.420588in}}%
\pgfpathcurveto{\pgfqpoint{2.246229in}{2.416681in}}{\pgfqpoint{2.244034in}{2.411381in}}{\pgfqpoint{2.244034in}{2.405856in}}%
\pgfpathcurveto{\pgfqpoint{2.244034in}{2.400331in}}{\pgfqpoint{2.246229in}{2.395032in}}{\pgfqpoint{2.250136in}{2.391125in}}%
\pgfpathcurveto{\pgfqpoint{2.254043in}{2.387218in}}{\pgfqpoint{2.259342in}{2.385023in}}{\pgfqpoint{2.264867in}{2.385023in}}%
\pgfpathclose%
\pgfusepath{fill}%
\end{pgfscope}%
\begin{pgfscope}%
\pgfpathrectangle{\pgfqpoint{2.051725in}{1.668832in}}{\pgfqpoint{1.162500in}{0.755000in}}%
\pgfusepath{clip}%
\pgfsetbuttcap%
\pgfsetroundjoin%
\definecolor{currentfill}{rgb}{0.000000,0.000000,0.000000}%
\pgfsetfillcolor{currentfill}%
\pgfsetfillopacity{0.500000}%
\pgfsetlinewidth{0.000000pt}%
\definecolor{currentstroke}{rgb}{0.000000,0.000000,0.000000}%
\pgfsetstrokecolor{currentstroke}%
\pgfsetdash{}{0pt}%
\pgfpathmoveto{\pgfqpoint{2.154193in}{2.103661in}}%
\pgfpathcurveto{\pgfqpoint{2.159718in}{2.103661in}}{\pgfqpoint{2.165018in}{2.105856in}}{\pgfqpoint{2.168924in}{2.109762in}}%
\pgfpathcurveto{\pgfqpoint{2.172831in}{2.113669in}}{\pgfqpoint{2.175026in}{2.118969in}}{\pgfqpoint{2.175026in}{2.124494in}}%
\pgfpathcurveto{\pgfqpoint{2.175026in}{2.130019in}}{\pgfqpoint{2.172831in}{2.135318in}}{\pgfqpoint{2.168924in}{2.139225in}}%
\pgfpathcurveto{\pgfqpoint{2.165018in}{2.143132in}}{\pgfqpoint{2.159718in}{2.145327in}}{\pgfqpoint{2.154193in}{2.145327in}}%
\pgfpathcurveto{\pgfqpoint{2.148668in}{2.145327in}}{\pgfqpoint{2.143368in}{2.143132in}}{\pgfqpoint{2.139462in}{2.139225in}}%
\pgfpathcurveto{\pgfqpoint{2.135555in}{2.135318in}}{\pgfqpoint{2.133360in}{2.130019in}}{\pgfqpoint{2.133360in}{2.124494in}}%
\pgfpathcurveto{\pgfqpoint{2.133360in}{2.118969in}}{\pgfqpoint{2.135555in}{2.113669in}}{\pgfqpoint{2.139462in}{2.109762in}}%
\pgfpathcurveto{\pgfqpoint{2.143368in}{2.105856in}}{\pgfqpoint{2.148668in}{2.103661in}}{\pgfqpoint{2.154193in}{2.103661in}}%
\pgfpathclose%
\pgfusepath{fill}%
\end{pgfscope}%
\begin{pgfscope}%
\pgfpathrectangle{\pgfqpoint{2.051725in}{1.668832in}}{\pgfqpoint{1.162500in}{0.755000in}}%
\pgfusepath{clip}%
\pgfsetbuttcap%
\pgfsetroundjoin%
\definecolor{currentfill}{rgb}{0.000000,0.000000,0.000000}%
\pgfsetfillcolor{currentfill}%
\pgfsetfillopacity{0.500000}%
\pgfsetlinewidth{0.000000pt}%
\definecolor{currentstroke}{rgb}{0.000000,0.000000,0.000000}%
\pgfsetstrokecolor{currentstroke}%
\pgfsetdash{}{0pt}%
\pgfpathmoveto{\pgfqpoint{2.183344in}{2.149550in}}%
\pgfpathcurveto{\pgfqpoint{2.188869in}{2.149550in}}{\pgfqpoint{2.194169in}{2.151746in}}{\pgfqpoint{2.198076in}{2.155652in}}%
\pgfpathcurveto{\pgfqpoint{2.201982in}{2.159559in}}{\pgfqpoint{2.204178in}{2.164859in}}{\pgfqpoint{2.204178in}{2.170384in}}%
\pgfpathcurveto{\pgfqpoint{2.204178in}{2.175909in}}{\pgfqpoint{2.201982in}{2.181208in}}{\pgfqpoint{2.198076in}{2.185115in}}%
\pgfpathcurveto{\pgfqpoint{2.194169in}{2.189022in}}{\pgfqpoint{2.188869in}{2.191217in}}{\pgfqpoint{2.183344in}{2.191217in}}%
\pgfpathcurveto{\pgfqpoint{2.177819in}{2.191217in}}{\pgfqpoint{2.172520in}{2.189022in}}{\pgfqpoint{2.168613in}{2.185115in}}%
\pgfpathcurveto{\pgfqpoint{2.164706in}{2.181208in}}{\pgfqpoint{2.162511in}{2.175909in}}{\pgfqpoint{2.162511in}{2.170384in}}%
\pgfpathcurveto{\pgfqpoint{2.162511in}{2.164859in}}{\pgfqpoint{2.164706in}{2.159559in}}{\pgfqpoint{2.168613in}{2.155652in}}%
\pgfpathcurveto{\pgfqpoint{2.172520in}{2.151746in}}{\pgfqpoint{2.177819in}{2.149550in}}{\pgfqpoint{2.183344in}{2.149550in}}%
\pgfpathclose%
\pgfusepath{fill}%
\end{pgfscope}%
\begin{pgfscope}%
\pgfpathrectangle{\pgfqpoint{2.051725in}{1.668832in}}{\pgfqpoint{1.162500in}{0.755000in}}%
\pgfusepath{clip}%
\pgfsetbuttcap%
\pgfsetroundjoin%
\definecolor{currentfill}{rgb}{0.000000,0.000000,0.000000}%
\pgfsetfillcolor{currentfill}%
\pgfsetfillopacity{0.500000}%
\pgfsetlinewidth{0.000000pt}%
\definecolor{currentstroke}{rgb}{0.000000,0.000000,0.000000}%
\pgfsetstrokecolor{currentstroke}%
\pgfsetdash{}{0pt}%
\pgfpathmoveto{\pgfqpoint{2.082232in}{2.049204in}}%
\pgfpathcurveto{\pgfqpoint{2.087757in}{2.049204in}}{\pgfqpoint{2.093056in}{2.051399in}}{\pgfqpoint{2.096963in}{2.055306in}}%
\pgfpathcurveto{\pgfqpoint{2.100870in}{2.059212in}}{\pgfqpoint{2.103065in}{2.064512in}}{\pgfqpoint{2.103065in}{2.070037in}}%
\pgfpathcurveto{\pgfqpoint{2.103065in}{2.075562in}}{\pgfqpoint{2.100870in}{2.080862in}}{\pgfqpoint{2.096963in}{2.084768in}}%
\pgfpathcurveto{\pgfqpoint{2.093056in}{2.088675in}}{\pgfqpoint{2.087757in}{2.090870in}}{\pgfqpoint{2.082232in}{2.090870in}}%
\pgfpathcurveto{\pgfqpoint{2.076707in}{2.090870in}}{\pgfqpoint{2.071407in}{2.088675in}}{\pgfqpoint{2.067500in}{2.084768in}}%
\pgfpathcurveto{\pgfqpoint{2.063594in}{2.080862in}}{\pgfqpoint{2.061398in}{2.075562in}}{\pgfqpoint{2.061398in}{2.070037in}}%
\pgfpathcurveto{\pgfqpoint{2.061398in}{2.064512in}}{\pgfqpoint{2.063594in}{2.059212in}}{\pgfqpoint{2.067500in}{2.055306in}}%
\pgfpathcurveto{\pgfqpoint{2.071407in}{2.051399in}}{\pgfqpoint{2.076707in}{2.049204in}}{\pgfqpoint{2.082232in}{2.049204in}}%
\pgfpathclose%
\pgfusepath{fill}%
\end{pgfscope}%
\begin{pgfscope}%
\pgfpathrectangle{\pgfqpoint{2.051725in}{1.668832in}}{\pgfqpoint{1.162500in}{0.755000in}}%
\pgfusepath{clip}%
\pgfsetbuttcap%
\pgfsetroundjoin%
\definecolor{currentfill}{rgb}{0.000000,0.000000,0.000000}%
\pgfsetfillcolor{currentfill}%
\pgfsetfillopacity{0.500000}%
\pgfsetlinewidth{0.000000pt}%
\definecolor{currentstroke}{rgb}{0.000000,0.000000,0.000000}%
\pgfsetstrokecolor{currentstroke}%
\pgfsetdash{}{0pt}%
\pgfpathmoveto{\pgfqpoint{2.079404in}{1.665975in}}%
\pgfpathcurveto{\pgfqpoint{2.084929in}{1.665975in}}{\pgfqpoint{2.090228in}{1.668170in}}{\pgfqpoint{2.094135in}{1.672077in}}%
\pgfpathcurveto{\pgfqpoint{2.098042in}{1.675984in}}{\pgfqpoint{2.100237in}{1.681283in}}{\pgfqpoint{2.100237in}{1.686809in}}%
\pgfpathcurveto{\pgfqpoint{2.100237in}{1.692334in}}{\pgfqpoint{2.098042in}{1.697633in}}{\pgfqpoint{2.094135in}{1.701540in}}%
\pgfpathcurveto{\pgfqpoint{2.090228in}{1.705447in}}{\pgfqpoint{2.084929in}{1.707642in}}{\pgfqpoint{2.079404in}{1.707642in}}%
\pgfpathcurveto{\pgfqpoint{2.073879in}{1.707642in}}{\pgfqpoint{2.068579in}{1.705447in}}{\pgfqpoint{2.064672in}{1.701540in}}%
\pgfpathcurveto{\pgfqpoint{2.060765in}{1.697633in}}{\pgfqpoint{2.058570in}{1.692334in}}{\pgfqpoint{2.058570in}{1.686809in}}%
\pgfpathcurveto{\pgfqpoint{2.058570in}{1.681283in}}{\pgfqpoint{2.060765in}{1.675984in}}{\pgfqpoint{2.064672in}{1.672077in}}%
\pgfpathcurveto{\pgfqpoint{2.068579in}{1.668170in}}{\pgfqpoint{2.073879in}{1.665975in}}{\pgfqpoint{2.079404in}{1.665975in}}%
\pgfpathclose%
\pgfusepath{fill}%
\end{pgfscope}%
\begin{pgfscope}%
\pgfpathrectangle{\pgfqpoint{2.051725in}{1.668832in}}{\pgfqpoint{1.162500in}{0.755000in}}%
\pgfusepath{clip}%
\pgfsetbuttcap%
\pgfsetroundjoin%
\definecolor{currentfill}{rgb}{0.000000,0.000000,0.000000}%
\pgfsetfillcolor{currentfill}%
\pgfsetfillopacity{0.500000}%
\pgfsetlinewidth{0.000000pt}%
\definecolor{currentstroke}{rgb}{0.000000,0.000000,0.000000}%
\pgfsetstrokecolor{currentstroke}%
\pgfsetdash{}{0pt}%
\pgfpathmoveto{\pgfqpoint{2.536683in}{2.301682in}}%
\pgfpathcurveto{\pgfqpoint{2.542208in}{2.301682in}}{\pgfqpoint{2.547508in}{2.303877in}}{\pgfqpoint{2.551415in}{2.307784in}}%
\pgfpathcurveto{\pgfqpoint{2.555322in}{2.311691in}}{\pgfqpoint{2.557517in}{2.316990in}}{\pgfqpoint{2.557517in}{2.322515in}}%
\pgfpathcurveto{\pgfqpoint{2.557517in}{2.328041in}}{\pgfqpoint{2.555322in}{2.333340in}}{\pgfqpoint{2.551415in}{2.337247in}}%
\pgfpathcurveto{\pgfqpoint{2.547508in}{2.341154in}}{\pgfqpoint{2.542208in}{2.343349in}}{\pgfqpoint{2.536683in}{2.343349in}}%
\pgfpathcurveto{\pgfqpoint{2.531158in}{2.343349in}}{\pgfqpoint{2.525859in}{2.341154in}}{\pgfqpoint{2.521952in}{2.337247in}}%
\pgfpathcurveto{\pgfqpoint{2.518045in}{2.333340in}}{\pgfqpoint{2.515850in}{2.328041in}}{\pgfqpoint{2.515850in}{2.322515in}}%
\pgfpathcurveto{\pgfqpoint{2.515850in}{2.316990in}}{\pgfqpoint{2.518045in}{2.311691in}}{\pgfqpoint{2.521952in}{2.307784in}}%
\pgfpathcurveto{\pgfqpoint{2.525859in}{2.303877in}}{\pgfqpoint{2.531158in}{2.301682in}}{\pgfqpoint{2.536683in}{2.301682in}}%
\pgfpathclose%
\pgfusepath{fill}%
\end{pgfscope}%
\begin{pgfscope}%
\pgfpathrectangle{\pgfqpoint{2.051725in}{1.668832in}}{\pgfqpoint{1.162500in}{0.755000in}}%
\pgfusepath{clip}%
\pgfsetbuttcap%
\pgfsetroundjoin%
\definecolor{currentfill}{rgb}{0.000000,0.000000,0.000000}%
\pgfsetfillcolor{currentfill}%
\pgfsetfillopacity{0.500000}%
\pgfsetlinewidth{0.000000pt}%
\definecolor{currentstroke}{rgb}{0.000000,0.000000,0.000000}%
\pgfsetstrokecolor{currentstroke}%
\pgfsetdash{}{0pt}%
\pgfpathmoveto{\pgfqpoint{2.417381in}{2.145886in}}%
\pgfpathcurveto{\pgfqpoint{2.422906in}{2.145886in}}{\pgfqpoint{2.428205in}{2.148081in}}{\pgfqpoint{2.432112in}{2.151988in}}%
\pgfpathcurveto{\pgfqpoint{2.436019in}{2.155894in}}{\pgfqpoint{2.438214in}{2.161194in}}{\pgfqpoint{2.438214in}{2.166719in}}%
\pgfpathcurveto{\pgfqpoint{2.438214in}{2.172244in}}{\pgfqpoint{2.436019in}{2.177544in}}{\pgfqpoint{2.432112in}{2.181450in}}%
\pgfpathcurveto{\pgfqpoint{2.428205in}{2.185357in}}{\pgfqpoint{2.422906in}{2.187552in}}{\pgfqpoint{2.417381in}{2.187552in}}%
\pgfpathcurveto{\pgfqpoint{2.411856in}{2.187552in}}{\pgfqpoint{2.406556in}{2.185357in}}{\pgfqpoint{2.402649in}{2.181450in}}%
\pgfpathcurveto{\pgfqpoint{2.398743in}{2.177544in}}{\pgfqpoint{2.396547in}{2.172244in}}{\pgfqpoint{2.396547in}{2.166719in}}%
\pgfpathcurveto{\pgfqpoint{2.396547in}{2.161194in}}{\pgfqpoint{2.398743in}{2.155894in}}{\pgfqpoint{2.402649in}{2.151988in}}%
\pgfpathcurveto{\pgfqpoint{2.406556in}{2.148081in}}{\pgfqpoint{2.411856in}{2.145886in}}{\pgfqpoint{2.417381in}{2.145886in}}%
\pgfpathclose%
\pgfusepath{fill}%
\end{pgfscope}%
\begin{pgfscope}%
\pgfpathrectangle{\pgfqpoint{2.051725in}{1.668832in}}{\pgfqpoint{1.162500in}{0.755000in}}%
\pgfusepath{clip}%
\pgfsetbuttcap%
\pgfsetroundjoin%
\definecolor{currentfill}{rgb}{0.000000,0.000000,0.000000}%
\pgfsetfillcolor{currentfill}%
\pgfsetfillopacity{0.500000}%
\pgfsetlinewidth{0.000000pt}%
\definecolor{currentstroke}{rgb}{0.000000,0.000000,0.000000}%
\pgfsetstrokecolor{currentstroke}%
\pgfsetdash{}{0pt}%
\pgfpathmoveto{\pgfqpoint{2.412630in}{1.969105in}}%
\pgfpathcurveto{\pgfqpoint{2.418155in}{1.969105in}}{\pgfqpoint{2.423454in}{1.971300in}}{\pgfqpoint{2.427361in}{1.975207in}}%
\pgfpathcurveto{\pgfqpoint{2.431268in}{1.979114in}}{\pgfqpoint{2.433463in}{1.984413in}}{\pgfqpoint{2.433463in}{1.989939in}}%
\pgfpathcurveto{\pgfqpoint{2.433463in}{1.995464in}}{\pgfqpoint{2.431268in}{2.000763in}}{\pgfqpoint{2.427361in}{2.004670in}}%
\pgfpathcurveto{\pgfqpoint{2.423454in}{2.008577in}}{\pgfqpoint{2.418155in}{2.010772in}}{\pgfqpoint{2.412630in}{2.010772in}}%
\pgfpathcurveto{\pgfqpoint{2.407105in}{2.010772in}}{\pgfqpoint{2.401805in}{2.008577in}}{\pgfqpoint{2.397898in}{2.004670in}}%
\pgfpathcurveto{\pgfqpoint{2.393991in}{2.000763in}}{\pgfqpoint{2.391796in}{1.995464in}}{\pgfqpoint{2.391796in}{1.989939in}}%
\pgfpathcurveto{\pgfqpoint{2.391796in}{1.984413in}}{\pgfqpoint{2.393991in}{1.979114in}}{\pgfqpoint{2.397898in}{1.975207in}}%
\pgfpathcurveto{\pgfqpoint{2.401805in}{1.971300in}}{\pgfqpoint{2.407105in}{1.969105in}}{\pgfqpoint{2.412630in}{1.969105in}}%
\pgfpathclose%
\pgfusepath{fill}%
\end{pgfscope}%
\begin{pgfscope}%
\pgfpathrectangle{\pgfqpoint{2.051725in}{1.668832in}}{\pgfqpoint{1.162500in}{0.755000in}}%
\pgfusepath{clip}%
\pgfsetbuttcap%
\pgfsetroundjoin%
\definecolor{currentfill}{rgb}{0.000000,0.000000,0.000000}%
\pgfsetfillcolor{currentfill}%
\pgfsetfillopacity{0.500000}%
\pgfsetlinewidth{0.000000pt}%
\definecolor{currentstroke}{rgb}{0.000000,0.000000,0.000000}%
\pgfsetstrokecolor{currentstroke}%
\pgfsetdash{}{0pt}%
\pgfpathmoveto{\pgfqpoint{2.478188in}{2.239104in}}%
\pgfpathcurveto{\pgfqpoint{2.483713in}{2.239104in}}{\pgfqpoint{2.489012in}{2.241300in}}{\pgfqpoint{2.492919in}{2.245206in}}%
\pgfpathcurveto{\pgfqpoint{2.496826in}{2.249113in}}{\pgfqpoint{2.499021in}{2.254413in}}{\pgfqpoint{2.499021in}{2.259938in}}%
\pgfpathcurveto{\pgfqpoint{2.499021in}{2.265463in}}{\pgfqpoint{2.496826in}{2.270762in}}{\pgfqpoint{2.492919in}{2.274669in}}%
\pgfpathcurveto{\pgfqpoint{2.489012in}{2.278576in}}{\pgfqpoint{2.483713in}{2.280771in}}{\pgfqpoint{2.478188in}{2.280771in}}%
\pgfpathcurveto{\pgfqpoint{2.472663in}{2.280771in}}{\pgfqpoint{2.467363in}{2.278576in}}{\pgfqpoint{2.463456in}{2.274669in}}%
\pgfpathcurveto{\pgfqpoint{2.459549in}{2.270762in}}{\pgfqpoint{2.457354in}{2.265463in}}{\pgfqpoint{2.457354in}{2.259938in}}%
\pgfpathcurveto{\pgfqpoint{2.457354in}{2.254413in}}{\pgfqpoint{2.459549in}{2.249113in}}{\pgfqpoint{2.463456in}{2.245206in}}%
\pgfpathcurveto{\pgfqpoint{2.467363in}{2.241300in}}{\pgfqpoint{2.472663in}{2.239104in}}{\pgfqpoint{2.478188in}{2.239104in}}%
\pgfpathclose%
\pgfusepath{fill}%
\end{pgfscope}%
\begin{pgfscope}%
\pgfpathrectangle{\pgfqpoint{2.051725in}{1.668832in}}{\pgfqpoint{1.162500in}{0.755000in}}%
\pgfusepath{clip}%
\pgfsetbuttcap%
\pgfsetroundjoin%
\definecolor{currentfill}{rgb}{0.000000,0.000000,0.000000}%
\pgfsetfillcolor{currentfill}%
\pgfsetfillopacity{0.500000}%
\pgfsetlinewidth{0.000000pt}%
\definecolor{currentstroke}{rgb}{0.000000,0.000000,0.000000}%
\pgfsetstrokecolor{currentstroke}%
\pgfsetdash{}{0pt}%
\pgfpathmoveto{\pgfqpoint{2.172291in}{1.960666in}}%
\pgfpathcurveto{\pgfqpoint{2.177816in}{1.960666in}}{\pgfqpoint{2.183116in}{1.962861in}}{\pgfqpoint{2.187022in}{1.966768in}}%
\pgfpathcurveto{\pgfqpoint{2.190929in}{1.970675in}}{\pgfqpoint{2.193124in}{1.975974in}}{\pgfqpoint{2.193124in}{1.981499in}}%
\pgfpathcurveto{\pgfqpoint{2.193124in}{1.987024in}}{\pgfqpoint{2.190929in}{1.992324in}}{\pgfqpoint{2.187022in}{1.996231in}}%
\pgfpathcurveto{\pgfqpoint{2.183116in}{2.000138in}}{\pgfqpoint{2.177816in}{2.002333in}}{\pgfqpoint{2.172291in}{2.002333in}}%
\pgfpathcurveto{\pgfqpoint{2.166766in}{2.002333in}}{\pgfqpoint{2.161467in}{2.000138in}}{\pgfqpoint{2.157560in}{1.996231in}}%
\pgfpathcurveto{\pgfqpoint{2.153653in}{1.992324in}}{\pgfqpoint{2.151458in}{1.987024in}}{\pgfqpoint{2.151458in}{1.981499in}}%
\pgfpathcurveto{\pgfqpoint{2.151458in}{1.975974in}}{\pgfqpoint{2.153653in}{1.970675in}}{\pgfqpoint{2.157560in}{1.966768in}}%
\pgfpathcurveto{\pgfqpoint{2.161467in}{1.962861in}}{\pgfqpoint{2.166766in}{1.960666in}}{\pgfqpoint{2.172291in}{1.960666in}}%
\pgfpathclose%
\pgfusepath{fill}%
\end{pgfscope}%
\begin{pgfscope}%
\pgfpathrectangle{\pgfqpoint{2.051725in}{1.668832in}}{\pgfqpoint{1.162500in}{0.755000in}}%
\pgfusepath{clip}%
\pgfsetbuttcap%
\pgfsetroundjoin%
\definecolor{currentfill}{rgb}{0.000000,0.000000,0.000000}%
\pgfsetfillcolor{currentfill}%
\pgfsetfillopacity{0.500000}%
\pgfsetlinewidth{0.000000pt}%
\definecolor{currentstroke}{rgb}{0.000000,0.000000,0.000000}%
\pgfsetstrokecolor{currentstroke}%
\pgfsetdash{}{0pt}%
\pgfpathmoveto{\pgfqpoint{2.205718in}{1.722959in}}%
\pgfpathcurveto{\pgfqpoint{2.211243in}{1.722959in}}{\pgfqpoint{2.216543in}{1.725154in}}{\pgfqpoint{2.220450in}{1.729061in}}%
\pgfpathcurveto{\pgfqpoint{2.224357in}{1.732968in}}{\pgfqpoint{2.226552in}{1.738267in}}{\pgfqpoint{2.226552in}{1.743792in}}%
\pgfpathcurveto{\pgfqpoint{2.226552in}{1.749317in}}{\pgfqpoint{2.224357in}{1.754617in}}{\pgfqpoint{2.220450in}{1.758523in}}%
\pgfpathcurveto{\pgfqpoint{2.216543in}{1.762430in}}{\pgfqpoint{2.211243in}{1.764625in}}{\pgfqpoint{2.205718in}{1.764625in}}%
\pgfpathcurveto{\pgfqpoint{2.200193in}{1.764625in}}{\pgfqpoint{2.194894in}{1.762430in}}{\pgfqpoint{2.190987in}{1.758523in}}%
\pgfpathcurveto{\pgfqpoint{2.187080in}{1.754617in}}{\pgfqpoint{2.184885in}{1.749317in}}{\pgfqpoint{2.184885in}{1.743792in}}%
\pgfpathcurveto{\pgfqpoint{2.184885in}{1.738267in}}{\pgfqpoint{2.187080in}{1.732968in}}{\pgfqpoint{2.190987in}{1.729061in}}%
\pgfpathcurveto{\pgfqpoint{2.194894in}{1.725154in}}{\pgfqpoint{2.200193in}{1.722959in}}{\pgfqpoint{2.205718in}{1.722959in}}%
\pgfpathclose%
\pgfusepath{fill}%
\end{pgfscope}%
\begin{pgfscope}%
\pgfpathrectangle{\pgfqpoint{2.051725in}{1.668832in}}{\pgfqpoint{1.162500in}{0.755000in}}%
\pgfusepath{clip}%
\pgfsetbuttcap%
\pgfsetroundjoin%
\definecolor{currentfill}{rgb}{0.000000,0.000000,0.000000}%
\pgfsetfillcolor{currentfill}%
\pgfsetfillopacity{0.500000}%
\pgfsetlinewidth{0.000000pt}%
\definecolor{currentstroke}{rgb}{0.000000,0.000000,0.000000}%
\pgfsetstrokecolor{currentstroke}%
\pgfsetdash{}{0pt}%
\pgfpathmoveto{\pgfqpoint{3.186546in}{2.267350in}}%
\pgfpathcurveto{\pgfqpoint{3.192071in}{2.267350in}}{\pgfqpoint{3.197371in}{2.269545in}}{\pgfqpoint{3.201278in}{2.273452in}}%
\pgfpathcurveto{\pgfqpoint{3.205185in}{2.277359in}}{\pgfqpoint{3.207380in}{2.282659in}}{\pgfqpoint{3.207380in}{2.288184in}}%
\pgfpathcurveto{\pgfqpoint{3.207380in}{2.293709in}}{\pgfqpoint{3.205185in}{2.299008in}}{\pgfqpoint{3.201278in}{2.302915in}}%
\pgfpathcurveto{\pgfqpoint{3.197371in}{2.306822in}}{\pgfqpoint{3.192071in}{2.309017in}}{\pgfqpoint{3.186546in}{2.309017in}}%
\pgfpathcurveto{\pgfqpoint{3.181021in}{2.309017in}}{\pgfqpoint{3.175722in}{2.306822in}}{\pgfqpoint{3.171815in}{2.302915in}}%
\pgfpathcurveto{\pgfqpoint{3.167908in}{2.299008in}}{\pgfqpoint{3.165713in}{2.293709in}}{\pgfqpoint{3.165713in}{2.288184in}}%
\pgfpathcurveto{\pgfqpoint{3.165713in}{2.282659in}}{\pgfqpoint{3.167908in}{2.277359in}}{\pgfqpoint{3.171815in}{2.273452in}}%
\pgfpathcurveto{\pgfqpoint{3.175722in}{2.269545in}}{\pgfqpoint{3.181021in}{2.267350in}}{\pgfqpoint{3.186546in}{2.267350in}}%
\pgfpathclose%
\pgfusepath{fill}%
\end{pgfscope}%
\begin{pgfscope}%
\pgfpathrectangle{\pgfqpoint{2.051725in}{1.668832in}}{\pgfqpoint{1.162500in}{0.755000in}}%
\pgfusepath{clip}%
\pgfsetbuttcap%
\pgfsetroundjoin%
\definecolor{currentfill}{rgb}{0.000000,0.000000,0.000000}%
\pgfsetfillcolor{currentfill}%
\pgfsetfillopacity{0.500000}%
\pgfsetlinewidth{0.000000pt}%
\definecolor{currentstroke}{rgb}{0.000000,0.000000,0.000000}%
\pgfsetstrokecolor{currentstroke}%
\pgfsetdash{}{0pt}%
\pgfpathmoveto{\pgfqpoint{2.394110in}{2.236796in}}%
\pgfpathcurveto{\pgfqpoint{2.399635in}{2.236796in}}{\pgfqpoint{2.404935in}{2.238991in}}{\pgfqpoint{2.408842in}{2.242898in}}%
\pgfpathcurveto{\pgfqpoint{2.412749in}{2.246805in}}{\pgfqpoint{2.414944in}{2.252104in}}{\pgfqpoint{2.414944in}{2.257629in}}%
\pgfpathcurveto{\pgfqpoint{2.414944in}{2.263154in}}{\pgfqpoint{2.412749in}{2.268454in}}{\pgfqpoint{2.408842in}{2.272361in}}%
\pgfpathcurveto{\pgfqpoint{2.404935in}{2.276267in}}{\pgfqpoint{2.399635in}{2.278463in}}{\pgfqpoint{2.394110in}{2.278463in}}%
\pgfpathcurveto{\pgfqpoint{2.388585in}{2.278463in}}{\pgfqpoint{2.383286in}{2.276267in}}{\pgfqpoint{2.379379in}{2.272361in}}%
\pgfpathcurveto{\pgfqpoint{2.375472in}{2.268454in}}{\pgfqpoint{2.373277in}{2.263154in}}{\pgfqpoint{2.373277in}{2.257629in}}%
\pgfpathcurveto{\pgfqpoint{2.373277in}{2.252104in}}{\pgfqpoint{2.375472in}{2.246805in}}{\pgfqpoint{2.379379in}{2.242898in}}%
\pgfpathcurveto{\pgfqpoint{2.383286in}{2.238991in}}{\pgfqpoint{2.388585in}{2.236796in}}{\pgfqpoint{2.394110in}{2.236796in}}%
\pgfpathclose%
\pgfusepath{fill}%
\end{pgfscope}%
\begin{pgfscope}%
\pgfpathrectangle{\pgfqpoint{2.051725in}{1.668832in}}{\pgfqpoint{1.162500in}{0.755000in}}%
\pgfusepath{clip}%
\pgfsetbuttcap%
\pgfsetroundjoin%
\definecolor{currentfill}{rgb}{0.000000,0.000000,0.000000}%
\pgfsetfillcolor{currentfill}%
\pgfsetfillopacity{0.500000}%
\pgfsetlinewidth{0.000000pt}%
\definecolor{currentstroke}{rgb}{0.000000,0.000000,0.000000}%
\pgfsetstrokecolor{currentstroke}%
\pgfsetdash{}{0pt}%
\pgfpathmoveto{\pgfqpoint{2.125339in}{2.023720in}}%
\pgfpathcurveto{\pgfqpoint{2.130865in}{2.023720in}}{\pgfqpoint{2.136164in}{2.025915in}}{\pgfqpoint{2.140071in}{2.029822in}}%
\pgfpathcurveto{\pgfqpoint{2.143978in}{2.033729in}}{\pgfqpoint{2.146173in}{2.039028in}}{\pgfqpoint{2.146173in}{2.044553in}}%
\pgfpathcurveto{\pgfqpoint{2.146173in}{2.050078in}}{\pgfqpoint{2.143978in}{2.055378in}}{\pgfqpoint{2.140071in}{2.059285in}}%
\pgfpathcurveto{\pgfqpoint{2.136164in}{2.063191in}}{\pgfqpoint{2.130865in}{2.065387in}}{\pgfqpoint{2.125339in}{2.065387in}}%
\pgfpathcurveto{\pgfqpoint{2.119814in}{2.065387in}}{\pgfqpoint{2.114515in}{2.063191in}}{\pgfqpoint{2.110608in}{2.059285in}}%
\pgfpathcurveto{\pgfqpoint{2.106701in}{2.055378in}}{\pgfqpoint{2.104506in}{2.050078in}}{\pgfqpoint{2.104506in}{2.044553in}}%
\pgfpathcurveto{\pgfqpoint{2.104506in}{2.039028in}}{\pgfqpoint{2.106701in}{2.033729in}}{\pgfqpoint{2.110608in}{2.029822in}}%
\pgfpathcurveto{\pgfqpoint{2.114515in}{2.025915in}}{\pgfqpoint{2.119814in}{2.023720in}}{\pgfqpoint{2.125339in}{2.023720in}}%
\pgfpathclose%
\pgfusepath{fill}%
\end{pgfscope}%
\begin{pgfscope}%
\pgfsetrectcap%
\pgfsetmiterjoin%
\pgfsetlinewidth{0.803000pt}%
\definecolor{currentstroke}{rgb}{0.501961,0.501961,0.501961}%
\pgfsetstrokecolor{currentstroke}%
\pgfsetdash{}{0pt}%
\pgfpathmoveto{\pgfqpoint{2.051725in}{1.668832in}}%
\pgfpathlineto{\pgfqpoint{2.051725in}{2.423832in}}%
\pgfusepath{stroke}%
\end{pgfscope}%
\begin{pgfscope}%
\pgfsetrectcap%
\pgfsetmiterjoin%
\pgfsetlinewidth{0.803000pt}%
\definecolor{currentstroke}{rgb}{0.501961,0.501961,0.501961}%
\pgfsetstrokecolor{currentstroke}%
\pgfsetdash{}{0pt}%
\pgfpathmoveto{\pgfqpoint{3.214225in}{1.668832in}}%
\pgfpathlineto{\pgfqpoint{3.214225in}{2.423832in}}%
\pgfusepath{stroke}%
\end{pgfscope}%
\begin{pgfscope}%
\pgfsetrectcap%
\pgfsetmiterjoin%
\pgfsetlinewidth{0.803000pt}%
\definecolor{currentstroke}{rgb}{0.501961,0.501961,0.501961}%
\pgfsetstrokecolor{currentstroke}%
\pgfsetdash{}{0pt}%
\pgfpathmoveto{\pgfqpoint{2.051725in}{1.668832in}}%
\pgfpathlineto{\pgfqpoint{3.214225in}{1.668832in}}%
\pgfusepath{stroke}%
\end{pgfscope}%
\begin{pgfscope}%
\pgfsetrectcap%
\pgfsetmiterjoin%
\pgfsetlinewidth{0.803000pt}%
\definecolor{currentstroke}{rgb}{0.501961,0.501961,0.501961}%
\pgfsetstrokecolor{currentstroke}%
\pgfsetdash{}{0pt}%
\pgfpathmoveto{\pgfqpoint{2.051725in}{2.423832in}}%
\pgfpathlineto{\pgfqpoint{3.214225in}{2.423832in}}%
\pgfusepath{stroke}%
\end{pgfscope}%
\begin{pgfscope}%
\pgfsetbuttcap%
\pgfsetmiterjoin%
\definecolor{currentfill}{rgb}{1.000000,1.000000,1.000000}%
\pgfsetfillcolor{currentfill}%
\pgfsetlinewidth{0.000000pt}%
\definecolor{currentstroke}{rgb}{0.000000,0.000000,0.000000}%
\pgfsetstrokecolor{currentstroke}%
\pgfsetstrokeopacity{0.000000}%
\pgfsetdash{}{0pt}%
\pgfpathmoveto{\pgfqpoint{3.214225in}{1.668832in}}%
\pgfpathlineto{\pgfqpoint{4.376725in}{1.668832in}}%
\pgfpathlineto{\pgfqpoint{4.376725in}{2.423832in}}%
\pgfpathlineto{\pgfqpoint{3.214225in}{2.423832in}}%
\pgfpathclose%
\pgfusepath{fill}%
\end{pgfscope}%
\begin{pgfscope}%
\pgfpathrectangle{\pgfqpoint{3.214225in}{1.668832in}}{\pgfqpoint{1.162500in}{0.755000in}}%
\pgfusepath{clip}%
\pgfsetrectcap%
\pgfsetroundjoin%
\pgfsetlinewidth{1.505625pt}%
\definecolor{currentstroke}{rgb}{0.121569,0.466667,0.705882}%
\pgfsetstrokecolor{currentstroke}%
\pgfsetdash{}{0pt}%
\pgfpathmoveto{\pgfqpoint{3.241904in}{1.703151in}}%
\pgfpathlineto{\pgfqpoint{3.274043in}{1.719497in}}%
\pgfpathlineto{\pgfqpoint{3.312832in}{1.736554in}}%
\pgfpathlineto{\pgfqpoint{3.382651in}{1.766730in}}%
\pgfpathlineto{\pgfqpoint{3.409249in}{1.781089in}}%
\pgfpathlineto{\pgfqpoint{3.432523in}{1.796220in}}%
\pgfpathlineto{\pgfqpoint{3.454688in}{1.813410in}}%
\pgfpathlineto{\pgfqpoint{3.475745in}{1.832588in}}%
\pgfpathlineto{\pgfqpoint{3.496801in}{1.854715in}}%
\pgfpathlineto{\pgfqpoint{3.518966in}{1.881209in}}%
\pgfpathlineto{\pgfqpoint{3.543348in}{1.913943in}}%
\pgfpathlineto{\pgfqpoint{3.571054in}{1.955094in}}%
\pgfpathlineto{\pgfqpoint{3.605410in}{2.010375in}}%
\pgfpathlineto{\pgfqpoint{3.705153in}{2.173572in}}%
\pgfpathlineto{\pgfqpoint{3.733967in}{2.215073in}}%
\pgfpathlineto{\pgfqpoint{3.759457in}{2.248062in}}%
\pgfpathlineto{\pgfqpoint{3.783838in}{2.276027in}}%
\pgfpathlineto{\pgfqpoint{3.807112in}{2.299374in}}%
\pgfpathlineto{\pgfqpoint{3.830385in}{2.319532in}}%
\pgfpathlineto{\pgfqpoint{3.853658in}{2.336684in}}%
\pgfpathlineto{\pgfqpoint{3.876931in}{2.351073in}}%
\pgfpathlineto{\pgfqpoint{3.901313in}{2.363461in}}%
\pgfpathlineto{\pgfqpoint{3.925695in}{2.373347in}}%
\pgfpathlineto{\pgfqpoint{3.950076in}{2.380899in}}%
\pgfpathlineto{\pgfqpoint{3.974458in}{2.386158in}}%
\pgfpathlineto{\pgfqpoint{3.997731in}{2.388954in}}%
\pgfpathlineto{\pgfqpoint{4.019896in}{2.389411in}}%
\pgfpathlineto{\pgfqpoint{4.040953in}{2.387609in}}%
\pgfpathlineto{\pgfqpoint{4.060901in}{2.383634in}}%
\pgfpathlineto{\pgfqpoint{4.079741in}{2.377611in}}%
\pgfpathlineto{\pgfqpoint{4.098582in}{2.369147in}}%
\pgfpathlineto{\pgfqpoint{4.116314in}{2.358752in}}%
\pgfpathlineto{\pgfqpoint{4.134046in}{2.345834in}}%
\pgfpathlineto{\pgfqpoint{4.151778in}{2.330258in}}%
\pgfpathlineto{\pgfqpoint{4.170618in}{2.310693in}}%
\pgfpathlineto{\pgfqpoint{4.190567in}{2.286536in}}%
\pgfpathlineto{\pgfqpoint{4.210515in}{2.258869in}}%
\pgfpathlineto{\pgfqpoint{4.232680in}{2.224164in}}%
\pgfpathlineto{\pgfqpoint{4.257062in}{2.181540in}}%
\pgfpathlineto{\pgfqpoint{4.283660in}{2.130435in}}%
\pgfpathlineto{\pgfqpoint{4.316907in}{2.061369in}}%
\pgfpathlineto{\pgfqpoint{4.349046in}{1.991285in}}%
\pgfpathlineto{\pgfqpoint{4.349046in}{1.991285in}}%
\pgfusepath{stroke}%
\end{pgfscope}%
\begin{pgfscope}%
\pgfsetrectcap%
\pgfsetmiterjoin%
\pgfsetlinewidth{0.803000pt}%
\definecolor{currentstroke}{rgb}{0.501961,0.501961,0.501961}%
\pgfsetstrokecolor{currentstroke}%
\pgfsetdash{}{0pt}%
\pgfpathmoveto{\pgfqpoint{3.214225in}{1.668832in}}%
\pgfpathlineto{\pgfqpoint{3.214225in}{2.423832in}}%
\pgfusepath{stroke}%
\end{pgfscope}%
\begin{pgfscope}%
\pgfsetrectcap%
\pgfsetmiterjoin%
\pgfsetlinewidth{0.803000pt}%
\definecolor{currentstroke}{rgb}{0.501961,0.501961,0.501961}%
\pgfsetstrokecolor{currentstroke}%
\pgfsetdash{}{0pt}%
\pgfpathmoveto{\pgfqpoint{4.376725in}{1.668832in}}%
\pgfpathlineto{\pgfqpoint{4.376725in}{2.423832in}}%
\pgfusepath{stroke}%
\end{pgfscope}%
\begin{pgfscope}%
\pgfsetrectcap%
\pgfsetmiterjoin%
\pgfsetlinewidth{0.803000pt}%
\definecolor{currentstroke}{rgb}{0.501961,0.501961,0.501961}%
\pgfsetstrokecolor{currentstroke}%
\pgfsetdash{}{0pt}%
\pgfpathmoveto{\pgfqpoint{3.214225in}{1.668832in}}%
\pgfpathlineto{\pgfqpoint{4.376725in}{1.668832in}}%
\pgfusepath{stroke}%
\end{pgfscope}%
\begin{pgfscope}%
\pgfsetrectcap%
\pgfsetmiterjoin%
\pgfsetlinewidth{0.803000pt}%
\definecolor{currentstroke}{rgb}{0.501961,0.501961,0.501961}%
\pgfsetstrokecolor{currentstroke}%
\pgfsetdash{}{0pt}%
\pgfpathmoveto{\pgfqpoint{3.214225in}{2.423832in}}%
\pgfpathlineto{\pgfqpoint{4.376725in}{2.423832in}}%
\pgfusepath{stroke}%
\end{pgfscope}%
\begin{pgfscope}%
\pgfsetbuttcap%
\pgfsetmiterjoin%
\definecolor{currentfill}{rgb}{1.000000,1.000000,1.000000}%
\pgfsetfillcolor{currentfill}%
\pgfsetlinewidth{0.000000pt}%
\definecolor{currentstroke}{rgb}{0.000000,0.000000,0.000000}%
\pgfsetstrokecolor{currentstroke}%
\pgfsetstrokeopacity{0.000000}%
\pgfsetdash{}{0pt}%
\pgfpathmoveto{\pgfqpoint{4.376725in}{1.668832in}}%
\pgfpathlineto{\pgfqpoint{5.539225in}{1.668832in}}%
\pgfpathlineto{\pgfqpoint{5.539225in}{2.423832in}}%
\pgfpathlineto{\pgfqpoint{4.376725in}{2.423832in}}%
\pgfpathclose%
\pgfusepath{fill}%
\end{pgfscope}%
\begin{pgfscope}%
\pgfpathrectangle{\pgfqpoint{4.376725in}{1.668832in}}{\pgfqpoint{1.162500in}{0.755000in}}%
\pgfusepath{clip}%
\pgfsetbuttcap%
\pgfsetroundjoin%
\definecolor{currentfill}{rgb}{0.000000,0.000000,0.000000}%
\pgfsetfillcolor{currentfill}%
\pgfsetfillopacity{0.500000}%
\pgfsetlinewidth{0.000000pt}%
\definecolor{currentstroke}{rgb}{0.000000,0.000000,0.000000}%
\pgfsetstrokecolor{currentstroke}%
\pgfsetdash{}{0pt}%
\pgfpathmoveto{\pgfqpoint{5.214960in}{1.935152in}}%
\pgfpathcurveto{\pgfqpoint{5.220485in}{1.935152in}}{\pgfqpoint{5.225785in}{1.937347in}}{\pgfqpoint{5.229692in}{1.941254in}}%
\pgfpathcurveto{\pgfqpoint{5.233598in}{1.945161in}}{\pgfqpoint{5.235794in}{1.950460in}}{\pgfqpoint{5.235794in}{1.955985in}}%
\pgfpathcurveto{\pgfqpoint{5.235794in}{1.961510in}}{\pgfqpoint{5.233598in}{1.966810in}}{\pgfqpoint{5.229692in}{1.970717in}}%
\pgfpathcurveto{\pgfqpoint{5.225785in}{1.974623in}}{\pgfqpoint{5.220485in}{1.976819in}}{\pgfqpoint{5.214960in}{1.976819in}}%
\pgfpathcurveto{\pgfqpoint{5.209435in}{1.976819in}}{\pgfqpoint{5.204136in}{1.974623in}}{\pgfqpoint{5.200229in}{1.970717in}}%
\pgfpathcurveto{\pgfqpoint{5.196322in}{1.966810in}}{\pgfqpoint{5.194127in}{1.961510in}}{\pgfqpoint{5.194127in}{1.955985in}}%
\pgfpathcurveto{\pgfqpoint{5.194127in}{1.950460in}}{\pgfqpoint{5.196322in}{1.945161in}}{\pgfqpoint{5.200229in}{1.941254in}}%
\pgfpathcurveto{\pgfqpoint{5.204136in}{1.937347in}}{\pgfqpoint{5.209435in}{1.935152in}}{\pgfqpoint{5.214960in}{1.935152in}}%
\pgfpathclose%
\pgfusepath{fill}%
\end{pgfscope}%
\begin{pgfscope}%
\pgfpathrectangle{\pgfqpoint{4.376725in}{1.668832in}}{\pgfqpoint{1.162500in}{0.755000in}}%
\pgfusepath{clip}%
\pgfsetbuttcap%
\pgfsetroundjoin%
\definecolor{currentfill}{rgb}{0.000000,0.000000,0.000000}%
\pgfsetfillcolor{currentfill}%
\pgfsetfillopacity{0.500000}%
\pgfsetlinewidth{0.000000pt}%
\definecolor{currentstroke}{rgb}{0.000000,0.000000,0.000000}%
\pgfsetstrokecolor{currentstroke}%
\pgfsetdash{}{0pt}%
\pgfpathmoveto{\pgfqpoint{4.521739in}{2.326950in}}%
\pgfpathcurveto{\pgfqpoint{4.527264in}{2.326950in}}{\pgfqpoint{4.532563in}{2.329145in}}{\pgfqpoint{4.536470in}{2.333052in}}%
\pgfpathcurveto{\pgfqpoint{4.540377in}{2.336959in}}{\pgfqpoint{4.542572in}{2.342258in}}{\pgfqpoint{4.542572in}{2.347783in}}%
\pgfpathcurveto{\pgfqpoint{4.542572in}{2.353308in}}{\pgfqpoint{4.540377in}{2.358608in}}{\pgfqpoint{4.536470in}{2.362515in}}%
\pgfpathcurveto{\pgfqpoint{4.532563in}{2.366421in}}{\pgfqpoint{4.527264in}{2.368616in}}{\pgfqpoint{4.521739in}{2.368616in}}%
\pgfpathcurveto{\pgfqpoint{4.516214in}{2.368616in}}{\pgfqpoint{4.510914in}{2.366421in}}{\pgfqpoint{4.507007in}{2.362515in}}%
\pgfpathcurveto{\pgfqpoint{4.503101in}{2.358608in}}{\pgfqpoint{4.500905in}{2.353308in}}{\pgfqpoint{4.500905in}{2.347783in}}%
\pgfpathcurveto{\pgfqpoint{4.500905in}{2.342258in}}{\pgfqpoint{4.503101in}{2.336959in}}{\pgfqpoint{4.507007in}{2.333052in}}%
\pgfpathcurveto{\pgfqpoint{4.510914in}{2.329145in}}{\pgfqpoint{4.516214in}{2.326950in}}{\pgfqpoint{4.521739in}{2.326950in}}%
\pgfpathclose%
\pgfusepath{fill}%
\end{pgfscope}%
\begin{pgfscope}%
\pgfpathrectangle{\pgfqpoint{4.376725in}{1.668832in}}{\pgfqpoint{1.162500in}{0.755000in}}%
\pgfusepath{clip}%
\pgfsetbuttcap%
\pgfsetroundjoin%
\definecolor{currentfill}{rgb}{0.000000,0.000000,0.000000}%
\pgfsetfillcolor{currentfill}%
\pgfsetfillopacity{0.500000}%
\pgfsetlinewidth{0.000000pt}%
\definecolor{currentstroke}{rgb}{0.000000,0.000000,0.000000}%
\pgfsetstrokecolor{currentstroke}%
\pgfsetdash{}{0pt}%
\pgfpathmoveto{\pgfqpoint{4.448919in}{2.385023in}}%
\pgfpathcurveto{\pgfqpoint{4.454444in}{2.385023in}}{\pgfqpoint{4.459744in}{2.387218in}}{\pgfqpoint{4.463650in}{2.391125in}}%
\pgfpathcurveto{\pgfqpoint{4.467557in}{2.395032in}}{\pgfqpoint{4.469752in}{2.400331in}}{\pgfqpoint{4.469752in}{2.405856in}}%
\pgfpathcurveto{\pgfqpoint{4.469752in}{2.411381in}}{\pgfqpoint{4.467557in}{2.416681in}}{\pgfqpoint{4.463650in}{2.420588in}}%
\pgfpathcurveto{\pgfqpoint{4.459744in}{2.424494in}}{\pgfqpoint{4.454444in}{2.426690in}}{\pgfqpoint{4.448919in}{2.426690in}}%
\pgfpathcurveto{\pgfqpoint{4.443394in}{2.426690in}}{\pgfqpoint{4.438094in}{2.424494in}}{\pgfqpoint{4.434188in}{2.420588in}}%
\pgfpathcurveto{\pgfqpoint{4.430281in}{2.416681in}}{\pgfqpoint{4.428086in}{2.411381in}}{\pgfqpoint{4.428086in}{2.405856in}}%
\pgfpathcurveto{\pgfqpoint{4.428086in}{2.400331in}}{\pgfqpoint{4.430281in}{2.395032in}}{\pgfqpoint{4.434188in}{2.391125in}}%
\pgfpathcurveto{\pgfqpoint{4.438094in}{2.387218in}}{\pgfqpoint{4.443394in}{2.385023in}}{\pgfqpoint{4.448919in}{2.385023in}}%
\pgfpathclose%
\pgfusepath{fill}%
\end{pgfscope}%
\begin{pgfscope}%
\pgfpathrectangle{\pgfqpoint{4.376725in}{1.668832in}}{\pgfqpoint{1.162500in}{0.755000in}}%
\pgfusepath{clip}%
\pgfsetbuttcap%
\pgfsetroundjoin%
\definecolor{currentfill}{rgb}{0.000000,0.000000,0.000000}%
\pgfsetfillcolor{currentfill}%
\pgfsetfillopacity{0.500000}%
\pgfsetlinewidth{0.000000pt}%
\definecolor{currentstroke}{rgb}{0.000000,0.000000,0.000000}%
\pgfsetstrokecolor{currentstroke}%
\pgfsetdash{}{0pt}%
\pgfpathmoveto{\pgfqpoint{4.704251in}{2.103661in}}%
\pgfpathcurveto{\pgfqpoint{4.709776in}{2.103661in}}{\pgfqpoint{4.715075in}{2.105856in}}{\pgfqpoint{4.718982in}{2.109762in}}%
\pgfpathcurveto{\pgfqpoint{4.722889in}{2.113669in}}{\pgfqpoint{4.725084in}{2.118969in}}{\pgfqpoint{4.725084in}{2.124494in}}%
\pgfpathcurveto{\pgfqpoint{4.725084in}{2.130019in}}{\pgfqpoint{4.722889in}{2.135318in}}{\pgfqpoint{4.718982in}{2.139225in}}%
\pgfpathcurveto{\pgfqpoint{4.715075in}{2.143132in}}{\pgfqpoint{4.709776in}{2.145327in}}{\pgfqpoint{4.704251in}{2.145327in}}%
\pgfpathcurveto{\pgfqpoint{4.698726in}{2.145327in}}{\pgfqpoint{4.693426in}{2.143132in}}{\pgfqpoint{4.689519in}{2.139225in}}%
\pgfpathcurveto{\pgfqpoint{4.685612in}{2.135318in}}{\pgfqpoint{4.683417in}{2.130019in}}{\pgfqpoint{4.683417in}{2.124494in}}%
\pgfpathcurveto{\pgfqpoint{4.683417in}{2.118969in}}{\pgfqpoint{4.685612in}{2.113669in}}{\pgfqpoint{4.689519in}{2.109762in}}%
\pgfpathcurveto{\pgfqpoint{4.693426in}{2.105856in}}{\pgfqpoint{4.698726in}{2.103661in}}{\pgfqpoint{4.704251in}{2.103661in}}%
\pgfpathclose%
\pgfusepath{fill}%
\end{pgfscope}%
\begin{pgfscope}%
\pgfpathrectangle{\pgfqpoint{4.376725in}{1.668832in}}{\pgfqpoint{1.162500in}{0.755000in}}%
\pgfusepath{clip}%
\pgfsetbuttcap%
\pgfsetroundjoin%
\definecolor{currentfill}{rgb}{0.000000,0.000000,0.000000}%
\pgfsetfillcolor{currentfill}%
\pgfsetfillopacity{0.500000}%
\pgfsetlinewidth{0.000000pt}%
\definecolor{currentstroke}{rgb}{0.000000,0.000000,0.000000}%
\pgfsetstrokecolor{currentstroke}%
\pgfsetdash{}{0pt}%
\pgfpathmoveto{\pgfqpoint{4.404404in}{2.149550in}}%
\pgfpathcurveto{\pgfqpoint{4.409929in}{2.149550in}}{\pgfqpoint{4.415228in}{2.151746in}}{\pgfqpoint{4.419135in}{2.155652in}}%
\pgfpathcurveto{\pgfqpoint{4.423042in}{2.159559in}}{\pgfqpoint{4.425237in}{2.164859in}}{\pgfqpoint{4.425237in}{2.170384in}}%
\pgfpathcurveto{\pgfqpoint{4.425237in}{2.175909in}}{\pgfqpoint{4.423042in}{2.181208in}}{\pgfqpoint{4.419135in}{2.185115in}}%
\pgfpathcurveto{\pgfqpoint{4.415228in}{2.189022in}}{\pgfqpoint{4.409929in}{2.191217in}}{\pgfqpoint{4.404404in}{2.191217in}}%
\pgfpathcurveto{\pgfqpoint{4.398879in}{2.191217in}}{\pgfqpoint{4.393579in}{2.189022in}}{\pgfqpoint{4.389672in}{2.185115in}}%
\pgfpathcurveto{\pgfqpoint{4.385765in}{2.181208in}}{\pgfqpoint{4.383570in}{2.175909in}}{\pgfqpoint{4.383570in}{2.170384in}}%
\pgfpathcurveto{\pgfqpoint{4.383570in}{2.164859in}}{\pgfqpoint{4.385765in}{2.159559in}}{\pgfqpoint{4.389672in}{2.155652in}}%
\pgfpathcurveto{\pgfqpoint{4.393579in}{2.151746in}}{\pgfqpoint{4.398879in}{2.149550in}}{\pgfqpoint{4.404404in}{2.149550in}}%
\pgfpathclose%
\pgfusepath{fill}%
\end{pgfscope}%
\begin{pgfscope}%
\pgfpathrectangle{\pgfqpoint{4.376725in}{1.668832in}}{\pgfqpoint{1.162500in}{0.755000in}}%
\pgfusepath{clip}%
\pgfsetbuttcap%
\pgfsetroundjoin%
\definecolor{currentfill}{rgb}{0.000000,0.000000,0.000000}%
\pgfsetfillcolor{currentfill}%
\pgfsetfillopacity{0.500000}%
\pgfsetlinewidth{0.000000pt}%
\definecolor{currentstroke}{rgb}{0.000000,0.000000,0.000000}%
\pgfsetstrokecolor{currentstroke}%
\pgfsetdash{}{0pt}%
\pgfpathmoveto{\pgfqpoint{4.607580in}{2.049204in}}%
\pgfpathcurveto{\pgfqpoint{4.613105in}{2.049204in}}{\pgfqpoint{4.618405in}{2.051399in}}{\pgfqpoint{4.622312in}{2.055306in}}%
\pgfpathcurveto{\pgfqpoint{4.626218in}{2.059212in}}{\pgfqpoint{4.628414in}{2.064512in}}{\pgfqpoint{4.628414in}{2.070037in}}%
\pgfpathcurveto{\pgfqpoint{4.628414in}{2.075562in}}{\pgfqpoint{4.626218in}{2.080862in}}{\pgfqpoint{4.622312in}{2.084768in}}%
\pgfpathcurveto{\pgfqpoint{4.618405in}{2.088675in}}{\pgfqpoint{4.613105in}{2.090870in}}{\pgfqpoint{4.607580in}{2.090870in}}%
\pgfpathcurveto{\pgfqpoint{4.602055in}{2.090870in}}{\pgfqpoint{4.596756in}{2.088675in}}{\pgfqpoint{4.592849in}{2.084768in}}%
\pgfpathcurveto{\pgfqpoint{4.588942in}{2.080862in}}{\pgfqpoint{4.586747in}{2.075562in}}{\pgfqpoint{4.586747in}{2.070037in}}%
\pgfpathcurveto{\pgfqpoint{4.586747in}{2.064512in}}{\pgfqpoint{4.588942in}{2.059212in}}{\pgfqpoint{4.592849in}{2.055306in}}%
\pgfpathcurveto{\pgfqpoint{4.596756in}{2.051399in}}{\pgfqpoint{4.602055in}{2.049204in}}{\pgfqpoint{4.607580in}{2.049204in}}%
\pgfpathclose%
\pgfusepath{fill}%
\end{pgfscope}%
\begin{pgfscope}%
\pgfpathrectangle{\pgfqpoint{4.376725in}{1.668832in}}{\pgfqpoint{1.162500in}{0.755000in}}%
\pgfusepath{clip}%
\pgfsetbuttcap%
\pgfsetroundjoin%
\definecolor{currentfill}{rgb}{0.000000,0.000000,0.000000}%
\pgfsetfillcolor{currentfill}%
\pgfsetfillopacity{0.500000}%
\pgfsetlinewidth{0.000000pt}%
\definecolor{currentstroke}{rgb}{0.000000,0.000000,0.000000}%
\pgfsetstrokecolor{currentstroke}%
\pgfsetdash{}{0pt}%
\pgfpathmoveto{\pgfqpoint{4.596304in}{1.665975in}}%
\pgfpathcurveto{\pgfqpoint{4.601829in}{1.665975in}}{\pgfqpoint{4.607129in}{1.668170in}}{\pgfqpoint{4.611036in}{1.672077in}}%
\pgfpathcurveto{\pgfqpoint{4.614942in}{1.675984in}}{\pgfqpoint{4.617137in}{1.681283in}}{\pgfqpoint{4.617137in}{1.686809in}}%
\pgfpathcurveto{\pgfqpoint{4.617137in}{1.692334in}}{\pgfqpoint{4.614942in}{1.697633in}}{\pgfqpoint{4.611036in}{1.701540in}}%
\pgfpathcurveto{\pgfqpoint{4.607129in}{1.705447in}}{\pgfqpoint{4.601829in}{1.707642in}}{\pgfqpoint{4.596304in}{1.707642in}}%
\pgfpathcurveto{\pgfqpoint{4.590779in}{1.707642in}}{\pgfqpoint{4.585480in}{1.705447in}}{\pgfqpoint{4.581573in}{1.701540in}}%
\pgfpathcurveto{\pgfqpoint{4.577666in}{1.697633in}}{\pgfqpoint{4.575471in}{1.692334in}}{\pgfqpoint{4.575471in}{1.686809in}}%
\pgfpathcurveto{\pgfqpoint{4.575471in}{1.681283in}}{\pgfqpoint{4.577666in}{1.675984in}}{\pgfqpoint{4.581573in}{1.672077in}}%
\pgfpathcurveto{\pgfqpoint{4.585480in}{1.668170in}}{\pgfqpoint{4.590779in}{1.665975in}}{\pgfqpoint{4.596304in}{1.665975in}}%
\pgfpathclose%
\pgfusepath{fill}%
\end{pgfscope}%
\begin{pgfscope}%
\pgfpathrectangle{\pgfqpoint{4.376725in}{1.668832in}}{\pgfqpoint{1.162500in}{0.755000in}}%
\pgfusepath{clip}%
\pgfsetbuttcap%
\pgfsetroundjoin%
\definecolor{currentfill}{rgb}{0.000000,0.000000,0.000000}%
\pgfsetfillcolor{currentfill}%
\pgfsetfillopacity{0.500000}%
\pgfsetlinewidth{0.000000pt}%
\definecolor{currentstroke}{rgb}{0.000000,0.000000,0.000000}%
\pgfsetstrokecolor{currentstroke}%
\pgfsetdash{}{0pt}%
\pgfpathmoveto{\pgfqpoint{5.423123in}{2.301682in}}%
\pgfpathcurveto{\pgfqpoint{5.428649in}{2.301682in}}{\pgfqpoint{5.433948in}{2.303877in}}{\pgfqpoint{5.437855in}{2.307784in}}%
\pgfpathcurveto{\pgfqpoint{5.441762in}{2.311691in}}{\pgfqpoint{5.443957in}{2.316990in}}{\pgfqpoint{5.443957in}{2.322515in}}%
\pgfpathcurveto{\pgfqpoint{5.443957in}{2.328041in}}{\pgfqpoint{5.441762in}{2.333340in}}{\pgfqpoint{5.437855in}{2.337247in}}%
\pgfpathcurveto{\pgfqpoint{5.433948in}{2.341154in}}{\pgfqpoint{5.428649in}{2.343349in}}{\pgfqpoint{5.423123in}{2.343349in}}%
\pgfpathcurveto{\pgfqpoint{5.417598in}{2.343349in}}{\pgfqpoint{5.412299in}{2.341154in}}{\pgfqpoint{5.408392in}{2.337247in}}%
\pgfpathcurveto{\pgfqpoint{5.404485in}{2.333340in}}{\pgfqpoint{5.402290in}{2.328041in}}{\pgfqpoint{5.402290in}{2.322515in}}%
\pgfpathcurveto{\pgfqpoint{5.402290in}{2.316990in}}{\pgfqpoint{5.404485in}{2.311691in}}{\pgfqpoint{5.408392in}{2.307784in}}%
\pgfpathcurveto{\pgfqpoint{5.412299in}{2.303877in}}{\pgfqpoint{5.417598in}{2.301682in}}{\pgfqpoint{5.423123in}{2.301682in}}%
\pgfpathclose%
\pgfusepath{fill}%
\end{pgfscope}%
\begin{pgfscope}%
\pgfpathrectangle{\pgfqpoint{4.376725in}{1.668832in}}{\pgfqpoint{1.162500in}{0.755000in}}%
\pgfusepath{clip}%
\pgfsetbuttcap%
\pgfsetroundjoin%
\definecolor{currentfill}{rgb}{0.000000,0.000000,0.000000}%
\pgfsetfillcolor{currentfill}%
\pgfsetfillopacity{0.500000}%
\pgfsetlinewidth{0.000000pt}%
\definecolor{currentstroke}{rgb}{0.000000,0.000000,0.000000}%
\pgfsetstrokecolor{currentstroke}%
\pgfsetdash{}{0pt}%
\pgfpathmoveto{\pgfqpoint{5.511546in}{2.145886in}}%
\pgfpathcurveto{\pgfqpoint{5.517071in}{2.145886in}}{\pgfqpoint{5.522371in}{2.148081in}}{\pgfqpoint{5.526278in}{2.151988in}}%
\pgfpathcurveto{\pgfqpoint{5.530185in}{2.155894in}}{\pgfqpoint{5.532380in}{2.161194in}}{\pgfqpoint{5.532380in}{2.166719in}}%
\pgfpathcurveto{\pgfqpoint{5.532380in}{2.172244in}}{\pgfqpoint{5.530185in}{2.177544in}}{\pgfqpoint{5.526278in}{2.181450in}}%
\pgfpathcurveto{\pgfqpoint{5.522371in}{2.185357in}}{\pgfqpoint{5.517071in}{2.187552in}}{\pgfqpoint{5.511546in}{2.187552in}}%
\pgfpathcurveto{\pgfqpoint{5.506021in}{2.187552in}}{\pgfqpoint{5.500722in}{2.185357in}}{\pgfqpoint{5.496815in}{2.181450in}}%
\pgfpathcurveto{\pgfqpoint{5.492908in}{2.177544in}}{\pgfqpoint{5.490713in}{2.172244in}}{\pgfqpoint{5.490713in}{2.166719in}}%
\pgfpathcurveto{\pgfqpoint{5.490713in}{2.161194in}}{\pgfqpoint{5.492908in}{2.155894in}}{\pgfqpoint{5.496815in}{2.151988in}}%
\pgfpathcurveto{\pgfqpoint{5.500722in}{2.148081in}}{\pgfqpoint{5.506021in}{2.145886in}}{\pgfqpoint{5.511546in}{2.145886in}}%
\pgfpathclose%
\pgfusepath{fill}%
\end{pgfscope}%
\begin{pgfscope}%
\pgfpathrectangle{\pgfqpoint{4.376725in}{1.668832in}}{\pgfqpoint{1.162500in}{0.755000in}}%
\pgfusepath{clip}%
\pgfsetbuttcap%
\pgfsetroundjoin%
\definecolor{currentfill}{rgb}{0.000000,0.000000,0.000000}%
\pgfsetfillcolor{currentfill}%
\pgfsetfillopacity{0.500000}%
\pgfsetlinewidth{0.000000pt}%
\definecolor{currentstroke}{rgb}{0.000000,0.000000,0.000000}%
\pgfsetstrokecolor{currentstroke}%
\pgfsetdash{}{0pt}%
\pgfpathmoveto{\pgfqpoint{5.286995in}{1.969105in}}%
\pgfpathcurveto{\pgfqpoint{5.292520in}{1.969105in}}{\pgfqpoint{5.297820in}{1.971300in}}{\pgfqpoint{5.301726in}{1.975207in}}%
\pgfpathcurveto{\pgfqpoint{5.305633in}{1.979114in}}{\pgfqpoint{5.307828in}{1.984413in}}{\pgfqpoint{5.307828in}{1.989939in}}%
\pgfpathcurveto{\pgfqpoint{5.307828in}{1.995464in}}{\pgfqpoint{5.305633in}{2.000763in}}{\pgfqpoint{5.301726in}{2.004670in}}%
\pgfpathcurveto{\pgfqpoint{5.297820in}{2.008577in}}{\pgfqpoint{5.292520in}{2.010772in}}{\pgfqpoint{5.286995in}{2.010772in}}%
\pgfpathcurveto{\pgfqpoint{5.281470in}{2.010772in}}{\pgfqpoint{5.276170in}{2.008577in}}{\pgfqpoint{5.272264in}{2.004670in}}%
\pgfpathcurveto{\pgfqpoint{5.268357in}{2.000763in}}{\pgfqpoint{5.266162in}{1.995464in}}{\pgfqpoint{5.266162in}{1.989939in}}%
\pgfpathcurveto{\pgfqpoint{5.266162in}{1.984413in}}{\pgfqpoint{5.268357in}{1.979114in}}{\pgfqpoint{5.272264in}{1.975207in}}%
\pgfpathcurveto{\pgfqpoint{5.276170in}{1.971300in}}{\pgfqpoint{5.281470in}{1.969105in}}{\pgfqpoint{5.286995in}{1.969105in}}%
\pgfpathclose%
\pgfusepath{fill}%
\end{pgfscope}%
\begin{pgfscope}%
\pgfpathrectangle{\pgfqpoint{4.376725in}{1.668832in}}{\pgfqpoint{1.162500in}{0.755000in}}%
\pgfusepath{clip}%
\pgfsetbuttcap%
\pgfsetroundjoin%
\definecolor{currentfill}{rgb}{0.000000,0.000000,0.000000}%
\pgfsetfillcolor{currentfill}%
\pgfsetfillopacity{0.500000}%
\pgfsetlinewidth{0.000000pt}%
\definecolor{currentstroke}{rgb}{0.000000,0.000000,0.000000}%
\pgfsetstrokecolor{currentstroke}%
\pgfsetdash{}{0pt}%
\pgfpathmoveto{\pgfqpoint{4.645822in}{2.239104in}}%
\pgfpathcurveto{\pgfqpoint{4.651348in}{2.239104in}}{\pgfqpoint{4.656647in}{2.241300in}}{\pgfqpoint{4.660554in}{2.245206in}}%
\pgfpathcurveto{\pgfqpoint{4.664461in}{2.249113in}}{\pgfqpoint{4.666656in}{2.254413in}}{\pgfqpoint{4.666656in}{2.259938in}}%
\pgfpathcurveto{\pgfqpoint{4.666656in}{2.265463in}}{\pgfqpoint{4.664461in}{2.270762in}}{\pgfqpoint{4.660554in}{2.274669in}}%
\pgfpathcurveto{\pgfqpoint{4.656647in}{2.278576in}}{\pgfqpoint{4.651348in}{2.280771in}}{\pgfqpoint{4.645822in}{2.280771in}}%
\pgfpathcurveto{\pgfqpoint{4.640297in}{2.280771in}}{\pgfqpoint{4.634998in}{2.278576in}}{\pgfqpoint{4.631091in}{2.274669in}}%
\pgfpathcurveto{\pgfqpoint{4.627184in}{2.270762in}}{\pgfqpoint{4.624989in}{2.265463in}}{\pgfqpoint{4.624989in}{2.259938in}}%
\pgfpathcurveto{\pgfqpoint{4.624989in}{2.254413in}}{\pgfqpoint{4.627184in}{2.249113in}}{\pgfqpoint{4.631091in}{2.245206in}}%
\pgfpathcurveto{\pgfqpoint{4.634998in}{2.241300in}}{\pgfqpoint{4.640297in}{2.239104in}}{\pgfqpoint{4.645822in}{2.239104in}}%
\pgfpathclose%
\pgfusepath{fill}%
\end{pgfscope}%
\begin{pgfscope}%
\pgfpathrectangle{\pgfqpoint{4.376725in}{1.668832in}}{\pgfqpoint{1.162500in}{0.755000in}}%
\pgfusepath{clip}%
\pgfsetbuttcap%
\pgfsetroundjoin%
\definecolor{currentfill}{rgb}{0.000000,0.000000,0.000000}%
\pgfsetfillcolor{currentfill}%
\pgfsetfillopacity{0.500000}%
\pgfsetlinewidth{0.000000pt}%
\definecolor{currentstroke}{rgb}{0.000000,0.000000,0.000000}%
\pgfsetstrokecolor{currentstroke}%
\pgfsetdash{}{0pt}%
\pgfpathmoveto{\pgfqpoint{5.069839in}{1.960666in}}%
\pgfpathcurveto{\pgfqpoint{5.075364in}{1.960666in}}{\pgfqpoint{5.080663in}{1.962861in}}{\pgfqpoint{5.084570in}{1.966768in}}%
\pgfpathcurveto{\pgfqpoint{5.088477in}{1.970675in}}{\pgfqpoint{5.090672in}{1.975974in}}{\pgfqpoint{5.090672in}{1.981499in}}%
\pgfpathcurveto{\pgfqpoint{5.090672in}{1.987024in}}{\pgfqpoint{5.088477in}{1.992324in}}{\pgfqpoint{5.084570in}{1.996231in}}%
\pgfpathcurveto{\pgfqpoint{5.080663in}{2.000138in}}{\pgfqpoint{5.075364in}{2.002333in}}{\pgfqpoint{5.069839in}{2.002333in}}%
\pgfpathcurveto{\pgfqpoint{5.064313in}{2.002333in}}{\pgfqpoint{5.059014in}{2.000138in}}{\pgfqpoint{5.055107in}{1.996231in}}%
\pgfpathcurveto{\pgfqpoint{5.051200in}{1.992324in}}{\pgfqpoint{5.049005in}{1.987024in}}{\pgfqpoint{5.049005in}{1.981499in}}%
\pgfpathcurveto{\pgfqpoint{5.049005in}{1.975974in}}{\pgfqpoint{5.051200in}{1.970675in}}{\pgfqpoint{5.055107in}{1.966768in}}%
\pgfpathcurveto{\pgfqpoint{5.059014in}{1.962861in}}{\pgfqpoint{5.064313in}{1.960666in}}{\pgfqpoint{5.069839in}{1.960666in}}%
\pgfpathclose%
\pgfusepath{fill}%
\end{pgfscope}%
\begin{pgfscope}%
\pgfpathrectangle{\pgfqpoint{4.376725in}{1.668832in}}{\pgfqpoint{1.162500in}{0.755000in}}%
\pgfusepath{clip}%
\pgfsetbuttcap%
\pgfsetroundjoin%
\definecolor{currentfill}{rgb}{0.000000,0.000000,0.000000}%
\pgfsetfillcolor{currentfill}%
\pgfsetfillopacity{0.500000}%
\pgfsetlinewidth{0.000000pt}%
\definecolor{currentstroke}{rgb}{0.000000,0.000000,0.000000}%
\pgfsetstrokecolor{currentstroke}%
\pgfsetdash{}{0pt}%
\pgfpathmoveto{\pgfqpoint{4.941375in}{1.722959in}}%
\pgfpathcurveto{\pgfqpoint{4.946900in}{1.722959in}}{\pgfqpoint{4.952199in}{1.725154in}}{\pgfqpoint{4.956106in}{1.729061in}}%
\pgfpathcurveto{\pgfqpoint{4.960013in}{1.732968in}}{\pgfqpoint{4.962208in}{1.738267in}}{\pgfqpoint{4.962208in}{1.743792in}}%
\pgfpathcurveto{\pgfqpoint{4.962208in}{1.749317in}}{\pgfqpoint{4.960013in}{1.754617in}}{\pgfqpoint{4.956106in}{1.758523in}}%
\pgfpathcurveto{\pgfqpoint{4.952199in}{1.762430in}}{\pgfqpoint{4.946900in}{1.764625in}}{\pgfqpoint{4.941375in}{1.764625in}}%
\pgfpathcurveto{\pgfqpoint{4.935850in}{1.764625in}}{\pgfqpoint{4.930550in}{1.762430in}}{\pgfqpoint{4.926643in}{1.758523in}}%
\pgfpathcurveto{\pgfqpoint{4.922736in}{1.754617in}}{\pgfqpoint{4.920541in}{1.749317in}}{\pgfqpoint{4.920541in}{1.743792in}}%
\pgfpathcurveto{\pgfqpoint{4.920541in}{1.738267in}}{\pgfqpoint{4.922736in}{1.732968in}}{\pgfqpoint{4.926643in}{1.729061in}}%
\pgfpathcurveto{\pgfqpoint{4.930550in}{1.725154in}}{\pgfqpoint{4.935850in}{1.722959in}}{\pgfqpoint{4.941375in}{1.722959in}}%
\pgfpathclose%
\pgfusepath{fill}%
\end{pgfscope}%
\begin{pgfscope}%
\pgfpathrectangle{\pgfqpoint{4.376725in}{1.668832in}}{\pgfqpoint{1.162500in}{0.755000in}}%
\pgfusepath{clip}%
\pgfsetbuttcap%
\pgfsetroundjoin%
\definecolor{currentfill}{rgb}{0.000000,0.000000,0.000000}%
\pgfsetfillcolor{currentfill}%
\pgfsetfillopacity{0.500000}%
\pgfsetlinewidth{0.000000pt}%
\definecolor{currentstroke}{rgb}{0.000000,0.000000,0.000000}%
\pgfsetstrokecolor{currentstroke}%
\pgfsetdash{}{0pt}%
\pgfpathmoveto{\pgfqpoint{4.799153in}{2.267350in}}%
\pgfpathcurveto{\pgfqpoint{4.804678in}{2.267350in}}{\pgfqpoint{4.809977in}{2.269545in}}{\pgfqpoint{4.813884in}{2.273452in}}%
\pgfpathcurveto{\pgfqpoint{4.817791in}{2.277359in}}{\pgfqpoint{4.819986in}{2.282659in}}{\pgfqpoint{4.819986in}{2.288184in}}%
\pgfpathcurveto{\pgfqpoint{4.819986in}{2.293709in}}{\pgfqpoint{4.817791in}{2.299008in}}{\pgfqpoint{4.813884in}{2.302915in}}%
\pgfpathcurveto{\pgfqpoint{4.809977in}{2.306822in}}{\pgfqpoint{4.804678in}{2.309017in}}{\pgfqpoint{4.799153in}{2.309017in}}%
\pgfpathcurveto{\pgfqpoint{4.793628in}{2.309017in}}{\pgfqpoint{4.788328in}{2.306822in}}{\pgfqpoint{4.784421in}{2.302915in}}%
\pgfpathcurveto{\pgfqpoint{4.780515in}{2.299008in}}{\pgfqpoint{4.778320in}{2.293709in}}{\pgfqpoint{4.778320in}{2.288184in}}%
\pgfpathcurveto{\pgfqpoint{4.778320in}{2.282659in}}{\pgfqpoint{4.780515in}{2.277359in}}{\pgfqpoint{4.784421in}{2.273452in}}%
\pgfpathcurveto{\pgfqpoint{4.788328in}{2.269545in}}{\pgfqpoint{4.793628in}{2.267350in}}{\pgfqpoint{4.799153in}{2.267350in}}%
\pgfpathclose%
\pgfusepath{fill}%
\end{pgfscope}%
\begin{pgfscope}%
\pgfpathrectangle{\pgfqpoint{4.376725in}{1.668832in}}{\pgfqpoint{1.162500in}{0.755000in}}%
\pgfusepath{clip}%
\pgfsetbuttcap%
\pgfsetroundjoin%
\definecolor{currentfill}{rgb}{0.000000,0.000000,0.000000}%
\pgfsetfillcolor{currentfill}%
\pgfsetfillopacity{0.500000}%
\pgfsetlinewidth{0.000000pt}%
\definecolor{currentstroke}{rgb}{0.000000,0.000000,0.000000}%
\pgfsetstrokecolor{currentstroke}%
\pgfsetdash{}{0pt}%
\pgfpathmoveto{\pgfqpoint{4.595218in}{2.236796in}}%
\pgfpathcurveto{\pgfqpoint{4.600743in}{2.236796in}}{\pgfqpoint{4.606043in}{2.238991in}}{\pgfqpoint{4.609950in}{2.242898in}}%
\pgfpathcurveto{\pgfqpoint{4.613857in}{2.246805in}}{\pgfqpoint{4.616052in}{2.252104in}}{\pgfqpoint{4.616052in}{2.257629in}}%
\pgfpathcurveto{\pgfqpoint{4.616052in}{2.263154in}}{\pgfqpoint{4.613857in}{2.268454in}}{\pgfqpoint{4.609950in}{2.272361in}}%
\pgfpathcurveto{\pgfqpoint{4.606043in}{2.276267in}}{\pgfqpoint{4.600743in}{2.278463in}}{\pgfqpoint{4.595218in}{2.278463in}}%
\pgfpathcurveto{\pgfqpoint{4.589693in}{2.278463in}}{\pgfqpoint{4.584394in}{2.276267in}}{\pgfqpoint{4.580487in}{2.272361in}}%
\pgfpathcurveto{\pgfqpoint{4.576580in}{2.268454in}}{\pgfqpoint{4.574385in}{2.263154in}}{\pgfqpoint{4.574385in}{2.257629in}}%
\pgfpathcurveto{\pgfqpoint{4.574385in}{2.252104in}}{\pgfqpoint{4.576580in}{2.246805in}}{\pgfqpoint{4.580487in}{2.242898in}}%
\pgfpathcurveto{\pgfqpoint{4.584394in}{2.238991in}}{\pgfqpoint{4.589693in}{2.236796in}}{\pgfqpoint{4.595218in}{2.236796in}}%
\pgfpathclose%
\pgfusepath{fill}%
\end{pgfscope}%
\begin{pgfscope}%
\pgfpathrectangle{\pgfqpoint{4.376725in}{1.668832in}}{\pgfqpoint{1.162500in}{0.755000in}}%
\pgfusepath{clip}%
\pgfsetbuttcap%
\pgfsetroundjoin%
\definecolor{currentfill}{rgb}{0.000000,0.000000,0.000000}%
\pgfsetfillcolor{currentfill}%
\pgfsetfillopacity{0.500000}%
\pgfsetlinewidth{0.000000pt}%
\definecolor{currentstroke}{rgb}{0.000000,0.000000,0.000000}%
\pgfsetstrokecolor{currentstroke}%
\pgfsetdash{}{0pt}%
\pgfpathmoveto{\pgfqpoint{4.834270in}{2.023720in}}%
\pgfpathcurveto{\pgfqpoint{4.839795in}{2.023720in}}{\pgfqpoint{4.845094in}{2.025915in}}{\pgfqpoint{4.849001in}{2.029822in}}%
\pgfpathcurveto{\pgfqpoint{4.852908in}{2.033729in}}{\pgfqpoint{4.855103in}{2.039028in}}{\pgfqpoint{4.855103in}{2.044553in}}%
\pgfpathcurveto{\pgfqpoint{4.855103in}{2.050078in}}{\pgfqpoint{4.852908in}{2.055378in}}{\pgfqpoint{4.849001in}{2.059285in}}%
\pgfpathcurveto{\pgfqpoint{4.845094in}{2.063191in}}{\pgfqpoint{4.839795in}{2.065387in}}{\pgfqpoint{4.834270in}{2.065387in}}%
\pgfpathcurveto{\pgfqpoint{4.828745in}{2.065387in}}{\pgfqpoint{4.823445in}{2.063191in}}{\pgfqpoint{4.819538in}{2.059285in}}%
\pgfpathcurveto{\pgfqpoint{4.815632in}{2.055378in}}{\pgfqpoint{4.813437in}{2.050078in}}{\pgfqpoint{4.813437in}{2.044553in}}%
\pgfpathcurveto{\pgfqpoint{4.813437in}{2.039028in}}{\pgfqpoint{4.815632in}{2.033729in}}{\pgfqpoint{4.819538in}{2.029822in}}%
\pgfpathcurveto{\pgfqpoint{4.823445in}{2.025915in}}{\pgfqpoint{4.828745in}{2.023720in}}{\pgfqpoint{4.834270in}{2.023720in}}%
\pgfpathclose%
\pgfusepath{fill}%
\end{pgfscope}%
\begin{pgfscope}%
\pgfsetrectcap%
\pgfsetmiterjoin%
\pgfsetlinewidth{0.803000pt}%
\definecolor{currentstroke}{rgb}{0.501961,0.501961,0.501961}%
\pgfsetstrokecolor{currentstroke}%
\pgfsetdash{}{0pt}%
\pgfpathmoveto{\pgfqpoint{4.376725in}{1.668832in}}%
\pgfpathlineto{\pgfqpoint{4.376725in}{2.423832in}}%
\pgfusepath{stroke}%
\end{pgfscope}%
\begin{pgfscope}%
\pgfsetrectcap%
\pgfsetmiterjoin%
\pgfsetlinewidth{0.803000pt}%
\definecolor{currentstroke}{rgb}{0.501961,0.501961,0.501961}%
\pgfsetstrokecolor{currentstroke}%
\pgfsetdash{}{0pt}%
\pgfpathmoveto{\pgfqpoint{5.539225in}{1.668832in}}%
\pgfpathlineto{\pgfqpoint{5.539225in}{2.423832in}}%
\pgfusepath{stroke}%
\end{pgfscope}%
\begin{pgfscope}%
\pgfsetrectcap%
\pgfsetmiterjoin%
\pgfsetlinewidth{0.803000pt}%
\definecolor{currentstroke}{rgb}{0.501961,0.501961,0.501961}%
\pgfsetstrokecolor{currentstroke}%
\pgfsetdash{}{0pt}%
\pgfpathmoveto{\pgfqpoint{4.376725in}{1.668832in}}%
\pgfpathlineto{\pgfqpoint{5.539225in}{1.668832in}}%
\pgfusepath{stroke}%
\end{pgfscope}%
\begin{pgfscope}%
\pgfsetrectcap%
\pgfsetmiterjoin%
\pgfsetlinewidth{0.803000pt}%
\definecolor{currentstroke}{rgb}{0.501961,0.501961,0.501961}%
\pgfsetstrokecolor{currentstroke}%
\pgfsetdash{}{0pt}%
\pgfpathmoveto{\pgfqpoint{4.376725in}{2.423832in}}%
\pgfpathlineto{\pgfqpoint{5.539225in}{2.423832in}}%
\pgfusepath{stroke}%
\end{pgfscope}%
\begin{pgfscope}%
\pgfsetbuttcap%
\pgfsetmiterjoin%
\definecolor{currentfill}{rgb}{1.000000,1.000000,1.000000}%
\pgfsetfillcolor{currentfill}%
\pgfsetlinewidth{0.000000pt}%
\definecolor{currentstroke}{rgb}{0.000000,0.000000,0.000000}%
\pgfsetstrokecolor{currentstroke}%
\pgfsetstrokeopacity{0.000000}%
\pgfsetdash{}{0pt}%
\pgfpathmoveto{\pgfqpoint{0.889225in}{0.913832in}}%
\pgfpathlineto{\pgfqpoint{2.051725in}{0.913832in}}%
\pgfpathlineto{\pgfqpoint{2.051725in}{1.668832in}}%
\pgfpathlineto{\pgfqpoint{0.889225in}{1.668832in}}%
\pgfpathclose%
\pgfusepath{fill}%
\end{pgfscope}%
\begin{pgfscope}%
\pgfpathrectangle{\pgfqpoint{0.889225in}{0.913832in}}{\pgfqpoint{1.162500in}{0.755000in}}%
\pgfusepath{clip}%
\pgfsetbuttcap%
\pgfsetroundjoin%
\definecolor{currentfill}{rgb}{0.000000,0.000000,0.000000}%
\pgfsetfillcolor{currentfill}%
\pgfsetfillopacity{0.500000}%
\pgfsetlinewidth{0.000000pt}%
\definecolor{currentstroke}{rgb}{0.000000,0.000000,0.000000}%
\pgfsetstrokecolor{currentstroke}%
\pgfsetdash{}{0pt}%
\pgfpathmoveto{\pgfqpoint{1.893746in}{1.437401in}}%
\pgfpathcurveto{\pgfqpoint{1.899271in}{1.437401in}}{\pgfqpoint{1.904570in}{1.439596in}}{\pgfqpoint{1.908477in}{1.443503in}}%
\pgfpathcurveto{\pgfqpoint{1.912384in}{1.447410in}}{\pgfqpoint{1.914579in}{1.452710in}}{\pgfqpoint{1.914579in}{1.458235in}}%
\pgfpathcurveto{\pgfqpoint{1.914579in}{1.463760in}}{\pgfqpoint{1.912384in}{1.469059in}}{\pgfqpoint{1.908477in}{1.472966in}}%
\pgfpathcurveto{\pgfqpoint{1.904570in}{1.476873in}}{\pgfqpoint{1.899271in}{1.479068in}}{\pgfqpoint{1.893746in}{1.479068in}}%
\pgfpathcurveto{\pgfqpoint{1.888221in}{1.479068in}}{\pgfqpoint{1.882921in}{1.476873in}}{\pgfqpoint{1.879015in}{1.472966in}}%
\pgfpathcurveto{\pgfqpoint{1.875108in}{1.469059in}}{\pgfqpoint{1.872913in}{1.463760in}}{\pgfqpoint{1.872913in}{1.458235in}}%
\pgfpathcurveto{\pgfqpoint{1.872913in}{1.452710in}}{\pgfqpoint{1.875108in}{1.447410in}}{\pgfqpoint{1.879015in}{1.443503in}}%
\pgfpathcurveto{\pgfqpoint{1.882921in}{1.439596in}}{\pgfqpoint{1.888221in}{1.437401in}}{\pgfqpoint{1.893746in}{1.437401in}}%
\pgfpathclose%
\pgfusepath{fill}%
\end{pgfscope}%
\begin{pgfscope}%
\pgfpathrectangle{\pgfqpoint{0.889225in}{0.913832in}}{\pgfqpoint{1.162500in}{0.755000in}}%
\pgfusepath{clip}%
\pgfsetbuttcap%
\pgfsetroundjoin%
\definecolor{currentfill}{rgb}{0.000000,0.000000,0.000000}%
\pgfsetfillcolor{currentfill}%
\pgfsetfillopacity{0.500000}%
\pgfsetlinewidth{0.000000pt}%
\definecolor{currentstroke}{rgb}{0.000000,0.000000,0.000000}%
\pgfsetstrokecolor{currentstroke}%
\pgfsetdash{}{0pt}%
\pgfpathmoveto{\pgfqpoint{1.476892in}{0.987180in}}%
\pgfpathcurveto{\pgfqpoint{1.482417in}{0.987180in}}{\pgfqpoint{1.487717in}{0.989375in}}{\pgfqpoint{1.491623in}{0.993282in}}%
\pgfpathcurveto{\pgfqpoint{1.495530in}{0.997189in}}{\pgfqpoint{1.497725in}{1.002488in}}{\pgfqpoint{1.497725in}{1.008013in}}%
\pgfpathcurveto{\pgfqpoint{1.497725in}{1.013538in}}{\pgfqpoint{1.495530in}{1.018838in}}{\pgfqpoint{1.491623in}{1.022745in}}%
\pgfpathcurveto{\pgfqpoint{1.487717in}{1.026652in}}{\pgfqpoint{1.482417in}{1.028847in}}{\pgfqpoint{1.476892in}{1.028847in}}%
\pgfpathcurveto{\pgfqpoint{1.471367in}{1.028847in}}{\pgfqpoint{1.466067in}{1.026652in}}{\pgfqpoint{1.462161in}{1.022745in}}%
\pgfpathcurveto{\pgfqpoint{1.458254in}{1.018838in}}{\pgfqpoint{1.456059in}{1.013538in}}{\pgfqpoint{1.456059in}{1.008013in}}%
\pgfpathcurveto{\pgfqpoint{1.456059in}{1.002488in}}{\pgfqpoint{1.458254in}{0.997189in}}{\pgfqpoint{1.462161in}{0.993282in}}%
\pgfpathcurveto{\pgfqpoint{1.466067in}{0.989375in}}{\pgfqpoint{1.471367in}{0.987180in}}{\pgfqpoint{1.476892in}{0.987180in}}%
\pgfpathclose%
\pgfusepath{fill}%
\end{pgfscope}%
\begin{pgfscope}%
\pgfpathrectangle{\pgfqpoint{0.889225in}{0.913832in}}{\pgfqpoint{1.162500in}{0.755000in}}%
\pgfusepath{clip}%
\pgfsetbuttcap%
\pgfsetroundjoin%
\definecolor{currentfill}{rgb}{0.000000,0.000000,0.000000}%
\pgfsetfillcolor{currentfill}%
\pgfsetfillopacity{0.500000}%
\pgfsetlinewidth{0.000000pt}%
\definecolor{currentstroke}{rgb}{0.000000,0.000000,0.000000}%
\pgfsetstrokecolor{currentstroke}%
\pgfsetdash{}{0pt}%
\pgfpathmoveto{\pgfqpoint{1.478974in}{0.939886in}}%
\pgfpathcurveto{\pgfqpoint{1.484499in}{0.939886in}}{\pgfqpoint{1.489799in}{0.942081in}}{\pgfqpoint{1.493705in}{0.945988in}}%
\pgfpathcurveto{\pgfqpoint{1.497612in}{0.949895in}}{\pgfqpoint{1.499807in}{0.955195in}}{\pgfqpoint{1.499807in}{0.960720in}}%
\pgfpathcurveto{\pgfqpoint{1.499807in}{0.966245in}}{\pgfqpoint{1.497612in}{0.971544in}}{\pgfqpoint{1.493705in}{0.975451in}}%
\pgfpathcurveto{\pgfqpoint{1.489799in}{0.979358in}}{\pgfqpoint{1.484499in}{0.981553in}}{\pgfqpoint{1.478974in}{0.981553in}}%
\pgfpathcurveto{\pgfqpoint{1.473449in}{0.981553in}}{\pgfqpoint{1.468149in}{0.979358in}}{\pgfqpoint{1.464243in}{0.975451in}}%
\pgfpathcurveto{\pgfqpoint{1.460336in}{0.971544in}}{\pgfqpoint{1.458141in}{0.966245in}}{\pgfqpoint{1.458141in}{0.960720in}}%
\pgfpathcurveto{\pgfqpoint{1.458141in}{0.955195in}}{\pgfqpoint{1.460336in}{0.949895in}}{\pgfqpoint{1.464243in}{0.945988in}}%
\pgfpathcurveto{\pgfqpoint{1.468149in}{0.942081in}}{\pgfqpoint{1.473449in}{0.939886in}}{\pgfqpoint{1.478974in}{0.939886in}}%
\pgfpathclose%
\pgfusepath{fill}%
\end{pgfscope}%
\begin{pgfscope}%
\pgfpathrectangle{\pgfqpoint{0.889225in}{0.913832in}}{\pgfqpoint{1.162500in}{0.755000in}}%
\pgfusepath{clip}%
\pgfsetbuttcap%
\pgfsetroundjoin%
\definecolor{currentfill}{rgb}{0.000000,0.000000,0.000000}%
\pgfsetfillcolor{currentfill}%
\pgfsetfillopacity{0.500000}%
\pgfsetlinewidth{0.000000pt}%
\definecolor{currentstroke}{rgb}{0.000000,0.000000,0.000000}%
\pgfsetstrokecolor{currentstroke}%
\pgfsetdash{}{0pt}%
\pgfpathmoveto{\pgfqpoint{1.120714in}{1.105715in}}%
\pgfpathcurveto{\pgfqpoint{1.126239in}{1.105715in}}{\pgfqpoint{1.131538in}{1.107910in}}{\pgfqpoint{1.135445in}{1.111817in}}%
\pgfpathcurveto{\pgfqpoint{1.139352in}{1.115723in}}{\pgfqpoint{1.141547in}{1.121023in}}{\pgfqpoint{1.141547in}{1.126548in}}%
\pgfpathcurveto{\pgfqpoint{1.141547in}{1.132073in}}{\pgfqpoint{1.139352in}{1.137372in}}{\pgfqpoint{1.135445in}{1.141279in}}%
\pgfpathcurveto{\pgfqpoint{1.131538in}{1.145186in}}{\pgfqpoint{1.126239in}{1.147381in}}{\pgfqpoint{1.120714in}{1.147381in}}%
\pgfpathcurveto{\pgfqpoint{1.115189in}{1.147381in}}{\pgfqpoint{1.109889in}{1.145186in}}{\pgfqpoint{1.105982in}{1.141279in}}%
\pgfpathcurveto{\pgfqpoint{1.102076in}{1.137372in}}{\pgfqpoint{1.099881in}{1.132073in}}{\pgfqpoint{1.099881in}{1.126548in}}%
\pgfpathcurveto{\pgfqpoint{1.099881in}{1.121023in}}{\pgfqpoint{1.102076in}{1.115723in}}{\pgfqpoint{1.105982in}{1.111817in}}%
\pgfpathcurveto{\pgfqpoint{1.109889in}{1.107910in}}{\pgfqpoint{1.115189in}{1.105715in}}{\pgfqpoint{1.120714in}{1.105715in}}%
\pgfpathclose%
\pgfusepath{fill}%
\end{pgfscope}%
\begin{pgfscope}%
\pgfpathrectangle{\pgfqpoint{0.889225in}{0.913832in}}{\pgfqpoint{1.162500in}{0.755000in}}%
\pgfusepath{clip}%
\pgfsetbuttcap%
\pgfsetroundjoin%
\definecolor{currentfill}{rgb}{0.000000,0.000000,0.000000}%
\pgfsetfillcolor{currentfill}%
\pgfsetfillopacity{0.500000}%
\pgfsetlinewidth{0.000000pt}%
\definecolor{currentstroke}{rgb}{0.000000,0.000000,0.000000}%
\pgfsetstrokecolor{currentstroke}%
\pgfsetdash{}{0pt}%
\pgfpathmoveto{\pgfqpoint{1.126248in}{0.910975in}}%
\pgfpathcurveto{\pgfqpoint{1.131773in}{0.910975in}}{\pgfqpoint{1.137073in}{0.913170in}}{\pgfqpoint{1.140980in}{0.917077in}}%
\pgfpathcurveto{\pgfqpoint{1.144886in}{0.920984in}}{\pgfqpoint{1.147082in}{0.926283in}}{\pgfqpoint{1.147082in}{0.931809in}}%
\pgfpathcurveto{\pgfqpoint{1.147082in}{0.937334in}}{\pgfqpoint{1.144886in}{0.942633in}}{\pgfqpoint{1.140980in}{0.946540in}}%
\pgfpathcurveto{\pgfqpoint{1.137073in}{0.950447in}}{\pgfqpoint{1.131773in}{0.952642in}}{\pgfqpoint{1.126248in}{0.952642in}}%
\pgfpathcurveto{\pgfqpoint{1.120723in}{0.952642in}}{\pgfqpoint{1.115424in}{0.950447in}}{\pgfqpoint{1.111517in}{0.946540in}}%
\pgfpathcurveto{\pgfqpoint{1.107610in}{0.942633in}}{\pgfqpoint{1.105415in}{0.937334in}}{\pgfqpoint{1.105415in}{0.931809in}}%
\pgfpathcurveto{\pgfqpoint{1.105415in}{0.926283in}}{\pgfqpoint{1.107610in}{0.920984in}}{\pgfqpoint{1.111517in}{0.917077in}}%
\pgfpathcurveto{\pgfqpoint{1.115424in}{0.913170in}}{\pgfqpoint{1.120723in}{0.910975in}}{\pgfqpoint{1.126248in}{0.910975in}}%
\pgfpathclose%
\pgfusepath{fill}%
\end{pgfscope}%
\begin{pgfscope}%
\pgfpathrectangle{\pgfqpoint{0.889225in}{0.913832in}}{\pgfqpoint{1.162500in}{0.755000in}}%
\pgfusepath{clip}%
\pgfsetbuttcap%
\pgfsetroundjoin%
\definecolor{currentfill}{rgb}{0.000000,0.000000,0.000000}%
\pgfsetfillcolor{currentfill}%
\pgfsetfillopacity{0.500000}%
\pgfsetlinewidth{0.000000pt}%
\definecolor{currentstroke}{rgb}{0.000000,0.000000,0.000000}%
\pgfsetstrokecolor{currentstroke}%
\pgfsetdash{}{0pt}%
\pgfpathmoveto{\pgfqpoint{0.977704in}{1.042931in}}%
\pgfpathcurveto{\pgfqpoint{0.983229in}{1.042931in}}{\pgfqpoint{0.988528in}{1.045126in}}{\pgfqpoint{0.992435in}{1.049033in}}%
\pgfpathcurveto{\pgfqpoint{0.996342in}{1.052940in}}{\pgfqpoint{0.998537in}{1.058239in}}{\pgfqpoint{0.998537in}{1.063764in}}%
\pgfpathcurveto{\pgfqpoint{0.998537in}{1.069289in}}{\pgfqpoint{0.996342in}{1.074589in}}{\pgfqpoint{0.992435in}{1.078495in}}%
\pgfpathcurveto{\pgfqpoint{0.988528in}{1.082402in}}{\pgfqpoint{0.983229in}{1.084597in}}{\pgfqpoint{0.977704in}{1.084597in}}%
\pgfpathcurveto{\pgfqpoint{0.972179in}{1.084597in}}{\pgfqpoint{0.966879in}{1.082402in}}{\pgfqpoint{0.962972in}{1.078495in}}%
\pgfpathcurveto{\pgfqpoint{0.959066in}{1.074589in}}{\pgfqpoint{0.956870in}{1.069289in}}{\pgfqpoint{0.956870in}{1.063764in}}%
\pgfpathcurveto{\pgfqpoint{0.956870in}{1.058239in}}{\pgfqpoint{0.959066in}{1.052940in}}{\pgfqpoint{0.962972in}{1.049033in}}%
\pgfpathcurveto{\pgfqpoint{0.966879in}{1.045126in}}{\pgfqpoint{0.972179in}{1.042931in}}{\pgfqpoint{0.977704in}{1.042931in}}%
\pgfpathclose%
\pgfusepath{fill}%
\end{pgfscope}%
\begin{pgfscope}%
\pgfpathrectangle{\pgfqpoint{0.889225in}{0.913832in}}{\pgfqpoint{1.162500in}{0.755000in}}%
\pgfusepath{clip}%
\pgfsetbuttcap%
\pgfsetroundjoin%
\definecolor{currentfill}{rgb}{0.000000,0.000000,0.000000}%
\pgfsetfillcolor{currentfill}%
\pgfsetfillopacity{0.500000}%
\pgfsetlinewidth{0.000000pt}%
\definecolor{currentstroke}{rgb}{0.000000,0.000000,0.000000}%
\pgfsetstrokecolor{currentstroke}%
\pgfsetdash{}{0pt}%
\pgfpathmoveto{\pgfqpoint{0.916904in}{1.035607in}}%
\pgfpathcurveto{\pgfqpoint{0.922429in}{1.035607in}}{\pgfqpoint{0.927728in}{1.037803in}}{\pgfqpoint{0.931635in}{1.041709in}}%
\pgfpathcurveto{\pgfqpoint{0.935542in}{1.045616in}}{\pgfqpoint{0.937737in}{1.050916in}}{\pgfqpoint{0.937737in}{1.056441in}}%
\pgfpathcurveto{\pgfqpoint{0.937737in}{1.061966in}}{\pgfqpoint{0.935542in}{1.067265in}}{\pgfqpoint{0.931635in}{1.071172in}}%
\pgfpathcurveto{\pgfqpoint{0.927728in}{1.075079in}}{\pgfqpoint{0.922429in}{1.077274in}}{\pgfqpoint{0.916904in}{1.077274in}}%
\pgfpathcurveto{\pgfqpoint{0.911379in}{1.077274in}}{\pgfqpoint{0.906079in}{1.075079in}}{\pgfqpoint{0.902172in}{1.071172in}}%
\pgfpathcurveto{\pgfqpoint{0.898265in}{1.067265in}}{\pgfqpoint{0.896070in}{1.061966in}}{\pgfqpoint{0.896070in}{1.056441in}}%
\pgfpathcurveto{\pgfqpoint{0.896070in}{1.050916in}}{\pgfqpoint{0.898265in}{1.045616in}}{\pgfqpoint{0.902172in}{1.041709in}}%
\pgfpathcurveto{\pgfqpoint{0.906079in}{1.037803in}}{\pgfqpoint{0.911379in}{1.035607in}}{\pgfqpoint{0.916904in}{1.035607in}}%
\pgfpathclose%
\pgfusepath{fill}%
\end{pgfscope}%
\begin{pgfscope}%
\pgfpathrectangle{\pgfqpoint{0.889225in}{0.913832in}}{\pgfqpoint{1.162500in}{0.755000in}}%
\pgfusepath{clip}%
\pgfsetbuttcap%
\pgfsetroundjoin%
\definecolor{currentfill}{rgb}{0.000000,0.000000,0.000000}%
\pgfsetfillcolor{currentfill}%
\pgfsetfillopacity{0.500000}%
\pgfsetlinewidth{0.000000pt}%
\definecolor{currentstroke}{rgb}{0.000000,0.000000,0.000000}%
\pgfsetstrokecolor{currentstroke}%
\pgfsetdash{}{0pt}%
\pgfpathmoveto{\pgfqpoint{1.691400in}{1.572595in}}%
\pgfpathcurveto{\pgfqpoint{1.696925in}{1.572595in}}{\pgfqpoint{1.702225in}{1.574791in}}{\pgfqpoint{1.706132in}{1.578697in}}%
\pgfpathcurveto{\pgfqpoint{1.710038in}{1.582604in}}{\pgfqpoint{1.712234in}{1.587904in}}{\pgfqpoint{1.712234in}{1.593429in}}%
\pgfpathcurveto{\pgfqpoint{1.712234in}{1.598954in}}{\pgfqpoint{1.710038in}{1.604253in}}{\pgfqpoint{1.706132in}{1.608160in}}%
\pgfpathcurveto{\pgfqpoint{1.702225in}{1.612067in}}{\pgfqpoint{1.696925in}{1.614262in}}{\pgfqpoint{1.691400in}{1.614262in}}%
\pgfpathcurveto{\pgfqpoint{1.685875in}{1.614262in}}{\pgfqpoint{1.680576in}{1.612067in}}{\pgfqpoint{1.676669in}{1.608160in}}%
\pgfpathcurveto{\pgfqpoint{1.672762in}{1.604253in}}{\pgfqpoint{1.670567in}{1.598954in}}{\pgfqpoint{1.670567in}{1.593429in}}%
\pgfpathcurveto{\pgfqpoint{1.670567in}{1.587904in}}{\pgfqpoint{1.672762in}{1.582604in}}{\pgfqpoint{1.676669in}{1.578697in}}%
\pgfpathcurveto{\pgfqpoint{1.680576in}{1.574791in}}{\pgfqpoint{1.685875in}{1.572595in}}{\pgfqpoint{1.691400in}{1.572595in}}%
\pgfpathclose%
\pgfusepath{fill}%
\end{pgfscope}%
\begin{pgfscope}%
\pgfpathrectangle{\pgfqpoint{0.889225in}{0.913832in}}{\pgfqpoint{1.162500in}{0.755000in}}%
\pgfusepath{clip}%
\pgfsetbuttcap%
\pgfsetroundjoin%
\definecolor{currentfill}{rgb}{0.000000,0.000000,0.000000}%
\pgfsetfillcolor{currentfill}%
\pgfsetfillopacity{0.500000}%
\pgfsetlinewidth{0.000000pt}%
\definecolor{currentstroke}{rgb}{0.000000,0.000000,0.000000}%
\pgfsetstrokecolor{currentstroke}%
\pgfsetdash{}{0pt}%
\pgfpathmoveto{\pgfqpoint{1.460463in}{1.630023in}}%
\pgfpathcurveto{\pgfqpoint{1.465988in}{1.630023in}}{\pgfqpoint{1.471288in}{1.632218in}}{\pgfqpoint{1.475194in}{1.636125in}}%
\pgfpathcurveto{\pgfqpoint{1.479101in}{1.640032in}}{\pgfqpoint{1.481296in}{1.645331in}}{\pgfqpoint{1.481296in}{1.650856in}}%
\pgfpathcurveto{\pgfqpoint{1.481296in}{1.656381in}}{\pgfqpoint{1.479101in}{1.661681in}}{\pgfqpoint{1.475194in}{1.665588in}}%
\pgfpathcurveto{\pgfqpoint{1.471288in}{1.669494in}}{\pgfqpoint{1.465988in}{1.671690in}}{\pgfqpoint{1.460463in}{1.671690in}}%
\pgfpathcurveto{\pgfqpoint{1.454938in}{1.671690in}}{\pgfqpoint{1.449638in}{1.669494in}}{\pgfqpoint{1.445732in}{1.665588in}}%
\pgfpathcurveto{\pgfqpoint{1.441825in}{1.661681in}}{\pgfqpoint{1.439630in}{1.656381in}}{\pgfqpoint{1.439630in}{1.650856in}}%
\pgfpathcurveto{\pgfqpoint{1.439630in}{1.645331in}}{\pgfqpoint{1.441825in}{1.640032in}}{\pgfqpoint{1.445732in}{1.636125in}}%
\pgfpathcurveto{\pgfqpoint{1.449638in}{1.632218in}}{\pgfqpoint{1.454938in}{1.630023in}}{\pgfqpoint{1.460463in}{1.630023in}}%
\pgfpathclose%
\pgfusepath{fill}%
\end{pgfscope}%
\begin{pgfscope}%
\pgfpathrectangle{\pgfqpoint{0.889225in}{0.913832in}}{\pgfqpoint{1.162500in}{0.755000in}}%
\pgfusepath{clip}%
\pgfsetbuttcap%
\pgfsetroundjoin%
\definecolor{currentfill}{rgb}{0.000000,0.000000,0.000000}%
\pgfsetfillcolor{currentfill}%
\pgfsetfillopacity{0.500000}%
\pgfsetlinewidth{0.000000pt}%
\definecolor{currentstroke}{rgb}{0.000000,0.000000,0.000000}%
\pgfsetstrokecolor{currentstroke}%
\pgfsetdash{}{0pt}%
\pgfpathmoveto{\pgfqpoint{1.303193in}{1.484185in}}%
\pgfpathcurveto{\pgfqpoint{1.308718in}{1.484185in}}{\pgfqpoint{1.314017in}{1.486380in}}{\pgfqpoint{1.317924in}{1.490287in}}%
\pgfpathcurveto{\pgfqpoint{1.321831in}{1.494194in}}{\pgfqpoint{1.324026in}{1.499493in}}{\pgfqpoint{1.324026in}{1.505018in}}%
\pgfpathcurveto{\pgfqpoint{1.324026in}{1.510544in}}{\pgfqpoint{1.321831in}{1.515843in}}{\pgfqpoint{1.317924in}{1.519750in}}%
\pgfpathcurveto{\pgfqpoint{1.314017in}{1.523657in}}{\pgfqpoint{1.308718in}{1.525852in}}{\pgfqpoint{1.303193in}{1.525852in}}%
\pgfpathcurveto{\pgfqpoint{1.297667in}{1.525852in}}{\pgfqpoint{1.292368in}{1.523657in}}{\pgfqpoint{1.288461in}{1.519750in}}%
\pgfpathcurveto{\pgfqpoint{1.284554in}{1.515843in}}{\pgfqpoint{1.282359in}{1.510544in}}{\pgfqpoint{1.282359in}{1.505018in}}%
\pgfpathcurveto{\pgfqpoint{1.282359in}{1.499493in}}{\pgfqpoint{1.284554in}{1.494194in}}{\pgfqpoint{1.288461in}{1.490287in}}%
\pgfpathcurveto{\pgfqpoint{1.292368in}{1.486380in}}{\pgfqpoint{1.297667in}{1.484185in}}{\pgfqpoint{1.303193in}{1.484185in}}%
\pgfpathclose%
\pgfusepath{fill}%
\end{pgfscope}%
\begin{pgfscope}%
\pgfpathrectangle{\pgfqpoint{0.889225in}{0.913832in}}{\pgfqpoint{1.162500in}{0.755000in}}%
\pgfusepath{clip}%
\pgfsetbuttcap%
\pgfsetroundjoin%
\definecolor{currentfill}{rgb}{0.000000,0.000000,0.000000}%
\pgfsetfillcolor{currentfill}%
\pgfsetfillopacity{0.500000}%
\pgfsetlinewidth{0.000000pt}%
\definecolor{currentstroke}{rgb}{0.000000,0.000000,0.000000}%
\pgfsetstrokecolor{currentstroke}%
\pgfsetdash{}{0pt}%
\pgfpathmoveto{\pgfqpoint{1.689350in}{1.067768in}}%
\pgfpathcurveto{\pgfqpoint{1.694875in}{1.067768in}}{\pgfqpoint{1.700175in}{1.069963in}}{\pgfqpoint{1.704082in}{1.073870in}}%
\pgfpathcurveto{\pgfqpoint{1.707989in}{1.077776in}}{\pgfqpoint{1.710184in}{1.083076in}}{\pgfqpoint{1.710184in}{1.088601in}}%
\pgfpathcurveto{\pgfqpoint{1.710184in}{1.094126in}}{\pgfqpoint{1.707989in}{1.099426in}}{\pgfqpoint{1.704082in}{1.103332in}}%
\pgfpathcurveto{\pgfqpoint{1.700175in}{1.107239in}}{\pgfqpoint{1.694875in}{1.109434in}}{\pgfqpoint{1.689350in}{1.109434in}}%
\pgfpathcurveto{\pgfqpoint{1.683825in}{1.109434in}}{\pgfqpoint{1.678526in}{1.107239in}}{\pgfqpoint{1.674619in}{1.103332in}}%
\pgfpathcurveto{\pgfqpoint{1.670712in}{1.099426in}}{\pgfqpoint{1.668517in}{1.094126in}}{\pgfqpoint{1.668517in}{1.088601in}}%
\pgfpathcurveto{\pgfqpoint{1.668517in}{1.083076in}}{\pgfqpoint{1.670712in}{1.077776in}}{\pgfqpoint{1.674619in}{1.073870in}}%
\pgfpathcurveto{\pgfqpoint{1.678526in}{1.069963in}}{\pgfqpoint{1.683825in}{1.067768in}}{\pgfqpoint{1.689350in}{1.067768in}}%
\pgfpathclose%
\pgfusepath{fill}%
\end{pgfscope}%
\begin{pgfscope}%
\pgfpathrectangle{\pgfqpoint{0.889225in}{0.913832in}}{\pgfqpoint{1.162500in}{0.755000in}}%
\pgfusepath{clip}%
\pgfsetbuttcap%
\pgfsetroundjoin%
\definecolor{currentfill}{rgb}{0.000000,0.000000,0.000000}%
\pgfsetfillcolor{currentfill}%
\pgfsetfillopacity{0.500000}%
\pgfsetlinewidth{0.000000pt}%
\definecolor{currentstroke}{rgb}{0.000000,0.000000,0.000000}%
\pgfsetstrokecolor{currentstroke}%
\pgfsetdash{}{0pt}%
\pgfpathmoveto{\pgfqpoint{1.102396in}{1.343150in}}%
\pgfpathcurveto{\pgfqpoint{1.107921in}{1.343150in}}{\pgfqpoint{1.113221in}{1.345345in}}{\pgfqpoint{1.117128in}{1.349252in}}%
\pgfpathcurveto{\pgfqpoint{1.121035in}{1.353159in}}{\pgfqpoint{1.123230in}{1.358458in}}{\pgfqpoint{1.123230in}{1.363984in}}%
\pgfpathcurveto{\pgfqpoint{1.123230in}{1.369509in}}{\pgfqpoint{1.121035in}{1.374808in}}{\pgfqpoint{1.117128in}{1.378715in}}%
\pgfpathcurveto{\pgfqpoint{1.113221in}{1.382622in}}{\pgfqpoint{1.107921in}{1.384817in}}{\pgfqpoint{1.102396in}{1.384817in}}%
\pgfpathcurveto{\pgfqpoint{1.096871in}{1.384817in}}{\pgfqpoint{1.091572in}{1.382622in}}{\pgfqpoint{1.087665in}{1.378715in}}%
\pgfpathcurveto{\pgfqpoint{1.083758in}{1.374808in}}{\pgfqpoint{1.081563in}{1.369509in}}{\pgfqpoint{1.081563in}{1.363984in}}%
\pgfpathcurveto{\pgfqpoint{1.081563in}{1.358458in}}{\pgfqpoint{1.083758in}{1.353159in}}{\pgfqpoint{1.087665in}{1.349252in}}%
\pgfpathcurveto{\pgfqpoint{1.091572in}{1.345345in}}{\pgfqpoint{1.096871in}{1.343150in}}{\pgfqpoint{1.102396in}{1.343150in}}%
\pgfpathclose%
\pgfusepath{fill}%
\end{pgfscope}%
\begin{pgfscope}%
\pgfpathrectangle{\pgfqpoint{0.889225in}{0.913832in}}{\pgfqpoint{1.162500in}{0.755000in}}%
\pgfusepath{clip}%
\pgfsetbuttcap%
\pgfsetroundjoin%
\definecolor{currentfill}{rgb}{0.000000,0.000000,0.000000}%
\pgfsetfillcolor{currentfill}%
\pgfsetfillopacity{0.500000}%
\pgfsetlinewidth{0.000000pt}%
\definecolor{currentstroke}{rgb}{0.000000,0.000000,0.000000}%
\pgfsetstrokecolor{currentstroke}%
\pgfsetdash{}{0pt}%
\pgfpathmoveto{\pgfqpoint{1.143186in}{1.259718in}}%
\pgfpathcurveto{\pgfqpoint{1.148711in}{1.259718in}}{\pgfqpoint{1.154011in}{1.261913in}}{\pgfqpoint{1.157918in}{1.265820in}}%
\pgfpathcurveto{\pgfqpoint{1.161825in}{1.269726in}}{\pgfqpoint{1.164020in}{1.275026in}}{\pgfqpoint{1.164020in}{1.280551in}}%
\pgfpathcurveto{\pgfqpoint{1.164020in}{1.286076in}}{\pgfqpoint{1.161825in}{1.291376in}}{\pgfqpoint{1.157918in}{1.295282in}}%
\pgfpathcurveto{\pgfqpoint{1.154011in}{1.299189in}}{\pgfqpoint{1.148711in}{1.301384in}}{\pgfqpoint{1.143186in}{1.301384in}}%
\pgfpathcurveto{\pgfqpoint{1.137661in}{1.301384in}}{\pgfqpoint{1.132362in}{1.299189in}}{\pgfqpoint{1.128455in}{1.295282in}}%
\pgfpathcurveto{\pgfqpoint{1.124548in}{1.291376in}}{\pgfqpoint{1.122353in}{1.286076in}}{\pgfqpoint{1.122353in}{1.280551in}}%
\pgfpathcurveto{\pgfqpoint{1.122353in}{1.275026in}}{\pgfqpoint{1.124548in}{1.269726in}}{\pgfqpoint{1.128455in}{1.265820in}}%
\pgfpathcurveto{\pgfqpoint{1.132362in}{1.261913in}}{\pgfqpoint{1.137661in}{1.259718in}}{\pgfqpoint{1.143186in}{1.259718in}}%
\pgfpathclose%
\pgfusepath{fill}%
\end{pgfscope}%
\begin{pgfscope}%
\pgfpathrectangle{\pgfqpoint{0.889225in}{0.913832in}}{\pgfqpoint{1.162500in}{0.755000in}}%
\pgfusepath{clip}%
\pgfsetbuttcap%
\pgfsetroundjoin%
\definecolor{currentfill}{rgb}{0.000000,0.000000,0.000000}%
\pgfsetfillcolor{currentfill}%
\pgfsetfillopacity{0.500000}%
\pgfsetlinewidth{0.000000pt}%
\definecolor{currentstroke}{rgb}{0.000000,0.000000,0.000000}%
\pgfsetstrokecolor{currentstroke}%
\pgfsetdash{}{0pt}%
\pgfpathmoveto{\pgfqpoint{2.024046in}{1.167350in}}%
\pgfpathcurveto{\pgfqpoint{2.029571in}{1.167350in}}{\pgfqpoint{2.034871in}{1.169545in}}{\pgfqpoint{2.038778in}{1.173452in}}%
\pgfpathcurveto{\pgfqpoint{2.042685in}{1.177359in}}{\pgfqpoint{2.044880in}{1.182658in}}{\pgfqpoint{2.044880in}{1.188183in}}%
\pgfpathcurveto{\pgfqpoint{2.044880in}{1.193708in}}{\pgfqpoint{2.042685in}{1.199008in}}{\pgfqpoint{2.038778in}{1.202915in}}%
\pgfpathcurveto{\pgfqpoint{2.034871in}{1.206822in}}{\pgfqpoint{2.029571in}{1.209017in}}{\pgfqpoint{2.024046in}{1.209017in}}%
\pgfpathcurveto{\pgfqpoint{2.018521in}{1.209017in}}{\pgfqpoint{2.013222in}{1.206822in}}{\pgfqpoint{2.009315in}{1.202915in}}%
\pgfpathcurveto{\pgfqpoint{2.005408in}{1.199008in}}{\pgfqpoint{2.003213in}{1.193708in}}{\pgfqpoint{2.003213in}{1.188183in}}%
\pgfpathcurveto{\pgfqpoint{2.003213in}{1.182658in}}{\pgfqpoint{2.005408in}{1.177359in}}{\pgfqpoint{2.009315in}{1.173452in}}%
\pgfpathcurveto{\pgfqpoint{2.013222in}{1.169545in}}{\pgfqpoint{2.018521in}{1.167350in}}{\pgfqpoint{2.024046in}{1.167350in}}%
\pgfpathclose%
\pgfusepath{fill}%
\end{pgfscope}%
\begin{pgfscope}%
\pgfpathrectangle{\pgfqpoint{0.889225in}{0.913832in}}{\pgfqpoint{1.162500in}{0.755000in}}%
\pgfusepath{clip}%
\pgfsetbuttcap%
\pgfsetroundjoin%
\definecolor{currentfill}{rgb}{0.000000,0.000000,0.000000}%
\pgfsetfillcolor{currentfill}%
\pgfsetfillopacity{0.500000}%
\pgfsetlinewidth{0.000000pt}%
\definecolor{currentstroke}{rgb}{0.000000,0.000000,0.000000}%
\pgfsetstrokecolor{currentstroke}%
\pgfsetdash{}{0pt}%
\pgfpathmoveto{\pgfqpoint{1.402285in}{1.034902in}}%
\pgfpathcurveto{\pgfqpoint{1.407810in}{1.034902in}}{\pgfqpoint{1.413110in}{1.037097in}}{\pgfqpoint{1.417016in}{1.041004in}}%
\pgfpathcurveto{\pgfqpoint{1.420923in}{1.044911in}}{\pgfqpoint{1.423118in}{1.050211in}}{\pgfqpoint{1.423118in}{1.055736in}}%
\pgfpathcurveto{\pgfqpoint{1.423118in}{1.061261in}}{\pgfqpoint{1.420923in}{1.066560in}}{\pgfqpoint{1.417016in}{1.070467in}}%
\pgfpathcurveto{\pgfqpoint{1.413110in}{1.074374in}}{\pgfqpoint{1.407810in}{1.076569in}}{\pgfqpoint{1.402285in}{1.076569in}}%
\pgfpathcurveto{\pgfqpoint{1.396760in}{1.076569in}}{\pgfqpoint{1.391460in}{1.074374in}}{\pgfqpoint{1.387554in}{1.070467in}}%
\pgfpathcurveto{\pgfqpoint{1.383647in}{1.066560in}}{\pgfqpoint{1.381452in}{1.061261in}}{\pgfqpoint{1.381452in}{1.055736in}}%
\pgfpathcurveto{\pgfqpoint{1.381452in}{1.050211in}}{\pgfqpoint{1.383647in}{1.044911in}}{\pgfqpoint{1.387554in}{1.041004in}}%
\pgfpathcurveto{\pgfqpoint{1.391460in}{1.037097in}}{\pgfqpoint{1.396760in}{1.034902in}}{\pgfqpoint{1.402285in}{1.034902in}}%
\pgfpathclose%
\pgfusepath{fill}%
\end{pgfscope}%
\begin{pgfscope}%
\pgfpathrectangle{\pgfqpoint{0.889225in}{0.913832in}}{\pgfqpoint{1.162500in}{0.755000in}}%
\pgfusepath{clip}%
\pgfsetbuttcap%
\pgfsetroundjoin%
\definecolor{currentfill}{rgb}{0.000000,0.000000,0.000000}%
\pgfsetfillcolor{currentfill}%
\pgfsetfillopacity{0.500000}%
\pgfsetlinewidth{0.000000pt}%
\definecolor{currentstroke}{rgb}{0.000000,0.000000,0.000000}%
\pgfsetstrokecolor{currentstroke}%
\pgfsetdash{}{0pt}%
\pgfpathmoveto{\pgfqpoint{0.973483in}{1.190157in}}%
\pgfpathcurveto{\pgfqpoint{0.979008in}{1.190157in}}{\pgfqpoint{0.984308in}{1.192352in}}{\pgfqpoint{0.988214in}{1.196259in}}%
\pgfpathcurveto{\pgfqpoint{0.992121in}{1.200166in}}{\pgfqpoint{0.994316in}{1.205465in}}{\pgfqpoint{0.994316in}{1.210991in}}%
\pgfpathcurveto{\pgfqpoint{0.994316in}{1.216516in}}{\pgfqpoint{0.992121in}{1.221815in}}{\pgfqpoint{0.988214in}{1.225722in}}%
\pgfpathcurveto{\pgfqpoint{0.984308in}{1.229629in}}{\pgfqpoint{0.979008in}{1.231824in}}{\pgfqpoint{0.973483in}{1.231824in}}%
\pgfpathcurveto{\pgfqpoint{0.967958in}{1.231824in}}{\pgfqpoint{0.962658in}{1.229629in}}{\pgfqpoint{0.958752in}{1.225722in}}%
\pgfpathcurveto{\pgfqpoint{0.954845in}{1.221815in}}{\pgfqpoint{0.952650in}{1.216516in}}{\pgfqpoint{0.952650in}{1.210991in}}%
\pgfpathcurveto{\pgfqpoint{0.952650in}{1.205465in}}{\pgfqpoint{0.954845in}{1.200166in}}{\pgfqpoint{0.958752in}{1.196259in}}%
\pgfpathcurveto{\pgfqpoint{0.962658in}{1.192352in}}{\pgfqpoint{0.967958in}{1.190157in}}{\pgfqpoint{0.973483in}{1.190157in}}%
\pgfpathclose%
\pgfusepath{fill}%
\end{pgfscope}%
\begin{pgfscope}%
\pgfsetbuttcap%
\pgfsetroundjoin%
\definecolor{currentfill}{rgb}{0.000000,0.000000,0.000000}%
\pgfsetfillcolor{currentfill}%
\pgfsetlinewidth{0.803000pt}%
\definecolor{currentstroke}{rgb}{0.000000,0.000000,0.000000}%
\pgfsetstrokecolor{currentstroke}%
\pgfsetdash{}{0pt}%
\pgfsys@defobject{currentmarker}{\pgfqpoint{0.000000in}{-0.048611in}}{\pgfqpoint{0.000000in}{0.000000in}}{%
\pgfpathmoveto{\pgfqpoint{0.000000in}{0.000000in}}%
\pgfpathlineto{\pgfqpoint{0.000000in}{-0.048611in}}%
\pgfusepath{stroke,fill}%
}%
\begin{pgfscope}%
\pgfsys@transformshift{1.236821in}{0.913832in}%
\pgfsys@useobject{currentmarker}{}%
\end{pgfscope}%
\end{pgfscope}%
\begin{pgfscope}%
\pgftext[x=1.267474in,y=0.370616in,left,base,rotate=90.000000]{\rmfamily\fontsize{8.000000}{9.600000}\selectfont \(\displaystyle 0.000025\)}%
\end{pgfscope}%
\begin{pgfscope}%
\pgfsetbuttcap%
\pgfsetroundjoin%
\definecolor{currentfill}{rgb}{0.000000,0.000000,0.000000}%
\pgfsetfillcolor{currentfill}%
\pgfsetlinewidth{0.803000pt}%
\definecolor{currentstroke}{rgb}{0.000000,0.000000,0.000000}%
\pgfsetstrokecolor{currentstroke}%
\pgfsetdash{}{0pt}%
\pgfsys@defobject{currentmarker}{\pgfqpoint{0.000000in}{-0.048611in}}{\pgfqpoint{0.000000in}{0.000000in}}{%
\pgfpathmoveto{\pgfqpoint{0.000000in}{0.000000in}}%
\pgfpathlineto{\pgfqpoint{0.000000in}{-0.048611in}}%
\pgfusepath{stroke,fill}%
}%
\begin{pgfscope}%
\pgfsys@transformshift{1.758350in}{0.913832in}%
\pgfsys@useobject{currentmarker}{}%
\end{pgfscope}%
\end{pgfscope}%
\begin{pgfscope}%
\pgftext[x=1.789004in,y=0.370616in,left,base,rotate=90.000000]{\rmfamily\fontsize{8.000000}{9.600000}\selectfont \(\displaystyle 0.000050\)}%
\end{pgfscope}%
\begin{pgfscope}%
\pgftext[x=1.470475in,y=0.315061in,,top]{\rmfamily\fontsize{16.000000}{19.200000}\selectfont area}%
\end{pgfscope}%
\begin{pgfscope}%
\pgfsetbuttcap%
\pgfsetroundjoin%
\definecolor{currentfill}{rgb}{0.000000,0.000000,0.000000}%
\pgfsetfillcolor{currentfill}%
\pgfsetlinewidth{0.803000pt}%
\definecolor{currentstroke}{rgb}{0.000000,0.000000,0.000000}%
\pgfsetstrokecolor{currentstroke}%
\pgfsetdash{}{0pt}%
\pgfsys@defobject{currentmarker}{\pgfqpoint{-0.048611in}{0.000000in}}{\pgfqpoint{0.000000in}{0.000000in}}{%
\pgfpathmoveto{\pgfqpoint{0.000000in}{0.000000in}}%
\pgfpathlineto{\pgfqpoint{-0.048611in}{0.000000in}}%
\pgfusepath{stroke,fill}%
}%
\begin{pgfscope}%
\pgfsys@transformshift{0.889225in}{0.965005in}%
\pgfsys@useobject{currentmarker}{}%
\end{pgfscope}%
\end{pgfscope}%
\begin{pgfscope}%
\pgftext[x=0.641152in,y=0.922796in,left,base]{\rmfamily\fontsize{8.000000}{9.600000}\selectfont \(\displaystyle 0.5\)}%
\end{pgfscope}%
\begin{pgfscope}%
\pgfsetbuttcap%
\pgfsetroundjoin%
\definecolor{currentfill}{rgb}{0.000000,0.000000,0.000000}%
\pgfsetfillcolor{currentfill}%
\pgfsetlinewidth{0.803000pt}%
\definecolor{currentstroke}{rgb}{0.000000,0.000000,0.000000}%
\pgfsetstrokecolor{currentstroke}%
\pgfsetdash{}{0pt}%
\pgfsys@defobject{currentmarker}{\pgfqpoint{-0.048611in}{0.000000in}}{\pgfqpoint{0.000000in}{0.000000in}}{%
\pgfpathmoveto{\pgfqpoint{0.000000in}{0.000000in}}%
\pgfpathlineto{\pgfqpoint{-0.048611in}{0.000000in}}%
\pgfusepath{stroke,fill}%
}%
\begin{pgfscope}%
\pgfsys@transformshift{0.889225in}{1.260040in}%
\pgfsys@useobject{currentmarker}{}%
\end{pgfscope}%
\end{pgfscope}%
\begin{pgfscope}%
\pgftext[x=0.641152in,y=1.217831in,left,base]{\rmfamily\fontsize{8.000000}{9.600000}\selectfont \(\displaystyle 1.0\)}%
\end{pgfscope}%
\begin{pgfscope}%
\pgfsetbuttcap%
\pgfsetroundjoin%
\definecolor{currentfill}{rgb}{0.000000,0.000000,0.000000}%
\pgfsetfillcolor{currentfill}%
\pgfsetlinewidth{0.803000pt}%
\definecolor{currentstroke}{rgb}{0.000000,0.000000,0.000000}%
\pgfsetstrokecolor{currentstroke}%
\pgfsetdash{}{0pt}%
\pgfsys@defobject{currentmarker}{\pgfqpoint{-0.048611in}{0.000000in}}{\pgfqpoint{0.000000in}{0.000000in}}{%
\pgfpathmoveto{\pgfqpoint{0.000000in}{0.000000in}}%
\pgfpathlineto{\pgfqpoint{-0.048611in}{0.000000in}}%
\pgfusepath{stroke,fill}%
}%
\begin{pgfscope}%
\pgfsys@transformshift{0.889225in}{1.555075in}%
\pgfsys@useobject{currentmarker}{}%
\end{pgfscope}%
\end{pgfscope}%
\begin{pgfscope}%
\pgftext[x=0.641152in,y=1.512866in,left,base]{\rmfamily\fontsize{8.000000}{9.600000}\selectfont \(\displaystyle 1.5\)}%
\end{pgfscope}%
\begin{pgfscope}%
\pgftext[x=0.585596in,y=1.291332in,,bottom,rotate=90.000000]{\rmfamily\fontsize{16.000000}{19.200000}\selectfont Ef0}%
\end{pgfscope}%
\begin{pgfscope}%
\pgfsetrectcap%
\pgfsetmiterjoin%
\pgfsetlinewidth{0.803000pt}%
\definecolor{currentstroke}{rgb}{0.501961,0.501961,0.501961}%
\pgfsetstrokecolor{currentstroke}%
\pgfsetdash{}{0pt}%
\pgfpathmoveto{\pgfqpoint{0.889225in}{0.913832in}}%
\pgfpathlineto{\pgfqpoint{0.889225in}{1.668832in}}%
\pgfusepath{stroke}%
\end{pgfscope}%
\begin{pgfscope}%
\pgfsetrectcap%
\pgfsetmiterjoin%
\pgfsetlinewidth{0.803000pt}%
\definecolor{currentstroke}{rgb}{0.501961,0.501961,0.501961}%
\pgfsetstrokecolor{currentstroke}%
\pgfsetdash{}{0pt}%
\pgfpathmoveto{\pgfqpoint{2.051725in}{0.913832in}}%
\pgfpathlineto{\pgfqpoint{2.051725in}{1.668832in}}%
\pgfusepath{stroke}%
\end{pgfscope}%
\begin{pgfscope}%
\pgfsetrectcap%
\pgfsetmiterjoin%
\pgfsetlinewidth{0.803000pt}%
\definecolor{currentstroke}{rgb}{0.501961,0.501961,0.501961}%
\pgfsetstrokecolor{currentstroke}%
\pgfsetdash{}{0pt}%
\pgfpathmoveto{\pgfqpoint{0.889225in}{0.913832in}}%
\pgfpathlineto{\pgfqpoint{2.051725in}{0.913832in}}%
\pgfusepath{stroke}%
\end{pgfscope}%
\begin{pgfscope}%
\pgfsetrectcap%
\pgfsetmiterjoin%
\pgfsetlinewidth{0.803000pt}%
\definecolor{currentstroke}{rgb}{0.501961,0.501961,0.501961}%
\pgfsetstrokecolor{currentstroke}%
\pgfsetdash{}{0pt}%
\pgfpathmoveto{\pgfqpoint{0.889225in}{1.668832in}}%
\pgfpathlineto{\pgfqpoint{2.051725in}{1.668832in}}%
\pgfusepath{stroke}%
\end{pgfscope}%
\begin{pgfscope}%
\pgfsetbuttcap%
\pgfsetmiterjoin%
\definecolor{currentfill}{rgb}{1.000000,1.000000,1.000000}%
\pgfsetfillcolor{currentfill}%
\pgfsetlinewidth{0.000000pt}%
\definecolor{currentstroke}{rgb}{0.000000,0.000000,0.000000}%
\pgfsetstrokecolor{currentstroke}%
\pgfsetstrokeopacity{0.000000}%
\pgfsetdash{}{0pt}%
\pgfpathmoveto{\pgfqpoint{2.051725in}{0.913832in}}%
\pgfpathlineto{\pgfqpoint{3.214225in}{0.913832in}}%
\pgfpathlineto{\pgfqpoint{3.214225in}{1.668832in}}%
\pgfpathlineto{\pgfqpoint{2.051725in}{1.668832in}}%
\pgfpathclose%
\pgfusepath{fill}%
\end{pgfscope}%
\begin{pgfscope}%
\pgfpathrectangle{\pgfqpoint{2.051725in}{0.913832in}}{\pgfqpoint{1.162500in}{0.755000in}}%
\pgfusepath{clip}%
\pgfsetbuttcap%
\pgfsetroundjoin%
\definecolor{currentfill}{rgb}{0.000000,0.000000,0.000000}%
\pgfsetfillcolor{currentfill}%
\pgfsetfillopacity{0.500000}%
\pgfsetlinewidth{0.000000pt}%
\definecolor{currentstroke}{rgb}{0.000000,0.000000,0.000000}%
\pgfsetstrokecolor{currentstroke}%
\pgfsetdash{}{0pt}%
\pgfpathmoveto{\pgfqpoint{3.107569in}{1.437401in}}%
\pgfpathcurveto{\pgfqpoint{3.113094in}{1.437401in}}{\pgfqpoint{3.118394in}{1.439596in}}{\pgfqpoint{3.122301in}{1.443503in}}%
\pgfpathcurveto{\pgfqpoint{3.126207in}{1.447410in}}{\pgfqpoint{3.128403in}{1.452710in}}{\pgfqpoint{3.128403in}{1.458235in}}%
\pgfpathcurveto{\pgfqpoint{3.128403in}{1.463760in}}{\pgfqpoint{3.126207in}{1.469059in}}{\pgfqpoint{3.122301in}{1.472966in}}%
\pgfpathcurveto{\pgfqpoint{3.118394in}{1.476873in}}{\pgfqpoint{3.113094in}{1.479068in}}{\pgfqpoint{3.107569in}{1.479068in}}%
\pgfpathcurveto{\pgfqpoint{3.102044in}{1.479068in}}{\pgfqpoint{3.096745in}{1.476873in}}{\pgfqpoint{3.092838in}{1.472966in}}%
\pgfpathcurveto{\pgfqpoint{3.088931in}{1.469059in}}{\pgfqpoint{3.086736in}{1.463760in}}{\pgfqpoint{3.086736in}{1.458235in}}%
\pgfpathcurveto{\pgfqpoint{3.086736in}{1.452710in}}{\pgfqpoint{3.088931in}{1.447410in}}{\pgfqpoint{3.092838in}{1.443503in}}%
\pgfpathcurveto{\pgfqpoint{3.096745in}{1.439596in}}{\pgfqpoint{3.102044in}{1.437401in}}{\pgfqpoint{3.107569in}{1.437401in}}%
\pgfpathclose%
\pgfusepath{fill}%
\end{pgfscope}%
\begin{pgfscope}%
\pgfpathrectangle{\pgfqpoint{2.051725in}{0.913832in}}{\pgfqpoint{1.162500in}{0.755000in}}%
\pgfusepath{clip}%
\pgfsetbuttcap%
\pgfsetroundjoin%
\definecolor{currentfill}{rgb}{0.000000,0.000000,0.000000}%
\pgfsetfillcolor{currentfill}%
\pgfsetfillopacity{0.500000}%
\pgfsetlinewidth{0.000000pt}%
\definecolor{currentstroke}{rgb}{0.000000,0.000000,0.000000}%
\pgfsetstrokecolor{currentstroke}%
\pgfsetdash{}{0pt}%
\pgfpathmoveto{\pgfqpoint{2.270173in}{0.987180in}}%
\pgfpathcurveto{\pgfqpoint{2.275698in}{0.987180in}}{\pgfqpoint{2.280997in}{0.989375in}}{\pgfqpoint{2.284904in}{0.993282in}}%
\pgfpathcurveto{\pgfqpoint{2.288811in}{0.997189in}}{\pgfqpoint{2.291006in}{1.002488in}}{\pgfqpoint{2.291006in}{1.008013in}}%
\pgfpathcurveto{\pgfqpoint{2.291006in}{1.013538in}}{\pgfqpoint{2.288811in}{1.018838in}}{\pgfqpoint{2.284904in}{1.022745in}}%
\pgfpathcurveto{\pgfqpoint{2.280997in}{1.026652in}}{\pgfqpoint{2.275698in}{1.028847in}}{\pgfqpoint{2.270173in}{1.028847in}}%
\pgfpathcurveto{\pgfqpoint{2.264648in}{1.028847in}}{\pgfqpoint{2.259348in}{1.026652in}}{\pgfqpoint{2.255441in}{1.022745in}}%
\pgfpathcurveto{\pgfqpoint{2.251535in}{1.018838in}}{\pgfqpoint{2.249339in}{1.013538in}}{\pgfqpoint{2.249339in}{1.008013in}}%
\pgfpathcurveto{\pgfqpoint{2.249339in}{1.002488in}}{\pgfqpoint{2.251535in}{0.997189in}}{\pgfqpoint{2.255441in}{0.993282in}}%
\pgfpathcurveto{\pgfqpoint{2.259348in}{0.989375in}}{\pgfqpoint{2.264648in}{0.987180in}}{\pgfqpoint{2.270173in}{0.987180in}}%
\pgfpathclose%
\pgfusepath{fill}%
\end{pgfscope}%
\begin{pgfscope}%
\pgfpathrectangle{\pgfqpoint{2.051725in}{0.913832in}}{\pgfqpoint{1.162500in}{0.755000in}}%
\pgfusepath{clip}%
\pgfsetbuttcap%
\pgfsetroundjoin%
\definecolor{currentfill}{rgb}{0.000000,0.000000,0.000000}%
\pgfsetfillcolor{currentfill}%
\pgfsetfillopacity{0.500000}%
\pgfsetlinewidth{0.000000pt}%
\definecolor{currentstroke}{rgb}{0.000000,0.000000,0.000000}%
\pgfsetstrokecolor{currentstroke}%
\pgfsetdash{}{0pt}%
\pgfpathmoveto{\pgfqpoint{2.264867in}{0.939886in}}%
\pgfpathcurveto{\pgfqpoint{2.270392in}{0.939886in}}{\pgfqpoint{2.275692in}{0.942081in}}{\pgfqpoint{2.279599in}{0.945988in}}%
\pgfpathcurveto{\pgfqpoint{2.283506in}{0.949895in}}{\pgfqpoint{2.285701in}{0.955195in}}{\pgfqpoint{2.285701in}{0.960720in}}%
\pgfpathcurveto{\pgfqpoint{2.285701in}{0.966245in}}{\pgfqpoint{2.283506in}{0.971544in}}{\pgfqpoint{2.279599in}{0.975451in}}%
\pgfpathcurveto{\pgfqpoint{2.275692in}{0.979358in}}{\pgfqpoint{2.270392in}{0.981553in}}{\pgfqpoint{2.264867in}{0.981553in}}%
\pgfpathcurveto{\pgfqpoint{2.259342in}{0.981553in}}{\pgfqpoint{2.254043in}{0.979358in}}{\pgfqpoint{2.250136in}{0.975451in}}%
\pgfpathcurveto{\pgfqpoint{2.246229in}{0.971544in}}{\pgfqpoint{2.244034in}{0.966245in}}{\pgfqpoint{2.244034in}{0.960720in}}%
\pgfpathcurveto{\pgfqpoint{2.244034in}{0.955195in}}{\pgfqpoint{2.246229in}{0.949895in}}{\pgfqpoint{2.250136in}{0.945988in}}%
\pgfpathcurveto{\pgfqpoint{2.254043in}{0.942081in}}{\pgfqpoint{2.259342in}{0.939886in}}{\pgfqpoint{2.264867in}{0.939886in}}%
\pgfpathclose%
\pgfusepath{fill}%
\end{pgfscope}%
\begin{pgfscope}%
\pgfpathrectangle{\pgfqpoint{2.051725in}{0.913832in}}{\pgfqpoint{1.162500in}{0.755000in}}%
\pgfusepath{clip}%
\pgfsetbuttcap%
\pgfsetroundjoin%
\definecolor{currentfill}{rgb}{0.000000,0.000000,0.000000}%
\pgfsetfillcolor{currentfill}%
\pgfsetfillopacity{0.500000}%
\pgfsetlinewidth{0.000000pt}%
\definecolor{currentstroke}{rgb}{0.000000,0.000000,0.000000}%
\pgfsetstrokecolor{currentstroke}%
\pgfsetdash{}{0pt}%
\pgfpathmoveto{\pgfqpoint{2.154193in}{1.105715in}}%
\pgfpathcurveto{\pgfqpoint{2.159718in}{1.105715in}}{\pgfqpoint{2.165018in}{1.107910in}}{\pgfqpoint{2.168924in}{1.111817in}}%
\pgfpathcurveto{\pgfqpoint{2.172831in}{1.115723in}}{\pgfqpoint{2.175026in}{1.121023in}}{\pgfqpoint{2.175026in}{1.126548in}}%
\pgfpathcurveto{\pgfqpoint{2.175026in}{1.132073in}}{\pgfqpoint{2.172831in}{1.137372in}}{\pgfqpoint{2.168924in}{1.141279in}}%
\pgfpathcurveto{\pgfqpoint{2.165018in}{1.145186in}}{\pgfqpoint{2.159718in}{1.147381in}}{\pgfqpoint{2.154193in}{1.147381in}}%
\pgfpathcurveto{\pgfqpoint{2.148668in}{1.147381in}}{\pgfqpoint{2.143368in}{1.145186in}}{\pgfqpoint{2.139462in}{1.141279in}}%
\pgfpathcurveto{\pgfqpoint{2.135555in}{1.137372in}}{\pgfqpoint{2.133360in}{1.132073in}}{\pgfqpoint{2.133360in}{1.126548in}}%
\pgfpathcurveto{\pgfqpoint{2.133360in}{1.121023in}}{\pgfqpoint{2.135555in}{1.115723in}}{\pgfqpoint{2.139462in}{1.111817in}}%
\pgfpathcurveto{\pgfqpoint{2.143368in}{1.107910in}}{\pgfqpoint{2.148668in}{1.105715in}}{\pgfqpoint{2.154193in}{1.105715in}}%
\pgfpathclose%
\pgfusepath{fill}%
\end{pgfscope}%
\begin{pgfscope}%
\pgfpathrectangle{\pgfqpoint{2.051725in}{0.913832in}}{\pgfqpoint{1.162500in}{0.755000in}}%
\pgfusepath{clip}%
\pgfsetbuttcap%
\pgfsetroundjoin%
\definecolor{currentfill}{rgb}{0.000000,0.000000,0.000000}%
\pgfsetfillcolor{currentfill}%
\pgfsetfillopacity{0.500000}%
\pgfsetlinewidth{0.000000pt}%
\definecolor{currentstroke}{rgb}{0.000000,0.000000,0.000000}%
\pgfsetstrokecolor{currentstroke}%
\pgfsetdash{}{0pt}%
\pgfpathmoveto{\pgfqpoint{2.183344in}{0.910975in}}%
\pgfpathcurveto{\pgfqpoint{2.188869in}{0.910975in}}{\pgfqpoint{2.194169in}{0.913170in}}{\pgfqpoint{2.198076in}{0.917077in}}%
\pgfpathcurveto{\pgfqpoint{2.201982in}{0.920984in}}{\pgfqpoint{2.204178in}{0.926283in}}{\pgfqpoint{2.204178in}{0.931809in}}%
\pgfpathcurveto{\pgfqpoint{2.204178in}{0.937334in}}{\pgfqpoint{2.201982in}{0.942633in}}{\pgfqpoint{2.198076in}{0.946540in}}%
\pgfpathcurveto{\pgfqpoint{2.194169in}{0.950447in}}{\pgfqpoint{2.188869in}{0.952642in}}{\pgfqpoint{2.183344in}{0.952642in}}%
\pgfpathcurveto{\pgfqpoint{2.177819in}{0.952642in}}{\pgfqpoint{2.172520in}{0.950447in}}{\pgfqpoint{2.168613in}{0.946540in}}%
\pgfpathcurveto{\pgfqpoint{2.164706in}{0.942633in}}{\pgfqpoint{2.162511in}{0.937334in}}{\pgfqpoint{2.162511in}{0.931809in}}%
\pgfpathcurveto{\pgfqpoint{2.162511in}{0.926283in}}{\pgfqpoint{2.164706in}{0.920984in}}{\pgfqpoint{2.168613in}{0.917077in}}%
\pgfpathcurveto{\pgfqpoint{2.172520in}{0.913170in}}{\pgfqpoint{2.177819in}{0.910975in}}{\pgfqpoint{2.183344in}{0.910975in}}%
\pgfpathclose%
\pgfusepath{fill}%
\end{pgfscope}%
\begin{pgfscope}%
\pgfpathrectangle{\pgfqpoint{2.051725in}{0.913832in}}{\pgfqpoint{1.162500in}{0.755000in}}%
\pgfusepath{clip}%
\pgfsetbuttcap%
\pgfsetroundjoin%
\definecolor{currentfill}{rgb}{0.000000,0.000000,0.000000}%
\pgfsetfillcolor{currentfill}%
\pgfsetfillopacity{0.500000}%
\pgfsetlinewidth{0.000000pt}%
\definecolor{currentstroke}{rgb}{0.000000,0.000000,0.000000}%
\pgfsetstrokecolor{currentstroke}%
\pgfsetdash{}{0pt}%
\pgfpathmoveto{\pgfqpoint{2.082232in}{1.042931in}}%
\pgfpathcurveto{\pgfqpoint{2.087757in}{1.042931in}}{\pgfqpoint{2.093056in}{1.045126in}}{\pgfqpoint{2.096963in}{1.049033in}}%
\pgfpathcurveto{\pgfqpoint{2.100870in}{1.052940in}}{\pgfqpoint{2.103065in}{1.058239in}}{\pgfqpoint{2.103065in}{1.063764in}}%
\pgfpathcurveto{\pgfqpoint{2.103065in}{1.069289in}}{\pgfqpoint{2.100870in}{1.074589in}}{\pgfqpoint{2.096963in}{1.078495in}}%
\pgfpathcurveto{\pgfqpoint{2.093056in}{1.082402in}}{\pgfqpoint{2.087757in}{1.084597in}}{\pgfqpoint{2.082232in}{1.084597in}}%
\pgfpathcurveto{\pgfqpoint{2.076707in}{1.084597in}}{\pgfqpoint{2.071407in}{1.082402in}}{\pgfqpoint{2.067500in}{1.078495in}}%
\pgfpathcurveto{\pgfqpoint{2.063594in}{1.074589in}}{\pgfqpoint{2.061398in}{1.069289in}}{\pgfqpoint{2.061398in}{1.063764in}}%
\pgfpathcurveto{\pgfqpoint{2.061398in}{1.058239in}}{\pgfqpoint{2.063594in}{1.052940in}}{\pgfqpoint{2.067500in}{1.049033in}}%
\pgfpathcurveto{\pgfqpoint{2.071407in}{1.045126in}}{\pgfqpoint{2.076707in}{1.042931in}}{\pgfqpoint{2.082232in}{1.042931in}}%
\pgfpathclose%
\pgfusepath{fill}%
\end{pgfscope}%
\begin{pgfscope}%
\pgfpathrectangle{\pgfqpoint{2.051725in}{0.913832in}}{\pgfqpoint{1.162500in}{0.755000in}}%
\pgfusepath{clip}%
\pgfsetbuttcap%
\pgfsetroundjoin%
\definecolor{currentfill}{rgb}{0.000000,0.000000,0.000000}%
\pgfsetfillcolor{currentfill}%
\pgfsetfillopacity{0.500000}%
\pgfsetlinewidth{0.000000pt}%
\definecolor{currentstroke}{rgb}{0.000000,0.000000,0.000000}%
\pgfsetstrokecolor{currentstroke}%
\pgfsetdash{}{0pt}%
\pgfpathmoveto{\pgfqpoint{2.079404in}{1.035607in}}%
\pgfpathcurveto{\pgfqpoint{2.084929in}{1.035607in}}{\pgfqpoint{2.090228in}{1.037803in}}{\pgfqpoint{2.094135in}{1.041709in}}%
\pgfpathcurveto{\pgfqpoint{2.098042in}{1.045616in}}{\pgfqpoint{2.100237in}{1.050916in}}{\pgfqpoint{2.100237in}{1.056441in}}%
\pgfpathcurveto{\pgfqpoint{2.100237in}{1.061966in}}{\pgfqpoint{2.098042in}{1.067265in}}{\pgfqpoint{2.094135in}{1.071172in}}%
\pgfpathcurveto{\pgfqpoint{2.090228in}{1.075079in}}{\pgfqpoint{2.084929in}{1.077274in}}{\pgfqpoint{2.079404in}{1.077274in}}%
\pgfpathcurveto{\pgfqpoint{2.073879in}{1.077274in}}{\pgfqpoint{2.068579in}{1.075079in}}{\pgfqpoint{2.064672in}{1.071172in}}%
\pgfpathcurveto{\pgfqpoint{2.060765in}{1.067265in}}{\pgfqpoint{2.058570in}{1.061966in}}{\pgfqpoint{2.058570in}{1.056441in}}%
\pgfpathcurveto{\pgfqpoint{2.058570in}{1.050916in}}{\pgfqpoint{2.060765in}{1.045616in}}{\pgfqpoint{2.064672in}{1.041709in}}%
\pgfpathcurveto{\pgfqpoint{2.068579in}{1.037803in}}{\pgfqpoint{2.073879in}{1.035607in}}{\pgfqpoint{2.079404in}{1.035607in}}%
\pgfpathclose%
\pgfusepath{fill}%
\end{pgfscope}%
\begin{pgfscope}%
\pgfpathrectangle{\pgfqpoint{2.051725in}{0.913832in}}{\pgfqpoint{1.162500in}{0.755000in}}%
\pgfusepath{clip}%
\pgfsetbuttcap%
\pgfsetroundjoin%
\definecolor{currentfill}{rgb}{0.000000,0.000000,0.000000}%
\pgfsetfillcolor{currentfill}%
\pgfsetfillopacity{0.500000}%
\pgfsetlinewidth{0.000000pt}%
\definecolor{currentstroke}{rgb}{0.000000,0.000000,0.000000}%
\pgfsetstrokecolor{currentstroke}%
\pgfsetdash{}{0pt}%
\pgfpathmoveto{\pgfqpoint{2.536683in}{1.572595in}}%
\pgfpathcurveto{\pgfqpoint{2.542208in}{1.572595in}}{\pgfqpoint{2.547508in}{1.574791in}}{\pgfqpoint{2.551415in}{1.578697in}}%
\pgfpathcurveto{\pgfqpoint{2.555322in}{1.582604in}}{\pgfqpoint{2.557517in}{1.587904in}}{\pgfqpoint{2.557517in}{1.593429in}}%
\pgfpathcurveto{\pgfqpoint{2.557517in}{1.598954in}}{\pgfqpoint{2.555322in}{1.604253in}}{\pgfqpoint{2.551415in}{1.608160in}}%
\pgfpathcurveto{\pgfqpoint{2.547508in}{1.612067in}}{\pgfqpoint{2.542208in}{1.614262in}}{\pgfqpoint{2.536683in}{1.614262in}}%
\pgfpathcurveto{\pgfqpoint{2.531158in}{1.614262in}}{\pgfqpoint{2.525859in}{1.612067in}}{\pgfqpoint{2.521952in}{1.608160in}}%
\pgfpathcurveto{\pgfqpoint{2.518045in}{1.604253in}}{\pgfqpoint{2.515850in}{1.598954in}}{\pgfqpoint{2.515850in}{1.593429in}}%
\pgfpathcurveto{\pgfqpoint{2.515850in}{1.587904in}}{\pgfqpoint{2.518045in}{1.582604in}}{\pgfqpoint{2.521952in}{1.578697in}}%
\pgfpathcurveto{\pgfqpoint{2.525859in}{1.574791in}}{\pgfqpoint{2.531158in}{1.572595in}}{\pgfqpoint{2.536683in}{1.572595in}}%
\pgfpathclose%
\pgfusepath{fill}%
\end{pgfscope}%
\begin{pgfscope}%
\pgfpathrectangle{\pgfqpoint{2.051725in}{0.913832in}}{\pgfqpoint{1.162500in}{0.755000in}}%
\pgfusepath{clip}%
\pgfsetbuttcap%
\pgfsetroundjoin%
\definecolor{currentfill}{rgb}{0.000000,0.000000,0.000000}%
\pgfsetfillcolor{currentfill}%
\pgfsetfillopacity{0.500000}%
\pgfsetlinewidth{0.000000pt}%
\definecolor{currentstroke}{rgb}{0.000000,0.000000,0.000000}%
\pgfsetstrokecolor{currentstroke}%
\pgfsetdash{}{0pt}%
\pgfpathmoveto{\pgfqpoint{2.417381in}{1.630023in}}%
\pgfpathcurveto{\pgfqpoint{2.422906in}{1.630023in}}{\pgfqpoint{2.428205in}{1.632218in}}{\pgfqpoint{2.432112in}{1.636125in}}%
\pgfpathcurveto{\pgfqpoint{2.436019in}{1.640032in}}{\pgfqpoint{2.438214in}{1.645331in}}{\pgfqpoint{2.438214in}{1.650856in}}%
\pgfpathcurveto{\pgfqpoint{2.438214in}{1.656381in}}{\pgfqpoint{2.436019in}{1.661681in}}{\pgfqpoint{2.432112in}{1.665588in}}%
\pgfpathcurveto{\pgfqpoint{2.428205in}{1.669494in}}{\pgfqpoint{2.422906in}{1.671690in}}{\pgfqpoint{2.417381in}{1.671690in}}%
\pgfpathcurveto{\pgfqpoint{2.411856in}{1.671690in}}{\pgfqpoint{2.406556in}{1.669494in}}{\pgfqpoint{2.402649in}{1.665588in}}%
\pgfpathcurveto{\pgfqpoint{2.398743in}{1.661681in}}{\pgfqpoint{2.396547in}{1.656381in}}{\pgfqpoint{2.396547in}{1.650856in}}%
\pgfpathcurveto{\pgfqpoint{2.396547in}{1.645331in}}{\pgfqpoint{2.398743in}{1.640032in}}{\pgfqpoint{2.402649in}{1.636125in}}%
\pgfpathcurveto{\pgfqpoint{2.406556in}{1.632218in}}{\pgfqpoint{2.411856in}{1.630023in}}{\pgfqpoint{2.417381in}{1.630023in}}%
\pgfpathclose%
\pgfusepath{fill}%
\end{pgfscope}%
\begin{pgfscope}%
\pgfpathrectangle{\pgfqpoint{2.051725in}{0.913832in}}{\pgfqpoint{1.162500in}{0.755000in}}%
\pgfusepath{clip}%
\pgfsetbuttcap%
\pgfsetroundjoin%
\definecolor{currentfill}{rgb}{0.000000,0.000000,0.000000}%
\pgfsetfillcolor{currentfill}%
\pgfsetfillopacity{0.500000}%
\pgfsetlinewidth{0.000000pt}%
\definecolor{currentstroke}{rgb}{0.000000,0.000000,0.000000}%
\pgfsetstrokecolor{currentstroke}%
\pgfsetdash{}{0pt}%
\pgfpathmoveto{\pgfqpoint{2.412630in}{1.484185in}}%
\pgfpathcurveto{\pgfqpoint{2.418155in}{1.484185in}}{\pgfqpoint{2.423454in}{1.486380in}}{\pgfqpoint{2.427361in}{1.490287in}}%
\pgfpathcurveto{\pgfqpoint{2.431268in}{1.494194in}}{\pgfqpoint{2.433463in}{1.499493in}}{\pgfqpoint{2.433463in}{1.505018in}}%
\pgfpathcurveto{\pgfqpoint{2.433463in}{1.510544in}}{\pgfqpoint{2.431268in}{1.515843in}}{\pgfqpoint{2.427361in}{1.519750in}}%
\pgfpathcurveto{\pgfqpoint{2.423454in}{1.523657in}}{\pgfqpoint{2.418155in}{1.525852in}}{\pgfqpoint{2.412630in}{1.525852in}}%
\pgfpathcurveto{\pgfqpoint{2.407105in}{1.525852in}}{\pgfqpoint{2.401805in}{1.523657in}}{\pgfqpoint{2.397898in}{1.519750in}}%
\pgfpathcurveto{\pgfqpoint{2.393991in}{1.515843in}}{\pgfqpoint{2.391796in}{1.510544in}}{\pgfqpoint{2.391796in}{1.505018in}}%
\pgfpathcurveto{\pgfqpoint{2.391796in}{1.499493in}}{\pgfqpoint{2.393991in}{1.494194in}}{\pgfqpoint{2.397898in}{1.490287in}}%
\pgfpathcurveto{\pgfqpoint{2.401805in}{1.486380in}}{\pgfqpoint{2.407105in}{1.484185in}}{\pgfqpoint{2.412630in}{1.484185in}}%
\pgfpathclose%
\pgfusepath{fill}%
\end{pgfscope}%
\begin{pgfscope}%
\pgfpathrectangle{\pgfqpoint{2.051725in}{0.913832in}}{\pgfqpoint{1.162500in}{0.755000in}}%
\pgfusepath{clip}%
\pgfsetbuttcap%
\pgfsetroundjoin%
\definecolor{currentfill}{rgb}{0.000000,0.000000,0.000000}%
\pgfsetfillcolor{currentfill}%
\pgfsetfillopacity{0.500000}%
\pgfsetlinewidth{0.000000pt}%
\definecolor{currentstroke}{rgb}{0.000000,0.000000,0.000000}%
\pgfsetstrokecolor{currentstroke}%
\pgfsetdash{}{0pt}%
\pgfpathmoveto{\pgfqpoint{2.478188in}{1.067768in}}%
\pgfpathcurveto{\pgfqpoint{2.483713in}{1.067768in}}{\pgfqpoint{2.489012in}{1.069963in}}{\pgfqpoint{2.492919in}{1.073870in}}%
\pgfpathcurveto{\pgfqpoint{2.496826in}{1.077776in}}{\pgfqpoint{2.499021in}{1.083076in}}{\pgfqpoint{2.499021in}{1.088601in}}%
\pgfpathcurveto{\pgfqpoint{2.499021in}{1.094126in}}{\pgfqpoint{2.496826in}{1.099426in}}{\pgfqpoint{2.492919in}{1.103332in}}%
\pgfpathcurveto{\pgfqpoint{2.489012in}{1.107239in}}{\pgfqpoint{2.483713in}{1.109434in}}{\pgfqpoint{2.478188in}{1.109434in}}%
\pgfpathcurveto{\pgfqpoint{2.472663in}{1.109434in}}{\pgfqpoint{2.467363in}{1.107239in}}{\pgfqpoint{2.463456in}{1.103332in}}%
\pgfpathcurveto{\pgfqpoint{2.459549in}{1.099426in}}{\pgfqpoint{2.457354in}{1.094126in}}{\pgfqpoint{2.457354in}{1.088601in}}%
\pgfpathcurveto{\pgfqpoint{2.457354in}{1.083076in}}{\pgfqpoint{2.459549in}{1.077776in}}{\pgfqpoint{2.463456in}{1.073870in}}%
\pgfpathcurveto{\pgfqpoint{2.467363in}{1.069963in}}{\pgfqpoint{2.472663in}{1.067768in}}{\pgfqpoint{2.478188in}{1.067768in}}%
\pgfpathclose%
\pgfusepath{fill}%
\end{pgfscope}%
\begin{pgfscope}%
\pgfpathrectangle{\pgfqpoint{2.051725in}{0.913832in}}{\pgfqpoint{1.162500in}{0.755000in}}%
\pgfusepath{clip}%
\pgfsetbuttcap%
\pgfsetroundjoin%
\definecolor{currentfill}{rgb}{0.000000,0.000000,0.000000}%
\pgfsetfillcolor{currentfill}%
\pgfsetfillopacity{0.500000}%
\pgfsetlinewidth{0.000000pt}%
\definecolor{currentstroke}{rgb}{0.000000,0.000000,0.000000}%
\pgfsetstrokecolor{currentstroke}%
\pgfsetdash{}{0pt}%
\pgfpathmoveto{\pgfqpoint{2.172291in}{1.343150in}}%
\pgfpathcurveto{\pgfqpoint{2.177816in}{1.343150in}}{\pgfqpoint{2.183116in}{1.345345in}}{\pgfqpoint{2.187022in}{1.349252in}}%
\pgfpathcurveto{\pgfqpoint{2.190929in}{1.353159in}}{\pgfqpoint{2.193124in}{1.358458in}}{\pgfqpoint{2.193124in}{1.363984in}}%
\pgfpathcurveto{\pgfqpoint{2.193124in}{1.369509in}}{\pgfqpoint{2.190929in}{1.374808in}}{\pgfqpoint{2.187022in}{1.378715in}}%
\pgfpathcurveto{\pgfqpoint{2.183116in}{1.382622in}}{\pgfqpoint{2.177816in}{1.384817in}}{\pgfqpoint{2.172291in}{1.384817in}}%
\pgfpathcurveto{\pgfqpoint{2.166766in}{1.384817in}}{\pgfqpoint{2.161467in}{1.382622in}}{\pgfqpoint{2.157560in}{1.378715in}}%
\pgfpathcurveto{\pgfqpoint{2.153653in}{1.374808in}}{\pgfqpoint{2.151458in}{1.369509in}}{\pgfqpoint{2.151458in}{1.363984in}}%
\pgfpathcurveto{\pgfqpoint{2.151458in}{1.358458in}}{\pgfqpoint{2.153653in}{1.353159in}}{\pgfqpoint{2.157560in}{1.349252in}}%
\pgfpathcurveto{\pgfqpoint{2.161467in}{1.345345in}}{\pgfqpoint{2.166766in}{1.343150in}}{\pgfqpoint{2.172291in}{1.343150in}}%
\pgfpathclose%
\pgfusepath{fill}%
\end{pgfscope}%
\begin{pgfscope}%
\pgfpathrectangle{\pgfqpoint{2.051725in}{0.913832in}}{\pgfqpoint{1.162500in}{0.755000in}}%
\pgfusepath{clip}%
\pgfsetbuttcap%
\pgfsetroundjoin%
\definecolor{currentfill}{rgb}{0.000000,0.000000,0.000000}%
\pgfsetfillcolor{currentfill}%
\pgfsetfillopacity{0.500000}%
\pgfsetlinewidth{0.000000pt}%
\definecolor{currentstroke}{rgb}{0.000000,0.000000,0.000000}%
\pgfsetstrokecolor{currentstroke}%
\pgfsetdash{}{0pt}%
\pgfpathmoveto{\pgfqpoint{2.205718in}{1.259718in}}%
\pgfpathcurveto{\pgfqpoint{2.211243in}{1.259718in}}{\pgfqpoint{2.216543in}{1.261913in}}{\pgfqpoint{2.220450in}{1.265820in}}%
\pgfpathcurveto{\pgfqpoint{2.224357in}{1.269726in}}{\pgfqpoint{2.226552in}{1.275026in}}{\pgfqpoint{2.226552in}{1.280551in}}%
\pgfpathcurveto{\pgfqpoint{2.226552in}{1.286076in}}{\pgfqpoint{2.224357in}{1.291376in}}{\pgfqpoint{2.220450in}{1.295282in}}%
\pgfpathcurveto{\pgfqpoint{2.216543in}{1.299189in}}{\pgfqpoint{2.211243in}{1.301384in}}{\pgfqpoint{2.205718in}{1.301384in}}%
\pgfpathcurveto{\pgfqpoint{2.200193in}{1.301384in}}{\pgfqpoint{2.194894in}{1.299189in}}{\pgfqpoint{2.190987in}{1.295282in}}%
\pgfpathcurveto{\pgfqpoint{2.187080in}{1.291376in}}{\pgfqpoint{2.184885in}{1.286076in}}{\pgfqpoint{2.184885in}{1.280551in}}%
\pgfpathcurveto{\pgfqpoint{2.184885in}{1.275026in}}{\pgfqpoint{2.187080in}{1.269726in}}{\pgfqpoint{2.190987in}{1.265820in}}%
\pgfpathcurveto{\pgfqpoint{2.194894in}{1.261913in}}{\pgfqpoint{2.200193in}{1.259718in}}{\pgfqpoint{2.205718in}{1.259718in}}%
\pgfpathclose%
\pgfusepath{fill}%
\end{pgfscope}%
\begin{pgfscope}%
\pgfpathrectangle{\pgfqpoint{2.051725in}{0.913832in}}{\pgfqpoint{1.162500in}{0.755000in}}%
\pgfusepath{clip}%
\pgfsetbuttcap%
\pgfsetroundjoin%
\definecolor{currentfill}{rgb}{0.000000,0.000000,0.000000}%
\pgfsetfillcolor{currentfill}%
\pgfsetfillopacity{0.500000}%
\pgfsetlinewidth{0.000000pt}%
\definecolor{currentstroke}{rgb}{0.000000,0.000000,0.000000}%
\pgfsetstrokecolor{currentstroke}%
\pgfsetdash{}{0pt}%
\pgfpathmoveto{\pgfqpoint{3.186546in}{1.167350in}}%
\pgfpathcurveto{\pgfqpoint{3.192071in}{1.167350in}}{\pgfqpoint{3.197371in}{1.169545in}}{\pgfqpoint{3.201278in}{1.173452in}}%
\pgfpathcurveto{\pgfqpoint{3.205185in}{1.177359in}}{\pgfqpoint{3.207380in}{1.182658in}}{\pgfqpoint{3.207380in}{1.188183in}}%
\pgfpathcurveto{\pgfqpoint{3.207380in}{1.193708in}}{\pgfqpoint{3.205185in}{1.199008in}}{\pgfqpoint{3.201278in}{1.202915in}}%
\pgfpathcurveto{\pgfqpoint{3.197371in}{1.206822in}}{\pgfqpoint{3.192071in}{1.209017in}}{\pgfqpoint{3.186546in}{1.209017in}}%
\pgfpathcurveto{\pgfqpoint{3.181021in}{1.209017in}}{\pgfqpoint{3.175722in}{1.206822in}}{\pgfqpoint{3.171815in}{1.202915in}}%
\pgfpathcurveto{\pgfqpoint{3.167908in}{1.199008in}}{\pgfqpoint{3.165713in}{1.193708in}}{\pgfqpoint{3.165713in}{1.188183in}}%
\pgfpathcurveto{\pgfqpoint{3.165713in}{1.182658in}}{\pgfqpoint{3.167908in}{1.177359in}}{\pgfqpoint{3.171815in}{1.173452in}}%
\pgfpathcurveto{\pgfqpoint{3.175722in}{1.169545in}}{\pgfqpoint{3.181021in}{1.167350in}}{\pgfqpoint{3.186546in}{1.167350in}}%
\pgfpathclose%
\pgfusepath{fill}%
\end{pgfscope}%
\begin{pgfscope}%
\pgfpathrectangle{\pgfqpoint{2.051725in}{0.913832in}}{\pgfqpoint{1.162500in}{0.755000in}}%
\pgfusepath{clip}%
\pgfsetbuttcap%
\pgfsetroundjoin%
\definecolor{currentfill}{rgb}{0.000000,0.000000,0.000000}%
\pgfsetfillcolor{currentfill}%
\pgfsetfillopacity{0.500000}%
\pgfsetlinewidth{0.000000pt}%
\definecolor{currentstroke}{rgb}{0.000000,0.000000,0.000000}%
\pgfsetstrokecolor{currentstroke}%
\pgfsetdash{}{0pt}%
\pgfpathmoveto{\pgfqpoint{2.394110in}{1.034902in}}%
\pgfpathcurveto{\pgfqpoint{2.399635in}{1.034902in}}{\pgfqpoint{2.404935in}{1.037097in}}{\pgfqpoint{2.408842in}{1.041004in}}%
\pgfpathcurveto{\pgfqpoint{2.412749in}{1.044911in}}{\pgfqpoint{2.414944in}{1.050211in}}{\pgfqpoint{2.414944in}{1.055736in}}%
\pgfpathcurveto{\pgfqpoint{2.414944in}{1.061261in}}{\pgfqpoint{2.412749in}{1.066560in}}{\pgfqpoint{2.408842in}{1.070467in}}%
\pgfpathcurveto{\pgfqpoint{2.404935in}{1.074374in}}{\pgfqpoint{2.399635in}{1.076569in}}{\pgfqpoint{2.394110in}{1.076569in}}%
\pgfpathcurveto{\pgfqpoint{2.388585in}{1.076569in}}{\pgfqpoint{2.383286in}{1.074374in}}{\pgfqpoint{2.379379in}{1.070467in}}%
\pgfpathcurveto{\pgfqpoint{2.375472in}{1.066560in}}{\pgfqpoint{2.373277in}{1.061261in}}{\pgfqpoint{2.373277in}{1.055736in}}%
\pgfpathcurveto{\pgfqpoint{2.373277in}{1.050211in}}{\pgfqpoint{2.375472in}{1.044911in}}{\pgfqpoint{2.379379in}{1.041004in}}%
\pgfpathcurveto{\pgfqpoint{2.383286in}{1.037097in}}{\pgfqpoint{2.388585in}{1.034902in}}{\pgfqpoint{2.394110in}{1.034902in}}%
\pgfpathclose%
\pgfusepath{fill}%
\end{pgfscope}%
\begin{pgfscope}%
\pgfpathrectangle{\pgfqpoint{2.051725in}{0.913832in}}{\pgfqpoint{1.162500in}{0.755000in}}%
\pgfusepath{clip}%
\pgfsetbuttcap%
\pgfsetroundjoin%
\definecolor{currentfill}{rgb}{0.000000,0.000000,0.000000}%
\pgfsetfillcolor{currentfill}%
\pgfsetfillopacity{0.500000}%
\pgfsetlinewidth{0.000000pt}%
\definecolor{currentstroke}{rgb}{0.000000,0.000000,0.000000}%
\pgfsetstrokecolor{currentstroke}%
\pgfsetdash{}{0pt}%
\pgfpathmoveto{\pgfqpoint{2.125339in}{1.190157in}}%
\pgfpathcurveto{\pgfqpoint{2.130865in}{1.190157in}}{\pgfqpoint{2.136164in}{1.192352in}}{\pgfqpoint{2.140071in}{1.196259in}}%
\pgfpathcurveto{\pgfqpoint{2.143978in}{1.200166in}}{\pgfqpoint{2.146173in}{1.205465in}}{\pgfqpoint{2.146173in}{1.210991in}}%
\pgfpathcurveto{\pgfqpoint{2.146173in}{1.216516in}}{\pgfqpoint{2.143978in}{1.221815in}}{\pgfqpoint{2.140071in}{1.225722in}}%
\pgfpathcurveto{\pgfqpoint{2.136164in}{1.229629in}}{\pgfqpoint{2.130865in}{1.231824in}}{\pgfqpoint{2.125339in}{1.231824in}}%
\pgfpathcurveto{\pgfqpoint{2.119814in}{1.231824in}}{\pgfqpoint{2.114515in}{1.229629in}}{\pgfqpoint{2.110608in}{1.225722in}}%
\pgfpathcurveto{\pgfqpoint{2.106701in}{1.221815in}}{\pgfqpoint{2.104506in}{1.216516in}}{\pgfqpoint{2.104506in}{1.210991in}}%
\pgfpathcurveto{\pgfqpoint{2.104506in}{1.205465in}}{\pgfqpoint{2.106701in}{1.200166in}}{\pgfqpoint{2.110608in}{1.196259in}}%
\pgfpathcurveto{\pgfqpoint{2.114515in}{1.192352in}}{\pgfqpoint{2.119814in}{1.190157in}}{\pgfqpoint{2.125339in}{1.190157in}}%
\pgfpathclose%
\pgfusepath{fill}%
\end{pgfscope}%
\begin{pgfscope}%
\pgfsetbuttcap%
\pgfsetroundjoin%
\definecolor{currentfill}{rgb}{0.000000,0.000000,0.000000}%
\pgfsetfillcolor{currentfill}%
\pgfsetlinewidth{0.803000pt}%
\definecolor{currentstroke}{rgb}{0.000000,0.000000,0.000000}%
\pgfsetstrokecolor{currentstroke}%
\pgfsetdash{}{0pt}%
\pgfsys@defobject{currentmarker}{\pgfqpoint{0.000000in}{-0.048611in}}{\pgfqpoint{0.000000in}{0.000000in}}{%
\pgfpathmoveto{\pgfqpoint{0.000000in}{0.000000in}}%
\pgfpathlineto{\pgfqpoint{0.000000in}{-0.048611in}}%
\pgfusepath{stroke,fill}%
}%
\begin{pgfscope}%
\pgfsys@transformshift{2.463280in}{0.913832in}%
\pgfsys@useobject{currentmarker}{}%
\end{pgfscope}%
\end{pgfscope}%
\begin{pgfscope}%
\pgftext[x=2.493933in,y=0.665759in,left,base,rotate=90.000000]{\rmfamily\fontsize{8.000000}{9.600000}\selectfont \(\displaystyle 0.5\)}%
\end{pgfscope}%
\begin{pgfscope}%
\pgfsetbuttcap%
\pgfsetroundjoin%
\definecolor{currentfill}{rgb}{0.000000,0.000000,0.000000}%
\pgfsetfillcolor{currentfill}%
\pgfsetlinewidth{0.803000pt}%
\definecolor{currentstroke}{rgb}{0.000000,0.000000,0.000000}%
\pgfsetstrokecolor{currentstroke}%
\pgfsetdash{}{0pt}%
\pgfsys@defobject{currentmarker}{\pgfqpoint{0.000000in}{-0.048611in}}{\pgfqpoint{0.000000in}{0.000000in}}{%
\pgfpathmoveto{\pgfqpoint{0.000000in}{0.000000in}}%
\pgfpathlineto{\pgfqpoint{0.000000in}{-0.048611in}}%
\pgfusepath{stroke,fill}%
}%
\begin{pgfscope}%
\pgfsys@transformshift{2.916252in}{0.913832in}%
\pgfsys@useobject{currentmarker}{}%
\end{pgfscope}%
\end{pgfscope}%
\begin{pgfscope}%
\pgftext[x=2.946906in,y=0.665759in,left,base,rotate=90.000000]{\rmfamily\fontsize{8.000000}{9.600000}\selectfont \(\displaystyle 1.0\)}%
\end{pgfscope}%
\begin{pgfscope}%
\pgftext[x=2.632975in,y=0.610204in,,top]{\rmfamily\fontsize{16.000000}{19.200000}\selectfont charge}%
\end{pgfscope}%
\begin{pgfscope}%
\pgftext[x=3.214225in,y=0.624092in,right,top]{\rmfamily\fontsize{16.000000}{19.200000}\selectfont \(\displaystyle \times10^{-9}\)}%
\end{pgfscope}%
\begin{pgfscope}%
\pgfsetrectcap%
\pgfsetmiterjoin%
\pgfsetlinewidth{0.803000pt}%
\definecolor{currentstroke}{rgb}{0.501961,0.501961,0.501961}%
\pgfsetstrokecolor{currentstroke}%
\pgfsetdash{}{0pt}%
\pgfpathmoveto{\pgfqpoint{2.051725in}{0.913832in}}%
\pgfpathlineto{\pgfqpoint{2.051725in}{1.668832in}}%
\pgfusepath{stroke}%
\end{pgfscope}%
\begin{pgfscope}%
\pgfsetrectcap%
\pgfsetmiterjoin%
\pgfsetlinewidth{0.803000pt}%
\definecolor{currentstroke}{rgb}{0.501961,0.501961,0.501961}%
\pgfsetstrokecolor{currentstroke}%
\pgfsetdash{}{0pt}%
\pgfpathmoveto{\pgfqpoint{3.214225in}{0.913832in}}%
\pgfpathlineto{\pgfqpoint{3.214225in}{1.668832in}}%
\pgfusepath{stroke}%
\end{pgfscope}%
\begin{pgfscope}%
\pgfsetrectcap%
\pgfsetmiterjoin%
\pgfsetlinewidth{0.803000pt}%
\definecolor{currentstroke}{rgb}{0.501961,0.501961,0.501961}%
\pgfsetstrokecolor{currentstroke}%
\pgfsetdash{}{0pt}%
\pgfpathmoveto{\pgfqpoint{2.051725in}{0.913832in}}%
\pgfpathlineto{\pgfqpoint{3.214225in}{0.913832in}}%
\pgfusepath{stroke}%
\end{pgfscope}%
\begin{pgfscope}%
\pgfsetrectcap%
\pgfsetmiterjoin%
\pgfsetlinewidth{0.803000pt}%
\definecolor{currentstroke}{rgb}{0.501961,0.501961,0.501961}%
\pgfsetstrokecolor{currentstroke}%
\pgfsetdash{}{0pt}%
\pgfpathmoveto{\pgfqpoint{2.051725in}{1.668832in}}%
\pgfpathlineto{\pgfqpoint{3.214225in}{1.668832in}}%
\pgfusepath{stroke}%
\end{pgfscope}%
\begin{pgfscope}%
\pgfsetbuttcap%
\pgfsetmiterjoin%
\definecolor{currentfill}{rgb}{1.000000,1.000000,1.000000}%
\pgfsetfillcolor{currentfill}%
\pgfsetlinewidth{0.000000pt}%
\definecolor{currentstroke}{rgb}{0.000000,0.000000,0.000000}%
\pgfsetstrokecolor{currentstroke}%
\pgfsetstrokeopacity{0.000000}%
\pgfsetdash{}{0pt}%
\pgfpathmoveto{\pgfqpoint{3.214225in}{0.913832in}}%
\pgfpathlineto{\pgfqpoint{4.376725in}{0.913832in}}%
\pgfpathlineto{\pgfqpoint{4.376725in}{1.668832in}}%
\pgfpathlineto{\pgfqpoint{3.214225in}{1.668832in}}%
\pgfpathclose%
\pgfusepath{fill}%
\end{pgfscope}%
\begin{pgfscope}%
\pgfpathrectangle{\pgfqpoint{3.214225in}{0.913832in}}{\pgfqpoint{1.162500in}{0.755000in}}%
\pgfusepath{clip}%
\pgfsetbuttcap%
\pgfsetroundjoin%
\definecolor{currentfill}{rgb}{0.000000,0.000000,0.000000}%
\pgfsetfillcolor{currentfill}%
\pgfsetfillopacity{0.500000}%
\pgfsetlinewidth{0.000000pt}%
\definecolor{currentstroke}{rgb}{0.000000,0.000000,0.000000}%
\pgfsetstrokecolor{currentstroke}%
\pgfsetdash{}{0pt}%
\pgfpathmoveto{\pgfqpoint{3.656364in}{1.437401in}}%
\pgfpathcurveto{\pgfqpoint{3.661889in}{1.437401in}}{\pgfqpoint{3.667189in}{1.439596in}}{\pgfqpoint{3.671096in}{1.443503in}}%
\pgfpathcurveto{\pgfqpoint{3.675002in}{1.447410in}}{\pgfqpoint{3.677198in}{1.452710in}}{\pgfqpoint{3.677198in}{1.458235in}}%
\pgfpathcurveto{\pgfqpoint{3.677198in}{1.463760in}}{\pgfqpoint{3.675002in}{1.469059in}}{\pgfqpoint{3.671096in}{1.472966in}}%
\pgfpathcurveto{\pgfqpoint{3.667189in}{1.476873in}}{\pgfqpoint{3.661889in}{1.479068in}}{\pgfqpoint{3.656364in}{1.479068in}}%
\pgfpathcurveto{\pgfqpoint{3.650839in}{1.479068in}}{\pgfqpoint{3.645540in}{1.476873in}}{\pgfqpoint{3.641633in}{1.472966in}}%
\pgfpathcurveto{\pgfqpoint{3.637726in}{1.469059in}}{\pgfqpoint{3.635531in}{1.463760in}}{\pgfqpoint{3.635531in}{1.458235in}}%
\pgfpathcurveto{\pgfqpoint{3.635531in}{1.452710in}}{\pgfqpoint{3.637726in}{1.447410in}}{\pgfqpoint{3.641633in}{1.443503in}}%
\pgfpathcurveto{\pgfqpoint{3.645540in}{1.439596in}}{\pgfqpoint{3.650839in}{1.437401in}}{\pgfqpoint{3.656364in}{1.437401in}}%
\pgfpathclose%
\pgfusepath{fill}%
\end{pgfscope}%
\begin{pgfscope}%
\pgfpathrectangle{\pgfqpoint{3.214225in}{0.913832in}}{\pgfqpoint{1.162500in}{0.755000in}}%
\pgfusepath{clip}%
\pgfsetbuttcap%
\pgfsetroundjoin%
\definecolor{currentfill}{rgb}{0.000000,0.000000,0.000000}%
\pgfsetfillcolor{currentfill}%
\pgfsetfillopacity{0.500000}%
\pgfsetlinewidth{0.000000pt}%
\definecolor{currentstroke}{rgb}{0.000000,0.000000,0.000000}%
\pgfsetstrokecolor{currentstroke}%
\pgfsetdash{}{0pt}%
\pgfpathmoveto{\pgfqpoint{4.259629in}{0.987180in}}%
\pgfpathcurveto{\pgfqpoint{4.265154in}{0.987180in}}{\pgfqpoint{4.270454in}{0.989375in}}{\pgfqpoint{4.274361in}{0.993282in}}%
\pgfpathcurveto{\pgfqpoint{4.278268in}{0.997189in}}{\pgfqpoint{4.280463in}{1.002488in}}{\pgfqpoint{4.280463in}{1.008013in}}%
\pgfpathcurveto{\pgfqpoint{4.280463in}{1.013538in}}{\pgfqpoint{4.278268in}{1.018838in}}{\pgfqpoint{4.274361in}{1.022745in}}%
\pgfpathcurveto{\pgfqpoint{4.270454in}{1.026652in}}{\pgfqpoint{4.265154in}{1.028847in}}{\pgfqpoint{4.259629in}{1.028847in}}%
\pgfpathcurveto{\pgfqpoint{4.254104in}{1.028847in}}{\pgfqpoint{4.248805in}{1.026652in}}{\pgfqpoint{4.244898in}{1.022745in}}%
\pgfpathcurveto{\pgfqpoint{4.240991in}{1.018838in}}{\pgfqpoint{4.238796in}{1.013538in}}{\pgfqpoint{4.238796in}{1.008013in}}%
\pgfpathcurveto{\pgfqpoint{4.238796in}{1.002488in}}{\pgfqpoint{4.240991in}{0.997189in}}{\pgfqpoint{4.244898in}{0.993282in}}%
\pgfpathcurveto{\pgfqpoint{4.248805in}{0.989375in}}{\pgfqpoint{4.254104in}{0.987180in}}{\pgfqpoint{4.259629in}{0.987180in}}%
\pgfpathclose%
\pgfusepath{fill}%
\end{pgfscope}%
\begin{pgfscope}%
\pgfpathrectangle{\pgfqpoint{3.214225in}{0.913832in}}{\pgfqpoint{1.162500in}{0.755000in}}%
\pgfusepath{clip}%
\pgfsetbuttcap%
\pgfsetroundjoin%
\definecolor{currentfill}{rgb}{0.000000,0.000000,0.000000}%
\pgfsetfillcolor{currentfill}%
\pgfsetfillopacity{0.500000}%
\pgfsetlinewidth{0.000000pt}%
\definecolor{currentstroke}{rgb}{0.000000,0.000000,0.000000}%
\pgfsetstrokecolor{currentstroke}%
\pgfsetdash{}{0pt}%
\pgfpathmoveto{\pgfqpoint{4.349046in}{0.939886in}}%
\pgfpathcurveto{\pgfqpoint{4.354571in}{0.939886in}}{\pgfqpoint{4.359871in}{0.942081in}}{\pgfqpoint{4.363778in}{0.945988in}}%
\pgfpathcurveto{\pgfqpoint{4.367685in}{0.949895in}}{\pgfqpoint{4.369880in}{0.955195in}}{\pgfqpoint{4.369880in}{0.960720in}}%
\pgfpathcurveto{\pgfqpoint{4.369880in}{0.966245in}}{\pgfqpoint{4.367685in}{0.971544in}}{\pgfqpoint{4.363778in}{0.975451in}}%
\pgfpathcurveto{\pgfqpoint{4.359871in}{0.979358in}}{\pgfqpoint{4.354571in}{0.981553in}}{\pgfqpoint{4.349046in}{0.981553in}}%
\pgfpathcurveto{\pgfqpoint{4.343521in}{0.981553in}}{\pgfqpoint{4.338222in}{0.979358in}}{\pgfqpoint{4.334315in}{0.975451in}}%
\pgfpathcurveto{\pgfqpoint{4.330408in}{0.971544in}}{\pgfqpoint{4.328213in}{0.966245in}}{\pgfqpoint{4.328213in}{0.960720in}}%
\pgfpathcurveto{\pgfqpoint{4.328213in}{0.955195in}}{\pgfqpoint{4.330408in}{0.949895in}}{\pgfqpoint{4.334315in}{0.945988in}}%
\pgfpathcurveto{\pgfqpoint{4.338222in}{0.942081in}}{\pgfqpoint{4.343521in}{0.939886in}}{\pgfqpoint{4.349046in}{0.939886in}}%
\pgfpathclose%
\pgfusepath{fill}%
\end{pgfscope}%
\begin{pgfscope}%
\pgfpathrectangle{\pgfqpoint{3.214225in}{0.913832in}}{\pgfqpoint{1.162500in}{0.755000in}}%
\pgfusepath{clip}%
\pgfsetbuttcap%
\pgfsetroundjoin%
\definecolor{currentfill}{rgb}{0.000000,0.000000,0.000000}%
\pgfsetfillcolor{currentfill}%
\pgfsetfillopacity{0.500000}%
\pgfsetlinewidth{0.000000pt}%
\definecolor{currentstroke}{rgb}{0.000000,0.000000,0.000000}%
\pgfsetstrokecolor{currentstroke}%
\pgfsetdash{}{0pt}%
\pgfpathmoveto{\pgfqpoint{3.915823in}{1.105715in}}%
\pgfpathcurveto{\pgfqpoint{3.921348in}{1.105715in}}{\pgfqpoint{3.926648in}{1.107910in}}{\pgfqpoint{3.930554in}{1.111817in}}%
\pgfpathcurveto{\pgfqpoint{3.934461in}{1.115723in}}{\pgfqpoint{3.936656in}{1.121023in}}{\pgfqpoint{3.936656in}{1.126548in}}%
\pgfpathcurveto{\pgfqpoint{3.936656in}{1.132073in}}{\pgfqpoint{3.934461in}{1.137372in}}{\pgfqpoint{3.930554in}{1.141279in}}%
\pgfpathcurveto{\pgfqpoint{3.926648in}{1.145186in}}{\pgfqpoint{3.921348in}{1.147381in}}{\pgfqpoint{3.915823in}{1.147381in}}%
\pgfpathcurveto{\pgfqpoint{3.910298in}{1.147381in}}{\pgfqpoint{3.904998in}{1.145186in}}{\pgfqpoint{3.901092in}{1.141279in}}%
\pgfpathcurveto{\pgfqpoint{3.897185in}{1.137372in}}{\pgfqpoint{3.894990in}{1.132073in}}{\pgfqpoint{3.894990in}{1.126548in}}%
\pgfpathcurveto{\pgfqpoint{3.894990in}{1.121023in}}{\pgfqpoint{3.897185in}{1.115723in}}{\pgfqpoint{3.901092in}{1.111817in}}%
\pgfpathcurveto{\pgfqpoint{3.904998in}{1.107910in}}{\pgfqpoint{3.910298in}{1.105715in}}{\pgfqpoint{3.915823in}{1.105715in}}%
\pgfpathclose%
\pgfusepath{fill}%
\end{pgfscope}%
\begin{pgfscope}%
\pgfpathrectangle{\pgfqpoint{3.214225in}{0.913832in}}{\pgfqpoint{1.162500in}{0.755000in}}%
\pgfusepath{clip}%
\pgfsetbuttcap%
\pgfsetroundjoin%
\definecolor{currentfill}{rgb}{0.000000,0.000000,0.000000}%
\pgfsetfillcolor{currentfill}%
\pgfsetfillopacity{0.500000}%
\pgfsetlinewidth{0.000000pt}%
\definecolor{currentstroke}{rgb}{0.000000,0.000000,0.000000}%
\pgfsetstrokecolor{currentstroke}%
\pgfsetdash{}{0pt}%
\pgfpathmoveto{\pgfqpoint{3.986481in}{0.910975in}}%
\pgfpathcurveto{\pgfqpoint{3.992006in}{0.910975in}}{\pgfqpoint{3.997306in}{0.913170in}}{\pgfqpoint{4.001213in}{0.917077in}}%
\pgfpathcurveto{\pgfqpoint{4.005120in}{0.920984in}}{\pgfqpoint{4.007315in}{0.926283in}}{\pgfqpoint{4.007315in}{0.931809in}}%
\pgfpathcurveto{\pgfqpoint{4.007315in}{0.937334in}}{\pgfqpoint{4.005120in}{0.942633in}}{\pgfqpoint{4.001213in}{0.946540in}}%
\pgfpathcurveto{\pgfqpoint{3.997306in}{0.950447in}}{\pgfqpoint{3.992006in}{0.952642in}}{\pgfqpoint{3.986481in}{0.952642in}}%
\pgfpathcurveto{\pgfqpoint{3.980956in}{0.952642in}}{\pgfqpoint{3.975657in}{0.950447in}}{\pgfqpoint{3.971750in}{0.946540in}}%
\pgfpathcurveto{\pgfqpoint{3.967843in}{0.942633in}}{\pgfqpoint{3.965648in}{0.937334in}}{\pgfqpoint{3.965648in}{0.931809in}}%
\pgfpathcurveto{\pgfqpoint{3.965648in}{0.926283in}}{\pgfqpoint{3.967843in}{0.920984in}}{\pgfqpoint{3.971750in}{0.917077in}}%
\pgfpathcurveto{\pgfqpoint{3.975657in}{0.913170in}}{\pgfqpoint{3.980956in}{0.910975in}}{\pgfqpoint{3.986481in}{0.910975in}}%
\pgfpathclose%
\pgfusepath{fill}%
\end{pgfscope}%
\begin{pgfscope}%
\pgfpathrectangle{\pgfqpoint{3.214225in}{0.913832in}}{\pgfqpoint{1.162500in}{0.755000in}}%
\pgfusepath{clip}%
\pgfsetbuttcap%
\pgfsetroundjoin%
\definecolor{currentfill}{rgb}{0.000000,0.000000,0.000000}%
\pgfsetfillcolor{currentfill}%
\pgfsetfillopacity{0.500000}%
\pgfsetlinewidth{0.000000pt}%
\definecolor{currentstroke}{rgb}{0.000000,0.000000,0.000000}%
\pgfsetstrokecolor{currentstroke}%
\pgfsetdash{}{0pt}%
\pgfpathmoveto{\pgfqpoint{3.831974in}{1.042931in}}%
\pgfpathcurveto{\pgfqpoint{3.837499in}{1.042931in}}{\pgfqpoint{3.842798in}{1.045126in}}{\pgfqpoint{3.846705in}{1.049033in}}%
\pgfpathcurveto{\pgfqpoint{3.850612in}{1.052940in}}{\pgfqpoint{3.852807in}{1.058239in}}{\pgfqpoint{3.852807in}{1.063764in}}%
\pgfpathcurveto{\pgfqpoint{3.852807in}{1.069289in}}{\pgfqpoint{3.850612in}{1.074589in}}{\pgfqpoint{3.846705in}{1.078495in}}%
\pgfpathcurveto{\pgfqpoint{3.842798in}{1.082402in}}{\pgfqpoint{3.837499in}{1.084597in}}{\pgfqpoint{3.831974in}{1.084597in}}%
\pgfpathcurveto{\pgfqpoint{3.826449in}{1.084597in}}{\pgfqpoint{3.821149in}{1.082402in}}{\pgfqpoint{3.817242in}{1.078495in}}%
\pgfpathcurveto{\pgfqpoint{3.813336in}{1.074589in}}{\pgfqpoint{3.811140in}{1.069289in}}{\pgfqpoint{3.811140in}{1.063764in}}%
\pgfpathcurveto{\pgfqpoint{3.811140in}{1.058239in}}{\pgfqpoint{3.813336in}{1.052940in}}{\pgfqpoint{3.817242in}{1.049033in}}%
\pgfpathcurveto{\pgfqpoint{3.821149in}{1.045126in}}{\pgfqpoint{3.826449in}{1.042931in}}{\pgfqpoint{3.831974in}{1.042931in}}%
\pgfpathclose%
\pgfusepath{fill}%
\end{pgfscope}%
\begin{pgfscope}%
\pgfpathrectangle{\pgfqpoint{3.214225in}{0.913832in}}{\pgfqpoint{1.162500in}{0.755000in}}%
\pgfusepath{clip}%
\pgfsetbuttcap%
\pgfsetroundjoin%
\definecolor{currentfill}{rgb}{0.000000,0.000000,0.000000}%
\pgfsetfillcolor{currentfill}%
\pgfsetfillopacity{0.500000}%
\pgfsetlinewidth{0.000000pt}%
\definecolor{currentstroke}{rgb}{0.000000,0.000000,0.000000}%
\pgfsetstrokecolor{currentstroke}%
\pgfsetdash{}{0pt}%
\pgfpathmoveto{\pgfqpoint{3.241904in}{1.035607in}}%
\pgfpathcurveto{\pgfqpoint{3.247429in}{1.035607in}}{\pgfqpoint{3.252728in}{1.037803in}}{\pgfqpoint{3.256635in}{1.041709in}}%
\pgfpathcurveto{\pgfqpoint{3.260542in}{1.045616in}}{\pgfqpoint{3.262737in}{1.050916in}}{\pgfqpoint{3.262737in}{1.056441in}}%
\pgfpathcurveto{\pgfqpoint{3.262737in}{1.061966in}}{\pgfqpoint{3.260542in}{1.067265in}}{\pgfqpoint{3.256635in}{1.071172in}}%
\pgfpathcurveto{\pgfqpoint{3.252728in}{1.075079in}}{\pgfqpoint{3.247429in}{1.077274in}}{\pgfqpoint{3.241904in}{1.077274in}}%
\pgfpathcurveto{\pgfqpoint{3.236379in}{1.077274in}}{\pgfqpoint{3.231079in}{1.075079in}}{\pgfqpoint{3.227172in}{1.071172in}}%
\pgfpathcurveto{\pgfqpoint{3.223265in}{1.067265in}}{\pgfqpoint{3.221070in}{1.061966in}}{\pgfqpoint{3.221070in}{1.056441in}}%
\pgfpathcurveto{\pgfqpoint{3.221070in}{1.050916in}}{\pgfqpoint{3.223265in}{1.045616in}}{\pgfqpoint{3.227172in}{1.041709in}}%
\pgfpathcurveto{\pgfqpoint{3.231079in}{1.037803in}}{\pgfqpoint{3.236379in}{1.035607in}}{\pgfqpoint{3.241904in}{1.035607in}}%
\pgfpathclose%
\pgfusepath{fill}%
\end{pgfscope}%
\begin{pgfscope}%
\pgfpathrectangle{\pgfqpoint{3.214225in}{0.913832in}}{\pgfqpoint{1.162500in}{0.755000in}}%
\pgfusepath{clip}%
\pgfsetbuttcap%
\pgfsetroundjoin%
\definecolor{currentfill}{rgb}{0.000000,0.000000,0.000000}%
\pgfsetfillcolor{currentfill}%
\pgfsetfillopacity{0.500000}%
\pgfsetlinewidth{0.000000pt}%
\definecolor{currentstroke}{rgb}{0.000000,0.000000,0.000000}%
\pgfsetstrokecolor{currentstroke}%
\pgfsetdash{}{0pt}%
\pgfpathmoveto{\pgfqpoint{4.220724in}{1.572595in}}%
\pgfpathcurveto{\pgfqpoint{4.226249in}{1.572595in}}{\pgfqpoint{4.231548in}{1.574791in}}{\pgfqpoint{4.235455in}{1.578697in}}%
\pgfpathcurveto{\pgfqpoint{4.239362in}{1.582604in}}{\pgfqpoint{4.241557in}{1.587904in}}{\pgfqpoint{4.241557in}{1.593429in}}%
\pgfpathcurveto{\pgfqpoint{4.241557in}{1.598954in}}{\pgfqpoint{4.239362in}{1.604253in}}{\pgfqpoint{4.235455in}{1.608160in}}%
\pgfpathcurveto{\pgfqpoint{4.231548in}{1.612067in}}{\pgfqpoint{4.226249in}{1.614262in}}{\pgfqpoint{4.220724in}{1.614262in}}%
\pgfpathcurveto{\pgfqpoint{4.215199in}{1.614262in}}{\pgfqpoint{4.209899in}{1.612067in}}{\pgfqpoint{4.205992in}{1.608160in}}%
\pgfpathcurveto{\pgfqpoint{4.202086in}{1.604253in}}{\pgfqpoint{4.199890in}{1.598954in}}{\pgfqpoint{4.199890in}{1.593429in}}%
\pgfpathcurveto{\pgfqpoint{4.199890in}{1.587904in}}{\pgfqpoint{4.202086in}{1.582604in}}{\pgfqpoint{4.205992in}{1.578697in}}%
\pgfpathcurveto{\pgfqpoint{4.209899in}{1.574791in}}{\pgfqpoint{4.215199in}{1.572595in}}{\pgfqpoint{4.220724in}{1.572595in}}%
\pgfpathclose%
\pgfusepath{fill}%
\end{pgfscope}%
\begin{pgfscope}%
\pgfpathrectangle{\pgfqpoint{3.214225in}{0.913832in}}{\pgfqpoint{1.162500in}{0.755000in}}%
\pgfusepath{clip}%
\pgfsetbuttcap%
\pgfsetroundjoin%
\definecolor{currentfill}{rgb}{0.000000,0.000000,0.000000}%
\pgfsetfillcolor{currentfill}%
\pgfsetfillopacity{0.500000}%
\pgfsetlinewidth{0.000000pt}%
\definecolor{currentstroke}{rgb}{0.000000,0.000000,0.000000}%
\pgfsetstrokecolor{currentstroke}%
\pgfsetdash{}{0pt}%
\pgfpathmoveto{\pgfqpoint{3.980838in}{1.630023in}}%
\pgfpathcurveto{\pgfqpoint{3.986364in}{1.630023in}}{\pgfqpoint{3.991663in}{1.632218in}}{\pgfqpoint{3.995570in}{1.636125in}}%
\pgfpathcurveto{\pgfqpoint{3.999477in}{1.640032in}}{\pgfqpoint{4.001672in}{1.645331in}}{\pgfqpoint{4.001672in}{1.650856in}}%
\pgfpathcurveto{\pgfqpoint{4.001672in}{1.656381in}}{\pgfqpoint{3.999477in}{1.661681in}}{\pgfqpoint{3.995570in}{1.665588in}}%
\pgfpathcurveto{\pgfqpoint{3.991663in}{1.669494in}}{\pgfqpoint{3.986364in}{1.671690in}}{\pgfqpoint{3.980838in}{1.671690in}}%
\pgfpathcurveto{\pgfqpoint{3.975313in}{1.671690in}}{\pgfqpoint{3.970014in}{1.669494in}}{\pgfqpoint{3.966107in}{1.665588in}}%
\pgfpathcurveto{\pgfqpoint{3.962200in}{1.661681in}}{\pgfqpoint{3.960005in}{1.656381in}}{\pgfqpoint{3.960005in}{1.650856in}}%
\pgfpathcurveto{\pgfqpoint{3.960005in}{1.645331in}}{\pgfqpoint{3.962200in}{1.640032in}}{\pgfqpoint{3.966107in}{1.636125in}}%
\pgfpathcurveto{\pgfqpoint{3.970014in}{1.632218in}}{\pgfqpoint{3.975313in}{1.630023in}}{\pgfqpoint{3.980838in}{1.630023in}}%
\pgfpathclose%
\pgfusepath{fill}%
\end{pgfscope}%
\begin{pgfscope}%
\pgfpathrectangle{\pgfqpoint{3.214225in}{0.913832in}}{\pgfqpoint{1.162500in}{0.755000in}}%
\pgfusepath{clip}%
\pgfsetbuttcap%
\pgfsetroundjoin%
\definecolor{currentfill}{rgb}{0.000000,0.000000,0.000000}%
\pgfsetfillcolor{currentfill}%
\pgfsetfillopacity{0.500000}%
\pgfsetlinewidth{0.000000pt}%
\definecolor{currentstroke}{rgb}{0.000000,0.000000,0.000000}%
\pgfsetstrokecolor{currentstroke}%
\pgfsetdash{}{0pt}%
\pgfpathmoveto{\pgfqpoint{3.708643in}{1.484185in}}%
\pgfpathcurveto{\pgfqpoint{3.714168in}{1.484185in}}{\pgfqpoint{3.719468in}{1.486380in}}{\pgfqpoint{3.723375in}{1.490287in}}%
\pgfpathcurveto{\pgfqpoint{3.727282in}{1.494194in}}{\pgfqpoint{3.729477in}{1.499493in}}{\pgfqpoint{3.729477in}{1.505018in}}%
\pgfpathcurveto{\pgfqpoint{3.729477in}{1.510544in}}{\pgfqpoint{3.727282in}{1.515843in}}{\pgfqpoint{3.723375in}{1.519750in}}%
\pgfpathcurveto{\pgfqpoint{3.719468in}{1.523657in}}{\pgfqpoint{3.714168in}{1.525852in}}{\pgfqpoint{3.708643in}{1.525852in}}%
\pgfpathcurveto{\pgfqpoint{3.703118in}{1.525852in}}{\pgfqpoint{3.697819in}{1.523657in}}{\pgfqpoint{3.693912in}{1.519750in}}%
\pgfpathcurveto{\pgfqpoint{3.690005in}{1.515843in}}{\pgfqpoint{3.687810in}{1.510544in}}{\pgfqpoint{3.687810in}{1.505018in}}%
\pgfpathcurveto{\pgfqpoint{3.687810in}{1.499493in}}{\pgfqpoint{3.690005in}{1.494194in}}{\pgfqpoint{3.693912in}{1.490287in}}%
\pgfpathcurveto{\pgfqpoint{3.697819in}{1.486380in}}{\pgfqpoint{3.703118in}{1.484185in}}{\pgfqpoint{3.708643in}{1.484185in}}%
\pgfpathclose%
\pgfusepath{fill}%
\end{pgfscope}%
\begin{pgfscope}%
\pgfpathrectangle{\pgfqpoint{3.214225in}{0.913832in}}{\pgfqpoint{1.162500in}{0.755000in}}%
\pgfusepath{clip}%
\pgfsetbuttcap%
\pgfsetroundjoin%
\definecolor{currentfill}{rgb}{0.000000,0.000000,0.000000}%
\pgfsetfillcolor{currentfill}%
\pgfsetfillopacity{0.500000}%
\pgfsetlinewidth{0.000000pt}%
\definecolor{currentstroke}{rgb}{0.000000,0.000000,0.000000}%
\pgfsetstrokecolor{currentstroke}%
\pgfsetdash{}{0pt}%
\pgfpathmoveto{\pgfqpoint{4.124371in}{1.067768in}}%
\pgfpathcurveto{\pgfqpoint{4.129896in}{1.067768in}}{\pgfqpoint{4.135195in}{1.069963in}}{\pgfqpoint{4.139102in}{1.073870in}}%
\pgfpathcurveto{\pgfqpoint{4.143009in}{1.077776in}}{\pgfqpoint{4.145204in}{1.083076in}}{\pgfqpoint{4.145204in}{1.088601in}}%
\pgfpathcurveto{\pgfqpoint{4.145204in}{1.094126in}}{\pgfqpoint{4.143009in}{1.099426in}}{\pgfqpoint{4.139102in}{1.103332in}}%
\pgfpathcurveto{\pgfqpoint{4.135195in}{1.107239in}}{\pgfqpoint{4.129896in}{1.109434in}}{\pgfqpoint{4.124371in}{1.109434in}}%
\pgfpathcurveto{\pgfqpoint{4.118846in}{1.109434in}}{\pgfqpoint{4.113546in}{1.107239in}}{\pgfqpoint{4.109639in}{1.103332in}}%
\pgfpathcurveto{\pgfqpoint{4.105732in}{1.099426in}}{\pgfqpoint{4.103537in}{1.094126in}}{\pgfqpoint{4.103537in}{1.088601in}}%
\pgfpathcurveto{\pgfqpoint{4.103537in}{1.083076in}}{\pgfqpoint{4.105732in}{1.077776in}}{\pgfqpoint{4.109639in}{1.073870in}}%
\pgfpathcurveto{\pgfqpoint{4.113546in}{1.069963in}}{\pgfqpoint{4.118846in}{1.067768in}}{\pgfqpoint{4.124371in}{1.067768in}}%
\pgfpathclose%
\pgfusepath{fill}%
\end{pgfscope}%
\begin{pgfscope}%
\pgfpathrectangle{\pgfqpoint{3.214225in}{0.913832in}}{\pgfqpoint{1.162500in}{0.755000in}}%
\pgfusepath{clip}%
\pgfsetbuttcap%
\pgfsetroundjoin%
\definecolor{currentfill}{rgb}{0.000000,0.000000,0.000000}%
\pgfsetfillcolor{currentfill}%
\pgfsetfillopacity{0.500000}%
\pgfsetlinewidth{0.000000pt}%
\definecolor{currentstroke}{rgb}{0.000000,0.000000,0.000000}%
\pgfsetstrokecolor{currentstroke}%
\pgfsetdash{}{0pt}%
\pgfpathmoveto{\pgfqpoint{3.695649in}{1.343150in}}%
\pgfpathcurveto{\pgfqpoint{3.701174in}{1.343150in}}{\pgfqpoint{3.706474in}{1.345345in}}{\pgfqpoint{3.710381in}{1.349252in}}%
\pgfpathcurveto{\pgfqpoint{3.714288in}{1.353159in}}{\pgfqpoint{3.716483in}{1.358458in}}{\pgfqpoint{3.716483in}{1.363984in}}%
\pgfpathcurveto{\pgfqpoint{3.716483in}{1.369509in}}{\pgfqpoint{3.714288in}{1.374808in}}{\pgfqpoint{3.710381in}{1.378715in}}%
\pgfpathcurveto{\pgfqpoint{3.706474in}{1.382622in}}{\pgfqpoint{3.701174in}{1.384817in}}{\pgfqpoint{3.695649in}{1.384817in}}%
\pgfpathcurveto{\pgfqpoint{3.690124in}{1.384817in}}{\pgfqpoint{3.684825in}{1.382622in}}{\pgfqpoint{3.680918in}{1.378715in}}%
\pgfpathcurveto{\pgfqpoint{3.677011in}{1.374808in}}{\pgfqpoint{3.674816in}{1.369509in}}{\pgfqpoint{3.674816in}{1.363984in}}%
\pgfpathcurveto{\pgfqpoint{3.674816in}{1.358458in}}{\pgfqpoint{3.677011in}{1.353159in}}{\pgfqpoint{3.680918in}{1.349252in}}%
\pgfpathcurveto{\pgfqpoint{3.684825in}{1.345345in}}{\pgfqpoint{3.690124in}{1.343150in}}{\pgfqpoint{3.695649in}{1.343150in}}%
\pgfpathclose%
\pgfusepath{fill}%
\end{pgfscope}%
\begin{pgfscope}%
\pgfpathrectangle{\pgfqpoint{3.214225in}{0.913832in}}{\pgfqpoint{1.162500in}{0.755000in}}%
\pgfusepath{clip}%
\pgfsetbuttcap%
\pgfsetroundjoin%
\definecolor{currentfill}{rgb}{0.000000,0.000000,0.000000}%
\pgfsetfillcolor{currentfill}%
\pgfsetfillopacity{0.500000}%
\pgfsetlinewidth{0.000000pt}%
\definecolor{currentstroke}{rgb}{0.000000,0.000000,0.000000}%
\pgfsetstrokecolor{currentstroke}%
\pgfsetdash{}{0pt}%
\pgfpathmoveto{\pgfqpoint{3.329643in}{1.259718in}}%
\pgfpathcurveto{\pgfqpoint{3.335168in}{1.259718in}}{\pgfqpoint{3.340468in}{1.261913in}}{\pgfqpoint{3.344375in}{1.265820in}}%
\pgfpathcurveto{\pgfqpoint{3.348281in}{1.269726in}}{\pgfqpoint{3.350476in}{1.275026in}}{\pgfqpoint{3.350476in}{1.280551in}}%
\pgfpathcurveto{\pgfqpoint{3.350476in}{1.286076in}}{\pgfqpoint{3.348281in}{1.291376in}}{\pgfqpoint{3.344375in}{1.295282in}}%
\pgfpathcurveto{\pgfqpoint{3.340468in}{1.299189in}}{\pgfqpoint{3.335168in}{1.301384in}}{\pgfqpoint{3.329643in}{1.301384in}}%
\pgfpathcurveto{\pgfqpoint{3.324118in}{1.301384in}}{\pgfqpoint{3.318819in}{1.299189in}}{\pgfqpoint{3.314912in}{1.295282in}}%
\pgfpathcurveto{\pgfqpoint{3.311005in}{1.291376in}}{\pgfqpoint{3.308810in}{1.286076in}}{\pgfqpoint{3.308810in}{1.280551in}}%
\pgfpathcurveto{\pgfqpoint{3.308810in}{1.275026in}}{\pgfqpoint{3.311005in}{1.269726in}}{\pgfqpoint{3.314912in}{1.265820in}}%
\pgfpathcurveto{\pgfqpoint{3.318819in}{1.261913in}}{\pgfqpoint{3.324118in}{1.259718in}}{\pgfqpoint{3.329643in}{1.259718in}}%
\pgfpathclose%
\pgfusepath{fill}%
\end{pgfscope}%
\begin{pgfscope}%
\pgfpathrectangle{\pgfqpoint{3.214225in}{0.913832in}}{\pgfqpoint{1.162500in}{0.755000in}}%
\pgfusepath{clip}%
\pgfsetbuttcap%
\pgfsetroundjoin%
\definecolor{currentfill}{rgb}{0.000000,0.000000,0.000000}%
\pgfsetfillcolor{currentfill}%
\pgfsetfillopacity{0.500000}%
\pgfsetlinewidth{0.000000pt}%
\definecolor{currentstroke}{rgb}{0.000000,0.000000,0.000000}%
\pgfsetstrokecolor{currentstroke}%
\pgfsetdash{}{0pt}%
\pgfpathmoveto{\pgfqpoint{4.167862in}{1.167350in}}%
\pgfpathcurveto{\pgfqpoint{4.173387in}{1.167350in}}{\pgfqpoint{4.178687in}{1.169545in}}{\pgfqpoint{4.182593in}{1.173452in}}%
\pgfpathcurveto{\pgfqpoint{4.186500in}{1.177359in}}{\pgfqpoint{4.188695in}{1.182658in}}{\pgfqpoint{4.188695in}{1.188183in}}%
\pgfpathcurveto{\pgfqpoint{4.188695in}{1.193708in}}{\pgfqpoint{4.186500in}{1.199008in}}{\pgfqpoint{4.182593in}{1.202915in}}%
\pgfpathcurveto{\pgfqpoint{4.178687in}{1.206822in}}{\pgfqpoint{4.173387in}{1.209017in}}{\pgfqpoint{4.167862in}{1.209017in}}%
\pgfpathcurveto{\pgfqpoint{4.162337in}{1.209017in}}{\pgfqpoint{4.157037in}{1.206822in}}{\pgfqpoint{4.153131in}{1.202915in}}%
\pgfpathcurveto{\pgfqpoint{4.149224in}{1.199008in}}{\pgfqpoint{4.147029in}{1.193708in}}{\pgfqpoint{4.147029in}{1.188183in}}%
\pgfpathcurveto{\pgfqpoint{4.147029in}{1.182658in}}{\pgfqpoint{4.149224in}{1.177359in}}{\pgfqpoint{4.153131in}{1.173452in}}%
\pgfpathcurveto{\pgfqpoint{4.157037in}{1.169545in}}{\pgfqpoint{4.162337in}{1.167350in}}{\pgfqpoint{4.167862in}{1.167350in}}%
\pgfpathclose%
\pgfusepath{fill}%
\end{pgfscope}%
\begin{pgfscope}%
\pgfpathrectangle{\pgfqpoint{3.214225in}{0.913832in}}{\pgfqpoint{1.162500in}{0.755000in}}%
\pgfusepath{clip}%
\pgfsetbuttcap%
\pgfsetroundjoin%
\definecolor{currentfill}{rgb}{0.000000,0.000000,0.000000}%
\pgfsetfillcolor{currentfill}%
\pgfsetfillopacity{0.500000}%
\pgfsetlinewidth{0.000000pt}%
\definecolor{currentstroke}{rgb}{0.000000,0.000000,0.000000}%
\pgfsetstrokecolor{currentstroke}%
\pgfsetdash{}{0pt}%
\pgfpathmoveto{\pgfqpoint{4.120816in}{1.034902in}}%
\pgfpathcurveto{\pgfqpoint{4.126341in}{1.034902in}}{\pgfqpoint{4.131641in}{1.037097in}}{\pgfqpoint{4.135548in}{1.041004in}}%
\pgfpathcurveto{\pgfqpoint{4.139455in}{1.044911in}}{\pgfqpoint{4.141650in}{1.050211in}}{\pgfqpoint{4.141650in}{1.055736in}}%
\pgfpathcurveto{\pgfqpoint{4.141650in}{1.061261in}}{\pgfqpoint{4.139455in}{1.066560in}}{\pgfqpoint{4.135548in}{1.070467in}}%
\pgfpathcurveto{\pgfqpoint{4.131641in}{1.074374in}}{\pgfqpoint{4.126341in}{1.076569in}}{\pgfqpoint{4.120816in}{1.076569in}}%
\pgfpathcurveto{\pgfqpoint{4.115291in}{1.076569in}}{\pgfqpoint{4.109992in}{1.074374in}}{\pgfqpoint{4.106085in}{1.070467in}}%
\pgfpathcurveto{\pgfqpoint{4.102178in}{1.066560in}}{\pgfqpoint{4.099983in}{1.061261in}}{\pgfqpoint{4.099983in}{1.055736in}}%
\pgfpathcurveto{\pgfqpoint{4.099983in}{1.050211in}}{\pgfqpoint{4.102178in}{1.044911in}}{\pgfqpoint{4.106085in}{1.041004in}}%
\pgfpathcurveto{\pgfqpoint{4.109992in}{1.037097in}}{\pgfqpoint{4.115291in}{1.034902in}}{\pgfqpoint{4.120816in}{1.034902in}}%
\pgfpathclose%
\pgfusepath{fill}%
\end{pgfscope}%
\begin{pgfscope}%
\pgfpathrectangle{\pgfqpoint{3.214225in}{0.913832in}}{\pgfqpoint{1.162500in}{0.755000in}}%
\pgfusepath{clip}%
\pgfsetbuttcap%
\pgfsetroundjoin%
\definecolor{currentfill}{rgb}{0.000000,0.000000,0.000000}%
\pgfsetfillcolor{currentfill}%
\pgfsetfillopacity{0.500000}%
\pgfsetlinewidth{0.000000pt}%
\definecolor{currentstroke}{rgb}{0.000000,0.000000,0.000000}%
\pgfsetstrokecolor{currentstroke}%
\pgfsetdash{}{0pt}%
\pgfpathmoveto{\pgfqpoint{3.792736in}{1.190157in}}%
\pgfpathcurveto{\pgfqpoint{3.798261in}{1.190157in}}{\pgfqpoint{3.803560in}{1.192352in}}{\pgfqpoint{3.807467in}{1.196259in}}%
\pgfpathcurveto{\pgfqpoint{3.811374in}{1.200166in}}{\pgfqpoint{3.813569in}{1.205465in}}{\pgfqpoint{3.813569in}{1.210991in}}%
\pgfpathcurveto{\pgfqpoint{3.813569in}{1.216516in}}{\pgfqpoint{3.811374in}{1.221815in}}{\pgfqpoint{3.807467in}{1.225722in}}%
\pgfpathcurveto{\pgfqpoint{3.803560in}{1.229629in}}{\pgfqpoint{3.798261in}{1.231824in}}{\pgfqpoint{3.792736in}{1.231824in}}%
\pgfpathcurveto{\pgfqpoint{3.787211in}{1.231824in}}{\pgfqpoint{3.781911in}{1.229629in}}{\pgfqpoint{3.778004in}{1.225722in}}%
\pgfpathcurveto{\pgfqpoint{3.774097in}{1.221815in}}{\pgfqpoint{3.771902in}{1.216516in}}{\pgfqpoint{3.771902in}{1.210991in}}%
\pgfpathcurveto{\pgfqpoint{3.771902in}{1.205465in}}{\pgfqpoint{3.774097in}{1.200166in}}{\pgfqpoint{3.778004in}{1.196259in}}%
\pgfpathcurveto{\pgfqpoint{3.781911in}{1.192352in}}{\pgfqpoint{3.787211in}{1.190157in}}{\pgfqpoint{3.792736in}{1.190157in}}%
\pgfpathclose%
\pgfusepath{fill}%
\end{pgfscope}%
\begin{pgfscope}%
\pgfsetbuttcap%
\pgfsetroundjoin%
\definecolor{currentfill}{rgb}{0.000000,0.000000,0.000000}%
\pgfsetfillcolor{currentfill}%
\pgfsetlinewidth{0.803000pt}%
\definecolor{currentstroke}{rgb}{0.000000,0.000000,0.000000}%
\pgfsetstrokecolor{currentstroke}%
\pgfsetdash{}{0pt}%
\pgfsys@defobject{currentmarker}{\pgfqpoint{0.000000in}{-0.048611in}}{\pgfqpoint{0.000000in}{0.000000in}}{%
\pgfpathmoveto{\pgfqpoint{0.000000in}{0.000000in}}%
\pgfpathlineto{\pgfqpoint{0.000000in}{-0.048611in}}%
\pgfusepath{stroke,fill}%
}%
\begin{pgfscope}%
\pgfsys@transformshift{3.342043in}{0.913832in}%
\pgfsys@useobject{currentmarker}{}%
\end{pgfscope}%
\end{pgfscope}%
\begin{pgfscope}%
\pgftext[x=3.372697in,y=0.547702in,left,base,rotate=90.000000]{\rmfamily\fontsize{8.000000}{9.600000}\selectfont \(\displaystyle 0.025\)}%
\end{pgfscope}%
\begin{pgfscope}%
\pgfsetbuttcap%
\pgfsetroundjoin%
\definecolor{currentfill}{rgb}{0.000000,0.000000,0.000000}%
\pgfsetfillcolor{currentfill}%
\pgfsetlinewidth{0.803000pt}%
\definecolor{currentstroke}{rgb}{0.000000,0.000000,0.000000}%
\pgfsetstrokecolor{currentstroke}%
\pgfsetdash{}{0pt}%
\pgfsys@defobject{currentmarker}{\pgfqpoint{0.000000in}{-0.048611in}}{\pgfqpoint{0.000000in}{0.000000in}}{%
\pgfpathmoveto{\pgfqpoint{0.000000in}{0.000000in}}%
\pgfpathlineto{\pgfqpoint{0.000000in}{-0.048611in}}%
\pgfusepath{stroke,fill}%
}%
\begin{pgfscope}%
\pgfsys@transformshift{3.752399in}{0.913832in}%
\pgfsys@useobject{currentmarker}{}%
\end{pgfscope}%
\end{pgfscope}%
\begin{pgfscope}%
\pgftext[x=3.783053in,y=0.547702in,left,base,rotate=90.000000]{\rmfamily\fontsize{8.000000}{9.600000}\selectfont \(\displaystyle 0.050\)}%
\end{pgfscope}%
\begin{pgfscope}%
\pgfsetbuttcap%
\pgfsetroundjoin%
\definecolor{currentfill}{rgb}{0.000000,0.000000,0.000000}%
\pgfsetfillcolor{currentfill}%
\pgfsetlinewidth{0.803000pt}%
\definecolor{currentstroke}{rgb}{0.000000,0.000000,0.000000}%
\pgfsetstrokecolor{currentstroke}%
\pgfsetdash{}{0pt}%
\pgfsys@defobject{currentmarker}{\pgfqpoint{0.000000in}{-0.048611in}}{\pgfqpoint{0.000000in}{0.000000in}}{%
\pgfpathmoveto{\pgfqpoint{0.000000in}{0.000000in}}%
\pgfpathlineto{\pgfqpoint{0.000000in}{-0.048611in}}%
\pgfusepath{stroke,fill}%
}%
\begin{pgfscope}%
\pgfsys@transformshift{4.162756in}{0.913832in}%
\pgfsys@useobject{currentmarker}{}%
\end{pgfscope}%
\end{pgfscope}%
\begin{pgfscope}%
\pgftext[x=4.193409in,y=0.547702in,left,base,rotate=90.000000]{\rmfamily\fontsize{8.000000}{9.600000}\selectfont \(\displaystyle 0.075\)}%
\end{pgfscope}%
\begin{pgfscope}%
\pgftext[x=3.795475in,y=0.492146in,,top]{\rmfamily\fontsize{16.000000}{19.200000}\selectfont u0}%
\end{pgfscope}%
\begin{pgfscope}%
\pgfsetrectcap%
\pgfsetmiterjoin%
\pgfsetlinewidth{0.803000pt}%
\definecolor{currentstroke}{rgb}{0.501961,0.501961,0.501961}%
\pgfsetstrokecolor{currentstroke}%
\pgfsetdash{}{0pt}%
\pgfpathmoveto{\pgfqpoint{3.214225in}{0.913832in}}%
\pgfpathlineto{\pgfqpoint{3.214225in}{1.668832in}}%
\pgfusepath{stroke}%
\end{pgfscope}%
\begin{pgfscope}%
\pgfsetrectcap%
\pgfsetmiterjoin%
\pgfsetlinewidth{0.803000pt}%
\definecolor{currentstroke}{rgb}{0.501961,0.501961,0.501961}%
\pgfsetstrokecolor{currentstroke}%
\pgfsetdash{}{0pt}%
\pgfpathmoveto{\pgfqpoint{4.376725in}{0.913832in}}%
\pgfpathlineto{\pgfqpoint{4.376725in}{1.668832in}}%
\pgfusepath{stroke}%
\end{pgfscope}%
\begin{pgfscope}%
\pgfsetrectcap%
\pgfsetmiterjoin%
\pgfsetlinewidth{0.803000pt}%
\definecolor{currentstroke}{rgb}{0.501961,0.501961,0.501961}%
\pgfsetstrokecolor{currentstroke}%
\pgfsetdash{}{0pt}%
\pgfpathmoveto{\pgfqpoint{3.214225in}{0.913832in}}%
\pgfpathlineto{\pgfqpoint{4.376725in}{0.913832in}}%
\pgfusepath{stroke}%
\end{pgfscope}%
\begin{pgfscope}%
\pgfsetrectcap%
\pgfsetmiterjoin%
\pgfsetlinewidth{0.803000pt}%
\definecolor{currentstroke}{rgb}{0.501961,0.501961,0.501961}%
\pgfsetstrokecolor{currentstroke}%
\pgfsetdash{}{0pt}%
\pgfpathmoveto{\pgfqpoint{3.214225in}{1.668832in}}%
\pgfpathlineto{\pgfqpoint{4.376725in}{1.668832in}}%
\pgfusepath{stroke}%
\end{pgfscope}%
\begin{pgfscope}%
\pgfsetbuttcap%
\pgfsetmiterjoin%
\definecolor{currentfill}{rgb}{1.000000,1.000000,1.000000}%
\pgfsetfillcolor{currentfill}%
\pgfsetlinewidth{0.000000pt}%
\definecolor{currentstroke}{rgb}{0.000000,0.000000,0.000000}%
\pgfsetstrokecolor{currentstroke}%
\pgfsetstrokeopacity{0.000000}%
\pgfsetdash{}{0pt}%
\pgfpathmoveto{\pgfqpoint{4.376725in}{0.913832in}}%
\pgfpathlineto{\pgfqpoint{5.539225in}{0.913832in}}%
\pgfpathlineto{\pgfqpoint{5.539225in}{1.668832in}}%
\pgfpathlineto{\pgfqpoint{4.376725in}{1.668832in}}%
\pgfpathclose%
\pgfusepath{fill}%
\end{pgfscope}%
\begin{pgfscope}%
\pgfsetbuttcap%
\pgfsetroundjoin%
\definecolor{currentfill}{rgb}{0.000000,0.000000,0.000000}%
\pgfsetfillcolor{currentfill}%
\pgfsetlinewidth{0.803000pt}%
\definecolor{currentstroke}{rgb}{0.000000,0.000000,0.000000}%
\pgfsetstrokecolor{currentstroke}%
\pgfsetdash{}{0pt}%
\pgfsys@defobject{currentmarker}{\pgfqpoint{0.000000in}{-0.048611in}}{\pgfqpoint{0.000000in}{0.000000in}}{%
\pgfpathmoveto{\pgfqpoint{0.000000in}{0.000000in}}%
\pgfpathlineto{\pgfqpoint{0.000000in}{-0.048611in}}%
\pgfusepath{stroke,fill}%
}%
\begin{pgfscope}%
\pgfsys@transformshift{4.455518in}{0.913832in}%
\pgfsys@useobject{currentmarker}{}%
\end{pgfscope}%
\end{pgfscope}%
\begin{pgfscope}%
\pgftext[x=4.486171in,y=0.665759in,left,base,rotate=90.000000]{\rmfamily\fontsize{8.000000}{9.600000}\selectfont \(\displaystyle 0.5\)}%
\end{pgfscope}%
\begin{pgfscope}%
\pgfsetbuttcap%
\pgfsetroundjoin%
\definecolor{currentfill}{rgb}{0.000000,0.000000,0.000000}%
\pgfsetfillcolor{currentfill}%
\pgfsetlinewidth{0.803000pt}%
\definecolor{currentstroke}{rgb}{0.000000,0.000000,0.000000}%
\pgfsetstrokecolor{currentstroke}%
\pgfsetdash{}{0pt}%
\pgfsys@defobject{currentmarker}{\pgfqpoint{0.000000in}{-0.048611in}}{\pgfqpoint{0.000000in}{0.000000in}}{%
\pgfpathmoveto{\pgfqpoint{0.000000in}{0.000000in}}%
\pgfpathlineto{\pgfqpoint{0.000000in}{-0.048611in}}%
\pgfusepath{stroke,fill}%
}%
\begin{pgfscope}%
\pgfsys@transformshift{4.909793in}{0.913832in}%
\pgfsys@useobject{currentmarker}{}%
\end{pgfscope}%
\end{pgfscope}%
\begin{pgfscope}%
\pgftext[x=4.940447in,y=0.665759in,left,base,rotate=90.000000]{\rmfamily\fontsize{8.000000}{9.600000}\selectfont \(\displaystyle 1.0\)}%
\end{pgfscope}%
\begin{pgfscope}%
\pgfsetbuttcap%
\pgfsetroundjoin%
\definecolor{currentfill}{rgb}{0.000000,0.000000,0.000000}%
\pgfsetfillcolor{currentfill}%
\pgfsetlinewidth{0.803000pt}%
\definecolor{currentstroke}{rgb}{0.000000,0.000000,0.000000}%
\pgfsetstrokecolor{currentstroke}%
\pgfsetdash{}{0pt}%
\pgfsys@defobject{currentmarker}{\pgfqpoint{0.000000in}{-0.048611in}}{\pgfqpoint{0.000000in}{0.000000in}}{%
\pgfpathmoveto{\pgfqpoint{0.000000in}{0.000000in}}%
\pgfpathlineto{\pgfqpoint{0.000000in}{-0.048611in}}%
\pgfusepath{stroke,fill}%
}%
\begin{pgfscope}%
\pgfsys@transformshift{5.364069in}{0.913832in}%
\pgfsys@useobject{currentmarker}{}%
\end{pgfscope}%
\end{pgfscope}%
\begin{pgfscope}%
\pgftext[x=5.394722in,y=0.665759in,left,base,rotate=90.000000]{\rmfamily\fontsize{8.000000}{9.600000}\selectfont \(\displaystyle 1.5\)}%
\end{pgfscope}%
\begin{pgfscope}%
\pgftext[x=4.957975in,y=0.610204in,,top]{\rmfamily\fontsize{16.000000}{19.200000}\selectfont Ef0}%
\end{pgfscope}%
\begin{pgfscope}%
\pgfpathrectangle{\pgfqpoint{4.376725in}{0.913832in}}{\pgfqpoint{1.162500in}{0.755000in}}%
\pgfusepath{clip}%
\pgfsetrectcap%
\pgfsetroundjoin%
\pgfsetlinewidth{1.505625pt}%
\definecolor{currentstroke}{rgb}{0.121569,0.466667,0.705882}%
\pgfsetstrokecolor{currentstroke}%
\pgfsetdash{}{0pt}%
\pgfpathmoveto{\pgfqpoint{4.404404in}{1.292854in}}%
\pgfpathlineto{\pgfqpoint{4.443192in}{1.385258in}}%
\pgfpathlineto{\pgfqpoint{4.470899in}{1.445863in}}%
\pgfpathlineto{\pgfqpoint{4.495280in}{1.493940in}}%
\pgfpathlineto{\pgfqpoint{4.516337in}{1.530674in}}%
\pgfpathlineto{\pgfqpoint{4.535177in}{1.559293in}}%
\pgfpathlineto{\pgfqpoint{4.552909in}{1.582261in}}%
\pgfpathlineto{\pgfqpoint{4.569533in}{1.600110in}}%
\pgfpathlineto{\pgfqpoint{4.585049in}{1.613447in}}%
\pgfpathlineto{\pgfqpoint{4.599456in}{1.622913in}}%
\pgfpathlineto{\pgfqpoint{4.612755in}{1.629148in}}%
\pgfpathlineto{\pgfqpoint{4.626054in}{1.632994in}}%
\pgfpathlineto{\pgfqpoint{4.639353in}{1.634490in}}%
\pgfpathlineto{\pgfqpoint{4.652652in}{1.633693in}}%
\pgfpathlineto{\pgfqpoint{4.665951in}{1.630682in}}%
\pgfpathlineto{\pgfqpoint{4.680358in}{1.625032in}}%
\pgfpathlineto{\pgfqpoint{4.695874in}{1.616332in}}%
\pgfpathlineto{\pgfqpoint{4.712497in}{1.604229in}}%
\pgfpathlineto{\pgfqpoint{4.730229in}{1.588458in}}%
\pgfpathlineto{\pgfqpoint{4.750178in}{1.567637in}}%
\pgfpathlineto{\pgfqpoint{4.773451in}{1.539943in}}%
\pgfpathlineto{\pgfqpoint{4.802266in}{1.501880in}}%
\pgfpathlineto{\pgfqpoint{4.844379in}{1.442067in}}%
\pgfpathlineto{\pgfqpoint{4.908658in}{1.351028in}}%
\pgfpathlineto{\pgfqpoint{4.941905in}{1.307935in}}%
\pgfpathlineto{\pgfqpoint{4.970720in}{1.274145in}}%
\pgfpathlineto{\pgfqpoint{4.997318in}{1.246340in}}%
\pgfpathlineto{\pgfqpoint{5.022808in}{1.222922in}}%
\pgfpathlineto{\pgfqpoint{5.047189in}{1.203484in}}%
\pgfpathlineto{\pgfqpoint{5.071571in}{1.186825in}}%
\pgfpathlineto{\pgfqpoint{5.097061in}{1.172140in}}%
\pgfpathlineto{\pgfqpoint{5.123659in}{1.159413in}}%
\pgfpathlineto{\pgfqpoint{5.153581in}{1.147648in}}%
\pgfpathlineto{\pgfqpoint{5.190154in}{1.135777in}}%
\pgfpathlineto{\pgfqpoint{5.293221in}{1.103752in}}%
\pgfpathlineto{\pgfqpoint{5.324252in}{1.090901in}}%
\pgfpathlineto{\pgfqpoint{5.353067in}{1.076511in}}%
\pgfpathlineto{\pgfqpoint{5.380773in}{1.060111in}}%
\pgfpathlineto{\pgfqpoint{5.408479in}{1.041034in}}%
\pgfpathlineto{\pgfqpoint{5.437294in}{1.018351in}}%
\pgfpathlineto{\pgfqpoint{5.467216in}{0.991904in}}%
\pgfpathlineto{\pgfqpoint{5.500464in}{0.959524in}}%
\pgfpathlineto{\pgfqpoint{5.511546in}{0.948151in}}%
\pgfpathlineto{\pgfqpoint{5.511546in}{0.948151in}}%
\pgfusepath{stroke}%
\end{pgfscope}%
\begin{pgfscope}%
\pgfsetrectcap%
\pgfsetmiterjoin%
\pgfsetlinewidth{0.803000pt}%
\definecolor{currentstroke}{rgb}{0.501961,0.501961,0.501961}%
\pgfsetstrokecolor{currentstroke}%
\pgfsetdash{}{0pt}%
\pgfpathmoveto{\pgfqpoint{4.376725in}{0.913832in}}%
\pgfpathlineto{\pgfqpoint{4.376725in}{1.668832in}}%
\pgfusepath{stroke}%
\end{pgfscope}%
\begin{pgfscope}%
\pgfsetrectcap%
\pgfsetmiterjoin%
\pgfsetlinewidth{0.803000pt}%
\definecolor{currentstroke}{rgb}{0.501961,0.501961,0.501961}%
\pgfsetstrokecolor{currentstroke}%
\pgfsetdash{}{0pt}%
\pgfpathmoveto{\pgfqpoint{5.539225in}{0.913832in}}%
\pgfpathlineto{\pgfqpoint{5.539225in}{1.668832in}}%
\pgfusepath{stroke}%
\end{pgfscope}%
\begin{pgfscope}%
\pgfsetrectcap%
\pgfsetmiterjoin%
\pgfsetlinewidth{0.803000pt}%
\definecolor{currentstroke}{rgb}{0.501961,0.501961,0.501961}%
\pgfsetstrokecolor{currentstroke}%
\pgfsetdash{}{0pt}%
\pgfpathmoveto{\pgfqpoint{4.376725in}{0.913832in}}%
\pgfpathlineto{\pgfqpoint{5.539225in}{0.913832in}}%
\pgfusepath{stroke}%
\end{pgfscope}%
\begin{pgfscope}%
\pgfsetrectcap%
\pgfsetmiterjoin%
\pgfsetlinewidth{0.803000pt}%
\definecolor{currentstroke}{rgb}{0.501961,0.501961,0.501961}%
\pgfsetstrokecolor{currentstroke}%
\pgfsetdash{}{0pt}%
\pgfpathmoveto{\pgfqpoint{4.376725in}{1.668832in}}%
\pgfpathlineto{\pgfqpoint{5.539225in}{1.668832in}}%
\pgfusepath{stroke}%
\end{pgfscope}%
\end{pgfpicture}%
\makeatother%
\endgroup%
}
    \caption{Scatter plot matrix of main parameters $E_0$, $U_0$, $V_d$, and $q$.\label{fig:scatter}}
\end{figure}

A two-way T-test comparison of charge distributions between the drop bounce experiment and a corollary experiment with zero electric field at the time of drop deposition on the superhydrophobic surface suggests that the drop charge is primarily induced by the electric field, rather than developed through contact charging on the PTFE layer ($t = 5.11$, $p = 0.0002$).

Incidentally, the model $q \sim kAE_0$ is similar to the classical solution for the surface charge of a half-spherical conductor with a field-induced dipole \cite{david_j._griffiths_introduction_1999}
\begin{eqnarray}
q &=& 3 \epsilon_0 E_0 \int_A \cos \theta dA \nonumber \\
&=& 3 \pi^{1/3} 6 \left(6 V_d \right)^{2/3} \epsilon_0 E_0 \int^{\pi / 2}_{0} \!\!\!\!\! \cos \theta d\theta \nonumber \\
&=& k A E_0 \label{dipole}
\end{eqnarray}
with $k \approx 1.3 \times 10^{-10}$ F/m. This is also of similar form to the charge found experimentally by Takamatsu and coauthors for drops falling from a grounded nozzle in an external electric field \cite{takamatsu_theoretical_1981}
\begin{equation}
q = 4 \pi \epsilon_0 \beta E_0 R_d^2
\label{Takamatsu}
\end{equation}
with $\beta \approx 2.63$. Casting Equation \ref{Takamatsu} in the same form as Equations \ref{ls_model} and \ref{dipole}, $k \approx 4 \pi \epsilon_0 \beta (4 \pi/3)^{-2/3} \approx 1.1 \times 10^{-10}$ F/m.

The effect of volume on jump velocity $U_0$ is not immediately evident in the data despite previous work having establishing this relationship \cite{attari_puddle_2016}. This likely results from large variance in $U_0$ due to contact line hysteresis during the drop roll-up. Contact line losses predominate in the sub 1 mL volume drops which are primarily the object of this study.  

\subsection{Model Validation}
Using the parameter estimates we find $\mathbb{E}\mbox{u}$ for each drop jump. Dimensional drop apoapses shown in Figure \ref{fig:series_s_eu} scale with $\mathbb{E}\mbox{u}$ as expected according to our earlier analysis. Electrostatic Euler numbers in the data set vary between $1.4 \lesssim \mathbb{E}\mbox{u} \lesssim 35.4$. \sout{The dielectrophoretic force plays a small role when drops have net charge in a DC field. The condition to neglect the DEP force was satisfied for all experiments in the dataset with typical values of the condition number $\kappa_w \epsilon_0 K R_d^2 E_0/q \approx \mathcal{O}(10^{-6})$}. In the non-dimensional trajectories with short-time scaling shown in Figure \ref{fig:series_s_ds}, we see that the scaled trajectory apoapses are consistently $\mathcal{O}(1)$, with all trajectories overshooting their characteristic time scale (which predicts returns at $t^*  =2$ at zeroth order). A trend of increasing $\mbox{max} \left(y^* \right)$ at apogee for decreasing $\mathbb{E}\mbox{u}$ is seen in Figure \ref{fig:yscale_trend}. This trend seems to decay to a constant $y_{max}/y_c \approx 1/2$ for $6 \lesssim \mathbb{E}\mbox{u} \lesssim 20$, and implies the existence an intermediate regime with a larger characteristic lengthscale for $\mathbb{E}\mbox{u} \lesssim 6$.
\begin{figure}[!htb]
    \centering
    \resizebox{0.5\textwidth}{!}{%% Creator: Matplotlib, PGF backend
%%
%% To include the figure in your LaTeX document, write
%%   \input{<filename>.pgf}
%%
%% Make sure the required packages are loaded in your preamble
%%   \usepackage{pgf}
%%
%% Figures using additional raster images can only be included by \input if
%% they are in the same directory as the main LaTeX file. For loading figures
%% from other directories you can use the `import` package
%%   \usepackage{import}
%% and then include the figures with
%%   \import{<path to file>}{<filename>.pgf}
%%
%% Matplotlib used the following preamble
%%   \usepackage[utf8x]{inputenc}
%%   \usepackage[T1]{fontenc}
%%
\begingroup%
\makeatletter%
\begin{pgfpicture}%
\pgfpathrectangle{\pgfpointorigin}{\pgfqpoint{5.770716in}{3.735201in}}%
\pgfusepath{use as bounding box, clip}%
\begin{pgfscope}%
\pgfsetbuttcap%
\pgfsetmiterjoin%
\definecolor{currentfill}{rgb}{1.000000,1.000000,1.000000}%
\pgfsetfillcolor{currentfill}%
\pgfsetlinewidth{0.000000pt}%
\definecolor{currentstroke}{rgb}{1.000000,1.000000,1.000000}%
\pgfsetstrokecolor{currentstroke}%
\pgfsetdash{}{0pt}%
\pgfpathmoveto{\pgfqpoint{0.000000in}{0.000000in}}%
\pgfpathlineto{\pgfqpoint{5.770716in}{0.000000in}}%
\pgfpathlineto{\pgfqpoint{5.770716in}{3.735201in}}%
\pgfpathlineto{\pgfqpoint{0.000000in}{3.735201in}}%
\pgfpathclose%
\pgfusepath{fill}%
\end{pgfscope}%
\begin{pgfscope}%
\pgfsetbuttcap%
\pgfsetmiterjoin%
\definecolor{currentfill}{rgb}{1.000000,1.000000,1.000000}%
\pgfsetfillcolor{currentfill}%
\pgfsetlinewidth{0.000000pt}%
\definecolor{currentstroke}{rgb}{0.000000,0.000000,0.000000}%
\pgfsetstrokecolor{currentstroke}%
\pgfsetstrokeopacity{0.000000}%
\pgfsetdash{}{0pt}%
\pgfpathmoveto{\pgfqpoint{0.501000in}{0.566590in}}%
\pgfpathlineto{\pgfqpoint{4.221000in}{0.566590in}}%
\pgfpathlineto{\pgfqpoint{4.221000in}{3.586590in}}%
\pgfpathlineto{\pgfqpoint{0.501000in}{3.586590in}}%
\pgfpathclose%
\pgfusepath{fill}%
\end{pgfscope}%
\begin{pgfscope}%
\pgfsetbuttcap%
\pgfsetroundjoin%
\definecolor{currentfill}{rgb}{0.000000,0.000000,0.000000}%
\pgfsetfillcolor{currentfill}%
\pgfsetlinewidth{0.803000pt}%
\definecolor{currentstroke}{rgb}{0.000000,0.000000,0.000000}%
\pgfsetstrokecolor{currentstroke}%
\pgfsetdash{}{0pt}%
\pgfsys@defobject{currentmarker}{\pgfqpoint{0.000000in}{-0.048611in}}{\pgfqpoint{0.000000in}{0.000000in}}{%
\pgfpathmoveto{\pgfqpoint{0.000000in}{0.000000in}}%
\pgfpathlineto{\pgfqpoint{0.000000in}{-0.048611in}}%
\pgfusepath{stroke,fill}%
}%
\begin{pgfscope}%
\pgfsys@transformshift{1.209572in}{0.566590in}%
\pgfsys@useobject{currentmarker}{}%
\end{pgfscope}%
\end{pgfscope}%
\begin{pgfscope}%
\pgftext[x=1.209572in,y=0.469368in,,top]{\rmfamily\fontsize{12.000000}{14.400000}\selectfont \(\displaystyle 0.5\)}%
\end{pgfscope}%
\begin{pgfscope}%
\pgfsetbuttcap%
\pgfsetroundjoin%
\definecolor{currentfill}{rgb}{0.000000,0.000000,0.000000}%
\pgfsetfillcolor{currentfill}%
\pgfsetlinewidth{0.803000pt}%
\definecolor{currentstroke}{rgb}{0.000000,0.000000,0.000000}%
\pgfsetstrokecolor{currentstroke}%
\pgfsetdash{}{0pt}%
\pgfsys@defobject{currentmarker}{\pgfqpoint{0.000000in}{-0.048611in}}{\pgfqpoint{0.000000in}{0.000000in}}{%
\pgfpathmoveto{\pgfqpoint{0.000000in}{0.000000in}}%
\pgfpathlineto{\pgfqpoint{0.000000in}{-0.048611in}}%
\pgfusepath{stroke,fill}%
}%
\begin{pgfscope}%
\pgfsys@transformshift{2.095286in}{0.566590in}%
\pgfsys@useobject{currentmarker}{}%
\end{pgfscope}%
\end{pgfscope}%
\begin{pgfscope}%
\pgftext[x=2.095286in,y=0.469368in,,top]{\rmfamily\fontsize{12.000000}{14.400000}\selectfont \(\displaystyle 1.0\)}%
\end{pgfscope}%
\begin{pgfscope}%
\pgfsetbuttcap%
\pgfsetroundjoin%
\definecolor{currentfill}{rgb}{0.000000,0.000000,0.000000}%
\pgfsetfillcolor{currentfill}%
\pgfsetlinewidth{0.803000pt}%
\definecolor{currentstroke}{rgb}{0.000000,0.000000,0.000000}%
\pgfsetstrokecolor{currentstroke}%
\pgfsetdash{}{0pt}%
\pgfsys@defobject{currentmarker}{\pgfqpoint{0.000000in}{-0.048611in}}{\pgfqpoint{0.000000in}{0.000000in}}{%
\pgfpathmoveto{\pgfqpoint{0.000000in}{0.000000in}}%
\pgfpathlineto{\pgfqpoint{0.000000in}{-0.048611in}}%
\pgfusepath{stroke,fill}%
}%
\begin{pgfscope}%
\pgfsys@transformshift{2.981000in}{0.566590in}%
\pgfsys@useobject{currentmarker}{}%
\end{pgfscope}%
\end{pgfscope}%
\begin{pgfscope}%
\pgftext[x=2.981000in,y=0.469368in,,top]{\rmfamily\fontsize{12.000000}{14.400000}\selectfont \(\displaystyle 1.5\)}%
\end{pgfscope}%
\begin{pgfscope}%
\pgfsetbuttcap%
\pgfsetroundjoin%
\definecolor{currentfill}{rgb}{0.000000,0.000000,0.000000}%
\pgfsetfillcolor{currentfill}%
\pgfsetlinewidth{0.803000pt}%
\definecolor{currentstroke}{rgb}{0.000000,0.000000,0.000000}%
\pgfsetstrokecolor{currentstroke}%
\pgfsetdash{}{0pt}%
\pgfsys@defobject{currentmarker}{\pgfqpoint{0.000000in}{-0.048611in}}{\pgfqpoint{0.000000in}{0.000000in}}{%
\pgfpathmoveto{\pgfqpoint{0.000000in}{0.000000in}}%
\pgfpathlineto{\pgfqpoint{0.000000in}{-0.048611in}}%
\pgfusepath{stroke,fill}%
}%
\begin{pgfscope}%
\pgfsys@transformshift{3.866714in}{0.566590in}%
\pgfsys@useobject{currentmarker}{}%
\end{pgfscope}%
\end{pgfscope}%
\begin{pgfscope}%
\pgftext[x=3.866714in,y=0.469368in,,top]{\rmfamily\fontsize{12.000000}{14.400000}\selectfont \(\displaystyle 2.0\)}%
\end{pgfscope}%
\begin{pgfscope}%
\pgftext[x=2.361000in,y=0.266626in,,top]{\rmfamily\fontsize{12.000000}{14.400000}\selectfont \(\displaystyle t\) (s)}%
\end{pgfscope}%
\begin{pgfscope}%
\pgfsetbuttcap%
\pgfsetroundjoin%
\definecolor{currentfill}{rgb}{0.000000,0.000000,0.000000}%
\pgfsetfillcolor{currentfill}%
\pgfsetlinewidth{0.803000pt}%
\definecolor{currentstroke}{rgb}{0.000000,0.000000,0.000000}%
\pgfsetstrokecolor{currentstroke}%
\pgfsetdash{}{0pt}%
\pgfsys@defobject{currentmarker}{\pgfqpoint{-0.048611in}{0.000000in}}{\pgfqpoint{0.000000in}{0.000000in}}{%
\pgfpathmoveto{\pgfqpoint{0.000000in}{0.000000in}}%
\pgfpathlineto{\pgfqpoint{-0.048611in}{0.000000in}}%
\pgfusepath{stroke,fill}%
}%
\begin{pgfscope}%
\pgfsys@transformshift{0.501000in}{0.642969in}%
\pgfsys@useobject{currentmarker}{}%
\end{pgfscope}%
\end{pgfscope}%
\begin{pgfscope}%
\pgftext[x=0.322182in,y=0.585576in,left,base]{\rmfamily\fontsize{12.000000}{14.400000}\selectfont \(\displaystyle 0\)}%
\end{pgfscope}%
\begin{pgfscope}%
\pgfsetbuttcap%
\pgfsetroundjoin%
\definecolor{currentfill}{rgb}{0.000000,0.000000,0.000000}%
\pgfsetfillcolor{currentfill}%
\pgfsetlinewidth{0.803000pt}%
\definecolor{currentstroke}{rgb}{0.000000,0.000000,0.000000}%
\pgfsetstrokecolor{currentstroke}%
\pgfsetdash{}{0pt}%
\pgfsys@defobject{currentmarker}{\pgfqpoint{-0.048611in}{0.000000in}}{\pgfqpoint{0.000000in}{0.000000in}}{%
\pgfpathmoveto{\pgfqpoint{0.000000in}{0.000000in}}%
\pgfpathlineto{\pgfqpoint{-0.048611in}{0.000000in}}%
\pgfusepath{stroke,fill}%
}%
\begin{pgfscope}%
\pgfsys@transformshift{0.501000in}{1.009698in}%
\pgfsys@useobject{currentmarker}{}%
\end{pgfscope}%
\end{pgfscope}%
\begin{pgfscope}%
\pgftext[x=0.322182in,y=0.952305in,left,base]{\rmfamily\fontsize{12.000000}{14.400000}\selectfont \(\displaystyle 1\)}%
\end{pgfscope}%
\begin{pgfscope}%
\pgfsetbuttcap%
\pgfsetroundjoin%
\definecolor{currentfill}{rgb}{0.000000,0.000000,0.000000}%
\pgfsetfillcolor{currentfill}%
\pgfsetlinewidth{0.803000pt}%
\definecolor{currentstroke}{rgb}{0.000000,0.000000,0.000000}%
\pgfsetstrokecolor{currentstroke}%
\pgfsetdash{}{0pt}%
\pgfsys@defobject{currentmarker}{\pgfqpoint{-0.048611in}{0.000000in}}{\pgfqpoint{0.000000in}{0.000000in}}{%
\pgfpathmoveto{\pgfqpoint{0.000000in}{0.000000in}}%
\pgfpathlineto{\pgfqpoint{-0.048611in}{0.000000in}}%
\pgfusepath{stroke,fill}%
}%
\begin{pgfscope}%
\pgfsys@transformshift{0.501000in}{1.376427in}%
\pgfsys@useobject{currentmarker}{}%
\end{pgfscope}%
\end{pgfscope}%
\begin{pgfscope}%
\pgftext[x=0.322182in,y=1.319033in,left,base]{\rmfamily\fontsize{12.000000}{14.400000}\selectfont \(\displaystyle 2\)}%
\end{pgfscope}%
\begin{pgfscope}%
\pgfsetbuttcap%
\pgfsetroundjoin%
\definecolor{currentfill}{rgb}{0.000000,0.000000,0.000000}%
\pgfsetfillcolor{currentfill}%
\pgfsetlinewidth{0.803000pt}%
\definecolor{currentstroke}{rgb}{0.000000,0.000000,0.000000}%
\pgfsetstrokecolor{currentstroke}%
\pgfsetdash{}{0pt}%
\pgfsys@defobject{currentmarker}{\pgfqpoint{-0.048611in}{0.000000in}}{\pgfqpoint{0.000000in}{0.000000in}}{%
\pgfpathmoveto{\pgfqpoint{0.000000in}{0.000000in}}%
\pgfpathlineto{\pgfqpoint{-0.048611in}{0.000000in}}%
\pgfusepath{stroke,fill}%
}%
\begin{pgfscope}%
\pgfsys@transformshift{0.501000in}{1.743156in}%
\pgfsys@useobject{currentmarker}{}%
\end{pgfscope}%
\end{pgfscope}%
\begin{pgfscope}%
\pgftext[x=0.322182in,y=1.685762in,left,base]{\rmfamily\fontsize{12.000000}{14.400000}\selectfont \(\displaystyle 3\)}%
\end{pgfscope}%
\begin{pgfscope}%
\pgfsetbuttcap%
\pgfsetroundjoin%
\definecolor{currentfill}{rgb}{0.000000,0.000000,0.000000}%
\pgfsetfillcolor{currentfill}%
\pgfsetlinewidth{0.803000pt}%
\definecolor{currentstroke}{rgb}{0.000000,0.000000,0.000000}%
\pgfsetstrokecolor{currentstroke}%
\pgfsetdash{}{0pt}%
\pgfsys@defobject{currentmarker}{\pgfqpoint{-0.048611in}{0.000000in}}{\pgfqpoint{0.000000in}{0.000000in}}{%
\pgfpathmoveto{\pgfqpoint{0.000000in}{0.000000in}}%
\pgfpathlineto{\pgfqpoint{-0.048611in}{0.000000in}}%
\pgfusepath{stroke,fill}%
}%
\begin{pgfscope}%
\pgfsys@transformshift{0.501000in}{2.109884in}%
\pgfsys@useobject{currentmarker}{}%
\end{pgfscope}%
\end{pgfscope}%
\begin{pgfscope}%
\pgftext[x=0.322182in,y=2.052491in,left,base]{\rmfamily\fontsize{12.000000}{14.400000}\selectfont \(\displaystyle 4\)}%
\end{pgfscope}%
\begin{pgfscope}%
\pgfsetbuttcap%
\pgfsetroundjoin%
\definecolor{currentfill}{rgb}{0.000000,0.000000,0.000000}%
\pgfsetfillcolor{currentfill}%
\pgfsetlinewidth{0.803000pt}%
\definecolor{currentstroke}{rgb}{0.000000,0.000000,0.000000}%
\pgfsetstrokecolor{currentstroke}%
\pgfsetdash{}{0pt}%
\pgfsys@defobject{currentmarker}{\pgfqpoint{-0.048611in}{0.000000in}}{\pgfqpoint{0.000000in}{0.000000in}}{%
\pgfpathmoveto{\pgfqpoint{0.000000in}{0.000000in}}%
\pgfpathlineto{\pgfqpoint{-0.048611in}{0.000000in}}%
\pgfusepath{stroke,fill}%
}%
\begin{pgfscope}%
\pgfsys@transformshift{0.501000in}{2.476613in}%
\pgfsys@useobject{currentmarker}{}%
\end{pgfscope}%
\end{pgfscope}%
\begin{pgfscope}%
\pgftext[x=0.322182in,y=2.419220in,left,base]{\rmfamily\fontsize{12.000000}{14.400000}\selectfont \(\displaystyle 5\)}%
\end{pgfscope}%
\begin{pgfscope}%
\pgfsetbuttcap%
\pgfsetroundjoin%
\definecolor{currentfill}{rgb}{0.000000,0.000000,0.000000}%
\pgfsetfillcolor{currentfill}%
\pgfsetlinewidth{0.803000pt}%
\definecolor{currentstroke}{rgb}{0.000000,0.000000,0.000000}%
\pgfsetstrokecolor{currentstroke}%
\pgfsetdash{}{0pt}%
\pgfsys@defobject{currentmarker}{\pgfqpoint{-0.048611in}{0.000000in}}{\pgfqpoint{0.000000in}{0.000000in}}{%
\pgfpathmoveto{\pgfqpoint{0.000000in}{0.000000in}}%
\pgfpathlineto{\pgfqpoint{-0.048611in}{0.000000in}}%
\pgfusepath{stroke,fill}%
}%
\begin{pgfscope}%
\pgfsys@transformshift{0.501000in}{2.843342in}%
\pgfsys@useobject{currentmarker}{}%
\end{pgfscope}%
\end{pgfscope}%
\begin{pgfscope}%
\pgftext[x=0.322182in,y=2.785949in,left,base]{\rmfamily\fontsize{12.000000}{14.400000}\selectfont \(\displaystyle 6\)}%
\end{pgfscope}%
\begin{pgfscope}%
\pgfsetbuttcap%
\pgfsetroundjoin%
\definecolor{currentfill}{rgb}{0.000000,0.000000,0.000000}%
\pgfsetfillcolor{currentfill}%
\pgfsetlinewidth{0.803000pt}%
\definecolor{currentstroke}{rgb}{0.000000,0.000000,0.000000}%
\pgfsetstrokecolor{currentstroke}%
\pgfsetdash{}{0pt}%
\pgfsys@defobject{currentmarker}{\pgfqpoint{-0.048611in}{0.000000in}}{\pgfqpoint{0.000000in}{0.000000in}}{%
\pgfpathmoveto{\pgfqpoint{0.000000in}{0.000000in}}%
\pgfpathlineto{\pgfqpoint{-0.048611in}{0.000000in}}%
\pgfusepath{stroke,fill}%
}%
\begin{pgfscope}%
\pgfsys@transformshift{0.501000in}{3.210071in}%
\pgfsys@useobject{currentmarker}{}%
\end{pgfscope}%
\end{pgfscope}%
\begin{pgfscope}%
\pgftext[x=0.322182in,y=3.152677in,left,base]{\rmfamily\fontsize{12.000000}{14.400000}\selectfont \(\displaystyle 7\)}%
\end{pgfscope}%
\begin{pgfscope}%
\pgfsetbuttcap%
\pgfsetroundjoin%
\definecolor{currentfill}{rgb}{0.000000,0.000000,0.000000}%
\pgfsetfillcolor{currentfill}%
\pgfsetlinewidth{0.803000pt}%
\definecolor{currentstroke}{rgb}{0.000000,0.000000,0.000000}%
\pgfsetstrokecolor{currentstroke}%
\pgfsetdash{}{0pt}%
\pgfsys@defobject{currentmarker}{\pgfqpoint{-0.048611in}{0.000000in}}{\pgfqpoint{0.000000in}{0.000000in}}{%
\pgfpathmoveto{\pgfqpoint{0.000000in}{0.000000in}}%
\pgfpathlineto{\pgfqpoint{-0.048611in}{0.000000in}}%
\pgfusepath{stroke,fill}%
}%
\begin{pgfscope}%
\pgfsys@transformshift{0.501000in}{3.576800in}%
\pgfsys@useobject{currentmarker}{}%
\end{pgfscope}%
\end{pgfscope}%
\begin{pgfscope}%
\pgftext[x=0.322182in,y=3.519406in,left,base]{\rmfamily\fontsize{12.000000}{14.400000}\selectfont \(\displaystyle 8\)}%
\end{pgfscope}%
\begin{pgfscope}%
\pgftext[x=0.266626in,y=2.076590in,,bottom,rotate=90.000000]{\rmfamily\fontsize{12.000000}{14.400000}\selectfont \(\displaystyle y\) (cm)}%
\end{pgfscope}%
\begin{pgfscope}%
\pgfpathrectangle{\pgfqpoint{0.501000in}{0.566590in}}{\pgfqpoint{3.720000in}{3.020000in}} %
\pgfusepath{clip}%
\pgfsetrectcap%
\pgfsetroundjoin%
\pgfsetlinewidth{1.505625pt}%
\definecolor{currentstroke}{rgb}{0.500000,0.000000,1.000000}%
\pgfsetstrokecolor{currentstroke}%
\pgfsetdash{}{0pt}%
\pgfpathmoveto{\pgfqpoint{0.487111in}{1.045226in}}%
\pgfpathlineto{\pgfqpoint{0.501000in}{1.074337in}}%
\pgfpathlineto{\pgfqpoint{0.515762in}{1.102645in}}%
\pgfpathlineto{\pgfqpoint{0.530524in}{1.126774in}}%
\pgfpathlineto{\pgfqpoint{0.545286in}{1.151928in}}%
\pgfpathlineto{\pgfqpoint{0.560048in}{1.174547in}}%
\pgfpathlineto{\pgfqpoint{0.574810in}{1.201417in}}%
\pgfpathlineto{\pgfqpoint{0.589572in}{1.230112in}}%
\pgfpathlineto{\pgfqpoint{0.604333in}{1.251935in}}%
\pgfpathlineto{\pgfqpoint{0.619095in}{1.275884in}}%
\pgfpathlineto{\pgfqpoint{0.633857in}{1.305137in}}%
\pgfpathlineto{\pgfqpoint{0.648619in}{1.331365in}}%
\pgfpathlineto{\pgfqpoint{0.663381in}{1.354408in}}%
\pgfpathlineto{\pgfqpoint{0.678143in}{1.374340in}}%
\pgfpathlineto{\pgfqpoint{0.692905in}{1.400486in}}%
\pgfpathlineto{\pgfqpoint{0.707667in}{1.427625in}}%
\pgfpathlineto{\pgfqpoint{0.722429in}{1.453041in}}%
\pgfpathlineto{\pgfqpoint{0.737191in}{1.474783in}}%
\pgfpathlineto{\pgfqpoint{0.751953in}{1.499082in}}%
\pgfpathlineto{\pgfqpoint{0.766714in}{1.525359in}}%
\pgfpathlineto{\pgfqpoint{0.781476in}{1.548829in}}%
\pgfpathlineto{\pgfqpoint{0.796238in}{1.569832in}}%
\pgfpathlineto{\pgfqpoint{0.811000in}{1.593525in}}%
\pgfpathlineto{\pgfqpoint{0.825762in}{1.618324in}}%
\pgfpathlineto{\pgfqpoint{0.840524in}{1.643381in}}%
\pgfpathlineto{\pgfqpoint{0.855286in}{1.663311in}}%
\pgfpathlineto{\pgfqpoint{0.870048in}{1.687126in}}%
\pgfpathlineto{\pgfqpoint{0.884810in}{1.711993in}}%
\pgfpathlineto{\pgfqpoint{0.899572in}{1.735587in}}%
\pgfpathlineto{\pgfqpoint{0.914333in}{1.756876in}}%
\pgfpathlineto{\pgfqpoint{0.929095in}{1.778567in}}%
\pgfpathlineto{\pgfqpoint{0.943857in}{1.802092in}}%
\pgfpathlineto{\pgfqpoint{0.958619in}{1.826311in}}%
\pgfpathlineto{\pgfqpoint{0.973381in}{1.845729in}}%
\pgfpathlineto{\pgfqpoint{0.988143in}{1.869787in}}%
\pgfpathlineto{\pgfqpoint{1.002905in}{1.892445in}}%
\pgfpathlineto{\pgfqpoint{1.017667in}{1.915958in}}%
\pgfpathlineto{\pgfqpoint{1.032429in}{1.937485in}}%
\pgfpathlineto{\pgfqpoint{1.047191in}{1.958490in}}%
\pgfpathlineto{\pgfqpoint{1.061953in}{1.978498in}}%
\pgfpathlineto{\pgfqpoint{1.076714in}{2.002561in}}%
\pgfpathlineto{\pgfqpoint{1.091476in}{2.023527in}}%
\pgfpathlineto{\pgfqpoint{1.106238in}{2.047054in}}%
\pgfpathlineto{\pgfqpoint{1.121000in}{2.069427in}}%
\pgfpathlineto{\pgfqpoint{1.135762in}{2.092499in}}%
\pgfpathlineto{\pgfqpoint{1.150524in}{2.114376in}}%
\pgfpathlineto{\pgfqpoint{1.165286in}{2.133825in}}%
\pgfpathlineto{\pgfqpoint{1.180048in}{2.153003in}}%
\pgfpathlineto{\pgfqpoint{1.194810in}{2.175907in}}%
\pgfpathlineto{\pgfqpoint{1.209572in}{2.197673in}}%
\pgfpathlineto{\pgfqpoint{1.224333in}{2.220081in}}%
\pgfpathlineto{\pgfqpoint{1.239095in}{2.242686in}}%
\pgfpathlineto{\pgfqpoint{1.253857in}{2.265069in}}%
\pgfpathlineto{\pgfqpoint{1.268619in}{2.287084in}}%
\pgfpathlineto{\pgfqpoint{1.283381in}{2.306296in}}%
\pgfpathlineto{\pgfqpoint{1.298143in}{2.324973in}}%
\pgfpathlineto{\pgfqpoint{1.312905in}{2.345597in}}%
\pgfpathlineto{\pgfqpoint{1.327667in}{2.368203in}}%
\pgfpathlineto{\pgfqpoint{1.342429in}{2.390363in}}%
\pgfpathlineto{\pgfqpoint{1.357191in}{2.413851in}}%
\pgfpathlineto{\pgfqpoint{1.371953in}{2.435551in}}%
\pgfpathlineto{\pgfqpoint{1.386714in}{2.457271in}}%
\pgfpathlineto{\pgfqpoint{1.401476in}{2.475980in}}%
\pgfpathlineto{\pgfqpoint{1.416238in}{2.494665in}}%
\pgfpathlineto{\pgfqpoint{1.431000in}{2.514310in}}%
\pgfpathlineto{\pgfqpoint{1.445762in}{2.535941in}}%
\pgfpathlineto{\pgfqpoint{1.460524in}{2.558888in}}%
\pgfpathlineto{\pgfqpoint{1.475286in}{2.582619in}}%
\pgfpathlineto{\pgfqpoint{1.490048in}{2.603774in}}%
\pgfpathlineto{\pgfqpoint{1.504810in}{2.624842in}}%
\pgfpathlineto{\pgfqpoint{1.519572in}{2.643625in}}%
\pgfpathlineto{\pgfqpoint{1.534333in}{2.661259in}}%
\pgfpathlineto{\pgfqpoint{1.549095in}{2.680627in}}%
\pgfpathlineto{\pgfqpoint{1.563857in}{2.701084in}}%
\pgfpathlineto{\pgfqpoint{1.578619in}{2.724323in}}%
\pgfpathlineto{\pgfqpoint{1.593381in}{2.747477in}}%
\pgfpathlineto{\pgfqpoint{1.608143in}{2.769253in}}%
\pgfpathlineto{\pgfqpoint{1.622905in}{2.790354in}}%
\pgfpathlineto{\pgfqpoint{1.637667in}{2.808290in}}%
\pgfpathlineto{\pgfqpoint{1.652429in}{2.825577in}}%
\pgfpathlineto{\pgfqpoint{1.667191in}{2.844883in}}%
\pgfpathlineto{\pgfqpoint{1.681953in}{2.865440in}}%
\pgfpathlineto{\pgfqpoint{1.696714in}{2.887808in}}%
\pgfpathlineto{\pgfqpoint{1.711476in}{2.911400in}}%
\pgfpathlineto{\pgfqpoint{1.726238in}{2.932485in}}%
\pgfpathlineto{\pgfqpoint{1.741000in}{2.952895in}}%
\pgfpathlineto{\pgfqpoint{1.755762in}{2.971551in}}%
\pgfpathlineto{\pgfqpoint{1.770524in}{2.987888in}}%
\pgfpathlineto{\pgfqpoint{1.785286in}{3.006680in}}%
\pgfpathlineto{\pgfqpoint{1.800048in}{3.026680in}}%
\pgfpathlineto{\pgfqpoint{1.814810in}{3.048746in}}%
\pgfpathlineto{\pgfqpoint{1.829572in}{3.072675in}}%
\pgfpathlineto{\pgfqpoint{1.844333in}{3.093598in}}%
\pgfpathlineto{\pgfqpoint{1.859095in}{3.113454in}}%
\pgfpathlineto{\pgfqpoint{1.873857in}{3.133018in}}%
\pgfpathlineto{\pgfqpoint{1.888619in}{3.148914in}}%
\pgfpathlineto{\pgfqpoint{1.903381in}{3.167103in}}%
\pgfpathlineto{\pgfqpoint{1.918143in}{3.187055in}}%
\pgfpathlineto{\pgfqpoint{1.932905in}{3.209492in}}%
\pgfpathlineto{\pgfqpoint{1.947667in}{3.232609in}}%
\pgfpathlineto{\pgfqpoint{1.962429in}{3.252922in}}%
\pgfpathlineto{\pgfqpoint{1.977191in}{3.272361in}}%
\pgfpathlineto{\pgfqpoint{1.991953in}{3.292171in}}%
\pgfpathlineto{\pgfqpoint{2.006714in}{3.308472in}}%
\pgfpathlineto{\pgfqpoint{2.021476in}{3.325817in}}%
\pgfpathlineto{\pgfqpoint{2.036238in}{3.345986in}}%
\pgfpathlineto{\pgfqpoint{2.051000in}{3.368580in}}%
\pgfpathlineto{\pgfqpoint{2.065762in}{3.391434in}}%
\pgfpathlineto{\pgfqpoint{2.080524in}{3.410321in}}%
\pgfpathlineto{\pgfqpoint{2.095286in}{3.431514in}}%
\pgfpathlineto{\pgfqpoint{2.110048in}{3.449317in}}%
\pgfusepath{stroke}%
\end{pgfscope}%
\begin{pgfscope}%
\pgfpathrectangle{\pgfqpoint{0.501000in}{0.566590in}}{\pgfqpoint{3.720000in}{3.020000in}} %
\pgfusepath{clip}%
\pgfsetrectcap%
\pgfsetroundjoin%
\pgfsetlinewidth{1.505625pt}%
\definecolor{currentstroke}{rgb}{0.013725,0.691698,0.927951}%
\pgfsetstrokecolor{currentstroke}%
\pgfsetdash{}{0pt}%
\pgfpathmoveto{\pgfqpoint{0.487111in}{0.979381in}}%
\pgfpathlineto{\pgfqpoint{0.501000in}{0.986006in}}%
\pgfpathlineto{\pgfqpoint{0.515762in}{1.018330in}}%
\pgfpathlineto{\pgfqpoint{0.530524in}{1.045355in}}%
\pgfpathlineto{\pgfqpoint{0.545286in}{1.060987in}}%
\pgfpathlineto{\pgfqpoint{0.560048in}{1.080810in}}%
\pgfpathlineto{\pgfqpoint{0.574810in}{1.105099in}}%
\pgfpathlineto{\pgfqpoint{0.589572in}{1.133063in}}%
\pgfpathlineto{\pgfqpoint{0.604333in}{1.152288in}}%
\pgfpathlineto{\pgfqpoint{0.619095in}{1.168475in}}%
\pgfpathlineto{\pgfqpoint{0.633857in}{1.190321in}}%
\pgfpathlineto{\pgfqpoint{0.648619in}{1.216028in}}%
\pgfpathlineto{\pgfqpoint{0.663381in}{1.235493in}}%
\pgfpathlineto{\pgfqpoint{0.692905in}{1.268933in}}%
\pgfpathlineto{\pgfqpoint{0.707667in}{1.290013in}}%
\pgfpathlineto{\pgfqpoint{0.722429in}{1.314282in}}%
\pgfpathlineto{\pgfqpoint{0.737191in}{1.332092in}}%
\pgfpathlineto{\pgfqpoint{0.751953in}{1.345020in}}%
\pgfpathlineto{\pgfqpoint{0.766714in}{1.364786in}}%
\pgfpathlineto{\pgfqpoint{0.781476in}{1.387358in}}%
\pgfpathlineto{\pgfqpoint{0.796238in}{1.405143in}}%
\pgfpathlineto{\pgfqpoint{0.825762in}{1.434087in}}%
\pgfpathlineto{\pgfqpoint{0.855286in}{1.474191in}}%
\pgfpathlineto{\pgfqpoint{0.870048in}{1.490116in}}%
\pgfpathlineto{\pgfqpoint{0.884810in}{1.501728in}}%
\pgfpathlineto{\pgfqpoint{0.929095in}{1.555884in}}%
\pgfpathlineto{\pgfqpoint{0.958619in}{1.581664in}}%
\pgfpathlineto{\pgfqpoint{0.973381in}{1.598595in}}%
\pgfpathlineto{\pgfqpoint{0.988143in}{1.618034in}}%
\pgfpathlineto{\pgfqpoint{1.032429in}{1.658006in}}%
\pgfpathlineto{\pgfqpoint{1.047191in}{1.676008in}}%
\pgfpathlineto{\pgfqpoint{1.061953in}{1.690595in}}%
\pgfpathlineto{\pgfqpoint{1.091476in}{1.713719in}}%
\pgfpathlineto{\pgfqpoint{1.135762in}{1.758477in}}%
\pgfpathlineto{\pgfqpoint{1.150524in}{1.768174in}}%
\pgfpathlineto{\pgfqpoint{1.194810in}{1.812710in}}%
\pgfpathlineto{\pgfqpoint{1.224333in}{1.833693in}}%
\pgfpathlineto{\pgfqpoint{1.253857in}{1.863761in}}%
\pgfpathlineto{\pgfqpoint{1.283381in}{1.883579in}}%
\pgfpathlineto{\pgfqpoint{1.327667in}{1.923974in}}%
\pgfpathlineto{\pgfqpoint{1.357191in}{1.943216in}}%
\pgfpathlineto{\pgfqpoint{1.386714in}{1.970883in}}%
\pgfpathlineto{\pgfqpoint{1.416238in}{1.989079in}}%
\pgfpathlineto{\pgfqpoint{1.460524in}{2.026049in}}%
\pgfpathlineto{\pgfqpoint{1.490048in}{2.043853in}}%
\pgfpathlineto{\pgfqpoint{1.519572in}{2.069316in}}%
\pgfpathlineto{\pgfqpoint{1.549095in}{2.086283in}}%
\pgfpathlineto{\pgfqpoint{1.593381in}{2.120465in}}%
\pgfpathlineto{\pgfqpoint{1.622905in}{2.136877in}}%
\pgfpathlineto{\pgfqpoint{1.652429in}{2.160377in}}%
\pgfpathlineto{\pgfqpoint{1.681953in}{2.176095in}}%
\pgfpathlineto{\pgfqpoint{1.726238in}{2.207610in}}%
\pgfpathlineto{\pgfqpoint{1.755762in}{2.222785in}}%
\pgfpathlineto{\pgfqpoint{1.785286in}{2.244347in}}%
\pgfpathlineto{\pgfqpoint{1.814810in}{2.259144in}}%
\pgfpathlineto{\pgfqpoint{1.859095in}{2.288288in}}%
\pgfpathlineto{\pgfqpoint{1.888619in}{2.302669in}}%
\pgfpathlineto{\pgfqpoint{1.918143in}{2.322160in}}%
\pgfpathlineto{\pgfqpoint{1.947667in}{2.336288in}}%
\pgfpathlineto{\pgfqpoint{1.991953in}{2.362964in}}%
\pgfpathlineto{\pgfqpoint{2.006714in}{2.369039in}}%
\pgfpathlineto{\pgfqpoint{2.036238in}{2.386228in}}%
\pgfpathlineto{\pgfqpoint{2.051000in}{2.394527in}}%
\pgfpathlineto{\pgfqpoint{2.080524in}{2.407860in}}%
\pgfpathlineto{\pgfqpoint{2.124810in}{2.432327in}}%
\pgfpathlineto{\pgfqpoint{2.154333in}{2.445373in}}%
\pgfpathlineto{\pgfqpoint{2.183857in}{2.461965in}}%
\pgfpathlineto{\pgfqpoint{2.213381in}{2.474317in}}%
\pgfpathlineto{\pgfqpoint{2.257667in}{2.497176in}}%
\pgfpathlineto{\pgfqpoint{2.287191in}{2.509391in}}%
\pgfpathlineto{\pgfqpoint{2.316714in}{2.524978in}}%
\pgfpathlineto{\pgfqpoint{2.346238in}{2.536699in}}%
\pgfpathlineto{\pgfqpoint{2.390524in}{2.557787in}}%
\pgfpathlineto{\pgfqpoint{2.420048in}{2.569541in}}%
\pgfpathlineto{\pgfqpoint{2.449572in}{2.584141in}}%
\pgfpathlineto{\pgfqpoint{2.479095in}{2.595001in}}%
\pgfpathlineto{\pgfqpoint{2.508619in}{2.609226in}}%
\pgfpathlineto{\pgfqpoint{2.552905in}{2.625835in}}%
\pgfpathlineto{\pgfqpoint{2.582429in}{2.639233in}}%
\pgfpathlineto{\pgfqpoint{2.611953in}{2.649410in}}%
\pgfpathlineto{\pgfqpoint{2.641476in}{2.662621in}}%
\pgfpathlineto{\pgfqpoint{2.685762in}{2.678389in}}%
\pgfpathlineto{\pgfqpoint{2.715286in}{2.690555in}}%
\pgfpathlineto{\pgfqpoint{2.744810in}{2.700127in}}%
\pgfpathlineto{\pgfqpoint{2.774333in}{2.712396in}}%
\pgfpathlineto{\pgfqpoint{2.803857in}{2.721179in}}%
\pgfpathlineto{\pgfqpoint{2.848143in}{2.738669in}}%
\pgfpathlineto{\pgfqpoint{2.877667in}{2.747785in}}%
\pgfpathlineto{\pgfqpoint{2.907191in}{2.758966in}}%
\pgfpathlineto{\pgfqpoint{2.936714in}{2.767464in}}%
\pgfpathlineto{\pgfqpoint{2.966238in}{2.779123in}}%
\pgfpathlineto{\pgfqpoint{3.069572in}{2.810585in}}%
\pgfpathlineto{\pgfqpoint{3.099095in}{2.821457in}}%
\pgfpathlineto{\pgfqpoint{3.158143in}{2.838513in}}%
\pgfpathlineto{\pgfqpoint{3.246714in}{2.864446in}}%
\pgfpathlineto{\pgfqpoint{3.276238in}{2.871767in}}%
\pgfpathlineto{\pgfqpoint{3.305762in}{2.880454in}}%
\pgfpathlineto{\pgfqpoint{3.335286in}{2.888133in}}%
\pgfpathlineto{\pgfqpoint{3.364810in}{2.897083in}}%
\pgfpathlineto{\pgfqpoint{3.468143in}{2.923006in}}%
\pgfpathlineto{\pgfqpoint{3.497667in}{2.931099in}}%
\pgfpathlineto{\pgfqpoint{3.541953in}{2.941220in}}%
\pgfpathlineto{\pgfqpoint{3.615762in}{2.959134in}}%
\pgfpathlineto{\pgfqpoint{3.704333in}{2.978463in}}%
\pgfpathlineto{\pgfqpoint{3.778143in}{2.993916in}}%
\pgfpathlineto{\pgfqpoint{3.778143in}{2.993916in}}%
\pgfusepath{stroke}%
\end{pgfscope}%
\begin{pgfscope}%
\pgfpathrectangle{\pgfqpoint{0.501000in}{0.566590in}}{\pgfqpoint{3.720000in}{3.020000in}} %
\pgfusepath{clip}%
\pgfsetrectcap%
\pgfsetroundjoin%
\pgfsetlinewidth{1.505625pt}%
\definecolor{currentstroke}{rgb}{0.374510,0.195845,0.995147}%
\pgfsetstrokecolor{currentstroke}%
\pgfsetdash{}{0pt}%
\pgfpathmoveto{\pgfqpoint{0.487111in}{0.982800in}}%
\pgfpathlineto{\pgfqpoint{0.501000in}{1.008468in}}%
\pgfpathlineto{\pgfqpoint{0.545286in}{1.072766in}}%
\pgfpathlineto{\pgfqpoint{0.574810in}{1.125558in}}%
\pgfpathlineto{\pgfqpoint{0.589572in}{1.143479in}}%
\pgfpathlineto{\pgfqpoint{0.604333in}{1.164436in}}%
\pgfpathlineto{\pgfqpoint{0.633857in}{1.215673in}}%
\pgfpathlineto{\pgfqpoint{0.663381in}{1.253527in}}%
\pgfpathlineto{\pgfqpoint{0.678143in}{1.275531in}}%
\pgfpathlineto{\pgfqpoint{0.692905in}{1.301573in}}%
\pgfpathlineto{\pgfqpoint{0.707667in}{1.323177in}}%
\pgfpathlineto{\pgfqpoint{0.722429in}{1.340508in}}%
\pgfpathlineto{\pgfqpoint{0.737191in}{1.361021in}}%
\pgfpathlineto{\pgfqpoint{0.766714in}{1.407964in}}%
\pgfpathlineto{\pgfqpoint{0.796238in}{1.444001in}}%
\pgfpathlineto{\pgfqpoint{0.840524in}{1.508850in}}%
\pgfpathlineto{\pgfqpoint{0.855286in}{1.525728in}}%
\pgfpathlineto{\pgfqpoint{0.914333in}{1.606001in}}%
\pgfpathlineto{\pgfqpoint{0.929095in}{1.624131in}}%
\pgfpathlineto{\pgfqpoint{0.958619in}{1.666897in}}%
\pgfpathlineto{\pgfqpoint{0.988143in}{1.701131in}}%
\pgfpathlineto{\pgfqpoint{1.032429in}{1.761374in}}%
\pgfpathlineto{\pgfqpoint{1.061953in}{1.795948in}}%
\pgfpathlineto{\pgfqpoint{1.091476in}{1.835967in}}%
\pgfpathlineto{\pgfqpoint{1.121000in}{1.869410in}}%
\pgfpathlineto{\pgfqpoint{1.150524in}{1.909393in}}%
\pgfpathlineto{\pgfqpoint{1.194810in}{1.960658in}}%
\pgfpathlineto{\pgfqpoint{1.224333in}{1.998255in}}%
\pgfpathlineto{\pgfqpoint{1.253857in}{2.031768in}}%
\pgfpathlineto{\pgfqpoint{1.283381in}{2.069315in}}%
\pgfpathlineto{\pgfqpoint{1.327667in}{2.119983in}}%
\pgfpathlineto{\pgfqpoint{1.357191in}{2.154947in}}%
\pgfpathlineto{\pgfqpoint{1.386714in}{2.188518in}}%
\pgfpathlineto{\pgfqpoint{1.416238in}{2.224273in}}%
\pgfpathlineto{\pgfqpoint{1.445762in}{2.256415in}}%
\pgfpathlineto{\pgfqpoint{1.475286in}{2.291984in}}%
\pgfpathlineto{\pgfqpoint{1.504810in}{2.323604in}}%
\pgfpathlineto{\pgfqpoint{1.549095in}{2.375305in}}%
\pgfpathlineto{\pgfqpoint{1.578619in}{2.407382in}}%
\pgfpathlineto{\pgfqpoint{1.608143in}{2.441462in}}%
\pgfpathlineto{\pgfqpoint{1.637667in}{2.472970in}}%
\pgfpathlineto{\pgfqpoint{1.681953in}{2.523057in}}%
\pgfpathlineto{\pgfqpoint{1.726238in}{2.572077in}}%
\pgfpathlineto{\pgfqpoint{2.124810in}{2.998282in}}%
\pgfpathlineto{\pgfqpoint{2.198619in}{3.075163in}}%
\pgfpathlineto{\pgfqpoint{2.464333in}{3.347645in}}%
\pgfpathlineto{\pgfqpoint{2.464333in}{3.347645in}}%
\pgfusepath{stroke}%
\end{pgfscope}%
\begin{pgfscope}%
\pgfpathrectangle{\pgfqpoint{0.501000in}{0.566590in}}{\pgfqpoint{3.720000in}{3.020000in}} %
\pgfusepath{clip}%
\pgfsetrectcap%
\pgfsetroundjoin%
\pgfsetlinewidth{1.505625pt}%
\definecolor{currentstroke}{rgb}{0.080392,0.790532,0.897892}%
\pgfsetstrokecolor{currentstroke}%
\pgfsetdash{}{0pt}%
\pgfpathmoveto{\pgfqpoint{0.487111in}{0.964344in}}%
\pgfpathlineto{\pgfqpoint{0.501000in}{0.983131in}}%
\pgfpathlineto{\pgfqpoint{0.515762in}{0.991148in}}%
\pgfpathlineto{\pgfqpoint{0.545286in}{1.045787in}}%
\pgfpathlineto{\pgfqpoint{0.560048in}{1.062268in}}%
\pgfpathlineto{\pgfqpoint{0.574810in}{1.076357in}}%
\pgfpathlineto{\pgfqpoint{0.604333in}{1.120632in}}%
\pgfpathlineto{\pgfqpoint{0.648619in}{1.169857in}}%
\pgfpathlineto{\pgfqpoint{0.663381in}{1.190919in}}%
\pgfpathlineto{\pgfqpoint{0.707667in}{1.239050in}}%
\pgfpathlineto{\pgfqpoint{0.722429in}{1.256579in}}%
\pgfpathlineto{\pgfqpoint{0.751953in}{1.283720in}}%
\pgfpathlineto{\pgfqpoint{0.781476in}{1.317965in}}%
\pgfpathlineto{\pgfqpoint{0.811000in}{1.343594in}}%
\pgfpathlineto{\pgfqpoint{0.840524in}{1.375086in}}%
\pgfpathlineto{\pgfqpoint{0.870048in}{1.398908in}}%
\pgfpathlineto{\pgfqpoint{0.899572in}{1.428454in}}%
\pgfpathlineto{\pgfqpoint{0.914333in}{1.438320in}}%
\pgfpathlineto{\pgfqpoint{0.958619in}{1.478013in}}%
\pgfpathlineto{\pgfqpoint{0.988143in}{1.499694in}}%
\pgfpathlineto{\pgfqpoint{1.002905in}{1.513732in}}%
\pgfpathlineto{\pgfqpoint{1.047191in}{1.545648in}}%
\pgfpathlineto{\pgfqpoint{1.061953in}{1.557969in}}%
\pgfpathlineto{\pgfqpoint{1.106238in}{1.587998in}}%
\pgfpathlineto{\pgfqpoint{1.121000in}{1.599418in}}%
\pgfpathlineto{\pgfqpoint{1.150524in}{1.616834in}}%
\pgfpathlineto{\pgfqpoint{1.180048in}{1.637659in}}%
\pgfpathlineto{\pgfqpoint{1.209572in}{1.653947in}}%
\pgfpathlineto{\pgfqpoint{1.239095in}{1.673428in}}%
\pgfpathlineto{\pgfqpoint{1.253857in}{1.680363in}}%
\pgfpathlineto{\pgfqpoint{1.298143in}{1.706686in}}%
\pgfpathlineto{\pgfqpoint{1.327667in}{1.721458in}}%
\pgfpathlineto{\pgfqpoint{1.342429in}{1.730556in}}%
\pgfpathlineto{\pgfqpoint{1.386714in}{1.751647in}}%
\pgfpathlineto{\pgfqpoint{1.416238in}{1.765744in}}%
\pgfpathlineto{\pgfqpoint{1.431000in}{1.771636in}}%
\pgfpathlineto{\pgfqpoint{1.460524in}{1.786412in}}%
\pgfpathlineto{\pgfqpoint{1.490048in}{1.797562in}}%
\pgfpathlineto{\pgfqpoint{1.519572in}{1.810903in}}%
\pgfpathlineto{\pgfqpoint{1.549095in}{1.821306in}}%
\pgfpathlineto{\pgfqpoint{1.578619in}{1.833181in}}%
\pgfpathlineto{\pgfqpoint{1.608143in}{1.843007in}}%
\pgfpathlineto{\pgfqpoint{1.637667in}{1.853648in}}%
\pgfpathlineto{\pgfqpoint{1.667191in}{1.862598in}}%
\pgfpathlineto{\pgfqpoint{1.696714in}{1.871652in}}%
\pgfpathlineto{\pgfqpoint{1.726238in}{1.880143in}}%
\pgfpathlineto{\pgfqpoint{1.755762in}{1.888206in}}%
\pgfpathlineto{\pgfqpoint{1.785286in}{1.895819in}}%
\pgfpathlineto{\pgfqpoint{1.844333in}{1.909287in}}%
\pgfpathlineto{\pgfqpoint{1.903381in}{1.920977in}}%
\pgfpathlineto{\pgfqpoint{1.977191in}{1.932716in}}%
\pgfpathlineto{\pgfqpoint{2.006714in}{1.936378in}}%
\pgfpathlineto{\pgfqpoint{2.036238in}{1.939832in}}%
\pgfpathlineto{\pgfqpoint{2.154333in}{1.948992in}}%
\pgfpathlineto{\pgfqpoint{2.242905in}{1.951525in}}%
\pgfpathlineto{\pgfqpoint{2.346238in}{1.948723in}}%
\pgfpathlineto{\pgfqpoint{2.390524in}{1.945470in}}%
\pgfpathlineto{\pgfqpoint{2.434810in}{1.941736in}}%
\pgfpathlineto{\pgfqpoint{2.538143in}{1.928816in}}%
\pgfpathlineto{\pgfqpoint{2.685762in}{1.899463in}}%
\pgfpathlineto{\pgfqpoint{2.744810in}{1.884189in}}%
\pgfpathlineto{\pgfqpoint{2.803857in}{1.866764in}}%
\pgfpathlineto{\pgfqpoint{2.877667in}{1.841532in}}%
\pgfpathlineto{\pgfqpoint{2.966238in}{1.806428in}}%
\pgfpathlineto{\pgfqpoint{3.025286in}{1.779864in}}%
\pgfpathlineto{\pgfqpoint{3.084333in}{1.750679in}}%
\pgfpathlineto{\pgfqpoint{3.158143in}{1.710212in}}%
\pgfpathlineto{\pgfqpoint{3.217191in}{1.674871in}}%
\pgfpathlineto{\pgfqpoint{3.305762in}{1.616529in}}%
\pgfpathlineto{\pgfqpoint{3.379572in}{1.562734in}}%
\pgfpathlineto{\pgfqpoint{3.453381in}{1.504184in}}%
\pgfpathlineto{\pgfqpoint{3.527191in}{1.440511in}}%
\pgfpathlineto{\pgfqpoint{3.586238in}{1.385666in}}%
\pgfpathlineto{\pgfqpoint{3.645286in}{1.327178in}}%
\pgfpathlineto{\pgfqpoint{3.704333in}{1.264811in}}%
\pgfpathlineto{\pgfqpoint{3.763381in}{1.197985in}}%
\pgfpathlineto{\pgfqpoint{3.763381in}{1.197985in}}%
\pgfusepath{stroke}%
\end{pgfscope}%
\begin{pgfscope}%
\pgfpathrectangle{\pgfqpoint{0.501000in}{0.566590in}}{\pgfqpoint{3.720000in}{3.020000in}} %
\pgfusepath{clip}%
\pgfsetrectcap%
\pgfsetroundjoin%
\pgfsetlinewidth{1.505625pt}%
\definecolor{currentstroke}{rgb}{0.323529,0.961826,0.798017}%
\pgfsetstrokecolor{currentstroke}%
\pgfsetdash{}{0pt}%
\pgfpathmoveto{\pgfqpoint{0.487111in}{0.992706in}}%
\pgfpathlineto{\pgfqpoint{0.501000in}{1.017999in}}%
\pgfpathlineto{\pgfqpoint{0.515762in}{1.036169in}}%
\pgfpathlineto{\pgfqpoint{0.530524in}{1.060186in}}%
\pgfpathlineto{\pgfqpoint{0.545286in}{1.077204in}}%
\pgfpathlineto{\pgfqpoint{0.560048in}{1.090859in}}%
\pgfpathlineto{\pgfqpoint{0.574810in}{1.109864in}}%
\pgfpathlineto{\pgfqpoint{0.589572in}{1.133990in}}%
\pgfpathlineto{\pgfqpoint{0.604333in}{1.153552in}}%
\pgfpathlineto{\pgfqpoint{0.633857in}{1.180233in}}%
\pgfpathlineto{\pgfqpoint{0.663381in}{1.219343in}}%
\pgfpathlineto{\pgfqpoint{0.678143in}{1.231969in}}%
\pgfpathlineto{\pgfqpoint{0.707667in}{1.253028in}}%
\pgfpathlineto{\pgfqpoint{0.737191in}{1.287429in}}%
\pgfpathlineto{\pgfqpoint{0.751953in}{1.297399in}}%
\pgfpathlineto{\pgfqpoint{0.766714in}{1.304472in}}%
\pgfpathlineto{\pgfqpoint{0.781476in}{1.316777in}}%
\pgfpathlineto{\pgfqpoint{0.796238in}{1.332119in}}%
\pgfpathlineto{\pgfqpoint{0.811000in}{1.343503in}}%
\pgfpathlineto{\pgfqpoint{0.840524in}{1.355887in}}%
\pgfpathlineto{\pgfqpoint{0.855286in}{1.366656in}}%
\pgfpathlineto{\pgfqpoint{0.870048in}{1.379668in}}%
\pgfpathlineto{\pgfqpoint{0.884810in}{1.388684in}}%
\pgfpathlineto{\pgfqpoint{0.899572in}{1.391548in}}%
\pgfpathlineto{\pgfqpoint{0.914333in}{1.398427in}}%
\pgfpathlineto{\pgfqpoint{0.929095in}{1.409215in}}%
\pgfpathlineto{\pgfqpoint{0.943857in}{1.417855in}}%
\pgfpathlineto{\pgfqpoint{0.958619in}{1.422258in}}%
\pgfpathlineto{\pgfqpoint{0.973381in}{1.424307in}}%
\pgfpathlineto{\pgfqpoint{0.988143in}{1.429401in}}%
\pgfpathlineto{\pgfqpoint{1.017667in}{1.444253in}}%
\pgfpathlineto{\pgfqpoint{1.047191in}{1.446493in}}%
\pgfpathlineto{\pgfqpoint{1.076714in}{1.458060in}}%
\pgfpathlineto{\pgfqpoint{1.091476in}{1.460291in}}%
\pgfpathlineto{\pgfqpoint{1.121000in}{1.459490in}}%
\pgfpathlineto{\pgfqpoint{1.150524in}{1.467128in}}%
\pgfpathlineto{\pgfqpoint{1.180048in}{1.463304in}}%
\pgfpathlineto{\pgfqpoint{1.224333in}{1.466421in}}%
\pgfpathlineto{\pgfqpoint{1.253857in}{1.459061in}}%
\pgfpathlineto{\pgfqpoint{1.268619in}{1.458002in}}%
\pgfpathlineto{\pgfqpoint{1.283381in}{1.458701in}}%
\pgfpathlineto{\pgfqpoint{1.298143in}{1.455001in}}%
\pgfpathlineto{\pgfqpoint{1.312905in}{1.448913in}}%
\pgfpathlineto{\pgfqpoint{1.327667in}{1.445356in}}%
\pgfpathlineto{\pgfqpoint{1.342429in}{1.443961in}}%
\pgfpathlineto{\pgfqpoint{1.357191in}{1.440947in}}%
\pgfpathlineto{\pgfqpoint{1.401476in}{1.421635in}}%
\pgfpathlineto{\pgfqpoint{1.416238in}{1.418394in}}%
\pgfpathlineto{\pgfqpoint{1.431000in}{1.412224in}}%
\pgfpathlineto{\pgfqpoint{1.460524in}{1.395183in}}%
\pgfpathlineto{\pgfqpoint{1.490048in}{1.383077in}}%
\pgfpathlineto{\pgfqpoint{1.519572in}{1.363001in}}%
\pgfpathlineto{\pgfqpoint{1.534333in}{1.352953in}}%
\pgfpathlineto{\pgfqpoint{1.549095in}{1.345628in}}%
\pgfpathlineto{\pgfqpoint{1.563857in}{1.336351in}}%
\pgfpathlineto{\pgfqpoint{1.593381in}{1.312045in}}%
\pgfpathlineto{\pgfqpoint{1.637667in}{1.278910in}}%
\pgfpathlineto{\pgfqpoint{1.667191in}{1.249516in}}%
\pgfpathlineto{\pgfqpoint{1.696714in}{1.223906in}}%
\pgfpathlineto{\pgfqpoint{1.785286in}{1.125098in}}%
\pgfpathlineto{\pgfqpoint{1.814810in}{1.087668in}}%
\pgfpathlineto{\pgfqpoint{1.829572in}{1.069469in}}%
\pgfpathlineto{\pgfqpoint{1.888619in}{0.984492in}}%
\pgfpathlineto{\pgfqpoint{1.918143in}{0.936806in}}%
\pgfpathlineto{\pgfqpoint{1.977191in}{0.832593in}}%
\pgfpathlineto{\pgfqpoint{1.991953in}{0.803748in}}%
\pgfpathlineto{\pgfqpoint{2.006714in}{0.784739in}}%
\pgfpathlineto{\pgfqpoint{2.036238in}{0.751317in}}%
\pgfpathlineto{\pgfqpoint{2.051000in}{0.745048in}}%
\pgfpathlineto{\pgfqpoint{2.065762in}{0.749573in}}%
\pgfpathlineto{\pgfqpoint{2.080524in}{0.758701in}}%
\pgfpathlineto{\pgfqpoint{2.095286in}{0.774875in}}%
\pgfpathlineto{\pgfqpoint{2.110048in}{0.797098in}}%
\pgfpathlineto{\pgfqpoint{2.124810in}{0.815800in}}%
\pgfpathlineto{\pgfqpoint{2.139572in}{0.845448in}}%
\pgfpathlineto{\pgfqpoint{2.169095in}{0.889823in}}%
\pgfpathlineto{\pgfqpoint{2.228143in}{0.956866in}}%
\pgfpathlineto{\pgfqpoint{2.257667in}{0.984631in}}%
\pgfpathlineto{\pgfqpoint{2.287191in}{1.011463in}}%
\pgfpathlineto{\pgfqpoint{2.301953in}{1.024877in}}%
\pgfpathlineto{\pgfqpoint{2.361000in}{1.063750in}}%
\pgfpathlineto{\pgfqpoint{2.375762in}{1.071895in}}%
\pgfpathlineto{\pgfqpoint{2.420048in}{1.091987in}}%
\pgfpathlineto{\pgfqpoint{2.434810in}{1.099063in}}%
\pgfpathlineto{\pgfqpoint{2.464333in}{1.106572in}}%
\pgfpathlineto{\pgfqpoint{2.479095in}{1.109338in}}%
\pgfpathlineto{\pgfqpoint{2.508619in}{1.116931in}}%
\pgfpathlineto{\pgfqpoint{2.582429in}{1.119872in}}%
\pgfpathlineto{\pgfqpoint{2.611953in}{1.114382in}}%
\pgfpathlineto{\pgfqpoint{2.641476in}{1.110337in}}%
\pgfpathlineto{\pgfqpoint{2.715286in}{1.084457in}}%
\pgfpathlineto{\pgfqpoint{2.774333in}{1.051216in}}%
\pgfpathlineto{\pgfqpoint{2.818619in}{1.017586in}}%
\pgfpathlineto{\pgfqpoint{2.848143in}{0.993503in}}%
\pgfpathlineto{\pgfqpoint{2.892429in}{0.947319in}}%
\pgfpathlineto{\pgfqpoint{2.921953in}{0.913050in}}%
\pgfpathlineto{\pgfqpoint{2.966238in}{0.851109in}}%
\pgfpathlineto{\pgfqpoint{2.995762in}{0.803736in}}%
\pgfpathlineto{\pgfqpoint{3.010524in}{0.782210in}}%
\pgfpathlineto{\pgfqpoint{3.025286in}{0.764813in}}%
\pgfpathlineto{\pgfqpoint{3.040048in}{0.751111in}}%
\pgfpathlineto{\pgfqpoint{3.054810in}{0.750314in}}%
\pgfpathlineto{\pgfqpoint{3.069572in}{0.758330in}}%
\pgfpathlineto{\pgfqpoint{3.084333in}{0.774908in}}%
\pgfpathlineto{\pgfqpoint{3.143381in}{0.857486in}}%
\pgfpathlineto{\pgfqpoint{3.158143in}{0.880507in}}%
\pgfpathlineto{\pgfqpoint{3.172905in}{0.899243in}}%
\pgfpathlineto{\pgfqpoint{3.202429in}{0.929419in}}%
\pgfpathlineto{\pgfqpoint{3.231953in}{0.954091in}}%
\pgfpathlineto{\pgfqpoint{3.261476in}{0.977379in}}%
\pgfpathlineto{\pgfqpoint{3.320524in}{1.011058in}}%
\pgfpathlineto{\pgfqpoint{3.335286in}{1.016483in}}%
\pgfpathlineto{\pgfqpoint{3.364810in}{1.024848in}}%
\pgfpathlineto{\pgfqpoint{3.379572in}{1.030297in}}%
\pgfpathlineto{\pgfqpoint{3.394333in}{1.033737in}}%
\pgfpathlineto{\pgfqpoint{3.453381in}{1.038750in}}%
\pgfpathlineto{\pgfqpoint{3.482905in}{1.034930in}}%
\pgfpathlineto{\pgfqpoint{3.541953in}{1.023060in}}%
\pgfpathlineto{\pgfqpoint{3.601000in}{0.997864in}}%
\pgfpathlineto{\pgfqpoint{3.645286in}{0.969388in}}%
\pgfpathlineto{\pgfqpoint{3.674810in}{0.947399in}}%
\pgfpathlineto{\pgfqpoint{3.719095in}{0.904803in}}%
\pgfpathlineto{\pgfqpoint{3.733857in}{0.889412in}}%
\pgfpathlineto{\pgfqpoint{3.733857in}{0.889412in}}%
\pgfusepath{stroke}%
\end{pgfscope}%
\begin{pgfscope}%
\pgfpathrectangle{\pgfqpoint{0.501000in}{0.566590in}}{\pgfqpoint{3.720000in}{3.020000in}} %
\pgfusepath{clip}%
\pgfsetrectcap%
\pgfsetroundjoin%
\pgfsetlinewidth{1.505625pt}%
\definecolor{currentstroke}{rgb}{1.000000,0.000000,0.000000}%
\pgfsetstrokecolor{currentstroke}%
\pgfsetdash{}{0pt}%
\pgfpathmoveto{\pgfqpoint{0.487111in}{0.800061in}}%
\pgfpathlineto{\pgfqpoint{0.501000in}{0.803724in}}%
\pgfpathlineto{\pgfqpoint{0.530524in}{0.813540in}}%
\pgfpathlineto{\pgfqpoint{0.560048in}{0.819223in}}%
\pgfpathlineto{\pgfqpoint{0.574810in}{0.823893in}}%
\pgfpathlineto{\pgfqpoint{0.589572in}{0.823453in}}%
\pgfpathlineto{\pgfqpoint{0.619095in}{0.827552in}}%
\pgfpathlineto{\pgfqpoint{0.648619in}{0.828042in}}%
\pgfpathlineto{\pgfqpoint{0.678143in}{0.826767in}}%
\pgfpathlineto{\pgfqpoint{0.707667in}{0.824241in}}%
\pgfpathlineto{\pgfqpoint{0.737191in}{0.819046in}}%
\pgfpathlineto{\pgfqpoint{0.781476in}{0.807804in}}%
\pgfpathlineto{\pgfqpoint{0.811000in}{0.798106in}}%
\pgfpathlineto{\pgfqpoint{0.855286in}{0.778974in}}%
\pgfpathlineto{\pgfqpoint{0.884810in}{0.761574in}}%
\pgfpathlineto{\pgfqpoint{0.899572in}{0.752119in}}%
\pgfpathlineto{\pgfqpoint{0.943857in}{0.712811in}}%
\pgfpathlineto{\pgfqpoint{0.958619in}{0.705759in}}%
\pgfpathlineto{\pgfqpoint{0.973381in}{0.708486in}}%
\pgfpathlineto{\pgfqpoint{1.002905in}{0.730583in}}%
\pgfpathlineto{\pgfqpoint{1.017667in}{0.741278in}}%
\pgfpathlineto{\pgfqpoint{1.032429in}{0.748566in}}%
\pgfpathlineto{\pgfqpoint{1.076714in}{0.763004in}}%
\pgfpathlineto{\pgfqpoint{1.106238in}{0.766303in}}%
\pgfpathlineto{\pgfqpoint{1.135762in}{0.764285in}}%
\pgfpathlineto{\pgfqpoint{1.165286in}{0.758076in}}%
\pgfpathlineto{\pgfqpoint{1.180048in}{0.753404in}}%
\pgfpathlineto{\pgfqpoint{1.209572in}{0.738365in}}%
\pgfpathlineto{\pgfqpoint{1.224333in}{0.728926in}}%
\pgfpathlineto{\pgfqpoint{1.253857in}{0.707436in}}%
\pgfpathlineto{\pgfqpoint{1.268619in}{0.704707in}}%
\pgfpathlineto{\pgfqpoint{1.283381in}{0.709282in}}%
\pgfpathlineto{\pgfqpoint{1.312905in}{0.727872in}}%
\pgfpathlineto{\pgfqpoint{1.327667in}{0.735487in}}%
\pgfpathlineto{\pgfqpoint{1.357191in}{0.743595in}}%
\pgfpathlineto{\pgfqpoint{1.371953in}{0.746996in}}%
\pgfpathlineto{\pgfqpoint{1.416238in}{0.744862in}}%
\pgfpathlineto{\pgfqpoint{1.445762in}{0.736147in}}%
\pgfpathlineto{\pgfqpoint{1.460524in}{0.729364in}}%
\pgfpathlineto{\pgfqpoint{1.490048in}{0.711687in}}%
\pgfpathlineto{\pgfqpoint{1.504810in}{0.705078in}}%
\pgfpathlineto{\pgfqpoint{1.519572in}{0.705329in}}%
\pgfpathlineto{\pgfqpoint{1.534333in}{0.710584in}}%
\pgfpathlineto{\pgfqpoint{1.549095in}{0.719492in}}%
\pgfpathlineto{\pgfqpoint{1.578619in}{0.730660in}}%
\pgfpathlineto{\pgfqpoint{1.593381in}{0.732710in}}%
\pgfpathlineto{\pgfqpoint{1.608143in}{0.732615in}}%
\pgfpathlineto{\pgfqpoint{1.622905in}{0.730844in}}%
\pgfpathlineto{\pgfqpoint{1.637667in}{0.727107in}}%
\pgfpathlineto{\pgfqpoint{1.667191in}{0.713321in}}%
\pgfpathlineto{\pgfqpoint{1.681953in}{0.706278in}}%
\pgfpathlineto{\pgfqpoint{1.696714in}{0.703863in}}%
\pgfpathlineto{\pgfqpoint{1.711476in}{0.707417in}}%
\pgfpathlineto{\pgfqpoint{1.741000in}{0.721095in}}%
\pgfpathlineto{\pgfqpoint{1.755762in}{0.724726in}}%
\pgfpathlineto{\pgfqpoint{1.770524in}{0.724968in}}%
\pgfpathlineto{\pgfqpoint{1.785286in}{0.722347in}}%
\pgfpathlineto{\pgfqpoint{1.829572in}{0.708157in}}%
\pgfpathlineto{\pgfqpoint{1.844333in}{0.707061in}}%
\pgfpathlineto{\pgfqpoint{1.859095in}{0.708903in}}%
\pgfpathlineto{\pgfqpoint{1.888619in}{0.718216in}}%
\pgfpathlineto{\pgfqpoint{1.903381in}{0.720836in}}%
\pgfpathlineto{\pgfqpoint{1.918143in}{0.720176in}}%
\pgfpathlineto{\pgfqpoint{1.962429in}{0.707044in}}%
\pgfpathlineto{\pgfqpoint{1.977191in}{0.706813in}}%
\pgfpathlineto{\pgfqpoint{2.006714in}{0.714311in}}%
\pgfpathlineto{\pgfqpoint{2.021476in}{0.717941in}}%
\pgfpathlineto{\pgfqpoint{2.036238in}{0.719282in}}%
\pgfpathlineto{\pgfqpoint{2.051000in}{0.717597in}}%
\pgfpathlineto{\pgfqpoint{2.095286in}{0.708797in}}%
\pgfpathlineto{\pgfqpoint{2.110048in}{0.709144in}}%
\pgfpathlineto{\pgfqpoint{2.169095in}{0.717378in}}%
\pgfpathlineto{\pgfqpoint{2.183857in}{0.713621in}}%
\pgfpathlineto{\pgfqpoint{2.228143in}{0.710358in}}%
\pgfpathlineto{\pgfqpoint{2.287191in}{0.716744in}}%
\pgfpathlineto{\pgfqpoint{2.316714in}{0.714408in}}%
\pgfpathlineto{\pgfqpoint{2.346238in}{0.711078in}}%
\pgfpathlineto{\pgfqpoint{2.375762in}{0.712004in}}%
\pgfpathlineto{\pgfqpoint{2.420048in}{0.716370in}}%
\pgfpathlineto{\pgfqpoint{2.449572in}{0.714887in}}%
\pgfpathlineto{\pgfqpoint{2.493857in}{0.711824in}}%
\pgfpathlineto{\pgfqpoint{2.567667in}{0.714722in}}%
\pgfpathlineto{\pgfqpoint{2.611953in}{0.711727in}}%
\pgfpathlineto{\pgfqpoint{2.656238in}{0.714342in}}%
\pgfpathlineto{\pgfqpoint{2.685762in}{0.714566in}}%
\pgfpathlineto{\pgfqpoint{2.730048in}{0.711484in}}%
\pgfpathlineto{\pgfqpoint{2.774333in}{0.713862in}}%
\pgfpathlineto{\pgfqpoint{2.803857in}{0.714902in}}%
\pgfpathlineto{\pgfqpoint{2.877667in}{0.712167in}}%
\pgfpathlineto{\pgfqpoint{2.936714in}{0.714558in}}%
\pgfpathlineto{\pgfqpoint{3.010524in}{0.712289in}}%
\pgfpathlineto{\pgfqpoint{3.069572in}{0.713467in}}%
\pgfpathlineto{\pgfqpoint{3.128619in}{0.712337in}}%
\pgfpathlineto{\pgfqpoint{3.187667in}{0.712530in}}%
\pgfpathlineto{\pgfqpoint{3.231953in}{0.711627in}}%
\pgfpathlineto{\pgfqpoint{3.305762in}{0.712417in}}%
\pgfpathlineto{\pgfqpoint{3.364810in}{0.712081in}}%
\pgfpathlineto{\pgfqpoint{3.423857in}{0.712607in}}%
\pgfpathlineto{\pgfqpoint{3.497667in}{0.712559in}}%
\pgfpathlineto{\pgfqpoint{3.541953in}{0.712352in}}%
\pgfpathlineto{\pgfqpoint{3.601000in}{0.712238in}}%
\pgfpathlineto{\pgfqpoint{3.660048in}{0.712759in}}%
\pgfpathlineto{\pgfqpoint{3.733857in}{0.712634in}}%
\pgfpathlineto{\pgfqpoint{3.748619in}{0.712974in}}%
\pgfpathlineto{\pgfqpoint{3.748619in}{0.712974in}}%
\pgfusepath{stroke}%
\end{pgfscope}%
\begin{pgfscope}%
\pgfpathrectangle{\pgfqpoint{0.501000in}{0.566590in}}{\pgfqpoint{3.720000in}{3.020000in}} %
\pgfusepath{clip}%
\pgfsetrectcap%
\pgfsetroundjoin%
\pgfsetlinewidth{1.505625pt}%
\definecolor{currentstroke}{rgb}{0.621569,0.981823,0.636474}%
\pgfsetstrokecolor{currentstroke}%
\pgfsetdash{}{0pt}%
\pgfpathmoveto{\pgfqpoint{0.487111in}{0.914199in}}%
\pgfpathlineto{\pgfqpoint{0.501000in}{0.934123in}}%
\pgfpathlineto{\pgfqpoint{0.515762in}{0.950220in}}%
\pgfpathlineto{\pgfqpoint{0.530524in}{0.961286in}}%
\pgfpathlineto{\pgfqpoint{0.560048in}{0.996134in}}%
\pgfpathlineto{\pgfqpoint{0.589572in}{1.016963in}}%
\pgfpathlineto{\pgfqpoint{0.604333in}{1.032217in}}%
\pgfpathlineto{\pgfqpoint{0.619095in}{1.045070in}}%
\pgfpathlineto{\pgfqpoint{0.648619in}{1.059524in}}%
\pgfpathlineto{\pgfqpoint{0.663381in}{1.071974in}}%
\pgfpathlineto{\pgfqpoint{0.678143in}{1.080393in}}%
\pgfpathlineto{\pgfqpoint{0.692905in}{1.084099in}}%
\pgfpathlineto{\pgfqpoint{0.722429in}{1.101309in}}%
\pgfpathlineto{\pgfqpoint{0.737191in}{1.105192in}}%
\pgfpathlineto{\pgfqpoint{0.751953in}{1.106314in}}%
\pgfpathlineto{\pgfqpoint{0.781476in}{1.118356in}}%
\pgfpathlineto{\pgfqpoint{0.811000in}{1.119126in}}%
\pgfpathlineto{\pgfqpoint{0.825762in}{1.123553in}}%
\pgfpathlineto{\pgfqpoint{0.840524in}{1.125305in}}%
\pgfpathlineto{\pgfqpoint{0.855286in}{1.122018in}}%
\pgfpathlineto{\pgfqpoint{0.870048in}{1.121040in}}%
\pgfpathlineto{\pgfqpoint{0.884810in}{1.122865in}}%
\pgfpathlineto{\pgfqpoint{0.899572in}{1.120118in}}%
\pgfpathlineto{\pgfqpoint{0.914333in}{1.114908in}}%
\pgfpathlineto{\pgfqpoint{0.943857in}{1.111905in}}%
\pgfpathlineto{\pgfqpoint{0.973381in}{1.097233in}}%
\pgfpathlineto{\pgfqpoint{0.988143in}{1.093441in}}%
\pgfpathlineto{\pgfqpoint{1.002905in}{1.088025in}}%
\pgfpathlineto{\pgfqpoint{1.017667in}{1.077559in}}%
\pgfpathlineto{\pgfqpoint{1.032429in}{1.069367in}}%
\pgfpathlineto{\pgfqpoint{1.047191in}{1.063378in}}%
\pgfpathlineto{\pgfqpoint{1.061953in}{1.053321in}}%
\pgfpathlineto{\pgfqpoint{1.076714in}{1.039748in}}%
\pgfpathlineto{\pgfqpoint{1.106238in}{1.019515in}}%
\pgfpathlineto{\pgfqpoint{1.135762in}{0.989636in}}%
\pgfpathlineto{\pgfqpoint{1.165286in}{0.962942in}}%
\pgfpathlineto{\pgfqpoint{1.194810in}{0.925443in}}%
\pgfpathlineto{\pgfqpoint{1.209572in}{0.909058in}}%
\pgfpathlineto{\pgfqpoint{1.224333in}{0.889448in}}%
\pgfpathlineto{\pgfqpoint{1.268619in}{0.820624in}}%
\pgfpathlineto{\pgfqpoint{1.283381in}{0.790239in}}%
\pgfpathlineto{\pgfqpoint{1.298143in}{0.764343in}}%
\pgfpathlineto{\pgfqpoint{1.327667in}{0.723945in}}%
\pgfpathlineto{\pgfqpoint{1.342429in}{0.724082in}}%
\pgfpathlineto{\pgfqpoint{1.357191in}{0.730953in}}%
\pgfpathlineto{\pgfqpoint{1.371953in}{0.753345in}}%
\pgfpathlineto{\pgfqpoint{1.401476in}{0.794568in}}%
\pgfpathlineto{\pgfqpoint{1.431000in}{0.845728in}}%
\pgfpathlineto{\pgfqpoint{1.445762in}{0.865581in}}%
\pgfpathlineto{\pgfqpoint{1.475286in}{0.895622in}}%
\pgfpathlineto{\pgfqpoint{1.504810in}{0.922543in}}%
\pgfpathlineto{\pgfqpoint{1.549095in}{0.951424in}}%
\pgfpathlineto{\pgfqpoint{1.563857in}{0.959091in}}%
\pgfpathlineto{\pgfqpoint{1.608143in}{0.975142in}}%
\pgfpathlineto{\pgfqpoint{1.637667in}{0.979750in}}%
\pgfpathlineto{\pgfqpoint{1.667191in}{0.981142in}}%
\pgfpathlineto{\pgfqpoint{1.681953in}{0.979913in}}%
\pgfpathlineto{\pgfqpoint{1.726238in}{0.970819in}}%
\pgfpathlineto{\pgfqpoint{1.755762in}{0.958935in}}%
\pgfpathlineto{\pgfqpoint{1.785286in}{0.943019in}}%
\pgfpathlineto{\pgfqpoint{1.829572in}{0.910762in}}%
\pgfpathlineto{\pgfqpoint{1.859095in}{0.881835in}}%
\pgfpathlineto{\pgfqpoint{1.888619in}{0.847853in}}%
\pgfpathlineto{\pgfqpoint{1.903381in}{0.827407in}}%
\pgfpathlineto{\pgfqpoint{1.932905in}{0.777341in}}%
\pgfpathlineto{\pgfqpoint{1.947667in}{0.760846in}}%
\pgfpathlineto{\pgfqpoint{1.962429in}{0.748553in}}%
\pgfpathlineto{\pgfqpoint{1.977191in}{0.734088in}}%
\pgfpathlineto{\pgfqpoint{1.991953in}{0.735792in}}%
\pgfpathlineto{\pgfqpoint{2.006714in}{0.743179in}}%
\pgfpathlineto{\pgfqpoint{2.021476in}{0.757666in}}%
\pgfpathlineto{\pgfqpoint{2.036238in}{0.774601in}}%
\pgfpathlineto{\pgfqpoint{2.051000in}{0.798233in}}%
\pgfpathlineto{\pgfqpoint{2.080524in}{0.830893in}}%
\pgfpathlineto{\pgfqpoint{2.095286in}{0.841194in}}%
\pgfpathlineto{\pgfqpoint{2.110048in}{0.856688in}}%
\pgfpathlineto{\pgfqpoint{2.124810in}{0.866363in}}%
\pgfpathlineto{\pgfqpoint{2.169095in}{0.882317in}}%
\pgfpathlineto{\pgfqpoint{2.183857in}{0.886215in}}%
\pgfpathlineto{\pgfqpoint{2.213381in}{0.883872in}}%
\pgfpathlineto{\pgfqpoint{2.228143in}{0.883819in}}%
\pgfpathlineto{\pgfqpoint{2.242905in}{0.881534in}}%
\pgfpathlineto{\pgfqpoint{2.272429in}{0.869678in}}%
\pgfpathlineto{\pgfqpoint{2.287191in}{0.864580in}}%
\pgfpathlineto{\pgfqpoint{2.301953in}{0.854391in}}%
\pgfpathlineto{\pgfqpoint{2.316714in}{0.837702in}}%
\pgfpathlineto{\pgfqpoint{2.331476in}{0.826722in}}%
\pgfpathlineto{\pgfqpoint{2.346238in}{0.813551in}}%
\pgfpathlineto{\pgfqpoint{2.361000in}{0.795621in}}%
\pgfpathlineto{\pgfqpoint{2.375762in}{0.774680in}}%
\pgfpathlineto{\pgfqpoint{2.390524in}{0.762129in}}%
\pgfpathlineto{\pgfqpoint{2.405286in}{0.754374in}}%
\pgfpathlineto{\pgfqpoint{2.420048in}{0.742729in}}%
\pgfpathlineto{\pgfqpoint{2.434810in}{0.743527in}}%
\pgfpathlineto{\pgfqpoint{2.449572in}{0.749403in}}%
\pgfpathlineto{\pgfqpoint{2.493857in}{0.787466in}}%
\pgfpathlineto{\pgfqpoint{2.508619in}{0.804290in}}%
\pgfpathlineto{\pgfqpoint{2.523381in}{0.818033in}}%
\pgfpathlineto{\pgfqpoint{2.552905in}{0.829263in}}%
\pgfpathlineto{\pgfqpoint{2.567667in}{0.838363in}}%
\pgfpathlineto{\pgfqpoint{2.582429in}{0.841739in}}%
\pgfpathlineto{\pgfqpoint{2.597191in}{0.838097in}}%
\pgfpathlineto{\pgfqpoint{2.611953in}{0.837335in}}%
\pgfpathlineto{\pgfqpoint{2.626714in}{0.838128in}}%
\pgfpathlineto{\pgfqpoint{2.641476in}{0.827012in}}%
\pgfpathlineto{\pgfqpoint{2.656238in}{0.819142in}}%
\pgfpathlineto{\pgfqpoint{2.671000in}{0.808035in}}%
\pgfpathlineto{\pgfqpoint{2.715286in}{0.763881in}}%
\pgfpathlineto{\pgfqpoint{2.744810in}{0.744656in}}%
\pgfpathlineto{\pgfqpoint{2.759572in}{0.744153in}}%
\pgfpathlineto{\pgfqpoint{2.774333in}{0.748366in}}%
\pgfpathlineto{\pgfqpoint{2.789095in}{0.758736in}}%
\pgfpathlineto{\pgfqpoint{2.803857in}{0.773705in}}%
\pgfpathlineto{\pgfqpoint{2.818619in}{0.783612in}}%
\pgfpathlineto{\pgfqpoint{2.848143in}{0.815167in}}%
\pgfpathlineto{\pgfqpoint{2.877667in}{0.829521in}}%
\pgfpathlineto{\pgfqpoint{2.907191in}{0.840587in}}%
\pgfpathlineto{\pgfqpoint{2.936714in}{0.839870in}}%
\pgfpathlineto{\pgfqpoint{2.951476in}{0.838734in}}%
\pgfpathlineto{\pgfqpoint{2.981000in}{0.829069in}}%
\pgfpathlineto{\pgfqpoint{3.010524in}{0.813523in}}%
\pgfpathlineto{\pgfqpoint{3.025286in}{0.795396in}}%
\pgfpathlineto{\pgfqpoint{3.084333in}{0.750204in}}%
\pgfpathlineto{\pgfqpoint{3.099095in}{0.747179in}}%
\pgfpathlineto{\pgfqpoint{3.113857in}{0.749263in}}%
\pgfpathlineto{\pgfqpoint{3.128619in}{0.756831in}}%
\pgfpathlineto{\pgfqpoint{3.158143in}{0.777634in}}%
\pgfpathlineto{\pgfqpoint{3.172905in}{0.790213in}}%
\pgfpathlineto{\pgfqpoint{3.217191in}{0.815558in}}%
\pgfpathlineto{\pgfqpoint{3.231953in}{0.822350in}}%
\pgfpathlineto{\pgfqpoint{3.261476in}{0.822541in}}%
\pgfpathlineto{\pgfqpoint{3.276238in}{0.821650in}}%
\pgfpathlineto{\pgfqpoint{3.291000in}{0.817666in}}%
\pgfpathlineto{\pgfqpoint{3.320524in}{0.798533in}}%
\pgfpathlineto{\pgfqpoint{3.350048in}{0.779205in}}%
\pgfpathlineto{\pgfqpoint{3.364810in}{0.767461in}}%
\pgfpathlineto{\pgfqpoint{3.394333in}{0.753356in}}%
\pgfpathlineto{\pgfqpoint{3.409095in}{0.751002in}}%
\pgfpathlineto{\pgfqpoint{3.423857in}{0.754328in}}%
\pgfpathlineto{\pgfqpoint{3.438619in}{0.762476in}}%
\pgfpathlineto{\pgfqpoint{3.482905in}{0.791435in}}%
\pgfpathlineto{\pgfqpoint{3.497667in}{0.800538in}}%
\pgfpathlineto{\pgfqpoint{3.512429in}{0.806803in}}%
\pgfpathlineto{\pgfqpoint{3.527191in}{0.818554in}}%
\pgfpathlineto{\pgfqpoint{3.541953in}{0.822553in}}%
\pgfpathlineto{\pgfqpoint{3.571476in}{0.824984in}}%
\pgfpathlineto{\pgfqpoint{3.601000in}{0.820773in}}%
\pgfpathlineto{\pgfqpoint{3.615762in}{0.817105in}}%
\pgfpathlineto{\pgfqpoint{3.630524in}{0.805668in}}%
\pgfpathlineto{\pgfqpoint{3.645286in}{0.798823in}}%
\pgfpathlineto{\pgfqpoint{3.674810in}{0.780285in}}%
\pgfpathlineto{\pgfqpoint{3.704333in}{0.762670in}}%
\pgfpathlineto{\pgfqpoint{3.719095in}{0.758724in}}%
\pgfpathlineto{\pgfqpoint{3.733857in}{0.756587in}}%
\pgfpathlineto{\pgfqpoint{3.748619in}{0.759980in}}%
\pgfpathlineto{\pgfqpoint{3.763381in}{0.766328in}}%
\pgfpathlineto{\pgfqpoint{3.763381in}{0.766328in}}%
\pgfusepath{stroke}%
\end{pgfscope}%
\begin{pgfscope}%
\pgfpathrectangle{\pgfqpoint{0.501000in}{0.566590in}}{\pgfqpoint{3.720000in}{3.020000in}} %
\pgfusepath{clip}%
\pgfsetrectcap%
\pgfsetroundjoin%
\pgfsetlinewidth{1.505625pt}%
\definecolor{currentstroke}{rgb}{0.982353,0.726434,0.395451}%
\pgfsetstrokecolor{currentstroke}%
\pgfsetdash{}{0pt}%
\pgfpathmoveto{\pgfqpoint{0.487111in}{0.878216in}}%
\pgfpathlineto{\pgfqpoint{0.515762in}{0.901328in}}%
\pgfpathlineto{\pgfqpoint{0.530524in}{0.906839in}}%
\pgfpathlineto{\pgfqpoint{0.545286in}{0.916274in}}%
\pgfpathlineto{\pgfqpoint{0.560048in}{0.922820in}}%
\pgfpathlineto{\pgfqpoint{0.574810in}{0.925842in}}%
\pgfpathlineto{\pgfqpoint{0.589572in}{0.932859in}}%
\pgfpathlineto{\pgfqpoint{0.604333in}{0.935130in}}%
\pgfpathlineto{\pgfqpoint{0.619095in}{0.934597in}}%
\pgfpathlineto{\pgfqpoint{0.633857in}{0.939641in}}%
\pgfpathlineto{\pgfqpoint{0.648619in}{0.938823in}}%
\pgfpathlineto{\pgfqpoint{0.663381in}{0.935896in}}%
\pgfpathlineto{\pgfqpoint{0.678143in}{0.937326in}}%
\pgfpathlineto{\pgfqpoint{0.707667in}{0.928179in}}%
\pgfpathlineto{\pgfqpoint{0.722429in}{0.925787in}}%
\pgfpathlineto{\pgfqpoint{0.766714in}{0.905013in}}%
\pgfpathlineto{\pgfqpoint{0.796238in}{0.884177in}}%
\pgfpathlineto{\pgfqpoint{0.811000in}{0.875039in}}%
\pgfpathlineto{\pgfqpoint{0.855286in}{0.833894in}}%
\pgfpathlineto{\pgfqpoint{0.899572in}{0.777702in}}%
\pgfpathlineto{\pgfqpoint{0.914333in}{0.755805in}}%
\pgfpathlineto{\pgfqpoint{0.929095in}{0.740769in}}%
\pgfpathlineto{\pgfqpoint{0.943857in}{0.722907in}}%
\pgfpathlineto{\pgfqpoint{0.958619in}{0.723059in}}%
\pgfpathlineto{\pgfqpoint{0.973381in}{0.729642in}}%
\pgfpathlineto{\pgfqpoint{0.988143in}{0.746215in}}%
\pgfpathlineto{\pgfqpoint{1.002905in}{0.765563in}}%
\pgfpathlineto{\pgfqpoint{1.017667in}{0.782313in}}%
\pgfpathlineto{\pgfqpoint{1.061953in}{0.818137in}}%
\pgfpathlineto{\pgfqpoint{1.076714in}{0.820530in}}%
\pgfpathlineto{\pgfqpoint{1.091476in}{0.831163in}}%
\pgfpathlineto{\pgfqpoint{1.106238in}{0.836108in}}%
\pgfpathlineto{\pgfqpoint{1.121000in}{0.834428in}}%
\pgfpathlineto{\pgfqpoint{1.135762in}{0.839080in}}%
\pgfpathlineto{\pgfqpoint{1.150524in}{0.841092in}}%
\pgfpathlineto{\pgfqpoint{1.165286in}{0.835601in}}%
\pgfpathlineto{\pgfqpoint{1.180048in}{0.836901in}}%
\pgfpathlineto{\pgfqpoint{1.194810in}{0.831763in}}%
\pgfpathlineto{\pgfqpoint{1.209572in}{0.823759in}}%
\pgfpathlineto{\pgfqpoint{1.224333in}{0.820443in}}%
\pgfpathlineto{\pgfqpoint{1.239095in}{0.810414in}}%
\pgfpathlineto{\pgfqpoint{1.253857in}{0.797493in}}%
\pgfpathlineto{\pgfqpoint{1.268619in}{0.788095in}}%
\pgfpathlineto{\pgfqpoint{1.312905in}{0.741689in}}%
\pgfpathlineto{\pgfqpoint{1.327667in}{0.729326in}}%
\pgfpathlineto{\pgfqpoint{1.342429in}{0.728399in}}%
\pgfpathlineto{\pgfqpoint{1.357191in}{0.732450in}}%
\pgfpathlineto{\pgfqpoint{1.386714in}{0.758013in}}%
\pgfpathlineto{\pgfqpoint{1.401476in}{0.773697in}}%
\pgfpathlineto{\pgfqpoint{1.416238in}{0.782803in}}%
\pgfpathlineto{\pgfqpoint{1.431000in}{0.794386in}}%
\pgfpathlineto{\pgfqpoint{1.445762in}{0.798669in}}%
\pgfpathlineto{\pgfqpoint{1.460524in}{0.798980in}}%
\pgfpathlineto{\pgfqpoint{1.475286in}{0.801807in}}%
\pgfpathlineto{\pgfqpoint{1.490048in}{0.799592in}}%
\pgfpathlineto{\pgfqpoint{1.519572in}{0.785545in}}%
\pgfpathlineto{\pgfqpoint{1.534333in}{0.774702in}}%
\pgfpathlineto{\pgfqpoint{1.549095in}{0.758738in}}%
\pgfpathlineto{\pgfqpoint{1.563857in}{0.749019in}}%
\pgfpathlineto{\pgfqpoint{1.578619in}{0.735811in}}%
\pgfpathlineto{\pgfqpoint{1.593381in}{0.731061in}}%
\pgfpathlineto{\pgfqpoint{1.608143in}{0.732893in}}%
\pgfpathlineto{\pgfqpoint{1.637667in}{0.752511in}}%
\pgfpathlineto{\pgfqpoint{1.652429in}{0.768229in}}%
\pgfpathlineto{\pgfqpoint{1.667191in}{0.778619in}}%
\pgfpathlineto{\pgfqpoint{1.696714in}{0.790177in}}%
\pgfpathlineto{\pgfqpoint{1.711476in}{0.791467in}}%
\pgfpathlineto{\pgfqpoint{1.741000in}{0.785647in}}%
\pgfpathlineto{\pgfqpoint{1.755762in}{0.779528in}}%
\pgfpathlineto{\pgfqpoint{1.770524in}{0.768542in}}%
\pgfpathlineto{\pgfqpoint{1.785286in}{0.760726in}}%
\pgfpathlineto{\pgfqpoint{1.800048in}{0.746996in}}%
\pgfpathlineto{\pgfqpoint{1.814810in}{0.740169in}}%
\pgfpathlineto{\pgfqpoint{1.829572in}{0.736057in}}%
\pgfpathlineto{\pgfqpoint{1.844333in}{0.739603in}}%
\pgfpathlineto{\pgfqpoint{1.859095in}{0.745862in}}%
\pgfpathlineto{\pgfqpoint{1.873857in}{0.758174in}}%
\pgfpathlineto{\pgfqpoint{1.903381in}{0.775764in}}%
\pgfpathlineto{\pgfqpoint{1.918143in}{0.781488in}}%
\pgfpathlineto{\pgfqpoint{1.932905in}{0.784017in}}%
\pgfpathlineto{\pgfqpoint{1.947667in}{0.782512in}}%
\pgfpathlineto{\pgfqpoint{1.962429in}{0.776859in}}%
\pgfpathlineto{\pgfqpoint{1.991953in}{0.758327in}}%
\pgfpathlineto{\pgfqpoint{2.006714in}{0.748162in}}%
\pgfpathlineto{\pgfqpoint{2.021476in}{0.740444in}}%
\pgfpathlineto{\pgfqpoint{2.036238in}{0.737955in}}%
\pgfpathlineto{\pgfqpoint{2.051000in}{0.740927in}}%
\pgfpathlineto{\pgfqpoint{2.110048in}{0.777457in}}%
\pgfpathlineto{\pgfqpoint{2.124810in}{0.782093in}}%
\pgfpathlineto{\pgfqpoint{2.139572in}{0.783904in}}%
\pgfpathlineto{\pgfqpoint{2.154333in}{0.781199in}}%
\pgfpathlineto{\pgfqpoint{2.169095in}{0.775340in}}%
\pgfpathlineto{\pgfqpoint{2.228143in}{0.742202in}}%
\pgfpathlineto{\pgfqpoint{2.242905in}{0.742182in}}%
\pgfpathlineto{\pgfqpoint{2.257667in}{0.747704in}}%
\pgfpathlineto{\pgfqpoint{2.301953in}{0.775485in}}%
\pgfpathlineto{\pgfqpoint{2.316714in}{0.781545in}}%
\pgfpathlineto{\pgfqpoint{2.331476in}{0.784297in}}%
\pgfpathlineto{\pgfqpoint{2.346238in}{0.783233in}}%
\pgfpathlineto{\pgfqpoint{2.361000in}{0.778546in}}%
\pgfpathlineto{\pgfqpoint{2.390524in}{0.762792in}}%
\pgfpathlineto{\pgfqpoint{2.420048in}{0.747305in}}%
\pgfpathlineto{\pgfqpoint{2.434810in}{0.746173in}}%
\pgfpathlineto{\pgfqpoint{2.449572in}{0.749239in}}%
\pgfpathlineto{\pgfqpoint{2.464333in}{0.756502in}}%
\pgfpathlineto{\pgfqpoint{2.493857in}{0.774199in}}%
\pgfpathlineto{\pgfqpoint{2.508619in}{0.781363in}}%
\pgfpathlineto{\pgfqpoint{2.523381in}{0.784729in}}%
\pgfpathlineto{\pgfqpoint{2.538143in}{0.785839in}}%
\pgfpathlineto{\pgfqpoint{2.552905in}{0.783171in}}%
\pgfpathlineto{\pgfqpoint{2.582429in}{0.770210in}}%
\pgfpathlineto{\pgfqpoint{2.611953in}{0.754544in}}%
\pgfpathlineto{\pgfqpoint{2.626714in}{0.749984in}}%
\pgfpathlineto{\pgfqpoint{2.641476in}{0.749784in}}%
\pgfpathlineto{\pgfqpoint{2.656238in}{0.754044in}}%
\pgfpathlineto{\pgfqpoint{2.700524in}{0.777326in}}%
\pgfpathlineto{\pgfqpoint{2.715286in}{0.783211in}}%
\pgfpathlineto{\pgfqpoint{2.730048in}{0.785746in}}%
\pgfpathlineto{\pgfqpoint{2.744810in}{0.785342in}}%
\pgfpathlineto{\pgfqpoint{2.759572in}{0.781906in}}%
\pgfpathlineto{\pgfqpoint{2.818619in}{0.754936in}}%
\pgfpathlineto{\pgfqpoint{2.833381in}{0.753060in}}%
\pgfpathlineto{\pgfqpoint{2.848143in}{0.754342in}}%
\pgfpathlineto{\pgfqpoint{2.877667in}{0.767891in}}%
\pgfpathlineto{\pgfqpoint{2.921953in}{0.786677in}}%
\pgfpathlineto{\pgfqpoint{2.936714in}{0.787218in}}%
\pgfpathlineto{\pgfqpoint{2.951476in}{0.786045in}}%
\pgfpathlineto{\pgfqpoint{2.966238in}{0.781405in}}%
\pgfpathlineto{\pgfqpoint{3.025286in}{0.756622in}}%
\pgfpathlineto{\pgfqpoint{3.040048in}{0.755457in}}%
\pgfpathlineto{\pgfqpoint{3.054810in}{0.757647in}}%
\pgfpathlineto{\pgfqpoint{3.084333in}{0.768612in}}%
\pgfpathlineto{\pgfqpoint{3.113857in}{0.780033in}}%
\pgfpathlineto{\pgfqpoint{3.128619in}{0.783640in}}%
\pgfpathlineto{\pgfqpoint{3.143381in}{0.784461in}}%
\pgfpathlineto{\pgfqpoint{3.172905in}{0.777679in}}%
\pgfpathlineto{\pgfqpoint{3.217191in}{0.761148in}}%
\pgfpathlineto{\pgfqpoint{3.231953in}{0.758195in}}%
\pgfpathlineto{\pgfqpoint{3.246714in}{0.757946in}}%
\pgfpathlineto{\pgfqpoint{3.276238in}{0.764718in}}%
\pgfpathlineto{\pgfqpoint{3.305762in}{0.775498in}}%
\pgfpathlineto{\pgfqpoint{3.320524in}{0.778805in}}%
\pgfpathlineto{\pgfqpoint{3.335286in}{0.780414in}}%
\pgfpathlineto{\pgfqpoint{3.350048in}{0.779933in}}%
\pgfpathlineto{\pgfqpoint{3.364810in}{0.776970in}}%
\pgfpathlineto{\pgfqpoint{3.409095in}{0.762863in}}%
\pgfpathlineto{\pgfqpoint{3.423857in}{0.760046in}}%
\pgfpathlineto{\pgfqpoint{3.438619in}{0.759337in}}%
\pgfpathlineto{\pgfqpoint{3.453381in}{0.760727in}}%
\pgfpathlineto{\pgfqpoint{3.527191in}{0.777859in}}%
\pgfpathlineto{\pgfqpoint{3.556714in}{0.774983in}}%
\pgfpathlineto{\pgfqpoint{3.615762in}{0.761264in}}%
\pgfpathlineto{\pgfqpoint{3.630524in}{0.760519in}}%
\pgfpathlineto{\pgfqpoint{3.660048in}{0.764770in}}%
\pgfpathlineto{\pgfqpoint{3.704333in}{0.774480in}}%
\pgfpathlineto{\pgfqpoint{3.733857in}{0.774946in}}%
\pgfpathlineto{\pgfqpoint{3.778143in}{0.766669in}}%
\pgfpathlineto{\pgfqpoint{3.778143in}{0.766669in}}%
\pgfusepath{stroke}%
\end{pgfscope}%
\begin{pgfscope}%
\pgfsetrectcap%
\pgfsetmiterjoin%
\pgfsetlinewidth{0.803000pt}%
\definecolor{currentstroke}{rgb}{0.000000,0.000000,0.000000}%
\pgfsetstrokecolor{currentstroke}%
\pgfsetdash{}{0pt}%
\pgfpathmoveto{\pgfqpoint{0.501000in}{0.566590in}}%
\pgfpathlineto{\pgfqpoint{0.501000in}{3.586590in}}%
\pgfusepath{stroke}%
\end{pgfscope}%
\begin{pgfscope}%
\pgfsetrectcap%
\pgfsetmiterjoin%
\pgfsetlinewidth{0.803000pt}%
\definecolor{currentstroke}{rgb}{0.000000,0.000000,0.000000}%
\pgfsetstrokecolor{currentstroke}%
\pgfsetdash{}{0pt}%
\pgfpathmoveto{\pgfqpoint{4.221000in}{0.566590in}}%
\pgfpathlineto{\pgfqpoint{4.221000in}{3.586590in}}%
\pgfusepath{stroke}%
\end{pgfscope}%
\begin{pgfscope}%
\pgfsetrectcap%
\pgfsetmiterjoin%
\pgfsetlinewidth{0.803000pt}%
\definecolor{currentstroke}{rgb}{0.000000,0.000000,0.000000}%
\pgfsetstrokecolor{currentstroke}%
\pgfsetdash{}{0pt}%
\pgfpathmoveto{\pgfqpoint{0.501000in}{0.566590in}}%
\pgfpathlineto{\pgfqpoint{4.221000in}{0.566590in}}%
\pgfusepath{stroke}%
\end{pgfscope}%
\begin{pgfscope}%
\pgfsetrectcap%
\pgfsetmiterjoin%
\pgfsetlinewidth{0.803000pt}%
\definecolor{currentstroke}{rgb}{0.000000,0.000000,0.000000}%
\pgfsetstrokecolor{currentstroke}%
\pgfsetdash{}{0pt}%
\pgfpathmoveto{\pgfqpoint{0.501000in}{3.586590in}}%
\pgfpathlineto{\pgfqpoint{4.221000in}{3.586590in}}%
\pgfusepath{stroke}%
\end{pgfscope}%
\begin{pgfscope}%
\pgfpathrectangle{\pgfqpoint{4.453500in}{0.566590in}}{\pgfqpoint{0.151000in}{3.020000in}} %
\pgfusepath{clip}%
\pgfsetbuttcap%
\pgfsetmiterjoin%
\definecolor{currentfill}{rgb}{1.000000,1.000000,1.000000}%
\pgfsetfillcolor{currentfill}%
\pgfsetlinewidth{0.010037pt}%
\definecolor{currentstroke}{rgb}{1.000000,1.000000,1.000000}%
\pgfsetstrokecolor{currentstroke}%
\pgfsetdash{}{0pt}%
\pgfpathmoveto{\pgfqpoint{4.453500in}{0.566590in}}%
\pgfpathlineto{\pgfqpoint{4.453500in}{0.578387in}}%
\pgfpathlineto{\pgfqpoint{4.453500in}{3.574793in}}%
\pgfpathlineto{\pgfqpoint{4.453500in}{3.586590in}}%
\pgfpathlineto{\pgfqpoint{4.604500in}{3.586590in}}%
\pgfpathlineto{\pgfqpoint{4.604500in}{3.574793in}}%
\pgfpathlineto{\pgfqpoint{4.604500in}{0.578387in}}%
\pgfpathlineto{\pgfqpoint{4.604500in}{0.566590in}}%
\pgfpathclose%
\pgfusepath{stroke,fill}%
\end{pgfscope}%
\begin{pgfscope}%
\pgfsys@transformshift{4.458333in}{0.582423in}%
\pgftext[left,bottom]{\pgfimage[interpolate=true,width=0.152778in,height=3.013889in]{series-img0.png}}%
\end{pgfscope}%
\begin{pgfscope}%
\pgfsetbuttcap%
\pgfsetroundjoin%
\definecolor{currentfill}{rgb}{0.000000,0.000000,0.000000}%
\pgfsetfillcolor{currentfill}%
\pgfsetlinewidth{0.803000pt}%
\definecolor{currentstroke}{rgb}{0.000000,0.000000,0.000000}%
\pgfsetstrokecolor{currentstroke}%
\pgfsetdash{}{0pt}%
\pgfsys@defobject{currentmarker}{\pgfqpoint{0.000000in}{0.000000in}}{\pgfqpoint{0.048611in}{0.000000in}}{%
\pgfpathmoveto{\pgfqpoint{0.000000in}{0.000000in}}%
\pgfpathlineto{\pgfqpoint{0.048611in}{0.000000in}}%
\pgfusepath{stroke,fill}%
}%
\begin{pgfscope}%
\pgfsys@transformshift{4.604500in}{0.601555in}%
\pgfsys@useobject{currentmarker}{}%
\end{pgfscope}%
\end{pgfscope}%
\begin{pgfscope}%
\pgfsetbuttcap%
\pgfsetroundjoin%
\definecolor{currentfill}{rgb}{0.000000,0.000000,0.000000}%
\pgfsetfillcolor{currentfill}%
\pgfsetlinewidth{0.803000pt}%
\definecolor{currentstroke}{rgb}{0.000000,0.000000,0.000000}%
\pgfsetstrokecolor{currentstroke}%
\pgfsetdash{}{0pt}%
\pgfsys@defobject{currentmarker}{\pgfqpoint{0.000000in}{0.000000in}}{\pgfqpoint{0.048611in}{0.000000in}}{%
\pgfpathmoveto{\pgfqpoint{0.000000in}{0.000000in}}%
\pgfpathlineto{\pgfqpoint{0.048611in}{0.000000in}}%
\pgfusepath{stroke,fill}%
}%
\begin{pgfscope}%
\pgfsys@transformshift{4.604500in}{0.683069in}%
\pgfsys@useobject{currentmarker}{}%
\end{pgfscope}%
\end{pgfscope}%
\begin{pgfscope}%
\pgftext[x=4.701722in,y=0.625676in,left,base]{\rmfamily\fontsize{12.000000}{14.400000}\selectfont \(\displaystyle 10^{-1}\)}%
\end{pgfscope}%
\begin{pgfscope}%
\pgfsetbuttcap%
\pgfsetroundjoin%
\definecolor{currentfill}{rgb}{0.000000,0.000000,0.000000}%
\pgfsetfillcolor{currentfill}%
\pgfsetlinewidth{0.803000pt}%
\definecolor{currentstroke}{rgb}{0.000000,0.000000,0.000000}%
\pgfsetstrokecolor{currentstroke}%
\pgfsetdash{}{0pt}%
\pgfsys@defobject{currentmarker}{\pgfqpoint{0.000000in}{0.000000in}}{\pgfqpoint{0.048611in}{0.000000in}}{%
\pgfpathmoveto{\pgfqpoint{0.000000in}{0.000000in}}%
\pgfpathlineto{\pgfqpoint{0.048611in}{0.000000in}}%
\pgfusepath{stroke,fill}%
}%
\begin{pgfscope}%
\pgfsys@transformshift{4.604500in}{1.219335in}%
\pgfsys@useobject{currentmarker}{}%
\end{pgfscope}%
\end{pgfscope}%
\begin{pgfscope}%
\pgfsetbuttcap%
\pgfsetroundjoin%
\definecolor{currentfill}{rgb}{0.000000,0.000000,0.000000}%
\pgfsetfillcolor{currentfill}%
\pgfsetlinewidth{0.803000pt}%
\definecolor{currentstroke}{rgb}{0.000000,0.000000,0.000000}%
\pgfsetstrokecolor{currentstroke}%
\pgfsetdash{}{0pt}%
\pgfsys@defobject{currentmarker}{\pgfqpoint{0.000000in}{0.000000in}}{\pgfqpoint{0.048611in}{0.000000in}}{%
\pgfpathmoveto{\pgfqpoint{0.000000in}{0.000000in}}%
\pgfpathlineto{\pgfqpoint{0.048611in}{0.000000in}}%
\pgfusepath{stroke,fill}%
}%
\begin{pgfscope}%
\pgfsys@transformshift{4.604500in}{1.533030in}%
\pgfsys@useobject{currentmarker}{}%
\end{pgfscope}%
\end{pgfscope}%
\begin{pgfscope}%
\pgfsetbuttcap%
\pgfsetroundjoin%
\definecolor{currentfill}{rgb}{0.000000,0.000000,0.000000}%
\pgfsetfillcolor{currentfill}%
\pgfsetlinewidth{0.803000pt}%
\definecolor{currentstroke}{rgb}{0.000000,0.000000,0.000000}%
\pgfsetstrokecolor{currentstroke}%
\pgfsetdash{}{0pt}%
\pgfsys@defobject{currentmarker}{\pgfqpoint{0.000000in}{0.000000in}}{\pgfqpoint{0.048611in}{0.000000in}}{%
\pgfpathmoveto{\pgfqpoint{0.000000in}{0.000000in}}%
\pgfpathlineto{\pgfqpoint{0.048611in}{0.000000in}}%
\pgfusepath{stroke,fill}%
}%
\begin{pgfscope}%
\pgfsys@transformshift{4.604500in}{1.755600in}%
\pgfsys@useobject{currentmarker}{}%
\end{pgfscope}%
\end{pgfscope}%
\begin{pgfscope}%
\pgfsetbuttcap%
\pgfsetroundjoin%
\definecolor{currentfill}{rgb}{0.000000,0.000000,0.000000}%
\pgfsetfillcolor{currentfill}%
\pgfsetlinewidth{0.803000pt}%
\definecolor{currentstroke}{rgb}{0.000000,0.000000,0.000000}%
\pgfsetstrokecolor{currentstroke}%
\pgfsetdash{}{0pt}%
\pgfsys@defobject{currentmarker}{\pgfqpoint{0.000000in}{0.000000in}}{\pgfqpoint{0.048611in}{0.000000in}}{%
\pgfpathmoveto{\pgfqpoint{0.000000in}{0.000000in}}%
\pgfpathlineto{\pgfqpoint{0.048611in}{0.000000in}}%
\pgfusepath{stroke,fill}%
}%
\begin{pgfscope}%
\pgfsys@transformshift{4.604500in}{1.928239in}%
\pgfsys@useobject{currentmarker}{}%
\end{pgfscope}%
\end{pgfscope}%
\begin{pgfscope}%
\pgfsetbuttcap%
\pgfsetroundjoin%
\definecolor{currentfill}{rgb}{0.000000,0.000000,0.000000}%
\pgfsetfillcolor{currentfill}%
\pgfsetlinewidth{0.803000pt}%
\definecolor{currentstroke}{rgb}{0.000000,0.000000,0.000000}%
\pgfsetstrokecolor{currentstroke}%
\pgfsetdash{}{0pt}%
\pgfsys@defobject{currentmarker}{\pgfqpoint{0.000000in}{0.000000in}}{\pgfqpoint{0.048611in}{0.000000in}}{%
\pgfpathmoveto{\pgfqpoint{0.000000in}{0.000000in}}%
\pgfpathlineto{\pgfqpoint{0.048611in}{0.000000in}}%
\pgfusepath{stroke,fill}%
}%
\begin{pgfscope}%
\pgfsys@transformshift{4.604500in}{2.069295in}%
\pgfsys@useobject{currentmarker}{}%
\end{pgfscope}%
\end{pgfscope}%
\begin{pgfscope}%
\pgfsetbuttcap%
\pgfsetroundjoin%
\definecolor{currentfill}{rgb}{0.000000,0.000000,0.000000}%
\pgfsetfillcolor{currentfill}%
\pgfsetlinewidth{0.803000pt}%
\definecolor{currentstroke}{rgb}{0.000000,0.000000,0.000000}%
\pgfsetstrokecolor{currentstroke}%
\pgfsetdash{}{0pt}%
\pgfsys@defobject{currentmarker}{\pgfqpoint{0.000000in}{0.000000in}}{\pgfqpoint{0.048611in}{0.000000in}}{%
\pgfpathmoveto{\pgfqpoint{0.000000in}{0.000000in}}%
\pgfpathlineto{\pgfqpoint{0.048611in}{0.000000in}}%
\pgfusepath{stroke,fill}%
}%
\begin{pgfscope}%
\pgfsys@transformshift{4.604500in}{2.188556in}%
\pgfsys@useobject{currentmarker}{}%
\end{pgfscope}%
\end{pgfscope}%
\begin{pgfscope}%
\pgfsetbuttcap%
\pgfsetroundjoin%
\definecolor{currentfill}{rgb}{0.000000,0.000000,0.000000}%
\pgfsetfillcolor{currentfill}%
\pgfsetlinewidth{0.803000pt}%
\definecolor{currentstroke}{rgb}{0.000000,0.000000,0.000000}%
\pgfsetstrokecolor{currentstroke}%
\pgfsetdash{}{0pt}%
\pgfsys@defobject{currentmarker}{\pgfqpoint{0.000000in}{0.000000in}}{\pgfqpoint{0.048611in}{0.000000in}}{%
\pgfpathmoveto{\pgfqpoint{0.000000in}{0.000000in}}%
\pgfpathlineto{\pgfqpoint{0.048611in}{0.000000in}}%
\pgfusepath{stroke,fill}%
}%
\begin{pgfscope}%
\pgfsys@transformshift{4.604500in}{2.291865in}%
\pgfsys@useobject{currentmarker}{}%
\end{pgfscope}%
\end{pgfscope}%
\begin{pgfscope}%
\pgfsetbuttcap%
\pgfsetroundjoin%
\definecolor{currentfill}{rgb}{0.000000,0.000000,0.000000}%
\pgfsetfillcolor{currentfill}%
\pgfsetlinewidth{0.803000pt}%
\definecolor{currentstroke}{rgb}{0.000000,0.000000,0.000000}%
\pgfsetstrokecolor{currentstroke}%
\pgfsetdash{}{0pt}%
\pgfsys@defobject{currentmarker}{\pgfqpoint{0.000000in}{0.000000in}}{\pgfqpoint{0.048611in}{0.000000in}}{%
\pgfpathmoveto{\pgfqpoint{0.000000in}{0.000000in}}%
\pgfpathlineto{\pgfqpoint{0.048611in}{0.000000in}}%
\pgfusepath{stroke,fill}%
}%
\begin{pgfscope}%
\pgfsys@transformshift{4.604500in}{2.382990in}%
\pgfsys@useobject{currentmarker}{}%
\end{pgfscope}%
\end{pgfscope}%
\begin{pgfscope}%
\pgfsetbuttcap%
\pgfsetroundjoin%
\definecolor{currentfill}{rgb}{0.000000,0.000000,0.000000}%
\pgfsetfillcolor{currentfill}%
\pgfsetlinewidth{0.803000pt}%
\definecolor{currentstroke}{rgb}{0.000000,0.000000,0.000000}%
\pgfsetstrokecolor{currentstroke}%
\pgfsetdash{}{0pt}%
\pgfsys@defobject{currentmarker}{\pgfqpoint{0.000000in}{0.000000in}}{\pgfqpoint{0.048611in}{0.000000in}}{%
\pgfpathmoveto{\pgfqpoint{0.000000in}{0.000000in}}%
\pgfpathlineto{\pgfqpoint{0.048611in}{0.000000in}}%
\pgfusepath{stroke,fill}%
}%
\begin{pgfscope}%
\pgfsys@transformshift{4.604500in}{2.464504in}%
\pgfsys@useobject{currentmarker}{}%
\end{pgfscope}%
\end{pgfscope}%
\begin{pgfscope}%
\pgftext[x=4.701722in,y=2.407111in,left,base]{\rmfamily\fontsize{12.000000}{14.400000}\selectfont \(\displaystyle 10^{0}\)}%
\end{pgfscope}%
\begin{pgfscope}%
\pgfsetbuttcap%
\pgfsetroundjoin%
\definecolor{currentfill}{rgb}{0.000000,0.000000,0.000000}%
\pgfsetfillcolor{currentfill}%
\pgfsetlinewidth{0.803000pt}%
\definecolor{currentstroke}{rgb}{0.000000,0.000000,0.000000}%
\pgfsetstrokecolor{currentstroke}%
\pgfsetdash{}{0pt}%
\pgfsys@defobject{currentmarker}{\pgfqpoint{0.000000in}{0.000000in}}{\pgfqpoint{0.048611in}{0.000000in}}{%
\pgfpathmoveto{\pgfqpoint{0.000000in}{0.000000in}}%
\pgfpathlineto{\pgfqpoint{0.048611in}{0.000000in}}%
\pgfusepath{stroke,fill}%
}%
\begin{pgfscope}%
\pgfsys@transformshift{4.604500in}{3.000770in}%
\pgfsys@useobject{currentmarker}{}%
\end{pgfscope}%
\end{pgfscope}%
\begin{pgfscope}%
\pgfsetbuttcap%
\pgfsetroundjoin%
\definecolor{currentfill}{rgb}{0.000000,0.000000,0.000000}%
\pgfsetfillcolor{currentfill}%
\pgfsetlinewidth{0.803000pt}%
\definecolor{currentstroke}{rgb}{0.000000,0.000000,0.000000}%
\pgfsetstrokecolor{currentstroke}%
\pgfsetdash{}{0pt}%
\pgfsys@defobject{currentmarker}{\pgfqpoint{0.000000in}{0.000000in}}{\pgfqpoint{0.048611in}{0.000000in}}{%
\pgfpathmoveto{\pgfqpoint{0.000000in}{0.000000in}}%
\pgfpathlineto{\pgfqpoint{0.048611in}{0.000000in}}%
\pgfusepath{stroke,fill}%
}%
\begin{pgfscope}%
\pgfsys@transformshift{4.604500in}{3.314465in}%
\pgfsys@useobject{currentmarker}{}%
\end{pgfscope}%
\end{pgfscope}%
\begin{pgfscope}%
\pgfsetbuttcap%
\pgfsetroundjoin%
\definecolor{currentfill}{rgb}{0.000000,0.000000,0.000000}%
\pgfsetfillcolor{currentfill}%
\pgfsetlinewidth{0.803000pt}%
\definecolor{currentstroke}{rgb}{0.000000,0.000000,0.000000}%
\pgfsetstrokecolor{currentstroke}%
\pgfsetdash{}{0pt}%
\pgfsys@defobject{currentmarker}{\pgfqpoint{0.000000in}{0.000000in}}{\pgfqpoint{0.048611in}{0.000000in}}{%
\pgfpathmoveto{\pgfqpoint{0.000000in}{0.000000in}}%
\pgfpathlineto{\pgfqpoint{0.048611in}{0.000000in}}%
\pgfusepath{stroke,fill}%
}%
\begin{pgfscope}%
\pgfsys@transformshift{4.604500in}{3.537035in}%
\pgfsys@useobject{currentmarker}{}%
\end{pgfscope}%
\end{pgfscope}%
\begin{pgfscope}%
\pgftext[x=5.217155in,y=2.076590in,,top]{\rmfamily\fontsize{12.000000}{14.400000}\selectfont \(\displaystyle {\mathbf{E} \mbox{u}}_e \equiv \frac{m U_0^2}{q E_0 L}\)}%
\end{pgfscope}%
\begin{pgfscope}%
\pgfsetbuttcap%
\pgfsetmiterjoin%
\pgfsetlinewidth{0.803000pt}%
\definecolor{currentstroke}{rgb}{0.000000,0.000000,0.000000}%
\pgfsetstrokecolor{currentstroke}%
\pgfsetdash{}{0pt}%
\pgfpathmoveto{\pgfqpoint{4.453500in}{0.566590in}}%
\pgfpathlineto{\pgfqpoint{4.453500in}{0.578387in}}%
\pgfpathlineto{\pgfqpoint{4.453500in}{3.574793in}}%
\pgfpathlineto{\pgfqpoint{4.453500in}{3.586590in}}%
\pgfpathlineto{\pgfqpoint{4.604500in}{3.586590in}}%
\pgfpathlineto{\pgfqpoint{4.604500in}{3.574793in}}%
\pgfpathlineto{\pgfqpoint{4.604500in}{0.578387in}}%
\pgfpathlineto{\pgfqpoint{4.604500in}{0.566590in}}%
\pgfpathclose%
\pgfusepath{stroke}%
\end{pgfscope}%
\end{pgfpicture}%
\makeatother%
\endgroup%
}
    \caption{Drop trajectories as a function of $\mathbb{E}\mbox{u}$.\label{fig:series_s_eu}}
\end{figure}
\begin{figure}[htb]
    \centering
    \resizebox{0.5\textwidth}{!}{%% Creator: Matplotlib, PGF backend
%%
%% To include the figure in your LaTeX document, write
%%   \input{<filename>.pgf}
%%
%% Make sure the required packages are loaded in your preamble
%%   \usepackage{pgf}
%%
%% Figures using additional raster images can only be included by \input if
%% they are in the same directory as the main LaTeX file. For loading figures
%% from other directories you can use the `import` package
%%   \usepackage{import}
%% and then include the figures with
%%   \import{<path to file>}{<filename>.pgf}
%%
%% Matplotlib used the following preamble
%%   \usepackage{fontspec}
%%   \setmainfont{DejaVuSerif.ttf}[Path=/home/erin/anaconda3/lib/python3.6/site-packages/matplotlib/mpl-data/fonts/ttf/]
%%   \setsansfont{DejaVuSans.ttf}[Path=/home/erin/anaconda3/lib/python3.6/site-packages/matplotlib/mpl-data/fonts/ttf/]
%%   \setmonofont{DejaVuSansMono.ttf}[Path=/home/erin/anaconda3/lib/python3.6/site-packages/matplotlib/mpl-data/fonts/ttf/]
%%
\begingroup%
\makeatletter%
\begin{pgfpicture}%
\pgfpathrectangle{\pgfpointorigin}{\pgfqpoint{5.311276in}{3.690214in}}%
\pgfusepath{use as bounding box, clip}%
\begin{pgfscope}%
\pgfsetbuttcap%
\pgfsetmiterjoin%
\definecolor{currentfill}{rgb}{1.000000,1.000000,1.000000}%
\pgfsetfillcolor{currentfill}%
\pgfsetlinewidth{0.000000pt}%
\definecolor{currentstroke}{rgb}{1.000000,1.000000,1.000000}%
\pgfsetstrokecolor{currentstroke}%
\pgfsetdash{}{0pt}%
\pgfpathmoveto{\pgfqpoint{0.000000in}{0.000000in}}%
\pgfpathlineto{\pgfqpoint{5.311276in}{0.000000in}}%
\pgfpathlineto{\pgfqpoint{5.311276in}{3.690214in}}%
\pgfpathlineto{\pgfqpoint{0.000000in}{3.690214in}}%
\pgfpathclose%
\pgfusepath{fill}%
\end{pgfscope}%
\begin{pgfscope}%
\pgfsetbuttcap%
\pgfsetmiterjoin%
\definecolor{currentfill}{rgb}{1.000000,1.000000,1.000000}%
\pgfsetfillcolor{currentfill}%
\pgfsetlinewidth{0.000000pt}%
\definecolor{currentstroke}{rgb}{0.000000,0.000000,0.000000}%
\pgfsetstrokecolor{currentstroke}%
\pgfsetstrokeopacity{0.000000}%
\pgfsetdash{}{0pt}%
\pgfpathmoveto{\pgfqpoint{0.564660in}{0.521603in}}%
\pgfpathlineto{\pgfqpoint{4.284660in}{0.521603in}}%
\pgfpathlineto{\pgfqpoint{4.284660in}{3.541603in}}%
\pgfpathlineto{\pgfqpoint{0.564660in}{3.541603in}}%
\pgfpathclose%
\pgfusepath{fill}%
\end{pgfscope}%
\begin{pgfscope}%
\pgfsetbuttcap%
\pgfsetroundjoin%
\definecolor{currentfill}{rgb}{0.000000,0.000000,0.000000}%
\pgfsetfillcolor{currentfill}%
\pgfsetlinewidth{0.803000pt}%
\definecolor{currentstroke}{rgb}{0.000000,0.000000,0.000000}%
\pgfsetstrokecolor{currentstroke}%
\pgfsetdash{}{0pt}%
\pgfsys@defobject{currentmarker}{\pgfqpoint{0.000000in}{-0.048611in}}{\pgfqpoint{0.000000in}{0.000000in}}{%
\pgfpathmoveto{\pgfqpoint{0.000000in}{0.000000in}}%
\pgfpathlineto{\pgfqpoint{0.000000in}{-0.048611in}}%
\pgfusepath{stroke,fill}%
}%
\begin{pgfscope}%
\pgfsys@transformshift{0.564660in}{0.521603in}%
\pgfsys@useobject{currentmarker}{}%
\end{pgfscope}%
\end{pgfscope}%
\begin{pgfscope}%
\definecolor{textcolor}{rgb}{0.000000,0.000000,0.000000}%
\pgfsetstrokecolor{textcolor}%
\pgfsetfillcolor{textcolor}%
\pgftext[x=0.564660in,y=0.424381in,,top]{\color{textcolor}\rmfamily\fontsize{10.000000}{12.000000}\selectfont \(\displaystyle 0.0\)}%
\end{pgfscope}%
\begin{pgfscope}%
\pgfsetbuttcap%
\pgfsetroundjoin%
\definecolor{currentfill}{rgb}{0.000000,0.000000,0.000000}%
\pgfsetfillcolor{currentfill}%
\pgfsetlinewidth{0.803000pt}%
\definecolor{currentstroke}{rgb}{0.000000,0.000000,0.000000}%
\pgfsetstrokecolor{currentstroke}%
\pgfsetdash{}{0pt}%
\pgfsys@defobject{currentmarker}{\pgfqpoint{0.000000in}{-0.048611in}}{\pgfqpoint{0.000000in}{0.000000in}}{%
\pgfpathmoveto{\pgfqpoint{0.000000in}{0.000000in}}%
\pgfpathlineto{\pgfqpoint{0.000000in}{-0.048611in}}%
\pgfusepath{stroke,fill}%
}%
\begin{pgfscope}%
\pgfsys@transformshift{1.029660in}{0.521603in}%
\pgfsys@useobject{currentmarker}{}%
\end{pgfscope}%
\end{pgfscope}%
\begin{pgfscope}%
\definecolor{textcolor}{rgb}{0.000000,0.000000,0.000000}%
\pgfsetstrokecolor{textcolor}%
\pgfsetfillcolor{textcolor}%
\pgftext[x=1.029660in,y=0.424381in,,top]{\color{textcolor}\rmfamily\fontsize{10.000000}{12.000000}\selectfont \(\displaystyle 0.5\)}%
\end{pgfscope}%
\begin{pgfscope}%
\pgfsetbuttcap%
\pgfsetroundjoin%
\definecolor{currentfill}{rgb}{0.000000,0.000000,0.000000}%
\pgfsetfillcolor{currentfill}%
\pgfsetlinewidth{0.803000pt}%
\definecolor{currentstroke}{rgb}{0.000000,0.000000,0.000000}%
\pgfsetstrokecolor{currentstroke}%
\pgfsetdash{}{0pt}%
\pgfsys@defobject{currentmarker}{\pgfqpoint{0.000000in}{-0.048611in}}{\pgfqpoint{0.000000in}{0.000000in}}{%
\pgfpathmoveto{\pgfqpoint{0.000000in}{0.000000in}}%
\pgfpathlineto{\pgfqpoint{0.000000in}{-0.048611in}}%
\pgfusepath{stroke,fill}%
}%
\begin{pgfscope}%
\pgfsys@transformshift{1.494660in}{0.521603in}%
\pgfsys@useobject{currentmarker}{}%
\end{pgfscope}%
\end{pgfscope}%
\begin{pgfscope}%
\definecolor{textcolor}{rgb}{0.000000,0.000000,0.000000}%
\pgfsetstrokecolor{textcolor}%
\pgfsetfillcolor{textcolor}%
\pgftext[x=1.494660in,y=0.424381in,,top]{\color{textcolor}\rmfamily\fontsize{10.000000}{12.000000}\selectfont \(\displaystyle 1.0\)}%
\end{pgfscope}%
\begin{pgfscope}%
\pgfsetbuttcap%
\pgfsetroundjoin%
\definecolor{currentfill}{rgb}{0.000000,0.000000,0.000000}%
\pgfsetfillcolor{currentfill}%
\pgfsetlinewidth{0.803000pt}%
\definecolor{currentstroke}{rgb}{0.000000,0.000000,0.000000}%
\pgfsetstrokecolor{currentstroke}%
\pgfsetdash{}{0pt}%
\pgfsys@defobject{currentmarker}{\pgfqpoint{0.000000in}{-0.048611in}}{\pgfqpoint{0.000000in}{0.000000in}}{%
\pgfpathmoveto{\pgfqpoint{0.000000in}{0.000000in}}%
\pgfpathlineto{\pgfqpoint{0.000000in}{-0.048611in}}%
\pgfusepath{stroke,fill}%
}%
\begin{pgfscope}%
\pgfsys@transformshift{1.959660in}{0.521603in}%
\pgfsys@useobject{currentmarker}{}%
\end{pgfscope}%
\end{pgfscope}%
\begin{pgfscope}%
\definecolor{textcolor}{rgb}{0.000000,0.000000,0.000000}%
\pgfsetstrokecolor{textcolor}%
\pgfsetfillcolor{textcolor}%
\pgftext[x=1.959660in,y=0.424381in,,top]{\color{textcolor}\rmfamily\fontsize{10.000000}{12.000000}\selectfont \(\displaystyle 1.5\)}%
\end{pgfscope}%
\begin{pgfscope}%
\pgfsetbuttcap%
\pgfsetroundjoin%
\definecolor{currentfill}{rgb}{0.000000,0.000000,0.000000}%
\pgfsetfillcolor{currentfill}%
\pgfsetlinewidth{0.803000pt}%
\definecolor{currentstroke}{rgb}{0.000000,0.000000,0.000000}%
\pgfsetstrokecolor{currentstroke}%
\pgfsetdash{}{0pt}%
\pgfsys@defobject{currentmarker}{\pgfqpoint{0.000000in}{-0.048611in}}{\pgfqpoint{0.000000in}{0.000000in}}{%
\pgfpathmoveto{\pgfqpoint{0.000000in}{0.000000in}}%
\pgfpathlineto{\pgfqpoint{0.000000in}{-0.048611in}}%
\pgfusepath{stroke,fill}%
}%
\begin{pgfscope}%
\pgfsys@transformshift{2.424660in}{0.521603in}%
\pgfsys@useobject{currentmarker}{}%
\end{pgfscope}%
\end{pgfscope}%
\begin{pgfscope}%
\definecolor{textcolor}{rgb}{0.000000,0.000000,0.000000}%
\pgfsetstrokecolor{textcolor}%
\pgfsetfillcolor{textcolor}%
\pgftext[x=2.424660in,y=0.424381in,,top]{\color{textcolor}\rmfamily\fontsize{10.000000}{12.000000}\selectfont \(\displaystyle 2.0\)}%
\end{pgfscope}%
\begin{pgfscope}%
\pgfsetbuttcap%
\pgfsetroundjoin%
\definecolor{currentfill}{rgb}{0.000000,0.000000,0.000000}%
\pgfsetfillcolor{currentfill}%
\pgfsetlinewidth{0.803000pt}%
\definecolor{currentstroke}{rgb}{0.000000,0.000000,0.000000}%
\pgfsetstrokecolor{currentstroke}%
\pgfsetdash{}{0pt}%
\pgfsys@defobject{currentmarker}{\pgfqpoint{0.000000in}{-0.048611in}}{\pgfqpoint{0.000000in}{0.000000in}}{%
\pgfpathmoveto{\pgfqpoint{0.000000in}{0.000000in}}%
\pgfpathlineto{\pgfqpoint{0.000000in}{-0.048611in}}%
\pgfusepath{stroke,fill}%
}%
\begin{pgfscope}%
\pgfsys@transformshift{2.889660in}{0.521603in}%
\pgfsys@useobject{currentmarker}{}%
\end{pgfscope}%
\end{pgfscope}%
\begin{pgfscope}%
\definecolor{textcolor}{rgb}{0.000000,0.000000,0.000000}%
\pgfsetstrokecolor{textcolor}%
\pgfsetfillcolor{textcolor}%
\pgftext[x=2.889660in,y=0.424381in,,top]{\color{textcolor}\rmfamily\fontsize{10.000000}{12.000000}\selectfont \(\displaystyle 2.5\)}%
\end{pgfscope}%
\begin{pgfscope}%
\pgfsetbuttcap%
\pgfsetroundjoin%
\definecolor{currentfill}{rgb}{0.000000,0.000000,0.000000}%
\pgfsetfillcolor{currentfill}%
\pgfsetlinewidth{0.803000pt}%
\definecolor{currentstroke}{rgb}{0.000000,0.000000,0.000000}%
\pgfsetstrokecolor{currentstroke}%
\pgfsetdash{}{0pt}%
\pgfsys@defobject{currentmarker}{\pgfqpoint{0.000000in}{-0.048611in}}{\pgfqpoint{0.000000in}{0.000000in}}{%
\pgfpathmoveto{\pgfqpoint{0.000000in}{0.000000in}}%
\pgfpathlineto{\pgfqpoint{0.000000in}{-0.048611in}}%
\pgfusepath{stroke,fill}%
}%
\begin{pgfscope}%
\pgfsys@transformshift{3.354660in}{0.521603in}%
\pgfsys@useobject{currentmarker}{}%
\end{pgfscope}%
\end{pgfscope}%
\begin{pgfscope}%
\definecolor{textcolor}{rgb}{0.000000,0.000000,0.000000}%
\pgfsetstrokecolor{textcolor}%
\pgfsetfillcolor{textcolor}%
\pgftext[x=3.354660in,y=0.424381in,,top]{\color{textcolor}\rmfamily\fontsize{10.000000}{12.000000}\selectfont \(\displaystyle 3.0\)}%
\end{pgfscope}%
\begin{pgfscope}%
\pgfsetbuttcap%
\pgfsetroundjoin%
\definecolor{currentfill}{rgb}{0.000000,0.000000,0.000000}%
\pgfsetfillcolor{currentfill}%
\pgfsetlinewidth{0.803000pt}%
\definecolor{currentstroke}{rgb}{0.000000,0.000000,0.000000}%
\pgfsetstrokecolor{currentstroke}%
\pgfsetdash{}{0pt}%
\pgfsys@defobject{currentmarker}{\pgfqpoint{0.000000in}{-0.048611in}}{\pgfqpoint{0.000000in}{0.000000in}}{%
\pgfpathmoveto{\pgfqpoint{0.000000in}{0.000000in}}%
\pgfpathlineto{\pgfqpoint{0.000000in}{-0.048611in}}%
\pgfusepath{stroke,fill}%
}%
\begin{pgfscope}%
\pgfsys@transformshift{3.819660in}{0.521603in}%
\pgfsys@useobject{currentmarker}{}%
\end{pgfscope}%
\end{pgfscope}%
\begin{pgfscope}%
\definecolor{textcolor}{rgb}{0.000000,0.000000,0.000000}%
\pgfsetstrokecolor{textcolor}%
\pgfsetfillcolor{textcolor}%
\pgftext[x=3.819660in,y=0.424381in,,top]{\color{textcolor}\rmfamily\fontsize{10.000000}{12.000000}\selectfont \(\displaystyle 3.5\)}%
\end{pgfscope}%
\begin{pgfscope}%
\pgfsetbuttcap%
\pgfsetroundjoin%
\definecolor{currentfill}{rgb}{0.000000,0.000000,0.000000}%
\pgfsetfillcolor{currentfill}%
\pgfsetlinewidth{0.803000pt}%
\definecolor{currentstroke}{rgb}{0.000000,0.000000,0.000000}%
\pgfsetstrokecolor{currentstroke}%
\pgfsetdash{}{0pt}%
\pgfsys@defobject{currentmarker}{\pgfqpoint{0.000000in}{-0.048611in}}{\pgfqpoint{0.000000in}{0.000000in}}{%
\pgfpathmoveto{\pgfqpoint{0.000000in}{0.000000in}}%
\pgfpathlineto{\pgfqpoint{0.000000in}{-0.048611in}}%
\pgfusepath{stroke,fill}%
}%
\begin{pgfscope}%
\pgfsys@transformshift{4.284660in}{0.521603in}%
\pgfsys@useobject{currentmarker}{}%
\end{pgfscope}%
\end{pgfscope}%
\begin{pgfscope}%
\definecolor{textcolor}{rgb}{0.000000,0.000000,0.000000}%
\pgfsetstrokecolor{textcolor}%
\pgfsetfillcolor{textcolor}%
\pgftext[x=4.284660in,y=0.424381in,,top]{\color{textcolor}\rmfamily\fontsize{10.000000}{12.000000}\selectfont \(\displaystyle 4.0\)}%
\end{pgfscope}%
\begin{pgfscope}%
\definecolor{textcolor}{rgb}{0.000000,0.000000,0.000000}%
\pgfsetstrokecolor{textcolor}%
\pgfsetfillcolor{textcolor}%
\pgftext[x=2.424660in,y=0.234413in,,top]{\color{textcolor}\rmfamily\fontsize{10.000000}{12.000000}\selectfont \(\displaystyle t^*\)}%
\end{pgfscope}%
\begin{pgfscope}%
\pgfsetbuttcap%
\pgfsetroundjoin%
\definecolor{currentfill}{rgb}{0.000000,0.000000,0.000000}%
\pgfsetfillcolor{currentfill}%
\pgfsetlinewidth{0.803000pt}%
\definecolor{currentstroke}{rgb}{0.000000,0.000000,0.000000}%
\pgfsetstrokecolor{currentstroke}%
\pgfsetdash{}{0pt}%
\pgfsys@defobject{currentmarker}{\pgfqpoint{-0.048611in}{0.000000in}}{\pgfqpoint{0.000000in}{0.000000in}}{%
\pgfpathmoveto{\pgfqpoint{0.000000in}{0.000000in}}%
\pgfpathlineto{\pgfqpoint{-0.048611in}{0.000000in}}%
\pgfusepath{stroke,fill}%
}%
\begin{pgfscope}%
\pgfsys@transformshift{0.564660in}{0.671953in}%
\pgfsys@useobject{currentmarker}{}%
\end{pgfscope}%
\end{pgfscope}%
\begin{pgfscope}%
\definecolor{textcolor}{rgb}{0.000000,0.000000,0.000000}%
\pgfsetstrokecolor{textcolor}%
\pgfsetfillcolor{textcolor}%
\pgftext[x=0.289968in,y=0.619192in,left,base]{\color{textcolor}\rmfamily\fontsize{10.000000}{12.000000}\selectfont \(\displaystyle 0.0\)}%
\end{pgfscope}%
\begin{pgfscope}%
\pgfsetbuttcap%
\pgfsetroundjoin%
\definecolor{currentfill}{rgb}{0.000000,0.000000,0.000000}%
\pgfsetfillcolor{currentfill}%
\pgfsetlinewidth{0.803000pt}%
\definecolor{currentstroke}{rgb}{0.000000,0.000000,0.000000}%
\pgfsetstrokecolor{currentstroke}%
\pgfsetdash{}{0pt}%
\pgfsys@defobject{currentmarker}{\pgfqpoint{-0.048611in}{0.000000in}}{\pgfqpoint{0.000000in}{0.000000in}}{%
\pgfpathmoveto{\pgfqpoint{0.000000in}{0.000000in}}%
\pgfpathlineto{\pgfqpoint{-0.048611in}{0.000000in}}%
\pgfusepath{stroke,fill}%
}%
\begin{pgfscope}%
\pgfsys@transformshift{0.564660in}{1.124453in}%
\pgfsys@useobject{currentmarker}{}%
\end{pgfscope}%
\end{pgfscope}%
\begin{pgfscope}%
\definecolor{textcolor}{rgb}{0.000000,0.000000,0.000000}%
\pgfsetstrokecolor{textcolor}%
\pgfsetfillcolor{textcolor}%
\pgftext[x=0.289968in,y=1.071692in,left,base]{\color{textcolor}\rmfamily\fontsize{10.000000}{12.000000}\selectfont \(\displaystyle 0.2\)}%
\end{pgfscope}%
\begin{pgfscope}%
\pgfsetbuttcap%
\pgfsetroundjoin%
\definecolor{currentfill}{rgb}{0.000000,0.000000,0.000000}%
\pgfsetfillcolor{currentfill}%
\pgfsetlinewidth{0.803000pt}%
\definecolor{currentstroke}{rgb}{0.000000,0.000000,0.000000}%
\pgfsetstrokecolor{currentstroke}%
\pgfsetdash{}{0pt}%
\pgfsys@defobject{currentmarker}{\pgfqpoint{-0.048611in}{0.000000in}}{\pgfqpoint{0.000000in}{0.000000in}}{%
\pgfpathmoveto{\pgfqpoint{0.000000in}{0.000000in}}%
\pgfpathlineto{\pgfqpoint{-0.048611in}{0.000000in}}%
\pgfusepath{stroke,fill}%
}%
\begin{pgfscope}%
\pgfsys@transformshift{0.564660in}{1.576953in}%
\pgfsys@useobject{currentmarker}{}%
\end{pgfscope}%
\end{pgfscope}%
\begin{pgfscope}%
\definecolor{textcolor}{rgb}{0.000000,0.000000,0.000000}%
\pgfsetstrokecolor{textcolor}%
\pgfsetfillcolor{textcolor}%
\pgftext[x=0.289968in,y=1.524192in,left,base]{\color{textcolor}\rmfamily\fontsize{10.000000}{12.000000}\selectfont \(\displaystyle 0.4\)}%
\end{pgfscope}%
\begin{pgfscope}%
\pgfsetbuttcap%
\pgfsetroundjoin%
\definecolor{currentfill}{rgb}{0.000000,0.000000,0.000000}%
\pgfsetfillcolor{currentfill}%
\pgfsetlinewidth{0.803000pt}%
\definecolor{currentstroke}{rgb}{0.000000,0.000000,0.000000}%
\pgfsetstrokecolor{currentstroke}%
\pgfsetdash{}{0pt}%
\pgfsys@defobject{currentmarker}{\pgfqpoint{-0.048611in}{0.000000in}}{\pgfqpoint{0.000000in}{0.000000in}}{%
\pgfpathmoveto{\pgfqpoint{0.000000in}{0.000000in}}%
\pgfpathlineto{\pgfqpoint{-0.048611in}{0.000000in}}%
\pgfusepath{stroke,fill}%
}%
\begin{pgfscope}%
\pgfsys@transformshift{0.564660in}{2.029453in}%
\pgfsys@useobject{currentmarker}{}%
\end{pgfscope}%
\end{pgfscope}%
\begin{pgfscope}%
\definecolor{textcolor}{rgb}{0.000000,0.000000,0.000000}%
\pgfsetstrokecolor{textcolor}%
\pgfsetfillcolor{textcolor}%
\pgftext[x=0.289968in,y=1.976692in,left,base]{\color{textcolor}\rmfamily\fontsize{10.000000}{12.000000}\selectfont \(\displaystyle 0.6\)}%
\end{pgfscope}%
\begin{pgfscope}%
\pgfsetbuttcap%
\pgfsetroundjoin%
\definecolor{currentfill}{rgb}{0.000000,0.000000,0.000000}%
\pgfsetfillcolor{currentfill}%
\pgfsetlinewidth{0.803000pt}%
\definecolor{currentstroke}{rgb}{0.000000,0.000000,0.000000}%
\pgfsetstrokecolor{currentstroke}%
\pgfsetdash{}{0pt}%
\pgfsys@defobject{currentmarker}{\pgfqpoint{-0.048611in}{0.000000in}}{\pgfqpoint{0.000000in}{0.000000in}}{%
\pgfpathmoveto{\pgfqpoint{0.000000in}{0.000000in}}%
\pgfpathlineto{\pgfqpoint{-0.048611in}{0.000000in}}%
\pgfusepath{stroke,fill}%
}%
\begin{pgfscope}%
\pgfsys@transformshift{0.564660in}{2.481953in}%
\pgfsys@useobject{currentmarker}{}%
\end{pgfscope}%
\end{pgfscope}%
\begin{pgfscope}%
\definecolor{textcolor}{rgb}{0.000000,0.000000,0.000000}%
\pgfsetstrokecolor{textcolor}%
\pgfsetfillcolor{textcolor}%
\pgftext[x=0.289968in,y=2.429192in,left,base]{\color{textcolor}\rmfamily\fontsize{10.000000}{12.000000}\selectfont \(\displaystyle 0.8\)}%
\end{pgfscope}%
\begin{pgfscope}%
\pgfsetbuttcap%
\pgfsetroundjoin%
\definecolor{currentfill}{rgb}{0.000000,0.000000,0.000000}%
\pgfsetfillcolor{currentfill}%
\pgfsetlinewidth{0.803000pt}%
\definecolor{currentstroke}{rgb}{0.000000,0.000000,0.000000}%
\pgfsetstrokecolor{currentstroke}%
\pgfsetdash{}{0pt}%
\pgfsys@defobject{currentmarker}{\pgfqpoint{-0.048611in}{0.000000in}}{\pgfqpoint{0.000000in}{0.000000in}}{%
\pgfpathmoveto{\pgfqpoint{0.000000in}{0.000000in}}%
\pgfpathlineto{\pgfqpoint{-0.048611in}{0.000000in}}%
\pgfusepath{stroke,fill}%
}%
\begin{pgfscope}%
\pgfsys@transformshift{0.564660in}{2.934453in}%
\pgfsys@useobject{currentmarker}{}%
\end{pgfscope}%
\end{pgfscope}%
\begin{pgfscope}%
\definecolor{textcolor}{rgb}{0.000000,0.000000,0.000000}%
\pgfsetstrokecolor{textcolor}%
\pgfsetfillcolor{textcolor}%
\pgftext[x=0.289968in,y=2.881692in,left,base]{\color{textcolor}\rmfamily\fontsize{10.000000}{12.000000}\selectfont \(\displaystyle 1.0\)}%
\end{pgfscope}%
\begin{pgfscope}%
\pgfsetbuttcap%
\pgfsetroundjoin%
\definecolor{currentfill}{rgb}{0.000000,0.000000,0.000000}%
\pgfsetfillcolor{currentfill}%
\pgfsetlinewidth{0.803000pt}%
\definecolor{currentstroke}{rgb}{0.000000,0.000000,0.000000}%
\pgfsetstrokecolor{currentstroke}%
\pgfsetdash{}{0pt}%
\pgfsys@defobject{currentmarker}{\pgfqpoint{-0.048611in}{0.000000in}}{\pgfqpoint{0.000000in}{0.000000in}}{%
\pgfpathmoveto{\pgfqpoint{0.000000in}{0.000000in}}%
\pgfpathlineto{\pgfqpoint{-0.048611in}{0.000000in}}%
\pgfusepath{stroke,fill}%
}%
\begin{pgfscope}%
\pgfsys@transformshift{0.564660in}{3.386954in}%
\pgfsys@useobject{currentmarker}{}%
\end{pgfscope}%
\end{pgfscope}%
\begin{pgfscope}%
\definecolor{textcolor}{rgb}{0.000000,0.000000,0.000000}%
\pgfsetstrokecolor{textcolor}%
\pgfsetfillcolor{textcolor}%
\pgftext[x=0.289968in,y=3.334192in,left,base]{\color{textcolor}\rmfamily\fontsize{10.000000}{12.000000}\selectfont \(\displaystyle 1.2\)}%
\end{pgfscope}%
\begin{pgfscope}%
\definecolor{textcolor}{rgb}{0.000000,0.000000,0.000000}%
\pgfsetstrokecolor{textcolor}%
\pgfsetfillcolor{textcolor}%
\pgftext[x=0.234413in,y=2.031603in,,bottom,rotate=90.000000]{\color{textcolor}\rmfamily\fontsize{10.000000}{12.000000}\selectfont \(\displaystyle y^*\)}%
\end{pgfscope}%
\begin{pgfscope}%
\pgfpathrectangle{\pgfqpoint{0.564660in}{0.521603in}}{\pgfqpoint{3.720000in}{3.020000in}}%
\pgfusepath{clip}%
\pgfsetrectcap%
\pgfsetroundjoin%
\pgfsetlinewidth{1.505625pt}%
\definecolor{currentstroke}{rgb}{0.993248,0.906157,0.143936}%
\pgfsetstrokecolor{currentstroke}%
\pgfsetdash{}{0pt}%
\pgfpathmoveto{\pgfqpoint{1.409488in}{2.299358in}}%
\pgfpathlineto{\pgfqpoint{1.486291in}{2.475880in}}%
\pgfpathlineto{\pgfqpoint{1.563093in}{2.637799in}}%
\pgfpathlineto{\pgfqpoint{1.639896in}{2.784889in}}%
\pgfpathlineto{\pgfqpoint{1.716698in}{2.916927in}}%
\pgfpathlineto{\pgfqpoint{1.793501in}{3.033686in}}%
\pgfpathlineto{\pgfqpoint{1.870303in}{3.134943in}}%
\pgfpathlineto{\pgfqpoint{1.947106in}{3.220472in}}%
\pgfpathlineto{\pgfqpoint{2.023908in}{3.290048in}}%
\pgfpathlineto{\pgfqpoint{2.100711in}{3.343447in}}%
\pgfpathlineto{\pgfqpoint{2.177513in}{3.380444in}}%
\pgfpathlineto{\pgfqpoint{2.254316in}{3.400813in}}%
\pgfpathlineto{\pgfqpoint{2.331118in}{3.404331in}}%
\pgfpathlineto{\pgfqpoint{2.407921in}{3.390904in}}%
\pgfpathlineto{\pgfqpoint{2.484723in}{3.360261in}}%
\pgfpathlineto{\pgfqpoint{2.561526in}{3.312160in}}%
\pgfpathlineto{\pgfqpoint{2.638328in}{3.246345in}}%
\pgfpathlineto{\pgfqpoint{2.715131in}{3.162550in}}%
\pgfpathlineto{\pgfqpoint{2.791933in}{3.060539in}}%
\pgfpathlineto{\pgfqpoint{2.868736in}{2.939892in}}%
\pgfpathlineto{\pgfqpoint{2.945539in}{2.800327in}}%
\pgfpathlineto{\pgfqpoint{3.022341in}{2.641559in}}%
\pgfpathlineto{\pgfqpoint{3.099144in}{2.463303in}}%
\pgfpathlineto{\pgfqpoint{3.175946in}{2.265275in}}%
\pgfpathlineto{\pgfqpoint{3.252749in}{2.047191in}}%
\pgfpathlineto{\pgfqpoint{3.329551in}{1.808766in}}%
\pgfpathlineto{\pgfqpoint{3.406354in}{1.549716in}}%
\pgfpathlineto{\pgfqpoint{3.483156in}{1.269756in}}%
\pgfpathlineto{\pgfqpoint{3.559959in}{0.968602in}}%
\pgfusepath{stroke}%
\end{pgfscope}%
\begin{pgfscope}%
\pgfpathrectangle{\pgfqpoint{0.564660in}{0.521603in}}{\pgfqpoint{3.720000in}{3.020000in}}%
\pgfusepath{clip}%
\pgfsetrectcap%
\pgfsetroundjoin%
\pgfsetlinewidth{1.505625pt}%
\definecolor{currentstroke}{rgb}{0.762373,0.876424,0.137064}%
\pgfsetstrokecolor{currentstroke}%
\pgfsetdash{}{0pt}%
\pgfpathmoveto{\pgfqpoint{1.233611in}{0.756271in}}%
\pgfpathlineto{\pgfqpoint{1.300506in}{0.938723in}}%
\pgfpathlineto{\pgfqpoint{1.367401in}{1.107373in}}%
\pgfpathlineto{\pgfqpoint{1.434296in}{1.262342in}}%
\pgfpathlineto{\pgfqpoint{1.501191in}{1.403753in}}%
\pgfpathlineto{\pgfqpoint{1.568086in}{1.531729in}}%
\pgfpathlineto{\pgfqpoint{1.634981in}{1.646390in}}%
\pgfpathlineto{\pgfqpoint{1.701876in}{1.747859in}}%
\pgfpathlineto{\pgfqpoint{1.768772in}{1.836259in}}%
\pgfpathlineto{\pgfqpoint{1.835667in}{1.911710in}}%
\pgfpathlineto{\pgfqpoint{1.902562in}{1.974336in}}%
\pgfpathlineto{\pgfqpoint{1.969457in}{2.024259in}}%
\pgfpathlineto{\pgfqpoint{2.036352in}{2.061600in}}%
\pgfpathlineto{\pgfqpoint{2.103247in}{2.086685in}}%
\pgfpathlineto{\pgfqpoint{2.170142in}{2.099560in}}%
\pgfpathlineto{\pgfqpoint{2.237037in}{2.100321in}}%
\pgfpathlineto{\pgfqpoint{2.303932in}{2.089048in}}%
\pgfpathlineto{\pgfqpoint{2.370827in}{2.065802in}}%
\pgfpathlineto{\pgfqpoint{2.437722in}{2.030627in}}%
\pgfpathlineto{\pgfqpoint{2.504617in}{1.983622in}}%
\pgfpathlineto{\pgfqpoint{2.571512in}{1.924599in}}%
\pgfpathlineto{\pgfqpoint{2.638407in}{1.853579in}}%
\pgfpathlineto{\pgfqpoint{2.705302in}{1.770581in}}%
\pgfpathlineto{\pgfqpoint{2.772198in}{1.675623in}}%
\pgfpathlineto{\pgfqpoint{2.839093in}{1.568725in}}%
\pgfpathlineto{\pgfqpoint{2.905988in}{1.449905in}}%
\pgfpathlineto{\pgfqpoint{2.972883in}{1.319184in}}%
\pgfpathlineto{\pgfqpoint{3.039778in}{1.176579in}}%
\pgfpathlineto{\pgfqpoint{3.106673in}{1.022111in}}%
\pgfpathlineto{\pgfqpoint{3.173568in}{0.855798in}}%
\pgfpathlineto{\pgfqpoint{3.240463in}{0.677659in}}%
\pgfusepath{stroke}%
\end{pgfscope}%
\begin{pgfscope}%
\pgfpathrectangle{\pgfqpoint{0.564660in}{0.521603in}}{\pgfqpoint{3.720000in}{3.020000in}}%
\pgfusepath{clip}%
\pgfsetrectcap%
\pgfsetroundjoin%
\pgfsetlinewidth{1.505625pt}%
\definecolor{currentstroke}{rgb}{0.751884,0.874951,0.143228}%
\pgfsetstrokecolor{currentstroke}%
\pgfsetdash{}{0pt}%
\pgfpathmoveto{\pgfqpoint{1.178273in}{2.036187in}}%
\pgfpathlineto{\pgfqpoint{1.246452in}{2.193330in}}%
\pgfpathlineto{\pgfqpoint{1.314631in}{2.339382in}}%
\pgfpathlineto{\pgfqpoint{1.382811in}{2.474114in}}%
\pgfpathlineto{\pgfqpoint{1.450990in}{2.597300in}}%
\pgfpathlineto{\pgfqpoint{1.519169in}{2.708710in}}%
\pgfpathlineto{\pgfqpoint{1.587348in}{2.808117in}}%
\pgfpathlineto{\pgfqpoint{1.655527in}{2.895293in}}%
\pgfpathlineto{\pgfqpoint{1.723707in}{2.970010in}}%
\pgfpathlineto{\pgfqpoint{1.791886in}{3.032040in}}%
\pgfpathlineto{\pgfqpoint{1.860065in}{3.081155in}}%
\pgfpathlineto{\pgfqpoint{1.928244in}{3.117128in}}%
\pgfpathlineto{\pgfqpoint{1.996423in}{3.139729in}}%
\pgfpathlineto{\pgfqpoint{2.064603in}{3.148971in}}%
\pgfpathlineto{\pgfqpoint{2.132782in}{3.144533in}}%
\pgfpathlineto{\pgfqpoint{2.200961in}{3.126161in}}%
\pgfpathlineto{\pgfqpoint{2.269140in}{3.093578in}}%
\pgfpathlineto{\pgfqpoint{2.337319in}{3.046489in}}%
\pgfpathlineto{\pgfqpoint{2.405499in}{2.984576in}}%
\pgfpathlineto{\pgfqpoint{2.473678in}{2.907503in}}%
\pgfpathlineto{\pgfqpoint{2.541857in}{2.814997in}}%
\pgfpathlineto{\pgfqpoint{2.610036in}{2.706455in}}%
\pgfpathlineto{\pgfqpoint{2.678215in}{2.581512in}}%
\pgfpathlineto{\pgfqpoint{2.746395in}{2.439805in}}%
\pgfpathlineto{\pgfqpoint{2.814574in}{2.280968in}}%
\pgfpathlineto{\pgfqpoint{2.882753in}{2.104637in}}%
\pgfpathlineto{\pgfqpoint{2.950932in}{1.910447in}}%
\pgfpathlineto{\pgfqpoint{3.019111in}{1.698036in}}%
\pgfpathlineto{\pgfqpoint{3.087291in}{1.467037in}}%
\pgfpathlineto{\pgfqpoint{3.155470in}{1.217087in}}%
\pgfpathlineto{\pgfqpoint{3.223649in}{0.947822in}}%
\pgfpathlineto{\pgfqpoint{3.291828in}{0.658876in}}%
\pgfusepath{stroke}%
\end{pgfscope}%
\begin{pgfscope}%
\pgfpathrectangle{\pgfqpoint{0.564660in}{0.521603in}}{\pgfqpoint{3.720000in}{3.020000in}}%
\pgfusepath{clip}%
\pgfsetrectcap%
\pgfsetroundjoin%
\pgfsetlinewidth{1.505625pt}%
\definecolor{currentstroke}{rgb}{0.377779,0.791781,0.377939}%
\pgfsetstrokecolor{currentstroke}%
\pgfsetdash{}{0pt}%
\pgfpathmoveto{\pgfqpoint{1.114895in}{1.137753in}}%
\pgfpathlineto{\pgfqpoint{1.183674in}{1.296900in}}%
\pgfpathlineto{\pgfqpoint{1.252453in}{1.444075in}}%
\pgfpathlineto{\pgfqpoint{1.321233in}{1.579273in}}%
\pgfpathlineto{\pgfqpoint{1.390012in}{1.702487in}}%
\pgfpathlineto{\pgfqpoint{1.458791in}{1.813712in}}%
\pgfpathlineto{\pgfqpoint{1.527570in}{1.912942in}}%
\pgfpathlineto{\pgfqpoint{1.596350in}{2.000173in}}%
\pgfpathlineto{\pgfqpoint{1.665129in}{2.075397in}}%
\pgfpathlineto{\pgfqpoint{1.733908in}{2.138609in}}%
\pgfpathlineto{\pgfqpoint{1.802688in}{2.189804in}}%
\pgfpathlineto{\pgfqpoint{1.871467in}{2.228976in}}%
\pgfpathlineto{\pgfqpoint{1.940246in}{2.256119in}}%
\pgfpathlineto{\pgfqpoint{2.009025in}{2.271336in}}%
\pgfpathlineto{\pgfqpoint{2.077805in}{2.274580in}}%
\pgfpathlineto{\pgfqpoint{2.146584in}{2.265833in}}%
\pgfpathlineto{\pgfqpoint{2.215363in}{2.245065in}}%
\pgfpathlineto{\pgfqpoint{2.284143in}{2.212239in}}%
\pgfpathlineto{\pgfqpoint{2.352922in}{2.167308in}}%
\pgfpathlineto{\pgfqpoint{2.421701in}{2.110253in}}%
\pgfpathlineto{\pgfqpoint{2.490481in}{2.040905in}}%
\pgfpathlineto{\pgfqpoint{2.559260in}{1.959203in}}%
\pgfpathlineto{\pgfqpoint{2.628039in}{1.865086in}}%
\pgfpathlineto{\pgfqpoint{2.696818in}{1.758495in}}%
\pgfpathlineto{\pgfqpoint{2.765598in}{1.639368in}}%
\pgfpathlineto{\pgfqpoint{2.834377in}{1.507646in}}%
\pgfpathlineto{\pgfqpoint{2.903156in}{1.363268in}}%
\pgfpathlineto{\pgfqpoint{2.971936in}{1.206173in}}%
\pgfpathlineto{\pgfqpoint{3.040715in}{1.036301in}}%
\pgfpathlineto{\pgfqpoint{3.109494in}{0.853591in}}%
\pgfusepath{stroke}%
\end{pgfscope}%
\begin{pgfscope}%
\pgfpathrectangle{\pgfqpoint{0.564660in}{0.521603in}}{\pgfqpoint{3.720000in}{3.020000in}}%
\pgfusepath{clip}%
\pgfsetrectcap%
\pgfsetroundjoin%
\pgfsetlinewidth{1.505625pt}%
\definecolor{currentstroke}{rgb}{0.128729,0.563265,0.551229}%
\pgfsetstrokecolor{currentstroke}%
\pgfsetdash{}{0pt}%
\pgfpathmoveto{\pgfqpoint{1.002735in}{1.139100in}}%
\pgfpathlineto{\pgfqpoint{1.046542in}{1.242081in}}%
\pgfpathlineto{\pgfqpoint{1.090350in}{1.340628in}}%
\pgfpathlineto{\pgfqpoint{1.134157in}{1.434751in}}%
\pgfpathlineto{\pgfqpoint{1.177965in}{1.524460in}}%
\pgfpathlineto{\pgfqpoint{1.221772in}{1.609765in}}%
\pgfpathlineto{\pgfqpoint{1.265580in}{1.690674in}}%
\pgfpathlineto{\pgfqpoint{1.309387in}{1.767198in}}%
\pgfpathlineto{\pgfqpoint{1.353195in}{1.839346in}}%
\pgfpathlineto{\pgfqpoint{1.397002in}{1.907128in}}%
\pgfpathlineto{\pgfqpoint{1.440810in}{1.970554in}}%
\pgfpathlineto{\pgfqpoint{1.484617in}{2.029633in}}%
\pgfpathlineto{\pgfqpoint{1.528425in}{2.084375in}}%
\pgfpathlineto{\pgfqpoint{1.572232in}{2.134787in}}%
\pgfpathlineto{\pgfqpoint{1.616040in}{2.180881in}}%
\pgfpathlineto{\pgfqpoint{1.659847in}{2.222665in}}%
\pgfpathlineto{\pgfqpoint{1.703655in}{2.260153in}}%
\pgfpathlineto{\pgfqpoint{1.747462in}{2.293356in}}%
\pgfpathlineto{\pgfqpoint{1.791269in}{2.322287in}}%
\pgfpathlineto{\pgfqpoint{1.835077in}{2.346957in}}%
\pgfpathlineto{\pgfqpoint{1.878884in}{2.367380in}}%
\pgfpathlineto{\pgfqpoint{1.922692in}{2.383567in}}%
\pgfpathlineto{\pgfqpoint{1.966499in}{2.395529in}}%
\pgfpathlineto{\pgfqpoint{2.010307in}{2.403277in}}%
\pgfpathlineto{\pgfqpoint{2.054114in}{2.406820in}}%
\pgfpathlineto{\pgfqpoint{2.097922in}{2.406163in}}%
\pgfpathlineto{\pgfqpoint{2.141729in}{2.401310in}}%
\pgfpathlineto{\pgfqpoint{2.185537in}{2.392259in}}%
\pgfpathlineto{\pgfqpoint{2.229344in}{2.379005in}}%
\pgfpathlineto{\pgfqpoint{2.273152in}{2.361536in}}%
\pgfpathlineto{\pgfqpoint{2.316959in}{2.339837in}}%
\pgfpathlineto{\pgfqpoint{2.360767in}{2.313886in}}%
\pgfpathlineto{\pgfqpoint{2.404574in}{2.283652in}}%
\pgfpathlineto{\pgfqpoint{2.448382in}{2.249093in}}%
\pgfpathlineto{\pgfqpoint{2.492189in}{2.210162in}}%
\pgfpathlineto{\pgfqpoint{2.535997in}{2.166798in}}%
\pgfpathlineto{\pgfqpoint{2.579804in}{2.118935in}}%
\pgfpathlineto{\pgfqpoint{2.623611in}{2.066498in}}%
\pgfpathlineto{\pgfqpoint{2.667419in}{2.009403in}}%
\pgfpathlineto{\pgfqpoint{2.711226in}{1.947561in}}%
\pgfpathlineto{\pgfqpoint{2.755034in}{1.880877in}}%
\pgfpathlineto{\pgfqpoint{2.798841in}{1.809275in}}%
\pgfpathlineto{\pgfqpoint{2.842649in}{1.732590in}}%
\pgfpathlineto{\pgfqpoint{2.886456in}{1.650724in}}%
\pgfpathlineto{\pgfqpoint{2.930264in}{1.563579in}}%
\pgfpathlineto{\pgfqpoint{2.974071in}{1.471057in}}%
\pgfpathlineto{\pgfqpoint{3.017879in}{1.373060in}}%
\pgfpathlineto{\pgfqpoint{3.061686in}{1.269489in}}%
\pgfpathlineto{\pgfqpoint{3.105494in}{1.160248in}}%
\pgfpathlineto{\pgfqpoint{3.149301in}{1.045237in}}%
\pgfpathlineto{\pgfqpoint{3.193109in}{0.924360in}}%
\pgfpathlineto{\pgfqpoint{3.236916in}{0.797517in}}%
\pgfusepath{stroke}%
\end{pgfscope}%
\begin{pgfscope}%
\pgfpathrectangle{\pgfqpoint{0.564660in}{0.521603in}}{\pgfqpoint{3.720000in}{3.020000in}}%
\pgfusepath{clip}%
\pgfsetrectcap%
\pgfsetroundjoin%
\pgfsetlinewidth{1.505625pt}%
\definecolor{currentstroke}{rgb}{0.131172,0.555899,0.552459}%
\pgfsetstrokecolor{currentstroke}%
\pgfsetdash{}{0pt}%
\pgfpathmoveto{\pgfqpoint{0.946461in}{0.834509in}}%
\pgfpathlineto{\pgfqpoint{0.988883in}{0.934924in}}%
\pgfpathlineto{\pgfqpoint{1.031305in}{1.030942in}}%
\pgfpathlineto{\pgfqpoint{1.073728in}{1.122608in}}%
\pgfpathlineto{\pgfqpoint{1.116150in}{1.209968in}}%
\pgfpathlineto{\pgfqpoint{1.158572in}{1.293067in}}%
\pgfpathlineto{\pgfqpoint{1.200994in}{1.371953in}}%
\pgfpathlineto{\pgfqpoint{1.243417in}{1.446670in}}%
\pgfpathlineto{\pgfqpoint{1.285839in}{1.517264in}}%
\pgfpathlineto{\pgfqpoint{1.328261in}{1.583780in}}%
\pgfpathlineto{\pgfqpoint{1.370683in}{1.646266in}}%
\pgfpathlineto{\pgfqpoint{1.413106in}{1.704766in}}%
\pgfpathlineto{\pgfqpoint{1.455528in}{1.759326in}}%
\pgfpathlineto{\pgfqpoint{1.497950in}{1.810015in}}%
\pgfpathlineto{\pgfqpoint{1.540372in}{1.856870in}}%
\pgfpathlineto{\pgfqpoint{1.582795in}{1.899937in}}%
\pgfpathlineto{\pgfqpoint{1.625217in}{1.939262in}}%
\pgfpathlineto{\pgfqpoint{1.667639in}{1.974887in}}%
\pgfpathlineto{\pgfqpoint{1.710062in}{2.006852in}}%
\pgfpathlineto{\pgfqpoint{1.752484in}{2.035195in}}%
\pgfpathlineto{\pgfqpoint{1.794906in}{2.059948in}}%
\pgfpathlineto{\pgfqpoint{1.837328in}{2.081141in}}%
\pgfpathlineto{\pgfqpoint{1.879751in}{2.098802in}}%
\pgfpathlineto{\pgfqpoint{1.922173in}{2.112950in}}%
\pgfpathlineto{\pgfqpoint{1.964595in}{2.123602in}}%
\pgfpathlineto{\pgfqpoint{2.007017in}{2.130770in}}%
\pgfpathlineto{\pgfqpoint{2.049440in}{2.134462in}}%
\pgfpathlineto{\pgfqpoint{2.091862in}{2.134679in}}%
\pgfpathlineto{\pgfqpoint{2.134284in}{2.131418in}}%
\pgfpathlineto{\pgfqpoint{2.176706in}{2.124673in}}%
\pgfpathlineto{\pgfqpoint{2.219129in}{2.114430in}}%
\pgfpathlineto{\pgfqpoint{2.261551in}{2.100668in}}%
\pgfpathlineto{\pgfqpoint{2.303973in}{2.083364in}}%
\pgfpathlineto{\pgfqpoint{2.346396in}{2.062487in}}%
\pgfpathlineto{\pgfqpoint{2.388818in}{2.037999in}}%
\pgfpathlineto{\pgfqpoint{2.431240in}{2.009850in}}%
\pgfpathlineto{\pgfqpoint{2.473662in}{1.977985in}}%
\pgfpathlineto{\pgfqpoint{2.516085in}{1.942336in}}%
\pgfpathlineto{\pgfqpoint{2.558507in}{1.902831in}}%
\pgfpathlineto{\pgfqpoint{2.600929in}{1.859388in}}%
\pgfpathlineto{\pgfqpoint{2.643351in}{1.811918in}}%
\pgfpathlineto{\pgfqpoint{2.685774in}{1.760326in}}%
\pgfpathlineto{\pgfqpoint{2.728196in}{1.704511in}}%
\pgfpathlineto{\pgfqpoint{2.770618in}{1.644390in}}%
\pgfpathlineto{\pgfqpoint{2.813041in}{1.579793in}}%
\pgfpathlineto{\pgfqpoint{2.855463in}{1.510617in}}%
\pgfpathlineto{\pgfqpoint{2.897885in}{1.436756in}}%
\pgfpathlineto{\pgfqpoint{2.940307in}{1.358108in}}%
\pgfpathlineto{\pgfqpoint{2.982730in}{1.274568in}}%
\pgfpathlineto{\pgfqpoint{3.025152in}{1.186032in}}%
\pgfpathlineto{\pgfqpoint{3.067574in}{1.092395in}}%
\pgfpathlineto{\pgfqpoint{3.109996in}{0.993555in}}%
\pgfpathlineto{\pgfqpoint{3.152419in}{0.889407in}}%
\pgfpathlineto{\pgfqpoint{3.194841in}{0.779846in}}%
\pgfusepath{stroke}%
\end{pgfscope}%
\begin{pgfscope}%
\pgfpathrectangle{\pgfqpoint{0.564660in}{0.521603in}}{\pgfqpoint{3.720000in}{3.020000in}}%
\pgfusepath{clip}%
\pgfsetrectcap%
\pgfsetroundjoin%
\pgfsetlinewidth{1.505625pt}%
\definecolor{currentstroke}{rgb}{0.192357,0.403199,0.555836}%
\pgfsetstrokecolor{currentstroke}%
\pgfsetdash{}{0pt}%
\pgfpathmoveto{\pgfqpoint{0.880673in}{1.105364in}}%
\pgfpathlineto{\pgfqpoint{0.920175in}{1.199183in}}%
\pgfpathlineto{\pgfqpoint{0.959676in}{1.289171in}}%
\pgfpathlineto{\pgfqpoint{0.999178in}{1.375366in}}%
\pgfpathlineto{\pgfqpoint{1.038679in}{1.457803in}}%
\pgfpathlineto{\pgfqpoint{1.078181in}{1.536521in}}%
\pgfpathlineto{\pgfqpoint{1.117683in}{1.611554in}}%
\pgfpathlineto{\pgfqpoint{1.157184in}{1.682940in}}%
\pgfpathlineto{\pgfqpoint{1.196686in}{1.750715in}}%
\pgfpathlineto{\pgfqpoint{1.236187in}{1.814915in}}%
\pgfpathlineto{\pgfqpoint{1.275689in}{1.875578in}}%
\pgfpathlineto{\pgfqpoint{1.315191in}{1.932740in}}%
\pgfpathlineto{\pgfqpoint{1.354692in}{1.986437in}}%
\pgfpathlineto{\pgfqpoint{1.394194in}{2.036717in}}%
\pgfpathlineto{\pgfqpoint{1.433695in}{2.083612in}}%
\pgfpathlineto{\pgfqpoint{1.473197in}{2.127160in}}%
\pgfpathlineto{\pgfqpoint{1.512698in}{2.167399in}}%
\pgfpathlineto{\pgfqpoint{1.552200in}{2.204364in}}%
\pgfpathlineto{\pgfqpoint{1.591702in}{2.238089in}}%
\pgfpathlineto{\pgfqpoint{1.631203in}{2.268605in}}%
\pgfpathlineto{\pgfqpoint{1.670705in}{2.295944in}}%
\pgfpathlineto{\pgfqpoint{1.710206in}{2.320132in}}%
\pgfpathlineto{\pgfqpoint{1.749708in}{2.341194in}}%
\pgfpathlineto{\pgfqpoint{1.789210in}{2.359153in}}%
\pgfpathlineto{\pgfqpoint{1.828711in}{2.374028in}}%
\pgfpathlineto{\pgfqpoint{1.868213in}{2.385835in}}%
\pgfpathlineto{\pgfqpoint{1.907714in}{2.394588in}}%
\pgfpathlineto{\pgfqpoint{1.947216in}{2.400298in}}%
\pgfpathlineto{\pgfqpoint{1.986718in}{2.402974in}}%
\pgfpathlineto{\pgfqpoint{2.026219in}{2.402622in}}%
\pgfpathlineto{\pgfqpoint{2.065721in}{2.399247in}}%
\pgfpathlineto{\pgfqpoint{2.105222in}{2.392852in}}%
\pgfpathlineto{\pgfqpoint{2.144724in}{2.383435in}}%
\pgfpathlineto{\pgfqpoint{2.184226in}{2.370997in}}%
\pgfpathlineto{\pgfqpoint{2.223727in}{2.355533in}}%
\pgfpathlineto{\pgfqpoint{2.263229in}{2.337035in}}%
\pgfpathlineto{\pgfqpoint{2.302730in}{2.315494in}}%
\pgfpathlineto{\pgfqpoint{2.342232in}{2.290894in}}%
\pgfpathlineto{\pgfqpoint{2.381734in}{2.263219in}}%
\pgfpathlineto{\pgfqpoint{2.421235in}{2.232447in}}%
\pgfpathlineto{\pgfqpoint{2.460737in}{2.198548in}}%
\pgfpathlineto{\pgfqpoint{2.500238in}{2.161482in}}%
\pgfpathlineto{\pgfqpoint{2.539740in}{2.121202in}}%
\pgfpathlineto{\pgfqpoint{2.579241in}{2.077655in}}%
\pgfpathlineto{\pgfqpoint{2.618743in}{2.030778in}}%
\pgfpathlineto{\pgfqpoint{2.658245in}{1.980499in}}%
\pgfpathlineto{\pgfqpoint{2.697746in}{1.926742in}}%
\pgfpathlineto{\pgfqpoint{2.737248in}{1.869423in}}%
\pgfpathlineto{\pgfqpoint{2.776749in}{1.808450in}}%
\pgfpathlineto{\pgfqpoint{2.816251in}{1.743754in}}%
\pgfpathlineto{\pgfqpoint{2.855753in}{1.675176in}}%
\pgfpathlineto{\pgfqpoint{2.895254in}{1.602623in}}%
\pgfpathlineto{\pgfqpoint{2.934756in}{1.526002in}}%
\pgfpathlineto{\pgfqpoint{2.974257in}{1.445220in}}%
\pgfpathlineto{\pgfqpoint{3.013759in}{1.360184in}}%
\pgfpathlineto{\pgfqpoint{3.053261in}{1.270802in}}%
\pgfpathlineto{\pgfqpoint{3.092762in}{1.176979in}}%
\pgfpathlineto{\pgfqpoint{3.132264in}{1.078623in}}%
\pgfpathlineto{\pgfqpoint{3.171765in}{0.975641in}}%
\pgfpathlineto{\pgfqpoint{3.211267in}{0.867940in}}%
\pgfpathlineto{\pgfqpoint{3.250769in}{0.755427in}}%
\pgfusepath{stroke}%
\end{pgfscope}%
\begin{pgfscope}%
\pgfpathrectangle{\pgfqpoint{0.564660in}{0.521603in}}{\pgfqpoint{3.720000in}{3.020000in}}%
\pgfusepath{clip}%
\pgfsetrectcap%
\pgfsetroundjoin%
\pgfsetlinewidth{1.505625pt}%
\definecolor{currentstroke}{rgb}{0.204903,0.375746,0.553533}%
\pgfsetstrokecolor{currentstroke}%
\pgfsetdash{}{0pt}%
\pgfpathmoveto{\pgfqpoint{0.777183in}{0.903667in}}%
\pgfpathlineto{\pgfqpoint{0.824410in}{1.028543in}}%
\pgfpathlineto{\pgfqpoint{0.871637in}{1.144664in}}%
\pgfpathlineto{\pgfqpoint{0.918864in}{1.252614in}}%
\pgfpathlineto{\pgfqpoint{0.966091in}{1.352977in}}%
\pgfpathlineto{\pgfqpoint{1.013319in}{1.446337in}}%
\pgfpathlineto{\pgfqpoint{1.060546in}{1.533277in}}%
\pgfpathlineto{\pgfqpoint{1.107773in}{1.614526in}}%
\pgfpathlineto{\pgfqpoint{1.155000in}{1.690645in}}%
\pgfpathlineto{\pgfqpoint{1.202227in}{1.762123in}}%
\pgfpathlineto{\pgfqpoint{1.249455in}{1.829368in}}%
\pgfpathlineto{\pgfqpoint{1.296682in}{1.892688in}}%
\pgfpathlineto{\pgfqpoint{1.343909in}{1.952300in}}%
\pgfpathlineto{\pgfqpoint{1.391136in}{2.008353in}}%
\pgfpathlineto{\pgfqpoint{1.438363in}{2.060952in}}%
\pgfpathlineto{\pgfqpoint{1.485591in}{2.110180in}}%
\pgfpathlineto{\pgfqpoint{1.532818in}{2.156125in}}%
\pgfpathlineto{\pgfqpoint{1.580045in}{2.198865in}}%
\pgfpathlineto{\pgfqpoint{1.627272in}{2.238491in}}%
\pgfpathlineto{\pgfqpoint{1.674499in}{2.275099in}}%
\pgfpathlineto{\pgfqpoint{1.721727in}{2.308759in}}%
\pgfpathlineto{\pgfqpoint{1.768954in}{2.339537in}}%
\pgfpathlineto{\pgfqpoint{1.816181in}{2.367501in}}%
\pgfpathlineto{\pgfqpoint{1.863408in}{2.392699in}}%
\pgfpathlineto{\pgfqpoint{1.910635in}{2.415187in}}%
\pgfpathlineto{\pgfqpoint{1.957863in}{2.435022in}}%
\pgfpathlineto{\pgfqpoint{2.005090in}{2.452236in}}%
\pgfpathlineto{\pgfqpoint{2.052317in}{2.466867in}}%
\pgfpathlineto{\pgfqpoint{2.099544in}{2.478946in}}%
\pgfpathlineto{\pgfqpoint{2.146771in}{2.488493in}}%
\pgfpathlineto{\pgfqpoint{2.193998in}{2.495539in}}%
\pgfpathlineto{\pgfqpoint{2.241226in}{2.500111in}}%
\pgfpathlineto{\pgfqpoint{2.288453in}{2.502214in}}%
\pgfpathlineto{\pgfqpoint{2.335680in}{2.501856in}}%
\pgfpathlineto{\pgfqpoint{2.382907in}{2.499038in}}%
\pgfpathlineto{\pgfqpoint{2.430134in}{2.493748in}}%
\pgfpathlineto{\pgfqpoint{2.477362in}{2.485987in}}%
\pgfpathlineto{\pgfqpoint{2.524589in}{2.475751in}}%
\pgfpathlineto{\pgfqpoint{2.571816in}{2.463015in}}%
\pgfpathlineto{\pgfqpoint{2.619043in}{2.447756in}}%
\pgfpathlineto{\pgfqpoint{2.666270in}{2.429945in}}%
\pgfpathlineto{\pgfqpoint{2.713498in}{2.409538in}}%
\pgfpathlineto{\pgfqpoint{2.760725in}{2.386510in}}%
\pgfpathlineto{\pgfqpoint{2.807952in}{2.360822in}}%
\pgfpathlineto{\pgfqpoint{2.855179in}{2.332412in}}%
\pgfpathlineto{\pgfqpoint{2.902406in}{2.301223in}}%
\pgfpathlineto{\pgfqpoint{2.949634in}{2.267176in}}%
\pgfpathlineto{\pgfqpoint{2.996861in}{2.230185in}}%
\pgfpathlineto{\pgfqpoint{3.044088in}{2.190174in}}%
\pgfpathlineto{\pgfqpoint{3.091315in}{2.147045in}}%
\pgfpathlineto{\pgfqpoint{3.138542in}{2.100688in}}%
\pgfpathlineto{\pgfqpoint{3.185770in}{2.050993in}}%
\pgfpathlineto{\pgfqpoint{3.232997in}{1.997837in}}%
\pgfpathlineto{\pgfqpoint{3.280224in}{1.941090in}}%
\pgfpathlineto{\pgfqpoint{3.327451in}{1.880616in}}%
\pgfpathlineto{\pgfqpoint{3.374678in}{1.816238in}}%
\pgfpathlineto{\pgfqpoint{3.421906in}{1.747737in}}%
\pgfpathlineto{\pgfqpoint{3.469133in}{1.674840in}}%
\pgfpathlineto{\pgfqpoint{3.516360in}{1.597184in}}%
\pgfpathlineto{\pgfqpoint{3.563587in}{1.514330in}}%
\pgfpathlineto{\pgfqpoint{3.610814in}{1.425765in}}%
\pgfpathlineto{\pgfqpoint{3.658041in}{1.330913in}}%
\pgfpathlineto{\pgfqpoint{3.705269in}{1.229080in}}%
\pgfpathlineto{\pgfqpoint{3.752496in}{1.119610in}}%
\pgfpathlineto{\pgfqpoint{3.799723in}{1.001890in}}%
\pgfpathlineto{\pgfqpoint{3.846950in}{0.875303in}}%
\pgfpathlineto{\pgfqpoint{3.894177in}{0.739236in}}%
\pgfpathlineto{\pgfqpoint{3.894177in}{0.739236in}}%
\pgfusepath{stroke}%
\end{pgfscope}%
\begin{pgfscope}%
\pgfpathrectangle{\pgfqpoint{0.564660in}{0.521603in}}{\pgfqpoint{3.720000in}{3.020000in}}%
\pgfusepath{clip}%
\pgfsetrectcap%
\pgfsetroundjoin%
\pgfsetlinewidth{1.505625pt}%
\definecolor{currentstroke}{rgb}{0.206756,0.371758,0.553117}%
\pgfsetstrokecolor{currentstroke}%
\pgfsetdash{}{0pt}%
\pgfpathmoveto{\pgfqpoint{0.760917in}{0.842885in}}%
\pgfpathlineto{\pgfqpoint{0.788954in}{0.915650in}}%
\pgfpathlineto{\pgfqpoint{0.816990in}{0.986020in}}%
\pgfpathlineto{\pgfqpoint{0.845027in}{1.054055in}}%
\pgfpathlineto{\pgfqpoint{0.873064in}{1.119814in}}%
\pgfpathlineto{\pgfqpoint{0.901100in}{1.183358in}}%
\pgfpathlineto{\pgfqpoint{0.929137in}{1.244745in}}%
\pgfpathlineto{\pgfqpoint{0.957174in}{1.304035in}}%
\pgfpathlineto{\pgfqpoint{0.985210in}{1.361287in}}%
\pgfpathlineto{\pgfqpoint{1.013247in}{1.416561in}}%
\pgfpathlineto{\pgfqpoint{1.041283in}{1.469916in}}%
\pgfpathlineto{\pgfqpoint{1.069320in}{1.521412in}}%
\pgfpathlineto{\pgfqpoint{1.097357in}{1.571108in}}%
\pgfpathlineto{\pgfqpoint{1.125393in}{1.619107in}}%
\pgfpathlineto{\pgfqpoint{1.153430in}{1.665454in}}%
\pgfpathlineto{\pgfqpoint{1.181467in}{1.710206in}}%
\pgfpathlineto{\pgfqpoint{1.209503in}{1.753415in}}%
\pgfpathlineto{\pgfqpoint{1.237540in}{1.795130in}}%
\pgfpathlineto{\pgfqpoint{1.265577in}{1.835393in}}%
\pgfpathlineto{\pgfqpoint{1.293613in}{1.874245in}}%
\pgfpathlineto{\pgfqpoint{1.321650in}{1.911718in}}%
\pgfpathlineto{\pgfqpoint{1.349687in}{1.947842in}}%
\pgfpathlineto{\pgfqpoint{1.377723in}{1.982638in}}%
\pgfpathlineto{\pgfqpoint{1.405760in}{2.016128in}}%
\pgfpathlineto{\pgfqpoint{1.433797in}{2.048325in}}%
\pgfpathlineto{\pgfqpoint{1.461833in}{2.079241in}}%
\pgfpathlineto{\pgfqpoint{1.489870in}{2.108889in}}%
\pgfpathlineto{\pgfqpoint{1.517907in}{2.137277in}}%
\pgfpathlineto{\pgfqpoint{1.545943in}{2.164419in}}%
\pgfpathlineto{\pgfqpoint{1.573980in}{2.190323in}}%
\pgfpathlineto{\pgfqpoint{1.602017in}{2.215000in}}%
\pgfpathlineto{\pgfqpoint{1.630053in}{2.238457in}}%
\pgfpathlineto{\pgfqpoint{1.658090in}{2.260704in}}%
\pgfpathlineto{\pgfqpoint{1.686127in}{2.281750in}}%
\pgfpathlineto{\pgfqpoint{1.714163in}{2.301607in}}%
\pgfpathlineto{\pgfqpoint{1.742200in}{2.320281in}}%
\pgfpathlineto{\pgfqpoint{1.770237in}{2.337783in}}%
\pgfpathlineto{\pgfqpoint{1.798273in}{2.354122in}}%
\pgfpathlineto{\pgfqpoint{1.826310in}{2.369308in}}%
\pgfpathlineto{\pgfqpoint{1.854347in}{2.383351in}}%
\pgfpathlineto{\pgfqpoint{1.882383in}{2.396257in}}%
\pgfpathlineto{\pgfqpoint{1.910420in}{2.408035in}}%
\pgfpathlineto{\pgfqpoint{1.938457in}{2.418693in}}%
\pgfpathlineto{\pgfqpoint{1.966493in}{2.428238in}}%
\pgfpathlineto{\pgfqpoint{1.994530in}{2.436674in}}%
\pgfpathlineto{\pgfqpoint{2.022567in}{2.444005in}}%
\pgfpathlineto{\pgfqpoint{2.050603in}{2.450235in}}%
\pgfpathlineto{\pgfqpoint{2.078640in}{2.455370in}}%
\pgfpathlineto{\pgfqpoint{2.106677in}{2.459411in}}%
\pgfpathlineto{\pgfqpoint{2.134713in}{2.462360in}}%
\pgfpathlineto{\pgfqpoint{2.162750in}{2.464218in}}%
\pgfpathlineto{\pgfqpoint{2.190787in}{2.464987in}}%
\pgfpathlineto{\pgfqpoint{2.218823in}{2.464668in}}%
\pgfpathlineto{\pgfqpoint{2.246860in}{2.463261in}}%
\pgfpathlineto{\pgfqpoint{2.274897in}{2.460761in}}%
\pgfpathlineto{\pgfqpoint{2.302933in}{2.457169in}}%
\pgfpathlineto{\pgfqpoint{2.330970in}{2.452482in}}%
\pgfpathlineto{\pgfqpoint{2.359006in}{2.446698in}}%
\pgfpathlineto{\pgfqpoint{2.387043in}{2.439814in}}%
\pgfpathlineto{\pgfqpoint{2.415080in}{2.431826in}}%
\pgfpathlineto{\pgfqpoint{2.443116in}{2.422731in}}%
\pgfpathlineto{\pgfqpoint{2.471153in}{2.412527in}}%
\pgfpathlineto{\pgfqpoint{2.499190in}{2.401211in}}%
\pgfpathlineto{\pgfqpoint{2.527226in}{2.388777in}}%
\pgfpathlineto{\pgfqpoint{2.555263in}{2.375221in}}%
\pgfpathlineto{\pgfqpoint{2.583300in}{2.360538in}}%
\pgfpathlineto{\pgfqpoint{2.611336in}{2.344725in}}%
\pgfpathlineto{\pgfqpoint{2.639373in}{2.327775in}}%
\pgfpathlineto{\pgfqpoint{2.667410in}{2.309681in}}%
\pgfpathlineto{\pgfqpoint{2.695446in}{2.290436in}}%
\pgfpathlineto{\pgfqpoint{2.723483in}{2.270033in}}%
\pgfpathlineto{\pgfqpoint{2.751520in}{2.248463in}}%
\pgfpathlineto{\pgfqpoint{2.779556in}{2.225716in}}%
\pgfpathlineto{\pgfqpoint{2.807593in}{2.201781in}}%
\pgfpathlineto{\pgfqpoint{2.835630in}{2.176646in}}%
\pgfpathlineto{\pgfqpoint{2.863666in}{2.150300in}}%
\pgfpathlineto{\pgfqpoint{2.891703in}{2.122729in}}%
\pgfpathlineto{\pgfqpoint{2.919740in}{2.093919in}}%
\pgfpathlineto{\pgfqpoint{2.947776in}{2.063854in}}%
\pgfpathlineto{\pgfqpoint{2.975813in}{2.032517in}}%
\pgfpathlineto{\pgfqpoint{3.003850in}{1.999893in}}%
\pgfpathlineto{\pgfqpoint{3.031886in}{1.965961in}}%
\pgfpathlineto{\pgfqpoint{3.059923in}{1.930700in}}%
\pgfpathlineto{\pgfqpoint{3.087960in}{1.894087in}}%
\pgfpathlineto{\pgfqpoint{3.115996in}{1.856099in}}%
\pgfpathlineto{\pgfqpoint{3.144033in}{1.816708in}}%
\pgfpathlineto{\pgfqpoint{3.172070in}{1.775884in}}%
\pgfpathlineto{\pgfqpoint{3.200106in}{1.733591in}}%
\pgfpathlineto{\pgfqpoint{3.228143in}{1.689794in}}%
\pgfpathlineto{\pgfqpoint{3.256180in}{1.644451in}}%
\pgfpathlineto{\pgfqpoint{3.284216in}{1.597521in}}%
\pgfpathlineto{\pgfqpoint{3.312253in}{1.548956in}}%
\pgfpathlineto{\pgfqpoint{3.340290in}{1.498708in}}%
\pgfpathlineto{\pgfqpoint{3.368326in}{1.446725in}}%
\pgfpathlineto{\pgfqpoint{3.396363in}{1.392964in}}%
\pgfpathlineto{\pgfqpoint{3.424400in}{1.337346in}}%
\pgfpathlineto{\pgfqpoint{3.452436in}{1.279820in}}%
\pgfpathlineto{\pgfqpoint{3.480473in}{1.220332in}}%
\pgfpathlineto{\pgfqpoint{3.508510in}{1.158830in}}%
\pgfpathlineto{\pgfqpoint{3.536546in}{1.095262in}}%
\pgfpathlineto{\pgfqpoint{3.564583in}{1.029575in}}%
\pgfpathlineto{\pgfqpoint{3.592620in}{0.961718in}}%
\pgfpathlineto{\pgfqpoint{3.620656in}{0.891637in}}%
\pgfpathlineto{\pgfqpoint{3.648693in}{0.819280in}}%
\pgfpathlineto{\pgfqpoint{3.676729in}{0.744596in}}%
\pgfusepath{stroke}%
\end{pgfscope}%
\begin{pgfscope}%
\pgfpathrectangle{\pgfqpoint{0.564660in}{0.521603in}}{\pgfqpoint{3.720000in}{3.020000in}}%
\pgfusepath{clip}%
\pgfsetrectcap%
\pgfsetroundjoin%
\pgfsetlinewidth{1.505625pt}%
\definecolor{currentstroke}{rgb}{0.282656,0.100196,0.422160}%
\pgfsetstrokecolor{currentstroke}%
\pgfsetdash{}{0pt}%
\pgfpathmoveto{\pgfqpoint{0.712990in}{0.794843in}}%
\pgfpathlineto{\pgfqpoint{0.768614in}{0.924476in}}%
\pgfpathlineto{\pgfqpoint{0.824238in}{1.046999in}}%
\pgfpathlineto{\pgfqpoint{0.879862in}{1.162599in}}%
\pgfpathlineto{\pgfqpoint{0.935485in}{1.271466in}}%
\pgfpathlineto{\pgfqpoint{0.991109in}{1.373802in}}%
\pgfpathlineto{\pgfqpoint{1.046733in}{1.469785in}}%
\pgfpathlineto{\pgfqpoint{1.102357in}{1.559566in}}%
\pgfpathlineto{\pgfqpoint{1.157980in}{1.643285in}}%
\pgfpathlineto{\pgfqpoint{1.213604in}{1.721089in}}%
\pgfpathlineto{\pgfqpoint{1.269228in}{1.793159in}}%
\pgfpathlineto{\pgfqpoint{1.324852in}{1.859702in}}%
\pgfpathlineto{\pgfqpoint{1.380475in}{1.920924in}}%
\pgfpathlineto{\pgfqpoint{1.436099in}{1.977014in}}%
\pgfpathlineto{\pgfqpoint{1.491723in}{2.028122in}}%
\pgfpathlineto{\pgfqpoint{1.547347in}{2.074359in}}%
\pgfpathlineto{\pgfqpoint{1.602970in}{2.115802in}}%
\pgfpathlineto{\pgfqpoint{1.658594in}{2.152519in}}%
\pgfpathlineto{\pgfqpoint{1.714218in}{2.184578in}}%
\pgfpathlineto{\pgfqpoint{1.769842in}{2.212052in}}%
\pgfpathlineto{\pgfqpoint{1.825465in}{2.235017in}}%
\pgfpathlineto{\pgfqpoint{1.881089in}{2.253552in}}%
\pgfpathlineto{\pgfqpoint{1.936713in}{2.267724in}}%
\pgfpathlineto{\pgfqpoint{1.992337in}{2.277577in}}%
\pgfpathlineto{\pgfqpoint{2.047960in}{2.283134in}}%
\pgfpathlineto{\pgfqpoint{2.103584in}{2.284398in}}%
\pgfpathlineto{\pgfqpoint{2.159208in}{2.281353in}}%
\pgfpathlineto{\pgfqpoint{2.214832in}{2.273974in}}%
\pgfpathlineto{\pgfqpoint{2.270455in}{2.262231in}}%
\pgfpathlineto{\pgfqpoint{2.326079in}{2.246084in}}%
\pgfpathlineto{\pgfqpoint{2.381703in}{2.225476in}}%
\pgfpathlineto{\pgfqpoint{2.437327in}{2.200336in}}%
\pgfpathlineto{\pgfqpoint{2.492950in}{2.170586in}}%
\pgfpathlineto{\pgfqpoint{2.548574in}{2.136148in}}%
\pgfpathlineto{\pgfqpoint{2.604198in}{2.096952in}}%
\pgfpathlineto{\pgfqpoint{2.659821in}{2.052936in}}%
\pgfpathlineto{\pgfqpoint{2.715445in}{2.004040in}}%
\pgfpathlineto{\pgfqpoint{2.771069in}{1.950187in}}%
\pgfpathlineto{\pgfqpoint{2.826693in}{1.891268in}}%
\pgfpathlineto{\pgfqpoint{2.882316in}{1.827141in}}%
\pgfpathlineto{\pgfqpoint{2.937940in}{1.757638in}}%
\pgfpathlineto{\pgfqpoint{2.993564in}{1.682578in}}%
\pgfpathlineto{\pgfqpoint{3.049188in}{1.601776in}}%
\pgfpathlineto{\pgfqpoint{3.104811in}{1.515043in}}%
\pgfpathlineto{\pgfqpoint{3.160435in}{1.422172in}}%
\pgfpathlineto{\pgfqpoint{3.216059in}{1.322907in}}%
\pgfpathlineto{\pgfqpoint{3.271683in}{1.216926in}}%
\pgfpathlineto{\pgfqpoint{3.327306in}{1.103849in}}%
\pgfpathlineto{\pgfqpoint{3.382930in}{0.983232in}}%
\pgfpathlineto{\pgfqpoint{3.438554in}{0.854656in}}%
\pgfpathlineto{\pgfqpoint{3.494178in}{0.717710in}}%
\pgfpathlineto{\pgfqpoint{3.512719in}{0.670130in}}%
\pgfpathlineto{\pgfqpoint{3.512719in}{0.670130in}}%
\pgfusepath{stroke}%
\end{pgfscope}%
\begin{pgfscope}%
\pgfsetrectcap%
\pgfsetmiterjoin%
\pgfsetlinewidth{0.803000pt}%
\definecolor{currentstroke}{rgb}{0.501961,0.501961,0.501961}%
\pgfsetstrokecolor{currentstroke}%
\pgfsetdash{}{0pt}%
\pgfpathmoveto{\pgfqpoint{0.564660in}{0.521603in}}%
\pgfpathlineto{\pgfqpoint{0.564660in}{3.541603in}}%
\pgfusepath{stroke}%
\end{pgfscope}%
\begin{pgfscope}%
\pgfsetrectcap%
\pgfsetmiterjoin%
\pgfsetlinewidth{0.803000pt}%
\definecolor{currentstroke}{rgb}{0.501961,0.501961,0.501961}%
\pgfsetstrokecolor{currentstroke}%
\pgfsetdash{}{0pt}%
\pgfpathmoveto{\pgfqpoint{4.284660in}{0.521603in}}%
\pgfpathlineto{\pgfqpoint{4.284660in}{3.541603in}}%
\pgfusepath{stroke}%
\end{pgfscope}%
\begin{pgfscope}%
\pgfsetrectcap%
\pgfsetmiterjoin%
\pgfsetlinewidth{0.803000pt}%
\definecolor{currentstroke}{rgb}{0.501961,0.501961,0.501961}%
\pgfsetstrokecolor{currentstroke}%
\pgfsetdash{}{0pt}%
\pgfpathmoveto{\pgfqpoint{0.564660in}{0.521603in}}%
\pgfpathlineto{\pgfqpoint{4.284660in}{0.521603in}}%
\pgfusepath{stroke}%
\end{pgfscope}%
\begin{pgfscope}%
\pgfsetrectcap%
\pgfsetmiterjoin%
\pgfsetlinewidth{0.803000pt}%
\definecolor{currentstroke}{rgb}{0.501961,0.501961,0.501961}%
\pgfsetstrokecolor{currentstroke}%
\pgfsetdash{}{0pt}%
\pgfpathmoveto{\pgfqpoint{0.564660in}{3.541603in}}%
\pgfpathlineto{\pgfqpoint{4.284660in}{3.541603in}}%
\pgfusepath{stroke}%
\end{pgfscope}%
\begin{pgfscope}%
\pgfpathrectangle{\pgfqpoint{4.517160in}{0.521603in}}{\pgfqpoint{0.151000in}{3.020000in}}%
\pgfusepath{clip}%
\pgfsetbuttcap%
\pgfsetmiterjoin%
\definecolor{currentfill}{rgb}{1.000000,1.000000,1.000000}%
\pgfsetfillcolor{currentfill}%
\pgfsetlinewidth{0.010037pt}%
\definecolor{currentstroke}{rgb}{1.000000,1.000000,1.000000}%
\pgfsetstrokecolor{currentstroke}%
\pgfsetdash{}{0pt}%
\pgfpathmoveto{\pgfqpoint{4.517160in}{0.521603in}}%
\pgfpathlineto{\pgfqpoint{4.517160in}{0.533400in}}%
\pgfpathlineto{\pgfqpoint{4.517160in}{3.529806in}}%
\pgfpathlineto{\pgfqpoint{4.517160in}{3.541603in}}%
\pgfpathlineto{\pgfqpoint{4.668160in}{3.541603in}}%
\pgfpathlineto{\pgfqpoint{4.668160in}{3.529806in}}%
\pgfpathlineto{\pgfqpoint{4.668160in}{0.533400in}}%
\pgfpathlineto{\pgfqpoint{4.668160in}{0.521603in}}%
\pgfpathclose%
\pgfusepath{stroke,fill}%
\end{pgfscope}%
\begin{pgfscope}%
\pgfsys@transformshift{4.513889in}{0.537437in}%
\pgftext[left,bottom]{\pgfimage[interpolate=true,width=0.152778in,height=3.013889in]{series_s_ds-img0.png}}%
\end{pgfscope}%
\begin{pgfscope}%
\pgfsetbuttcap%
\pgfsetroundjoin%
\definecolor{currentfill}{rgb}{0.000000,0.000000,0.000000}%
\pgfsetfillcolor{currentfill}%
\pgfsetlinewidth{0.803000pt}%
\definecolor{currentstroke}{rgb}{0.000000,0.000000,0.000000}%
\pgfsetstrokecolor{currentstroke}%
\pgfsetdash{}{0pt}%
\pgfsys@defobject{currentmarker}{\pgfqpoint{0.000000in}{0.000000in}}{\pgfqpoint{0.048611in}{0.000000in}}{%
\pgfpathmoveto{\pgfqpoint{0.000000in}{0.000000in}}%
\pgfpathlineto{\pgfqpoint{0.048611in}{0.000000in}}%
\pgfusepath{stroke,fill}%
}%
\begin{pgfscope}%
\pgfsys@transformshift{4.668160in}{0.687294in}%
\pgfsys@useobject{currentmarker}{}%
\end{pgfscope}%
\end{pgfscope}%
\begin{pgfscope}%
\definecolor{textcolor}{rgb}{0.000000,0.000000,0.000000}%
\pgfsetstrokecolor{textcolor}%
\pgfsetfillcolor{textcolor}%
\pgftext[x=4.765383in,y=0.634533in,left,base]{\color{textcolor}\rmfamily\fontsize{10.000000}{12.000000}\selectfont \(\displaystyle 10^{0}\)}%
\end{pgfscope}%
\begin{pgfscope}%
\pgfsetbuttcap%
\pgfsetroundjoin%
\definecolor{currentfill}{rgb}{0.000000,0.000000,0.000000}%
\pgfsetfillcolor{currentfill}%
\pgfsetlinewidth{0.803000pt}%
\definecolor{currentstroke}{rgb}{0.000000,0.000000,0.000000}%
\pgfsetstrokecolor{currentstroke}%
\pgfsetdash{}{0pt}%
\pgfsys@defobject{currentmarker}{\pgfqpoint{0.000000in}{0.000000in}}{\pgfqpoint{0.048611in}{0.000000in}}{%
\pgfpathmoveto{\pgfqpoint{0.000000in}{0.000000in}}%
\pgfpathlineto{\pgfqpoint{0.048611in}{0.000000in}}%
\pgfusepath{stroke,fill}%
}%
\begin{pgfscope}%
\pgfsys@transformshift{4.668160in}{3.444394in}%
\pgfsys@useobject{currentmarker}{}%
\end{pgfscope}%
\end{pgfscope}%
\begin{pgfscope}%
\definecolor{textcolor}{rgb}{0.000000,0.000000,0.000000}%
\pgfsetstrokecolor{textcolor}%
\pgfsetfillcolor{textcolor}%
\pgftext[x=4.765383in,y=3.391633in,left,base]{\color{textcolor}\rmfamily\fontsize{10.000000}{12.000000}\selectfont \(\displaystyle 10^{1}\)}%
\end{pgfscope}%
\begin{pgfscope}%
\pgfsetbuttcap%
\pgfsetroundjoin%
\definecolor{currentfill}{rgb}{0.000000,0.000000,0.000000}%
\pgfsetfillcolor{currentfill}%
\pgfsetlinewidth{0.602250pt}%
\definecolor{currentstroke}{rgb}{0.000000,0.000000,0.000000}%
\pgfsetstrokecolor{currentstroke}%
\pgfsetdash{}{0pt}%
\pgfsys@defobject{currentmarker}{\pgfqpoint{0.000000in}{0.000000in}}{\pgfqpoint{0.027778in}{0.000000in}}{%
\pgfpathmoveto{\pgfqpoint{0.000000in}{0.000000in}}%
\pgfpathlineto{\pgfqpoint{0.027778in}{0.000000in}}%
\pgfusepath{stroke,fill}%
}%
\begin{pgfscope}%
\pgfsys@transformshift{4.668160in}{0.561136in}%
\pgfsys@useobject{currentmarker}{}%
\end{pgfscope}%
\end{pgfscope}%
\begin{pgfscope}%
\pgfsetbuttcap%
\pgfsetroundjoin%
\definecolor{currentfill}{rgb}{0.000000,0.000000,0.000000}%
\pgfsetfillcolor{currentfill}%
\pgfsetlinewidth{0.602250pt}%
\definecolor{currentstroke}{rgb}{0.000000,0.000000,0.000000}%
\pgfsetstrokecolor{currentstroke}%
\pgfsetdash{}{0pt}%
\pgfsys@defobject{currentmarker}{\pgfqpoint{0.000000in}{0.000000in}}{\pgfqpoint{0.027778in}{0.000000in}}{%
\pgfpathmoveto{\pgfqpoint{0.000000in}{0.000000in}}%
\pgfpathlineto{\pgfqpoint{0.027778in}{0.000000in}}%
\pgfusepath{stroke,fill}%
}%
\begin{pgfscope}%
\pgfsys@transformshift{4.668160in}{1.517264in}%
\pgfsys@useobject{currentmarker}{}%
\end{pgfscope}%
\end{pgfscope}%
\begin{pgfscope}%
\pgfsetbuttcap%
\pgfsetroundjoin%
\definecolor{currentfill}{rgb}{0.000000,0.000000,0.000000}%
\pgfsetfillcolor{currentfill}%
\pgfsetlinewidth{0.602250pt}%
\definecolor{currentstroke}{rgb}{0.000000,0.000000,0.000000}%
\pgfsetstrokecolor{currentstroke}%
\pgfsetdash{}{0pt}%
\pgfsys@defobject{currentmarker}{\pgfqpoint{0.000000in}{0.000000in}}{\pgfqpoint{0.027778in}{0.000000in}}{%
\pgfpathmoveto{\pgfqpoint{0.000000in}{0.000000in}}%
\pgfpathlineto{\pgfqpoint{0.027778in}{0.000000in}}%
\pgfusepath{stroke,fill}%
}%
\begin{pgfscope}%
\pgfsys@transformshift{4.668160in}{2.002765in}%
\pgfsys@useobject{currentmarker}{}%
\end{pgfscope}%
\end{pgfscope}%
\begin{pgfscope}%
\pgfsetbuttcap%
\pgfsetroundjoin%
\definecolor{currentfill}{rgb}{0.000000,0.000000,0.000000}%
\pgfsetfillcolor{currentfill}%
\pgfsetlinewidth{0.602250pt}%
\definecolor{currentstroke}{rgb}{0.000000,0.000000,0.000000}%
\pgfsetstrokecolor{currentstroke}%
\pgfsetdash{}{0pt}%
\pgfsys@defobject{currentmarker}{\pgfqpoint{0.000000in}{0.000000in}}{\pgfqpoint{0.027778in}{0.000000in}}{%
\pgfpathmoveto{\pgfqpoint{0.000000in}{0.000000in}}%
\pgfpathlineto{\pgfqpoint{0.027778in}{0.000000in}}%
\pgfusepath{stroke,fill}%
}%
\begin{pgfscope}%
\pgfsys@transformshift{4.668160in}{2.347234in}%
\pgfsys@useobject{currentmarker}{}%
\end{pgfscope}%
\end{pgfscope}%
\begin{pgfscope}%
\pgfsetbuttcap%
\pgfsetroundjoin%
\definecolor{currentfill}{rgb}{0.000000,0.000000,0.000000}%
\pgfsetfillcolor{currentfill}%
\pgfsetlinewidth{0.602250pt}%
\definecolor{currentstroke}{rgb}{0.000000,0.000000,0.000000}%
\pgfsetstrokecolor{currentstroke}%
\pgfsetdash{}{0pt}%
\pgfsys@defobject{currentmarker}{\pgfqpoint{0.000000in}{0.000000in}}{\pgfqpoint{0.027778in}{0.000000in}}{%
\pgfpathmoveto{\pgfqpoint{0.000000in}{0.000000in}}%
\pgfpathlineto{\pgfqpoint{0.027778in}{0.000000in}}%
\pgfusepath{stroke,fill}%
}%
\begin{pgfscope}%
\pgfsys@transformshift{4.668160in}{2.614424in}%
\pgfsys@useobject{currentmarker}{}%
\end{pgfscope}%
\end{pgfscope}%
\begin{pgfscope}%
\pgfsetbuttcap%
\pgfsetroundjoin%
\definecolor{currentfill}{rgb}{0.000000,0.000000,0.000000}%
\pgfsetfillcolor{currentfill}%
\pgfsetlinewidth{0.602250pt}%
\definecolor{currentstroke}{rgb}{0.000000,0.000000,0.000000}%
\pgfsetstrokecolor{currentstroke}%
\pgfsetdash{}{0pt}%
\pgfsys@defobject{currentmarker}{\pgfqpoint{0.000000in}{0.000000in}}{\pgfqpoint{0.027778in}{0.000000in}}{%
\pgfpathmoveto{\pgfqpoint{0.000000in}{0.000000in}}%
\pgfpathlineto{\pgfqpoint{0.027778in}{0.000000in}}%
\pgfusepath{stroke,fill}%
}%
\begin{pgfscope}%
\pgfsys@transformshift{4.668160in}{2.832735in}%
\pgfsys@useobject{currentmarker}{}%
\end{pgfscope}%
\end{pgfscope}%
\begin{pgfscope}%
\pgfsetbuttcap%
\pgfsetroundjoin%
\definecolor{currentfill}{rgb}{0.000000,0.000000,0.000000}%
\pgfsetfillcolor{currentfill}%
\pgfsetlinewidth{0.602250pt}%
\definecolor{currentstroke}{rgb}{0.000000,0.000000,0.000000}%
\pgfsetstrokecolor{currentstroke}%
\pgfsetdash{}{0pt}%
\pgfsys@defobject{currentmarker}{\pgfqpoint{0.000000in}{0.000000in}}{\pgfqpoint{0.027778in}{0.000000in}}{%
\pgfpathmoveto{\pgfqpoint{0.000000in}{0.000000in}}%
\pgfpathlineto{\pgfqpoint{0.027778in}{0.000000in}}%
\pgfusepath{stroke,fill}%
}%
\begin{pgfscope}%
\pgfsys@transformshift{4.668160in}{3.017314in}%
\pgfsys@useobject{currentmarker}{}%
\end{pgfscope}%
\end{pgfscope}%
\begin{pgfscope}%
\pgfsetbuttcap%
\pgfsetroundjoin%
\definecolor{currentfill}{rgb}{0.000000,0.000000,0.000000}%
\pgfsetfillcolor{currentfill}%
\pgfsetlinewidth{0.602250pt}%
\definecolor{currentstroke}{rgb}{0.000000,0.000000,0.000000}%
\pgfsetstrokecolor{currentstroke}%
\pgfsetdash{}{0pt}%
\pgfsys@defobject{currentmarker}{\pgfqpoint{0.000000in}{0.000000in}}{\pgfqpoint{0.027778in}{0.000000in}}{%
\pgfpathmoveto{\pgfqpoint{0.000000in}{0.000000in}}%
\pgfpathlineto{\pgfqpoint{0.027778in}{0.000000in}}%
\pgfusepath{stroke,fill}%
}%
\begin{pgfscope}%
\pgfsys@transformshift{4.668160in}{3.177204in}%
\pgfsys@useobject{currentmarker}{}%
\end{pgfscope}%
\end{pgfscope}%
\begin{pgfscope}%
\pgfsetbuttcap%
\pgfsetroundjoin%
\definecolor{currentfill}{rgb}{0.000000,0.000000,0.000000}%
\pgfsetfillcolor{currentfill}%
\pgfsetlinewidth{0.602250pt}%
\definecolor{currentstroke}{rgb}{0.000000,0.000000,0.000000}%
\pgfsetstrokecolor{currentstroke}%
\pgfsetdash{}{0pt}%
\pgfsys@defobject{currentmarker}{\pgfqpoint{0.000000in}{0.000000in}}{\pgfqpoint{0.027778in}{0.000000in}}{%
\pgfpathmoveto{\pgfqpoint{0.000000in}{0.000000in}}%
\pgfpathlineto{\pgfqpoint{0.027778in}{0.000000in}}%
\pgfusepath{stroke,fill}%
}%
\begin{pgfscope}%
\pgfsys@transformshift{4.668160in}{3.318236in}%
\pgfsys@useobject{currentmarker}{}%
\end{pgfscope}%
\end{pgfscope}%
\begin{pgfscope}%
\definecolor{textcolor}{rgb}{0.000000,0.000000,0.000000}%
\pgfsetstrokecolor{textcolor}%
\pgfsetfillcolor{textcolor}%
\pgftext[x=5.077690in,y=2.031603in,,top]{\color{textcolor}\rmfamily\fontsize{14.000000}{16.800000}\selectfont \(\displaystyle {\mathbf{E} \mbox{u}}\)}%
\end{pgfscope}%
\begin{pgfscope}%
\pgfsetbuttcap%
\pgfsetmiterjoin%
\pgfsetlinewidth{0.803000pt}%
\definecolor{currentstroke}{rgb}{0.501961,0.501961,0.501961}%
\pgfsetstrokecolor{currentstroke}%
\pgfsetdash{}{0pt}%
\pgfpathmoveto{\pgfqpoint{4.517160in}{0.521603in}}%
\pgfpathlineto{\pgfqpoint{4.517160in}{0.533400in}}%
\pgfpathlineto{\pgfqpoint{4.517160in}{3.529806in}}%
\pgfpathlineto{\pgfqpoint{4.517160in}{3.541603in}}%
\pgfpathlineto{\pgfqpoint{4.668160in}{3.541603in}}%
\pgfpathlineto{\pgfqpoint{4.668160in}{3.529806in}}%
\pgfpathlineto{\pgfqpoint{4.668160in}{0.533400in}}%
\pgfpathlineto{\pgfqpoint{4.668160in}{0.521603in}}%
\pgfpathclose%
\pgfusepath{stroke}%
\end{pgfscope}%
\end{pgfpicture}%
\makeatother%
\endgroup%
}
    \caption{Non-dimensional trajectories with the short-time scaling.\label{fig:series_s_ds}}
\end{figure}
\begin{figure}[htb]
    \centering
    \resizebox{0.5\textwidth}{!}{%% Creator: Matplotlib, PGF backend
%%
%% To include the figure in your LaTeX document, write
%%   \input{<filename>.pgf}
%%
%% Make sure the required packages are loaded in your preamble
%%   \usepackage{pgf}
%%
%% Figures using additional raster images can only be included by \input if
%% they are in the same directory as the main LaTeX file. For loading figures
%% from other directories you can use the `import` package
%%   \usepackage{import}
%% and then include the figures with
%%   \import{<path to file>}{<filename>.pgf}
%%
%% Matplotlib used the following preamble
%%   \usepackage{fontspec}
%%   \setmainfont{DejaVu Serif}
%%   \setsansfont{DejaVu Sans}
%%   \setmonofont{DejaVu Sans Mono}
%%
\begingroup%
\makeatletter%
\begin{pgfpicture}%
\pgfpathrectangle{\pgfpointorigin}{\pgfqpoint{5.497497in}{3.865458in}}%
\pgfusepath{use as bounding box, clip}%
\begin{pgfscope}%
\pgfsetbuttcap%
\pgfsetmiterjoin%
\definecolor{currentfill}{rgb}{1.000000,1.000000,1.000000}%
\pgfsetfillcolor{currentfill}%
\pgfsetlinewidth{0.000000pt}%
\definecolor{currentstroke}{rgb}{1.000000,1.000000,1.000000}%
\pgfsetstrokecolor{currentstroke}%
\pgfsetdash{}{0pt}%
\pgfpathmoveto{\pgfqpoint{0.000000in}{0.000000in}}%
\pgfpathlineto{\pgfqpoint{5.497497in}{0.000000in}}%
\pgfpathlineto{\pgfqpoint{5.497497in}{3.865458in}}%
\pgfpathlineto{\pgfqpoint{0.000000in}{3.865458in}}%
\pgfpathclose%
\pgfusepath{fill}%
\end{pgfscope}%
\begin{pgfscope}%
\pgfsetbuttcap%
\pgfsetmiterjoin%
\definecolor{currentfill}{rgb}{1.000000,1.000000,1.000000}%
\pgfsetfillcolor{currentfill}%
\pgfsetlinewidth{0.000000pt}%
\definecolor{currentstroke}{rgb}{0.000000,0.000000,0.000000}%
\pgfsetstrokecolor{currentstroke}%
\pgfsetstrokeopacity{0.000000}%
\pgfsetdash{}{0pt}%
\pgfpathmoveto{\pgfqpoint{0.712497in}{0.682899in}}%
\pgfpathlineto{\pgfqpoint{5.362497in}{0.682899in}}%
\pgfpathlineto{\pgfqpoint{5.362497in}{3.702899in}}%
\pgfpathlineto{\pgfqpoint{0.712497in}{3.702899in}}%
\pgfpathclose%
\pgfusepath{fill}%
\end{pgfscope}%
\begin{pgfscope}%
\pgfpathrectangle{\pgfqpoint{0.712497in}{0.682899in}}{\pgfqpoint{4.650000in}{3.020000in}}%
\pgfusepath{clip}%
\pgfsetbuttcap%
\pgfsetroundjoin%
\definecolor{currentfill}{rgb}{1.000000,1.000000,1.000000}%
\pgfsetfillcolor{currentfill}%
\pgfsetlinewidth{1.003750pt}%
\definecolor{currentstroke}{rgb}{0.000000,0.000000,0.000000}%
\pgfsetstrokecolor{currentstroke}%
\pgfsetdash{}{0pt}%
\pgfpathmoveto{\pgfqpoint{4.395745in}{0.781098in}}%
\pgfpathcurveto{\pgfqpoint{4.406795in}{0.781098in}}{\pgfqpoint{4.417394in}{0.785488in}}{\pgfqpoint{4.425208in}{0.793302in}}%
\pgfpathcurveto{\pgfqpoint{4.433022in}{0.801116in}}{\pgfqpoint{4.437412in}{0.811715in}}{\pgfqpoint{4.437412in}{0.822765in}}%
\pgfpathcurveto{\pgfqpoint{4.437412in}{0.833815in}}{\pgfqpoint{4.433022in}{0.844414in}}{\pgfqpoint{4.425208in}{0.852228in}}%
\pgfpathcurveto{\pgfqpoint{4.417394in}{0.860041in}}{\pgfqpoint{4.406795in}{0.864431in}}{\pgfqpoint{4.395745in}{0.864431in}}%
\pgfpathcurveto{\pgfqpoint{4.384695in}{0.864431in}}{\pgfqpoint{4.374096in}{0.860041in}}{\pgfqpoint{4.366282in}{0.852228in}}%
\pgfpathcurveto{\pgfqpoint{4.358469in}{0.844414in}}{\pgfqpoint{4.354078in}{0.833815in}}{\pgfqpoint{4.354078in}{0.822765in}}%
\pgfpathcurveto{\pgfqpoint{4.354078in}{0.811715in}}{\pgfqpoint{4.358469in}{0.801116in}}{\pgfqpoint{4.366282in}{0.793302in}}%
\pgfpathcurveto{\pgfqpoint{4.374096in}{0.785488in}}{\pgfqpoint{4.384695in}{0.781098in}}{\pgfqpoint{4.395745in}{0.781098in}}%
\pgfpathclose%
\pgfusepath{stroke,fill}%
\end{pgfscope}%
\begin{pgfscope}%
\pgfpathrectangle{\pgfqpoint{0.712497in}{0.682899in}}{\pgfqpoint{4.650000in}{3.020000in}}%
\pgfusepath{clip}%
\pgfsetbuttcap%
\pgfsetroundjoin%
\definecolor{currentfill}{rgb}{1.000000,1.000000,1.000000}%
\pgfsetfillcolor{currentfill}%
\pgfsetlinewidth{1.003750pt}%
\definecolor{currentstroke}{rgb}{0.000000,0.000000,0.000000}%
\pgfsetstrokecolor{currentstroke}%
\pgfsetdash{}{0pt}%
\pgfpathmoveto{\pgfqpoint{3.384709in}{0.872078in}}%
\pgfpathcurveto{\pgfqpoint{3.395759in}{0.872078in}}{\pgfqpoint{3.406358in}{0.876468in}}{\pgfqpoint{3.414172in}{0.884282in}}%
\pgfpathcurveto{\pgfqpoint{3.421986in}{0.892095in}}{\pgfqpoint{3.426376in}{0.902694in}}{\pgfqpoint{3.426376in}{0.913744in}}%
\pgfpathcurveto{\pgfqpoint{3.426376in}{0.924795in}}{\pgfqpoint{3.421986in}{0.935394in}}{\pgfqpoint{3.414172in}{0.943207in}}%
\pgfpathcurveto{\pgfqpoint{3.406358in}{0.951021in}}{\pgfqpoint{3.395759in}{0.955411in}}{\pgfqpoint{3.384709in}{0.955411in}}%
\pgfpathcurveto{\pgfqpoint{3.373659in}{0.955411in}}{\pgfqpoint{3.363060in}{0.951021in}}{\pgfqpoint{3.355246in}{0.943207in}}%
\pgfpathcurveto{\pgfqpoint{3.347433in}{0.935394in}}{\pgfqpoint{3.343043in}{0.924795in}}{\pgfqpoint{3.343043in}{0.913744in}}%
\pgfpathcurveto{\pgfqpoint{3.343043in}{0.902694in}}{\pgfqpoint{3.347433in}{0.892095in}}{\pgfqpoint{3.355246in}{0.884282in}}%
\pgfpathcurveto{\pgfqpoint{3.363060in}{0.876468in}}{\pgfqpoint{3.373659in}{0.872078in}}{\pgfqpoint{3.384709in}{0.872078in}}%
\pgfpathclose%
\pgfusepath{stroke,fill}%
\end{pgfscope}%
\begin{pgfscope}%
\pgfpathrectangle{\pgfqpoint{0.712497in}{0.682899in}}{\pgfqpoint{4.650000in}{3.020000in}}%
\pgfusepath{clip}%
\pgfsetbuttcap%
\pgfsetroundjoin%
\definecolor{currentfill}{rgb}{1.000000,1.000000,1.000000}%
\pgfsetfillcolor{currentfill}%
\pgfsetlinewidth{1.003750pt}%
\definecolor{currentstroke}{rgb}{0.000000,0.000000,0.000000}%
\pgfsetstrokecolor{currentstroke}%
\pgfsetdash{}{0pt}%
\pgfpathmoveto{\pgfqpoint{5.141466in}{0.894493in}}%
\pgfpathcurveto{\pgfqpoint{5.152516in}{0.894493in}}{\pgfqpoint{5.163115in}{0.898883in}}{\pgfqpoint{5.170928in}{0.906697in}}%
\pgfpathcurveto{\pgfqpoint{5.178742in}{0.914510in}}{\pgfqpoint{5.183132in}{0.925109in}}{\pgfqpoint{5.183132in}{0.936160in}}%
\pgfpathcurveto{\pgfqpoint{5.183132in}{0.947210in}}{\pgfqpoint{5.178742in}{0.957809in}}{\pgfqpoint{5.170928in}{0.965622in}}%
\pgfpathcurveto{\pgfqpoint{5.163115in}{0.973436in}}{\pgfqpoint{5.152516in}{0.977826in}}{\pgfqpoint{5.141466in}{0.977826in}}%
\pgfpathcurveto{\pgfqpoint{5.130416in}{0.977826in}}{\pgfqpoint{5.119816in}{0.973436in}}{\pgfqpoint{5.112003in}{0.965622in}}%
\pgfpathcurveto{\pgfqpoint{5.104189in}{0.957809in}}{\pgfqpoint{5.099799in}{0.947210in}}{\pgfqpoint{5.099799in}{0.936160in}}%
\pgfpathcurveto{\pgfqpoint{5.099799in}{0.925109in}}{\pgfqpoint{5.104189in}{0.914510in}}{\pgfqpoint{5.112003in}{0.906697in}}%
\pgfpathcurveto{\pgfqpoint{5.119816in}{0.898883in}}{\pgfqpoint{5.130416in}{0.894493in}}{\pgfqpoint{5.141466in}{0.894493in}}%
\pgfpathclose%
\pgfusepath{stroke,fill}%
\end{pgfscope}%
\begin{pgfscope}%
\pgfpathrectangle{\pgfqpoint{0.712497in}{0.682899in}}{\pgfqpoint{4.650000in}{3.020000in}}%
\pgfusepath{clip}%
\pgfsetbuttcap%
\pgfsetroundjoin%
\definecolor{currentfill}{rgb}{1.000000,1.000000,1.000000}%
\pgfsetfillcolor{currentfill}%
\pgfsetlinewidth{1.003750pt}%
\definecolor{currentstroke}{rgb}{0.000000,0.000000,0.000000}%
\pgfsetstrokecolor{currentstroke}%
\pgfsetdash{}{0pt}%
\pgfpathmoveto{\pgfqpoint{2.376928in}{1.033990in}}%
\pgfpathcurveto{\pgfqpoint{2.387978in}{1.033990in}}{\pgfqpoint{2.398577in}{1.038380in}}{\pgfqpoint{2.406391in}{1.046194in}}%
\pgfpathcurveto{\pgfqpoint{2.414205in}{1.054007in}}{\pgfqpoint{2.418595in}{1.064606in}}{\pgfqpoint{2.418595in}{1.075657in}}%
\pgfpathcurveto{\pgfqpoint{2.418595in}{1.086707in}}{\pgfqpoint{2.414205in}{1.097306in}}{\pgfqpoint{2.406391in}{1.105119in}}%
\pgfpathcurveto{\pgfqpoint{2.398577in}{1.112933in}}{\pgfqpoint{2.387978in}{1.117323in}}{\pgfqpoint{2.376928in}{1.117323in}}%
\pgfpathcurveto{\pgfqpoint{2.365878in}{1.117323in}}{\pgfqpoint{2.355279in}{1.112933in}}{\pgfqpoint{2.347465in}{1.105119in}}%
\pgfpathcurveto{\pgfqpoint{2.339652in}{1.097306in}}{\pgfqpoint{2.335262in}{1.086707in}}{\pgfqpoint{2.335262in}{1.075657in}}%
\pgfpathcurveto{\pgfqpoint{2.335262in}{1.064606in}}{\pgfqpoint{2.339652in}{1.054007in}}{\pgfqpoint{2.347465in}{1.046194in}}%
\pgfpathcurveto{\pgfqpoint{2.355279in}{1.038380in}}{\pgfqpoint{2.365878in}{1.033990in}}{\pgfqpoint{2.376928in}{1.033990in}}%
\pgfpathclose%
\pgfusepath{stroke,fill}%
\end{pgfscope}%
\begin{pgfscope}%
\pgfpathrectangle{\pgfqpoint{0.712497in}{0.682899in}}{\pgfqpoint{4.650000in}{3.020000in}}%
\pgfusepath{clip}%
\pgfsetbuttcap%
\pgfsetroundjoin%
\definecolor{currentfill}{rgb}{1.000000,1.000000,1.000000}%
\pgfsetfillcolor{currentfill}%
\pgfsetlinewidth{1.003750pt}%
\definecolor{currentstroke}{rgb}{0.000000,0.000000,0.000000}%
\pgfsetstrokecolor{currentstroke}%
\pgfsetdash{}{0pt}%
\pgfpathmoveto{\pgfqpoint{1.930775in}{1.400028in}}%
\pgfpathcurveto{\pgfqpoint{1.941825in}{1.400028in}}{\pgfqpoint{1.952424in}{1.404418in}}{\pgfqpoint{1.960238in}{1.412232in}}%
\pgfpathcurveto{\pgfqpoint{1.968051in}{1.420045in}}{\pgfqpoint{1.972442in}{1.430644in}}{\pgfqpoint{1.972442in}{1.441695in}}%
\pgfpathcurveto{\pgfqpoint{1.972442in}{1.452745in}}{\pgfqpoint{1.968051in}{1.463344in}}{\pgfqpoint{1.960238in}{1.471157in}}%
\pgfpathcurveto{\pgfqpoint{1.952424in}{1.478971in}}{\pgfqpoint{1.941825in}{1.483361in}}{\pgfqpoint{1.930775in}{1.483361in}}%
\pgfpathcurveto{\pgfqpoint{1.919725in}{1.483361in}}{\pgfqpoint{1.909126in}{1.478971in}}{\pgfqpoint{1.901312in}{1.471157in}}%
\pgfpathcurveto{\pgfqpoint{1.893499in}{1.463344in}}{\pgfqpoint{1.889108in}{1.452745in}}{\pgfqpoint{1.889108in}{1.441695in}}%
\pgfpathcurveto{\pgfqpoint{1.889108in}{1.430644in}}{\pgfqpoint{1.893499in}{1.420045in}}{\pgfqpoint{1.901312in}{1.412232in}}%
\pgfpathcurveto{\pgfqpoint{1.909126in}{1.404418in}}{\pgfqpoint{1.919725in}{1.400028in}}{\pgfqpoint{1.930775in}{1.400028in}}%
\pgfpathclose%
\pgfusepath{stroke,fill}%
\end{pgfscope}%
\begin{pgfscope}%
\pgfpathrectangle{\pgfqpoint{0.712497in}{0.682899in}}{\pgfqpoint{4.650000in}{3.020000in}}%
\pgfusepath{clip}%
\pgfsetbuttcap%
\pgfsetroundjoin%
\definecolor{currentfill}{rgb}{1.000000,1.000000,1.000000}%
\pgfsetfillcolor{currentfill}%
\pgfsetlinewidth{1.003750pt}%
\definecolor{currentstroke}{rgb}{0.000000,0.000000,0.000000}%
\pgfsetstrokecolor{currentstroke}%
\pgfsetdash{}{0pt}%
\pgfpathmoveto{\pgfqpoint{1.627718in}{1.466608in}}%
\pgfpathcurveto{\pgfqpoint{1.638768in}{1.466608in}}{\pgfqpoint{1.649367in}{1.470999in}}{\pgfqpoint{1.657181in}{1.478812in}}%
\pgfpathcurveto{\pgfqpoint{1.664994in}{1.486626in}}{\pgfqpoint{1.669384in}{1.497225in}}{\pgfqpoint{1.669384in}{1.508275in}}%
\pgfpathcurveto{\pgfqpoint{1.669384in}{1.519325in}}{\pgfqpoint{1.664994in}{1.529924in}}{\pgfqpoint{1.657181in}{1.537738in}}%
\pgfpathcurveto{\pgfqpoint{1.649367in}{1.545551in}}{\pgfqpoint{1.638768in}{1.549942in}}{\pgfqpoint{1.627718in}{1.549942in}}%
\pgfpathcurveto{\pgfqpoint{1.616668in}{1.549942in}}{\pgfqpoint{1.606069in}{1.545551in}}{\pgfqpoint{1.598255in}{1.537738in}}%
\pgfpathcurveto{\pgfqpoint{1.590441in}{1.529924in}}{\pgfqpoint{1.586051in}{1.519325in}}{\pgfqpoint{1.586051in}{1.508275in}}%
\pgfpathcurveto{\pgfqpoint{1.586051in}{1.497225in}}{\pgfqpoint{1.590441in}{1.486626in}}{\pgfqpoint{1.598255in}{1.478812in}}%
\pgfpathcurveto{\pgfqpoint{1.606069in}{1.470999in}}{\pgfqpoint{1.616668in}{1.466608in}}{\pgfqpoint{1.627718in}{1.466608in}}%
\pgfpathclose%
\pgfusepath{stroke,fill}%
\end{pgfscope}%
\begin{pgfscope}%
\pgfpathrectangle{\pgfqpoint{0.712497in}{0.682899in}}{\pgfqpoint{4.650000in}{3.020000in}}%
\pgfusepath{clip}%
\pgfsetbuttcap%
\pgfsetroundjoin%
\definecolor{currentfill}{rgb}{1.000000,1.000000,1.000000}%
\pgfsetfillcolor{currentfill}%
\pgfsetlinewidth{1.003750pt}%
\definecolor{currentstroke}{rgb}{0.000000,0.000000,0.000000}%
\pgfsetstrokecolor{currentstroke}%
\pgfsetdash{}{0pt}%
\pgfpathmoveto{\pgfqpoint{1.726414in}{1.946673in}}%
\pgfpathcurveto{\pgfqpoint{1.737464in}{1.946673in}}{\pgfqpoint{1.748063in}{1.951064in}}{\pgfqpoint{1.755877in}{1.958877in}}%
\pgfpathcurveto{\pgfqpoint{1.763691in}{1.966691in}}{\pgfqpoint{1.768081in}{1.977290in}}{\pgfqpoint{1.768081in}{1.988340in}}%
\pgfpathcurveto{\pgfqpoint{1.768081in}{1.999390in}}{\pgfqpoint{1.763691in}{2.009989in}}{\pgfqpoint{1.755877in}{2.017803in}}%
\pgfpathcurveto{\pgfqpoint{1.748063in}{2.025616in}}{\pgfqpoint{1.737464in}{2.030007in}}{\pgfqpoint{1.726414in}{2.030007in}}%
\pgfpathcurveto{\pgfqpoint{1.715364in}{2.030007in}}{\pgfqpoint{1.704765in}{2.025616in}}{\pgfqpoint{1.696951in}{2.017803in}}%
\pgfpathcurveto{\pgfqpoint{1.689138in}{2.009989in}}{\pgfqpoint{1.684748in}{1.999390in}}{\pgfqpoint{1.684748in}{1.988340in}}%
\pgfpathcurveto{\pgfqpoint{1.684748in}{1.977290in}}{\pgfqpoint{1.689138in}{1.966691in}}{\pgfqpoint{1.696951in}{1.958877in}}%
\pgfpathcurveto{\pgfqpoint{1.704765in}{1.951064in}}{\pgfqpoint{1.715364in}{1.946673in}}{\pgfqpoint{1.726414in}{1.946673in}}%
\pgfpathclose%
\pgfusepath{stroke,fill}%
\end{pgfscope}%
\begin{pgfscope}%
\pgfpathrectangle{\pgfqpoint{0.712497in}{0.682899in}}{\pgfqpoint{4.650000in}{3.020000in}}%
\pgfusepath{clip}%
\pgfsetbuttcap%
\pgfsetroundjoin%
\definecolor{currentfill}{rgb}{1.000000,1.000000,1.000000}%
\pgfsetfillcolor{currentfill}%
\pgfsetlinewidth{1.003750pt}%
\definecolor{currentstroke}{rgb}{0.000000,0.000000,0.000000}%
\pgfsetstrokecolor{currentstroke}%
\pgfsetdash{}{0pt}%
\pgfpathmoveto{\pgfqpoint{0.933529in}{3.170274in}}%
\pgfpathcurveto{\pgfqpoint{0.944579in}{3.170274in}}{\pgfqpoint{0.955178in}{3.174664in}}{\pgfqpoint{0.962991in}{3.182478in}}%
\pgfpathcurveto{\pgfqpoint{0.970805in}{3.190291in}}{\pgfqpoint{0.975195in}{3.200890in}}{\pgfqpoint{0.975195in}{3.211940in}}%
\pgfpathcurveto{\pgfqpoint{0.975195in}{3.222990in}}{\pgfqpoint{0.970805in}{3.233589in}}{\pgfqpoint{0.962991in}{3.241403in}}%
\pgfpathcurveto{\pgfqpoint{0.955178in}{3.249217in}}{\pgfqpoint{0.944579in}{3.253607in}}{\pgfqpoint{0.933529in}{3.253607in}}%
\pgfpathcurveto{\pgfqpoint{0.922478in}{3.253607in}}{\pgfqpoint{0.911879in}{3.249217in}}{\pgfqpoint{0.904066in}{3.241403in}}%
\pgfpathcurveto{\pgfqpoint{0.896252in}{3.233589in}}{\pgfqpoint{0.891862in}{3.222990in}}{\pgfqpoint{0.891862in}{3.211940in}}%
\pgfpathcurveto{\pgfqpoint{0.891862in}{3.200890in}}{\pgfqpoint{0.896252in}{3.190291in}}{\pgfqpoint{0.904066in}{3.182478in}}%
\pgfpathcurveto{\pgfqpoint{0.911879in}{3.174664in}}{\pgfqpoint{0.922478in}{3.170274in}}{\pgfqpoint{0.933529in}{3.170274in}}%
\pgfpathclose%
\pgfusepath{stroke,fill}%
\end{pgfscope}%
\begin{pgfscope}%
\pgfpathrectangle{\pgfqpoint{0.712497in}{0.682899in}}{\pgfqpoint{4.650000in}{3.020000in}}%
\pgfusepath{clip}%
\pgfsetbuttcap%
\pgfsetroundjoin%
\definecolor{currentfill}{rgb}{1.000000,1.000000,1.000000}%
\pgfsetfillcolor{currentfill}%
\pgfsetlinewidth{1.003750pt}%
\definecolor{currentstroke}{rgb}{0.000000,0.000000,0.000000}%
\pgfsetstrokecolor{currentstroke}%
\pgfsetdash{}{0pt}%
\pgfpathmoveto{\pgfqpoint{1.015676in}{3.521366in}}%
\pgfpathcurveto{\pgfqpoint{1.026726in}{3.521366in}}{\pgfqpoint{1.037325in}{3.525757in}}{\pgfqpoint{1.045139in}{3.533570in}}%
\pgfpathcurveto{\pgfqpoint{1.052953in}{3.541384in}}{\pgfqpoint{1.057343in}{3.551983in}}{\pgfqpoint{1.057343in}{3.563033in}}%
\pgfpathcurveto{\pgfqpoint{1.057343in}{3.574083in}}{\pgfqpoint{1.052953in}{3.584682in}}{\pgfqpoint{1.045139in}{3.592496in}}%
\pgfpathcurveto{\pgfqpoint{1.037325in}{3.600309in}}{\pgfqpoint{1.026726in}{3.604700in}}{\pgfqpoint{1.015676in}{3.604700in}}%
\pgfpathcurveto{\pgfqpoint{1.004626in}{3.604700in}}{\pgfqpoint{0.994027in}{3.600309in}}{\pgfqpoint{0.986213in}{3.592496in}}%
\pgfpathcurveto{\pgfqpoint{0.978400in}{3.584682in}}{\pgfqpoint{0.974010in}{3.574083in}}{\pgfqpoint{0.974010in}{3.563033in}}%
\pgfpathcurveto{\pgfqpoint{0.974010in}{3.551983in}}{\pgfqpoint{0.978400in}{3.541384in}}{\pgfqpoint{0.986213in}{3.533570in}}%
\pgfpathcurveto{\pgfqpoint{0.994027in}{3.525757in}}{\pgfqpoint{1.004626in}{3.521366in}}{\pgfqpoint{1.015676in}{3.521366in}}%
\pgfpathclose%
\pgfusepath{stroke,fill}%
\end{pgfscope}%
\begin{pgfscope}%
\pgfsetbuttcap%
\pgfsetroundjoin%
\definecolor{currentfill}{rgb}{0.000000,0.000000,0.000000}%
\pgfsetfillcolor{currentfill}%
\pgfsetlinewidth{0.803000pt}%
\definecolor{currentstroke}{rgb}{0.000000,0.000000,0.000000}%
\pgfsetstrokecolor{currentstroke}%
\pgfsetdash{}{0pt}%
\pgfsys@defobject{currentmarker}{\pgfqpoint{0.000000in}{-0.048611in}}{\pgfqpoint{0.000000in}{0.000000in}}{%
\pgfpathmoveto{\pgfqpoint{0.000000in}{0.000000in}}%
\pgfpathlineto{\pgfqpoint{0.000000in}{-0.048611in}}%
\pgfusepath{stroke,fill}%
}%
\begin{pgfscope}%
\pgfsys@transformshift{1.544241in}{0.682899in}%
\pgfsys@useobject{currentmarker}{}%
\end{pgfscope}%
\end{pgfscope}%
\begin{pgfscope}%
\pgftext[x=1.544241in,y=0.585677in,,top]{\rmfamily\fontsize{16.000000}{19.200000}\selectfont \(\displaystyle 2\)}%
\end{pgfscope}%
\begin{pgfscope}%
\pgfsetbuttcap%
\pgfsetroundjoin%
\definecolor{currentfill}{rgb}{0.000000,0.000000,0.000000}%
\pgfsetfillcolor{currentfill}%
\pgfsetlinewidth{0.803000pt}%
\definecolor{currentstroke}{rgb}{0.000000,0.000000,0.000000}%
\pgfsetstrokecolor{currentstroke}%
\pgfsetdash{}{0pt}%
\pgfsys@defobject{currentmarker}{\pgfqpoint{0.000000in}{-0.048611in}}{\pgfqpoint{0.000000in}{0.000000in}}{%
\pgfpathmoveto{\pgfqpoint{0.000000in}{0.000000in}}%
\pgfpathlineto{\pgfqpoint{0.000000in}{-0.048611in}}%
\pgfusepath{stroke,fill}%
}%
\begin{pgfscope}%
\pgfsys@transformshift{3.042751in}{0.682899in}%
\pgfsys@useobject{currentmarker}{}%
\end{pgfscope}%
\end{pgfscope}%
\begin{pgfscope}%
\pgftext[x=3.042751in,y=0.585677in,,top]{\rmfamily\fontsize{16.000000}{19.200000}\selectfont \(\displaystyle 3\)}%
\end{pgfscope}%
\begin{pgfscope}%
\pgfsetbuttcap%
\pgfsetroundjoin%
\definecolor{currentfill}{rgb}{0.000000,0.000000,0.000000}%
\pgfsetfillcolor{currentfill}%
\pgfsetlinewidth{0.803000pt}%
\definecolor{currentstroke}{rgb}{0.000000,0.000000,0.000000}%
\pgfsetstrokecolor{currentstroke}%
\pgfsetdash{}{0pt}%
\pgfsys@defobject{currentmarker}{\pgfqpoint{0.000000in}{-0.048611in}}{\pgfqpoint{0.000000in}{0.000000in}}{%
\pgfpathmoveto{\pgfqpoint{0.000000in}{0.000000in}}%
\pgfpathlineto{\pgfqpoint{0.000000in}{-0.048611in}}%
\pgfusepath{stroke,fill}%
}%
\begin{pgfscope}%
\pgfsys@transformshift{4.541261in}{0.682899in}%
\pgfsys@useobject{currentmarker}{}%
\end{pgfscope}%
\end{pgfscope}%
\begin{pgfscope}%
\pgftext[x=4.541261in,y=0.585677in,,top]{\rmfamily\fontsize{16.000000}{19.200000}\selectfont \(\displaystyle 4\)}%
\end{pgfscope}%
\begin{pgfscope}%
\pgftext[x=3.037497in,y=0.315061in,,top]{\rmfamily\fontsize{16.000000}{19.200000}\selectfont \(\displaystyle \mathbf{E}\mbox{u}\)}%
\end{pgfscope}%
\begin{pgfscope}%
\pgfsetbuttcap%
\pgfsetroundjoin%
\definecolor{currentfill}{rgb}{0.000000,0.000000,0.000000}%
\pgfsetfillcolor{currentfill}%
\pgfsetlinewidth{0.803000pt}%
\definecolor{currentstroke}{rgb}{0.000000,0.000000,0.000000}%
\pgfsetstrokecolor{currentstroke}%
\pgfsetdash{}{0pt}%
\pgfsys@defobject{currentmarker}{\pgfqpoint{-0.048611in}{0.000000in}}{\pgfqpoint{0.000000in}{0.000000in}}{%
\pgfpathmoveto{\pgfqpoint{0.000000in}{0.000000in}}%
\pgfpathlineto{\pgfqpoint{-0.048611in}{0.000000in}}%
\pgfusepath{stroke,fill}%
}%
\begin{pgfscope}%
\pgfsys@transformshift{0.712497in}{1.070553in}%
\pgfsys@useobject{currentmarker}{}%
\end{pgfscope}%
\end{pgfscope}%
\begin{pgfscope}%
\pgftext[x=0.505207in,y=0.986135in,left,base]{\rmfamily\fontsize{16.000000}{19.200000}\selectfont \(\displaystyle 2\)}%
\end{pgfscope}%
\begin{pgfscope}%
\pgfsetbuttcap%
\pgfsetroundjoin%
\definecolor{currentfill}{rgb}{0.000000,0.000000,0.000000}%
\pgfsetfillcolor{currentfill}%
\pgfsetlinewidth{0.803000pt}%
\definecolor{currentstroke}{rgb}{0.000000,0.000000,0.000000}%
\pgfsetstrokecolor{currentstroke}%
\pgfsetdash{}{0pt}%
\pgfsys@defobject{currentmarker}{\pgfqpoint{-0.048611in}{0.000000in}}{\pgfqpoint{0.000000in}{0.000000in}}{%
\pgfpathmoveto{\pgfqpoint{0.000000in}{0.000000in}}%
\pgfpathlineto{\pgfqpoint{-0.048611in}{0.000000in}}%
\pgfusepath{stroke,fill}%
}%
\begin{pgfscope}%
\pgfsys@transformshift{0.712497in}{1.592650in}%
\pgfsys@useobject{currentmarker}{}%
\end{pgfscope}%
\end{pgfscope}%
\begin{pgfscope}%
\pgftext[x=0.505207in,y=1.508232in,left,base]{\rmfamily\fontsize{16.000000}{19.200000}\selectfont \(\displaystyle 4\)}%
\end{pgfscope}%
\begin{pgfscope}%
\pgfsetbuttcap%
\pgfsetroundjoin%
\definecolor{currentfill}{rgb}{0.000000,0.000000,0.000000}%
\pgfsetfillcolor{currentfill}%
\pgfsetlinewidth{0.803000pt}%
\definecolor{currentstroke}{rgb}{0.000000,0.000000,0.000000}%
\pgfsetstrokecolor{currentstroke}%
\pgfsetdash{}{0pt}%
\pgfsys@defobject{currentmarker}{\pgfqpoint{-0.048611in}{0.000000in}}{\pgfqpoint{0.000000in}{0.000000in}}{%
\pgfpathmoveto{\pgfqpoint{0.000000in}{0.000000in}}%
\pgfpathlineto{\pgfqpoint{-0.048611in}{0.000000in}}%
\pgfusepath{stroke,fill}%
}%
\begin{pgfscope}%
\pgfsys@transformshift{0.712497in}{2.114748in}%
\pgfsys@useobject{currentmarker}{}%
\end{pgfscope}%
\end{pgfscope}%
\begin{pgfscope}%
\pgftext[x=0.505207in,y=2.030329in,left,base]{\rmfamily\fontsize{16.000000}{19.200000}\selectfont \(\displaystyle 6\)}%
\end{pgfscope}%
\begin{pgfscope}%
\pgfsetbuttcap%
\pgfsetroundjoin%
\definecolor{currentfill}{rgb}{0.000000,0.000000,0.000000}%
\pgfsetfillcolor{currentfill}%
\pgfsetlinewidth{0.803000pt}%
\definecolor{currentstroke}{rgb}{0.000000,0.000000,0.000000}%
\pgfsetstrokecolor{currentstroke}%
\pgfsetdash{}{0pt}%
\pgfsys@defobject{currentmarker}{\pgfqpoint{-0.048611in}{0.000000in}}{\pgfqpoint{0.000000in}{0.000000in}}{%
\pgfpathmoveto{\pgfqpoint{0.000000in}{0.000000in}}%
\pgfpathlineto{\pgfqpoint{-0.048611in}{0.000000in}}%
\pgfusepath{stroke,fill}%
}%
\begin{pgfscope}%
\pgfsys@transformshift{0.712497in}{2.636845in}%
\pgfsys@useobject{currentmarker}{}%
\end{pgfscope}%
\end{pgfscope}%
\begin{pgfscope}%
\pgftext[x=0.505207in,y=2.552427in,left,base]{\rmfamily\fontsize{16.000000}{19.200000}\selectfont \(\displaystyle 8\)}%
\end{pgfscope}%
\begin{pgfscope}%
\pgfsetbuttcap%
\pgfsetroundjoin%
\definecolor{currentfill}{rgb}{0.000000,0.000000,0.000000}%
\pgfsetfillcolor{currentfill}%
\pgfsetlinewidth{0.803000pt}%
\definecolor{currentstroke}{rgb}{0.000000,0.000000,0.000000}%
\pgfsetstrokecolor{currentstroke}%
\pgfsetdash{}{0pt}%
\pgfsys@defobject{currentmarker}{\pgfqpoint{-0.048611in}{0.000000in}}{\pgfqpoint{0.000000in}{0.000000in}}{%
\pgfpathmoveto{\pgfqpoint{0.000000in}{0.000000in}}%
\pgfpathlineto{\pgfqpoint{-0.048611in}{0.000000in}}%
\pgfusepath{stroke,fill}%
}%
\begin{pgfscope}%
\pgfsys@transformshift{0.712497in}{3.158942in}%
\pgfsys@useobject{currentmarker}{}%
\end{pgfscope}%
\end{pgfscope}%
\begin{pgfscope}%
\pgftext[x=0.395138in,y=3.074524in,left,base]{\rmfamily\fontsize{16.000000}{19.200000}\selectfont \(\displaystyle 10\)}%
\end{pgfscope}%
\begin{pgfscope}%
\pgfsetbuttcap%
\pgfsetroundjoin%
\definecolor{currentfill}{rgb}{0.000000,0.000000,0.000000}%
\pgfsetfillcolor{currentfill}%
\pgfsetlinewidth{0.803000pt}%
\definecolor{currentstroke}{rgb}{0.000000,0.000000,0.000000}%
\pgfsetstrokecolor{currentstroke}%
\pgfsetdash{}{0pt}%
\pgfsys@defobject{currentmarker}{\pgfqpoint{-0.048611in}{0.000000in}}{\pgfqpoint{0.000000in}{0.000000in}}{%
\pgfpathmoveto{\pgfqpoint{0.000000in}{0.000000in}}%
\pgfpathlineto{\pgfqpoint{-0.048611in}{0.000000in}}%
\pgfusepath{stroke,fill}%
}%
\begin{pgfscope}%
\pgfsys@transformshift{0.712497in}{3.681040in}%
\pgfsys@useobject{currentmarker}{}%
\end{pgfscope}%
\end{pgfscope}%
\begin{pgfscope}%
\pgftext[x=0.395138in,y=3.596621in,left,base]{\rmfamily\fontsize{16.000000}{19.200000}\selectfont \(\displaystyle 12\)}%
\end{pgfscope}%
\begin{pgfscope}%
\pgftext[x=0.339583in,y=2.192899in,,bottom,rotate=90.000000]{\rmfamily\fontsize{16.000000}{19.200000}\selectfont \(\displaystyle {y_{max}}/y_c\)}%
\end{pgfscope}%
\begin{pgfscope}%
\pgfsetrectcap%
\pgfsetmiterjoin%
\pgfsetlinewidth{0.803000pt}%
\definecolor{currentstroke}{rgb}{0.501961,0.501961,0.501961}%
\pgfsetstrokecolor{currentstroke}%
\pgfsetdash{}{0pt}%
\pgfpathmoveto{\pgfqpoint{0.712497in}{0.682899in}}%
\pgfpathlineto{\pgfqpoint{0.712497in}{3.702899in}}%
\pgfusepath{stroke}%
\end{pgfscope}%
\begin{pgfscope}%
\pgfsetrectcap%
\pgfsetmiterjoin%
\pgfsetlinewidth{0.803000pt}%
\definecolor{currentstroke}{rgb}{0.501961,0.501961,0.501961}%
\pgfsetstrokecolor{currentstroke}%
\pgfsetdash{}{0pt}%
\pgfpathmoveto{\pgfqpoint{5.362497in}{0.682899in}}%
\pgfpathlineto{\pgfqpoint{5.362497in}{3.702899in}}%
\pgfusepath{stroke}%
\end{pgfscope}%
\begin{pgfscope}%
\pgfsetrectcap%
\pgfsetmiterjoin%
\pgfsetlinewidth{0.803000pt}%
\definecolor{currentstroke}{rgb}{0.501961,0.501961,0.501961}%
\pgfsetstrokecolor{currentstroke}%
\pgfsetdash{}{0pt}%
\pgfpathmoveto{\pgfqpoint{0.712497in}{0.682899in}}%
\pgfpathlineto{\pgfqpoint{5.362497in}{0.682899in}}%
\pgfusepath{stroke}%
\end{pgfscope}%
\begin{pgfscope}%
\pgfsetrectcap%
\pgfsetmiterjoin%
\pgfsetlinewidth{0.803000pt}%
\definecolor{currentstroke}{rgb}{0.501961,0.501961,0.501961}%
\pgfsetstrokecolor{currentstroke}%
\pgfsetdash{}{0pt}%
\pgfpathmoveto{\pgfqpoint{0.712497in}{3.702899in}}%
\pgfpathlineto{\pgfqpoint{5.362497in}{3.702899in}}%
\pgfusepath{stroke}%
\end{pgfscope}%
\end{pgfpicture}%
\makeatother%
\endgroup%
}
    \caption{Non-dimensional apoapse height $y_{max}/y_c$ compared with $\mathbb{E}\mbox{u}$.\label{fig:yscale_trend}}
\end{figure}

Because $\mathbb{E}\mbox{u} \ll 1$ is not satisfied for any of the experimental drops the entire dataset is beyond the regime where the asymptotic results for short-times, given by Equation \ref{perturb_image}, are valid. However, the long-time scaled asymptotic trajectory   $y_c$ and time-of-flight $t_f$ given by Equations \ref{perturb_viscous} and \ref{time_of_flight}, when redimensionalized by the short-time characteristic time $t_c=\mathbb{E}\mbox{u} R_d$, compare favorably to the experimental trajectory apoapses $y_{max}$ and time-of-flight $t_b$ as shown in Figures \ref{fig:times2} and \ref{fig:ymaxes}. This allows an improvement to be made to the asymptotic time-of-flight result of Equation \ref{time_of_flight} by multiplying the series by the empirical coefficient $a = 1.31$. The semi-analytic time-of-flight is then given by
\begin{equation}
\label{time_improved}
t^* = 2.62 + 1.75\phi\mathbb{E}\mbox{u} + 1.05\phi^2\mathbb{E}\mbox{u}^{2} + \mathcal{O}(\phi^3\mathbb{E}\mbox{u}^{3}).
\end{equation}
\begin{figure}[htb]
    \centering
    \resizebox{0.5\textwidth}{!}{%% Creator: Matplotlib, PGF backend
%%
%% To include the figure in your LaTeX document, write
%%   \input{<filename>.pgf}
%%
%% Make sure the required packages are loaded in your preamble
%%   \usepackage{pgf}
%%
%% Figures using additional raster images can only be included by \input if
%% they are in the same directory as the main LaTeX file. For loading figures
%% from other directories you can use the `import` package
%%   \usepackage{import}
%% and then include the figures with
%%   \import{<path to file>}{<filename>.pgf}
%%
%% Matplotlib used the following preamble
%%   \usepackage{fontspec}
%%   \setmainfont{DejaVu Serif}
%%   \setsansfont{DejaVu Sans}
%%   \setmonofont{DejaVu Sans Mono}
%%
\begingroup%
\makeatletter%
\begin{pgfpicture}%
\pgfpathrectangle{\pgfpointorigin}{\pgfqpoint{5.360508in}{3.676603in}}%
\pgfusepath{use as bounding box, clip}%
\begin{pgfscope}%
\pgfsetbuttcap%
\pgfsetmiterjoin%
\definecolor{currentfill}{rgb}{1.000000,1.000000,1.000000}%
\pgfsetfillcolor{currentfill}%
\pgfsetlinewidth{0.000000pt}%
\definecolor{currentstroke}{rgb}{1.000000,1.000000,1.000000}%
\pgfsetstrokecolor{currentstroke}%
\pgfsetdash{}{0pt}%
\pgfpathmoveto{\pgfqpoint{0.000000in}{0.000000in}}%
\pgfpathlineto{\pgfqpoint{5.360508in}{0.000000in}}%
\pgfpathlineto{\pgfqpoint{5.360508in}{3.676603in}}%
\pgfpathlineto{\pgfqpoint{0.000000in}{3.676603in}}%
\pgfpathclose%
\pgfusepath{fill}%
\end{pgfscope}%
\begin{pgfscope}%
\pgfsetbuttcap%
\pgfsetmiterjoin%
\definecolor{currentfill}{rgb}{1.000000,1.000000,1.000000}%
\pgfsetfillcolor{currentfill}%
\pgfsetlinewidth{0.000000pt}%
\definecolor{currentstroke}{rgb}{0.000000,0.000000,0.000000}%
\pgfsetstrokecolor{currentstroke}%
\pgfsetstrokeopacity{0.000000}%
\pgfsetdash{}{0pt}%
\pgfpathmoveto{\pgfqpoint{0.575508in}{0.521603in}}%
\pgfpathlineto{\pgfqpoint{5.225508in}{0.521603in}}%
\pgfpathlineto{\pgfqpoint{5.225508in}{3.541603in}}%
\pgfpathlineto{\pgfqpoint{0.575508in}{3.541603in}}%
\pgfpathclose%
\pgfusepath{fill}%
\end{pgfscope}%
\begin{pgfscope}%
\pgfpathrectangle{\pgfqpoint{0.575508in}{0.521603in}}{\pgfqpoint{4.650000in}{3.020000in}} %
\pgfusepath{clip}%
\pgfsetbuttcap%
\pgfsetroundjoin%
\definecolor{currentfill}{rgb}{1.000000,1.000000,1.000000}%
\pgfsetfillcolor{currentfill}%
\pgfsetlinewidth{1.003750pt}%
\definecolor{currentstroke}{rgb}{0.000000,0.000000,0.000000}%
\pgfsetstrokecolor{currentstroke}%
\pgfsetdash{}{0pt}%
\pgfpathmoveto{\pgfqpoint{0.876702in}{0.652871in}}%
\pgfpathcurveto{\pgfqpoint{0.887752in}{0.652871in}}{\pgfqpoint{0.898351in}{0.657261in}}{\pgfqpoint{0.906165in}{0.665075in}}%
\pgfpathcurveto{\pgfqpoint{0.913979in}{0.672888in}}{\pgfqpoint{0.918369in}{0.683487in}}{\pgfqpoint{0.918369in}{0.694537in}}%
\pgfpathcurveto{\pgfqpoint{0.918369in}{0.705588in}}{\pgfqpoint{0.913979in}{0.716187in}}{\pgfqpoint{0.906165in}{0.724000in}}%
\pgfpathcurveto{\pgfqpoint{0.898351in}{0.731814in}}{\pgfqpoint{0.887752in}{0.736204in}}{\pgfqpoint{0.876702in}{0.736204in}}%
\pgfpathcurveto{\pgfqpoint{0.865652in}{0.736204in}}{\pgfqpoint{0.855053in}{0.731814in}}{\pgfqpoint{0.847239in}{0.724000in}}%
\pgfpathcurveto{\pgfqpoint{0.839426in}{0.716187in}}{\pgfqpoint{0.835036in}{0.705588in}}{\pgfqpoint{0.835036in}{0.694537in}}%
\pgfpathcurveto{\pgfqpoint{0.835036in}{0.683487in}}{\pgfqpoint{0.839426in}{0.672888in}}{\pgfqpoint{0.847239in}{0.665075in}}%
\pgfpathcurveto{\pgfqpoint{0.855053in}{0.657261in}}{\pgfqpoint{0.865652in}{0.652871in}}{\pgfqpoint{0.876702in}{0.652871in}}%
\pgfpathclose%
\pgfusepath{stroke,fill}%
\end{pgfscope}%
\begin{pgfscope}%
\pgfpathrectangle{\pgfqpoint{0.575508in}{0.521603in}}{\pgfqpoint{4.650000in}{3.020000in}} %
\pgfusepath{clip}%
\pgfsetbuttcap%
\pgfsetroundjoin%
\definecolor{currentfill}{rgb}{1.000000,1.000000,1.000000}%
\pgfsetfillcolor{currentfill}%
\pgfsetlinewidth{1.003750pt}%
\definecolor{currentstroke}{rgb}{0.000000,0.000000,0.000000}%
\pgfsetstrokecolor{currentstroke}%
\pgfsetdash{}{0pt}%
\pgfpathmoveto{\pgfqpoint{0.876702in}{0.690201in}}%
\pgfpathcurveto{\pgfqpoint{0.887752in}{0.690201in}}{\pgfqpoint{0.898351in}{0.694591in}}{\pgfqpoint{0.906165in}{0.702405in}}%
\pgfpathcurveto{\pgfqpoint{0.913979in}{0.710218in}}{\pgfqpoint{0.918369in}{0.720818in}}{\pgfqpoint{0.918369in}{0.731868in}}%
\pgfpathcurveto{\pgfqpoint{0.918369in}{0.742918in}}{\pgfqpoint{0.913979in}{0.753517in}}{\pgfqpoint{0.906165in}{0.761330in}}%
\pgfpathcurveto{\pgfqpoint{0.898351in}{0.769144in}}{\pgfqpoint{0.887752in}{0.773534in}}{\pgfqpoint{0.876702in}{0.773534in}}%
\pgfpathcurveto{\pgfqpoint{0.865652in}{0.773534in}}{\pgfqpoint{0.855053in}{0.769144in}}{\pgfqpoint{0.847239in}{0.761330in}}%
\pgfpathcurveto{\pgfqpoint{0.839426in}{0.753517in}}{\pgfqpoint{0.835036in}{0.742918in}}{\pgfqpoint{0.835036in}{0.731868in}}%
\pgfpathcurveto{\pgfqpoint{0.835036in}{0.720818in}}{\pgfqpoint{0.839426in}{0.710218in}}{\pgfqpoint{0.847239in}{0.702405in}}%
\pgfpathcurveto{\pgfqpoint{0.855053in}{0.694591in}}{\pgfqpoint{0.865652in}{0.690201in}}{\pgfqpoint{0.876702in}{0.690201in}}%
\pgfpathclose%
\pgfusepath{stroke,fill}%
\end{pgfscope}%
\begin{pgfscope}%
\pgfpathrectangle{\pgfqpoint{0.575508in}{0.521603in}}{\pgfqpoint{4.650000in}{3.020000in}} %
\pgfusepath{clip}%
\pgfsetbuttcap%
\pgfsetroundjoin%
\definecolor{currentfill}{rgb}{1.000000,1.000000,1.000000}%
\pgfsetfillcolor{currentfill}%
\pgfsetlinewidth{1.003750pt}%
\definecolor{currentstroke}{rgb}{0.000000,0.000000,0.000000}%
\pgfsetstrokecolor{currentstroke}%
\pgfsetdash{}{0pt}%
\pgfpathmoveto{\pgfqpoint{0.811940in}{0.739534in}}%
\pgfpathcurveto{\pgfqpoint{0.822991in}{0.739534in}}{\pgfqpoint{0.833590in}{0.743925in}}{\pgfqpoint{0.841403in}{0.751738in}}%
\pgfpathcurveto{\pgfqpoint{0.849217in}{0.759552in}}{\pgfqpoint{0.853607in}{0.770151in}}{\pgfqpoint{0.853607in}{0.781201in}}%
\pgfpathcurveto{\pgfqpoint{0.853607in}{0.792251in}}{\pgfqpoint{0.849217in}{0.802850in}}{\pgfqpoint{0.841403in}{0.810664in}}%
\pgfpathcurveto{\pgfqpoint{0.833590in}{0.818478in}}{\pgfqpoint{0.822991in}{0.822868in}}{\pgfqpoint{0.811940in}{0.822868in}}%
\pgfpathcurveto{\pgfqpoint{0.800890in}{0.822868in}}{\pgfqpoint{0.790291in}{0.818478in}}{\pgfqpoint{0.782478in}{0.810664in}}%
\pgfpathcurveto{\pgfqpoint{0.774664in}{0.802850in}}{\pgfqpoint{0.770274in}{0.792251in}}{\pgfqpoint{0.770274in}{0.781201in}}%
\pgfpathcurveto{\pgfqpoint{0.770274in}{0.770151in}}{\pgfqpoint{0.774664in}{0.759552in}}{\pgfqpoint{0.782478in}{0.751738in}}%
\pgfpathcurveto{\pgfqpoint{0.790291in}{0.743925in}}{\pgfqpoint{0.800890in}{0.739534in}}{\pgfqpoint{0.811940in}{0.739534in}}%
\pgfpathclose%
\pgfusepath{stroke,fill}%
\end{pgfscope}%
\begin{pgfscope}%
\pgfpathrectangle{\pgfqpoint{0.575508in}{0.521603in}}{\pgfqpoint{4.650000in}{3.020000in}} %
\pgfusepath{clip}%
\pgfsetbuttcap%
\pgfsetroundjoin%
\definecolor{currentfill}{rgb}{1.000000,1.000000,1.000000}%
\pgfsetfillcolor{currentfill}%
\pgfsetlinewidth{1.003750pt}%
\definecolor{currentstroke}{rgb}{0.000000,0.000000,0.000000}%
\pgfsetstrokecolor{currentstroke}%
\pgfsetdash{}{0pt}%
\pgfpathmoveto{\pgfqpoint{0.876702in}{0.807081in}}%
\pgfpathcurveto{\pgfqpoint{0.887752in}{0.807081in}}{\pgfqpoint{0.898351in}{0.811471in}}{\pgfqpoint{0.906165in}{0.819285in}}%
\pgfpathcurveto{\pgfqpoint{0.913979in}{0.827098in}}{\pgfqpoint{0.918369in}{0.837697in}}{\pgfqpoint{0.918369in}{0.848748in}}%
\pgfpathcurveto{\pgfqpoint{0.918369in}{0.859798in}}{\pgfqpoint{0.913979in}{0.870397in}}{\pgfqpoint{0.906165in}{0.878210in}}%
\pgfpathcurveto{\pgfqpoint{0.898351in}{0.886024in}}{\pgfqpoint{0.887752in}{0.890414in}}{\pgfqpoint{0.876702in}{0.890414in}}%
\pgfpathcurveto{\pgfqpoint{0.865652in}{0.890414in}}{\pgfqpoint{0.855053in}{0.886024in}}{\pgfqpoint{0.847239in}{0.878210in}}%
\pgfpathcurveto{\pgfqpoint{0.839426in}{0.870397in}}{\pgfqpoint{0.835036in}{0.859798in}}{\pgfqpoint{0.835036in}{0.848748in}}%
\pgfpathcurveto{\pgfqpoint{0.835036in}{0.837697in}}{\pgfqpoint{0.839426in}{0.827098in}}{\pgfqpoint{0.847239in}{0.819285in}}%
\pgfpathcurveto{\pgfqpoint{0.855053in}{0.811471in}}{\pgfqpoint{0.865652in}{0.807081in}}{\pgfqpoint{0.876702in}{0.807081in}}%
\pgfpathclose%
\pgfusepath{stroke,fill}%
\end{pgfscope}%
\begin{pgfscope}%
\pgfpathrectangle{\pgfqpoint{0.575508in}{0.521603in}}{\pgfqpoint{4.650000in}{3.020000in}} %
\pgfusepath{clip}%
\pgfsetbuttcap%
\pgfsetroundjoin%
\definecolor{currentfill}{rgb}{1.000000,1.000000,1.000000}%
\pgfsetfillcolor{currentfill}%
\pgfsetlinewidth{1.003750pt}%
\definecolor{currentstroke}{rgb}{0.000000,0.000000,0.000000}%
\pgfsetstrokecolor{currentstroke}%
\pgfsetdash{}{0pt}%
\pgfpathmoveto{\pgfqpoint{1.589082in}{1.189637in}}%
\pgfpathcurveto{\pgfqpoint{1.600132in}{1.189637in}}{\pgfqpoint{1.610731in}{1.194027in}}{\pgfqpoint{1.618545in}{1.201841in}}%
\pgfpathcurveto{\pgfqpoint{1.626358in}{1.209655in}}{\pgfqpoint{1.630748in}{1.220254in}}{\pgfqpoint{1.630748in}{1.231304in}}%
\pgfpathcurveto{\pgfqpoint{1.630748in}{1.242354in}}{\pgfqpoint{1.626358in}{1.252953in}}{\pgfqpoint{1.618545in}{1.260767in}}%
\pgfpathcurveto{\pgfqpoint{1.610731in}{1.268580in}}{\pgfqpoint{1.600132in}{1.272970in}}{\pgfqpoint{1.589082in}{1.272970in}}%
\pgfpathcurveto{\pgfqpoint{1.578032in}{1.272970in}}{\pgfqpoint{1.567433in}{1.268580in}}{\pgfqpoint{1.559619in}{1.260767in}}%
\pgfpathcurveto{\pgfqpoint{1.551805in}{1.252953in}}{\pgfqpoint{1.547415in}{1.242354in}}{\pgfqpoint{1.547415in}{1.231304in}}%
\pgfpathcurveto{\pgfqpoint{1.547415in}{1.220254in}}{\pgfqpoint{1.551805in}{1.209655in}}{\pgfqpoint{1.559619in}{1.201841in}}%
\pgfpathcurveto{\pgfqpoint{1.567433in}{1.194027in}}{\pgfqpoint{1.578032in}{1.189637in}}{\pgfqpoint{1.589082in}{1.189637in}}%
\pgfpathclose%
\pgfusepath{stroke,fill}%
\end{pgfscope}%
\begin{pgfscope}%
\pgfpathrectangle{\pgfqpoint{0.575508in}{0.521603in}}{\pgfqpoint{4.650000in}{3.020000in}} %
\pgfusepath{clip}%
\pgfsetbuttcap%
\pgfsetroundjoin%
\definecolor{currentfill}{rgb}{1.000000,1.000000,1.000000}%
\pgfsetfillcolor{currentfill}%
\pgfsetlinewidth{1.003750pt}%
\definecolor{currentstroke}{rgb}{0.000000,0.000000,0.000000}%
\pgfsetstrokecolor{currentstroke}%
\pgfsetdash{}{0pt}%
\pgfpathmoveto{\pgfqpoint{1.783367in}{1.230970in}}%
\pgfpathcurveto{\pgfqpoint{1.794417in}{1.230970in}}{\pgfqpoint{1.805016in}{1.235361in}}{\pgfqpoint{1.812830in}{1.243174in}}%
\pgfpathcurveto{\pgfqpoint{1.820644in}{1.250988in}}{\pgfqpoint{1.825034in}{1.261587in}}{\pgfqpoint{1.825034in}{1.272637in}}%
\pgfpathcurveto{\pgfqpoint{1.825034in}{1.283687in}}{\pgfqpoint{1.820644in}{1.294286in}}{\pgfqpoint{1.812830in}{1.302100in}}%
\pgfpathcurveto{\pgfqpoint{1.805016in}{1.309913in}}{\pgfqpoint{1.794417in}{1.314304in}}{\pgfqpoint{1.783367in}{1.314304in}}%
\pgfpathcurveto{\pgfqpoint{1.772317in}{1.314304in}}{\pgfqpoint{1.761718in}{1.309913in}}{\pgfqpoint{1.753904in}{1.302100in}}%
\pgfpathcurveto{\pgfqpoint{1.746091in}{1.294286in}}{\pgfqpoint{1.741700in}{1.283687in}}{\pgfqpoint{1.741700in}{1.272637in}}%
\pgfpathcurveto{\pgfqpoint{1.741700in}{1.261587in}}{\pgfqpoint{1.746091in}{1.250988in}}{\pgfqpoint{1.753904in}{1.243174in}}%
\pgfpathcurveto{\pgfqpoint{1.761718in}{1.235361in}}{\pgfqpoint{1.772317in}{1.230970in}}{\pgfqpoint{1.783367in}{1.230970in}}%
\pgfpathclose%
\pgfusepath{stroke,fill}%
\end{pgfscope}%
\begin{pgfscope}%
\pgfpathrectangle{\pgfqpoint{0.575508in}{0.521603in}}{\pgfqpoint{4.650000in}{3.020000in}} %
\pgfusepath{clip}%
\pgfsetbuttcap%
\pgfsetroundjoin%
\definecolor{currentfill}{rgb}{1.000000,1.000000,1.000000}%
\pgfsetfillcolor{currentfill}%
\pgfsetlinewidth{1.003750pt}%
\definecolor{currentstroke}{rgb}{0.000000,0.000000,0.000000}%
\pgfsetstrokecolor{currentstroke}%
\pgfsetdash{}{0pt}%
\pgfpathmoveto{\pgfqpoint{1.621463in}{1.241729in}}%
\pgfpathcurveto{\pgfqpoint{1.632513in}{1.241729in}}{\pgfqpoint{1.643112in}{1.246119in}}{\pgfqpoint{1.650925in}{1.253933in}}%
\pgfpathcurveto{\pgfqpoint{1.658739in}{1.261746in}}{\pgfqpoint{1.663129in}{1.272345in}}{\pgfqpoint{1.663129in}{1.283395in}}%
\pgfpathcurveto{\pgfqpoint{1.663129in}{1.294445in}}{\pgfqpoint{1.658739in}{1.305044in}}{\pgfqpoint{1.650925in}{1.312858in}}%
\pgfpathcurveto{\pgfqpoint{1.643112in}{1.320672in}}{\pgfqpoint{1.632513in}{1.325062in}}{\pgfqpoint{1.621463in}{1.325062in}}%
\pgfpathcurveto{\pgfqpoint{1.610413in}{1.325062in}}{\pgfqpoint{1.599814in}{1.320672in}}{\pgfqpoint{1.592000in}{1.312858in}}%
\pgfpathcurveto{\pgfqpoint{1.584186in}{1.305044in}}{\pgfqpoint{1.579796in}{1.294445in}}{\pgfqpoint{1.579796in}{1.283395in}}%
\pgfpathcurveto{\pgfqpoint{1.579796in}{1.272345in}}{\pgfqpoint{1.584186in}{1.261746in}}{\pgfqpoint{1.592000in}{1.253933in}}%
\pgfpathcurveto{\pgfqpoint{1.599814in}{1.246119in}}{\pgfqpoint{1.610413in}{1.241729in}}{\pgfqpoint{1.621463in}{1.241729in}}%
\pgfpathclose%
\pgfusepath{stroke,fill}%
\end{pgfscope}%
\begin{pgfscope}%
\pgfpathrectangle{\pgfqpoint{0.575508in}{0.521603in}}{\pgfqpoint{4.650000in}{3.020000in}} %
\pgfusepath{clip}%
\pgfsetbuttcap%
\pgfsetroundjoin%
\definecolor{currentfill}{rgb}{1.000000,1.000000,1.000000}%
\pgfsetfillcolor{currentfill}%
\pgfsetlinewidth{1.003750pt}%
\definecolor{currentstroke}{rgb}{0.000000,0.000000,0.000000}%
\pgfsetstrokecolor{currentstroke}%
\pgfsetdash{}{0pt}%
\pgfpathmoveto{\pgfqpoint{4.989075in}{3.327003in}}%
\pgfpathcurveto{\pgfqpoint{5.000125in}{3.327003in}}{\pgfqpoint{5.010724in}{3.331393in}}{\pgfqpoint{5.018538in}{3.339206in}}%
\pgfpathcurveto{\pgfqpoint{5.026351in}{3.347020in}}{\pgfqpoint{5.030742in}{3.357619in}}{\pgfqpoint{5.030742in}{3.368669in}}%
\pgfpathcurveto{\pgfqpoint{5.030742in}{3.379719in}}{\pgfqpoint{5.026351in}{3.390318in}}{\pgfqpoint{5.018538in}{3.398132in}}%
\pgfpathcurveto{\pgfqpoint{5.010724in}{3.405946in}}{\pgfqpoint{5.000125in}{3.410336in}}{\pgfqpoint{4.989075in}{3.410336in}}%
\pgfpathcurveto{\pgfqpoint{4.978025in}{3.410336in}}{\pgfqpoint{4.967426in}{3.405946in}}{\pgfqpoint{4.959612in}{3.398132in}}%
\pgfpathcurveto{\pgfqpoint{4.951799in}{3.390318in}}{\pgfqpoint{4.947408in}{3.379719in}}{\pgfqpoint{4.947408in}{3.368669in}}%
\pgfpathcurveto{\pgfqpoint{4.947408in}{3.357619in}}{\pgfqpoint{4.951799in}{3.347020in}}{\pgfqpoint{4.959612in}{3.339206in}}%
\pgfpathcurveto{\pgfqpoint{4.967426in}{3.331393in}}{\pgfqpoint{4.978025in}{3.327003in}}{\pgfqpoint{4.989075in}{3.327003in}}%
\pgfpathclose%
\pgfusepath{stroke,fill}%
\end{pgfscope}%
\begin{pgfscope}%
\pgfpathrectangle{\pgfqpoint{0.575508in}{0.521603in}}{\pgfqpoint{4.650000in}{3.020000in}} %
\pgfusepath{clip}%
\pgfsetbuttcap%
\pgfsetroundjoin%
\definecolor{currentfill}{rgb}{1.000000,1.000000,1.000000}%
\pgfsetfillcolor{currentfill}%
\pgfsetlinewidth{1.003750pt}%
\definecolor{currentstroke}{rgb}{0.000000,0.000000,0.000000}%
\pgfsetstrokecolor{currentstroke}%
\pgfsetdash{}{0pt}%
\pgfpathmoveto{\pgfqpoint{4.730028in}{3.200774in}}%
\pgfpathcurveto{\pgfqpoint{4.741078in}{3.200774in}}{\pgfqpoint{4.751677in}{3.205165in}}{\pgfqpoint{4.759491in}{3.212978in}}%
\pgfpathcurveto{\pgfqpoint{4.767304in}{3.220792in}}{\pgfqpoint{4.771695in}{3.231391in}}{\pgfqpoint{4.771695in}{3.242441in}}%
\pgfpathcurveto{\pgfqpoint{4.771695in}{3.253491in}}{\pgfqpoint{4.767304in}{3.264090in}}{\pgfqpoint{4.759491in}{3.271904in}}%
\pgfpathcurveto{\pgfqpoint{4.751677in}{3.279718in}}{\pgfqpoint{4.741078in}{3.284108in}}{\pgfqpoint{4.730028in}{3.284108in}}%
\pgfpathcurveto{\pgfqpoint{4.718978in}{3.284108in}}{\pgfqpoint{4.708379in}{3.279718in}}{\pgfqpoint{4.700565in}{3.271904in}}%
\pgfpathcurveto{\pgfqpoint{4.692752in}{3.264090in}}{\pgfqpoint{4.688361in}{3.253491in}}{\pgfqpoint{4.688361in}{3.242441in}}%
\pgfpathcurveto{\pgfqpoint{4.688361in}{3.231391in}}{\pgfqpoint{4.692752in}{3.220792in}}{\pgfqpoint{4.700565in}{3.212978in}}%
\pgfpathcurveto{\pgfqpoint{4.708379in}{3.205165in}}{\pgfqpoint{4.718978in}{3.200774in}}{\pgfqpoint{4.730028in}{3.200774in}}%
\pgfpathclose%
\pgfusepath{stroke,fill}%
\end{pgfscope}%
\begin{pgfscope}%
\pgfpathrectangle{\pgfqpoint{0.575508in}{0.521603in}}{\pgfqpoint{4.650000in}{3.020000in}} %
\pgfusepath{clip}%
\pgfsetbuttcap%
\pgfsetroundjoin%
\definecolor{currentfill}{rgb}{1.000000,1.000000,1.000000}%
\pgfsetfillcolor{currentfill}%
\pgfsetlinewidth{1.003750pt}%
\definecolor{currentstroke}{rgb}{0.000000,0.000000,0.000000}%
\pgfsetstrokecolor{currentstroke}%
\pgfsetdash{}{0pt}%
\pgfpathmoveto{\pgfqpoint{3.208126in}{2.165670in}}%
\pgfpathcurveto{\pgfqpoint{3.219176in}{2.165670in}}{\pgfqpoint{3.229775in}{2.170060in}}{\pgfqpoint{3.237589in}{2.177874in}}%
\pgfpathcurveto{\pgfqpoint{3.245403in}{2.185687in}}{\pgfqpoint{3.249793in}{2.196286in}}{\pgfqpoint{3.249793in}{2.207337in}}%
\pgfpathcurveto{\pgfqpoint{3.249793in}{2.218387in}}{\pgfqpoint{3.245403in}{2.228986in}}{\pgfqpoint{3.237589in}{2.236799in}}%
\pgfpathcurveto{\pgfqpoint{3.229775in}{2.244613in}}{\pgfqpoint{3.219176in}{2.249003in}}{\pgfqpoint{3.208126in}{2.249003in}}%
\pgfpathcurveto{\pgfqpoint{3.197076in}{2.249003in}}{\pgfqpoint{3.186477in}{2.244613in}}{\pgfqpoint{3.178663in}{2.236799in}}%
\pgfpathcurveto{\pgfqpoint{3.170850in}{2.228986in}}{\pgfqpoint{3.166460in}{2.218387in}}{\pgfqpoint{3.166460in}{2.207337in}}%
\pgfpathcurveto{\pgfqpoint{3.166460in}{2.196286in}}{\pgfqpoint{3.170850in}{2.185687in}}{\pgfqpoint{3.178663in}{2.177874in}}%
\pgfpathcurveto{\pgfqpoint{3.186477in}{2.170060in}}{\pgfqpoint{3.197076in}{2.165670in}}{\pgfqpoint{3.208126in}{2.165670in}}%
\pgfpathclose%
\pgfusepath{stroke,fill}%
\end{pgfscope}%
\begin{pgfscope}%
\pgfsetbuttcap%
\pgfsetroundjoin%
\definecolor{currentfill}{rgb}{0.000000,0.000000,0.000000}%
\pgfsetfillcolor{currentfill}%
\pgfsetlinewidth{0.803000pt}%
\definecolor{currentstroke}{rgb}{0.000000,0.000000,0.000000}%
\pgfsetstrokecolor{currentstroke}%
\pgfsetdash{}{0pt}%
\pgfsys@defobject{currentmarker}{\pgfqpoint{0.000000in}{-0.048611in}}{\pgfqpoint{0.000000in}{0.000000in}}{%
\pgfpathmoveto{\pgfqpoint{0.000000in}{0.000000in}}%
\pgfpathlineto{\pgfqpoint{0.000000in}{-0.048611in}}%
\pgfusepath{stroke,fill}%
}%
\begin{pgfscope}%
\pgfsys@transformshift{1.135749in}{0.521603in}%
\pgfsys@useobject{currentmarker}{}%
\end{pgfscope}%
\end{pgfscope}%
\begin{pgfscope}%
\pgftext[x=1.135749in,y=0.424381in,,top]{\rmfamily\fontsize{10.000000}{12.000000}\selectfont \(\displaystyle 0.4\)}%
\end{pgfscope}%
\begin{pgfscope}%
\pgfsetbuttcap%
\pgfsetroundjoin%
\definecolor{currentfill}{rgb}{0.000000,0.000000,0.000000}%
\pgfsetfillcolor{currentfill}%
\pgfsetlinewidth{0.803000pt}%
\definecolor{currentstroke}{rgb}{0.000000,0.000000,0.000000}%
\pgfsetstrokecolor{currentstroke}%
\pgfsetdash{}{0pt}%
\pgfsys@defobject{currentmarker}{\pgfqpoint{0.000000in}{-0.048611in}}{\pgfqpoint{0.000000in}{0.000000in}}{%
\pgfpathmoveto{\pgfqpoint{0.000000in}{0.000000in}}%
\pgfpathlineto{\pgfqpoint{0.000000in}{-0.048611in}}%
\pgfusepath{stroke,fill}%
}%
\begin{pgfscope}%
\pgfsys@transformshift{1.912891in}{0.521603in}%
\pgfsys@useobject{currentmarker}{}%
\end{pgfscope}%
\end{pgfscope}%
\begin{pgfscope}%
\pgftext[x=1.912891in,y=0.424381in,,top]{\rmfamily\fontsize{10.000000}{12.000000}\selectfont \(\displaystyle 0.6\)}%
\end{pgfscope}%
\begin{pgfscope}%
\pgfsetbuttcap%
\pgfsetroundjoin%
\definecolor{currentfill}{rgb}{0.000000,0.000000,0.000000}%
\pgfsetfillcolor{currentfill}%
\pgfsetlinewidth{0.803000pt}%
\definecolor{currentstroke}{rgb}{0.000000,0.000000,0.000000}%
\pgfsetstrokecolor{currentstroke}%
\pgfsetdash{}{0pt}%
\pgfsys@defobject{currentmarker}{\pgfqpoint{0.000000in}{-0.048611in}}{\pgfqpoint{0.000000in}{0.000000in}}{%
\pgfpathmoveto{\pgfqpoint{0.000000in}{0.000000in}}%
\pgfpathlineto{\pgfqpoint{0.000000in}{-0.048611in}}%
\pgfusepath{stroke,fill}%
}%
\begin{pgfscope}%
\pgfsys@transformshift{2.690032in}{0.521603in}%
\pgfsys@useobject{currentmarker}{}%
\end{pgfscope}%
\end{pgfscope}%
\begin{pgfscope}%
\pgftext[x=2.690032in,y=0.424381in,,top]{\rmfamily\fontsize{10.000000}{12.000000}\selectfont \(\displaystyle 0.8\)}%
\end{pgfscope}%
\begin{pgfscope}%
\pgfsetbuttcap%
\pgfsetroundjoin%
\definecolor{currentfill}{rgb}{0.000000,0.000000,0.000000}%
\pgfsetfillcolor{currentfill}%
\pgfsetlinewidth{0.803000pt}%
\definecolor{currentstroke}{rgb}{0.000000,0.000000,0.000000}%
\pgfsetstrokecolor{currentstroke}%
\pgfsetdash{}{0pt}%
\pgfsys@defobject{currentmarker}{\pgfqpoint{0.000000in}{-0.048611in}}{\pgfqpoint{0.000000in}{0.000000in}}{%
\pgfpathmoveto{\pgfqpoint{0.000000in}{0.000000in}}%
\pgfpathlineto{\pgfqpoint{0.000000in}{-0.048611in}}%
\pgfusepath{stroke,fill}%
}%
\begin{pgfscope}%
\pgfsys@transformshift{3.467173in}{0.521603in}%
\pgfsys@useobject{currentmarker}{}%
\end{pgfscope}%
\end{pgfscope}%
\begin{pgfscope}%
\pgftext[x=3.467173in,y=0.424381in,,top]{\rmfamily\fontsize{10.000000}{12.000000}\selectfont \(\displaystyle 1.0\)}%
\end{pgfscope}%
\begin{pgfscope}%
\pgfsetbuttcap%
\pgfsetroundjoin%
\definecolor{currentfill}{rgb}{0.000000,0.000000,0.000000}%
\pgfsetfillcolor{currentfill}%
\pgfsetlinewidth{0.803000pt}%
\definecolor{currentstroke}{rgb}{0.000000,0.000000,0.000000}%
\pgfsetstrokecolor{currentstroke}%
\pgfsetdash{}{0pt}%
\pgfsys@defobject{currentmarker}{\pgfqpoint{0.000000in}{-0.048611in}}{\pgfqpoint{0.000000in}{0.000000in}}{%
\pgfpathmoveto{\pgfqpoint{0.000000in}{0.000000in}}%
\pgfpathlineto{\pgfqpoint{0.000000in}{-0.048611in}}%
\pgfusepath{stroke,fill}%
}%
\begin{pgfscope}%
\pgfsys@transformshift{4.244315in}{0.521603in}%
\pgfsys@useobject{currentmarker}{}%
\end{pgfscope}%
\end{pgfscope}%
\begin{pgfscope}%
\pgftext[x=4.244315in,y=0.424381in,,top]{\rmfamily\fontsize{10.000000}{12.000000}\selectfont \(\displaystyle 1.2\)}%
\end{pgfscope}%
\begin{pgfscope}%
\pgfsetbuttcap%
\pgfsetroundjoin%
\definecolor{currentfill}{rgb}{0.000000,0.000000,0.000000}%
\pgfsetfillcolor{currentfill}%
\pgfsetlinewidth{0.803000pt}%
\definecolor{currentstroke}{rgb}{0.000000,0.000000,0.000000}%
\pgfsetstrokecolor{currentstroke}%
\pgfsetdash{}{0pt}%
\pgfsys@defobject{currentmarker}{\pgfqpoint{0.000000in}{-0.048611in}}{\pgfqpoint{0.000000in}{0.000000in}}{%
\pgfpathmoveto{\pgfqpoint{0.000000in}{0.000000in}}%
\pgfpathlineto{\pgfqpoint{0.000000in}{-0.048611in}}%
\pgfusepath{stroke,fill}%
}%
\begin{pgfscope}%
\pgfsys@transformshift{5.021456in}{0.521603in}%
\pgfsys@useobject{currentmarker}{}%
\end{pgfscope}%
\end{pgfscope}%
\begin{pgfscope}%
\pgftext[x=5.021456in,y=0.424381in,,top]{\rmfamily\fontsize{10.000000}{12.000000}\selectfont \(\displaystyle 1.4\)}%
\end{pgfscope}%
\begin{pgfscope}%
\pgftext[x=2.900508in,y=0.234413in,,top]{\rmfamily\fontsize{10.000000}{12.000000}\selectfont \(\displaystyle {t_b}\) (s)}%
\end{pgfscope}%
\begin{pgfscope}%
\pgfsetbuttcap%
\pgfsetroundjoin%
\definecolor{currentfill}{rgb}{0.000000,0.000000,0.000000}%
\pgfsetfillcolor{currentfill}%
\pgfsetlinewidth{0.803000pt}%
\definecolor{currentstroke}{rgb}{0.000000,0.000000,0.000000}%
\pgfsetstrokecolor{currentstroke}%
\pgfsetdash{}{0pt}%
\pgfsys@defobject{currentmarker}{\pgfqpoint{-0.048611in}{0.000000in}}{\pgfqpoint{0.000000in}{0.000000in}}{%
\pgfpathmoveto{\pgfqpoint{0.000000in}{0.000000in}}%
\pgfpathlineto{\pgfqpoint{-0.048611in}{0.000000in}}%
\pgfusepath{stroke,fill}%
}%
\begin{pgfscope}%
\pgfsys@transformshift{0.575508in}{0.542723in}%
\pgfsys@useobject{currentmarker}{}%
\end{pgfscope}%
\end{pgfscope}%
\begin{pgfscope}%
\pgftext[x=0.300816in,y=0.489962in,left,base]{\rmfamily\fontsize{10.000000}{12.000000}\selectfont \(\displaystyle 0.2\)}%
\end{pgfscope}%
\begin{pgfscope}%
\pgfsetbuttcap%
\pgfsetroundjoin%
\definecolor{currentfill}{rgb}{0.000000,0.000000,0.000000}%
\pgfsetfillcolor{currentfill}%
\pgfsetlinewidth{0.803000pt}%
\definecolor{currentstroke}{rgb}{0.000000,0.000000,0.000000}%
\pgfsetstrokecolor{currentstroke}%
\pgfsetdash{}{0pt}%
\pgfsys@defobject{currentmarker}{\pgfqpoint{-0.048611in}{0.000000in}}{\pgfqpoint{0.000000in}{0.000000in}}{%
\pgfpathmoveto{\pgfqpoint{0.000000in}{0.000000in}}%
\pgfpathlineto{\pgfqpoint{-0.048611in}{0.000000in}}%
\pgfusepath{stroke,fill}%
}%
\begin{pgfscope}%
\pgfsys@transformshift{0.575508in}{0.901714in}%
\pgfsys@useobject{currentmarker}{}%
\end{pgfscope}%
\end{pgfscope}%
\begin{pgfscope}%
\pgftext[x=0.300816in,y=0.848953in,left,base]{\rmfamily\fontsize{10.000000}{12.000000}\selectfont \(\displaystyle 0.3\)}%
\end{pgfscope}%
\begin{pgfscope}%
\pgfsetbuttcap%
\pgfsetroundjoin%
\definecolor{currentfill}{rgb}{0.000000,0.000000,0.000000}%
\pgfsetfillcolor{currentfill}%
\pgfsetlinewidth{0.803000pt}%
\definecolor{currentstroke}{rgb}{0.000000,0.000000,0.000000}%
\pgfsetstrokecolor{currentstroke}%
\pgfsetdash{}{0pt}%
\pgfsys@defobject{currentmarker}{\pgfqpoint{-0.048611in}{0.000000in}}{\pgfqpoint{0.000000in}{0.000000in}}{%
\pgfpathmoveto{\pgfqpoint{0.000000in}{0.000000in}}%
\pgfpathlineto{\pgfqpoint{-0.048611in}{0.000000in}}%
\pgfusepath{stroke,fill}%
}%
\begin{pgfscope}%
\pgfsys@transformshift{0.575508in}{1.260705in}%
\pgfsys@useobject{currentmarker}{}%
\end{pgfscope}%
\end{pgfscope}%
\begin{pgfscope}%
\pgftext[x=0.300816in,y=1.207944in,left,base]{\rmfamily\fontsize{10.000000}{12.000000}\selectfont \(\displaystyle 0.4\)}%
\end{pgfscope}%
\begin{pgfscope}%
\pgfsetbuttcap%
\pgfsetroundjoin%
\definecolor{currentfill}{rgb}{0.000000,0.000000,0.000000}%
\pgfsetfillcolor{currentfill}%
\pgfsetlinewidth{0.803000pt}%
\definecolor{currentstroke}{rgb}{0.000000,0.000000,0.000000}%
\pgfsetstrokecolor{currentstroke}%
\pgfsetdash{}{0pt}%
\pgfsys@defobject{currentmarker}{\pgfqpoint{-0.048611in}{0.000000in}}{\pgfqpoint{0.000000in}{0.000000in}}{%
\pgfpathmoveto{\pgfqpoint{0.000000in}{0.000000in}}%
\pgfpathlineto{\pgfqpoint{-0.048611in}{0.000000in}}%
\pgfusepath{stroke,fill}%
}%
\begin{pgfscope}%
\pgfsys@transformshift{0.575508in}{1.619697in}%
\pgfsys@useobject{currentmarker}{}%
\end{pgfscope}%
\end{pgfscope}%
\begin{pgfscope}%
\pgftext[x=0.300816in,y=1.566935in,left,base]{\rmfamily\fontsize{10.000000}{12.000000}\selectfont \(\displaystyle 0.5\)}%
\end{pgfscope}%
\begin{pgfscope}%
\pgfsetbuttcap%
\pgfsetroundjoin%
\definecolor{currentfill}{rgb}{0.000000,0.000000,0.000000}%
\pgfsetfillcolor{currentfill}%
\pgfsetlinewidth{0.803000pt}%
\definecolor{currentstroke}{rgb}{0.000000,0.000000,0.000000}%
\pgfsetstrokecolor{currentstroke}%
\pgfsetdash{}{0pt}%
\pgfsys@defobject{currentmarker}{\pgfqpoint{-0.048611in}{0.000000in}}{\pgfqpoint{0.000000in}{0.000000in}}{%
\pgfpathmoveto{\pgfqpoint{0.000000in}{0.000000in}}%
\pgfpathlineto{\pgfqpoint{-0.048611in}{0.000000in}}%
\pgfusepath{stroke,fill}%
}%
\begin{pgfscope}%
\pgfsys@transformshift{0.575508in}{1.978688in}%
\pgfsys@useobject{currentmarker}{}%
\end{pgfscope}%
\end{pgfscope}%
\begin{pgfscope}%
\pgftext[x=0.300816in,y=1.925926in,left,base]{\rmfamily\fontsize{10.000000}{12.000000}\selectfont \(\displaystyle 0.6\)}%
\end{pgfscope}%
\begin{pgfscope}%
\pgfsetbuttcap%
\pgfsetroundjoin%
\definecolor{currentfill}{rgb}{0.000000,0.000000,0.000000}%
\pgfsetfillcolor{currentfill}%
\pgfsetlinewidth{0.803000pt}%
\definecolor{currentstroke}{rgb}{0.000000,0.000000,0.000000}%
\pgfsetstrokecolor{currentstroke}%
\pgfsetdash{}{0pt}%
\pgfsys@defobject{currentmarker}{\pgfqpoint{-0.048611in}{0.000000in}}{\pgfqpoint{0.000000in}{0.000000in}}{%
\pgfpathmoveto{\pgfqpoint{0.000000in}{0.000000in}}%
\pgfpathlineto{\pgfqpoint{-0.048611in}{0.000000in}}%
\pgfusepath{stroke,fill}%
}%
\begin{pgfscope}%
\pgfsys@transformshift{0.575508in}{2.337679in}%
\pgfsys@useobject{currentmarker}{}%
\end{pgfscope}%
\end{pgfscope}%
\begin{pgfscope}%
\pgftext[x=0.300816in,y=2.284918in,left,base]{\rmfamily\fontsize{10.000000}{12.000000}\selectfont \(\displaystyle 0.7\)}%
\end{pgfscope}%
\begin{pgfscope}%
\pgfsetbuttcap%
\pgfsetroundjoin%
\definecolor{currentfill}{rgb}{0.000000,0.000000,0.000000}%
\pgfsetfillcolor{currentfill}%
\pgfsetlinewidth{0.803000pt}%
\definecolor{currentstroke}{rgb}{0.000000,0.000000,0.000000}%
\pgfsetstrokecolor{currentstroke}%
\pgfsetdash{}{0pt}%
\pgfsys@defobject{currentmarker}{\pgfqpoint{-0.048611in}{0.000000in}}{\pgfqpoint{0.000000in}{0.000000in}}{%
\pgfpathmoveto{\pgfqpoint{0.000000in}{0.000000in}}%
\pgfpathlineto{\pgfqpoint{-0.048611in}{0.000000in}}%
\pgfusepath{stroke,fill}%
}%
\begin{pgfscope}%
\pgfsys@transformshift{0.575508in}{2.696670in}%
\pgfsys@useobject{currentmarker}{}%
\end{pgfscope}%
\end{pgfscope}%
\begin{pgfscope}%
\pgftext[x=0.300816in,y=2.643909in,left,base]{\rmfamily\fontsize{10.000000}{12.000000}\selectfont \(\displaystyle 0.8\)}%
\end{pgfscope}%
\begin{pgfscope}%
\pgfsetbuttcap%
\pgfsetroundjoin%
\definecolor{currentfill}{rgb}{0.000000,0.000000,0.000000}%
\pgfsetfillcolor{currentfill}%
\pgfsetlinewidth{0.803000pt}%
\definecolor{currentstroke}{rgb}{0.000000,0.000000,0.000000}%
\pgfsetstrokecolor{currentstroke}%
\pgfsetdash{}{0pt}%
\pgfsys@defobject{currentmarker}{\pgfqpoint{-0.048611in}{0.000000in}}{\pgfqpoint{0.000000in}{0.000000in}}{%
\pgfpathmoveto{\pgfqpoint{0.000000in}{0.000000in}}%
\pgfpathlineto{\pgfqpoint{-0.048611in}{0.000000in}}%
\pgfusepath{stroke,fill}%
}%
\begin{pgfscope}%
\pgfsys@transformshift{0.575508in}{3.055661in}%
\pgfsys@useobject{currentmarker}{}%
\end{pgfscope}%
\end{pgfscope}%
\begin{pgfscope}%
\pgftext[x=0.300816in,y=3.002900in,left,base]{\rmfamily\fontsize{10.000000}{12.000000}\selectfont \(\displaystyle 0.9\)}%
\end{pgfscope}%
\begin{pgfscope}%
\pgfsetbuttcap%
\pgfsetroundjoin%
\definecolor{currentfill}{rgb}{0.000000,0.000000,0.000000}%
\pgfsetfillcolor{currentfill}%
\pgfsetlinewidth{0.803000pt}%
\definecolor{currentstroke}{rgb}{0.000000,0.000000,0.000000}%
\pgfsetstrokecolor{currentstroke}%
\pgfsetdash{}{0pt}%
\pgfsys@defobject{currentmarker}{\pgfqpoint{-0.048611in}{0.000000in}}{\pgfqpoint{0.000000in}{0.000000in}}{%
\pgfpathmoveto{\pgfqpoint{0.000000in}{0.000000in}}%
\pgfpathlineto{\pgfqpoint{-0.048611in}{0.000000in}}%
\pgfusepath{stroke,fill}%
}%
\begin{pgfscope}%
\pgfsys@transformshift{0.575508in}{3.414653in}%
\pgfsys@useobject{currentmarker}{}%
\end{pgfscope}%
\end{pgfscope}%
\begin{pgfscope}%
\pgftext[x=0.300816in,y=3.361891in,left,base]{\rmfamily\fontsize{10.000000}{12.000000}\selectfont \(\displaystyle 1.0\)}%
\end{pgfscope}%
\begin{pgfscope}%
\pgftext[x=0.245260in,y=2.031603in,,bottom,rotate=90.000000]{\rmfamily\fontsize{10.000000}{12.000000}\selectfont \(\displaystyle t_c t_f\) (s)}%
\end{pgfscope}%
\begin{pgfscope}%
\pgfsetrectcap%
\pgfsetmiterjoin%
\pgfsetlinewidth{0.803000pt}%
\definecolor{currentstroke}{rgb}{0.000000,0.000000,0.000000}%
\pgfsetstrokecolor{currentstroke}%
\pgfsetdash{}{0pt}%
\pgfpathmoveto{\pgfqpoint{0.575508in}{0.521603in}}%
\pgfpathlineto{\pgfqpoint{0.575508in}{3.541603in}}%
\pgfusepath{stroke}%
\end{pgfscope}%
\begin{pgfscope}%
\pgfsetrectcap%
\pgfsetmiterjoin%
\pgfsetlinewidth{0.803000pt}%
\definecolor{currentstroke}{rgb}{0.000000,0.000000,0.000000}%
\pgfsetstrokecolor{currentstroke}%
\pgfsetdash{}{0pt}%
\pgfpathmoveto{\pgfqpoint{5.225508in}{0.521603in}}%
\pgfpathlineto{\pgfqpoint{5.225508in}{3.541603in}}%
\pgfusepath{stroke}%
\end{pgfscope}%
\begin{pgfscope}%
\pgfsetrectcap%
\pgfsetmiterjoin%
\pgfsetlinewidth{0.803000pt}%
\definecolor{currentstroke}{rgb}{0.000000,0.000000,0.000000}%
\pgfsetstrokecolor{currentstroke}%
\pgfsetdash{}{0pt}%
\pgfpathmoveto{\pgfqpoint{0.575508in}{0.521603in}}%
\pgfpathlineto{\pgfqpoint{5.225508in}{0.521603in}}%
\pgfusepath{stroke}%
\end{pgfscope}%
\begin{pgfscope}%
\pgfsetrectcap%
\pgfsetmiterjoin%
\pgfsetlinewidth{0.803000pt}%
\definecolor{currentstroke}{rgb}{0.000000,0.000000,0.000000}%
\pgfsetstrokecolor{currentstroke}%
\pgfsetdash{}{0pt}%
\pgfpathmoveto{\pgfqpoint{0.575508in}{3.541603in}}%
\pgfpathlineto{\pgfqpoint{5.225508in}{3.541603in}}%
\pgfusepath{stroke}%
\end{pgfscope}%
\end{pgfpicture}%
\makeatother%
\endgroup%
}
    \caption{Dimensional time-of-flight asymptotic estimates $t_c t_f$ compared with experimental time-of-flight $t_b$. These results compare $\mathcal{O}(1) \equiv 2 t_c$ and $\mathcal{O}(\phi^3\mathbb{E}\mbox{u}^{3})$ accurate variations of Equation \ref{time_of_flight}. \label{fig:times2}}
\end{figure}
\begin{figure}[htb]
    \centering
    \resizebox{0.5\textwidth}{!}{%% Creator: Matplotlib, PGF backend
%%
%% To include the figure in your LaTeX document, write
%%   \input{<filename>.pgf}
%%
%% Make sure the required packages are loaded in your preamble
%%   \usepackage{pgf}
%%
%% Figures using additional raster images can only be included by \input if
%% they are in the same directory as the main LaTeX file. For loading figures
%% from other directories you can use the `import` package
%%   \usepackage{import}
%% and then include the figures with
%%   \import{<path to file>}{<filename>.pgf}
%%
%% Matplotlib used the following preamble
%%   \usepackage{fontspec}
%%   \setmainfont{DejaVu Serif}
%%   \setsansfont{DejaVu Sans}
%%   \setmonofont{DejaVu Sans Mono}
%%
\begingroup%
\makeatletter%
\begin{pgfpicture}%
\pgfpathrectangle{\pgfpointorigin}{\pgfqpoint{5.557599in}{3.838342in}}%
\pgfusepath{use as bounding box, clip}%
\begin{pgfscope}%
\pgfsetbuttcap%
\pgfsetmiterjoin%
\definecolor{currentfill}{rgb}{1.000000,1.000000,1.000000}%
\pgfsetfillcolor{currentfill}%
\pgfsetlinewidth{0.000000pt}%
\definecolor{currentstroke}{rgb}{1.000000,1.000000,1.000000}%
\pgfsetstrokecolor{currentstroke}%
\pgfsetdash{}{0pt}%
\pgfpathmoveto{\pgfqpoint{0.000000in}{0.000000in}}%
\pgfpathlineto{\pgfqpoint{5.557599in}{0.000000in}}%
\pgfpathlineto{\pgfqpoint{5.557599in}{3.838342in}}%
\pgfpathlineto{\pgfqpoint{0.000000in}{3.838342in}}%
\pgfpathclose%
\pgfusepath{fill}%
\end{pgfscope}%
\begin{pgfscope}%
\pgfsetbuttcap%
\pgfsetmiterjoin%
\definecolor{currentfill}{rgb}{1.000000,1.000000,1.000000}%
\pgfsetfillcolor{currentfill}%
\pgfsetlinewidth{0.000000pt}%
\definecolor{currentstroke}{rgb}{0.000000,0.000000,0.000000}%
\pgfsetstrokecolor{currentstroke}%
\pgfsetstrokeopacity{0.000000}%
\pgfsetdash{}{0pt}%
\pgfpathmoveto{\pgfqpoint{0.772599in}{0.683342in}}%
\pgfpathlineto{\pgfqpoint{5.422599in}{0.683342in}}%
\pgfpathlineto{\pgfqpoint{5.422599in}{3.703342in}}%
\pgfpathlineto{\pgfqpoint{0.772599in}{3.703342in}}%
\pgfpathclose%
\pgfusepath{fill}%
\end{pgfscope}%
\begin{pgfscope}%
\pgfpathrectangle{\pgfqpoint{0.772599in}{0.683342in}}{\pgfqpoint{4.650000in}{3.020000in}}%
\pgfusepath{clip}%
\pgfsetbuttcap%
\pgfsetroundjoin%
\pgfsetlinewidth{1.003750pt}%
\definecolor{currentstroke}{rgb}{0.000000,0.000000,0.000000}%
\pgfsetstrokecolor{currentstroke}%
\pgfsetdash{}{0pt}%
\pgfpathmoveto{\pgfqpoint{1.144483in}{0.853606in}}%
\pgfpathcurveto{\pgfqpoint{1.155533in}{0.853606in}}{\pgfqpoint{1.166132in}{0.857997in}}{\pgfqpoint{1.173946in}{0.865810in}}%
\pgfpathcurveto{\pgfqpoint{1.181759in}{0.873624in}}{\pgfqpoint{1.186150in}{0.884223in}}{\pgfqpoint{1.186150in}{0.895273in}}%
\pgfpathcurveto{\pgfqpoint{1.186150in}{0.906323in}}{\pgfqpoint{1.181759in}{0.916922in}}{\pgfqpoint{1.173946in}{0.924736in}}%
\pgfpathcurveto{\pgfqpoint{1.166132in}{0.932549in}}{\pgfqpoint{1.155533in}{0.936940in}}{\pgfqpoint{1.144483in}{0.936940in}}%
\pgfpathcurveto{\pgfqpoint{1.133433in}{0.936940in}}{\pgfqpoint{1.122834in}{0.932549in}}{\pgfqpoint{1.115020in}{0.924736in}}%
\pgfpathcurveto{\pgfqpoint{1.107207in}{0.916922in}}{\pgfqpoint{1.102816in}{0.906323in}}{\pgfqpoint{1.102816in}{0.895273in}}%
\pgfpathcurveto{\pgfqpoint{1.102816in}{0.884223in}}{\pgfqpoint{1.107207in}{0.873624in}}{\pgfqpoint{1.115020in}{0.865810in}}%
\pgfpathcurveto{\pgfqpoint{1.122834in}{0.857997in}}{\pgfqpoint{1.133433in}{0.853606in}}{\pgfqpoint{1.144483in}{0.853606in}}%
\pgfpathclose%
\pgfusepath{stroke}%
\end{pgfscope}%
\begin{pgfscope}%
\pgfpathrectangle{\pgfqpoint{0.772599in}{0.683342in}}{\pgfqpoint{4.650000in}{3.020000in}}%
\pgfusepath{clip}%
\pgfsetbuttcap%
\pgfsetroundjoin%
\pgfsetlinewidth{1.003750pt}%
\definecolor{currentstroke}{rgb}{0.000000,0.000000,0.000000}%
\pgfsetstrokecolor{currentstroke}%
\pgfsetdash{}{0pt}%
\pgfpathmoveto{\pgfqpoint{1.705631in}{1.038997in}}%
\pgfpathcurveto{\pgfqpoint{1.716681in}{1.038997in}}{\pgfqpoint{1.727280in}{1.043387in}}{\pgfqpoint{1.735094in}{1.051200in}}%
\pgfpathcurveto{\pgfqpoint{1.742907in}{1.059014in}}{\pgfqpoint{1.747298in}{1.069613in}}{\pgfqpoint{1.747298in}{1.080663in}}%
\pgfpathcurveto{\pgfqpoint{1.747298in}{1.091713in}}{\pgfqpoint{1.742907in}{1.102312in}}{\pgfqpoint{1.735094in}{1.110126in}}%
\pgfpathcurveto{\pgfqpoint{1.727280in}{1.117940in}}{\pgfqpoint{1.716681in}{1.122330in}}{\pgfqpoint{1.705631in}{1.122330in}}%
\pgfpathcurveto{\pgfqpoint{1.694581in}{1.122330in}}{\pgfqpoint{1.683982in}{1.117940in}}{\pgfqpoint{1.676168in}{1.110126in}}%
\pgfpathcurveto{\pgfqpoint{1.668354in}{1.102312in}}{\pgfqpoint{1.663964in}{1.091713in}}{\pgfqpoint{1.663964in}{1.080663in}}%
\pgfpathcurveto{\pgfqpoint{1.663964in}{1.069613in}}{\pgfqpoint{1.668354in}{1.059014in}}{\pgfqpoint{1.676168in}{1.051200in}}%
\pgfpathcurveto{\pgfqpoint{1.683982in}{1.043387in}}{\pgfqpoint{1.694581in}{1.038997in}}{\pgfqpoint{1.705631in}{1.038997in}}%
\pgfpathclose%
\pgfusepath{stroke}%
\end{pgfscope}%
\begin{pgfscope}%
\pgfpathrectangle{\pgfqpoint{0.772599in}{0.683342in}}{\pgfqpoint{4.650000in}{3.020000in}}%
\pgfusepath{clip}%
\pgfsetbuttcap%
\pgfsetroundjoin%
\pgfsetlinewidth{1.003750pt}%
\definecolor{currentstroke}{rgb}{0.000000,0.000000,0.000000}%
\pgfsetstrokecolor{currentstroke}%
\pgfsetdash{}{0pt}%
\pgfpathmoveto{\pgfqpoint{1.013666in}{0.809392in}}%
\pgfpathcurveto{\pgfqpoint{1.024716in}{0.809392in}}{\pgfqpoint{1.035315in}{0.813782in}}{\pgfqpoint{1.043129in}{0.821596in}}%
\pgfpathcurveto{\pgfqpoint{1.050942in}{0.829409in}}{\pgfqpoint{1.055333in}{0.840008in}}{\pgfqpoint{1.055333in}{0.851059in}}%
\pgfpathcurveto{\pgfqpoint{1.055333in}{0.862109in}}{\pgfqpoint{1.050942in}{0.872708in}}{\pgfqpoint{1.043129in}{0.880521in}}%
\pgfpathcurveto{\pgfqpoint{1.035315in}{0.888335in}}{\pgfqpoint{1.024716in}{0.892725in}}{\pgfqpoint{1.013666in}{0.892725in}}%
\pgfpathcurveto{\pgfqpoint{1.002616in}{0.892725in}}{\pgfqpoint{0.992017in}{0.888335in}}{\pgfqpoint{0.984203in}{0.880521in}}%
\pgfpathcurveto{\pgfqpoint{0.976390in}{0.872708in}}{\pgfqpoint{0.971999in}{0.862109in}}{\pgfqpoint{0.971999in}{0.851059in}}%
\pgfpathcurveto{\pgfqpoint{0.971999in}{0.840008in}}{\pgfqpoint{0.976390in}{0.829409in}}{\pgfqpoint{0.984203in}{0.821596in}}%
\pgfpathcurveto{\pgfqpoint{0.992017in}{0.813782in}}{\pgfqpoint{1.002616in}{0.809392in}}{\pgfqpoint{1.013666in}{0.809392in}}%
\pgfpathclose%
\pgfusepath{stroke}%
\end{pgfscope}%
\begin{pgfscope}%
\pgfpathrectangle{\pgfqpoint{0.772599in}{0.683342in}}{\pgfqpoint{4.650000in}{3.020000in}}%
\pgfusepath{clip}%
\pgfsetbuttcap%
\pgfsetroundjoin%
\pgfsetlinewidth{1.003750pt}%
\definecolor{currentstroke}{rgb}{0.000000,0.000000,0.000000}%
\pgfsetstrokecolor{currentstroke}%
\pgfsetdash{}{0pt}%
\pgfpathmoveto{\pgfqpoint{1.427674in}{1.037944in}}%
\pgfpathcurveto{\pgfqpoint{1.438724in}{1.037944in}}{\pgfqpoint{1.449323in}{1.042335in}}{\pgfqpoint{1.457137in}{1.050148in}}%
\pgfpathcurveto{\pgfqpoint{1.464950in}{1.057962in}}{\pgfqpoint{1.469340in}{1.068561in}}{\pgfqpoint{1.469340in}{1.079611in}}%
\pgfpathcurveto{\pgfqpoint{1.469340in}{1.090661in}}{\pgfqpoint{1.464950in}{1.101260in}}{\pgfqpoint{1.457137in}{1.109074in}}%
\pgfpathcurveto{\pgfqpoint{1.449323in}{1.116887in}}{\pgfqpoint{1.438724in}{1.121278in}}{\pgfqpoint{1.427674in}{1.121278in}}%
\pgfpathcurveto{\pgfqpoint{1.416624in}{1.121278in}}{\pgfqpoint{1.406025in}{1.116887in}}{\pgfqpoint{1.398211in}{1.109074in}}%
\pgfpathcurveto{\pgfqpoint{1.390397in}{1.101260in}}{\pgfqpoint{1.386007in}{1.090661in}}{\pgfqpoint{1.386007in}{1.079611in}}%
\pgfpathcurveto{\pgfqpoint{1.386007in}{1.068561in}}{\pgfqpoint{1.390397in}{1.057962in}}{\pgfqpoint{1.398211in}{1.050148in}}%
\pgfpathcurveto{\pgfqpoint{1.406025in}{1.042335in}}{\pgfqpoint{1.416624in}{1.037944in}}{\pgfqpoint{1.427674in}{1.037944in}}%
\pgfpathclose%
\pgfusepath{stroke}%
\end{pgfscope}%
\begin{pgfscope}%
\pgfpathrectangle{\pgfqpoint{0.772599in}{0.683342in}}{\pgfqpoint{4.650000in}{3.020000in}}%
\pgfusepath{clip}%
\pgfsetbuttcap%
\pgfsetroundjoin%
\pgfsetlinewidth{1.003750pt}%
\definecolor{currentstroke}{rgb}{0.000000,0.000000,0.000000}%
\pgfsetstrokecolor{currentstroke}%
\pgfsetdash{}{0pt}%
\pgfpathmoveto{\pgfqpoint{1.728671in}{1.197262in}}%
\pgfpathcurveto{\pgfqpoint{1.739721in}{1.197262in}}{\pgfqpoint{1.750320in}{1.201653in}}{\pgfqpoint{1.758133in}{1.209466in}}%
\pgfpathcurveto{\pgfqpoint{1.765947in}{1.217280in}}{\pgfqpoint{1.770337in}{1.227879in}}{\pgfqpoint{1.770337in}{1.238929in}}%
\pgfpathcurveto{\pgfqpoint{1.770337in}{1.249979in}}{\pgfqpoint{1.765947in}{1.260578in}}{\pgfqpoint{1.758133in}{1.268392in}}%
\pgfpathcurveto{\pgfqpoint{1.750320in}{1.276205in}}{\pgfqpoint{1.739721in}{1.280596in}}{\pgfqpoint{1.728671in}{1.280596in}}%
\pgfpathcurveto{\pgfqpoint{1.717620in}{1.280596in}}{\pgfqpoint{1.707021in}{1.276205in}}{\pgfqpoint{1.699208in}{1.268392in}}%
\pgfpathcurveto{\pgfqpoint{1.691394in}{1.260578in}}{\pgfqpoint{1.687004in}{1.249979in}}{\pgfqpoint{1.687004in}{1.238929in}}%
\pgfpathcurveto{\pgfqpoint{1.687004in}{1.227879in}}{\pgfqpoint{1.691394in}{1.217280in}}{\pgfqpoint{1.699208in}{1.209466in}}%
\pgfpathcurveto{\pgfqpoint{1.707021in}{1.201653in}}{\pgfqpoint{1.717620in}{1.197262in}}{\pgfqpoint{1.728671in}{1.197262in}}%
\pgfpathclose%
\pgfusepath{stroke}%
\end{pgfscope}%
\begin{pgfscope}%
\pgfpathrectangle{\pgfqpoint{0.772599in}{0.683342in}}{\pgfqpoint{4.650000in}{3.020000in}}%
\pgfusepath{clip}%
\pgfsetbuttcap%
\pgfsetroundjoin%
\pgfsetlinewidth{1.003750pt}%
\definecolor{currentstroke}{rgb}{0.000000,0.000000,0.000000}%
\pgfsetstrokecolor{currentstroke}%
\pgfsetdash{}{0pt}%
\pgfpathmoveto{\pgfqpoint{2.116441in}{1.423062in}}%
\pgfpathcurveto{\pgfqpoint{2.127492in}{1.423062in}}{\pgfqpoint{2.138091in}{1.427452in}}{\pgfqpoint{2.145904in}{1.435266in}}%
\pgfpathcurveto{\pgfqpoint{2.153718in}{1.443079in}}{\pgfqpoint{2.158108in}{1.453678in}}{\pgfqpoint{2.158108in}{1.464728in}}%
\pgfpathcurveto{\pgfqpoint{2.158108in}{1.475779in}}{\pgfqpoint{2.153718in}{1.486378in}}{\pgfqpoint{2.145904in}{1.494191in}}%
\pgfpathcurveto{\pgfqpoint{2.138091in}{1.502005in}}{\pgfqpoint{2.127492in}{1.506395in}}{\pgfqpoint{2.116441in}{1.506395in}}%
\pgfpathcurveto{\pgfqpoint{2.105391in}{1.506395in}}{\pgfqpoint{2.094792in}{1.502005in}}{\pgfqpoint{2.086979in}{1.494191in}}%
\pgfpathcurveto{\pgfqpoint{2.079165in}{1.486378in}}{\pgfqpoint{2.074775in}{1.475779in}}{\pgfqpoint{2.074775in}{1.464728in}}%
\pgfpathcurveto{\pgfqpoint{2.074775in}{1.453678in}}{\pgfqpoint{2.079165in}{1.443079in}}{\pgfqpoint{2.086979in}{1.435266in}}%
\pgfpathcurveto{\pgfqpoint{2.094792in}{1.427452in}}{\pgfqpoint{2.105391in}{1.423062in}}{\pgfqpoint{2.116441in}{1.423062in}}%
\pgfpathclose%
\pgfusepath{stroke}%
\end{pgfscope}%
\begin{pgfscope}%
\pgfpathrectangle{\pgfqpoint{0.772599in}{0.683342in}}{\pgfqpoint{4.650000in}{3.020000in}}%
\pgfusepath{clip}%
\pgfsetbuttcap%
\pgfsetroundjoin%
\pgfsetlinewidth{1.003750pt}%
\definecolor{currentstroke}{rgb}{0.000000,0.000000,0.000000}%
\pgfsetstrokecolor{currentstroke}%
\pgfsetdash{}{0pt}%
\pgfpathmoveto{\pgfqpoint{1.851327in}{1.274673in}}%
\pgfpathcurveto{\pgfqpoint{1.862378in}{1.274673in}}{\pgfqpoint{1.872977in}{1.279064in}}{\pgfqpoint{1.880790in}{1.286877in}}%
\pgfpathcurveto{\pgfqpoint{1.888604in}{1.294691in}}{\pgfqpoint{1.892994in}{1.305290in}}{\pgfqpoint{1.892994in}{1.316340in}}%
\pgfpathcurveto{\pgfqpoint{1.892994in}{1.327390in}}{\pgfqpoint{1.888604in}{1.337989in}}{\pgfqpoint{1.880790in}{1.345803in}}%
\pgfpathcurveto{\pgfqpoint{1.872977in}{1.353616in}}{\pgfqpoint{1.862378in}{1.358007in}}{\pgfqpoint{1.851327in}{1.358007in}}%
\pgfpathcurveto{\pgfqpoint{1.840277in}{1.358007in}}{\pgfqpoint{1.829678in}{1.353616in}}{\pgfqpoint{1.821865in}{1.345803in}}%
\pgfpathcurveto{\pgfqpoint{1.814051in}{1.337989in}}{\pgfqpoint{1.809661in}{1.327390in}}{\pgfqpoint{1.809661in}{1.316340in}}%
\pgfpathcurveto{\pgfqpoint{1.809661in}{1.305290in}}{\pgfqpoint{1.814051in}{1.294691in}}{\pgfqpoint{1.821865in}{1.286877in}}%
\pgfpathcurveto{\pgfqpoint{1.829678in}{1.279064in}}{\pgfqpoint{1.840277in}{1.274673in}}{\pgfqpoint{1.851327in}{1.274673in}}%
\pgfpathclose%
\pgfusepath{stroke}%
\end{pgfscope}%
\begin{pgfscope}%
\pgfpathrectangle{\pgfqpoint{0.772599in}{0.683342in}}{\pgfqpoint{4.650000in}{3.020000in}}%
\pgfusepath{clip}%
\pgfsetbuttcap%
\pgfsetroundjoin%
\pgfsetlinewidth{1.003750pt}%
\definecolor{currentstroke}{rgb}{0.000000,0.000000,0.000000}%
\pgfsetstrokecolor{currentstroke}%
\pgfsetdash{}{0pt}%
\pgfpathmoveto{\pgfqpoint{3.270481in}{2.306811in}}%
\pgfpathcurveto{\pgfqpoint{3.281531in}{2.306811in}}{\pgfqpoint{3.292130in}{2.311201in}}{\pgfqpoint{3.299944in}{2.319015in}}%
\pgfpathcurveto{\pgfqpoint{3.307758in}{2.326828in}}{\pgfqpoint{3.312148in}{2.337427in}}{\pgfqpoint{3.312148in}{2.348477in}}%
\pgfpathcurveto{\pgfqpoint{3.312148in}{2.359528in}}{\pgfqpoint{3.307758in}{2.370127in}}{\pgfqpoint{3.299944in}{2.377940in}}%
\pgfpathcurveto{\pgfqpoint{3.292130in}{2.385754in}}{\pgfqpoint{3.281531in}{2.390144in}}{\pgfqpoint{3.270481in}{2.390144in}}%
\pgfpathcurveto{\pgfqpoint{3.259431in}{2.390144in}}{\pgfqpoint{3.248832in}{2.385754in}}{\pgfqpoint{3.241018in}{2.377940in}}%
\pgfpathcurveto{\pgfqpoint{3.233205in}{2.370127in}}{\pgfqpoint{3.228814in}{2.359528in}}{\pgfqpoint{3.228814in}{2.348477in}}%
\pgfpathcurveto{\pgfqpoint{3.228814in}{2.337427in}}{\pgfqpoint{3.233205in}{2.326828in}}{\pgfqpoint{3.241018in}{2.319015in}}%
\pgfpathcurveto{\pgfqpoint{3.248832in}{2.311201in}}{\pgfqpoint{3.259431in}{2.306811in}}{\pgfqpoint{3.270481in}{2.306811in}}%
\pgfpathclose%
\pgfusepath{stroke}%
\end{pgfscope}%
\begin{pgfscope}%
\pgfpathrectangle{\pgfqpoint{0.772599in}{0.683342in}}{\pgfqpoint{4.650000in}{3.020000in}}%
\pgfusepath{clip}%
\pgfsetbuttcap%
\pgfsetroundjoin%
\pgfsetlinewidth{1.003750pt}%
\definecolor{currentstroke}{rgb}{0.000000,0.000000,0.000000}%
\pgfsetstrokecolor{currentstroke}%
\pgfsetdash{}{0pt}%
\pgfpathmoveto{\pgfqpoint{5.181533in}{3.513614in}}%
\pgfpathcurveto{\pgfqpoint{5.192583in}{3.513614in}}{\pgfqpoint{5.203182in}{3.518005in}}{\pgfqpoint{5.210996in}{3.525818in}}%
\pgfpathcurveto{\pgfqpoint{5.218809in}{3.533632in}}{\pgfqpoint{5.223199in}{3.544231in}}{\pgfqpoint{5.223199in}{3.555281in}}%
\pgfpathcurveto{\pgfqpoint{5.223199in}{3.566331in}}{\pgfqpoint{5.218809in}{3.576930in}}{\pgfqpoint{5.210996in}{3.584744in}}%
\pgfpathcurveto{\pgfqpoint{5.203182in}{3.592557in}}{\pgfqpoint{5.192583in}{3.596948in}}{\pgfqpoint{5.181533in}{3.596948in}}%
\pgfpathcurveto{\pgfqpoint{5.170483in}{3.596948in}}{\pgfqpoint{5.159884in}{3.592557in}}{\pgfqpoint{5.152070in}{3.584744in}}%
\pgfpathcurveto{\pgfqpoint{5.144256in}{3.576930in}}{\pgfqpoint{5.139866in}{3.566331in}}{\pgfqpoint{5.139866in}{3.555281in}}%
\pgfpathcurveto{\pgfqpoint{5.139866in}{3.544231in}}{\pgfqpoint{5.144256in}{3.533632in}}{\pgfqpoint{5.152070in}{3.525818in}}%
\pgfpathcurveto{\pgfqpoint{5.159884in}{3.518005in}}{\pgfqpoint{5.170483in}{3.513614in}}{\pgfqpoint{5.181533in}{3.513614in}}%
\pgfpathclose%
\pgfusepath{stroke}%
\end{pgfscope}%
\begin{pgfscope}%
\pgfpathrectangle{\pgfqpoint{0.772599in}{0.683342in}}{\pgfqpoint{4.650000in}{3.020000in}}%
\pgfusepath{clip}%
\pgfsetbuttcap%
\pgfsetroundjoin%
\pgfsetlinewidth{1.003750pt}%
\definecolor{currentstroke}{rgb}{1.000000,0.000000,0.000000}%
\pgfsetstrokecolor{currentstroke}%
\pgfsetdash{}{0pt}%
\pgfpathmoveto{\pgfqpoint{1.144483in}{0.851368in}}%
\pgfpathcurveto{\pgfqpoint{1.155533in}{0.851368in}}{\pgfqpoint{1.166132in}{0.855758in}}{\pgfqpoint{1.173946in}{0.863572in}}%
\pgfpathcurveto{\pgfqpoint{1.181759in}{0.871385in}}{\pgfqpoint{1.186150in}{0.881985in}}{\pgfqpoint{1.186150in}{0.893035in}}%
\pgfpathcurveto{\pgfqpoint{1.186150in}{0.904085in}}{\pgfqpoint{1.181759in}{0.914684in}}{\pgfqpoint{1.173946in}{0.922497in}}%
\pgfpathcurveto{\pgfqpoint{1.166132in}{0.930311in}}{\pgfqpoint{1.155533in}{0.934701in}}{\pgfqpoint{1.144483in}{0.934701in}}%
\pgfpathcurveto{\pgfqpoint{1.133433in}{0.934701in}}{\pgfqpoint{1.122834in}{0.930311in}}{\pgfqpoint{1.115020in}{0.922497in}}%
\pgfpathcurveto{\pgfqpoint{1.107207in}{0.914684in}}{\pgfqpoint{1.102816in}{0.904085in}}{\pgfqpoint{1.102816in}{0.893035in}}%
\pgfpathcurveto{\pgfqpoint{1.102816in}{0.881985in}}{\pgfqpoint{1.107207in}{0.871385in}}{\pgfqpoint{1.115020in}{0.863572in}}%
\pgfpathcurveto{\pgfqpoint{1.122834in}{0.855758in}}{\pgfqpoint{1.133433in}{0.851368in}}{\pgfqpoint{1.144483in}{0.851368in}}%
\pgfpathclose%
\pgfusepath{stroke}%
\end{pgfscope}%
\begin{pgfscope}%
\pgfpathrectangle{\pgfqpoint{0.772599in}{0.683342in}}{\pgfqpoint{4.650000in}{3.020000in}}%
\pgfusepath{clip}%
\pgfsetbuttcap%
\pgfsetroundjoin%
\pgfsetlinewidth{1.003750pt}%
\definecolor{currentstroke}{rgb}{1.000000,0.000000,0.000000}%
\pgfsetstrokecolor{currentstroke}%
\pgfsetdash{}{0pt}%
\pgfpathmoveto{\pgfqpoint{1.705631in}{1.025063in}}%
\pgfpathcurveto{\pgfqpoint{1.716681in}{1.025063in}}{\pgfqpoint{1.727280in}{1.029453in}}{\pgfqpoint{1.735094in}{1.037267in}}%
\pgfpathcurveto{\pgfqpoint{1.742907in}{1.045080in}}{\pgfqpoint{1.747298in}{1.055679in}}{\pgfqpoint{1.747298in}{1.066730in}}%
\pgfpathcurveto{\pgfqpoint{1.747298in}{1.077780in}}{\pgfqpoint{1.742907in}{1.088379in}}{\pgfqpoint{1.735094in}{1.096192in}}%
\pgfpathcurveto{\pgfqpoint{1.727280in}{1.104006in}}{\pgfqpoint{1.716681in}{1.108396in}}{\pgfqpoint{1.705631in}{1.108396in}}%
\pgfpathcurveto{\pgfqpoint{1.694581in}{1.108396in}}{\pgfqpoint{1.683982in}{1.104006in}}{\pgfqpoint{1.676168in}{1.096192in}}%
\pgfpathcurveto{\pgfqpoint{1.668354in}{1.088379in}}{\pgfqpoint{1.663964in}{1.077780in}}{\pgfqpoint{1.663964in}{1.066730in}}%
\pgfpathcurveto{\pgfqpoint{1.663964in}{1.055679in}}{\pgfqpoint{1.668354in}{1.045080in}}{\pgfqpoint{1.676168in}{1.037267in}}%
\pgfpathcurveto{\pgfqpoint{1.683982in}{1.029453in}}{\pgfqpoint{1.694581in}{1.025063in}}{\pgfqpoint{1.705631in}{1.025063in}}%
\pgfpathclose%
\pgfusepath{stroke}%
\end{pgfscope}%
\begin{pgfscope}%
\pgfpathrectangle{\pgfqpoint{0.772599in}{0.683342in}}{\pgfqpoint{4.650000in}{3.020000in}}%
\pgfusepath{clip}%
\pgfsetbuttcap%
\pgfsetroundjoin%
\pgfsetlinewidth{1.003750pt}%
\definecolor{currentstroke}{rgb}{1.000000,0.000000,0.000000}%
\pgfsetstrokecolor{currentstroke}%
\pgfsetdash{}{0pt}%
\pgfpathmoveto{\pgfqpoint{1.013666in}{0.808733in}}%
\pgfpathcurveto{\pgfqpoint{1.024716in}{0.808733in}}{\pgfqpoint{1.035315in}{0.813123in}}{\pgfqpoint{1.043129in}{0.820937in}}%
\pgfpathcurveto{\pgfqpoint{1.050942in}{0.828750in}}{\pgfqpoint{1.055333in}{0.839349in}}{\pgfqpoint{1.055333in}{0.850400in}}%
\pgfpathcurveto{\pgfqpoint{1.055333in}{0.861450in}}{\pgfqpoint{1.050942in}{0.872049in}}{\pgfqpoint{1.043129in}{0.879862in}}%
\pgfpathcurveto{\pgfqpoint{1.035315in}{0.887676in}}{\pgfqpoint{1.024716in}{0.892066in}}{\pgfqpoint{1.013666in}{0.892066in}}%
\pgfpathcurveto{\pgfqpoint{1.002616in}{0.892066in}}{\pgfqpoint{0.992017in}{0.887676in}}{\pgfqpoint{0.984203in}{0.879862in}}%
\pgfpathcurveto{\pgfqpoint{0.976390in}{0.872049in}}{\pgfqpoint{0.971999in}{0.861450in}}{\pgfqpoint{0.971999in}{0.850400in}}%
\pgfpathcurveto{\pgfqpoint{0.971999in}{0.839349in}}{\pgfqpoint{0.976390in}{0.828750in}}{\pgfqpoint{0.984203in}{0.820937in}}%
\pgfpathcurveto{\pgfqpoint{0.992017in}{0.813123in}}{\pgfqpoint{1.002616in}{0.808733in}}{\pgfqpoint{1.013666in}{0.808733in}}%
\pgfpathclose%
\pgfusepath{stroke}%
\end{pgfscope}%
\begin{pgfscope}%
\pgfpathrectangle{\pgfqpoint{0.772599in}{0.683342in}}{\pgfqpoint{4.650000in}{3.020000in}}%
\pgfusepath{clip}%
\pgfsetbuttcap%
\pgfsetroundjoin%
\pgfsetlinewidth{1.003750pt}%
\definecolor{currentstroke}{rgb}{1.000000,0.000000,0.000000}%
\pgfsetstrokecolor{currentstroke}%
\pgfsetdash{}{0pt}%
\pgfpathmoveto{\pgfqpoint{1.427674in}{1.027946in}}%
\pgfpathcurveto{\pgfqpoint{1.438724in}{1.027946in}}{\pgfqpoint{1.449323in}{1.032336in}}{\pgfqpoint{1.457137in}{1.040150in}}%
\pgfpathcurveto{\pgfqpoint{1.464950in}{1.047964in}}{\pgfqpoint{1.469340in}{1.058563in}}{\pgfqpoint{1.469340in}{1.069613in}}%
\pgfpathcurveto{\pgfqpoint{1.469340in}{1.080663in}}{\pgfqpoint{1.464950in}{1.091262in}}{\pgfqpoint{1.457137in}{1.099076in}}%
\pgfpathcurveto{\pgfqpoint{1.449323in}{1.106889in}}{\pgfqpoint{1.438724in}{1.111279in}}{\pgfqpoint{1.427674in}{1.111279in}}%
\pgfpathcurveto{\pgfqpoint{1.416624in}{1.111279in}}{\pgfqpoint{1.406025in}{1.106889in}}{\pgfqpoint{1.398211in}{1.099076in}}%
\pgfpathcurveto{\pgfqpoint{1.390397in}{1.091262in}}{\pgfqpoint{1.386007in}{1.080663in}}{\pgfqpoint{1.386007in}{1.069613in}}%
\pgfpathcurveto{\pgfqpoint{1.386007in}{1.058563in}}{\pgfqpoint{1.390397in}{1.047964in}}{\pgfqpoint{1.398211in}{1.040150in}}%
\pgfpathcurveto{\pgfqpoint{1.406025in}{1.032336in}}{\pgfqpoint{1.416624in}{1.027946in}}{\pgfqpoint{1.427674in}{1.027946in}}%
\pgfpathclose%
\pgfusepath{stroke}%
\end{pgfscope}%
\begin{pgfscope}%
\pgfpathrectangle{\pgfqpoint{0.772599in}{0.683342in}}{\pgfqpoint{4.650000in}{3.020000in}}%
\pgfusepath{clip}%
\pgfsetbuttcap%
\pgfsetroundjoin%
\pgfsetlinewidth{1.003750pt}%
\definecolor{currentstroke}{rgb}{1.000000,0.000000,0.000000}%
\pgfsetstrokecolor{currentstroke}%
\pgfsetdash{}{0pt}%
\pgfpathmoveto{\pgfqpoint{1.728671in}{1.179683in}}%
\pgfpathcurveto{\pgfqpoint{1.739721in}{1.179683in}}{\pgfqpoint{1.750320in}{1.184073in}}{\pgfqpoint{1.758133in}{1.191887in}}%
\pgfpathcurveto{\pgfqpoint{1.765947in}{1.199700in}}{\pgfqpoint{1.770337in}{1.210299in}}{\pgfqpoint{1.770337in}{1.221349in}}%
\pgfpathcurveto{\pgfqpoint{1.770337in}{1.232400in}}{\pgfqpoint{1.765947in}{1.242999in}}{\pgfqpoint{1.758133in}{1.250812in}}%
\pgfpathcurveto{\pgfqpoint{1.750320in}{1.258626in}}{\pgfqpoint{1.739721in}{1.263016in}}{\pgfqpoint{1.728671in}{1.263016in}}%
\pgfpathcurveto{\pgfqpoint{1.717620in}{1.263016in}}{\pgfqpoint{1.707021in}{1.258626in}}{\pgfqpoint{1.699208in}{1.250812in}}%
\pgfpathcurveto{\pgfqpoint{1.691394in}{1.242999in}}{\pgfqpoint{1.687004in}{1.232400in}}{\pgfqpoint{1.687004in}{1.221349in}}%
\pgfpathcurveto{\pgfqpoint{1.687004in}{1.210299in}}{\pgfqpoint{1.691394in}{1.199700in}}{\pgfqpoint{1.699208in}{1.191887in}}%
\pgfpathcurveto{\pgfqpoint{1.707021in}{1.184073in}}{\pgfqpoint{1.717620in}{1.179683in}}{\pgfqpoint{1.728671in}{1.179683in}}%
\pgfpathclose%
\pgfusepath{stroke}%
\end{pgfscope}%
\begin{pgfscope}%
\pgfpathrectangle{\pgfqpoint{0.772599in}{0.683342in}}{\pgfqpoint{4.650000in}{3.020000in}}%
\pgfusepath{clip}%
\pgfsetbuttcap%
\pgfsetroundjoin%
\pgfsetlinewidth{1.003750pt}%
\definecolor{currentstroke}{rgb}{1.000000,0.000000,0.000000}%
\pgfsetstrokecolor{currentstroke}%
\pgfsetdash{}{0pt}%
\pgfpathmoveto{\pgfqpoint{2.116441in}{1.379001in}}%
\pgfpathcurveto{\pgfqpoint{2.127492in}{1.379001in}}{\pgfqpoint{2.138091in}{1.383392in}}{\pgfqpoint{2.145904in}{1.391205in}}%
\pgfpathcurveto{\pgfqpoint{2.153718in}{1.399019in}}{\pgfqpoint{2.158108in}{1.409618in}}{\pgfqpoint{2.158108in}{1.420668in}}%
\pgfpathcurveto{\pgfqpoint{2.158108in}{1.431718in}}{\pgfqpoint{2.153718in}{1.442317in}}{\pgfqpoint{2.145904in}{1.450131in}}%
\pgfpathcurveto{\pgfqpoint{2.138091in}{1.457944in}}{\pgfqpoint{2.127492in}{1.462335in}}{\pgfqpoint{2.116441in}{1.462335in}}%
\pgfpathcurveto{\pgfqpoint{2.105391in}{1.462335in}}{\pgfqpoint{2.094792in}{1.457944in}}{\pgfqpoint{2.086979in}{1.450131in}}%
\pgfpathcurveto{\pgfqpoint{2.079165in}{1.442317in}}{\pgfqpoint{2.074775in}{1.431718in}}{\pgfqpoint{2.074775in}{1.420668in}}%
\pgfpathcurveto{\pgfqpoint{2.074775in}{1.409618in}}{\pgfqpoint{2.079165in}{1.399019in}}{\pgfqpoint{2.086979in}{1.391205in}}%
\pgfpathcurveto{\pgfqpoint{2.094792in}{1.383392in}}{\pgfqpoint{2.105391in}{1.379001in}}{\pgfqpoint{2.116441in}{1.379001in}}%
\pgfpathclose%
\pgfusepath{stroke}%
\end{pgfscope}%
\begin{pgfscope}%
\pgfpathrectangle{\pgfqpoint{0.772599in}{0.683342in}}{\pgfqpoint{4.650000in}{3.020000in}}%
\pgfusepath{clip}%
\pgfsetbuttcap%
\pgfsetroundjoin%
\pgfsetlinewidth{1.003750pt}%
\definecolor{currentstroke}{rgb}{1.000000,0.000000,0.000000}%
\pgfsetstrokecolor{currentstroke}%
\pgfsetdash{}{0pt}%
\pgfpathmoveto{\pgfqpoint{1.851327in}{1.257683in}}%
\pgfpathcurveto{\pgfqpoint{1.862378in}{1.257683in}}{\pgfqpoint{1.872977in}{1.262073in}}{\pgfqpoint{1.880790in}{1.269887in}}%
\pgfpathcurveto{\pgfqpoint{1.888604in}{1.277700in}}{\pgfqpoint{1.892994in}{1.288299in}}{\pgfqpoint{1.892994in}{1.299349in}}%
\pgfpathcurveto{\pgfqpoint{1.892994in}{1.310399in}}{\pgfqpoint{1.888604in}{1.320999in}}{\pgfqpoint{1.880790in}{1.328812in}}%
\pgfpathcurveto{\pgfqpoint{1.872977in}{1.336626in}}{\pgfqpoint{1.862378in}{1.341016in}}{\pgfqpoint{1.851327in}{1.341016in}}%
\pgfpathcurveto{\pgfqpoint{1.840277in}{1.341016in}}{\pgfqpoint{1.829678in}{1.336626in}}{\pgfqpoint{1.821865in}{1.328812in}}%
\pgfpathcurveto{\pgfqpoint{1.814051in}{1.320999in}}{\pgfqpoint{1.809661in}{1.310399in}}{\pgfqpoint{1.809661in}{1.299349in}}%
\pgfpathcurveto{\pgfqpoint{1.809661in}{1.288299in}}{\pgfqpoint{1.814051in}{1.277700in}}{\pgfqpoint{1.821865in}{1.269887in}}%
\pgfpathcurveto{\pgfqpoint{1.829678in}{1.262073in}}{\pgfqpoint{1.840277in}{1.257683in}}{\pgfqpoint{1.851327in}{1.257683in}}%
\pgfpathclose%
\pgfusepath{stroke}%
\end{pgfscope}%
\begin{pgfscope}%
\pgfpathrectangle{\pgfqpoint{0.772599in}{0.683342in}}{\pgfqpoint{4.650000in}{3.020000in}}%
\pgfusepath{clip}%
\pgfsetbuttcap%
\pgfsetroundjoin%
\pgfsetlinewidth{1.003750pt}%
\definecolor{currentstroke}{rgb}{1.000000,0.000000,0.000000}%
\pgfsetstrokecolor{currentstroke}%
\pgfsetdash{}{0pt}%
\pgfpathmoveto{\pgfqpoint{3.270481in}{2.159601in}}%
\pgfpathcurveto{\pgfqpoint{3.281531in}{2.159601in}}{\pgfqpoint{3.292130in}{2.163991in}}{\pgfqpoint{3.299944in}{2.171805in}}%
\pgfpathcurveto{\pgfqpoint{3.307758in}{2.179618in}}{\pgfqpoint{3.312148in}{2.190218in}}{\pgfqpoint{3.312148in}{2.201268in}}%
\pgfpathcurveto{\pgfqpoint{3.312148in}{2.212318in}}{\pgfqpoint{3.307758in}{2.222917in}}{\pgfqpoint{3.299944in}{2.230730in}}%
\pgfpathcurveto{\pgfqpoint{3.292130in}{2.238544in}}{\pgfqpoint{3.281531in}{2.242934in}}{\pgfqpoint{3.270481in}{2.242934in}}%
\pgfpathcurveto{\pgfqpoint{3.259431in}{2.242934in}}{\pgfqpoint{3.248832in}{2.238544in}}{\pgfqpoint{3.241018in}{2.230730in}}%
\pgfpathcurveto{\pgfqpoint{3.233205in}{2.222917in}}{\pgfqpoint{3.228814in}{2.212318in}}{\pgfqpoint{3.228814in}{2.201268in}}%
\pgfpathcurveto{\pgfqpoint{3.228814in}{2.190218in}}{\pgfqpoint{3.233205in}{2.179618in}}{\pgfqpoint{3.241018in}{2.171805in}}%
\pgfpathcurveto{\pgfqpoint{3.248832in}{2.163991in}}{\pgfqpoint{3.259431in}{2.159601in}}{\pgfqpoint{3.270481in}{2.159601in}}%
\pgfpathclose%
\pgfusepath{stroke}%
\end{pgfscope}%
\begin{pgfscope}%
\pgfpathrectangle{\pgfqpoint{0.772599in}{0.683342in}}{\pgfqpoint{4.650000in}{3.020000in}}%
\pgfusepath{clip}%
\pgfsetbuttcap%
\pgfsetroundjoin%
\pgfsetlinewidth{1.003750pt}%
\definecolor{currentstroke}{rgb}{1.000000,0.000000,0.000000}%
\pgfsetstrokecolor{currentstroke}%
\pgfsetdash{}{0pt}%
\pgfpathmoveto{\pgfqpoint{5.181533in}{3.036558in}}%
\pgfpathcurveto{\pgfqpoint{5.192583in}{3.036558in}}{\pgfqpoint{5.203182in}{3.040949in}}{\pgfqpoint{5.210996in}{3.048762in}}%
\pgfpathcurveto{\pgfqpoint{5.218809in}{3.056576in}}{\pgfqpoint{5.223199in}{3.067175in}}{\pgfqpoint{5.223199in}{3.078225in}}%
\pgfpathcurveto{\pgfqpoint{5.223199in}{3.089275in}}{\pgfqpoint{5.218809in}{3.099874in}}{\pgfqpoint{5.210996in}{3.107688in}}%
\pgfpathcurveto{\pgfqpoint{5.203182in}{3.115501in}}{\pgfqpoint{5.192583in}{3.119892in}}{\pgfqpoint{5.181533in}{3.119892in}}%
\pgfpathcurveto{\pgfqpoint{5.170483in}{3.119892in}}{\pgfqpoint{5.159884in}{3.115501in}}{\pgfqpoint{5.152070in}{3.107688in}}%
\pgfpathcurveto{\pgfqpoint{5.144256in}{3.099874in}}{\pgfqpoint{5.139866in}{3.089275in}}{\pgfqpoint{5.139866in}{3.078225in}}%
\pgfpathcurveto{\pgfqpoint{5.139866in}{3.067175in}}{\pgfqpoint{5.144256in}{3.056576in}}{\pgfqpoint{5.152070in}{3.048762in}}%
\pgfpathcurveto{\pgfqpoint{5.159884in}{3.040949in}}{\pgfqpoint{5.170483in}{3.036558in}}{\pgfqpoint{5.181533in}{3.036558in}}%
\pgfpathclose%
\pgfusepath{stroke}%
\end{pgfscope}%
\begin{pgfscope}%
\pgfsetbuttcap%
\pgfsetroundjoin%
\definecolor{currentfill}{rgb}{0.000000,0.000000,0.000000}%
\pgfsetfillcolor{currentfill}%
\pgfsetlinewidth{0.803000pt}%
\definecolor{currentstroke}{rgb}{0.000000,0.000000,0.000000}%
\pgfsetstrokecolor{currentstroke}%
\pgfsetdash{}{0pt}%
\pgfsys@defobject{currentmarker}{\pgfqpoint{0.000000in}{-0.048611in}}{\pgfqpoint{0.000000in}{0.000000in}}{%
\pgfpathmoveto{\pgfqpoint{0.000000in}{0.000000in}}%
\pgfpathlineto{\pgfqpoint{0.000000in}{-0.048611in}}%
\pgfusepath{stroke,fill}%
}%
\begin{pgfscope}%
\pgfsys@transformshift{1.687568in}{0.683342in}%
\pgfsys@useobject{currentmarker}{}%
\end{pgfscope}%
\end{pgfscope}%
\begin{pgfscope}%
\pgftext[x=1.687568in,y=0.586119in,,top]{\rmfamily\fontsize{16.000000}{19.200000}\selectfont \(\displaystyle 1\)}%
\end{pgfscope}%
\begin{pgfscope}%
\pgfsetbuttcap%
\pgfsetroundjoin%
\definecolor{currentfill}{rgb}{0.000000,0.000000,0.000000}%
\pgfsetfillcolor{currentfill}%
\pgfsetlinewidth{0.803000pt}%
\definecolor{currentstroke}{rgb}{0.000000,0.000000,0.000000}%
\pgfsetstrokecolor{currentstroke}%
\pgfsetdash{}{0pt}%
\pgfsys@defobject{currentmarker}{\pgfqpoint{0.000000in}{-0.048611in}}{\pgfqpoint{0.000000in}{0.000000in}}{%
\pgfpathmoveto{\pgfqpoint{0.000000in}{0.000000in}}%
\pgfpathlineto{\pgfqpoint{0.000000in}{-0.048611in}}%
\pgfusepath{stroke,fill}%
}%
\begin{pgfscope}%
\pgfsys@transformshift{3.048048in}{0.683342in}%
\pgfsys@useobject{currentmarker}{}%
\end{pgfscope}%
\end{pgfscope}%
\begin{pgfscope}%
\pgftext[x=3.048048in,y=0.586119in,,top]{\rmfamily\fontsize{16.000000}{19.200000}\selectfont \(\displaystyle 2\)}%
\end{pgfscope}%
\begin{pgfscope}%
\pgfsetbuttcap%
\pgfsetroundjoin%
\definecolor{currentfill}{rgb}{0.000000,0.000000,0.000000}%
\pgfsetfillcolor{currentfill}%
\pgfsetlinewidth{0.803000pt}%
\definecolor{currentstroke}{rgb}{0.000000,0.000000,0.000000}%
\pgfsetstrokecolor{currentstroke}%
\pgfsetdash{}{0pt}%
\pgfsys@defobject{currentmarker}{\pgfqpoint{0.000000in}{-0.048611in}}{\pgfqpoint{0.000000in}{0.000000in}}{%
\pgfpathmoveto{\pgfqpoint{0.000000in}{0.000000in}}%
\pgfpathlineto{\pgfqpoint{0.000000in}{-0.048611in}}%
\pgfusepath{stroke,fill}%
}%
\begin{pgfscope}%
\pgfsys@transformshift{4.408528in}{0.683342in}%
\pgfsys@useobject{currentmarker}{}%
\end{pgfscope}%
\end{pgfscope}%
\begin{pgfscope}%
\pgftext[x=4.408528in,y=0.586119in,,top]{\rmfamily\fontsize{16.000000}{19.200000}\selectfont \(\displaystyle 3\)}%
\end{pgfscope}%
\begin{pgfscope}%
\pgftext[x=3.097599in,y=0.315503in,,top]{\rmfamily\fontsize{16.000000}{19.200000}\selectfont \(\displaystyle y_{max}\) (cm)}%
\end{pgfscope}%
\begin{pgfscope}%
\pgfsetbuttcap%
\pgfsetroundjoin%
\definecolor{currentfill}{rgb}{0.000000,0.000000,0.000000}%
\pgfsetfillcolor{currentfill}%
\pgfsetlinewidth{0.803000pt}%
\definecolor{currentstroke}{rgb}{0.000000,0.000000,0.000000}%
\pgfsetstrokecolor{currentstroke}%
\pgfsetdash{}{0pt}%
\pgfsys@defobject{currentmarker}{\pgfqpoint{-0.048611in}{0.000000in}}{\pgfqpoint{0.000000in}{0.000000in}}{%
\pgfpathmoveto{\pgfqpoint{0.000000in}{0.000000in}}%
\pgfpathlineto{\pgfqpoint{-0.048611in}{0.000000in}}%
\pgfusepath{stroke,fill}%
}%
\begin{pgfscope}%
\pgfsys@transformshift{0.772599in}{0.725858in}%
\pgfsys@useobject{currentmarker}{}%
\end{pgfscope}%
\end{pgfscope}%
\begin{pgfscope}%
\pgftext[x=0.389964in,y=0.641439in,left,base]{\rmfamily\fontsize{16.000000}{19.200000}\selectfont \(\displaystyle 0.0\)}%
\end{pgfscope}%
\begin{pgfscope}%
\pgfsetbuttcap%
\pgfsetroundjoin%
\definecolor{currentfill}{rgb}{0.000000,0.000000,0.000000}%
\pgfsetfillcolor{currentfill}%
\pgfsetlinewidth{0.803000pt}%
\definecolor{currentstroke}{rgb}{0.000000,0.000000,0.000000}%
\pgfsetstrokecolor{currentstroke}%
\pgfsetdash{}{0pt}%
\pgfsys@defobject{currentmarker}{\pgfqpoint{-0.048611in}{0.000000in}}{\pgfqpoint{0.000000in}{0.000000in}}{%
\pgfpathmoveto{\pgfqpoint{0.000000in}{0.000000in}}%
\pgfpathlineto{\pgfqpoint{-0.048611in}{0.000000in}}%
\pgfusepath{stroke,fill}%
}%
\begin{pgfscope}%
\pgfsys@transformshift{0.772599in}{1.268849in}%
\pgfsys@useobject{currentmarker}{}%
\end{pgfscope}%
\end{pgfscope}%
\begin{pgfscope}%
\pgftext[x=0.389964in,y=1.184431in,left,base]{\rmfamily\fontsize{16.000000}{19.200000}\selectfont \(\displaystyle 0.5\)}%
\end{pgfscope}%
\begin{pgfscope}%
\pgfsetbuttcap%
\pgfsetroundjoin%
\definecolor{currentfill}{rgb}{0.000000,0.000000,0.000000}%
\pgfsetfillcolor{currentfill}%
\pgfsetlinewidth{0.803000pt}%
\definecolor{currentstroke}{rgb}{0.000000,0.000000,0.000000}%
\pgfsetstrokecolor{currentstroke}%
\pgfsetdash{}{0pt}%
\pgfsys@defobject{currentmarker}{\pgfqpoint{-0.048611in}{0.000000in}}{\pgfqpoint{0.000000in}{0.000000in}}{%
\pgfpathmoveto{\pgfqpoint{0.000000in}{0.000000in}}%
\pgfpathlineto{\pgfqpoint{-0.048611in}{0.000000in}}%
\pgfusepath{stroke,fill}%
}%
\begin{pgfscope}%
\pgfsys@transformshift{0.772599in}{1.811841in}%
\pgfsys@useobject{currentmarker}{}%
\end{pgfscope}%
\end{pgfscope}%
\begin{pgfscope}%
\pgftext[x=0.389964in,y=1.727423in,left,base]{\rmfamily\fontsize{16.000000}{19.200000}\selectfont \(\displaystyle 1.0\)}%
\end{pgfscope}%
\begin{pgfscope}%
\pgfsetbuttcap%
\pgfsetroundjoin%
\definecolor{currentfill}{rgb}{0.000000,0.000000,0.000000}%
\pgfsetfillcolor{currentfill}%
\pgfsetlinewidth{0.803000pt}%
\definecolor{currentstroke}{rgb}{0.000000,0.000000,0.000000}%
\pgfsetstrokecolor{currentstroke}%
\pgfsetdash{}{0pt}%
\pgfsys@defobject{currentmarker}{\pgfqpoint{-0.048611in}{0.000000in}}{\pgfqpoint{0.000000in}{0.000000in}}{%
\pgfpathmoveto{\pgfqpoint{0.000000in}{0.000000in}}%
\pgfpathlineto{\pgfqpoint{-0.048611in}{0.000000in}}%
\pgfusepath{stroke,fill}%
}%
\begin{pgfscope}%
\pgfsys@transformshift{0.772599in}{2.354833in}%
\pgfsys@useobject{currentmarker}{}%
\end{pgfscope}%
\end{pgfscope}%
\begin{pgfscope}%
\pgftext[x=0.389964in,y=2.270415in,left,base]{\rmfamily\fontsize{16.000000}{19.200000}\selectfont \(\displaystyle 1.5\)}%
\end{pgfscope}%
\begin{pgfscope}%
\pgfsetbuttcap%
\pgfsetroundjoin%
\definecolor{currentfill}{rgb}{0.000000,0.000000,0.000000}%
\pgfsetfillcolor{currentfill}%
\pgfsetlinewidth{0.803000pt}%
\definecolor{currentstroke}{rgb}{0.000000,0.000000,0.000000}%
\pgfsetstrokecolor{currentstroke}%
\pgfsetdash{}{0pt}%
\pgfsys@defobject{currentmarker}{\pgfqpoint{-0.048611in}{0.000000in}}{\pgfqpoint{0.000000in}{0.000000in}}{%
\pgfpathmoveto{\pgfqpoint{0.000000in}{0.000000in}}%
\pgfpathlineto{\pgfqpoint{-0.048611in}{0.000000in}}%
\pgfusepath{stroke,fill}%
}%
\begin{pgfscope}%
\pgfsys@transformshift{0.772599in}{2.897825in}%
\pgfsys@useobject{currentmarker}{}%
\end{pgfscope}%
\end{pgfscope}%
\begin{pgfscope}%
\pgftext[x=0.389964in,y=2.813407in,left,base]{\rmfamily\fontsize{16.000000}{19.200000}\selectfont \(\displaystyle 2.0\)}%
\end{pgfscope}%
\begin{pgfscope}%
\pgfsetbuttcap%
\pgfsetroundjoin%
\definecolor{currentfill}{rgb}{0.000000,0.000000,0.000000}%
\pgfsetfillcolor{currentfill}%
\pgfsetlinewidth{0.803000pt}%
\definecolor{currentstroke}{rgb}{0.000000,0.000000,0.000000}%
\pgfsetstrokecolor{currentstroke}%
\pgfsetdash{}{0pt}%
\pgfsys@defobject{currentmarker}{\pgfqpoint{-0.048611in}{0.000000in}}{\pgfqpoint{0.000000in}{0.000000in}}{%
\pgfpathmoveto{\pgfqpoint{0.000000in}{0.000000in}}%
\pgfpathlineto{\pgfqpoint{-0.048611in}{0.000000in}}%
\pgfusepath{stroke,fill}%
}%
\begin{pgfscope}%
\pgfsys@transformshift{0.772599in}{3.440817in}%
\pgfsys@useobject{currentmarker}{}%
\end{pgfscope}%
\end{pgfscope}%
\begin{pgfscope}%
\pgftext[x=0.389964in,y=3.356399in,left,base]{\rmfamily\fontsize{16.000000}{19.200000}\selectfont \(\displaystyle 2.5\)}%
\end{pgfscope}%
\begin{pgfscope}%
\pgftext[x=0.334408in,y=2.193342in,,bottom,rotate=90.000000]{\rmfamily\fontsize{16.000000}{19.200000}\selectfont \(\displaystyle y_c y_f\) (cm)}%
\end{pgfscope}%
\begin{pgfscope}%
\pgfsetrectcap%
\pgfsetmiterjoin%
\pgfsetlinewidth{0.803000pt}%
\definecolor{currentstroke}{rgb}{0.501961,0.501961,0.501961}%
\pgfsetstrokecolor{currentstroke}%
\pgfsetdash{}{0pt}%
\pgfpathmoveto{\pgfqpoint{0.772599in}{0.683342in}}%
\pgfpathlineto{\pgfqpoint{0.772599in}{3.703342in}}%
\pgfusepath{stroke}%
\end{pgfscope}%
\begin{pgfscope}%
\pgfsetrectcap%
\pgfsetmiterjoin%
\pgfsetlinewidth{0.803000pt}%
\definecolor{currentstroke}{rgb}{0.501961,0.501961,0.501961}%
\pgfsetstrokecolor{currentstroke}%
\pgfsetdash{}{0pt}%
\pgfpathmoveto{\pgfqpoint{5.422599in}{0.683342in}}%
\pgfpathlineto{\pgfqpoint{5.422599in}{3.703342in}}%
\pgfusepath{stroke}%
\end{pgfscope}%
\begin{pgfscope}%
\pgfsetrectcap%
\pgfsetmiterjoin%
\pgfsetlinewidth{0.803000pt}%
\definecolor{currentstroke}{rgb}{0.501961,0.501961,0.501961}%
\pgfsetstrokecolor{currentstroke}%
\pgfsetdash{}{0pt}%
\pgfpathmoveto{\pgfqpoint{0.772599in}{0.683342in}}%
\pgfpathlineto{\pgfqpoint{5.422599in}{0.683342in}}%
\pgfusepath{stroke}%
\end{pgfscope}%
\begin{pgfscope}%
\pgfsetrectcap%
\pgfsetmiterjoin%
\pgfsetlinewidth{0.803000pt}%
\definecolor{currentstroke}{rgb}{0.501961,0.501961,0.501961}%
\pgfsetstrokecolor{currentstroke}%
\pgfsetdash{}{0pt}%
\pgfpathmoveto{\pgfqpoint{0.772599in}{3.703342in}}%
\pgfpathlineto{\pgfqpoint{5.422599in}{3.703342in}}%
\pgfusepath{stroke}%
\end{pgfscope}%
\begin{pgfscope}%
\pgfsetbuttcap%
\pgfsetmiterjoin%
\definecolor{currentfill}{rgb}{1.000000,1.000000,1.000000}%
\pgfsetfillcolor{currentfill}%
\pgfsetfillopacity{0.800000}%
\pgfsetlinewidth{0.000000pt}%
\definecolor{currentstroke}{rgb}{0.800000,0.800000,0.800000}%
\pgfsetstrokecolor{currentstroke}%
\pgfsetstrokeopacity{0.800000}%
\pgfsetdash{}{0pt}%
\pgfpathmoveto{\pgfqpoint{0.928155in}{2.806857in}}%
\pgfpathlineto{\pgfqpoint{2.624537in}{2.806857in}}%
\pgfpathquadraticcurveto{\pgfqpoint{2.668981in}{2.806857in}}{\pgfqpoint{2.668981in}{2.851301in}}%
\pgfpathlineto{\pgfqpoint{2.668981in}{3.547786in}}%
\pgfpathquadraticcurveto{\pgfqpoint{2.668981in}{3.592231in}}{\pgfqpoint{2.624537in}{3.592231in}}%
\pgfpathlineto{\pgfqpoint{0.928155in}{3.592231in}}%
\pgfpathquadraticcurveto{\pgfqpoint{0.883711in}{3.592231in}}{\pgfqpoint{0.883711in}{3.547786in}}%
\pgfpathlineto{\pgfqpoint{0.883711in}{2.851301in}}%
\pgfpathquadraticcurveto{\pgfqpoint{0.883711in}{2.806857in}}{\pgfqpoint{0.928155in}{2.806857in}}%
\pgfpathclose%
\pgfusepath{fill}%
\end{pgfscope}%
\begin{pgfscope}%
\pgfsetbuttcap%
\pgfsetroundjoin%
\pgfsetlinewidth{1.003750pt}%
\definecolor{currentstroke}{rgb}{0.000000,0.000000,0.000000}%
\pgfsetstrokecolor{currentstroke}%
\pgfsetdash{}{0pt}%
\pgfpathmoveto{\pgfqpoint{1.194822in}{3.322905in}}%
\pgfpathcurveto{\pgfqpoint{1.205872in}{3.322905in}}{\pgfqpoint{1.216471in}{3.327296in}}{\pgfqpoint{1.224284in}{3.335109in}}%
\pgfpathcurveto{\pgfqpoint{1.232098in}{3.342923in}}{\pgfqpoint{1.236488in}{3.353522in}}{\pgfqpoint{1.236488in}{3.364572in}}%
\pgfpathcurveto{\pgfqpoint{1.236488in}{3.375622in}}{\pgfqpoint{1.232098in}{3.386221in}}{\pgfqpoint{1.224284in}{3.394035in}}%
\pgfpathcurveto{\pgfqpoint{1.216471in}{3.401848in}}{\pgfqpoint{1.205872in}{3.406239in}}{\pgfqpoint{1.194822in}{3.406239in}}%
\pgfpathcurveto{\pgfqpoint{1.183772in}{3.406239in}}{\pgfqpoint{1.173173in}{3.401848in}}{\pgfqpoint{1.165359in}{3.394035in}}%
\pgfpathcurveto{\pgfqpoint{1.157545in}{3.386221in}}{\pgfqpoint{1.153155in}{3.375622in}}{\pgfqpoint{1.153155in}{3.364572in}}%
\pgfpathcurveto{\pgfqpoint{1.153155in}{3.353522in}}{\pgfqpoint{1.157545in}{3.342923in}}{\pgfqpoint{1.165359in}{3.335109in}}%
\pgfpathcurveto{\pgfqpoint{1.173173in}{3.327296in}}{\pgfqpoint{1.183772in}{3.322905in}}{\pgfqpoint{1.194822in}{3.322905in}}%
\pgfpathclose%
\pgfusepath{stroke}%
\end{pgfscope}%
\begin{pgfscope}%
\pgftext[x=1.594822in,y=3.306239in,left,base]{\rmfamily\fontsize{16.000000}{19.200000}\selectfont \(\displaystyle \mathcal{O}(\phi^3 \mathbf{E}\mbox{u}^3)\)}%
\end{pgfscope}%
\begin{pgfscope}%
\pgfsetbuttcap%
\pgfsetroundjoin%
\pgfsetlinewidth{1.003750pt}%
\definecolor{currentstroke}{rgb}{1.000000,0.000000,0.000000}%
\pgfsetstrokecolor{currentstroke}%
\pgfsetdash{}{0pt}%
\pgfpathmoveto{\pgfqpoint{1.194822in}{2.972211in}}%
\pgfpathcurveto{\pgfqpoint{1.205872in}{2.972211in}}{\pgfqpoint{1.216471in}{2.976602in}}{\pgfqpoint{1.224284in}{2.984415in}}%
\pgfpathcurveto{\pgfqpoint{1.232098in}{2.992229in}}{\pgfqpoint{1.236488in}{3.002828in}}{\pgfqpoint{1.236488in}{3.013878in}}%
\pgfpathcurveto{\pgfqpoint{1.236488in}{3.024928in}}{\pgfqpoint{1.232098in}{3.035527in}}{\pgfqpoint{1.224284in}{3.043341in}}%
\pgfpathcurveto{\pgfqpoint{1.216471in}{3.051154in}}{\pgfqpoint{1.205872in}{3.055545in}}{\pgfqpoint{1.194822in}{3.055545in}}%
\pgfpathcurveto{\pgfqpoint{1.183772in}{3.055545in}}{\pgfqpoint{1.173173in}{3.051154in}}{\pgfqpoint{1.165359in}{3.043341in}}%
\pgfpathcurveto{\pgfqpoint{1.157545in}{3.035527in}}{\pgfqpoint{1.153155in}{3.024928in}}{\pgfqpoint{1.153155in}{3.013878in}}%
\pgfpathcurveto{\pgfqpoint{1.153155in}{3.002828in}}{\pgfqpoint{1.157545in}{2.992229in}}{\pgfqpoint{1.165359in}{2.984415in}}%
\pgfpathcurveto{\pgfqpoint{1.173173in}{2.976602in}}{\pgfqpoint{1.183772in}{2.972211in}}{\pgfqpoint{1.194822in}{2.972211in}}%
\pgfpathclose%
\pgfusepath{stroke}%
\end{pgfscope}%
\begin{pgfscope}%
\pgftext[x=1.594822in,y=2.955545in,left,base]{\rmfamily\fontsize{16.000000}{19.200000}\selectfont \(\displaystyle \mathcal{O}(1)\)}%
\end{pgfscope}%
\end{pgfpicture}%
\makeatother%
\endgroup%
}
    \caption{Dimensional asymptotic trajectory apoapses, found from Equation \ref{perturb_viscous} with $t^*=1$ and $\mathcal{O}(1) \equiv y_c/2$, as compared to experimental apoapses $y_{max}$. \label{fig:ymaxes}}
\end{figure}
A similar empirical coefficient $a = 1.22$ is found for Equation \ref{perturb_viscous}. 
The relative magnitudes of the forces acting on the drops determined by the MLE numerical solutions to Equation \ref{gov_eqn_subs} for the entire population of drop tower experiments are shown in Figure \ref{fig:forces}. Coulomb, image, and drag forces acting on the drops vary in typical magnitude between $\mathcal{O}(10^{-6})$-$\mathcal{O}(10^{-4})$ N. We see that of the drops in the experimental dataset only the two with the largest $\mathbb{E}\mbox{u} \sim \mathcal{O}(10)$ could appropriately be said to be in the inertial electro-viscous regime. In all other cases image forces are much stronger than drag. As expected, the image forces themselves rapidly become small compared to Coulomb forces for drops with apoapses $\mbox{max}\left( y\right) \gtrsim L$. The drop with largest $\mathbb{E}\mbox{u}$ in our dataset failed to escape the electric field as the escape condition $\phi \mathbb{E}\mbox{u} / 8\pi = 0.2 < 1$ was unsatisfied. Equation \ref{time_improved} predicts that this drop will return to the substrate after 7.4 s, a period of free-fall which is lamentably well out of reach of the Dryden Drop Tower. However, such an experiment could be performed in a drop tower facility with a longer free fall period, aboard the ISS or on certain suborbital parabolic flights.
\begin{figure}[!htb]
    \centering
    \resizebox{0.5\textwidth}{!}{%% Creator: Matplotlib, PGF backend
%%
%% To include the figure in your LaTeX document, write
%%   \input{<filename>.pgf}
%%
%% Make sure the required packages are loaded in your preamble
%%   \usepackage{pgf}
%%
%% Figures using additional raster images can only be included by \input if
%% they are in the same directory as the main LaTeX file. For loading figures
%% from other directories you can use the `import` package
%%   \usepackage{import}
%% and then include the figures with
%%   \import{<path to file>}{<filename>.pgf}
%%
%% Matplotlib used the following preamble
%%   \usepackage{fontspec}
%%   \setmainfont{DejaVuSerif.ttf}[Path=/home/erin/anaconda3/lib/python3.6/site-packages/matplotlib/mpl-data/fonts/ttf/]
%%   \setsansfont{DejaVuSans.ttf}[Path=/home/erin/anaconda3/lib/python3.6/site-packages/matplotlib/mpl-data/fonts/ttf/]
%%   \setmonofont{DejaVuSansMono.ttf}[Path=/home/erin/anaconda3/lib/python3.6/site-packages/matplotlib/mpl-data/fonts/ttf/]
%%
\begingroup%
\makeatletter%
\begin{pgfpicture}%
\pgfpathrectangle{\pgfpointorigin}{\pgfqpoint{5.473804in}{3.694691in}}%
\pgfusepath{use as bounding box, clip}%
\begin{pgfscope}%
\pgfsetbuttcap%
\pgfsetmiterjoin%
\definecolor{currentfill}{rgb}{1.000000,1.000000,1.000000}%
\pgfsetfillcolor{currentfill}%
\pgfsetlinewidth{0.000000pt}%
\definecolor{currentstroke}{rgb}{1.000000,1.000000,1.000000}%
\pgfsetstrokecolor{currentstroke}%
\pgfsetdash{}{0pt}%
\pgfpathmoveto{\pgfqpoint{0.000000in}{0.000000in}}%
\pgfpathlineto{\pgfqpoint{5.473804in}{0.000000in}}%
\pgfpathlineto{\pgfqpoint{5.473804in}{3.694691in}}%
\pgfpathlineto{\pgfqpoint{0.000000in}{3.694691in}}%
\pgfpathclose%
\pgfusepath{fill}%
\end{pgfscope}%
\begin{pgfscope}%
\pgfsetbuttcap%
\pgfsetmiterjoin%
\definecolor{currentfill}{rgb}{1.000000,1.000000,1.000000}%
\pgfsetfillcolor{currentfill}%
\pgfsetlinewidth{0.000000pt}%
\definecolor{currentstroke}{rgb}{0.000000,0.000000,0.000000}%
\pgfsetstrokecolor{currentstroke}%
\pgfsetstrokeopacity{0.000000}%
\pgfsetdash{}{0pt}%
\pgfpathmoveto{\pgfqpoint{0.675193in}{0.526079in}}%
\pgfpathlineto{\pgfqpoint{5.325193in}{0.526079in}}%
\pgfpathlineto{\pgfqpoint{5.325193in}{3.546079in}}%
\pgfpathlineto{\pgfqpoint{0.675193in}{3.546079in}}%
\pgfpathclose%
\pgfusepath{fill}%
\end{pgfscope}%
\begin{pgfscope}%
\pgfsetbuttcap%
\pgfsetroundjoin%
\definecolor{currentfill}{rgb}{0.000000,0.000000,0.000000}%
\pgfsetfillcolor{currentfill}%
\pgfsetlinewidth{0.803000pt}%
\definecolor{currentstroke}{rgb}{0.000000,0.000000,0.000000}%
\pgfsetstrokecolor{currentstroke}%
\pgfsetdash{}{0pt}%
\pgfsys@defobject{currentmarker}{\pgfqpoint{0.000000in}{-0.048611in}}{\pgfqpoint{0.000000in}{0.000000in}}{%
\pgfpathmoveto{\pgfqpoint{0.000000in}{0.000000in}}%
\pgfpathlineto{\pgfqpoint{0.000000in}{-0.048611in}}%
\pgfusepath{stroke,fill}%
}%
\begin{pgfscope}%
\pgfsys@transformshift{1.495889in}{0.526079in}%
\pgfsys@useobject{currentmarker}{}%
\end{pgfscope}%
\end{pgfscope}%
\begin{pgfscope}%
\definecolor{textcolor}{rgb}{0.000000,0.000000,0.000000}%
\pgfsetstrokecolor{textcolor}%
\pgfsetfillcolor{textcolor}%
\pgftext[x=1.495889in,y=0.428857in,,top]{\color{textcolor}\rmfamily\fontsize{10.000000}{12.000000}\selectfont \(\displaystyle 0.2\)}%
\end{pgfscope}%
\begin{pgfscope}%
\pgfsetbuttcap%
\pgfsetroundjoin%
\definecolor{currentfill}{rgb}{0.000000,0.000000,0.000000}%
\pgfsetfillcolor{currentfill}%
\pgfsetlinewidth{0.803000pt}%
\definecolor{currentstroke}{rgb}{0.000000,0.000000,0.000000}%
\pgfsetstrokecolor{currentstroke}%
\pgfsetdash{}{0pt}%
\pgfsys@defobject{currentmarker}{\pgfqpoint{0.000000in}{-0.048611in}}{\pgfqpoint{0.000000in}{0.000000in}}{%
\pgfpathmoveto{\pgfqpoint{0.000000in}{0.000000in}}%
\pgfpathlineto{\pgfqpoint{0.000000in}{-0.048611in}}%
\pgfusepath{stroke,fill}%
}%
\begin{pgfscope}%
\pgfsys@transformshift{2.393477in}{0.526079in}%
\pgfsys@useobject{currentmarker}{}%
\end{pgfscope}%
\end{pgfscope}%
\begin{pgfscope}%
\definecolor{textcolor}{rgb}{0.000000,0.000000,0.000000}%
\pgfsetstrokecolor{textcolor}%
\pgfsetfillcolor{textcolor}%
\pgftext[x=2.393477in,y=0.428857in,,top]{\color{textcolor}\rmfamily\fontsize{10.000000}{12.000000}\selectfont \(\displaystyle 0.4\)}%
\end{pgfscope}%
\begin{pgfscope}%
\pgfsetbuttcap%
\pgfsetroundjoin%
\definecolor{currentfill}{rgb}{0.000000,0.000000,0.000000}%
\pgfsetfillcolor{currentfill}%
\pgfsetlinewidth{0.803000pt}%
\definecolor{currentstroke}{rgb}{0.000000,0.000000,0.000000}%
\pgfsetstrokecolor{currentstroke}%
\pgfsetdash{}{0pt}%
\pgfsys@defobject{currentmarker}{\pgfqpoint{0.000000in}{-0.048611in}}{\pgfqpoint{0.000000in}{0.000000in}}{%
\pgfpathmoveto{\pgfqpoint{0.000000in}{0.000000in}}%
\pgfpathlineto{\pgfqpoint{0.000000in}{-0.048611in}}%
\pgfusepath{stroke,fill}%
}%
\begin{pgfscope}%
\pgfsys@transformshift{3.291064in}{0.526079in}%
\pgfsys@useobject{currentmarker}{}%
\end{pgfscope}%
\end{pgfscope}%
\begin{pgfscope}%
\definecolor{textcolor}{rgb}{0.000000,0.000000,0.000000}%
\pgfsetstrokecolor{textcolor}%
\pgfsetfillcolor{textcolor}%
\pgftext[x=3.291064in,y=0.428857in,,top]{\color{textcolor}\rmfamily\fontsize{10.000000}{12.000000}\selectfont \(\displaystyle 0.6\)}%
\end{pgfscope}%
\begin{pgfscope}%
\pgfsetbuttcap%
\pgfsetroundjoin%
\definecolor{currentfill}{rgb}{0.000000,0.000000,0.000000}%
\pgfsetfillcolor{currentfill}%
\pgfsetlinewidth{0.803000pt}%
\definecolor{currentstroke}{rgb}{0.000000,0.000000,0.000000}%
\pgfsetstrokecolor{currentstroke}%
\pgfsetdash{}{0pt}%
\pgfsys@defobject{currentmarker}{\pgfqpoint{0.000000in}{-0.048611in}}{\pgfqpoint{0.000000in}{0.000000in}}{%
\pgfpathmoveto{\pgfqpoint{0.000000in}{0.000000in}}%
\pgfpathlineto{\pgfqpoint{0.000000in}{-0.048611in}}%
\pgfusepath{stroke,fill}%
}%
\begin{pgfscope}%
\pgfsys@transformshift{4.188652in}{0.526079in}%
\pgfsys@useobject{currentmarker}{}%
\end{pgfscope}%
\end{pgfscope}%
\begin{pgfscope}%
\definecolor{textcolor}{rgb}{0.000000,0.000000,0.000000}%
\pgfsetstrokecolor{textcolor}%
\pgfsetfillcolor{textcolor}%
\pgftext[x=4.188652in,y=0.428857in,,top]{\color{textcolor}\rmfamily\fontsize{10.000000}{12.000000}\selectfont \(\displaystyle 0.8\)}%
\end{pgfscope}%
\begin{pgfscope}%
\pgfsetbuttcap%
\pgfsetroundjoin%
\definecolor{currentfill}{rgb}{0.000000,0.000000,0.000000}%
\pgfsetfillcolor{currentfill}%
\pgfsetlinewidth{0.803000pt}%
\definecolor{currentstroke}{rgb}{0.000000,0.000000,0.000000}%
\pgfsetstrokecolor{currentstroke}%
\pgfsetdash{}{0pt}%
\pgfsys@defobject{currentmarker}{\pgfqpoint{0.000000in}{-0.048611in}}{\pgfqpoint{0.000000in}{0.000000in}}{%
\pgfpathmoveto{\pgfqpoint{0.000000in}{0.000000in}}%
\pgfpathlineto{\pgfqpoint{0.000000in}{-0.048611in}}%
\pgfusepath{stroke,fill}%
}%
\begin{pgfscope}%
\pgfsys@transformshift{5.086239in}{0.526079in}%
\pgfsys@useobject{currentmarker}{}%
\end{pgfscope}%
\end{pgfscope}%
\begin{pgfscope}%
\definecolor{textcolor}{rgb}{0.000000,0.000000,0.000000}%
\pgfsetstrokecolor{textcolor}%
\pgfsetfillcolor{textcolor}%
\pgftext[x=5.086239in,y=0.428857in,,top]{\color{textcolor}\rmfamily\fontsize{10.000000}{12.000000}\selectfont \(\displaystyle 1.0\)}%
\end{pgfscope}%
\begin{pgfscope}%
\definecolor{textcolor}{rgb}{0.000000,0.000000,0.000000}%
\pgfsetstrokecolor{textcolor}%
\pgfsetfillcolor{textcolor}%
\pgftext[x=3.000193in,y=0.238889in,,top]{\color{textcolor}\rmfamily\fontsize{10.000000}{12.000000}\selectfont \(\displaystyle y/L\)}%
\end{pgfscope}%
\begin{pgfscope}%
\pgfsetbuttcap%
\pgfsetroundjoin%
\definecolor{currentfill}{rgb}{0.000000,0.000000,0.000000}%
\pgfsetfillcolor{currentfill}%
\pgfsetlinewidth{0.803000pt}%
\definecolor{currentstroke}{rgb}{0.000000,0.000000,0.000000}%
\pgfsetstrokecolor{currentstroke}%
\pgfsetdash{}{0pt}%
\pgfsys@defobject{currentmarker}{\pgfqpoint{-0.048611in}{0.000000in}}{\pgfqpoint{0.000000in}{0.000000in}}{%
\pgfpathmoveto{\pgfqpoint{0.000000in}{0.000000in}}%
\pgfpathlineto{\pgfqpoint{-0.048611in}{0.000000in}}%
\pgfusepath{stroke,fill}%
}%
\begin{pgfscope}%
\pgfsys@transformshift{0.675193in}{0.903536in}%
\pgfsys@useobject{currentmarker}{}%
\end{pgfscope}%
\end{pgfscope}%
\begin{pgfscope}%
\definecolor{textcolor}{rgb}{0.000000,0.000000,0.000000}%
\pgfsetstrokecolor{textcolor}%
\pgfsetfillcolor{textcolor}%
\pgftext[x=0.289968in,y=0.850774in,left,base]{\color{textcolor}\rmfamily\fontsize{10.000000}{12.000000}\selectfont \(\displaystyle 10^{-2}\)}%
\end{pgfscope}%
\begin{pgfscope}%
\pgfsetbuttcap%
\pgfsetroundjoin%
\definecolor{currentfill}{rgb}{0.000000,0.000000,0.000000}%
\pgfsetfillcolor{currentfill}%
\pgfsetlinewidth{0.803000pt}%
\definecolor{currentstroke}{rgb}{0.000000,0.000000,0.000000}%
\pgfsetstrokecolor{currentstroke}%
\pgfsetdash{}{0pt}%
\pgfsys@defobject{currentmarker}{\pgfqpoint{-0.048611in}{0.000000in}}{\pgfqpoint{0.000000in}{0.000000in}}{%
\pgfpathmoveto{\pgfqpoint{0.000000in}{0.000000in}}%
\pgfpathlineto{\pgfqpoint{-0.048611in}{0.000000in}}%
\pgfusepath{stroke,fill}%
}%
\begin{pgfscope}%
\pgfsys@transformshift{0.675193in}{2.157420in}%
\pgfsys@useobject{currentmarker}{}%
\end{pgfscope}%
\end{pgfscope}%
\begin{pgfscope}%
\definecolor{textcolor}{rgb}{0.000000,0.000000,0.000000}%
\pgfsetstrokecolor{textcolor}%
\pgfsetfillcolor{textcolor}%
\pgftext[x=0.289968in,y=2.104658in,left,base]{\color{textcolor}\rmfamily\fontsize{10.000000}{12.000000}\selectfont \(\displaystyle 10^{-1}\)}%
\end{pgfscope}%
\begin{pgfscope}%
\pgfsetbuttcap%
\pgfsetroundjoin%
\definecolor{currentfill}{rgb}{0.000000,0.000000,0.000000}%
\pgfsetfillcolor{currentfill}%
\pgfsetlinewidth{0.803000pt}%
\definecolor{currentstroke}{rgb}{0.000000,0.000000,0.000000}%
\pgfsetstrokecolor{currentstroke}%
\pgfsetdash{}{0pt}%
\pgfsys@defobject{currentmarker}{\pgfqpoint{-0.048611in}{0.000000in}}{\pgfqpoint{0.000000in}{0.000000in}}{%
\pgfpathmoveto{\pgfqpoint{0.000000in}{0.000000in}}%
\pgfpathlineto{\pgfqpoint{-0.048611in}{0.000000in}}%
\pgfusepath{stroke,fill}%
}%
\begin{pgfscope}%
\pgfsys@transformshift{0.675193in}{3.411303in}%
\pgfsys@useobject{currentmarker}{}%
\end{pgfscope}%
\end{pgfscope}%
\begin{pgfscope}%
\definecolor{textcolor}{rgb}{0.000000,0.000000,0.000000}%
\pgfsetstrokecolor{textcolor}%
\pgfsetfillcolor{textcolor}%
\pgftext[x=0.376774in,y=3.358542in,left,base]{\color{textcolor}\rmfamily\fontsize{10.000000}{12.000000}\selectfont \(\displaystyle 10^{0}\)}%
\end{pgfscope}%
\begin{pgfscope}%
\pgfsetbuttcap%
\pgfsetroundjoin%
\definecolor{currentfill}{rgb}{0.000000,0.000000,0.000000}%
\pgfsetfillcolor{currentfill}%
\pgfsetlinewidth{0.602250pt}%
\definecolor{currentstroke}{rgb}{0.000000,0.000000,0.000000}%
\pgfsetstrokecolor{currentstroke}%
\pgfsetdash{}{0pt}%
\pgfsys@defobject{currentmarker}{\pgfqpoint{-0.027778in}{0.000000in}}{\pgfqpoint{0.000000in}{0.000000in}}{%
\pgfpathmoveto{\pgfqpoint{0.000000in}{0.000000in}}%
\pgfpathlineto{\pgfqpoint{-0.027778in}{0.000000in}}%
\pgfusepath{stroke,fill}%
}%
\begin{pgfscope}%
\pgfsys@transformshift{0.675193in}{0.526079in}%
\pgfsys@useobject{currentmarker}{}%
\end{pgfscope}%
\end{pgfscope}%
\begin{pgfscope}%
\pgfsetbuttcap%
\pgfsetroundjoin%
\definecolor{currentfill}{rgb}{0.000000,0.000000,0.000000}%
\pgfsetfillcolor{currentfill}%
\pgfsetlinewidth{0.602250pt}%
\definecolor{currentstroke}{rgb}{0.000000,0.000000,0.000000}%
\pgfsetstrokecolor{currentstroke}%
\pgfsetdash{}{0pt}%
\pgfsys@defobject{currentmarker}{\pgfqpoint{-0.027778in}{0.000000in}}{\pgfqpoint{0.000000in}{0.000000in}}{%
\pgfpathmoveto{\pgfqpoint{0.000000in}{0.000000in}}%
\pgfpathlineto{\pgfqpoint{-0.027778in}{0.000000in}}%
\pgfusepath{stroke,fill}%
}%
\begin{pgfscope}%
\pgfsys@transformshift{0.675193in}{0.625364in}%
\pgfsys@useobject{currentmarker}{}%
\end{pgfscope}%
\end{pgfscope}%
\begin{pgfscope}%
\pgfsetbuttcap%
\pgfsetroundjoin%
\definecolor{currentfill}{rgb}{0.000000,0.000000,0.000000}%
\pgfsetfillcolor{currentfill}%
\pgfsetlinewidth{0.602250pt}%
\definecolor{currentstroke}{rgb}{0.000000,0.000000,0.000000}%
\pgfsetstrokecolor{currentstroke}%
\pgfsetdash{}{0pt}%
\pgfsys@defobject{currentmarker}{\pgfqpoint{-0.027778in}{0.000000in}}{\pgfqpoint{0.000000in}{0.000000in}}{%
\pgfpathmoveto{\pgfqpoint{0.000000in}{0.000000in}}%
\pgfpathlineto{\pgfqpoint{-0.027778in}{0.000000in}}%
\pgfusepath{stroke,fill}%
}%
\begin{pgfscope}%
\pgfsys@transformshift{0.675193in}{0.709307in}%
\pgfsys@useobject{currentmarker}{}%
\end{pgfscope}%
\end{pgfscope}%
\begin{pgfscope}%
\pgfsetbuttcap%
\pgfsetroundjoin%
\definecolor{currentfill}{rgb}{0.000000,0.000000,0.000000}%
\pgfsetfillcolor{currentfill}%
\pgfsetlinewidth{0.602250pt}%
\definecolor{currentstroke}{rgb}{0.000000,0.000000,0.000000}%
\pgfsetstrokecolor{currentstroke}%
\pgfsetdash{}{0pt}%
\pgfsys@defobject{currentmarker}{\pgfqpoint{-0.027778in}{0.000000in}}{\pgfqpoint{0.000000in}{0.000000in}}{%
\pgfpathmoveto{\pgfqpoint{0.000000in}{0.000000in}}%
\pgfpathlineto{\pgfqpoint{-0.027778in}{0.000000in}}%
\pgfusepath{stroke,fill}%
}%
\begin{pgfscope}%
\pgfsys@transformshift{0.675193in}{0.782022in}%
\pgfsys@useobject{currentmarker}{}%
\end{pgfscope}%
\end{pgfscope}%
\begin{pgfscope}%
\pgfsetbuttcap%
\pgfsetroundjoin%
\definecolor{currentfill}{rgb}{0.000000,0.000000,0.000000}%
\pgfsetfillcolor{currentfill}%
\pgfsetlinewidth{0.602250pt}%
\definecolor{currentstroke}{rgb}{0.000000,0.000000,0.000000}%
\pgfsetstrokecolor{currentstroke}%
\pgfsetdash{}{0pt}%
\pgfsys@defobject{currentmarker}{\pgfqpoint{-0.027778in}{0.000000in}}{\pgfqpoint{0.000000in}{0.000000in}}{%
\pgfpathmoveto{\pgfqpoint{0.000000in}{0.000000in}}%
\pgfpathlineto{\pgfqpoint{-0.027778in}{0.000000in}}%
\pgfusepath{stroke,fill}%
}%
\begin{pgfscope}%
\pgfsys@transformshift{0.675193in}{0.846161in}%
\pgfsys@useobject{currentmarker}{}%
\end{pgfscope}%
\end{pgfscope}%
\begin{pgfscope}%
\pgfsetbuttcap%
\pgfsetroundjoin%
\definecolor{currentfill}{rgb}{0.000000,0.000000,0.000000}%
\pgfsetfillcolor{currentfill}%
\pgfsetlinewidth{0.602250pt}%
\definecolor{currentstroke}{rgb}{0.000000,0.000000,0.000000}%
\pgfsetstrokecolor{currentstroke}%
\pgfsetdash{}{0pt}%
\pgfsys@defobject{currentmarker}{\pgfqpoint{-0.027778in}{0.000000in}}{\pgfqpoint{0.000000in}{0.000000in}}{%
\pgfpathmoveto{\pgfqpoint{0.000000in}{0.000000in}}%
\pgfpathlineto{\pgfqpoint{-0.027778in}{0.000000in}}%
\pgfusepath{stroke,fill}%
}%
\begin{pgfscope}%
\pgfsys@transformshift{0.675193in}{1.280993in}%
\pgfsys@useobject{currentmarker}{}%
\end{pgfscope}%
\end{pgfscope}%
\begin{pgfscope}%
\pgfsetbuttcap%
\pgfsetroundjoin%
\definecolor{currentfill}{rgb}{0.000000,0.000000,0.000000}%
\pgfsetfillcolor{currentfill}%
\pgfsetlinewidth{0.602250pt}%
\definecolor{currentstroke}{rgb}{0.000000,0.000000,0.000000}%
\pgfsetstrokecolor{currentstroke}%
\pgfsetdash{}{0pt}%
\pgfsys@defobject{currentmarker}{\pgfqpoint{-0.027778in}{0.000000in}}{\pgfqpoint{0.000000in}{0.000000in}}{%
\pgfpathmoveto{\pgfqpoint{0.000000in}{0.000000in}}%
\pgfpathlineto{\pgfqpoint{-0.027778in}{0.000000in}}%
\pgfusepath{stroke,fill}%
}%
\begin{pgfscope}%
\pgfsys@transformshift{0.675193in}{1.501791in}%
\pgfsys@useobject{currentmarker}{}%
\end{pgfscope}%
\end{pgfscope}%
\begin{pgfscope}%
\pgfsetbuttcap%
\pgfsetroundjoin%
\definecolor{currentfill}{rgb}{0.000000,0.000000,0.000000}%
\pgfsetfillcolor{currentfill}%
\pgfsetlinewidth{0.602250pt}%
\definecolor{currentstroke}{rgb}{0.000000,0.000000,0.000000}%
\pgfsetstrokecolor{currentstroke}%
\pgfsetdash{}{0pt}%
\pgfsys@defobject{currentmarker}{\pgfqpoint{-0.027778in}{0.000000in}}{\pgfqpoint{0.000000in}{0.000000in}}{%
\pgfpathmoveto{\pgfqpoint{0.000000in}{0.000000in}}%
\pgfpathlineto{\pgfqpoint{-0.027778in}{0.000000in}}%
\pgfusepath{stroke,fill}%
}%
\begin{pgfscope}%
\pgfsys@transformshift{0.675193in}{1.658449in}%
\pgfsys@useobject{currentmarker}{}%
\end{pgfscope}%
\end{pgfscope}%
\begin{pgfscope}%
\pgfsetbuttcap%
\pgfsetroundjoin%
\definecolor{currentfill}{rgb}{0.000000,0.000000,0.000000}%
\pgfsetfillcolor{currentfill}%
\pgfsetlinewidth{0.602250pt}%
\definecolor{currentstroke}{rgb}{0.000000,0.000000,0.000000}%
\pgfsetstrokecolor{currentstroke}%
\pgfsetdash{}{0pt}%
\pgfsys@defobject{currentmarker}{\pgfqpoint{-0.027778in}{0.000000in}}{\pgfqpoint{0.000000in}{0.000000in}}{%
\pgfpathmoveto{\pgfqpoint{0.000000in}{0.000000in}}%
\pgfpathlineto{\pgfqpoint{-0.027778in}{0.000000in}}%
\pgfusepath{stroke,fill}%
}%
\begin{pgfscope}%
\pgfsys@transformshift{0.675193in}{1.779963in}%
\pgfsys@useobject{currentmarker}{}%
\end{pgfscope}%
\end{pgfscope}%
\begin{pgfscope}%
\pgfsetbuttcap%
\pgfsetroundjoin%
\definecolor{currentfill}{rgb}{0.000000,0.000000,0.000000}%
\pgfsetfillcolor{currentfill}%
\pgfsetlinewidth{0.602250pt}%
\definecolor{currentstroke}{rgb}{0.000000,0.000000,0.000000}%
\pgfsetstrokecolor{currentstroke}%
\pgfsetdash{}{0pt}%
\pgfsys@defobject{currentmarker}{\pgfqpoint{-0.027778in}{0.000000in}}{\pgfqpoint{0.000000in}{0.000000in}}{%
\pgfpathmoveto{\pgfqpoint{0.000000in}{0.000000in}}%
\pgfpathlineto{\pgfqpoint{-0.027778in}{0.000000in}}%
\pgfusepath{stroke,fill}%
}%
\begin{pgfscope}%
\pgfsys@transformshift{0.675193in}{1.879247in}%
\pgfsys@useobject{currentmarker}{}%
\end{pgfscope}%
\end{pgfscope}%
\begin{pgfscope}%
\pgfsetbuttcap%
\pgfsetroundjoin%
\definecolor{currentfill}{rgb}{0.000000,0.000000,0.000000}%
\pgfsetfillcolor{currentfill}%
\pgfsetlinewidth{0.602250pt}%
\definecolor{currentstroke}{rgb}{0.000000,0.000000,0.000000}%
\pgfsetstrokecolor{currentstroke}%
\pgfsetdash{}{0pt}%
\pgfsys@defobject{currentmarker}{\pgfqpoint{-0.027778in}{0.000000in}}{\pgfqpoint{0.000000in}{0.000000in}}{%
\pgfpathmoveto{\pgfqpoint{0.000000in}{0.000000in}}%
\pgfpathlineto{\pgfqpoint{-0.027778in}{0.000000in}}%
\pgfusepath{stroke,fill}%
}%
\begin{pgfscope}%
\pgfsys@transformshift{0.675193in}{1.963191in}%
\pgfsys@useobject{currentmarker}{}%
\end{pgfscope}%
\end{pgfscope}%
\begin{pgfscope}%
\pgfsetbuttcap%
\pgfsetroundjoin%
\definecolor{currentfill}{rgb}{0.000000,0.000000,0.000000}%
\pgfsetfillcolor{currentfill}%
\pgfsetlinewidth{0.602250pt}%
\definecolor{currentstroke}{rgb}{0.000000,0.000000,0.000000}%
\pgfsetstrokecolor{currentstroke}%
\pgfsetdash{}{0pt}%
\pgfsys@defobject{currentmarker}{\pgfqpoint{-0.027778in}{0.000000in}}{\pgfqpoint{0.000000in}{0.000000in}}{%
\pgfpathmoveto{\pgfqpoint{0.000000in}{0.000000in}}%
\pgfpathlineto{\pgfqpoint{-0.027778in}{0.000000in}}%
\pgfusepath{stroke,fill}%
}%
\begin{pgfscope}%
\pgfsys@transformshift{0.675193in}{2.035906in}%
\pgfsys@useobject{currentmarker}{}%
\end{pgfscope}%
\end{pgfscope}%
\begin{pgfscope}%
\pgfsetbuttcap%
\pgfsetroundjoin%
\definecolor{currentfill}{rgb}{0.000000,0.000000,0.000000}%
\pgfsetfillcolor{currentfill}%
\pgfsetlinewidth{0.602250pt}%
\definecolor{currentstroke}{rgb}{0.000000,0.000000,0.000000}%
\pgfsetstrokecolor{currentstroke}%
\pgfsetdash{}{0pt}%
\pgfsys@defobject{currentmarker}{\pgfqpoint{-0.027778in}{0.000000in}}{\pgfqpoint{0.000000in}{0.000000in}}{%
\pgfpathmoveto{\pgfqpoint{0.000000in}{0.000000in}}%
\pgfpathlineto{\pgfqpoint{-0.027778in}{0.000000in}}%
\pgfusepath{stroke,fill}%
}%
\begin{pgfscope}%
\pgfsys@transformshift{0.675193in}{2.100045in}%
\pgfsys@useobject{currentmarker}{}%
\end{pgfscope}%
\end{pgfscope}%
\begin{pgfscope}%
\pgfsetbuttcap%
\pgfsetroundjoin%
\definecolor{currentfill}{rgb}{0.000000,0.000000,0.000000}%
\pgfsetfillcolor{currentfill}%
\pgfsetlinewidth{0.602250pt}%
\definecolor{currentstroke}{rgb}{0.000000,0.000000,0.000000}%
\pgfsetstrokecolor{currentstroke}%
\pgfsetdash{}{0pt}%
\pgfsys@defobject{currentmarker}{\pgfqpoint{-0.027778in}{0.000000in}}{\pgfqpoint{0.000000in}{0.000000in}}{%
\pgfpathmoveto{\pgfqpoint{0.000000in}{0.000000in}}%
\pgfpathlineto{\pgfqpoint{-0.027778in}{0.000000in}}%
\pgfusepath{stroke,fill}%
}%
\begin{pgfscope}%
\pgfsys@transformshift{0.675193in}{2.534876in}%
\pgfsys@useobject{currentmarker}{}%
\end{pgfscope}%
\end{pgfscope}%
\begin{pgfscope}%
\pgfsetbuttcap%
\pgfsetroundjoin%
\definecolor{currentfill}{rgb}{0.000000,0.000000,0.000000}%
\pgfsetfillcolor{currentfill}%
\pgfsetlinewidth{0.602250pt}%
\definecolor{currentstroke}{rgb}{0.000000,0.000000,0.000000}%
\pgfsetstrokecolor{currentstroke}%
\pgfsetdash{}{0pt}%
\pgfsys@defobject{currentmarker}{\pgfqpoint{-0.027778in}{0.000000in}}{\pgfqpoint{0.000000in}{0.000000in}}{%
\pgfpathmoveto{\pgfqpoint{0.000000in}{0.000000in}}%
\pgfpathlineto{\pgfqpoint{-0.027778in}{0.000000in}}%
\pgfusepath{stroke,fill}%
}%
\begin{pgfscope}%
\pgfsys@transformshift{0.675193in}{2.755674in}%
\pgfsys@useobject{currentmarker}{}%
\end{pgfscope}%
\end{pgfscope}%
\begin{pgfscope}%
\pgfsetbuttcap%
\pgfsetroundjoin%
\definecolor{currentfill}{rgb}{0.000000,0.000000,0.000000}%
\pgfsetfillcolor{currentfill}%
\pgfsetlinewidth{0.602250pt}%
\definecolor{currentstroke}{rgb}{0.000000,0.000000,0.000000}%
\pgfsetstrokecolor{currentstroke}%
\pgfsetdash{}{0pt}%
\pgfsys@defobject{currentmarker}{\pgfqpoint{-0.027778in}{0.000000in}}{\pgfqpoint{0.000000in}{0.000000in}}{%
\pgfpathmoveto{\pgfqpoint{0.000000in}{0.000000in}}%
\pgfpathlineto{\pgfqpoint{-0.027778in}{0.000000in}}%
\pgfusepath{stroke,fill}%
}%
\begin{pgfscope}%
\pgfsys@transformshift{0.675193in}{2.912333in}%
\pgfsys@useobject{currentmarker}{}%
\end{pgfscope}%
\end{pgfscope}%
\begin{pgfscope}%
\pgfsetbuttcap%
\pgfsetroundjoin%
\definecolor{currentfill}{rgb}{0.000000,0.000000,0.000000}%
\pgfsetfillcolor{currentfill}%
\pgfsetlinewidth{0.602250pt}%
\definecolor{currentstroke}{rgb}{0.000000,0.000000,0.000000}%
\pgfsetstrokecolor{currentstroke}%
\pgfsetdash{}{0pt}%
\pgfsys@defobject{currentmarker}{\pgfqpoint{-0.027778in}{0.000000in}}{\pgfqpoint{0.000000in}{0.000000in}}{%
\pgfpathmoveto{\pgfqpoint{0.000000in}{0.000000in}}%
\pgfpathlineto{\pgfqpoint{-0.027778in}{0.000000in}}%
\pgfusepath{stroke,fill}%
}%
\begin{pgfscope}%
\pgfsys@transformshift{0.675193in}{3.033847in}%
\pgfsys@useobject{currentmarker}{}%
\end{pgfscope}%
\end{pgfscope}%
\begin{pgfscope}%
\pgfsetbuttcap%
\pgfsetroundjoin%
\definecolor{currentfill}{rgb}{0.000000,0.000000,0.000000}%
\pgfsetfillcolor{currentfill}%
\pgfsetlinewidth{0.602250pt}%
\definecolor{currentstroke}{rgb}{0.000000,0.000000,0.000000}%
\pgfsetstrokecolor{currentstroke}%
\pgfsetdash{}{0pt}%
\pgfsys@defobject{currentmarker}{\pgfqpoint{-0.027778in}{0.000000in}}{\pgfqpoint{0.000000in}{0.000000in}}{%
\pgfpathmoveto{\pgfqpoint{0.000000in}{0.000000in}}%
\pgfpathlineto{\pgfqpoint{-0.027778in}{0.000000in}}%
\pgfusepath{stroke,fill}%
}%
\begin{pgfscope}%
\pgfsys@transformshift{0.675193in}{3.133131in}%
\pgfsys@useobject{currentmarker}{}%
\end{pgfscope}%
\end{pgfscope}%
\begin{pgfscope}%
\pgfsetbuttcap%
\pgfsetroundjoin%
\definecolor{currentfill}{rgb}{0.000000,0.000000,0.000000}%
\pgfsetfillcolor{currentfill}%
\pgfsetlinewidth{0.602250pt}%
\definecolor{currentstroke}{rgb}{0.000000,0.000000,0.000000}%
\pgfsetstrokecolor{currentstroke}%
\pgfsetdash{}{0pt}%
\pgfsys@defobject{currentmarker}{\pgfqpoint{-0.027778in}{0.000000in}}{\pgfqpoint{0.000000in}{0.000000in}}{%
\pgfpathmoveto{\pgfqpoint{0.000000in}{0.000000in}}%
\pgfpathlineto{\pgfqpoint{-0.027778in}{0.000000in}}%
\pgfusepath{stroke,fill}%
}%
\begin{pgfscope}%
\pgfsys@transformshift{0.675193in}{3.217074in}%
\pgfsys@useobject{currentmarker}{}%
\end{pgfscope}%
\end{pgfscope}%
\begin{pgfscope}%
\pgfsetbuttcap%
\pgfsetroundjoin%
\definecolor{currentfill}{rgb}{0.000000,0.000000,0.000000}%
\pgfsetfillcolor{currentfill}%
\pgfsetlinewidth{0.602250pt}%
\definecolor{currentstroke}{rgb}{0.000000,0.000000,0.000000}%
\pgfsetstrokecolor{currentstroke}%
\pgfsetdash{}{0pt}%
\pgfsys@defobject{currentmarker}{\pgfqpoint{-0.027778in}{0.000000in}}{\pgfqpoint{0.000000in}{0.000000in}}{%
\pgfpathmoveto{\pgfqpoint{0.000000in}{0.000000in}}%
\pgfpathlineto{\pgfqpoint{-0.027778in}{0.000000in}}%
\pgfusepath{stroke,fill}%
}%
\begin{pgfscope}%
\pgfsys@transformshift{0.675193in}{3.289789in}%
\pgfsys@useobject{currentmarker}{}%
\end{pgfscope}%
\end{pgfscope}%
\begin{pgfscope}%
\pgfsetbuttcap%
\pgfsetroundjoin%
\definecolor{currentfill}{rgb}{0.000000,0.000000,0.000000}%
\pgfsetfillcolor{currentfill}%
\pgfsetlinewidth{0.602250pt}%
\definecolor{currentstroke}{rgb}{0.000000,0.000000,0.000000}%
\pgfsetstrokecolor{currentstroke}%
\pgfsetdash{}{0pt}%
\pgfsys@defobject{currentmarker}{\pgfqpoint{-0.027778in}{0.000000in}}{\pgfqpoint{0.000000in}{0.000000in}}{%
\pgfpathmoveto{\pgfqpoint{0.000000in}{0.000000in}}%
\pgfpathlineto{\pgfqpoint{-0.027778in}{0.000000in}}%
\pgfusepath{stroke,fill}%
}%
\begin{pgfscope}%
\pgfsys@transformshift{0.675193in}{3.353929in}%
\pgfsys@useobject{currentmarker}{}%
\end{pgfscope}%
\end{pgfscope}%
\begin{pgfscope}%
\definecolor{textcolor}{rgb}{0.000000,0.000000,0.000000}%
\pgfsetstrokecolor{textcolor}%
\pgfsetfillcolor{textcolor}%
\pgftext[x=0.234413in,y=2.036079in,,bottom,rotate=90.000000]{\color{textcolor}\rmfamily\fontsize{10.000000}{12.000000}\selectfont Dimensionless force}%
\end{pgfscope}%
\begin{pgfscope}%
\pgfpathrectangle{\pgfqpoint{0.675193in}{0.526079in}}{\pgfqpoint{4.650000in}{3.020000in}}%
\pgfusepath{clip}%
\pgfsetrectcap%
\pgfsetroundjoin%
\pgfsetlinewidth{1.505625pt}%
\definecolor{currentstroke}{rgb}{1.000000,0.000000,0.000000}%
\pgfsetstrokecolor{currentstroke}%
\pgfsetstrokeopacity{0.750000}%
\pgfsetdash{}{0pt}%
\pgfpathmoveto{\pgfqpoint{0.991862in}{3.368536in}}%
\pgfpathlineto{\pgfqpoint{1.017859in}{3.371958in}}%
\pgfpathlineto{\pgfqpoint{1.042736in}{3.375299in}}%
\pgfpathlineto{\pgfqpoint{1.066497in}{3.378100in}}%
\pgfusepath{stroke}%
\end{pgfscope}%
\begin{pgfscope}%
\pgfpathrectangle{\pgfqpoint{0.675193in}{0.526079in}}{\pgfqpoint{4.650000in}{3.020000in}}%
\pgfusepath{clip}%
\pgfsetrectcap%
\pgfsetroundjoin%
\pgfsetlinewidth{1.505625pt}%
\definecolor{currentstroke}{rgb}{0.000000,0.000000,1.000000}%
\pgfsetstrokecolor{currentstroke}%
\pgfsetstrokeopacity{0.750000}%
\pgfsetdash{}{0pt}%
\pgfpathmoveto{\pgfqpoint{0.991862in}{1.209667in}}%
\pgfpathlineto{\pgfqpoint{1.017859in}{1.186125in}}%
\pgfpathlineto{\pgfqpoint{1.042736in}{1.161377in}}%
\pgfpathlineto{\pgfqpoint{1.066497in}{1.134828in}}%
\pgfusepath{stroke}%
\end{pgfscope}%
\begin{pgfscope}%
\pgfpathrectangle{\pgfqpoint{0.675193in}{0.526079in}}{\pgfqpoint{4.650000in}{3.020000in}}%
\pgfusepath{clip}%
\pgfsetrectcap%
\pgfsetroundjoin%
\pgfsetlinewidth{1.505625pt}%
\definecolor{currentstroke}{rgb}{0.000000,0.750000,0.750000}%
\pgfsetstrokecolor{currentstroke}%
\pgfsetstrokeopacity{0.750000}%
\pgfsetdash{}{0pt}%
\pgfpathmoveto{\pgfqpoint{0.991862in}{1.867466in}}%
\pgfpathlineto{\pgfqpoint{1.017859in}{1.807053in}}%
\pgfpathlineto{\pgfqpoint{1.042736in}{1.753572in}}%
\pgfpathlineto{\pgfqpoint{1.066497in}{1.705539in}}%
\pgfusepath{stroke}%
\end{pgfscope}%
\begin{pgfscope}%
\pgfpathrectangle{\pgfqpoint{0.675193in}{0.526079in}}{\pgfqpoint{4.650000in}{3.020000in}}%
\pgfusepath{clip}%
\pgfsetrectcap%
\pgfsetroundjoin%
\pgfsetlinewidth{1.505625pt}%
\definecolor{currentstroke}{rgb}{1.000000,0.000000,0.000000}%
\pgfsetstrokecolor{currentstroke}%
\pgfsetstrokeopacity{0.750000}%
\pgfsetdash{}{0pt}%
\pgfpathmoveto{\pgfqpoint{0.965930in}{3.361237in}}%
\pgfpathlineto{\pgfqpoint{0.995533in}{3.365490in}}%
\pgfpathlineto{\pgfqpoint{1.023927in}{3.369587in}}%
\pgfpathlineto{\pgfqpoint{1.051125in}{3.372966in}}%
\pgfpathlineto{\pgfqpoint{1.077137in}{3.375820in}}%
\pgfpathlineto{\pgfqpoint{1.101975in}{3.378274in}}%
\pgfpathlineto{\pgfqpoint{1.125650in}{3.380419in}}%
\pgfpathlineto{\pgfqpoint{1.148175in}{3.382321in}}%
\pgfusepath{stroke}%
\end{pgfscope}%
\begin{pgfscope}%
\pgfpathrectangle{\pgfqpoint{0.675193in}{0.526079in}}{\pgfqpoint{4.650000in}{3.020000in}}%
\pgfusepath{clip}%
\pgfsetrectcap%
\pgfsetroundjoin%
\pgfsetlinewidth{1.505625pt}%
\definecolor{currentstroke}{rgb}{0.000000,0.000000,1.000000}%
\pgfsetstrokecolor{currentstroke}%
\pgfsetstrokeopacity{0.750000}%
\pgfsetdash{}{0pt}%
\pgfpathmoveto{\pgfqpoint{0.965930in}{1.466956in}}%
\pgfpathlineto{\pgfqpoint{0.995533in}{1.448329in}}%
\pgfpathlineto{\pgfqpoint{1.023927in}{1.428733in}}%
\pgfpathlineto{\pgfqpoint{1.051125in}{1.407502in}}%
\pgfpathlineto{\pgfqpoint{1.077137in}{1.384725in}}%
\pgfpathlineto{\pgfqpoint{1.101975in}{1.360424in}}%
\pgfpathlineto{\pgfqpoint{1.125650in}{1.334579in}}%
\pgfpathlineto{\pgfqpoint{1.148175in}{1.307137in}}%
\pgfusepath{stroke}%
\end{pgfscope}%
\begin{pgfscope}%
\pgfpathrectangle{\pgfqpoint{0.675193in}{0.526079in}}{\pgfqpoint{4.650000in}{3.020000in}}%
\pgfusepath{clip}%
\pgfsetrectcap%
\pgfsetroundjoin%
\pgfsetlinewidth{1.505625pt}%
\definecolor{currentstroke}{rgb}{0.000000,0.750000,0.750000}%
\pgfsetstrokecolor{currentstroke}%
\pgfsetstrokeopacity{0.750000}%
\pgfsetdash{}{0pt}%
\pgfpathmoveto{\pgfqpoint{0.965930in}{1.885088in}}%
\pgfpathlineto{\pgfqpoint{0.995533in}{1.812143in}}%
\pgfpathlineto{\pgfqpoint{1.023927in}{1.748092in}}%
\pgfpathlineto{\pgfqpoint{1.051125in}{1.690887in}}%
\pgfpathlineto{\pgfqpoint{1.077137in}{1.639571in}}%
\pgfpathlineto{\pgfqpoint{1.101975in}{1.593370in}}%
\pgfpathlineto{\pgfqpoint{1.125650in}{1.551648in}}%
\pgfpathlineto{\pgfqpoint{1.148175in}{1.513884in}}%
\pgfusepath{stroke}%
\end{pgfscope}%
\begin{pgfscope}%
\pgfpathrectangle{\pgfqpoint{0.675193in}{0.526079in}}{\pgfqpoint{4.650000in}{3.020000in}}%
\pgfusepath{clip}%
\pgfsetrectcap%
\pgfsetroundjoin%
\pgfsetlinewidth{1.505625pt}%
\definecolor{currentstroke}{rgb}{1.000000,0.000000,0.000000}%
\pgfsetstrokecolor{currentstroke}%
\pgfsetstrokeopacity{0.750000}%
\pgfsetdash{}{0pt}%
\pgfpathmoveto{\pgfqpoint{1.041681in}{3.359676in}}%
\pgfpathlineto{\pgfqpoint{1.077630in}{3.364661in}}%
\pgfpathlineto{\pgfqpoint{1.112004in}{3.369467in}}%
\pgfpathlineto{\pgfqpoint{1.144820in}{3.373331in}}%
\pgfpathlineto{\pgfqpoint{1.176094in}{3.376501in}}%
\pgfpathlineto{\pgfqpoint{1.205844in}{3.379147in}}%
\pgfpathlineto{\pgfqpoint{1.234084in}{3.381388in}}%
\pgfusepath{stroke}%
\end{pgfscope}%
\begin{pgfscope}%
\pgfpathrectangle{\pgfqpoint{0.675193in}{0.526079in}}{\pgfqpoint{4.650000in}{3.020000in}}%
\pgfusepath{clip}%
\pgfsetrectcap%
\pgfsetroundjoin%
\pgfsetlinewidth{1.505625pt}%
\definecolor{currentstroke}{rgb}{0.000000,0.000000,1.000000}%
\pgfsetstrokecolor{currentstroke}%
\pgfsetstrokeopacity{0.750000}%
\pgfsetdash{}{0pt}%
\pgfpathmoveto{\pgfqpoint{1.041681in}{0.978295in}}%
\pgfpathlineto{\pgfqpoint{1.077630in}{0.957448in}}%
\pgfpathlineto{\pgfqpoint{1.112004in}{0.935486in}}%
\pgfpathlineto{\pgfqpoint{1.144820in}{0.911499in}}%
\pgfpathlineto{\pgfqpoint{1.176094in}{0.885598in}}%
\pgfpathlineto{\pgfqpoint{1.205844in}{0.857811in}}%
\pgfpathlineto{\pgfqpoint{1.234084in}{0.828109in}}%
\pgfusepath{stroke}%
\end{pgfscope}%
\begin{pgfscope}%
\pgfpathrectangle{\pgfqpoint{0.675193in}{0.526079in}}{\pgfqpoint{4.650000in}{3.020000in}}%
\pgfusepath{clip}%
\pgfsetrectcap%
\pgfsetroundjoin%
\pgfsetlinewidth{1.505625pt}%
\definecolor{currentstroke}{rgb}{0.000000,0.750000,0.750000}%
\pgfsetstrokecolor{currentstroke}%
\pgfsetstrokeopacity{0.750000}%
\pgfsetdash{}{0pt}%
\pgfpathmoveto{\pgfqpoint{1.041681in}{2.037695in}}%
\pgfpathlineto{\pgfqpoint{1.077630in}{1.966579in}}%
\pgfpathlineto{\pgfqpoint{1.112004in}{1.904641in}}%
\pgfpathlineto{\pgfqpoint{1.144820in}{1.849532in}}%
\pgfpathlineto{\pgfqpoint{1.176094in}{1.800289in}}%
\pgfpathlineto{\pgfqpoint{1.205844in}{1.756128in}}%
\pgfpathlineto{\pgfqpoint{1.234084in}{1.716412in}}%
\pgfusepath{stroke}%
\end{pgfscope}%
\begin{pgfscope}%
\pgfpathrectangle{\pgfqpoint{0.675193in}{0.526079in}}{\pgfqpoint{4.650000in}{3.020000in}}%
\pgfusepath{clip}%
\pgfsetrectcap%
\pgfsetroundjoin%
\pgfsetlinewidth{1.505625pt}%
\definecolor{currentstroke}{rgb}{1.000000,0.000000,0.000000}%
\pgfsetstrokecolor{currentstroke}%
\pgfsetstrokeopacity{0.750000}%
\pgfsetdash{}{0pt}%
\pgfpathmoveto{\pgfqpoint{0.886557in}{3.331498in}}%
\pgfpathlineto{\pgfqpoint{0.917018in}{3.338080in}}%
\pgfpathlineto{\pgfqpoint{0.947047in}{3.344108in}}%
\pgfpathlineto{\pgfqpoint{0.976642in}{3.348797in}}%
\pgfpathlineto{\pgfqpoint{1.005800in}{3.352551in}}%
\pgfpathlineto{\pgfqpoint{1.034518in}{3.355628in}}%
\pgfpathlineto{\pgfqpoint{1.062793in}{3.358199in}}%
\pgfpathlineto{\pgfqpoint{1.090622in}{3.360390in}}%
\pgfpathlineto{\pgfqpoint{1.118001in}{3.362289in}}%
\pgfpathlineto{\pgfqpoint{1.144929in}{3.363957in}}%
\pgfpathlineto{\pgfqpoint{1.171401in}{3.365442in}}%
\pgfpathlineto{\pgfqpoint{1.197416in}{3.366780in}}%
\pgfpathlineto{\pgfqpoint{1.222969in}{3.367998in}}%
\pgfpathlineto{\pgfqpoint{1.248048in}{3.369118in}}%
\pgfpathlineto{\pgfqpoint{1.272654in}{3.370159in}}%
\pgfpathlineto{\pgfqpoint{1.296784in}{3.371131in}}%
\pgfpathlineto{\pgfqpoint{1.320436in}{3.372047in}}%
\pgfpathlineto{\pgfqpoint{1.343612in}{3.372915in}}%
\pgfpathlineto{\pgfqpoint{1.366312in}{3.373743in}}%
\pgfpathlineto{\pgfqpoint{1.388539in}{3.374537in}}%
\pgfpathlineto{\pgfqpoint{1.410295in}{3.375300in}}%
\pgfpathlineto{\pgfqpoint{1.431585in}{3.376039in}}%
\pgfpathlineto{\pgfqpoint{1.452415in}{3.376757in}}%
\pgfpathlineto{\pgfqpoint{1.472790in}{3.377455in}}%
\pgfpathlineto{\pgfqpoint{1.492716in}{3.378138in}}%
\pgfpathlineto{\pgfqpoint{1.512201in}{3.378806in}}%
\pgfpathlineto{\pgfqpoint{1.531251in}{3.379462in}}%
\pgfpathlineto{\pgfqpoint{1.549873in}{3.380107in}}%
\pgfpathlineto{\pgfqpoint{1.568071in}{3.380743in}}%
\pgfpathlineto{\pgfqpoint{1.585852in}{3.381371in}}%
\pgfpathlineto{\pgfqpoint{1.603221in}{3.381991in}}%
\pgfpathlineto{\pgfqpoint{1.620183in}{3.382605in}}%
\pgfpathlineto{\pgfqpoint{1.636740in}{3.383213in}}%
\pgfpathlineto{\pgfqpoint{1.652898in}{3.383815in}}%
\pgfpathlineto{\pgfqpoint{1.668658in}{3.384413in}}%
\pgfpathlineto{\pgfqpoint{1.684024in}{3.385008in}}%
\pgfpathlineto{\pgfqpoint{1.698999in}{3.385598in}}%
\pgfpathlineto{\pgfqpoint{1.713585in}{3.386185in}}%
\pgfpathlineto{\pgfqpoint{1.727784in}{3.386769in}}%
\pgfpathlineto{\pgfqpoint{1.741600in}{3.387350in}}%
\pgfpathlineto{\pgfqpoint{1.755033in}{3.387927in}}%
\pgfpathlineto{\pgfqpoint{1.768088in}{3.388503in}}%
\pgfpathlineto{\pgfqpoint{1.780767in}{3.389076in}}%
\pgfpathlineto{\pgfqpoint{1.793072in}{3.389645in}}%
\pgfpathlineto{\pgfqpoint{1.805007in}{3.390213in}}%
\pgfpathlineto{\pgfqpoint{1.816573in}{3.390778in}}%
\pgfpathlineto{\pgfqpoint{1.827775in}{3.391340in}}%
\pgfpathlineto{\pgfqpoint{1.838615in}{3.391900in}}%
\pgfpathlineto{\pgfqpoint{1.849095in}{3.392458in}}%
\pgfpathlineto{\pgfqpoint{1.859219in}{3.393013in}}%
\pgfpathlineto{\pgfqpoint{1.868987in}{3.393564in}}%
\pgfpathlineto{\pgfqpoint{1.878404in}{3.394113in}}%
\pgfpathlineto{\pgfqpoint{1.887470in}{3.394660in}}%
\pgfpathlineto{\pgfqpoint{1.896187in}{3.395203in}}%
\pgfpathlineto{\pgfqpoint{1.904557in}{3.395744in}}%
\pgfpathlineto{\pgfqpoint{1.912581in}{3.396281in}}%
\pgfpathlineto{\pgfqpoint{1.920262in}{3.396815in}}%
\pgfpathlineto{\pgfqpoint{1.927599in}{3.397344in}}%
\pgfpathlineto{\pgfqpoint{1.934594in}{3.397868in}}%
\pgfusepath{stroke}%
\end{pgfscope}%
\begin{pgfscope}%
\pgfpathrectangle{\pgfqpoint{0.675193in}{0.526079in}}{\pgfqpoint{4.650000in}{3.020000in}}%
\pgfusepath{clip}%
\pgfsetrectcap%
\pgfsetroundjoin%
\pgfsetlinewidth{1.505625pt}%
\definecolor{currentstroke}{rgb}{0.000000,0.000000,1.000000}%
\pgfsetstrokecolor{currentstroke}%
\pgfsetstrokeopacity{0.750000}%
\pgfsetdash{}{0pt}%
\pgfpathmoveto{\pgfqpoint{0.886557in}{1.883580in}}%
\pgfpathlineto{\pgfqpoint{0.917018in}{1.883838in}}%
\pgfpathlineto{\pgfqpoint{0.947047in}{1.883579in}}%
\pgfpathlineto{\pgfqpoint{0.976642in}{1.881975in}}%
\pgfpathlineto{\pgfqpoint{1.005800in}{1.879400in}}%
\pgfpathlineto{\pgfqpoint{1.034518in}{1.876086in}}%
\pgfpathlineto{\pgfqpoint{1.062793in}{1.872188in}}%
\pgfpathlineto{\pgfqpoint{1.090622in}{1.867814in}}%
\pgfpathlineto{\pgfqpoint{1.118001in}{1.863040in}}%
\pgfpathlineto{\pgfqpoint{1.144929in}{1.857914in}}%
\pgfpathlineto{\pgfqpoint{1.171401in}{1.852476in}}%
\pgfpathlineto{\pgfqpoint{1.197416in}{1.846752in}}%
\pgfpathlineto{\pgfqpoint{1.222969in}{1.840759in}}%
\pgfpathlineto{\pgfqpoint{1.248048in}{1.834514in}}%
\pgfpathlineto{\pgfqpoint{1.272654in}{1.828024in}}%
\pgfpathlineto{\pgfqpoint{1.296784in}{1.821296in}}%
\pgfpathlineto{\pgfqpoint{1.320436in}{1.814333in}}%
\pgfpathlineto{\pgfqpoint{1.343612in}{1.807136in}}%
\pgfpathlineto{\pgfqpoint{1.366312in}{1.799706in}}%
\pgfpathlineto{\pgfqpoint{1.388539in}{1.792043in}}%
\pgfpathlineto{\pgfqpoint{1.410295in}{1.784142in}}%
\pgfpathlineto{\pgfqpoint{1.431585in}{1.776003in}}%
\pgfpathlineto{\pgfqpoint{1.452415in}{1.767621in}}%
\pgfpathlineto{\pgfqpoint{1.472790in}{1.758991in}}%
\pgfpathlineto{\pgfqpoint{1.492716in}{1.750109in}}%
\pgfpathlineto{\pgfqpoint{1.512201in}{1.740968in}}%
\pgfpathlineto{\pgfqpoint{1.531251in}{1.731562in}}%
\pgfpathlineto{\pgfqpoint{1.549873in}{1.721885in}}%
\pgfpathlineto{\pgfqpoint{1.568071in}{1.711928in}}%
\pgfpathlineto{\pgfqpoint{1.585852in}{1.701684in}}%
\pgfpathlineto{\pgfqpoint{1.603221in}{1.691144in}}%
\pgfpathlineto{\pgfqpoint{1.620183in}{1.680299in}}%
\pgfpathlineto{\pgfqpoint{1.636740in}{1.669139in}}%
\pgfpathlineto{\pgfqpoint{1.652898in}{1.657653in}}%
\pgfpathlineto{\pgfqpoint{1.668658in}{1.645830in}}%
\pgfpathlineto{\pgfqpoint{1.684024in}{1.633660in}}%
\pgfpathlineto{\pgfqpoint{1.698999in}{1.621126in}}%
\pgfpathlineto{\pgfqpoint{1.713585in}{1.608217in}}%
\pgfpathlineto{\pgfqpoint{1.727784in}{1.594918in}}%
\pgfpathlineto{\pgfqpoint{1.741600in}{1.581213in}}%
\pgfpathlineto{\pgfqpoint{1.755033in}{1.567083in}}%
\pgfpathlineto{\pgfqpoint{1.768088in}{1.552512in}}%
\pgfpathlineto{\pgfqpoint{1.780767in}{1.537479in}}%
\pgfpathlineto{\pgfqpoint{1.793072in}{1.521963in}}%
\pgfpathlineto{\pgfqpoint{1.805007in}{1.505941in}}%
\pgfpathlineto{\pgfqpoint{1.816573in}{1.489387in}}%
\pgfpathlineto{\pgfqpoint{1.827775in}{1.472274in}}%
\pgfpathlineto{\pgfqpoint{1.838615in}{1.454573in}}%
\pgfpathlineto{\pgfqpoint{1.849095in}{1.436251in}}%
\pgfpathlineto{\pgfqpoint{1.859219in}{1.417273in}}%
\pgfpathlineto{\pgfqpoint{1.868987in}{1.397599in}}%
\pgfpathlineto{\pgfqpoint{1.878404in}{1.377187in}}%
\pgfpathlineto{\pgfqpoint{1.887470in}{1.355989in}}%
\pgfpathlineto{\pgfqpoint{1.896187in}{1.333952in}}%
\pgfpathlineto{\pgfqpoint{1.904557in}{1.311019in}}%
\pgfpathlineto{\pgfqpoint{1.912581in}{1.287123in}}%
\pgfpathlineto{\pgfqpoint{1.920262in}{1.262190in}}%
\pgfpathlineto{\pgfqpoint{1.927599in}{1.236138in}}%
\pgfpathlineto{\pgfqpoint{1.934594in}{1.208873in}}%
\pgfusepath{stroke}%
\end{pgfscope}%
\begin{pgfscope}%
\pgfpathrectangle{\pgfqpoint{0.675193in}{0.526079in}}{\pgfqpoint{4.650000in}{3.020000in}}%
\pgfusepath{clip}%
\pgfsetrectcap%
\pgfsetroundjoin%
\pgfsetlinewidth{1.505625pt}%
\definecolor{currentstroke}{rgb}{0.000000,0.750000,0.750000}%
\pgfsetstrokecolor{currentstroke}%
\pgfsetstrokeopacity{0.750000}%
\pgfsetdash{}{0pt}%
\pgfpathmoveto{\pgfqpoint{0.886557in}{2.017788in}}%
\pgfpathlineto{\pgfqpoint{0.917018in}{1.919406in}}%
\pgfpathlineto{\pgfqpoint{0.947047in}{1.832835in}}%
\pgfpathlineto{\pgfqpoint{0.976642in}{1.754977in}}%
\pgfpathlineto{\pgfqpoint{1.005800in}{1.684516in}}%
\pgfpathlineto{\pgfqpoint{1.034518in}{1.620389in}}%
\pgfpathlineto{\pgfqpoint{1.062793in}{1.561739in}}%
\pgfpathlineto{\pgfqpoint{1.090622in}{1.507868in}}%
\pgfpathlineto{\pgfqpoint{1.118001in}{1.458199in}}%
\pgfpathlineto{\pgfqpoint{1.144929in}{1.412245in}}%
\pgfpathlineto{\pgfqpoint{1.171401in}{1.369602in}}%
\pgfpathlineto{\pgfqpoint{1.197416in}{1.329922in}}%
\pgfpathlineto{\pgfqpoint{1.222969in}{1.292908in}}%
\pgfpathlineto{\pgfqpoint{1.248048in}{1.258306in}}%
\pgfpathlineto{\pgfqpoint{1.272654in}{1.225894in}}%
\pgfpathlineto{\pgfqpoint{1.296784in}{1.195476in}}%
\pgfpathlineto{\pgfqpoint{1.320436in}{1.166883in}}%
\pgfpathlineto{\pgfqpoint{1.343612in}{1.139964in}}%
\pgfpathlineto{\pgfqpoint{1.366312in}{1.114586in}}%
\pgfpathlineto{\pgfqpoint{1.388539in}{1.090630in}}%
\pgfpathlineto{\pgfqpoint{1.410295in}{1.067989in}}%
\pgfpathlineto{\pgfqpoint{1.431585in}{1.046569in}}%
\pgfpathlineto{\pgfqpoint{1.452415in}{1.026285in}}%
\pgfpathlineto{\pgfqpoint{1.472790in}{1.007057in}}%
\pgfpathlineto{\pgfqpoint{1.492716in}{0.988817in}}%
\pgfpathlineto{\pgfqpoint{1.512201in}{0.971500in}}%
\pgfpathlineto{\pgfqpoint{1.531251in}{0.955048in}}%
\pgfpathlineto{\pgfqpoint{1.549873in}{0.939407in}}%
\pgfpathlineto{\pgfqpoint{1.568071in}{0.924531in}}%
\pgfpathlineto{\pgfqpoint{1.585852in}{0.910373in}}%
\pgfpathlineto{\pgfqpoint{1.603221in}{0.896893in}}%
\pgfpathlineto{\pgfqpoint{1.620183in}{0.884053in}}%
\pgfpathlineto{\pgfqpoint{1.636740in}{0.871818in}}%
\pgfpathlineto{\pgfqpoint{1.652898in}{0.860157in}}%
\pgfpathlineto{\pgfqpoint{1.668658in}{0.849040in}}%
\pgfpathlineto{\pgfqpoint{1.684024in}{0.838439in}}%
\pgfpathlineto{\pgfqpoint{1.698999in}{0.828328in}}%
\pgfpathlineto{\pgfqpoint{1.713585in}{0.818683in}}%
\pgfpathlineto{\pgfqpoint{1.727784in}{0.809484in}}%
\pgfpathlineto{\pgfqpoint{1.741600in}{0.800708in}}%
\pgfpathlineto{\pgfqpoint{1.755033in}{0.792337in}}%
\pgfpathlineto{\pgfqpoint{1.768088in}{0.784353in}}%
\pgfpathlineto{\pgfqpoint{1.780767in}{0.776739in}}%
\pgfpathlineto{\pgfqpoint{1.793072in}{0.769478in}}%
\pgfpathlineto{\pgfqpoint{1.805007in}{0.762558in}}%
\pgfpathlineto{\pgfqpoint{1.816573in}{0.755964in}}%
\pgfpathlineto{\pgfqpoint{1.827775in}{0.749683in}}%
\pgfpathlineto{\pgfqpoint{1.838615in}{0.743702in}}%
\pgfpathlineto{\pgfqpoint{1.849095in}{0.738012in}}%
\pgfpathlineto{\pgfqpoint{1.859219in}{0.732600in}}%
\pgfpathlineto{\pgfqpoint{1.868987in}{0.727457in}}%
\pgfpathlineto{\pgfqpoint{1.878404in}{0.722574in}}%
\pgfpathlineto{\pgfqpoint{1.887470in}{0.717942in}}%
\pgfpathlineto{\pgfqpoint{1.896187in}{0.713552in}}%
\pgfpathlineto{\pgfqpoint{1.904557in}{0.709398in}}%
\pgfpathlineto{\pgfqpoint{1.912581in}{0.705470in}}%
\pgfpathlineto{\pgfqpoint{1.920262in}{0.701763in}}%
\pgfpathlineto{\pgfqpoint{1.927599in}{0.698270in}}%
\pgfpathlineto{\pgfqpoint{1.934594in}{0.694983in}}%
\pgfusepath{stroke}%
\end{pgfscope}%
\begin{pgfscope}%
\pgfpathrectangle{\pgfqpoint{0.675193in}{0.526079in}}{\pgfqpoint{4.650000in}{3.020000in}}%
\pgfusepath{clip}%
\pgfsetrectcap%
\pgfsetroundjoin%
\pgfsetlinewidth{1.505625pt}%
\definecolor{currentstroke}{rgb}{1.000000,0.000000,0.000000}%
\pgfsetstrokecolor{currentstroke}%
\pgfsetstrokeopacity{0.750000}%
\pgfsetdash{}{0pt}%
\pgfpathmoveto{\pgfqpoint{0.954489in}{3.336626in}}%
\pgfpathlineto{\pgfqpoint{0.988306in}{3.343309in}}%
\pgfpathlineto{\pgfqpoint{1.021505in}{3.349518in}}%
\pgfpathlineto{\pgfqpoint{1.054091in}{3.354442in}}%
\pgfpathlineto{\pgfqpoint{1.086070in}{3.358437in}}%
\pgfpathlineto{\pgfqpoint{1.117448in}{3.361742in}}%
\pgfpathlineto{\pgfqpoint{1.148230in}{3.364519in}}%
\pgfpathlineto{\pgfqpoint{1.178420in}{3.366889in}}%
\pgfpathlineto{\pgfqpoint{1.208025in}{3.368941in}}%
\pgfpathlineto{\pgfqpoint{1.237049in}{3.370738in}}%
\pgfpathlineto{\pgfqpoint{1.265499in}{3.372329in}}%
\pgfpathlineto{\pgfqpoint{1.293379in}{3.373752in}}%
\pgfpathlineto{\pgfqpoint{1.320695in}{3.375036in}}%
\pgfpathlineto{\pgfqpoint{1.347455in}{3.376206in}}%
\pgfpathlineto{\pgfqpoint{1.373663in}{3.377279in}}%
\pgfpathlineto{\pgfqpoint{1.399324in}{3.378272in}}%
\pgfpathlineto{\pgfqpoint{1.424442in}{3.379195in}}%
\pgfpathlineto{\pgfqpoint{1.449023in}{3.380059in}}%
\pgfpathlineto{\pgfqpoint{1.473071in}{3.380873in}}%
\pgfpathlineto{\pgfqpoint{1.496589in}{3.381643in}}%
\pgfpathlineto{\pgfqpoint{1.519582in}{3.382374in}}%
\pgfpathlineto{\pgfqpoint{1.542053in}{3.383073in}}%
\pgfpathlineto{\pgfqpoint{1.564006in}{3.383743in}}%
\pgfpathlineto{\pgfqpoint{1.585446in}{3.384387in}}%
\pgfpathlineto{\pgfqpoint{1.606377in}{3.385010in}}%
\pgfpathlineto{\pgfqpoint{1.626802in}{3.385612in}}%
\pgfpathlineto{\pgfqpoint{1.646726in}{3.386196in}}%
\pgfpathlineto{\pgfqpoint{1.666156in}{3.386765in}}%
\pgfpathlineto{\pgfqpoint{1.685096in}{3.387319in}}%
\pgfpathlineto{\pgfqpoint{1.703553in}{3.387861in}}%
\pgfpathlineto{\pgfqpoint{1.721531in}{3.388391in}}%
\pgfpathlineto{\pgfqpoint{1.739037in}{3.388911in}}%
\pgfpathlineto{\pgfqpoint{1.756077in}{3.389421in}}%
\pgfpathlineto{\pgfqpoint{1.772657in}{3.389923in}}%
\pgfpathlineto{\pgfqpoint{1.788784in}{3.390417in}}%
\pgfpathlineto{\pgfqpoint{1.804462in}{3.390903in}}%
\pgfpathlineto{\pgfqpoint{1.819697in}{3.391383in}}%
\pgfpathlineto{\pgfqpoint{1.834494in}{3.391856in}}%
\pgfpathlineto{\pgfqpoint{1.848858in}{3.392323in}}%
\pgfpathlineto{\pgfqpoint{1.862793in}{3.392785in}}%
\pgfpathlineto{\pgfqpoint{1.876303in}{3.393241in}}%
\pgfpathlineto{\pgfqpoint{1.889391in}{3.393692in}}%
\pgfpathlineto{\pgfqpoint{1.902061in}{3.394138in}}%
\pgfpathlineto{\pgfqpoint{1.914315in}{3.394579in}}%
\pgfpathlineto{\pgfqpoint{1.926156in}{3.395015in}}%
\pgfpathlineto{\pgfqpoint{1.937586in}{3.395446in}}%
\pgfpathlineto{\pgfqpoint{1.948607in}{3.395874in}}%
\pgfpathlineto{\pgfqpoint{1.959221in}{3.396296in}}%
\pgfpathlineto{\pgfqpoint{1.969429in}{3.396714in}}%
\pgfpathlineto{\pgfqpoint{1.979233in}{3.397128in}}%
\pgfpathlineto{\pgfqpoint{1.988636in}{3.397537in}}%
\pgfpathlineto{\pgfqpoint{1.997640in}{3.397943in}}%
\pgfpathlineto{\pgfqpoint{2.006245in}{3.398343in}}%
\pgfpathlineto{\pgfqpoint{2.014455in}{3.398738in}}%
\pgfusepath{stroke}%
\end{pgfscope}%
\begin{pgfscope}%
\pgfpathrectangle{\pgfqpoint{0.675193in}{0.526079in}}{\pgfqpoint{4.650000in}{3.020000in}}%
\pgfusepath{clip}%
\pgfsetrectcap%
\pgfsetroundjoin%
\pgfsetlinewidth{1.505625pt}%
\definecolor{currentstroke}{rgb}{0.000000,0.000000,1.000000}%
\pgfsetstrokecolor{currentstroke}%
\pgfsetstrokeopacity{0.750000}%
\pgfsetdash{}{0pt}%
\pgfpathmoveto{\pgfqpoint{0.954489in}{1.659320in}}%
\pgfpathlineto{\pgfqpoint{0.988306in}{1.659068in}}%
\pgfpathlineto{\pgfqpoint{1.021505in}{1.658378in}}%
\pgfpathlineto{\pgfqpoint{1.054091in}{1.656393in}}%
\pgfpathlineto{\pgfqpoint{1.086070in}{1.653435in}}%
\pgfpathlineto{\pgfqpoint{1.117448in}{1.649713in}}%
\pgfpathlineto{\pgfqpoint{1.148230in}{1.645369in}}%
\pgfpathlineto{\pgfqpoint{1.178420in}{1.640506in}}%
\pgfpathlineto{\pgfqpoint{1.208025in}{1.635196in}}%
\pgfpathlineto{\pgfqpoint{1.237049in}{1.629489in}}%
\pgfpathlineto{\pgfqpoint{1.265499in}{1.623421in}}%
\pgfpathlineto{\pgfqpoint{1.293379in}{1.617019in}}%
\pgfpathlineto{\pgfqpoint{1.320695in}{1.610302in}}%
\pgfpathlineto{\pgfqpoint{1.347455in}{1.603283in}}%
\pgfpathlineto{\pgfqpoint{1.373663in}{1.595972in}}%
\pgfpathlineto{\pgfqpoint{1.399324in}{1.588374in}}%
\pgfpathlineto{\pgfqpoint{1.424442in}{1.580491in}}%
\pgfpathlineto{\pgfqpoint{1.449023in}{1.572326in}}%
\pgfpathlineto{\pgfqpoint{1.473071in}{1.563877in}}%
\pgfpathlineto{\pgfqpoint{1.496589in}{1.555143in}}%
\pgfpathlineto{\pgfqpoint{1.519582in}{1.546118in}}%
\pgfpathlineto{\pgfqpoint{1.542053in}{1.536802in}}%
\pgfpathlineto{\pgfqpoint{1.564006in}{1.527186in}}%
\pgfpathlineto{\pgfqpoint{1.585446in}{1.517265in}}%
\pgfpathlineto{\pgfqpoint{1.606377in}{1.507033in}}%
\pgfpathlineto{\pgfqpoint{1.626802in}{1.496480in}}%
\pgfpathlineto{\pgfqpoint{1.646726in}{1.485599in}}%
\pgfpathlineto{\pgfqpoint{1.666156in}{1.474379in}}%
\pgfpathlineto{\pgfqpoint{1.685096in}{1.462812in}}%
\pgfpathlineto{\pgfqpoint{1.703553in}{1.450885in}}%
\pgfpathlineto{\pgfqpoint{1.721531in}{1.438586in}}%
\pgfpathlineto{\pgfqpoint{1.739037in}{1.425903in}}%
\pgfpathlineto{\pgfqpoint{1.756077in}{1.412822in}}%
\pgfpathlineto{\pgfqpoint{1.772657in}{1.399329in}}%
\pgfpathlineto{\pgfqpoint{1.788784in}{1.385406in}}%
\pgfpathlineto{\pgfqpoint{1.804462in}{1.371037in}}%
\pgfpathlineto{\pgfqpoint{1.819697in}{1.356202in}}%
\pgfpathlineto{\pgfqpoint{1.834494in}{1.340883in}}%
\pgfpathlineto{\pgfqpoint{1.848858in}{1.325055in}}%
\pgfpathlineto{\pgfqpoint{1.862793in}{1.308697in}}%
\pgfpathlineto{\pgfqpoint{1.876303in}{1.291781in}}%
\pgfpathlineto{\pgfqpoint{1.889391in}{1.274281in}}%
\pgfpathlineto{\pgfqpoint{1.902061in}{1.256165in}}%
\pgfpathlineto{\pgfqpoint{1.914315in}{1.237401in}}%
\pgfpathlineto{\pgfqpoint{1.926156in}{1.217950in}}%
\pgfpathlineto{\pgfqpoint{1.937586in}{1.197775in}}%
\pgfpathlineto{\pgfqpoint{1.948607in}{1.176829in}}%
\pgfpathlineto{\pgfqpoint{1.959221in}{1.155064in}}%
\pgfpathlineto{\pgfqpoint{1.969429in}{1.132426in}}%
\pgfpathlineto{\pgfqpoint{1.979233in}{1.108854in}}%
\pgfpathlineto{\pgfqpoint{1.988636in}{1.084280in}}%
\pgfpathlineto{\pgfqpoint{1.997640in}{1.058628in}}%
\pgfpathlineto{\pgfqpoint{2.006245in}{1.031812in}}%
\pgfpathlineto{\pgfqpoint{2.014455in}{1.003735in}}%
\pgfusepath{stroke}%
\end{pgfscope}%
\begin{pgfscope}%
\pgfpathrectangle{\pgfqpoint{0.675193in}{0.526079in}}{\pgfqpoint{4.650000in}{3.020000in}}%
\pgfusepath{clip}%
\pgfsetrectcap%
\pgfsetroundjoin%
\pgfsetlinewidth{1.505625pt}%
\definecolor{currentstroke}{rgb}{0.000000,0.750000,0.750000}%
\pgfsetstrokecolor{currentstroke}%
\pgfsetstrokeopacity{0.750000}%
\pgfsetdash{}{0pt}%
\pgfpathmoveto{\pgfqpoint{0.954489in}{2.097449in}}%
\pgfpathlineto{\pgfqpoint{0.988306in}{2.012521in}}%
\pgfpathlineto{\pgfqpoint{1.021505in}{1.937287in}}%
\pgfpathlineto{\pgfqpoint{1.054091in}{1.869173in}}%
\pgfpathlineto{\pgfqpoint{1.086070in}{1.807193in}}%
\pgfpathlineto{\pgfqpoint{1.117448in}{1.750533in}}%
\pgfpathlineto{\pgfqpoint{1.148230in}{1.698523in}}%
\pgfpathlineto{\pgfqpoint{1.178420in}{1.650608in}}%
\pgfpathlineto{\pgfqpoint{1.208025in}{1.606322in}}%
\pgfpathlineto{\pgfqpoint{1.237049in}{1.565270in}}%
\pgfpathlineto{\pgfqpoint{1.265499in}{1.527114in}}%
\pgfpathlineto{\pgfqpoint{1.293379in}{1.491564in}}%
\pgfpathlineto{\pgfqpoint{1.320695in}{1.458371in}}%
\pgfpathlineto{\pgfqpoint{1.347455in}{1.427319in}}%
\pgfpathlineto{\pgfqpoint{1.373663in}{1.398215in}}%
\pgfpathlineto{\pgfqpoint{1.399324in}{1.370894in}}%
\pgfpathlineto{\pgfqpoint{1.424442in}{1.345207in}}%
\pgfpathlineto{\pgfqpoint{1.449023in}{1.321022in}}%
\pgfpathlineto{\pgfqpoint{1.473071in}{1.298224in}}%
\pgfpathlineto{\pgfqpoint{1.496589in}{1.276708in}}%
\pgfpathlineto{\pgfqpoint{1.519582in}{1.256379in}}%
\pgfpathlineto{\pgfqpoint{1.542053in}{1.237155in}}%
\pgfpathlineto{\pgfqpoint{1.564006in}{1.218958in}}%
\pgfpathlineto{\pgfqpoint{1.585446in}{1.201719in}}%
\pgfpathlineto{\pgfqpoint{1.606377in}{1.185376in}}%
\pgfpathlineto{\pgfqpoint{1.626802in}{1.169871in}}%
\pgfpathlineto{\pgfqpoint{1.646726in}{1.155153in}}%
\pgfpathlineto{\pgfqpoint{1.666156in}{1.141173in}}%
\pgfpathlineto{\pgfqpoint{1.685096in}{1.127889in}}%
\pgfpathlineto{\pgfqpoint{1.703553in}{1.115259in}}%
\pgfpathlineto{\pgfqpoint{1.721531in}{1.103247in}}%
\pgfpathlineto{\pgfqpoint{1.739037in}{1.091819in}}%
\pgfpathlineto{\pgfqpoint{1.756077in}{1.080943in}}%
\pgfpathlineto{\pgfqpoint{1.772657in}{1.070592in}}%
\pgfpathlineto{\pgfqpoint{1.788784in}{1.060737in}}%
\pgfpathlineto{\pgfqpoint{1.804462in}{1.051354in}}%
\pgfpathlineto{\pgfqpoint{1.819697in}{1.042420in}}%
\pgfpathlineto{\pgfqpoint{1.834494in}{1.033913in}}%
\pgfpathlineto{\pgfqpoint{1.848858in}{1.025813in}}%
\pgfpathlineto{\pgfqpoint{1.862793in}{1.018103in}}%
\pgfpathlineto{\pgfqpoint{1.876303in}{1.010763in}}%
\pgfpathlineto{\pgfqpoint{1.889391in}{1.003779in}}%
\pgfpathlineto{\pgfqpoint{1.902061in}{0.997135in}}%
\pgfpathlineto{\pgfqpoint{1.914315in}{0.990817in}}%
\pgfpathlineto{\pgfqpoint{1.926156in}{0.984812in}}%
\pgfpathlineto{\pgfqpoint{1.937586in}{0.979108in}}%
\pgfpathlineto{\pgfqpoint{1.948607in}{0.973693in}}%
\pgfpathlineto{\pgfqpoint{1.959221in}{0.968557in}}%
\pgfpathlineto{\pgfqpoint{1.969429in}{0.963688in}}%
\pgfpathlineto{\pgfqpoint{1.979233in}{0.959079in}}%
\pgfpathlineto{\pgfqpoint{1.988636in}{0.954720in}}%
\pgfpathlineto{\pgfqpoint{1.997640in}{0.950604in}}%
\pgfpathlineto{\pgfqpoint{2.006245in}{0.946721in}}%
\pgfpathlineto{\pgfqpoint{2.014455in}{0.943065in}}%
\pgfusepath{stroke}%
\end{pgfscope}%
\begin{pgfscope}%
\pgfpathrectangle{\pgfqpoint{0.675193in}{0.526079in}}{\pgfqpoint{4.650000in}{3.020000in}}%
\pgfusepath{clip}%
\pgfsetrectcap%
\pgfsetroundjoin%
\pgfsetlinewidth{1.505625pt}%
\definecolor{currentstroke}{rgb}{1.000000,0.000000,0.000000}%
\pgfsetstrokecolor{currentstroke}%
\pgfsetstrokeopacity{0.750000}%
\pgfsetdash{}{0pt}%
\pgfpathmoveto{\pgfqpoint{1.132009in}{3.358593in}}%
\pgfpathlineto{\pgfqpoint{1.176445in}{3.363332in}}%
\pgfpathlineto{\pgfqpoint{1.219682in}{3.367933in}}%
\pgfpathlineto{\pgfqpoint{1.261735in}{3.371645in}}%
\pgfpathlineto{\pgfqpoint{1.302618in}{3.374699in}}%
\pgfpathlineto{\pgfqpoint{1.342347in}{3.377254in}}%
\pgfpathlineto{\pgfqpoint{1.380936in}{3.379421in}}%
\pgfpathlineto{\pgfqpoint{1.418401in}{3.381284in}}%
\pgfpathlineto{\pgfqpoint{1.454756in}{3.382905in}}%
\pgfpathlineto{\pgfqpoint{1.490016in}{3.384329in}}%
\pgfpathlineto{\pgfqpoint{1.524195in}{3.385593in}}%
\pgfpathlineto{\pgfqpoint{1.557309in}{3.386724in}}%
\pgfpathlineto{\pgfqpoint{1.589373in}{3.387744in}}%
\pgfpathlineto{\pgfqpoint{1.620401in}{3.388671in}}%
\pgfpathlineto{\pgfqpoint{1.650410in}{3.389519in}}%
\pgfpathlineto{\pgfqpoint{1.679414in}{3.390298in}}%
\pgfpathlineto{\pgfqpoint{1.707427in}{3.391020in}}%
\pgfpathlineto{\pgfqpoint{1.734466in}{3.391690in}}%
\pgfpathlineto{\pgfqpoint{1.760544in}{3.392316in}}%
\pgfpathlineto{\pgfqpoint{1.785678in}{3.392903in}}%
\pgfpathlineto{\pgfqpoint{1.809881in}{3.393456in}}%
\pgfpathlineto{\pgfqpoint{1.833167in}{3.393978in}}%
\pgfpathlineto{\pgfqpoint{1.855549in}{3.394472in}}%
\pgfpathlineto{\pgfqpoint{1.877041in}{3.394942in}}%
\pgfpathlineto{\pgfqpoint{1.897656in}{3.395388in}}%
\pgfpathlineto{\pgfqpoint{1.917406in}{3.395815in}}%
\pgfpathlineto{\pgfqpoint{1.936302in}{3.396222in}}%
\pgfpathlineto{\pgfqpoint{1.954357in}{3.396613in}}%
\pgfpathlineto{\pgfqpoint{1.971581in}{3.396987in}}%
\pgfpathlineto{\pgfqpoint{1.987984in}{3.397346in}}%
\pgfusepath{stroke}%
\end{pgfscope}%
\begin{pgfscope}%
\pgfpathrectangle{\pgfqpoint{0.675193in}{0.526079in}}{\pgfqpoint{4.650000in}{3.020000in}}%
\pgfusepath{clip}%
\pgfsetrectcap%
\pgfsetroundjoin%
\pgfsetlinewidth{1.505625pt}%
\definecolor{currentstroke}{rgb}{0.000000,0.000000,1.000000}%
\pgfsetstrokecolor{currentstroke}%
\pgfsetstrokeopacity{0.750000}%
\pgfsetdash{}{0pt}%
\pgfpathmoveto{\pgfqpoint{1.132009in}{1.144534in}}%
\pgfpathlineto{\pgfqpoint{1.176445in}{1.137841in}}%
\pgfpathlineto{\pgfqpoint{1.219682in}{1.130848in}}%
\pgfpathlineto{\pgfqpoint{1.261735in}{1.122747in}}%
\pgfpathlineto{\pgfqpoint{1.302618in}{1.113720in}}%
\pgfpathlineto{\pgfqpoint{1.342347in}{1.103886in}}%
\pgfpathlineto{\pgfqpoint{1.380936in}{1.093320in}}%
\pgfpathlineto{\pgfqpoint{1.418401in}{1.082073in}}%
\pgfpathlineto{\pgfqpoint{1.454756in}{1.070177in}}%
\pgfpathlineto{\pgfqpoint{1.490016in}{1.057648in}}%
\pgfpathlineto{\pgfqpoint{1.524195in}{1.044494in}}%
\pgfpathlineto{\pgfqpoint{1.557309in}{1.030714in}}%
\pgfpathlineto{\pgfqpoint{1.589373in}{1.016301in}}%
\pgfpathlineto{\pgfqpoint{1.620401in}{1.001245in}}%
\pgfpathlineto{\pgfqpoint{1.650410in}{0.985529in}}%
\pgfpathlineto{\pgfqpoint{1.679414in}{0.969133in}}%
\pgfpathlineto{\pgfqpoint{1.707427in}{0.952034in}}%
\pgfpathlineto{\pgfqpoint{1.734466in}{0.934204in}}%
\pgfpathlineto{\pgfqpoint{1.760544in}{0.915613in}}%
\pgfpathlineto{\pgfqpoint{1.785678in}{0.896225in}}%
\pgfpathlineto{\pgfqpoint{1.809881in}{0.876001in}}%
\pgfpathlineto{\pgfqpoint{1.833167in}{0.854899in}}%
\pgfpathlineto{\pgfqpoint{1.855549in}{0.832867in}}%
\pgfpathlineto{\pgfqpoint{1.877041in}{0.809853in}}%
\pgfpathlineto{\pgfqpoint{1.897656in}{0.785796in}}%
\pgfpathlineto{\pgfqpoint{1.917406in}{0.760627in}}%
\pgfpathlineto{\pgfqpoint{1.936302in}{0.734269in}}%
\pgfpathlineto{\pgfqpoint{1.954357in}{0.706637in}}%
\pgfpathlineto{\pgfqpoint{1.971581in}{0.677631in}}%
\pgfpathlineto{\pgfqpoint{1.987984in}{0.647140in}}%
\pgfusepath{stroke}%
\end{pgfscope}%
\begin{pgfscope}%
\pgfpathrectangle{\pgfqpoint{0.675193in}{0.526079in}}{\pgfqpoint{4.650000in}{3.020000in}}%
\pgfusepath{clip}%
\pgfsetrectcap%
\pgfsetroundjoin%
\pgfsetlinewidth{1.505625pt}%
\definecolor{currentstroke}{rgb}{0.000000,0.750000,0.750000}%
\pgfsetstrokecolor{currentstroke}%
\pgfsetstrokeopacity{0.750000}%
\pgfsetdash{}{0pt}%
\pgfpathmoveto{\pgfqpoint{1.132009in}{2.021584in}}%
\pgfpathlineto{\pgfqpoint{1.176445in}{1.950266in}}%
\pgfpathlineto{\pgfqpoint{1.219682in}{1.887398in}}%
\pgfpathlineto{\pgfqpoint{1.261735in}{1.830806in}}%
\pgfpathlineto{\pgfqpoint{1.302618in}{1.779619in}}%
\pgfpathlineto{\pgfqpoint{1.342347in}{1.733121in}}%
\pgfpathlineto{\pgfqpoint{1.380936in}{1.690722in}}%
\pgfpathlineto{\pgfqpoint{1.418401in}{1.651930in}}%
\pgfpathlineto{\pgfqpoint{1.454756in}{1.616332in}}%
\pgfpathlineto{\pgfqpoint{1.490016in}{1.583577in}}%
\pgfpathlineto{\pgfqpoint{1.524195in}{1.553366in}}%
\pgfpathlineto{\pgfqpoint{1.557309in}{1.525441in}}%
\pgfpathlineto{\pgfqpoint{1.589373in}{1.499579in}}%
\pgfpathlineto{\pgfqpoint{1.620401in}{1.475589in}}%
\pgfpathlineto{\pgfqpoint{1.650410in}{1.453300in}}%
\pgfpathlineto{\pgfqpoint{1.679414in}{1.432563in}}%
\pgfpathlineto{\pgfqpoint{1.707427in}{1.413248in}}%
\pgfpathlineto{\pgfqpoint{1.734466in}{1.395238in}}%
\pgfpathlineto{\pgfqpoint{1.760544in}{1.378430in}}%
\pgfpathlineto{\pgfqpoint{1.785678in}{1.362733in}}%
\pgfpathlineto{\pgfqpoint{1.809881in}{1.348065in}}%
\pgfpathlineto{\pgfqpoint{1.833167in}{1.334352in}}%
\pgfpathlineto{\pgfqpoint{1.855549in}{1.321526in}}%
\pgfpathlineto{\pgfqpoint{1.877041in}{1.309529in}}%
\pgfpathlineto{\pgfqpoint{1.897656in}{1.298307in}}%
\pgfpathlineto{\pgfqpoint{1.917406in}{1.287811in}}%
\pgfpathlineto{\pgfqpoint{1.936302in}{1.277997in}}%
\pgfpathlineto{\pgfqpoint{1.954357in}{1.268826in}}%
\pgfpathlineto{\pgfqpoint{1.971581in}{1.260260in}}%
\pgfpathlineto{\pgfqpoint{1.987984in}{1.252266in}}%
\pgfusepath{stroke}%
\end{pgfscope}%
\begin{pgfscope}%
\pgfpathrectangle{\pgfqpoint{0.675193in}{0.526079in}}{\pgfqpoint{4.650000in}{3.020000in}}%
\pgfusepath{clip}%
\pgfsetrectcap%
\pgfsetroundjoin%
\pgfsetlinewidth{1.505625pt}%
\definecolor{currentstroke}{rgb}{1.000000,0.000000,0.000000}%
\pgfsetstrokecolor{currentstroke}%
\pgfsetstrokeopacity{0.750000}%
\pgfsetdash{}{0pt}%
\pgfpathmoveto{\pgfqpoint{1.223532in}{3.281749in}}%
\pgfpathlineto{\pgfqpoint{1.267962in}{3.291974in}}%
\pgfpathlineto{\pgfqpoint{1.311046in}{3.301864in}}%
\pgfpathlineto{\pgfqpoint{1.352819in}{3.310144in}}%
\pgfpathlineto{\pgfqpoint{1.393311in}{3.317161in}}%
\pgfpathlineto{\pgfqpoint{1.432557in}{3.323168in}}%
\pgfpathlineto{\pgfqpoint{1.470589in}{3.328361in}}%
\pgfpathlineto{\pgfqpoint{1.507439in}{3.332887in}}%
\pgfpathlineto{\pgfqpoint{1.543140in}{3.336861in}}%
\pgfpathlineto{\pgfqpoint{1.577725in}{3.340376in}}%
\pgfpathlineto{\pgfqpoint{1.611227in}{3.343504in}}%
\pgfpathlineto{\pgfqpoint{1.643678in}{3.346304in}}%
\pgfpathlineto{\pgfqpoint{1.675112in}{3.348825in}}%
\pgfpathlineto{\pgfqpoint{1.705577in}{3.351104in}}%
\pgfpathlineto{\pgfqpoint{1.735101in}{3.353174in}}%
\pgfpathlineto{\pgfqpoint{1.763713in}{3.355063in}}%
\pgfpathlineto{\pgfqpoint{1.791444in}{3.356793in}}%
\pgfpathlineto{\pgfqpoint{1.818323in}{3.358383in}}%
\pgfpathlineto{\pgfqpoint{1.844379in}{3.359849in}}%
\pgfpathlineto{\pgfqpoint{1.869636in}{3.361206in}}%
\pgfpathlineto{\pgfqpoint{1.894120in}{3.362465in}}%
\pgfpathlineto{\pgfqpoint{1.917850in}{3.363637in}}%
\pgfpathlineto{\pgfqpoint{1.940850in}{3.364730in}}%
\pgfpathlineto{\pgfqpoint{1.963137in}{3.365751in}}%
\pgfpathlineto{\pgfqpoint{1.984729in}{3.366708in}}%
\pgfpathlineto{\pgfqpoint{2.005641in}{3.367607in}}%
\pgfpathlineto{\pgfqpoint{2.025885in}{3.368452in}}%
\pgfpathlineto{\pgfqpoint{2.045473in}{3.369249in}}%
\pgfpathlineto{\pgfqpoint{2.064418in}{3.370001in}}%
\pgfpathlineto{\pgfqpoint{2.082730in}{3.370711in}}%
\pgfpathlineto{\pgfqpoint{2.100420in}{3.371384in}}%
\pgfpathlineto{\pgfqpoint{2.117499in}{3.372021in}}%
\pgfpathlineto{\pgfqpoint{2.133975in}{3.372625in}}%
\pgfpathlineto{\pgfqpoint{2.149860in}{3.373199in}}%
\pgfpathlineto{\pgfqpoint{2.165164in}{3.373744in}}%
\pgfpathlineto{\pgfqpoint{2.179897in}{3.374263in}}%
\pgfpathlineto{\pgfqpoint{2.194068in}{3.374755in}}%
\pgfpathlineto{\pgfqpoint{2.207685in}{3.375224in}}%
\pgfpathlineto{\pgfqpoint{2.220755in}{3.375670in}}%
\pgfusepath{stroke}%
\end{pgfscope}%
\begin{pgfscope}%
\pgfpathrectangle{\pgfqpoint{0.675193in}{0.526079in}}{\pgfqpoint{4.650000in}{3.020000in}}%
\pgfusepath{clip}%
\pgfsetrectcap%
\pgfsetroundjoin%
\pgfsetlinewidth{1.505625pt}%
\definecolor{currentstroke}{rgb}{0.000000,0.000000,1.000000}%
\pgfsetstrokecolor{currentstroke}%
\pgfsetstrokeopacity{0.750000}%
\pgfsetdash{}{0pt}%
\pgfpathmoveto{\pgfqpoint{1.223532in}{1.038478in}}%
\pgfpathlineto{\pgfqpoint{1.267962in}{1.038131in}}%
\pgfpathlineto{\pgfqpoint{1.311046in}{1.037547in}}%
\pgfpathlineto{\pgfqpoint{1.352819in}{1.035369in}}%
\pgfpathlineto{\pgfqpoint{1.393311in}{1.031885in}}%
\pgfpathlineto{\pgfqpoint{1.432557in}{1.027297in}}%
\pgfpathlineto{\pgfqpoint{1.470589in}{1.021756in}}%
\pgfpathlineto{\pgfqpoint{1.507439in}{1.015374in}}%
\pgfpathlineto{\pgfqpoint{1.543140in}{1.008235in}}%
\pgfpathlineto{\pgfqpoint{1.577725in}{1.000402in}}%
\pgfpathlineto{\pgfqpoint{1.611227in}{0.991924in}}%
\pgfpathlineto{\pgfqpoint{1.643678in}{0.982834in}}%
\pgfpathlineto{\pgfqpoint{1.675112in}{0.973157in}}%
\pgfpathlineto{\pgfqpoint{1.705577in}{0.962912in}}%
\pgfpathlineto{\pgfqpoint{1.735101in}{0.952111in}}%
\pgfpathlineto{\pgfqpoint{1.763713in}{0.940759in}}%
\pgfpathlineto{\pgfqpoint{1.791444in}{0.928859in}}%
\pgfpathlineto{\pgfqpoint{1.818323in}{0.916411in}}%
\pgfpathlineto{\pgfqpoint{1.844379in}{0.903408in}}%
\pgfpathlineto{\pgfqpoint{1.869636in}{0.889845in}}%
\pgfpathlineto{\pgfqpoint{1.894120in}{0.875710in}}%
\pgfpathlineto{\pgfqpoint{1.917850in}{0.860991in}}%
\pgfpathlineto{\pgfqpoint{1.940850in}{0.845673in}}%
\pgfpathlineto{\pgfqpoint{1.963137in}{0.829736in}}%
\pgfpathlineto{\pgfqpoint{1.984729in}{0.813161in}}%
\pgfpathlineto{\pgfqpoint{2.005641in}{0.795924in}}%
\pgfpathlineto{\pgfqpoint{2.025885in}{0.778000in}}%
\pgfpathlineto{\pgfqpoint{2.045473in}{0.759358in}}%
\pgfpathlineto{\pgfqpoint{2.064418in}{0.739967in}}%
\pgfpathlineto{\pgfqpoint{2.082730in}{0.719790in}}%
\pgfpathlineto{\pgfqpoint{2.100420in}{0.698786in}}%
\pgfpathlineto{\pgfqpoint{2.117499in}{0.676910in}}%
\pgfpathlineto{\pgfqpoint{2.133975in}{0.654113in}}%
\pgfpathlineto{\pgfqpoint{2.149860in}{0.630338in}}%
\pgfpathlineto{\pgfqpoint{2.165164in}{0.605522in}}%
\pgfpathlineto{\pgfqpoint{2.179897in}{0.579596in}}%
\pgfpathlineto{\pgfqpoint{2.194068in}{0.552479in}}%
\pgfpathlineto{\pgfqpoint{2.207685in}{0.524082in}}%
\pgfpathlineto{\pgfqpoint{2.212904in}{0.512191in}}%
\pgfusepath{stroke}%
\end{pgfscope}%
\begin{pgfscope}%
\pgfpathrectangle{\pgfqpoint{0.675193in}{0.526079in}}{\pgfqpoint{4.650000in}{3.020000in}}%
\pgfusepath{clip}%
\pgfsetrectcap%
\pgfsetroundjoin%
\pgfsetlinewidth{1.505625pt}%
\definecolor{currentstroke}{rgb}{0.000000,0.750000,0.750000}%
\pgfsetstrokecolor{currentstroke}%
\pgfsetstrokeopacity{0.750000}%
\pgfsetdash{}{0pt}%
\pgfpathmoveto{\pgfqpoint{1.223532in}{2.537324in}}%
\pgfpathlineto{\pgfqpoint{1.267962in}{2.486556in}}%
\pgfpathlineto{\pgfqpoint{1.311046in}{2.441538in}}%
\pgfpathlineto{\pgfqpoint{1.352819in}{2.400101in}}%
\pgfpathlineto{\pgfqpoint{1.393311in}{2.361877in}}%
\pgfpathlineto{\pgfqpoint{1.432557in}{2.326538in}}%
\pgfpathlineto{\pgfqpoint{1.470589in}{2.293799in}}%
\pgfpathlineto{\pgfqpoint{1.507439in}{2.263409in}}%
\pgfpathlineto{\pgfqpoint{1.543140in}{2.235148in}}%
\pgfpathlineto{\pgfqpoint{1.577725in}{2.208822in}}%
\pgfpathlineto{\pgfqpoint{1.611227in}{2.184259in}}%
\pgfpathlineto{\pgfqpoint{1.643678in}{2.161308in}}%
\pgfpathlineto{\pgfqpoint{1.675112in}{2.139834in}}%
\pgfpathlineto{\pgfqpoint{1.705577in}{2.119715in}}%
\pgfpathlineto{\pgfqpoint{1.735101in}{2.100844in}}%
\pgfpathlineto{\pgfqpoint{1.763713in}{2.083125in}}%
\pgfpathlineto{\pgfqpoint{1.791444in}{2.066471in}}%
\pgfpathlineto{\pgfqpoint{1.818323in}{2.050804in}}%
\pgfpathlineto{\pgfqpoint{1.844379in}{2.036053in}}%
\pgfpathlineto{\pgfqpoint{1.869636in}{2.022154in}}%
\pgfpathlineto{\pgfqpoint{1.894120in}{2.009050in}}%
\pgfpathlineto{\pgfqpoint{1.917850in}{1.996687in}}%
\pgfpathlineto{\pgfqpoint{1.940850in}{1.985019in}}%
\pgfpathlineto{\pgfqpoint{1.963137in}{1.974000in}}%
\pgfpathlineto{\pgfqpoint{1.984729in}{1.963591in}}%
\pgfpathlineto{\pgfqpoint{2.005641in}{1.953756in}}%
\pgfpathlineto{\pgfqpoint{2.025885in}{1.944462in}}%
\pgfpathlineto{\pgfqpoint{2.045473in}{1.935676in}}%
\pgfpathlineto{\pgfqpoint{2.064418in}{1.927372in}}%
\pgfpathlineto{\pgfqpoint{2.082730in}{1.919523in}}%
\pgfpathlineto{\pgfqpoint{2.100420in}{1.912105in}}%
\pgfpathlineto{\pgfqpoint{2.117499in}{1.905097in}}%
\pgfpathlineto{\pgfqpoint{2.133975in}{1.898477in}}%
\pgfpathlineto{\pgfqpoint{2.149860in}{1.892228in}}%
\pgfpathlineto{\pgfqpoint{2.165164in}{1.886331in}}%
\pgfpathlineto{\pgfqpoint{2.179897in}{1.880772in}}%
\pgfpathlineto{\pgfqpoint{2.194068in}{1.875533in}}%
\pgfpathlineto{\pgfqpoint{2.207685in}{1.870603in}}%
\pgfpathlineto{\pgfqpoint{2.220755in}{1.865969in}}%
\pgfusepath{stroke}%
\end{pgfscope}%
\begin{pgfscope}%
\pgfpathrectangle{\pgfqpoint{0.675193in}{0.526079in}}{\pgfqpoint{4.650000in}{3.020000in}}%
\pgfusepath{clip}%
\pgfsetrectcap%
\pgfsetroundjoin%
\pgfsetlinewidth{1.505625pt}%
\definecolor{currentstroke}{rgb}{1.000000,0.000000,0.000000}%
\pgfsetstrokecolor{currentstroke}%
\pgfsetstrokeopacity{0.750000}%
\pgfsetdash{}{0pt}%
\pgfpathmoveto{\pgfqpoint{1.067481in}{3.299343in}}%
\pgfpathlineto{\pgfqpoint{1.110135in}{3.309660in}}%
\pgfpathlineto{\pgfqpoint{1.152001in}{3.319374in}}%
\pgfpathlineto{\pgfqpoint{1.193092in}{3.327203in}}%
\pgfpathlineto{\pgfqpoint{1.233421in}{3.333621in}}%
\pgfpathlineto{\pgfqpoint{1.273001in}{3.338956in}}%
\pgfpathlineto{\pgfqpoint{1.311843in}{3.343447in}}%
\pgfpathlineto{\pgfqpoint{1.349961in}{3.347272in}}%
\pgfpathlineto{\pgfqpoint{1.387368in}{3.350562in}}%
\pgfpathlineto{\pgfqpoint{1.424075in}{3.353418in}}%
\pgfpathlineto{\pgfqpoint{1.460095in}{3.355917in}}%
\pgfpathlineto{\pgfqpoint{1.495442in}{3.358121in}}%
\pgfpathlineto{\pgfqpoint{1.530127in}{3.360078in}}%
\pgfpathlineto{\pgfqpoint{1.564159in}{3.361828in}}%
\pgfpathlineto{\pgfqpoint{1.597553in}{3.363402in}}%
\pgfpathlineto{\pgfqpoint{1.630322in}{3.364825in}}%
\pgfpathlineto{\pgfqpoint{1.662479in}{3.366119in}}%
\pgfpathlineto{\pgfqpoint{1.694039in}{3.367300in}}%
\pgfpathlineto{\pgfqpoint{1.725017in}{3.368385in}}%
\pgfpathlineto{\pgfqpoint{1.755427in}{3.369386in}}%
\pgfpathlineto{\pgfqpoint{1.785284in}{3.370313in}}%
\pgfpathlineto{\pgfqpoint{1.814604in}{3.371174in}}%
\pgfpathlineto{\pgfqpoint{1.843400in}{3.371978in}}%
\pgfpathlineto{\pgfqpoint{1.871689in}{3.372732in}}%
\pgfpathlineto{\pgfqpoint{1.899483in}{3.373440in}}%
\pgfpathlineto{\pgfqpoint{1.926797in}{3.374108in}}%
\pgfpathlineto{\pgfqpoint{1.953643in}{3.374740in}}%
\pgfpathlineto{\pgfqpoint{1.980032in}{3.375339in}}%
\pgfpathlineto{\pgfqpoint{2.005975in}{3.375910in}}%
\pgfpathlineto{\pgfqpoint{2.031482in}{3.376456in}}%
\pgfpathlineto{\pgfqpoint{2.056561in}{3.376977in}}%
\pgfpathlineto{\pgfqpoint{2.081220in}{3.377478in}}%
\pgfpathlineto{\pgfqpoint{2.105467in}{3.377960in}}%
\pgfpathlineto{\pgfqpoint{2.129309in}{3.378424in}}%
\pgfpathlineto{\pgfqpoint{2.152752in}{3.378872in}}%
\pgfpathlineto{\pgfqpoint{2.175801in}{3.379307in}}%
\pgfpathlineto{\pgfqpoint{2.198463in}{3.379728in}}%
\pgfpathlineto{\pgfqpoint{2.220743in}{3.380137in}}%
\pgfpathlineto{\pgfqpoint{2.242646in}{3.380536in}}%
\pgfpathlineto{\pgfqpoint{2.264177in}{3.380925in}}%
\pgfpathlineto{\pgfqpoint{2.285342in}{3.381305in}}%
\pgfpathlineto{\pgfqpoint{2.306146in}{3.381676in}}%
\pgfpathlineto{\pgfqpoint{2.326594in}{3.382040in}}%
\pgfpathlineto{\pgfqpoint{2.346691in}{3.382398in}}%
\pgfpathlineto{\pgfqpoint{2.366442in}{3.382749in}}%
\pgfpathlineto{\pgfqpoint{2.385853in}{3.383092in}}%
\pgfpathlineto{\pgfqpoint{2.404929in}{3.383430in}}%
\pgfpathlineto{\pgfqpoint{2.423673in}{3.383764in}}%
\pgfpathlineto{\pgfqpoint{2.442092in}{3.384093in}}%
\pgfpathlineto{\pgfqpoint{2.460188in}{3.384418in}}%
\pgfpathlineto{\pgfqpoint{2.477967in}{3.384738in}}%
\pgfpathlineto{\pgfqpoint{2.495432in}{3.385055in}}%
\pgfpathlineto{\pgfqpoint{2.512587in}{3.385369in}}%
\pgfpathlineto{\pgfqpoint{2.529435in}{3.385679in}}%
\pgfpathlineto{\pgfqpoint{2.545981in}{3.385986in}}%
\pgfpathlineto{\pgfqpoint{2.562226in}{3.386290in}}%
\pgfpathlineto{\pgfqpoint{2.578173in}{3.386591in}}%
\pgfpathlineto{\pgfqpoint{2.593827in}{3.386890in}}%
\pgfpathlineto{\pgfqpoint{2.609189in}{3.387186in}}%
\pgfpathlineto{\pgfqpoint{2.624262in}{3.387480in}}%
\pgfpathlineto{\pgfqpoint{2.639048in}{3.387772in}}%
\pgfpathlineto{\pgfqpoint{2.653551in}{3.388062in}}%
\pgfpathlineto{\pgfqpoint{2.667773in}{3.388350in}}%
\pgfpathlineto{\pgfqpoint{2.681715in}{3.388636in}}%
\pgfpathlineto{\pgfqpoint{2.695381in}{3.388919in}}%
\pgfpathlineto{\pgfqpoint{2.708772in}{3.389201in}}%
\pgfpathlineto{\pgfqpoint{2.721892in}{3.389481in}}%
\pgfpathlineto{\pgfqpoint{2.734743in}{3.389760in}}%
\pgfpathlineto{\pgfqpoint{2.747326in}{3.390036in}}%
\pgfpathlineto{\pgfqpoint{2.759646in}{3.390311in}}%
\pgfpathlineto{\pgfqpoint{2.771703in}{3.390584in}}%
\pgfpathlineto{\pgfqpoint{2.783500in}{3.390856in}}%
\pgfpathlineto{\pgfqpoint{2.795040in}{3.391126in}}%
\pgfpathlineto{\pgfqpoint{2.806324in}{3.391394in}}%
\pgfpathlineto{\pgfqpoint{2.817355in}{3.391661in}}%
\pgfpathlineto{\pgfqpoint{2.828134in}{3.391926in}}%
\pgfpathlineto{\pgfqpoint{2.838665in}{3.392189in}}%
\pgfpathlineto{\pgfqpoint{2.848947in}{3.392451in}}%
\pgfpathlineto{\pgfqpoint{2.858984in}{3.392711in}}%
\pgfpathlineto{\pgfqpoint{2.868775in}{3.392970in}}%
\pgfpathlineto{\pgfqpoint{2.878324in}{3.393226in}}%
\pgfpathlineto{\pgfqpoint{2.887631in}{3.393481in}}%
\pgfpathlineto{\pgfqpoint{2.896696in}{3.393734in}}%
\pgfpathlineto{\pgfqpoint{2.905523in}{3.393986in}}%
\pgfpathlineto{\pgfqpoint{2.914111in}{3.394235in}}%
\pgfpathlineto{\pgfqpoint{2.922462in}{3.394483in}}%
\pgfpathlineto{\pgfqpoint{2.930576in}{3.394728in}}%
\pgfpathlineto{\pgfqpoint{2.938455in}{3.394972in}}%
\pgfpathlineto{\pgfqpoint{2.946100in}{3.395213in}}%
\pgfpathlineto{\pgfqpoint{2.953512in}{3.395453in}}%
\pgfpathlineto{\pgfqpoint{2.960692in}{3.395691in}}%
\pgfpathlineto{\pgfqpoint{2.967640in}{3.395926in}}%
\pgfpathlineto{\pgfqpoint{2.974359in}{3.396159in}}%
\pgfpathlineto{\pgfqpoint{2.980849in}{3.396390in}}%
\pgfpathlineto{\pgfqpoint{2.987110in}{3.396619in}}%
\pgfpathlineto{\pgfqpoint{2.993145in}{3.396845in}}%
\pgfpathlineto{\pgfqpoint{2.998953in}{3.397069in}}%
\pgfpathlineto{\pgfqpoint{3.004537in}{3.397290in}}%
\pgfpathlineto{\pgfqpoint{3.009897in}{3.397508in}}%
\pgfpathlineto{\pgfqpoint{3.015034in}{3.397723in}}%
\pgfpathlineto{\pgfqpoint{3.019948in}{3.397936in}}%
\pgfpathlineto{\pgfqpoint{3.024641in}{3.398146in}}%
\pgfusepath{stroke}%
\end{pgfscope}%
\begin{pgfscope}%
\pgfpathrectangle{\pgfqpoint{0.675193in}{0.526079in}}{\pgfqpoint{4.650000in}{3.020000in}}%
\pgfusepath{clip}%
\pgfsetrectcap%
\pgfsetroundjoin%
\pgfsetlinewidth{1.505625pt}%
\definecolor{currentstroke}{rgb}{0.000000,0.000000,1.000000}%
\pgfsetstrokecolor{currentstroke}%
\pgfsetstrokeopacity{0.750000}%
\pgfsetdash{}{0pt}%
\pgfpathmoveto{\pgfqpoint{1.067481in}{1.531089in}}%
\pgfpathlineto{\pgfqpoint{1.110135in}{1.537955in}}%
\pgfpathlineto{\pgfqpoint{1.152001in}{1.544434in}}%
\pgfpathlineto{\pgfqpoint{1.193092in}{1.549183in}}%
\pgfpathlineto{\pgfqpoint{1.233421in}{1.552631in}}%
\pgfpathlineto{\pgfqpoint{1.273001in}{1.555071in}}%
\pgfpathlineto{\pgfqpoint{1.311843in}{1.556713in}}%
\pgfpathlineto{\pgfqpoint{1.349961in}{1.557713in}}%
\pgfpathlineto{\pgfqpoint{1.387368in}{1.558183in}}%
\pgfpathlineto{\pgfqpoint{1.424075in}{1.558209in}}%
\pgfpathlineto{\pgfqpoint{1.460095in}{1.557855in}}%
\pgfpathlineto{\pgfqpoint{1.495442in}{1.557170in}}%
\pgfpathlineto{\pgfqpoint{1.530127in}{1.556195in}}%
\pgfpathlineto{\pgfqpoint{1.564159in}{1.554958in}}%
\pgfpathlineto{\pgfqpoint{1.597553in}{1.553483in}}%
\pgfpathlineto{\pgfqpoint{1.630322in}{1.551790in}}%
\pgfpathlineto{\pgfqpoint{1.662479in}{1.549892in}}%
\pgfpathlineto{\pgfqpoint{1.694039in}{1.547801in}}%
\pgfpathlineto{\pgfqpoint{1.725017in}{1.545528in}}%
\pgfpathlineto{\pgfqpoint{1.755427in}{1.543081in}}%
\pgfpathlineto{\pgfqpoint{1.785284in}{1.540465in}}%
\pgfpathlineto{\pgfqpoint{1.814604in}{1.537684in}}%
\pgfpathlineto{\pgfqpoint{1.843400in}{1.534743in}}%
\pgfpathlineto{\pgfqpoint{1.871689in}{1.531645in}}%
\pgfpathlineto{\pgfqpoint{1.899483in}{1.528392in}}%
\pgfpathlineto{\pgfqpoint{1.926797in}{1.524986in}}%
\pgfpathlineto{\pgfqpoint{1.953643in}{1.521427in}}%
\pgfpathlineto{\pgfqpoint{1.980032in}{1.517717in}}%
\pgfpathlineto{\pgfqpoint{2.005975in}{1.513857in}}%
\pgfpathlineto{\pgfqpoint{2.031482in}{1.509846in}}%
\pgfpathlineto{\pgfqpoint{2.056561in}{1.505685in}}%
\pgfpathlineto{\pgfqpoint{2.081220in}{1.501373in}}%
\pgfpathlineto{\pgfqpoint{2.105467in}{1.496910in}}%
\pgfpathlineto{\pgfqpoint{2.129309in}{1.492294in}}%
\pgfpathlineto{\pgfqpoint{2.152752in}{1.487526in}}%
\pgfpathlineto{\pgfqpoint{2.175801in}{1.482603in}}%
\pgfpathlineto{\pgfqpoint{2.198463in}{1.477525in}}%
\pgfpathlineto{\pgfqpoint{2.220743in}{1.472291in}}%
\pgfpathlineto{\pgfqpoint{2.242646in}{1.466898in}}%
\pgfpathlineto{\pgfqpoint{2.264177in}{1.461346in}}%
\pgfpathlineto{\pgfqpoint{2.285342in}{1.455632in}}%
\pgfpathlineto{\pgfqpoint{2.306146in}{1.449755in}}%
\pgfpathlineto{\pgfqpoint{2.326594in}{1.443712in}}%
\pgfpathlineto{\pgfqpoint{2.346691in}{1.437502in}}%
\pgfpathlineto{\pgfqpoint{2.366442in}{1.431123in}}%
\pgfpathlineto{\pgfqpoint{2.385853in}{1.424569in}}%
\pgfpathlineto{\pgfqpoint{2.404929in}{1.417840in}}%
\pgfpathlineto{\pgfqpoint{2.423673in}{1.410934in}}%
\pgfpathlineto{\pgfqpoint{2.442092in}{1.403847in}}%
\pgfpathlineto{\pgfqpoint{2.460188in}{1.396577in}}%
\pgfpathlineto{\pgfqpoint{2.477967in}{1.389119in}}%
\pgfpathlineto{\pgfqpoint{2.495432in}{1.381471in}}%
\pgfpathlineto{\pgfqpoint{2.512587in}{1.373628in}}%
\pgfpathlineto{\pgfqpoint{2.529435in}{1.365587in}}%
\pgfpathlineto{\pgfqpoint{2.545981in}{1.357345in}}%
\pgfpathlineto{\pgfqpoint{2.562226in}{1.348895in}}%
\pgfpathlineto{\pgfqpoint{2.578173in}{1.340235in}}%
\pgfpathlineto{\pgfqpoint{2.593827in}{1.331359in}}%
\pgfpathlineto{\pgfqpoint{2.609189in}{1.322262in}}%
\pgfpathlineto{\pgfqpoint{2.624262in}{1.312940in}}%
\pgfpathlineto{\pgfqpoint{2.639048in}{1.303385in}}%
\pgfpathlineto{\pgfqpoint{2.653551in}{1.293594in}}%
\pgfpathlineto{\pgfqpoint{2.667773in}{1.283558in}}%
\pgfpathlineto{\pgfqpoint{2.681715in}{1.273271in}}%
\pgfpathlineto{\pgfqpoint{2.695381in}{1.262726in}}%
\pgfpathlineto{\pgfqpoint{2.708772in}{1.251916in}}%
\pgfpathlineto{\pgfqpoint{2.721892in}{1.240834in}}%
\pgfpathlineto{\pgfqpoint{2.734743in}{1.229470in}}%
\pgfpathlineto{\pgfqpoint{2.747326in}{1.217815in}}%
\pgfpathlineto{\pgfqpoint{2.759646in}{1.205861in}}%
\pgfpathlineto{\pgfqpoint{2.771703in}{1.193596in}}%
\pgfpathlineto{\pgfqpoint{2.783500in}{1.181010in}}%
\pgfpathlineto{\pgfqpoint{2.795040in}{1.168092in}}%
\pgfpathlineto{\pgfqpoint{2.806324in}{1.154830in}}%
\pgfpathlineto{\pgfqpoint{2.817355in}{1.141210in}}%
\pgfpathlineto{\pgfqpoint{2.828134in}{1.127218in}}%
\pgfpathlineto{\pgfqpoint{2.838665in}{1.112839in}}%
\pgfpathlineto{\pgfqpoint{2.848947in}{1.098057in}}%
\pgfpathlineto{\pgfqpoint{2.858984in}{1.082854in}}%
\pgfpathlineto{\pgfqpoint{2.868775in}{1.067213in}}%
\pgfpathlineto{\pgfqpoint{2.878324in}{1.051111in}}%
\pgfpathlineto{\pgfqpoint{2.887631in}{1.034529in}}%
\pgfpathlineto{\pgfqpoint{2.896696in}{1.017441in}}%
\pgfpathlineto{\pgfqpoint{2.905523in}{0.999823in}}%
\pgfpathlineto{\pgfqpoint{2.914111in}{0.981646in}}%
\pgfpathlineto{\pgfqpoint{2.922462in}{0.962880in}}%
\pgfpathlineto{\pgfqpoint{2.930576in}{0.943492in}}%
\pgfpathlineto{\pgfqpoint{2.938455in}{0.923444in}}%
\pgfpathlineto{\pgfqpoint{2.946100in}{0.902698in}}%
\pgfpathlineto{\pgfqpoint{2.953512in}{0.881208in}}%
\pgfpathlineto{\pgfqpoint{2.960692in}{0.858926in}}%
\pgfpathlineto{\pgfqpoint{2.967640in}{0.835797in}}%
\pgfpathlineto{\pgfqpoint{2.974359in}{0.811761in}}%
\pgfpathlineto{\pgfqpoint{2.980849in}{0.786751in}}%
\pgfpathlineto{\pgfqpoint{2.987110in}{0.760689in}}%
\pgfpathlineto{\pgfqpoint{2.993145in}{0.733491in}}%
\pgfpathlineto{\pgfqpoint{2.998953in}{0.705059in}}%
\pgfpathlineto{\pgfqpoint{3.004537in}{0.675281in}}%
\pgfpathlineto{\pgfqpoint{3.009897in}{0.644030in}}%
\pgfpathlineto{\pgfqpoint{3.015034in}{0.611158in}}%
\pgfpathlineto{\pgfqpoint{3.019948in}{0.576494in}}%
\pgfpathlineto{\pgfqpoint{3.024641in}{0.539836in}}%
\pgfusepath{stroke}%
\end{pgfscope}%
\begin{pgfscope}%
\pgfpathrectangle{\pgfqpoint{0.675193in}{0.526079in}}{\pgfqpoint{4.650000in}{3.020000in}}%
\pgfusepath{clip}%
\pgfsetrectcap%
\pgfsetroundjoin%
\pgfsetlinewidth{1.505625pt}%
\definecolor{currentstroke}{rgb}{0.000000,0.750000,0.750000}%
\pgfsetstrokecolor{currentstroke}%
\pgfsetstrokeopacity{0.750000}%
\pgfsetdash{}{0pt}%
\pgfpathmoveto{\pgfqpoint{1.067481in}{2.400885in}}%
\pgfpathlineto{\pgfqpoint{1.110135in}{2.329509in}}%
\pgfpathlineto{\pgfqpoint{1.152001in}{2.266137in}}%
\pgfpathlineto{\pgfqpoint{1.193092in}{2.208080in}}%
\pgfpathlineto{\pgfqpoint{1.233421in}{2.154721in}}%
\pgfpathlineto{\pgfqpoint{1.273001in}{2.105523in}}%
\pgfpathlineto{\pgfqpoint{1.311843in}{2.060026in}}%
\pgfpathlineto{\pgfqpoint{1.349961in}{2.017839in}}%
\pgfpathlineto{\pgfqpoint{1.387368in}{1.978622in}}%
\pgfpathlineto{\pgfqpoint{1.424075in}{1.942077in}}%
\pgfpathlineto{\pgfqpoint{1.460095in}{1.907950in}}%
\pgfpathlineto{\pgfqpoint{1.495442in}{1.876015in}}%
\pgfpathlineto{\pgfqpoint{1.530127in}{1.846076in}}%
\pgfpathlineto{\pgfqpoint{1.564159in}{1.817958in}}%
\pgfpathlineto{\pgfqpoint{1.597553in}{1.791509in}}%
\pgfpathlineto{\pgfqpoint{1.630322in}{1.766590in}}%
\pgfpathlineto{\pgfqpoint{1.662479in}{1.743079in}}%
\pgfpathlineto{\pgfqpoint{1.694039in}{1.720867in}}%
\pgfpathlineto{\pgfqpoint{1.725017in}{1.699856in}}%
\pgfpathlineto{\pgfqpoint{1.755427in}{1.679957in}}%
\pgfpathlineto{\pgfqpoint{1.785284in}{1.661092in}}%
\pgfpathlineto{\pgfqpoint{1.814604in}{1.643186in}}%
\pgfpathlineto{\pgfqpoint{1.843400in}{1.626175in}}%
\pgfpathlineto{\pgfqpoint{1.871689in}{1.609998in}}%
\pgfpathlineto{\pgfqpoint{1.899483in}{1.594601in}}%
\pgfpathlineto{\pgfqpoint{1.926797in}{1.579934in}}%
\pgfpathlineto{\pgfqpoint{1.953643in}{1.565950in}}%
\pgfpathlineto{\pgfqpoint{1.980032in}{1.552608in}}%
\pgfpathlineto{\pgfqpoint{2.005975in}{1.539870in}}%
\pgfpathlineto{\pgfqpoint{2.031482in}{1.527700in}}%
\pgfpathlineto{\pgfqpoint{2.056561in}{1.516064in}}%
\pgfpathlineto{\pgfqpoint{2.081220in}{1.504932in}}%
\pgfpathlineto{\pgfqpoint{2.105467in}{1.494277in}}%
\pgfpathlineto{\pgfqpoint{2.129309in}{1.484071in}}%
\pgfpathlineto{\pgfqpoint{2.152752in}{1.474292in}}%
\pgfpathlineto{\pgfqpoint{2.175801in}{1.464915in}}%
\pgfpathlineto{\pgfqpoint{2.198463in}{1.455921in}}%
\pgfpathlineto{\pgfqpoint{2.220743in}{1.447289in}}%
\pgfpathlineto{\pgfqpoint{2.242646in}{1.439002in}}%
\pgfpathlineto{\pgfqpoint{2.264177in}{1.431042in}}%
\pgfpathlineto{\pgfqpoint{2.285342in}{1.423394in}}%
\pgfpathlineto{\pgfqpoint{2.306146in}{1.416042in}}%
\pgfpathlineto{\pgfqpoint{2.326594in}{1.408973in}}%
\pgfpathlineto{\pgfqpoint{2.346691in}{1.402174in}}%
\pgfpathlineto{\pgfqpoint{2.366442in}{1.395631in}}%
\pgfpathlineto{\pgfqpoint{2.385853in}{1.389333in}}%
\pgfpathlineto{\pgfqpoint{2.404929in}{1.383269in}}%
\pgfpathlineto{\pgfqpoint{2.423673in}{1.377430in}}%
\pgfpathlineto{\pgfqpoint{2.442092in}{1.371806in}}%
\pgfpathlineto{\pgfqpoint{2.460188in}{1.366387in}}%
\pgfpathlineto{\pgfqpoint{2.477967in}{1.361165in}}%
\pgfpathlineto{\pgfqpoint{2.495432in}{1.356132in}}%
\pgfpathlineto{\pgfqpoint{2.512587in}{1.351280in}}%
\pgfpathlineto{\pgfqpoint{2.529435in}{1.346602in}}%
\pgfpathlineto{\pgfqpoint{2.545981in}{1.342092in}}%
\pgfpathlineto{\pgfqpoint{2.562226in}{1.337741in}}%
\pgfpathlineto{\pgfqpoint{2.578173in}{1.333545in}}%
\pgfpathlineto{\pgfqpoint{2.593827in}{1.329498in}}%
\pgfpathlineto{\pgfqpoint{2.609189in}{1.325593in}}%
\pgfpathlineto{\pgfqpoint{2.624262in}{1.321827in}}%
\pgfpathlineto{\pgfqpoint{2.639048in}{1.318193in}}%
\pgfpathlineto{\pgfqpoint{2.653551in}{1.314688in}}%
\pgfpathlineto{\pgfqpoint{2.667773in}{1.311306in}}%
\pgfpathlineto{\pgfqpoint{2.681715in}{1.308043in}}%
\pgfpathlineto{\pgfqpoint{2.695381in}{1.304895in}}%
\pgfpathlineto{\pgfqpoint{2.708772in}{1.301858in}}%
\pgfpathlineto{\pgfqpoint{2.721892in}{1.298929in}}%
\pgfpathlineto{\pgfqpoint{2.734743in}{1.296105in}}%
\pgfpathlineto{\pgfqpoint{2.747326in}{1.293382in}}%
\pgfpathlineto{\pgfqpoint{2.759646in}{1.290755in}}%
\pgfpathlineto{\pgfqpoint{2.771703in}{1.288223in}}%
\pgfpathlineto{\pgfqpoint{2.783500in}{1.285783in}}%
\pgfpathlineto{\pgfqpoint{2.795040in}{1.283433in}}%
\pgfpathlineto{\pgfqpoint{2.806324in}{1.281168in}}%
\pgfpathlineto{\pgfqpoint{2.817355in}{1.278987in}}%
\pgfpathlineto{\pgfqpoint{2.828134in}{1.276887in}}%
\pgfpathlineto{\pgfqpoint{2.838665in}{1.274867in}}%
\pgfpathlineto{\pgfqpoint{2.848947in}{1.272923in}}%
\pgfpathlineto{\pgfqpoint{2.858984in}{1.271054in}}%
\pgfpathlineto{\pgfqpoint{2.868775in}{1.269258in}}%
\pgfpathlineto{\pgfqpoint{2.878324in}{1.267533in}}%
\pgfpathlineto{\pgfqpoint{2.887631in}{1.265877in}}%
\pgfpathlineto{\pgfqpoint{2.896696in}{1.264288in}}%
\pgfpathlineto{\pgfqpoint{2.905523in}{1.262765in}}%
\pgfpathlineto{\pgfqpoint{2.914111in}{1.261306in}}%
\pgfpathlineto{\pgfqpoint{2.922462in}{1.259909in}}%
\pgfpathlineto{\pgfqpoint{2.930576in}{1.258573in}}%
\pgfpathlineto{\pgfqpoint{2.938455in}{1.257297in}}%
\pgfpathlineto{\pgfqpoint{2.946100in}{1.256079in}}%
\pgfpathlineto{\pgfqpoint{2.953512in}{1.254919in}}%
\pgfpathlineto{\pgfqpoint{2.960692in}{1.253814in}}%
\pgfpathlineto{\pgfqpoint{2.967640in}{1.252764in}}%
\pgfpathlineto{\pgfqpoint{2.974359in}{1.251768in}}%
\pgfpathlineto{\pgfqpoint{2.980849in}{1.250824in}}%
\pgfpathlineto{\pgfqpoint{2.987110in}{1.249932in}}%
\pgfpathlineto{\pgfqpoint{2.993145in}{1.249091in}}%
\pgfpathlineto{\pgfqpoint{2.998953in}{1.248299in}}%
\pgfpathlineto{\pgfqpoint{3.004537in}{1.247555in}}%
\pgfpathlineto{\pgfqpoint{3.009897in}{1.246859in}}%
\pgfpathlineto{\pgfqpoint{3.015034in}{1.246211in}}%
\pgfpathlineto{\pgfqpoint{3.019948in}{1.245609in}}%
\pgfpathlineto{\pgfqpoint{3.024641in}{1.245054in}}%
\pgfusepath{stroke}%
\end{pgfscope}%
\begin{pgfscope}%
\pgfpathrectangle{\pgfqpoint{0.675193in}{0.526079in}}{\pgfqpoint{4.650000in}{3.020000in}}%
\pgfusepath{clip}%
\pgfsetrectcap%
\pgfsetroundjoin%
\pgfsetlinewidth{1.505625pt}%
\definecolor{currentstroke}{rgb}{1.000000,0.000000,0.000000}%
\pgfsetstrokecolor{currentstroke}%
\pgfsetstrokeopacity{0.750000}%
\pgfsetdash{}{0pt}%
\pgfpathmoveto{\pgfqpoint{0.943739in}{3.269665in}}%
\pgfpathlineto{\pgfqpoint{1.017119in}{3.296514in}}%
\pgfpathlineto{\pgfqpoint{1.053078in}{3.306553in}}%
\pgfpathlineto{\pgfqpoint{1.123568in}{3.321021in}}%
\pgfpathlineto{\pgfqpoint{1.192204in}{3.330787in}}%
\pgfpathlineto{\pgfqpoint{1.291806in}{3.340473in}}%
\pgfpathlineto{\pgfqpoint{1.418661in}{3.348417in}}%
\pgfpathlineto{\pgfqpoint{1.568331in}{3.354407in}}%
\pgfpathlineto{\pgfqpoint{1.789059in}{3.359981in}}%
\pgfpathlineto{\pgfqpoint{2.212153in}{3.366789in}}%
\pgfpathlineto{\pgfqpoint{2.793221in}{3.376590in}}%
\pgfpathlineto{\pgfqpoint{3.079157in}{3.384386in}}%
\pgfpathlineto{\pgfqpoint{3.271738in}{3.392491in}}%
\pgfpathlineto{\pgfqpoint{3.376355in}{3.399608in}}%
\pgfpathlineto{\pgfqpoint{3.387937in}{3.400757in}}%
\pgfpathlineto{\pgfqpoint{3.387937in}{3.400757in}}%
\pgfusepath{stroke}%
\end{pgfscope}%
\begin{pgfscope}%
\pgfpathrectangle{\pgfqpoint{0.675193in}{0.526079in}}{\pgfqpoint{4.650000in}{3.020000in}}%
\pgfusepath{clip}%
\pgfsetrectcap%
\pgfsetroundjoin%
\pgfsetlinewidth{1.505625pt}%
\definecolor{currentstroke}{rgb}{0.000000,0.000000,1.000000}%
\pgfsetstrokecolor{currentstroke}%
\pgfsetstrokeopacity{0.750000}%
\pgfsetdash{}{0pt}%
\pgfpathmoveto{\pgfqpoint{0.943739in}{1.847438in}}%
\pgfpathlineto{\pgfqpoint{1.017119in}{1.870742in}}%
\pgfpathlineto{\pgfqpoint{1.053078in}{1.879355in}}%
\pgfpathlineto{\pgfqpoint{1.123568in}{1.891406in}}%
\pgfpathlineto{\pgfqpoint{1.192204in}{1.899127in}}%
\pgfpathlineto{\pgfqpoint{1.291806in}{1.906111in}}%
\pgfpathlineto{\pgfqpoint{1.418661in}{1.910655in}}%
\pgfpathlineto{\pgfqpoint{1.568331in}{1.912107in}}%
\pgfpathlineto{\pgfqpoint{1.735927in}{1.910011in}}%
\pgfpathlineto{\pgfqpoint{1.891425in}{1.904755in}}%
\pgfpathlineto{\pgfqpoint{2.035702in}{1.896701in}}%
\pgfpathlineto{\pgfqpoint{2.169646in}{1.885950in}}%
\pgfpathlineto{\pgfqpoint{2.294129in}{1.872502in}}%
\pgfpathlineto{\pgfqpoint{2.391121in}{1.859198in}}%
\pgfpathlineto{\pgfqpoint{2.482260in}{1.843940in}}%
\pgfpathlineto{\pgfqpoint{2.567776in}{1.826667in}}%
\pgfpathlineto{\pgfqpoint{2.647929in}{1.807295in}}%
\pgfpathlineto{\pgfqpoint{2.722995in}{1.785733in}}%
\pgfpathlineto{\pgfqpoint{2.793221in}{1.761861in}}%
\pgfpathlineto{\pgfqpoint{2.846058in}{1.741004in}}%
\pgfpathlineto{\pgfqpoint{2.896038in}{1.718490in}}%
\pgfpathlineto{\pgfqpoint{2.943273in}{1.694213in}}%
\pgfpathlineto{\pgfqpoint{2.987880in}{1.668053in}}%
\pgfpathlineto{\pgfqpoint{3.029967in}{1.639867in}}%
\pgfpathlineto{\pgfqpoint{3.069618in}{1.609487in}}%
\pgfpathlineto{\pgfqpoint{3.106895in}{1.576710in}}%
\pgfpathlineto{\pgfqpoint{3.141850in}{1.541295in}}%
\pgfpathlineto{\pgfqpoint{3.174533in}{1.502951in}}%
\pgfpathlineto{\pgfqpoint{3.204996in}{1.461321in}}%
\pgfpathlineto{\pgfqpoint{3.233290in}{1.415966in}}%
\pgfpathlineto{\pgfqpoint{3.259450in}{1.366334in}}%
\pgfpathlineto{\pgfqpoint{3.277684in}{1.325887in}}%
\pgfpathlineto{\pgfqpoint{3.294737in}{1.282270in}}%
\pgfpathlineto{\pgfqpoint{3.310620in}{1.235024in}}%
\pgfpathlineto{\pgfqpoint{3.325349in}{1.183572in}}%
\pgfpathlineto{\pgfqpoint{3.338946in}{1.127178in}}%
\pgfpathlineto{\pgfqpoint{3.351435in}{1.064877in}}%
\pgfpathlineto{\pgfqpoint{3.362836in}{0.995375in}}%
\pgfpathlineto{\pgfqpoint{3.373156in}{0.916874in}}%
\pgfpathlineto{\pgfqpoint{3.382390in}{0.826773in}}%
\pgfpathlineto{\pgfqpoint{3.387937in}{0.758388in}}%
\pgfpathlineto{\pgfqpoint{3.387937in}{0.758388in}}%
\pgfusepath{stroke}%
\end{pgfscope}%
\begin{pgfscope}%
\pgfpathrectangle{\pgfqpoint{0.675193in}{0.526079in}}{\pgfqpoint{4.650000in}{3.020000in}}%
\pgfusepath{clip}%
\pgfsetrectcap%
\pgfsetroundjoin%
\pgfsetlinewidth{1.505625pt}%
\definecolor{currentstroke}{rgb}{0.000000,0.750000,0.750000}%
\pgfsetstrokecolor{currentstroke}%
\pgfsetstrokeopacity{0.750000}%
\pgfsetdash{}{0pt}%
\pgfpathmoveto{\pgfqpoint{0.943739in}{2.464167in}}%
\pgfpathlineto{\pgfqpoint{0.980675in}{2.376812in}}%
\pgfpathlineto{\pgfqpoint{1.017119in}{2.299901in}}%
\pgfpathlineto{\pgfqpoint{1.053078in}{2.229584in}}%
\pgfpathlineto{\pgfqpoint{1.088558in}{2.165074in}}%
\pgfpathlineto{\pgfqpoint{1.123568in}{2.105678in}}%
\pgfpathlineto{\pgfqpoint{1.158114in}{2.050812in}}%
\pgfpathlineto{\pgfqpoint{1.192204in}{1.999973in}}%
\pgfpathlineto{\pgfqpoint{1.259043in}{1.908706in}}%
\pgfpathlineto{\pgfqpoint{1.324142in}{1.829086in}}%
\pgfpathlineto{\pgfqpoint{1.387562in}{1.759017in}}%
\pgfpathlineto{\pgfqpoint{1.449363in}{1.696883in}}%
\pgfpathlineto{\pgfqpoint{1.509602in}{1.641425in}}%
\pgfpathlineto{\pgfqpoint{1.568331in}{1.591646in}}%
\pgfpathlineto{\pgfqpoint{1.625600in}{1.546740in}}%
\pgfpathlineto{\pgfqpoint{1.681452in}{1.506050in}}%
\pgfpathlineto{\pgfqpoint{1.762658in}{1.451760in}}%
\pgfpathlineto{\pgfqpoint{1.840881in}{1.404297in}}%
\pgfpathlineto{\pgfqpoint{1.916227in}{1.362523in}}%
\pgfpathlineto{\pgfqpoint{1.988803in}{1.325542in}}%
\pgfpathlineto{\pgfqpoint{2.081453in}{1.282471in}}%
\pgfpathlineto{\pgfqpoint{2.169646in}{1.245329in}}%
\pgfpathlineto{\pgfqpoint{2.253639in}{1.213086in}}%
\pgfpathlineto{\pgfqpoint{2.353039in}{1.178461in}}%
\pgfpathlineto{\pgfqpoint{2.446492in}{1.149026in}}%
\pgfpathlineto{\pgfqpoint{2.551111in}{1.119265in}}%
\pgfpathlineto{\pgfqpoint{2.663341in}{1.090694in}}%
\pgfpathlineto{\pgfqpoint{2.779552in}{1.064448in}}%
\pgfpathlineto{\pgfqpoint{2.896038in}{1.041278in}}%
\pgfpathlineto{\pgfqpoint{3.019676in}{1.019959in}}%
\pgfpathlineto{\pgfqpoint{3.141850in}{1.002245in}}%
\pgfpathlineto{\pgfqpoint{3.253109in}{0.989271in}}%
\pgfpathlineto{\pgfqpoint{3.343231in}{0.981744in}}%
\pgfpathlineto{\pgfqpoint{3.387937in}{0.979978in}}%
\pgfpathlineto{\pgfqpoint{3.387937in}{0.979978in}}%
\pgfusepath{stroke}%
\end{pgfscope}%
\begin{pgfscope}%
\pgfpathrectangle{\pgfqpoint{0.675193in}{0.526079in}}{\pgfqpoint{4.650000in}{3.020000in}}%
\pgfusepath{clip}%
\pgfsetrectcap%
\pgfsetroundjoin%
\pgfsetlinewidth{1.505625pt}%
\definecolor{currentstroke}{rgb}{1.000000,0.000000,0.000000}%
\pgfsetstrokecolor{currentstroke}%
\pgfsetstrokeopacity{0.750000}%
\pgfsetdash{}{0pt}%
\pgfpathmoveto{\pgfqpoint{1.220940in}{3.326916in}}%
\pgfpathlineto{\pgfqpoint{1.350641in}{3.341971in}}%
\pgfpathlineto{\pgfqpoint{1.475918in}{3.351628in}}%
\pgfpathlineto{\pgfqpoint{1.636036in}{3.359560in}}%
\pgfpathlineto{\pgfqpoint{1.824922in}{3.365419in}}%
\pgfpathlineto{\pgfqpoint{2.101271in}{3.370481in}}%
\pgfpathlineto{\pgfqpoint{2.520930in}{3.374478in}}%
\pgfpathlineto{\pgfqpoint{3.572486in}{3.379817in}}%
\pgfpathlineto{\pgfqpoint{4.244009in}{3.385221in}}%
\pgfpathlineto{\pgfqpoint{4.742371in}{3.392243in}}%
\pgfpathlineto{\pgfqpoint{4.911127in}{3.395497in}}%
\pgfpathlineto{\pgfqpoint{4.911127in}{3.395497in}}%
\pgfusepath{stroke}%
\end{pgfscope}%
\begin{pgfscope}%
\pgfpathrectangle{\pgfqpoint{0.675193in}{0.526079in}}{\pgfqpoint{4.650000in}{3.020000in}}%
\pgfusepath{clip}%
\pgfsetrectcap%
\pgfsetroundjoin%
\pgfsetlinewidth{1.505625pt}%
\definecolor{currentstroke}{rgb}{0.000000,0.000000,1.000000}%
\pgfsetstrokecolor{currentstroke}%
\pgfsetstrokeopacity{0.750000}%
\pgfsetdash{}{0pt}%
\pgfpathmoveto{\pgfqpoint{1.220940in}{1.570241in}}%
\pgfpathlineto{\pgfqpoint{1.350641in}{1.588461in}}%
\pgfpathlineto{\pgfqpoint{1.475918in}{1.602061in}}%
\pgfpathlineto{\pgfqpoint{1.636036in}{1.615832in}}%
\pgfpathlineto{\pgfqpoint{1.861176in}{1.631534in}}%
\pgfpathlineto{\pgfqpoint{2.133731in}{1.647119in}}%
\pgfpathlineto{\pgfqpoint{2.379087in}{1.658289in}}%
\pgfpathlineto{\pgfqpoint{2.602242in}{1.665647in}}%
\pgfpathlineto{\pgfqpoint{2.806534in}{1.669463in}}%
\pgfpathlineto{\pgfqpoint{2.994763in}{1.669925in}}%
\pgfpathlineto{\pgfqpoint{3.169252in}{1.667185in}}%
\pgfpathlineto{\pgfqpoint{3.331694in}{1.661369in}}%
\pgfpathlineto{\pgfqpoint{3.483097in}{1.652575in}}%
\pgfpathlineto{\pgfqpoint{3.607180in}{1.642492in}}%
\pgfpathlineto{\pgfqpoint{3.724111in}{1.630207in}}%
\pgfpathlineto{\pgfqpoint{3.834420in}{1.615728in}}%
\pgfpathlineto{\pgfqpoint{3.938531in}{1.599038in}}%
\pgfpathlineto{\pgfqpoint{4.037009in}{1.580103in}}%
\pgfpathlineto{\pgfqpoint{4.130507in}{1.558864in}}%
\pgfpathlineto{\pgfqpoint{4.219430in}{1.535237in}}%
\pgfpathlineto{\pgfqpoint{4.292060in}{1.513000in}}%
\pgfpathlineto{\pgfqpoint{4.361377in}{1.488835in}}%
\pgfpathlineto{\pgfqpoint{4.427512in}{1.462634in}}%
\pgfpathlineto{\pgfqpoint{4.490710in}{1.434260in}}%
\pgfpathlineto{\pgfqpoint{4.551186in}{1.403554in}}%
\pgfpathlineto{\pgfqpoint{4.609035in}{1.370316in}}%
\pgfpathlineto{\pgfqpoint{4.655255in}{1.340518in}}%
\pgfpathlineto{\pgfqpoint{4.699688in}{1.308633in}}%
\pgfpathlineto{\pgfqpoint{4.742371in}{1.274468in}}%
\pgfpathlineto{\pgfqpoint{4.783363in}{1.237788in}}%
\pgfpathlineto{\pgfqpoint{4.822752in}{1.198315in}}%
\pgfpathlineto{\pgfqpoint{4.860616in}{1.155705in}}%
\pgfpathlineto{\pgfqpoint{4.896993in}{1.109541in}}%
\pgfpathlineto{\pgfqpoint{4.911127in}{1.089966in}}%
\pgfpathlineto{\pgfqpoint{4.911127in}{1.089966in}}%
\pgfusepath{stroke}%
\end{pgfscope}%
\begin{pgfscope}%
\pgfpathrectangle{\pgfqpoint{0.675193in}{0.526079in}}{\pgfqpoint{4.650000in}{3.020000in}}%
\pgfusepath{clip}%
\pgfsetrectcap%
\pgfsetroundjoin%
\pgfsetlinewidth{1.505625pt}%
\definecolor{currentstroke}{rgb}{0.000000,0.750000,0.750000}%
\pgfsetstrokecolor{currentstroke}%
\pgfsetstrokeopacity{0.750000}%
\pgfsetdash{}{0pt}%
\pgfpathmoveto{\pgfqpoint{1.220940in}{2.211949in}}%
\pgfpathlineto{\pgfqpoint{1.264664in}{2.157737in}}%
\pgfpathlineto{\pgfqpoint{1.307898in}{2.108407in}}%
\pgfpathlineto{\pgfqpoint{1.350641in}{2.062642in}}%
\pgfpathlineto{\pgfqpoint{1.392893in}{2.020085in}}%
\pgfpathlineto{\pgfqpoint{1.434652in}{1.980418in}}%
\pgfpathlineto{\pgfqpoint{1.516690in}{1.908663in}}%
\pgfpathlineto{\pgfqpoint{1.596750in}{1.845527in}}%
\pgfpathlineto{\pgfqpoint{1.674825in}{1.789587in}}%
\pgfpathlineto{\pgfqpoint{1.750894in}{1.739711in}}%
\pgfpathlineto{\pgfqpoint{1.824922in}{1.694997in}}%
\pgfpathlineto{\pgfqpoint{1.896926in}{1.654713in}}%
\pgfpathlineto{\pgfqpoint{1.966942in}{1.618263in}}%
\pgfpathlineto{\pgfqpoint{2.068375in}{1.569711in}}%
\pgfpathlineto{\pgfqpoint{2.165767in}{1.527337in}}%
\pgfpathlineto{\pgfqpoint{2.259447in}{1.490096in}}%
\pgfpathlineto{\pgfqpoint{2.349713in}{1.457162in}}%
\pgfpathlineto{\pgfqpoint{2.465171in}{1.418828in}}%
\pgfpathlineto{\pgfqpoint{2.575441in}{1.385762in}}%
\pgfpathlineto{\pgfqpoint{2.680903in}{1.357021in}}%
\pgfpathlineto{\pgfqpoint{2.806534in}{1.326069in}}%
\pgfpathlineto{\pgfqpoint{2.925900in}{1.299629in}}%
\pgfpathlineto{\pgfqpoint{3.061699in}{1.272686in}}%
\pgfpathlineto{\pgfqpoint{3.210943in}{1.246500in}}%
\pgfpathlineto{\pgfqpoint{3.370549in}{1.221962in}}%
\pgfpathlineto{\pgfqpoint{3.537193in}{1.199639in}}%
\pgfpathlineto{\pgfqpoint{3.707822in}{1.179820in}}%
\pgfpathlineto{\pgfqpoint{3.894639in}{1.161238in}}%
\pgfpathlineto{\pgfqpoint{4.091011in}{1.144845in}}%
\pgfpathlineto{\pgfqpoint{4.292060in}{1.131169in}}%
\pgfpathlineto{\pgfqpoint{4.511164in}{1.119461in}}%
\pgfpathlineto{\pgfqpoint{4.725505in}{1.111165in}}%
\pgfpathlineto{\pgfqpoint{4.911127in}{1.106607in}}%
\pgfpathlineto{\pgfqpoint{4.911127in}{1.106607in}}%
\pgfusepath{stroke}%
\end{pgfscope}%
\begin{pgfscope}%
\pgfpathrectangle{\pgfqpoint{0.675193in}{0.526079in}}{\pgfqpoint{4.650000in}{3.020000in}}%
\pgfusepath{clip}%
\pgfsetrectcap%
\pgfsetroundjoin%
\pgfsetlinewidth{1.505625pt}%
\definecolor{currentstroke}{rgb}{1.000000,0.000000,0.000000}%
\pgfsetstrokecolor{currentstroke}%
\pgfsetstrokeopacity{0.750000}%
\pgfsetdash{}{0pt}%
\pgfpathmoveto{\pgfqpoint{1.005049in}{3.282734in}}%
\pgfpathlineto{\pgfqpoint{1.052517in}{3.294674in}}%
\pgfpathlineto{\pgfqpoint{1.099547in}{3.305606in}}%
\pgfpathlineto{\pgfqpoint{1.146148in}{3.313898in}}%
\pgfpathlineto{\pgfqpoint{1.192328in}{3.320315in}}%
\pgfpathlineto{\pgfqpoint{1.238095in}{3.325357in}}%
\pgfpathlineto{\pgfqpoint{1.283458in}{3.329364in}}%
\pgfpathlineto{\pgfqpoint{1.328425in}{3.332583in}}%
\pgfpathlineto{\pgfqpoint{1.373005in}{3.335186in}}%
\pgfpathlineto{\pgfqpoint{1.417205in}{3.337300in}}%
\pgfpathlineto{\pgfqpoint{1.461033in}{3.339021in}}%
\pgfpathlineto{\pgfqpoint{1.504500in}{3.340421in}}%
\pgfpathlineto{\pgfqpoint{1.547611in}{3.341558in}}%
\pgfpathlineto{\pgfqpoint{1.590380in}{3.342476in}}%
\pgfpathlineto{\pgfqpoint{1.632814in}{3.343209in}}%
\pgfpathlineto{\pgfqpoint{1.674921in}{3.343785in}}%
\pgfpathlineto{\pgfqpoint{1.716707in}{3.344227in}}%
\pgfpathlineto{\pgfqpoint{1.758181in}{3.344553in}}%
\pgfpathlineto{\pgfqpoint{1.799349in}{3.344777in}}%
\pgfpathlineto{\pgfqpoint{1.840218in}{3.344914in}}%
\pgfpathlineto{\pgfqpoint{1.880793in}{3.344975in}}%
\pgfpathlineto{\pgfqpoint{1.921080in}{3.344968in}}%
\pgfpathlineto{\pgfqpoint{1.961085in}{3.344900in}}%
\pgfpathlineto{\pgfqpoint{2.000814in}{3.344778in}}%
\pgfpathlineto{\pgfqpoint{2.040271in}{3.344607in}}%
\pgfpathlineto{\pgfqpoint{2.079463in}{3.344393in}}%
\pgfpathlineto{\pgfqpoint{2.118395in}{3.344139in}}%
\pgfpathlineto{\pgfqpoint{2.157075in}{3.343850in}}%
\pgfpathlineto{\pgfqpoint{2.195507in}{3.343527in}}%
\pgfpathlineto{\pgfqpoint{2.233698in}{3.343175in}}%
\pgfpathlineto{\pgfqpoint{2.271655in}{3.342794in}}%
\pgfpathlineto{\pgfqpoint{2.309385in}{3.342391in}}%
\pgfpathlineto{\pgfqpoint{2.346894in}{3.341964in}}%
\pgfpathlineto{\pgfqpoint{2.384188in}{3.341514in}}%
\pgfpathlineto{\pgfqpoint{2.421275in}{3.341046in}}%
\pgfpathlineto{\pgfqpoint{2.458159in}{3.340560in}}%
\pgfpathlineto{\pgfqpoint{2.494848in}{3.340055in}}%
\pgfpathlineto{\pgfqpoint{2.531347in}{3.339535in}}%
\pgfpathlineto{\pgfqpoint{2.567662in}{3.339001in}}%
\pgfpathlineto{\pgfqpoint{2.603796in}{3.338452in}}%
\pgfpathlineto{\pgfqpoint{2.639756in}{3.337890in}}%
\pgfpathlineto{\pgfqpoint{2.675544in}{3.337317in}}%
\pgfpathlineto{\pgfqpoint{2.711164in}{3.336731in}}%
\pgfpathlineto{\pgfqpoint{2.746620in}{3.336135in}}%
\pgfpathlineto{\pgfqpoint{2.781914in}{3.335529in}}%
\pgfpathlineto{\pgfqpoint{2.817049in}{3.334913in}}%
\pgfpathlineto{\pgfqpoint{2.852028in}{3.334288in}}%
\pgfpathlineto{\pgfqpoint{2.886853in}{3.333655in}}%
\pgfpathlineto{\pgfqpoint{2.921526in}{3.333014in}}%
\pgfpathlineto{\pgfqpoint{2.956050in}{3.332366in}}%
\pgfpathlineto{\pgfqpoint{2.990428in}{3.331711in}}%
\pgfpathlineto{\pgfqpoint{3.024662in}{3.331049in}}%
\pgfpathlineto{\pgfqpoint{3.058754in}{3.330378in}}%
\pgfpathlineto{\pgfqpoint{3.092707in}{3.329700in}}%
\pgfpathlineto{\pgfqpoint{3.126524in}{3.329018in}}%
\pgfpathlineto{\pgfqpoint{3.160209in}{3.328331in}}%
\pgfpathlineto{\pgfqpoint{3.193764in}{3.327638in}}%
\pgfpathlineto{\pgfqpoint{3.227192in}{3.326940in}}%
\pgfpathlineto{\pgfqpoint{3.260498in}{3.326237in}}%
\pgfpathlineto{\pgfqpoint{3.293683in}{3.325530in}}%
\pgfpathlineto{\pgfqpoint{3.326753in}{3.324819in}}%
\pgfpathlineto{\pgfqpoint{3.359709in}{3.324103in}}%
\pgfpathlineto{\pgfqpoint{3.392556in}{3.323383in}}%
\pgfpathlineto{\pgfqpoint{3.425296in}{3.322660in}}%
\pgfpathlineto{\pgfqpoint{3.457932in}{3.321933in}}%
\pgfpathlineto{\pgfqpoint{3.490468in}{3.321203in}}%
\pgfpathlineto{\pgfqpoint{3.522906in}{3.320469in}}%
\pgfpathlineto{\pgfqpoint{3.555248in}{3.319732in}}%
\pgfpathlineto{\pgfqpoint{3.587498in}{3.318993in}}%
\pgfpathlineto{\pgfqpoint{3.619656in}{3.318250in}}%
\pgfpathlineto{\pgfqpoint{3.651726in}{3.317505in}}%
\pgfpathlineto{\pgfqpoint{3.683709in}{3.316757in}}%
\pgfpathlineto{\pgfqpoint{3.715606in}{3.316007in}}%
\pgfpathlineto{\pgfqpoint{3.747421in}{3.315255in}}%
\pgfpathlineto{\pgfqpoint{3.779153in}{3.314501in}}%
\pgfpathlineto{\pgfqpoint{3.810805in}{3.313744in}}%
\pgfpathlineto{\pgfqpoint{3.842378in}{3.312985in}}%
\pgfpathlineto{\pgfqpoint{3.873873in}{3.312223in}}%
\pgfpathlineto{\pgfqpoint{3.905292in}{3.311460in}}%
\pgfpathlineto{\pgfqpoint{3.936636in}{3.310695in}}%
\pgfpathlineto{\pgfqpoint{3.967907in}{3.309928in}}%
\pgfpathlineto{\pgfqpoint{3.999105in}{3.309160in}}%
\pgfpathlineto{\pgfqpoint{4.030233in}{3.308390in}}%
\pgfpathlineto{\pgfqpoint{4.061290in}{3.307620in}}%
\pgfpathlineto{\pgfqpoint{4.092280in}{3.306849in}}%
\pgfpathlineto{\pgfqpoint{4.123202in}{3.306077in}}%
\pgfpathlineto{\pgfqpoint{4.154058in}{3.305303in}}%
\pgfpathlineto{\pgfqpoint{4.184849in}{3.304527in}}%
\pgfpathlineto{\pgfqpoint{4.215576in}{3.303750in}}%
\pgfpathlineto{\pgfqpoint{4.246240in}{3.302973in}}%
\pgfpathlineto{\pgfqpoint{4.276843in}{3.302194in}}%
\pgfpathlineto{\pgfqpoint{4.307384in}{3.301416in}}%
\pgfpathlineto{\pgfqpoint{4.337864in}{3.300637in}}%
\pgfpathlineto{\pgfqpoint{4.368286in}{3.299856in}}%
\pgfpathlineto{\pgfqpoint{4.398648in}{3.299074in}}%
\pgfpathlineto{\pgfqpoint{4.428952in}{3.298292in}}%
\pgfpathlineto{\pgfqpoint{4.459198in}{3.297509in}}%
\pgfpathlineto{\pgfqpoint{4.489388in}{3.296726in}}%
\pgfpathlineto{\pgfqpoint{4.519521in}{3.295943in}}%
\pgfpathlineto{\pgfqpoint{4.549598in}{3.295159in}}%
\pgfpathlineto{\pgfqpoint{4.579619in}{3.294374in}}%
\pgfpathlineto{\pgfqpoint{4.609587in}{3.293589in}}%
\pgfpathlineto{\pgfqpoint{4.639500in}{3.292803in}}%
\pgfpathlineto{\pgfqpoint{4.669360in}{3.292018in}}%
\pgfpathlineto{\pgfqpoint{4.699168in}{3.291232in}}%
\pgfpathlineto{\pgfqpoint{4.728924in}{3.290446in}}%
\pgfpathlineto{\pgfqpoint{4.758629in}{3.289660in}}%
\pgfpathlineto{\pgfqpoint{4.788284in}{3.288873in}}%
\pgfpathlineto{\pgfqpoint{4.817890in}{3.288087in}}%
\pgfpathlineto{\pgfqpoint{4.847448in}{3.287300in}}%
\pgfpathlineto{\pgfqpoint{4.876959in}{3.286513in}}%
\pgfpathlineto{\pgfqpoint{4.906424in}{3.285727in}}%
\pgfpathlineto{\pgfqpoint{4.935843in}{3.284940in}}%
\pgfpathlineto{\pgfqpoint{4.965218in}{3.284154in}}%
\pgfpathlineto{\pgfqpoint{4.994548in}{3.283367in}}%
\pgfpathlineto{\pgfqpoint{5.023836in}{3.282581in}}%
\pgfpathlineto{\pgfqpoint{5.053081in}{3.281795in}}%
\pgfpathlineto{\pgfqpoint{5.082285in}{3.281008in}}%
\pgfpathlineto{\pgfqpoint{5.111447in}{3.280222in}}%
\pgfusepath{stroke}%
\end{pgfscope}%
\begin{pgfscope}%
\pgfpathrectangle{\pgfqpoint{0.675193in}{0.526079in}}{\pgfqpoint{4.650000in}{3.020000in}}%
\pgfusepath{clip}%
\pgfsetrectcap%
\pgfsetroundjoin%
\pgfsetlinewidth{1.505625pt}%
\definecolor{currentstroke}{rgb}{0.000000,0.000000,1.000000}%
\pgfsetstrokecolor{currentstroke}%
\pgfsetstrokeopacity{0.750000}%
\pgfsetdash{}{0pt}%
\pgfpathmoveto{\pgfqpoint{1.005049in}{1.903514in}}%
\pgfpathlineto{\pgfqpoint{1.052517in}{1.920295in}}%
\pgfpathlineto{\pgfqpoint{1.099547in}{1.936347in}}%
\pgfpathlineto{\pgfqpoint{1.146148in}{1.949985in}}%
\pgfpathlineto{\pgfqpoint{1.192328in}{1.961941in}}%
\pgfpathlineto{\pgfqpoint{1.238095in}{1.972687in}}%
\pgfpathlineto{\pgfqpoint{1.283458in}{1.982544in}}%
\pgfpathlineto{\pgfqpoint{1.328425in}{1.991737in}}%
\pgfpathlineto{\pgfqpoint{1.373005in}{2.000426in}}%
\pgfpathlineto{\pgfqpoint{1.417205in}{2.008724in}}%
\pgfpathlineto{\pgfqpoint{1.461033in}{2.016716in}}%
\pgfpathlineto{\pgfqpoint{1.504500in}{2.024464in}}%
\pgfpathlineto{\pgfqpoint{1.547611in}{2.032017in}}%
\pgfpathlineto{\pgfqpoint{1.590380in}{2.039412in}}%
\pgfpathlineto{\pgfqpoint{1.632814in}{2.046672in}}%
\pgfpathlineto{\pgfqpoint{1.674921in}{2.053821in}}%
\pgfpathlineto{\pgfqpoint{1.716707in}{2.060874in}}%
\pgfpathlineto{\pgfqpoint{1.758181in}{2.067845in}}%
\pgfpathlineto{\pgfqpoint{1.799349in}{2.074741in}}%
\pgfpathlineto{\pgfqpoint{1.840218in}{2.081571in}}%
\pgfpathlineto{\pgfqpoint{1.880793in}{2.088343in}}%
\pgfpathlineto{\pgfqpoint{1.921080in}{2.095060in}}%
\pgfpathlineto{\pgfqpoint{1.961085in}{2.101724in}}%
\pgfpathlineto{\pgfqpoint{2.000814in}{2.108339in}}%
\pgfpathlineto{\pgfqpoint{2.040271in}{2.114907in}}%
\pgfpathlineto{\pgfqpoint{2.079463in}{2.121429in}}%
\pgfpathlineto{\pgfqpoint{2.118395in}{2.127907in}}%
\pgfpathlineto{\pgfqpoint{2.157075in}{2.134340in}}%
\pgfpathlineto{\pgfqpoint{2.195507in}{2.140729in}}%
\pgfpathlineto{\pgfqpoint{2.233698in}{2.147075in}}%
\pgfpathlineto{\pgfqpoint{2.271655in}{2.153378in}}%
\pgfpathlineto{\pgfqpoint{2.309385in}{2.159639in}}%
\pgfpathlineto{\pgfqpoint{2.346894in}{2.165858in}}%
\pgfpathlineto{\pgfqpoint{2.384188in}{2.172032in}}%
\pgfpathlineto{\pgfqpoint{2.421275in}{2.178164in}}%
\pgfpathlineto{\pgfqpoint{2.458159in}{2.184253in}}%
\pgfpathlineto{\pgfqpoint{2.494848in}{2.190297in}}%
\pgfpathlineto{\pgfqpoint{2.531347in}{2.196299in}}%
\pgfpathlineto{\pgfqpoint{2.567662in}{2.202257in}}%
\pgfpathlineto{\pgfqpoint{2.603796in}{2.208171in}}%
\pgfpathlineto{\pgfqpoint{2.639756in}{2.214041in}}%
\pgfpathlineto{\pgfqpoint{2.675544in}{2.219867in}}%
\pgfpathlineto{\pgfqpoint{2.711164in}{2.225648in}}%
\pgfpathlineto{\pgfqpoint{2.746620in}{2.231386in}}%
\pgfpathlineto{\pgfqpoint{2.781914in}{2.237079in}}%
\pgfpathlineto{\pgfqpoint{2.817049in}{2.242728in}}%
\pgfpathlineto{\pgfqpoint{2.852028in}{2.248333in}}%
\pgfpathlineto{\pgfqpoint{2.886853in}{2.253893in}}%
\pgfpathlineto{\pgfqpoint{2.921526in}{2.259410in}}%
\pgfpathlineto{\pgfqpoint{2.956050in}{2.264883in}}%
\pgfpathlineto{\pgfqpoint{2.990428in}{2.270312in}}%
\pgfpathlineto{\pgfqpoint{3.024662in}{2.275697in}}%
\pgfpathlineto{\pgfqpoint{3.058754in}{2.281036in}}%
\pgfpathlineto{\pgfqpoint{3.092707in}{2.286331in}}%
\pgfpathlineto{\pgfqpoint{3.126524in}{2.291583in}}%
\pgfpathlineto{\pgfqpoint{3.160209in}{2.296793in}}%
\pgfpathlineto{\pgfqpoint{3.193764in}{2.301959in}}%
\pgfpathlineto{\pgfqpoint{3.227192in}{2.307082in}}%
\pgfpathlineto{\pgfqpoint{3.260498in}{2.312162in}}%
\pgfpathlineto{\pgfqpoint{3.293683in}{2.317200in}}%
\pgfpathlineto{\pgfqpoint{3.326753in}{2.322196in}}%
\pgfpathlineto{\pgfqpoint{3.359709in}{2.327150in}}%
\pgfpathlineto{\pgfqpoint{3.392556in}{2.332062in}}%
\pgfpathlineto{\pgfqpoint{3.425296in}{2.336933in}}%
\pgfpathlineto{\pgfqpoint{3.457932in}{2.341763in}}%
\pgfpathlineto{\pgfqpoint{3.490468in}{2.346552in}}%
\pgfpathlineto{\pgfqpoint{3.522906in}{2.351300in}}%
\pgfpathlineto{\pgfqpoint{3.555248in}{2.356008in}}%
\pgfpathlineto{\pgfqpoint{3.587498in}{2.360676in}}%
\pgfpathlineto{\pgfqpoint{3.619656in}{2.365305in}}%
\pgfpathlineto{\pgfqpoint{3.651726in}{2.369894in}}%
\pgfpathlineto{\pgfqpoint{3.683709in}{2.374444in}}%
\pgfpathlineto{\pgfqpoint{3.715606in}{2.378957in}}%
\pgfpathlineto{\pgfqpoint{3.747421in}{2.383431in}}%
\pgfpathlineto{\pgfqpoint{3.779153in}{2.387867in}}%
\pgfpathlineto{\pgfqpoint{3.810805in}{2.392265in}}%
\pgfpathlineto{\pgfqpoint{3.842378in}{2.396626in}}%
\pgfpathlineto{\pgfqpoint{3.873873in}{2.400949in}}%
\pgfpathlineto{\pgfqpoint{3.905292in}{2.405235in}}%
\pgfpathlineto{\pgfqpoint{3.936636in}{2.409486in}}%
\pgfpathlineto{\pgfqpoint{3.967907in}{2.413700in}}%
\pgfpathlineto{\pgfqpoint{3.999105in}{2.417880in}}%
\pgfpathlineto{\pgfqpoint{4.030233in}{2.422024in}}%
\pgfpathlineto{\pgfqpoint{4.061290in}{2.426134in}}%
\pgfpathlineto{\pgfqpoint{4.092280in}{2.430211in}}%
\pgfpathlineto{\pgfqpoint{4.123202in}{2.434253in}}%
\pgfpathlineto{\pgfqpoint{4.154058in}{2.438260in}}%
\pgfpathlineto{\pgfqpoint{4.184849in}{2.442234in}}%
\pgfpathlineto{\pgfqpoint{4.215576in}{2.446175in}}%
\pgfpathlineto{\pgfqpoint{4.246240in}{2.450083in}}%
\pgfpathlineto{\pgfqpoint{4.276843in}{2.453959in}}%
\pgfpathlineto{\pgfqpoint{4.307384in}{2.457803in}}%
\pgfpathlineto{\pgfqpoint{4.337864in}{2.461616in}}%
\pgfpathlineto{\pgfqpoint{4.368286in}{2.465396in}}%
\pgfpathlineto{\pgfqpoint{4.398648in}{2.469146in}}%
\pgfpathlineto{\pgfqpoint{4.428952in}{2.472864in}}%
\pgfpathlineto{\pgfqpoint{4.459198in}{2.476553in}}%
\pgfpathlineto{\pgfqpoint{4.489388in}{2.480211in}}%
\pgfpathlineto{\pgfqpoint{4.519521in}{2.483839in}}%
\pgfpathlineto{\pgfqpoint{4.549598in}{2.487438in}}%
\pgfpathlineto{\pgfqpoint{4.579619in}{2.491008in}}%
\pgfpathlineto{\pgfqpoint{4.609587in}{2.494548in}}%
\pgfpathlineto{\pgfqpoint{4.639500in}{2.498061in}}%
\pgfpathlineto{\pgfqpoint{4.669360in}{2.501544in}}%
\pgfpathlineto{\pgfqpoint{4.699168in}{2.505000in}}%
\pgfpathlineto{\pgfqpoint{4.728924in}{2.508429in}}%
\pgfpathlineto{\pgfqpoint{4.758629in}{2.511830in}}%
\pgfpathlineto{\pgfqpoint{4.788284in}{2.515204in}}%
\pgfpathlineto{\pgfqpoint{4.817890in}{2.518550in}}%
\pgfpathlineto{\pgfqpoint{4.847448in}{2.521871in}}%
\pgfpathlineto{\pgfqpoint{4.876959in}{2.525165in}}%
\pgfpathlineto{\pgfqpoint{4.906424in}{2.528433in}}%
\pgfpathlineto{\pgfqpoint{4.935843in}{2.531676in}}%
\pgfpathlineto{\pgfqpoint{4.965218in}{2.534893in}}%
\pgfpathlineto{\pgfqpoint{4.994548in}{2.538086in}}%
\pgfpathlineto{\pgfqpoint{5.023836in}{2.541253in}}%
\pgfpathlineto{\pgfqpoint{5.053081in}{2.544395in}}%
\pgfpathlineto{\pgfqpoint{5.082285in}{2.547513in}}%
\pgfpathlineto{\pgfqpoint{5.111447in}{2.550608in}}%
\pgfusepath{stroke}%
\end{pgfscope}%
\begin{pgfscope}%
\pgfpathrectangle{\pgfqpoint{0.675193in}{0.526079in}}{\pgfqpoint{4.650000in}{3.020000in}}%
\pgfusepath{clip}%
\pgfsetrectcap%
\pgfsetroundjoin%
\pgfsetlinewidth{1.505625pt}%
\definecolor{currentstroke}{rgb}{0.000000,0.750000,0.750000}%
\pgfsetstrokecolor{currentstroke}%
\pgfsetstrokeopacity{0.750000}%
\pgfsetdash{}{0pt}%
\pgfpathmoveto{\pgfqpoint{1.005049in}{2.380485in}}%
\pgfpathlineto{\pgfqpoint{1.052517in}{2.286113in}}%
\pgfpathlineto{\pgfqpoint{1.099547in}{2.203202in}}%
\pgfpathlineto{\pgfqpoint{1.146148in}{2.127793in}}%
\pgfpathlineto{\pgfqpoint{1.192328in}{2.058923in}}%
\pgfpathlineto{\pgfqpoint{1.238095in}{1.995763in}}%
\pgfpathlineto{\pgfqpoint{1.283458in}{1.937618in}}%
\pgfpathlineto{\pgfqpoint{1.328425in}{1.883906in}}%
\pgfpathlineto{\pgfqpoint{1.373005in}{1.834129in}}%
\pgfpathlineto{\pgfqpoint{1.417205in}{1.787862in}}%
\pgfpathlineto{\pgfqpoint{1.461033in}{1.744740in}}%
\pgfpathlineto{\pgfqpoint{1.504500in}{1.704450in}}%
\pgfpathlineto{\pgfqpoint{1.547611in}{1.666721in}}%
\pgfpathlineto{\pgfqpoint{1.590380in}{1.631316in}}%
\pgfpathlineto{\pgfqpoint{1.632814in}{1.598025in}}%
\pgfpathlineto{\pgfqpoint{1.674921in}{1.566665in}}%
\pgfpathlineto{\pgfqpoint{1.716707in}{1.537074in}}%
\pgfpathlineto{\pgfqpoint{1.758181in}{1.509108in}}%
\pgfpathlineto{\pgfqpoint{1.799349in}{1.482637in}}%
\pgfpathlineto{\pgfqpoint{1.840218in}{1.457547in}}%
\pgfpathlineto{\pgfqpoint{1.880793in}{1.433736in}}%
\pgfpathlineto{\pgfqpoint{1.921080in}{1.411108in}}%
\pgfpathlineto{\pgfqpoint{1.961085in}{1.389581in}}%
\pgfpathlineto{\pgfqpoint{2.000814in}{1.369076in}}%
\pgfpathlineto{\pgfqpoint{2.040271in}{1.349524in}}%
\pgfpathlineto{\pgfqpoint{2.079463in}{1.330863in}}%
\pgfpathlineto{\pgfqpoint{2.118395in}{1.313033in}}%
\pgfpathlineto{\pgfqpoint{2.157075in}{1.295983in}}%
\pgfpathlineto{\pgfqpoint{2.195507in}{1.279662in}}%
\pgfpathlineto{\pgfqpoint{2.233698in}{1.264026in}}%
\pgfpathlineto{\pgfqpoint{2.271655in}{1.249033in}}%
\pgfpathlineto{\pgfqpoint{2.309385in}{1.234649in}}%
\pgfpathlineto{\pgfqpoint{2.346894in}{1.220835in}}%
\pgfpathlineto{\pgfqpoint{2.384188in}{1.207558in}}%
\pgfpathlineto{\pgfqpoint{2.421275in}{1.194790in}}%
\pgfpathlineto{\pgfqpoint{2.458159in}{1.182501in}}%
\pgfpathlineto{\pgfqpoint{2.494848in}{1.170666in}}%
\pgfpathlineto{\pgfqpoint{2.531347in}{1.159260in}}%
\pgfpathlineto{\pgfqpoint{2.567662in}{1.148262in}}%
\pgfpathlineto{\pgfqpoint{2.603796in}{1.137649in}}%
\pgfpathlineto{\pgfqpoint{2.639756in}{1.127402in}}%
\pgfpathlineto{\pgfqpoint{2.675544in}{1.117502in}}%
\pgfpathlineto{\pgfqpoint{2.711164in}{1.107932in}}%
\pgfpathlineto{\pgfqpoint{2.746620in}{1.098676in}}%
\pgfpathlineto{\pgfqpoint{2.781914in}{1.089720in}}%
\pgfpathlineto{\pgfqpoint{2.817049in}{1.081047in}}%
\pgfpathlineto{\pgfqpoint{2.852028in}{1.072645in}}%
\pgfpathlineto{\pgfqpoint{2.886853in}{1.064503in}}%
\pgfpathlineto{\pgfqpoint{2.921526in}{1.056606in}}%
\pgfpathlineto{\pgfqpoint{2.956050in}{1.048946in}}%
\pgfpathlineto{\pgfqpoint{2.990428in}{1.041511in}}%
\pgfpathlineto{\pgfqpoint{3.024662in}{1.034291in}}%
\pgfpathlineto{\pgfqpoint{3.058754in}{1.027273in}}%
\pgfpathlineto{\pgfqpoint{3.092707in}{1.020451in}}%
\pgfpathlineto{\pgfqpoint{3.126524in}{1.013819in}}%
\pgfpathlineto{\pgfqpoint{3.160209in}{1.007368in}}%
\pgfpathlineto{\pgfqpoint{3.193764in}{1.001089in}}%
\pgfpathlineto{\pgfqpoint{3.227192in}{0.994975in}}%
\pgfpathlineto{\pgfqpoint{3.260498in}{0.989019in}}%
\pgfpathlineto{\pgfqpoint{3.293683in}{0.983217in}}%
\pgfpathlineto{\pgfqpoint{3.326753in}{0.977560in}}%
\pgfpathlineto{\pgfqpoint{3.359709in}{0.972044in}}%
\pgfpathlineto{\pgfqpoint{3.392556in}{0.966663in}}%
\pgfpathlineto{\pgfqpoint{3.425296in}{0.961411in}}%
\pgfpathlineto{\pgfqpoint{3.457932in}{0.956284in}}%
\pgfpathlineto{\pgfqpoint{3.490468in}{0.951277in}}%
\pgfpathlineto{\pgfqpoint{3.522906in}{0.946386in}}%
\pgfpathlineto{\pgfqpoint{3.555248in}{0.941605in}}%
\pgfpathlineto{\pgfqpoint{3.587498in}{0.936932in}}%
\pgfpathlineto{\pgfqpoint{3.619656in}{0.932362in}}%
\pgfpathlineto{\pgfqpoint{3.651726in}{0.927891in}}%
\pgfpathlineto{\pgfqpoint{3.683709in}{0.923516in}}%
\pgfpathlineto{\pgfqpoint{3.715606in}{0.919234in}}%
\pgfpathlineto{\pgfqpoint{3.747421in}{0.915042in}}%
\pgfpathlineto{\pgfqpoint{3.779153in}{0.910935in}}%
\pgfpathlineto{\pgfqpoint{3.810805in}{0.906912in}}%
\pgfpathlineto{\pgfqpoint{3.842378in}{0.902968in}}%
\pgfpathlineto{\pgfqpoint{3.873873in}{0.899102in}}%
\pgfpathlineto{\pgfqpoint{3.905292in}{0.895310in}}%
\pgfpathlineto{\pgfqpoint{3.936636in}{0.891591in}}%
\pgfpathlineto{\pgfqpoint{3.967907in}{0.887942in}}%
\pgfpathlineto{\pgfqpoint{3.999105in}{0.884362in}}%
\pgfpathlineto{\pgfqpoint{4.030233in}{0.880847in}}%
\pgfpathlineto{\pgfqpoint{4.061290in}{0.877396in}}%
\pgfpathlineto{\pgfqpoint{4.092280in}{0.874008in}}%
\pgfpathlineto{\pgfqpoint{4.123202in}{0.870680in}}%
\pgfpathlineto{\pgfqpoint{4.154058in}{0.867408in}}%
\pgfpathlineto{\pgfqpoint{4.184849in}{0.864192in}}%
\pgfpathlineto{\pgfqpoint{4.215576in}{0.861031in}}%
\pgfpathlineto{\pgfqpoint{4.246240in}{0.857922in}}%
\pgfpathlineto{\pgfqpoint{4.276843in}{0.854865in}}%
\pgfpathlineto{\pgfqpoint{4.307384in}{0.851858in}}%
\pgfpathlineto{\pgfqpoint{4.337864in}{0.848899in}}%
\pgfpathlineto{\pgfqpoint{4.368286in}{0.845986in}}%
\pgfpathlineto{\pgfqpoint{4.398648in}{0.843118in}}%
\pgfpathlineto{\pgfqpoint{4.428952in}{0.840294in}}%
\pgfpathlineto{\pgfqpoint{4.459198in}{0.837514in}}%
\pgfpathlineto{\pgfqpoint{4.489388in}{0.834775in}}%
\pgfpathlineto{\pgfqpoint{4.519521in}{0.832076in}}%
\pgfpathlineto{\pgfqpoint{4.549598in}{0.829417in}}%
\pgfpathlineto{\pgfqpoint{4.579619in}{0.826796in}}%
\pgfpathlineto{\pgfqpoint{4.609587in}{0.824213in}}%
\pgfpathlineto{\pgfqpoint{4.639500in}{0.821665in}}%
\pgfpathlineto{\pgfqpoint{4.669360in}{0.819153in}}%
\pgfpathlineto{\pgfqpoint{4.699168in}{0.816675in}}%
\pgfpathlineto{\pgfqpoint{4.728924in}{0.814230in}}%
\pgfpathlineto{\pgfqpoint{4.758629in}{0.811818in}}%
\pgfpathlineto{\pgfqpoint{4.788284in}{0.809438in}}%
\pgfpathlineto{\pgfqpoint{4.817890in}{0.807088in}}%
\pgfpathlineto{\pgfqpoint{4.847448in}{0.804768in}}%
\pgfpathlineto{\pgfqpoint{4.876959in}{0.802478in}}%
\pgfpathlineto{\pgfqpoint{4.906424in}{0.800217in}}%
\pgfpathlineto{\pgfqpoint{4.935843in}{0.797983in}}%
\pgfpathlineto{\pgfqpoint{4.965218in}{0.795777in}}%
\pgfpathlineto{\pgfqpoint{4.994548in}{0.793597in}}%
\pgfpathlineto{\pgfqpoint{5.023836in}{0.791442in}}%
\pgfpathlineto{\pgfqpoint{5.053081in}{0.789314in}}%
\pgfpathlineto{\pgfqpoint{5.082285in}{0.787209in}}%
\pgfpathlineto{\pgfqpoint{5.111447in}{0.785129in}}%
\pgfusepath{stroke}%
\end{pgfscope}%
\begin{pgfscope}%
\pgfpathrectangle{\pgfqpoint{0.675193in}{0.526079in}}{\pgfqpoint{4.650000in}{3.020000in}}%
\pgfusepath{clip}%
\pgfsetrectcap%
\pgfsetroundjoin%
\pgfsetlinewidth{1.505625pt}%
\definecolor{currentstroke}{rgb}{1.000000,0.000000,0.000000}%
\pgfsetstrokecolor{currentstroke}%
\pgfsetstrokeopacity{0.750000}%
\pgfsetdash{}{0pt}%
\pgfpathmoveto{\pgfqpoint{1.266608in}{3.308691in}}%
\pgfpathlineto{\pgfqpoint{1.315801in}{3.311820in}}%
\pgfpathlineto{\pgfqpoint{1.364866in}{3.314782in}}%
\pgfpathlineto{\pgfqpoint{1.413797in}{3.317039in}}%
\pgfpathlineto{\pgfqpoint{1.462586in}{3.318755in}}%
\pgfpathlineto{\pgfqpoint{1.511228in}{3.320036in}}%
\pgfpathlineto{\pgfqpoint{1.559715in}{3.320965in}}%
\pgfpathlineto{\pgfqpoint{1.608040in}{3.321602in}}%
\pgfpathlineto{\pgfqpoint{1.656198in}{3.321994in}}%
\pgfpathlineto{\pgfqpoint{1.704181in}{3.322179in}}%
\pgfpathlineto{\pgfqpoint{1.751983in}{3.322185in}}%
\pgfpathlineto{\pgfqpoint{1.799596in}{3.322040in}}%
\pgfpathlineto{\pgfqpoint{1.847015in}{3.321764in}}%
\pgfpathlineto{\pgfqpoint{1.894222in}{3.321372in}}%
\pgfpathlineto{\pgfqpoint{1.941215in}{3.320877in}}%
\pgfpathlineto{\pgfqpoint{1.987986in}{3.320291in}}%
\pgfpathlineto{\pgfqpoint{2.034532in}{3.319624in}}%
\pgfpathlineto{\pgfqpoint{2.080847in}{3.318883in}}%
\pgfpathlineto{\pgfqpoint{2.126928in}{3.318075in}}%
\pgfpathlineto{\pgfqpoint{2.172772in}{3.317206in}}%
\pgfpathlineto{\pgfqpoint{2.218380in}{3.316281in}}%
\pgfpathlineto{\pgfqpoint{2.263751in}{3.315305in}}%
\pgfpathlineto{\pgfqpoint{2.308887in}{3.314284in}}%
\pgfpathlineto{\pgfqpoint{2.353790in}{3.313218in}}%
\pgfpathlineto{\pgfqpoint{2.398465in}{3.312111in}}%
\pgfpathlineto{\pgfqpoint{2.442915in}{3.310967in}}%
\pgfpathlineto{\pgfqpoint{2.487147in}{3.309787in}}%
\pgfpathlineto{\pgfqpoint{2.531166in}{3.308573in}}%
\pgfpathlineto{\pgfqpoint{2.574980in}{3.307328in}}%
\pgfpathlineto{\pgfqpoint{2.618594in}{3.306053in}}%
\pgfpathlineto{\pgfqpoint{2.662017in}{3.304751in}}%
\pgfpathlineto{\pgfqpoint{2.705255in}{3.303422in}}%
\pgfpathlineto{\pgfqpoint{2.748315in}{3.302066in}}%
\pgfpathlineto{\pgfqpoint{2.791204in}{3.300688in}}%
\pgfpathlineto{\pgfqpoint{2.833930in}{3.299287in}}%
\pgfpathlineto{\pgfqpoint{2.876500in}{3.297864in}}%
\pgfpathlineto{\pgfqpoint{2.918921in}{3.296420in}}%
\pgfpathlineto{\pgfqpoint{2.961199in}{3.294956in}}%
\pgfpathlineto{\pgfqpoint{3.003342in}{3.293474in}}%
\pgfpathlineto{\pgfqpoint{3.045355in}{3.291975in}}%
\pgfpathlineto{\pgfqpoint{3.087242in}{3.290457in}}%
\pgfpathlineto{\pgfqpoint{3.129009in}{3.288920in}}%
\pgfpathlineto{\pgfqpoint{3.170661in}{3.287366in}}%
\pgfpathlineto{\pgfqpoint{3.212203in}{3.285798in}}%
\pgfpathlineto{\pgfqpoint{3.253639in}{3.284216in}}%
\pgfpathlineto{\pgfqpoint{3.294973in}{3.282618in}}%
\pgfpathlineto{\pgfqpoint{3.336209in}{3.281007in}}%
\pgfpathlineto{\pgfqpoint{3.377348in}{3.279383in}}%
\pgfpathlineto{\pgfqpoint{3.418395in}{3.277746in}}%
\pgfpathlineto{\pgfqpoint{3.459351in}{3.276096in}}%
\pgfpathlineto{\pgfqpoint{3.500221in}{3.274434in}}%
\pgfpathlineto{\pgfqpoint{3.541007in}{3.272760in}}%
\pgfpathlineto{\pgfqpoint{3.581713in}{3.271076in}}%
\pgfpathlineto{\pgfqpoint{3.622340in}{3.269380in}}%
\pgfpathlineto{\pgfqpoint{3.662890in}{3.267674in}}%
\pgfpathlineto{\pgfqpoint{3.703364in}{3.265958in}}%
\pgfpathlineto{\pgfqpoint{3.743763in}{3.264232in}}%
\pgfpathlineto{\pgfqpoint{3.784089in}{3.262497in}}%
\pgfpathlineto{\pgfqpoint{3.824343in}{3.260753in}}%
\pgfpathlineto{\pgfqpoint{3.864526in}{3.259000in}}%
\pgfpathlineto{\pgfqpoint{3.904640in}{3.257239in}}%
\pgfpathlineto{\pgfqpoint{3.944684in}{3.255470in}}%
\pgfpathlineto{\pgfqpoint{3.984657in}{3.253692in}}%
\pgfpathlineto{\pgfqpoint{4.024559in}{3.251906in}}%
\pgfpathlineto{\pgfqpoint{4.064388in}{3.250113in}}%
\pgfpathlineto{\pgfqpoint{4.104146in}{3.248311in}}%
\pgfpathlineto{\pgfqpoint{4.143832in}{3.246503in}}%
\pgfpathlineto{\pgfqpoint{4.183446in}{3.244688in}}%
\pgfpathlineto{\pgfqpoint{4.222988in}{3.242867in}}%
\pgfpathlineto{\pgfqpoint{4.262458in}{3.241041in}}%
\pgfpathlineto{\pgfqpoint{4.301855in}{3.239209in}}%
\pgfpathlineto{\pgfqpoint{4.341179in}{3.237372in}}%
\pgfpathlineto{\pgfqpoint{4.380429in}{3.235529in}}%
\pgfpathlineto{\pgfqpoint{4.419609in}{3.233680in}}%
\pgfpathlineto{\pgfqpoint{4.458719in}{3.231825in}}%
\pgfpathlineto{\pgfqpoint{4.497761in}{3.229966in}}%
\pgfpathlineto{\pgfqpoint{4.536737in}{3.228102in}}%
\pgfpathlineto{\pgfqpoint{4.575646in}{3.226233in}}%
\pgfpathlineto{\pgfqpoint{4.614490in}{3.224362in}}%
\pgfpathlineto{\pgfqpoint{4.653269in}{3.222485in}}%
\pgfpathlineto{\pgfqpoint{4.691985in}{3.220604in}}%
\pgfpathlineto{\pgfqpoint{4.730638in}{3.218718in}}%
\pgfpathlineto{\pgfqpoint{4.769231in}{3.216829in}}%
\pgfpathlineto{\pgfqpoint{4.807764in}{3.214937in}}%
\pgfpathlineto{\pgfqpoint{4.846237in}{3.213041in}}%
\pgfpathlineto{\pgfqpoint{4.884649in}{3.211142in}}%
\pgfpathlineto{\pgfqpoint{4.923000in}{3.209240in}}%
\pgfpathlineto{\pgfqpoint{4.961289in}{3.207335in}}%
\pgfpathlineto{\pgfqpoint{4.999516in}{3.205426in}}%
\pgfpathlineto{\pgfqpoint{5.037682in}{3.203515in}}%
\pgfpathlineto{\pgfqpoint{5.075786in}{3.201602in}}%
\pgfpathlineto{\pgfqpoint{5.113829in}{3.199686in}}%
\pgfusepath{stroke}%
\end{pgfscope}%
\begin{pgfscope}%
\pgfpathrectangle{\pgfqpoint{0.675193in}{0.526079in}}{\pgfqpoint{4.650000in}{3.020000in}}%
\pgfusepath{clip}%
\pgfsetrectcap%
\pgfsetroundjoin%
\pgfsetlinewidth{1.505625pt}%
\definecolor{currentstroke}{rgb}{0.000000,0.000000,1.000000}%
\pgfsetstrokecolor{currentstroke}%
\pgfsetstrokeopacity{0.750000}%
\pgfsetdash{}{0pt}%
\pgfpathmoveto{\pgfqpoint{1.266608in}{2.141431in}}%
\pgfpathlineto{\pgfqpoint{1.315801in}{2.153201in}}%
\pgfpathlineto{\pgfqpoint{1.364866in}{2.164948in}}%
\pgfpathlineto{\pgfqpoint{1.413797in}{2.176117in}}%
\pgfpathlineto{\pgfqpoint{1.462586in}{2.186858in}}%
\pgfpathlineto{\pgfqpoint{1.511228in}{2.197264in}}%
\pgfpathlineto{\pgfqpoint{1.559715in}{2.207406in}}%
\pgfpathlineto{\pgfqpoint{1.608040in}{2.217333in}}%
\pgfpathlineto{\pgfqpoint{1.656198in}{2.227082in}}%
\pgfpathlineto{\pgfqpoint{1.704181in}{2.236681in}}%
\pgfpathlineto{\pgfqpoint{1.751983in}{2.246150in}}%
\pgfpathlineto{\pgfqpoint{1.799596in}{2.255509in}}%
\pgfpathlineto{\pgfqpoint{1.847015in}{2.264769in}}%
\pgfpathlineto{\pgfqpoint{1.894222in}{2.273940in}}%
\pgfpathlineto{\pgfqpoint{1.941215in}{2.283027in}}%
\pgfpathlineto{\pgfqpoint{1.987986in}{2.292036in}}%
\pgfpathlineto{\pgfqpoint{2.034532in}{2.300971in}}%
\pgfpathlineto{\pgfqpoint{2.080847in}{2.309834in}}%
\pgfpathlineto{\pgfqpoint{2.126928in}{2.318627in}}%
\pgfpathlineto{\pgfqpoint{2.172772in}{2.327352in}}%
\pgfpathlineto{\pgfqpoint{2.218380in}{2.336007in}}%
\pgfpathlineto{\pgfqpoint{2.263751in}{2.344595in}}%
\pgfpathlineto{\pgfqpoint{2.308887in}{2.353118in}}%
\pgfpathlineto{\pgfqpoint{2.353790in}{2.361573in}}%
\pgfpathlineto{\pgfqpoint{2.398465in}{2.369959in}}%
\pgfpathlineto{\pgfqpoint{2.442915in}{2.378279in}}%
\pgfpathlineto{\pgfqpoint{2.487147in}{2.386530in}}%
\pgfpathlineto{\pgfqpoint{2.531166in}{2.394711in}}%
\pgfpathlineto{\pgfqpoint{2.574980in}{2.402825in}}%
\pgfpathlineto{\pgfqpoint{2.618594in}{2.410868in}}%
\pgfpathlineto{\pgfqpoint{2.662017in}{2.418842in}}%
\pgfpathlineto{\pgfqpoint{2.705255in}{2.426746in}}%
\pgfpathlineto{\pgfqpoint{2.748315in}{2.434579in}}%
\pgfpathlineto{\pgfqpoint{2.791204in}{2.442341in}}%
\pgfpathlineto{\pgfqpoint{2.833930in}{2.450033in}}%
\pgfpathlineto{\pgfqpoint{2.876500in}{2.457654in}}%
\pgfpathlineto{\pgfqpoint{2.918921in}{2.465203in}}%
\pgfpathlineto{\pgfqpoint{2.961199in}{2.472682in}}%
\pgfpathlineto{\pgfqpoint{3.003342in}{2.480090in}}%
\pgfpathlineto{\pgfqpoint{3.045355in}{2.487427in}}%
\pgfpathlineto{\pgfqpoint{3.087242in}{2.494692in}}%
\pgfpathlineto{\pgfqpoint{3.129009in}{2.501885in}}%
\pgfpathlineto{\pgfqpoint{3.170661in}{2.509005in}}%
\pgfpathlineto{\pgfqpoint{3.212203in}{2.516057in}}%
\pgfpathlineto{\pgfqpoint{3.253639in}{2.523038in}}%
\pgfpathlineto{\pgfqpoint{3.294973in}{2.529949in}}%
\pgfpathlineto{\pgfqpoint{3.336209in}{2.536790in}}%
\pgfpathlineto{\pgfqpoint{3.377348in}{2.543561in}}%
\pgfpathlineto{\pgfqpoint{3.418395in}{2.550264in}}%
\pgfpathlineto{\pgfqpoint{3.459351in}{2.556897in}}%
\pgfpathlineto{\pgfqpoint{3.500221in}{2.563462in}}%
\pgfpathlineto{\pgfqpoint{3.541007in}{2.569959in}}%
\pgfpathlineto{\pgfqpoint{3.581713in}{2.576389in}}%
\pgfpathlineto{\pgfqpoint{3.622340in}{2.582752in}}%
\pgfpathlineto{\pgfqpoint{3.662890in}{2.589048in}}%
\pgfpathlineto{\pgfqpoint{3.703364in}{2.595278in}}%
\pgfpathlineto{\pgfqpoint{3.743763in}{2.601443in}}%
\pgfpathlineto{\pgfqpoint{3.784089in}{2.607543in}}%
\pgfpathlineto{\pgfqpoint{3.824343in}{2.613578in}}%
\pgfpathlineto{\pgfqpoint{3.864526in}{2.619551in}}%
\pgfpathlineto{\pgfqpoint{3.904640in}{2.625460in}}%
\pgfpathlineto{\pgfqpoint{3.944684in}{2.631307in}}%
\pgfpathlineto{\pgfqpoint{3.984657in}{2.637091in}}%
\pgfpathlineto{\pgfqpoint{4.024559in}{2.642814in}}%
\pgfpathlineto{\pgfqpoint{4.064388in}{2.648475in}}%
\pgfpathlineto{\pgfqpoint{4.104146in}{2.654075in}}%
\pgfpathlineto{\pgfqpoint{4.143832in}{2.659616in}}%
\pgfpathlineto{\pgfqpoint{4.183446in}{2.665098in}}%
\pgfpathlineto{\pgfqpoint{4.222988in}{2.670522in}}%
\pgfpathlineto{\pgfqpoint{4.262458in}{2.675889in}}%
\pgfpathlineto{\pgfqpoint{4.301855in}{2.681200in}}%
\pgfpathlineto{\pgfqpoint{4.341179in}{2.686455in}}%
\pgfpathlineto{\pgfqpoint{4.380429in}{2.691654in}}%
\pgfpathlineto{\pgfqpoint{4.419609in}{2.696797in}}%
\pgfpathlineto{\pgfqpoint{4.458719in}{2.701884in}}%
\pgfpathlineto{\pgfqpoint{4.497761in}{2.706918in}}%
\pgfpathlineto{\pgfqpoint{4.536737in}{2.711899in}}%
\pgfpathlineto{\pgfqpoint{4.575646in}{2.716827in}}%
\pgfpathlineto{\pgfqpoint{4.614490in}{2.721705in}}%
\pgfpathlineto{\pgfqpoint{4.653269in}{2.726530in}}%
\pgfpathlineto{\pgfqpoint{4.691985in}{2.731304in}}%
\pgfpathlineto{\pgfqpoint{4.730638in}{2.736027in}}%
\pgfpathlineto{\pgfqpoint{4.769231in}{2.740701in}}%
\pgfpathlineto{\pgfqpoint{4.807764in}{2.745325in}}%
\pgfpathlineto{\pgfqpoint{4.846237in}{2.749902in}}%
\pgfpathlineto{\pgfqpoint{4.884649in}{2.754430in}}%
\pgfpathlineto{\pgfqpoint{4.923000in}{2.758911in}}%
\pgfpathlineto{\pgfqpoint{4.961289in}{2.763345in}}%
\pgfpathlineto{\pgfqpoint{4.999516in}{2.767733in}}%
\pgfpathlineto{\pgfqpoint{5.037682in}{2.772075in}}%
\pgfpathlineto{\pgfqpoint{5.075786in}{2.776372in}}%
\pgfpathlineto{\pgfqpoint{5.113829in}{2.780625in}}%
\pgfusepath{stroke}%
\end{pgfscope}%
\begin{pgfscope}%
\pgfpathrectangle{\pgfqpoint{0.675193in}{0.526079in}}{\pgfqpoint{4.650000in}{3.020000in}}%
\pgfusepath{clip}%
\pgfsetrectcap%
\pgfsetroundjoin%
\pgfsetlinewidth{1.505625pt}%
\definecolor{currentstroke}{rgb}{0.000000,0.750000,0.750000}%
\pgfsetstrokecolor{currentstroke}%
\pgfsetstrokeopacity{0.750000}%
\pgfsetdash{}{0pt}%
\pgfpathmoveto{\pgfqpoint{1.266608in}{2.003850in}}%
\pgfpathlineto{\pgfqpoint{1.315801in}{1.941998in}}%
\pgfpathlineto{\pgfqpoint{1.364866in}{1.885667in}}%
\pgfpathlineto{\pgfqpoint{1.413797in}{1.833564in}}%
\pgfpathlineto{\pgfqpoint{1.462586in}{1.785238in}}%
\pgfpathlineto{\pgfqpoint{1.511228in}{1.740285in}}%
\pgfpathlineto{\pgfqpoint{1.559715in}{1.698362in}}%
\pgfpathlineto{\pgfqpoint{1.608040in}{1.659167in}}%
\pgfpathlineto{\pgfqpoint{1.656198in}{1.622438in}}%
\pgfpathlineto{\pgfqpoint{1.704181in}{1.587948in}}%
\pgfpathlineto{\pgfqpoint{1.751983in}{1.555497in}}%
\pgfpathlineto{\pgfqpoint{1.799596in}{1.524910in}}%
\pgfpathlineto{\pgfqpoint{1.847015in}{1.496031in}}%
\pgfpathlineto{\pgfqpoint{1.894222in}{1.468721in}}%
\pgfpathlineto{\pgfqpoint{1.941215in}{1.442854in}}%
\pgfpathlineto{\pgfqpoint{1.987986in}{1.418319in}}%
\pgfpathlineto{\pgfqpoint{2.034532in}{1.395016in}}%
\pgfpathlineto{\pgfqpoint{2.080847in}{1.372854in}}%
\pgfpathlineto{\pgfqpoint{2.126928in}{1.351751in}}%
\pgfpathlineto{\pgfqpoint{2.172772in}{1.331631in}}%
\pgfpathlineto{\pgfqpoint{2.218380in}{1.312428in}}%
\pgfpathlineto{\pgfqpoint{2.263751in}{1.294078in}}%
\pgfpathlineto{\pgfqpoint{2.308887in}{1.276530in}}%
\pgfpathlineto{\pgfqpoint{2.353790in}{1.259726in}}%
\pgfpathlineto{\pgfqpoint{2.398465in}{1.243621in}}%
\pgfpathlineto{\pgfqpoint{2.442915in}{1.228171in}}%
\pgfpathlineto{\pgfqpoint{2.487147in}{1.213336in}}%
\pgfpathlineto{\pgfqpoint{2.531166in}{1.199077in}}%
\pgfpathlineto{\pgfqpoint{2.574980in}{1.185362in}}%
\pgfpathlineto{\pgfqpoint{2.618594in}{1.172157in}}%
\pgfpathlineto{\pgfqpoint{2.662017in}{1.159434in}}%
\pgfpathlineto{\pgfqpoint{2.705255in}{1.147165in}}%
\pgfpathlineto{\pgfqpoint{2.748315in}{1.135324in}}%
\pgfpathlineto{\pgfqpoint{2.791204in}{1.123889in}}%
\pgfpathlineto{\pgfqpoint{2.833930in}{1.112837in}}%
\pgfpathlineto{\pgfqpoint{2.876500in}{1.102146in}}%
\pgfpathlineto{\pgfqpoint{2.918921in}{1.091799in}}%
\pgfpathlineto{\pgfqpoint{2.961199in}{1.081777in}}%
\pgfpathlineto{\pgfqpoint{3.003342in}{1.072064in}}%
\pgfpathlineto{\pgfqpoint{3.045355in}{1.062645in}}%
\pgfpathlineto{\pgfqpoint{3.087242in}{1.053502in}}%
\pgfpathlineto{\pgfqpoint{3.129009in}{1.044621in}}%
\pgfpathlineto{\pgfqpoint{3.170661in}{1.035990in}}%
\pgfpathlineto{\pgfqpoint{3.212203in}{1.027600in}}%
\pgfpathlineto{\pgfqpoint{3.253639in}{1.019437in}}%
\pgfpathlineto{\pgfqpoint{3.294973in}{1.011490in}}%
\pgfpathlineto{\pgfqpoint{3.336209in}{1.003750in}}%
\pgfpathlineto{\pgfqpoint{3.377348in}{0.996207in}}%
\pgfpathlineto{\pgfqpoint{3.418395in}{0.988852in}}%
\pgfpathlineto{\pgfqpoint{3.459351in}{0.981675in}}%
\pgfpathlineto{\pgfqpoint{3.500221in}{0.974670in}}%
\pgfpathlineto{\pgfqpoint{3.541007in}{0.967829in}}%
\pgfpathlineto{\pgfqpoint{3.581713in}{0.961144in}}%
\pgfpathlineto{\pgfqpoint{3.622340in}{0.954609in}}%
\pgfpathlineto{\pgfqpoint{3.662890in}{0.948216in}}%
\pgfpathlineto{\pgfqpoint{3.703364in}{0.941961in}}%
\pgfpathlineto{\pgfqpoint{3.743763in}{0.935838in}}%
\pgfpathlineto{\pgfqpoint{3.784089in}{0.929840in}}%
\pgfpathlineto{\pgfqpoint{3.824343in}{0.923963in}}%
\pgfpathlineto{\pgfqpoint{3.864526in}{0.918202in}}%
\pgfpathlineto{\pgfqpoint{3.904640in}{0.912553in}}%
\pgfpathlineto{\pgfqpoint{3.944684in}{0.907010in}}%
\pgfpathlineto{\pgfqpoint{3.984657in}{0.901569in}}%
\pgfpathlineto{\pgfqpoint{4.024559in}{0.896227in}}%
\pgfpathlineto{\pgfqpoint{4.064388in}{0.890978in}}%
\pgfpathlineto{\pgfqpoint{4.104146in}{0.885820in}}%
\pgfpathlineto{\pgfqpoint{4.143832in}{0.880750in}}%
\pgfpathlineto{\pgfqpoint{4.183446in}{0.875764in}}%
\pgfpathlineto{\pgfqpoint{4.222988in}{0.870860in}}%
\pgfpathlineto{\pgfqpoint{4.262458in}{0.866034in}}%
\pgfpathlineto{\pgfqpoint{4.301855in}{0.861285in}}%
\pgfpathlineto{\pgfqpoint{4.341179in}{0.856609in}}%
\pgfpathlineto{\pgfqpoint{4.380429in}{0.852003in}}%
\pgfpathlineto{\pgfqpoint{4.419609in}{0.847463in}}%
\pgfpathlineto{\pgfqpoint{4.458719in}{0.842989in}}%
\pgfpathlineto{\pgfqpoint{4.497761in}{0.838577in}}%
\pgfpathlineto{\pgfqpoint{4.536737in}{0.834226in}}%
\pgfpathlineto{\pgfqpoint{4.575646in}{0.829935in}}%
\pgfpathlineto{\pgfqpoint{4.614490in}{0.825701in}}%
\pgfpathlineto{\pgfqpoint{4.653269in}{0.821522in}}%
\pgfpathlineto{\pgfqpoint{4.691985in}{0.817396in}}%
\pgfpathlineto{\pgfqpoint{4.730638in}{0.813320in}}%
\pgfpathlineto{\pgfqpoint{4.769231in}{0.809295in}}%
\pgfpathlineto{\pgfqpoint{4.807764in}{0.805318in}}%
\pgfpathlineto{\pgfqpoint{4.846237in}{0.801387in}}%
\pgfpathlineto{\pgfqpoint{4.884649in}{0.797503in}}%
\pgfpathlineto{\pgfqpoint{4.923000in}{0.793661in}}%
\pgfpathlineto{\pgfqpoint{4.961289in}{0.789862in}}%
\pgfpathlineto{\pgfqpoint{4.999516in}{0.786105in}}%
\pgfpathlineto{\pgfqpoint{5.037682in}{0.782387in}}%
\pgfpathlineto{\pgfqpoint{5.075786in}{0.778708in}}%
\pgfpathlineto{\pgfqpoint{5.113829in}{0.775067in}}%
\pgfusepath{stroke}%
\end{pgfscope}%
\begin{pgfscope}%
\pgfsetrectcap%
\pgfsetmiterjoin%
\pgfsetlinewidth{0.803000pt}%
\definecolor{currentstroke}{rgb}{0.501961,0.501961,0.501961}%
\pgfsetstrokecolor{currentstroke}%
\pgfsetdash{}{0pt}%
\pgfpathmoveto{\pgfqpoint{0.675193in}{0.526079in}}%
\pgfpathlineto{\pgfqpoint{0.675193in}{3.546079in}}%
\pgfusepath{stroke}%
\end{pgfscope}%
\begin{pgfscope}%
\pgfsetrectcap%
\pgfsetmiterjoin%
\pgfsetlinewidth{0.803000pt}%
\definecolor{currentstroke}{rgb}{0.501961,0.501961,0.501961}%
\pgfsetstrokecolor{currentstroke}%
\pgfsetdash{}{0pt}%
\pgfpathmoveto{\pgfqpoint{5.325193in}{0.526079in}}%
\pgfpathlineto{\pgfqpoint{5.325193in}{3.546079in}}%
\pgfusepath{stroke}%
\end{pgfscope}%
\begin{pgfscope}%
\pgfsetrectcap%
\pgfsetmiterjoin%
\pgfsetlinewidth{0.803000pt}%
\definecolor{currentstroke}{rgb}{0.501961,0.501961,0.501961}%
\pgfsetstrokecolor{currentstroke}%
\pgfsetdash{}{0pt}%
\pgfpathmoveto{\pgfqpoint{0.675193in}{0.526079in}}%
\pgfpathlineto{\pgfqpoint{5.325193in}{0.526079in}}%
\pgfusepath{stroke}%
\end{pgfscope}%
\begin{pgfscope}%
\pgfsetrectcap%
\pgfsetmiterjoin%
\pgfsetlinewidth{0.803000pt}%
\definecolor{currentstroke}{rgb}{0.501961,0.501961,0.501961}%
\pgfsetstrokecolor{currentstroke}%
\pgfsetdash{}{0pt}%
\pgfpathmoveto{\pgfqpoint{0.675193in}{3.546079in}}%
\pgfpathlineto{\pgfqpoint{5.325193in}{3.546079in}}%
\pgfusepath{stroke}%
\end{pgfscope}%
\end{pgfpicture}%
\makeatother%
\endgroup%
}
    \caption{Simulated forces acting on drops up to the first apoapse as determined by the MLE solutions to Equation \ref{gov_eqn_subs} for all drops in the dataset. The (\protect\redline) lines denote the Coulomb force, the (\protect\blueline) lines the drag force, and the (\protect\cyanline) lines the foce due to image charge respectively. The forces are scaled by the drop inertia.\label{fig:forces}}
\end{figure}

\subsection{Drop Impacts}
\sout{In 1-$g_0$, for Weber numbers $\mathbb{W}\mbox{e} \equiv \rho U^2 R_d/\gamma >$ 0.4, impact rebound behavior on a superhydrophobic surface can be strongly influenced by damping from contact line hysteresis. For the low Bond and Ohnesorge numbers occurring in free-fall, the drop impact dynamics may additionally include electrohydrodynamic surface wettability effects. To date there has been little work in general on drop impacts outside of two regimes: (1) very low $\mathbb{R}\mbox{e}$ viscous drop spreading driven by capillary forces at the contact line and (2) impacts at `high' Weber numbers. Models for dynamic contact lines in general remain controversial, even for ordinary spreading of liquids, despite decades of work in the area.} 

\sout{If we naively neglect the contact line dynamics, the dimensionless groups for isothermal drop impacts are the Ohnesorge number $\mathbb{O}\mbox{h} \equiv \mu/\sqrt{\rho \gamma R_d}$, the Weber number $\mathbb{W}\mbox{e}$, and Bond number $\mathbb{B}\mbox{o}$. The Weber number scales the driving force of drop spreading. In the well-studied case of high $\mathbb{W}\mbox{e}$ impacts the drop bulk is driven radially outward by the impact induced pressure gradient, whereas in the case of small $\mathbb{W}\mbox{e}$ wetting impacts the liquid is driven outwards by capillary force. The Ohnesorge number, by contrast, is a measure of viscous effects in an inertial-capillary flow and scales the force that resists spreading. Previous empirical work for low $\mathbb{O}\mbox{h}$ and low $\mathbb{W}\mbox{e}$ impacts includes that of Schiaffino and Sonin \cite{schiaffino_molten_1997} with wetting and non-wetting impacts of molten metal drops on cold surfaces. Mol\'{a}\u{c}ek and Bush \cite{molacek_quasi-static_2012}, Gopinath and Koch \cite{gopinath_collision_2002}, and Okamura \cite{okumura_water_2003} have all developed analytical models of drop impacts at low $\mathbb{W}\mbox{e}$ and low $\mathbb{B}\mbox{o}$. These works show increasing dimensionless contact time $\tau$ as Weber number decreases, in opposition to the results of Richard and Qu\'{e}r\'{e} \cite{richard_surface_2002} which show experimentally that the dimensionless contact time is approximately a constant $\tau \approx 2.6(\rho R^3_d/\sigma)^{1/2}$ with respect to $\mathbb{W}\mbox{e}$ at large $\mathbb{W}\mbox{e}$. Mol\'{a}\u{c}ek's work also shows that impact coefficients of restitution $C_r$ depend non-linearly on $\mathbb{B}\mbox{o}$. To date little experimental work has been performed on low $\mathbb{B}\mbox{o}$ impacts as well. An exception is provided by Duvivier \emph{et al.} \cite{duvivier_drop_2012} who studied regimes of aqueous ferrofluid drop impacts on superhydrophobic substrates under the influence of an external magnetic field. In this case the magnetic body force acts as an ersatz gravity.}

\sout{When the electro-drop bounce occurs, if the drop has enough time to return a non-wetting impact occurs on the charged substrate usually followed by rebound. We observe average drop impact $\mathbb{O}\mbox{h} \approx 2.18 \pm 0.36$ and  $\mathbb{W}\mbox{e} \approx 0.28 \pm 0.22$. Thus impact velocity plays little role in the spreading dynamics of the bounces and viscous effects are important but do not dominate inertia. Notably we observe underdamped oscillations of drop interfaces during impact. Access to such relatively low $\mathbb{O}\mbox{h}$ and low $\mathbb{W}\mbox{e}$ drop impacts enabled by the low-gravity environment raises an interesting possibilities for new work on the basic science of drop impacts. This work also intersects the burgeoning field which spans the intersection of electrowetting on patterned surfaces and drop impacts. Various authors \cite{bartolo_bouncing_2006, reyssat_bouncing_2006} have suggested that a wetting transition called Fakir impalement can occur during impacts on patterned hydrophobic surfaces if a certain critical pressure $p_c \sim \gamma h/l$ is exceeded, where $h/l$ is the pillar aspect ratio of the patterned surface. This pressure can result from fluid inertia in a high $\mathbb{W}\mbox{e}$ impact regime, or can result from an electrostatic pressure due to an external electric field. The latter case is responsible for the irreversiblity that notoriously plagues static EWOD experiments. There is hope that additional work in this area could produce engineered surfaces that are tuneably wetting under drop impacts by leveraging electrostatic forces.}
 
\sout{Our preliminary results showing the influence of $\mathbb{W}\mbox{e}$ and electrostatic Bond number $\mathbb{B}\mbox{o}_e \equiv \epsilon E_0^2 R_d/\gamma $ on drop impact dimensionless contact time $\tau$ and coefficient of restitution $C_r$ are shown in Figures \ref{fig:contact} and \ref{fig:restitution}.}

\begin{figure}[htb]
    \centering
    \resizebox{0.5\textwidth}{!}{%% Creator: Matplotlib, PGF backend
%%
%% To include the figure in your LaTeX document, write
%%   \input{<filename>.pgf}
%%
%% Make sure the required packages are loaded in your preamble
%%   \usepackage{pgf}
%%
%% Figures using additional raster images can only be included by \input if
%% they are in the same directory as the main LaTeX file. For loading figures
%% from other directories you can use the `import` package
%%   \usepackage{import}
%% and then include the figures with
%%   \import{<path to file>}{<filename>.pgf}
%%
%% Matplotlib used the following preamble
%%   \usepackage{fontspec}
%%   \setmainfont{DejaVuSerif.ttf}[Path=/home/erin/anaconda3/lib/python3.6/site-packages/matplotlib/mpl-data/fonts/ttf/]
%%   \setsansfont{DejaVuSans.ttf}[Path=/home/erin/anaconda3/lib/python3.6/site-packages/matplotlib/mpl-data/fonts/ttf/]
%%   \setmonofont{DejaVuSansMono.ttf}[Path=/home/erin/anaconda3/lib/python3.6/site-packages/matplotlib/mpl-data/fonts/ttf/]
%%
\begingroup%
\makeatletter%
\begin{pgfpicture}%
\pgfpathrectangle{\pgfpointorigin}{\pgfqpoint{5.637669in}{3.790756in}}%
\pgfusepath{use as bounding box, clip}%
\begin{pgfscope}%
\pgfsetbuttcap%
\pgfsetmiterjoin%
\definecolor{currentfill}{rgb}{1.000000,1.000000,1.000000}%
\pgfsetfillcolor{currentfill}%
\pgfsetlinewidth{0.000000pt}%
\definecolor{currentstroke}{rgb}{1.000000,1.000000,1.000000}%
\pgfsetstrokecolor{currentstroke}%
\pgfsetdash{}{0pt}%
\pgfpathmoveto{\pgfqpoint{-0.000000in}{0.000000in}}%
\pgfpathlineto{\pgfqpoint{5.637669in}{0.000000in}}%
\pgfpathlineto{\pgfqpoint{5.637669in}{3.790756in}}%
\pgfpathlineto{\pgfqpoint{-0.000000in}{3.790756in}}%
\pgfpathclose%
\pgfusepath{fill}%
\end{pgfscope}%
\begin{pgfscope}%
\pgfsetbuttcap%
\pgfsetmiterjoin%
\definecolor{currentfill}{rgb}{1.000000,1.000000,1.000000}%
\pgfsetfillcolor{currentfill}%
\pgfsetlinewidth{0.000000pt}%
\definecolor{currentstroke}{rgb}{0.000000,0.000000,0.000000}%
\pgfsetstrokecolor{currentstroke}%
\pgfsetstrokeopacity{0.000000}%
\pgfsetdash{}{0pt}%
\pgfpathmoveto{\pgfqpoint{0.557699in}{0.629134in}}%
\pgfpathlineto{\pgfqpoint{4.277699in}{0.629134in}}%
\pgfpathlineto{\pgfqpoint{4.277699in}{3.649134in}}%
\pgfpathlineto{\pgfqpoint{0.557699in}{3.649134in}}%
\pgfpathclose%
\pgfusepath{fill}%
\end{pgfscope}%
\begin{pgfscope}%
\pgfpathrectangle{\pgfqpoint{0.557699in}{0.629134in}}{\pgfqpoint{3.720000in}{3.020000in}}%
\pgfusepath{clip}%
\pgfsetbuttcap%
\pgfsetroundjoin%
\definecolor{currentfill}{rgb}{1.000000,0.626924,0.332355}%
\pgfsetfillcolor{currentfill}%
\pgfsetlinewidth{1.003750pt}%
\definecolor{currentstroke}{rgb}{1.000000,0.626924,0.332355}%
\pgfsetstrokecolor{currentstroke}%
\pgfsetdash{}{0pt}%
\pgfpathmoveto{\pgfqpoint{2.668907in}{1.309675in}}%
\pgfpathcurveto{\pgfqpoint{2.679957in}{1.309675in}}{\pgfqpoint{2.690556in}{1.314066in}}{\pgfqpoint{2.698370in}{1.321879in}}%
\pgfpathcurveto{\pgfqpoint{2.706183in}{1.329693in}}{\pgfqpoint{2.710574in}{1.340292in}}{\pgfqpoint{2.710574in}{1.351342in}}%
\pgfpathcurveto{\pgfqpoint{2.710574in}{1.362392in}}{\pgfqpoint{2.706183in}{1.372991in}}{\pgfqpoint{2.698370in}{1.380805in}}%
\pgfpathcurveto{\pgfqpoint{2.690556in}{1.388618in}}{\pgfqpoint{2.679957in}{1.393009in}}{\pgfqpoint{2.668907in}{1.393009in}}%
\pgfpathcurveto{\pgfqpoint{2.657857in}{1.393009in}}{\pgfqpoint{2.647258in}{1.388618in}}{\pgfqpoint{2.639444in}{1.380805in}}%
\pgfpathcurveto{\pgfqpoint{2.631631in}{1.372991in}}{\pgfqpoint{2.627240in}{1.362392in}}{\pgfqpoint{2.627240in}{1.351342in}}%
\pgfpathcurveto{\pgfqpoint{2.627240in}{1.340292in}}{\pgfqpoint{2.631631in}{1.329693in}}{\pgfqpoint{2.639444in}{1.321879in}}%
\pgfpathcurveto{\pgfqpoint{2.647258in}{1.314066in}}{\pgfqpoint{2.657857in}{1.309675in}}{\pgfqpoint{2.668907in}{1.309675in}}%
\pgfpathclose%
\pgfusepath{stroke,fill}%
\end{pgfscope}%
\begin{pgfscope}%
\pgfpathrectangle{\pgfqpoint{0.557699in}{0.629134in}}{\pgfqpoint{3.720000in}{3.020000in}}%
\pgfusepath{clip}%
\pgfsetbuttcap%
\pgfsetroundjoin%
\definecolor{currentfill}{rgb}{0.335294,0.255843,0.991645}%
\pgfsetfillcolor{currentfill}%
\pgfsetlinewidth{1.003750pt}%
\definecolor{currentstroke}{rgb}{0.335294,0.255843,0.991645}%
\pgfsetstrokecolor{currentstroke}%
\pgfsetdash{}{0pt}%
\pgfpathmoveto{\pgfqpoint{2.713932in}{1.520210in}}%
\pgfpathcurveto{\pgfqpoint{2.724982in}{1.520210in}}{\pgfqpoint{2.735581in}{1.524600in}}{\pgfqpoint{2.743395in}{1.532414in}}%
\pgfpathcurveto{\pgfqpoint{2.751208in}{1.540227in}}{\pgfqpoint{2.755598in}{1.550827in}}{\pgfqpoint{2.755598in}{1.561877in}}%
\pgfpathcurveto{\pgfqpoint{2.755598in}{1.572927in}}{\pgfqpoint{2.751208in}{1.583526in}}{\pgfqpoint{2.743395in}{1.591339in}}%
\pgfpathcurveto{\pgfqpoint{2.735581in}{1.599153in}}{\pgfqpoint{2.724982in}{1.603543in}}{\pgfqpoint{2.713932in}{1.603543in}}%
\pgfpathcurveto{\pgfqpoint{2.702882in}{1.603543in}}{\pgfqpoint{2.692283in}{1.599153in}}{\pgfqpoint{2.684469in}{1.591339in}}%
\pgfpathcurveto{\pgfqpoint{2.676655in}{1.583526in}}{\pgfqpoint{2.672265in}{1.572927in}}{\pgfqpoint{2.672265in}{1.561877in}}%
\pgfpathcurveto{\pgfqpoint{2.672265in}{1.550827in}}{\pgfqpoint{2.676655in}{1.540227in}}{\pgfqpoint{2.684469in}{1.532414in}}%
\pgfpathcurveto{\pgfqpoint{2.692283in}{1.524600in}}{\pgfqpoint{2.702882in}{1.520210in}}{\pgfqpoint{2.713932in}{1.520210in}}%
\pgfpathclose%
\pgfusepath{stroke,fill}%
\end{pgfscope}%
\begin{pgfscope}%
\pgfpathrectangle{\pgfqpoint{0.557699in}{0.629134in}}{\pgfqpoint{3.720000in}{3.020000in}}%
\pgfusepath{clip}%
\pgfsetbuttcap%
\pgfsetroundjoin%
\definecolor{currentfill}{rgb}{0.468627,0.049260,0.999696}%
\pgfsetfillcolor{currentfill}%
\pgfsetlinewidth{1.003750pt}%
\definecolor{currentstroke}{rgb}{0.468627,0.049260,0.999696}%
\pgfsetstrokecolor{currentstroke}%
\pgfsetdash{}{0pt}%
\pgfpathmoveto{\pgfqpoint{2.084667in}{2.094532in}}%
\pgfpathcurveto{\pgfqpoint{2.095717in}{2.094532in}}{\pgfqpoint{2.106317in}{2.098922in}}{\pgfqpoint{2.114130in}{2.106736in}}%
\pgfpathcurveto{\pgfqpoint{2.121944in}{2.114550in}}{\pgfqpoint{2.126334in}{2.125149in}}{\pgfqpoint{2.126334in}{2.136199in}}%
\pgfpathcurveto{\pgfqpoint{2.126334in}{2.147249in}}{\pgfqpoint{2.121944in}{2.157848in}}{\pgfqpoint{2.114130in}{2.165662in}}%
\pgfpathcurveto{\pgfqpoint{2.106317in}{2.173475in}}{\pgfqpoint{2.095717in}{2.177866in}}{\pgfqpoint{2.084667in}{2.177866in}}%
\pgfpathcurveto{\pgfqpoint{2.073617in}{2.177866in}}{\pgfqpoint{2.063018in}{2.173475in}}{\pgfqpoint{2.055205in}{2.165662in}}%
\pgfpathcurveto{\pgfqpoint{2.047391in}{2.157848in}}{\pgfqpoint{2.043001in}{2.147249in}}{\pgfqpoint{2.043001in}{2.136199in}}%
\pgfpathcurveto{\pgfqpoint{2.043001in}{2.125149in}}{\pgfqpoint{2.047391in}{2.114550in}}{\pgfqpoint{2.055205in}{2.106736in}}%
\pgfpathcurveto{\pgfqpoint{2.063018in}{2.098922in}}{\pgfqpoint{2.073617in}{2.094532in}}{\pgfqpoint{2.084667in}{2.094532in}}%
\pgfpathclose%
\pgfusepath{stroke,fill}%
\end{pgfscope}%
\begin{pgfscope}%
\pgfpathrectangle{\pgfqpoint{0.557699in}{0.629134in}}{\pgfqpoint{3.720000in}{3.020000in}}%
\pgfusepath{clip}%
\pgfsetbuttcap%
\pgfsetroundjoin%
\definecolor{currentfill}{rgb}{0.500000,0.000000,1.000000}%
\pgfsetfillcolor{currentfill}%
\pgfsetlinewidth{1.003750pt}%
\definecolor{currentstroke}{rgb}{0.500000,0.000000,1.000000}%
\pgfsetstrokecolor{currentstroke}%
\pgfsetdash{}{0pt}%
\pgfpathmoveto{\pgfqpoint{0.788018in}{2.105537in}}%
\pgfpathcurveto{\pgfqpoint{0.799068in}{2.105537in}}{\pgfqpoint{0.809667in}{2.109927in}}{\pgfqpoint{0.817480in}{2.117741in}}%
\pgfpathcurveto{\pgfqpoint{0.825294in}{2.125555in}}{\pgfqpoint{0.829684in}{2.136154in}}{\pgfqpoint{0.829684in}{2.147204in}}%
\pgfpathcurveto{\pgfqpoint{0.829684in}{2.158254in}}{\pgfqpoint{0.825294in}{2.168853in}}{\pgfqpoint{0.817480in}{2.176666in}}%
\pgfpathcurveto{\pgfqpoint{0.809667in}{2.184480in}}{\pgfqpoint{0.799068in}{2.188870in}}{\pgfqpoint{0.788018in}{2.188870in}}%
\pgfpathcurveto{\pgfqpoint{0.776967in}{2.188870in}}{\pgfqpoint{0.766368in}{2.184480in}}{\pgfqpoint{0.758555in}{2.176666in}}%
\pgfpathcurveto{\pgfqpoint{0.750741in}{2.168853in}}{\pgfqpoint{0.746351in}{2.158254in}}{\pgfqpoint{0.746351in}{2.147204in}}%
\pgfpathcurveto{\pgfqpoint{0.746351in}{2.136154in}}{\pgfqpoint{0.750741in}{2.125555in}}{\pgfqpoint{0.758555in}{2.117741in}}%
\pgfpathcurveto{\pgfqpoint{0.766368in}{2.109927in}}{\pgfqpoint{0.776967in}{2.105537in}}{\pgfqpoint{0.788018in}{2.105537in}}%
\pgfpathclose%
\pgfusepath{stroke,fill}%
\end{pgfscope}%
\begin{pgfscope}%
\pgfpathrectangle{\pgfqpoint{0.557699in}{0.629134in}}{\pgfqpoint{3.720000in}{3.020000in}}%
\pgfusepath{clip}%
\pgfsetbuttcap%
\pgfsetroundjoin%
\definecolor{currentfill}{rgb}{1.000000,0.000000,0.000000}%
\pgfsetfillcolor{currentfill}%
\pgfsetlinewidth{1.003750pt}%
\definecolor{currentstroke}{rgb}{1.000000,0.000000,0.000000}%
\pgfsetstrokecolor{currentstroke}%
\pgfsetdash{}{0pt}%
\pgfpathmoveto{\pgfqpoint{3.931832in}{0.873490in}}%
\pgfpathcurveto{\pgfqpoint{3.942882in}{0.873490in}}{\pgfqpoint{3.953481in}{0.877880in}}{\pgfqpoint{3.961295in}{0.885694in}}%
\pgfpathcurveto{\pgfqpoint{3.969108in}{0.893508in}}{\pgfqpoint{3.973499in}{0.904107in}}{\pgfqpoint{3.973499in}{0.915157in}}%
\pgfpathcurveto{\pgfqpoint{3.973499in}{0.926207in}}{\pgfqpoint{3.969108in}{0.936806in}}{\pgfqpoint{3.961295in}{0.944620in}}%
\pgfpathcurveto{\pgfqpoint{3.953481in}{0.952433in}}{\pgfqpoint{3.942882in}{0.956824in}}{\pgfqpoint{3.931832in}{0.956824in}}%
\pgfpathcurveto{\pgfqpoint{3.920782in}{0.956824in}}{\pgfqpoint{3.910183in}{0.952433in}}{\pgfqpoint{3.902369in}{0.944620in}}%
\pgfpathcurveto{\pgfqpoint{3.894556in}{0.936806in}}{\pgfqpoint{3.890165in}{0.926207in}}{\pgfqpoint{3.890165in}{0.915157in}}%
\pgfpathcurveto{\pgfqpoint{3.890165in}{0.904107in}}{\pgfqpoint{3.894556in}{0.893508in}}{\pgfqpoint{3.902369in}{0.885694in}}%
\pgfpathcurveto{\pgfqpoint{3.910183in}{0.877880in}}{\pgfqpoint{3.920782in}{0.873490in}}{\pgfqpoint{3.931832in}{0.873490in}}%
\pgfpathclose%
\pgfusepath{stroke,fill}%
\end{pgfscope}%
\begin{pgfscope}%
\pgfpathrectangle{\pgfqpoint{0.557699in}{0.629134in}}{\pgfqpoint{3.720000in}{3.020000in}}%
\pgfusepath{clip}%
\pgfsetbuttcap%
\pgfsetroundjoin%
\definecolor{currentfill}{rgb}{1.000000,0.000000,0.000000}%
\pgfsetfillcolor{currentfill}%
\pgfsetlinewidth{1.003750pt}%
\definecolor{currentstroke}{rgb}{1.000000,0.000000,0.000000}%
\pgfsetstrokecolor{currentstroke}%
\pgfsetdash{}{0pt}%
\pgfpathmoveto{\pgfqpoint{3.360356in}{0.873490in}}%
\pgfpathcurveto{\pgfqpoint{3.371406in}{0.873490in}}{\pgfqpoint{3.382005in}{0.877880in}}{\pgfqpoint{3.389818in}{0.885694in}}%
\pgfpathcurveto{\pgfqpoint{3.397632in}{0.893508in}}{\pgfqpoint{3.402022in}{0.904107in}}{\pgfqpoint{3.402022in}{0.915157in}}%
\pgfpathcurveto{\pgfqpoint{3.402022in}{0.926207in}}{\pgfqpoint{3.397632in}{0.936806in}}{\pgfqpoint{3.389818in}{0.944620in}}%
\pgfpathcurveto{\pgfqpoint{3.382005in}{0.952433in}}{\pgfqpoint{3.371406in}{0.956824in}}{\pgfqpoint{3.360356in}{0.956824in}}%
\pgfpathcurveto{\pgfqpoint{3.349306in}{0.956824in}}{\pgfqpoint{3.338707in}{0.952433in}}{\pgfqpoint{3.330893in}{0.944620in}}%
\pgfpathcurveto{\pgfqpoint{3.323079in}{0.936806in}}{\pgfqpoint{3.318689in}{0.926207in}}{\pgfqpoint{3.318689in}{0.915157in}}%
\pgfpathcurveto{\pgfqpoint{3.318689in}{0.904107in}}{\pgfqpoint{3.323079in}{0.893508in}}{\pgfqpoint{3.330893in}{0.885694in}}%
\pgfpathcurveto{\pgfqpoint{3.338707in}{0.877880in}}{\pgfqpoint{3.349306in}{0.873490in}}{\pgfqpoint{3.360356in}{0.873490in}}%
\pgfpathclose%
\pgfusepath{stroke,fill}%
\end{pgfscope}%
\begin{pgfscope}%
\pgfpathrectangle{\pgfqpoint{0.557699in}{0.629134in}}{\pgfqpoint{3.720000in}{3.020000in}}%
\pgfusepath{clip}%
\pgfsetbuttcap%
\pgfsetroundjoin%
\definecolor{currentfill}{rgb}{1.000000,0.036951,0.018479}%
\pgfsetfillcolor{currentfill}%
\pgfsetlinewidth{1.003750pt}%
\definecolor{currentstroke}{rgb}{1.000000,0.036951,0.018479}%
\pgfsetstrokecolor{currentstroke}%
\pgfsetdash{}{0pt}%
\pgfpathmoveto{\pgfqpoint{3.286283in}{1.201309in}}%
\pgfpathcurveto{\pgfqpoint{3.297333in}{1.201309in}}{\pgfqpoint{3.307932in}{1.205699in}}{\pgfqpoint{3.315745in}{1.213513in}}%
\pgfpathcurveto{\pgfqpoint{3.323559in}{1.221327in}}{\pgfqpoint{3.327949in}{1.231926in}}{\pgfqpoint{3.327949in}{1.242976in}}%
\pgfpathcurveto{\pgfqpoint{3.327949in}{1.254026in}}{\pgfqpoint{3.323559in}{1.264625in}}{\pgfqpoint{3.315745in}{1.272439in}}%
\pgfpathcurveto{\pgfqpoint{3.307932in}{1.280252in}}{\pgfqpoint{3.297333in}{1.284642in}}{\pgfqpoint{3.286283in}{1.284642in}}%
\pgfpathcurveto{\pgfqpoint{3.275232in}{1.284642in}}{\pgfqpoint{3.264633in}{1.280252in}}{\pgfqpoint{3.256820in}{1.272439in}}%
\pgfpathcurveto{\pgfqpoint{3.249006in}{1.264625in}}{\pgfqpoint{3.244616in}{1.254026in}}{\pgfqpoint{3.244616in}{1.242976in}}%
\pgfpathcurveto{\pgfqpoint{3.244616in}{1.231926in}}{\pgfqpoint{3.249006in}{1.221327in}}{\pgfqpoint{3.256820in}{1.213513in}}%
\pgfpathcurveto{\pgfqpoint{3.264633in}{1.205699in}}{\pgfqpoint{3.275232in}{1.201309in}}{\pgfqpoint{3.286283in}{1.201309in}}%
\pgfpathclose%
\pgfusepath{stroke,fill}%
\end{pgfscope}%
\begin{pgfscope}%
\pgfpathrectangle{\pgfqpoint{0.557699in}{0.629134in}}{\pgfqpoint{3.720000in}{3.020000in}}%
\pgfusepath{clip}%
\pgfsetbuttcap%
\pgfsetroundjoin%
\definecolor{currentfill}{rgb}{1.000000,0.036951,0.018479}%
\pgfsetfillcolor{currentfill}%
\pgfsetlinewidth{1.003750pt}%
\definecolor{currentstroke}{rgb}{1.000000,0.036951,0.018479}%
\pgfsetstrokecolor{currentstroke}%
\pgfsetdash{}{0pt}%
\pgfpathmoveto{\pgfqpoint{2.917715in}{1.499050in}}%
\pgfpathcurveto{\pgfqpoint{2.928765in}{1.499050in}}{\pgfqpoint{2.939364in}{1.503440in}}{\pgfqpoint{2.947178in}{1.511254in}}%
\pgfpathcurveto{\pgfqpoint{2.954991in}{1.519068in}}{\pgfqpoint{2.959382in}{1.529667in}}{\pgfqpoint{2.959382in}{1.540717in}}%
\pgfpathcurveto{\pgfqpoint{2.959382in}{1.551767in}}{\pgfqpoint{2.954991in}{1.562366in}}{\pgfqpoint{2.947178in}{1.570180in}}%
\pgfpathcurveto{\pgfqpoint{2.939364in}{1.577993in}}{\pgfqpoint{2.928765in}{1.582383in}}{\pgfqpoint{2.917715in}{1.582383in}}%
\pgfpathcurveto{\pgfqpoint{2.906665in}{1.582383in}}{\pgfqpoint{2.896066in}{1.577993in}}{\pgfqpoint{2.888252in}{1.570180in}}%
\pgfpathcurveto{\pgfqpoint{2.880439in}{1.562366in}}{\pgfqpoint{2.876048in}{1.551767in}}{\pgfqpoint{2.876048in}{1.540717in}}%
\pgfpathcurveto{\pgfqpoint{2.876048in}{1.529667in}}{\pgfqpoint{2.880439in}{1.519068in}}{\pgfqpoint{2.888252in}{1.511254in}}%
\pgfpathcurveto{\pgfqpoint{2.896066in}{1.503440in}}{\pgfqpoint{2.906665in}{1.499050in}}{\pgfqpoint{2.917715in}{1.499050in}}%
\pgfpathclose%
\pgfusepath{stroke,fill}%
\end{pgfscope}%
\begin{pgfscope}%
\pgfpathrectangle{\pgfqpoint{0.557699in}{0.629134in}}{\pgfqpoint{3.720000in}{3.020000in}}%
\pgfusepath{clip}%
\pgfsetbuttcap%
\pgfsetroundjoin%
\definecolor{currentfill}{rgb}{0.707843,0.947177,0.582791}%
\pgfsetfillcolor{currentfill}%
\pgfsetlinewidth{1.003750pt}%
\definecolor{currentstroke}{rgb}{0.707843,0.947177,0.582791}%
\pgfsetstrokecolor{currentstroke}%
\pgfsetdash{}{0pt}%
\pgfpathmoveto{\pgfqpoint{2.608964in}{1.305772in}}%
\pgfpathcurveto{\pgfqpoint{2.620014in}{1.305772in}}{\pgfqpoint{2.630613in}{1.310163in}}{\pgfqpoint{2.638427in}{1.317976in}}%
\pgfpathcurveto{\pgfqpoint{2.646241in}{1.325790in}}{\pgfqpoint{2.650631in}{1.336389in}}{\pgfqpoint{2.650631in}{1.347439in}}%
\pgfpathcurveto{\pgfqpoint{2.650631in}{1.358489in}}{\pgfqpoint{2.646241in}{1.369088in}}{\pgfqpoint{2.638427in}{1.376902in}}%
\pgfpathcurveto{\pgfqpoint{2.630613in}{1.384716in}}{\pgfqpoint{2.620014in}{1.389106in}}{\pgfqpoint{2.608964in}{1.389106in}}%
\pgfpathcurveto{\pgfqpoint{2.597914in}{1.389106in}}{\pgfqpoint{2.587315in}{1.384716in}}{\pgfqpoint{2.579501in}{1.376902in}}%
\pgfpathcurveto{\pgfqpoint{2.571688in}{1.369088in}}{\pgfqpoint{2.567298in}{1.358489in}}{\pgfqpoint{2.567298in}{1.347439in}}%
\pgfpathcurveto{\pgfqpoint{2.567298in}{1.336389in}}{\pgfqpoint{2.571688in}{1.325790in}}{\pgfqpoint{2.579501in}{1.317976in}}%
\pgfpathcurveto{\pgfqpoint{2.587315in}{1.310163in}}{\pgfqpoint{2.597914in}{1.305772in}}{\pgfqpoint{2.608964in}{1.305772in}}%
\pgfpathclose%
\pgfusepath{stroke,fill}%
\end{pgfscope}%
\begin{pgfscope}%
\pgfpathrectangle{\pgfqpoint{0.557699in}{0.629134in}}{\pgfqpoint{3.720000in}{3.020000in}}%
\pgfusepath{clip}%
\pgfsetbuttcap%
\pgfsetroundjoin%
\definecolor{currentfill}{rgb}{0.096078,0.805381,0.892401}%
\pgfsetfillcolor{currentfill}%
\pgfsetlinewidth{1.003750pt}%
\definecolor{currentstroke}{rgb}{0.096078,0.805381,0.892401}%
\pgfsetstrokecolor{currentstroke}%
\pgfsetdash{}{0pt}%
\pgfpathmoveto{\pgfqpoint{2.541691in}{1.520210in}}%
\pgfpathcurveto{\pgfqpoint{2.552741in}{1.520210in}}{\pgfqpoint{2.563340in}{1.524600in}}{\pgfqpoint{2.571154in}{1.532414in}}%
\pgfpathcurveto{\pgfqpoint{2.578967in}{1.540227in}}{\pgfqpoint{2.583358in}{1.550827in}}{\pgfqpoint{2.583358in}{1.561877in}}%
\pgfpathcurveto{\pgfqpoint{2.583358in}{1.572927in}}{\pgfqpoint{2.578967in}{1.583526in}}{\pgfqpoint{2.571154in}{1.591339in}}%
\pgfpathcurveto{\pgfqpoint{2.563340in}{1.599153in}}{\pgfqpoint{2.552741in}{1.603543in}}{\pgfqpoint{2.541691in}{1.603543in}}%
\pgfpathcurveto{\pgfqpoint{2.530641in}{1.603543in}}{\pgfqpoint{2.520042in}{1.599153in}}{\pgfqpoint{2.512228in}{1.591339in}}%
\pgfpathcurveto{\pgfqpoint{2.504415in}{1.583526in}}{\pgfqpoint{2.500024in}{1.572927in}}{\pgfqpoint{2.500024in}{1.561877in}}%
\pgfpathcurveto{\pgfqpoint{2.500024in}{1.550827in}}{\pgfqpoint{2.504415in}{1.540227in}}{\pgfqpoint{2.512228in}{1.532414in}}%
\pgfpathcurveto{\pgfqpoint{2.520042in}{1.524600in}}{\pgfqpoint{2.530641in}{1.520210in}}{\pgfqpoint{2.541691in}{1.520210in}}%
\pgfpathclose%
\pgfusepath{stroke,fill}%
\end{pgfscope}%
\begin{pgfscope}%
\pgfpathrectangle{\pgfqpoint{0.557699in}{0.629134in}}{\pgfqpoint{3.720000in}{3.020000in}}%
\pgfusepath{clip}%
\pgfsetbuttcap%
\pgfsetroundjoin%
\definecolor{currentfill}{rgb}{0.045098,0.655284,0.936852}%
\pgfsetfillcolor{currentfill}%
\pgfsetlinewidth{1.003750pt}%
\definecolor{currentstroke}{rgb}{0.045098,0.655284,0.936852}%
\pgfsetstrokecolor{currentstroke}%
\pgfsetdash{}{0pt}%
\pgfpathmoveto{\pgfqpoint{1.809304in}{1.099141in}}%
\pgfpathcurveto{\pgfqpoint{1.820355in}{1.099141in}}{\pgfqpoint{1.830954in}{1.103531in}}{\pgfqpoint{1.838767in}{1.111344in}}%
\pgfpathcurveto{\pgfqpoint{1.846581in}{1.119158in}}{\pgfqpoint{1.850971in}{1.129757in}}{\pgfqpoint{1.850971in}{1.140807in}}%
\pgfpathcurveto{\pgfqpoint{1.850971in}{1.151857in}}{\pgfqpoint{1.846581in}{1.162456in}}{\pgfqpoint{1.838767in}{1.170270in}}%
\pgfpathcurveto{\pgfqpoint{1.830954in}{1.178084in}}{\pgfqpoint{1.820355in}{1.182474in}}{\pgfqpoint{1.809304in}{1.182474in}}%
\pgfpathcurveto{\pgfqpoint{1.798254in}{1.182474in}}{\pgfqpoint{1.787655in}{1.178084in}}{\pgfqpoint{1.779842in}{1.170270in}}%
\pgfpathcurveto{\pgfqpoint{1.772028in}{1.162456in}}{\pgfqpoint{1.767638in}{1.151857in}}{\pgfqpoint{1.767638in}{1.140807in}}%
\pgfpathcurveto{\pgfqpoint{1.767638in}{1.129757in}}{\pgfqpoint{1.772028in}{1.119158in}}{\pgfqpoint{1.779842in}{1.111344in}}%
\pgfpathcurveto{\pgfqpoint{1.787655in}{1.103531in}}{\pgfqpoint{1.798254in}{1.099141in}}{\pgfqpoint{1.809304in}{1.099141in}}%
\pgfpathclose%
\pgfusepath{stroke,fill}%
\end{pgfscope}%
\begin{pgfscope}%
\pgfpathrectangle{\pgfqpoint{0.557699in}{0.629134in}}{\pgfqpoint{3.720000in}{3.020000in}}%
\pgfusepath{clip}%
\pgfsetbuttcap%
\pgfsetroundjoin%
\definecolor{currentfill}{rgb}{0.045098,0.655284,0.936852}%
\pgfsetfillcolor{currentfill}%
\pgfsetlinewidth{1.003750pt}%
\definecolor{currentstroke}{rgb}{0.045098,0.655284,0.936852}%
\pgfsetstrokecolor{currentstroke}%
\pgfsetdash{}{0pt}%
\pgfpathmoveto{\pgfqpoint{1.068655in}{0.888606in}}%
\pgfpathcurveto{\pgfqpoint{1.079705in}{0.888606in}}{\pgfqpoint{1.090304in}{0.892996in}}{\pgfqpoint{1.098118in}{0.900810in}}%
\pgfpathcurveto{\pgfqpoint{1.105932in}{0.908623in}}{\pgfqpoint{1.110322in}{0.919222in}}{\pgfqpoint{1.110322in}{0.930273in}}%
\pgfpathcurveto{\pgfqpoint{1.110322in}{0.941323in}}{\pgfqpoint{1.105932in}{0.951922in}}{\pgfqpoint{1.098118in}{0.959735in}}%
\pgfpathcurveto{\pgfqpoint{1.090304in}{0.967549in}}{\pgfqpoint{1.079705in}{0.971939in}}{\pgfqpoint{1.068655in}{0.971939in}}%
\pgfpathcurveto{\pgfqpoint{1.057605in}{0.971939in}}{\pgfqpoint{1.047006in}{0.967549in}}{\pgfqpoint{1.039192in}{0.959735in}}%
\pgfpathcurveto{\pgfqpoint{1.031379in}{0.951922in}}{\pgfqpoint{1.026989in}{0.941323in}}{\pgfqpoint{1.026989in}{0.930273in}}%
\pgfpathcurveto{\pgfqpoint{1.026989in}{0.919222in}}{\pgfqpoint{1.031379in}{0.908623in}}{\pgfqpoint{1.039192in}{0.900810in}}%
\pgfpathcurveto{\pgfqpoint{1.047006in}{0.892996in}}{\pgfqpoint{1.057605in}{0.888606in}}{\pgfqpoint{1.068655in}{0.888606in}}%
\pgfpathclose%
\pgfusepath{stroke,fill}%
\end{pgfscope}%
\begin{pgfscope}%
\pgfpathrectangle{\pgfqpoint{0.557699in}{0.629134in}}{\pgfqpoint{3.720000in}{3.020000in}}%
\pgfusepath{clip}%
\pgfsetbuttcap%
\pgfsetroundjoin%
\definecolor{currentfill}{rgb}{0.296078,0.314870,0.987202}%
\pgfsetfillcolor{currentfill}%
\pgfsetlinewidth{1.003750pt}%
\definecolor{currentstroke}{rgb}{0.296078,0.314870,0.987202}%
\pgfsetstrokecolor{currentstroke}%
\pgfsetdash{}{0pt}%
\pgfpathmoveto{\pgfqpoint{2.401394in}{1.499050in}}%
\pgfpathcurveto{\pgfqpoint{2.412444in}{1.499050in}}{\pgfqpoint{2.423043in}{1.503440in}}{\pgfqpoint{2.430857in}{1.511254in}}%
\pgfpathcurveto{\pgfqpoint{2.438671in}{1.519068in}}{\pgfqpoint{2.443061in}{1.529667in}}{\pgfqpoint{2.443061in}{1.540717in}}%
\pgfpathcurveto{\pgfqpoint{2.443061in}{1.551767in}}{\pgfqpoint{2.438671in}{1.562366in}}{\pgfqpoint{2.430857in}{1.570180in}}%
\pgfpathcurveto{\pgfqpoint{2.423043in}{1.577993in}}{\pgfqpoint{2.412444in}{1.582383in}}{\pgfqpoint{2.401394in}{1.582383in}}%
\pgfpathcurveto{\pgfqpoint{2.390344in}{1.582383in}}{\pgfqpoint{2.379745in}{1.577993in}}{\pgfqpoint{2.371931in}{1.570180in}}%
\pgfpathcurveto{\pgfqpoint{2.364118in}{1.562366in}}{\pgfqpoint{2.359728in}{1.551767in}}{\pgfqpoint{2.359728in}{1.540717in}}%
\pgfpathcurveto{\pgfqpoint{2.359728in}{1.529667in}}{\pgfqpoint{2.364118in}{1.519068in}}{\pgfqpoint{2.371931in}{1.511254in}}%
\pgfpathcurveto{\pgfqpoint{2.379745in}{1.503440in}}{\pgfqpoint{2.390344in}{1.499050in}}{\pgfqpoint{2.401394in}{1.499050in}}%
\pgfpathclose%
\pgfusepath{stroke,fill}%
\end{pgfscope}%
\begin{pgfscope}%
\pgfpathrectangle{\pgfqpoint{0.557699in}{0.629134in}}{\pgfqpoint{3.720000in}{3.020000in}}%
\pgfusepath{clip}%
\pgfsetbuttcap%
\pgfsetroundjoin%
\definecolor{currentfill}{rgb}{0.296078,0.314870,0.987202}%
\pgfsetfillcolor{currentfill}%
\pgfsetlinewidth{1.003750pt}%
\definecolor{currentstroke}{rgb}{0.296078,0.314870,0.987202}%
\pgfsetstrokecolor{currentstroke}%
\pgfsetdash{}{0pt}%
\pgfpathmoveto{\pgfqpoint{1.963328in}{1.499050in}}%
\pgfpathcurveto{\pgfqpoint{1.974378in}{1.499050in}}{\pgfqpoint{1.984977in}{1.503440in}}{\pgfqpoint{1.992791in}{1.511254in}}%
\pgfpathcurveto{\pgfqpoint{2.000604in}{1.519068in}}{\pgfqpoint{2.004995in}{1.529667in}}{\pgfqpoint{2.004995in}{1.540717in}}%
\pgfpathcurveto{\pgfqpoint{2.004995in}{1.551767in}}{\pgfqpoint{2.000604in}{1.562366in}}{\pgfqpoint{1.992791in}{1.570180in}}%
\pgfpathcurveto{\pgfqpoint{1.984977in}{1.577993in}}{\pgfqpoint{1.974378in}{1.582383in}}{\pgfqpoint{1.963328in}{1.582383in}}%
\pgfpathcurveto{\pgfqpoint{1.952278in}{1.582383in}}{\pgfqpoint{1.941679in}{1.577993in}}{\pgfqpoint{1.933865in}{1.570180in}}%
\pgfpathcurveto{\pgfqpoint{1.926052in}{1.562366in}}{\pgfqpoint{1.921661in}{1.551767in}}{\pgfqpoint{1.921661in}{1.540717in}}%
\pgfpathcurveto{\pgfqpoint{1.921661in}{1.529667in}}{\pgfqpoint{1.926052in}{1.519068in}}{\pgfqpoint{1.933865in}{1.511254in}}%
\pgfpathcurveto{\pgfqpoint{1.941679in}{1.503440in}}{\pgfqpoint{1.952278in}{1.499050in}}{\pgfqpoint{1.963328in}{1.499050in}}%
\pgfpathclose%
\pgfusepath{stroke,fill}%
\end{pgfscope}%
\begin{pgfscope}%
\pgfpathrectangle{\pgfqpoint{0.557699in}{0.629134in}}{\pgfqpoint{3.720000in}{3.020000in}}%
\pgfusepath{clip}%
\pgfsetbuttcap%
\pgfsetroundjoin%
\definecolor{currentfill}{rgb}{0.296078,0.314870,0.987202}%
\pgfsetfillcolor{currentfill}%
\pgfsetlinewidth{1.003750pt}%
\definecolor{currentstroke}{rgb}{0.296078,0.314870,0.987202}%
\pgfsetstrokecolor{currentstroke}%
\pgfsetdash{}{0pt}%
\pgfpathmoveto{\pgfqpoint{1.705161in}{1.499050in}}%
\pgfpathcurveto{\pgfqpoint{1.716211in}{1.499050in}}{\pgfqpoint{1.726810in}{1.503440in}}{\pgfqpoint{1.734624in}{1.511254in}}%
\pgfpathcurveto{\pgfqpoint{1.742437in}{1.519068in}}{\pgfqpoint{1.746828in}{1.529667in}}{\pgfqpoint{1.746828in}{1.540717in}}%
\pgfpathcurveto{\pgfqpoint{1.746828in}{1.551767in}}{\pgfqpoint{1.742437in}{1.562366in}}{\pgfqpoint{1.734624in}{1.570180in}}%
\pgfpathcurveto{\pgfqpoint{1.726810in}{1.577993in}}{\pgfqpoint{1.716211in}{1.582383in}}{\pgfqpoint{1.705161in}{1.582383in}}%
\pgfpathcurveto{\pgfqpoint{1.694111in}{1.582383in}}{\pgfqpoint{1.683512in}{1.577993in}}{\pgfqpoint{1.675698in}{1.570180in}}%
\pgfpathcurveto{\pgfqpoint{1.667884in}{1.562366in}}{\pgfqpoint{1.663494in}{1.551767in}}{\pgfqpoint{1.663494in}{1.540717in}}%
\pgfpathcurveto{\pgfqpoint{1.663494in}{1.529667in}}{\pgfqpoint{1.667884in}{1.519068in}}{\pgfqpoint{1.675698in}{1.511254in}}%
\pgfpathcurveto{\pgfqpoint{1.683512in}{1.503440in}}{\pgfqpoint{1.694111in}{1.499050in}}{\pgfqpoint{1.705161in}{1.499050in}}%
\pgfpathclose%
\pgfusepath{stroke,fill}%
\end{pgfscope}%
\begin{pgfscope}%
\pgfpathrectangle{\pgfqpoint{0.557699in}{0.629134in}}{\pgfqpoint{3.720000in}{3.020000in}}%
\pgfusepath{clip}%
\pgfsetbuttcap%
\pgfsetroundjoin%
\definecolor{currentfill}{rgb}{0.296078,0.314870,0.987202}%
\pgfsetfillcolor{currentfill}%
\pgfsetlinewidth{1.003750pt}%
\definecolor{currentstroke}{rgb}{0.296078,0.314870,0.987202}%
\pgfsetstrokecolor{currentstroke}%
\pgfsetdash{}{0pt}%
\pgfpathmoveto{\pgfqpoint{1.464401in}{1.499050in}}%
\pgfpathcurveto{\pgfqpoint{1.475451in}{1.499050in}}{\pgfqpoint{1.486050in}{1.503440in}}{\pgfqpoint{1.493864in}{1.511254in}}%
\pgfpathcurveto{\pgfqpoint{1.501678in}{1.519068in}}{\pgfqpoint{1.506068in}{1.529667in}}{\pgfqpoint{1.506068in}{1.540717in}}%
\pgfpathcurveto{\pgfqpoint{1.506068in}{1.551767in}}{\pgfqpoint{1.501678in}{1.562366in}}{\pgfqpoint{1.493864in}{1.570180in}}%
\pgfpathcurveto{\pgfqpoint{1.486050in}{1.577993in}}{\pgfqpoint{1.475451in}{1.582383in}}{\pgfqpoint{1.464401in}{1.582383in}}%
\pgfpathcurveto{\pgfqpoint{1.453351in}{1.582383in}}{\pgfqpoint{1.442752in}{1.577993in}}{\pgfqpoint{1.434938in}{1.570180in}}%
\pgfpathcurveto{\pgfqpoint{1.427125in}{1.562366in}}{\pgfqpoint{1.422735in}{1.551767in}}{\pgfqpoint{1.422735in}{1.540717in}}%
\pgfpathcurveto{\pgfqpoint{1.422735in}{1.529667in}}{\pgfqpoint{1.427125in}{1.519068in}}{\pgfqpoint{1.434938in}{1.511254in}}%
\pgfpathcurveto{\pgfqpoint{1.442752in}{1.503440in}}{\pgfqpoint{1.453351in}{1.499050in}}{\pgfqpoint{1.464401in}{1.499050in}}%
\pgfpathclose%
\pgfusepath{stroke,fill}%
\end{pgfscope}%
\begin{pgfscope}%
\pgfpathrectangle{\pgfqpoint{0.557699in}{0.629134in}}{\pgfqpoint{3.720000in}{3.020000in}}%
\pgfusepath{clip}%
\pgfsetbuttcap%
\pgfsetroundjoin%
\definecolor{currentfill}{rgb}{0.296078,0.314870,0.987202}%
\pgfsetfillcolor{currentfill}%
\pgfsetlinewidth{1.003750pt}%
\definecolor{currentstroke}{rgb}{0.296078,0.314870,0.987202}%
\pgfsetstrokecolor{currentstroke}%
\pgfsetdash{}{0pt}%
\pgfpathmoveto{\pgfqpoint{0.799570in}{1.201309in}}%
\pgfpathcurveto{\pgfqpoint{0.810620in}{1.201309in}}{\pgfqpoint{0.821219in}{1.205699in}}{\pgfqpoint{0.829033in}{1.213513in}}%
\pgfpathcurveto{\pgfqpoint{0.836847in}{1.221327in}}{\pgfqpoint{0.841237in}{1.231926in}}{\pgfqpoint{0.841237in}{1.242976in}}%
\pgfpathcurveto{\pgfqpoint{0.841237in}{1.254026in}}{\pgfqpoint{0.836847in}{1.264625in}}{\pgfqpoint{0.829033in}{1.272439in}}%
\pgfpathcurveto{\pgfqpoint{0.821219in}{1.280252in}}{\pgfqpoint{0.810620in}{1.284642in}}{\pgfqpoint{0.799570in}{1.284642in}}%
\pgfpathcurveto{\pgfqpoint{0.788520in}{1.284642in}}{\pgfqpoint{0.777921in}{1.280252in}}{\pgfqpoint{0.770107in}{1.272439in}}%
\pgfpathcurveto{\pgfqpoint{0.762294in}{1.264625in}}{\pgfqpoint{0.757904in}{1.254026in}}{\pgfqpoint{0.757904in}{1.242976in}}%
\pgfpathcurveto{\pgfqpoint{0.757904in}{1.231926in}}{\pgfqpoint{0.762294in}{1.221327in}}{\pgfqpoint{0.770107in}{1.213513in}}%
\pgfpathcurveto{\pgfqpoint{0.777921in}{1.205699in}}{\pgfqpoint{0.788520in}{1.201309in}}{\pgfqpoint{0.799570in}{1.201309in}}%
\pgfpathclose%
\pgfusepath{stroke,fill}%
\end{pgfscope}%
\begin{pgfscope}%
\pgfsetbuttcap%
\pgfsetroundjoin%
\definecolor{currentfill}{rgb}{0.000000,0.000000,0.000000}%
\pgfsetfillcolor{currentfill}%
\pgfsetlinewidth{0.803000pt}%
\definecolor{currentstroke}{rgb}{0.000000,0.000000,0.000000}%
\pgfsetstrokecolor{currentstroke}%
\pgfsetdash{}{0pt}%
\pgfsys@defobject{currentmarker}{\pgfqpoint{0.000000in}{-0.048611in}}{\pgfqpoint{0.000000in}{0.000000in}}{%
\pgfpathmoveto{\pgfqpoint{0.000000in}{0.000000in}}%
\pgfpathlineto{\pgfqpoint{0.000000in}{-0.048611in}}%
\pgfusepath{stroke,fill}%
}%
\begin{pgfscope}%
\pgfsys@transformshift{1.359614in}{0.629134in}%
\pgfsys@useobject{currentmarker}{}%
\end{pgfscope}%
\end{pgfscope}%
\begin{pgfscope}%
\definecolor{textcolor}{rgb}{0.000000,0.000000,0.000000}%
\pgfsetstrokecolor{textcolor}%
\pgfsetfillcolor{textcolor}%
\pgftext[x=1.359614in,y=0.531912in,,top]{\color{textcolor}\rmfamily\fontsize{14.000000}{16.800000}\selectfont \(\displaystyle 10^{-1}\)}%
\end{pgfscope}%
\begin{pgfscope}%
\pgfsetbuttcap%
\pgfsetroundjoin%
\definecolor{currentfill}{rgb}{0.000000,0.000000,0.000000}%
\pgfsetfillcolor{currentfill}%
\pgfsetlinewidth{0.803000pt}%
\definecolor{currentstroke}{rgb}{0.000000,0.000000,0.000000}%
\pgfsetstrokecolor{currentstroke}%
\pgfsetdash{}{0pt}%
\pgfsys@defobject{currentmarker}{\pgfqpoint{0.000000in}{-0.048611in}}{\pgfqpoint{0.000000in}{0.000000in}}{%
\pgfpathmoveto{\pgfqpoint{0.000000in}{0.000000in}}%
\pgfpathlineto{\pgfqpoint{0.000000in}{-0.048611in}}%
\pgfusepath{stroke,fill}%
}%
\begin{pgfscope}%
\pgfsys@transformshift{4.023517in}{0.629134in}%
\pgfsys@useobject{currentmarker}{}%
\end{pgfscope}%
\end{pgfscope}%
\begin{pgfscope}%
\definecolor{textcolor}{rgb}{0.000000,0.000000,0.000000}%
\pgfsetstrokecolor{textcolor}%
\pgfsetfillcolor{textcolor}%
\pgftext[x=4.023517in,y=0.531912in,,top]{\color{textcolor}\rmfamily\fontsize{14.000000}{16.800000}\selectfont \(\displaystyle 10^{0}\)}%
\end{pgfscope}%
\begin{pgfscope}%
\pgfsetbuttcap%
\pgfsetroundjoin%
\definecolor{currentfill}{rgb}{0.000000,0.000000,0.000000}%
\pgfsetfillcolor{currentfill}%
\pgfsetlinewidth{0.602250pt}%
\definecolor{currentstroke}{rgb}{0.000000,0.000000,0.000000}%
\pgfsetstrokecolor{currentstroke}%
\pgfsetdash{}{0pt}%
\pgfsys@defobject{currentmarker}{\pgfqpoint{0.000000in}{-0.027778in}}{\pgfqpoint{0.000000in}{0.000000in}}{%
\pgfpathmoveto{\pgfqpoint{0.000000in}{0.000000in}}%
\pgfpathlineto{\pgfqpoint{0.000000in}{-0.027778in}}%
\pgfusepath{stroke,fill}%
}%
\begin{pgfscope}%
\pgfsys@transformshift{0.557699in}{0.629134in}%
\pgfsys@useobject{currentmarker}{}%
\end{pgfscope}%
\end{pgfscope}%
\begin{pgfscope}%
\pgfsetbuttcap%
\pgfsetroundjoin%
\definecolor{currentfill}{rgb}{0.000000,0.000000,0.000000}%
\pgfsetfillcolor{currentfill}%
\pgfsetlinewidth{0.602250pt}%
\definecolor{currentstroke}{rgb}{0.000000,0.000000,0.000000}%
\pgfsetstrokecolor{currentstroke}%
\pgfsetdash{}{0pt}%
\pgfsys@defobject{currentmarker}{\pgfqpoint{0.000000in}{-0.027778in}}{\pgfqpoint{0.000000in}{0.000000in}}{%
\pgfpathmoveto{\pgfqpoint{0.000000in}{0.000000in}}%
\pgfpathlineto{\pgfqpoint{0.000000in}{-0.027778in}}%
\pgfusepath{stroke,fill}%
}%
\begin{pgfscope}%
\pgfsys@transformshift{0.768630in}{0.629134in}%
\pgfsys@useobject{currentmarker}{}%
\end{pgfscope}%
\end{pgfscope}%
\begin{pgfscope}%
\pgfsetbuttcap%
\pgfsetroundjoin%
\definecolor{currentfill}{rgb}{0.000000,0.000000,0.000000}%
\pgfsetfillcolor{currentfill}%
\pgfsetlinewidth{0.602250pt}%
\definecolor{currentstroke}{rgb}{0.000000,0.000000,0.000000}%
\pgfsetstrokecolor{currentstroke}%
\pgfsetdash{}{0pt}%
\pgfsys@defobject{currentmarker}{\pgfqpoint{0.000000in}{-0.027778in}}{\pgfqpoint{0.000000in}{0.000000in}}{%
\pgfpathmoveto{\pgfqpoint{0.000000in}{0.000000in}}%
\pgfpathlineto{\pgfqpoint{0.000000in}{-0.027778in}}%
\pgfusepath{stroke,fill}%
}%
\begin{pgfscope}%
\pgfsys@transformshift{0.946970in}{0.629134in}%
\pgfsys@useobject{currentmarker}{}%
\end{pgfscope}%
\end{pgfscope}%
\begin{pgfscope}%
\pgfsetbuttcap%
\pgfsetroundjoin%
\definecolor{currentfill}{rgb}{0.000000,0.000000,0.000000}%
\pgfsetfillcolor{currentfill}%
\pgfsetlinewidth{0.602250pt}%
\definecolor{currentstroke}{rgb}{0.000000,0.000000,0.000000}%
\pgfsetstrokecolor{currentstroke}%
\pgfsetdash{}{0pt}%
\pgfsys@defobject{currentmarker}{\pgfqpoint{0.000000in}{-0.027778in}}{\pgfqpoint{0.000000in}{0.000000in}}{%
\pgfpathmoveto{\pgfqpoint{0.000000in}{0.000000in}}%
\pgfpathlineto{\pgfqpoint{0.000000in}{-0.027778in}}%
\pgfusepath{stroke,fill}%
}%
\begin{pgfscope}%
\pgfsys@transformshift{1.101455in}{0.629134in}%
\pgfsys@useobject{currentmarker}{}%
\end{pgfscope}%
\end{pgfscope}%
\begin{pgfscope}%
\pgfsetbuttcap%
\pgfsetroundjoin%
\definecolor{currentfill}{rgb}{0.000000,0.000000,0.000000}%
\pgfsetfillcolor{currentfill}%
\pgfsetlinewidth{0.602250pt}%
\definecolor{currentstroke}{rgb}{0.000000,0.000000,0.000000}%
\pgfsetstrokecolor{currentstroke}%
\pgfsetdash{}{0pt}%
\pgfsys@defobject{currentmarker}{\pgfqpoint{0.000000in}{-0.027778in}}{\pgfqpoint{0.000000in}{0.000000in}}{%
\pgfpathmoveto{\pgfqpoint{0.000000in}{0.000000in}}%
\pgfpathlineto{\pgfqpoint{0.000000in}{-0.027778in}}%
\pgfusepath{stroke,fill}%
}%
\begin{pgfscope}%
\pgfsys@transformshift{1.237720in}{0.629134in}%
\pgfsys@useobject{currentmarker}{}%
\end{pgfscope}%
\end{pgfscope}%
\begin{pgfscope}%
\pgfsetbuttcap%
\pgfsetroundjoin%
\definecolor{currentfill}{rgb}{0.000000,0.000000,0.000000}%
\pgfsetfillcolor{currentfill}%
\pgfsetlinewidth{0.602250pt}%
\definecolor{currentstroke}{rgb}{0.000000,0.000000,0.000000}%
\pgfsetstrokecolor{currentstroke}%
\pgfsetdash{}{0pt}%
\pgfsys@defobject{currentmarker}{\pgfqpoint{0.000000in}{-0.027778in}}{\pgfqpoint{0.000000in}{0.000000in}}{%
\pgfpathmoveto{\pgfqpoint{0.000000in}{0.000000in}}%
\pgfpathlineto{\pgfqpoint{0.000000in}{-0.027778in}}%
\pgfusepath{stroke,fill}%
}%
\begin{pgfscope}%
\pgfsys@transformshift{2.161529in}{0.629134in}%
\pgfsys@useobject{currentmarker}{}%
\end{pgfscope}%
\end{pgfscope}%
\begin{pgfscope}%
\pgfsetbuttcap%
\pgfsetroundjoin%
\definecolor{currentfill}{rgb}{0.000000,0.000000,0.000000}%
\pgfsetfillcolor{currentfill}%
\pgfsetlinewidth{0.602250pt}%
\definecolor{currentstroke}{rgb}{0.000000,0.000000,0.000000}%
\pgfsetstrokecolor{currentstroke}%
\pgfsetdash{}{0pt}%
\pgfsys@defobject{currentmarker}{\pgfqpoint{0.000000in}{-0.027778in}}{\pgfqpoint{0.000000in}{0.000000in}}{%
\pgfpathmoveto{\pgfqpoint{0.000000in}{0.000000in}}%
\pgfpathlineto{\pgfqpoint{0.000000in}{-0.027778in}}%
\pgfusepath{stroke,fill}%
}%
\begin{pgfscope}%
\pgfsys@transformshift{2.630619in}{0.629134in}%
\pgfsys@useobject{currentmarker}{}%
\end{pgfscope}%
\end{pgfscope}%
\begin{pgfscope}%
\pgfsetbuttcap%
\pgfsetroundjoin%
\definecolor{currentfill}{rgb}{0.000000,0.000000,0.000000}%
\pgfsetfillcolor{currentfill}%
\pgfsetlinewidth{0.602250pt}%
\definecolor{currentstroke}{rgb}{0.000000,0.000000,0.000000}%
\pgfsetstrokecolor{currentstroke}%
\pgfsetdash{}{0pt}%
\pgfsys@defobject{currentmarker}{\pgfqpoint{0.000000in}{-0.027778in}}{\pgfqpoint{0.000000in}{0.000000in}}{%
\pgfpathmoveto{\pgfqpoint{0.000000in}{0.000000in}}%
\pgfpathlineto{\pgfqpoint{0.000000in}{-0.027778in}}%
\pgfusepath{stroke,fill}%
}%
\begin{pgfscope}%
\pgfsys@transformshift{2.963444in}{0.629134in}%
\pgfsys@useobject{currentmarker}{}%
\end{pgfscope}%
\end{pgfscope}%
\begin{pgfscope}%
\pgfsetbuttcap%
\pgfsetroundjoin%
\definecolor{currentfill}{rgb}{0.000000,0.000000,0.000000}%
\pgfsetfillcolor{currentfill}%
\pgfsetlinewidth{0.602250pt}%
\definecolor{currentstroke}{rgb}{0.000000,0.000000,0.000000}%
\pgfsetstrokecolor{currentstroke}%
\pgfsetdash{}{0pt}%
\pgfsys@defobject{currentmarker}{\pgfqpoint{0.000000in}{-0.027778in}}{\pgfqpoint{0.000000in}{0.000000in}}{%
\pgfpathmoveto{\pgfqpoint{0.000000in}{0.000000in}}%
\pgfpathlineto{\pgfqpoint{0.000000in}{-0.027778in}}%
\pgfusepath{stroke,fill}%
}%
\begin{pgfscope}%
\pgfsys@transformshift{3.221603in}{0.629134in}%
\pgfsys@useobject{currentmarker}{}%
\end{pgfscope}%
\end{pgfscope}%
\begin{pgfscope}%
\pgfsetbuttcap%
\pgfsetroundjoin%
\definecolor{currentfill}{rgb}{0.000000,0.000000,0.000000}%
\pgfsetfillcolor{currentfill}%
\pgfsetlinewidth{0.602250pt}%
\definecolor{currentstroke}{rgb}{0.000000,0.000000,0.000000}%
\pgfsetstrokecolor{currentstroke}%
\pgfsetdash{}{0pt}%
\pgfsys@defobject{currentmarker}{\pgfqpoint{0.000000in}{-0.027778in}}{\pgfqpoint{0.000000in}{0.000000in}}{%
\pgfpathmoveto{\pgfqpoint{0.000000in}{0.000000in}}%
\pgfpathlineto{\pgfqpoint{0.000000in}{-0.027778in}}%
\pgfusepath{stroke,fill}%
}%
\begin{pgfscope}%
\pgfsys@transformshift{3.432534in}{0.629134in}%
\pgfsys@useobject{currentmarker}{}%
\end{pgfscope}%
\end{pgfscope}%
\begin{pgfscope}%
\pgfsetbuttcap%
\pgfsetroundjoin%
\definecolor{currentfill}{rgb}{0.000000,0.000000,0.000000}%
\pgfsetfillcolor{currentfill}%
\pgfsetlinewidth{0.602250pt}%
\definecolor{currentstroke}{rgb}{0.000000,0.000000,0.000000}%
\pgfsetstrokecolor{currentstroke}%
\pgfsetdash{}{0pt}%
\pgfsys@defobject{currentmarker}{\pgfqpoint{0.000000in}{-0.027778in}}{\pgfqpoint{0.000000in}{0.000000in}}{%
\pgfpathmoveto{\pgfqpoint{0.000000in}{0.000000in}}%
\pgfpathlineto{\pgfqpoint{0.000000in}{-0.027778in}}%
\pgfusepath{stroke,fill}%
}%
\begin{pgfscope}%
\pgfsys@transformshift{3.610874in}{0.629134in}%
\pgfsys@useobject{currentmarker}{}%
\end{pgfscope}%
\end{pgfscope}%
\begin{pgfscope}%
\pgfsetbuttcap%
\pgfsetroundjoin%
\definecolor{currentfill}{rgb}{0.000000,0.000000,0.000000}%
\pgfsetfillcolor{currentfill}%
\pgfsetlinewidth{0.602250pt}%
\definecolor{currentstroke}{rgb}{0.000000,0.000000,0.000000}%
\pgfsetstrokecolor{currentstroke}%
\pgfsetdash{}{0pt}%
\pgfsys@defobject{currentmarker}{\pgfqpoint{0.000000in}{-0.027778in}}{\pgfqpoint{0.000000in}{0.000000in}}{%
\pgfpathmoveto{\pgfqpoint{0.000000in}{0.000000in}}%
\pgfpathlineto{\pgfqpoint{0.000000in}{-0.027778in}}%
\pgfusepath{stroke,fill}%
}%
\begin{pgfscope}%
\pgfsys@transformshift{3.765359in}{0.629134in}%
\pgfsys@useobject{currentmarker}{}%
\end{pgfscope}%
\end{pgfscope}%
\begin{pgfscope}%
\pgfsetbuttcap%
\pgfsetroundjoin%
\definecolor{currentfill}{rgb}{0.000000,0.000000,0.000000}%
\pgfsetfillcolor{currentfill}%
\pgfsetlinewidth{0.602250pt}%
\definecolor{currentstroke}{rgb}{0.000000,0.000000,0.000000}%
\pgfsetstrokecolor{currentstroke}%
\pgfsetdash{}{0pt}%
\pgfsys@defobject{currentmarker}{\pgfqpoint{0.000000in}{-0.027778in}}{\pgfqpoint{0.000000in}{0.000000in}}{%
\pgfpathmoveto{\pgfqpoint{0.000000in}{0.000000in}}%
\pgfpathlineto{\pgfqpoint{0.000000in}{-0.027778in}}%
\pgfusepath{stroke,fill}%
}%
\begin{pgfscope}%
\pgfsys@transformshift{3.901624in}{0.629134in}%
\pgfsys@useobject{currentmarker}{}%
\end{pgfscope}%
\end{pgfscope}%
\begin{pgfscope}%
\definecolor{textcolor}{rgb}{0.000000,0.000000,0.000000}%
\pgfsetstrokecolor{textcolor}%
\pgfsetfillcolor{textcolor}%
\pgftext[x=2.417699in,y=0.288178in,,top]{\color{textcolor}\rmfamily\fontsize{14.000000}{16.800000}\selectfont \(\displaystyle \mathbf{W}\mbox{e}\)}%
\end{pgfscope}%
\begin{pgfscope}%
\pgfsetbuttcap%
\pgfsetroundjoin%
\definecolor{currentfill}{rgb}{0.000000,0.000000,0.000000}%
\pgfsetfillcolor{currentfill}%
\pgfsetlinewidth{0.803000pt}%
\definecolor{currentstroke}{rgb}{0.000000,0.000000,0.000000}%
\pgfsetstrokecolor{currentstroke}%
\pgfsetdash{}{0pt}%
\pgfsys@defobject{currentmarker}{\pgfqpoint{-0.048611in}{0.000000in}}{\pgfqpoint{0.000000in}{0.000000in}}{%
\pgfpathmoveto{\pgfqpoint{0.000000in}{0.000000in}}%
\pgfpathlineto{\pgfqpoint{-0.048611in}{0.000000in}}%
\pgfusepath{stroke,fill}%
}%
\begin{pgfscope}%
\pgfsys@transformshift{0.557699in}{0.629134in}%
\pgfsys@useobject{currentmarker}{}%
\end{pgfscope}%
\end{pgfscope}%
\begin{pgfscope}%
\definecolor{textcolor}{rgb}{0.000000,0.000000,0.000000}%
\pgfsetstrokecolor{textcolor}%
\pgfsetfillcolor{textcolor}%
\pgftext[x=0.362561in,y=0.555268in,left,base]{\color{textcolor}\rmfamily\fontsize{14.000000}{16.800000}\selectfont \(\displaystyle 2\)}%
\end{pgfscope}%
\begin{pgfscope}%
\pgfsetbuttcap%
\pgfsetroundjoin%
\definecolor{currentfill}{rgb}{0.000000,0.000000,0.000000}%
\pgfsetfillcolor{currentfill}%
\pgfsetlinewidth{0.803000pt}%
\definecolor{currentstroke}{rgb}{0.000000,0.000000,0.000000}%
\pgfsetstrokecolor{currentstroke}%
\pgfsetdash{}{0pt}%
\pgfsys@defobject{currentmarker}{\pgfqpoint{-0.048611in}{0.000000in}}{\pgfqpoint{0.000000in}{0.000000in}}{%
\pgfpathmoveto{\pgfqpoint{0.000000in}{0.000000in}}%
\pgfpathlineto{\pgfqpoint{-0.048611in}{0.000000in}}%
\pgfusepath{stroke,fill}%
}%
\begin{pgfscope}%
\pgfsys@transformshift{0.557699in}{1.215436in}%
\pgfsys@useobject{currentmarker}{}%
\end{pgfscope}%
\end{pgfscope}%
\begin{pgfscope}%
\definecolor{textcolor}{rgb}{0.000000,0.000000,0.000000}%
\pgfsetstrokecolor{textcolor}%
\pgfsetfillcolor{textcolor}%
\pgftext[x=0.362561in,y=1.141570in,left,base]{\color{textcolor}\rmfamily\fontsize{14.000000}{16.800000}\selectfont \(\displaystyle 3\)}%
\end{pgfscope}%
\begin{pgfscope}%
\pgfsetbuttcap%
\pgfsetroundjoin%
\definecolor{currentfill}{rgb}{0.000000,0.000000,0.000000}%
\pgfsetfillcolor{currentfill}%
\pgfsetlinewidth{0.803000pt}%
\definecolor{currentstroke}{rgb}{0.000000,0.000000,0.000000}%
\pgfsetstrokecolor{currentstroke}%
\pgfsetdash{}{0pt}%
\pgfsys@defobject{currentmarker}{\pgfqpoint{-0.048611in}{0.000000in}}{\pgfqpoint{0.000000in}{0.000000in}}{%
\pgfpathmoveto{\pgfqpoint{0.000000in}{0.000000in}}%
\pgfpathlineto{\pgfqpoint{-0.048611in}{0.000000in}}%
\pgfusepath{stroke,fill}%
}%
\begin{pgfscope}%
\pgfsys@transformshift{0.557699in}{1.801738in}%
\pgfsys@useobject{currentmarker}{}%
\end{pgfscope}%
\end{pgfscope}%
\begin{pgfscope}%
\definecolor{textcolor}{rgb}{0.000000,0.000000,0.000000}%
\pgfsetstrokecolor{textcolor}%
\pgfsetfillcolor{textcolor}%
\pgftext[x=0.362561in,y=1.727872in,left,base]{\color{textcolor}\rmfamily\fontsize{14.000000}{16.800000}\selectfont \(\displaystyle 4\)}%
\end{pgfscope}%
\begin{pgfscope}%
\pgfsetbuttcap%
\pgfsetroundjoin%
\definecolor{currentfill}{rgb}{0.000000,0.000000,0.000000}%
\pgfsetfillcolor{currentfill}%
\pgfsetlinewidth{0.803000pt}%
\definecolor{currentstroke}{rgb}{0.000000,0.000000,0.000000}%
\pgfsetstrokecolor{currentstroke}%
\pgfsetdash{}{0pt}%
\pgfsys@defobject{currentmarker}{\pgfqpoint{-0.048611in}{0.000000in}}{\pgfqpoint{0.000000in}{0.000000in}}{%
\pgfpathmoveto{\pgfqpoint{0.000000in}{0.000000in}}%
\pgfpathlineto{\pgfqpoint{-0.048611in}{0.000000in}}%
\pgfusepath{stroke,fill}%
}%
\begin{pgfscope}%
\pgfsys@transformshift{0.557699in}{2.388040in}%
\pgfsys@useobject{currentmarker}{}%
\end{pgfscope}%
\end{pgfscope}%
\begin{pgfscope}%
\definecolor{textcolor}{rgb}{0.000000,0.000000,0.000000}%
\pgfsetstrokecolor{textcolor}%
\pgfsetfillcolor{textcolor}%
\pgftext[x=0.362561in,y=2.314174in,left,base]{\color{textcolor}\rmfamily\fontsize{14.000000}{16.800000}\selectfont \(\displaystyle 5\)}%
\end{pgfscope}%
\begin{pgfscope}%
\pgfsetbuttcap%
\pgfsetroundjoin%
\definecolor{currentfill}{rgb}{0.000000,0.000000,0.000000}%
\pgfsetfillcolor{currentfill}%
\pgfsetlinewidth{0.803000pt}%
\definecolor{currentstroke}{rgb}{0.000000,0.000000,0.000000}%
\pgfsetstrokecolor{currentstroke}%
\pgfsetdash{}{0pt}%
\pgfsys@defobject{currentmarker}{\pgfqpoint{-0.048611in}{0.000000in}}{\pgfqpoint{0.000000in}{0.000000in}}{%
\pgfpathmoveto{\pgfqpoint{0.000000in}{0.000000in}}%
\pgfpathlineto{\pgfqpoint{-0.048611in}{0.000000in}}%
\pgfusepath{stroke,fill}%
}%
\begin{pgfscope}%
\pgfsys@transformshift{0.557699in}{2.974342in}%
\pgfsys@useobject{currentmarker}{}%
\end{pgfscope}%
\end{pgfscope}%
\begin{pgfscope}%
\definecolor{textcolor}{rgb}{0.000000,0.000000,0.000000}%
\pgfsetstrokecolor{textcolor}%
\pgfsetfillcolor{textcolor}%
\pgftext[x=0.362561in,y=2.900476in,left,base]{\color{textcolor}\rmfamily\fontsize{14.000000}{16.800000}\selectfont \(\displaystyle 6\)}%
\end{pgfscope}%
\begin{pgfscope}%
\pgfsetbuttcap%
\pgfsetroundjoin%
\definecolor{currentfill}{rgb}{0.000000,0.000000,0.000000}%
\pgfsetfillcolor{currentfill}%
\pgfsetlinewidth{0.803000pt}%
\definecolor{currentstroke}{rgb}{0.000000,0.000000,0.000000}%
\pgfsetstrokecolor{currentstroke}%
\pgfsetdash{}{0pt}%
\pgfsys@defobject{currentmarker}{\pgfqpoint{-0.048611in}{0.000000in}}{\pgfqpoint{0.000000in}{0.000000in}}{%
\pgfpathmoveto{\pgfqpoint{0.000000in}{0.000000in}}%
\pgfpathlineto{\pgfqpoint{-0.048611in}{0.000000in}}%
\pgfusepath{stroke,fill}%
}%
\begin{pgfscope}%
\pgfsys@transformshift{0.557699in}{3.560644in}%
\pgfsys@useobject{currentmarker}{}%
\end{pgfscope}%
\end{pgfscope}%
\begin{pgfscope}%
\definecolor{textcolor}{rgb}{0.000000,0.000000,0.000000}%
\pgfsetstrokecolor{textcolor}%
\pgfsetfillcolor{textcolor}%
\pgftext[x=0.362561in,y=3.486778in,left,base]{\color{textcolor}\rmfamily\fontsize{14.000000}{16.800000}\selectfont \(\displaystyle 7\)}%
\end{pgfscope}%
\begin{pgfscope}%
\definecolor{textcolor}{rgb}{0.000000,0.000000,0.000000}%
\pgfsetstrokecolor{textcolor}%
\pgfsetfillcolor{textcolor}%
\pgftext[x=0.307006in,y=2.139134in,,bottom,rotate=90.000000]{\color{textcolor}\rmfamily\fontsize{14.000000}{16.800000}\selectfont \(\displaystyle t_j/ \tau\)}%
\end{pgfscope}%
\begin{pgfscope}%
\pgfpathrectangle{\pgfqpoint{0.557699in}{0.629134in}}{\pgfqpoint{3.720000in}{3.020000in}}%
\pgfusepath{clip}%
\pgfsetbuttcap%
\pgfsetroundjoin%
\pgfsetlinewidth{1.505625pt}%
\definecolor{currentstroke}{rgb}{0.000000,0.000000,0.000000}%
\pgfsetstrokecolor{currentstroke}%
\pgfsetdash{{5.550000pt}{2.400000pt}}{0.000000pt}%
\pgfpathmoveto{\pgfqpoint{0.554366in}{0.746394in}}%
\pgfpathlineto{\pgfqpoint{0.758851in}{0.746394in}}%
\pgfpathlineto{\pgfqpoint{0.936910in}{0.746394in}}%
\pgfpathlineto{\pgfqpoint{1.091184in}{0.746394in}}%
\pgfpathlineto{\pgfqpoint{1.227286in}{0.746394in}}%
\pgfpathlineto{\pgfqpoint{1.349048in}{0.746394in}}%
\pgfpathlineto{\pgfqpoint{1.459207in}{0.746394in}}%
\pgfpathlineto{\pgfqpoint{1.559783in}{0.746394in}}%
\pgfpathlineto{\pgfqpoint{1.652310in}{0.746394in}}%
\pgfpathlineto{\pgfqpoint{1.737982in}{0.746394in}}%
\pgfpathlineto{\pgfqpoint{1.817745in}{0.746394in}}%
\pgfpathlineto{\pgfqpoint{1.892362in}{0.746394in}}%
\pgfpathlineto{\pgfqpoint{1.962456in}{0.746394in}}%
\pgfpathlineto{\pgfqpoint{2.028545in}{0.746394in}}%
\pgfpathlineto{\pgfqpoint{2.091062in}{0.746394in}}%
\pgfpathlineto{\pgfqpoint{2.150373in}{0.746394in}}%
\pgfpathlineto{\pgfqpoint{2.206792in}{0.746394in}}%
\pgfpathlineto{\pgfqpoint{2.260586in}{0.746394in}}%
\pgfpathlineto{\pgfqpoint{2.311990in}{0.746394in}}%
\pgfpathlineto{\pgfqpoint{2.361206in}{0.746394in}}%
\pgfpathlineto{\pgfqpoint{2.408414in}{0.746394in}}%
\pgfpathlineto{\pgfqpoint{2.453771in}{0.746394in}}%
\pgfpathlineto{\pgfqpoint{2.497417in}{0.746394in}}%
\pgfpathlineto{\pgfqpoint{2.539476in}{0.746394in}}%
\pgfpathlineto{\pgfqpoint{2.580059in}{0.746394in}}%
\pgfpathlineto{\pgfqpoint{2.619267in}{0.746394in}}%
\pgfpathlineto{\pgfqpoint{2.657189in}{0.746394in}}%
\pgfpathlineto{\pgfqpoint{2.693908in}{0.746394in}}%
\pgfpathlineto{\pgfqpoint{2.729497in}{0.746394in}}%
\pgfpathlineto{\pgfqpoint{2.764024in}{0.746394in}}%
\pgfpathlineto{\pgfqpoint{2.797550in}{0.746394in}}%
\pgfpathlineto{\pgfqpoint{2.830133in}{0.746394in}}%
\pgfpathlineto{\pgfqpoint{2.861822in}{0.746394in}}%
\pgfpathlineto{\pgfqpoint{2.892667in}{0.746394in}}%
\pgfpathlineto{\pgfqpoint{2.922710in}{0.746394in}}%
\pgfpathlineto{\pgfqpoint{2.951993in}{0.746394in}}%
\pgfpathlineto{\pgfqpoint{2.980553in}{0.746394in}}%
\pgfpathlineto{\pgfqpoint{3.008425in}{0.746394in}}%
\pgfpathlineto{\pgfqpoint{3.035642in}{0.746394in}}%
\pgfpathlineto{\pgfqpoint{3.062233in}{0.746394in}}%
\pgfpathlineto{\pgfqpoint{3.088226in}{0.746394in}}%
\pgfpathlineto{\pgfqpoint{3.113648in}{0.746394in}}%
\pgfpathlineto{\pgfqpoint{3.138523in}{0.746394in}}%
\pgfpathlineto{\pgfqpoint{3.162875in}{0.746394in}}%
\pgfpathlineto{\pgfqpoint{3.186725in}{0.746394in}}%
\pgfpathlineto{\pgfqpoint{3.210093in}{0.746394in}}%
\pgfpathlineto{\pgfqpoint{3.232999in}{0.746394in}}%
\pgfpathlineto{\pgfqpoint{3.255459in}{0.746394in}}%
\pgfpathlineto{\pgfqpoint{3.277492in}{0.746394in}}%
\pgfpathlineto{\pgfqpoint{3.299113in}{0.746394in}}%
\pgfpathlineto{\pgfqpoint{3.320338in}{0.746394in}}%
\pgfpathlineto{\pgfqpoint{3.341180in}{0.746394in}}%
\pgfpathlineto{\pgfqpoint{3.361653in}{0.746394in}}%
\pgfpathlineto{\pgfqpoint{3.381771in}{0.746394in}}%
\pgfpathlineto{\pgfqpoint{3.401544in}{0.746394in}}%
\pgfpathlineto{\pgfqpoint{3.420985in}{0.746394in}}%
\pgfpathlineto{\pgfqpoint{3.440105in}{0.746394in}}%
\pgfpathlineto{\pgfqpoint{3.458914in}{0.746394in}}%
\pgfpathlineto{\pgfqpoint{3.477422in}{0.746394in}}%
\pgfpathlineto{\pgfqpoint{3.495639in}{0.746394in}}%
\pgfpathlineto{\pgfqpoint{3.513573in}{0.746394in}}%
\pgfpathlineto{\pgfqpoint{3.531233in}{0.746394in}}%
\pgfpathlineto{\pgfqpoint{3.548628in}{0.746394in}}%
\pgfpathlineto{\pgfqpoint{3.565765in}{0.746394in}}%
\pgfpathlineto{\pgfqpoint{3.582653in}{0.746394in}}%
\pgfpathlineto{\pgfqpoint{3.599297in}{0.746394in}}%
\pgfpathlineto{\pgfqpoint{3.615705in}{0.746394in}}%
\pgfpathlineto{\pgfqpoint{3.631883in}{0.746394in}}%
\pgfpathlineto{\pgfqpoint{3.647839in}{0.746394in}}%
\pgfpathlineto{\pgfqpoint{3.663577in}{0.746394in}}%
\pgfpathlineto{\pgfqpoint{3.679105in}{0.746394in}}%
\pgfpathlineto{\pgfqpoint{3.694426in}{0.746394in}}%
\pgfpathlineto{\pgfqpoint{3.709548in}{0.746394in}}%
\pgfpathlineto{\pgfqpoint{3.724474in}{0.746394in}}%
\pgfpathlineto{\pgfqpoint{3.739210in}{0.746394in}}%
\pgfpathlineto{\pgfqpoint{3.753761in}{0.746394in}}%
\pgfpathlineto{\pgfqpoint{3.768131in}{0.746394in}}%
\pgfpathlineto{\pgfqpoint{3.782324in}{0.746394in}}%
\pgfpathlineto{\pgfqpoint{3.796346in}{0.746394in}}%
\pgfpathlineto{\pgfqpoint{3.810200in}{0.746394in}}%
\pgfpathlineto{\pgfqpoint{3.823890in}{0.746394in}}%
\pgfpathlineto{\pgfqpoint{3.837419in}{0.746394in}}%
\pgfpathlineto{\pgfqpoint{3.850793in}{0.746394in}}%
\pgfpathlineto{\pgfqpoint{3.864013in}{0.746394in}}%
\pgfpathlineto{\pgfqpoint{3.877084in}{0.746394in}}%
\pgfpathlineto{\pgfqpoint{3.890010in}{0.746394in}}%
\pgfpathlineto{\pgfqpoint{3.902792in}{0.746394in}}%
\pgfpathlineto{\pgfqpoint{3.915435in}{0.746394in}}%
\pgfpathlineto{\pgfqpoint{3.927941in}{0.746394in}}%
\pgfpathlineto{\pgfqpoint{3.940313in}{0.746394in}}%
\pgfpathlineto{\pgfqpoint{3.952554in}{0.746394in}}%
\pgfpathlineto{\pgfqpoint{3.964667in}{0.746394in}}%
\pgfpathlineto{\pgfqpoint{3.976655in}{0.746394in}}%
\pgfpathlineto{\pgfqpoint{3.988520in}{0.746394in}}%
\pgfpathlineto{\pgfqpoint{4.000264in}{0.746394in}}%
\pgfpathlineto{\pgfqpoint{4.011890in}{0.746394in}}%
\pgfusepath{stroke}%
\end{pgfscope}%
\begin{pgfscope}%
\pgfpathrectangle{\pgfqpoint{0.557699in}{0.629134in}}{\pgfqpoint{3.720000in}{3.020000in}}%
\pgfusepath{clip}%
\pgfsetbuttcap%
\pgfsetroundjoin%
\pgfsetlinewidth{1.505625pt}%
\definecolor{currentstroke}{rgb}{0.501961,0.501961,0.501961}%
\pgfsetstrokecolor{currentstroke}%
\pgfsetdash{{9.600000pt}{2.400000pt}{1.500000pt}{2.400000pt}}{0.000000pt}%
\pgfpathmoveto{\pgfqpoint{0.554366in}{2.568981in}}%
\pgfpathlineto{\pgfqpoint{0.758851in}{2.465352in}}%
\pgfpathlineto{\pgfqpoint{0.936910in}{2.375115in}}%
\pgfpathlineto{\pgfqpoint{1.091184in}{2.296932in}}%
\pgfpathlineto{\pgfqpoint{1.227286in}{2.227959in}}%
\pgfpathlineto{\pgfqpoint{1.349048in}{2.166252in}}%
\pgfpathlineto{\pgfqpoint{1.459207in}{2.110426in}}%
\pgfpathlineto{\pgfqpoint{1.559783in}{2.059456in}}%
\pgfpathlineto{\pgfqpoint{1.652310in}{2.012565in}}%
\pgfpathlineto{\pgfqpoint{1.737982in}{1.969149in}}%
\pgfpathlineto{\pgfqpoint{1.817745in}{1.928726in}}%
\pgfpathlineto{\pgfqpoint{1.892362in}{1.890912in}}%
\pgfpathlineto{\pgfqpoint{1.962456in}{1.855390in}}%
\pgfpathlineto{\pgfqpoint{2.028545in}{1.821897in}}%
\pgfpathlineto{\pgfqpoint{2.091062in}{1.790215in}}%
\pgfpathlineto{\pgfqpoint{2.150373in}{1.760158in}}%
\pgfpathlineto{\pgfqpoint{2.206792in}{1.731566in}}%
\pgfpathlineto{\pgfqpoint{2.260586in}{1.704304in}}%
\pgfpathlineto{\pgfqpoint{2.311990in}{1.678254in}}%
\pgfpathlineto{\pgfqpoint{2.361206in}{1.653312in}}%
\pgfpathlineto{\pgfqpoint{2.408414in}{1.629388in}}%
\pgfpathlineto{\pgfqpoint{2.453771in}{1.606402in}}%
\pgfpathlineto{\pgfqpoint{2.497417in}{1.584283in}}%
\pgfpathlineto{\pgfqpoint{2.539476in}{1.562969in}}%
\pgfpathlineto{\pgfqpoint{2.580059in}{1.542402in}}%
\pgfpathlineto{\pgfqpoint{2.619267in}{1.522532in}}%
\pgfpathlineto{\pgfqpoint{2.657189in}{1.503314in}}%
\pgfpathlineto{\pgfqpoint{2.693908in}{1.484706in}}%
\pgfpathlineto{\pgfqpoint{2.729497in}{1.466670in}}%
\pgfpathlineto{\pgfqpoint{2.764024in}{1.449172in}}%
\pgfpathlineto{\pgfqpoint{2.797550in}{1.432182in}}%
\pgfpathlineto{\pgfqpoint{2.830133in}{1.415670in}}%
\pgfpathlineto{\pgfqpoint{2.861822in}{1.399610in}}%
\pgfpathlineto{\pgfqpoint{2.892667in}{1.383979in}}%
\pgfpathlineto{\pgfqpoint{2.922710in}{1.368754in}}%
\pgfpathlineto{\pgfqpoint{2.951993in}{1.353913in}}%
\pgfpathlineto{\pgfqpoint{2.980553in}{1.339440in}}%
\pgfpathlineto{\pgfqpoint{3.008425in}{1.325315in}}%
\pgfpathlineto{\pgfqpoint{3.035642in}{1.311522in}}%
\pgfpathlineto{\pgfqpoint{3.062233in}{1.298046in}}%
\pgfpathlineto{\pgfqpoint{3.088226in}{1.284874in}}%
\pgfpathlineto{\pgfqpoint{3.113648in}{1.271990in}}%
\pgfpathlineto{\pgfqpoint{3.138523in}{1.259384in}}%
\pgfpathlineto{\pgfqpoint{3.162875in}{1.247043in}}%
\pgfpathlineto{\pgfqpoint{3.186725in}{1.234956in}}%
\pgfpathlineto{\pgfqpoint{3.210093in}{1.223114in}}%
\pgfpathlineto{\pgfqpoint{3.232999in}{1.211506in}}%
\pgfpathlineto{\pgfqpoint{3.255459in}{1.200123in}}%
\pgfpathlineto{\pgfqpoint{3.277492in}{1.188957in}}%
\pgfpathlineto{\pgfqpoint{3.299113in}{1.178000in}}%
\pgfpathlineto{\pgfqpoint{3.320338in}{1.167244in}}%
\pgfpathlineto{\pgfqpoint{3.341180in}{1.156682in}}%
\pgfpathlineto{\pgfqpoint{3.361653in}{1.146306in}}%
\pgfpathlineto{\pgfqpoint{3.381771in}{1.136111in}}%
\pgfpathlineto{\pgfqpoint{3.401544in}{1.126091in}}%
\pgfpathlineto{\pgfqpoint{3.420985in}{1.116238in}}%
\pgfpathlineto{\pgfqpoint{3.440105in}{1.106549in}}%
\pgfpathlineto{\pgfqpoint{3.458914in}{1.097017in}}%
\pgfpathlineto{\pgfqpoint{3.477422in}{1.087637in}}%
\pgfpathlineto{\pgfqpoint{3.495639in}{1.078405in}}%
\pgfpathlineto{\pgfqpoint{3.513573in}{1.069317in}}%
\pgfpathlineto{\pgfqpoint{3.531233in}{1.060367in}}%
\pgfpathlineto{\pgfqpoint{3.548628in}{1.051551in}}%
\pgfpathlineto{\pgfqpoint{3.565765in}{1.042867in}}%
\pgfpathlineto{\pgfqpoint{3.582653in}{1.034309in}}%
\pgfpathlineto{\pgfqpoint{3.599297in}{1.025874in}}%
\pgfpathlineto{\pgfqpoint{3.615705in}{1.017558in}}%
\pgfpathlineto{\pgfqpoint{3.631883in}{1.009359in}}%
\pgfpathlineto{\pgfqpoint{3.647839in}{1.001273in}}%
\pgfpathlineto{\pgfqpoint{3.663577in}{0.993298in}}%
\pgfpathlineto{\pgfqpoint{3.679105in}{0.985429in}}%
\pgfpathlineto{\pgfqpoint{3.694426in}{0.977664in}}%
\pgfpathlineto{\pgfqpoint{3.709548in}{0.970001in}}%
\pgfpathlineto{\pgfqpoint{3.724474in}{0.962437in}}%
\pgfpathlineto{\pgfqpoint{3.739210in}{0.954969in}}%
\pgfpathlineto{\pgfqpoint{3.753761in}{0.947595in}}%
\pgfpathlineto{\pgfqpoint{3.768131in}{0.940312in}}%
\pgfpathlineto{\pgfqpoint{3.782324in}{0.933119in}}%
\pgfpathlineto{\pgfqpoint{3.796346in}{0.926013in}}%
\pgfpathlineto{\pgfqpoint{3.810200in}{0.918992in}}%
\pgfpathlineto{\pgfqpoint{3.823890in}{0.912055in}}%
\pgfpathlineto{\pgfqpoint{3.837419in}{0.905198in}}%
\pgfpathlineto{\pgfqpoint{3.850793in}{0.898421in}}%
\pgfpathlineto{\pgfqpoint{3.864013in}{0.891721in}}%
\pgfpathlineto{\pgfqpoint{3.877084in}{0.885097in}}%
\pgfpathlineto{\pgfqpoint{3.890010in}{0.878546in}}%
\pgfpathlineto{\pgfqpoint{3.902792in}{0.872069in}}%
\pgfpathlineto{\pgfqpoint{3.915435in}{0.865662in}}%
\pgfpathlineto{\pgfqpoint{3.927941in}{0.859324in}}%
\pgfpathlineto{\pgfqpoint{3.940313in}{0.853054in}}%
\pgfpathlineto{\pgfqpoint{3.952554in}{0.846850in}}%
\pgfpathlineto{\pgfqpoint{3.964667in}{0.840712in}}%
\pgfpathlineto{\pgfqpoint{3.976655in}{0.834636in}}%
\pgfpathlineto{\pgfqpoint{3.988520in}{0.828624in}}%
\pgfpathlineto{\pgfqpoint{4.000264in}{0.822672in}}%
\pgfpathlineto{\pgfqpoint{4.011890in}{0.816780in}}%
\pgfusepath{stroke}%
\end{pgfscope}%
\begin{pgfscope}%
\pgfpathrectangle{\pgfqpoint{0.557699in}{0.629134in}}{\pgfqpoint{3.720000in}{3.020000in}}%
\pgfusepath{clip}%
\pgfsetrectcap%
\pgfsetroundjoin%
\pgfsetlinewidth{1.505625pt}%
\definecolor{currentstroke}{rgb}{0.190196,0.883910,0.856638}%
\pgfsetstrokecolor{currentstroke}%
\pgfsetdash{}{0pt}%
\pgfpathmoveto{\pgfqpoint{0.554366in}{1.560082in}}%
\pgfpathlineto{\pgfqpoint{0.758851in}{1.528597in}}%
\pgfpathlineto{\pgfqpoint{0.936910in}{1.502896in}}%
\pgfpathlineto{\pgfqpoint{1.091184in}{1.481872in}}%
\pgfpathlineto{\pgfqpoint{1.227286in}{1.464257in}}%
\pgfpathlineto{\pgfqpoint{1.349048in}{1.449220in}}%
\pgfpathlineto{\pgfqpoint{1.459207in}{1.436187in}}%
\pgfpathlineto{\pgfqpoint{1.559783in}{1.424749in}}%
\pgfpathlineto{\pgfqpoint{1.652310in}{1.414605in}}%
\pgfpathlineto{\pgfqpoint{1.737982in}{1.405528in}}%
\pgfpathlineto{\pgfqpoint{1.817745in}{1.397343in}}%
\pgfpathlineto{\pgfqpoint{1.892362in}{1.389914in}}%
\pgfpathlineto{\pgfqpoint{1.962456in}{1.383130in}}%
\pgfpathlineto{\pgfqpoint{2.028545in}{1.376903in}}%
\pgfpathlineto{\pgfqpoint{2.091062in}{1.371161in}}%
\pgfpathlineto{\pgfqpoint{2.150373in}{1.365844in}}%
\pgfpathlineto{\pgfqpoint{2.206792in}{1.360902in}}%
\pgfpathlineto{\pgfqpoint{2.260586in}{1.356293in}}%
\pgfpathlineto{\pgfqpoint{2.311990in}{1.351981in}}%
\pgfpathlineto{\pgfqpoint{2.361206in}{1.347935in}}%
\pgfpathlineto{\pgfqpoint{2.408414in}{1.344130in}}%
\pgfpathlineto{\pgfqpoint{2.453771in}{1.340542in}}%
\pgfpathlineto{\pgfqpoint{2.497417in}{1.337152in}}%
\pgfpathlineto{\pgfqpoint{2.539476in}{1.333941in}}%
\pgfpathlineto{\pgfqpoint{2.580059in}{1.330895in}}%
\pgfpathlineto{\pgfqpoint{2.619267in}{1.328000in}}%
\pgfpathlineto{\pgfqpoint{2.657189in}{1.325244in}}%
\pgfpathlineto{\pgfqpoint{2.693908in}{1.322616in}}%
\pgfpathlineto{\pgfqpoint{2.729497in}{1.320106in}}%
\pgfpathlineto{\pgfqpoint{2.764024in}{1.317706in}}%
\pgfpathlineto{\pgfqpoint{2.797550in}{1.315408in}}%
\pgfpathlineto{\pgfqpoint{2.830133in}{1.313206in}}%
\pgfpathlineto{\pgfqpoint{2.861822in}{1.311092in}}%
\pgfpathlineto{\pgfqpoint{2.892667in}{1.309060in}}%
\pgfpathlineto{\pgfqpoint{2.922710in}{1.307107in}}%
\pgfpathlineto{\pgfqpoint{2.951993in}{1.305226in}}%
\pgfpathlineto{\pgfqpoint{2.980553in}{1.303413in}}%
\pgfpathlineto{\pgfqpoint{3.008425in}{1.301665in}}%
\pgfpathlineto{\pgfqpoint{3.035642in}{1.299978in}}%
\pgfpathlineto{\pgfqpoint{3.062233in}{1.298347in}}%
\pgfpathlineto{\pgfqpoint{3.088226in}{1.296771in}}%
\pgfpathlineto{\pgfqpoint{3.113648in}{1.295246in}}%
\pgfpathlineto{\pgfqpoint{3.138523in}{1.293769in}}%
\pgfpathlineto{\pgfqpoint{3.162875in}{1.292338in}}%
\pgfpathlineto{\pgfqpoint{3.186725in}{1.290950in}}%
\pgfpathlineto{\pgfqpoint{3.210093in}{1.289604in}}%
\pgfpathlineto{\pgfqpoint{3.232999in}{1.288298in}}%
\pgfpathlineto{\pgfqpoint{3.255459in}{1.287029in}}%
\pgfpathlineto{\pgfqpoint{3.277492in}{1.285795in}}%
\pgfpathlineto{\pgfqpoint{3.299113in}{1.284596in}}%
\pgfpathlineto{\pgfqpoint{3.320338in}{1.283429in}}%
\pgfpathlineto{\pgfqpoint{3.341180in}{1.282293in}}%
\pgfpathlineto{\pgfqpoint{3.361653in}{1.281187in}}%
\pgfpathlineto{\pgfqpoint{3.381771in}{1.280110in}}%
\pgfpathlineto{\pgfqpoint{3.401544in}{1.279060in}}%
\pgfpathlineto{\pgfqpoint{3.420985in}{1.278036in}}%
\pgfpathlineto{\pgfqpoint{3.440105in}{1.277037in}}%
\pgfpathlineto{\pgfqpoint{3.458914in}{1.276062in}}%
\pgfpathlineto{\pgfqpoint{3.477422in}{1.275110in}}%
\pgfpathlineto{\pgfqpoint{3.495639in}{1.274180in}}%
\pgfpathlineto{\pgfqpoint{3.513573in}{1.273272in}}%
\pgfpathlineto{\pgfqpoint{3.531233in}{1.272384in}}%
\pgfpathlineto{\pgfqpoint{3.548628in}{1.271517in}}%
\pgfpathlineto{\pgfqpoint{3.565765in}{1.270668in}}%
\pgfpathlineto{\pgfqpoint{3.582653in}{1.269837in}}%
\pgfpathlineto{\pgfqpoint{3.599297in}{1.269025in}}%
\pgfpathlineto{\pgfqpoint{3.615705in}{1.268229in}}%
\pgfpathlineto{\pgfqpoint{3.631883in}{1.267450in}}%
\pgfpathlineto{\pgfqpoint{3.647839in}{1.266687in}}%
\pgfpathlineto{\pgfqpoint{3.663577in}{1.265939in}}%
\pgfpathlineto{\pgfqpoint{3.679105in}{1.265206in}}%
\pgfpathlineto{\pgfqpoint{3.694426in}{1.264488in}}%
\pgfpathlineto{\pgfqpoint{3.709548in}{1.263783in}}%
\pgfpathlineto{\pgfqpoint{3.724474in}{1.263092in}}%
\pgfpathlineto{\pgfqpoint{3.739210in}{1.262414in}}%
\pgfpathlineto{\pgfqpoint{3.753761in}{1.261749in}}%
\pgfpathlineto{\pgfqpoint{3.768131in}{1.261096in}}%
\pgfpathlineto{\pgfqpoint{3.782324in}{1.260455in}}%
\pgfpathlineto{\pgfqpoint{3.796346in}{1.259826in}}%
\pgfpathlineto{\pgfqpoint{3.810200in}{1.259207in}}%
\pgfpathlineto{\pgfqpoint{3.823890in}{1.258600in}}%
\pgfpathlineto{\pgfqpoint{3.837419in}{1.258003in}}%
\pgfpathlineto{\pgfqpoint{3.850793in}{1.257416in}}%
\pgfpathlineto{\pgfqpoint{3.864013in}{1.256840in}}%
\pgfpathlineto{\pgfqpoint{3.877084in}{1.256273in}}%
\pgfpathlineto{\pgfqpoint{3.890010in}{1.255715in}}%
\pgfpathlineto{\pgfqpoint{3.902792in}{1.255167in}}%
\pgfpathlineto{\pgfqpoint{3.915435in}{1.254627in}}%
\pgfpathlineto{\pgfqpoint{3.927941in}{1.254097in}}%
\pgfpathlineto{\pgfqpoint{3.940313in}{1.253574in}}%
\pgfpathlineto{\pgfqpoint{3.952554in}{1.253060in}}%
\pgfpathlineto{\pgfqpoint{3.964667in}{1.252554in}}%
\pgfpathlineto{\pgfqpoint{3.976655in}{1.252056in}}%
\pgfpathlineto{\pgfqpoint{3.988520in}{1.251565in}}%
\pgfpathlineto{\pgfqpoint{4.000264in}{1.251082in}}%
\pgfpathlineto{\pgfqpoint{4.011890in}{1.250606in}}%
\pgfusepath{stroke}%
\end{pgfscope}%
\begin{pgfscope}%
\pgfpathrectangle{\pgfqpoint{0.557699in}{0.629134in}}{\pgfqpoint{3.720000in}{3.020000in}}%
\pgfusepath{clip}%
\pgfsetrectcap%
\pgfsetroundjoin%
\pgfsetlinewidth{1.505625pt}%
\definecolor{currentstroke}{rgb}{0.190196,0.883910,0.856638}%
\pgfsetstrokecolor{currentstroke}%
\pgfsetdash{}{0pt}%
\pgfpathmoveto{\pgfqpoint{0.554366in}{1.435009in}}%
\pgfpathlineto{\pgfqpoint{0.758851in}{1.397891in}}%
\pgfpathlineto{\pgfqpoint{0.936910in}{1.365822in}}%
\pgfpathlineto{\pgfqpoint{1.091184in}{1.338374in}}%
\pgfpathlineto{\pgfqpoint{1.227286in}{1.314513in}}%
\pgfpathlineto{\pgfqpoint{1.349048in}{1.293508in}}%
\pgfpathlineto{\pgfqpoint{1.459207in}{1.274824in}}%
\pgfpathlineto{\pgfqpoint{1.559783in}{1.258058in}}%
\pgfpathlineto{\pgfqpoint{1.652310in}{1.242899in}}%
\pgfpathlineto{\pgfqpoint{1.737982in}{1.229104in}}%
\pgfpathlineto{\pgfqpoint{1.817745in}{1.216478in}}%
\pgfpathlineto{\pgfqpoint{1.892362in}{1.204863in}}%
\pgfpathlineto{\pgfqpoint{1.962456in}{1.194132in}}%
\pgfpathlineto{\pgfqpoint{2.028545in}{1.184175in}}%
\pgfpathlineto{\pgfqpoint{2.091062in}{1.174905in}}%
\pgfpathlineto{\pgfqpoint{2.150373in}{1.166245in}}%
\pgfpathlineto{\pgfqpoint{2.206792in}{1.158131in}}%
\pgfpathlineto{\pgfqpoint{2.260586in}{1.150508in}}%
\pgfpathlineto{\pgfqpoint{2.311990in}{1.143327in}}%
\pgfpathlineto{\pgfqpoint{2.361206in}{1.136548in}}%
\pgfpathlineto{\pgfqpoint{2.408414in}{1.130135in}}%
\pgfpathlineto{\pgfqpoint{2.453771in}{1.124055in}}%
\pgfpathlineto{\pgfqpoint{2.497417in}{1.118280in}}%
\pgfpathlineto{\pgfqpoint{2.539476in}{1.112786in}}%
\pgfpathlineto{\pgfqpoint{2.580059in}{1.107551in}}%
\pgfpathlineto{\pgfqpoint{2.619267in}{1.102554in}}%
\pgfpathlineto{\pgfqpoint{2.657189in}{1.097779in}}%
\pgfpathlineto{\pgfqpoint{2.693908in}{1.093209in}}%
\pgfpathlineto{\pgfqpoint{2.729497in}{1.088830in}}%
\pgfpathlineto{\pgfqpoint{2.764024in}{1.084629in}}%
\pgfpathlineto{\pgfqpoint{2.797550in}{1.080595in}}%
\pgfpathlineto{\pgfqpoint{2.830133in}{1.076716in}}%
\pgfpathlineto{\pgfqpoint{2.861822in}{1.072983in}}%
\pgfpathlineto{\pgfqpoint{2.892667in}{1.069387in}}%
\pgfpathlineto{\pgfqpoint{2.922710in}{1.065919in}}%
\pgfpathlineto{\pgfqpoint{2.951993in}{1.062573in}}%
\pgfpathlineto{\pgfqpoint{2.980553in}{1.059340in}}%
\pgfpathlineto{\pgfqpoint{3.008425in}{1.056216in}}%
\pgfpathlineto{\pgfqpoint{3.035642in}{1.053194in}}%
\pgfpathlineto{\pgfqpoint{3.062233in}{1.050269in}}%
\pgfpathlineto{\pgfqpoint{3.088226in}{1.047435in}}%
\pgfpathlineto{\pgfqpoint{3.113648in}{1.044688in}}%
\pgfpathlineto{\pgfqpoint{3.138523in}{1.042023in}}%
\pgfpathlineto{\pgfqpoint{3.162875in}{1.039437in}}%
\pgfpathlineto{\pgfqpoint{3.186725in}{1.036926in}}%
\pgfpathlineto{\pgfqpoint{3.210093in}{1.034485in}}%
\pgfpathlineto{\pgfqpoint{3.232999in}{1.032113in}}%
\pgfpathlineto{\pgfqpoint{3.255459in}{1.029805in}}%
\pgfpathlineto{\pgfqpoint{3.277492in}{1.027559in}}%
\pgfpathlineto{\pgfqpoint{3.299113in}{1.025373in}}%
\pgfpathlineto{\pgfqpoint{3.320338in}{1.023243in}}%
\pgfpathlineto{\pgfqpoint{3.341180in}{1.021167in}}%
\pgfpathlineto{\pgfqpoint{3.361653in}{1.019143in}}%
\pgfpathlineto{\pgfqpoint{3.381771in}{1.017169in}}%
\pgfpathlineto{\pgfqpoint{3.401544in}{1.015243in}}%
\pgfpathlineto{\pgfqpoint{3.420985in}{1.013362in}}%
\pgfpathlineto{\pgfqpoint{3.440105in}{1.011526in}}%
\pgfpathlineto{\pgfqpoint{3.458914in}{1.009732in}}%
\pgfpathlineto{\pgfqpoint{3.477422in}{1.007979in}}%
\pgfpathlineto{\pgfqpoint{3.495639in}{1.006265in}}%
\pgfpathlineto{\pgfqpoint{3.513573in}{1.004589in}}%
\pgfpathlineto{\pgfqpoint{3.531233in}{1.002950in}}%
\pgfpathlineto{\pgfqpoint{3.548628in}{1.001345in}}%
\pgfpathlineto{\pgfqpoint{3.565765in}{0.999775in}}%
\pgfpathlineto{\pgfqpoint{3.582653in}{0.998237in}}%
\pgfpathlineto{\pgfqpoint{3.599297in}{0.996731in}}%
\pgfpathlineto{\pgfqpoint{3.615705in}{0.995255in}}%
\pgfpathlineto{\pgfqpoint{3.631883in}{0.993809in}}%
\pgfpathlineto{\pgfqpoint{3.647839in}{0.992391in}}%
\pgfpathlineto{\pgfqpoint{3.663577in}{0.991001in}}%
\pgfpathlineto{\pgfqpoint{3.679105in}{0.989638in}}%
\pgfpathlineto{\pgfqpoint{3.694426in}{0.988301in}}%
\pgfpathlineto{\pgfqpoint{3.709548in}{0.986989in}}%
\pgfpathlineto{\pgfqpoint{3.724474in}{0.985701in}}%
\pgfpathlineto{\pgfqpoint{3.739210in}{0.984436in}}%
\pgfpathlineto{\pgfqpoint{3.753761in}{0.983195in}}%
\pgfpathlineto{\pgfqpoint{3.768131in}{0.981976in}}%
\pgfpathlineto{\pgfqpoint{3.782324in}{0.980778in}}%
\pgfpathlineto{\pgfqpoint{3.796346in}{0.979601in}}%
\pgfpathlineto{\pgfqpoint{3.810200in}{0.978444in}}%
\pgfpathlineto{\pgfqpoint{3.823890in}{0.977307in}}%
\pgfpathlineto{\pgfqpoint{3.837419in}{0.976190in}}%
\pgfpathlineto{\pgfqpoint{3.850793in}{0.975090in}}%
\pgfpathlineto{\pgfqpoint{3.864013in}{0.974009in}}%
\pgfpathlineto{\pgfqpoint{3.877084in}{0.972946in}}%
\pgfpathlineto{\pgfqpoint{3.890010in}{0.971900in}}%
\pgfpathlineto{\pgfqpoint{3.902792in}{0.970870in}}%
\pgfpathlineto{\pgfqpoint{3.915435in}{0.969857in}}%
\pgfpathlineto{\pgfqpoint{3.927941in}{0.968860in}}%
\pgfpathlineto{\pgfqpoint{3.940313in}{0.967878in}}%
\pgfpathlineto{\pgfqpoint{3.952554in}{0.966911in}}%
\pgfpathlineto{\pgfqpoint{3.964667in}{0.965958in}}%
\pgfpathlineto{\pgfqpoint{3.976655in}{0.965020in}}%
\pgfpathlineto{\pgfqpoint{3.988520in}{0.964096in}}%
\pgfpathlineto{\pgfqpoint{4.000264in}{0.963186in}}%
\pgfpathlineto{\pgfqpoint{4.011890in}{0.962289in}}%
\pgfusepath{stroke}%
\end{pgfscope}%
\begin{pgfscope}%
\pgfpathrectangle{\pgfqpoint{0.557699in}{0.629134in}}{\pgfqpoint{3.720000in}{3.020000in}}%
\pgfusepath{clip}%
\pgfsetrectcap%
\pgfsetroundjoin%
\pgfsetlinewidth{1.505625pt}%
\definecolor{currentstroke}{rgb}{0.621569,0.981823,0.636474}%
\pgfsetstrokecolor{currentstroke}%
\pgfsetdash{}{0pt}%
\pgfpathmoveto{\pgfqpoint{0.554366in}{1.543011in}}%
\pgfpathlineto{\pgfqpoint{0.758851in}{1.504803in}}%
\pgfpathlineto{\pgfqpoint{0.936910in}{1.471477in}}%
\pgfpathlineto{\pgfqpoint{1.091184in}{1.442724in}}%
\pgfpathlineto{\pgfqpoint{1.227286in}{1.417557in}}%
\pgfpathlineto{\pgfqpoint{1.349048in}{1.395270in}}%
\pgfpathlineto{\pgfqpoint{1.459207in}{1.375341in}}%
\pgfpathlineto{\pgfqpoint{1.559783in}{1.357376in}}%
\pgfpathlineto{\pgfqpoint{1.652310in}{1.341067in}}%
\pgfpathlineto{\pgfqpoint{1.737982in}{1.326170in}}%
\pgfpathlineto{\pgfqpoint{1.817745in}{1.312492in}}%
\pgfpathlineto{\pgfqpoint{1.892362in}{1.299871in}}%
\pgfpathlineto{\pgfqpoint{1.962456in}{1.288179in}}%
\pgfpathlineto{\pgfqpoint{2.028545in}{1.277304in}}%
\pgfpathlineto{\pgfqpoint{2.091062in}{1.267156in}}%
\pgfpathlineto{\pgfqpoint{2.150373in}{1.257656in}}%
\pgfpathlineto{\pgfqpoint{2.206792in}{1.248737in}}%
\pgfpathlineto{\pgfqpoint{2.260586in}{1.240343in}}%
\pgfpathlineto{\pgfqpoint{2.311990in}{1.232424in}}%
\pgfpathlineto{\pgfqpoint{2.361206in}{1.224936in}}%
\pgfpathlineto{\pgfqpoint{2.408414in}{1.217841in}}%
\pgfpathlineto{\pgfqpoint{2.453771in}{1.211106in}}%
\pgfpathlineto{\pgfqpoint{2.497417in}{1.204702in}}%
\pgfpathlineto{\pgfqpoint{2.539476in}{1.198601in}}%
\pgfpathlineto{\pgfqpoint{2.580059in}{1.192782in}}%
\pgfpathlineto{\pgfqpoint{2.619267in}{1.187221in}}%
\pgfpathlineto{\pgfqpoint{2.657189in}{1.181902in}}%
\pgfpathlineto{\pgfqpoint{2.693908in}{1.176807in}}%
\pgfpathlineto{\pgfqpoint{2.729497in}{1.171921in}}%
\pgfpathlineto{\pgfqpoint{2.764024in}{1.167229in}}%
\pgfpathlineto{\pgfqpoint{2.797550in}{1.162719in}}%
\pgfpathlineto{\pgfqpoint{2.830133in}{1.158380in}}%
\pgfpathlineto{\pgfqpoint{2.861822in}{1.154201in}}%
\pgfpathlineto{\pgfqpoint{2.892667in}{1.150172in}}%
\pgfpathlineto{\pgfqpoint{2.922710in}{1.146285in}}%
\pgfpathlineto{\pgfqpoint{2.951993in}{1.142531in}}%
\pgfpathlineto{\pgfqpoint{2.980553in}{1.138904in}}%
\pgfpathlineto{\pgfqpoint{3.008425in}{1.135395in}}%
\pgfpathlineto{\pgfqpoint{3.035642in}{1.132000in}}%
\pgfpathlineto{\pgfqpoint{3.062233in}{1.128711in}}%
\pgfpathlineto{\pgfqpoint{3.088226in}{1.125523in}}%
\pgfpathlineto{\pgfqpoint{3.113648in}{1.122432in}}%
\pgfpathlineto{\pgfqpoint{3.138523in}{1.119432in}}%
\pgfpathlineto{\pgfqpoint{3.162875in}{1.116519in}}%
\pgfpathlineto{\pgfqpoint{3.186725in}{1.113688in}}%
\pgfpathlineto{\pgfqpoint{3.210093in}{1.110937in}}%
\pgfpathlineto{\pgfqpoint{3.232999in}{1.108262in}}%
\pgfpathlineto{\pgfqpoint{3.255459in}{1.105658in}}%
\pgfpathlineto{\pgfqpoint{3.277492in}{1.103123in}}%
\pgfpathlineto{\pgfqpoint{3.299113in}{1.100654in}}%
\pgfpathlineto{\pgfqpoint{3.320338in}{1.098248in}}%
\pgfpathlineto{\pgfqpoint{3.341180in}{1.095903in}}%
\pgfpathlineto{\pgfqpoint{3.361653in}{1.093615in}}%
\pgfpathlineto{\pgfqpoint{3.381771in}{1.091383in}}%
\pgfpathlineto{\pgfqpoint{3.401544in}{1.089205in}}%
\pgfpathlineto{\pgfqpoint{3.420985in}{1.087077in}}%
\pgfpathlineto{\pgfqpoint{3.440105in}{1.084999in}}%
\pgfpathlineto{\pgfqpoint{3.458914in}{1.082969in}}%
\pgfpathlineto{\pgfqpoint{3.477422in}{1.080984in}}%
\pgfpathlineto{\pgfqpoint{3.495639in}{1.079042in}}%
\pgfpathlineto{\pgfqpoint{3.513573in}{1.077144in}}%
\pgfpathlineto{\pgfqpoint{3.531233in}{1.075286in}}%
\pgfpathlineto{\pgfqpoint{3.548628in}{1.073467in}}%
\pgfpathlineto{\pgfqpoint{3.565765in}{1.071687in}}%
\pgfpathlineto{\pgfqpoint{3.582653in}{1.069943in}}%
\pgfpathlineto{\pgfqpoint{3.599297in}{1.068234in}}%
\pgfpathlineto{\pgfqpoint{3.615705in}{1.066560in}}%
\pgfpathlineto{\pgfqpoint{3.631883in}{1.064919in}}%
\pgfpathlineto{\pgfqpoint{3.647839in}{1.063311in}}%
\pgfpathlineto{\pgfqpoint{3.663577in}{1.061733in}}%
\pgfpathlineto{\pgfqpoint{3.679105in}{1.060185in}}%
\pgfpathlineto{\pgfqpoint{3.694426in}{1.058666in}}%
\pgfpathlineto{\pgfqpoint{3.709548in}{1.057176in}}%
\pgfpathlineto{\pgfqpoint{3.724474in}{1.055712in}}%
\pgfpathlineto{\pgfqpoint{3.739210in}{1.054276in}}%
\pgfpathlineto{\pgfqpoint{3.753761in}{1.052865in}}%
\pgfpathlineto{\pgfqpoint{3.768131in}{1.051479in}}%
\pgfpathlineto{\pgfqpoint{3.782324in}{1.050117in}}%
\pgfpathlineto{\pgfqpoint{3.796346in}{1.048779in}}%
\pgfpathlineto{\pgfqpoint{3.810200in}{1.047464in}}%
\pgfpathlineto{\pgfqpoint{3.823890in}{1.046170in}}%
\pgfpathlineto{\pgfqpoint{3.837419in}{1.044899in}}%
\pgfpathlineto{\pgfqpoint{3.850793in}{1.043648in}}%
\pgfpathlineto{\pgfqpoint{3.864013in}{1.042418in}}%
\pgfpathlineto{\pgfqpoint{3.877084in}{1.041208in}}%
\pgfpathlineto{\pgfqpoint{3.890010in}{1.040017in}}%
\pgfpathlineto{\pgfqpoint{3.902792in}{1.038845in}}%
\pgfpathlineto{\pgfqpoint{3.915435in}{1.037692in}}%
\pgfpathlineto{\pgfqpoint{3.927941in}{1.036556in}}%
\pgfpathlineto{\pgfqpoint{3.940313in}{1.035438in}}%
\pgfpathlineto{\pgfqpoint{3.952554in}{1.034336in}}%
\pgfpathlineto{\pgfqpoint{3.964667in}{1.033251in}}%
\pgfpathlineto{\pgfqpoint{3.976655in}{1.032183in}}%
\pgfpathlineto{\pgfqpoint{3.988520in}{1.031130in}}%
\pgfpathlineto{\pgfqpoint{4.000264in}{1.030092in}}%
\pgfpathlineto{\pgfqpoint{4.011890in}{1.029070in}}%
\pgfusepath{stroke}%
\end{pgfscope}%
\begin{pgfscope}%
\pgfpathrectangle{\pgfqpoint{0.557699in}{0.629134in}}{\pgfqpoint{3.720000in}{3.020000in}}%
\pgfusepath{clip}%
\pgfsetrectcap%
\pgfsetroundjoin%
\pgfsetlinewidth{1.505625pt}%
\definecolor{currentstroke}{rgb}{1.000000,0.645928,0.343949}%
\pgfsetstrokecolor{currentstroke}%
\pgfsetdash{}{0pt}%
\pgfpathmoveto{\pgfqpoint{0.554366in}{2.114747in}}%
\pgfpathlineto{\pgfqpoint{0.758851in}{2.064776in}}%
\pgfpathlineto{\pgfqpoint{0.936910in}{2.021649in}}%
\pgfpathlineto{\pgfqpoint{1.091184in}{1.984770in}}%
\pgfpathlineto{\pgfqpoint{1.227286in}{1.952736in}}%
\pgfpathlineto{\pgfqpoint{1.349048in}{1.924555in}}%
\pgfpathlineto{\pgfqpoint{1.459207in}{1.899502in}}%
\pgfpathlineto{\pgfqpoint{1.559783in}{1.877032in}}%
\pgfpathlineto{\pgfqpoint{1.652310in}{1.856725in}}%
\pgfpathlineto{\pgfqpoint{1.737982in}{1.838253in}}%
\pgfpathlineto{\pgfqpoint{1.817745in}{1.821352in}}%
\pgfpathlineto{\pgfqpoint{1.892362in}{1.805810in}}%
\pgfpathlineto{\pgfqpoint{1.962456in}{1.791454in}}%
\pgfpathlineto{\pgfqpoint{2.028545in}{1.778138in}}%
\pgfpathlineto{\pgfqpoint{2.091062in}{1.765744in}}%
\pgfpathlineto{\pgfqpoint{2.150373in}{1.754167in}}%
\pgfpathlineto{\pgfqpoint{2.206792in}{1.743323in}}%
\pgfpathlineto{\pgfqpoint{2.260586in}{1.733137in}}%
\pgfpathlineto{\pgfqpoint{2.311990in}{1.723544in}}%
\pgfpathlineto{\pgfqpoint{2.361206in}{1.714488in}}%
\pgfpathlineto{\pgfqpoint{2.408414in}{1.705922in}}%
\pgfpathlineto{\pgfqpoint{2.453771in}{1.697803in}}%
\pgfpathlineto{\pgfqpoint{2.497417in}{1.690093in}}%
\pgfpathlineto{\pgfqpoint{2.539476in}{1.682758in}}%
\pgfpathlineto{\pgfqpoint{2.580059in}{1.675770in}}%
\pgfpathlineto{\pgfqpoint{2.619267in}{1.669101in}}%
\pgfpathlineto{\pgfqpoint{2.657189in}{1.662728in}}%
\pgfpathlineto{\pgfqpoint{2.693908in}{1.656630in}}%
\pgfpathlineto{\pgfqpoint{2.729497in}{1.650787in}}%
\pgfpathlineto{\pgfqpoint{2.764024in}{1.645182in}}%
\pgfpathlineto{\pgfqpoint{2.797550in}{1.639800in}}%
\pgfpathlineto{\pgfqpoint{2.830133in}{1.634625in}}%
\pgfpathlineto{\pgfqpoint{2.861822in}{1.629646in}}%
\pgfpathlineto{\pgfqpoint{2.892667in}{1.624849in}}%
\pgfpathlineto{\pgfqpoint{2.922710in}{1.620224in}}%
\pgfpathlineto{\pgfqpoint{2.951993in}{1.615762in}}%
\pgfpathlineto{\pgfqpoint{2.980553in}{1.611452in}}%
\pgfpathlineto{\pgfqpoint{3.008425in}{1.607286in}}%
\pgfpathlineto{\pgfqpoint{3.035642in}{1.603256in}}%
\pgfpathlineto{\pgfqpoint{3.062233in}{1.599356in}}%
\pgfpathlineto{\pgfqpoint{3.088226in}{1.595577in}}%
\pgfpathlineto{\pgfqpoint{3.113648in}{1.591915in}}%
\pgfpathlineto{\pgfqpoint{3.138523in}{1.588363in}}%
\pgfpathlineto{\pgfqpoint{3.162875in}{1.584916in}}%
\pgfpathlineto{\pgfqpoint{3.186725in}{1.581568in}}%
\pgfpathlineto{\pgfqpoint{3.210093in}{1.578315in}}%
\pgfpathlineto{\pgfqpoint{3.232999in}{1.575153in}}%
\pgfpathlineto{\pgfqpoint{3.255459in}{1.572078in}}%
\pgfpathlineto{\pgfqpoint{3.277492in}{1.569085in}}%
\pgfpathlineto{\pgfqpoint{3.299113in}{1.566171in}}%
\pgfpathlineto{\pgfqpoint{3.320338in}{1.563332in}}%
\pgfpathlineto{\pgfqpoint{3.341180in}{1.560566in}}%
\pgfpathlineto{\pgfqpoint{3.361653in}{1.557869in}}%
\pgfpathlineto{\pgfqpoint{3.381771in}{1.555238in}}%
\pgfpathlineto{\pgfqpoint{3.401544in}{1.552672in}}%
\pgfpathlineto{\pgfqpoint{3.420985in}{1.550166in}}%
\pgfpathlineto{\pgfqpoint{3.440105in}{1.547720in}}%
\pgfpathlineto{\pgfqpoint{3.458914in}{1.545329in}}%
\pgfpathlineto{\pgfqpoint{3.477422in}{1.542994in}}%
\pgfpathlineto{\pgfqpoint{3.495639in}{1.540710in}}%
\pgfpathlineto{\pgfqpoint{3.513573in}{1.538478in}}%
\pgfpathlineto{\pgfqpoint{3.531233in}{1.536293in}}%
\pgfpathlineto{\pgfqpoint{3.548628in}{1.534156in}}%
\pgfpathlineto{\pgfqpoint{3.565765in}{1.532064in}}%
\pgfpathlineto{\pgfqpoint{3.582653in}{1.530015in}}%
\pgfpathlineto{\pgfqpoint{3.599297in}{1.528009in}}%
\pgfpathlineto{\pgfqpoint{3.615705in}{1.526044in}}%
\pgfpathlineto{\pgfqpoint{3.631883in}{1.524117in}}%
\pgfpathlineto{\pgfqpoint{3.647839in}{1.522229in}}%
\pgfpathlineto{\pgfqpoint{3.663577in}{1.520378in}}%
\pgfpathlineto{\pgfqpoint{3.679105in}{1.518562in}}%
\pgfpathlineto{\pgfqpoint{3.694426in}{1.516781in}}%
\pgfpathlineto{\pgfqpoint{3.709548in}{1.515033in}}%
\pgfpathlineto{\pgfqpoint{3.724474in}{1.513318in}}%
\pgfpathlineto{\pgfqpoint{3.739210in}{1.511634in}}%
\pgfpathlineto{\pgfqpoint{3.753761in}{1.509981in}}%
\pgfpathlineto{\pgfqpoint{3.768131in}{1.508357in}}%
\pgfpathlineto{\pgfqpoint{3.782324in}{1.506762in}}%
\pgfpathlineto{\pgfqpoint{3.796346in}{1.505194in}}%
\pgfpathlineto{\pgfqpoint{3.810200in}{1.503654in}}%
\pgfpathlineto{\pgfqpoint{3.823890in}{1.502140in}}%
\pgfpathlineto{\pgfqpoint{3.837419in}{1.500652in}}%
\pgfpathlineto{\pgfqpoint{3.850793in}{1.499188in}}%
\pgfpathlineto{\pgfqpoint{3.864013in}{1.497749in}}%
\pgfpathlineto{\pgfqpoint{3.877084in}{1.496333in}}%
\pgfpathlineto{\pgfqpoint{3.890010in}{1.494940in}}%
\pgfpathlineto{\pgfqpoint{3.902792in}{1.493569in}}%
\pgfpathlineto{\pgfqpoint{3.915435in}{1.492220in}}%
\pgfpathlineto{\pgfqpoint{3.927941in}{1.490892in}}%
\pgfpathlineto{\pgfqpoint{3.940313in}{1.489584in}}%
\pgfpathlineto{\pgfqpoint{3.952554in}{1.488297in}}%
\pgfpathlineto{\pgfqpoint{3.964667in}{1.487029in}}%
\pgfpathlineto{\pgfqpoint{3.976655in}{1.485780in}}%
\pgfpathlineto{\pgfqpoint{3.988520in}{1.484549in}}%
\pgfpathlineto{\pgfqpoint{4.000264in}{1.483337in}}%
\pgfpathlineto{\pgfqpoint{4.011890in}{1.482142in}}%
\pgfusepath{stroke}%
\end{pgfscope}%
\begin{pgfscope}%
\pgfsetrectcap%
\pgfsetmiterjoin%
\pgfsetlinewidth{0.803000pt}%
\definecolor{currentstroke}{rgb}{0.501961,0.501961,0.501961}%
\pgfsetstrokecolor{currentstroke}%
\pgfsetdash{}{0pt}%
\pgfpathmoveto{\pgfqpoint{0.557699in}{0.629134in}}%
\pgfpathlineto{\pgfqpoint{0.557699in}{3.649134in}}%
\pgfusepath{stroke}%
\end{pgfscope}%
\begin{pgfscope}%
\pgfsetrectcap%
\pgfsetmiterjoin%
\pgfsetlinewidth{0.803000pt}%
\definecolor{currentstroke}{rgb}{0.501961,0.501961,0.501961}%
\pgfsetstrokecolor{currentstroke}%
\pgfsetdash{}{0pt}%
\pgfpathmoveto{\pgfqpoint{4.277699in}{0.629134in}}%
\pgfpathlineto{\pgfqpoint{4.277699in}{3.649134in}}%
\pgfusepath{stroke}%
\end{pgfscope}%
\begin{pgfscope}%
\pgfsetrectcap%
\pgfsetmiterjoin%
\pgfsetlinewidth{0.803000pt}%
\definecolor{currentstroke}{rgb}{0.501961,0.501961,0.501961}%
\pgfsetstrokecolor{currentstroke}%
\pgfsetdash{}{0pt}%
\pgfpathmoveto{\pgfqpoint{0.557699in}{0.629134in}}%
\pgfpathlineto{\pgfqpoint{4.277699in}{0.629134in}}%
\pgfusepath{stroke}%
\end{pgfscope}%
\begin{pgfscope}%
\pgfsetrectcap%
\pgfsetmiterjoin%
\pgfsetlinewidth{0.803000pt}%
\definecolor{currentstroke}{rgb}{0.501961,0.501961,0.501961}%
\pgfsetstrokecolor{currentstroke}%
\pgfsetdash{}{0pt}%
\pgfpathmoveto{\pgfqpoint{0.557699in}{3.649134in}}%
\pgfpathlineto{\pgfqpoint{4.277699in}{3.649134in}}%
\pgfusepath{stroke}%
\end{pgfscope}%
\begin{pgfscope}%
\pgfsetbuttcap%
\pgfsetmiterjoin%
\definecolor{currentfill}{rgb}{1.000000,1.000000,1.000000}%
\pgfsetfillcolor{currentfill}%
\pgfsetfillopacity{0.800000}%
\pgfsetlinewidth{1.003750pt}%
\definecolor{currentstroke}{rgb}{0.800000,0.800000,0.800000}%
\pgfsetstrokecolor{currentstroke}%
\pgfsetstrokeopacity{0.800000}%
\pgfsetdash{}{0pt}%
\pgfpathmoveto{\pgfqpoint{0.693810in}{1.781177in}}%
\pgfpathlineto{\pgfqpoint{4.227988in}{1.781177in}}%
\pgfpathquadraticcurveto{\pgfqpoint{4.266877in}{1.781177in}}{\pgfqpoint{4.266877in}{1.820066in}}%
\pgfpathlineto{\pgfqpoint{4.266877in}{3.513023in}}%
\pgfpathquadraticcurveto{\pgfqpoint{4.266877in}{3.551912in}}{\pgfqpoint{4.227988in}{3.551912in}}%
\pgfpathlineto{\pgfqpoint{0.693810in}{3.551912in}}%
\pgfpathquadraticcurveto{\pgfqpoint{0.654921in}{3.551912in}}{\pgfqpoint{0.654921in}{3.513023in}}%
\pgfpathlineto{\pgfqpoint{0.654921in}{1.820066in}}%
\pgfpathquadraticcurveto{\pgfqpoint{0.654921in}{1.781177in}}{\pgfqpoint{0.693810in}{1.781177in}}%
\pgfpathclose%
\pgfusepath{stroke,fill}%
\end{pgfscope}%
\begin{pgfscope}%
\pgfsetbuttcap%
\pgfsetroundjoin%
\pgfsetlinewidth{1.505625pt}%
\definecolor{currentstroke}{rgb}{0.000000,0.000000,0.000000}%
\pgfsetstrokecolor{currentstroke}%
\pgfsetdash{{5.550000pt}{2.400000pt}}{0.000000pt}%
\pgfpathmoveto{\pgfqpoint{0.732699in}{3.394457in}}%
\pgfpathlineto{\pgfqpoint{1.121588in}{3.394457in}}%
\pgfusepath{stroke}%
\end{pgfscope}%
\begin{pgfscope}%
\definecolor{textcolor}{rgb}{0.501961,0.501961,0.501961}%
\pgfsetstrokecolor{textcolor}%
\pgfsetfillcolor{textcolor}%
\pgftext[x=1.277143in,y=3.326402in,left,base]{\color{textcolor}\rmfamily\fontsize{14.000000}{16.800000}\selectfont Richards, 2001}%
\end{pgfscope}%
\begin{pgfscope}%
\pgfsetbuttcap%
\pgfsetroundjoin%
\pgfsetlinewidth{1.505625pt}%
\definecolor{currentstroke}{rgb}{0.501961,0.501961,0.501961}%
\pgfsetstrokecolor{currentstroke}%
\pgfsetdash{{9.600000pt}{2.400000pt}{1.500000pt}{2.400000pt}}{0.000000pt}%
\pgfpathmoveto{\pgfqpoint{0.732699in}{3.109057in}}%
\pgfpathlineto{\pgfqpoint{1.121588in}{3.109057in}}%
\pgfusepath{stroke}%
\end{pgfscope}%
\begin{pgfscope}%
\definecolor{textcolor}{rgb}{0.501961,0.501961,0.501961}%
\pgfsetstrokecolor{textcolor}%
\pgfsetfillcolor{textcolor}%
\pgftext[x=1.277143in,y=3.041001in,left,base]{\color{textcolor}\rmfamily\fontsize{14.000000}{16.800000}\selectfont Gopinath et al., 2002}%
\end{pgfscope}%
\begin{pgfscope}%
\pgfsetrectcap%
\pgfsetroundjoin%
\pgfsetlinewidth{1.505625pt}%
\definecolor{currentstroke}{rgb}{0.190196,0.883910,0.856638}%
\pgfsetstrokecolor{currentstroke}%
\pgfsetdash{}{0pt}%
\pgfpathmoveto{\pgfqpoint{0.732699in}{2.823657in}}%
\pgfpathlineto{\pgfqpoint{1.121588in}{2.823657in}}%
\pgfusepath{stroke}%
\end{pgfscope}%
\begin{pgfscope}%
\definecolor{textcolor}{rgb}{0.501961,0.501961,0.501961}%
\pgfsetstrokecolor{textcolor}%
\pgfsetfillcolor{textcolor}%
\pgftext[x=1.277143in,y=2.755601in,left,base]{\color{textcolor}\rmfamily\fontsize{14.000000}{16.800000}\selectfont Molacek et al., 2012, \(\displaystyle \mathbf{B}\mbox{o} = \)0.2}%
\end{pgfscope}%
\begin{pgfscope}%
\pgfsetrectcap%
\pgfsetroundjoin%
\pgfsetlinewidth{1.505625pt}%
\definecolor{currentstroke}{rgb}{0.190196,0.883910,0.856638}%
\pgfsetstrokecolor{currentstroke}%
\pgfsetdash{}{0pt}%
\pgfpathmoveto{\pgfqpoint{0.732699in}{2.538256in}}%
\pgfpathlineto{\pgfqpoint{1.121588in}{2.538256in}}%
\pgfusepath{stroke}%
\end{pgfscope}%
\begin{pgfscope}%
\definecolor{textcolor}{rgb}{0.501961,0.501961,0.501961}%
\pgfsetstrokecolor{textcolor}%
\pgfsetfillcolor{textcolor}%
\pgftext[x=1.277143in,y=2.470201in,left,base]{\color{textcolor}\rmfamily\fontsize{14.000000}{16.800000}\selectfont ..., \(\displaystyle \mathbf{B}\mbox{o} = \)0.4}%
\end{pgfscope}%
\begin{pgfscope}%
\pgfsetrectcap%
\pgfsetroundjoin%
\pgfsetlinewidth{1.505625pt}%
\definecolor{currentstroke}{rgb}{0.621569,0.981823,0.636474}%
\pgfsetstrokecolor{currentstroke}%
\pgfsetdash{}{0pt}%
\pgfpathmoveto{\pgfqpoint{0.732699in}{2.252856in}}%
\pgfpathlineto{\pgfqpoint{1.121588in}{2.252856in}}%
\pgfusepath{stroke}%
\end{pgfscope}%
\begin{pgfscope}%
\definecolor{textcolor}{rgb}{0.501961,0.501961,0.501961}%
\pgfsetstrokecolor{textcolor}%
\pgfsetfillcolor{textcolor}%
\pgftext[x=1.277143in,y=2.184801in,left,base]{\color{textcolor}\rmfamily\fontsize{14.000000}{16.800000}\selectfont ..., \(\displaystyle \mathbf{B}\mbox{o} = \)0.6}%
\end{pgfscope}%
\begin{pgfscope}%
\pgfsetrectcap%
\pgfsetroundjoin%
\pgfsetlinewidth{1.505625pt}%
\definecolor{currentstroke}{rgb}{1.000000,0.645928,0.343949}%
\pgfsetstrokecolor{currentstroke}%
\pgfsetdash{}{0pt}%
\pgfpathmoveto{\pgfqpoint{0.732699in}{1.967456in}}%
\pgfpathlineto{\pgfqpoint{1.121588in}{1.967456in}}%
\pgfusepath{stroke}%
\end{pgfscope}%
\begin{pgfscope}%
\definecolor{textcolor}{rgb}{0.501961,0.501961,0.501961}%
\pgfsetstrokecolor{textcolor}%
\pgfsetfillcolor{textcolor}%
\pgftext[x=1.277143in,y=1.899400in,left,base]{\color{textcolor}\rmfamily\fontsize{14.000000}{16.800000}\selectfont ..., \(\displaystyle \mathbf{B}\mbox{o} = \)0.8}%
\end{pgfscope}%
\begin{pgfscope}%
\pgfpathrectangle{\pgfqpoint{4.510199in}{0.629134in}}{\pgfqpoint{0.151000in}{3.020000in}}%
\pgfusepath{clip}%
\pgfsetbuttcap%
\pgfsetmiterjoin%
\definecolor{currentfill}{rgb}{1.000000,1.000000,1.000000}%
\pgfsetfillcolor{currentfill}%
\pgfsetlinewidth{0.010037pt}%
\definecolor{currentstroke}{rgb}{1.000000,1.000000,1.000000}%
\pgfsetstrokecolor{currentstroke}%
\pgfsetdash{}{0pt}%
\pgfpathmoveto{\pgfqpoint{4.510199in}{0.629134in}}%
\pgfpathlineto{\pgfqpoint{4.510199in}{0.640931in}}%
\pgfpathlineto{\pgfqpoint{4.510199in}{3.637337in}}%
\pgfpathlineto{\pgfqpoint{4.510199in}{3.649134in}}%
\pgfpathlineto{\pgfqpoint{4.661199in}{3.649134in}}%
\pgfpathlineto{\pgfqpoint{4.661199in}{3.637337in}}%
\pgfpathlineto{\pgfqpoint{4.661199in}{0.640931in}}%
\pgfpathlineto{\pgfqpoint{4.661199in}{0.629134in}}%
\pgfpathclose%
\pgfusepath{stroke,fill}%
\end{pgfscope}%
\begin{pgfscope}%
\pgfsys@transformshift{4.510000in}{0.630756in}%
\pgftext[left,bottom]{\pgfimage[interpolate=true,width=0.150000in,height=3.020000in]{contact2-img0.png}}%
\end{pgfscope}%
\begin{pgfscope}%
\pgfsetbuttcap%
\pgfsetroundjoin%
\definecolor{currentfill}{rgb}{0.000000,0.000000,0.000000}%
\pgfsetfillcolor{currentfill}%
\pgfsetlinewidth{0.803000pt}%
\definecolor{currentstroke}{rgb}{0.000000,0.000000,0.000000}%
\pgfsetstrokecolor{currentstroke}%
\pgfsetdash{}{0pt}%
\pgfsys@defobject{currentmarker}{\pgfqpoint{0.000000in}{0.000000in}}{\pgfqpoint{0.048611in}{0.000000in}}{%
\pgfpathmoveto{\pgfqpoint{0.000000in}{0.000000in}}%
\pgfpathlineto{\pgfqpoint{0.048611in}{0.000000in}}%
\pgfusepath{stroke,fill}%
}%
\begin{pgfscope}%
\pgfsys@transformshift{4.661199in}{1.017743in}%
\pgfsys@useobject{currentmarker}{}%
\end{pgfscope}%
\end{pgfscope}%
\begin{pgfscope}%
\definecolor{textcolor}{rgb}{0.000000,0.000000,0.000000}%
\pgfsetstrokecolor{textcolor}%
\pgfsetfillcolor{textcolor}%
\pgftext[x=4.758421in,y=0.943877in,left,base]{\color{textcolor}\rmfamily\fontsize{14.000000}{16.800000}\selectfont \(\displaystyle 0.2\)}%
\end{pgfscope}%
\begin{pgfscope}%
\pgfsetbuttcap%
\pgfsetroundjoin%
\definecolor{currentfill}{rgb}{0.000000,0.000000,0.000000}%
\pgfsetfillcolor{currentfill}%
\pgfsetlinewidth{0.803000pt}%
\definecolor{currentstroke}{rgb}{0.000000,0.000000,0.000000}%
\pgfsetstrokecolor{currentstroke}%
\pgfsetdash{}{0pt}%
\pgfsys@defobject{currentmarker}{\pgfqpoint{0.000000in}{0.000000in}}{\pgfqpoint{0.048611in}{0.000000in}}{%
\pgfpathmoveto{\pgfqpoint{0.000000in}{0.000000in}}%
\pgfpathlineto{\pgfqpoint{0.048611in}{0.000000in}}%
\pgfusepath{stroke,fill}%
}%
\begin{pgfscope}%
\pgfsys@transformshift{4.661199in}{1.667530in}%
\pgfsys@useobject{currentmarker}{}%
\end{pgfscope}%
\end{pgfscope}%
\begin{pgfscope}%
\definecolor{textcolor}{rgb}{0.000000,0.000000,0.000000}%
\pgfsetstrokecolor{textcolor}%
\pgfsetfillcolor{textcolor}%
\pgftext[x=4.758421in,y=1.593663in,left,base]{\color{textcolor}\rmfamily\fontsize{14.000000}{16.800000}\selectfont \(\displaystyle 0.4\)}%
\end{pgfscope}%
\begin{pgfscope}%
\pgfsetbuttcap%
\pgfsetroundjoin%
\definecolor{currentfill}{rgb}{0.000000,0.000000,0.000000}%
\pgfsetfillcolor{currentfill}%
\pgfsetlinewidth{0.803000pt}%
\definecolor{currentstroke}{rgb}{0.000000,0.000000,0.000000}%
\pgfsetstrokecolor{currentstroke}%
\pgfsetdash{}{0pt}%
\pgfsys@defobject{currentmarker}{\pgfqpoint{0.000000in}{0.000000in}}{\pgfqpoint{0.048611in}{0.000000in}}{%
\pgfpathmoveto{\pgfqpoint{0.000000in}{0.000000in}}%
\pgfpathlineto{\pgfqpoint{0.048611in}{0.000000in}}%
\pgfusepath{stroke,fill}%
}%
\begin{pgfscope}%
\pgfsys@transformshift{4.661199in}{2.317316in}%
\pgfsys@useobject{currentmarker}{}%
\end{pgfscope}%
\end{pgfscope}%
\begin{pgfscope}%
\definecolor{textcolor}{rgb}{0.000000,0.000000,0.000000}%
\pgfsetstrokecolor{textcolor}%
\pgfsetfillcolor{textcolor}%
\pgftext[x=4.758421in,y=2.243450in,left,base]{\color{textcolor}\rmfamily\fontsize{14.000000}{16.800000}\selectfont \(\displaystyle 0.6\)}%
\end{pgfscope}%
\begin{pgfscope}%
\pgfsetbuttcap%
\pgfsetroundjoin%
\definecolor{currentfill}{rgb}{0.000000,0.000000,0.000000}%
\pgfsetfillcolor{currentfill}%
\pgfsetlinewidth{0.803000pt}%
\definecolor{currentstroke}{rgb}{0.000000,0.000000,0.000000}%
\pgfsetstrokecolor{currentstroke}%
\pgfsetdash{}{0pt}%
\pgfsys@defobject{currentmarker}{\pgfqpoint{0.000000in}{0.000000in}}{\pgfqpoint{0.048611in}{0.000000in}}{%
\pgfpathmoveto{\pgfqpoint{0.000000in}{0.000000in}}%
\pgfpathlineto{\pgfqpoint{0.048611in}{0.000000in}}%
\pgfusepath{stroke,fill}%
}%
\begin{pgfscope}%
\pgfsys@transformshift{4.661199in}{2.967103in}%
\pgfsys@useobject{currentmarker}{}%
\end{pgfscope}%
\end{pgfscope}%
\begin{pgfscope}%
\definecolor{textcolor}{rgb}{0.000000,0.000000,0.000000}%
\pgfsetstrokecolor{textcolor}%
\pgfsetfillcolor{textcolor}%
\pgftext[x=4.758421in,y=2.893237in,left,base]{\color{textcolor}\rmfamily\fontsize{14.000000}{16.800000}\selectfont \(\displaystyle 0.8\)}%
\end{pgfscope}%
\begin{pgfscope}%
\pgfsetbuttcap%
\pgfsetroundjoin%
\definecolor{currentfill}{rgb}{0.000000,0.000000,0.000000}%
\pgfsetfillcolor{currentfill}%
\pgfsetlinewidth{0.803000pt}%
\definecolor{currentstroke}{rgb}{0.000000,0.000000,0.000000}%
\pgfsetstrokecolor{currentstroke}%
\pgfsetdash{}{0pt}%
\pgfsys@defobject{currentmarker}{\pgfqpoint{0.000000in}{0.000000in}}{\pgfqpoint{0.048611in}{0.000000in}}{%
\pgfpathmoveto{\pgfqpoint{0.000000in}{0.000000in}}%
\pgfpathlineto{\pgfqpoint{0.048611in}{0.000000in}}%
\pgfusepath{stroke,fill}%
}%
\begin{pgfscope}%
\pgfsys@transformshift{4.661199in}{3.616890in}%
\pgfsys@useobject{currentmarker}{}%
\end{pgfscope}%
\end{pgfscope}%
\begin{pgfscope}%
\definecolor{textcolor}{rgb}{0.000000,0.000000,0.000000}%
\pgfsetstrokecolor{textcolor}%
\pgfsetfillcolor{textcolor}%
\pgftext[x=4.758421in,y=3.543024in,left,base]{\color{textcolor}\rmfamily\fontsize{14.000000}{16.800000}\selectfont \(\displaystyle 1.0\)}%
\end{pgfscope}%
\begin{pgfscope}%
\definecolor{textcolor}{rgb}{0.000000,0.000000,0.000000}%
\pgfsetstrokecolor{textcolor}%
\pgfsetfillcolor{textcolor}%
\pgftext[x=5.064205in,y=2.139134in,,top,rotate=90.000000]{\color{textcolor}\rmfamily\fontsize{14.000000}{16.800000}\selectfont \(\displaystyle \mathrm{\mathit{Bo_e}} \equiv \frac{\epsilon E_0^2 R_d}{\gamma}\)}%
\end{pgfscope}%
\begin{pgfscope}%
\pgfsetbuttcap%
\pgfsetmiterjoin%
\pgfsetlinewidth{0.803000pt}%
\definecolor{currentstroke}{rgb}{0.501961,0.501961,0.501961}%
\pgfsetstrokecolor{currentstroke}%
\pgfsetdash{}{0pt}%
\pgfpathmoveto{\pgfqpoint{4.510199in}{0.629134in}}%
\pgfpathlineto{\pgfqpoint{4.510199in}{0.640931in}}%
\pgfpathlineto{\pgfqpoint{4.510199in}{3.637337in}}%
\pgfpathlineto{\pgfqpoint{4.510199in}{3.649134in}}%
\pgfpathlineto{\pgfqpoint{4.661199in}{3.649134in}}%
\pgfpathlineto{\pgfqpoint{4.661199in}{3.637337in}}%
\pgfpathlineto{\pgfqpoint{4.661199in}{0.640931in}}%
\pgfpathlineto{\pgfqpoint{4.661199in}{0.629134in}}%
\pgfpathclose%
\pgfusepath{stroke}%
\end{pgfscope}%
\end{pgfpicture}%
\makeatother%
\endgroup%
}
    \caption{Dimensionless contact time $t_j/\tau$ compared with impact $\mathbb{W}\mbox{e}$ and $\mathbb{B}\mbox{o}_e$.\label{fig:contact}}
\end{figure}

\begin{figure}[htb]
    \centering
    \resizebox{0.5\textwidth}{!}{%% Creator: Matplotlib, PGF backend
%%
%% To include the figure in your LaTeX document, write
%%   \input{<filename>.pgf}
%%
%% Make sure the required packages are loaded in your preamble
%%   \usepackage{pgf}
%%
%% Figures using additional raster images can only be included by \input if
%% they are in the same directory as the main LaTeX file. For loading figures
%% from other directories you can use the `import` package
%%   \usepackage{import}
%% and then include the figures with
%%   \import{<path to file>}{<filename>.pgf}
%%
%% Matplotlib used the following preamble
%%   \usepackage{fontspec}
%%   \setmainfont{DejaVu Serif}
%%   \setsansfont{DejaVu Sans}
%%   \setmonofont{DejaVu Sans Mono}
%%
\begingroup%
\makeatletter%
\begin{pgfpicture}%
\pgfpathrectangle{\pgfpointorigin}{\pgfqpoint{5.427700in}{3.676603in}}%
\pgfusepath{use as bounding box, clip}%
\begin{pgfscope}%
\pgfsetbuttcap%
\pgfsetmiterjoin%
\definecolor{currentfill}{rgb}{1.000000,1.000000,1.000000}%
\pgfsetfillcolor{currentfill}%
\pgfsetlinewidth{0.000000pt}%
\definecolor{currentstroke}{rgb}{1.000000,1.000000,1.000000}%
\pgfsetstrokecolor{currentstroke}%
\pgfsetdash{}{0pt}%
\pgfpathmoveto{\pgfqpoint{0.000000in}{0.000000in}}%
\pgfpathlineto{\pgfqpoint{5.427700in}{0.000000in}}%
\pgfpathlineto{\pgfqpoint{5.427700in}{3.676603in}}%
\pgfpathlineto{\pgfqpoint{0.000000in}{3.676603in}}%
\pgfpathclose%
\pgfusepath{fill}%
\end{pgfscope}%
\begin{pgfscope}%
\pgfsetbuttcap%
\pgfsetmiterjoin%
\definecolor{currentfill}{rgb}{1.000000,1.000000,1.000000}%
\pgfsetfillcolor{currentfill}%
\pgfsetlinewidth{0.000000pt}%
\definecolor{currentstroke}{rgb}{0.000000,0.000000,0.000000}%
\pgfsetstrokecolor{currentstroke}%
\pgfsetstrokeopacity{0.000000}%
\pgfsetdash{}{0pt}%
\pgfpathmoveto{\pgfqpoint{0.564660in}{0.521603in}}%
\pgfpathlineto{\pgfqpoint{4.284660in}{0.521603in}}%
\pgfpathlineto{\pgfqpoint{4.284660in}{3.541603in}}%
\pgfpathlineto{\pgfqpoint{0.564660in}{3.541603in}}%
\pgfpathclose%
\pgfusepath{fill}%
\end{pgfscope}%
\begin{pgfscope}%
\pgfpathrectangle{\pgfqpoint{0.564660in}{0.521603in}}{\pgfqpoint{3.720000in}{3.020000in}} %
\pgfusepath{clip}%
\pgfsetbuttcap%
\pgfsetroundjoin%
\definecolor{currentfill}{rgb}{1.000000,0.255843,0.128999}%
\pgfsetfillcolor{currentfill}%
\pgfsetlinewidth{1.003750pt}%
\definecolor{currentstroke}{rgb}{1.000000,0.255843,0.128999}%
\pgfsetstrokecolor{currentstroke}%
\pgfsetdash{}{0pt}%
\pgfpathmoveto{\pgfqpoint{3.297585in}{2.468980in}}%
\pgfpathcurveto{\pgfqpoint{3.308636in}{2.468980in}}{\pgfqpoint{3.319235in}{2.473370in}}{\pgfqpoint{3.327048in}{2.481184in}}%
\pgfpathcurveto{\pgfqpoint{3.334862in}{2.488998in}}{\pgfqpoint{3.339252in}{2.499597in}}{\pgfqpoint{3.339252in}{2.510647in}}%
\pgfpathcurveto{\pgfqpoint{3.339252in}{2.521697in}}{\pgfqpoint{3.334862in}{2.532296in}}{\pgfqpoint{3.327048in}{2.540110in}}%
\pgfpathcurveto{\pgfqpoint{3.319235in}{2.547923in}}{\pgfqpoint{3.308636in}{2.552314in}}{\pgfqpoint{3.297585in}{2.552314in}}%
\pgfpathcurveto{\pgfqpoint{3.286535in}{2.552314in}}{\pgfqpoint{3.275936in}{2.547923in}}{\pgfqpoint{3.268123in}{2.540110in}}%
\pgfpathcurveto{\pgfqpoint{3.260309in}{2.532296in}}{\pgfqpoint{3.255919in}{2.521697in}}{\pgfqpoint{3.255919in}{2.510647in}}%
\pgfpathcurveto{\pgfqpoint{3.255919in}{2.499597in}}{\pgfqpoint{3.260309in}{2.488998in}}{\pgfqpoint{3.268123in}{2.481184in}}%
\pgfpathcurveto{\pgfqpoint{3.275936in}{2.473370in}}{\pgfqpoint{3.286535in}{2.468980in}}{\pgfqpoint{3.297585in}{2.468980in}}%
\pgfpathclose%
\pgfusepath{stroke,fill}%
\end{pgfscope}%
\begin{pgfscope}%
\pgfpathrectangle{\pgfqpoint{0.564660in}{0.521603in}}{\pgfqpoint{3.720000in}{3.020000in}} %
\pgfusepath{clip}%
\pgfsetbuttcap%
\pgfsetroundjoin%
\definecolor{currentfill}{rgb}{0.319608,0.279583,0.989980}%
\pgfsetfillcolor{currentfill}%
\pgfsetlinewidth{1.003750pt}%
\definecolor{currentstroke}{rgb}{0.319608,0.279583,0.989980}%
\pgfsetstrokecolor{currentstroke}%
\pgfsetdash{}{0pt}%
\pgfpathmoveto{\pgfqpoint{3.328555in}{1.873234in}}%
\pgfpathcurveto{\pgfqpoint{3.339606in}{1.873234in}}{\pgfqpoint{3.350205in}{1.877624in}}{\pgfqpoint{3.358018in}{1.885438in}}%
\pgfpathcurveto{\pgfqpoint{3.365832in}{1.893251in}}{\pgfqpoint{3.370222in}{1.903850in}}{\pgfqpoint{3.370222in}{1.914900in}}%
\pgfpathcurveto{\pgfqpoint{3.370222in}{1.925951in}}{\pgfqpoint{3.365832in}{1.936550in}}{\pgfqpoint{3.358018in}{1.944363in}}%
\pgfpathcurveto{\pgfqpoint{3.350205in}{1.952177in}}{\pgfqpoint{3.339606in}{1.956567in}}{\pgfqpoint{3.328555in}{1.956567in}}%
\pgfpathcurveto{\pgfqpoint{3.317505in}{1.956567in}}{\pgfqpoint{3.306906in}{1.952177in}}{\pgfqpoint{3.299093in}{1.944363in}}%
\pgfpathcurveto{\pgfqpoint{3.291279in}{1.936550in}}{\pgfqpoint{3.286889in}{1.925951in}}{\pgfqpoint{3.286889in}{1.914900in}}%
\pgfpathcurveto{\pgfqpoint{3.286889in}{1.903850in}}{\pgfqpoint{3.291279in}{1.893251in}}{\pgfqpoint{3.299093in}{1.885438in}}%
\pgfpathcurveto{\pgfqpoint{3.306906in}{1.877624in}}{\pgfqpoint{3.317505in}{1.873234in}}{\pgfqpoint{3.328555in}{1.873234in}}%
\pgfpathclose%
\pgfusepath{stroke,fill}%
\end{pgfscope}%
\begin{pgfscope}%
\pgfpathrectangle{\pgfqpoint{0.564660in}{0.521603in}}{\pgfqpoint{3.720000in}{3.020000in}} %
\pgfusepath{clip}%
\pgfsetbuttcap%
\pgfsetroundjoin%
\definecolor{currentfill}{rgb}{0.460784,0.061561,0.999526}%
\pgfsetfillcolor{currentfill}%
\pgfsetlinewidth{1.003750pt}%
\definecolor{currentstroke}{rgb}{0.460784,0.061561,0.999526}%
\pgfsetstrokecolor{currentstroke}%
\pgfsetdash{}{0pt}%
\pgfpathmoveto{\pgfqpoint{2.895722in}{1.799203in}}%
\pgfpathcurveto{\pgfqpoint{2.906772in}{1.799203in}}{\pgfqpoint{2.917371in}{1.803593in}}{\pgfqpoint{2.925185in}{1.811407in}}%
\pgfpathcurveto{\pgfqpoint{2.932998in}{1.819220in}}{\pgfqpoint{2.937389in}{1.829819in}}{\pgfqpoint{2.937389in}{1.840870in}}%
\pgfpathcurveto{\pgfqpoint{2.937389in}{1.851920in}}{\pgfqpoint{2.932998in}{1.862519in}}{\pgfqpoint{2.925185in}{1.870332in}}%
\pgfpathcurveto{\pgfqpoint{2.917371in}{1.878146in}}{\pgfqpoint{2.906772in}{1.882536in}}{\pgfqpoint{2.895722in}{1.882536in}}%
\pgfpathcurveto{\pgfqpoint{2.884672in}{1.882536in}}{\pgfqpoint{2.874073in}{1.878146in}}{\pgfqpoint{2.866259in}{1.870332in}}%
\pgfpathcurveto{\pgfqpoint{2.858445in}{1.862519in}}{\pgfqpoint{2.854055in}{1.851920in}}{\pgfqpoint{2.854055in}{1.840870in}}%
\pgfpathcurveto{\pgfqpoint{2.854055in}{1.829819in}}{\pgfqpoint{2.858445in}{1.819220in}}{\pgfqpoint{2.866259in}{1.811407in}}%
\pgfpathcurveto{\pgfqpoint{2.874073in}{1.803593in}}{\pgfqpoint{2.884672in}{1.799203in}}{\pgfqpoint{2.895722in}{1.799203in}}%
\pgfpathclose%
\pgfusepath{stroke,fill}%
\end{pgfscope}%
\begin{pgfscope}%
\pgfpathrectangle{\pgfqpoint{0.564660in}{0.521603in}}{\pgfqpoint{3.720000in}{3.020000in}} %
\pgfusepath{clip}%
\pgfsetbuttcap%
\pgfsetroundjoin%
\definecolor{currentfill}{rgb}{0.500000,0.000000,1.000000}%
\pgfsetfillcolor{currentfill}%
\pgfsetlinewidth{1.003750pt}%
\definecolor{currentstroke}{rgb}{0.500000,0.000000,1.000000}%
\pgfsetstrokecolor{currentstroke}%
\pgfsetdash{}{0pt}%
\pgfpathmoveto{\pgfqpoint{2.003834in}{2.169242in}}%
\pgfpathcurveto{\pgfqpoint{2.014884in}{2.169242in}}{\pgfqpoint{2.025483in}{2.173632in}}{\pgfqpoint{2.033297in}{2.181446in}}%
\pgfpathcurveto{\pgfqpoint{2.041110in}{2.189259in}}{\pgfqpoint{2.045500in}{2.199858in}}{\pgfqpoint{2.045500in}{2.210908in}}%
\pgfpathcurveto{\pgfqpoint{2.045500in}{2.221959in}}{\pgfqpoint{2.041110in}{2.232558in}}{\pgfqpoint{2.033297in}{2.240371in}}%
\pgfpathcurveto{\pgfqpoint{2.025483in}{2.248185in}}{\pgfqpoint{2.014884in}{2.252575in}}{\pgfqpoint{2.003834in}{2.252575in}}%
\pgfpathcurveto{\pgfqpoint{1.992784in}{2.252575in}}{\pgfqpoint{1.982185in}{2.248185in}}{\pgfqpoint{1.974371in}{2.240371in}}%
\pgfpathcurveto{\pgfqpoint{1.966557in}{2.232558in}}{\pgfqpoint{1.962167in}{2.221959in}}{\pgfqpoint{1.962167in}{2.210908in}}%
\pgfpathcurveto{\pgfqpoint{1.962167in}{2.199858in}}{\pgfqpoint{1.966557in}{2.189259in}}{\pgfqpoint{1.974371in}{2.181446in}}%
\pgfpathcurveto{\pgfqpoint{1.982185in}{2.173632in}}{\pgfqpoint{1.992784in}{2.169242in}}{\pgfqpoint{2.003834in}{2.169242in}}%
\pgfpathclose%
\pgfusepath{stroke,fill}%
\end{pgfscope}%
\begin{pgfscope}%
\pgfpathrectangle{\pgfqpoint{0.564660in}{0.521603in}}{\pgfqpoint{3.720000in}{3.020000in}} %
\pgfusepath{clip}%
\pgfsetbuttcap%
\pgfsetroundjoin%
\definecolor{currentfill}{rgb}{1.000000,0.000000,0.000000}%
\pgfsetfillcolor{currentfill}%
\pgfsetlinewidth{1.003750pt}%
\definecolor{currentstroke}{rgb}{1.000000,0.000000,0.000000}%
\pgfsetstrokecolor{currentstroke}%
\pgfsetdash{}{0pt}%
\pgfpathmoveto{\pgfqpoint{4.166276in}{1.932500in}}%
\pgfpathcurveto{\pgfqpoint{4.177326in}{1.932500in}}{\pgfqpoint{4.187926in}{1.936890in}}{\pgfqpoint{4.195739in}{1.944704in}}%
\pgfpathcurveto{\pgfqpoint{4.203553in}{1.952518in}}{\pgfqpoint{4.207943in}{1.963117in}}{\pgfqpoint{4.207943in}{1.974167in}}%
\pgfpathcurveto{\pgfqpoint{4.207943in}{1.985217in}}{\pgfqpoint{4.203553in}{1.995816in}}{\pgfqpoint{4.195739in}{2.003630in}}%
\pgfpathcurveto{\pgfqpoint{4.187926in}{2.011443in}}{\pgfqpoint{4.177326in}{2.015834in}}{\pgfqpoint{4.166276in}{2.015834in}}%
\pgfpathcurveto{\pgfqpoint{4.155226in}{2.015834in}}{\pgfqpoint{4.144627in}{2.011443in}}{\pgfqpoint{4.136814in}{2.003630in}}%
\pgfpathcurveto{\pgfqpoint{4.129000in}{1.995816in}}{\pgfqpoint{4.124610in}{1.985217in}}{\pgfqpoint{4.124610in}{1.974167in}}%
\pgfpathcurveto{\pgfqpoint{4.124610in}{1.963117in}}{\pgfqpoint{4.129000in}{1.952518in}}{\pgfqpoint{4.136814in}{1.944704in}}%
\pgfpathcurveto{\pgfqpoint{4.144627in}{1.936890in}}{\pgfqpoint{4.155226in}{1.932500in}}{\pgfqpoint{4.166276in}{1.932500in}}%
\pgfpathclose%
\pgfusepath{stroke,fill}%
\end{pgfscope}%
\begin{pgfscope}%
\pgfpathrectangle{\pgfqpoint{0.564660in}{0.521603in}}{\pgfqpoint{3.720000in}{3.020000in}} %
\pgfusepath{clip}%
\pgfsetbuttcap%
\pgfsetroundjoin%
\definecolor{currentfill}{rgb}{1.000000,0.000000,0.000000}%
\pgfsetfillcolor{currentfill}%
\pgfsetlinewidth{1.003750pt}%
\definecolor{currentstroke}{rgb}{1.000000,0.000000,0.000000}%
\pgfsetstrokecolor{currentstroke}%
\pgfsetdash{}{0pt}%
\pgfpathmoveto{\pgfqpoint{3.773192in}{2.692753in}}%
\pgfpathcurveto{\pgfqpoint{3.784242in}{2.692753in}}{\pgfqpoint{3.794841in}{2.697143in}}{\pgfqpoint{3.802655in}{2.704957in}}%
\pgfpathcurveto{\pgfqpoint{3.810468in}{2.712770in}}{\pgfqpoint{3.814859in}{2.723369in}}{\pgfqpoint{3.814859in}{2.734420in}}%
\pgfpathcurveto{\pgfqpoint{3.814859in}{2.745470in}}{\pgfqpoint{3.810468in}{2.756069in}}{\pgfqpoint{3.802655in}{2.763882in}}%
\pgfpathcurveto{\pgfqpoint{3.794841in}{2.771696in}}{\pgfqpoint{3.784242in}{2.776086in}}{\pgfqpoint{3.773192in}{2.776086in}}%
\pgfpathcurveto{\pgfqpoint{3.762142in}{2.776086in}}{\pgfqpoint{3.751543in}{2.771696in}}{\pgfqpoint{3.743729in}{2.763882in}}%
\pgfpathcurveto{\pgfqpoint{3.735915in}{2.756069in}}{\pgfqpoint{3.731525in}{2.745470in}}{\pgfqpoint{3.731525in}{2.734420in}}%
\pgfpathcurveto{\pgfqpoint{3.731525in}{2.723369in}}{\pgfqpoint{3.735915in}{2.712770in}}{\pgfqpoint{3.743729in}{2.704957in}}%
\pgfpathcurveto{\pgfqpoint{3.751543in}{2.697143in}}{\pgfqpoint{3.762142in}{2.692753in}}{\pgfqpoint{3.773192in}{2.692753in}}%
\pgfpathclose%
\pgfusepath{stroke,fill}%
\end{pgfscope}%
\begin{pgfscope}%
\pgfpathrectangle{\pgfqpoint{0.564660in}{0.521603in}}{\pgfqpoint{3.720000in}{3.020000in}} %
\pgfusepath{clip}%
\pgfsetbuttcap%
\pgfsetroundjoin%
\definecolor{currentfill}{rgb}{1.000000,0.171626,0.086133}%
\pgfsetfillcolor{currentfill}%
\pgfsetlinewidth{1.003750pt}%
\definecolor{currentstroke}{rgb}{1.000000,0.171626,0.086133}%
\pgfsetstrokecolor{currentstroke}%
\pgfsetdash{}{0pt}%
\pgfpathmoveto{\pgfqpoint{3.722241in}{2.302709in}}%
\pgfpathcurveto{\pgfqpoint{3.733292in}{2.302709in}}{\pgfqpoint{3.743891in}{2.307099in}}{\pgfqpoint{3.751704in}{2.314913in}}%
\pgfpathcurveto{\pgfqpoint{3.759518in}{2.322726in}}{\pgfqpoint{3.763908in}{2.333325in}}{\pgfqpoint{3.763908in}{2.344375in}}%
\pgfpathcurveto{\pgfqpoint{3.763908in}{2.355425in}}{\pgfqpoint{3.759518in}{2.366025in}}{\pgfqpoint{3.751704in}{2.373838in}}%
\pgfpathcurveto{\pgfqpoint{3.743891in}{2.381652in}}{\pgfqpoint{3.733292in}{2.386042in}}{\pgfqpoint{3.722241in}{2.386042in}}%
\pgfpathcurveto{\pgfqpoint{3.711191in}{2.386042in}}{\pgfqpoint{3.700592in}{2.381652in}}{\pgfqpoint{3.692779in}{2.373838in}}%
\pgfpathcurveto{\pgfqpoint{3.684965in}{2.366025in}}{\pgfqpoint{3.680575in}{2.355425in}}{\pgfqpoint{3.680575in}{2.344375in}}%
\pgfpathcurveto{\pgfqpoint{3.680575in}{2.333325in}}{\pgfqpoint{3.684965in}{2.322726in}}{\pgfqpoint{3.692779in}{2.314913in}}%
\pgfpathcurveto{\pgfqpoint{3.700592in}{2.307099in}}{\pgfqpoint{3.711191in}{2.302709in}}{\pgfqpoint{3.722241in}{2.302709in}}%
\pgfpathclose%
\pgfusepath{stroke,fill}%
\end{pgfscope}%
\begin{pgfscope}%
\pgfpathrectangle{\pgfqpoint{0.564660in}{0.521603in}}{\pgfqpoint{3.720000in}{3.020000in}} %
\pgfusepath{clip}%
\pgfsetbuttcap%
\pgfsetroundjoin%
\definecolor{currentfill}{rgb}{1.000000,0.171626,0.086133}%
\pgfsetfillcolor{currentfill}%
\pgfsetlinewidth{1.003750pt}%
\definecolor{currentstroke}{rgb}{1.000000,0.171626,0.086133}%
\pgfsetstrokecolor{currentstroke}%
\pgfsetdash{}{0pt}%
\pgfpathmoveto{\pgfqpoint{3.468726in}{0.678635in}}%
\pgfpathcurveto{\pgfqpoint{3.479776in}{0.678635in}}{\pgfqpoint{3.490375in}{0.683025in}}{\pgfqpoint{3.498189in}{0.690839in}}%
\pgfpathcurveto{\pgfqpoint{3.506002in}{0.698653in}}{\pgfqpoint{3.510392in}{0.709252in}}{\pgfqpoint{3.510392in}{0.720302in}}%
\pgfpathcurveto{\pgfqpoint{3.510392in}{0.731352in}}{\pgfqpoint{3.506002in}{0.741951in}}{\pgfqpoint{3.498189in}{0.749765in}}%
\pgfpathcurveto{\pgfqpoint{3.490375in}{0.757578in}}{\pgfqpoint{3.479776in}{0.761968in}}{\pgfqpoint{3.468726in}{0.761968in}}%
\pgfpathcurveto{\pgfqpoint{3.457676in}{0.761968in}}{\pgfqpoint{3.447077in}{0.757578in}}{\pgfqpoint{3.439263in}{0.749765in}}%
\pgfpathcurveto{\pgfqpoint{3.431449in}{0.741951in}}{\pgfqpoint{3.427059in}{0.731352in}}{\pgfqpoint{3.427059in}{0.720302in}}%
\pgfpathcurveto{\pgfqpoint{3.427059in}{0.709252in}}{\pgfqpoint{3.431449in}{0.698653in}}{\pgfqpoint{3.439263in}{0.690839in}}%
\pgfpathcurveto{\pgfqpoint{3.447077in}{0.683025in}}{\pgfqpoint{3.457676in}{0.678635in}}{\pgfqpoint{3.468726in}{0.678635in}}%
\pgfpathclose%
\pgfusepath{stroke,fill}%
\end{pgfscope}%
\begin{pgfscope}%
\pgfpathrectangle{\pgfqpoint{0.564660in}{0.521603in}}{\pgfqpoint{3.720000in}{3.020000in}} %
\pgfusepath{clip}%
\pgfsetbuttcap%
\pgfsetroundjoin%
\definecolor{currentfill}{rgb}{0.660784,0.968276,0.612420}%
\pgfsetfillcolor{currentfill}%
\pgfsetlinewidth{1.003750pt}%
\definecolor{currentstroke}{rgb}{0.660784,0.968276,0.612420}%
\pgfsetstrokecolor{currentstroke}%
\pgfsetdash{}{0pt}%
\pgfpathmoveto{\pgfqpoint{3.256354in}{0.943869in}}%
\pgfpathcurveto{\pgfqpoint{3.267405in}{0.943869in}}{\pgfqpoint{3.278004in}{0.948259in}}{\pgfqpoint{3.285817in}{0.956072in}}%
\pgfpathcurveto{\pgfqpoint{3.293631in}{0.963886in}}{\pgfqpoint{3.298021in}{0.974485in}}{\pgfqpoint{3.298021in}{0.985535in}}%
\pgfpathcurveto{\pgfqpoint{3.298021in}{0.996585in}}{\pgfqpoint{3.293631in}{1.007184in}}{\pgfqpoint{3.285817in}{1.014998in}}%
\pgfpathcurveto{\pgfqpoint{3.278004in}{1.022812in}}{\pgfqpoint{3.267405in}{1.027202in}}{\pgfqpoint{3.256354in}{1.027202in}}%
\pgfpathcurveto{\pgfqpoint{3.245304in}{1.027202in}}{\pgfqpoint{3.234705in}{1.022812in}}{\pgfqpoint{3.226892in}{1.014998in}}%
\pgfpathcurveto{\pgfqpoint{3.219078in}{1.007184in}}{\pgfqpoint{3.214688in}{0.996585in}}{\pgfqpoint{3.214688in}{0.985535in}}%
\pgfpathcurveto{\pgfqpoint{3.214688in}{0.974485in}}{\pgfqpoint{3.219078in}{0.963886in}}{\pgfqpoint{3.226892in}{0.956072in}}%
\pgfpathcurveto{\pgfqpoint{3.234705in}{0.948259in}}{\pgfqpoint{3.245304in}{0.943869in}}{\pgfqpoint{3.256354in}{0.943869in}}%
\pgfpathclose%
\pgfusepath{stroke,fill}%
\end{pgfscope}%
\begin{pgfscope}%
\pgfpathrectangle{\pgfqpoint{0.564660in}{0.521603in}}{\pgfqpoint{3.720000in}{3.020000in}} %
\pgfusepath{clip}%
\pgfsetbuttcap%
\pgfsetroundjoin%
\definecolor{currentfill}{rgb}{0.103922,0.812622,0.889604}%
\pgfsetfillcolor{currentfill}%
\pgfsetlinewidth{1.003750pt}%
\definecolor{currentstroke}{rgb}{0.103922,0.812622,0.889604}%
\pgfsetstrokecolor{currentstroke}%
\pgfsetdash{}{0pt}%
\pgfpathmoveto{\pgfqpoint{3.210081in}{1.002082in}}%
\pgfpathcurveto{\pgfqpoint{3.221131in}{1.002082in}}{\pgfqpoint{3.231730in}{1.006472in}}{\pgfqpoint{3.239544in}{1.014286in}}%
\pgfpathcurveto{\pgfqpoint{3.247358in}{1.022099in}}{\pgfqpoint{3.251748in}{1.032698in}}{\pgfqpoint{3.251748in}{1.043748in}}%
\pgfpathcurveto{\pgfqpoint{3.251748in}{1.054799in}}{\pgfqpoint{3.247358in}{1.065398in}}{\pgfqpoint{3.239544in}{1.073211in}}%
\pgfpathcurveto{\pgfqpoint{3.231730in}{1.081025in}}{\pgfqpoint{3.221131in}{1.085415in}}{\pgfqpoint{3.210081in}{1.085415in}}%
\pgfpathcurveto{\pgfqpoint{3.199031in}{1.085415in}}{\pgfqpoint{3.188432in}{1.081025in}}{\pgfqpoint{3.180618in}{1.073211in}}%
\pgfpathcurveto{\pgfqpoint{3.172805in}{1.065398in}}{\pgfqpoint{3.168415in}{1.054799in}}{\pgfqpoint{3.168415in}{1.043748in}}%
\pgfpathcurveto{\pgfqpoint{3.168415in}{1.032698in}}{\pgfqpoint{3.172805in}{1.022099in}}{\pgfqpoint{3.180618in}{1.014286in}}%
\pgfpathcurveto{\pgfqpoint{3.188432in}{1.006472in}}{\pgfqpoint{3.199031in}{1.002082in}}{\pgfqpoint{3.210081in}{1.002082in}}%
\pgfpathclose%
\pgfusepath{stroke,fill}%
\end{pgfscope}%
\begin{pgfscope}%
\pgfpathrectangle{\pgfqpoint{0.564660in}{0.521603in}}{\pgfqpoint{3.720000in}{3.020000in}} %
\pgfusepath{clip}%
\pgfsetbuttcap%
\pgfsetroundjoin%
\definecolor{currentfill}{rgb}{0.005882,0.700543,0.925638}%
\pgfsetfillcolor{currentfill}%
\pgfsetlinewidth{1.003750pt}%
\definecolor{currentstroke}{rgb}{0.005882,0.700543,0.925638}%
\pgfsetstrokecolor{currentstroke}%
\pgfsetdash{}{0pt}%
\pgfpathmoveto{\pgfqpoint{2.706316in}{1.513387in}}%
\pgfpathcurveto{\pgfqpoint{2.717366in}{1.513387in}}{\pgfqpoint{2.727965in}{1.517777in}}{\pgfqpoint{2.735779in}{1.525591in}}%
\pgfpathcurveto{\pgfqpoint{2.743593in}{1.533405in}}{\pgfqpoint{2.747983in}{1.544004in}}{\pgfqpoint{2.747983in}{1.555054in}}%
\pgfpathcurveto{\pgfqpoint{2.747983in}{1.566104in}}{\pgfqpoint{2.743593in}{1.576703in}}{\pgfqpoint{2.735779in}{1.584517in}}%
\pgfpathcurveto{\pgfqpoint{2.727965in}{1.592330in}}{\pgfqpoint{2.717366in}{1.596720in}}{\pgfqpoint{2.706316in}{1.596720in}}%
\pgfpathcurveto{\pgfqpoint{2.695266in}{1.596720in}}{\pgfqpoint{2.684667in}{1.592330in}}{\pgfqpoint{2.676853in}{1.584517in}}%
\pgfpathcurveto{\pgfqpoint{2.669040in}{1.576703in}}{\pgfqpoint{2.664649in}{1.566104in}}{\pgfqpoint{2.664649in}{1.555054in}}%
\pgfpathcurveto{\pgfqpoint{2.664649in}{1.544004in}}{\pgfqpoint{2.669040in}{1.533405in}}{\pgfqpoint{2.676853in}{1.525591in}}%
\pgfpathcurveto{\pgfqpoint{2.684667in}{1.517777in}}{\pgfqpoint{2.695266in}{1.513387in}}{\pgfqpoint{2.706316in}{1.513387in}}%
\pgfpathclose%
\pgfusepath{stroke,fill}%
\end{pgfscope}%
\begin{pgfscope}%
\pgfpathrectangle{\pgfqpoint{0.564660in}{0.521603in}}{\pgfqpoint{3.720000in}{3.020000in}} %
\pgfusepath{clip}%
\pgfsetbuttcap%
\pgfsetroundjoin%
\definecolor{currentfill}{rgb}{0.005882,0.700543,0.925638}%
\pgfsetfillcolor{currentfill}%
\pgfsetlinewidth{1.003750pt}%
\definecolor{currentstroke}{rgb}{0.005882,0.700543,0.925638}%
\pgfsetstrokecolor{currentstroke}%
\pgfsetdash{}{0pt}%
\pgfpathmoveto{\pgfqpoint{2.196868in}{2.803645in}}%
\pgfpathcurveto{\pgfqpoint{2.207918in}{2.803645in}}{\pgfqpoint{2.218517in}{2.808035in}}{\pgfqpoint{2.226330in}{2.815849in}}%
\pgfpathcurveto{\pgfqpoint{2.234144in}{2.823663in}}{\pgfqpoint{2.238534in}{2.834262in}}{\pgfqpoint{2.238534in}{2.845312in}}%
\pgfpathcurveto{\pgfqpoint{2.238534in}{2.856362in}}{\pgfqpoint{2.234144in}{2.866961in}}{\pgfqpoint{2.226330in}{2.874775in}}%
\pgfpathcurveto{\pgfqpoint{2.218517in}{2.882588in}}{\pgfqpoint{2.207918in}{2.886978in}}{\pgfqpoint{2.196868in}{2.886978in}}%
\pgfpathcurveto{\pgfqpoint{2.185818in}{2.886978in}}{\pgfqpoint{2.175219in}{2.882588in}}{\pgfqpoint{2.167405in}{2.874775in}}%
\pgfpathcurveto{\pgfqpoint{2.159591in}{2.866961in}}{\pgfqpoint{2.155201in}{2.856362in}}{\pgfqpoint{2.155201in}{2.845312in}}%
\pgfpathcurveto{\pgfqpoint{2.155201in}{2.834262in}}{\pgfqpoint{2.159591in}{2.823663in}}{\pgfqpoint{2.167405in}{2.815849in}}%
\pgfpathcurveto{\pgfqpoint{2.175219in}{2.808035in}}{\pgfqpoint{2.185818in}{2.803645in}}{\pgfqpoint{2.196868in}{2.803645in}}%
\pgfpathclose%
\pgfusepath{stroke,fill}%
\end{pgfscope}%
\begin{pgfscope}%
\pgfpathrectangle{\pgfqpoint{0.564660in}{0.521603in}}{\pgfqpoint{3.720000in}{3.020000in}} %
\pgfusepath{clip}%
\pgfsetbuttcap%
\pgfsetroundjoin%
\definecolor{currentfill}{rgb}{0.139216,0.536867,0.960122}%
\pgfsetfillcolor{currentfill}%
\pgfsetlinewidth{1.003750pt}%
\definecolor{currentstroke}{rgb}{0.139216,0.536867,0.960122}%
\pgfsetstrokecolor{currentstroke}%
\pgfsetdash{}{0pt}%
\pgfpathmoveto{\pgfqpoint{3.113579in}{2.722935in}}%
\pgfpathcurveto{\pgfqpoint{3.124629in}{2.722935in}}{\pgfqpoint{3.135229in}{2.727325in}}{\pgfqpoint{3.143042in}{2.735139in}}%
\pgfpathcurveto{\pgfqpoint{3.150856in}{2.742952in}}{\pgfqpoint{3.155246in}{2.753551in}}{\pgfqpoint{3.155246in}{2.764601in}}%
\pgfpathcurveto{\pgfqpoint{3.155246in}{2.775651in}}{\pgfqpoint{3.150856in}{2.786250in}}{\pgfqpoint{3.143042in}{2.794064in}}%
\pgfpathcurveto{\pgfqpoint{3.135229in}{2.801878in}}{\pgfqpoint{3.124629in}{2.806268in}}{\pgfqpoint{3.113579in}{2.806268in}}%
\pgfpathcurveto{\pgfqpoint{3.102529in}{2.806268in}}{\pgfqpoint{3.091930in}{2.801878in}}{\pgfqpoint{3.084117in}{2.794064in}}%
\pgfpathcurveto{\pgfqpoint{3.076303in}{2.786250in}}{\pgfqpoint{3.071913in}{2.775651in}}{\pgfqpoint{3.071913in}{2.764601in}}%
\pgfpathcurveto{\pgfqpoint{3.071913in}{2.753551in}}{\pgfqpoint{3.076303in}{2.742952in}}{\pgfqpoint{3.084117in}{2.735139in}}%
\pgfpathcurveto{\pgfqpoint{3.091930in}{2.727325in}}{\pgfqpoint{3.102529in}{2.722935in}}{\pgfqpoint{3.113579in}{2.722935in}}%
\pgfpathclose%
\pgfusepath{stroke,fill}%
\end{pgfscope}%
\begin{pgfscope}%
\pgfpathrectangle{\pgfqpoint{0.564660in}{0.521603in}}{\pgfqpoint{3.720000in}{3.020000in}} %
\pgfusepath{clip}%
\pgfsetbuttcap%
\pgfsetroundjoin%
\definecolor{currentfill}{rgb}{0.139216,0.536867,0.960122}%
\pgfsetfillcolor{currentfill}%
\pgfsetlinewidth{1.003750pt}%
\definecolor{currentstroke}{rgb}{0.139216,0.536867,0.960122}%
\pgfsetstrokecolor{currentstroke}%
\pgfsetdash{}{0pt}%
\pgfpathmoveto{\pgfqpoint{2.812260in}{2.289241in}}%
\pgfpathcurveto{\pgfqpoint{2.823310in}{2.289241in}}{\pgfqpoint{2.833909in}{2.293632in}}{\pgfqpoint{2.841723in}{2.301445in}}%
\pgfpathcurveto{\pgfqpoint{2.849536in}{2.309259in}}{\pgfqpoint{2.853926in}{2.319858in}}{\pgfqpoint{2.853926in}{2.330908in}}%
\pgfpathcurveto{\pgfqpoint{2.853926in}{2.341958in}}{\pgfqpoint{2.849536in}{2.352557in}}{\pgfqpoint{2.841723in}{2.360371in}}%
\pgfpathcurveto{\pgfqpoint{2.833909in}{2.368185in}}{\pgfqpoint{2.823310in}{2.372575in}}{\pgfqpoint{2.812260in}{2.372575in}}%
\pgfpathcurveto{\pgfqpoint{2.801210in}{2.372575in}}{\pgfqpoint{2.790611in}{2.368185in}}{\pgfqpoint{2.782797in}{2.360371in}}%
\pgfpathcurveto{\pgfqpoint{2.774983in}{2.352557in}}{\pgfqpoint{2.770593in}{2.341958in}}{\pgfqpoint{2.770593in}{2.330908in}}%
\pgfpathcurveto{\pgfqpoint{2.770593in}{2.319858in}}{\pgfqpoint{2.774983in}{2.309259in}}{\pgfqpoint{2.782797in}{2.301445in}}%
\pgfpathcurveto{\pgfqpoint{2.790611in}{2.293632in}}{\pgfqpoint{2.801210in}{2.289241in}}{\pgfqpoint{2.812260in}{2.289241in}}%
\pgfpathclose%
\pgfusepath{stroke,fill}%
\end{pgfscope}%
\begin{pgfscope}%
\pgfpathrectangle{\pgfqpoint{0.564660in}{0.521603in}}{\pgfqpoint{3.720000in}{3.020000in}} %
\pgfusepath{clip}%
\pgfsetbuttcap%
\pgfsetroundjoin%
\definecolor{currentfill}{rgb}{0.139216,0.536867,0.960122}%
\pgfsetfillcolor{currentfill}%
\pgfsetlinewidth{1.003750pt}%
\definecolor{currentstroke}{rgb}{0.139216,0.536867,0.960122}%
\pgfsetstrokecolor{currentstroke}%
\pgfsetdash{}{0pt}%
\pgfpathmoveto{\pgfqpoint{2.634682in}{2.190627in}}%
\pgfpathcurveto{\pgfqpoint{2.645732in}{2.190627in}}{\pgfqpoint{2.656331in}{2.195017in}}{\pgfqpoint{2.664145in}{2.202831in}}%
\pgfpathcurveto{\pgfqpoint{2.671958in}{2.210644in}}{\pgfqpoint{2.676349in}{2.221243in}}{\pgfqpoint{2.676349in}{2.232294in}}%
\pgfpathcurveto{\pgfqpoint{2.676349in}{2.243344in}}{\pgfqpoint{2.671958in}{2.253943in}}{\pgfqpoint{2.664145in}{2.261756in}}%
\pgfpathcurveto{\pgfqpoint{2.656331in}{2.269570in}}{\pgfqpoint{2.645732in}{2.273960in}}{\pgfqpoint{2.634682in}{2.273960in}}%
\pgfpathcurveto{\pgfqpoint{2.623632in}{2.273960in}}{\pgfqpoint{2.613033in}{2.269570in}}{\pgfqpoint{2.605219in}{2.261756in}}%
\pgfpathcurveto{\pgfqpoint{2.597406in}{2.253943in}}{\pgfqpoint{2.593015in}{2.243344in}}{\pgfqpoint{2.593015in}{2.232294in}}%
\pgfpathcurveto{\pgfqpoint{2.593015in}{2.221243in}}{\pgfqpoint{2.597406in}{2.210644in}}{\pgfqpoint{2.605219in}{2.202831in}}%
\pgfpathcurveto{\pgfqpoint{2.613033in}{2.195017in}}{\pgfqpoint{2.623632in}{2.190627in}}{\pgfqpoint{2.634682in}{2.190627in}}%
\pgfpathclose%
\pgfusepath{stroke,fill}%
\end{pgfscope}%
\begin{pgfscope}%
\pgfpathrectangle{\pgfqpoint{0.564660in}{0.521603in}}{\pgfqpoint{3.720000in}{3.020000in}} %
\pgfusepath{clip}%
\pgfsetbuttcap%
\pgfsetroundjoin%
\definecolor{currentfill}{rgb}{0.139216,0.536867,0.960122}%
\pgfsetfillcolor{currentfill}%
\pgfsetlinewidth{1.003750pt}%
\definecolor{currentstroke}{rgb}{0.139216,0.536867,0.960122}%
\pgfsetstrokecolor{currentstroke}%
\pgfsetdash{}{0pt}%
\pgfpathmoveto{\pgfqpoint{2.469078in}{1.382766in}}%
\pgfpathcurveto{\pgfqpoint{2.480128in}{1.382766in}}{\pgfqpoint{2.490727in}{1.387157in}}{\pgfqpoint{2.498541in}{1.394970in}}%
\pgfpathcurveto{\pgfqpoint{2.506354in}{1.402784in}}{\pgfqpoint{2.510744in}{1.413383in}}{\pgfqpoint{2.510744in}{1.424433in}}%
\pgfpathcurveto{\pgfqpoint{2.510744in}{1.435483in}}{\pgfqpoint{2.506354in}{1.446082in}}{\pgfqpoint{2.498541in}{1.453896in}}%
\pgfpathcurveto{\pgfqpoint{2.490727in}{1.461710in}}{\pgfqpoint{2.480128in}{1.466100in}}{\pgfqpoint{2.469078in}{1.466100in}}%
\pgfpathcurveto{\pgfqpoint{2.458028in}{1.466100in}}{\pgfqpoint{2.447429in}{1.461710in}}{\pgfqpoint{2.439615in}{1.453896in}}%
\pgfpathcurveto{\pgfqpoint{2.431801in}{1.446082in}}{\pgfqpoint{2.427411in}{1.435483in}}{\pgfqpoint{2.427411in}{1.424433in}}%
\pgfpathcurveto{\pgfqpoint{2.427411in}{1.413383in}}{\pgfqpoint{2.431801in}{1.402784in}}{\pgfqpoint{2.439615in}{1.394970in}}%
\pgfpathcurveto{\pgfqpoint{2.447429in}{1.387157in}}{\pgfqpoint{2.458028in}{1.382766in}}{\pgfqpoint{2.469078in}{1.382766in}}%
\pgfpathclose%
\pgfusepath{stroke,fill}%
\end{pgfscope}%
\begin{pgfscope}%
\pgfpathrectangle{\pgfqpoint{0.564660in}{0.521603in}}{\pgfqpoint{3.720000in}{3.020000in}} %
\pgfusepath{clip}%
\pgfsetbuttcap%
\pgfsetroundjoin%
\definecolor{currentfill}{rgb}{0.139216,0.536867,0.960122}%
\pgfsetfillcolor{currentfill}%
\pgfsetlinewidth{1.003750pt}%
\definecolor{currentstroke}{rgb}{0.139216,0.536867,0.960122}%
\pgfsetstrokecolor{currentstroke}%
\pgfsetdash{}{0pt}%
\pgfpathmoveto{\pgfqpoint{2.011780in}{3.301238in}}%
\pgfpathcurveto{\pgfqpoint{2.022830in}{3.301238in}}{\pgfqpoint{2.033429in}{3.305628in}}{\pgfqpoint{2.041243in}{3.313442in}}%
\pgfpathcurveto{\pgfqpoint{2.049057in}{3.321256in}}{\pgfqpoint{2.053447in}{3.331855in}}{\pgfqpoint{2.053447in}{3.342905in}}%
\pgfpathcurveto{\pgfqpoint{2.053447in}{3.353955in}}{\pgfqpoint{2.049057in}{3.364554in}}{\pgfqpoint{2.041243in}{3.372368in}}%
\pgfpathcurveto{\pgfqpoint{2.033429in}{3.380181in}}{\pgfqpoint{2.022830in}{3.384572in}}{\pgfqpoint{2.011780in}{3.384572in}}%
\pgfpathcurveto{\pgfqpoint{2.000730in}{3.384572in}}{\pgfqpoint{1.990131in}{3.380181in}}{\pgfqpoint{1.982317in}{3.372368in}}%
\pgfpathcurveto{\pgfqpoint{1.974504in}{3.364554in}}{\pgfqpoint{1.970113in}{3.353955in}}{\pgfqpoint{1.970113in}{3.342905in}}%
\pgfpathcurveto{\pgfqpoint{1.970113in}{3.331855in}}{\pgfqpoint{1.974504in}{3.321256in}}{\pgfqpoint{1.982317in}{3.313442in}}%
\pgfpathcurveto{\pgfqpoint{1.990131in}{3.305628in}}{\pgfqpoint{2.000730in}{3.301238in}}{\pgfqpoint{2.011780in}{3.301238in}}%
\pgfpathclose%
\pgfusepath{stroke,fill}%
\end{pgfscope}%
\begin{pgfscope}%
\pgfsetbuttcap%
\pgfsetroundjoin%
\definecolor{currentfill}{rgb}{0.000000,0.000000,0.000000}%
\pgfsetfillcolor{currentfill}%
\pgfsetlinewidth{0.803000pt}%
\definecolor{currentstroke}{rgb}{0.000000,0.000000,0.000000}%
\pgfsetstrokecolor{currentstroke}%
\pgfsetdash{}{0pt}%
\pgfsys@defobject{currentmarker}{\pgfqpoint{0.000000in}{-0.048611in}}{\pgfqpoint{0.000000in}{0.000000in}}{%
\pgfpathmoveto{\pgfqpoint{0.000000in}{0.000000in}}%
\pgfpathlineto{\pgfqpoint{0.000000in}{-0.048611in}}%
\pgfusepath{stroke,fill}%
}%
\begin{pgfscope}%
\pgfsys@transformshift{0.564660in}{0.521603in}%
\pgfsys@useobject{currentmarker}{}%
\end{pgfscope}%
\end{pgfscope}%
\begin{pgfscope}%
\pgftext[x=0.564660in,y=0.424381in,,top]{\rmfamily\fontsize{10.000000}{12.000000}\selectfont \(\displaystyle 10^{-2}\)}%
\end{pgfscope}%
\begin{pgfscope}%
\pgfsetbuttcap%
\pgfsetroundjoin%
\definecolor{currentfill}{rgb}{0.000000,0.000000,0.000000}%
\pgfsetfillcolor{currentfill}%
\pgfsetlinewidth{0.803000pt}%
\definecolor{currentstroke}{rgb}{0.000000,0.000000,0.000000}%
\pgfsetstrokecolor{currentstroke}%
\pgfsetdash{}{0pt}%
\pgfsys@defobject{currentmarker}{\pgfqpoint{0.000000in}{-0.048611in}}{\pgfqpoint{0.000000in}{0.000000in}}{%
\pgfpathmoveto{\pgfqpoint{0.000000in}{0.000000in}}%
\pgfpathlineto{\pgfqpoint{0.000000in}{-0.048611in}}%
\pgfusepath{stroke,fill}%
}%
\begin{pgfscope}%
\pgfsys@transformshift{2.397001in}{0.521603in}%
\pgfsys@useobject{currentmarker}{}%
\end{pgfscope}%
\end{pgfscope}%
\begin{pgfscope}%
\pgftext[x=2.397001in,y=0.424381in,,top]{\rmfamily\fontsize{10.000000}{12.000000}\selectfont \(\displaystyle 10^{-1}\)}%
\end{pgfscope}%
\begin{pgfscope}%
\pgfsetbuttcap%
\pgfsetroundjoin%
\definecolor{currentfill}{rgb}{0.000000,0.000000,0.000000}%
\pgfsetfillcolor{currentfill}%
\pgfsetlinewidth{0.803000pt}%
\definecolor{currentstroke}{rgb}{0.000000,0.000000,0.000000}%
\pgfsetstrokecolor{currentstroke}%
\pgfsetdash{}{0pt}%
\pgfsys@defobject{currentmarker}{\pgfqpoint{0.000000in}{-0.048611in}}{\pgfqpoint{0.000000in}{0.000000in}}{%
\pgfpathmoveto{\pgfqpoint{0.000000in}{0.000000in}}%
\pgfpathlineto{\pgfqpoint{0.000000in}{-0.048611in}}%
\pgfusepath{stroke,fill}%
}%
\begin{pgfscope}%
\pgfsys@transformshift{4.229341in}{0.521603in}%
\pgfsys@useobject{currentmarker}{}%
\end{pgfscope}%
\end{pgfscope}%
\begin{pgfscope}%
\pgftext[x=4.229341in,y=0.424381in,,top]{\rmfamily\fontsize{10.000000}{12.000000}\selectfont \(\displaystyle 10^{0}\)}%
\end{pgfscope}%
\begin{pgfscope}%
\pgfsetbuttcap%
\pgfsetroundjoin%
\definecolor{currentfill}{rgb}{0.000000,0.000000,0.000000}%
\pgfsetfillcolor{currentfill}%
\pgfsetlinewidth{0.602250pt}%
\definecolor{currentstroke}{rgb}{0.000000,0.000000,0.000000}%
\pgfsetstrokecolor{currentstroke}%
\pgfsetdash{}{0pt}%
\pgfsys@defobject{currentmarker}{\pgfqpoint{0.000000in}{-0.027778in}}{\pgfqpoint{0.000000in}{0.000000in}}{%
\pgfpathmoveto{\pgfqpoint{0.000000in}{0.000000in}}%
\pgfpathlineto{\pgfqpoint{0.000000in}{-0.027778in}}%
\pgfusepath{stroke,fill}%
}%
\begin{pgfscope}%
\pgfsys@transformshift{1.116250in}{0.521603in}%
\pgfsys@useobject{currentmarker}{}%
\end{pgfscope}%
\end{pgfscope}%
\begin{pgfscope}%
\pgfsetbuttcap%
\pgfsetroundjoin%
\definecolor{currentfill}{rgb}{0.000000,0.000000,0.000000}%
\pgfsetfillcolor{currentfill}%
\pgfsetlinewidth{0.602250pt}%
\definecolor{currentstroke}{rgb}{0.000000,0.000000,0.000000}%
\pgfsetstrokecolor{currentstroke}%
\pgfsetdash{}{0pt}%
\pgfsys@defobject{currentmarker}{\pgfqpoint{0.000000in}{-0.027778in}}{\pgfqpoint{0.000000in}{0.000000in}}{%
\pgfpathmoveto{\pgfqpoint{0.000000in}{0.000000in}}%
\pgfpathlineto{\pgfqpoint{0.000000in}{-0.027778in}}%
\pgfusepath{stroke,fill}%
}%
\begin{pgfscope}%
\pgfsys@transformshift{1.438909in}{0.521603in}%
\pgfsys@useobject{currentmarker}{}%
\end{pgfscope}%
\end{pgfscope}%
\begin{pgfscope}%
\pgfsetbuttcap%
\pgfsetroundjoin%
\definecolor{currentfill}{rgb}{0.000000,0.000000,0.000000}%
\pgfsetfillcolor{currentfill}%
\pgfsetlinewidth{0.602250pt}%
\definecolor{currentstroke}{rgb}{0.000000,0.000000,0.000000}%
\pgfsetstrokecolor{currentstroke}%
\pgfsetdash{}{0pt}%
\pgfsys@defobject{currentmarker}{\pgfqpoint{0.000000in}{-0.027778in}}{\pgfqpoint{0.000000in}{0.000000in}}{%
\pgfpathmoveto{\pgfqpoint{0.000000in}{0.000000in}}%
\pgfpathlineto{\pgfqpoint{0.000000in}{-0.027778in}}%
\pgfusepath{stroke,fill}%
}%
\begin{pgfscope}%
\pgfsys@transformshift{1.667839in}{0.521603in}%
\pgfsys@useobject{currentmarker}{}%
\end{pgfscope}%
\end{pgfscope}%
\begin{pgfscope}%
\pgfsetbuttcap%
\pgfsetroundjoin%
\definecolor{currentfill}{rgb}{0.000000,0.000000,0.000000}%
\pgfsetfillcolor{currentfill}%
\pgfsetlinewidth{0.602250pt}%
\definecolor{currentstroke}{rgb}{0.000000,0.000000,0.000000}%
\pgfsetstrokecolor{currentstroke}%
\pgfsetdash{}{0pt}%
\pgfsys@defobject{currentmarker}{\pgfqpoint{0.000000in}{-0.027778in}}{\pgfqpoint{0.000000in}{0.000000in}}{%
\pgfpathmoveto{\pgfqpoint{0.000000in}{0.000000in}}%
\pgfpathlineto{\pgfqpoint{0.000000in}{-0.027778in}}%
\pgfusepath{stroke,fill}%
}%
\begin{pgfscope}%
\pgfsys@transformshift{1.845411in}{0.521603in}%
\pgfsys@useobject{currentmarker}{}%
\end{pgfscope}%
\end{pgfscope}%
\begin{pgfscope}%
\pgfsetbuttcap%
\pgfsetroundjoin%
\definecolor{currentfill}{rgb}{0.000000,0.000000,0.000000}%
\pgfsetfillcolor{currentfill}%
\pgfsetlinewidth{0.602250pt}%
\definecolor{currentstroke}{rgb}{0.000000,0.000000,0.000000}%
\pgfsetstrokecolor{currentstroke}%
\pgfsetdash{}{0pt}%
\pgfsys@defobject{currentmarker}{\pgfqpoint{0.000000in}{-0.027778in}}{\pgfqpoint{0.000000in}{0.000000in}}{%
\pgfpathmoveto{\pgfqpoint{0.000000in}{0.000000in}}%
\pgfpathlineto{\pgfqpoint{0.000000in}{-0.027778in}}%
\pgfusepath{stroke,fill}%
}%
\begin{pgfscope}%
\pgfsys@transformshift{1.990498in}{0.521603in}%
\pgfsys@useobject{currentmarker}{}%
\end{pgfscope}%
\end{pgfscope}%
\begin{pgfscope}%
\pgfsetbuttcap%
\pgfsetroundjoin%
\definecolor{currentfill}{rgb}{0.000000,0.000000,0.000000}%
\pgfsetfillcolor{currentfill}%
\pgfsetlinewidth{0.602250pt}%
\definecolor{currentstroke}{rgb}{0.000000,0.000000,0.000000}%
\pgfsetstrokecolor{currentstroke}%
\pgfsetdash{}{0pt}%
\pgfsys@defobject{currentmarker}{\pgfqpoint{0.000000in}{-0.027778in}}{\pgfqpoint{0.000000in}{0.000000in}}{%
\pgfpathmoveto{\pgfqpoint{0.000000in}{0.000000in}}%
\pgfpathlineto{\pgfqpoint{0.000000in}{-0.027778in}}%
\pgfusepath{stroke,fill}%
}%
\begin{pgfscope}%
\pgfsys@transformshift{2.113168in}{0.521603in}%
\pgfsys@useobject{currentmarker}{}%
\end{pgfscope}%
\end{pgfscope}%
\begin{pgfscope}%
\pgfsetbuttcap%
\pgfsetroundjoin%
\definecolor{currentfill}{rgb}{0.000000,0.000000,0.000000}%
\pgfsetfillcolor{currentfill}%
\pgfsetlinewidth{0.602250pt}%
\definecolor{currentstroke}{rgb}{0.000000,0.000000,0.000000}%
\pgfsetstrokecolor{currentstroke}%
\pgfsetdash{}{0pt}%
\pgfsys@defobject{currentmarker}{\pgfqpoint{0.000000in}{-0.027778in}}{\pgfqpoint{0.000000in}{0.000000in}}{%
\pgfpathmoveto{\pgfqpoint{0.000000in}{0.000000in}}%
\pgfpathlineto{\pgfqpoint{0.000000in}{-0.027778in}}%
\pgfusepath{stroke,fill}%
}%
\begin{pgfscope}%
\pgfsys@transformshift{2.219429in}{0.521603in}%
\pgfsys@useobject{currentmarker}{}%
\end{pgfscope}%
\end{pgfscope}%
\begin{pgfscope}%
\pgfsetbuttcap%
\pgfsetroundjoin%
\definecolor{currentfill}{rgb}{0.000000,0.000000,0.000000}%
\pgfsetfillcolor{currentfill}%
\pgfsetlinewidth{0.602250pt}%
\definecolor{currentstroke}{rgb}{0.000000,0.000000,0.000000}%
\pgfsetstrokecolor{currentstroke}%
\pgfsetdash{}{0pt}%
\pgfsys@defobject{currentmarker}{\pgfqpoint{0.000000in}{-0.027778in}}{\pgfqpoint{0.000000in}{0.000000in}}{%
\pgfpathmoveto{\pgfqpoint{0.000000in}{0.000000in}}%
\pgfpathlineto{\pgfqpoint{0.000000in}{-0.027778in}}%
\pgfusepath{stroke,fill}%
}%
\begin{pgfscope}%
\pgfsys@transformshift{2.313157in}{0.521603in}%
\pgfsys@useobject{currentmarker}{}%
\end{pgfscope}%
\end{pgfscope}%
\begin{pgfscope}%
\pgfsetbuttcap%
\pgfsetroundjoin%
\definecolor{currentfill}{rgb}{0.000000,0.000000,0.000000}%
\pgfsetfillcolor{currentfill}%
\pgfsetlinewidth{0.602250pt}%
\definecolor{currentstroke}{rgb}{0.000000,0.000000,0.000000}%
\pgfsetstrokecolor{currentstroke}%
\pgfsetdash{}{0pt}%
\pgfsys@defobject{currentmarker}{\pgfqpoint{0.000000in}{-0.027778in}}{\pgfqpoint{0.000000in}{0.000000in}}{%
\pgfpathmoveto{\pgfqpoint{0.000000in}{0.000000in}}%
\pgfpathlineto{\pgfqpoint{0.000000in}{-0.027778in}}%
\pgfusepath{stroke,fill}%
}%
\begin{pgfscope}%
\pgfsys@transformshift{2.948590in}{0.521603in}%
\pgfsys@useobject{currentmarker}{}%
\end{pgfscope}%
\end{pgfscope}%
\begin{pgfscope}%
\pgfsetbuttcap%
\pgfsetroundjoin%
\definecolor{currentfill}{rgb}{0.000000,0.000000,0.000000}%
\pgfsetfillcolor{currentfill}%
\pgfsetlinewidth{0.602250pt}%
\definecolor{currentstroke}{rgb}{0.000000,0.000000,0.000000}%
\pgfsetstrokecolor{currentstroke}%
\pgfsetdash{}{0pt}%
\pgfsys@defobject{currentmarker}{\pgfqpoint{0.000000in}{-0.027778in}}{\pgfqpoint{0.000000in}{0.000000in}}{%
\pgfpathmoveto{\pgfqpoint{0.000000in}{0.000000in}}%
\pgfpathlineto{\pgfqpoint{0.000000in}{-0.027778in}}%
\pgfusepath{stroke,fill}%
}%
\begin{pgfscope}%
\pgfsys@transformshift{3.271249in}{0.521603in}%
\pgfsys@useobject{currentmarker}{}%
\end{pgfscope}%
\end{pgfscope}%
\begin{pgfscope}%
\pgfsetbuttcap%
\pgfsetroundjoin%
\definecolor{currentfill}{rgb}{0.000000,0.000000,0.000000}%
\pgfsetfillcolor{currentfill}%
\pgfsetlinewidth{0.602250pt}%
\definecolor{currentstroke}{rgb}{0.000000,0.000000,0.000000}%
\pgfsetstrokecolor{currentstroke}%
\pgfsetdash{}{0pt}%
\pgfsys@defobject{currentmarker}{\pgfqpoint{0.000000in}{-0.027778in}}{\pgfqpoint{0.000000in}{0.000000in}}{%
\pgfpathmoveto{\pgfqpoint{0.000000in}{0.000000in}}%
\pgfpathlineto{\pgfqpoint{0.000000in}{-0.027778in}}%
\pgfusepath{stroke,fill}%
}%
\begin{pgfscope}%
\pgfsys@transformshift{3.500180in}{0.521603in}%
\pgfsys@useobject{currentmarker}{}%
\end{pgfscope}%
\end{pgfscope}%
\begin{pgfscope}%
\pgfsetbuttcap%
\pgfsetroundjoin%
\definecolor{currentfill}{rgb}{0.000000,0.000000,0.000000}%
\pgfsetfillcolor{currentfill}%
\pgfsetlinewidth{0.602250pt}%
\definecolor{currentstroke}{rgb}{0.000000,0.000000,0.000000}%
\pgfsetstrokecolor{currentstroke}%
\pgfsetdash{}{0pt}%
\pgfsys@defobject{currentmarker}{\pgfqpoint{0.000000in}{-0.027778in}}{\pgfqpoint{0.000000in}{0.000000in}}{%
\pgfpathmoveto{\pgfqpoint{0.000000in}{0.000000in}}%
\pgfpathlineto{\pgfqpoint{0.000000in}{-0.027778in}}%
\pgfusepath{stroke,fill}%
}%
\begin{pgfscope}%
\pgfsys@transformshift{3.677752in}{0.521603in}%
\pgfsys@useobject{currentmarker}{}%
\end{pgfscope}%
\end{pgfscope}%
\begin{pgfscope}%
\pgfsetbuttcap%
\pgfsetroundjoin%
\definecolor{currentfill}{rgb}{0.000000,0.000000,0.000000}%
\pgfsetfillcolor{currentfill}%
\pgfsetlinewidth{0.602250pt}%
\definecolor{currentstroke}{rgb}{0.000000,0.000000,0.000000}%
\pgfsetstrokecolor{currentstroke}%
\pgfsetdash{}{0pt}%
\pgfsys@defobject{currentmarker}{\pgfqpoint{0.000000in}{-0.027778in}}{\pgfqpoint{0.000000in}{0.000000in}}{%
\pgfpathmoveto{\pgfqpoint{0.000000in}{0.000000in}}%
\pgfpathlineto{\pgfqpoint{0.000000in}{-0.027778in}}%
\pgfusepath{stroke,fill}%
}%
\begin{pgfscope}%
\pgfsys@transformshift{3.822839in}{0.521603in}%
\pgfsys@useobject{currentmarker}{}%
\end{pgfscope}%
\end{pgfscope}%
\begin{pgfscope}%
\pgfsetbuttcap%
\pgfsetroundjoin%
\definecolor{currentfill}{rgb}{0.000000,0.000000,0.000000}%
\pgfsetfillcolor{currentfill}%
\pgfsetlinewidth{0.602250pt}%
\definecolor{currentstroke}{rgb}{0.000000,0.000000,0.000000}%
\pgfsetstrokecolor{currentstroke}%
\pgfsetdash{}{0pt}%
\pgfsys@defobject{currentmarker}{\pgfqpoint{0.000000in}{-0.027778in}}{\pgfqpoint{0.000000in}{0.000000in}}{%
\pgfpathmoveto{\pgfqpoint{0.000000in}{0.000000in}}%
\pgfpathlineto{\pgfqpoint{0.000000in}{-0.027778in}}%
\pgfusepath{stroke,fill}%
}%
\begin{pgfscope}%
\pgfsys@transformshift{3.945508in}{0.521603in}%
\pgfsys@useobject{currentmarker}{}%
\end{pgfscope}%
\end{pgfscope}%
\begin{pgfscope}%
\pgfsetbuttcap%
\pgfsetroundjoin%
\definecolor{currentfill}{rgb}{0.000000,0.000000,0.000000}%
\pgfsetfillcolor{currentfill}%
\pgfsetlinewidth{0.602250pt}%
\definecolor{currentstroke}{rgb}{0.000000,0.000000,0.000000}%
\pgfsetstrokecolor{currentstroke}%
\pgfsetdash{}{0pt}%
\pgfsys@defobject{currentmarker}{\pgfqpoint{0.000000in}{-0.027778in}}{\pgfqpoint{0.000000in}{0.000000in}}{%
\pgfpathmoveto{\pgfqpoint{0.000000in}{0.000000in}}%
\pgfpathlineto{\pgfqpoint{0.000000in}{-0.027778in}}%
\pgfusepath{stroke,fill}%
}%
\begin{pgfscope}%
\pgfsys@transformshift{4.051769in}{0.521603in}%
\pgfsys@useobject{currentmarker}{}%
\end{pgfscope}%
\end{pgfscope}%
\begin{pgfscope}%
\pgfsetbuttcap%
\pgfsetroundjoin%
\definecolor{currentfill}{rgb}{0.000000,0.000000,0.000000}%
\pgfsetfillcolor{currentfill}%
\pgfsetlinewidth{0.602250pt}%
\definecolor{currentstroke}{rgb}{0.000000,0.000000,0.000000}%
\pgfsetstrokecolor{currentstroke}%
\pgfsetdash{}{0pt}%
\pgfsys@defobject{currentmarker}{\pgfqpoint{0.000000in}{-0.027778in}}{\pgfqpoint{0.000000in}{0.000000in}}{%
\pgfpathmoveto{\pgfqpoint{0.000000in}{0.000000in}}%
\pgfpathlineto{\pgfqpoint{0.000000in}{-0.027778in}}%
\pgfusepath{stroke,fill}%
}%
\begin{pgfscope}%
\pgfsys@transformshift{4.145498in}{0.521603in}%
\pgfsys@useobject{currentmarker}{}%
\end{pgfscope}%
\end{pgfscope}%
\begin{pgfscope}%
\pgftext[x=2.424660in,y=0.234413in,,top]{\rmfamily\fontsize{10.000000}{12.000000}\selectfont \(\displaystyle \mathbf{W}\mbox{e}\)}%
\end{pgfscope}%
\begin{pgfscope}%
\pgfsetbuttcap%
\pgfsetroundjoin%
\definecolor{currentfill}{rgb}{0.000000,0.000000,0.000000}%
\pgfsetfillcolor{currentfill}%
\pgfsetlinewidth{0.803000pt}%
\definecolor{currentstroke}{rgb}{0.000000,0.000000,0.000000}%
\pgfsetstrokecolor{currentstroke}%
\pgfsetdash{}{0pt}%
\pgfsys@defobject{currentmarker}{\pgfqpoint{-0.048611in}{0.000000in}}{\pgfqpoint{0.000000in}{0.000000in}}{%
\pgfpathmoveto{\pgfqpoint{0.000000in}{0.000000in}}%
\pgfpathlineto{\pgfqpoint{-0.048611in}{0.000000in}}%
\pgfusepath{stroke,fill}%
}%
\begin{pgfscope}%
\pgfsys@transformshift{0.564660in}{1.024740in}%
\pgfsys@useobject{currentmarker}{}%
\end{pgfscope}%
\end{pgfscope}%
\begin{pgfscope}%
\pgftext[x=0.289968in,y=0.971978in,left,base]{\rmfamily\fontsize{10.000000}{12.000000}\selectfont \(\displaystyle 0.4\)}%
\end{pgfscope}%
\begin{pgfscope}%
\pgfsetbuttcap%
\pgfsetroundjoin%
\definecolor{currentfill}{rgb}{0.000000,0.000000,0.000000}%
\pgfsetfillcolor{currentfill}%
\pgfsetlinewidth{0.803000pt}%
\definecolor{currentstroke}{rgb}{0.000000,0.000000,0.000000}%
\pgfsetstrokecolor{currentstroke}%
\pgfsetdash{}{0pt}%
\pgfsys@defobject{currentmarker}{\pgfqpoint{-0.048611in}{0.000000in}}{\pgfqpoint{0.000000in}{0.000000in}}{%
\pgfpathmoveto{\pgfqpoint{0.000000in}{0.000000in}}%
\pgfpathlineto{\pgfqpoint{-0.048611in}{0.000000in}}%
\pgfusepath{stroke,fill}%
}%
\begin{pgfscope}%
\pgfsys@transformshift{0.564660in}{1.643093in}%
\pgfsys@useobject{currentmarker}{}%
\end{pgfscope}%
\end{pgfscope}%
\begin{pgfscope}%
\pgftext[x=0.289968in,y=1.590331in,left,base]{\rmfamily\fontsize{10.000000}{12.000000}\selectfont \(\displaystyle 0.5\)}%
\end{pgfscope}%
\begin{pgfscope}%
\pgfsetbuttcap%
\pgfsetroundjoin%
\definecolor{currentfill}{rgb}{0.000000,0.000000,0.000000}%
\pgfsetfillcolor{currentfill}%
\pgfsetlinewidth{0.803000pt}%
\definecolor{currentstroke}{rgb}{0.000000,0.000000,0.000000}%
\pgfsetstrokecolor{currentstroke}%
\pgfsetdash{}{0pt}%
\pgfsys@defobject{currentmarker}{\pgfqpoint{-0.048611in}{0.000000in}}{\pgfqpoint{0.000000in}{0.000000in}}{%
\pgfpathmoveto{\pgfqpoint{0.000000in}{0.000000in}}%
\pgfpathlineto{\pgfqpoint{-0.048611in}{0.000000in}}%
\pgfusepath{stroke,fill}%
}%
\begin{pgfscope}%
\pgfsys@transformshift{0.564660in}{2.261445in}%
\pgfsys@useobject{currentmarker}{}%
\end{pgfscope}%
\end{pgfscope}%
\begin{pgfscope}%
\pgftext[x=0.289968in,y=2.208684in,left,base]{\rmfamily\fontsize{10.000000}{12.000000}\selectfont \(\displaystyle 0.6\)}%
\end{pgfscope}%
\begin{pgfscope}%
\pgfsetbuttcap%
\pgfsetroundjoin%
\definecolor{currentfill}{rgb}{0.000000,0.000000,0.000000}%
\pgfsetfillcolor{currentfill}%
\pgfsetlinewidth{0.803000pt}%
\definecolor{currentstroke}{rgb}{0.000000,0.000000,0.000000}%
\pgfsetstrokecolor{currentstroke}%
\pgfsetdash{}{0pt}%
\pgfsys@defobject{currentmarker}{\pgfqpoint{-0.048611in}{0.000000in}}{\pgfqpoint{0.000000in}{0.000000in}}{%
\pgfpathmoveto{\pgfqpoint{0.000000in}{0.000000in}}%
\pgfpathlineto{\pgfqpoint{-0.048611in}{0.000000in}}%
\pgfusepath{stroke,fill}%
}%
\begin{pgfscope}%
\pgfsys@transformshift{0.564660in}{2.879798in}%
\pgfsys@useobject{currentmarker}{}%
\end{pgfscope}%
\end{pgfscope}%
\begin{pgfscope}%
\pgftext[x=0.289968in,y=2.827036in,left,base]{\rmfamily\fontsize{10.000000}{12.000000}\selectfont \(\displaystyle 0.7\)}%
\end{pgfscope}%
\begin{pgfscope}%
\pgfsetbuttcap%
\pgfsetroundjoin%
\definecolor{currentfill}{rgb}{0.000000,0.000000,0.000000}%
\pgfsetfillcolor{currentfill}%
\pgfsetlinewidth{0.803000pt}%
\definecolor{currentstroke}{rgb}{0.000000,0.000000,0.000000}%
\pgfsetstrokecolor{currentstroke}%
\pgfsetdash{}{0pt}%
\pgfsys@defobject{currentmarker}{\pgfqpoint{-0.048611in}{0.000000in}}{\pgfqpoint{0.000000in}{0.000000in}}{%
\pgfpathmoveto{\pgfqpoint{0.000000in}{0.000000in}}%
\pgfpathlineto{\pgfqpoint{-0.048611in}{0.000000in}}%
\pgfusepath{stroke,fill}%
}%
\begin{pgfscope}%
\pgfsys@transformshift{0.564660in}{3.498150in}%
\pgfsys@useobject{currentmarker}{}%
\end{pgfscope}%
\end{pgfscope}%
\begin{pgfscope}%
\pgftext[x=0.289968in,y=3.445389in,left,base]{\rmfamily\fontsize{10.000000}{12.000000}\selectfont \(\displaystyle 0.8\)}%
\end{pgfscope}%
\begin{pgfscope}%
\pgftext[x=0.234413in,y=2.031603in,,bottom,rotate=90.000000]{\rmfamily\fontsize{10.000000}{12.000000}\selectfont \(\displaystyle C_r\)}%
\end{pgfscope}%
\begin{pgfscope}%
\pgfsetrectcap%
\pgfsetmiterjoin%
\pgfsetlinewidth{0.803000pt}%
\definecolor{currentstroke}{rgb}{0.000000,0.000000,0.000000}%
\pgfsetstrokecolor{currentstroke}%
\pgfsetdash{}{0pt}%
\pgfpathmoveto{\pgfqpoint{0.564660in}{0.521603in}}%
\pgfpathlineto{\pgfqpoint{0.564660in}{3.541603in}}%
\pgfusepath{stroke}%
\end{pgfscope}%
\begin{pgfscope}%
\pgfsetrectcap%
\pgfsetmiterjoin%
\pgfsetlinewidth{0.803000pt}%
\definecolor{currentstroke}{rgb}{0.000000,0.000000,0.000000}%
\pgfsetstrokecolor{currentstroke}%
\pgfsetdash{}{0pt}%
\pgfpathmoveto{\pgfqpoint{4.284660in}{0.521603in}}%
\pgfpathlineto{\pgfqpoint{4.284660in}{3.541603in}}%
\pgfusepath{stroke}%
\end{pgfscope}%
\begin{pgfscope}%
\pgfsetrectcap%
\pgfsetmiterjoin%
\pgfsetlinewidth{0.803000pt}%
\definecolor{currentstroke}{rgb}{0.000000,0.000000,0.000000}%
\pgfsetstrokecolor{currentstroke}%
\pgfsetdash{}{0pt}%
\pgfpathmoveto{\pgfqpoint{0.564660in}{0.521603in}}%
\pgfpathlineto{\pgfqpoint{4.284660in}{0.521603in}}%
\pgfusepath{stroke}%
\end{pgfscope}%
\begin{pgfscope}%
\pgfsetrectcap%
\pgfsetmiterjoin%
\pgfsetlinewidth{0.803000pt}%
\definecolor{currentstroke}{rgb}{0.000000,0.000000,0.000000}%
\pgfsetstrokecolor{currentstroke}%
\pgfsetdash{}{0pt}%
\pgfpathmoveto{\pgfqpoint{0.564660in}{3.541603in}}%
\pgfpathlineto{\pgfqpoint{4.284660in}{3.541603in}}%
\pgfusepath{stroke}%
\end{pgfscope}%
\begin{pgfscope}%
\pgfsetbuttcap%
\pgfsetmiterjoin%
\definecolor{currentfill}{rgb}{1.000000,1.000000,1.000000}%
\pgfsetfillcolor{currentfill}%
\pgfsetfillopacity{0.700000}%
\pgfsetlinewidth{1.003750pt}%
\definecolor{currentstroke}{rgb}{0.500000,0.500000,0.500000}%
\pgfsetstrokecolor{currentstroke}%
\pgfsetstrokeopacity{0.700000}%
\pgfsetdash{}{0pt}%
\pgfpathmoveto{\pgfqpoint{0.887319in}{3.107570in}}%
\pgfpathlineto{\pgfqpoint{1.534885in}{3.107570in}}%
\pgfpathquadraticcurveto{\pgfqpoint{1.576552in}{3.107570in}}{\pgfqpoint{1.576552in}{3.149237in}}%
\pgfpathlineto{\pgfqpoint{1.576552in}{3.294497in}}%
\pgfpathquadraticcurveto{\pgfqpoint{1.576552in}{3.336164in}}{\pgfqpoint{1.534885in}{3.336164in}}%
\pgfpathlineto{\pgfqpoint{0.887319in}{3.336164in}}%
\pgfpathquadraticcurveto{\pgfqpoint{0.845653in}{3.336164in}}{\pgfqpoint{0.845653in}{3.294497in}}%
\pgfpathlineto{\pgfqpoint{0.845653in}{3.149237in}}%
\pgfpathquadraticcurveto{\pgfqpoint{0.845653in}{3.107570in}}{\pgfqpoint{0.887319in}{3.107570in}}%
\pgfpathclose%
\pgfusepath{stroke,fill}%
\end{pgfscope}%
\begin{pgfscope}%
\pgftext[x=0.887319in,y=3.188974in,left,base]{\rmfamily\fontsize{10.000000}{12.000000}\selectfont \(\displaystyle \mathbf{O}\mbox{h}_{\mu} = 2.2\)}%
\end{pgfscope}%
\begin{pgfscope}%
\pgfpathrectangle{\pgfqpoint{4.517160in}{0.521603in}}{\pgfqpoint{0.151000in}{3.020000in}} %
\pgfusepath{clip}%
\pgfsetbuttcap%
\pgfsetmiterjoin%
\definecolor{currentfill}{rgb}{1.000000,1.000000,1.000000}%
\pgfsetfillcolor{currentfill}%
\pgfsetlinewidth{0.010037pt}%
\definecolor{currentstroke}{rgb}{1.000000,1.000000,1.000000}%
\pgfsetstrokecolor{currentstroke}%
\pgfsetdash{}{0pt}%
\pgfpathmoveto{\pgfqpoint{4.517160in}{0.521603in}}%
\pgfpathlineto{\pgfqpoint{4.517160in}{0.533400in}}%
\pgfpathlineto{\pgfqpoint{4.517160in}{3.529806in}}%
\pgfpathlineto{\pgfqpoint{4.517160in}{3.541603in}}%
\pgfpathlineto{\pgfqpoint{4.668160in}{3.541603in}}%
\pgfpathlineto{\pgfqpoint{4.668160in}{3.529806in}}%
\pgfpathlineto{\pgfqpoint{4.668160in}{0.533400in}}%
\pgfpathlineto{\pgfqpoint{4.668160in}{0.521603in}}%
\pgfpathclose%
\pgfusepath{stroke,fill}%
\end{pgfscope}%
\begin{pgfscope}%
\pgfsys@transformshift{4.520000in}{0.526603in}%
\pgftext[left,bottom]{\pgfimage[interpolate=true,width=0.150000in,height=3.020000in]{restitution-img0.png}}%
\end{pgfscope}%
\begin{pgfscope}%
\pgfsetbuttcap%
\pgfsetroundjoin%
\definecolor{currentfill}{rgb}{0.000000,0.000000,0.000000}%
\pgfsetfillcolor{currentfill}%
\pgfsetlinewidth{0.803000pt}%
\definecolor{currentstroke}{rgb}{0.000000,0.000000,0.000000}%
\pgfsetstrokecolor{currentstroke}%
\pgfsetdash{}{0pt}%
\pgfsys@defobject{currentmarker}{\pgfqpoint{0.000000in}{0.000000in}}{\pgfqpoint{0.048611in}{0.000000in}}{%
\pgfpathmoveto{\pgfqpoint{0.000000in}{0.000000in}}%
\pgfpathlineto{\pgfqpoint{0.048611in}{0.000000in}}%
\pgfusepath{stroke,fill}%
}%
\begin{pgfscope}%
\pgfsys@transformshift{4.668160in}{0.925593in}%
\pgfsys@useobject{currentmarker}{}%
\end{pgfscope}%
\end{pgfscope}%
\begin{pgfscope}%
\pgftext[x=4.765383in,y=0.872831in,left,base]{\rmfamily\fontsize{10.000000}{12.000000}\selectfont \(\displaystyle 0.2\)}%
\end{pgfscope}%
\begin{pgfscope}%
\pgfsetbuttcap%
\pgfsetroundjoin%
\definecolor{currentfill}{rgb}{0.000000,0.000000,0.000000}%
\pgfsetfillcolor{currentfill}%
\pgfsetlinewidth{0.803000pt}%
\definecolor{currentstroke}{rgb}{0.000000,0.000000,0.000000}%
\pgfsetstrokecolor{currentstroke}%
\pgfsetdash{}{0pt}%
\pgfsys@defobject{currentmarker}{\pgfqpoint{0.000000in}{0.000000in}}{\pgfqpoint{0.048611in}{0.000000in}}{%
\pgfpathmoveto{\pgfqpoint{0.000000in}{0.000000in}}%
\pgfpathlineto{\pgfqpoint{0.048611in}{0.000000in}}%
\pgfusepath{stroke,fill}%
}%
\begin{pgfscope}%
\pgfsys@transformshift{4.668160in}{1.540182in}%
\pgfsys@useobject{currentmarker}{}%
\end{pgfscope}%
\end{pgfscope}%
\begin{pgfscope}%
\pgftext[x=4.765383in,y=1.487421in,left,base]{\rmfamily\fontsize{10.000000}{12.000000}\selectfont \(\displaystyle 0.4\)}%
\end{pgfscope}%
\begin{pgfscope}%
\pgfsetbuttcap%
\pgfsetroundjoin%
\definecolor{currentfill}{rgb}{0.000000,0.000000,0.000000}%
\pgfsetfillcolor{currentfill}%
\pgfsetlinewidth{0.803000pt}%
\definecolor{currentstroke}{rgb}{0.000000,0.000000,0.000000}%
\pgfsetstrokecolor{currentstroke}%
\pgfsetdash{}{0pt}%
\pgfsys@defobject{currentmarker}{\pgfqpoint{0.000000in}{0.000000in}}{\pgfqpoint{0.048611in}{0.000000in}}{%
\pgfpathmoveto{\pgfqpoint{0.000000in}{0.000000in}}%
\pgfpathlineto{\pgfqpoint{0.048611in}{0.000000in}}%
\pgfusepath{stroke,fill}%
}%
\begin{pgfscope}%
\pgfsys@transformshift{4.668160in}{2.154772in}%
\pgfsys@useobject{currentmarker}{}%
\end{pgfscope}%
\end{pgfscope}%
\begin{pgfscope}%
\pgftext[x=4.765383in,y=2.102010in,left,base]{\rmfamily\fontsize{10.000000}{12.000000}\selectfont \(\displaystyle 0.6\)}%
\end{pgfscope}%
\begin{pgfscope}%
\pgfsetbuttcap%
\pgfsetroundjoin%
\definecolor{currentfill}{rgb}{0.000000,0.000000,0.000000}%
\pgfsetfillcolor{currentfill}%
\pgfsetlinewidth{0.803000pt}%
\definecolor{currentstroke}{rgb}{0.000000,0.000000,0.000000}%
\pgfsetstrokecolor{currentstroke}%
\pgfsetdash{}{0pt}%
\pgfsys@defobject{currentmarker}{\pgfqpoint{0.000000in}{0.000000in}}{\pgfqpoint{0.048611in}{0.000000in}}{%
\pgfpathmoveto{\pgfqpoint{0.000000in}{0.000000in}}%
\pgfpathlineto{\pgfqpoint{0.048611in}{0.000000in}}%
\pgfusepath{stroke,fill}%
}%
\begin{pgfscope}%
\pgfsys@transformshift{4.668160in}{2.769361in}%
\pgfsys@useobject{currentmarker}{}%
\end{pgfscope}%
\end{pgfscope}%
\begin{pgfscope}%
\pgftext[x=4.765383in,y=2.716599in,left,base]{\rmfamily\fontsize{10.000000}{12.000000}\selectfont \(\displaystyle 0.8\)}%
\end{pgfscope}%
\begin{pgfscope}%
\pgfsetbuttcap%
\pgfsetroundjoin%
\definecolor{currentfill}{rgb}{0.000000,0.000000,0.000000}%
\pgfsetfillcolor{currentfill}%
\pgfsetlinewidth{0.803000pt}%
\definecolor{currentstroke}{rgb}{0.000000,0.000000,0.000000}%
\pgfsetstrokecolor{currentstroke}%
\pgfsetdash{}{0pt}%
\pgfsys@defobject{currentmarker}{\pgfqpoint{0.000000in}{0.000000in}}{\pgfqpoint{0.048611in}{0.000000in}}{%
\pgfpathmoveto{\pgfqpoint{0.000000in}{0.000000in}}%
\pgfpathlineto{\pgfqpoint{0.048611in}{0.000000in}}%
\pgfusepath{stroke,fill}%
}%
\begin{pgfscope}%
\pgfsys@transformshift{4.668160in}{3.383950in}%
\pgfsys@useobject{currentmarker}{}%
\end{pgfscope}%
\end{pgfscope}%
\begin{pgfscope}%
\pgftext[x=4.765383in,y=3.331189in,left,base]{\rmfamily\fontsize{10.000000}{12.000000}\selectfont \(\displaystyle 1.0\)}%
\end{pgfscope}%
\begin{pgfscope}%
\pgftext[x=4.998408in,y=2.031603in,,top,rotate=90.000000]{\rmfamily\fontsize{10.000000}{12.000000}\selectfont \(\displaystyle \mathrm{\mathit{Bo_e}} \equiv \frac{\epsilon E_0^2 R_0}{\gamma}\)}%
\end{pgfscope}%
\begin{pgfscope}%
\pgfsetbuttcap%
\pgfsetmiterjoin%
\pgfsetlinewidth{0.803000pt}%
\definecolor{currentstroke}{rgb}{0.000000,0.000000,0.000000}%
\pgfsetstrokecolor{currentstroke}%
\pgfsetdash{}{0pt}%
\pgfpathmoveto{\pgfqpoint{4.517160in}{0.521603in}}%
\pgfpathlineto{\pgfqpoint{4.517160in}{0.533400in}}%
\pgfpathlineto{\pgfqpoint{4.517160in}{3.529806in}}%
\pgfpathlineto{\pgfqpoint{4.517160in}{3.541603in}}%
\pgfpathlineto{\pgfqpoint{4.668160in}{3.541603in}}%
\pgfpathlineto{\pgfqpoint{4.668160in}{3.529806in}}%
\pgfpathlineto{\pgfqpoint{4.668160in}{0.533400in}}%
\pgfpathlineto{\pgfqpoint{4.668160in}{0.521603in}}%
\pgfpathclose%
\pgfusepath{stroke}%
\end{pgfscope}%
\end{pgfpicture}%
\makeatother%
\endgroup%
}
    \caption{Impact coefficient of restitution $C_r$ compared with impact $\mathbb{W}\mbox{e}$ and $\mathbb{B}\mbox{o}_e$.\label{fig:restitution}}
\end{figure}

\begin{acknowledgments}
This work was supported by NASA Cooperative Agreement NNX12A047A. We are also indebted to Andrew Greenberg for his frequent technical advice, and to Joe Shields for assistance with drop tower experiments.
\end{acknowledgments}

% If in two-column mode, this environment will change to single-column format so that long equations can be displayed. 
% Use only when necessary.
%\begin{widetext}
%$$\mbox{put long equation here}$$
%\end{widetext}

% Figures should be put into the text as floats. 
% Use the graphics or graphicx packages (distributed with LaTeX2e).
% See the LaTeX Graphics Companion by Michel Goosens, Sebastian Rahtz, and Frank Mittelbach for examples. 
%
% Here is an example of the general form of a figure:
% Fill in the caption in the braces of the \caption{} command. 
% Put the label that you will use with \ref{} command in the braces of the \label{} command.
%
% \begin{figure}
% \includegraphics{}%
% \caption{\label{}}%
% \end{figure}

% Tables may be be put in the text as floats.
% Here is an example of the general form of a table:
% Fill in the caption in the braces of the \caption{} command. Put the label
% that you will use with \ref{} command in the braces of the \label{} command.
% Insert the column specifiers (l, r, c, d, etc.) in the empty braces of the
% \begin{tabular}{} command.
%
% \begin{table}
% \caption{\label{} }
% \begin{tabular}{}
% \end{tabular}
% \end{table}

% If you have acknowledgments, this puts in the proper section head.
%\begin{acknowledgments}
% Put your acknowledgments here.
%\end{acknowledgments}

% Create the reference section using BibTeX:
\bibliography{thesis}

\end{document}
%
% ****** End of file aiptemplate.tex ******

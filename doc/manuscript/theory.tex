\documentclass[a4paper, 12pt]{article}
\usepackage{changepage}
\usepackage{color,soul}
\usepackage{listings}
\usepackage{verbatim}
\usepackage{pgfplots}
\usepackage{pgf}
\usepackage[english]{babel}
\usepackage{amsmath}
\usepackage{amsfonts}
\usepackage{amssymb}
\usepackage{graphicx}
\usepackage{setspace}
\lstset{
basicstyle=\small\ttfamily,
columns=flexible,
breaklines=true
}
\title{\textsf{\textbf{Droplet Electro-Bouncing in $\mu$-Gravity}}}
\vspace{-25mm}
\author{Erin S. Schmidt, Mark M. Weislogel}
\date{}

\usepackage{abstract}
\renewcommand{\abstractnamefont}{\normalfont\bfseries}
\renewcommand{\abstracttextfont}{\normalfont\small\itshape}

\usepackage{setspace}
\begin{document}
%\maketitle

\doublespacing

\section{Trajectory Model}
\subsection{Forces and the Equation of Motion}
Treating a droplet as a particle, with radius $R_d$, the  equation of motion for a charged droplet translating vertically along the central axis of a finite square charged dielectric sheet is given by
\begin{equation}
m y'' = - \mathbf{F}_D - \mathbf{F}_E, \hspace{5 mm} y(0) = R_d, \hspace{5 mm} y'(0) = U_0,
\label{gov_eqn}
\end{equation}
where $m$ is the droplet mass, $y'' = \frac{d^2 y}{d t^2}$ is the droplet acceleration, $\mathbf{F}_D$ is the drag force, and $\mathbf{F}_E$ is the electrostatic force. The initial conditions are such that the droplet leaves its resting position $R_d$ at $t=0$, with initial velocity $U_0$. The signs of the forces on the right side of Equation \ref{gov_eqn} indicate that they act in the opposite direction of the droplet motion. In this section we aim specifiy models for the various forces in this equation.

If we assume an intermediate range of Reynolds numbers $\mathbb{R}\mbox{e} \equiv \frac{2UR_d}{\nu}$, $1 <\mathbb{R}\mbox{e} < 1000 $ then the drag is quadratic,
\begin{equation*}\label{drag_force}
\mathbf{F}_D = \frac{1}{2}C_D \rho A {y'}^2,
\end{equation*}
where $C_D$ is the drag coefficient, $\rho$ is the density of the sorrounding fluid medium (air, in this case), and $A$ is the frontal area of the droplet. For this range of Reynolds numbers we may also approximate the drag coefficient by the well known Abraham correlation \hl{ref.}
\[C_D = \frac{24}{9.06^2} \left( 1 + \frac{9.06}{\sqrt{\mathbb{R}\mbox{e}}} \right)^2\]

Modeling the electrostatic force is somehwhat more involved, but we will adopt the standard electrohydrodynamic (EHD) approximation model because of the dramatic simplifications it offers \hl{ref}. We first assume a DC electric field, such that $Re \langle \epsilon \rangle \approx  \mbox{constant}$, where $\epsilon$ is the dielectric permittivity of the respective media. We also assume that currents are small such that the effects of magnetic fields can be neglected. For the validity of this assumption to hold the characteristic time scale for electrical phenomena $\tau_e = \epsilon \epsilon_0/\sigma_e \ll 1$, where $\tau$ is the ratio of absolute dielectric permittivity $\kappa = \epsilon \epsilon_0$, to conductivity $\sigma_e$, of the medium [ref]. \hl{Given the respective conductivity, and permittivity of water ($\sigma_e = 2.5 \cdot 10^{-4}$ $\Omega^{-1} \mbox{cm}^{-1}$), we estimate $\tau_e \approx 7 \times 10^{-8}$ s}. This assumption also allows us to assume that the net charge present in the medium surrounding the droplets remains approximately constant during the typical time interval of a low-gravity experiment, and no transfers of charge occur after the droplet leaves the surface.

If we suppose that electrical forces acting on free charges and dipoles in a fluid are transferred directly to the fluid itself, then this overall electrical body force will be the the divergence of the Maxwell stress tensor $\tau_m $, by

\[ \mathbf{F}_E = \nabla \cdot \tau_m = \nabla \cdot \left( \epsilon \epsilon_0 \mathbf{E} \mathbf{E} - \frac{1}{2} \epsilon \epsilon_0 \mathbf{E} \cdot \mathbf{E} \delta \right) ,\]
where $\mathbf{F}_E$ is the electric body force per unit volume, and $\delta$ is the delta function. The product of the electric field vectors is the dyadic product.  

The classical Korteweg-Helmholtz force density formulation of the Maxwell stress tensor is usually expressed as \hl{[ref]}
\begin{equation}\label{force_density}
\mathbf{F}_E = \rho_f \mathbf{E} + \frac{1}{2} \left| \mathbf{E} \right|^2 \nabla \epsilon - \nabla \left( \frac{1}{2} \rho \left( \frac{\partial \epsilon}{\partial \rho} \right)_T \left| \mathbf{E} \right|^2 \right) .
\end{equation}

The first term in this expression, equivalently written as $q\mathbf{E}$, is the well known Coulombic force or electrophoretic force, which arises from the presence of free charge in an external electric field. The second term is the force arising from polarization stresses due to a nonuniform field acting across a gradient in permittivity. This force is widely termed the dielectrophoretic force (DEP). The third term describes forces due to electrostriction. It has been noted by Melcher and Hurwitz that the electrostriction term is the gradient of a scalar and can thus be cannononically lumped togeather with the hydrostatic pressure for incompressible fluids)\hl{[ref]}; we neglect it in our analysis. 

It is common to approximate the polarization stress by idealizing the droplet as a simple dipole using the effective dipole moment method first suggested by Pohl and Jones\hl{[ref][ref]}. This approach can be related back to the force density by means of a Taylor series expansion of $\mathbf{E}$ in the limit of a small gradient \hl{[ref]}. The DEP force is distinct from the Coulombic force in that net charge is not required, and that the force vector goes in the direction of the gradient of the field, $\nabla \left| \mathbf{E} \right|^2$, rather than in the direction of $\mathbf{E}$. The DEP force is related to the dipole moment (induced or permanent) of polarizable media which has a tendency to align the dipole with the electric field. If there is a gradient in the field then for a finite separation of charge one end of the dipole will feel a stronger electric field than the other, resulting in a net force. Whether the force is positive or negative in the direction of the electric field gradient depends on the difference of dielectric permittivites between the fluids, rather than on the polarity of $\mathbf{E}$ itself. It bears repeating that droplets will polarize in a uniform field, but since there is no gradient in the field the forces felt by the dipoles are symmetric and there is no net force. The dipole moment of a spherical linear-dielectric particle immersed in a dielectric medium is given by
\begin{equation} \label{dipole_m_1}
\mu = V_d \mathbf{P} = \frac{4}{3} \pi R_d^3 \mathbf{P},
\end{equation} 
where $\mathbf{P} = \left(\kappa_1 - 1 \right) \epsilon_0 \mathbf{E}_{iz} = \chi_e \epsilon_0 \mathbf{E}_{iz}$ is the polarization moment, and $R_d$ is the particle radius, \hl{$\kappa_1 = \frac{\epsilon}{\epsilon_0}$ being the relative dielectric constant of the medium (air in this case)}, $\chi_e = \kappa_1 - 1$ being the electric susceptibility of the dielectric medium, and $\mathbf{E}_{iz}$ is the $z$-coordinate component of the electric field internal to the sphere, assuming the external electric field to be oriented parallel to the $z$-axis. The excess polarization $\mathbf{P}_e$, in the sphere is given by
\begin{equation} \label{polarization}
\mathbf{P}_e = \left( \kappa_2 - \kappa_1 \right) \epsilon_0 \mathbf{E}_{iz} = \frac{3 \kappa_1}{\kappa_2 +2\kappa_1}\mathbf{E}_{iz},
\end{equation}
where $\kappa_2$ is the relative dielectric constant of the spherical particle. Taking together equations \ref{dipole_m_1}, and \ref{polarization} we find that the effective dipole moment of the particle is given by 
\begin{equation}\label{dipole_m_2}
\mu = 4 \pi R_d^3 \left( \frac{\kappa_2 - \kappa_1}{\kappa_2 + 2 \kappa_1} \right) \kappa_1 \epsilon_0 \mathbf{E},
\end{equation}
and the force felt by the dipole is 
\begin{eqnarray} \label{dep_force}
\mathbf{F}_{DEP} &=& \left( \mathbf{P}_e \cdot \nabla \right) \mathbf{E} \nonumber \\
&=& 2 \pi R_d^3 \kappa_1 \epsilon_0 K \nabla E^2,
\end{eqnarray}
where it is an asthetically pleasing shorthand to refer to the permittivity ratio by $K = \frac{\kappa_2 - \kappa_1}{\kappa_2 + 2 \kappa_1}$, which is also known as the Clausius-Mossotti factor. In cases where $K <$ 0, or $K>$ 0 the particle will be repelled or attracted to regions of strong field respectively. In our experiment, taking the relative dielectric constants to be $\kappa_1 \approx$ 1 and $\kappa_2 \approx$ 80, we have $K \approx$ 0.96. We also note that the equivalent dipole approximation requires an assumption of small physical scale of the particle relative to the lengthscale of nonuniformity of the field, which in this case we take to be the length of the charged superhydrophobic surface ($L =$ 25 mm $\gg a \approx$ 2.5 mm).

When the droplet is close to the dielectric surface, the net charge on the droplet will tend to polarize the dielectric, perturbing the electric field. The polarization bound charge in the dielectric will be of the opposite sign of the net droplet charge and thus there will be a force of attraction. This so-called image force is a correction to the Colomb force due to the external electric field only, and can be found by the method of images \hl{ref}. The image force $\mathbf{F}_I$, is given by

\begin{equation}
\mathbf{F}_I = \frac{k q^2}{16 \pi \epsilon_0} y^{-2} \hat{\mathbf{j}},
\label{image_force}
\end{equation}
where the factor $k$ is a function of the dielectric surface susceptibility $k = \frac{\chi_e}{\chi_e + 2}$, and $\hat{\mathbf{j}}$ is a unit vector normal to the dielectric surface.

By substituting Equations \ref{dep_force}, \ref{image_force} into Equation \ref{force_density} we have
\begin{eqnarray*}
 \mathbf{F}_E &=& q \mathbf{E} + \mathbf{F}_{DEP} + \mathbf{F}_I \\
 &=& q \mathbf{E} + \frac{k q^2}{16 \pi \epsilon_0 } y^{-2} \hat{\mathbf{j}} + 2 \pi R_d^3 \kappa_1 \epsilon_0 K \nabla E^2, 
\end{eqnarray*}

and the 1-D governing equation becomes
\begin{eqnarray} \label{gov_eqn_subs}
&m y'' = - \frac{1}{2} C_D \rho A {y'}^2 - q E - \frac{k q^2}{16 \pi \epsilon_0} y^{-2}- 2 \pi R_d^3 \kappa_1 \epsilon_0 K \nabla E^2,& \nonumber \\
&y(0) = R, \hspace{1 mm} y'(0) = U_0 .&
\end{eqnarray}

By comparing DEP and Coulombic terms in Equation \ref{gov_eqn_subs}, we note that a condition to neglect the DEP term is
\begin{eqnarray}
1 &\gg& \frac{R_d^2 \kappa_2 \epsilon_0 K E_0}{q} \nonumber
\end{eqnarray}
As this condition holds in all cases under study we herefore neglect the DEP force in our analysis.


\subsection{The Electric Field}
If we consider the charged dielectric surface of our experiments to be a square sheet of charge lying in the $xz$-plane with width $L$, the symmetry of the problem happily lets us obtain the $y$-component of the electric field $\mathbf{E}_y$ by direct integration \hl{ref}. In particular it is easy to constuct the electric field due to a finite plane of charge by supoerposition of the electric fields of a series of line charges. By symmetry the electric field points along the y-axis; for a point along the $y$-axis the position vector is $\mathbf{r} = \left( x^2 + y^2 + z^2 \right)^{1/2} \hat{\mathbf{r}}$. The $y$-component of $\mathbf{E}$ is given by $\mathbf{E}_y \cos \theta = \mathbf{E}_y y/ \mathbf{r}$. Then if the charge in an element of area, $dx dz$ is $\sigma dx dz$ the electric field $\mathbf{E}_y$ is
\[ \mathbf{E}_y = \frac{\sigma y}{4 \pi \epsilon_0} \int^{L/2}_{L/2} \int^{L/2}_{L/2} \left( x^2 + y^2 + z^2 \right)^{3/2} dx dz \hspace{1mm}\hat{\mathbf{r}} 
,\]
where $\sigma$ is the surface charge density. This can be easily integrated to obtain an expression for the electric field in terms of $y$, 
\begin{equation}
\mathbf{E}_y = \frac{\sigma y}{ \pi \epsilon_0} \tan^{-1} \left( \frac{L^2}{y \sqrt{2L^2 + y^2}}\right)
.\end{equation}

By taking Taylor series expansions in large and small limits we can intuit a bit about the behavior of this field. In the limit $L \rightarrow \infty, \hspace{1mm} y \ll L$ the argument of the function tends towards infinity and
\[ \lim_{x\to\infty} \tan^{-1}(x) = \frac{\pi}{2}
,\]
and thus
\begin{equation}
\mathbf{E}_y \approx \frac{\sigma}{4 \pi \epsilon_0} \hat{\mathbf{j}} \hspace{10mm} y \ll L
,\end{equation}
which is constant, and equivalent to the electric field due to an infinite plane of charge. In the limit of $y \gg L$, the argument of the arctangent function can be approximated by
\[ \frac{L^2}{2 y \left( 2 L^2 + 4 y^2\right)^{1/2}} = \frac{L^2}{4 y^2 \left( 1 + L^2/2y^2 \right)^{1/2}} \approx \frac{L^2}{4 y^2}
.\]
For small $x$, $\tan^{-1}(x) \sim x$ and we thus find the familiar electric field due to a point charge
\begin{equation}
\mathbf{E}_y \approx \frac{\sigma L^2}{4 \pi \epsilon_0} y^{-2} \hat{\mathbf{j}} \hspace{10mm} y \gg L
.\end{equation}
With the characteristic electric field given by $E_0 = \frac{\sigma}{4 \pi \epsilon_0}$, both these regiemes can be clearly seen in the plot of $\mathbf{E}_y$ shown in Figure \ref{fig:E0}.
\begin{figure}[htb]
    \centering
    %% Creator: Matplotlib, PGF backend
%%
%% To include the figure in your LaTeX document, write
%%   \input{<filename>.pgf}
%%
%% Make sure the required packages are loaded in your preamble
%%   \usepackage{pgf}
%%
%% Figures using additional raster images can only be included by \input if
%% they are in the same directory as the main LaTeX file. For loading figures
%% from other directories you can use the `import` package
%%   \usepackage{import}
%% and then include the figures with
%%   \import{<path to file>}{<filename>.pgf}
%%
%% Matplotlib used the following preamble
%%   \usepackage{fontspec}
%%   \setmainfont{DejaVu Serif}
%%   \setsansfont{DejaVu Sans}
%%   \setmonofont{DejaVu Sans Mono}
%%
\begingroup%
\makeatletter%
\begin{pgfpicture}%
\pgfpathrectangle{\pgfpointorigin}{\pgfqpoint{5.464669in}{3.681079in}}%
\pgfusepath{use as bounding box, clip}%
\begin{pgfscope}%
\pgfsetbuttcap%
\pgfsetmiterjoin%
\definecolor{currentfill}{rgb}{1.000000,1.000000,1.000000}%
\pgfsetfillcolor{currentfill}%
\pgfsetlinewidth{0.000000pt}%
\definecolor{currentstroke}{rgb}{1.000000,1.000000,1.000000}%
\pgfsetstrokecolor{currentstroke}%
\pgfsetdash{}{0pt}%
\pgfpathmoveto{\pgfqpoint{0.000000in}{0.000000in}}%
\pgfpathlineto{\pgfqpoint{5.464669in}{0.000000in}}%
\pgfpathlineto{\pgfqpoint{5.464669in}{3.681079in}}%
\pgfpathlineto{\pgfqpoint{0.000000in}{3.681079in}}%
\pgfpathclose%
\pgfusepath{fill}%
\end{pgfscope}%
\begin{pgfscope}%
\pgfsetbuttcap%
\pgfsetmiterjoin%
\definecolor{currentfill}{rgb}{1.000000,1.000000,1.000000}%
\pgfsetfillcolor{currentfill}%
\pgfsetlinewidth{0.000000pt}%
\definecolor{currentstroke}{rgb}{0.000000,0.000000,0.000000}%
\pgfsetstrokecolor{currentstroke}%
\pgfsetstrokeopacity{0.000000}%
\pgfsetdash{}{0pt}%
\pgfpathmoveto{\pgfqpoint{0.679669in}{0.526079in}}%
\pgfpathlineto{\pgfqpoint{5.329669in}{0.526079in}}%
\pgfpathlineto{\pgfqpoint{5.329669in}{3.546079in}}%
\pgfpathlineto{\pgfqpoint{0.679669in}{3.546079in}}%
\pgfpathclose%
\pgfusepath{fill}%
\end{pgfscope}%
\begin{pgfscope}%
\pgfsetbuttcap%
\pgfsetroundjoin%
\definecolor{currentfill}{rgb}{0.000000,0.000000,0.000000}%
\pgfsetfillcolor{currentfill}%
\pgfsetlinewidth{0.803000pt}%
\definecolor{currentstroke}{rgb}{0.000000,0.000000,0.000000}%
\pgfsetstrokecolor{currentstroke}%
\pgfsetdash{}{0pt}%
\pgfsys@defobject{currentmarker}{\pgfqpoint{0.000000in}{-0.048611in}}{\pgfqpoint{0.000000in}{0.000000in}}{%
\pgfpathmoveto{\pgfqpoint{0.000000in}{0.000000in}}%
\pgfpathlineto{\pgfqpoint{0.000000in}{-0.048611in}}%
\pgfusepath{stroke,fill}%
}%
\begin{pgfscope}%
\pgfsys@transformshift{2.211681in}{0.526079in}%
\pgfsys@useobject{currentmarker}{}%
\end{pgfscope}%
\end{pgfscope}%
\begin{pgfscope}%
\pgftext[x=2.211681in,y=0.428857in,,top]{\rmfamily\fontsize{10.000000}{12.000000}\selectfont \(\displaystyle 10^{-1}\)}%
\end{pgfscope}%
\begin{pgfscope}%
\pgfsetbuttcap%
\pgfsetroundjoin%
\definecolor{currentfill}{rgb}{0.000000,0.000000,0.000000}%
\pgfsetfillcolor{currentfill}%
\pgfsetlinewidth{0.803000pt}%
\definecolor{currentstroke}{rgb}{0.000000,0.000000,0.000000}%
\pgfsetstrokecolor{currentstroke}%
\pgfsetdash{}{0pt}%
\pgfsys@defobject{currentmarker}{\pgfqpoint{0.000000in}{-0.048611in}}{\pgfqpoint{0.000000in}{0.000000in}}{%
\pgfpathmoveto{\pgfqpoint{0.000000in}{0.000000in}}%
\pgfpathlineto{\pgfqpoint{0.000000in}{-0.048611in}}%
\pgfusepath{stroke,fill}%
}%
\begin{pgfscope}%
\pgfsys@transformshift{3.836268in}{0.526079in}%
\pgfsys@useobject{currentmarker}{}%
\end{pgfscope}%
\end{pgfscope}%
\begin{pgfscope}%
\pgftext[x=3.836268in,y=0.428857in,,top]{\rmfamily\fontsize{10.000000}{12.000000}\selectfont \(\displaystyle 10^{0}\)}%
\end{pgfscope}%
\begin{pgfscope}%
\pgfsetbuttcap%
\pgfsetroundjoin%
\definecolor{currentfill}{rgb}{0.000000,0.000000,0.000000}%
\pgfsetfillcolor{currentfill}%
\pgfsetlinewidth{0.602250pt}%
\definecolor{currentstroke}{rgb}{0.000000,0.000000,0.000000}%
\pgfsetstrokecolor{currentstroke}%
\pgfsetdash{}{0pt}%
\pgfsys@defobject{currentmarker}{\pgfqpoint{0.000000in}{-0.027778in}}{\pgfqpoint{0.000000in}{0.000000in}}{%
\pgfpathmoveto{\pgfqpoint{0.000000in}{0.000000in}}%
\pgfpathlineto{\pgfqpoint{0.000000in}{-0.027778in}}%
\pgfusepath{stroke,fill}%
}%
\begin{pgfscope}%
\pgfsys@transformshift{1.076144in}{0.526079in}%
\pgfsys@useobject{currentmarker}{}%
\end{pgfscope}%
\end{pgfscope}%
\begin{pgfscope}%
\pgfsetbuttcap%
\pgfsetroundjoin%
\definecolor{currentfill}{rgb}{0.000000,0.000000,0.000000}%
\pgfsetfillcolor{currentfill}%
\pgfsetlinewidth{0.602250pt}%
\definecolor{currentstroke}{rgb}{0.000000,0.000000,0.000000}%
\pgfsetstrokecolor{currentstroke}%
\pgfsetdash{}{0pt}%
\pgfsys@defobject{currentmarker}{\pgfqpoint{0.000000in}{-0.027778in}}{\pgfqpoint{0.000000in}{0.000000in}}{%
\pgfpathmoveto{\pgfqpoint{0.000000in}{0.000000in}}%
\pgfpathlineto{\pgfqpoint{0.000000in}{-0.027778in}}%
\pgfusepath{stroke,fill}%
}%
\begin{pgfscope}%
\pgfsys@transformshift{1.362219in}{0.526079in}%
\pgfsys@useobject{currentmarker}{}%
\end{pgfscope}%
\end{pgfscope}%
\begin{pgfscope}%
\pgfsetbuttcap%
\pgfsetroundjoin%
\definecolor{currentfill}{rgb}{0.000000,0.000000,0.000000}%
\pgfsetfillcolor{currentfill}%
\pgfsetlinewidth{0.602250pt}%
\definecolor{currentstroke}{rgb}{0.000000,0.000000,0.000000}%
\pgfsetstrokecolor{currentstroke}%
\pgfsetdash{}{0pt}%
\pgfsys@defobject{currentmarker}{\pgfqpoint{0.000000in}{-0.027778in}}{\pgfqpoint{0.000000in}{0.000000in}}{%
\pgfpathmoveto{\pgfqpoint{0.000000in}{0.000000in}}%
\pgfpathlineto{\pgfqpoint{0.000000in}{-0.027778in}}%
\pgfusepath{stroke,fill}%
}%
\begin{pgfscope}%
\pgfsys@transformshift{1.565193in}{0.526079in}%
\pgfsys@useobject{currentmarker}{}%
\end{pgfscope}%
\end{pgfscope}%
\begin{pgfscope}%
\pgfsetbuttcap%
\pgfsetroundjoin%
\definecolor{currentfill}{rgb}{0.000000,0.000000,0.000000}%
\pgfsetfillcolor{currentfill}%
\pgfsetlinewidth{0.602250pt}%
\definecolor{currentstroke}{rgb}{0.000000,0.000000,0.000000}%
\pgfsetstrokecolor{currentstroke}%
\pgfsetdash{}{0pt}%
\pgfsys@defobject{currentmarker}{\pgfqpoint{0.000000in}{-0.027778in}}{\pgfqpoint{0.000000in}{0.000000in}}{%
\pgfpathmoveto{\pgfqpoint{0.000000in}{0.000000in}}%
\pgfpathlineto{\pgfqpoint{0.000000in}{-0.027778in}}%
\pgfusepath{stroke,fill}%
}%
\begin{pgfscope}%
\pgfsys@transformshift{1.722632in}{0.526079in}%
\pgfsys@useobject{currentmarker}{}%
\end{pgfscope}%
\end{pgfscope}%
\begin{pgfscope}%
\pgfsetbuttcap%
\pgfsetroundjoin%
\definecolor{currentfill}{rgb}{0.000000,0.000000,0.000000}%
\pgfsetfillcolor{currentfill}%
\pgfsetlinewidth{0.602250pt}%
\definecolor{currentstroke}{rgb}{0.000000,0.000000,0.000000}%
\pgfsetstrokecolor{currentstroke}%
\pgfsetdash{}{0pt}%
\pgfsys@defobject{currentmarker}{\pgfqpoint{0.000000in}{-0.027778in}}{\pgfqpoint{0.000000in}{0.000000in}}{%
\pgfpathmoveto{\pgfqpoint{0.000000in}{0.000000in}}%
\pgfpathlineto{\pgfqpoint{0.000000in}{-0.027778in}}%
\pgfusepath{stroke,fill}%
}%
\begin{pgfscope}%
\pgfsys@transformshift{1.851269in}{0.526079in}%
\pgfsys@useobject{currentmarker}{}%
\end{pgfscope}%
\end{pgfscope}%
\begin{pgfscope}%
\pgfsetbuttcap%
\pgfsetroundjoin%
\definecolor{currentfill}{rgb}{0.000000,0.000000,0.000000}%
\pgfsetfillcolor{currentfill}%
\pgfsetlinewidth{0.602250pt}%
\definecolor{currentstroke}{rgb}{0.000000,0.000000,0.000000}%
\pgfsetstrokecolor{currentstroke}%
\pgfsetdash{}{0pt}%
\pgfsys@defobject{currentmarker}{\pgfqpoint{0.000000in}{-0.027778in}}{\pgfqpoint{0.000000in}{0.000000in}}{%
\pgfpathmoveto{\pgfqpoint{0.000000in}{0.000000in}}%
\pgfpathlineto{\pgfqpoint{0.000000in}{-0.027778in}}%
\pgfusepath{stroke,fill}%
}%
\begin{pgfscope}%
\pgfsys@transformshift{1.960030in}{0.526079in}%
\pgfsys@useobject{currentmarker}{}%
\end{pgfscope}%
\end{pgfscope}%
\begin{pgfscope}%
\pgfsetbuttcap%
\pgfsetroundjoin%
\definecolor{currentfill}{rgb}{0.000000,0.000000,0.000000}%
\pgfsetfillcolor{currentfill}%
\pgfsetlinewidth{0.602250pt}%
\definecolor{currentstroke}{rgb}{0.000000,0.000000,0.000000}%
\pgfsetstrokecolor{currentstroke}%
\pgfsetdash{}{0pt}%
\pgfsys@defobject{currentmarker}{\pgfqpoint{0.000000in}{-0.027778in}}{\pgfqpoint{0.000000in}{0.000000in}}{%
\pgfpathmoveto{\pgfqpoint{0.000000in}{0.000000in}}%
\pgfpathlineto{\pgfqpoint{0.000000in}{-0.027778in}}%
\pgfusepath{stroke,fill}%
}%
\begin{pgfscope}%
\pgfsys@transformshift{2.054242in}{0.526079in}%
\pgfsys@useobject{currentmarker}{}%
\end{pgfscope}%
\end{pgfscope}%
\begin{pgfscope}%
\pgfsetbuttcap%
\pgfsetroundjoin%
\definecolor{currentfill}{rgb}{0.000000,0.000000,0.000000}%
\pgfsetfillcolor{currentfill}%
\pgfsetlinewidth{0.602250pt}%
\definecolor{currentstroke}{rgb}{0.000000,0.000000,0.000000}%
\pgfsetstrokecolor{currentstroke}%
\pgfsetdash{}{0pt}%
\pgfsys@defobject{currentmarker}{\pgfqpoint{0.000000in}{-0.027778in}}{\pgfqpoint{0.000000in}{0.000000in}}{%
\pgfpathmoveto{\pgfqpoint{0.000000in}{0.000000in}}%
\pgfpathlineto{\pgfqpoint{0.000000in}{-0.027778in}}%
\pgfusepath{stroke,fill}%
}%
\begin{pgfscope}%
\pgfsys@transformshift{2.137344in}{0.526079in}%
\pgfsys@useobject{currentmarker}{}%
\end{pgfscope}%
\end{pgfscope}%
\begin{pgfscope}%
\pgfsetbuttcap%
\pgfsetroundjoin%
\definecolor{currentfill}{rgb}{0.000000,0.000000,0.000000}%
\pgfsetfillcolor{currentfill}%
\pgfsetlinewidth{0.602250pt}%
\definecolor{currentstroke}{rgb}{0.000000,0.000000,0.000000}%
\pgfsetstrokecolor{currentstroke}%
\pgfsetdash{}{0pt}%
\pgfsys@defobject{currentmarker}{\pgfqpoint{0.000000in}{-0.027778in}}{\pgfqpoint{0.000000in}{0.000000in}}{%
\pgfpathmoveto{\pgfqpoint{0.000000in}{0.000000in}}%
\pgfpathlineto{\pgfqpoint{0.000000in}{-0.027778in}}%
\pgfusepath{stroke,fill}%
}%
\begin{pgfscope}%
\pgfsys@transformshift{2.700731in}{0.526079in}%
\pgfsys@useobject{currentmarker}{}%
\end{pgfscope}%
\end{pgfscope}%
\begin{pgfscope}%
\pgfsetbuttcap%
\pgfsetroundjoin%
\definecolor{currentfill}{rgb}{0.000000,0.000000,0.000000}%
\pgfsetfillcolor{currentfill}%
\pgfsetlinewidth{0.602250pt}%
\definecolor{currentstroke}{rgb}{0.000000,0.000000,0.000000}%
\pgfsetstrokecolor{currentstroke}%
\pgfsetdash{}{0pt}%
\pgfsys@defobject{currentmarker}{\pgfqpoint{0.000000in}{-0.027778in}}{\pgfqpoint{0.000000in}{0.000000in}}{%
\pgfpathmoveto{\pgfqpoint{0.000000in}{0.000000in}}%
\pgfpathlineto{\pgfqpoint{0.000000in}{-0.027778in}}%
\pgfusepath{stroke,fill}%
}%
\begin{pgfscope}%
\pgfsys@transformshift{2.986806in}{0.526079in}%
\pgfsys@useobject{currentmarker}{}%
\end{pgfscope}%
\end{pgfscope}%
\begin{pgfscope}%
\pgfsetbuttcap%
\pgfsetroundjoin%
\definecolor{currentfill}{rgb}{0.000000,0.000000,0.000000}%
\pgfsetfillcolor{currentfill}%
\pgfsetlinewidth{0.602250pt}%
\definecolor{currentstroke}{rgb}{0.000000,0.000000,0.000000}%
\pgfsetstrokecolor{currentstroke}%
\pgfsetdash{}{0pt}%
\pgfsys@defobject{currentmarker}{\pgfqpoint{0.000000in}{-0.027778in}}{\pgfqpoint{0.000000in}{0.000000in}}{%
\pgfpathmoveto{\pgfqpoint{0.000000in}{0.000000in}}%
\pgfpathlineto{\pgfqpoint{0.000000in}{-0.027778in}}%
\pgfusepath{stroke,fill}%
}%
\begin{pgfscope}%
\pgfsys@transformshift{3.189780in}{0.526079in}%
\pgfsys@useobject{currentmarker}{}%
\end{pgfscope}%
\end{pgfscope}%
\begin{pgfscope}%
\pgfsetbuttcap%
\pgfsetroundjoin%
\definecolor{currentfill}{rgb}{0.000000,0.000000,0.000000}%
\pgfsetfillcolor{currentfill}%
\pgfsetlinewidth{0.602250pt}%
\definecolor{currentstroke}{rgb}{0.000000,0.000000,0.000000}%
\pgfsetstrokecolor{currentstroke}%
\pgfsetdash{}{0pt}%
\pgfsys@defobject{currentmarker}{\pgfqpoint{0.000000in}{-0.027778in}}{\pgfqpoint{0.000000in}{0.000000in}}{%
\pgfpathmoveto{\pgfqpoint{0.000000in}{0.000000in}}%
\pgfpathlineto{\pgfqpoint{0.000000in}{-0.027778in}}%
\pgfusepath{stroke,fill}%
}%
\begin{pgfscope}%
\pgfsys@transformshift{3.347219in}{0.526079in}%
\pgfsys@useobject{currentmarker}{}%
\end{pgfscope}%
\end{pgfscope}%
\begin{pgfscope}%
\pgfsetbuttcap%
\pgfsetroundjoin%
\definecolor{currentfill}{rgb}{0.000000,0.000000,0.000000}%
\pgfsetfillcolor{currentfill}%
\pgfsetlinewidth{0.602250pt}%
\definecolor{currentstroke}{rgb}{0.000000,0.000000,0.000000}%
\pgfsetstrokecolor{currentstroke}%
\pgfsetdash{}{0pt}%
\pgfsys@defobject{currentmarker}{\pgfqpoint{0.000000in}{-0.027778in}}{\pgfqpoint{0.000000in}{0.000000in}}{%
\pgfpathmoveto{\pgfqpoint{0.000000in}{0.000000in}}%
\pgfpathlineto{\pgfqpoint{0.000000in}{-0.027778in}}%
\pgfusepath{stroke,fill}%
}%
\begin{pgfscope}%
\pgfsys@transformshift{3.475856in}{0.526079in}%
\pgfsys@useobject{currentmarker}{}%
\end{pgfscope}%
\end{pgfscope}%
\begin{pgfscope}%
\pgfsetbuttcap%
\pgfsetroundjoin%
\definecolor{currentfill}{rgb}{0.000000,0.000000,0.000000}%
\pgfsetfillcolor{currentfill}%
\pgfsetlinewidth{0.602250pt}%
\definecolor{currentstroke}{rgb}{0.000000,0.000000,0.000000}%
\pgfsetstrokecolor{currentstroke}%
\pgfsetdash{}{0pt}%
\pgfsys@defobject{currentmarker}{\pgfqpoint{0.000000in}{-0.027778in}}{\pgfqpoint{0.000000in}{0.000000in}}{%
\pgfpathmoveto{\pgfqpoint{0.000000in}{0.000000in}}%
\pgfpathlineto{\pgfqpoint{0.000000in}{-0.027778in}}%
\pgfusepath{stroke,fill}%
}%
\begin{pgfscope}%
\pgfsys@transformshift{3.584616in}{0.526079in}%
\pgfsys@useobject{currentmarker}{}%
\end{pgfscope}%
\end{pgfscope}%
\begin{pgfscope}%
\pgfsetbuttcap%
\pgfsetroundjoin%
\definecolor{currentfill}{rgb}{0.000000,0.000000,0.000000}%
\pgfsetfillcolor{currentfill}%
\pgfsetlinewidth{0.602250pt}%
\definecolor{currentstroke}{rgb}{0.000000,0.000000,0.000000}%
\pgfsetstrokecolor{currentstroke}%
\pgfsetdash{}{0pt}%
\pgfsys@defobject{currentmarker}{\pgfqpoint{0.000000in}{-0.027778in}}{\pgfqpoint{0.000000in}{0.000000in}}{%
\pgfpathmoveto{\pgfqpoint{0.000000in}{0.000000in}}%
\pgfpathlineto{\pgfqpoint{0.000000in}{-0.027778in}}%
\pgfusepath{stroke,fill}%
}%
\begin{pgfscope}%
\pgfsys@transformshift{3.678829in}{0.526079in}%
\pgfsys@useobject{currentmarker}{}%
\end{pgfscope}%
\end{pgfscope}%
\begin{pgfscope}%
\pgfsetbuttcap%
\pgfsetroundjoin%
\definecolor{currentfill}{rgb}{0.000000,0.000000,0.000000}%
\pgfsetfillcolor{currentfill}%
\pgfsetlinewidth{0.602250pt}%
\definecolor{currentstroke}{rgb}{0.000000,0.000000,0.000000}%
\pgfsetstrokecolor{currentstroke}%
\pgfsetdash{}{0pt}%
\pgfsys@defobject{currentmarker}{\pgfqpoint{0.000000in}{-0.027778in}}{\pgfqpoint{0.000000in}{0.000000in}}{%
\pgfpathmoveto{\pgfqpoint{0.000000in}{0.000000in}}%
\pgfpathlineto{\pgfqpoint{0.000000in}{-0.027778in}}%
\pgfusepath{stroke,fill}%
}%
\begin{pgfscope}%
\pgfsys@transformshift{3.761931in}{0.526079in}%
\pgfsys@useobject{currentmarker}{}%
\end{pgfscope}%
\end{pgfscope}%
\begin{pgfscope}%
\pgfsetbuttcap%
\pgfsetroundjoin%
\definecolor{currentfill}{rgb}{0.000000,0.000000,0.000000}%
\pgfsetfillcolor{currentfill}%
\pgfsetlinewidth{0.602250pt}%
\definecolor{currentstroke}{rgb}{0.000000,0.000000,0.000000}%
\pgfsetstrokecolor{currentstroke}%
\pgfsetdash{}{0pt}%
\pgfsys@defobject{currentmarker}{\pgfqpoint{0.000000in}{-0.027778in}}{\pgfqpoint{0.000000in}{0.000000in}}{%
\pgfpathmoveto{\pgfqpoint{0.000000in}{0.000000in}}%
\pgfpathlineto{\pgfqpoint{0.000000in}{-0.027778in}}%
\pgfusepath{stroke,fill}%
}%
\begin{pgfscope}%
\pgfsys@transformshift{4.325318in}{0.526079in}%
\pgfsys@useobject{currentmarker}{}%
\end{pgfscope}%
\end{pgfscope}%
\begin{pgfscope}%
\pgfsetbuttcap%
\pgfsetroundjoin%
\definecolor{currentfill}{rgb}{0.000000,0.000000,0.000000}%
\pgfsetfillcolor{currentfill}%
\pgfsetlinewidth{0.602250pt}%
\definecolor{currentstroke}{rgb}{0.000000,0.000000,0.000000}%
\pgfsetstrokecolor{currentstroke}%
\pgfsetdash{}{0pt}%
\pgfsys@defobject{currentmarker}{\pgfqpoint{0.000000in}{-0.027778in}}{\pgfqpoint{0.000000in}{0.000000in}}{%
\pgfpathmoveto{\pgfqpoint{0.000000in}{0.000000in}}%
\pgfpathlineto{\pgfqpoint{0.000000in}{-0.027778in}}%
\pgfusepath{stroke,fill}%
}%
\begin{pgfscope}%
\pgfsys@transformshift{4.611393in}{0.526079in}%
\pgfsys@useobject{currentmarker}{}%
\end{pgfscope}%
\end{pgfscope}%
\begin{pgfscope}%
\pgfsetbuttcap%
\pgfsetroundjoin%
\definecolor{currentfill}{rgb}{0.000000,0.000000,0.000000}%
\pgfsetfillcolor{currentfill}%
\pgfsetlinewidth{0.602250pt}%
\definecolor{currentstroke}{rgb}{0.000000,0.000000,0.000000}%
\pgfsetstrokecolor{currentstroke}%
\pgfsetdash{}{0pt}%
\pgfsys@defobject{currentmarker}{\pgfqpoint{0.000000in}{-0.027778in}}{\pgfqpoint{0.000000in}{0.000000in}}{%
\pgfpathmoveto{\pgfqpoint{0.000000in}{0.000000in}}%
\pgfpathlineto{\pgfqpoint{0.000000in}{-0.027778in}}%
\pgfusepath{stroke,fill}%
}%
\begin{pgfscope}%
\pgfsys@transformshift{4.814367in}{0.526079in}%
\pgfsys@useobject{currentmarker}{}%
\end{pgfscope}%
\end{pgfscope}%
\begin{pgfscope}%
\pgfsetbuttcap%
\pgfsetroundjoin%
\definecolor{currentfill}{rgb}{0.000000,0.000000,0.000000}%
\pgfsetfillcolor{currentfill}%
\pgfsetlinewidth{0.602250pt}%
\definecolor{currentstroke}{rgb}{0.000000,0.000000,0.000000}%
\pgfsetstrokecolor{currentstroke}%
\pgfsetdash{}{0pt}%
\pgfsys@defobject{currentmarker}{\pgfqpoint{0.000000in}{-0.027778in}}{\pgfqpoint{0.000000in}{0.000000in}}{%
\pgfpathmoveto{\pgfqpoint{0.000000in}{0.000000in}}%
\pgfpathlineto{\pgfqpoint{0.000000in}{-0.027778in}}%
\pgfusepath{stroke,fill}%
}%
\begin{pgfscope}%
\pgfsys@transformshift{4.971806in}{0.526079in}%
\pgfsys@useobject{currentmarker}{}%
\end{pgfscope}%
\end{pgfscope}%
\begin{pgfscope}%
\pgfsetbuttcap%
\pgfsetroundjoin%
\definecolor{currentfill}{rgb}{0.000000,0.000000,0.000000}%
\pgfsetfillcolor{currentfill}%
\pgfsetlinewidth{0.602250pt}%
\definecolor{currentstroke}{rgb}{0.000000,0.000000,0.000000}%
\pgfsetstrokecolor{currentstroke}%
\pgfsetdash{}{0pt}%
\pgfsys@defobject{currentmarker}{\pgfqpoint{0.000000in}{-0.027778in}}{\pgfqpoint{0.000000in}{0.000000in}}{%
\pgfpathmoveto{\pgfqpoint{0.000000in}{0.000000in}}%
\pgfpathlineto{\pgfqpoint{0.000000in}{-0.027778in}}%
\pgfusepath{stroke,fill}%
}%
\begin{pgfscope}%
\pgfsys@transformshift{5.100443in}{0.526079in}%
\pgfsys@useobject{currentmarker}{}%
\end{pgfscope}%
\end{pgfscope}%
\begin{pgfscope}%
\pgfsetbuttcap%
\pgfsetroundjoin%
\definecolor{currentfill}{rgb}{0.000000,0.000000,0.000000}%
\pgfsetfillcolor{currentfill}%
\pgfsetlinewidth{0.602250pt}%
\definecolor{currentstroke}{rgb}{0.000000,0.000000,0.000000}%
\pgfsetstrokecolor{currentstroke}%
\pgfsetdash{}{0pt}%
\pgfsys@defobject{currentmarker}{\pgfqpoint{0.000000in}{-0.027778in}}{\pgfqpoint{0.000000in}{0.000000in}}{%
\pgfpathmoveto{\pgfqpoint{0.000000in}{0.000000in}}%
\pgfpathlineto{\pgfqpoint{0.000000in}{-0.027778in}}%
\pgfusepath{stroke,fill}%
}%
\begin{pgfscope}%
\pgfsys@transformshift{5.209203in}{0.526079in}%
\pgfsys@useobject{currentmarker}{}%
\end{pgfscope}%
\end{pgfscope}%
\begin{pgfscope}%
\pgfsetbuttcap%
\pgfsetroundjoin%
\definecolor{currentfill}{rgb}{0.000000,0.000000,0.000000}%
\pgfsetfillcolor{currentfill}%
\pgfsetlinewidth{0.602250pt}%
\definecolor{currentstroke}{rgb}{0.000000,0.000000,0.000000}%
\pgfsetstrokecolor{currentstroke}%
\pgfsetdash{}{0pt}%
\pgfsys@defobject{currentmarker}{\pgfqpoint{0.000000in}{-0.027778in}}{\pgfqpoint{0.000000in}{0.000000in}}{%
\pgfpathmoveto{\pgfqpoint{0.000000in}{0.000000in}}%
\pgfpathlineto{\pgfqpoint{0.000000in}{-0.027778in}}%
\pgfusepath{stroke,fill}%
}%
\begin{pgfscope}%
\pgfsys@transformshift{5.303416in}{0.526079in}%
\pgfsys@useobject{currentmarker}{}%
\end{pgfscope}%
\end{pgfscope}%
\begin{pgfscope}%
\pgftext[x=3.004669in,y=0.238889in,,top]{\rmfamily\fontsize{10.000000}{12.000000}\selectfont \(\displaystyle y/L\)}%
\end{pgfscope}%
\begin{pgfscope}%
\pgfsetbuttcap%
\pgfsetroundjoin%
\definecolor{currentfill}{rgb}{0.000000,0.000000,0.000000}%
\pgfsetfillcolor{currentfill}%
\pgfsetlinewidth{0.803000pt}%
\definecolor{currentstroke}{rgb}{0.000000,0.000000,0.000000}%
\pgfsetstrokecolor{currentstroke}%
\pgfsetdash{}{0pt}%
\pgfsys@defobject{currentmarker}{\pgfqpoint{-0.048611in}{0.000000in}}{\pgfqpoint{0.000000in}{0.000000in}}{%
\pgfpathmoveto{\pgfqpoint{0.000000in}{0.000000in}}%
\pgfpathlineto{\pgfqpoint{-0.048611in}{0.000000in}}%
\pgfusepath{stroke,fill}%
}%
\begin{pgfscope}%
\pgfsys@transformshift{0.679669in}{1.104918in}%
\pgfsys@useobject{currentmarker}{}%
\end{pgfscope}%
\end{pgfscope}%
\begin{pgfscope}%
\pgftext[x=0.294444in,y=1.052156in,left,base]{\rmfamily\fontsize{10.000000}{12.000000}\selectfont \(\displaystyle 10^{-2}\)}%
\end{pgfscope}%
\begin{pgfscope}%
\pgfsetbuttcap%
\pgfsetroundjoin%
\definecolor{currentfill}{rgb}{0.000000,0.000000,0.000000}%
\pgfsetfillcolor{currentfill}%
\pgfsetlinewidth{0.803000pt}%
\definecolor{currentstroke}{rgb}{0.000000,0.000000,0.000000}%
\pgfsetstrokecolor{currentstroke}%
\pgfsetdash{}{0pt}%
\pgfsys@defobject{currentmarker}{\pgfqpoint{-0.048611in}{0.000000in}}{\pgfqpoint{0.000000in}{0.000000in}}{%
\pgfpathmoveto{\pgfqpoint{0.000000in}{0.000000in}}%
\pgfpathlineto{\pgfqpoint{-0.048611in}{0.000000in}}%
\pgfusepath{stroke,fill}%
}%
\begin{pgfscope}%
\pgfsys@transformshift{0.679669in}{2.256869in}%
\pgfsys@useobject{currentmarker}{}%
\end{pgfscope}%
\end{pgfscope}%
\begin{pgfscope}%
\pgftext[x=0.294444in,y=2.204108in,left,base]{\rmfamily\fontsize{10.000000}{12.000000}\selectfont \(\displaystyle 10^{-1}\)}%
\end{pgfscope}%
\begin{pgfscope}%
\pgfsetbuttcap%
\pgfsetroundjoin%
\definecolor{currentfill}{rgb}{0.000000,0.000000,0.000000}%
\pgfsetfillcolor{currentfill}%
\pgfsetlinewidth{0.803000pt}%
\definecolor{currentstroke}{rgb}{0.000000,0.000000,0.000000}%
\pgfsetstrokecolor{currentstroke}%
\pgfsetdash{}{0pt}%
\pgfsys@defobject{currentmarker}{\pgfqpoint{-0.048611in}{0.000000in}}{\pgfqpoint{0.000000in}{0.000000in}}{%
\pgfpathmoveto{\pgfqpoint{0.000000in}{0.000000in}}%
\pgfpathlineto{\pgfqpoint{-0.048611in}{0.000000in}}%
\pgfusepath{stroke,fill}%
}%
\begin{pgfscope}%
\pgfsys@transformshift{0.679669in}{3.408821in}%
\pgfsys@useobject{currentmarker}{}%
\end{pgfscope}%
\end{pgfscope}%
\begin{pgfscope}%
\pgftext[x=0.381250in,y=3.356059in,left,base]{\rmfamily\fontsize{10.000000}{12.000000}\selectfont \(\displaystyle 10^{0}\)}%
\end{pgfscope}%
\begin{pgfscope}%
\pgfsetbuttcap%
\pgfsetroundjoin%
\definecolor{currentfill}{rgb}{0.000000,0.000000,0.000000}%
\pgfsetfillcolor{currentfill}%
\pgfsetlinewidth{0.602250pt}%
\definecolor{currentstroke}{rgb}{0.000000,0.000000,0.000000}%
\pgfsetstrokecolor{currentstroke}%
\pgfsetdash{}{0pt}%
\pgfsys@defobject{currentmarker}{\pgfqpoint{-0.027778in}{0.000000in}}{\pgfqpoint{0.000000in}{0.000000in}}{%
\pgfpathmoveto{\pgfqpoint{0.000000in}{0.000000in}}%
\pgfpathlineto{\pgfqpoint{-0.027778in}{0.000000in}}%
\pgfusepath{stroke,fill}%
}%
\begin{pgfscope}%
\pgfsys@transformshift{0.679669in}{0.646510in}%
\pgfsys@useobject{currentmarker}{}%
\end{pgfscope}%
\end{pgfscope}%
\begin{pgfscope}%
\pgfsetbuttcap%
\pgfsetroundjoin%
\definecolor{currentfill}{rgb}{0.000000,0.000000,0.000000}%
\pgfsetfillcolor{currentfill}%
\pgfsetlinewidth{0.602250pt}%
\definecolor{currentstroke}{rgb}{0.000000,0.000000,0.000000}%
\pgfsetstrokecolor{currentstroke}%
\pgfsetdash{}{0pt}%
\pgfsys@defobject{currentmarker}{\pgfqpoint{-0.027778in}{0.000000in}}{\pgfqpoint{0.000000in}{0.000000in}}{%
\pgfpathmoveto{\pgfqpoint{0.000000in}{0.000000in}}%
\pgfpathlineto{\pgfqpoint{-0.027778in}{0.000000in}}%
\pgfusepath{stroke,fill}%
}%
\begin{pgfscope}%
\pgfsys@transformshift{0.679669in}{0.758146in}%
\pgfsys@useobject{currentmarker}{}%
\end{pgfscope}%
\end{pgfscope}%
\begin{pgfscope}%
\pgfsetbuttcap%
\pgfsetroundjoin%
\definecolor{currentfill}{rgb}{0.000000,0.000000,0.000000}%
\pgfsetfillcolor{currentfill}%
\pgfsetlinewidth{0.602250pt}%
\definecolor{currentstroke}{rgb}{0.000000,0.000000,0.000000}%
\pgfsetstrokecolor{currentstroke}%
\pgfsetdash{}{0pt}%
\pgfsys@defobject{currentmarker}{\pgfqpoint{-0.027778in}{0.000000in}}{\pgfqpoint{0.000000in}{0.000000in}}{%
\pgfpathmoveto{\pgfqpoint{0.000000in}{0.000000in}}%
\pgfpathlineto{\pgfqpoint{-0.027778in}{0.000000in}}%
\pgfusepath{stroke,fill}%
}%
\begin{pgfscope}%
\pgfsys@transformshift{0.679669in}{0.849359in}%
\pgfsys@useobject{currentmarker}{}%
\end{pgfscope}%
\end{pgfscope}%
\begin{pgfscope}%
\pgfsetbuttcap%
\pgfsetroundjoin%
\definecolor{currentfill}{rgb}{0.000000,0.000000,0.000000}%
\pgfsetfillcolor{currentfill}%
\pgfsetlinewidth{0.602250pt}%
\definecolor{currentstroke}{rgb}{0.000000,0.000000,0.000000}%
\pgfsetstrokecolor{currentstroke}%
\pgfsetdash{}{0pt}%
\pgfsys@defobject{currentmarker}{\pgfqpoint{-0.027778in}{0.000000in}}{\pgfqpoint{0.000000in}{0.000000in}}{%
\pgfpathmoveto{\pgfqpoint{0.000000in}{0.000000in}}%
\pgfpathlineto{\pgfqpoint{-0.027778in}{0.000000in}}%
\pgfusepath{stroke,fill}%
}%
\begin{pgfscope}%
\pgfsys@transformshift{0.679669in}{0.926478in}%
\pgfsys@useobject{currentmarker}{}%
\end{pgfscope}%
\end{pgfscope}%
\begin{pgfscope}%
\pgfsetbuttcap%
\pgfsetroundjoin%
\definecolor{currentfill}{rgb}{0.000000,0.000000,0.000000}%
\pgfsetfillcolor{currentfill}%
\pgfsetlinewidth{0.602250pt}%
\definecolor{currentstroke}{rgb}{0.000000,0.000000,0.000000}%
\pgfsetstrokecolor{currentstroke}%
\pgfsetdash{}{0pt}%
\pgfsys@defobject{currentmarker}{\pgfqpoint{-0.027778in}{0.000000in}}{\pgfqpoint{0.000000in}{0.000000in}}{%
\pgfpathmoveto{\pgfqpoint{0.000000in}{0.000000in}}%
\pgfpathlineto{\pgfqpoint{-0.027778in}{0.000000in}}%
\pgfusepath{stroke,fill}%
}%
\begin{pgfscope}%
\pgfsys@transformshift{0.679669in}{0.993282in}%
\pgfsys@useobject{currentmarker}{}%
\end{pgfscope}%
\end{pgfscope}%
\begin{pgfscope}%
\pgfsetbuttcap%
\pgfsetroundjoin%
\definecolor{currentfill}{rgb}{0.000000,0.000000,0.000000}%
\pgfsetfillcolor{currentfill}%
\pgfsetlinewidth{0.602250pt}%
\definecolor{currentstroke}{rgb}{0.000000,0.000000,0.000000}%
\pgfsetstrokecolor{currentstroke}%
\pgfsetdash{}{0pt}%
\pgfsys@defobject{currentmarker}{\pgfqpoint{-0.027778in}{0.000000in}}{\pgfqpoint{0.000000in}{0.000000in}}{%
\pgfpathmoveto{\pgfqpoint{0.000000in}{0.000000in}}%
\pgfpathlineto{\pgfqpoint{-0.027778in}{0.000000in}}%
\pgfusepath{stroke,fill}%
}%
\begin{pgfscope}%
\pgfsys@transformshift{0.679669in}{1.052207in}%
\pgfsys@useobject{currentmarker}{}%
\end{pgfscope}%
\end{pgfscope}%
\begin{pgfscope}%
\pgfsetbuttcap%
\pgfsetroundjoin%
\definecolor{currentfill}{rgb}{0.000000,0.000000,0.000000}%
\pgfsetfillcolor{currentfill}%
\pgfsetlinewidth{0.602250pt}%
\definecolor{currentstroke}{rgb}{0.000000,0.000000,0.000000}%
\pgfsetstrokecolor{currentstroke}%
\pgfsetdash{}{0pt}%
\pgfsys@defobject{currentmarker}{\pgfqpoint{-0.027778in}{0.000000in}}{\pgfqpoint{0.000000in}{0.000000in}}{%
\pgfpathmoveto{\pgfqpoint{0.000000in}{0.000000in}}%
\pgfpathlineto{\pgfqpoint{-0.027778in}{0.000000in}}%
\pgfusepath{stroke,fill}%
}%
\begin{pgfscope}%
\pgfsys@transformshift{0.679669in}{1.451690in}%
\pgfsys@useobject{currentmarker}{}%
\end{pgfscope}%
\end{pgfscope}%
\begin{pgfscope}%
\pgfsetbuttcap%
\pgfsetroundjoin%
\definecolor{currentfill}{rgb}{0.000000,0.000000,0.000000}%
\pgfsetfillcolor{currentfill}%
\pgfsetlinewidth{0.602250pt}%
\definecolor{currentstroke}{rgb}{0.000000,0.000000,0.000000}%
\pgfsetstrokecolor{currentstroke}%
\pgfsetdash{}{0pt}%
\pgfsys@defobject{currentmarker}{\pgfqpoint{-0.027778in}{0.000000in}}{\pgfqpoint{0.000000in}{0.000000in}}{%
\pgfpathmoveto{\pgfqpoint{0.000000in}{0.000000in}}%
\pgfpathlineto{\pgfqpoint{-0.027778in}{0.000000in}}%
\pgfusepath{stroke,fill}%
}%
\begin{pgfscope}%
\pgfsys@transformshift{0.679669in}{1.654538in}%
\pgfsys@useobject{currentmarker}{}%
\end{pgfscope}%
\end{pgfscope}%
\begin{pgfscope}%
\pgfsetbuttcap%
\pgfsetroundjoin%
\definecolor{currentfill}{rgb}{0.000000,0.000000,0.000000}%
\pgfsetfillcolor{currentfill}%
\pgfsetlinewidth{0.602250pt}%
\definecolor{currentstroke}{rgb}{0.000000,0.000000,0.000000}%
\pgfsetstrokecolor{currentstroke}%
\pgfsetdash{}{0pt}%
\pgfsys@defobject{currentmarker}{\pgfqpoint{-0.027778in}{0.000000in}}{\pgfqpoint{0.000000in}{0.000000in}}{%
\pgfpathmoveto{\pgfqpoint{0.000000in}{0.000000in}}%
\pgfpathlineto{\pgfqpoint{-0.027778in}{0.000000in}}%
\pgfusepath{stroke,fill}%
}%
\begin{pgfscope}%
\pgfsys@transformshift{0.679669in}{1.798462in}%
\pgfsys@useobject{currentmarker}{}%
\end{pgfscope}%
\end{pgfscope}%
\begin{pgfscope}%
\pgfsetbuttcap%
\pgfsetroundjoin%
\definecolor{currentfill}{rgb}{0.000000,0.000000,0.000000}%
\pgfsetfillcolor{currentfill}%
\pgfsetlinewidth{0.602250pt}%
\definecolor{currentstroke}{rgb}{0.000000,0.000000,0.000000}%
\pgfsetstrokecolor{currentstroke}%
\pgfsetdash{}{0pt}%
\pgfsys@defobject{currentmarker}{\pgfqpoint{-0.027778in}{0.000000in}}{\pgfqpoint{0.000000in}{0.000000in}}{%
\pgfpathmoveto{\pgfqpoint{0.000000in}{0.000000in}}%
\pgfpathlineto{\pgfqpoint{-0.027778in}{0.000000in}}%
\pgfusepath{stroke,fill}%
}%
\begin{pgfscope}%
\pgfsys@transformshift{0.679669in}{1.910097in}%
\pgfsys@useobject{currentmarker}{}%
\end{pgfscope}%
\end{pgfscope}%
\begin{pgfscope}%
\pgfsetbuttcap%
\pgfsetroundjoin%
\definecolor{currentfill}{rgb}{0.000000,0.000000,0.000000}%
\pgfsetfillcolor{currentfill}%
\pgfsetlinewidth{0.602250pt}%
\definecolor{currentstroke}{rgb}{0.000000,0.000000,0.000000}%
\pgfsetstrokecolor{currentstroke}%
\pgfsetdash{}{0pt}%
\pgfsys@defobject{currentmarker}{\pgfqpoint{-0.027778in}{0.000000in}}{\pgfqpoint{0.000000in}{0.000000in}}{%
\pgfpathmoveto{\pgfqpoint{0.000000in}{0.000000in}}%
\pgfpathlineto{\pgfqpoint{-0.027778in}{0.000000in}}%
\pgfusepath{stroke,fill}%
}%
\begin{pgfscope}%
\pgfsys@transformshift{0.679669in}{2.001310in}%
\pgfsys@useobject{currentmarker}{}%
\end{pgfscope}%
\end{pgfscope}%
\begin{pgfscope}%
\pgfsetbuttcap%
\pgfsetroundjoin%
\definecolor{currentfill}{rgb}{0.000000,0.000000,0.000000}%
\pgfsetfillcolor{currentfill}%
\pgfsetlinewidth{0.602250pt}%
\definecolor{currentstroke}{rgb}{0.000000,0.000000,0.000000}%
\pgfsetstrokecolor{currentstroke}%
\pgfsetdash{}{0pt}%
\pgfsys@defobject{currentmarker}{\pgfqpoint{-0.027778in}{0.000000in}}{\pgfqpoint{0.000000in}{0.000000in}}{%
\pgfpathmoveto{\pgfqpoint{0.000000in}{0.000000in}}%
\pgfpathlineto{\pgfqpoint{-0.027778in}{0.000000in}}%
\pgfusepath{stroke,fill}%
}%
\begin{pgfscope}%
\pgfsys@transformshift{0.679669in}{2.078430in}%
\pgfsys@useobject{currentmarker}{}%
\end{pgfscope}%
\end{pgfscope}%
\begin{pgfscope}%
\pgfsetbuttcap%
\pgfsetroundjoin%
\definecolor{currentfill}{rgb}{0.000000,0.000000,0.000000}%
\pgfsetfillcolor{currentfill}%
\pgfsetlinewidth{0.602250pt}%
\definecolor{currentstroke}{rgb}{0.000000,0.000000,0.000000}%
\pgfsetstrokecolor{currentstroke}%
\pgfsetdash{}{0pt}%
\pgfsys@defobject{currentmarker}{\pgfqpoint{-0.027778in}{0.000000in}}{\pgfqpoint{0.000000in}{0.000000in}}{%
\pgfpathmoveto{\pgfqpoint{0.000000in}{0.000000in}}%
\pgfpathlineto{\pgfqpoint{-0.027778in}{0.000000in}}%
\pgfusepath{stroke,fill}%
}%
\begin{pgfscope}%
\pgfsys@transformshift{0.679669in}{2.145234in}%
\pgfsys@useobject{currentmarker}{}%
\end{pgfscope}%
\end{pgfscope}%
\begin{pgfscope}%
\pgfsetbuttcap%
\pgfsetroundjoin%
\definecolor{currentfill}{rgb}{0.000000,0.000000,0.000000}%
\pgfsetfillcolor{currentfill}%
\pgfsetlinewidth{0.602250pt}%
\definecolor{currentstroke}{rgb}{0.000000,0.000000,0.000000}%
\pgfsetstrokecolor{currentstroke}%
\pgfsetdash{}{0pt}%
\pgfsys@defobject{currentmarker}{\pgfqpoint{-0.027778in}{0.000000in}}{\pgfqpoint{0.000000in}{0.000000in}}{%
\pgfpathmoveto{\pgfqpoint{0.000000in}{0.000000in}}%
\pgfpathlineto{\pgfqpoint{-0.027778in}{0.000000in}}%
\pgfusepath{stroke,fill}%
}%
\begin{pgfscope}%
\pgfsys@transformshift{0.679669in}{2.204159in}%
\pgfsys@useobject{currentmarker}{}%
\end{pgfscope}%
\end{pgfscope}%
\begin{pgfscope}%
\pgfsetbuttcap%
\pgfsetroundjoin%
\definecolor{currentfill}{rgb}{0.000000,0.000000,0.000000}%
\pgfsetfillcolor{currentfill}%
\pgfsetlinewidth{0.602250pt}%
\definecolor{currentstroke}{rgb}{0.000000,0.000000,0.000000}%
\pgfsetstrokecolor{currentstroke}%
\pgfsetdash{}{0pt}%
\pgfsys@defobject{currentmarker}{\pgfqpoint{-0.027778in}{0.000000in}}{\pgfqpoint{0.000000in}{0.000000in}}{%
\pgfpathmoveto{\pgfqpoint{0.000000in}{0.000000in}}%
\pgfpathlineto{\pgfqpoint{-0.027778in}{0.000000in}}%
\pgfusepath{stroke,fill}%
}%
\begin{pgfscope}%
\pgfsys@transformshift{0.679669in}{2.603641in}%
\pgfsys@useobject{currentmarker}{}%
\end{pgfscope}%
\end{pgfscope}%
\begin{pgfscope}%
\pgfsetbuttcap%
\pgfsetroundjoin%
\definecolor{currentfill}{rgb}{0.000000,0.000000,0.000000}%
\pgfsetfillcolor{currentfill}%
\pgfsetlinewidth{0.602250pt}%
\definecolor{currentstroke}{rgb}{0.000000,0.000000,0.000000}%
\pgfsetstrokecolor{currentstroke}%
\pgfsetdash{}{0pt}%
\pgfsys@defobject{currentmarker}{\pgfqpoint{-0.027778in}{0.000000in}}{\pgfqpoint{0.000000in}{0.000000in}}{%
\pgfpathmoveto{\pgfqpoint{0.000000in}{0.000000in}}%
\pgfpathlineto{\pgfqpoint{-0.027778in}{0.000000in}}%
\pgfusepath{stroke,fill}%
}%
\begin{pgfscope}%
\pgfsys@transformshift{0.679669in}{2.806490in}%
\pgfsys@useobject{currentmarker}{}%
\end{pgfscope}%
\end{pgfscope}%
\begin{pgfscope}%
\pgfsetbuttcap%
\pgfsetroundjoin%
\definecolor{currentfill}{rgb}{0.000000,0.000000,0.000000}%
\pgfsetfillcolor{currentfill}%
\pgfsetlinewidth{0.602250pt}%
\definecolor{currentstroke}{rgb}{0.000000,0.000000,0.000000}%
\pgfsetstrokecolor{currentstroke}%
\pgfsetdash{}{0pt}%
\pgfsys@defobject{currentmarker}{\pgfqpoint{-0.027778in}{0.000000in}}{\pgfqpoint{0.000000in}{0.000000in}}{%
\pgfpathmoveto{\pgfqpoint{0.000000in}{0.000000in}}%
\pgfpathlineto{\pgfqpoint{-0.027778in}{0.000000in}}%
\pgfusepath{stroke,fill}%
}%
\begin{pgfscope}%
\pgfsys@transformshift{0.679669in}{2.950413in}%
\pgfsys@useobject{currentmarker}{}%
\end{pgfscope}%
\end{pgfscope}%
\begin{pgfscope}%
\pgfsetbuttcap%
\pgfsetroundjoin%
\definecolor{currentfill}{rgb}{0.000000,0.000000,0.000000}%
\pgfsetfillcolor{currentfill}%
\pgfsetlinewidth{0.602250pt}%
\definecolor{currentstroke}{rgb}{0.000000,0.000000,0.000000}%
\pgfsetstrokecolor{currentstroke}%
\pgfsetdash{}{0pt}%
\pgfsys@defobject{currentmarker}{\pgfqpoint{-0.027778in}{0.000000in}}{\pgfqpoint{0.000000in}{0.000000in}}{%
\pgfpathmoveto{\pgfqpoint{0.000000in}{0.000000in}}%
\pgfpathlineto{\pgfqpoint{-0.027778in}{0.000000in}}%
\pgfusepath{stroke,fill}%
}%
\begin{pgfscope}%
\pgfsys@transformshift{0.679669in}{3.062049in}%
\pgfsys@useobject{currentmarker}{}%
\end{pgfscope}%
\end{pgfscope}%
\begin{pgfscope}%
\pgfsetbuttcap%
\pgfsetroundjoin%
\definecolor{currentfill}{rgb}{0.000000,0.000000,0.000000}%
\pgfsetfillcolor{currentfill}%
\pgfsetlinewidth{0.602250pt}%
\definecolor{currentstroke}{rgb}{0.000000,0.000000,0.000000}%
\pgfsetstrokecolor{currentstroke}%
\pgfsetdash{}{0pt}%
\pgfsys@defobject{currentmarker}{\pgfqpoint{-0.027778in}{0.000000in}}{\pgfqpoint{0.000000in}{0.000000in}}{%
\pgfpathmoveto{\pgfqpoint{0.000000in}{0.000000in}}%
\pgfpathlineto{\pgfqpoint{-0.027778in}{0.000000in}}%
\pgfusepath{stroke,fill}%
}%
\begin{pgfscope}%
\pgfsys@transformshift{0.679669in}{3.153262in}%
\pgfsys@useobject{currentmarker}{}%
\end{pgfscope}%
\end{pgfscope}%
\begin{pgfscope}%
\pgfsetbuttcap%
\pgfsetroundjoin%
\definecolor{currentfill}{rgb}{0.000000,0.000000,0.000000}%
\pgfsetfillcolor{currentfill}%
\pgfsetlinewidth{0.602250pt}%
\definecolor{currentstroke}{rgb}{0.000000,0.000000,0.000000}%
\pgfsetstrokecolor{currentstroke}%
\pgfsetdash{}{0pt}%
\pgfsys@defobject{currentmarker}{\pgfqpoint{-0.027778in}{0.000000in}}{\pgfqpoint{0.000000in}{0.000000in}}{%
\pgfpathmoveto{\pgfqpoint{0.000000in}{0.000000in}}%
\pgfpathlineto{\pgfqpoint{-0.027778in}{0.000000in}}%
\pgfusepath{stroke,fill}%
}%
\begin{pgfscope}%
\pgfsys@transformshift{0.679669in}{3.230381in}%
\pgfsys@useobject{currentmarker}{}%
\end{pgfscope}%
\end{pgfscope}%
\begin{pgfscope}%
\pgfsetbuttcap%
\pgfsetroundjoin%
\definecolor{currentfill}{rgb}{0.000000,0.000000,0.000000}%
\pgfsetfillcolor{currentfill}%
\pgfsetlinewidth{0.602250pt}%
\definecolor{currentstroke}{rgb}{0.000000,0.000000,0.000000}%
\pgfsetstrokecolor{currentstroke}%
\pgfsetdash{}{0pt}%
\pgfsys@defobject{currentmarker}{\pgfqpoint{-0.027778in}{0.000000in}}{\pgfqpoint{0.000000in}{0.000000in}}{%
\pgfpathmoveto{\pgfqpoint{0.000000in}{0.000000in}}%
\pgfpathlineto{\pgfqpoint{-0.027778in}{0.000000in}}%
\pgfusepath{stroke,fill}%
}%
\begin{pgfscope}%
\pgfsys@transformshift{0.679669in}{3.297185in}%
\pgfsys@useobject{currentmarker}{}%
\end{pgfscope}%
\end{pgfscope}%
\begin{pgfscope}%
\pgfsetbuttcap%
\pgfsetroundjoin%
\definecolor{currentfill}{rgb}{0.000000,0.000000,0.000000}%
\pgfsetfillcolor{currentfill}%
\pgfsetlinewidth{0.602250pt}%
\definecolor{currentstroke}{rgb}{0.000000,0.000000,0.000000}%
\pgfsetstrokecolor{currentstroke}%
\pgfsetdash{}{0pt}%
\pgfsys@defobject{currentmarker}{\pgfqpoint{-0.027778in}{0.000000in}}{\pgfqpoint{0.000000in}{0.000000in}}{%
\pgfpathmoveto{\pgfqpoint{0.000000in}{0.000000in}}%
\pgfpathlineto{\pgfqpoint{-0.027778in}{0.000000in}}%
\pgfusepath{stroke,fill}%
}%
\begin{pgfscope}%
\pgfsys@transformshift{0.679669in}{3.356110in}%
\pgfsys@useobject{currentmarker}{}%
\end{pgfscope}%
\end{pgfscope}%
\begin{pgfscope}%
\pgftext[x=0.238889in,y=2.036079in,,bottom,rotate=90.000000]{\rmfamily\fontsize{10.000000}{12.000000}\selectfont \(\displaystyle E/E_0\)}%
\end{pgfscope}%
\begin{pgfscope}%
\pgfpathrectangle{\pgfqpoint{0.679669in}{0.526079in}}{\pgfqpoint{4.650000in}{3.020000in}} %
\pgfusepath{clip}%
\pgfsetrectcap%
\pgfsetroundjoin%
\pgfsetlinewidth{1.505625pt}%
\definecolor{currentstroke}{rgb}{0.000000,0.000000,0.000000}%
\pgfsetstrokecolor{currentstroke}%
\pgfsetdash{}{0pt}%
\pgfpathmoveto{\pgfqpoint{0.891033in}{3.408807in}}%
\pgfpathlineto{\pgfqpoint{1.147789in}{3.402642in}}%
\pgfpathlineto{\pgfqpoint{1.364998in}{3.395262in}}%
\pgfpathlineto{\pgfqpoint{1.553221in}{3.386620in}}%
\pgfpathlineto{\pgfqpoint{1.710528in}{3.377257in}}%
\pgfpathlineto{\pgfqpoint{1.853615in}{3.366548in}}%
\pgfpathlineto{\pgfqpoint{1.978573in}{3.355050in}}%
\pgfpathlineto{\pgfqpoint{2.094989in}{3.342118in}}%
\pgfpathlineto{\pgfqpoint{2.199312in}{3.328331in}}%
\pgfpathlineto{\pgfqpoint{2.297886in}{3.313026in}}%
\pgfpathlineto{\pgfqpoint{2.387745in}{3.296809in}}%
\pgfpathlineto{\pgfqpoint{2.470449in}{3.279667in}}%
\pgfpathlineto{\pgfqpoint{2.549852in}{3.260922in}}%
\pgfpathlineto{\pgfqpoint{2.623636in}{3.241226in}}%
\pgfpathlineto{\pgfqpoint{2.694811in}{3.219895in}}%
\pgfpathlineto{\pgfqpoint{2.761449in}{3.197620in}}%
\pgfpathlineto{\pgfqpoint{2.825970in}{3.173718in}}%
\pgfpathlineto{\pgfqpoint{2.888408in}{3.148203in}}%
\pgfpathlineto{\pgfqpoint{2.948822in}{3.121106in}}%
\pgfpathlineto{\pgfqpoint{3.007279in}{3.092471in}}%
\pgfpathlineto{\pgfqpoint{3.065147in}{3.061641in}}%
\pgfpathlineto{\pgfqpoint{3.122212in}{3.028694in}}%
\pgfpathlineto{\pgfqpoint{3.178313in}{2.993735in}}%
\pgfpathlineto{\pgfqpoint{3.234359in}{2.956183in}}%
\pgfpathlineto{\pgfqpoint{3.290046in}{2.916209in}}%
\pgfpathlineto{\pgfqpoint{3.346877in}{2.872639in}}%
\pgfpathlineto{\pgfqpoint{3.404280in}{2.825776in}}%
\pgfpathlineto{\pgfqpoint{3.462530in}{2.775309in}}%
\pgfpathlineto{\pgfqpoint{3.521768in}{2.721033in}}%
\pgfpathlineto{\pgfqpoint{3.583268in}{2.661631in}}%
\pgfpathlineto{\pgfqpoint{3.646668in}{2.597281in}}%
\pgfpathlineto{\pgfqpoint{3.713138in}{2.526615in}}%
\pgfpathlineto{\pgfqpoint{3.783282in}{2.448754in}}%
\pgfpathlineto{\pgfqpoint{3.858083in}{2.362341in}}%
\pgfpathlineto{\pgfqpoint{3.938564in}{2.265896in}}%
\pgfpathlineto{\pgfqpoint{4.026186in}{2.157350in}}%
\pgfpathlineto{\pgfqpoint{4.123122in}{2.033645in}}%
\pgfpathlineto{\pgfqpoint{4.232457in}{1.890416in}}%
\pgfpathlineto{\pgfqpoint{4.358782in}{1.721155in}}%
\pgfpathlineto{\pgfqpoint{4.509577in}{1.515251in}}%
\pgfpathlineto{\pgfqpoint{4.698525in}{1.253313in}}%
\pgfpathlineto{\pgfqpoint{4.953954in}{0.895157in}}%
\pgfpathlineto{\pgfqpoint{5.118306in}{0.663352in}}%
\pgfpathlineto{\pgfqpoint{5.118306in}{0.663352in}}%
\pgfusepath{stroke}%
\end{pgfscope}%
\begin{pgfscope}%
\pgfsetrectcap%
\pgfsetmiterjoin%
\pgfsetlinewidth{0.803000pt}%
\definecolor{currentstroke}{rgb}{0.000000,0.000000,0.000000}%
\pgfsetstrokecolor{currentstroke}%
\pgfsetdash{}{0pt}%
\pgfpathmoveto{\pgfqpoint{0.679669in}{0.526079in}}%
\pgfpathlineto{\pgfqpoint{0.679669in}{3.546079in}}%
\pgfusepath{stroke}%
\end{pgfscope}%
\begin{pgfscope}%
\pgfsetrectcap%
\pgfsetmiterjoin%
\pgfsetlinewidth{0.803000pt}%
\definecolor{currentstroke}{rgb}{0.000000,0.000000,0.000000}%
\pgfsetstrokecolor{currentstroke}%
\pgfsetdash{}{0pt}%
\pgfpathmoveto{\pgfqpoint{5.329669in}{0.526079in}}%
\pgfpathlineto{\pgfqpoint{5.329669in}{3.546079in}}%
\pgfusepath{stroke}%
\end{pgfscope}%
\begin{pgfscope}%
\pgfsetrectcap%
\pgfsetmiterjoin%
\pgfsetlinewidth{0.803000pt}%
\definecolor{currentstroke}{rgb}{0.000000,0.000000,0.000000}%
\pgfsetstrokecolor{currentstroke}%
\pgfsetdash{}{0pt}%
\pgfpathmoveto{\pgfqpoint{0.679669in}{0.526079in}}%
\pgfpathlineto{\pgfqpoint{5.329669in}{0.526079in}}%
\pgfusepath{stroke}%
\end{pgfscope}%
\begin{pgfscope}%
\pgfsetrectcap%
\pgfsetmiterjoin%
\pgfsetlinewidth{0.803000pt}%
\definecolor{currentstroke}{rgb}{0.000000,0.000000,0.000000}%
\pgfsetstrokecolor{currentstroke}%
\pgfsetdash{}{0pt}%
\pgfpathmoveto{\pgfqpoint{0.679669in}{3.546079in}}%
\pgfpathlineto{\pgfqpoint{5.329669in}{3.546079in}}%
\pgfusepath{stroke}%
\end{pgfscope}%
\end{pgfpicture}%
\makeatother%
\endgroup%

    \caption{A log-log plot of the magnitude of the non-dimensional electric field, $\mathbf{E}_y$.\label{fig:E0}}
\end{figure}



\subsection{Scaling}
Introducing scaled variables
\[ \bar{t} = \frac{t}{t_c}, \hspace{10 mm} \bar{y} = \frac{y}{y_c}, \]
where $y_c$ and $t_c$ are characteristic length and time scales respectively, and using the cooridnate transformation $y(0) - R = 0$, the governing equation becomes
\begin{eqnarray}
& \bar{y}'' = - \frac{1}{2} \frac{C_D \rho A y_c}{m} \bar{y}'^2
- \frac{q E_0 t_c^2}{m y_c} \bar{E} 
- \frac{k q^2 t_c^2}{16 \pi \epsilon_0 R^2 m y_c} \frac{1}{\left( \frac{y_c}{R} \bar{y} + 1 \right)^2}, & \nonumber \\
& \bar{y}(0) = 0, \hspace{1 mm} \bar{y}'(0) = \frac{U_0 t_c}{y_c} .&
\end{eqnarray}
We note several dimensionless groups
\[ \mathbf{\Pi}_1 = \frac{C_D \rho A y_c}{2 m}, \hspace{5 mm}
\mathbf{\Pi}_2 = \frac{q E_0 t_c^2}{m y_c}, \hspace{5 mm}
\mathbf{\Pi}_3 = \frac{k q^2 t_c^2}{16 \pi \epsilon_0 R^2 m y_c}, \hspace{5 mm}
\mathbf{\Pi}_4 = \frac{y_c}{R}.\]

\subsubsection{Inertial Electro-Image Limit}
In the limit of small $y, \hspace{1mm} t$ we expect intertia to scale as Coulombic and image force. With $y_c \sim U_0 t_c$ and picking $t_c$ such that Coulombic force is $\mathcal{O}(1)$
\[ t_c \sim \frac{m U_0}{q E_0}, \hspace{5 mm}
y_c \sim \frac{m U_0^2}{q E_0} .
\]
With these scales the governing equation then becomes
\begin{eqnarray}
& \bar{y}'' = -1 - \mathbb{I}\mbox{m} \frac{1}{\left( \mathbb{E}\mbox{u}\bar{y} + 1 \right)^2} ,& \nonumber \\
& \bar{y}(0) = 0, \hspace{1 mm} \bar{y}'(0) = 1 .&
\end{eqnarray} 
with 
\[ \mathbb{I}\mbox{m} \equiv \frac{k q}{16 \pi \epsilon_0 R_d^2 E_0} = \mathbf{\Pi}_3, \hspace{5 mm}
\mathbb{E}\mbox{u} \equiv \frac{m U_0^2}{q E_0} = \mathbf{\Pi}_4 ,
\]
where $\mathbb{I}\mbox{m}$ is the Image number, and denotes the ratio of image forces to the Coulombic force of the unperturbed field, and where where $\mathbb{E}\mbox{u}$ is the electrostatic Euler number, and is a ratio of inertia to electrostatic force.

\subsubsection{Inertial Electro-Viscous Limit}
In the limit of large $y, \hspace{1mm} t$ we expect droplet inertia to scale as Coulombic force and drag. In this case there are several obvious choices of scales:
\begin{enumerate}
\item $y_c \sim U_0 t_c$ and make Coulomb force $\mathcal{O}(1)$.
\item $y_c \sim R_d$ or $L$, $t_c \sim \frac{L}{U_0}$ but this makes the governing equation singular.
\item $y_c \sim R_d$ or $L$, $t_c \sim \left( \frac{L m}{q E_0} \right)^{1/2}$.
\item $y_c \sim R_d$ or $L$, and making Coulomb force $\mathcal{O}(1)$.
\end{enumerate}

In Case 3, the characterisitc time is $t_c \sim \left( \frac{4 \pi R_d^2}{q E_0 L} \right)^2 $ and the non-dimensional governing equation is given by
\begin{eqnarray}
&\bar{y}'' = - \frac{C_D \rho A L}{2 m} \bar{y}'^2 - \frac{1}{\left( \frac{L}{R} \bar{y} + 1\right)^2} ,& \nonumber \\
& \bar{y}(0) = \frac{R}{L}, \hspace{5 mm} \bar{y}'(0) = \left( \frac{4 \pi U_0^2 R_d^2}{q E_0 L^3}\right)^{1/2} = \frac{R}{L} \sqrt{\mathbb{E}\mbox{u}_+}.& \nonumber
\end{eqnarray}
where $\mathbb{E}\mbox{u}_+ = \frac{4 \pi m U_0^2}{q E_0 L}$ is a long time scaled electrostatic Euler number. We prefer the approach with the greatest physical simplicity, fewest Pi terms, and has homogoenous initial conditions.

In Case 1, the characteristic dimensions are
\[ t_c \sim \frac{R^2}{L^2} \frac{4 \pi m U_0}{q E_0}, \hspace{5mm} y_c \sim \frac{R^2}{L^2} \frac{4 \pi m U_0^2}{q E_0}.
\]
With this scaling the non-dimensional governing equation is 
\begin{eqnarray}
&\bar{y}'' = - \mathbb{D}\mbox{g} \mathbb{E}\mbox{u}_+ \bar{y}'^2 - \frac{1}{\left( \mathbb{E}\mbox{u}_+ \bar{y} + 1 \right)^2}, & \nonumber \\
& \bar{y}(0) = 0, \hspace{1 mm} \bar{y}'(0) = 1 & 
\end{eqnarray}
where $\mathbb{D}\mbox{g} = \frac{C_D \rho_a}{\rho_l}$. This is the preferred scaling.

\subsection{Asymptotic Estimates}


\subsubsection{Inertial Electro-Image Limit}
With $\epsilon = \mathbb{E}\mbox{u}$, where $\epsilon$ is a small parameter, and $\alpha = \mathbb{I}\mbox{m}$,
\begin{eqnarray*}
&\bar{y}(\bar{t}) = \bar{t} + \frac{\bar{t}^{2}}{2} \left(-1 - \alpha\right) + \epsilon \left(\frac{\alpha \bar{t}^{3}}{3} + \frac{\alpha \bar{t}^{4}}{12} \left(-1 - \alpha\right)\right)& \\
&+ \epsilon^{2} \left(- \frac{\alpha \bar{t}^{4}}{4} + \frac{\alpha \bar{t}^{5}}{60} \left(9 + 11 \alpha\right) + \frac{\alpha \bar{t}^{6}}{360} \left(-9 - 20 \alpha - 11 \alpha^{2}\right)\right) + \mathcal{O}(\epsilon^3)&
\end{eqnarray*}

\newpage
\begin{figure}[htb]
    \centering
    \resizebox{1\textwidth}{!}{%% Creator: Matplotlib, PGF backend
%%
%% To include the figure in your LaTeX document, write
%%   \input{<filename>.pgf}
%%
%% Make sure the required packages are loaded in your preamble
%%   \usepackage{pgf}
%%
%% Figures using additional raster images can only be included by \input if
%% they are in the same directory as the main LaTeX file. For loading figures
%% from other directories you can use the `import` package
%%   \usepackage{import}
%% and then include the figures with
%%   \import{<path to file>}{<filename>.pgf}
%%
%% Matplotlib used the following preamble
%%   \usepackage{fontspec}
%%   \setmainfont{DejaVu Serif}
%%   \setsansfont{DejaVu Sans}
%%   \setmonofont{DejaVu Sans Mono}
%%
\begingroup%
\makeatletter%
\begin{pgfpicture}%
\pgfpathrectangle{\pgfpointorigin}{\pgfqpoint{10.267664in}{8.135326in}}%
\pgfusepath{use as bounding box, clip}%
\begin{pgfscope}%
\pgfsetbuttcap%
\pgfsetmiterjoin%
\definecolor{currentfill}{rgb}{1.000000,1.000000,1.000000}%
\pgfsetfillcolor{currentfill}%
\pgfsetlinewidth{0.000000pt}%
\definecolor{currentstroke}{rgb}{1.000000,1.000000,1.000000}%
\pgfsetstrokecolor{currentstroke}%
\pgfsetdash{}{0pt}%
\pgfpathmoveto{\pgfqpoint{0.000000in}{0.000000in}}%
\pgfpathlineto{\pgfqpoint{10.267664in}{0.000000in}}%
\pgfpathlineto{\pgfqpoint{10.267664in}{8.135326in}}%
\pgfpathlineto{\pgfqpoint{0.000000in}{8.135326in}}%
\pgfpathclose%
\pgfusepath{fill}%
\end{pgfscope}%
\begin{pgfscope}%
\pgfsetbuttcap%
\pgfsetmiterjoin%
\definecolor{currentfill}{rgb}{1.000000,1.000000,1.000000}%
\pgfsetfillcolor{currentfill}%
\pgfsetlinewidth{0.000000pt}%
\definecolor{currentstroke}{rgb}{0.000000,0.000000,0.000000}%
\pgfsetstrokecolor{currentstroke}%
\pgfsetstrokeopacity{0.000000}%
\pgfsetdash{}{0pt}%
\pgfpathmoveto{\pgfqpoint{0.984216in}{4.549747in}}%
\pgfpathlineto{\pgfqpoint{5.442272in}{4.549747in}}%
\pgfpathlineto{\pgfqpoint{5.442272in}{7.950908in}}%
\pgfpathlineto{\pgfqpoint{0.984216in}{7.950908in}}%
\pgfpathclose%
\pgfusepath{fill}%
\end{pgfscope}%
\begin{pgfscope}%
\pgfsetbuttcap%
\pgfsetroundjoin%
\definecolor{currentfill}{rgb}{0.000000,0.000000,0.000000}%
\pgfsetfillcolor{currentfill}%
\pgfsetlinewidth{0.803000pt}%
\definecolor{currentstroke}{rgb}{0.000000,0.000000,0.000000}%
\pgfsetstrokecolor{currentstroke}%
\pgfsetdash{}{0pt}%
\pgfsys@defobject{currentmarker}{\pgfqpoint{0.000000in}{-0.048611in}}{\pgfqpoint{0.000000in}{0.000000in}}{%
\pgfpathmoveto{\pgfqpoint{0.000000in}{0.000000in}}%
\pgfpathlineto{\pgfqpoint{0.000000in}{-0.048611in}}%
\pgfusepath{stroke,fill}%
}%
\begin{pgfscope}%
\pgfsys@transformshift{1.225002in}{4.549747in}%
\pgfsys@useobject{currentmarker}{}%
\end{pgfscope}%
\end{pgfscope}%
\begin{pgfscope}%
\pgfsetbuttcap%
\pgfsetroundjoin%
\definecolor{currentfill}{rgb}{0.000000,0.000000,0.000000}%
\pgfsetfillcolor{currentfill}%
\pgfsetlinewidth{0.803000pt}%
\definecolor{currentstroke}{rgb}{0.000000,0.000000,0.000000}%
\pgfsetstrokecolor{currentstroke}%
\pgfsetdash{}{0pt}%
\pgfsys@defobject{currentmarker}{\pgfqpoint{0.000000in}{-0.048611in}}{\pgfqpoint{0.000000in}{0.000000in}}{%
\pgfpathmoveto{\pgfqpoint{0.000000in}{0.000000in}}%
\pgfpathlineto{\pgfqpoint{0.000000in}{-0.048611in}}%
\pgfusepath{stroke,fill}%
}%
\begin{pgfscope}%
\pgfsys@transformshift{2.028291in}{4.549747in}%
\pgfsys@useobject{currentmarker}{}%
\end{pgfscope}%
\end{pgfscope}%
\begin{pgfscope}%
\pgfsetbuttcap%
\pgfsetroundjoin%
\definecolor{currentfill}{rgb}{0.000000,0.000000,0.000000}%
\pgfsetfillcolor{currentfill}%
\pgfsetlinewidth{0.803000pt}%
\definecolor{currentstroke}{rgb}{0.000000,0.000000,0.000000}%
\pgfsetstrokecolor{currentstroke}%
\pgfsetdash{}{0pt}%
\pgfsys@defobject{currentmarker}{\pgfqpoint{0.000000in}{-0.048611in}}{\pgfqpoint{0.000000in}{0.000000in}}{%
\pgfpathmoveto{\pgfqpoint{0.000000in}{0.000000in}}%
\pgfpathlineto{\pgfqpoint{0.000000in}{-0.048611in}}%
\pgfusepath{stroke,fill}%
}%
\begin{pgfscope}%
\pgfsys@transformshift{2.831581in}{4.549747in}%
\pgfsys@useobject{currentmarker}{}%
\end{pgfscope}%
\end{pgfscope}%
\begin{pgfscope}%
\pgfsetbuttcap%
\pgfsetroundjoin%
\definecolor{currentfill}{rgb}{0.000000,0.000000,0.000000}%
\pgfsetfillcolor{currentfill}%
\pgfsetlinewidth{0.803000pt}%
\definecolor{currentstroke}{rgb}{0.000000,0.000000,0.000000}%
\pgfsetstrokecolor{currentstroke}%
\pgfsetdash{}{0pt}%
\pgfsys@defobject{currentmarker}{\pgfqpoint{0.000000in}{-0.048611in}}{\pgfqpoint{0.000000in}{0.000000in}}{%
\pgfpathmoveto{\pgfqpoint{0.000000in}{0.000000in}}%
\pgfpathlineto{\pgfqpoint{0.000000in}{-0.048611in}}%
\pgfusepath{stroke,fill}%
}%
\begin{pgfscope}%
\pgfsys@transformshift{3.634870in}{4.549747in}%
\pgfsys@useobject{currentmarker}{}%
\end{pgfscope}%
\end{pgfscope}%
\begin{pgfscope}%
\pgfsetbuttcap%
\pgfsetroundjoin%
\definecolor{currentfill}{rgb}{0.000000,0.000000,0.000000}%
\pgfsetfillcolor{currentfill}%
\pgfsetlinewidth{0.803000pt}%
\definecolor{currentstroke}{rgb}{0.000000,0.000000,0.000000}%
\pgfsetstrokecolor{currentstroke}%
\pgfsetdash{}{0pt}%
\pgfsys@defobject{currentmarker}{\pgfqpoint{0.000000in}{-0.048611in}}{\pgfqpoint{0.000000in}{0.000000in}}{%
\pgfpathmoveto{\pgfqpoint{0.000000in}{0.000000in}}%
\pgfpathlineto{\pgfqpoint{0.000000in}{-0.048611in}}%
\pgfusepath{stroke,fill}%
}%
\begin{pgfscope}%
\pgfsys@transformshift{4.438160in}{4.549747in}%
\pgfsys@useobject{currentmarker}{}%
\end{pgfscope}%
\end{pgfscope}%
\begin{pgfscope}%
\pgfsetbuttcap%
\pgfsetroundjoin%
\definecolor{currentfill}{rgb}{0.000000,0.000000,0.000000}%
\pgfsetfillcolor{currentfill}%
\pgfsetlinewidth{0.803000pt}%
\definecolor{currentstroke}{rgb}{0.000000,0.000000,0.000000}%
\pgfsetstrokecolor{currentstroke}%
\pgfsetdash{}{0pt}%
\pgfsys@defobject{currentmarker}{\pgfqpoint{0.000000in}{-0.048611in}}{\pgfqpoint{0.000000in}{0.000000in}}{%
\pgfpathmoveto{\pgfqpoint{0.000000in}{0.000000in}}%
\pgfpathlineto{\pgfqpoint{0.000000in}{-0.048611in}}%
\pgfusepath{stroke,fill}%
}%
\begin{pgfscope}%
\pgfsys@transformshift{5.241450in}{4.549747in}%
\pgfsys@useobject{currentmarker}{}%
\end{pgfscope}%
\end{pgfscope}%
\begin{pgfscope}%
\pgfsetbuttcap%
\pgfsetroundjoin%
\definecolor{currentfill}{rgb}{0.000000,0.000000,0.000000}%
\pgfsetfillcolor{currentfill}%
\pgfsetlinewidth{0.803000pt}%
\definecolor{currentstroke}{rgb}{0.000000,0.000000,0.000000}%
\pgfsetstrokecolor{currentstroke}%
\pgfsetdash{}{0pt}%
\pgfsys@defobject{currentmarker}{\pgfqpoint{-0.048611in}{0.000000in}}{\pgfqpoint{0.000000in}{0.000000in}}{%
\pgfpathmoveto{\pgfqpoint{0.000000in}{0.000000in}}%
\pgfpathlineto{\pgfqpoint{-0.048611in}{0.000000in}}%
\pgfusepath{stroke,fill}%
}%
\begin{pgfscope}%
\pgfsys@transformshift{0.984216in}{4.549747in}%
\pgfsys@useobject{currentmarker}{}%
\end{pgfscope}%
\end{pgfscope}%
\begin{pgfscope}%
\pgftext[x=0.601580in,y=4.465329in,left,base]{\rmfamily\fontsize{16.000000}{19.200000}\selectfont \(\displaystyle 0.0\)}%
\end{pgfscope}%
\begin{pgfscope}%
\pgfsetbuttcap%
\pgfsetroundjoin%
\definecolor{currentfill}{rgb}{0.000000,0.000000,0.000000}%
\pgfsetfillcolor{currentfill}%
\pgfsetlinewidth{0.803000pt}%
\definecolor{currentstroke}{rgb}{0.000000,0.000000,0.000000}%
\pgfsetstrokecolor{currentstroke}%
\pgfsetdash{}{0pt}%
\pgfsys@defobject{currentmarker}{\pgfqpoint{-0.048611in}{0.000000in}}{\pgfqpoint{0.000000in}{0.000000in}}{%
\pgfpathmoveto{\pgfqpoint{0.000000in}{0.000000in}}%
\pgfpathlineto{\pgfqpoint{-0.048611in}{0.000000in}}%
\pgfusepath{stroke,fill}%
}%
\begin{pgfscope}%
\pgfsys@transformshift{0.984216in}{5.229979in}%
\pgfsys@useobject{currentmarker}{}%
\end{pgfscope}%
\end{pgfscope}%
\begin{pgfscope}%
\pgftext[x=0.601580in,y=5.145561in,left,base]{\rmfamily\fontsize{16.000000}{19.200000}\selectfont \(\displaystyle 0.1\)}%
\end{pgfscope}%
\begin{pgfscope}%
\pgfsetbuttcap%
\pgfsetroundjoin%
\definecolor{currentfill}{rgb}{0.000000,0.000000,0.000000}%
\pgfsetfillcolor{currentfill}%
\pgfsetlinewidth{0.803000pt}%
\definecolor{currentstroke}{rgb}{0.000000,0.000000,0.000000}%
\pgfsetstrokecolor{currentstroke}%
\pgfsetdash{}{0pt}%
\pgfsys@defobject{currentmarker}{\pgfqpoint{-0.048611in}{0.000000in}}{\pgfqpoint{0.000000in}{0.000000in}}{%
\pgfpathmoveto{\pgfqpoint{0.000000in}{0.000000in}}%
\pgfpathlineto{\pgfqpoint{-0.048611in}{0.000000in}}%
\pgfusepath{stroke,fill}%
}%
\begin{pgfscope}%
\pgfsys@transformshift{0.984216in}{5.910211in}%
\pgfsys@useobject{currentmarker}{}%
\end{pgfscope}%
\end{pgfscope}%
\begin{pgfscope}%
\pgftext[x=0.601580in,y=5.825793in,left,base]{\rmfamily\fontsize{16.000000}{19.200000}\selectfont \(\displaystyle 0.2\)}%
\end{pgfscope}%
\begin{pgfscope}%
\pgfsetbuttcap%
\pgfsetroundjoin%
\definecolor{currentfill}{rgb}{0.000000,0.000000,0.000000}%
\pgfsetfillcolor{currentfill}%
\pgfsetlinewidth{0.803000pt}%
\definecolor{currentstroke}{rgb}{0.000000,0.000000,0.000000}%
\pgfsetstrokecolor{currentstroke}%
\pgfsetdash{}{0pt}%
\pgfsys@defobject{currentmarker}{\pgfqpoint{-0.048611in}{0.000000in}}{\pgfqpoint{0.000000in}{0.000000in}}{%
\pgfpathmoveto{\pgfqpoint{0.000000in}{0.000000in}}%
\pgfpathlineto{\pgfqpoint{-0.048611in}{0.000000in}}%
\pgfusepath{stroke,fill}%
}%
\begin{pgfscope}%
\pgfsys@transformshift{0.984216in}{6.590443in}%
\pgfsys@useobject{currentmarker}{}%
\end{pgfscope}%
\end{pgfscope}%
\begin{pgfscope}%
\pgftext[x=0.601580in,y=6.506025in,left,base]{\rmfamily\fontsize{16.000000}{19.200000}\selectfont \(\displaystyle 0.3\)}%
\end{pgfscope}%
\begin{pgfscope}%
\pgfsetbuttcap%
\pgfsetroundjoin%
\definecolor{currentfill}{rgb}{0.000000,0.000000,0.000000}%
\pgfsetfillcolor{currentfill}%
\pgfsetlinewidth{0.803000pt}%
\definecolor{currentstroke}{rgb}{0.000000,0.000000,0.000000}%
\pgfsetstrokecolor{currentstroke}%
\pgfsetdash{}{0pt}%
\pgfsys@defobject{currentmarker}{\pgfqpoint{-0.048611in}{0.000000in}}{\pgfqpoint{0.000000in}{0.000000in}}{%
\pgfpathmoveto{\pgfqpoint{0.000000in}{0.000000in}}%
\pgfpathlineto{\pgfqpoint{-0.048611in}{0.000000in}}%
\pgfusepath{stroke,fill}%
}%
\begin{pgfscope}%
\pgfsys@transformshift{0.984216in}{7.270676in}%
\pgfsys@useobject{currentmarker}{}%
\end{pgfscope}%
\end{pgfscope}%
\begin{pgfscope}%
\pgftext[x=0.601580in,y=7.186257in,left,base]{\rmfamily\fontsize{16.000000}{19.200000}\selectfont \(\displaystyle 0.4\)}%
\end{pgfscope}%
\begin{pgfscope}%
\pgfsetbuttcap%
\pgfsetroundjoin%
\definecolor{currentfill}{rgb}{0.000000,0.000000,0.000000}%
\pgfsetfillcolor{currentfill}%
\pgfsetlinewidth{0.803000pt}%
\definecolor{currentstroke}{rgb}{0.000000,0.000000,0.000000}%
\pgfsetstrokecolor{currentstroke}%
\pgfsetdash{}{0pt}%
\pgfsys@defobject{currentmarker}{\pgfqpoint{-0.048611in}{0.000000in}}{\pgfqpoint{0.000000in}{0.000000in}}{%
\pgfpathmoveto{\pgfqpoint{0.000000in}{0.000000in}}%
\pgfpathlineto{\pgfqpoint{-0.048611in}{0.000000in}}%
\pgfusepath{stroke,fill}%
}%
\begin{pgfscope}%
\pgfsys@transformshift{0.984216in}{7.950908in}%
\pgfsys@useobject{currentmarker}{}%
\end{pgfscope}%
\end{pgfscope}%
\begin{pgfscope}%
\pgftext[x=0.601580in,y=7.866489in,left,base]{\rmfamily\fontsize{16.000000}{19.200000}\selectfont \(\displaystyle 0.5\)}%
\end{pgfscope}%
\begin{pgfscope}%
\pgfpathrectangle{\pgfqpoint{0.984216in}{4.549747in}}{\pgfqpoint{4.458056in}{3.401160in}} %
\pgfusepath{clip}%
\pgfsetbuttcap%
\pgfsetroundjoin%
\pgfsetlinewidth{1.505625pt}%
\definecolor{currentstroke}{rgb}{1.000000,0.000000,0.000000}%
\pgfsetstrokecolor{currentstroke}%
\pgfsetdash{{5.550000pt}{2.400000pt}}{0.000000pt}%
\pgfpathmoveto{\pgfqpoint{1.225002in}{4.549747in}}%
\pgfpathlineto{\pgfqpoint{1.257133in}{4.603515in}}%
\pgfpathlineto{\pgfqpoint{1.289265in}{4.655991in}}%
\pgfpathlineto{\pgfqpoint{1.321396in}{4.707186in}}%
\pgfpathlineto{\pgfqpoint{1.353528in}{4.757115in}}%
\pgfpathlineto{\pgfqpoint{1.385660in}{4.805788in}}%
\pgfpathlineto{\pgfqpoint{1.417791in}{4.853218in}}%
\pgfpathlineto{\pgfqpoint{1.449923in}{4.899414in}}%
\pgfpathlineto{\pgfqpoint{1.482054in}{4.944387in}}%
\pgfpathlineto{\pgfqpoint{1.514186in}{4.988146in}}%
\pgfpathlineto{\pgfqpoint{1.546317in}{5.030702in}}%
\pgfpathlineto{\pgfqpoint{1.578449in}{5.072063in}}%
\pgfpathlineto{\pgfqpoint{1.610581in}{5.112238in}}%
\pgfpathlineto{\pgfqpoint{1.642712in}{5.151234in}}%
\pgfpathlineto{\pgfqpoint{1.674844in}{5.189060in}}%
\pgfpathlineto{\pgfqpoint{1.706975in}{5.225724in}}%
\pgfpathlineto{\pgfqpoint{1.739107in}{5.261231in}}%
\pgfpathlineto{\pgfqpoint{1.771239in}{5.295589in}}%
\pgfpathlineto{\pgfqpoint{1.803370in}{5.328805in}}%
\pgfpathlineto{\pgfqpoint{1.835502in}{5.360885in}}%
\pgfpathlineto{\pgfqpoint{1.867633in}{5.391834in}}%
\pgfpathlineto{\pgfqpoint{1.899765in}{5.421658in}}%
\pgfpathlineto{\pgfqpoint{1.931896in}{5.450363in}}%
\pgfpathlineto{\pgfqpoint{1.964028in}{5.477953in}}%
\pgfpathlineto{\pgfqpoint{1.992143in}{5.501185in}}%
\pgfpathlineto{\pgfqpoint{2.020258in}{5.523570in}}%
\pgfpathlineto{\pgfqpoint{2.048373in}{5.545111in}}%
\pgfpathlineto{\pgfqpoint{2.076489in}{5.565812in}}%
\pgfpathlineto{\pgfqpoint{2.104604in}{5.585675in}}%
\pgfpathlineto{\pgfqpoint{2.132719in}{5.604703in}}%
\pgfpathlineto{\pgfqpoint{2.160834in}{5.622897in}}%
\pgfpathlineto{\pgfqpoint{2.188949in}{5.640262in}}%
\pgfpathlineto{\pgfqpoint{2.217064in}{5.656798in}}%
\pgfpathlineto{\pgfqpoint{2.245179in}{5.672507in}}%
\pgfpathlineto{\pgfqpoint{2.273295in}{5.687393in}}%
\pgfpathlineto{\pgfqpoint{2.301410in}{5.701456in}}%
\pgfpathlineto{\pgfqpoint{2.329525in}{5.714698in}}%
\pgfpathlineto{\pgfqpoint{2.357640in}{5.727122in}}%
\pgfpathlineto{\pgfqpoint{2.385755in}{5.738728in}}%
\pgfpathlineto{\pgfqpoint{2.413870in}{5.749518in}}%
\pgfpathlineto{\pgfqpoint{2.441985in}{5.759494in}}%
\pgfpathlineto{\pgfqpoint{2.470101in}{5.768656in}}%
\pgfpathlineto{\pgfqpoint{2.498216in}{5.777006in}}%
\pgfpathlineto{\pgfqpoint{2.526331in}{5.784545in}}%
\pgfpathlineto{\pgfqpoint{2.554446in}{5.791273in}}%
\pgfpathlineto{\pgfqpoint{2.582561in}{5.797192in}}%
\pgfpathlineto{\pgfqpoint{2.610676in}{5.802303in}}%
\pgfpathlineto{\pgfqpoint{2.638791in}{5.806605in}}%
\pgfpathlineto{\pgfqpoint{2.666906in}{5.810100in}}%
\pgfpathlineto{\pgfqpoint{2.695022in}{5.812788in}}%
\pgfpathlineto{\pgfqpoint{2.723137in}{5.814670in}}%
\pgfpathlineto{\pgfqpoint{2.751252in}{5.815744in}}%
\pgfpathlineto{\pgfqpoint{2.779367in}{5.816013in}}%
\pgfpathlineto{\pgfqpoint{2.807482in}{5.815475in}}%
\pgfpathlineto{\pgfqpoint{2.835597in}{5.814130in}}%
\pgfpathlineto{\pgfqpoint{2.863712in}{5.811979in}}%
\pgfpathlineto{\pgfqpoint{2.891828in}{5.809021in}}%
\pgfpathlineto{\pgfqpoint{2.919943in}{5.805256in}}%
\pgfpathlineto{\pgfqpoint{2.948058in}{5.800683in}}%
\pgfpathlineto{\pgfqpoint{2.976173in}{5.795301in}}%
\pgfpathlineto{\pgfqpoint{3.004288in}{5.789110in}}%
\pgfpathlineto{\pgfqpoint{3.032403in}{5.782109in}}%
\pgfpathlineto{\pgfqpoint{3.060518in}{5.774297in}}%
\pgfpathlineto{\pgfqpoint{3.088633in}{5.765673in}}%
\pgfpathlineto{\pgfqpoint{3.116749in}{5.756235in}}%
\pgfpathlineto{\pgfqpoint{3.144864in}{5.745983in}}%
\pgfpathlineto{\pgfqpoint{3.172979in}{5.734914in}}%
\pgfpathlineto{\pgfqpoint{3.201094in}{5.723028in}}%
\pgfpathlineto{\pgfqpoint{3.229209in}{5.710322in}}%
\pgfpathlineto{\pgfqpoint{3.257324in}{5.696796in}}%
\pgfpathlineto{\pgfqpoint{3.285439in}{5.682446in}}%
\pgfpathlineto{\pgfqpoint{3.313555in}{5.667271in}}%
\pgfpathlineto{\pgfqpoint{3.341670in}{5.651268in}}%
\pgfpathlineto{\pgfqpoint{3.369785in}{5.634436in}}%
\pgfpathlineto{\pgfqpoint{3.397900in}{5.616772in}}%
\pgfpathlineto{\pgfqpoint{3.426015in}{5.598273in}}%
\pgfpathlineto{\pgfqpoint{3.454130in}{5.578937in}}%
\pgfpathlineto{\pgfqpoint{3.482245in}{5.558761in}}%
\pgfpathlineto{\pgfqpoint{3.510361in}{5.537742in}}%
\pgfpathlineto{\pgfqpoint{3.538476in}{5.515876in}}%
\pgfpathlineto{\pgfqpoint{3.566591in}{5.493162in}}%
\pgfpathlineto{\pgfqpoint{3.594706in}{5.469594in}}%
\pgfpathlineto{\pgfqpoint{3.622821in}{5.445171in}}%
\pgfpathlineto{\pgfqpoint{3.654953in}{5.416206in}}%
\pgfpathlineto{\pgfqpoint{3.687084in}{5.386113in}}%
\pgfpathlineto{\pgfqpoint{3.719216in}{5.354886in}}%
\pgfpathlineto{\pgfqpoint{3.751347in}{5.322519in}}%
\pgfpathlineto{\pgfqpoint{3.783479in}{5.289006in}}%
\pgfpathlineto{\pgfqpoint{3.815611in}{5.254340in}}%
\pgfpathlineto{\pgfqpoint{3.847742in}{5.218516in}}%
\pgfpathlineto{\pgfqpoint{3.879874in}{5.181527in}}%
\pgfpathlineto{\pgfqpoint{3.912005in}{5.143364in}}%
\pgfpathlineto{\pgfqpoint{3.944137in}{5.104022in}}%
\pgfpathlineto{\pgfqpoint{3.976268in}{5.063492in}}%
\pgfpathlineto{\pgfqpoint{4.008400in}{5.021768in}}%
\pgfpathlineto{\pgfqpoint{4.040532in}{4.978841in}}%
\pgfpathlineto{\pgfqpoint{4.072663in}{4.934703in}}%
\pgfpathlineto{\pgfqpoint{4.104795in}{4.889345in}}%
\pgfpathlineto{\pgfqpoint{4.136926in}{4.842760in}}%
\pgfpathlineto{\pgfqpoint{4.169058in}{4.794937in}}%
\pgfpathlineto{\pgfqpoint{4.201190in}{4.745868in}}%
\pgfpathlineto{\pgfqpoint{4.233321in}{4.695543in}}%
\pgfpathlineto{\pgfqpoint{4.265453in}{4.643952in}}%
\pgfpathlineto{\pgfqpoint{4.297584in}{4.591085in}}%
\pgfpathlineto{\pgfqpoint{4.324142in}{4.546414in}}%
\pgfpathlineto{\pgfqpoint{4.324142in}{4.546414in}}%
\pgfusepath{stroke}%
\end{pgfscope}%
\begin{pgfscope}%
\pgfpathrectangle{\pgfqpoint{0.984216in}{4.549747in}}{\pgfqpoint{4.458056in}{3.401160in}} %
\pgfusepath{clip}%
\pgfsetbuttcap%
\pgfsetmiterjoin%
\definecolor{currentfill}{rgb}{1.000000,0.000000,0.000000}%
\pgfsetfillcolor{currentfill}%
\pgfsetlinewidth{1.003750pt}%
\definecolor{currentstroke}{rgb}{1.000000,0.000000,0.000000}%
\pgfsetstrokecolor{currentstroke}%
\pgfsetdash{}{0pt}%
\pgfsys@defobject{currentmarker}{\pgfqpoint{-0.041667in}{-0.041667in}}{\pgfqpoint{0.041667in}{0.041667in}}{%
\pgfpathmoveto{\pgfqpoint{-0.041667in}{-0.041667in}}%
\pgfpathlineto{\pgfqpoint{0.041667in}{-0.041667in}}%
\pgfpathlineto{\pgfqpoint{0.041667in}{0.041667in}}%
\pgfpathlineto{\pgfqpoint{-0.041667in}{0.041667in}}%
\pgfpathclose%
\pgfusepath{stroke,fill}%
}%
\begin{pgfscope}%
\pgfsys@transformshift{1.225002in}{4.549747in}%
\pgfsys@useobject{currentmarker}{}%
\end{pgfscope}%
\begin{pgfscope}%
\pgfsys@transformshift{1.626646in}{5.131883in}%
\pgfsys@useobject{currentmarker}{}%
\end{pgfscope}%
\begin{pgfscope}%
\pgfsys@transformshift{2.028291in}{5.529810in}%
\pgfsys@useobject{currentmarker}{}%
\end{pgfscope}%
\begin{pgfscope}%
\pgfsys@transformshift{2.429936in}{5.755318in}%
\pgfsys@useobject{currentmarker}{}%
\end{pgfscope}%
\begin{pgfscope}%
\pgfsys@transformshift{2.831581in}{5.814372in}%
\pgfsys@useobject{currentmarker}{}%
\end{pgfscope}%
\begin{pgfscope}%
\pgfsys@transformshift{3.233226in}{5.708440in}%
\pgfsys@useobject{currentmarker}{}%
\end{pgfscope}%
\begin{pgfscope}%
\pgfsys@transformshift{3.634870in}{5.434441in}%
\pgfsys@useobject{currentmarker}{}%
\end{pgfscope}%
\begin{pgfscope}%
\pgfsys@transformshift{4.036515in}{4.984273in}%
\pgfsys@useobject{currentmarker}{}%
\end{pgfscope}%
\begin{pgfscope}%
\pgfsys@transformshift{4.438160in}{4.344498in}%
\pgfsys@useobject{currentmarker}{}%
\end{pgfscope}%
\begin{pgfscope}%
\pgfsys@transformshift{4.839805in}{3.493435in}%
\pgfsys@useobject{currentmarker}{}%
\end{pgfscope}%
\begin{pgfscope}%
\pgfsys@transformshift{5.241450in}{2.384192in}%
\pgfsys@useobject{currentmarker}{}%
\end{pgfscope}%
\begin{pgfscope}%
\pgfsys@transformshift{5.643094in}{0.885517in}%
\pgfsys@useobject{currentmarker}{}%
\end{pgfscope}%
\end{pgfscope}%
\begin{pgfscope}%
\pgfpathrectangle{\pgfqpoint{0.984216in}{4.549747in}}{\pgfqpoint{4.458056in}{3.401160in}} %
\pgfusepath{clip}%
\pgfsetrectcap%
\pgfsetroundjoin%
\pgfsetlinewidth{1.505625pt}%
\definecolor{currentstroke}{rgb}{0.000000,0.000000,1.000000}%
\pgfsetstrokecolor{currentstroke}%
\pgfsetdash{}{0pt}%
\pgfpathmoveto{\pgfqpoint{1.225002in}{4.549747in}}%
\pgfpathlineto{\pgfqpoint{1.257133in}{4.603513in}}%
\pgfpathlineto{\pgfqpoint{1.289265in}{4.655974in}}%
\pgfpathlineto{\pgfqpoint{1.321396in}{4.707132in}}%
\pgfpathlineto{\pgfqpoint{1.353528in}{4.756988in}}%
\pgfpathlineto{\pgfqpoint{1.385660in}{4.805543in}}%
\pgfpathlineto{\pgfqpoint{1.417791in}{4.852798in}}%
\pgfpathlineto{\pgfqpoint{1.449923in}{4.898755in}}%
\pgfpathlineto{\pgfqpoint{1.482054in}{4.943415in}}%
\pgfpathlineto{\pgfqpoint{1.514186in}{4.986779in}}%
\pgfpathlineto{\pgfqpoint{1.546317in}{5.028848in}}%
\pgfpathlineto{\pgfqpoint{1.578449in}{5.069623in}}%
\pgfpathlineto{\pgfqpoint{1.610581in}{5.109105in}}%
\pgfpathlineto{\pgfqpoint{1.642712in}{5.147295in}}%
\pgfpathlineto{\pgfqpoint{1.674844in}{5.184194in}}%
\pgfpathlineto{\pgfqpoint{1.702959in}{5.215423in}}%
\pgfpathlineto{\pgfqpoint{1.731074in}{5.245664in}}%
\pgfpathlineto{\pgfqpoint{1.759189in}{5.274919in}}%
\pgfpathlineto{\pgfqpoint{1.787304in}{5.303188in}}%
\pgfpathlineto{\pgfqpoint{1.815419in}{5.330472in}}%
\pgfpathlineto{\pgfqpoint{1.843535in}{5.356771in}}%
\pgfpathlineto{\pgfqpoint{1.871650in}{5.382085in}}%
\pgfpathlineto{\pgfqpoint{1.899765in}{5.406415in}}%
\pgfpathlineto{\pgfqpoint{1.927880in}{5.429762in}}%
\pgfpathlineto{\pgfqpoint{1.955995in}{5.452126in}}%
\pgfpathlineto{\pgfqpoint{1.984110in}{5.473507in}}%
\pgfpathlineto{\pgfqpoint{2.012225in}{5.493906in}}%
\pgfpathlineto{\pgfqpoint{2.040341in}{5.513324in}}%
\pgfpathlineto{\pgfqpoint{2.068456in}{5.531759in}}%
\pgfpathlineto{\pgfqpoint{2.096571in}{5.549214in}}%
\pgfpathlineto{\pgfqpoint{2.124686in}{5.565688in}}%
\pgfpathlineto{\pgfqpoint{2.152801in}{5.581181in}}%
\pgfpathlineto{\pgfqpoint{2.176900in}{5.593681in}}%
\pgfpathlineto{\pgfqpoint{2.200998in}{5.605461in}}%
\pgfpathlineto{\pgfqpoint{2.225097in}{5.616522in}}%
\pgfpathlineto{\pgfqpoint{2.249196in}{5.626862in}}%
\pgfpathlineto{\pgfqpoint{2.273295in}{5.636483in}}%
\pgfpathlineto{\pgfqpoint{2.297393in}{5.645385in}}%
\pgfpathlineto{\pgfqpoint{2.321492in}{5.653567in}}%
\pgfpathlineto{\pgfqpoint{2.345591in}{5.661031in}}%
\pgfpathlineto{\pgfqpoint{2.369689in}{5.667775in}}%
\pgfpathlineto{\pgfqpoint{2.393788in}{5.673800in}}%
\pgfpathlineto{\pgfqpoint{2.417887in}{5.679107in}}%
\pgfpathlineto{\pgfqpoint{2.441985in}{5.683695in}}%
\pgfpathlineto{\pgfqpoint{2.466084in}{5.687564in}}%
\pgfpathlineto{\pgfqpoint{2.490183in}{5.690714in}}%
\pgfpathlineto{\pgfqpoint{2.514281in}{5.693145in}}%
\pgfpathlineto{\pgfqpoint{2.538380in}{5.694859in}}%
\pgfpathlineto{\pgfqpoint{2.562479in}{5.695853in}}%
\pgfpathlineto{\pgfqpoint{2.586577in}{5.696129in}}%
\pgfpathlineto{\pgfqpoint{2.610676in}{5.695686in}}%
\pgfpathlineto{\pgfqpoint{2.634775in}{5.694525in}}%
\pgfpathlineto{\pgfqpoint{2.658874in}{5.692646in}}%
\pgfpathlineto{\pgfqpoint{2.682972in}{5.690047in}}%
\pgfpathlineto{\pgfqpoint{2.707071in}{5.686730in}}%
\pgfpathlineto{\pgfqpoint{2.731170in}{5.682695in}}%
\pgfpathlineto{\pgfqpoint{2.755268in}{5.677940in}}%
\pgfpathlineto{\pgfqpoint{2.779367in}{5.672467in}}%
\pgfpathlineto{\pgfqpoint{2.803466in}{5.666275in}}%
\pgfpathlineto{\pgfqpoint{2.827564in}{5.659364in}}%
\pgfpathlineto{\pgfqpoint{2.851663in}{5.651734in}}%
\pgfpathlineto{\pgfqpoint{2.875762in}{5.643385in}}%
\pgfpathlineto{\pgfqpoint{2.899860in}{5.634316in}}%
\pgfpathlineto{\pgfqpoint{2.923959in}{5.624528in}}%
\pgfpathlineto{\pgfqpoint{2.948058in}{5.614021in}}%
\pgfpathlineto{\pgfqpoint{2.972156in}{5.602794in}}%
\pgfpathlineto{\pgfqpoint{2.996255in}{5.590847in}}%
\pgfpathlineto{\pgfqpoint{3.020354in}{5.578180in}}%
\pgfpathlineto{\pgfqpoint{3.048469in}{5.562492in}}%
\pgfpathlineto{\pgfqpoint{3.076584in}{5.545823in}}%
\pgfpathlineto{\pgfqpoint{3.104699in}{5.528173in}}%
\pgfpathlineto{\pgfqpoint{3.132814in}{5.509542in}}%
\pgfpathlineto{\pgfqpoint{3.160930in}{5.489930in}}%
\pgfpathlineto{\pgfqpoint{3.189045in}{5.469336in}}%
\pgfpathlineto{\pgfqpoint{3.217160in}{5.447759in}}%
\pgfpathlineto{\pgfqpoint{3.245275in}{5.425200in}}%
\pgfpathlineto{\pgfqpoint{3.273390in}{5.401657in}}%
\pgfpathlineto{\pgfqpoint{3.301505in}{5.377131in}}%
\pgfpathlineto{\pgfqpoint{3.329620in}{5.351622in}}%
\pgfpathlineto{\pgfqpoint{3.357735in}{5.325127in}}%
\pgfpathlineto{\pgfqpoint{3.385851in}{5.297648in}}%
\pgfpathlineto{\pgfqpoint{3.413966in}{5.269183in}}%
\pgfpathlineto{\pgfqpoint{3.442081in}{5.239732in}}%
\pgfpathlineto{\pgfqpoint{3.470196in}{5.209294in}}%
\pgfpathlineto{\pgfqpoint{3.498311in}{5.177869in}}%
\pgfpathlineto{\pgfqpoint{3.526426in}{5.145457in}}%
\pgfpathlineto{\pgfqpoint{3.558558in}{5.107203in}}%
\pgfpathlineto{\pgfqpoint{3.590689in}{5.067658in}}%
\pgfpathlineto{\pgfqpoint{3.622821in}{5.026820in}}%
\pgfpathlineto{\pgfqpoint{3.654953in}{4.984688in}}%
\pgfpathlineto{\pgfqpoint{3.687084in}{4.941260in}}%
\pgfpathlineto{\pgfqpoint{3.719216in}{4.896537in}}%
\pgfpathlineto{\pgfqpoint{3.751347in}{4.850516in}}%
\pgfpathlineto{\pgfqpoint{3.783479in}{4.803197in}}%
\pgfpathlineto{\pgfqpoint{3.815611in}{4.754579in}}%
\pgfpathlineto{\pgfqpoint{3.847742in}{4.704660in}}%
\pgfpathlineto{\pgfqpoint{3.879874in}{4.653438in}}%
\pgfpathlineto{\pgfqpoint{3.912005in}{4.600913in}}%
\pgfpathlineto{\pgfqpoint{3.944531in}{4.546414in}}%
\pgfpathlineto{\pgfqpoint{3.944531in}{4.546414in}}%
\pgfusepath{stroke}%
\end{pgfscope}%
\begin{pgfscope}%
\pgfpathrectangle{\pgfqpoint{0.984216in}{4.549747in}}{\pgfqpoint{4.458056in}{3.401160in}} %
\pgfusepath{clip}%
\pgfsetbuttcap%
\pgfsetroundjoin%
\definecolor{currentfill}{rgb}{0.000000,0.000000,1.000000}%
\pgfsetfillcolor{currentfill}%
\pgfsetlinewidth{1.003750pt}%
\definecolor{currentstroke}{rgb}{0.000000,0.000000,1.000000}%
\pgfsetstrokecolor{currentstroke}%
\pgfsetdash{}{0pt}%
\pgfsys@defobject{currentmarker}{\pgfqpoint{-0.041667in}{-0.041667in}}{\pgfqpoint{0.041667in}{0.041667in}}{%
\pgfpathmoveto{\pgfqpoint{0.000000in}{-0.041667in}}%
\pgfpathcurveto{\pgfqpoint{0.011050in}{-0.041667in}}{\pgfqpoint{0.021649in}{-0.037276in}}{\pgfqpoint{0.029463in}{-0.029463in}}%
\pgfpathcurveto{\pgfqpoint{0.037276in}{-0.021649in}}{\pgfqpoint{0.041667in}{-0.011050in}}{\pgfqpoint{0.041667in}{0.000000in}}%
\pgfpathcurveto{\pgfqpoint{0.041667in}{0.011050in}}{\pgfqpoint{0.037276in}{0.021649in}}{\pgfqpoint{0.029463in}{0.029463in}}%
\pgfpathcurveto{\pgfqpoint{0.021649in}{0.037276in}}{\pgfqpoint{0.011050in}{0.041667in}}{\pgfqpoint{0.000000in}{0.041667in}}%
\pgfpathcurveto{\pgfqpoint{-0.011050in}{0.041667in}}{\pgfqpoint{-0.021649in}{0.037276in}}{\pgfqpoint{-0.029463in}{0.029463in}}%
\pgfpathcurveto{\pgfqpoint{-0.037276in}{0.021649in}}{\pgfqpoint{-0.041667in}{0.011050in}}{\pgfqpoint{-0.041667in}{0.000000in}}%
\pgfpathcurveto{\pgfqpoint{-0.041667in}{-0.011050in}}{\pgfqpoint{-0.037276in}{-0.021649in}}{\pgfqpoint{-0.029463in}{-0.029463in}}%
\pgfpathcurveto{\pgfqpoint{-0.021649in}{-0.037276in}}{\pgfqpoint{-0.011050in}{-0.041667in}}{\pgfqpoint{0.000000in}{-0.041667in}}%
\pgfpathclose%
\pgfusepath{stroke,fill}%
}%
\begin{pgfscope}%
\pgfsys@transformshift{1.225002in}{4.549747in}%
\pgfsys@useobject{currentmarker}{}%
\end{pgfscope}%
\begin{pgfscope}%
\pgfsys@transformshift{1.626646in}{5.128361in}%
\pgfsys@useobject{currentmarker}{}%
\end{pgfscope}%
\begin{pgfscope}%
\pgfsys@transformshift{2.028291in}{5.505122in}%
\pgfsys@useobject{currentmarker}{}%
\end{pgfscope}%
\begin{pgfscope}%
\pgfsys@transformshift{2.429936in}{5.681491in}%
\pgfsys@useobject{currentmarker}{}%
\end{pgfscope}%
\begin{pgfscope}%
\pgfsys@transformshift{2.831581in}{5.658142in}%
\pgfsys@useobject{currentmarker}{}%
\end{pgfscope}%
\begin{pgfscope}%
\pgfsys@transformshift{3.233226in}{5.434988in}%
\pgfsys@useobject{currentmarker}{}%
\end{pgfscope}%
\begin{pgfscope}%
\pgfsys@transformshift{3.634870in}{5.011172in}%
\pgfsys@useobject{currentmarker}{}%
\end{pgfscope}%
\begin{pgfscope}%
\pgfsys@transformshift{4.036515in}{4.385043in}%
\pgfsys@useobject{currentmarker}{}%
\end{pgfscope}%
\begin{pgfscope}%
\pgfsys@transformshift{4.438160in}{3.554106in}%
\pgfsys@useobject{currentmarker}{}%
\end{pgfscope}%
\begin{pgfscope}%
\pgfsys@transformshift{4.839805in}{2.514942in}%
\pgfsys@useobject{currentmarker}{}%
\end{pgfscope}%
\begin{pgfscope}%
\pgfsys@transformshift{5.241450in}{1.263089in}%
\pgfsys@useobject{currentmarker}{}%
\end{pgfscope}%
\begin{pgfscope}%
\pgfsys@transformshift{5.643094in}{-0.207116in}%
\pgfsys@useobject{currentmarker}{}%
\end{pgfscope}%
\end{pgfscope}%
\begin{pgfscope}%
\pgfpathrectangle{\pgfqpoint{0.984216in}{4.549747in}}{\pgfqpoint{4.458056in}{3.401160in}} %
\pgfusepath{clip}%
\pgfsetbuttcap%
\pgfsetroundjoin%
\pgfsetlinewidth{1.505625pt}%
\definecolor{currentstroke}{rgb}{0.000000,0.750000,0.750000}%
\pgfsetstrokecolor{currentstroke}%
\pgfsetdash{{9.600000pt}{2.400000pt}{1.500000pt}{2.400000pt}}{0.000000pt}%
\pgfpathmoveto{\pgfqpoint{1.225002in}{4.549747in}}%
\pgfpathlineto{\pgfqpoint{1.257133in}{4.603513in}}%
\pgfpathlineto{\pgfqpoint{1.289265in}{4.655973in}}%
\pgfpathlineto{\pgfqpoint{1.321396in}{4.707126in}}%
\pgfpathlineto{\pgfqpoint{1.353528in}{4.756975in}}%
\pgfpathlineto{\pgfqpoint{1.385660in}{4.805517in}}%
\pgfpathlineto{\pgfqpoint{1.417791in}{4.852755in}}%
\pgfpathlineto{\pgfqpoint{1.449923in}{4.898687in}}%
\pgfpathlineto{\pgfqpoint{1.482054in}{4.943314in}}%
\pgfpathlineto{\pgfqpoint{1.514186in}{4.986636in}}%
\pgfpathlineto{\pgfqpoint{1.546317in}{5.028653in}}%
\pgfpathlineto{\pgfqpoint{1.578449in}{5.069365in}}%
\pgfpathlineto{\pgfqpoint{1.610581in}{5.108772in}}%
\pgfpathlineto{\pgfqpoint{1.642712in}{5.146875in}}%
\pgfpathlineto{\pgfqpoint{1.670827in}{5.179145in}}%
\pgfpathlineto{\pgfqpoint{1.698942in}{5.210416in}}%
\pgfpathlineto{\pgfqpoint{1.727058in}{5.240688in}}%
\pgfpathlineto{\pgfqpoint{1.755173in}{5.269962in}}%
\pgfpathlineto{\pgfqpoint{1.783288in}{5.298237in}}%
\pgfpathlineto{\pgfqpoint{1.811403in}{5.325514in}}%
\pgfpathlineto{\pgfqpoint{1.839518in}{5.351793in}}%
\pgfpathlineto{\pgfqpoint{1.867633in}{5.377073in}}%
\pgfpathlineto{\pgfqpoint{1.895748in}{5.401355in}}%
\pgfpathlineto{\pgfqpoint{1.923864in}{5.424638in}}%
\pgfpathlineto{\pgfqpoint{1.951979in}{5.446923in}}%
\pgfpathlineto{\pgfqpoint{1.980094in}{5.468210in}}%
\pgfpathlineto{\pgfqpoint{2.008209in}{5.488499in}}%
\pgfpathlineto{\pgfqpoint{2.036324in}{5.507790in}}%
\pgfpathlineto{\pgfqpoint{2.064439in}{5.526083in}}%
\pgfpathlineto{\pgfqpoint{2.092554in}{5.543377in}}%
\pgfpathlineto{\pgfqpoint{2.120670in}{5.559674in}}%
\pgfpathlineto{\pgfqpoint{2.144768in}{5.572849in}}%
\pgfpathlineto{\pgfqpoint{2.168867in}{5.585290in}}%
\pgfpathlineto{\pgfqpoint{2.192966in}{5.596998in}}%
\pgfpathlineto{\pgfqpoint{2.217064in}{5.607972in}}%
\pgfpathlineto{\pgfqpoint{2.241163in}{5.618214in}}%
\pgfpathlineto{\pgfqpoint{2.265262in}{5.627723in}}%
\pgfpathlineto{\pgfqpoint{2.289360in}{5.636498in}}%
\pgfpathlineto{\pgfqpoint{2.313459in}{5.644541in}}%
\pgfpathlineto{\pgfqpoint{2.337558in}{5.651850in}}%
\pgfpathlineto{\pgfqpoint{2.361656in}{5.658426in}}%
\pgfpathlineto{\pgfqpoint{2.385755in}{5.664269in}}%
\pgfpathlineto{\pgfqpoint{2.409854in}{5.669379in}}%
\pgfpathlineto{\pgfqpoint{2.433952in}{5.673756in}}%
\pgfpathlineto{\pgfqpoint{2.458051in}{5.677400in}}%
\pgfpathlineto{\pgfqpoint{2.482150in}{5.680312in}}%
\pgfpathlineto{\pgfqpoint{2.506249in}{5.682489in}}%
\pgfpathlineto{\pgfqpoint{2.530347in}{5.683934in}}%
\pgfpathlineto{\pgfqpoint{2.554446in}{5.684646in}}%
\pgfpathlineto{\pgfqpoint{2.578545in}{5.684625in}}%
\pgfpathlineto{\pgfqpoint{2.602643in}{5.683871in}}%
\pgfpathlineto{\pgfqpoint{2.626742in}{5.682384in}}%
\pgfpathlineto{\pgfqpoint{2.650841in}{5.680164in}}%
\pgfpathlineto{\pgfqpoint{2.674939in}{5.677211in}}%
\pgfpathlineto{\pgfqpoint{2.699038in}{5.673525in}}%
\pgfpathlineto{\pgfqpoint{2.723137in}{5.669105in}}%
\pgfpathlineto{\pgfqpoint{2.747235in}{5.663953in}}%
\pgfpathlineto{\pgfqpoint{2.771334in}{5.658068in}}%
\pgfpathlineto{\pgfqpoint{2.795433in}{5.651449in}}%
\pgfpathlineto{\pgfqpoint{2.819531in}{5.644098in}}%
\pgfpathlineto{\pgfqpoint{2.843630in}{5.636013in}}%
\pgfpathlineto{\pgfqpoint{2.867729in}{5.627196in}}%
\pgfpathlineto{\pgfqpoint{2.891828in}{5.617645in}}%
\pgfpathlineto{\pgfqpoint{2.915926in}{5.607361in}}%
\pgfpathlineto{\pgfqpoint{2.940025in}{5.596344in}}%
\pgfpathlineto{\pgfqpoint{2.964124in}{5.584594in}}%
\pgfpathlineto{\pgfqpoint{2.988222in}{5.572111in}}%
\pgfpathlineto{\pgfqpoint{3.012321in}{5.558894in}}%
\pgfpathlineto{\pgfqpoint{3.036420in}{5.544945in}}%
\pgfpathlineto{\pgfqpoint{3.064535in}{5.527743in}}%
\pgfpathlineto{\pgfqpoint{3.092650in}{5.509544in}}%
\pgfpathlineto{\pgfqpoint{3.120765in}{5.490347in}}%
\pgfpathlineto{\pgfqpoint{3.148880in}{5.470151in}}%
\pgfpathlineto{\pgfqpoint{3.176995in}{5.448957in}}%
\pgfpathlineto{\pgfqpoint{3.205110in}{5.426766in}}%
\pgfpathlineto{\pgfqpoint{3.233226in}{5.403575in}}%
\pgfpathlineto{\pgfqpoint{3.261341in}{5.379387in}}%
\pgfpathlineto{\pgfqpoint{3.289456in}{5.354200in}}%
\pgfpathlineto{\pgfqpoint{3.317571in}{5.328015in}}%
\pgfpathlineto{\pgfqpoint{3.345686in}{5.300832in}}%
\pgfpathlineto{\pgfqpoint{3.373801in}{5.272650in}}%
\pgfpathlineto{\pgfqpoint{3.401916in}{5.243470in}}%
\pgfpathlineto{\pgfqpoint{3.430032in}{5.213291in}}%
\pgfpathlineto{\pgfqpoint{3.458147in}{5.182113in}}%
\pgfpathlineto{\pgfqpoint{3.486262in}{5.149937in}}%
\pgfpathlineto{\pgfqpoint{3.514377in}{5.116761in}}%
\pgfpathlineto{\pgfqpoint{3.546509in}{5.077624in}}%
\pgfpathlineto{\pgfqpoint{3.578640in}{5.037182in}}%
\pgfpathlineto{\pgfqpoint{3.610772in}{4.995435in}}%
\pgfpathlineto{\pgfqpoint{3.642903in}{4.952383in}}%
\pgfpathlineto{\pgfqpoint{3.675035in}{4.908026in}}%
\pgfpathlineto{\pgfqpoint{3.707166in}{4.862364in}}%
\pgfpathlineto{\pgfqpoint{3.739298in}{4.815397in}}%
\pgfpathlineto{\pgfqpoint{3.771430in}{4.767124in}}%
\pgfpathlineto{\pgfqpoint{3.803561in}{4.717546in}}%
\pgfpathlineto{\pgfqpoint{3.835693in}{4.666662in}}%
\pgfpathlineto{\pgfqpoint{3.867824in}{4.614472in}}%
\pgfpathlineto{\pgfqpoint{3.899956in}{4.560977in}}%
\pgfpathlineto{\pgfqpoint{3.908569in}{4.546414in}}%
\pgfpathlineto{\pgfqpoint{3.908569in}{4.546414in}}%
\pgfusepath{stroke}%
\end{pgfscope}%
\begin{pgfscope}%
\pgfpathrectangle{\pgfqpoint{0.984216in}{4.549747in}}{\pgfqpoint{4.458056in}{3.401160in}} %
\pgfusepath{clip}%
\pgfsetbuttcap%
\pgfsetmiterjoin%
\definecolor{currentfill}{rgb}{0.000000,0.750000,0.750000}%
\pgfsetfillcolor{currentfill}%
\pgfsetlinewidth{1.003750pt}%
\definecolor{currentstroke}{rgb}{0.000000,0.750000,0.750000}%
\pgfsetstrokecolor{currentstroke}%
\pgfsetdash{}{0pt}%
\pgfsys@defobject{currentmarker}{\pgfqpoint{-0.041667in}{-0.041667in}}{\pgfqpoint{0.041667in}{0.041667in}}{%
\pgfpathmoveto{\pgfqpoint{-0.000000in}{-0.041667in}}%
\pgfpathlineto{\pgfqpoint{0.041667in}{0.041667in}}%
\pgfpathlineto{\pgfqpoint{-0.041667in}{0.041667in}}%
\pgfpathclose%
\pgfusepath{stroke,fill}%
}%
\begin{pgfscope}%
\pgfsys@transformshift{1.225002in}{4.549747in}%
\pgfsys@useobject{currentmarker}{}%
\end{pgfscope}%
\begin{pgfscope}%
\pgfsys@transformshift{1.626646in}{5.127986in}%
\pgfsys@useobject{currentmarker}{}%
\end{pgfscope}%
\begin{pgfscope}%
\pgfsys@transformshift{2.028291in}{5.502380in}%
\pgfsys@useobject{currentmarker}{}%
\end{pgfscope}%
\begin{pgfscope}%
\pgfsys@transformshift{2.429936in}{5.673078in}%
\pgfsys@useobject{currentmarker}{}%
\end{pgfscope}%
\begin{pgfscope}%
\pgfsys@transformshift{2.831581in}{5.640147in}%
\pgfsys@useobject{currentmarker}{}%
\end{pgfscope}%
\begin{pgfscope}%
\pgfsys@transformshift{3.233226in}{5.403575in}%
\pgfsys@useobject{currentmarker}{}%
\end{pgfscope}%
\begin{pgfscope}%
\pgfsys@transformshift{3.634870in}{4.963268in}%
\pgfsys@useobject{currentmarker}{}%
\end{pgfscope}%
\begin{pgfscope}%
\pgfsys@transformshift{4.036515in}{4.319050in}%
\pgfsys@useobject{currentmarker}{}%
\end{pgfscope}%
\begin{pgfscope}%
\pgfsys@transformshift{4.438160in}{3.470662in}%
\pgfsys@useobject{currentmarker}{}%
\end{pgfscope}%
\begin{pgfscope}%
\pgfsys@transformshift{4.839805in}{2.417766in}%
\pgfsys@useobject{currentmarker}{}%
\end{pgfscope}%
\begin{pgfscope}%
\pgfsys@transformshift{5.241450in}{1.159936in}%
\pgfsys@useobject{currentmarker}{}%
\end{pgfscope}%
\begin{pgfscope}%
\pgfsys@transformshift{5.643094in}{-0.303337in}%
\pgfsys@useobject{currentmarker}{}%
\end{pgfscope}%
\end{pgfscope}%
\begin{pgfscope}%
\pgfsetrectcap%
\pgfsetmiterjoin%
\pgfsetlinewidth{0.803000pt}%
\definecolor{currentstroke}{rgb}{0.000000,0.000000,0.000000}%
\pgfsetstrokecolor{currentstroke}%
\pgfsetdash{}{0pt}%
\pgfpathmoveto{\pgfqpoint{0.984216in}{4.549747in}}%
\pgfpathlineto{\pgfqpoint{0.984216in}{7.950908in}}%
\pgfusepath{stroke}%
\end{pgfscope}%
\begin{pgfscope}%
\pgfsetrectcap%
\pgfsetmiterjoin%
\pgfsetlinewidth{0.803000pt}%
\definecolor{currentstroke}{rgb}{0.000000,0.000000,0.000000}%
\pgfsetstrokecolor{currentstroke}%
\pgfsetdash{}{0pt}%
\pgfpathmoveto{\pgfqpoint{5.442272in}{4.549747in}}%
\pgfpathlineto{\pgfqpoint{5.442272in}{7.950908in}}%
\pgfusepath{stroke}%
\end{pgfscope}%
\begin{pgfscope}%
\pgfsetrectcap%
\pgfsetmiterjoin%
\pgfsetlinewidth{0.803000pt}%
\definecolor{currentstroke}{rgb}{0.000000,0.000000,0.000000}%
\pgfsetstrokecolor{currentstroke}%
\pgfsetdash{}{0pt}%
\pgfpathmoveto{\pgfqpoint{0.984216in}{4.549747in}}%
\pgfpathlineto{\pgfqpoint{5.442272in}{4.549747in}}%
\pgfusepath{stroke}%
\end{pgfscope}%
\begin{pgfscope}%
\pgfsetrectcap%
\pgfsetmiterjoin%
\pgfsetlinewidth{0.803000pt}%
\definecolor{currentstroke}{rgb}{0.000000,0.000000,0.000000}%
\pgfsetstrokecolor{currentstroke}%
\pgfsetdash{}{0pt}%
\pgfpathmoveto{\pgfqpoint{0.984216in}{7.950908in}}%
\pgfpathlineto{\pgfqpoint{5.442272in}{7.950908in}}%
\pgfusepath{stroke}%
\end{pgfscope}%
\begin{pgfscope}%
\pgfsetbuttcap%
\pgfsetmiterjoin%
\definecolor{currentfill}{rgb}{1.000000,1.000000,1.000000}%
\pgfsetfillcolor{currentfill}%
\pgfsetfillopacity{0.800000}%
\pgfsetlinewidth{1.003750pt}%
\definecolor{currentstroke}{rgb}{0.800000,0.800000,0.800000}%
\pgfsetstrokecolor{currentstroke}%
\pgfsetstrokeopacity{0.800000}%
\pgfsetdash{}{0pt}%
\pgfpathmoveto{\pgfqpoint{1.120327in}{6.650998in}}%
\pgfpathlineto{\pgfqpoint{2.718605in}{6.650998in}}%
\pgfpathquadraticcurveto{\pgfqpoint{2.757494in}{6.650998in}}{\pgfqpoint{2.757494in}{6.689886in}}%
\pgfpathlineto{\pgfqpoint{2.757494in}{7.814796in}}%
\pgfpathquadraticcurveto{\pgfqpoint{2.757494in}{7.853685in}}{\pgfqpoint{2.718605in}{7.853685in}}%
\pgfpathlineto{\pgfqpoint{1.120327in}{7.853685in}}%
\pgfpathquadraticcurveto{\pgfqpoint{1.081438in}{7.853685in}}{\pgfqpoint{1.081438in}{7.814796in}}%
\pgfpathlineto{\pgfqpoint{1.081438in}{6.689886in}}%
\pgfpathquadraticcurveto{\pgfqpoint{1.081438in}{6.650998in}}{\pgfqpoint{1.120327in}{6.650998in}}%
\pgfpathclose%
\pgfusepath{stroke,fill}%
\end{pgfscope}%
\begin{pgfscope}%
\pgftext[x=1.615059in,y=7.628175in,left,base]{\rmfamily\fontsize{14.000000}{16.800000}\selectfont \(\displaystyle \mathbf{I}\mbox{g} = \) 2}%
\end{pgfscope}%
\begin{pgfscope}%
\pgfsetbuttcap%
\pgfsetroundjoin%
\pgfsetlinewidth{1.505625pt}%
\definecolor{currentstroke}{rgb}{1.000000,0.000000,0.000000}%
\pgfsetstrokecolor{currentstroke}%
\pgfsetdash{{5.550000pt}{2.400000pt}}{0.000000pt}%
\pgfpathmoveto{\pgfqpoint{1.159216in}{7.408077in}}%
\pgfpathlineto{\pgfqpoint{1.548104in}{7.408077in}}%
\pgfusepath{stroke}%
\end{pgfscope}%
\begin{pgfscope}%
\pgfsetbuttcap%
\pgfsetmiterjoin%
\definecolor{currentfill}{rgb}{1.000000,0.000000,0.000000}%
\pgfsetfillcolor{currentfill}%
\pgfsetlinewidth{1.003750pt}%
\definecolor{currentstroke}{rgb}{1.000000,0.000000,0.000000}%
\pgfsetstrokecolor{currentstroke}%
\pgfsetdash{}{0pt}%
\pgfsys@defobject{currentmarker}{\pgfqpoint{-0.041667in}{-0.041667in}}{\pgfqpoint{0.041667in}{0.041667in}}{%
\pgfpathmoveto{\pgfqpoint{-0.041667in}{-0.041667in}}%
\pgfpathlineto{\pgfqpoint{0.041667in}{-0.041667in}}%
\pgfpathlineto{\pgfqpoint{0.041667in}{0.041667in}}%
\pgfpathlineto{\pgfqpoint{-0.041667in}{0.041667in}}%
\pgfpathclose%
\pgfusepath{stroke,fill}%
}%
\begin{pgfscope}%
\pgfsys@transformshift{1.353660in}{7.408077in}%
\pgfsys@useobject{currentmarker}{}%
\end{pgfscope}%
\end{pgfscope}%
\begin{pgfscope}%
\pgftext[x=1.703660in,y=7.340022in,left,base]{\rmfamily\fontsize{14.000000}{16.800000}\selectfont \(\displaystyle \mathbf{E}\mbox{u}=\) 1}%
\end{pgfscope}%
\begin{pgfscope}%
\pgfsetrectcap%
\pgfsetroundjoin%
\pgfsetlinewidth{1.505625pt}%
\definecolor{currentstroke}{rgb}{0.000000,0.000000,1.000000}%
\pgfsetstrokecolor{currentstroke}%
\pgfsetdash{}{0pt}%
\pgfpathmoveto{\pgfqpoint{1.159216in}{7.122677in}}%
\pgfpathlineto{\pgfqpoint{1.548104in}{7.122677in}}%
\pgfusepath{stroke}%
\end{pgfscope}%
\begin{pgfscope}%
\pgfsetbuttcap%
\pgfsetroundjoin%
\definecolor{currentfill}{rgb}{0.000000,0.000000,1.000000}%
\pgfsetfillcolor{currentfill}%
\pgfsetlinewidth{1.003750pt}%
\definecolor{currentstroke}{rgb}{0.000000,0.000000,1.000000}%
\pgfsetstrokecolor{currentstroke}%
\pgfsetdash{}{0pt}%
\pgfsys@defobject{currentmarker}{\pgfqpoint{-0.041667in}{-0.041667in}}{\pgfqpoint{0.041667in}{0.041667in}}{%
\pgfpathmoveto{\pgfqpoint{0.000000in}{-0.041667in}}%
\pgfpathcurveto{\pgfqpoint{0.011050in}{-0.041667in}}{\pgfqpoint{0.021649in}{-0.037276in}}{\pgfqpoint{0.029463in}{-0.029463in}}%
\pgfpathcurveto{\pgfqpoint{0.037276in}{-0.021649in}}{\pgfqpoint{0.041667in}{-0.011050in}}{\pgfqpoint{0.041667in}{0.000000in}}%
\pgfpathcurveto{\pgfqpoint{0.041667in}{0.011050in}}{\pgfqpoint{0.037276in}{0.021649in}}{\pgfqpoint{0.029463in}{0.029463in}}%
\pgfpathcurveto{\pgfqpoint{0.021649in}{0.037276in}}{\pgfqpoint{0.011050in}{0.041667in}}{\pgfqpoint{0.000000in}{0.041667in}}%
\pgfpathcurveto{\pgfqpoint{-0.011050in}{0.041667in}}{\pgfqpoint{-0.021649in}{0.037276in}}{\pgfqpoint{-0.029463in}{0.029463in}}%
\pgfpathcurveto{\pgfqpoint{-0.037276in}{0.021649in}}{\pgfqpoint{-0.041667in}{0.011050in}}{\pgfqpoint{-0.041667in}{0.000000in}}%
\pgfpathcurveto{\pgfqpoint{-0.041667in}{-0.011050in}}{\pgfqpoint{-0.037276in}{-0.021649in}}{\pgfqpoint{-0.029463in}{-0.029463in}}%
\pgfpathcurveto{\pgfqpoint{-0.021649in}{-0.037276in}}{\pgfqpoint{-0.011050in}{-0.041667in}}{\pgfqpoint{0.000000in}{-0.041667in}}%
\pgfpathclose%
\pgfusepath{stroke,fill}%
}%
\begin{pgfscope}%
\pgfsys@transformshift{1.353660in}{7.122677in}%
\pgfsys@useobject{currentmarker}{}%
\end{pgfscope}%
\end{pgfscope}%
\begin{pgfscope}%
\pgftext[x=1.703660in,y=7.054621in,left,base]{\rmfamily\fontsize{14.000000}{16.800000}\selectfont \(\displaystyle \mathbf{E}\mbox{u}=\) 0.1}%
\end{pgfscope}%
\begin{pgfscope}%
\pgfsetbuttcap%
\pgfsetroundjoin%
\pgfsetlinewidth{1.505625pt}%
\definecolor{currentstroke}{rgb}{0.000000,0.750000,0.750000}%
\pgfsetstrokecolor{currentstroke}%
\pgfsetdash{{9.600000pt}{2.400000pt}{1.500000pt}{2.400000pt}}{0.000000pt}%
\pgfpathmoveto{\pgfqpoint{1.159216in}{6.837277in}}%
\pgfpathlineto{\pgfqpoint{1.548104in}{6.837277in}}%
\pgfusepath{stroke}%
\end{pgfscope}%
\begin{pgfscope}%
\pgfsetbuttcap%
\pgfsetmiterjoin%
\definecolor{currentfill}{rgb}{0.000000,0.750000,0.750000}%
\pgfsetfillcolor{currentfill}%
\pgfsetlinewidth{1.003750pt}%
\definecolor{currentstroke}{rgb}{0.000000,0.750000,0.750000}%
\pgfsetstrokecolor{currentstroke}%
\pgfsetdash{}{0pt}%
\pgfsys@defobject{currentmarker}{\pgfqpoint{-0.041667in}{-0.041667in}}{\pgfqpoint{0.041667in}{0.041667in}}{%
\pgfpathmoveto{\pgfqpoint{-0.000000in}{-0.041667in}}%
\pgfpathlineto{\pgfqpoint{0.041667in}{0.041667in}}%
\pgfpathlineto{\pgfqpoint{-0.041667in}{0.041667in}}%
\pgfpathclose%
\pgfusepath{stroke,fill}%
}%
\begin{pgfscope}%
\pgfsys@transformshift{1.353660in}{6.837277in}%
\pgfsys@useobject{currentmarker}{}%
\end{pgfscope}%
\end{pgfscope}%
\begin{pgfscope}%
\pgftext[x=1.703660in,y=6.769221in,left,base]{\rmfamily\fontsize{14.000000}{16.800000}\selectfont \(\displaystyle \mathbf{E}\mbox{u}=\) 0.01}%
\end{pgfscope}%
\begin{pgfscope}%
\pgfsetbuttcap%
\pgfsetmiterjoin%
\definecolor{currentfill}{rgb}{1.000000,1.000000,1.000000}%
\pgfsetfillcolor{currentfill}%
\pgfsetlinewidth{0.000000pt}%
\definecolor{currentstroke}{rgb}{0.000000,0.000000,0.000000}%
\pgfsetstrokecolor{currentstroke}%
\pgfsetstrokeopacity{0.000000}%
\pgfsetdash{}{0pt}%
\pgfpathmoveto{\pgfqpoint{5.697941in}{4.549747in}}%
\pgfpathlineto{\pgfqpoint{10.155998in}{4.549747in}}%
\pgfpathlineto{\pgfqpoint{10.155998in}{7.950908in}}%
\pgfpathlineto{\pgfqpoint{5.697941in}{7.950908in}}%
\pgfpathclose%
\pgfusepath{fill}%
\end{pgfscope}%
\begin{pgfscope}%
\pgfsetbuttcap%
\pgfsetroundjoin%
\definecolor{currentfill}{rgb}{0.000000,0.000000,0.000000}%
\pgfsetfillcolor{currentfill}%
\pgfsetlinewidth{0.803000pt}%
\definecolor{currentstroke}{rgb}{0.000000,0.000000,0.000000}%
\pgfsetstrokecolor{currentstroke}%
\pgfsetdash{}{0pt}%
\pgfsys@defobject{currentmarker}{\pgfqpoint{0.000000in}{-0.048611in}}{\pgfqpoint{0.000000in}{0.000000in}}{%
\pgfpathmoveto{\pgfqpoint{0.000000in}{0.000000in}}%
\pgfpathlineto{\pgfqpoint{0.000000in}{-0.048611in}}%
\pgfusepath{stroke,fill}%
}%
\begin{pgfscope}%
\pgfsys@transformshift{5.938727in}{4.549747in}%
\pgfsys@useobject{currentmarker}{}%
\end{pgfscope}%
\end{pgfscope}%
\begin{pgfscope}%
\pgfsetbuttcap%
\pgfsetroundjoin%
\definecolor{currentfill}{rgb}{0.000000,0.000000,0.000000}%
\pgfsetfillcolor{currentfill}%
\pgfsetlinewidth{0.803000pt}%
\definecolor{currentstroke}{rgb}{0.000000,0.000000,0.000000}%
\pgfsetstrokecolor{currentstroke}%
\pgfsetdash{}{0pt}%
\pgfsys@defobject{currentmarker}{\pgfqpoint{0.000000in}{-0.048611in}}{\pgfqpoint{0.000000in}{0.000000in}}{%
\pgfpathmoveto{\pgfqpoint{0.000000in}{0.000000in}}%
\pgfpathlineto{\pgfqpoint{0.000000in}{-0.048611in}}%
\pgfusepath{stroke,fill}%
}%
\begin{pgfscope}%
\pgfsys@transformshift{6.742017in}{4.549747in}%
\pgfsys@useobject{currentmarker}{}%
\end{pgfscope}%
\end{pgfscope}%
\begin{pgfscope}%
\pgfsetbuttcap%
\pgfsetroundjoin%
\definecolor{currentfill}{rgb}{0.000000,0.000000,0.000000}%
\pgfsetfillcolor{currentfill}%
\pgfsetlinewidth{0.803000pt}%
\definecolor{currentstroke}{rgb}{0.000000,0.000000,0.000000}%
\pgfsetstrokecolor{currentstroke}%
\pgfsetdash{}{0pt}%
\pgfsys@defobject{currentmarker}{\pgfqpoint{0.000000in}{-0.048611in}}{\pgfqpoint{0.000000in}{0.000000in}}{%
\pgfpathmoveto{\pgfqpoint{0.000000in}{0.000000in}}%
\pgfpathlineto{\pgfqpoint{0.000000in}{-0.048611in}}%
\pgfusepath{stroke,fill}%
}%
\begin{pgfscope}%
\pgfsys@transformshift{7.545306in}{4.549747in}%
\pgfsys@useobject{currentmarker}{}%
\end{pgfscope}%
\end{pgfscope}%
\begin{pgfscope}%
\pgfsetbuttcap%
\pgfsetroundjoin%
\definecolor{currentfill}{rgb}{0.000000,0.000000,0.000000}%
\pgfsetfillcolor{currentfill}%
\pgfsetlinewidth{0.803000pt}%
\definecolor{currentstroke}{rgb}{0.000000,0.000000,0.000000}%
\pgfsetstrokecolor{currentstroke}%
\pgfsetdash{}{0pt}%
\pgfsys@defobject{currentmarker}{\pgfqpoint{0.000000in}{-0.048611in}}{\pgfqpoint{0.000000in}{0.000000in}}{%
\pgfpathmoveto{\pgfqpoint{0.000000in}{0.000000in}}%
\pgfpathlineto{\pgfqpoint{0.000000in}{-0.048611in}}%
\pgfusepath{stroke,fill}%
}%
\begin{pgfscope}%
\pgfsys@transformshift{8.348596in}{4.549747in}%
\pgfsys@useobject{currentmarker}{}%
\end{pgfscope}%
\end{pgfscope}%
\begin{pgfscope}%
\pgfsetbuttcap%
\pgfsetroundjoin%
\definecolor{currentfill}{rgb}{0.000000,0.000000,0.000000}%
\pgfsetfillcolor{currentfill}%
\pgfsetlinewidth{0.803000pt}%
\definecolor{currentstroke}{rgb}{0.000000,0.000000,0.000000}%
\pgfsetstrokecolor{currentstroke}%
\pgfsetdash{}{0pt}%
\pgfsys@defobject{currentmarker}{\pgfqpoint{0.000000in}{-0.048611in}}{\pgfqpoint{0.000000in}{0.000000in}}{%
\pgfpathmoveto{\pgfqpoint{0.000000in}{0.000000in}}%
\pgfpathlineto{\pgfqpoint{0.000000in}{-0.048611in}}%
\pgfusepath{stroke,fill}%
}%
\begin{pgfscope}%
\pgfsys@transformshift{9.151886in}{4.549747in}%
\pgfsys@useobject{currentmarker}{}%
\end{pgfscope}%
\end{pgfscope}%
\begin{pgfscope}%
\pgfsetbuttcap%
\pgfsetroundjoin%
\definecolor{currentfill}{rgb}{0.000000,0.000000,0.000000}%
\pgfsetfillcolor{currentfill}%
\pgfsetlinewidth{0.803000pt}%
\definecolor{currentstroke}{rgb}{0.000000,0.000000,0.000000}%
\pgfsetstrokecolor{currentstroke}%
\pgfsetdash{}{0pt}%
\pgfsys@defobject{currentmarker}{\pgfqpoint{0.000000in}{-0.048611in}}{\pgfqpoint{0.000000in}{0.000000in}}{%
\pgfpathmoveto{\pgfqpoint{0.000000in}{0.000000in}}%
\pgfpathlineto{\pgfqpoint{0.000000in}{-0.048611in}}%
\pgfusepath{stroke,fill}%
}%
\begin{pgfscope}%
\pgfsys@transformshift{9.955175in}{4.549747in}%
\pgfsys@useobject{currentmarker}{}%
\end{pgfscope}%
\end{pgfscope}%
\begin{pgfscope}%
\pgfsetbuttcap%
\pgfsetroundjoin%
\definecolor{currentfill}{rgb}{0.000000,0.000000,0.000000}%
\pgfsetfillcolor{currentfill}%
\pgfsetlinewidth{0.803000pt}%
\definecolor{currentstroke}{rgb}{0.000000,0.000000,0.000000}%
\pgfsetstrokecolor{currentstroke}%
\pgfsetdash{}{0pt}%
\pgfsys@defobject{currentmarker}{\pgfqpoint{-0.048611in}{0.000000in}}{\pgfqpoint{0.000000in}{0.000000in}}{%
\pgfpathmoveto{\pgfqpoint{0.000000in}{0.000000in}}%
\pgfpathlineto{\pgfqpoint{-0.048611in}{0.000000in}}%
\pgfusepath{stroke,fill}%
}%
\begin{pgfscope}%
\pgfsys@transformshift{5.697941in}{4.549747in}%
\pgfsys@useobject{currentmarker}{}%
\end{pgfscope}%
\end{pgfscope}%
\begin{pgfscope}%
\pgfsetbuttcap%
\pgfsetroundjoin%
\definecolor{currentfill}{rgb}{0.000000,0.000000,0.000000}%
\pgfsetfillcolor{currentfill}%
\pgfsetlinewidth{0.803000pt}%
\definecolor{currentstroke}{rgb}{0.000000,0.000000,0.000000}%
\pgfsetstrokecolor{currentstroke}%
\pgfsetdash{}{0pt}%
\pgfsys@defobject{currentmarker}{\pgfqpoint{-0.048611in}{0.000000in}}{\pgfqpoint{0.000000in}{0.000000in}}{%
\pgfpathmoveto{\pgfqpoint{0.000000in}{0.000000in}}%
\pgfpathlineto{\pgfqpoint{-0.048611in}{0.000000in}}%
\pgfusepath{stroke,fill}%
}%
\begin{pgfscope}%
\pgfsys@transformshift{5.697941in}{5.229979in}%
\pgfsys@useobject{currentmarker}{}%
\end{pgfscope}%
\end{pgfscope}%
\begin{pgfscope}%
\pgfsetbuttcap%
\pgfsetroundjoin%
\definecolor{currentfill}{rgb}{0.000000,0.000000,0.000000}%
\pgfsetfillcolor{currentfill}%
\pgfsetlinewidth{0.803000pt}%
\definecolor{currentstroke}{rgb}{0.000000,0.000000,0.000000}%
\pgfsetstrokecolor{currentstroke}%
\pgfsetdash{}{0pt}%
\pgfsys@defobject{currentmarker}{\pgfqpoint{-0.048611in}{0.000000in}}{\pgfqpoint{0.000000in}{0.000000in}}{%
\pgfpathmoveto{\pgfqpoint{0.000000in}{0.000000in}}%
\pgfpathlineto{\pgfqpoint{-0.048611in}{0.000000in}}%
\pgfusepath{stroke,fill}%
}%
\begin{pgfscope}%
\pgfsys@transformshift{5.697941in}{5.910211in}%
\pgfsys@useobject{currentmarker}{}%
\end{pgfscope}%
\end{pgfscope}%
\begin{pgfscope}%
\pgfsetbuttcap%
\pgfsetroundjoin%
\definecolor{currentfill}{rgb}{0.000000,0.000000,0.000000}%
\pgfsetfillcolor{currentfill}%
\pgfsetlinewidth{0.803000pt}%
\definecolor{currentstroke}{rgb}{0.000000,0.000000,0.000000}%
\pgfsetstrokecolor{currentstroke}%
\pgfsetdash{}{0pt}%
\pgfsys@defobject{currentmarker}{\pgfqpoint{-0.048611in}{0.000000in}}{\pgfqpoint{0.000000in}{0.000000in}}{%
\pgfpathmoveto{\pgfqpoint{0.000000in}{0.000000in}}%
\pgfpathlineto{\pgfqpoint{-0.048611in}{0.000000in}}%
\pgfusepath{stroke,fill}%
}%
\begin{pgfscope}%
\pgfsys@transformshift{5.697941in}{6.590443in}%
\pgfsys@useobject{currentmarker}{}%
\end{pgfscope}%
\end{pgfscope}%
\begin{pgfscope}%
\pgfsetbuttcap%
\pgfsetroundjoin%
\definecolor{currentfill}{rgb}{0.000000,0.000000,0.000000}%
\pgfsetfillcolor{currentfill}%
\pgfsetlinewidth{0.803000pt}%
\definecolor{currentstroke}{rgb}{0.000000,0.000000,0.000000}%
\pgfsetstrokecolor{currentstroke}%
\pgfsetdash{}{0pt}%
\pgfsys@defobject{currentmarker}{\pgfqpoint{-0.048611in}{0.000000in}}{\pgfqpoint{0.000000in}{0.000000in}}{%
\pgfpathmoveto{\pgfqpoint{0.000000in}{0.000000in}}%
\pgfpathlineto{\pgfqpoint{-0.048611in}{0.000000in}}%
\pgfusepath{stroke,fill}%
}%
\begin{pgfscope}%
\pgfsys@transformshift{5.697941in}{7.270676in}%
\pgfsys@useobject{currentmarker}{}%
\end{pgfscope}%
\end{pgfscope}%
\begin{pgfscope}%
\pgfsetbuttcap%
\pgfsetroundjoin%
\definecolor{currentfill}{rgb}{0.000000,0.000000,0.000000}%
\pgfsetfillcolor{currentfill}%
\pgfsetlinewidth{0.803000pt}%
\definecolor{currentstroke}{rgb}{0.000000,0.000000,0.000000}%
\pgfsetstrokecolor{currentstroke}%
\pgfsetdash{}{0pt}%
\pgfsys@defobject{currentmarker}{\pgfqpoint{-0.048611in}{0.000000in}}{\pgfqpoint{0.000000in}{0.000000in}}{%
\pgfpathmoveto{\pgfqpoint{0.000000in}{0.000000in}}%
\pgfpathlineto{\pgfqpoint{-0.048611in}{0.000000in}}%
\pgfusepath{stroke,fill}%
}%
\begin{pgfscope}%
\pgfsys@transformshift{5.697941in}{7.950908in}%
\pgfsys@useobject{currentmarker}{}%
\end{pgfscope}%
\end{pgfscope}%
\begin{pgfscope}%
\pgfpathrectangle{\pgfqpoint{5.697941in}{4.549747in}}{\pgfqpoint{4.458056in}{3.401160in}} %
\pgfusepath{clip}%
\pgfsetbuttcap%
\pgfsetroundjoin%
\pgfsetlinewidth{1.505625pt}%
\definecolor{currentstroke}{rgb}{1.000000,0.000000,0.000000}%
\pgfsetstrokecolor{currentstroke}%
\pgfsetdash{{5.550000pt}{2.400000pt}}{0.000000pt}%
\pgfpathmoveto{\pgfqpoint{5.938727in}{4.549747in}}%
\pgfpathlineto{\pgfqpoint{5.978892in}{4.617093in}}%
\pgfpathlineto{\pgfqpoint{6.019056in}{4.683091in}}%
\pgfpathlineto{\pgfqpoint{6.059221in}{4.747754in}}%
\pgfpathlineto{\pgfqpoint{6.099385in}{4.811095in}}%
\pgfpathlineto{\pgfqpoint{6.139550in}{4.873124in}}%
\pgfpathlineto{\pgfqpoint{6.179714in}{4.933854in}}%
\pgfpathlineto{\pgfqpoint{6.219879in}{4.993294in}}%
\pgfpathlineto{\pgfqpoint{6.260043in}{5.051454in}}%
\pgfpathlineto{\pgfqpoint{6.300207in}{5.108343in}}%
\pgfpathlineto{\pgfqpoint{6.340372in}{5.163972in}}%
\pgfpathlineto{\pgfqpoint{6.380536in}{5.218348in}}%
\pgfpathlineto{\pgfqpoint{6.420701in}{5.271480in}}%
\pgfpathlineto{\pgfqpoint{6.460865in}{5.323375in}}%
\pgfpathlineto{\pgfqpoint{6.501030in}{5.374042in}}%
\pgfpathlineto{\pgfqpoint{6.541194in}{5.423488in}}%
\pgfpathlineto{\pgfqpoint{6.581359in}{5.471718in}}%
\pgfpathlineto{\pgfqpoint{6.621523in}{5.518741in}}%
\pgfpathlineto{\pgfqpoint{6.661688in}{5.564563in}}%
\pgfpathlineto{\pgfqpoint{6.701852in}{5.609189in}}%
\pgfpathlineto{\pgfqpoint{6.742017in}{5.652625in}}%
\pgfpathlineto{\pgfqpoint{6.782181in}{5.694877in}}%
\pgfpathlineto{\pgfqpoint{6.822346in}{5.735950in}}%
\pgfpathlineto{\pgfqpoint{6.858494in}{5.771913in}}%
\pgfpathlineto{\pgfqpoint{6.894642in}{5.806928in}}%
\pgfpathlineto{\pgfqpoint{6.930790in}{5.841000in}}%
\pgfpathlineto{\pgfqpoint{6.966938in}{5.874131in}}%
\pgfpathlineto{\pgfqpoint{7.003086in}{5.906325in}}%
\pgfpathlineto{\pgfqpoint{7.039234in}{5.937584in}}%
\pgfpathlineto{\pgfqpoint{7.075382in}{5.967913in}}%
\pgfpathlineto{\pgfqpoint{7.111530in}{5.997312in}}%
\pgfpathlineto{\pgfqpoint{7.147678in}{6.025787in}}%
\pgfpathlineto{\pgfqpoint{7.183826in}{6.053337in}}%
\pgfpathlineto{\pgfqpoint{7.219974in}{6.079968in}}%
\pgfpathlineto{\pgfqpoint{7.256122in}{6.105680in}}%
\pgfpathlineto{\pgfqpoint{7.292270in}{6.130476in}}%
\pgfpathlineto{\pgfqpoint{7.328418in}{6.154358in}}%
\pgfpathlineto{\pgfqpoint{7.364566in}{6.177329in}}%
\pgfpathlineto{\pgfqpoint{7.400714in}{6.199390in}}%
\pgfpathlineto{\pgfqpoint{7.436862in}{6.220543in}}%
\pgfpathlineto{\pgfqpoint{7.473010in}{6.240791in}}%
\pgfpathlineto{\pgfqpoint{7.509158in}{6.260135in}}%
\pgfpathlineto{\pgfqpoint{7.545306in}{6.278577in}}%
\pgfpathlineto{\pgfqpoint{7.581454in}{6.296118in}}%
\pgfpathlineto{\pgfqpoint{7.617602in}{6.312761in}}%
\pgfpathlineto{\pgfqpoint{7.653750in}{6.328506in}}%
\pgfpathlineto{\pgfqpoint{7.689898in}{6.343355in}}%
\pgfpathlineto{\pgfqpoint{7.726047in}{6.357309in}}%
\pgfpathlineto{\pgfqpoint{7.758178in}{6.368964in}}%
\pgfpathlineto{\pgfqpoint{7.790310in}{6.379913in}}%
\pgfpathlineto{\pgfqpoint{7.822441in}{6.390158in}}%
\pgfpathlineto{\pgfqpoint{7.854573in}{6.399700in}}%
\pgfpathlineto{\pgfqpoint{7.886704in}{6.408539in}}%
\pgfpathlineto{\pgfqpoint{7.918836in}{6.416677in}}%
\pgfpathlineto{\pgfqpoint{7.950968in}{6.424113in}}%
\pgfpathlineto{\pgfqpoint{7.983099in}{6.430849in}}%
\pgfpathlineto{\pgfqpoint{8.015231in}{6.436885in}}%
\pgfpathlineto{\pgfqpoint{8.047362in}{6.442221in}}%
\pgfpathlineto{\pgfqpoint{8.079494in}{6.446858in}}%
\pgfpathlineto{\pgfqpoint{8.111626in}{6.450797in}}%
\pgfpathlineto{\pgfqpoint{8.143757in}{6.454037in}}%
\pgfpathlineto{\pgfqpoint{8.175889in}{6.456580in}}%
\pgfpathlineto{\pgfqpoint{8.208020in}{6.458425in}}%
\pgfpathlineto{\pgfqpoint{8.240152in}{6.459572in}}%
\pgfpathlineto{\pgfqpoint{8.272283in}{6.460022in}}%
\pgfpathlineto{\pgfqpoint{8.304415in}{6.459775in}}%
\pgfpathlineto{\pgfqpoint{8.336547in}{6.458831in}}%
\pgfpathlineto{\pgfqpoint{8.368678in}{6.457190in}}%
\pgfpathlineto{\pgfqpoint{8.400810in}{6.454851in}}%
\pgfpathlineto{\pgfqpoint{8.432941in}{6.451815in}}%
\pgfpathlineto{\pgfqpoint{8.465073in}{6.448082in}}%
\pgfpathlineto{\pgfqpoint{8.497205in}{6.443650in}}%
\pgfpathlineto{\pgfqpoint{8.529336in}{6.438520in}}%
\pgfpathlineto{\pgfqpoint{8.561468in}{6.432692in}}%
\pgfpathlineto{\pgfqpoint{8.593599in}{6.426165in}}%
\pgfpathlineto{\pgfqpoint{8.625731in}{6.418938in}}%
\pgfpathlineto{\pgfqpoint{8.657862in}{6.411011in}}%
\pgfpathlineto{\pgfqpoint{8.689994in}{6.402382in}}%
\pgfpathlineto{\pgfqpoint{8.722126in}{6.393053in}}%
\pgfpathlineto{\pgfqpoint{8.754257in}{6.383020in}}%
\pgfpathlineto{\pgfqpoint{8.786389in}{6.372284in}}%
\pgfpathlineto{\pgfqpoint{8.818520in}{6.360844in}}%
\pgfpathlineto{\pgfqpoint{8.850652in}{6.348697in}}%
\pgfpathlineto{\pgfqpoint{8.882784in}{6.335844in}}%
\pgfpathlineto{\pgfqpoint{8.918932in}{6.320539in}}%
\pgfpathlineto{\pgfqpoint{8.955080in}{6.304335in}}%
\pgfpathlineto{\pgfqpoint{8.991228in}{6.287231in}}%
\pgfpathlineto{\pgfqpoint{9.027376in}{6.269225in}}%
\pgfpathlineto{\pgfqpoint{9.063524in}{6.250315in}}%
\pgfpathlineto{\pgfqpoint{9.099672in}{6.230497in}}%
\pgfpathlineto{\pgfqpoint{9.135820in}{6.209771in}}%
\pgfpathlineto{\pgfqpoint{9.171968in}{6.188132in}}%
\pgfpathlineto{\pgfqpoint{9.208116in}{6.165578in}}%
\pgfpathlineto{\pgfqpoint{9.244264in}{6.142107in}}%
\pgfpathlineto{\pgfqpoint{9.280412in}{6.117714in}}%
\pgfpathlineto{\pgfqpoint{9.316560in}{6.092398in}}%
\pgfpathlineto{\pgfqpoint{9.352708in}{6.066155in}}%
\pgfpathlineto{\pgfqpoint{9.388856in}{6.038980in}}%
\pgfpathlineto{\pgfqpoint{9.425004in}{6.010872in}}%
\pgfpathlineto{\pgfqpoint{9.461152in}{5.981826in}}%
\pgfpathlineto{\pgfqpoint{9.497300in}{5.951838in}}%
\pgfpathlineto{\pgfqpoint{9.533448in}{5.920906in}}%
\pgfpathlineto{\pgfqpoint{9.569596in}{5.889024in}}%
\pgfpathlineto{\pgfqpoint{9.605744in}{5.856189in}}%
\pgfpathlineto{\pgfqpoint{9.641892in}{5.822397in}}%
\pgfpathlineto{\pgfqpoint{9.678040in}{5.787644in}}%
\pgfpathlineto{\pgfqpoint{9.714188in}{5.751925in}}%
\pgfpathlineto{\pgfqpoint{9.750336in}{5.715237in}}%
\pgfpathlineto{\pgfqpoint{9.786484in}{5.677575in}}%
\pgfpathlineto{\pgfqpoint{9.822632in}{5.638935in}}%
\pgfpathlineto{\pgfqpoint{9.858780in}{5.599313in}}%
\pgfpathlineto{\pgfqpoint{9.898945in}{5.554130in}}%
\pgfpathlineto{\pgfqpoint{9.939109in}{5.507723in}}%
\pgfpathlineto{\pgfqpoint{9.979274in}{5.460085in}}%
\pgfpathlineto{\pgfqpoint{10.019438in}{5.411210in}}%
\pgfpathlineto{\pgfqpoint{10.059603in}{5.361093in}}%
\pgfpathlineto{\pgfqpoint{10.099767in}{5.309726in}}%
\pgfpathlineto{\pgfqpoint{10.139932in}{5.257104in}}%
\pgfpathlineto{\pgfqpoint{10.159331in}{5.231236in}}%
\pgfpathlineto{\pgfqpoint{10.159331in}{5.231236in}}%
\pgfusepath{stroke}%
\end{pgfscope}%
\begin{pgfscope}%
\pgfpathrectangle{\pgfqpoint{5.697941in}{4.549747in}}{\pgfqpoint{4.458056in}{3.401160in}} %
\pgfusepath{clip}%
\pgfsetbuttcap%
\pgfsetmiterjoin%
\definecolor{currentfill}{rgb}{1.000000,0.000000,0.000000}%
\pgfsetfillcolor{currentfill}%
\pgfsetlinewidth{1.003750pt}%
\definecolor{currentstroke}{rgb}{1.000000,0.000000,0.000000}%
\pgfsetstrokecolor{currentstroke}%
\pgfsetdash{}{0pt}%
\pgfsys@defobject{currentmarker}{\pgfqpoint{-0.041667in}{-0.041667in}}{\pgfqpoint{0.041667in}{0.041667in}}{%
\pgfpathmoveto{\pgfqpoint{-0.041667in}{-0.041667in}}%
\pgfpathlineto{\pgfqpoint{0.041667in}{-0.041667in}}%
\pgfpathlineto{\pgfqpoint{0.041667in}{0.041667in}}%
\pgfpathlineto{\pgfqpoint{-0.041667in}{0.041667in}}%
\pgfpathclose%
\pgfusepath{stroke,fill}%
}%
\begin{pgfscope}%
\pgfsys@transformshift{5.938727in}{4.549747in}%
\pgfsys@useobject{currentmarker}{}%
\end{pgfscope}%
\begin{pgfscope}%
\pgfsys@transformshift{6.340372in}{5.163972in}%
\pgfsys@useobject{currentmarker}{}%
\end{pgfscope}%
\begin{pgfscope}%
\pgfsys@transformshift{6.742017in}{5.652625in}%
\pgfsys@useobject{currentmarker}{}%
\end{pgfscope}%
\begin{pgfscope}%
\pgfsys@transformshift{7.143662in}{6.022668in}%
\pgfsys@useobject{currentmarker}{}%
\end{pgfscope}%
\begin{pgfscope}%
\pgfsys@transformshift{7.545306in}{6.278577in}%
\pgfsys@useobject{currentmarker}{}%
\end{pgfscope}%
\begin{pgfscope}%
\pgfsys@transformshift{7.946951in}{6.423222in}%
\pgfsys@useobject{currentmarker}{}%
\end{pgfscope}%
\begin{pgfscope}%
\pgfsys@transformshift{8.348596in}{6.458297in}%
\pgfsys@useobject{currentmarker}{}%
\end{pgfscope}%
\begin{pgfscope}%
\pgfsys@transformshift{8.750241in}{6.384313in}%
\pgfsys@useobject{currentmarker}{}%
\end{pgfscope}%
\begin{pgfscope}%
\pgfsys@transformshift{9.151886in}{6.200266in}%
\pgfsys@useobject{currentmarker}{}%
\end{pgfscope}%
\begin{pgfscope}%
\pgfsys@transformshift{9.553530in}{5.903311in}%
\pgfsys@useobject{currentmarker}{}%
\end{pgfscope}%
\begin{pgfscope}%
\pgfsys@transformshift{9.955175in}{5.488815in}%
\pgfsys@useobject{currentmarker}{}%
\end{pgfscope}%
\begin{pgfscope}%
\pgfsys@transformshift{10.356820in}{4.950985in}%
\pgfsys@useobject{currentmarker}{}%
\end{pgfscope}%
\end{pgfscope}%
\begin{pgfscope}%
\pgfpathrectangle{\pgfqpoint{5.697941in}{4.549747in}}{\pgfqpoint{4.458056in}{3.401160in}} %
\pgfusepath{clip}%
\pgfsetrectcap%
\pgfsetroundjoin%
\pgfsetlinewidth{1.505625pt}%
\definecolor{currentstroke}{rgb}{0.000000,0.000000,1.000000}%
\pgfsetstrokecolor{currentstroke}%
\pgfsetdash{}{0pt}%
\pgfpathmoveto{\pgfqpoint{5.938727in}{4.549747in}}%
\pgfpathlineto{\pgfqpoint{5.978892in}{4.617091in}}%
\pgfpathlineto{\pgfqpoint{6.019056in}{4.683075in}}%
\pgfpathlineto{\pgfqpoint{6.059221in}{4.747701in}}%
\pgfpathlineto{\pgfqpoint{6.099385in}{4.810971in}}%
\pgfpathlineto{\pgfqpoint{6.139550in}{4.872885in}}%
\pgfpathlineto{\pgfqpoint{6.179714in}{4.933446in}}%
\pgfpathlineto{\pgfqpoint{6.219879in}{4.992653in}}%
\pgfpathlineto{\pgfqpoint{6.260043in}{5.050509in}}%
\pgfpathlineto{\pgfqpoint{6.300207in}{5.107014in}}%
\pgfpathlineto{\pgfqpoint{6.340372in}{5.162170in}}%
\pgfpathlineto{\pgfqpoint{6.380536in}{5.215978in}}%
\pgfpathlineto{\pgfqpoint{6.420701in}{5.268438in}}%
\pgfpathlineto{\pgfqpoint{6.456849in}{5.314501in}}%
\pgfpathlineto{\pgfqpoint{6.492997in}{5.359474in}}%
\pgfpathlineto{\pgfqpoint{6.529145in}{5.403358in}}%
\pgfpathlineto{\pgfqpoint{6.565293in}{5.446154in}}%
\pgfpathlineto{\pgfqpoint{6.601441in}{5.487862in}}%
\pgfpathlineto{\pgfqpoint{6.637589in}{5.528483in}}%
\pgfpathlineto{\pgfqpoint{6.673737in}{5.568017in}}%
\pgfpathlineto{\pgfqpoint{6.709885in}{5.606466in}}%
\pgfpathlineto{\pgfqpoint{6.746033in}{5.643829in}}%
\pgfpathlineto{\pgfqpoint{6.782181in}{5.680108in}}%
\pgfpathlineto{\pgfqpoint{6.818329in}{5.715302in}}%
\pgfpathlineto{\pgfqpoint{6.854477in}{5.749413in}}%
\pgfpathlineto{\pgfqpoint{6.890625in}{5.782441in}}%
\pgfpathlineto{\pgfqpoint{6.926773in}{5.814386in}}%
\pgfpathlineto{\pgfqpoint{6.962921in}{5.845250in}}%
\pgfpathlineto{\pgfqpoint{6.999069in}{5.875031in}}%
\pgfpathlineto{\pgfqpoint{7.031201in}{5.900597in}}%
\pgfpathlineto{\pgfqpoint{7.063333in}{5.925308in}}%
\pgfpathlineto{\pgfqpoint{7.095464in}{5.949165in}}%
\pgfpathlineto{\pgfqpoint{7.127596in}{5.972169in}}%
\pgfpathlineto{\pgfqpoint{7.159727in}{5.994321in}}%
\pgfpathlineto{\pgfqpoint{7.191859in}{6.015619in}}%
\pgfpathlineto{\pgfqpoint{7.223991in}{6.036065in}}%
\pgfpathlineto{\pgfqpoint{7.256122in}{6.055658in}}%
\pgfpathlineto{\pgfqpoint{7.288254in}{6.074400in}}%
\pgfpathlineto{\pgfqpoint{7.320385in}{6.092290in}}%
\pgfpathlineto{\pgfqpoint{7.352517in}{6.109328in}}%
\pgfpathlineto{\pgfqpoint{7.384648in}{6.125515in}}%
\pgfpathlineto{\pgfqpoint{7.416780in}{6.140850in}}%
\pgfpathlineto{\pgfqpoint{7.448912in}{6.155335in}}%
\pgfpathlineto{\pgfqpoint{7.481043in}{6.168968in}}%
\pgfpathlineto{\pgfqpoint{7.513175in}{6.181751in}}%
\pgfpathlineto{\pgfqpoint{7.545306in}{6.193684in}}%
\pgfpathlineto{\pgfqpoint{7.577438in}{6.204766in}}%
\pgfpathlineto{\pgfqpoint{7.609570in}{6.214997in}}%
\pgfpathlineto{\pgfqpoint{7.641701in}{6.224379in}}%
\pgfpathlineto{\pgfqpoint{7.673833in}{6.232911in}}%
\pgfpathlineto{\pgfqpoint{7.705964in}{6.240592in}}%
\pgfpathlineto{\pgfqpoint{7.738096in}{6.247424in}}%
\pgfpathlineto{\pgfqpoint{7.770227in}{6.253406in}}%
\pgfpathlineto{\pgfqpoint{7.802359in}{6.258538in}}%
\pgfpathlineto{\pgfqpoint{7.834491in}{6.262821in}}%
\pgfpathlineto{\pgfqpoint{7.866622in}{6.266254in}}%
\pgfpathlineto{\pgfqpoint{7.898754in}{6.268838in}}%
\pgfpathlineto{\pgfqpoint{7.926869in}{6.270402in}}%
\pgfpathlineto{\pgfqpoint{7.954984in}{6.271315in}}%
\pgfpathlineto{\pgfqpoint{7.983099in}{6.271578in}}%
\pgfpathlineto{\pgfqpoint{8.011214in}{6.271191in}}%
\pgfpathlineto{\pgfqpoint{8.039329in}{6.270153in}}%
\pgfpathlineto{\pgfqpoint{8.067445in}{6.268465in}}%
\pgfpathlineto{\pgfqpoint{8.095560in}{6.266127in}}%
\pgfpathlineto{\pgfqpoint{8.127691in}{6.262658in}}%
\pgfpathlineto{\pgfqpoint{8.159823in}{6.258339in}}%
\pgfpathlineto{\pgfqpoint{8.191954in}{6.253171in}}%
\pgfpathlineto{\pgfqpoint{8.224086in}{6.247153in}}%
\pgfpathlineto{\pgfqpoint{8.256218in}{6.240286in}}%
\pgfpathlineto{\pgfqpoint{8.288349in}{6.232568in}}%
\pgfpathlineto{\pgfqpoint{8.320481in}{6.224001in}}%
\pgfpathlineto{\pgfqpoint{8.352612in}{6.214584in}}%
\pgfpathlineto{\pgfqpoint{8.384744in}{6.204316in}}%
\pgfpathlineto{\pgfqpoint{8.416876in}{6.193198in}}%
\pgfpathlineto{\pgfqpoint{8.449007in}{6.181230in}}%
\pgfpathlineto{\pgfqpoint{8.481139in}{6.168411in}}%
\pgfpathlineto{\pgfqpoint{8.513270in}{6.154742in}}%
\pgfpathlineto{\pgfqpoint{8.545402in}{6.140221in}}%
\pgfpathlineto{\pgfqpoint{8.577533in}{6.124850in}}%
\pgfpathlineto{\pgfqpoint{8.609665in}{6.108628in}}%
\pgfpathlineto{\pgfqpoint{8.641797in}{6.091554in}}%
\pgfpathlineto{\pgfqpoint{8.673928in}{6.073628in}}%
\pgfpathlineto{\pgfqpoint{8.706060in}{6.054851in}}%
\pgfpathlineto{\pgfqpoint{8.738191in}{6.035221in}}%
\pgfpathlineto{\pgfqpoint{8.770323in}{6.014739in}}%
\pgfpathlineto{\pgfqpoint{8.802455in}{5.993405in}}%
\pgfpathlineto{\pgfqpoint{8.834586in}{5.971218in}}%
\pgfpathlineto{\pgfqpoint{8.866718in}{5.948178in}}%
\pgfpathlineto{\pgfqpoint{8.898849in}{5.924284in}}%
\pgfpathlineto{\pgfqpoint{8.930981in}{5.899537in}}%
\pgfpathlineto{\pgfqpoint{8.963112in}{5.873936in}}%
\pgfpathlineto{\pgfqpoint{8.995244in}{5.847481in}}%
\pgfpathlineto{\pgfqpoint{9.031392in}{5.816697in}}%
\pgfpathlineto{\pgfqpoint{9.067540in}{5.784832in}}%
\pgfpathlineto{\pgfqpoint{9.103688in}{5.751883in}}%
\pgfpathlineto{\pgfqpoint{9.139836in}{5.717852in}}%
\pgfpathlineto{\pgfqpoint{9.175984in}{5.682738in}}%
\pgfpathlineto{\pgfqpoint{9.212132in}{5.646539in}}%
\pgfpathlineto{\pgfqpoint{9.248280in}{5.609255in}}%
\pgfpathlineto{\pgfqpoint{9.284428in}{5.570887in}}%
\pgfpathlineto{\pgfqpoint{9.320576in}{5.531432in}}%
\pgfpathlineto{\pgfqpoint{9.356724in}{5.490892in}}%
\pgfpathlineto{\pgfqpoint{9.392872in}{5.449264in}}%
\pgfpathlineto{\pgfqpoint{9.429020in}{5.406548in}}%
\pgfpathlineto{\pgfqpoint{9.465168in}{5.362744in}}%
\pgfpathlineto{\pgfqpoint{9.501317in}{5.317851in}}%
\pgfpathlineto{\pgfqpoint{9.537465in}{5.271869in}}%
\pgfpathlineto{\pgfqpoint{9.573613in}{5.224796in}}%
\pgfpathlineto{\pgfqpoint{9.613777in}{5.171212in}}%
\pgfpathlineto{\pgfqpoint{9.653942in}{5.116281in}}%
\pgfpathlineto{\pgfqpoint{9.694106in}{5.060000in}}%
\pgfpathlineto{\pgfqpoint{9.734270in}{5.002369in}}%
\pgfpathlineto{\pgfqpoint{9.774435in}{4.943386in}}%
\pgfpathlineto{\pgfqpoint{9.814599in}{4.883051in}}%
\pgfpathlineto{\pgfqpoint{9.854764in}{4.821362in}}%
\pgfpathlineto{\pgfqpoint{9.894928in}{4.758317in}}%
\pgfpathlineto{\pgfqpoint{9.935093in}{4.693917in}}%
\pgfpathlineto{\pgfqpoint{9.975257in}{4.628158in}}%
\pgfpathlineto{\pgfqpoint{10.015422in}{4.561041in}}%
\pgfpathlineto{\pgfqpoint{10.024068in}{4.546414in}}%
\pgfpathlineto{\pgfqpoint{10.024068in}{4.546414in}}%
\pgfusepath{stroke}%
\end{pgfscope}%
\begin{pgfscope}%
\pgfpathrectangle{\pgfqpoint{5.697941in}{4.549747in}}{\pgfqpoint{4.458056in}{3.401160in}} %
\pgfusepath{clip}%
\pgfsetbuttcap%
\pgfsetroundjoin%
\definecolor{currentfill}{rgb}{0.000000,0.000000,1.000000}%
\pgfsetfillcolor{currentfill}%
\pgfsetlinewidth{1.003750pt}%
\definecolor{currentstroke}{rgb}{0.000000,0.000000,1.000000}%
\pgfsetstrokecolor{currentstroke}%
\pgfsetdash{}{0pt}%
\pgfsys@defobject{currentmarker}{\pgfqpoint{-0.041667in}{-0.041667in}}{\pgfqpoint{0.041667in}{0.041667in}}{%
\pgfpathmoveto{\pgfqpoint{0.000000in}{-0.041667in}}%
\pgfpathcurveto{\pgfqpoint{0.011050in}{-0.041667in}}{\pgfqpoint{0.021649in}{-0.037276in}}{\pgfqpoint{0.029463in}{-0.029463in}}%
\pgfpathcurveto{\pgfqpoint{0.037276in}{-0.021649in}}{\pgfqpoint{0.041667in}{-0.011050in}}{\pgfqpoint{0.041667in}{0.000000in}}%
\pgfpathcurveto{\pgfqpoint{0.041667in}{0.011050in}}{\pgfqpoint{0.037276in}{0.021649in}}{\pgfqpoint{0.029463in}{0.029463in}}%
\pgfpathcurveto{\pgfqpoint{0.021649in}{0.037276in}}{\pgfqpoint{0.011050in}{0.041667in}}{\pgfqpoint{0.000000in}{0.041667in}}%
\pgfpathcurveto{\pgfqpoint{-0.011050in}{0.041667in}}{\pgfqpoint{-0.021649in}{0.037276in}}{\pgfqpoint{-0.029463in}{0.029463in}}%
\pgfpathcurveto{\pgfqpoint{-0.037276in}{0.021649in}}{\pgfqpoint{-0.041667in}{0.011050in}}{\pgfqpoint{-0.041667in}{0.000000in}}%
\pgfpathcurveto{\pgfqpoint{-0.041667in}{-0.011050in}}{\pgfqpoint{-0.037276in}{-0.021649in}}{\pgfqpoint{-0.029463in}{-0.029463in}}%
\pgfpathcurveto{\pgfqpoint{-0.021649in}{-0.037276in}}{\pgfqpoint{-0.011050in}{-0.041667in}}{\pgfqpoint{0.000000in}{-0.041667in}}%
\pgfpathclose%
\pgfusepath{stroke,fill}%
}%
\begin{pgfscope}%
\pgfsys@transformshift{5.938727in}{4.549747in}%
\pgfsys@useobject{currentmarker}{}%
\end{pgfscope}%
\begin{pgfscope}%
\pgfsys@transformshift{6.340372in}{5.162170in}%
\pgfsys@useobject{currentmarker}{}%
\end{pgfscope}%
\begin{pgfscope}%
\pgfsys@transformshift{6.742017in}{5.639731in}%
\pgfsys@useobject{currentmarker}{}%
\end{pgfscope}%
\begin{pgfscope}%
\pgfsys@transformshift{7.143662in}{5.983352in}%
\pgfsys@useobject{currentmarker}{}%
\end{pgfscope}%
\begin{pgfscope}%
\pgfsys@transformshift{7.545306in}{6.193684in}%
\pgfsys@useobject{currentmarker}{}%
\end{pgfscope}%
\begin{pgfscope}%
\pgfsys@transformshift{7.946951in}{6.271121in}%
\pgfsys@useobject{currentmarker}{}%
\end{pgfscope}%
\begin{pgfscope}%
\pgfsys@transformshift{8.348596in}{6.215807in}%
\pgfsys@useobject{currentmarker}{}%
\end{pgfscope}%
\begin{pgfscope}%
\pgfsys@transformshift{8.750241in}{6.027640in}%
\pgfsys@useobject{currentmarker}{}%
\end{pgfscope}%
\begin{pgfscope}%
\pgfsys@transformshift{9.151886in}{5.706268in}%
\pgfsys@useobject{currentmarker}{}%
\end{pgfscope}%
\begin{pgfscope}%
\pgfsys@transformshift{9.553530in}{5.251082in}%
\pgfsys@useobject{currentmarker}{}%
\end{pgfscope}%
\begin{pgfscope}%
\pgfsys@transformshift{9.955175in}{4.661207in}%
\pgfsys@useobject{currentmarker}{}%
\end{pgfscope}%
\begin{pgfscope}%
\pgfsys@transformshift{10.356820in}{3.935483in}%
\pgfsys@useobject{currentmarker}{}%
\end{pgfscope}%
\end{pgfscope}%
\begin{pgfscope}%
\pgfpathrectangle{\pgfqpoint{5.697941in}{4.549747in}}{\pgfqpoint{4.458056in}{3.401160in}} %
\pgfusepath{clip}%
\pgfsetbuttcap%
\pgfsetroundjoin%
\pgfsetlinewidth{1.505625pt}%
\definecolor{currentstroke}{rgb}{0.000000,0.750000,0.750000}%
\pgfsetstrokecolor{currentstroke}%
\pgfsetdash{{9.600000pt}{2.400000pt}{1.500000pt}{2.400000pt}}{0.000000pt}%
\pgfpathmoveto{\pgfqpoint{5.938727in}{4.549747in}}%
\pgfpathlineto{\pgfqpoint{5.978892in}{4.617090in}}%
\pgfpathlineto{\pgfqpoint{6.019056in}{4.683073in}}%
\pgfpathlineto{\pgfqpoint{6.059221in}{4.747695in}}%
\pgfpathlineto{\pgfqpoint{6.099385in}{4.810958in}}%
\pgfpathlineto{\pgfqpoint{6.139550in}{4.872860in}}%
\pgfpathlineto{\pgfqpoint{6.179714in}{4.933403in}}%
\pgfpathlineto{\pgfqpoint{6.219879in}{4.992586in}}%
\pgfpathlineto{\pgfqpoint{6.260043in}{5.050409in}}%
\pgfpathlineto{\pgfqpoint{6.300207in}{5.106873in}}%
\pgfpathlineto{\pgfqpoint{6.340372in}{5.161978in}}%
\pgfpathlineto{\pgfqpoint{6.380536in}{5.215723in}}%
\pgfpathlineto{\pgfqpoint{6.420701in}{5.268109in}}%
\pgfpathlineto{\pgfqpoint{6.456849in}{5.314095in}}%
\pgfpathlineto{\pgfqpoint{6.492997in}{5.358980in}}%
\pgfpathlineto{\pgfqpoint{6.529145in}{5.402764in}}%
\pgfpathlineto{\pgfqpoint{6.565293in}{5.445447in}}%
\pgfpathlineto{\pgfqpoint{6.601441in}{5.487030in}}%
\pgfpathlineto{\pgfqpoint{6.637589in}{5.527513in}}%
\pgfpathlineto{\pgfqpoint{6.673737in}{5.566895in}}%
\pgfpathlineto{\pgfqpoint{6.709885in}{5.605177in}}%
\pgfpathlineto{\pgfqpoint{6.746033in}{5.642359in}}%
\pgfpathlineto{\pgfqpoint{6.782181in}{5.678440in}}%
\pgfpathlineto{\pgfqpoint{6.818329in}{5.713421in}}%
\pgfpathlineto{\pgfqpoint{6.854477in}{5.747302in}}%
\pgfpathlineto{\pgfqpoint{6.890625in}{5.780083in}}%
\pgfpathlineto{\pgfqpoint{6.926773in}{5.811765in}}%
\pgfpathlineto{\pgfqpoint{6.958905in}{5.839002in}}%
\pgfpathlineto{\pgfqpoint{6.991037in}{5.865371in}}%
\pgfpathlineto{\pgfqpoint{7.023168in}{5.890870in}}%
\pgfpathlineto{\pgfqpoint{7.055300in}{5.915501in}}%
\pgfpathlineto{\pgfqpoint{7.087431in}{5.939262in}}%
\pgfpathlineto{\pgfqpoint{7.119563in}{5.962155in}}%
\pgfpathlineto{\pgfqpoint{7.151694in}{5.984179in}}%
\pgfpathlineto{\pgfqpoint{7.183826in}{6.005333in}}%
\pgfpathlineto{\pgfqpoint{7.215958in}{6.025620in}}%
\pgfpathlineto{\pgfqpoint{7.248089in}{6.045037in}}%
\pgfpathlineto{\pgfqpoint{7.280221in}{6.063585in}}%
\pgfpathlineto{\pgfqpoint{7.312352in}{6.081265in}}%
\pgfpathlineto{\pgfqpoint{7.344484in}{6.098076in}}%
\pgfpathlineto{\pgfqpoint{7.376616in}{6.114018in}}%
\pgfpathlineto{\pgfqpoint{7.408747in}{6.129091in}}%
\pgfpathlineto{\pgfqpoint{7.440879in}{6.143296in}}%
\pgfpathlineto{\pgfqpoint{7.473010in}{6.156632in}}%
\pgfpathlineto{\pgfqpoint{7.505142in}{6.169100in}}%
\pgfpathlineto{\pgfqpoint{7.537273in}{6.180699in}}%
\pgfpathlineto{\pgfqpoint{7.569405in}{6.191429in}}%
\pgfpathlineto{\pgfqpoint{7.601537in}{6.201291in}}%
\pgfpathlineto{\pgfqpoint{7.633668in}{6.210284in}}%
\pgfpathlineto{\pgfqpoint{7.665800in}{6.218408in}}%
\pgfpathlineto{\pgfqpoint{7.697931in}{6.225664in}}%
\pgfpathlineto{\pgfqpoint{7.726047in}{6.231301in}}%
\pgfpathlineto{\pgfqpoint{7.754162in}{6.236272in}}%
\pgfpathlineto{\pgfqpoint{7.782277in}{6.240579in}}%
\pgfpathlineto{\pgfqpoint{7.810392in}{6.244220in}}%
\pgfpathlineto{\pgfqpoint{7.838507in}{6.247197in}}%
\pgfpathlineto{\pgfqpoint{7.866622in}{6.249508in}}%
\pgfpathlineto{\pgfqpoint{7.894737in}{6.251155in}}%
\pgfpathlineto{\pgfqpoint{7.922852in}{6.252136in}}%
\pgfpathlineto{\pgfqpoint{7.950968in}{6.252453in}}%
\pgfpathlineto{\pgfqpoint{7.979083in}{6.252105in}}%
\pgfpathlineto{\pgfqpoint{8.007198in}{6.251091in}}%
\pgfpathlineto{\pgfqpoint{8.035313in}{6.249413in}}%
\pgfpathlineto{\pgfqpoint{8.063428in}{6.247070in}}%
\pgfpathlineto{\pgfqpoint{8.091543in}{6.244062in}}%
\pgfpathlineto{\pgfqpoint{8.119658in}{6.240389in}}%
\pgfpathlineto{\pgfqpoint{8.147774in}{6.236050in}}%
\pgfpathlineto{\pgfqpoint{8.175889in}{6.231047in}}%
\pgfpathlineto{\pgfqpoint{8.204004in}{6.225379in}}%
\pgfpathlineto{\pgfqpoint{8.232119in}{6.219046in}}%
\pgfpathlineto{\pgfqpoint{8.264251in}{6.210994in}}%
\pgfpathlineto{\pgfqpoint{8.296382in}{6.202073in}}%
\pgfpathlineto{\pgfqpoint{8.328514in}{6.192284in}}%
\pgfpathlineto{\pgfqpoint{8.360645in}{6.181626in}}%
\pgfpathlineto{\pgfqpoint{8.392777in}{6.170100in}}%
\pgfpathlineto{\pgfqpoint{8.424908in}{6.157705in}}%
\pgfpathlineto{\pgfqpoint{8.457040in}{6.144441in}}%
\pgfpathlineto{\pgfqpoint{8.489172in}{6.130308in}}%
\pgfpathlineto{\pgfqpoint{8.521303in}{6.115307in}}%
\pgfpathlineto{\pgfqpoint{8.553435in}{6.099437in}}%
\pgfpathlineto{\pgfqpoint{8.585566in}{6.082699in}}%
\pgfpathlineto{\pgfqpoint{8.617698in}{6.065092in}}%
\pgfpathlineto{\pgfqpoint{8.649830in}{6.046616in}}%
\pgfpathlineto{\pgfqpoint{8.681961in}{6.027271in}}%
\pgfpathlineto{\pgfqpoint{8.714093in}{6.007057in}}%
\pgfpathlineto{\pgfqpoint{8.746224in}{5.985975in}}%
\pgfpathlineto{\pgfqpoint{8.778356in}{5.964023in}}%
\pgfpathlineto{\pgfqpoint{8.810487in}{5.941203in}}%
\pgfpathlineto{\pgfqpoint{8.842619in}{5.917514in}}%
\pgfpathlineto{\pgfqpoint{8.874751in}{5.892956in}}%
\pgfpathlineto{\pgfqpoint{8.906882in}{5.867529in}}%
\pgfpathlineto{\pgfqpoint{8.939014in}{5.841233in}}%
\pgfpathlineto{\pgfqpoint{8.971145in}{5.814068in}}%
\pgfpathlineto{\pgfqpoint{9.003277in}{5.786033in}}%
\pgfpathlineto{\pgfqpoint{9.039425in}{5.753456in}}%
\pgfpathlineto{\pgfqpoint{9.075573in}{5.719778in}}%
\pgfpathlineto{\pgfqpoint{9.111721in}{5.685001in}}%
\pgfpathlineto{\pgfqpoint{9.147869in}{5.649123in}}%
\pgfpathlineto{\pgfqpoint{9.184017in}{5.612145in}}%
\pgfpathlineto{\pgfqpoint{9.220165in}{5.574067in}}%
\pgfpathlineto{\pgfqpoint{9.256313in}{5.534889in}}%
\pgfpathlineto{\pgfqpoint{9.292461in}{5.494610in}}%
\pgfpathlineto{\pgfqpoint{9.328609in}{5.453231in}}%
\pgfpathlineto{\pgfqpoint{9.364757in}{5.410751in}}%
\pgfpathlineto{\pgfqpoint{9.400905in}{5.367171in}}%
\pgfpathlineto{\pgfqpoint{9.437053in}{5.322490in}}%
\pgfpathlineto{\pgfqpoint{9.473201in}{5.276708in}}%
\pgfpathlineto{\pgfqpoint{9.509349in}{5.229826in}}%
\pgfpathlineto{\pgfqpoint{9.545497in}{5.181842in}}%
\pgfpathlineto{\pgfqpoint{9.585662in}{5.127236in}}%
\pgfpathlineto{\pgfqpoint{9.625826in}{5.071271in}}%
\pgfpathlineto{\pgfqpoint{9.665991in}{5.013946in}}%
\pgfpathlineto{\pgfqpoint{9.706155in}{4.955261in}}%
\pgfpathlineto{\pgfqpoint{9.746320in}{4.895217in}}%
\pgfpathlineto{\pgfqpoint{9.786484in}{4.833813in}}%
\pgfpathlineto{\pgfqpoint{9.826649in}{4.771050in}}%
\pgfpathlineto{\pgfqpoint{9.866813in}{4.706926in}}%
\pgfpathlineto{\pgfqpoint{9.906978in}{4.641442in}}%
\pgfpathlineto{\pgfqpoint{9.947142in}{4.574598in}}%
\pgfpathlineto{\pgfqpoint{9.963836in}{4.546414in}}%
\pgfpathlineto{\pgfqpoint{9.963836in}{4.546414in}}%
\pgfusepath{stroke}%
\end{pgfscope}%
\begin{pgfscope}%
\pgfpathrectangle{\pgfqpoint{5.697941in}{4.549747in}}{\pgfqpoint{4.458056in}{3.401160in}} %
\pgfusepath{clip}%
\pgfsetbuttcap%
\pgfsetmiterjoin%
\definecolor{currentfill}{rgb}{0.000000,0.750000,0.750000}%
\pgfsetfillcolor{currentfill}%
\pgfsetlinewidth{1.003750pt}%
\definecolor{currentstroke}{rgb}{0.000000,0.750000,0.750000}%
\pgfsetstrokecolor{currentstroke}%
\pgfsetdash{}{0pt}%
\pgfsys@defobject{currentmarker}{\pgfqpoint{-0.041667in}{-0.041667in}}{\pgfqpoint{0.041667in}{0.041667in}}{%
\pgfpathmoveto{\pgfqpoint{-0.000000in}{-0.041667in}}%
\pgfpathlineto{\pgfqpoint{0.041667in}{0.041667in}}%
\pgfpathlineto{\pgfqpoint{-0.041667in}{0.041667in}}%
\pgfpathclose%
\pgfusepath{stroke,fill}%
}%
\begin{pgfscope}%
\pgfsys@transformshift{5.938727in}{4.549747in}%
\pgfsys@useobject{currentmarker}{}%
\end{pgfscope}%
\begin{pgfscope}%
\pgfsys@transformshift{6.340372in}{5.161978in}%
\pgfsys@useobject{currentmarker}{}%
\end{pgfscope}%
\begin{pgfscope}%
\pgfsys@transformshift{6.742017in}{5.638282in}%
\pgfsys@useobject{currentmarker}{}%
\end{pgfscope}%
\begin{pgfscope}%
\pgfsys@transformshift{7.143662in}{5.978754in}%
\pgfsys@useobject{currentmarker}{}%
\end{pgfscope}%
\begin{pgfscope}%
\pgfsys@transformshift{7.545306in}{6.183463in}%
\pgfsys@useobject{currentmarker}{}%
\end{pgfscope}%
\begin{pgfscope}%
\pgfsys@transformshift{7.946951in}{6.252448in}%
\pgfsys@useobject{currentmarker}{}%
\end{pgfscope}%
\begin{pgfscope}%
\pgfsys@transformshift{8.348596in}{6.185725in}%
\pgfsys@useobject{currentmarker}{}%
\end{pgfscope}%
\begin{pgfscope}%
\pgfsys@transformshift{8.750241in}{5.983278in}%
\pgfsys@useobject{currentmarker}{}%
\end{pgfscope}%
\begin{pgfscope}%
\pgfsys@transformshift{9.151886in}{5.645069in}%
\pgfsys@useobject{currentmarker}{}%
\end{pgfscope}%
\begin{pgfscope}%
\pgfsys@transformshift{9.553530in}{5.171030in}%
\pgfsys@useobject{currentmarker}{}%
\end{pgfscope}%
\begin{pgfscope}%
\pgfsys@transformshift{9.955175in}{4.561066in}%
\pgfsys@useobject{currentmarker}{}%
\end{pgfscope}%
\begin{pgfscope}%
\pgfsys@transformshift{10.356820in}{3.815055in}%
\pgfsys@useobject{currentmarker}{}%
\end{pgfscope}%
\end{pgfscope}%
\begin{pgfscope}%
\pgfsetrectcap%
\pgfsetmiterjoin%
\pgfsetlinewidth{0.803000pt}%
\definecolor{currentstroke}{rgb}{0.000000,0.000000,0.000000}%
\pgfsetstrokecolor{currentstroke}%
\pgfsetdash{}{0pt}%
\pgfpathmoveto{\pgfqpoint{5.697941in}{4.549747in}}%
\pgfpathlineto{\pgfqpoint{5.697941in}{7.950908in}}%
\pgfusepath{stroke}%
\end{pgfscope}%
\begin{pgfscope}%
\pgfsetrectcap%
\pgfsetmiterjoin%
\pgfsetlinewidth{0.803000pt}%
\definecolor{currentstroke}{rgb}{0.000000,0.000000,0.000000}%
\pgfsetstrokecolor{currentstroke}%
\pgfsetdash{}{0pt}%
\pgfpathmoveto{\pgfqpoint{10.155998in}{4.549747in}}%
\pgfpathlineto{\pgfqpoint{10.155998in}{7.950908in}}%
\pgfusepath{stroke}%
\end{pgfscope}%
\begin{pgfscope}%
\pgfsetrectcap%
\pgfsetmiterjoin%
\pgfsetlinewidth{0.803000pt}%
\definecolor{currentstroke}{rgb}{0.000000,0.000000,0.000000}%
\pgfsetstrokecolor{currentstroke}%
\pgfsetdash{}{0pt}%
\pgfpathmoveto{\pgfqpoint{5.697941in}{4.549747in}}%
\pgfpathlineto{\pgfqpoint{10.155998in}{4.549747in}}%
\pgfusepath{stroke}%
\end{pgfscope}%
\begin{pgfscope}%
\pgfsetrectcap%
\pgfsetmiterjoin%
\pgfsetlinewidth{0.803000pt}%
\definecolor{currentstroke}{rgb}{0.000000,0.000000,0.000000}%
\pgfsetstrokecolor{currentstroke}%
\pgfsetdash{}{0pt}%
\pgfpathmoveto{\pgfqpoint{5.697941in}{7.950908in}}%
\pgfpathlineto{\pgfqpoint{10.155998in}{7.950908in}}%
\pgfusepath{stroke}%
\end{pgfscope}%
\begin{pgfscope}%
\pgfsetbuttcap%
\pgfsetmiterjoin%
\definecolor{currentfill}{rgb}{1.000000,1.000000,1.000000}%
\pgfsetfillcolor{currentfill}%
\pgfsetfillopacity{0.800000}%
\pgfsetlinewidth{1.003750pt}%
\definecolor{currentstroke}{rgb}{0.800000,0.800000,0.800000}%
\pgfsetstrokecolor{currentstroke}%
\pgfsetstrokeopacity{0.800000}%
\pgfsetdash{}{0pt}%
\pgfpathmoveto{\pgfqpoint{5.834052in}{7.409976in}}%
\pgfpathlineto{\pgfqpoint{6.520644in}{7.409976in}}%
\pgfpathquadraticcurveto{\pgfqpoint{6.559533in}{7.409976in}}{\pgfqpoint{6.559533in}{7.448865in}}%
\pgfpathlineto{\pgfqpoint{6.559533in}{7.814796in}}%
\pgfpathquadraticcurveto{\pgfqpoint{6.559533in}{7.853685in}}{\pgfqpoint{6.520644in}{7.853685in}}%
\pgfpathlineto{\pgfqpoint{5.834052in}{7.853685in}}%
\pgfpathquadraticcurveto{\pgfqpoint{5.795163in}{7.853685in}}{\pgfqpoint{5.795163in}{7.814796in}}%
\pgfpathlineto{\pgfqpoint{5.795163in}{7.448865in}}%
\pgfpathquadraticcurveto{\pgfqpoint{5.795163in}{7.409976in}}{\pgfqpoint{5.834052in}{7.409976in}}%
\pgfpathclose%
\pgfusepath{stroke,fill}%
\end{pgfscope}%
\begin{pgfscope}%
\pgftext[x=5.872941in,y=7.628175in,left,base]{\rmfamily\fontsize{14.000000}{16.800000}\selectfont \(\displaystyle \mathbf{I}\mbox{g} = \) 1}%
\end{pgfscope}%
\begin{pgfscope}%
\pgfsetbuttcap%
\pgfsetmiterjoin%
\definecolor{currentfill}{rgb}{1.000000,1.000000,1.000000}%
\pgfsetfillcolor{currentfill}%
\pgfsetlinewidth{0.000000pt}%
\definecolor{currentstroke}{rgb}{0.000000,0.000000,0.000000}%
\pgfsetstrokecolor{currentstroke}%
\pgfsetstrokeopacity{0.000000}%
\pgfsetdash{}{0pt}%
\pgfpathmoveto{\pgfqpoint{0.984216in}{0.741795in}}%
\pgfpathlineto{\pgfqpoint{5.442272in}{0.741795in}}%
\pgfpathlineto{\pgfqpoint{5.442272in}{4.142955in}}%
\pgfpathlineto{\pgfqpoint{0.984216in}{4.142955in}}%
\pgfpathclose%
\pgfusepath{fill}%
\end{pgfscope}%
\begin{pgfscope}%
\pgfsetbuttcap%
\pgfsetroundjoin%
\definecolor{currentfill}{rgb}{0.000000,0.000000,0.000000}%
\pgfsetfillcolor{currentfill}%
\pgfsetlinewidth{0.803000pt}%
\definecolor{currentstroke}{rgb}{0.000000,0.000000,0.000000}%
\pgfsetstrokecolor{currentstroke}%
\pgfsetdash{}{0pt}%
\pgfsys@defobject{currentmarker}{\pgfqpoint{0.000000in}{-0.048611in}}{\pgfqpoint{0.000000in}{0.000000in}}{%
\pgfpathmoveto{\pgfqpoint{0.000000in}{0.000000in}}%
\pgfpathlineto{\pgfqpoint{0.000000in}{-0.048611in}}%
\pgfusepath{stroke,fill}%
}%
\begin{pgfscope}%
\pgfsys@transformshift{1.225002in}{0.741795in}%
\pgfsys@useobject{currentmarker}{}%
\end{pgfscope}%
\end{pgfscope}%
\begin{pgfscope}%
\pgftext[x=1.225002in,y=0.644572in,,top]{\rmfamily\fontsize{16.000000}{19.200000}\selectfont \(\displaystyle 0.0\)}%
\end{pgfscope}%
\begin{pgfscope}%
\pgfsetbuttcap%
\pgfsetroundjoin%
\definecolor{currentfill}{rgb}{0.000000,0.000000,0.000000}%
\pgfsetfillcolor{currentfill}%
\pgfsetlinewidth{0.803000pt}%
\definecolor{currentstroke}{rgb}{0.000000,0.000000,0.000000}%
\pgfsetstrokecolor{currentstroke}%
\pgfsetdash{}{0pt}%
\pgfsys@defobject{currentmarker}{\pgfqpoint{0.000000in}{-0.048611in}}{\pgfqpoint{0.000000in}{0.000000in}}{%
\pgfpathmoveto{\pgfqpoint{0.000000in}{0.000000in}}%
\pgfpathlineto{\pgfqpoint{0.000000in}{-0.048611in}}%
\pgfusepath{stroke,fill}%
}%
\begin{pgfscope}%
\pgfsys@transformshift{2.028291in}{0.741795in}%
\pgfsys@useobject{currentmarker}{}%
\end{pgfscope}%
\end{pgfscope}%
\begin{pgfscope}%
\pgftext[x=2.028291in,y=0.644572in,,top]{\rmfamily\fontsize{16.000000}{19.200000}\selectfont \(\displaystyle 0.2\)}%
\end{pgfscope}%
\begin{pgfscope}%
\pgfsetbuttcap%
\pgfsetroundjoin%
\definecolor{currentfill}{rgb}{0.000000,0.000000,0.000000}%
\pgfsetfillcolor{currentfill}%
\pgfsetlinewidth{0.803000pt}%
\definecolor{currentstroke}{rgb}{0.000000,0.000000,0.000000}%
\pgfsetstrokecolor{currentstroke}%
\pgfsetdash{}{0pt}%
\pgfsys@defobject{currentmarker}{\pgfqpoint{0.000000in}{-0.048611in}}{\pgfqpoint{0.000000in}{0.000000in}}{%
\pgfpathmoveto{\pgfqpoint{0.000000in}{0.000000in}}%
\pgfpathlineto{\pgfqpoint{0.000000in}{-0.048611in}}%
\pgfusepath{stroke,fill}%
}%
\begin{pgfscope}%
\pgfsys@transformshift{2.831581in}{0.741795in}%
\pgfsys@useobject{currentmarker}{}%
\end{pgfscope}%
\end{pgfscope}%
\begin{pgfscope}%
\pgftext[x=2.831581in,y=0.644572in,,top]{\rmfamily\fontsize{16.000000}{19.200000}\selectfont \(\displaystyle 0.4\)}%
\end{pgfscope}%
\begin{pgfscope}%
\pgfsetbuttcap%
\pgfsetroundjoin%
\definecolor{currentfill}{rgb}{0.000000,0.000000,0.000000}%
\pgfsetfillcolor{currentfill}%
\pgfsetlinewidth{0.803000pt}%
\definecolor{currentstroke}{rgb}{0.000000,0.000000,0.000000}%
\pgfsetstrokecolor{currentstroke}%
\pgfsetdash{}{0pt}%
\pgfsys@defobject{currentmarker}{\pgfqpoint{0.000000in}{-0.048611in}}{\pgfqpoint{0.000000in}{0.000000in}}{%
\pgfpathmoveto{\pgfqpoint{0.000000in}{0.000000in}}%
\pgfpathlineto{\pgfqpoint{0.000000in}{-0.048611in}}%
\pgfusepath{stroke,fill}%
}%
\begin{pgfscope}%
\pgfsys@transformshift{3.634870in}{0.741795in}%
\pgfsys@useobject{currentmarker}{}%
\end{pgfscope}%
\end{pgfscope}%
\begin{pgfscope}%
\pgftext[x=3.634870in,y=0.644572in,,top]{\rmfamily\fontsize{16.000000}{19.200000}\selectfont \(\displaystyle 0.6\)}%
\end{pgfscope}%
\begin{pgfscope}%
\pgfsetbuttcap%
\pgfsetroundjoin%
\definecolor{currentfill}{rgb}{0.000000,0.000000,0.000000}%
\pgfsetfillcolor{currentfill}%
\pgfsetlinewidth{0.803000pt}%
\definecolor{currentstroke}{rgb}{0.000000,0.000000,0.000000}%
\pgfsetstrokecolor{currentstroke}%
\pgfsetdash{}{0pt}%
\pgfsys@defobject{currentmarker}{\pgfqpoint{0.000000in}{-0.048611in}}{\pgfqpoint{0.000000in}{0.000000in}}{%
\pgfpathmoveto{\pgfqpoint{0.000000in}{0.000000in}}%
\pgfpathlineto{\pgfqpoint{0.000000in}{-0.048611in}}%
\pgfusepath{stroke,fill}%
}%
\begin{pgfscope}%
\pgfsys@transformshift{4.438160in}{0.741795in}%
\pgfsys@useobject{currentmarker}{}%
\end{pgfscope}%
\end{pgfscope}%
\begin{pgfscope}%
\pgftext[x=4.438160in,y=0.644572in,,top]{\rmfamily\fontsize{16.000000}{19.200000}\selectfont \(\displaystyle 0.8\)}%
\end{pgfscope}%
\begin{pgfscope}%
\pgfsetbuttcap%
\pgfsetroundjoin%
\definecolor{currentfill}{rgb}{0.000000,0.000000,0.000000}%
\pgfsetfillcolor{currentfill}%
\pgfsetlinewidth{0.803000pt}%
\definecolor{currentstroke}{rgb}{0.000000,0.000000,0.000000}%
\pgfsetstrokecolor{currentstroke}%
\pgfsetdash{}{0pt}%
\pgfsys@defobject{currentmarker}{\pgfqpoint{0.000000in}{-0.048611in}}{\pgfqpoint{0.000000in}{0.000000in}}{%
\pgfpathmoveto{\pgfqpoint{0.000000in}{0.000000in}}%
\pgfpathlineto{\pgfqpoint{0.000000in}{-0.048611in}}%
\pgfusepath{stroke,fill}%
}%
\begin{pgfscope}%
\pgfsys@transformshift{5.241450in}{0.741795in}%
\pgfsys@useobject{currentmarker}{}%
\end{pgfscope}%
\end{pgfscope}%
\begin{pgfscope}%
\pgftext[x=5.241450in,y=0.644572in,,top]{\rmfamily\fontsize{16.000000}{19.200000}\selectfont \(\displaystyle 1.0\)}%
\end{pgfscope}%
\begin{pgfscope}%
\pgfsetbuttcap%
\pgfsetroundjoin%
\definecolor{currentfill}{rgb}{0.000000,0.000000,0.000000}%
\pgfsetfillcolor{currentfill}%
\pgfsetlinewidth{0.803000pt}%
\definecolor{currentstroke}{rgb}{0.000000,0.000000,0.000000}%
\pgfsetstrokecolor{currentstroke}%
\pgfsetdash{}{0pt}%
\pgfsys@defobject{currentmarker}{\pgfqpoint{-0.048611in}{0.000000in}}{\pgfqpoint{0.000000in}{0.000000in}}{%
\pgfpathmoveto{\pgfqpoint{0.000000in}{0.000000in}}%
\pgfpathlineto{\pgfqpoint{-0.048611in}{0.000000in}}%
\pgfusepath{stroke,fill}%
}%
\begin{pgfscope}%
\pgfsys@transformshift{0.984216in}{0.741795in}%
\pgfsys@useobject{currentmarker}{}%
\end{pgfscope}%
\end{pgfscope}%
\begin{pgfscope}%
\pgftext[x=0.601580in,y=0.657376in,left,base]{\rmfamily\fontsize{16.000000}{19.200000}\selectfont \(\displaystyle 0.0\)}%
\end{pgfscope}%
\begin{pgfscope}%
\pgfsetbuttcap%
\pgfsetroundjoin%
\definecolor{currentfill}{rgb}{0.000000,0.000000,0.000000}%
\pgfsetfillcolor{currentfill}%
\pgfsetlinewidth{0.803000pt}%
\definecolor{currentstroke}{rgb}{0.000000,0.000000,0.000000}%
\pgfsetstrokecolor{currentstroke}%
\pgfsetdash{}{0pt}%
\pgfsys@defobject{currentmarker}{\pgfqpoint{-0.048611in}{0.000000in}}{\pgfqpoint{0.000000in}{0.000000in}}{%
\pgfpathmoveto{\pgfqpoint{0.000000in}{0.000000in}}%
\pgfpathlineto{\pgfqpoint{-0.048611in}{0.000000in}}%
\pgfusepath{stroke,fill}%
}%
\begin{pgfscope}%
\pgfsys@transformshift{0.984216in}{1.422027in}%
\pgfsys@useobject{currentmarker}{}%
\end{pgfscope}%
\end{pgfscope}%
\begin{pgfscope}%
\pgftext[x=0.601580in,y=1.337608in,left,base]{\rmfamily\fontsize{16.000000}{19.200000}\selectfont \(\displaystyle 0.1\)}%
\end{pgfscope}%
\begin{pgfscope}%
\pgfsetbuttcap%
\pgfsetroundjoin%
\definecolor{currentfill}{rgb}{0.000000,0.000000,0.000000}%
\pgfsetfillcolor{currentfill}%
\pgfsetlinewidth{0.803000pt}%
\definecolor{currentstroke}{rgb}{0.000000,0.000000,0.000000}%
\pgfsetstrokecolor{currentstroke}%
\pgfsetdash{}{0pt}%
\pgfsys@defobject{currentmarker}{\pgfqpoint{-0.048611in}{0.000000in}}{\pgfqpoint{0.000000in}{0.000000in}}{%
\pgfpathmoveto{\pgfqpoint{0.000000in}{0.000000in}}%
\pgfpathlineto{\pgfqpoint{-0.048611in}{0.000000in}}%
\pgfusepath{stroke,fill}%
}%
\begin{pgfscope}%
\pgfsys@transformshift{0.984216in}{2.102259in}%
\pgfsys@useobject{currentmarker}{}%
\end{pgfscope}%
\end{pgfscope}%
\begin{pgfscope}%
\pgftext[x=0.601580in,y=2.017840in,left,base]{\rmfamily\fontsize{16.000000}{19.200000}\selectfont \(\displaystyle 0.2\)}%
\end{pgfscope}%
\begin{pgfscope}%
\pgfsetbuttcap%
\pgfsetroundjoin%
\definecolor{currentfill}{rgb}{0.000000,0.000000,0.000000}%
\pgfsetfillcolor{currentfill}%
\pgfsetlinewidth{0.803000pt}%
\definecolor{currentstroke}{rgb}{0.000000,0.000000,0.000000}%
\pgfsetstrokecolor{currentstroke}%
\pgfsetdash{}{0pt}%
\pgfsys@defobject{currentmarker}{\pgfqpoint{-0.048611in}{0.000000in}}{\pgfqpoint{0.000000in}{0.000000in}}{%
\pgfpathmoveto{\pgfqpoint{0.000000in}{0.000000in}}%
\pgfpathlineto{\pgfqpoint{-0.048611in}{0.000000in}}%
\pgfusepath{stroke,fill}%
}%
\begin{pgfscope}%
\pgfsys@transformshift{0.984216in}{2.782491in}%
\pgfsys@useobject{currentmarker}{}%
\end{pgfscope}%
\end{pgfscope}%
\begin{pgfscope}%
\pgftext[x=0.601580in,y=2.698072in,left,base]{\rmfamily\fontsize{16.000000}{19.200000}\selectfont \(\displaystyle 0.3\)}%
\end{pgfscope}%
\begin{pgfscope}%
\pgfsetbuttcap%
\pgfsetroundjoin%
\definecolor{currentfill}{rgb}{0.000000,0.000000,0.000000}%
\pgfsetfillcolor{currentfill}%
\pgfsetlinewidth{0.803000pt}%
\definecolor{currentstroke}{rgb}{0.000000,0.000000,0.000000}%
\pgfsetstrokecolor{currentstroke}%
\pgfsetdash{}{0pt}%
\pgfsys@defobject{currentmarker}{\pgfqpoint{-0.048611in}{0.000000in}}{\pgfqpoint{0.000000in}{0.000000in}}{%
\pgfpathmoveto{\pgfqpoint{0.000000in}{0.000000in}}%
\pgfpathlineto{\pgfqpoint{-0.048611in}{0.000000in}}%
\pgfusepath{stroke,fill}%
}%
\begin{pgfscope}%
\pgfsys@transformshift{0.984216in}{3.462723in}%
\pgfsys@useobject{currentmarker}{}%
\end{pgfscope}%
\end{pgfscope}%
\begin{pgfscope}%
\pgftext[x=0.601580in,y=3.378304in,left,base]{\rmfamily\fontsize{16.000000}{19.200000}\selectfont \(\displaystyle 0.4\)}%
\end{pgfscope}%
\begin{pgfscope}%
\pgfsetbuttcap%
\pgfsetroundjoin%
\definecolor{currentfill}{rgb}{0.000000,0.000000,0.000000}%
\pgfsetfillcolor{currentfill}%
\pgfsetlinewidth{0.803000pt}%
\definecolor{currentstroke}{rgb}{0.000000,0.000000,0.000000}%
\pgfsetstrokecolor{currentstroke}%
\pgfsetdash{}{0pt}%
\pgfsys@defobject{currentmarker}{\pgfqpoint{-0.048611in}{0.000000in}}{\pgfqpoint{0.000000in}{0.000000in}}{%
\pgfpathmoveto{\pgfqpoint{0.000000in}{0.000000in}}%
\pgfpathlineto{\pgfqpoint{-0.048611in}{0.000000in}}%
\pgfusepath{stroke,fill}%
}%
\begin{pgfscope}%
\pgfsys@transformshift{0.984216in}{4.142955in}%
\pgfsys@useobject{currentmarker}{}%
\end{pgfscope}%
\end{pgfscope}%
\begin{pgfscope}%
\pgftext[x=0.601580in,y=4.058536in,left,base]{\rmfamily\fontsize{16.000000}{19.200000}\selectfont \(\displaystyle 0.5\)}%
\end{pgfscope}%
\begin{pgfscope}%
\pgfpathrectangle{\pgfqpoint{0.984216in}{0.741795in}}{\pgfqpoint{4.458056in}{3.401160in}} %
\pgfusepath{clip}%
\pgfsetbuttcap%
\pgfsetroundjoin%
\pgfsetlinewidth{1.505625pt}%
\definecolor{currentstroke}{rgb}{1.000000,0.000000,0.000000}%
\pgfsetstrokecolor{currentstroke}%
\pgfsetdash{{5.550000pt}{2.400000pt}}{0.000000pt}%
\pgfpathmoveto{\pgfqpoint{1.225002in}{0.741795in}}%
\pgfpathlineto{\pgfqpoint{1.273199in}{0.822690in}}%
\pgfpathlineto{\pgfqpoint{1.321396in}{0.902127in}}%
\pgfpathlineto{\pgfqpoint{1.369594in}{0.980117in}}%
\pgfpathlineto{\pgfqpoint{1.417791in}{1.056670in}}%
\pgfpathlineto{\pgfqpoint{1.465989in}{1.131797in}}%
\pgfpathlineto{\pgfqpoint{1.514186in}{1.205506in}}%
\pgfpathlineto{\pgfqpoint{1.562383in}{1.277806in}}%
\pgfpathlineto{\pgfqpoint{1.610581in}{1.348705in}}%
\pgfpathlineto{\pgfqpoint{1.658778in}{1.418212in}}%
\pgfpathlineto{\pgfqpoint{1.706975in}{1.486335in}}%
\pgfpathlineto{\pgfqpoint{1.755173in}{1.553080in}}%
\pgfpathlineto{\pgfqpoint{1.803370in}{1.618454in}}%
\pgfpathlineto{\pgfqpoint{1.847551in}{1.677182in}}%
\pgfpathlineto{\pgfqpoint{1.891732in}{1.734769in}}%
\pgfpathlineto{\pgfqpoint{1.935913in}{1.791219in}}%
\pgfpathlineto{\pgfqpoint{1.980094in}{1.846538in}}%
\pgfpathlineto{\pgfqpoint{2.024275in}{1.900728in}}%
\pgfpathlineto{\pgfqpoint{2.068456in}{1.953794in}}%
\pgfpathlineto{\pgfqpoint{2.112637in}{2.005741in}}%
\pgfpathlineto{\pgfqpoint{2.156818in}{2.056572in}}%
\pgfpathlineto{\pgfqpoint{2.200998in}{2.106290in}}%
\pgfpathlineto{\pgfqpoint{2.245179in}{2.154899in}}%
\pgfpathlineto{\pgfqpoint{2.289360in}{2.202402in}}%
\pgfpathlineto{\pgfqpoint{2.333541in}{2.248803in}}%
\pgfpathlineto{\pgfqpoint{2.377722in}{2.294104in}}%
\pgfpathlineto{\pgfqpoint{2.421903in}{2.338309in}}%
\pgfpathlineto{\pgfqpoint{2.466084in}{2.381420in}}%
\pgfpathlineto{\pgfqpoint{2.510265in}{2.423439in}}%
\pgfpathlineto{\pgfqpoint{2.554446in}{2.464370in}}%
\pgfpathlineto{\pgfqpoint{2.598627in}{2.504215in}}%
\pgfpathlineto{\pgfqpoint{2.642808in}{2.542976in}}%
\pgfpathlineto{\pgfqpoint{2.682972in}{2.577275in}}%
\pgfpathlineto{\pgfqpoint{2.723137in}{2.610681in}}%
\pgfpathlineto{\pgfqpoint{2.763301in}{2.643197in}}%
\pgfpathlineto{\pgfqpoint{2.803466in}{2.674824in}}%
\pgfpathlineto{\pgfqpoint{2.843630in}{2.705563in}}%
\pgfpathlineto{\pgfqpoint{2.883795in}{2.735416in}}%
\pgfpathlineto{\pgfqpoint{2.923959in}{2.764385in}}%
\pgfpathlineto{\pgfqpoint{2.964124in}{2.792470in}}%
\pgfpathlineto{\pgfqpoint{3.004288in}{2.819673in}}%
\pgfpathlineto{\pgfqpoint{3.044453in}{2.845995in}}%
\pgfpathlineto{\pgfqpoint{3.084617in}{2.871437in}}%
\pgfpathlineto{\pgfqpoint{3.124782in}{2.896001in}}%
\pgfpathlineto{\pgfqpoint{3.164946in}{2.919687in}}%
\pgfpathlineto{\pgfqpoint{3.205110in}{2.942498in}}%
\pgfpathlineto{\pgfqpoint{3.245275in}{2.964433in}}%
\pgfpathlineto{\pgfqpoint{3.285439in}{2.985494in}}%
\pgfpathlineto{\pgfqpoint{3.325604in}{3.005682in}}%
\pgfpathlineto{\pgfqpoint{3.365768in}{3.024998in}}%
\pgfpathlineto{\pgfqpoint{3.405933in}{3.043443in}}%
\pgfpathlineto{\pgfqpoint{3.446097in}{3.061018in}}%
\pgfpathlineto{\pgfqpoint{3.486262in}{3.077724in}}%
\pgfpathlineto{\pgfqpoint{3.526426in}{3.093562in}}%
\pgfpathlineto{\pgfqpoint{3.566591in}{3.108533in}}%
\pgfpathlineto{\pgfqpoint{3.606755in}{3.122636in}}%
\pgfpathlineto{\pgfqpoint{3.646920in}{3.135875in}}%
\pgfpathlineto{\pgfqpoint{3.683068in}{3.147050in}}%
\pgfpathlineto{\pgfqpoint{3.719216in}{3.157525in}}%
\pgfpathlineto{\pgfqpoint{3.755364in}{3.167301in}}%
\pgfpathlineto{\pgfqpoint{3.791512in}{3.176378in}}%
\pgfpathlineto{\pgfqpoint{3.827660in}{3.184757in}}%
\pgfpathlineto{\pgfqpoint{3.863808in}{3.192438in}}%
\pgfpathlineto{\pgfqpoint{3.899956in}{3.199422in}}%
\pgfpathlineto{\pgfqpoint{3.936104in}{3.205710in}}%
\pgfpathlineto{\pgfqpoint{3.972252in}{3.211301in}}%
\pgfpathlineto{\pgfqpoint{4.008400in}{3.216196in}}%
\pgfpathlineto{\pgfqpoint{4.044548in}{3.220396in}}%
\pgfpathlineto{\pgfqpoint{4.080696in}{3.223901in}}%
\pgfpathlineto{\pgfqpoint{4.116844in}{3.226712in}}%
\pgfpathlineto{\pgfqpoint{4.152992in}{3.228829in}}%
\pgfpathlineto{\pgfqpoint{4.189140in}{3.230251in}}%
\pgfpathlineto{\pgfqpoint{4.225288in}{3.230980in}}%
\pgfpathlineto{\pgfqpoint{4.261436in}{3.231016in}}%
\pgfpathlineto{\pgfqpoint{4.297584in}{3.230359in}}%
\pgfpathlineto{\pgfqpoint{4.333732in}{3.229010in}}%
\pgfpathlineto{\pgfqpoint{4.369880in}{3.226967in}}%
\pgfpathlineto{\pgfqpoint{4.406028in}{3.224233in}}%
\pgfpathlineto{\pgfqpoint{4.442176in}{3.220806in}}%
\pgfpathlineto{\pgfqpoint{4.478324in}{3.216687in}}%
\pgfpathlineto{\pgfqpoint{4.514473in}{3.211877in}}%
\pgfpathlineto{\pgfqpoint{4.550621in}{3.206374in}}%
\pgfpathlineto{\pgfqpoint{4.586769in}{3.200179in}}%
\pgfpathlineto{\pgfqpoint{4.622917in}{3.193292in}}%
\pgfpathlineto{\pgfqpoint{4.659065in}{3.185713in}}%
\pgfpathlineto{\pgfqpoint{4.695213in}{3.177441in}}%
\pgfpathlineto{\pgfqpoint{4.731361in}{3.168477in}}%
\pgfpathlineto{\pgfqpoint{4.767509in}{3.158820in}}%
\pgfpathlineto{\pgfqpoint{4.803657in}{3.148471in}}%
\pgfpathlineto{\pgfqpoint{4.843821in}{3.136157in}}%
\pgfpathlineto{\pgfqpoint{4.883986in}{3.122988in}}%
\pgfpathlineto{\pgfqpoint{4.924150in}{3.108961in}}%
\pgfpathlineto{\pgfqpoint{4.964315in}{3.094076in}}%
\pgfpathlineto{\pgfqpoint{5.004479in}{3.078332in}}%
\pgfpathlineto{\pgfqpoint{5.044644in}{3.061728in}}%
\pgfpathlineto{\pgfqpoint{5.084808in}{3.044264in}}%
\pgfpathlineto{\pgfqpoint{5.124973in}{3.025938in}}%
\pgfpathlineto{\pgfqpoint{5.165137in}{3.006749in}}%
\pgfpathlineto{\pgfqpoint{5.205302in}{2.986696in}}%
\pgfpathlineto{\pgfqpoint{5.245466in}{2.965778in}}%
\pgfpathlineto{\pgfqpoint{5.285631in}{2.943992in}}%
\pgfpathlineto{\pgfqpoint{5.325795in}{2.921338in}}%
\pgfpathlineto{\pgfqpoint{5.365959in}{2.897814in}}%
\pgfpathlineto{\pgfqpoint{5.406124in}{2.873417in}}%
\pgfpathlineto{\pgfqpoint{5.445605in}{2.848584in}}%
\pgfpathlineto{\pgfqpoint{5.445605in}{2.848584in}}%
\pgfusepath{stroke}%
\end{pgfscope}%
\begin{pgfscope}%
\pgfpathrectangle{\pgfqpoint{0.984216in}{0.741795in}}{\pgfqpoint{4.458056in}{3.401160in}} %
\pgfusepath{clip}%
\pgfsetbuttcap%
\pgfsetmiterjoin%
\definecolor{currentfill}{rgb}{1.000000,0.000000,0.000000}%
\pgfsetfillcolor{currentfill}%
\pgfsetlinewidth{1.003750pt}%
\definecolor{currentstroke}{rgb}{1.000000,0.000000,0.000000}%
\pgfsetstrokecolor{currentstroke}%
\pgfsetdash{}{0pt}%
\pgfsys@defobject{currentmarker}{\pgfqpoint{-0.041667in}{-0.041667in}}{\pgfqpoint{0.041667in}{0.041667in}}{%
\pgfpathmoveto{\pgfqpoint{-0.041667in}{-0.041667in}}%
\pgfpathlineto{\pgfqpoint{0.041667in}{-0.041667in}}%
\pgfpathlineto{\pgfqpoint{0.041667in}{0.041667in}}%
\pgfpathlineto{\pgfqpoint{-0.041667in}{0.041667in}}%
\pgfpathclose%
\pgfusepath{stroke,fill}%
}%
\begin{pgfscope}%
\pgfsys@transformshift{1.225002in}{0.741795in}%
\pgfsys@useobject{currentmarker}{}%
\end{pgfscope}%
\begin{pgfscope}%
\pgfsys@transformshift{1.626646in}{1.372029in}%
\pgfsys@useobject{currentmarker}{}%
\end{pgfscope}%
\begin{pgfscope}%
\pgfsys@transformshift{2.028291in}{1.905598in}%
\pgfsys@useobject{currentmarker}{}%
\end{pgfscope}%
\begin{pgfscope}%
\pgfsys@transformshift{2.429936in}{2.346228in}%
\pgfsys@useobject{currentmarker}{}%
\end{pgfscope}%
\begin{pgfscope}%
\pgfsys@transformshift{2.831581in}{2.696435in}%
\pgfsys@useobject{currentmarker}{}%
\end{pgfscope}%
\begin{pgfscope}%
\pgfsys@transformshift{3.233226in}{2.957944in}%
\pgfsys@useobject{currentmarker}{}%
\end{pgfscope}%
\begin{pgfscope}%
\pgfsys@transformshift{3.634870in}{3.131994in}%
\pgfsys@useobject{currentmarker}{}%
\end{pgfscope}%
\begin{pgfscope}%
\pgfsys@transformshift{4.036515in}{3.219523in}%
\pgfsys@useobject{currentmarker}{}%
\end{pgfscope}%
\begin{pgfscope}%
\pgfsys@transformshift{4.438160in}{3.221221in}%
\pgfsys@useobject{currentmarker}{}%
\end{pgfscope}%
\begin{pgfscope}%
\pgfsys@transformshift{4.839805in}{3.137427in}%
\pgfsys@useobject{currentmarker}{}%
\end{pgfscope}%
\begin{pgfscope}%
\pgfsys@transformshift{5.241450in}{2.967909in}%
\pgfsys@useobject{currentmarker}{}%
\end{pgfscope}%
\begin{pgfscope}%
\pgfsys@transformshift{5.643094in}{2.711624in}%
\pgfsys@useobject{currentmarker}{}%
\end{pgfscope}%
\end{pgfscope}%
\begin{pgfscope}%
\pgfpathrectangle{\pgfqpoint{0.984216in}{0.741795in}}{\pgfqpoint{4.458056in}{3.401160in}} %
\pgfusepath{clip}%
\pgfsetrectcap%
\pgfsetroundjoin%
\pgfsetlinewidth{1.505625pt}%
\definecolor{currentstroke}{rgb}{0.000000,0.000000,1.000000}%
\pgfsetstrokecolor{currentstroke}%
\pgfsetdash{}{0pt}%
\pgfpathmoveto{\pgfqpoint{1.225002in}{0.741795in}}%
\pgfpathlineto{\pgfqpoint{1.273199in}{0.822688in}}%
\pgfpathlineto{\pgfqpoint{1.321396in}{0.902113in}}%
\pgfpathlineto{\pgfqpoint{1.369594in}{0.980071in}}%
\pgfpathlineto{\pgfqpoint{1.417791in}{1.056564in}}%
\pgfpathlineto{\pgfqpoint{1.465989in}{1.131591in}}%
\pgfpathlineto{\pgfqpoint{1.514186in}{1.205155in}}%
\pgfpathlineto{\pgfqpoint{1.562383in}{1.277256in}}%
\pgfpathlineto{\pgfqpoint{1.606564in}{1.342065in}}%
\pgfpathlineto{\pgfqpoint{1.650745in}{1.405646in}}%
\pgfpathlineto{\pgfqpoint{1.694926in}{1.468000in}}%
\pgfpathlineto{\pgfqpoint{1.739107in}{1.529129in}}%
\pgfpathlineto{\pgfqpoint{1.783288in}{1.589032in}}%
\pgfpathlineto{\pgfqpoint{1.827469in}{1.647711in}}%
\pgfpathlineto{\pgfqpoint{1.871650in}{1.705166in}}%
\pgfpathlineto{\pgfqpoint{1.915831in}{1.761397in}}%
\pgfpathlineto{\pgfqpoint{1.960012in}{1.816406in}}%
\pgfpathlineto{\pgfqpoint{2.004193in}{1.870193in}}%
\pgfpathlineto{\pgfqpoint{2.048373in}{1.922759in}}%
\pgfpathlineto{\pgfqpoint{2.092554in}{1.974104in}}%
\pgfpathlineto{\pgfqpoint{2.136735in}{2.024229in}}%
\pgfpathlineto{\pgfqpoint{2.180916in}{2.073135in}}%
\pgfpathlineto{\pgfqpoint{2.221081in}{2.116536in}}%
\pgfpathlineto{\pgfqpoint{2.261245in}{2.158931in}}%
\pgfpathlineto{\pgfqpoint{2.301410in}{2.200319in}}%
\pgfpathlineto{\pgfqpoint{2.341574in}{2.240701in}}%
\pgfpathlineto{\pgfqpoint{2.381739in}{2.280077in}}%
\pgfpathlineto{\pgfqpoint{2.421903in}{2.318448in}}%
\pgfpathlineto{\pgfqpoint{2.462068in}{2.355813in}}%
\pgfpathlineto{\pgfqpoint{2.502232in}{2.392174in}}%
\pgfpathlineto{\pgfqpoint{2.542397in}{2.427530in}}%
\pgfpathlineto{\pgfqpoint{2.582561in}{2.461882in}}%
\pgfpathlineto{\pgfqpoint{2.622726in}{2.495231in}}%
\pgfpathlineto{\pgfqpoint{2.662890in}{2.527576in}}%
\pgfpathlineto{\pgfqpoint{2.703054in}{2.558917in}}%
\pgfpathlineto{\pgfqpoint{2.743219in}{2.589256in}}%
\pgfpathlineto{\pgfqpoint{2.783383in}{2.618592in}}%
\pgfpathlineto{\pgfqpoint{2.823548in}{2.646926in}}%
\pgfpathlineto{\pgfqpoint{2.859696in}{2.671570in}}%
\pgfpathlineto{\pgfqpoint{2.895844in}{2.695402in}}%
\pgfpathlineto{\pgfqpoint{2.931992in}{2.718423in}}%
\pgfpathlineto{\pgfqpoint{2.968140in}{2.740633in}}%
\pgfpathlineto{\pgfqpoint{3.004288in}{2.762032in}}%
\pgfpathlineto{\pgfqpoint{3.040436in}{2.782620in}}%
\pgfpathlineto{\pgfqpoint{3.076584in}{2.802397in}}%
\pgfpathlineto{\pgfqpoint{3.112732in}{2.821364in}}%
\pgfpathlineto{\pgfqpoint{3.148880in}{2.839521in}}%
\pgfpathlineto{\pgfqpoint{3.185028in}{2.856867in}}%
\pgfpathlineto{\pgfqpoint{3.221176in}{2.873403in}}%
\pgfpathlineto{\pgfqpoint{3.257324in}{2.889129in}}%
\pgfpathlineto{\pgfqpoint{3.293472in}{2.904045in}}%
\pgfpathlineto{\pgfqpoint{3.329620in}{2.918152in}}%
\pgfpathlineto{\pgfqpoint{3.365768in}{2.931448in}}%
\pgfpathlineto{\pgfqpoint{3.401916in}{2.943936in}}%
\pgfpathlineto{\pgfqpoint{3.438064in}{2.955613in}}%
\pgfpathlineto{\pgfqpoint{3.474212in}{2.966482in}}%
\pgfpathlineto{\pgfqpoint{3.510361in}{2.976541in}}%
\pgfpathlineto{\pgfqpoint{3.546509in}{2.985790in}}%
\pgfpathlineto{\pgfqpoint{3.582657in}{2.994231in}}%
\pgfpathlineto{\pgfqpoint{3.618805in}{3.001863in}}%
\pgfpathlineto{\pgfqpoint{3.654953in}{3.008685in}}%
\pgfpathlineto{\pgfqpoint{3.691101in}{3.014699in}}%
\pgfpathlineto{\pgfqpoint{3.727249in}{3.019903in}}%
\pgfpathlineto{\pgfqpoint{3.763397in}{3.024299in}}%
\pgfpathlineto{\pgfqpoint{3.799545in}{3.027885in}}%
\pgfpathlineto{\pgfqpoint{3.835693in}{3.030663in}}%
\pgfpathlineto{\pgfqpoint{3.871841in}{3.032632in}}%
\pgfpathlineto{\pgfqpoint{3.907989in}{3.033792in}}%
\pgfpathlineto{\pgfqpoint{3.944137in}{3.034144in}}%
\pgfpathlineto{\pgfqpoint{3.980285in}{3.033686in}}%
\pgfpathlineto{\pgfqpoint{4.016433in}{3.032420in}}%
\pgfpathlineto{\pgfqpoint{4.052581in}{3.030345in}}%
\pgfpathlineto{\pgfqpoint{4.088729in}{3.027461in}}%
\pgfpathlineto{\pgfqpoint{4.124877in}{3.023768in}}%
\pgfpathlineto{\pgfqpoint{4.161025in}{3.019267in}}%
\pgfpathlineto{\pgfqpoint{4.197173in}{3.013956in}}%
\pgfpathlineto{\pgfqpoint{4.233321in}{3.007837in}}%
\pgfpathlineto{\pgfqpoint{4.269469in}{3.000908in}}%
\pgfpathlineto{\pgfqpoint{4.305617in}{2.993170in}}%
\pgfpathlineto{\pgfqpoint{4.341765in}{2.984624in}}%
\pgfpathlineto{\pgfqpoint{4.377913in}{2.975268in}}%
\pgfpathlineto{\pgfqpoint{4.414061in}{2.965103in}}%
\pgfpathlineto{\pgfqpoint{4.450209in}{2.954128in}}%
\pgfpathlineto{\pgfqpoint{4.486357in}{2.942344in}}%
\pgfpathlineto{\pgfqpoint{4.522505in}{2.929751in}}%
\pgfpathlineto{\pgfqpoint{4.558653in}{2.916348in}}%
\pgfpathlineto{\pgfqpoint{4.594801in}{2.902135in}}%
\pgfpathlineto{\pgfqpoint{4.630949in}{2.887113in}}%
\pgfpathlineto{\pgfqpoint{4.667098in}{2.871281in}}%
\pgfpathlineto{\pgfqpoint{4.703246in}{2.854639in}}%
\pgfpathlineto{\pgfqpoint{4.739394in}{2.837186in}}%
\pgfpathlineto{\pgfqpoint{4.775542in}{2.818923in}}%
\pgfpathlineto{\pgfqpoint{4.811690in}{2.799850in}}%
\pgfpathlineto{\pgfqpoint{4.847838in}{2.779967in}}%
\pgfpathlineto{\pgfqpoint{4.883986in}{2.759272in}}%
\pgfpathlineto{\pgfqpoint{4.920134in}{2.737767in}}%
\pgfpathlineto{\pgfqpoint{4.956282in}{2.715451in}}%
\pgfpathlineto{\pgfqpoint{4.992430in}{2.692324in}}%
\pgfpathlineto{\pgfqpoint{5.028578in}{2.668385in}}%
\pgfpathlineto{\pgfqpoint{5.064726in}{2.643635in}}%
\pgfpathlineto{\pgfqpoint{5.104890in}{2.615183in}}%
\pgfpathlineto{\pgfqpoint{5.145055in}{2.585728in}}%
\pgfpathlineto{\pgfqpoint{5.185219in}{2.555271in}}%
\pgfpathlineto{\pgfqpoint{5.225384in}{2.523811in}}%
\pgfpathlineto{\pgfqpoint{5.265548in}{2.491348in}}%
\pgfpathlineto{\pgfqpoint{5.305713in}{2.457881in}}%
\pgfpathlineto{\pgfqpoint{5.345877in}{2.423410in}}%
\pgfpathlineto{\pgfqpoint{5.386042in}{2.387935in}}%
\pgfpathlineto{\pgfqpoint{5.426206in}{2.351456in}}%
\pgfpathlineto{\pgfqpoint{5.445605in}{2.333476in}}%
\pgfpathlineto{\pgfqpoint{5.445605in}{2.333476in}}%
\pgfusepath{stroke}%
\end{pgfscope}%
\begin{pgfscope}%
\pgfpathrectangle{\pgfqpoint{0.984216in}{0.741795in}}{\pgfqpoint{4.458056in}{3.401160in}} %
\pgfusepath{clip}%
\pgfsetbuttcap%
\pgfsetroundjoin%
\definecolor{currentfill}{rgb}{0.000000,0.000000,1.000000}%
\pgfsetfillcolor{currentfill}%
\pgfsetlinewidth{1.003750pt}%
\definecolor{currentstroke}{rgb}{0.000000,0.000000,1.000000}%
\pgfsetstrokecolor{currentstroke}%
\pgfsetdash{}{0pt}%
\pgfsys@defobject{currentmarker}{\pgfqpoint{-0.041667in}{-0.041667in}}{\pgfqpoint{0.041667in}{0.041667in}}{%
\pgfpathmoveto{\pgfqpoint{0.000000in}{-0.041667in}}%
\pgfpathcurveto{\pgfqpoint{0.011050in}{-0.041667in}}{\pgfqpoint{0.021649in}{-0.037276in}}{\pgfqpoint{0.029463in}{-0.029463in}}%
\pgfpathcurveto{\pgfqpoint{0.037276in}{-0.021649in}}{\pgfqpoint{0.041667in}{-0.011050in}}{\pgfqpoint{0.041667in}{0.000000in}}%
\pgfpathcurveto{\pgfqpoint{0.041667in}{0.011050in}}{\pgfqpoint{0.037276in}{0.021649in}}{\pgfqpoint{0.029463in}{0.029463in}}%
\pgfpathcurveto{\pgfqpoint{0.021649in}{0.037276in}}{\pgfqpoint{0.011050in}{0.041667in}}{\pgfqpoint{0.000000in}{0.041667in}}%
\pgfpathcurveto{\pgfqpoint{-0.011050in}{0.041667in}}{\pgfqpoint{-0.021649in}{0.037276in}}{\pgfqpoint{-0.029463in}{0.029463in}}%
\pgfpathcurveto{\pgfqpoint{-0.037276in}{0.021649in}}{\pgfqpoint{-0.041667in}{0.011050in}}{\pgfqpoint{-0.041667in}{0.000000in}}%
\pgfpathcurveto{\pgfqpoint{-0.041667in}{-0.011050in}}{\pgfqpoint{-0.037276in}{-0.021649in}}{\pgfqpoint{-0.029463in}{-0.029463in}}%
\pgfpathcurveto{\pgfqpoint{-0.021649in}{-0.037276in}}{\pgfqpoint{-0.011050in}{-0.041667in}}{\pgfqpoint{0.000000in}{-0.041667in}}%
\pgfpathclose%
\pgfusepath{stroke,fill}%
}%
\begin{pgfscope}%
\pgfsys@transformshift{1.225002in}{0.741795in}%
\pgfsys@useobject{currentmarker}{}%
\end{pgfscope}%
\begin{pgfscope}%
\pgfsys@transformshift{1.626646in}{1.371118in}%
\pgfsys@useobject{currentmarker}{}%
\end{pgfscope}%
\begin{pgfscope}%
\pgfsys@transformshift{2.028291in}{1.899017in}%
\pgfsys@useobject{currentmarker}{}%
\end{pgfscope}%
\begin{pgfscope}%
\pgfsys@transformshift{2.429936in}{2.326001in}%
\pgfsys@useobject{currentmarker}{}%
\end{pgfscope}%
\begin{pgfscope}%
\pgfsys@transformshift{2.831581in}{2.652473in}%
\pgfsys@useobject{currentmarker}{}%
\end{pgfscope}%
\begin{pgfscope}%
\pgfsys@transformshift{3.233226in}{2.878735in}%
\pgfsys@useobject{currentmarker}{}%
\end{pgfscope}%
\begin{pgfscope}%
\pgfsys@transformshift{3.634870in}{3.004995in}%
\pgfsys@useobject{currentmarker}{}%
\end{pgfscope}%
\begin{pgfscope}%
\pgfsys@transformshift{4.036515in}{3.031367in}%
\pgfsys@useobject{currentmarker}{}%
\end{pgfscope}%
\begin{pgfscope}%
\pgfsys@transformshift{4.438160in}{2.957876in}%
\pgfsys@useobject{currentmarker}{}%
\end{pgfscope}%
\begin{pgfscope}%
\pgfsys@transformshift{4.839805in}{2.784455in}%
\pgfsys@useobject{currentmarker}{}%
\end{pgfscope}%
\begin{pgfscope}%
\pgfsys@transformshift{5.241450in}{2.510946in}%
\pgfsys@useobject{currentmarker}{}%
\end{pgfscope}%
\begin{pgfscope}%
\pgfsys@transformshift{5.643094in}{2.137096in}%
\pgfsys@useobject{currentmarker}{}%
\end{pgfscope}%
\end{pgfscope}%
\begin{pgfscope}%
\pgfpathrectangle{\pgfqpoint{0.984216in}{0.741795in}}{\pgfqpoint{4.458056in}{3.401160in}} %
\pgfusepath{clip}%
\pgfsetbuttcap%
\pgfsetroundjoin%
\pgfsetlinewidth{1.505625pt}%
\definecolor{currentstroke}{rgb}{0.000000,0.750000,0.750000}%
\pgfsetstrokecolor{currentstroke}%
\pgfsetdash{{9.600000pt}{2.400000pt}{1.500000pt}{2.400000pt}}{0.000000pt}%
\pgfpathmoveto{\pgfqpoint{1.225002in}{0.741795in}}%
\pgfpathlineto{\pgfqpoint{1.273199in}{0.822688in}}%
\pgfpathlineto{\pgfqpoint{1.321396in}{0.902112in}}%
\pgfpathlineto{\pgfqpoint{1.369594in}{0.980067in}}%
\pgfpathlineto{\pgfqpoint{1.417791in}{1.056553in}}%
\pgfpathlineto{\pgfqpoint{1.465989in}{1.131570in}}%
\pgfpathlineto{\pgfqpoint{1.514186in}{1.205118in}}%
\pgfpathlineto{\pgfqpoint{1.558367in}{1.271247in}}%
\pgfpathlineto{\pgfqpoint{1.602548in}{1.336143in}}%
\pgfpathlineto{\pgfqpoint{1.646729in}{1.399804in}}%
\pgfpathlineto{\pgfqpoint{1.690910in}{1.462232in}}%
\pgfpathlineto{\pgfqpoint{1.735091in}{1.523425in}}%
\pgfpathlineto{\pgfqpoint{1.779271in}{1.583385in}}%
\pgfpathlineto{\pgfqpoint{1.823452in}{1.642112in}}%
\pgfpathlineto{\pgfqpoint{1.867633in}{1.699605in}}%
\pgfpathlineto{\pgfqpoint{1.911814in}{1.755864in}}%
\pgfpathlineto{\pgfqpoint{1.955995in}{1.810890in}}%
\pgfpathlineto{\pgfqpoint{2.000176in}{1.864683in}}%
\pgfpathlineto{\pgfqpoint{2.044357in}{1.917243in}}%
\pgfpathlineto{\pgfqpoint{2.088538in}{1.968569in}}%
\pgfpathlineto{\pgfqpoint{2.132719in}{2.018662in}}%
\pgfpathlineto{\pgfqpoint{2.172883in}{2.063131in}}%
\pgfpathlineto{\pgfqpoint{2.213048in}{2.106581in}}%
\pgfpathlineto{\pgfqpoint{2.253212in}{2.149013in}}%
\pgfpathlineto{\pgfqpoint{2.293377in}{2.190425in}}%
\pgfpathlineto{\pgfqpoint{2.333541in}{2.230818in}}%
\pgfpathlineto{\pgfqpoint{2.373706in}{2.270193in}}%
\pgfpathlineto{\pgfqpoint{2.413870in}{2.308548in}}%
\pgfpathlineto{\pgfqpoint{2.454035in}{2.345885in}}%
\pgfpathlineto{\pgfqpoint{2.494199in}{2.382203in}}%
\pgfpathlineto{\pgfqpoint{2.534364in}{2.417503in}}%
\pgfpathlineto{\pgfqpoint{2.574528in}{2.451783in}}%
\pgfpathlineto{\pgfqpoint{2.614693in}{2.485045in}}%
\pgfpathlineto{\pgfqpoint{2.654857in}{2.517289in}}%
\pgfpathlineto{\pgfqpoint{2.695022in}{2.548514in}}%
\pgfpathlineto{\pgfqpoint{2.735186in}{2.578720in}}%
\pgfpathlineto{\pgfqpoint{2.775351in}{2.607908in}}%
\pgfpathlineto{\pgfqpoint{2.811499in}{2.633306in}}%
\pgfpathlineto{\pgfqpoint{2.847647in}{2.657879in}}%
\pgfpathlineto{\pgfqpoint{2.883795in}{2.681627in}}%
\pgfpathlineto{\pgfqpoint{2.919943in}{2.704551in}}%
\pgfpathlineto{\pgfqpoint{2.956091in}{2.726649in}}%
\pgfpathlineto{\pgfqpoint{2.992239in}{2.747923in}}%
\pgfpathlineto{\pgfqpoint{3.028387in}{2.768372in}}%
\pgfpathlineto{\pgfqpoint{3.064535in}{2.787996in}}%
\pgfpathlineto{\pgfqpoint{3.100683in}{2.806795in}}%
\pgfpathlineto{\pgfqpoint{3.136831in}{2.824769in}}%
\pgfpathlineto{\pgfqpoint{3.172979in}{2.841918in}}%
\pgfpathlineto{\pgfqpoint{3.209127in}{2.858243in}}%
\pgfpathlineto{\pgfqpoint{3.245275in}{2.873743in}}%
\pgfpathlineto{\pgfqpoint{3.281423in}{2.888418in}}%
\pgfpathlineto{\pgfqpoint{3.317571in}{2.902269in}}%
\pgfpathlineto{\pgfqpoint{3.353719in}{2.915294in}}%
\pgfpathlineto{\pgfqpoint{3.389867in}{2.927495in}}%
\pgfpathlineto{\pgfqpoint{3.426015in}{2.938871in}}%
\pgfpathlineto{\pgfqpoint{3.462163in}{2.949423in}}%
\pgfpathlineto{\pgfqpoint{3.498311in}{2.959150in}}%
\pgfpathlineto{\pgfqpoint{3.534459in}{2.968052in}}%
\pgfpathlineto{\pgfqpoint{3.570607in}{2.976129in}}%
\pgfpathlineto{\pgfqpoint{3.606755in}{2.983382in}}%
\pgfpathlineto{\pgfqpoint{3.642903in}{2.989810in}}%
\pgfpathlineto{\pgfqpoint{3.679051in}{2.995413in}}%
\pgfpathlineto{\pgfqpoint{3.715199in}{3.000192in}}%
\pgfpathlineto{\pgfqpoint{3.751347in}{3.004146in}}%
\pgfpathlineto{\pgfqpoint{3.787495in}{3.007275in}}%
\pgfpathlineto{\pgfqpoint{3.823643in}{3.009580in}}%
\pgfpathlineto{\pgfqpoint{3.859791in}{3.011060in}}%
\pgfpathlineto{\pgfqpoint{3.895940in}{3.011715in}}%
\pgfpathlineto{\pgfqpoint{3.932088in}{3.011546in}}%
\pgfpathlineto{\pgfqpoint{3.968236in}{3.010552in}}%
\pgfpathlineto{\pgfqpoint{4.004384in}{3.008734in}}%
\pgfpathlineto{\pgfqpoint{4.040532in}{3.006091in}}%
\pgfpathlineto{\pgfqpoint{4.076680in}{3.002623in}}%
\pgfpathlineto{\pgfqpoint{4.112828in}{2.998330in}}%
\pgfpathlineto{\pgfqpoint{4.148976in}{2.993213in}}%
\pgfpathlineto{\pgfqpoint{4.185124in}{2.987271in}}%
\pgfpathlineto{\pgfqpoint{4.221272in}{2.980505in}}%
\pgfpathlineto{\pgfqpoint{4.257420in}{2.972913in}}%
\pgfpathlineto{\pgfqpoint{4.293568in}{2.964498in}}%
\pgfpathlineto{\pgfqpoint{4.329716in}{2.955257in}}%
\pgfpathlineto{\pgfqpoint{4.365864in}{2.945192in}}%
\pgfpathlineto{\pgfqpoint{4.402012in}{2.934302in}}%
\pgfpathlineto{\pgfqpoint{4.438160in}{2.922587in}}%
\pgfpathlineto{\pgfqpoint{4.474308in}{2.910048in}}%
\pgfpathlineto{\pgfqpoint{4.510456in}{2.896684in}}%
\pgfpathlineto{\pgfqpoint{4.546604in}{2.882495in}}%
\pgfpathlineto{\pgfqpoint{4.582752in}{2.867481in}}%
\pgfpathlineto{\pgfqpoint{4.618900in}{2.851643in}}%
\pgfpathlineto{\pgfqpoint{4.655048in}{2.834979in}}%
\pgfpathlineto{\pgfqpoint{4.691196in}{2.817491in}}%
\pgfpathlineto{\pgfqpoint{4.727344in}{2.799179in}}%
\pgfpathlineto{\pgfqpoint{4.763492in}{2.780041in}}%
\pgfpathlineto{\pgfqpoint{4.799640in}{2.760079in}}%
\pgfpathlineto{\pgfqpoint{4.835788in}{2.739291in}}%
\pgfpathlineto{\pgfqpoint{4.871936in}{2.717679in}}%
\pgfpathlineto{\pgfqpoint{4.908084in}{2.695242in}}%
\pgfpathlineto{\pgfqpoint{4.944232in}{2.671980in}}%
\pgfpathlineto{\pgfqpoint{4.980380in}{2.647893in}}%
\pgfpathlineto{\pgfqpoint{5.016528in}{2.622981in}}%
\pgfpathlineto{\pgfqpoint{5.052677in}{2.597244in}}%
\pgfpathlineto{\pgfqpoint{5.088825in}{2.570683in}}%
\pgfpathlineto{\pgfqpoint{5.128989in}{2.540202in}}%
\pgfpathlineto{\pgfqpoint{5.169154in}{2.508703in}}%
\pgfpathlineto{\pgfqpoint{5.209318in}{2.476185in}}%
\pgfpathlineto{\pgfqpoint{5.249482in}{2.442648in}}%
\pgfpathlineto{\pgfqpoint{5.289647in}{2.408093in}}%
\pgfpathlineto{\pgfqpoint{5.329811in}{2.372519in}}%
\pgfpathlineto{\pgfqpoint{5.369976in}{2.335927in}}%
\pgfpathlineto{\pgfqpoint{5.410140in}{2.298315in}}%
\pgfpathlineto{\pgfqpoint{5.445605in}{2.264257in}}%
\pgfpathlineto{\pgfqpoint{5.445605in}{2.264257in}}%
\pgfusepath{stroke}%
\end{pgfscope}%
\begin{pgfscope}%
\pgfpathrectangle{\pgfqpoint{0.984216in}{0.741795in}}{\pgfqpoint{4.458056in}{3.401160in}} %
\pgfusepath{clip}%
\pgfsetbuttcap%
\pgfsetmiterjoin%
\definecolor{currentfill}{rgb}{0.000000,0.750000,0.750000}%
\pgfsetfillcolor{currentfill}%
\pgfsetlinewidth{1.003750pt}%
\definecolor{currentstroke}{rgb}{0.000000,0.750000,0.750000}%
\pgfsetstrokecolor{currentstroke}%
\pgfsetdash{}{0pt}%
\pgfsys@defobject{currentmarker}{\pgfqpoint{-0.041667in}{-0.041667in}}{\pgfqpoint{0.041667in}{0.041667in}}{%
\pgfpathmoveto{\pgfqpoint{-0.000000in}{-0.041667in}}%
\pgfpathlineto{\pgfqpoint{0.041667in}{0.041667in}}%
\pgfpathlineto{\pgfqpoint{-0.041667in}{0.041667in}}%
\pgfpathclose%
\pgfusepath{stroke,fill}%
}%
\begin{pgfscope}%
\pgfsys@transformshift{1.225002in}{0.741795in}%
\pgfsys@useobject{currentmarker}{}%
\end{pgfscope}%
\begin{pgfscope}%
\pgfsys@transformshift{1.626646in}{1.371020in}%
\pgfsys@useobject{currentmarker}{}%
\end{pgfscope}%
\begin{pgfscope}%
\pgfsys@transformshift{2.028291in}{1.898273in}%
\pgfsys@useobject{currentmarker}{}%
\end{pgfscope}%
\begin{pgfscope}%
\pgfsys@transformshift{2.429936in}{2.323605in}%
\pgfsys@useobject{currentmarker}{}%
\end{pgfscope}%
\begin{pgfscope}%
\pgfsys@transformshift{2.831581in}{2.647060in}%
\pgfsys@useobject{currentmarker}{}%
\end{pgfscope}%
\begin{pgfscope}%
\pgfsys@transformshift{3.233226in}{2.868668in}%
\pgfsys@useobject{currentmarker}{}%
\end{pgfscope}%
\begin{pgfscope}%
\pgfsys@transformshift{3.634870in}{2.988453in}%
\pgfsys@useobject{currentmarker}{}%
\end{pgfscope}%
\begin{pgfscope}%
\pgfsys@transformshift{4.036515in}{3.006425in}%
\pgfsys@useobject{currentmarker}{}%
\end{pgfscope}%
\begin{pgfscope}%
\pgfsys@transformshift{4.438160in}{2.922587in}%
\pgfsys@useobject{currentmarker}{}%
\end{pgfscope}%
\begin{pgfscope}%
\pgfsys@transformshift{4.839805in}{2.736931in}%
\pgfsys@useobject{currentmarker}{}%
\end{pgfscope}%
\begin{pgfscope}%
\pgfsys@transformshift{5.241450in}{2.449437in}%
\pgfsys@useobject{currentmarker}{}%
\end{pgfscope}%
\begin{pgfscope}%
\pgfsys@transformshift{5.643094in}{2.060078in}%
\pgfsys@useobject{currentmarker}{}%
\end{pgfscope}%
\end{pgfscope}%
\begin{pgfscope}%
\pgfsetrectcap%
\pgfsetmiterjoin%
\pgfsetlinewidth{0.803000pt}%
\definecolor{currentstroke}{rgb}{0.000000,0.000000,0.000000}%
\pgfsetstrokecolor{currentstroke}%
\pgfsetdash{}{0pt}%
\pgfpathmoveto{\pgfqpoint{0.984216in}{0.741795in}}%
\pgfpathlineto{\pgfqpoint{0.984216in}{4.142955in}}%
\pgfusepath{stroke}%
\end{pgfscope}%
\begin{pgfscope}%
\pgfsetrectcap%
\pgfsetmiterjoin%
\pgfsetlinewidth{0.803000pt}%
\definecolor{currentstroke}{rgb}{0.000000,0.000000,0.000000}%
\pgfsetstrokecolor{currentstroke}%
\pgfsetdash{}{0pt}%
\pgfpathmoveto{\pgfqpoint{5.442272in}{0.741795in}}%
\pgfpathlineto{\pgfqpoint{5.442272in}{4.142955in}}%
\pgfusepath{stroke}%
\end{pgfscope}%
\begin{pgfscope}%
\pgfsetrectcap%
\pgfsetmiterjoin%
\pgfsetlinewidth{0.803000pt}%
\definecolor{currentstroke}{rgb}{0.000000,0.000000,0.000000}%
\pgfsetstrokecolor{currentstroke}%
\pgfsetdash{}{0pt}%
\pgfpathmoveto{\pgfqpoint{0.984216in}{0.741795in}}%
\pgfpathlineto{\pgfqpoint{5.442272in}{0.741795in}}%
\pgfusepath{stroke}%
\end{pgfscope}%
\begin{pgfscope}%
\pgfsetrectcap%
\pgfsetmiterjoin%
\pgfsetlinewidth{0.803000pt}%
\definecolor{currentstroke}{rgb}{0.000000,0.000000,0.000000}%
\pgfsetstrokecolor{currentstroke}%
\pgfsetdash{}{0pt}%
\pgfpathmoveto{\pgfqpoint{0.984216in}{4.142955in}}%
\pgfpathlineto{\pgfqpoint{5.442272in}{4.142955in}}%
\pgfusepath{stroke}%
\end{pgfscope}%
\begin{pgfscope}%
\pgfsetbuttcap%
\pgfsetmiterjoin%
\definecolor{currentfill}{rgb}{1.000000,1.000000,1.000000}%
\pgfsetfillcolor{currentfill}%
\pgfsetfillopacity{0.800000}%
\pgfsetlinewidth{1.003750pt}%
\definecolor{currentstroke}{rgb}{0.800000,0.800000,0.800000}%
\pgfsetstrokecolor{currentstroke}%
\pgfsetstrokeopacity{0.800000}%
\pgfsetdash{}{0pt}%
\pgfpathmoveto{\pgfqpoint{1.120327in}{3.602023in}}%
\pgfpathlineto{\pgfqpoint{1.992439in}{3.602023in}}%
\pgfpathquadraticcurveto{\pgfqpoint{2.031328in}{3.602023in}}{\pgfqpoint{2.031328in}{3.640912in}}%
\pgfpathlineto{\pgfqpoint{2.031328in}{4.006844in}}%
\pgfpathquadraticcurveto{\pgfqpoint{2.031328in}{4.045733in}}{\pgfqpoint{1.992439in}{4.045733in}}%
\pgfpathlineto{\pgfqpoint{1.120327in}{4.045733in}}%
\pgfpathquadraticcurveto{\pgfqpoint{1.081438in}{4.045733in}}{\pgfqpoint{1.081438in}{4.006844in}}%
\pgfpathlineto{\pgfqpoint{1.081438in}{3.640912in}}%
\pgfpathquadraticcurveto{\pgfqpoint{1.081438in}{3.602023in}}{\pgfqpoint{1.120327in}{3.602023in}}%
\pgfpathclose%
\pgfusepath{stroke,fill}%
\end{pgfscope}%
\begin{pgfscope}%
\pgftext[x=1.159216in,y=3.820223in,left,base]{\rmfamily\fontsize{14.000000}{16.800000}\selectfont \(\displaystyle \mathbf{I}\mbox{g} = \) 0.5}%
\end{pgfscope}%
\begin{pgfscope}%
\pgfsetbuttcap%
\pgfsetmiterjoin%
\definecolor{currentfill}{rgb}{1.000000,1.000000,1.000000}%
\pgfsetfillcolor{currentfill}%
\pgfsetlinewidth{0.000000pt}%
\definecolor{currentstroke}{rgb}{0.000000,0.000000,0.000000}%
\pgfsetstrokecolor{currentstroke}%
\pgfsetstrokeopacity{0.000000}%
\pgfsetdash{}{0pt}%
\pgfpathmoveto{\pgfqpoint{5.697941in}{0.741795in}}%
\pgfpathlineto{\pgfqpoint{10.155998in}{0.741795in}}%
\pgfpathlineto{\pgfqpoint{10.155998in}{4.142955in}}%
\pgfpathlineto{\pgfqpoint{5.697941in}{4.142955in}}%
\pgfpathclose%
\pgfusepath{fill}%
\end{pgfscope}%
\begin{pgfscope}%
\pgfsetbuttcap%
\pgfsetroundjoin%
\definecolor{currentfill}{rgb}{0.000000,0.000000,0.000000}%
\pgfsetfillcolor{currentfill}%
\pgfsetlinewidth{0.803000pt}%
\definecolor{currentstroke}{rgb}{0.000000,0.000000,0.000000}%
\pgfsetstrokecolor{currentstroke}%
\pgfsetdash{}{0pt}%
\pgfsys@defobject{currentmarker}{\pgfqpoint{0.000000in}{-0.048611in}}{\pgfqpoint{0.000000in}{0.000000in}}{%
\pgfpathmoveto{\pgfqpoint{0.000000in}{0.000000in}}%
\pgfpathlineto{\pgfqpoint{0.000000in}{-0.048611in}}%
\pgfusepath{stroke,fill}%
}%
\begin{pgfscope}%
\pgfsys@transformshift{5.938727in}{0.741795in}%
\pgfsys@useobject{currentmarker}{}%
\end{pgfscope}%
\end{pgfscope}%
\begin{pgfscope}%
\pgftext[x=5.938727in,y=0.644572in,,top]{\rmfamily\fontsize{16.000000}{19.200000}\selectfont \(\displaystyle 0.0\)}%
\end{pgfscope}%
\begin{pgfscope}%
\pgfsetbuttcap%
\pgfsetroundjoin%
\definecolor{currentfill}{rgb}{0.000000,0.000000,0.000000}%
\pgfsetfillcolor{currentfill}%
\pgfsetlinewidth{0.803000pt}%
\definecolor{currentstroke}{rgb}{0.000000,0.000000,0.000000}%
\pgfsetstrokecolor{currentstroke}%
\pgfsetdash{}{0pt}%
\pgfsys@defobject{currentmarker}{\pgfqpoint{0.000000in}{-0.048611in}}{\pgfqpoint{0.000000in}{0.000000in}}{%
\pgfpathmoveto{\pgfqpoint{0.000000in}{0.000000in}}%
\pgfpathlineto{\pgfqpoint{0.000000in}{-0.048611in}}%
\pgfusepath{stroke,fill}%
}%
\begin{pgfscope}%
\pgfsys@transformshift{6.742017in}{0.741795in}%
\pgfsys@useobject{currentmarker}{}%
\end{pgfscope}%
\end{pgfscope}%
\begin{pgfscope}%
\pgftext[x=6.742017in,y=0.644572in,,top]{\rmfamily\fontsize{16.000000}{19.200000}\selectfont \(\displaystyle 0.2\)}%
\end{pgfscope}%
\begin{pgfscope}%
\pgfsetbuttcap%
\pgfsetroundjoin%
\definecolor{currentfill}{rgb}{0.000000,0.000000,0.000000}%
\pgfsetfillcolor{currentfill}%
\pgfsetlinewidth{0.803000pt}%
\definecolor{currentstroke}{rgb}{0.000000,0.000000,0.000000}%
\pgfsetstrokecolor{currentstroke}%
\pgfsetdash{}{0pt}%
\pgfsys@defobject{currentmarker}{\pgfqpoint{0.000000in}{-0.048611in}}{\pgfqpoint{0.000000in}{0.000000in}}{%
\pgfpathmoveto{\pgfqpoint{0.000000in}{0.000000in}}%
\pgfpathlineto{\pgfqpoint{0.000000in}{-0.048611in}}%
\pgfusepath{stroke,fill}%
}%
\begin{pgfscope}%
\pgfsys@transformshift{7.545306in}{0.741795in}%
\pgfsys@useobject{currentmarker}{}%
\end{pgfscope}%
\end{pgfscope}%
\begin{pgfscope}%
\pgftext[x=7.545306in,y=0.644572in,,top]{\rmfamily\fontsize{16.000000}{19.200000}\selectfont \(\displaystyle 0.4\)}%
\end{pgfscope}%
\begin{pgfscope}%
\pgfsetbuttcap%
\pgfsetroundjoin%
\definecolor{currentfill}{rgb}{0.000000,0.000000,0.000000}%
\pgfsetfillcolor{currentfill}%
\pgfsetlinewidth{0.803000pt}%
\definecolor{currentstroke}{rgb}{0.000000,0.000000,0.000000}%
\pgfsetstrokecolor{currentstroke}%
\pgfsetdash{}{0pt}%
\pgfsys@defobject{currentmarker}{\pgfqpoint{0.000000in}{-0.048611in}}{\pgfqpoint{0.000000in}{0.000000in}}{%
\pgfpathmoveto{\pgfqpoint{0.000000in}{0.000000in}}%
\pgfpathlineto{\pgfqpoint{0.000000in}{-0.048611in}}%
\pgfusepath{stroke,fill}%
}%
\begin{pgfscope}%
\pgfsys@transformshift{8.348596in}{0.741795in}%
\pgfsys@useobject{currentmarker}{}%
\end{pgfscope}%
\end{pgfscope}%
\begin{pgfscope}%
\pgftext[x=8.348596in,y=0.644572in,,top]{\rmfamily\fontsize{16.000000}{19.200000}\selectfont \(\displaystyle 0.6\)}%
\end{pgfscope}%
\begin{pgfscope}%
\pgfsetbuttcap%
\pgfsetroundjoin%
\definecolor{currentfill}{rgb}{0.000000,0.000000,0.000000}%
\pgfsetfillcolor{currentfill}%
\pgfsetlinewidth{0.803000pt}%
\definecolor{currentstroke}{rgb}{0.000000,0.000000,0.000000}%
\pgfsetstrokecolor{currentstroke}%
\pgfsetdash{}{0pt}%
\pgfsys@defobject{currentmarker}{\pgfqpoint{0.000000in}{-0.048611in}}{\pgfqpoint{0.000000in}{0.000000in}}{%
\pgfpathmoveto{\pgfqpoint{0.000000in}{0.000000in}}%
\pgfpathlineto{\pgfqpoint{0.000000in}{-0.048611in}}%
\pgfusepath{stroke,fill}%
}%
\begin{pgfscope}%
\pgfsys@transformshift{9.151886in}{0.741795in}%
\pgfsys@useobject{currentmarker}{}%
\end{pgfscope}%
\end{pgfscope}%
\begin{pgfscope}%
\pgftext[x=9.151886in,y=0.644572in,,top]{\rmfamily\fontsize{16.000000}{19.200000}\selectfont \(\displaystyle 0.8\)}%
\end{pgfscope}%
\begin{pgfscope}%
\pgfsetbuttcap%
\pgfsetroundjoin%
\definecolor{currentfill}{rgb}{0.000000,0.000000,0.000000}%
\pgfsetfillcolor{currentfill}%
\pgfsetlinewidth{0.803000pt}%
\definecolor{currentstroke}{rgb}{0.000000,0.000000,0.000000}%
\pgfsetstrokecolor{currentstroke}%
\pgfsetdash{}{0pt}%
\pgfsys@defobject{currentmarker}{\pgfqpoint{0.000000in}{-0.048611in}}{\pgfqpoint{0.000000in}{0.000000in}}{%
\pgfpathmoveto{\pgfqpoint{0.000000in}{0.000000in}}%
\pgfpathlineto{\pgfqpoint{0.000000in}{-0.048611in}}%
\pgfusepath{stroke,fill}%
}%
\begin{pgfscope}%
\pgfsys@transformshift{9.955175in}{0.741795in}%
\pgfsys@useobject{currentmarker}{}%
\end{pgfscope}%
\end{pgfscope}%
\begin{pgfscope}%
\pgftext[x=9.955175in,y=0.644572in,,top]{\rmfamily\fontsize{16.000000}{19.200000}\selectfont \(\displaystyle 1.0\)}%
\end{pgfscope}%
\begin{pgfscope}%
\pgfsetbuttcap%
\pgfsetroundjoin%
\definecolor{currentfill}{rgb}{0.000000,0.000000,0.000000}%
\pgfsetfillcolor{currentfill}%
\pgfsetlinewidth{0.803000pt}%
\definecolor{currentstroke}{rgb}{0.000000,0.000000,0.000000}%
\pgfsetstrokecolor{currentstroke}%
\pgfsetdash{}{0pt}%
\pgfsys@defobject{currentmarker}{\pgfqpoint{-0.048611in}{0.000000in}}{\pgfqpoint{0.000000in}{0.000000in}}{%
\pgfpathmoveto{\pgfqpoint{0.000000in}{0.000000in}}%
\pgfpathlineto{\pgfqpoint{-0.048611in}{0.000000in}}%
\pgfusepath{stroke,fill}%
}%
\begin{pgfscope}%
\pgfsys@transformshift{5.697941in}{0.741795in}%
\pgfsys@useobject{currentmarker}{}%
\end{pgfscope}%
\end{pgfscope}%
\begin{pgfscope}%
\pgfsetbuttcap%
\pgfsetroundjoin%
\definecolor{currentfill}{rgb}{0.000000,0.000000,0.000000}%
\pgfsetfillcolor{currentfill}%
\pgfsetlinewidth{0.803000pt}%
\definecolor{currentstroke}{rgb}{0.000000,0.000000,0.000000}%
\pgfsetstrokecolor{currentstroke}%
\pgfsetdash{}{0pt}%
\pgfsys@defobject{currentmarker}{\pgfqpoint{-0.048611in}{0.000000in}}{\pgfqpoint{0.000000in}{0.000000in}}{%
\pgfpathmoveto{\pgfqpoint{0.000000in}{0.000000in}}%
\pgfpathlineto{\pgfqpoint{-0.048611in}{0.000000in}}%
\pgfusepath{stroke,fill}%
}%
\begin{pgfscope}%
\pgfsys@transformshift{5.697941in}{1.422027in}%
\pgfsys@useobject{currentmarker}{}%
\end{pgfscope}%
\end{pgfscope}%
\begin{pgfscope}%
\pgfsetbuttcap%
\pgfsetroundjoin%
\definecolor{currentfill}{rgb}{0.000000,0.000000,0.000000}%
\pgfsetfillcolor{currentfill}%
\pgfsetlinewidth{0.803000pt}%
\definecolor{currentstroke}{rgb}{0.000000,0.000000,0.000000}%
\pgfsetstrokecolor{currentstroke}%
\pgfsetdash{}{0pt}%
\pgfsys@defobject{currentmarker}{\pgfqpoint{-0.048611in}{0.000000in}}{\pgfqpoint{0.000000in}{0.000000in}}{%
\pgfpathmoveto{\pgfqpoint{0.000000in}{0.000000in}}%
\pgfpathlineto{\pgfqpoint{-0.048611in}{0.000000in}}%
\pgfusepath{stroke,fill}%
}%
\begin{pgfscope}%
\pgfsys@transformshift{5.697941in}{2.102259in}%
\pgfsys@useobject{currentmarker}{}%
\end{pgfscope}%
\end{pgfscope}%
\begin{pgfscope}%
\pgfsetbuttcap%
\pgfsetroundjoin%
\definecolor{currentfill}{rgb}{0.000000,0.000000,0.000000}%
\pgfsetfillcolor{currentfill}%
\pgfsetlinewidth{0.803000pt}%
\definecolor{currentstroke}{rgb}{0.000000,0.000000,0.000000}%
\pgfsetstrokecolor{currentstroke}%
\pgfsetdash{}{0pt}%
\pgfsys@defobject{currentmarker}{\pgfqpoint{-0.048611in}{0.000000in}}{\pgfqpoint{0.000000in}{0.000000in}}{%
\pgfpathmoveto{\pgfqpoint{0.000000in}{0.000000in}}%
\pgfpathlineto{\pgfqpoint{-0.048611in}{0.000000in}}%
\pgfusepath{stroke,fill}%
}%
\begin{pgfscope}%
\pgfsys@transformshift{5.697941in}{2.782491in}%
\pgfsys@useobject{currentmarker}{}%
\end{pgfscope}%
\end{pgfscope}%
\begin{pgfscope}%
\pgfsetbuttcap%
\pgfsetroundjoin%
\definecolor{currentfill}{rgb}{0.000000,0.000000,0.000000}%
\pgfsetfillcolor{currentfill}%
\pgfsetlinewidth{0.803000pt}%
\definecolor{currentstroke}{rgb}{0.000000,0.000000,0.000000}%
\pgfsetstrokecolor{currentstroke}%
\pgfsetdash{}{0pt}%
\pgfsys@defobject{currentmarker}{\pgfqpoint{-0.048611in}{0.000000in}}{\pgfqpoint{0.000000in}{0.000000in}}{%
\pgfpathmoveto{\pgfqpoint{0.000000in}{0.000000in}}%
\pgfpathlineto{\pgfqpoint{-0.048611in}{0.000000in}}%
\pgfusepath{stroke,fill}%
}%
\begin{pgfscope}%
\pgfsys@transformshift{5.697941in}{3.462723in}%
\pgfsys@useobject{currentmarker}{}%
\end{pgfscope}%
\end{pgfscope}%
\begin{pgfscope}%
\pgfsetbuttcap%
\pgfsetroundjoin%
\definecolor{currentfill}{rgb}{0.000000,0.000000,0.000000}%
\pgfsetfillcolor{currentfill}%
\pgfsetlinewidth{0.803000pt}%
\definecolor{currentstroke}{rgb}{0.000000,0.000000,0.000000}%
\pgfsetstrokecolor{currentstroke}%
\pgfsetdash{}{0pt}%
\pgfsys@defobject{currentmarker}{\pgfqpoint{-0.048611in}{0.000000in}}{\pgfqpoint{0.000000in}{0.000000in}}{%
\pgfpathmoveto{\pgfqpoint{0.000000in}{0.000000in}}%
\pgfpathlineto{\pgfqpoint{-0.048611in}{0.000000in}}%
\pgfusepath{stroke,fill}%
}%
\begin{pgfscope}%
\pgfsys@transformshift{5.697941in}{4.142955in}%
\pgfsys@useobject{currentmarker}{}%
\end{pgfscope}%
\end{pgfscope}%
\begin{pgfscope}%
\pgfpathrectangle{\pgfqpoint{5.697941in}{0.741795in}}{\pgfqpoint{4.458056in}{3.401160in}} %
\pgfusepath{clip}%
\pgfsetbuttcap%
\pgfsetroundjoin%
\pgfsetlinewidth{1.505625pt}%
\definecolor{currentstroke}{rgb}{1.000000,0.000000,0.000000}%
\pgfsetstrokecolor{currentstroke}%
\pgfsetdash{{5.550000pt}{2.400000pt}}{0.000000pt}%
\pgfpathmoveto{\pgfqpoint{5.938727in}{0.741795in}}%
\pgfpathlineto{\pgfqpoint{5.998974in}{0.843056in}}%
\pgfpathlineto{\pgfqpoint{6.055204in}{0.936173in}}%
\pgfpathlineto{\pgfqpoint{6.111434in}{1.027944in}}%
\pgfpathlineto{\pgfqpoint{6.167665in}{1.118370in}}%
\pgfpathlineto{\pgfqpoint{6.223895in}{1.207450in}}%
\pgfpathlineto{\pgfqpoint{6.280125in}{1.295185in}}%
\pgfpathlineto{\pgfqpoint{6.336356in}{1.381576in}}%
\pgfpathlineto{\pgfqpoint{6.392586in}{1.466622in}}%
\pgfpathlineto{\pgfqpoint{6.448816in}{1.550325in}}%
\pgfpathlineto{\pgfqpoint{6.505046in}{1.632683in}}%
\pgfpathlineto{\pgfqpoint{6.561277in}{1.713697in}}%
\pgfpathlineto{\pgfqpoint{6.617507in}{1.793368in}}%
\pgfpathlineto{\pgfqpoint{6.673737in}{1.871696in}}%
\pgfpathlineto{\pgfqpoint{6.729967in}{1.948681in}}%
\pgfpathlineto{\pgfqpoint{6.786198in}{2.024323in}}%
\pgfpathlineto{\pgfqpoint{6.838412in}{2.093359in}}%
\pgfpathlineto{\pgfqpoint{6.890625in}{2.161238in}}%
\pgfpathlineto{\pgfqpoint{6.942839in}{2.227960in}}%
\pgfpathlineto{\pgfqpoint{6.995053in}{2.293524in}}%
\pgfpathlineto{\pgfqpoint{7.047267in}{2.357931in}}%
\pgfpathlineto{\pgfqpoint{7.099481in}{2.421181in}}%
\pgfpathlineto{\pgfqpoint{7.151694in}{2.483273in}}%
\pgfpathlineto{\pgfqpoint{7.203908in}{2.544209in}}%
\pgfpathlineto{\pgfqpoint{7.256122in}{2.603988in}}%
\pgfpathlineto{\pgfqpoint{7.308336in}{2.662610in}}%
\pgfpathlineto{\pgfqpoint{7.360550in}{2.720076in}}%
\pgfpathlineto{\pgfqpoint{7.412764in}{2.776385in}}%
\pgfpathlineto{\pgfqpoint{7.464977in}{2.831537in}}%
\pgfpathlineto{\pgfqpoint{7.517191in}{2.885533in}}%
\pgfpathlineto{\pgfqpoint{7.565389in}{2.934349in}}%
\pgfpathlineto{\pgfqpoint{7.613586in}{2.982180in}}%
\pgfpathlineto{\pgfqpoint{7.661783in}{3.029025in}}%
\pgfpathlineto{\pgfqpoint{7.709981in}{3.074885in}}%
\pgfpathlineto{\pgfqpoint{7.758178in}{3.119760in}}%
\pgfpathlineto{\pgfqpoint{7.806375in}{3.163650in}}%
\pgfpathlineto{\pgfqpoint{7.854573in}{3.206555in}}%
\pgfpathlineto{\pgfqpoint{7.902770in}{3.248475in}}%
\pgfpathlineto{\pgfqpoint{7.950968in}{3.289410in}}%
\pgfpathlineto{\pgfqpoint{7.999165in}{3.329360in}}%
\pgfpathlineto{\pgfqpoint{8.047362in}{3.368325in}}%
\pgfpathlineto{\pgfqpoint{8.095560in}{3.406304in}}%
\pgfpathlineto{\pgfqpoint{8.143757in}{3.443299in}}%
\pgfpathlineto{\pgfqpoint{8.191954in}{3.479309in}}%
\pgfpathlineto{\pgfqpoint{8.240152in}{3.514334in}}%
\pgfpathlineto{\pgfqpoint{8.288349in}{3.548374in}}%
\pgfpathlineto{\pgfqpoint{8.332530in}{3.578712in}}%
\pgfpathlineto{\pgfqpoint{8.376711in}{3.608223in}}%
\pgfpathlineto{\pgfqpoint{8.420892in}{3.636906in}}%
\pgfpathlineto{\pgfqpoint{8.465073in}{3.664761in}}%
\pgfpathlineto{\pgfqpoint{8.509254in}{3.691789in}}%
\pgfpathlineto{\pgfqpoint{8.553435in}{3.717990in}}%
\pgfpathlineto{\pgfqpoint{8.597616in}{3.743362in}}%
\pgfpathlineto{\pgfqpoint{8.641797in}{3.767907in}}%
\pgfpathlineto{\pgfqpoint{8.685978in}{3.791625in}}%
\pgfpathlineto{\pgfqpoint{8.730159in}{3.814515in}}%
\pgfpathlineto{\pgfqpoint{8.774339in}{3.836577in}}%
\pgfpathlineto{\pgfqpoint{8.818520in}{3.857812in}}%
\pgfpathlineto{\pgfqpoint{8.862701in}{3.878219in}}%
\pgfpathlineto{\pgfqpoint{8.906882in}{3.897799in}}%
\pgfpathlineto{\pgfqpoint{8.951063in}{3.916551in}}%
\pgfpathlineto{\pgfqpoint{8.995244in}{3.934476in}}%
\pgfpathlineto{\pgfqpoint{9.039425in}{3.951573in}}%
\pgfpathlineto{\pgfqpoint{9.083606in}{3.967842in}}%
\pgfpathlineto{\pgfqpoint{9.127787in}{3.983284in}}%
\pgfpathlineto{\pgfqpoint{9.171968in}{3.997898in}}%
\pgfpathlineto{\pgfqpoint{9.216149in}{4.011684in}}%
\pgfpathlineto{\pgfqpoint{9.260330in}{4.024643in}}%
\pgfpathlineto{\pgfqpoint{9.304511in}{4.036774in}}%
\pgfpathlineto{\pgfqpoint{9.348691in}{4.048078in}}%
\pgfpathlineto{\pgfqpoint{9.392872in}{4.058554in}}%
\pgfpathlineto{\pgfqpoint{9.437053in}{4.068203in}}%
\pgfpathlineto{\pgfqpoint{9.481234in}{4.077023in}}%
\pgfpathlineto{\pgfqpoint{9.525415in}{4.085017in}}%
\pgfpathlineto{\pgfqpoint{9.565580in}{4.091565in}}%
\pgfpathlineto{\pgfqpoint{9.605744in}{4.097429in}}%
\pgfpathlineto{\pgfqpoint{9.645909in}{4.102610in}}%
\pgfpathlineto{\pgfqpoint{9.686073in}{4.107106in}}%
\pgfpathlineto{\pgfqpoint{9.726238in}{4.110918in}}%
\pgfpathlineto{\pgfqpoint{9.766402in}{4.114047in}}%
\pgfpathlineto{\pgfqpoint{9.806567in}{4.116491in}}%
\pgfpathlineto{\pgfqpoint{9.846731in}{4.118251in}}%
\pgfpathlineto{\pgfqpoint{9.886896in}{4.119328in}}%
\pgfpathlineto{\pgfqpoint{9.927060in}{4.119720in}}%
\pgfpathlineto{\pgfqpoint{9.967224in}{4.119428in}}%
\pgfpathlineto{\pgfqpoint{10.007389in}{4.118453in}}%
\pgfpathlineto{\pgfqpoint{10.047553in}{4.116793in}}%
\pgfpathlineto{\pgfqpoint{10.087718in}{4.114450in}}%
\pgfpathlineto{\pgfqpoint{10.127882in}{4.111422in}}%
\pgfpathlineto{\pgfqpoint{10.159331in}{4.108573in}}%
\pgfpathlineto{\pgfqpoint{10.159331in}{4.108573in}}%
\pgfusepath{stroke}%
\end{pgfscope}%
\begin{pgfscope}%
\pgfpathrectangle{\pgfqpoint{5.697941in}{0.741795in}}{\pgfqpoint{4.458056in}{3.401160in}} %
\pgfusepath{clip}%
\pgfsetbuttcap%
\pgfsetmiterjoin%
\definecolor{currentfill}{rgb}{1.000000,0.000000,0.000000}%
\pgfsetfillcolor{currentfill}%
\pgfsetlinewidth{1.003750pt}%
\definecolor{currentstroke}{rgb}{1.000000,0.000000,0.000000}%
\pgfsetstrokecolor{currentstroke}%
\pgfsetdash{}{0pt}%
\pgfsys@defobject{currentmarker}{\pgfqpoint{-0.041667in}{-0.041667in}}{\pgfqpoint{0.041667in}{0.041667in}}{%
\pgfpathmoveto{\pgfqpoint{-0.041667in}{-0.041667in}}%
\pgfpathlineto{\pgfqpoint{0.041667in}{-0.041667in}}%
\pgfpathlineto{\pgfqpoint{0.041667in}{0.041667in}}%
\pgfpathlineto{\pgfqpoint{-0.041667in}{0.041667in}}%
\pgfpathclose%
\pgfusepath{stroke,fill}%
}%
\begin{pgfscope}%
\pgfsys@transformshift{5.938727in}{0.741795in}%
\pgfsys@useobject{currentmarker}{}%
\end{pgfscope}%
\begin{pgfscope}%
\pgfsys@transformshift{6.340372in}{1.387695in}%
\pgfsys@useobject{currentmarker}{}%
\end{pgfscope}%
\begin{pgfscope}%
\pgfsys@transformshift{6.742017in}{1.965003in}%
\pgfsys@useobject{currentmarker}{}%
\end{pgfscope}%
\begin{pgfscope}%
\pgfsys@transformshift{7.143662in}{2.473796in}%
\pgfsys@useobject{currentmarker}{}%
\end{pgfscope}%
\begin{pgfscope}%
\pgfsys@transformshift{7.545306in}{2.914129in}%
\pgfsys@useobject{currentmarker}{}%
\end{pgfscope}%
\begin{pgfscope}%
\pgfsys@transformshift{7.946951in}{3.286036in}%
\pgfsys@useobject{currentmarker}{}%
\end{pgfscope}%
\begin{pgfscope}%
\pgfsys@transformshift{8.348596in}{3.589539in}%
\pgfsys@useobject{currentmarker}{}%
\end{pgfscope}%
\begin{pgfscope}%
\pgfsys@transformshift{8.750241in}{3.824646in}%
\pgfsys@useobject{currentmarker}{}%
\end{pgfscope}%
\begin{pgfscope}%
\pgfsys@transformshift{9.151886in}{3.991358in}%
\pgfsys@useobject{currentmarker}{}%
\end{pgfscope}%
\begin{pgfscope}%
\pgfsys@transformshift{9.553530in}{4.089672in}%
\pgfsys@useobject{currentmarker}{}%
\end{pgfscope}%
\begin{pgfscope}%
\pgfsys@transformshift{9.955175in}{4.119588in}%
\pgfsys@useobject{currentmarker}{}%
\end{pgfscope}%
\begin{pgfscope}%
\pgfsys@transformshift{10.356820in}{4.081104in}%
\pgfsys@useobject{currentmarker}{}%
\end{pgfscope}%
\end{pgfscope}%
\begin{pgfscope}%
\pgfpathrectangle{\pgfqpoint{5.697941in}{0.741795in}}{\pgfqpoint{4.458056in}{3.401160in}} %
\pgfusepath{clip}%
\pgfsetrectcap%
\pgfsetroundjoin%
\pgfsetlinewidth{1.505625pt}%
\definecolor{currentstroke}{rgb}{0.000000,0.000000,1.000000}%
\pgfsetstrokecolor{currentstroke}%
\pgfsetdash{}{0pt}%
\pgfpathmoveto{\pgfqpoint{5.938727in}{0.741795in}}%
\pgfpathlineto{\pgfqpoint{5.998974in}{0.843056in}}%
\pgfpathlineto{\pgfqpoint{6.055204in}{0.936173in}}%
\pgfpathlineto{\pgfqpoint{6.111434in}{1.027943in}}%
\pgfpathlineto{\pgfqpoint{6.167665in}{1.118366in}}%
\pgfpathlineto{\pgfqpoint{6.223895in}{1.207443in}}%
\pgfpathlineto{\pgfqpoint{6.280125in}{1.295174in}}%
\pgfpathlineto{\pgfqpoint{6.336356in}{1.381558in}}%
\pgfpathlineto{\pgfqpoint{6.392586in}{1.466596in}}%
\pgfpathlineto{\pgfqpoint{6.448816in}{1.550288in}}%
\pgfpathlineto{\pgfqpoint{6.505046in}{1.632633in}}%
\pgfpathlineto{\pgfqpoint{6.561277in}{1.713632in}}%
\pgfpathlineto{\pgfqpoint{6.617507in}{1.793285in}}%
\pgfpathlineto{\pgfqpoint{6.673737in}{1.871592in}}%
\pgfpathlineto{\pgfqpoint{6.729967in}{1.948552in}}%
\pgfpathlineto{\pgfqpoint{6.782181in}{2.018810in}}%
\pgfpathlineto{\pgfqpoint{6.834395in}{2.087908in}}%
\pgfpathlineto{\pgfqpoint{6.886609in}{2.155844in}}%
\pgfpathlineto{\pgfqpoint{6.938823in}{2.222620in}}%
\pgfpathlineto{\pgfqpoint{6.991037in}{2.288236in}}%
\pgfpathlineto{\pgfqpoint{7.043250in}{2.352691in}}%
\pgfpathlineto{\pgfqpoint{7.095464in}{2.415985in}}%
\pgfpathlineto{\pgfqpoint{7.147678in}{2.478119in}}%
\pgfpathlineto{\pgfqpoint{7.199892in}{2.539092in}}%
\pgfpathlineto{\pgfqpoint{7.252106in}{2.598905in}}%
\pgfpathlineto{\pgfqpoint{7.304319in}{2.657557in}}%
\pgfpathlineto{\pgfqpoint{7.356533in}{2.715049in}}%
\pgfpathlineto{\pgfqpoint{7.408747in}{2.771380in}}%
\pgfpathlineto{\pgfqpoint{7.460961in}{2.826551in}}%
\pgfpathlineto{\pgfqpoint{7.513175in}{2.880562in}}%
\pgfpathlineto{\pgfqpoint{7.561372in}{2.929388in}}%
\pgfpathlineto{\pgfqpoint{7.609570in}{2.977225in}}%
\pgfpathlineto{\pgfqpoint{7.657767in}{3.024074in}}%
\pgfpathlineto{\pgfqpoint{7.705964in}{3.069933in}}%
\pgfpathlineto{\pgfqpoint{7.754162in}{3.114804in}}%
\pgfpathlineto{\pgfqpoint{7.802359in}{3.158687in}}%
\pgfpathlineto{\pgfqpoint{7.850556in}{3.201581in}}%
\pgfpathlineto{\pgfqpoint{7.898754in}{3.243486in}}%
\pgfpathlineto{\pgfqpoint{7.946951in}{3.284402in}}%
\pgfpathlineto{\pgfqpoint{7.995149in}{3.324330in}}%
\pgfpathlineto{\pgfqpoint{8.043346in}{3.363269in}}%
\pgfpathlineto{\pgfqpoint{8.091543in}{3.401220in}}%
\pgfpathlineto{\pgfqpoint{8.139741in}{3.438181in}}%
\pgfpathlineto{\pgfqpoint{8.187938in}{3.474155in}}%
\pgfpathlineto{\pgfqpoint{8.236135in}{3.509140in}}%
\pgfpathlineto{\pgfqpoint{8.284333in}{3.543136in}}%
\pgfpathlineto{\pgfqpoint{8.328514in}{3.573430in}}%
\pgfpathlineto{\pgfqpoint{8.372695in}{3.602895in}}%
\pgfpathlineto{\pgfqpoint{8.416876in}{3.631528in}}%
\pgfpathlineto{\pgfqpoint{8.461056in}{3.659331in}}%
\pgfpathlineto{\pgfqpoint{8.505237in}{3.686303in}}%
\pgfpathlineto{\pgfqpoint{8.549418in}{3.712444in}}%
\pgfpathlineto{\pgfqpoint{8.593599in}{3.737755in}}%
\pgfpathlineto{\pgfqpoint{8.637780in}{3.762235in}}%
\pgfpathlineto{\pgfqpoint{8.681961in}{3.785885in}}%
\pgfpathlineto{\pgfqpoint{8.726142in}{3.808704in}}%
\pgfpathlineto{\pgfqpoint{8.770323in}{3.830692in}}%
\pgfpathlineto{\pgfqpoint{8.814504in}{3.851850in}}%
\pgfpathlineto{\pgfqpoint{8.858685in}{3.872177in}}%
\pgfpathlineto{\pgfqpoint{8.902866in}{3.891673in}}%
\pgfpathlineto{\pgfqpoint{8.947047in}{3.910339in}}%
\pgfpathlineto{\pgfqpoint{8.991228in}{3.928175in}}%
\pgfpathlineto{\pgfqpoint{9.035409in}{3.945179in}}%
\pgfpathlineto{\pgfqpoint{9.079589in}{3.961354in}}%
\pgfpathlineto{\pgfqpoint{9.123770in}{3.976697in}}%
\pgfpathlineto{\pgfqpoint{9.167951in}{3.991210in}}%
\pgfpathlineto{\pgfqpoint{9.212132in}{4.004892in}}%
\pgfpathlineto{\pgfqpoint{9.256313in}{4.017744in}}%
\pgfpathlineto{\pgfqpoint{9.300494in}{4.029765in}}%
\pgfpathlineto{\pgfqpoint{9.344675in}{4.040956in}}%
\pgfpathlineto{\pgfqpoint{9.388856in}{4.051316in}}%
\pgfpathlineto{\pgfqpoint{9.433037in}{4.060846in}}%
\pgfpathlineto{\pgfqpoint{9.477218in}{4.069545in}}%
\pgfpathlineto{\pgfqpoint{9.517382in}{4.076732in}}%
\pgfpathlineto{\pgfqpoint{9.557547in}{4.083233in}}%
\pgfpathlineto{\pgfqpoint{9.597711in}{4.089048in}}%
\pgfpathlineto{\pgfqpoint{9.637876in}{4.094176in}}%
\pgfpathlineto{\pgfqpoint{9.678040in}{4.098618in}}%
\pgfpathlineto{\pgfqpoint{9.718205in}{4.102373in}}%
\pgfpathlineto{\pgfqpoint{9.758369in}{4.105442in}}%
\pgfpathlineto{\pgfqpoint{9.798534in}{4.107825in}}%
\pgfpathlineto{\pgfqpoint{9.838698in}{4.109521in}}%
\pgfpathlineto{\pgfqpoint{9.878863in}{4.110530in}}%
\pgfpathlineto{\pgfqpoint{9.919027in}{4.110854in}}%
\pgfpathlineto{\pgfqpoint{9.959192in}{4.110491in}}%
\pgfpathlineto{\pgfqpoint{9.999356in}{4.109441in}}%
\pgfpathlineto{\pgfqpoint{10.039521in}{4.107705in}}%
\pgfpathlineto{\pgfqpoint{10.079685in}{4.105283in}}%
\pgfpathlineto{\pgfqpoint{10.119849in}{4.102174in}}%
\pgfpathlineto{\pgfqpoint{10.159331in}{4.098449in}}%
\pgfpathlineto{\pgfqpoint{10.159331in}{4.098449in}}%
\pgfusepath{stroke}%
\end{pgfscope}%
\begin{pgfscope}%
\pgfpathrectangle{\pgfqpoint{5.697941in}{0.741795in}}{\pgfqpoint{4.458056in}{3.401160in}} %
\pgfusepath{clip}%
\pgfsetbuttcap%
\pgfsetroundjoin%
\definecolor{currentfill}{rgb}{0.000000,0.000000,1.000000}%
\pgfsetfillcolor{currentfill}%
\pgfsetlinewidth{1.003750pt}%
\definecolor{currentstroke}{rgb}{0.000000,0.000000,1.000000}%
\pgfsetstrokecolor{currentstroke}%
\pgfsetdash{}{0pt}%
\pgfsys@defobject{currentmarker}{\pgfqpoint{-0.041667in}{-0.041667in}}{\pgfqpoint{0.041667in}{0.041667in}}{%
\pgfpathmoveto{\pgfqpoint{0.000000in}{-0.041667in}}%
\pgfpathcurveto{\pgfqpoint{0.011050in}{-0.041667in}}{\pgfqpoint{0.021649in}{-0.037276in}}{\pgfqpoint{0.029463in}{-0.029463in}}%
\pgfpathcurveto{\pgfqpoint{0.037276in}{-0.021649in}}{\pgfqpoint{0.041667in}{-0.011050in}}{\pgfqpoint{0.041667in}{0.000000in}}%
\pgfpathcurveto{\pgfqpoint{0.041667in}{0.011050in}}{\pgfqpoint{0.037276in}{0.021649in}}{\pgfqpoint{0.029463in}{0.029463in}}%
\pgfpathcurveto{\pgfqpoint{0.021649in}{0.037276in}}{\pgfqpoint{0.011050in}{0.041667in}}{\pgfqpoint{0.000000in}{0.041667in}}%
\pgfpathcurveto{\pgfqpoint{-0.011050in}{0.041667in}}{\pgfqpoint{-0.021649in}{0.037276in}}{\pgfqpoint{-0.029463in}{0.029463in}}%
\pgfpathcurveto{\pgfqpoint{-0.037276in}{0.021649in}}{\pgfqpoint{-0.041667in}{0.011050in}}{\pgfqpoint{-0.041667in}{0.000000in}}%
\pgfpathcurveto{\pgfqpoint{-0.041667in}{-0.011050in}}{\pgfqpoint{-0.037276in}{-0.021649in}}{\pgfqpoint{-0.029463in}{-0.029463in}}%
\pgfpathcurveto{\pgfqpoint{-0.021649in}{-0.037276in}}{\pgfqpoint{-0.011050in}{-0.041667in}}{\pgfqpoint{0.000000in}{-0.041667in}}%
\pgfpathclose%
\pgfusepath{stroke,fill}%
}%
\begin{pgfscope}%
\pgfsys@transformshift{5.938727in}{0.741795in}%
\pgfsys@useobject{currentmarker}{}%
\end{pgfscope}%
\begin{pgfscope}%
\pgfsys@transformshift{6.340372in}{1.387677in}%
\pgfsys@useobject{currentmarker}{}%
\end{pgfscope}%
\begin{pgfscope}%
\pgfsys@transformshift{6.742017in}{1.964869in}%
\pgfsys@useobject{currentmarker}{}%
\end{pgfscope}%
\begin{pgfscope}%
\pgfsys@transformshift{7.143662in}{2.473381in}%
\pgfsys@useobject{currentmarker}{}%
\end{pgfscope}%
\begin{pgfscope}%
\pgfsys@transformshift{7.545306in}{2.913222in}%
\pgfsys@useobject{currentmarker}{}%
\end{pgfscope}%
\begin{pgfscope}%
\pgfsys@transformshift{7.946951in}{3.284402in}%
\pgfsys@useobject{currentmarker}{}%
\end{pgfscope}%
\begin{pgfscope}%
\pgfsys@transformshift{8.348596in}{3.586926in}%
\pgfsys@useobject{currentmarker}{}%
\end{pgfscope}%
\begin{pgfscope}%
\pgfsys@transformshift{8.750241in}{3.820800in}%
\pgfsys@useobject{currentmarker}{}%
\end{pgfscope}%
\begin{pgfscope}%
\pgfsys@transformshift{9.151886in}{3.986029in}%
\pgfsys@useobject{currentmarker}{}%
\end{pgfscope}%
\begin{pgfscope}%
\pgfsys@transformshift{9.553530in}{4.082614in}%
\pgfsys@useobject{currentmarker}{}%
\end{pgfscope}%
\begin{pgfscope}%
\pgfsys@transformshift{9.955175in}{4.110558in}%
\pgfsys@useobject{currentmarker}{}%
\end{pgfscope}%
\begin{pgfscope}%
\pgfsys@transformshift{10.356820in}{4.069861in}%
\pgfsys@useobject{currentmarker}{}%
\end{pgfscope}%
\end{pgfscope}%
\begin{pgfscope}%
\pgfpathrectangle{\pgfqpoint{5.697941in}{0.741795in}}{\pgfqpoint{4.458056in}{3.401160in}} %
\pgfusepath{clip}%
\pgfsetbuttcap%
\pgfsetroundjoin%
\pgfsetlinewidth{1.505625pt}%
\definecolor{currentstroke}{rgb}{0.000000,0.750000,0.750000}%
\pgfsetstrokecolor{currentstroke}%
\pgfsetdash{{9.600000pt}{2.400000pt}{1.500000pt}{2.400000pt}}{0.000000pt}%
\pgfpathmoveto{\pgfqpoint{5.938727in}{0.741795in}}%
\pgfpathlineto{\pgfqpoint{5.998974in}{0.843056in}}%
\pgfpathlineto{\pgfqpoint{6.055204in}{0.936173in}}%
\pgfpathlineto{\pgfqpoint{6.111434in}{1.027943in}}%
\pgfpathlineto{\pgfqpoint{6.167665in}{1.118366in}}%
\pgfpathlineto{\pgfqpoint{6.223895in}{1.207443in}}%
\pgfpathlineto{\pgfqpoint{6.280125in}{1.295173in}}%
\pgfpathlineto{\pgfqpoint{6.336356in}{1.381556in}}%
\pgfpathlineto{\pgfqpoint{6.392586in}{1.466593in}}%
\pgfpathlineto{\pgfqpoint{6.448816in}{1.550284in}}%
\pgfpathlineto{\pgfqpoint{6.505046in}{1.632628in}}%
\pgfpathlineto{\pgfqpoint{6.561277in}{1.713625in}}%
\pgfpathlineto{\pgfqpoint{6.617507in}{1.793276in}}%
\pgfpathlineto{\pgfqpoint{6.673737in}{1.871580in}}%
\pgfpathlineto{\pgfqpoint{6.729967in}{1.948538in}}%
\pgfpathlineto{\pgfqpoint{6.782181in}{2.018793in}}%
\pgfpathlineto{\pgfqpoint{6.834395in}{2.087887in}}%
\pgfpathlineto{\pgfqpoint{6.886609in}{2.155820in}}%
\pgfpathlineto{\pgfqpoint{6.938823in}{2.222591in}}%
\pgfpathlineto{\pgfqpoint{6.991037in}{2.288202in}}%
\pgfpathlineto{\pgfqpoint{7.043250in}{2.352652in}}%
\pgfpathlineto{\pgfqpoint{7.095464in}{2.415941in}}%
\pgfpathlineto{\pgfqpoint{7.147678in}{2.478069in}}%
\pgfpathlineto{\pgfqpoint{7.199892in}{2.539035in}}%
\pgfpathlineto{\pgfqpoint{7.252106in}{2.598841in}}%
\pgfpathlineto{\pgfqpoint{7.304319in}{2.657486in}}%
\pgfpathlineto{\pgfqpoint{7.356533in}{2.714969in}}%
\pgfpathlineto{\pgfqpoint{7.408747in}{2.771292in}}%
\pgfpathlineto{\pgfqpoint{7.460961in}{2.826454in}}%
\pgfpathlineto{\pgfqpoint{7.509158in}{2.876342in}}%
\pgfpathlineto{\pgfqpoint{7.557356in}{2.925240in}}%
\pgfpathlineto{\pgfqpoint{7.605553in}{2.973150in}}%
\pgfpathlineto{\pgfqpoint{7.653750in}{3.020070in}}%
\pgfpathlineto{\pgfqpoint{7.701948in}{3.066000in}}%
\pgfpathlineto{\pgfqpoint{7.750145in}{3.110942in}}%
\pgfpathlineto{\pgfqpoint{7.798343in}{3.154894in}}%
\pgfpathlineto{\pgfqpoint{7.846540in}{3.197857in}}%
\pgfpathlineto{\pgfqpoint{7.894737in}{3.239831in}}%
\pgfpathlineto{\pgfqpoint{7.942935in}{3.280816in}}%
\pgfpathlineto{\pgfqpoint{7.991132in}{3.320811in}}%
\pgfpathlineto{\pgfqpoint{8.039329in}{3.359817in}}%
\pgfpathlineto{\pgfqpoint{8.087527in}{3.397834in}}%
\pgfpathlineto{\pgfqpoint{8.135724in}{3.434861in}}%
\pgfpathlineto{\pgfqpoint{8.183922in}{3.470900in}}%
\pgfpathlineto{\pgfqpoint{8.232119in}{3.505949in}}%
\pgfpathlineto{\pgfqpoint{8.280316in}{3.540008in}}%
\pgfpathlineto{\pgfqpoint{8.324497in}{3.570361in}}%
\pgfpathlineto{\pgfqpoint{8.368678in}{3.599882in}}%
\pgfpathlineto{\pgfqpoint{8.412859in}{3.628572in}}%
\pgfpathlineto{\pgfqpoint{8.457040in}{3.656431in}}%
\pgfpathlineto{\pgfqpoint{8.501221in}{3.683458in}}%
\pgfpathlineto{\pgfqpoint{8.545402in}{3.709654in}}%
\pgfpathlineto{\pgfqpoint{8.589583in}{3.735019in}}%
\pgfpathlineto{\pgfqpoint{8.633764in}{3.759553in}}%
\pgfpathlineto{\pgfqpoint{8.677945in}{3.783256in}}%
\pgfpathlineto{\pgfqpoint{8.722126in}{3.806127in}}%
\pgfpathlineto{\pgfqpoint{8.766307in}{3.828167in}}%
\pgfpathlineto{\pgfqpoint{8.810487in}{3.849376in}}%
\pgfpathlineto{\pgfqpoint{8.854668in}{3.869753in}}%
\pgfpathlineto{\pgfqpoint{8.898849in}{3.889299in}}%
\pgfpathlineto{\pgfqpoint{8.943030in}{3.908015in}}%
\pgfpathlineto{\pgfqpoint{8.987211in}{3.925898in}}%
\pgfpathlineto{\pgfqpoint{9.031392in}{3.942951in}}%
\pgfpathlineto{\pgfqpoint{9.075573in}{3.959172in}}%
\pgfpathlineto{\pgfqpoint{9.119754in}{3.974562in}}%
\pgfpathlineto{\pgfqpoint{9.163935in}{3.989121in}}%
\pgfpathlineto{\pgfqpoint{9.208116in}{4.002849in}}%
\pgfpathlineto{\pgfqpoint{9.252297in}{4.015745in}}%
\pgfpathlineto{\pgfqpoint{9.296478in}{4.027811in}}%
\pgfpathlineto{\pgfqpoint{9.340659in}{4.039045in}}%
\pgfpathlineto{\pgfqpoint{9.384840in}{4.049447in}}%
\pgfpathlineto{\pgfqpoint{9.429020in}{4.059019in}}%
\pgfpathlineto{\pgfqpoint{9.473201in}{4.067759in}}%
\pgfpathlineto{\pgfqpoint{9.513366in}{4.074983in}}%
\pgfpathlineto{\pgfqpoint{9.553530in}{4.081521in}}%
\pgfpathlineto{\pgfqpoint{9.593695in}{4.087371in}}%
\pgfpathlineto{\pgfqpoint{9.633859in}{4.092535in}}%
\pgfpathlineto{\pgfqpoint{9.674024in}{4.097011in}}%
\pgfpathlineto{\pgfqpoint{9.714188in}{4.100801in}}%
\pgfpathlineto{\pgfqpoint{9.754353in}{4.103903in}}%
\pgfpathlineto{\pgfqpoint{9.794517in}{4.106319in}}%
\pgfpathlineto{\pgfqpoint{9.834682in}{4.108048in}}%
\pgfpathlineto{\pgfqpoint{9.874846in}{4.109089in}}%
\pgfpathlineto{\pgfqpoint{9.915011in}{4.109444in}}%
\pgfpathlineto{\pgfqpoint{9.955175in}{4.109112in}}%
\pgfpathlineto{\pgfqpoint{9.995340in}{4.108093in}}%
\pgfpathlineto{\pgfqpoint{10.035504in}{4.106386in}}%
\pgfpathlineto{\pgfqpoint{10.075669in}{4.103993in}}%
\pgfpathlineto{\pgfqpoint{10.115833in}{4.100913in}}%
\pgfpathlineto{\pgfqpoint{10.159331in}{4.096802in}}%
\pgfpathlineto{\pgfqpoint{10.159331in}{4.096802in}}%
\pgfusepath{stroke}%
\end{pgfscope}%
\begin{pgfscope}%
\pgfpathrectangle{\pgfqpoint{5.697941in}{0.741795in}}{\pgfqpoint{4.458056in}{3.401160in}} %
\pgfusepath{clip}%
\pgfsetbuttcap%
\pgfsetmiterjoin%
\definecolor{currentfill}{rgb}{0.000000,0.750000,0.750000}%
\pgfsetfillcolor{currentfill}%
\pgfsetlinewidth{1.003750pt}%
\definecolor{currentstroke}{rgb}{0.000000,0.750000,0.750000}%
\pgfsetstrokecolor{currentstroke}%
\pgfsetdash{}{0pt}%
\pgfsys@defobject{currentmarker}{\pgfqpoint{-0.041667in}{-0.041667in}}{\pgfqpoint{0.041667in}{0.041667in}}{%
\pgfpathmoveto{\pgfqpoint{-0.000000in}{-0.041667in}}%
\pgfpathlineto{\pgfqpoint{0.041667in}{0.041667in}}%
\pgfpathlineto{\pgfqpoint{-0.041667in}{0.041667in}}%
\pgfpathclose%
\pgfusepath{stroke,fill}%
}%
\begin{pgfscope}%
\pgfsys@transformshift{5.938727in}{0.741795in}%
\pgfsys@useobject{currentmarker}{}%
\end{pgfscope}%
\begin{pgfscope}%
\pgfsys@transformshift{6.340372in}{1.387675in}%
\pgfsys@useobject{currentmarker}{}%
\end{pgfscope}%
\begin{pgfscope}%
\pgfsys@transformshift{6.742017in}{1.964853in}%
\pgfsys@useobject{currentmarker}{}%
\end{pgfscope}%
\begin{pgfscope}%
\pgfsys@transformshift{7.143662in}{2.473331in}%
\pgfsys@useobject{currentmarker}{}%
\end{pgfscope}%
\begin{pgfscope}%
\pgfsys@transformshift{7.545306in}{2.913108in}%
\pgfsys@useobject{currentmarker}{}%
\end{pgfscope}%
\begin{pgfscope}%
\pgfsys@transformshift{7.946951in}{3.284186in}%
\pgfsys@useobject{currentmarker}{}%
\end{pgfscope}%
\begin{pgfscope}%
\pgfsys@transformshift{8.348596in}{3.586566in}%
\pgfsys@useobject{currentmarker}{}%
\end{pgfscope}%
\begin{pgfscope}%
\pgfsys@transformshift{8.750241in}{3.820248in}%
\pgfsys@useobject{currentmarker}{}%
\end{pgfscope}%
\begin{pgfscope}%
\pgfsys@transformshift{9.151886in}{3.985233in}%
\pgfsys@useobject{currentmarker}{}%
\end{pgfscope}%
\begin{pgfscope}%
\pgfsys@transformshift{9.553530in}{4.081521in}%
\pgfsys@useobject{currentmarker}{}%
\end{pgfscope}%
\begin{pgfscope}%
\pgfsys@transformshift{9.955175in}{4.109112in}%
\pgfsys@useobject{currentmarker}{}%
\end{pgfscope}%
\begin{pgfscope}%
\pgfsys@transformshift{10.356820in}{4.068006in}%
\pgfsys@useobject{currentmarker}{}%
\end{pgfscope}%
\end{pgfscope}%
\begin{pgfscope}%
\pgfsetrectcap%
\pgfsetmiterjoin%
\pgfsetlinewidth{0.803000pt}%
\definecolor{currentstroke}{rgb}{0.000000,0.000000,0.000000}%
\pgfsetstrokecolor{currentstroke}%
\pgfsetdash{}{0pt}%
\pgfpathmoveto{\pgfqpoint{5.697941in}{0.741795in}}%
\pgfpathlineto{\pgfqpoint{5.697941in}{4.142955in}}%
\pgfusepath{stroke}%
\end{pgfscope}%
\begin{pgfscope}%
\pgfsetrectcap%
\pgfsetmiterjoin%
\pgfsetlinewidth{0.803000pt}%
\definecolor{currentstroke}{rgb}{0.000000,0.000000,0.000000}%
\pgfsetstrokecolor{currentstroke}%
\pgfsetdash{}{0pt}%
\pgfpathmoveto{\pgfqpoint{10.155998in}{0.741795in}}%
\pgfpathlineto{\pgfqpoint{10.155998in}{4.142955in}}%
\pgfusepath{stroke}%
\end{pgfscope}%
\begin{pgfscope}%
\pgfsetrectcap%
\pgfsetmiterjoin%
\pgfsetlinewidth{0.803000pt}%
\definecolor{currentstroke}{rgb}{0.000000,0.000000,0.000000}%
\pgfsetstrokecolor{currentstroke}%
\pgfsetdash{}{0pt}%
\pgfpathmoveto{\pgfqpoint{5.697941in}{0.741795in}}%
\pgfpathlineto{\pgfqpoint{10.155998in}{0.741795in}}%
\pgfusepath{stroke}%
\end{pgfscope}%
\begin{pgfscope}%
\pgfsetrectcap%
\pgfsetmiterjoin%
\pgfsetlinewidth{0.803000pt}%
\definecolor{currentstroke}{rgb}{0.000000,0.000000,0.000000}%
\pgfsetstrokecolor{currentstroke}%
\pgfsetdash{}{0pt}%
\pgfpathmoveto{\pgfqpoint{5.697941in}{4.142955in}}%
\pgfpathlineto{\pgfqpoint{10.155998in}{4.142955in}}%
\pgfusepath{stroke}%
\end{pgfscope}%
\begin{pgfscope}%
\pgfsetbuttcap%
\pgfsetmiterjoin%
\definecolor{currentfill}{rgb}{1.000000,1.000000,1.000000}%
\pgfsetfillcolor{currentfill}%
\pgfsetfillopacity{0.800000}%
\pgfsetlinewidth{1.003750pt}%
\definecolor{currentstroke}{rgb}{0.800000,0.800000,0.800000}%
\pgfsetstrokecolor{currentstroke}%
\pgfsetstrokeopacity{0.800000}%
\pgfsetdash{}{0pt}%
\pgfpathmoveto{\pgfqpoint{5.834052in}{3.602023in}}%
\pgfpathlineto{\pgfqpoint{6.829876in}{3.602023in}}%
\pgfpathquadraticcurveto{\pgfqpoint{6.868765in}{3.602023in}}{\pgfqpoint{6.868765in}{3.640912in}}%
\pgfpathlineto{\pgfqpoint{6.868765in}{4.006844in}}%
\pgfpathquadraticcurveto{\pgfqpoint{6.868765in}{4.045733in}}{\pgfqpoint{6.829876in}{4.045733in}}%
\pgfpathlineto{\pgfqpoint{5.834052in}{4.045733in}}%
\pgfpathquadraticcurveto{\pgfqpoint{5.795163in}{4.045733in}}{\pgfqpoint{5.795163in}{4.006844in}}%
\pgfpathlineto{\pgfqpoint{5.795163in}{3.640912in}}%
\pgfpathquadraticcurveto{\pgfqpoint{5.795163in}{3.602023in}}{\pgfqpoint{5.834052in}{3.602023in}}%
\pgfpathclose%
\pgfusepath{stroke,fill}%
\end{pgfscope}%
\begin{pgfscope}%
\pgftext[x=5.872941in,y=3.820223in,left,base]{\rmfamily\fontsize{14.000000}{16.800000}\selectfont \(\displaystyle \mathbf{I}\mbox{g} = \) 0.01}%
\end{pgfscope}%
\begin{pgfscope}%
\pgftext[x=5.400000in,y=0.157780in,,base]{\rmfamily\fontsize{20.000000}{24.000000}\selectfont \(\displaystyle t^*\)}%
\end{pgfscope}%
\begin{pgfscope}%
\pgftext[x=0.311046in,y=4.110377in,left,base,rotate=90.000000]{\rmfamily\fontsize{20.000000}{24.000000}\selectfont \(\displaystyle y^*\)}%
\end{pgfscope}%
\end{pgfpicture}%
\makeatother%
\endgroup%
}
    \caption{Text.}
    \label{fig:short_times}
\end{figure}

\subsubsection{Inertial Electro-Viscous Limit}
Asymptotic estimate of the trajectory. With $\epsilon = {\mathbb{E}\mbox{u}}_+$, where $\epsilon$ is a small parameter, and $\beta = \mathbb{D}\mbox{g}$,
\begin{eqnarray*}
&\bar{y}(\bar{t}) = \bar{t} - \frac{\bar{t}^{2}}{2} + \epsilon \left(\frac{\bar{t}^{3}}{3} \left(1 + \beta\right) + \frac{\bar{t}^{4}}{12} \left(-1 - \beta\right) - \frac{\beta \bar{t}^{2}}{2}\right)&  \\
&+ \epsilon^{2} \left(\frac{\bar{t}^{4}}{12} \left(-3 - 3 \beta - 4 \beta^{2}\right) + \frac{\bar{t}^{5}}{60} \left(11 + 10 \beta + 8 \beta^{2}\right)+ \frac{\bar{t}^{6}}{360} \left(-11 - 10 \beta - 8 \beta^{2}\right) + \frac{\beta^{2} \bar{t}^{3}}{3}\right)& \\
 &+ \mathcal{O}(\epsilon^3)&
\end{eqnarray*}

\newpage
\begin{figure}[htb]
    \centering
    %% Creator: Matplotlib, PGF backend
%%
%% To include the figure in your LaTeX document, write
%%   \input{<filename>.pgf}
%%
%% Make sure the required packages are loaded in your preamble
%%   \usepackage{pgf}
%%
%% Figures using additional raster images can only be included by \input if
%% they are in the same directory as the main LaTeX file. For loading figures
%% from other directories you can use the `import` package
%%   \usepackage{import}
%% and then include the figures with
%%   \import{<path to file>}{<filename>.pgf}
%%
%% Matplotlib used the following preamble
%%   \usepackage{fontspec}
%%   \setmainfont{DejaVu Serif}
%%   \setsansfont{DejaVu Sans}
%%   \setmonofont{DejaVu Sans Mono}
%%
\begingroup%
\makeatletter%
\begin{pgfpicture}%
\pgfpathrectangle{\pgfpointorigin}{\pgfqpoint{5.403395in}{3.694365in}}%
\pgfusepath{use as bounding box, clip}%
\begin{pgfscope}%
\pgfsetbuttcap%
\pgfsetmiterjoin%
\definecolor{currentfill}{rgb}{1.000000,1.000000,1.000000}%
\pgfsetfillcolor{currentfill}%
\pgfsetlinewidth{0.000000pt}%
\definecolor{currentstroke}{rgb}{1.000000,1.000000,1.000000}%
\pgfsetstrokecolor{currentstroke}%
\pgfsetdash{}{0pt}%
\pgfpathmoveto{\pgfqpoint{0.000000in}{0.000000in}}%
\pgfpathlineto{\pgfqpoint{5.403395in}{0.000000in}}%
\pgfpathlineto{\pgfqpoint{5.403395in}{3.694365in}}%
\pgfpathlineto{\pgfqpoint{0.000000in}{3.694365in}}%
\pgfpathclose%
\pgfusepath{fill}%
\end{pgfscope}%
\begin{pgfscope}%
\pgfsetbuttcap%
\pgfsetmiterjoin%
\definecolor{currentfill}{rgb}{1.000000,1.000000,1.000000}%
\pgfsetfillcolor{currentfill}%
\pgfsetlinewidth{0.000000pt}%
\definecolor{currentstroke}{rgb}{0.000000,0.000000,0.000000}%
\pgfsetstrokecolor{currentstroke}%
\pgfsetstrokeopacity{0.000000}%
\pgfsetdash{}{0pt}%
\pgfpathmoveto{\pgfqpoint{0.564660in}{0.521603in}}%
\pgfpathlineto{\pgfqpoint{5.214660in}{0.521603in}}%
\pgfpathlineto{\pgfqpoint{5.214660in}{3.541603in}}%
\pgfpathlineto{\pgfqpoint{0.564660in}{3.541603in}}%
\pgfpathclose%
\pgfusepath{fill}%
\end{pgfscope}%
\begin{pgfscope}%
\pgfsetbuttcap%
\pgfsetroundjoin%
\definecolor{currentfill}{rgb}{0.000000,0.000000,0.000000}%
\pgfsetfillcolor{currentfill}%
\pgfsetlinewidth{0.803000pt}%
\definecolor{currentstroke}{rgb}{0.000000,0.000000,0.000000}%
\pgfsetstrokecolor{currentstroke}%
\pgfsetdash{}{0pt}%
\pgfsys@defobject{currentmarker}{\pgfqpoint{0.000000in}{-0.048611in}}{\pgfqpoint{0.000000in}{0.000000in}}{%
\pgfpathmoveto{\pgfqpoint{0.000000in}{0.000000in}}%
\pgfpathlineto{\pgfqpoint{0.000000in}{-0.048611in}}%
\pgfusepath{stroke,fill}%
}%
\begin{pgfscope}%
\pgfsys@transformshift{0.785878in}{0.521603in}%
\pgfsys@useobject{currentmarker}{}%
\end{pgfscope}%
\end{pgfscope}%
\begin{pgfscope}%
\pgftext[x=0.785878in,y=0.424381in,,top]{\rmfamily\fontsize{10.000000}{12.000000}\selectfont \(\displaystyle 0.0\)}%
\end{pgfscope}%
\begin{pgfscope}%
\pgfsetbuttcap%
\pgfsetroundjoin%
\definecolor{currentfill}{rgb}{0.000000,0.000000,0.000000}%
\pgfsetfillcolor{currentfill}%
\pgfsetlinewidth{0.803000pt}%
\definecolor{currentstroke}{rgb}{0.000000,0.000000,0.000000}%
\pgfsetstrokecolor{currentstroke}%
\pgfsetdash{}{0pt}%
\pgfsys@defobject{currentmarker}{\pgfqpoint{0.000000in}{-0.048611in}}{\pgfqpoint{0.000000in}{0.000000in}}{%
\pgfpathmoveto{\pgfqpoint{0.000000in}{0.000000in}}%
\pgfpathlineto{\pgfqpoint{0.000000in}{-0.048611in}}%
\pgfusepath{stroke,fill}%
}%
\begin{pgfscope}%
\pgfsys@transformshift{1.671634in}{0.521603in}%
\pgfsys@useobject{currentmarker}{}%
\end{pgfscope}%
\end{pgfscope}%
\begin{pgfscope}%
\pgftext[x=1.671634in,y=0.424381in,,top]{\rmfamily\fontsize{10.000000}{12.000000}\selectfont \(\displaystyle 0.2\)}%
\end{pgfscope}%
\begin{pgfscope}%
\pgfsetbuttcap%
\pgfsetroundjoin%
\definecolor{currentfill}{rgb}{0.000000,0.000000,0.000000}%
\pgfsetfillcolor{currentfill}%
\pgfsetlinewidth{0.803000pt}%
\definecolor{currentstroke}{rgb}{0.000000,0.000000,0.000000}%
\pgfsetstrokecolor{currentstroke}%
\pgfsetdash{}{0pt}%
\pgfsys@defobject{currentmarker}{\pgfqpoint{0.000000in}{-0.048611in}}{\pgfqpoint{0.000000in}{0.000000in}}{%
\pgfpathmoveto{\pgfqpoint{0.000000in}{0.000000in}}%
\pgfpathlineto{\pgfqpoint{0.000000in}{-0.048611in}}%
\pgfusepath{stroke,fill}%
}%
\begin{pgfscope}%
\pgfsys@transformshift{2.557391in}{0.521603in}%
\pgfsys@useobject{currentmarker}{}%
\end{pgfscope}%
\end{pgfscope}%
\begin{pgfscope}%
\pgftext[x=2.557391in,y=0.424381in,,top]{\rmfamily\fontsize{10.000000}{12.000000}\selectfont \(\displaystyle 0.4\)}%
\end{pgfscope}%
\begin{pgfscope}%
\pgfsetbuttcap%
\pgfsetroundjoin%
\definecolor{currentfill}{rgb}{0.000000,0.000000,0.000000}%
\pgfsetfillcolor{currentfill}%
\pgfsetlinewidth{0.803000pt}%
\definecolor{currentstroke}{rgb}{0.000000,0.000000,0.000000}%
\pgfsetstrokecolor{currentstroke}%
\pgfsetdash{}{0pt}%
\pgfsys@defobject{currentmarker}{\pgfqpoint{0.000000in}{-0.048611in}}{\pgfqpoint{0.000000in}{0.000000in}}{%
\pgfpathmoveto{\pgfqpoint{0.000000in}{0.000000in}}%
\pgfpathlineto{\pgfqpoint{0.000000in}{-0.048611in}}%
\pgfusepath{stroke,fill}%
}%
\begin{pgfscope}%
\pgfsys@transformshift{3.443147in}{0.521603in}%
\pgfsys@useobject{currentmarker}{}%
\end{pgfscope}%
\end{pgfscope}%
\begin{pgfscope}%
\pgftext[x=3.443147in,y=0.424381in,,top]{\rmfamily\fontsize{10.000000}{12.000000}\selectfont \(\displaystyle 0.6\)}%
\end{pgfscope}%
\begin{pgfscope}%
\pgfsetbuttcap%
\pgfsetroundjoin%
\definecolor{currentfill}{rgb}{0.000000,0.000000,0.000000}%
\pgfsetfillcolor{currentfill}%
\pgfsetlinewidth{0.803000pt}%
\definecolor{currentstroke}{rgb}{0.000000,0.000000,0.000000}%
\pgfsetstrokecolor{currentstroke}%
\pgfsetdash{}{0pt}%
\pgfsys@defobject{currentmarker}{\pgfqpoint{0.000000in}{-0.048611in}}{\pgfqpoint{0.000000in}{0.000000in}}{%
\pgfpathmoveto{\pgfqpoint{0.000000in}{0.000000in}}%
\pgfpathlineto{\pgfqpoint{0.000000in}{-0.048611in}}%
\pgfusepath{stroke,fill}%
}%
\begin{pgfscope}%
\pgfsys@transformshift{4.328904in}{0.521603in}%
\pgfsys@useobject{currentmarker}{}%
\end{pgfscope}%
\end{pgfscope}%
\begin{pgfscope}%
\pgftext[x=4.328904in,y=0.424381in,,top]{\rmfamily\fontsize{10.000000}{12.000000}\selectfont \(\displaystyle 0.8\)}%
\end{pgfscope}%
\begin{pgfscope}%
\pgfsetbuttcap%
\pgfsetroundjoin%
\definecolor{currentfill}{rgb}{0.000000,0.000000,0.000000}%
\pgfsetfillcolor{currentfill}%
\pgfsetlinewidth{0.803000pt}%
\definecolor{currentstroke}{rgb}{0.000000,0.000000,0.000000}%
\pgfsetstrokecolor{currentstroke}%
\pgfsetdash{}{0pt}%
\pgfsys@defobject{currentmarker}{\pgfqpoint{0.000000in}{-0.048611in}}{\pgfqpoint{0.000000in}{0.000000in}}{%
\pgfpathmoveto{\pgfqpoint{0.000000in}{0.000000in}}%
\pgfpathlineto{\pgfqpoint{0.000000in}{-0.048611in}}%
\pgfusepath{stroke,fill}%
}%
\begin{pgfscope}%
\pgfsys@transformshift{5.214660in}{0.521603in}%
\pgfsys@useobject{currentmarker}{}%
\end{pgfscope}%
\end{pgfscope}%
\begin{pgfscope}%
\pgftext[x=5.214660in,y=0.424381in,,top]{\rmfamily\fontsize{10.000000}{12.000000}\selectfont \(\displaystyle 1.0\)}%
\end{pgfscope}%
\begin{pgfscope}%
\pgftext[x=2.889660in,y=0.234413in,,top]{\rmfamily\fontsize{10.000000}{12.000000}\selectfont \(\displaystyle \bar{t}\)}%
\end{pgfscope}%
\begin{pgfscope}%
\pgfsetbuttcap%
\pgfsetroundjoin%
\definecolor{currentfill}{rgb}{0.000000,0.000000,0.000000}%
\pgfsetfillcolor{currentfill}%
\pgfsetlinewidth{0.803000pt}%
\definecolor{currentstroke}{rgb}{0.000000,0.000000,0.000000}%
\pgfsetstrokecolor{currentstroke}%
\pgfsetdash{}{0pt}%
\pgfsys@defobject{currentmarker}{\pgfqpoint{-0.048611in}{0.000000in}}{\pgfqpoint{0.000000in}{0.000000in}}{%
\pgfpathmoveto{\pgfqpoint{0.000000in}{0.000000in}}%
\pgfpathlineto{\pgfqpoint{-0.048611in}{0.000000in}}%
\pgfusepath{stroke,fill}%
}%
\begin{pgfscope}%
\pgfsys@transformshift{0.564660in}{0.521603in}%
\pgfsys@useobject{currentmarker}{}%
\end{pgfscope}%
\end{pgfscope}%
\begin{pgfscope}%
\pgftext[x=0.289968in,y=0.468842in,left,base]{\rmfamily\fontsize{10.000000}{12.000000}\selectfont \(\displaystyle 0.0\)}%
\end{pgfscope}%
\begin{pgfscope}%
\pgfsetbuttcap%
\pgfsetroundjoin%
\definecolor{currentfill}{rgb}{0.000000,0.000000,0.000000}%
\pgfsetfillcolor{currentfill}%
\pgfsetlinewidth{0.803000pt}%
\definecolor{currentstroke}{rgb}{0.000000,0.000000,0.000000}%
\pgfsetstrokecolor{currentstroke}%
\pgfsetdash{}{0pt}%
\pgfsys@defobject{currentmarker}{\pgfqpoint{-0.048611in}{0.000000in}}{\pgfqpoint{0.000000in}{0.000000in}}{%
\pgfpathmoveto{\pgfqpoint{0.000000in}{0.000000in}}%
\pgfpathlineto{\pgfqpoint{-0.048611in}{0.000000in}}%
\pgfusepath{stroke,fill}%
}%
\begin{pgfscope}%
\pgfsys@transformshift{0.564660in}{0.953032in}%
\pgfsys@useobject{currentmarker}{}%
\end{pgfscope}%
\end{pgfscope}%
\begin{pgfscope}%
\pgftext[x=0.289968in,y=0.900270in,left,base]{\rmfamily\fontsize{10.000000}{12.000000}\selectfont \(\displaystyle 0.1\)}%
\end{pgfscope}%
\begin{pgfscope}%
\pgfsetbuttcap%
\pgfsetroundjoin%
\definecolor{currentfill}{rgb}{0.000000,0.000000,0.000000}%
\pgfsetfillcolor{currentfill}%
\pgfsetlinewidth{0.803000pt}%
\definecolor{currentstroke}{rgb}{0.000000,0.000000,0.000000}%
\pgfsetstrokecolor{currentstroke}%
\pgfsetdash{}{0pt}%
\pgfsys@defobject{currentmarker}{\pgfqpoint{-0.048611in}{0.000000in}}{\pgfqpoint{0.000000in}{0.000000in}}{%
\pgfpathmoveto{\pgfqpoint{0.000000in}{0.000000in}}%
\pgfpathlineto{\pgfqpoint{-0.048611in}{0.000000in}}%
\pgfusepath{stroke,fill}%
}%
\begin{pgfscope}%
\pgfsys@transformshift{0.564660in}{1.384460in}%
\pgfsys@useobject{currentmarker}{}%
\end{pgfscope}%
\end{pgfscope}%
\begin{pgfscope}%
\pgftext[x=0.289968in,y=1.331699in,left,base]{\rmfamily\fontsize{10.000000}{12.000000}\selectfont \(\displaystyle 0.2\)}%
\end{pgfscope}%
\begin{pgfscope}%
\pgfsetbuttcap%
\pgfsetroundjoin%
\definecolor{currentfill}{rgb}{0.000000,0.000000,0.000000}%
\pgfsetfillcolor{currentfill}%
\pgfsetlinewidth{0.803000pt}%
\definecolor{currentstroke}{rgb}{0.000000,0.000000,0.000000}%
\pgfsetstrokecolor{currentstroke}%
\pgfsetdash{}{0pt}%
\pgfsys@defobject{currentmarker}{\pgfqpoint{-0.048611in}{0.000000in}}{\pgfqpoint{0.000000in}{0.000000in}}{%
\pgfpathmoveto{\pgfqpoint{0.000000in}{0.000000in}}%
\pgfpathlineto{\pgfqpoint{-0.048611in}{0.000000in}}%
\pgfusepath{stroke,fill}%
}%
\begin{pgfscope}%
\pgfsys@transformshift{0.564660in}{1.815889in}%
\pgfsys@useobject{currentmarker}{}%
\end{pgfscope}%
\end{pgfscope}%
\begin{pgfscope}%
\pgftext[x=0.289968in,y=1.763128in,left,base]{\rmfamily\fontsize{10.000000}{12.000000}\selectfont \(\displaystyle 0.3\)}%
\end{pgfscope}%
\begin{pgfscope}%
\pgfsetbuttcap%
\pgfsetroundjoin%
\definecolor{currentfill}{rgb}{0.000000,0.000000,0.000000}%
\pgfsetfillcolor{currentfill}%
\pgfsetlinewidth{0.803000pt}%
\definecolor{currentstroke}{rgb}{0.000000,0.000000,0.000000}%
\pgfsetstrokecolor{currentstroke}%
\pgfsetdash{}{0pt}%
\pgfsys@defobject{currentmarker}{\pgfqpoint{-0.048611in}{0.000000in}}{\pgfqpoint{0.000000in}{0.000000in}}{%
\pgfpathmoveto{\pgfqpoint{0.000000in}{0.000000in}}%
\pgfpathlineto{\pgfqpoint{-0.048611in}{0.000000in}}%
\pgfusepath{stroke,fill}%
}%
\begin{pgfscope}%
\pgfsys@transformshift{0.564660in}{2.247318in}%
\pgfsys@useobject{currentmarker}{}%
\end{pgfscope}%
\end{pgfscope}%
\begin{pgfscope}%
\pgftext[x=0.289968in,y=2.194556in,left,base]{\rmfamily\fontsize{10.000000}{12.000000}\selectfont \(\displaystyle 0.4\)}%
\end{pgfscope}%
\begin{pgfscope}%
\pgfsetbuttcap%
\pgfsetroundjoin%
\definecolor{currentfill}{rgb}{0.000000,0.000000,0.000000}%
\pgfsetfillcolor{currentfill}%
\pgfsetlinewidth{0.803000pt}%
\definecolor{currentstroke}{rgb}{0.000000,0.000000,0.000000}%
\pgfsetstrokecolor{currentstroke}%
\pgfsetdash{}{0pt}%
\pgfsys@defobject{currentmarker}{\pgfqpoint{-0.048611in}{0.000000in}}{\pgfqpoint{0.000000in}{0.000000in}}{%
\pgfpathmoveto{\pgfqpoint{0.000000in}{0.000000in}}%
\pgfpathlineto{\pgfqpoint{-0.048611in}{0.000000in}}%
\pgfusepath{stroke,fill}%
}%
\begin{pgfscope}%
\pgfsys@transformshift{0.564660in}{2.678746in}%
\pgfsys@useobject{currentmarker}{}%
\end{pgfscope}%
\end{pgfscope}%
\begin{pgfscope}%
\pgftext[x=0.289968in,y=2.625985in,left,base]{\rmfamily\fontsize{10.000000}{12.000000}\selectfont \(\displaystyle 0.5\)}%
\end{pgfscope}%
\begin{pgfscope}%
\pgfsetbuttcap%
\pgfsetroundjoin%
\definecolor{currentfill}{rgb}{0.000000,0.000000,0.000000}%
\pgfsetfillcolor{currentfill}%
\pgfsetlinewidth{0.803000pt}%
\definecolor{currentstroke}{rgb}{0.000000,0.000000,0.000000}%
\pgfsetstrokecolor{currentstroke}%
\pgfsetdash{}{0pt}%
\pgfsys@defobject{currentmarker}{\pgfqpoint{-0.048611in}{0.000000in}}{\pgfqpoint{0.000000in}{0.000000in}}{%
\pgfpathmoveto{\pgfqpoint{0.000000in}{0.000000in}}%
\pgfpathlineto{\pgfqpoint{-0.048611in}{0.000000in}}%
\pgfusepath{stroke,fill}%
}%
\begin{pgfscope}%
\pgfsys@transformshift{0.564660in}{3.110175in}%
\pgfsys@useobject{currentmarker}{}%
\end{pgfscope}%
\end{pgfscope}%
\begin{pgfscope}%
\pgftext[x=0.289968in,y=3.057413in,left,base]{\rmfamily\fontsize{10.000000}{12.000000}\selectfont \(\displaystyle 0.6\)}%
\end{pgfscope}%
\begin{pgfscope}%
\pgfsetbuttcap%
\pgfsetroundjoin%
\definecolor{currentfill}{rgb}{0.000000,0.000000,0.000000}%
\pgfsetfillcolor{currentfill}%
\pgfsetlinewidth{0.803000pt}%
\definecolor{currentstroke}{rgb}{0.000000,0.000000,0.000000}%
\pgfsetstrokecolor{currentstroke}%
\pgfsetdash{}{0pt}%
\pgfsys@defobject{currentmarker}{\pgfqpoint{-0.048611in}{0.000000in}}{\pgfqpoint{0.000000in}{0.000000in}}{%
\pgfpathmoveto{\pgfqpoint{0.000000in}{0.000000in}}%
\pgfpathlineto{\pgfqpoint{-0.048611in}{0.000000in}}%
\pgfusepath{stroke,fill}%
}%
\begin{pgfscope}%
\pgfsys@transformshift{0.564660in}{3.541603in}%
\pgfsys@useobject{currentmarker}{}%
\end{pgfscope}%
\end{pgfscope}%
\begin{pgfscope}%
\pgftext[x=0.289968in,y=3.488842in,left,base]{\rmfamily\fontsize{10.000000}{12.000000}\selectfont \(\displaystyle 0.7\)}%
\end{pgfscope}%
\begin{pgfscope}%
\pgftext[x=0.234413in,y=2.031603in,,bottom,rotate=90.000000]{\rmfamily\fontsize{10.000000}{12.000000}\selectfont \(\displaystyle \bar{y}\)}%
\end{pgfscope}%
\begin{pgfscope}%
\pgfpathrectangle{\pgfqpoint{0.564660in}{0.521603in}}{\pgfqpoint{4.650000in}{3.020000in}} %
\pgfusepath{clip}%
\pgfsetbuttcap%
\pgfsetroundjoin%
\pgfsetlinewidth{1.505625pt}%
\definecolor{currentstroke}{rgb}{1.000000,0.000000,0.000000}%
\pgfsetstrokecolor{currentstroke}%
\pgfsetdash{{5.550000pt}{2.400000pt}}{0.000000pt}%
\pgfpathmoveto{\pgfqpoint{0.785878in}{0.521603in}}%
\pgfpathlineto{\pgfqpoint{0.905455in}{0.636543in}}%
\pgfpathlineto{\pgfqpoint{1.029461in}{0.752586in}}%
\pgfpathlineto{\pgfqpoint{1.153467in}{0.865579in}}%
\pgfpathlineto{\pgfqpoint{1.281902in}{0.979547in}}%
\pgfpathlineto{\pgfqpoint{1.410336in}{1.090547in}}%
\pgfpathlineto{\pgfqpoint{1.543200in}{1.202392in}}%
\pgfpathlineto{\pgfqpoint{1.676063in}{1.311336in}}%
\pgfpathlineto{\pgfqpoint{1.813355in}{1.420993in}}%
\pgfpathlineto{\pgfqpoint{1.950648in}{1.527805in}}%
\pgfpathlineto{\pgfqpoint{2.092369in}{1.635195in}}%
\pgfpathlineto{\pgfqpoint{2.234090in}{1.739783in}}%
\pgfpathlineto{\pgfqpoint{2.380240in}{1.844815in}}%
\pgfpathlineto{\pgfqpoint{2.526389in}{1.947082in}}%
\pgfpathlineto{\pgfqpoint{2.672539in}{2.046682in}}%
\pgfpathlineto{\pgfqpoint{2.823118in}{2.146611in}}%
\pgfpathlineto{\pgfqpoint{2.973696in}{2.243910in}}%
\pgfpathlineto{\pgfqpoint{3.128704in}{2.341429in}}%
\pgfpathlineto{\pgfqpoint{3.288140in}{2.439054in}}%
\pgfpathlineto{\pgfqpoint{3.447576in}{2.534084in}}%
\pgfpathlineto{\pgfqpoint{3.611441in}{2.629188in}}%
\pgfpathlineto{\pgfqpoint{3.779735in}{2.724312in}}%
\pgfpathlineto{\pgfqpoint{3.956886in}{2.821844in}}%
\pgfpathlineto{\pgfqpoint{4.138466in}{2.919273in}}%
\pgfpathlineto{\pgfqpoint{4.333333in}{3.021252in}}%
\pgfpathlineto{\pgfqpoint{4.541485in}{3.127596in}}%
\pgfpathlineto{\pgfqpoint{4.767353in}{3.240428in}}%
\pgfpathlineto{\pgfqpoint{5.019794in}{3.363988in}}%
\pgfpathlineto{\pgfqpoint{5.210231in}{3.455803in}}%
\pgfpathlineto{\pgfqpoint{5.210231in}{3.455803in}}%
\pgfusepath{stroke}%
\end{pgfscope}%
\begin{pgfscope}%
\pgfpathrectangle{\pgfqpoint{0.564660in}{0.521603in}}{\pgfqpoint{4.650000in}{3.020000in}} %
\pgfusepath{clip}%
\pgfsetbuttcap%
\pgfsetmiterjoin%
\definecolor{currentfill}{rgb}{1.000000,0.000000,0.000000}%
\pgfsetfillcolor{currentfill}%
\pgfsetlinewidth{1.003750pt}%
\definecolor{currentstroke}{rgb}{1.000000,0.000000,0.000000}%
\pgfsetstrokecolor{currentstroke}%
\pgfsetdash{}{0pt}%
\pgfsys@defobject{currentmarker}{\pgfqpoint{-0.041667in}{-0.041667in}}{\pgfqpoint{0.041667in}{0.041667in}}{%
\pgfpathmoveto{\pgfqpoint{-0.041667in}{-0.041667in}}%
\pgfpathlineto{\pgfqpoint{0.041667in}{-0.041667in}}%
\pgfpathlineto{\pgfqpoint{0.041667in}{0.041667in}}%
\pgfpathlineto{\pgfqpoint{-0.041667in}{0.041667in}}%
\pgfpathclose%
\pgfusepath{stroke,fill}%
}%
\begin{pgfscope}%
\pgfsys@transformshift{0.785878in}{0.521603in}%
\pgfsys@useobject{currentmarker}{}%
\end{pgfscope}%
\begin{pgfscope}%
\pgfsys@transformshift{1.228756in}{0.932756in}%
\pgfsys@useobject{currentmarker}{}%
\end{pgfscope}%
\begin{pgfscope}%
\pgfsys@transformshift{1.671634in}{1.307750in}%
\pgfsys@useobject{currentmarker}{}%
\end{pgfscope}%
\begin{pgfscope}%
\pgfsys@transformshift{2.114513in}{1.651719in}%
\pgfsys@useobject{currentmarker}{}%
\end{pgfscope}%
\begin{pgfscope}%
\pgfsys@transformshift{2.557391in}{1.968429in}%
\pgfsys@useobject{currentmarker}{}%
\end{pgfscope}%
\begin{pgfscope}%
\pgfsys@transformshift{3.000269in}{2.260815in}%
\pgfsys@useobject{currentmarker}{}%
\end{pgfscope}%
\begin{pgfscope}%
\pgfsys@transformshift{3.443147in}{2.531478in}%
\pgfsys@useobject{currentmarker}{}%
\end{pgfscope}%
\begin{pgfscope}%
\pgfsys@transformshift{3.886026in}{2.783138in}%
\pgfsys@useobject{currentmarker}{}%
\end{pgfscope}%
\begin{pgfscope}%
\pgfsys@transformshift{4.328904in}{3.018962in}%
\pgfsys@useobject{currentmarker}{}%
\end{pgfscope}%
\begin{pgfscope}%
\pgfsys@transformshift{4.771782in}{3.242617in}%
\pgfsys@useobject{currentmarker}{}%
\end{pgfscope}%
\end{pgfscope}%
\begin{pgfscope}%
\pgfpathrectangle{\pgfqpoint{0.564660in}{0.521603in}}{\pgfqpoint{4.650000in}{3.020000in}} %
\pgfusepath{clip}%
\pgfsetrectcap%
\pgfsetroundjoin%
\pgfsetlinewidth{1.505625pt}%
\definecolor{currentstroke}{rgb}{0.000000,0.000000,1.000000}%
\pgfsetstrokecolor{currentstroke}%
\pgfsetdash{}{0pt}%
\pgfpathmoveto{\pgfqpoint{0.785878in}{0.521603in}}%
\pgfpathlineto{\pgfqpoint{0.905455in}{0.636519in}}%
\pgfpathlineto{\pgfqpoint{1.025032in}{0.748306in}}%
\pgfpathlineto{\pgfqpoint{1.144609in}{0.856981in}}%
\pgfpathlineto{\pgfqpoint{1.264186in}{0.962559in}}%
\pgfpathlineto{\pgfqpoint{1.379335in}{1.061313in}}%
\pgfpathlineto{\pgfqpoint{1.494483in}{1.157222in}}%
\pgfpathlineto{\pgfqpoint{1.609631in}{1.250299in}}%
\pgfpathlineto{\pgfqpoint{1.724780in}{1.340555in}}%
\pgfpathlineto{\pgfqpoint{1.839928in}{1.428002in}}%
\pgfpathlineto{\pgfqpoint{1.955076in}{1.512652in}}%
\pgfpathlineto{\pgfqpoint{2.070225in}{1.594514in}}%
\pgfpathlineto{\pgfqpoint{2.185373in}{1.673600in}}%
\pgfpathlineto{\pgfqpoint{2.300522in}{1.749918in}}%
\pgfpathlineto{\pgfqpoint{2.415670in}{1.823480in}}%
\pgfpathlineto{\pgfqpoint{2.526389in}{1.891620in}}%
\pgfpathlineto{\pgfqpoint{2.637109in}{1.957227in}}%
\pgfpathlineto{\pgfqpoint{2.747829in}{2.020309in}}%
\pgfpathlineto{\pgfqpoint{2.858548in}{2.080872in}}%
\pgfpathlineto{\pgfqpoint{2.969268in}{2.138923in}}%
\pgfpathlineto{\pgfqpoint{3.079987in}{2.194469in}}%
\pgfpathlineto{\pgfqpoint{3.190707in}{2.247517in}}%
\pgfpathlineto{\pgfqpoint{3.301426in}{2.298071in}}%
\pgfpathlineto{\pgfqpoint{3.412146in}{2.346137in}}%
\pgfpathlineto{\pgfqpoint{3.522865in}{2.391722in}}%
\pgfpathlineto{\pgfqpoint{3.633585in}{2.434830in}}%
\pgfpathlineto{\pgfqpoint{3.744305in}{2.475465in}}%
\pgfpathlineto{\pgfqpoint{3.850595in}{2.512153in}}%
\pgfpathlineto{\pgfqpoint{3.956886in}{2.546570in}}%
\pgfpathlineto{\pgfqpoint{4.063177in}{2.578720in}}%
\pgfpathlineto{\pgfqpoint{4.169468in}{2.608606in}}%
\pgfpathlineto{\pgfqpoint{4.275758in}{2.636231in}}%
\pgfpathlineto{\pgfqpoint{4.382049in}{2.661598in}}%
\pgfpathlineto{\pgfqpoint{4.488340in}{2.684709in}}%
\pgfpathlineto{\pgfqpoint{4.594631in}{2.705567in}}%
\pgfpathlineto{\pgfqpoint{4.700922in}{2.724173in}}%
\pgfpathlineto{\pgfqpoint{4.807212in}{2.740530in}}%
\pgfpathlineto{\pgfqpoint{4.913503in}{2.754639in}}%
\pgfpathlineto{\pgfqpoint{5.019794in}{2.766501in}}%
\pgfpathlineto{\pgfqpoint{5.126085in}{2.776118in}}%
\pgfpathlineto{\pgfqpoint{5.210231in}{2.782140in}}%
\pgfpathlineto{\pgfqpoint{5.210231in}{2.782140in}}%
\pgfusepath{stroke}%
\end{pgfscope}%
\begin{pgfscope}%
\pgfpathrectangle{\pgfqpoint{0.564660in}{0.521603in}}{\pgfqpoint{4.650000in}{3.020000in}} %
\pgfusepath{clip}%
\pgfsetbuttcap%
\pgfsetroundjoin%
\definecolor{currentfill}{rgb}{0.000000,0.000000,1.000000}%
\pgfsetfillcolor{currentfill}%
\pgfsetlinewidth{1.003750pt}%
\definecolor{currentstroke}{rgb}{0.000000,0.000000,1.000000}%
\pgfsetstrokecolor{currentstroke}%
\pgfsetdash{}{0pt}%
\pgfsys@defobject{currentmarker}{\pgfqpoint{-0.041667in}{-0.041667in}}{\pgfqpoint{0.041667in}{0.041667in}}{%
\pgfpathmoveto{\pgfqpoint{0.000000in}{-0.041667in}}%
\pgfpathcurveto{\pgfqpoint{0.011050in}{-0.041667in}}{\pgfqpoint{0.021649in}{-0.037276in}}{\pgfqpoint{0.029463in}{-0.029463in}}%
\pgfpathcurveto{\pgfqpoint{0.037276in}{-0.021649in}}{\pgfqpoint{0.041667in}{-0.011050in}}{\pgfqpoint{0.041667in}{0.000000in}}%
\pgfpathcurveto{\pgfqpoint{0.041667in}{0.011050in}}{\pgfqpoint{0.037276in}{0.021649in}}{\pgfqpoint{0.029463in}{0.029463in}}%
\pgfpathcurveto{\pgfqpoint{0.021649in}{0.037276in}}{\pgfqpoint{0.011050in}{0.041667in}}{\pgfqpoint{0.000000in}{0.041667in}}%
\pgfpathcurveto{\pgfqpoint{-0.011050in}{0.041667in}}{\pgfqpoint{-0.021649in}{0.037276in}}{\pgfqpoint{-0.029463in}{0.029463in}}%
\pgfpathcurveto{\pgfqpoint{-0.037276in}{0.021649in}}{\pgfqpoint{-0.041667in}{0.011050in}}{\pgfqpoint{-0.041667in}{0.000000in}}%
\pgfpathcurveto{\pgfqpoint{-0.041667in}{-0.011050in}}{\pgfqpoint{-0.037276in}{-0.021649in}}{\pgfqpoint{-0.029463in}{-0.029463in}}%
\pgfpathcurveto{\pgfqpoint{-0.021649in}{-0.037276in}}{\pgfqpoint{-0.011050in}{-0.041667in}}{\pgfqpoint{0.000000in}{-0.041667in}}%
\pgfpathclose%
\pgfusepath{stroke,fill}%
}%
\begin{pgfscope}%
\pgfsys@transformshift{0.785878in}{0.521603in}%
\pgfsys@useobject{currentmarker}{}%
\end{pgfscope}%
\begin{pgfscope}%
\pgfsys@transformshift{1.228756in}{0.931598in}%
\pgfsys@useobject{currentmarker}{}%
\end{pgfscope}%
\begin{pgfscope}%
\pgfsys@transformshift{1.671634in}{1.299248in}%
\pgfsys@useobject{currentmarker}{}%
\end{pgfscope}%
\begin{pgfscope}%
\pgfsys@transformshift{2.114513in}{1.625260in}%
\pgfsys@useobject{currentmarker}{}%
\end{pgfscope}%
\begin{pgfscope}%
\pgfsys@transformshift{2.557391in}{1.910245in}%
\pgfsys@useobject{currentmarker}{}%
\end{pgfscope}%
\begin{pgfscope}%
\pgfsys@transformshift{3.000269in}{2.154728in}%
\pgfsys@useobject{currentmarker}{}%
\end{pgfscope}%
\begin{pgfscope}%
\pgfsys@transformshift{3.443147in}{2.359151in}%
\pgfsys@useobject{currentmarker}{}%
\end{pgfscope}%
\begin{pgfscope}%
\pgfsys@transformshift{3.886026in}{2.523878in}%
\pgfsys@useobject{currentmarker}{}%
\end{pgfscope}%
\begin{pgfscope}%
\pgfsys@transformshift{4.328904in}{2.649197in}%
\pgfsys@useobject{currentmarker}{}%
\end{pgfscope}%
\begin{pgfscope}%
\pgfsys@transformshift{4.771782in}{2.735327in}%
\pgfsys@useobject{currentmarker}{}%
\end{pgfscope}%
\end{pgfscope}%
\begin{pgfscope}%
\pgfpathrectangle{\pgfqpoint{0.564660in}{0.521603in}}{\pgfqpoint{4.650000in}{3.020000in}} %
\pgfusepath{clip}%
\pgfsetbuttcap%
\pgfsetroundjoin%
\pgfsetlinewidth{1.505625pt}%
\definecolor{currentstroke}{rgb}{0.000000,0.750000,0.750000}%
\pgfsetstrokecolor{currentstroke}%
\pgfsetdash{{9.600000pt}{2.400000pt}{1.500000pt}{2.400000pt}}{0.000000pt}%
\pgfpathmoveto{\pgfqpoint{0.785878in}{0.521603in}}%
\pgfpathlineto{\pgfqpoint{0.905455in}{0.636517in}}%
\pgfpathlineto{\pgfqpoint{1.025032in}{0.748287in}}%
\pgfpathlineto{\pgfqpoint{1.140181in}{0.852948in}}%
\pgfpathlineto{\pgfqpoint{1.255329in}{0.954696in}}%
\pgfpathlineto{\pgfqpoint{1.370477in}{1.053535in}}%
\pgfpathlineto{\pgfqpoint{1.485626in}{1.149464in}}%
\pgfpathlineto{\pgfqpoint{1.600774in}{1.242485in}}%
\pgfpathlineto{\pgfqpoint{1.715922in}{1.332599in}}%
\pgfpathlineto{\pgfqpoint{1.831071in}{1.419807in}}%
\pgfpathlineto{\pgfqpoint{1.941790in}{1.500923in}}%
\pgfpathlineto{\pgfqpoint{2.052510in}{1.579354in}}%
\pgfpathlineto{\pgfqpoint{2.163229in}{1.655102in}}%
\pgfpathlineto{\pgfqpoint{2.273949in}{1.728168in}}%
\pgfpathlineto{\pgfqpoint{2.384668in}{1.798552in}}%
\pgfpathlineto{\pgfqpoint{2.495388in}{1.866256in}}%
\pgfpathlineto{\pgfqpoint{2.606107in}{1.931280in}}%
\pgfpathlineto{\pgfqpoint{2.712398in}{1.991183in}}%
\pgfpathlineto{\pgfqpoint{2.818689in}{2.048617in}}%
\pgfpathlineto{\pgfqpoint{2.924980in}{2.103584in}}%
\pgfpathlineto{\pgfqpoint{3.031271in}{2.156084in}}%
\pgfpathlineto{\pgfqpoint{3.137561in}{2.206118in}}%
\pgfpathlineto{\pgfqpoint{3.243852in}{2.253686in}}%
\pgfpathlineto{\pgfqpoint{3.350143in}{2.298789in}}%
\pgfpathlineto{\pgfqpoint{3.456434in}{2.341427in}}%
\pgfpathlineto{\pgfqpoint{3.562724in}{2.381601in}}%
\pgfpathlineto{\pgfqpoint{3.669015in}{2.419311in}}%
\pgfpathlineto{\pgfqpoint{3.770877in}{2.453138in}}%
\pgfpathlineto{\pgfqpoint{3.872739in}{2.484704in}}%
\pgfpathlineto{\pgfqpoint{3.974601in}{2.514008in}}%
\pgfpathlineto{\pgfqpoint{4.076463in}{2.541050in}}%
\pgfpathlineto{\pgfqpoint{4.178325in}{2.565832in}}%
\pgfpathlineto{\pgfqpoint{4.280187in}{2.588352in}}%
\pgfpathlineto{\pgfqpoint{4.382049in}{2.608613in}}%
\pgfpathlineto{\pgfqpoint{4.483911in}{2.626612in}}%
\pgfpathlineto{\pgfqpoint{4.585773in}{2.642352in}}%
\pgfpathlineto{\pgfqpoint{4.687635in}{2.655832in}}%
\pgfpathlineto{\pgfqpoint{4.789497in}{2.667051in}}%
\pgfpathlineto{\pgfqpoint{4.891359in}{2.676012in}}%
\pgfpathlineto{\pgfqpoint{4.993221in}{2.682712in}}%
\pgfpathlineto{\pgfqpoint{5.095083in}{2.687153in}}%
\pgfpathlineto{\pgfqpoint{5.196945in}{2.689334in}}%
\pgfpathlineto{\pgfqpoint{5.210231in}{2.689452in}}%
\pgfpathlineto{\pgfqpoint{5.210231in}{2.689452in}}%
\pgfusepath{stroke}%
\end{pgfscope}%
\begin{pgfscope}%
\pgfpathrectangle{\pgfqpoint{0.564660in}{0.521603in}}{\pgfqpoint{4.650000in}{3.020000in}} %
\pgfusepath{clip}%
\pgfsetbuttcap%
\pgfsetmiterjoin%
\definecolor{currentfill}{rgb}{0.000000,0.750000,0.750000}%
\pgfsetfillcolor{currentfill}%
\pgfsetlinewidth{1.003750pt}%
\definecolor{currentstroke}{rgb}{0.000000,0.750000,0.750000}%
\pgfsetstrokecolor{currentstroke}%
\pgfsetdash{}{0pt}%
\pgfsys@defobject{currentmarker}{\pgfqpoint{-0.041667in}{-0.041667in}}{\pgfqpoint{0.041667in}{0.041667in}}{%
\pgfpathmoveto{\pgfqpoint{-0.000000in}{-0.041667in}}%
\pgfpathlineto{\pgfqpoint{0.041667in}{0.041667in}}%
\pgfpathlineto{\pgfqpoint{-0.041667in}{0.041667in}}%
\pgfpathclose%
\pgfusepath{stroke,fill}%
}%
\begin{pgfscope}%
\pgfsys@transformshift{0.785878in}{0.521603in}%
\pgfsys@useobject{currentmarker}{}%
\end{pgfscope}%
\begin{pgfscope}%
\pgfsys@transformshift{1.228756in}{0.931474in}%
\pgfsys@useobject{currentmarker}{}%
\end{pgfscope}%
\begin{pgfscope}%
\pgfsys@transformshift{1.671634in}{1.298283in}%
\pgfsys@useobject{currentmarker}{}%
\end{pgfscope}%
\begin{pgfscope}%
\pgfsys@transformshift{2.114513in}{1.622104in}%
\pgfsys@useobject{currentmarker}{}%
\end{pgfscope}%
\begin{pgfscope}%
\pgfsys@transformshift{2.557391in}{1.902999in}%
\pgfsys@useobject{currentmarker}{}%
\end{pgfscope}%
\begin{pgfscope}%
\pgfsys@transformshift{3.000269in}{2.141026in}%
\pgfsys@useobject{currentmarker}{}%
\end{pgfscope}%
\begin{pgfscope}%
\pgfsys@transformshift{3.443147in}{2.336232in}%
\pgfsys@useobject{currentmarker}{}%
\end{pgfscope}%
\begin{pgfscope}%
\pgfsys@transformshift{3.886026in}{2.488654in}%
\pgfsys@useobject{currentmarker}{}%
\end{pgfscope}%
\begin{pgfscope}%
\pgfsys@transformshift{4.328904in}{2.598324in}%
\pgfsys@useobject{currentmarker}{}%
\end{pgfscope}%
\begin{pgfscope}%
\pgfsys@transformshift{4.771782in}{2.665263in}%
\pgfsys@useobject{currentmarker}{}%
\end{pgfscope}%
\end{pgfscope}%
\begin{pgfscope}%
\pgfsetrectcap%
\pgfsetmiterjoin%
\pgfsetlinewidth{0.803000pt}%
\definecolor{currentstroke}{rgb}{0.000000,0.000000,0.000000}%
\pgfsetstrokecolor{currentstroke}%
\pgfsetdash{}{0pt}%
\pgfpathmoveto{\pgfqpoint{0.564660in}{0.521603in}}%
\pgfpathlineto{\pgfqpoint{0.564660in}{3.541603in}}%
\pgfusepath{stroke}%
\end{pgfscope}%
\begin{pgfscope}%
\pgfsetrectcap%
\pgfsetmiterjoin%
\pgfsetlinewidth{0.803000pt}%
\definecolor{currentstroke}{rgb}{0.000000,0.000000,0.000000}%
\pgfsetstrokecolor{currentstroke}%
\pgfsetdash{}{0pt}%
\pgfpathmoveto{\pgfqpoint{5.214660in}{0.521603in}}%
\pgfpathlineto{\pgfqpoint{5.214660in}{3.541603in}}%
\pgfusepath{stroke}%
\end{pgfscope}%
\begin{pgfscope}%
\pgfsetrectcap%
\pgfsetmiterjoin%
\pgfsetlinewidth{0.803000pt}%
\definecolor{currentstroke}{rgb}{0.000000,0.000000,0.000000}%
\pgfsetstrokecolor{currentstroke}%
\pgfsetdash{}{0pt}%
\pgfpathmoveto{\pgfqpoint{0.564660in}{0.521603in}}%
\pgfpathlineto{\pgfqpoint{5.214660in}{0.521603in}}%
\pgfusepath{stroke}%
\end{pgfscope}%
\begin{pgfscope}%
\pgfsetrectcap%
\pgfsetmiterjoin%
\pgfsetlinewidth{0.803000pt}%
\definecolor{currentstroke}{rgb}{0.000000,0.000000,0.000000}%
\pgfsetstrokecolor{currentstroke}%
\pgfsetdash{}{0pt}%
\pgfpathmoveto{\pgfqpoint{0.564660in}{3.541603in}}%
\pgfpathlineto{\pgfqpoint{5.214660in}{3.541603in}}%
\pgfusepath{stroke}%
\end{pgfscope}%
\begin{pgfscope}%
\pgfsetbuttcap%
\pgfsetmiterjoin%
\definecolor{currentfill}{rgb}{1.000000,1.000000,1.000000}%
\pgfsetfillcolor{currentfill}%
\pgfsetfillopacity{0.800000}%
\pgfsetlinewidth{1.003750pt}%
\definecolor{currentstroke}{rgb}{0.800000,0.800000,0.800000}%
\pgfsetstrokecolor{currentstroke}%
\pgfsetstrokeopacity{0.800000}%
\pgfsetdash{}{0pt}%
\pgfpathmoveto{\pgfqpoint{0.661883in}{2.602470in}}%
\pgfpathlineto{\pgfqpoint{1.676618in}{2.602470in}}%
\pgfpathquadraticcurveto{\pgfqpoint{1.704396in}{2.602470in}}{\pgfqpoint{1.704396in}{2.630247in}}%
\pgfpathlineto{\pgfqpoint{1.704396in}{3.444381in}}%
\pgfpathquadraticcurveto{\pgfqpoint{1.704396in}{3.472159in}}{\pgfqpoint{1.676618in}{3.472159in}}%
\pgfpathlineto{\pgfqpoint{0.661883in}{3.472159in}}%
\pgfpathquadraticcurveto{\pgfqpoint{0.634105in}{3.472159in}}{\pgfqpoint{0.634105in}{3.444381in}}%
\pgfpathlineto{\pgfqpoint{0.634105in}{2.630247in}}%
\pgfpathquadraticcurveto{\pgfqpoint{0.634105in}{2.602470in}}{\pgfqpoint{0.661883in}{2.602470in}}%
\pgfpathclose%
\pgfusepath{stroke,fill}%
\end{pgfscope}%
\begin{pgfscope}%
\pgftext[x=0.781774in,y=3.298487in,left,base]{\rmfamily\fontsize{10.000000}{12.000000}\selectfont \(\displaystyle \beta\) = 6\(\displaystyle \times 10^{-4}\)}%
\end{pgfscope}%
\begin{pgfscope}%
\pgfsetbuttcap%
\pgfsetroundjoin%
\pgfsetlinewidth{1.505625pt}%
\definecolor{currentstroke}{rgb}{1.000000,0.000000,0.000000}%
\pgfsetstrokecolor{currentstroke}%
\pgfsetdash{{5.550000pt}{2.400000pt}}{0.000000pt}%
\pgfpathmoveto{\pgfqpoint{0.689660in}{3.143240in}}%
\pgfpathlineto{\pgfqpoint{0.967438in}{3.143240in}}%
\pgfusepath{stroke}%
\end{pgfscope}%
\begin{pgfscope}%
\pgfsetbuttcap%
\pgfsetmiterjoin%
\definecolor{currentfill}{rgb}{1.000000,0.000000,0.000000}%
\pgfsetfillcolor{currentfill}%
\pgfsetlinewidth{1.003750pt}%
\definecolor{currentstroke}{rgb}{1.000000,0.000000,0.000000}%
\pgfsetstrokecolor{currentstroke}%
\pgfsetdash{}{0pt}%
\pgfsys@defobject{currentmarker}{\pgfqpoint{-0.041667in}{-0.041667in}}{\pgfqpoint{0.041667in}{0.041667in}}{%
\pgfpathmoveto{\pgfqpoint{-0.041667in}{-0.041667in}}%
\pgfpathlineto{\pgfqpoint{0.041667in}{-0.041667in}}%
\pgfpathlineto{\pgfqpoint{0.041667in}{0.041667in}}%
\pgfpathlineto{\pgfqpoint{-0.041667in}{0.041667in}}%
\pgfpathclose%
\pgfusepath{stroke,fill}%
}%
\begin{pgfscope}%
\pgfsys@transformshift{0.828549in}{3.143240in}%
\pgfsys@useobject{currentmarker}{}%
\end{pgfscope}%
\end{pgfscope}%
\begin{pgfscope}%
\pgftext[x=1.078549in,y=3.094629in,left,base]{\rmfamily\fontsize{10.000000}{12.000000}\selectfont \(\displaystyle \epsilon\) = 1}%
\end{pgfscope}%
\begin{pgfscope}%
\pgfsetrectcap%
\pgfsetroundjoin%
\pgfsetlinewidth{1.505625pt}%
\definecolor{currentstroke}{rgb}{0.000000,0.000000,1.000000}%
\pgfsetstrokecolor{currentstroke}%
\pgfsetdash{}{0pt}%
\pgfpathmoveto{\pgfqpoint{0.689660in}{2.939383in}}%
\pgfpathlineto{\pgfqpoint{0.967438in}{2.939383in}}%
\pgfusepath{stroke}%
\end{pgfscope}%
\begin{pgfscope}%
\pgfsetbuttcap%
\pgfsetroundjoin%
\definecolor{currentfill}{rgb}{0.000000,0.000000,1.000000}%
\pgfsetfillcolor{currentfill}%
\pgfsetlinewidth{1.003750pt}%
\definecolor{currentstroke}{rgb}{0.000000,0.000000,1.000000}%
\pgfsetstrokecolor{currentstroke}%
\pgfsetdash{}{0pt}%
\pgfsys@defobject{currentmarker}{\pgfqpoint{-0.041667in}{-0.041667in}}{\pgfqpoint{0.041667in}{0.041667in}}{%
\pgfpathmoveto{\pgfqpoint{0.000000in}{-0.041667in}}%
\pgfpathcurveto{\pgfqpoint{0.011050in}{-0.041667in}}{\pgfqpoint{0.021649in}{-0.037276in}}{\pgfqpoint{0.029463in}{-0.029463in}}%
\pgfpathcurveto{\pgfqpoint{0.037276in}{-0.021649in}}{\pgfqpoint{0.041667in}{-0.011050in}}{\pgfqpoint{0.041667in}{0.000000in}}%
\pgfpathcurveto{\pgfqpoint{0.041667in}{0.011050in}}{\pgfqpoint{0.037276in}{0.021649in}}{\pgfqpoint{0.029463in}{0.029463in}}%
\pgfpathcurveto{\pgfqpoint{0.021649in}{0.037276in}}{\pgfqpoint{0.011050in}{0.041667in}}{\pgfqpoint{0.000000in}{0.041667in}}%
\pgfpathcurveto{\pgfqpoint{-0.011050in}{0.041667in}}{\pgfqpoint{-0.021649in}{0.037276in}}{\pgfqpoint{-0.029463in}{0.029463in}}%
\pgfpathcurveto{\pgfqpoint{-0.037276in}{0.021649in}}{\pgfqpoint{-0.041667in}{0.011050in}}{\pgfqpoint{-0.041667in}{0.000000in}}%
\pgfpathcurveto{\pgfqpoint{-0.041667in}{-0.011050in}}{\pgfqpoint{-0.037276in}{-0.021649in}}{\pgfqpoint{-0.029463in}{-0.029463in}}%
\pgfpathcurveto{\pgfqpoint{-0.021649in}{-0.037276in}}{\pgfqpoint{-0.011050in}{-0.041667in}}{\pgfqpoint{0.000000in}{-0.041667in}}%
\pgfpathclose%
\pgfusepath{stroke,fill}%
}%
\begin{pgfscope}%
\pgfsys@transformshift{0.828549in}{2.939383in}%
\pgfsys@useobject{currentmarker}{}%
\end{pgfscope}%
\end{pgfscope}%
\begin{pgfscope}%
\pgftext[x=1.078549in,y=2.890772in,left,base]{\rmfamily\fontsize{10.000000}{12.000000}\selectfont \(\displaystyle \epsilon\) = 0.1}%
\end{pgfscope}%
\begin{pgfscope}%
\pgfsetbuttcap%
\pgfsetroundjoin%
\pgfsetlinewidth{1.505625pt}%
\definecolor{currentstroke}{rgb}{0.000000,0.750000,0.750000}%
\pgfsetstrokecolor{currentstroke}%
\pgfsetdash{{9.600000pt}{2.400000pt}{1.500000pt}{2.400000pt}}{0.000000pt}%
\pgfpathmoveto{\pgfqpoint{0.689660in}{2.735526in}}%
\pgfpathlineto{\pgfqpoint{0.967438in}{2.735526in}}%
\pgfusepath{stroke}%
\end{pgfscope}%
\begin{pgfscope}%
\pgfsetbuttcap%
\pgfsetmiterjoin%
\definecolor{currentfill}{rgb}{0.000000,0.750000,0.750000}%
\pgfsetfillcolor{currentfill}%
\pgfsetlinewidth{1.003750pt}%
\definecolor{currentstroke}{rgb}{0.000000,0.750000,0.750000}%
\pgfsetstrokecolor{currentstroke}%
\pgfsetdash{}{0pt}%
\pgfsys@defobject{currentmarker}{\pgfqpoint{-0.041667in}{-0.041667in}}{\pgfqpoint{0.041667in}{0.041667in}}{%
\pgfpathmoveto{\pgfqpoint{-0.000000in}{-0.041667in}}%
\pgfpathlineto{\pgfqpoint{0.041667in}{0.041667in}}%
\pgfpathlineto{\pgfqpoint{-0.041667in}{0.041667in}}%
\pgfpathclose%
\pgfusepath{stroke,fill}%
}%
\begin{pgfscope}%
\pgfsys@transformshift{0.828549in}{2.735526in}%
\pgfsys@useobject{currentmarker}{}%
\end{pgfscope}%
\end{pgfscope}%
\begin{pgfscope}%
\pgftext[x=1.078549in,y=2.686915in,left,base]{\rmfamily\fontsize{10.000000}{12.000000}\selectfont \(\displaystyle \epsilon\) = 0.01}%
\end{pgfscope}%
\end{pgfpicture}%
\makeatother%
\endgroup%

    \caption{Text.}
     \label{fig:long_times}
\end{figure}

Times for returns
\[t_c = 2 + \epsilon \left(\frac{4}{3} - \frac{2 \beta}{3}\right) + \epsilon^{2} \left(\frac{4}{5} - \frac{4 \beta}{3} + \frac{2 \beta^{2}}{5}\right)
\]
\begin{figure}[htb]
    \centering
    %% Creator: Matplotlib, PGF backend
%%
%% To include the figure in your LaTeX document, write
%%   \input{<filename>.pgf}
%%
%% Make sure the required packages are loaded in your preamble
%%   \usepackage{pgf}
%%
%% Figures using additional raster images can only be included by \input if
%% they are in the same directory as the main LaTeX file. For loading figures
%% from other directories you can use the `import` package
%%   \usepackage{import}
%% and then include the figures with
%%   \import{<path to file>}{<filename>.pgf}
%%
%% Matplotlib used the following preamble
%%   \usepackage{fontspec}
%%   \setmainfont{DejaVu Serif}
%%   \setsansfont{DejaVu Sans}
%%   \setmonofont{DejaVu Sans Mono}
%%
\begingroup%
\makeatletter%
\begin{pgfpicture}%
\pgfpathrectangle{\pgfpointorigin}{\pgfqpoint{5.349660in}{3.676603in}}%
\pgfusepath{use as bounding box, clip}%
\begin{pgfscope}%
\pgfsetbuttcap%
\pgfsetmiterjoin%
\definecolor{currentfill}{rgb}{1.000000,1.000000,1.000000}%
\pgfsetfillcolor{currentfill}%
\pgfsetlinewidth{0.000000pt}%
\definecolor{currentstroke}{rgb}{1.000000,1.000000,1.000000}%
\pgfsetstrokecolor{currentstroke}%
\pgfsetdash{}{0pt}%
\pgfpathmoveto{\pgfqpoint{0.000000in}{0.000000in}}%
\pgfpathlineto{\pgfqpoint{5.349660in}{0.000000in}}%
\pgfpathlineto{\pgfqpoint{5.349660in}{3.676603in}}%
\pgfpathlineto{\pgfqpoint{0.000000in}{3.676603in}}%
\pgfpathclose%
\pgfusepath{fill}%
\end{pgfscope}%
\begin{pgfscope}%
\pgfsetbuttcap%
\pgfsetmiterjoin%
\definecolor{currentfill}{rgb}{1.000000,1.000000,1.000000}%
\pgfsetfillcolor{currentfill}%
\pgfsetlinewidth{0.000000pt}%
\definecolor{currentstroke}{rgb}{0.000000,0.000000,0.000000}%
\pgfsetstrokecolor{currentstroke}%
\pgfsetstrokeopacity{0.000000}%
\pgfsetdash{}{0pt}%
\pgfpathmoveto{\pgfqpoint{0.564660in}{0.521603in}}%
\pgfpathlineto{\pgfqpoint{5.214660in}{0.521603in}}%
\pgfpathlineto{\pgfqpoint{5.214660in}{3.541603in}}%
\pgfpathlineto{\pgfqpoint{0.564660in}{3.541603in}}%
\pgfpathclose%
\pgfusepath{fill}%
\end{pgfscope}%
\begin{pgfscope}%
\pgfsetbuttcap%
\pgfsetroundjoin%
\definecolor{currentfill}{rgb}{0.000000,0.000000,0.000000}%
\pgfsetfillcolor{currentfill}%
\pgfsetlinewidth{0.803000pt}%
\definecolor{currentstroke}{rgb}{0.000000,0.000000,0.000000}%
\pgfsetstrokecolor{currentstroke}%
\pgfsetdash{}{0pt}%
\pgfsys@defobject{currentmarker}{\pgfqpoint{0.000000in}{-0.048611in}}{\pgfqpoint{0.000000in}{0.000000in}}{%
\pgfpathmoveto{\pgfqpoint{0.000000in}{0.000000in}}%
\pgfpathlineto{\pgfqpoint{0.000000in}{-0.048611in}}%
\pgfusepath{stroke,fill}%
}%
\begin{pgfscope}%
\pgfsys@transformshift{0.776024in}{0.521603in}%
\pgfsys@useobject{currentmarker}{}%
\end{pgfscope}%
\end{pgfscope}%
\begin{pgfscope}%
\pgftext[x=0.776024in,y=0.424381in,,top]{\rmfamily\fontsize{10.000000}{12.000000}\selectfont \(\displaystyle 10^{-2}\)}%
\end{pgfscope}%
\begin{pgfscope}%
\pgfsetbuttcap%
\pgfsetroundjoin%
\definecolor{currentfill}{rgb}{0.000000,0.000000,0.000000}%
\pgfsetfillcolor{currentfill}%
\pgfsetlinewidth{0.803000pt}%
\definecolor{currentstroke}{rgb}{0.000000,0.000000,0.000000}%
\pgfsetstrokecolor{currentstroke}%
\pgfsetdash{}{0pt}%
\pgfsys@defobject{currentmarker}{\pgfqpoint{0.000000in}{-0.048611in}}{\pgfqpoint{0.000000in}{0.000000in}}{%
\pgfpathmoveto{\pgfqpoint{0.000000in}{0.000000in}}%
\pgfpathlineto{\pgfqpoint{0.000000in}{-0.048611in}}%
\pgfusepath{stroke,fill}%
}%
\begin{pgfscope}%
\pgfsys@transformshift{2.890120in}{0.521603in}%
\pgfsys@useobject{currentmarker}{}%
\end{pgfscope}%
\end{pgfscope}%
\begin{pgfscope}%
\pgftext[x=2.890120in,y=0.424381in,,top]{\rmfamily\fontsize{10.000000}{12.000000}\selectfont \(\displaystyle 10^{-1}\)}%
\end{pgfscope}%
\begin{pgfscope}%
\pgfsetbuttcap%
\pgfsetroundjoin%
\definecolor{currentfill}{rgb}{0.000000,0.000000,0.000000}%
\pgfsetfillcolor{currentfill}%
\pgfsetlinewidth{0.803000pt}%
\definecolor{currentstroke}{rgb}{0.000000,0.000000,0.000000}%
\pgfsetstrokecolor{currentstroke}%
\pgfsetdash{}{0pt}%
\pgfsys@defobject{currentmarker}{\pgfqpoint{0.000000in}{-0.048611in}}{\pgfqpoint{0.000000in}{0.000000in}}{%
\pgfpathmoveto{\pgfqpoint{0.000000in}{0.000000in}}%
\pgfpathlineto{\pgfqpoint{0.000000in}{-0.048611in}}%
\pgfusepath{stroke,fill}%
}%
\begin{pgfscope}%
\pgfsys@transformshift{5.004215in}{0.521603in}%
\pgfsys@useobject{currentmarker}{}%
\end{pgfscope}%
\end{pgfscope}%
\begin{pgfscope}%
\pgftext[x=5.004215in,y=0.424381in,,top]{\rmfamily\fontsize{10.000000}{12.000000}\selectfont \(\displaystyle 10^{0}\)}%
\end{pgfscope}%
\begin{pgfscope}%
\pgfsetbuttcap%
\pgfsetroundjoin%
\definecolor{currentfill}{rgb}{0.000000,0.000000,0.000000}%
\pgfsetfillcolor{currentfill}%
\pgfsetlinewidth{0.602250pt}%
\definecolor{currentstroke}{rgb}{0.000000,0.000000,0.000000}%
\pgfsetstrokecolor{currentstroke}%
\pgfsetdash{}{0pt}%
\pgfsys@defobject{currentmarker}{\pgfqpoint{0.000000in}{-0.027778in}}{\pgfqpoint{0.000000in}{0.000000in}}{%
\pgfpathmoveto{\pgfqpoint{0.000000in}{0.000000in}}%
\pgfpathlineto{\pgfqpoint{0.000000in}{-0.027778in}}%
\pgfusepath{stroke,fill}%
}%
\begin{pgfscope}%
\pgfsys@transformshift{0.571147in}{0.521603in}%
\pgfsys@useobject{currentmarker}{}%
\end{pgfscope}%
\end{pgfscope}%
\begin{pgfscope}%
\pgfsetbuttcap%
\pgfsetroundjoin%
\definecolor{currentfill}{rgb}{0.000000,0.000000,0.000000}%
\pgfsetfillcolor{currentfill}%
\pgfsetlinewidth{0.602250pt}%
\definecolor{currentstroke}{rgb}{0.000000,0.000000,0.000000}%
\pgfsetstrokecolor{currentstroke}%
\pgfsetdash{}{0pt}%
\pgfsys@defobject{currentmarker}{\pgfqpoint{0.000000in}{-0.027778in}}{\pgfqpoint{0.000000in}{0.000000in}}{%
\pgfpathmoveto{\pgfqpoint{0.000000in}{0.000000in}}%
\pgfpathlineto{\pgfqpoint{0.000000in}{-0.027778in}}%
\pgfusepath{stroke,fill}%
}%
\begin{pgfscope}%
\pgfsys@transformshift{0.679288in}{0.521603in}%
\pgfsys@useobject{currentmarker}{}%
\end{pgfscope}%
\end{pgfscope}%
\begin{pgfscope}%
\pgfsetbuttcap%
\pgfsetroundjoin%
\definecolor{currentfill}{rgb}{0.000000,0.000000,0.000000}%
\pgfsetfillcolor{currentfill}%
\pgfsetlinewidth{0.602250pt}%
\definecolor{currentstroke}{rgb}{0.000000,0.000000,0.000000}%
\pgfsetstrokecolor{currentstroke}%
\pgfsetdash{}{0pt}%
\pgfsys@defobject{currentmarker}{\pgfqpoint{0.000000in}{-0.027778in}}{\pgfqpoint{0.000000in}{0.000000in}}{%
\pgfpathmoveto{\pgfqpoint{0.000000in}{0.000000in}}%
\pgfpathlineto{\pgfqpoint{0.000000in}{-0.027778in}}%
\pgfusepath{stroke,fill}%
}%
\begin{pgfscope}%
\pgfsys@transformshift{1.412430in}{0.521603in}%
\pgfsys@useobject{currentmarker}{}%
\end{pgfscope}%
\end{pgfscope}%
\begin{pgfscope}%
\pgfsetbuttcap%
\pgfsetroundjoin%
\definecolor{currentfill}{rgb}{0.000000,0.000000,0.000000}%
\pgfsetfillcolor{currentfill}%
\pgfsetlinewidth{0.602250pt}%
\definecolor{currentstroke}{rgb}{0.000000,0.000000,0.000000}%
\pgfsetstrokecolor{currentstroke}%
\pgfsetdash{}{0pt}%
\pgfsys@defobject{currentmarker}{\pgfqpoint{0.000000in}{-0.027778in}}{\pgfqpoint{0.000000in}{0.000000in}}{%
\pgfpathmoveto{\pgfqpoint{0.000000in}{0.000000in}}%
\pgfpathlineto{\pgfqpoint{0.000000in}{-0.027778in}}%
\pgfusepath{stroke,fill}%
}%
\begin{pgfscope}%
\pgfsys@transformshift{1.784704in}{0.521603in}%
\pgfsys@useobject{currentmarker}{}%
\end{pgfscope}%
\end{pgfscope}%
\begin{pgfscope}%
\pgfsetbuttcap%
\pgfsetroundjoin%
\definecolor{currentfill}{rgb}{0.000000,0.000000,0.000000}%
\pgfsetfillcolor{currentfill}%
\pgfsetlinewidth{0.602250pt}%
\definecolor{currentstroke}{rgb}{0.000000,0.000000,0.000000}%
\pgfsetstrokecolor{currentstroke}%
\pgfsetdash{}{0pt}%
\pgfsys@defobject{currentmarker}{\pgfqpoint{0.000000in}{-0.027778in}}{\pgfqpoint{0.000000in}{0.000000in}}{%
\pgfpathmoveto{\pgfqpoint{0.000000in}{0.000000in}}%
\pgfpathlineto{\pgfqpoint{0.000000in}{-0.027778in}}%
\pgfusepath{stroke,fill}%
}%
\begin{pgfscope}%
\pgfsys@transformshift{2.048836in}{0.521603in}%
\pgfsys@useobject{currentmarker}{}%
\end{pgfscope}%
\end{pgfscope}%
\begin{pgfscope}%
\pgfsetbuttcap%
\pgfsetroundjoin%
\definecolor{currentfill}{rgb}{0.000000,0.000000,0.000000}%
\pgfsetfillcolor{currentfill}%
\pgfsetlinewidth{0.602250pt}%
\definecolor{currentstroke}{rgb}{0.000000,0.000000,0.000000}%
\pgfsetstrokecolor{currentstroke}%
\pgfsetdash{}{0pt}%
\pgfsys@defobject{currentmarker}{\pgfqpoint{0.000000in}{-0.027778in}}{\pgfqpoint{0.000000in}{0.000000in}}{%
\pgfpathmoveto{\pgfqpoint{0.000000in}{0.000000in}}%
\pgfpathlineto{\pgfqpoint{0.000000in}{-0.027778in}}%
\pgfusepath{stroke,fill}%
}%
\begin{pgfscope}%
\pgfsys@transformshift{2.253713in}{0.521603in}%
\pgfsys@useobject{currentmarker}{}%
\end{pgfscope}%
\end{pgfscope}%
\begin{pgfscope}%
\pgfsetbuttcap%
\pgfsetroundjoin%
\definecolor{currentfill}{rgb}{0.000000,0.000000,0.000000}%
\pgfsetfillcolor{currentfill}%
\pgfsetlinewidth{0.602250pt}%
\definecolor{currentstroke}{rgb}{0.000000,0.000000,0.000000}%
\pgfsetstrokecolor{currentstroke}%
\pgfsetdash{}{0pt}%
\pgfsys@defobject{currentmarker}{\pgfqpoint{0.000000in}{-0.027778in}}{\pgfqpoint{0.000000in}{0.000000in}}{%
\pgfpathmoveto{\pgfqpoint{0.000000in}{0.000000in}}%
\pgfpathlineto{\pgfqpoint{0.000000in}{-0.027778in}}%
\pgfusepath{stroke,fill}%
}%
\begin{pgfscope}%
\pgfsys@transformshift{2.421110in}{0.521603in}%
\pgfsys@useobject{currentmarker}{}%
\end{pgfscope}%
\end{pgfscope}%
\begin{pgfscope}%
\pgfsetbuttcap%
\pgfsetroundjoin%
\definecolor{currentfill}{rgb}{0.000000,0.000000,0.000000}%
\pgfsetfillcolor{currentfill}%
\pgfsetlinewidth{0.602250pt}%
\definecolor{currentstroke}{rgb}{0.000000,0.000000,0.000000}%
\pgfsetstrokecolor{currentstroke}%
\pgfsetdash{}{0pt}%
\pgfsys@defobject{currentmarker}{\pgfqpoint{0.000000in}{-0.027778in}}{\pgfqpoint{0.000000in}{0.000000in}}{%
\pgfpathmoveto{\pgfqpoint{0.000000in}{0.000000in}}%
\pgfpathlineto{\pgfqpoint{0.000000in}{-0.027778in}}%
\pgfusepath{stroke,fill}%
}%
\begin{pgfscope}%
\pgfsys@transformshift{2.562642in}{0.521603in}%
\pgfsys@useobject{currentmarker}{}%
\end{pgfscope}%
\end{pgfscope}%
\begin{pgfscope}%
\pgfsetbuttcap%
\pgfsetroundjoin%
\definecolor{currentfill}{rgb}{0.000000,0.000000,0.000000}%
\pgfsetfillcolor{currentfill}%
\pgfsetlinewidth{0.602250pt}%
\definecolor{currentstroke}{rgb}{0.000000,0.000000,0.000000}%
\pgfsetstrokecolor{currentstroke}%
\pgfsetdash{}{0pt}%
\pgfsys@defobject{currentmarker}{\pgfqpoint{0.000000in}{-0.027778in}}{\pgfqpoint{0.000000in}{0.000000in}}{%
\pgfpathmoveto{\pgfqpoint{0.000000in}{0.000000in}}%
\pgfpathlineto{\pgfqpoint{0.000000in}{-0.027778in}}%
\pgfusepath{stroke,fill}%
}%
\begin{pgfscope}%
\pgfsys@transformshift{2.685243in}{0.521603in}%
\pgfsys@useobject{currentmarker}{}%
\end{pgfscope}%
\end{pgfscope}%
\begin{pgfscope}%
\pgfsetbuttcap%
\pgfsetroundjoin%
\definecolor{currentfill}{rgb}{0.000000,0.000000,0.000000}%
\pgfsetfillcolor{currentfill}%
\pgfsetlinewidth{0.602250pt}%
\definecolor{currentstroke}{rgb}{0.000000,0.000000,0.000000}%
\pgfsetstrokecolor{currentstroke}%
\pgfsetdash{}{0pt}%
\pgfsys@defobject{currentmarker}{\pgfqpoint{0.000000in}{-0.027778in}}{\pgfqpoint{0.000000in}{0.000000in}}{%
\pgfpathmoveto{\pgfqpoint{0.000000in}{0.000000in}}%
\pgfpathlineto{\pgfqpoint{0.000000in}{-0.027778in}}%
\pgfusepath{stroke,fill}%
}%
\begin{pgfscope}%
\pgfsys@transformshift{2.793384in}{0.521603in}%
\pgfsys@useobject{currentmarker}{}%
\end{pgfscope}%
\end{pgfscope}%
\begin{pgfscope}%
\pgfsetbuttcap%
\pgfsetroundjoin%
\definecolor{currentfill}{rgb}{0.000000,0.000000,0.000000}%
\pgfsetfillcolor{currentfill}%
\pgfsetlinewidth{0.602250pt}%
\definecolor{currentstroke}{rgb}{0.000000,0.000000,0.000000}%
\pgfsetstrokecolor{currentstroke}%
\pgfsetdash{}{0pt}%
\pgfsys@defobject{currentmarker}{\pgfqpoint{0.000000in}{-0.027778in}}{\pgfqpoint{0.000000in}{0.000000in}}{%
\pgfpathmoveto{\pgfqpoint{0.000000in}{0.000000in}}%
\pgfpathlineto{\pgfqpoint{0.000000in}{-0.027778in}}%
\pgfusepath{stroke,fill}%
}%
\begin{pgfscope}%
\pgfsys@transformshift{3.526526in}{0.521603in}%
\pgfsys@useobject{currentmarker}{}%
\end{pgfscope}%
\end{pgfscope}%
\begin{pgfscope}%
\pgfsetbuttcap%
\pgfsetroundjoin%
\definecolor{currentfill}{rgb}{0.000000,0.000000,0.000000}%
\pgfsetfillcolor{currentfill}%
\pgfsetlinewidth{0.602250pt}%
\definecolor{currentstroke}{rgb}{0.000000,0.000000,0.000000}%
\pgfsetstrokecolor{currentstroke}%
\pgfsetdash{}{0pt}%
\pgfsys@defobject{currentmarker}{\pgfqpoint{0.000000in}{-0.027778in}}{\pgfqpoint{0.000000in}{0.000000in}}{%
\pgfpathmoveto{\pgfqpoint{0.000000in}{0.000000in}}%
\pgfpathlineto{\pgfqpoint{0.000000in}{-0.027778in}}%
\pgfusepath{stroke,fill}%
}%
\begin{pgfscope}%
\pgfsys@transformshift{3.898800in}{0.521603in}%
\pgfsys@useobject{currentmarker}{}%
\end{pgfscope}%
\end{pgfscope}%
\begin{pgfscope}%
\pgfsetbuttcap%
\pgfsetroundjoin%
\definecolor{currentfill}{rgb}{0.000000,0.000000,0.000000}%
\pgfsetfillcolor{currentfill}%
\pgfsetlinewidth{0.602250pt}%
\definecolor{currentstroke}{rgb}{0.000000,0.000000,0.000000}%
\pgfsetstrokecolor{currentstroke}%
\pgfsetdash{}{0pt}%
\pgfsys@defobject{currentmarker}{\pgfqpoint{0.000000in}{-0.027778in}}{\pgfqpoint{0.000000in}{0.000000in}}{%
\pgfpathmoveto{\pgfqpoint{0.000000in}{0.000000in}}%
\pgfpathlineto{\pgfqpoint{0.000000in}{-0.027778in}}%
\pgfusepath{stroke,fill}%
}%
\begin{pgfscope}%
\pgfsys@transformshift{4.162932in}{0.521603in}%
\pgfsys@useobject{currentmarker}{}%
\end{pgfscope}%
\end{pgfscope}%
\begin{pgfscope}%
\pgfsetbuttcap%
\pgfsetroundjoin%
\definecolor{currentfill}{rgb}{0.000000,0.000000,0.000000}%
\pgfsetfillcolor{currentfill}%
\pgfsetlinewidth{0.602250pt}%
\definecolor{currentstroke}{rgb}{0.000000,0.000000,0.000000}%
\pgfsetstrokecolor{currentstroke}%
\pgfsetdash{}{0pt}%
\pgfsys@defobject{currentmarker}{\pgfqpoint{0.000000in}{-0.027778in}}{\pgfqpoint{0.000000in}{0.000000in}}{%
\pgfpathmoveto{\pgfqpoint{0.000000in}{0.000000in}}%
\pgfpathlineto{\pgfqpoint{0.000000in}{-0.027778in}}%
\pgfusepath{stroke,fill}%
}%
\begin{pgfscope}%
\pgfsys@transformshift{4.367809in}{0.521603in}%
\pgfsys@useobject{currentmarker}{}%
\end{pgfscope}%
\end{pgfscope}%
\begin{pgfscope}%
\pgfsetbuttcap%
\pgfsetroundjoin%
\definecolor{currentfill}{rgb}{0.000000,0.000000,0.000000}%
\pgfsetfillcolor{currentfill}%
\pgfsetlinewidth{0.602250pt}%
\definecolor{currentstroke}{rgb}{0.000000,0.000000,0.000000}%
\pgfsetstrokecolor{currentstroke}%
\pgfsetdash{}{0pt}%
\pgfsys@defobject{currentmarker}{\pgfqpoint{0.000000in}{-0.027778in}}{\pgfqpoint{0.000000in}{0.000000in}}{%
\pgfpathmoveto{\pgfqpoint{0.000000in}{0.000000in}}%
\pgfpathlineto{\pgfqpoint{0.000000in}{-0.027778in}}%
\pgfusepath{stroke,fill}%
}%
\begin{pgfscope}%
\pgfsys@transformshift{4.535206in}{0.521603in}%
\pgfsys@useobject{currentmarker}{}%
\end{pgfscope}%
\end{pgfscope}%
\begin{pgfscope}%
\pgfsetbuttcap%
\pgfsetroundjoin%
\definecolor{currentfill}{rgb}{0.000000,0.000000,0.000000}%
\pgfsetfillcolor{currentfill}%
\pgfsetlinewidth{0.602250pt}%
\definecolor{currentstroke}{rgb}{0.000000,0.000000,0.000000}%
\pgfsetstrokecolor{currentstroke}%
\pgfsetdash{}{0pt}%
\pgfsys@defobject{currentmarker}{\pgfqpoint{0.000000in}{-0.027778in}}{\pgfqpoint{0.000000in}{0.000000in}}{%
\pgfpathmoveto{\pgfqpoint{0.000000in}{0.000000in}}%
\pgfpathlineto{\pgfqpoint{0.000000in}{-0.027778in}}%
\pgfusepath{stroke,fill}%
}%
\begin{pgfscope}%
\pgfsys@transformshift{4.676738in}{0.521603in}%
\pgfsys@useobject{currentmarker}{}%
\end{pgfscope}%
\end{pgfscope}%
\begin{pgfscope}%
\pgfsetbuttcap%
\pgfsetroundjoin%
\definecolor{currentfill}{rgb}{0.000000,0.000000,0.000000}%
\pgfsetfillcolor{currentfill}%
\pgfsetlinewidth{0.602250pt}%
\definecolor{currentstroke}{rgb}{0.000000,0.000000,0.000000}%
\pgfsetstrokecolor{currentstroke}%
\pgfsetdash{}{0pt}%
\pgfsys@defobject{currentmarker}{\pgfqpoint{0.000000in}{-0.027778in}}{\pgfqpoint{0.000000in}{0.000000in}}{%
\pgfpathmoveto{\pgfqpoint{0.000000in}{0.000000in}}%
\pgfpathlineto{\pgfqpoint{0.000000in}{-0.027778in}}%
\pgfusepath{stroke,fill}%
}%
\begin{pgfscope}%
\pgfsys@transformshift{4.799338in}{0.521603in}%
\pgfsys@useobject{currentmarker}{}%
\end{pgfscope}%
\end{pgfscope}%
\begin{pgfscope}%
\pgfsetbuttcap%
\pgfsetroundjoin%
\definecolor{currentfill}{rgb}{0.000000,0.000000,0.000000}%
\pgfsetfillcolor{currentfill}%
\pgfsetlinewidth{0.602250pt}%
\definecolor{currentstroke}{rgb}{0.000000,0.000000,0.000000}%
\pgfsetstrokecolor{currentstroke}%
\pgfsetdash{}{0pt}%
\pgfsys@defobject{currentmarker}{\pgfqpoint{0.000000in}{-0.027778in}}{\pgfqpoint{0.000000in}{0.000000in}}{%
\pgfpathmoveto{\pgfqpoint{0.000000in}{0.000000in}}%
\pgfpathlineto{\pgfqpoint{0.000000in}{-0.027778in}}%
\pgfusepath{stroke,fill}%
}%
\begin{pgfscope}%
\pgfsys@transformshift{4.907480in}{0.521603in}%
\pgfsys@useobject{currentmarker}{}%
\end{pgfscope}%
\end{pgfscope}%
\begin{pgfscope}%
\pgftext[x=2.889660in,y=0.234413in,,top]{\rmfamily\fontsize{10.000000}{12.000000}\selectfont \(\displaystyle \epsilon\)}%
\end{pgfscope}%
\begin{pgfscope}%
\pgfsetbuttcap%
\pgfsetroundjoin%
\definecolor{currentfill}{rgb}{0.000000,0.000000,0.000000}%
\pgfsetfillcolor{currentfill}%
\pgfsetlinewidth{0.803000pt}%
\definecolor{currentstroke}{rgb}{0.000000,0.000000,0.000000}%
\pgfsetstrokecolor{currentstroke}%
\pgfsetdash{}{0pt}%
\pgfsys@defobject{currentmarker}{\pgfqpoint{-0.048611in}{0.000000in}}{\pgfqpoint{0.000000in}{0.000000in}}{%
\pgfpathmoveto{\pgfqpoint{0.000000in}{0.000000in}}%
\pgfpathlineto{\pgfqpoint{-0.048611in}{0.000000in}}%
\pgfusepath{stroke,fill}%
}%
\begin{pgfscope}%
\pgfsys@transformshift{0.564660in}{0.641477in}%
\pgfsys@useobject{currentmarker}{}%
\end{pgfscope}%
\end{pgfscope}%
\begin{pgfscope}%
\pgftext[x=0.289968in,y=0.588715in,left,base]{\rmfamily\fontsize{10.000000}{12.000000}\selectfont \(\displaystyle 2.0\)}%
\end{pgfscope}%
\begin{pgfscope}%
\pgfsetbuttcap%
\pgfsetroundjoin%
\definecolor{currentfill}{rgb}{0.000000,0.000000,0.000000}%
\pgfsetfillcolor{currentfill}%
\pgfsetlinewidth{0.803000pt}%
\definecolor{currentstroke}{rgb}{0.000000,0.000000,0.000000}%
\pgfsetstrokecolor{currentstroke}%
\pgfsetdash{}{0pt}%
\pgfsys@defobject{currentmarker}{\pgfqpoint{-0.048611in}{0.000000in}}{\pgfqpoint{0.000000in}{0.000000in}}{%
\pgfpathmoveto{\pgfqpoint{0.000000in}{0.000000in}}%
\pgfpathlineto{\pgfqpoint{-0.048611in}{0.000000in}}%
\pgfusepath{stroke,fill}%
}%
\begin{pgfscope}%
\pgfsys@transformshift{0.564660in}{1.290216in}%
\pgfsys@useobject{currentmarker}{}%
\end{pgfscope}%
\end{pgfscope}%
\begin{pgfscope}%
\pgftext[x=0.289968in,y=1.237455in,left,base]{\rmfamily\fontsize{10.000000}{12.000000}\selectfont \(\displaystyle 2.5\)}%
\end{pgfscope}%
\begin{pgfscope}%
\pgfsetbuttcap%
\pgfsetroundjoin%
\definecolor{currentfill}{rgb}{0.000000,0.000000,0.000000}%
\pgfsetfillcolor{currentfill}%
\pgfsetlinewidth{0.803000pt}%
\definecolor{currentstroke}{rgb}{0.000000,0.000000,0.000000}%
\pgfsetstrokecolor{currentstroke}%
\pgfsetdash{}{0pt}%
\pgfsys@defobject{currentmarker}{\pgfqpoint{-0.048611in}{0.000000in}}{\pgfqpoint{0.000000in}{0.000000in}}{%
\pgfpathmoveto{\pgfqpoint{0.000000in}{0.000000in}}%
\pgfpathlineto{\pgfqpoint{-0.048611in}{0.000000in}}%
\pgfusepath{stroke,fill}%
}%
\begin{pgfscope}%
\pgfsys@transformshift{0.564660in}{1.938955in}%
\pgfsys@useobject{currentmarker}{}%
\end{pgfscope}%
\end{pgfscope}%
\begin{pgfscope}%
\pgftext[x=0.289968in,y=1.886194in,left,base]{\rmfamily\fontsize{10.000000}{12.000000}\selectfont \(\displaystyle 3.0\)}%
\end{pgfscope}%
\begin{pgfscope}%
\pgfsetbuttcap%
\pgfsetroundjoin%
\definecolor{currentfill}{rgb}{0.000000,0.000000,0.000000}%
\pgfsetfillcolor{currentfill}%
\pgfsetlinewidth{0.803000pt}%
\definecolor{currentstroke}{rgb}{0.000000,0.000000,0.000000}%
\pgfsetstrokecolor{currentstroke}%
\pgfsetdash{}{0pt}%
\pgfsys@defobject{currentmarker}{\pgfqpoint{-0.048611in}{0.000000in}}{\pgfqpoint{0.000000in}{0.000000in}}{%
\pgfpathmoveto{\pgfqpoint{0.000000in}{0.000000in}}%
\pgfpathlineto{\pgfqpoint{-0.048611in}{0.000000in}}%
\pgfusepath{stroke,fill}%
}%
\begin{pgfscope}%
\pgfsys@transformshift{0.564660in}{2.587694in}%
\pgfsys@useobject{currentmarker}{}%
\end{pgfscope}%
\end{pgfscope}%
\begin{pgfscope}%
\pgftext[x=0.289968in,y=2.534933in,left,base]{\rmfamily\fontsize{10.000000}{12.000000}\selectfont \(\displaystyle 3.5\)}%
\end{pgfscope}%
\begin{pgfscope}%
\pgfsetbuttcap%
\pgfsetroundjoin%
\definecolor{currentfill}{rgb}{0.000000,0.000000,0.000000}%
\pgfsetfillcolor{currentfill}%
\pgfsetlinewidth{0.803000pt}%
\definecolor{currentstroke}{rgb}{0.000000,0.000000,0.000000}%
\pgfsetstrokecolor{currentstroke}%
\pgfsetdash{}{0pt}%
\pgfsys@defobject{currentmarker}{\pgfqpoint{-0.048611in}{0.000000in}}{\pgfqpoint{0.000000in}{0.000000in}}{%
\pgfpathmoveto{\pgfqpoint{0.000000in}{0.000000in}}%
\pgfpathlineto{\pgfqpoint{-0.048611in}{0.000000in}}%
\pgfusepath{stroke,fill}%
}%
\begin{pgfscope}%
\pgfsys@transformshift{0.564660in}{3.236434in}%
\pgfsys@useobject{currentmarker}{}%
\end{pgfscope}%
\end{pgfscope}%
\begin{pgfscope}%
\pgftext[x=0.289968in,y=3.183672in,left,base]{\rmfamily\fontsize{10.000000}{12.000000}\selectfont \(\displaystyle 4.0\)}%
\end{pgfscope}%
\begin{pgfscope}%
\pgftext[x=0.234413in,y=2.031603in,,bottom,rotate=90.000000]{\rmfamily\fontsize{10.000000}{12.000000}\selectfont \(\displaystyle \bar{t}\)}%
\end{pgfscope}%
\begin{pgfscope}%
\pgfpathrectangle{\pgfqpoint{0.564660in}{0.521603in}}{\pgfqpoint{4.650000in}{3.020000in}} %
\pgfusepath{clip}%
\pgfsetrectcap%
\pgfsetroundjoin%
\pgfsetlinewidth{1.505625pt}%
\definecolor{currentstroke}{rgb}{0.000000,0.000000,0.000000}%
\pgfsetstrokecolor{currentstroke}%
\pgfsetdash{}{0pt}%
\pgfpathmoveto{\pgfqpoint{0.776024in}{0.658876in}}%
\pgfpathlineto{\pgfqpoint{1.084953in}{0.665894in}}%
\pgfpathlineto{\pgfqpoint{1.365336in}{0.674713in}}%
\pgfpathlineto{\pgfqpoint{1.579827in}{0.683583in}}%
\pgfpathlineto{\pgfqpoint{1.784704in}{0.694297in}}%
\pgfpathlineto{\pgfqpoint{1.952101in}{0.705084in}}%
\pgfpathlineto{\pgfqpoint{2.115237in}{0.717765in}}%
\pgfpathlineto{\pgfqpoint{2.271895in}{0.732381in}}%
\pgfpathlineto{\pgfqpoint{2.405679in}{0.747130in}}%
\pgfpathlineto{\pgfqpoint{2.536027in}{0.763881in}}%
\pgfpathlineto{\pgfqpoint{2.650150in}{0.780801in}}%
\pgfpathlineto{\pgfqpoint{2.762257in}{0.799797in}}%
\pgfpathlineto{\pgfqpoint{2.862154in}{0.819000in}}%
\pgfpathlineto{\pgfqpoint{2.960781in}{0.840364in}}%
\pgfpathlineto{\pgfqpoint{3.049833in}{0.861979in}}%
\pgfpathlineto{\pgfqpoint{3.131007in}{0.883844in}}%
\pgfpathlineto{\pgfqpoint{3.212072in}{0.907984in}}%
\pgfpathlineto{\pgfqpoint{3.286556in}{0.932422in}}%
\pgfpathlineto{\pgfqpoint{3.360964in}{0.959234in}}%
\pgfpathlineto{\pgfqpoint{3.429790in}{0.986397in}}%
\pgfpathlineto{\pgfqpoint{3.498560in}{1.016040in}}%
\pgfpathlineto{\pgfqpoint{3.562536in}{1.046091in}}%
\pgfpathlineto{\pgfqpoint{3.622343in}{1.076548in}}%
\pgfpathlineto{\pgfqpoint{3.682373in}{1.109631in}}%
\pgfpathlineto{\pgfqpoint{3.738719in}{1.143182in}}%
\pgfpathlineto{\pgfqpoint{3.795237in}{1.179483in}}%
\pgfpathlineto{\pgfqpoint{3.848478in}{1.216316in}}%
\pgfpathlineto{\pgfqpoint{3.901855in}{1.256032in}}%
\pgfpathlineto{\pgfqpoint{3.952299in}{1.296348in}}%
\pgfpathlineto{\pgfqpoint{4.002851in}{1.339688in}}%
\pgfpathlineto{\pgfqpoint{4.050765in}{1.383701in}}%
\pgfpathlineto{\pgfqpoint{4.098767in}{1.430888in}}%
\pgfpathlineto{\pgfqpoint{4.144383in}{1.478824in}}%
\pgfpathlineto{\pgfqpoint{4.190071in}{1.530091in}}%
\pgfpathlineto{\pgfqpoint{4.233593in}{1.582189in}}%
\pgfpathlineto{\pgfqpoint{4.277174in}{1.637784in}}%
\pgfpathlineto{\pgfqpoint{4.320715in}{1.697007in}}%
\pgfpathlineto{\pgfqpoint{4.362284in}{1.757235in}}%
\pgfpathlineto{\pgfqpoint{4.403819in}{1.821274in}}%
\pgfpathlineto{\pgfqpoint{4.445246in}{1.889266in}}%
\pgfpathlineto{\pgfqpoint{4.486499in}{1.961363in}}%
\pgfpathlineto{\pgfqpoint{4.527523in}{2.037718in}}%
\pgfpathlineto{\pgfqpoint{4.566791in}{2.115476in}}%
\pgfpathlineto{\pgfqpoint{4.605867in}{2.197708in}}%
\pgfpathlineto{\pgfqpoint{4.644706in}{2.284583in}}%
\pgfpathlineto{\pgfqpoint{4.683273in}{2.376275in}}%
\pgfpathlineto{\pgfqpoint{4.721534in}{2.472965in}}%
\pgfpathlineto{\pgfqpoint{4.759464in}{2.574839in}}%
\pgfpathlineto{\pgfqpoint{4.797040in}{2.682090in}}%
\pgfpathlineto{\pgfqpoint{4.834244in}{2.794917in}}%
\pgfpathlineto{\pgfqpoint{4.872122in}{2.917050in}}%
\pgfpathlineto{\pgfqpoint{4.909518in}{3.045324in}}%
\pgfpathlineto{\pgfqpoint{4.946428in}{3.179962in}}%
\pgfpathlineto{\pgfqpoint{4.982851in}{3.321195in}}%
\pgfpathlineto{\pgfqpoint{5.003297in}{3.404331in}}%
\pgfpathlineto{\pgfqpoint{5.003297in}{3.404331in}}%
\pgfusepath{stroke}%
\end{pgfscope}%
\begin{pgfscope}%
\pgfsetrectcap%
\pgfsetmiterjoin%
\pgfsetlinewidth{0.803000pt}%
\definecolor{currentstroke}{rgb}{0.000000,0.000000,0.000000}%
\pgfsetstrokecolor{currentstroke}%
\pgfsetdash{}{0pt}%
\pgfpathmoveto{\pgfqpoint{0.564660in}{0.521603in}}%
\pgfpathlineto{\pgfqpoint{0.564660in}{3.541603in}}%
\pgfusepath{stroke}%
\end{pgfscope}%
\begin{pgfscope}%
\pgfsetrectcap%
\pgfsetmiterjoin%
\pgfsetlinewidth{0.803000pt}%
\definecolor{currentstroke}{rgb}{0.000000,0.000000,0.000000}%
\pgfsetstrokecolor{currentstroke}%
\pgfsetdash{}{0pt}%
\pgfpathmoveto{\pgfqpoint{5.214660in}{0.521603in}}%
\pgfpathlineto{\pgfqpoint{5.214660in}{3.541603in}}%
\pgfusepath{stroke}%
\end{pgfscope}%
\begin{pgfscope}%
\pgfsetrectcap%
\pgfsetmiterjoin%
\pgfsetlinewidth{0.803000pt}%
\definecolor{currentstroke}{rgb}{0.000000,0.000000,0.000000}%
\pgfsetstrokecolor{currentstroke}%
\pgfsetdash{}{0pt}%
\pgfpathmoveto{\pgfqpoint{0.564660in}{0.521603in}}%
\pgfpathlineto{\pgfqpoint{5.214660in}{0.521603in}}%
\pgfusepath{stroke}%
\end{pgfscope}%
\begin{pgfscope}%
\pgfsetrectcap%
\pgfsetmiterjoin%
\pgfsetlinewidth{0.803000pt}%
\definecolor{currentstroke}{rgb}{0.000000,0.000000,0.000000}%
\pgfsetstrokecolor{currentstroke}%
\pgfsetdash{}{0pt}%
\pgfpathmoveto{\pgfqpoint{0.564660in}{3.541603in}}%
\pgfpathlineto{\pgfqpoint{5.214660in}{3.541603in}}%
\pgfusepath{stroke}%
\end{pgfscope}%
\begin{pgfscope}%
\pgfsetbuttcap%
\pgfsetmiterjoin%
\definecolor{currentfill}{rgb}{1.000000,1.000000,1.000000}%
\pgfsetfillcolor{currentfill}%
\pgfsetfillopacity{0.800000}%
\pgfsetlinewidth{1.003750pt}%
\definecolor{currentstroke}{rgb}{0.800000,0.800000,0.800000}%
\pgfsetstrokecolor{currentstroke}%
\pgfsetstrokeopacity{0.800000}%
\pgfsetdash{}{0pt}%
\pgfpathmoveto{\pgfqpoint{0.661883in}{3.226635in}}%
\pgfpathlineto{\pgfqpoint{1.882863in}{3.226635in}}%
\pgfpathquadraticcurveto{\pgfqpoint{1.910641in}{3.226635in}}{\pgfqpoint{1.910641in}{3.254413in}}%
\pgfpathlineto{\pgfqpoint{1.910641in}{3.444381in}}%
\pgfpathquadraticcurveto{\pgfqpoint{1.910641in}{3.472159in}}{\pgfqpoint{1.882863in}{3.472159in}}%
\pgfpathlineto{\pgfqpoint{0.661883in}{3.472159in}}%
\pgfpathquadraticcurveto{\pgfqpoint{0.634105in}{3.472159in}}{\pgfqpoint{0.634105in}{3.444381in}}%
\pgfpathlineto{\pgfqpoint{0.634105in}{3.254413in}}%
\pgfpathquadraticcurveto{\pgfqpoint{0.634105in}{3.226635in}}{\pgfqpoint{0.661883in}{3.226635in}}%
\pgfpathclose%
\pgfusepath{stroke,fill}%
\end{pgfscope}%
\begin{pgfscope}%
\pgfsetrectcap%
\pgfsetroundjoin%
\pgfsetlinewidth{1.505625pt}%
\definecolor{currentstroke}{rgb}{0.000000,0.000000,0.000000}%
\pgfsetstrokecolor{currentstroke}%
\pgfsetdash{}{0pt}%
\pgfpathmoveto{\pgfqpoint{0.689660in}{3.359691in}}%
\pgfpathlineto{\pgfqpoint{0.967438in}{3.359691in}}%
\pgfusepath{stroke}%
\end{pgfscope}%
\begin{pgfscope}%
\pgftext[x=1.078549in,y=3.311080in,left,base]{\rmfamily\fontsize{10.000000}{12.000000}\selectfont \(\displaystyle \beta\) = 0.0005}%
\end{pgfscope}%
\end{pgfpicture}%
\makeatother%
\endgroup%

    \caption{Text.}
    \label{fig:drag}
\end{figure}
\end{document}

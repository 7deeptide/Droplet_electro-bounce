% This is file JFM2esam.tex
% first release v1.0, 20th October 1996
%       release v1.01, 29th October 1996
%       release v1.1, 25th June 1997
%       release v2.0, 27th July 2004
%       release v3.0, 16th July 2014
%   (based on JFMsampl.tex v1.3 for LaTeX2.09)
% Copyright (C) 1996, 1997, 2014 Cambridge University Press

\documentclass{jfm}
\usepackage{graphicx}
\usepackage{epstopdf, epsfig}

\newtheorem{lemma}{Lemma}
\newtheorem{corollary}{Corollary}

\shorttitle{Droplet electro-bounce}
\shortauthor{E. S. Schmidt and M. W. Weislogel}

\title{Spontaneous Droplet Jump with Electro-bouncing}

\author{Erin S. Schmidt\aff{1}
  \corresp{\email{esch2@pdx.edu}},
 \and M. W.  Weislogel\aff{1}}

\affiliation{\aff{1}Department of Mechanical Engineering, Portland State University,
Portland, OR 97211, USA
}

\begin{document}

\maketitle

\begin{abstract}
We investigate the dynamics of water droplet jumps from superhydrophobic surfaces in the presence of an electric field during a step reduction in gravity level. In the brief free-fall environment of a drop tower, when a strong non-homogeneous electric field (with a measured strength between $0.39$ and $2.36$ kV/cm) is imposed, body forces acting on the jumped droplets are primarily supplied by polarization stress and Coulombic attraction instead of gravity. The droplet charge, measured to be on the order of $2.3 \cdot(10^{-11})$ C, originates by electro-osmosis of charged species at the (PTFE coated) hydrophobic surface interface. This electric body force leads to a droplet bouncing behavior similar to well-known phenomena in 1-g, though occurring for larger drops $\sim \! \!$ 0.1 mL for a given range of impact Weber numbers, $\mathbf{We} < 20$. In 1-g, for $\mathbf{We} > 0.4$, impact recoil behavior on a super-hydrophobic surface is normally dominated by damping from contact line hysteresis and by air-layer interactions. However, in the strong electric field, the droplet bounce dynamics additionally include electrohydrodynamic effects on wettability and Cassie-Wenzel transition. This is qualitatively discussed in terms of coefficients of restitution and trends in contact time. \end{abstract}

\section{Introduction}\label{sec:intro}



%\bibliographystyle{jfm}
% Note the spaces between the initials
%\bibliography{jfm-instructions}

\end{document}

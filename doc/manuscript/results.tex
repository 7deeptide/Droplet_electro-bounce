\documentclass[10pt,a4paper]{article}
\usepackage[utf8]{inputenc}
%\usepackage{fontspec} % This line only for XeLaTeX and LuaLaTeX
\usepackage{pgfplots}
\usepackage{pgf}
\usepackage[english]{babel}
\usepackage{amsmath}
\usepackage{amsfonts}
\usepackage{amssymb}
\usepackage{graphicx}
\graphicspath{ {../figures/} }
%\usepackage{svg}
\usepackage{verbatim}
\usepackage{color,soul}
\usepackage{listings}
\usepackage{setspace}
\usepackage{float}
\author{Erin Schmidt}

\newlength\figureheight
\newlength\figurewidth
\setlength\figureheight{7cm}
\setlength\figurewidth{10cm}

\let\pgfimageWithoutPath\pgfimage 
\renewcommand{\pgfimage}[2][]{\pgfimageWithoutPath[#1]{../figures/#2}}

\usepackage[
backend=biber,
style=phys,
sorting=none
]{biblatex}
\addbibresource{thesis.bib}

\begin{document}

\doublespacing
\section{Charge Estimates}
We found the distribution of mostly likely experimental net charges for a population of the drops jumped in low-gravity. A covariance plot of the model variables is shown in Figure \ref{fig:scatter}. The multicollinear dependence of charge on drop surface area, $A$, and the characteristic electric field, $E_0$, is evident. Assuming the main effect is the interaction between charge and electric field, a Robust Least Squares model fit $q \sim kAE_0$ (using the Python \verb|statsmodels.formula.api.RLM| function), with the non-linear transformation $A = V_d^{2/3}$, found that $k=5.01 \times 10^{-11} \pm  2.85 \times 10^{-11}$ with $R^2 = 0.946$. This model uses Huber's T norm, median absolute scaling, and H1 covariance estimation. A contour plot showing the estimated drop free charge as a function of $V_d$ and $\varphi_s$ is shown in Figure \ref{fig:charge}.
\begin{figure}[H]
    \centering
    \resizebox{12cm}{!}{%% Creator: Matplotlib, PGF backend
%%
%% To include the figure in your LaTeX document, write
%%   \input{<filename>.pgf}
%%
%% Make sure the required packages are loaded in your preamble
%%   \usepackage{pgf}
%%
%% Figures using additional raster images can only be included by \input if
%% they are in the same directory as the main LaTeX file. For loading figures
%% from other directories you can use the `import` package
%%   \usepackage{import}
%% and then include the figures with
%%   \import{<path to file>}{<filename>.pgf}
%%
%% Matplotlib used the following preamble
%%   \usepackage{fontspec}
%%   \setmainfont{DejaVu Serif}
%%   \setsansfont{DejaVu Sans}
%%   \setmonofont{DejaVu Sans Mono}
%%
\begingroup%
\makeatletter%
\begin{pgfpicture}%
\pgfpathrectangle{\pgfpointorigin}{\pgfqpoint{5.618185in}{3.988185in}}%
\pgfusepath{use as bounding box, clip}%
\begin{pgfscope}%
\pgfsetbuttcap%
\pgfsetmiterjoin%
\definecolor{currentfill}{rgb}{1.000000,1.000000,1.000000}%
\pgfsetfillcolor{currentfill}%
\pgfsetlinewidth{0.000000pt}%
\definecolor{currentstroke}{rgb}{1.000000,1.000000,1.000000}%
\pgfsetstrokecolor{currentstroke}%
\pgfsetdash{}{0pt}%
\pgfpathmoveto{\pgfqpoint{0.000000in}{0.000000in}}%
\pgfpathlineto{\pgfqpoint{5.618185in}{0.000000in}}%
\pgfpathlineto{\pgfqpoint{5.618185in}{3.988185in}}%
\pgfpathlineto{\pgfqpoint{0.000000in}{3.988185in}}%
\pgfpathclose%
\pgfusepath{fill}%
\end{pgfscope}%
\begin{pgfscope}%
\pgfsetbuttcap%
\pgfsetmiterjoin%
\definecolor{currentfill}{rgb}{1.000000,1.000000,1.000000}%
\pgfsetfillcolor{currentfill}%
\pgfsetlinewidth{0.000000pt}%
\definecolor{currentstroke}{rgb}{0.000000,0.000000,0.000000}%
\pgfsetstrokecolor{currentstroke}%
\pgfsetstrokeopacity{0.000000}%
\pgfsetdash{}{0pt}%
\pgfpathmoveto{\pgfqpoint{0.833185in}{3.098185in}}%
\pgfpathlineto{\pgfqpoint{1.995685in}{3.098185in}}%
\pgfpathlineto{\pgfqpoint{1.995685in}{3.853185in}}%
\pgfpathlineto{\pgfqpoint{0.833185in}{3.853185in}}%
\pgfpathclose%
\pgfusepath{fill}%
\end{pgfscope}%
\begin{pgfscope}%
\pgfsetbuttcap%
\pgfsetroundjoin%
\definecolor{currentfill}{rgb}{0.000000,0.000000,0.000000}%
\pgfsetfillcolor{currentfill}%
\pgfsetlinewidth{0.803000pt}%
\definecolor{currentstroke}{rgb}{0.000000,0.000000,0.000000}%
\pgfsetstrokecolor{currentstroke}%
\pgfsetdash{}{0pt}%
\pgfsys@defobject{currentmarker}{\pgfqpoint{-0.048611in}{0.000000in}}{\pgfqpoint{0.000000in}{0.000000in}}{%
\pgfpathmoveto{\pgfqpoint{0.000000in}{0.000000in}}%
\pgfpathlineto{\pgfqpoint{-0.048611in}{0.000000in}}%
\pgfusepath{stroke,fill}%
}%
\begin{pgfscope}%
\pgfsys@transformshift{0.833185in}{3.164777in}%
\pgfsys@useobject{currentmarker}{}%
\end{pgfscope}%
\end{pgfscope}%
\begin{pgfscope}%
\pgftext[x=0.559259in,y=3.122567in,left,base]{\rmfamily\fontsize{8.000000}{9.600000}\selectfont 0.5}%
\end{pgfscope}%
\begin{pgfscope}%
\pgfsetbuttcap%
\pgfsetroundjoin%
\definecolor{currentfill}{rgb}{0.000000,0.000000,0.000000}%
\pgfsetfillcolor{currentfill}%
\pgfsetlinewidth{0.803000pt}%
\definecolor{currentstroke}{rgb}{0.000000,0.000000,0.000000}%
\pgfsetstrokecolor{currentstroke}%
\pgfsetdash{}{0pt}%
\pgfsys@defobject{currentmarker}{\pgfqpoint{-0.048611in}{0.000000in}}{\pgfqpoint{0.000000in}{0.000000in}}{%
\pgfpathmoveto{\pgfqpoint{0.000000in}{0.000000in}}%
\pgfpathlineto{\pgfqpoint{-0.048611in}{0.000000in}}%
\pgfusepath{stroke,fill}%
}%
\begin{pgfscope}%
\pgfsys@transformshift{0.833185in}{3.440784in}%
\pgfsys@useobject{currentmarker}{}%
\end{pgfscope}%
\end{pgfscope}%
\begin{pgfscope}%
\pgftext[x=0.559259in,y=3.398574in,left,base]{\rmfamily\fontsize{8.000000}{9.600000}\selectfont 1.0}%
\end{pgfscope}%
\begin{pgfscope}%
\pgfsetbuttcap%
\pgfsetroundjoin%
\definecolor{currentfill}{rgb}{0.000000,0.000000,0.000000}%
\pgfsetfillcolor{currentfill}%
\pgfsetlinewidth{0.803000pt}%
\definecolor{currentstroke}{rgb}{0.000000,0.000000,0.000000}%
\pgfsetstrokecolor{currentstroke}%
\pgfsetdash{}{0pt}%
\pgfsys@defobject{currentmarker}{\pgfqpoint{-0.048611in}{0.000000in}}{\pgfqpoint{0.000000in}{0.000000in}}{%
\pgfpathmoveto{\pgfqpoint{0.000000in}{0.000000in}}%
\pgfpathlineto{\pgfqpoint{-0.048611in}{0.000000in}}%
\pgfusepath{stroke,fill}%
}%
\begin{pgfscope}%
\pgfsys@transformshift{0.833185in}{3.716791in}%
\pgfsys@useobject{currentmarker}{}%
\end{pgfscope}%
\end{pgfscope}%
\begin{pgfscope}%
\pgftext[x=0.559259in,y=3.674581in,left,base]{\rmfamily\fontsize{8.000000}{9.600000}\selectfont 1.5}%
\end{pgfscope}%
\begin{pgfscope}%
\pgftext[x=0.503703in,y=3.475685in,,bottom,rotate=90.000000]{\rmfamily\fontsize{10.000000}{12.000000}\selectfont Ef0}%
\end{pgfscope}%
\begin{pgfscope}%
\pgfpathrectangle{\pgfqpoint{0.833185in}{3.098185in}}{\pgfqpoint{1.162500in}{0.755000in}} %
\pgfusepath{clip}%
\pgfsetrectcap%
\pgfsetroundjoin%
\pgfsetlinewidth{1.505625pt}%
\definecolor{currentstroke}{rgb}{0.121569,0.466667,0.705882}%
\pgfsetstrokecolor{currentstroke}%
\pgfsetdash{}{0pt}%
\pgfpathmoveto{\pgfqpoint{0.860863in}{3.508850in}}%
\pgfpathlineto{\pgfqpoint{0.889678in}{3.584059in}}%
\pgfpathlineto{\pgfqpoint{0.914059in}{3.641668in}}%
\pgfpathlineto{\pgfqpoint{0.935116in}{3.685989in}}%
\pgfpathlineto{\pgfqpoint{0.953956in}{3.720805in}}%
\pgfpathlineto{\pgfqpoint{0.971688in}{3.749038in}}%
\pgfpathlineto{\pgfqpoint{0.988312in}{3.771288in}}%
\pgfpathlineto{\pgfqpoint{1.003828in}{3.788247in}}%
\pgfpathlineto{\pgfqpoint{1.018235in}{3.800646in}}%
\pgfpathlineto{\pgfqpoint{1.031534in}{3.809221in}}%
\pgfpathlineto{\pgfqpoint{1.044833in}{3.815062in}}%
\pgfpathlineto{\pgfqpoint{1.057024in}{3.818055in}}%
\pgfpathlineto{\pgfqpoint{1.069214in}{3.818850in}}%
\pgfpathlineto{\pgfqpoint{1.081405in}{3.817519in}}%
\pgfpathlineto{\pgfqpoint{1.094704in}{3.813751in}}%
\pgfpathlineto{\pgfqpoint{1.108003in}{3.807704in}}%
\pgfpathlineto{\pgfqpoint{1.122410in}{3.798763in}}%
\pgfpathlineto{\pgfqpoint{1.139034in}{3.785647in}}%
\pgfpathlineto{\pgfqpoint{1.156766in}{3.768754in}}%
\pgfpathlineto{\pgfqpoint{1.177823in}{3.745460in}}%
\pgfpathlineto{\pgfqpoint{1.203313in}{3.713710in}}%
\pgfpathlineto{\pgfqpoint{1.239885in}{3.664179in}}%
\pgfpathlineto{\pgfqpoint{1.305272in}{3.575335in}}%
\pgfpathlineto{\pgfqpoint{1.336303in}{3.537262in}}%
\pgfpathlineto{\pgfqpoint{1.362901in}{3.508055in}}%
\pgfpathlineto{\pgfqpoint{1.387282in}{3.484432in}}%
\pgfpathlineto{\pgfqpoint{1.410556in}{3.464808in}}%
\pgfpathlineto{\pgfqpoint{1.433829in}{3.448013in}}%
\pgfpathlineto{\pgfqpoint{1.457102in}{3.433906in}}%
\pgfpathlineto{\pgfqpoint{1.481484in}{3.421768in}}%
\pgfpathlineto{\pgfqpoint{1.506974in}{3.411611in}}%
\pgfpathlineto{\pgfqpoint{1.534680in}{3.403002in}}%
\pgfpathlineto{\pgfqpoint{1.566819in}{3.395392in}}%
\pgfpathlineto{\pgfqpoint{1.613366in}{3.386905in}}%
\pgfpathlineto{\pgfqpoint{1.674319in}{3.375490in}}%
\pgfpathlineto{\pgfqpoint{1.706459in}{3.367112in}}%
\pgfpathlineto{\pgfqpoint{1.733057in}{3.357976in}}%
\pgfpathlineto{\pgfqpoint{1.757438in}{3.347350in}}%
\pgfpathlineto{\pgfqpoint{1.780712in}{3.334841in}}%
\pgfpathlineto{\pgfqpoint{1.803985in}{3.319740in}}%
\pgfpathlineto{\pgfqpoint{1.826150in}{3.302753in}}%
\pgfpathlineto{\pgfqpoint{1.849423in}{3.282046in}}%
\pgfpathlineto{\pgfqpoint{1.872696in}{3.258348in}}%
\pgfpathlineto{\pgfqpoint{1.897078in}{3.230355in}}%
\pgfpathlineto{\pgfqpoint{1.923676in}{3.196315in}}%
\pgfpathlineto{\pgfqpoint{1.952490in}{3.155732in}}%
\pgfpathlineto{\pgfqpoint{1.968006in}{3.132503in}}%
\pgfpathlineto{\pgfqpoint{1.968006in}{3.132503in}}%
\pgfusepath{stroke}%
\end{pgfscope}%
\begin{pgfscope}%
\pgfsetrectcap%
\pgfsetmiterjoin%
\pgfsetlinewidth{0.803000pt}%
\definecolor{currentstroke}{rgb}{0.000000,0.000000,0.000000}%
\pgfsetstrokecolor{currentstroke}%
\pgfsetdash{}{0pt}%
\pgfpathmoveto{\pgfqpoint{0.833185in}{3.098185in}}%
\pgfpathlineto{\pgfqpoint{0.833185in}{3.853185in}}%
\pgfusepath{stroke}%
\end{pgfscope}%
\begin{pgfscope}%
\pgfsetrectcap%
\pgfsetmiterjoin%
\pgfsetlinewidth{0.803000pt}%
\definecolor{currentstroke}{rgb}{0.000000,0.000000,0.000000}%
\pgfsetstrokecolor{currentstroke}%
\pgfsetdash{}{0pt}%
\pgfpathmoveto{\pgfqpoint{1.995685in}{3.098185in}}%
\pgfpathlineto{\pgfqpoint{1.995685in}{3.853185in}}%
\pgfusepath{stroke}%
\end{pgfscope}%
\begin{pgfscope}%
\pgfsetrectcap%
\pgfsetmiterjoin%
\pgfsetlinewidth{0.803000pt}%
\definecolor{currentstroke}{rgb}{0.000000,0.000000,0.000000}%
\pgfsetstrokecolor{currentstroke}%
\pgfsetdash{}{0pt}%
\pgfpathmoveto{\pgfqpoint{0.833185in}{3.098185in}}%
\pgfpathlineto{\pgfqpoint{1.995685in}{3.098185in}}%
\pgfusepath{stroke}%
\end{pgfscope}%
\begin{pgfscope}%
\pgfsetrectcap%
\pgfsetmiterjoin%
\pgfsetlinewidth{0.803000pt}%
\definecolor{currentstroke}{rgb}{0.000000,0.000000,0.000000}%
\pgfsetstrokecolor{currentstroke}%
\pgfsetdash{}{0pt}%
\pgfpathmoveto{\pgfqpoint{0.833185in}{3.853185in}}%
\pgfpathlineto{\pgfqpoint{1.995685in}{3.853185in}}%
\pgfusepath{stroke}%
\end{pgfscope}%
\begin{pgfscope}%
\pgfsetbuttcap%
\pgfsetmiterjoin%
\definecolor{currentfill}{rgb}{1.000000,1.000000,1.000000}%
\pgfsetfillcolor{currentfill}%
\pgfsetlinewidth{0.000000pt}%
\definecolor{currentstroke}{rgb}{0.000000,0.000000,0.000000}%
\pgfsetstrokecolor{currentstroke}%
\pgfsetstrokeopacity{0.000000}%
\pgfsetdash{}{0pt}%
\pgfpathmoveto{\pgfqpoint{1.995685in}{3.098185in}}%
\pgfpathlineto{\pgfqpoint{3.158185in}{3.098185in}}%
\pgfpathlineto{\pgfqpoint{3.158185in}{3.853185in}}%
\pgfpathlineto{\pgfqpoint{1.995685in}{3.853185in}}%
\pgfpathclose%
\pgfusepath{fill}%
\end{pgfscope}%
\begin{pgfscope}%
\pgfpathrectangle{\pgfqpoint{1.995685in}{3.098185in}}{\pgfqpoint{1.162500in}{0.755000in}} %
\pgfusepath{clip}%
\pgfsetbuttcap%
\pgfsetroundjoin%
\definecolor{currentfill}{rgb}{0.000000,0.000000,0.000000}%
\pgfsetfillcolor{currentfill}%
\pgfsetfillopacity{0.500000}%
\pgfsetlinewidth{0.000000pt}%
\definecolor{currentstroke}{rgb}{0.000000,0.000000,0.000000}%
\pgfsetstrokecolor{currentstroke}%
\pgfsetdash{}{0pt}%
\pgfpathmoveto{\pgfqpoint{3.130506in}{3.691634in}}%
\pgfpathcurveto{\pgfqpoint{3.136031in}{3.691634in}}{\pgfqpoint{3.141331in}{3.693829in}}{\pgfqpoint{3.145237in}{3.697736in}}%
\pgfpathcurveto{\pgfqpoint{3.149144in}{3.701643in}}{\pgfqpoint{3.151339in}{3.706942in}}{\pgfqpoint{3.151339in}{3.712467in}}%
\pgfpathcurveto{\pgfqpoint{3.151339in}{3.717993in}}{\pgfqpoint{3.149144in}{3.723292in}}{\pgfqpoint{3.145237in}{3.727199in}}%
\pgfpathcurveto{\pgfqpoint{3.141331in}{3.731106in}}{\pgfqpoint{3.136031in}{3.733301in}}{\pgfqpoint{3.130506in}{3.733301in}}%
\pgfpathcurveto{\pgfqpoint{3.124981in}{3.733301in}}{\pgfqpoint{3.119681in}{3.731106in}}{\pgfqpoint{3.115775in}{3.727199in}}%
\pgfpathcurveto{\pgfqpoint{3.111868in}{3.723292in}}{\pgfqpoint{3.109673in}{3.717993in}}{\pgfqpoint{3.109673in}{3.712467in}}%
\pgfpathcurveto{\pgfqpoint{3.109673in}{3.706942in}}{\pgfqpoint{3.111868in}{3.701643in}}{\pgfqpoint{3.115775in}{3.697736in}}%
\pgfpathcurveto{\pgfqpoint{3.119681in}{3.693829in}}{\pgfqpoint{3.124981in}{3.691634in}}{\pgfqpoint{3.130506in}{3.691634in}}%
\pgfpathclose%
\pgfusepath{fill}%
\end{pgfscope}%
\begin{pgfscope}%
\pgfpathrectangle{\pgfqpoint{1.995685in}{3.098185in}}{\pgfqpoint{1.162500in}{0.755000in}} %
\pgfusepath{clip}%
\pgfsetbuttcap%
\pgfsetroundjoin%
\definecolor{currentfill}{rgb}{0.000000,0.000000,0.000000}%
\pgfsetfillcolor{currentfill}%
\pgfsetfillopacity{0.500000}%
\pgfsetlinewidth{0.000000pt}%
\definecolor{currentstroke}{rgb}{0.000000,0.000000,0.000000}%
\pgfsetstrokecolor{currentstroke}%
\pgfsetdash{}{0pt}%
\pgfpathmoveto{\pgfqpoint{2.755547in}{3.134292in}}%
\pgfpathcurveto{\pgfqpoint{2.761072in}{3.134292in}}{\pgfqpoint{2.766372in}{3.136487in}}{\pgfqpoint{2.770279in}{3.140394in}}%
\pgfpathcurveto{\pgfqpoint{2.774186in}{3.144301in}}{\pgfqpoint{2.776381in}{3.149600in}}{\pgfqpoint{2.776381in}{3.155125in}}%
\pgfpathcurveto{\pgfqpoint{2.776381in}{3.160650in}}{\pgfqpoint{2.774186in}{3.165950in}}{\pgfqpoint{2.770279in}{3.169857in}}%
\pgfpathcurveto{\pgfqpoint{2.766372in}{3.173763in}}{\pgfqpoint{2.761072in}{3.175959in}}{\pgfqpoint{2.755547in}{3.175959in}}%
\pgfpathcurveto{\pgfqpoint{2.750022in}{3.175959in}}{\pgfqpoint{2.744723in}{3.173763in}}{\pgfqpoint{2.740816in}{3.169857in}}%
\pgfpathcurveto{\pgfqpoint{2.736909in}{3.165950in}}{\pgfqpoint{2.734714in}{3.160650in}}{\pgfqpoint{2.734714in}{3.155125in}}%
\pgfpathcurveto{\pgfqpoint{2.734714in}{3.149600in}}{\pgfqpoint{2.736909in}{3.144301in}}{\pgfqpoint{2.740816in}{3.140394in}}%
\pgfpathcurveto{\pgfqpoint{2.744723in}{3.136487in}}{\pgfqpoint{2.750022in}{3.134292in}}{\pgfqpoint{2.755547in}{3.134292in}}%
\pgfpathclose%
\pgfusepath{fill}%
\end{pgfscope}%
\begin{pgfscope}%
\pgfpathrectangle{\pgfqpoint{1.995685in}{3.098185in}}{\pgfqpoint{1.162500in}{0.755000in}} %
\pgfusepath{clip}%
\pgfsetbuttcap%
\pgfsetroundjoin%
\definecolor{currentfill}{rgb}{0.000000,0.000000,0.000000}%
\pgfsetfillcolor{currentfill}%
\pgfsetfillopacity{0.500000}%
\pgfsetlinewidth{0.000000pt}%
\definecolor{currentstroke}{rgb}{0.000000,0.000000,0.000000}%
\pgfsetstrokecolor{currentstroke}%
\pgfsetdash{}{0pt}%
\pgfpathmoveto{\pgfqpoint{2.755127in}{3.095327in}}%
\pgfpathcurveto{\pgfqpoint{2.760652in}{3.095327in}}{\pgfqpoint{2.765952in}{3.097523in}}{\pgfqpoint{2.769858in}{3.101429in}}%
\pgfpathcurveto{\pgfqpoint{2.773765in}{3.105336in}}{\pgfqpoint{2.775960in}{3.110636in}}{\pgfqpoint{2.775960in}{3.116161in}}%
\pgfpathcurveto{\pgfqpoint{2.775960in}{3.121686in}}{\pgfqpoint{2.773765in}{3.126985in}}{\pgfqpoint{2.769858in}{3.130892in}}%
\pgfpathcurveto{\pgfqpoint{2.765952in}{3.134799in}}{\pgfqpoint{2.760652in}{3.136994in}}{\pgfqpoint{2.755127in}{3.136994in}}%
\pgfpathcurveto{\pgfqpoint{2.749602in}{3.136994in}}{\pgfqpoint{2.744302in}{3.134799in}}{\pgfqpoint{2.740396in}{3.130892in}}%
\pgfpathcurveto{\pgfqpoint{2.736489in}{3.126985in}}{\pgfqpoint{2.734294in}{3.121686in}}{\pgfqpoint{2.734294in}{3.116161in}}%
\pgfpathcurveto{\pgfqpoint{2.734294in}{3.110636in}}{\pgfqpoint{2.736489in}{3.105336in}}{\pgfqpoint{2.740396in}{3.101429in}}%
\pgfpathcurveto{\pgfqpoint{2.744302in}{3.097523in}}{\pgfqpoint{2.749602in}{3.095327in}}{\pgfqpoint{2.755127in}{3.095327in}}%
\pgfpathclose%
\pgfusepath{fill}%
\end{pgfscope}%
\begin{pgfscope}%
\pgfpathrectangle{\pgfqpoint{1.995685in}{3.098185in}}{\pgfqpoint{1.162500in}{0.755000in}} %
\pgfusepath{clip}%
\pgfsetbuttcap%
\pgfsetroundjoin%
\definecolor{currentfill}{rgb}{0.000000,0.000000,0.000000}%
\pgfsetfillcolor{currentfill}%
\pgfsetfillopacity{0.500000}%
\pgfsetlinewidth{0.000000pt}%
\definecolor{currentstroke}{rgb}{0.000000,0.000000,0.000000}%
\pgfsetstrokecolor{currentstroke}%
\pgfsetdash{}{0pt}%
\pgfpathmoveto{\pgfqpoint{2.317114in}{3.302828in}}%
\pgfpathcurveto{\pgfqpoint{2.322639in}{3.302828in}}{\pgfqpoint{2.327939in}{3.305023in}}{\pgfqpoint{2.331845in}{3.308930in}}%
\pgfpathcurveto{\pgfqpoint{2.335752in}{3.312837in}}{\pgfqpoint{2.337947in}{3.318136in}}{\pgfqpoint{2.337947in}{3.323661in}}%
\pgfpathcurveto{\pgfqpoint{2.337947in}{3.329186in}}{\pgfqpoint{2.335752in}{3.334486in}}{\pgfqpoint{2.331845in}{3.338393in}}%
\pgfpathcurveto{\pgfqpoint{2.327939in}{3.342300in}}{\pgfqpoint{2.322639in}{3.344495in}}{\pgfqpoint{2.317114in}{3.344495in}}%
\pgfpathcurveto{\pgfqpoint{2.311589in}{3.344495in}}{\pgfqpoint{2.306289in}{3.342300in}}{\pgfqpoint{2.302383in}{3.338393in}}%
\pgfpathcurveto{\pgfqpoint{2.298476in}{3.334486in}}{\pgfqpoint{2.296281in}{3.329186in}}{\pgfqpoint{2.296281in}{3.323661in}}%
\pgfpathcurveto{\pgfqpoint{2.296281in}{3.318136in}}{\pgfqpoint{2.298476in}{3.312837in}}{\pgfqpoint{2.302383in}{3.308930in}}%
\pgfpathcurveto{\pgfqpoint{2.306289in}{3.305023in}}{\pgfqpoint{2.311589in}{3.302828in}}{\pgfqpoint{2.317114in}{3.302828in}}%
\pgfpathclose%
\pgfusepath{fill}%
\end{pgfscope}%
\begin{pgfscope}%
\pgfpathrectangle{\pgfqpoint{1.995685in}{3.098185in}}{\pgfqpoint{1.162500in}{0.755000in}} %
\pgfusepath{clip}%
\pgfsetbuttcap%
\pgfsetroundjoin%
\definecolor{currentfill}{rgb}{0.000000,0.000000,0.000000}%
\pgfsetfillcolor{currentfill}%
\pgfsetfillopacity{0.500000}%
\pgfsetlinewidth{0.000000pt}%
\definecolor{currentstroke}{rgb}{0.000000,0.000000,0.000000}%
\pgfsetstrokecolor{currentstroke}%
\pgfsetdash{}{0pt}%
\pgfpathmoveto{\pgfqpoint{2.241285in}{3.145618in}}%
\pgfpathcurveto{\pgfqpoint{2.246810in}{3.145618in}}{\pgfqpoint{2.252110in}{3.147813in}}{\pgfqpoint{2.256017in}{3.151720in}}%
\pgfpathcurveto{\pgfqpoint{2.259923in}{3.155626in}}{\pgfqpoint{2.262118in}{3.160926in}}{\pgfqpoint{2.262118in}{3.166451in}}%
\pgfpathcurveto{\pgfqpoint{2.262118in}{3.171976in}}{\pgfqpoint{2.259923in}{3.177276in}}{\pgfqpoint{2.256017in}{3.181182in}}%
\pgfpathcurveto{\pgfqpoint{2.252110in}{3.185089in}}{\pgfqpoint{2.246810in}{3.187284in}}{\pgfqpoint{2.241285in}{3.187284in}}%
\pgfpathcurveto{\pgfqpoint{2.235760in}{3.187284in}}{\pgfqpoint{2.230461in}{3.185089in}}{\pgfqpoint{2.226554in}{3.181182in}}%
\pgfpathcurveto{\pgfqpoint{2.222647in}{3.177276in}}{\pgfqpoint{2.220452in}{3.171976in}}{\pgfqpoint{2.220452in}{3.166451in}}%
\pgfpathcurveto{\pgfqpoint{2.220452in}{3.160926in}}{\pgfqpoint{2.222647in}{3.155626in}}{\pgfqpoint{2.226554in}{3.151720in}}%
\pgfpathcurveto{\pgfqpoint{2.230461in}{3.147813in}}{\pgfqpoint{2.235760in}{3.145618in}}{\pgfqpoint{2.241285in}{3.145618in}}%
\pgfpathclose%
\pgfusepath{fill}%
\end{pgfscope}%
\begin{pgfscope}%
\pgfpathrectangle{\pgfqpoint{1.995685in}{3.098185in}}{\pgfqpoint{1.162500in}{0.755000in}} %
\pgfusepath{clip}%
\pgfsetbuttcap%
\pgfsetroundjoin%
\definecolor{currentfill}{rgb}{0.000000,0.000000,0.000000}%
\pgfsetfillcolor{currentfill}%
\pgfsetfillopacity{0.500000}%
\pgfsetlinewidth{0.000000pt}%
\definecolor{currentstroke}{rgb}{0.000000,0.000000,0.000000}%
\pgfsetstrokecolor{currentstroke}%
\pgfsetdash{}{0pt}%
\pgfpathmoveto{\pgfqpoint{2.106066in}{3.249924in}}%
\pgfpathcurveto{\pgfqpoint{2.111591in}{3.249924in}}{\pgfqpoint{2.116890in}{3.252119in}}{\pgfqpoint{2.120797in}{3.256026in}}%
\pgfpathcurveto{\pgfqpoint{2.124704in}{3.259932in}}{\pgfqpoint{2.126899in}{3.265232in}}{\pgfqpoint{2.126899in}{3.270757in}}%
\pgfpathcurveto{\pgfqpoint{2.126899in}{3.276282in}}{\pgfqpoint{2.124704in}{3.281582in}}{\pgfqpoint{2.120797in}{3.285488in}}%
\pgfpathcurveto{\pgfqpoint{2.116890in}{3.289395in}}{\pgfqpoint{2.111591in}{3.291590in}}{\pgfqpoint{2.106066in}{3.291590in}}%
\pgfpathcurveto{\pgfqpoint{2.100541in}{3.291590in}}{\pgfqpoint{2.095241in}{3.289395in}}{\pgfqpoint{2.091334in}{3.285488in}}%
\pgfpathcurveto{\pgfqpoint{2.087428in}{3.281582in}}{\pgfqpoint{2.085232in}{3.276282in}}{\pgfqpoint{2.085232in}{3.270757in}}%
\pgfpathcurveto{\pgfqpoint{2.085232in}{3.265232in}}{\pgfqpoint{2.087428in}{3.259932in}}{\pgfqpoint{2.091334in}{3.256026in}}%
\pgfpathcurveto{\pgfqpoint{2.095241in}{3.252119in}}{\pgfqpoint{2.100541in}{3.249924in}}{\pgfqpoint{2.106066in}{3.249924in}}%
\pgfpathclose%
\pgfusepath{fill}%
\end{pgfscope}%
\begin{pgfscope}%
\pgfpathrectangle{\pgfqpoint{1.995685in}{3.098185in}}{\pgfqpoint{1.162500in}{0.755000in}} %
\pgfusepath{clip}%
\pgfsetbuttcap%
\pgfsetroundjoin%
\definecolor{currentfill}{rgb}{0.000000,0.000000,0.000000}%
\pgfsetfillcolor{currentfill}%
\pgfsetfillopacity{0.500000}%
\pgfsetlinewidth{0.000000pt}%
\definecolor{currentstroke}{rgb}{0.000000,0.000000,0.000000}%
\pgfsetstrokecolor{currentstroke}%
\pgfsetdash{}{0pt}%
\pgfpathmoveto{\pgfqpoint{2.023363in}{3.240978in}}%
\pgfpathcurveto{\pgfqpoint{2.028888in}{3.240978in}}{\pgfqpoint{2.034188in}{3.243173in}}{\pgfqpoint{2.038095in}{3.247080in}}%
\pgfpathcurveto{\pgfqpoint{2.042001in}{3.250987in}}{\pgfqpoint{2.044196in}{3.256286in}}{\pgfqpoint{2.044196in}{3.261812in}}%
\pgfpathcurveto{\pgfqpoint{2.044196in}{3.267337in}}{\pgfqpoint{2.042001in}{3.272636in}}{\pgfqpoint{2.038095in}{3.276543in}}%
\pgfpathcurveto{\pgfqpoint{2.034188in}{3.280450in}}{\pgfqpoint{2.028888in}{3.282645in}}{\pgfqpoint{2.023363in}{3.282645in}}%
\pgfpathcurveto{\pgfqpoint{2.017838in}{3.282645in}}{\pgfqpoint{2.012539in}{3.280450in}}{\pgfqpoint{2.008632in}{3.276543in}}%
\pgfpathcurveto{\pgfqpoint{2.004725in}{3.272636in}}{\pgfqpoint{2.002530in}{3.267337in}}{\pgfqpoint{2.002530in}{3.261812in}}%
\pgfpathcurveto{\pgfqpoint{2.002530in}{3.256286in}}{\pgfqpoint{2.004725in}{3.250987in}}{\pgfqpoint{2.008632in}{3.247080in}}%
\pgfpathcurveto{\pgfqpoint{2.012539in}{3.243173in}}{\pgfqpoint{2.017838in}{3.240978in}}{\pgfqpoint{2.023363in}{3.240978in}}%
\pgfpathclose%
\pgfusepath{fill}%
\end{pgfscope}%
\begin{pgfscope}%
\pgfpathrectangle{\pgfqpoint{1.995685in}{3.098185in}}{\pgfqpoint{1.162500in}{0.755000in}} %
\pgfusepath{clip}%
\pgfsetbuttcap%
\pgfsetroundjoin%
\definecolor{currentfill}{rgb}{0.000000,0.000000,0.000000}%
\pgfsetfillcolor{currentfill}%
\pgfsetfillopacity{0.500000}%
\pgfsetlinewidth{0.000000pt}%
\definecolor{currentstroke}{rgb}{0.000000,0.000000,0.000000}%
\pgfsetstrokecolor{currentstroke}%
\pgfsetdash{}{0pt}%
\pgfpathmoveto{\pgfqpoint{2.907651in}{3.768072in}}%
\pgfpathcurveto{\pgfqpoint{2.913176in}{3.768072in}}{\pgfqpoint{2.918476in}{3.770267in}}{\pgfqpoint{2.922382in}{3.774174in}}%
\pgfpathcurveto{\pgfqpoint{2.926289in}{3.778081in}}{\pgfqpoint{2.928484in}{3.783381in}}{\pgfqpoint{2.928484in}{3.788906in}}%
\pgfpathcurveto{\pgfqpoint{2.928484in}{3.794431in}}{\pgfqpoint{2.926289in}{3.799730in}}{\pgfqpoint{2.922382in}{3.803637in}}%
\pgfpathcurveto{\pgfqpoint{2.918476in}{3.807544in}}{\pgfqpoint{2.913176in}{3.809739in}}{\pgfqpoint{2.907651in}{3.809739in}}%
\pgfpathcurveto{\pgfqpoint{2.902126in}{3.809739in}}{\pgfqpoint{2.896827in}{3.807544in}}{\pgfqpoint{2.892920in}{3.803637in}}%
\pgfpathcurveto{\pgfqpoint{2.889013in}{3.799730in}}{\pgfqpoint{2.886818in}{3.794431in}}{\pgfqpoint{2.886818in}{3.788906in}}%
\pgfpathcurveto{\pgfqpoint{2.886818in}{3.783381in}}{\pgfqpoint{2.889013in}{3.778081in}}{\pgfqpoint{2.892920in}{3.774174in}}%
\pgfpathcurveto{\pgfqpoint{2.896827in}{3.770267in}}{\pgfqpoint{2.902126in}{3.768072in}}{\pgfqpoint{2.907651in}{3.768072in}}%
\pgfpathclose%
\pgfusepath{fill}%
\end{pgfscope}%
\begin{pgfscope}%
\pgfpathrectangle{\pgfqpoint{1.995685in}{3.098185in}}{\pgfqpoint{1.162500in}{0.755000in}} %
\pgfusepath{clip}%
\pgfsetbuttcap%
\pgfsetroundjoin%
\definecolor{currentfill}{rgb}{0.000000,0.000000,0.000000}%
\pgfsetfillcolor{currentfill}%
\pgfsetfillopacity{0.500000}%
\pgfsetlinewidth{0.000000pt}%
\definecolor{currentstroke}{rgb}{0.000000,0.000000,0.000000}%
\pgfsetstrokecolor{currentstroke}%
\pgfsetdash{}{0pt}%
\pgfpathmoveto{\pgfqpoint{2.630361in}{3.814375in}}%
\pgfpathcurveto{\pgfqpoint{2.635886in}{3.814375in}}{\pgfqpoint{2.641186in}{3.816570in}}{\pgfqpoint{2.645093in}{3.820477in}}%
\pgfpathcurveto{\pgfqpoint{2.649000in}{3.824384in}}{\pgfqpoint{2.651195in}{3.829683in}}{\pgfqpoint{2.651195in}{3.835208in}}%
\pgfpathcurveto{\pgfqpoint{2.651195in}{3.840733in}}{\pgfqpoint{2.649000in}{3.846033in}}{\pgfqpoint{2.645093in}{3.849940in}}%
\pgfpathcurveto{\pgfqpoint{2.641186in}{3.853847in}}{\pgfqpoint{2.635886in}{3.856042in}}{\pgfqpoint{2.630361in}{3.856042in}}%
\pgfpathcurveto{\pgfqpoint{2.624836in}{3.856042in}}{\pgfqpoint{2.619537in}{3.853847in}}{\pgfqpoint{2.615630in}{3.849940in}}%
\pgfpathcurveto{\pgfqpoint{2.611723in}{3.846033in}}{\pgfqpoint{2.609528in}{3.840733in}}{\pgfqpoint{2.609528in}{3.835208in}}%
\pgfpathcurveto{\pgfqpoint{2.609528in}{3.829683in}}{\pgfqpoint{2.611723in}{3.824384in}}{\pgfqpoint{2.615630in}{3.820477in}}%
\pgfpathcurveto{\pgfqpoint{2.619537in}{3.816570in}}{\pgfqpoint{2.624836in}{3.814375in}}{\pgfqpoint{2.630361in}{3.814375in}}%
\pgfpathclose%
\pgfusepath{fill}%
\end{pgfscope}%
\begin{pgfscope}%
\pgfpathrectangle{\pgfqpoint{1.995685in}{3.098185in}}{\pgfqpoint{1.162500in}{0.755000in}} %
\pgfusepath{clip}%
\pgfsetbuttcap%
\pgfsetroundjoin%
\definecolor{currentfill}{rgb}{0.000000,0.000000,0.000000}%
\pgfsetfillcolor{currentfill}%
\pgfsetfillopacity{0.500000}%
\pgfsetlinewidth{0.000000pt}%
\definecolor{currentstroke}{rgb}{0.000000,0.000000,0.000000}%
\pgfsetstrokecolor{currentstroke}%
\pgfsetdash{}{0pt}%
\pgfpathmoveto{\pgfqpoint{2.487752in}{3.667724in}}%
\pgfpathcurveto{\pgfqpoint{2.493277in}{3.667724in}}{\pgfqpoint{2.498576in}{3.669919in}}{\pgfqpoint{2.502483in}{3.673826in}}%
\pgfpathcurveto{\pgfqpoint{2.506390in}{3.677733in}}{\pgfqpoint{2.508585in}{3.683032in}}{\pgfqpoint{2.508585in}{3.688558in}}%
\pgfpathcurveto{\pgfqpoint{2.508585in}{3.694083in}}{\pgfqpoint{2.506390in}{3.699382in}}{\pgfqpoint{2.502483in}{3.703289in}}%
\pgfpathcurveto{\pgfqpoint{2.498576in}{3.707196in}}{\pgfqpoint{2.493277in}{3.709391in}}{\pgfqpoint{2.487752in}{3.709391in}}%
\pgfpathcurveto{\pgfqpoint{2.482227in}{3.709391in}}{\pgfqpoint{2.476927in}{3.707196in}}{\pgfqpoint{2.473020in}{3.703289in}}%
\pgfpathcurveto{\pgfqpoint{2.469113in}{3.699382in}}{\pgfqpoint{2.466918in}{3.694083in}}{\pgfqpoint{2.466918in}{3.688558in}}%
\pgfpathcurveto{\pgfqpoint{2.466918in}{3.683032in}}{\pgfqpoint{2.469113in}{3.677733in}}{\pgfqpoint{2.473020in}{3.673826in}}%
\pgfpathcurveto{\pgfqpoint{2.476927in}{3.669919in}}{\pgfqpoint{2.482227in}{3.667724in}}{\pgfqpoint{2.487752in}{3.667724in}}%
\pgfpathclose%
\pgfusepath{fill}%
\end{pgfscope}%
\begin{pgfscope}%
\pgfpathrectangle{\pgfqpoint{1.995685in}{3.098185in}}{\pgfqpoint{1.162500in}{0.755000in}} %
\pgfusepath{clip}%
\pgfsetbuttcap%
\pgfsetroundjoin%
\definecolor{currentfill}{rgb}{0.000000,0.000000,0.000000}%
\pgfsetfillcolor{currentfill}%
\pgfsetfillopacity{0.500000}%
\pgfsetlinewidth{0.000000pt}%
\definecolor{currentstroke}{rgb}{0.000000,0.000000,0.000000}%
\pgfsetstrokecolor{currentstroke}%
\pgfsetdash{}{0pt}%
\pgfpathmoveto{\pgfqpoint{3.000738in}{3.223705in}}%
\pgfpathcurveto{\pgfqpoint{3.006263in}{3.223705in}}{\pgfqpoint{3.011563in}{3.225900in}}{\pgfqpoint{3.015470in}{3.229807in}}%
\pgfpathcurveto{\pgfqpoint{3.019376in}{3.233713in}}{\pgfqpoint{3.021572in}{3.239013in}}{\pgfqpoint{3.021572in}{3.244538in}}%
\pgfpathcurveto{\pgfqpoint{3.021572in}{3.250063in}}{\pgfqpoint{3.019376in}{3.255363in}}{\pgfqpoint{3.015470in}{3.259269in}}%
\pgfpathcurveto{\pgfqpoint{3.011563in}{3.263176in}}{\pgfqpoint{3.006263in}{3.265371in}}{\pgfqpoint{3.000738in}{3.265371in}}%
\pgfpathcurveto{\pgfqpoint{2.995213in}{3.265371in}}{\pgfqpoint{2.989914in}{3.263176in}}{\pgfqpoint{2.986007in}{3.259269in}}%
\pgfpathcurveto{\pgfqpoint{2.982100in}{3.255363in}}{\pgfqpoint{2.979905in}{3.250063in}}{\pgfqpoint{2.979905in}{3.244538in}}%
\pgfpathcurveto{\pgfqpoint{2.979905in}{3.239013in}}{\pgfqpoint{2.982100in}{3.233713in}}{\pgfqpoint{2.986007in}{3.229807in}}%
\pgfpathcurveto{\pgfqpoint{2.989914in}{3.225900in}}{\pgfqpoint{2.995213in}{3.223705in}}{\pgfqpoint{3.000738in}{3.223705in}}%
\pgfpathclose%
\pgfusepath{fill}%
\end{pgfscope}%
\begin{pgfscope}%
\pgfpathrectangle{\pgfqpoint{1.995685in}{3.098185in}}{\pgfqpoint{1.162500in}{0.755000in}} %
\pgfusepath{clip}%
\pgfsetbuttcap%
\pgfsetroundjoin%
\definecolor{currentfill}{rgb}{0.000000,0.000000,0.000000}%
\pgfsetfillcolor{currentfill}%
\pgfsetfillopacity{0.500000}%
\pgfsetlinewidth{0.000000pt}%
\definecolor{currentstroke}{rgb}{0.000000,0.000000,0.000000}%
\pgfsetstrokecolor{currentstroke}%
\pgfsetdash{}{0pt}%
\pgfpathmoveto{\pgfqpoint{2.318352in}{3.524055in}}%
\pgfpathcurveto{\pgfqpoint{2.323877in}{3.524055in}}{\pgfqpoint{2.329176in}{3.526250in}}{\pgfqpoint{2.333083in}{3.530157in}}%
\pgfpathcurveto{\pgfqpoint{2.336990in}{3.534064in}}{\pgfqpoint{2.339185in}{3.539363in}}{\pgfqpoint{2.339185in}{3.544888in}}%
\pgfpathcurveto{\pgfqpoint{2.339185in}{3.550413in}}{\pgfqpoint{2.336990in}{3.555713in}}{\pgfqpoint{2.333083in}{3.559620in}}%
\pgfpathcurveto{\pgfqpoint{2.329176in}{3.563526in}}{\pgfqpoint{2.323877in}{3.565721in}}{\pgfqpoint{2.318352in}{3.565721in}}%
\pgfpathcurveto{\pgfqpoint{2.312827in}{3.565721in}}{\pgfqpoint{2.307527in}{3.563526in}}{\pgfqpoint{2.303620in}{3.559620in}}%
\pgfpathcurveto{\pgfqpoint{2.299713in}{3.555713in}}{\pgfqpoint{2.297518in}{3.550413in}}{\pgfqpoint{2.297518in}{3.544888in}}%
\pgfpathcurveto{\pgfqpoint{2.297518in}{3.539363in}}{\pgfqpoint{2.299713in}{3.534064in}}{\pgfqpoint{2.303620in}{3.530157in}}%
\pgfpathcurveto{\pgfqpoint{2.307527in}{3.526250in}}{\pgfqpoint{2.312827in}{3.524055in}}{\pgfqpoint{2.318352in}{3.524055in}}%
\pgfpathclose%
\pgfusepath{fill}%
\end{pgfscope}%
\begin{pgfscope}%
\pgfpathrectangle{\pgfqpoint{1.995685in}{3.098185in}}{\pgfqpoint{1.162500in}{0.755000in}} %
\pgfusepath{clip}%
\pgfsetbuttcap%
\pgfsetroundjoin%
\definecolor{currentfill}{rgb}{0.000000,0.000000,0.000000}%
\pgfsetfillcolor{currentfill}%
\pgfsetfillopacity{0.500000}%
\pgfsetlinewidth{0.000000pt}%
\definecolor{currentstroke}{rgb}{0.000000,0.000000,0.000000}%
\pgfsetstrokecolor{currentstroke}%
\pgfsetdash{}{0pt}%
\pgfpathmoveto{\pgfqpoint{2.410257in}{3.446907in}}%
\pgfpathcurveto{\pgfqpoint{2.415782in}{3.446907in}}{\pgfqpoint{2.421082in}{3.449102in}}{\pgfqpoint{2.424989in}{3.453009in}}%
\pgfpathcurveto{\pgfqpoint{2.428896in}{3.456916in}}{\pgfqpoint{2.431091in}{3.462215in}}{\pgfqpoint{2.431091in}{3.467740in}}%
\pgfpathcurveto{\pgfqpoint{2.431091in}{3.473265in}}{\pgfqpoint{2.428896in}{3.478565in}}{\pgfqpoint{2.424989in}{3.482472in}}%
\pgfpathcurveto{\pgfqpoint{2.421082in}{3.486379in}}{\pgfqpoint{2.415782in}{3.488574in}}{\pgfqpoint{2.410257in}{3.488574in}}%
\pgfpathcurveto{\pgfqpoint{2.404732in}{3.488574in}}{\pgfqpoint{2.399433in}{3.486379in}}{\pgfqpoint{2.395526in}{3.482472in}}%
\pgfpathcurveto{\pgfqpoint{2.391619in}{3.478565in}}{\pgfqpoint{2.389424in}{3.473265in}}{\pgfqpoint{2.389424in}{3.467740in}}%
\pgfpathcurveto{\pgfqpoint{2.389424in}{3.462215in}}{\pgfqpoint{2.391619in}{3.456916in}}{\pgfqpoint{2.395526in}{3.453009in}}%
\pgfpathcurveto{\pgfqpoint{2.399433in}{3.449102in}}{\pgfqpoint{2.404732in}{3.446907in}}{\pgfqpoint{2.410257in}{3.446907in}}%
\pgfpathclose%
\pgfusepath{fill}%
\end{pgfscope}%
\begin{pgfscope}%
\pgfpathrectangle{\pgfqpoint{1.995685in}{3.098185in}}{\pgfqpoint{1.162500in}{0.755000in}} %
\pgfusepath{clip}%
\pgfsetbuttcap%
\pgfsetroundjoin%
\definecolor{currentfill}{rgb}{0.000000,0.000000,0.000000}%
\pgfsetfillcolor{currentfill}%
\pgfsetfillopacity{0.500000}%
\pgfsetlinewidth{0.000000pt}%
\definecolor{currentstroke}{rgb}{0.000000,0.000000,0.000000}%
\pgfsetstrokecolor{currentstroke}%
\pgfsetdash{}{0pt}%
\pgfpathmoveto{\pgfqpoint{2.752801in}{3.220444in}}%
\pgfpathcurveto{\pgfqpoint{2.758327in}{3.220444in}}{\pgfqpoint{2.763626in}{3.222640in}}{\pgfqpoint{2.767533in}{3.226546in}}%
\pgfpathcurveto{\pgfqpoint{2.771440in}{3.230453in}}{\pgfqpoint{2.773635in}{3.235753in}}{\pgfqpoint{2.773635in}{3.241278in}}%
\pgfpathcurveto{\pgfqpoint{2.773635in}{3.246803in}}{\pgfqpoint{2.771440in}{3.252102in}}{\pgfqpoint{2.767533in}{3.256009in}}%
\pgfpathcurveto{\pgfqpoint{2.763626in}{3.259916in}}{\pgfqpoint{2.758327in}{3.262111in}}{\pgfqpoint{2.752801in}{3.262111in}}%
\pgfpathcurveto{\pgfqpoint{2.747276in}{3.262111in}}{\pgfqpoint{2.741977in}{3.259916in}}{\pgfqpoint{2.738070in}{3.256009in}}%
\pgfpathcurveto{\pgfqpoint{2.734163in}{3.252102in}}{\pgfqpoint{2.731968in}{3.246803in}}{\pgfqpoint{2.731968in}{3.241278in}}%
\pgfpathcurveto{\pgfqpoint{2.731968in}{3.235753in}}{\pgfqpoint{2.734163in}{3.230453in}}{\pgfqpoint{2.738070in}{3.226546in}}%
\pgfpathcurveto{\pgfqpoint{2.741977in}{3.222640in}}{\pgfqpoint{2.747276in}{3.220444in}}{\pgfqpoint{2.752801in}{3.220444in}}%
\pgfpathclose%
\pgfusepath{fill}%
\end{pgfscope}%
\begin{pgfscope}%
\pgfpathrectangle{\pgfqpoint{1.995685in}{3.098185in}}{\pgfqpoint{1.162500in}{0.755000in}} %
\pgfusepath{clip}%
\pgfsetbuttcap%
\pgfsetroundjoin%
\definecolor{currentfill}{rgb}{0.000000,0.000000,0.000000}%
\pgfsetfillcolor{currentfill}%
\pgfsetfillopacity{0.500000}%
\pgfsetlinewidth{0.000000pt}%
\definecolor{currentstroke}{rgb}{0.000000,0.000000,0.000000}%
\pgfsetstrokecolor{currentstroke}%
\pgfsetdash{}{0pt}%
\pgfpathmoveto{\pgfqpoint{2.103337in}{3.505459in}}%
\pgfpathcurveto{\pgfqpoint{2.108862in}{3.505459in}}{\pgfqpoint{2.114161in}{3.507654in}}{\pgfqpoint{2.118068in}{3.511561in}}%
\pgfpathcurveto{\pgfqpoint{2.121975in}{3.515468in}}{\pgfqpoint{2.124170in}{3.520768in}}{\pgfqpoint{2.124170in}{3.526293in}}%
\pgfpathcurveto{\pgfqpoint{2.124170in}{3.531818in}}{\pgfqpoint{2.121975in}{3.537117in}}{\pgfqpoint{2.118068in}{3.541024in}}%
\pgfpathcurveto{\pgfqpoint{2.114161in}{3.544931in}}{\pgfqpoint{2.108862in}{3.547126in}}{\pgfqpoint{2.103337in}{3.547126in}}%
\pgfpathcurveto{\pgfqpoint{2.097812in}{3.547126in}}{\pgfqpoint{2.092512in}{3.544931in}}{\pgfqpoint{2.088605in}{3.541024in}}%
\pgfpathcurveto{\pgfqpoint{2.084698in}{3.537117in}}{\pgfqpoint{2.082503in}{3.531818in}}{\pgfqpoint{2.082503in}{3.526293in}}%
\pgfpathcurveto{\pgfqpoint{2.082503in}{3.520768in}}{\pgfqpoint{2.084698in}{3.515468in}}{\pgfqpoint{2.088605in}{3.511561in}}%
\pgfpathcurveto{\pgfqpoint{2.092512in}{3.507654in}}{\pgfqpoint{2.097812in}{3.505459in}}{\pgfqpoint{2.103337in}{3.505459in}}%
\pgfpathclose%
\pgfusepath{fill}%
\end{pgfscope}%
\begin{pgfscope}%
\pgfsetrectcap%
\pgfsetmiterjoin%
\pgfsetlinewidth{0.803000pt}%
\definecolor{currentstroke}{rgb}{0.000000,0.000000,0.000000}%
\pgfsetstrokecolor{currentstroke}%
\pgfsetdash{}{0pt}%
\pgfpathmoveto{\pgfqpoint{1.995685in}{3.098185in}}%
\pgfpathlineto{\pgfqpoint{1.995685in}{3.853185in}}%
\pgfusepath{stroke}%
\end{pgfscope}%
\begin{pgfscope}%
\pgfsetrectcap%
\pgfsetmiterjoin%
\pgfsetlinewidth{0.803000pt}%
\definecolor{currentstroke}{rgb}{0.000000,0.000000,0.000000}%
\pgfsetstrokecolor{currentstroke}%
\pgfsetdash{}{0pt}%
\pgfpathmoveto{\pgfqpoint{3.158185in}{3.098185in}}%
\pgfpathlineto{\pgfqpoint{3.158185in}{3.853185in}}%
\pgfusepath{stroke}%
\end{pgfscope}%
\begin{pgfscope}%
\pgfsetrectcap%
\pgfsetmiterjoin%
\pgfsetlinewidth{0.803000pt}%
\definecolor{currentstroke}{rgb}{0.000000,0.000000,0.000000}%
\pgfsetstrokecolor{currentstroke}%
\pgfsetdash{}{0pt}%
\pgfpathmoveto{\pgfqpoint{1.995685in}{3.098185in}}%
\pgfpathlineto{\pgfqpoint{3.158185in}{3.098185in}}%
\pgfusepath{stroke}%
\end{pgfscope}%
\begin{pgfscope}%
\pgfsetrectcap%
\pgfsetmiterjoin%
\pgfsetlinewidth{0.803000pt}%
\definecolor{currentstroke}{rgb}{0.000000,0.000000,0.000000}%
\pgfsetstrokecolor{currentstroke}%
\pgfsetdash{}{0pt}%
\pgfpathmoveto{\pgfqpoint{1.995685in}{3.853185in}}%
\pgfpathlineto{\pgfqpoint{3.158185in}{3.853185in}}%
\pgfusepath{stroke}%
\end{pgfscope}%
\begin{pgfscope}%
\pgfsetbuttcap%
\pgfsetmiterjoin%
\definecolor{currentfill}{rgb}{1.000000,1.000000,1.000000}%
\pgfsetfillcolor{currentfill}%
\pgfsetlinewidth{0.000000pt}%
\definecolor{currentstroke}{rgb}{0.000000,0.000000,0.000000}%
\pgfsetstrokecolor{currentstroke}%
\pgfsetstrokeopacity{0.000000}%
\pgfsetdash{}{0pt}%
\pgfpathmoveto{\pgfqpoint{3.158185in}{3.098185in}}%
\pgfpathlineto{\pgfqpoint{4.320685in}{3.098185in}}%
\pgfpathlineto{\pgfqpoint{4.320685in}{3.853185in}}%
\pgfpathlineto{\pgfqpoint{3.158185in}{3.853185in}}%
\pgfpathclose%
\pgfusepath{fill}%
\end{pgfscope}%
\begin{pgfscope}%
\pgfpathrectangle{\pgfqpoint{3.158185in}{3.098185in}}{\pgfqpoint{1.162500in}{0.755000in}} %
\pgfusepath{clip}%
\pgfsetbuttcap%
\pgfsetroundjoin%
\definecolor{currentfill}{rgb}{0.000000,0.000000,0.000000}%
\pgfsetfillcolor{currentfill}%
\pgfsetfillopacity{0.500000}%
\pgfsetlinewidth{0.000000pt}%
\definecolor{currentstroke}{rgb}{0.000000,0.000000,0.000000}%
\pgfsetstrokecolor{currentstroke}%
\pgfsetdash{}{0pt}%
\pgfpathmoveto{\pgfqpoint{4.293006in}{3.691634in}}%
\pgfpathcurveto{\pgfqpoint{4.298531in}{3.691634in}}{\pgfqpoint{4.303831in}{3.693829in}}{\pgfqpoint{4.307737in}{3.697736in}}%
\pgfpathcurveto{\pgfqpoint{4.311644in}{3.701643in}}{\pgfqpoint{4.313839in}{3.706942in}}{\pgfqpoint{4.313839in}{3.712467in}}%
\pgfpathcurveto{\pgfqpoint{4.313839in}{3.717993in}}{\pgfqpoint{4.311644in}{3.723292in}}{\pgfqpoint{4.307737in}{3.727199in}}%
\pgfpathcurveto{\pgfqpoint{4.303831in}{3.731106in}}{\pgfqpoint{4.298531in}{3.733301in}}{\pgfqpoint{4.293006in}{3.733301in}}%
\pgfpathcurveto{\pgfqpoint{4.287481in}{3.733301in}}{\pgfqpoint{4.282181in}{3.731106in}}{\pgfqpoint{4.278275in}{3.727199in}}%
\pgfpathcurveto{\pgfqpoint{4.274368in}{3.723292in}}{\pgfqpoint{4.272173in}{3.717993in}}{\pgfqpoint{4.272173in}{3.712467in}}%
\pgfpathcurveto{\pgfqpoint{4.272173in}{3.706942in}}{\pgfqpoint{4.274368in}{3.701643in}}{\pgfqpoint{4.278275in}{3.697736in}}%
\pgfpathcurveto{\pgfqpoint{4.282181in}{3.693829in}}{\pgfqpoint{4.287481in}{3.691634in}}{\pgfqpoint{4.293006in}{3.691634in}}%
\pgfpathclose%
\pgfusepath{fill}%
\end{pgfscope}%
\begin{pgfscope}%
\pgfpathrectangle{\pgfqpoint{3.158185in}{3.098185in}}{\pgfqpoint{1.162500in}{0.755000in}} %
\pgfusepath{clip}%
\pgfsetbuttcap%
\pgfsetroundjoin%
\definecolor{currentfill}{rgb}{0.000000,0.000000,0.000000}%
\pgfsetfillcolor{currentfill}%
\pgfsetfillopacity{0.500000}%
\pgfsetlinewidth{0.000000pt}%
\definecolor{currentstroke}{rgb}{0.000000,0.000000,0.000000}%
\pgfsetstrokecolor{currentstroke}%
\pgfsetdash{}{0pt}%
\pgfpathmoveto{\pgfqpoint{3.586831in}{3.134292in}}%
\pgfpathcurveto{\pgfqpoint{3.592356in}{3.134292in}}{\pgfqpoint{3.597655in}{3.136487in}}{\pgfqpoint{3.601562in}{3.140394in}}%
\pgfpathcurveto{\pgfqpoint{3.605469in}{3.144301in}}{\pgfqpoint{3.607664in}{3.149600in}}{\pgfqpoint{3.607664in}{3.155125in}}%
\pgfpathcurveto{\pgfqpoint{3.607664in}{3.160650in}}{\pgfqpoint{3.605469in}{3.165950in}}{\pgfqpoint{3.601562in}{3.169857in}}%
\pgfpathcurveto{\pgfqpoint{3.597655in}{3.173763in}}{\pgfqpoint{3.592356in}{3.175959in}}{\pgfqpoint{3.586831in}{3.175959in}}%
\pgfpathcurveto{\pgfqpoint{3.581306in}{3.175959in}}{\pgfqpoint{3.576006in}{3.173763in}}{\pgfqpoint{3.572099in}{3.169857in}}%
\pgfpathcurveto{\pgfqpoint{3.568193in}{3.165950in}}{\pgfqpoint{3.565998in}{3.160650in}}{\pgfqpoint{3.565998in}{3.155125in}}%
\pgfpathcurveto{\pgfqpoint{3.565998in}{3.149600in}}{\pgfqpoint{3.568193in}{3.144301in}}{\pgfqpoint{3.572099in}{3.140394in}}%
\pgfpathcurveto{\pgfqpoint{3.576006in}{3.136487in}}{\pgfqpoint{3.581306in}{3.134292in}}{\pgfqpoint{3.586831in}{3.134292in}}%
\pgfpathclose%
\pgfusepath{fill}%
\end{pgfscope}%
\begin{pgfscope}%
\pgfpathrectangle{\pgfqpoint{3.158185in}{3.098185in}}{\pgfqpoint{1.162500in}{0.755000in}} %
\pgfusepath{clip}%
\pgfsetbuttcap%
\pgfsetroundjoin%
\definecolor{currentfill}{rgb}{0.000000,0.000000,0.000000}%
\pgfsetfillcolor{currentfill}%
\pgfsetfillopacity{0.500000}%
\pgfsetlinewidth{0.000000pt}%
\definecolor{currentstroke}{rgb}{0.000000,0.000000,0.000000}%
\pgfsetstrokecolor{currentstroke}%
\pgfsetdash{}{0pt}%
\pgfpathmoveto{\pgfqpoint{3.578870in}{3.095327in}}%
\pgfpathcurveto{\pgfqpoint{3.584395in}{3.095327in}}{\pgfqpoint{3.589694in}{3.097523in}}{\pgfqpoint{3.593601in}{3.101429in}}%
\pgfpathcurveto{\pgfqpoint{3.597508in}{3.105336in}}{\pgfqpoint{3.599703in}{3.110636in}}{\pgfqpoint{3.599703in}{3.116161in}}%
\pgfpathcurveto{\pgfqpoint{3.599703in}{3.121686in}}{\pgfqpoint{3.597508in}{3.126985in}}{\pgfqpoint{3.593601in}{3.130892in}}%
\pgfpathcurveto{\pgfqpoint{3.589694in}{3.134799in}}{\pgfqpoint{3.584395in}{3.136994in}}{\pgfqpoint{3.578870in}{3.136994in}}%
\pgfpathcurveto{\pgfqpoint{3.573345in}{3.136994in}}{\pgfqpoint{3.568045in}{3.134799in}}{\pgfqpoint{3.564138in}{3.130892in}}%
\pgfpathcurveto{\pgfqpoint{3.560231in}{3.126985in}}{\pgfqpoint{3.558036in}{3.121686in}}{\pgfqpoint{3.558036in}{3.116161in}}%
\pgfpathcurveto{\pgfqpoint{3.558036in}{3.110636in}}{\pgfqpoint{3.560231in}{3.105336in}}{\pgfqpoint{3.564138in}{3.101429in}}%
\pgfpathcurveto{\pgfqpoint{3.568045in}{3.097523in}}{\pgfqpoint{3.573345in}{3.095327in}}{\pgfqpoint{3.578870in}{3.095327in}}%
\pgfpathclose%
\pgfusepath{fill}%
\end{pgfscope}%
\begin{pgfscope}%
\pgfpathrectangle{\pgfqpoint{3.158185in}{3.098185in}}{\pgfqpoint{1.162500in}{0.755000in}} %
\pgfusepath{clip}%
\pgfsetbuttcap%
\pgfsetroundjoin%
\definecolor{currentfill}{rgb}{0.000000,0.000000,0.000000}%
\pgfsetfillcolor{currentfill}%
\pgfsetfillopacity{0.500000}%
\pgfsetlinewidth{0.000000pt}%
\definecolor{currentstroke}{rgb}{0.000000,0.000000,0.000000}%
\pgfsetstrokecolor{currentstroke}%
\pgfsetdash{}{0pt}%
\pgfpathmoveto{\pgfqpoint{3.333069in}{3.302828in}}%
\pgfpathcurveto{\pgfqpoint{3.338594in}{3.302828in}}{\pgfqpoint{3.343894in}{3.305023in}}{\pgfqpoint{3.347801in}{3.308930in}}%
\pgfpathcurveto{\pgfqpoint{3.351708in}{3.312837in}}{\pgfqpoint{3.353903in}{3.318136in}}{\pgfqpoint{3.353903in}{3.323661in}}%
\pgfpathcurveto{\pgfqpoint{3.353903in}{3.329186in}}{\pgfqpoint{3.351708in}{3.334486in}}{\pgfqpoint{3.347801in}{3.338393in}}%
\pgfpathcurveto{\pgfqpoint{3.343894in}{3.342300in}}{\pgfqpoint{3.338594in}{3.344495in}}{\pgfqpoint{3.333069in}{3.344495in}}%
\pgfpathcurveto{\pgfqpoint{3.327544in}{3.344495in}}{\pgfqpoint{3.322245in}{3.342300in}}{\pgfqpoint{3.318338in}{3.338393in}}%
\pgfpathcurveto{\pgfqpoint{3.314431in}{3.334486in}}{\pgfqpoint{3.312236in}{3.329186in}}{\pgfqpoint{3.312236in}{3.323661in}}%
\pgfpathcurveto{\pgfqpoint{3.312236in}{3.318136in}}{\pgfqpoint{3.314431in}{3.312837in}}{\pgfqpoint{3.318338in}{3.308930in}}%
\pgfpathcurveto{\pgfqpoint{3.322245in}{3.305023in}}{\pgfqpoint{3.327544in}{3.302828in}}{\pgfqpoint{3.333069in}{3.302828in}}%
\pgfpathclose%
\pgfusepath{fill}%
\end{pgfscope}%
\begin{pgfscope}%
\pgfpathrectangle{\pgfqpoint{3.158185in}{3.098185in}}{\pgfqpoint{1.162500in}{0.755000in}} %
\pgfusepath{clip}%
\pgfsetbuttcap%
\pgfsetroundjoin%
\definecolor{currentfill}{rgb}{0.000000,0.000000,0.000000}%
\pgfsetfillcolor{currentfill}%
\pgfsetfillopacity{0.500000}%
\pgfsetlinewidth{0.000000pt}%
\definecolor{currentstroke}{rgb}{0.000000,0.000000,0.000000}%
\pgfsetstrokecolor{currentstroke}%
\pgfsetdash{}{0pt}%
\pgfpathmoveto{\pgfqpoint{3.296013in}{3.145618in}}%
\pgfpathcurveto{\pgfqpoint{3.301538in}{3.145618in}}{\pgfqpoint{3.306838in}{3.147813in}}{\pgfqpoint{3.310744in}{3.151720in}}%
\pgfpathcurveto{\pgfqpoint{3.314651in}{3.155626in}}{\pgfqpoint{3.316846in}{3.160926in}}{\pgfqpoint{3.316846in}{3.166451in}}%
\pgfpathcurveto{\pgfqpoint{3.316846in}{3.171976in}}{\pgfqpoint{3.314651in}{3.177276in}}{\pgfqpoint{3.310744in}{3.181182in}}%
\pgfpathcurveto{\pgfqpoint{3.306838in}{3.185089in}}{\pgfqpoint{3.301538in}{3.187284in}}{\pgfqpoint{3.296013in}{3.187284in}}%
\pgfpathcurveto{\pgfqpoint{3.290488in}{3.187284in}}{\pgfqpoint{3.285188in}{3.185089in}}{\pgfqpoint{3.281282in}{3.181182in}}%
\pgfpathcurveto{\pgfqpoint{3.277375in}{3.177276in}}{\pgfqpoint{3.275180in}{3.171976in}}{\pgfqpoint{3.275180in}{3.166451in}}%
\pgfpathcurveto{\pgfqpoint{3.275180in}{3.160926in}}{\pgfqpoint{3.277375in}{3.155626in}}{\pgfqpoint{3.281282in}{3.151720in}}%
\pgfpathcurveto{\pgfqpoint{3.285188in}{3.147813in}}{\pgfqpoint{3.290488in}{3.145618in}}{\pgfqpoint{3.296013in}{3.145618in}}%
\pgfpathclose%
\pgfusepath{fill}%
\end{pgfscope}%
\begin{pgfscope}%
\pgfpathrectangle{\pgfqpoint{3.158185in}{3.098185in}}{\pgfqpoint{1.162500in}{0.755000in}} %
\pgfusepath{clip}%
\pgfsetbuttcap%
\pgfsetroundjoin%
\definecolor{currentfill}{rgb}{0.000000,0.000000,0.000000}%
\pgfsetfillcolor{currentfill}%
\pgfsetfillopacity{0.500000}%
\pgfsetlinewidth{0.000000pt}%
\definecolor{currentstroke}{rgb}{0.000000,0.000000,0.000000}%
\pgfsetstrokecolor{currentstroke}%
\pgfsetdash{}{0pt}%
\pgfpathmoveto{\pgfqpoint{3.206735in}{3.249924in}}%
\pgfpathcurveto{\pgfqpoint{3.212260in}{3.249924in}}{\pgfqpoint{3.217559in}{3.252119in}}{\pgfqpoint{3.221466in}{3.256026in}}%
\pgfpathcurveto{\pgfqpoint{3.225373in}{3.259932in}}{\pgfqpoint{3.227568in}{3.265232in}}{\pgfqpoint{3.227568in}{3.270757in}}%
\pgfpathcurveto{\pgfqpoint{3.227568in}{3.276282in}}{\pgfqpoint{3.225373in}{3.281582in}}{\pgfqpoint{3.221466in}{3.285488in}}%
\pgfpathcurveto{\pgfqpoint{3.217559in}{3.289395in}}{\pgfqpoint{3.212260in}{3.291590in}}{\pgfqpoint{3.206735in}{3.291590in}}%
\pgfpathcurveto{\pgfqpoint{3.201210in}{3.291590in}}{\pgfqpoint{3.195910in}{3.289395in}}{\pgfqpoint{3.192003in}{3.285488in}}%
\pgfpathcurveto{\pgfqpoint{3.188097in}{3.281582in}}{\pgfqpoint{3.185901in}{3.276282in}}{\pgfqpoint{3.185901in}{3.270757in}}%
\pgfpathcurveto{\pgfqpoint{3.185901in}{3.265232in}}{\pgfqpoint{3.188097in}{3.259932in}}{\pgfqpoint{3.192003in}{3.256026in}}%
\pgfpathcurveto{\pgfqpoint{3.195910in}{3.252119in}}{\pgfqpoint{3.201210in}{3.249924in}}{\pgfqpoint{3.206735in}{3.249924in}}%
\pgfpathclose%
\pgfusepath{fill}%
\end{pgfscope}%
\begin{pgfscope}%
\pgfpathrectangle{\pgfqpoint{3.158185in}{3.098185in}}{\pgfqpoint{1.162500in}{0.755000in}} %
\pgfusepath{clip}%
\pgfsetbuttcap%
\pgfsetroundjoin%
\definecolor{currentfill}{rgb}{0.000000,0.000000,0.000000}%
\pgfsetfillcolor{currentfill}%
\pgfsetfillopacity{0.500000}%
\pgfsetlinewidth{0.000000pt}%
\definecolor{currentstroke}{rgb}{0.000000,0.000000,0.000000}%
\pgfsetstrokecolor{currentstroke}%
\pgfsetdash{}{0pt}%
\pgfpathmoveto{\pgfqpoint{3.185863in}{3.240978in}}%
\pgfpathcurveto{\pgfqpoint{3.191388in}{3.240978in}}{\pgfqpoint{3.196688in}{3.243173in}}{\pgfqpoint{3.200595in}{3.247080in}}%
\pgfpathcurveto{\pgfqpoint{3.204501in}{3.250987in}}{\pgfqpoint{3.206696in}{3.256286in}}{\pgfqpoint{3.206696in}{3.261812in}}%
\pgfpathcurveto{\pgfqpoint{3.206696in}{3.267337in}}{\pgfqpoint{3.204501in}{3.272636in}}{\pgfqpoint{3.200595in}{3.276543in}}%
\pgfpathcurveto{\pgfqpoint{3.196688in}{3.280450in}}{\pgfqpoint{3.191388in}{3.282645in}}{\pgfqpoint{3.185863in}{3.282645in}}%
\pgfpathcurveto{\pgfqpoint{3.180338in}{3.282645in}}{\pgfqpoint{3.175039in}{3.280450in}}{\pgfqpoint{3.171132in}{3.276543in}}%
\pgfpathcurveto{\pgfqpoint{3.167225in}{3.272636in}}{\pgfqpoint{3.165030in}{3.267337in}}{\pgfqpoint{3.165030in}{3.261812in}}%
\pgfpathcurveto{\pgfqpoint{3.165030in}{3.256286in}}{\pgfqpoint{3.167225in}{3.250987in}}{\pgfqpoint{3.171132in}{3.247080in}}%
\pgfpathcurveto{\pgfqpoint{3.175039in}{3.243173in}}{\pgfqpoint{3.180338in}{3.240978in}}{\pgfqpoint{3.185863in}{3.240978in}}%
\pgfpathclose%
\pgfusepath{fill}%
\end{pgfscope}%
\begin{pgfscope}%
\pgfpathrectangle{\pgfqpoint{3.158185in}{3.098185in}}{\pgfqpoint{1.162500in}{0.755000in}} %
\pgfusepath{clip}%
\pgfsetbuttcap%
\pgfsetroundjoin%
\definecolor{currentfill}{rgb}{0.000000,0.000000,0.000000}%
\pgfsetfillcolor{currentfill}%
\pgfsetfillopacity{0.500000}%
\pgfsetlinewidth{0.000000pt}%
\definecolor{currentstroke}{rgb}{0.000000,0.000000,0.000000}%
\pgfsetstrokecolor{currentstroke}%
\pgfsetdash{}{0pt}%
\pgfpathmoveto{\pgfqpoint{3.808562in}{3.768072in}}%
\pgfpathcurveto{\pgfqpoint{3.814087in}{3.768072in}}{\pgfqpoint{3.819386in}{3.770267in}}{\pgfqpoint{3.823293in}{3.774174in}}%
\pgfpathcurveto{\pgfqpoint{3.827200in}{3.778081in}}{\pgfqpoint{3.829395in}{3.783381in}}{\pgfqpoint{3.829395in}{3.788906in}}%
\pgfpathcurveto{\pgfqpoint{3.829395in}{3.794431in}}{\pgfqpoint{3.827200in}{3.799730in}}{\pgfqpoint{3.823293in}{3.803637in}}%
\pgfpathcurveto{\pgfqpoint{3.819386in}{3.807544in}}{\pgfqpoint{3.814087in}{3.809739in}}{\pgfqpoint{3.808562in}{3.809739in}}%
\pgfpathcurveto{\pgfqpoint{3.803037in}{3.809739in}}{\pgfqpoint{3.797737in}{3.807544in}}{\pgfqpoint{3.793830in}{3.803637in}}%
\pgfpathcurveto{\pgfqpoint{3.789924in}{3.799730in}}{\pgfqpoint{3.787728in}{3.794431in}}{\pgfqpoint{3.787728in}{3.788906in}}%
\pgfpathcurveto{\pgfqpoint{3.787728in}{3.783381in}}{\pgfqpoint{3.789924in}{3.778081in}}{\pgfqpoint{3.793830in}{3.774174in}}%
\pgfpathcurveto{\pgfqpoint{3.797737in}{3.770267in}}{\pgfqpoint{3.803037in}{3.768072in}}{\pgfqpoint{3.808562in}{3.768072in}}%
\pgfpathclose%
\pgfusepath{fill}%
\end{pgfscope}%
\begin{pgfscope}%
\pgfpathrectangle{\pgfqpoint{3.158185in}{3.098185in}}{\pgfqpoint{1.162500in}{0.755000in}} %
\pgfusepath{clip}%
\pgfsetbuttcap%
\pgfsetroundjoin%
\definecolor{currentfill}{rgb}{0.000000,0.000000,0.000000}%
\pgfsetfillcolor{currentfill}%
\pgfsetfillopacity{0.500000}%
\pgfsetlinewidth{0.000000pt}%
\definecolor{currentstroke}{rgb}{0.000000,0.000000,0.000000}%
\pgfsetstrokecolor{currentstroke}%
\pgfsetdash{}{0pt}%
\pgfpathmoveto{\pgfqpoint{3.620933in}{3.814375in}}%
\pgfpathcurveto{\pgfqpoint{3.626458in}{3.814375in}}{\pgfqpoint{3.631758in}{3.816570in}}{\pgfqpoint{3.635664in}{3.820477in}}%
\pgfpathcurveto{\pgfqpoint{3.639571in}{3.824384in}}{\pgfqpoint{3.641766in}{3.829683in}}{\pgfqpoint{3.641766in}{3.835208in}}%
\pgfpathcurveto{\pgfqpoint{3.641766in}{3.840733in}}{\pgfqpoint{3.639571in}{3.846033in}}{\pgfqpoint{3.635664in}{3.849940in}}%
\pgfpathcurveto{\pgfqpoint{3.631758in}{3.853847in}}{\pgfqpoint{3.626458in}{3.856042in}}{\pgfqpoint{3.620933in}{3.856042in}}%
\pgfpathcurveto{\pgfqpoint{3.615408in}{3.856042in}}{\pgfqpoint{3.610108in}{3.853847in}}{\pgfqpoint{3.606202in}{3.849940in}}%
\pgfpathcurveto{\pgfqpoint{3.602295in}{3.846033in}}{\pgfqpoint{3.600100in}{3.840733in}}{\pgfqpoint{3.600100in}{3.835208in}}%
\pgfpathcurveto{\pgfqpoint{3.600100in}{3.829683in}}{\pgfqpoint{3.602295in}{3.824384in}}{\pgfqpoint{3.606202in}{3.820477in}}%
\pgfpathcurveto{\pgfqpoint{3.610108in}{3.816570in}}{\pgfqpoint{3.615408in}{3.814375in}}{\pgfqpoint{3.620933in}{3.814375in}}%
\pgfpathclose%
\pgfusepath{fill}%
\end{pgfscope}%
\begin{pgfscope}%
\pgfpathrectangle{\pgfqpoint{3.158185in}{3.098185in}}{\pgfqpoint{1.162500in}{0.755000in}} %
\pgfusepath{clip}%
\pgfsetbuttcap%
\pgfsetroundjoin%
\definecolor{currentfill}{rgb}{0.000000,0.000000,0.000000}%
\pgfsetfillcolor{currentfill}%
\pgfsetfillopacity{0.500000}%
\pgfsetlinewidth{0.000000pt}%
\definecolor{currentstroke}{rgb}{0.000000,0.000000,0.000000}%
\pgfsetstrokecolor{currentstroke}%
\pgfsetdash{}{0pt}%
\pgfpathmoveto{\pgfqpoint{3.600191in}{3.667724in}}%
\pgfpathcurveto{\pgfqpoint{3.605716in}{3.667724in}}{\pgfqpoint{3.611016in}{3.669919in}}{\pgfqpoint{3.614922in}{3.673826in}}%
\pgfpathcurveto{\pgfqpoint{3.618829in}{3.677733in}}{\pgfqpoint{3.621024in}{3.683032in}}{\pgfqpoint{3.621024in}{3.688558in}}%
\pgfpathcurveto{\pgfqpoint{3.621024in}{3.694083in}}{\pgfqpoint{3.618829in}{3.699382in}}{\pgfqpoint{3.614922in}{3.703289in}}%
\pgfpathcurveto{\pgfqpoint{3.611016in}{3.707196in}}{\pgfqpoint{3.605716in}{3.709391in}}{\pgfqpoint{3.600191in}{3.709391in}}%
\pgfpathcurveto{\pgfqpoint{3.594666in}{3.709391in}}{\pgfqpoint{3.589366in}{3.707196in}}{\pgfqpoint{3.585460in}{3.703289in}}%
\pgfpathcurveto{\pgfqpoint{3.581553in}{3.699382in}}{\pgfqpoint{3.579358in}{3.694083in}}{\pgfqpoint{3.579358in}{3.688558in}}%
\pgfpathcurveto{\pgfqpoint{3.579358in}{3.683032in}}{\pgfqpoint{3.581553in}{3.677733in}}{\pgfqpoint{3.585460in}{3.673826in}}%
\pgfpathcurveto{\pgfqpoint{3.589366in}{3.669919in}}{\pgfqpoint{3.594666in}{3.667724in}}{\pgfqpoint{3.600191in}{3.667724in}}%
\pgfpathclose%
\pgfusepath{fill}%
\end{pgfscope}%
\begin{pgfscope}%
\pgfpathrectangle{\pgfqpoint{3.158185in}{3.098185in}}{\pgfqpoint{1.162500in}{0.755000in}} %
\pgfusepath{clip}%
\pgfsetbuttcap%
\pgfsetroundjoin%
\definecolor{currentfill}{rgb}{0.000000,0.000000,0.000000}%
\pgfsetfillcolor{currentfill}%
\pgfsetfillopacity{0.500000}%
\pgfsetlinewidth{0.000000pt}%
\definecolor{currentstroke}{rgb}{0.000000,0.000000,0.000000}%
\pgfsetstrokecolor{currentstroke}%
\pgfsetdash{}{0pt}%
\pgfpathmoveto{\pgfqpoint{3.899960in}{3.223705in}}%
\pgfpathcurveto{\pgfqpoint{3.905485in}{3.223705in}}{\pgfqpoint{3.910785in}{3.225900in}}{\pgfqpoint{3.914692in}{3.229807in}}%
\pgfpathcurveto{\pgfqpoint{3.918598in}{3.233713in}}{\pgfqpoint{3.920794in}{3.239013in}}{\pgfqpoint{3.920794in}{3.244538in}}%
\pgfpathcurveto{\pgfqpoint{3.920794in}{3.250063in}}{\pgfqpoint{3.918598in}{3.255363in}}{\pgfqpoint{3.914692in}{3.259269in}}%
\pgfpathcurveto{\pgfqpoint{3.910785in}{3.263176in}}{\pgfqpoint{3.905485in}{3.265371in}}{\pgfqpoint{3.899960in}{3.265371in}}%
\pgfpathcurveto{\pgfqpoint{3.894435in}{3.265371in}}{\pgfqpoint{3.889136in}{3.263176in}}{\pgfqpoint{3.885229in}{3.259269in}}%
\pgfpathcurveto{\pgfqpoint{3.881322in}{3.255363in}}{\pgfqpoint{3.879127in}{3.250063in}}{\pgfqpoint{3.879127in}{3.244538in}}%
\pgfpathcurveto{\pgfqpoint{3.879127in}{3.239013in}}{\pgfqpoint{3.881322in}{3.233713in}}{\pgfqpoint{3.885229in}{3.229807in}}%
\pgfpathcurveto{\pgfqpoint{3.889136in}{3.225900in}}{\pgfqpoint{3.894435in}{3.223705in}}{\pgfqpoint{3.899960in}{3.223705in}}%
\pgfpathclose%
\pgfusepath{fill}%
\end{pgfscope}%
\begin{pgfscope}%
\pgfpathrectangle{\pgfqpoint{3.158185in}{3.098185in}}{\pgfqpoint{1.162500in}{0.755000in}} %
\pgfusepath{clip}%
\pgfsetbuttcap%
\pgfsetroundjoin%
\definecolor{currentfill}{rgb}{0.000000,0.000000,0.000000}%
\pgfsetfillcolor{currentfill}%
\pgfsetfillopacity{0.500000}%
\pgfsetlinewidth{0.000000pt}%
\definecolor{currentstroke}{rgb}{0.000000,0.000000,0.000000}%
\pgfsetstrokecolor{currentstroke}%
\pgfsetdash{}{0pt}%
\pgfpathmoveto{\pgfqpoint{3.368320in}{3.524055in}}%
\pgfpathcurveto{\pgfqpoint{3.373845in}{3.524055in}}{\pgfqpoint{3.379145in}{3.526250in}}{\pgfqpoint{3.383052in}{3.530157in}}%
\pgfpathcurveto{\pgfqpoint{3.386958in}{3.534064in}}{\pgfqpoint{3.389154in}{3.539363in}}{\pgfqpoint{3.389154in}{3.544888in}}%
\pgfpathcurveto{\pgfqpoint{3.389154in}{3.550413in}}{\pgfqpoint{3.386958in}{3.555713in}}{\pgfqpoint{3.383052in}{3.559620in}}%
\pgfpathcurveto{\pgfqpoint{3.379145in}{3.563526in}}{\pgfqpoint{3.373845in}{3.565721in}}{\pgfqpoint{3.368320in}{3.565721in}}%
\pgfpathcurveto{\pgfqpoint{3.362795in}{3.565721in}}{\pgfqpoint{3.357496in}{3.563526in}}{\pgfqpoint{3.353589in}{3.559620in}}%
\pgfpathcurveto{\pgfqpoint{3.349682in}{3.555713in}}{\pgfqpoint{3.347487in}{3.550413in}}{\pgfqpoint{3.347487in}{3.544888in}}%
\pgfpathcurveto{\pgfqpoint{3.347487in}{3.539363in}}{\pgfqpoint{3.349682in}{3.534064in}}{\pgfqpoint{3.353589in}{3.530157in}}%
\pgfpathcurveto{\pgfqpoint{3.357496in}{3.526250in}}{\pgfqpoint{3.362795in}{3.524055in}}{\pgfqpoint{3.368320in}{3.524055in}}%
\pgfpathclose%
\pgfusepath{fill}%
\end{pgfscope}%
\begin{pgfscope}%
\pgfpathrectangle{\pgfqpoint{3.158185in}{3.098185in}}{\pgfqpoint{1.162500in}{0.755000in}} %
\pgfusepath{clip}%
\pgfsetbuttcap%
\pgfsetroundjoin%
\definecolor{currentfill}{rgb}{0.000000,0.000000,0.000000}%
\pgfsetfillcolor{currentfill}%
\pgfsetfillopacity{0.500000}%
\pgfsetlinewidth{0.000000pt}%
\definecolor{currentstroke}{rgb}{0.000000,0.000000,0.000000}%
\pgfsetstrokecolor{currentstroke}%
\pgfsetdash{}{0pt}%
\pgfpathmoveto{\pgfqpoint{3.416175in}{3.446907in}}%
\pgfpathcurveto{\pgfqpoint{3.421700in}{3.446907in}}{\pgfqpoint{3.427000in}{3.449102in}}{\pgfqpoint{3.430907in}{3.453009in}}%
\pgfpathcurveto{\pgfqpoint{3.434814in}{3.456916in}}{\pgfqpoint{3.437009in}{3.462215in}}{\pgfqpoint{3.437009in}{3.467740in}}%
\pgfpathcurveto{\pgfqpoint{3.437009in}{3.473265in}}{\pgfqpoint{3.434814in}{3.478565in}}{\pgfqpoint{3.430907in}{3.482472in}}%
\pgfpathcurveto{\pgfqpoint{3.427000in}{3.486379in}}{\pgfqpoint{3.421700in}{3.488574in}}{\pgfqpoint{3.416175in}{3.488574in}}%
\pgfpathcurveto{\pgfqpoint{3.410650in}{3.488574in}}{\pgfqpoint{3.405351in}{3.486379in}}{\pgfqpoint{3.401444in}{3.482472in}}%
\pgfpathcurveto{\pgfqpoint{3.397537in}{3.478565in}}{\pgfqpoint{3.395342in}{3.473265in}}{\pgfqpoint{3.395342in}{3.467740in}}%
\pgfpathcurveto{\pgfqpoint{3.395342in}{3.462215in}}{\pgfqpoint{3.397537in}{3.456916in}}{\pgfqpoint{3.401444in}{3.453009in}}%
\pgfpathcurveto{\pgfqpoint{3.405351in}{3.449102in}}{\pgfqpoint{3.410650in}{3.446907in}}{\pgfqpoint{3.416175in}{3.446907in}}%
\pgfpathclose%
\pgfusepath{fill}%
\end{pgfscope}%
\begin{pgfscope}%
\pgfpathrectangle{\pgfqpoint{3.158185in}{3.098185in}}{\pgfqpoint{1.162500in}{0.755000in}} %
\pgfusepath{clip}%
\pgfsetbuttcap%
\pgfsetroundjoin%
\definecolor{currentfill}{rgb}{0.000000,0.000000,0.000000}%
\pgfsetfillcolor{currentfill}%
\pgfsetfillopacity{0.500000}%
\pgfsetlinewidth{0.000000pt}%
\definecolor{currentstroke}{rgb}{0.000000,0.000000,0.000000}%
\pgfsetstrokecolor{currentstroke}%
\pgfsetdash{}{0pt}%
\pgfpathmoveto{\pgfqpoint{3.789669in}{3.220444in}}%
\pgfpathcurveto{\pgfqpoint{3.795194in}{3.220444in}}{\pgfqpoint{3.800493in}{3.222640in}}{\pgfqpoint{3.804400in}{3.226546in}}%
\pgfpathcurveto{\pgfqpoint{3.808307in}{3.230453in}}{\pgfqpoint{3.810502in}{3.235753in}}{\pgfqpoint{3.810502in}{3.241278in}}%
\pgfpathcurveto{\pgfqpoint{3.810502in}{3.246803in}}{\pgfqpoint{3.808307in}{3.252102in}}{\pgfqpoint{3.804400in}{3.256009in}}%
\pgfpathcurveto{\pgfqpoint{3.800493in}{3.259916in}}{\pgfqpoint{3.795194in}{3.262111in}}{\pgfqpoint{3.789669in}{3.262111in}}%
\pgfpathcurveto{\pgfqpoint{3.784143in}{3.262111in}}{\pgfqpoint{3.778844in}{3.259916in}}{\pgfqpoint{3.774937in}{3.256009in}}%
\pgfpathcurveto{\pgfqpoint{3.771030in}{3.252102in}}{\pgfqpoint{3.768835in}{3.246803in}}{\pgfqpoint{3.768835in}{3.241278in}}%
\pgfpathcurveto{\pgfqpoint{3.768835in}{3.235753in}}{\pgfqpoint{3.771030in}{3.230453in}}{\pgfqpoint{3.774937in}{3.226546in}}%
\pgfpathcurveto{\pgfqpoint{3.778844in}{3.222640in}}{\pgfqpoint{3.784143in}{3.220444in}}{\pgfqpoint{3.789669in}{3.220444in}}%
\pgfpathclose%
\pgfusepath{fill}%
\end{pgfscope}%
\begin{pgfscope}%
\pgfpathrectangle{\pgfqpoint{3.158185in}{3.098185in}}{\pgfqpoint{1.162500in}{0.755000in}} %
\pgfusepath{clip}%
\pgfsetbuttcap%
\pgfsetroundjoin%
\definecolor{currentfill}{rgb}{0.000000,0.000000,0.000000}%
\pgfsetfillcolor{currentfill}%
\pgfsetfillopacity{0.500000}%
\pgfsetlinewidth{0.000000pt}%
\definecolor{currentstroke}{rgb}{0.000000,0.000000,0.000000}%
\pgfsetstrokecolor{currentstroke}%
\pgfsetdash{}{0pt}%
\pgfpathmoveto{\pgfqpoint{3.231064in}{3.505459in}}%
\pgfpathcurveto{\pgfqpoint{3.236589in}{3.505459in}}{\pgfqpoint{3.241888in}{3.507654in}}{\pgfqpoint{3.245795in}{3.511561in}}%
\pgfpathcurveto{\pgfqpoint{3.249702in}{3.515468in}}{\pgfqpoint{3.251897in}{3.520768in}}{\pgfqpoint{3.251897in}{3.526293in}}%
\pgfpathcurveto{\pgfqpoint{3.251897in}{3.531818in}}{\pgfqpoint{3.249702in}{3.537117in}}{\pgfqpoint{3.245795in}{3.541024in}}%
\pgfpathcurveto{\pgfqpoint{3.241888in}{3.544931in}}{\pgfqpoint{3.236589in}{3.547126in}}{\pgfqpoint{3.231064in}{3.547126in}}%
\pgfpathcurveto{\pgfqpoint{3.225539in}{3.547126in}}{\pgfqpoint{3.220239in}{3.544931in}}{\pgfqpoint{3.216332in}{3.541024in}}%
\pgfpathcurveto{\pgfqpoint{3.212425in}{3.537117in}}{\pgfqpoint{3.210230in}{3.531818in}}{\pgfqpoint{3.210230in}{3.526293in}}%
\pgfpathcurveto{\pgfqpoint{3.210230in}{3.520768in}}{\pgfqpoint{3.212425in}{3.515468in}}{\pgfqpoint{3.216332in}{3.511561in}}%
\pgfpathcurveto{\pgfqpoint{3.220239in}{3.507654in}}{\pgfqpoint{3.225539in}{3.505459in}}{\pgfqpoint{3.231064in}{3.505459in}}%
\pgfpathclose%
\pgfusepath{fill}%
\end{pgfscope}%
\begin{pgfscope}%
\pgfsetrectcap%
\pgfsetmiterjoin%
\pgfsetlinewidth{0.803000pt}%
\definecolor{currentstroke}{rgb}{0.000000,0.000000,0.000000}%
\pgfsetstrokecolor{currentstroke}%
\pgfsetdash{}{0pt}%
\pgfpathmoveto{\pgfqpoint{3.158185in}{3.098185in}}%
\pgfpathlineto{\pgfqpoint{3.158185in}{3.853185in}}%
\pgfusepath{stroke}%
\end{pgfscope}%
\begin{pgfscope}%
\pgfsetrectcap%
\pgfsetmiterjoin%
\pgfsetlinewidth{0.803000pt}%
\definecolor{currentstroke}{rgb}{0.000000,0.000000,0.000000}%
\pgfsetstrokecolor{currentstroke}%
\pgfsetdash{}{0pt}%
\pgfpathmoveto{\pgfqpoint{4.320685in}{3.098185in}}%
\pgfpathlineto{\pgfqpoint{4.320685in}{3.853185in}}%
\pgfusepath{stroke}%
\end{pgfscope}%
\begin{pgfscope}%
\pgfsetrectcap%
\pgfsetmiterjoin%
\pgfsetlinewidth{0.803000pt}%
\definecolor{currentstroke}{rgb}{0.000000,0.000000,0.000000}%
\pgfsetstrokecolor{currentstroke}%
\pgfsetdash{}{0pt}%
\pgfpathmoveto{\pgfqpoint{3.158185in}{3.098185in}}%
\pgfpathlineto{\pgfqpoint{4.320685in}{3.098185in}}%
\pgfusepath{stroke}%
\end{pgfscope}%
\begin{pgfscope}%
\pgfsetrectcap%
\pgfsetmiterjoin%
\pgfsetlinewidth{0.803000pt}%
\definecolor{currentstroke}{rgb}{0.000000,0.000000,0.000000}%
\pgfsetstrokecolor{currentstroke}%
\pgfsetdash{}{0pt}%
\pgfpathmoveto{\pgfqpoint{3.158185in}{3.853185in}}%
\pgfpathlineto{\pgfqpoint{4.320685in}{3.853185in}}%
\pgfusepath{stroke}%
\end{pgfscope}%
\begin{pgfscope}%
\pgfsetbuttcap%
\pgfsetmiterjoin%
\definecolor{currentfill}{rgb}{1.000000,1.000000,1.000000}%
\pgfsetfillcolor{currentfill}%
\pgfsetlinewidth{0.000000pt}%
\definecolor{currentstroke}{rgb}{0.000000,0.000000,0.000000}%
\pgfsetstrokecolor{currentstroke}%
\pgfsetstrokeopacity{0.000000}%
\pgfsetdash{}{0pt}%
\pgfpathmoveto{\pgfqpoint{4.320685in}{3.098185in}}%
\pgfpathlineto{\pgfqpoint{5.483185in}{3.098185in}}%
\pgfpathlineto{\pgfqpoint{5.483185in}{3.853185in}}%
\pgfpathlineto{\pgfqpoint{4.320685in}{3.853185in}}%
\pgfpathclose%
\pgfusepath{fill}%
\end{pgfscope}%
\begin{pgfscope}%
\pgfpathrectangle{\pgfqpoint{4.320685in}{3.098185in}}{\pgfqpoint{1.162500in}{0.755000in}} %
\pgfusepath{clip}%
\pgfsetbuttcap%
\pgfsetroundjoin%
\definecolor{currentfill}{rgb}{0.000000,0.000000,0.000000}%
\pgfsetfillcolor{currentfill}%
\pgfsetfillopacity{0.500000}%
\pgfsetlinewidth{0.000000pt}%
\definecolor{currentstroke}{rgb}{0.000000,0.000000,0.000000}%
\pgfsetstrokecolor{currentstroke}%
\pgfsetdash{}{0pt}%
\pgfpathmoveto{\pgfqpoint{4.763543in}{3.691634in}}%
\pgfpathcurveto{\pgfqpoint{4.769068in}{3.691634in}}{\pgfqpoint{4.774367in}{3.693829in}}{\pgfqpoint{4.778274in}{3.697736in}}%
\pgfpathcurveto{\pgfqpoint{4.782181in}{3.701643in}}{\pgfqpoint{4.784376in}{3.706942in}}{\pgfqpoint{4.784376in}{3.712467in}}%
\pgfpathcurveto{\pgfqpoint{4.784376in}{3.717993in}}{\pgfqpoint{4.782181in}{3.723292in}}{\pgfqpoint{4.778274in}{3.727199in}}%
\pgfpathcurveto{\pgfqpoint{4.774367in}{3.731106in}}{\pgfqpoint{4.769068in}{3.733301in}}{\pgfqpoint{4.763543in}{3.733301in}}%
\pgfpathcurveto{\pgfqpoint{4.758018in}{3.733301in}}{\pgfqpoint{4.752718in}{3.731106in}}{\pgfqpoint{4.748811in}{3.727199in}}%
\pgfpathcurveto{\pgfqpoint{4.744905in}{3.723292in}}{\pgfqpoint{4.742709in}{3.717993in}}{\pgfqpoint{4.742709in}{3.712467in}}%
\pgfpathcurveto{\pgfqpoint{4.742709in}{3.706942in}}{\pgfqpoint{4.744905in}{3.701643in}}{\pgfqpoint{4.748811in}{3.697736in}}%
\pgfpathcurveto{\pgfqpoint{4.752718in}{3.693829in}}{\pgfqpoint{4.758018in}{3.691634in}}{\pgfqpoint{4.763543in}{3.691634in}}%
\pgfpathclose%
\pgfusepath{fill}%
\end{pgfscope}%
\begin{pgfscope}%
\pgfpathrectangle{\pgfqpoint{4.320685in}{3.098185in}}{\pgfqpoint{1.162500in}{0.755000in}} %
\pgfusepath{clip}%
\pgfsetbuttcap%
\pgfsetroundjoin%
\definecolor{currentfill}{rgb}{0.000000,0.000000,0.000000}%
\pgfsetfillcolor{currentfill}%
\pgfsetfillopacity{0.500000}%
\pgfsetlinewidth{0.000000pt}%
\definecolor{currentstroke}{rgb}{0.000000,0.000000,0.000000}%
\pgfsetstrokecolor{currentstroke}%
\pgfsetdash{}{0pt}%
\pgfpathmoveto{\pgfqpoint{5.385388in}{3.134292in}}%
\pgfpathcurveto{\pgfqpoint{5.390913in}{3.134292in}}{\pgfqpoint{5.396213in}{3.136487in}}{\pgfqpoint{5.400119in}{3.140394in}}%
\pgfpathcurveto{\pgfqpoint{5.404026in}{3.144301in}}{\pgfqpoint{5.406221in}{3.149600in}}{\pgfqpoint{5.406221in}{3.155125in}}%
\pgfpathcurveto{\pgfqpoint{5.406221in}{3.160650in}}{\pgfqpoint{5.404026in}{3.165950in}}{\pgfqpoint{5.400119in}{3.169857in}}%
\pgfpathcurveto{\pgfqpoint{5.396213in}{3.173763in}}{\pgfqpoint{5.390913in}{3.175959in}}{\pgfqpoint{5.385388in}{3.175959in}}%
\pgfpathcurveto{\pgfqpoint{5.379863in}{3.175959in}}{\pgfqpoint{5.374564in}{3.173763in}}{\pgfqpoint{5.370657in}{3.169857in}}%
\pgfpathcurveto{\pgfqpoint{5.366750in}{3.165950in}}{\pgfqpoint{5.364555in}{3.160650in}}{\pgfqpoint{5.364555in}{3.155125in}}%
\pgfpathcurveto{\pgfqpoint{5.364555in}{3.149600in}}{\pgfqpoint{5.366750in}{3.144301in}}{\pgfqpoint{5.370657in}{3.140394in}}%
\pgfpathcurveto{\pgfqpoint{5.374564in}{3.136487in}}{\pgfqpoint{5.379863in}{3.134292in}}{\pgfqpoint{5.385388in}{3.134292in}}%
\pgfpathclose%
\pgfusepath{fill}%
\end{pgfscope}%
\begin{pgfscope}%
\pgfpathrectangle{\pgfqpoint{4.320685in}{3.098185in}}{\pgfqpoint{1.162500in}{0.755000in}} %
\pgfusepath{clip}%
\pgfsetbuttcap%
\pgfsetroundjoin%
\definecolor{currentfill}{rgb}{0.000000,0.000000,0.000000}%
\pgfsetfillcolor{currentfill}%
\pgfsetfillopacity{0.500000}%
\pgfsetlinewidth{0.000000pt}%
\definecolor{currentstroke}{rgb}{0.000000,0.000000,0.000000}%
\pgfsetstrokecolor{currentstroke}%
\pgfsetdash{}{0pt}%
\pgfpathmoveto{\pgfqpoint{5.455506in}{3.095327in}}%
\pgfpathcurveto{\pgfqpoint{5.461031in}{3.095327in}}{\pgfqpoint{5.466331in}{3.097523in}}{\pgfqpoint{5.470237in}{3.101429in}}%
\pgfpathcurveto{\pgfqpoint{5.474144in}{3.105336in}}{\pgfqpoint{5.476339in}{3.110636in}}{\pgfqpoint{5.476339in}{3.116161in}}%
\pgfpathcurveto{\pgfqpoint{5.476339in}{3.121686in}}{\pgfqpoint{5.474144in}{3.126985in}}{\pgfqpoint{5.470237in}{3.130892in}}%
\pgfpathcurveto{\pgfqpoint{5.466331in}{3.134799in}}{\pgfqpoint{5.461031in}{3.136994in}}{\pgfqpoint{5.455506in}{3.136994in}}%
\pgfpathcurveto{\pgfqpoint{5.449981in}{3.136994in}}{\pgfqpoint{5.444681in}{3.134799in}}{\pgfqpoint{5.440775in}{3.130892in}}%
\pgfpathcurveto{\pgfqpoint{5.436868in}{3.126985in}}{\pgfqpoint{5.434673in}{3.121686in}}{\pgfqpoint{5.434673in}{3.116161in}}%
\pgfpathcurveto{\pgfqpoint{5.434673in}{3.110636in}}{\pgfqpoint{5.436868in}{3.105336in}}{\pgfqpoint{5.440775in}{3.101429in}}%
\pgfpathcurveto{\pgfqpoint{5.444681in}{3.097523in}}{\pgfqpoint{5.449981in}{3.095327in}}{\pgfqpoint{5.455506in}{3.095327in}}%
\pgfpathclose%
\pgfusepath{fill}%
\end{pgfscope}%
\begin{pgfscope}%
\pgfpathrectangle{\pgfqpoint{4.320685in}{3.098185in}}{\pgfqpoint{1.162500in}{0.755000in}} %
\pgfusepath{clip}%
\pgfsetbuttcap%
\pgfsetroundjoin%
\definecolor{currentfill}{rgb}{0.000000,0.000000,0.000000}%
\pgfsetfillcolor{currentfill}%
\pgfsetfillopacity{0.500000}%
\pgfsetlinewidth{0.000000pt}%
\definecolor{currentstroke}{rgb}{0.000000,0.000000,0.000000}%
\pgfsetstrokecolor{currentstroke}%
\pgfsetdash{}{0pt}%
\pgfpathmoveto{\pgfqpoint{5.031128in}{3.302828in}}%
\pgfpathcurveto{\pgfqpoint{5.036653in}{3.302828in}}{\pgfqpoint{5.041953in}{3.305023in}}{\pgfqpoint{5.045859in}{3.308930in}}%
\pgfpathcurveto{\pgfqpoint{5.049766in}{3.312837in}}{\pgfqpoint{5.051961in}{3.318136in}}{\pgfqpoint{5.051961in}{3.323661in}}%
\pgfpathcurveto{\pgfqpoint{5.051961in}{3.329186in}}{\pgfqpoint{5.049766in}{3.334486in}}{\pgfqpoint{5.045859in}{3.338393in}}%
\pgfpathcurveto{\pgfqpoint{5.041953in}{3.342300in}}{\pgfqpoint{5.036653in}{3.344495in}}{\pgfqpoint{5.031128in}{3.344495in}}%
\pgfpathcurveto{\pgfqpoint{5.025603in}{3.344495in}}{\pgfqpoint{5.020303in}{3.342300in}}{\pgfqpoint{5.016397in}{3.338393in}}%
\pgfpathcurveto{\pgfqpoint{5.012490in}{3.334486in}}{\pgfqpoint{5.010295in}{3.329186in}}{\pgfqpoint{5.010295in}{3.323661in}}%
\pgfpathcurveto{\pgfqpoint{5.010295in}{3.318136in}}{\pgfqpoint{5.012490in}{3.312837in}}{\pgfqpoint{5.016397in}{3.308930in}}%
\pgfpathcurveto{\pgfqpoint{5.020303in}{3.305023in}}{\pgfqpoint{5.025603in}{3.302828in}}{\pgfqpoint{5.031128in}{3.302828in}}%
\pgfpathclose%
\pgfusepath{fill}%
\end{pgfscope}%
\begin{pgfscope}%
\pgfpathrectangle{\pgfqpoint{4.320685in}{3.098185in}}{\pgfqpoint{1.162500in}{0.755000in}} %
\pgfusepath{clip}%
\pgfsetbuttcap%
\pgfsetroundjoin%
\definecolor{currentfill}{rgb}{0.000000,0.000000,0.000000}%
\pgfsetfillcolor{currentfill}%
\pgfsetfillopacity{0.500000}%
\pgfsetlinewidth{0.000000pt}%
\definecolor{currentstroke}{rgb}{0.000000,0.000000,0.000000}%
\pgfsetstrokecolor{currentstroke}%
\pgfsetdash{}{0pt}%
\pgfpathmoveto{\pgfqpoint{5.102087in}{3.145618in}}%
\pgfpathcurveto{\pgfqpoint{5.107612in}{3.145618in}}{\pgfqpoint{5.112912in}{3.147813in}}{\pgfqpoint{5.116819in}{3.151720in}}%
\pgfpathcurveto{\pgfqpoint{5.120726in}{3.155626in}}{\pgfqpoint{5.122921in}{3.160926in}}{\pgfqpoint{5.122921in}{3.166451in}}%
\pgfpathcurveto{\pgfqpoint{5.122921in}{3.171976in}}{\pgfqpoint{5.120726in}{3.177276in}}{\pgfqpoint{5.116819in}{3.181182in}}%
\pgfpathcurveto{\pgfqpoint{5.112912in}{3.185089in}}{\pgfqpoint{5.107612in}{3.187284in}}{\pgfqpoint{5.102087in}{3.187284in}}%
\pgfpathcurveto{\pgfqpoint{5.096562in}{3.187284in}}{\pgfqpoint{5.091263in}{3.185089in}}{\pgfqpoint{5.087356in}{3.181182in}}%
\pgfpathcurveto{\pgfqpoint{5.083449in}{3.177276in}}{\pgfqpoint{5.081254in}{3.171976in}}{\pgfqpoint{5.081254in}{3.166451in}}%
\pgfpathcurveto{\pgfqpoint{5.081254in}{3.160926in}}{\pgfqpoint{5.083449in}{3.155626in}}{\pgfqpoint{5.087356in}{3.151720in}}%
\pgfpathcurveto{\pgfqpoint{5.091263in}{3.147813in}}{\pgfqpoint{5.096562in}{3.145618in}}{\pgfqpoint{5.102087in}{3.145618in}}%
\pgfpathclose%
\pgfusepath{fill}%
\end{pgfscope}%
\begin{pgfscope}%
\pgfpathrectangle{\pgfqpoint{4.320685in}{3.098185in}}{\pgfqpoint{1.162500in}{0.755000in}} %
\pgfusepath{clip}%
\pgfsetbuttcap%
\pgfsetroundjoin%
\definecolor{currentfill}{rgb}{0.000000,0.000000,0.000000}%
\pgfsetfillcolor{currentfill}%
\pgfsetfillopacity{0.500000}%
\pgfsetlinewidth{0.000000pt}%
\definecolor{currentstroke}{rgb}{0.000000,0.000000,0.000000}%
\pgfsetstrokecolor{currentstroke}%
\pgfsetdash{}{0pt}%
\pgfpathmoveto{\pgfqpoint{4.940844in}{3.249924in}}%
\pgfpathcurveto{\pgfqpoint{4.946369in}{3.249924in}}{\pgfqpoint{4.951669in}{3.252119in}}{\pgfqpoint{4.955576in}{3.256026in}}%
\pgfpathcurveto{\pgfqpoint{4.959482in}{3.259932in}}{\pgfqpoint{4.961678in}{3.265232in}}{\pgfqpoint{4.961678in}{3.270757in}}%
\pgfpathcurveto{\pgfqpoint{4.961678in}{3.276282in}}{\pgfqpoint{4.959482in}{3.281582in}}{\pgfqpoint{4.955576in}{3.285488in}}%
\pgfpathcurveto{\pgfqpoint{4.951669in}{3.289395in}}{\pgfqpoint{4.946369in}{3.291590in}}{\pgfqpoint{4.940844in}{3.291590in}}%
\pgfpathcurveto{\pgfqpoint{4.935319in}{3.291590in}}{\pgfqpoint{4.930020in}{3.289395in}}{\pgfqpoint{4.926113in}{3.285488in}}%
\pgfpathcurveto{\pgfqpoint{4.922206in}{3.281582in}}{\pgfqpoint{4.920011in}{3.276282in}}{\pgfqpoint{4.920011in}{3.270757in}}%
\pgfpathcurveto{\pgfqpoint{4.920011in}{3.265232in}}{\pgfqpoint{4.922206in}{3.259932in}}{\pgfqpoint{4.926113in}{3.256026in}}%
\pgfpathcurveto{\pgfqpoint{4.930020in}{3.252119in}}{\pgfqpoint{4.935319in}{3.249924in}}{\pgfqpoint{4.940844in}{3.249924in}}%
\pgfpathclose%
\pgfusepath{fill}%
\end{pgfscope}%
\begin{pgfscope}%
\pgfpathrectangle{\pgfqpoint{4.320685in}{3.098185in}}{\pgfqpoint{1.162500in}{0.755000in}} %
\pgfusepath{clip}%
\pgfsetbuttcap%
\pgfsetroundjoin%
\definecolor{currentfill}{rgb}{0.000000,0.000000,0.000000}%
\pgfsetfillcolor{currentfill}%
\pgfsetfillopacity{0.500000}%
\pgfsetlinewidth{0.000000pt}%
\definecolor{currentstroke}{rgb}{0.000000,0.000000,0.000000}%
\pgfsetstrokecolor{currentstroke}%
\pgfsetdash{}{0pt}%
\pgfpathmoveto{\pgfqpoint{4.348363in}{3.240978in}}%
\pgfpathcurveto{\pgfqpoint{4.353888in}{3.240978in}}{\pgfqpoint{4.359188in}{3.243173in}}{\pgfqpoint{4.363095in}{3.247080in}}%
\pgfpathcurveto{\pgfqpoint{4.367001in}{3.250987in}}{\pgfqpoint{4.369196in}{3.256286in}}{\pgfqpoint{4.369196in}{3.261812in}}%
\pgfpathcurveto{\pgfqpoint{4.369196in}{3.267337in}}{\pgfqpoint{4.367001in}{3.272636in}}{\pgfqpoint{4.363095in}{3.276543in}}%
\pgfpathcurveto{\pgfqpoint{4.359188in}{3.280450in}}{\pgfqpoint{4.353888in}{3.282645in}}{\pgfqpoint{4.348363in}{3.282645in}}%
\pgfpathcurveto{\pgfqpoint{4.342838in}{3.282645in}}{\pgfqpoint{4.337539in}{3.280450in}}{\pgfqpoint{4.333632in}{3.276543in}}%
\pgfpathcurveto{\pgfqpoint{4.329725in}{3.272636in}}{\pgfqpoint{4.327530in}{3.267337in}}{\pgfqpoint{4.327530in}{3.261812in}}%
\pgfpathcurveto{\pgfqpoint{4.327530in}{3.256286in}}{\pgfqpoint{4.329725in}{3.250987in}}{\pgfqpoint{4.333632in}{3.247080in}}%
\pgfpathcurveto{\pgfqpoint{4.337539in}{3.243173in}}{\pgfqpoint{4.342838in}{3.240978in}}{\pgfqpoint{4.348363in}{3.240978in}}%
\pgfpathclose%
\pgfusepath{fill}%
\end{pgfscope}%
\begin{pgfscope}%
\pgfpathrectangle{\pgfqpoint{4.320685in}{3.098185in}}{\pgfqpoint{1.162500in}{0.755000in}} %
\pgfusepath{clip}%
\pgfsetbuttcap%
\pgfsetroundjoin%
\definecolor{currentfill}{rgb}{0.000000,0.000000,0.000000}%
\pgfsetfillcolor{currentfill}%
\pgfsetfillopacity{0.500000}%
\pgfsetlinewidth{0.000000pt}%
\definecolor{currentstroke}{rgb}{0.000000,0.000000,0.000000}%
\pgfsetstrokecolor{currentstroke}%
\pgfsetdash{}{0pt}%
\pgfpathmoveto{\pgfqpoint{5.332823in}{3.768072in}}%
\pgfpathcurveto{\pgfqpoint{5.338348in}{3.768072in}}{\pgfqpoint{5.343648in}{3.770267in}}{\pgfqpoint{5.347555in}{3.774174in}}%
\pgfpathcurveto{\pgfqpoint{5.351461in}{3.778081in}}{\pgfqpoint{5.353656in}{3.783381in}}{\pgfqpoint{5.353656in}{3.788906in}}%
\pgfpathcurveto{\pgfqpoint{5.353656in}{3.794431in}}{\pgfqpoint{5.351461in}{3.799730in}}{\pgfqpoint{5.347555in}{3.803637in}}%
\pgfpathcurveto{\pgfqpoint{5.343648in}{3.807544in}}{\pgfqpoint{5.338348in}{3.809739in}}{\pgfqpoint{5.332823in}{3.809739in}}%
\pgfpathcurveto{\pgfqpoint{5.327298in}{3.809739in}}{\pgfqpoint{5.321999in}{3.807544in}}{\pgfqpoint{5.318092in}{3.803637in}}%
\pgfpathcurveto{\pgfqpoint{5.314185in}{3.799730in}}{\pgfqpoint{5.311990in}{3.794431in}}{\pgfqpoint{5.311990in}{3.788906in}}%
\pgfpathcurveto{\pgfqpoint{5.311990in}{3.783381in}}{\pgfqpoint{5.314185in}{3.778081in}}{\pgfqpoint{5.318092in}{3.774174in}}%
\pgfpathcurveto{\pgfqpoint{5.321999in}{3.770267in}}{\pgfqpoint{5.327298in}{3.768072in}}{\pgfqpoint{5.332823in}{3.768072in}}%
\pgfpathclose%
\pgfusepath{fill}%
\end{pgfscope}%
\begin{pgfscope}%
\pgfpathrectangle{\pgfqpoint{4.320685in}{3.098185in}}{\pgfqpoint{1.162500in}{0.755000in}} %
\pgfusepath{clip}%
\pgfsetbuttcap%
\pgfsetroundjoin%
\definecolor{currentfill}{rgb}{0.000000,0.000000,0.000000}%
\pgfsetfillcolor{currentfill}%
\pgfsetfillopacity{0.500000}%
\pgfsetlinewidth{0.000000pt}%
\definecolor{currentstroke}{rgb}{0.000000,0.000000,0.000000}%
\pgfsetstrokecolor{currentstroke}%
\pgfsetdash{}{0pt}%
\pgfpathmoveto{\pgfqpoint{5.090412in}{3.814375in}}%
\pgfpathcurveto{\pgfqpoint{5.095937in}{3.814375in}}{\pgfqpoint{5.101237in}{3.816570in}}{\pgfqpoint{5.105143in}{3.820477in}}%
\pgfpathcurveto{\pgfqpoint{5.109050in}{3.824384in}}{\pgfqpoint{5.111245in}{3.829683in}}{\pgfqpoint{5.111245in}{3.835208in}}%
\pgfpathcurveto{\pgfqpoint{5.111245in}{3.840733in}}{\pgfqpoint{5.109050in}{3.846033in}}{\pgfqpoint{5.105143in}{3.849940in}}%
\pgfpathcurveto{\pgfqpoint{5.101237in}{3.853847in}}{\pgfqpoint{5.095937in}{3.856042in}}{\pgfqpoint{5.090412in}{3.856042in}}%
\pgfpathcurveto{\pgfqpoint{5.084887in}{3.856042in}}{\pgfqpoint{5.079587in}{3.853847in}}{\pgfqpoint{5.075681in}{3.849940in}}%
\pgfpathcurveto{\pgfqpoint{5.071774in}{3.846033in}}{\pgfqpoint{5.069579in}{3.840733in}}{\pgfqpoint{5.069579in}{3.835208in}}%
\pgfpathcurveto{\pgfqpoint{5.069579in}{3.829683in}}{\pgfqpoint{5.071774in}{3.824384in}}{\pgfqpoint{5.075681in}{3.820477in}}%
\pgfpathcurveto{\pgfqpoint{5.079587in}{3.816570in}}{\pgfqpoint{5.084887in}{3.814375in}}{\pgfqpoint{5.090412in}{3.814375in}}%
\pgfpathclose%
\pgfusepath{fill}%
\end{pgfscope}%
\begin{pgfscope}%
\pgfpathrectangle{\pgfqpoint{4.320685in}{3.098185in}}{\pgfqpoint{1.162500in}{0.755000in}} %
\pgfusepath{clip}%
\pgfsetbuttcap%
\pgfsetroundjoin%
\definecolor{currentfill}{rgb}{0.000000,0.000000,0.000000}%
\pgfsetfillcolor{currentfill}%
\pgfsetfillopacity{0.500000}%
\pgfsetlinewidth{0.000000pt}%
\definecolor{currentstroke}{rgb}{0.000000,0.000000,0.000000}%
\pgfsetstrokecolor{currentstroke}%
\pgfsetdash{}{0pt}%
\pgfpathmoveto{\pgfqpoint{4.817426in}{3.667724in}}%
\pgfpathcurveto{\pgfqpoint{4.822951in}{3.667724in}}{\pgfqpoint{4.828251in}{3.669919in}}{\pgfqpoint{4.832158in}{3.673826in}}%
\pgfpathcurveto{\pgfqpoint{4.836065in}{3.677733in}}{\pgfqpoint{4.838260in}{3.683032in}}{\pgfqpoint{4.838260in}{3.688558in}}%
\pgfpathcurveto{\pgfqpoint{4.838260in}{3.694083in}}{\pgfqpoint{4.836065in}{3.699382in}}{\pgfqpoint{4.832158in}{3.703289in}}%
\pgfpathcurveto{\pgfqpoint{4.828251in}{3.707196in}}{\pgfqpoint{4.822951in}{3.709391in}}{\pgfqpoint{4.817426in}{3.709391in}}%
\pgfpathcurveto{\pgfqpoint{4.811901in}{3.709391in}}{\pgfqpoint{4.806602in}{3.707196in}}{\pgfqpoint{4.802695in}{3.703289in}}%
\pgfpathcurveto{\pgfqpoint{4.798788in}{3.699382in}}{\pgfqpoint{4.796593in}{3.694083in}}{\pgfqpoint{4.796593in}{3.688558in}}%
\pgfpathcurveto{\pgfqpoint{4.796593in}{3.683032in}}{\pgfqpoint{4.798788in}{3.677733in}}{\pgfqpoint{4.802695in}{3.673826in}}%
\pgfpathcurveto{\pgfqpoint{4.806602in}{3.669919in}}{\pgfqpoint{4.811901in}{3.667724in}}{\pgfqpoint{4.817426in}{3.667724in}}%
\pgfpathclose%
\pgfusepath{fill}%
\end{pgfscope}%
\begin{pgfscope}%
\pgfpathrectangle{\pgfqpoint{4.320685in}{3.098185in}}{\pgfqpoint{1.162500in}{0.755000in}} %
\pgfusepath{clip}%
\pgfsetbuttcap%
\pgfsetroundjoin%
\definecolor{currentfill}{rgb}{0.000000,0.000000,0.000000}%
\pgfsetfillcolor{currentfill}%
\pgfsetfillopacity{0.500000}%
\pgfsetlinewidth{0.000000pt}%
\definecolor{currentstroke}{rgb}{0.000000,0.000000,0.000000}%
\pgfsetstrokecolor{currentstroke}%
\pgfsetdash{}{0pt}%
\pgfpathmoveto{\pgfqpoint{5.258215in}{3.223705in}}%
\pgfpathcurveto{\pgfqpoint{5.263740in}{3.223705in}}{\pgfqpoint{5.269040in}{3.225900in}}{\pgfqpoint{5.272947in}{3.229807in}}%
\pgfpathcurveto{\pgfqpoint{5.276853in}{3.233713in}}{\pgfqpoint{5.279048in}{3.239013in}}{\pgfqpoint{5.279048in}{3.244538in}}%
\pgfpathcurveto{\pgfqpoint{5.279048in}{3.250063in}}{\pgfqpoint{5.276853in}{3.255363in}}{\pgfqpoint{5.272947in}{3.259269in}}%
\pgfpathcurveto{\pgfqpoint{5.269040in}{3.263176in}}{\pgfqpoint{5.263740in}{3.265371in}}{\pgfqpoint{5.258215in}{3.265371in}}%
\pgfpathcurveto{\pgfqpoint{5.252690in}{3.265371in}}{\pgfqpoint{5.247391in}{3.263176in}}{\pgfqpoint{5.243484in}{3.259269in}}%
\pgfpathcurveto{\pgfqpoint{5.239577in}{3.255363in}}{\pgfqpoint{5.237382in}{3.250063in}}{\pgfqpoint{5.237382in}{3.244538in}}%
\pgfpathcurveto{\pgfqpoint{5.237382in}{3.239013in}}{\pgfqpoint{5.239577in}{3.233713in}}{\pgfqpoint{5.243484in}{3.229807in}}%
\pgfpathcurveto{\pgfqpoint{5.247391in}{3.225900in}}{\pgfqpoint{5.252690in}{3.223705in}}{\pgfqpoint{5.258215in}{3.223705in}}%
\pgfpathclose%
\pgfusepath{fill}%
\end{pgfscope}%
\begin{pgfscope}%
\pgfpathrectangle{\pgfqpoint{4.320685in}{3.098185in}}{\pgfqpoint{1.162500in}{0.755000in}} %
\pgfusepath{clip}%
\pgfsetbuttcap%
\pgfsetroundjoin%
\definecolor{currentfill}{rgb}{0.000000,0.000000,0.000000}%
\pgfsetfillcolor{currentfill}%
\pgfsetfillopacity{0.500000}%
\pgfsetlinewidth{0.000000pt}%
\definecolor{currentstroke}{rgb}{0.000000,0.000000,0.000000}%
\pgfsetstrokecolor{currentstroke}%
\pgfsetdash{}{0pt}%
\pgfpathmoveto{\pgfqpoint{4.803827in}{3.524055in}}%
\pgfpathcurveto{\pgfqpoint{4.809352in}{3.524055in}}{\pgfqpoint{4.814651in}{3.526250in}}{\pgfqpoint{4.818558in}{3.530157in}}%
\pgfpathcurveto{\pgfqpoint{4.822465in}{3.534064in}}{\pgfqpoint{4.824660in}{3.539363in}}{\pgfqpoint{4.824660in}{3.544888in}}%
\pgfpathcurveto{\pgfqpoint{4.824660in}{3.550413in}}{\pgfqpoint{4.822465in}{3.555713in}}{\pgfqpoint{4.818558in}{3.559620in}}%
\pgfpathcurveto{\pgfqpoint{4.814651in}{3.563526in}}{\pgfqpoint{4.809352in}{3.565721in}}{\pgfqpoint{4.803827in}{3.565721in}}%
\pgfpathcurveto{\pgfqpoint{4.798302in}{3.565721in}}{\pgfqpoint{4.793002in}{3.563526in}}{\pgfqpoint{4.789096in}{3.559620in}}%
\pgfpathcurveto{\pgfqpoint{4.785189in}{3.555713in}}{\pgfqpoint{4.782994in}{3.550413in}}{\pgfqpoint{4.782994in}{3.544888in}}%
\pgfpathcurveto{\pgfqpoint{4.782994in}{3.539363in}}{\pgfqpoint{4.785189in}{3.534064in}}{\pgfqpoint{4.789096in}{3.530157in}}%
\pgfpathcurveto{\pgfqpoint{4.793002in}{3.526250in}}{\pgfqpoint{4.798302in}{3.524055in}}{\pgfqpoint{4.803827in}{3.524055in}}%
\pgfpathclose%
\pgfusepath{fill}%
\end{pgfscope}%
\begin{pgfscope}%
\pgfpathrectangle{\pgfqpoint{4.320685in}{3.098185in}}{\pgfqpoint{1.162500in}{0.755000in}} %
\pgfusepath{clip}%
\pgfsetbuttcap%
\pgfsetroundjoin%
\definecolor{currentfill}{rgb}{0.000000,0.000000,0.000000}%
\pgfsetfillcolor{currentfill}%
\pgfsetfillopacity{0.500000}%
\pgfsetlinewidth{0.000000pt}%
\definecolor{currentstroke}{rgb}{0.000000,0.000000,0.000000}%
\pgfsetstrokecolor{currentstroke}%
\pgfsetdash{}{0pt}%
\pgfpathmoveto{\pgfqpoint{4.437194in}{3.446907in}}%
\pgfpathcurveto{\pgfqpoint{4.442719in}{3.446907in}}{\pgfqpoint{4.448018in}{3.449102in}}{\pgfqpoint{4.451925in}{3.453009in}}%
\pgfpathcurveto{\pgfqpoint{4.455832in}{3.456916in}}{\pgfqpoint{4.458027in}{3.462215in}}{\pgfqpoint{4.458027in}{3.467740in}}%
\pgfpathcurveto{\pgfqpoint{4.458027in}{3.473265in}}{\pgfqpoint{4.455832in}{3.478565in}}{\pgfqpoint{4.451925in}{3.482472in}}%
\pgfpathcurveto{\pgfqpoint{4.448018in}{3.486379in}}{\pgfqpoint{4.442719in}{3.488574in}}{\pgfqpoint{4.437194in}{3.488574in}}%
\pgfpathcurveto{\pgfqpoint{4.431669in}{3.488574in}}{\pgfqpoint{4.426369in}{3.486379in}}{\pgfqpoint{4.422462in}{3.482472in}}%
\pgfpathcurveto{\pgfqpoint{4.418556in}{3.478565in}}{\pgfqpoint{4.416360in}{3.473265in}}{\pgfqpoint{4.416360in}{3.467740in}}%
\pgfpathcurveto{\pgfqpoint{4.416360in}{3.462215in}}{\pgfqpoint{4.418556in}{3.456916in}}{\pgfqpoint{4.422462in}{3.453009in}}%
\pgfpathcurveto{\pgfqpoint{4.426369in}{3.449102in}}{\pgfqpoint{4.431669in}{3.446907in}}{\pgfqpoint{4.437194in}{3.446907in}}%
\pgfpathclose%
\pgfusepath{fill}%
\end{pgfscope}%
\begin{pgfscope}%
\pgfpathrectangle{\pgfqpoint{4.320685in}{3.098185in}}{\pgfqpoint{1.162500in}{0.755000in}} %
\pgfusepath{clip}%
\pgfsetbuttcap%
\pgfsetroundjoin%
\definecolor{currentfill}{rgb}{0.000000,0.000000,0.000000}%
\pgfsetfillcolor{currentfill}%
\pgfsetfillopacity{0.500000}%
\pgfsetlinewidth{0.000000pt}%
\definecolor{currentstroke}{rgb}{0.000000,0.000000,0.000000}%
\pgfsetstrokecolor{currentstroke}%
\pgfsetdash{}{0pt}%
\pgfpathmoveto{\pgfqpoint{5.262967in}{3.220444in}}%
\pgfpathcurveto{\pgfqpoint{5.268492in}{3.220444in}}{\pgfqpoint{5.273791in}{3.222640in}}{\pgfqpoint{5.277698in}{3.226546in}}%
\pgfpathcurveto{\pgfqpoint{5.281605in}{3.230453in}}{\pgfqpoint{5.283800in}{3.235753in}}{\pgfqpoint{5.283800in}{3.241278in}}%
\pgfpathcurveto{\pgfqpoint{5.283800in}{3.246803in}}{\pgfqpoint{5.281605in}{3.252102in}}{\pgfqpoint{5.277698in}{3.256009in}}%
\pgfpathcurveto{\pgfqpoint{5.273791in}{3.259916in}}{\pgfqpoint{5.268492in}{3.262111in}}{\pgfqpoint{5.262967in}{3.262111in}}%
\pgfpathcurveto{\pgfqpoint{5.257442in}{3.262111in}}{\pgfqpoint{5.252142in}{3.259916in}}{\pgfqpoint{5.248235in}{3.256009in}}%
\pgfpathcurveto{\pgfqpoint{5.244329in}{3.252102in}}{\pgfqpoint{5.242134in}{3.246803in}}{\pgfqpoint{5.242134in}{3.241278in}}%
\pgfpathcurveto{\pgfqpoint{5.242134in}{3.235753in}}{\pgfqpoint{5.244329in}{3.230453in}}{\pgfqpoint{5.248235in}{3.226546in}}%
\pgfpathcurveto{\pgfqpoint{5.252142in}{3.222640in}}{\pgfqpoint{5.257442in}{3.220444in}}{\pgfqpoint{5.262967in}{3.220444in}}%
\pgfpathclose%
\pgfusepath{fill}%
\end{pgfscope}%
\begin{pgfscope}%
\pgfpathrectangle{\pgfqpoint{4.320685in}{3.098185in}}{\pgfqpoint{1.162500in}{0.755000in}} %
\pgfusepath{clip}%
\pgfsetbuttcap%
\pgfsetroundjoin%
\definecolor{currentfill}{rgb}{0.000000,0.000000,0.000000}%
\pgfsetfillcolor{currentfill}%
\pgfsetfillopacity{0.500000}%
\pgfsetlinewidth{0.000000pt}%
\definecolor{currentstroke}{rgb}{0.000000,0.000000,0.000000}%
\pgfsetstrokecolor{currentstroke}%
\pgfsetdash{}{0pt}%
\pgfpathmoveto{\pgfqpoint{4.896922in}{3.505459in}}%
\pgfpathcurveto{\pgfqpoint{4.902447in}{3.505459in}}{\pgfqpoint{4.907746in}{3.507654in}}{\pgfqpoint{4.911653in}{3.511561in}}%
\pgfpathcurveto{\pgfqpoint{4.915560in}{3.515468in}}{\pgfqpoint{4.917755in}{3.520768in}}{\pgfqpoint{4.917755in}{3.526293in}}%
\pgfpathcurveto{\pgfqpoint{4.917755in}{3.531818in}}{\pgfqpoint{4.915560in}{3.537117in}}{\pgfqpoint{4.911653in}{3.541024in}}%
\pgfpathcurveto{\pgfqpoint{4.907746in}{3.544931in}}{\pgfqpoint{4.902447in}{3.547126in}}{\pgfqpoint{4.896922in}{3.547126in}}%
\pgfpathcurveto{\pgfqpoint{4.891397in}{3.547126in}}{\pgfqpoint{4.886097in}{3.544931in}}{\pgfqpoint{4.882190in}{3.541024in}}%
\pgfpathcurveto{\pgfqpoint{4.878284in}{3.537117in}}{\pgfqpoint{4.876088in}{3.531818in}}{\pgfqpoint{4.876088in}{3.526293in}}%
\pgfpathcurveto{\pgfqpoint{4.876088in}{3.520768in}}{\pgfqpoint{4.878284in}{3.515468in}}{\pgfqpoint{4.882190in}{3.511561in}}%
\pgfpathcurveto{\pgfqpoint{4.886097in}{3.507654in}}{\pgfqpoint{4.891397in}{3.505459in}}{\pgfqpoint{4.896922in}{3.505459in}}%
\pgfpathclose%
\pgfusepath{fill}%
\end{pgfscope}%
\begin{pgfscope}%
\pgfsetrectcap%
\pgfsetmiterjoin%
\pgfsetlinewidth{0.803000pt}%
\definecolor{currentstroke}{rgb}{0.000000,0.000000,0.000000}%
\pgfsetstrokecolor{currentstroke}%
\pgfsetdash{}{0pt}%
\pgfpathmoveto{\pgfqpoint{4.320685in}{3.098185in}}%
\pgfpathlineto{\pgfqpoint{4.320685in}{3.853185in}}%
\pgfusepath{stroke}%
\end{pgfscope}%
\begin{pgfscope}%
\pgfsetrectcap%
\pgfsetmiterjoin%
\pgfsetlinewidth{0.803000pt}%
\definecolor{currentstroke}{rgb}{0.000000,0.000000,0.000000}%
\pgfsetstrokecolor{currentstroke}%
\pgfsetdash{}{0pt}%
\pgfpathmoveto{\pgfqpoint{5.483185in}{3.098185in}}%
\pgfpathlineto{\pgfqpoint{5.483185in}{3.853185in}}%
\pgfusepath{stroke}%
\end{pgfscope}%
\begin{pgfscope}%
\pgfsetrectcap%
\pgfsetmiterjoin%
\pgfsetlinewidth{0.803000pt}%
\definecolor{currentstroke}{rgb}{0.000000,0.000000,0.000000}%
\pgfsetstrokecolor{currentstroke}%
\pgfsetdash{}{0pt}%
\pgfpathmoveto{\pgfqpoint{4.320685in}{3.098185in}}%
\pgfpathlineto{\pgfqpoint{5.483185in}{3.098185in}}%
\pgfusepath{stroke}%
\end{pgfscope}%
\begin{pgfscope}%
\pgfsetrectcap%
\pgfsetmiterjoin%
\pgfsetlinewidth{0.803000pt}%
\definecolor{currentstroke}{rgb}{0.000000,0.000000,0.000000}%
\pgfsetstrokecolor{currentstroke}%
\pgfsetdash{}{0pt}%
\pgfpathmoveto{\pgfqpoint{4.320685in}{3.853185in}}%
\pgfpathlineto{\pgfqpoint{5.483185in}{3.853185in}}%
\pgfusepath{stroke}%
\end{pgfscope}%
\begin{pgfscope}%
\pgfsetbuttcap%
\pgfsetmiterjoin%
\definecolor{currentfill}{rgb}{1.000000,1.000000,1.000000}%
\pgfsetfillcolor{currentfill}%
\pgfsetlinewidth{0.000000pt}%
\definecolor{currentstroke}{rgb}{0.000000,0.000000,0.000000}%
\pgfsetstrokecolor{currentstroke}%
\pgfsetstrokeopacity{0.000000}%
\pgfsetdash{}{0pt}%
\pgfpathmoveto{\pgfqpoint{0.833185in}{2.343185in}}%
\pgfpathlineto{\pgfqpoint{1.995685in}{2.343185in}}%
\pgfpathlineto{\pgfqpoint{1.995685in}{3.098185in}}%
\pgfpathlineto{\pgfqpoint{0.833185in}{3.098185in}}%
\pgfpathclose%
\pgfusepath{fill}%
\end{pgfscope}%
\begin{pgfscope}%
\pgfpathrectangle{\pgfqpoint{0.833185in}{2.343185in}}{\pgfqpoint{1.162500in}{0.755000in}} %
\pgfusepath{clip}%
\pgfsetbuttcap%
\pgfsetroundjoin%
\definecolor{currentfill}{rgb}{0.000000,0.000000,0.000000}%
\pgfsetfillcolor{currentfill}%
\pgfsetfillopacity{0.500000}%
\pgfsetlinewidth{0.000000pt}%
\definecolor{currentstroke}{rgb}{0.000000,0.000000,0.000000}%
\pgfsetstrokecolor{currentstroke}%
\pgfsetdash{}{0pt}%
\pgfpathmoveto{\pgfqpoint{1.779018in}{3.059375in}}%
\pgfpathcurveto{\pgfqpoint{1.784543in}{3.059375in}}{\pgfqpoint{1.789842in}{3.061570in}}{\pgfqpoint{1.793749in}{3.065477in}}%
\pgfpathcurveto{\pgfqpoint{1.797656in}{3.069384in}}{\pgfqpoint{1.799851in}{3.074683in}}{\pgfqpoint{1.799851in}{3.080208in}}%
\pgfpathcurveto{\pgfqpoint{1.799851in}{3.085733in}}{\pgfqpoint{1.797656in}{3.091033in}}{\pgfqpoint{1.793749in}{3.094940in}}%
\pgfpathcurveto{\pgfqpoint{1.789842in}{3.098847in}}{\pgfqpoint{1.784543in}{3.101042in}}{\pgfqpoint{1.779018in}{3.101042in}}%
\pgfpathcurveto{\pgfqpoint{1.773492in}{3.101042in}}{\pgfqpoint{1.768193in}{3.098847in}}{\pgfqpoint{1.764286in}{3.094940in}}%
\pgfpathcurveto{\pgfqpoint{1.760379in}{3.091033in}}{\pgfqpoint{1.758184in}{3.085733in}}{\pgfqpoint{1.758184in}{3.080208in}}%
\pgfpathcurveto{\pgfqpoint{1.758184in}{3.074683in}}{\pgfqpoint{1.760379in}{3.069384in}}{\pgfqpoint{1.764286in}{3.065477in}}%
\pgfpathcurveto{\pgfqpoint{1.768193in}{3.061570in}}{\pgfqpoint{1.773492in}{3.059375in}}{\pgfqpoint{1.779018in}{3.059375in}}%
\pgfpathclose%
\pgfusepath{fill}%
\end{pgfscope}%
\begin{pgfscope}%
\pgfpathrectangle{\pgfqpoint{0.833185in}{2.343185in}}{\pgfqpoint{1.162500in}{0.755000in}} %
\pgfusepath{clip}%
\pgfsetbuttcap%
\pgfsetroundjoin%
\definecolor{currentfill}{rgb}{0.000000,0.000000,0.000000}%
\pgfsetfillcolor{currentfill}%
\pgfsetfillopacity{0.500000}%
\pgfsetlinewidth{0.000000pt}%
\definecolor{currentstroke}{rgb}{0.000000,0.000000,0.000000}%
\pgfsetstrokecolor{currentstroke}%
\pgfsetdash{}{0pt}%
\pgfpathmoveto{\pgfqpoint{0.920858in}{2.815854in}}%
\pgfpathcurveto{\pgfqpoint{0.926383in}{2.815854in}}{\pgfqpoint{0.931683in}{2.818049in}}{\pgfqpoint{0.935589in}{2.821955in}}%
\pgfpathcurveto{\pgfqpoint{0.939496in}{2.825862in}}{\pgfqpoint{0.941691in}{2.831162in}}{\pgfqpoint{0.941691in}{2.836687in}}%
\pgfpathcurveto{\pgfqpoint{0.941691in}{2.842212in}}{\pgfqpoint{0.939496in}{2.847511in}}{\pgfqpoint{0.935589in}{2.851418in}}%
\pgfpathcurveto{\pgfqpoint{0.931683in}{2.855325in}}{\pgfqpoint{0.926383in}{2.857520in}}{\pgfqpoint{0.920858in}{2.857520in}}%
\pgfpathcurveto{\pgfqpoint{0.915333in}{2.857520in}}{\pgfqpoint{0.910034in}{2.855325in}}{\pgfqpoint{0.906127in}{2.851418in}}%
\pgfpathcurveto{\pgfqpoint{0.902220in}{2.847511in}}{\pgfqpoint{0.900025in}{2.842212in}}{\pgfqpoint{0.900025in}{2.836687in}}%
\pgfpathcurveto{\pgfqpoint{0.900025in}{2.831162in}}{\pgfqpoint{0.902220in}{2.825862in}}{\pgfqpoint{0.906127in}{2.821955in}}%
\pgfpathcurveto{\pgfqpoint{0.910034in}{2.818049in}}{\pgfqpoint{0.915333in}{2.815854in}}{\pgfqpoint{0.920858in}{2.815854in}}%
\pgfpathclose%
\pgfusepath{fill}%
\end{pgfscope}%
\begin{pgfscope}%
\pgfpathrectangle{\pgfqpoint{0.833185in}{2.343185in}}{\pgfqpoint{1.162500in}{0.755000in}} %
\pgfusepath{clip}%
\pgfsetbuttcap%
\pgfsetroundjoin%
\definecolor{currentfill}{rgb}{0.000000,0.000000,0.000000}%
\pgfsetfillcolor{currentfill}%
\pgfsetfillopacity{0.500000}%
\pgfsetlinewidth{0.000000pt}%
\definecolor{currentstroke}{rgb}{0.000000,0.000000,0.000000}%
\pgfsetstrokecolor{currentstroke}%
\pgfsetdash{}{0pt}%
\pgfpathmoveto{\pgfqpoint{0.860863in}{2.815581in}}%
\pgfpathcurveto{\pgfqpoint{0.866388in}{2.815581in}}{\pgfqpoint{0.871688in}{2.817776in}}{\pgfqpoint{0.875595in}{2.821682in}}%
\pgfpathcurveto{\pgfqpoint{0.879501in}{2.825589in}}{\pgfqpoint{0.881696in}{2.830889in}}{\pgfqpoint{0.881696in}{2.836414in}}%
\pgfpathcurveto{\pgfqpoint{0.881696in}{2.841939in}}{\pgfqpoint{0.879501in}{2.847238in}}{\pgfqpoint{0.875595in}{2.851145in}}%
\pgfpathcurveto{\pgfqpoint{0.871688in}{2.855052in}}{\pgfqpoint{0.866388in}{2.857247in}}{\pgfqpoint{0.860863in}{2.857247in}}%
\pgfpathcurveto{\pgfqpoint{0.855338in}{2.857247in}}{\pgfqpoint{0.850039in}{2.855052in}}{\pgfqpoint{0.846132in}{2.851145in}}%
\pgfpathcurveto{\pgfqpoint{0.842225in}{2.847238in}}{\pgfqpoint{0.840030in}{2.841939in}}{\pgfqpoint{0.840030in}{2.836414in}}%
\pgfpathcurveto{\pgfqpoint{0.840030in}{2.830889in}}{\pgfqpoint{0.842225in}{2.825589in}}{\pgfqpoint{0.846132in}{2.821682in}}%
\pgfpathcurveto{\pgfqpoint{0.850039in}{2.817776in}}{\pgfqpoint{0.855338in}{2.815581in}}{\pgfqpoint{0.860863in}{2.815581in}}%
\pgfpathclose%
\pgfusepath{fill}%
\end{pgfscope}%
\begin{pgfscope}%
\pgfpathrectangle{\pgfqpoint{0.833185in}{2.343185in}}{\pgfqpoint{1.162500in}{0.755000in}} %
\pgfusepath{clip}%
\pgfsetbuttcap%
\pgfsetroundjoin%
\definecolor{currentfill}{rgb}{0.000000,0.000000,0.000000}%
\pgfsetfillcolor{currentfill}%
\pgfsetfillopacity{0.500000}%
\pgfsetlinewidth{0.000000pt}%
\definecolor{currentstroke}{rgb}{0.000000,0.000000,0.000000}%
\pgfsetstrokecolor{currentstroke}%
\pgfsetdash{}{0pt}%
\pgfpathmoveto{\pgfqpoint{1.180359in}{2.531108in}}%
\pgfpathcurveto{\pgfqpoint{1.185884in}{2.531108in}}{\pgfqpoint{1.191184in}{2.533303in}}{\pgfqpoint{1.195090in}{2.537210in}}%
\pgfpathcurveto{\pgfqpoint{1.198997in}{2.541116in}}{\pgfqpoint{1.201192in}{2.546416in}}{\pgfqpoint{1.201192in}{2.551941in}}%
\pgfpathcurveto{\pgfqpoint{1.201192in}{2.557466in}}{\pgfqpoint{1.198997in}{2.562766in}}{\pgfqpoint{1.195090in}{2.566672in}}%
\pgfpathcurveto{\pgfqpoint{1.191184in}{2.570579in}}{\pgfqpoint{1.185884in}{2.572774in}}{\pgfqpoint{1.180359in}{2.572774in}}%
\pgfpathcurveto{\pgfqpoint{1.174834in}{2.572774in}}{\pgfqpoint{1.169535in}{2.570579in}}{\pgfqpoint{1.165628in}{2.566672in}}%
\pgfpathcurveto{\pgfqpoint{1.161721in}{2.562766in}}{\pgfqpoint{1.159526in}{2.557466in}}{\pgfqpoint{1.159526in}{2.551941in}}%
\pgfpathcurveto{\pgfqpoint{1.159526in}{2.546416in}}{\pgfqpoint{1.161721in}{2.541116in}}{\pgfqpoint{1.165628in}{2.537210in}}%
\pgfpathcurveto{\pgfqpoint{1.169535in}{2.533303in}}{\pgfqpoint{1.174834in}{2.531108in}}{\pgfqpoint{1.180359in}{2.531108in}}%
\pgfpathclose%
\pgfusepath{fill}%
\end{pgfscope}%
\begin{pgfscope}%
\pgfpathrectangle{\pgfqpoint{0.833185in}{2.343185in}}{\pgfqpoint{1.162500in}{0.755000in}} %
\pgfusepath{clip}%
\pgfsetbuttcap%
\pgfsetroundjoin%
\definecolor{currentfill}{rgb}{0.000000,0.000000,0.000000}%
\pgfsetfillcolor{currentfill}%
\pgfsetfillopacity{0.500000}%
\pgfsetlinewidth{0.000000pt}%
\definecolor{currentstroke}{rgb}{0.000000,0.000000,0.000000}%
\pgfsetstrokecolor{currentstroke}%
\pgfsetdash{}{0pt}%
\pgfpathmoveto{\pgfqpoint{0.938297in}{2.481860in}}%
\pgfpathcurveto{\pgfqpoint{0.943822in}{2.481860in}}{\pgfqpoint{0.949121in}{2.484055in}}{\pgfqpoint{0.953028in}{2.487962in}}%
\pgfpathcurveto{\pgfqpoint{0.956935in}{2.491868in}}{\pgfqpoint{0.959130in}{2.497168in}}{\pgfqpoint{0.959130in}{2.502693in}}%
\pgfpathcurveto{\pgfqpoint{0.959130in}{2.508218in}}{\pgfqpoint{0.956935in}{2.513517in}}{\pgfqpoint{0.953028in}{2.517424in}}%
\pgfpathcurveto{\pgfqpoint{0.949121in}{2.521331in}}{\pgfqpoint{0.943822in}{2.523526in}}{\pgfqpoint{0.938297in}{2.523526in}}%
\pgfpathcurveto{\pgfqpoint{0.932772in}{2.523526in}}{\pgfqpoint{0.927472in}{2.521331in}}{\pgfqpoint{0.923565in}{2.517424in}}%
\pgfpathcurveto{\pgfqpoint{0.919659in}{2.513517in}}{\pgfqpoint{0.917464in}{2.508218in}}{\pgfqpoint{0.917464in}{2.502693in}}%
\pgfpathcurveto{\pgfqpoint{0.917464in}{2.497168in}}{\pgfqpoint{0.919659in}{2.491868in}}{\pgfqpoint{0.923565in}{2.487962in}}%
\pgfpathcurveto{\pgfqpoint{0.927472in}{2.484055in}}{\pgfqpoint{0.932772in}{2.481860in}}{\pgfqpoint{0.938297in}{2.481860in}}%
\pgfpathclose%
\pgfusepath{fill}%
\end{pgfscope}%
\begin{pgfscope}%
\pgfpathrectangle{\pgfqpoint{0.833185in}{2.343185in}}{\pgfqpoint{1.162500in}{0.755000in}} %
\pgfusepath{clip}%
\pgfsetbuttcap%
\pgfsetroundjoin%
\definecolor{currentfill}{rgb}{0.000000,0.000000,0.000000}%
\pgfsetfillcolor{currentfill}%
\pgfsetfillopacity{0.500000}%
\pgfsetlinewidth{0.000000pt}%
\definecolor{currentstroke}{rgb}{0.000000,0.000000,0.000000}%
\pgfsetstrokecolor{currentstroke}%
\pgfsetdash{}{0pt}%
\pgfpathmoveto{\pgfqpoint{1.098900in}{2.394040in}}%
\pgfpathcurveto{\pgfqpoint{1.104426in}{2.394040in}}{\pgfqpoint{1.109725in}{2.396235in}}{\pgfqpoint{1.113632in}{2.400142in}}%
\pgfpathcurveto{\pgfqpoint{1.117539in}{2.404048in}}{\pgfqpoint{1.119734in}{2.409348in}}{\pgfqpoint{1.119734in}{2.414873in}}%
\pgfpathcurveto{\pgfqpoint{1.119734in}{2.420398in}}{\pgfqpoint{1.117539in}{2.425698in}}{\pgfqpoint{1.113632in}{2.429604in}}%
\pgfpathcurveto{\pgfqpoint{1.109725in}{2.433511in}}{\pgfqpoint{1.104426in}{2.435706in}}{\pgfqpoint{1.098900in}{2.435706in}}%
\pgfpathcurveto{\pgfqpoint{1.093375in}{2.435706in}}{\pgfqpoint{1.088076in}{2.433511in}}{\pgfqpoint{1.084169in}{2.429604in}}%
\pgfpathcurveto{\pgfqpoint{1.080262in}{2.425698in}}{\pgfqpoint{1.078067in}{2.420398in}}{\pgfqpoint{1.078067in}{2.414873in}}%
\pgfpathcurveto{\pgfqpoint{1.078067in}{2.409348in}}{\pgfqpoint{1.080262in}{2.404048in}}{\pgfqpoint{1.084169in}{2.400142in}}%
\pgfpathcurveto{\pgfqpoint{1.088076in}{2.396235in}}{\pgfqpoint{1.093375in}{2.394040in}}{\pgfqpoint{1.098900in}{2.394040in}}%
\pgfpathclose%
\pgfusepath{fill}%
\end{pgfscope}%
\begin{pgfscope}%
\pgfpathrectangle{\pgfqpoint{0.833185in}{2.343185in}}{\pgfqpoint{1.162500in}{0.755000in}} %
\pgfusepath{clip}%
\pgfsetbuttcap%
\pgfsetroundjoin%
\definecolor{currentfill}{rgb}{0.000000,0.000000,0.000000}%
\pgfsetfillcolor{currentfill}%
\pgfsetfillopacity{0.500000}%
\pgfsetlinewidth{0.000000pt}%
\definecolor{currentstroke}{rgb}{0.000000,0.000000,0.000000}%
\pgfsetstrokecolor{currentstroke}%
\pgfsetdash{}{0pt}%
\pgfpathmoveto{\pgfqpoint{1.085127in}{2.340327in}}%
\pgfpathcurveto{\pgfqpoint{1.090652in}{2.340327in}}{\pgfqpoint{1.095951in}{2.342523in}}{\pgfqpoint{1.099858in}{2.346429in}}%
\pgfpathcurveto{\pgfqpoint{1.103765in}{2.350336in}}{\pgfqpoint{1.105960in}{2.355636in}}{\pgfqpoint{1.105960in}{2.361161in}}%
\pgfpathcurveto{\pgfqpoint{1.105960in}{2.366686in}}{\pgfqpoint{1.103765in}{2.371985in}}{\pgfqpoint{1.099858in}{2.375892in}}%
\pgfpathcurveto{\pgfqpoint{1.095951in}{2.379799in}}{\pgfqpoint{1.090652in}{2.381994in}}{\pgfqpoint{1.085127in}{2.381994in}}%
\pgfpathcurveto{\pgfqpoint{1.079602in}{2.381994in}}{\pgfqpoint{1.074302in}{2.379799in}}{\pgfqpoint{1.070395in}{2.375892in}}%
\pgfpathcurveto{\pgfqpoint{1.066488in}{2.371985in}}{\pgfqpoint{1.064293in}{2.366686in}}{\pgfqpoint{1.064293in}{2.361161in}}%
\pgfpathcurveto{\pgfqpoint{1.064293in}{2.355636in}}{\pgfqpoint{1.066488in}{2.350336in}}{\pgfqpoint{1.070395in}{2.346429in}}%
\pgfpathcurveto{\pgfqpoint{1.074302in}{2.342523in}}{\pgfqpoint{1.079602in}{2.340327in}}{\pgfqpoint{1.085127in}{2.340327in}}%
\pgfpathclose%
\pgfusepath{fill}%
\end{pgfscope}%
\begin{pgfscope}%
\pgfpathrectangle{\pgfqpoint{0.833185in}{2.343185in}}{\pgfqpoint{1.162500in}{0.755000in}} %
\pgfusepath{clip}%
\pgfsetbuttcap%
\pgfsetroundjoin%
\definecolor{currentfill}{rgb}{0.000000,0.000000,0.000000}%
\pgfsetfillcolor{currentfill}%
\pgfsetfillopacity{0.500000}%
\pgfsetlinewidth{0.000000pt}%
\definecolor{currentstroke}{rgb}{0.000000,0.000000,0.000000}%
\pgfsetstrokecolor{currentstroke}%
\pgfsetdash{}{0pt}%
\pgfpathmoveto{\pgfqpoint{1.896712in}{2.914639in}}%
\pgfpathcurveto{\pgfqpoint{1.902237in}{2.914639in}}{\pgfqpoint{1.907537in}{2.916834in}}{\pgfqpoint{1.911443in}{2.920741in}}%
\pgfpathcurveto{\pgfqpoint{1.915350in}{2.924648in}}{\pgfqpoint{1.917545in}{2.929947in}}{\pgfqpoint{1.917545in}{2.935473in}}%
\pgfpathcurveto{\pgfqpoint{1.917545in}{2.940998in}}{\pgfqpoint{1.915350in}{2.946297in}}{\pgfqpoint{1.911443in}{2.950204in}}%
\pgfpathcurveto{\pgfqpoint{1.907537in}{2.954111in}}{\pgfqpoint{1.902237in}{2.956306in}}{\pgfqpoint{1.896712in}{2.956306in}}%
\pgfpathcurveto{\pgfqpoint{1.891187in}{2.956306in}}{\pgfqpoint{1.885887in}{2.954111in}}{\pgfqpoint{1.881981in}{2.950204in}}%
\pgfpathcurveto{\pgfqpoint{1.878074in}{2.946297in}}{\pgfqpoint{1.875879in}{2.940998in}}{\pgfqpoint{1.875879in}{2.935473in}}%
\pgfpathcurveto{\pgfqpoint{1.875879in}{2.929947in}}{\pgfqpoint{1.878074in}{2.924648in}}{\pgfqpoint{1.881981in}{2.920741in}}%
\pgfpathcurveto{\pgfqpoint{1.885887in}{2.916834in}}{\pgfqpoint{1.891187in}{2.914639in}}{\pgfqpoint{1.896712in}{2.914639in}}%
\pgfpathclose%
\pgfusepath{fill}%
\end{pgfscope}%
\begin{pgfscope}%
\pgfpathrectangle{\pgfqpoint{0.833185in}{2.343185in}}{\pgfqpoint{1.162500in}{0.755000in}} %
\pgfusepath{clip}%
\pgfsetbuttcap%
\pgfsetroundjoin%
\definecolor{currentfill}{rgb}{0.000000,0.000000,0.000000}%
\pgfsetfillcolor{currentfill}%
\pgfsetfillopacity{0.500000}%
\pgfsetlinewidth{0.000000pt}%
\definecolor{currentstroke}{rgb}{0.000000,0.000000,0.000000}%
\pgfsetstrokecolor{currentstroke}%
\pgfsetdash{}{0pt}%
\pgfpathmoveto{\pgfqpoint{1.968006in}{2.734550in}}%
\pgfpathcurveto{\pgfqpoint{1.973531in}{2.734550in}}{\pgfqpoint{1.978831in}{2.736745in}}{\pgfqpoint{1.982737in}{2.740652in}}%
\pgfpathcurveto{\pgfqpoint{1.986644in}{2.744559in}}{\pgfqpoint{1.988839in}{2.749858in}}{\pgfqpoint{1.988839in}{2.755383in}}%
\pgfpathcurveto{\pgfqpoint{1.988839in}{2.760908in}}{\pgfqpoint{1.986644in}{2.766208in}}{\pgfqpoint{1.982737in}{2.770115in}}%
\pgfpathcurveto{\pgfqpoint{1.978831in}{2.774021in}}{\pgfqpoint{1.973531in}{2.776217in}}{\pgfqpoint{1.968006in}{2.776217in}}%
\pgfpathcurveto{\pgfqpoint{1.962481in}{2.776217in}}{\pgfqpoint{1.957181in}{2.774021in}}{\pgfqpoint{1.953275in}{2.770115in}}%
\pgfpathcurveto{\pgfqpoint{1.949368in}{2.766208in}}{\pgfqpoint{1.947173in}{2.760908in}}{\pgfqpoint{1.947173in}{2.755383in}}%
\pgfpathcurveto{\pgfqpoint{1.947173in}{2.749858in}}{\pgfqpoint{1.949368in}{2.744559in}}{\pgfqpoint{1.953275in}{2.740652in}}%
\pgfpathcurveto{\pgfqpoint{1.957181in}{2.736745in}}{\pgfqpoint{1.962481in}{2.734550in}}{\pgfqpoint{1.968006in}{2.734550in}}%
\pgfpathclose%
\pgfusepath{fill}%
\end{pgfscope}%
\begin{pgfscope}%
\pgfpathrectangle{\pgfqpoint{0.833185in}{2.343185in}}{\pgfqpoint{1.162500in}{0.755000in}} %
\pgfusepath{clip}%
\pgfsetbuttcap%
\pgfsetroundjoin%
\definecolor{currentfill}{rgb}{0.000000,0.000000,0.000000}%
\pgfsetfillcolor{currentfill}%
\pgfsetfillopacity{0.500000}%
\pgfsetlinewidth{0.000000pt}%
\definecolor{currentstroke}{rgb}{0.000000,0.000000,0.000000}%
\pgfsetstrokecolor{currentstroke}%
\pgfsetdash{}{0pt}%
\pgfpathmoveto{\pgfqpoint{1.742203in}{2.641930in}}%
\pgfpathcurveto{\pgfqpoint{1.747728in}{2.641930in}}{\pgfqpoint{1.753027in}{2.644125in}}{\pgfqpoint{1.756934in}{2.648032in}}%
\pgfpathcurveto{\pgfqpoint{1.760841in}{2.651939in}}{\pgfqpoint{1.763036in}{2.657239in}}{\pgfqpoint{1.763036in}{2.662764in}}%
\pgfpathcurveto{\pgfqpoint{1.763036in}{2.668289in}}{\pgfqpoint{1.760841in}{2.673588in}}{\pgfqpoint{1.756934in}{2.677495in}}%
\pgfpathcurveto{\pgfqpoint{1.753027in}{2.681402in}}{\pgfqpoint{1.747728in}{2.683597in}}{\pgfqpoint{1.742203in}{2.683597in}}%
\pgfpathcurveto{\pgfqpoint{1.736677in}{2.683597in}}{\pgfqpoint{1.731378in}{2.681402in}}{\pgfqpoint{1.727471in}{2.677495in}}%
\pgfpathcurveto{\pgfqpoint{1.723564in}{2.673588in}}{\pgfqpoint{1.721369in}{2.668289in}}{\pgfqpoint{1.721369in}{2.662764in}}%
\pgfpathcurveto{\pgfqpoint{1.721369in}{2.657239in}}{\pgfqpoint{1.723564in}{2.651939in}}{\pgfqpoint{1.727471in}{2.648032in}}%
\pgfpathcurveto{\pgfqpoint{1.731378in}{2.644125in}}{\pgfqpoint{1.736677in}{2.641930in}}{\pgfqpoint{1.742203in}{2.641930in}}%
\pgfpathclose%
\pgfusepath{fill}%
\end{pgfscope}%
\begin{pgfscope}%
\pgfpathrectangle{\pgfqpoint{0.833185in}{2.343185in}}{\pgfqpoint{1.162500in}{0.755000in}} %
\pgfusepath{clip}%
\pgfsetbuttcap%
\pgfsetroundjoin%
\definecolor{currentfill}{rgb}{0.000000,0.000000,0.000000}%
\pgfsetfillcolor{currentfill}%
\pgfsetfillopacity{0.500000}%
\pgfsetlinewidth{0.000000pt}%
\definecolor{currentstroke}{rgb}{0.000000,0.000000,0.000000}%
\pgfsetstrokecolor{currentstroke}%
\pgfsetdash{}{0pt}%
\pgfpathmoveto{\pgfqpoint{1.058530in}{2.975096in}}%
\pgfpathcurveto{\pgfqpoint{1.064055in}{2.975096in}}{\pgfqpoint{1.069355in}{2.977291in}}{\pgfqpoint{1.073261in}{2.981198in}}%
\pgfpathcurveto{\pgfqpoint{1.077168in}{2.985104in}}{\pgfqpoint{1.079363in}{2.990404in}}{\pgfqpoint{1.079363in}{2.995929in}}%
\pgfpathcurveto{\pgfqpoint{1.079363in}{3.001454in}}{\pgfqpoint{1.077168in}{3.006754in}}{\pgfqpoint{1.073261in}{3.010660in}}%
\pgfpathcurveto{\pgfqpoint{1.069355in}{3.014567in}}{\pgfqpoint{1.064055in}{3.016762in}}{\pgfqpoint{1.058530in}{3.016762in}}%
\pgfpathcurveto{\pgfqpoint{1.053005in}{3.016762in}}{\pgfqpoint{1.047706in}{3.014567in}}{\pgfqpoint{1.043799in}{3.010660in}}%
\pgfpathcurveto{\pgfqpoint{1.039892in}{3.006754in}}{\pgfqpoint{1.037697in}{3.001454in}}{\pgfqpoint{1.037697in}{2.995929in}}%
\pgfpathcurveto{\pgfqpoint{1.037697in}{2.990404in}}{\pgfqpoint{1.039892in}{2.985104in}}{\pgfqpoint{1.043799in}{2.981198in}}%
\pgfpathcurveto{\pgfqpoint{1.047706in}{2.977291in}}{\pgfqpoint{1.053005in}{2.975096in}}{\pgfqpoint{1.058530in}{2.975096in}}%
\pgfpathclose%
\pgfusepath{fill}%
\end{pgfscope}%
\begin{pgfscope}%
\pgfpathrectangle{\pgfqpoint{0.833185in}{2.343185in}}{\pgfqpoint{1.162500in}{0.755000in}} %
\pgfusepath{clip}%
\pgfsetbuttcap%
\pgfsetroundjoin%
\definecolor{currentfill}{rgb}{0.000000,0.000000,0.000000}%
\pgfsetfillcolor{currentfill}%
\pgfsetfillopacity{0.500000}%
\pgfsetlinewidth{0.000000pt}%
\definecolor{currentstroke}{rgb}{0.000000,0.000000,0.000000}%
\pgfsetstrokecolor{currentstroke}%
\pgfsetdash{}{0pt}%
\pgfpathmoveto{\pgfqpoint{1.520990in}{2.531911in}}%
\pgfpathcurveto{\pgfqpoint{1.526515in}{2.531911in}}{\pgfqpoint{1.531814in}{2.534106in}}{\pgfqpoint{1.535721in}{2.538013in}}%
\pgfpathcurveto{\pgfqpoint{1.539628in}{2.541920in}}{\pgfqpoint{1.541823in}{2.547220in}}{\pgfqpoint{1.541823in}{2.552745in}}%
\pgfpathcurveto{\pgfqpoint{1.541823in}{2.558270in}}{\pgfqpoint{1.539628in}{2.563569in}}{\pgfqpoint{1.535721in}{2.567476in}}%
\pgfpathcurveto{\pgfqpoint{1.531814in}{2.571383in}}{\pgfqpoint{1.526515in}{2.573578in}}{\pgfqpoint{1.520990in}{2.573578in}}%
\pgfpathcurveto{\pgfqpoint{1.515465in}{2.573578in}}{\pgfqpoint{1.510165in}{2.571383in}}{\pgfqpoint{1.506258in}{2.567476in}}%
\pgfpathcurveto{\pgfqpoint{1.502352in}{2.563569in}}{\pgfqpoint{1.500156in}{2.558270in}}{\pgfqpoint{1.500156in}{2.552745in}}%
\pgfpathcurveto{\pgfqpoint{1.500156in}{2.547220in}}{\pgfqpoint{1.502352in}{2.541920in}}{\pgfqpoint{1.506258in}{2.538013in}}%
\pgfpathcurveto{\pgfqpoint{1.510165in}{2.534106in}}{\pgfqpoint{1.515465in}{2.531911in}}{\pgfqpoint{1.520990in}{2.531911in}}%
\pgfpathclose%
\pgfusepath{fill}%
\end{pgfscope}%
\begin{pgfscope}%
\pgfpathrectangle{\pgfqpoint{0.833185in}{2.343185in}}{\pgfqpoint{1.162500in}{0.755000in}} %
\pgfusepath{clip}%
\pgfsetbuttcap%
\pgfsetroundjoin%
\definecolor{currentfill}{rgb}{0.000000,0.000000,0.000000}%
\pgfsetfillcolor{currentfill}%
\pgfsetfillopacity{0.500000}%
\pgfsetlinewidth{0.000000pt}%
\definecolor{currentstroke}{rgb}{0.000000,0.000000,0.000000}%
\pgfsetstrokecolor{currentstroke}%
\pgfsetdash{}{0pt}%
\pgfpathmoveto{\pgfqpoint{1.402203in}{2.591601in}}%
\pgfpathcurveto{\pgfqpoint{1.407728in}{2.591601in}}{\pgfqpoint{1.413027in}{2.593796in}}{\pgfqpoint{1.416934in}{2.597703in}}%
\pgfpathcurveto{\pgfqpoint{1.420841in}{2.601609in}}{\pgfqpoint{1.423036in}{2.606909in}}{\pgfqpoint{1.423036in}{2.612434in}}%
\pgfpathcurveto{\pgfqpoint{1.423036in}{2.617959in}}{\pgfqpoint{1.420841in}{2.623259in}}{\pgfqpoint{1.416934in}{2.627165in}}%
\pgfpathcurveto{\pgfqpoint{1.413027in}{2.631072in}}{\pgfqpoint{1.407728in}{2.633267in}}{\pgfqpoint{1.402203in}{2.633267in}}%
\pgfpathcurveto{\pgfqpoint{1.396677in}{2.633267in}}{\pgfqpoint{1.391378in}{2.631072in}}{\pgfqpoint{1.387471in}{2.627165in}}%
\pgfpathcurveto{\pgfqpoint{1.383564in}{2.623259in}}{\pgfqpoint{1.381369in}{2.617959in}}{\pgfqpoint{1.381369in}{2.612434in}}%
\pgfpathcurveto{\pgfqpoint{1.381369in}{2.606909in}}{\pgfqpoint{1.383564in}{2.601609in}}{\pgfqpoint{1.387471in}{2.597703in}}%
\pgfpathcurveto{\pgfqpoint{1.391378in}{2.593796in}}{\pgfqpoint{1.396677in}{2.591601in}}{\pgfqpoint{1.402203in}{2.591601in}}%
\pgfpathclose%
\pgfusepath{fill}%
\end{pgfscope}%
\begin{pgfscope}%
\pgfpathrectangle{\pgfqpoint{0.833185in}{2.343185in}}{\pgfqpoint{1.162500in}{0.755000in}} %
\pgfusepath{clip}%
\pgfsetbuttcap%
\pgfsetroundjoin%
\definecolor{currentfill}{rgb}{0.000000,0.000000,0.000000}%
\pgfsetfillcolor{currentfill}%
\pgfsetfillopacity{0.500000}%
\pgfsetlinewidth{0.000000pt}%
\definecolor{currentstroke}{rgb}{0.000000,0.000000,0.000000}%
\pgfsetstrokecolor{currentstroke}%
\pgfsetdash{}{0pt}%
\pgfpathmoveto{\pgfqpoint{1.053510in}{2.814070in}}%
\pgfpathcurveto{\pgfqpoint{1.059035in}{2.814070in}}{\pgfqpoint{1.064335in}{2.816265in}}{\pgfqpoint{1.068242in}{2.820172in}}%
\pgfpathcurveto{\pgfqpoint{1.072148in}{2.824079in}}{\pgfqpoint{1.074344in}{2.829378in}}{\pgfqpoint{1.074344in}{2.834904in}}%
\pgfpathcurveto{\pgfqpoint{1.074344in}{2.840429in}}{\pgfqpoint{1.072148in}{2.845728in}}{\pgfqpoint{1.068242in}{2.849635in}}%
\pgfpathcurveto{\pgfqpoint{1.064335in}{2.853542in}}{\pgfqpoint{1.059035in}{2.855737in}}{\pgfqpoint{1.053510in}{2.855737in}}%
\pgfpathcurveto{\pgfqpoint{1.047985in}{2.855737in}}{\pgfqpoint{1.042686in}{2.853542in}}{\pgfqpoint{1.038779in}{2.849635in}}%
\pgfpathcurveto{\pgfqpoint{1.034872in}{2.845728in}}{\pgfqpoint{1.032677in}{2.840429in}}{\pgfqpoint{1.032677in}{2.834904in}}%
\pgfpathcurveto{\pgfqpoint{1.032677in}{2.829378in}}{\pgfqpoint{1.034872in}{2.824079in}}{\pgfqpoint{1.038779in}{2.820172in}}%
\pgfpathcurveto{\pgfqpoint{1.042686in}{2.816265in}}{\pgfqpoint{1.047985in}{2.814070in}}{\pgfqpoint{1.053510in}{2.814070in}}%
\pgfpathclose%
\pgfusepath{fill}%
\end{pgfscope}%
\begin{pgfscope}%
\pgfpathrectangle{\pgfqpoint{0.833185in}{2.343185in}}{\pgfqpoint{1.162500in}{0.755000in}} %
\pgfusepath{clip}%
\pgfsetbuttcap%
\pgfsetroundjoin%
\definecolor{currentfill}{rgb}{0.000000,0.000000,0.000000}%
\pgfsetfillcolor{currentfill}%
\pgfsetfillopacity{0.500000}%
\pgfsetlinewidth{0.000000pt}%
\definecolor{currentstroke}{rgb}{0.000000,0.000000,0.000000}%
\pgfsetstrokecolor{currentstroke}%
\pgfsetdash{}{0pt}%
\pgfpathmoveto{\pgfqpoint{1.492358in}{2.392267in}}%
\pgfpathcurveto{\pgfqpoint{1.497883in}{2.392267in}}{\pgfqpoint{1.503182in}{2.394462in}}{\pgfqpoint{1.507089in}{2.398369in}}%
\pgfpathcurveto{\pgfqpoint{1.510996in}{2.402276in}}{\pgfqpoint{1.513191in}{2.407575in}}{\pgfqpoint{1.513191in}{2.413101in}}%
\pgfpathcurveto{\pgfqpoint{1.513191in}{2.418626in}}{\pgfqpoint{1.510996in}{2.423925in}}{\pgfqpoint{1.507089in}{2.427832in}}%
\pgfpathcurveto{\pgfqpoint{1.503182in}{2.431739in}}{\pgfqpoint{1.497883in}{2.433934in}}{\pgfqpoint{1.492358in}{2.433934in}}%
\pgfpathcurveto{\pgfqpoint{1.486832in}{2.433934in}}{\pgfqpoint{1.481533in}{2.431739in}}{\pgfqpoint{1.477626in}{2.427832in}}%
\pgfpathcurveto{\pgfqpoint{1.473719in}{2.423925in}}{\pgfqpoint{1.471524in}{2.418626in}}{\pgfqpoint{1.471524in}{2.413101in}}%
\pgfpathcurveto{\pgfqpoint{1.471524in}{2.407575in}}{\pgfqpoint{1.473719in}{2.402276in}}{\pgfqpoint{1.477626in}{2.398369in}}%
\pgfpathcurveto{\pgfqpoint{1.481533in}{2.394462in}}{\pgfqpoint{1.486832in}{2.392267in}}{\pgfqpoint{1.492358in}{2.392267in}}%
\pgfpathclose%
\pgfusepath{fill}%
\end{pgfscope}%
\begin{pgfscope}%
\pgfsetbuttcap%
\pgfsetroundjoin%
\definecolor{currentfill}{rgb}{0.000000,0.000000,0.000000}%
\pgfsetfillcolor{currentfill}%
\pgfsetlinewidth{0.803000pt}%
\definecolor{currentstroke}{rgb}{0.000000,0.000000,0.000000}%
\pgfsetstrokecolor{currentstroke}%
\pgfsetdash{}{0pt}%
\pgfsys@defobject{currentmarker}{\pgfqpoint{-0.048611in}{0.000000in}}{\pgfqpoint{0.000000in}{0.000000in}}{%
\pgfpathmoveto{\pgfqpoint{0.000000in}{0.000000in}}%
\pgfpathlineto{\pgfqpoint{-0.048611in}{0.000000in}}%
\pgfusepath{stroke,fill}%
}%
\begin{pgfscope}%
\pgfsys@transformshift{0.833185in}{2.654426in}%
\pgfsys@useobject{currentmarker}{}%
\end{pgfscope}%
\end{pgfscope}%
\begin{pgfscope}%
\pgftext[x=0.289968in,y=2.612217in,left,base]{\rmfamily\fontsize{8.000000}{9.600000}\selectfont \(\displaystyle 0.000025\)}%
\end{pgfscope}%
\begin{pgfscope}%
\pgfsetbuttcap%
\pgfsetroundjoin%
\definecolor{currentfill}{rgb}{0.000000,0.000000,0.000000}%
\pgfsetfillcolor{currentfill}%
\pgfsetlinewidth{0.803000pt}%
\definecolor{currentstroke}{rgb}{0.000000,0.000000,0.000000}%
\pgfsetstrokecolor{currentstroke}%
\pgfsetdash{}{0pt}%
\pgfsys@defobject{currentmarker}{\pgfqpoint{-0.048611in}{0.000000in}}{\pgfqpoint{0.000000in}{0.000000in}}{%
\pgfpathmoveto{\pgfqpoint{0.000000in}{0.000000in}}%
\pgfpathlineto{\pgfqpoint{-0.048611in}{0.000000in}}%
\pgfusepath{stroke,fill}%
}%
\begin{pgfscope}%
\pgfsys@transformshift{0.833185in}{3.044353in}%
\pgfsys@useobject{currentmarker}{}%
\end{pgfscope}%
\end{pgfscope}%
\begin{pgfscope}%
\pgftext[x=0.289968in,y=3.002144in,left,base]{\rmfamily\fontsize{8.000000}{9.600000}\selectfont \(\displaystyle 0.000050\)}%
\end{pgfscope}%
\begin{pgfscope}%
\pgftext[x=0.234413in,y=2.720685in,,bottom,rotate=90.000000]{\rmfamily\fontsize{10.000000}{12.000000}\selectfont area}%
\end{pgfscope}%
\begin{pgfscope}%
\pgfsetrectcap%
\pgfsetmiterjoin%
\pgfsetlinewidth{0.803000pt}%
\definecolor{currentstroke}{rgb}{0.000000,0.000000,0.000000}%
\pgfsetstrokecolor{currentstroke}%
\pgfsetdash{}{0pt}%
\pgfpathmoveto{\pgfqpoint{0.833185in}{2.343185in}}%
\pgfpathlineto{\pgfqpoint{0.833185in}{3.098185in}}%
\pgfusepath{stroke}%
\end{pgfscope}%
\begin{pgfscope}%
\pgfsetrectcap%
\pgfsetmiterjoin%
\pgfsetlinewidth{0.803000pt}%
\definecolor{currentstroke}{rgb}{0.000000,0.000000,0.000000}%
\pgfsetstrokecolor{currentstroke}%
\pgfsetdash{}{0pt}%
\pgfpathmoveto{\pgfqpoint{1.995685in}{2.343185in}}%
\pgfpathlineto{\pgfqpoint{1.995685in}{3.098185in}}%
\pgfusepath{stroke}%
\end{pgfscope}%
\begin{pgfscope}%
\pgfsetrectcap%
\pgfsetmiterjoin%
\pgfsetlinewidth{0.803000pt}%
\definecolor{currentstroke}{rgb}{0.000000,0.000000,0.000000}%
\pgfsetstrokecolor{currentstroke}%
\pgfsetdash{}{0pt}%
\pgfpathmoveto{\pgfqpoint{0.833185in}{2.343185in}}%
\pgfpathlineto{\pgfqpoint{1.995685in}{2.343185in}}%
\pgfusepath{stroke}%
\end{pgfscope}%
\begin{pgfscope}%
\pgfsetrectcap%
\pgfsetmiterjoin%
\pgfsetlinewidth{0.803000pt}%
\definecolor{currentstroke}{rgb}{0.000000,0.000000,0.000000}%
\pgfsetstrokecolor{currentstroke}%
\pgfsetdash{}{0pt}%
\pgfpathmoveto{\pgfqpoint{0.833185in}{3.098185in}}%
\pgfpathlineto{\pgfqpoint{1.995685in}{3.098185in}}%
\pgfusepath{stroke}%
\end{pgfscope}%
\begin{pgfscope}%
\pgfsetbuttcap%
\pgfsetmiterjoin%
\definecolor{currentfill}{rgb}{1.000000,1.000000,1.000000}%
\pgfsetfillcolor{currentfill}%
\pgfsetlinewidth{0.000000pt}%
\definecolor{currentstroke}{rgb}{0.000000,0.000000,0.000000}%
\pgfsetstrokecolor{currentstroke}%
\pgfsetstrokeopacity{0.000000}%
\pgfsetdash{}{0pt}%
\pgfpathmoveto{\pgfqpoint{1.995685in}{2.343185in}}%
\pgfpathlineto{\pgfqpoint{3.158185in}{2.343185in}}%
\pgfpathlineto{\pgfqpoint{3.158185in}{3.098185in}}%
\pgfpathlineto{\pgfqpoint{1.995685in}{3.098185in}}%
\pgfpathclose%
\pgfusepath{fill}%
\end{pgfscope}%
\begin{pgfscope}%
\pgfpathrectangle{\pgfqpoint{1.995685in}{2.343185in}}{\pgfqpoint{1.162500in}{0.755000in}} %
\pgfusepath{clip}%
\pgfsetrectcap%
\pgfsetroundjoin%
\pgfsetlinewidth{1.505625pt}%
\definecolor{currentstroke}{rgb}{0.121569,0.466667,0.705882}%
\pgfsetstrokecolor{currentstroke}%
\pgfsetdash{}{0pt}%
\pgfpathmoveto{\pgfqpoint{2.023363in}{2.636992in}}%
\pgfpathlineto{\pgfqpoint{2.061044in}{2.733201in}}%
\pgfpathlineto{\pgfqpoint{2.090966in}{2.803557in}}%
\pgfpathlineto{\pgfqpoint{2.116456in}{2.858016in}}%
\pgfpathlineto{\pgfqpoint{2.139730in}{2.902658in}}%
\pgfpathlineto{\pgfqpoint{2.160786in}{2.938493in}}%
\pgfpathlineto{\pgfqpoint{2.180735in}{2.968245in}}%
\pgfpathlineto{\pgfqpoint{2.199575in}{2.992499in}}%
\pgfpathlineto{\pgfqpoint{2.217307in}{3.011894in}}%
\pgfpathlineto{\pgfqpoint{2.235039in}{3.027999in}}%
\pgfpathlineto{\pgfqpoint{2.251663in}{3.040184in}}%
\pgfpathlineto{\pgfqpoint{2.268287in}{3.049660in}}%
\pgfpathlineto{\pgfqpoint{2.284910in}{3.056569in}}%
\pgfpathlineto{\pgfqpoint{2.301534in}{3.061086in}}%
\pgfpathlineto{\pgfqpoint{2.319266in}{3.063499in}}%
\pgfpathlineto{\pgfqpoint{2.336998in}{3.063703in}}%
\pgfpathlineto{\pgfqpoint{2.356947in}{3.061671in}}%
\pgfpathlineto{\pgfqpoint{2.379112in}{3.057152in}}%
\pgfpathlineto{\pgfqpoint{2.406818in}{3.049123in}}%
\pgfpathlineto{\pgfqpoint{2.455581in}{3.032151in}}%
\pgfpathlineto{\pgfqpoint{2.499911in}{3.017708in}}%
\pgfpathlineto{\pgfqpoint{2.532050in}{3.009550in}}%
\pgfpathlineto{\pgfqpoint{2.563081in}{3.003938in}}%
\pgfpathlineto{\pgfqpoint{2.598545in}{2.999866in}}%
\pgfpathlineto{\pgfqpoint{2.687206in}{2.990950in}}%
\pgfpathlineto{\pgfqpoint{2.711587in}{2.985549in}}%
\pgfpathlineto{\pgfqpoint{2.733752in}{2.978361in}}%
\pgfpathlineto{\pgfqpoint{2.753701in}{2.969630in}}%
\pgfpathlineto{\pgfqpoint{2.773649in}{2.958460in}}%
\pgfpathlineto{\pgfqpoint{2.793598in}{2.944633in}}%
\pgfpathlineto{\pgfqpoint{2.813546in}{2.928005in}}%
\pgfpathlineto{\pgfqpoint{2.833495in}{2.908505in}}%
\pgfpathlineto{\pgfqpoint{2.854551in}{2.884806in}}%
\pgfpathlineto{\pgfqpoint{2.876717in}{2.856493in}}%
\pgfpathlineto{\pgfqpoint{2.901098in}{2.821570in}}%
\pgfpathlineto{\pgfqpoint{2.927696in}{2.779342in}}%
\pgfpathlineto{\pgfqpoint{2.956511in}{2.729334in}}%
\pgfpathlineto{\pgfqpoint{2.989758in}{2.667097in}}%
\pgfpathlineto{\pgfqpoint{3.030763in}{2.585401in}}%
\pgfpathlineto{\pgfqpoint{3.095042in}{2.451609in}}%
\pgfpathlineto{\pgfqpoint{3.130506in}{2.377503in}}%
\pgfpathlineto{\pgfqpoint{3.130506in}{2.377503in}}%
\pgfusepath{stroke}%
\end{pgfscope}%
\begin{pgfscope}%
\pgfsetrectcap%
\pgfsetmiterjoin%
\pgfsetlinewidth{0.803000pt}%
\definecolor{currentstroke}{rgb}{0.000000,0.000000,0.000000}%
\pgfsetstrokecolor{currentstroke}%
\pgfsetdash{}{0pt}%
\pgfpathmoveto{\pgfqpoint{1.995685in}{2.343185in}}%
\pgfpathlineto{\pgfqpoint{1.995685in}{3.098185in}}%
\pgfusepath{stroke}%
\end{pgfscope}%
\begin{pgfscope}%
\pgfsetrectcap%
\pgfsetmiterjoin%
\pgfsetlinewidth{0.803000pt}%
\definecolor{currentstroke}{rgb}{0.000000,0.000000,0.000000}%
\pgfsetstrokecolor{currentstroke}%
\pgfsetdash{}{0pt}%
\pgfpathmoveto{\pgfqpoint{3.158185in}{2.343185in}}%
\pgfpathlineto{\pgfqpoint{3.158185in}{3.098185in}}%
\pgfusepath{stroke}%
\end{pgfscope}%
\begin{pgfscope}%
\pgfsetrectcap%
\pgfsetmiterjoin%
\pgfsetlinewidth{0.803000pt}%
\definecolor{currentstroke}{rgb}{0.000000,0.000000,0.000000}%
\pgfsetstrokecolor{currentstroke}%
\pgfsetdash{}{0pt}%
\pgfpathmoveto{\pgfqpoint{1.995685in}{2.343185in}}%
\pgfpathlineto{\pgfqpoint{3.158185in}{2.343185in}}%
\pgfusepath{stroke}%
\end{pgfscope}%
\begin{pgfscope}%
\pgfsetrectcap%
\pgfsetmiterjoin%
\pgfsetlinewidth{0.803000pt}%
\definecolor{currentstroke}{rgb}{0.000000,0.000000,0.000000}%
\pgfsetstrokecolor{currentstroke}%
\pgfsetdash{}{0pt}%
\pgfpathmoveto{\pgfqpoint{1.995685in}{3.098185in}}%
\pgfpathlineto{\pgfqpoint{3.158185in}{3.098185in}}%
\pgfusepath{stroke}%
\end{pgfscope}%
\begin{pgfscope}%
\pgfsetbuttcap%
\pgfsetmiterjoin%
\definecolor{currentfill}{rgb}{1.000000,1.000000,1.000000}%
\pgfsetfillcolor{currentfill}%
\pgfsetlinewidth{0.000000pt}%
\definecolor{currentstroke}{rgb}{0.000000,0.000000,0.000000}%
\pgfsetstrokecolor{currentstroke}%
\pgfsetstrokeopacity{0.000000}%
\pgfsetdash{}{0pt}%
\pgfpathmoveto{\pgfqpoint{3.158185in}{2.343185in}}%
\pgfpathlineto{\pgfqpoint{4.320685in}{2.343185in}}%
\pgfpathlineto{\pgfqpoint{4.320685in}{3.098185in}}%
\pgfpathlineto{\pgfqpoint{3.158185in}{3.098185in}}%
\pgfpathclose%
\pgfusepath{fill}%
\end{pgfscope}%
\begin{pgfscope}%
\pgfpathrectangle{\pgfqpoint{3.158185in}{2.343185in}}{\pgfqpoint{1.162500in}{0.755000in}} %
\pgfusepath{clip}%
\pgfsetbuttcap%
\pgfsetroundjoin%
\definecolor{currentfill}{rgb}{0.000000,0.000000,0.000000}%
\pgfsetfillcolor{currentfill}%
\pgfsetfillopacity{0.500000}%
\pgfsetlinewidth{0.000000pt}%
\definecolor{currentstroke}{rgb}{0.000000,0.000000,0.000000}%
\pgfsetstrokecolor{currentstroke}%
\pgfsetdash{}{0pt}%
\pgfpathmoveto{\pgfqpoint{4.293006in}{3.059375in}}%
\pgfpathcurveto{\pgfqpoint{4.298531in}{3.059375in}}{\pgfqpoint{4.303831in}{3.061570in}}{\pgfqpoint{4.307737in}{3.065477in}}%
\pgfpathcurveto{\pgfqpoint{4.311644in}{3.069384in}}{\pgfqpoint{4.313839in}{3.074683in}}{\pgfqpoint{4.313839in}{3.080208in}}%
\pgfpathcurveto{\pgfqpoint{4.313839in}{3.085733in}}{\pgfqpoint{4.311644in}{3.091033in}}{\pgfqpoint{4.307737in}{3.094940in}}%
\pgfpathcurveto{\pgfqpoint{4.303831in}{3.098847in}}{\pgfqpoint{4.298531in}{3.101042in}}{\pgfqpoint{4.293006in}{3.101042in}}%
\pgfpathcurveto{\pgfqpoint{4.287481in}{3.101042in}}{\pgfqpoint{4.282181in}{3.098847in}}{\pgfqpoint{4.278275in}{3.094940in}}%
\pgfpathcurveto{\pgfqpoint{4.274368in}{3.091033in}}{\pgfqpoint{4.272173in}{3.085733in}}{\pgfqpoint{4.272173in}{3.080208in}}%
\pgfpathcurveto{\pgfqpoint{4.272173in}{3.074683in}}{\pgfqpoint{4.274368in}{3.069384in}}{\pgfqpoint{4.278275in}{3.065477in}}%
\pgfpathcurveto{\pgfqpoint{4.282181in}{3.061570in}}{\pgfqpoint{4.287481in}{3.059375in}}{\pgfqpoint{4.293006in}{3.059375in}}%
\pgfpathclose%
\pgfusepath{fill}%
\end{pgfscope}%
\begin{pgfscope}%
\pgfpathrectangle{\pgfqpoint{3.158185in}{2.343185in}}{\pgfqpoint{1.162500in}{0.755000in}} %
\pgfusepath{clip}%
\pgfsetbuttcap%
\pgfsetroundjoin%
\definecolor{currentfill}{rgb}{0.000000,0.000000,0.000000}%
\pgfsetfillcolor{currentfill}%
\pgfsetfillopacity{0.500000}%
\pgfsetlinewidth{0.000000pt}%
\definecolor{currentstroke}{rgb}{0.000000,0.000000,0.000000}%
\pgfsetstrokecolor{currentstroke}%
\pgfsetdash{}{0pt}%
\pgfpathmoveto{\pgfqpoint{3.586831in}{2.815854in}}%
\pgfpathcurveto{\pgfqpoint{3.592356in}{2.815854in}}{\pgfqpoint{3.597655in}{2.818049in}}{\pgfqpoint{3.601562in}{2.821955in}}%
\pgfpathcurveto{\pgfqpoint{3.605469in}{2.825862in}}{\pgfqpoint{3.607664in}{2.831162in}}{\pgfqpoint{3.607664in}{2.836687in}}%
\pgfpathcurveto{\pgfqpoint{3.607664in}{2.842212in}}{\pgfqpoint{3.605469in}{2.847511in}}{\pgfqpoint{3.601562in}{2.851418in}}%
\pgfpathcurveto{\pgfqpoint{3.597655in}{2.855325in}}{\pgfqpoint{3.592356in}{2.857520in}}{\pgfqpoint{3.586831in}{2.857520in}}%
\pgfpathcurveto{\pgfqpoint{3.581306in}{2.857520in}}{\pgfqpoint{3.576006in}{2.855325in}}{\pgfqpoint{3.572099in}{2.851418in}}%
\pgfpathcurveto{\pgfqpoint{3.568193in}{2.847511in}}{\pgfqpoint{3.565998in}{2.842212in}}{\pgfqpoint{3.565998in}{2.836687in}}%
\pgfpathcurveto{\pgfqpoint{3.565998in}{2.831162in}}{\pgfqpoint{3.568193in}{2.825862in}}{\pgfqpoint{3.572099in}{2.821955in}}%
\pgfpathcurveto{\pgfqpoint{3.576006in}{2.818049in}}{\pgfqpoint{3.581306in}{2.815854in}}{\pgfqpoint{3.586831in}{2.815854in}}%
\pgfpathclose%
\pgfusepath{fill}%
\end{pgfscope}%
\begin{pgfscope}%
\pgfpathrectangle{\pgfqpoint{3.158185in}{2.343185in}}{\pgfqpoint{1.162500in}{0.755000in}} %
\pgfusepath{clip}%
\pgfsetbuttcap%
\pgfsetroundjoin%
\definecolor{currentfill}{rgb}{0.000000,0.000000,0.000000}%
\pgfsetfillcolor{currentfill}%
\pgfsetfillopacity{0.500000}%
\pgfsetlinewidth{0.000000pt}%
\definecolor{currentstroke}{rgb}{0.000000,0.000000,0.000000}%
\pgfsetstrokecolor{currentstroke}%
\pgfsetdash{}{0pt}%
\pgfpathmoveto{\pgfqpoint{3.578870in}{2.815581in}}%
\pgfpathcurveto{\pgfqpoint{3.584395in}{2.815581in}}{\pgfqpoint{3.589694in}{2.817776in}}{\pgfqpoint{3.593601in}{2.821682in}}%
\pgfpathcurveto{\pgfqpoint{3.597508in}{2.825589in}}{\pgfqpoint{3.599703in}{2.830889in}}{\pgfqpoint{3.599703in}{2.836414in}}%
\pgfpathcurveto{\pgfqpoint{3.599703in}{2.841939in}}{\pgfqpoint{3.597508in}{2.847238in}}{\pgfqpoint{3.593601in}{2.851145in}}%
\pgfpathcurveto{\pgfqpoint{3.589694in}{2.855052in}}{\pgfqpoint{3.584395in}{2.857247in}}{\pgfqpoint{3.578870in}{2.857247in}}%
\pgfpathcurveto{\pgfqpoint{3.573345in}{2.857247in}}{\pgfqpoint{3.568045in}{2.855052in}}{\pgfqpoint{3.564138in}{2.851145in}}%
\pgfpathcurveto{\pgfqpoint{3.560231in}{2.847238in}}{\pgfqpoint{3.558036in}{2.841939in}}{\pgfqpoint{3.558036in}{2.836414in}}%
\pgfpathcurveto{\pgfqpoint{3.558036in}{2.830889in}}{\pgfqpoint{3.560231in}{2.825589in}}{\pgfqpoint{3.564138in}{2.821682in}}%
\pgfpathcurveto{\pgfqpoint{3.568045in}{2.817776in}}{\pgfqpoint{3.573345in}{2.815581in}}{\pgfqpoint{3.578870in}{2.815581in}}%
\pgfpathclose%
\pgfusepath{fill}%
\end{pgfscope}%
\begin{pgfscope}%
\pgfpathrectangle{\pgfqpoint{3.158185in}{2.343185in}}{\pgfqpoint{1.162500in}{0.755000in}} %
\pgfusepath{clip}%
\pgfsetbuttcap%
\pgfsetroundjoin%
\definecolor{currentfill}{rgb}{0.000000,0.000000,0.000000}%
\pgfsetfillcolor{currentfill}%
\pgfsetfillopacity{0.500000}%
\pgfsetlinewidth{0.000000pt}%
\definecolor{currentstroke}{rgb}{0.000000,0.000000,0.000000}%
\pgfsetstrokecolor{currentstroke}%
\pgfsetdash{}{0pt}%
\pgfpathmoveto{\pgfqpoint{3.333069in}{2.531108in}}%
\pgfpathcurveto{\pgfqpoint{3.338594in}{2.531108in}}{\pgfqpoint{3.343894in}{2.533303in}}{\pgfqpoint{3.347801in}{2.537210in}}%
\pgfpathcurveto{\pgfqpoint{3.351708in}{2.541116in}}{\pgfqpoint{3.353903in}{2.546416in}}{\pgfqpoint{3.353903in}{2.551941in}}%
\pgfpathcurveto{\pgfqpoint{3.353903in}{2.557466in}}{\pgfqpoint{3.351708in}{2.562766in}}{\pgfqpoint{3.347801in}{2.566672in}}%
\pgfpathcurveto{\pgfqpoint{3.343894in}{2.570579in}}{\pgfqpoint{3.338594in}{2.572774in}}{\pgfqpoint{3.333069in}{2.572774in}}%
\pgfpathcurveto{\pgfqpoint{3.327544in}{2.572774in}}{\pgfqpoint{3.322245in}{2.570579in}}{\pgfqpoint{3.318338in}{2.566672in}}%
\pgfpathcurveto{\pgfqpoint{3.314431in}{2.562766in}}{\pgfqpoint{3.312236in}{2.557466in}}{\pgfqpoint{3.312236in}{2.551941in}}%
\pgfpathcurveto{\pgfqpoint{3.312236in}{2.546416in}}{\pgfqpoint{3.314431in}{2.541116in}}{\pgfqpoint{3.318338in}{2.537210in}}%
\pgfpathcurveto{\pgfqpoint{3.322245in}{2.533303in}}{\pgfqpoint{3.327544in}{2.531108in}}{\pgfqpoint{3.333069in}{2.531108in}}%
\pgfpathclose%
\pgfusepath{fill}%
\end{pgfscope}%
\begin{pgfscope}%
\pgfpathrectangle{\pgfqpoint{3.158185in}{2.343185in}}{\pgfqpoint{1.162500in}{0.755000in}} %
\pgfusepath{clip}%
\pgfsetbuttcap%
\pgfsetroundjoin%
\definecolor{currentfill}{rgb}{0.000000,0.000000,0.000000}%
\pgfsetfillcolor{currentfill}%
\pgfsetfillopacity{0.500000}%
\pgfsetlinewidth{0.000000pt}%
\definecolor{currentstroke}{rgb}{0.000000,0.000000,0.000000}%
\pgfsetstrokecolor{currentstroke}%
\pgfsetdash{}{0pt}%
\pgfpathmoveto{\pgfqpoint{3.296013in}{2.481860in}}%
\pgfpathcurveto{\pgfqpoint{3.301538in}{2.481860in}}{\pgfqpoint{3.306838in}{2.484055in}}{\pgfqpoint{3.310744in}{2.487962in}}%
\pgfpathcurveto{\pgfqpoint{3.314651in}{2.491868in}}{\pgfqpoint{3.316846in}{2.497168in}}{\pgfqpoint{3.316846in}{2.502693in}}%
\pgfpathcurveto{\pgfqpoint{3.316846in}{2.508218in}}{\pgfqpoint{3.314651in}{2.513517in}}{\pgfqpoint{3.310744in}{2.517424in}}%
\pgfpathcurveto{\pgfqpoint{3.306838in}{2.521331in}}{\pgfqpoint{3.301538in}{2.523526in}}{\pgfqpoint{3.296013in}{2.523526in}}%
\pgfpathcurveto{\pgfqpoint{3.290488in}{2.523526in}}{\pgfqpoint{3.285188in}{2.521331in}}{\pgfqpoint{3.281282in}{2.517424in}}%
\pgfpathcurveto{\pgfqpoint{3.277375in}{2.513517in}}{\pgfqpoint{3.275180in}{2.508218in}}{\pgfqpoint{3.275180in}{2.502693in}}%
\pgfpathcurveto{\pgfqpoint{3.275180in}{2.497168in}}{\pgfqpoint{3.277375in}{2.491868in}}{\pgfqpoint{3.281282in}{2.487962in}}%
\pgfpathcurveto{\pgfqpoint{3.285188in}{2.484055in}}{\pgfqpoint{3.290488in}{2.481860in}}{\pgfqpoint{3.296013in}{2.481860in}}%
\pgfpathclose%
\pgfusepath{fill}%
\end{pgfscope}%
\begin{pgfscope}%
\pgfpathrectangle{\pgfqpoint{3.158185in}{2.343185in}}{\pgfqpoint{1.162500in}{0.755000in}} %
\pgfusepath{clip}%
\pgfsetbuttcap%
\pgfsetroundjoin%
\definecolor{currentfill}{rgb}{0.000000,0.000000,0.000000}%
\pgfsetfillcolor{currentfill}%
\pgfsetfillopacity{0.500000}%
\pgfsetlinewidth{0.000000pt}%
\definecolor{currentstroke}{rgb}{0.000000,0.000000,0.000000}%
\pgfsetstrokecolor{currentstroke}%
\pgfsetdash{}{0pt}%
\pgfpathmoveto{\pgfqpoint{3.206735in}{2.394040in}}%
\pgfpathcurveto{\pgfqpoint{3.212260in}{2.394040in}}{\pgfqpoint{3.217559in}{2.396235in}}{\pgfqpoint{3.221466in}{2.400142in}}%
\pgfpathcurveto{\pgfqpoint{3.225373in}{2.404048in}}{\pgfqpoint{3.227568in}{2.409348in}}{\pgfqpoint{3.227568in}{2.414873in}}%
\pgfpathcurveto{\pgfqpoint{3.227568in}{2.420398in}}{\pgfqpoint{3.225373in}{2.425698in}}{\pgfqpoint{3.221466in}{2.429604in}}%
\pgfpathcurveto{\pgfqpoint{3.217559in}{2.433511in}}{\pgfqpoint{3.212260in}{2.435706in}}{\pgfqpoint{3.206735in}{2.435706in}}%
\pgfpathcurveto{\pgfqpoint{3.201210in}{2.435706in}}{\pgfqpoint{3.195910in}{2.433511in}}{\pgfqpoint{3.192003in}{2.429604in}}%
\pgfpathcurveto{\pgfqpoint{3.188097in}{2.425698in}}{\pgfqpoint{3.185901in}{2.420398in}}{\pgfqpoint{3.185901in}{2.414873in}}%
\pgfpathcurveto{\pgfqpoint{3.185901in}{2.409348in}}{\pgfqpoint{3.188097in}{2.404048in}}{\pgfqpoint{3.192003in}{2.400142in}}%
\pgfpathcurveto{\pgfqpoint{3.195910in}{2.396235in}}{\pgfqpoint{3.201210in}{2.394040in}}{\pgfqpoint{3.206735in}{2.394040in}}%
\pgfpathclose%
\pgfusepath{fill}%
\end{pgfscope}%
\begin{pgfscope}%
\pgfpathrectangle{\pgfqpoint{3.158185in}{2.343185in}}{\pgfqpoint{1.162500in}{0.755000in}} %
\pgfusepath{clip}%
\pgfsetbuttcap%
\pgfsetroundjoin%
\definecolor{currentfill}{rgb}{0.000000,0.000000,0.000000}%
\pgfsetfillcolor{currentfill}%
\pgfsetfillopacity{0.500000}%
\pgfsetlinewidth{0.000000pt}%
\definecolor{currentstroke}{rgb}{0.000000,0.000000,0.000000}%
\pgfsetstrokecolor{currentstroke}%
\pgfsetdash{}{0pt}%
\pgfpathmoveto{\pgfqpoint{3.185863in}{2.340327in}}%
\pgfpathcurveto{\pgfqpoint{3.191388in}{2.340327in}}{\pgfqpoint{3.196688in}{2.342523in}}{\pgfqpoint{3.200595in}{2.346429in}}%
\pgfpathcurveto{\pgfqpoint{3.204501in}{2.350336in}}{\pgfqpoint{3.206696in}{2.355636in}}{\pgfqpoint{3.206696in}{2.361161in}}%
\pgfpathcurveto{\pgfqpoint{3.206696in}{2.366686in}}{\pgfqpoint{3.204501in}{2.371985in}}{\pgfqpoint{3.200595in}{2.375892in}}%
\pgfpathcurveto{\pgfqpoint{3.196688in}{2.379799in}}{\pgfqpoint{3.191388in}{2.381994in}}{\pgfqpoint{3.185863in}{2.381994in}}%
\pgfpathcurveto{\pgfqpoint{3.180338in}{2.381994in}}{\pgfqpoint{3.175039in}{2.379799in}}{\pgfqpoint{3.171132in}{2.375892in}}%
\pgfpathcurveto{\pgfqpoint{3.167225in}{2.371985in}}{\pgfqpoint{3.165030in}{2.366686in}}{\pgfqpoint{3.165030in}{2.361161in}}%
\pgfpathcurveto{\pgfqpoint{3.165030in}{2.355636in}}{\pgfqpoint{3.167225in}{2.350336in}}{\pgfqpoint{3.171132in}{2.346429in}}%
\pgfpathcurveto{\pgfqpoint{3.175039in}{2.342523in}}{\pgfqpoint{3.180338in}{2.340327in}}{\pgfqpoint{3.185863in}{2.340327in}}%
\pgfpathclose%
\pgfusepath{fill}%
\end{pgfscope}%
\begin{pgfscope}%
\pgfpathrectangle{\pgfqpoint{3.158185in}{2.343185in}}{\pgfqpoint{1.162500in}{0.755000in}} %
\pgfusepath{clip}%
\pgfsetbuttcap%
\pgfsetroundjoin%
\definecolor{currentfill}{rgb}{0.000000,0.000000,0.000000}%
\pgfsetfillcolor{currentfill}%
\pgfsetfillopacity{0.500000}%
\pgfsetlinewidth{0.000000pt}%
\definecolor{currentstroke}{rgb}{0.000000,0.000000,0.000000}%
\pgfsetstrokecolor{currentstroke}%
\pgfsetdash{}{0pt}%
\pgfpathmoveto{\pgfqpoint{3.808562in}{2.914639in}}%
\pgfpathcurveto{\pgfqpoint{3.814087in}{2.914639in}}{\pgfqpoint{3.819386in}{2.916834in}}{\pgfqpoint{3.823293in}{2.920741in}}%
\pgfpathcurveto{\pgfqpoint{3.827200in}{2.924648in}}{\pgfqpoint{3.829395in}{2.929947in}}{\pgfqpoint{3.829395in}{2.935473in}}%
\pgfpathcurveto{\pgfqpoint{3.829395in}{2.940998in}}{\pgfqpoint{3.827200in}{2.946297in}}{\pgfqpoint{3.823293in}{2.950204in}}%
\pgfpathcurveto{\pgfqpoint{3.819386in}{2.954111in}}{\pgfqpoint{3.814087in}{2.956306in}}{\pgfqpoint{3.808562in}{2.956306in}}%
\pgfpathcurveto{\pgfqpoint{3.803037in}{2.956306in}}{\pgfqpoint{3.797737in}{2.954111in}}{\pgfqpoint{3.793830in}{2.950204in}}%
\pgfpathcurveto{\pgfqpoint{3.789924in}{2.946297in}}{\pgfqpoint{3.787728in}{2.940998in}}{\pgfqpoint{3.787728in}{2.935473in}}%
\pgfpathcurveto{\pgfqpoint{3.787728in}{2.929947in}}{\pgfqpoint{3.789924in}{2.924648in}}{\pgfqpoint{3.793830in}{2.920741in}}%
\pgfpathcurveto{\pgfqpoint{3.797737in}{2.916834in}}{\pgfqpoint{3.803037in}{2.914639in}}{\pgfqpoint{3.808562in}{2.914639in}}%
\pgfpathclose%
\pgfusepath{fill}%
\end{pgfscope}%
\begin{pgfscope}%
\pgfpathrectangle{\pgfqpoint{3.158185in}{2.343185in}}{\pgfqpoint{1.162500in}{0.755000in}} %
\pgfusepath{clip}%
\pgfsetbuttcap%
\pgfsetroundjoin%
\definecolor{currentfill}{rgb}{0.000000,0.000000,0.000000}%
\pgfsetfillcolor{currentfill}%
\pgfsetfillopacity{0.500000}%
\pgfsetlinewidth{0.000000pt}%
\definecolor{currentstroke}{rgb}{0.000000,0.000000,0.000000}%
\pgfsetstrokecolor{currentstroke}%
\pgfsetdash{}{0pt}%
\pgfpathmoveto{\pgfqpoint{3.620933in}{2.734550in}}%
\pgfpathcurveto{\pgfqpoint{3.626458in}{2.734550in}}{\pgfqpoint{3.631758in}{2.736745in}}{\pgfqpoint{3.635664in}{2.740652in}}%
\pgfpathcurveto{\pgfqpoint{3.639571in}{2.744559in}}{\pgfqpoint{3.641766in}{2.749858in}}{\pgfqpoint{3.641766in}{2.755383in}}%
\pgfpathcurveto{\pgfqpoint{3.641766in}{2.760908in}}{\pgfqpoint{3.639571in}{2.766208in}}{\pgfqpoint{3.635664in}{2.770115in}}%
\pgfpathcurveto{\pgfqpoint{3.631758in}{2.774021in}}{\pgfqpoint{3.626458in}{2.776217in}}{\pgfqpoint{3.620933in}{2.776217in}}%
\pgfpathcurveto{\pgfqpoint{3.615408in}{2.776217in}}{\pgfqpoint{3.610108in}{2.774021in}}{\pgfqpoint{3.606202in}{2.770115in}}%
\pgfpathcurveto{\pgfqpoint{3.602295in}{2.766208in}}{\pgfqpoint{3.600100in}{2.760908in}}{\pgfqpoint{3.600100in}{2.755383in}}%
\pgfpathcurveto{\pgfqpoint{3.600100in}{2.749858in}}{\pgfqpoint{3.602295in}{2.744559in}}{\pgfqpoint{3.606202in}{2.740652in}}%
\pgfpathcurveto{\pgfqpoint{3.610108in}{2.736745in}}{\pgfqpoint{3.615408in}{2.734550in}}{\pgfqpoint{3.620933in}{2.734550in}}%
\pgfpathclose%
\pgfusepath{fill}%
\end{pgfscope}%
\begin{pgfscope}%
\pgfpathrectangle{\pgfqpoint{3.158185in}{2.343185in}}{\pgfqpoint{1.162500in}{0.755000in}} %
\pgfusepath{clip}%
\pgfsetbuttcap%
\pgfsetroundjoin%
\definecolor{currentfill}{rgb}{0.000000,0.000000,0.000000}%
\pgfsetfillcolor{currentfill}%
\pgfsetfillopacity{0.500000}%
\pgfsetlinewidth{0.000000pt}%
\definecolor{currentstroke}{rgb}{0.000000,0.000000,0.000000}%
\pgfsetstrokecolor{currentstroke}%
\pgfsetdash{}{0pt}%
\pgfpathmoveto{\pgfqpoint{3.600191in}{2.641930in}}%
\pgfpathcurveto{\pgfqpoint{3.605716in}{2.641930in}}{\pgfqpoint{3.611016in}{2.644125in}}{\pgfqpoint{3.614922in}{2.648032in}}%
\pgfpathcurveto{\pgfqpoint{3.618829in}{2.651939in}}{\pgfqpoint{3.621024in}{2.657239in}}{\pgfqpoint{3.621024in}{2.662764in}}%
\pgfpathcurveto{\pgfqpoint{3.621024in}{2.668289in}}{\pgfqpoint{3.618829in}{2.673588in}}{\pgfqpoint{3.614922in}{2.677495in}}%
\pgfpathcurveto{\pgfqpoint{3.611016in}{2.681402in}}{\pgfqpoint{3.605716in}{2.683597in}}{\pgfqpoint{3.600191in}{2.683597in}}%
\pgfpathcurveto{\pgfqpoint{3.594666in}{2.683597in}}{\pgfqpoint{3.589366in}{2.681402in}}{\pgfqpoint{3.585460in}{2.677495in}}%
\pgfpathcurveto{\pgfqpoint{3.581553in}{2.673588in}}{\pgfqpoint{3.579358in}{2.668289in}}{\pgfqpoint{3.579358in}{2.662764in}}%
\pgfpathcurveto{\pgfqpoint{3.579358in}{2.657239in}}{\pgfqpoint{3.581553in}{2.651939in}}{\pgfqpoint{3.585460in}{2.648032in}}%
\pgfpathcurveto{\pgfqpoint{3.589366in}{2.644125in}}{\pgfqpoint{3.594666in}{2.641930in}}{\pgfqpoint{3.600191in}{2.641930in}}%
\pgfpathclose%
\pgfusepath{fill}%
\end{pgfscope}%
\begin{pgfscope}%
\pgfpathrectangle{\pgfqpoint{3.158185in}{2.343185in}}{\pgfqpoint{1.162500in}{0.755000in}} %
\pgfusepath{clip}%
\pgfsetbuttcap%
\pgfsetroundjoin%
\definecolor{currentfill}{rgb}{0.000000,0.000000,0.000000}%
\pgfsetfillcolor{currentfill}%
\pgfsetfillopacity{0.500000}%
\pgfsetlinewidth{0.000000pt}%
\definecolor{currentstroke}{rgb}{0.000000,0.000000,0.000000}%
\pgfsetstrokecolor{currentstroke}%
\pgfsetdash{}{0pt}%
\pgfpathmoveto{\pgfqpoint{3.899960in}{2.975096in}}%
\pgfpathcurveto{\pgfqpoint{3.905485in}{2.975096in}}{\pgfqpoint{3.910785in}{2.977291in}}{\pgfqpoint{3.914692in}{2.981198in}}%
\pgfpathcurveto{\pgfqpoint{3.918598in}{2.985104in}}{\pgfqpoint{3.920794in}{2.990404in}}{\pgfqpoint{3.920794in}{2.995929in}}%
\pgfpathcurveto{\pgfqpoint{3.920794in}{3.001454in}}{\pgfqpoint{3.918598in}{3.006754in}}{\pgfqpoint{3.914692in}{3.010660in}}%
\pgfpathcurveto{\pgfqpoint{3.910785in}{3.014567in}}{\pgfqpoint{3.905485in}{3.016762in}}{\pgfqpoint{3.899960in}{3.016762in}}%
\pgfpathcurveto{\pgfqpoint{3.894435in}{3.016762in}}{\pgfqpoint{3.889136in}{3.014567in}}{\pgfqpoint{3.885229in}{3.010660in}}%
\pgfpathcurveto{\pgfqpoint{3.881322in}{3.006754in}}{\pgfqpoint{3.879127in}{3.001454in}}{\pgfqpoint{3.879127in}{2.995929in}}%
\pgfpathcurveto{\pgfqpoint{3.879127in}{2.990404in}}{\pgfqpoint{3.881322in}{2.985104in}}{\pgfqpoint{3.885229in}{2.981198in}}%
\pgfpathcurveto{\pgfqpoint{3.889136in}{2.977291in}}{\pgfqpoint{3.894435in}{2.975096in}}{\pgfqpoint{3.899960in}{2.975096in}}%
\pgfpathclose%
\pgfusepath{fill}%
\end{pgfscope}%
\begin{pgfscope}%
\pgfpathrectangle{\pgfqpoint{3.158185in}{2.343185in}}{\pgfqpoint{1.162500in}{0.755000in}} %
\pgfusepath{clip}%
\pgfsetbuttcap%
\pgfsetroundjoin%
\definecolor{currentfill}{rgb}{0.000000,0.000000,0.000000}%
\pgfsetfillcolor{currentfill}%
\pgfsetfillopacity{0.500000}%
\pgfsetlinewidth{0.000000pt}%
\definecolor{currentstroke}{rgb}{0.000000,0.000000,0.000000}%
\pgfsetstrokecolor{currentstroke}%
\pgfsetdash{}{0pt}%
\pgfpathmoveto{\pgfqpoint{3.368320in}{2.531911in}}%
\pgfpathcurveto{\pgfqpoint{3.373845in}{2.531911in}}{\pgfqpoint{3.379145in}{2.534106in}}{\pgfqpoint{3.383052in}{2.538013in}}%
\pgfpathcurveto{\pgfqpoint{3.386958in}{2.541920in}}{\pgfqpoint{3.389154in}{2.547220in}}{\pgfqpoint{3.389154in}{2.552745in}}%
\pgfpathcurveto{\pgfqpoint{3.389154in}{2.558270in}}{\pgfqpoint{3.386958in}{2.563569in}}{\pgfqpoint{3.383052in}{2.567476in}}%
\pgfpathcurveto{\pgfqpoint{3.379145in}{2.571383in}}{\pgfqpoint{3.373845in}{2.573578in}}{\pgfqpoint{3.368320in}{2.573578in}}%
\pgfpathcurveto{\pgfqpoint{3.362795in}{2.573578in}}{\pgfqpoint{3.357496in}{2.571383in}}{\pgfqpoint{3.353589in}{2.567476in}}%
\pgfpathcurveto{\pgfqpoint{3.349682in}{2.563569in}}{\pgfqpoint{3.347487in}{2.558270in}}{\pgfqpoint{3.347487in}{2.552745in}}%
\pgfpathcurveto{\pgfqpoint{3.347487in}{2.547220in}}{\pgfqpoint{3.349682in}{2.541920in}}{\pgfqpoint{3.353589in}{2.538013in}}%
\pgfpathcurveto{\pgfqpoint{3.357496in}{2.534106in}}{\pgfqpoint{3.362795in}{2.531911in}}{\pgfqpoint{3.368320in}{2.531911in}}%
\pgfpathclose%
\pgfusepath{fill}%
\end{pgfscope}%
\begin{pgfscope}%
\pgfpathrectangle{\pgfqpoint{3.158185in}{2.343185in}}{\pgfqpoint{1.162500in}{0.755000in}} %
\pgfusepath{clip}%
\pgfsetbuttcap%
\pgfsetroundjoin%
\definecolor{currentfill}{rgb}{0.000000,0.000000,0.000000}%
\pgfsetfillcolor{currentfill}%
\pgfsetfillopacity{0.500000}%
\pgfsetlinewidth{0.000000pt}%
\definecolor{currentstroke}{rgb}{0.000000,0.000000,0.000000}%
\pgfsetstrokecolor{currentstroke}%
\pgfsetdash{}{0pt}%
\pgfpathmoveto{\pgfqpoint{3.416175in}{2.591601in}}%
\pgfpathcurveto{\pgfqpoint{3.421700in}{2.591601in}}{\pgfqpoint{3.427000in}{2.593796in}}{\pgfqpoint{3.430907in}{2.597703in}}%
\pgfpathcurveto{\pgfqpoint{3.434814in}{2.601609in}}{\pgfqpoint{3.437009in}{2.606909in}}{\pgfqpoint{3.437009in}{2.612434in}}%
\pgfpathcurveto{\pgfqpoint{3.437009in}{2.617959in}}{\pgfqpoint{3.434814in}{2.623259in}}{\pgfqpoint{3.430907in}{2.627165in}}%
\pgfpathcurveto{\pgfqpoint{3.427000in}{2.631072in}}{\pgfqpoint{3.421700in}{2.633267in}}{\pgfqpoint{3.416175in}{2.633267in}}%
\pgfpathcurveto{\pgfqpoint{3.410650in}{2.633267in}}{\pgfqpoint{3.405351in}{2.631072in}}{\pgfqpoint{3.401444in}{2.627165in}}%
\pgfpathcurveto{\pgfqpoint{3.397537in}{2.623259in}}{\pgfqpoint{3.395342in}{2.617959in}}{\pgfqpoint{3.395342in}{2.612434in}}%
\pgfpathcurveto{\pgfqpoint{3.395342in}{2.606909in}}{\pgfqpoint{3.397537in}{2.601609in}}{\pgfqpoint{3.401444in}{2.597703in}}%
\pgfpathcurveto{\pgfqpoint{3.405351in}{2.593796in}}{\pgfqpoint{3.410650in}{2.591601in}}{\pgfqpoint{3.416175in}{2.591601in}}%
\pgfpathclose%
\pgfusepath{fill}%
\end{pgfscope}%
\begin{pgfscope}%
\pgfpathrectangle{\pgfqpoint{3.158185in}{2.343185in}}{\pgfqpoint{1.162500in}{0.755000in}} %
\pgfusepath{clip}%
\pgfsetbuttcap%
\pgfsetroundjoin%
\definecolor{currentfill}{rgb}{0.000000,0.000000,0.000000}%
\pgfsetfillcolor{currentfill}%
\pgfsetfillopacity{0.500000}%
\pgfsetlinewidth{0.000000pt}%
\definecolor{currentstroke}{rgb}{0.000000,0.000000,0.000000}%
\pgfsetstrokecolor{currentstroke}%
\pgfsetdash{}{0pt}%
\pgfpathmoveto{\pgfqpoint{3.789669in}{2.814070in}}%
\pgfpathcurveto{\pgfqpoint{3.795194in}{2.814070in}}{\pgfqpoint{3.800493in}{2.816265in}}{\pgfqpoint{3.804400in}{2.820172in}}%
\pgfpathcurveto{\pgfqpoint{3.808307in}{2.824079in}}{\pgfqpoint{3.810502in}{2.829378in}}{\pgfqpoint{3.810502in}{2.834904in}}%
\pgfpathcurveto{\pgfqpoint{3.810502in}{2.840429in}}{\pgfqpoint{3.808307in}{2.845728in}}{\pgfqpoint{3.804400in}{2.849635in}}%
\pgfpathcurveto{\pgfqpoint{3.800493in}{2.853542in}}{\pgfqpoint{3.795194in}{2.855737in}}{\pgfqpoint{3.789669in}{2.855737in}}%
\pgfpathcurveto{\pgfqpoint{3.784143in}{2.855737in}}{\pgfqpoint{3.778844in}{2.853542in}}{\pgfqpoint{3.774937in}{2.849635in}}%
\pgfpathcurveto{\pgfqpoint{3.771030in}{2.845728in}}{\pgfqpoint{3.768835in}{2.840429in}}{\pgfqpoint{3.768835in}{2.834904in}}%
\pgfpathcurveto{\pgfqpoint{3.768835in}{2.829378in}}{\pgfqpoint{3.771030in}{2.824079in}}{\pgfqpoint{3.774937in}{2.820172in}}%
\pgfpathcurveto{\pgfqpoint{3.778844in}{2.816265in}}{\pgfqpoint{3.784143in}{2.814070in}}{\pgfqpoint{3.789669in}{2.814070in}}%
\pgfpathclose%
\pgfusepath{fill}%
\end{pgfscope}%
\begin{pgfscope}%
\pgfpathrectangle{\pgfqpoint{3.158185in}{2.343185in}}{\pgfqpoint{1.162500in}{0.755000in}} %
\pgfusepath{clip}%
\pgfsetbuttcap%
\pgfsetroundjoin%
\definecolor{currentfill}{rgb}{0.000000,0.000000,0.000000}%
\pgfsetfillcolor{currentfill}%
\pgfsetfillopacity{0.500000}%
\pgfsetlinewidth{0.000000pt}%
\definecolor{currentstroke}{rgb}{0.000000,0.000000,0.000000}%
\pgfsetstrokecolor{currentstroke}%
\pgfsetdash{}{0pt}%
\pgfpathmoveto{\pgfqpoint{3.231064in}{2.392267in}}%
\pgfpathcurveto{\pgfqpoint{3.236589in}{2.392267in}}{\pgfqpoint{3.241888in}{2.394462in}}{\pgfqpoint{3.245795in}{2.398369in}}%
\pgfpathcurveto{\pgfqpoint{3.249702in}{2.402276in}}{\pgfqpoint{3.251897in}{2.407575in}}{\pgfqpoint{3.251897in}{2.413101in}}%
\pgfpathcurveto{\pgfqpoint{3.251897in}{2.418626in}}{\pgfqpoint{3.249702in}{2.423925in}}{\pgfqpoint{3.245795in}{2.427832in}}%
\pgfpathcurveto{\pgfqpoint{3.241888in}{2.431739in}}{\pgfqpoint{3.236589in}{2.433934in}}{\pgfqpoint{3.231064in}{2.433934in}}%
\pgfpathcurveto{\pgfqpoint{3.225539in}{2.433934in}}{\pgfqpoint{3.220239in}{2.431739in}}{\pgfqpoint{3.216332in}{2.427832in}}%
\pgfpathcurveto{\pgfqpoint{3.212425in}{2.423925in}}{\pgfqpoint{3.210230in}{2.418626in}}{\pgfqpoint{3.210230in}{2.413101in}}%
\pgfpathcurveto{\pgfqpoint{3.210230in}{2.407575in}}{\pgfqpoint{3.212425in}{2.402276in}}{\pgfqpoint{3.216332in}{2.398369in}}%
\pgfpathcurveto{\pgfqpoint{3.220239in}{2.394462in}}{\pgfqpoint{3.225539in}{2.392267in}}{\pgfqpoint{3.231064in}{2.392267in}}%
\pgfpathclose%
\pgfusepath{fill}%
\end{pgfscope}%
\begin{pgfscope}%
\pgfsetrectcap%
\pgfsetmiterjoin%
\pgfsetlinewidth{0.803000pt}%
\definecolor{currentstroke}{rgb}{0.000000,0.000000,0.000000}%
\pgfsetstrokecolor{currentstroke}%
\pgfsetdash{}{0pt}%
\pgfpathmoveto{\pgfqpoint{3.158185in}{2.343185in}}%
\pgfpathlineto{\pgfqpoint{3.158185in}{3.098185in}}%
\pgfusepath{stroke}%
\end{pgfscope}%
\begin{pgfscope}%
\pgfsetrectcap%
\pgfsetmiterjoin%
\pgfsetlinewidth{0.803000pt}%
\definecolor{currentstroke}{rgb}{0.000000,0.000000,0.000000}%
\pgfsetstrokecolor{currentstroke}%
\pgfsetdash{}{0pt}%
\pgfpathmoveto{\pgfqpoint{4.320685in}{2.343185in}}%
\pgfpathlineto{\pgfqpoint{4.320685in}{3.098185in}}%
\pgfusepath{stroke}%
\end{pgfscope}%
\begin{pgfscope}%
\pgfsetrectcap%
\pgfsetmiterjoin%
\pgfsetlinewidth{0.803000pt}%
\definecolor{currentstroke}{rgb}{0.000000,0.000000,0.000000}%
\pgfsetstrokecolor{currentstroke}%
\pgfsetdash{}{0pt}%
\pgfpathmoveto{\pgfqpoint{3.158185in}{2.343185in}}%
\pgfpathlineto{\pgfqpoint{4.320685in}{2.343185in}}%
\pgfusepath{stroke}%
\end{pgfscope}%
\begin{pgfscope}%
\pgfsetrectcap%
\pgfsetmiterjoin%
\pgfsetlinewidth{0.803000pt}%
\definecolor{currentstroke}{rgb}{0.000000,0.000000,0.000000}%
\pgfsetstrokecolor{currentstroke}%
\pgfsetdash{}{0pt}%
\pgfpathmoveto{\pgfqpoint{3.158185in}{3.098185in}}%
\pgfpathlineto{\pgfqpoint{4.320685in}{3.098185in}}%
\pgfusepath{stroke}%
\end{pgfscope}%
\begin{pgfscope}%
\pgfsetbuttcap%
\pgfsetmiterjoin%
\definecolor{currentfill}{rgb}{1.000000,1.000000,1.000000}%
\pgfsetfillcolor{currentfill}%
\pgfsetlinewidth{0.000000pt}%
\definecolor{currentstroke}{rgb}{0.000000,0.000000,0.000000}%
\pgfsetstrokecolor{currentstroke}%
\pgfsetstrokeopacity{0.000000}%
\pgfsetdash{}{0pt}%
\pgfpathmoveto{\pgfqpoint{4.320685in}{2.343185in}}%
\pgfpathlineto{\pgfqpoint{5.483185in}{2.343185in}}%
\pgfpathlineto{\pgfqpoint{5.483185in}{3.098185in}}%
\pgfpathlineto{\pgfqpoint{4.320685in}{3.098185in}}%
\pgfpathclose%
\pgfusepath{fill}%
\end{pgfscope}%
\begin{pgfscope}%
\pgfpathrectangle{\pgfqpoint{4.320685in}{2.343185in}}{\pgfqpoint{1.162500in}{0.755000in}} %
\pgfusepath{clip}%
\pgfsetbuttcap%
\pgfsetroundjoin%
\definecolor{currentfill}{rgb}{0.000000,0.000000,0.000000}%
\pgfsetfillcolor{currentfill}%
\pgfsetfillopacity{0.500000}%
\pgfsetlinewidth{0.000000pt}%
\definecolor{currentstroke}{rgb}{0.000000,0.000000,0.000000}%
\pgfsetstrokecolor{currentstroke}%
\pgfsetdash{}{0pt}%
\pgfpathmoveto{\pgfqpoint{4.763543in}{3.059375in}}%
\pgfpathcurveto{\pgfqpoint{4.769068in}{3.059375in}}{\pgfqpoint{4.774367in}{3.061570in}}{\pgfqpoint{4.778274in}{3.065477in}}%
\pgfpathcurveto{\pgfqpoint{4.782181in}{3.069384in}}{\pgfqpoint{4.784376in}{3.074683in}}{\pgfqpoint{4.784376in}{3.080208in}}%
\pgfpathcurveto{\pgfqpoint{4.784376in}{3.085733in}}{\pgfqpoint{4.782181in}{3.091033in}}{\pgfqpoint{4.778274in}{3.094940in}}%
\pgfpathcurveto{\pgfqpoint{4.774367in}{3.098847in}}{\pgfqpoint{4.769068in}{3.101042in}}{\pgfqpoint{4.763543in}{3.101042in}}%
\pgfpathcurveto{\pgfqpoint{4.758018in}{3.101042in}}{\pgfqpoint{4.752718in}{3.098847in}}{\pgfqpoint{4.748811in}{3.094940in}}%
\pgfpathcurveto{\pgfqpoint{4.744905in}{3.091033in}}{\pgfqpoint{4.742709in}{3.085733in}}{\pgfqpoint{4.742709in}{3.080208in}}%
\pgfpathcurveto{\pgfqpoint{4.742709in}{3.074683in}}{\pgfqpoint{4.744905in}{3.069384in}}{\pgfqpoint{4.748811in}{3.065477in}}%
\pgfpathcurveto{\pgfqpoint{4.752718in}{3.061570in}}{\pgfqpoint{4.758018in}{3.059375in}}{\pgfqpoint{4.763543in}{3.059375in}}%
\pgfpathclose%
\pgfusepath{fill}%
\end{pgfscope}%
\begin{pgfscope}%
\pgfpathrectangle{\pgfqpoint{4.320685in}{2.343185in}}{\pgfqpoint{1.162500in}{0.755000in}} %
\pgfusepath{clip}%
\pgfsetbuttcap%
\pgfsetroundjoin%
\definecolor{currentfill}{rgb}{0.000000,0.000000,0.000000}%
\pgfsetfillcolor{currentfill}%
\pgfsetfillopacity{0.500000}%
\pgfsetlinewidth{0.000000pt}%
\definecolor{currentstroke}{rgb}{0.000000,0.000000,0.000000}%
\pgfsetstrokecolor{currentstroke}%
\pgfsetdash{}{0pt}%
\pgfpathmoveto{\pgfqpoint{5.385388in}{2.815854in}}%
\pgfpathcurveto{\pgfqpoint{5.390913in}{2.815854in}}{\pgfqpoint{5.396213in}{2.818049in}}{\pgfqpoint{5.400119in}{2.821955in}}%
\pgfpathcurveto{\pgfqpoint{5.404026in}{2.825862in}}{\pgfqpoint{5.406221in}{2.831162in}}{\pgfqpoint{5.406221in}{2.836687in}}%
\pgfpathcurveto{\pgfqpoint{5.406221in}{2.842212in}}{\pgfqpoint{5.404026in}{2.847511in}}{\pgfqpoint{5.400119in}{2.851418in}}%
\pgfpathcurveto{\pgfqpoint{5.396213in}{2.855325in}}{\pgfqpoint{5.390913in}{2.857520in}}{\pgfqpoint{5.385388in}{2.857520in}}%
\pgfpathcurveto{\pgfqpoint{5.379863in}{2.857520in}}{\pgfqpoint{5.374564in}{2.855325in}}{\pgfqpoint{5.370657in}{2.851418in}}%
\pgfpathcurveto{\pgfqpoint{5.366750in}{2.847511in}}{\pgfqpoint{5.364555in}{2.842212in}}{\pgfqpoint{5.364555in}{2.836687in}}%
\pgfpathcurveto{\pgfqpoint{5.364555in}{2.831162in}}{\pgfqpoint{5.366750in}{2.825862in}}{\pgfqpoint{5.370657in}{2.821955in}}%
\pgfpathcurveto{\pgfqpoint{5.374564in}{2.818049in}}{\pgfqpoint{5.379863in}{2.815854in}}{\pgfqpoint{5.385388in}{2.815854in}}%
\pgfpathclose%
\pgfusepath{fill}%
\end{pgfscope}%
\begin{pgfscope}%
\pgfpathrectangle{\pgfqpoint{4.320685in}{2.343185in}}{\pgfqpoint{1.162500in}{0.755000in}} %
\pgfusepath{clip}%
\pgfsetbuttcap%
\pgfsetroundjoin%
\definecolor{currentfill}{rgb}{0.000000,0.000000,0.000000}%
\pgfsetfillcolor{currentfill}%
\pgfsetfillopacity{0.500000}%
\pgfsetlinewidth{0.000000pt}%
\definecolor{currentstroke}{rgb}{0.000000,0.000000,0.000000}%
\pgfsetstrokecolor{currentstroke}%
\pgfsetdash{}{0pt}%
\pgfpathmoveto{\pgfqpoint{5.455506in}{2.815581in}}%
\pgfpathcurveto{\pgfqpoint{5.461031in}{2.815581in}}{\pgfqpoint{5.466331in}{2.817776in}}{\pgfqpoint{5.470237in}{2.821682in}}%
\pgfpathcurveto{\pgfqpoint{5.474144in}{2.825589in}}{\pgfqpoint{5.476339in}{2.830889in}}{\pgfqpoint{5.476339in}{2.836414in}}%
\pgfpathcurveto{\pgfqpoint{5.476339in}{2.841939in}}{\pgfqpoint{5.474144in}{2.847238in}}{\pgfqpoint{5.470237in}{2.851145in}}%
\pgfpathcurveto{\pgfqpoint{5.466331in}{2.855052in}}{\pgfqpoint{5.461031in}{2.857247in}}{\pgfqpoint{5.455506in}{2.857247in}}%
\pgfpathcurveto{\pgfqpoint{5.449981in}{2.857247in}}{\pgfqpoint{5.444681in}{2.855052in}}{\pgfqpoint{5.440775in}{2.851145in}}%
\pgfpathcurveto{\pgfqpoint{5.436868in}{2.847238in}}{\pgfqpoint{5.434673in}{2.841939in}}{\pgfqpoint{5.434673in}{2.836414in}}%
\pgfpathcurveto{\pgfqpoint{5.434673in}{2.830889in}}{\pgfqpoint{5.436868in}{2.825589in}}{\pgfqpoint{5.440775in}{2.821682in}}%
\pgfpathcurveto{\pgfqpoint{5.444681in}{2.817776in}}{\pgfqpoint{5.449981in}{2.815581in}}{\pgfqpoint{5.455506in}{2.815581in}}%
\pgfpathclose%
\pgfusepath{fill}%
\end{pgfscope}%
\begin{pgfscope}%
\pgfpathrectangle{\pgfqpoint{4.320685in}{2.343185in}}{\pgfqpoint{1.162500in}{0.755000in}} %
\pgfusepath{clip}%
\pgfsetbuttcap%
\pgfsetroundjoin%
\definecolor{currentfill}{rgb}{0.000000,0.000000,0.000000}%
\pgfsetfillcolor{currentfill}%
\pgfsetfillopacity{0.500000}%
\pgfsetlinewidth{0.000000pt}%
\definecolor{currentstroke}{rgb}{0.000000,0.000000,0.000000}%
\pgfsetstrokecolor{currentstroke}%
\pgfsetdash{}{0pt}%
\pgfpathmoveto{\pgfqpoint{5.031128in}{2.531108in}}%
\pgfpathcurveto{\pgfqpoint{5.036653in}{2.531108in}}{\pgfqpoint{5.041953in}{2.533303in}}{\pgfqpoint{5.045859in}{2.537210in}}%
\pgfpathcurveto{\pgfqpoint{5.049766in}{2.541116in}}{\pgfqpoint{5.051961in}{2.546416in}}{\pgfqpoint{5.051961in}{2.551941in}}%
\pgfpathcurveto{\pgfqpoint{5.051961in}{2.557466in}}{\pgfqpoint{5.049766in}{2.562766in}}{\pgfqpoint{5.045859in}{2.566672in}}%
\pgfpathcurveto{\pgfqpoint{5.041953in}{2.570579in}}{\pgfqpoint{5.036653in}{2.572774in}}{\pgfqpoint{5.031128in}{2.572774in}}%
\pgfpathcurveto{\pgfqpoint{5.025603in}{2.572774in}}{\pgfqpoint{5.020303in}{2.570579in}}{\pgfqpoint{5.016397in}{2.566672in}}%
\pgfpathcurveto{\pgfqpoint{5.012490in}{2.562766in}}{\pgfqpoint{5.010295in}{2.557466in}}{\pgfqpoint{5.010295in}{2.551941in}}%
\pgfpathcurveto{\pgfqpoint{5.010295in}{2.546416in}}{\pgfqpoint{5.012490in}{2.541116in}}{\pgfqpoint{5.016397in}{2.537210in}}%
\pgfpathcurveto{\pgfqpoint{5.020303in}{2.533303in}}{\pgfqpoint{5.025603in}{2.531108in}}{\pgfqpoint{5.031128in}{2.531108in}}%
\pgfpathclose%
\pgfusepath{fill}%
\end{pgfscope}%
\begin{pgfscope}%
\pgfpathrectangle{\pgfqpoint{4.320685in}{2.343185in}}{\pgfqpoint{1.162500in}{0.755000in}} %
\pgfusepath{clip}%
\pgfsetbuttcap%
\pgfsetroundjoin%
\definecolor{currentfill}{rgb}{0.000000,0.000000,0.000000}%
\pgfsetfillcolor{currentfill}%
\pgfsetfillopacity{0.500000}%
\pgfsetlinewidth{0.000000pt}%
\definecolor{currentstroke}{rgb}{0.000000,0.000000,0.000000}%
\pgfsetstrokecolor{currentstroke}%
\pgfsetdash{}{0pt}%
\pgfpathmoveto{\pgfqpoint{5.102087in}{2.481860in}}%
\pgfpathcurveto{\pgfqpoint{5.107612in}{2.481860in}}{\pgfqpoint{5.112912in}{2.484055in}}{\pgfqpoint{5.116819in}{2.487962in}}%
\pgfpathcurveto{\pgfqpoint{5.120726in}{2.491868in}}{\pgfqpoint{5.122921in}{2.497168in}}{\pgfqpoint{5.122921in}{2.502693in}}%
\pgfpathcurveto{\pgfqpoint{5.122921in}{2.508218in}}{\pgfqpoint{5.120726in}{2.513517in}}{\pgfqpoint{5.116819in}{2.517424in}}%
\pgfpathcurveto{\pgfqpoint{5.112912in}{2.521331in}}{\pgfqpoint{5.107612in}{2.523526in}}{\pgfqpoint{5.102087in}{2.523526in}}%
\pgfpathcurveto{\pgfqpoint{5.096562in}{2.523526in}}{\pgfqpoint{5.091263in}{2.521331in}}{\pgfqpoint{5.087356in}{2.517424in}}%
\pgfpathcurveto{\pgfqpoint{5.083449in}{2.513517in}}{\pgfqpoint{5.081254in}{2.508218in}}{\pgfqpoint{5.081254in}{2.502693in}}%
\pgfpathcurveto{\pgfqpoint{5.081254in}{2.497168in}}{\pgfqpoint{5.083449in}{2.491868in}}{\pgfqpoint{5.087356in}{2.487962in}}%
\pgfpathcurveto{\pgfqpoint{5.091263in}{2.484055in}}{\pgfqpoint{5.096562in}{2.481860in}}{\pgfqpoint{5.102087in}{2.481860in}}%
\pgfpathclose%
\pgfusepath{fill}%
\end{pgfscope}%
\begin{pgfscope}%
\pgfpathrectangle{\pgfqpoint{4.320685in}{2.343185in}}{\pgfqpoint{1.162500in}{0.755000in}} %
\pgfusepath{clip}%
\pgfsetbuttcap%
\pgfsetroundjoin%
\definecolor{currentfill}{rgb}{0.000000,0.000000,0.000000}%
\pgfsetfillcolor{currentfill}%
\pgfsetfillopacity{0.500000}%
\pgfsetlinewidth{0.000000pt}%
\definecolor{currentstroke}{rgb}{0.000000,0.000000,0.000000}%
\pgfsetstrokecolor{currentstroke}%
\pgfsetdash{}{0pt}%
\pgfpathmoveto{\pgfqpoint{4.940844in}{2.394040in}}%
\pgfpathcurveto{\pgfqpoint{4.946369in}{2.394040in}}{\pgfqpoint{4.951669in}{2.396235in}}{\pgfqpoint{4.955576in}{2.400142in}}%
\pgfpathcurveto{\pgfqpoint{4.959482in}{2.404048in}}{\pgfqpoint{4.961678in}{2.409348in}}{\pgfqpoint{4.961678in}{2.414873in}}%
\pgfpathcurveto{\pgfqpoint{4.961678in}{2.420398in}}{\pgfqpoint{4.959482in}{2.425698in}}{\pgfqpoint{4.955576in}{2.429604in}}%
\pgfpathcurveto{\pgfqpoint{4.951669in}{2.433511in}}{\pgfqpoint{4.946369in}{2.435706in}}{\pgfqpoint{4.940844in}{2.435706in}}%
\pgfpathcurveto{\pgfqpoint{4.935319in}{2.435706in}}{\pgfqpoint{4.930020in}{2.433511in}}{\pgfqpoint{4.926113in}{2.429604in}}%
\pgfpathcurveto{\pgfqpoint{4.922206in}{2.425698in}}{\pgfqpoint{4.920011in}{2.420398in}}{\pgfqpoint{4.920011in}{2.414873in}}%
\pgfpathcurveto{\pgfqpoint{4.920011in}{2.409348in}}{\pgfqpoint{4.922206in}{2.404048in}}{\pgfqpoint{4.926113in}{2.400142in}}%
\pgfpathcurveto{\pgfqpoint{4.930020in}{2.396235in}}{\pgfqpoint{4.935319in}{2.394040in}}{\pgfqpoint{4.940844in}{2.394040in}}%
\pgfpathclose%
\pgfusepath{fill}%
\end{pgfscope}%
\begin{pgfscope}%
\pgfpathrectangle{\pgfqpoint{4.320685in}{2.343185in}}{\pgfqpoint{1.162500in}{0.755000in}} %
\pgfusepath{clip}%
\pgfsetbuttcap%
\pgfsetroundjoin%
\definecolor{currentfill}{rgb}{0.000000,0.000000,0.000000}%
\pgfsetfillcolor{currentfill}%
\pgfsetfillopacity{0.500000}%
\pgfsetlinewidth{0.000000pt}%
\definecolor{currentstroke}{rgb}{0.000000,0.000000,0.000000}%
\pgfsetstrokecolor{currentstroke}%
\pgfsetdash{}{0pt}%
\pgfpathmoveto{\pgfqpoint{4.348363in}{2.340327in}}%
\pgfpathcurveto{\pgfqpoint{4.353888in}{2.340327in}}{\pgfqpoint{4.359188in}{2.342523in}}{\pgfqpoint{4.363095in}{2.346429in}}%
\pgfpathcurveto{\pgfqpoint{4.367001in}{2.350336in}}{\pgfqpoint{4.369196in}{2.355636in}}{\pgfqpoint{4.369196in}{2.361161in}}%
\pgfpathcurveto{\pgfqpoint{4.369196in}{2.366686in}}{\pgfqpoint{4.367001in}{2.371985in}}{\pgfqpoint{4.363095in}{2.375892in}}%
\pgfpathcurveto{\pgfqpoint{4.359188in}{2.379799in}}{\pgfqpoint{4.353888in}{2.381994in}}{\pgfqpoint{4.348363in}{2.381994in}}%
\pgfpathcurveto{\pgfqpoint{4.342838in}{2.381994in}}{\pgfqpoint{4.337539in}{2.379799in}}{\pgfqpoint{4.333632in}{2.375892in}}%
\pgfpathcurveto{\pgfqpoint{4.329725in}{2.371985in}}{\pgfqpoint{4.327530in}{2.366686in}}{\pgfqpoint{4.327530in}{2.361161in}}%
\pgfpathcurveto{\pgfqpoint{4.327530in}{2.355636in}}{\pgfqpoint{4.329725in}{2.350336in}}{\pgfqpoint{4.333632in}{2.346429in}}%
\pgfpathcurveto{\pgfqpoint{4.337539in}{2.342523in}}{\pgfqpoint{4.342838in}{2.340327in}}{\pgfqpoint{4.348363in}{2.340327in}}%
\pgfpathclose%
\pgfusepath{fill}%
\end{pgfscope}%
\begin{pgfscope}%
\pgfpathrectangle{\pgfqpoint{4.320685in}{2.343185in}}{\pgfqpoint{1.162500in}{0.755000in}} %
\pgfusepath{clip}%
\pgfsetbuttcap%
\pgfsetroundjoin%
\definecolor{currentfill}{rgb}{0.000000,0.000000,0.000000}%
\pgfsetfillcolor{currentfill}%
\pgfsetfillopacity{0.500000}%
\pgfsetlinewidth{0.000000pt}%
\definecolor{currentstroke}{rgb}{0.000000,0.000000,0.000000}%
\pgfsetstrokecolor{currentstroke}%
\pgfsetdash{}{0pt}%
\pgfpathmoveto{\pgfqpoint{5.332823in}{2.914639in}}%
\pgfpathcurveto{\pgfqpoint{5.338348in}{2.914639in}}{\pgfqpoint{5.343648in}{2.916834in}}{\pgfqpoint{5.347555in}{2.920741in}}%
\pgfpathcurveto{\pgfqpoint{5.351461in}{2.924648in}}{\pgfqpoint{5.353656in}{2.929947in}}{\pgfqpoint{5.353656in}{2.935473in}}%
\pgfpathcurveto{\pgfqpoint{5.353656in}{2.940998in}}{\pgfqpoint{5.351461in}{2.946297in}}{\pgfqpoint{5.347555in}{2.950204in}}%
\pgfpathcurveto{\pgfqpoint{5.343648in}{2.954111in}}{\pgfqpoint{5.338348in}{2.956306in}}{\pgfqpoint{5.332823in}{2.956306in}}%
\pgfpathcurveto{\pgfqpoint{5.327298in}{2.956306in}}{\pgfqpoint{5.321999in}{2.954111in}}{\pgfqpoint{5.318092in}{2.950204in}}%
\pgfpathcurveto{\pgfqpoint{5.314185in}{2.946297in}}{\pgfqpoint{5.311990in}{2.940998in}}{\pgfqpoint{5.311990in}{2.935473in}}%
\pgfpathcurveto{\pgfqpoint{5.311990in}{2.929947in}}{\pgfqpoint{5.314185in}{2.924648in}}{\pgfqpoint{5.318092in}{2.920741in}}%
\pgfpathcurveto{\pgfqpoint{5.321999in}{2.916834in}}{\pgfqpoint{5.327298in}{2.914639in}}{\pgfqpoint{5.332823in}{2.914639in}}%
\pgfpathclose%
\pgfusepath{fill}%
\end{pgfscope}%
\begin{pgfscope}%
\pgfpathrectangle{\pgfqpoint{4.320685in}{2.343185in}}{\pgfqpoint{1.162500in}{0.755000in}} %
\pgfusepath{clip}%
\pgfsetbuttcap%
\pgfsetroundjoin%
\definecolor{currentfill}{rgb}{0.000000,0.000000,0.000000}%
\pgfsetfillcolor{currentfill}%
\pgfsetfillopacity{0.500000}%
\pgfsetlinewidth{0.000000pt}%
\definecolor{currentstroke}{rgb}{0.000000,0.000000,0.000000}%
\pgfsetstrokecolor{currentstroke}%
\pgfsetdash{}{0pt}%
\pgfpathmoveto{\pgfqpoint{5.090412in}{2.734550in}}%
\pgfpathcurveto{\pgfqpoint{5.095937in}{2.734550in}}{\pgfqpoint{5.101237in}{2.736745in}}{\pgfqpoint{5.105143in}{2.740652in}}%
\pgfpathcurveto{\pgfqpoint{5.109050in}{2.744559in}}{\pgfqpoint{5.111245in}{2.749858in}}{\pgfqpoint{5.111245in}{2.755383in}}%
\pgfpathcurveto{\pgfqpoint{5.111245in}{2.760908in}}{\pgfqpoint{5.109050in}{2.766208in}}{\pgfqpoint{5.105143in}{2.770115in}}%
\pgfpathcurveto{\pgfqpoint{5.101237in}{2.774021in}}{\pgfqpoint{5.095937in}{2.776217in}}{\pgfqpoint{5.090412in}{2.776217in}}%
\pgfpathcurveto{\pgfqpoint{5.084887in}{2.776217in}}{\pgfqpoint{5.079587in}{2.774021in}}{\pgfqpoint{5.075681in}{2.770115in}}%
\pgfpathcurveto{\pgfqpoint{5.071774in}{2.766208in}}{\pgfqpoint{5.069579in}{2.760908in}}{\pgfqpoint{5.069579in}{2.755383in}}%
\pgfpathcurveto{\pgfqpoint{5.069579in}{2.749858in}}{\pgfqpoint{5.071774in}{2.744559in}}{\pgfqpoint{5.075681in}{2.740652in}}%
\pgfpathcurveto{\pgfqpoint{5.079587in}{2.736745in}}{\pgfqpoint{5.084887in}{2.734550in}}{\pgfqpoint{5.090412in}{2.734550in}}%
\pgfpathclose%
\pgfusepath{fill}%
\end{pgfscope}%
\begin{pgfscope}%
\pgfpathrectangle{\pgfqpoint{4.320685in}{2.343185in}}{\pgfqpoint{1.162500in}{0.755000in}} %
\pgfusepath{clip}%
\pgfsetbuttcap%
\pgfsetroundjoin%
\definecolor{currentfill}{rgb}{0.000000,0.000000,0.000000}%
\pgfsetfillcolor{currentfill}%
\pgfsetfillopacity{0.500000}%
\pgfsetlinewidth{0.000000pt}%
\definecolor{currentstroke}{rgb}{0.000000,0.000000,0.000000}%
\pgfsetstrokecolor{currentstroke}%
\pgfsetdash{}{0pt}%
\pgfpathmoveto{\pgfqpoint{4.817426in}{2.641930in}}%
\pgfpathcurveto{\pgfqpoint{4.822951in}{2.641930in}}{\pgfqpoint{4.828251in}{2.644125in}}{\pgfqpoint{4.832158in}{2.648032in}}%
\pgfpathcurveto{\pgfqpoint{4.836065in}{2.651939in}}{\pgfqpoint{4.838260in}{2.657239in}}{\pgfqpoint{4.838260in}{2.662764in}}%
\pgfpathcurveto{\pgfqpoint{4.838260in}{2.668289in}}{\pgfqpoint{4.836065in}{2.673588in}}{\pgfqpoint{4.832158in}{2.677495in}}%
\pgfpathcurveto{\pgfqpoint{4.828251in}{2.681402in}}{\pgfqpoint{4.822951in}{2.683597in}}{\pgfqpoint{4.817426in}{2.683597in}}%
\pgfpathcurveto{\pgfqpoint{4.811901in}{2.683597in}}{\pgfqpoint{4.806602in}{2.681402in}}{\pgfqpoint{4.802695in}{2.677495in}}%
\pgfpathcurveto{\pgfqpoint{4.798788in}{2.673588in}}{\pgfqpoint{4.796593in}{2.668289in}}{\pgfqpoint{4.796593in}{2.662764in}}%
\pgfpathcurveto{\pgfqpoint{4.796593in}{2.657239in}}{\pgfqpoint{4.798788in}{2.651939in}}{\pgfqpoint{4.802695in}{2.648032in}}%
\pgfpathcurveto{\pgfqpoint{4.806602in}{2.644125in}}{\pgfqpoint{4.811901in}{2.641930in}}{\pgfqpoint{4.817426in}{2.641930in}}%
\pgfpathclose%
\pgfusepath{fill}%
\end{pgfscope}%
\begin{pgfscope}%
\pgfpathrectangle{\pgfqpoint{4.320685in}{2.343185in}}{\pgfqpoint{1.162500in}{0.755000in}} %
\pgfusepath{clip}%
\pgfsetbuttcap%
\pgfsetroundjoin%
\definecolor{currentfill}{rgb}{0.000000,0.000000,0.000000}%
\pgfsetfillcolor{currentfill}%
\pgfsetfillopacity{0.500000}%
\pgfsetlinewidth{0.000000pt}%
\definecolor{currentstroke}{rgb}{0.000000,0.000000,0.000000}%
\pgfsetstrokecolor{currentstroke}%
\pgfsetdash{}{0pt}%
\pgfpathmoveto{\pgfqpoint{5.258215in}{2.975096in}}%
\pgfpathcurveto{\pgfqpoint{5.263740in}{2.975096in}}{\pgfqpoint{5.269040in}{2.977291in}}{\pgfqpoint{5.272947in}{2.981198in}}%
\pgfpathcurveto{\pgfqpoint{5.276853in}{2.985104in}}{\pgfqpoint{5.279048in}{2.990404in}}{\pgfqpoint{5.279048in}{2.995929in}}%
\pgfpathcurveto{\pgfqpoint{5.279048in}{3.001454in}}{\pgfqpoint{5.276853in}{3.006754in}}{\pgfqpoint{5.272947in}{3.010660in}}%
\pgfpathcurveto{\pgfqpoint{5.269040in}{3.014567in}}{\pgfqpoint{5.263740in}{3.016762in}}{\pgfqpoint{5.258215in}{3.016762in}}%
\pgfpathcurveto{\pgfqpoint{5.252690in}{3.016762in}}{\pgfqpoint{5.247391in}{3.014567in}}{\pgfqpoint{5.243484in}{3.010660in}}%
\pgfpathcurveto{\pgfqpoint{5.239577in}{3.006754in}}{\pgfqpoint{5.237382in}{3.001454in}}{\pgfqpoint{5.237382in}{2.995929in}}%
\pgfpathcurveto{\pgfqpoint{5.237382in}{2.990404in}}{\pgfqpoint{5.239577in}{2.985104in}}{\pgfqpoint{5.243484in}{2.981198in}}%
\pgfpathcurveto{\pgfqpoint{5.247391in}{2.977291in}}{\pgfqpoint{5.252690in}{2.975096in}}{\pgfqpoint{5.258215in}{2.975096in}}%
\pgfpathclose%
\pgfusepath{fill}%
\end{pgfscope}%
\begin{pgfscope}%
\pgfpathrectangle{\pgfqpoint{4.320685in}{2.343185in}}{\pgfqpoint{1.162500in}{0.755000in}} %
\pgfusepath{clip}%
\pgfsetbuttcap%
\pgfsetroundjoin%
\definecolor{currentfill}{rgb}{0.000000,0.000000,0.000000}%
\pgfsetfillcolor{currentfill}%
\pgfsetfillopacity{0.500000}%
\pgfsetlinewidth{0.000000pt}%
\definecolor{currentstroke}{rgb}{0.000000,0.000000,0.000000}%
\pgfsetstrokecolor{currentstroke}%
\pgfsetdash{}{0pt}%
\pgfpathmoveto{\pgfqpoint{4.803827in}{2.531911in}}%
\pgfpathcurveto{\pgfqpoint{4.809352in}{2.531911in}}{\pgfqpoint{4.814651in}{2.534106in}}{\pgfqpoint{4.818558in}{2.538013in}}%
\pgfpathcurveto{\pgfqpoint{4.822465in}{2.541920in}}{\pgfqpoint{4.824660in}{2.547220in}}{\pgfqpoint{4.824660in}{2.552745in}}%
\pgfpathcurveto{\pgfqpoint{4.824660in}{2.558270in}}{\pgfqpoint{4.822465in}{2.563569in}}{\pgfqpoint{4.818558in}{2.567476in}}%
\pgfpathcurveto{\pgfqpoint{4.814651in}{2.571383in}}{\pgfqpoint{4.809352in}{2.573578in}}{\pgfqpoint{4.803827in}{2.573578in}}%
\pgfpathcurveto{\pgfqpoint{4.798302in}{2.573578in}}{\pgfqpoint{4.793002in}{2.571383in}}{\pgfqpoint{4.789096in}{2.567476in}}%
\pgfpathcurveto{\pgfqpoint{4.785189in}{2.563569in}}{\pgfqpoint{4.782994in}{2.558270in}}{\pgfqpoint{4.782994in}{2.552745in}}%
\pgfpathcurveto{\pgfqpoint{4.782994in}{2.547220in}}{\pgfqpoint{4.785189in}{2.541920in}}{\pgfqpoint{4.789096in}{2.538013in}}%
\pgfpathcurveto{\pgfqpoint{4.793002in}{2.534106in}}{\pgfqpoint{4.798302in}{2.531911in}}{\pgfqpoint{4.803827in}{2.531911in}}%
\pgfpathclose%
\pgfusepath{fill}%
\end{pgfscope}%
\begin{pgfscope}%
\pgfpathrectangle{\pgfqpoint{4.320685in}{2.343185in}}{\pgfqpoint{1.162500in}{0.755000in}} %
\pgfusepath{clip}%
\pgfsetbuttcap%
\pgfsetroundjoin%
\definecolor{currentfill}{rgb}{0.000000,0.000000,0.000000}%
\pgfsetfillcolor{currentfill}%
\pgfsetfillopacity{0.500000}%
\pgfsetlinewidth{0.000000pt}%
\definecolor{currentstroke}{rgb}{0.000000,0.000000,0.000000}%
\pgfsetstrokecolor{currentstroke}%
\pgfsetdash{}{0pt}%
\pgfpathmoveto{\pgfqpoint{4.437194in}{2.591601in}}%
\pgfpathcurveto{\pgfqpoint{4.442719in}{2.591601in}}{\pgfqpoint{4.448018in}{2.593796in}}{\pgfqpoint{4.451925in}{2.597703in}}%
\pgfpathcurveto{\pgfqpoint{4.455832in}{2.601609in}}{\pgfqpoint{4.458027in}{2.606909in}}{\pgfqpoint{4.458027in}{2.612434in}}%
\pgfpathcurveto{\pgfqpoint{4.458027in}{2.617959in}}{\pgfqpoint{4.455832in}{2.623259in}}{\pgfqpoint{4.451925in}{2.627165in}}%
\pgfpathcurveto{\pgfqpoint{4.448018in}{2.631072in}}{\pgfqpoint{4.442719in}{2.633267in}}{\pgfqpoint{4.437194in}{2.633267in}}%
\pgfpathcurveto{\pgfqpoint{4.431669in}{2.633267in}}{\pgfqpoint{4.426369in}{2.631072in}}{\pgfqpoint{4.422462in}{2.627165in}}%
\pgfpathcurveto{\pgfqpoint{4.418556in}{2.623259in}}{\pgfqpoint{4.416360in}{2.617959in}}{\pgfqpoint{4.416360in}{2.612434in}}%
\pgfpathcurveto{\pgfqpoint{4.416360in}{2.606909in}}{\pgfqpoint{4.418556in}{2.601609in}}{\pgfqpoint{4.422462in}{2.597703in}}%
\pgfpathcurveto{\pgfqpoint{4.426369in}{2.593796in}}{\pgfqpoint{4.431669in}{2.591601in}}{\pgfqpoint{4.437194in}{2.591601in}}%
\pgfpathclose%
\pgfusepath{fill}%
\end{pgfscope}%
\begin{pgfscope}%
\pgfpathrectangle{\pgfqpoint{4.320685in}{2.343185in}}{\pgfqpoint{1.162500in}{0.755000in}} %
\pgfusepath{clip}%
\pgfsetbuttcap%
\pgfsetroundjoin%
\definecolor{currentfill}{rgb}{0.000000,0.000000,0.000000}%
\pgfsetfillcolor{currentfill}%
\pgfsetfillopacity{0.500000}%
\pgfsetlinewidth{0.000000pt}%
\definecolor{currentstroke}{rgb}{0.000000,0.000000,0.000000}%
\pgfsetstrokecolor{currentstroke}%
\pgfsetdash{}{0pt}%
\pgfpathmoveto{\pgfqpoint{5.262967in}{2.814070in}}%
\pgfpathcurveto{\pgfqpoint{5.268492in}{2.814070in}}{\pgfqpoint{5.273791in}{2.816265in}}{\pgfqpoint{5.277698in}{2.820172in}}%
\pgfpathcurveto{\pgfqpoint{5.281605in}{2.824079in}}{\pgfqpoint{5.283800in}{2.829378in}}{\pgfqpoint{5.283800in}{2.834904in}}%
\pgfpathcurveto{\pgfqpoint{5.283800in}{2.840429in}}{\pgfqpoint{5.281605in}{2.845728in}}{\pgfqpoint{5.277698in}{2.849635in}}%
\pgfpathcurveto{\pgfqpoint{5.273791in}{2.853542in}}{\pgfqpoint{5.268492in}{2.855737in}}{\pgfqpoint{5.262967in}{2.855737in}}%
\pgfpathcurveto{\pgfqpoint{5.257442in}{2.855737in}}{\pgfqpoint{5.252142in}{2.853542in}}{\pgfqpoint{5.248235in}{2.849635in}}%
\pgfpathcurveto{\pgfqpoint{5.244329in}{2.845728in}}{\pgfqpoint{5.242134in}{2.840429in}}{\pgfqpoint{5.242134in}{2.834904in}}%
\pgfpathcurveto{\pgfqpoint{5.242134in}{2.829378in}}{\pgfqpoint{5.244329in}{2.824079in}}{\pgfqpoint{5.248235in}{2.820172in}}%
\pgfpathcurveto{\pgfqpoint{5.252142in}{2.816265in}}{\pgfqpoint{5.257442in}{2.814070in}}{\pgfqpoint{5.262967in}{2.814070in}}%
\pgfpathclose%
\pgfusepath{fill}%
\end{pgfscope}%
\begin{pgfscope}%
\pgfpathrectangle{\pgfqpoint{4.320685in}{2.343185in}}{\pgfqpoint{1.162500in}{0.755000in}} %
\pgfusepath{clip}%
\pgfsetbuttcap%
\pgfsetroundjoin%
\definecolor{currentfill}{rgb}{0.000000,0.000000,0.000000}%
\pgfsetfillcolor{currentfill}%
\pgfsetfillopacity{0.500000}%
\pgfsetlinewidth{0.000000pt}%
\definecolor{currentstroke}{rgb}{0.000000,0.000000,0.000000}%
\pgfsetstrokecolor{currentstroke}%
\pgfsetdash{}{0pt}%
\pgfpathmoveto{\pgfqpoint{4.896922in}{2.392267in}}%
\pgfpathcurveto{\pgfqpoint{4.902447in}{2.392267in}}{\pgfqpoint{4.907746in}{2.394462in}}{\pgfqpoint{4.911653in}{2.398369in}}%
\pgfpathcurveto{\pgfqpoint{4.915560in}{2.402276in}}{\pgfqpoint{4.917755in}{2.407575in}}{\pgfqpoint{4.917755in}{2.413101in}}%
\pgfpathcurveto{\pgfqpoint{4.917755in}{2.418626in}}{\pgfqpoint{4.915560in}{2.423925in}}{\pgfqpoint{4.911653in}{2.427832in}}%
\pgfpathcurveto{\pgfqpoint{4.907746in}{2.431739in}}{\pgfqpoint{4.902447in}{2.433934in}}{\pgfqpoint{4.896922in}{2.433934in}}%
\pgfpathcurveto{\pgfqpoint{4.891397in}{2.433934in}}{\pgfqpoint{4.886097in}{2.431739in}}{\pgfqpoint{4.882190in}{2.427832in}}%
\pgfpathcurveto{\pgfqpoint{4.878284in}{2.423925in}}{\pgfqpoint{4.876088in}{2.418626in}}{\pgfqpoint{4.876088in}{2.413101in}}%
\pgfpathcurveto{\pgfqpoint{4.876088in}{2.407575in}}{\pgfqpoint{4.878284in}{2.402276in}}{\pgfqpoint{4.882190in}{2.398369in}}%
\pgfpathcurveto{\pgfqpoint{4.886097in}{2.394462in}}{\pgfqpoint{4.891397in}{2.392267in}}{\pgfqpoint{4.896922in}{2.392267in}}%
\pgfpathclose%
\pgfusepath{fill}%
\end{pgfscope}%
\begin{pgfscope}%
\pgfsetrectcap%
\pgfsetmiterjoin%
\pgfsetlinewidth{0.803000pt}%
\definecolor{currentstroke}{rgb}{0.000000,0.000000,0.000000}%
\pgfsetstrokecolor{currentstroke}%
\pgfsetdash{}{0pt}%
\pgfpathmoveto{\pgfqpoint{4.320685in}{2.343185in}}%
\pgfpathlineto{\pgfqpoint{4.320685in}{3.098185in}}%
\pgfusepath{stroke}%
\end{pgfscope}%
\begin{pgfscope}%
\pgfsetrectcap%
\pgfsetmiterjoin%
\pgfsetlinewidth{0.803000pt}%
\definecolor{currentstroke}{rgb}{0.000000,0.000000,0.000000}%
\pgfsetstrokecolor{currentstroke}%
\pgfsetdash{}{0pt}%
\pgfpathmoveto{\pgfqpoint{5.483185in}{2.343185in}}%
\pgfpathlineto{\pgfqpoint{5.483185in}{3.098185in}}%
\pgfusepath{stroke}%
\end{pgfscope}%
\begin{pgfscope}%
\pgfsetrectcap%
\pgfsetmiterjoin%
\pgfsetlinewidth{0.803000pt}%
\definecolor{currentstroke}{rgb}{0.000000,0.000000,0.000000}%
\pgfsetstrokecolor{currentstroke}%
\pgfsetdash{}{0pt}%
\pgfpathmoveto{\pgfqpoint{4.320685in}{2.343185in}}%
\pgfpathlineto{\pgfqpoint{5.483185in}{2.343185in}}%
\pgfusepath{stroke}%
\end{pgfscope}%
\begin{pgfscope}%
\pgfsetrectcap%
\pgfsetmiterjoin%
\pgfsetlinewidth{0.803000pt}%
\definecolor{currentstroke}{rgb}{0.000000,0.000000,0.000000}%
\pgfsetstrokecolor{currentstroke}%
\pgfsetdash{}{0pt}%
\pgfpathmoveto{\pgfqpoint{4.320685in}{3.098185in}}%
\pgfpathlineto{\pgfqpoint{5.483185in}{3.098185in}}%
\pgfusepath{stroke}%
\end{pgfscope}%
\begin{pgfscope}%
\pgfsetbuttcap%
\pgfsetmiterjoin%
\definecolor{currentfill}{rgb}{1.000000,1.000000,1.000000}%
\pgfsetfillcolor{currentfill}%
\pgfsetlinewidth{0.000000pt}%
\definecolor{currentstroke}{rgb}{0.000000,0.000000,0.000000}%
\pgfsetstrokecolor{currentstroke}%
\pgfsetstrokeopacity{0.000000}%
\pgfsetdash{}{0pt}%
\pgfpathmoveto{\pgfqpoint{0.833185in}{1.588185in}}%
\pgfpathlineto{\pgfqpoint{1.995685in}{1.588185in}}%
\pgfpathlineto{\pgfqpoint{1.995685in}{2.343185in}}%
\pgfpathlineto{\pgfqpoint{0.833185in}{2.343185in}}%
\pgfpathclose%
\pgfusepath{fill}%
\end{pgfscope}%
\begin{pgfscope}%
\pgfpathrectangle{\pgfqpoint{0.833185in}{1.588185in}}{\pgfqpoint{1.162500in}{0.755000in}} %
\pgfusepath{clip}%
\pgfsetbuttcap%
\pgfsetroundjoin%
\definecolor{currentfill}{rgb}{0.000000,0.000000,0.000000}%
\pgfsetfillcolor{currentfill}%
\pgfsetfillopacity{0.500000}%
\pgfsetlinewidth{0.000000pt}%
\definecolor{currentstroke}{rgb}{0.000000,0.000000,0.000000}%
\pgfsetstrokecolor{currentstroke}%
\pgfsetdash{}{0pt}%
\pgfpathmoveto{\pgfqpoint{1.779018in}{2.304375in}}%
\pgfpathcurveto{\pgfqpoint{1.784543in}{2.304375in}}{\pgfqpoint{1.789842in}{2.306570in}}{\pgfqpoint{1.793749in}{2.310477in}}%
\pgfpathcurveto{\pgfqpoint{1.797656in}{2.314384in}}{\pgfqpoint{1.799851in}{2.319683in}}{\pgfqpoint{1.799851in}{2.325208in}}%
\pgfpathcurveto{\pgfqpoint{1.799851in}{2.330733in}}{\pgfqpoint{1.797656in}{2.336033in}}{\pgfqpoint{1.793749in}{2.339940in}}%
\pgfpathcurveto{\pgfqpoint{1.789842in}{2.343847in}}{\pgfqpoint{1.784543in}{2.346042in}}{\pgfqpoint{1.779018in}{2.346042in}}%
\pgfpathcurveto{\pgfqpoint{1.773492in}{2.346042in}}{\pgfqpoint{1.768193in}{2.343847in}}{\pgfqpoint{1.764286in}{2.339940in}}%
\pgfpathcurveto{\pgfqpoint{1.760379in}{2.336033in}}{\pgfqpoint{1.758184in}{2.330733in}}{\pgfqpoint{1.758184in}{2.325208in}}%
\pgfpathcurveto{\pgfqpoint{1.758184in}{2.319683in}}{\pgfqpoint{1.760379in}{2.314384in}}{\pgfqpoint{1.764286in}{2.310477in}}%
\pgfpathcurveto{\pgfqpoint{1.768193in}{2.306570in}}{\pgfqpoint{1.773492in}{2.304375in}}{\pgfqpoint{1.779018in}{2.304375in}}%
\pgfpathclose%
\pgfusepath{fill}%
\end{pgfscope}%
\begin{pgfscope}%
\pgfpathrectangle{\pgfqpoint{0.833185in}{1.588185in}}{\pgfqpoint{1.162500in}{0.755000in}} %
\pgfusepath{clip}%
\pgfsetbuttcap%
\pgfsetroundjoin%
\definecolor{currentfill}{rgb}{0.000000,0.000000,0.000000}%
\pgfsetfillcolor{currentfill}%
\pgfsetfillopacity{0.500000}%
\pgfsetlinewidth{0.000000pt}%
\definecolor{currentstroke}{rgb}{0.000000,0.000000,0.000000}%
\pgfsetstrokecolor{currentstroke}%
\pgfsetdash{}{0pt}%
\pgfpathmoveto{\pgfqpoint{0.920858in}{1.845741in}}%
\pgfpathcurveto{\pgfqpoint{0.926383in}{1.845741in}}{\pgfqpoint{0.931683in}{1.847936in}}{\pgfqpoint{0.935589in}{1.851843in}}%
\pgfpathcurveto{\pgfqpoint{0.939496in}{1.855750in}}{\pgfqpoint{0.941691in}{1.861049in}}{\pgfqpoint{0.941691in}{1.866574in}}%
\pgfpathcurveto{\pgfqpoint{0.941691in}{1.872099in}}{\pgfqpoint{0.939496in}{1.877399in}}{\pgfqpoint{0.935589in}{1.881306in}}%
\pgfpathcurveto{\pgfqpoint{0.931683in}{1.885212in}}{\pgfqpoint{0.926383in}{1.887408in}}{\pgfqpoint{0.920858in}{1.887408in}}%
\pgfpathcurveto{\pgfqpoint{0.915333in}{1.887408in}}{\pgfqpoint{0.910034in}{1.885212in}}{\pgfqpoint{0.906127in}{1.881306in}}%
\pgfpathcurveto{\pgfqpoint{0.902220in}{1.877399in}}{\pgfqpoint{0.900025in}{1.872099in}}{\pgfqpoint{0.900025in}{1.866574in}}%
\pgfpathcurveto{\pgfqpoint{0.900025in}{1.861049in}}{\pgfqpoint{0.902220in}{1.855750in}}{\pgfqpoint{0.906127in}{1.851843in}}%
\pgfpathcurveto{\pgfqpoint{0.910034in}{1.847936in}}{\pgfqpoint{0.915333in}{1.845741in}}{\pgfqpoint{0.920858in}{1.845741in}}%
\pgfpathclose%
\pgfusepath{fill}%
\end{pgfscope}%
\begin{pgfscope}%
\pgfpathrectangle{\pgfqpoint{0.833185in}{1.588185in}}{\pgfqpoint{1.162500in}{0.755000in}} %
\pgfusepath{clip}%
\pgfsetbuttcap%
\pgfsetroundjoin%
\definecolor{currentfill}{rgb}{0.000000,0.000000,0.000000}%
\pgfsetfillcolor{currentfill}%
\pgfsetfillopacity{0.500000}%
\pgfsetlinewidth{0.000000pt}%
\definecolor{currentstroke}{rgb}{0.000000,0.000000,0.000000}%
\pgfsetstrokecolor{currentstroke}%
\pgfsetdash{}{0pt}%
\pgfpathmoveto{\pgfqpoint{0.860863in}{1.840570in}}%
\pgfpathcurveto{\pgfqpoint{0.866388in}{1.840570in}}{\pgfqpoint{0.871688in}{1.842766in}}{\pgfqpoint{0.875595in}{1.846672in}}%
\pgfpathcurveto{\pgfqpoint{0.879501in}{1.850579in}}{\pgfqpoint{0.881696in}{1.855879in}}{\pgfqpoint{0.881696in}{1.861404in}}%
\pgfpathcurveto{\pgfqpoint{0.881696in}{1.866929in}}{\pgfqpoint{0.879501in}{1.872228in}}{\pgfqpoint{0.875595in}{1.876135in}}%
\pgfpathcurveto{\pgfqpoint{0.871688in}{1.880042in}}{\pgfqpoint{0.866388in}{1.882237in}}{\pgfqpoint{0.860863in}{1.882237in}}%
\pgfpathcurveto{\pgfqpoint{0.855338in}{1.882237in}}{\pgfqpoint{0.850039in}{1.880042in}}{\pgfqpoint{0.846132in}{1.876135in}}%
\pgfpathcurveto{\pgfqpoint{0.842225in}{1.872228in}}{\pgfqpoint{0.840030in}{1.866929in}}{\pgfqpoint{0.840030in}{1.861404in}}%
\pgfpathcurveto{\pgfqpoint{0.840030in}{1.855879in}}{\pgfqpoint{0.842225in}{1.850579in}}{\pgfqpoint{0.846132in}{1.846672in}}%
\pgfpathcurveto{\pgfqpoint{0.850039in}{1.842766in}}{\pgfqpoint{0.855338in}{1.840570in}}{\pgfqpoint{0.860863in}{1.840570in}}%
\pgfpathclose%
\pgfusepath{fill}%
\end{pgfscope}%
\begin{pgfscope}%
\pgfpathrectangle{\pgfqpoint{0.833185in}{1.588185in}}{\pgfqpoint{1.162500in}{0.755000in}} %
\pgfusepath{clip}%
\pgfsetbuttcap%
\pgfsetroundjoin%
\definecolor{currentfill}{rgb}{0.000000,0.000000,0.000000}%
\pgfsetfillcolor{currentfill}%
\pgfsetfillopacity{0.500000}%
\pgfsetlinewidth{0.000000pt}%
\definecolor{currentstroke}{rgb}{0.000000,0.000000,0.000000}%
\pgfsetstrokecolor{currentstroke}%
\pgfsetdash{}{0pt}%
\pgfpathmoveto{\pgfqpoint{1.180359in}{1.680932in}}%
\pgfpathcurveto{\pgfqpoint{1.185884in}{1.680932in}}{\pgfqpoint{1.191184in}{1.683127in}}{\pgfqpoint{1.195090in}{1.687034in}}%
\pgfpathcurveto{\pgfqpoint{1.198997in}{1.690941in}}{\pgfqpoint{1.201192in}{1.696241in}}{\pgfqpoint{1.201192in}{1.701766in}}%
\pgfpathcurveto{\pgfqpoint{1.201192in}{1.707291in}}{\pgfqpoint{1.198997in}{1.712590in}}{\pgfqpoint{1.195090in}{1.716497in}}%
\pgfpathcurveto{\pgfqpoint{1.191184in}{1.720404in}}{\pgfqpoint{1.185884in}{1.722599in}}{\pgfqpoint{1.180359in}{1.722599in}}%
\pgfpathcurveto{\pgfqpoint{1.174834in}{1.722599in}}{\pgfqpoint{1.169535in}{1.720404in}}{\pgfqpoint{1.165628in}{1.716497in}}%
\pgfpathcurveto{\pgfqpoint{1.161721in}{1.712590in}}{\pgfqpoint{1.159526in}{1.707291in}}{\pgfqpoint{1.159526in}{1.701766in}}%
\pgfpathcurveto{\pgfqpoint{1.159526in}{1.696241in}}{\pgfqpoint{1.161721in}{1.690941in}}{\pgfqpoint{1.165628in}{1.687034in}}%
\pgfpathcurveto{\pgfqpoint{1.169535in}{1.683127in}}{\pgfqpoint{1.174834in}{1.680932in}}{\pgfqpoint{1.180359in}{1.680932in}}%
\pgfpathclose%
\pgfusepath{fill}%
\end{pgfscope}%
\begin{pgfscope}%
\pgfpathrectangle{\pgfqpoint{0.833185in}{1.588185in}}{\pgfqpoint{1.162500in}{0.755000in}} %
\pgfusepath{clip}%
\pgfsetbuttcap%
\pgfsetroundjoin%
\definecolor{currentfill}{rgb}{0.000000,0.000000,0.000000}%
\pgfsetfillcolor{currentfill}%
\pgfsetfillopacity{0.500000}%
\pgfsetlinewidth{0.000000pt}%
\definecolor{currentstroke}{rgb}{0.000000,0.000000,0.000000}%
\pgfsetstrokecolor{currentstroke}%
\pgfsetdash{}{0pt}%
\pgfpathmoveto{\pgfqpoint{0.938297in}{1.656866in}}%
\pgfpathcurveto{\pgfqpoint{0.943822in}{1.656866in}}{\pgfqpoint{0.949121in}{1.659061in}}{\pgfqpoint{0.953028in}{1.662968in}}%
\pgfpathcurveto{\pgfqpoint{0.956935in}{1.666874in}}{\pgfqpoint{0.959130in}{1.672174in}}{\pgfqpoint{0.959130in}{1.677699in}}%
\pgfpathcurveto{\pgfqpoint{0.959130in}{1.683224in}}{\pgfqpoint{0.956935in}{1.688523in}}{\pgfqpoint{0.953028in}{1.692430in}}%
\pgfpathcurveto{\pgfqpoint{0.949121in}{1.696337in}}{\pgfqpoint{0.943822in}{1.698532in}}{\pgfqpoint{0.938297in}{1.698532in}}%
\pgfpathcurveto{\pgfqpoint{0.932772in}{1.698532in}}{\pgfqpoint{0.927472in}{1.696337in}}{\pgfqpoint{0.923565in}{1.692430in}}%
\pgfpathcurveto{\pgfqpoint{0.919659in}{1.688523in}}{\pgfqpoint{0.917464in}{1.683224in}}{\pgfqpoint{0.917464in}{1.677699in}}%
\pgfpathcurveto{\pgfqpoint{0.917464in}{1.672174in}}{\pgfqpoint{0.919659in}{1.666874in}}{\pgfqpoint{0.923565in}{1.662968in}}%
\pgfpathcurveto{\pgfqpoint{0.927472in}{1.659061in}}{\pgfqpoint{0.932772in}{1.656866in}}{\pgfqpoint{0.938297in}{1.656866in}}%
\pgfpathclose%
\pgfusepath{fill}%
\end{pgfscope}%
\begin{pgfscope}%
\pgfpathrectangle{\pgfqpoint{0.833185in}{1.588185in}}{\pgfqpoint{1.162500in}{0.755000in}} %
\pgfusepath{clip}%
\pgfsetbuttcap%
\pgfsetroundjoin%
\definecolor{currentfill}{rgb}{0.000000,0.000000,0.000000}%
\pgfsetfillcolor{currentfill}%
\pgfsetfillopacity{0.500000}%
\pgfsetlinewidth{0.000000pt}%
\definecolor{currentstroke}{rgb}{0.000000,0.000000,0.000000}%
\pgfsetstrokecolor{currentstroke}%
\pgfsetdash{}{0pt}%
\pgfpathmoveto{\pgfqpoint{1.098900in}{1.598883in}}%
\pgfpathcurveto{\pgfqpoint{1.104426in}{1.598883in}}{\pgfqpoint{1.109725in}{1.601078in}}{\pgfqpoint{1.113632in}{1.604985in}}%
\pgfpathcurveto{\pgfqpoint{1.117539in}{1.608892in}}{\pgfqpoint{1.119734in}{1.614191in}}{\pgfqpoint{1.119734in}{1.619716in}}%
\pgfpathcurveto{\pgfqpoint{1.119734in}{1.625241in}}{\pgfqpoint{1.117539in}{1.630541in}}{\pgfqpoint{1.113632in}{1.634447in}}%
\pgfpathcurveto{\pgfqpoint{1.109725in}{1.638354in}}{\pgfqpoint{1.104426in}{1.640549in}}{\pgfqpoint{1.098900in}{1.640549in}}%
\pgfpathcurveto{\pgfqpoint{1.093375in}{1.640549in}}{\pgfqpoint{1.088076in}{1.638354in}}{\pgfqpoint{1.084169in}{1.634447in}}%
\pgfpathcurveto{\pgfqpoint{1.080262in}{1.630541in}}{\pgfqpoint{1.078067in}{1.625241in}}{\pgfqpoint{1.078067in}{1.619716in}}%
\pgfpathcurveto{\pgfqpoint{1.078067in}{1.614191in}}{\pgfqpoint{1.080262in}{1.608892in}}{\pgfqpoint{1.084169in}{1.604985in}}%
\pgfpathcurveto{\pgfqpoint{1.088076in}{1.601078in}}{\pgfqpoint{1.093375in}{1.598883in}}{\pgfqpoint{1.098900in}{1.598883in}}%
\pgfpathclose%
\pgfusepath{fill}%
\end{pgfscope}%
\begin{pgfscope}%
\pgfpathrectangle{\pgfqpoint{0.833185in}{1.588185in}}{\pgfqpoint{1.162500in}{0.755000in}} %
\pgfusepath{clip}%
\pgfsetbuttcap%
\pgfsetroundjoin%
\definecolor{currentfill}{rgb}{0.000000,0.000000,0.000000}%
\pgfsetfillcolor{currentfill}%
\pgfsetfillopacity{0.500000}%
\pgfsetlinewidth{0.000000pt}%
\definecolor{currentstroke}{rgb}{0.000000,0.000000,0.000000}%
\pgfsetstrokecolor{currentstroke}%
\pgfsetdash{}{0pt}%
\pgfpathmoveto{\pgfqpoint{1.085127in}{1.585327in}}%
\pgfpathcurveto{\pgfqpoint{1.090652in}{1.585327in}}{\pgfqpoint{1.095951in}{1.587523in}}{\pgfqpoint{1.099858in}{1.591429in}}%
\pgfpathcurveto{\pgfqpoint{1.103765in}{1.595336in}}{\pgfqpoint{1.105960in}{1.600636in}}{\pgfqpoint{1.105960in}{1.606161in}}%
\pgfpathcurveto{\pgfqpoint{1.105960in}{1.611686in}}{\pgfqpoint{1.103765in}{1.616985in}}{\pgfqpoint{1.099858in}{1.620892in}}%
\pgfpathcurveto{\pgfqpoint{1.095951in}{1.624799in}}{\pgfqpoint{1.090652in}{1.626994in}}{\pgfqpoint{1.085127in}{1.626994in}}%
\pgfpathcurveto{\pgfqpoint{1.079602in}{1.626994in}}{\pgfqpoint{1.074302in}{1.624799in}}{\pgfqpoint{1.070395in}{1.620892in}}%
\pgfpathcurveto{\pgfqpoint{1.066488in}{1.616985in}}{\pgfqpoint{1.064293in}{1.611686in}}{\pgfqpoint{1.064293in}{1.606161in}}%
\pgfpathcurveto{\pgfqpoint{1.064293in}{1.600636in}}{\pgfqpoint{1.066488in}{1.595336in}}{\pgfqpoint{1.070395in}{1.591429in}}%
\pgfpathcurveto{\pgfqpoint{1.074302in}{1.587523in}}{\pgfqpoint{1.079602in}{1.585327in}}{\pgfqpoint{1.085127in}{1.585327in}}%
\pgfpathclose%
\pgfusepath{fill}%
\end{pgfscope}%
\begin{pgfscope}%
\pgfpathrectangle{\pgfqpoint{0.833185in}{1.588185in}}{\pgfqpoint{1.162500in}{0.755000in}} %
\pgfusepath{clip}%
\pgfsetbuttcap%
\pgfsetroundjoin%
\definecolor{currentfill}{rgb}{0.000000,0.000000,0.000000}%
\pgfsetfillcolor{currentfill}%
\pgfsetfillopacity{0.500000}%
\pgfsetlinewidth{0.000000pt}%
\definecolor{currentstroke}{rgb}{0.000000,0.000000,0.000000}%
\pgfsetstrokecolor{currentstroke}%
\pgfsetdash{}{0pt}%
\pgfpathmoveto{\pgfqpoint{1.896712in}{1.989747in}}%
\pgfpathcurveto{\pgfqpoint{1.902237in}{1.989747in}}{\pgfqpoint{1.907537in}{1.991942in}}{\pgfqpoint{1.911443in}{1.995849in}}%
\pgfpathcurveto{\pgfqpoint{1.915350in}{1.999756in}}{\pgfqpoint{1.917545in}{2.005055in}}{\pgfqpoint{1.917545in}{2.010580in}}%
\pgfpathcurveto{\pgfqpoint{1.917545in}{2.016105in}}{\pgfqpoint{1.915350in}{2.021405in}}{\pgfqpoint{1.911443in}{2.025311in}}%
\pgfpathcurveto{\pgfqpoint{1.907537in}{2.029218in}}{\pgfqpoint{1.902237in}{2.031413in}}{\pgfqpoint{1.896712in}{2.031413in}}%
\pgfpathcurveto{\pgfqpoint{1.891187in}{2.031413in}}{\pgfqpoint{1.885887in}{2.029218in}}{\pgfqpoint{1.881981in}{2.025311in}}%
\pgfpathcurveto{\pgfqpoint{1.878074in}{2.021405in}}{\pgfqpoint{1.875879in}{2.016105in}}{\pgfqpoint{1.875879in}{2.010580in}}%
\pgfpathcurveto{\pgfqpoint{1.875879in}{2.005055in}}{\pgfqpoint{1.878074in}{1.999756in}}{\pgfqpoint{1.881981in}{1.995849in}}%
\pgfpathcurveto{\pgfqpoint{1.885887in}{1.991942in}}{\pgfqpoint{1.891187in}{1.989747in}}{\pgfqpoint{1.896712in}{1.989747in}}%
\pgfpathclose%
\pgfusepath{fill}%
\end{pgfscope}%
\begin{pgfscope}%
\pgfpathrectangle{\pgfqpoint{0.833185in}{1.588185in}}{\pgfqpoint{1.162500in}{0.755000in}} %
\pgfusepath{clip}%
\pgfsetbuttcap%
\pgfsetroundjoin%
\definecolor{currentfill}{rgb}{0.000000,0.000000,0.000000}%
\pgfsetfillcolor{currentfill}%
\pgfsetfillopacity{0.500000}%
\pgfsetlinewidth{0.000000pt}%
\definecolor{currentstroke}{rgb}{0.000000,0.000000,0.000000}%
\pgfsetstrokecolor{currentstroke}%
\pgfsetdash{}{0pt}%
\pgfpathmoveto{\pgfqpoint{1.968006in}{1.867889in}}%
\pgfpathcurveto{\pgfqpoint{1.973531in}{1.867889in}}{\pgfqpoint{1.978831in}{1.870084in}}{\pgfqpoint{1.982737in}{1.873991in}}%
\pgfpathcurveto{\pgfqpoint{1.986644in}{1.877898in}}{\pgfqpoint{1.988839in}{1.883197in}}{\pgfqpoint{1.988839in}{1.888722in}}%
\pgfpathcurveto{\pgfqpoint{1.988839in}{1.894247in}}{\pgfqpoint{1.986644in}{1.899547in}}{\pgfqpoint{1.982737in}{1.903454in}}%
\pgfpathcurveto{\pgfqpoint{1.978831in}{1.907360in}}{\pgfqpoint{1.973531in}{1.909556in}}{\pgfqpoint{1.968006in}{1.909556in}}%
\pgfpathcurveto{\pgfqpoint{1.962481in}{1.909556in}}{\pgfqpoint{1.957181in}{1.907360in}}{\pgfqpoint{1.953275in}{1.903454in}}%
\pgfpathcurveto{\pgfqpoint{1.949368in}{1.899547in}}{\pgfqpoint{1.947173in}{1.894247in}}{\pgfqpoint{1.947173in}{1.888722in}}%
\pgfpathcurveto{\pgfqpoint{1.947173in}{1.883197in}}{\pgfqpoint{1.949368in}{1.877898in}}{\pgfqpoint{1.953275in}{1.873991in}}%
\pgfpathcurveto{\pgfqpoint{1.957181in}{1.870084in}}{\pgfqpoint{1.962481in}{1.867889in}}{\pgfqpoint{1.968006in}{1.867889in}}%
\pgfpathclose%
\pgfusepath{fill}%
\end{pgfscope}%
\begin{pgfscope}%
\pgfpathrectangle{\pgfqpoint{0.833185in}{1.588185in}}{\pgfqpoint{1.162500in}{0.755000in}} %
\pgfusepath{clip}%
\pgfsetbuttcap%
\pgfsetroundjoin%
\definecolor{currentfill}{rgb}{0.000000,0.000000,0.000000}%
\pgfsetfillcolor{currentfill}%
\pgfsetfillopacity{0.500000}%
\pgfsetlinewidth{0.000000pt}%
\definecolor{currentstroke}{rgb}{0.000000,0.000000,0.000000}%
\pgfsetstrokecolor{currentstroke}%
\pgfsetdash{}{0pt}%
\pgfpathmoveto{\pgfqpoint{1.742203in}{1.854418in}}%
\pgfpathcurveto{\pgfqpoint{1.747728in}{1.854418in}}{\pgfqpoint{1.753027in}{1.856613in}}{\pgfqpoint{1.756934in}{1.860520in}}%
\pgfpathcurveto{\pgfqpoint{1.760841in}{1.864427in}}{\pgfqpoint{1.763036in}{1.869726in}}{\pgfqpoint{1.763036in}{1.875251in}}%
\pgfpathcurveto{\pgfqpoint{1.763036in}{1.880776in}}{\pgfqpoint{1.760841in}{1.886076in}}{\pgfqpoint{1.756934in}{1.889982in}}%
\pgfpathcurveto{\pgfqpoint{1.753027in}{1.893889in}}{\pgfqpoint{1.747728in}{1.896084in}}{\pgfqpoint{1.742203in}{1.896084in}}%
\pgfpathcurveto{\pgfqpoint{1.736677in}{1.896084in}}{\pgfqpoint{1.731378in}{1.893889in}}{\pgfqpoint{1.727471in}{1.889982in}}%
\pgfpathcurveto{\pgfqpoint{1.723564in}{1.886076in}}{\pgfqpoint{1.721369in}{1.880776in}}{\pgfqpoint{1.721369in}{1.875251in}}%
\pgfpathcurveto{\pgfqpoint{1.721369in}{1.869726in}}{\pgfqpoint{1.723564in}{1.864427in}}{\pgfqpoint{1.727471in}{1.860520in}}%
\pgfpathcurveto{\pgfqpoint{1.731378in}{1.856613in}}{\pgfqpoint{1.736677in}{1.854418in}}{\pgfqpoint{1.742203in}{1.854418in}}%
\pgfpathclose%
\pgfusepath{fill}%
\end{pgfscope}%
\begin{pgfscope}%
\pgfpathrectangle{\pgfqpoint{0.833185in}{1.588185in}}{\pgfqpoint{1.162500in}{0.755000in}} %
\pgfusepath{clip}%
\pgfsetbuttcap%
\pgfsetroundjoin%
\definecolor{currentfill}{rgb}{0.000000,0.000000,0.000000}%
\pgfsetfillcolor{currentfill}%
\pgfsetfillopacity{0.500000}%
\pgfsetlinewidth{0.000000pt}%
\definecolor{currentstroke}{rgb}{0.000000,0.000000,0.000000}%
\pgfsetstrokecolor{currentstroke}%
\pgfsetdash{}{0pt}%
\pgfpathmoveto{\pgfqpoint{1.058530in}{2.049107in}}%
\pgfpathcurveto{\pgfqpoint{1.064055in}{2.049107in}}{\pgfqpoint{1.069355in}{2.051302in}}{\pgfqpoint{1.073261in}{2.055209in}}%
\pgfpathcurveto{\pgfqpoint{1.077168in}{2.059115in}}{\pgfqpoint{1.079363in}{2.064415in}}{\pgfqpoint{1.079363in}{2.069940in}}%
\pgfpathcurveto{\pgfqpoint{1.079363in}{2.075465in}}{\pgfqpoint{1.077168in}{2.080764in}}{\pgfqpoint{1.073261in}{2.084671in}}%
\pgfpathcurveto{\pgfqpoint{1.069355in}{2.088578in}}{\pgfqpoint{1.064055in}{2.090773in}}{\pgfqpoint{1.058530in}{2.090773in}}%
\pgfpathcurveto{\pgfqpoint{1.053005in}{2.090773in}}{\pgfqpoint{1.047706in}{2.088578in}}{\pgfqpoint{1.043799in}{2.084671in}}%
\pgfpathcurveto{\pgfqpoint{1.039892in}{2.080764in}}{\pgfqpoint{1.037697in}{2.075465in}}{\pgfqpoint{1.037697in}{2.069940in}}%
\pgfpathcurveto{\pgfqpoint{1.037697in}{2.064415in}}{\pgfqpoint{1.039892in}{2.059115in}}{\pgfqpoint{1.043799in}{2.055209in}}%
\pgfpathcurveto{\pgfqpoint{1.047706in}{2.051302in}}{\pgfqpoint{1.053005in}{2.049107in}}{\pgfqpoint{1.058530in}{2.049107in}}%
\pgfpathclose%
\pgfusepath{fill}%
\end{pgfscope}%
\begin{pgfscope}%
\pgfpathrectangle{\pgfqpoint{0.833185in}{1.588185in}}{\pgfqpoint{1.162500in}{0.755000in}} %
\pgfusepath{clip}%
\pgfsetbuttcap%
\pgfsetroundjoin%
\definecolor{currentfill}{rgb}{0.000000,0.000000,0.000000}%
\pgfsetfillcolor{currentfill}%
\pgfsetfillopacity{0.500000}%
\pgfsetlinewidth{0.000000pt}%
\definecolor{currentstroke}{rgb}{0.000000,0.000000,0.000000}%
\pgfsetstrokecolor{currentstroke}%
\pgfsetdash{}{0pt}%
\pgfpathmoveto{\pgfqpoint{1.520990in}{1.703826in}}%
\pgfpathcurveto{\pgfqpoint{1.526515in}{1.703826in}}{\pgfqpoint{1.531814in}{1.706022in}}{\pgfqpoint{1.535721in}{1.709928in}}%
\pgfpathcurveto{\pgfqpoint{1.539628in}{1.713835in}}{\pgfqpoint{1.541823in}{1.719135in}}{\pgfqpoint{1.541823in}{1.724660in}}%
\pgfpathcurveto{\pgfqpoint{1.541823in}{1.730185in}}{\pgfqpoint{1.539628in}{1.735484in}}{\pgfqpoint{1.535721in}{1.739391in}}%
\pgfpathcurveto{\pgfqpoint{1.531814in}{1.743298in}}{\pgfqpoint{1.526515in}{1.745493in}}{\pgfqpoint{1.520990in}{1.745493in}}%
\pgfpathcurveto{\pgfqpoint{1.515465in}{1.745493in}}{\pgfqpoint{1.510165in}{1.743298in}}{\pgfqpoint{1.506258in}{1.739391in}}%
\pgfpathcurveto{\pgfqpoint{1.502352in}{1.735484in}}{\pgfqpoint{1.500156in}{1.730185in}}{\pgfqpoint{1.500156in}{1.724660in}}%
\pgfpathcurveto{\pgfqpoint{1.500156in}{1.719135in}}{\pgfqpoint{1.502352in}{1.713835in}}{\pgfqpoint{1.506258in}{1.709928in}}%
\pgfpathcurveto{\pgfqpoint{1.510165in}{1.706022in}}{\pgfqpoint{1.515465in}{1.703826in}}{\pgfqpoint{1.520990in}{1.703826in}}%
\pgfpathclose%
\pgfusepath{fill}%
\end{pgfscope}%
\begin{pgfscope}%
\pgfpathrectangle{\pgfqpoint{0.833185in}{1.588185in}}{\pgfqpoint{1.162500in}{0.755000in}} %
\pgfusepath{clip}%
\pgfsetbuttcap%
\pgfsetroundjoin%
\definecolor{currentfill}{rgb}{0.000000,0.000000,0.000000}%
\pgfsetfillcolor{currentfill}%
\pgfsetfillopacity{0.500000}%
\pgfsetlinewidth{0.000000pt}%
\definecolor{currentstroke}{rgb}{0.000000,0.000000,0.000000}%
\pgfsetstrokecolor{currentstroke}%
\pgfsetdash{}{0pt}%
\pgfpathmoveto{\pgfqpoint{1.402203in}{1.734907in}}%
\pgfpathcurveto{\pgfqpoint{1.407728in}{1.734907in}}{\pgfqpoint{1.413027in}{1.737102in}}{\pgfqpoint{1.416934in}{1.741009in}}%
\pgfpathcurveto{\pgfqpoint{1.420841in}{1.744915in}}{\pgfqpoint{1.423036in}{1.750215in}}{\pgfqpoint{1.423036in}{1.755740in}}%
\pgfpathcurveto{\pgfqpoint{1.423036in}{1.761265in}}{\pgfqpoint{1.420841in}{1.766565in}}{\pgfqpoint{1.416934in}{1.770471in}}%
\pgfpathcurveto{\pgfqpoint{1.413027in}{1.774378in}}{\pgfqpoint{1.407728in}{1.776573in}}{\pgfqpoint{1.402203in}{1.776573in}}%
\pgfpathcurveto{\pgfqpoint{1.396677in}{1.776573in}}{\pgfqpoint{1.391378in}{1.774378in}}{\pgfqpoint{1.387471in}{1.770471in}}%
\pgfpathcurveto{\pgfqpoint{1.383564in}{1.766565in}}{\pgfqpoint{1.381369in}{1.761265in}}{\pgfqpoint{1.381369in}{1.755740in}}%
\pgfpathcurveto{\pgfqpoint{1.381369in}{1.750215in}}{\pgfqpoint{1.383564in}{1.744915in}}{\pgfqpoint{1.387471in}{1.741009in}}%
\pgfpathcurveto{\pgfqpoint{1.391378in}{1.737102in}}{\pgfqpoint{1.396677in}{1.734907in}}{\pgfqpoint{1.402203in}{1.734907in}}%
\pgfpathclose%
\pgfusepath{fill}%
\end{pgfscope}%
\begin{pgfscope}%
\pgfpathrectangle{\pgfqpoint{0.833185in}{1.588185in}}{\pgfqpoint{1.162500in}{0.755000in}} %
\pgfusepath{clip}%
\pgfsetbuttcap%
\pgfsetroundjoin%
\definecolor{currentfill}{rgb}{0.000000,0.000000,0.000000}%
\pgfsetfillcolor{currentfill}%
\pgfsetfillopacity{0.500000}%
\pgfsetlinewidth{0.000000pt}%
\definecolor{currentstroke}{rgb}{0.000000,0.000000,0.000000}%
\pgfsetstrokecolor{currentstroke}%
\pgfsetdash{}{0pt}%
\pgfpathmoveto{\pgfqpoint{1.053510in}{1.977476in}}%
\pgfpathcurveto{\pgfqpoint{1.059035in}{1.977476in}}{\pgfqpoint{1.064335in}{1.979671in}}{\pgfqpoint{1.068242in}{1.983578in}}%
\pgfpathcurveto{\pgfqpoint{1.072148in}{1.987485in}}{\pgfqpoint{1.074344in}{1.992785in}}{\pgfqpoint{1.074344in}{1.998310in}}%
\pgfpathcurveto{\pgfqpoint{1.074344in}{2.003835in}}{\pgfqpoint{1.072148in}{2.009134in}}{\pgfqpoint{1.068242in}{2.013041in}}%
\pgfpathcurveto{\pgfqpoint{1.064335in}{2.016948in}}{\pgfqpoint{1.059035in}{2.019143in}}{\pgfqpoint{1.053510in}{2.019143in}}%
\pgfpathcurveto{\pgfqpoint{1.047985in}{2.019143in}}{\pgfqpoint{1.042686in}{2.016948in}}{\pgfqpoint{1.038779in}{2.013041in}}%
\pgfpathcurveto{\pgfqpoint{1.034872in}{2.009134in}}{\pgfqpoint{1.032677in}{2.003835in}}{\pgfqpoint{1.032677in}{1.998310in}}%
\pgfpathcurveto{\pgfqpoint{1.032677in}{1.992785in}}{\pgfqpoint{1.034872in}{1.987485in}}{\pgfqpoint{1.038779in}{1.983578in}}%
\pgfpathcurveto{\pgfqpoint{1.042686in}{1.979671in}}{\pgfqpoint{1.047985in}{1.977476in}}{\pgfqpoint{1.053510in}{1.977476in}}%
\pgfpathclose%
\pgfusepath{fill}%
\end{pgfscope}%
\begin{pgfscope}%
\pgfpathrectangle{\pgfqpoint{0.833185in}{1.588185in}}{\pgfqpoint{1.162500in}{0.755000in}} %
\pgfusepath{clip}%
\pgfsetbuttcap%
\pgfsetroundjoin%
\definecolor{currentfill}{rgb}{0.000000,0.000000,0.000000}%
\pgfsetfillcolor{currentfill}%
\pgfsetfillopacity{0.500000}%
\pgfsetlinewidth{0.000000pt}%
\definecolor{currentstroke}{rgb}{0.000000,0.000000,0.000000}%
\pgfsetstrokecolor{currentstroke}%
\pgfsetdash{}{0pt}%
\pgfpathmoveto{\pgfqpoint{1.492358in}{1.614683in}}%
\pgfpathcurveto{\pgfqpoint{1.497883in}{1.614683in}}{\pgfqpoint{1.503182in}{1.616879in}}{\pgfqpoint{1.507089in}{1.620785in}}%
\pgfpathcurveto{\pgfqpoint{1.510996in}{1.624692in}}{\pgfqpoint{1.513191in}{1.629992in}}{\pgfqpoint{1.513191in}{1.635517in}}%
\pgfpathcurveto{\pgfqpoint{1.513191in}{1.641042in}}{\pgfqpoint{1.510996in}{1.646341in}}{\pgfqpoint{1.507089in}{1.650248in}}%
\pgfpathcurveto{\pgfqpoint{1.503182in}{1.654155in}}{\pgfqpoint{1.497883in}{1.656350in}}{\pgfqpoint{1.492358in}{1.656350in}}%
\pgfpathcurveto{\pgfqpoint{1.486832in}{1.656350in}}{\pgfqpoint{1.481533in}{1.654155in}}{\pgfqpoint{1.477626in}{1.650248in}}%
\pgfpathcurveto{\pgfqpoint{1.473719in}{1.646341in}}{\pgfqpoint{1.471524in}{1.641042in}}{\pgfqpoint{1.471524in}{1.635517in}}%
\pgfpathcurveto{\pgfqpoint{1.471524in}{1.629992in}}{\pgfqpoint{1.473719in}{1.624692in}}{\pgfqpoint{1.477626in}{1.620785in}}%
\pgfpathcurveto{\pgfqpoint{1.481533in}{1.616879in}}{\pgfqpoint{1.486832in}{1.614683in}}{\pgfqpoint{1.492358in}{1.614683in}}%
\pgfpathclose%
\pgfusepath{fill}%
\end{pgfscope}%
\begin{pgfscope}%
\pgfsetbuttcap%
\pgfsetroundjoin%
\definecolor{currentfill}{rgb}{0.000000,0.000000,0.000000}%
\pgfsetfillcolor{currentfill}%
\pgfsetlinewidth{0.803000pt}%
\definecolor{currentstroke}{rgb}{0.000000,0.000000,0.000000}%
\pgfsetstrokecolor{currentstroke}%
\pgfsetdash{}{0pt}%
\pgfsys@defobject{currentmarker}{\pgfqpoint{-0.048611in}{0.000000in}}{\pgfqpoint{0.000000in}{0.000000in}}{%
\pgfpathmoveto{\pgfqpoint{0.000000in}{0.000000in}}%
\pgfpathlineto{\pgfqpoint{-0.048611in}{0.000000in}}%
\pgfusepath{stroke,fill}%
}%
\begin{pgfscope}%
\pgfsys@transformshift{0.833185in}{1.808481in}%
\pgfsys@useobject{currentmarker}{}%
\end{pgfscope}%
\end{pgfscope}%
\begin{pgfscope}%
\pgftext[x=0.585111in,y=1.766272in,left,base]{\rmfamily\fontsize{8.000000}{9.600000}\selectfont \(\displaystyle 2.5\)}%
\end{pgfscope}%
\begin{pgfscope}%
\pgfsetbuttcap%
\pgfsetroundjoin%
\definecolor{currentfill}{rgb}{0.000000,0.000000,0.000000}%
\pgfsetfillcolor{currentfill}%
\pgfsetlinewidth{0.803000pt}%
\definecolor{currentstroke}{rgb}{0.000000,0.000000,0.000000}%
\pgfsetstrokecolor{currentstroke}%
\pgfsetdash{}{0pt}%
\pgfsys@defobject{currentmarker}{\pgfqpoint{-0.048611in}{0.000000in}}{\pgfqpoint{0.000000in}{0.000000in}}{%
\pgfpathmoveto{\pgfqpoint{0.000000in}{0.000000in}}%
\pgfpathlineto{\pgfqpoint{-0.048611in}{0.000000in}}%
\pgfusepath{stroke,fill}%
}%
\begin{pgfscope}%
\pgfsys@transformshift{0.833185in}{2.044496in}%
\pgfsys@useobject{currentmarker}{}%
\end{pgfscope}%
\end{pgfscope}%
\begin{pgfscope}%
\pgftext[x=0.585111in,y=2.002287in,left,base]{\rmfamily\fontsize{8.000000}{9.600000}\selectfont \(\displaystyle 5.0\)}%
\end{pgfscope}%
\begin{pgfscope}%
\pgfsetbuttcap%
\pgfsetroundjoin%
\definecolor{currentfill}{rgb}{0.000000,0.000000,0.000000}%
\pgfsetfillcolor{currentfill}%
\pgfsetlinewidth{0.803000pt}%
\definecolor{currentstroke}{rgb}{0.000000,0.000000,0.000000}%
\pgfsetstrokecolor{currentstroke}%
\pgfsetdash{}{0pt}%
\pgfsys@defobject{currentmarker}{\pgfqpoint{-0.048611in}{0.000000in}}{\pgfqpoint{0.000000in}{0.000000in}}{%
\pgfpathmoveto{\pgfqpoint{0.000000in}{0.000000in}}%
\pgfpathlineto{\pgfqpoint{-0.048611in}{0.000000in}}%
\pgfusepath{stroke,fill}%
}%
\begin{pgfscope}%
\pgfsys@transformshift{0.833185in}{2.280511in}%
\pgfsys@useobject{currentmarker}{}%
\end{pgfscope}%
\end{pgfscope}%
\begin{pgfscope}%
\pgftext[x=0.585111in,y=2.238302in,left,base]{\rmfamily\fontsize{8.000000}{9.600000}\selectfont \(\displaystyle 7.5\)}%
\end{pgfscope}%
\begin{pgfscope}%
\pgftext[x=0.529556in,y=1.965685in,,bottom,rotate=90.000000]{\rmfamily\fontsize{10.000000}{12.000000}\selectfont charge}%
\end{pgfscope}%
\begin{pgfscope}%
\pgftext[x=0.833185in,y=2.384851in,left,base]{\rmfamily\fontsize{10.000000}{12.000000}\selectfont \(\displaystyle \times10^{-10}\)}%
\end{pgfscope}%
\begin{pgfscope}%
\pgfsetrectcap%
\pgfsetmiterjoin%
\pgfsetlinewidth{0.803000pt}%
\definecolor{currentstroke}{rgb}{0.000000,0.000000,0.000000}%
\pgfsetstrokecolor{currentstroke}%
\pgfsetdash{}{0pt}%
\pgfpathmoveto{\pgfqpoint{0.833185in}{1.588185in}}%
\pgfpathlineto{\pgfqpoint{0.833185in}{2.343185in}}%
\pgfusepath{stroke}%
\end{pgfscope}%
\begin{pgfscope}%
\pgfsetrectcap%
\pgfsetmiterjoin%
\pgfsetlinewidth{0.803000pt}%
\definecolor{currentstroke}{rgb}{0.000000,0.000000,0.000000}%
\pgfsetstrokecolor{currentstroke}%
\pgfsetdash{}{0pt}%
\pgfpathmoveto{\pgfqpoint{1.995685in}{1.588185in}}%
\pgfpathlineto{\pgfqpoint{1.995685in}{2.343185in}}%
\pgfusepath{stroke}%
\end{pgfscope}%
\begin{pgfscope}%
\pgfsetrectcap%
\pgfsetmiterjoin%
\pgfsetlinewidth{0.803000pt}%
\definecolor{currentstroke}{rgb}{0.000000,0.000000,0.000000}%
\pgfsetstrokecolor{currentstroke}%
\pgfsetdash{}{0pt}%
\pgfpathmoveto{\pgfqpoint{0.833185in}{1.588185in}}%
\pgfpathlineto{\pgfqpoint{1.995685in}{1.588185in}}%
\pgfusepath{stroke}%
\end{pgfscope}%
\begin{pgfscope}%
\pgfsetrectcap%
\pgfsetmiterjoin%
\pgfsetlinewidth{0.803000pt}%
\definecolor{currentstroke}{rgb}{0.000000,0.000000,0.000000}%
\pgfsetstrokecolor{currentstroke}%
\pgfsetdash{}{0pt}%
\pgfpathmoveto{\pgfqpoint{0.833185in}{2.343185in}}%
\pgfpathlineto{\pgfqpoint{1.995685in}{2.343185in}}%
\pgfusepath{stroke}%
\end{pgfscope}%
\begin{pgfscope}%
\pgfsetbuttcap%
\pgfsetmiterjoin%
\definecolor{currentfill}{rgb}{1.000000,1.000000,1.000000}%
\pgfsetfillcolor{currentfill}%
\pgfsetlinewidth{0.000000pt}%
\definecolor{currentstroke}{rgb}{0.000000,0.000000,0.000000}%
\pgfsetstrokecolor{currentstroke}%
\pgfsetstrokeopacity{0.000000}%
\pgfsetdash{}{0pt}%
\pgfpathmoveto{\pgfqpoint{1.995685in}{1.588185in}}%
\pgfpathlineto{\pgfqpoint{3.158185in}{1.588185in}}%
\pgfpathlineto{\pgfqpoint{3.158185in}{2.343185in}}%
\pgfpathlineto{\pgfqpoint{1.995685in}{2.343185in}}%
\pgfpathclose%
\pgfusepath{fill}%
\end{pgfscope}%
\begin{pgfscope}%
\pgfpathrectangle{\pgfqpoint{1.995685in}{1.588185in}}{\pgfqpoint{1.162500in}{0.755000in}} %
\pgfusepath{clip}%
\pgfsetbuttcap%
\pgfsetroundjoin%
\definecolor{currentfill}{rgb}{0.000000,0.000000,0.000000}%
\pgfsetfillcolor{currentfill}%
\pgfsetfillopacity{0.500000}%
\pgfsetlinewidth{0.000000pt}%
\definecolor{currentstroke}{rgb}{0.000000,0.000000,0.000000}%
\pgfsetstrokecolor{currentstroke}%
\pgfsetdash{}{0pt}%
\pgfpathmoveto{\pgfqpoint{3.130506in}{2.304375in}}%
\pgfpathcurveto{\pgfqpoint{3.136031in}{2.304375in}}{\pgfqpoint{3.141331in}{2.306570in}}{\pgfqpoint{3.145237in}{2.310477in}}%
\pgfpathcurveto{\pgfqpoint{3.149144in}{2.314384in}}{\pgfqpoint{3.151339in}{2.319683in}}{\pgfqpoint{3.151339in}{2.325208in}}%
\pgfpathcurveto{\pgfqpoint{3.151339in}{2.330733in}}{\pgfqpoint{3.149144in}{2.336033in}}{\pgfqpoint{3.145237in}{2.339940in}}%
\pgfpathcurveto{\pgfqpoint{3.141331in}{2.343847in}}{\pgfqpoint{3.136031in}{2.346042in}}{\pgfqpoint{3.130506in}{2.346042in}}%
\pgfpathcurveto{\pgfqpoint{3.124981in}{2.346042in}}{\pgfqpoint{3.119681in}{2.343847in}}{\pgfqpoint{3.115775in}{2.339940in}}%
\pgfpathcurveto{\pgfqpoint{3.111868in}{2.336033in}}{\pgfqpoint{3.109673in}{2.330733in}}{\pgfqpoint{3.109673in}{2.325208in}}%
\pgfpathcurveto{\pgfqpoint{3.109673in}{2.319683in}}{\pgfqpoint{3.111868in}{2.314384in}}{\pgfqpoint{3.115775in}{2.310477in}}%
\pgfpathcurveto{\pgfqpoint{3.119681in}{2.306570in}}{\pgfqpoint{3.124981in}{2.304375in}}{\pgfqpoint{3.130506in}{2.304375in}}%
\pgfpathclose%
\pgfusepath{fill}%
\end{pgfscope}%
\begin{pgfscope}%
\pgfpathrectangle{\pgfqpoint{1.995685in}{1.588185in}}{\pgfqpoint{1.162500in}{0.755000in}} %
\pgfusepath{clip}%
\pgfsetbuttcap%
\pgfsetroundjoin%
\definecolor{currentfill}{rgb}{0.000000,0.000000,0.000000}%
\pgfsetfillcolor{currentfill}%
\pgfsetfillopacity{0.500000}%
\pgfsetlinewidth{0.000000pt}%
\definecolor{currentstroke}{rgb}{0.000000,0.000000,0.000000}%
\pgfsetstrokecolor{currentstroke}%
\pgfsetdash{}{0pt}%
\pgfpathmoveto{\pgfqpoint{2.755547in}{1.845741in}}%
\pgfpathcurveto{\pgfqpoint{2.761072in}{1.845741in}}{\pgfqpoint{2.766372in}{1.847936in}}{\pgfqpoint{2.770279in}{1.851843in}}%
\pgfpathcurveto{\pgfqpoint{2.774186in}{1.855750in}}{\pgfqpoint{2.776381in}{1.861049in}}{\pgfqpoint{2.776381in}{1.866574in}}%
\pgfpathcurveto{\pgfqpoint{2.776381in}{1.872099in}}{\pgfqpoint{2.774186in}{1.877399in}}{\pgfqpoint{2.770279in}{1.881306in}}%
\pgfpathcurveto{\pgfqpoint{2.766372in}{1.885212in}}{\pgfqpoint{2.761072in}{1.887408in}}{\pgfqpoint{2.755547in}{1.887408in}}%
\pgfpathcurveto{\pgfqpoint{2.750022in}{1.887408in}}{\pgfqpoint{2.744723in}{1.885212in}}{\pgfqpoint{2.740816in}{1.881306in}}%
\pgfpathcurveto{\pgfqpoint{2.736909in}{1.877399in}}{\pgfqpoint{2.734714in}{1.872099in}}{\pgfqpoint{2.734714in}{1.866574in}}%
\pgfpathcurveto{\pgfqpoint{2.734714in}{1.861049in}}{\pgfqpoint{2.736909in}{1.855750in}}{\pgfqpoint{2.740816in}{1.851843in}}%
\pgfpathcurveto{\pgfqpoint{2.744723in}{1.847936in}}{\pgfqpoint{2.750022in}{1.845741in}}{\pgfqpoint{2.755547in}{1.845741in}}%
\pgfpathclose%
\pgfusepath{fill}%
\end{pgfscope}%
\begin{pgfscope}%
\pgfpathrectangle{\pgfqpoint{1.995685in}{1.588185in}}{\pgfqpoint{1.162500in}{0.755000in}} %
\pgfusepath{clip}%
\pgfsetbuttcap%
\pgfsetroundjoin%
\definecolor{currentfill}{rgb}{0.000000,0.000000,0.000000}%
\pgfsetfillcolor{currentfill}%
\pgfsetfillopacity{0.500000}%
\pgfsetlinewidth{0.000000pt}%
\definecolor{currentstroke}{rgb}{0.000000,0.000000,0.000000}%
\pgfsetstrokecolor{currentstroke}%
\pgfsetdash{}{0pt}%
\pgfpathmoveto{\pgfqpoint{2.755127in}{1.840570in}}%
\pgfpathcurveto{\pgfqpoint{2.760652in}{1.840570in}}{\pgfqpoint{2.765952in}{1.842766in}}{\pgfqpoint{2.769858in}{1.846672in}}%
\pgfpathcurveto{\pgfqpoint{2.773765in}{1.850579in}}{\pgfqpoint{2.775960in}{1.855879in}}{\pgfqpoint{2.775960in}{1.861404in}}%
\pgfpathcurveto{\pgfqpoint{2.775960in}{1.866929in}}{\pgfqpoint{2.773765in}{1.872228in}}{\pgfqpoint{2.769858in}{1.876135in}}%
\pgfpathcurveto{\pgfqpoint{2.765952in}{1.880042in}}{\pgfqpoint{2.760652in}{1.882237in}}{\pgfqpoint{2.755127in}{1.882237in}}%
\pgfpathcurveto{\pgfqpoint{2.749602in}{1.882237in}}{\pgfqpoint{2.744302in}{1.880042in}}{\pgfqpoint{2.740396in}{1.876135in}}%
\pgfpathcurveto{\pgfqpoint{2.736489in}{1.872228in}}{\pgfqpoint{2.734294in}{1.866929in}}{\pgfqpoint{2.734294in}{1.861404in}}%
\pgfpathcurveto{\pgfqpoint{2.734294in}{1.855879in}}{\pgfqpoint{2.736489in}{1.850579in}}{\pgfqpoint{2.740396in}{1.846672in}}%
\pgfpathcurveto{\pgfqpoint{2.744302in}{1.842766in}}{\pgfqpoint{2.749602in}{1.840570in}}{\pgfqpoint{2.755127in}{1.840570in}}%
\pgfpathclose%
\pgfusepath{fill}%
\end{pgfscope}%
\begin{pgfscope}%
\pgfpathrectangle{\pgfqpoint{1.995685in}{1.588185in}}{\pgfqpoint{1.162500in}{0.755000in}} %
\pgfusepath{clip}%
\pgfsetbuttcap%
\pgfsetroundjoin%
\definecolor{currentfill}{rgb}{0.000000,0.000000,0.000000}%
\pgfsetfillcolor{currentfill}%
\pgfsetfillopacity{0.500000}%
\pgfsetlinewidth{0.000000pt}%
\definecolor{currentstroke}{rgb}{0.000000,0.000000,0.000000}%
\pgfsetstrokecolor{currentstroke}%
\pgfsetdash{}{0pt}%
\pgfpathmoveto{\pgfqpoint{2.317114in}{1.680932in}}%
\pgfpathcurveto{\pgfqpoint{2.322639in}{1.680932in}}{\pgfqpoint{2.327939in}{1.683127in}}{\pgfqpoint{2.331845in}{1.687034in}}%
\pgfpathcurveto{\pgfqpoint{2.335752in}{1.690941in}}{\pgfqpoint{2.337947in}{1.696241in}}{\pgfqpoint{2.337947in}{1.701766in}}%
\pgfpathcurveto{\pgfqpoint{2.337947in}{1.707291in}}{\pgfqpoint{2.335752in}{1.712590in}}{\pgfqpoint{2.331845in}{1.716497in}}%
\pgfpathcurveto{\pgfqpoint{2.327939in}{1.720404in}}{\pgfqpoint{2.322639in}{1.722599in}}{\pgfqpoint{2.317114in}{1.722599in}}%
\pgfpathcurveto{\pgfqpoint{2.311589in}{1.722599in}}{\pgfqpoint{2.306289in}{1.720404in}}{\pgfqpoint{2.302383in}{1.716497in}}%
\pgfpathcurveto{\pgfqpoint{2.298476in}{1.712590in}}{\pgfqpoint{2.296281in}{1.707291in}}{\pgfqpoint{2.296281in}{1.701766in}}%
\pgfpathcurveto{\pgfqpoint{2.296281in}{1.696241in}}{\pgfqpoint{2.298476in}{1.690941in}}{\pgfqpoint{2.302383in}{1.687034in}}%
\pgfpathcurveto{\pgfqpoint{2.306289in}{1.683127in}}{\pgfqpoint{2.311589in}{1.680932in}}{\pgfqpoint{2.317114in}{1.680932in}}%
\pgfpathclose%
\pgfusepath{fill}%
\end{pgfscope}%
\begin{pgfscope}%
\pgfpathrectangle{\pgfqpoint{1.995685in}{1.588185in}}{\pgfqpoint{1.162500in}{0.755000in}} %
\pgfusepath{clip}%
\pgfsetbuttcap%
\pgfsetroundjoin%
\definecolor{currentfill}{rgb}{0.000000,0.000000,0.000000}%
\pgfsetfillcolor{currentfill}%
\pgfsetfillopacity{0.500000}%
\pgfsetlinewidth{0.000000pt}%
\definecolor{currentstroke}{rgb}{0.000000,0.000000,0.000000}%
\pgfsetstrokecolor{currentstroke}%
\pgfsetdash{}{0pt}%
\pgfpathmoveto{\pgfqpoint{2.241285in}{1.656866in}}%
\pgfpathcurveto{\pgfqpoint{2.246810in}{1.656866in}}{\pgfqpoint{2.252110in}{1.659061in}}{\pgfqpoint{2.256017in}{1.662968in}}%
\pgfpathcurveto{\pgfqpoint{2.259923in}{1.666874in}}{\pgfqpoint{2.262118in}{1.672174in}}{\pgfqpoint{2.262118in}{1.677699in}}%
\pgfpathcurveto{\pgfqpoint{2.262118in}{1.683224in}}{\pgfqpoint{2.259923in}{1.688523in}}{\pgfqpoint{2.256017in}{1.692430in}}%
\pgfpathcurveto{\pgfqpoint{2.252110in}{1.696337in}}{\pgfqpoint{2.246810in}{1.698532in}}{\pgfqpoint{2.241285in}{1.698532in}}%
\pgfpathcurveto{\pgfqpoint{2.235760in}{1.698532in}}{\pgfqpoint{2.230461in}{1.696337in}}{\pgfqpoint{2.226554in}{1.692430in}}%
\pgfpathcurveto{\pgfqpoint{2.222647in}{1.688523in}}{\pgfqpoint{2.220452in}{1.683224in}}{\pgfqpoint{2.220452in}{1.677699in}}%
\pgfpathcurveto{\pgfqpoint{2.220452in}{1.672174in}}{\pgfqpoint{2.222647in}{1.666874in}}{\pgfqpoint{2.226554in}{1.662968in}}%
\pgfpathcurveto{\pgfqpoint{2.230461in}{1.659061in}}{\pgfqpoint{2.235760in}{1.656866in}}{\pgfqpoint{2.241285in}{1.656866in}}%
\pgfpathclose%
\pgfusepath{fill}%
\end{pgfscope}%
\begin{pgfscope}%
\pgfpathrectangle{\pgfqpoint{1.995685in}{1.588185in}}{\pgfqpoint{1.162500in}{0.755000in}} %
\pgfusepath{clip}%
\pgfsetbuttcap%
\pgfsetroundjoin%
\definecolor{currentfill}{rgb}{0.000000,0.000000,0.000000}%
\pgfsetfillcolor{currentfill}%
\pgfsetfillopacity{0.500000}%
\pgfsetlinewidth{0.000000pt}%
\definecolor{currentstroke}{rgb}{0.000000,0.000000,0.000000}%
\pgfsetstrokecolor{currentstroke}%
\pgfsetdash{}{0pt}%
\pgfpathmoveto{\pgfqpoint{2.106066in}{1.598883in}}%
\pgfpathcurveto{\pgfqpoint{2.111591in}{1.598883in}}{\pgfqpoint{2.116890in}{1.601078in}}{\pgfqpoint{2.120797in}{1.604985in}}%
\pgfpathcurveto{\pgfqpoint{2.124704in}{1.608892in}}{\pgfqpoint{2.126899in}{1.614191in}}{\pgfqpoint{2.126899in}{1.619716in}}%
\pgfpathcurveto{\pgfqpoint{2.126899in}{1.625241in}}{\pgfqpoint{2.124704in}{1.630541in}}{\pgfqpoint{2.120797in}{1.634447in}}%
\pgfpathcurveto{\pgfqpoint{2.116890in}{1.638354in}}{\pgfqpoint{2.111591in}{1.640549in}}{\pgfqpoint{2.106066in}{1.640549in}}%
\pgfpathcurveto{\pgfqpoint{2.100541in}{1.640549in}}{\pgfqpoint{2.095241in}{1.638354in}}{\pgfqpoint{2.091334in}{1.634447in}}%
\pgfpathcurveto{\pgfqpoint{2.087428in}{1.630541in}}{\pgfqpoint{2.085232in}{1.625241in}}{\pgfqpoint{2.085232in}{1.619716in}}%
\pgfpathcurveto{\pgfqpoint{2.085232in}{1.614191in}}{\pgfqpoint{2.087428in}{1.608892in}}{\pgfqpoint{2.091334in}{1.604985in}}%
\pgfpathcurveto{\pgfqpoint{2.095241in}{1.601078in}}{\pgfqpoint{2.100541in}{1.598883in}}{\pgfqpoint{2.106066in}{1.598883in}}%
\pgfpathclose%
\pgfusepath{fill}%
\end{pgfscope}%
\begin{pgfscope}%
\pgfpathrectangle{\pgfqpoint{1.995685in}{1.588185in}}{\pgfqpoint{1.162500in}{0.755000in}} %
\pgfusepath{clip}%
\pgfsetbuttcap%
\pgfsetroundjoin%
\definecolor{currentfill}{rgb}{0.000000,0.000000,0.000000}%
\pgfsetfillcolor{currentfill}%
\pgfsetfillopacity{0.500000}%
\pgfsetlinewidth{0.000000pt}%
\definecolor{currentstroke}{rgb}{0.000000,0.000000,0.000000}%
\pgfsetstrokecolor{currentstroke}%
\pgfsetdash{}{0pt}%
\pgfpathmoveto{\pgfqpoint{2.023363in}{1.585327in}}%
\pgfpathcurveto{\pgfqpoint{2.028888in}{1.585327in}}{\pgfqpoint{2.034188in}{1.587523in}}{\pgfqpoint{2.038095in}{1.591429in}}%
\pgfpathcurveto{\pgfqpoint{2.042001in}{1.595336in}}{\pgfqpoint{2.044196in}{1.600636in}}{\pgfqpoint{2.044196in}{1.606161in}}%
\pgfpathcurveto{\pgfqpoint{2.044196in}{1.611686in}}{\pgfqpoint{2.042001in}{1.616985in}}{\pgfqpoint{2.038095in}{1.620892in}}%
\pgfpathcurveto{\pgfqpoint{2.034188in}{1.624799in}}{\pgfqpoint{2.028888in}{1.626994in}}{\pgfqpoint{2.023363in}{1.626994in}}%
\pgfpathcurveto{\pgfqpoint{2.017838in}{1.626994in}}{\pgfqpoint{2.012539in}{1.624799in}}{\pgfqpoint{2.008632in}{1.620892in}}%
\pgfpathcurveto{\pgfqpoint{2.004725in}{1.616985in}}{\pgfqpoint{2.002530in}{1.611686in}}{\pgfqpoint{2.002530in}{1.606161in}}%
\pgfpathcurveto{\pgfqpoint{2.002530in}{1.600636in}}{\pgfqpoint{2.004725in}{1.595336in}}{\pgfqpoint{2.008632in}{1.591429in}}%
\pgfpathcurveto{\pgfqpoint{2.012539in}{1.587523in}}{\pgfqpoint{2.017838in}{1.585327in}}{\pgfqpoint{2.023363in}{1.585327in}}%
\pgfpathclose%
\pgfusepath{fill}%
\end{pgfscope}%
\begin{pgfscope}%
\pgfpathrectangle{\pgfqpoint{1.995685in}{1.588185in}}{\pgfqpoint{1.162500in}{0.755000in}} %
\pgfusepath{clip}%
\pgfsetbuttcap%
\pgfsetroundjoin%
\definecolor{currentfill}{rgb}{0.000000,0.000000,0.000000}%
\pgfsetfillcolor{currentfill}%
\pgfsetfillopacity{0.500000}%
\pgfsetlinewidth{0.000000pt}%
\definecolor{currentstroke}{rgb}{0.000000,0.000000,0.000000}%
\pgfsetstrokecolor{currentstroke}%
\pgfsetdash{}{0pt}%
\pgfpathmoveto{\pgfqpoint{2.907651in}{1.989747in}}%
\pgfpathcurveto{\pgfqpoint{2.913176in}{1.989747in}}{\pgfqpoint{2.918476in}{1.991942in}}{\pgfqpoint{2.922382in}{1.995849in}}%
\pgfpathcurveto{\pgfqpoint{2.926289in}{1.999756in}}{\pgfqpoint{2.928484in}{2.005055in}}{\pgfqpoint{2.928484in}{2.010580in}}%
\pgfpathcurveto{\pgfqpoint{2.928484in}{2.016105in}}{\pgfqpoint{2.926289in}{2.021405in}}{\pgfqpoint{2.922382in}{2.025311in}}%
\pgfpathcurveto{\pgfqpoint{2.918476in}{2.029218in}}{\pgfqpoint{2.913176in}{2.031413in}}{\pgfqpoint{2.907651in}{2.031413in}}%
\pgfpathcurveto{\pgfqpoint{2.902126in}{2.031413in}}{\pgfqpoint{2.896827in}{2.029218in}}{\pgfqpoint{2.892920in}{2.025311in}}%
\pgfpathcurveto{\pgfqpoint{2.889013in}{2.021405in}}{\pgfqpoint{2.886818in}{2.016105in}}{\pgfqpoint{2.886818in}{2.010580in}}%
\pgfpathcurveto{\pgfqpoint{2.886818in}{2.005055in}}{\pgfqpoint{2.889013in}{1.999756in}}{\pgfqpoint{2.892920in}{1.995849in}}%
\pgfpathcurveto{\pgfqpoint{2.896827in}{1.991942in}}{\pgfqpoint{2.902126in}{1.989747in}}{\pgfqpoint{2.907651in}{1.989747in}}%
\pgfpathclose%
\pgfusepath{fill}%
\end{pgfscope}%
\begin{pgfscope}%
\pgfpathrectangle{\pgfqpoint{1.995685in}{1.588185in}}{\pgfqpoint{1.162500in}{0.755000in}} %
\pgfusepath{clip}%
\pgfsetbuttcap%
\pgfsetroundjoin%
\definecolor{currentfill}{rgb}{0.000000,0.000000,0.000000}%
\pgfsetfillcolor{currentfill}%
\pgfsetfillopacity{0.500000}%
\pgfsetlinewidth{0.000000pt}%
\definecolor{currentstroke}{rgb}{0.000000,0.000000,0.000000}%
\pgfsetstrokecolor{currentstroke}%
\pgfsetdash{}{0pt}%
\pgfpathmoveto{\pgfqpoint{2.630361in}{1.867889in}}%
\pgfpathcurveto{\pgfqpoint{2.635886in}{1.867889in}}{\pgfqpoint{2.641186in}{1.870084in}}{\pgfqpoint{2.645093in}{1.873991in}}%
\pgfpathcurveto{\pgfqpoint{2.649000in}{1.877898in}}{\pgfqpoint{2.651195in}{1.883197in}}{\pgfqpoint{2.651195in}{1.888722in}}%
\pgfpathcurveto{\pgfqpoint{2.651195in}{1.894247in}}{\pgfqpoint{2.649000in}{1.899547in}}{\pgfqpoint{2.645093in}{1.903454in}}%
\pgfpathcurveto{\pgfqpoint{2.641186in}{1.907360in}}{\pgfqpoint{2.635886in}{1.909556in}}{\pgfqpoint{2.630361in}{1.909556in}}%
\pgfpathcurveto{\pgfqpoint{2.624836in}{1.909556in}}{\pgfqpoint{2.619537in}{1.907360in}}{\pgfqpoint{2.615630in}{1.903454in}}%
\pgfpathcurveto{\pgfqpoint{2.611723in}{1.899547in}}{\pgfqpoint{2.609528in}{1.894247in}}{\pgfqpoint{2.609528in}{1.888722in}}%
\pgfpathcurveto{\pgfqpoint{2.609528in}{1.883197in}}{\pgfqpoint{2.611723in}{1.877898in}}{\pgfqpoint{2.615630in}{1.873991in}}%
\pgfpathcurveto{\pgfqpoint{2.619537in}{1.870084in}}{\pgfqpoint{2.624836in}{1.867889in}}{\pgfqpoint{2.630361in}{1.867889in}}%
\pgfpathclose%
\pgfusepath{fill}%
\end{pgfscope}%
\begin{pgfscope}%
\pgfpathrectangle{\pgfqpoint{1.995685in}{1.588185in}}{\pgfqpoint{1.162500in}{0.755000in}} %
\pgfusepath{clip}%
\pgfsetbuttcap%
\pgfsetroundjoin%
\definecolor{currentfill}{rgb}{0.000000,0.000000,0.000000}%
\pgfsetfillcolor{currentfill}%
\pgfsetfillopacity{0.500000}%
\pgfsetlinewidth{0.000000pt}%
\definecolor{currentstroke}{rgb}{0.000000,0.000000,0.000000}%
\pgfsetstrokecolor{currentstroke}%
\pgfsetdash{}{0pt}%
\pgfpathmoveto{\pgfqpoint{2.487752in}{1.854418in}}%
\pgfpathcurveto{\pgfqpoint{2.493277in}{1.854418in}}{\pgfqpoint{2.498576in}{1.856613in}}{\pgfqpoint{2.502483in}{1.860520in}}%
\pgfpathcurveto{\pgfqpoint{2.506390in}{1.864427in}}{\pgfqpoint{2.508585in}{1.869726in}}{\pgfqpoint{2.508585in}{1.875251in}}%
\pgfpathcurveto{\pgfqpoint{2.508585in}{1.880776in}}{\pgfqpoint{2.506390in}{1.886076in}}{\pgfqpoint{2.502483in}{1.889982in}}%
\pgfpathcurveto{\pgfqpoint{2.498576in}{1.893889in}}{\pgfqpoint{2.493277in}{1.896084in}}{\pgfqpoint{2.487752in}{1.896084in}}%
\pgfpathcurveto{\pgfqpoint{2.482227in}{1.896084in}}{\pgfqpoint{2.476927in}{1.893889in}}{\pgfqpoint{2.473020in}{1.889982in}}%
\pgfpathcurveto{\pgfqpoint{2.469113in}{1.886076in}}{\pgfqpoint{2.466918in}{1.880776in}}{\pgfqpoint{2.466918in}{1.875251in}}%
\pgfpathcurveto{\pgfqpoint{2.466918in}{1.869726in}}{\pgfqpoint{2.469113in}{1.864427in}}{\pgfqpoint{2.473020in}{1.860520in}}%
\pgfpathcurveto{\pgfqpoint{2.476927in}{1.856613in}}{\pgfqpoint{2.482227in}{1.854418in}}{\pgfqpoint{2.487752in}{1.854418in}}%
\pgfpathclose%
\pgfusepath{fill}%
\end{pgfscope}%
\begin{pgfscope}%
\pgfpathrectangle{\pgfqpoint{1.995685in}{1.588185in}}{\pgfqpoint{1.162500in}{0.755000in}} %
\pgfusepath{clip}%
\pgfsetbuttcap%
\pgfsetroundjoin%
\definecolor{currentfill}{rgb}{0.000000,0.000000,0.000000}%
\pgfsetfillcolor{currentfill}%
\pgfsetfillopacity{0.500000}%
\pgfsetlinewidth{0.000000pt}%
\definecolor{currentstroke}{rgb}{0.000000,0.000000,0.000000}%
\pgfsetstrokecolor{currentstroke}%
\pgfsetdash{}{0pt}%
\pgfpathmoveto{\pgfqpoint{3.000738in}{2.049107in}}%
\pgfpathcurveto{\pgfqpoint{3.006263in}{2.049107in}}{\pgfqpoint{3.011563in}{2.051302in}}{\pgfqpoint{3.015470in}{2.055209in}}%
\pgfpathcurveto{\pgfqpoint{3.019376in}{2.059115in}}{\pgfqpoint{3.021572in}{2.064415in}}{\pgfqpoint{3.021572in}{2.069940in}}%
\pgfpathcurveto{\pgfqpoint{3.021572in}{2.075465in}}{\pgfqpoint{3.019376in}{2.080764in}}{\pgfqpoint{3.015470in}{2.084671in}}%
\pgfpathcurveto{\pgfqpoint{3.011563in}{2.088578in}}{\pgfqpoint{3.006263in}{2.090773in}}{\pgfqpoint{3.000738in}{2.090773in}}%
\pgfpathcurveto{\pgfqpoint{2.995213in}{2.090773in}}{\pgfqpoint{2.989914in}{2.088578in}}{\pgfqpoint{2.986007in}{2.084671in}}%
\pgfpathcurveto{\pgfqpoint{2.982100in}{2.080764in}}{\pgfqpoint{2.979905in}{2.075465in}}{\pgfqpoint{2.979905in}{2.069940in}}%
\pgfpathcurveto{\pgfqpoint{2.979905in}{2.064415in}}{\pgfqpoint{2.982100in}{2.059115in}}{\pgfqpoint{2.986007in}{2.055209in}}%
\pgfpathcurveto{\pgfqpoint{2.989914in}{2.051302in}}{\pgfqpoint{2.995213in}{2.049107in}}{\pgfqpoint{3.000738in}{2.049107in}}%
\pgfpathclose%
\pgfusepath{fill}%
\end{pgfscope}%
\begin{pgfscope}%
\pgfpathrectangle{\pgfqpoint{1.995685in}{1.588185in}}{\pgfqpoint{1.162500in}{0.755000in}} %
\pgfusepath{clip}%
\pgfsetbuttcap%
\pgfsetroundjoin%
\definecolor{currentfill}{rgb}{0.000000,0.000000,0.000000}%
\pgfsetfillcolor{currentfill}%
\pgfsetfillopacity{0.500000}%
\pgfsetlinewidth{0.000000pt}%
\definecolor{currentstroke}{rgb}{0.000000,0.000000,0.000000}%
\pgfsetstrokecolor{currentstroke}%
\pgfsetdash{}{0pt}%
\pgfpathmoveto{\pgfqpoint{2.318352in}{1.703826in}}%
\pgfpathcurveto{\pgfqpoint{2.323877in}{1.703826in}}{\pgfqpoint{2.329176in}{1.706022in}}{\pgfqpoint{2.333083in}{1.709928in}}%
\pgfpathcurveto{\pgfqpoint{2.336990in}{1.713835in}}{\pgfqpoint{2.339185in}{1.719135in}}{\pgfqpoint{2.339185in}{1.724660in}}%
\pgfpathcurveto{\pgfqpoint{2.339185in}{1.730185in}}{\pgfqpoint{2.336990in}{1.735484in}}{\pgfqpoint{2.333083in}{1.739391in}}%
\pgfpathcurveto{\pgfqpoint{2.329176in}{1.743298in}}{\pgfqpoint{2.323877in}{1.745493in}}{\pgfqpoint{2.318352in}{1.745493in}}%
\pgfpathcurveto{\pgfqpoint{2.312827in}{1.745493in}}{\pgfqpoint{2.307527in}{1.743298in}}{\pgfqpoint{2.303620in}{1.739391in}}%
\pgfpathcurveto{\pgfqpoint{2.299713in}{1.735484in}}{\pgfqpoint{2.297518in}{1.730185in}}{\pgfqpoint{2.297518in}{1.724660in}}%
\pgfpathcurveto{\pgfqpoint{2.297518in}{1.719135in}}{\pgfqpoint{2.299713in}{1.713835in}}{\pgfqpoint{2.303620in}{1.709928in}}%
\pgfpathcurveto{\pgfqpoint{2.307527in}{1.706022in}}{\pgfqpoint{2.312827in}{1.703826in}}{\pgfqpoint{2.318352in}{1.703826in}}%
\pgfpathclose%
\pgfusepath{fill}%
\end{pgfscope}%
\begin{pgfscope}%
\pgfpathrectangle{\pgfqpoint{1.995685in}{1.588185in}}{\pgfqpoint{1.162500in}{0.755000in}} %
\pgfusepath{clip}%
\pgfsetbuttcap%
\pgfsetroundjoin%
\definecolor{currentfill}{rgb}{0.000000,0.000000,0.000000}%
\pgfsetfillcolor{currentfill}%
\pgfsetfillopacity{0.500000}%
\pgfsetlinewidth{0.000000pt}%
\definecolor{currentstroke}{rgb}{0.000000,0.000000,0.000000}%
\pgfsetstrokecolor{currentstroke}%
\pgfsetdash{}{0pt}%
\pgfpathmoveto{\pgfqpoint{2.410257in}{1.734907in}}%
\pgfpathcurveto{\pgfqpoint{2.415782in}{1.734907in}}{\pgfqpoint{2.421082in}{1.737102in}}{\pgfqpoint{2.424989in}{1.741009in}}%
\pgfpathcurveto{\pgfqpoint{2.428896in}{1.744915in}}{\pgfqpoint{2.431091in}{1.750215in}}{\pgfqpoint{2.431091in}{1.755740in}}%
\pgfpathcurveto{\pgfqpoint{2.431091in}{1.761265in}}{\pgfqpoint{2.428896in}{1.766565in}}{\pgfqpoint{2.424989in}{1.770471in}}%
\pgfpathcurveto{\pgfqpoint{2.421082in}{1.774378in}}{\pgfqpoint{2.415782in}{1.776573in}}{\pgfqpoint{2.410257in}{1.776573in}}%
\pgfpathcurveto{\pgfqpoint{2.404732in}{1.776573in}}{\pgfqpoint{2.399433in}{1.774378in}}{\pgfqpoint{2.395526in}{1.770471in}}%
\pgfpathcurveto{\pgfqpoint{2.391619in}{1.766565in}}{\pgfqpoint{2.389424in}{1.761265in}}{\pgfqpoint{2.389424in}{1.755740in}}%
\pgfpathcurveto{\pgfqpoint{2.389424in}{1.750215in}}{\pgfqpoint{2.391619in}{1.744915in}}{\pgfqpoint{2.395526in}{1.741009in}}%
\pgfpathcurveto{\pgfqpoint{2.399433in}{1.737102in}}{\pgfqpoint{2.404732in}{1.734907in}}{\pgfqpoint{2.410257in}{1.734907in}}%
\pgfpathclose%
\pgfusepath{fill}%
\end{pgfscope}%
\begin{pgfscope}%
\pgfpathrectangle{\pgfqpoint{1.995685in}{1.588185in}}{\pgfqpoint{1.162500in}{0.755000in}} %
\pgfusepath{clip}%
\pgfsetbuttcap%
\pgfsetroundjoin%
\definecolor{currentfill}{rgb}{0.000000,0.000000,0.000000}%
\pgfsetfillcolor{currentfill}%
\pgfsetfillopacity{0.500000}%
\pgfsetlinewidth{0.000000pt}%
\definecolor{currentstroke}{rgb}{0.000000,0.000000,0.000000}%
\pgfsetstrokecolor{currentstroke}%
\pgfsetdash{}{0pt}%
\pgfpathmoveto{\pgfqpoint{2.752801in}{1.977476in}}%
\pgfpathcurveto{\pgfqpoint{2.758327in}{1.977476in}}{\pgfqpoint{2.763626in}{1.979671in}}{\pgfqpoint{2.767533in}{1.983578in}}%
\pgfpathcurveto{\pgfqpoint{2.771440in}{1.987485in}}{\pgfqpoint{2.773635in}{1.992785in}}{\pgfqpoint{2.773635in}{1.998310in}}%
\pgfpathcurveto{\pgfqpoint{2.773635in}{2.003835in}}{\pgfqpoint{2.771440in}{2.009134in}}{\pgfqpoint{2.767533in}{2.013041in}}%
\pgfpathcurveto{\pgfqpoint{2.763626in}{2.016948in}}{\pgfqpoint{2.758327in}{2.019143in}}{\pgfqpoint{2.752801in}{2.019143in}}%
\pgfpathcurveto{\pgfqpoint{2.747276in}{2.019143in}}{\pgfqpoint{2.741977in}{2.016948in}}{\pgfqpoint{2.738070in}{2.013041in}}%
\pgfpathcurveto{\pgfqpoint{2.734163in}{2.009134in}}{\pgfqpoint{2.731968in}{2.003835in}}{\pgfqpoint{2.731968in}{1.998310in}}%
\pgfpathcurveto{\pgfqpoint{2.731968in}{1.992785in}}{\pgfqpoint{2.734163in}{1.987485in}}{\pgfqpoint{2.738070in}{1.983578in}}%
\pgfpathcurveto{\pgfqpoint{2.741977in}{1.979671in}}{\pgfqpoint{2.747276in}{1.977476in}}{\pgfqpoint{2.752801in}{1.977476in}}%
\pgfpathclose%
\pgfusepath{fill}%
\end{pgfscope}%
\begin{pgfscope}%
\pgfpathrectangle{\pgfqpoint{1.995685in}{1.588185in}}{\pgfqpoint{1.162500in}{0.755000in}} %
\pgfusepath{clip}%
\pgfsetbuttcap%
\pgfsetroundjoin%
\definecolor{currentfill}{rgb}{0.000000,0.000000,0.000000}%
\pgfsetfillcolor{currentfill}%
\pgfsetfillopacity{0.500000}%
\pgfsetlinewidth{0.000000pt}%
\definecolor{currentstroke}{rgb}{0.000000,0.000000,0.000000}%
\pgfsetstrokecolor{currentstroke}%
\pgfsetdash{}{0pt}%
\pgfpathmoveto{\pgfqpoint{2.103337in}{1.614683in}}%
\pgfpathcurveto{\pgfqpoint{2.108862in}{1.614683in}}{\pgfqpoint{2.114161in}{1.616879in}}{\pgfqpoint{2.118068in}{1.620785in}}%
\pgfpathcurveto{\pgfqpoint{2.121975in}{1.624692in}}{\pgfqpoint{2.124170in}{1.629992in}}{\pgfqpoint{2.124170in}{1.635517in}}%
\pgfpathcurveto{\pgfqpoint{2.124170in}{1.641042in}}{\pgfqpoint{2.121975in}{1.646341in}}{\pgfqpoint{2.118068in}{1.650248in}}%
\pgfpathcurveto{\pgfqpoint{2.114161in}{1.654155in}}{\pgfqpoint{2.108862in}{1.656350in}}{\pgfqpoint{2.103337in}{1.656350in}}%
\pgfpathcurveto{\pgfqpoint{2.097812in}{1.656350in}}{\pgfqpoint{2.092512in}{1.654155in}}{\pgfqpoint{2.088605in}{1.650248in}}%
\pgfpathcurveto{\pgfqpoint{2.084698in}{1.646341in}}{\pgfqpoint{2.082503in}{1.641042in}}{\pgfqpoint{2.082503in}{1.635517in}}%
\pgfpathcurveto{\pgfqpoint{2.082503in}{1.629992in}}{\pgfqpoint{2.084698in}{1.624692in}}{\pgfqpoint{2.088605in}{1.620785in}}%
\pgfpathcurveto{\pgfqpoint{2.092512in}{1.616879in}}{\pgfqpoint{2.097812in}{1.614683in}}{\pgfqpoint{2.103337in}{1.614683in}}%
\pgfpathclose%
\pgfusepath{fill}%
\end{pgfscope}%
\begin{pgfscope}%
\pgfsetrectcap%
\pgfsetmiterjoin%
\pgfsetlinewidth{0.803000pt}%
\definecolor{currentstroke}{rgb}{0.000000,0.000000,0.000000}%
\pgfsetstrokecolor{currentstroke}%
\pgfsetdash{}{0pt}%
\pgfpathmoveto{\pgfqpoint{1.995685in}{1.588185in}}%
\pgfpathlineto{\pgfqpoint{1.995685in}{2.343185in}}%
\pgfusepath{stroke}%
\end{pgfscope}%
\begin{pgfscope}%
\pgfsetrectcap%
\pgfsetmiterjoin%
\pgfsetlinewidth{0.803000pt}%
\definecolor{currentstroke}{rgb}{0.000000,0.000000,0.000000}%
\pgfsetstrokecolor{currentstroke}%
\pgfsetdash{}{0pt}%
\pgfpathmoveto{\pgfqpoint{3.158185in}{1.588185in}}%
\pgfpathlineto{\pgfqpoint{3.158185in}{2.343185in}}%
\pgfusepath{stroke}%
\end{pgfscope}%
\begin{pgfscope}%
\pgfsetrectcap%
\pgfsetmiterjoin%
\pgfsetlinewidth{0.803000pt}%
\definecolor{currentstroke}{rgb}{0.000000,0.000000,0.000000}%
\pgfsetstrokecolor{currentstroke}%
\pgfsetdash{}{0pt}%
\pgfpathmoveto{\pgfqpoint{1.995685in}{1.588185in}}%
\pgfpathlineto{\pgfqpoint{3.158185in}{1.588185in}}%
\pgfusepath{stroke}%
\end{pgfscope}%
\begin{pgfscope}%
\pgfsetrectcap%
\pgfsetmiterjoin%
\pgfsetlinewidth{0.803000pt}%
\definecolor{currentstroke}{rgb}{0.000000,0.000000,0.000000}%
\pgfsetstrokecolor{currentstroke}%
\pgfsetdash{}{0pt}%
\pgfpathmoveto{\pgfqpoint{1.995685in}{2.343185in}}%
\pgfpathlineto{\pgfqpoint{3.158185in}{2.343185in}}%
\pgfusepath{stroke}%
\end{pgfscope}%
\begin{pgfscope}%
\pgfsetbuttcap%
\pgfsetmiterjoin%
\definecolor{currentfill}{rgb}{1.000000,1.000000,1.000000}%
\pgfsetfillcolor{currentfill}%
\pgfsetlinewidth{0.000000pt}%
\definecolor{currentstroke}{rgb}{0.000000,0.000000,0.000000}%
\pgfsetstrokecolor{currentstroke}%
\pgfsetstrokeopacity{0.000000}%
\pgfsetdash{}{0pt}%
\pgfpathmoveto{\pgfqpoint{3.158185in}{1.588185in}}%
\pgfpathlineto{\pgfqpoint{4.320685in}{1.588185in}}%
\pgfpathlineto{\pgfqpoint{4.320685in}{2.343185in}}%
\pgfpathlineto{\pgfqpoint{3.158185in}{2.343185in}}%
\pgfpathclose%
\pgfusepath{fill}%
\end{pgfscope}%
\begin{pgfscope}%
\pgfpathrectangle{\pgfqpoint{3.158185in}{1.588185in}}{\pgfqpoint{1.162500in}{0.755000in}} %
\pgfusepath{clip}%
\pgfsetrectcap%
\pgfsetroundjoin%
\pgfsetlinewidth{1.505625pt}%
\definecolor{currentstroke}{rgb}{0.121569,0.466667,0.705882}%
\pgfsetstrokecolor{currentstroke}%
\pgfsetdash{}{0pt}%
\pgfpathmoveto{\pgfqpoint{3.185863in}{2.104447in}}%
\pgfpathlineto{\pgfqpoint{3.211353in}{2.150561in}}%
\pgfpathlineto{\pgfqpoint{3.233518in}{2.186403in}}%
\pgfpathlineto{\pgfqpoint{3.254575in}{2.216289in}}%
\pgfpathlineto{\pgfqpoint{3.273415in}{2.239364in}}%
\pgfpathlineto{\pgfqpoint{3.292255in}{2.258893in}}%
\pgfpathlineto{\pgfqpoint{3.309987in}{2.274058in}}%
\pgfpathlineto{\pgfqpoint{3.327719in}{2.286206in}}%
\pgfpathlineto{\pgfqpoint{3.344343in}{2.294999in}}%
\pgfpathlineto{\pgfqpoint{3.362075in}{2.301807in}}%
\pgfpathlineto{\pgfqpoint{3.379807in}{2.306200in}}%
\pgfpathlineto{\pgfqpoint{3.398647in}{2.308514in}}%
\pgfpathlineto{\pgfqpoint{3.418596in}{2.308664in}}%
\pgfpathlineto{\pgfqpoint{3.440761in}{2.306475in}}%
\pgfpathlineto{\pgfqpoint{3.465142in}{2.301687in}}%
\pgfpathlineto{\pgfqpoint{3.491740in}{2.294102in}}%
\pgfpathlineto{\pgfqpoint{3.520555in}{2.283496in}}%
\pgfpathlineto{\pgfqpoint{3.550478in}{2.270030in}}%
\pgfpathlineto{\pgfqpoint{3.579292in}{2.254637in}}%
\pgfpathlineto{\pgfqpoint{3.606999in}{2.237399in}}%
\pgfpathlineto{\pgfqpoint{3.634705in}{2.217533in}}%
\pgfpathlineto{\pgfqpoint{3.662411in}{2.194829in}}%
\pgfpathlineto{\pgfqpoint{3.690117in}{2.169146in}}%
\pgfpathlineto{\pgfqpoint{3.717824in}{2.140453in}}%
\pgfpathlineto{\pgfqpoint{3.747746in}{2.106189in}}%
\pgfpathlineto{\pgfqpoint{3.779886in}{2.065923in}}%
\pgfpathlineto{\pgfqpoint{3.817566in}{2.014928in}}%
\pgfpathlineto{\pgfqpoint{3.869654in}{1.940191in}}%
\pgfpathlineto{\pgfqpoint{3.940582in}{1.838745in}}%
\pgfpathlineto{\pgfqpoint{3.974938in}{1.793588in}}%
\pgfpathlineto{\pgfqpoint{4.003752in}{1.759310in}}%
\pgfpathlineto{\pgfqpoint{4.029242in}{1.732355in}}%
\pgfpathlineto{\pgfqpoint{4.052516in}{1.710813in}}%
\pgfpathlineto{\pgfqpoint{4.074681in}{1.693141in}}%
\pgfpathlineto{\pgfqpoint{4.096846in}{1.678234in}}%
\pgfpathlineto{\pgfqpoint{4.119011in}{1.665964in}}%
\pgfpathlineto{\pgfqpoint{4.142284in}{1.655659in}}%
\pgfpathlineto{\pgfqpoint{4.166665in}{1.647288in}}%
\pgfpathlineto{\pgfqpoint{4.195480in}{1.639872in}}%
\pgfpathlineto{\pgfqpoint{4.233160in}{1.632692in}}%
\pgfpathlineto{\pgfqpoint{4.293006in}{1.622503in}}%
\pgfpathlineto{\pgfqpoint{4.293006in}{1.622503in}}%
\pgfusepath{stroke}%
\end{pgfscope}%
\begin{pgfscope}%
\pgfsetrectcap%
\pgfsetmiterjoin%
\pgfsetlinewidth{0.803000pt}%
\definecolor{currentstroke}{rgb}{0.000000,0.000000,0.000000}%
\pgfsetstrokecolor{currentstroke}%
\pgfsetdash{}{0pt}%
\pgfpathmoveto{\pgfqpoint{3.158185in}{1.588185in}}%
\pgfpathlineto{\pgfqpoint{3.158185in}{2.343185in}}%
\pgfusepath{stroke}%
\end{pgfscope}%
\begin{pgfscope}%
\pgfsetrectcap%
\pgfsetmiterjoin%
\pgfsetlinewidth{0.803000pt}%
\definecolor{currentstroke}{rgb}{0.000000,0.000000,0.000000}%
\pgfsetstrokecolor{currentstroke}%
\pgfsetdash{}{0pt}%
\pgfpathmoveto{\pgfqpoint{4.320685in}{1.588185in}}%
\pgfpathlineto{\pgfqpoint{4.320685in}{2.343185in}}%
\pgfusepath{stroke}%
\end{pgfscope}%
\begin{pgfscope}%
\pgfsetrectcap%
\pgfsetmiterjoin%
\pgfsetlinewidth{0.803000pt}%
\definecolor{currentstroke}{rgb}{0.000000,0.000000,0.000000}%
\pgfsetstrokecolor{currentstroke}%
\pgfsetdash{}{0pt}%
\pgfpathmoveto{\pgfqpoint{3.158185in}{1.588185in}}%
\pgfpathlineto{\pgfqpoint{4.320685in}{1.588185in}}%
\pgfusepath{stroke}%
\end{pgfscope}%
\begin{pgfscope}%
\pgfsetrectcap%
\pgfsetmiterjoin%
\pgfsetlinewidth{0.803000pt}%
\definecolor{currentstroke}{rgb}{0.000000,0.000000,0.000000}%
\pgfsetstrokecolor{currentstroke}%
\pgfsetdash{}{0pt}%
\pgfpathmoveto{\pgfqpoint{3.158185in}{2.343185in}}%
\pgfpathlineto{\pgfqpoint{4.320685in}{2.343185in}}%
\pgfusepath{stroke}%
\end{pgfscope}%
\begin{pgfscope}%
\pgfsetbuttcap%
\pgfsetmiterjoin%
\definecolor{currentfill}{rgb}{1.000000,1.000000,1.000000}%
\pgfsetfillcolor{currentfill}%
\pgfsetlinewidth{0.000000pt}%
\definecolor{currentstroke}{rgb}{0.000000,0.000000,0.000000}%
\pgfsetstrokecolor{currentstroke}%
\pgfsetstrokeopacity{0.000000}%
\pgfsetdash{}{0pt}%
\pgfpathmoveto{\pgfqpoint{4.320685in}{1.588185in}}%
\pgfpathlineto{\pgfqpoint{5.483185in}{1.588185in}}%
\pgfpathlineto{\pgfqpoint{5.483185in}{2.343185in}}%
\pgfpathlineto{\pgfqpoint{4.320685in}{2.343185in}}%
\pgfpathclose%
\pgfusepath{fill}%
\end{pgfscope}%
\begin{pgfscope}%
\pgfpathrectangle{\pgfqpoint{4.320685in}{1.588185in}}{\pgfqpoint{1.162500in}{0.755000in}} %
\pgfusepath{clip}%
\pgfsetbuttcap%
\pgfsetroundjoin%
\definecolor{currentfill}{rgb}{0.000000,0.000000,0.000000}%
\pgfsetfillcolor{currentfill}%
\pgfsetfillopacity{0.500000}%
\pgfsetlinewidth{0.000000pt}%
\definecolor{currentstroke}{rgb}{0.000000,0.000000,0.000000}%
\pgfsetstrokecolor{currentstroke}%
\pgfsetdash{}{0pt}%
\pgfpathmoveto{\pgfqpoint{4.763543in}{2.304375in}}%
\pgfpathcurveto{\pgfqpoint{4.769068in}{2.304375in}}{\pgfqpoint{4.774367in}{2.306570in}}{\pgfqpoint{4.778274in}{2.310477in}}%
\pgfpathcurveto{\pgfqpoint{4.782181in}{2.314384in}}{\pgfqpoint{4.784376in}{2.319683in}}{\pgfqpoint{4.784376in}{2.325208in}}%
\pgfpathcurveto{\pgfqpoint{4.784376in}{2.330733in}}{\pgfqpoint{4.782181in}{2.336033in}}{\pgfqpoint{4.778274in}{2.339940in}}%
\pgfpathcurveto{\pgfqpoint{4.774367in}{2.343847in}}{\pgfqpoint{4.769068in}{2.346042in}}{\pgfqpoint{4.763543in}{2.346042in}}%
\pgfpathcurveto{\pgfqpoint{4.758018in}{2.346042in}}{\pgfqpoint{4.752718in}{2.343847in}}{\pgfqpoint{4.748811in}{2.339940in}}%
\pgfpathcurveto{\pgfqpoint{4.744905in}{2.336033in}}{\pgfqpoint{4.742709in}{2.330733in}}{\pgfqpoint{4.742709in}{2.325208in}}%
\pgfpathcurveto{\pgfqpoint{4.742709in}{2.319683in}}{\pgfqpoint{4.744905in}{2.314384in}}{\pgfqpoint{4.748811in}{2.310477in}}%
\pgfpathcurveto{\pgfqpoint{4.752718in}{2.306570in}}{\pgfqpoint{4.758018in}{2.304375in}}{\pgfqpoint{4.763543in}{2.304375in}}%
\pgfpathclose%
\pgfusepath{fill}%
\end{pgfscope}%
\begin{pgfscope}%
\pgfpathrectangle{\pgfqpoint{4.320685in}{1.588185in}}{\pgfqpoint{1.162500in}{0.755000in}} %
\pgfusepath{clip}%
\pgfsetbuttcap%
\pgfsetroundjoin%
\definecolor{currentfill}{rgb}{0.000000,0.000000,0.000000}%
\pgfsetfillcolor{currentfill}%
\pgfsetfillopacity{0.500000}%
\pgfsetlinewidth{0.000000pt}%
\definecolor{currentstroke}{rgb}{0.000000,0.000000,0.000000}%
\pgfsetstrokecolor{currentstroke}%
\pgfsetdash{}{0pt}%
\pgfpathmoveto{\pgfqpoint{5.385388in}{1.845741in}}%
\pgfpathcurveto{\pgfqpoint{5.390913in}{1.845741in}}{\pgfqpoint{5.396213in}{1.847936in}}{\pgfqpoint{5.400119in}{1.851843in}}%
\pgfpathcurveto{\pgfqpoint{5.404026in}{1.855750in}}{\pgfqpoint{5.406221in}{1.861049in}}{\pgfqpoint{5.406221in}{1.866574in}}%
\pgfpathcurveto{\pgfqpoint{5.406221in}{1.872099in}}{\pgfqpoint{5.404026in}{1.877399in}}{\pgfqpoint{5.400119in}{1.881306in}}%
\pgfpathcurveto{\pgfqpoint{5.396213in}{1.885212in}}{\pgfqpoint{5.390913in}{1.887408in}}{\pgfqpoint{5.385388in}{1.887408in}}%
\pgfpathcurveto{\pgfqpoint{5.379863in}{1.887408in}}{\pgfqpoint{5.374564in}{1.885212in}}{\pgfqpoint{5.370657in}{1.881306in}}%
\pgfpathcurveto{\pgfqpoint{5.366750in}{1.877399in}}{\pgfqpoint{5.364555in}{1.872099in}}{\pgfqpoint{5.364555in}{1.866574in}}%
\pgfpathcurveto{\pgfqpoint{5.364555in}{1.861049in}}{\pgfqpoint{5.366750in}{1.855750in}}{\pgfqpoint{5.370657in}{1.851843in}}%
\pgfpathcurveto{\pgfqpoint{5.374564in}{1.847936in}}{\pgfqpoint{5.379863in}{1.845741in}}{\pgfqpoint{5.385388in}{1.845741in}}%
\pgfpathclose%
\pgfusepath{fill}%
\end{pgfscope}%
\begin{pgfscope}%
\pgfpathrectangle{\pgfqpoint{4.320685in}{1.588185in}}{\pgfqpoint{1.162500in}{0.755000in}} %
\pgfusepath{clip}%
\pgfsetbuttcap%
\pgfsetroundjoin%
\definecolor{currentfill}{rgb}{0.000000,0.000000,0.000000}%
\pgfsetfillcolor{currentfill}%
\pgfsetfillopacity{0.500000}%
\pgfsetlinewidth{0.000000pt}%
\definecolor{currentstroke}{rgb}{0.000000,0.000000,0.000000}%
\pgfsetstrokecolor{currentstroke}%
\pgfsetdash{}{0pt}%
\pgfpathmoveto{\pgfqpoint{5.455506in}{1.840570in}}%
\pgfpathcurveto{\pgfqpoint{5.461031in}{1.840570in}}{\pgfqpoint{5.466331in}{1.842766in}}{\pgfqpoint{5.470237in}{1.846672in}}%
\pgfpathcurveto{\pgfqpoint{5.474144in}{1.850579in}}{\pgfqpoint{5.476339in}{1.855879in}}{\pgfqpoint{5.476339in}{1.861404in}}%
\pgfpathcurveto{\pgfqpoint{5.476339in}{1.866929in}}{\pgfqpoint{5.474144in}{1.872228in}}{\pgfqpoint{5.470237in}{1.876135in}}%
\pgfpathcurveto{\pgfqpoint{5.466331in}{1.880042in}}{\pgfqpoint{5.461031in}{1.882237in}}{\pgfqpoint{5.455506in}{1.882237in}}%
\pgfpathcurveto{\pgfqpoint{5.449981in}{1.882237in}}{\pgfqpoint{5.444681in}{1.880042in}}{\pgfqpoint{5.440775in}{1.876135in}}%
\pgfpathcurveto{\pgfqpoint{5.436868in}{1.872228in}}{\pgfqpoint{5.434673in}{1.866929in}}{\pgfqpoint{5.434673in}{1.861404in}}%
\pgfpathcurveto{\pgfqpoint{5.434673in}{1.855879in}}{\pgfqpoint{5.436868in}{1.850579in}}{\pgfqpoint{5.440775in}{1.846672in}}%
\pgfpathcurveto{\pgfqpoint{5.444681in}{1.842766in}}{\pgfqpoint{5.449981in}{1.840570in}}{\pgfqpoint{5.455506in}{1.840570in}}%
\pgfpathclose%
\pgfusepath{fill}%
\end{pgfscope}%
\begin{pgfscope}%
\pgfpathrectangle{\pgfqpoint{4.320685in}{1.588185in}}{\pgfqpoint{1.162500in}{0.755000in}} %
\pgfusepath{clip}%
\pgfsetbuttcap%
\pgfsetroundjoin%
\definecolor{currentfill}{rgb}{0.000000,0.000000,0.000000}%
\pgfsetfillcolor{currentfill}%
\pgfsetfillopacity{0.500000}%
\pgfsetlinewidth{0.000000pt}%
\definecolor{currentstroke}{rgb}{0.000000,0.000000,0.000000}%
\pgfsetstrokecolor{currentstroke}%
\pgfsetdash{}{0pt}%
\pgfpathmoveto{\pgfqpoint{5.031128in}{1.680932in}}%
\pgfpathcurveto{\pgfqpoint{5.036653in}{1.680932in}}{\pgfqpoint{5.041953in}{1.683127in}}{\pgfqpoint{5.045859in}{1.687034in}}%
\pgfpathcurveto{\pgfqpoint{5.049766in}{1.690941in}}{\pgfqpoint{5.051961in}{1.696241in}}{\pgfqpoint{5.051961in}{1.701766in}}%
\pgfpathcurveto{\pgfqpoint{5.051961in}{1.707291in}}{\pgfqpoint{5.049766in}{1.712590in}}{\pgfqpoint{5.045859in}{1.716497in}}%
\pgfpathcurveto{\pgfqpoint{5.041953in}{1.720404in}}{\pgfqpoint{5.036653in}{1.722599in}}{\pgfqpoint{5.031128in}{1.722599in}}%
\pgfpathcurveto{\pgfqpoint{5.025603in}{1.722599in}}{\pgfqpoint{5.020303in}{1.720404in}}{\pgfqpoint{5.016397in}{1.716497in}}%
\pgfpathcurveto{\pgfqpoint{5.012490in}{1.712590in}}{\pgfqpoint{5.010295in}{1.707291in}}{\pgfqpoint{5.010295in}{1.701766in}}%
\pgfpathcurveto{\pgfqpoint{5.010295in}{1.696241in}}{\pgfqpoint{5.012490in}{1.690941in}}{\pgfqpoint{5.016397in}{1.687034in}}%
\pgfpathcurveto{\pgfqpoint{5.020303in}{1.683127in}}{\pgfqpoint{5.025603in}{1.680932in}}{\pgfqpoint{5.031128in}{1.680932in}}%
\pgfpathclose%
\pgfusepath{fill}%
\end{pgfscope}%
\begin{pgfscope}%
\pgfpathrectangle{\pgfqpoint{4.320685in}{1.588185in}}{\pgfqpoint{1.162500in}{0.755000in}} %
\pgfusepath{clip}%
\pgfsetbuttcap%
\pgfsetroundjoin%
\definecolor{currentfill}{rgb}{0.000000,0.000000,0.000000}%
\pgfsetfillcolor{currentfill}%
\pgfsetfillopacity{0.500000}%
\pgfsetlinewidth{0.000000pt}%
\definecolor{currentstroke}{rgb}{0.000000,0.000000,0.000000}%
\pgfsetstrokecolor{currentstroke}%
\pgfsetdash{}{0pt}%
\pgfpathmoveto{\pgfqpoint{5.102087in}{1.656866in}}%
\pgfpathcurveto{\pgfqpoint{5.107612in}{1.656866in}}{\pgfqpoint{5.112912in}{1.659061in}}{\pgfqpoint{5.116819in}{1.662968in}}%
\pgfpathcurveto{\pgfqpoint{5.120726in}{1.666874in}}{\pgfqpoint{5.122921in}{1.672174in}}{\pgfqpoint{5.122921in}{1.677699in}}%
\pgfpathcurveto{\pgfqpoint{5.122921in}{1.683224in}}{\pgfqpoint{5.120726in}{1.688523in}}{\pgfqpoint{5.116819in}{1.692430in}}%
\pgfpathcurveto{\pgfqpoint{5.112912in}{1.696337in}}{\pgfqpoint{5.107612in}{1.698532in}}{\pgfqpoint{5.102087in}{1.698532in}}%
\pgfpathcurveto{\pgfqpoint{5.096562in}{1.698532in}}{\pgfqpoint{5.091263in}{1.696337in}}{\pgfqpoint{5.087356in}{1.692430in}}%
\pgfpathcurveto{\pgfqpoint{5.083449in}{1.688523in}}{\pgfqpoint{5.081254in}{1.683224in}}{\pgfqpoint{5.081254in}{1.677699in}}%
\pgfpathcurveto{\pgfqpoint{5.081254in}{1.672174in}}{\pgfqpoint{5.083449in}{1.666874in}}{\pgfqpoint{5.087356in}{1.662968in}}%
\pgfpathcurveto{\pgfqpoint{5.091263in}{1.659061in}}{\pgfqpoint{5.096562in}{1.656866in}}{\pgfqpoint{5.102087in}{1.656866in}}%
\pgfpathclose%
\pgfusepath{fill}%
\end{pgfscope}%
\begin{pgfscope}%
\pgfpathrectangle{\pgfqpoint{4.320685in}{1.588185in}}{\pgfqpoint{1.162500in}{0.755000in}} %
\pgfusepath{clip}%
\pgfsetbuttcap%
\pgfsetroundjoin%
\definecolor{currentfill}{rgb}{0.000000,0.000000,0.000000}%
\pgfsetfillcolor{currentfill}%
\pgfsetfillopacity{0.500000}%
\pgfsetlinewidth{0.000000pt}%
\definecolor{currentstroke}{rgb}{0.000000,0.000000,0.000000}%
\pgfsetstrokecolor{currentstroke}%
\pgfsetdash{}{0pt}%
\pgfpathmoveto{\pgfqpoint{4.940844in}{1.598883in}}%
\pgfpathcurveto{\pgfqpoint{4.946369in}{1.598883in}}{\pgfqpoint{4.951669in}{1.601078in}}{\pgfqpoint{4.955576in}{1.604985in}}%
\pgfpathcurveto{\pgfqpoint{4.959482in}{1.608892in}}{\pgfqpoint{4.961678in}{1.614191in}}{\pgfqpoint{4.961678in}{1.619716in}}%
\pgfpathcurveto{\pgfqpoint{4.961678in}{1.625241in}}{\pgfqpoint{4.959482in}{1.630541in}}{\pgfqpoint{4.955576in}{1.634447in}}%
\pgfpathcurveto{\pgfqpoint{4.951669in}{1.638354in}}{\pgfqpoint{4.946369in}{1.640549in}}{\pgfqpoint{4.940844in}{1.640549in}}%
\pgfpathcurveto{\pgfqpoint{4.935319in}{1.640549in}}{\pgfqpoint{4.930020in}{1.638354in}}{\pgfqpoint{4.926113in}{1.634447in}}%
\pgfpathcurveto{\pgfqpoint{4.922206in}{1.630541in}}{\pgfqpoint{4.920011in}{1.625241in}}{\pgfqpoint{4.920011in}{1.619716in}}%
\pgfpathcurveto{\pgfqpoint{4.920011in}{1.614191in}}{\pgfqpoint{4.922206in}{1.608892in}}{\pgfqpoint{4.926113in}{1.604985in}}%
\pgfpathcurveto{\pgfqpoint{4.930020in}{1.601078in}}{\pgfqpoint{4.935319in}{1.598883in}}{\pgfqpoint{4.940844in}{1.598883in}}%
\pgfpathclose%
\pgfusepath{fill}%
\end{pgfscope}%
\begin{pgfscope}%
\pgfpathrectangle{\pgfqpoint{4.320685in}{1.588185in}}{\pgfqpoint{1.162500in}{0.755000in}} %
\pgfusepath{clip}%
\pgfsetbuttcap%
\pgfsetroundjoin%
\definecolor{currentfill}{rgb}{0.000000,0.000000,0.000000}%
\pgfsetfillcolor{currentfill}%
\pgfsetfillopacity{0.500000}%
\pgfsetlinewidth{0.000000pt}%
\definecolor{currentstroke}{rgb}{0.000000,0.000000,0.000000}%
\pgfsetstrokecolor{currentstroke}%
\pgfsetdash{}{0pt}%
\pgfpathmoveto{\pgfqpoint{4.348363in}{1.585327in}}%
\pgfpathcurveto{\pgfqpoint{4.353888in}{1.585327in}}{\pgfqpoint{4.359188in}{1.587523in}}{\pgfqpoint{4.363095in}{1.591429in}}%
\pgfpathcurveto{\pgfqpoint{4.367001in}{1.595336in}}{\pgfqpoint{4.369196in}{1.600636in}}{\pgfqpoint{4.369196in}{1.606161in}}%
\pgfpathcurveto{\pgfqpoint{4.369196in}{1.611686in}}{\pgfqpoint{4.367001in}{1.616985in}}{\pgfqpoint{4.363095in}{1.620892in}}%
\pgfpathcurveto{\pgfqpoint{4.359188in}{1.624799in}}{\pgfqpoint{4.353888in}{1.626994in}}{\pgfqpoint{4.348363in}{1.626994in}}%
\pgfpathcurveto{\pgfqpoint{4.342838in}{1.626994in}}{\pgfqpoint{4.337539in}{1.624799in}}{\pgfqpoint{4.333632in}{1.620892in}}%
\pgfpathcurveto{\pgfqpoint{4.329725in}{1.616985in}}{\pgfqpoint{4.327530in}{1.611686in}}{\pgfqpoint{4.327530in}{1.606161in}}%
\pgfpathcurveto{\pgfqpoint{4.327530in}{1.600636in}}{\pgfqpoint{4.329725in}{1.595336in}}{\pgfqpoint{4.333632in}{1.591429in}}%
\pgfpathcurveto{\pgfqpoint{4.337539in}{1.587523in}}{\pgfqpoint{4.342838in}{1.585327in}}{\pgfqpoint{4.348363in}{1.585327in}}%
\pgfpathclose%
\pgfusepath{fill}%
\end{pgfscope}%
\begin{pgfscope}%
\pgfpathrectangle{\pgfqpoint{4.320685in}{1.588185in}}{\pgfqpoint{1.162500in}{0.755000in}} %
\pgfusepath{clip}%
\pgfsetbuttcap%
\pgfsetroundjoin%
\definecolor{currentfill}{rgb}{0.000000,0.000000,0.000000}%
\pgfsetfillcolor{currentfill}%
\pgfsetfillopacity{0.500000}%
\pgfsetlinewidth{0.000000pt}%
\definecolor{currentstroke}{rgb}{0.000000,0.000000,0.000000}%
\pgfsetstrokecolor{currentstroke}%
\pgfsetdash{}{0pt}%
\pgfpathmoveto{\pgfqpoint{5.332823in}{1.989747in}}%
\pgfpathcurveto{\pgfqpoint{5.338348in}{1.989747in}}{\pgfqpoint{5.343648in}{1.991942in}}{\pgfqpoint{5.347555in}{1.995849in}}%
\pgfpathcurveto{\pgfqpoint{5.351461in}{1.999756in}}{\pgfqpoint{5.353656in}{2.005055in}}{\pgfqpoint{5.353656in}{2.010580in}}%
\pgfpathcurveto{\pgfqpoint{5.353656in}{2.016105in}}{\pgfqpoint{5.351461in}{2.021405in}}{\pgfqpoint{5.347555in}{2.025311in}}%
\pgfpathcurveto{\pgfqpoint{5.343648in}{2.029218in}}{\pgfqpoint{5.338348in}{2.031413in}}{\pgfqpoint{5.332823in}{2.031413in}}%
\pgfpathcurveto{\pgfqpoint{5.327298in}{2.031413in}}{\pgfqpoint{5.321999in}{2.029218in}}{\pgfqpoint{5.318092in}{2.025311in}}%
\pgfpathcurveto{\pgfqpoint{5.314185in}{2.021405in}}{\pgfqpoint{5.311990in}{2.016105in}}{\pgfqpoint{5.311990in}{2.010580in}}%
\pgfpathcurveto{\pgfqpoint{5.311990in}{2.005055in}}{\pgfqpoint{5.314185in}{1.999756in}}{\pgfqpoint{5.318092in}{1.995849in}}%
\pgfpathcurveto{\pgfqpoint{5.321999in}{1.991942in}}{\pgfqpoint{5.327298in}{1.989747in}}{\pgfqpoint{5.332823in}{1.989747in}}%
\pgfpathclose%
\pgfusepath{fill}%
\end{pgfscope}%
\begin{pgfscope}%
\pgfpathrectangle{\pgfqpoint{4.320685in}{1.588185in}}{\pgfqpoint{1.162500in}{0.755000in}} %
\pgfusepath{clip}%
\pgfsetbuttcap%
\pgfsetroundjoin%
\definecolor{currentfill}{rgb}{0.000000,0.000000,0.000000}%
\pgfsetfillcolor{currentfill}%
\pgfsetfillopacity{0.500000}%
\pgfsetlinewidth{0.000000pt}%
\definecolor{currentstroke}{rgb}{0.000000,0.000000,0.000000}%
\pgfsetstrokecolor{currentstroke}%
\pgfsetdash{}{0pt}%
\pgfpathmoveto{\pgfqpoint{5.090412in}{1.867889in}}%
\pgfpathcurveto{\pgfqpoint{5.095937in}{1.867889in}}{\pgfqpoint{5.101237in}{1.870084in}}{\pgfqpoint{5.105143in}{1.873991in}}%
\pgfpathcurveto{\pgfqpoint{5.109050in}{1.877898in}}{\pgfqpoint{5.111245in}{1.883197in}}{\pgfqpoint{5.111245in}{1.888722in}}%
\pgfpathcurveto{\pgfqpoint{5.111245in}{1.894247in}}{\pgfqpoint{5.109050in}{1.899547in}}{\pgfqpoint{5.105143in}{1.903454in}}%
\pgfpathcurveto{\pgfqpoint{5.101237in}{1.907360in}}{\pgfqpoint{5.095937in}{1.909556in}}{\pgfqpoint{5.090412in}{1.909556in}}%
\pgfpathcurveto{\pgfqpoint{5.084887in}{1.909556in}}{\pgfqpoint{5.079587in}{1.907360in}}{\pgfqpoint{5.075681in}{1.903454in}}%
\pgfpathcurveto{\pgfqpoint{5.071774in}{1.899547in}}{\pgfqpoint{5.069579in}{1.894247in}}{\pgfqpoint{5.069579in}{1.888722in}}%
\pgfpathcurveto{\pgfqpoint{5.069579in}{1.883197in}}{\pgfqpoint{5.071774in}{1.877898in}}{\pgfqpoint{5.075681in}{1.873991in}}%
\pgfpathcurveto{\pgfqpoint{5.079587in}{1.870084in}}{\pgfqpoint{5.084887in}{1.867889in}}{\pgfqpoint{5.090412in}{1.867889in}}%
\pgfpathclose%
\pgfusepath{fill}%
\end{pgfscope}%
\begin{pgfscope}%
\pgfpathrectangle{\pgfqpoint{4.320685in}{1.588185in}}{\pgfqpoint{1.162500in}{0.755000in}} %
\pgfusepath{clip}%
\pgfsetbuttcap%
\pgfsetroundjoin%
\definecolor{currentfill}{rgb}{0.000000,0.000000,0.000000}%
\pgfsetfillcolor{currentfill}%
\pgfsetfillopacity{0.500000}%
\pgfsetlinewidth{0.000000pt}%
\definecolor{currentstroke}{rgb}{0.000000,0.000000,0.000000}%
\pgfsetstrokecolor{currentstroke}%
\pgfsetdash{}{0pt}%
\pgfpathmoveto{\pgfqpoint{4.817426in}{1.854418in}}%
\pgfpathcurveto{\pgfqpoint{4.822951in}{1.854418in}}{\pgfqpoint{4.828251in}{1.856613in}}{\pgfqpoint{4.832158in}{1.860520in}}%
\pgfpathcurveto{\pgfqpoint{4.836065in}{1.864427in}}{\pgfqpoint{4.838260in}{1.869726in}}{\pgfqpoint{4.838260in}{1.875251in}}%
\pgfpathcurveto{\pgfqpoint{4.838260in}{1.880776in}}{\pgfqpoint{4.836065in}{1.886076in}}{\pgfqpoint{4.832158in}{1.889982in}}%
\pgfpathcurveto{\pgfqpoint{4.828251in}{1.893889in}}{\pgfqpoint{4.822951in}{1.896084in}}{\pgfqpoint{4.817426in}{1.896084in}}%
\pgfpathcurveto{\pgfqpoint{4.811901in}{1.896084in}}{\pgfqpoint{4.806602in}{1.893889in}}{\pgfqpoint{4.802695in}{1.889982in}}%
\pgfpathcurveto{\pgfqpoint{4.798788in}{1.886076in}}{\pgfqpoint{4.796593in}{1.880776in}}{\pgfqpoint{4.796593in}{1.875251in}}%
\pgfpathcurveto{\pgfqpoint{4.796593in}{1.869726in}}{\pgfqpoint{4.798788in}{1.864427in}}{\pgfqpoint{4.802695in}{1.860520in}}%
\pgfpathcurveto{\pgfqpoint{4.806602in}{1.856613in}}{\pgfqpoint{4.811901in}{1.854418in}}{\pgfqpoint{4.817426in}{1.854418in}}%
\pgfpathclose%
\pgfusepath{fill}%
\end{pgfscope}%
\begin{pgfscope}%
\pgfpathrectangle{\pgfqpoint{4.320685in}{1.588185in}}{\pgfqpoint{1.162500in}{0.755000in}} %
\pgfusepath{clip}%
\pgfsetbuttcap%
\pgfsetroundjoin%
\definecolor{currentfill}{rgb}{0.000000,0.000000,0.000000}%
\pgfsetfillcolor{currentfill}%
\pgfsetfillopacity{0.500000}%
\pgfsetlinewidth{0.000000pt}%
\definecolor{currentstroke}{rgb}{0.000000,0.000000,0.000000}%
\pgfsetstrokecolor{currentstroke}%
\pgfsetdash{}{0pt}%
\pgfpathmoveto{\pgfqpoint{5.258215in}{2.049107in}}%
\pgfpathcurveto{\pgfqpoint{5.263740in}{2.049107in}}{\pgfqpoint{5.269040in}{2.051302in}}{\pgfqpoint{5.272947in}{2.055209in}}%
\pgfpathcurveto{\pgfqpoint{5.276853in}{2.059115in}}{\pgfqpoint{5.279048in}{2.064415in}}{\pgfqpoint{5.279048in}{2.069940in}}%
\pgfpathcurveto{\pgfqpoint{5.279048in}{2.075465in}}{\pgfqpoint{5.276853in}{2.080764in}}{\pgfqpoint{5.272947in}{2.084671in}}%
\pgfpathcurveto{\pgfqpoint{5.269040in}{2.088578in}}{\pgfqpoint{5.263740in}{2.090773in}}{\pgfqpoint{5.258215in}{2.090773in}}%
\pgfpathcurveto{\pgfqpoint{5.252690in}{2.090773in}}{\pgfqpoint{5.247391in}{2.088578in}}{\pgfqpoint{5.243484in}{2.084671in}}%
\pgfpathcurveto{\pgfqpoint{5.239577in}{2.080764in}}{\pgfqpoint{5.237382in}{2.075465in}}{\pgfqpoint{5.237382in}{2.069940in}}%
\pgfpathcurveto{\pgfqpoint{5.237382in}{2.064415in}}{\pgfqpoint{5.239577in}{2.059115in}}{\pgfqpoint{5.243484in}{2.055209in}}%
\pgfpathcurveto{\pgfqpoint{5.247391in}{2.051302in}}{\pgfqpoint{5.252690in}{2.049107in}}{\pgfqpoint{5.258215in}{2.049107in}}%
\pgfpathclose%
\pgfusepath{fill}%
\end{pgfscope}%
\begin{pgfscope}%
\pgfpathrectangle{\pgfqpoint{4.320685in}{1.588185in}}{\pgfqpoint{1.162500in}{0.755000in}} %
\pgfusepath{clip}%
\pgfsetbuttcap%
\pgfsetroundjoin%
\definecolor{currentfill}{rgb}{0.000000,0.000000,0.000000}%
\pgfsetfillcolor{currentfill}%
\pgfsetfillopacity{0.500000}%
\pgfsetlinewidth{0.000000pt}%
\definecolor{currentstroke}{rgb}{0.000000,0.000000,0.000000}%
\pgfsetstrokecolor{currentstroke}%
\pgfsetdash{}{0pt}%
\pgfpathmoveto{\pgfqpoint{4.803827in}{1.703826in}}%
\pgfpathcurveto{\pgfqpoint{4.809352in}{1.703826in}}{\pgfqpoint{4.814651in}{1.706022in}}{\pgfqpoint{4.818558in}{1.709928in}}%
\pgfpathcurveto{\pgfqpoint{4.822465in}{1.713835in}}{\pgfqpoint{4.824660in}{1.719135in}}{\pgfqpoint{4.824660in}{1.724660in}}%
\pgfpathcurveto{\pgfqpoint{4.824660in}{1.730185in}}{\pgfqpoint{4.822465in}{1.735484in}}{\pgfqpoint{4.818558in}{1.739391in}}%
\pgfpathcurveto{\pgfqpoint{4.814651in}{1.743298in}}{\pgfqpoint{4.809352in}{1.745493in}}{\pgfqpoint{4.803827in}{1.745493in}}%
\pgfpathcurveto{\pgfqpoint{4.798302in}{1.745493in}}{\pgfqpoint{4.793002in}{1.743298in}}{\pgfqpoint{4.789096in}{1.739391in}}%
\pgfpathcurveto{\pgfqpoint{4.785189in}{1.735484in}}{\pgfqpoint{4.782994in}{1.730185in}}{\pgfqpoint{4.782994in}{1.724660in}}%
\pgfpathcurveto{\pgfqpoint{4.782994in}{1.719135in}}{\pgfqpoint{4.785189in}{1.713835in}}{\pgfqpoint{4.789096in}{1.709928in}}%
\pgfpathcurveto{\pgfqpoint{4.793002in}{1.706022in}}{\pgfqpoint{4.798302in}{1.703826in}}{\pgfqpoint{4.803827in}{1.703826in}}%
\pgfpathclose%
\pgfusepath{fill}%
\end{pgfscope}%
\begin{pgfscope}%
\pgfpathrectangle{\pgfqpoint{4.320685in}{1.588185in}}{\pgfqpoint{1.162500in}{0.755000in}} %
\pgfusepath{clip}%
\pgfsetbuttcap%
\pgfsetroundjoin%
\definecolor{currentfill}{rgb}{0.000000,0.000000,0.000000}%
\pgfsetfillcolor{currentfill}%
\pgfsetfillopacity{0.500000}%
\pgfsetlinewidth{0.000000pt}%
\definecolor{currentstroke}{rgb}{0.000000,0.000000,0.000000}%
\pgfsetstrokecolor{currentstroke}%
\pgfsetdash{}{0pt}%
\pgfpathmoveto{\pgfqpoint{4.437194in}{1.734907in}}%
\pgfpathcurveto{\pgfqpoint{4.442719in}{1.734907in}}{\pgfqpoint{4.448018in}{1.737102in}}{\pgfqpoint{4.451925in}{1.741009in}}%
\pgfpathcurveto{\pgfqpoint{4.455832in}{1.744915in}}{\pgfqpoint{4.458027in}{1.750215in}}{\pgfqpoint{4.458027in}{1.755740in}}%
\pgfpathcurveto{\pgfqpoint{4.458027in}{1.761265in}}{\pgfqpoint{4.455832in}{1.766565in}}{\pgfqpoint{4.451925in}{1.770471in}}%
\pgfpathcurveto{\pgfqpoint{4.448018in}{1.774378in}}{\pgfqpoint{4.442719in}{1.776573in}}{\pgfqpoint{4.437194in}{1.776573in}}%
\pgfpathcurveto{\pgfqpoint{4.431669in}{1.776573in}}{\pgfqpoint{4.426369in}{1.774378in}}{\pgfqpoint{4.422462in}{1.770471in}}%
\pgfpathcurveto{\pgfqpoint{4.418556in}{1.766565in}}{\pgfqpoint{4.416360in}{1.761265in}}{\pgfqpoint{4.416360in}{1.755740in}}%
\pgfpathcurveto{\pgfqpoint{4.416360in}{1.750215in}}{\pgfqpoint{4.418556in}{1.744915in}}{\pgfqpoint{4.422462in}{1.741009in}}%
\pgfpathcurveto{\pgfqpoint{4.426369in}{1.737102in}}{\pgfqpoint{4.431669in}{1.734907in}}{\pgfqpoint{4.437194in}{1.734907in}}%
\pgfpathclose%
\pgfusepath{fill}%
\end{pgfscope}%
\begin{pgfscope}%
\pgfpathrectangle{\pgfqpoint{4.320685in}{1.588185in}}{\pgfqpoint{1.162500in}{0.755000in}} %
\pgfusepath{clip}%
\pgfsetbuttcap%
\pgfsetroundjoin%
\definecolor{currentfill}{rgb}{0.000000,0.000000,0.000000}%
\pgfsetfillcolor{currentfill}%
\pgfsetfillopacity{0.500000}%
\pgfsetlinewidth{0.000000pt}%
\definecolor{currentstroke}{rgb}{0.000000,0.000000,0.000000}%
\pgfsetstrokecolor{currentstroke}%
\pgfsetdash{}{0pt}%
\pgfpathmoveto{\pgfqpoint{5.262967in}{1.977476in}}%
\pgfpathcurveto{\pgfqpoint{5.268492in}{1.977476in}}{\pgfqpoint{5.273791in}{1.979671in}}{\pgfqpoint{5.277698in}{1.983578in}}%
\pgfpathcurveto{\pgfqpoint{5.281605in}{1.987485in}}{\pgfqpoint{5.283800in}{1.992785in}}{\pgfqpoint{5.283800in}{1.998310in}}%
\pgfpathcurveto{\pgfqpoint{5.283800in}{2.003835in}}{\pgfqpoint{5.281605in}{2.009134in}}{\pgfqpoint{5.277698in}{2.013041in}}%
\pgfpathcurveto{\pgfqpoint{5.273791in}{2.016948in}}{\pgfqpoint{5.268492in}{2.019143in}}{\pgfqpoint{5.262967in}{2.019143in}}%
\pgfpathcurveto{\pgfqpoint{5.257442in}{2.019143in}}{\pgfqpoint{5.252142in}{2.016948in}}{\pgfqpoint{5.248235in}{2.013041in}}%
\pgfpathcurveto{\pgfqpoint{5.244329in}{2.009134in}}{\pgfqpoint{5.242134in}{2.003835in}}{\pgfqpoint{5.242134in}{1.998310in}}%
\pgfpathcurveto{\pgfqpoint{5.242134in}{1.992785in}}{\pgfqpoint{5.244329in}{1.987485in}}{\pgfqpoint{5.248235in}{1.983578in}}%
\pgfpathcurveto{\pgfqpoint{5.252142in}{1.979671in}}{\pgfqpoint{5.257442in}{1.977476in}}{\pgfqpoint{5.262967in}{1.977476in}}%
\pgfpathclose%
\pgfusepath{fill}%
\end{pgfscope}%
\begin{pgfscope}%
\pgfpathrectangle{\pgfqpoint{4.320685in}{1.588185in}}{\pgfqpoint{1.162500in}{0.755000in}} %
\pgfusepath{clip}%
\pgfsetbuttcap%
\pgfsetroundjoin%
\definecolor{currentfill}{rgb}{0.000000,0.000000,0.000000}%
\pgfsetfillcolor{currentfill}%
\pgfsetfillopacity{0.500000}%
\pgfsetlinewidth{0.000000pt}%
\definecolor{currentstroke}{rgb}{0.000000,0.000000,0.000000}%
\pgfsetstrokecolor{currentstroke}%
\pgfsetdash{}{0pt}%
\pgfpathmoveto{\pgfqpoint{4.896922in}{1.614683in}}%
\pgfpathcurveto{\pgfqpoint{4.902447in}{1.614683in}}{\pgfqpoint{4.907746in}{1.616879in}}{\pgfqpoint{4.911653in}{1.620785in}}%
\pgfpathcurveto{\pgfqpoint{4.915560in}{1.624692in}}{\pgfqpoint{4.917755in}{1.629992in}}{\pgfqpoint{4.917755in}{1.635517in}}%
\pgfpathcurveto{\pgfqpoint{4.917755in}{1.641042in}}{\pgfqpoint{4.915560in}{1.646341in}}{\pgfqpoint{4.911653in}{1.650248in}}%
\pgfpathcurveto{\pgfqpoint{4.907746in}{1.654155in}}{\pgfqpoint{4.902447in}{1.656350in}}{\pgfqpoint{4.896922in}{1.656350in}}%
\pgfpathcurveto{\pgfqpoint{4.891397in}{1.656350in}}{\pgfqpoint{4.886097in}{1.654155in}}{\pgfqpoint{4.882190in}{1.650248in}}%
\pgfpathcurveto{\pgfqpoint{4.878284in}{1.646341in}}{\pgfqpoint{4.876088in}{1.641042in}}{\pgfqpoint{4.876088in}{1.635517in}}%
\pgfpathcurveto{\pgfqpoint{4.876088in}{1.629992in}}{\pgfqpoint{4.878284in}{1.624692in}}{\pgfqpoint{4.882190in}{1.620785in}}%
\pgfpathcurveto{\pgfqpoint{4.886097in}{1.616879in}}{\pgfqpoint{4.891397in}{1.614683in}}{\pgfqpoint{4.896922in}{1.614683in}}%
\pgfpathclose%
\pgfusepath{fill}%
\end{pgfscope}%
\begin{pgfscope}%
\pgfsetrectcap%
\pgfsetmiterjoin%
\pgfsetlinewidth{0.803000pt}%
\definecolor{currentstroke}{rgb}{0.000000,0.000000,0.000000}%
\pgfsetstrokecolor{currentstroke}%
\pgfsetdash{}{0pt}%
\pgfpathmoveto{\pgfqpoint{4.320685in}{1.588185in}}%
\pgfpathlineto{\pgfqpoint{4.320685in}{2.343185in}}%
\pgfusepath{stroke}%
\end{pgfscope}%
\begin{pgfscope}%
\pgfsetrectcap%
\pgfsetmiterjoin%
\pgfsetlinewidth{0.803000pt}%
\definecolor{currentstroke}{rgb}{0.000000,0.000000,0.000000}%
\pgfsetstrokecolor{currentstroke}%
\pgfsetdash{}{0pt}%
\pgfpathmoveto{\pgfqpoint{5.483185in}{1.588185in}}%
\pgfpathlineto{\pgfqpoint{5.483185in}{2.343185in}}%
\pgfusepath{stroke}%
\end{pgfscope}%
\begin{pgfscope}%
\pgfsetrectcap%
\pgfsetmiterjoin%
\pgfsetlinewidth{0.803000pt}%
\definecolor{currentstroke}{rgb}{0.000000,0.000000,0.000000}%
\pgfsetstrokecolor{currentstroke}%
\pgfsetdash{}{0pt}%
\pgfpathmoveto{\pgfqpoint{4.320685in}{1.588185in}}%
\pgfpathlineto{\pgfqpoint{5.483185in}{1.588185in}}%
\pgfusepath{stroke}%
\end{pgfscope}%
\begin{pgfscope}%
\pgfsetrectcap%
\pgfsetmiterjoin%
\pgfsetlinewidth{0.803000pt}%
\definecolor{currentstroke}{rgb}{0.000000,0.000000,0.000000}%
\pgfsetstrokecolor{currentstroke}%
\pgfsetdash{}{0pt}%
\pgfpathmoveto{\pgfqpoint{4.320685in}{2.343185in}}%
\pgfpathlineto{\pgfqpoint{5.483185in}{2.343185in}}%
\pgfusepath{stroke}%
\end{pgfscope}%
\begin{pgfscope}%
\pgfsetbuttcap%
\pgfsetmiterjoin%
\definecolor{currentfill}{rgb}{1.000000,1.000000,1.000000}%
\pgfsetfillcolor{currentfill}%
\pgfsetlinewidth{0.000000pt}%
\definecolor{currentstroke}{rgb}{0.000000,0.000000,0.000000}%
\pgfsetstrokecolor{currentstroke}%
\pgfsetstrokeopacity{0.000000}%
\pgfsetdash{}{0pt}%
\pgfpathmoveto{\pgfqpoint{0.833185in}{0.833185in}}%
\pgfpathlineto{\pgfqpoint{1.995685in}{0.833185in}}%
\pgfpathlineto{\pgfqpoint{1.995685in}{1.588185in}}%
\pgfpathlineto{\pgfqpoint{0.833185in}{1.588185in}}%
\pgfpathclose%
\pgfusepath{fill}%
\end{pgfscope}%
\begin{pgfscope}%
\pgfpathrectangle{\pgfqpoint{0.833185in}{0.833185in}}{\pgfqpoint{1.162500in}{0.755000in}} %
\pgfusepath{clip}%
\pgfsetbuttcap%
\pgfsetroundjoin%
\definecolor{currentfill}{rgb}{0.000000,0.000000,0.000000}%
\pgfsetfillcolor{currentfill}%
\pgfsetfillopacity{0.500000}%
\pgfsetlinewidth{0.000000pt}%
\definecolor{currentstroke}{rgb}{0.000000,0.000000,0.000000}%
\pgfsetstrokecolor{currentstroke}%
\pgfsetdash{}{0pt}%
\pgfpathmoveto{\pgfqpoint{1.779018in}{1.099971in}}%
\pgfpathcurveto{\pgfqpoint{1.784543in}{1.099971in}}{\pgfqpoint{1.789842in}{1.102166in}}{\pgfqpoint{1.793749in}{1.106073in}}%
\pgfpathcurveto{\pgfqpoint{1.797656in}{1.109980in}}{\pgfqpoint{1.799851in}{1.115279in}}{\pgfqpoint{1.799851in}{1.120804in}}%
\pgfpathcurveto{\pgfqpoint{1.799851in}{1.126329in}}{\pgfqpoint{1.797656in}{1.131629in}}{\pgfqpoint{1.793749in}{1.135536in}}%
\pgfpathcurveto{\pgfqpoint{1.789842in}{1.139442in}}{\pgfqpoint{1.784543in}{1.141638in}}{\pgfqpoint{1.779018in}{1.141638in}}%
\pgfpathcurveto{\pgfqpoint{1.773492in}{1.141638in}}{\pgfqpoint{1.768193in}{1.139442in}}{\pgfqpoint{1.764286in}{1.135536in}}%
\pgfpathcurveto{\pgfqpoint{1.760379in}{1.131629in}}{\pgfqpoint{1.758184in}{1.126329in}}{\pgfqpoint{1.758184in}{1.120804in}}%
\pgfpathcurveto{\pgfqpoint{1.758184in}{1.115279in}}{\pgfqpoint{1.760379in}{1.109980in}}{\pgfqpoint{1.764286in}{1.106073in}}%
\pgfpathcurveto{\pgfqpoint{1.768193in}{1.102166in}}{\pgfqpoint{1.773492in}{1.099971in}}{\pgfqpoint{1.779018in}{1.099971in}}%
\pgfpathclose%
\pgfusepath{fill}%
\end{pgfscope}%
\begin{pgfscope}%
\pgfpathrectangle{\pgfqpoint{0.833185in}{0.833185in}}{\pgfqpoint{1.162500in}{0.755000in}} %
\pgfusepath{clip}%
\pgfsetbuttcap%
\pgfsetroundjoin%
\definecolor{currentfill}{rgb}{0.000000,0.000000,0.000000}%
\pgfsetfillcolor{currentfill}%
\pgfsetfillopacity{0.500000}%
\pgfsetlinewidth{0.000000pt}%
\definecolor{currentstroke}{rgb}{0.000000,0.000000,0.000000}%
\pgfsetstrokecolor{currentstroke}%
\pgfsetdash{}{0pt}%
\pgfpathmoveto{\pgfqpoint{0.920858in}{1.503836in}}%
\pgfpathcurveto{\pgfqpoint{0.926383in}{1.503836in}}{\pgfqpoint{0.931683in}{1.506031in}}{\pgfqpoint{0.935589in}{1.509938in}}%
\pgfpathcurveto{\pgfqpoint{0.939496in}{1.513845in}}{\pgfqpoint{0.941691in}{1.519144in}}{\pgfqpoint{0.941691in}{1.524669in}}%
\pgfpathcurveto{\pgfqpoint{0.941691in}{1.530195in}}{\pgfqpoint{0.939496in}{1.535494in}}{\pgfqpoint{0.935589in}{1.539401in}}%
\pgfpathcurveto{\pgfqpoint{0.931683in}{1.543308in}}{\pgfqpoint{0.926383in}{1.545503in}}{\pgfqpoint{0.920858in}{1.545503in}}%
\pgfpathcurveto{\pgfqpoint{0.915333in}{1.545503in}}{\pgfqpoint{0.910034in}{1.543308in}}{\pgfqpoint{0.906127in}{1.539401in}}%
\pgfpathcurveto{\pgfqpoint{0.902220in}{1.535494in}}{\pgfqpoint{0.900025in}{1.530195in}}{\pgfqpoint{0.900025in}{1.524669in}}%
\pgfpathcurveto{\pgfqpoint{0.900025in}{1.519144in}}{\pgfqpoint{0.902220in}{1.513845in}}{\pgfqpoint{0.906127in}{1.509938in}}%
\pgfpathcurveto{\pgfqpoint{0.910034in}{1.506031in}}{\pgfqpoint{0.915333in}{1.503836in}}{\pgfqpoint{0.920858in}{1.503836in}}%
\pgfpathclose%
\pgfusepath{fill}%
\end{pgfscope}%
\begin{pgfscope}%
\pgfpathrectangle{\pgfqpoint{0.833185in}{0.833185in}}{\pgfqpoint{1.162500in}{0.755000in}} %
\pgfusepath{clip}%
\pgfsetbuttcap%
\pgfsetroundjoin%
\definecolor{currentfill}{rgb}{0.000000,0.000000,0.000000}%
\pgfsetfillcolor{currentfill}%
\pgfsetfillopacity{0.500000}%
\pgfsetlinewidth{0.000000pt}%
\definecolor{currentstroke}{rgb}{0.000000,0.000000,0.000000}%
\pgfsetstrokecolor{currentstroke}%
\pgfsetdash{}{0pt}%
\pgfpathmoveto{\pgfqpoint{0.860863in}{1.549375in}}%
\pgfpathcurveto{\pgfqpoint{0.866388in}{1.549375in}}{\pgfqpoint{0.871688in}{1.551570in}}{\pgfqpoint{0.875595in}{1.555477in}}%
\pgfpathcurveto{\pgfqpoint{0.879501in}{1.559384in}}{\pgfqpoint{0.881696in}{1.564683in}}{\pgfqpoint{0.881696in}{1.570208in}}%
\pgfpathcurveto{\pgfqpoint{0.881696in}{1.575733in}}{\pgfqpoint{0.879501in}{1.581033in}}{\pgfqpoint{0.875595in}{1.584940in}}%
\pgfpathcurveto{\pgfqpoint{0.871688in}{1.588847in}}{\pgfqpoint{0.866388in}{1.591042in}}{\pgfqpoint{0.860863in}{1.591042in}}%
\pgfpathcurveto{\pgfqpoint{0.855338in}{1.591042in}}{\pgfqpoint{0.850039in}{1.588847in}}{\pgfqpoint{0.846132in}{1.584940in}}%
\pgfpathcurveto{\pgfqpoint{0.842225in}{1.581033in}}{\pgfqpoint{0.840030in}{1.575733in}}{\pgfqpoint{0.840030in}{1.570208in}}%
\pgfpathcurveto{\pgfqpoint{0.840030in}{1.564683in}}{\pgfqpoint{0.842225in}{1.559384in}}{\pgfqpoint{0.846132in}{1.555477in}}%
\pgfpathcurveto{\pgfqpoint{0.850039in}{1.551570in}}{\pgfqpoint{0.855338in}{1.549375in}}{\pgfqpoint{0.860863in}{1.549375in}}%
\pgfpathclose%
\pgfusepath{fill}%
\end{pgfscope}%
\begin{pgfscope}%
\pgfpathrectangle{\pgfqpoint{0.833185in}{0.833185in}}{\pgfqpoint{1.162500in}{0.755000in}} %
\pgfusepath{clip}%
\pgfsetbuttcap%
\pgfsetroundjoin%
\definecolor{currentfill}{rgb}{0.000000,0.000000,0.000000}%
\pgfsetfillcolor{currentfill}%
\pgfsetfillopacity{0.500000}%
\pgfsetlinewidth{0.000000pt}%
\definecolor{currentstroke}{rgb}{0.000000,0.000000,0.000000}%
\pgfsetstrokecolor{currentstroke}%
\pgfsetdash{}{0pt}%
\pgfpathmoveto{\pgfqpoint{1.180359in}{1.273758in}}%
\pgfpathcurveto{\pgfqpoint{1.185884in}{1.273758in}}{\pgfqpoint{1.191184in}{1.275953in}}{\pgfqpoint{1.195090in}{1.279859in}}%
\pgfpathcurveto{\pgfqpoint{1.198997in}{1.283766in}}{\pgfqpoint{1.201192in}{1.289066in}}{\pgfqpoint{1.201192in}{1.294591in}}%
\pgfpathcurveto{\pgfqpoint{1.201192in}{1.300116in}}{\pgfqpoint{1.198997in}{1.305415in}}{\pgfqpoint{1.195090in}{1.309322in}}%
\pgfpathcurveto{\pgfqpoint{1.191184in}{1.313229in}}{\pgfqpoint{1.185884in}{1.315424in}}{\pgfqpoint{1.180359in}{1.315424in}}%
\pgfpathcurveto{\pgfqpoint{1.174834in}{1.315424in}}{\pgfqpoint{1.169535in}{1.313229in}}{\pgfqpoint{1.165628in}{1.309322in}}%
\pgfpathcurveto{\pgfqpoint{1.161721in}{1.305415in}}{\pgfqpoint{1.159526in}{1.300116in}}{\pgfqpoint{1.159526in}{1.294591in}}%
\pgfpathcurveto{\pgfqpoint{1.159526in}{1.289066in}}{\pgfqpoint{1.161721in}{1.283766in}}{\pgfqpoint{1.165628in}{1.279859in}}%
\pgfpathcurveto{\pgfqpoint{1.169535in}{1.275953in}}{\pgfqpoint{1.174834in}{1.273758in}}{\pgfqpoint{1.180359in}{1.273758in}}%
\pgfpathclose%
\pgfusepath{fill}%
\end{pgfscope}%
\begin{pgfscope}%
\pgfpathrectangle{\pgfqpoint{0.833185in}{0.833185in}}{\pgfqpoint{1.162500in}{0.755000in}} %
\pgfusepath{clip}%
\pgfsetbuttcap%
\pgfsetroundjoin%
\definecolor{currentfill}{rgb}{0.000000,0.000000,0.000000}%
\pgfsetfillcolor{currentfill}%
\pgfsetfillopacity{0.500000}%
\pgfsetlinewidth{0.000000pt}%
\definecolor{currentstroke}{rgb}{0.000000,0.000000,0.000000}%
\pgfsetstrokecolor{currentstroke}%
\pgfsetdash{}{0pt}%
\pgfpathmoveto{\pgfqpoint{0.938297in}{1.319843in}}%
\pgfpathcurveto{\pgfqpoint{0.943822in}{1.319843in}}{\pgfqpoint{0.949121in}{1.322038in}}{\pgfqpoint{0.953028in}{1.325945in}}%
\pgfpathcurveto{\pgfqpoint{0.956935in}{1.329852in}}{\pgfqpoint{0.959130in}{1.335151in}}{\pgfqpoint{0.959130in}{1.340676in}}%
\pgfpathcurveto{\pgfqpoint{0.959130in}{1.346201in}}{\pgfqpoint{0.956935in}{1.351501in}}{\pgfqpoint{0.953028in}{1.355408in}}%
\pgfpathcurveto{\pgfqpoint{0.949121in}{1.359314in}}{\pgfqpoint{0.943822in}{1.361510in}}{\pgfqpoint{0.938297in}{1.361510in}}%
\pgfpathcurveto{\pgfqpoint{0.932772in}{1.361510in}}{\pgfqpoint{0.927472in}{1.359314in}}{\pgfqpoint{0.923565in}{1.355408in}}%
\pgfpathcurveto{\pgfqpoint{0.919659in}{1.351501in}}{\pgfqpoint{0.917464in}{1.346201in}}{\pgfqpoint{0.917464in}{1.340676in}}%
\pgfpathcurveto{\pgfqpoint{0.917464in}{1.335151in}}{\pgfqpoint{0.919659in}{1.329852in}}{\pgfqpoint{0.923565in}{1.325945in}}%
\pgfpathcurveto{\pgfqpoint{0.927472in}{1.322038in}}{\pgfqpoint{0.932772in}{1.319843in}}{\pgfqpoint{0.938297in}{1.319843in}}%
\pgfpathclose%
\pgfusepath{fill}%
\end{pgfscope}%
\begin{pgfscope}%
\pgfpathrectangle{\pgfqpoint{0.833185in}{0.833185in}}{\pgfqpoint{1.162500in}{0.755000in}} %
\pgfusepath{clip}%
\pgfsetbuttcap%
\pgfsetroundjoin%
\definecolor{currentfill}{rgb}{0.000000,0.000000,0.000000}%
\pgfsetfillcolor{currentfill}%
\pgfsetfillopacity{0.500000}%
\pgfsetlinewidth{0.000000pt}%
\definecolor{currentstroke}{rgb}{0.000000,0.000000,0.000000}%
\pgfsetstrokecolor{currentstroke}%
\pgfsetdash{}{0pt}%
\pgfpathmoveto{\pgfqpoint{1.098900in}{1.215122in}}%
\pgfpathcurveto{\pgfqpoint{1.104426in}{1.215122in}}{\pgfqpoint{1.109725in}{1.217317in}}{\pgfqpoint{1.113632in}{1.221224in}}%
\pgfpathcurveto{\pgfqpoint{1.117539in}{1.225130in}}{\pgfqpoint{1.119734in}{1.230430in}}{\pgfqpoint{1.119734in}{1.235955in}}%
\pgfpathcurveto{\pgfqpoint{1.119734in}{1.241480in}}{\pgfqpoint{1.117539in}{1.246780in}}{\pgfqpoint{1.113632in}{1.250686in}}%
\pgfpathcurveto{\pgfqpoint{1.109725in}{1.254593in}}{\pgfqpoint{1.104426in}{1.256788in}}{\pgfqpoint{1.098900in}{1.256788in}}%
\pgfpathcurveto{\pgfqpoint{1.093375in}{1.256788in}}{\pgfqpoint{1.088076in}{1.254593in}}{\pgfqpoint{1.084169in}{1.250686in}}%
\pgfpathcurveto{\pgfqpoint{1.080262in}{1.246780in}}{\pgfqpoint{1.078067in}{1.241480in}}{\pgfqpoint{1.078067in}{1.235955in}}%
\pgfpathcurveto{\pgfqpoint{1.078067in}{1.230430in}}{\pgfqpoint{1.080262in}{1.225130in}}{\pgfqpoint{1.084169in}{1.221224in}}%
\pgfpathcurveto{\pgfqpoint{1.088076in}{1.217317in}}{\pgfqpoint{1.093375in}{1.215122in}}{\pgfqpoint{1.098900in}{1.215122in}}%
\pgfpathclose%
\pgfusepath{fill}%
\end{pgfscope}%
\begin{pgfscope}%
\pgfpathrectangle{\pgfqpoint{0.833185in}{0.833185in}}{\pgfqpoint{1.162500in}{0.755000in}} %
\pgfusepath{clip}%
\pgfsetbuttcap%
\pgfsetroundjoin%
\definecolor{currentfill}{rgb}{0.000000,0.000000,0.000000}%
\pgfsetfillcolor{currentfill}%
\pgfsetfillopacity{0.500000}%
\pgfsetlinewidth{0.000000pt}%
\definecolor{currentstroke}{rgb}{0.000000,0.000000,0.000000}%
\pgfsetstrokecolor{currentstroke}%
\pgfsetdash{}{0pt}%
\pgfpathmoveto{\pgfqpoint{1.085127in}{0.830327in}}%
\pgfpathcurveto{\pgfqpoint{1.090652in}{0.830327in}}{\pgfqpoint{1.095951in}{0.832523in}}{\pgfqpoint{1.099858in}{0.836429in}}%
\pgfpathcurveto{\pgfqpoint{1.103765in}{0.840336in}}{\pgfqpoint{1.105960in}{0.845636in}}{\pgfqpoint{1.105960in}{0.851161in}}%
\pgfpathcurveto{\pgfqpoint{1.105960in}{0.856686in}}{\pgfqpoint{1.103765in}{0.861985in}}{\pgfqpoint{1.099858in}{0.865892in}}%
\pgfpathcurveto{\pgfqpoint{1.095951in}{0.869799in}}{\pgfqpoint{1.090652in}{0.871994in}}{\pgfqpoint{1.085127in}{0.871994in}}%
\pgfpathcurveto{\pgfqpoint{1.079602in}{0.871994in}}{\pgfqpoint{1.074302in}{0.869799in}}{\pgfqpoint{1.070395in}{0.865892in}}%
\pgfpathcurveto{\pgfqpoint{1.066488in}{0.861985in}}{\pgfqpoint{1.064293in}{0.856686in}}{\pgfqpoint{1.064293in}{0.851161in}}%
\pgfpathcurveto{\pgfqpoint{1.064293in}{0.845636in}}{\pgfqpoint{1.066488in}{0.840336in}}{\pgfqpoint{1.070395in}{0.836429in}}%
\pgfpathcurveto{\pgfqpoint{1.074302in}{0.832523in}}{\pgfqpoint{1.079602in}{0.830327in}}{\pgfqpoint{1.085127in}{0.830327in}}%
\pgfpathclose%
\pgfusepath{fill}%
\end{pgfscope}%
\begin{pgfscope}%
\pgfpathrectangle{\pgfqpoint{0.833185in}{0.833185in}}{\pgfqpoint{1.162500in}{0.755000in}} %
\pgfusepath{clip}%
\pgfsetbuttcap%
\pgfsetroundjoin%
\definecolor{currentfill}{rgb}{0.000000,0.000000,0.000000}%
\pgfsetfillcolor{currentfill}%
\pgfsetfillopacity{0.500000}%
\pgfsetlinewidth{0.000000pt}%
\definecolor{currentstroke}{rgb}{0.000000,0.000000,0.000000}%
\pgfsetstrokecolor{currentstroke}%
\pgfsetdash{}{0pt}%
\pgfpathmoveto{\pgfqpoint{1.896712in}{1.469697in}}%
\pgfpathcurveto{\pgfqpoint{1.902237in}{1.469697in}}{\pgfqpoint{1.907537in}{1.471892in}}{\pgfqpoint{1.911443in}{1.475799in}}%
\pgfpathcurveto{\pgfqpoint{1.915350in}{1.479706in}}{\pgfqpoint{1.917545in}{1.485005in}}{\pgfqpoint{1.917545in}{1.490530in}}%
\pgfpathcurveto{\pgfqpoint{1.917545in}{1.496056in}}{\pgfqpoint{1.915350in}{1.501355in}}{\pgfqpoint{1.911443in}{1.505262in}}%
\pgfpathcurveto{\pgfqpoint{1.907537in}{1.509169in}}{\pgfqpoint{1.902237in}{1.511364in}}{\pgfqpoint{1.896712in}{1.511364in}}%
\pgfpathcurveto{\pgfqpoint{1.891187in}{1.511364in}}{\pgfqpoint{1.885887in}{1.509169in}}{\pgfqpoint{1.881981in}{1.505262in}}%
\pgfpathcurveto{\pgfqpoint{1.878074in}{1.501355in}}{\pgfqpoint{1.875879in}{1.496056in}}{\pgfqpoint{1.875879in}{1.490530in}}%
\pgfpathcurveto{\pgfqpoint{1.875879in}{1.485005in}}{\pgfqpoint{1.878074in}{1.479706in}}{\pgfqpoint{1.881981in}{1.475799in}}%
\pgfpathcurveto{\pgfqpoint{1.885887in}{1.471892in}}{\pgfqpoint{1.891187in}{1.469697in}}{\pgfqpoint{1.896712in}{1.469697in}}%
\pgfpathclose%
\pgfusepath{fill}%
\end{pgfscope}%
\begin{pgfscope}%
\pgfpathrectangle{\pgfqpoint{0.833185in}{0.833185in}}{\pgfqpoint{1.162500in}{0.755000in}} %
\pgfusepath{clip}%
\pgfsetbuttcap%
\pgfsetroundjoin%
\definecolor{currentfill}{rgb}{0.000000,0.000000,0.000000}%
\pgfsetfillcolor{currentfill}%
\pgfsetfillopacity{0.500000}%
\pgfsetlinewidth{0.000000pt}%
\definecolor{currentstroke}{rgb}{0.000000,0.000000,0.000000}%
\pgfsetstrokecolor{currentstroke}%
\pgfsetdash{}{0pt}%
\pgfpathmoveto{\pgfqpoint{1.968006in}{1.312260in}}%
\pgfpathcurveto{\pgfqpoint{1.973531in}{1.312260in}}{\pgfqpoint{1.978831in}{1.314455in}}{\pgfqpoint{1.982737in}{1.318362in}}%
\pgfpathcurveto{\pgfqpoint{1.986644in}{1.322269in}}{\pgfqpoint{1.988839in}{1.327569in}}{\pgfqpoint{1.988839in}{1.333094in}}%
\pgfpathcurveto{\pgfqpoint{1.988839in}{1.338619in}}{\pgfqpoint{1.986644in}{1.343918in}}{\pgfqpoint{1.982737in}{1.347825in}}%
\pgfpathcurveto{\pgfqpoint{1.978831in}{1.351732in}}{\pgfqpoint{1.973531in}{1.353927in}}{\pgfqpoint{1.968006in}{1.353927in}}%
\pgfpathcurveto{\pgfqpoint{1.962481in}{1.353927in}}{\pgfqpoint{1.957181in}{1.351732in}}{\pgfqpoint{1.953275in}{1.347825in}}%
\pgfpathcurveto{\pgfqpoint{1.949368in}{1.343918in}}{\pgfqpoint{1.947173in}{1.338619in}}{\pgfqpoint{1.947173in}{1.333094in}}%
\pgfpathcurveto{\pgfqpoint{1.947173in}{1.327569in}}{\pgfqpoint{1.949368in}{1.322269in}}{\pgfqpoint{1.953275in}{1.318362in}}%
\pgfpathcurveto{\pgfqpoint{1.957181in}{1.314455in}}{\pgfqpoint{1.962481in}{1.312260in}}{\pgfqpoint{1.968006in}{1.312260in}}%
\pgfpathclose%
\pgfusepath{fill}%
\end{pgfscope}%
\begin{pgfscope}%
\pgfpathrectangle{\pgfqpoint{0.833185in}{0.833185in}}{\pgfqpoint{1.162500in}{0.755000in}} %
\pgfusepath{clip}%
\pgfsetbuttcap%
\pgfsetroundjoin%
\definecolor{currentfill}{rgb}{0.000000,0.000000,0.000000}%
\pgfsetfillcolor{currentfill}%
\pgfsetfillopacity{0.500000}%
\pgfsetlinewidth{0.000000pt}%
\definecolor{currentstroke}{rgb}{0.000000,0.000000,0.000000}%
\pgfsetstrokecolor{currentstroke}%
\pgfsetdash{}{0pt}%
\pgfpathmoveto{\pgfqpoint{1.742203in}{1.134966in}}%
\pgfpathcurveto{\pgfqpoint{1.747728in}{1.134966in}}{\pgfqpoint{1.753027in}{1.137161in}}{\pgfqpoint{1.756934in}{1.141068in}}%
\pgfpathcurveto{\pgfqpoint{1.760841in}{1.144975in}}{\pgfqpoint{1.763036in}{1.150275in}}{\pgfqpoint{1.763036in}{1.155800in}}%
\pgfpathcurveto{\pgfqpoint{1.763036in}{1.161325in}}{\pgfqpoint{1.760841in}{1.166624in}}{\pgfqpoint{1.756934in}{1.170531in}}%
\pgfpathcurveto{\pgfqpoint{1.753027in}{1.174438in}}{\pgfqpoint{1.747728in}{1.176633in}}{\pgfqpoint{1.742203in}{1.176633in}}%
\pgfpathcurveto{\pgfqpoint{1.736677in}{1.176633in}}{\pgfqpoint{1.731378in}{1.174438in}}{\pgfqpoint{1.727471in}{1.170531in}}%
\pgfpathcurveto{\pgfqpoint{1.723564in}{1.166624in}}{\pgfqpoint{1.721369in}{1.161325in}}{\pgfqpoint{1.721369in}{1.155800in}}%
\pgfpathcurveto{\pgfqpoint{1.721369in}{1.150275in}}{\pgfqpoint{1.723564in}{1.144975in}}{\pgfqpoint{1.727471in}{1.141068in}}%
\pgfpathcurveto{\pgfqpoint{1.731378in}{1.137161in}}{\pgfqpoint{1.736677in}{1.134966in}}{\pgfqpoint{1.742203in}{1.134966in}}%
\pgfpathclose%
\pgfusepath{fill}%
\end{pgfscope}%
\begin{pgfscope}%
\pgfpathrectangle{\pgfqpoint{0.833185in}{0.833185in}}{\pgfqpoint{1.162500in}{0.755000in}} %
\pgfusepath{clip}%
\pgfsetbuttcap%
\pgfsetroundjoin%
\definecolor{currentfill}{rgb}{0.000000,0.000000,0.000000}%
\pgfsetfillcolor{currentfill}%
\pgfsetfillopacity{0.500000}%
\pgfsetlinewidth{0.000000pt}%
\definecolor{currentstroke}{rgb}{0.000000,0.000000,0.000000}%
\pgfsetstrokecolor{currentstroke}%
\pgfsetdash{}{0pt}%
\pgfpathmoveto{\pgfqpoint{1.058530in}{1.421242in}}%
\pgfpathcurveto{\pgfqpoint{1.064055in}{1.421242in}}{\pgfqpoint{1.069355in}{1.423437in}}{\pgfqpoint{1.073261in}{1.427344in}}%
\pgfpathcurveto{\pgfqpoint{1.077168in}{1.431251in}}{\pgfqpoint{1.079363in}{1.436550in}}{\pgfqpoint{1.079363in}{1.442075in}}%
\pgfpathcurveto{\pgfqpoint{1.079363in}{1.447600in}}{\pgfqpoint{1.077168in}{1.452900in}}{\pgfqpoint{1.073261in}{1.456807in}}%
\pgfpathcurveto{\pgfqpoint{1.069355in}{1.460714in}}{\pgfqpoint{1.064055in}{1.462909in}}{\pgfqpoint{1.058530in}{1.462909in}}%
\pgfpathcurveto{\pgfqpoint{1.053005in}{1.462909in}}{\pgfqpoint{1.047706in}{1.460714in}}{\pgfqpoint{1.043799in}{1.456807in}}%
\pgfpathcurveto{\pgfqpoint{1.039892in}{1.452900in}}{\pgfqpoint{1.037697in}{1.447600in}}{\pgfqpoint{1.037697in}{1.442075in}}%
\pgfpathcurveto{\pgfqpoint{1.037697in}{1.436550in}}{\pgfqpoint{1.039892in}{1.431251in}}{\pgfqpoint{1.043799in}{1.427344in}}%
\pgfpathcurveto{\pgfqpoint{1.047706in}{1.423437in}}{\pgfqpoint{1.053005in}{1.421242in}}{\pgfqpoint{1.058530in}{1.421242in}}%
\pgfpathclose%
\pgfusepath{fill}%
\end{pgfscope}%
\begin{pgfscope}%
\pgfpathrectangle{\pgfqpoint{0.833185in}{0.833185in}}{\pgfqpoint{1.162500in}{0.755000in}} %
\pgfusepath{clip}%
\pgfsetbuttcap%
\pgfsetroundjoin%
\definecolor{currentfill}{rgb}{0.000000,0.000000,0.000000}%
\pgfsetfillcolor{currentfill}%
\pgfsetfillopacity{0.500000}%
\pgfsetlinewidth{0.000000pt}%
\definecolor{currentstroke}{rgb}{0.000000,0.000000,0.000000}%
\pgfsetstrokecolor{currentstroke}%
\pgfsetdash{}{0pt}%
\pgfpathmoveto{\pgfqpoint{1.520990in}{1.126134in}}%
\pgfpathcurveto{\pgfqpoint{1.526515in}{1.126134in}}{\pgfqpoint{1.531814in}{1.128329in}}{\pgfqpoint{1.535721in}{1.132236in}}%
\pgfpathcurveto{\pgfqpoint{1.539628in}{1.136143in}}{\pgfqpoint{1.541823in}{1.141442in}}{\pgfqpoint{1.541823in}{1.146967in}}%
\pgfpathcurveto{\pgfqpoint{1.541823in}{1.152492in}}{\pgfqpoint{1.539628in}{1.157792in}}{\pgfqpoint{1.535721in}{1.161699in}}%
\pgfpathcurveto{\pgfqpoint{1.531814in}{1.165606in}}{\pgfqpoint{1.526515in}{1.167801in}}{\pgfqpoint{1.520990in}{1.167801in}}%
\pgfpathcurveto{\pgfqpoint{1.515465in}{1.167801in}}{\pgfqpoint{1.510165in}{1.165606in}}{\pgfqpoint{1.506258in}{1.161699in}}%
\pgfpathcurveto{\pgfqpoint{1.502352in}{1.157792in}}{\pgfqpoint{1.500156in}{1.152492in}}{\pgfqpoint{1.500156in}{1.146967in}}%
\pgfpathcurveto{\pgfqpoint{1.500156in}{1.141442in}}{\pgfqpoint{1.502352in}{1.136143in}}{\pgfqpoint{1.506258in}{1.132236in}}%
\pgfpathcurveto{\pgfqpoint{1.510165in}{1.128329in}}{\pgfqpoint{1.515465in}{1.126134in}}{\pgfqpoint{1.520990in}{1.126134in}}%
\pgfpathclose%
\pgfusepath{fill}%
\end{pgfscope}%
\begin{pgfscope}%
\pgfpathrectangle{\pgfqpoint{0.833185in}{0.833185in}}{\pgfqpoint{1.162500in}{0.755000in}} %
\pgfusepath{clip}%
\pgfsetbuttcap%
\pgfsetroundjoin%
\definecolor{currentfill}{rgb}{0.000000,0.000000,0.000000}%
\pgfsetfillcolor{currentfill}%
\pgfsetfillopacity{0.500000}%
\pgfsetlinewidth{0.000000pt}%
\definecolor{currentstroke}{rgb}{0.000000,0.000000,0.000000}%
\pgfsetstrokecolor{currentstroke}%
\pgfsetdash{}{0pt}%
\pgfpathmoveto{\pgfqpoint{1.402203in}{0.888020in}}%
\pgfpathcurveto{\pgfqpoint{1.407728in}{0.888020in}}{\pgfqpoint{1.413027in}{0.890215in}}{\pgfqpoint{1.416934in}{0.894122in}}%
\pgfpathcurveto{\pgfqpoint{1.420841in}{0.898028in}}{\pgfqpoint{1.423036in}{0.903328in}}{\pgfqpoint{1.423036in}{0.908853in}}%
\pgfpathcurveto{\pgfqpoint{1.423036in}{0.914378in}}{\pgfqpoint{1.420841in}{0.919677in}}{\pgfqpoint{1.416934in}{0.923584in}}%
\pgfpathcurveto{\pgfqpoint{1.413027in}{0.927491in}}{\pgfqpoint{1.407728in}{0.929686in}}{\pgfqpoint{1.402203in}{0.929686in}}%
\pgfpathcurveto{\pgfqpoint{1.396677in}{0.929686in}}{\pgfqpoint{1.391378in}{0.927491in}}{\pgfqpoint{1.387471in}{0.923584in}}%
\pgfpathcurveto{\pgfqpoint{1.383564in}{0.919677in}}{\pgfqpoint{1.381369in}{0.914378in}}{\pgfqpoint{1.381369in}{0.908853in}}%
\pgfpathcurveto{\pgfqpoint{1.381369in}{0.903328in}}{\pgfqpoint{1.383564in}{0.898028in}}{\pgfqpoint{1.387471in}{0.894122in}}%
\pgfpathcurveto{\pgfqpoint{1.391378in}{0.890215in}}{\pgfqpoint{1.396677in}{0.888020in}}{\pgfqpoint{1.402203in}{0.888020in}}%
\pgfpathclose%
\pgfusepath{fill}%
\end{pgfscope}%
\begin{pgfscope}%
\pgfpathrectangle{\pgfqpoint{0.833185in}{0.833185in}}{\pgfqpoint{1.162500in}{0.755000in}} %
\pgfusepath{clip}%
\pgfsetbuttcap%
\pgfsetroundjoin%
\definecolor{currentfill}{rgb}{0.000000,0.000000,0.000000}%
\pgfsetfillcolor{currentfill}%
\pgfsetfillopacity{0.500000}%
\pgfsetlinewidth{0.000000pt}%
\definecolor{currentstroke}{rgb}{0.000000,0.000000,0.000000}%
\pgfsetstrokecolor{currentstroke}%
\pgfsetdash{}{0pt}%
\pgfpathmoveto{\pgfqpoint{1.053510in}{1.424328in}}%
\pgfpathcurveto{\pgfqpoint{1.059035in}{1.424328in}}{\pgfqpoint{1.064335in}{1.426523in}}{\pgfqpoint{1.068242in}{1.430430in}}%
\pgfpathcurveto{\pgfqpoint{1.072148in}{1.434337in}}{\pgfqpoint{1.074344in}{1.439636in}}{\pgfqpoint{1.074344in}{1.445161in}}%
\pgfpathcurveto{\pgfqpoint{1.074344in}{1.450687in}}{\pgfqpoint{1.072148in}{1.455986in}}{\pgfqpoint{1.068242in}{1.459893in}}%
\pgfpathcurveto{\pgfqpoint{1.064335in}{1.463800in}}{\pgfqpoint{1.059035in}{1.465995in}}{\pgfqpoint{1.053510in}{1.465995in}}%
\pgfpathcurveto{\pgfqpoint{1.047985in}{1.465995in}}{\pgfqpoint{1.042686in}{1.463800in}}{\pgfqpoint{1.038779in}{1.459893in}}%
\pgfpathcurveto{\pgfqpoint{1.034872in}{1.455986in}}{\pgfqpoint{1.032677in}{1.450687in}}{\pgfqpoint{1.032677in}{1.445161in}}%
\pgfpathcurveto{\pgfqpoint{1.032677in}{1.439636in}}{\pgfqpoint{1.034872in}{1.434337in}}{\pgfqpoint{1.038779in}{1.430430in}}%
\pgfpathcurveto{\pgfqpoint{1.042686in}{1.426523in}}{\pgfqpoint{1.047985in}{1.424328in}}{\pgfqpoint{1.053510in}{1.424328in}}%
\pgfpathclose%
\pgfusepath{fill}%
\end{pgfscope}%
\begin{pgfscope}%
\pgfpathrectangle{\pgfqpoint{0.833185in}{0.833185in}}{\pgfqpoint{1.162500in}{0.755000in}} %
\pgfusepath{clip}%
\pgfsetbuttcap%
\pgfsetroundjoin%
\definecolor{currentfill}{rgb}{0.000000,0.000000,0.000000}%
\pgfsetfillcolor{currentfill}%
\pgfsetfillopacity{0.500000}%
\pgfsetlinewidth{0.000000pt}%
\definecolor{currentstroke}{rgb}{0.000000,0.000000,0.000000}%
\pgfsetstrokecolor{currentstroke}%
\pgfsetdash{}{0pt}%
\pgfpathmoveto{\pgfqpoint{1.492358in}{1.186596in}}%
\pgfpathcurveto{\pgfqpoint{1.497883in}{1.186596in}}{\pgfqpoint{1.503182in}{1.188791in}}{\pgfqpoint{1.507089in}{1.192698in}}%
\pgfpathcurveto{\pgfqpoint{1.510996in}{1.196604in}}{\pgfqpoint{1.513191in}{1.201904in}}{\pgfqpoint{1.513191in}{1.207429in}}%
\pgfpathcurveto{\pgfqpoint{1.513191in}{1.212954in}}{\pgfqpoint{1.510996in}{1.218254in}}{\pgfqpoint{1.507089in}{1.222160in}}%
\pgfpathcurveto{\pgfqpoint{1.503182in}{1.226067in}}{\pgfqpoint{1.497883in}{1.228262in}}{\pgfqpoint{1.492358in}{1.228262in}}%
\pgfpathcurveto{\pgfqpoint{1.486832in}{1.228262in}}{\pgfqpoint{1.481533in}{1.226067in}}{\pgfqpoint{1.477626in}{1.222160in}}%
\pgfpathcurveto{\pgfqpoint{1.473719in}{1.218254in}}{\pgfqpoint{1.471524in}{1.212954in}}{\pgfqpoint{1.471524in}{1.207429in}}%
\pgfpathcurveto{\pgfqpoint{1.471524in}{1.201904in}}{\pgfqpoint{1.473719in}{1.196604in}}{\pgfqpoint{1.477626in}{1.192698in}}%
\pgfpathcurveto{\pgfqpoint{1.481533in}{1.188791in}}{\pgfqpoint{1.486832in}{1.186596in}}{\pgfqpoint{1.492358in}{1.186596in}}%
\pgfpathclose%
\pgfusepath{fill}%
\end{pgfscope}%
\begin{pgfscope}%
\pgfsetbuttcap%
\pgfsetroundjoin%
\definecolor{currentfill}{rgb}{0.000000,0.000000,0.000000}%
\pgfsetfillcolor{currentfill}%
\pgfsetlinewidth{0.803000pt}%
\definecolor{currentstroke}{rgb}{0.000000,0.000000,0.000000}%
\pgfsetstrokecolor{currentstroke}%
\pgfsetdash{}{0pt}%
\pgfsys@defobject{currentmarker}{\pgfqpoint{0.000000in}{-0.048611in}}{\pgfqpoint{0.000000in}{0.000000in}}{%
\pgfpathmoveto{\pgfqpoint{0.000000in}{0.000000in}}%
\pgfpathlineto{\pgfqpoint{0.000000in}{-0.048611in}}%
\pgfusepath{stroke,fill}%
}%
\begin{pgfscope}%
\pgfsys@transformshift{0.935719in}{0.833185in}%
\pgfsys@useobject{currentmarker}{}%
\end{pgfscope}%
\end{pgfscope}%
\begin{pgfscope}%
\pgftext[x=0.966372in,y=0.585111in,left,base,rotate=90.000000]{\rmfamily\fontsize{8.000000}{9.600000}\selectfont \(\displaystyle 0.5\)}%
\end{pgfscope}%
\begin{pgfscope}%
\pgfsetbuttcap%
\pgfsetroundjoin%
\definecolor{currentfill}{rgb}{0.000000,0.000000,0.000000}%
\pgfsetfillcolor{currentfill}%
\pgfsetlinewidth{0.803000pt}%
\definecolor{currentstroke}{rgb}{0.000000,0.000000,0.000000}%
\pgfsetstrokecolor{currentstroke}%
\pgfsetdash{}{0pt}%
\pgfsys@defobject{currentmarker}{\pgfqpoint{0.000000in}{-0.048611in}}{\pgfqpoint{0.000000in}{0.000000in}}{%
\pgfpathmoveto{\pgfqpoint{0.000000in}{0.000000in}}%
\pgfpathlineto{\pgfqpoint{0.000000in}{-0.048611in}}%
\pgfusepath{stroke,fill}%
}%
\begin{pgfscope}%
\pgfsys@transformshift{1.360696in}{0.833185in}%
\pgfsys@useobject{currentmarker}{}%
\end{pgfscope}%
\end{pgfscope}%
\begin{pgfscope}%
\pgftext[x=1.391350in,y=0.585111in,left,base,rotate=90.000000]{\rmfamily\fontsize{8.000000}{9.600000}\selectfont \(\displaystyle 1.0\)}%
\end{pgfscope}%
\begin{pgfscope}%
\pgfsetbuttcap%
\pgfsetroundjoin%
\definecolor{currentfill}{rgb}{0.000000,0.000000,0.000000}%
\pgfsetfillcolor{currentfill}%
\pgfsetlinewidth{0.803000pt}%
\definecolor{currentstroke}{rgb}{0.000000,0.000000,0.000000}%
\pgfsetstrokecolor{currentstroke}%
\pgfsetdash{}{0pt}%
\pgfsys@defobject{currentmarker}{\pgfqpoint{0.000000in}{-0.048611in}}{\pgfqpoint{0.000000in}{0.000000in}}{%
\pgfpathmoveto{\pgfqpoint{0.000000in}{0.000000in}}%
\pgfpathlineto{\pgfqpoint{0.000000in}{-0.048611in}}%
\pgfusepath{stroke,fill}%
}%
\begin{pgfscope}%
\pgfsys@transformshift{1.785674in}{0.833185in}%
\pgfsys@useobject{currentmarker}{}%
\end{pgfscope}%
\end{pgfscope}%
\begin{pgfscope}%
\pgftext[x=1.816328in,y=0.585111in,left,base,rotate=90.000000]{\rmfamily\fontsize{8.000000}{9.600000}\selectfont \(\displaystyle 1.5\)}%
\end{pgfscope}%
\begin{pgfscope}%
\pgftext[x=1.414435in,y=0.529556in,,top]{\rmfamily\fontsize{10.000000}{12.000000}\selectfont Ef0}%
\end{pgfscope}%
\begin{pgfscope}%
\pgfsetbuttcap%
\pgfsetroundjoin%
\definecolor{currentfill}{rgb}{0.000000,0.000000,0.000000}%
\pgfsetfillcolor{currentfill}%
\pgfsetlinewidth{0.803000pt}%
\definecolor{currentstroke}{rgb}{0.000000,0.000000,0.000000}%
\pgfsetstrokecolor{currentstroke}%
\pgfsetdash{}{0pt}%
\pgfsys@defobject{currentmarker}{\pgfqpoint{-0.048611in}{0.000000in}}{\pgfqpoint{0.000000in}{0.000000in}}{%
\pgfpathmoveto{\pgfqpoint{0.000000in}{0.000000in}}%
\pgfpathlineto{\pgfqpoint{-0.048611in}{0.000000in}}%
\pgfusepath{stroke,fill}%
}%
\begin{pgfscope}%
\pgfsys@transformshift{0.833185in}{0.915277in}%
\pgfsys@useobject{currentmarker}{}%
\end{pgfscope}%
\end{pgfscope}%
\begin{pgfscope}%
\pgftext[x=0.467054in,y=0.873068in,left,base]{\rmfamily\fontsize{8.000000}{9.600000}\selectfont \(\displaystyle 0.025\)}%
\end{pgfscope}%
\begin{pgfscope}%
\pgfsetbuttcap%
\pgfsetroundjoin%
\definecolor{currentfill}{rgb}{0.000000,0.000000,0.000000}%
\pgfsetfillcolor{currentfill}%
\pgfsetlinewidth{0.803000pt}%
\definecolor{currentstroke}{rgb}{0.000000,0.000000,0.000000}%
\pgfsetstrokecolor{currentstroke}%
\pgfsetdash{}{0pt}%
\pgfsys@defobject{currentmarker}{\pgfqpoint{-0.048611in}{0.000000in}}{\pgfqpoint{0.000000in}{0.000000in}}{%
\pgfpathmoveto{\pgfqpoint{0.000000in}{0.000000in}}%
\pgfpathlineto{\pgfqpoint{-0.048611in}{0.000000in}}%
\pgfusepath{stroke,fill}%
}%
\begin{pgfscope}%
\pgfsys@transformshift{0.833185in}{1.179603in}%
\pgfsys@useobject{currentmarker}{}%
\end{pgfscope}%
\end{pgfscope}%
\begin{pgfscope}%
\pgftext[x=0.467054in,y=1.137394in,left,base]{\rmfamily\fontsize{8.000000}{9.600000}\selectfont \(\displaystyle 0.050\)}%
\end{pgfscope}%
\begin{pgfscope}%
\pgfsetbuttcap%
\pgfsetroundjoin%
\definecolor{currentfill}{rgb}{0.000000,0.000000,0.000000}%
\pgfsetfillcolor{currentfill}%
\pgfsetlinewidth{0.803000pt}%
\definecolor{currentstroke}{rgb}{0.000000,0.000000,0.000000}%
\pgfsetstrokecolor{currentstroke}%
\pgfsetdash{}{0pt}%
\pgfsys@defobject{currentmarker}{\pgfqpoint{-0.048611in}{0.000000in}}{\pgfqpoint{0.000000in}{0.000000in}}{%
\pgfpathmoveto{\pgfqpoint{0.000000in}{0.000000in}}%
\pgfpathlineto{\pgfqpoint{-0.048611in}{0.000000in}}%
\pgfusepath{stroke,fill}%
}%
\begin{pgfscope}%
\pgfsys@transformshift{0.833185in}{1.443929in}%
\pgfsys@useobject{currentmarker}{}%
\end{pgfscope}%
\end{pgfscope}%
\begin{pgfscope}%
\pgftext[x=0.467054in,y=1.401719in,left,base]{\rmfamily\fontsize{8.000000}{9.600000}\selectfont \(\displaystyle 0.075\)}%
\end{pgfscope}%
\begin{pgfscope}%
\pgftext[x=0.411499in,y=1.210685in,,bottom,rotate=90.000000]{\rmfamily\fontsize{10.000000}{12.000000}\selectfont u0}%
\end{pgfscope}%
\begin{pgfscope}%
\pgfsetrectcap%
\pgfsetmiterjoin%
\pgfsetlinewidth{0.803000pt}%
\definecolor{currentstroke}{rgb}{0.000000,0.000000,0.000000}%
\pgfsetstrokecolor{currentstroke}%
\pgfsetdash{}{0pt}%
\pgfpathmoveto{\pgfqpoint{0.833185in}{0.833185in}}%
\pgfpathlineto{\pgfqpoint{0.833185in}{1.588185in}}%
\pgfusepath{stroke}%
\end{pgfscope}%
\begin{pgfscope}%
\pgfsetrectcap%
\pgfsetmiterjoin%
\pgfsetlinewidth{0.803000pt}%
\definecolor{currentstroke}{rgb}{0.000000,0.000000,0.000000}%
\pgfsetstrokecolor{currentstroke}%
\pgfsetdash{}{0pt}%
\pgfpathmoveto{\pgfqpoint{1.995685in}{0.833185in}}%
\pgfpathlineto{\pgfqpoint{1.995685in}{1.588185in}}%
\pgfusepath{stroke}%
\end{pgfscope}%
\begin{pgfscope}%
\pgfsetrectcap%
\pgfsetmiterjoin%
\pgfsetlinewidth{0.803000pt}%
\definecolor{currentstroke}{rgb}{0.000000,0.000000,0.000000}%
\pgfsetstrokecolor{currentstroke}%
\pgfsetdash{}{0pt}%
\pgfpathmoveto{\pgfqpoint{0.833185in}{0.833185in}}%
\pgfpathlineto{\pgfqpoint{1.995685in}{0.833185in}}%
\pgfusepath{stroke}%
\end{pgfscope}%
\begin{pgfscope}%
\pgfsetrectcap%
\pgfsetmiterjoin%
\pgfsetlinewidth{0.803000pt}%
\definecolor{currentstroke}{rgb}{0.000000,0.000000,0.000000}%
\pgfsetstrokecolor{currentstroke}%
\pgfsetdash{}{0pt}%
\pgfpathmoveto{\pgfqpoint{0.833185in}{1.588185in}}%
\pgfpathlineto{\pgfqpoint{1.995685in}{1.588185in}}%
\pgfusepath{stroke}%
\end{pgfscope}%
\begin{pgfscope}%
\pgfsetbuttcap%
\pgfsetmiterjoin%
\definecolor{currentfill}{rgb}{1.000000,1.000000,1.000000}%
\pgfsetfillcolor{currentfill}%
\pgfsetlinewidth{0.000000pt}%
\definecolor{currentstroke}{rgb}{0.000000,0.000000,0.000000}%
\pgfsetstrokecolor{currentstroke}%
\pgfsetstrokeopacity{0.000000}%
\pgfsetdash{}{0pt}%
\pgfpathmoveto{\pgfqpoint{1.995685in}{0.833185in}}%
\pgfpathlineto{\pgfqpoint{3.158185in}{0.833185in}}%
\pgfpathlineto{\pgfqpoint{3.158185in}{1.588185in}}%
\pgfpathlineto{\pgfqpoint{1.995685in}{1.588185in}}%
\pgfpathclose%
\pgfusepath{fill}%
\end{pgfscope}%
\begin{pgfscope}%
\pgfpathrectangle{\pgfqpoint{1.995685in}{0.833185in}}{\pgfqpoint{1.162500in}{0.755000in}} %
\pgfusepath{clip}%
\pgfsetbuttcap%
\pgfsetroundjoin%
\definecolor{currentfill}{rgb}{0.000000,0.000000,0.000000}%
\pgfsetfillcolor{currentfill}%
\pgfsetfillopacity{0.500000}%
\pgfsetlinewidth{0.000000pt}%
\definecolor{currentstroke}{rgb}{0.000000,0.000000,0.000000}%
\pgfsetstrokecolor{currentstroke}%
\pgfsetdash{}{0pt}%
\pgfpathmoveto{\pgfqpoint{3.130506in}{1.099971in}}%
\pgfpathcurveto{\pgfqpoint{3.136031in}{1.099971in}}{\pgfqpoint{3.141331in}{1.102166in}}{\pgfqpoint{3.145237in}{1.106073in}}%
\pgfpathcurveto{\pgfqpoint{3.149144in}{1.109980in}}{\pgfqpoint{3.151339in}{1.115279in}}{\pgfqpoint{3.151339in}{1.120804in}}%
\pgfpathcurveto{\pgfqpoint{3.151339in}{1.126329in}}{\pgfqpoint{3.149144in}{1.131629in}}{\pgfqpoint{3.145237in}{1.135536in}}%
\pgfpathcurveto{\pgfqpoint{3.141331in}{1.139442in}}{\pgfqpoint{3.136031in}{1.141638in}}{\pgfqpoint{3.130506in}{1.141638in}}%
\pgfpathcurveto{\pgfqpoint{3.124981in}{1.141638in}}{\pgfqpoint{3.119681in}{1.139442in}}{\pgfqpoint{3.115775in}{1.135536in}}%
\pgfpathcurveto{\pgfqpoint{3.111868in}{1.131629in}}{\pgfqpoint{3.109673in}{1.126329in}}{\pgfqpoint{3.109673in}{1.120804in}}%
\pgfpathcurveto{\pgfqpoint{3.109673in}{1.115279in}}{\pgfqpoint{3.111868in}{1.109980in}}{\pgfqpoint{3.115775in}{1.106073in}}%
\pgfpathcurveto{\pgfqpoint{3.119681in}{1.102166in}}{\pgfqpoint{3.124981in}{1.099971in}}{\pgfqpoint{3.130506in}{1.099971in}}%
\pgfpathclose%
\pgfusepath{fill}%
\end{pgfscope}%
\begin{pgfscope}%
\pgfpathrectangle{\pgfqpoint{1.995685in}{0.833185in}}{\pgfqpoint{1.162500in}{0.755000in}} %
\pgfusepath{clip}%
\pgfsetbuttcap%
\pgfsetroundjoin%
\definecolor{currentfill}{rgb}{0.000000,0.000000,0.000000}%
\pgfsetfillcolor{currentfill}%
\pgfsetfillopacity{0.500000}%
\pgfsetlinewidth{0.000000pt}%
\definecolor{currentstroke}{rgb}{0.000000,0.000000,0.000000}%
\pgfsetstrokecolor{currentstroke}%
\pgfsetdash{}{0pt}%
\pgfpathmoveto{\pgfqpoint{2.755547in}{1.503836in}}%
\pgfpathcurveto{\pgfqpoint{2.761072in}{1.503836in}}{\pgfqpoint{2.766372in}{1.506031in}}{\pgfqpoint{2.770279in}{1.509938in}}%
\pgfpathcurveto{\pgfqpoint{2.774186in}{1.513845in}}{\pgfqpoint{2.776381in}{1.519144in}}{\pgfqpoint{2.776381in}{1.524669in}}%
\pgfpathcurveto{\pgfqpoint{2.776381in}{1.530195in}}{\pgfqpoint{2.774186in}{1.535494in}}{\pgfqpoint{2.770279in}{1.539401in}}%
\pgfpathcurveto{\pgfqpoint{2.766372in}{1.543308in}}{\pgfqpoint{2.761072in}{1.545503in}}{\pgfqpoint{2.755547in}{1.545503in}}%
\pgfpathcurveto{\pgfqpoint{2.750022in}{1.545503in}}{\pgfqpoint{2.744723in}{1.543308in}}{\pgfqpoint{2.740816in}{1.539401in}}%
\pgfpathcurveto{\pgfqpoint{2.736909in}{1.535494in}}{\pgfqpoint{2.734714in}{1.530195in}}{\pgfqpoint{2.734714in}{1.524669in}}%
\pgfpathcurveto{\pgfqpoint{2.734714in}{1.519144in}}{\pgfqpoint{2.736909in}{1.513845in}}{\pgfqpoint{2.740816in}{1.509938in}}%
\pgfpathcurveto{\pgfqpoint{2.744723in}{1.506031in}}{\pgfqpoint{2.750022in}{1.503836in}}{\pgfqpoint{2.755547in}{1.503836in}}%
\pgfpathclose%
\pgfusepath{fill}%
\end{pgfscope}%
\begin{pgfscope}%
\pgfpathrectangle{\pgfqpoint{1.995685in}{0.833185in}}{\pgfqpoint{1.162500in}{0.755000in}} %
\pgfusepath{clip}%
\pgfsetbuttcap%
\pgfsetroundjoin%
\definecolor{currentfill}{rgb}{0.000000,0.000000,0.000000}%
\pgfsetfillcolor{currentfill}%
\pgfsetfillopacity{0.500000}%
\pgfsetlinewidth{0.000000pt}%
\definecolor{currentstroke}{rgb}{0.000000,0.000000,0.000000}%
\pgfsetstrokecolor{currentstroke}%
\pgfsetdash{}{0pt}%
\pgfpathmoveto{\pgfqpoint{2.755127in}{1.549375in}}%
\pgfpathcurveto{\pgfqpoint{2.760652in}{1.549375in}}{\pgfqpoint{2.765952in}{1.551570in}}{\pgfqpoint{2.769858in}{1.555477in}}%
\pgfpathcurveto{\pgfqpoint{2.773765in}{1.559384in}}{\pgfqpoint{2.775960in}{1.564683in}}{\pgfqpoint{2.775960in}{1.570208in}}%
\pgfpathcurveto{\pgfqpoint{2.775960in}{1.575733in}}{\pgfqpoint{2.773765in}{1.581033in}}{\pgfqpoint{2.769858in}{1.584940in}}%
\pgfpathcurveto{\pgfqpoint{2.765952in}{1.588847in}}{\pgfqpoint{2.760652in}{1.591042in}}{\pgfqpoint{2.755127in}{1.591042in}}%
\pgfpathcurveto{\pgfqpoint{2.749602in}{1.591042in}}{\pgfqpoint{2.744302in}{1.588847in}}{\pgfqpoint{2.740396in}{1.584940in}}%
\pgfpathcurveto{\pgfqpoint{2.736489in}{1.581033in}}{\pgfqpoint{2.734294in}{1.575733in}}{\pgfqpoint{2.734294in}{1.570208in}}%
\pgfpathcurveto{\pgfqpoint{2.734294in}{1.564683in}}{\pgfqpoint{2.736489in}{1.559384in}}{\pgfqpoint{2.740396in}{1.555477in}}%
\pgfpathcurveto{\pgfqpoint{2.744302in}{1.551570in}}{\pgfqpoint{2.749602in}{1.549375in}}{\pgfqpoint{2.755127in}{1.549375in}}%
\pgfpathclose%
\pgfusepath{fill}%
\end{pgfscope}%
\begin{pgfscope}%
\pgfpathrectangle{\pgfqpoint{1.995685in}{0.833185in}}{\pgfqpoint{1.162500in}{0.755000in}} %
\pgfusepath{clip}%
\pgfsetbuttcap%
\pgfsetroundjoin%
\definecolor{currentfill}{rgb}{0.000000,0.000000,0.000000}%
\pgfsetfillcolor{currentfill}%
\pgfsetfillopacity{0.500000}%
\pgfsetlinewidth{0.000000pt}%
\definecolor{currentstroke}{rgb}{0.000000,0.000000,0.000000}%
\pgfsetstrokecolor{currentstroke}%
\pgfsetdash{}{0pt}%
\pgfpathmoveto{\pgfqpoint{2.317114in}{1.273758in}}%
\pgfpathcurveto{\pgfqpoint{2.322639in}{1.273758in}}{\pgfqpoint{2.327939in}{1.275953in}}{\pgfqpoint{2.331845in}{1.279859in}}%
\pgfpathcurveto{\pgfqpoint{2.335752in}{1.283766in}}{\pgfqpoint{2.337947in}{1.289066in}}{\pgfqpoint{2.337947in}{1.294591in}}%
\pgfpathcurveto{\pgfqpoint{2.337947in}{1.300116in}}{\pgfqpoint{2.335752in}{1.305415in}}{\pgfqpoint{2.331845in}{1.309322in}}%
\pgfpathcurveto{\pgfqpoint{2.327939in}{1.313229in}}{\pgfqpoint{2.322639in}{1.315424in}}{\pgfqpoint{2.317114in}{1.315424in}}%
\pgfpathcurveto{\pgfqpoint{2.311589in}{1.315424in}}{\pgfqpoint{2.306289in}{1.313229in}}{\pgfqpoint{2.302383in}{1.309322in}}%
\pgfpathcurveto{\pgfqpoint{2.298476in}{1.305415in}}{\pgfqpoint{2.296281in}{1.300116in}}{\pgfqpoint{2.296281in}{1.294591in}}%
\pgfpathcurveto{\pgfqpoint{2.296281in}{1.289066in}}{\pgfqpoint{2.298476in}{1.283766in}}{\pgfqpoint{2.302383in}{1.279859in}}%
\pgfpathcurveto{\pgfqpoint{2.306289in}{1.275953in}}{\pgfqpoint{2.311589in}{1.273758in}}{\pgfqpoint{2.317114in}{1.273758in}}%
\pgfpathclose%
\pgfusepath{fill}%
\end{pgfscope}%
\begin{pgfscope}%
\pgfpathrectangle{\pgfqpoint{1.995685in}{0.833185in}}{\pgfqpoint{1.162500in}{0.755000in}} %
\pgfusepath{clip}%
\pgfsetbuttcap%
\pgfsetroundjoin%
\definecolor{currentfill}{rgb}{0.000000,0.000000,0.000000}%
\pgfsetfillcolor{currentfill}%
\pgfsetfillopacity{0.500000}%
\pgfsetlinewidth{0.000000pt}%
\definecolor{currentstroke}{rgb}{0.000000,0.000000,0.000000}%
\pgfsetstrokecolor{currentstroke}%
\pgfsetdash{}{0pt}%
\pgfpathmoveto{\pgfqpoint{2.241285in}{1.319843in}}%
\pgfpathcurveto{\pgfqpoint{2.246810in}{1.319843in}}{\pgfqpoint{2.252110in}{1.322038in}}{\pgfqpoint{2.256017in}{1.325945in}}%
\pgfpathcurveto{\pgfqpoint{2.259923in}{1.329852in}}{\pgfqpoint{2.262118in}{1.335151in}}{\pgfqpoint{2.262118in}{1.340676in}}%
\pgfpathcurveto{\pgfqpoint{2.262118in}{1.346201in}}{\pgfqpoint{2.259923in}{1.351501in}}{\pgfqpoint{2.256017in}{1.355408in}}%
\pgfpathcurveto{\pgfqpoint{2.252110in}{1.359314in}}{\pgfqpoint{2.246810in}{1.361510in}}{\pgfqpoint{2.241285in}{1.361510in}}%
\pgfpathcurveto{\pgfqpoint{2.235760in}{1.361510in}}{\pgfqpoint{2.230461in}{1.359314in}}{\pgfqpoint{2.226554in}{1.355408in}}%
\pgfpathcurveto{\pgfqpoint{2.222647in}{1.351501in}}{\pgfqpoint{2.220452in}{1.346201in}}{\pgfqpoint{2.220452in}{1.340676in}}%
\pgfpathcurveto{\pgfqpoint{2.220452in}{1.335151in}}{\pgfqpoint{2.222647in}{1.329852in}}{\pgfqpoint{2.226554in}{1.325945in}}%
\pgfpathcurveto{\pgfqpoint{2.230461in}{1.322038in}}{\pgfqpoint{2.235760in}{1.319843in}}{\pgfqpoint{2.241285in}{1.319843in}}%
\pgfpathclose%
\pgfusepath{fill}%
\end{pgfscope}%
\begin{pgfscope}%
\pgfpathrectangle{\pgfqpoint{1.995685in}{0.833185in}}{\pgfqpoint{1.162500in}{0.755000in}} %
\pgfusepath{clip}%
\pgfsetbuttcap%
\pgfsetroundjoin%
\definecolor{currentfill}{rgb}{0.000000,0.000000,0.000000}%
\pgfsetfillcolor{currentfill}%
\pgfsetfillopacity{0.500000}%
\pgfsetlinewidth{0.000000pt}%
\definecolor{currentstroke}{rgb}{0.000000,0.000000,0.000000}%
\pgfsetstrokecolor{currentstroke}%
\pgfsetdash{}{0pt}%
\pgfpathmoveto{\pgfqpoint{2.106066in}{1.215122in}}%
\pgfpathcurveto{\pgfqpoint{2.111591in}{1.215122in}}{\pgfqpoint{2.116890in}{1.217317in}}{\pgfqpoint{2.120797in}{1.221224in}}%
\pgfpathcurveto{\pgfqpoint{2.124704in}{1.225130in}}{\pgfqpoint{2.126899in}{1.230430in}}{\pgfqpoint{2.126899in}{1.235955in}}%
\pgfpathcurveto{\pgfqpoint{2.126899in}{1.241480in}}{\pgfqpoint{2.124704in}{1.246780in}}{\pgfqpoint{2.120797in}{1.250686in}}%
\pgfpathcurveto{\pgfqpoint{2.116890in}{1.254593in}}{\pgfqpoint{2.111591in}{1.256788in}}{\pgfqpoint{2.106066in}{1.256788in}}%
\pgfpathcurveto{\pgfqpoint{2.100541in}{1.256788in}}{\pgfqpoint{2.095241in}{1.254593in}}{\pgfqpoint{2.091334in}{1.250686in}}%
\pgfpathcurveto{\pgfqpoint{2.087428in}{1.246780in}}{\pgfqpoint{2.085232in}{1.241480in}}{\pgfqpoint{2.085232in}{1.235955in}}%
\pgfpathcurveto{\pgfqpoint{2.085232in}{1.230430in}}{\pgfqpoint{2.087428in}{1.225130in}}{\pgfqpoint{2.091334in}{1.221224in}}%
\pgfpathcurveto{\pgfqpoint{2.095241in}{1.217317in}}{\pgfqpoint{2.100541in}{1.215122in}}{\pgfqpoint{2.106066in}{1.215122in}}%
\pgfpathclose%
\pgfusepath{fill}%
\end{pgfscope}%
\begin{pgfscope}%
\pgfpathrectangle{\pgfqpoint{1.995685in}{0.833185in}}{\pgfqpoint{1.162500in}{0.755000in}} %
\pgfusepath{clip}%
\pgfsetbuttcap%
\pgfsetroundjoin%
\definecolor{currentfill}{rgb}{0.000000,0.000000,0.000000}%
\pgfsetfillcolor{currentfill}%
\pgfsetfillopacity{0.500000}%
\pgfsetlinewidth{0.000000pt}%
\definecolor{currentstroke}{rgb}{0.000000,0.000000,0.000000}%
\pgfsetstrokecolor{currentstroke}%
\pgfsetdash{}{0pt}%
\pgfpathmoveto{\pgfqpoint{2.023363in}{0.830327in}}%
\pgfpathcurveto{\pgfqpoint{2.028888in}{0.830327in}}{\pgfqpoint{2.034188in}{0.832523in}}{\pgfqpoint{2.038095in}{0.836429in}}%
\pgfpathcurveto{\pgfqpoint{2.042001in}{0.840336in}}{\pgfqpoint{2.044196in}{0.845636in}}{\pgfqpoint{2.044196in}{0.851161in}}%
\pgfpathcurveto{\pgfqpoint{2.044196in}{0.856686in}}{\pgfqpoint{2.042001in}{0.861985in}}{\pgfqpoint{2.038095in}{0.865892in}}%
\pgfpathcurveto{\pgfqpoint{2.034188in}{0.869799in}}{\pgfqpoint{2.028888in}{0.871994in}}{\pgfqpoint{2.023363in}{0.871994in}}%
\pgfpathcurveto{\pgfqpoint{2.017838in}{0.871994in}}{\pgfqpoint{2.012539in}{0.869799in}}{\pgfqpoint{2.008632in}{0.865892in}}%
\pgfpathcurveto{\pgfqpoint{2.004725in}{0.861985in}}{\pgfqpoint{2.002530in}{0.856686in}}{\pgfqpoint{2.002530in}{0.851161in}}%
\pgfpathcurveto{\pgfqpoint{2.002530in}{0.845636in}}{\pgfqpoint{2.004725in}{0.840336in}}{\pgfqpoint{2.008632in}{0.836429in}}%
\pgfpathcurveto{\pgfqpoint{2.012539in}{0.832523in}}{\pgfqpoint{2.017838in}{0.830327in}}{\pgfqpoint{2.023363in}{0.830327in}}%
\pgfpathclose%
\pgfusepath{fill}%
\end{pgfscope}%
\begin{pgfscope}%
\pgfpathrectangle{\pgfqpoint{1.995685in}{0.833185in}}{\pgfqpoint{1.162500in}{0.755000in}} %
\pgfusepath{clip}%
\pgfsetbuttcap%
\pgfsetroundjoin%
\definecolor{currentfill}{rgb}{0.000000,0.000000,0.000000}%
\pgfsetfillcolor{currentfill}%
\pgfsetfillopacity{0.500000}%
\pgfsetlinewidth{0.000000pt}%
\definecolor{currentstroke}{rgb}{0.000000,0.000000,0.000000}%
\pgfsetstrokecolor{currentstroke}%
\pgfsetdash{}{0pt}%
\pgfpathmoveto{\pgfqpoint{2.907651in}{1.469697in}}%
\pgfpathcurveto{\pgfqpoint{2.913176in}{1.469697in}}{\pgfqpoint{2.918476in}{1.471892in}}{\pgfqpoint{2.922382in}{1.475799in}}%
\pgfpathcurveto{\pgfqpoint{2.926289in}{1.479706in}}{\pgfqpoint{2.928484in}{1.485005in}}{\pgfqpoint{2.928484in}{1.490530in}}%
\pgfpathcurveto{\pgfqpoint{2.928484in}{1.496056in}}{\pgfqpoint{2.926289in}{1.501355in}}{\pgfqpoint{2.922382in}{1.505262in}}%
\pgfpathcurveto{\pgfqpoint{2.918476in}{1.509169in}}{\pgfqpoint{2.913176in}{1.511364in}}{\pgfqpoint{2.907651in}{1.511364in}}%
\pgfpathcurveto{\pgfqpoint{2.902126in}{1.511364in}}{\pgfqpoint{2.896827in}{1.509169in}}{\pgfqpoint{2.892920in}{1.505262in}}%
\pgfpathcurveto{\pgfqpoint{2.889013in}{1.501355in}}{\pgfqpoint{2.886818in}{1.496056in}}{\pgfqpoint{2.886818in}{1.490530in}}%
\pgfpathcurveto{\pgfqpoint{2.886818in}{1.485005in}}{\pgfqpoint{2.889013in}{1.479706in}}{\pgfqpoint{2.892920in}{1.475799in}}%
\pgfpathcurveto{\pgfqpoint{2.896827in}{1.471892in}}{\pgfqpoint{2.902126in}{1.469697in}}{\pgfqpoint{2.907651in}{1.469697in}}%
\pgfpathclose%
\pgfusepath{fill}%
\end{pgfscope}%
\begin{pgfscope}%
\pgfpathrectangle{\pgfqpoint{1.995685in}{0.833185in}}{\pgfqpoint{1.162500in}{0.755000in}} %
\pgfusepath{clip}%
\pgfsetbuttcap%
\pgfsetroundjoin%
\definecolor{currentfill}{rgb}{0.000000,0.000000,0.000000}%
\pgfsetfillcolor{currentfill}%
\pgfsetfillopacity{0.500000}%
\pgfsetlinewidth{0.000000pt}%
\definecolor{currentstroke}{rgb}{0.000000,0.000000,0.000000}%
\pgfsetstrokecolor{currentstroke}%
\pgfsetdash{}{0pt}%
\pgfpathmoveto{\pgfqpoint{2.630361in}{1.312260in}}%
\pgfpathcurveto{\pgfqpoint{2.635886in}{1.312260in}}{\pgfqpoint{2.641186in}{1.314455in}}{\pgfqpoint{2.645093in}{1.318362in}}%
\pgfpathcurveto{\pgfqpoint{2.649000in}{1.322269in}}{\pgfqpoint{2.651195in}{1.327569in}}{\pgfqpoint{2.651195in}{1.333094in}}%
\pgfpathcurveto{\pgfqpoint{2.651195in}{1.338619in}}{\pgfqpoint{2.649000in}{1.343918in}}{\pgfqpoint{2.645093in}{1.347825in}}%
\pgfpathcurveto{\pgfqpoint{2.641186in}{1.351732in}}{\pgfqpoint{2.635886in}{1.353927in}}{\pgfqpoint{2.630361in}{1.353927in}}%
\pgfpathcurveto{\pgfqpoint{2.624836in}{1.353927in}}{\pgfqpoint{2.619537in}{1.351732in}}{\pgfqpoint{2.615630in}{1.347825in}}%
\pgfpathcurveto{\pgfqpoint{2.611723in}{1.343918in}}{\pgfqpoint{2.609528in}{1.338619in}}{\pgfqpoint{2.609528in}{1.333094in}}%
\pgfpathcurveto{\pgfqpoint{2.609528in}{1.327569in}}{\pgfqpoint{2.611723in}{1.322269in}}{\pgfqpoint{2.615630in}{1.318362in}}%
\pgfpathcurveto{\pgfqpoint{2.619537in}{1.314455in}}{\pgfqpoint{2.624836in}{1.312260in}}{\pgfqpoint{2.630361in}{1.312260in}}%
\pgfpathclose%
\pgfusepath{fill}%
\end{pgfscope}%
\begin{pgfscope}%
\pgfpathrectangle{\pgfqpoint{1.995685in}{0.833185in}}{\pgfqpoint{1.162500in}{0.755000in}} %
\pgfusepath{clip}%
\pgfsetbuttcap%
\pgfsetroundjoin%
\definecolor{currentfill}{rgb}{0.000000,0.000000,0.000000}%
\pgfsetfillcolor{currentfill}%
\pgfsetfillopacity{0.500000}%
\pgfsetlinewidth{0.000000pt}%
\definecolor{currentstroke}{rgb}{0.000000,0.000000,0.000000}%
\pgfsetstrokecolor{currentstroke}%
\pgfsetdash{}{0pt}%
\pgfpathmoveto{\pgfqpoint{2.487752in}{1.134966in}}%
\pgfpathcurveto{\pgfqpoint{2.493277in}{1.134966in}}{\pgfqpoint{2.498576in}{1.137161in}}{\pgfqpoint{2.502483in}{1.141068in}}%
\pgfpathcurveto{\pgfqpoint{2.506390in}{1.144975in}}{\pgfqpoint{2.508585in}{1.150275in}}{\pgfqpoint{2.508585in}{1.155800in}}%
\pgfpathcurveto{\pgfqpoint{2.508585in}{1.161325in}}{\pgfqpoint{2.506390in}{1.166624in}}{\pgfqpoint{2.502483in}{1.170531in}}%
\pgfpathcurveto{\pgfqpoint{2.498576in}{1.174438in}}{\pgfqpoint{2.493277in}{1.176633in}}{\pgfqpoint{2.487752in}{1.176633in}}%
\pgfpathcurveto{\pgfqpoint{2.482227in}{1.176633in}}{\pgfqpoint{2.476927in}{1.174438in}}{\pgfqpoint{2.473020in}{1.170531in}}%
\pgfpathcurveto{\pgfqpoint{2.469113in}{1.166624in}}{\pgfqpoint{2.466918in}{1.161325in}}{\pgfqpoint{2.466918in}{1.155800in}}%
\pgfpathcurveto{\pgfqpoint{2.466918in}{1.150275in}}{\pgfqpoint{2.469113in}{1.144975in}}{\pgfqpoint{2.473020in}{1.141068in}}%
\pgfpathcurveto{\pgfqpoint{2.476927in}{1.137161in}}{\pgfqpoint{2.482227in}{1.134966in}}{\pgfqpoint{2.487752in}{1.134966in}}%
\pgfpathclose%
\pgfusepath{fill}%
\end{pgfscope}%
\begin{pgfscope}%
\pgfpathrectangle{\pgfqpoint{1.995685in}{0.833185in}}{\pgfqpoint{1.162500in}{0.755000in}} %
\pgfusepath{clip}%
\pgfsetbuttcap%
\pgfsetroundjoin%
\definecolor{currentfill}{rgb}{0.000000,0.000000,0.000000}%
\pgfsetfillcolor{currentfill}%
\pgfsetfillopacity{0.500000}%
\pgfsetlinewidth{0.000000pt}%
\definecolor{currentstroke}{rgb}{0.000000,0.000000,0.000000}%
\pgfsetstrokecolor{currentstroke}%
\pgfsetdash{}{0pt}%
\pgfpathmoveto{\pgfqpoint{3.000738in}{1.421242in}}%
\pgfpathcurveto{\pgfqpoint{3.006263in}{1.421242in}}{\pgfqpoint{3.011563in}{1.423437in}}{\pgfqpoint{3.015470in}{1.427344in}}%
\pgfpathcurveto{\pgfqpoint{3.019376in}{1.431251in}}{\pgfqpoint{3.021572in}{1.436550in}}{\pgfqpoint{3.021572in}{1.442075in}}%
\pgfpathcurveto{\pgfqpoint{3.021572in}{1.447600in}}{\pgfqpoint{3.019376in}{1.452900in}}{\pgfqpoint{3.015470in}{1.456807in}}%
\pgfpathcurveto{\pgfqpoint{3.011563in}{1.460714in}}{\pgfqpoint{3.006263in}{1.462909in}}{\pgfqpoint{3.000738in}{1.462909in}}%
\pgfpathcurveto{\pgfqpoint{2.995213in}{1.462909in}}{\pgfqpoint{2.989914in}{1.460714in}}{\pgfqpoint{2.986007in}{1.456807in}}%
\pgfpathcurveto{\pgfqpoint{2.982100in}{1.452900in}}{\pgfqpoint{2.979905in}{1.447600in}}{\pgfqpoint{2.979905in}{1.442075in}}%
\pgfpathcurveto{\pgfqpoint{2.979905in}{1.436550in}}{\pgfqpoint{2.982100in}{1.431251in}}{\pgfqpoint{2.986007in}{1.427344in}}%
\pgfpathcurveto{\pgfqpoint{2.989914in}{1.423437in}}{\pgfqpoint{2.995213in}{1.421242in}}{\pgfqpoint{3.000738in}{1.421242in}}%
\pgfpathclose%
\pgfusepath{fill}%
\end{pgfscope}%
\begin{pgfscope}%
\pgfpathrectangle{\pgfqpoint{1.995685in}{0.833185in}}{\pgfqpoint{1.162500in}{0.755000in}} %
\pgfusepath{clip}%
\pgfsetbuttcap%
\pgfsetroundjoin%
\definecolor{currentfill}{rgb}{0.000000,0.000000,0.000000}%
\pgfsetfillcolor{currentfill}%
\pgfsetfillopacity{0.500000}%
\pgfsetlinewidth{0.000000pt}%
\definecolor{currentstroke}{rgb}{0.000000,0.000000,0.000000}%
\pgfsetstrokecolor{currentstroke}%
\pgfsetdash{}{0pt}%
\pgfpathmoveto{\pgfqpoint{2.318352in}{1.126134in}}%
\pgfpathcurveto{\pgfqpoint{2.323877in}{1.126134in}}{\pgfqpoint{2.329176in}{1.128329in}}{\pgfqpoint{2.333083in}{1.132236in}}%
\pgfpathcurveto{\pgfqpoint{2.336990in}{1.136143in}}{\pgfqpoint{2.339185in}{1.141442in}}{\pgfqpoint{2.339185in}{1.146967in}}%
\pgfpathcurveto{\pgfqpoint{2.339185in}{1.152492in}}{\pgfqpoint{2.336990in}{1.157792in}}{\pgfqpoint{2.333083in}{1.161699in}}%
\pgfpathcurveto{\pgfqpoint{2.329176in}{1.165606in}}{\pgfqpoint{2.323877in}{1.167801in}}{\pgfqpoint{2.318352in}{1.167801in}}%
\pgfpathcurveto{\pgfqpoint{2.312827in}{1.167801in}}{\pgfqpoint{2.307527in}{1.165606in}}{\pgfqpoint{2.303620in}{1.161699in}}%
\pgfpathcurveto{\pgfqpoint{2.299713in}{1.157792in}}{\pgfqpoint{2.297518in}{1.152492in}}{\pgfqpoint{2.297518in}{1.146967in}}%
\pgfpathcurveto{\pgfqpoint{2.297518in}{1.141442in}}{\pgfqpoint{2.299713in}{1.136143in}}{\pgfqpoint{2.303620in}{1.132236in}}%
\pgfpathcurveto{\pgfqpoint{2.307527in}{1.128329in}}{\pgfqpoint{2.312827in}{1.126134in}}{\pgfqpoint{2.318352in}{1.126134in}}%
\pgfpathclose%
\pgfusepath{fill}%
\end{pgfscope}%
\begin{pgfscope}%
\pgfpathrectangle{\pgfqpoint{1.995685in}{0.833185in}}{\pgfqpoint{1.162500in}{0.755000in}} %
\pgfusepath{clip}%
\pgfsetbuttcap%
\pgfsetroundjoin%
\definecolor{currentfill}{rgb}{0.000000,0.000000,0.000000}%
\pgfsetfillcolor{currentfill}%
\pgfsetfillopacity{0.500000}%
\pgfsetlinewidth{0.000000pt}%
\definecolor{currentstroke}{rgb}{0.000000,0.000000,0.000000}%
\pgfsetstrokecolor{currentstroke}%
\pgfsetdash{}{0pt}%
\pgfpathmoveto{\pgfqpoint{2.410257in}{0.888020in}}%
\pgfpathcurveto{\pgfqpoint{2.415782in}{0.888020in}}{\pgfqpoint{2.421082in}{0.890215in}}{\pgfqpoint{2.424989in}{0.894122in}}%
\pgfpathcurveto{\pgfqpoint{2.428896in}{0.898028in}}{\pgfqpoint{2.431091in}{0.903328in}}{\pgfqpoint{2.431091in}{0.908853in}}%
\pgfpathcurveto{\pgfqpoint{2.431091in}{0.914378in}}{\pgfqpoint{2.428896in}{0.919677in}}{\pgfqpoint{2.424989in}{0.923584in}}%
\pgfpathcurveto{\pgfqpoint{2.421082in}{0.927491in}}{\pgfqpoint{2.415782in}{0.929686in}}{\pgfqpoint{2.410257in}{0.929686in}}%
\pgfpathcurveto{\pgfqpoint{2.404732in}{0.929686in}}{\pgfqpoint{2.399433in}{0.927491in}}{\pgfqpoint{2.395526in}{0.923584in}}%
\pgfpathcurveto{\pgfqpoint{2.391619in}{0.919677in}}{\pgfqpoint{2.389424in}{0.914378in}}{\pgfqpoint{2.389424in}{0.908853in}}%
\pgfpathcurveto{\pgfqpoint{2.389424in}{0.903328in}}{\pgfqpoint{2.391619in}{0.898028in}}{\pgfqpoint{2.395526in}{0.894122in}}%
\pgfpathcurveto{\pgfqpoint{2.399433in}{0.890215in}}{\pgfqpoint{2.404732in}{0.888020in}}{\pgfqpoint{2.410257in}{0.888020in}}%
\pgfpathclose%
\pgfusepath{fill}%
\end{pgfscope}%
\begin{pgfscope}%
\pgfpathrectangle{\pgfqpoint{1.995685in}{0.833185in}}{\pgfqpoint{1.162500in}{0.755000in}} %
\pgfusepath{clip}%
\pgfsetbuttcap%
\pgfsetroundjoin%
\definecolor{currentfill}{rgb}{0.000000,0.000000,0.000000}%
\pgfsetfillcolor{currentfill}%
\pgfsetfillopacity{0.500000}%
\pgfsetlinewidth{0.000000pt}%
\definecolor{currentstroke}{rgb}{0.000000,0.000000,0.000000}%
\pgfsetstrokecolor{currentstroke}%
\pgfsetdash{}{0pt}%
\pgfpathmoveto{\pgfqpoint{2.752801in}{1.424328in}}%
\pgfpathcurveto{\pgfqpoint{2.758327in}{1.424328in}}{\pgfqpoint{2.763626in}{1.426523in}}{\pgfqpoint{2.767533in}{1.430430in}}%
\pgfpathcurveto{\pgfqpoint{2.771440in}{1.434337in}}{\pgfqpoint{2.773635in}{1.439636in}}{\pgfqpoint{2.773635in}{1.445161in}}%
\pgfpathcurveto{\pgfqpoint{2.773635in}{1.450687in}}{\pgfqpoint{2.771440in}{1.455986in}}{\pgfqpoint{2.767533in}{1.459893in}}%
\pgfpathcurveto{\pgfqpoint{2.763626in}{1.463800in}}{\pgfqpoint{2.758327in}{1.465995in}}{\pgfqpoint{2.752801in}{1.465995in}}%
\pgfpathcurveto{\pgfqpoint{2.747276in}{1.465995in}}{\pgfqpoint{2.741977in}{1.463800in}}{\pgfqpoint{2.738070in}{1.459893in}}%
\pgfpathcurveto{\pgfqpoint{2.734163in}{1.455986in}}{\pgfqpoint{2.731968in}{1.450687in}}{\pgfqpoint{2.731968in}{1.445161in}}%
\pgfpathcurveto{\pgfqpoint{2.731968in}{1.439636in}}{\pgfqpoint{2.734163in}{1.434337in}}{\pgfqpoint{2.738070in}{1.430430in}}%
\pgfpathcurveto{\pgfqpoint{2.741977in}{1.426523in}}{\pgfqpoint{2.747276in}{1.424328in}}{\pgfqpoint{2.752801in}{1.424328in}}%
\pgfpathclose%
\pgfusepath{fill}%
\end{pgfscope}%
\begin{pgfscope}%
\pgfpathrectangle{\pgfqpoint{1.995685in}{0.833185in}}{\pgfqpoint{1.162500in}{0.755000in}} %
\pgfusepath{clip}%
\pgfsetbuttcap%
\pgfsetroundjoin%
\definecolor{currentfill}{rgb}{0.000000,0.000000,0.000000}%
\pgfsetfillcolor{currentfill}%
\pgfsetfillopacity{0.500000}%
\pgfsetlinewidth{0.000000pt}%
\definecolor{currentstroke}{rgb}{0.000000,0.000000,0.000000}%
\pgfsetstrokecolor{currentstroke}%
\pgfsetdash{}{0pt}%
\pgfpathmoveto{\pgfqpoint{2.103337in}{1.186596in}}%
\pgfpathcurveto{\pgfqpoint{2.108862in}{1.186596in}}{\pgfqpoint{2.114161in}{1.188791in}}{\pgfqpoint{2.118068in}{1.192698in}}%
\pgfpathcurveto{\pgfqpoint{2.121975in}{1.196604in}}{\pgfqpoint{2.124170in}{1.201904in}}{\pgfqpoint{2.124170in}{1.207429in}}%
\pgfpathcurveto{\pgfqpoint{2.124170in}{1.212954in}}{\pgfqpoint{2.121975in}{1.218254in}}{\pgfqpoint{2.118068in}{1.222160in}}%
\pgfpathcurveto{\pgfqpoint{2.114161in}{1.226067in}}{\pgfqpoint{2.108862in}{1.228262in}}{\pgfqpoint{2.103337in}{1.228262in}}%
\pgfpathcurveto{\pgfqpoint{2.097812in}{1.228262in}}{\pgfqpoint{2.092512in}{1.226067in}}{\pgfqpoint{2.088605in}{1.222160in}}%
\pgfpathcurveto{\pgfqpoint{2.084698in}{1.218254in}}{\pgfqpoint{2.082503in}{1.212954in}}{\pgfqpoint{2.082503in}{1.207429in}}%
\pgfpathcurveto{\pgfqpoint{2.082503in}{1.201904in}}{\pgfqpoint{2.084698in}{1.196604in}}{\pgfqpoint{2.088605in}{1.192698in}}%
\pgfpathcurveto{\pgfqpoint{2.092512in}{1.188791in}}{\pgfqpoint{2.097812in}{1.186596in}}{\pgfqpoint{2.103337in}{1.186596in}}%
\pgfpathclose%
\pgfusepath{fill}%
\end{pgfscope}%
\begin{pgfscope}%
\pgfsetbuttcap%
\pgfsetroundjoin%
\definecolor{currentfill}{rgb}{0.000000,0.000000,0.000000}%
\pgfsetfillcolor{currentfill}%
\pgfsetlinewidth{0.803000pt}%
\definecolor{currentstroke}{rgb}{0.000000,0.000000,0.000000}%
\pgfsetstrokecolor{currentstroke}%
\pgfsetdash{}{0pt}%
\pgfsys@defobject{currentmarker}{\pgfqpoint{0.000000in}{-0.048611in}}{\pgfqpoint{0.000000in}{0.000000in}}{%
\pgfpathmoveto{\pgfqpoint{0.000000in}{0.000000in}}%
\pgfpathlineto{\pgfqpoint{0.000000in}{-0.048611in}}%
\pgfusepath{stroke,fill}%
}%
\begin{pgfscope}%
\pgfsys@transformshift{2.474914in}{0.833185in}%
\pgfsys@useobject{currentmarker}{}%
\end{pgfscope}%
\end{pgfscope}%
\begin{pgfscope}%
\pgftext[x=2.505567in,y=0.289968in,left,base,rotate=90.000000]{\rmfamily\fontsize{8.000000}{9.600000}\selectfont \(\displaystyle 0.000025\)}%
\end{pgfscope}%
\begin{pgfscope}%
\pgfsetbuttcap%
\pgfsetroundjoin%
\definecolor{currentfill}{rgb}{0.000000,0.000000,0.000000}%
\pgfsetfillcolor{currentfill}%
\pgfsetlinewidth{0.803000pt}%
\definecolor{currentstroke}{rgb}{0.000000,0.000000,0.000000}%
\pgfsetstrokecolor{currentstroke}%
\pgfsetdash{}{0pt}%
\pgfsys@defobject{currentmarker}{\pgfqpoint{0.000000in}{-0.048611in}}{\pgfqpoint{0.000000in}{0.000000in}}{%
\pgfpathmoveto{\pgfqpoint{0.000000in}{0.000000in}}%
\pgfpathlineto{\pgfqpoint{0.000000in}{-0.048611in}}%
\pgfusepath{stroke,fill}%
}%
\begin{pgfscope}%
\pgfsys@transformshift{3.075299in}{0.833185in}%
\pgfsys@useobject{currentmarker}{}%
\end{pgfscope}%
\end{pgfscope}%
\begin{pgfscope}%
\pgftext[x=3.105952in,y=0.289968in,left,base,rotate=90.000000]{\rmfamily\fontsize{8.000000}{9.600000}\selectfont \(\displaystyle 0.000050\)}%
\end{pgfscope}%
\begin{pgfscope}%
\pgftext[x=2.576935in,y=0.234413in,,top]{\rmfamily\fontsize{10.000000}{12.000000}\selectfont area}%
\end{pgfscope}%
\begin{pgfscope}%
\pgfsetrectcap%
\pgfsetmiterjoin%
\pgfsetlinewidth{0.803000pt}%
\definecolor{currentstroke}{rgb}{0.000000,0.000000,0.000000}%
\pgfsetstrokecolor{currentstroke}%
\pgfsetdash{}{0pt}%
\pgfpathmoveto{\pgfqpoint{1.995685in}{0.833185in}}%
\pgfpathlineto{\pgfqpoint{1.995685in}{1.588185in}}%
\pgfusepath{stroke}%
\end{pgfscope}%
\begin{pgfscope}%
\pgfsetrectcap%
\pgfsetmiterjoin%
\pgfsetlinewidth{0.803000pt}%
\definecolor{currentstroke}{rgb}{0.000000,0.000000,0.000000}%
\pgfsetstrokecolor{currentstroke}%
\pgfsetdash{}{0pt}%
\pgfpathmoveto{\pgfqpoint{3.158185in}{0.833185in}}%
\pgfpathlineto{\pgfqpoint{3.158185in}{1.588185in}}%
\pgfusepath{stroke}%
\end{pgfscope}%
\begin{pgfscope}%
\pgfsetrectcap%
\pgfsetmiterjoin%
\pgfsetlinewidth{0.803000pt}%
\definecolor{currentstroke}{rgb}{0.000000,0.000000,0.000000}%
\pgfsetstrokecolor{currentstroke}%
\pgfsetdash{}{0pt}%
\pgfpathmoveto{\pgfqpoint{1.995685in}{0.833185in}}%
\pgfpathlineto{\pgfqpoint{3.158185in}{0.833185in}}%
\pgfusepath{stroke}%
\end{pgfscope}%
\begin{pgfscope}%
\pgfsetrectcap%
\pgfsetmiterjoin%
\pgfsetlinewidth{0.803000pt}%
\definecolor{currentstroke}{rgb}{0.000000,0.000000,0.000000}%
\pgfsetstrokecolor{currentstroke}%
\pgfsetdash{}{0pt}%
\pgfpathmoveto{\pgfqpoint{1.995685in}{1.588185in}}%
\pgfpathlineto{\pgfqpoint{3.158185in}{1.588185in}}%
\pgfusepath{stroke}%
\end{pgfscope}%
\begin{pgfscope}%
\pgfsetbuttcap%
\pgfsetmiterjoin%
\definecolor{currentfill}{rgb}{1.000000,1.000000,1.000000}%
\pgfsetfillcolor{currentfill}%
\pgfsetlinewidth{0.000000pt}%
\definecolor{currentstroke}{rgb}{0.000000,0.000000,0.000000}%
\pgfsetstrokecolor{currentstroke}%
\pgfsetstrokeopacity{0.000000}%
\pgfsetdash{}{0pt}%
\pgfpathmoveto{\pgfqpoint{3.158185in}{0.833185in}}%
\pgfpathlineto{\pgfqpoint{4.320685in}{0.833185in}}%
\pgfpathlineto{\pgfqpoint{4.320685in}{1.588185in}}%
\pgfpathlineto{\pgfqpoint{3.158185in}{1.588185in}}%
\pgfpathclose%
\pgfusepath{fill}%
\end{pgfscope}%
\begin{pgfscope}%
\pgfpathrectangle{\pgfqpoint{3.158185in}{0.833185in}}{\pgfqpoint{1.162500in}{0.755000in}} %
\pgfusepath{clip}%
\pgfsetbuttcap%
\pgfsetroundjoin%
\definecolor{currentfill}{rgb}{0.000000,0.000000,0.000000}%
\pgfsetfillcolor{currentfill}%
\pgfsetfillopacity{0.500000}%
\pgfsetlinewidth{0.000000pt}%
\definecolor{currentstroke}{rgb}{0.000000,0.000000,0.000000}%
\pgfsetstrokecolor{currentstroke}%
\pgfsetdash{}{0pt}%
\pgfpathmoveto{\pgfqpoint{4.293006in}{1.099971in}}%
\pgfpathcurveto{\pgfqpoint{4.298531in}{1.099971in}}{\pgfqpoint{4.303831in}{1.102166in}}{\pgfqpoint{4.307737in}{1.106073in}}%
\pgfpathcurveto{\pgfqpoint{4.311644in}{1.109980in}}{\pgfqpoint{4.313839in}{1.115279in}}{\pgfqpoint{4.313839in}{1.120804in}}%
\pgfpathcurveto{\pgfqpoint{4.313839in}{1.126329in}}{\pgfqpoint{4.311644in}{1.131629in}}{\pgfqpoint{4.307737in}{1.135536in}}%
\pgfpathcurveto{\pgfqpoint{4.303831in}{1.139442in}}{\pgfqpoint{4.298531in}{1.141638in}}{\pgfqpoint{4.293006in}{1.141638in}}%
\pgfpathcurveto{\pgfqpoint{4.287481in}{1.141638in}}{\pgfqpoint{4.282181in}{1.139442in}}{\pgfqpoint{4.278275in}{1.135536in}}%
\pgfpathcurveto{\pgfqpoint{4.274368in}{1.131629in}}{\pgfqpoint{4.272173in}{1.126329in}}{\pgfqpoint{4.272173in}{1.120804in}}%
\pgfpathcurveto{\pgfqpoint{4.272173in}{1.115279in}}{\pgfqpoint{4.274368in}{1.109980in}}{\pgfqpoint{4.278275in}{1.106073in}}%
\pgfpathcurveto{\pgfqpoint{4.282181in}{1.102166in}}{\pgfqpoint{4.287481in}{1.099971in}}{\pgfqpoint{4.293006in}{1.099971in}}%
\pgfpathclose%
\pgfusepath{fill}%
\end{pgfscope}%
\begin{pgfscope}%
\pgfpathrectangle{\pgfqpoint{3.158185in}{0.833185in}}{\pgfqpoint{1.162500in}{0.755000in}} %
\pgfusepath{clip}%
\pgfsetbuttcap%
\pgfsetroundjoin%
\definecolor{currentfill}{rgb}{0.000000,0.000000,0.000000}%
\pgfsetfillcolor{currentfill}%
\pgfsetfillopacity{0.500000}%
\pgfsetlinewidth{0.000000pt}%
\definecolor{currentstroke}{rgb}{0.000000,0.000000,0.000000}%
\pgfsetstrokecolor{currentstroke}%
\pgfsetdash{}{0pt}%
\pgfpathmoveto{\pgfqpoint{3.586831in}{1.503836in}}%
\pgfpathcurveto{\pgfqpoint{3.592356in}{1.503836in}}{\pgfqpoint{3.597655in}{1.506031in}}{\pgfqpoint{3.601562in}{1.509938in}}%
\pgfpathcurveto{\pgfqpoint{3.605469in}{1.513845in}}{\pgfqpoint{3.607664in}{1.519144in}}{\pgfqpoint{3.607664in}{1.524669in}}%
\pgfpathcurveto{\pgfqpoint{3.607664in}{1.530195in}}{\pgfqpoint{3.605469in}{1.535494in}}{\pgfqpoint{3.601562in}{1.539401in}}%
\pgfpathcurveto{\pgfqpoint{3.597655in}{1.543308in}}{\pgfqpoint{3.592356in}{1.545503in}}{\pgfqpoint{3.586831in}{1.545503in}}%
\pgfpathcurveto{\pgfqpoint{3.581306in}{1.545503in}}{\pgfqpoint{3.576006in}{1.543308in}}{\pgfqpoint{3.572099in}{1.539401in}}%
\pgfpathcurveto{\pgfqpoint{3.568193in}{1.535494in}}{\pgfqpoint{3.565998in}{1.530195in}}{\pgfqpoint{3.565998in}{1.524669in}}%
\pgfpathcurveto{\pgfqpoint{3.565998in}{1.519144in}}{\pgfqpoint{3.568193in}{1.513845in}}{\pgfqpoint{3.572099in}{1.509938in}}%
\pgfpathcurveto{\pgfqpoint{3.576006in}{1.506031in}}{\pgfqpoint{3.581306in}{1.503836in}}{\pgfqpoint{3.586831in}{1.503836in}}%
\pgfpathclose%
\pgfusepath{fill}%
\end{pgfscope}%
\begin{pgfscope}%
\pgfpathrectangle{\pgfqpoint{3.158185in}{0.833185in}}{\pgfqpoint{1.162500in}{0.755000in}} %
\pgfusepath{clip}%
\pgfsetbuttcap%
\pgfsetroundjoin%
\definecolor{currentfill}{rgb}{0.000000,0.000000,0.000000}%
\pgfsetfillcolor{currentfill}%
\pgfsetfillopacity{0.500000}%
\pgfsetlinewidth{0.000000pt}%
\definecolor{currentstroke}{rgb}{0.000000,0.000000,0.000000}%
\pgfsetstrokecolor{currentstroke}%
\pgfsetdash{}{0pt}%
\pgfpathmoveto{\pgfqpoint{3.578870in}{1.549375in}}%
\pgfpathcurveto{\pgfqpoint{3.584395in}{1.549375in}}{\pgfqpoint{3.589694in}{1.551570in}}{\pgfqpoint{3.593601in}{1.555477in}}%
\pgfpathcurveto{\pgfqpoint{3.597508in}{1.559384in}}{\pgfqpoint{3.599703in}{1.564683in}}{\pgfqpoint{3.599703in}{1.570208in}}%
\pgfpathcurveto{\pgfqpoint{3.599703in}{1.575733in}}{\pgfqpoint{3.597508in}{1.581033in}}{\pgfqpoint{3.593601in}{1.584940in}}%
\pgfpathcurveto{\pgfqpoint{3.589694in}{1.588847in}}{\pgfqpoint{3.584395in}{1.591042in}}{\pgfqpoint{3.578870in}{1.591042in}}%
\pgfpathcurveto{\pgfqpoint{3.573345in}{1.591042in}}{\pgfqpoint{3.568045in}{1.588847in}}{\pgfqpoint{3.564138in}{1.584940in}}%
\pgfpathcurveto{\pgfqpoint{3.560231in}{1.581033in}}{\pgfqpoint{3.558036in}{1.575733in}}{\pgfqpoint{3.558036in}{1.570208in}}%
\pgfpathcurveto{\pgfqpoint{3.558036in}{1.564683in}}{\pgfqpoint{3.560231in}{1.559384in}}{\pgfqpoint{3.564138in}{1.555477in}}%
\pgfpathcurveto{\pgfqpoint{3.568045in}{1.551570in}}{\pgfqpoint{3.573345in}{1.549375in}}{\pgfqpoint{3.578870in}{1.549375in}}%
\pgfpathclose%
\pgfusepath{fill}%
\end{pgfscope}%
\begin{pgfscope}%
\pgfpathrectangle{\pgfqpoint{3.158185in}{0.833185in}}{\pgfqpoint{1.162500in}{0.755000in}} %
\pgfusepath{clip}%
\pgfsetbuttcap%
\pgfsetroundjoin%
\definecolor{currentfill}{rgb}{0.000000,0.000000,0.000000}%
\pgfsetfillcolor{currentfill}%
\pgfsetfillopacity{0.500000}%
\pgfsetlinewidth{0.000000pt}%
\definecolor{currentstroke}{rgb}{0.000000,0.000000,0.000000}%
\pgfsetstrokecolor{currentstroke}%
\pgfsetdash{}{0pt}%
\pgfpathmoveto{\pgfqpoint{3.333069in}{1.273758in}}%
\pgfpathcurveto{\pgfqpoint{3.338594in}{1.273758in}}{\pgfqpoint{3.343894in}{1.275953in}}{\pgfqpoint{3.347801in}{1.279859in}}%
\pgfpathcurveto{\pgfqpoint{3.351708in}{1.283766in}}{\pgfqpoint{3.353903in}{1.289066in}}{\pgfqpoint{3.353903in}{1.294591in}}%
\pgfpathcurveto{\pgfqpoint{3.353903in}{1.300116in}}{\pgfqpoint{3.351708in}{1.305415in}}{\pgfqpoint{3.347801in}{1.309322in}}%
\pgfpathcurveto{\pgfqpoint{3.343894in}{1.313229in}}{\pgfqpoint{3.338594in}{1.315424in}}{\pgfqpoint{3.333069in}{1.315424in}}%
\pgfpathcurveto{\pgfqpoint{3.327544in}{1.315424in}}{\pgfqpoint{3.322245in}{1.313229in}}{\pgfqpoint{3.318338in}{1.309322in}}%
\pgfpathcurveto{\pgfqpoint{3.314431in}{1.305415in}}{\pgfqpoint{3.312236in}{1.300116in}}{\pgfqpoint{3.312236in}{1.294591in}}%
\pgfpathcurveto{\pgfqpoint{3.312236in}{1.289066in}}{\pgfqpoint{3.314431in}{1.283766in}}{\pgfqpoint{3.318338in}{1.279859in}}%
\pgfpathcurveto{\pgfqpoint{3.322245in}{1.275953in}}{\pgfqpoint{3.327544in}{1.273758in}}{\pgfqpoint{3.333069in}{1.273758in}}%
\pgfpathclose%
\pgfusepath{fill}%
\end{pgfscope}%
\begin{pgfscope}%
\pgfpathrectangle{\pgfqpoint{3.158185in}{0.833185in}}{\pgfqpoint{1.162500in}{0.755000in}} %
\pgfusepath{clip}%
\pgfsetbuttcap%
\pgfsetroundjoin%
\definecolor{currentfill}{rgb}{0.000000,0.000000,0.000000}%
\pgfsetfillcolor{currentfill}%
\pgfsetfillopacity{0.500000}%
\pgfsetlinewidth{0.000000pt}%
\definecolor{currentstroke}{rgb}{0.000000,0.000000,0.000000}%
\pgfsetstrokecolor{currentstroke}%
\pgfsetdash{}{0pt}%
\pgfpathmoveto{\pgfqpoint{3.296013in}{1.319843in}}%
\pgfpathcurveto{\pgfqpoint{3.301538in}{1.319843in}}{\pgfqpoint{3.306838in}{1.322038in}}{\pgfqpoint{3.310744in}{1.325945in}}%
\pgfpathcurveto{\pgfqpoint{3.314651in}{1.329852in}}{\pgfqpoint{3.316846in}{1.335151in}}{\pgfqpoint{3.316846in}{1.340676in}}%
\pgfpathcurveto{\pgfqpoint{3.316846in}{1.346201in}}{\pgfqpoint{3.314651in}{1.351501in}}{\pgfqpoint{3.310744in}{1.355408in}}%
\pgfpathcurveto{\pgfqpoint{3.306838in}{1.359314in}}{\pgfqpoint{3.301538in}{1.361510in}}{\pgfqpoint{3.296013in}{1.361510in}}%
\pgfpathcurveto{\pgfqpoint{3.290488in}{1.361510in}}{\pgfqpoint{3.285188in}{1.359314in}}{\pgfqpoint{3.281282in}{1.355408in}}%
\pgfpathcurveto{\pgfqpoint{3.277375in}{1.351501in}}{\pgfqpoint{3.275180in}{1.346201in}}{\pgfqpoint{3.275180in}{1.340676in}}%
\pgfpathcurveto{\pgfqpoint{3.275180in}{1.335151in}}{\pgfqpoint{3.277375in}{1.329852in}}{\pgfqpoint{3.281282in}{1.325945in}}%
\pgfpathcurveto{\pgfqpoint{3.285188in}{1.322038in}}{\pgfqpoint{3.290488in}{1.319843in}}{\pgfqpoint{3.296013in}{1.319843in}}%
\pgfpathclose%
\pgfusepath{fill}%
\end{pgfscope}%
\begin{pgfscope}%
\pgfpathrectangle{\pgfqpoint{3.158185in}{0.833185in}}{\pgfqpoint{1.162500in}{0.755000in}} %
\pgfusepath{clip}%
\pgfsetbuttcap%
\pgfsetroundjoin%
\definecolor{currentfill}{rgb}{0.000000,0.000000,0.000000}%
\pgfsetfillcolor{currentfill}%
\pgfsetfillopacity{0.500000}%
\pgfsetlinewidth{0.000000pt}%
\definecolor{currentstroke}{rgb}{0.000000,0.000000,0.000000}%
\pgfsetstrokecolor{currentstroke}%
\pgfsetdash{}{0pt}%
\pgfpathmoveto{\pgfqpoint{3.206735in}{1.215122in}}%
\pgfpathcurveto{\pgfqpoint{3.212260in}{1.215122in}}{\pgfqpoint{3.217559in}{1.217317in}}{\pgfqpoint{3.221466in}{1.221224in}}%
\pgfpathcurveto{\pgfqpoint{3.225373in}{1.225130in}}{\pgfqpoint{3.227568in}{1.230430in}}{\pgfqpoint{3.227568in}{1.235955in}}%
\pgfpathcurveto{\pgfqpoint{3.227568in}{1.241480in}}{\pgfqpoint{3.225373in}{1.246780in}}{\pgfqpoint{3.221466in}{1.250686in}}%
\pgfpathcurveto{\pgfqpoint{3.217559in}{1.254593in}}{\pgfqpoint{3.212260in}{1.256788in}}{\pgfqpoint{3.206735in}{1.256788in}}%
\pgfpathcurveto{\pgfqpoint{3.201210in}{1.256788in}}{\pgfqpoint{3.195910in}{1.254593in}}{\pgfqpoint{3.192003in}{1.250686in}}%
\pgfpathcurveto{\pgfqpoint{3.188097in}{1.246780in}}{\pgfqpoint{3.185901in}{1.241480in}}{\pgfqpoint{3.185901in}{1.235955in}}%
\pgfpathcurveto{\pgfqpoint{3.185901in}{1.230430in}}{\pgfqpoint{3.188097in}{1.225130in}}{\pgfqpoint{3.192003in}{1.221224in}}%
\pgfpathcurveto{\pgfqpoint{3.195910in}{1.217317in}}{\pgfqpoint{3.201210in}{1.215122in}}{\pgfqpoint{3.206735in}{1.215122in}}%
\pgfpathclose%
\pgfusepath{fill}%
\end{pgfscope}%
\begin{pgfscope}%
\pgfpathrectangle{\pgfqpoint{3.158185in}{0.833185in}}{\pgfqpoint{1.162500in}{0.755000in}} %
\pgfusepath{clip}%
\pgfsetbuttcap%
\pgfsetroundjoin%
\definecolor{currentfill}{rgb}{0.000000,0.000000,0.000000}%
\pgfsetfillcolor{currentfill}%
\pgfsetfillopacity{0.500000}%
\pgfsetlinewidth{0.000000pt}%
\definecolor{currentstroke}{rgb}{0.000000,0.000000,0.000000}%
\pgfsetstrokecolor{currentstroke}%
\pgfsetdash{}{0pt}%
\pgfpathmoveto{\pgfqpoint{3.185863in}{0.830327in}}%
\pgfpathcurveto{\pgfqpoint{3.191388in}{0.830327in}}{\pgfqpoint{3.196688in}{0.832523in}}{\pgfqpoint{3.200595in}{0.836429in}}%
\pgfpathcurveto{\pgfqpoint{3.204501in}{0.840336in}}{\pgfqpoint{3.206696in}{0.845636in}}{\pgfqpoint{3.206696in}{0.851161in}}%
\pgfpathcurveto{\pgfqpoint{3.206696in}{0.856686in}}{\pgfqpoint{3.204501in}{0.861985in}}{\pgfqpoint{3.200595in}{0.865892in}}%
\pgfpathcurveto{\pgfqpoint{3.196688in}{0.869799in}}{\pgfqpoint{3.191388in}{0.871994in}}{\pgfqpoint{3.185863in}{0.871994in}}%
\pgfpathcurveto{\pgfqpoint{3.180338in}{0.871994in}}{\pgfqpoint{3.175039in}{0.869799in}}{\pgfqpoint{3.171132in}{0.865892in}}%
\pgfpathcurveto{\pgfqpoint{3.167225in}{0.861985in}}{\pgfqpoint{3.165030in}{0.856686in}}{\pgfqpoint{3.165030in}{0.851161in}}%
\pgfpathcurveto{\pgfqpoint{3.165030in}{0.845636in}}{\pgfqpoint{3.167225in}{0.840336in}}{\pgfqpoint{3.171132in}{0.836429in}}%
\pgfpathcurveto{\pgfqpoint{3.175039in}{0.832523in}}{\pgfqpoint{3.180338in}{0.830327in}}{\pgfqpoint{3.185863in}{0.830327in}}%
\pgfpathclose%
\pgfusepath{fill}%
\end{pgfscope}%
\begin{pgfscope}%
\pgfpathrectangle{\pgfqpoint{3.158185in}{0.833185in}}{\pgfqpoint{1.162500in}{0.755000in}} %
\pgfusepath{clip}%
\pgfsetbuttcap%
\pgfsetroundjoin%
\definecolor{currentfill}{rgb}{0.000000,0.000000,0.000000}%
\pgfsetfillcolor{currentfill}%
\pgfsetfillopacity{0.500000}%
\pgfsetlinewidth{0.000000pt}%
\definecolor{currentstroke}{rgb}{0.000000,0.000000,0.000000}%
\pgfsetstrokecolor{currentstroke}%
\pgfsetdash{}{0pt}%
\pgfpathmoveto{\pgfqpoint{3.808562in}{1.469697in}}%
\pgfpathcurveto{\pgfqpoint{3.814087in}{1.469697in}}{\pgfqpoint{3.819386in}{1.471892in}}{\pgfqpoint{3.823293in}{1.475799in}}%
\pgfpathcurveto{\pgfqpoint{3.827200in}{1.479706in}}{\pgfqpoint{3.829395in}{1.485005in}}{\pgfqpoint{3.829395in}{1.490530in}}%
\pgfpathcurveto{\pgfqpoint{3.829395in}{1.496056in}}{\pgfqpoint{3.827200in}{1.501355in}}{\pgfqpoint{3.823293in}{1.505262in}}%
\pgfpathcurveto{\pgfqpoint{3.819386in}{1.509169in}}{\pgfqpoint{3.814087in}{1.511364in}}{\pgfqpoint{3.808562in}{1.511364in}}%
\pgfpathcurveto{\pgfqpoint{3.803037in}{1.511364in}}{\pgfqpoint{3.797737in}{1.509169in}}{\pgfqpoint{3.793830in}{1.505262in}}%
\pgfpathcurveto{\pgfqpoint{3.789924in}{1.501355in}}{\pgfqpoint{3.787728in}{1.496056in}}{\pgfqpoint{3.787728in}{1.490530in}}%
\pgfpathcurveto{\pgfqpoint{3.787728in}{1.485005in}}{\pgfqpoint{3.789924in}{1.479706in}}{\pgfqpoint{3.793830in}{1.475799in}}%
\pgfpathcurveto{\pgfqpoint{3.797737in}{1.471892in}}{\pgfqpoint{3.803037in}{1.469697in}}{\pgfqpoint{3.808562in}{1.469697in}}%
\pgfpathclose%
\pgfusepath{fill}%
\end{pgfscope}%
\begin{pgfscope}%
\pgfpathrectangle{\pgfqpoint{3.158185in}{0.833185in}}{\pgfqpoint{1.162500in}{0.755000in}} %
\pgfusepath{clip}%
\pgfsetbuttcap%
\pgfsetroundjoin%
\definecolor{currentfill}{rgb}{0.000000,0.000000,0.000000}%
\pgfsetfillcolor{currentfill}%
\pgfsetfillopacity{0.500000}%
\pgfsetlinewidth{0.000000pt}%
\definecolor{currentstroke}{rgb}{0.000000,0.000000,0.000000}%
\pgfsetstrokecolor{currentstroke}%
\pgfsetdash{}{0pt}%
\pgfpathmoveto{\pgfqpoint{3.620933in}{1.312260in}}%
\pgfpathcurveto{\pgfqpoint{3.626458in}{1.312260in}}{\pgfqpoint{3.631758in}{1.314455in}}{\pgfqpoint{3.635664in}{1.318362in}}%
\pgfpathcurveto{\pgfqpoint{3.639571in}{1.322269in}}{\pgfqpoint{3.641766in}{1.327569in}}{\pgfqpoint{3.641766in}{1.333094in}}%
\pgfpathcurveto{\pgfqpoint{3.641766in}{1.338619in}}{\pgfqpoint{3.639571in}{1.343918in}}{\pgfqpoint{3.635664in}{1.347825in}}%
\pgfpathcurveto{\pgfqpoint{3.631758in}{1.351732in}}{\pgfqpoint{3.626458in}{1.353927in}}{\pgfqpoint{3.620933in}{1.353927in}}%
\pgfpathcurveto{\pgfqpoint{3.615408in}{1.353927in}}{\pgfqpoint{3.610108in}{1.351732in}}{\pgfqpoint{3.606202in}{1.347825in}}%
\pgfpathcurveto{\pgfqpoint{3.602295in}{1.343918in}}{\pgfqpoint{3.600100in}{1.338619in}}{\pgfqpoint{3.600100in}{1.333094in}}%
\pgfpathcurveto{\pgfqpoint{3.600100in}{1.327569in}}{\pgfqpoint{3.602295in}{1.322269in}}{\pgfqpoint{3.606202in}{1.318362in}}%
\pgfpathcurveto{\pgfqpoint{3.610108in}{1.314455in}}{\pgfqpoint{3.615408in}{1.312260in}}{\pgfqpoint{3.620933in}{1.312260in}}%
\pgfpathclose%
\pgfusepath{fill}%
\end{pgfscope}%
\begin{pgfscope}%
\pgfpathrectangle{\pgfqpoint{3.158185in}{0.833185in}}{\pgfqpoint{1.162500in}{0.755000in}} %
\pgfusepath{clip}%
\pgfsetbuttcap%
\pgfsetroundjoin%
\definecolor{currentfill}{rgb}{0.000000,0.000000,0.000000}%
\pgfsetfillcolor{currentfill}%
\pgfsetfillopacity{0.500000}%
\pgfsetlinewidth{0.000000pt}%
\definecolor{currentstroke}{rgb}{0.000000,0.000000,0.000000}%
\pgfsetstrokecolor{currentstroke}%
\pgfsetdash{}{0pt}%
\pgfpathmoveto{\pgfqpoint{3.600191in}{1.134966in}}%
\pgfpathcurveto{\pgfqpoint{3.605716in}{1.134966in}}{\pgfqpoint{3.611016in}{1.137161in}}{\pgfqpoint{3.614922in}{1.141068in}}%
\pgfpathcurveto{\pgfqpoint{3.618829in}{1.144975in}}{\pgfqpoint{3.621024in}{1.150275in}}{\pgfqpoint{3.621024in}{1.155800in}}%
\pgfpathcurveto{\pgfqpoint{3.621024in}{1.161325in}}{\pgfqpoint{3.618829in}{1.166624in}}{\pgfqpoint{3.614922in}{1.170531in}}%
\pgfpathcurveto{\pgfqpoint{3.611016in}{1.174438in}}{\pgfqpoint{3.605716in}{1.176633in}}{\pgfqpoint{3.600191in}{1.176633in}}%
\pgfpathcurveto{\pgfqpoint{3.594666in}{1.176633in}}{\pgfqpoint{3.589366in}{1.174438in}}{\pgfqpoint{3.585460in}{1.170531in}}%
\pgfpathcurveto{\pgfqpoint{3.581553in}{1.166624in}}{\pgfqpoint{3.579358in}{1.161325in}}{\pgfqpoint{3.579358in}{1.155800in}}%
\pgfpathcurveto{\pgfqpoint{3.579358in}{1.150275in}}{\pgfqpoint{3.581553in}{1.144975in}}{\pgfqpoint{3.585460in}{1.141068in}}%
\pgfpathcurveto{\pgfqpoint{3.589366in}{1.137161in}}{\pgfqpoint{3.594666in}{1.134966in}}{\pgfqpoint{3.600191in}{1.134966in}}%
\pgfpathclose%
\pgfusepath{fill}%
\end{pgfscope}%
\begin{pgfscope}%
\pgfpathrectangle{\pgfqpoint{3.158185in}{0.833185in}}{\pgfqpoint{1.162500in}{0.755000in}} %
\pgfusepath{clip}%
\pgfsetbuttcap%
\pgfsetroundjoin%
\definecolor{currentfill}{rgb}{0.000000,0.000000,0.000000}%
\pgfsetfillcolor{currentfill}%
\pgfsetfillopacity{0.500000}%
\pgfsetlinewidth{0.000000pt}%
\definecolor{currentstroke}{rgb}{0.000000,0.000000,0.000000}%
\pgfsetstrokecolor{currentstroke}%
\pgfsetdash{}{0pt}%
\pgfpathmoveto{\pgfqpoint{3.899960in}{1.421242in}}%
\pgfpathcurveto{\pgfqpoint{3.905485in}{1.421242in}}{\pgfqpoint{3.910785in}{1.423437in}}{\pgfqpoint{3.914692in}{1.427344in}}%
\pgfpathcurveto{\pgfqpoint{3.918598in}{1.431251in}}{\pgfqpoint{3.920794in}{1.436550in}}{\pgfqpoint{3.920794in}{1.442075in}}%
\pgfpathcurveto{\pgfqpoint{3.920794in}{1.447600in}}{\pgfqpoint{3.918598in}{1.452900in}}{\pgfqpoint{3.914692in}{1.456807in}}%
\pgfpathcurveto{\pgfqpoint{3.910785in}{1.460714in}}{\pgfqpoint{3.905485in}{1.462909in}}{\pgfqpoint{3.899960in}{1.462909in}}%
\pgfpathcurveto{\pgfqpoint{3.894435in}{1.462909in}}{\pgfqpoint{3.889136in}{1.460714in}}{\pgfqpoint{3.885229in}{1.456807in}}%
\pgfpathcurveto{\pgfqpoint{3.881322in}{1.452900in}}{\pgfqpoint{3.879127in}{1.447600in}}{\pgfqpoint{3.879127in}{1.442075in}}%
\pgfpathcurveto{\pgfqpoint{3.879127in}{1.436550in}}{\pgfqpoint{3.881322in}{1.431251in}}{\pgfqpoint{3.885229in}{1.427344in}}%
\pgfpathcurveto{\pgfqpoint{3.889136in}{1.423437in}}{\pgfqpoint{3.894435in}{1.421242in}}{\pgfqpoint{3.899960in}{1.421242in}}%
\pgfpathclose%
\pgfusepath{fill}%
\end{pgfscope}%
\begin{pgfscope}%
\pgfpathrectangle{\pgfqpoint{3.158185in}{0.833185in}}{\pgfqpoint{1.162500in}{0.755000in}} %
\pgfusepath{clip}%
\pgfsetbuttcap%
\pgfsetroundjoin%
\definecolor{currentfill}{rgb}{0.000000,0.000000,0.000000}%
\pgfsetfillcolor{currentfill}%
\pgfsetfillopacity{0.500000}%
\pgfsetlinewidth{0.000000pt}%
\definecolor{currentstroke}{rgb}{0.000000,0.000000,0.000000}%
\pgfsetstrokecolor{currentstroke}%
\pgfsetdash{}{0pt}%
\pgfpathmoveto{\pgfqpoint{3.368320in}{1.126134in}}%
\pgfpathcurveto{\pgfqpoint{3.373845in}{1.126134in}}{\pgfqpoint{3.379145in}{1.128329in}}{\pgfqpoint{3.383052in}{1.132236in}}%
\pgfpathcurveto{\pgfqpoint{3.386958in}{1.136143in}}{\pgfqpoint{3.389154in}{1.141442in}}{\pgfqpoint{3.389154in}{1.146967in}}%
\pgfpathcurveto{\pgfqpoint{3.389154in}{1.152492in}}{\pgfqpoint{3.386958in}{1.157792in}}{\pgfqpoint{3.383052in}{1.161699in}}%
\pgfpathcurveto{\pgfqpoint{3.379145in}{1.165606in}}{\pgfqpoint{3.373845in}{1.167801in}}{\pgfqpoint{3.368320in}{1.167801in}}%
\pgfpathcurveto{\pgfqpoint{3.362795in}{1.167801in}}{\pgfqpoint{3.357496in}{1.165606in}}{\pgfqpoint{3.353589in}{1.161699in}}%
\pgfpathcurveto{\pgfqpoint{3.349682in}{1.157792in}}{\pgfqpoint{3.347487in}{1.152492in}}{\pgfqpoint{3.347487in}{1.146967in}}%
\pgfpathcurveto{\pgfqpoint{3.347487in}{1.141442in}}{\pgfqpoint{3.349682in}{1.136143in}}{\pgfqpoint{3.353589in}{1.132236in}}%
\pgfpathcurveto{\pgfqpoint{3.357496in}{1.128329in}}{\pgfqpoint{3.362795in}{1.126134in}}{\pgfqpoint{3.368320in}{1.126134in}}%
\pgfpathclose%
\pgfusepath{fill}%
\end{pgfscope}%
\begin{pgfscope}%
\pgfpathrectangle{\pgfqpoint{3.158185in}{0.833185in}}{\pgfqpoint{1.162500in}{0.755000in}} %
\pgfusepath{clip}%
\pgfsetbuttcap%
\pgfsetroundjoin%
\definecolor{currentfill}{rgb}{0.000000,0.000000,0.000000}%
\pgfsetfillcolor{currentfill}%
\pgfsetfillopacity{0.500000}%
\pgfsetlinewidth{0.000000pt}%
\definecolor{currentstroke}{rgb}{0.000000,0.000000,0.000000}%
\pgfsetstrokecolor{currentstroke}%
\pgfsetdash{}{0pt}%
\pgfpathmoveto{\pgfqpoint{3.416175in}{0.888020in}}%
\pgfpathcurveto{\pgfqpoint{3.421700in}{0.888020in}}{\pgfqpoint{3.427000in}{0.890215in}}{\pgfqpoint{3.430907in}{0.894122in}}%
\pgfpathcurveto{\pgfqpoint{3.434814in}{0.898028in}}{\pgfqpoint{3.437009in}{0.903328in}}{\pgfqpoint{3.437009in}{0.908853in}}%
\pgfpathcurveto{\pgfqpoint{3.437009in}{0.914378in}}{\pgfqpoint{3.434814in}{0.919677in}}{\pgfqpoint{3.430907in}{0.923584in}}%
\pgfpathcurveto{\pgfqpoint{3.427000in}{0.927491in}}{\pgfqpoint{3.421700in}{0.929686in}}{\pgfqpoint{3.416175in}{0.929686in}}%
\pgfpathcurveto{\pgfqpoint{3.410650in}{0.929686in}}{\pgfqpoint{3.405351in}{0.927491in}}{\pgfqpoint{3.401444in}{0.923584in}}%
\pgfpathcurveto{\pgfqpoint{3.397537in}{0.919677in}}{\pgfqpoint{3.395342in}{0.914378in}}{\pgfqpoint{3.395342in}{0.908853in}}%
\pgfpathcurveto{\pgfqpoint{3.395342in}{0.903328in}}{\pgfqpoint{3.397537in}{0.898028in}}{\pgfqpoint{3.401444in}{0.894122in}}%
\pgfpathcurveto{\pgfqpoint{3.405351in}{0.890215in}}{\pgfqpoint{3.410650in}{0.888020in}}{\pgfqpoint{3.416175in}{0.888020in}}%
\pgfpathclose%
\pgfusepath{fill}%
\end{pgfscope}%
\begin{pgfscope}%
\pgfpathrectangle{\pgfqpoint{3.158185in}{0.833185in}}{\pgfqpoint{1.162500in}{0.755000in}} %
\pgfusepath{clip}%
\pgfsetbuttcap%
\pgfsetroundjoin%
\definecolor{currentfill}{rgb}{0.000000,0.000000,0.000000}%
\pgfsetfillcolor{currentfill}%
\pgfsetfillopacity{0.500000}%
\pgfsetlinewidth{0.000000pt}%
\definecolor{currentstroke}{rgb}{0.000000,0.000000,0.000000}%
\pgfsetstrokecolor{currentstroke}%
\pgfsetdash{}{0pt}%
\pgfpathmoveto{\pgfqpoint{3.789669in}{1.424328in}}%
\pgfpathcurveto{\pgfqpoint{3.795194in}{1.424328in}}{\pgfqpoint{3.800493in}{1.426523in}}{\pgfqpoint{3.804400in}{1.430430in}}%
\pgfpathcurveto{\pgfqpoint{3.808307in}{1.434337in}}{\pgfqpoint{3.810502in}{1.439636in}}{\pgfqpoint{3.810502in}{1.445161in}}%
\pgfpathcurveto{\pgfqpoint{3.810502in}{1.450687in}}{\pgfqpoint{3.808307in}{1.455986in}}{\pgfqpoint{3.804400in}{1.459893in}}%
\pgfpathcurveto{\pgfqpoint{3.800493in}{1.463800in}}{\pgfqpoint{3.795194in}{1.465995in}}{\pgfqpoint{3.789669in}{1.465995in}}%
\pgfpathcurveto{\pgfqpoint{3.784143in}{1.465995in}}{\pgfqpoint{3.778844in}{1.463800in}}{\pgfqpoint{3.774937in}{1.459893in}}%
\pgfpathcurveto{\pgfqpoint{3.771030in}{1.455986in}}{\pgfqpoint{3.768835in}{1.450687in}}{\pgfqpoint{3.768835in}{1.445161in}}%
\pgfpathcurveto{\pgfqpoint{3.768835in}{1.439636in}}{\pgfqpoint{3.771030in}{1.434337in}}{\pgfqpoint{3.774937in}{1.430430in}}%
\pgfpathcurveto{\pgfqpoint{3.778844in}{1.426523in}}{\pgfqpoint{3.784143in}{1.424328in}}{\pgfqpoint{3.789669in}{1.424328in}}%
\pgfpathclose%
\pgfusepath{fill}%
\end{pgfscope}%
\begin{pgfscope}%
\pgfpathrectangle{\pgfqpoint{3.158185in}{0.833185in}}{\pgfqpoint{1.162500in}{0.755000in}} %
\pgfusepath{clip}%
\pgfsetbuttcap%
\pgfsetroundjoin%
\definecolor{currentfill}{rgb}{0.000000,0.000000,0.000000}%
\pgfsetfillcolor{currentfill}%
\pgfsetfillopacity{0.500000}%
\pgfsetlinewidth{0.000000pt}%
\definecolor{currentstroke}{rgb}{0.000000,0.000000,0.000000}%
\pgfsetstrokecolor{currentstroke}%
\pgfsetdash{}{0pt}%
\pgfpathmoveto{\pgfqpoint{3.231064in}{1.186596in}}%
\pgfpathcurveto{\pgfqpoint{3.236589in}{1.186596in}}{\pgfqpoint{3.241888in}{1.188791in}}{\pgfqpoint{3.245795in}{1.192698in}}%
\pgfpathcurveto{\pgfqpoint{3.249702in}{1.196604in}}{\pgfqpoint{3.251897in}{1.201904in}}{\pgfqpoint{3.251897in}{1.207429in}}%
\pgfpathcurveto{\pgfqpoint{3.251897in}{1.212954in}}{\pgfqpoint{3.249702in}{1.218254in}}{\pgfqpoint{3.245795in}{1.222160in}}%
\pgfpathcurveto{\pgfqpoint{3.241888in}{1.226067in}}{\pgfqpoint{3.236589in}{1.228262in}}{\pgfqpoint{3.231064in}{1.228262in}}%
\pgfpathcurveto{\pgfqpoint{3.225539in}{1.228262in}}{\pgfqpoint{3.220239in}{1.226067in}}{\pgfqpoint{3.216332in}{1.222160in}}%
\pgfpathcurveto{\pgfqpoint{3.212425in}{1.218254in}}{\pgfqpoint{3.210230in}{1.212954in}}{\pgfqpoint{3.210230in}{1.207429in}}%
\pgfpathcurveto{\pgfqpoint{3.210230in}{1.201904in}}{\pgfqpoint{3.212425in}{1.196604in}}{\pgfqpoint{3.216332in}{1.192698in}}%
\pgfpathcurveto{\pgfqpoint{3.220239in}{1.188791in}}{\pgfqpoint{3.225539in}{1.186596in}}{\pgfqpoint{3.231064in}{1.186596in}}%
\pgfpathclose%
\pgfusepath{fill}%
\end{pgfscope}%
\begin{pgfscope}%
\pgfsetbuttcap%
\pgfsetroundjoin%
\definecolor{currentfill}{rgb}{0.000000,0.000000,0.000000}%
\pgfsetfillcolor{currentfill}%
\pgfsetlinewidth{0.803000pt}%
\definecolor{currentstroke}{rgb}{0.000000,0.000000,0.000000}%
\pgfsetstrokecolor{currentstroke}%
\pgfsetdash{}{0pt}%
\pgfsys@defobject{currentmarker}{\pgfqpoint{0.000000in}{-0.048611in}}{\pgfqpoint{0.000000in}{0.000000in}}{%
\pgfpathmoveto{\pgfqpoint{0.000000in}{0.000000in}}%
\pgfpathlineto{\pgfqpoint{0.000000in}{-0.048611in}}%
\pgfusepath{stroke,fill}%
}%
\begin{pgfscope}%
\pgfsys@transformshift{3.497383in}{0.833185in}%
\pgfsys@useobject{currentmarker}{}%
\end{pgfscope}%
\end{pgfscope}%
\begin{pgfscope}%
\pgftext[x=3.528037in,y=0.585111in,left,base,rotate=90.000000]{\rmfamily\fontsize{8.000000}{9.600000}\selectfont \(\displaystyle 2.5\)}%
\end{pgfscope}%
\begin{pgfscope}%
\pgfsetbuttcap%
\pgfsetroundjoin%
\definecolor{currentfill}{rgb}{0.000000,0.000000,0.000000}%
\pgfsetfillcolor{currentfill}%
\pgfsetlinewidth{0.803000pt}%
\definecolor{currentstroke}{rgb}{0.000000,0.000000,0.000000}%
\pgfsetstrokecolor{currentstroke}%
\pgfsetdash{}{0pt}%
\pgfsys@defobject{currentmarker}{\pgfqpoint{0.000000in}{-0.048611in}}{\pgfqpoint{0.000000in}{0.000000in}}{%
\pgfpathmoveto{\pgfqpoint{0.000000in}{0.000000in}}%
\pgfpathlineto{\pgfqpoint{0.000000in}{-0.048611in}}%
\pgfusepath{stroke,fill}%
}%
\begin{pgfscope}%
\pgfsys@transformshift{3.860784in}{0.833185in}%
\pgfsys@useobject{currentmarker}{}%
\end{pgfscope}%
\end{pgfscope}%
\begin{pgfscope}%
\pgftext[x=3.891437in,y=0.585111in,left,base,rotate=90.000000]{\rmfamily\fontsize{8.000000}{9.600000}\selectfont \(\displaystyle 5.0\)}%
\end{pgfscope}%
\begin{pgfscope}%
\pgfsetbuttcap%
\pgfsetroundjoin%
\definecolor{currentfill}{rgb}{0.000000,0.000000,0.000000}%
\pgfsetfillcolor{currentfill}%
\pgfsetlinewidth{0.803000pt}%
\definecolor{currentstroke}{rgb}{0.000000,0.000000,0.000000}%
\pgfsetstrokecolor{currentstroke}%
\pgfsetdash{}{0pt}%
\pgfsys@defobject{currentmarker}{\pgfqpoint{0.000000in}{-0.048611in}}{\pgfqpoint{0.000000in}{0.000000in}}{%
\pgfpathmoveto{\pgfqpoint{0.000000in}{0.000000in}}%
\pgfpathlineto{\pgfqpoint{0.000000in}{-0.048611in}}%
\pgfusepath{stroke,fill}%
}%
\begin{pgfscope}%
\pgfsys@transformshift{4.224184in}{0.833185in}%
\pgfsys@useobject{currentmarker}{}%
\end{pgfscope}%
\end{pgfscope}%
\begin{pgfscope}%
\pgftext[x=4.254838in,y=0.585111in,left,base,rotate=90.000000]{\rmfamily\fontsize{8.000000}{9.600000}\selectfont \(\displaystyle 7.5\)}%
\end{pgfscope}%
\begin{pgfscope}%
\pgftext[x=3.739435in,y=0.529556in,,top]{\rmfamily\fontsize{10.000000}{12.000000}\selectfont charge}%
\end{pgfscope}%
\begin{pgfscope}%
\pgftext[x=4.320685in,y=0.543445in,right,top]{\rmfamily\fontsize{10.000000}{12.000000}\selectfont \(\displaystyle \times10^{-10}\)}%
\end{pgfscope}%
\begin{pgfscope}%
\pgfsetrectcap%
\pgfsetmiterjoin%
\pgfsetlinewidth{0.803000pt}%
\definecolor{currentstroke}{rgb}{0.000000,0.000000,0.000000}%
\pgfsetstrokecolor{currentstroke}%
\pgfsetdash{}{0pt}%
\pgfpathmoveto{\pgfqpoint{3.158185in}{0.833185in}}%
\pgfpathlineto{\pgfqpoint{3.158185in}{1.588185in}}%
\pgfusepath{stroke}%
\end{pgfscope}%
\begin{pgfscope}%
\pgfsetrectcap%
\pgfsetmiterjoin%
\pgfsetlinewidth{0.803000pt}%
\definecolor{currentstroke}{rgb}{0.000000,0.000000,0.000000}%
\pgfsetstrokecolor{currentstroke}%
\pgfsetdash{}{0pt}%
\pgfpathmoveto{\pgfqpoint{4.320685in}{0.833185in}}%
\pgfpathlineto{\pgfqpoint{4.320685in}{1.588185in}}%
\pgfusepath{stroke}%
\end{pgfscope}%
\begin{pgfscope}%
\pgfsetrectcap%
\pgfsetmiterjoin%
\pgfsetlinewidth{0.803000pt}%
\definecolor{currentstroke}{rgb}{0.000000,0.000000,0.000000}%
\pgfsetstrokecolor{currentstroke}%
\pgfsetdash{}{0pt}%
\pgfpathmoveto{\pgfqpoint{3.158185in}{0.833185in}}%
\pgfpathlineto{\pgfqpoint{4.320685in}{0.833185in}}%
\pgfusepath{stroke}%
\end{pgfscope}%
\begin{pgfscope}%
\pgfsetrectcap%
\pgfsetmiterjoin%
\pgfsetlinewidth{0.803000pt}%
\definecolor{currentstroke}{rgb}{0.000000,0.000000,0.000000}%
\pgfsetstrokecolor{currentstroke}%
\pgfsetdash{}{0pt}%
\pgfpathmoveto{\pgfqpoint{3.158185in}{1.588185in}}%
\pgfpathlineto{\pgfqpoint{4.320685in}{1.588185in}}%
\pgfusepath{stroke}%
\end{pgfscope}%
\begin{pgfscope}%
\pgfsetbuttcap%
\pgfsetmiterjoin%
\definecolor{currentfill}{rgb}{1.000000,1.000000,1.000000}%
\pgfsetfillcolor{currentfill}%
\pgfsetlinewidth{0.000000pt}%
\definecolor{currentstroke}{rgb}{0.000000,0.000000,0.000000}%
\pgfsetstrokecolor{currentstroke}%
\pgfsetstrokeopacity{0.000000}%
\pgfsetdash{}{0pt}%
\pgfpathmoveto{\pgfqpoint{4.320685in}{0.833185in}}%
\pgfpathlineto{\pgfqpoint{5.483185in}{0.833185in}}%
\pgfpathlineto{\pgfqpoint{5.483185in}{1.588185in}}%
\pgfpathlineto{\pgfqpoint{4.320685in}{1.588185in}}%
\pgfpathclose%
\pgfusepath{fill}%
\end{pgfscope}%
\begin{pgfscope}%
\pgfsetbuttcap%
\pgfsetroundjoin%
\definecolor{currentfill}{rgb}{0.000000,0.000000,0.000000}%
\pgfsetfillcolor{currentfill}%
\pgfsetlinewidth{0.803000pt}%
\definecolor{currentstroke}{rgb}{0.000000,0.000000,0.000000}%
\pgfsetstrokecolor{currentstroke}%
\pgfsetdash{}{0pt}%
\pgfsys@defobject{currentmarker}{\pgfqpoint{0.000000in}{-0.048611in}}{\pgfqpoint{0.000000in}{0.000000in}}{%
\pgfpathmoveto{\pgfqpoint{0.000000in}{0.000000in}}%
\pgfpathlineto{\pgfqpoint{0.000000in}{-0.048611in}}%
\pgfusepath{stroke,fill}%
}%
\begin{pgfscope}%
\pgfsys@transformshift{4.447086in}{0.833185in}%
\pgfsys@useobject{currentmarker}{}%
\end{pgfscope}%
\end{pgfscope}%
\begin{pgfscope}%
\pgftext[x=4.477739in,y=0.467054in,left,base,rotate=90.000000]{\rmfamily\fontsize{8.000000}{9.600000}\selectfont \(\displaystyle 0.025\)}%
\end{pgfscope}%
\begin{pgfscope}%
\pgfsetbuttcap%
\pgfsetroundjoin%
\definecolor{currentfill}{rgb}{0.000000,0.000000,0.000000}%
\pgfsetfillcolor{currentfill}%
\pgfsetlinewidth{0.803000pt}%
\definecolor{currentstroke}{rgb}{0.000000,0.000000,0.000000}%
\pgfsetstrokecolor{currentstroke}%
\pgfsetdash{}{0pt}%
\pgfsys@defobject{currentmarker}{\pgfqpoint{0.000000in}{-0.048611in}}{\pgfqpoint{0.000000in}{0.000000in}}{%
\pgfpathmoveto{\pgfqpoint{0.000000in}{0.000000in}}%
\pgfpathlineto{\pgfqpoint{0.000000in}{-0.048611in}}%
\pgfusepath{stroke,fill}%
}%
\begin{pgfscope}%
\pgfsys@transformshift{4.854077in}{0.833185in}%
\pgfsys@useobject{currentmarker}{}%
\end{pgfscope}%
\end{pgfscope}%
\begin{pgfscope}%
\pgftext[x=4.884731in,y=0.467054in,left,base,rotate=90.000000]{\rmfamily\fontsize{8.000000}{9.600000}\selectfont \(\displaystyle 0.050\)}%
\end{pgfscope}%
\begin{pgfscope}%
\pgfsetbuttcap%
\pgfsetroundjoin%
\definecolor{currentfill}{rgb}{0.000000,0.000000,0.000000}%
\pgfsetfillcolor{currentfill}%
\pgfsetlinewidth{0.803000pt}%
\definecolor{currentstroke}{rgb}{0.000000,0.000000,0.000000}%
\pgfsetstrokecolor{currentstroke}%
\pgfsetdash{}{0pt}%
\pgfsys@defobject{currentmarker}{\pgfqpoint{0.000000in}{-0.048611in}}{\pgfqpoint{0.000000in}{0.000000in}}{%
\pgfpathmoveto{\pgfqpoint{0.000000in}{0.000000in}}%
\pgfpathlineto{\pgfqpoint{0.000000in}{-0.048611in}}%
\pgfusepath{stroke,fill}%
}%
\begin{pgfscope}%
\pgfsys@transformshift{5.261069in}{0.833185in}%
\pgfsys@useobject{currentmarker}{}%
\end{pgfscope}%
\end{pgfscope}%
\begin{pgfscope}%
\pgftext[x=5.291722in,y=0.467054in,left,base,rotate=90.000000]{\rmfamily\fontsize{8.000000}{9.600000}\selectfont \(\displaystyle 0.075\)}%
\end{pgfscope}%
\begin{pgfscope}%
\pgftext[x=4.901935in,y=0.411499in,,top]{\rmfamily\fontsize{10.000000}{12.000000}\selectfont u0}%
\end{pgfscope}%
\begin{pgfscope}%
\pgfpathrectangle{\pgfqpoint{4.320685in}{0.833185in}}{\pgfqpoint{1.162500in}{0.755000in}} %
\pgfusepath{clip}%
\pgfsetrectcap%
\pgfsetroundjoin%
\pgfsetlinewidth{1.505625pt}%
\definecolor{currentstroke}{rgb}{0.121569,0.466667,0.705882}%
\pgfsetstrokecolor{currentstroke}%
\pgfsetdash{}{0pt}%
\pgfpathmoveto{\pgfqpoint{4.348363in}{0.867503in}}%
\pgfpathlineto{\pgfqpoint{4.384935in}{0.889698in}}%
\pgfpathlineto{\pgfqpoint{4.438131in}{0.918946in}}%
\pgfpathlineto{\pgfqpoint{4.481353in}{0.943686in}}%
\pgfpathlineto{\pgfqpoint{4.509060in}{0.962052in}}%
\pgfpathlineto{\pgfqpoint{4.533441in}{0.980861in}}%
\pgfpathlineto{\pgfqpoint{4.556714in}{1.001727in}}%
\pgfpathlineto{\pgfqpoint{4.578879in}{1.024610in}}%
\pgfpathlineto{\pgfqpoint{4.602153in}{1.051996in}}%
\pgfpathlineto{\pgfqpoint{4.626534in}{1.084371in}}%
\pgfpathlineto{\pgfqpoint{4.653132in}{1.123666in}}%
\pgfpathlineto{\pgfqpoint{4.684163in}{1.173821in}}%
\pgfpathlineto{\pgfqpoint{4.728493in}{1.250270in}}%
\pgfpathlineto{\pgfqpoint{4.786122in}{1.348907in}}%
\pgfpathlineto{\pgfqpoint{4.816045in}{1.395699in}}%
\pgfpathlineto{\pgfqpoint{4.841535in}{1.431586in}}%
\pgfpathlineto{\pgfqpoint{4.864808in}{1.460534in}}%
\pgfpathlineto{\pgfqpoint{4.885865in}{1.483292in}}%
\pgfpathlineto{\pgfqpoint{4.905813in}{1.501731in}}%
\pgfpathlineto{\pgfqpoint{4.925762in}{1.517130in}}%
\pgfpathlineto{\pgfqpoint{4.944602in}{1.528953in}}%
\pgfpathlineto{\pgfqpoint{4.963443in}{1.538257in}}%
\pgfpathlineto{\pgfqpoint{4.982283in}{1.545206in}}%
\pgfpathlineto{\pgfqpoint{5.002231in}{1.550207in}}%
\pgfpathlineto{\pgfqpoint{5.023288in}{1.553112in}}%
\pgfpathlineto{\pgfqpoint{5.045453in}{1.553833in}}%
\pgfpathlineto{\pgfqpoint{5.068726in}{1.552312in}}%
\pgfpathlineto{\pgfqpoint{5.093108in}{1.548490in}}%
\pgfpathlineto{\pgfqpoint{5.118598in}{1.542249in}}%
\pgfpathlineto{\pgfqpoint{5.144087in}{1.533767in}}%
\pgfpathlineto{\pgfqpoint{5.169577in}{1.522946in}}%
\pgfpathlineto{\pgfqpoint{5.193959in}{1.510184in}}%
\pgfpathlineto{\pgfqpoint{5.217232in}{1.495522in}}%
\pgfpathlineto{\pgfqpoint{5.239397in}{1.479020in}}%
\pgfpathlineto{\pgfqpoint{5.261562in}{1.459760in}}%
\pgfpathlineto{\pgfqpoint{5.283727in}{1.437493in}}%
\pgfpathlineto{\pgfqpoint{5.305892in}{1.412020in}}%
\pgfpathlineto{\pgfqpoint{5.329165in}{1.381694in}}%
\pgfpathlineto{\pgfqpoint{5.352439in}{1.347687in}}%
\pgfpathlineto{\pgfqpoint{5.377928in}{1.306375in}}%
\pgfpathlineto{\pgfqpoint{5.405635in}{1.257087in}}%
\pgfpathlineto{\pgfqpoint{5.437774in}{1.195202in}}%
\pgfpathlineto{\pgfqpoint{5.455506in}{1.159442in}}%
\pgfpathlineto{\pgfqpoint{5.455506in}{1.159442in}}%
\pgfusepath{stroke}%
\end{pgfscope}%
\begin{pgfscope}%
\pgfsetrectcap%
\pgfsetmiterjoin%
\pgfsetlinewidth{0.803000pt}%
\definecolor{currentstroke}{rgb}{0.000000,0.000000,0.000000}%
\pgfsetstrokecolor{currentstroke}%
\pgfsetdash{}{0pt}%
\pgfpathmoveto{\pgfqpoint{4.320685in}{0.833185in}}%
\pgfpathlineto{\pgfqpoint{4.320685in}{1.588185in}}%
\pgfusepath{stroke}%
\end{pgfscope}%
\begin{pgfscope}%
\pgfsetrectcap%
\pgfsetmiterjoin%
\pgfsetlinewidth{0.803000pt}%
\definecolor{currentstroke}{rgb}{0.000000,0.000000,0.000000}%
\pgfsetstrokecolor{currentstroke}%
\pgfsetdash{}{0pt}%
\pgfpathmoveto{\pgfqpoint{5.483185in}{0.833185in}}%
\pgfpathlineto{\pgfqpoint{5.483185in}{1.588185in}}%
\pgfusepath{stroke}%
\end{pgfscope}%
\begin{pgfscope}%
\pgfsetrectcap%
\pgfsetmiterjoin%
\pgfsetlinewidth{0.803000pt}%
\definecolor{currentstroke}{rgb}{0.000000,0.000000,0.000000}%
\pgfsetstrokecolor{currentstroke}%
\pgfsetdash{}{0pt}%
\pgfpathmoveto{\pgfqpoint{4.320685in}{0.833185in}}%
\pgfpathlineto{\pgfqpoint{5.483185in}{0.833185in}}%
\pgfusepath{stroke}%
\end{pgfscope}%
\begin{pgfscope}%
\pgfsetrectcap%
\pgfsetmiterjoin%
\pgfsetlinewidth{0.803000pt}%
\definecolor{currentstroke}{rgb}{0.000000,0.000000,0.000000}%
\pgfsetstrokecolor{currentstroke}%
\pgfsetdash{}{0pt}%
\pgfpathmoveto{\pgfqpoint{4.320685in}{1.588185in}}%
\pgfpathlineto{\pgfqpoint{5.483185in}{1.588185in}}%
\pgfusepath{stroke}%
\end{pgfscope}%
\end{pgfpicture}%
\makeatother%
\endgroup%
}
    \caption{Covariance plot of main parameters $E_0$, $U_0$, droplet area ($V_d^{2/3})$, and $q$.\label{fig:scatter}}
\end{figure}
\begin{figure}[H]
    \centering
    %% Creator: Matplotlib, PGF backend
%%
%% To include the figure in your LaTeX document, write
%%   \input{<filename>.pgf}
%%
%% Make sure the required packages are loaded in your preamble
%%   \usepackage{pgf}
%%
%% Figures using additional raster images can only be included by \input if
%% they are in the same directory as the main LaTeX file. For loading figures
%% from other directories you can use the `import` package
%%   \usepackage{import}
%% and then include the figures with
%%   \import{<path to file>}{<filename>.pgf}
%%
%% Matplotlib used the following preamble
%%   \usepackage{fontspec}
%%   \setmainfont{DejaVu Serif}
%%   \setsansfont{DejaVu Sans}
%%   \setmonofont{DejaVu Sans Mono}
%%
\begingroup%
\makeatletter%
\begin{pgfpicture}%
\pgfpathrectangle{\pgfpointorigin}{\pgfqpoint{5.194240in}{3.788793in}}%
\pgfusepath{use as bounding box, clip}%
\begin{pgfscope}%
\pgfsetbuttcap%
\pgfsetmiterjoin%
\definecolor{currentfill}{rgb}{1.000000,1.000000,1.000000}%
\pgfsetfillcolor{currentfill}%
\pgfsetlinewidth{0.000000pt}%
\definecolor{currentstroke}{rgb}{1.000000,1.000000,1.000000}%
\pgfsetstrokecolor{currentstroke}%
\pgfsetdash{}{0pt}%
\pgfpathmoveto{\pgfqpoint{0.000000in}{0.000000in}}%
\pgfpathlineto{\pgfqpoint{5.194240in}{0.000000in}}%
\pgfpathlineto{\pgfqpoint{5.194240in}{3.788793in}}%
\pgfpathlineto{\pgfqpoint{0.000000in}{3.788793in}}%
\pgfpathclose%
\pgfusepath{fill}%
\end{pgfscope}%
\begin{pgfscope}%
\pgfsetbuttcap%
\pgfsetmiterjoin%
\definecolor{currentfill}{rgb}{1.000000,1.000000,1.000000}%
\pgfsetfillcolor{currentfill}%
\pgfsetlinewidth{0.000000pt}%
\definecolor{currentstroke}{rgb}{0.000000,0.000000,0.000000}%
\pgfsetstrokecolor{currentstroke}%
\pgfsetstrokeopacity{0.000000}%
\pgfsetdash{}{0pt}%
\pgfpathmoveto{\pgfqpoint{0.634105in}{0.521603in}}%
\pgfpathlineto{\pgfqpoint{4.354105in}{0.521603in}}%
\pgfpathlineto{\pgfqpoint{4.354105in}{3.541603in}}%
\pgfpathlineto{\pgfqpoint{0.634105in}{3.541603in}}%
\pgfpathclose%
\pgfusepath{fill}%
\end{pgfscope}%
\begin{pgfscope}%
\pgfpathrectangle{\pgfqpoint{0.634105in}{0.521603in}}{\pgfqpoint{3.720000in}{3.020000in}} %
\pgfusepath{clip}%
\pgfsetbuttcap%
\pgfsetroundjoin%
\definecolor{currentfill}{rgb}{0.061765,0.061765,0.085934}%
\pgfsetfillcolor{currentfill}%
\pgfsetlinewidth{0.000000pt}%
\definecolor{currentstroke}{rgb}{0.000000,0.000000,0.000000}%
\pgfsetstrokecolor{currentstroke}%
\pgfsetdash{}{0pt}%
\pgfpathmoveto{\pgfqpoint{1.369836in}{0.740556in}}%
\pgfpathlineto{\pgfqpoint{1.437974in}{0.740556in}}%
\pgfpathlineto{\pgfqpoint{1.506111in}{0.740556in}}%
\pgfpathlineto{\pgfqpoint{1.574249in}{0.740556in}}%
\pgfpathlineto{\pgfqpoint{1.642386in}{0.740556in}}%
\pgfpathlineto{\pgfqpoint{1.710524in}{0.740556in}}%
\pgfpathlineto{\pgfqpoint{1.778661in}{0.740556in}}%
\pgfpathlineto{\pgfqpoint{1.846799in}{0.795494in}}%
\pgfpathlineto{\pgfqpoint{1.914936in}{0.795494in}}%
\pgfpathlineto{\pgfqpoint{1.983074in}{0.795494in}}%
\pgfpathlineto{\pgfqpoint{2.051211in}{0.795494in}}%
\pgfpathlineto{\pgfqpoint{2.119349in}{0.795494in}}%
\pgfpathlineto{\pgfqpoint{2.187486in}{0.795494in}}%
\pgfpathlineto{\pgfqpoint{2.255624in}{0.850432in}}%
\pgfpathlineto{\pgfqpoint{2.323761in}{0.905370in}}%
\pgfpathlineto{\pgfqpoint{2.337491in}{0.916441in}}%
\pgfpathlineto{\pgfqpoint{2.323761in}{0.911739in}}%
\pgfpathlineto{\pgfqpoint{2.269121in}{0.905370in}}%
\pgfpathlineto{\pgfqpoint{2.255624in}{0.903185in}}%
\pgfpathlineto{\pgfqpoint{2.247795in}{0.905370in}}%
\pgfpathlineto{\pgfqpoint{2.187486in}{0.927654in}}%
\pgfpathlineto{\pgfqpoint{2.119349in}{0.940331in}}%
\pgfpathlineto{\pgfqpoint{2.051211in}{0.952591in}}%
\pgfpathlineto{\pgfqpoint{2.005953in}{0.960308in}}%
\pgfpathlineto{\pgfqpoint{1.983074in}{0.964247in}}%
\pgfpathlineto{\pgfqpoint{1.914936in}{0.972393in}}%
\pgfpathlineto{\pgfqpoint{1.846799in}{0.980237in}}%
\pgfpathlineto{\pgfqpoint{1.778661in}{0.988081in}}%
\pgfpathlineto{\pgfqpoint{1.710524in}{0.995925in}}%
\pgfpathlineto{\pgfqpoint{1.642386in}{1.000048in}}%
\pgfpathlineto{\pgfqpoint{1.574249in}{0.991101in}}%
\pgfpathlineto{\pgfqpoint{1.506111in}{0.980260in}}%
\pgfpathlineto{\pgfqpoint{1.437974in}{0.966635in}}%
\pgfpathlineto{\pgfqpoint{1.428669in}{0.960308in}}%
\pgfpathlineto{\pgfqpoint{1.369836in}{0.923458in}}%
\pgfpathlineto{\pgfqpoint{1.301699in}{0.933018in}}%
\pgfpathlineto{\pgfqpoint{1.233561in}{0.942578in}}%
\pgfpathlineto{\pgfqpoint{1.165424in}{0.952138in}}%
\pgfpathlineto{\pgfqpoint{1.107195in}{0.960308in}}%
\pgfpathlineto{\pgfqpoint{1.097286in}{0.961699in}}%
\pgfpathlineto{\pgfqpoint{1.095198in}{0.961992in}}%
\pgfpathlineto{\pgfqpoint{1.097286in}{0.960308in}}%
\pgfpathlineto{\pgfqpoint{1.165424in}{0.905370in}}%
\pgfpathlineto{\pgfqpoint{1.233561in}{0.850432in}}%
\pgfpathlineto{\pgfqpoint{1.301699in}{0.795494in}}%
\pgfpathclose%
\pgfusepath{fill}%
\end{pgfscope}%
\begin{pgfscope}%
\pgfpathrectangle{\pgfqpoint{0.634105in}{0.521603in}}{\pgfqpoint{3.720000in}{3.020000in}} %
\pgfusepath{clip}%
\pgfsetbuttcap%
\pgfsetroundjoin%
\definecolor{currentfill}{rgb}{0.185294,0.185294,0.257801}%
\pgfsetfillcolor{currentfill}%
\pgfsetlinewidth{0.000000pt}%
\definecolor{currentstroke}{rgb}{0.000000,0.000000,0.000000}%
\pgfsetstrokecolor{currentstroke}%
\pgfsetdash{}{0pt}%
\pgfpathmoveto{\pgfqpoint{2.255624in}{0.903185in}}%
\pgfpathlineto{\pgfqpoint{2.269121in}{0.905370in}}%
\pgfpathlineto{\pgfqpoint{2.323761in}{0.911739in}}%
\pgfpathlineto{\pgfqpoint{2.337491in}{0.916441in}}%
\pgfpathlineto{\pgfqpoint{2.391899in}{0.960308in}}%
\pgfpathlineto{\pgfqpoint{2.460036in}{1.015247in}}%
\pgfpathlineto{\pgfqpoint{2.528174in}{1.070185in}}%
\pgfpathlineto{\pgfqpoint{2.596311in}{1.125123in}}%
\pgfpathlineto{\pgfqpoint{2.664449in}{1.180061in}}%
\pgfpathlineto{\pgfqpoint{2.732586in}{1.234999in}}%
\pgfpathlineto{\pgfqpoint{2.800724in}{1.289938in}}%
\pgfpathlineto{\pgfqpoint{2.828867in}{1.312630in}}%
\pgfpathlineto{\pgfqpoint{2.800724in}{1.297732in}}%
\pgfpathlineto{\pgfqpoint{2.785999in}{1.289938in}}%
\pgfpathlineto{\pgfqpoint{2.732586in}{1.261664in}}%
\pgfpathlineto{\pgfqpoint{2.682214in}{1.234999in}}%
\pgfpathlineto{\pgfqpoint{2.664449in}{1.225595in}}%
\pgfpathlineto{\pgfqpoint{2.596311in}{1.197344in}}%
\pgfpathlineto{\pgfqpoint{2.528174in}{1.206434in}}%
\pgfpathlineto{\pgfqpoint{2.484547in}{1.234999in}}%
\pgfpathlineto{\pgfqpoint{2.460036in}{1.247206in}}%
\pgfpathlineto{\pgfqpoint{2.391899in}{1.269643in}}%
\pgfpathlineto{\pgfqpoint{2.323761in}{1.288767in}}%
\pgfpathlineto{\pgfqpoint{2.319592in}{1.289938in}}%
\pgfpathlineto{\pgfqpoint{2.255624in}{1.307892in}}%
\pgfpathlineto{\pgfqpoint{2.187486in}{1.327016in}}%
\pgfpathlineto{\pgfqpoint{2.123852in}{1.344876in}}%
\pgfpathlineto{\pgfqpoint{2.119349in}{1.346235in}}%
\pgfpathlineto{\pgfqpoint{2.051211in}{1.367605in}}%
\pgfpathlineto{\pgfqpoint{1.983074in}{1.386557in}}%
\pgfpathlineto{\pgfqpoint{1.917821in}{1.399814in}}%
\pgfpathlineto{\pgfqpoint{1.914936in}{1.400400in}}%
\pgfpathlineto{\pgfqpoint{1.846799in}{1.414243in}}%
\pgfpathlineto{\pgfqpoint{1.778661in}{1.428085in}}%
\pgfpathlineto{\pgfqpoint{1.710524in}{1.441928in}}%
\pgfpathlineto{\pgfqpoint{1.666521in}{1.399814in}}%
\pgfpathlineto{\pgfqpoint{1.642386in}{1.374990in}}%
\pgfpathlineto{\pgfqpoint{1.574249in}{1.357931in}}%
\pgfpathlineto{\pgfqpoint{1.506111in}{1.347090in}}%
\pgfpathlineto{\pgfqpoint{1.492193in}{1.344876in}}%
\pgfpathlineto{\pgfqpoint{1.437974in}{1.336250in}}%
\pgfpathlineto{\pgfqpoint{1.369836in}{1.325409in}}%
\pgfpathlineto{\pgfqpoint{1.301699in}{1.291509in}}%
\pgfpathlineto{\pgfqpoint{1.233561in}{1.301069in}}%
\pgfpathlineto{\pgfqpoint{1.165424in}{1.310629in}}%
\pgfpathlineto{\pgfqpoint{1.097286in}{1.320189in}}%
\pgfpathlineto{\pgfqpoint{1.080739in}{1.344876in}}%
\pgfpathlineto{\pgfqpoint{1.053387in}{1.399814in}}%
\pgfpathlineto{\pgfqpoint{1.034280in}{1.454752in}}%
\pgfpathlineto{\pgfqpoint{1.029149in}{1.472995in}}%
\pgfpathlineto{\pgfqpoint{0.999191in}{1.509690in}}%
\pgfpathlineto{\pgfqpoint{0.961011in}{1.556918in}}%
\pgfpathlineto{\pgfqpoint{0.961011in}{1.509690in}}%
\pgfpathlineto{\pgfqpoint{0.961011in}{1.454752in}}%
\pgfpathlineto{\pgfqpoint{0.961011in}{1.399814in}}%
\pgfpathlineto{\pgfqpoint{1.029149in}{1.344876in}}%
\pgfpathlineto{\pgfqpoint{1.029149in}{1.289938in}}%
\pgfpathlineto{\pgfqpoint{1.029149in}{1.234999in}}%
\pgfpathlineto{\pgfqpoint{1.029149in}{1.180061in}}%
\pgfpathlineto{\pgfqpoint{1.029149in}{1.125123in}}%
\pgfpathlineto{\pgfqpoint{1.029149in}{1.070185in}}%
\pgfpathlineto{\pgfqpoint{1.029149in}{1.015247in}}%
\pgfpathlineto{\pgfqpoint{1.095198in}{0.961992in}}%
\pgfpathlineto{\pgfqpoint{1.097286in}{0.961699in}}%
\pgfpathlineto{\pgfqpoint{1.107195in}{0.960308in}}%
\pgfpathlineto{\pgfqpoint{1.165424in}{0.952138in}}%
\pgfpathlineto{\pgfqpoint{1.233561in}{0.942578in}}%
\pgfpathlineto{\pgfqpoint{1.301699in}{0.933018in}}%
\pgfpathlineto{\pgfqpoint{1.369836in}{0.923458in}}%
\pgfpathlineto{\pgfqpoint{1.428669in}{0.960308in}}%
\pgfpathlineto{\pgfqpoint{1.437974in}{0.966635in}}%
\pgfpathlineto{\pgfqpoint{1.506111in}{0.980260in}}%
\pgfpathlineto{\pgfqpoint{1.574249in}{0.991101in}}%
\pgfpathlineto{\pgfqpoint{1.642386in}{1.000048in}}%
\pgfpathlineto{\pgfqpoint{1.710524in}{0.995925in}}%
\pgfpathlineto{\pgfqpoint{1.778661in}{0.988081in}}%
\pgfpathlineto{\pgfqpoint{1.846799in}{0.980237in}}%
\pgfpathlineto{\pgfqpoint{1.914936in}{0.972393in}}%
\pgfpathlineto{\pgfqpoint{1.983074in}{0.964247in}}%
\pgfpathlineto{\pgfqpoint{2.005953in}{0.960308in}}%
\pgfpathlineto{\pgfqpoint{2.051211in}{0.952591in}}%
\pgfpathlineto{\pgfqpoint{2.119349in}{0.940331in}}%
\pgfpathlineto{\pgfqpoint{2.187486in}{0.927654in}}%
\pgfpathlineto{\pgfqpoint{2.247795in}{0.905370in}}%
\pgfpathclose%
\pgfusepath{fill}%
\end{pgfscope}%
\begin{pgfscope}%
\pgfpathrectangle{\pgfqpoint{0.634105in}{0.521603in}}{\pgfqpoint{3.720000in}{3.020000in}} %
\pgfusepath{clip}%
\pgfsetbuttcap%
\pgfsetroundjoin%
\definecolor{currentfill}{rgb}{0.312255,0.312255,0.434442}%
\pgfsetfillcolor{currentfill}%
\pgfsetlinewidth{0.000000pt}%
\definecolor{currentstroke}{rgb}{0.000000,0.000000,0.000000}%
\pgfsetstrokecolor{currentstroke}%
\pgfsetdash{}{0pt}%
\pgfpathmoveto{\pgfqpoint{2.528174in}{1.206434in}}%
\pgfpathlineto{\pgfqpoint{2.596311in}{1.197344in}}%
\pgfpathlineto{\pgfqpoint{2.664449in}{1.225595in}}%
\pgfpathlineto{\pgfqpoint{2.682214in}{1.234999in}}%
\pgfpathlineto{\pgfqpoint{2.732586in}{1.261664in}}%
\pgfpathlineto{\pgfqpoint{2.785999in}{1.289938in}}%
\pgfpathlineto{\pgfqpoint{2.800724in}{1.297732in}}%
\pgfpathlineto{\pgfqpoint{2.828867in}{1.312630in}}%
\pgfpathlineto{\pgfqpoint{2.868861in}{1.344876in}}%
\pgfpathlineto{\pgfqpoint{2.936999in}{1.399814in}}%
\pgfpathlineto{\pgfqpoint{3.005136in}{1.454752in}}%
\pgfpathlineto{\pgfqpoint{3.073274in}{1.509690in}}%
\pgfpathlineto{\pgfqpoint{3.141411in}{1.564629in}}%
\pgfpathlineto{\pgfqpoint{3.209549in}{1.619567in}}%
\pgfpathlineto{\pgfqpoint{3.277686in}{1.674505in}}%
\pgfpathlineto{\pgfqpoint{3.320244in}{1.708818in}}%
\pgfpathlineto{\pgfqpoint{3.277686in}{1.686784in}}%
\pgfpathlineto{\pgfqpoint{3.255421in}{1.674505in}}%
\pgfpathlineto{\pgfqpoint{3.209549in}{1.650223in}}%
\pgfpathlineto{\pgfqpoint{3.151636in}{1.619567in}}%
\pgfpathlineto{\pgfqpoint{3.141411in}{1.614155in}}%
\pgfpathlineto{\pgfqpoint{3.073274in}{1.579343in}}%
\pgfpathlineto{\pgfqpoint{3.047850in}{1.564629in}}%
\pgfpathlineto{\pgfqpoint{3.005136in}{1.542018in}}%
\pgfpathlineto{\pgfqpoint{2.944065in}{1.509690in}}%
\pgfpathlineto{\pgfqpoint{2.936999in}{1.505950in}}%
\pgfpathlineto{\pgfqpoint{2.868861in}{1.483166in}}%
\pgfpathlineto{\pgfqpoint{2.800724in}{1.494202in}}%
\pgfpathlineto{\pgfqpoint{2.750992in}{1.509690in}}%
\pgfpathlineto{\pgfqpoint{2.732586in}{1.516233in}}%
\pgfpathlineto{\pgfqpoint{2.664449in}{1.540451in}}%
\pgfpathlineto{\pgfqpoint{2.596426in}{1.564629in}}%
\pgfpathlineto{\pgfqpoint{2.596311in}{1.564669in}}%
\pgfpathlineto{\pgfqpoint{2.528174in}{1.588888in}}%
\pgfpathlineto{\pgfqpoint{2.460036in}{1.613106in}}%
\pgfpathlineto{\pgfqpoint{2.441859in}{1.619567in}}%
\pgfpathlineto{\pgfqpoint{2.391899in}{1.637324in}}%
\pgfpathlineto{\pgfqpoint{2.323761in}{1.661543in}}%
\pgfpathlineto{\pgfqpoint{2.287292in}{1.674505in}}%
\pgfpathlineto{\pgfqpoint{2.255624in}{1.685761in}}%
\pgfpathlineto{\pgfqpoint{2.187486in}{1.709979in}}%
\pgfpathlineto{\pgfqpoint{2.132725in}{1.729443in}}%
\pgfpathlineto{\pgfqpoint{2.119349in}{1.734198in}}%
\pgfpathlineto{\pgfqpoint{2.051211in}{1.758416in}}%
\pgfpathlineto{\pgfqpoint{1.983074in}{1.782635in}}%
\pgfpathlineto{\pgfqpoint{1.978159in}{1.784381in}}%
\pgfpathlineto{\pgfqpoint{1.914936in}{1.806853in}}%
\pgfpathlineto{\pgfqpoint{1.846799in}{1.831071in}}%
\pgfpathlineto{\pgfqpoint{1.823592in}{1.839320in}}%
\pgfpathlineto{\pgfqpoint{1.778661in}{1.855290in}}%
\pgfpathlineto{\pgfqpoint{1.710524in}{1.872116in}}%
\pgfpathlineto{\pgfqpoint{1.642386in}{1.885959in}}%
\pgfpathlineto{\pgfqpoint{1.601535in}{1.894258in}}%
\pgfpathlineto{\pgfqpoint{1.574249in}{1.899801in}}%
\pgfpathlineto{\pgfqpoint{1.561047in}{1.894258in}}%
\pgfpathlineto{\pgfqpoint{1.506111in}{1.848486in}}%
\pgfpathlineto{\pgfqpoint{1.502632in}{1.839320in}}%
\pgfpathlineto{\pgfqpoint{1.481783in}{1.784381in}}%
\pgfpathlineto{\pgfqpoint{1.460934in}{1.729443in}}%
\pgfpathlineto{\pgfqpoint{1.437974in}{1.703270in}}%
\pgfpathlineto{\pgfqpoint{1.369836in}{1.692239in}}%
\pgfpathlineto{\pgfqpoint{1.301699in}{1.681399in}}%
\pgfpathlineto{\pgfqpoint{1.280527in}{1.674505in}}%
\pgfpathlineto{\pgfqpoint{1.233561in}{1.659560in}}%
\pgfpathlineto{\pgfqpoint{1.165424in}{1.669120in}}%
\pgfpathlineto{\pgfqpoint{1.127043in}{1.674505in}}%
\pgfpathlineto{\pgfqpoint{1.097286in}{1.679372in}}%
\pgfpathlineto{\pgfqpoint{1.085455in}{1.729443in}}%
\pgfpathlineto{\pgfqpoint{1.076365in}{1.784381in}}%
\pgfpathlineto{\pgfqpoint{1.067275in}{1.839320in}}%
\pgfpathlineto{\pgfqpoint{1.058184in}{1.894258in}}%
\pgfpathlineto{\pgfqpoint{1.049094in}{1.949196in}}%
\pgfpathlineto{\pgfqpoint{1.040004in}{2.004134in}}%
\pgfpathlineto{\pgfqpoint{1.030914in}{2.059072in}}%
\pgfpathlineto{\pgfqpoint{1.029149in}{2.069741in}}%
\pgfpathlineto{\pgfqpoint{0.995383in}{2.114011in}}%
\pgfpathlineto{\pgfqpoint{0.961011in}{2.159517in}}%
\pgfpathlineto{\pgfqpoint{0.904418in}{2.168949in}}%
\pgfpathlineto{\pgfqpoint{0.892874in}{2.170759in}}%
\pgfpathlineto{\pgfqpoint{0.892874in}{2.168949in}}%
\pgfpathlineto{\pgfqpoint{0.892874in}{2.114011in}}%
\pgfpathlineto{\pgfqpoint{0.892874in}{2.059072in}}%
\pgfpathlineto{\pgfqpoint{0.892874in}{2.004134in}}%
\pgfpathlineto{\pgfqpoint{0.892874in}{1.949196in}}%
\pgfpathlineto{\pgfqpoint{0.892874in}{1.894258in}}%
\pgfpathlineto{\pgfqpoint{0.892874in}{1.839320in}}%
\pgfpathlineto{\pgfqpoint{0.961011in}{1.784381in}}%
\pgfpathlineto{\pgfqpoint{0.961011in}{1.729443in}}%
\pgfpathlineto{\pgfqpoint{0.961011in}{1.674505in}}%
\pgfpathlineto{\pgfqpoint{0.961011in}{1.619567in}}%
\pgfpathlineto{\pgfqpoint{0.961011in}{1.564629in}}%
\pgfpathlineto{\pgfqpoint{0.961011in}{1.556918in}}%
\pgfpathlineto{\pgfqpoint{0.999191in}{1.509690in}}%
\pgfpathlineto{\pgfqpoint{1.029149in}{1.472995in}}%
\pgfpathlineto{\pgfqpoint{1.034280in}{1.454752in}}%
\pgfpathlineto{\pgfqpoint{1.053387in}{1.399814in}}%
\pgfpathlineto{\pgfqpoint{1.080739in}{1.344876in}}%
\pgfpathlineto{\pgfqpoint{1.097286in}{1.320189in}}%
\pgfpathlineto{\pgfqpoint{1.165424in}{1.310629in}}%
\pgfpathlineto{\pgfqpoint{1.233561in}{1.301069in}}%
\pgfpathlineto{\pgfqpoint{1.301699in}{1.291509in}}%
\pgfpathlineto{\pgfqpoint{1.369836in}{1.325409in}}%
\pgfpathlineto{\pgfqpoint{1.437974in}{1.336250in}}%
\pgfpathlineto{\pgfqpoint{1.492193in}{1.344876in}}%
\pgfpathlineto{\pgfqpoint{1.506111in}{1.347090in}}%
\pgfpathlineto{\pgfqpoint{1.574249in}{1.357931in}}%
\pgfpathlineto{\pgfqpoint{1.642386in}{1.374990in}}%
\pgfpathlineto{\pgfqpoint{1.666521in}{1.399814in}}%
\pgfpathlineto{\pgfqpoint{1.710524in}{1.441928in}}%
\pgfpathlineto{\pgfqpoint{1.778661in}{1.428085in}}%
\pgfpathlineto{\pgfqpoint{1.846799in}{1.414243in}}%
\pgfpathlineto{\pgfqpoint{1.914936in}{1.400400in}}%
\pgfpathlineto{\pgfqpoint{1.917821in}{1.399814in}}%
\pgfpathlineto{\pgfqpoint{1.983074in}{1.386557in}}%
\pgfpathlineto{\pgfqpoint{2.051211in}{1.367605in}}%
\pgfpathlineto{\pgfqpoint{2.119349in}{1.346235in}}%
\pgfpathlineto{\pgfqpoint{2.123852in}{1.344876in}}%
\pgfpathlineto{\pgfqpoint{2.187486in}{1.327016in}}%
\pgfpathlineto{\pgfqpoint{2.255624in}{1.307892in}}%
\pgfpathlineto{\pgfqpoint{2.319592in}{1.289938in}}%
\pgfpathlineto{\pgfqpoint{2.323761in}{1.288767in}}%
\pgfpathlineto{\pgfqpoint{2.391899in}{1.269643in}}%
\pgfpathlineto{\pgfqpoint{2.460036in}{1.247206in}}%
\pgfpathlineto{\pgfqpoint{2.484547in}{1.234999in}}%
\pgfpathclose%
\pgfusepath{fill}%
\end{pgfscope}%
\begin{pgfscope}%
\pgfpathrectangle{\pgfqpoint{0.634105in}{0.521603in}}{\pgfqpoint{3.720000in}{3.020000in}} %
\pgfusepath{clip}%
\pgfsetbuttcap%
\pgfsetroundjoin%
\definecolor{currentfill}{rgb}{0.439216,0.484130,0.564216}%
\pgfsetfillcolor{currentfill}%
\pgfsetlinewidth{0.000000pt}%
\definecolor{currentstroke}{rgb}{0.000000,0.000000,0.000000}%
\pgfsetstrokecolor{currentstroke}%
\pgfsetdash{}{0pt}%
\pgfpathmoveto{\pgfqpoint{2.800724in}{1.494202in}}%
\pgfpathlineto{\pgfqpoint{2.868861in}{1.483166in}}%
\pgfpathlineto{\pgfqpoint{2.936999in}{1.505950in}}%
\pgfpathlineto{\pgfqpoint{2.944065in}{1.509690in}}%
\pgfpathlineto{\pgfqpoint{3.005136in}{1.542018in}}%
\pgfpathlineto{\pgfqpoint{3.047850in}{1.564629in}}%
\pgfpathlineto{\pgfqpoint{3.073274in}{1.579343in}}%
\pgfpathlineto{\pgfqpoint{3.141411in}{1.614155in}}%
\pgfpathlineto{\pgfqpoint{3.151636in}{1.619567in}}%
\pgfpathlineto{\pgfqpoint{3.209549in}{1.650223in}}%
\pgfpathlineto{\pgfqpoint{3.255421in}{1.674505in}}%
\pgfpathlineto{\pgfqpoint{3.277686in}{1.686784in}}%
\pgfpathlineto{\pgfqpoint{3.320244in}{1.708818in}}%
\pgfpathlineto{\pgfqpoint{3.345824in}{1.729443in}}%
\pgfpathlineto{\pgfqpoint{3.413961in}{1.784381in}}%
\pgfpathlineto{\pgfqpoint{3.482099in}{1.839320in}}%
\pgfpathlineto{\pgfqpoint{3.550236in}{1.894258in}}%
\pgfpathlineto{\pgfqpoint{3.550236in}{1.949196in}}%
\pgfpathlineto{\pgfqpoint{3.618374in}{2.004134in}}%
\pgfpathlineto{\pgfqpoint{3.686511in}{2.059072in}}%
\pgfpathlineto{\pgfqpoint{3.686511in}{2.114011in}}%
\pgfpathlineto{\pgfqpoint{3.754649in}{2.168949in}}%
\pgfpathlineto{\pgfqpoint{3.754649in}{2.201612in}}%
\pgfpathlineto{\pgfqpoint{3.707448in}{2.168949in}}%
\pgfpathlineto{\pgfqpoint{3.686511in}{2.154460in}}%
\pgfpathlineto{\pgfqpoint{3.628058in}{2.114011in}}%
\pgfpathlineto{\pgfqpoint{3.618374in}{2.107309in}}%
\pgfpathlineto{\pgfqpoint{3.550236in}{2.069188in}}%
\pgfpathlineto{\pgfqpoint{3.529745in}{2.059072in}}%
\pgfpathlineto{\pgfqpoint{3.482099in}{2.035551in}}%
\pgfpathlineto{\pgfqpoint{3.418459in}{2.004134in}}%
\pgfpathlineto{\pgfqpoint{3.413961in}{2.001914in}}%
\pgfpathlineto{\pgfqpoint{3.345824in}{1.968276in}}%
\pgfpathlineto{\pgfqpoint{3.307173in}{1.949196in}}%
\pgfpathlineto{\pgfqpoint{3.277686in}{1.934639in}}%
\pgfpathlineto{\pgfqpoint{3.209549in}{1.901002in}}%
\pgfpathlineto{\pgfqpoint{3.195887in}{1.894258in}}%
\pgfpathlineto{\pgfqpoint{3.141411in}{1.867365in}}%
\pgfpathlineto{\pgfqpoint{3.076025in}{1.839320in}}%
\pgfpathlineto{\pgfqpoint{3.073274in}{1.838037in}}%
\pgfpathlineto{\pgfqpoint{3.067192in}{1.839320in}}%
\pgfpathlineto{\pgfqpoint{3.005136in}{1.857538in}}%
\pgfpathlineto{\pgfqpoint{2.936999in}{1.877542in}}%
\pgfpathlineto{\pgfqpoint{2.880062in}{1.894258in}}%
\pgfpathlineto{\pgfqpoint{2.868861in}{1.897546in}}%
\pgfpathlineto{\pgfqpoint{2.800724in}{1.917550in}}%
\pgfpathlineto{\pgfqpoint{2.732586in}{1.937554in}}%
\pgfpathlineto{\pgfqpoint{2.692931in}{1.949196in}}%
\pgfpathlineto{\pgfqpoint{2.664449in}{1.957558in}}%
\pgfpathlineto{\pgfqpoint{2.596311in}{1.977562in}}%
\pgfpathlineto{\pgfqpoint{2.528174in}{1.997566in}}%
\pgfpathlineto{\pgfqpoint{2.505801in}{2.004134in}}%
\pgfpathlineto{\pgfqpoint{2.460036in}{2.017570in}}%
\pgfpathlineto{\pgfqpoint{2.391899in}{2.037574in}}%
\pgfpathlineto{\pgfqpoint{2.323761in}{2.057645in}}%
\pgfpathlineto{\pgfqpoint{2.319278in}{2.059072in}}%
\pgfpathlineto{\pgfqpoint{2.255624in}{2.079292in}}%
\pgfpathlineto{\pgfqpoint{2.187486in}{2.102019in}}%
\pgfpathlineto{\pgfqpoint{2.153747in}{2.114011in}}%
\pgfpathlineto{\pgfqpoint{2.119349in}{2.126237in}}%
\pgfpathlineto{\pgfqpoint{2.051211in}{2.150455in}}%
\pgfpathlineto{\pgfqpoint{1.999180in}{2.168949in}}%
\pgfpathlineto{\pgfqpoint{1.983074in}{2.174674in}}%
\pgfpathlineto{\pgfqpoint{1.914936in}{2.198892in}}%
\pgfpathlineto{\pgfqpoint{1.846799in}{2.223110in}}%
\pgfpathlineto{\pgfqpoint{1.844613in}{2.223887in}}%
\pgfpathlineto{\pgfqpoint{1.778661in}{2.247329in}}%
\pgfpathlineto{\pgfqpoint{1.710524in}{2.271547in}}%
\pgfpathlineto{\pgfqpoint{1.690047in}{2.278825in}}%
\pgfpathlineto{\pgfqpoint{1.642386in}{2.295765in}}%
\pgfpathlineto{\pgfqpoint{1.574249in}{2.319984in}}%
\pgfpathlineto{\pgfqpoint{1.533347in}{2.333763in}}%
\pgfpathlineto{\pgfqpoint{1.506111in}{2.343155in}}%
\pgfpathlineto{\pgfqpoint{1.437974in}{2.357675in}}%
\pgfpathlineto{\pgfqpoint{1.369836in}{2.342991in}}%
\pgfpathlineto{\pgfqpoint{1.366334in}{2.333763in}}%
\pgfpathlineto{\pgfqpoint{1.345485in}{2.278825in}}%
\pgfpathlineto{\pgfqpoint{1.324635in}{2.223887in}}%
\pgfpathlineto{\pgfqpoint{1.303786in}{2.168949in}}%
\pgfpathlineto{\pgfqpoint{1.301699in}{2.163449in}}%
\pgfpathlineto{\pgfqpoint{1.282936in}{2.114011in}}%
\pgfpathlineto{\pgfqpoint{1.255766in}{2.059072in}}%
\pgfpathlineto{\pgfqpoint{1.233561in}{2.037389in}}%
\pgfpathlineto{\pgfqpoint{1.165424in}{2.027611in}}%
\pgfpathlineto{\pgfqpoint{1.151034in}{2.059072in}}%
\pgfpathlineto{\pgfqpoint{1.130978in}{2.114011in}}%
\pgfpathlineto{\pgfqpoint{1.115692in}{2.168949in}}%
\pgfpathlineto{\pgfqpoint{1.104839in}{2.223887in}}%
\pgfpathlineto{\pgfqpoint{1.100966in}{2.278825in}}%
\pgfpathlineto{\pgfqpoint{1.097526in}{2.333763in}}%
\pgfpathlineto{\pgfqpoint{1.097286in}{2.337842in}}%
\pgfpathlineto{\pgfqpoint{1.092891in}{2.388702in}}%
\pgfpathlineto{\pgfqpoint{1.084182in}{2.443640in}}%
\pgfpathlineto{\pgfqpoint{1.062163in}{2.498578in}}%
\pgfpathlineto{\pgfqpoint{1.029149in}{2.526883in}}%
\pgfpathlineto{\pgfqpoint{1.029149in}{2.498578in}}%
\pgfpathlineto{\pgfqpoint{0.961011in}{2.443640in}}%
\pgfpathlineto{\pgfqpoint{0.961011in}{2.388702in}}%
\pgfpathlineto{\pgfqpoint{0.892874in}{2.333763in}}%
\pgfpathlineto{\pgfqpoint{0.892874in}{2.278825in}}%
\pgfpathlineto{\pgfqpoint{0.892874in}{2.223887in}}%
\pgfpathlineto{\pgfqpoint{0.892874in}{2.170759in}}%
\pgfpathlineto{\pgfqpoint{0.904418in}{2.168949in}}%
\pgfpathlineto{\pgfqpoint{0.961011in}{2.159517in}}%
\pgfpathlineto{\pgfqpoint{0.995383in}{2.114011in}}%
\pgfpathlineto{\pgfqpoint{1.029149in}{2.069741in}}%
\pgfpathlineto{\pgfqpoint{1.030914in}{2.059072in}}%
\pgfpathlineto{\pgfqpoint{1.040004in}{2.004134in}}%
\pgfpathlineto{\pgfqpoint{1.049094in}{1.949196in}}%
\pgfpathlineto{\pgfqpoint{1.058184in}{1.894258in}}%
\pgfpathlineto{\pgfqpoint{1.067275in}{1.839320in}}%
\pgfpathlineto{\pgfqpoint{1.076365in}{1.784381in}}%
\pgfpathlineto{\pgfqpoint{1.085455in}{1.729443in}}%
\pgfpathlineto{\pgfqpoint{1.097286in}{1.679372in}}%
\pgfpathlineto{\pgfqpoint{1.127043in}{1.674505in}}%
\pgfpathlineto{\pgfqpoint{1.165424in}{1.669120in}}%
\pgfpathlineto{\pgfqpoint{1.233561in}{1.659560in}}%
\pgfpathlineto{\pgfqpoint{1.280527in}{1.674505in}}%
\pgfpathlineto{\pgfqpoint{1.301699in}{1.681399in}}%
\pgfpathlineto{\pgfqpoint{1.369836in}{1.692239in}}%
\pgfpathlineto{\pgfqpoint{1.437974in}{1.703270in}}%
\pgfpathlineto{\pgfqpoint{1.460934in}{1.729443in}}%
\pgfpathlineto{\pgfqpoint{1.481783in}{1.784381in}}%
\pgfpathlineto{\pgfqpoint{1.502632in}{1.839320in}}%
\pgfpathlineto{\pgfqpoint{1.506111in}{1.848486in}}%
\pgfpathlineto{\pgfqpoint{1.561047in}{1.894258in}}%
\pgfpathlineto{\pgfqpoint{1.574249in}{1.899801in}}%
\pgfpathlineto{\pgfqpoint{1.601535in}{1.894258in}}%
\pgfpathlineto{\pgfqpoint{1.642386in}{1.885959in}}%
\pgfpathlineto{\pgfqpoint{1.710524in}{1.872116in}}%
\pgfpathlineto{\pgfqpoint{1.778661in}{1.855290in}}%
\pgfpathlineto{\pgfqpoint{1.823592in}{1.839320in}}%
\pgfpathlineto{\pgfqpoint{1.846799in}{1.831071in}}%
\pgfpathlineto{\pgfqpoint{1.914936in}{1.806853in}}%
\pgfpathlineto{\pgfqpoint{1.978159in}{1.784381in}}%
\pgfpathlineto{\pgfqpoint{1.983074in}{1.782635in}}%
\pgfpathlineto{\pgfqpoint{2.051211in}{1.758416in}}%
\pgfpathlineto{\pgfqpoint{2.119349in}{1.734198in}}%
\pgfpathlineto{\pgfqpoint{2.132725in}{1.729443in}}%
\pgfpathlineto{\pgfqpoint{2.187486in}{1.709979in}}%
\pgfpathlineto{\pgfqpoint{2.255624in}{1.685761in}}%
\pgfpathlineto{\pgfqpoint{2.287292in}{1.674505in}}%
\pgfpathlineto{\pgfqpoint{2.323761in}{1.661543in}}%
\pgfpathlineto{\pgfqpoint{2.391899in}{1.637324in}}%
\pgfpathlineto{\pgfqpoint{2.441859in}{1.619567in}}%
\pgfpathlineto{\pgfqpoint{2.460036in}{1.613106in}}%
\pgfpathlineto{\pgfqpoint{2.528174in}{1.588888in}}%
\pgfpathlineto{\pgfqpoint{2.596311in}{1.564669in}}%
\pgfpathlineto{\pgfqpoint{2.596426in}{1.564629in}}%
\pgfpathlineto{\pgfqpoint{2.664449in}{1.540451in}}%
\pgfpathlineto{\pgfqpoint{2.732586in}{1.516233in}}%
\pgfpathlineto{\pgfqpoint{2.750992in}{1.509690in}}%
\pgfpathclose%
\pgfusepath{fill}%
\end{pgfscope}%
\begin{pgfscope}%
\pgfpathrectangle{\pgfqpoint{0.634105in}{0.521603in}}{\pgfqpoint{3.720000in}{3.020000in}} %
\pgfusepath{clip}%
\pgfsetbuttcap%
\pgfsetroundjoin%
\definecolor{currentfill}{rgb}{0.562745,0.653983,0.687745}%
\pgfsetfillcolor{currentfill}%
\pgfsetlinewidth{0.000000pt}%
\definecolor{currentstroke}{rgb}{0.000000,0.000000,0.000000}%
\pgfsetstrokecolor{currentstroke}%
\pgfsetdash{}{0pt}%
\pgfpathmoveto{\pgfqpoint{3.073274in}{1.838037in}}%
\pgfpathlineto{\pgfqpoint{3.076025in}{1.839320in}}%
\pgfpathlineto{\pgfqpoint{3.141411in}{1.867365in}}%
\pgfpathlineto{\pgfqpoint{3.195887in}{1.894258in}}%
\pgfpathlineto{\pgfqpoint{3.209549in}{1.901002in}}%
\pgfpathlineto{\pgfqpoint{3.277686in}{1.934639in}}%
\pgfpathlineto{\pgfqpoint{3.307173in}{1.949196in}}%
\pgfpathlineto{\pgfqpoint{3.345824in}{1.968276in}}%
\pgfpathlineto{\pgfqpoint{3.413961in}{2.001914in}}%
\pgfpathlineto{\pgfqpoint{3.418459in}{2.004134in}}%
\pgfpathlineto{\pgfqpoint{3.482099in}{2.035551in}}%
\pgfpathlineto{\pgfqpoint{3.529745in}{2.059072in}}%
\pgfpathlineto{\pgfqpoint{3.550236in}{2.069188in}}%
\pgfpathlineto{\pgfqpoint{3.618374in}{2.107309in}}%
\pgfpathlineto{\pgfqpoint{3.628058in}{2.114011in}}%
\pgfpathlineto{\pgfqpoint{3.686511in}{2.154460in}}%
\pgfpathlineto{\pgfqpoint{3.707448in}{2.168949in}}%
\pgfpathlineto{\pgfqpoint{3.754649in}{2.201612in}}%
\pgfpathlineto{\pgfqpoint{3.754649in}{2.223887in}}%
\pgfpathlineto{\pgfqpoint{3.822786in}{2.278825in}}%
\pgfpathlineto{\pgfqpoint{3.890924in}{2.333763in}}%
\pgfpathlineto{\pgfqpoint{3.890924in}{2.388702in}}%
\pgfpathlineto{\pgfqpoint{3.959061in}{2.443640in}}%
\pgfpathlineto{\pgfqpoint{4.027199in}{2.498578in}}%
\pgfpathlineto{\pgfqpoint{4.027199in}{2.553516in}}%
\pgfpathlineto{\pgfqpoint{4.095336in}{2.608454in}}%
\pgfpathlineto{\pgfqpoint{4.095336in}{2.663393in}}%
\pgfpathlineto{\pgfqpoint{4.095336in}{2.718331in}}%
\pgfpathlineto{\pgfqpoint{4.095336in}{2.718774in}}%
\pgfpathlineto{\pgfqpoint{4.094696in}{2.718331in}}%
\pgfpathlineto{\pgfqpoint{4.027199in}{2.671622in}}%
\pgfpathlineto{\pgfqpoint{4.015306in}{2.663393in}}%
\pgfpathlineto{\pgfqpoint{3.959061in}{2.624471in}}%
\pgfpathlineto{\pgfqpoint{3.935916in}{2.608454in}}%
\pgfpathlineto{\pgfqpoint{3.890924in}{2.577320in}}%
\pgfpathlineto{\pgfqpoint{3.856525in}{2.553516in}}%
\pgfpathlineto{\pgfqpoint{3.822786in}{2.530169in}}%
\pgfpathlineto{\pgfqpoint{3.777135in}{2.498578in}}%
\pgfpathlineto{\pgfqpoint{3.754649in}{2.483017in}}%
\pgfpathlineto{\pgfqpoint{3.697745in}{2.443640in}}%
\pgfpathlineto{\pgfqpoint{3.686511in}{2.435866in}}%
\pgfpathlineto{\pgfqpoint{3.618374in}{2.396333in}}%
\pgfpathlineto{\pgfqpoint{3.602916in}{2.388702in}}%
\pgfpathlineto{\pgfqpoint{3.550236in}{2.362695in}}%
\pgfpathlineto{\pgfqpoint{3.491630in}{2.333763in}}%
\pgfpathlineto{\pgfqpoint{3.482099in}{2.329058in}}%
\pgfpathlineto{\pgfqpoint{3.413961in}{2.295421in}}%
\pgfpathlineto{\pgfqpoint{3.380344in}{2.278825in}}%
\pgfpathlineto{\pgfqpoint{3.345824in}{2.261784in}}%
\pgfpathlineto{\pgfqpoint{3.277686in}{2.229450in}}%
\pgfpathlineto{\pgfqpoint{3.209549in}{2.241630in}}%
\pgfpathlineto{\pgfqpoint{3.141411in}{2.261634in}}%
\pgfpathlineto{\pgfqpoint{3.082854in}{2.278825in}}%
\pgfpathlineto{\pgfqpoint{3.073274in}{2.281638in}}%
\pgfpathlineto{\pgfqpoint{3.005136in}{2.301642in}}%
\pgfpathlineto{\pgfqpoint{2.936999in}{2.321646in}}%
\pgfpathlineto{\pgfqpoint{2.895724in}{2.333763in}}%
\pgfpathlineto{\pgfqpoint{2.868861in}{2.341650in}}%
\pgfpathlineto{\pgfqpoint{2.800724in}{2.361654in}}%
\pgfpathlineto{\pgfqpoint{2.732586in}{2.381658in}}%
\pgfpathlineto{\pgfqpoint{2.708594in}{2.388702in}}%
\pgfpathlineto{\pgfqpoint{2.664449in}{2.401662in}}%
\pgfpathlineto{\pgfqpoint{2.596311in}{2.421666in}}%
\pgfpathlineto{\pgfqpoint{2.528174in}{2.441670in}}%
\pgfpathlineto{\pgfqpoint{2.521464in}{2.443640in}}%
\pgfpathlineto{\pgfqpoint{2.460036in}{2.461674in}}%
\pgfpathlineto{\pgfqpoint{2.391899in}{2.481678in}}%
\pgfpathlineto{\pgfqpoint{2.334334in}{2.498578in}}%
\pgfpathlineto{\pgfqpoint{2.323761in}{2.501682in}}%
\pgfpathlineto{\pgfqpoint{2.255624in}{2.521686in}}%
\pgfpathlineto{\pgfqpoint{2.187486in}{2.541690in}}%
\pgfpathlineto{\pgfqpoint{2.147204in}{2.553516in}}%
\pgfpathlineto{\pgfqpoint{2.119349in}{2.561694in}}%
\pgfpathlineto{\pgfqpoint{2.051211in}{2.581698in}}%
\pgfpathlineto{\pgfqpoint{1.983074in}{2.601702in}}%
\pgfpathlineto{\pgfqpoint{1.960073in}{2.608454in}}%
\pgfpathlineto{\pgfqpoint{1.914936in}{2.621706in}}%
\pgfpathlineto{\pgfqpoint{1.846799in}{2.641710in}}%
\pgfpathlineto{\pgfqpoint{1.778661in}{2.661714in}}%
\pgfpathlineto{\pgfqpoint{1.772943in}{2.663393in}}%
\pgfpathlineto{\pgfqpoint{1.710524in}{2.681718in}}%
\pgfpathlineto{\pgfqpoint{1.642386in}{2.701722in}}%
\pgfpathlineto{\pgfqpoint{1.585813in}{2.718331in}}%
\pgfpathlineto{\pgfqpoint{1.574249in}{2.721726in}}%
\pgfpathlineto{\pgfqpoint{1.506111in}{2.741730in}}%
\pgfpathlineto{\pgfqpoint{1.437974in}{2.762575in}}%
\pgfpathlineto{\pgfqpoint{1.405840in}{2.773269in}}%
\pgfpathlineto{\pgfqpoint{1.369836in}{2.785224in}}%
\pgfpathlineto{\pgfqpoint{1.301699in}{2.808896in}}%
\pgfpathlineto{\pgfqpoint{1.237165in}{2.828207in}}%
\pgfpathlineto{\pgfqpoint{1.233561in}{2.829366in}}%
\pgfpathlineto{\pgfqpoint{1.165424in}{2.853515in}}%
\pgfpathlineto{\pgfqpoint{1.165424in}{2.828207in}}%
\pgfpathlineto{\pgfqpoint{1.165424in}{2.773269in}}%
\pgfpathlineto{\pgfqpoint{1.097286in}{2.718331in}}%
\pgfpathlineto{\pgfqpoint{1.097286in}{2.663393in}}%
\pgfpathlineto{\pgfqpoint{1.029149in}{2.608454in}}%
\pgfpathlineto{\pgfqpoint{1.029149in}{2.553516in}}%
\pgfpathlineto{\pgfqpoint{1.029149in}{2.526883in}}%
\pgfpathlineto{\pgfqpoint{1.062163in}{2.498578in}}%
\pgfpathlineto{\pgfqpoint{1.084182in}{2.443640in}}%
\pgfpathlineto{\pgfqpoint{1.092891in}{2.388702in}}%
\pgfpathlineto{\pgfqpoint{1.097286in}{2.337842in}}%
\pgfpathlineto{\pgfqpoint{1.097526in}{2.333763in}}%
\pgfpathlineto{\pgfqpoint{1.100966in}{2.278825in}}%
\pgfpathlineto{\pgfqpoint{1.104839in}{2.223887in}}%
\pgfpathlineto{\pgfqpoint{1.115692in}{2.168949in}}%
\pgfpathlineto{\pgfqpoint{1.130978in}{2.114011in}}%
\pgfpathlineto{\pgfqpoint{1.151034in}{2.059072in}}%
\pgfpathlineto{\pgfqpoint{1.165424in}{2.027611in}}%
\pgfpathlineto{\pgfqpoint{1.233561in}{2.037389in}}%
\pgfpathlineto{\pgfqpoint{1.255766in}{2.059072in}}%
\pgfpathlineto{\pgfqpoint{1.282936in}{2.114011in}}%
\pgfpathlineto{\pgfqpoint{1.301699in}{2.163449in}}%
\pgfpathlineto{\pgfqpoint{1.303786in}{2.168949in}}%
\pgfpathlineto{\pgfqpoint{1.324635in}{2.223887in}}%
\pgfpathlineto{\pgfqpoint{1.345485in}{2.278825in}}%
\pgfpathlineto{\pgfqpoint{1.366334in}{2.333763in}}%
\pgfpathlineto{\pgfqpoint{1.369836in}{2.342991in}}%
\pgfpathlineto{\pgfqpoint{1.437974in}{2.357675in}}%
\pgfpathlineto{\pgfqpoint{1.506111in}{2.343155in}}%
\pgfpathlineto{\pgfqpoint{1.533347in}{2.333763in}}%
\pgfpathlineto{\pgfqpoint{1.574249in}{2.319984in}}%
\pgfpathlineto{\pgfqpoint{1.642386in}{2.295765in}}%
\pgfpathlineto{\pgfqpoint{1.690047in}{2.278825in}}%
\pgfpathlineto{\pgfqpoint{1.710524in}{2.271547in}}%
\pgfpathlineto{\pgfqpoint{1.778661in}{2.247329in}}%
\pgfpathlineto{\pgfqpoint{1.844613in}{2.223887in}}%
\pgfpathlineto{\pgfqpoint{1.846799in}{2.223110in}}%
\pgfpathlineto{\pgfqpoint{1.914936in}{2.198892in}}%
\pgfpathlineto{\pgfqpoint{1.983074in}{2.174674in}}%
\pgfpathlineto{\pgfqpoint{1.999180in}{2.168949in}}%
\pgfpathlineto{\pgfqpoint{2.051211in}{2.150455in}}%
\pgfpathlineto{\pgfqpoint{2.119349in}{2.126237in}}%
\pgfpathlineto{\pgfqpoint{2.153747in}{2.114011in}}%
\pgfpathlineto{\pgfqpoint{2.187486in}{2.102019in}}%
\pgfpathlineto{\pgfqpoint{2.255624in}{2.079292in}}%
\pgfpathlineto{\pgfqpoint{2.319278in}{2.059072in}}%
\pgfpathlineto{\pgfqpoint{2.323761in}{2.057645in}}%
\pgfpathlineto{\pgfqpoint{2.391899in}{2.037574in}}%
\pgfpathlineto{\pgfqpoint{2.460036in}{2.017570in}}%
\pgfpathlineto{\pgfqpoint{2.505801in}{2.004134in}}%
\pgfpathlineto{\pgfqpoint{2.528174in}{1.997566in}}%
\pgfpathlineto{\pgfqpoint{2.596311in}{1.977562in}}%
\pgfpathlineto{\pgfqpoint{2.664449in}{1.957558in}}%
\pgfpathlineto{\pgfqpoint{2.692931in}{1.949196in}}%
\pgfpathlineto{\pgfqpoint{2.732586in}{1.937554in}}%
\pgfpathlineto{\pgfqpoint{2.800724in}{1.917550in}}%
\pgfpathlineto{\pgfqpoint{2.868861in}{1.897546in}}%
\pgfpathlineto{\pgfqpoint{2.880062in}{1.894258in}}%
\pgfpathlineto{\pgfqpoint{2.936999in}{1.877542in}}%
\pgfpathlineto{\pgfqpoint{3.005136in}{1.857538in}}%
\pgfpathlineto{\pgfqpoint{3.067192in}{1.839320in}}%
\pgfpathclose%
\pgfusepath{fill}%
\end{pgfscope}%
\begin{pgfscope}%
\pgfpathrectangle{\pgfqpoint{0.634105in}{0.521603in}}{\pgfqpoint{3.720000in}{3.020000in}} %
\pgfusepath{clip}%
\pgfsetbuttcap%
\pgfsetroundjoin%
\definecolor{currentfill}{rgb}{0.710478,0.814706,0.814706}%
\pgfsetfillcolor{currentfill}%
\pgfsetlinewidth{0.000000pt}%
\definecolor{currentstroke}{rgb}{0.000000,0.000000,0.000000}%
\pgfsetstrokecolor{currentstroke}%
\pgfsetdash{}{0pt}%
\pgfpathmoveto{\pgfqpoint{3.141411in}{2.261634in}}%
\pgfpathlineto{\pgfqpoint{3.209549in}{2.241630in}}%
\pgfpathlineto{\pgfqpoint{3.277686in}{2.229450in}}%
\pgfpathlineto{\pgfqpoint{3.345824in}{2.261784in}}%
\pgfpathlineto{\pgfqpoint{3.380344in}{2.278825in}}%
\pgfpathlineto{\pgfqpoint{3.413961in}{2.295421in}}%
\pgfpathlineto{\pgfqpoint{3.482099in}{2.329058in}}%
\pgfpathlineto{\pgfqpoint{3.491630in}{2.333763in}}%
\pgfpathlineto{\pgfqpoint{3.550236in}{2.362695in}}%
\pgfpathlineto{\pgfqpoint{3.602916in}{2.388702in}}%
\pgfpathlineto{\pgfqpoint{3.618374in}{2.396333in}}%
\pgfpathlineto{\pgfqpoint{3.686511in}{2.435866in}}%
\pgfpathlineto{\pgfqpoint{3.697745in}{2.443640in}}%
\pgfpathlineto{\pgfqpoint{3.754649in}{2.483017in}}%
\pgfpathlineto{\pgfqpoint{3.777135in}{2.498578in}}%
\pgfpathlineto{\pgfqpoint{3.822786in}{2.530169in}}%
\pgfpathlineto{\pgfqpoint{3.856525in}{2.553516in}}%
\pgfpathlineto{\pgfqpoint{3.890924in}{2.577320in}}%
\pgfpathlineto{\pgfqpoint{3.935916in}{2.608454in}}%
\pgfpathlineto{\pgfqpoint{3.959061in}{2.624471in}}%
\pgfpathlineto{\pgfqpoint{4.015306in}{2.663393in}}%
\pgfpathlineto{\pgfqpoint{4.027199in}{2.671622in}}%
\pgfpathlineto{\pgfqpoint{4.094696in}{2.718331in}}%
\pgfpathlineto{\pgfqpoint{4.095336in}{2.718774in}}%
\pgfpathlineto{\pgfqpoint{4.095336in}{2.773269in}}%
\pgfpathlineto{\pgfqpoint{4.027199in}{2.828207in}}%
\pgfpathlineto{\pgfqpoint{4.027199in}{2.883145in}}%
\pgfpathlineto{\pgfqpoint{4.027199in}{2.938084in}}%
\pgfpathlineto{\pgfqpoint{4.017224in}{2.946126in}}%
\pgfpathlineto{\pgfqpoint{4.005603in}{2.938084in}}%
\pgfpathlineto{\pgfqpoint{3.959061in}{2.905877in}}%
\pgfpathlineto{\pgfqpoint{3.926213in}{2.883145in}}%
\pgfpathlineto{\pgfqpoint{3.890924in}{2.858725in}}%
\pgfpathlineto{\pgfqpoint{3.846822in}{2.828207in}}%
\pgfpathlineto{\pgfqpoint{3.822786in}{2.811574in}}%
\pgfpathlineto{\pgfqpoint{3.767432in}{2.773269in}}%
\pgfpathlineto{\pgfqpoint{3.754649in}{2.764423in}}%
\pgfpathlineto{\pgfqpoint{3.686511in}{2.723477in}}%
\pgfpathlineto{\pgfqpoint{3.676087in}{2.718331in}}%
\pgfpathlineto{\pgfqpoint{3.618374in}{2.689840in}}%
\pgfpathlineto{\pgfqpoint{3.564801in}{2.663393in}}%
\pgfpathlineto{\pgfqpoint{3.550236in}{2.656202in}}%
\pgfpathlineto{\pgfqpoint{3.482099in}{2.623297in}}%
\pgfpathlineto{\pgfqpoint{3.413961in}{2.625722in}}%
\pgfpathlineto{\pgfqpoint{3.345824in}{2.645726in}}%
\pgfpathlineto{\pgfqpoint{3.285647in}{2.663393in}}%
\pgfpathlineto{\pgfqpoint{3.277686in}{2.665730in}}%
\pgfpathlineto{\pgfqpoint{3.209549in}{2.685734in}}%
\pgfpathlineto{\pgfqpoint{3.141411in}{2.705738in}}%
\pgfpathlineto{\pgfqpoint{3.098517in}{2.718331in}}%
\pgfpathlineto{\pgfqpoint{3.073274in}{2.725742in}}%
\pgfpathlineto{\pgfqpoint{3.005136in}{2.745746in}}%
\pgfpathlineto{\pgfqpoint{2.936999in}{2.765750in}}%
\pgfpathlineto{\pgfqpoint{2.911387in}{2.773269in}}%
\pgfpathlineto{\pgfqpoint{2.868861in}{2.785754in}}%
\pgfpathlineto{\pgfqpoint{2.800724in}{2.805758in}}%
\pgfpathlineto{\pgfqpoint{2.732586in}{2.825762in}}%
\pgfpathlineto{\pgfqpoint{2.724257in}{2.828207in}}%
\pgfpathlineto{\pgfqpoint{2.664449in}{2.845766in}}%
\pgfpathlineto{\pgfqpoint{2.596311in}{2.865770in}}%
\pgfpathlineto{\pgfqpoint{2.537126in}{2.883145in}}%
\pgfpathlineto{\pgfqpoint{2.528174in}{2.885774in}}%
\pgfpathlineto{\pgfqpoint{2.460036in}{2.905778in}}%
\pgfpathlineto{\pgfqpoint{2.391899in}{2.925782in}}%
\pgfpathlineto{\pgfqpoint{2.349996in}{2.938084in}}%
\pgfpathlineto{\pgfqpoint{2.323761in}{2.945786in}}%
\pgfpathlineto{\pgfqpoint{2.255624in}{2.965790in}}%
\pgfpathlineto{\pgfqpoint{2.187486in}{2.985794in}}%
\pgfpathlineto{\pgfqpoint{2.162866in}{2.993022in}}%
\pgfpathlineto{\pgfqpoint{2.119349in}{3.005798in}}%
\pgfpathlineto{\pgfqpoint{2.051211in}{3.025802in}}%
\pgfpathlineto{\pgfqpoint{1.983074in}{3.045806in}}%
\pgfpathlineto{\pgfqpoint{1.975736in}{3.047960in}}%
\pgfpathlineto{\pgfqpoint{1.914936in}{3.047960in}}%
\pgfpathlineto{\pgfqpoint{1.846799in}{3.047960in}}%
\pgfpathlineto{\pgfqpoint{1.778661in}{3.047960in}}%
\pgfpathlineto{\pgfqpoint{1.710524in}{2.993022in}}%
\pgfpathlineto{\pgfqpoint{1.642386in}{2.993022in}}%
\pgfpathlineto{\pgfqpoint{1.574249in}{2.993022in}}%
\pgfpathlineto{\pgfqpoint{1.506111in}{2.993022in}}%
\pgfpathlineto{\pgfqpoint{1.437974in}{2.993022in}}%
\pgfpathlineto{\pgfqpoint{1.369836in}{2.938084in}}%
\pgfpathlineto{\pgfqpoint{1.301699in}{2.938084in}}%
\pgfpathlineto{\pgfqpoint{1.233561in}{2.938084in}}%
\pgfpathlineto{\pgfqpoint{1.165424in}{2.883145in}}%
\pgfpathlineto{\pgfqpoint{1.165424in}{2.853515in}}%
\pgfpathlineto{\pgfqpoint{1.233561in}{2.829366in}}%
\pgfpathlineto{\pgfqpoint{1.237165in}{2.828207in}}%
\pgfpathlineto{\pgfqpoint{1.301699in}{2.808896in}}%
\pgfpathlineto{\pgfqpoint{1.369836in}{2.785224in}}%
\pgfpathlineto{\pgfqpoint{1.405840in}{2.773269in}}%
\pgfpathlineto{\pgfqpoint{1.437974in}{2.762575in}}%
\pgfpathlineto{\pgfqpoint{1.506111in}{2.741730in}}%
\pgfpathlineto{\pgfqpoint{1.574249in}{2.721726in}}%
\pgfpathlineto{\pgfqpoint{1.585813in}{2.718331in}}%
\pgfpathlineto{\pgfqpoint{1.642386in}{2.701722in}}%
\pgfpathlineto{\pgfqpoint{1.710524in}{2.681718in}}%
\pgfpathlineto{\pgfqpoint{1.772943in}{2.663393in}}%
\pgfpathlineto{\pgfqpoint{1.778661in}{2.661714in}}%
\pgfpathlineto{\pgfqpoint{1.846799in}{2.641710in}}%
\pgfpathlineto{\pgfqpoint{1.914936in}{2.621706in}}%
\pgfpathlineto{\pgfqpoint{1.960073in}{2.608454in}}%
\pgfpathlineto{\pgfqpoint{1.983074in}{2.601702in}}%
\pgfpathlineto{\pgfqpoint{2.051211in}{2.581698in}}%
\pgfpathlineto{\pgfqpoint{2.119349in}{2.561694in}}%
\pgfpathlineto{\pgfqpoint{2.147204in}{2.553516in}}%
\pgfpathlineto{\pgfqpoint{2.187486in}{2.541690in}}%
\pgfpathlineto{\pgfqpoint{2.255624in}{2.521686in}}%
\pgfpathlineto{\pgfqpoint{2.323761in}{2.501682in}}%
\pgfpathlineto{\pgfqpoint{2.334334in}{2.498578in}}%
\pgfpathlineto{\pgfqpoint{2.391899in}{2.481678in}}%
\pgfpathlineto{\pgfqpoint{2.460036in}{2.461674in}}%
\pgfpathlineto{\pgfqpoint{2.521464in}{2.443640in}}%
\pgfpathlineto{\pgfqpoint{2.528174in}{2.441670in}}%
\pgfpathlineto{\pgfqpoint{2.596311in}{2.421666in}}%
\pgfpathlineto{\pgfqpoint{2.664449in}{2.401662in}}%
\pgfpathlineto{\pgfqpoint{2.708594in}{2.388702in}}%
\pgfpathlineto{\pgfqpoint{2.732586in}{2.381658in}}%
\pgfpathlineto{\pgfqpoint{2.800724in}{2.361654in}}%
\pgfpathlineto{\pgfqpoint{2.868861in}{2.341650in}}%
\pgfpathlineto{\pgfqpoint{2.895724in}{2.333763in}}%
\pgfpathlineto{\pgfqpoint{2.936999in}{2.321646in}}%
\pgfpathlineto{\pgfqpoint{3.005136in}{2.301642in}}%
\pgfpathlineto{\pgfqpoint{3.073274in}{2.281638in}}%
\pgfpathlineto{\pgfqpoint{3.082854in}{2.278825in}}%
\pgfpathclose%
\pgfusepath{fill}%
\end{pgfscope}%
\begin{pgfscope}%
\pgfpathrectangle{\pgfqpoint{0.634105in}{0.521603in}}{\pgfqpoint{3.720000in}{3.020000in}} %
\pgfusepath{clip}%
\pgfsetbuttcap%
\pgfsetroundjoin%
\definecolor{currentfill}{rgb}{0.903493,0.938235,0.938235}%
\pgfsetfillcolor{currentfill}%
\pgfsetlinewidth{0.000000pt}%
\definecolor{currentstroke}{rgb}{0.000000,0.000000,0.000000}%
\pgfsetstrokecolor{currentstroke}%
\pgfsetdash{}{0pt}%
\pgfpathmoveto{\pgfqpoint{3.345824in}{2.645726in}}%
\pgfpathlineto{\pgfqpoint{3.413961in}{2.625722in}}%
\pgfpathlineto{\pgfqpoint{3.482099in}{2.623297in}}%
\pgfpathlineto{\pgfqpoint{3.550236in}{2.656202in}}%
\pgfpathlineto{\pgfqpoint{3.564801in}{2.663393in}}%
\pgfpathlineto{\pgfqpoint{3.618374in}{2.689840in}}%
\pgfpathlineto{\pgfqpoint{3.676087in}{2.718331in}}%
\pgfpathlineto{\pgfqpoint{3.686511in}{2.723477in}}%
\pgfpathlineto{\pgfqpoint{3.754649in}{2.764423in}}%
\pgfpathlineto{\pgfqpoint{3.767432in}{2.773269in}}%
\pgfpathlineto{\pgfqpoint{3.822786in}{2.811574in}}%
\pgfpathlineto{\pgfqpoint{3.846822in}{2.828207in}}%
\pgfpathlineto{\pgfqpoint{3.890924in}{2.858725in}}%
\pgfpathlineto{\pgfqpoint{3.926213in}{2.883145in}}%
\pgfpathlineto{\pgfqpoint{3.959061in}{2.905877in}}%
\pgfpathlineto{\pgfqpoint{4.005603in}{2.938084in}}%
\pgfpathlineto{\pgfqpoint{4.017224in}{2.946126in}}%
\pgfpathlineto{\pgfqpoint{3.959061in}{2.993022in}}%
\pgfpathlineto{\pgfqpoint{3.959061in}{3.047960in}}%
\pgfpathlineto{\pgfqpoint{3.959061in}{3.102898in}}%
\pgfpathlineto{\pgfqpoint{3.902741in}{3.148308in}}%
\pgfpathlineto{\pgfqpoint{3.890924in}{3.140131in}}%
\pgfpathlineto{\pgfqpoint{3.837119in}{3.102898in}}%
\pgfpathlineto{\pgfqpoint{3.822786in}{3.092980in}}%
\pgfpathlineto{\pgfqpoint{3.754649in}{3.050621in}}%
\pgfpathlineto{\pgfqpoint{3.749258in}{3.047960in}}%
\pgfpathlineto{\pgfqpoint{3.686511in}{3.016984in}}%
\pgfpathlineto{\pgfqpoint{3.618374in}{3.009814in}}%
\pgfpathlineto{\pgfqpoint{3.550236in}{3.029818in}}%
\pgfpathlineto{\pgfqpoint{3.488440in}{3.047960in}}%
\pgfpathlineto{\pgfqpoint{3.482099in}{3.049822in}}%
\pgfpathlineto{\pgfqpoint{3.413961in}{3.069826in}}%
\pgfpathlineto{\pgfqpoint{3.345824in}{3.089830in}}%
\pgfpathlineto{\pgfqpoint{3.301309in}{3.102898in}}%
\pgfpathlineto{\pgfqpoint{3.277686in}{3.109834in}}%
\pgfpathlineto{\pgfqpoint{3.209549in}{3.129838in}}%
\pgfpathlineto{\pgfqpoint{3.141411in}{3.149842in}}%
\pgfpathlineto{\pgfqpoint{3.114179in}{3.157836in}}%
\pgfpathlineto{\pgfqpoint{3.073274in}{3.169846in}}%
\pgfpathlineto{\pgfqpoint{3.005136in}{3.189850in}}%
\pgfpathlineto{\pgfqpoint{2.936999in}{3.209854in}}%
\pgfpathlineto{\pgfqpoint{2.927049in}{3.212775in}}%
\pgfpathlineto{\pgfqpoint{2.868861in}{3.212775in}}%
\pgfpathlineto{\pgfqpoint{2.800724in}{3.157836in}}%
\pgfpathlineto{\pgfqpoint{2.732586in}{3.157836in}}%
\pgfpathlineto{\pgfqpoint{2.664449in}{3.157836in}}%
\pgfpathlineto{\pgfqpoint{2.596311in}{3.157836in}}%
\pgfpathlineto{\pgfqpoint{2.528174in}{3.157836in}}%
\pgfpathlineto{\pgfqpoint{2.460036in}{3.102898in}}%
\pgfpathlineto{\pgfqpoint{2.391899in}{3.102898in}}%
\pgfpathlineto{\pgfqpoint{2.323761in}{3.102898in}}%
\pgfpathlineto{\pgfqpoint{2.255624in}{3.102898in}}%
\pgfpathlineto{\pgfqpoint{2.187486in}{3.102898in}}%
\pgfpathlineto{\pgfqpoint{2.119349in}{3.047960in}}%
\pgfpathlineto{\pgfqpoint{2.051211in}{3.047960in}}%
\pgfpathlineto{\pgfqpoint{1.983074in}{3.047960in}}%
\pgfpathlineto{\pgfqpoint{1.975736in}{3.047960in}}%
\pgfpathlineto{\pgfqpoint{1.983074in}{3.045806in}}%
\pgfpathlineto{\pgfqpoint{2.051211in}{3.025802in}}%
\pgfpathlineto{\pgfqpoint{2.119349in}{3.005798in}}%
\pgfpathlineto{\pgfqpoint{2.162866in}{2.993022in}}%
\pgfpathlineto{\pgfqpoint{2.187486in}{2.985794in}}%
\pgfpathlineto{\pgfqpoint{2.255624in}{2.965790in}}%
\pgfpathlineto{\pgfqpoint{2.323761in}{2.945786in}}%
\pgfpathlineto{\pgfqpoint{2.349996in}{2.938084in}}%
\pgfpathlineto{\pgfqpoint{2.391899in}{2.925782in}}%
\pgfpathlineto{\pgfqpoint{2.460036in}{2.905778in}}%
\pgfpathlineto{\pgfqpoint{2.528174in}{2.885774in}}%
\pgfpathlineto{\pgfqpoint{2.537126in}{2.883145in}}%
\pgfpathlineto{\pgfqpoint{2.596311in}{2.865770in}}%
\pgfpathlineto{\pgfqpoint{2.664449in}{2.845766in}}%
\pgfpathlineto{\pgfqpoint{2.724257in}{2.828207in}}%
\pgfpathlineto{\pgfqpoint{2.732586in}{2.825762in}}%
\pgfpathlineto{\pgfqpoint{2.800724in}{2.805758in}}%
\pgfpathlineto{\pgfqpoint{2.868861in}{2.785754in}}%
\pgfpathlineto{\pgfqpoint{2.911387in}{2.773269in}}%
\pgfpathlineto{\pgfqpoint{2.936999in}{2.765750in}}%
\pgfpathlineto{\pgfqpoint{3.005136in}{2.745746in}}%
\pgfpathlineto{\pgfqpoint{3.073274in}{2.725742in}}%
\pgfpathlineto{\pgfqpoint{3.098517in}{2.718331in}}%
\pgfpathlineto{\pgfqpoint{3.141411in}{2.705738in}}%
\pgfpathlineto{\pgfqpoint{3.209549in}{2.685734in}}%
\pgfpathlineto{\pgfqpoint{3.277686in}{2.665730in}}%
\pgfpathlineto{\pgfqpoint{3.285647in}{2.663393in}}%
\pgfpathclose%
\pgfusepath{fill}%
\end{pgfscope}%
\begin{pgfscope}%
\pgfpathrectangle{\pgfqpoint{0.634105in}{0.521603in}}{\pgfqpoint{3.720000in}{3.020000in}} %
\pgfusepath{clip}%
\pgfsetbuttcap%
\pgfsetroundjoin%
\definecolor{currentfill}{rgb}{1.000000,1.000000,1.000000}%
\pgfsetfillcolor{currentfill}%
\pgfsetlinewidth{1.003750pt}%
\definecolor{currentstroke}{rgb}{0.000000,0.000000,0.000000}%
\pgfsetstrokecolor{currentstroke}%
\pgfsetdash{}{0pt}%
\pgfpathmoveto{\pgfqpoint{3.838210in}{3.335922in}}%
\pgfpathcurveto{\pgfqpoint{3.849260in}{3.335922in}}{\pgfqpoint{3.859859in}{3.340313in}}{\pgfqpoint{3.867673in}{3.348126in}}%
\pgfpathcurveto{\pgfqpoint{3.875486in}{3.355940in}}{\pgfqpoint{3.879877in}{3.366539in}}{\pgfqpoint{3.879877in}{3.377589in}}%
\pgfpathcurveto{\pgfqpoint{3.879877in}{3.388639in}}{\pgfqpoint{3.875486in}{3.399238in}}{\pgfqpoint{3.867673in}{3.407052in}}%
\pgfpathcurveto{\pgfqpoint{3.859859in}{3.414866in}}{\pgfqpoint{3.849260in}{3.419256in}}{\pgfqpoint{3.838210in}{3.419256in}}%
\pgfpathcurveto{\pgfqpoint{3.827160in}{3.419256in}}{\pgfqpoint{3.816561in}{3.414866in}}{\pgfqpoint{3.808747in}{3.407052in}}%
\pgfpathcurveto{\pgfqpoint{3.800934in}{3.399238in}}{\pgfqpoint{3.796543in}{3.388639in}}{\pgfqpoint{3.796543in}{3.377589in}}%
\pgfpathcurveto{\pgfqpoint{3.796543in}{3.366539in}}{\pgfqpoint{3.800934in}{3.355940in}}{\pgfqpoint{3.808747in}{3.348126in}}%
\pgfpathcurveto{\pgfqpoint{3.816561in}{3.340313in}}{\pgfqpoint{3.827160in}{3.335922in}}{\pgfqpoint{3.838210in}{3.335922in}}%
\pgfpathclose%
\pgfusepath{stroke,fill}%
\end{pgfscope}%
\begin{pgfscope}%
\pgfpathrectangle{\pgfqpoint{0.634105in}{0.521603in}}{\pgfqpoint{3.720000in}{3.020000in}} %
\pgfusepath{clip}%
\pgfsetbuttcap%
\pgfsetroundjoin%
\definecolor{currentfill}{rgb}{1.000000,1.000000,1.000000}%
\pgfsetfillcolor{currentfill}%
\pgfsetlinewidth{1.003750pt}%
\definecolor{currentstroke}{rgb}{0.000000,0.000000,0.000000}%
\pgfsetstrokecolor{currentstroke}%
\pgfsetdash{}{0pt}%
\pgfpathmoveto{\pgfqpoint{1.015690in}{2.178249in}}%
\pgfpathcurveto{\pgfqpoint{1.026740in}{2.178249in}}{\pgfqpoint{1.037339in}{2.182639in}}{\pgfqpoint{1.045153in}{2.190453in}}%
\pgfpathcurveto{\pgfqpoint{1.052966in}{2.198266in}}{\pgfqpoint{1.057357in}{2.208866in}}{\pgfqpoint{1.057357in}{2.219916in}}%
\pgfpathcurveto{\pgfqpoint{1.057357in}{2.230966in}}{\pgfqpoint{1.052966in}{2.241565in}}{\pgfqpoint{1.045153in}{2.249378in}}%
\pgfpathcurveto{\pgfqpoint{1.037339in}{2.257192in}}{\pgfqpoint{1.026740in}{2.261582in}}{\pgfqpoint{1.015690in}{2.261582in}}%
\pgfpathcurveto{\pgfqpoint{1.004640in}{2.261582in}}{\pgfqpoint{0.994041in}{2.257192in}}{\pgfqpoint{0.986227in}{2.249378in}}%
\pgfpathcurveto{\pgfqpoint{0.978414in}{2.241565in}}{\pgfqpoint{0.974023in}{2.230966in}}{\pgfqpoint{0.974023in}{2.219916in}}%
\pgfpathcurveto{\pgfqpoint{0.974023in}{2.208866in}}{\pgfqpoint{0.978414in}{2.198266in}}{\pgfqpoint{0.986227in}{2.190453in}}%
\pgfpathcurveto{\pgfqpoint{0.994041in}{2.182639in}}{\pgfqpoint{1.004640in}{2.178249in}}{\pgfqpoint{1.015690in}{2.178249in}}%
\pgfpathclose%
\pgfusepath{stroke,fill}%
\end{pgfscope}%
\begin{pgfscope}%
\pgfpathrectangle{\pgfqpoint{0.634105in}{0.521603in}}{\pgfqpoint{3.720000in}{3.020000in}} %
\pgfusepath{clip}%
\pgfsetbuttcap%
\pgfsetroundjoin%
\definecolor{currentfill}{rgb}{1.000000,1.000000,1.000000}%
\pgfsetfillcolor{currentfill}%
\pgfsetlinewidth{1.003750pt}%
\definecolor{currentstroke}{rgb}{0.000000,0.000000,0.000000}%
\pgfsetstrokecolor{currentstroke}%
\pgfsetdash{}{0pt}%
\pgfpathmoveto{\pgfqpoint{0.824736in}{2.177069in}}%
\pgfpathcurveto{\pgfqpoint{0.835786in}{2.177069in}}{\pgfqpoint{0.846385in}{2.181459in}}{\pgfqpoint{0.854199in}{2.189273in}}%
\pgfpathcurveto{\pgfqpoint{0.862012in}{2.197087in}}{\pgfqpoint{0.866403in}{2.207686in}}{\pgfqpoint{0.866403in}{2.218736in}}%
\pgfpathcurveto{\pgfqpoint{0.866403in}{2.229786in}}{\pgfqpoint{0.862012in}{2.240385in}}{\pgfqpoint{0.854199in}{2.248199in}}%
\pgfpathcurveto{\pgfqpoint{0.846385in}{2.256012in}}{\pgfqpoint{0.835786in}{2.260402in}}{\pgfqpoint{0.824736in}{2.260402in}}%
\pgfpathcurveto{\pgfqpoint{0.813686in}{2.260402in}}{\pgfqpoint{0.803087in}{2.256012in}}{\pgfqpoint{0.795273in}{2.248199in}}%
\pgfpathcurveto{\pgfqpoint{0.787460in}{2.240385in}}{\pgfqpoint{0.783069in}{2.229786in}}{\pgfqpoint{0.783069in}{2.218736in}}%
\pgfpathcurveto{\pgfqpoint{0.783069in}{2.207686in}}{\pgfqpoint{0.787460in}{2.197087in}}{\pgfqpoint{0.795273in}{2.189273in}}%
\pgfpathcurveto{\pgfqpoint{0.803087in}{2.181459in}}{\pgfqpoint{0.813686in}{2.177069in}}{\pgfqpoint{0.824736in}{2.177069in}}%
\pgfpathclose%
\pgfusepath{stroke,fill}%
\end{pgfscope}%
\begin{pgfscope}%
\pgfpathrectangle{\pgfqpoint{0.634105in}{0.521603in}}{\pgfqpoint{3.720000in}{3.020000in}} %
\pgfusepath{clip}%
\pgfsetbuttcap%
\pgfsetroundjoin%
\definecolor{currentfill}{rgb}{1.000000,1.000000,1.000000}%
\pgfsetfillcolor{currentfill}%
\pgfsetlinewidth{1.003750pt}%
\definecolor{currentstroke}{rgb}{0.000000,0.000000,0.000000}%
\pgfsetstrokecolor{currentstroke}%
\pgfsetdash{}{0pt}%
\pgfpathmoveto{\pgfqpoint{1.772294in}{1.116526in}}%
\pgfpathcurveto{\pgfqpoint{1.783344in}{1.116526in}}{\pgfqpoint{1.793943in}{1.120916in}}{\pgfqpoint{1.801757in}{1.128730in}}%
\pgfpathcurveto{\pgfqpoint{1.809570in}{1.136543in}}{\pgfqpoint{1.813960in}{1.147142in}}{\pgfqpoint{1.813960in}{1.158192in}}%
\pgfpathcurveto{\pgfqpoint{1.813960in}{1.169243in}}{\pgfqpoint{1.809570in}{1.179842in}}{\pgfqpoint{1.801757in}{1.187655in}}%
\pgfpathcurveto{\pgfqpoint{1.793943in}{1.195469in}}{\pgfqpoint{1.783344in}{1.199859in}}{\pgfqpoint{1.772294in}{1.199859in}}%
\pgfpathcurveto{\pgfqpoint{1.761244in}{1.199859in}}{\pgfqpoint{1.750645in}{1.195469in}}{\pgfqpoint{1.742831in}{1.187655in}}%
\pgfpathcurveto{\pgfqpoint{1.735017in}{1.179842in}}{\pgfqpoint{1.730627in}{1.169243in}}{\pgfqpoint{1.730627in}{1.158192in}}%
\pgfpathcurveto{\pgfqpoint{1.730627in}{1.147142in}}{\pgfqpoint{1.735017in}{1.136543in}}{\pgfqpoint{1.742831in}{1.128730in}}%
\pgfpathcurveto{\pgfqpoint{1.750645in}{1.120916in}}{\pgfqpoint{1.761244in}{1.116526in}}{\pgfqpoint{1.772294in}{1.116526in}}%
\pgfpathclose%
\pgfusepath{stroke,fill}%
\end{pgfscope}%
\begin{pgfscope}%
\pgfpathrectangle{\pgfqpoint{0.634105in}{0.521603in}}{\pgfqpoint{3.720000in}{3.020000in}} %
\pgfusepath{clip}%
\pgfsetbuttcap%
\pgfsetroundjoin%
\definecolor{currentfill}{rgb}{1.000000,1.000000,1.000000}%
\pgfsetfillcolor{currentfill}%
\pgfsetlinewidth{1.003750pt}%
\definecolor{currentstroke}{rgb}{0.000000,0.000000,0.000000}%
\pgfsetstrokecolor{currentstroke}%
\pgfsetdash{}{0pt}%
\pgfpathmoveto{\pgfqpoint{1.017683in}{0.972313in}}%
\pgfpathcurveto{\pgfqpoint{1.028733in}{0.972313in}}{\pgfqpoint{1.039332in}{0.976703in}}{\pgfqpoint{1.047146in}{0.984517in}}%
\pgfpathcurveto{\pgfqpoint{1.054959in}{0.992331in}}{\pgfqpoint{1.059349in}{1.002930in}}{\pgfqpoint{1.059349in}{1.013980in}}%
\pgfpathcurveto{\pgfqpoint{1.059349in}{1.025030in}}{\pgfqpoint{1.054959in}{1.035629in}}{\pgfqpoint{1.047146in}{1.043443in}}%
\pgfpathcurveto{\pgfqpoint{1.039332in}{1.051256in}}{\pgfqpoint{1.028733in}{1.055647in}}{\pgfqpoint{1.017683in}{1.055647in}}%
\pgfpathcurveto{\pgfqpoint{1.006633in}{1.055647in}}{\pgfqpoint{0.996034in}{1.051256in}}{\pgfqpoint{0.988220in}{1.043443in}}%
\pgfpathcurveto{\pgfqpoint{0.980406in}{1.035629in}}{\pgfqpoint{0.976016in}{1.025030in}}{\pgfqpoint{0.976016in}{1.013980in}}%
\pgfpathcurveto{\pgfqpoint{0.976016in}{1.002930in}}{\pgfqpoint{0.980406in}{0.992331in}}{\pgfqpoint{0.988220in}{0.984517in}}%
\pgfpathcurveto{\pgfqpoint{0.996034in}{0.976703in}}{\pgfqpoint{1.006633in}{0.972313in}}{\pgfqpoint{1.017683in}{0.972313in}}%
\pgfpathclose%
\pgfusepath{stroke,fill}%
\end{pgfscope}%
\begin{pgfscope}%
\pgfpathrectangle{\pgfqpoint{0.634105in}{0.521603in}}{\pgfqpoint{3.720000in}{3.020000in}} %
\pgfusepath{clip}%
\pgfsetbuttcap%
\pgfsetroundjoin%
\definecolor{currentfill}{rgb}{1.000000,1.000000,1.000000}%
\pgfsetfillcolor{currentfill}%
\pgfsetlinewidth{1.003750pt}%
\definecolor{currentstroke}{rgb}{0.000000,0.000000,0.000000}%
\pgfsetstrokecolor{currentstroke}%
\pgfsetdash{}{0pt}%
\pgfpathmoveto{\pgfqpoint{1.482493in}{0.751603in}}%
\pgfpathcurveto{\pgfqpoint{1.493543in}{0.751603in}}{\pgfqpoint{1.504142in}{0.755993in}}{\pgfqpoint{1.511956in}{0.763806in}}%
\pgfpathcurveto{\pgfqpoint{1.519770in}{0.771620in}}{\pgfqpoint{1.524160in}{0.782219in}}{\pgfqpoint{1.524160in}{0.793269in}}%
\pgfpathcurveto{\pgfqpoint{1.524160in}{0.804319in}}{\pgfqpoint{1.519770in}{0.814918in}}{\pgfqpoint{1.511956in}{0.822732in}}%
\pgfpathcurveto{\pgfqpoint{1.504142in}{0.830546in}}{\pgfqpoint{1.493543in}{0.834936in}}{\pgfqpoint{1.482493in}{0.834936in}}%
\pgfpathcurveto{\pgfqpoint{1.471443in}{0.834936in}}{\pgfqpoint{1.460844in}{0.830546in}}{\pgfqpoint{1.453030in}{0.822732in}}%
\pgfpathcurveto{\pgfqpoint{1.445217in}{0.814918in}}{\pgfqpoint{1.440826in}{0.804319in}}{\pgfqpoint{1.440826in}{0.793269in}}%
\pgfpathcurveto{\pgfqpoint{1.440826in}{0.782219in}}{\pgfqpoint{1.445217in}{0.771620in}}{\pgfqpoint{1.453030in}{0.763806in}}%
\pgfpathcurveto{\pgfqpoint{1.460844in}{0.755993in}}{\pgfqpoint{1.471443in}{0.751603in}}{\pgfqpoint{1.482493in}{0.751603in}}%
\pgfpathclose%
\pgfusepath{stroke,fill}%
\end{pgfscope}%
\begin{pgfscope}%
\pgfpathrectangle{\pgfqpoint{0.634105in}{0.521603in}}{\pgfqpoint{3.720000in}{3.020000in}} %
\pgfusepath{clip}%
\pgfsetbuttcap%
\pgfsetroundjoin%
\definecolor{currentfill}{rgb}{1.000000,1.000000,1.000000}%
\pgfsetfillcolor{currentfill}%
\pgfsetlinewidth{1.003750pt}%
\definecolor{currentstroke}{rgb}{0.000000,0.000000,0.000000}%
\pgfsetstrokecolor{currentstroke}%
\pgfsetdash{}{0pt}%
\pgfpathmoveto{\pgfqpoint{1.421851in}{0.643951in}}%
\pgfpathcurveto{\pgfqpoint{1.432901in}{0.643951in}}{\pgfqpoint{1.443501in}{0.648341in}}{\pgfqpoint{1.451314in}{0.656155in}}%
\pgfpathcurveto{\pgfqpoint{1.459128in}{0.663968in}}{\pgfqpoint{1.463518in}{0.674567in}}{\pgfqpoint{1.463518in}{0.685618in}}%
\pgfpathcurveto{\pgfqpoint{1.463518in}{0.696668in}}{\pgfqpoint{1.459128in}{0.707267in}}{\pgfqpoint{1.451314in}{0.715080in}}%
\pgfpathcurveto{\pgfqpoint{1.443501in}{0.722894in}}{\pgfqpoint{1.432901in}{0.727284in}}{\pgfqpoint{1.421851in}{0.727284in}}%
\pgfpathcurveto{\pgfqpoint{1.410801in}{0.727284in}}{\pgfqpoint{1.400202in}{0.722894in}}{\pgfqpoint{1.392389in}{0.715080in}}%
\pgfpathcurveto{\pgfqpoint{1.384575in}{0.707267in}}{\pgfqpoint{1.380185in}{0.696668in}}{\pgfqpoint{1.380185in}{0.685618in}}%
\pgfpathcurveto{\pgfqpoint{1.380185in}{0.674567in}}{\pgfqpoint{1.384575in}{0.663968in}}{\pgfqpoint{1.392389in}{0.656155in}}%
\pgfpathcurveto{\pgfqpoint{1.400202in}{0.648341in}}{\pgfqpoint{1.410801in}{0.643951in}}{\pgfqpoint{1.421851in}{0.643951in}}%
\pgfpathclose%
\pgfusepath{stroke,fill}%
\end{pgfscope}%
\begin{pgfscope}%
\pgfpathrectangle{\pgfqpoint{0.634105in}{0.521603in}}{\pgfqpoint{3.720000in}{3.020000in}} %
\pgfusepath{clip}%
\pgfsetbuttcap%
\pgfsetroundjoin%
\definecolor{currentfill}{rgb}{1.000000,1.000000,1.000000}%
\pgfsetfillcolor{currentfill}%
\pgfsetlinewidth{1.003750pt}%
\definecolor{currentstroke}{rgb}{0.000000,0.000000,0.000000}%
\pgfsetstrokecolor{currentstroke}%
\pgfsetdash{}{0pt}%
\pgfpathmoveto{\pgfqpoint{4.163474in}{2.623161in}}%
\pgfpathcurveto{\pgfqpoint{4.174524in}{2.623161in}}{\pgfqpoint{4.185123in}{2.627551in}}{\pgfqpoint{4.192936in}{2.635365in}}%
\pgfpathcurveto{\pgfqpoint{4.200750in}{2.643178in}}{\pgfqpoint{4.205140in}{2.653777in}}{\pgfqpoint{4.205140in}{2.664828in}}%
\pgfpathcurveto{\pgfqpoint{4.205140in}{2.675878in}}{\pgfqpoint{4.200750in}{2.686477in}}{\pgfqpoint{4.192936in}{2.694290in}}%
\pgfpathcurveto{\pgfqpoint{4.185123in}{2.702104in}}{\pgfqpoint{4.174524in}{2.706494in}}{\pgfqpoint{4.163474in}{2.706494in}}%
\pgfpathcurveto{\pgfqpoint{4.152424in}{2.706494in}}{\pgfqpoint{4.141825in}{2.702104in}}{\pgfqpoint{4.134011in}{2.694290in}}%
\pgfpathcurveto{\pgfqpoint{4.126197in}{2.686477in}}{\pgfqpoint{4.121807in}{2.675878in}}{\pgfqpoint{4.121807in}{2.664828in}}%
\pgfpathcurveto{\pgfqpoint{4.121807in}{2.653777in}}{\pgfqpoint{4.126197in}{2.643178in}}{\pgfqpoint{4.134011in}{2.635365in}}%
\pgfpathcurveto{\pgfqpoint{4.141825in}{2.627551in}}{\pgfqpoint{4.152424in}{2.623161in}}{\pgfqpoint{4.163474in}{2.623161in}}%
\pgfpathclose%
\pgfusepath{stroke,fill}%
\end{pgfscope}%
\begin{pgfscope}%
\pgfpathrectangle{\pgfqpoint{0.634105in}{0.521603in}}{\pgfqpoint{3.720000in}{3.020000in}} %
\pgfusepath{clip}%
\pgfsetbuttcap%
\pgfsetroundjoin%
\definecolor{currentfill}{rgb}{1.000000,1.000000,1.000000}%
\pgfsetfillcolor{currentfill}%
\pgfsetlinewidth{1.003750pt}%
\definecolor{currentstroke}{rgb}{0.000000,0.000000,0.000000}%
\pgfsetstrokecolor{currentstroke}%
\pgfsetdash{}{0pt}%
\pgfpathmoveto{\pgfqpoint{3.556105in}{1.839626in}}%
\pgfpathcurveto{\pgfqpoint{3.567155in}{1.839626in}}{\pgfqpoint{3.577754in}{1.844017in}}{\pgfqpoint{3.585568in}{1.851830in}}%
\pgfpathcurveto{\pgfqpoint{3.593381in}{1.859644in}}{\pgfqpoint{3.597772in}{1.870243in}}{\pgfqpoint{3.597772in}{1.881293in}}%
\pgfpathcurveto{\pgfqpoint{3.597772in}{1.892343in}}{\pgfqpoint{3.593381in}{1.902942in}}{\pgfqpoint{3.585568in}{1.910756in}}%
\pgfpathcurveto{\pgfqpoint{3.577754in}{1.918569in}}{\pgfqpoint{3.567155in}{1.922960in}}{\pgfqpoint{3.556105in}{1.922960in}}%
\pgfpathcurveto{\pgfqpoint{3.545055in}{1.922960in}}{\pgfqpoint{3.534456in}{1.918569in}}{\pgfqpoint{3.526642in}{1.910756in}}%
\pgfpathcurveto{\pgfqpoint{3.518829in}{1.902942in}}{\pgfqpoint{3.514438in}{1.892343in}}{\pgfqpoint{3.514438in}{1.881293in}}%
\pgfpathcurveto{\pgfqpoint{3.514438in}{1.870243in}}{\pgfqpoint{3.518829in}{1.859644in}}{\pgfqpoint{3.526642in}{1.851830in}}%
\pgfpathcurveto{\pgfqpoint{3.534456in}{1.844017in}}{\pgfqpoint{3.545055in}{1.839626in}}{\pgfqpoint{3.556105in}{1.839626in}}%
\pgfpathclose%
\pgfusepath{stroke,fill}%
\end{pgfscope}%
\begin{pgfscope}%
\pgfpathrectangle{\pgfqpoint{0.634105in}{0.521603in}}{\pgfqpoint{3.720000in}{3.020000in}} %
\pgfusepath{clip}%
\pgfsetbuttcap%
\pgfsetroundjoin%
\definecolor{currentfill}{rgb}{1.000000,1.000000,1.000000}%
\pgfsetfillcolor{currentfill}%
\pgfsetlinewidth{1.003750pt}%
\definecolor{currentstroke}{rgb}{0.000000,0.000000,0.000000}%
\pgfsetstrokecolor{currentstroke}%
\pgfsetdash{}{0pt}%
\pgfpathmoveto{\pgfqpoint{2.926462in}{1.486911in}}%
\pgfpathcurveto{\pgfqpoint{2.937513in}{1.486911in}}{\pgfqpoint{2.948112in}{1.491301in}}{\pgfqpoint{2.955925in}{1.499115in}}%
\pgfpathcurveto{\pgfqpoint{2.963739in}{1.506928in}}{\pgfqpoint{2.968129in}{1.517527in}}{\pgfqpoint{2.968129in}{1.528577in}}%
\pgfpathcurveto{\pgfqpoint{2.968129in}{1.539628in}}{\pgfqpoint{2.963739in}{1.550227in}}{\pgfqpoint{2.955925in}{1.558040in}}%
\pgfpathcurveto{\pgfqpoint{2.948112in}{1.565854in}}{\pgfqpoint{2.937513in}{1.570244in}}{\pgfqpoint{2.926462in}{1.570244in}}%
\pgfpathcurveto{\pgfqpoint{2.915412in}{1.570244in}}{\pgfqpoint{2.904813in}{1.565854in}}{\pgfqpoint{2.897000in}{1.558040in}}%
\pgfpathcurveto{\pgfqpoint{2.889186in}{1.550227in}}{\pgfqpoint{2.884796in}{1.539628in}}{\pgfqpoint{2.884796in}{1.528577in}}%
\pgfpathcurveto{\pgfqpoint{2.884796in}{1.517527in}}{\pgfqpoint{2.889186in}{1.506928in}}{\pgfqpoint{2.897000in}{1.499115in}}%
\pgfpathcurveto{\pgfqpoint{2.904813in}{1.491301in}}{\pgfqpoint{2.915412in}{1.486911in}}{\pgfqpoint{2.926462in}{1.486911in}}%
\pgfpathclose%
\pgfusepath{stroke,fill}%
\end{pgfscope}%
\begin{pgfscope}%
\pgfpathrectangle{\pgfqpoint{0.634105in}{0.521603in}}{\pgfqpoint{3.720000in}{3.020000in}} %
\pgfusepath{clip}%
\pgfsetbuttcap%
\pgfsetroundjoin%
\definecolor{currentfill}{rgb}{1.000000,1.000000,1.000000}%
\pgfsetfillcolor{currentfill}%
\pgfsetlinewidth{1.003750pt}%
\definecolor{currentstroke}{rgb}{0.000000,0.000000,0.000000}%
\pgfsetstrokecolor{currentstroke}%
\pgfsetdash{}{0pt}%
\pgfpathmoveto{\pgfqpoint{1.190499in}{2.912413in}}%
\pgfpathcurveto{\pgfqpoint{1.201549in}{2.912413in}}{\pgfqpoint{1.212148in}{2.916804in}}{\pgfqpoint{1.219962in}{2.924617in}}%
\pgfpathcurveto{\pgfqpoint{1.227775in}{2.932431in}}{\pgfqpoint{1.232166in}{2.943030in}}{\pgfqpoint{1.232166in}{2.954080in}}%
\pgfpathcurveto{\pgfqpoint{1.232166in}{2.965130in}}{\pgfqpoint{1.227775in}{2.975729in}}{\pgfqpoint{1.219962in}{2.983543in}}%
\pgfpathcurveto{\pgfqpoint{1.212148in}{2.991356in}}{\pgfqpoint{1.201549in}{2.995747in}}{\pgfqpoint{1.190499in}{2.995747in}}%
\pgfpathcurveto{\pgfqpoint{1.179449in}{2.995747in}}{\pgfqpoint{1.168850in}{2.991356in}}{\pgfqpoint{1.161036in}{2.983543in}}%
\pgfpathcurveto{\pgfqpoint{1.153222in}{2.975729in}}{\pgfqpoint{1.148832in}{2.965130in}}{\pgfqpoint{1.148832in}{2.954080in}}%
\pgfpathcurveto{\pgfqpoint{1.148832in}{2.943030in}}{\pgfqpoint{1.153222in}{2.932431in}}{\pgfqpoint{1.161036in}{2.924617in}}%
\pgfpathcurveto{\pgfqpoint{1.168850in}{2.916804in}}{\pgfqpoint{1.179449in}{2.912413in}}{\pgfqpoint{1.190499in}{2.912413in}}%
\pgfpathclose%
\pgfusepath{stroke,fill}%
\end{pgfscope}%
\begin{pgfscope}%
\pgfpathrectangle{\pgfqpoint{0.634105in}{0.521603in}}{\pgfqpoint{3.720000in}{3.020000in}} %
\pgfusepath{clip}%
\pgfsetbuttcap%
\pgfsetroundjoin%
\definecolor{currentfill}{rgb}{1.000000,1.000000,1.000000}%
\pgfsetfillcolor{currentfill}%
\pgfsetlinewidth{1.003750pt}%
\definecolor{currentstroke}{rgb}{0.000000,0.000000,0.000000}%
\pgfsetstrokecolor{currentstroke}%
\pgfsetdash{}{0pt}%
\pgfpathmoveto{\pgfqpoint{2.312047in}{1.118990in}}%
\pgfpathcurveto{\pgfqpoint{2.323097in}{1.118990in}}{\pgfqpoint{2.333696in}{1.123380in}}{\pgfqpoint{2.341510in}{1.131194in}}%
\pgfpathcurveto{\pgfqpoint{2.349323in}{1.139007in}}{\pgfqpoint{2.353714in}{1.149606in}}{\pgfqpoint{2.353714in}{1.160656in}}%
\pgfpathcurveto{\pgfqpoint{2.353714in}{1.171707in}}{\pgfqpoint{2.349323in}{1.182306in}}{\pgfqpoint{2.341510in}{1.190119in}}%
\pgfpathcurveto{\pgfqpoint{2.333696in}{1.197933in}}{\pgfqpoint{2.323097in}{1.202323in}}{\pgfqpoint{2.312047in}{1.202323in}}%
\pgfpathcurveto{\pgfqpoint{2.300997in}{1.202323in}}{\pgfqpoint{2.290398in}{1.197933in}}{\pgfqpoint{2.282584in}{1.190119in}}%
\pgfpathcurveto{\pgfqpoint{2.274770in}{1.182306in}}{\pgfqpoint{2.270380in}{1.171707in}}{\pgfqpoint{2.270380in}{1.160656in}}%
\pgfpathcurveto{\pgfqpoint{2.270380in}{1.149606in}}{\pgfqpoint{2.274770in}{1.139007in}}{\pgfqpoint{2.282584in}{1.131194in}}%
\pgfpathcurveto{\pgfqpoint{2.290398in}{1.123380in}}{\pgfqpoint{2.300997in}{1.118990in}}{\pgfqpoint{2.312047in}{1.118990in}}%
\pgfpathclose%
\pgfusepath{stroke,fill}%
\end{pgfscope}%
\begin{pgfscope}%
\pgfpathrectangle{\pgfqpoint{0.634105in}{0.521603in}}{\pgfqpoint{3.720000in}{3.020000in}} %
\pgfusepath{clip}%
\pgfsetbuttcap%
\pgfsetroundjoin%
\definecolor{currentfill}{rgb}{1.000000,1.000000,1.000000}%
\pgfsetfillcolor{currentfill}%
\pgfsetlinewidth{1.003750pt}%
\definecolor{currentstroke}{rgb}{0.000000,0.000000,0.000000}%
\pgfsetstrokecolor{currentstroke}%
\pgfsetdash{}{0pt}%
\pgfpathmoveto{\pgfqpoint{2.024287in}{1.311281in}}%
\pgfpathcurveto{\pgfqpoint{2.035338in}{1.311281in}}{\pgfqpoint{2.045937in}{1.315671in}}{\pgfqpoint{2.053750in}{1.323484in}}%
\pgfpathcurveto{\pgfqpoint{2.061564in}{1.331298in}}{\pgfqpoint{2.065954in}{1.341897in}}{\pgfqpoint{2.065954in}{1.352947in}}%
\pgfpathcurveto{\pgfqpoint{2.065954in}{1.363997in}}{\pgfqpoint{2.061564in}{1.374596in}}{\pgfqpoint{2.053750in}{1.382410in}}%
\pgfpathcurveto{\pgfqpoint{2.045937in}{1.390224in}}{\pgfqpoint{2.035338in}{1.394614in}}{\pgfqpoint{2.024287in}{1.394614in}}%
\pgfpathcurveto{\pgfqpoint{2.013237in}{1.394614in}}{\pgfqpoint{2.002638in}{1.390224in}}{\pgfqpoint{1.994825in}{1.382410in}}%
\pgfpathcurveto{\pgfqpoint{1.987011in}{1.374596in}}{\pgfqpoint{1.982621in}{1.363997in}}{\pgfqpoint{1.982621in}{1.352947in}}%
\pgfpathcurveto{\pgfqpoint{1.982621in}{1.341897in}}{\pgfqpoint{1.987011in}{1.331298in}}{\pgfqpoint{1.994825in}{1.323484in}}%
\pgfpathcurveto{\pgfqpoint{2.002638in}{1.315671in}}{\pgfqpoint{2.013237in}{1.311281in}}{\pgfqpoint{2.024287in}{1.311281in}}%
\pgfpathclose%
\pgfusepath{stroke,fill}%
\end{pgfscope}%
\begin{pgfscope}%
\pgfpathrectangle{\pgfqpoint{0.634105in}{0.521603in}}{\pgfqpoint{3.720000in}{3.020000in}} %
\pgfusepath{clip}%
\pgfsetbuttcap%
\pgfsetroundjoin%
\definecolor{currentfill}{rgb}{1.000000,1.000000,1.000000}%
\pgfsetfillcolor{currentfill}%
\pgfsetlinewidth{1.003750pt}%
\definecolor{currentstroke}{rgb}{0.000000,0.000000,0.000000}%
\pgfsetstrokecolor{currentstroke}%
\pgfsetdash{}{0pt}%
\pgfpathmoveto{\pgfqpoint{1.154746in}{2.170547in}}%
\pgfpathcurveto{\pgfqpoint{1.165796in}{2.170547in}}{\pgfqpoint{1.176395in}{2.174937in}}{\pgfqpoint{1.184209in}{2.182751in}}%
\pgfpathcurveto{\pgfqpoint{1.192022in}{2.190564in}}{\pgfqpoint{1.196413in}{2.201163in}}{\pgfqpoint{1.196413in}{2.212213in}}%
\pgfpathcurveto{\pgfqpoint{1.196413in}{2.223264in}}{\pgfqpoint{1.192022in}{2.233863in}}{\pgfqpoint{1.184209in}{2.241676in}}%
\pgfpathcurveto{\pgfqpoint{1.176395in}{2.249490in}}{\pgfqpoint{1.165796in}{2.253880in}}{\pgfqpoint{1.154746in}{2.253880in}}%
\pgfpathcurveto{\pgfqpoint{1.143696in}{2.253880in}}{\pgfqpoint{1.133097in}{2.249490in}}{\pgfqpoint{1.125283in}{2.241676in}}%
\pgfpathcurveto{\pgfqpoint{1.117470in}{2.233863in}}{\pgfqpoint{1.113079in}{2.223264in}}{\pgfqpoint{1.113079in}{2.212213in}}%
\pgfpathcurveto{\pgfqpoint{1.113079in}{2.201163in}}{\pgfqpoint{1.117470in}{2.190564in}}{\pgfqpoint{1.125283in}{2.182751in}}%
\pgfpathcurveto{\pgfqpoint{1.133097in}{2.174937in}}{\pgfqpoint{1.143696in}{2.170547in}}{\pgfqpoint{1.154746in}{2.170547in}}%
\pgfpathclose%
\pgfusepath{stroke,fill}%
\end{pgfscope}%
\begin{pgfscope}%
\pgfpathrectangle{\pgfqpoint{0.634105in}{0.521603in}}{\pgfqpoint{3.720000in}{3.020000in}} %
\pgfusepath{clip}%
\pgfsetbuttcap%
\pgfsetroundjoin%
\definecolor{currentfill}{rgb}{1.000000,1.000000,1.000000}%
\pgfsetfillcolor{currentfill}%
\pgfsetlinewidth{1.003750pt}%
\definecolor{currentstroke}{rgb}{0.000000,0.000000,0.000000}%
\pgfsetstrokecolor{currentstroke}%
\pgfsetdash{}{0pt}%
\pgfpathmoveto{\pgfqpoint{2.182791in}{0.747687in}}%
\pgfpathcurveto{\pgfqpoint{2.193841in}{0.747687in}}{\pgfqpoint{2.204440in}{0.752077in}}{\pgfqpoint{2.212254in}{0.759891in}}%
\pgfpathcurveto{\pgfqpoint{2.220067in}{0.767704in}}{\pgfqpoint{2.224457in}{0.778303in}}{\pgfqpoint{2.224457in}{0.789353in}}%
\pgfpathcurveto{\pgfqpoint{2.224457in}{0.800403in}}{\pgfqpoint{2.220067in}{0.811002in}}{\pgfqpoint{2.212254in}{0.818816in}}%
\pgfpathcurveto{\pgfqpoint{2.204440in}{0.826630in}}{\pgfqpoint{2.193841in}{0.831020in}}{\pgfqpoint{2.182791in}{0.831020in}}%
\pgfpathcurveto{\pgfqpoint{2.171741in}{0.831020in}}{\pgfqpoint{2.161142in}{0.826630in}}{\pgfqpoint{2.153328in}{0.818816in}}%
\pgfpathcurveto{\pgfqpoint{2.145514in}{0.811002in}}{\pgfqpoint{2.141124in}{0.800403in}}{\pgfqpoint{2.141124in}{0.789353in}}%
\pgfpathcurveto{\pgfqpoint{2.141124in}{0.778303in}}{\pgfqpoint{2.145514in}{0.767704in}}{\pgfqpoint{2.153328in}{0.759891in}}%
\pgfpathcurveto{\pgfqpoint{2.161142in}{0.752077in}}{\pgfqpoint{2.171741in}{0.747687in}}{\pgfqpoint{2.182791in}{0.747687in}}%
\pgfpathclose%
\pgfusepath{stroke,fill}%
\end{pgfscope}%
\begin{pgfscope}%
\pgfsetbuttcap%
\pgfsetroundjoin%
\definecolor{currentfill}{rgb}{0.000000,0.000000,0.000000}%
\pgfsetfillcolor{currentfill}%
\pgfsetlinewidth{0.803000pt}%
\definecolor{currentstroke}{rgb}{0.000000,0.000000,0.000000}%
\pgfsetstrokecolor{currentstroke}%
\pgfsetdash{}{0pt}%
\pgfsys@defobject{currentmarker}{\pgfqpoint{0.000000in}{-0.048611in}}{\pgfqpoint{0.000000in}{0.000000in}}{%
\pgfpathmoveto{\pgfqpoint{0.000000in}{0.000000in}}%
\pgfpathlineto{\pgfqpoint{0.000000in}{-0.048611in}}%
\pgfusepath{stroke,fill}%
}%
\begin{pgfscope}%
\pgfsys@transformshift{0.674322in}{0.521603in}%
\pgfsys@useobject{currentmarker}{}%
\end{pgfscope}%
\end{pgfscope}%
\begin{pgfscope}%
\pgftext[x=0.674322in,y=0.424381in,,top]{\rmfamily\fontsize{10.000000}{12.000000}\selectfont \(\displaystyle 0.4\)}%
\end{pgfscope}%
\begin{pgfscope}%
\pgfsetbuttcap%
\pgfsetroundjoin%
\definecolor{currentfill}{rgb}{0.000000,0.000000,0.000000}%
\pgfsetfillcolor{currentfill}%
\pgfsetlinewidth{0.803000pt}%
\definecolor{currentstroke}{rgb}{0.000000,0.000000,0.000000}%
\pgfsetstrokecolor{currentstroke}%
\pgfsetdash{}{0pt}%
\pgfsys@defobject{currentmarker}{\pgfqpoint{0.000000in}{-0.048611in}}{\pgfqpoint{0.000000in}{0.000000in}}{%
\pgfpathmoveto{\pgfqpoint{0.000000in}{0.000000in}}%
\pgfpathlineto{\pgfqpoint{0.000000in}{-0.048611in}}%
\pgfusepath{stroke,fill}%
}%
\begin{pgfscope}%
\pgfsys@transformshift{1.156200in}{0.521603in}%
\pgfsys@useobject{currentmarker}{}%
\end{pgfscope}%
\end{pgfscope}%
\begin{pgfscope}%
\pgftext[x=1.156200in,y=0.424381in,,top]{\rmfamily\fontsize{10.000000}{12.000000}\selectfont \(\displaystyle 0.6\)}%
\end{pgfscope}%
\begin{pgfscope}%
\pgfsetbuttcap%
\pgfsetroundjoin%
\definecolor{currentfill}{rgb}{0.000000,0.000000,0.000000}%
\pgfsetfillcolor{currentfill}%
\pgfsetlinewidth{0.803000pt}%
\definecolor{currentstroke}{rgb}{0.000000,0.000000,0.000000}%
\pgfsetstrokecolor{currentstroke}%
\pgfsetdash{}{0pt}%
\pgfsys@defobject{currentmarker}{\pgfqpoint{0.000000in}{-0.048611in}}{\pgfqpoint{0.000000in}{0.000000in}}{%
\pgfpathmoveto{\pgfqpoint{0.000000in}{0.000000in}}%
\pgfpathlineto{\pgfqpoint{0.000000in}{-0.048611in}}%
\pgfusepath{stroke,fill}%
}%
\begin{pgfscope}%
\pgfsys@transformshift{1.638079in}{0.521603in}%
\pgfsys@useobject{currentmarker}{}%
\end{pgfscope}%
\end{pgfscope}%
\begin{pgfscope}%
\pgftext[x=1.638079in,y=0.424381in,,top]{\rmfamily\fontsize{10.000000}{12.000000}\selectfont \(\displaystyle 0.8\)}%
\end{pgfscope}%
\begin{pgfscope}%
\pgfsetbuttcap%
\pgfsetroundjoin%
\definecolor{currentfill}{rgb}{0.000000,0.000000,0.000000}%
\pgfsetfillcolor{currentfill}%
\pgfsetlinewidth{0.803000pt}%
\definecolor{currentstroke}{rgb}{0.000000,0.000000,0.000000}%
\pgfsetstrokecolor{currentstroke}%
\pgfsetdash{}{0pt}%
\pgfsys@defobject{currentmarker}{\pgfqpoint{0.000000in}{-0.048611in}}{\pgfqpoint{0.000000in}{0.000000in}}{%
\pgfpathmoveto{\pgfqpoint{0.000000in}{0.000000in}}%
\pgfpathlineto{\pgfqpoint{0.000000in}{-0.048611in}}%
\pgfusepath{stroke,fill}%
}%
\begin{pgfscope}%
\pgfsys@transformshift{2.119958in}{0.521603in}%
\pgfsys@useobject{currentmarker}{}%
\end{pgfscope}%
\end{pgfscope}%
\begin{pgfscope}%
\pgftext[x=2.119958in,y=0.424381in,,top]{\rmfamily\fontsize{10.000000}{12.000000}\selectfont \(\displaystyle 1.0\)}%
\end{pgfscope}%
\begin{pgfscope}%
\pgfsetbuttcap%
\pgfsetroundjoin%
\definecolor{currentfill}{rgb}{0.000000,0.000000,0.000000}%
\pgfsetfillcolor{currentfill}%
\pgfsetlinewidth{0.803000pt}%
\definecolor{currentstroke}{rgb}{0.000000,0.000000,0.000000}%
\pgfsetstrokecolor{currentstroke}%
\pgfsetdash{}{0pt}%
\pgfsys@defobject{currentmarker}{\pgfqpoint{0.000000in}{-0.048611in}}{\pgfqpoint{0.000000in}{0.000000in}}{%
\pgfpathmoveto{\pgfqpoint{0.000000in}{0.000000in}}%
\pgfpathlineto{\pgfqpoint{0.000000in}{-0.048611in}}%
\pgfusepath{stroke,fill}%
}%
\begin{pgfscope}%
\pgfsys@transformshift{2.601837in}{0.521603in}%
\pgfsys@useobject{currentmarker}{}%
\end{pgfscope}%
\end{pgfscope}%
\begin{pgfscope}%
\pgftext[x=2.601837in,y=0.424381in,,top]{\rmfamily\fontsize{10.000000}{12.000000}\selectfont \(\displaystyle 1.2\)}%
\end{pgfscope}%
\begin{pgfscope}%
\pgfsetbuttcap%
\pgfsetroundjoin%
\definecolor{currentfill}{rgb}{0.000000,0.000000,0.000000}%
\pgfsetfillcolor{currentfill}%
\pgfsetlinewidth{0.803000pt}%
\definecolor{currentstroke}{rgb}{0.000000,0.000000,0.000000}%
\pgfsetstrokecolor{currentstroke}%
\pgfsetdash{}{0pt}%
\pgfsys@defobject{currentmarker}{\pgfqpoint{0.000000in}{-0.048611in}}{\pgfqpoint{0.000000in}{0.000000in}}{%
\pgfpathmoveto{\pgfqpoint{0.000000in}{0.000000in}}%
\pgfpathlineto{\pgfqpoint{0.000000in}{-0.048611in}}%
\pgfusepath{stroke,fill}%
}%
\begin{pgfscope}%
\pgfsys@transformshift{3.083716in}{0.521603in}%
\pgfsys@useobject{currentmarker}{}%
\end{pgfscope}%
\end{pgfscope}%
\begin{pgfscope}%
\pgftext[x=3.083716in,y=0.424381in,,top]{\rmfamily\fontsize{10.000000}{12.000000}\selectfont \(\displaystyle 1.4\)}%
\end{pgfscope}%
\begin{pgfscope}%
\pgfsetbuttcap%
\pgfsetroundjoin%
\definecolor{currentfill}{rgb}{0.000000,0.000000,0.000000}%
\pgfsetfillcolor{currentfill}%
\pgfsetlinewidth{0.803000pt}%
\definecolor{currentstroke}{rgb}{0.000000,0.000000,0.000000}%
\pgfsetstrokecolor{currentstroke}%
\pgfsetdash{}{0pt}%
\pgfsys@defobject{currentmarker}{\pgfqpoint{0.000000in}{-0.048611in}}{\pgfqpoint{0.000000in}{0.000000in}}{%
\pgfpathmoveto{\pgfqpoint{0.000000in}{0.000000in}}%
\pgfpathlineto{\pgfqpoint{0.000000in}{-0.048611in}}%
\pgfusepath{stroke,fill}%
}%
\begin{pgfscope}%
\pgfsys@transformshift{3.565595in}{0.521603in}%
\pgfsys@useobject{currentmarker}{}%
\end{pgfscope}%
\end{pgfscope}%
\begin{pgfscope}%
\pgftext[x=3.565595in,y=0.424381in,,top]{\rmfamily\fontsize{10.000000}{12.000000}\selectfont \(\displaystyle 1.6\)}%
\end{pgfscope}%
\begin{pgfscope}%
\pgfsetbuttcap%
\pgfsetroundjoin%
\definecolor{currentfill}{rgb}{0.000000,0.000000,0.000000}%
\pgfsetfillcolor{currentfill}%
\pgfsetlinewidth{0.803000pt}%
\definecolor{currentstroke}{rgb}{0.000000,0.000000,0.000000}%
\pgfsetstrokecolor{currentstroke}%
\pgfsetdash{}{0pt}%
\pgfsys@defobject{currentmarker}{\pgfqpoint{0.000000in}{-0.048611in}}{\pgfqpoint{0.000000in}{0.000000in}}{%
\pgfpathmoveto{\pgfqpoint{0.000000in}{0.000000in}}%
\pgfpathlineto{\pgfqpoint{0.000000in}{-0.048611in}}%
\pgfusepath{stroke,fill}%
}%
\begin{pgfscope}%
\pgfsys@transformshift{4.047474in}{0.521603in}%
\pgfsys@useobject{currentmarker}{}%
\end{pgfscope}%
\end{pgfscope}%
\begin{pgfscope}%
\pgftext[x=4.047474in,y=0.424381in,,top]{\rmfamily\fontsize{10.000000}{12.000000}\selectfont \(\displaystyle 1.8\)}%
\end{pgfscope}%
\begin{pgfscope}%
\pgftext[x=2.494105in,y=0.234413in,,top]{\rmfamily\fontsize{10.000000}{12.000000}\selectfont \(\displaystyle \varphi_s\) (kV)}%
\end{pgfscope}%
\begin{pgfscope}%
\pgfsetbuttcap%
\pgfsetroundjoin%
\definecolor{currentfill}{rgb}{0.000000,0.000000,0.000000}%
\pgfsetfillcolor{currentfill}%
\pgfsetlinewidth{0.803000pt}%
\definecolor{currentstroke}{rgb}{0.000000,0.000000,0.000000}%
\pgfsetstrokecolor{currentstroke}%
\pgfsetdash{}{0pt}%
\pgfsys@defobject{currentmarker}{\pgfqpoint{-0.048611in}{0.000000in}}{\pgfqpoint{0.000000in}{0.000000in}}{%
\pgfpathmoveto{\pgfqpoint{0.000000in}{0.000000in}}%
\pgfpathlineto{\pgfqpoint{-0.048611in}{0.000000in}}%
\pgfusepath{stroke,fill}%
}%
\begin{pgfscope}%
\pgfsys@transformshift{0.634105in}{0.571135in}%
\pgfsys@useobject{currentmarker}{}%
\end{pgfscope}%
\end{pgfscope}%
\begin{pgfscope}%
\pgftext[x=0.289968in,y=0.518373in,left,base]{\rmfamily\fontsize{10.000000}{12.000000}\selectfont \(\displaystyle 0.00\)}%
\end{pgfscope}%
\begin{pgfscope}%
\pgfsetbuttcap%
\pgfsetroundjoin%
\definecolor{currentfill}{rgb}{0.000000,0.000000,0.000000}%
\pgfsetfillcolor{currentfill}%
\pgfsetlinewidth{0.803000pt}%
\definecolor{currentstroke}{rgb}{0.000000,0.000000,0.000000}%
\pgfsetstrokecolor{currentstroke}%
\pgfsetdash{}{0pt}%
\pgfsys@defobject{currentmarker}{\pgfqpoint{-0.048611in}{0.000000in}}{\pgfqpoint{0.000000in}{0.000000in}}{%
\pgfpathmoveto{\pgfqpoint{0.000000in}{0.000000in}}%
\pgfpathlineto{\pgfqpoint{-0.048611in}{0.000000in}}%
\pgfusepath{stroke,fill}%
}%
\begin{pgfscope}%
\pgfsys@transformshift{0.634105in}{0.942149in}%
\pgfsys@useobject{currentmarker}{}%
\end{pgfscope}%
\end{pgfscope}%
\begin{pgfscope}%
\pgftext[x=0.289968in,y=0.889387in,left,base]{\rmfamily\fontsize{10.000000}{12.000000}\selectfont \(\displaystyle 0.05\)}%
\end{pgfscope}%
\begin{pgfscope}%
\pgfsetbuttcap%
\pgfsetroundjoin%
\definecolor{currentfill}{rgb}{0.000000,0.000000,0.000000}%
\pgfsetfillcolor{currentfill}%
\pgfsetlinewidth{0.803000pt}%
\definecolor{currentstroke}{rgb}{0.000000,0.000000,0.000000}%
\pgfsetstrokecolor{currentstroke}%
\pgfsetdash{}{0pt}%
\pgfsys@defobject{currentmarker}{\pgfqpoint{-0.048611in}{0.000000in}}{\pgfqpoint{0.000000in}{0.000000in}}{%
\pgfpathmoveto{\pgfqpoint{0.000000in}{0.000000in}}%
\pgfpathlineto{\pgfqpoint{-0.048611in}{0.000000in}}%
\pgfusepath{stroke,fill}%
}%
\begin{pgfscope}%
\pgfsys@transformshift{0.634105in}{1.313162in}%
\pgfsys@useobject{currentmarker}{}%
\end{pgfscope}%
\end{pgfscope}%
\begin{pgfscope}%
\pgftext[x=0.289968in,y=1.260401in,left,base]{\rmfamily\fontsize{10.000000}{12.000000}\selectfont \(\displaystyle 0.10\)}%
\end{pgfscope}%
\begin{pgfscope}%
\pgfsetbuttcap%
\pgfsetroundjoin%
\definecolor{currentfill}{rgb}{0.000000,0.000000,0.000000}%
\pgfsetfillcolor{currentfill}%
\pgfsetlinewidth{0.803000pt}%
\definecolor{currentstroke}{rgb}{0.000000,0.000000,0.000000}%
\pgfsetstrokecolor{currentstroke}%
\pgfsetdash{}{0pt}%
\pgfsys@defobject{currentmarker}{\pgfqpoint{-0.048611in}{0.000000in}}{\pgfqpoint{0.000000in}{0.000000in}}{%
\pgfpathmoveto{\pgfqpoint{0.000000in}{0.000000in}}%
\pgfpathlineto{\pgfqpoint{-0.048611in}{0.000000in}}%
\pgfusepath{stroke,fill}%
}%
\begin{pgfscope}%
\pgfsys@transformshift{0.634105in}{1.684176in}%
\pgfsys@useobject{currentmarker}{}%
\end{pgfscope}%
\end{pgfscope}%
\begin{pgfscope}%
\pgftext[x=0.289968in,y=1.631415in,left,base]{\rmfamily\fontsize{10.000000}{12.000000}\selectfont \(\displaystyle 0.15\)}%
\end{pgfscope}%
\begin{pgfscope}%
\pgfsetbuttcap%
\pgfsetroundjoin%
\definecolor{currentfill}{rgb}{0.000000,0.000000,0.000000}%
\pgfsetfillcolor{currentfill}%
\pgfsetlinewidth{0.803000pt}%
\definecolor{currentstroke}{rgb}{0.000000,0.000000,0.000000}%
\pgfsetstrokecolor{currentstroke}%
\pgfsetdash{}{0pt}%
\pgfsys@defobject{currentmarker}{\pgfqpoint{-0.048611in}{0.000000in}}{\pgfqpoint{0.000000in}{0.000000in}}{%
\pgfpathmoveto{\pgfqpoint{0.000000in}{0.000000in}}%
\pgfpathlineto{\pgfqpoint{-0.048611in}{0.000000in}}%
\pgfusepath{stroke,fill}%
}%
\begin{pgfscope}%
\pgfsys@transformshift{0.634105in}{2.055190in}%
\pgfsys@useobject{currentmarker}{}%
\end{pgfscope}%
\end{pgfscope}%
\begin{pgfscope}%
\pgftext[x=0.289968in,y=2.002429in,left,base]{\rmfamily\fontsize{10.000000}{12.000000}\selectfont \(\displaystyle 0.20\)}%
\end{pgfscope}%
\begin{pgfscope}%
\pgfsetbuttcap%
\pgfsetroundjoin%
\definecolor{currentfill}{rgb}{0.000000,0.000000,0.000000}%
\pgfsetfillcolor{currentfill}%
\pgfsetlinewidth{0.803000pt}%
\definecolor{currentstroke}{rgb}{0.000000,0.000000,0.000000}%
\pgfsetstrokecolor{currentstroke}%
\pgfsetdash{}{0pt}%
\pgfsys@defobject{currentmarker}{\pgfqpoint{-0.048611in}{0.000000in}}{\pgfqpoint{0.000000in}{0.000000in}}{%
\pgfpathmoveto{\pgfqpoint{0.000000in}{0.000000in}}%
\pgfpathlineto{\pgfqpoint{-0.048611in}{0.000000in}}%
\pgfusepath{stroke,fill}%
}%
\begin{pgfscope}%
\pgfsys@transformshift{0.634105in}{2.426204in}%
\pgfsys@useobject{currentmarker}{}%
\end{pgfscope}%
\end{pgfscope}%
\begin{pgfscope}%
\pgftext[x=0.289968in,y=2.373443in,left,base]{\rmfamily\fontsize{10.000000}{12.000000}\selectfont \(\displaystyle 0.25\)}%
\end{pgfscope}%
\begin{pgfscope}%
\pgfsetbuttcap%
\pgfsetroundjoin%
\definecolor{currentfill}{rgb}{0.000000,0.000000,0.000000}%
\pgfsetfillcolor{currentfill}%
\pgfsetlinewidth{0.803000pt}%
\definecolor{currentstroke}{rgb}{0.000000,0.000000,0.000000}%
\pgfsetstrokecolor{currentstroke}%
\pgfsetdash{}{0pt}%
\pgfsys@defobject{currentmarker}{\pgfqpoint{-0.048611in}{0.000000in}}{\pgfqpoint{0.000000in}{0.000000in}}{%
\pgfpathmoveto{\pgfqpoint{0.000000in}{0.000000in}}%
\pgfpathlineto{\pgfqpoint{-0.048611in}{0.000000in}}%
\pgfusepath{stroke,fill}%
}%
\begin{pgfscope}%
\pgfsys@transformshift{0.634105in}{2.797218in}%
\pgfsys@useobject{currentmarker}{}%
\end{pgfscope}%
\end{pgfscope}%
\begin{pgfscope}%
\pgftext[x=0.289968in,y=2.744457in,left,base]{\rmfamily\fontsize{10.000000}{12.000000}\selectfont \(\displaystyle 0.30\)}%
\end{pgfscope}%
\begin{pgfscope}%
\pgfsetbuttcap%
\pgfsetroundjoin%
\definecolor{currentfill}{rgb}{0.000000,0.000000,0.000000}%
\pgfsetfillcolor{currentfill}%
\pgfsetlinewidth{0.803000pt}%
\definecolor{currentstroke}{rgb}{0.000000,0.000000,0.000000}%
\pgfsetstrokecolor{currentstroke}%
\pgfsetdash{}{0pt}%
\pgfsys@defobject{currentmarker}{\pgfqpoint{-0.048611in}{0.000000in}}{\pgfqpoint{0.000000in}{0.000000in}}{%
\pgfpathmoveto{\pgfqpoint{0.000000in}{0.000000in}}%
\pgfpathlineto{\pgfqpoint{-0.048611in}{0.000000in}}%
\pgfusepath{stroke,fill}%
}%
\begin{pgfscope}%
\pgfsys@transformshift{0.634105in}{3.168232in}%
\pgfsys@useobject{currentmarker}{}%
\end{pgfscope}%
\end{pgfscope}%
\begin{pgfscope}%
\pgftext[x=0.289968in,y=3.115470in,left,base]{\rmfamily\fontsize{10.000000}{12.000000}\selectfont \(\displaystyle 0.35\)}%
\end{pgfscope}%
\begin{pgfscope}%
\pgfsetbuttcap%
\pgfsetroundjoin%
\definecolor{currentfill}{rgb}{0.000000,0.000000,0.000000}%
\pgfsetfillcolor{currentfill}%
\pgfsetlinewidth{0.803000pt}%
\definecolor{currentstroke}{rgb}{0.000000,0.000000,0.000000}%
\pgfsetstrokecolor{currentstroke}%
\pgfsetdash{}{0pt}%
\pgfsys@defobject{currentmarker}{\pgfqpoint{-0.048611in}{0.000000in}}{\pgfqpoint{0.000000in}{0.000000in}}{%
\pgfpathmoveto{\pgfqpoint{0.000000in}{0.000000in}}%
\pgfpathlineto{\pgfqpoint{-0.048611in}{0.000000in}}%
\pgfusepath{stroke,fill}%
}%
\begin{pgfscope}%
\pgfsys@transformshift{0.634105in}{3.539246in}%
\pgfsys@useobject{currentmarker}{}%
\end{pgfscope}%
\end{pgfscope}%
\begin{pgfscope}%
\pgftext[x=0.289968in,y=3.486484in,left,base]{\rmfamily\fontsize{10.000000}{12.000000}\selectfont \(\displaystyle 0.40\)}%
\end{pgfscope}%
\begin{pgfscope}%
\pgftext[x=0.234413in,y=2.031603in,,bottom,rotate=90.000000]{\rmfamily\fontsize{10.000000}{12.000000}\selectfont \(\displaystyle V_d\) (mL)}%
\end{pgfscope}%
\begin{pgfscope}%
\pgfsetrectcap%
\pgfsetmiterjoin%
\pgfsetlinewidth{0.803000pt}%
\definecolor{currentstroke}{rgb}{0.000000,0.000000,0.000000}%
\pgfsetstrokecolor{currentstroke}%
\pgfsetdash{}{0pt}%
\pgfpathmoveto{\pgfqpoint{0.634105in}{0.521603in}}%
\pgfpathlineto{\pgfqpoint{0.634105in}{3.541603in}}%
\pgfusepath{stroke}%
\end{pgfscope}%
\begin{pgfscope}%
\pgfsetrectcap%
\pgfsetmiterjoin%
\pgfsetlinewidth{0.803000pt}%
\definecolor{currentstroke}{rgb}{0.000000,0.000000,0.000000}%
\pgfsetstrokecolor{currentstroke}%
\pgfsetdash{}{0pt}%
\pgfpathmoveto{\pgfqpoint{4.354105in}{0.521603in}}%
\pgfpathlineto{\pgfqpoint{4.354105in}{3.541603in}}%
\pgfusepath{stroke}%
\end{pgfscope}%
\begin{pgfscope}%
\pgfsetrectcap%
\pgfsetmiterjoin%
\pgfsetlinewidth{0.803000pt}%
\definecolor{currentstroke}{rgb}{0.000000,0.000000,0.000000}%
\pgfsetstrokecolor{currentstroke}%
\pgfsetdash{}{0pt}%
\pgfpathmoveto{\pgfqpoint{0.634105in}{0.521603in}}%
\pgfpathlineto{\pgfqpoint{4.354105in}{0.521603in}}%
\pgfusepath{stroke}%
\end{pgfscope}%
\begin{pgfscope}%
\pgfsetrectcap%
\pgfsetmiterjoin%
\pgfsetlinewidth{0.803000pt}%
\definecolor{currentstroke}{rgb}{0.000000,0.000000,0.000000}%
\pgfsetstrokecolor{currentstroke}%
\pgfsetdash{}{0pt}%
\pgfpathmoveto{\pgfqpoint{0.634105in}{3.541603in}}%
\pgfpathlineto{\pgfqpoint{4.354105in}{3.541603in}}%
\pgfusepath{stroke}%
\end{pgfscope}%
\begin{pgfscope}%
\pgfpathrectangle{\pgfqpoint{4.586605in}{0.521603in}}{\pgfqpoint{0.151000in}{3.020000in}} %
\pgfusepath{clip}%
\pgfsetbuttcap%
\pgfsetmiterjoin%
\definecolor{currentfill}{rgb}{1.000000,1.000000,1.000000}%
\pgfsetfillcolor{currentfill}%
\pgfsetlinewidth{0.010037pt}%
\definecolor{currentstroke}{rgb}{1.000000,1.000000,1.000000}%
\pgfsetstrokecolor{currentstroke}%
\pgfsetdash{}{0pt}%
\pgfpathmoveto{\pgfqpoint{4.586605in}{0.521603in}}%
\pgfpathlineto{\pgfqpoint{4.586605in}{0.953032in}}%
\pgfpathlineto{\pgfqpoint{4.586605in}{3.110175in}}%
\pgfpathlineto{\pgfqpoint{4.586605in}{3.541603in}}%
\pgfpathlineto{\pgfqpoint{4.737605in}{3.541603in}}%
\pgfpathlineto{\pgfqpoint{4.737605in}{3.110175in}}%
\pgfpathlineto{\pgfqpoint{4.737605in}{0.953032in}}%
\pgfpathlineto{\pgfqpoint{4.737605in}{0.521603in}}%
\pgfpathclose%
\pgfusepath{stroke,fill}%
\end{pgfscope}%
\begin{pgfscope}%
\pgfpathrectangle{\pgfqpoint{4.586605in}{0.521603in}}{\pgfqpoint{0.151000in}{3.020000in}} %
\pgfusepath{clip}%
\pgfsetbuttcap%
\pgfsetroundjoin%
\definecolor{currentfill}{rgb}{0.061765,0.061765,0.085934}%
\pgfsetfillcolor{currentfill}%
\pgfsetlinewidth{0.000000pt}%
\definecolor{currentstroke}{rgb}{0.000000,0.000000,0.000000}%
\pgfsetstrokecolor{currentstroke}%
\pgfsetdash{}{0pt}%
\pgfpathmoveto{\pgfqpoint{4.586605in}{0.521603in}}%
\pgfpathlineto{\pgfqpoint{4.737605in}{0.521603in}}%
\pgfpathlineto{\pgfqpoint{4.737605in}{0.953032in}}%
\pgfpathlineto{\pgfqpoint{4.586605in}{0.953032in}}%
\pgfpathlineto{\pgfqpoint{4.586605in}{0.521603in}}%
\pgfusepath{fill}%
\end{pgfscope}%
\begin{pgfscope}%
\pgfpathrectangle{\pgfqpoint{4.586605in}{0.521603in}}{\pgfqpoint{0.151000in}{3.020000in}} %
\pgfusepath{clip}%
\pgfsetbuttcap%
\pgfsetroundjoin%
\definecolor{currentfill}{rgb}{0.185294,0.185294,0.257801}%
\pgfsetfillcolor{currentfill}%
\pgfsetlinewidth{0.000000pt}%
\definecolor{currentstroke}{rgb}{0.000000,0.000000,0.000000}%
\pgfsetstrokecolor{currentstroke}%
\pgfsetdash{}{0pt}%
\pgfpathmoveto{\pgfqpoint{4.586605in}{0.953032in}}%
\pgfpathlineto{\pgfqpoint{4.737605in}{0.953032in}}%
\pgfpathlineto{\pgfqpoint{4.737605in}{1.384460in}}%
\pgfpathlineto{\pgfqpoint{4.586605in}{1.384460in}}%
\pgfpathlineto{\pgfqpoint{4.586605in}{0.953032in}}%
\pgfusepath{fill}%
\end{pgfscope}%
\begin{pgfscope}%
\pgfpathrectangle{\pgfqpoint{4.586605in}{0.521603in}}{\pgfqpoint{0.151000in}{3.020000in}} %
\pgfusepath{clip}%
\pgfsetbuttcap%
\pgfsetroundjoin%
\definecolor{currentfill}{rgb}{0.312255,0.312255,0.434442}%
\pgfsetfillcolor{currentfill}%
\pgfsetlinewidth{0.000000pt}%
\definecolor{currentstroke}{rgb}{0.000000,0.000000,0.000000}%
\pgfsetstrokecolor{currentstroke}%
\pgfsetdash{}{0pt}%
\pgfpathmoveto{\pgfqpoint{4.586605in}{1.384460in}}%
\pgfpathlineto{\pgfqpoint{4.737605in}{1.384460in}}%
\pgfpathlineto{\pgfqpoint{4.737605in}{1.815889in}}%
\pgfpathlineto{\pgfqpoint{4.586605in}{1.815889in}}%
\pgfpathlineto{\pgfqpoint{4.586605in}{1.384460in}}%
\pgfusepath{fill}%
\end{pgfscope}%
\begin{pgfscope}%
\pgfpathrectangle{\pgfqpoint{4.586605in}{0.521603in}}{\pgfqpoint{0.151000in}{3.020000in}} %
\pgfusepath{clip}%
\pgfsetbuttcap%
\pgfsetroundjoin%
\definecolor{currentfill}{rgb}{0.439216,0.484130,0.564216}%
\pgfsetfillcolor{currentfill}%
\pgfsetlinewidth{0.000000pt}%
\definecolor{currentstroke}{rgb}{0.000000,0.000000,0.000000}%
\pgfsetstrokecolor{currentstroke}%
\pgfsetdash{}{0pt}%
\pgfpathmoveto{\pgfqpoint{4.586605in}{1.815889in}}%
\pgfpathlineto{\pgfqpoint{4.737605in}{1.815889in}}%
\pgfpathlineto{\pgfqpoint{4.737605in}{2.247318in}}%
\pgfpathlineto{\pgfqpoint{4.586605in}{2.247318in}}%
\pgfpathlineto{\pgfqpoint{4.586605in}{1.815889in}}%
\pgfusepath{fill}%
\end{pgfscope}%
\begin{pgfscope}%
\pgfpathrectangle{\pgfqpoint{4.586605in}{0.521603in}}{\pgfqpoint{0.151000in}{3.020000in}} %
\pgfusepath{clip}%
\pgfsetbuttcap%
\pgfsetroundjoin%
\definecolor{currentfill}{rgb}{0.562745,0.653983,0.687745}%
\pgfsetfillcolor{currentfill}%
\pgfsetlinewidth{0.000000pt}%
\definecolor{currentstroke}{rgb}{0.000000,0.000000,0.000000}%
\pgfsetstrokecolor{currentstroke}%
\pgfsetdash{}{0pt}%
\pgfpathmoveto{\pgfqpoint{4.586605in}{2.247318in}}%
\pgfpathlineto{\pgfqpoint{4.737605in}{2.247318in}}%
\pgfpathlineto{\pgfqpoint{4.737605in}{2.678746in}}%
\pgfpathlineto{\pgfqpoint{4.586605in}{2.678746in}}%
\pgfpathlineto{\pgfqpoint{4.586605in}{2.247318in}}%
\pgfusepath{fill}%
\end{pgfscope}%
\begin{pgfscope}%
\pgfpathrectangle{\pgfqpoint{4.586605in}{0.521603in}}{\pgfqpoint{0.151000in}{3.020000in}} %
\pgfusepath{clip}%
\pgfsetbuttcap%
\pgfsetroundjoin%
\definecolor{currentfill}{rgb}{0.710478,0.814706,0.814706}%
\pgfsetfillcolor{currentfill}%
\pgfsetlinewidth{0.000000pt}%
\definecolor{currentstroke}{rgb}{0.000000,0.000000,0.000000}%
\pgfsetstrokecolor{currentstroke}%
\pgfsetdash{}{0pt}%
\pgfpathmoveto{\pgfqpoint{4.586605in}{2.678746in}}%
\pgfpathlineto{\pgfqpoint{4.737605in}{2.678746in}}%
\pgfpathlineto{\pgfqpoint{4.737605in}{3.110175in}}%
\pgfpathlineto{\pgfqpoint{4.586605in}{3.110175in}}%
\pgfpathlineto{\pgfqpoint{4.586605in}{2.678746in}}%
\pgfusepath{fill}%
\end{pgfscope}%
\begin{pgfscope}%
\pgfpathrectangle{\pgfqpoint{4.586605in}{0.521603in}}{\pgfqpoint{0.151000in}{3.020000in}} %
\pgfusepath{clip}%
\pgfsetbuttcap%
\pgfsetroundjoin%
\definecolor{currentfill}{rgb}{0.903493,0.938235,0.938235}%
\pgfsetfillcolor{currentfill}%
\pgfsetlinewidth{0.000000pt}%
\definecolor{currentstroke}{rgb}{0.000000,0.000000,0.000000}%
\pgfsetstrokecolor{currentstroke}%
\pgfsetdash{}{0pt}%
\pgfpathmoveto{\pgfqpoint{4.586605in}{3.110175in}}%
\pgfpathlineto{\pgfqpoint{4.737605in}{3.110175in}}%
\pgfpathlineto{\pgfqpoint{4.737605in}{3.541603in}}%
\pgfpathlineto{\pgfqpoint{4.586605in}{3.541603in}}%
\pgfpathlineto{\pgfqpoint{4.586605in}{3.110175in}}%
\pgfusepath{fill}%
\end{pgfscope}%
\begin{pgfscope}%
\pgfsetbuttcap%
\pgfsetroundjoin%
\definecolor{currentfill}{rgb}{0.000000,0.000000,0.000000}%
\pgfsetfillcolor{currentfill}%
\pgfsetlinewidth{0.803000pt}%
\definecolor{currentstroke}{rgb}{0.000000,0.000000,0.000000}%
\pgfsetstrokecolor{currentstroke}%
\pgfsetdash{}{0pt}%
\pgfsys@defobject{currentmarker}{\pgfqpoint{0.000000in}{0.000000in}}{\pgfqpoint{0.048611in}{0.000000in}}{%
\pgfpathmoveto{\pgfqpoint{0.000000in}{0.000000in}}%
\pgfpathlineto{\pgfqpoint{0.048611in}{0.000000in}}%
\pgfusepath{stroke,fill}%
}%
\begin{pgfscope}%
\pgfsys@transformshift{4.737605in}{0.521603in}%
\pgfsys@useobject{currentmarker}{}%
\end{pgfscope}%
\end{pgfscope}%
\begin{pgfscope}%
\pgftext[x=4.834827in,y=0.468842in,left,base]{\rmfamily\fontsize{10.000000}{12.000000}\selectfont \(\displaystyle 0\)}%
\end{pgfscope}%
\begin{pgfscope}%
\pgfsetbuttcap%
\pgfsetroundjoin%
\definecolor{currentfill}{rgb}{0.000000,0.000000,0.000000}%
\pgfsetfillcolor{currentfill}%
\pgfsetlinewidth{0.803000pt}%
\definecolor{currentstroke}{rgb}{0.000000,0.000000,0.000000}%
\pgfsetstrokecolor{currentstroke}%
\pgfsetdash{}{0pt}%
\pgfsys@defobject{currentmarker}{\pgfqpoint{0.000000in}{0.000000in}}{\pgfqpoint{0.048611in}{0.000000in}}{%
\pgfpathmoveto{\pgfqpoint{0.000000in}{0.000000in}}%
\pgfpathlineto{\pgfqpoint{0.048611in}{0.000000in}}%
\pgfusepath{stroke,fill}%
}%
\begin{pgfscope}%
\pgfsys@transformshift{4.737605in}{0.953032in}%
\pgfsys@useobject{currentmarker}{}%
\end{pgfscope}%
\end{pgfscope}%
\begin{pgfscope}%
\pgftext[x=4.834827in,y=0.900270in,left,base]{\rmfamily\fontsize{10.000000}{12.000000}\selectfont \(\displaystyle 1\)}%
\end{pgfscope}%
\begin{pgfscope}%
\pgfsetbuttcap%
\pgfsetroundjoin%
\definecolor{currentfill}{rgb}{0.000000,0.000000,0.000000}%
\pgfsetfillcolor{currentfill}%
\pgfsetlinewidth{0.803000pt}%
\definecolor{currentstroke}{rgb}{0.000000,0.000000,0.000000}%
\pgfsetstrokecolor{currentstroke}%
\pgfsetdash{}{0pt}%
\pgfsys@defobject{currentmarker}{\pgfqpoint{0.000000in}{0.000000in}}{\pgfqpoint{0.048611in}{0.000000in}}{%
\pgfpathmoveto{\pgfqpoint{0.000000in}{0.000000in}}%
\pgfpathlineto{\pgfqpoint{0.048611in}{0.000000in}}%
\pgfusepath{stroke,fill}%
}%
\begin{pgfscope}%
\pgfsys@transformshift{4.737605in}{1.384460in}%
\pgfsys@useobject{currentmarker}{}%
\end{pgfscope}%
\end{pgfscope}%
\begin{pgfscope}%
\pgftext[x=4.834827in,y=1.331699in,left,base]{\rmfamily\fontsize{10.000000}{12.000000}\selectfont \(\displaystyle 2\)}%
\end{pgfscope}%
\begin{pgfscope}%
\pgfsetbuttcap%
\pgfsetroundjoin%
\definecolor{currentfill}{rgb}{0.000000,0.000000,0.000000}%
\pgfsetfillcolor{currentfill}%
\pgfsetlinewidth{0.803000pt}%
\definecolor{currentstroke}{rgb}{0.000000,0.000000,0.000000}%
\pgfsetstrokecolor{currentstroke}%
\pgfsetdash{}{0pt}%
\pgfsys@defobject{currentmarker}{\pgfqpoint{0.000000in}{0.000000in}}{\pgfqpoint{0.048611in}{0.000000in}}{%
\pgfpathmoveto{\pgfqpoint{0.000000in}{0.000000in}}%
\pgfpathlineto{\pgfqpoint{0.048611in}{0.000000in}}%
\pgfusepath{stroke,fill}%
}%
\begin{pgfscope}%
\pgfsys@transformshift{4.737605in}{1.815889in}%
\pgfsys@useobject{currentmarker}{}%
\end{pgfscope}%
\end{pgfscope}%
\begin{pgfscope}%
\pgftext[x=4.834827in,y=1.763128in,left,base]{\rmfamily\fontsize{10.000000}{12.000000}\selectfont \(\displaystyle 3\)}%
\end{pgfscope}%
\begin{pgfscope}%
\pgfsetbuttcap%
\pgfsetroundjoin%
\definecolor{currentfill}{rgb}{0.000000,0.000000,0.000000}%
\pgfsetfillcolor{currentfill}%
\pgfsetlinewidth{0.803000pt}%
\definecolor{currentstroke}{rgb}{0.000000,0.000000,0.000000}%
\pgfsetstrokecolor{currentstroke}%
\pgfsetdash{}{0pt}%
\pgfsys@defobject{currentmarker}{\pgfqpoint{0.000000in}{0.000000in}}{\pgfqpoint{0.048611in}{0.000000in}}{%
\pgfpathmoveto{\pgfqpoint{0.000000in}{0.000000in}}%
\pgfpathlineto{\pgfqpoint{0.048611in}{0.000000in}}%
\pgfusepath{stroke,fill}%
}%
\begin{pgfscope}%
\pgfsys@transformshift{4.737605in}{2.247318in}%
\pgfsys@useobject{currentmarker}{}%
\end{pgfscope}%
\end{pgfscope}%
\begin{pgfscope}%
\pgftext[x=4.834827in,y=2.194556in,left,base]{\rmfamily\fontsize{10.000000}{12.000000}\selectfont \(\displaystyle 4\)}%
\end{pgfscope}%
\begin{pgfscope}%
\pgfsetbuttcap%
\pgfsetroundjoin%
\definecolor{currentfill}{rgb}{0.000000,0.000000,0.000000}%
\pgfsetfillcolor{currentfill}%
\pgfsetlinewidth{0.803000pt}%
\definecolor{currentstroke}{rgb}{0.000000,0.000000,0.000000}%
\pgfsetstrokecolor{currentstroke}%
\pgfsetdash{}{0pt}%
\pgfsys@defobject{currentmarker}{\pgfqpoint{0.000000in}{0.000000in}}{\pgfqpoint{0.048611in}{0.000000in}}{%
\pgfpathmoveto{\pgfqpoint{0.000000in}{0.000000in}}%
\pgfpathlineto{\pgfqpoint{0.048611in}{0.000000in}}%
\pgfusepath{stroke,fill}%
}%
\begin{pgfscope}%
\pgfsys@transformshift{4.737605in}{2.678746in}%
\pgfsys@useobject{currentmarker}{}%
\end{pgfscope}%
\end{pgfscope}%
\begin{pgfscope}%
\pgftext[x=4.834827in,y=2.625985in,left,base]{\rmfamily\fontsize{10.000000}{12.000000}\selectfont \(\displaystyle 5\)}%
\end{pgfscope}%
\begin{pgfscope}%
\pgfsetbuttcap%
\pgfsetroundjoin%
\definecolor{currentfill}{rgb}{0.000000,0.000000,0.000000}%
\pgfsetfillcolor{currentfill}%
\pgfsetlinewidth{0.803000pt}%
\definecolor{currentstroke}{rgb}{0.000000,0.000000,0.000000}%
\pgfsetstrokecolor{currentstroke}%
\pgfsetdash{}{0pt}%
\pgfsys@defobject{currentmarker}{\pgfqpoint{0.000000in}{0.000000in}}{\pgfqpoint{0.048611in}{0.000000in}}{%
\pgfpathmoveto{\pgfqpoint{0.000000in}{0.000000in}}%
\pgfpathlineto{\pgfqpoint{0.048611in}{0.000000in}}%
\pgfusepath{stroke,fill}%
}%
\begin{pgfscope}%
\pgfsys@transformshift{4.737605in}{3.110175in}%
\pgfsys@useobject{currentmarker}{}%
\end{pgfscope}%
\end{pgfscope}%
\begin{pgfscope}%
\pgftext[x=4.834827in,y=3.057413in,left,base]{\rmfamily\fontsize{10.000000}{12.000000}\selectfont \(\displaystyle 6\)}%
\end{pgfscope}%
\begin{pgfscope}%
\pgfsetbuttcap%
\pgfsetroundjoin%
\definecolor{currentfill}{rgb}{0.000000,0.000000,0.000000}%
\pgfsetfillcolor{currentfill}%
\pgfsetlinewidth{0.803000pt}%
\definecolor{currentstroke}{rgb}{0.000000,0.000000,0.000000}%
\pgfsetstrokecolor{currentstroke}%
\pgfsetdash{}{0pt}%
\pgfsys@defobject{currentmarker}{\pgfqpoint{0.000000in}{0.000000in}}{\pgfqpoint{0.048611in}{0.000000in}}{%
\pgfpathmoveto{\pgfqpoint{0.000000in}{0.000000in}}%
\pgfpathlineto{\pgfqpoint{0.048611in}{0.000000in}}%
\pgfusepath{stroke,fill}%
}%
\begin{pgfscope}%
\pgfsys@transformshift{4.737605in}{3.541603in}%
\pgfsys@useobject{currentmarker}{}%
\end{pgfscope}%
\end{pgfscope}%
\begin{pgfscope}%
\pgftext[x=4.834827in,y=3.488842in,left,base]{\rmfamily\fontsize{10.000000}{12.000000}\selectfont \(\displaystyle 7\)}%
\end{pgfscope}%
\begin{pgfscope}%
\pgftext[x=4.959827in,y=2.031603in,,top,rotate=90.000000]{\rmfamily\fontsize{10.000000}{12.000000}\selectfont \(\displaystyle q\) (C)}%
\end{pgfscope}%
\begin{pgfscope}%
\pgftext[x=4.737605in,y=3.583270in,right,base]{\rmfamily\fontsize{10.000000}{12.000000}\selectfont \(\displaystyle \times10^{-10}\)}%
\end{pgfscope}%
\begin{pgfscope}%
\pgfsetbuttcap%
\pgfsetmiterjoin%
\pgfsetlinewidth{0.803000pt}%
\definecolor{currentstroke}{rgb}{0.000000,0.000000,0.000000}%
\pgfsetstrokecolor{currentstroke}%
\pgfsetdash{}{0pt}%
\pgfpathmoveto{\pgfqpoint{4.586605in}{0.521603in}}%
\pgfpathlineto{\pgfqpoint{4.586605in}{0.953032in}}%
\pgfpathlineto{\pgfqpoint{4.586605in}{3.110175in}}%
\pgfpathlineto{\pgfqpoint{4.586605in}{3.541603in}}%
\pgfpathlineto{\pgfqpoint{4.737605in}{3.541603in}}%
\pgfpathlineto{\pgfqpoint{4.737605in}{3.110175in}}%
\pgfpathlineto{\pgfqpoint{4.737605in}{0.953032in}}%
\pgfpathlineto{\pgfqpoint{4.737605in}{0.521603in}}%
\pgfpathclose%
\pgfusepath{stroke}%
\end{pgfscope}%
\end{pgfpicture}%
\makeatother%
\endgroup%

    \caption{Charge $q$, as a function of $V_d$, $\varphi_s$.\label{fig:charge}}
\end{figure}

A two-ways T-test comparison of charge distributions between the drop bounce experiment and a corollary experiment with zero electric field at the time of drop deposition on the superhydrophobic surface suggests that the drop charge is induced by the electric field, rather than through contact charging on the PTFE layer ($t = 5.11, p = 0.0002$). The T-test informs us that the charge distribution  are about 5 times more different from each other as they are within each other, and there is a 0.02$\%$ probability that this result happened by chance. This corollary experiment is documented in Appendix \ref{sec.drop_charge}.

The model $q \sim kAE_0$ is incidentally very similar to the classical solution for the surface charge density of a half-spherical conductor with a field-induced dipole \cite{david_j._griffiths_introduction_1999}
\begin{eqnarray*}
q &=& 3 \epsilon_0 E_0 \int_A \cos \theta dA \\
&=& 3 \pi^{1/3} 6 \left(6 V_d \right)^{2/3} \epsilon_0 E_0 \int^{4 \pi/2}_{\pi / 2} \cos \theta d\theta \\
&=& k E_0 V_d^{2/3}
\end{eqnarray*}
with $k \approx 1.3 \times 10^{-10}$. This is also of a similar form to the charge found by Takamatsu and coauthors for drops falling from a grounded nozzle in an external electric field \cite{takamatsu_theoretical_1981}
\[q = 4 \pi \epsilon_0 \beta E_0 R_d^2 \]
with $\beta \approx 2.63$.

\section{Scale Quantities}
The dielectrophoretic force plays a very small role when drops have net charge in a DC field; the condition to neglect the DEP force was satisfied for all experiments in the dataset. Dimensional drop apoapses scale closely with $\mathbb{E}\mbox{u}$ as seen in Figure \ref{fig:series_s_eu}. The relative magnitudes of the simulated forces felt by the drops is shown in Figure \ref{fig:forces}. Forces acting on the drops vary in magnitude between $\mathcal{O}(10^{-6})$-$\mathcal{O}(10^{-4})$ N. We see that, of the drops in the experimental dataset only the two with the largest $\mathbb{E}\mbox{u}$, $\mathbb{E}\mbox{u} \sim \mathcal{O}(1)$ could appropriately be said to be in the inertial electro-viscous regime. In all other cases image forces are much stronger than drag. For these drops $\mathbb{E}\mbox{u} \gg 1/8 \pi$, and are likely on escape trajectories. The image forces themselves rapidly become small compared to Coulomb forces for drops with apoapses $\mbox{max}\left( y\right) \gtrapprox L$, thus it is reasonable to claim that for intermediate drops Coulomb force scales as inertia, and we can neglect the effects of drag and image forces.

\begin{figure}[!htb]
    \centering
    %% Creator: Matplotlib, PGF backend
%%
%% To include the figure in your LaTeX document, write
%%   \input{<filename>.pgf}
%%
%% Make sure the required packages are loaded in your preamble
%%   \usepackage{pgf}
%%
%% Figures using additional raster images can only be included by \input if
%% they are in the same directory as the main LaTeX file. For loading figures
%% from other directories you can use the `import` package
%%   \usepackage{import}
%% and then include the figures with
%%   \import{<path to file>}{<filename>.pgf}
%%
%% Matplotlib used the following preamble
%%   \usepackage{fontspec}
%%   \setmainfont{DejaVu Serif}
%%   \setsansfont{DejaVu Sans}
%%   \setmonofont{DejaVu Sans Mono}
%%
\begingroup%
\makeatletter%
\begin{pgfpicture}%
\pgfpathrectangle{\pgfpointorigin}{\pgfqpoint{5.270186in}{3.684574in}}%
\pgfusepath{use as bounding box, clip}%
\begin{pgfscope}%
\pgfsetbuttcap%
\pgfsetmiterjoin%
\definecolor{currentfill}{rgb}{1.000000,1.000000,1.000000}%
\pgfsetfillcolor{currentfill}%
\pgfsetlinewidth{0.000000pt}%
\definecolor{currentstroke}{rgb}{1.000000,1.000000,1.000000}%
\pgfsetstrokecolor{currentstroke}%
\pgfsetdash{}{0pt}%
\pgfpathmoveto{\pgfqpoint{0.000000in}{0.000000in}}%
\pgfpathlineto{\pgfqpoint{5.270186in}{0.000000in}}%
\pgfpathlineto{\pgfqpoint{5.270186in}{3.684574in}}%
\pgfpathlineto{\pgfqpoint{0.000000in}{3.684574in}}%
\pgfpathclose%
\pgfusepath{fill}%
\end{pgfscope}%
\begin{pgfscope}%
\pgfsetbuttcap%
\pgfsetmiterjoin%
\definecolor{currentfill}{rgb}{1.000000,1.000000,1.000000}%
\pgfsetfillcolor{currentfill}%
\pgfsetlinewidth{0.000000pt}%
\definecolor{currentstroke}{rgb}{0.000000,0.000000,0.000000}%
\pgfsetstrokecolor{currentstroke}%
\pgfsetstrokeopacity{0.000000}%
\pgfsetdash{}{0pt}%
\pgfpathmoveto{\pgfqpoint{0.456635in}{0.521603in}}%
\pgfpathlineto{\pgfqpoint{4.176635in}{0.521603in}}%
\pgfpathlineto{\pgfqpoint{4.176635in}{3.541603in}}%
\pgfpathlineto{\pgfqpoint{0.456635in}{3.541603in}}%
\pgfpathclose%
\pgfusepath{fill}%
\end{pgfscope}%
\begin{pgfscope}%
\pgfsetbuttcap%
\pgfsetroundjoin%
\definecolor{currentfill}{rgb}{0.000000,0.000000,0.000000}%
\pgfsetfillcolor{currentfill}%
\pgfsetlinewidth{0.803000pt}%
\definecolor{currentstroke}{rgb}{0.000000,0.000000,0.000000}%
\pgfsetstrokecolor{currentstroke}%
\pgfsetdash{}{0pt}%
\pgfsys@defobject{currentmarker}{\pgfqpoint{0.000000in}{-0.048611in}}{\pgfqpoint{0.000000in}{0.000000in}}{%
\pgfpathmoveto{\pgfqpoint{0.000000in}{0.000000in}}%
\pgfpathlineto{\pgfqpoint{0.000000in}{-0.048611in}}%
\pgfusepath{stroke,fill}%
}%
\begin{pgfscope}%
\pgfsys@transformshift{0.581806in}{0.521603in}%
\pgfsys@useobject{currentmarker}{}%
\end{pgfscope}%
\end{pgfscope}%
\begin{pgfscope}%
\pgftext[x=0.581806in,y=0.424381in,,top]{\rmfamily\fontsize{10.000000}{12.000000}\selectfont \(\displaystyle 0.0\)}%
\end{pgfscope}%
\begin{pgfscope}%
\pgfsetbuttcap%
\pgfsetroundjoin%
\definecolor{currentfill}{rgb}{0.000000,0.000000,0.000000}%
\pgfsetfillcolor{currentfill}%
\pgfsetlinewidth{0.803000pt}%
\definecolor{currentstroke}{rgb}{0.000000,0.000000,0.000000}%
\pgfsetstrokecolor{currentstroke}%
\pgfsetdash{}{0pt}%
\pgfsys@defobject{currentmarker}{\pgfqpoint{0.000000in}{-0.048611in}}{\pgfqpoint{0.000000in}{0.000000in}}{%
\pgfpathmoveto{\pgfqpoint{0.000000in}{0.000000in}}%
\pgfpathlineto{\pgfqpoint{0.000000in}{-0.048611in}}%
\pgfusepath{stroke,fill}%
}%
\begin{pgfscope}%
\pgfsys@transformshift{1.460201in}{0.521603in}%
\pgfsys@useobject{currentmarker}{}%
\end{pgfscope}%
\end{pgfscope}%
\begin{pgfscope}%
\pgftext[x=1.460201in,y=0.424381in,,top]{\rmfamily\fontsize{10.000000}{12.000000}\selectfont \(\displaystyle 0.5\)}%
\end{pgfscope}%
\begin{pgfscope}%
\pgfsetbuttcap%
\pgfsetroundjoin%
\definecolor{currentfill}{rgb}{0.000000,0.000000,0.000000}%
\pgfsetfillcolor{currentfill}%
\pgfsetlinewidth{0.803000pt}%
\definecolor{currentstroke}{rgb}{0.000000,0.000000,0.000000}%
\pgfsetstrokecolor{currentstroke}%
\pgfsetdash{}{0pt}%
\pgfsys@defobject{currentmarker}{\pgfqpoint{0.000000in}{-0.048611in}}{\pgfqpoint{0.000000in}{0.000000in}}{%
\pgfpathmoveto{\pgfqpoint{0.000000in}{0.000000in}}%
\pgfpathlineto{\pgfqpoint{0.000000in}{-0.048611in}}%
\pgfusepath{stroke,fill}%
}%
\begin{pgfscope}%
\pgfsys@transformshift{2.338595in}{0.521603in}%
\pgfsys@useobject{currentmarker}{}%
\end{pgfscope}%
\end{pgfscope}%
\begin{pgfscope}%
\pgftext[x=2.338595in,y=0.424381in,,top]{\rmfamily\fontsize{10.000000}{12.000000}\selectfont \(\displaystyle 1.0\)}%
\end{pgfscope}%
\begin{pgfscope}%
\pgfsetbuttcap%
\pgfsetroundjoin%
\definecolor{currentfill}{rgb}{0.000000,0.000000,0.000000}%
\pgfsetfillcolor{currentfill}%
\pgfsetlinewidth{0.803000pt}%
\definecolor{currentstroke}{rgb}{0.000000,0.000000,0.000000}%
\pgfsetstrokecolor{currentstroke}%
\pgfsetdash{}{0pt}%
\pgfsys@defobject{currentmarker}{\pgfqpoint{0.000000in}{-0.048611in}}{\pgfqpoint{0.000000in}{0.000000in}}{%
\pgfpathmoveto{\pgfqpoint{0.000000in}{0.000000in}}%
\pgfpathlineto{\pgfqpoint{0.000000in}{-0.048611in}}%
\pgfusepath{stroke,fill}%
}%
\begin{pgfscope}%
\pgfsys@transformshift{3.216989in}{0.521603in}%
\pgfsys@useobject{currentmarker}{}%
\end{pgfscope}%
\end{pgfscope}%
\begin{pgfscope}%
\pgftext[x=3.216989in,y=0.424381in,,top]{\rmfamily\fontsize{10.000000}{12.000000}\selectfont \(\displaystyle 1.5\)}%
\end{pgfscope}%
\begin{pgfscope}%
\pgfsetbuttcap%
\pgfsetroundjoin%
\definecolor{currentfill}{rgb}{0.000000,0.000000,0.000000}%
\pgfsetfillcolor{currentfill}%
\pgfsetlinewidth{0.803000pt}%
\definecolor{currentstroke}{rgb}{0.000000,0.000000,0.000000}%
\pgfsetstrokecolor{currentstroke}%
\pgfsetdash{}{0pt}%
\pgfsys@defobject{currentmarker}{\pgfqpoint{0.000000in}{-0.048611in}}{\pgfqpoint{0.000000in}{0.000000in}}{%
\pgfpathmoveto{\pgfqpoint{0.000000in}{0.000000in}}%
\pgfpathlineto{\pgfqpoint{0.000000in}{-0.048611in}}%
\pgfusepath{stroke,fill}%
}%
\begin{pgfscope}%
\pgfsys@transformshift{4.095384in}{0.521603in}%
\pgfsys@useobject{currentmarker}{}%
\end{pgfscope}%
\end{pgfscope}%
\begin{pgfscope}%
\pgftext[x=4.095384in,y=0.424381in,,top]{\rmfamily\fontsize{10.000000}{12.000000}\selectfont \(\displaystyle 2.0\)}%
\end{pgfscope}%
\begin{pgfscope}%
\pgftext[x=2.316635in,y=0.234413in,,top]{\rmfamily\fontsize{10.000000}{12.000000}\selectfont \(\displaystyle t\) (s)}%
\end{pgfscope}%
\begin{pgfscope}%
\pgfsetbuttcap%
\pgfsetroundjoin%
\definecolor{currentfill}{rgb}{0.000000,0.000000,0.000000}%
\pgfsetfillcolor{currentfill}%
\pgfsetlinewidth{0.803000pt}%
\definecolor{currentstroke}{rgb}{0.000000,0.000000,0.000000}%
\pgfsetstrokecolor{currentstroke}%
\pgfsetdash{}{0pt}%
\pgfsys@defobject{currentmarker}{\pgfqpoint{-0.048611in}{0.000000in}}{\pgfqpoint{0.000000in}{0.000000in}}{%
\pgfpathmoveto{\pgfqpoint{0.000000in}{0.000000in}}%
\pgfpathlineto{\pgfqpoint{-0.048611in}{0.000000in}}%
\pgfusepath{stroke,fill}%
}%
\begin{pgfscope}%
\pgfsys@transformshift{0.456635in}{0.597983in}%
\pgfsys@useobject{currentmarker}{}%
\end{pgfscope}%
\end{pgfscope}%
\begin{pgfscope}%
\pgftext[x=0.289968in,y=0.545221in,left,base]{\rmfamily\fontsize{10.000000}{12.000000}\selectfont \(\displaystyle 0\)}%
\end{pgfscope}%
\begin{pgfscope}%
\pgfsetbuttcap%
\pgfsetroundjoin%
\definecolor{currentfill}{rgb}{0.000000,0.000000,0.000000}%
\pgfsetfillcolor{currentfill}%
\pgfsetlinewidth{0.803000pt}%
\definecolor{currentstroke}{rgb}{0.000000,0.000000,0.000000}%
\pgfsetstrokecolor{currentstroke}%
\pgfsetdash{}{0pt}%
\pgfsys@defobject{currentmarker}{\pgfqpoint{-0.048611in}{0.000000in}}{\pgfqpoint{0.000000in}{0.000000in}}{%
\pgfpathmoveto{\pgfqpoint{0.000000in}{0.000000in}}%
\pgfpathlineto{\pgfqpoint{-0.048611in}{0.000000in}}%
\pgfusepath{stroke,fill}%
}%
\begin{pgfscope}%
\pgfsys@transformshift{0.456635in}{0.964711in}%
\pgfsys@useobject{currentmarker}{}%
\end{pgfscope}%
\end{pgfscope}%
\begin{pgfscope}%
\pgftext[x=0.289968in,y=0.911950in,left,base]{\rmfamily\fontsize{10.000000}{12.000000}\selectfont \(\displaystyle 1\)}%
\end{pgfscope}%
\begin{pgfscope}%
\pgfsetbuttcap%
\pgfsetroundjoin%
\definecolor{currentfill}{rgb}{0.000000,0.000000,0.000000}%
\pgfsetfillcolor{currentfill}%
\pgfsetlinewidth{0.803000pt}%
\definecolor{currentstroke}{rgb}{0.000000,0.000000,0.000000}%
\pgfsetstrokecolor{currentstroke}%
\pgfsetdash{}{0pt}%
\pgfsys@defobject{currentmarker}{\pgfqpoint{-0.048611in}{0.000000in}}{\pgfqpoint{0.000000in}{0.000000in}}{%
\pgfpathmoveto{\pgfqpoint{0.000000in}{0.000000in}}%
\pgfpathlineto{\pgfqpoint{-0.048611in}{0.000000in}}%
\pgfusepath{stroke,fill}%
}%
\begin{pgfscope}%
\pgfsys@transformshift{0.456635in}{1.331440in}%
\pgfsys@useobject{currentmarker}{}%
\end{pgfscope}%
\end{pgfscope}%
\begin{pgfscope}%
\pgftext[x=0.289968in,y=1.278679in,left,base]{\rmfamily\fontsize{10.000000}{12.000000}\selectfont \(\displaystyle 2\)}%
\end{pgfscope}%
\begin{pgfscope}%
\pgfsetbuttcap%
\pgfsetroundjoin%
\definecolor{currentfill}{rgb}{0.000000,0.000000,0.000000}%
\pgfsetfillcolor{currentfill}%
\pgfsetlinewidth{0.803000pt}%
\definecolor{currentstroke}{rgb}{0.000000,0.000000,0.000000}%
\pgfsetstrokecolor{currentstroke}%
\pgfsetdash{}{0pt}%
\pgfsys@defobject{currentmarker}{\pgfqpoint{-0.048611in}{0.000000in}}{\pgfqpoint{0.000000in}{0.000000in}}{%
\pgfpathmoveto{\pgfqpoint{0.000000in}{0.000000in}}%
\pgfpathlineto{\pgfqpoint{-0.048611in}{0.000000in}}%
\pgfusepath{stroke,fill}%
}%
\begin{pgfscope}%
\pgfsys@transformshift{0.456635in}{1.698169in}%
\pgfsys@useobject{currentmarker}{}%
\end{pgfscope}%
\end{pgfscope}%
\begin{pgfscope}%
\pgftext[x=0.289968in,y=1.645407in,left,base]{\rmfamily\fontsize{10.000000}{12.000000}\selectfont \(\displaystyle 3\)}%
\end{pgfscope}%
\begin{pgfscope}%
\pgfsetbuttcap%
\pgfsetroundjoin%
\definecolor{currentfill}{rgb}{0.000000,0.000000,0.000000}%
\pgfsetfillcolor{currentfill}%
\pgfsetlinewidth{0.803000pt}%
\definecolor{currentstroke}{rgb}{0.000000,0.000000,0.000000}%
\pgfsetstrokecolor{currentstroke}%
\pgfsetdash{}{0pt}%
\pgfsys@defobject{currentmarker}{\pgfqpoint{-0.048611in}{0.000000in}}{\pgfqpoint{0.000000in}{0.000000in}}{%
\pgfpathmoveto{\pgfqpoint{0.000000in}{0.000000in}}%
\pgfpathlineto{\pgfqpoint{-0.048611in}{0.000000in}}%
\pgfusepath{stroke,fill}%
}%
\begin{pgfscope}%
\pgfsys@transformshift{0.456635in}{2.064898in}%
\pgfsys@useobject{currentmarker}{}%
\end{pgfscope}%
\end{pgfscope}%
\begin{pgfscope}%
\pgftext[x=0.289968in,y=2.012136in,left,base]{\rmfamily\fontsize{10.000000}{12.000000}\selectfont \(\displaystyle 4\)}%
\end{pgfscope}%
\begin{pgfscope}%
\pgfsetbuttcap%
\pgfsetroundjoin%
\definecolor{currentfill}{rgb}{0.000000,0.000000,0.000000}%
\pgfsetfillcolor{currentfill}%
\pgfsetlinewidth{0.803000pt}%
\definecolor{currentstroke}{rgb}{0.000000,0.000000,0.000000}%
\pgfsetstrokecolor{currentstroke}%
\pgfsetdash{}{0pt}%
\pgfsys@defobject{currentmarker}{\pgfqpoint{-0.048611in}{0.000000in}}{\pgfqpoint{0.000000in}{0.000000in}}{%
\pgfpathmoveto{\pgfqpoint{0.000000in}{0.000000in}}%
\pgfpathlineto{\pgfqpoint{-0.048611in}{0.000000in}}%
\pgfusepath{stroke,fill}%
}%
\begin{pgfscope}%
\pgfsys@transformshift{0.456635in}{2.431627in}%
\pgfsys@useobject{currentmarker}{}%
\end{pgfscope}%
\end{pgfscope}%
\begin{pgfscope}%
\pgftext[x=0.289968in,y=2.378865in,left,base]{\rmfamily\fontsize{10.000000}{12.000000}\selectfont \(\displaystyle 5\)}%
\end{pgfscope}%
\begin{pgfscope}%
\pgfsetbuttcap%
\pgfsetroundjoin%
\definecolor{currentfill}{rgb}{0.000000,0.000000,0.000000}%
\pgfsetfillcolor{currentfill}%
\pgfsetlinewidth{0.803000pt}%
\definecolor{currentstroke}{rgb}{0.000000,0.000000,0.000000}%
\pgfsetstrokecolor{currentstroke}%
\pgfsetdash{}{0pt}%
\pgfsys@defobject{currentmarker}{\pgfqpoint{-0.048611in}{0.000000in}}{\pgfqpoint{0.000000in}{0.000000in}}{%
\pgfpathmoveto{\pgfqpoint{0.000000in}{0.000000in}}%
\pgfpathlineto{\pgfqpoint{-0.048611in}{0.000000in}}%
\pgfusepath{stroke,fill}%
}%
\begin{pgfscope}%
\pgfsys@transformshift{0.456635in}{2.798355in}%
\pgfsys@useobject{currentmarker}{}%
\end{pgfscope}%
\end{pgfscope}%
\begin{pgfscope}%
\pgftext[x=0.289968in,y=2.745594in,left,base]{\rmfamily\fontsize{10.000000}{12.000000}\selectfont \(\displaystyle 6\)}%
\end{pgfscope}%
\begin{pgfscope}%
\pgfsetbuttcap%
\pgfsetroundjoin%
\definecolor{currentfill}{rgb}{0.000000,0.000000,0.000000}%
\pgfsetfillcolor{currentfill}%
\pgfsetlinewidth{0.803000pt}%
\definecolor{currentstroke}{rgb}{0.000000,0.000000,0.000000}%
\pgfsetstrokecolor{currentstroke}%
\pgfsetdash{}{0pt}%
\pgfsys@defobject{currentmarker}{\pgfqpoint{-0.048611in}{0.000000in}}{\pgfqpoint{0.000000in}{0.000000in}}{%
\pgfpathmoveto{\pgfqpoint{0.000000in}{0.000000in}}%
\pgfpathlineto{\pgfqpoint{-0.048611in}{0.000000in}}%
\pgfusepath{stroke,fill}%
}%
\begin{pgfscope}%
\pgfsys@transformshift{0.456635in}{3.165084in}%
\pgfsys@useobject{currentmarker}{}%
\end{pgfscope}%
\end{pgfscope}%
\begin{pgfscope}%
\pgftext[x=0.289968in,y=3.112323in,left,base]{\rmfamily\fontsize{10.000000}{12.000000}\selectfont \(\displaystyle 7\)}%
\end{pgfscope}%
\begin{pgfscope}%
\pgfsetbuttcap%
\pgfsetroundjoin%
\definecolor{currentfill}{rgb}{0.000000,0.000000,0.000000}%
\pgfsetfillcolor{currentfill}%
\pgfsetlinewidth{0.803000pt}%
\definecolor{currentstroke}{rgb}{0.000000,0.000000,0.000000}%
\pgfsetstrokecolor{currentstroke}%
\pgfsetdash{}{0pt}%
\pgfsys@defobject{currentmarker}{\pgfqpoint{-0.048611in}{0.000000in}}{\pgfqpoint{0.000000in}{0.000000in}}{%
\pgfpathmoveto{\pgfqpoint{0.000000in}{0.000000in}}%
\pgfpathlineto{\pgfqpoint{-0.048611in}{0.000000in}}%
\pgfusepath{stroke,fill}%
}%
\begin{pgfscope}%
\pgfsys@transformshift{0.456635in}{3.531813in}%
\pgfsys@useobject{currentmarker}{}%
\end{pgfscope}%
\end{pgfscope}%
\begin{pgfscope}%
\pgftext[x=0.289968in,y=3.479051in,left,base]{\rmfamily\fontsize{10.000000}{12.000000}\selectfont \(\displaystyle 8\)}%
\end{pgfscope}%
\begin{pgfscope}%
\pgftext[x=0.234413in,y=2.031603in,,bottom,rotate=90.000000]{\rmfamily\fontsize{10.000000}{12.000000}\selectfont \(\displaystyle y\) (cm)}%
\end{pgfscope}%
\begin{pgfscope}%
\pgfpathrectangle{\pgfqpoint{0.456635in}{0.521603in}}{\pgfqpoint{3.720000in}{3.020000in}} %
\pgfusepath{clip}%
\pgfsetrectcap%
\pgfsetroundjoin%
\pgfsetlinewidth{1.505625pt}%
\definecolor{currentstroke}{rgb}{0.500000,0.000000,1.000000}%
\pgfsetstrokecolor{currentstroke}%
\pgfsetdash{}{0pt}%
\pgfpathmoveto{\pgfqpoint{0.713566in}{0.953990in}}%
\pgfpathlineto{\pgfqpoint{0.728206in}{0.976249in}}%
\pgfpathlineto{\pgfqpoint{0.742845in}{0.998409in}}%
\pgfpathlineto{\pgfqpoint{0.757485in}{1.029351in}}%
\pgfpathlineto{\pgfqpoint{0.772125in}{1.057658in}}%
\pgfpathlineto{\pgfqpoint{0.786765in}{1.081788in}}%
\pgfpathlineto{\pgfqpoint{0.801405in}{1.106941in}}%
\pgfpathlineto{\pgfqpoint{0.816045in}{1.129560in}}%
\pgfpathlineto{\pgfqpoint{0.830685in}{1.156430in}}%
\pgfpathlineto{\pgfqpoint{0.845325in}{1.185125in}}%
\pgfpathlineto{\pgfqpoint{0.859965in}{1.206949in}}%
\pgfpathlineto{\pgfqpoint{0.874605in}{1.230897in}}%
\pgfpathlineto{\pgfqpoint{0.889244in}{1.260151in}}%
\pgfpathlineto{\pgfqpoint{0.903884in}{1.286379in}}%
\pgfpathlineto{\pgfqpoint{0.918524in}{1.309421in}}%
\pgfpathlineto{\pgfqpoint{0.933164in}{1.329354in}}%
\pgfpathlineto{\pgfqpoint{0.947804in}{1.355500in}}%
\pgfpathlineto{\pgfqpoint{0.962444in}{1.382638in}}%
\pgfpathlineto{\pgfqpoint{0.977084in}{1.408055in}}%
\pgfpathlineto{\pgfqpoint{0.991724in}{1.429796in}}%
\pgfpathlineto{\pgfqpoint{1.006364in}{1.454095in}}%
\pgfpathlineto{\pgfqpoint{1.021004in}{1.480372in}}%
\pgfpathlineto{\pgfqpoint{1.035644in}{1.503842in}}%
\pgfpathlineto{\pgfqpoint{1.050283in}{1.524845in}}%
\pgfpathlineto{\pgfqpoint{1.064923in}{1.548538in}}%
\pgfpathlineto{\pgfqpoint{1.079563in}{1.573337in}}%
\pgfpathlineto{\pgfqpoint{1.094203in}{1.598394in}}%
\pgfpathlineto{\pgfqpoint{1.108843in}{1.618325in}}%
\pgfpathlineto{\pgfqpoint{1.123483in}{1.642140in}}%
\pgfpathlineto{\pgfqpoint{1.138123in}{1.667006in}}%
\pgfpathlineto{\pgfqpoint{1.152763in}{1.690600in}}%
\pgfpathlineto{\pgfqpoint{1.167403in}{1.711890in}}%
\pgfpathlineto{\pgfqpoint{1.182043in}{1.733580in}}%
\pgfpathlineto{\pgfqpoint{1.196683in}{1.757106in}}%
\pgfpathlineto{\pgfqpoint{1.211322in}{1.781324in}}%
\pgfpathlineto{\pgfqpoint{1.225962in}{1.800742in}}%
\pgfpathlineto{\pgfqpoint{1.240602in}{1.824801in}}%
\pgfpathlineto{\pgfqpoint{1.255242in}{1.847458in}}%
\pgfpathlineto{\pgfqpoint{1.269882in}{1.870972in}}%
\pgfpathlineto{\pgfqpoint{1.284522in}{1.892498in}}%
\pgfpathlineto{\pgfqpoint{1.299162in}{1.913503in}}%
\pgfpathlineto{\pgfqpoint{1.313802in}{1.933511in}}%
\pgfpathlineto{\pgfqpoint{1.328442in}{1.957575in}}%
\pgfpathlineto{\pgfqpoint{1.343082in}{1.978541in}}%
\pgfpathlineto{\pgfqpoint{1.357721in}{2.002067in}}%
\pgfpathlineto{\pgfqpoint{1.372361in}{2.024440in}}%
\pgfpathlineto{\pgfqpoint{1.387001in}{2.047513in}}%
\pgfpathlineto{\pgfqpoint{1.401641in}{2.069389in}}%
\pgfpathlineto{\pgfqpoint{1.416281in}{2.088838in}}%
\pgfpathlineto{\pgfqpoint{1.430921in}{2.108017in}}%
\pgfpathlineto{\pgfqpoint{1.445561in}{2.130921in}}%
\pgfpathlineto{\pgfqpoint{1.460201in}{2.152687in}}%
\pgfpathlineto{\pgfqpoint{1.474841in}{2.175094in}}%
\pgfpathlineto{\pgfqpoint{1.489481in}{2.197700in}}%
\pgfpathlineto{\pgfqpoint{1.504121in}{2.220082in}}%
\pgfpathlineto{\pgfqpoint{1.518760in}{2.242097in}}%
\pgfpathlineto{\pgfqpoint{1.533400in}{2.261309in}}%
\pgfpathlineto{\pgfqpoint{1.548040in}{2.279987in}}%
\pgfpathlineto{\pgfqpoint{1.562680in}{2.300611in}}%
\pgfpathlineto{\pgfqpoint{1.577320in}{2.323216in}}%
\pgfpathlineto{\pgfqpoint{1.591960in}{2.345376in}}%
\pgfpathlineto{\pgfqpoint{1.606600in}{2.368865in}}%
\pgfpathlineto{\pgfqpoint{1.621240in}{2.390565in}}%
\pgfpathlineto{\pgfqpoint{1.635880in}{2.412284in}}%
\pgfpathlineto{\pgfqpoint{1.650520in}{2.430994in}}%
\pgfpathlineto{\pgfqpoint{1.665159in}{2.449679in}}%
\pgfpathlineto{\pgfqpoint{1.679799in}{2.469323in}}%
\pgfpathlineto{\pgfqpoint{1.694439in}{2.490954in}}%
\pgfpathlineto{\pgfqpoint{1.709079in}{2.513901in}}%
\pgfpathlineto{\pgfqpoint{1.723719in}{2.537633in}}%
\pgfpathlineto{\pgfqpoint{1.738359in}{2.558787in}}%
\pgfpathlineto{\pgfqpoint{1.752999in}{2.579855in}}%
\pgfpathlineto{\pgfqpoint{1.767639in}{2.598638in}}%
\pgfpathlineto{\pgfqpoint{1.782279in}{2.616273in}}%
\pgfpathlineto{\pgfqpoint{1.796919in}{2.635641in}}%
\pgfpathlineto{\pgfqpoint{1.811559in}{2.656098in}}%
\pgfpathlineto{\pgfqpoint{1.826198in}{2.679336in}}%
\pgfpathlineto{\pgfqpoint{1.840838in}{2.702490in}}%
\pgfpathlineto{\pgfqpoint{1.855478in}{2.724267in}}%
\pgfpathlineto{\pgfqpoint{1.870118in}{2.745367in}}%
\pgfpathlineto{\pgfqpoint{1.884758in}{2.763303in}}%
\pgfpathlineto{\pgfqpoint{1.899398in}{2.780590in}}%
\pgfpathlineto{\pgfqpoint{1.914038in}{2.799896in}}%
\pgfpathlineto{\pgfqpoint{1.928678in}{2.820454in}}%
\pgfpathlineto{\pgfqpoint{1.943318in}{2.842822in}}%
\pgfpathlineto{\pgfqpoint{1.957958in}{2.866414in}}%
\pgfpathlineto{\pgfqpoint{1.972597in}{2.887499in}}%
\pgfpathlineto{\pgfqpoint{1.987237in}{2.907908in}}%
\pgfpathlineto{\pgfqpoint{2.001877in}{2.926565in}}%
\pgfpathlineto{\pgfqpoint{2.016517in}{2.942901in}}%
\pgfpathlineto{\pgfqpoint{2.031157in}{2.961694in}}%
\pgfpathlineto{\pgfqpoint{2.045797in}{2.981693in}}%
\pgfpathlineto{\pgfqpoint{2.060437in}{3.003759in}}%
\pgfpathlineto{\pgfqpoint{2.075077in}{3.027689in}}%
\pgfpathlineto{\pgfqpoint{2.089717in}{3.048612in}}%
\pgfpathlineto{\pgfqpoint{2.104357in}{3.068468in}}%
\pgfpathlineto{\pgfqpoint{2.118997in}{3.088032in}}%
\pgfpathlineto{\pgfqpoint{2.133636in}{3.103928in}}%
\pgfpathlineto{\pgfqpoint{2.148276in}{3.122116in}}%
\pgfpathlineto{\pgfqpoint{2.162916in}{3.142069in}}%
\pgfpathlineto{\pgfqpoint{2.177556in}{3.164506in}}%
\pgfpathlineto{\pgfqpoint{2.192196in}{3.187622in}}%
\pgfpathlineto{\pgfqpoint{2.206836in}{3.207936in}}%
\pgfpathlineto{\pgfqpoint{2.221476in}{3.227375in}}%
\pgfpathlineto{\pgfqpoint{2.236116in}{3.247184in}}%
\pgfpathlineto{\pgfqpoint{2.250756in}{3.263485in}}%
\pgfpathlineto{\pgfqpoint{2.265396in}{3.280831in}}%
\pgfpathlineto{\pgfqpoint{2.280036in}{3.301000in}}%
\pgfpathlineto{\pgfqpoint{2.294675in}{3.323594in}}%
\pgfpathlineto{\pgfqpoint{2.309315in}{3.346447in}}%
\pgfpathlineto{\pgfqpoint{2.323955in}{3.365335in}}%
\pgfpathlineto{\pgfqpoint{2.338595in}{3.386528in}}%
\pgfpathlineto{\pgfqpoint{2.353235in}{3.404331in}}%
\pgfusepath{stroke}%
\end{pgfscope}%
\begin{pgfscope}%
\pgfpathrectangle{\pgfqpoint{0.456635in}{0.521603in}}{\pgfqpoint{3.720000in}{3.020000in}} %
\pgfusepath{clip}%
\pgfsetrectcap%
\pgfsetroundjoin%
\pgfsetlinewidth{1.505625pt}%
\definecolor{currentstroke}{rgb}{0.182353,0.878081,0.859800}%
\pgfsetstrokecolor{currentstroke}%
\pgfsetdash{}{0pt}%
\pgfpathmoveto{\pgfqpoint{0.742845in}{0.933978in}}%
\pgfpathlineto{\pgfqpoint{0.757485in}{0.941019in}}%
\pgfpathlineto{\pgfqpoint{0.772125in}{0.973343in}}%
\pgfpathlineto{\pgfqpoint{0.786765in}{1.000368in}}%
\pgfpathlineto{\pgfqpoint{0.801405in}{1.016000in}}%
\pgfpathlineto{\pgfqpoint{0.816045in}{1.035824in}}%
\pgfpathlineto{\pgfqpoint{0.830685in}{1.060113in}}%
\pgfpathlineto{\pgfqpoint{0.845325in}{1.088076in}}%
\pgfpathlineto{\pgfqpoint{0.859965in}{1.107301in}}%
\pgfpathlineto{\pgfqpoint{0.874605in}{1.123489in}}%
\pgfpathlineto{\pgfqpoint{0.889244in}{1.145334in}}%
\pgfpathlineto{\pgfqpoint{0.903884in}{1.171042in}}%
\pgfpathlineto{\pgfqpoint{0.918524in}{1.190507in}}%
\pgfpathlineto{\pgfqpoint{0.947804in}{1.223946in}}%
\pgfpathlineto{\pgfqpoint{0.962444in}{1.245026in}}%
\pgfpathlineto{\pgfqpoint{0.977084in}{1.269296in}}%
\pgfpathlineto{\pgfqpoint{0.991724in}{1.287106in}}%
\pgfpathlineto{\pgfqpoint{1.006364in}{1.300033in}}%
\pgfpathlineto{\pgfqpoint{1.021004in}{1.319799in}}%
\pgfpathlineto{\pgfqpoint{1.035644in}{1.342371in}}%
\pgfpathlineto{\pgfqpoint{1.050283in}{1.360156in}}%
\pgfpathlineto{\pgfqpoint{1.079563in}{1.389101in}}%
\pgfpathlineto{\pgfqpoint{1.108843in}{1.429204in}}%
\pgfpathlineto{\pgfqpoint{1.123483in}{1.445130in}}%
\pgfpathlineto{\pgfqpoint{1.138123in}{1.456741in}}%
\pgfpathlineto{\pgfqpoint{1.182043in}{1.510897in}}%
\pgfpathlineto{\pgfqpoint{1.211322in}{1.536678in}}%
\pgfpathlineto{\pgfqpoint{1.225962in}{1.553608in}}%
\pgfpathlineto{\pgfqpoint{1.240602in}{1.573048in}}%
\pgfpathlineto{\pgfqpoint{1.255242in}{1.585839in}}%
\pgfpathlineto{\pgfqpoint{1.269882in}{1.597138in}}%
\pgfpathlineto{\pgfqpoint{1.284522in}{1.613020in}}%
\pgfpathlineto{\pgfqpoint{1.299162in}{1.631021in}}%
\pgfpathlineto{\pgfqpoint{1.313802in}{1.645608in}}%
\pgfpathlineto{\pgfqpoint{1.343082in}{1.668732in}}%
\pgfpathlineto{\pgfqpoint{1.357721in}{1.684388in}}%
\pgfpathlineto{\pgfqpoint{1.372361in}{1.701768in}}%
\pgfpathlineto{\pgfqpoint{1.387001in}{1.713490in}}%
\pgfpathlineto{\pgfqpoint{1.401641in}{1.723187in}}%
\pgfpathlineto{\pgfqpoint{1.445561in}{1.767723in}}%
\pgfpathlineto{\pgfqpoint{1.474841in}{1.788706in}}%
\pgfpathlineto{\pgfqpoint{1.489481in}{1.802843in}}%
\pgfpathlineto{\pgfqpoint{1.504121in}{1.818774in}}%
\pgfpathlineto{\pgfqpoint{1.533400in}{1.838593in}}%
\pgfpathlineto{\pgfqpoint{1.577320in}{1.878987in}}%
\pgfpathlineto{\pgfqpoint{1.606600in}{1.898229in}}%
\pgfpathlineto{\pgfqpoint{1.621240in}{1.911232in}}%
\pgfpathlineto{\pgfqpoint{1.635880in}{1.925897in}}%
\pgfpathlineto{\pgfqpoint{1.665159in}{1.944092in}}%
\pgfpathlineto{\pgfqpoint{1.679799in}{1.956146in}}%
\pgfpathlineto{\pgfqpoint{1.694439in}{1.969727in}}%
\pgfpathlineto{\pgfqpoint{1.709079in}{1.981062in}}%
\pgfpathlineto{\pgfqpoint{1.738359in}{1.998867in}}%
\pgfpathlineto{\pgfqpoint{1.767639in}{2.024330in}}%
\pgfpathlineto{\pgfqpoint{1.796919in}{2.041296in}}%
\pgfpathlineto{\pgfqpoint{1.840838in}{2.075478in}}%
\pgfpathlineto{\pgfqpoint{1.870118in}{2.091891in}}%
\pgfpathlineto{\pgfqpoint{1.899398in}{2.115390in}}%
\pgfpathlineto{\pgfqpoint{1.928678in}{2.131108in}}%
\pgfpathlineto{\pgfqpoint{1.972597in}{2.162623in}}%
\pgfpathlineto{\pgfqpoint{2.001877in}{2.177798in}}%
\pgfpathlineto{\pgfqpoint{2.031157in}{2.199360in}}%
\pgfpathlineto{\pgfqpoint{2.060437in}{2.214157in}}%
\pgfpathlineto{\pgfqpoint{2.104357in}{2.243302in}}%
\pgfpathlineto{\pgfqpoint{2.118997in}{2.249723in}}%
\pgfpathlineto{\pgfqpoint{2.133636in}{2.257683in}}%
\pgfpathlineto{\pgfqpoint{2.148276in}{2.268149in}}%
\pgfpathlineto{\pgfqpoint{2.162916in}{2.277173in}}%
\pgfpathlineto{\pgfqpoint{2.192196in}{2.291301in}}%
\pgfpathlineto{\pgfqpoint{2.236116in}{2.317978in}}%
\pgfpathlineto{\pgfqpoint{2.250756in}{2.324052in}}%
\pgfpathlineto{\pgfqpoint{2.265396in}{2.331888in}}%
\pgfpathlineto{\pgfqpoint{2.294675in}{2.349540in}}%
\pgfpathlineto{\pgfqpoint{2.323955in}{2.362874in}}%
\pgfpathlineto{\pgfqpoint{2.367875in}{2.387341in}}%
\pgfpathlineto{\pgfqpoint{2.382515in}{2.393125in}}%
\pgfpathlineto{\pgfqpoint{2.397155in}{2.400387in}}%
\pgfpathlineto{\pgfqpoint{2.426435in}{2.416978in}}%
\pgfpathlineto{\pgfqpoint{2.455714in}{2.429330in}}%
\pgfpathlineto{\pgfqpoint{2.499634in}{2.452189in}}%
\pgfpathlineto{\pgfqpoint{2.514274in}{2.457637in}}%
\pgfpathlineto{\pgfqpoint{2.528914in}{2.464404in}}%
\pgfpathlineto{\pgfqpoint{2.543554in}{2.472880in}}%
\pgfpathlineto{\pgfqpoint{2.558194in}{2.479991in}}%
\pgfpathlineto{\pgfqpoint{2.587474in}{2.491712in}}%
\pgfpathlineto{\pgfqpoint{2.616753in}{2.506825in}}%
\pgfpathlineto{\pgfqpoint{2.660673in}{2.524554in}}%
\pgfpathlineto{\pgfqpoint{2.675313in}{2.532584in}}%
\pgfpathlineto{\pgfqpoint{2.689953in}{2.539155in}}%
\pgfpathlineto{\pgfqpoint{2.719233in}{2.550014in}}%
\pgfpathlineto{\pgfqpoint{2.748512in}{2.564240in}}%
\pgfpathlineto{\pgfqpoint{2.792432in}{2.580848in}}%
\pgfpathlineto{\pgfqpoint{2.807072in}{2.588388in}}%
\pgfpathlineto{\pgfqpoint{2.836352in}{2.599127in}}%
\pgfpathlineto{\pgfqpoint{2.850992in}{2.604424in}}%
\pgfpathlineto{\pgfqpoint{2.880272in}{2.617634in}}%
\pgfpathlineto{\pgfqpoint{2.909551in}{2.627015in}}%
\pgfpathlineto{\pgfqpoint{2.953471in}{2.645568in}}%
\pgfpathlineto{\pgfqpoint{2.982751in}{2.655140in}}%
\pgfpathlineto{\pgfqpoint{3.012031in}{2.667409in}}%
\pgfpathlineto{\pgfqpoint{3.041311in}{2.676193in}}%
\pgfpathlineto{\pgfqpoint{3.070590in}{2.688902in}}%
\pgfpathlineto{\pgfqpoint{3.129150in}{2.708443in}}%
\pgfpathlineto{\pgfqpoint{3.143790in}{2.713979in}}%
\pgfpathlineto{\pgfqpoint{3.173070in}{2.722477in}}%
\pgfpathlineto{\pgfqpoint{3.202350in}{2.734136in}}%
\pgfpathlineto{\pgfqpoint{3.260909in}{2.752555in}}%
\pgfpathlineto{\pgfqpoint{3.275549in}{2.757441in}}%
\pgfpathlineto{\pgfqpoint{3.304829in}{2.765599in}}%
\pgfpathlineto{\pgfqpoint{3.334109in}{2.776470in}}%
\pgfpathlineto{\pgfqpoint{3.378028in}{2.788514in}}%
\pgfpathlineto{\pgfqpoint{3.407308in}{2.797945in}}%
\pgfpathlineto{\pgfqpoint{3.436588in}{2.805763in}}%
\pgfpathlineto{\pgfqpoint{3.465868in}{2.815708in}}%
\pgfpathlineto{\pgfqpoint{3.553707in}{2.838818in}}%
\pgfpathlineto{\pgfqpoint{3.612267in}{2.855578in}}%
\pgfpathlineto{\pgfqpoint{3.641547in}{2.862772in}}%
\pgfpathlineto{\pgfqpoint{3.670827in}{2.870655in}}%
\pgfpathlineto{\pgfqpoint{3.700106in}{2.878019in}}%
\pgfpathlineto{\pgfqpoint{3.729386in}{2.886112in}}%
\pgfpathlineto{\pgfqpoint{3.758666in}{2.892053in}}%
\pgfpathlineto{\pgfqpoint{3.787946in}{2.900089in}}%
\pgfpathlineto{\pgfqpoint{3.831866in}{2.910066in}}%
\pgfpathlineto{\pgfqpoint{3.861145in}{2.917565in}}%
\pgfpathlineto{\pgfqpoint{3.905065in}{2.926960in}}%
\pgfpathlineto{\pgfqpoint{3.992904in}{2.946717in}}%
\pgfpathlineto{\pgfqpoint{4.007544in}{2.948929in}}%
\pgfpathlineto{\pgfqpoint{4.007544in}{2.948929in}}%
\pgfusepath{stroke}%
\end{pgfscope}%
\begin{pgfscope}%
\pgfpathrectangle{\pgfqpoint{0.456635in}{0.521603in}}{\pgfqpoint{3.720000in}{3.020000in}} %
\pgfusepath{clip}%
\pgfsetrectcap%
\pgfsetroundjoin%
\pgfsetlinewidth{1.505625pt}%
\definecolor{currentstroke}{rgb}{0.327451,0.267733,0.990831}%
\pgfsetstrokecolor{currentstroke}%
\pgfsetdash{}{0pt}%
\pgfpathmoveto{\pgfqpoint{0.669646in}{0.811525in}}%
\pgfpathlineto{\pgfqpoint{0.698926in}{0.865918in}}%
\pgfpathlineto{\pgfqpoint{0.728206in}{0.908725in}}%
\pgfpathlineto{\pgfqpoint{0.757485in}{0.963482in}}%
\pgfpathlineto{\pgfqpoint{0.801405in}{1.027780in}}%
\pgfpathlineto{\pgfqpoint{0.830685in}{1.080571in}}%
\pgfpathlineto{\pgfqpoint{0.845325in}{1.098492in}}%
\pgfpathlineto{\pgfqpoint{0.859965in}{1.119449in}}%
\pgfpathlineto{\pgfqpoint{0.889244in}{1.170686in}}%
\pgfpathlineto{\pgfqpoint{0.918524in}{1.208540in}}%
\pgfpathlineto{\pgfqpoint{0.933164in}{1.230544in}}%
\pgfpathlineto{\pgfqpoint{0.947804in}{1.256586in}}%
\pgfpathlineto{\pgfqpoint{0.962444in}{1.278190in}}%
\pgfpathlineto{\pgfqpoint{0.977084in}{1.295522in}}%
\pgfpathlineto{\pgfqpoint{0.991724in}{1.316035in}}%
\pgfpathlineto{\pgfqpoint{1.006364in}{1.340894in}}%
\pgfpathlineto{\pgfqpoint{1.021004in}{1.362977in}}%
\pgfpathlineto{\pgfqpoint{1.050283in}{1.399015in}}%
\pgfpathlineto{\pgfqpoint{1.094203in}{1.463863in}}%
\pgfpathlineto{\pgfqpoint{1.108843in}{1.480742in}}%
\pgfpathlineto{\pgfqpoint{1.123483in}{1.501349in}}%
\pgfpathlineto{\pgfqpoint{1.138123in}{1.524051in}}%
\pgfpathlineto{\pgfqpoint{1.152763in}{1.544073in}}%
\pgfpathlineto{\pgfqpoint{1.182043in}{1.579144in}}%
\pgfpathlineto{\pgfqpoint{1.211322in}{1.621910in}}%
\pgfpathlineto{\pgfqpoint{1.240602in}{1.656144in}}%
\pgfpathlineto{\pgfqpoint{1.284522in}{1.716388in}}%
\pgfpathlineto{\pgfqpoint{1.313802in}{1.750962in}}%
\pgfpathlineto{\pgfqpoint{1.343082in}{1.790981in}}%
\pgfpathlineto{\pgfqpoint{1.372361in}{1.824423in}}%
\pgfpathlineto{\pgfqpoint{1.401641in}{1.864406in}}%
\pgfpathlineto{\pgfqpoint{1.445561in}{1.915671in}}%
\pgfpathlineto{\pgfqpoint{1.460201in}{1.935707in}}%
\pgfpathlineto{\pgfqpoint{1.606600in}{2.109960in}}%
\pgfpathlineto{\pgfqpoint{1.621240in}{2.125842in}}%
\pgfpathlineto{\pgfqpoint{1.679799in}{2.195059in}}%
\pgfpathlineto{\pgfqpoint{1.709079in}{2.229423in}}%
\pgfpathlineto{\pgfqpoint{1.723719in}{2.246998in}}%
\pgfpathlineto{\pgfqpoint{1.752999in}{2.278617in}}%
\pgfpathlineto{\pgfqpoint{1.796919in}{2.330318in}}%
\pgfpathlineto{\pgfqpoint{1.826198in}{2.362395in}}%
\pgfpathlineto{\pgfqpoint{1.855478in}{2.396475in}}%
\pgfpathlineto{\pgfqpoint{1.884758in}{2.427983in}}%
\pgfpathlineto{\pgfqpoint{1.914038in}{2.462437in}}%
\pgfpathlineto{\pgfqpoint{1.957958in}{2.510295in}}%
\pgfpathlineto{\pgfqpoint{1.987237in}{2.542933in}}%
\pgfpathlineto{\pgfqpoint{2.016517in}{2.574534in}}%
\pgfpathlineto{\pgfqpoint{2.045797in}{2.607833in}}%
\pgfpathlineto{\pgfqpoint{2.075077in}{2.638864in}}%
\pgfpathlineto{\pgfqpoint{2.118997in}{2.686924in}}%
\pgfpathlineto{\pgfqpoint{2.148276in}{2.718456in}}%
\pgfpathlineto{\pgfqpoint{2.177556in}{2.750600in}}%
\pgfpathlineto{\pgfqpoint{2.221476in}{2.797740in}}%
\pgfpathlineto{\pgfqpoint{2.309315in}{2.891284in}}%
\pgfpathlineto{\pgfqpoint{2.689953in}{3.287956in}}%
\pgfpathlineto{\pgfqpoint{2.704593in}{3.302659in}}%
\pgfpathlineto{\pgfqpoint{2.704593in}{3.302659in}}%
\pgfusepath{stroke}%
\end{pgfscope}%
\begin{pgfscope}%
\pgfpathrectangle{\pgfqpoint{0.456635in}{0.521603in}}{\pgfqpoint{3.720000in}{3.020000in}} %
\pgfusepath{clip}%
\pgfsetrectcap%
\pgfsetroundjoin%
\pgfsetlinewidth{1.505625pt}%
\definecolor{currentstroke}{rgb}{0.221569,0.905873,0.843667}%
\pgfsetstrokecolor{currentstroke}%
\pgfsetdash{}{0pt}%
\pgfpathmoveto{\pgfqpoint{0.698926in}{0.851858in}}%
\pgfpathlineto{\pgfqpoint{0.713566in}{0.864296in}}%
\pgfpathlineto{\pgfqpoint{0.728206in}{0.883058in}}%
\pgfpathlineto{\pgfqpoint{0.742845in}{0.918177in}}%
\pgfpathlineto{\pgfqpoint{0.757485in}{0.938144in}}%
\pgfpathlineto{\pgfqpoint{0.772125in}{0.946161in}}%
\pgfpathlineto{\pgfqpoint{0.801405in}{1.000800in}}%
\pgfpathlineto{\pgfqpoint{0.816045in}{1.017281in}}%
\pgfpathlineto{\pgfqpoint{0.830685in}{1.031370in}}%
\pgfpathlineto{\pgfqpoint{0.859965in}{1.075645in}}%
\pgfpathlineto{\pgfqpoint{0.903884in}{1.124870in}}%
\pgfpathlineto{\pgfqpoint{0.918524in}{1.145932in}}%
\pgfpathlineto{\pgfqpoint{0.933164in}{1.161308in}}%
\pgfpathlineto{\pgfqpoint{0.947804in}{1.174899in}}%
\pgfpathlineto{\pgfqpoint{0.977084in}{1.211593in}}%
\pgfpathlineto{\pgfqpoint{1.006364in}{1.238733in}}%
\pgfpathlineto{\pgfqpoint{1.035644in}{1.272978in}}%
\pgfpathlineto{\pgfqpoint{1.064923in}{1.298607in}}%
\pgfpathlineto{\pgfqpoint{1.079563in}{1.315231in}}%
\pgfpathlineto{\pgfqpoint{1.094203in}{1.330100in}}%
\pgfpathlineto{\pgfqpoint{1.123483in}{1.353921in}}%
\pgfpathlineto{\pgfqpoint{1.152763in}{1.383467in}}%
\pgfpathlineto{\pgfqpoint{1.167403in}{1.393333in}}%
\pgfpathlineto{\pgfqpoint{1.182043in}{1.406430in}}%
\pgfpathlineto{\pgfqpoint{1.196683in}{1.421021in}}%
\pgfpathlineto{\pgfqpoint{1.211322in}{1.433027in}}%
\pgfpathlineto{\pgfqpoint{1.225962in}{1.442974in}}%
\pgfpathlineto{\pgfqpoint{1.240602in}{1.454708in}}%
\pgfpathlineto{\pgfqpoint{1.255242in}{1.468745in}}%
\pgfpathlineto{\pgfqpoint{1.284522in}{1.488491in}}%
\pgfpathlineto{\pgfqpoint{1.313802in}{1.512982in}}%
\pgfpathlineto{\pgfqpoint{1.357721in}{1.543011in}}%
\pgfpathlineto{\pgfqpoint{1.372361in}{1.554431in}}%
\pgfpathlineto{\pgfqpoint{1.387001in}{1.562419in}}%
\pgfpathlineto{\pgfqpoint{1.401641in}{1.571848in}}%
\pgfpathlineto{\pgfqpoint{1.430921in}{1.592672in}}%
\pgfpathlineto{\pgfqpoint{1.460201in}{1.608960in}}%
\pgfpathlineto{\pgfqpoint{1.474841in}{1.619591in}}%
\pgfpathlineto{\pgfqpoint{1.489481in}{1.628442in}}%
\pgfpathlineto{\pgfqpoint{1.504121in}{1.635377in}}%
\pgfpathlineto{\pgfqpoint{1.548040in}{1.661700in}}%
\pgfpathlineto{\pgfqpoint{1.562680in}{1.668304in}}%
\pgfpathlineto{\pgfqpoint{1.606600in}{1.692133in}}%
\pgfpathlineto{\pgfqpoint{1.621240in}{1.698642in}}%
\pgfpathlineto{\pgfqpoint{1.650520in}{1.714561in}}%
\pgfpathlineto{\pgfqpoint{1.679799in}{1.726649in}}%
\pgfpathlineto{\pgfqpoint{1.709079in}{1.741425in}}%
\pgfpathlineto{\pgfqpoint{1.738359in}{1.752576in}}%
\pgfpathlineto{\pgfqpoint{1.767639in}{1.765916in}}%
\pgfpathlineto{\pgfqpoint{1.796919in}{1.776319in}}%
\pgfpathlineto{\pgfqpoint{1.811559in}{1.782916in}}%
\pgfpathlineto{\pgfqpoint{1.884758in}{1.808662in}}%
\pgfpathlineto{\pgfqpoint{1.914038in}{1.817611in}}%
\pgfpathlineto{\pgfqpoint{1.928678in}{1.822955in}}%
\pgfpathlineto{\pgfqpoint{1.972597in}{1.835156in}}%
\pgfpathlineto{\pgfqpoint{2.001877in}{1.843219in}}%
\pgfpathlineto{\pgfqpoint{2.016517in}{1.846409in}}%
\pgfpathlineto{\pgfqpoint{2.045797in}{1.854518in}}%
\pgfpathlineto{\pgfqpoint{2.089717in}{1.864300in}}%
\pgfpathlineto{\pgfqpoint{2.118997in}{1.869836in}}%
\pgfpathlineto{\pgfqpoint{2.206836in}{1.885592in}}%
\pgfpathlineto{\pgfqpoint{2.265396in}{1.893569in}}%
\pgfpathlineto{\pgfqpoint{2.397155in}{1.904005in}}%
\pgfpathlineto{\pgfqpoint{2.455714in}{1.905659in}}%
\pgfpathlineto{\pgfqpoint{2.514274in}{1.905938in}}%
\pgfpathlineto{\pgfqpoint{2.558194in}{1.904842in}}%
\pgfpathlineto{\pgfqpoint{2.660673in}{1.898446in}}%
\pgfpathlineto{\pgfqpoint{2.777792in}{1.883829in}}%
\pgfpathlineto{\pgfqpoint{2.924191in}{1.854476in}}%
\pgfpathlineto{\pgfqpoint{2.953471in}{1.847007in}}%
\pgfpathlineto{\pgfqpoint{3.041311in}{1.821777in}}%
\pgfpathlineto{\pgfqpoint{3.099870in}{1.802002in}}%
\pgfpathlineto{\pgfqpoint{3.173070in}{1.773770in}}%
\pgfpathlineto{\pgfqpoint{3.260909in}{1.734878in}}%
\pgfpathlineto{\pgfqpoint{3.319469in}{1.705692in}}%
\pgfpathlineto{\pgfqpoint{3.378028in}{1.673804in}}%
\pgfpathlineto{\pgfqpoint{3.451228in}{1.629885in}}%
\pgfpathlineto{\pgfqpoint{3.524427in}{1.581609in}}%
\pgfpathlineto{\pgfqpoint{3.568347in}{1.550435in}}%
\pgfpathlineto{\pgfqpoint{3.612267in}{1.517747in}}%
\pgfpathlineto{\pgfqpoint{3.670827in}{1.471263in}}%
\pgfpathlineto{\pgfqpoint{3.714746in}{1.434195in}}%
\pgfpathlineto{\pgfqpoint{3.773306in}{1.382222in}}%
\pgfpathlineto{\pgfqpoint{3.831866in}{1.326426in}}%
\pgfpathlineto{\pgfqpoint{3.890425in}{1.267035in}}%
\pgfpathlineto{\pgfqpoint{3.948985in}{1.203506in}}%
\pgfpathlineto{\pgfqpoint{3.992904in}{1.152998in}}%
\pgfpathlineto{\pgfqpoint{3.992904in}{1.152998in}}%
\pgfusepath{stroke}%
\end{pgfscope}%
\begin{pgfscope}%
\pgfpathrectangle{\pgfqpoint{0.456635in}{0.521603in}}{\pgfqpoint{3.720000in}{3.020000in}} %
\pgfusepath{clip}%
\pgfsetrectcap%
\pgfsetroundjoin%
\pgfsetlinewidth{1.505625pt}%
\definecolor{currentstroke}{rgb}{0.676471,0.961826,0.602635}%
\pgfsetstrokecolor{currentstroke}%
\pgfsetdash{}{0pt}%
\pgfpathmoveto{\pgfqpoint{0.625726in}{0.723991in}}%
\pgfpathlineto{\pgfqpoint{0.640366in}{0.769626in}}%
\pgfpathlineto{\pgfqpoint{0.655006in}{0.798991in}}%
\pgfpathlineto{\pgfqpoint{0.669646in}{0.818221in}}%
\pgfpathlineto{\pgfqpoint{0.684286in}{0.841608in}}%
\pgfpathlineto{\pgfqpoint{0.698926in}{0.877155in}}%
\pgfpathlineto{\pgfqpoint{0.713566in}{0.907224in}}%
\pgfpathlineto{\pgfqpoint{0.742845in}{0.946129in}}%
\pgfpathlineto{\pgfqpoint{0.757485in}{0.973013in}}%
\pgfpathlineto{\pgfqpoint{0.772125in}{0.991182in}}%
\pgfpathlineto{\pgfqpoint{0.786765in}{1.015200in}}%
\pgfpathlineto{\pgfqpoint{0.801405in}{1.032217in}}%
\pgfpathlineto{\pgfqpoint{0.816045in}{1.045872in}}%
\pgfpathlineto{\pgfqpoint{0.830685in}{1.064878in}}%
\pgfpathlineto{\pgfqpoint{0.845325in}{1.089004in}}%
\pgfpathlineto{\pgfqpoint{0.859965in}{1.108565in}}%
\pgfpathlineto{\pgfqpoint{0.889244in}{1.135246in}}%
\pgfpathlineto{\pgfqpoint{0.918524in}{1.174356in}}%
\pgfpathlineto{\pgfqpoint{0.933164in}{1.186982in}}%
\pgfpathlineto{\pgfqpoint{0.947804in}{1.196767in}}%
\pgfpathlineto{\pgfqpoint{0.962444in}{1.208042in}}%
\pgfpathlineto{\pgfqpoint{0.991724in}{1.242442in}}%
\pgfpathlineto{\pgfqpoint{1.006364in}{1.252412in}}%
\pgfpathlineto{\pgfqpoint{1.021004in}{1.259485in}}%
\pgfpathlineto{\pgfqpoint{1.035644in}{1.271790in}}%
\pgfpathlineto{\pgfqpoint{1.050283in}{1.287133in}}%
\pgfpathlineto{\pgfqpoint{1.064923in}{1.298516in}}%
\pgfpathlineto{\pgfqpoint{1.094203in}{1.310900in}}%
\pgfpathlineto{\pgfqpoint{1.108843in}{1.321669in}}%
\pgfpathlineto{\pgfqpoint{1.123483in}{1.334681in}}%
\pgfpathlineto{\pgfqpoint{1.138123in}{1.343697in}}%
\pgfpathlineto{\pgfqpoint{1.152763in}{1.346561in}}%
\pgfpathlineto{\pgfqpoint{1.167403in}{1.353440in}}%
\pgfpathlineto{\pgfqpoint{1.182043in}{1.364228in}}%
\pgfpathlineto{\pgfqpoint{1.196683in}{1.372868in}}%
\pgfpathlineto{\pgfqpoint{1.211322in}{1.377271in}}%
\pgfpathlineto{\pgfqpoint{1.225962in}{1.379321in}}%
\pgfpathlineto{\pgfqpoint{1.240602in}{1.384414in}}%
\pgfpathlineto{\pgfqpoint{1.269882in}{1.399266in}}%
\pgfpathlineto{\pgfqpoint{1.284522in}{1.399731in}}%
\pgfpathlineto{\pgfqpoint{1.299162in}{1.401507in}}%
\pgfpathlineto{\pgfqpoint{1.328442in}{1.413073in}}%
\pgfpathlineto{\pgfqpoint{1.343082in}{1.415304in}}%
\pgfpathlineto{\pgfqpoint{1.357721in}{1.414213in}}%
\pgfpathlineto{\pgfqpoint{1.372361in}{1.414503in}}%
\pgfpathlineto{\pgfqpoint{1.401641in}{1.422141in}}%
\pgfpathlineto{\pgfqpoint{1.430921in}{1.418317in}}%
\pgfpathlineto{\pgfqpoint{1.445561in}{1.419228in}}%
\pgfpathlineto{\pgfqpoint{1.460201in}{1.421553in}}%
\pgfpathlineto{\pgfqpoint{1.474841in}{1.421434in}}%
\pgfpathlineto{\pgfqpoint{1.504121in}{1.414074in}}%
\pgfpathlineto{\pgfqpoint{1.518760in}{1.413015in}}%
\pgfpathlineto{\pgfqpoint{1.533400in}{1.413715in}}%
\pgfpathlineto{\pgfqpoint{1.548040in}{1.410014in}}%
\pgfpathlineto{\pgfqpoint{1.562680in}{1.403926in}}%
\pgfpathlineto{\pgfqpoint{1.577320in}{1.400369in}}%
\pgfpathlineto{\pgfqpoint{1.591960in}{1.398974in}}%
\pgfpathlineto{\pgfqpoint{1.606600in}{1.395960in}}%
\pgfpathlineto{\pgfqpoint{1.650520in}{1.376649in}}%
\pgfpathlineto{\pgfqpoint{1.665159in}{1.373407in}}%
\pgfpathlineto{\pgfqpoint{1.679799in}{1.367237in}}%
\pgfpathlineto{\pgfqpoint{1.694439in}{1.358001in}}%
\pgfpathlineto{\pgfqpoint{1.709079in}{1.350196in}}%
\pgfpathlineto{\pgfqpoint{1.738359in}{1.338091in}}%
\pgfpathlineto{\pgfqpoint{1.752999in}{1.328956in}}%
\pgfpathlineto{\pgfqpoint{1.782279in}{1.307966in}}%
\pgfpathlineto{\pgfqpoint{1.796919in}{1.300642in}}%
\pgfpathlineto{\pgfqpoint{1.811559in}{1.291365in}}%
\pgfpathlineto{\pgfqpoint{1.840838in}{1.267059in}}%
\pgfpathlineto{\pgfqpoint{1.884758in}{1.233923in}}%
\pgfpathlineto{\pgfqpoint{1.914038in}{1.204529in}}%
\pgfpathlineto{\pgfqpoint{1.943318in}{1.178919in}}%
\pgfpathlineto{\pgfqpoint{2.016517in}{1.099160in}}%
\pgfpathlineto{\pgfqpoint{2.089717in}{1.003535in}}%
\pgfpathlineto{\pgfqpoint{2.148276in}{0.916469in}}%
\pgfpathlineto{\pgfqpoint{2.206836in}{0.816016in}}%
\pgfpathlineto{\pgfqpoint{2.236116in}{0.758762in}}%
\pgfpathlineto{\pgfqpoint{2.250756in}{0.739752in}}%
\pgfpathlineto{\pgfqpoint{2.280036in}{0.706330in}}%
\pgfpathlineto{\pgfqpoint{2.294675in}{0.700062in}}%
\pgfpathlineto{\pgfqpoint{2.309315in}{0.704587in}}%
\pgfpathlineto{\pgfqpoint{2.323955in}{0.713714in}}%
\pgfpathlineto{\pgfqpoint{2.338595in}{0.729888in}}%
\pgfpathlineto{\pgfqpoint{2.353235in}{0.752111in}}%
\pgfpathlineto{\pgfqpoint{2.367875in}{0.770814in}}%
\pgfpathlineto{\pgfqpoint{2.382515in}{0.800461in}}%
\pgfpathlineto{\pgfqpoint{2.411795in}{0.844837in}}%
\pgfpathlineto{\pgfqpoint{2.426435in}{0.861958in}}%
\pgfpathlineto{\pgfqpoint{2.441074in}{0.877195in}}%
\pgfpathlineto{\pgfqpoint{2.470354in}{0.911880in}}%
\pgfpathlineto{\pgfqpoint{2.484994in}{0.926760in}}%
\pgfpathlineto{\pgfqpoint{2.528914in}{0.966476in}}%
\pgfpathlineto{\pgfqpoint{2.543554in}{0.979890in}}%
\pgfpathlineto{\pgfqpoint{2.572834in}{0.998856in}}%
\pgfpathlineto{\pgfqpoint{2.587474in}{1.008032in}}%
\pgfpathlineto{\pgfqpoint{2.602113in}{1.018763in}}%
\pgfpathlineto{\pgfqpoint{2.616753in}{1.026908in}}%
\pgfpathlineto{\pgfqpoint{2.660673in}{1.047001in}}%
\pgfpathlineto{\pgfqpoint{2.675313in}{1.054076in}}%
\pgfpathlineto{\pgfqpoint{2.689953in}{1.058535in}}%
\pgfpathlineto{\pgfqpoint{2.719233in}{1.064352in}}%
\pgfpathlineto{\pgfqpoint{2.733873in}{1.068835in}}%
\pgfpathlineto{\pgfqpoint{2.748512in}{1.071944in}}%
\pgfpathlineto{\pgfqpoint{2.792432in}{1.074202in}}%
\pgfpathlineto{\pgfqpoint{2.807072in}{1.075131in}}%
\pgfpathlineto{\pgfqpoint{2.821712in}{1.074885in}}%
\pgfpathlineto{\pgfqpoint{2.850992in}{1.069395in}}%
\pgfpathlineto{\pgfqpoint{2.880272in}{1.065350in}}%
\pgfpathlineto{\pgfqpoint{2.953471in}{1.039470in}}%
\pgfpathlineto{\pgfqpoint{3.012031in}{1.006229in}}%
\pgfpathlineto{\pgfqpoint{3.041311in}{0.983998in}}%
\pgfpathlineto{\pgfqpoint{3.085230in}{0.948516in}}%
\pgfpathlineto{\pgfqpoint{3.114510in}{0.918175in}}%
\pgfpathlineto{\pgfqpoint{3.143790in}{0.886120in}}%
\pgfpathlineto{\pgfqpoint{3.173070in}{0.848274in}}%
\pgfpathlineto{\pgfqpoint{3.202350in}{0.806122in}}%
\pgfpathlineto{\pgfqpoint{3.231629in}{0.758749in}}%
\pgfpathlineto{\pgfqpoint{3.246269in}{0.737224in}}%
\pgfpathlineto{\pgfqpoint{3.260909in}{0.719826in}}%
\pgfpathlineto{\pgfqpoint{3.275549in}{0.706124in}}%
\pgfpathlineto{\pgfqpoint{3.290189in}{0.705328in}}%
\pgfpathlineto{\pgfqpoint{3.304829in}{0.713343in}}%
\pgfpathlineto{\pgfqpoint{3.319469in}{0.729921in}}%
\pgfpathlineto{\pgfqpoint{3.348749in}{0.771368in}}%
\pgfpathlineto{\pgfqpoint{3.363389in}{0.793717in}}%
\pgfpathlineto{\pgfqpoint{3.378028in}{0.812499in}}%
\pgfpathlineto{\pgfqpoint{3.392668in}{0.835521in}}%
\pgfpathlineto{\pgfqpoint{3.407308in}{0.854257in}}%
\pgfpathlineto{\pgfqpoint{3.436588in}{0.884432in}}%
\pgfpathlineto{\pgfqpoint{3.451228in}{0.895825in}}%
\pgfpathlineto{\pgfqpoint{3.480508in}{0.921222in}}%
\pgfpathlineto{\pgfqpoint{3.495148in}{0.932392in}}%
\pgfpathlineto{\pgfqpoint{3.553707in}{0.966072in}}%
\pgfpathlineto{\pgfqpoint{3.568347in}{0.971496in}}%
\pgfpathlineto{\pgfqpoint{3.597627in}{0.979862in}}%
\pgfpathlineto{\pgfqpoint{3.612267in}{0.985311in}}%
\pgfpathlineto{\pgfqpoint{3.626907in}{0.988750in}}%
\pgfpathlineto{\pgfqpoint{3.685466in}{0.993763in}}%
\pgfpathlineto{\pgfqpoint{3.700106in}{0.992602in}}%
\pgfpathlineto{\pgfqpoint{3.729386in}{0.987230in}}%
\pgfpathlineto{\pgfqpoint{3.744026in}{0.985985in}}%
\pgfpathlineto{\pgfqpoint{3.758666in}{0.982802in}}%
\pgfpathlineto{\pgfqpoint{3.773306in}{0.978073in}}%
\pgfpathlineto{\pgfqpoint{3.831866in}{0.952877in}}%
\pgfpathlineto{\pgfqpoint{3.875785in}{0.924402in}}%
\pgfpathlineto{\pgfqpoint{3.890425in}{0.914319in}}%
\pgfpathlineto{\pgfqpoint{3.905065in}{0.902412in}}%
\pgfpathlineto{\pgfqpoint{3.948985in}{0.859816in}}%
\pgfpathlineto{\pgfqpoint{3.963625in}{0.844425in}}%
\pgfpathlineto{\pgfqpoint{3.963625in}{0.844425in}}%
\pgfusepath{stroke}%
\end{pgfscope}%
\begin{pgfscope}%
\pgfpathrectangle{\pgfqpoint{0.456635in}{0.521603in}}{\pgfqpoint{3.720000in}{3.020000in}} %
\pgfusepath{clip}%
\pgfsetrectcap%
\pgfsetroundjoin%
\pgfsetlinewidth{1.505625pt}%
\definecolor{currentstroke}{rgb}{1.000000,0.000000,0.000000}%
\pgfsetstrokecolor{currentstroke}%
\pgfsetdash{}{0pt}%
\pgfpathmoveto{\pgfqpoint{0.713566in}{0.744513in}}%
\pgfpathlineto{\pgfqpoint{0.742845in}{0.754844in}}%
\pgfpathlineto{\pgfqpoint{0.757485in}{0.758737in}}%
\pgfpathlineto{\pgfqpoint{0.772125in}{0.764319in}}%
\pgfpathlineto{\pgfqpoint{0.786765in}{0.768553in}}%
\pgfpathlineto{\pgfqpoint{0.801405in}{0.770678in}}%
\pgfpathlineto{\pgfqpoint{0.830685in}{0.778906in}}%
\pgfpathlineto{\pgfqpoint{0.845325in}{0.778466in}}%
\pgfpathlineto{\pgfqpoint{0.874605in}{0.782565in}}%
\pgfpathlineto{\pgfqpoint{0.889244in}{0.782131in}}%
\pgfpathlineto{\pgfqpoint{0.918524in}{0.783032in}}%
\pgfpathlineto{\pgfqpoint{0.977084in}{0.777015in}}%
\pgfpathlineto{\pgfqpoint{1.021004in}{0.767397in}}%
\pgfpathlineto{\pgfqpoint{1.064923in}{0.753119in}}%
\pgfpathlineto{\pgfqpoint{1.108843in}{0.733987in}}%
\pgfpathlineto{\pgfqpoint{1.123483in}{0.725958in}}%
\pgfpathlineto{\pgfqpoint{1.152763in}{0.707133in}}%
\pgfpathlineto{\pgfqpoint{1.196683in}{0.667825in}}%
\pgfpathlineto{\pgfqpoint{1.211322in}{0.660773in}}%
\pgfpathlineto{\pgfqpoint{1.225962in}{0.663499in}}%
\pgfpathlineto{\pgfqpoint{1.240602in}{0.673788in}}%
\pgfpathlineto{\pgfqpoint{1.269882in}{0.696291in}}%
\pgfpathlineto{\pgfqpoint{1.284522in}{0.703579in}}%
\pgfpathlineto{\pgfqpoint{1.313802in}{0.715027in}}%
\pgfpathlineto{\pgfqpoint{1.328442in}{0.718017in}}%
\pgfpathlineto{\pgfqpoint{1.357721in}{0.721316in}}%
\pgfpathlineto{\pgfqpoint{1.372361in}{0.721022in}}%
\pgfpathlineto{\pgfqpoint{1.401641in}{0.716713in}}%
\pgfpathlineto{\pgfqpoint{1.430921in}{0.708418in}}%
\pgfpathlineto{\pgfqpoint{1.460201in}{0.693378in}}%
\pgfpathlineto{\pgfqpoint{1.474841in}{0.683940in}}%
\pgfpathlineto{\pgfqpoint{1.489481in}{0.672239in}}%
\pgfpathlineto{\pgfqpoint{1.504121in}{0.662450in}}%
\pgfpathlineto{\pgfqpoint{1.518760in}{0.659721in}}%
\pgfpathlineto{\pgfqpoint{1.533400in}{0.664295in}}%
\pgfpathlineto{\pgfqpoint{1.562680in}{0.682885in}}%
\pgfpathlineto{\pgfqpoint{1.577320in}{0.690500in}}%
\pgfpathlineto{\pgfqpoint{1.591960in}{0.695235in}}%
\pgfpathlineto{\pgfqpoint{1.621240in}{0.702010in}}%
\pgfpathlineto{\pgfqpoint{1.650520in}{0.701157in}}%
\pgfpathlineto{\pgfqpoint{1.665159in}{0.699875in}}%
\pgfpathlineto{\pgfqpoint{1.694439in}{0.691160in}}%
\pgfpathlineto{\pgfqpoint{1.709079in}{0.684377in}}%
\pgfpathlineto{\pgfqpoint{1.738359in}{0.666700in}}%
\pgfpathlineto{\pgfqpoint{1.752999in}{0.660091in}}%
\pgfpathlineto{\pgfqpoint{1.767639in}{0.660342in}}%
\pgfpathlineto{\pgfqpoint{1.782279in}{0.665597in}}%
\pgfpathlineto{\pgfqpoint{1.796919in}{0.674506in}}%
\pgfpathlineto{\pgfqpoint{1.811559in}{0.680747in}}%
\pgfpathlineto{\pgfqpoint{1.826198in}{0.685673in}}%
\pgfpathlineto{\pgfqpoint{1.840838in}{0.687723in}}%
\pgfpathlineto{\pgfqpoint{1.855478in}{0.687629in}}%
\pgfpathlineto{\pgfqpoint{1.870118in}{0.685857in}}%
\pgfpathlineto{\pgfqpoint{1.884758in}{0.682120in}}%
\pgfpathlineto{\pgfqpoint{1.899398in}{0.675901in}}%
\pgfpathlineto{\pgfqpoint{1.928678in}{0.661291in}}%
\pgfpathlineto{\pgfqpoint{1.943318in}{0.658876in}}%
\pgfpathlineto{\pgfqpoint{1.957958in}{0.662430in}}%
\pgfpathlineto{\pgfqpoint{1.987237in}{0.676108in}}%
\pgfpathlineto{\pgfqpoint{2.001877in}{0.679739in}}%
\pgfpathlineto{\pgfqpoint{2.016517in}{0.679982in}}%
\pgfpathlineto{\pgfqpoint{2.031157in}{0.677360in}}%
\pgfpathlineto{\pgfqpoint{2.075077in}{0.663171in}}%
\pgfpathlineto{\pgfqpoint{2.089717in}{0.662075in}}%
\pgfpathlineto{\pgfqpoint{2.104357in}{0.663916in}}%
\pgfpathlineto{\pgfqpoint{2.118997in}{0.669254in}}%
\pgfpathlineto{\pgfqpoint{2.133636in}{0.673229in}}%
\pgfpathlineto{\pgfqpoint{2.148276in}{0.675850in}}%
\pgfpathlineto{\pgfqpoint{2.162916in}{0.675189in}}%
\pgfpathlineto{\pgfqpoint{2.177556in}{0.671194in}}%
\pgfpathlineto{\pgfqpoint{2.192196in}{0.666026in}}%
\pgfpathlineto{\pgfqpoint{2.206836in}{0.662058in}}%
\pgfpathlineto{\pgfqpoint{2.221476in}{0.661826in}}%
\pgfpathlineto{\pgfqpoint{2.236116in}{0.664860in}}%
\pgfpathlineto{\pgfqpoint{2.265396in}{0.672954in}}%
\pgfpathlineto{\pgfqpoint{2.280036in}{0.674296in}}%
\pgfpathlineto{\pgfqpoint{2.294675in}{0.672611in}}%
\pgfpathlineto{\pgfqpoint{2.338595in}{0.663810in}}%
\pgfpathlineto{\pgfqpoint{2.353235in}{0.664157in}}%
\pgfpathlineto{\pgfqpoint{2.397155in}{0.671991in}}%
\pgfpathlineto{\pgfqpoint{2.411795in}{0.672391in}}%
\pgfpathlineto{\pgfqpoint{2.426435in}{0.668634in}}%
\pgfpathlineto{\pgfqpoint{2.441074in}{0.667368in}}%
\pgfpathlineto{\pgfqpoint{2.455714in}{0.664927in}}%
\pgfpathlineto{\pgfqpoint{2.484994in}{0.666911in}}%
\pgfpathlineto{\pgfqpoint{2.514274in}{0.671071in}}%
\pgfpathlineto{\pgfqpoint{2.528914in}{0.671758in}}%
\pgfpathlineto{\pgfqpoint{2.558194in}{0.669422in}}%
\pgfpathlineto{\pgfqpoint{2.587474in}{0.666092in}}%
\pgfpathlineto{\pgfqpoint{2.616753in}{0.667017in}}%
\pgfpathlineto{\pgfqpoint{2.660673in}{0.671383in}}%
\pgfpathlineto{\pgfqpoint{2.675313in}{0.671291in}}%
\pgfpathlineto{\pgfqpoint{2.733873in}{0.666837in}}%
\pgfpathlineto{\pgfqpoint{2.807072in}{0.669736in}}%
\pgfpathlineto{\pgfqpoint{2.836352in}{0.667182in}}%
\pgfpathlineto{\pgfqpoint{2.865632in}{0.667123in}}%
\pgfpathlineto{\pgfqpoint{2.909551in}{0.670099in}}%
\pgfpathlineto{\pgfqpoint{2.938831in}{0.668344in}}%
\pgfpathlineto{\pgfqpoint{2.968111in}{0.666498in}}%
\pgfpathlineto{\pgfqpoint{2.997391in}{0.667777in}}%
\pgfpathlineto{\pgfqpoint{3.026671in}{0.669857in}}%
\pgfpathlineto{\pgfqpoint{3.055951in}{0.669227in}}%
\pgfpathlineto{\pgfqpoint{3.085230in}{0.667153in}}%
\pgfpathlineto{\pgfqpoint{3.114510in}{0.667180in}}%
\pgfpathlineto{\pgfqpoint{3.173070in}{0.669572in}}%
\pgfpathlineto{\pgfqpoint{3.231629in}{0.666723in}}%
\pgfpathlineto{\pgfqpoint{3.290189in}{0.668790in}}%
\pgfpathlineto{\pgfqpoint{3.378028in}{0.667594in}}%
\pgfpathlineto{\pgfqpoint{3.407308in}{0.668180in}}%
\pgfpathlineto{\pgfqpoint{3.465868in}{0.666640in}}%
\pgfpathlineto{\pgfqpoint{3.539067in}{0.667431in}}%
\pgfpathlineto{\pgfqpoint{3.597627in}{0.667095in}}%
\pgfpathlineto{\pgfqpoint{3.656187in}{0.667620in}}%
\pgfpathlineto{\pgfqpoint{3.714746in}{0.667098in}}%
\pgfpathlineto{\pgfqpoint{3.758666in}{0.667772in}}%
\pgfpathlineto{\pgfqpoint{3.817226in}{0.666881in}}%
\pgfpathlineto{\pgfqpoint{3.890425in}{0.667773in}}%
\pgfpathlineto{\pgfqpoint{3.948985in}{0.667250in}}%
\pgfpathlineto{\pgfqpoint{3.978265in}{0.667987in}}%
\pgfpathlineto{\pgfqpoint{3.978265in}{0.667987in}}%
\pgfusepath{stroke}%
\end{pgfscope}%
\begin{pgfscope}%
\pgfpathrectangle{\pgfqpoint{0.456635in}{0.521603in}}{\pgfqpoint{3.720000in}{3.020000in}} %
\pgfusepath{clip}%
\pgfsetrectcap%
\pgfsetroundjoin%
\pgfsetlinewidth{1.505625pt}%
\definecolor{currentstroke}{rgb}{0.958824,0.751332,0.412356}%
\pgfsetstrokecolor{currentstroke}%
\pgfsetdash{}{0pt}%
\pgfpathmoveto{\pgfqpoint{0.713566in}{0.835471in}}%
\pgfpathlineto{\pgfqpoint{0.728206in}{0.847490in}}%
\pgfpathlineto{\pgfqpoint{0.757485in}{0.889137in}}%
\pgfpathlineto{\pgfqpoint{0.772125in}{0.905233in}}%
\pgfpathlineto{\pgfqpoint{0.786765in}{0.916299in}}%
\pgfpathlineto{\pgfqpoint{0.816045in}{0.951147in}}%
\pgfpathlineto{\pgfqpoint{0.845325in}{0.971976in}}%
\pgfpathlineto{\pgfqpoint{0.859965in}{0.987231in}}%
\pgfpathlineto{\pgfqpoint{0.874605in}{1.000083in}}%
\pgfpathlineto{\pgfqpoint{0.889244in}{1.006651in}}%
\pgfpathlineto{\pgfqpoint{0.903884in}{1.014538in}}%
\pgfpathlineto{\pgfqpoint{0.918524in}{1.026987in}}%
\pgfpathlineto{\pgfqpoint{0.933164in}{1.035406in}}%
\pgfpathlineto{\pgfqpoint{0.947804in}{1.039112in}}%
\pgfpathlineto{\pgfqpoint{0.977084in}{1.056323in}}%
\pgfpathlineto{\pgfqpoint{0.991724in}{1.060205in}}%
\pgfpathlineto{\pgfqpoint{1.006364in}{1.061328in}}%
\pgfpathlineto{\pgfqpoint{1.035644in}{1.073370in}}%
\pgfpathlineto{\pgfqpoint{1.050283in}{1.073156in}}%
\pgfpathlineto{\pgfqpoint{1.064923in}{1.074139in}}%
\pgfpathlineto{\pgfqpoint{1.079563in}{1.078566in}}%
\pgfpathlineto{\pgfqpoint{1.094203in}{1.080318in}}%
\pgfpathlineto{\pgfqpoint{1.108843in}{1.077031in}}%
\pgfpathlineto{\pgfqpoint{1.123483in}{1.076053in}}%
\pgfpathlineto{\pgfqpoint{1.138123in}{1.077879in}}%
\pgfpathlineto{\pgfqpoint{1.152763in}{1.075132in}}%
\pgfpathlineto{\pgfqpoint{1.167403in}{1.069922in}}%
\pgfpathlineto{\pgfqpoint{1.196683in}{1.066918in}}%
\pgfpathlineto{\pgfqpoint{1.225962in}{1.052246in}}%
\pgfpathlineto{\pgfqpoint{1.240602in}{1.048455in}}%
\pgfpathlineto{\pgfqpoint{1.255242in}{1.043038in}}%
\pgfpathlineto{\pgfqpoint{1.269882in}{1.032572in}}%
\pgfpathlineto{\pgfqpoint{1.284522in}{1.024380in}}%
\pgfpathlineto{\pgfqpoint{1.299162in}{1.018391in}}%
\pgfpathlineto{\pgfqpoint{1.313802in}{1.008334in}}%
\pgfpathlineto{\pgfqpoint{1.328442in}{0.994761in}}%
\pgfpathlineto{\pgfqpoint{1.357721in}{0.974528in}}%
\pgfpathlineto{\pgfqpoint{1.387001in}{0.944649in}}%
\pgfpathlineto{\pgfqpoint{1.401641in}{0.932223in}}%
\pgfpathlineto{\pgfqpoint{1.416281in}{0.917956in}}%
\pgfpathlineto{\pgfqpoint{1.445561in}{0.880457in}}%
\pgfpathlineto{\pgfqpoint{1.460201in}{0.864071in}}%
\pgfpathlineto{\pgfqpoint{1.474841in}{0.844462in}}%
\pgfpathlineto{\pgfqpoint{1.518760in}{0.775638in}}%
\pgfpathlineto{\pgfqpoint{1.533400in}{0.745253in}}%
\pgfpathlineto{\pgfqpoint{1.548040in}{0.719356in}}%
\pgfpathlineto{\pgfqpoint{1.562680in}{0.698014in}}%
\pgfpathlineto{\pgfqpoint{1.577320in}{0.678958in}}%
\pgfpathlineto{\pgfqpoint{1.591960in}{0.679096in}}%
\pgfpathlineto{\pgfqpoint{1.606600in}{0.685966in}}%
\pgfpathlineto{\pgfqpoint{1.621240in}{0.708358in}}%
\pgfpathlineto{\pgfqpoint{1.635880in}{0.727794in}}%
\pgfpathlineto{\pgfqpoint{1.650520in}{0.749581in}}%
\pgfpathlineto{\pgfqpoint{1.679799in}{0.800741in}}%
\pgfpathlineto{\pgfqpoint{1.694439in}{0.820594in}}%
\pgfpathlineto{\pgfqpoint{1.709079in}{0.836573in}}%
\pgfpathlineto{\pgfqpoint{1.752999in}{0.877556in}}%
\pgfpathlineto{\pgfqpoint{1.782279in}{0.897944in}}%
\pgfpathlineto{\pgfqpoint{1.811559in}{0.914104in}}%
\pgfpathlineto{\pgfqpoint{1.840838in}{0.925081in}}%
\pgfpathlineto{\pgfqpoint{1.855478in}{0.930155in}}%
\pgfpathlineto{\pgfqpoint{1.884758in}{0.934764in}}%
\pgfpathlineto{\pgfqpoint{1.914038in}{0.936155in}}%
\pgfpathlineto{\pgfqpoint{1.928678in}{0.934927in}}%
\pgfpathlineto{\pgfqpoint{1.972597in}{0.925832in}}%
\pgfpathlineto{\pgfqpoint{2.001877in}{0.913948in}}%
\pgfpathlineto{\pgfqpoint{2.016517in}{0.906638in}}%
\pgfpathlineto{\pgfqpoint{2.031157in}{0.898033in}}%
\pgfpathlineto{\pgfqpoint{2.060437in}{0.877335in}}%
\pgfpathlineto{\pgfqpoint{2.075077in}{0.865775in}}%
\pgfpathlineto{\pgfqpoint{2.104357in}{0.836848in}}%
\pgfpathlineto{\pgfqpoint{2.133636in}{0.802866in}}%
\pgfpathlineto{\pgfqpoint{2.148276in}{0.782420in}}%
\pgfpathlineto{\pgfqpoint{2.177556in}{0.732354in}}%
\pgfpathlineto{\pgfqpoint{2.192196in}{0.715859in}}%
\pgfpathlineto{\pgfqpoint{2.206836in}{0.703566in}}%
\pgfpathlineto{\pgfqpoint{2.221476in}{0.689101in}}%
\pgfpathlineto{\pgfqpoint{2.236116in}{0.690806in}}%
\pgfpathlineto{\pgfqpoint{2.250756in}{0.698193in}}%
\pgfpathlineto{\pgfqpoint{2.265396in}{0.712680in}}%
\pgfpathlineto{\pgfqpoint{2.280036in}{0.729615in}}%
\pgfpathlineto{\pgfqpoint{2.294675in}{0.753247in}}%
\pgfpathlineto{\pgfqpoint{2.309315in}{0.770631in}}%
\pgfpathlineto{\pgfqpoint{2.323955in}{0.785907in}}%
\pgfpathlineto{\pgfqpoint{2.338595in}{0.796207in}}%
\pgfpathlineto{\pgfqpoint{2.353235in}{0.811701in}}%
\pgfpathlineto{\pgfqpoint{2.367875in}{0.821377in}}%
\pgfpathlineto{\pgfqpoint{2.411795in}{0.837330in}}%
\pgfpathlineto{\pgfqpoint{2.426435in}{0.841228in}}%
\pgfpathlineto{\pgfqpoint{2.455714in}{0.838885in}}%
\pgfpathlineto{\pgfqpoint{2.470354in}{0.838832in}}%
\pgfpathlineto{\pgfqpoint{2.484994in}{0.836547in}}%
\pgfpathlineto{\pgfqpoint{2.499634in}{0.829860in}}%
\pgfpathlineto{\pgfqpoint{2.528914in}{0.819593in}}%
\pgfpathlineto{\pgfqpoint{2.543554in}{0.809405in}}%
\pgfpathlineto{\pgfqpoint{2.558194in}{0.792715in}}%
\pgfpathlineto{\pgfqpoint{2.572834in}{0.781736in}}%
\pgfpathlineto{\pgfqpoint{2.587474in}{0.768565in}}%
\pgfpathlineto{\pgfqpoint{2.602113in}{0.750635in}}%
\pgfpathlineto{\pgfqpoint{2.616753in}{0.729694in}}%
\pgfpathlineto{\pgfqpoint{2.631393in}{0.717143in}}%
\pgfpathlineto{\pgfqpoint{2.646033in}{0.709388in}}%
\pgfpathlineto{\pgfqpoint{2.660673in}{0.697743in}}%
\pgfpathlineto{\pgfqpoint{2.675313in}{0.698540in}}%
\pgfpathlineto{\pgfqpoint{2.689953in}{0.704416in}}%
\pgfpathlineto{\pgfqpoint{2.719233in}{0.728937in}}%
\pgfpathlineto{\pgfqpoint{2.733873in}{0.742480in}}%
\pgfpathlineto{\pgfqpoint{2.748512in}{0.759303in}}%
\pgfpathlineto{\pgfqpoint{2.763152in}{0.773047in}}%
\pgfpathlineto{\pgfqpoint{2.792432in}{0.784277in}}%
\pgfpathlineto{\pgfqpoint{2.807072in}{0.793376in}}%
\pgfpathlineto{\pgfqpoint{2.821712in}{0.796753in}}%
\pgfpathlineto{\pgfqpoint{2.836352in}{0.793110in}}%
\pgfpathlineto{\pgfqpoint{2.850992in}{0.792348in}}%
\pgfpathlineto{\pgfqpoint{2.865632in}{0.793142in}}%
\pgfpathlineto{\pgfqpoint{2.880272in}{0.782026in}}%
\pgfpathlineto{\pgfqpoint{2.894912in}{0.774155in}}%
\pgfpathlineto{\pgfqpoint{2.909551in}{0.763049in}}%
\pgfpathlineto{\pgfqpoint{2.953471in}{0.718895in}}%
\pgfpathlineto{\pgfqpoint{2.968111in}{0.710040in}}%
\pgfpathlineto{\pgfqpoint{2.982751in}{0.699669in}}%
\pgfpathlineto{\pgfqpoint{2.997391in}{0.699166in}}%
\pgfpathlineto{\pgfqpoint{3.012031in}{0.703379in}}%
\pgfpathlineto{\pgfqpoint{3.026671in}{0.713750in}}%
\pgfpathlineto{\pgfqpoint{3.041311in}{0.728719in}}%
\pgfpathlineto{\pgfqpoint{3.055951in}{0.738625in}}%
\pgfpathlineto{\pgfqpoint{3.070590in}{0.755275in}}%
\pgfpathlineto{\pgfqpoint{3.085230in}{0.770180in}}%
\pgfpathlineto{\pgfqpoint{3.114510in}{0.784535in}}%
\pgfpathlineto{\pgfqpoint{3.143790in}{0.795600in}}%
\pgfpathlineto{\pgfqpoint{3.173070in}{0.794884in}}%
\pgfpathlineto{\pgfqpoint{3.187710in}{0.793747in}}%
\pgfpathlineto{\pgfqpoint{3.216989in}{0.784082in}}%
\pgfpathlineto{\pgfqpoint{3.246269in}{0.768536in}}%
\pgfpathlineto{\pgfqpoint{3.260909in}{0.750410in}}%
\pgfpathlineto{\pgfqpoint{3.290189in}{0.727749in}}%
\pgfpathlineto{\pgfqpoint{3.319469in}{0.705218in}}%
\pgfpathlineto{\pgfqpoint{3.334109in}{0.702192in}}%
\pgfpathlineto{\pgfqpoint{3.348749in}{0.704276in}}%
\pgfpathlineto{\pgfqpoint{3.363389in}{0.711844in}}%
\pgfpathlineto{\pgfqpoint{3.378028in}{0.721462in}}%
\pgfpathlineto{\pgfqpoint{3.421948in}{0.754120in}}%
\pgfpathlineto{\pgfqpoint{3.451228in}{0.770571in}}%
\pgfpathlineto{\pgfqpoint{3.465868in}{0.777363in}}%
\pgfpathlineto{\pgfqpoint{3.480508in}{0.778059in}}%
\pgfpathlineto{\pgfqpoint{3.509788in}{0.776663in}}%
\pgfpathlineto{\pgfqpoint{3.524427in}{0.772679in}}%
\pgfpathlineto{\pgfqpoint{3.553707in}{0.753547in}}%
\pgfpathlineto{\pgfqpoint{3.582987in}{0.734218in}}%
\pgfpathlineto{\pgfqpoint{3.597627in}{0.722475in}}%
\pgfpathlineto{\pgfqpoint{3.626907in}{0.708369in}}%
\pgfpathlineto{\pgfqpoint{3.641547in}{0.706015in}}%
\pgfpathlineto{\pgfqpoint{3.656187in}{0.709342in}}%
\pgfpathlineto{\pgfqpoint{3.670827in}{0.717489in}}%
\pgfpathlineto{\pgfqpoint{3.700106in}{0.736799in}}%
\pgfpathlineto{\pgfqpoint{3.729386in}{0.755551in}}%
\pgfpathlineto{\pgfqpoint{3.744026in}{0.761817in}}%
\pgfpathlineto{\pgfqpoint{3.758666in}{0.773568in}}%
\pgfpathlineto{\pgfqpoint{3.773306in}{0.777566in}}%
\pgfpathlineto{\pgfqpoint{3.802586in}{0.779997in}}%
\pgfpathlineto{\pgfqpoint{3.817226in}{0.778640in}}%
\pgfpathlineto{\pgfqpoint{3.846505in}{0.772118in}}%
\pgfpathlineto{\pgfqpoint{3.861145in}{0.760681in}}%
\pgfpathlineto{\pgfqpoint{3.875785in}{0.753836in}}%
\pgfpathlineto{\pgfqpoint{3.905065in}{0.735299in}}%
\pgfpathlineto{\pgfqpoint{3.934345in}{0.717684in}}%
\pgfpathlineto{\pgfqpoint{3.948985in}{0.713737in}}%
\pgfpathlineto{\pgfqpoint{3.963625in}{0.711600in}}%
\pgfpathlineto{\pgfqpoint{3.978265in}{0.714993in}}%
\pgfpathlineto{\pgfqpoint{3.992904in}{0.721341in}}%
\pgfpathlineto{\pgfqpoint{3.992904in}{0.721341in}}%
\pgfusepath{stroke}%
\end{pgfscope}%
\begin{pgfscope}%
\pgfpathrectangle{\pgfqpoint{0.456635in}{0.521603in}}{\pgfqpoint{3.720000in}{3.020000in}} %
\pgfusepath{clip}%
\pgfsetrectcap%
\pgfsetroundjoin%
\pgfsetlinewidth{1.505625pt}%
\definecolor{currentstroke}{rgb}{1.000000,0.255843,0.128999}%
\pgfsetstrokecolor{currentstroke}%
\pgfsetdash{}{0pt}%
\pgfpathmoveto{\pgfqpoint{0.698926in}{0.793976in}}%
\pgfpathlineto{\pgfqpoint{0.713566in}{0.810676in}}%
\pgfpathlineto{\pgfqpoint{0.728206in}{0.823919in}}%
\pgfpathlineto{\pgfqpoint{0.742845in}{0.832514in}}%
\pgfpathlineto{\pgfqpoint{0.772125in}{0.856341in}}%
\pgfpathlineto{\pgfqpoint{0.786765in}{0.861852in}}%
\pgfpathlineto{\pgfqpoint{0.801405in}{0.871288in}}%
\pgfpathlineto{\pgfqpoint{0.816045in}{0.877833in}}%
\pgfpathlineto{\pgfqpoint{0.830685in}{0.880855in}}%
\pgfpathlineto{\pgfqpoint{0.845325in}{0.887872in}}%
\pgfpathlineto{\pgfqpoint{0.859965in}{0.890144in}}%
\pgfpathlineto{\pgfqpoint{0.874605in}{0.889611in}}%
\pgfpathlineto{\pgfqpoint{0.889244in}{0.894655in}}%
\pgfpathlineto{\pgfqpoint{0.903884in}{0.893836in}}%
\pgfpathlineto{\pgfqpoint{0.918524in}{0.890909in}}%
\pgfpathlineto{\pgfqpoint{0.933164in}{0.892340in}}%
\pgfpathlineto{\pgfqpoint{0.962444in}{0.883192in}}%
\pgfpathlineto{\pgfqpoint{0.977084in}{0.880800in}}%
\pgfpathlineto{\pgfqpoint{1.006364in}{0.865331in}}%
\pgfpathlineto{\pgfqpoint{1.021004in}{0.860026in}}%
\pgfpathlineto{\pgfqpoint{1.035644in}{0.848672in}}%
\pgfpathlineto{\pgfqpoint{1.064923in}{0.830052in}}%
\pgfpathlineto{\pgfqpoint{1.094203in}{0.801535in}}%
\pgfpathlineto{\pgfqpoint{1.108843in}{0.788907in}}%
\pgfpathlineto{\pgfqpoint{1.152763in}{0.732715in}}%
\pgfpathlineto{\pgfqpoint{1.167403in}{0.710818in}}%
\pgfpathlineto{\pgfqpoint{1.182043in}{0.695783in}}%
\pgfpathlineto{\pgfqpoint{1.196683in}{0.677920in}}%
\pgfpathlineto{\pgfqpoint{1.211322in}{0.678072in}}%
\pgfpathlineto{\pgfqpoint{1.225962in}{0.684655in}}%
\pgfpathlineto{\pgfqpoint{1.240602in}{0.701228in}}%
\pgfpathlineto{\pgfqpoint{1.255242in}{0.720577in}}%
\pgfpathlineto{\pgfqpoint{1.269882in}{0.737327in}}%
\pgfpathlineto{\pgfqpoint{1.313802in}{0.773150in}}%
\pgfpathlineto{\pgfqpoint{1.328442in}{0.775543in}}%
\pgfpathlineto{\pgfqpoint{1.343082in}{0.786176in}}%
\pgfpathlineto{\pgfqpoint{1.357721in}{0.791121in}}%
\pgfpathlineto{\pgfqpoint{1.372361in}{0.789441in}}%
\pgfpathlineto{\pgfqpoint{1.387001in}{0.794093in}}%
\pgfpathlineto{\pgfqpoint{1.401641in}{0.796106in}}%
\pgfpathlineto{\pgfqpoint{1.416281in}{0.790614in}}%
\pgfpathlineto{\pgfqpoint{1.430921in}{0.791915in}}%
\pgfpathlineto{\pgfqpoint{1.445561in}{0.786776in}}%
\pgfpathlineto{\pgfqpoint{1.460201in}{0.778773in}}%
\pgfpathlineto{\pgfqpoint{1.474841in}{0.775456in}}%
\pgfpathlineto{\pgfqpoint{1.489481in}{0.765427in}}%
\pgfpathlineto{\pgfqpoint{1.504121in}{0.752507in}}%
\pgfpathlineto{\pgfqpoint{1.518760in}{0.743108in}}%
\pgfpathlineto{\pgfqpoint{1.548040in}{0.710709in}}%
\pgfpathlineto{\pgfqpoint{1.562680in}{0.696702in}}%
\pgfpathlineto{\pgfqpoint{1.577320in}{0.684339in}}%
\pgfpathlineto{\pgfqpoint{1.591960in}{0.683412in}}%
\pgfpathlineto{\pgfqpoint{1.606600in}{0.687464in}}%
\pgfpathlineto{\pgfqpoint{1.635880in}{0.713026in}}%
\pgfpathlineto{\pgfqpoint{1.650520in}{0.728710in}}%
\pgfpathlineto{\pgfqpoint{1.665159in}{0.737816in}}%
\pgfpathlineto{\pgfqpoint{1.679799in}{0.749399in}}%
\pgfpathlineto{\pgfqpoint{1.694439in}{0.753682in}}%
\pgfpathlineto{\pgfqpoint{1.709079in}{0.753994in}}%
\pgfpathlineto{\pgfqpoint{1.723719in}{0.756821in}}%
\pgfpathlineto{\pgfqpoint{1.738359in}{0.754605in}}%
\pgfpathlineto{\pgfqpoint{1.767639in}{0.740558in}}%
\pgfpathlineto{\pgfqpoint{1.782279in}{0.729716in}}%
\pgfpathlineto{\pgfqpoint{1.796919in}{0.713752in}}%
\pgfpathlineto{\pgfqpoint{1.811559in}{0.704032in}}%
\pgfpathlineto{\pgfqpoint{1.826198in}{0.690824in}}%
\pgfpathlineto{\pgfqpoint{1.840838in}{0.686075in}}%
\pgfpathlineto{\pgfqpoint{1.855478in}{0.687906in}}%
\pgfpathlineto{\pgfqpoint{1.884758in}{0.707524in}}%
\pgfpathlineto{\pgfqpoint{1.899398in}{0.723243in}}%
\pgfpathlineto{\pgfqpoint{1.914038in}{0.733632in}}%
\pgfpathlineto{\pgfqpoint{1.928678in}{0.740188in}}%
\pgfpathlineto{\pgfqpoint{1.943318in}{0.745190in}}%
\pgfpathlineto{\pgfqpoint{1.957958in}{0.746480in}}%
\pgfpathlineto{\pgfqpoint{1.987237in}{0.740660in}}%
\pgfpathlineto{\pgfqpoint{2.001877in}{0.734541in}}%
\pgfpathlineto{\pgfqpoint{2.016517in}{0.723556in}}%
\pgfpathlineto{\pgfqpoint{2.031157in}{0.715739in}}%
\pgfpathlineto{\pgfqpoint{2.045797in}{0.702010in}}%
\pgfpathlineto{\pgfqpoint{2.060437in}{0.695183in}}%
\pgfpathlineto{\pgfqpoint{2.075077in}{0.691070in}}%
\pgfpathlineto{\pgfqpoint{2.089717in}{0.694616in}}%
\pgfpathlineto{\pgfqpoint{2.104357in}{0.700876in}}%
\pgfpathlineto{\pgfqpoint{2.118997in}{0.713187in}}%
\pgfpathlineto{\pgfqpoint{2.148276in}{0.730778in}}%
\pgfpathlineto{\pgfqpoint{2.162916in}{0.736501in}}%
\pgfpathlineto{\pgfqpoint{2.177556in}{0.739030in}}%
\pgfpathlineto{\pgfqpoint{2.192196in}{0.737526in}}%
\pgfpathlineto{\pgfqpoint{2.206836in}{0.731872in}}%
\pgfpathlineto{\pgfqpoint{2.236116in}{0.713340in}}%
\pgfpathlineto{\pgfqpoint{2.250756in}{0.703175in}}%
\pgfpathlineto{\pgfqpoint{2.265396in}{0.695458in}}%
\pgfpathlineto{\pgfqpoint{2.280036in}{0.692968in}}%
\pgfpathlineto{\pgfqpoint{2.294675in}{0.695941in}}%
\pgfpathlineto{\pgfqpoint{2.353235in}{0.732471in}}%
\pgfpathlineto{\pgfqpoint{2.367875in}{0.737107in}}%
\pgfpathlineto{\pgfqpoint{2.382515in}{0.738917in}}%
\pgfpathlineto{\pgfqpoint{2.397155in}{0.736212in}}%
\pgfpathlineto{\pgfqpoint{2.411795in}{0.730353in}}%
\pgfpathlineto{\pgfqpoint{2.455714in}{0.703239in}}%
\pgfpathlineto{\pgfqpoint{2.470354in}{0.697216in}}%
\pgfpathlineto{\pgfqpoint{2.484994in}{0.697196in}}%
\pgfpathlineto{\pgfqpoint{2.499634in}{0.702717in}}%
\pgfpathlineto{\pgfqpoint{2.543554in}{0.730498in}}%
\pgfpathlineto{\pgfqpoint{2.558194in}{0.736559in}}%
\pgfpathlineto{\pgfqpoint{2.572834in}{0.739310in}}%
\pgfpathlineto{\pgfqpoint{2.587474in}{0.738246in}}%
\pgfpathlineto{\pgfqpoint{2.602113in}{0.733559in}}%
\pgfpathlineto{\pgfqpoint{2.631393in}{0.717805in}}%
\pgfpathlineto{\pgfqpoint{2.646033in}{0.709189in}}%
\pgfpathlineto{\pgfqpoint{2.660673in}{0.702319in}}%
\pgfpathlineto{\pgfqpoint{2.675313in}{0.701187in}}%
\pgfpathlineto{\pgfqpoint{2.689953in}{0.704252in}}%
\pgfpathlineto{\pgfqpoint{2.704593in}{0.711515in}}%
\pgfpathlineto{\pgfqpoint{2.733873in}{0.729212in}}%
\pgfpathlineto{\pgfqpoint{2.748512in}{0.736376in}}%
\pgfpathlineto{\pgfqpoint{2.763152in}{0.739742in}}%
\pgfpathlineto{\pgfqpoint{2.777792in}{0.740852in}}%
\pgfpathlineto{\pgfqpoint{2.792432in}{0.738184in}}%
\pgfpathlineto{\pgfqpoint{2.821712in}{0.725223in}}%
\pgfpathlineto{\pgfqpoint{2.850992in}{0.709557in}}%
\pgfpathlineto{\pgfqpoint{2.865632in}{0.704998in}}%
\pgfpathlineto{\pgfqpoint{2.880272in}{0.704797in}}%
\pgfpathlineto{\pgfqpoint{2.894912in}{0.709057in}}%
\pgfpathlineto{\pgfqpoint{2.938831in}{0.732340in}}%
\pgfpathlineto{\pgfqpoint{2.953471in}{0.738224in}}%
\pgfpathlineto{\pgfqpoint{2.968111in}{0.740760in}}%
\pgfpathlineto{\pgfqpoint{2.982751in}{0.740355in}}%
\pgfpathlineto{\pgfqpoint{2.997391in}{0.736920in}}%
\pgfpathlineto{\pgfqpoint{3.055951in}{0.709950in}}%
\pgfpathlineto{\pgfqpoint{3.070590in}{0.708073in}}%
\pgfpathlineto{\pgfqpoint{3.085230in}{0.709356in}}%
\pgfpathlineto{\pgfqpoint{3.099870in}{0.715305in}}%
\pgfpathlineto{\pgfqpoint{3.129150in}{0.729659in}}%
\pgfpathlineto{\pgfqpoint{3.143790in}{0.736704in}}%
\pgfpathlineto{\pgfqpoint{3.158430in}{0.741691in}}%
\pgfpathlineto{\pgfqpoint{3.173070in}{0.742231in}}%
\pgfpathlineto{\pgfqpoint{3.187710in}{0.741058in}}%
\pgfpathlineto{\pgfqpoint{3.202350in}{0.736418in}}%
\pgfpathlineto{\pgfqpoint{3.260909in}{0.711635in}}%
\pgfpathlineto{\pgfqpoint{3.275549in}{0.710470in}}%
\pgfpathlineto{\pgfqpoint{3.290189in}{0.712661in}}%
\pgfpathlineto{\pgfqpoint{3.304829in}{0.717345in}}%
\pgfpathlineto{\pgfqpoint{3.348749in}{0.735047in}}%
\pgfpathlineto{\pgfqpoint{3.363389in}{0.738654in}}%
\pgfpathlineto{\pgfqpoint{3.378028in}{0.739474in}}%
\pgfpathlineto{\pgfqpoint{3.392668in}{0.736780in}}%
\pgfpathlineto{\pgfqpoint{3.407308in}{0.732693in}}%
\pgfpathlineto{\pgfqpoint{3.451228in}{0.716161in}}%
\pgfpathlineto{\pgfqpoint{3.465868in}{0.713208in}}%
\pgfpathlineto{\pgfqpoint{3.480508in}{0.712959in}}%
\pgfpathlineto{\pgfqpoint{3.495148in}{0.715585in}}%
\pgfpathlineto{\pgfqpoint{3.509788in}{0.719731in}}%
\pgfpathlineto{\pgfqpoint{3.539067in}{0.730512in}}%
\pgfpathlineto{\pgfqpoint{3.553707in}{0.733819in}}%
\pgfpathlineto{\pgfqpoint{3.568347in}{0.735427in}}%
\pgfpathlineto{\pgfqpoint{3.582987in}{0.734946in}}%
\pgfpathlineto{\pgfqpoint{3.597627in}{0.731983in}}%
\pgfpathlineto{\pgfqpoint{3.641547in}{0.717876in}}%
\pgfpathlineto{\pgfqpoint{3.656187in}{0.715059in}}%
\pgfpathlineto{\pgfqpoint{3.670827in}{0.714350in}}%
\pgfpathlineto{\pgfqpoint{3.685466in}{0.715740in}}%
\pgfpathlineto{\pgfqpoint{3.700106in}{0.719377in}}%
\pgfpathlineto{\pgfqpoint{3.714746in}{0.724174in}}%
\pgfpathlineto{\pgfqpoint{3.744026in}{0.731310in}}%
\pgfpathlineto{\pgfqpoint{3.758666in}{0.732873in}}%
\pgfpathlineto{\pgfqpoint{3.773306in}{0.732185in}}%
\pgfpathlineto{\pgfqpoint{3.787946in}{0.729996in}}%
\pgfpathlineto{\pgfqpoint{3.817226in}{0.722389in}}%
\pgfpathlineto{\pgfqpoint{3.831866in}{0.718663in}}%
\pgfpathlineto{\pgfqpoint{3.846505in}{0.716278in}}%
\pgfpathlineto{\pgfqpoint{3.861145in}{0.715532in}}%
\pgfpathlineto{\pgfqpoint{3.890425in}{0.719783in}}%
\pgfpathlineto{\pgfqpoint{3.934345in}{0.729494in}}%
\pgfpathlineto{\pgfqpoint{3.948985in}{0.730440in}}%
\pgfpathlineto{\pgfqpoint{3.963625in}{0.729959in}}%
\pgfpathlineto{\pgfqpoint{4.007544in}{0.721682in}}%
\pgfpathlineto{\pgfqpoint{4.007544in}{0.721682in}}%
\pgfusepath{stroke}%
\end{pgfscope}%
\begin{pgfscope}%
\pgfsetrectcap%
\pgfsetmiterjoin%
\pgfsetlinewidth{0.803000pt}%
\definecolor{currentstroke}{rgb}{0.000000,0.000000,0.000000}%
\pgfsetstrokecolor{currentstroke}%
\pgfsetdash{}{0pt}%
\pgfpathmoveto{\pgfqpoint{0.456635in}{0.521603in}}%
\pgfpathlineto{\pgfqpoint{0.456635in}{3.541603in}}%
\pgfusepath{stroke}%
\end{pgfscope}%
\begin{pgfscope}%
\pgfsetrectcap%
\pgfsetmiterjoin%
\pgfsetlinewidth{0.803000pt}%
\definecolor{currentstroke}{rgb}{0.000000,0.000000,0.000000}%
\pgfsetstrokecolor{currentstroke}%
\pgfsetdash{}{0pt}%
\pgfpathmoveto{\pgfqpoint{4.176635in}{0.521603in}}%
\pgfpathlineto{\pgfqpoint{4.176635in}{3.541603in}}%
\pgfusepath{stroke}%
\end{pgfscope}%
\begin{pgfscope}%
\pgfsetrectcap%
\pgfsetmiterjoin%
\pgfsetlinewidth{0.803000pt}%
\definecolor{currentstroke}{rgb}{0.000000,0.000000,0.000000}%
\pgfsetstrokecolor{currentstroke}%
\pgfsetdash{}{0pt}%
\pgfpathmoveto{\pgfqpoint{0.456635in}{0.521603in}}%
\pgfpathlineto{\pgfqpoint{4.176635in}{0.521603in}}%
\pgfusepath{stroke}%
\end{pgfscope}%
\begin{pgfscope}%
\pgfsetrectcap%
\pgfsetmiterjoin%
\pgfsetlinewidth{0.803000pt}%
\definecolor{currentstroke}{rgb}{0.000000,0.000000,0.000000}%
\pgfsetstrokecolor{currentstroke}%
\pgfsetdash{}{0pt}%
\pgfpathmoveto{\pgfqpoint{0.456635in}{3.541603in}}%
\pgfpathlineto{\pgfqpoint{4.176635in}{3.541603in}}%
\pgfusepath{stroke}%
\end{pgfscope}%
\begin{pgfscope}%
\pgfpathrectangle{\pgfqpoint{4.409135in}{0.521603in}}{\pgfqpoint{0.151000in}{3.020000in}} %
\pgfusepath{clip}%
\pgfsetbuttcap%
\pgfsetmiterjoin%
\definecolor{currentfill}{rgb}{1.000000,1.000000,1.000000}%
\pgfsetfillcolor{currentfill}%
\pgfsetlinewidth{0.010037pt}%
\definecolor{currentstroke}{rgb}{1.000000,1.000000,1.000000}%
\pgfsetstrokecolor{currentstroke}%
\pgfsetdash{}{0pt}%
\pgfpathmoveto{\pgfqpoint{4.409135in}{0.521603in}}%
\pgfpathlineto{\pgfqpoint{4.409135in}{0.533400in}}%
\pgfpathlineto{\pgfqpoint{4.409135in}{3.529806in}}%
\pgfpathlineto{\pgfqpoint{4.409135in}{3.541603in}}%
\pgfpathlineto{\pgfqpoint{4.560135in}{3.541603in}}%
\pgfpathlineto{\pgfqpoint{4.560135in}{3.529806in}}%
\pgfpathlineto{\pgfqpoint{4.560135in}{0.533400in}}%
\pgfpathlineto{\pgfqpoint{4.560135in}{0.521603in}}%
\pgfpathclose%
\pgfusepath{stroke,fill}%
\end{pgfscope}%
\begin{pgfscope}%
\pgfsys@transformshift{4.410000in}{0.524574in}%
\pgftext[left,bottom]{\pgfimage[interpolate=true,width=0.150000in,height=3.020000in]{series_s_eu-img0.png}}%
\end{pgfscope}%
\begin{pgfscope}%
\pgfsetbuttcap%
\pgfsetroundjoin%
\definecolor{currentfill}{rgb}{0.000000,0.000000,0.000000}%
\pgfsetfillcolor{currentfill}%
\pgfsetlinewidth{0.803000pt}%
\definecolor{currentstroke}{rgb}{0.000000,0.000000,0.000000}%
\pgfsetstrokecolor{currentstroke}%
\pgfsetdash{}{0pt}%
\pgfsys@defobject{currentmarker}{\pgfqpoint{0.000000in}{0.000000in}}{\pgfqpoint{0.048611in}{0.000000in}}{%
\pgfpathmoveto{\pgfqpoint{0.000000in}{0.000000in}}%
\pgfpathlineto{\pgfqpoint{0.048611in}{0.000000in}}%
\pgfusepath{stroke,fill}%
}%
\begin{pgfscope}%
\pgfsys@transformshift{4.560135in}{0.853575in}%
\pgfsys@useobject{currentmarker}{}%
\end{pgfscope}%
\end{pgfscope}%
\begin{pgfscope}%
\pgfsetbuttcap%
\pgfsetroundjoin%
\definecolor{currentfill}{rgb}{0.000000,0.000000,0.000000}%
\pgfsetfillcolor{currentfill}%
\pgfsetlinewidth{0.803000pt}%
\definecolor{currentstroke}{rgb}{0.000000,0.000000,0.000000}%
\pgfsetstrokecolor{currentstroke}%
\pgfsetdash{}{0pt}%
\pgfsys@defobject{currentmarker}{\pgfqpoint{0.000000in}{0.000000in}}{\pgfqpoint{0.048611in}{0.000000in}}{%
\pgfpathmoveto{\pgfqpoint{0.000000in}{0.000000in}}%
\pgfpathlineto{\pgfqpoint{0.048611in}{0.000000in}}%
\pgfusepath{stroke,fill}%
}%
\begin{pgfscope}%
\pgfsys@transformshift{4.560135in}{1.247759in}%
\pgfsys@useobject{currentmarker}{}%
\end{pgfscope}%
\end{pgfscope}%
\begin{pgfscope}%
\pgfsetbuttcap%
\pgfsetroundjoin%
\definecolor{currentfill}{rgb}{0.000000,0.000000,0.000000}%
\pgfsetfillcolor{currentfill}%
\pgfsetlinewidth{0.803000pt}%
\definecolor{currentstroke}{rgb}{0.000000,0.000000,0.000000}%
\pgfsetstrokecolor{currentstroke}%
\pgfsetdash{}{0pt}%
\pgfsys@defobject{currentmarker}{\pgfqpoint{0.000000in}{0.000000in}}{\pgfqpoint{0.048611in}{0.000000in}}{%
\pgfpathmoveto{\pgfqpoint{0.000000in}{0.000000in}}%
\pgfpathlineto{\pgfqpoint{0.048611in}{0.000000in}}%
\pgfusepath{stroke,fill}%
}%
\begin{pgfscope}%
\pgfsys@transformshift{4.560135in}{1.527436in}%
\pgfsys@useobject{currentmarker}{}%
\end{pgfscope}%
\end{pgfscope}%
\begin{pgfscope}%
\pgfsetbuttcap%
\pgfsetroundjoin%
\definecolor{currentfill}{rgb}{0.000000,0.000000,0.000000}%
\pgfsetfillcolor{currentfill}%
\pgfsetlinewidth{0.803000pt}%
\definecolor{currentstroke}{rgb}{0.000000,0.000000,0.000000}%
\pgfsetstrokecolor{currentstroke}%
\pgfsetdash{}{0pt}%
\pgfsys@defobject{currentmarker}{\pgfqpoint{0.000000in}{0.000000in}}{\pgfqpoint{0.048611in}{0.000000in}}{%
\pgfpathmoveto{\pgfqpoint{0.000000in}{0.000000in}}%
\pgfpathlineto{\pgfqpoint{0.048611in}{0.000000in}}%
\pgfusepath{stroke,fill}%
}%
\begin{pgfscope}%
\pgfsys@transformshift{4.560135in}{1.744371in}%
\pgfsys@useobject{currentmarker}{}%
\end{pgfscope}%
\end{pgfscope}%
\begin{pgfscope}%
\pgfsetbuttcap%
\pgfsetroundjoin%
\definecolor{currentfill}{rgb}{0.000000,0.000000,0.000000}%
\pgfsetfillcolor{currentfill}%
\pgfsetlinewidth{0.803000pt}%
\definecolor{currentstroke}{rgb}{0.000000,0.000000,0.000000}%
\pgfsetstrokecolor{currentstroke}%
\pgfsetdash{}{0pt}%
\pgfsys@defobject{currentmarker}{\pgfqpoint{0.000000in}{0.000000in}}{\pgfqpoint{0.048611in}{0.000000in}}{%
\pgfpathmoveto{\pgfqpoint{0.000000in}{0.000000in}}%
\pgfpathlineto{\pgfqpoint{0.048611in}{0.000000in}}%
\pgfusepath{stroke,fill}%
}%
\begin{pgfscope}%
\pgfsys@transformshift{4.560135in}{1.921620in}%
\pgfsys@useobject{currentmarker}{}%
\end{pgfscope}%
\end{pgfscope}%
\begin{pgfscope}%
\pgfsetbuttcap%
\pgfsetroundjoin%
\definecolor{currentfill}{rgb}{0.000000,0.000000,0.000000}%
\pgfsetfillcolor{currentfill}%
\pgfsetlinewidth{0.803000pt}%
\definecolor{currentstroke}{rgb}{0.000000,0.000000,0.000000}%
\pgfsetstrokecolor{currentstroke}%
\pgfsetdash{}{0pt}%
\pgfsys@defobject{currentmarker}{\pgfqpoint{0.000000in}{0.000000in}}{\pgfqpoint{0.048611in}{0.000000in}}{%
\pgfpathmoveto{\pgfqpoint{0.000000in}{0.000000in}}%
\pgfpathlineto{\pgfqpoint{0.048611in}{0.000000in}}%
\pgfusepath{stroke,fill}%
}%
\begin{pgfscope}%
\pgfsys@transformshift{4.560135in}{2.071482in}%
\pgfsys@useobject{currentmarker}{}%
\end{pgfscope}%
\end{pgfscope}%
\begin{pgfscope}%
\pgfsetbuttcap%
\pgfsetroundjoin%
\definecolor{currentfill}{rgb}{0.000000,0.000000,0.000000}%
\pgfsetfillcolor{currentfill}%
\pgfsetlinewidth{0.803000pt}%
\definecolor{currentstroke}{rgb}{0.000000,0.000000,0.000000}%
\pgfsetstrokecolor{currentstroke}%
\pgfsetdash{}{0pt}%
\pgfsys@defobject{currentmarker}{\pgfqpoint{0.000000in}{0.000000in}}{\pgfqpoint{0.048611in}{0.000000in}}{%
\pgfpathmoveto{\pgfqpoint{0.000000in}{0.000000in}}%
\pgfpathlineto{\pgfqpoint{0.048611in}{0.000000in}}%
\pgfusepath{stroke,fill}%
}%
\begin{pgfscope}%
\pgfsys@transformshift{4.560135in}{2.201298in}%
\pgfsys@useobject{currentmarker}{}%
\end{pgfscope}%
\end{pgfscope}%
\begin{pgfscope}%
\pgfsetbuttcap%
\pgfsetroundjoin%
\definecolor{currentfill}{rgb}{0.000000,0.000000,0.000000}%
\pgfsetfillcolor{currentfill}%
\pgfsetlinewidth{0.803000pt}%
\definecolor{currentstroke}{rgb}{0.000000,0.000000,0.000000}%
\pgfsetstrokecolor{currentstroke}%
\pgfsetdash{}{0pt}%
\pgfsys@defobject{currentmarker}{\pgfqpoint{0.000000in}{0.000000in}}{\pgfqpoint{0.048611in}{0.000000in}}{%
\pgfpathmoveto{\pgfqpoint{0.000000in}{0.000000in}}%
\pgfpathlineto{\pgfqpoint{0.048611in}{0.000000in}}%
\pgfusepath{stroke,fill}%
}%
\begin{pgfscope}%
\pgfsys@transformshift{4.560135in}{2.315804in}%
\pgfsys@useobject{currentmarker}{}%
\end{pgfscope}%
\end{pgfscope}%
\begin{pgfscope}%
\pgfsetbuttcap%
\pgfsetroundjoin%
\definecolor{currentfill}{rgb}{0.000000,0.000000,0.000000}%
\pgfsetfillcolor{currentfill}%
\pgfsetlinewidth{0.803000pt}%
\definecolor{currentstroke}{rgb}{0.000000,0.000000,0.000000}%
\pgfsetstrokecolor{currentstroke}%
\pgfsetdash{}{0pt}%
\pgfsys@defobject{currentmarker}{\pgfqpoint{0.000000in}{0.000000in}}{\pgfqpoint{0.048611in}{0.000000in}}{%
\pgfpathmoveto{\pgfqpoint{0.000000in}{0.000000in}}%
\pgfpathlineto{\pgfqpoint{0.048611in}{0.000000in}}%
\pgfusepath{stroke,fill}%
}%
\begin{pgfscope}%
\pgfsys@transformshift{4.560135in}{2.418233in}%
\pgfsys@useobject{currentmarker}{}%
\end{pgfscope}%
\end{pgfscope}%
\begin{pgfscope}%
\pgftext[x=4.657357in,y=2.365471in,left,base]{\rmfamily\fontsize{10.000000}{12.000000}\selectfont \(\displaystyle 10^{1}\)}%
\end{pgfscope}%
\begin{pgfscope}%
\pgfsetbuttcap%
\pgfsetroundjoin%
\definecolor{currentfill}{rgb}{0.000000,0.000000,0.000000}%
\pgfsetfillcolor{currentfill}%
\pgfsetlinewidth{0.803000pt}%
\definecolor{currentstroke}{rgb}{0.000000,0.000000,0.000000}%
\pgfsetstrokecolor{currentstroke}%
\pgfsetdash{}{0pt}%
\pgfsys@defobject{currentmarker}{\pgfqpoint{0.000000in}{0.000000in}}{\pgfqpoint{0.048611in}{0.000000in}}{%
\pgfpathmoveto{\pgfqpoint{0.000000in}{0.000000in}}%
\pgfpathlineto{\pgfqpoint{0.048611in}{0.000000in}}%
\pgfusepath{stroke,fill}%
}%
\begin{pgfscope}%
\pgfsys@transformshift{4.560135in}{3.092094in}%
\pgfsys@useobject{currentmarker}{}%
\end{pgfscope}%
\end{pgfscope}%
\begin{pgfscope}%
\pgfsetbuttcap%
\pgfsetroundjoin%
\definecolor{currentfill}{rgb}{0.000000,0.000000,0.000000}%
\pgfsetfillcolor{currentfill}%
\pgfsetlinewidth{0.803000pt}%
\definecolor{currentstroke}{rgb}{0.000000,0.000000,0.000000}%
\pgfsetstrokecolor{currentstroke}%
\pgfsetdash{}{0pt}%
\pgfsys@defobject{currentmarker}{\pgfqpoint{0.000000in}{0.000000in}}{\pgfqpoint{0.048611in}{0.000000in}}{%
\pgfpathmoveto{\pgfqpoint{0.000000in}{0.000000in}}%
\pgfpathlineto{\pgfqpoint{0.048611in}{0.000000in}}%
\pgfusepath{stroke,fill}%
}%
\begin{pgfscope}%
\pgfsys@transformshift{4.560135in}{3.486278in}%
\pgfsys@useobject{currentmarker}{}%
\end{pgfscope}%
\end{pgfscope}%
\begin{pgfscope}%
\pgftext[x=5.052999in,y=2.031603in,,top]{\rmfamily\fontsize{12.000000}{14.400000}\selectfont \(\displaystyle {\mathbf{E} \mbox{u}}\)}%
\end{pgfscope}%
\begin{pgfscope}%
\pgfsetbuttcap%
\pgfsetmiterjoin%
\pgfsetlinewidth{0.803000pt}%
\definecolor{currentstroke}{rgb}{0.000000,0.000000,0.000000}%
\pgfsetstrokecolor{currentstroke}%
\pgfsetdash{}{0pt}%
\pgfpathmoveto{\pgfqpoint{4.409135in}{0.521603in}}%
\pgfpathlineto{\pgfqpoint{4.409135in}{0.533400in}}%
\pgfpathlineto{\pgfqpoint{4.409135in}{3.529806in}}%
\pgfpathlineto{\pgfqpoint{4.409135in}{3.541603in}}%
\pgfpathlineto{\pgfqpoint{4.560135in}{3.541603in}}%
\pgfpathlineto{\pgfqpoint{4.560135in}{3.529806in}}%
\pgfpathlineto{\pgfqpoint{4.560135in}{0.533400in}}%
\pgfpathlineto{\pgfqpoint{4.560135in}{0.521603in}}%
\pgfpathclose%
\pgfusepath{stroke}%
\end{pgfscope}%
\end{pgfpicture}%
\makeatother%
\endgroup%

    \caption{Drop trajectories as a function of $\mathbb{E}\mbox{u}$.\label{fig:series_s_eu}}
\end{figure}
\begin{figure}[!htb]
    \centering
    \resizebox{14cm}{!}{%% Creator: Matplotlib, PGF backend
%%
%% To include the figure in your LaTeX document, write
%%   \input{<filename>.pgf}
%%
%% Make sure the required packages are loaded in your preamble
%%   \usepackage{pgf}
%%
%% Figures using additional raster images can only be included by \input if
%% they are in the same directory as the main LaTeX file. For loading figures
%% from other directories you can use the `import` package
%%   \usepackage{import}
%% and then include the figures with
%%   \import{<path to file>}{<filename>.pgf}
%%
%% Matplotlib used the following preamble
%%   \usepackage{fontspec}
%%   \setmainfont{DejaVu Serif}
%%   \setsansfont{DejaVu Sans}
%%   \setmonofont{DejaVu Sans Mono}
%%
\begingroup%
\makeatletter%
\begin{pgfpicture}%
\pgfpathrectangle{\pgfpointorigin}{\pgfqpoint{5.460193in}{3.681079in}}%
\pgfusepath{use as bounding box, clip}%
\begin{pgfscope}%
\pgfsetbuttcap%
\pgfsetmiterjoin%
\definecolor{currentfill}{rgb}{1.000000,1.000000,1.000000}%
\pgfsetfillcolor{currentfill}%
\pgfsetlinewidth{0.000000pt}%
\definecolor{currentstroke}{rgb}{1.000000,1.000000,1.000000}%
\pgfsetstrokecolor{currentstroke}%
\pgfsetdash{}{0pt}%
\pgfpathmoveto{\pgfqpoint{0.000000in}{0.000000in}}%
\pgfpathlineto{\pgfqpoint{5.460193in}{0.000000in}}%
\pgfpathlineto{\pgfqpoint{5.460193in}{3.681079in}}%
\pgfpathlineto{\pgfqpoint{0.000000in}{3.681079in}}%
\pgfpathclose%
\pgfusepath{fill}%
\end{pgfscope}%
\begin{pgfscope}%
\pgfsetbuttcap%
\pgfsetmiterjoin%
\definecolor{currentfill}{rgb}{1.000000,1.000000,1.000000}%
\pgfsetfillcolor{currentfill}%
\pgfsetlinewidth{0.000000pt}%
\definecolor{currentstroke}{rgb}{0.000000,0.000000,0.000000}%
\pgfsetstrokecolor{currentstroke}%
\pgfsetstrokeopacity{0.000000}%
\pgfsetdash{}{0pt}%
\pgfpathmoveto{\pgfqpoint{0.675193in}{0.526079in}}%
\pgfpathlineto{\pgfqpoint{5.325193in}{0.526079in}}%
\pgfpathlineto{\pgfqpoint{5.325193in}{3.546079in}}%
\pgfpathlineto{\pgfqpoint{0.675193in}{3.546079in}}%
\pgfpathclose%
\pgfusepath{fill}%
\end{pgfscope}%
\begin{pgfscope}%
\pgfsetbuttcap%
\pgfsetroundjoin%
\definecolor{currentfill}{rgb}{0.000000,0.000000,0.000000}%
\pgfsetfillcolor{currentfill}%
\pgfsetlinewidth{0.803000pt}%
\definecolor{currentstroke}{rgb}{0.000000,0.000000,0.000000}%
\pgfsetstrokecolor{currentstroke}%
\pgfsetdash{}{0pt}%
\pgfsys@defobject{currentmarker}{\pgfqpoint{0.000000in}{-0.048611in}}{\pgfqpoint{0.000000in}{0.000000in}}{%
\pgfpathmoveto{\pgfqpoint{0.000000in}{0.000000in}}%
\pgfpathlineto{\pgfqpoint{0.000000in}{-0.048611in}}%
\pgfusepath{stroke,fill}%
}%
\begin{pgfscope}%
\pgfsys@transformshift{1.495889in}{0.526079in}%
\pgfsys@useobject{currentmarker}{}%
\end{pgfscope}%
\end{pgfscope}%
\begin{pgfscope}%
\pgftext[x=1.495889in,y=0.428857in,,top]{\rmfamily\fontsize{10.000000}{12.000000}\selectfont \(\displaystyle 0.2\)}%
\end{pgfscope}%
\begin{pgfscope}%
\pgfsetbuttcap%
\pgfsetroundjoin%
\definecolor{currentfill}{rgb}{0.000000,0.000000,0.000000}%
\pgfsetfillcolor{currentfill}%
\pgfsetlinewidth{0.803000pt}%
\definecolor{currentstroke}{rgb}{0.000000,0.000000,0.000000}%
\pgfsetstrokecolor{currentstroke}%
\pgfsetdash{}{0pt}%
\pgfsys@defobject{currentmarker}{\pgfqpoint{0.000000in}{-0.048611in}}{\pgfqpoint{0.000000in}{0.000000in}}{%
\pgfpathmoveto{\pgfqpoint{0.000000in}{0.000000in}}%
\pgfpathlineto{\pgfqpoint{0.000000in}{-0.048611in}}%
\pgfusepath{stroke,fill}%
}%
\begin{pgfscope}%
\pgfsys@transformshift{2.393477in}{0.526079in}%
\pgfsys@useobject{currentmarker}{}%
\end{pgfscope}%
\end{pgfscope}%
\begin{pgfscope}%
\pgftext[x=2.393477in,y=0.428857in,,top]{\rmfamily\fontsize{10.000000}{12.000000}\selectfont \(\displaystyle 0.4\)}%
\end{pgfscope}%
\begin{pgfscope}%
\pgfsetbuttcap%
\pgfsetroundjoin%
\definecolor{currentfill}{rgb}{0.000000,0.000000,0.000000}%
\pgfsetfillcolor{currentfill}%
\pgfsetlinewidth{0.803000pt}%
\definecolor{currentstroke}{rgb}{0.000000,0.000000,0.000000}%
\pgfsetstrokecolor{currentstroke}%
\pgfsetdash{}{0pt}%
\pgfsys@defobject{currentmarker}{\pgfqpoint{0.000000in}{-0.048611in}}{\pgfqpoint{0.000000in}{0.000000in}}{%
\pgfpathmoveto{\pgfqpoint{0.000000in}{0.000000in}}%
\pgfpathlineto{\pgfqpoint{0.000000in}{-0.048611in}}%
\pgfusepath{stroke,fill}%
}%
\begin{pgfscope}%
\pgfsys@transformshift{3.291064in}{0.526079in}%
\pgfsys@useobject{currentmarker}{}%
\end{pgfscope}%
\end{pgfscope}%
\begin{pgfscope}%
\pgftext[x=3.291064in,y=0.428857in,,top]{\rmfamily\fontsize{10.000000}{12.000000}\selectfont \(\displaystyle 0.6\)}%
\end{pgfscope}%
\begin{pgfscope}%
\pgfsetbuttcap%
\pgfsetroundjoin%
\definecolor{currentfill}{rgb}{0.000000,0.000000,0.000000}%
\pgfsetfillcolor{currentfill}%
\pgfsetlinewidth{0.803000pt}%
\definecolor{currentstroke}{rgb}{0.000000,0.000000,0.000000}%
\pgfsetstrokecolor{currentstroke}%
\pgfsetdash{}{0pt}%
\pgfsys@defobject{currentmarker}{\pgfqpoint{0.000000in}{-0.048611in}}{\pgfqpoint{0.000000in}{0.000000in}}{%
\pgfpathmoveto{\pgfqpoint{0.000000in}{0.000000in}}%
\pgfpathlineto{\pgfqpoint{0.000000in}{-0.048611in}}%
\pgfusepath{stroke,fill}%
}%
\begin{pgfscope}%
\pgfsys@transformshift{4.188652in}{0.526079in}%
\pgfsys@useobject{currentmarker}{}%
\end{pgfscope}%
\end{pgfscope}%
\begin{pgfscope}%
\pgftext[x=4.188652in,y=0.428857in,,top]{\rmfamily\fontsize{10.000000}{12.000000}\selectfont \(\displaystyle 0.8\)}%
\end{pgfscope}%
\begin{pgfscope}%
\pgfsetbuttcap%
\pgfsetroundjoin%
\definecolor{currentfill}{rgb}{0.000000,0.000000,0.000000}%
\pgfsetfillcolor{currentfill}%
\pgfsetlinewidth{0.803000pt}%
\definecolor{currentstroke}{rgb}{0.000000,0.000000,0.000000}%
\pgfsetstrokecolor{currentstroke}%
\pgfsetdash{}{0pt}%
\pgfsys@defobject{currentmarker}{\pgfqpoint{0.000000in}{-0.048611in}}{\pgfqpoint{0.000000in}{0.000000in}}{%
\pgfpathmoveto{\pgfqpoint{0.000000in}{0.000000in}}%
\pgfpathlineto{\pgfqpoint{0.000000in}{-0.048611in}}%
\pgfusepath{stroke,fill}%
}%
\begin{pgfscope}%
\pgfsys@transformshift{5.086239in}{0.526079in}%
\pgfsys@useobject{currentmarker}{}%
\end{pgfscope}%
\end{pgfscope}%
\begin{pgfscope}%
\pgftext[x=5.086239in,y=0.428857in,,top]{\rmfamily\fontsize{10.000000}{12.000000}\selectfont \(\displaystyle 1.0\)}%
\end{pgfscope}%
\begin{pgfscope}%
\pgftext[x=3.000193in,y=0.238889in,,top]{\rmfamily\fontsize{10.000000}{12.000000}\selectfont \(\displaystyle y/L\)}%
\end{pgfscope}%
\begin{pgfscope}%
\pgfsetbuttcap%
\pgfsetroundjoin%
\definecolor{currentfill}{rgb}{0.000000,0.000000,0.000000}%
\pgfsetfillcolor{currentfill}%
\pgfsetlinewidth{0.803000pt}%
\definecolor{currentstroke}{rgb}{0.000000,0.000000,0.000000}%
\pgfsetstrokecolor{currentstroke}%
\pgfsetdash{}{0pt}%
\pgfsys@defobject{currentmarker}{\pgfqpoint{-0.048611in}{0.000000in}}{\pgfqpoint{0.000000in}{0.000000in}}{%
\pgfpathmoveto{\pgfqpoint{0.000000in}{0.000000in}}%
\pgfpathlineto{\pgfqpoint{-0.048611in}{0.000000in}}%
\pgfusepath{stroke,fill}%
}%
\begin{pgfscope}%
\pgfsys@transformshift{0.675193in}{0.526079in}%
\pgfsys@useobject{currentmarker}{}%
\end{pgfscope}%
\end{pgfscope}%
\begin{pgfscope}%
\pgftext[x=0.289968in,y=0.473318in,left,base]{\rmfamily\fontsize{10.000000}{12.000000}\selectfont \(\displaystyle 10^{-2}\)}%
\end{pgfscope}%
\begin{pgfscope}%
\pgfsetbuttcap%
\pgfsetroundjoin%
\definecolor{currentfill}{rgb}{0.000000,0.000000,0.000000}%
\pgfsetfillcolor{currentfill}%
\pgfsetlinewidth{0.803000pt}%
\definecolor{currentstroke}{rgb}{0.000000,0.000000,0.000000}%
\pgfsetstrokecolor{currentstroke}%
\pgfsetdash{}{0pt}%
\pgfsys@defobject{currentmarker}{\pgfqpoint{-0.048611in}{0.000000in}}{\pgfqpoint{0.000000in}{0.000000in}}{%
\pgfpathmoveto{\pgfqpoint{0.000000in}{0.000000in}}%
\pgfpathlineto{\pgfqpoint{-0.048611in}{0.000000in}}%
\pgfusepath{stroke,fill}%
}%
\begin{pgfscope}%
\pgfsys@transformshift{0.675193in}{1.966187in}%
\pgfsys@useobject{currentmarker}{}%
\end{pgfscope}%
\end{pgfscope}%
\begin{pgfscope}%
\pgftext[x=0.289968in,y=1.913426in,left,base]{\rmfamily\fontsize{10.000000}{12.000000}\selectfont \(\displaystyle 10^{-1}\)}%
\end{pgfscope}%
\begin{pgfscope}%
\pgfsetbuttcap%
\pgfsetroundjoin%
\definecolor{currentfill}{rgb}{0.000000,0.000000,0.000000}%
\pgfsetfillcolor{currentfill}%
\pgfsetlinewidth{0.803000pt}%
\definecolor{currentstroke}{rgb}{0.000000,0.000000,0.000000}%
\pgfsetstrokecolor{currentstroke}%
\pgfsetdash{}{0pt}%
\pgfsys@defobject{currentmarker}{\pgfqpoint{-0.048611in}{0.000000in}}{\pgfqpoint{0.000000in}{0.000000in}}{%
\pgfpathmoveto{\pgfqpoint{0.000000in}{0.000000in}}%
\pgfpathlineto{\pgfqpoint{-0.048611in}{0.000000in}}%
\pgfusepath{stroke,fill}%
}%
\begin{pgfscope}%
\pgfsys@transformshift{0.675193in}{3.406295in}%
\pgfsys@useobject{currentmarker}{}%
\end{pgfscope}%
\end{pgfscope}%
\begin{pgfscope}%
\pgftext[x=0.376774in,y=3.353534in,left,base]{\rmfamily\fontsize{10.000000}{12.000000}\selectfont \(\displaystyle 10^{0}\)}%
\end{pgfscope}%
\begin{pgfscope}%
\pgfsetbuttcap%
\pgfsetroundjoin%
\definecolor{currentfill}{rgb}{0.000000,0.000000,0.000000}%
\pgfsetfillcolor{currentfill}%
\pgfsetlinewidth{0.602250pt}%
\definecolor{currentstroke}{rgb}{0.000000,0.000000,0.000000}%
\pgfsetstrokecolor{currentstroke}%
\pgfsetdash{}{0pt}%
\pgfsys@defobject{currentmarker}{\pgfqpoint{-0.027778in}{0.000000in}}{\pgfqpoint{0.000000in}{0.000000in}}{%
\pgfpathmoveto{\pgfqpoint{0.000000in}{0.000000in}}%
\pgfpathlineto{\pgfqpoint{-0.027778in}{0.000000in}}%
\pgfusepath{stroke,fill}%
}%
\begin{pgfscope}%
\pgfsys@transformshift{0.675193in}{0.959595in}%
\pgfsys@useobject{currentmarker}{}%
\end{pgfscope}%
\end{pgfscope}%
\begin{pgfscope}%
\pgfsetbuttcap%
\pgfsetroundjoin%
\definecolor{currentfill}{rgb}{0.000000,0.000000,0.000000}%
\pgfsetfillcolor{currentfill}%
\pgfsetlinewidth{0.602250pt}%
\definecolor{currentstroke}{rgb}{0.000000,0.000000,0.000000}%
\pgfsetstrokecolor{currentstroke}%
\pgfsetdash{}{0pt}%
\pgfsys@defobject{currentmarker}{\pgfqpoint{-0.027778in}{0.000000in}}{\pgfqpoint{0.000000in}{0.000000in}}{%
\pgfpathmoveto{\pgfqpoint{0.000000in}{0.000000in}}%
\pgfpathlineto{\pgfqpoint{-0.027778in}{0.000000in}}%
\pgfusepath{stroke,fill}%
}%
\begin{pgfscope}%
\pgfsys@transformshift{0.675193in}{1.213186in}%
\pgfsys@useobject{currentmarker}{}%
\end{pgfscope}%
\end{pgfscope}%
\begin{pgfscope}%
\pgfsetbuttcap%
\pgfsetroundjoin%
\definecolor{currentfill}{rgb}{0.000000,0.000000,0.000000}%
\pgfsetfillcolor{currentfill}%
\pgfsetlinewidth{0.602250pt}%
\definecolor{currentstroke}{rgb}{0.000000,0.000000,0.000000}%
\pgfsetstrokecolor{currentstroke}%
\pgfsetdash{}{0pt}%
\pgfsys@defobject{currentmarker}{\pgfqpoint{-0.027778in}{0.000000in}}{\pgfqpoint{0.000000in}{0.000000in}}{%
\pgfpathmoveto{\pgfqpoint{0.000000in}{0.000000in}}%
\pgfpathlineto{\pgfqpoint{-0.027778in}{0.000000in}}%
\pgfusepath{stroke,fill}%
}%
\begin{pgfscope}%
\pgfsys@transformshift{0.675193in}{1.393111in}%
\pgfsys@useobject{currentmarker}{}%
\end{pgfscope}%
\end{pgfscope}%
\begin{pgfscope}%
\pgfsetbuttcap%
\pgfsetroundjoin%
\definecolor{currentfill}{rgb}{0.000000,0.000000,0.000000}%
\pgfsetfillcolor{currentfill}%
\pgfsetlinewidth{0.602250pt}%
\definecolor{currentstroke}{rgb}{0.000000,0.000000,0.000000}%
\pgfsetstrokecolor{currentstroke}%
\pgfsetdash{}{0pt}%
\pgfsys@defobject{currentmarker}{\pgfqpoint{-0.027778in}{0.000000in}}{\pgfqpoint{0.000000in}{0.000000in}}{%
\pgfpathmoveto{\pgfqpoint{0.000000in}{0.000000in}}%
\pgfpathlineto{\pgfqpoint{-0.027778in}{0.000000in}}%
\pgfusepath{stroke,fill}%
}%
\begin{pgfscope}%
\pgfsys@transformshift{0.675193in}{1.532672in}%
\pgfsys@useobject{currentmarker}{}%
\end{pgfscope}%
\end{pgfscope}%
\begin{pgfscope}%
\pgfsetbuttcap%
\pgfsetroundjoin%
\definecolor{currentfill}{rgb}{0.000000,0.000000,0.000000}%
\pgfsetfillcolor{currentfill}%
\pgfsetlinewidth{0.602250pt}%
\definecolor{currentstroke}{rgb}{0.000000,0.000000,0.000000}%
\pgfsetstrokecolor{currentstroke}%
\pgfsetdash{}{0pt}%
\pgfsys@defobject{currentmarker}{\pgfqpoint{-0.027778in}{0.000000in}}{\pgfqpoint{0.000000in}{0.000000in}}{%
\pgfpathmoveto{\pgfqpoint{0.000000in}{0.000000in}}%
\pgfpathlineto{\pgfqpoint{-0.027778in}{0.000000in}}%
\pgfusepath{stroke,fill}%
}%
\begin{pgfscope}%
\pgfsys@transformshift{0.675193in}{1.646701in}%
\pgfsys@useobject{currentmarker}{}%
\end{pgfscope}%
\end{pgfscope}%
\begin{pgfscope}%
\pgfsetbuttcap%
\pgfsetroundjoin%
\definecolor{currentfill}{rgb}{0.000000,0.000000,0.000000}%
\pgfsetfillcolor{currentfill}%
\pgfsetlinewidth{0.602250pt}%
\definecolor{currentstroke}{rgb}{0.000000,0.000000,0.000000}%
\pgfsetstrokecolor{currentstroke}%
\pgfsetdash{}{0pt}%
\pgfsys@defobject{currentmarker}{\pgfqpoint{-0.027778in}{0.000000in}}{\pgfqpoint{0.000000in}{0.000000in}}{%
\pgfpathmoveto{\pgfqpoint{0.000000in}{0.000000in}}%
\pgfpathlineto{\pgfqpoint{-0.027778in}{0.000000in}}%
\pgfusepath{stroke,fill}%
}%
\begin{pgfscope}%
\pgfsys@transformshift{0.675193in}{1.743112in}%
\pgfsys@useobject{currentmarker}{}%
\end{pgfscope}%
\end{pgfscope}%
\begin{pgfscope}%
\pgfsetbuttcap%
\pgfsetroundjoin%
\definecolor{currentfill}{rgb}{0.000000,0.000000,0.000000}%
\pgfsetfillcolor{currentfill}%
\pgfsetlinewidth{0.602250pt}%
\definecolor{currentstroke}{rgb}{0.000000,0.000000,0.000000}%
\pgfsetstrokecolor{currentstroke}%
\pgfsetdash{}{0pt}%
\pgfsys@defobject{currentmarker}{\pgfqpoint{-0.027778in}{0.000000in}}{\pgfqpoint{0.000000in}{0.000000in}}{%
\pgfpathmoveto{\pgfqpoint{0.000000in}{0.000000in}}%
\pgfpathlineto{\pgfqpoint{-0.027778in}{0.000000in}}%
\pgfusepath{stroke,fill}%
}%
\begin{pgfscope}%
\pgfsys@transformshift{0.675193in}{1.826626in}%
\pgfsys@useobject{currentmarker}{}%
\end{pgfscope}%
\end{pgfscope}%
\begin{pgfscope}%
\pgfsetbuttcap%
\pgfsetroundjoin%
\definecolor{currentfill}{rgb}{0.000000,0.000000,0.000000}%
\pgfsetfillcolor{currentfill}%
\pgfsetlinewidth{0.602250pt}%
\definecolor{currentstroke}{rgb}{0.000000,0.000000,0.000000}%
\pgfsetstrokecolor{currentstroke}%
\pgfsetdash{}{0pt}%
\pgfsys@defobject{currentmarker}{\pgfqpoint{-0.027778in}{0.000000in}}{\pgfqpoint{0.000000in}{0.000000in}}{%
\pgfpathmoveto{\pgfqpoint{0.000000in}{0.000000in}}%
\pgfpathlineto{\pgfqpoint{-0.027778in}{0.000000in}}%
\pgfusepath{stroke,fill}%
}%
\begin{pgfscope}%
\pgfsys@transformshift{0.675193in}{1.900292in}%
\pgfsys@useobject{currentmarker}{}%
\end{pgfscope}%
\end{pgfscope}%
\begin{pgfscope}%
\pgfsetbuttcap%
\pgfsetroundjoin%
\definecolor{currentfill}{rgb}{0.000000,0.000000,0.000000}%
\pgfsetfillcolor{currentfill}%
\pgfsetlinewidth{0.602250pt}%
\definecolor{currentstroke}{rgb}{0.000000,0.000000,0.000000}%
\pgfsetstrokecolor{currentstroke}%
\pgfsetdash{}{0pt}%
\pgfsys@defobject{currentmarker}{\pgfqpoint{-0.027778in}{0.000000in}}{\pgfqpoint{0.000000in}{0.000000in}}{%
\pgfpathmoveto{\pgfqpoint{0.000000in}{0.000000in}}%
\pgfpathlineto{\pgfqpoint{-0.027778in}{0.000000in}}%
\pgfusepath{stroke,fill}%
}%
\begin{pgfscope}%
\pgfsys@transformshift{0.675193in}{2.399703in}%
\pgfsys@useobject{currentmarker}{}%
\end{pgfscope}%
\end{pgfscope}%
\begin{pgfscope}%
\pgfsetbuttcap%
\pgfsetroundjoin%
\definecolor{currentfill}{rgb}{0.000000,0.000000,0.000000}%
\pgfsetfillcolor{currentfill}%
\pgfsetlinewidth{0.602250pt}%
\definecolor{currentstroke}{rgb}{0.000000,0.000000,0.000000}%
\pgfsetstrokecolor{currentstroke}%
\pgfsetdash{}{0pt}%
\pgfsys@defobject{currentmarker}{\pgfqpoint{-0.027778in}{0.000000in}}{\pgfqpoint{0.000000in}{0.000000in}}{%
\pgfpathmoveto{\pgfqpoint{0.000000in}{0.000000in}}%
\pgfpathlineto{\pgfqpoint{-0.027778in}{0.000000in}}%
\pgfusepath{stroke,fill}%
}%
\begin{pgfscope}%
\pgfsys@transformshift{0.675193in}{2.653293in}%
\pgfsys@useobject{currentmarker}{}%
\end{pgfscope}%
\end{pgfscope}%
\begin{pgfscope}%
\pgfsetbuttcap%
\pgfsetroundjoin%
\definecolor{currentfill}{rgb}{0.000000,0.000000,0.000000}%
\pgfsetfillcolor{currentfill}%
\pgfsetlinewidth{0.602250pt}%
\definecolor{currentstroke}{rgb}{0.000000,0.000000,0.000000}%
\pgfsetstrokecolor{currentstroke}%
\pgfsetdash{}{0pt}%
\pgfsys@defobject{currentmarker}{\pgfqpoint{-0.027778in}{0.000000in}}{\pgfqpoint{0.000000in}{0.000000in}}{%
\pgfpathmoveto{\pgfqpoint{0.000000in}{0.000000in}}%
\pgfpathlineto{\pgfqpoint{-0.027778in}{0.000000in}}%
\pgfusepath{stroke,fill}%
}%
\begin{pgfscope}%
\pgfsys@transformshift{0.675193in}{2.833219in}%
\pgfsys@useobject{currentmarker}{}%
\end{pgfscope}%
\end{pgfscope}%
\begin{pgfscope}%
\pgfsetbuttcap%
\pgfsetroundjoin%
\definecolor{currentfill}{rgb}{0.000000,0.000000,0.000000}%
\pgfsetfillcolor{currentfill}%
\pgfsetlinewidth{0.602250pt}%
\definecolor{currentstroke}{rgb}{0.000000,0.000000,0.000000}%
\pgfsetstrokecolor{currentstroke}%
\pgfsetdash{}{0pt}%
\pgfsys@defobject{currentmarker}{\pgfqpoint{-0.027778in}{0.000000in}}{\pgfqpoint{0.000000in}{0.000000in}}{%
\pgfpathmoveto{\pgfqpoint{0.000000in}{0.000000in}}%
\pgfpathlineto{\pgfqpoint{-0.027778in}{0.000000in}}%
\pgfusepath{stroke,fill}%
}%
\begin{pgfscope}%
\pgfsys@transformshift{0.675193in}{2.972780in}%
\pgfsys@useobject{currentmarker}{}%
\end{pgfscope}%
\end{pgfscope}%
\begin{pgfscope}%
\pgfsetbuttcap%
\pgfsetroundjoin%
\definecolor{currentfill}{rgb}{0.000000,0.000000,0.000000}%
\pgfsetfillcolor{currentfill}%
\pgfsetlinewidth{0.602250pt}%
\definecolor{currentstroke}{rgb}{0.000000,0.000000,0.000000}%
\pgfsetstrokecolor{currentstroke}%
\pgfsetdash{}{0pt}%
\pgfsys@defobject{currentmarker}{\pgfqpoint{-0.027778in}{0.000000in}}{\pgfqpoint{0.000000in}{0.000000in}}{%
\pgfpathmoveto{\pgfqpoint{0.000000in}{0.000000in}}%
\pgfpathlineto{\pgfqpoint{-0.027778in}{0.000000in}}%
\pgfusepath{stroke,fill}%
}%
\begin{pgfscope}%
\pgfsys@transformshift{0.675193in}{3.086809in}%
\pgfsys@useobject{currentmarker}{}%
\end{pgfscope}%
\end{pgfscope}%
\begin{pgfscope}%
\pgfsetbuttcap%
\pgfsetroundjoin%
\definecolor{currentfill}{rgb}{0.000000,0.000000,0.000000}%
\pgfsetfillcolor{currentfill}%
\pgfsetlinewidth{0.602250pt}%
\definecolor{currentstroke}{rgb}{0.000000,0.000000,0.000000}%
\pgfsetstrokecolor{currentstroke}%
\pgfsetdash{}{0pt}%
\pgfsys@defobject{currentmarker}{\pgfqpoint{-0.027778in}{0.000000in}}{\pgfqpoint{0.000000in}{0.000000in}}{%
\pgfpathmoveto{\pgfqpoint{0.000000in}{0.000000in}}%
\pgfpathlineto{\pgfqpoint{-0.027778in}{0.000000in}}%
\pgfusepath{stroke,fill}%
}%
\begin{pgfscope}%
\pgfsys@transformshift{0.675193in}{3.183220in}%
\pgfsys@useobject{currentmarker}{}%
\end{pgfscope}%
\end{pgfscope}%
\begin{pgfscope}%
\pgfsetbuttcap%
\pgfsetroundjoin%
\definecolor{currentfill}{rgb}{0.000000,0.000000,0.000000}%
\pgfsetfillcolor{currentfill}%
\pgfsetlinewidth{0.602250pt}%
\definecolor{currentstroke}{rgb}{0.000000,0.000000,0.000000}%
\pgfsetstrokecolor{currentstroke}%
\pgfsetdash{}{0pt}%
\pgfsys@defobject{currentmarker}{\pgfqpoint{-0.027778in}{0.000000in}}{\pgfqpoint{0.000000in}{0.000000in}}{%
\pgfpathmoveto{\pgfqpoint{0.000000in}{0.000000in}}%
\pgfpathlineto{\pgfqpoint{-0.027778in}{0.000000in}}%
\pgfusepath{stroke,fill}%
}%
\begin{pgfscope}%
\pgfsys@transformshift{0.675193in}{3.266734in}%
\pgfsys@useobject{currentmarker}{}%
\end{pgfscope}%
\end{pgfscope}%
\begin{pgfscope}%
\pgfsetbuttcap%
\pgfsetroundjoin%
\definecolor{currentfill}{rgb}{0.000000,0.000000,0.000000}%
\pgfsetfillcolor{currentfill}%
\pgfsetlinewidth{0.602250pt}%
\definecolor{currentstroke}{rgb}{0.000000,0.000000,0.000000}%
\pgfsetstrokecolor{currentstroke}%
\pgfsetdash{}{0pt}%
\pgfsys@defobject{currentmarker}{\pgfqpoint{-0.027778in}{0.000000in}}{\pgfqpoint{0.000000in}{0.000000in}}{%
\pgfpathmoveto{\pgfqpoint{0.000000in}{0.000000in}}%
\pgfpathlineto{\pgfqpoint{-0.027778in}{0.000000in}}%
\pgfusepath{stroke,fill}%
}%
\begin{pgfscope}%
\pgfsys@transformshift{0.675193in}{3.340399in}%
\pgfsys@useobject{currentmarker}{}%
\end{pgfscope}%
\end{pgfscope}%
\begin{pgfscope}%
\pgftext[x=0.234413in,y=2.036079in,,bottom,rotate=90.000000]{\rmfamily\fontsize{10.000000}{12.000000}\selectfont Dimensionless force}%
\end{pgfscope}%
\begin{pgfscope}%
\pgfpathrectangle{\pgfqpoint{0.675193in}{0.526079in}}{\pgfqpoint{4.650000in}{3.020000in}} %
\pgfusepath{clip}%
\pgfsetrectcap%
\pgfsetroundjoin%
\pgfsetlinewidth{1.505625pt}%
\definecolor{currentstroke}{rgb}{1.000000,0.000000,0.000000}%
\pgfsetstrokecolor{currentstroke}%
\pgfsetstrokeopacity{0.750000}%
\pgfsetdash{}{0pt}%
\pgfpathmoveto{\pgfqpoint{0.991862in}{3.257881in}}%
\pgfpathlineto{\pgfqpoint{1.017859in}{3.270387in}}%
\pgfpathlineto{\pgfqpoint{1.042736in}{3.282256in}}%
\pgfpathlineto{\pgfqpoint{1.066497in}{3.292094in}}%
\pgfusepath{stroke}%
\end{pgfscope}%
\begin{pgfscope}%
\pgfpathrectangle{\pgfqpoint{0.675193in}{0.526079in}}{\pgfqpoint{4.650000in}{3.020000in}} %
\pgfusepath{clip}%
\pgfsetrectcap%
\pgfsetroundjoin%
\pgfsetlinewidth{1.505625pt}%
\definecolor{currentstroke}{rgb}{0.000000,0.000000,1.000000}%
\pgfsetstrokecolor{currentstroke}%
\pgfsetstrokeopacity{0.750000}%
\pgfsetdash{}{0pt}%
\pgfpathmoveto{\pgfqpoint{0.991862in}{0.841496in}}%
\pgfpathlineto{\pgfqpoint{1.017859in}{0.820755in}}%
\pgfpathlineto{\pgfqpoint{1.042736in}{0.798492in}}%
\pgfpathlineto{\pgfqpoint{1.066497in}{0.773090in}}%
\pgfusepath{stroke}%
\end{pgfscope}%
\begin{pgfscope}%
\pgfpathrectangle{\pgfqpoint{0.675193in}{0.526079in}}{\pgfqpoint{4.650000in}{3.020000in}} %
\pgfusepath{clip}%
\pgfsetrectcap%
\pgfsetroundjoin%
\pgfsetlinewidth{1.505625pt}%
\definecolor{currentstroke}{rgb}{0.000000,0.750000,0.750000}%
\pgfsetstrokecolor{currentstroke}%
\pgfsetstrokeopacity{0.750000}%
\pgfsetdash{}{0pt}%
\pgfpathmoveto{\pgfqpoint{0.991862in}{2.389275in}}%
\pgfpathlineto{\pgfqpoint{1.017859in}{2.328006in}}%
\pgfpathlineto{\pgfqpoint{1.042736in}{2.274395in}}%
\pgfpathlineto{\pgfqpoint{1.066497in}{2.225786in}}%
\pgfusepath{stroke}%
\end{pgfscope}%
\begin{pgfscope}%
\pgfpathrectangle{\pgfqpoint{0.675193in}{0.526079in}}{\pgfqpoint{4.650000in}{3.020000in}} %
\pgfusepath{clip}%
\pgfsetrectcap%
\pgfsetroundjoin%
\pgfsetlinewidth{1.505625pt}%
\definecolor{currentstroke}{rgb}{1.000000,0.000000,0.000000}%
\pgfsetstrokecolor{currentstroke}%
\pgfsetstrokeopacity{0.750000}%
\pgfsetdash{}{0pt}%
\pgfpathmoveto{\pgfqpoint{0.965930in}{3.312843in}}%
\pgfpathlineto{\pgfqpoint{0.995533in}{3.321196in}}%
\pgfpathlineto{\pgfqpoint{1.023927in}{3.329119in}}%
\pgfpathlineto{\pgfqpoint{1.051125in}{3.335610in}}%
\pgfpathlineto{\pgfqpoint{1.077137in}{3.341042in}}%
\pgfpathlineto{\pgfqpoint{1.101975in}{3.345668in}}%
\pgfpathlineto{\pgfqpoint{1.125650in}{3.349670in}}%
\pgfpathlineto{\pgfqpoint{1.148175in}{3.353180in}}%
\pgfusepath{stroke}%
\end{pgfscope}%
\begin{pgfscope}%
\pgfpathrectangle{\pgfqpoint{0.675193in}{0.526079in}}{\pgfqpoint{4.650000in}{3.020000in}} %
\pgfusepath{clip}%
\pgfsetrectcap%
\pgfsetroundjoin%
\pgfsetlinewidth{1.505625pt}%
\definecolor{currentstroke}{rgb}{0.000000,0.000000,1.000000}%
\pgfsetstrokecolor{currentstroke}%
\pgfsetstrokeopacity{0.750000}%
\pgfsetdash{}{0pt}%
\pgfpathmoveto{\pgfqpoint{0.965930in}{1.296446in}}%
\pgfpathlineto{\pgfqpoint{0.995533in}{1.278034in}}%
\pgfpathlineto{\pgfqpoint{1.023927in}{1.258410in}}%
\pgfpathlineto{\pgfqpoint{1.051125in}{1.236424in}}%
\pgfpathlineto{\pgfqpoint{1.077137in}{1.212311in}}%
\pgfpathlineto{\pgfqpoint{1.101975in}{1.186190in}}%
\pgfpathlineto{\pgfqpoint{1.125650in}{1.158107in}}%
\pgfpathlineto{\pgfqpoint{1.148175in}{1.128050in}}%
\pgfusepath{stroke}%
\end{pgfscope}%
\begin{pgfscope}%
\pgfpathrectangle{\pgfqpoint{0.675193in}{0.526079in}}{\pgfqpoint{4.650000in}{3.020000in}} %
\pgfusepath{clip}%
\pgfsetrectcap%
\pgfsetroundjoin%
\pgfsetlinewidth{1.505625pt}%
\definecolor{currentstroke}{rgb}{0.000000,0.750000,0.750000}%
\pgfsetstrokecolor{currentstroke}%
\pgfsetstrokeopacity{0.750000}%
\pgfsetdash{}{0pt}%
\pgfpathmoveto{\pgfqpoint{0.965930in}{2.002835in}}%
\pgfpathlineto{\pgfqpoint{0.995533in}{1.922291in}}%
\pgfpathlineto{\pgfqpoint{1.023927in}{1.851826in}}%
\pgfpathlineto{\pgfqpoint{1.051125in}{1.788693in}}%
\pgfpathlineto{\pgfqpoint{1.077137in}{1.731917in}}%
\pgfpathlineto{\pgfqpoint{1.101975in}{1.680697in}}%
\pgfpathlineto{\pgfqpoint{1.125650in}{1.634371in}}%
\pgfpathlineto{\pgfqpoint{1.148175in}{1.592387in}}%
\pgfusepath{stroke}%
\end{pgfscope}%
\begin{pgfscope}%
\pgfpathrectangle{\pgfqpoint{0.675193in}{0.526079in}}{\pgfqpoint{4.650000in}{3.020000in}} %
\pgfusepath{clip}%
\pgfsetrectcap%
\pgfsetroundjoin%
\pgfsetlinewidth{1.505625pt}%
\definecolor{currentstroke}{rgb}{1.000000,0.000000,0.000000}%
\pgfsetstrokecolor{currentstroke}%
\pgfsetstrokeopacity{0.750000}%
\pgfsetdash{}{0pt}%
\pgfpathmoveto{\pgfqpoint{1.041681in}{3.245098in}}%
\pgfpathlineto{\pgfqpoint{1.077630in}{3.260812in}}%
\pgfpathlineto{\pgfqpoint{1.112004in}{3.275682in}}%
\pgfpathlineto{\pgfqpoint{1.144820in}{3.287646in}}%
\pgfpathlineto{\pgfqpoint{1.176094in}{3.297441in}}%
\pgfpathlineto{\pgfqpoint{1.205844in}{3.305580in}}%
\pgfpathlineto{\pgfqpoint{1.234084in}{3.312429in}}%
\pgfusepath{stroke}%
\end{pgfscope}%
\begin{pgfscope}%
\pgfpathrectangle{\pgfqpoint{0.675193in}{0.526079in}}{\pgfqpoint{4.650000in}{3.020000in}} %
\pgfusepath{clip}%
\pgfsetrectcap%
\pgfsetroundjoin%
\pgfsetlinewidth{1.505625pt}%
\definecolor{currentstroke}{rgb}{0.000000,0.000000,1.000000}%
\pgfsetstrokecolor{currentstroke}%
\pgfsetstrokeopacity{0.750000}%
\pgfsetdash{}{0pt}%
\pgfpathmoveto{\pgfqpoint{1.041681in}{0.632160in}}%
\pgfpathlineto{\pgfqpoint{1.077630in}{0.615684in}}%
\pgfpathlineto{\pgfqpoint{1.112004in}{0.597813in}}%
\pgfpathlineto{\pgfqpoint{1.144820in}{0.576211in}}%
\pgfpathlineto{\pgfqpoint{1.176094in}{0.551378in}}%
\pgfpathlineto{\pgfqpoint{1.205844in}{0.523611in}}%
\pgfpathlineto{\pgfqpoint{1.212807in}{0.516079in}}%
\pgfusepath{stroke}%
\end{pgfscope}%
\begin{pgfscope}%
\pgfpathrectangle{\pgfqpoint{0.675193in}{0.526079in}}{\pgfqpoint{4.650000in}{3.020000in}} %
\pgfusepath{clip}%
\pgfsetrectcap%
\pgfsetroundjoin%
\pgfsetlinewidth{1.505625pt}%
\definecolor{currentstroke}{rgb}{0.000000,0.750000,0.750000}%
\pgfsetstrokecolor{currentstroke}%
\pgfsetstrokeopacity{0.750000}%
\pgfsetdash{}{0pt}%
\pgfpathmoveto{\pgfqpoint{1.041681in}{2.454753in}}%
\pgfpathlineto{\pgfqpoint{1.077630in}{2.382421in}}%
\pgfpathlineto{\pgfqpoint{1.112004in}{2.320332in}}%
\pgfpathlineto{\pgfqpoint{1.144820in}{2.264476in}}%
\pgfpathlineto{\pgfqpoint{1.176094in}{2.214113in}}%
\pgfpathlineto{\pgfqpoint{1.205844in}{2.168611in}}%
\pgfpathlineto{\pgfqpoint{1.234084in}{2.127430in}}%
\pgfusepath{stroke}%
\end{pgfscope}%
\begin{pgfscope}%
\pgfpathrectangle{\pgfqpoint{0.675193in}{0.526079in}}{\pgfqpoint{4.650000in}{3.020000in}} %
\pgfusepath{clip}%
\pgfsetrectcap%
\pgfsetroundjoin%
\pgfsetlinewidth{1.505625pt}%
\definecolor{currentstroke}{rgb}{1.000000,0.000000,0.000000}%
\pgfsetstrokecolor{currentstroke}%
\pgfsetstrokeopacity{0.750000}%
\pgfsetdash{}{0pt}%
\pgfpathmoveto{\pgfqpoint{0.886557in}{3.228510in}}%
\pgfpathlineto{\pgfqpoint{0.917018in}{3.246177in}}%
\pgfpathlineto{\pgfqpoint{0.947047in}{3.262417in}}%
\pgfpathlineto{\pgfqpoint{0.976642in}{3.275073in}}%
\pgfpathlineto{\pgfqpoint{1.005800in}{3.285178in}}%
\pgfpathlineto{\pgfqpoint{1.034518in}{3.293411in}}%
\pgfpathlineto{\pgfqpoint{1.062793in}{3.300242in}}%
\pgfpathlineto{\pgfqpoint{1.090622in}{3.306003in}}%
\pgfpathlineto{\pgfqpoint{1.118001in}{3.310929in}}%
\pgfpathlineto{\pgfqpoint{1.144929in}{3.315195in}}%
\pgfpathlineto{\pgfqpoint{1.171401in}{3.318935in}}%
\pgfpathlineto{\pgfqpoint{1.197416in}{3.322248in}}%
\pgfpathlineto{\pgfqpoint{1.222969in}{3.325212in}}%
\pgfpathlineto{\pgfqpoint{1.248048in}{3.327887in}}%
\pgfpathlineto{\pgfqpoint{1.272654in}{3.330321in}}%
\pgfpathlineto{\pgfqpoint{1.296784in}{3.332552in}}%
\pgfpathlineto{\pgfqpoint{1.320436in}{3.334613in}}%
\pgfpathlineto{\pgfqpoint{1.343612in}{3.336529in}}%
\pgfpathlineto{\pgfqpoint{1.366312in}{3.338321in}}%
\pgfpathlineto{\pgfqpoint{1.388539in}{3.340005in}}%
\pgfpathlineto{\pgfqpoint{1.410295in}{3.341597in}}%
\pgfpathlineto{\pgfqpoint{1.431585in}{3.343108in}}%
\pgfpathlineto{\pgfqpoint{1.452415in}{3.344549in}}%
\pgfpathlineto{\pgfqpoint{1.472790in}{3.345928in}}%
\pgfpathlineto{\pgfqpoint{1.492716in}{3.347253in}}%
\pgfpathlineto{\pgfqpoint{1.512201in}{3.348532in}}%
\pgfpathlineto{\pgfqpoint{1.531251in}{3.349768in}}%
\pgfpathlineto{\pgfqpoint{1.549873in}{3.350967in}}%
\pgfpathlineto{\pgfqpoint{1.568071in}{3.352132in}}%
\pgfpathlineto{\pgfqpoint{1.585852in}{3.353268in}}%
\pgfpathlineto{\pgfqpoint{1.603221in}{3.354377in}}%
\pgfpathlineto{\pgfqpoint{1.620183in}{3.355462in}}%
\pgfpathlineto{\pgfqpoint{1.636740in}{3.356526in}}%
\pgfpathlineto{\pgfqpoint{1.652898in}{3.357570in}}%
\pgfpathlineto{\pgfqpoint{1.668658in}{3.358596in}}%
\pgfpathlineto{\pgfqpoint{1.684024in}{3.359605in}}%
\pgfpathlineto{\pgfqpoint{1.698999in}{3.360600in}}%
\pgfpathlineto{\pgfqpoint{1.713585in}{3.361582in}}%
\pgfpathlineto{\pgfqpoint{1.727784in}{3.362551in}}%
\pgfpathlineto{\pgfqpoint{1.741600in}{3.363508in}}%
\pgfpathlineto{\pgfqpoint{1.755033in}{3.364455in}}%
\pgfpathlineto{\pgfqpoint{1.768088in}{3.365391in}}%
\pgfpathlineto{\pgfqpoint{1.780767in}{3.366318in}}%
\pgfpathlineto{\pgfqpoint{1.793072in}{3.367236in}}%
\pgfpathlineto{\pgfqpoint{1.805007in}{3.368145in}}%
\pgfpathlineto{\pgfqpoint{1.816573in}{3.369046in}}%
\pgfpathlineto{\pgfqpoint{1.827775in}{3.369939in}}%
\pgfpathlineto{\pgfqpoint{1.838615in}{3.370824in}}%
\pgfpathlineto{\pgfqpoint{1.849095in}{3.371701in}}%
\pgfpathlineto{\pgfqpoint{1.859219in}{3.372572in}}%
\pgfpathlineto{\pgfqpoint{1.868987in}{3.373435in}}%
\pgfpathlineto{\pgfqpoint{1.878404in}{3.374291in}}%
\pgfpathlineto{\pgfqpoint{1.887470in}{3.375140in}}%
\pgfpathlineto{\pgfqpoint{1.896187in}{3.375982in}}%
\pgfpathlineto{\pgfqpoint{1.904557in}{3.376817in}}%
\pgfpathlineto{\pgfqpoint{1.912581in}{3.377645in}}%
\pgfpathlineto{\pgfqpoint{1.920262in}{3.378466in}}%
\pgfpathlineto{\pgfqpoint{1.927599in}{3.379278in}}%
\pgfpathlineto{\pgfqpoint{1.934594in}{3.380082in}}%
\pgfusepath{stroke}%
\end{pgfscope}%
\begin{pgfscope}%
\pgfpathrectangle{\pgfqpoint{0.675193in}{0.526079in}}{\pgfqpoint{4.650000in}{3.020000in}} %
\pgfusepath{clip}%
\pgfsetrectcap%
\pgfsetroundjoin%
\pgfsetlinewidth{1.505625pt}%
\definecolor{currentstroke}{rgb}{0.000000,0.000000,1.000000}%
\pgfsetstrokecolor{currentstroke}%
\pgfsetstrokeopacity{0.750000}%
\pgfsetdash{}{0pt}%
\pgfpathmoveto{\pgfqpoint{0.886557in}{1.735498in}}%
\pgfpathlineto{\pgfqpoint{0.917018in}{1.744678in}}%
\pgfpathlineto{\pgfqpoint{0.947047in}{1.752732in}}%
\pgfpathlineto{\pgfqpoint{0.976642in}{1.757386in}}%
\pgfpathlineto{\pgfqpoint{1.005800in}{1.759593in}}%
\pgfpathlineto{\pgfqpoint{1.034518in}{1.759970in}}%
\pgfpathlineto{\pgfqpoint{1.062793in}{1.758946in}}%
\pgfpathlineto{\pgfqpoint{1.090622in}{1.756815in}}%
\pgfpathlineto{\pgfqpoint{1.118001in}{1.753786in}}%
\pgfpathlineto{\pgfqpoint{1.144929in}{1.750011in}}%
\pgfpathlineto{\pgfqpoint{1.171401in}{1.745604in}}%
\pgfpathlineto{\pgfqpoint{1.197416in}{1.740649in}}%
\pgfpathlineto{\pgfqpoint{1.222969in}{1.735208in}}%
\pgfpathlineto{\pgfqpoint{1.248048in}{1.729327in}}%
\pgfpathlineto{\pgfqpoint{1.272654in}{1.723043in}}%
\pgfpathlineto{\pgfqpoint{1.296784in}{1.716383in}}%
\pgfpathlineto{\pgfqpoint{1.320436in}{1.709370in}}%
\pgfpathlineto{\pgfqpoint{1.343612in}{1.702018in}}%
\pgfpathlineto{\pgfqpoint{1.366312in}{1.694337in}}%
\pgfpathlineto{\pgfqpoint{1.388539in}{1.686337in}}%
\pgfpathlineto{\pgfqpoint{1.410295in}{1.678024in}}%
\pgfpathlineto{\pgfqpoint{1.431585in}{1.669398in}}%
\pgfpathlineto{\pgfqpoint{1.452415in}{1.660463in}}%
\pgfpathlineto{\pgfqpoint{1.472790in}{1.651217in}}%
\pgfpathlineto{\pgfqpoint{1.492716in}{1.641660in}}%
\pgfpathlineto{\pgfqpoint{1.512201in}{1.631788in}}%
\pgfpathlineto{\pgfqpoint{1.531251in}{1.621597in}}%
\pgfpathlineto{\pgfqpoint{1.549873in}{1.611083in}}%
\pgfpathlineto{\pgfqpoint{1.568071in}{1.600238in}}%
\pgfpathlineto{\pgfqpoint{1.585852in}{1.589056in}}%
\pgfpathlineto{\pgfqpoint{1.603221in}{1.577530in}}%
\pgfpathlineto{\pgfqpoint{1.620183in}{1.565650in}}%
\pgfpathlineto{\pgfqpoint{1.636740in}{1.553407in}}%
\pgfpathlineto{\pgfqpoint{1.652898in}{1.540791in}}%
\pgfpathlineto{\pgfqpoint{1.668658in}{1.527789in}}%
\pgfpathlineto{\pgfqpoint{1.684024in}{1.514390in}}%
\pgfpathlineto{\pgfqpoint{1.698999in}{1.500582in}}%
\pgfpathlineto{\pgfqpoint{1.713585in}{1.486349in}}%
\pgfpathlineto{\pgfqpoint{1.727784in}{1.471675in}}%
\pgfpathlineto{\pgfqpoint{1.741600in}{1.456544in}}%
\pgfpathlineto{\pgfqpoint{1.755033in}{1.440937in}}%
\pgfpathlineto{\pgfqpoint{1.768088in}{1.424835in}}%
\pgfpathlineto{\pgfqpoint{1.780767in}{1.408218in}}%
\pgfpathlineto{\pgfqpoint{1.793072in}{1.391061in}}%
\pgfpathlineto{\pgfqpoint{1.805007in}{1.373340in}}%
\pgfpathlineto{\pgfqpoint{1.816573in}{1.355029in}}%
\pgfpathlineto{\pgfqpoint{1.827775in}{1.336098in}}%
\pgfpathlineto{\pgfqpoint{1.838615in}{1.316514in}}%
\pgfpathlineto{\pgfqpoint{1.849095in}{1.296243in}}%
\pgfpathlineto{\pgfqpoint{1.859219in}{1.275248in}}%
\pgfpathlineto{\pgfqpoint{1.868987in}{1.253485in}}%
\pgfpathlineto{\pgfqpoint{1.878404in}{1.230910in}}%
\pgfpathlineto{\pgfqpoint{1.887470in}{1.207471in}}%
\pgfpathlineto{\pgfqpoint{1.896187in}{1.183113in}}%
\pgfpathlineto{\pgfqpoint{1.904557in}{1.157773in}}%
\pgfpathlineto{\pgfqpoint{1.912581in}{1.131381in}}%
\pgfpathlineto{\pgfqpoint{1.920262in}{1.103859in}}%
\pgfpathlineto{\pgfqpoint{1.927599in}{1.075118in}}%
\pgfpathlineto{\pgfqpoint{1.934594in}{1.045061in}}%
\pgfusepath{stroke}%
\end{pgfscope}%
\begin{pgfscope}%
\pgfpathrectangle{\pgfqpoint{0.675193in}{0.526079in}}{\pgfqpoint{4.650000in}{3.020000in}} %
\pgfusepath{clip}%
\pgfsetrectcap%
\pgfsetroundjoin%
\pgfsetlinewidth{1.505625pt}%
\definecolor{currentstroke}{rgb}{0.000000,0.750000,0.750000}%
\pgfsetstrokecolor{currentstroke}%
\pgfsetstrokeopacity{0.750000}%
\pgfsetdash{}{0pt}%
\pgfpathmoveto{\pgfqpoint{0.886557in}{2.340150in}}%
\pgfpathlineto{\pgfqpoint{0.917018in}{2.236368in}}%
\pgfpathlineto{\pgfqpoint{0.947047in}{2.145714in}}%
\pgfpathlineto{\pgfqpoint{0.976642in}{2.063248in}}%
\pgfpathlineto{\pgfqpoint{1.005800in}{1.987950in}}%
\pgfpathlineto{\pgfqpoint{1.034518in}{1.918932in}}%
\pgfpathlineto{\pgfqpoint{1.062793in}{1.855451in}}%
\pgfpathlineto{\pgfqpoint{1.090622in}{1.796871in}}%
\pgfpathlineto{\pgfqpoint{1.118001in}{1.742649in}}%
\pgfpathlineto{\pgfqpoint{1.144929in}{1.692320in}}%
\pgfpathlineto{\pgfqpoint{1.171401in}{1.645489in}}%
\pgfpathlineto{\pgfqpoint{1.197416in}{1.601812in}}%
\pgfpathlineto{\pgfqpoint{1.222969in}{1.560990in}}%
\pgfpathlineto{\pgfqpoint{1.248048in}{1.522761in}}%
\pgfpathlineto{\pgfqpoint{1.272654in}{1.486898in}}%
\pgfpathlineto{\pgfqpoint{1.296784in}{1.453198in}}%
\pgfpathlineto{\pgfqpoint{1.320436in}{1.421485in}}%
\pgfpathlineto{\pgfqpoint{1.343612in}{1.391601in}}%
\pgfpathlineto{\pgfqpoint{1.366312in}{1.363404in}}%
\pgfpathlineto{\pgfqpoint{1.388539in}{1.336767in}}%
\pgfpathlineto{\pgfqpoint{1.410295in}{1.311578in}}%
\pgfpathlineto{\pgfqpoint{1.431585in}{1.287735in}}%
\pgfpathlineto{\pgfqpoint{1.452415in}{1.265144in}}%
\pgfpathlineto{\pgfqpoint{1.472790in}{1.243722in}}%
\pgfpathlineto{\pgfqpoint{1.492716in}{1.223394in}}%
\pgfpathlineto{\pgfqpoint{1.512201in}{1.204091in}}%
\pgfpathlineto{\pgfqpoint{1.531251in}{1.185748in}}%
\pgfpathlineto{\pgfqpoint{1.549873in}{1.168308in}}%
\pgfpathlineto{\pgfqpoint{1.568071in}{1.151718in}}%
\pgfpathlineto{\pgfqpoint{1.585852in}{1.135929in}}%
\pgfpathlineto{\pgfqpoint{1.603221in}{1.120896in}}%
\pgfpathlineto{\pgfqpoint{1.620183in}{1.106578in}}%
\pgfpathlineto{\pgfqpoint{1.636740in}{1.092936in}}%
\pgfpathlineto{\pgfqpoint{1.652898in}{1.079935in}}%
\pgfpathlineto{\pgfqpoint{1.668658in}{1.067542in}}%
\pgfpathlineto{\pgfqpoint{1.684024in}{1.055726in}}%
\pgfpathlineto{\pgfqpoint{1.698999in}{1.044461in}}%
\pgfpathlineto{\pgfqpoint{1.713585in}{1.033718in}}%
\pgfpathlineto{\pgfqpoint{1.727784in}{1.023473in}}%
\pgfpathlineto{\pgfqpoint{1.741600in}{1.013704in}}%
\pgfpathlineto{\pgfqpoint{1.755033in}{1.004389in}}%
\pgfpathlineto{\pgfqpoint{1.768088in}{0.995508in}}%
\pgfpathlineto{\pgfqpoint{1.780767in}{0.987042in}}%
\pgfpathlineto{\pgfqpoint{1.793072in}{0.978975in}}%
\pgfpathlineto{\pgfqpoint{1.805007in}{0.971288in}}%
\pgfpathlineto{\pgfqpoint{1.816573in}{0.963968in}}%
\pgfpathlineto{\pgfqpoint{1.827775in}{0.957000in}}%
\pgfpathlineto{\pgfqpoint{1.838615in}{0.950369in}}%
\pgfpathlineto{\pgfqpoint{1.849095in}{0.944064in}}%
\pgfpathlineto{\pgfqpoint{1.859219in}{0.938073in}}%
\pgfpathlineto{\pgfqpoint{1.868987in}{0.932384in}}%
\pgfpathlineto{\pgfqpoint{1.878404in}{0.926986in}}%
\pgfpathlineto{\pgfqpoint{1.887470in}{0.921870in}}%
\pgfpathlineto{\pgfqpoint{1.896187in}{0.917026in}}%
\pgfpathlineto{\pgfqpoint{1.904557in}{0.912446in}}%
\pgfpathlineto{\pgfqpoint{1.912581in}{0.908122in}}%
\pgfpathlineto{\pgfqpoint{1.920262in}{0.904044in}}%
\pgfpathlineto{\pgfqpoint{1.927599in}{0.900205in}}%
\pgfpathlineto{\pgfqpoint{1.934594in}{0.896599in}}%
\pgfusepath{stroke}%
\end{pgfscope}%
\begin{pgfscope}%
\pgfpathrectangle{\pgfqpoint{0.675193in}{0.526079in}}{\pgfqpoint{4.650000in}{3.020000in}} %
\pgfusepath{clip}%
\pgfsetrectcap%
\pgfsetroundjoin%
\pgfsetlinewidth{1.505625pt}%
\definecolor{currentstroke}{rgb}{1.000000,0.000000,0.000000}%
\pgfsetstrokecolor{currentstroke}%
\pgfsetstrokeopacity{0.750000}%
\pgfsetdash{}{0pt}%
\pgfpathmoveto{\pgfqpoint{0.954489in}{3.182972in}}%
\pgfpathlineto{\pgfqpoint{0.988306in}{3.206121in}}%
\pgfpathlineto{\pgfqpoint{1.021505in}{3.227816in}}%
\pgfpathlineto{\pgfqpoint{1.054091in}{3.245175in}}%
\pgfpathlineto{\pgfqpoint{1.086070in}{3.259318in}}%
\pgfpathlineto{\pgfqpoint{1.117448in}{3.271020in}}%
\pgfpathlineto{\pgfqpoint{1.148230in}{3.280835in}}%
\pgfpathlineto{\pgfqpoint{1.178420in}{3.289169in}}%
\pgfpathlineto{\pgfqpoint{1.208025in}{3.296323in}}%
\pgfpathlineto{\pgfqpoint{1.237049in}{3.302527in}}%
\pgfpathlineto{\pgfqpoint{1.265499in}{3.307954in}}%
\pgfpathlineto{\pgfqpoint{1.293379in}{3.312740in}}%
\pgfpathlineto{\pgfqpoint{1.320695in}{3.316992in}}%
\pgfpathlineto{\pgfqpoint{1.347455in}{3.320798in}}%
\pgfpathlineto{\pgfqpoint{1.373663in}{3.324225in}}%
\pgfpathlineto{\pgfqpoint{1.399324in}{3.327329in}}%
\pgfpathlineto{\pgfqpoint{1.424442in}{3.330157in}}%
\pgfpathlineto{\pgfqpoint{1.449023in}{3.332747in}}%
\pgfpathlineto{\pgfqpoint{1.473071in}{3.335129in}}%
\pgfpathlineto{\pgfqpoint{1.496589in}{3.337331in}}%
\pgfpathlineto{\pgfqpoint{1.519582in}{3.339375in}}%
\pgfpathlineto{\pgfqpoint{1.542053in}{3.341279in}}%
\pgfpathlineto{\pgfqpoint{1.564006in}{3.343060in}}%
\pgfpathlineto{\pgfqpoint{1.585446in}{3.344732in}}%
\pgfpathlineto{\pgfqpoint{1.606377in}{3.346307in}}%
\pgfpathlineto{\pgfqpoint{1.626802in}{3.347795in}}%
\pgfpathlineto{\pgfqpoint{1.646726in}{3.349205in}}%
\pgfpathlineto{\pgfqpoint{1.666156in}{3.350545in}}%
\pgfpathlineto{\pgfqpoint{1.685096in}{3.351822in}}%
\pgfpathlineto{\pgfqpoint{1.703553in}{3.353040in}}%
\pgfpathlineto{\pgfqpoint{1.721531in}{3.354206in}}%
\pgfpathlineto{\pgfqpoint{1.739037in}{3.355324in}}%
\pgfpathlineto{\pgfqpoint{1.756077in}{3.356399in}}%
\pgfpathlineto{\pgfqpoint{1.772657in}{3.357434in}}%
\pgfpathlineto{\pgfqpoint{1.788784in}{3.358432in}}%
\pgfpathlineto{\pgfqpoint{1.804462in}{3.359396in}}%
\pgfpathlineto{\pgfqpoint{1.819697in}{3.360329in}}%
\pgfpathlineto{\pgfqpoint{1.834494in}{3.361232in}}%
\pgfpathlineto{\pgfqpoint{1.848858in}{3.362109in}}%
\pgfpathlineto{\pgfqpoint{1.862793in}{3.362960in}}%
\pgfpathlineto{\pgfqpoint{1.876303in}{3.363787in}}%
\pgfpathlineto{\pgfqpoint{1.889391in}{3.364592in}}%
\pgfpathlineto{\pgfqpoint{1.902061in}{3.365376in}}%
\pgfpathlineto{\pgfqpoint{1.914315in}{3.366140in}}%
\pgfpathlineto{\pgfqpoint{1.926156in}{3.366885in}}%
\pgfpathlineto{\pgfqpoint{1.937586in}{3.367613in}}%
\pgfpathlineto{\pgfqpoint{1.948607in}{3.368323in}}%
\pgfpathlineto{\pgfqpoint{1.959221in}{3.369016in}}%
\pgfpathlineto{\pgfqpoint{1.969429in}{3.369694in}}%
\pgfpathlineto{\pgfqpoint{1.979233in}{3.370357in}}%
\pgfpathlineto{\pgfqpoint{1.988636in}{3.371004in}}%
\pgfpathlineto{\pgfqpoint{1.997640in}{3.371638in}}%
\pgfpathlineto{\pgfqpoint{2.006245in}{3.372257in}}%
\pgfpathlineto{\pgfqpoint{2.014455in}{3.372864in}}%
\pgfusepath{stroke}%
\end{pgfscope}%
\begin{pgfscope}%
\pgfpathrectangle{\pgfqpoint{0.675193in}{0.526079in}}{\pgfqpoint{4.650000in}{3.020000in}} %
\pgfusepath{clip}%
\pgfsetrectcap%
\pgfsetroundjoin%
\pgfsetlinewidth{1.505625pt}%
\definecolor{currentstroke}{rgb}{0.000000,0.000000,1.000000}%
\pgfsetstrokecolor{currentstroke}%
\pgfsetstrokeopacity{0.750000}%
\pgfsetdash{}{0pt}%
\pgfpathmoveto{\pgfqpoint{0.954489in}{1.329763in}}%
\pgfpathlineto{\pgfqpoint{0.988306in}{1.342409in}}%
\pgfpathlineto{\pgfqpoint{1.021505in}{1.354103in}}%
\pgfpathlineto{\pgfqpoint{1.054091in}{1.361798in}}%
\pgfpathlineto{\pgfqpoint{1.086070in}{1.366498in}}%
\pgfpathlineto{\pgfqpoint{1.117448in}{1.368891in}}%
\pgfpathlineto{\pgfqpoint{1.148230in}{1.369465in}}%
\pgfpathlineto{\pgfqpoint{1.178420in}{1.368572in}}%
\pgfpathlineto{\pgfqpoint{1.208025in}{1.366474in}}%
\pgfpathlineto{\pgfqpoint{1.237049in}{1.363365in}}%
\pgfpathlineto{\pgfqpoint{1.265499in}{1.359389in}}%
\pgfpathlineto{\pgfqpoint{1.293379in}{1.354657in}}%
\pgfpathlineto{\pgfqpoint{1.320695in}{1.349256in}}%
\pgfpathlineto{\pgfqpoint{1.347455in}{1.343252in}}%
\pgfpathlineto{\pgfqpoint{1.373663in}{1.336696in}}%
\pgfpathlineto{\pgfqpoint{1.399324in}{1.329628in}}%
\pgfpathlineto{\pgfqpoint{1.424442in}{1.322080in}}%
\pgfpathlineto{\pgfqpoint{1.449023in}{1.314076in}}%
\pgfpathlineto{\pgfqpoint{1.473071in}{1.305632in}}%
\pgfpathlineto{\pgfqpoint{1.496589in}{1.296763in}}%
\pgfpathlineto{\pgfqpoint{1.519582in}{1.287479in}}%
\pgfpathlineto{\pgfqpoint{1.542053in}{1.277786in}}%
\pgfpathlineto{\pgfqpoint{1.564006in}{1.267688in}}%
\pgfpathlineto{\pgfqpoint{1.585446in}{1.257186in}}%
\pgfpathlineto{\pgfqpoint{1.606377in}{1.246280in}}%
\pgfpathlineto{\pgfqpoint{1.626802in}{1.234967in}}%
\pgfpathlineto{\pgfqpoint{1.646726in}{1.223243in}}%
\pgfpathlineto{\pgfqpoint{1.666156in}{1.211102in}}%
\pgfpathlineto{\pgfqpoint{1.685096in}{1.198538in}}%
\pgfpathlineto{\pgfqpoint{1.703553in}{1.185540in}}%
\pgfpathlineto{\pgfqpoint{1.721531in}{1.172100in}}%
\pgfpathlineto{\pgfqpoint{1.739037in}{1.158207in}}%
\pgfpathlineto{\pgfqpoint{1.756077in}{1.143846in}}%
\pgfpathlineto{\pgfqpoint{1.772657in}{1.129005in}}%
\pgfpathlineto{\pgfqpoint{1.788784in}{1.113668in}}%
\pgfpathlineto{\pgfqpoint{1.804462in}{1.097818in}}%
\pgfpathlineto{\pgfqpoint{1.819697in}{1.081435in}}%
\pgfpathlineto{\pgfqpoint{1.834494in}{1.064500in}}%
\pgfpathlineto{\pgfqpoint{1.848858in}{1.046989in}}%
\pgfpathlineto{\pgfqpoint{1.862793in}{1.028879in}}%
\pgfpathlineto{\pgfqpoint{1.876303in}{1.010143in}}%
\pgfpathlineto{\pgfqpoint{1.889391in}{0.990751in}}%
\pgfpathlineto{\pgfqpoint{1.902061in}{0.970671in}}%
\pgfpathlineto{\pgfqpoint{1.914315in}{0.949868in}}%
\pgfpathlineto{\pgfqpoint{1.926156in}{0.928305in}}%
\pgfpathlineto{\pgfqpoint{1.937586in}{0.905938in}}%
\pgfpathlineto{\pgfqpoint{1.948607in}{0.882719in}}%
\pgfpathlineto{\pgfqpoint{1.959221in}{0.858599in}}%
\pgfpathlineto{\pgfqpoint{1.969429in}{0.833519in}}%
\pgfpathlineto{\pgfqpoint{1.979233in}{0.807417in}}%
\pgfpathlineto{\pgfqpoint{1.988636in}{0.780220in}}%
\pgfpathlineto{\pgfqpoint{1.997640in}{0.751849in}}%
\pgfpathlineto{\pgfqpoint{2.006245in}{0.722215in}}%
\pgfpathlineto{\pgfqpoint{2.014455in}{0.691217in}}%
\pgfusepath{stroke}%
\end{pgfscope}%
\begin{pgfscope}%
\pgfpathrectangle{\pgfqpoint{0.675193in}{0.526079in}}{\pgfqpoint{4.650000in}{3.020000in}} %
\pgfusepath{clip}%
\pgfsetrectcap%
\pgfsetroundjoin%
\pgfsetlinewidth{1.505625pt}%
\definecolor{currentstroke}{rgb}{0.000000,0.750000,0.750000}%
\pgfsetstrokecolor{currentstroke}%
\pgfsetstrokeopacity{0.750000}%
\pgfsetdash{}{0pt}%
\pgfpathmoveto{\pgfqpoint{0.954489in}{2.583847in}}%
\pgfpathlineto{\pgfqpoint{0.988306in}{2.500413in}}%
\pgfpathlineto{\pgfqpoint{1.021505in}{2.427736in}}%
\pgfpathlineto{\pgfqpoint{1.054091in}{2.360725in}}%
\pgfpathlineto{\pgfqpoint{1.086070in}{2.298841in}}%
\pgfpathlineto{\pgfqpoint{1.117448in}{2.241579in}}%
\pgfpathlineto{\pgfqpoint{1.148230in}{2.188484in}}%
\pgfpathlineto{\pgfqpoint{1.178420in}{2.139149in}}%
\pgfpathlineto{\pgfqpoint{1.208025in}{2.093219in}}%
\pgfpathlineto{\pgfqpoint{1.237049in}{2.050375in}}%
\pgfpathlineto{\pgfqpoint{1.265499in}{2.010337in}}%
\pgfpathlineto{\pgfqpoint{1.293379in}{1.972855in}}%
\pgfpathlineto{\pgfqpoint{1.320695in}{1.937711in}}%
\pgfpathlineto{\pgfqpoint{1.347455in}{1.904709in}}%
\pgfpathlineto{\pgfqpoint{1.373663in}{1.873675in}}%
\pgfpathlineto{\pgfqpoint{1.399324in}{1.844452in}}%
\pgfpathlineto{\pgfqpoint{1.424442in}{1.816903in}}%
\pgfpathlineto{\pgfqpoint{1.449023in}{1.790902in}}%
\pgfpathlineto{\pgfqpoint{1.473071in}{1.766335in}}%
\pgfpathlineto{\pgfqpoint{1.496589in}{1.743101in}}%
\pgfpathlineto{\pgfqpoint{1.519582in}{1.721108in}}%
\pgfpathlineto{\pgfqpoint{1.542053in}{1.700272in}}%
\pgfpathlineto{\pgfqpoint{1.564006in}{1.680517in}}%
\pgfpathlineto{\pgfqpoint{1.585446in}{1.661774in}}%
\pgfpathlineto{\pgfqpoint{1.606377in}{1.643979in}}%
\pgfpathlineto{\pgfqpoint{1.626802in}{1.627075in}}%
\pgfpathlineto{\pgfqpoint{1.646726in}{1.611008in}}%
\pgfpathlineto{\pgfqpoint{1.666156in}{1.595729in}}%
\pgfpathlineto{\pgfqpoint{1.685096in}{1.581194in}}%
\pgfpathlineto{\pgfqpoint{1.703553in}{1.567360in}}%
\pgfpathlineto{\pgfqpoint{1.721531in}{1.554190in}}%
\pgfpathlineto{\pgfqpoint{1.739037in}{1.541648in}}%
\pgfpathlineto{\pgfqpoint{1.756077in}{1.529701in}}%
\pgfpathlineto{\pgfqpoint{1.772657in}{1.518319in}}%
\pgfpathlineto{\pgfqpoint{1.788784in}{1.507474in}}%
\pgfpathlineto{\pgfqpoint{1.804462in}{1.497140in}}%
\pgfpathlineto{\pgfqpoint{1.819697in}{1.487292in}}%
\pgfpathlineto{\pgfqpoint{1.834494in}{1.477907in}}%
\pgfpathlineto{\pgfqpoint{1.848858in}{1.468965in}}%
\pgfpathlineto{\pgfqpoint{1.862793in}{1.460445in}}%
\pgfpathlineto{\pgfqpoint{1.876303in}{1.452330in}}%
\pgfpathlineto{\pgfqpoint{1.889391in}{1.444602in}}%
\pgfpathlineto{\pgfqpoint{1.902061in}{1.437244in}}%
\pgfpathlineto{\pgfqpoint{1.914315in}{1.430242in}}%
\pgfpathlineto{\pgfqpoint{1.926156in}{1.423583in}}%
\pgfpathlineto{\pgfqpoint{1.937586in}{1.417252in}}%
\pgfpathlineto{\pgfqpoint{1.948607in}{1.411236in}}%
\pgfpathlineto{\pgfqpoint{1.959221in}{1.405525in}}%
\pgfpathlineto{\pgfqpoint{1.969429in}{1.400107in}}%
\pgfpathlineto{\pgfqpoint{1.979233in}{1.394974in}}%
\pgfpathlineto{\pgfqpoint{1.988636in}{1.390114in}}%
\pgfpathlineto{\pgfqpoint{1.997640in}{1.385518in}}%
\pgfpathlineto{\pgfqpoint{2.006245in}{1.381180in}}%
\pgfpathlineto{\pgfqpoint{2.014455in}{1.377090in}}%
\pgfusepath{stroke}%
\end{pgfscope}%
\begin{pgfscope}%
\pgfpathrectangle{\pgfqpoint{0.675193in}{0.526079in}}{\pgfqpoint{4.650000in}{3.020000in}} %
\pgfusepath{clip}%
\pgfsetrectcap%
\pgfsetroundjoin%
\pgfsetlinewidth{1.505625pt}%
\definecolor{currentstroke}{rgb}{1.000000,0.000000,0.000000}%
\pgfsetstrokecolor{currentstroke}%
\pgfsetstrokeopacity{0.750000}%
\pgfsetdash{}{0pt}%
\pgfpathmoveto{\pgfqpoint{1.132009in}{3.238889in}}%
\pgfpathlineto{\pgfqpoint{1.176445in}{3.254816in}}%
\pgfpathlineto{\pgfqpoint{1.219682in}{3.270004in}}%
\pgfpathlineto{\pgfqpoint{1.261735in}{3.282318in}}%
\pgfpathlineto{\pgfqpoint{1.302618in}{3.292468in}}%
\pgfpathlineto{\pgfqpoint{1.342347in}{3.300953in}}%
\pgfpathlineto{\pgfqpoint{1.380936in}{3.308134in}}%
\pgfpathlineto{\pgfqpoint{1.418401in}{3.314281in}}%
\pgfpathlineto{\pgfqpoint{1.454756in}{3.319594in}}%
\pgfpathlineto{\pgfqpoint{1.490016in}{3.324227in}}%
\pgfpathlineto{\pgfqpoint{1.524195in}{3.328301in}}%
\pgfpathlineto{\pgfqpoint{1.557309in}{3.331907in}}%
\pgfpathlineto{\pgfqpoint{1.589373in}{3.335122in}}%
\pgfpathlineto{\pgfqpoint{1.620401in}{3.338005in}}%
\pgfpathlineto{\pgfqpoint{1.650410in}{3.340604in}}%
\pgfpathlineto{\pgfqpoint{1.679414in}{3.342960in}}%
\pgfpathlineto{\pgfqpoint{1.707427in}{3.345104in}}%
\pgfpathlineto{\pgfqpoint{1.734466in}{3.347066in}}%
\pgfpathlineto{\pgfqpoint{1.760544in}{3.348866in}}%
\pgfpathlineto{\pgfqpoint{1.785678in}{3.350524in}}%
\pgfpathlineto{\pgfqpoint{1.809881in}{3.352056in}}%
\pgfpathlineto{\pgfqpoint{1.833167in}{3.353477in}}%
\pgfpathlineto{\pgfqpoint{1.855549in}{3.354798in}}%
\pgfpathlineto{\pgfqpoint{1.877041in}{3.356029in}}%
\pgfpathlineto{\pgfqpoint{1.897656in}{3.357179in}}%
\pgfpathlineto{\pgfqpoint{1.917406in}{3.358255in}}%
\pgfpathlineto{\pgfqpoint{1.936302in}{3.359264in}}%
\pgfpathlineto{\pgfqpoint{1.954357in}{3.360213in}}%
\pgfpathlineto{\pgfqpoint{1.971581in}{3.361104in}}%
\pgfpathlineto{\pgfqpoint{1.987984in}{3.361944in}}%
\pgfusepath{stroke}%
\end{pgfscope}%
\begin{pgfscope}%
\pgfpathrectangle{\pgfqpoint{0.675193in}{0.526079in}}{\pgfqpoint{4.650000in}{3.020000in}} %
\pgfusepath{clip}%
\pgfsetrectcap%
\pgfsetroundjoin%
\pgfsetlinewidth{1.505625pt}%
\definecolor{currentstroke}{rgb}{0.000000,0.000000,1.000000}%
\pgfsetstrokecolor{currentstroke}%
\pgfsetstrokeopacity{0.750000}%
\pgfsetdash{}{0pt}%
\pgfpathmoveto{\pgfqpoint{1.132009in}{0.786746in}}%
\pgfpathlineto{\pgfqpoint{1.176445in}{0.787267in}}%
\pgfpathlineto{\pgfqpoint{1.219682in}{0.787273in}}%
\pgfpathlineto{\pgfqpoint{1.261735in}{0.784472in}}%
\pgfpathlineto{\pgfqpoint{1.302618in}{0.779453in}}%
\pgfpathlineto{\pgfqpoint{1.342347in}{0.772622in}}%
\pgfpathlineto{\pgfqpoint{1.380936in}{0.764265in}}%
\pgfpathlineto{\pgfqpoint{1.418401in}{0.754584in}}%
\pgfpathlineto{\pgfqpoint{1.454756in}{0.743727in}}%
\pgfpathlineto{\pgfqpoint{1.490016in}{0.731796in}}%
\pgfpathlineto{\pgfqpoint{1.524195in}{0.718866in}}%
\pgfpathlineto{\pgfqpoint{1.557309in}{0.704988in}}%
\pgfpathlineto{\pgfqpoint{1.589373in}{0.690195in}}%
\pgfpathlineto{\pgfqpoint{1.620401in}{0.674508in}}%
\pgfpathlineto{\pgfqpoint{1.650410in}{0.657936in}}%
\pgfpathlineto{\pgfqpoint{1.679414in}{0.640478in}}%
\pgfpathlineto{\pgfqpoint{1.707427in}{0.622127in}}%
\pgfpathlineto{\pgfqpoint{1.734466in}{0.602867in}}%
\pgfpathlineto{\pgfqpoint{1.760544in}{0.582676in}}%
\pgfpathlineto{\pgfqpoint{1.785678in}{0.561527in}}%
\pgfpathlineto{\pgfqpoint{1.809881in}{0.539385in}}%
\pgfpathlineto{\pgfqpoint{1.833167in}{0.516210in}}%
\pgfpathlineto{\pgfqpoint{1.833287in}{0.516079in}}%
\pgfusepath{stroke}%
\end{pgfscope}%
\begin{pgfscope}%
\pgfpathrectangle{\pgfqpoint{0.675193in}{0.526079in}}{\pgfqpoint{4.650000in}{3.020000in}} %
\pgfusepath{clip}%
\pgfsetrectcap%
\pgfsetroundjoin%
\pgfsetlinewidth{1.505625pt}%
\definecolor{currentstroke}{rgb}{0.000000,0.750000,0.750000}%
\pgfsetstrokecolor{currentstroke}%
\pgfsetstrokeopacity{0.750000}%
\pgfsetdash{}{0pt}%
\pgfpathmoveto{\pgfqpoint{1.132009in}{2.467053in}}%
\pgfpathlineto{\pgfqpoint{1.176445in}{2.394820in}}%
\pgfpathlineto{\pgfqpoint{1.219682in}{2.332055in}}%
\pgfpathlineto{\pgfqpoint{1.261735in}{2.274873in}}%
\pgfpathlineto{\pgfqpoint{1.302618in}{2.222638in}}%
\pgfpathlineto{\pgfqpoint{1.342347in}{2.174796in}}%
\pgfpathlineto{\pgfqpoint{1.380936in}{2.130866in}}%
\pgfpathlineto{\pgfqpoint{1.418401in}{2.090433in}}%
\pgfpathlineto{\pgfqpoint{1.454756in}{2.053138in}}%
\pgfpathlineto{\pgfqpoint{1.490016in}{2.018667in}}%
\pgfpathlineto{\pgfqpoint{1.524195in}{1.986748in}}%
\pgfpathlineto{\pgfqpoint{1.557309in}{1.957140in}}%
\pgfpathlineto{\pgfqpoint{1.589373in}{1.929634in}}%
\pgfpathlineto{\pgfqpoint{1.620401in}{1.904046in}}%
\pgfpathlineto{\pgfqpoint{1.650410in}{1.880211in}}%
\pgfpathlineto{\pgfqpoint{1.679414in}{1.857984in}}%
\pgfpathlineto{\pgfqpoint{1.707427in}{1.837236in}}%
\pgfpathlineto{\pgfqpoint{1.734466in}{1.817853in}}%
\pgfpathlineto{\pgfqpoint{1.760544in}{1.799730in}}%
\pgfpathlineto{\pgfqpoint{1.785678in}{1.782775in}}%
\pgfpathlineto{\pgfqpoint{1.809881in}{1.766905in}}%
\pgfpathlineto{\pgfqpoint{1.833167in}{1.752046in}}%
\pgfpathlineto{\pgfqpoint{1.855549in}{1.738129in}}%
\pgfpathlineto{\pgfqpoint{1.877041in}{1.725094in}}%
\pgfpathlineto{\pgfqpoint{1.897656in}{1.712883in}}%
\pgfpathlineto{\pgfqpoint{1.917406in}{1.701449in}}%
\pgfpathlineto{\pgfqpoint{1.936302in}{1.690744in}}%
\pgfpathlineto{\pgfqpoint{1.954357in}{1.680726in}}%
\pgfpathlineto{\pgfqpoint{1.971581in}{1.671359in}}%
\pgfpathlineto{\pgfqpoint{1.987984in}{1.662606in}}%
\pgfusepath{stroke}%
\end{pgfscope}%
\begin{pgfscope}%
\pgfpathrectangle{\pgfqpoint{0.675193in}{0.526079in}}{\pgfqpoint{4.650000in}{3.020000in}} %
\pgfusepath{clip}%
\pgfsetrectcap%
\pgfsetroundjoin%
\pgfsetlinewidth{1.505625pt}%
\definecolor{currentstroke}{rgb}{1.000000,0.000000,0.000000}%
\pgfsetstrokecolor{currentstroke}%
\pgfsetstrokeopacity{0.750000}%
\pgfsetdash{}{0pt}%
\pgfpathmoveto{\pgfqpoint{1.223532in}{2.745340in}}%
\pgfpathlineto{\pgfqpoint{1.267962in}{2.785727in}}%
\pgfpathlineto{\pgfqpoint{1.311046in}{2.825820in}}%
\pgfpathlineto{\pgfqpoint{1.352819in}{2.860299in}}%
\pgfpathlineto{\pgfqpoint{1.393311in}{2.890229in}}%
\pgfpathlineto{\pgfqpoint{1.432557in}{2.916422in}}%
\pgfpathlineto{\pgfqpoint{1.470589in}{2.939509in}}%
\pgfpathlineto{\pgfqpoint{1.507439in}{2.959991in}}%
\pgfpathlineto{\pgfqpoint{1.543140in}{2.978265in}}%
\pgfpathlineto{\pgfqpoint{1.577725in}{2.994654in}}%
\pgfpathlineto{\pgfqpoint{1.611227in}{3.009421in}}%
\pgfpathlineto{\pgfqpoint{1.643678in}{3.022784in}}%
\pgfpathlineto{\pgfqpoint{1.675112in}{3.034923in}}%
\pgfpathlineto{\pgfqpoint{1.705577in}{3.045991in}}%
\pgfpathlineto{\pgfqpoint{1.735101in}{3.056113in}}%
\pgfpathlineto{\pgfqpoint{1.763713in}{3.065400in}}%
\pgfpathlineto{\pgfqpoint{1.791444in}{3.073944in}}%
\pgfpathlineto{\pgfqpoint{1.818323in}{3.081825in}}%
\pgfpathlineto{\pgfqpoint{1.844379in}{3.089111in}}%
\pgfpathlineto{\pgfqpoint{1.869636in}{3.095863in}}%
\pgfpathlineto{\pgfqpoint{1.894120in}{3.102132in}}%
\pgfpathlineto{\pgfqpoint{1.917850in}{3.107965in}}%
\pgfpathlineto{\pgfqpoint{1.940850in}{3.113400in}}%
\pgfpathlineto{\pgfqpoint{1.963137in}{3.118472in}}%
\pgfpathlineto{\pgfqpoint{1.984729in}{3.123214in}}%
\pgfpathlineto{\pgfqpoint{2.005641in}{3.127653in}}%
\pgfpathlineto{\pgfqpoint{2.025885in}{3.131813in}}%
\pgfpathlineto{\pgfqpoint{2.045473in}{3.135716in}}%
\pgfpathlineto{\pgfqpoint{2.064418in}{3.139382in}}%
\pgfpathlineto{\pgfqpoint{2.082730in}{3.142829in}}%
\pgfpathlineto{\pgfqpoint{2.100420in}{3.146071in}}%
\pgfpathlineto{\pgfqpoint{2.117499in}{3.149125in}}%
\pgfpathlineto{\pgfqpoint{2.133975in}{3.152001in}}%
\pgfpathlineto{\pgfqpoint{2.149860in}{3.154711in}}%
\pgfpathlineto{\pgfqpoint{2.165164in}{3.157267in}}%
\pgfpathlineto{\pgfqpoint{2.179897in}{3.159677in}}%
\pgfpathlineto{\pgfqpoint{2.194068in}{3.161950in}}%
\pgfpathlineto{\pgfqpoint{2.207685in}{3.164094in}}%
\pgfpathlineto{\pgfqpoint{2.220755in}{3.166116in}}%
\pgfusepath{stroke}%
\end{pgfscope}%
\begin{pgfscope}%
\pgfpathrectangle{\pgfqpoint{0.675193in}{0.526079in}}{\pgfqpoint{4.650000in}{3.020000in}} %
\pgfusepath{clip}%
\pgfsetrectcap%
\pgfsetroundjoin%
\pgfsetlinewidth{1.505625pt}%
\definecolor{currentstroke}{rgb}{0.000000,0.000000,1.000000}%
\pgfsetstrokecolor{currentstroke}%
\pgfsetstrokeopacity{0.750000}%
\pgfsetdash{}{0pt}%
\pgfpathmoveto{\pgfqpoint{1.307406in}{0.516079in}}%
\pgfpathlineto{\pgfqpoint{1.311046in}{0.517760in}}%
\pgfpathlineto{\pgfqpoint{1.352819in}{0.533176in}}%
\pgfpathlineto{\pgfqpoint{1.393311in}{0.544906in}}%
\pgfpathlineto{\pgfqpoint{1.432557in}{0.553551in}}%
\pgfpathlineto{\pgfqpoint{1.470589in}{0.559579in}}%
\pgfpathlineto{\pgfqpoint{1.507439in}{0.563356in}}%
\pgfpathlineto{\pgfqpoint{1.543140in}{0.565171in}}%
\pgfpathlineto{\pgfqpoint{1.577725in}{0.565254in}}%
\pgfpathlineto{\pgfqpoint{1.611227in}{0.563793in}}%
\pgfpathlineto{\pgfqpoint{1.643678in}{0.560939in}}%
\pgfpathlineto{\pgfqpoint{1.675112in}{0.556814in}}%
\pgfpathlineto{\pgfqpoint{1.705577in}{0.551520in}}%
\pgfpathlineto{\pgfqpoint{1.735101in}{0.545137in}}%
\pgfpathlineto{\pgfqpoint{1.763713in}{0.537733in}}%
\pgfpathlineto{\pgfqpoint{1.791444in}{0.529361in}}%
\pgfpathlineto{\pgfqpoint{1.818323in}{0.520067in}}%
\pgfpathlineto{\pgfqpoint{1.828526in}{0.516079in}}%
\pgfusepath{stroke}%
\end{pgfscope}%
\begin{pgfscope}%
\pgfpathrectangle{\pgfqpoint{0.675193in}{0.526079in}}{\pgfqpoint{4.650000in}{3.020000in}} %
\pgfusepath{clip}%
\pgfsetrectcap%
\pgfsetroundjoin%
\pgfsetlinewidth{1.505625pt}%
\definecolor{currentstroke}{rgb}{0.000000,0.750000,0.750000}%
\pgfsetstrokecolor{currentstroke}%
\pgfsetstrokeopacity{0.750000}%
\pgfsetdash{}{0pt}%
\pgfpathmoveto{\pgfqpoint{1.223532in}{3.136623in}}%
\pgfpathlineto{\pgfqpoint{1.267962in}{3.103672in}}%
\pgfpathlineto{\pgfqpoint{1.311046in}{3.078634in}}%
\pgfpathlineto{\pgfqpoint{1.352819in}{3.054786in}}%
\pgfpathlineto{\pgfqpoint{1.393311in}{3.032119in}}%
\pgfpathlineto{\pgfqpoint{1.432557in}{3.010599in}}%
\pgfpathlineto{\pgfqpoint{1.470589in}{2.990185in}}%
\pgfpathlineto{\pgfqpoint{1.507439in}{2.970830in}}%
\pgfpathlineto{\pgfqpoint{1.543140in}{2.952482in}}%
\pgfpathlineto{\pgfqpoint{1.577725in}{2.935089in}}%
\pgfpathlineto{\pgfqpoint{1.611227in}{2.918599in}}%
\pgfpathlineto{\pgfqpoint{1.643678in}{2.902965in}}%
\pgfpathlineto{\pgfqpoint{1.675112in}{2.888137in}}%
\pgfpathlineto{\pgfqpoint{1.705577in}{2.874071in}}%
\pgfpathlineto{\pgfqpoint{1.735101in}{2.860723in}}%
\pgfpathlineto{\pgfqpoint{1.763713in}{2.848053in}}%
\pgfpathlineto{\pgfqpoint{1.791444in}{2.836023in}}%
\pgfpathlineto{\pgfqpoint{1.818323in}{2.824598in}}%
\pgfpathlineto{\pgfqpoint{1.844379in}{2.813744in}}%
\pgfpathlineto{\pgfqpoint{1.869636in}{2.803431in}}%
\pgfpathlineto{\pgfqpoint{1.894120in}{2.793629in}}%
\pgfpathlineto{\pgfqpoint{1.917850in}{2.784311in}}%
\pgfpathlineto{\pgfqpoint{1.940850in}{2.775452in}}%
\pgfpathlineto{\pgfqpoint{1.963137in}{2.767028in}}%
\pgfpathlineto{\pgfqpoint{1.984729in}{2.759018in}}%
\pgfpathlineto{\pgfqpoint{2.005641in}{2.751401in}}%
\pgfpathlineto{\pgfqpoint{2.025885in}{2.744157in}}%
\pgfpathlineto{\pgfqpoint{2.045473in}{2.737269in}}%
\pgfpathlineto{\pgfqpoint{2.064418in}{2.730720in}}%
\pgfpathlineto{\pgfqpoint{2.082730in}{2.724495in}}%
\pgfpathlineto{\pgfqpoint{2.100420in}{2.718579in}}%
\pgfpathlineto{\pgfqpoint{2.117499in}{2.712959in}}%
\pgfpathlineto{\pgfqpoint{2.133975in}{2.707621in}}%
\pgfpathlineto{\pgfqpoint{2.149860in}{2.702554in}}%
\pgfpathlineto{\pgfqpoint{2.165164in}{2.697746in}}%
\pgfpathlineto{\pgfqpoint{2.179897in}{2.693188in}}%
\pgfpathlineto{\pgfqpoint{2.194068in}{2.688869in}}%
\pgfpathlineto{\pgfqpoint{2.207685in}{2.684781in}}%
\pgfpathlineto{\pgfqpoint{2.220755in}{2.680915in}}%
\pgfusepath{stroke}%
\end{pgfscope}%
\begin{pgfscope}%
\pgfpathrectangle{\pgfqpoint{0.675193in}{0.526079in}}{\pgfqpoint{4.650000in}{3.020000in}} %
\pgfusepath{clip}%
\pgfsetrectcap%
\pgfsetroundjoin%
\pgfsetlinewidth{1.505625pt}%
\definecolor{currentstroke}{rgb}{1.000000,0.000000,0.000000}%
\pgfsetstrokecolor{currentstroke}%
\pgfsetstrokeopacity{0.750000}%
\pgfsetdash{}{0pt}%
\pgfpathmoveto{\pgfqpoint{1.067481in}{2.990292in}}%
\pgfpathlineto{\pgfqpoint{1.110135in}{3.027644in}}%
\pgfpathlineto{\pgfqpoint{1.152001in}{3.063799in}}%
\pgfpathlineto{\pgfqpoint{1.193092in}{3.093700in}}%
\pgfpathlineto{\pgfqpoint{1.233421in}{3.118747in}}%
\pgfpathlineto{\pgfqpoint{1.273001in}{3.139961in}}%
\pgfpathlineto{\pgfqpoint{1.311843in}{3.158107in}}%
\pgfpathlineto{\pgfqpoint{1.349961in}{3.173768in}}%
\pgfpathlineto{\pgfqpoint{1.387368in}{3.187392in}}%
\pgfpathlineto{\pgfqpoint{1.424075in}{3.199331in}}%
\pgfpathlineto{\pgfqpoint{1.460095in}{3.209860in}}%
\pgfpathlineto{\pgfqpoint{1.495442in}{3.219202in}}%
\pgfpathlineto{\pgfqpoint{1.530127in}{3.227538in}}%
\pgfpathlineto{\pgfqpoint{1.564159in}{3.235013in}}%
\pgfpathlineto{\pgfqpoint{1.597553in}{3.241748in}}%
\pgfpathlineto{\pgfqpoint{1.630322in}{3.247843in}}%
\pgfpathlineto{\pgfqpoint{1.662479in}{3.253380in}}%
\pgfpathlineto{\pgfqpoint{1.694039in}{3.258431in}}%
\pgfpathlineto{\pgfqpoint{1.725017in}{3.263054in}}%
\pgfpathlineto{\pgfqpoint{1.755427in}{3.267300in}}%
\pgfpathlineto{\pgfqpoint{1.785284in}{3.271211in}}%
\pgfpathlineto{\pgfqpoint{1.814604in}{3.274826in}}%
\pgfpathlineto{\pgfqpoint{1.843400in}{3.278175in}}%
\pgfpathlineto{\pgfqpoint{1.871689in}{3.281288in}}%
\pgfpathlineto{\pgfqpoint{1.899483in}{3.284187in}}%
\pgfpathlineto{\pgfqpoint{1.926797in}{3.286894in}}%
\pgfpathlineto{\pgfqpoint{1.953643in}{3.289427in}}%
\pgfpathlineto{\pgfqpoint{1.980032in}{3.291803in}}%
\pgfpathlineto{\pgfqpoint{2.005975in}{3.294037in}}%
\pgfpathlineto{\pgfqpoint{2.031482in}{3.296140in}}%
\pgfpathlineto{\pgfqpoint{2.056561in}{3.298125in}}%
\pgfpathlineto{\pgfqpoint{2.081220in}{3.300002in}}%
\pgfpathlineto{\pgfqpoint{2.105467in}{3.301779in}}%
\pgfpathlineto{\pgfqpoint{2.129309in}{3.303465in}}%
\pgfpathlineto{\pgfqpoint{2.152752in}{3.305067in}}%
\pgfpathlineto{\pgfqpoint{2.175801in}{3.306592in}}%
\pgfpathlineto{\pgfqpoint{2.198463in}{3.308046in}}%
\pgfpathlineto{\pgfqpoint{2.220743in}{3.309434in}}%
\pgfpathlineto{\pgfqpoint{2.242646in}{3.310759in}}%
\pgfpathlineto{\pgfqpoint{2.264177in}{3.312029in}}%
\pgfpathlineto{\pgfqpoint{2.285342in}{3.313246in}}%
\pgfpathlineto{\pgfqpoint{2.306146in}{3.314414in}}%
\pgfpathlineto{\pgfqpoint{2.326594in}{3.315536in}}%
\pgfpathlineto{\pgfqpoint{2.346691in}{3.316615in}}%
\pgfpathlineto{\pgfqpoint{2.366442in}{3.317655in}}%
\pgfpathlineto{\pgfqpoint{2.385853in}{3.318657in}}%
\pgfpathlineto{\pgfqpoint{2.404929in}{3.319624in}}%
\pgfpathlineto{\pgfqpoint{2.423673in}{3.320559in}}%
\pgfpathlineto{\pgfqpoint{2.442092in}{3.321463in}}%
\pgfpathlineto{\pgfqpoint{2.460188in}{3.322338in}}%
\pgfpathlineto{\pgfqpoint{2.477967in}{3.323185in}}%
\pgfpathlineto{\pgfqpoint{2.495432in}{3.324007in}}%
\pgfpathlineto{\pgfqpoint{2.512587in}{3.324805in}}%
\pgfpathlineto{\pgfqpoint{2.529435in}{3.325580in}}%
\pgfpathlineto{\pgfqpoint{2.545981in}{3.326333in}}%
\pgfpathlineto{\pgfqpoint{2.562226in}{3.327066in}}%
\pgfpathlineto{\pgfqpoint{2.578173in}{3.327778in}}%
\pgfpathlineto{\pgfqpoint{2.593827in}{3.328472in}}%
\pgfpathlineto{\pgfqpoint{2.609189in}{3.329149in}}%
\pgfpathlineto{\pgfqpoint{2.624262in}{3.329810in}}%
\pgfpathlineto{\pgfqpoint{2.639048in}{3.330454in}}%
\pgfpathlineto{\pgfqpoint{2.653551in}{3.331082in}}%
\pgfpathlineto{\pgfqpoint{2.667773in}{3.331697in}}%
\pgfpathlineto{\pgfqpoint{2.681715in}{3.332297in}}%
\pgfpathlineto{\pgfqpoint{2.695381in}{3.332884in}}%
\pgfpathlineto{\pgfqpoint{2.708772in}{3.333458in}}%
\pgfpathlineto{\pgfqpoint{2.721892in}{3.334020in}}%
\pgfpathlineto{\pgfqpoint{2.734743in}{3.334571in}}%
\pgfpathlineto{\pgfqpoint{2.747326in}{3.335110in}}%
\pgfpathlineto{\pgfqpoint{2.759646in}{3.335638in}}%
\pgfpathlineto{\pgfqpoint{2.771703in}{3.336155in}}%
\pgfpathlineto{\pgfqpoint{2.783500in}{3.336663in}}%
\pgfpathlineto{\pgfqpoint{2.795040in}{3.337161in}}%
\pgfpathlineto{\pgfqpoint{2.806324in}{3.337648in}}%
\pgfpathlineto{\pgfqpoint{2.817355in}{3.338127in}}%
\pgfpathlineto{\pgfqpoint{2.828134in}{3.338597in}}%
\pgfpathlineto{\pgfqpoint{2.838665in}{3.339058in}}%
\pgfpathlineto{\pgfqpoint{2.848947in}{3.339511in}}%
\pgfpathlineto{\pgfqpoint{2.858984in}{3.339956in}}%
\pgfpathlineto{\pgfqpoint{2.868775in}{3.340392in}}%
\pgfpathlineto{\pgfqpoint{2.878324in}{3.340821in}}%
\pgfpathlineto{\pgfqpoint{2.887631in}{3.341242in}}%
\pgfpathlineto{\pgfqpoint{2.896696in}{3.341656in}}%
\pgfpathlineto{\pgfqpoint{2.905523in}{3.342061in}}%
\pgfpathlineto{\pgfqpoint{2.914111in}{3.342459in}}%
\pgfpathlineto{\pgfqpoint{2.922462in}{3.342851in}}%
\pgfpathlineto{\pgfqpoint{2.930576in}{3.343236in}}%
\pgfpathlineto{\pgfqpoint{2.938455in}{3.343613in}}%
\pgfpathlineto{\pgfqpoint{2.946100in}{3.343984in}}%
\pgfpathlineto{\pgfqpoint{2.953512in}{3.344347in}}%
\pgfpathlineto{\pgfqpoint{2.960692in}{3.344705in}}%
\pgfpathlineto{\pgfqpoint{2.967640in}{3.345055in}}%
\pgfpathlineto{\pgfqpoint{2.974359in}{3.345400in}}%
\pgfpathlineto{\pgfqpoint{2.980849in}{3.345737in}}%
\pgfpathlineto{\pgfqpoint{2.987110in}{3.346068in}}%
\pgfpathlineto{\pgfqpoint{2.993145in}{3.346392in}}%
\pgfpathlineto{\pgfqpoint{2.998953in}{3.346710in}}%
\pgfpathlineto{\pgfqpoint{3.004537in}{3.347022in}}%
\pgfpathlineto{\pgfqpoint{3.009897in}{3.347328in}}%
\pgfpathlineto{\pgfqpoint{3.015034in}{3.347627in}}%
\pgfpathlineto{\pgfqpoint{3.019948in}{3.347919in}}%
\pgfpathlineto{\pgfqpoint{3.024641in}{3.348205in}}%
\pgfusepath{stroke}%
\end{pgfscope}%
\begin{pgfscope}%
\pgfpathrectangle{\pgfqpoint{0.675193in}{0.526079in}}{\pgfqpoint{4.650000in}{3.020000in}} %
\pgfusepath{clip}%
\pgfsetrectcap%
\pgfsetroundjoin%
\pgfsetlinewidth{1.505625pt}%
\definecolor{currentstroke}{rgb}{0.000000,0.000000,1.000000}%
\pgfsetstrokecolor{currentstroke}%
\pgfsetstrokeopacity{0.750000}%
\pgfsetdash{}{0pt}%
\pgfpathmoveto{\pgfqpoint{1.067481in}{1.008681in}}%
\pgfpathlineto{\pgfqpoint{1.110135in}{1.036801in}}%
\pgfpathlineto{\pgfqpoint{1.152001in}{1.064836in}}%
\pgfpathlineto{\pgfqpoint{1.193092in}{1.087459in}}%
\pgfpathlineto{\pgfqpoint{1.233421in}{1.105877in}}%
\pgfpathlineto{\pgfqpoint{1.273001in}{1.120968in}}%
\pgfpathlineto{\pgfqpoint{1.311843in}{1.133386in}}%
\pgfpathlineto{\pgfqpoint{1.349961in}{1.143632in}}%
\pgfpathlineto{\pgfqpoint{1.387368in}{1.152086in}}%
\pgfpathlineto{\pgfqpoint{1.424075in}{1.159043in}}%
\pgfpathlineto{\pgfqpoint{1.460095in}{1.164736in}}%
\pgfpathlineto{\pgfqpoint{1.495442in}{1.169352in}}%
\pgfpathlineto{\pgfqpoint{1.530127in}{1.173039in}}%
\pgfpathlineto{\pgfqpoint{1.564159in}{1.175918in}}%
\pgfpathlineto{\pgfqpoint{1.597553in}{1.178087in}}%
\pgfpathlineto{\pgfqpoint{1.630322in}{1.179625in}}%
\pgfpathlineto{\pgfqpoint{1.662479in}{1.180600in}}%
\pgfpathlineto{\pgfqpoint{1.694039in}{1.181068in}}%
\pgfpathlineto{\pgfqpoint{1.725017in}{1.181075in}}%
\pgfpathlineto{\pgfqpoint{1.755427in}{1.180659in}}%
\pgfpathlineto{\pgfqpoint{1.785284in}{1.179854in}}%
\pgfpathlineto{\pgfqpoint{1.814604in}{1.178688in}}%
\pgfpathlineto{\pgfqpoint{1.843400in}{1.177184in}}%
\pgfpathlineto{\pgfqpoint{1.871689in}{1.175363in}}%
\pgfpathlineto{\pgfqpoint{1.899483in}{1.173241in}}%
\pgfpathlineto{\pgfqpoint{1.926797in}{1.170834in}}%
\pgfpathlineto{\pgfqpoint{1.953643in}{1.168154in}}%
\pgfpathlineto{\pgfqpoint{1.980032in}{1.165212in}}%
\pgfpathlineto{\pgfqpoint{2.005975in}{1.162018in}}%
\pgfpathlineto{\pgfqpoint{2.031482in}{1.158579in}}%
\pgfpathlineto{\pgfqpoint{2.056561in}{1.154903in}}%
\pgfpathlineto{\pgfqpoint{2.081220in}{1.150996in}}%
\pgfpathlineto{\pgfqpoint{2.105467in}{1.146862in}}%
\pgfpathlineto{\pgfqpoint{2.129309in}{1.142506in}}%
\pgfpathlineto{\pgfqpoint{2.152752in}{1.137931in}}%
\pgfpathlineto{\pgfqpoint{2.175801in}{1.133140in}}%
\pgfpathlineto{\pgfqpoint{2.198463in}{1.128136in}}%
\pgfpathlineto{\pgfqpoint{2.220743in}{1.122920in}}%
\pgfpathlineto{\pgfqpoint{2.242646in}{1.117493in}}%
\pgfpathlineto{\pgfqpoint{2.264177in}{1.111857in}}%
\pgfpathlineto{\pgfqpoint{2.285342in}{1.106012in}}%
\pgfpathlineto{\pgfqpoint{2.306146in}{1.099959in}}%
\pgfpathlineto{\pgfqpoint{2.326594in}{1.093697in}}%
\pgfpathlineto{\pgfqpoint{2.346691in}{1.087226in}}%
\pgfpathlineto{\pgfqpoint{2.366442in}{1.080546in}}%
\pgfpathlineto{\pgfqpoint{2.385853in}{1.073655in}}%
\pgfpathlineto{\pgfqpoint{2.404929in}{1.066551in}}%
\pgfpathlineto{\pgfqpoint{2.423673in}{1.059234in}}%
\pgfpathlineto{\pgfqpoint{2.442092in}{1.051701in}}%
\pgfpathlineto{\pgfqpoint{2.460188in}{1.043951in}}%
\pgfpathlineto{\pgfqpoint{2.477967in}{1.035981in}}%
\pgfpathlineto{\pgfqpoint{2.495432in}{1.027788in}}%
\pgfpathlineto{\pgfqpoint{2.512587in}{1.019369in}}%
\pgfpathlineto{\pgfqpoint{2.529435in}{1.010722in}}%
\pgfpathlineto{\pgfqpoint{2.545981in}{1.001844in}}%
\pgfpathlineto{\pgfqpoint{2.562226in}{0.992729in}}%
\pgfpathlineto{\pgfqpoint{2.578173in}{0.983373in}}%
\pgfpathlineto{\pgfqpoint{2.593827in}{0.973773in}}%
\pgfpathlineto{\pgfqpoint{2.609189in}{0.963925in}}%
\pgfpathlineto{\pgfqpoint{2.624262in}{0.953823in}}%
\pgfpathlineto{\pgfqpoint{2.639048in}{0.943461in}}%
\pgfpathlineto{\pgfqpoint{2.653551in}{0.932834in}}%
\pgfpathlineto{\pgfqpoint{2.667773in}{0.921937in}}%
\pgfpathlineto{\pgfqpoint{2.681715in}{0.910763in}}%
\pgfpathlineto{\pgfqpoint{2.695381in}{0.899304in}}%
\pgfpathlineto{\pgfqpoint{2.708772in}{0.887554in}}%
\pgfpathlineto{\pgfqpoint{2.721892in}{0.875505in}}%
\pgfpathlineto{\pgfqpoint{2.734743in}{0.863149in}}%
\pgfpathlineto{\pgfqpoint{2.747326in}{0.850476in}}%
\pgfpathlineto{\pgfqpoint{2.759646in}{0.837478in}}%
\pgfpathlineto{\pgfqpoint{2.771703in}{0.824144in}}%
\pgfpathlineto{\pgfqpoint{2.783500in}{0.810464in}}%
\pgfpathlineto{\pgfqpoint{2.795040in}{0.796426in}}%
\pgfpathlineto{\pgfqpoint{2.806324in}{0.782019in}}%
\pgfpathlineto{\pgfqpoint{2.817355in}{0.767229in}}%
\pgfpathlineto{\pgfqpoint{2.828134in}{0.752044in}}%
\pgfpathlineto{\pgfqpoint{2.838665in}{0.736446in}}%
\pgfpathlineto{\pgfqpoint{2.848947in}{0.720422in}}%
\pgfpathlineto{\pgfqpoint{2.858984in}{0.703954in}}%
\pgfpathlineto{\pgfqpoint{2.868775in}{0.687023in}}%
\pgfpathlineto{\pgfqpoint{2.878324in}{0.669611in}}%
\pgfpathlineto{\pgfqpoint{2.887631in}{0.651695in}}%
\pgfpathlineto{\pgfqpoint{2.896696in}{0.633252in}}%
\pgfpathlineto{\pgfqpoint{2.905523in}{0.614258in}}%
\pgfpathlineto{\pgfqpoint{2.914111in}{0.594686in}}%
\pgfpathlineto{\pgfqpoint{2.922462in}{0.574507in}}%
\pgfpathlineto{\pgfqpoint{2.930576in}{0.553690in}}%
\pgfpathlineto{\pgfqpoint{2.938455in}{0.532199in}}%
\pgfpathlineto{\pgfqpoint{2.944006in}{0.516079in}}%
\pgfusepath{stroke}%
\end{pgfscope}%
\begin{pgfscope}%
\pgfpathrectangle{\pgfqpoint{0.675193in}{0.526079in}}{\pgfqpoint{4.650000in}{3.020000in}} %
\pgfusepath{clip}%
\pgfsetrectcap%
\pgfsetroundjoin%
\pgfsetlinewidth{1.505625pt}%
\definecolor{currentstroke}{rgb}{0.000000,0.750000,0.750000}%
\pgfsetstrokecolor{currentstroke}%
\pgfsetstrokeopacity{0.750000}%
\pgfsetdash{}{0pt}%
\pgfpathmoveto{\pgfqpoint{1.067481in}{2.935311in}}%
\pgfpathlineto{\pgfqpoint{1.110135in}{2.875847in}}%
\pgfpathlineto{\pgfqpoint{1.152001in}{2.826114in}}%
\pgfpathlineto{\pgfqpoint{1.193092in}{2.779108in}}%
\pgfpathlineto{\pgfqpoint{1.233421in}{2.734756in}}%
\pgfpathlineto{\pgfqpoint{1.273001in}{2.692938in}}%
\pgfpathlineto{\pgfqpoint{1.311843in}{2.653516in}}%
\pgfpathlineto{\pgfqpoint{1.349961in}{2.616349in}}%
\pgfpathlineto{\pgfqpoint{1.387368in}{2.581291in}}%
\pgfpathlineto{\pgfqpoint{1.424075in}{2.548202in}}%
\pgfpathlineto{\pgfqpoint{1.460095in}{2.516948in}}%
\pgfpathlineto{\pgfqpoint{1.495442in}{2.487404in}}%
\pgfpathlineto{\pgfqpoint{1.530127in}{2.459452in}}%
\pgfpathlineto{\pgfqpoint{1.564159in}{2.432984in}}%
\pgfpathlineto{\pgfqpoint{1.597553in}{2.407901in}}%
\pgfpathlineto{\pgfqpoint{1.630322in}{2.384107in}}%
\pgfpathlineto{\pgfqpoint{1.662479in}{2.361518in}}%
\pgfpathlineto{\pgfqpoint{1.694039in}{2.340055in}}%
\pgfpathlineto{\pgfqpoint{1.725017in}{2.319645in}}%
\pgfpathlineto{\pgfqpoint{1.755427in}{2.300222in}}%
\pgfpathlineto{\pgfqpoint{1.785284in}{2.281723in}}%
\pgfpathlineto{\pgfqpoint{1.814604in}{2.264091in}}%
\pgfpathlineto{\pgfqpoint{1.843400in}{2.247274in}}%
\pgfpathlineto{\pgfqpoint{1.871689in}{2.231223in}}%
\pgfpathlineto{\pgfqpoint{1.899483in}{2.215893in}}%
\pgfpathlineto{\pgfqpoint{1.926797in}{2.201241in}}%
\pgfpathlineto{\pgfqpoint{1.953643in}{2.187230in}}%
\pgfpathlineto{\pgfqpoint{1.980032in}{2.173823in}}%
\pgfpathlineto{\pgfqpoint{2.005975in}{2.160987in}}%
\pgfpathlineto{\pgfqpoint{2.031482in}{2.148691in}}%
\pgfpathlineto{\pgfqpoint{2.056561in}{2.136906in}}%
\pgfpathlineto{\pgfqpoint{2.081220in}{2.125604in}}%
\pgfpathlineto{\pgfqpoint{2.105467in}{2.114762in}}%
\pgfpathlineto{\pgfqpoint{2.129309in}{2.104355in}}%
\pgfpathlineto{\pgfqpoint{2.152752in}{2.094362in}}%
\pgfpathlineto{\pgfqpoint{2.175801in}{2.084762in}}%
\pgfpathlineto{\pgfqpoint{2.198463in}{2.075536in}}%
\pgfpathlineto{\pgfqpoint{2.220743in}{2.066664in}}%
\pgfpathlineto{\pgfqpoint{2.242646in}{2.058132in}}%
\pgfpathlineto{\pgfqpoint{2.264177in}{2.049923in}}%
\pgfpathlineto{\pgfqpoint{2.285342in}{2.042021in}}%
\pgfpathlineto{\pgfqpoint{2.306146in}{2.034414in}}%
\pgfpathlineto{\pgfqpoint{2.326594in}{2.027087in}}%
\pgfpathlineto{\pgfqpoint{2.346691in}{2.020029in}}%
\pgfpathlineto{\pgfqpoint{2.366442in}{2.013228in}}%
\pgfpathlineto{\pgfqpoint{2.385853in}{2.006673in}}%
\pgfpathlineto{\pgfqpoint{2.404929in}{2.000352in}}%
\pgfpathlineto{\pgfqpoint{2.423673in}{1.994257in}}%
\pgfpathlineto{\pgfqpoint{2.442092in}{1.988379in}}%
\pgfpathlineto{\pgfqpoint{2.460188in}{1.982708in}}%
\pgfpathlineto{\pgfqpoint{2.477967in}{1.977235in}}%
\pgfpathlineto{\pgfqpoint{2.495432in}{1.971954in}}%
\pgfpathlineto{\pgfqpoint{2.512587in}{1.966857in}}%
\pgfpathlineto{\pgfqpoint{2.529435in}{1.961937in}}%
\pgfpathlineto{\pgfqpoint{2.545981in}{1.957186in}}%
\pgfpathlineto{\pgfqpoint{2.562226in}{1.952599in}}%
\pgfpathlineto{\pgfqpoint{2.578173in}{1.948169in}}%
\pgfpathlineto{\pgfqpoint{2.593827in}{1.943891in}}%
\pgfpathlineto{\pgfqpoint{2.609189in}{1.939759in}}%
\pgfpathlineto{\pgfqpoint{2.624262in}{1.935769in}}%
\pgfpathlineto{\pgfqpoint{2.639048in}{1.931914in}}%
\pgfpathlineto{\pgfqpoint{2.653551in}{1.928191in}}%
\pgfpathlineto{\pgfqpoint{2.667773in}{1.924595in}}%
\pgfpathlineto{\pgfqpoint{2.681715in}{1.921122in}}%
\pgfpathlineto{\pgfqpoint{2.695381in}{1.917768in}}%
\pgfpathlineto{\pgfqpoint{2.708772in}{1.914528in}}%
\pgfpathlineto{\pgfqpoint{2.721892in}{1.911399in}}%
\pgfpathlineto{\pgfqpoint{2.734743in}{1.908378in}}%
\pgfpathlineto{\pgfqpoint{2.747326in}{1.905462in}}%
\pgfpathlineto{\pgfqpoint{2.759646in}{1.902646in}}%
\pgfpathlineto{\pgfqpoint{2.771703in}{1.899928in}}%
\pgfpathlineto{\pgfqpoint{2.783500in}{1.897305in}}%
\pgfpathlineto{\pgfqpoint{2.795040in}{1.894775in}}%
\pgfpathlineto{\pgfqpoint{2.806324in}{1.892333in}}%
\pgfpathlineto{\pgfqpoint{2.817355in}{1.889979in}}%
\pgfpathlineto{\pgfqpoint{2.828134in}{1.887710in}}%
\pgfpathlineto{\pgfqpoint{2.838665in}{1.885523in}}%
\pgfpathlineto{\pgfqpoint{2.848947in}{1.883415in}}%
\pgfpathlineto{\pgfqpoint{2.858984in}{1.881386in}}%
\pgfpathlineto{\pgfqpoint{2.868775in}{1.879433in}}%
\pgfpathlineto{\pgfqpoint{2.878324in}{1.877553in}}%
\pgfpathlineto{\pgfqpoint{2.887631in}{1.875746in}}%
\pgfpathlineto{\pgfqpoint{2.896696in}{1.874008in}}%
\pgfpathlineto{\pgfqpoint{2.905523in}{1.872338in}}%
\pgfpathlineto{\pgfqpoint{2.914111in}{1.870735in}}%
\pgfpathlineto{\pgfqpoint{2.922462in}{1.869198in}}%
\pgfpathlineto{\pgfqpoint{2.930576in}{1.867725in}}%
\pgfpathlineto{\pgfqpoint{2.938455in}{1.866314in}}%
\pgfpathlineto{\pgfqpoint{2.946100in}{1.864964in}}%
\pgfpathlineto{\pgfqpoint{2.953512in}{1.863673in}}%
\pgfpathlineto{\pgfqpoint{2.960692in}{1.862440in}}%
\pgfpathlineto{\pgfqpoint{2.967640in}{1.861265in}}%
\pgfpathlineto{\pgfqpoint{2.974359in}{1.860146in}}%
\pgfpathlineto{\pgfqpoint{2.980849in}{1.859082in}}%
\pgfpathlineto{\pgfqpoint{2.987110in}{1.858071in}}%
\pgfpathlineto{\pgfqpoint{2.993145in}{1.857114in}}%
\pgfpathlineto{\pgfqpoint{2.998953in}{1.856208in}}%
\pgfpathlineto{\pgfqpoint{3.004537in}{1.855353in}}%
\pgfpathlineto{\pgfqpoint{3.009897in}{1.854548in}}%
\pgfpathlineto{\pgfqpoint{3.015034in}{1.853793in}}%
\pgfpathlineto{\pgfqpoint{3.019948in}{1.853086in}}%
\pgfpathlineto{\pgfqpoint{3.024641in}{1.852426in}}%
\pgfusepath{stroke}%
\end{pgfscope}%
\begin{pgfscope}%
\pgfpathrectangle{\pgfqpoint{0.675193in}{0.526079in}}{\pgfqpoint{4.650000in}{3.020000in}} %
\pgfusepath{clip}%
\pgfsetrectcap%
\pgfsetroundjoin%
\pgfsetlinewidth{1.505625pt}%
\definecolor{currentstroke}{rgb}{1.000000,0.000000,0.000000}%
\pgfsetstrokecolor{currentstroke}%
\pgfsetstrokeopacity{0.750000}%
\pgfsetdash{}{0pt}%
\pgfpathmoveto{\pgfqpoint{0.943739in}{3.109294in}}%
\pgfpathlineto{\pgfqpoint{0.980675in}{3.139589in}}%
\pgfpathlineto{\pgfqpoint{1.017119in}{3.168014in}}%
\pgfpathlineto{\pgfqpoint{1.053078in}{3.190522in}}%
\pgfpathlineto{\pgfqpoint{1.088558in}{3.208666in}}%
\pgfpathlineto{\pgfqpoint{1.123568in}{3.223519in}}%
\pgfpathlineto{\pgfqpoint{1.158114in}{3.235842in}}%
\pgfpathlineto{\pgfqpoint{1.192204in}{3.246189in}}%
\pgfpathlineto{\pgfqpoint{1.259043in}{3.262498in}}%
\pgfpathlineto{\pgfqpoint{1.324142in}{3.274668in}}%
\pgfpathlineto{\pgfqpoint{1.387562in}{3.284027in}}%
\pgfpathlineto{\pgfqpoint{1.479675in}{3.294541in}}%
\pgfpathlineto{\pgfqpoint{1.568331in}{3.302257in}}%
\pgfpathlineto{\pgfqpoint{1.681452in}{3.309815in}}%
\pgfpathlineto{\pgfqpoint{1.840881in}{3.317704in}}%
\pgfpathlineto{\pgfqpoint{2.035702in}{3.324844in}}%
\pgfpathlineto{\pgfqpoint{2.372198in}{3.334503in}}%
\pgfpathlineto{\pgfqpoint{2.793221in}{3.346979in}}%
\pgfpathlineto{\pgfqpoint{3.009233in}{3.355503in}}%
\pgfpathlineto{\pgfqpoint{3.174533in}{3.364164in}}%
\pgfpathlineto{\pgfqpoint{3.289183in}{3.372278in}}%
\pgfpathlineto{\pgfqpoint{3.362836in}{3.379594in}}%
\pgfpathlineto{\pgfqpoint{3.387937in}{3.383035in}}%
\pgfpathlineto{\pgfqpoint{3.387937in}{3.383035in}}%
\pgfusepath{stroke}%
\end{pgfscope}%
\begin{pgfscope}%
\pgfpathrectangle{\pgfqpoint{0.675193in}{0.526079in}}{\pgfqpoint{4.650000in}{3.020000in}} %
\pgfusepath{clip}%
\pgfsetrectcap%
\pgfsetroundjoin%
\pgfsetlinewidth{1.505625pt}%
\definecolor{currentstroke}{rgb}{0.000000,0.000000,1.000000}%
\pgfsetstrokecolor{currentstroke}%
\pgfsetstrokeopacity{0.750000}%
\pgfsetdash{}{0pt}%
\pgfpathmoveto{\pgfqpoint{0.943739in}{1.646129in}}%
\pgfpathlineto{\pgfqpoint{1.017119in}{1.697632in}}%
\pgfpathlineto{\pgfqpoint{1.053078in}{1.717357in}}%
\pgfpathlineto{\pgfqpoint{1.088558in}{1.733094in}}%
\pgfpathlineto{\pgfqpoint{1.123568in}{1.745829in}}%
\pgfpathlineto{\pgfqpoint{1.158114in}{1.756260in}}%
\pgfpathlineto{\pgfqpoint{1.192204in}{1.764893in}}%
\pgfpathlineto{\pgfqpoint{1.259043in}{1.778167in}}%
\pgfpathlineto{\pgfqpoint{1.324142in}{1.787649in}}%
\pgfpathlineto{\pgfqpoint{1.387562in}{1.794521in}}%
\pgfpathlineto{\pgfqpoint{1.479675in}{1.801452in}}%
\pgfpathlineto{\pgfqpoint{1.568331in}{1.805548in}}%
\pgfpathlineto{\pgfqpoint{1.681452in}{1.807932in}}%
\pgfpathlineto{\pgfqpoint{1.789059in}{1.807697in}}%
\pgfpathlineto{\pgfqpoint{1.891425in}{1.805373in}}%
\pgfpathlineto{\pgfqpoint{1.988803in}{1.801250in}}%
\pgfpathlineto{\pgfqpoint{2.081453in}{1.795491in}}%
\pgfpathlineto{\pgfqpoint{2.169646in}{1.788188in}}%
\pgfpathlineto{\pgfqpoint{2.253639in}{1.779389in}}%
\pgfpathlineto{\pgfqpoint{2.333643in}{1.769120in}}%
\pgfpathlineto{\pgfqpoint{2.409810in}{1.757385in}}%
\pgfpathlineto{\pgfqpoint{2.482260in}{1.744176in}}%
\pgfpathlineto{\pgfqpoint{2.551111in}{1.729472in}}%
\pgfpathlineto{\pgfqpoint{2.616493in}{1.713243in}}%
\pgfpathlineto{\pgfqpoint{2.678551in}{1.695445in}}%
\pgfpathlineto{\pgfqpoint{2.737421in}{1.676028in}}%
\pgfpathlineto{\pgfqpoint{2.793221in}{1.654924in}}%
\pgfpathlineto{\pgfqpoint{2.846058in}{1.632059in}}%
\pgfpathlineto{\pgfqpoint{2.896038in}{1.607340in}}%
\pgfpathlineto{\pgfqpoint{2.931715in}{1.587518in}}%
\pgfpathlineto{\pgfqpoint{2.965898in}{1.566539in}}%
\pgfpathlineto{\pgfqpoint{2.998635in}{1.544342in}}%
\pgfpathlineto{\pgfqpoint{3.029967in}{1.520860in}}%
\pgfpathlineto{\pgfqpoint{3.059929in}{1.496013in}}%
\pgfpathlineto{\pgfqpoint{3.088550in}{1.469716in}}%
\pgfpathlineto{\pgfqpoint{3.115850in}{1.441863in}}%
\pgfpathlineto{\pgfqpoint{3.141850in}{1.412342in}}%
\pgfpathlineto{\pgfqpoint{3.166572in}{1.381018in}}%
\pgfpathlineto{\pgfqpoint{3.190038in}{1.347735in}}%
\pgfpathlineto{\pgfqpoint{3.212271in}{1.312316in}}%
\pgfpathlineto{\pgfqpoint{3.233290in}{1.274547in}}%
\pgfpathlineto{\pgfqpoint{3.253109in}{1.234176in}}%
\pgfpathlineto{\pgfqpoint{3.271738in}{1.190908in}}%
\pgfpathlineto{\pgfqpoint{3.289183in}{1.144379in}}%
\pgfpathlineto{\pgfqpoint{3.305455in}{1.094149in}}%
\pgfpathlineto{\pgfqpoint{3.320566in}{1.039672in}}%
\pgfpathlineto{\pgfqpoint{3.334538in}{0.980260in}}%
\pgfpathlineto{\pgfqpoint{3.347394in}{0.915030in}}%
\pgfpathlineto{\pgfqpoint{3.359156in}{0.842824in}}%
\pgfpathlineto{\pgfqpoint{3.369836in}{0.762073in}}%
\pgfpathlineto{\pgfqpoint{3.379433in}{0.670591in}}%
\pgfpathlineto{\pgfqpoint{3.387937in}{0.565182in}}%
\pgfpathlineto{\pgfqpoint{3.387937in}{0.565182in}}%
\pgfusepath{stroke}%
\end{pgfscope}%
\begin{pgfscope}%
\pgfpathrectangle{\pgfqpoint{0.675193in}{0.526079in}}{\pgfqpoint{4.650000in}{3.020000in}} %
\pgfusepath{clip}%
\pgfsetrectcap%
\pgfsetroundjoin%
\pgfsetlinewidth{1.505625pt}%
\definecolor{currentstroke}{rgb}{0.000000,0.750000,0.750000}%
\pgfsetstrokecolor{currentstroke}%
\pgfsetstrokeopacity{0.750000}%
\pgfsetdash{}{0pt}%
\pgfpathmoveto{\pgfqpoint{0.943739in}{2.701670in}}%
\pgfpathlineto{\pgfqpoint{0.980675in}{2.614061in}}%
\pgfpathlineto{\pgfqpoint{1.017119in}{2.538367in}}%
\pgfpathlineto{\pgfqpoint{1.053078in}{2.467940in}}%
\pgfpathlineto{\pgfqpoint{1.088558in}{2.402408in}}%
\pgfpathlineto{\pgfqpoint{1.123568in}{2.341370in}}%
\pgfpathlineto{\pgfqpoint{1.158114in}{2.284440in}}%
\pgfpathlineto{\pgfqpoint{1.192204in}{2.231258in}}%
\pgfpathlineto{\pgfqpoint{1.225844in}{2.181497in}}%
\pgfpathlineto{\pgfqpoint{1.259043in}{2.134857in}}%
\pgfpathlineto{\pgfqpoint{1.291806in}{2.091070in}}%
\pgfpathlineto{\pgfqpoint{1.324142in}{2.049893in}}%
\pgfpathlineto{\pgfqpoint{1.387562in}{1.974529in}}%
\pgfpathlineto{\pgfqpoint{1.449363in}{1.907290in}}%
\pgfpathlineto{\pgfqpoint{1.509602in}{1.846983in}}%
\pgfpathlineto{\pgfqpoint{1.568331in}{1.792632in}}%
\pgfpathlineto{\pgfqpoint{1.625600in}{1.743440in}}%
\pgfpathlineto{\pgfqpoint{1.681452in}{1.698744in}}%
\pgfpathlineto{\pgfqpoint{1.735927in}{1.657992in}}%
\pgfpathlineto{\pgfqpoint{1.789059in}{1.620718in}}%
\pgfpathlineto{\pgfqpoint{1.840881in}{1.586524in}}%
\pgfpathlineto{\pgfqpoint{1.916227in}{1.540286in}}%
\pgfpathlineto{\pgfqpoint{1.988803in}{1.499286in}}%
\pgfpathlineto{\pgfqpoint{2.058719in}{1.462759in}}%
\pgfpathlineto{\pgfqpoint{2.126090in}{1.430082in}}%
\pgfpathlineto{\pgfqpoint{2.191029in}{1.400744in}}%
\pgfpathlineto{\pgfqpoint{2.274007in}{1.366089in}}%
\pgfpathlineto{\pgfqpoint{2.353039in}{1.335790in}}%
\pgfpathlineto{\pgfqpoint{2.428266in}{1.309187in}}%
\pgfpathlineto{\pgfqpoint{2.517127in}{1.280326in}}%
\pgfpathlineto{\pgfqpoint{2.600465in}{1.255576in}}%
\pgfpathlineto{\pgfqpoint{2.693563in}{1.230385in}}%
\pgfpathlineto{\pgfqpoint{2.779552in}{1.209282in}}%
\pgfpathlineto{\pgfqpoint{2.871399in}{1.188916in}}%
\pgfpathlineto{\pgfqpoint{2.965898in}{1.170229in}}%
\pgfpathlineto{\pgfqpoint{3.059929in}{1.153933in}}%
\pgfpathlineto{\pgfqpoint{3.150232in}{1.140528in}}%
\pgfpathlineto{\pgfqpoint{3.233290in}{1.130341in}}%
\pgfpathlineto{\pgfqpoint{3.305455in}{1.123571in}}%
\pgfpathlineto{\pgfqpoint{3.362836in}{1.120341in}}%
\pgfpathlineto{\pgfqpoint{3.387937in}{1.120121in}}%
\pgfpathlineto{\pgfqpoint{3.387937in}{1.120121in}}%
\pgfusepath{stroke}%
\end{pgfscope}%
\begin{pgfscope}%
\pgfpathrectangle{\pgfqpoint{0.675193in}{0.526079in}}{\pgfqpoint{4.650000in}{3.020000in}} %
\pgfusepath{clip}%
\pgfsetrectcap%
\pgfsetroundjoin%
\pgfsetlinewidth{1.505625pt}%
\definecolor{currentstroke}{rgb}{1.000000,0.000000,0.000000}%
\pgfsetstrokecolor{currentstroke}%
\pgfsetstrokeopacity{0.750000}%
\pgfsetdash{}{0pt}%
\pgfpathmoveto{\pgfqpoint{1.220940in}{3.083002in}}%
\pgfpathlineto{\pgfqpoint{1.307898in}{3.127765in}}%
\pgfpathlineto{\pgfqpoint{1.350641in}{3.146660in}}%
\pgfpathlineto{\pgfqpoint{1.392893in}{3.162970in}}%
\pgfpathlineto{\pgfqpoint{1.434652in}{3.177151in}}%
\pgfpathlineto{\pgfqpoint{1.475918in}{3.189567in}}%
\pgfpathlineto{\pgfqpoint{1.556968in}{3.210187in}}%
\pgfpathlineto{\pgfqpoint{1.636036in}{3.226528in}}%
\pgfpathlineto{\pgfqpoint{1.713116in}{3.239719in}}%
\pgfpathlineto{\pgfqpoint{1.788162in}{3.250536in}}%
\pgfpathlineto{\pgfqpoint{1.896926in}{3.263468in}}%
\pgfpathlineto{\pgfqpoint{2.001222in}{3.273527in}}%
\pgfpathlineto{\pgfqpoint{2.133731in}{3.283832in}}%
\pgfpathlineto{\pgfqpoint{2.289903in}{3.293339in}}%
\pgfpathlineto{\pgfqpoint{2.465171in}{3.301568in}}%
\pgfpathlineto{\pgfqpoint{2.680903in}{3.309252in}}%
\pgfpathlineto{\pgfqpoint{2.949075in}{3.316363in}}%
\pgfpathlineto{\pgfqpoint{3.312009in}{3.323484in}}%
\pgfpathlineto{\pgfqpoint{4.853163in}{3.350875in}}%
\pgfpathlineto{\pgfqpoint{4.911127in}{3.352364in}}%
\pgfpathlineto{\pgfqpoint{4.911127in}{3.352364in}}%
\pgfusepath{stroke}%
\end{pgfscope}%
\begin{pgfscope}%
\pgfpathrectangle{\pgfqpoint{0.675193in}{0.526079in}}{\pgfqpoint{4.650000in}{3.020000in}} %
\pgfusepath{clip}%
\pgfsetrectcap%
\pgfsetroundjoin%
\pgfsetlinewidth{1.505625pt}%
\definecolor{currentstroke}{rgb}{0.000000,0.000000,1.000000}%
\pgfsetstrokecolor{currentstroke}%
\pgfsetstrokeopacity{0.750000}%
\pgfsetdash{}{0pt}%
\pgfpathmoveto{\pgfqpoint{1.220940in}{1.140623in}}%
\pgfpathlineto{\pgfqpoint{1.307898in}{1.182442in}}%
\pgfpathlineto{\pgfqpoint{1.350641in}{1.200548in}}%
\pgfpathlineto{\pgfqpoint{1.392893in}{1.216417in}}%
\pgfpathlineto{\pgfqpoint{1.434652in}{1.230445in}}%
\pgfpathlineto{\pgfqpoint{1.516690in}{1.254171in}}%
\pgfpathlineto{\pgfqpoint{1.596750in}{1.273529in}}%
\pgfpathlineto{\pgfqpoint{1.674825in}{1.289699in}}%
\pgfpathlineto{\pgfqpoint{1.750894in}{1.303462in}}%
\pgfpathlineto{\pgfqpoint{1.861176in}{1.320728in}}%
\pgfpathlineto{\pgfqpoint{1.966942in}{1.334964in}}%
\pgfpathlineto{\pgfqpoint{2.101271in}{1.350484in}}%
\pgfpathlineto{\pgfqpoint{2.228613in}{1.363006in}}%
\pgfpathlineto{\pgfqpoint{2.379087in}{1.375409in}}%
\pgfpathlineto{\pgfqpoint{2.520930in}{1.384900in}}%
\pgfpathlineto{\pgfqpoint{2.680903in}{1.393090in}}%
\pgfpathlineto{\pgfqpoint{2.830888in}{1.398269in}}%
\pgfpathlineto{\pgfqpoint{2.972028in}{1.400775in}}%
\pgfpathlineto{\pgfqpoint{3.105305in}{1.400840in}}%
\pgfpathlineto{\pgfqpoint{3.231514in}{1.398640in}}%
\pgfpathlineto{\pgfqpoint{3.351207in}{1.394301in}}%
\pgfpathlineto{\pgfqpoint{3.464748in}{1.387922in}}%
\pgfpathlineto{\pgfqpoint{3.572486in}{1.379574in}}%
\pgfpathlineto{\pgfqpoint{3.674835in}{1.369308in}}%
\pgfpathlineto{\pgfqpoint{3.772173in}{1.357157in}}%
\pgfpathlineto{\pgfqpoint{3.864778in}{1.343136in}}%
\pgfpathlineto{\pgfqpoint{3.952928in}{1.327248in}}%
\pgfpathlineto{\pgfqpoint{4.037009in}{1.309475in}}%
\pgfpathlineto{\pgfqpoint{4.117435in}{1.289790in}}%
\pgfpathlineto{\pgfqpoint{4.181871in}{1.271887in}}%
\pgfpathlineto{\pgfqpoint{4.244009in}{1.252587in}}%
\pgfpathlineto{\pgfqpoint{4.303841in}{1.231845in}}%
\pgfpathlineto{\pgfqpoint{4.361377in}{1.209604in}}%
\pgfpathlineto{\pgfqpoint{4.416701in}{1.185802in}}%
\pgfpathlineto{\pgfqpoint{4.469954in}{1.160359in}}%
\pgfpathlineto{\pgfqpoint{4.521280in}{1.133185in}}%
\pgfpathlineto{\pgfqpoint{4.570759in}{1.104173in}}%
\pgfpathlineto{\pgfqpoint{4.618423in}{1.073197in}}%
\pgfpathlineto{\pgfqpoint{4.655255in}{1.046906in}}%
\pgfpathlineto{\pgfqpoint{4.690943in}{1.019176in}}%
\pgfpathlineto{\pgfqpoint{4.725505in}{0.989908in}}%
\pgfpathlineto{\pgfqpoint{4.758966in}{0.958990in}}%
\pgfpathlineto{\pgfqpoint{4.791366in}{0.926290in}}%
\pgfpathlineto{\pgfqpoint{4.822752in}{0.891655in}}%
\pgfpathlineto{\pgfqpoint{4.853163in}{0.854909in}}%
\pgfpathlineto{\pgfqpoint{4.882620in}{0.815844in}}%
\pgfpathlineto{\pgfqpoint{4.911127in}{0.774217in}}%
\pgfpathlineto{\pgfqpoint{4.911127in}{0.774217in}}%
\pgfusepath{stroke}%
\end{pgfscope}%
\begin{pgfscope}%
\pgfpathrectangle{\pgfqpoint{0.675193in}{0.526079in}}{\pgfqpoint{4.650000in}{3.020000in}} %
\pgfusepath{clip}%
\pgfsetrectcap%
\pgfsetroundjoin%
\pgfsetlinewidth{1.505625pt}%
\definecolor{currentstroke}{rgb}{0.000000,0.750000,0.750000}%
\pgfsetstrokecolor{currentstroke}%
\pgfsetstrokeopacity{0.750000}%
\pgfsetdash{}{0pt}%
\pgfpathmoveto{\pgfqpoint{1.220940in}{2.801556in}}%
\pgfpathlineto{\pgfqpoint{1.264664in}{2.753902in}}%
\pgfpathlineto{\pgfqpoint{1.307898in}{2.711963in}}%
\pgfpathlineto{\pgfqpoint{1.350641in}{2.672275in}}%
\pgfpathlineto{\pgfqpoint{1.392893in}{2.634729in}}%
\pgfpathlineto{\pgfqpoint{1.434652in}{2.599196in}}%
\pgfpathlineto{\pgfqpoint{1.516690in}{2.533692in}}%
\pgfpathlineto{\pgfqpoint{1.596750in}{2.474832in}}%
\pgfpathlineto{\pgfqpoint{1.674825in}{2.421792in}}%
\pgfpathlineto{\pgfqpoint{1.750894in}{2.373846in}}%
\pgfpathlineto{\pgfqpoint{1.824922in}{2.330364in}}%
\pgfpathlineto{\pgfqpoint{1.896926in}{2.290811in}}%
\pgfpathlineto{\pgfqpoint{1.966942in}{2.254722in}}%
\pgfpathlineto{\pgfqpoint{2.035029in}{2.221702in}}%
\pgfpathlineto{\pgfqpoint{2.101271in}{2.191405in}}%
\pgfpathlineto{\pgfqpoint{2.197390in}{2.150431in}}%
\pgfpathlineto{\pgfqpoint{2.289903in}{2.114090in}}%
\pgfpathlineto{\pgfqpoint{2.379087in}{2.081699in}}%
\pgfpathlineto{\pgfqpoint{2.465171in}{2.052698in}}%
\pgfpathlineto{\pgfqpoint{2.548339in}{2.026621in}}%
\pgfpathlineto{\pgfqpoint{2.654968in}{1.995746in}}%
\pgfpathlineto{\pgfqpoint{2.757068in}{1.968648in}}%
\pgfpathlineto{\pgfqpoint{2.854997in}{1.944733in}}%
\pgfpathlineto{\pgfqpoint{2.972028in}{1.918591in}}%
\pgfpathlineto{\pgfqpoint{3.083601in}{1.895927in}}%
\pgfpathlineto{\pgfqpoint{3.210943in}{1.872510in}}%
\pgfpathlineto{\pgfqpoint{3.331694in}{1.852484in}}%
\pgfpathlineto{\pgfqpoint{3.464748in}{1.832604in}}%
\pgfpathlineto{\pgfqpoint{3.607180in}{1.813597in}}%
\pgfpathlineto{\pgfqpoint{3.756285in}{1.795980in}}%
\pgfpathlineto{\pgfqpoint{3.909388in}{1.780088in}}%
\pgfpathlineto{\pgfqpoint{4.077656in}{1.764992in}}%
\pgfpathlineto{\pgfqpoint{4.244009in}{1.752392in}}%
\pgfpathlineto{\pgfqpoint{4.416701in}{1.741534in}}%
\pgfpathlineto{\pgfqpoint{4.590042in}{1.732817in}}%
\pgfpathlineto{\pgfqpoint{4.767164in}{1.726189in}}%
\pgfpathlineto{\pgfqpoint{4.911127in}{1.722602in}}%
\pgfpathlineto{\pgfqpoint{4.911127in}{1.722602in}}%
\pgfusepath{stroke}%
\end{pgfscope}%
\begin{pgfscope}%
\pgfpathrectangle{\pgfqpoint{0.675193in}{0.526079in}}{\pgfqpoint{4.650000in}{3.020000in}} %
\pgfusepath{clip}%
\pgfsetrectcap%
\pgfsetroundjoin%
\pgfsetlinewidth{1.505625pt}%
\definecolor{currentstroke}{rgb}{1.000000,0.000000,0.000000}%
\pgfsetstrokecolor{currentstroke}%
\pgfsetstrokeopacity{0.750000}%
\pgfsetdash{}{0pt}%
\pgfpathmoveto{\pgfqpoint{1.005049in}{2.927752in}}%
\pgfpathlineto{\pgfqpoint{1.052517in}{2.976819in}}%
\pgfpathlineto{\pgfqpoint{1.099547in}{3.024147in}}%
\pgfpathlineto{\pgfqpoint{1.146148in}{3.062068in}}%
\pgfpathlineto{\pgfqpoint{1.192328in}{3.092901in}}%
\pgfpathlineto{\pgfqpoint{1.238095in}{3.118282in}}%
\pgfpathlineto{\pgfqpoint{1.283458in}{3.139406in}}%
\pgfpathlineto{\pgfqpoint{1.328425in}{3.157162in}}%
\pgfpathlineto{\pgfqpoint{1.373005in}{3.172212in}}%
\pgfpathlineto{\pgfqpoint{1.417205in}{3.185063in}}%
\pgfpathlineto{\pgfqpoint{1.461033in}{3.196110in}}%
\pgfpathlineto{\pgfqpoint{1.504500in}{3.205663in}}%
\pgfpathlineto{\pgfqpoint{1.547611in}{3.213966in}}%
\pgfpathlineto{\pgfqpoint{1.590380in}{3.221217in}}%
\pgfpathlineto{\pgfqpoint{1.632814in}{3.227574in}}%
\pgfpathlineto{\pgfqpoint{1.674921in}{3.233165in}}%
\pgfpathlineto{\pgfqpoint{1.716707in}{3.238097in}}%
\pgfpathlineto{\pgfqpoint{1.758181in}{3.242459in}}%
\pgfpathlineto{\pgfqpoint{1.799349in}{3.246326in}}%
\pgfpathlineto{\pgfqpoint{1.840218in}{3.249759in}}%
\pgfpathlineto{\pgfqpoint{1.880793in}{3.252810in}}%
\pgfpathlineto{\pgfqpoint{1.921080in}{3.255524in}}%
\pgfpathlineto{\pgfqpoint{1.961085in}{3.257940in}}%
\pgfpathlineto{\pgfqpoint{2.000814in}{3.260091in}}%
\pgfpathlineto{\pgfqpoint{2.040271in}{3.262002in}}%
\pgfpathlineto{\pgfqpoint{2.079463in}{3.263698in}}%
\pgfpathlineto{\pgfqpoint{2.118395in}{3.265204in}}%
\pgfpathlineto{\pgfqpoint{2.157075in}{3.266535in}}%
\pgfpathlineto{\pgfqpoint{2.195507in}{3.267710in}}%
\pgfpathlineto{\pgfqpoint{2.233698in}{3.268741in}}%
\pgfpathlineto{\pgfqpoint{2.271655in}{3.269643in}}%
\pgfpathlineto{\pgfqpoint{2.309385in}{3.270426in}}%
\pgfpathlineto{\pgfqpoint{2.346894in}{3.271100in}}%
\pgfpathlineto{\pgfqpoint{2.384188in}{3.271675in}}%
\pgfpathlineto{\pgfqpoint{2.421275in}{3.272158in}}%
\pgfpathlineto{\pgfqpoint{2.458159in}{3.272557in}}%
\pgfpathlineto{\pgfqpoint{2.494848in}{3.272879in}}%
\pgfpathlineto{\pgfqpoint{2.531347in}{3.273128in}}%
\pgfpathlineto{\pgfqpoint{2.567662in}{3.273309in}}%
\pgfpathlineto{\pgfqpoint{2.603796in}{3.273428in}}%
\pgfpathlineto{\pgfqpoint{2.639756in}{3.273491in}}%
\pgfpathlineto{\pgfqpoint{2.675544in}{3.273499in}}%
\pgfpathlineto{\pgfqpoint{2.711164in}{3.273456in}}%
\pgfpathlineto{\pgfqpoint{2.746620in}{3.273368in}}%
\pgfpathlineto{\pgfqpoint{2.781914in}{3.273236in}}%
\pgfpathlineto{\pgfqpoint{2.817049in}{3.273062in}}%
\pgfpathlineto{\pgfqpoint{2.852028in}{3.272851in}}%
\pgfpathlineto{\pgfqpoint{2.886853in}{3.272603in}}%
\pgfpathlineto{\pgfqpoint{2.921526in}{3.272321in}}%
\pgfpathlineto{\pgfqpoint{2.956050in}{3.272008in}}%
\pgfpathlineto{\pgfqpoint{2.990428in}{3.271664in}}%
\pgfpathlineto{\pgfqpoint{3.024662in}{3.271292in}}%
\pgfpathlineto{\pgfqpoint{3.058754in}{3.270893in}}%
\pgfpathlineto{\pgfqpoint{3.092707in}{3.270469in}}%
\pgfpathlineto{\pgfqpoint{3.126524in}{3.270021in}}%
\pgfpathlineto{\pgfqpoint{3.160209in}{3.269551in}}%
\pgfpathlineto{\pgfqpoint{3.193764in}{3.269061in}}%
\pgfpathlineto{\pgfqpoint{3.227192in}{3.268549in}}%
\pgfpathlineto{\pgfqpoint{3.260498in}{3.268018in}}%
\pgfpathlineto{\pgfqpoint{3.293683in}{3.267466in}}%
\pgfpathlineto{\pgfqpoint{3.326753in}{3.266899in}}%
\pgfpathlineto{\pgfqpoint{3.359709in}{3.266316in}}%
\pgfpathlineto{\pgfqpoint{3.392556in}{3.265717in}}%
\pgfpathlineto{\pgfqpoint{3.425296in}{3.265102in}}%
\pgfpathlineto{\pgfqpoint{3.457932in}{3.264473in}}%
\pgfpathlineto{\pgfqpoint{3.490468in}{3.263830in}}%
\pgfpathlineto{\pgfqpoint{3.522906in}{3.263176in}}%
\pgfpathlineto{\pgfqpoint{3.555248in}{3.262509in}}%
\pgfpathlineto{\pgfqpoint{3.587498in}{3.261829in}}%
\pgfpathlineto{\pgfqpoint{3.619656in}{3.261138in}}%
\pgfpathlineto{\pgfqpoint{3.651726in}{3.260437in}}%
\pgfpathlineto{\pgfqpoint{3.683709in}{3.259726in}}%
\pgfpathlineto{\pgfqpoint{3.715606in}{3.259005in}}%
\pgfpathlineto{\pgfqpoint{3.747421in}{3.258274in}}%
\pgfpathlineto{\pgfqpoint{3.779153in}{3.257534in}}%
\pgfpathlineto{\pgfqpoint{3.810805in}{3.256786in}}%
\pgfpathlineto{\pgfqpoint{3.842378in}{3.256030in}}%
\pgfpathlineto{\pgfqpoint{3.873873in}{3.255266in}}%
\pgfpathlineto{\pgfqpoint{3.905292in}{3.254494in}}%
\pgfpathlineto{\pgfqpoint{3.936636in}{3.253715in}}%
\pgfpathlineto{\pgfqpoint{3.967907in}{3.252930in}}%
\pgfpathlineto{\pgfqpoint{3.999105in}{3.252137in}}%
\pgfpathlineto{\pgfqpoint{4.030233in}{3.251339in}}%
\pgfpathlineto{\pgfqpoint{4.061290in}{3.250534in}}%
\pgfpathlineto{\pgfqpoint{4.092280in}{3.249723in}}%
\pgfpathlineto{\pgfqpoint{4.123202in}{3.248908in}}%
\pgfpathlineto{\pgfqpoint{4.154058in}{3.248088in}}%
\pgfpathlineto{\pgfqpoint{4.184849in}{3.247262in}}%
\pgfpathlineto{\pgfqpoint{4.215576in}{3.246431in}}%
\pgfpathlineto{\pgfqpoint{4.246240in}{3.245594in}}%
\pgfpathlineto{\pgfqpoint{4.276843in}{3.244751in}}%
\pgfpathlineto{\pgfqpoint{4.307384in}{3.243906in}}%
\pgfpathlineto{\pgfqpoint{4.337864in}{3.243057in}}%
\pgfpathlineto{\pgfqpoint{4.368286in}{3.242205in}}%
\pgfpathlineto{\pgfqpoint{4.398648in}{3.241348in}}%
\pgfpathlineto{\pgfqpoint{4.428952in}{3.240487in}}%
\pgfpathlineto{\pgfqpoint{4.459198in}{3.239622in}}%
\pgfpathlineto{\pgfqpoint{4.489388in}{3.238754in}}%
\pgfpathlineto{\pgfqpoint{4.519521in}{3.237883in}}%
\pgfpathlineto{\pgfqpoint{4.549598in}{3.237010in}}%
\pgfpathlineto{\pgfqpoint{4.579619in}{3.236134in}}%
\pgfpathlineto{\pgfqpoint{4.609587in}{3.235254in}}%
\pgfpathlineto{\pgfqpoint{4.639500in}{3.234372in}}%
\pgfpathlineto{\pgfqpoint{4.669360in}{3.233486in}}%
\pgfpathlineto{\pgfqpoint{4.699168in}{3.232598in}}%
\pgfpathlineto{\pgfqpoint{4.728924in}{3.231709in}}%
\pgfpathlineto{\pgfqpoint{4.758629in}{3.230817in}}%
\pgfpathlineto{\pgfqpoint{4.788284in}{3.229924in}}%
\pgfpathlineto{\pgfqpoint{4.817890in}{3.229028in}}%
\pgfpathlineto{\pgfqpoint{4.847448in}{3.228130in}}%
\pgfpathlineto{\pgfqpoint{4.876959in}{3.227229in}}%
\pgfpathlineto{\pgfqpoint{4.906424in}{3.226327in}}%
\pgfpathlineto{\pgfqpoint{4.935843in}{3.225424in}}%
\pgfpathlineto{\pgfqpoint{4.965218in}{3.224519in}}%
\pgfpathlineto{\pgfqpoint{4.994548in}{3.223613in}}%
\pgfpathlineto{\pgfqpoint{5.023836in}{3.222705in}}%
\pgfpathlineto{\pgfqpoint{5.053081in}{3.221795in}}%
\pgfpathlineto{\pgfqpoint{5.082285in}{3.220884in}}%
\pgfpathlineto{\pgfqpoint{5.111447in}{3.219972in}}%
\pgfusepath{stroke}%
\end{pgfscope}%
\begin{pgfscope}%
\pgfpathrectangle{\pgfqpoint{0.675193in}{0.526079in}}{\pgfqpoint{4.650000in}{3.020000in}} %
\pgfusepath{clip}%
\pgfsetrectcap%
\pgfsetroundjoin%
\pgfsetlinewidth{1.505625pt}%
\definecolor{currentstroke}{rgb}{0.000000,0.000000,1.000000}%
\pgfsetstrokecolor{currentstroke}%
\pgfsetstrokeopacity{0.750000}%
\pgfsetdash{}{0pt}%
\pgfpathmoveto{\pgfqpoint{1.005049in}{1.420136in}}%
\pgfpathlineto{\pgfqpoint{1.052517in}{1.471280in}}%
\pgfpathlineto{\pgfqpoint{1.099547in}{1.521719in}}%
\pgfpathlineto{\pgfqpoint{1.146148in}{1.563537in}}%
\pgfpathlineto{\pgfqpoint{1.192328in}{1.598882in}}%
\pgfpathlineto{\pgfqpoint{1.238095in}{1.629271in}}%
\pgfpathlineto{\pgfqpoint{1.283458in}{1.655807in}}%
\pgfpathlineto{\pgfqpoint{1.328425in}{1.679313in}}%
\pgfpathlineto{\pgfqpoint{1.373005in}{1.700397in}}%
\pgfpathlineto{\pgfqpoint{1.417205in}{1.719525in}}%
\pgfpathlineto{\pgfqpoint{1.461033in}{1.737056in}}%
\pgfpathlineto{\pgfqpoint{1.504500in}{1.753272in}}%
\pgfpathlineto{\pgfqpoint{1.547611in}{1.768392in}}%
\pgfpathlineto{\pgfqpoint{1.590380in}{1.782595in}}%
\pgfpathlineto{\pgfqpoint{1.632814in}{1.796020in}}%
\pgfpathlineto{\pgfqpoint{1.674921in}{1.808780in}}%
\pgfpathlineto{\pgfqpoint{1.716707in}{1.820968in}}%
\pgfpathlineto{\pgfqpoint{1.758181in}{1.832663in}}%
\pgfpathlineto{\pgfqpoint{1.799349in}{1.843926in}}%
\pgfpathlineto{\pgfqpoint{1.840218in}{1.854809in}}%
\pgfpathlineto{\pgfqpoint{1.880793in}{1.865358in}}%
\pgfpathlineto{\pgfqpoint{1.921080in}{1.875608in}}%
\pgfpathlineto{\pgfqpoint{1.961085in}{1.885590in}}%
\pgfpathlineto{\pgfqpoint{2.000814in}{1.895332in}}%
\pgfpathlineto{\pgfqpoint{2.040271in}{1.904852in}}%
\pgfpathlineto{\pgfqpoint{2.079463in}{1.914173in}}%
\pgfpathlineto{\pgfqpoint{2.118395in}{1.923309in}}%
\pgfpathlineto{\pgfqpoint{2.157075in}{1.932276in}}%
\pgfpathlineto{\pgfqpoint{2.195507in}{1.941085in}}%
\pgfpathlineto{\pgfqpoint{2.233698in}{1.949747in}}%
\pgfpathlineto{\pgfqpoint{2.271655in}{1.958272in}}%
\pgfpathlineto{\pgfqpoint{2.309385in}{1.966666in}}%
\pgfpathlineto{\pgfqpoint{2.346894in}{1.974939in}}%
\pgfpathlineto{\pgfqpoint{2.384188in}{1.983094in}}%
\pgfpathlineto{\pgfqpoint{2.421275in}{1.991139in}}%
\pgfpathlineto{\pgfqpoint{2.458159in}{1.999078in}}%
\pgfpathlineto{\pgfqpoint{2.494848in}{2.006916in}}%
\pgfpathlineto{\pgfqpoint{2.531347in}{2.014657in}}%
\pgfpathlineto{\pgfqpoint{2.567662in}{2.022301in}}%
\pgfpathlineto{\pgfqpoint{2.603796in}{2.029854in}}%
\pgfpathlineto{\pgfqpoint{2.639756in}{2.037321in}}%
\pgfpathlineto{\pgfqpoint{2.675544in}{2.044701in}}%
\pgfpathlineto{\pgfqpoint{2.711164in}{2.051998in}}%
\pgfpathlineto{\pgfqpoint{2.746620in}{2.059215in}}%
\pgfpathlineto{\pgfqpoint{2.781914in}{2.066352in}}%
\pgfpathlineto{\pgfqpoint{2.817049in}{2.073412in}}%
\pgfpathlineto{\pgfqpoint{2.852028in}{2.080398in}}%
\pgfpathlineto{\pgfqpoint{2.886853in}{2.087309in}}%
\pgfpathlineto{\pgfqpoint{2.921526in}{2.094148in}}%
\pgfpathlineto{\pgfqpoint{2.956050in}{2.100916in}}%
\pgfpathlineto{\pgfqpoint{2.990428in}{2.107616in}}%
\pgfpathlineto{\pgfqpoint{3.024662in}{2.114246in}}%
\pgfpathlineto{\pgfqpoint{3.058754in}{2.120810in}}%
\pgfpathlineto{\pgfqpoint{3.092707in}{2.127308in}}%
\pgfpathlineto{\pgfqpoint{3.126524in}{2.133741in}}%
\pgfpathlineto{\pgfqpoint{3.160209in}{2.140110in}}%
\pgfpathlineto{\pgfqpoint{3.193764in}{2.146418in}}%
\pgfpathlineto{\pgfqpoint{3.227192in}{2.152664in}}%
\pgfpathlineto{\pgfqpoint{3.260498in}{2.158847in}}%
\pgfpathlineto{\pgfqpoint{3.293683in}{2.164970in}}%
\pgfpathlineto{\pgfqpoint{3.326753in}{2.171034in}}%
\pgfpathlineto{\pgfqpoint{3.359709in}{2.177041in}}%
\pgfpathlineto{\pgfqpoint{3.392556in}{2.182991in}}%
\pgfpathlineto{\pgfqpoint{3.425296in}{2.188883in}}%
\pgfpathlineto{\pgfqpoint{3.457932in}{2.194719in}}%
\pgfpathlineto{\pgfqpoint{3.490468in}{2.200500in}}%
\pgfpathlineto{\pgfqpoint{3.522906in}{2.206227in}}%
\pgfpathlineto{\pgfqpoint{3.555248in}{2.211901in}}%
\pgfpathlineto{\pgfqpoint{3.587498in}{2.217521in}}%
\pgfpathlineto{\pgfqpoint{3.619656in}{2.223088in}}%
\pgfpathlineto{\pgfqpoint{3.651726in}{2.228605in}}%
\pgfpathlineto{\pgfqpoint{3.683709in}{2.234071in}}%
\pgfpathlineto{\pgfqpoint{3.715606in}{2.239486in}}%
\pgfpathlineto{\pgfqpoint{3.747421in}{2.244851in}}%
\pgfpathlineto{\pgfqpoint{3.779153in}{2.250167in}}%
\pgfpathlineto{\pgfqpoint{3.810805in}{2.255435in}}%
\pgfpathlineto{\pgfqpoint{3.842378in}{2.260656in}}%
\pgfpathlineto{\pgfqpoint{3.873873in}{2.265829in}}%
\pgfpathlineto{\pgfqpoint{3.905292in}{2.270955in}}%
\pgfpathlineto{\pgfqpoint{3.936636in}{2.276035in}}%
\pgfpathlineto{\pgfqpoint{3.967907in}{2.281070in}}%
\pgfpathlineto{\pgfqpoint{3.999105in}{2.286060in}}%
\pgfpathlineto{\pgfqpoint{4.030233in}{2.291005in}}%
\pgfpathlineto{\pgfqpoint{4.061290in}{2.295907in}}%
\pgfpathlineto{\pgfqpoint{4.092280in}{2.300765in}}%
\pgfpathlineto{\pgfqpoint{4.123202in}{2.305582in}}%
\pgfpathlineto{\pgfqpoint{4.154058in}{2.310356in}}%
\pgfpathlineto{\pgfqpoint{4.184849in}{2.315088in}}%
\pgfpathlineto{\pgfqpoint{4.215576in}{2.319779in}}%
\pgfpathlineto{\pgfqpoint{4.246240in}{2.324427in}}%
\pgfpathlineto{\pgfqpoint{4.276843in}{2.329035in}}%
\pgfpathlineto{\pgfqpoint{4.307384in}{2.333604in}}%
\pgfpathlineto{\pgfqpoint{4.337864in}{2.338134in}}%
\pgfpathlineto{\pgfqpoint{4.368286in}{2.342626in}}%
\pgfpathlineto{\pgfqpoint{4.398648in}{2.347080in}}%
\pgfpathlineto{\pgfqpoint{4.428952in}{2.351494in}}%
\pgfpathlineto{\pgfqpoint{4.459198in}{2.355871in}}%
\pgfpathlineto{\pgfqpoint{4.489388in}{2.360210in}}%
\pgfpathlineto{\pgfqpoint{4.519521in}{2.364514in}}%
\pgfpathlineto{\pgfqpoint{4.549598in}{2.368782in}}%
\pgfpathlineto{\pgfqpoint{4.579619in}{2.373014in}}%
\pgfpathlineto{\pgfqpoint{4.609587in}{2.377211in}}%
\pgfpathlineto{\pgfqpoint{4.639500in}{2.381372in}}%
\pgfpathlineto{\pgfqpoint{4.669360in}{2.385498in}}%
\pgfpathlineto{\pgfqpoint{4.699168in}{2.389590in}}%
\pgfpathlineto{\pgfqpoint{4.728924in}{2.393649in}}%
\pgfpathlineto{\pgfqpoint{4.758629in}{2.397675in}}%
\pgfpathlineto{\pgfqpoint{4.788284in}{2.401668in}}%
\pgfpathlineto{\pgfqpoint{4.817890in}{2.405629in}}%
\pgfpathlineto{\pgfqpoint{4.847448in}{2.409556in}}%
\pgfpathlineto{\pgfqpoint{4.876959in}{2.413452in}}%
\pgfpathlineto{\pgfqpoint{4.906424in}{2.417316in}}%
\pgfpathlineto{\pgfqpoint{4.935843in}{2.421149in}}%
\pgfpathlineto{\pgfqpoint{4.965218in}{2.424952in}}%
\pgfpathlineto{\pgfqpoint{4.994548in}{2.428725in}}%
\pgfpathlineto{\pgfqpoint{5.023836in}{2.432467in}}%
\pgfpathlineto{\pgfqpoint{5.053081in}{2.436179in}}%
\pgfpathlineto{\pgfqpoint{5.082285in}{2.439861in}}%
\pgfpathlineto{\pgfqpoint{5.111447in}{2.443515in}}%
\pgfusepath{stroke}%
\end{pgfscope}%
\begin{pgfscope}%
\pgfpathrectangle{\pgfqpoint{0.675193in}{0.526079in}}{\pgfqpoint{4.650000in}{3.020000in}} %
\pgfusepath{clip}%
\pgfsetrectcap%
\pgfsetroundjoin%
\pgfsetlinewidth{1.505625pt}%
\definecolor{currentstroke}{rgb}{0.000000,0.750000,0.750000}%
\pgfsetstrokecolor{currentstroke}%
\pgfsetstrokeopacity{0.750000}%
\pgfsetdash{}{0pt}%
\pgfpathmoveto{\pgfqpoint{1.005049in}{2.976983in}}%
\pgfpathlineto{\pgfqpoint{1.052517in}{2.901744in}}%
\pgfpathlineto{\pgfqpoint{1.099547in}{2.839958in}}%
\pgfpathlineto{\pgfqpoint{1.146148in}{2.780969in}}%
\pgfpathlineto{\pgfqpoint{1.192328in}{2.724913in}}%
\pgfpathlineto{\pgfqpoint{1.238095in}{2.671785in}}%
\pgfpathlineto{\pgfqpoint{1.283458in}{2.621506in}}%
\pgfpathlineto{\pgfqpoint{1.328425in}{2.573961in}}%
\pgfpathlineto{\pgfqpoint{1.373005in}{2.529006in}}%
\pgfpathlineto{\pgfqpoint{1.417205in}{2.486489in}}%
\pgfpathlineto{\pgfqpoint{1.461033in}{2.446260in}}%
\pgfpathlineto{\pgfqpoint{1.504500in}{2.408173in}}%
\pgfpathlineto{\pgfqpoint{1.547611in}{2.372085in}}%
\pgfpathlineto{\pgfqpoint{1.590380in}{2.337866in}}%
\pgfpathlineto{\pgfqpoint{1.632814in}{2.305391in}}%
\pgfpathlineto{\pgfqpoint{1.674921in}{2.274541in}}%
\pgfpathlineto{\pgfqpoint{1.716707in}{2.245210in}}%
\pgfpathlineto{\pgfqpoint{1.758181in}{2.217298in}}%
\pgfpathlineto{\pgfqpoint{1.799349in}{2.190714in}}%
\pgfpathlineto{\pgfqpoint{1.840218in}{2.165372in}}%
\pgfpathlineto{\pgfqpoint{1.880793in}{2.141193in}}%
\pgfpathlineto{\pgfqpoint{1.921080in}{2.118104in}}%
\pgfpathlineto{\pgfqpoint{1.961085in}{2.096039in}}%
\pgfpathlineto{\pgfqpoint{2.000814in}{2.074935in}}%
\pgfpathlineto{\pgfqpoint{2.040271in}{2.054732in}}%
\pgfpathlineto{\pgfqpoint{2.079463in}{2.035378in}}%
\pgfpathlineto{\pgfqpoint{2.118395in}{2.016824in}}%
\pgfpathlineto{\pgfqpoint{2.157075in}{1.999023in}}%
\pgfpathlineto{\pgfqpoint{2.195507in}{1.981933in}}%
\pgfpathlineto{\pgfqpoint{2.233698in}{1.965514in}}%
\pgfpathlineto{\pgfqpoint{2.271655in}{1.949729in}}%
\pgfpathlineto{\pgfqpoint{2.309385in}{1.934542in}}%
\pgfpathlineto{\pgfqpoint{2.346894in}{1.919923in}}%
\pgfpathlineto{\pgfqpoint{2.384188in}{1.905840in}}%
\pgfpathlineto{\pgfqpoint{2.421275in}{1.892266in}}%
\pgfpathlineto{\pgfqpoint{2.458159in}{1.879175in}}%
\pgfpathlineto{\pgfqpoint{2.494848in}{1.866542in}}%
\pgfpathlineto{\pgfqpoint{2.531347in}{1.854344in}}%
\pgfpathlineto{\pgfqpoint{2.567662in}{1.842557in}}%
\pgfpathlineto{\pgfqpoint{2.603796in}{1.831163in}}%
\pgfpathlineto{\pgfqpoint{2.639756in}{1.820144in}}%
\pgfpathlineto{\pgfqpoint{2.675544in}{1.809480in}}%
\pgfpathlineto{\pgfqpoint{2.711164in}{1.799154in}}%
\pgfpathlineto{\pgfqpoint{2.746620in}{1.789153in}}%
\pgfpathlineto{\pgfqpoint{2.781914in}{1.779459in}}%
\pgfpathlineto{\pgfqpoint{2.817049in}{1.770058in}}%
\pgfpathlineto{\pgfqpoint{2.852028in}{1.760939in}}%
\pgfpathlineto{\pgfqpoint{2.886853in}{1.752088in}}%
\pgfpathlineto{\pgfqpoint{2.921526in}{1.743493in}}%
\pgfpathlineto{\pgfqpoint{2.956050in}{1.735143in}}%
\pgfpathlineto{\pgfqpoint{2.990428in}{1.727028in}}%
\pgfpathlineto{\pgfqpoint{3.024662in}{1.719137in}}%
\pgfpathlineto{\pgfqpoint{3.058754in}{1.711461in}}%
\pgfpathlineto{\pgfqpoint{3.092707in}{1.703991in}}%
\pgfpathlineto{\pgfqpoint{3.126524in}{1.696719in}}%
\pgfpathlineto{\pgfqpoint{3.160209in}{1.689636in}}%
\pgfpathlineto{\pgfqpoint{3.193764in}{1.682735in}}%
\pgfpathlineto{\pgfqpoint{3.227192in}{1.676009in}}%
\pgfpathlineto{\pgfqpoint{3.260498in}{1.669449in}}%
\pgfpathlineto{\pgfqpoint{3.293683in}{1.663049in}}%
\pgfpathlineto{\pgfqpoint{3.326753in}{1.656804in}}%
\pgfpathlineto{\pgfqpoint{3.359709in}{1.650709in}}%
\pgfpathlineto{\pgfqpoint{3.392556in}{1.644757in}}%
\pgfpathlineto{\pgfqpoint{3.425296in}{1.638941in}}%
\pgfpathlineto{\pgfqpoint{3.457932in}{1.633258in}}%
\pgfpathlineto{\pgfqpoint{3.490468in}{1.627702in}}%
\pgfpathlineto{\pgfqpoint{3.522906in}{1.622270in}}%
\pgfpathlineto{\pgfqpoint{3.555248in}{1.616957in}}%
\pgfpathlineto{\pgfqpoint{3.587498in}{1.611756in}}%
\pgfpathlineto{\pgfqpoint{3.619656in}{1.606666in}}%
\pgfpathlineto{\pgfqpoint{3.651726in}{1.601682in}}%
\pgfpathlineto{\pgfqpoint{3.683709in}{1.596801in}}%
\pgfpathlineto{\pgfqpoint{3.715606in}{1.592019in}}%
\pgfpathlineto{\pgfqpoint{3.747421in}{1.587332in}}%
\pgfpathlineto{\pgfqpoint{3.779153in}{1.582737in}}%
\pgfpathlineto{\pgfqpoint{3.810805in}{1.578232in}}%
\pgfpathlineto{\pgfqpoint{3.842378in}{1.573813in}}%
\pgfpathlineto{\pgfqpoint{3.873873in}{1.569477in}}%
\pgfpathlineto{\pgfqpoint{3.905292in}{1.565221in}}%
\pgfpathlineto{\pgfqpoint{3.936636in}{1.561043in}}%
\pgfpathlineto{\pgfqpoint{3.967907in}{1.556941in}}%
\pgfpathlineto{\pgfqpoint{3.999105in}{1.552912in}}%
\pgfpathlineto{\pgfqpoint{4.030233in}{1.548954in}}%
\pgfpathlineto{\pgfqpoint{4.061290in}{1.545064in}}%
\pgfpathlineto{\pgfqpoint{4.092280in}{1.541241in}}%
\pgfpathlineto{\pgfqpoint{4.123202in}{1.537483in}}%
\pgfpathlineto{\pgfqpoint{4.154058in}{1.533787in}}%
\pgfpathlineto{\pgfqpoint{4.184849in}{1.530152in}}%
\pgfpathlineto{\pgfqpoint{4.215576in}{1.526574in}}%
\pgfpathlineto{\pgfqpoint{4.246240in}{1.523053in}}%
\pgfpathlineto{\pgfqpoint{4.276843in}{1.519586in}}%
\pgfpathlineto{\pgfqpoint{4.307384in}{1.516173in}}%
\pgfpathlineto{\pgfqpoint{4.337864in}{1.512814in}}%
\pgfpathlineto{\pgfqpoint{4.368286in}{1.509505in}}%
\pgfpathlineto{\pgfqpoint{4.398648in}{1.506246in}}%
\pgfpathlineto{\pgfqpoint{4.428952in}{1.503033in}}%
\pgfpathlineto{\pgfqpoint{4.459198in}{1.499867in}}%
\pgfpathlineto{\pgfqpoint{4.489388in}{1.496745in}}%
\pgfpathlineto{\pgfqpoint{4.519521in}{1.493668in}}%
\pgfpathlineto{\pgfqpoint{4.549598in}{1.490634in}}%
\pgfpathlineto{\pgfqpoint{4.579619in}{1.487642in}}%
\pgfpathlineto{\pgfqpoint{4.609587in}{1.484689in}}%
\pgfpathlineto{\pgfqpoint{4.639500in}{1.481776in}}%
\pgfpathlineto{\pgfqpoint{4.669360in}{1.478900in}}%
\pgfpathlineto{\pgfqpoint{4.699168in}{1.476062in}}%
\pgfpathlineto{\pgfqpoint{4.728924in}{1.473261in}}%
\pgfpathlineto{\pgfqpoint{4.758629in}{1.470495in}}%
\pgfpathlineto{\pgfqpoint{4.788284in}{1.467764in}}%
\pgfpathlineto{\pgfqpoint{4.817890in}{1.465066in}}%
\pgfpathlineto{\pgfqpoint{4.847448in}{1.462401in}}%
\pgfpathlineto{\pgfqpoint{4.876959in}{1.459767in}}%
\pgfpathlineto{\pgfqpoint{4.906424in}{1.457164in}}%
\pgfpathlineto{\pgfqpoint{4.935843in}{1.454592in}}%
\pgfpathlineto{\pgfqpoint{4.965218in}{1.452051in}}%
\pgfpathlineto{\pgfqpoint{4.994548in}{1.449537in}}%
\pgfpathlineto{\pgfqpoint{5.023836in}{1.447052in}}%
\pgfpathlineto{\pgfqpoint{5.053081in}{1.444595in}}%
\pgfpathlineto{\pgfqpoint{5.082285in}{1.442163in}}%
\pgfpathlineto{\pgfqpoint{5.111447in}{1.439759in}}%
\pgfusepath{stroke}%
\end{pgfscope}%
\begin{pgfscope}%
\pgfpathrectangle{\pgfqpoint{0.675193in}{0.526079in}}{\pgfqpoint{4.650000in}{3.020000in}} %
\pgfusepath{clip}%
\pgfsetrectcap%
\pgfsetroundjoin%
\pgfsetlinewidth{1.505625pt}%
\definecolor{currentstroke}{rgb}{1.000000,0.000000,0.000000}%
\pgfsetstrokecolor{currentstroke}%
\pgfsetstrokeopacity{0.750000}%
\pgfsetdash{}{0pt}%
\pgfpathmoveto{\pgfqpoint{1.266608in}{3.094201in}}%
\pgfpathlineto{\pgfqpoint{1.315801in}{3.113358in}}%
\pgfpathlineto{\pgfqpoint{1.364866in}{3.131846in}}%
\pgfpathlineto{\pgfqpoint{1.413797in}{3.147300in}}%
\pgfpathlineto{\pgfqpoint{1.462586in}{3.160319in}}%
\pgfpathlineto{\pgfqpoint{1.511228in}{3.171351in}}%
\pgfpathlineto{\pgfqpoint{1.559715in}{3.180737in}}%
\pgfpathlineto{\pgfqpoint{1.608040in}{3.188757in}}%
\pgfpathlineto{\pgfqpoint{1.656198in}{3.195628in}}%
\pgfpathlineto{\pgfqpoint{1.704181in}{3.201523in}}%
\pgfpathlineto{\pgfqpoint{1.751983in}{3.206587in}}%
\pgfpathlineto{\pgfqpoint{1.799596in}{3.210938in}}%
\pgfpathlineto{\pgfqpoint{1.847015in}{3.214673in}}%
\pgfpathlineto{\pgfqpoint{1.894222in}{3.217872in}}%
\pgfpathlineto{\pgfqpoint{1.941215in}{3.220603in}}%
\pgfpathlineto{\pgfqpoint{1.987986in}{3.222919in}}%
\pgfpathlineto{\pgfqpoint{2.034532in}{3.224869in}}%
\pgfpathlineto{\pgfqpoint{2.080847in}{3.226496in}}%
\pgfpathlineto{\pgfqpoint{2.126928in}{3.227832in}}%
\pgfpathlineto{\pgfqpoint{2.172772in}{3.228910in}}%
\pgfpathlineto{\pgfqpoint{2.218380in}{3.229755in}}%
\pgfpathlineto{\pgfqpoint{2.263751in}{3.230389in}}%
\pgfpathlineto{\pgfqpoint{2.308887in}{3.230832in}}%
\pgfpathlineto{\pgfqpoint{2.353790in}{3.231102in}}%
\pgfpathlineto{\pgfqpoint{2.398465in}{3.231213in}}%
\pgfpathlineto{\pgfqpoint{2.442915in}{3.231181in}}%
\pgfpathlineto{\pgfqpoint{2.487147in}{3.231017in}}%
\pgfpathlineto{\pgfqpoint{2.531166in}{3.230728in}}%
\pgfpathlineto{\pgfqpoint{2.574980in}{3.230326in}}%
\pgfpathlineto{\pgfqpoint{2.618594in}{3.229822in}}%
\pgfpathlineto{\pgfqpoint{2.662017in}{3.229219in}}%
\pgfpathlineto{\pgfqpoint{2.705255in}{3.228528in}}%
\pgfpathlineto{\pgfqpoint{2.748315in}{3.227753in}}%
\pgfpathlineto{\pgfqpoint{2.791204in}{3.226900in}}%
\pgfpathlineto{\pgfqpoint{2.833930in}{3.225975in}}%
\pgfpathlineto{\pgfqpoint{2.876500in}{3.224981in}}%
\pgfpathlineto{\pgfqpoint{2.918921in}{3.223923in}}%
\pgfpathlineto{\pgfqpoint{2.961199in}{3.222806in}}%
\pgfpathlineto{\pgfqpoint{3.003342in}{3.221632in}}%
\pgfpathlineto{\pgfqpoint{3.045355in}{3.220405in}}%
\pgfpathlineto{\pgfqpoint{3.087242in}{3.219128in}}%
\pgfpathlineto{\pgfqpoint{3.129009in}{3.217803in}}%
\pgfpathlineto{\pgfqpoint{3.170661in}{3.216434in}}%
\pgfpathlineto{\pgfqpoint{3.212203in}{3.215024in}}%
\pgfpathlineto{\pgfqpoint{3.253639in}{3.213572in}}%
\pgfpathlineto{\pgfqpoint{3.294973in}{3.212081in}}%
\pgfpathlineto{\pgfqpoint{3.336209in}{3.210553in}}%
\pgfpathlineto{\pgfqpoint{3.377348in}{3.208993in}}%
\pgfpathlineto{\pgfqpoint{3.418395in}{3.207399in}}%
\pgfpathlineto{\pgfqpoint{3.459351in}{3.205773in}}%
\pgfpathlineto{\pgfqpoint{3.500221in}{3.204117in}}%
\pgfpathlineto{\pgfqpoint{3.541007in}{3.202434in}}%
\pgfpathlineto{\pgfqpoint{3.581713in}{3.200724in}}%
\pgfpathlineto{\pgfqpoint{3.622340in}{3.198987in}}%
\pgfpathlineto{\pgfqpoint{3.662890in}{3.197225in}}%
\pgfpathlineto{\pgfqpoint{3.703364in}{3.195440in}}%
\pgfpathlineto{\pgfqpoint{3.743763in}{3.193633in}}%
\pgfpathlineto{\pgfqpoint{3.784089in}{3.191803in}}%
\pgfpathlineto{\pgfqpoint{3.824343in}{3.189952in}}%
\pgfpathlineto{\pgfqpoint{3.864526in}{3.188082in}}%
\pgfpathlineto{\pgfqpoint{3.904640in}{3.186193in}}%
\pgfpathlineto{\pgfqpoint{3.944684in}{3.184285in}}%
\pgfpathlineto{\pgfqpoint{3.984657in}{3.182360in}}%
\pgfpathlineto{\pgfqpoint{4.024559in}{3.180418in}}%
\pgfpathlineto{\pgfqpoint{4.064388in}{3.178460in}}%
\pgfpathlineto{\pgfqpoint{4.104146in}{3.176486in}}%
\pgfpathlineto{\pgfqpoint{4.143832in}{3.174497in}}%
\pgfpathlineto{\pgfqpoint{4.183446in}{3.172494in}}%
\pgfpathlineto{\pgfqpoint{4.222988in}{3.170479in}}%
\pgfpathlineto{\pgfqpoint{4.262458in}{3.168450in}}%
\pgfpathlineto{\pgfqpoint{4.301855in}{3.166408in}}%
\pgfpathlineto{\pgfqpoint{4.341179in}{3.164354in}}%
\pgfpathlineto{\pgfqpoint{4.380429in}{3.162286in}}%
\pgfpathlineto{\pgfqpoint{4.419609in}{3.160207in}}%
\pgfpathlineto{\pgfqpoint{4.458719in}{3.158118in}}%
\pgfpathlineto{\pgfqpoint{4.497761in}{3.156020in}}%
\pgfpathlineto{\pgfqpoint{4.536737in}{3.153912in}}%
\pgfpathlineto{\pgfqpoint{4.575646in}{3.151794in}}%
\pgfpathlineto{\pgfqpoint{4.614490in}{3.149666in}}%
\pgfpathlineto{\pgfqpoint{4.653269in}{3.147528in}}%
\pgfpathlineto{\pgfqpoint{4.691985in}{3.145383in}}%
\pgfpathlineto{\pgfqpoint{4.730638in}{3.143231in}}%
\pgfpathlineto{\pgfqpoint{4.769231in}{3.141071in}}%
\pgfpathlineto{\pgfqpoint{4.807764in}{3.138903in}}%
\pgfpathlineto{\pgfqpoint{4.846237in}{3.136727in}}%
\pgfpathlineto{\pgfqpoint{4.884649in}{3.134544in}}%
\pgfpathlineto{\pgfqpoint{4.923000in}{3.132355in}}%
\pgfpathlineto{\pgfqpoint{4.961289in}{3.130161in}}%
\pgfpathlineto{\pgfqpoint{4.999516in}{3.127961in}}%
\pgfpathlineto{\pgfqpoint{5.037682in}{3.125754in}}%
\pgfpathlineto{\pgfqpoint{5.075786in}{3.123542in}}%
\pgfpathlineto{\pgfqpoint{5.113829in}{3.121324in}}%
\pgfusepath{stroke}%
\end{pgfscope}%
\begin{pgfscope}%
\pgfpathrectangle{\pgfqpoint{0.675193in}{0.526079in}}{\pgfqpoint{4.650000in}{3.020000in}} %
\pgfusepath{clip}%
\pgfsetrectcap%
\pgfsetroundjoin%
\pgfsetlinewidth{1.505625pt}%
\definecolor{currentstroke}{rgb}{0.000000,0.000000,1.000000}%
\pgfsetstrokecolor{currentstroke}%
\pgfsetstrokeopacity{0.750000}%
\pgfsetdash{}{0pt}%
\pgfpathmoveto{\pgfqpoint{1.266608in}{1.818921in}}%
\pgfpathlineto{\pgfqpoint{1.315801in}{1.846948in}}%
\pgfpathlineto{\pgfqpoint{1.364866in}{1.874638in}}%
\pgfpathlineto{\pgfqpoint{1.413797in}{1.899575in}}%
\pgfpathlineto{\pgfqpoint{1.462586in}{1.922316in}}%
\pgfpathlineto{\pgfqpoint{1.511228in}{1.943273in}}%
\pgfpathlineto{\pgfqpoint{1.559715in}{1.962761in}}%
\pgfpathlineto{\pgfqpoint{1.608040in}{1.981034in}}%
\pgfpathlineto{\pgfqpoint{1.656198in}{1.998288in}}%
\pgfpathlineto{\pgfqpoint{1.704181in}{2.014679in}}%
\pgfpathlineto{\pgfqpoint{1.751983in}{2.030332in}}%
\pgfpathlineto{\pgfqpoint{1.799596in}{2.045355in}}%
\pgfpathlineto{\pgfqpoint{1.847015in}{2.059828in}}%
\pgfpathlineto{\pgfqpoint{1.894222in}{2.073821in}}%
\pgfpathlineto{\pgfqpoint{1.941215in}{2.087391in}}%
\pgfpathlineto{\pgfqpoint{1.987986in}{2.100581in}}%
\pgfpathlineto{\pgfqpoint{2.034532in}{2.113431in}}%
\pgfpathlineto{\pgfqpoint{2.080847in}{2.125975in}}%
\pgfpathlineto{\pgfqpoint{2.126928in}{2.138240in}}%
\pgfpathlineto{\pgfqpoint{2.172772in}{2.150250in}}%
\pgfpathlineto{\pgfqpoint{2.218380in}{2.162024in}}%
\pgfpathlineto{\pgfqpoint{2.263751in}{2.173580in}}%
\pgfpathlineto{\pgfqpoint{2.308887in}{2.184930in}}%
\pgfpathlineto{\pgfqpoint{2.353790in}{2.196089in}}%
\pgfpathlineto{\pgfqpoint{2.398465in}{2.207066in}}%
\pgfpathlineto{\pgfqpoint{2.442915in}{2.217872in}}%
\pgfpathlineto{\pgfqpoint{2.487147in}{2.228515in}}%
\pgfpathlineto{\pgfqpoint{2.531166in}{2.238998in}}%
\pgfpathlineto{\pgfqpoint{2.574980in}{2.249332in}}%
\pgfpathlineto{\pgfqpoint{2.618594in}{2.259521in}}%
\pgfpathlineto{\pgfqpoint{2.662017in}{2.269570in}}%
\pgfpathlineto{\pgfqpoint{2.705255in}{2.279484in}}%
\pgfpathlineto{\pgfqpoint{2.748315in}{2.289267in}}%
\pgfpathlineto{\pgfqpoint{2.791204in}{2.298921in}}%
\pgfpathlineto{\pgfqpoint{2.833930in}{2.308451in}}%
\pgfpathlineto{\pgfqpoint{2.876500in}{2.317860in}}%
\pgfpathlineto{\pgfqpoint{2.918921in}{2.327150in}}%
\pgfpathlineto{\pgfqpoint{2.961199in}{2.336324in}}%
\pgfpathlineto{\pgfqpoint{3.003342in}{2.345384in}}%
\pgfpathlineto{\pgfqpoint{3.045355in}{2.354332in}}%
\pgfpathlineto{\pgfqpoint{3.087242in}{2.363171in}}%
\pgfpathlineto{\pgfqpoint{3.129009in}{2.371902in}}%
\pgfpathlineto{\pgfqpoint{3.170661in}{2.380529in}}%
\pgfpathlineto{\pgfqpoint{3.212203in}{2.389052in}}%
\pgfpathlineto{\pgfqpoint{3.253639in}{2.397472in}}%
\pgfpathlineto{\pgfqpoint{3.294973in}{2.405789in}}%
\pgfpathlineto{\pgfqpoint{3.336209in}{2.414007in}}%
\pgfpathlineto{\pgfqpoint{3.377348in}{2.422130in}}%
\pgfpathlineto{\pgfqpoint{3.418395in}{2.430156in}}%
\pgfpathlineto{\pgfqpoint{3.459351in}{2.438087in}}%
\pgfpathlineto{\pgfqpoint{3.500221in}{2.445924in}}%
\pgfpathlineto{\pgfqpoint{3.541007in}{2.453670in}}%
\pgfpathlineto{\pgfqpoint{3.581713in}{2.461325in}}%
\pgfpathlineto{\pgfqpoint{3.622340in}{2.468890in}}%
\pgfpathlineto{\pgfqpoint{3.662890in}{2.476366in}}%
\pgfpathlineto{\pgfqpoint{3.703364in}{2.483756in}}%
\pgfpathlineto{\pgfqpoint{3.743763in}{2.491061in}}%
\pgfpathlineto{\pgfqpoint{3.784089in}{2.498280in}}%
\pgfpathlineto{\pgfqpoint{3.824343in}{2.505416in}}%
\pgfpathlineto{\pgfqpoint{3.864526in}{2.512469in}}%
\pgfpathlineto{\pgfqpoint{3.904640in}{2.519442in}}%
\pgfpathlineto{\pgfqpoint{3.944684in}{2.526334in}}%
\pgfpathlineto{\pgfqpoint{3.984657in}{2.533146in}}%
\pgfpathlineto{\pgfqpoint{4.024559in}{2.539881in}}%
\pgfpathlineto{\pgfqpoint{4.064388in}{2.546538in}}%
\pgfpathlineto{\pgfqpoint{4.104146in}{2.553120in}}%
\pgfpathlineto{\pgfqpoint{4.143832in}{2.559627in}}%
\pgfpathlineto{\pgfqpoint{4.183446in}{2.566059in}}%
\pgfpathlineto{\pgfqpoint{4.222988in}{2.572420in}}%
\pgfpathlineto{\pgfqpoint{4.262458in}{2.578709in}}%
\pgfpathlineto{\pgfqpoint{4.301855in}{2.584927in}}%
\pgfpathlineto{\pgfqpoint{4.341179in}{2.591073in}}%
\pgfpathlineto{\pgfqpoint{4.380429in}{2.597149in}}%
\pgfpathlineto{\pgfqpoint{4.419609in}{2.603156in}}%
\pgfpathlineto{\pgfqpoint{4.458719in}{2.609098in}}%
\pgfpathlineto{\pgfqpoint{4.497761in}{2.614974in}}%
\pgfpathlineto{\pgfqpoint{4.536737in}{2.620784in}}%
\pgfpathlineto{\pgfqpoint{4.575646in}{2.626530in}}%
\pgfpathlineto{\pgfqpoint{4.614490in}{2.632210in}}%
\pgfpathlineto{\pgfqpoint{4.653269in}{2.637827in}}%
\pgfpathlineto{\pgfqpoint{4.691985in}{2.643382in}}%
\pgfpathlineto{\pgfqpoint{4.730638in}{2.648878in}}%
\pgfpathlineto{\pgfqpoint{4.769231in}{2.654312in}}%
\pgfpathlineto{\pgfqpoint{4.807764in}{2.659687in}}%
\pgfpathlineto{\pgfqpoint{4.846237in}{2.665002in}}%
\pgfpathlineto{\pgfqpoint{4.884649in}{2.670258in}}%
\pgfpathlineto{\pgfqpoint{4.923000in}{2.675458in}}%
\pgfpathlineto{\pgfqpoint{4.961289in}{2.680602in}}%
\pgfpathlineto{\pgfqpoint{4.999516in}{2.685690in}}%
\pgfpathlineto{\pgfqpoint{5.037682in}{2.690723in}}%
\pgfpathlineto{\pgfqpoint{5.075786in}{2.695701in}}%
\pgfpathlineto{\pgfqpoint{5.113829in}{2.700624in}}%
\pgfusepath{stroke}%
\end{pgfscope}%
\begin{pgfscope}%
\pgfpathrectangle{\pgfqpoint{0.675193in}{0.526079in}}{\pgfqpoint{4.650000in}{3.020000in}} %
\pgfusepath{clip}%
\pgfsetrectcap%
\pgfsetroundjoin%
\pgfsetlinewidth{1.505625pt}%
\definecolor{currentstroke}{rgb}{0.000000,0.750000,0.750000}%
\pgfsetstrokecolor{currentstroke}%
\pgfsetstrokeopacity{0.750000}%
\pgfsetdash{}{0pt}%
\pgfpathmoveto{\pgfqpoint{1.266608in}{2.687629in}}%
\pgfpathlineto{\pgfqpoint{1.315801in}{2.631711in}}%
\pgfpathlineto{\pgfqpoint{1.364866in}{2.581807in}}%
\pgfpathlineto{\pgfqpoint{1.413797in}{2.534643in}}%
\pgfpathlineto{\pgfqpoint{1.462586in}{2.490080in}}%
\pgfpathlineto{\pgfqpoint{1.511228in}{2.447958in}}%
\pgfpathlineto{\pgfqpoint{1.559715in}{2.408113in}}%
\pgfpathlineto{\pgfqpoint{1.608040in}{2.370397in}}%
\pgfpathlineto{\pgfqpoint{1.656198in}{2.334667in}}%
\pgfpathlineto{\pgfqpoint{1.704181in}{2.300785in}}%
\pgfpathlineto{\pgfqpoint{1.751983in}{2.268626in}}%
\pgfpathlineto{\pgfqpoint{1.799596in}{2.238075in}}%
\pgfpathlineto{\pgfqpoint{1.847015in}{2.209023in}}%
\pgfpathlineto{\pgfqpoint{1.894222in}{2.181371in}}%
\pgfpathlineto{\pgfqpoint{1.941215in}{2.155025in}}%
\pgfpathlineto{\pgfqpoint{1.987986in}{2.129899in}}%
\pgfpathlineto{\pgfqpoint{2.034532in}{2.105912in}}%
\pgfpathlineto{\pgfqpoint{2.080847in}{2.082995in}}%
\pgfpathlineto{\pgfqpoint{2.126928in}{2.061077in}}%
\pgfpathlineto{\pgfqpoint{2.172772in}{2.040098in}}%
\pgfpathlineto{\pgfqpoint{2.218380in}{2.019999in}}%
\pgfpathlineto{\pgfqpoint{2.263751in}{2.000727in}}%
\pgfpathlineto{\pgfqpoint{2.308887in}{1.982232in}}%
\pgfpathlineto{\pgfqpoint{2.353790in}{1.964467in}}%
\pgfpathlineto{\pgfqpoint{2.398465in}{1.947391in}}%
\pgfpathlineto{\pgfqpoint{2.442915in}{1.930963in}}%
\pgfpathlineto{\pgfqpoint{2.487147in}{1.915147in}}%
\pgfpathlineto{\pgfqpoint{2.531166in}{1.899905in}}%
\pgfpathlineto{\pgfqpoint{2.574980in}{1.885208in}}%
\pgfpathlineto{\pgfqpoint{2.618594in}{1.871027in}}%
\pgfpathlineto{\pgfqpoint{2.662017in}{1.857330in}}%
\pgfpathlineto{\pgfqpoint{2.705255in}{1.844095in}}%
\pgfpathlineto{\pgfqpoint{2.748315in}{1.831296in}}%
\pgfpathlineto{\pgfqpoint{2.791204in}{1.818909in}}%
\pgfpathlineto{\pgfqpoint{2.833930in}{1.806914in}}%
\pgfpathlineto{\pgfqpoint{2.876500in}{1.795290in}}%
\pgfpathlineto{\pgfqpoint{2.918921in}{1.784019in}}%
\pgfpathlineto{\pgfqpoint{2.961199in}{1.773083in}}%
\pgfpathlineto{\pgfqpoint{3.003342in}{1.762464in}}%
\pgfpathlineto{\pgfqpoint{3.045355in}{1.752149in}}%
\pgfpathlineto{\pgfqpoint{3.087242in}{1.742122in}}%
\pgfpathlineto{\pgfqpoint{3.129009in}{1.732368in}}%
\pgfpathlineto{\pgfqpoint{3.170661in}{1.722877in}}%
\pgfpathlineto{\pgfqpoint{3.212203in}{1.713635in}}%
\pgfpathlineto{\pgfqpoint{3.253639in}{1.704629in}}%
\pgfpathlineto{\pgfqpoint{3.294973in}{1.695848in}}%
\pgfpathlineto{\pgfqpoint{3.336209in}{1.687283in}}%
\pgfpathlineto{\pgfqpoint{3.377348in}{1.678927in}}%
\pgfpathlineto{\pgfqpoint{3.418395in}{1.670767in}}%
\pgfpathlineto{\pgfqpoint{3.459351in}{1.662794in}}%
\pgfpathlineto{\pgfqpoint{3.500221in}{1.655001in}}%
\pgfpathlineto{\pgfqpoint{3.541007in}{1.647382in}}%
\pgfpathlineto{\pgfqpoint{3.581713in}{1.639928in}}%
\pgfpathlineto{\pgfqpoint{3.622340in}{1.632631in}}%
\pgfpathlineto{\pgfqpoint{3.662890in}{1.625485in}}%
\pgfpathlineto{\pgfqpoint{3.703364in}{1.618485in}}%
\pgfpathlineto{\pgfqpoint{3.743763in}{1.611625in}}%
\pgfpathlineto{\pgfqpoint{3.784089in}{1.604898in}}%
\pgfpathlineto{\pgfqpoint{3.824343in}{1.598298in}}%
\pgfpathlineto{\pgfqpoint{3.864526in}{1.591821in}}%
\pgfpathlineto{\pgfqpoint{3.904640in}{1.585463in}}%
\pgfpathlineto{\pgfqpoint{3.944684in}{1.579219in}}%
\pgfpathlineto{\pgfqpoint{3.984657in}{1.573083in}}%
\pgfpathlineto{\pgfqpoint{4.024559in}{1.567052in}}%
\pgfpathlineto{\pgfqpoint{4.064388in}{1.561123in}}%
\pgfpathlineto{\pgfqpoint{4.104146in}{1.555291in}}%
\pgfpathlineto{\pgfqpoint{4.143832in}{1.549552in}}%
\pgfpathlineto{\pgfqpoint{4.183446in}{1.543903in}}%
\pgfpathlineto{\pgfqpoint{4.222988in}{1.538343in}}%
\pgfpathlineto{\pgfqpoint{4.262458in}{1.532866in}}%
\pgfpathlineto{\pgfqpoint{4.301855in}{1.527470in}}%
\pgfpathlineto{\pgfqpoint{4.341179in}{1.522150in}}%
\pgfpathlineto{\pgfqpoint{4.380429in}{1.516905in}}%
\pgfpathlineto{\pgfqpoint{4.419609in}{1.511732in}}%
\pgfpathlineto{\pgfqpoint{4.458719in}{1.506630in}}%
\pgfpathlineto{\pgfqpoint{4.497761in}{1.501597in}}%
\pgfpathlineto{\pgfqpoint{4.536737in}{1.496629in}}%
\pgfpathlineto{\pgfqpoint{4.575646in}{1.491725in}}%
\pgfpathlineto{\pgfqpoint{4.614490in}{1.486880in}}%
\pgfpathlineto{\pgfqpoint{4.653269in}{1.482094in}}%
\pgfpathlineto{\pgfqpoint{4.691985in}{1.477366in}}%
\pgfpathlineto{\pgfqpoint{4.730638in}{1.472695in}}%
\pgfpathlineto{\pgfqpoint{4.769231in}{1.468078in}}%
\pgfpathlineto{\pgfqpoint{4.807764in}{1.463512in}}%
\pgfpathlineto{\pgfqpoint{4.846237in}{1.458996in}}%
\pgfpathlineto{\pgfqpoint{4.884649in}{1.454528in}}%
\pgfpathlineto{\pgfqpoint{4.923000in}{1.450108in}}%
\pgfpathlineto{\pgfqpoint{4.961289in}{1.445736in}}%
\pgfpathlineto{\pgfqpoint{4.999516in}{1.441408in}}%
\pgfpathlineto{\pgfqpoint{5.037682in}{1.437123in}}%
\pgfpathlineto{\pgfqpoint{5.075786in}{1.432879in}}%
\pgfpathlineto{\pgfqpoint{5.113829in}{1.428676in}}%
\pgfusepath{stroke}%
\end{pgfscope}%
\begin{pgfscope}%
\pgfsetrectcap%
\pgfsetmiterjoin%
\pgfsetlinewidth{0.803000pt}%
\definecolor{currentstroke}{rgb}{0.000000,0.000000,0.000000}%
\pgfsetstrokecolor{currentstroke}%
\pgfsetdash{}{0pt}%
\pgfpathmoveto{\pgfqpoint{0.675193in}{0.526079in}}%
\pgfpathlineto{\pgfqpoint{0.675193in}{3.546079in}}%
\pgfusepath{stroke}%
\end{pgfscope}%
\begin{pgfscope}%
\pgfsetrectcap%
\pgfsetmiterjoin%
\pgfsetlinewidth{0.803000pt}%
\definecolor{currentstroke}{rgb}{0.000000,0.000000,0.000000}%
\pgfsetstrokecolor{currentstroke}%
\pgfsetdash{}{0pt}%
\pgfpathmoveto{\pgfqpoint{5.325193in}{0.526079in}}%
\pgfpathlineto{\pgfqpoint{5.325193in}{3.546079in}}%
\pgfusepath{stroke}%
\end{pgfscope}%
\begin{pgfscope}%
\pgfsetrectcap%
\pgfsetmiterjoin%
\pgfsetlinewidth{0.803000pt}%
\definecolor{currentstroke}{rgb}{0.000000,0.000000,0.000000}%
\pgfsetstrokecolor{currentstroke}%
\pgfsetdash{}{0pt}%
\pgfpathmoveto{\pgfqpoint{0.675193in}{0.526079in}}%
\pgfpathlineto{\pgfqpoint{5.325193in}{0.526079in}}%
\pgfusepath{stroke}%
\end{pgfscope}%
\begin{pgfscope}%
\pgfsetrectcap%
\pgfsetmiterjoin%
\pgfsetlinewidth{0.803000pt}%
\definecolor{currentstroke}{rgb}{0.000000,0.000000,0.000000}%
\pgfsetstrokecolor{currentstroke}%
\pgfsetdash{}{0pt}%
\pgfpathmoveto{\pgfqpoint{0.675193in}{3.546079in}}%
\pgfpathlineto{\pgfqpoint{5.325193in}{3.546079in}}%
\pgfusepath{stroke}%
\end{pgfscope}%
\begin{pgfscope}%
\pgfsetbuttcap%
\pgfsetmiterjoin%
\definecolor{currentfill}{rgb}{1.000000,1.000000,1.000000}%
\pgfsetfillcolor{currentfill}%
\pgfsetfillopacity{0.800000}%
\pgfsetlinewidth{1.003750pt}%
\definecolor{currentstroke}{rgb}{0.800000,0.800000,0.800000}%
\pgfsetstrokecolor{currentstroke}%
\pgfsetstrokeopacity{0.800000}%
\pgfsetdash{}{0pt}%
\pgfpathmoveto{\pgfqpoint{3.749998in}{0.595524in}}%
\pgfpathlineto{\pgfqpoint{5.227971in}{0.595524in}}%
\pgfpathquadraticcurveto{\pgfqpoint{5.255749in}{0.595524in}}{\pgfqpoint{5.255749in}{0.623302in}}%
\pgfpathlineto{\pgfqpoint{5.255749in}{1.224918in}}%
\pgfpathquadraticcurveto{\pgfqpoint{5.255749in}{1.252696in}}{\pgfqpoint{5.227971in}{1.252696in}}%
\pgfpathlineto{\pgfqpoint{3.749998in}{1.252696in}}%
\pgfpathquadraticcurveto{\pgfqpoint{3.722220in}{1.252696in}}{\pgfqpoint{3.722220in}{1.224918in}}%
\pgfpathlineto{\pgfqpoint{3.722220in}{0.623302in}}%
\pgfpathquadraticcurveto{\pgfqpoint{3.722220in}{0.595524in}}{\pgfqpoint{3.749998in}{0.595524in}}%
\pgfpathclose%
\pgfusepath{stroke,fill}%
\end{pgfscope}%
\begin{pgfscope}%
\pgfsetrectcap%
\pgfsetroundjoin%
\pgfsetlinewidth{1.505625pt}%
\definecolor{currentstroke}{rgb}{1.000000,0.000000,0.000000}%
\pgfsetstrokecolor{currentstroke}%
\pgfsetstrokeopacity{0.750000}%
\pgfsetdash{}{0pt}%
\pgfpathmoveto{\pgfqpoint{3.777776in}{1.140228in}}%
\pgfpathlineto{\pgfqpoint{4.055553in}{1.140228in}}%
\pgfusepath{stroke}%
\end{pgfscope}%
\begin{pgfscope}%
\pgftext[x=4.166664in,y=1.091617in,left,base]{\rmfamily\fontsize{10.000000}{12.000000}\selectfont Coulomb force}%
\end{pgfscope}%
\begin{pgfscope}%
\pgfsetrectcap%
\pgfsetroundjoin%
\pgfsetlinewidth{1.505625pt}%
\definecolor{currentstroke}{rgb}{0.000000,0.000000,1.000000}%
\pgfsetstrokecolor{currentstroke}%
\pgfsetstrokeopacity{0.750000}%
\pgfsetdash{}{0pt}%
\pgfpathmoveto{\pgfqpoint{3.777776in}{0.936371in}}%
\pgfpathlineto{\pgfqpoint{4.055553in}{0.936371in}}%
\pgfusepath{stroke}%
\end{pgfscope}%
\begin{pgfscope}%
\pgftext[x=4.166664in,y=0.887760in,left,base]{\rmfamily\fontsize{10.000000}{12.000000}\selectfont Drag force}%
\end{pgfscope}%
\begin{pgfscope}%
\pgfsetrectcap%
\pgfsetroundjoin%
\pgfsetlinewidth{1.505625pt}%
\definecolor{currentstroke}{rgb}{0.000000,0.750000,0.750000}%
\pgfsetstrokecolor{currentstroke}%
\pgfsetstrokeopacity{0.750000}%
\pgfsetdash{}{0pt}%
\pgfpathmoveto{\pgfqpoint{3.777776in}{0.730547in}}%
\pgfpathlineto{\pgfqpoint{4.055553in}{0.730547in}}%
\pgfusepath{stroke}%
\end{pgfscope}%
\begin{pgfscope}%
\pgftext[x=4.166664in,y=0.681936in,left,base]{\rmfamily\fontsize{10.000000}{12.000000}\selectfont Image force}%
\end{pgfscope}%
\end{pgfpicture}%
\makeatother%
\endgroup%
}
    \caption{Simulated forces acting on the drop. Experiments are shown by order of increasing apoapse.\label{fig:forces}}
\end{figure}

In the non-dimensional trajectories with short-time scaling shown in Figure \ref{fig:series_s_ds}, we see that the trajectory apoapses are consistently $\mathcal{O}(1)$, but most trajectories overshoot their characteristic time scale (which predicts returns at $\bar{t}  =2$ to the first order). We also observe that that $\mathbb{E}\mbox{u}$ is not typically a small number in this regime, imperiling our use of asymptotic estimates in this regime. We can perhaps gain some insight by comparing the asymptotic estimate for return times to the scaled experimental return times. We see in Figure \ref{fig:times} that the long-time scaled non-dimensional time of first bounce in the experiment $t_b / t_c$, compares poorly to the asymptotic estimate for returns $t_f$ in the limit of small $\mathbb{E}\mbox{u}_+$, this is due to the use on long-time scaling for drops with $y/L \ll 1$.   
\begin{figure}[H]
    \centering
    %% Creator: Matplotlib, PGF backend
%%
%% To include the figure in your LaTeX document, write
%%   \input{<filename>.pgf}
%%
%% Make sure the required packages are loaded in your preamble
%%   \usepackage{pgf}
%%
%% Figures using additional raster images can only be included by \input if
%% they are in the same directory as the main LaTeX file. For loading figures
%% from other directories you can use the `import` package
%%   \usepackage{import}
%% and then include the figures with
%%   \import{<path to file>}{<filename>.pgf}
%%
%% Matplotlib used the following preamble
%%   \usepackage{fontspec}
%%   \setmainfont{DejaVu Serif}
%%   \setsansfont{DejaVu Sans}
%%   \setmonofont{DejaVu Sans Mono}
%%
\begingroup%
\makeatletter%
\begin{pgfpicture}%
\pgfpathrectangle{\pgfpointorigin}{\pgfqpoint{5.606387in}{3.837899in}}%
\pgfusepath{use as bounding box, clip}%
\begin{pgfscope}%
\pgfsetbuttcap%
\pgfsetmiterjoin%
\definecolor{currentfill}{rgb}{1.000000,1.000000,1.000000}%
\pgfsetfillcolor{currentfill}%
\pgfsetlinewidth{0.000000pt}%
\definecolor{currentstroke}{rgb}{1.000000,1.000000,1.000000}%
\pgfsetstrokecolor{currentstroke}%
\pgfsetdash{}{0pt}%
\pgfpathmoveto{\pgfqpoint{0.000000in}{0.000000in}}%
\pgfpathlineto{\pgfqpoint{5.606387in}{0.000000in}}%
\pgfpathlineto{\pgfqpoint{5.606387in}{3.837899in}}%
\pgfpathlineto{\pgfqpoint{0.000000in}{3.837899in}}%
\pgfpathclose%
\pgfusepath{fill}%
\end{pgfscope}%
\begin{pgfscope}%
\pgfsetbuttcap%
\pgfsetmiterjoin%
\definecolor{currentfill}{rgb}{1.000000,1.000000,1.000000}%
\pgfsetfillcolor{currentfill}%
\pgfsetlinewidth{0.000000pt}%
\definecolor{currentstroke}{rgb}{0.000000,0.000000,0.000000}%
\pgfsetstrokecolor{currentstroke}%
\pgfsetstrokeopacity{0.000000}%
\pgfsetdash{}{0pt}%
\pgfpathmoveto{\pgfqpoint{0.754713in}{0.682899in}}%
\pgfpathlineto{\pgfqpoint{4.474713in}{0.682899in}}%
\pgfpathlineto{\pgfqpoint{4.474713in}{3.702899in}}%
\pgfpathlineto{\pgfqpoint{0.754713in}{3.702899in}}%
\pgfpathclose%
\pgfusepath{fill}%
\end{pgfscope}%
\begin{pgfscope}%
\pgfsetbuttcap%
\pgfsetroundjoin%
\definecolor{currentfill}{rgb}{0.000000,0.000000,0.000000}%
\pgfsetfillcolor{currentfill}%
\pgfsetlinewidth{0.803000pt}%
\definecolor{currentstroke}{rgb}{0.000000,0.000000,0.000000}%
\pgfsetstrokecolor{currentstroke}%
\pgfsetdash{}{0pt}%
\pgfsys@defobject{currentmarker}{\pgfqpoint{0.000000in}{-0.048611in}}{\pgfqpoint{0.000000in}{0.000000in}}{%
\pgfpathmoveto{\pgfqpoint{0.000000in}{0.000000in}}%
\pgfpathlineto{\pgfqpoint{0.000000in}{-0.048611in}}%
\pgfusepath{stroke,fill}%
}%
\begin{pgfscope}%
\pgfsys@transformshift{0.754713in}{0.682899in}%
\pgfsys@useobject{currentmarker}{}%
\end{pgfscope}%
\end{pgfscope}%
\begin{pgfscope}%
\pgftext[x=0.754713in,y=0.585677in,,top]{\rmfamily\fontsize{16.000000}{19.200000}\selectfont \(\displaystyle 0\)}%
\end{pgfscope}%
\begin{pgfscope}%
\pgfsetbuttcap%
\pgfsetroundjoin%
\definecolor{currentfill}{rgb}{0.000000,0.000000,0.000000}%
\pgfsetfillcolor{currentfill}%
\pgfsetlinewidth{0.803000pt}%
\definecolor{currentstroke}{rgb}{0.000000,0.000000,0.000000}%
\pgfsetstrokecolor{currentstroke}%
\pgfsetdash{}{0pt}%
\pgfsys@defobject{currentmarker}{\pgfqpoint{0.000000in}{-0.048611in}}{\pgfqpoint{0.000000in}{0.000000in}}{%
\pgfpathmoveto{\pgfqpoint{0.000000in}{0.000000in}}%
\pgfpathlineto{\pgfqpoint{0.000000in}{-0.048611in}}%
\pgfusepath{stroke,fill}%
}%
\begin{pgfscope}%
\pgfsys@transformshift{1.817570in}{0.682899in}%
\pgfsys@useobject{currentmarker}{}%
\end{pgfscope}%
\end{pgfscope}%
\begin{pgfscope}%
\pgftext[x=1.817570in,y=0.585677in,,top]{\rmfamily\fontsize{16.000000}{19.200000}\selectfont \(\displaystyle 1\)}%
\end{pgfscope}%
\begin{pgfscope}%
\pgfsetbuttcap%
\pgfsetroundjoin%
\definecolor{currentfill}{rgb}{0.000000,0.000000,0.000000}%
\pgfsetfillcolor{currentfill}%
\pgfsetlinewidth{0.803000pt}%
\definecolor{currentstroke}{rgb}{0.000000,0.000000,0.000000}%
\pgfsetstrokecolor{currentstroke}%
\pgfsetdash{}{0pt}%
\pgfsys@defobject{currentmarker}{\pgfqpoint{0.000000in}{-0.048611in}}{\pgfqpoint{0.000000in}{0.000000in}}{%
\pgfpathmoveto{\pgfqpoint{0.000000in}{0.000000in}}%
\pgfpathlineto{\pgfqpoint{0.000000in}{-0.048611in}}%
\pgfusepath{stroke,fill}%
}%
\begin{pgfscope}%
\pgfsys@transformshift{2.880427in}{0.682899in}%
\pgfsys@useobject{currentmarker}{}%
\end{pgfscope}%
\end{pgfscope}%
\begin{pgfscope}%
\pgftext[x=2.880427in,y=0.585677in,,top]{\rmfamily\fontsize{16.000000}{19.200000}\selectfont \(\displaystyle 2\)}%
\end{pgfscope}%
\begin{pgfscope}%
\pgfsetbuttcap%
\pgfsetroundjoin%
\definecolor{currentfill}{rgb}{0.000000,0.000000,0.000000}%
\pgfsetfillcolor{currentfill}%
\pgfsetlinewidth{0.803000pt}%
\definecolor{currentstroke}{rgb}{0.000000,0.000000,0.000000}%
\pgfsetstrokecolor{currentstroke}%
\pgfsetdash{}{0pt}%
\pgfsys@defobject{currentmarker}{\pgfqpoint{0.000000in}{-0.048611in}}{\pgfqpoint{0.000000in}{0.000000in}}{%
\pgfpathmoveto{\pgfqpoint{0.000000in}{0.000000in}}%
\pgfpathlineto{\pgfqpoint{0.000000in}{-0.048611in}}%
\pgfusepath{stroke,fill}%
}%
\begin{pgfscope}%
\pgfsys@transformshift{3.943284in}{0.682899in}%
\pgfsys@useobject{currentmarker}{}%
\end{pgfscope}%
\end{pgfscope}%
\begin{pgfscope}%
\pgftext[x=3.943284in,y=0.585677in,,top]{\rmfamily\fontsize{16.000000}{19.200000}\selectfont \(\displaystyle 3\)}%
\end{pgfscope}%
\begin{pgfscope}%
\pgftext[x=2.614713in,y=0.315061in,,top]{\rmfamily\fontsize{16.000000}{19.200000}\selectfont \(\displaystyle t^*\)}%
\end{pgfscope}%
\begin{pgfscope}%
\pgfsetbuttcap%
\pgfsetroundjoin%
\definecolor{currentfill}{rgb}{0.000000,0.000000,0.000000}%
\pgfsetfillcolor{currentfill}%
\pgfsetlinewidth{0.803000pt}%
\definecolor{currentstroke}{rgb}{0.000000,0.000000,0.000000}%
\pgfsetstrokecolor{currentstroke}%
\pgfsetdash{}{0pt}%
\pgfsys@defobject{currentmarker}{\pgfqpoint{-0.048611in}{0.000000in}}{\pgfqpoint{0.000000in}{0.000000in}}{%
\pgfpathmoveto{\pgfqpoint{0.000000in}{0.000000in}}%
\pgfpathlineto{\pgfqpoint{-0.048611in}{0.000000in}}%
\pgfusepath{stroke,fill}%
}%
\begin{pgfscope}%
\pgfsys@transformshift{0.754713in}{0.933165in}%
\pgfsys@useobject{currentmarker}{}%
\end{pgfscope}%
\end{pgfscope}%
\begin{pgfscope}%
\pgftext[x=0.372077in,y=0.848746in,left,base]{\rmfamily\fontsize{16.000000}{19.200000}\selectfont \(\displaystyle 0.0\)}%
\end{pgfscope}%
\begin{pgfscope}%
\pgfsetbuttcap%
\pgfsetroundjoin%
\definecolor{currentfill}{rgb}{0.000000,0.000000,0.000000}%
\pgfsetfillcolor{currentfill}%
\pgfsetlinewidth{0.803000pt}%
\definecolor{currentstroke}{rgb}{0.000000,0.000000,0.000000}%
\pgfsetstrokecolor{currentstroke}%
\pgfsetdash{}{0pt}%
\pgfsys@defobject{currentmarker}{\pgfqpoint{-0.048611in}{0.000000in}}{\pgfqpoint{0.000000in}{0.000000in}}{%
\pgfpathmoveto{\pgfqpoint{0.000000in}{0.000000in}}%
\pgfpathlineto{\pgfqpoint{-0.048611in}{0.000000in}}%
\pgfusepath{stroke,fill}%
}%
\begin{pgfscope}%
\pgfsys@transformshift{0.754713in}{1.830828in}%
\pgfsys@useobject{currentmarker}{}%
\end{pgfscope}%
\end{pgfscope}%
\begin{pgfscope}%
\pgftext[x=0.372077in,y=1.746410in,left,base]{\rmfamily\fontsize{16.000000}{19.200000}\selectfont \(\displaystyle 0.5\)}%
\end{pgfscope}%
\begin{pgfscope}%
\pgfsetbuttcap%
\pgfsetroundjoin%
\definecolor{currentfill}{rgb}{0.000000,0.000000,0.000000}%
\pgfsetfillcolor{currentfill}%
\pgfsetlinewidth{0.803000pt}%
\definecolor{currentstroke}{rgb}{0.000000,0.000000,0.000000}%
\pgfsetstrokecolor{currentstroke}%
\pgfsetdash{}{0pt}%
\pgfsys@defobject{currentmarker}{\pgfqpoint{-0.048611in}{0.000000in}}{\pgfqpoint{0.000000in}{0.000000in}}{%
\pgfpathmoveto{\pgfqpoint{0.000000in}{0.000000in}}%
\pgfpathlineto{\pgfqpoint{-0.048611in}{0.000000in}}%
\pgfusepath{stroke,fill}%
}%
\begin{pgfscope}%
\pgfsys@transformshift{0.754713in}{2.728492in}%
\pgfsys@useobject{currentmarker}{}%
\end{pgfscope}%
\end{pgfscope}%
\begin{pgfscope}%
\pgftext[x=0.372077in,y=2.644073in,left,base]{\rmfamily\fontsize{16.000000}{19.200000}\selectfont \(\displaystyle 1.0\)}%
\end{pgfscope}%
\begin{pgfscope}%
\pgfsetbuttcap%
\pgfsetroundjoin%
\definecolor{currentfill}{rgb}{0.000000,0.000000,0.000000}%
\pgfsetfillcolor{currentfill}%
\pgfsetlinewidth{0.803000pt}%
\definecolor{currentstroke}{rgb}{0.000000,0.000000,0.000000}%
\pgfsetstrokecolor{currentstroke}%
\pgfsetdash{}{0pt}%
\pgfsys@defobject{currentmarker}{\pgfqpoint{-0.048611in}{0.000000in}}{\pgfqpoint{0.000000in}{0.000000in}}{%
\pgfpathmoveto{\pgfqpoint{0.000000in}{0.000000in}}%
\pgfpathlineto{\pgfqpoint{-0.048611in}{0.000000in}}%
\pgfusepath{stroke,fill}%
}%
\begin{pgfscope}%
\pgfsys@transformshift{0.754713in}{3.626155in}%
\pgfsys@useobject{currentmarker}{}%
\end{pgfscope}%
\end{pgfscope}%
\begin{pgfscope}%
\pgftext[x=0.372077in,y=3.541737in,left,base]{\rmfamily\fontsize{16.000000}{19.200000}\selectfont \(\displaystyle 1.5\)}%
\end{pgfscope}%
\begin{pgfscope}%
\pgftext[x=0.316521in,y=2.192899in,,bottom,rotate=90.000000]{\rmfamily\fontsize{16.000000}{19.200000}\selectfont \(\displaystyle y^*\)}%
\end{pgfscope}%
\begin{pgfscope}%
\pgfpathrectangle{\pgfqpoint{0.754713in}{0.682899in}}{\pgfqpoint{3.720000in}{3.020000in}}%
\pgfusepath{clip}%
\pgfsetrectcap%
\pgfsetroundjoin%
\pgfsetlinewidth{1.505625pt}%
\definecolor{currentstroke}{rgb}{0.993248,0.906157,0.143936}%
\pgfsetstrokecolor{currentstroke}%
\pgfsetdash{}{0pt}%
\pgfpathmoveto{\pgfqpoint{1.526605in}{2.540456in}}%
\pgfpathlineto{\pgfqpoint{1.596777in}{2.665308in}}%
\pgfpathlineto{\pgfqpoint{1.666949in}{2.817882in}}%
\pgfpathlineto{\pgfqpoint{1.737121in}{2.841919in}}%
\pgfpathlineto{\pgfqpoint{1.807293in}{2.914832in}}%
\pgfpathlineto{\pgfqpoint{1.877465in}{3.038647in}}%
\pgfpathlineto{\pgfqpoint{1.947637in}{3.082049in}}%
\pgfpathlineto{\pgfqpoint{2.017809in}{3.138944in}}%
\pgfpathlineto{\pgfqpoint{2.087981in}{3.186031in}}%
\pgfpathlineto{\pgfqpoint{2.158153in}{3.225182in}}%
\pgfpathlineto{\pgfqpoint{2.228325in}{3.238547in}}%
\pgfpathlineto{\pgfqpoint{2.298497in}{3.244670in}}%
\pgfpathlineto{\pgfqpoint{2.368669in}{3.251294in}}%
\pgfpathlineto{\pgfqpoint{2.438841in}{3.230319in}}%
\pgfpathlineto{\pgfqpoint{2.509013in}{3.220226in}}%
\pgfpathlineto{\pgfqpoint{2.579185in}{3.182532in}}%
\pgfpathlineto{\pgfqpoint{2.649357in}{3.145323in}}%
\pgfpathlineto{\pgfqpoint{2.719529in}{3.086988in}}%
\pgfpathlineto{\pgfqpoint{2.789702in}{3.029451in}}%
\pgfpathlineto{\pgfqpoint{2.859874in}{2.949412in}}%
\pgfpathlineto{\pgfqpoint{2.930046in}{2.865500in}}%
\pgfpathlineto{\pgfqpoint{3.000218in}{2.770906in}}%
\pgfpathlineto{\pgfqpoint{3.070390in}{2.658771in}}%
\pgfpathlineto{\pgfqpoint{3.140562in}{2.541389in}}%
\pgfpathlineto{\pgfqpoint{3.210734in}{2.401807in}}%
\pgfpathlineto{\pgfqpoint{3.280906in}{2.264913in}}%
\pgfpathlineto{\pgfqpoint{3.351078in}{2.098262in}}%
\pgfpathlineto{\pgfqpoint{3.421250in}{1.920170in}}%
\pgfpathlineto{\pgfqpoint{3.491422in}{1.707511in}}%
\pgfpathlineto{\pgfqpoint{3.561594in}{1.493756in}}%
\pgfpathlineto{\pgfqpoint{3.631766in}{1.222808in}}%
\pgfpathlineto{\pgfqpoint{3.701938in}{1.018307in}}%
\pgfpathlineto{\pgfqpoint{3.772110in}{0.853977in}}%
\pgfpathlineto{\pgfqpoint{3.842282in}{0.869925in}}%
\pgfpathlineto{\pgfqpoint{3.912454in}{1.015648in}}%
\pgfpathlineto{\pgfqpoint{3.982626in}{1.219280in}}%
\pgfpathlineto{\pgfqpoint{4.052798in}{1.416638in}}%
\pgfpathlineto{\pgfqpoint{4.122970in}{1.565620in}}%
\pgfpathlineto{\pgfqpoint{4.193142in}{1.681545in}}%
\pgfpathlineto{\pgfqpoint{4.263314in}{1.773229in}}%
\pgfpathlineto{\pgfqpoint{4.333486in}{1.827833in}}%
\pgfpathlineto{\pgfqpoint{4.403658in}{1.874022in}}%
\pgfpathlineto{\pgfqpoint{4.473830in}{1.873142in}}%
\pgfpathlineto{\pgfqpoint{4.484713in}{1.870017in}}%
\pgfusepath{stroke}%
\end{pgfscope}%
\begin{pgfscope}%
\pgfpathrectangle{\pgfqpoint{0.754713in}{0.682899in}}{\pgfqpoint{3.720000in}{3.020000in}}%
\pgfusepath{clip}%
\pgfsetrectcap%
\pgfsetroundjoin%
\pgfsetlinewidth{1.505625pt}%
\definecolor{currentstroke}{rgb}{0.699415,0.867117,0.175971}%
\pgfsetstrokecolor{currentstroke}%
\pgfsetdash{}{0pt}%
\pgfpathmoveto{\pgfqpoint{1.005507in}{1.122611in}}%
\pgfpathlineto{\pgfqpoint{1.068206in}{1.371153in}}%
\pgfpathlineto{\pgfqpoint{1.130904in}{1.583661in}}%
\pgfpathlineto{\pgfqpoint{1.193603in}{1.736381in}}%
\pgfpathlineto{\pgfqpoint{1.256301in}{1.864552in}}%
\pgfpathlineto{\pgfqpoint{1.319000in}{2.008757in}}%
\pgfpathlineto{\pgfqpoint{1.381699in}{2.143646in}}%
\pgfpathlineto{\pgfqpoint{1.444397in}{2.340164in}}%
\pgfpathlineto{\pgfqpoint{1.507096in}{2.473534in}}%
\pgfpathlineto{\pgfqpoint{1.569795in}{2.551234in}}%
\pgfpathlineto{\pgfqpoint{1.632493in}{2.622196in}}%
\pgfpathlineto{\pgfqpoint{1.695192in}{2.674623in}}%
\pgfpathlineto{\pgfqpoint{1.757890in}{2.782897in}}%
\pgfpathlineto{\pgfqpoint{1.820589in}{2.860214in}}%
\pgfpathlineto{\pgfqpoint{1.883288in}{2.930209in}}%
\pgfpathlineto{\pgfqpoint{1.945986in}{2.922035in}}%
\pgfpathlineto{\pgfqpoint{2.008685in}{2.933715in}}%
\pgfpathlineto{\pgfqpoint{2.071384in}{2.988627in}}%
\pgfpathlineto{\pgfqpoint{2.134082in}{3.034831in}}%
\pgfpathlineto{\pgfqpoint{2.196781in}{3.093865in}}%
\pgfpathlineto{\pgfqpoint{2.259479in}{3.080572in}}%
\pgfpathlineto{\pgfqpoint{2.322178in}{3.040494in}}%
\pgfpathlineto{\pgfqpoint{2.384877in}{3.031214in}}%
\pgfpathlineto{\pgfqpoint{2.447575in}{3.054173in}}%
\pgfpathlineto{\pgfqpoint{2.510274in}{3.067809in}}%
\pgfpathlineto{\pgfqpoint{2.572972in}{2.999990in}}%
\pgfpathlineto{\pgfqpoint{2.635671in}{2.937788in}}%
\pgfpathlineto{\pgfqpoint{2.698370in}{2.873438in}}%
\pgfpathlineto{\pgfqpoint{2.761068in}{2.828260in}}%
\pgfpathlineto{\pgfqpoint{2.823767in}{2.827357in}}%
\pgfpathlineto{\pgfqpoint{2.886466in}{2.759907in}}%
\pgfpathlineto{\pgfqpoint{2.949164in}{2.654501in}}%
\pgfpathlineto{\pgfqpoint{3.011863in}{2.527271in}}%
\pgfpathlineto{\pgfqpoint{3.074561in}{2.465584in}}%
\pgfpathlineto{\pgfqpoint{3.137260in}{2.389752in}}%
\pgfpathlineto{\pgfqpoint{3.199959in}{2.281062in}}%
\pgfpathlineto{\pgfqpoint{3.262657in}{2.135655in}}%
\pgfpathlineto{\pgfqpoint{3.325356in}{1.955252in}}%
\pgfpathlineto{\pgfqpoint{3.388055in}{1.781112in}}%
\pgfpathlineto{\pgfqpoint{3.450753in}{1.632073in}}%
\pgfpathlineto{\pgfqpoint{3.513452in}{1.442164in}}%
\pgfpathlineto{\pgfqpoint{3.576150in}{1.237340in}}%
\pgfpathlineto{\pgfqpoint{3.638849in}{1.063732in}}%
\pgfpathlineto{\pgfqpoint{3.701548in}{0.930760in}}%
\pgfpathlineto{\pgfqpoint{3.764246in}{0.842899in}}%
\pgfpathlineto{\pgfqpoint{3.826945in}{0.820172in}}%
\pgfpathlineto{\pgfqpoint{3.889643in}{0.899559in}}%
\pgfpathlineto{\pgfqpoint{3.952342in}{1.028284in}}%
\pgfpathlineto{\pgfqpoint{4.077739in}{1.358084in}}%
\pgfpathlineto{\pgfqpoint{4.140438in}{1.539273in}}%
\pgfpathlineto{\pgfqpoint{4.203137in}{1.677699in}}%
\pgfpathlineto{\pgfqpoint{4.265835in}{1.825798in}}%
\pgfpathlineto{\pgfqpoint{4.328534in}{1.978689in}}%
\pgfpathlineto{\pgfqpoint{4.391232in}{2.116514in}}%
\pgfpathlineto{\pgfqpoint{4.453931in}{2.243527in}}%
\pgfpathlineto{\pgfqpoint{4.484713in}{2.280774in}}%
\pgfpathlineto{\pgfqpoint{4.484713in}{2.280774in}}%
\pgfusepath{stroke}%
\end{pgfscope}%
\begin{pgfscope}%
\pgfpathrectangle{\pgfqpoint{0.754713in}{0.682899in}}{\pgfqpoint{3.720000in}{3.020000in}}%
\pgfusepath{clip}%
\pgfsetrectcap%
\pgfsetroundjoin%
\pgfsetlinewidth{1.505625pt}%
\definecolor{currentstroke}{rgb}{0.636902,0.856542,0.216620}%
\pgfsetstrokecolor{currentstroke}%
\pgfsetdash{}{0pt}%
\pgfpathmoveto{\pgfqpoint{1.412821in}{2.742976in}}%
\pgfpathlineto{\pgfqpoint{1.485944in}{2.862986in}}%
\pgfpathlineto{\pgfqpoint{1.559068in}{2.963475in}}%
\pgfpathlineto{\pgfqpoint{1.632191in}{3.046579in}}%
\pgfpathlineto{\pgfqpoint{1.705314in}{3.165706in}}%
\pgfpathlineto{\pgfqpoint{1.778437in}{3.256081in}}%
\pgfpathlineto{\pgfqpoint{1.851560in}{3.301449in}}%
\pgfpathlineto{\pgfqpoint{1.924684in}{3.377384in}}%
\pgfpathlineto{\pgfqpoint{1.997807in}{3.477058in}}%
\pgfpathlineto{\pgfqpoint{2.070930in}{3.467665in}}%
\pgfpathlineto{\pgfqpoint{2.144053in}{3.505926in}}%
\pgfpathlineto{\pgfqpoint{2.217176in}{3.555161in}}%
\pgfpathlineto{\pgfqpoint{2.290299in}{3.545901in}}%
\pgfpathlineto{\pgfqpoint{2.363423in}{3.565626in}}%
\pgfpathlineto{\pgfqpoint{2.436546in}{3.565125in}}%
\pgfpathlineto{\pgfqpoint{2.509669in}{3.538410in}}%
\pgfpathlineto{\pgfqpoint{2.582792in}{3.521777in}}%
\pgfpathlineto{\pgfqpoint{2.655915in}{3.484502in}}%
\pgfpathlineto{\pgfqpoint{2.729039in}{3.436700in}}%
\pgfpathlineto{\pgfqpoint{2.802162in}{3.373611in}}%
\pgfpathlineto{\pgfqpoint{2.875285in}{3.298757in}}%
\pgfpathlineto{\pgfqpoint{2.948408in}{3.231401in}}%
\pgfpathlineto{\pgfqpoint{3.021531in}{3.133660in}}%
\pgfpathlineto{\pgfqpoint{3.094655in}{3.032529in}}%
\pgfpathlineto{\pgfqpoint{3.167778in}{2.926671in}}%
\pgfpathlineto{\pgfqpoint{3.240901in}{2.793674in}}%
\pgfpathlineto{\pgfqpoint{3.314024in}{2.653696in}}%
\pgfpathlineto{\pgfqpoint{3.387147in}{2.518312in}}%
\pgfpathlineto{\pgfqpoint{3.460271in}{2.346945in}}%
\pgfpathlineto{\pgfqpoint{3.606517in}{1.945130in}}%
\pgfpathlineto{\pgfqpoint{3.679640in}{1.669486in}}%
\pgfpathlineto{\pgfqpoint{3.752763in}{1.381765in}}%
\pgfpathlineto{\pgfqpoint{3.825886in}{1.106136in}}%
\pgfpathlineto{\pgfqpoint{3.899010in}{0.955622in}}%
\pgfpathlineto{\pgfqpoint{3.972133in}{1.013819in}}%
\pgfpathlineto{\pgfqpoint{4.045256in}{1.233426in}}%
\pgfpathlineto{\pgfqpoint{4.118379in}{1.485455in}}%
\pgfpathlineto{\pgfqpoint{4.191502in}{1.713727in}}%
\pgfpathlineto{\pgfqpoint{4.264626in}{1.869282in}}%
\pgfpathlineto{\pgfqpoint{4.337749in}{1.982193in}}%
\pgfpathlineto{\pgfqpoint{4.410872in}{2.113632in}}%
\pgfpathlineto{\pgfqpoint{4.484713in}{2.177766in}}%
\pgfpathlineto{\pgfqpoint{4.484713in}{2.177766in}}%
\pgfusepath{stroke}%
\end{pgfscope}%
\begin{pgfscope}%
\pgfpathrectangle{\pgfqpoint{0.754713in}{0.682899in}}{\pgfqpoint{3.720000in}{3.020000in}}%
\pgfusepath{clip}%
\pgfsetrectcap%
\pgfsetroundjoin%
\pgfsetlinewidth{1.505625pt}%
\definecolor{currentstroke}{rgb}{0.335885,0.777018,0.402049}%
\pgfsetstrokecolor{currentstroke}%
\pgfsetdash{}{0pt}%
\pgfpathmoveto{\pgfqpoint{1.289140in}{1.788140in}}%
\pgfpathlineto{\pgfqpoint{1.355944in}{1.917285in}}%
\pgfpathlineto{\pgfqpoint{1.422747in}{2.019686in}}%
\pgfpathlineto{\pgfqpoint{1.489550in}{2.086154in}}%
\pgfpathlineto{\pgfqpoint{1.556354in}{2.179740in}}%
\pgfpathlineto{\pgfqpoint{1.623157in}{2.270406in}}%
\pgfpathlineto{\pgfqpoint{1.689961in}{2.313024in}}%
\pgfpathlineto{\pgfqpoint{1.756764in}{2.385985in}}%
\pgfpathlineto{\pgfqpoint{1.823568in}{2.436603in}}%
\pgfpathlineto{\pgfqpoint{1.890371in}{2.459970in}}%
\pgfpathlineto{\pgfqpoint{1.957175in}{2.514232in}}%
\pgfpathlineto{\pgfqpoint{2.023978in}{2.531798in}}%
\pgfpathlineto{\pgfqpoint{2.090782in}{2.527677in}}%
\pgfpathlineto{\pgfqpoint{2.157585in}{2.566682in}}%
\pgfpathlineto{\pgfqpoint{2.224388in}{2.560349in}}%
\pgfpathlineto{\pgfqpoint{2.291192in}{2.537716in}}%
\pgfpathlineto{\pgfqpoint{2.357995in}{2.548779in}}%
\pgfpathlineto{\pgfqpoint{2.491602in}{2.478044in}}%
\pgfpathlineto{\pgfqpoint{2.558406in}{2.459545in}}%
\pgfpathlineto{\pgfqpoint{2.625209in}{2.401991in}}%
\pgfpathlineto{\pgfqpoint{2.692013in}{2.339925in}}%
\pgfpathlineto{\pgfqpoint{2.758816in}{2.298901in}}%
\pgfpathlineto{\pgfqpoint{2.825619in}{2.211102in}}%
\pgfpathlineto{\pgfqpoint{2.892423in}{2.137781in}}%
\pgfpathlineto{\pgfqpoint{2.959226in}{2.067113in}}%
\pgfpathlineto{\pgfqpoint{3.092833in}{1.846593in}}%
\pgfpathlineto{\pgfqpoint{3.159637in}{1.748945in}}%
\pgfpathlineto{\pgfqpoint{3.226440in}{1.604854in}}%
\pgfpathlineto{\pgfqpoint{3.293244in}{1.472373in}}%
\pgfpathlineto{\pgfqpoint{3.360047in}{1.314418in}}%
\pgfpathlineto{\pgfqpoint{3.426851in}{1.145090in}}%
\pgfpathlineto{\pgfqpoint{3.493654in}{1.028822in}}%
\pgfpathlineto{\pgfqpoint{3.560457in}{0.890692in}}%
\pgfpathlineto{\pgfqpoint{3.627261in}{0.891872in}}%
\pgfpathlineto{\pgfqpoint{3.694064in}{0.942776in}}%
\pgfpathlineto{\pgfqpoint{3.760868in}{1.070935in}}%
\pgfpathlineto{\pgfqpoint{3.827671in}{1.220553in}}%
\pgfpathlineto{\pgfqpoint{3.894475in}{1.350079in}}%
\pgfpathlineto{\pgfqpoint{3.961278in}{1.443507in}}%
\pgfpathlineto{\pgfqpoint{4.028082in}{1.547237in}}%
\pgfpathlineto{\pgfqpoint{4.094885in}{1.627096in}}%
\pgfpathlineto{\pgfqpoint{4.161688in}{1.645603in}}%
\pgfpathlineto{\pgfqpoint{4.228492in}{1.727829in}}%
\pgfpathlineto{\pgfqpoint{4.295295in}{1.766065in}}%
\pgfpathlineto{\pgfqpoint{4.362099in}{1.753077in}}%
\pgfpathlineto{\pgfqpoint{4.428902in}{1.789050in}}%
\pgfpathlineto{\pgfqpoint{4.484713in}{1.802050in}}%
\pgfpathlineto{\pgfqpoint{4.484713in}{1.802050in}}%
\pgfusepath{stroke}%
\end{pgfscope}%
\begin{pgfscope}%
\pgfpathrectangle{\pgfqpoint{0.754713in}{0.682899in}}{\pgfqpoint{3.720000in}{3.020000in}}%
\pgfusepath{clip}%
\pgfsetrectcap%
\pgfsetroundjoin%
\pgfsetlinewidth{1.505625pt}%
\definecolor{currentstroke}{rgb}{0.122606,0.585371,0.546557}%
\pgfsetstrokecolor{currentstroke}%
\pgfsetdash{}{0pt}%
\pgfpathmoveto{\pgfqpoint{1.210064in}{2.041114in}}%
\pgfpathlineto{\pgfqpoint{1.255599in}{2.126493in}}%
\pgfpathlineto{\pgfqpoint{1.301134in}{2.186878in}}%
\pgfpathlineto{\pgfqpoint{1.346669in}{2.247201in}}%
\pgfpathlineto{\pgfqpoint{1.392204in}{2.326849in}}%
\pgfpathlineto{\pgfqpoint{1.437739in}{2.380003in}}%
\pgfpathlineto{\pgfqpoint{1.483274in}{2.431756in}}%
\pgfpathlineto{\pgfqpoint{1.528810in}{2.500981in}}%
\pgfpathlineto{\pgfqpoint{1.574345in}{2.543991in}}%
\pgfpathlineto{\pgfqpoint{1.619880in}{2.592931in}}%
\pgfpathlineto{\pgfqpoint{1.665415in}{2.644533in}}%
\pgfpathlineto{\pgfqpoint{1.710950in}{2.676291in}}%
\pgfpathlineto{\pgfqpoint{1.756485in}{2.724918in}}%
\pgfpathlineto{\pgfqpoint{1.802020in}{2.757566in}}%
\pgfpathlineto{\pgfqpoint{1.847555in}{2.789444in}}%
\pgfpathlineto{\pgfqpoint{1.893090in}{2.825149in}}%
\pgfpathlineto{\pgfqpoint{1.938626in}{2.843228in}}%
\pgfpathlineto{\pgfqpoint{1.984161in}{2.872786in}}%
\pgfpathlineto{\pgfqpoint{2.029696in}{2.890440in}}%
\pgfpathlineto{\pgfqpoint{2.075231in}{2.904761in}}%
\pgfpathlineto{\pgfqpoint{2.120766in}{2.928425in}}%
\pgfpathlineto{\pgfqpoint{2.166301in}{2.931459in}}%
\pgfpathlineto{\pgfqpoint{2.211836in}{2.943441in}}%
\pgfpathlineto{\pgfqpoint{2.257371in}{2.949398in}}%
\pgfpathlineto{\pgfqpoint{2.302907in}{2.946453in}}%
\pgfpathlineto{\pgfqpoint{2.348442in}{2.954793in}}%
\pgfpathlineto{\pgfqpoint{2.393977in}{2.945397in}}%
\pgfpathlineto{\pgfqpoint{2.439512in}{2.940366in}}%
\pgfpathlineto{\pgfqpoint{2.485047in}{2.932110in}}%
\pgfpathlineto{\pgfqpoint{2.530582in}{2.914650in}}%
\pgfpathlineto{\pgfqpoint{2.576117in}{2.907126in}}%
\pgfpathlineto{\pgfqpoint{2.621652in}{2.883169in}}%
\pgfpathlineto{\pgfqpoint{2.667187in}{2.860990in}}%
\pgfpathlineto{\pgfqpoint{2.712723in}{2.840137in}}%
\pgfpathlineto{\pgfqpoint{2.758258in}{2.806159in}}%
\pgfpathlineto{\pgfqpoint{2.803793in}{2.780714in}}%
\pgfpathlineto{\pgfqpoint{2.849328in}{2.743544in}}%
\pgfpathlineto{\pgfqpoint{2.894863in}{2.703260in}}%
\pgfpathlineto{\pgfqpoint{2.940398in}{2.667485in}}%
\pgfpathlineto{\pgfqpoint{2.985933in}{2.616876in}}%
\pgfpathlineto{\pgfqpoint{3.031468in}{2.575302in}}%
\pgfpathlineto{\pgfqpoint{3.077004in}{2.521629in}}%
\pgfpathlineto{\pgfqpoint{3.122539in}{2.464028in}}%
\pgfpathlineto{\pgfqpoint{3.168074in}{2.410573in}}%
\pgfpathlineto{\pgfqpoint{3.213609in}{2.344589in}}%
\pgfpathlineto{\pgfqpoint{3.259144in}{2.284588in}}%
\pgfpathlineto{\pgfqpoint{3.304679in}{2.214609in}}%
\pgfpathlineto{\pgfqpoint{3.350214in}{2.137290in}}%
\pgfpathlineto{\pgfqpoint{3.395749in}{2.064979in}}%
\pgfpathlineto{\pgfqpoint{3.441285in}{1.978347in}}%
\pgfpathlineto{\pgfqpoint{3.486820in}{1.895585in}}%
\pgfpathlineto{\pgfqpoint{3.532355in}{1.802502in}}%
\pgfpathlineto{\pgfqpoint{3.577890in}{1.694714in}}%
\pgfpathlineto{\pgfqpoint{3.623425in}{1.586464in}}%
\pgfpathlineto{\pgfqpoint{3.668960in}{1.471758in}}%
\pgfpathlineto{\pgfqpoint{3.714495in}{1.403724in}}%
\pgfpathlineto{\pgfqpoint{3.760030in}{1.333833in}}%
\pgfpathlineto{\pgfqpoint{3.805565in}{1.362562in}}%
\pgfpathlineto{\pgfqpoint{3.851101in}{1.415907in}}%
\pgfpathlineto{\pgfqpoint{3.896636in}{1.497923in}}%
\pgfpathlineto{\pgfqpoint{3.942171in}{1.596677in}}%
\pgfpathlineto{\pgfqpoint{3.987706in}{1.677273in}}%
\pgfpathlineto{\pgfqpoint{4.033241in}{1.736359in}}%
\pgfpathlineto{\pgfqpoint{4.078776in}{1.792979in}}%
\pgfpathlineto{\pgfqpoint{4.124311in}{1.828230in}}%
\pgfpathlineto{\pgfqpoint{4.169846in}{1.861568in}}%
\pgfpathlineto{\pgfqpoint{4.215382in}{1.897358in}}%
\pgfpathlineto{\pgfqpoint{4.260917in}{1.912466in}}%
\pgfpathlineto{\pgfqpoint{4.306452in}{1.929575in}}%
\pgfpathlineto{\pgfqpoint{4.351987in}{1.932783in}}%
\pgfpathlineto{\pgfqpoint{4.397522in}{1.926322in}}%
\pgfpathlineto{\pgfqpoint{4.443057in}{1.921874in}}%
\pgfpathlineto{\pgfqpoint{4.484713in}{1.902623in}}%
\pgfusepath{stroke}%
\end{pgfscope}%
\begin{pgfscope}%
\pgfpathrectangle{\pgfqpoint{0.754713in}{0.682899in}}{\pgfqpoint{3.720000in}{3.020000in}}%
\pgfusepath{clip}%
\pgfsetrectcap%
\pgfsetroundjoin%
\pgfsetlinewidth{1.505625pt}%
\definecolor{currentstroke}{rgb}{0.129933,0.559582,0.551864}%
\pgfsetstrokecolor{currentstroke}%
\pgfsetdash{}{0pt}%
\pgfpathmoveto{\pgfqpoint{1.156632in}{1.493148in}}%
\pgfpathlineto{\pgfqpoint{1.201289in}{1.539133in}}%
\pgfpathlineto{\pgfqpoint{1.245947in}{1.617447in}}%
\pgfpathlineto{\pgfqpoint{1.290605in}{1.698465in}}%
\pgfpathlineto{\pgfqpoint{1.335262in}{1.760049in}}%
\pgfpathlineto{\pgfqpoint{1.379920in}{1.802384in}}%
\pgfpathlineto{\pgfqpoint{1.424578in}{1.866192in}}%
\pgfpathlineto{\pgfqpoint{1.469235in}{1.935707in}}%
\pgfpathlineto{\pgfqpoint{1.513893in}{1.977048in}}%
\pgfpathlineto{\pgfqpoint{1.558551in}{2.015396in}}%
\pgfpathlineto{\pgfqpoint{1.603208in}{2.073756in}}%
\pgfpathlineto{\pgfqpoint{1.647866in}{2.122927in}}%
\pgfpathlineto{\pgfqpoint{1.692524in}{2.148056in}}%
\pgfpathlineto{\pgfqpoint{1.737181in}{2.178228in}}%
\pgfpathlineto{\pgfqpoint{1.781839in}{2.225857in}}%
\pgfpathlineto{\pgfqpoint{1.826497in}{2.258068in}}%
\pgfpathlineto{\pgfqpoint{1.871154in}{2.272246in}}%
\pgfpathlineto{\pgfqpoint{1.915812in}{2.303287in}}%
\pgfpathlineto{\pgfqpoint{1.960470in}{2.338091in}}%
\pgfpathlineto{\pgfqpoint{2.005127in}{2.352945in}}%
\pgfpathlineto{\pgfqpoint{2.049785in}{2.357238in}}%
\pgfpathlineto{\pgfqpoint{2.139100in}{2.403310in}}%
\pgfpathlineto{\pgfqpoint{2.183758in}{2.402491in}}%
\pgfpathlineto{\pgfqpoint{2.228416in}{2.406252in}}%
\pgfpathlineto{\pgfqpoint{2.273073in}{2.423190in}}%
\pgfpathlineto{\pgfqpoint{2.317731in}{2.429892in}}%
\pgfpathlineto{\pgfqpoint{2.362389in}{2.417316in}}%
\pgfpathlineto{\pgfqpoint{2.407047in}{2.413576in}}%
\pgfpathlineto{\pgfqpoint{2.451704in}{2.420560in}}%
\pgfpathlineto{\pgfqpoint{2.496362in}{2.410050in}}%
\pgfpathlineto{\pgfqpoint{2.541020in}{2.390118in}}%
\pgfpathlineto{\pgfqpoint{2.585677in}{2.383808in}}%
\pgfpathlineto{\pgfqpoint{2.630335in}{2.378628in}}%
\pgfpathlineto{\pgfqpoint{2.674993in}{2.351402in}}%
\pgfpathlineto{\pgfqpoint{2.719650in}{2.322495in}}%
\pgfpathlineto{\pgfqpoint{2.764308in}{2.307989in}}%
\pgfpathlineto{\pgfqpoint{2.808966in}{2.287267in}}%
\pgfpathlineto{\pgfqpoint{2.853623in}{2.247226in}}%
\pgfpathlineto{\pgfqpoint{2.898281in}{2.215885in}}%
\pgfpathlineto{\pgfqpoint{2.942939in}{2.192971in}}%
\pgfpathlineto{\pgfqpoint{2.987596in}{2.154496in}}%
\pgfpathlineto{\pgfqpoint{3.032254in}{2.102566in}}%
\pgfpathlineto{\pgfqpoint{3.076912in}{2.063105in}}%
\pgfpathlineto{\pgfqpoint{3.121569in}{2.025159in}}%
\pgfpathlineto{\pgfqpoint{3.210885in}{1.910846in}}%
\pgfpathlineto{\pgfqpoint{3.255542in}{1.863306in}}%
\pgfpathlineto{\pgfqpoint{3.300200in}{1.808722in}}%
\pgfpathlineto{\pgfqpoint{3.344858in}{1.735306in}}%
\pgfpathlineto{\pgfqpoint{3.389515in}{1.665257in}}%
\pgfpathlineto{\pgfqpoint{3.434173in}{1.602569in}}%
\pgfpathlineto{\pgfqpoint{3.478831in}{1.527547in}}%
\pgfpathlineto{\pgfqpoint{3.612804in}{1.264238in}}%
\pgfpathlineto{\pgfqpoint{3.657461in}{1.147990in}}%
\pgfpathlineto{\pgfqpoint{3.702119in}{1.048913in}}%
\pgfpathlineto{\pgfqpoint{3.746777in}{0.967262in}}%
\pgfpathlineto{\pgfqpoint{3.791434in}{0.894358in}}%
\pgfpathlineto{\pgfqpoint{3.836092in}{0.894884in}}%
\pgfpathlineto{\pgfqpoint{3.880750in}{0.921169in}}%
\pgfpathlineto{\pgfqpoint{3.925407in}{1.006837in}}%
\pgfpathlineto{\pgfqpoint{3.970065in}{1.081196in}}%
\pgfpathlineto{\pgfqpoint{4.014723in}{1.164549in}}%
\pgfpathlineto{\pgfqpoint{4.059381in}{1.265393in}}%
\pgfpathlineto{\pgfqpoint{4.104038in}{1.360280in}}%
\pgfpathlineto{\pgfqpoint{4.148696in}{1.436234in}}%
\pgfpathlineto{\pgfqpoint{4.193354in}{1.497367in}}%
\pgfpathlineto{\pgfqpoint{4.282669in}{1.604668in}}%
\pgfpathlineto{\pgfqpoint{4.327327in}{1.654161in}}%
\pgfpathlineto{\pgfqpoint{4.416642in}{1.732159in}}%
\pgfpathlineto{\pgfqpoint{4.461300in}{1.764655in}}%
\pgfpathlineto{\pgfqpoint{4.484713in}{1.780032in}}%
\pgfpathlineto{\pgfqpoint{4.484713in}{1.780032in}}%
\pgfusepath{stroke}%
\end{pgfscope}%
\begin{pgfscope}%
\pgfpathrectangle{\pgfqpoint{0.754713in}{0.682899in}}{\pgfqpoint{3.720000in}{3.020000in}}%
\pgfusepath{clip}%
\pgfsetrectcap%
\pgfsetroundjoin%
\pgfsetlinewidth{1.505625pt}%
\definecolor{currentstroke}{rgb}{0.192357,0.403199,0.555836}%
\pgfsetstrokecolor{currentstroke}%
\pgfsetdash{}{0pt}%
\pgfpathmoveto{\pgfqpoint{1.107813in}{1.491525in}}%
\pgfpathlineto{\pgfqpoint{1.151950in}{1.564134in}}%
\pgfpathlineto{\pgfqpoint{1.196088in}{1.620209in}}%
\pgfpathlineto{\pgfqpoint{1.240225in}{1.699304in}}%
\pgfpathlineto{\pgfqpoint{1.328500in}{1.821472in}}%
\pgfpathlineto{\pgfqpoint{1.372638in}{1.875380in}}%
\pgfpathlineto{\pgfqpoint{1.416775in}{1.938685in}}%
\pgfpathlineto{\pgfqpoint{1.460913in}{1.985363in}}%
\pgfpathlineto{\pgfqpoint{1.505050in}{2.041424in}}%
\pgfpathlineto{\pgfqpoint{1.549188in}{2.082394in}}%
\pgfpathlineto{\pgfqpoint{1.593325in}{2.131038in}}%
\pgfpathlineto{\pgfqpoint{1.637463in}{2.169230in}}%
\pgfpathlineto{\pgfqpoint{1.681600in}{2.210368in}}%
\pgfpathlineto{\pgfqpoint{1.725738in}{2.242567in}}%
\pgfpathlineto{\pgfqpoint{1.769875in}{2.280484in}}%
\pgfpathlineto{\pgfqpoint{1.814013in}{2.306282in}}%
\pgfpathlineto{\pgfqpoint{1.858150in}{2.339004in}}%
\pgfpathlineto{\pgfqpoint{1.902288in}{2.360698in}}%
\pgfpathlineto{\pgfqpoint{1.946425in}{2.388125in}}%
\pgfpathlineto{\pgfqpoint{1.990563in}{2.406323in}}%
\pgfpathlineto{\pgfqpoint{2.034700in}{2.428472in}}%
\pgfpathlineto{\pgfqpoint{2.078838in}{2.441495in}}%
\pgfpathlineto{\pgfqpoint{2.122975in}{2.458371in}}%
\pgfpathlineto{\pgfqpoint{2.167113in}{2.466987in}}%
\pgfpathlineto{\pgfqpoint{2.211251in}{2.479218in}}%
\pgfpathlineto{\pgfqpoint{2.255388in}{2.483119in}}%
\pgfpathlineto{\pgfqpoint{2.299526in}{2.490789in}}%
\pgfpathlineto{\pgfqpoint{2.343663in}{2.490365in}}%
\pgfpathlineto{\pgfqpoint{2.387801in}{2.491859in}}%
\pgfpathlineto{\pgfqpoint{2.431938in}{2.486728in}}%
\pgfpathlineto{\pgfqpoint{2.476076in}{2.484550in}}%
\pgfpathlineto{\pgfqpoint{2.520213in}{2.474764in}}%
\pgfpathlineto{\pgfqpoint{2.564351in}{2.467465in}}%
\pgfpathlineto{\pgfqpoint{2.608488in}{2.453227in}}%
\pgfpathlineto{\pgfqpoint{2.652626in}{2.441374in}}%
\pgfpathlineto{\pgfqpoint{2.696763in}{2.422071in}}%
\pgfpathlineto{\pgfqpoint{2.740901in}{2.404905in}}%
\pgfpathlineto{\pgfqpoint{2.785038in}{2.381198in}}%
\pgfpathlineto{\pgfqpoint{2.829176in}{2.359499in}}%
\pgfpathlineto{\pgfqpoint{2.873313in}{2.330524in}}%
\pgfpathlineto{\pgfqpoint{2.917451in}{2.303663in}}%
\pgfpathlineto{\pgfqpoint{3.005726in}{2.238625in}}%
\pgfpathlineto{\pgfqpoint{3.094001in}{2.162599in}}%
\pgfpathlineto{\pgfqpoint{3.138138in}{2.120650in}}%
\pgfpathlineto{\pgfqpoint{3.182276in}{2.076737in}}%
\pgfpathlineto{\pgfqpoint{3.226413in}{2.029277in}}%
\pgfpathlineto{\pgfqpoint{3.270551in}{1.980048in}}%
\pgfpathlineto{\pgfqpoint{3.314688in}{1.927228in}}%
\pgfpathlineto{\pgfqpoint{3.358826in}{1.871820in}}%
\pgfpathlineto{\pgfqpoint{3.447101in}{1.750963in}}%
\pgfpathlineto{\pgfqpoint{3.491238in}{1.686212in}}%
\pgfpathlineto{\pgfqpoint{3.535376in}{1.618502in}}%
\pgfpathlineto{\pgfqpoint{3.579513in}{1.546449in}}%
\pgfpathlineto{\pgfqpoint{3.623651in}{1.471732in}}%
\pgfpathlineto{\pgfqpoint{3.667789in}{1.392300in}}%
\pgfpathlineto{\pgfqpoint{3.711926in}{1.306109in}}%
\pgfpathlineto{\pgfqpoint{3.756064in}{1.209419in}}%
\pgfpathlineto{\pgfqpoint{3.844339in}{1.007562in}}%
\pgfpathlineto{\pgfqpoint{3.888476in}{0.935297in}}%
\pgfpathlineto{\pgfqpoint{3.932614in}{0.912801in}}%
\pgfpathlineto{\pgfqpoint{3.976751in}{0.958649in}}%
\pgfpathlineto{\pgfqpoint{4.020889in}{1.044428in}}%
\pgfpathlineto{\pgfqpoint{4.065026in}{1.140789in}}%
\pgfpathlineto{\pgfqpoint{4.153301in}{1.303261in}}%
\pgfpathlineto{\pgfqpoint{4.241576in}{1.433018in}}%
\pgfpathlineto{\pgfqpoint{4.329851in}{1.543138in}}%
\pgfpathlineto{\pgfqpoint{4.418126in}{1.639662in}}%
\pgfpathlineto{\pgfqpoint{4.484713in}{1.701058in}}%
\pgfpathlineto{\pgfqpoint{4.484713in}{1.701058in}}%
\pgfusepath{stroke}%
\end{pgfscope}%
\begin{pgfscope}%
\pgfpathrectangle{\pgfqpoint{0.754713in}{0.682899in}}{\pgfqpoint{3.720000in}{3.020000in}}%
\pgfusepath{clip}%
\pgfsetrectcap%
\pgfsetroundjoin%
\pgfsetlinewidth{1.505625pt}%
\definecolor{currentstroke}{rgb}{0.201239,0.383670,0.554294}%
\pgfsetstrokecolor{currentstroke}%
\pgfsetdash{}{0pt}%
\pgfpathmoveto{\pgfqpoint{0.844456in}{0.972025in}}%
\pgfpathlineto{\pgfqpoint{0.874371in}{1.067048in}}%
\pgfpathlineto{\pgfqpoint{0.904286in}{1.128192in}}%
\pgfpathlineto{\pgfqpoint{0.934200in}{1.168233in}}%
\pgfpathlineto{\pgfqpoint{0.964115in}{1.216930in}}%
\pgfpathlineto{\pgfqpoint{0.994030in}{1.290947in}}%
\pgfpathlineto{\pgfqpoint{1.023944in}{1.353557in}}%
\pgfpathlineto{\pgfqpoint{1.083774in}{1.434567in}}%
\pgfpathlineto{\pgfqpoint{1.113688in}{1.490544in}}%
\pgfpathlineto{\pgfqpoint{1.143603in}{1.528376in}}%
\pgfpathlineto{\pgfqpoint{1.173518in}{1.578386in}}%
\pgfpathlineto{\pgfqpoint{1.203432in}{1.613820in}}%
\pgfpathlineto{\pgfqpoint{1.233347in}{1.642253in}}%
\pgfpathlineto{\pgfqpoint{1.263262in}{1.681827in}}%
\pgfpathlineto{\pgfqpoint{1.293176in}{1.732063in}}%
\pgfpathlineto{\pgfqpoint{1.323091in}{1.772794in}}%
\pgfpathlineto{\pgfqpoint{1.353006in}{1.799500in}}%
\pgfpathlineto{\pgfqpoint{1.382920in}{1.828350in}}%
\pgfpathlineto{\pgfqpoint{1.412835in}{1.867490in}}%
\pgfpathlineto{\pgfqpoint{1.442750in}{1.909786in}}%
\pgfpathlineto{\pgfqpoint{1.472664in}{1.936075in}}%
\pgfpathlineto{\pgfqpoint{1.502579in}{1.956450in}}%
\pgfpathlineto{\pgfqpoint{1.532494in}{1.979926in}}%
\pgfpathlineto{\pgfqpoint{1.562408in}{2.014665in}}%
\pgfpathlineto{\pgfqpoint{1.592323in}{2.051555in}}%
\pgfpathlineto{\pgfqpoint{1.622237in}{2.072315in}}%
\pgfpathlineto{\pgfqpoint{1.652152in}{2.087043in}}%
\pgfpathlineto{\pgfqpoint{1.682067in}{2.112664in}}%
\pgfpathlineto{\pgfqpoint{1.711981in}{2.144611in}}%
\pgfpathlineto{\pgfqpoint{1.741896in}{2.168314in}}%
\pgfpathlineto{\pgfqpoint{1.771811in}{2.182412in}}%
\pgfpathlineto{\pgfqpoint{1.801725in}{2.194101in}}%
\pgfpathlineto{\pgfqpoint{1.831640in}{2.216523in}}%
\pgfpathlineto{\pgfqpoint{1.861555in}{2.243618in}}%
\pgfpathlineto{\pgfqpoint{1.891469in}{2.262391in}}%
\pgfpathlineto{\pgfqpoint{1.921384in}{2.268354in}}%
\pgfpathlineto{\pgfqpoint{1.951299in}{2.282678in}}%
\pgfpathlineto{\pgfqpoint{1.981213in}{2.305140in}}%
\pgfpathlineto{\pgfqpoint{2.011128in}{2.323131in}}%
\pgfpathlineto{\pgfqpoint{2.041043in}{2.332299in}}%
\pgfpathlineto{\pgfqpoint{2.070957in}{2.336567in}}%
\pgfpathlineto{\pgfqpoint{2.100872in}{2.347173in}}%
\pgfpathlineto{\pgfqpoint{2.130787in}{2.363800in}}%
\pgfpathlineto{\pgfqpoint{2.160701in}{2.378097in}}%
\pgfpathlineto{\pgfqpoint{2.190616in}{2.379066in}}%
\pgfpathlineto{\pgfqpoint{2.220531in}{2.382763in}}%
\pgfpathlineto{\pgfqpoint{2.280360in}{2.406846in}}%
\pgfpathlineto{\pgfqpoint{2.310275in}{2.411492in}}%
\pgfpathlineto{\pgfqpoint{2.340189in}{2.409220in}}%
\pgfpathlineto{\pgfqpoint{2.370104in}{2.409825in}}%
\pgfpathlineto{\pgfqpoint{2.429933in}{2.425728in}}%
\pgfpathlineto{\pgfqpoint{2.489762in}{2.417767in}}%
\pgfpathlineto{\pgfqpoint{2.519677in}{2.419663in}}%
\pgfpathlineto{\pgfqpoint{2.549592in}{2.424505in}}%
\pgfpathlineto{\pgfqpoint{2.579506in}{2.424256in}}%
\pgfpathlineto{\pgfqpoint{2.639336in}{2.408932in}}%
\pgfpathlineto{\pgfqpoint{2.669250in}{2.406726in}}%
\pgfpathlineto{\pgfqpoint{2.699165in}{2.408183in}}%
\pgfpathlineto{\pgfqpoint{2.729080in}{2.400477in}}%
\pgfpathlineto{\pgfqpoint{2.758994in}{2.387800in}}%
\pgfpathlineto{\pgfqpoint{2.788909in}{2.380394in}}%
\pgfpathlineto{\pgfqpoint{2.818824in}{2.377489in}}%
\pgfpathlineto{\pgfqpoint{2.848738in}{2.371214in}}%
\pgfpathlineto{\pgfqpoint{2.878653in}{2.358256in}}%
\pgfpathlineto{\pgfqpoint{2.908568in}{2.342972in}}%
\pgfpathlineto{\pgfqpoint{2.938482in}{2.331003in}}%
\pgfpathlineto{\pgfqpoint{2.968397in}{2.324253in}}%
\pgfpathlineto{\pgfqpoint{2.998312in}{2.311406in}}%
\pgfpathlineto{\pgfqpoint{3.028226in}{2.292175in}}%
\pgfpathlineto{\pgfqpoint{3.058141in}{2.275924in}}%
\pgfpathlineto{\pgfqpoint{3.117970in}{2.250717in}}%
\pgfpathlineto{\pgfqpoint{3.147885in}{2.231697in}}%
\pgfpathlineto{\pgfqpoint{3.177799in}{2.208913in}}%
\pgfpathlineto{\pgfqpoint{3.207714in}{2.187991in}}%
\pgfpathlineto{\pgfqpoint{3.237629in}{2.172739in}}%
\pgfpathlineto{\pgfqpoint{3.267543in}{2.153423in}}%
\pgfpathlineto{\pgfqpoint{3.297458in}{2.127367in}}%
\pgfpathlineto{\pgfqpoint{3.327373in}{2.102813in}}%
\pgfpathlineto{\pgfqpoint{3.387202in}{2.059566in}}%
\pgfpathlineto{\pgfqpoint{3.417117in}{2.033817in}}%
\pgfpathlineto{\pgfqpoint{3.476946in}{1.972612in}}%
\pgfpathlineto{\pgfqpoint{3.506861in}{1.947077in}}%
\pgfpathlineto{\pgfqpoint{3.536775in}{1.919287in}}%
\pgfpathlineto{\pgfqpoint{3.626519in}{1.818747in}}%
\pgfpathlineto{\pgfqpoint{3.656434in}{1.787868in}}%
\pgfpathlineto{\pgfqpoint{3.686349in}{1.753210in}}%
\pgfpathlineto{\pgfqpoint{3.776093in}{1.635609in}}%
\pgfpathlineto{\pgfqpoint{3.806007in}{1.597715in}}%
\pgfpathlineto{\pgfqpoint{3.895751in}{1.465065in}}%
\pgfpathlineto{\pgfqpoint{3.925666in}{1.420774in}}%
\pgfpathlineto{\pgfqpoint{3.955580in}{1.372808in}}%
\pgfpathlineto{\pgfqpoint{4.015410in}{1.268647in}}%
\pgfpathlineto{\pgfqpoint{4.075239in}{1.163642in}}%
\pgfpathlineto{\pgfqpoint{4.135068in}{1.044426in}}%
\pgfpathlineto{\pgfqpoint{4.164983in}{1.004844in}}%
\pgfpathlineto{\pgfqpoint{4.194898in}{0.971173in}}%
\pgfpathlineto{\pgfqpoint{4.224812in}{0.935253in}}%
\pgfpathlineto{\pgfqpoint{4.254727in}{0.922199in}}%
\pgfpathlineto{\pgfqpoint{4.284642in}{0.931622in}}%
\pgfpathlineto{\pgfqpoint{4.314556in}{0.950627in}}%
\pgfpathlineto{\pgfqpoint{4.344471in}{0.984305in}}%
\pgfpathlineto{\pgfqpoint{4.374386in}{1.030579in}}%
\pgfpathlineto{\pgfqpoint{4.404300in}{1.069521in}}%
\pgfpathlineto{\pgfqpoint{4.434215in}{1.131253in}}%
\pgfpathlineto{\pgfqpoint{4.484713in}{1.209165in}}%
\pgfpathlineto{\pgfqpoint{4.484713in}{1.209165in}}%
\pgfusepath{stroke}%
\end{pgfscope}%
\begin{pgfscope}%
\pgfpathrectangle{\pgfqpoint{0.754713in}{0.682899in}}{\pgfqpoint{3.720000in}{3.020000in}}%
\pgfusepath{clip}%
\pgfsetrectcap%
\pgfsetroundjoin%
\pgfsetlinewidth{1.505625pt}%
\definecolor{currentstroke}{rgb}{0.283197,0.115680,0.436115}%
\pgfsetstrokecolor{currentstroke}%
\pgfsetdash{}{0pt}%
\pgfpathmoveto{\pgfqpoint{0.913460in}{1.266667in}}%
\pgfpathlineto{\pgfqpoint{0.933304in}{1.305795in}}%
\pgfpathlineto{\pgfqpoint{0.953147in}{1.337200in}}%
\pgfpathlineto{\pgfqpoint{0.972991in}{1.365133in}}%
\pgfpathlineto{\pgfqpoint{1.012677in}{1.426777in}}%
\pgfpathlineto{\pgfqpoint{1.032521in}{1.453695in}}%
\pgfpathlineto{\pgfqpoint{1.052364in}{1.484859in}}%
\pgfpathlineto{\pgfqpoint{1.171425in}{1.646308in}}%
\pgfpathlineto{\pgfqpoint{1.211112in}{1.693424in}}%
\pgfpathlineto{\pgfqpoint{1.230956in}{1.719020in}}%
\pgfpathlineto{\pgfqpoint{1.250799in}{1.740528in}}%
\pgfpathlineto{\pgfqpoint{1.290486in}{1.788559in}}%
\pgfpathlineto{\pgfqpoint{1.330173in}{1.830148in}}%
\pgfpathlineto{\pgfqpoint{1.350016in}{1.852619in}}%
\pgfpathlineto{\pgfqpoint{1.369860in}{1.871162in}}%
\pgfpathlineto{\pgfqpoint{1.389703in}{1.892235in}}%
\pgfpathlineto{\pgfqpoint{1.449234in}{1.948412in}}%
\pgfpathlineto{\pgfqpoint{1.508764in}{2.002449in}}%
\pgfpathlineto{\pgfqpoint{1.548451in}{2.032969in}}%
\pgfpathlineto{\pgfqpoint{1.568294in}{2.050861in}}%
\pgfpathlineto{\pgfqpoint{1.588138in}{2.064129in}}%
\pgfpathlineto{\pgfqpoint{1.607981in}{2.080662in}}%
\pgfpathlineto{\pgfqpoint{1.627825in}{2.095365in}}%
\pgfpathlineto{\pgfqpoint{1.647668in}{2.108057in}}%
\pgfpathlineto{\pgfqpoint{1.687355in}{2.136979in}}%
\pgfpathlineto{\pgfqpoint{1.707199in}{2.148404in}}%
\pgfpathlineto{\pgfqpoint{1.727042in}{2.161483in}}%
\pgfpathlineto{\pgfqpoint{1.826259in}{2.218520in}}%
\pgfpathlineto{\pgfqpoint{1.865946in}{2.237153in}}%
\pgfpathlineto{\pgfqpoint{1.885790in}{2.245724in}}%
\pgfpathlineto{\pgfqpoint{1.905633in}{2.255906in}}%
\pgfpathlineto{\pgfqpoint{1.985007in}{2.287646in}}%
\pgfpathlineto{\pgfqpoint{2.064381in}{2.314493in}}%
\pgfpathlineto{\pgfqpoint{2.163598in}{2.338359in}}%
\pgfpathlineto{\pgfqpoint{2.203285in}{2.345159in}}%
\pgfpathlineto{\pgfqpoint{2.282659in}{2.356536in}}%
\pgfpathlineto{\pgfqpoint{2.362033in}{2.360399in}}%
\pgfpathlineto{\pgfqpoint{2.421563in}{2.360495in}}%
\pgfpathlineto{\pgfqpoint{2.461250in}{2.358375in}}%
\pgfpathlineto{\pgfqpoint{2.520780in}{2.353657in}}%
\pgfpathlineto{\pgfqpoint{2.540624in}{2.350216in}}%
\pgfpathlineto{\pgfqpoint{2.560467in}{2.348075in}}%
\pgfpathlineto{\pgfqpoint{2.600154in}{2.340426in}}%
\pgfpathlineto{\pgfqpoint{2.619998in}{2.337736in}}%
\pgfpathlineto{\pgfqpoint{2.679528in}{2.323278in}}%
\pgfpathlineto{\pgfqpoint{2.778745in}{2.292490in}}%
\pgfpathlineto{\pgfqpoint{2.877963in}{2.252488in}}%
\pgfpathlineto{\pgfqpoint{2.897806in}{2.243874in}}%
\pgfpathlineto{\pgfqpoint{2.937493in}{2.223731in}}%
\pgfpathlineto{\pgfqpoint{2.997023in}{2.191897in}}%
\pgfpathlineto{\pgfqpoint{3.116084in}{2.116549in}}%
\pgfpathlineto{\pgfqpoint{3.155771in}{2.087757in}}%
\pgfpathlineto{\pgfqpoint{3.175615in}{2.073911in}}%
\pgfpathlineto{\pgfqpoint{3.274832in}{1.992583in}}%
\pgfpathlineto{\pgfqpoint{3.314519in}{1.957115in}}%
\pgfpathlineto{\pgfqpoint{3.354206in}{1.920238in}}%
\pgfpathlineto{\pgfqpoint{3.413736in}{1.861849in}}%
\pgfpathlineto{\pgfqpoint{3.473267in}{1.797496in}}%
\pgfpathlineto{\pgfqpoint{3.512953in}{1.752941in}}%
\pgfpathlineto{\pgfqpoint{3.592327in}{1.657123in}}%
\pgfpathlineto{\pgfqpoint{3.632014in}{1.605852in}}%
\pgfpathlineto{\pgfqpoint{3.711388in}{1.496544in}}%
\pgfpathlineto{\pgfqpoint{3.751075in}{1.438797in}}%
\pgfpathlineto{\pgfqpoint{3.810605in}{1.346958in}}%
\pgfpathlineto{\pgfqpoint{3.889979in}{1.214432in}}%
\pgfpathlineto{\pgfqpoint{3.909823in}{1.178908in}}%
\pgfpathlineto{\pgfqpoint{3.949510in}{1.101522in}}%
\pgfpathlineto{\pgfqpoint{3.989196in}{1.046803in}}%
\pgfpathlineto{\pgfqpoint{4.009040in}{1.046427in}}%
\pgfpathlineto{\pgfqpoint{4.028883in}{1.063931in}}%
\pgfpathlineto{\pgfqpoint{4.048727in}{1.089470in}}%
\pgfpathlineto{\pgfqpoint{4.068570in}{1.119620in}}%
\pgfpathlineto{\pgfqpoint{4.088414in}{1.152975in}}%
\pgfpathlineto{\pgfqpoint{4.128101in}{1.203496in}}%
\pgfpathlineto{\pgfqpoint{4.187631in}{1.271802in}}%
\pgfpathlineto{\pgfqpoint{4.227318in}{1.308723in}}%
\pgfpathlineto{\pgfqpoint{4.247161in}{1.330310in}}%
\pgfpathlineto{\pgfqpoint{4.286848in}{1.365899in}}%
\pgfpathlineto{\pgfqpoint{4.306692in}{1.382980in}}%
\pgfpathlineto{\pgfqpoint{4.326535in}{1.397554in}}%
\pgfpathlineto{\pgfqpoint{4.366222in}{1.428849in}}%
\pgfpathlineto{\pgfqpoint{4.386066in}{1.440781in}}%
\pgfpathlineto{\pgfqpoint{4.425753in}{1.467700in}}%
\pgfpathlineto{\pgfqpoint{4.445596in}{1.477794in}}%
\pgfpathlineto{\pgfqpoint{4.484713in}{1.500267in}}%
\pgfpathlineto{\pgfqpoint{4.484713in}{1.500267in}}%
\pgfusepath{stroke}%
\end{pgfscope}%
\begin{pgfscope}%
\pgfpathrectangle{\pgfqpoint{0.754713in}{0.682899in}}{\pgfqpoint{3.720000in}{3.020000in}}%
\pgfusepath{clip}%
\pgfsetrectcap%
\pgfsetroundjoin%
\pgfsetlinewidth{1.505625pt}%
\definecolor{currentstroke}{rgb}{0.276022,0.044167,0.370164}%
\pgfsetstrokecolor{currentstroke}%
\pgfsetdash{}{0pt}%
\pgfpathmoveto{\pgfqpoint{0.877572in}{1.218494in}}%
\pgfpathlineto{\pgfqpoint{0.892929in}{1.232550in}}%
\pgfpathlineto{\pgfqpoint{0.908287in}{1.253752in}}%
\pgfpathlineto{\pgfqpoint{0.923644in}{1.293437in}}%
\pgfpathlineto{\pgfqpoint{0.939001in}{1.316000in}}%
\pgfpathlineto{\pgfqpoint{0.954359in}{1.325059in}}%
\pgfpathlineto{\pgfqpoint{0.985074in}{1.386803in}}%
\pgfpathlineto{\pgfqpoint{1.000431in}{1.405426in}}%
\pgfpathlineto{\pgfqpoint{1.015788in}{1.421348in}}%
\pgfpathlineto{\pgfqpoint{1.046503in}{1.471379in}}%
\pgfpathlineto{\pgfqpoint{1.092575in}{1.527004in}}%
\pgfpathlineto{\pgfqpoint{1.107933in}{1.550805in}}%
\pgfpathlineto{\pgfqpoint{1.123290in}{1.568180in}}%
\pgfpathlineto{\pgfqpoint{1.138648in}{1.583538in}}%
\pgfpathlineto{\pgfqpoint{1.169362in}{1.625003in}}%
\pgfpathlineto{\pgfqpoint{1.200077in}{1.655672in}}%
\pgfpathlineto{\pgfqpoint{1.230792in}{1.694370in}}%
\pgfpathlineto{\pgfqpoint{1.246149in}{1.708065in}}%
\pgfpathlineto{\pgfqpoint{1.261507in}{1.723331in}}%
\pgfpathlineto{\pgfqpoint{1.276864in}{1.742117in}}%
\pgfpathlineto{\pgfqpoint{1.292222in}{1.758918in}}%
\pgfpathlineto{\pgfqpoint{1.322936in}{1.785837in}}%
\pgfpathlineto{\pgfqpoint{1.338294in}{1.803432in}}%
\pgfpathlineto{\pgfqpoint{1.353651in}{1.819225in}}%
\pgfpathlineto{\pgfqpoint{1.369009in}{1.830374in}}%
\pgfpathlineto{\pgfqpoint{1.384366in}{1.845173in}}%
\pgfpathlineto{\pgfqpoint{1.399723in}{1.861661in}}%
\pgfpathlineto{\pgfqpoint{1.415081in}{1.875228in}}%
\pgfpathlineto{\pgfqpoint{1.430438in}{1.886468in}}%
\pgfpathlineto{\pgfqpoint{1.445796in}{1.899728in}}%
\pgfpathlineto{\pgfqpoint{1.461153in}{1.915591in}}%
\pgfpathlineto{\pgfqpoint{1.491868in}{1.937904in}}%
\pgfpathlineto{\pgfqpoint{1.522583in}{1.965579in}}%
\pgfpathlineto{\pgfqpoint{1.553297in}{1.986579in}}%
\pgfpathlineto{\pgfqpoint{1.584012in}{2.012418in}}%
\pgfpathlineto{\pgfqpoint{1.599370in}{2.021444in}}%
\pgfpathlineto{\pgfqpoint{1.614727in}{2.032099in}}%
\pgfpathlineto{\pgfqpoint{1.645442in}{2.055631in}}%
\pgfpathlineto{\pgfqpoint{1.676157in}{2.074037in}}%
\pgfpathlineto{\pgfqpoint{1.691514in}{2.086049in}}%
\pgfpathlineto{\pgfqpoint{1.706871in}{2.096051in}}%
\pgfpathlineto{\pgfqpoint{1.722229in}{2.103888in}}%
\pgfpathlineto{\pgfqpoint{1.768301in}{2.133633in}}%
\pgfpathlineto{\pgfqpoint{1.783658in}{2.141097in}}%
\pgfpathlineto{\pgfqpoint{1.829731in}{2.168024in}}%
\pgfpathlineto{\pgfqpoint{1.845088in}{2.175379in}}%
\pgfpathlineto{\pgfqpoint{1.875803in}{2.193368in}}%
\pgfpathlineto{\pgfqpoint{1.906518in}{2.207028in}}%
\pgfpathlineto{\pgfqpoint{1.937232in}{2.223725in}}%
\pgfpathlineto{\pgfqpoint{1.967947in}{2.236325in}}%
\pgfpathlineto{\pgfqpoint{1.998662in}{2.251400in}}%
\pgfpathlineto{\pgfqpoint{2.029377in}{2.263156in}}%
\pgfpathlineto{\pgfqpoint{2.044734in}{2.270610in}}%
\pgfpathlineto{\pgfqpoint{2.121521in}{2.299704in}}%
\pgfpathlineto{\pgfqpoint{2.136879in}{2.304181in}}%
\pgfpathlineto{\pgfqpoint{2.182951in}{2.320048in}}%
\pgfpathlineto{\pgfqpoint{2.213666in}{2.329643in}}%
\pgfpathlineto{\pgfqpoint{2.229023in}{2.334848in}}%
\pgfpathlineto{\pgfqpoint{2.351882in}{2.366294in}}%
\pgfpathlineto{\pgfqpoint{2.382597in}{2.371963in}}%
\pgfpathlineto{\pgfqpoint{2.413312in}{2.378622in}}%
\pgfpathlineto{\pgfqpoint{2.444027in}{2.383693in}}%
\pgfpathlineto{\pgfqpoint{2.474742in}{2.389051in}}%
\pgfpathlineto{\pgfqpoint{2.505456in}{2.393190in}}%
\pgfpathlineto{\pgfqpoint{2.536171in}{2.397093in}}%
\pgfpathlineto{\pgfqpoint{2.643673in}{2.407047in}}%
\pgfpathlineto{\pgfqpoint{2.781890in}{2.409628in}}%
\pgfpathlineto{\pgfqpoint{2.827962in}{2.408390in}}%
\pgfpathlineto{\pgfqpoint{2.935464in}{2.401162in}}%
\pgfpathlineto{\pgfqpoint{3.042965in}{2.387017in}}%
\pgfpathlineto{\pgfqpoint{3.089038in}{2.378855in}}%
\pgfpathlineto{\pgfqpoint{3.119752in}{2.373002in}}%
\pgfpathlineto{\pgfqpoint{3.242612in}{2.343035in}}%
\pgfpathlineto{\pgfqpoint{3.334756in}{2.314524in}}%
\pgfpathlineto{\pgfqpoint{3.396186in}{2.292178in}}%
\pgfpathlineto{\pgfqpoint{3.472973in}{2.260275in}}%
\pgfpathlineto{\pgfqpoint{3.549760in}{2.224028in}}%
\pgfpathlineto{\pgfqpoint{3.595832in}{2.200086in}}%
\pgfpathlineto{\pgfqpoint{3.657262in}{2.165524in}}%
\pgfpathlineto{\pgfqpoint{3.703334in}{2.137618in}}%
\pgfpathlineto{\pgfqpoint{3.749406in}{2.108167in}}%
\pgfpathlineto{\pgfqpoint{3.810836in}{2.065600in}}%
\pgfpathlineto{\pgfqpoint{3.872265in}{2.019850in}}%
\pgfpathlineto{\pgfqpoint{3.933695in}{1.970964in}}%
\pgfpathlineto{\pgfqpoint{3.995124in}{1.918436in}}%
\pgfpathlineto{\pgfqpoint{4.041197in}{1.876548in}}%
\pgfpathlineto{\pgfqpoint{4.102626in}{1.817817in}}%
\pgfpathlineto{\pgfqpoint{4.164056in}{1.754767in}}%
\pgfpathlineto{\pgfqpoint{4.225485in}{1.687653in}}%
\pgfpathlineto{\pgfqpoint{4.286915in}{1.615865in}}%
\pgfpathlineto{\pgfqpoint{4.332987in}{1.558790in}}%
\pgfpathlineto{\pgfqpoint{4.332987in}{1.558790in}}%
\pgfusepath{stroke}%
\end{pgfscope}%
\begin{pgfscope}%
\pgfpathrectangle{\pgfqpoint{0.754713in}{0.682899in}}{\pgfqpoint{3.720000in}{3.020000in}}%
\pgfusepath{clip}%
\pgfsetrectcap%
\pgfsetroundjoin%
\pgfsetlinewidth{1.505625pt}%
\definecolor{currentstroke}{rgb}{0.267004,0.004874,0.329415}%
\pgfsetstrokecolor{currentstroke}%
\pgfsetdash{}{0pt}%
\pgfpathmoveto{\pgfqpoint{0.892413in}{1.242420in}}%
\pgfpathlineto{\pgfqpoint{0.904931in}{1.248931in}}%
\pgfpathlineto{\pgfqpoint{0.917449in}{1.278821in}}%
\pgfpathlineto{\pgfqpoint{0.929967in}{1.303812in}}%
\pgfpathlineto{\pgfqpoint{0.942486in}{1.318266in}}%
\pgfpathlineto{\pgfqpoint{0.955004in}{1.336597in}}%
\pgfpathlineto{\pgfqpoint{0.967522in}{1.359057in}}%
\pgfpathlineto{\pgfqpoint{0.980040in}{1.384915in}}%
\pgfpathlineto{\pgfqpoint{0.992559in}{1.402692in}}%
\pgfpathlineto{\pgfqpoint{1.005077in}{1.417661in}}%
\pgfpathlineto{\pgfqpoint{1.017595in}{1.437862in}}%
\pgfpathlineto{\pgfqpoint{1.030113in}{1.461634in}}%
\pgfpathlineto{\pgfqpoint{1.042631in}{1.479633in}}%
\pgfpathlineto{\pgfqpoint{1.067668in}{1.510554in}}%
\pgfpathlineto{\pgfqpoint{1.080186in}{1.530047in}}%
\pgfpathlineto{\pgfqpoint{1.092704in}{1.552489in}}%
\pgfpathlineto{\pgfqpoint{1.105222in}{1.568958in}}%
\pgfpathlineto{\pgfqpoint{1.117741in}{1.580912in}}%
\pgfpathlineto{\pgfqpoint{1.130259in}{1.599190in}}%
\pgfpathlineto{\pgfqpoint{1.142777in}{1.620063in}}%
\pgfpathlineto{\pgfqpoint{1.155295in}{1.636508in}}%
\pgfpathlineto{\pgfqpoint{1.180332in}{1.663273in}}%
\pgfpathlineto{\pgfqpoint{1.205368in}{1.700357in}}%
\pgfpathlineto{\pgfqpoint{1.217886in}{1.715084in}}%
\pgfpathlineto{\pgfqpoint{1.230405in}{1.725821in}}%
\pgfpathlineto{\pgfqpoint{1.267959in}{1.775899in}}%
\pgfpathlineto{\pgfqpoint{1.292996in}{1.799738in}}%
\pgfpathlineto{\pgfqpoint{1.305514in}{1.815394in}}%
\pgfpathlineto{\pgfqpoint{1.318032in}{1.833370in}}%
\pgfpathlineto{\pgfqpoint{1.355587in}{1.870332in}}%
\pgfpathlineto{\pgfqpoint{1.368105in}{1.886978in}}%
\pgfpathlineto{\pgfqpoint{1.380623in}{1.900467in}}%
\pgfpathlineto{\pgfqpoint{1.405660in}{1.921849in}}%
\pgfpathlineto{\pgfqpoint{1.430696in}{1.952398in}}%
\pgfpathlineto{\pgfqpoint{1.443214in}{1.963237in}}%
\pgfpathlineto{\pgfqpoint{1.455732in}{1.972204in}}%
\pgfpathlineto{\pgfqpoint{1.493287in}{2.013387in}}%
\pgfpathlineto{\pgfqpoint{1.518324in}{2.032789in}}%
\pgfpathlineto{\pgfqpoint{1.530842in}{2.045862in}}%
\pgfpathlineto{\pgfqpoint{1.543360in}{2.060594in}}%
\pgfpathlineto{\pgfqpoint{1.568396in}{2.078920in}}%
\pgfpathlineto{\pgfqpoint{1.605951in}{2.116273in}}%
\pgfpathlineto{\pgfqpoint{1.630987in}{2.134066in}}%
\pgfpathlineto{\pgfqpoint{1.656024in}{2.159650in}}%
\pgfpathlineto{\pgfqpoint{1.681060in}{2.176476in}}%
\pgfpathlineto{\pgfqpoint{1.718615in}{2.210662in}}%
\pgfpathlineto{\pgfqpoint{1.743651in}{2.227126in}}%
\pgfpathlineto{\pgfqpoint{1.768688in}{2.250671in}}%
\pgfpathlineto{\pgfqpoint{1.793724in}{2.266360in}}%
\pgfpathlineto{\pgfqpoint{1.831279in}{2.297969in}}%
\pgfpathlineto{\pgfqpoint{1.856315in}{2.313145in}}%
\pgfpathlineto{\pgfqpoint{1.881352in}{2.334875in}}%
\pgfpathlineto{\pgfqpoint{1.906388in}{2.349410in}}%
\pgfpathlineto{\pgfqpoint{1.943943in}{2.378551in}}%
\pgfpathlineto{\pgfqpoint{1.968979in}{2.392584in}}%
\pgfpathlineto{\pgfqpoint{1.994016in}{2.412523in}}%
\pgfpathlineto{\pgfqpoint{2.019052in}{2.426206in}}%
\pgfpathlineto{\pgfqpoint{2.056607in}{2.453155in}}%
\pgfpathlineto{\pgfqpoint{2.069125in}{2.459093in}}%
\pgfpathlineto{\pgfqpoint{2.081643in}{2.466453in}}%
\pgfpathlineto{\pgfqpoint{2.106680in}{2.484476in}}%
\pgfpathlineto{\pgfqpoint{2.131716in}{2.497541in}}%
\pgfpathlineto{\pgfqpoint{2.169271in}{2.522208in}}%
\pgfpathlineto{\pgfqpoint{2.181789in}{2.527826in}}%
\pgfpathlineto{\pgfqpoint{2.194307in}{2.535071in}}%
\pgfpathlineto{\pgfqpoint{2.219343in}{2.551394in}}%
\pgfpathlineto{\pgfqpoint{2.244380in}{2.563724in}}%
\pgfpathlineto{\pgfqpoint{2.281935in}{2.586349in}}%
\pgfpathlineto{\pgfqpoint{2.294453in}{2.591697in}}%
\pgfpathlineto{\pgfqpoint{2.306971in}{2.598412in}}%
\pgfpathlineto{\pgfqpoint{2.332007in}{2.613755in}}%
\pgfpathlineto{\pgfqpoint{2.357044in}{2.625177in}}%
\pgfpathlineto{\pgfqpoint{2.394598in}{2.646314in}}%
\pgfpathlineto{\pgfqpoint{2.407117in}{2.651352in}}%
\pgfpathlineto{\pgfqpoint{2.419635in}{2.657610in}}%
\pgfpathlineto{\pgfqpoint{2.444671in}{2.672023in}}%
\pgfpathlineto{\pgfqpoint{2.469708in}{2.682861in}}%
\pgfpathlineto{\pgfqpoint{2.507262in}{2.702361in}}%
\pgfpathlineto{\pgfqpoint{2.532299in}{2.713231in}}%
\pgfpathlineto{\pgfqpoint{2.544817in}{2.720655in}}%
\pgfpathlineto{\pgfqpoint{2.569853in}{2.731583in}}%
\pgfpathlineto{\pgfqpoint{2.582372in}{2.736773in}}%
\pgfpathlineto{\pgfqpoint{2.607408in}{2.749928in}}%
\pgfpathlineto{\pgfqpoint{2.644963in}{2.765285in}}%
\pgfpathlineto{\pgfqpoint{2.657481in}{2.772257in}}%
\pgfpathlineto{\pgfqpoint{2.682517in}{2.782188in}}%
\pgfpathlineto{\pgfqpoint{2.695036in}{2.787086in}}%
\pgfpathlineto{\pgfqpoint{2.720072in}{2.799302in}}%
\pgfpathlineto{\pgfqpoint{2.745108in}{2.807976in}}%
\pgfpathlineto{\pgfqpoint{2.782663in}{2.825132in}}%
\pgfpathlineto{\pgfqpoint{2.807700in}{2.833983in}}%
\pgfpathlineto{\pgfqpoint{2.832736in}{2.845328in}}%
\pgfpathlineto{\pgfqpoint{2.857772in}{2.853451in}}%
\pgfpathlineto{\pgfqpoint{2.882809in}{2.865203in}}%
\pgfpathlineto{\pgfqpoint{2.970436in}{2.896250in}}%
\pgfpathlineto{\pgfqpoint{2.995473in}{2.907031in}}%
\pgfpathlineto{\pgfqpoint{3.083100in}{2.936125in}}%
\pgfpathlineto{\pgfqpoint{3.108137in}{2.946178in}}%
\pgfpathlineto{\pgfqpoint{3.158210in}{2.961949in}}%
\pgfpathlineto{\pgfqpoint{3.233319in}{2.985930in}}%
\pgfpathlineto{\pgfqpoint{3.258355in}{2.992700in}}%
\pgfpathlineto{\pgfqpoint{3.283392in}{3.000733in}}%
\pgfpathlineto{\pgfqpoint{3.308428in}{3.007833in}}%
\pgfpathlineto{\pgfqpoint{3.333464in}{3.016110in}}%
\pgfpathlineto{\pgfqpoint{3.408574in}{3.036362in}}%
\pgfpathlineto{\pgfqpoint{3.458647in}{3.050294in}}%
\pgfpathlineto{\pgfqpoint{3.483683in}{3.056923in}}%
\pgfpathlineto{\pgfqpoint{3.508719in}{3.063399in}}%
\pgfpathlineto{\pgfqpoint{3.533756in}{3.069714in}}%
\pgfpathlineto{\pgfqpoint{3.558792in}{3.076648in}}%
\pgfpathlineto{\pgfqpoint{3.596347in}{3.085337in}}%
\pgfpathlineto{\pgfqpoint{3.671456in}{3.103605in}}%
\pgfpathlineto{\pgfqpoint{3.683974in}{3.105651in}}%
\pgfpathlineto{\pgfqpoint{3.683974in}{3.105651in}}%
\pgfusepath{stroke}%
\end{pgfscope}%
\begin{pgfscope}%
\pgfsetrectcap%
\pgfsetmiterjoin%
\pgfsetlinewidth{0.803000pt}%
\definecolor{currentstroke}{rgb}{0.501961,0.501961,0.501961}%
\pgfsetstrokecolor{currentstroke}%
\pgfsetdash{}{0pt}%
\pgfpathmoveto{\pgfqpoint{0.754713in}{0.682899in}}%
\pgfpathlineto{\pgfqpoint{0.754713in}{3.702899in}}%
\pgfusepath{stroke}%
\end{pgfscope}%
\begin{pgfscope}%
\pgfsetrectcap%
\pgfsetmiterjoin%
\pgfsetlinewidth{0.803000pt}%
\definecolor{currentstroke}{rgb}{0.501961,0.501961,0.501961}%
\pgfsetstrokecolor{currentstroke}%
\pgfsetdash{}{0pt}%
\pgfpathmoveto{\pgfqpoint{4.474713in}{0.682899in}}%
\pgfpathlineto{\pgfqpoint{4.474713in}{3.702899in}}%
\pgfusepath{stroke}%
\end{pgfscope}%
\begin{pgfscope}%
\pgfsetrectcap%
\pgfsetmiterjoin%
\pgfsetlinewidth{0.803000pt}%
\definecolor{currentstroke}{rgb}{0.501961,0.501961,0.501961}%
\pgfsetstrokecolor{currentstroke}%
\pgfsetdash{}{0pt}%
\pgfpathmoveto{\pgfqpoint{0.754713in}{0.682899in}}%
\pgfpathlineto{\pgfqpoint{4.474712in}{0.682899in}}%
\pgfusepath{stroke}%
\end{pgfscope}%
\begin{pgfscope}%
\pgfsetrectcap%
\pgfsetmiterjoin%
\pgfsetlinewidth{0.803000pt}%
\definecolor{currentstroke}{rgb}{0.501961,0.501961,0.501961}%
\pgfsetstrokecolor{currentstroke}%
\pgfsetdash{}{0pt}%
\pgfpathmoveto{\pgfqpoint{0.754713in}{3.702899in}}%
\pgfpathlineto{\pgfqpoint{4.474712in}{3.702899in}}%
\pgfusepath{stroke}%
\end{pgfscope}%
\begin{pgfscope}%
\pgfpathrectangle{\pgfqpoint{4.707213in}{0.682899in}}{\pgfqpoint{0.151000in}{3.020000in}}%
\pgfusepath{clip}%
\pgfsetbuttcap%
\pgfsetmiterjoin%
\definecolor{currentfill}{rgb}{1.000000,1.000000,1.000000}%
\pgfsetfillcolor{currentfill}%
\pgfsetlinewidth{0.010037pt}%
\definecolor{currentstroke}{rgb}{1.000000,1.000000,1.000000}%
\pgfsetstrokecolor{currentstroke}%
\pgfsetdash{}{0pt}%
\pgfpathmoveto{\pgfqpoint{4.707213in}{0.682899in}}%
\pgfpathlineto{\pgfqpoint{4.707213in}{0.694696in}}%
\pgfpathlineto{\pgfqpoint{4.707213in}{3.691102in}}%
\pgfpathlineto{\pgfqpoint{4.707213in}{3.702899in}}%
\pgfpathlineto{\pgfqpoint{4.858213in}{3.702899in}}%
\pgfpathlineto{\pgfqpoint{4.858213in}{3.691102in}}%
\pgfpathlineto{\pgfqpoint{4.858213in}{0.694696in}}%
\pgfpathlineto{\pgfqpoint{4.858213in}{0.682899in}}%
\pgfpathclose%
\pgfusepath{stroke,fill}%
\end{pgfscope}%
\begin{pgfscope}%
\pgfsys@transformshift{4.710000in}{0.687899in}%
\pgftext[left,bottom]{\pgfimage[interpolate=true,width=0.150000in,height=3.020000in]{series_s_ds-img0.png}}%
\end{pgfscope}%
\begin{pgfscope}%
\pgfsetbuttcap%
\pgfsetroundjoin%
\definecolor{currentfill}{rgb}{0.000000,0.000000,0.000000}%
\pgfsetfillcolor{currentfill}%
\pgfsetlinewidth{0.803000pt}%
\definecolor{currentstroke}{rgb}{0.000000,0.000000,0.000000}%
\pgfsetstrokecolor{currentstroke}%
\pgfsetdash{}{0pt}%
\pgfsys@defobject{currentmarker}{\pgfqpoint{0.000000in}{0.000000in}}{\pgfqpoint{0.048611in}{0.000000in}}{%
\pgfpathmoveto{\pgfqpoint{0.000000in}{0.000000in}}%
\pgfpathlineto{\pgfqpoint{0.048611in}{0.000000in}}%
\pgfusepath{stroke,fill}%
}%
\begin{pgfscope}%
\pgfsys@transformshift{4.858213in}{1.468806in}%
\pgfsys@useobject{currentmarker}{}%
\end{pgfscope}%
\end{pgfscope}%
\begin{pgfscope}%
\pgfsetbuttcap%
\pgfsetroundjoin%
\definecolor{currentfill}{rgb}{0.000000,0.000000,0.000000}%
\pgfsetfillcolor{currentfill}%
\pgfsetlinewidth{0.803000pt}%
\definecolor{currentstroke}{rgb}{0.000000,0.000000,0.000000}%
\pgfsetstrokecolor{currentstroke}%
\pgfsetdash{}{0pt}%
\pgfsys@defobject{currentmarker}{\pgfqpoint{0.000000in}{0.000000in}}{\pgfqpoint{0.048611in}{0.000000in}}{%
\pgfpathmoveto{\pgfqpoint{0.000000in}{0.000000in}}%
\pgfpathlineto{\pgfqpoint{0.048611in}{0.000000in}}%
\pgfusepath{stroke,fill}%
}%
\begin{pgfscope}%
\pgfsys@transformshift{4.858213in}{1.963925in}%
\pgfsys@useobject{currentmarker}{}%
\end{pgfscope}%
\end{pgfscope}%
\begin{pgfscope}%
\pgfsetbuttcap%
\pgfsetroundjoin%
\definecolor{currentfill}{rgb}{0.000000,0.000000,0.000000}%
\pgfsetfillcolor{currentfill}%
\pgfsetlinewidth{0.803000pt}%
\definecolor{currentstroke}{rgb}{0.000000,0.000000,0.000000}%
\pgfsetstrokecolor{currentstroke}%
\pgfsetdash{}{0pt}%
\pgfsys@defobject{currentmarker}{\pgfqpoint{0.000000in}{0.000000in}}{\pgfqpoint{0.048611in}{0.000000in}}{%
\pgfpathmoveto{\pgfqpoint{0.000000in}{0.000000in}}%
\pgfpathlineto{\pgfqpoint{0.048611in}{0.000000in}}%
\pgfusepath{stroke,fill}%
}%
\begin{pgfscope}%
\pgfsys@transformshift{4.858213in}{2.315218in}%
\pgfsys@useobject{currentmarker}{}%
\end{pgfscope}%
\end{pgfscope}%
\begin{pgfscope}%
\pgfsetbuttcap%
\pgfsetroundjoin%
\definecolor{currentfill}{rgb}{0.000000,0.000000,0.000000}%
\pgfsetfillcolor{currentfill}%
\pgfsetlinewidth{0.803000pt}%
\definecolor{currentstroke}{rgb}{0.000000,0.000000,0.000000}%
\pgfsetstrokecolor{currentstroke}%
\pgfsetdash{}{0pt}%
\pgfsys@defobject{currentmarker}{\pgfqpoint{0.000000in}{0.000000in}}{\pgfqpoint{0.048611in}{0.000000in}}{%
\pgfpathmoveto{\pgfqpoint{0.000000in}{0.000000in}}%
\pgfpathlineto{\pgfqpoint{0.048611in}{0.000000in}}%
\pgfusepath{stroke,fill}%
}%
\begin{pgfscope}%
\pgfsys@transformshift{4.858213in}{2.587702in}%
\pgfsys@useobject{currentmarker}{}%
\end{pgfscope}%
\end{pgfscope}%
\begin{pgfscope}%
\pgfsetbuttcap%
\pgfsetroundjoin%
\definecolor{currentfill}{rgb}{0.000000,0.000000,0.000000}%
\pgfsetfillcolor{currentfill}%
\pgfsetlinewidth{0.803000pt}%
\definecolor{currentstroke}{rgb}{0.000000,0.000000,0.000000}%
\pgfsetstrokecolor{currentstroke}%
\pgfsetdash{}{0pt}%
\pgfsys@defobject{currentmarker}{\pgfqpoint{0.000000in}{0.000000in}}{\pgfqpoint{0.048611in}{0.000000in}}{%
\pgfpathmoveto{\pgfqpoint{0.000000in}{0.000000in}}%
\pgfpathlineto{\pgfqpoint{0.048611in}{0.000000in}}%
\pgfusepath{stroke,fill}%
}%
\begin{pgfscope}%
\pgfsys@transformshift{4.858213in}{2.810337in}%
\pgfsys@useobject{currentmarker}{}%
\end{pgfscope}%
\end{pgfscope}%
\begin{pgfscope}%
\pgfsetbuttcap%
\pgfsetroundjoin%
\definecolor{currentfill}{rgb}{0.000000,0.000000,0.000000}%
\pgfsetfillcolor{currentfill}%
\pgfsetlinewidth{0.803000pt}%
\definecolor{currentstroke}{rgb}{0.000000,0.000000,0.000000}%
\pgfsetstrokecolor{currentstroke}%
\pgfsetdash{}{0pt}%
\pgfsys@defobject{currentmarker}{\pgfqpoint{0.000000in}{0.000000in}}{\pgfqpoint{0.048611in}{0.000000in}}{%
\pgfpathmoveto{\pgfqpoint{0.000000in}{0.000000in}}%
\pgfpathlineto{\pgfqpoint{0.048611in}{0.000000in}}%
\pgfusepath{stroke,fill}%
}%
\begin{pgfscope}%
\pgfsys@transformshift{4.858213in}{2.998573in}%
\pgfsys@useobject{currentmarker}{}%
\end{pgfscope}%
\end{pgfscope}%
\begin{pgfscope}%
\pgfsetbuttcap%
\pgfsetroundjoin%
\definecolor{currentfill}{rgb}{0.000000,0.000000,0.000000}%
\pgfsetfillcolor{currentfill}%
\pgfsetlinewidth{0.803000pt}%
\definecolor{currentstroke}{rgb}{0.000000,0.000000,0.000000}%
\pgfsetstrokecolor{currentstroke}%
\pgfsetdash{}{0pt}%
\pgfsys@defobject{currentmarker}{\pgfqpoint{0.000000in}{0.000000in}}{\pgfqpoint{0.048611in}{0.000000in}}{%
\pgfpathmoveto{\pgfqpoint{0.000000in}{0.000000in}}%
\pgfpathlineto{\pgfqpoint{0.048611in}{0.000000in}}%
\pgfusepath{stroke,fill}%
}%
\begin{pgfscope}%
\pgfsys@transformshift{4.858213in}{3.161630in}%
\pgfsys@useobject{currentmarker}{}%
\end{pgfscope}%
\end{pgfscope}%
\begin{pgfscope}%
\pgfsetbuttcap%
\pgfsetroundjoin%
\definecolor{currentfill}{rgb}{0.000000,0.000000,0.000000}%
\pgfsetfillcolor{currentfill}%
\pgfsetlinewidth{0.803000pt}%
\definecolor{currentstroke}{rgb}{0.000000,0.000000,0.000000}%
\pgfsetstrokecolor{currentstroke}%
\pgfsetdash{}{0pt}%
\pgfsys@defobject{currentmarker}{\pgfqpoint{0.000000in}{0.000000in}}{\pgfqpoint{0.048611in}{0.000000in}}{%
\pgfpathmoveto{\pgfqpoint{0.000000in}{0.000000in}}%
\pgfpathlineto{\pgfqpoint{0.048611in}{0.000000in}}%
\pgfusepath{stroke,fill}%
}%
\begin{pgfscope}%
\pgfsys@transformshift{4.858213in}{3.305456in}%
\pgfsys@useobject{currentmarker}{}%
\end{pgfscope}%
\end{pgfscope}%
\begin{pgfscope}%
\pgfsetbuttcap%
\pgfsetroundjoin%
\definecolor{currentfill}{rgb}{0.000000,0.000000,0.000000}%
\pgfsetfillcolor{currentfill}%
\pgfsetlinewidth{0.803000pt}%
\definecolor{currentstroke}{rgb}{0.000000,0.000000,0.000000}%
\pgfsetstrokecolor{currentstroke}%
\pgfsetdash{}{0pt}%
\pgfsys@defobject{currentmarker}{\pgfqpoint{0.000000in}{0.000000in}}{\pgfqpoint{0.048611in}{0.000000in}}{%
\pgfpathmoveto{\pgfqpoint{0.000000in}{0.000000in}}%
\pgfpathlineto{\pgfqpoint{0.048611in}{0.000000in}}%
\pgfusepath{stroke,fill}%
}%
\begin{pgfscope}%
\pgfsys@transformshift{4.858213in}{3.434114in}%
\pgfsys@useobject{currentmarker}{}%
\end{pgfscope}%
\end{pgfscope}%
\begin{pgfscope}%
\pgftext[x=4.955435in,y=3.349695in,left,base]{\rmfamily\fontsize{16.000000}{19.200000}\selectfont \(\displaystyle 10^{1}\)}%
\end{pgfscope}%
\begin{pgfscope}%
\pgftext[x=5.369668in,y=2.192899in,,top]{\rmfamily\fontsize{14.000000}{16.800000}\selectfont \(\displaystyle {\mathbf{E} \mbox{u}}\)}%
\end{pgfscope}%
\begin{pgfscope}%
\pgfsetbuttcap%
\pgfsetmiterjoin%
\pgfsetlinewidth{0.803000pt}%
\definecolor{currentstroke}{rgb}{0.501961,0.501961,0.501961}%
\pgfsetstrokecolor{currentstroke}%
\pgfsetdash{}{0pt}%
\pgfpathmoveto{\pgfqpoint{4.707213in}{0.682899in}}%
\pgfpathlineto{\pgfqpoint{4.707213in}{0.694696in}}%
\pgfpathlineto{\pgfqpoint{4.707213in}{3.691102in}}%
\pgfpathlineto{\pgfqpoint{4.707213in}{3.702899in}}%
\pgfpathlineto{\pgfqpoint{4.858213in}{3.702899in}}%
\pgfpathlineto{\pgfqpoint{4.858213in}{3.691102in}}%
\pgfpathlineto{\pgfqpoint{4.858213in}{0.694696in}}%
\pgfpathlineto{\pgfqpoint{4.858213in}{0.682899in}}%
\pgfpathclose%
\pgfusepath{stroke}%
\end{pgfscope}%
\end{pgfpicture}%
\makeatother%
\endgroup%

    \caption{Non-dimensional trajectories with the short-time scaling.\label{fig:series_s_ds}}
\end{figure}

The covariance of $\mathbb{I}\mbox{m}$ with $\mathbb{E}\mbox{u}$ is shown in Figure \ref{fig:dnumbs}. Predictably, there is quite strong correlation between the dimensionless groups. We also see that $\mathbb{I}\mbox{m} < 1$ for all drops. Using an OLS regression, we find the model $\mathbb{I}\mbox{m} \sim (0.012 \pm 0.003) \mathbb{E}\mbox{u} + (0.212 \pm 0.036) $ with $R^2 =0.59$.
\begin{figure}[H]
    \centering
    %% Creator: Matplotlib, PGF backend
%%
%% To include the figure in your LaTeX document, write
%%   \input{<filename>.pgf}
%%
%% Make sure the required packages are loaded in your preamble
%%   \usepackage{pgf}
%%
%% Figures using additional raster images can only be included by \input if
%% they are in the same directory as the main LaTeX file. For loading figures
%% from other directories you can use the `import` package
%%   \usepackage{import}
%% and then include the figures with
%%   \import{<path to file>}{<filename>.pgf}
%%
%% Matplotlib used the following preamble
%%   \usepackage{fontspec}
%%   \setmainfont{DejaVuSerif.ttf}[Path=/home/erin/anaconda3/lib/python3.6/site-packages/matplotlib/mpl-data/fonts/ttf/]
%%   \setsansfont{DejaVuSans.ttf}[Path=/home/erin/anaconda3/lib/python3.6/site-packages/matplotlib/mpl-data/fonts/ttf/]
%%   \setmonofont{DejaVuSansMono.ttf}[Path=/home/erin/anaconda3/lib/python3.6/site-packages/matplotlib/mpl-data/fonts/ttf/]
%%
\begingroup%
\makeatletter%
\begin{pgfpicture}%
\pgfpathrectangle{\pgfpointorigin}{\pgfqpoint{5.314660in}{3.641603in}}%
\pgfusepath{use as bounding box, clip}%
\begin{pgfscope}%
\pgfsetbuttcap%
\pgfsetmiterjoin%
\definecolor{currentfill}{rgb}{1.000000,1.000000,1.000000}%
\pgfsetfillcolor{currentfill}%
\pgfsetlinewidth{0.000000pt}%
\definecolor{currentstroke}{rgb}{1.000000,1.000000,1.000000}%
\pgfsetstrokecolor{currentstroke}%
\pgfsetdash{}{0pt}%
\pgfpathmoveto{\pgfqpoint{0.000000in}{0.000000in}}%
\pgfpathlineto{\pgfqpoint{5.314660in}{0.000000in}}%
\pgfpathlineto{\pgfqpoint{5.314660in}{3.641603in}}%
\pgfpathlineto{\pgfqpoint{0.000000in}{3.641603in}}%
\pgfpathclose%
\pgfusepath{fill}%
\end{pgfscope}%
\begin{pgfscope}%
\pgfsetbuttcap%
\pgfsetmiterjoin%
\definecolor{currentfill}{rgb}{1.000000,1.000000,1.000000}%
\pgfsetfillcolor{currentfill}%
\pgfsetlinewidth{0.000000pt}%
\definecolor{currentstroke}{rgb}{0.000000,0.000000,0.000000}%
\pgfsetstrokecolor{currentstroke}%
\pgfsetstrokeopacity{0.000000}%
\pgfsetdash{}{0pt}%
\pgfpathmoveto{\pgfqpoint{0.564660in}{0.521603in}}%
\pgfpathlineto{\pgfqpoint{5.214660in}{0.521603in}}%
\pgfpathlineto{\pgfqpoint{5.214660in}{3.541603in}}%
\pgfpathlineto{\pgfqpoint{0.564660in}{3.541603in}}%
\pgfpathclose%
\pgfusepath{fill}%
\end{pgfscope}%
\begin{pgfscope}%
\pgfpathrectangle{\pgfqpoint{0.564660in}{0.521603in}}{\pgfqpoint{4.650000in}{3.020000in}}%
\pgfusepath{clip}%
\pgfsetbuttcap%
\pgfsetroundjoin%
\definecolor{currentfill}{rgb}{1.000000,1.000000,1.000000}%
\pgfsetfillcolor{currentfill}%
\pgfsetlinewidth{1.003750pt}%
\definecolor{currentstroke}{rgb}{0.000000,0.000000,0.000000}%
\pgfsetstrokecolor{currentstroke}%
\pgfsetdash{}{0pt}%
\pgfpathmoveto{\pgfqpoint{0.816771in}{2.246882in}}%
\pgfpathcurveto{\pgfqpoint{0.827822in}{2.246882in}}{\pgfqpoint{0.838421in}{2.251272in}}{\pgfqpoint{0.846234in}{2.259086in}}%
\pgfpathcurveto{\pgfqpoint{0.854048in}{2.266900in}}{\pgfqpoint{0.858438in}{2.277499in}}{\pgfqpoint{0.858438in}{2.288549in}}%
\pgfpathcurveto{\pgfqpoint{0.858438in}{2.299599in}}{\pgfqpoint{0.854048in}{2.310198in}}{\pgfqpoint{0.846234in}{2.318012in}}%
\pgfpathcurveto{\pgfqpoint{0.838421in}{2.325825in}}{\pgfqpoint{0.827822in}{2.330215in}}{\pgfqpoint{0.816771in}{2.330215in}}%
\pgfpathcurveto{\pgfqpoint{0.805721in}{2.330215in}}{\pgfqpoint{0.795122in}{2.325825in}}{\pgfqpoint{0.787309in}{2.318012in}}%
\pgfpathcurveto{\pgfqpoint{0.779495in}{2.310198in}}{\pgfqpoint{0.775105in}{2.299599in}}{\pgfqpoint{0.775105in}{2.288549in}}%
\pgfpathcurveto{\pgfqpoint{0.775105in}{2.277499in}}{\pgfqpoint{0.779495in}{2.266900in}}{\pgfqpoint{0.787309in}{2.259086in}}%
\pgfpathcurveto{\pgfqpoint{0.795122in}{2.251272in}}{\pgfqpoint{0.805721in}{2.246882in}}{\pgfqpoint{0.816771in}{2.246882in}}%
\pgfpathclose%
\pgfusepath{stroke,fill}%
\end{pgfscope}%
\begin{pgfscope}%
\pgfpathrectangle{\pgfqpoint{0.564660in}{0.521603in}}{\pgfqpoint{4.650000in}{3.020000in}}%
\pgfusepath{clip}%
\pgfsetbuttcap%
\pgfsetroundjoin%
\definecolor{currentfill}{rgb}{1.000000,1.000000,1.000000}%
\pgfsetfillcolor{currentfill}%
\pgfsetlinewidth{1.003750pt}%
\definecolor{currentstroke}{rgb}{0.000000,0.000000,0.000000}%
\pgfsetstrokecolor{currentstroke}%
\pgfsetdash{}{0pt}%
\pgfpathmoveto{\pgfqpoint{3.858085in}{1.799540in}}%
\pgfpathcurveto{\pgfqpoint{3.869135in}{1.799540in}}{\pgfqpoint{3.879734in}{1.803931in}}{\pgfqpoint{3.887547in}{1.811744in}}%
\pgfpathcurveto{\pgfqpoint{3.895361in}{1.819558in}}{\pgfqpoint{3.899751in}{1.830157in}}{\pgfqpoint{3.899751in}{1.841207in}}%
\pgfpathcurveto{\pgfqpoint{3.899751in}{1.852257in}}{\pgfqpoint{3.895361in}{1.862856in}}{\pgfqpoint{3.887547in}{1.870670in}}%
\pgfpathcurveto{\pgfqpoint{3.879734in}{1.878484in}}{\pgfqpoint{3.869135in}{1.882874in}}{\pgfqpoint{3.858085in}{1.882874in}}%
\pgfpathcurveto{\pgfqpoint{3.847034in}{1.882874in}}{\pgfqpoint{3.836435in}{1.878484in}}{\pgfqpoint{3.828622in}{1.870670in}}%
\pgfpathcurveto{\pgfqpoint{3.820808in}{1.862856in}}{\pgfqpoint{3.816418in}{1.852257in}}{\pgfqpoint{3.816418in}{1.841207in}}%
\pgfpathcurveto{\pgfqpoint{3.816418in}{1.830157in}}{\pgfqpoint{3.820808in}{1.819558in}}{\pgfqpoint{3.828622in}{1.811744in}}%
\pgfpathcurveto{\pgfqpoint{3.836435in}{1.803931in}}{\pgfqpoint{3.847034in}{1.799540in}}{\pgfqpoint{3.858085in}{1.799540in}}%
\pgfpathclose%
\pgfusepath{stroke,fill}%
\end{pgfscope}%
\begin{pgfscope}%
\pgfpathrectangle{\pgfqpoint{0.564660in}{0.521603in}}{\pgfqpoint{4.650000in}{3.020000in}}%
\pgfusepath{clip}%
\pgfsetbuttcap%
\pgfsetroundjoin%
\definecolor{currentfill}{rgb}{1.000000,1.000000,1.000000}%
\pgfsetfillcolor{currentfill}%
\pgfsetlinewidth{1.003750pt}%
\definecolor{currentstroke}{rgb}{0.000000,0.000000,0.000000}%
\pgfsetstrokecolor{currentstroke}%
\pgfsetdash{}{0pt}%
\pgfpathmoveto{\pgfqpoint{4.993344in}{2.156424in}}%
\pgfpathcurveto{\pgfqpoint{5.004394in}{2.156424in}}{\pgfqpoint{5.014993in}{2.160815in}}{\pgfqpoint{5.022807in}{2.168628in}}%
\pgfpathcurveto{\pgfqpoint{5.030620in}{2.176442in}}{\pgfqpoint{5.035010in}{2.187041in}}{\pgfqpoint{5.035010in}{2.198091in}}%
\pgfpathcurveto{\pgfqpoint{5.035010in}{2.209141in}}{\pgfqpoint{5.030620in}{2.219740in}}{\pgfqpoint{5.022807in}{2.227554in}}%
\pgfpathcurveto{\pgfqpoint{5.014993in}{2.235367in}}{\pgfqpoint{5.004394in}{2.239758in}}{\pgfqpoint{4.993344in}{2.239758in}}%
\pgfpathcurveto{\pgfqpoint{4.982294in}{2.239758in}}{\pgfqpoint{4.971695in}{2.235367in}}{\pgfqpoint{4.963881in}{2.227554in}}%
\pgfpathcurveto{\pgfqpoint{4.956067in}{2.219740in}}{\pgfqpoint{4.951677in}{2.209141in}}{\pgfqpoint{4.951677in}{2.198091in}}%
\pgfpathcurveto{\pgfqpoint{4.951677in}{2.187041in}}{\pgfqpoint{4.956067in}{2.176442in}}{\pgfqpoint{4.963881in}{2.168628in}}%
\pgfpathcurveto{\pgfqpoint{4.971695in}{2.160815in}}{\pgfqpoint{4.982294in}{2.156424in}}{\pgfqpoint{4.993344in}{2.156424in}}%
\pgfpathclose%
\pgfusepath{stroke,fill}%
\end{pgfscope}%
\begin{pgfscope}%
\pgfpathrectangle{\pgfqpoint{0.564660in}{0.521603in}}{\pgfqpoint{4.650000in}{3.020000in}}%
\pgfusepath{clip}%
\pgfsetbuttcap%
\pgfsetroundjoin%
\definecolor{currentfill}{rgb}{1.000000,1.000000,1.000000}%
\pgfsetfillcolor{currentfill}%
\pgfsetlinewidth{1.003750pt}%
\definecolor{currentstroke}{rgb}{0.000000,0.000000,0.000000}%
\pgfsetstrokecolor{currentstroke}%
\pgfsetdash{}{0pt}%
\pgfpathmoveto{\pgfqpoint{1.914828in}{1.195437in}}%
\pgfpathcurveto{\pgfqpoint{1.925878in}{1.195437in}}{\pgfqpoint{1.936477in}{1.199828in}}{\pgfqpoint{1.944290in}{1.207641in}}%
\pgfpathcurveto{\pgfqpoint{1.952104in}{1.215455in}}{\pgfqpoint{1.956494in}{1.226054in}}{\pgfqpoint{1.956494in}{1.237104in}}%
\pgfpathcurveto{\pgfqpoint{1.956494in}{1.248154in}}{\pgfqpoint{1.952104in}{1.258753in}}{\pgfqpoint{1.944290in}{1.266567in}}%
\pgfpathcurveto{\pgfqpoint{1.936477in}{1.274380in}}{\pgfqpoint{1.925878in}{1.278771in}}{\pgfqpoint{1.914828in}{1.278771in}}%
\pgfpathcurveto{\pgfqpoint{1.903777in}{1.278771in}}{\pgfqpoint{1.893178in}{1.274380in}}{\pgfqpoint{1.885365in}{1.266567in}}%
\pgfpathcurveto{\pgfqpoint{1.877551in}{1.258753in}}{\pgfqpoint{1.873161in}{1.248154in}}{\pgfqpoint{1.873161in}{1.237104in}}%
\pgfpathcurveto{\pgfqpoint{1.873161in}{1.226054in}}{\pgfqpoint{1.877551in}{1.215455in}}{\pgfqpoint{1.885365in}{1.207641in}}%
\pgfpathcurveto{\pgfqpoint{1.893178in}{1.199828in}}{\pgfqpoint{1.903777in}{1.195437in}}{\pgfqpoint{1.914828in}{1.195437in}}%
\pgfpathclose%
\pgfusepath{stroke,fill}%
\end{pgfscope}%
\begin{pgfscope}%
\pgfpathrectangle{\pgfqpoint{0.564660in}{0.521603in}}{\pgfqpoint{4.650000in}{3.020000in}}%
\pgfusepath{clip}%
\pgfsetbuttcap%
\pgfsetroundjoin%
\definecolor{currentfill}{rgb}{1.000000,1.000000,1.000000}%
\pgfsetfillcolor{currentfill}%
\pgfsetlinewidth{1.003750pt}%
\definecolor{currentstroke}{rgb}{0.000000,0.000000,0.000000}%
\pgfsetstrokecolor{currentstroke}%
\pgfsetdash{}{0pt}%
\pgfpathmoveto{\pgfqpoint{2.773519in}{3.358147in}}%
\pgfpathcurveto{\pgfqpoint{2.784569in}{3.358147in}}{\pgfqpoint{2.795168in}{3.362537in}}{\pgfqpoint{2.802981in}{3.370350in}}%
\pgfpathcurveto{\pgfqpoint{2.810795in}{3.378164in}}{\pgfqpoint{2.815185in}{3.388763in}}{\pgfqpoint{2.815185in}{3.399813in}}%
\pgfpathcurveto{\pgfqpoint{2.815185in}{3.410863in}}{\pgfqpoint{2.810795in}{3.421462in}}{\pgfqpoint{2.802981in}{3.429276in}}%
\pgfpathcurveto{\pgfqpoint{2.795168in}{3.437090in}}{\pgfqpoint{2.784569in}{3.441480in}}{\pgfqpoint{2.773519in}{3.441480in}}%
\pgfpathcurveto{\pgfqpoint{2.762469in}{3.441480in}}{\pgfqpoint{2.751870in}{3.437090in}}{\pgfqpoint{2.744056in}{3.429276in}}%
\pgfpathcurveto{\pgfqpoint{2.736242in}{3.421462in}}{\pgfqpoint{2.731852in}{3.410863in}}{\pgfqpoint{2.731852in}{3.399813in}}%
\pgfpathcurveto{\pgfqpoint{2.731852in}{3.388763in}}{\pgfqpoint{2.736242in}{3.378164in}}{\pgfqpoint{2.744056in}{3.370350in}}%
\pgfpathcurveto{\pgfqpoint{2.751870in}{3.362537in}}{\pgfqpoint{2.762469in}{3.358147in}}{\pgfqpoint{2.773519in}{3.358147in}}%
\pgfpathclose%
\pgfusepath{stroke,fill}%
\end{pgfscope}%
\begin{pgfscope}%
\pgfpathrectangle{\pgfqpoint{0.564660in}{0.521603in}}{\pgfqpoint{4.650000in}{3.020000in}}%
\pgfusepath{clip}%
\pgfsetbuttcap%
\pgfsetroundjoin%
\definecolor{currentfill}{rgb}{1.000000,1.000000,1.000000}%
\pgfsetfillcolor{currentfill}%
\pgfsetlinewidth{1.003750pt}%
\definecolor{currentstroke}{rgb}{0.000000,0.000000,0.000000}%
\pgfsetstrokecolor{currentstroke}%
\pgfsetdash{}{0pt}%
\pgfpathmoveto{\pgfqpoint{2.234616in}{0.983815in}}%
\pgfpathcurveto{\pgfqpoint{2.245666in}{0.983815in}}{\pgfqpoint{2.256265in}{0.988205in}}{\pgfqpoint{2.264079in}{0.996018in}}%
\pgfpathcurveto{\pgfqpoint{2.271892in}{1.003832in}}{\pgfqpoint{2.276283in}{1.014431in}}{\pgfqpoint{2.276283in}{1.025481in}}%
\pgfpathcurveto{\pgfqpoint{2.276283in}{1.036531in}}{\pgfqpoint{2.271892in}{1.047130in}}{\pgfqpoint{2.264079in}{1.054944in}}%
\pgfpathcurveto{\pgfqpoint{2.256265in}{1.062758in}}{\pgfqpoint{2.245666in}{1.067148in}}{\pgfqpoint{2.234616in}{1.067148in}}%
\pgfpathcurveto{\pgfqpoint{2.223566in}{1.067148in}}{\pgfqpoint{2.212967in}{1.062758in}}{\pgfqpoint{2.205153in}{1.054944in}}%
\pgfpathcurveto{\pgfqpoint{2.197340in}{1.047130in}}{\pgfqpoint{2.192949in}{1.036531in}}{\pgfqpoint{2.192949in}{1.025481in}}%
\pgfpathcurveto{\pgfqpoint{2.192949in}{1.014431in}}{\pgfqpoint{2.197340in}{1.003832in}}{\pgfqpoint{2.205153in}{0.996018in}}%
\pgfpathcurveto{\pgfqpoint{2.212967in}{0.988205in}}{\pgfqpoint{2.223566in}{0.983815in}}{\pgfqpoint{2.234616in}{0.983815in}}%
\pgfpathclose%
\pgfusepath{stroke,fill}%
\end{pgfscope}%
\begin{pgfscope}%
\pgfpathrectangle{\pgfqpoint{0.564660in}{0.521603in}}{\pgfqpoint{4.650000in}{3.020000in}}%
\pgfusepath{clip}%
\pgfsetbuttcap%
\pgfsetroundjoin%
\definecolor{currentfill}{rgb}{1.000000,1.000000,1.000000}%
\pgfsetfillcolor{currentfill}%
\pgfsetlinewidth{1.003750pt}%
\definecolor{currentstroke}{rgb}{0.000000,0.000000,0.000000}%
\pgfsetstrokecolor{currentstroke}%
\pgfsetdash{}{0pt}%
\pgfpathmoveto{\pgfqpoint{0.819079in}{1.480427in}}%
\pgfpathcurveto{\pgfqpoint{0.830129in}{1.480427in}}{\pgfqpoint{0.840728in}{1.484817in}}{\pgfqpoint{0.848542in}{1.492631in}}%
\pgfpathcurveto{\pgfqpoint{0.856355in}{1.500444in}}{\pgfqpoint{0.860746in}{1.511043in}}{\pgfqpoint{0.860746in}{1.522093in}}%
\pgfpathcurveto{\pgfqpoint{0.860746in}{1.533144in}}{\pgfqpoint{0.856355in}{1.543743in}}{\pgfqpoint{0.848542in}{1.551556in}}%
\pgfpathcurveto{\pgfqpoint{0.840728in}{1.559370in}}{\pgfqpoint{0.830129in}{1.563760in}}{\pgfqpoint{0.819079in}{1.563760in}}%
\pgfpathcurveto{\pgfqpoint{0.808029in}{1.563760in}}{\pgfqpoint{0.797430in}{1.559370in}}{\pgfqpoint{0.789616in}{1.551556in}}%
\pgfpathcurveto{\pgfqpoint{0.781803in}{1.543743in}}{\pgfqpoint{0.777412in}{1.533144in}}{\pgfqpoint{0.777412in}{1.522093in}}%
\pgfpathcurveto{\pgfqpoint{0.777412in}{1.511043in}}{\pgfqpoint{0.781803in}{1.500444in}}{\pgfqpoint{0.789616in}{1.492631in}}%
\pgfpathcurveto{\pgfqpoint{0.797430in}{1.484817in}}{\pgfqpoint{0.808029in}{1.480427in}}{\pgfqpoint{0.819079in}{1.480427in}}%
\pgfpathclose%
\pgfusepath{stroke,fill}%
\end{pgfscope}%
\begin{pgfscope}%
\pgfpathrectangle{\pgfqpoint{0.564660in}{0.521603in}}{\pgfqpoint{4.650000in}{3.020000in}}%
\pgfusepath{clip}%
\pgfsetbuttcap%
\pgfsetroundjoin%
\definecolor{currentfill}{rgb}{1.000000,1.000000,1.000000}%
\pgfsetfillcolor{currentfill}%
\pgfsetlinewidth{1.003750pt}%
\definecolor{currentstroke}{rgb}{0.000000,0.000000,0.000000}%
\pgfsetstrokecolor{currentstroke}%
\pgfsetdash{}{0pt}%
\pgfpathmoveto{\pgfqpoint{1.363796in}{0.692093in}}%
\pgfpathcurveto{\pgfqpoint{1.374846in}{0.692093in}}{\pgfqpoint{1.385445in}{0.696483in}}{\pgfqpoint{1.393259in}{0.704297in}}%
\pgfpathcurveto{\pgfqpoint{1.401072in}{0.712110in}}{\pgfqpoint{1.405462in}{0.722709in}}{\pgfqpoint{1.405462in}{0.733759in}}%
\pgfpathcurveto{\pgfqpoint{1.405462in}{0.744810in}}{\pgfqpoint{1.401072in}{0.755409in}}{\pgfqpoint{1.393259in}{0.763222in}}%
\pgfpathcurveto{\pgfqpoint{1.385445in}{0.771036in}}{\pgfqpoint{1.374846in}{0.775426in}}{\pgfqpoint{1.363796in}{0.775426in}}%
\pgfpathcurveto{\pgfqpoint{1.352746in}{0.775426in}}{\pgfqpoint{1.342147in}{0.771036in}}{\pgfqpoint{1.334333in}{0.763222in}}%
\pgfpathcurveto{\pgfqpoint{1.326519in}{0.755409in}}{\pgfqpoint{1.322129in}{0.744810in}}{\pgfqpoint{1.322129in}{0.733759in}}%
\pgfpathcurveto{\pgfqpoint{1.322129in}{0.722709in}}{\pgfqpoint{1.326519in}{0.712110in}}{\pgfqpoint{1.334333in}{0.704297in}}%
\pgfpathcurveto{\pgfqpoint{1.342147in}{0.696483in}}{\pgfqpoint{1.352746in}{0.692093in}}{\pgfqpoint{1.363796in}{0.692093in}}%
\pgfpathclose%
\pgfusepath{stroke,fill}%
\end{pgfscope}%
\begin{pgfscope}%
\pgfpathrectangle{\pgfqpoint{0.564660in}{0.521603in}}{\pgfqpoint{4.650000in}{3.020000in}}%
\pgfusepath{clip}%
\pgfsetbuttcap%
\pgfsetroundjoin%
\definecolor{currentfill}{rgb}{1.000000,1.000000,1.000000}%
\pgfsetfillcolor{currentfill}%
\pgfsetlinewidth{1.003750pt}%
\definecolor{currentstroke}{rgb}{0.000000,0.000000,0.000000}%
\pgfsetstrokecolor{currentstroke}%
\pgfsetdash{}{0pt}%
\pgfpathmoveto{\pgfqpoint{1.096147in}{0.621727in}}%
\pgfpathcurveto{\pgfqpoint{1.107197in}{0.621727in}}{\pgfqpoint{1.117796in}{0.626117in}}{\pgfqpoint{1.125609in}{0.633931in}}%
\pgfpathcurveto{\pgfqpoint{1.133423in}{0.641744in}}{\pgfqpoint{1.137813in}{0.652343in}}{\pgfqpoint{1.137813in}{0.663393in}}%
\pgfpathcurveto{\pgfqpoint{1.137813in}{0.674444in}}{\pgfqpoint{1.133423in}{0.685043in}}{\pgfqpoint{1.125609in}{0.692856in}}%
\pgfpathcurveto{\pgfqpoint{1.117796in}{0.700670in}}{\pgfqpoint{1.107197in}{0.705060in}}{\pgfqpoint{1.096147in}{0.705060in}}%
\pgfpathcurveto{\pgfqpoint{1.085096in}{0.705060in}}{\pgfqpoint{1.074497in}{0.700670in}}{\pgfqpoint{1.066684in}{0.692856in}}%
\pgfpathcurveto{\pgfqpoint{1.058870in}{0.685043in}}{\pgfqpoint{1.054480in}{0.674444in}}{\pgfqpoint{1.054480in}{0.663393in}}%
\pgfpathcurveto{\pgfqpoint{1.054480in}{0.652343in}}{\pgfqpoint{1.058870in}{0.641744in}}{\pgfqpoint{1.066684in}{0.633931in}}%
\pgfpathcurveto{\pgfqpoint{1.074497in}{0.626117in}}{\pgfqpoint{1.085096in}{0.621727in}}{\pgfqpoint{1.096147in}{0.621727in}}%
\pgfpathclose%
\pgfusepath{stroke,fill}%
\end{pgfscope}%
\begin{pgfscope}%
\pgfpathrectangle{\pgfqpoint{0.564660in}{0.521603in}}{\pgfqpoint{4.650000in}{3.020000in}}%
\pgfusepath{clip}%
\pgfsetbuttcap%
\pgfsetroundjoin%
\definecolor{currentfill}{rgb}{1.000000,1.000000,1.000000}%
\pgfsetfillcolor{currentfill}%
\pgfsetlinewidth{1.003750pt}%
\definecolor{currentstroke}{rgb}{0.000000,0.000000,0.000000}%
\pgfsetstrokecolor{currentstroke}%
\pgfsetdash{}{0pt}%
\pgfpathmoveto{\pgfqpoint{0.887904in}{1.305315in}}%
\pgfpathcurveto{\pgfqpoint{0.898954in}{1.305315in}}{\pgfqpoint{0.909553in}{1.309705in}}{\pgfqpoint{0.917367in}{1.317519in}}%
\pgfpathcurveto{\pgfqpoint{0.925180in}{1.325332in}}{\pgfqpoint{0.929570in}{1.335932in}}{\pgfqpoint{0.929570in}{1.346982in}}%
\pgfpathcurveto{\pgfqpoint{0.929570in}{1.358032in}}{\pgfqpoint{0.925180in}{1.368631in}}{\pgfqpoint{0.917367in}{1.376444in}}%
\pgfpathcurveto{\pgfqpoint{0.909553in}{1.384258in}}{\pgfqpoint{0.898954in}{1.388648in}}{\pgfqpoint{0.887904in}{1.388648in}}%
\pgfpathcurveto{\pgfqpoint{0.876854in}{1.388648in}}{\pgfqpoint{0.866255in}{1.384258in}}{\pgfqpoint{0.858441in}{1.376444in}}%
\pgfpathcurveto{\pgfqpoint{0.850627in}{1.368631in}}{\pgfqpoint{0.846237in}{1.358032in}}{\pgfqpoint{0.846237in}{1.346982in}}%
\pgfpathcurveto{\pgfqpoint{0.846237in}{1.335932in}}{\pgfqpoint{0.850627in}{1.325332in}}{\pgfqpoint{0.858441in}{1.317519in}}%
\pgfpathcurveto{\pgfqpoint{0.866255in}{1.309705in}}{\pgfqpoint{0.876854in}{1.305315in}}{\pgfqpoint{0.887904in}{1.305315in}}%
\pgfpathclose%
\pgfusepath{stroke,fill}%
\end{pgfscope}%
\begin{pgfscope}%
\pgfpathrectangle{\pgfqpoint{0.564660in}{0.521603in}}{\pgfqpoint{4.650000in}{3.020000in}}%
\pgfusepath{clip}%
\pgfsetbuttcap%
\pgfsetroundjoin%
\definecolor{currentfill}{rgb}{1.000000,1.000000,1.000000}%
\pgfsetfillcolor{currentfill}%
\pgfsetlinewidth{1.003750pt}%
\definecolor{currentstroke}{rgb}{0.000000,0.000000,0.000000}%
\pgfsetstrokecolor{currentstroke}%
\pgfsetdash{}{0pt}%
\pgfpathmoveto{\pgfqpoint{2.143771in}{2.158900in}}%
\pgfpathcurveto{\pgfqpoint{2.154821in}{2.158900in}}{\pgfqpoint{2.165420in}{2.163290in}}{\pgfqpoint{2.173234in}{2.171104in}}%
\pgfpathcurveto{\pgfqpoint{2.181047in}{2.178918in}}{\pgfqpoint{2.185438in}{2.189517in}}{\pgfqpoint{2.185438in}{2.200567in}}%
\pgfpathcurveto{\pgfqpoint{2.185438in}{2.211617in}}{\pgfqpoint{2.181047in}{2.222216in}}{\pgfqpoint{2.173234in}{2.230029in}}%
\pgfpathcurveto{\pgfqpoint{2.165420in}{2.237843in}}{\pgfqpoint{2.154821in}{2.242233in}}{\pgfqpoint{2.143771in}{2.242233in}}%
\pgfpathcurveto{\pgfqpoint{2.132721in}{2.242233in}}{\pgfqpoint{2.122122in}{2.237843in}}{\pgfqpoint{2.114308in}{2.230029in}}%
\pgfpathcurveto{\pgfqpoint{2.106494in}{2.222216in}}{\pgfqpoint{2.102104in}{2.211617in}}{\pgfqpoint{2.102104in}{2.200567in}}%
\pgfpathcurveto{\pgfqpoint{2.102104in}{2.189517in}}{\pgfqpoint{2.106494in}{2.178918in}}{\pgfqpoint{2.114308in}{2.171104in}}%
\pgfpathcurveto{\pgfqpoint{2.122122in}{2.163290in}}{\pgfqpoint{2.132721in}{2.158900in}}{\pgfqpoint{2.143771in}{2.158900in}}%
\pgfpathclose%
\pgfusepath{stroke,fill}%
\end{pgfscope}%
\begin{pgfscope}%
\pgfpathrectangle{\pgfqpoint{0.564660in}{0.521603in}}{\pgfqpoint{4.650000in}{3.020000in}}%
\pgfusepath{clip}%
\pgfsetbuttcap%
\pgfsetroundjoin%
\definecolor{currentfill}{rgb}{1.000000,1.000000,1.000000}%
\pgfsetfillcolor{currentfill}%
\pgfsetlinewidth{1.003750pt}%
\definecolor{currentstroke}{rgb}{0.000000,0.000000,0.000000}%
\pgfsetstrokecolor{currentstroke}%
\pgfsetdash{}{0pt}%
\pgfpathmoveto{\pgfqpoint{1.086864in}{0.741510in}}%
\pgfpathcurveto{\pgfqpoint{1.097914in}{0.741510in}}{\pgfqpoint{1.108513in}{0.745901in}}{\pgfqpoint{1.116327in}{0.753714in}}%
\pgfpathcurveto{\pgfqpoint{1.124141in}{0.761528in}}{\pgfqpoint{1.128531in}{0.772127in}}{\pgfqpoint{1.128531in}{0.783177in}}%
\pgfpathcurveto{\pgfqpoint{1.128531in}{0.794227in}}{\pgfqpoint{1.124141in}{0.804826in}}{\pgfqpoint{1.116327in}{0.812640in}}%
\pgfpathcurveto{\pgfqpoint{1.108513in}{0.820453in}}{\pgfqpoint{1.097914in}{0.824844in}}{\pgfqpoint{1.086864in}{0.824844in}}%
\pgfpathcurveto{\pgfqpoint{1.075814in}{0.824844in}}{\pgfqpoint{1.065215in}{0.820453in}}{\pgfqpoint{1.057401in}{0.812640in}}%
\pgfpathcurveto{\pgfqpoint{1.049588in}{0.804826in}}{\pgfqpoint{1.045197in}{0.794227in}}{\pgfqpoint{1.045197in}{0.783177in}}%
\pgfpathcurveto{\pgfqpoint{1.045197in}{0.772127in}}{\pgfqpoint{1.049588in}{0.761528in}}{\pgfqpoint{1.057401in}{0.753714in}}%
\pgfpathcurveto{\pgfqpoint{1.065215in}{0.745901in}}{\pgfqpoint{1.075814in}{0.741510in}}{\pgfqpoint{1.086864in}{0.741510in}}%
\pgfpathclose%
\pgfusepath{stroke,fill}%
\end{pgfscope}%
\begin{pgfscope}%
\pgfpathrectangle{\pgfqpoint{0.564660in}{0.521603in}}{\pgfqpoint{4.650000in}{3.020000in}}%
\pgfusepath{clip}%
\pgfsetbuttcap%
\pgfsetroundjoin%
\definecolor{currentfill}{rgb}{1.000000,1.000000,1.000000}%
\pgfsetfillcolor{currentfill}%
\pgfsetlinewidth{1.003750pt}%
\definecolor{currentstroke}{rgb}{0.000000,0.000000,0.000000}%
\pgfsetstrokecolor{currentstroke}%
\pgfsetdash{}{0pt}%
\pgfpathmoveto{\pgfqpoint{0.785977in}{1.118987in}}%
\pgfpathcurveto{\pgfqpoint{0.797027in}{1.118987in}}{\pgfqpoint{0.807626in}{1.123377in}}{\pgfqpoint{0.815440in}{1.131190in}}%
\pgfpathcurveto{\pgfqpoint{0.823253in}{1.139004in}}{\pgfqpoint{0.827643in}{1.149603in}}{\pgfqpoint{0.827643in}{1.160653in}}%
\pgfpathcurveto{\pgfqpoint{0.827643in}{1.171703in}}{\pgfqpoint{0.823253in}{1.182302in}}{\pgfqpoint{0.815440in}{1.190116in}}%
\pgfpathcurveto{\pgfqpoint{0.807626in}{1.197930in}}{\pgfqpoint{0.797027in}{1.202320in}}{\pgfqpoint{0.785977in}{1.202320in}}%
\pgfpathcurveto{\pgfqpoint{0.774927in}{1.202320in}}{\pgfqpoint{0.764328in}{1.197930in}}{\pgfqpoint{0.756514in}{1.190116in}}%
\pgfpathcurveto{\pgfqpoint{0.748700in}{1.182302in}}{\pgfqpoint{0.744310in}{1.171703in}}{\pgfqpoint{0.744310in}{1.160653in}}%
\pgfpathcurveto{\pgfqpoint{0.744310in}{1.149603in}}{\pgfqpoint{0.748700in}{1.139004in}}{\pgfqpoint{0.756514in}{1.131190in}}%
\pgfpathcurveto{\pgfqpoint{0.764328in}{1.123377in}}{\pgfqpoint{0.774927in}{1.118987in}}{\pgfqpoint{0.785977in}{1.118987in}}%
\pgfpathclose%
\pgfusepath{stroke,fill}%
\end{pgfscope}%
\begin{pgfscope}%
\pgfpathrectangle{\pgfqpoint{0.564660in}{0.521603in}}{\pgfqpoint{4.650000in}{3.020000in}}%
\pgfusepath{clip}%
\pgfsetbuttcap%
\pgfsetroundjoin%
\definecolor{currentfill}{rgb}{1.000000,1.000000,1.000000}%
\pgfsetfillcolor{currentfill}%
\pgfsetlinewidth{1.003750pt}%
\definecolor{currentstroke}{rgb}{0.000000,0.000000,0.000000}%
\pgfsetstrokecolor{currentstroke}%
\pgfsetdash{}{0pt}%
\pgfpathmoveto{\pgfqpoint{2.036724in}{2.941966in}}%
\pgfpathcurveto{\pgfqpoint{2.047774in}{2.941966in}}{\pgfqpoint{2.058373in}{2.946356in}}{\pgfqpoint{2.066187in}{2.954170in}}%
\pgfpathcurveto{\pgfqpoint{2.074000in}{2.961983in}}{\pgfqpoint{2.078390in}{2.972582in}}{\pgfqpoint{2.078390in}{2.983632in}}%
\pgfpathcurveto{\pgfqpoint{2.078390in}{2.994682in}}{\pgfqpoint{2.074000in}{3.005282in}}{\pgfqpoint{2.066187in}{3.013095in}}%
\pgfpathcurveto{\pgfqpoint{2.058373in}{3.020909in}}{\pgfqpoint{2.047774in}{3.025299in}}{\pgfqpoint{2.036724in}{3.025299in}}%
\pgfpathcurveto{\pgfqpoint{2.025674in}{3.025299in}}{\pgfqpoint{2.015075in}{3.020909in}}{\pgfqpoint{2.007261in}{3.013095in}}%
\pgfpathcurveto{\pgfqpoint{1.999447in}{3.005282in}}{\pgfqpoint{1.995057in}{2.994682in}}{\pgfqpoint{1.995057in}{2.983632in}}%
\pgfpathcurveto{\pgfqpoint{1.995057in}{2.972582in}}{\pgfqpoint{1.999447in}{2.961983in}}{\pgfqpoint{2.007261in}{2.954170in}}%
\pgfpathcurveto{\pgfqpoint{2.015075in}{2.946356in}}{\pgfqpoint{2.025674in}{2.941966in}}{\pgfqpoint{2.036724in}{2.941966in}}%
\pgfpathclose%
\pgfusepath{stroke,fill}%
\end{pgfscope}%
\begin{pgfscope}%
\pgfpathrectangle{\pgfqpoint{0.564660in}{0.521603in}}{\pgfqpoint{4.650000in}{3.020000in}}%
\pgfusepath{clip}%
\pgfsetbuttcap%
\pgfsetroundjoin%
\definecolor{currentfill}{rgb}{1.000000,1.000000,1.000000}%
\pgfsetfillcolor{currentfill}%
\pgfsetlinewidth{1.003750pt}%
\definecolor{currentstroke}{rgb}{0.000000,0.000000,0.000000}%
\pgfsetstrokecolor{currentstroke}%
\pgfsetdash{}{0pt}%
\pgfpathmoveto{\pgfqpoint{1.309364in}{1.315672in}}%
\pgfpathcurveto{\pgfqpoint{1.320414in}{1.315672in}}{\pgfqpoint{1.331013in}{1.320062in}}{\pgfqpoint{1.338826in}{1.327876in}}%
\pgfpathcurveto{\pgfqpoint{1.346640in}{1.335690in}}{\pgfqpoint{1.351030in}{1.346289in}}{\pgfqpoint{1.351030in}{1.357339in}}%
\pgfpathcurveto{\pgfqpoint{1.351030in}{1.368389in}}{\pgfqpoint{1.346640in}{1.378988in}}{\pgfqpoint{1.338826in}{1.386802in}}%
\pgfpathcurveto{\pgfqpoint{1.331013in}{1.394615in}}{\pgfqpoint{1.320414in}{1.399006in}}{\pgfqpoint{1.309364in}{1.399006in}}%
\pgfpathcurveto{\pgfqpoint{1.298314in}{1.399006in}}{\pgfqpoint{1.287715in}{1.394615in}}{\pgfqpoint{1.279901in}{1.386802in}}%
\pgfpathcurveto{\pgfqpoint{1.272087in}{1.378988in}}{\pgfqpoint{1.267697in}{1.368389in}}{\pgfqpoint{1.267697in}{1.357339in}}%
\pgfpathcurveto{\pgfqpoint{1.267697in}{1.346289in}}{\pgfqpoint{1.272087in}{1.335690in}}{\pgfqpoint{1.279901in}{1.327876in}}%
\pgfpathcurveto{\pgfqpoint{1.287715in}{1.320062in}}{\pgfqpoint{1.298314in}{1.315672in}}{\pgfqpoint{1.309364in}{1.315672in}}%
\pgfpathclose%
\pgfusepath{stroke,fill}%
\end{pgfscope}%
\begin{pgfscope}%
\pgfsetbuttcap%
\pgfsetroundjoin%
\definecolor{currentfill}{rgb}{0.000000,0.000000,0.000000}%
\pgfsetfillcolor{currentfill}%
\pgfsetlinewidth{0.803000pt}%
\definecolor{currentstroke}{rgb}{0.000000,0.000000,0.000000}%
\pgfsetstrokecolor{currentstroke}%
\pgfsetdash{}{0pt}%
\pgfsys@defobject{currentmarker}{\pgfqpoint{0.000000in}{-0.048611in}}{\pgfqpoint{0.000000in}{0.000000in}}{%
\pgfpathmoveto{\pgfqpoint{0.000000in}{0.000000in}}%
\pgfpathlineto{\pgfqpoint{0.000000in}{-0.048611in}}%
\pgfusepath{stroke,fill}%
}%
\begin{pgfscope}%
\pgfsys@transformshift{0.667447in}{0.521603in}%
\pgfsys@useobject{currentmarker}{}%
\end{pgfscope}%
\end{pgfscope}%
\begin{pgfscope}%
\definecolor{textcolor}{rgb}{0.000000,0.000000,0.000000}%
\pgfsetstrokecolor{textcolor}%
\pgfsetfillcolor{textcolor}%
\pgftext[x=0.667447in,y=0.424381in,,top]{\color{textcolor}\rmfamily\fontsize{10.000000}{12.000000}\selectfont \(\displaystyle 0\)}%
\end{pgfscope}%
\begin{pgfscope}%
\pgfsetbuttcap%
\pgfsetroundjoin%
\definecolor{currentfill}{rgb}{0.000000,0.000000,0.000000}%
\pgfsetfillcolor{currentfill}%
\pgfsetlinewidth{0.803000pt}%
\definecolor{currentstroke}{rgb}{0.000000,0.000000,0.000000}%
\pgfsetstrokecolor{currentstroke}%
\pgfsetdash{}{0pt}%
\pgfsys@defobject{currentmarker}{\pgfqpoint{0.000000in}{-0.048611in}}{\pgfqpoint{0.000000in}{0.000000in}}{%
\pgfpathmoveto{\pgfqpoint{0.000000in}{0.000000in}}%
\pgfpathlineto{\pgfqpoint{0.000000in}{-0.048611in}}%
\pgfusepath{stroke,fill}%
}%
\begin{pgfscope}%
\pgfsys@transformshift{1.348050in}{0.521603in}%
\pgfsys@useobject{currentmarker}{}%
\end{pgfscope}%
\end{pgfscope}%
\begin{pgfscope}%
\definecolor{textcolor}{rgb}{0.000000,0.000000,0.000000}%
\pgfsetstrokecolor{textcolor}%
\pgfsetfillcolor{textcolor}%
\pgftext[x=1.348050in,y=0.424381in,,top]{\color{textcolor}\rmfamily\fontsize{10.000000}{12.000000}\selectfont \(\displaystyle 5\)}%
\end{pgfscope}%
\begin{pgfscope}%
\pgfsetbuttcap%
\pgfsetroundjoin%
\definecolor{currentfill}{rgb}{0.000000,0.000000,0.000000}%
\pgfsetfillcolor{currentfill}%
\pgfsetlinewidth{0.803000pt}%
\definecolor{currentstroke}{rgb}{0.000000,0.000000,0.000000}%
\pgfsetstrokecolor{currentstroke}%
\pgfsetdash{}{0pt}%
\pgfsys@defobject{currentmarker}{\pgfqpoint{0.000000in}{-0.048611in}}{\pgfqpoint{0.000000in}{0.000000in}}{%
\pgfpathmoveto{\pgfqpoint{0.000000in}{0.000000in}}%
\pgfpathlineto{\pgfqpoint{0.000000in}{-0.048611in}}%
\pgfusepath{stroke,fill}%
}%
\begin{pgfscope}%
\pgfsys@transformshift{2.028653in}{0.521603in}%
\pgfsys@useobject{currentmarker}{}%
\end{pgfscope}%
\end{pgfscope}%
\begin{pgfscope}%
\definecolor{textcolor}{rgb}{0.000000,0.000000,0.000000}%
\pgfsetstrokecolor{textcolor}%
\pgfsetfillcolor{textcolor}%
\pgftext[x=2.028653in,y=0.424381in,,top]{\color{textcolor}\rmfamily\fontsize{10.000000}{12.000000}\selectfont \(\displaystyle 10\)}%
\end{pgfscope}%
\begin{pgfscope}%
\pgfsetbuttcap%
\pgfsetroundjoin%
\definecolor{currentfill}{rgb}{0.000000,0.000000,0.000000}%
\pgfsetfillcolor{currentfill}%
\pgfsetlinewidth{0.803000pt}%
\definecolor{currentstroke}{rgb}{0.000000,0.000000,0.000000}%
\pgfsetstrokecolor{currentstroke}%
\pgfsetdash{}{0pt}%
\pgfsys@defobject{currentmarker}{\pgfqpoint{0.000000in}{-0.048611in}}{\pgfqpoint{0.000000in}{0.000000in}}{%
\pgfpathmoveto{\pgfqpoint{0.000000in}{0.000000in}}%
\pgfpathlineto{\pgfqpoint{0.000000in}{-0.048611in}}%
\pgfusepath{stroke,fill}%
}%
\begin{pgfscope}%
\pgfsys@transformshift{2.709256in}{0.521603in}%
\pgfsys@useobject{currentmarker}{}%
\end{pgfscope}%
\end{pgfscope}%
\begin{pgfscope}%
\definecolor{textcolor}{rgb}{0.000000,0.000000,0.000000}%
\pgfsetstrokecolor{textcolor}%
\pgfsetfillcolor{textcolor}%
\pgftext[x=2.709256in,y=0.424381in,,top]{\color{textcolor}\rmfamily\fontsize{10.000000}{12.000000}\selectfont \(\displaystyle 15\)}%
\end{pgfscope}%
\begin{pgfscope}%
\pgfsetbuttcap%
\pgfsetroundjoin%
\definecolor{currentfill}{rgb}{0.000000,0.000000,0.000000}%
\pgfsetfillcolor{currentfill}%
\pgfsetlinewidth{0.803000pt}%
\definecolor{currentstroke}{rgb}{0.000000,0.000000,0.000000}%
\pgfsetstrokecolor{currentstroke}%
\pgfsetdash{}{0pt}%
\pgfsys@defobject{currentmarker}{\pgfqpoint{0.000000in}{-0.048611in}}{\pgfqpoint{0.000000in}{0.000000in}}{%
\pgfpathmoveto{\pgfqpoint{0.000000in}{0.000000in}}%
\pgfpathlineto{\pgfqpoint{0.000000in}{-0.048611in}}%
\pgfusepath{stroke,fill}%
}%
\begin{pgfscope}%
\pgfsys@transformshift{3.389859in}{0.521603in}%
\pgfsys@useobject{currentmarker}{}%
\end{pgfscope}%
\end{pgfscope}%
\begin{pgfscope}%
\definecolor{textcolor}{rgb}{0.000000,0.000000,0.000000}%
\pgfsetstrokecolor{textcolor}%
\pgfsetfillcolor{textcolor}%
\pgftext[x=3.389859in,y=0.424381in,,top]{\color{textcolor}\rmfamily\fontsize{10.000000}{12.000000}\selectfont \(\displaystyle 20\)}%
\end{pgfscope}%
\begin{pgfscope}%
\pgfsetbuttcap%
\pgfsetroundjoin%
\definecolor{currentfill}{rgb}{0.000000,0.000000,0.000000}%
\pgfsetfillcolor{currentfill}%
\pgfsetlinewidth{0.803000pt}%
\definecolor{currentstroke}{rgb}{0.000000,0.000000,0.000000}%
\pgfsetstrokecolor{currentstroke}%
\pgfsetdash{}{0pt}%
\pgfsys@defobject{currentmarker}{\pgfqpoint{0.000000in}{-0.048611in}}{\pgfqpoint{0.000000in}{0.000000in}}{%
\pgfpathmoveto{\pgfqpoint{0.000000in}{0.000000in}}%
\pgfpathlineto{\pgfqpoint{0.000000in}{-0.048611in}}%
\pgfusepath{stroke,fill}%
}%
\begin{pgfscope}%
\pgfsys@transformshift{4.070462in}{0.521603in}%
\pgfsys@useobject{currentmarker}{}%
\end{pgfscope}%
\end{pgfscope}%
\begin{pgfscope}%
\definecolor{textcolor}{rgb}{0.000000,0.000000,0.000000}%
\pgfsetstrokecolor{textcolor}%
\pgfsetfillcolor{textcolor}%
\pgftext[x=4.070462in,y=0.424381in,,top]{\color{textcolor}\rmfamily\fontsize{10.000000}{12.000000}\selectfont \(\displaystyle 25\)}%
\end{pgfscope}%
\begin{pgfscope}%
\pgfsetbuttcap%
\pgfsetroundjoin%
\definecolor{currentfill}{rgb}{0.000000,0.000000,0.000000}%
\pgfsetfillcolor{currentfill}%
\pgfsetlinewidth{0.803000pt}%
\definecolor{currentstroke}{rgb}{0.000000,0.000000,0.000000}%
\pgfsetstrokecolor{currentstroke}%
\pgfsetdash{}{0pt}%
\pgfsys@defobject{currentmarker}{\pgfqpoint{0.000000in}{-0.048611in}}{\pgfqpoint{0.000000in}{0.000000in}}{%
\pgfpathmoveto{\pgfqpoint{0.000000in}{0.000000in}}%
\pgfpathlineto{\pgfqpoint{0.000000in}{-0.048611in}}%
\pgfusepath{stroke,fill}%
}%
\begin{pgfscope}%
\pgfsys@transformshift{4.751065in}{0.521603in}%
\pgfsys@useobject{currentmarker}{}%
\end{pgfscope}%
\end{pgfscope}%
\begin{pgfscope}%
\definecolor{textcolor}{rgb}{0.000000,0.000000,0.000000}%
\pgfsetstrokecolor{textcolor}%
\pgfsetfillcolor{textcolor}%
\pgftext[x=4.751065in,y=0.424381in,,top]{\color{textcolor}\rmfamily\fontsize{10.000000}{12.000000}\selectfont \(\displaystyle 30\)}%
\end{pgfscope}%
\begin{pgfscope}%
\definecolor{textcolor}{rgb}{0.000000,0.000000,0.000000}%
\pgfsetstrokecolor{textcolor}%
\pgfsetfillcolor{textcolor}%
\pgftext[x=2.889660in,y=0.234413in,,top]{\color{textcolor}\rmfamily\fontsize{10.000000}{12.000000}\selectfont \(\displaystyle \mathbf{E}\mbox{u}\)}%
\end{pgfscope}%
\begin{pgfscope}%
\pgfsetbuttcap%
\pgfsetroundjoin%
\definecolor{currentfill}{rgb}{0.000000,0.000000,0.000000}%
\pgfsetfillcolor{currentfill}%
\pgfsetlinewidth{0.803000pt}%
\definecolor{currentstroke}{rgb}{0.000000,0.000000,0.000000}%
\pgfsetstrokecolor{currentstroke}%
\pgfsetdash{}{0pt}%
\pgfsys@defobject{currentmarker}{\pgfqpoint{-0.048611in}{0.000000in}}{\pgfqpoint{0.000000in}{0.000000in}}{%
\pgfpathmoveto{\pgfqpoint{0.000000in}{0.000000in}}%
\pgfpathlineto{\pgfqpoint{-0.048611in}{0.000000in}}%
\pgfusepath{stroke,fill}%
}%
\begin{pgfscope}%
\pgfsys@transformshift{0.564660in}{1.017529in}%
\pgfsys@useobject{currentmarker}{}%
\end{pgfscope}%
\end{pgfscope}%
\begin{pgfscope}%
\definecolor{textcolor}{rgb}{0.000000,0.000000,0.000000}%
\pgfsetstrokecolor{textcolor}%
\pgfsetfillcolor{textcolor}%
\pgftext[x=0.289968in,y=0.964768in,left,base]{\color{textcolor}\rmfamily\fontsize{10.000000}{12.000000}\selectfont \(\displaystyle 0.3\)}%
\end{pgfscope}%
\begin{pgfscope}%
\pgfsetbuttcap%
\pgfsetroundjoin%
\definecolor{currentfill}{rgb}{0.000000,0.000000,0.000000}%
\pgfsetfillcolor{currentfill}%
\pgfsetlinewidth{0.803000pt}%
\definecolor{currentstroke}{rgb}{0.000000,0.000000,0.000000}%
\pgfsetstrokecolor{currentstroke}%
\pgfsetdash{}{0pt}%
\pgfsys@defobject{currentmarker}{\pgfqpoint{-0.048611in}{0.000000in}}{\pgfqpoint{0.000000in}{0.000000in}}{%
\pgfpathmoveto{\pgfqpoint{0.000000in}{0.000000in}}%
\pgfpathlineto{\pgfqpoint{-0.048611in}{0.000000in}}%
\pgfusepath{stroke,fill}%
}%
\begin{pgfscope}%
\pgfsys@transformshift{0.564660in}{1.618847in}%
\pgfsys@useobject{currentmarker}{}%
\end{pgfscope}%
\end{pgfscope}%
\begin{pgfscope}%
\definecolor{textcolor}{rgb}{0.000000,0.000000,0.000000}%
\pgfsetstrokecolor{textcolor}%
\pgfsetfillcolor{textcolor}%
\pgftext[x=0.289968in,y=1.566085in,left,base]{\color{textcolor}\rmfamily\fontsize{10.000000}{12.000000}\selectfont \(\displaystyle 0.4\)}%
\end{pgfscope}%
\begin{pgfscope}%
\pgfsetbuttcap%
\pgfsetroundjoin%
\definecolor{currentfill}{rgb}{0.000000,0.000000,0.000000}%
\pgfsetfillcolor{currentfill}%
\pgfsetlinewidth{0.803000pt}%
\definecolor{currentstroke}{rgb}{0.000000,0.000000,0.000000}%
\pgfsetstrokecolor{currentstroke}%
\pgfsetdash{}{0pt}%
\pgfsys@defobject{currentmarker}{\pgfqpoint{-0.048611in}{0.000000in}}{\pgfqpoint{0.000000in}{0.000000in}}{%
\pgfpathmoveto{\pgfqpoint{0.000000in}{0.000000in}}%
\pgfpathlineto{\pgfqpoint{-0.048611in}{0.000000in}}%
\pgfusepath{stroke,fill}%
}%
\begin{pgfscope}%
\pgfsys@transformshift{0.564660in}{2.220164in}%
\pgfsys@useobject{currentmarker}{}%
\end{pgfscope}%
\end{pgfscope}%
\begin{pgfscope}%
\definecolor{textcolor}{rgb}{0.000000,0.000000,0.000000}%
\pgfsetstrokecolor{textcolor}%
\pgfsetfillcolor{textcolor}%
\pgftext[x=0.289968in,y=2.167403in,left,base]{\color{textcolor}\rmfamily\fontsize{10.000000}{12.000000}\selectfont \(\displaystyle 0.5\)}%
\end{pgfscope}%
\begin{pgfscope}%
\pgfsetbuttcap%
\pgfsetroundjoin%
\definecolor{currentfill}{rgb}{0.000000,0.000000,0.000000}%
\pgfsetfillcolor{currentfill}%
\pgfsetlinewidth{0.803000pt}%
\definecolor{currentstroke}{rgb}{0.000000,0.000000,0.000000}%
\pgfsetstrokecolor{currentstroke}%
\pgfsetdash{}{0pt}%
\pgfsys@defobject{currentmarker}{\pgfqpoint{-0.048611in}{0.000000in}}{\pgfqpoint{0.000000in}{0.000000in}}{%
\pgfpathmoveto{\pgfqpoint{0.000000in}{0.000000in}}%
\pgfpathlineto{\pgfqpoint{-0.048611in}{0.000000in}}%
\pgfusepath{stroke,fill}%
}%
\begin{pgfscope}%
\pgfsys@transformshift{0.564660in}{2.821481in}%
\pgfsys@useobject{currentmarker}{}%
\end{pgfscope}%
\end{pgfscope}%
\begin{pgfscope}%
\definecolor{textcolor}{rgb}{0.000000,0.000000,0.000000}%
\pgfsetstrokecolor{textcolor}%
\pgfsetfillcolor{textcolor}%
\pgftext[x=0.289968in,y=2.768720in,left,base]{\color{textcolor}\rmfamily\fontsize{10.000000}{12.000000}\selectfont \(\displaystyle 0.6\)}%
\end{pgfscope}%
\begin{pgfscope}%
\pgfsetbuttcap%
\pgfsetroundjoin%
\definecolor{currentfill}{rgb}{0.000000,0.000000,0.000000}%
\pgfsetfillcolor{currentfill}%
\pgfsetlinewidth{0.803000pt}%
\definecolor{currentstroke}{rgb}{0.000000,0.000000,0.000000}%
\pgfsetstrokecolor{currentstroke}%
\pgfsetdash{}{0pt}%
\pgfsys@defobject{currentmarker}{\pgfqpoint{-0.048611in}{0.000000in}}{\pgfqpoint{0.000000in}{0.000000in}}{%
\pgfpathmoveto{\pgfqpoint{0.000000in}{0.000000in}}%
\pgfpathlineto{\pgfqpoint{-0.048611in}{0.000000in}}%
\pgfusepath{stroke,fill}%
}%
\begin{pgfscope}%
\pgfsys@transformshift{0.564660in}{3.422799in}%
\pgfsys@useobject{currentmarker}{}%
\end{pgfscope}%
\end{pgfscope}%
\begin{pgfscope}%
\definecolor{textcolor}{rgb}{0.000000,0.000000,0.000000}%
\pgfsetstrokecolor{textcolor}%
\pgfsetfillcolor{textcolor}%
\pgftext[x=0.289968in,y=3.370037in,left,base]{\color{textcolor}\rmfamily\fontsize{10.000000}{12.000000}\selectfont \(\displaystyle 0.7\)}%
\end{pgfscope}%
\begin{pgfscope}%
\definecolor{textcolor}{rgb}{0.000000,0.000000,0.000000}%
\pgfsetstrokecolor{textcolor}%
\pgfsetfillcolor{textcolor}%
\pgftext[x=0.234413in,y=2.031603in,,bottom,rotate=90.000000]{\color{textcolor}\rmfamily\fontsize{10.000000}{12.000000}\selectfont \(\displaystyle \mathbf{I}\mbox{g}\)}%
\end{pgfscope}%
\begin{pgfscope}%
\pgfpathrectangle{\pgfqpoint{0.564660in}{0.521603in}}{\pgfqpoint{4.650000in}{3.020000in}}%
\pgfusepath{clip}%
\pgfsetrectcap%
\pgfsetroundjoin%
\pgfsetlinewidth{1.505625pt}%
\definecolor{currentstroke}{rgb}{0.000000,0.000000,0.000000}%
\pgfsetstrokecolor{currentstroke}%
\pgfsetstrokeopacity{0.300000}%
\pgfsetdash{}{0pt}%
\pgfpathmoveto{\pgfqpoint{0.816771in}{1.337601in}}%
\pgfpathlineto{\pgfqpoint{3.858085in}{2.234101in}}%
\pgfpathlineto{\pgfqpoint{4.993344in}{2.568746in}}%
\pgfpathlineto{\pgfqpoint{1.914828in}{1.661279in}}%
\pgfpathlineto{\pgfqpoint{2.773519in}{1.914399in}}%
\pgfpathlineto{\pgfqpoint{2.234616in}{1.755544in}}%
\pgfpathlineto{\pgfqpoint{0.819079in}{1.338281in}}%
\pgfpathlineto{\pgfqpoint{1.363796in}{1.498849in}}%
\pgfpathlineto{\pgfqpoint{1.096147in}{1.419953in}}%
\pgfpathlineto{\pgfqpoint{0.887904in}{1.358569in}}%
\pgfpathlineto{\pgfqpoint{2.143771in}{1.728766in}}%
\pgfpathlineto{\pgfqpoint{1.086864in}{1.417217in}}%
\pgfpathlineto{\pgfqpoint{0.785977in}{1.328523in}}%
\pgfpathlineto{\pgfqpoint{2.036724in}{1.697211in}}%
\pgfpathlineto{\pgfqpoint{1.309364in}{1.482804in}}%
\pgfusepath{stroke}%
\end{pgfscope}%
\begin{pgfscope}%
\pgfsetrectcap%
\pgfsetmiterjoin%
\pgfsetlinewidth{0.803000pt}%
\definecolor{currentstroke}{rgb}{0.501961,0.501961,0.501961}%
\pgfsetstrokecolor{currentstroke}%
\pgfsetdash{}{0pt}%
\pgfpathmoveto{\pgfqpoint{0.564660in}{0.521603in}}%
\pgfpathlineto{\pgfqpoint{0.564660in}{3.541603in}}%
\pgfusepath{stroke}%
\end{pgfscope}%
\begin{pgfscope}%
\pgfsetrectcap%
\pgfsetmiterjoin%
\pgfsetlinewidth{0.803000pt}%
\definecolor{currentstroke}{rgb}{0.501961,0.501961,0.501961}%
\pgfsetstrokecolor{currentstroke}%
\pgfsetdash{}{0pt}%
\pgfpathmoveto{\pgfqpoint{5.214660in}{0.521603in}}%
\pgfpathlineto{\pgfqpoint{5.214660in}{3.541603in}}%
\pgfusepath{stroke}%
\end{pgfscope}%
\begin{pgfscope}%
\pgfsetrectcap%
\pgfsetmiterjoin%
\pgfsetlinewidth{0.803000pt}%
\definecolor{currentstroke}{rgb}{0.501961,0.501961,0.501961}%
\pgfsetstrokecolor{currentstroke}%
\pgfsetdash{}{0pt}%
\pgfpathmoveto{\pgfqpoint{0.564660in}{0.521603in}}%
\pgfpathlineto{\pgfqpoint{5.214660in}{0.521603in}}%
\pgfusepath{stroke}%
\end{pgfscope}%
\begin{pgfscope}%
\pgfsetrectcap%
\pgfsetmiterjoin%
\pgfsetlinewidth{0.803000pt}%
\definecolor{currentstroke}{rgb}{0.501961,0.501961,0.501961}%
\pgfsetstrokecolor{currentstroke}%
\pgfsetdash{}{0pt}%
\pgfpathmoveto{\pgfqpoint{0.564660in}{3.541603in}}%
\pgfpathlineto{\pgfqpoint{5.214660in}{3.541603in}}%
\pgfusepath{stroke}%
\end{pgfscope}%
\begin{pgfscope}%
\pgfsetbuttcap%
\pgfsetmiterjoin%
\definecolor{currentfill}{rgb}{1.000000,1.000000,1.000000}%
\pgfsetfillcolor{currentfill}%
\pgfsetfillopacity{0.800000}%
\pgfsetlinewidth{1.003750pt}%
\definecolor{currentstroke}{rgb}{0.800000,0.800000,0.800000}%
\pgfsetstrokecolor{currentstroke}%
\pgfsetstrokeopacity{0.800000}%
\pgfsetdash{}{0pt}%
\pgfpathmoveto{\pgfqpoint{2.661499in}{0.591048in}}%
\pgfpathlineto{\pgfqpoint{5.117438in}{0.591048in}}%
\pgfpathquadraticcurveto{\pgfqpoint{5.145216in}{0.591048in}}{\pgfqpoint{5.145216in}{0.618826in}}%
\pgfpathlineto{\pgfqpoint{5.145216in}{0.825238in}}%
\pgfpathquadraticcurveto{\pgfqpoint{5.145216in}{0.853016in}}{\pgfqpoint{5.117438in}{0.853016in}}%
\pgfpathlineto{\pgfqpoint{2.661499in}{0.853016in}}%
\pgfpathquadraticcurveto{\pgfqpoint{2.633721in}{0.853016in}}{\pgfqpoint{2.633721in}{0.825238in}}%
\pgfpathlineto{\pgfqpoint{2.633721in}{0.618826in}}%
\pgfpathquadraticcurveto{\pgfqpoint{2.633721in}{0.591048in}}{\pgfqpoint{2.661499in}{0.591048in}}%
\pgfpathclose%
\pgfusepath{stroke,fill}%
\end{pgfscope}%
\begin{pgfscope}%
\pgfsetrectcap%
\pgfsetroundjoin%
\pgfsetlinewidth{1.505625pt}%
\definecolor{currentstroke}{rgb}{0.000000,0.000000,0.000000}%
\pgfsetstrokecolor{currentstroke}%
\pgfsetstrokeopacity{0.300000}%
\pgfsetdash{}{0pt}%
\pgfpathmoveto{\pgfqpoint{2.689277in}{0.726071in}}%
\pgfpathlineto{\pgfqpoint{2.967055in}{0.726071in}}%
\pgfusepath{stroke}%
\end{pgfscope}%
\begin{pgfscope}%
\definecolor{textcolor}{rgb}{0.501961,0.501961,0.501961}%
\pgfsetstrokecolor{textcolor}%
\pgfsetfillcolor{textcolor}%
\pgftext[x=3.078166in,y=0.677460in,left,base]{\color{textcolor}\rmfamily\fontsize{10.000000}{12.000000}\selectfont \(\displaystyle \mathbf{I}\mbox{g} \approx 0.013 \mathbf{E}\mbox{u} + 0.230\), \(\displaystyle R^2=0.57\)}%
\end{pgfscope}%
\end{pgfpicture}%
\makeatother%
\endgroup%

    \caption{Experimental covariance between $\mathbb{E}\mbox{u}$ and $\mathbb{I}\mbox{m}$.\label{fig:dnumbs}}
\end{figure}

We should note that there are several kinds of systematic error that influence our data. We assume that drop translate purely along the central axis of the electric field, but in practice, despite the improvement in surface charge density uniformity produced by corona charging, there are still local areas of especially high charge density. In principle, this kind of error should become small for drops which are far enough away from the charge distribution, that the geometry of the charge distribution disappears, and the electric field looks like that due to a point charge. Another form of error is in the initial velocity as it appears in $\mathbb{E}\mbox{u}$. Because we usually lose the first few frames of video due to camera shake transients at the start of the low-gravity experiment, we will consistently underestimate $U_0$ because the drop will already have decelerated significantly during that period of time. The primary sources of random error are the effect of contact line hysteresis on the drop initial velocity, and of the variance in the MLE parameter estimates.

\section{Impact Dynamics}
\hl{Check the Oh numbers}. Very little work to date on drop impacts outside of two regimes: first, very low $\mathbb{R}\mbox{e}$ viscous drop spreading driven by capillary forces at the contact line, and second, impacts at ``high'' $\mathbb{W}\mbox{e}$. There has been some computational work with impacts at $\mathbb{R}\mbox{e}=1.4$ (Fukai \emph{et al.}), but the results are suspect as the model neglects uncompensated Young's force at the contact line (which is the dominant spreading force for low $\mathbb{R}\mbox{e}$ impacts). Of course models for dynamic contact lines in general remain controversial, even for ordinary spreading of liquids, despite decades of work in the area. 

Naively neglecting dimensionless groups governing the dynamic contact line, the pertinent dimensionless groups for isothermal droplet impacts are the Weber number $\mathbb{W}\mbox{e} = \frac{\rho U^2 R_d}{\sigma}$, which is a ratio of droplet inertia to surface tension, the Ohnesorge number, $\mathbb{O}\mbox{h} = \frac{\mu}{\sqrt{\rho \sigma R_d}}$, which is a ratio of viscous to inertial and surface tension forces, and the Bond number, $\mathbb{B}\mbox{o}$, defined previously. The dynamics of spreading are characterized primarily by $\mathbb{W}\mbox{e}$ and $\mathbb{O}\mbox{h}$. Additionally the final stages of spreading depend strongly on $\theta_e$, which is the static contact angle. The Weber number scales the \emph{driving force} of the impact. In the more familiar case of high $\mathbb{W}\mbox{e}$ the drop liquid bulk is driven radially outward by the impact induced pressure gradient, whereas in the case of small $\mathbb{W}\mbox{e}$ the liquid is pulled outwards by capillary force (e.g. uncompensated Young's force) at the contact line. The Ohnesorge number, by contrast, scales the force that \emph{resists} spreading. For large $\mathbb{O}\mbox{h}$ the resistive force is viscous, whereas for low $\mathbb{O}\mbox{h}$ the force is inertia.

We observe average drop impact $\mathbb{O}\mbox{h} \approx 2.18 \pm 0.36$, and  $\mathbb{W}\mbox{e} \approx 0.28 \pm 0.22$. Thus impact velocity plays little role in the spreading dynamics of the bounces, and viscous effects are important but do not dominate inertia. Notably we observe underdamped oscillations of drop interfaces during impact. according to (Schiaffino, 1997) viscous forces play a role in the final (which? describe) stage of spreading even for drop impact at very low $\mathbb{O}\mbox{h}$. In the regime of intermediate viscosity, following spreading the oscillations of the interface are damped with a characteristic time that is generally longer than the spreading time (again, given by $t_c \sim \left( \sigma / \rho R_d^3 \right)^{1/2}$). The celebrated Hoffman-Tanner-Voinov law, which relates the contact line velocity $U_c$ of an isothermal spreading process to the static contact angle $\theta_e$ to the dynamic advancing contact angle $\theta_a$ by
\[
\frac{\mu U_c}{\sigma} \approx \kappa \left( \theta_a^3 - \theta_e^3 \right),
\]
where $\kappa \approx 0.013$ is an empirical coefficient extracted from Hoffman's data. The Hoffman-Tanner-Voinov law implies that the transition between low $\mathbb{W}\mbox{e}$ inviscid and viscous regimes occurs at $\mathbb{O}\mbox{h} \sim \mathcal{O}(10^{-2})$ rather than $\mathbb{O}\mbox{h} \sim \mathcal{O}(1)$ given by a naive scaling.

Some notes:
\begin{itemize}
\item{\textbf{Schiaffino, 1997}} 
\end{itemize}

%\newpage
\end{document}

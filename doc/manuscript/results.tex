\documentclass[10pt,a4paper]{article}
\usepackage[utf8]{inputenc}
%\usepackage{fontspec} % This line only for XeLaTeX and LuaLaTeX
\usepackage{pgfplots}
\usepackage{pgf}
\usepackage[english]{babel}
\usepackage{amsmath}
\usepackage{amsfonts}
\usepackage{amssymb}
\usepackage{graphicx}
\graphicspath{ {../figures/} }
%\usepackage{svg}
\usepackage{verbatim}
\usepackage{color,soul}
\usepackage{listings}
\usepackage{setspace}
\author{Erin Schmidt}

\newlength\figureheight
\newlength\figurewidth
\setlength\figureheight{7cm}
\setlength\figurewidth{10cm}

\begin{document}

\doublespacing
\section{Overview}
\begin{itemize}
\item We found the distribution of mostly likely experimental net charges for a population of the droplets jumped in low-gravity. We found the charge to be a function of droplet volume and surface potential of the dielectric substrate. A two-ways T-test with a charge distribution determined by a corollary experiment suggests that the droplet charge is induced by the electric field (rather than through contact charging on the PTFE layer).

\item The bounce dynamics are controlled by a dimensionless ratio of electrostatic force to inertia. The dielectrophoretic force plays a very small role when droplets have net charge in a DC field.  

\item Using the unique capabilities of the low-gravity environment we obtained data on dimensionless contact time and coefficients of restitution at very low Ohnesorge numbers for a range of electric Bond numbers. Despite strong electric fields (20-30 $kV/cm$) we found little evidence for wetting transitions due to excession of a critical pressure (the ``Fakir impalement''). There is no obvious trend in dimensionless contact time or coefficient of restitution with electric Bond number.

\item Jump velocities are more strongly damped for relatively small droplet volumes in the presence of the electric fields than was shown by Attari \emph{et. al.}. This may be evidence for electrowetting paradoxically enhancing the effect of contact angle hysteresis pinning on sharp corners. (How does this tie into the coefficients of restitution problem?)

\item By scale arguments and perturbation of solutions to the equations of motion we find several simple rules of thumb in droplet ``escape velocity'' of impacts, length scales and time scales for returns.
\end{itemize}

\begin{figure}[htb]
    \centering
    %% Creator: Matplotlib, PGF backend
%%
%% To include the figure in your LaTeX document, write
%%   \input{<filename>.pgf}
%%
%% Make sure the required packages are loaded in your preamble
%%   \usepackage{pgf}
%%
%% Figures using additional raster images can only be included by \input if
%% they are in the same directory as the main LaTeX file. For loading figures
%% from other directories you can use the `import` package
%%   \usepackage{import}
%% and then include the figures with
%%   \import{<path to file>}{<filename>.pgf}
%%
%% Matplotlib used the following preamble
%%   \usepackage{fontspec}
%%   \setmainfont{DejaVuSerif.ttf}[Path=/home/erin/anaconda3/lib/python3.6/site-packages/matplotlib/mpl-data/fonts/ttf/]
%%   \setsansfont{DejaVuSans.ttf}[Path=/home/erin/anaconda3/lib/python3.6/site-packages/matplotlib/mpl-data/fonts/ttf/]
%%   \setmonofont{DejaVuSansMono.ttf}[Path=/home/erin/anaconda3/lib/python3.6/site-packages/matplotlib/mpl-data/fonts/ttf/]
%%
\begingroup%
\makeatletter%
\begin{pgfpicture}%
\pgfpathrectangle{\pgfpointorigin}{\pgfqpoint{5.232821in}{3.788793in}}%
\pgfusepath{use as bounding box, clip}%
\begin{pgfscope}%
\pgfsetbuttcap%
\pgfsetmiterjoin%
\definecolor{currentfill}{rgb}{1.000000,1.000000,1.000000}%
\pgfsetfillcolor{currentfill}%
\pgfsetlinewidth{0.000000pt}%
\definecolor{currentstroke}{rgb}{1.000000,1.000000,1.000000}%
\pgfsetstrokecolor{currentstroke}%
\pgfsetdash{}{0pt}%
\pgfpathmoveto{\pgfqpoint{0.000000in}{0.000000in}}%
\pgfpathlineto{\pgfqpoint{5.232821in}{0.000000in}}%
\pgfpathlineto{\pgfqpoint{5.232821in}{3.788793in}}%
\pgfpathlineto{\pgfqpoint{0.000000in}{3.788793in}}%
\pgfpathclose%
\pgfusepath{fill}%
\end{pgfscope}%
\begin{pgfscope}%
\pgfsetbuttcap%
\pgfsetmiterjoin%
\definecolor{currentfill}{rgb}{1.000000,1.000000,1.000000}%
\pgfsetfillcolor{currentfill}%
\pgfsetlinewidth{0.000000pt}%
\definecolor{currentstroke}{rgb}{0.000000,0.000000,0.000000}%
\pgfsetstrokecolor{currentstroke}%
\pgfsetstrokeopacity{0.000000}%
\pgfsetdash{}{0pt}%
\pgfpathmoveto{\pgfqpoint{0.564660in}{0.521603in}}%
\pgfpathlineto{\pgfqpoint{4.284660in}{0.521603in}}%
\pgfpathlineto{\pgfqpoint{4.284660in}{3.541603in}}%
\pgfpathlineto{\pgfqpoint{0.564660in}{3.541603in}}%
\pgfpathclose%
\pgfusepath{fill}%
\end{pgfscope}%
\begin{pgfscope}%
\pgfpathrectangle{\pgfqpoint{0.564660in}{0.521603in}}{\pgfqpoint{3.720000in}{3.020000in}}%
\pgfusepath{clip}%
\pgfsetbuttcap%
\pgfsetroundjoin%
\definecolor{currentfill}{rgb}{0.061765,0.061765,0.085934}%
\pgfsetfillcolor{currentfill}%
\pgfsetlinewidth{0.000000pt}%
\definecolor{currentstroke}{rgb}{0.000000,0.000000,0.000000}%
\pgfsetstrokecolor{currentstroke}%
\pgfsetdash{}{0pt}%
\pgfpathmoveto{\pgfqpoint{1.238132in}{0.750263in}}%
\pgfpathlineto{\pgfqpoint{1.305934in}{0.750263in}}%
\pgfpathlineto{\pgfqpoint{1.373735in}{0.750263in}}%
\pgfpathlineto{\pgfqpoint{1.441537in}{0.750263in}}%
\pgfpathlineto{\pgfqpoint{1.509338in}{0.750263in}}%
\pgfpathlineto{\pgfqpoint{1.577140in}{0.804788in}}%
\pgfpathlineto{\pgfqpoint{1.644942in}{0.804788in}}%
\pgfpathlineto{\pgfqpoint{1.712743in}{0.804788in}}%
\pgfpathlineto{\pgfqpoint{1.780545in}{0.859313in}}%
\pgfpathlineto{\pgfqpoint{1.848347in}{0.859313in}}%
\pgfpathlineto{\pgfqpoint{1.916148in}{0.913838in}}%
\pgfpathlineto{\pgfqpoint{1.983950in}{0.913838in}}%
\pgfpathlineto{\pgfqpoint{2.051751in}{0.913838in}}%
\pgfpathlineto{\pgfqpoint{2.119553in}{0.968363in}}%
\pgfpathlineto{\pgfqpoint{2.187355in}{0.968363in}}%
\pgfpathlineto{\pgfqpoint{2.255156in}{1.022888in}}%
\pgfpathlineto{\pgfqpoint{2.264819in}{1.022888in}}%
\pgfpathlineto{\pgfqpoint{2.255156in}{1.025047in}}%
\pgfpathlineto{\pgfqpoint{2.187355in}{1.025510in}}%
\pgfpathlineto{\pgfqpoint{2.119553in}{1.024502in}}%
\pgfpathlineto{\pgfqpoint{2.078575in}{1.022888in}}%
\pgfpathlineto{\pgfqpoint{2.051751in}{1.021485in}}%
\pgfpathlineto{\pgfqpoint{1.983950in}{1.014951in}}%
\pgfpathlineto{\pgfqpoint{1.916148in}{1.008416in}}%
\pgfpathlineto{\pgfqpoint{1.885240in}{1.022888in}}%
\pgfpathlineto{\pgfqpoint{1.848347in}{1.047771in}}%
\pgfpathlineto{\pgfqpoint{1.805159in}{1.077414in}}%
\pgfpathlineto{\pgfqpoint{1.780545in}{1.090266in}}%
\pgfpathlineto{\pgfqpoint{1.712743in}{1.125668in}}%
\pgfpathlineto{\pgfqpoint{1.700735in}{1.131939in}}%
\pgfpathlineto{\pgfqpoint{1.644942in}{1.161071in}}%
\pgfpathlineto{\pgfqpoint{1.613824in}{1.186464in}}%
\pgfpathlineto{\pgfqpoint{1.577140in}{1.212962in}}%
\pgfpathlineto{\pgfqpoint{1.532843in}{1.240989in}}%
\pgfpathlineto{\pgfqpoint{1.509338in}{1.260474in}}%
\pgfpathlineto{\pgfqpoint{1.441537in}{1.291166in}}%
\pgfpathlineto{\pgfqpoint{1.430114in}{1.295514in}}%
\pgfpathlineto{\pgfqpoint{1.373735in}{1.316975in}}%
\pgfpathlineto{\pgfqpoint{1.305934in}{1.342785in}}%
\pgfpathlineto{\pgfqpoint{1.282780in}{1.295514in}}%
\pgfpathlineto{\pgfqpoint{1.272931in}{1.240989in}}%
\pgfpathlineto{\pgfqpoint{1.268059in}{1.186464in}}%
\pgfpathlineto{\pgfqpoint{1.263374in}{1.131939in}}%
\pgfpathlineto{\pgfqpoint{1.249947in}{1.077414in}}%
\pgfpathlineto{\pgfqpoint{1.238132in}{1.033194in}}%
\pgfpathlineto{\pgfqpoint{1.170330in}{1.023200in}}%
\pgfpathlineto{\pgfqpoint{1.102529in}{1.027231in}}%
\pgfpathlineto{\pgfqpoint{1.034727in}{1.031263in}}%
\pgfpathlineto{\pgfqpoint{0.966926in}{1.035294in}}%
\pgfpathlineto{\pgfqpoint{0.899124in}{1.039326in}}%
\pgfpathlineto{\pgfqpoint{0.831322in}{1.043357in}}%
\pgfpathlineto{\pgfqpoint{0.831322in}{1.022888in}}%
\pgfpathlineto{\pgfqpoint{0.899124in}{0.968363in}}%
\pgfpathlineto{\pgfqpoint{0.966926in}{0.913838in}}%
\pgfpathlineto{\pgfqpoint{1.034727in}{0.913838in}}%
\pgfpathlineto{\pgfqpoint{1.102529in}{0.859313in}}%
\pgfpathlineto{\pgfqpoint{1.170330in}{0.804788in}}%
\pgfpathclose%
\pgfusepath{fill}%
\end{pgfscope}%
\begin{pgfscope}%
\pgfpathrectangle{\pgfqpoint{0.564660in}{0.521603in}}{\pgfqpoint{3.720000in}{3.020000in}}%
\pgfusepath{clip}%
\pgfsetbuttcap%
\pgfsetroundjoin%
\definecolor{currentfill}{rgb}{0.185294,0.185294,0.257801}%
\pgfsetfillcolor{currentfill}%
\pgfsetlinewidth{0.000000pt}%
\definecolor{currentstroke}{rgb}{0.000000,0.000000,0.000000}%
\pgfsetstrokecolor{currentstroke}%
\pgfsetdash{}{0pt}%
\pgfpathmoveto{\pgfqpoint{1.916148in}{1.008416in}}%
\pgfpathlineto{\pgfqpoint{1.983950in}{1.014951in}}%
\pgfpathlineto{\pgfqpoint{2.051751in}{1.021485in}}%
\pgfpathlineto{\pgfqpoint{2.078575in}{1.022888in}}%
\pgfpathlineto{\pgfqpoint{2.119553in}{1.024502in}}%
\pgfpathlineto{\pgfqpoint{2.187355in}{1.025510in}}%
\pgfpathlineto{\pgfqpoint{2.255156in}{1.025047in}}%
\pgfpathlineto{\pgfqpoint{2.264819in}{1.022888in}}%
\pgfpathlineto{\pgfqpoint{2.322958in}{1.022888in}}%
\pgfpathlineto{\pgfqpoint{2.390759in}{1.077414in}}%
\pgfpathlineto{\pgfqpoint{2.458561in}{1.131939in}}%
\pgfpathlineto{\pgfqpoint{2.526363in}{1.186464in}}%
\pgfpathlineto{\pgfqpoint{2.594164in}{1.186464in}}%
\pgfpathlineto{\pgfqpoint{2.661966in}{1.240989in}}%
\pgfpathlineto{\pgfqpoint{2.729768in}{1.295514in}}%
\pgfpathlineto{\pgfqpoint{2.756979in}{1.317397in}}%
\pgfpathlineto{\pgfqpoint{2.729768in}{1.323475in}}%
\pgfpathlineto{\pgfqpoint{2.661966in}{1.338621in}}%
\pgfpathlineto{\pgfqpoint{2.610848in}{1.350039in}}%
\pgfpathlineto{\pgfqpoint{2.594164in}{1.353766in}}%
\pgfpathlineto{\pgfqpoint{2.526363in}{1.368911in}}%
\pgfpathlineto{\pgfqpoint{2.458561in}{1.384057in}}%
\pgfpathlineto{\pgfqpoint{2.390759in}{1.399202in}}%
\pgfpathlineto{\pgfqpoint{2.366754in}{1.404564in}}%
\pgfpathlineto{\pgfqpoint{2.322958in}{1.414347in}}%
\pgfpathlineto{\pgfqpoint{2.255156in}{1.429493in}}%
\pgfpathlineto{\pgfqpoint{2.187355in}{1.444638in}}%
\pgfpathlineto{\pgfqpoint{2.119553in}{1.452237in}}%
\pgfpathlineto{\pgfqpoint{2.051751in}{1.451228in}}%
\pgfpathlineto{\pgfqpoint{1.983950in}{1.450219in}}%
\pgfpathlineto{\pgfqpoint{1.916148in}{1.449211in}}%
\pgfpathlineto{\pgfqpoint{1.905252in}{1.459089in}}%
\pgfpathlineto{\pgfqpoint{1.848347in}{1.505259in}}%
\pgfpathlineto{\pgfqpoint{1.840382in}{1.513615in}}%
\pgfpathlineto{\pgfqpoint{1.788412in}{1.568140in}}%
\pgfpathlineto{\pgfqpoint{1.780545in}{1.576394in}}%
\pgfpathlineto{\pgfqpoint{1.736442in}{1.622665in}}%
\pgfpathlineto{\pgfqpoint{1.712743in}{1.647529in}}%
\pgfpathlineto{\pgfqpoint{1.684473in}{1.677190in}}%
\pgfpathlineto{\pgfqpoint{1.644942in}{1.718664in}}%
\pgfpathlineto{\pgfqpoint{1.631711in}{1.731715in}}%
\pgfpathlineto{\pgfqpoint{1.580533in}{1.786240in}}%
\pgfpathlineto{\pgfqpoint{1.577140in}{1.789800in}}%
\pgfpathlineto{\pgfqpoint{1.519093in}{1.840765in}}%
\pgfpathlineto{\pgfqpoint{1.509338in}{1.851646in}}%
\pgfpathlineto{\pgfqpoint{1.441537in}{1.890104in}}%
\pgfpathlineto{\pgfqpoint{1.433944in}{1.895291in}}%
\pgfpathlineto{\pgfqpoint{1.373735in}{1.931697in}}%
\pgfpathlineto{\pgfqpoint{1.351722in}{1.949816in}}%
\pgfpathlineto{\pgfqpoint{1.305934in}{1.987504in}}%
\pgfpathlineto{\pgfqpoint{1.285478in}{2.004341in}}%
\pgfpathlineto{\pgfqpoint{1.238132in}{2.043312in}}%
\pgfpathlineto{\pgfqpoint{1.196249in}{2.058866in}}%
\pgfpathlineto{\pgfqpoint{1.170330in}{2.067546in}}%
\pgfpathlineto{\pgfqpoint{1.102529in}{2.062611in}}%
\pgfpathlineto{\pgfqpoint{1.102529in}{2.058866in}}%
\pgfpathlineto{\pgfqpoint{1.102529in}{2.004341in}}%
\pgfpathlineto{\pgfqpoint{1.034727in}{1.949816in}}%
\pgfpathlineto{\pgfqpoint{1.034727in}{1.895291in}}%
\pgfpathlineto{\pgfqpoint{0.966926in}{1.840765in}}%
\pgfpathlineto{\pgfqpoint{0.966926in}{1.786240in}}%
\pgfpathlineto{\pgfqpoint{0.966926in}{1.731715in}}%
\pgfpathlineto{\pgfqpoint{0.966926in}{1.677190in}}%
\pgfpathlineto{\pgfqpoint{0.966926in}{1.622665in}}%
\pgfpathlineto{\pgfqpoint{0.899124in}{1.568140in}}%
\pgfpathlineto{\pgfqpoint{0.899124in}{1.513615in}}%
\pgfpathlineto{\pgfqpoint{0.899124in}{1.459089in}}%
\pgfpathlineto{\pgfqpoint{0.899124in}{1.404564in}}%
\pgfpathlineto{\pgfqpoint{0.899124in}{1.350039in}}%
\pgfpathlineto{\pgfqpoint{0.831322in}{1.295514in}}%
\pgfpathlineto{\pgfqpoint{0.831322in}{1.240989in}}%
\pgfpathlineto{\pgfqpoint{0.831322in}{1.186464in}}%
\pgfpathlineto{\pgfqpoint{0.831322in}{1.131939in}}%
\pgfpathlineto{\pgfqpoint{0.831322in}{1.077414in}}%
\pgfpathlineto{\pgfqpoint{0.831322in}{1.043357in}}%
\pgfpathlineto{\pgfqpoint{0.899124in}{1.039326in}}%
\pgfpathlineto{\pgfqpoint{0.966926in}{1.035294in}}%
\pgfpathlineto{\pgfqpoint{1.034727in}{1.031263in}}%
\pgfpathlineto{\pgfqpoint{1.102529in}{1.027231in}}%
\pgfpathlineto{\pgfqpoint{1.170330in}{1.023200in}}%
\pgfpathlineto{\pgfqpoint{1.238132in}{1.033194in}}%
\pgfpathlineto{\pgfqpoint{1.249947in}{1.077414in}}%
\pgfpathlineto{\pgfqpoint{1.263374in}{1.131939in}}%
\pgfpathlineto{\pgfqpoint{1.268059in}{1.186464in}}%
\pgfpathlineto{\pgfqpoint{1.272931in}{1.240989in}}%
\pgfpathlineto{\pgfqpoint{1.282780in}{1.295514in}}%
\pgfpathlineto{\pgfqpoint{1.305934in}{1.342785in}}%
\pgfpathlineto{\pgfqpoint{1.373735in}{1.316975in}}%
\pgfpathlineto{\pgfqpoint{1.430114in}{1.295514in}}%
\pgfpathlineto{\pgfqpoint{1.441537in}{1.291166in}}%
\pgfpathlineto{\pgfqpoint{1.509338in}{1.260474in}}%
\pgfpathlineto{\pgfqpoint{1.532843in}{1.240989in}}%
\pgfpathlineto{\pgfqpoint{1.577140in}{1.212962in}}%
\pgfpathlineto{\pgfqpoint{1.613824in}{1.186464in}}%
\pgfpathlineto{\pgfqpoint{1.644942in}{1.161071in}}%
\pgfpathlineto{\pgfqpoint{1.700735in}{1.131939in}}%
\pgfpathlineto{\pgfqpoint{1.712743in}{1.125668in}}%
\pgfpathlineto{\pgfqpoint{1.780545in}{1.090266in}}%
\pgfpathlineto{\pgfqpoint{1.805159in}{1.077414in}}%
\pgfpathlineto{\pgfqpoint{1.848347in}{1.047771in}}%
\pgfpathlineto{\pgfqpoint{1.885240in}{1.022888in}}%
\pgfpathclose%
\pgfpathmoveto{\pgfqpoint{1.091247in}{1.568140in}}%
\pgfpathlineto{\pgfqpoint{1.101834in}{1.622665in}}%
\pgfpathlineto{\pgfqpoint{1.102529in}{1.623097in}}%
\pgfpathlineto{\pgfqpoint{1.102695in}{1.622665in}}%
\pgfpathlineto{\pgfqpoint{1.105105in}{1.568140in}}%
\pgfpathlineto{\pgfqpoint{1.102529in}{1.554543in}}%
\pgfpathclose%
\pgfusepath{fill}%
\end{pgfscope}%
\begin{pgfscope}%
\pgfpathrectangle{\pgfqpoint{0.564660in}{0.521603in}}{\pgfqpoint{3.720000in}{3.020000in}}%
\pgfusepath{clip}%
\pgfsetbuttcap%
\pgfsetroundjoin%
\definecolor{currentfill}{rgb}{0.185294,0.185294,0.257801}%
\pgfsetfillcolor{currentfill}%
\pgfsetlinewidth{0.000000pt}%
\definecolor{currentstroke}{rgb}{0.000000,0.000000,0.000000}%
\pgfsetstrokecolor{currentstroke}%
\pgfsetdash{}{0pt}%
\pgfpathmoveto{\pgfqpoint{2.933172in}{1.404564in}}%
\pgfpathlineto{\pgfqpoint{2.939480in}{1.409637in}}%
\pgfpathlineto{\pgfqpoint{2.933172in}{1.408107in}}%
\pgfpathlineto{\pgfqpoint{2.927424in}{1.404564in}}%
\pgfpathclose%
\pgfusepath{fill}%
\end{pgfscope}%
\begin{pgfscope}%
\pgfpathrectangle{\pgfqpoint{0.564660in}{0.521603in}}{\pgfqpoint{3.720000in}{3.020000in}}%
\pgfusepath{clip}%
\pgfsetbuttcap%
\pgfsetroundjoin%
\definecolor{currentfill}{rgb}{0.312255,0.312255,0.434442}%
\pgfsetfillcolor{currentfill}%
\pgfsetlinewidth{0.000000pt}%
\definecolor{currentstroke}{rgb}{0.000000,0.000000,0.000000}%
\pgfsetstrokecolor{currentstroke}%
\pgfsetdash{}{0pt}%
\pgfpathmoveto{\pgfqpoint{2.661966in}{1.338621in}}%
\pgfpathlineto{\pgfqpoint{2.729768in}{1.323475in}}%
\pgfpathlineto{\pgfqpoint{2.756979in}{1.317397in}}%
\pgfpathlineto{\pgfqpoint{2.797569in}{1.350039in}}%
\pgfpathlineto{\pgfqpoint{2.865371in}{1.404564in}}%
\pgfpathlineto{\pgfqpoint{2.927424in}{1.404564in}}%
\pgfpathlineto{\pgfqpoint{2.933172in}{1.408107in}}%
\pgfpathlineto{\pgfqpoint{2.939480in}{1.409637in}}%
\pgfpathlineto{\pgfqpoint{3.000974in}{1.459089in}}%
\pgfpathlineto{\pgfqpoint{3.068776in}{1.513615in}}%
\pgfpathlineto{\pgfqpoint{3.136577in}{1.568140in}}%
\pgfpathlineto{\pgfqpoint{3.204379in}{1.568140in}}%
\pgfpathlineto{\pgfqpoint{3.272180in}{1.622665in}}%
\pgfpathlineto{\pgfqpoint{3.339982in}{1.677190in}}%
\pgfpathlineto{\pgfqpoint{3.407784in}{1.731715in}}%
\pgfpathlineto{\pgfqpoint{3.475585in}{1.731715in}}%
\pgfpathlineto{\pgfqpoint{3.543387in}{1.786240in}}%
\pgfpathlineto{\pgfqpoint{3.543387in}{1.840765in}}%
\pgfpathlineto{\pgfqpoint{3.611189in}{1.895291in}}%
\pgfpathlineto{\pgfqpoint{3.678990in}{1.949816in}}%
\pgfpathlineto{\pgfqpoint{3.746792in}{2.004341in}}%
\pgfpathlineto{\pgfqpoint{3.746792in}{2.058866in}}%
\pgfpathlineto{\pgfqpoint{3.814593in}{2.113391in}}%
\pgfpathlineto{\pgfqpoint{3.882395in}{2.167916in}}%
\pgfpathlineto{\pgfqpoint{3.950197in}{2.222441in}}%
\pgfpathlineto{\pgfqpoint{3.950197in}{2.276966in}}%
\pgfpathlineto{\pgfqpoint{4.017998in}{2.331492in}}%
\pgfpathlineto{\pgfqpoint{4.017998in}{2.339530in}}%
\pgfpathlineto{\pgfqpoint{4.006327in}{2.331492in}}%
\pgfpathlineto{\pgfqpoint{3.950197in}{2.292831in}}%
\pgfpathlineto{\pgfqpoint{3.927163in}{2.276966in}}%
\pgfpathlineto{\pgfqpoint{3.882395in}{2.246132in}}%
\pgfpathlineto{\pgfqpoint{3.847998in}{2.222441in}}%
\pgfpathlineto{\pgfqpoint{3.814593in}{2.199433in}}%
\pgfpathlineto{\pgfqpoint{3.768834in}{2.167916in}}%
\pgfpathlineto{\pgfqpoint{3.746792in}{2.152734in}}%
\pgfpathlineto{\pgfqpoint{3.689670in}{2.113391in}}%
\pgfpathlineto{\pgfqpoint{3.678990in}{2.106035in}}%
\pgfpathlineto{\pgfqpoint{3.611189in}{2.059336in}}%
\pgfpathlineto{\pgfqpoint{3.610505in}{2.058866in}}%
\pgfpathlineto{\pgfqpoint{3.543387in}{2.012638in}}%
\pgfpathlineto{\pgfqpoint{3.530445in}{2.004341in}}%
\pgfpathlineto{\pgfqpoint{3.475585in}{1.969319in}}%
\pgfpathlineto{\pgfqpoint{3.440935in}{1.949816in}}%
\pgfpathlineto{\pgfqpoint{3.407784in}{1.931156in}}%
\pgfpathlineto{\pgfqpoint{3.344064in}{1.895291in}}%
\pgfpathlineto{\pgfqpoint{3.339982in}{1.892993in}}%
\pgfpathlineto{\pgfqpoint{3.272180in}{1.854830in}}%
\pgfpathlineto{\pgfqpoint{3.247194in}{1.840765in}}%
\pgfpathlineto{\pgfqpoint{3.204379in}{1.816666in}}%
\pgfpathlineto{\pgfqpoint{3.150323in}{1.786240in}}%
\pgfpathlineto{\pgfqpoint{3.136577in}{1.778503in}}%
\pgfpathlineto{\pgfqpoint{3.068776in}{1.740340in}}%
\pgfpathlineto{\pgfqpoint{3.051071in}{1.731715in}}%
\pgfpathlineto{\pgfqpoint{3.000974in}{1.704773in}}%
\pgfpathlineto{\pgfqpoint{2.933172in}{1.707964in}}%
\pgfpathlineto{\pgfqpoint{2.865371in}{1.716699in}}%
\pgfpathlineto{\pgfqpoint{2.797569in}{1.725434in}}%
\pgfpathlineto{\pgfqpoint{2.748816in}{1.731715in}}%
\pgfpathlineto{\pgfqpoint{2.729768in}{1.734169in}}%
\pgfpathlineto{\pgfqpoint{2.661966in}{1.743597in}}%
\pgfpathlineto{\pgfqpoint{2.594164in}{1.758212in}}%
\pgfpathlineto{\pgfqpoint{2.526363in}{1.773357in}}%
\pgfpathlineto{\pgfqpoint{2.468688in}{1.786240in}}%
\pgfpathlineto{\pgfqpoint{2.458561in}{1.788502in}}%
\pgfpathlineto{\pgfqpoint{2.390759in}{1.803648in}}%
\pgfpathlineto{\pgfqpoint{2.322958in}{1.818793in}}%
\pgfpathlineto{\pgfqpoint{2.255156in}{1.833938in}}%
\pgfpathlineto{\pgfqpoint{2.224593in}{1.840765in}}%
\pgfpathlineto{\pgfqpoint{2.187355in}{1.849084in}}%
\pgfpathlineto{\pgfqpoint{2.119553in}{1.864229in}}%
\pgfpathlineto{\pgfqpoint{2.051751in}{1.878612in}}%
\pgfpathlineto{\pgfqpoint{1.983950in}{1.877955in}}%
\pgfpathlineto{\pgfqpoint{1.916148in}{1.876946in}}%
\pgfpathlineto{\pgfqpoint{1.848347in}{1.888029in}}%
\pgfpathlineto{\pgfqpoint{1.841425in}{1.895291in}}%
\pgfpathlineto{\pgfqpoint{1.789455in}{1.949816in}}%
\pgfpathlineto{\pgfqpoint{1.780545in}{1.959164in}}%
\pgfpathlineto{\pgfqpoint{1.737486in}{2.004341in}}%
\pgfpathlineto{\pgfqpoint{1.712743in}{2.030299in}}%
\pgfpathlineto{\pgfqpoint{1.685516in}{2.058866in}}%
\pgfpathlineto{\pgfqpoint{1.644942in}{2.101435in}}%
\pgfpathlineto{\pgfqpoint{1.633546in}{2.113391in}}%
\pgfpathlineto{\pgfqpoint{1.581576in}{2.167916in}}%
\pgfpathlineto{\pgfqpoint{1.577140in}{2.172570in}}%
\pgfpathlineto{\pgfqpoint{1.529606in}{2.222441in}}%
\pgfpathlineto{\pgfqpoint{1.509338in}{2.243705in}}%
\pgfpathlineto{\pgfqpoint{1.477636in}{2.276966in}}%
\pgfpathlineto{\pgfqpoint{1.441537in}{2.314840in}}%
\pgfpathlineto{\pgfqpoint{1.425666in}{2.331492in}}%
\pgfpathlineto{\pgfqpoint{1.373735in}{2.385975in}}%
\pgfpathlineto{\pgfqpoint{1.373696in}{2.386017in}}%
\pgfpathlineto{\pgfqpoint{1.321726in}{2.440542in}}%
\pgfpathlineto{\pgfqpoint{1.305934in}{2.446237in}}%
\pgfpathlineto{\pgfqpoint{1.305934in}{2.440542in}}%
\pgfpathlineto{\pgfqpoint{1.295552in}{2.432193in}}%
\pgfpathlineto{\pgfqpoint{1.276312in}{2.386017in}}%
\pgfpathlineto{\pgfqpoint{1.238132in}{2.363523in}}%
\pgfpathlineto{\pgfqpoint{1.238132in}{2.331492in}}%
\pgfpathlineto{\pgfqpoint{1.238132in}{2.276966in}}%
\pgfpathlineto{\pgfqpoint{1.170330in}{2.222441in}}%
\pgfpathlineto{\pgfqpoint{1.170330in}{2.167916in}}%
\pgfpathlineto{\pgfqpoint{1.102529in}{2.113391in}}%
\pgfpathlineto{\pgfqpoint{1.102529in}{2.062611in}}%
\pgfpathlineto{\pgfqpoint{1.170330in}{2.067546in}}%
\pgfpathlineto{\pgfqpoint{1.196249in}{2.058866in}}%
\pgfpathlineto{\pgfqpoint{1.238132in}{2.043312in}}%
\pgfpathlineto{\pgfqpoint{1.285478in}{2.004341in}}%
\pgfpathlineto{\pgfqpoint{1.305934in}{1.987504in}}%
\pgfpathlineto{\pgfqpoint{1.351722in}{1.949816in}}%
\pgfpathlineto{\pgfqpoint{1.373735in}{1.931697in}}%
\pgfpathlineto{\pgfqpoint{1.433944in}{1.895291in}}%
\pgfpathlineto{\pgfqpoint{1.441537in}{1.890104in}}%
\pgfpathlineto{\pgfqpoint{1.509338in}{1.851646in}}%
\pgfpathlineto{\pgfqpoint{1.519093in}{1.840765in}}%
\pgfpathlineto{\pgfqpoint{1.577140in}{1.789800in}}%
\pgfpathlineto{\pgfqpoint{1.580533in}{1.786240in}}%
\pgfpathlineto{\pgfqpoint{1.631711in}{1.731715in}}%
\pgfpathlineto{\pgfqpoint{1.644942in}{1.718664in}}%
\pgfpathlineto{\pgfqpoint{1.684473in}{1.677190in}}%
\pgfpathlineto{\pgfqpoint{1.712743in}{1.647529in}}%
\pgfpathlineto{\pgfqpoint{1.736442in}{1.622665in}}%
\pgfpathlineto{\pgfqpoint{1.780545in}{1.576394in}}%
\pgfpathlineto{\pgfqpoint{1.788412in}{1.568140in}}%
\pgfpathlineto{\pgfqpoint{1.840382in}{1.513615in}}%
\pgfpathlineto{\pgfqpoint{1.848347in}{1.505259in}}%
\pgfpathlineto{\pgfqpoint{1.905252in}{1.459089in}}%
\pgfpathlineto{\pgfqpoint{1.916148in}{1.449211in}}%
\pgfpathlineto{\pgfqpoint{1.983950in}{1.450219in}}%
\pgfpathlineto{\pgfqpoint{2.051751in}{1.451228in}}%
\pgfpathlineto{\pgfqpoint{2.119553in}{1.452237in}}%
\pgfpathlineto{\pgfqpoint{2.187355in}{1.444638in}}%
\pgfpathlineto{\pgfqpoint{2.255156in}{1.429493in}}%
\pgfpathlineto{\pgfqpoint{2.322958in}{1.414347in}}%
\pgfpathlineto{\pgfqpoint{2.366754in}{1.404564in}}%
\pgfpathlineto{\pgfqpoint{2.390759in}{1.399202in}}%
\pgfpathlineto{\pgfqpoint{2.458561in}{1.384057in}}%
\pgfpathlineto{\pgfqpoint{2.526363in}{1.368911in}}%
\pgfpathlineto{\pgfqpoint{2.594164in}{1.353766in}}%
\pgfpathlineto{\pgfqpoint{2.610848in}{1.350039in}}%
\pgfpathclose%
\pgfusepath{fill}%
\end{pgfscope}%
\begin{pgfscope}%
\pgfpathrectangle{\pgfqpoint{0.564660in}{0.521603in}}{\pgfqpoint{3.720000in}{3.020000in}}%
\pgfusepath{clip}%
\pgfsetbuttcap%
\pgfsetroundjoin%
\definecolor{currentfill}{rgb}{0.312255,0.312255,0.434442}%
\pgfsetfillcolor{currentfill}%
\pgfsetlinewidth{0.000000pt}%
\definecolor{currentstroke}{rgb}{0.000000,0.000000,0.000000}%
\pgfsetstrokecolor{currentstroke}%
\pgfsetdash{}{0pt}%
\pgfpathmoveto{\pgfqpoint{1.102529in}{1.554543in}}%
\pgfpathlineto{\pgfqpoint{1.105105in}{1.568140in}}%
\pgfpathlineto{\pgfqpoint{1.102695in}{1.622665in}}%
\pgfpathlineto{\pgfqpoint{1.102529in}{1.623097in}}%
\pgfpathlineto{\pgfqpoint{1.101834in}{1.622665in}}%
\pgfpathlineto{\pgfqpoint{1.091247in}{1.568140in}}%
\pgfpathclose%
\pgfusepath{fill}%
\end{pgfscope}%
\begin{pgfscope}%
\pgfpathrectangle{\pgfqpoint{0.564660in}{0.521603in}}{\pgfqpoint{3.720000in}{3.020000in}}%
\pgfusepath{clip}%
\pgfsetbuttcap%
\pgfsetroundjoin%
\definecolor{currentfill}{rgb}{0.439216,0.484130,0.564216}%
\pgfsetfillcolor{currentfill}%
\pgfsetlinewidth{0.000000pt}%
\definecolor{currentstroke}{rgb}{0.000000,0.000000,0.000000}%
\pgfsetstrokecolor{currentstroke}%
\pgfsetdash{}{0pt}%
\pgfpathmoveto{\pgfqpoint{2.797569in}{1.725434in}}%
\pgfpathlineto{\pgfqpoint{2.865371in}{1.716699in}}%
\pgfpathlineto{\pgfqpoint{2.933172in}{1.707964in}}%
\pgfpathlineto{\pgfqpoint{3.000974in}{1.704773in}}%
\pgfpathlineto{\pgfqpoint{3.051071in}{1.731715in}}%
\pgfpathlineto{\pgfqpoint{3.068776in}{1.740340in}}%
\pgfpathlineto{\pgfqpoint{3.136577in}{1.778503in}}%
\pgfpathlineto{\pgfqpoint{3.150323in}{1.786240in}}%
\pgfpathlineto{\pgfqpoint{3.204379in}{1.816666in}}%
\pgfpathlineto{\pgfqpoint{3.247194in}{1.840765in}}%
\pgfpathlineto{\pgfqpoint{3.272180in}{1.854830in}}%
\pgfpathlineto{\pgfqpoint{3.339982in}{1.892993in}}%
\pgfpathlineto{\pgfqpoint{3.344064in}{1.895291in}}%
\pgfpathlineto{\pgfqpoint{3.407784in}{1.931156in}}%
\pgfpathlineto{\pgfqpoint{3.440935in}{1.949816in}}%
\pgfpathlineto{\pgfqpoint{3.475585in}{1.969319in}}%
\pgfpathlineto{\pgfqpoint{3.530445in}{2.004341in}}%
\pgfpathlineto{\pgfqpoint{3.543387in}{2.012638in}}%
\pgfpathlineto{\pgfqpoint{3.610505in}{2.058866in}}%
\pgfpathlineto{\pgfqpoint{3.611189in}{2.059336in}}%
\pgfpathlineto{\pgfqpoint{3.678990in}{2.106035in}}%
\pgfpathlineto{\pgfqpoint{3.689670in}{2.113391in}}%
\pgfpathlineto{\pgfqpoint{3.746792in}{2.152734in}}%
\pgfpathlineto{\pgfqpoint{3.768834in}{2.167916in}}%
\pgfpathlineto{\pgfqpoint{3.814593in}{2.199433in}}%
\pgfpathlineto{\pgfqpoint{3.847998in}{2.222441in}}%
\pgfpathlineto{\pgfqpoint{3.882395in}{2.246132in}}%
\pgfpathlineto{\pgfqpoint{3.927163in}{2.276966in}}%
\pgfpathlineto{\pgfqpoint{3.950197in}{2.292831in}}%
\pgfpathlineto{\pgfqpoint{4.006327in}{2.331492in}}%
\pgfpathlineto{\pgfqpoint{4.017998in}{2.339530in}}%
\pgfpathlineto{\pgfqpoint{4.017998in}{2.386017in}}%
\pgfpathlineto{\pgfqpoint{3.950197in}{2.440542in}}%
\pgfpathlineto{\pgfqpoint{3.882395in}{2.495067in}}%
\pgfpathlineto{\pgfqpoint{3.882395in}{2.549592in}}%
\pgfpathlineto{\pgfqpoint{3.868264in}{2.560956in}}%
\pgfpathlineto{\pgfqpoint{3.851765in}{2.549592in}}%
\pgfpathlineto{\pgfqpoint{3.814593in}{2.523990in}}%
\pgfpathlineto{\pgfqpoint{3.772600in}{2.495067in}}%
\pgfpathlineto{\pgfqpoint{3.746792in}{2.477291in}}%
\pgfpathlineto{\pgfqpoint{3.693436in}{2.440542in}}%
\pgfpathlineto{\pgfqpoint{3.678990in}{2.430592in}}%
\pgfpathlineto{\pgfqpoint{3.614272in}{2.386017in}}%
\pgfpathlineto{\pgfqpoint{3.611189in}{2.383893in}}%
\pgfpathlineto{\pgfqpoint{3.543387in}{2.337194in}}%
\pgfpathlineto{\pgfqpoint{3.534764in}{2.331492in}}%
\pgfpathlineto{\pgfqpoint{3.475585in}{2.292518in}}%
\pgfpathlineto{\pgfqpoint{3.447956in}{2.276966in}}%
\pgfpathlineto{\pgfqpoint{3.407784in}{2.254355in}}%
\pgfpathlineto{\pgfqpoint{3.351085in}{2.222441in}}%
\pgfpathlineto{\pgfqpoint{3.339982in}{2.216192in}}%
\pgfpathlineto{\pgfqpoint{3.272180in}{2.178028in}}%
\pgfpathlineto{\pgfqpoint{3.254215in}{2.167916in}}%
\pgfpathlineto{\pgfqpoint{3.204379in}{2.139865in}}%
\pgfpathlineto{\pgfqpoint{3.139209in}{2.113391in}}%
\pgfpathlineto{\pgfqpoint{3.136577in}{2.112063in}}%
\pgfpathlineto{\pgfqpoint{3.125530in}{2.113391in}}%
\pgfpathlineto{\pgfqpoint{3.068776in}{2.120703in}}%
\pgfpathlineto{\pgfqpoint{3.000974in}{2.129438in}}%
\pgfpathlineto{\pgfqpoint{2.933172in}{2.138173in}}%
\pgfpathlineto{\pgfqpoint{2.865371in}{2.146908in}}%
\pgfpathlineto{\pgfqpoint{2.797569in}{2.155643in}}%
\pgfpathlineto{\pgfqpoint{2.729768in}{2.164378in}}%
\pgfpathlineto{\pgfqpoint{2.702308in}{2.167916in}}%
\pgfpathlineto{\pgfqpoint{2.661966in}{2.173114in}}%
\pgfpathlineto{\pgfqpoint{2.594164in}{2.181849in}}%
\pgfpathlineto{\pgfqpoint{2.526363in}{2.190584in}}%
\pgfpathlineto{\pgfqpoint{2.458561in}{2.199319in}}%
\pgfpathlineto{\pgfqpoint{2.390759in}{2.208709in}}%
\pgfpathlineto{\pgfqpoint{2.326733in}{2.222441in}}%
\pgfpathlineto{\pgfqpoint{2.322958in}{2.223239in}}%
\pgfpathlineto{\pgfqpoint{2.255156in}{2.238384in}}%
\pgfpathlineto{\pgfqpoint{2.187355in}{2.253529in}}%
\pgfpathlineto{\pgfqpoint{2.119553in}{2.268675in}}%
\pgfpathlineto{\pgfqpoint{2.082433in}{2.276966in}}%
\pgfpathlineto{\pgfqpoint{2.051751in}{2.283820in}}%
\pgfpathlineto{\pgfqpoint{1.983950in}{2.298965in}}%
\pgfpathlineto{\pgfqpoint{1.916148in}{2.304681in}}%
\pgfpathlineto{\pgfqpoint{1.848347in}{2.303672in}}%
\pgfpathlineto{\pgfqpoint{1.800582in}{2.331492in}}%
\pgfpathlineto{\pgfqpoint{1.780545in}{2.341934in}}%
\pgfpathlineto{\pgfqpoint{1.738529in}{2.386017in}}%
\pgfpathlineto{\pgfqpoint{1.712743in}{2.413070in}}%
\pgfpathlineto{\pgfqpoint{1.686559in}{2.440542in}}%
\pgfpathlineto{\pgfqpoint{1.644942in}{2.484205in}}%
\pgfpathlineto{\pgfqpoint{1.634589in}{2.495067in}}%
\pgfpathlineto{\pgfqpoint{1.582619in}{2.549592in}}%
\pgfpathlineto{\pgfqpoint{1.577140in}{2.555340in}}%
\pgfpathlineto{\pgfqpoint{1.530649in}{2.604117in}}%
\pgfpathlineto{\pgfqpoint{1.509338in}{2.626475in}}%
\pgfpathlineto{\pgfqpoint{1.478679in}{2.658642in}}%
\pgfpathlineto{\pgfqpoint{1.441537in}{2.697610in}}%
\pgfpathlineto{\pgfqpoint{1.404225in}{2.658642in}}%
\pgfpathlineto{\pgfqpoint{1.373735in}{2.649805in}}%
\pgfpathlineto{\pgfqpoint{1.373735in}{2.604117in}}%
\pgfpathlineto{\pgfqpoint{1.373735in}{2.549592in}}%
\pgfpathlineto{\pgfqpoint{1.305934in}{2.495067in}}%
\pgfpathlineto{\pgfqpoint{1.305934in}{2.446237in}}%
\pgfpathlineto{\pgfqpoint{1.321726in}{2.440542in}}%
\pgfpathlineto{\pgfqpoint{1.373696in}{2.386017in}}%
\pgfpathlineto{\pgfqpoint{1.373735in}{2.385975in}}%
\pgfpathlineto{\pgfqpoint{1.425666in}{2.331492in}}%
\pgfpathlineto{\pgfqpoint{1.441537in}{2.314840in}}%
\pgfpathlineto{\pgfqpoint{1.477636in}{2.276966in}}%
\pgfpathlineto{\pgfqpoint{1.509338in}{2.243705in}}%
\pgfpathlineto{\pgfqpoint{1.529606in}{2.222441in}}%
\pgfpathlineto{\pgfqpoint{1.577140in}{2.172570in}}%
\pgfpathlineto{\pgfqpoint{1.581576in}{2.167916in}}%
\pgfpathlineto{\pgfqpoint{1.633546in}{2.113391in}}%
\pgfpathlineto{\pgfqpoint{1.644942in}{2.101435in}}%
\pgfpathlineto{\pgfqpoint{1.685516in}{2.058866in}}%
\pgfpathlineto{\pgfqpoint{1.712743in}{2.030299in}}%
\pgfpathlineto{\pgfqpoint{1.737486in}{2.004341in}}%
\pgfpathlineto{\pgfqpoint{1.780545in}{1.959164in}}%
\pgfpathlineto{\pgfqpoint{1.789455in}{1.949816in}}%
\pgfpathlineto{\pgfqpoint{1.841425in}{1.895291in}}%
\pgfpathlineto{\pgfqpoint{1.848347in}{1.888029in}}%
\pgfpathlineto{\pgfqpoint{1.916148in}{1.876946in}}%
\pgfpathlineto{\pgfqpoint{1.983950in}{1.877955in}}%
\pgfpathlineto{\pgfqpoint{2.051751in}{1.878612in}}%
\pgfpathlineto{\pgfqpoint{2.119553in}{1.864229in}}%
\pgfpathlineto{\pgfqpoint{2.187355in}{1.849084in}}%
\pgfpathlineto{\pgfqpoint{2.224593in}{1.840765in}}%
\pgfpathlineto{\pgfqpoint{2.255156in}{1.833938in}}%
\pgfpathlineto{\pgfqpoint{2.322958in}{1.818793in}}%
\pgfpathlineto{\pgfqpoint{2.390759in}{1.803648in}}%
\pgfpathlineto{\pgfqpoint{2.458561in}{1.788502in}}%
\pgfpathlineto{\pgfqpoint{2.468688in}{1.786240in}}%
\pgfpathlineto{\pgfqpoint{2.526363in}{1.773357in}}%
\pgfpathlineto{\pgfqpoint{2.594164in}{1.758212in}}%
\pgfpathlineto{\pgfqpoint{2.661966in}{1.743597in}}%
\pgfpathlineto{\pgfqpoint{2.729768in}{1.734169in}}%
\pgfpathlineto{\pgfqpoint{2.748816in}{1.731715in}}%
\pgfpathclose%
\pgfusepath{fill}%
\end{pgfscope}%
\begin{pgfscope}%
\pgfpathrectangle{\pgfqpoint{0.564660in}{0.521603in}}{\pgfqpoint{3.720000in}{3.020000in}}%
\pgfusepath{clip}%
\pgfsetbuttcap%
\pgfsetroundjoin%
\definecolor{currentfill}{rgb}{0.439216,0.484130,0.564216}%
\pgfsetfillcolor{currentfill}%
\pgfsetlinewidth{0.000000pt}%
\definecolor{currentstroke}{rgb}{0.000000,0.000000,0.000000}%
\pgfsetstrokecolor{currentstroke}%
\pgfsetdash{}{0pt}%
\pgfpathmoveto{\pgfqpoint{1.276312in}{2.386017in}}%
\pgfpathlineto{\pgfqpoint{1.295552in}{2.432193in}}%
\pgfpathlineto{\pgfqpoint{1.238132in}{2.386017in}}%
\pgfpathlineto{\pgfqpoint{1.238132in}{2.363523in}}%
\pgfpathclose%
\pgfusepath{fill}%
\end{pgfscope}%
\begin{pgfscope}%
\pgfpathrectangle{\pgfqpoint{0.564660in}{0.521603in}}{\pgfqpoint{3.720000in}{3.020000in}}%
\pgfusepath{clip}%
\pgfsetbuttcap%
\pgfsetroundjoin%
\definecolor{currentfill}{rgb}{0.562745,0.653983,0.687745}%
\pgfsetfillcolor{currentfill}%
\pgfsetlinewidth{0.000000pt}%
\definecolor{currentstroke}{rgb}{0.000000,0.000000,0.000000}%
\pgfsetstrokecolor{currentstroke}%
\pgfsetdash{}{0pt}%
\pgfpathmoveto{\pgfqpoint{3.136577in}{2.112063in}}%
\pgfpathlineto{\pgfqpoint{3.139209in}{2.113391in}}%
\pgfpathlineto{\pgfqpoint{3.204379in}{2.139865in}}%
\pgfpathlineto{\pgfqpoint{3.254215in}{2.167916in}}%
\pgfpathlineto{\pgfqpoint{3.272180in}{2.178028in}}%
\pgfpathlineto{\pgfqpoint{3.339982in}{2.216192in}}%
\pgfpathlineto{\pgfqpoint{3.351085in}{2.222441in}}%
\pgfpathlineto{\pgfqpoint{3.407784in}{2.254355in}}%
\pgfpathlineto{\pgfqpoint{3.447956in}{2.276966in}}%
\pgfpathlineto{\pgfqpoint{3.475585in}{2.292518in}}%
\pgfpathlineto{\pgfqpoint{3.534764in}{2.331492in}}%
\pgfpathlineto{\pgfqpoint{3.543387in}{2.337194in}}%
\pgfpathlineto{\pgfqpoint{3.611189in}{2.383893in}}%
\pgfpathlineto{\pgfqpoint{3.614272in}{2.386017in}}%
\pgfpathlineto{\pgfqpoint{3.678990in}{2.430592in}}%
\pgfpathlineto{\pgfqpoint{3.693436in}{2.440542in}}%
\pgfpathlineto{\pgfqpoint{3.746792in}{2.477291in}}%
\pgfpathlineto{\pgfqpoint{3.772600in}{2.495067in}}%
\pgfpathlineto{\pgfqpoint{3.814593in}{2.523990in}}%
\pgfpathlineto{\pgfqpoint{3.851765in}{2.549592in}}%
\pgfpathlineto{\pgfqpoint{3.868264in}{2.560956in}}%
\pgfpathlineto{\pgfqpoint{3.814593in}{2.604117in}}%
\pgfpathlineto{\pgfqpoint{3.746792in}{2.658642in}}%
\pgfpathlineto{\pgfqpoint{3.678990in}{2.713167in}}%
\pgfpathlineto{\pgfqpoint{3.650870in}{2.735781in}}%
\pgfpathlineto{\pgfqpoint{3.618038in}{2.713167in}}%
\pgfpathlineto{\pgfqpoint{3.611189in}{2.708450in}}%
\pgfpathlineto{\pgfqpoint{3.543387in}{2.661751in}}%
\pgfpathlineto{\pgfqpoint{3.538827in}{2.658642in}}%
\pgfpathlineto{\pgfqpoint{3.475585in}{2.615717in}}%
\pgfpathlineto{\pgfqpoint{3.454977in}{2.604117in}}%
\pgfpathlineto{\pgfqpoint{3.407784in}{2.577554in}}%
\pgfpathlineto{\pgfqpoint{3.358106in}{2.549592in}}%
\pgfpathlineto{\pgfqpoint{3.339982in}{2.539390in}}%
\pgfpathlineto{\pgfqpoint{3.272180in}{2.524707in}}%
\pgfpathlineto{\pgfqpoint{3.204379in}{2.533442in}}%
\pgfpathlineto{\pgfqpoint{3.136577in}{2.542177in}}%
\pgfpathlineto{\pgfqpoint{3.079023in}{2.549592in}}%
\pgfpathlineto{\pgfqpoint{3.068776in}{2.550912in}}%
\pgfpathlineto{\pgfqpoint{3.000974in}{2.559647in}}%
\pgfpathlineto{\pgfqpoint{2.933172in}{2.568382in}}%
\pgfpathlineto{\pgfqpoint{2.865371in}{2.577118in}}%
\pgfpathlineto{\pgfqpoint{2.797569in}{2.585853in}}%
\pgfpathlineto{\pgfqpoint{2.729768in}{2.594588in}}%
\pgfpathlineto{\pgfqpoint{2.661966in}{2.603323in}}%
\pgfpathlineto{\pgfqpoint{2.655801in}{2.604117in}}%
\pgfpathlineto{\pgfqpoint{2.594164in}{2.612058in}}%
\pgfpathlineto{\pgfqpoint{2.526363in}{2.620793in}}%
\pgfpathlineto{\pgfqpoint{2.458561in}{2.629528in}}%
\pgfpathlineto{\pgfqpoint{2.390759in}{2.638263in}}%
\pgfpathlineto{\pgfqpoint{2.322958in}{2.646998in}}%
\pgfpathlineto{\pgfqpoint{2.255156in}{2.655734in}}%
\pgfpathlineto{\pgfqpoint{2.232579in}{2.658642in}}%
\pgfpathlineto{\pgfqpoint{2.187355in}{2.664469in}}%
\pgfpathlineto{\pgfqpoint{2.119553in}{2.673809in}}%
\pgfpathlineto{\pgfqpoint{2.051751in}{2.688266in}}%
\pgfpathlineto{\pgfqpoint{1.983950in}{2.703411in}}%
\pgfpathlineto{\pgfqpoint{1.940273in}{2.713167in}}%
\pgfpathlineto{\pgfqpoint{1.916148in}{2.718556in}}%
\pgfpathlineto{\pgfqpoint{1.848347in}{2.731408in}}%
\pgfpathlineto{\pgfqpoint{1.780545in}{2.730399in}}%
\pgfpathlineto{\pgfqpoint{1.743764in}{2.767693in}}%
\pgfpathlineto{\pgfqpoint{1.712743in}{2.795840in}}%
\pgfpathlineto{\pgfqpoint{1.687602in}{2.822218in}}%
\pgfpathlineto{\pgfqpoint{1.644942in}{2.866975in}}%
\pgfpathlineto{\pgfqpoint{1.635632in}{2.876743in}}%
\pgfpathlineto{\pgfqpoint{1.583662in}{2.931268in}}%
\pgfpathlineto{\pgfqpoint{1.577140in}{2.938110in}}%
\pgfpathlineto{\pgfqpoint{1.534395in}{2.951418in}}%
\pgfpathlineto{\pgfqpoint{1.509338in}{2.931268in}}%
\pgfpathlineto{\pgfqpoint{1.509338in}{2.876743in}}%
\pgfpathlineto{\pgfqpoint{1.509338in}{2.822218in}}%
\pgfpathlineto{\pgfqpoint{1.441537in}{2.767693in}}%
\pgfpathlineto{\pgfqpoint{1.441537in}{2.713167in}}%
\pgfpathlineto{\pgfqpoint{1.373735in}{2.658642in}}%
\pgfpathlineto{\pgfqpoint{1.373735in}{2.649805in}}%
\pgfpathlineto{\pgfqpoint{1.404225in}{2.658642in}}%
\pgfpathlineto{\pgfqpoint{1.441537in}{2.697610in}}%
\pgfpathlineto{\pgfqpoint{1.478679in}{2.658642in}}%
\pgfpathlineto{\pgfqpoint{1.509338in}{2.626475in}}%
\pgfpathlineto{\pgfqpoint{1.530649in}{2.604117in}}%
\pgfpathlineto{\pgfqpoint{1.577140in}{2.555340in}}%
\pgfpathlineto{\pgfqpoint{1.582619in}{2.549592in}}%
\pgfpathlineto{\pgfqpoint{1.634589in}{2.495067in}}%
\pgfpathlineto{\pgfqpoint{1.644942in}{2.484205in}}%
\pgfpathlineto{\pgfqpoint{1.686559in}{2.440542in}}%
\pgfpathlineto{\pgfqpoint{1.712743in}{2.413070in}}%
\pgfpathlineto{\pgfqpoint{1.738529in}{2.386017in}}%
\pgfpathlineto{\pgfqpoint{1.780545in}{2.341934in}}%
\pgfpathlineto{\pgfqpoint{1.800582in}{2.331492in}}%
\pgfpathlineto{\pgfqpoint{1.848347in}{2.303672in}}%
\pgfpathlineto{\pgfqpoint{1.916148in}{2.304681in}}%
\pgfpathlineto{\pgfqpoint{1.983950in}{2.298965in}}%
\pgfpathlineto{\pgfqpoint{2.051751in}{2.283820in}}%
\pgfpathlineto{\pgfqpoint{2.082433in}{2.276966in}}%
\pgfpathlineto{\pgfqpoint{2.119553in}{2.268675in}}%
\pgfpathlineto{\pgfqpoint{2.187355in}{2.253529in}}%
\pgfpathlineto{\pgfqpoint{2.255156in}{2.238384in}}%
\pgfpathlineto{\pgfqpoint{2.322958in}{2.223239in}}%
\pgfpathlineto{\pgfqpoint{2.326733in}{2.222441in}}%
\pgfpathlineto{\pgfqpoint{2.390759in}{2.208709in}}%
\pgfpathlineto{\pgfqpoint{2.458561in}{2.199319in}}%
\pgfpathlineto{\pgfqpoint{2.526363in}{2.190584in}}%
\pgfpathlineto{\pgfqpoint{2.594164in}{2.181849in}}%
\pgfpathlineto{\pgfqpoint{2.661966in}{2.173114in}}%
\pgfpathlineto{\pgfqpoint{2.702308in}{2.167916in}}%
\pgfpathlineto{\pgfqpoint{2.729768in}{2.164378in}}%
\pgfpathlineto{\pgfqpoint{2.797569in}{2.155643in}}%
\pgfpathlineto{\pgfqpoint{2.865371in}{2.146908in}}%
\pgfpathlineto{\pgfqpoint{2.933172in}{2.138173in}}%
\pgfpathlineto{\pgfqpoint{3.000974in}{2.129438in}}%
\pgfpathlineto{\pgfqpoint{3.068776in}{2.120703in}}%
\pgfpathlineto{\pgfqpoint{3.125530in}{2.113391in}}%
\pgfpathclose%
\pgfusepath{fill}%
\end{pgfscope}%
\begin{pgfscope}%
\pgfpathrectangle{\pgfqpoint{0.564660in}{0.521603in}}{\pgfqpoint{3.720000in}{3.020000in}}%
\pgfusepath{clip}%
\pgfsetbuttcap%
\pgfsetroundjoin%
\definecolor{currentfill}{rgb}{0.710478,0.814706,0.814706}%
\pgfsetfillcolor{currentfill}%
\pgfsetlinewidth{0.000000pt}%
\definecolor{currentstroke}{rgb}{0.000000,0.000000,0.000000}%
\pgfsetstrokecolor{currentstroke}%
\pgfsetdash{}{0pt}%
\pgfpathmoveto{\pgfqpoint{3.136577in}{2.542177in}}%
\pgfpathlineto{\pgfqpoint{3.204379in}{2.533442in}}%
\pgfpathlineto{\pgfqpoint{3.272180in}{2.524707in}}%
\pgfpathlineto{\pgfqpoint{3.339982in}{2.539390in}}%
\pgfpathlineto{\pgfqpoint{3.358106in}{2.549592in}}%
\pgfpathlineto{\pgfqpoint{3.407784in}{2.577554in}}%
\pgfpathlineto{\pgfqpoint{3.454977in}{2.604117in}}%
\pgfpathlineto{\pgfqpoint{3.475585in}{2.615717in}}%
\pgfpathlineto{\pgfqpoint{3.538827in}{2.658642in}}%
\pgfpathlineto{\pgfqpoint{3.543387in}{2.661751in}}%
\pgfpathlineto{\pgfqpoint{3.611189in}{2.708450in}}%
\pgfpathlineto{\pgfqpoint{3.618038in}{2.713167in}}%
\pgfpathlineto{\pgfqpoint{3.650870in}{2.735781in}}%
\pgfpathlineto{\pgfqpoint{3.611189in}{2.767693in}}%
\pgfpathlineto{\pgfqpoint{3.543387in}{2.822218in}}%
\pgfpathlineto{\pgfqpoint{3.543387in}{2.876743in}}%
\pgfpathlineto{\pgfqpoint{3.475585in}{2.931268in}}%
\pgfpathlineto{\pgfqpoint{3.407784in}{2.931268in}}%
\pgfpathlineto{\pgfqpoint{3.339982in}{2.931268in}}%
\pgfpathlineto{\pgfqpoint{3.317900in}{2.949026in}}%
\pgfpathlineto{\pgfqpoint{3.272180in}{2.954916in}}%
\pgfpathlineto{\pgfqpoint{3.204379in}{2.963651in}}%
\pgfpathlineto{\pgfqpoint{3.136577in}{2.972386in}}%
\pgfpathlineto{\pgfqpoint{3.068776in}{2.981122in}}%
\pgfpathlineto{\pgfqpoint{3.032515in}{2.985793in}}%
\pgfpathlineto{\pgfqpoint{3.000974in}{2.989857in}}%
\pgfpathlineto{\pgfqpoint{2.933172in}{2.998592in}}%
\pgfpathlineto{\pgfqpoint{2.865371in}{3.007327in}}%
\pgfpathlineto{\pgfqpoint{2.797569in}{3.016062in}}%
\pgfpathlineto{\pgfqpoint{2.729768in}{3.024797in}}%
\pgfpathlineto{\pgfqpoint{2.661966in}{3.033532in}}%
\pgfpathlineto{\pgfqpoint{2.609293in}{3.040318in}}%
\pgfpathlineto{\pgfqpoint{2.594164in}{3.042267in}}%
\pgfpathlineto{\pgfqpoint{2.526363in}{3.051002in}}%
\pgfpathlineto{\pgfqpoint{2.458561in}{3.059738in}}%
\pgfpathlineto{\pgfqpoint{2.390759in}{3.068473in}}%
\pgfpathlineto{\pgfqpoint{2.322958in}{3.077208in}}%
\pgfpathlineto{\pgfqpoint{2.255156in}{3.085943in}}%
\pgfpathlineto{\pgfqpoint{2.187355in}{3.094678in}}%
\pgfpathlineto{\pgfqpoint{2.186072in}{3.094843in}}%
\pgfpathlineto{\pgfqpoint{2.119553in}{3.103413in}}%
\pgfpathlineto{\pgfqpoint{2.051751in}{3.112148in}}%
\pgfpathlineto{\pgfqpoint{1.983950in}{3.120883in}}%
\pgfpathlineto{\pgfqpoint{1.916148in}{3.129618in}}%
\pgfpathlineto{\pgfqpoint{1.848347in}{3.138846in}}%
\pgfpathlineto{\pgfqpoint{1.799178in}{3.149368in}}%
\pgfpathlineto{\pgfqpoint{1.780545in}{3.153293in}}%
\pgfpathlineto{\pgfqpoint{1.712743in}{3.178610in}}%
\pgfpathlineto{\pgfqpoint{1.688645in}{3.203894in}}%
\pgfpathlineto{\pgfqpoint{1.669682in}{3.223789in}}%
\pgfpathlineto{\pgfqpoint{1.644942in}{3.203894in}}%
\pgfpathlineto{\pgfqpoint{1.644942in}{3.149368in}}%
\pgfpathlineto{\pgfqpoint{1.644942in}{3.094843in}}%
\pgfpathlineto{\pgfqpoint{1.577140in}{3.040318in}}%
\pgfpathlineto{\pgfqpoint{1.577140in}{2.985793in}}%
\pgfpathlineto{\pgfqpoint{1.534395in}{2.951418in}}%
\pgfpathlineto{\pgfqpoint{1.577140in}{2.938110in}}%
\pgfpathlineto{\pgfqpoint{1.583662in}{2.931268in}}%
\pgfpathlineto{\pgfqpoint{1.635632in}{2.876743in}}%
\pgfpathlineto{\pgfqpoint{1.644942in}{2.866975in}}%
\pgfpathlineto{\pgfqpoint{1.687602in}{2.822218in}}%
\pgfpathlineto{\pgfqpoint{1.712743in}{2.795840in}}%
\pgfpathlineto{\pgfqpoint{1.743764in}{2.767693in}}%
\pgfpathlineto{\pgfqpoint{1.780545in}{2.730399in}}%
\pgfpathlineto{\pgfqpoint{1.848347in}{2.731408in}}%
\pgfpathlineto{\pgfqpoint{1.916148in}{2.718556in}}%
\pgfpathlineto{\pgfqpoint{1.940273in}{2.713167in}}%
\pgfpathlineto{\pgfqpoint{1.983950in}{2.703411in}}%
\pgfpathlineto{\pgfqpoint{2.051751in}{2.688266in}}%
\pgfpathlineto{\pgfqpoint{2.119553in}{2.673809in}}%
\pgfpathlineto{\pgfqpoint{2.187355in}{2.664469in}}%
\pgfpathlineto{\pgfqpoint{2.232579in}{2.658642in}}%
\pgfpathlineto{\pgfqpoint{2.255156in}{2.655734in}}%
\pgfpathlineto{\pgfqpoint{2.322958in}{2.646998in}}%
\pgfpathlineto{\pgfqpoint{2.390759in}{2.638263in}}%
\pgfpathlineto{\pgfqpoint{2.458561in}{2.629528in}}%
\pgfpathlineto{\pgfqpoint{2.526363in}{2.620793in}}%
\pgfpathlineto{\pgfqpoint{2.594164in}{2.612058in}}%
\pgfpathlineto{\pgfqpoint{2.655801in}{2.604117in}}%
\pgfpathlineto{\pgfqpoint{2.661966in}{2.603323in}}%
\pgfpathlineto{\pgfqpoint{2.729768in}{2.594588in}}%
\pgfpathlineto{\pgfqpoint{2.797569in}{2.585853in}}%
\pgfpathlineto{\pgfqpoint{2.865371in}{2.577118in}}%
\pgfpathlineto{\pgfqpoint{2.933172in}{2.568382in}}%
\pgfpathlineto{\pgfqpoint{3.000974in}{2.559647in}}%
\pgfpathlineto{\pgfqpoint{3.068776in}{2.550912in}}%
\pgfpathlineto{\pgfqpoint{3.079023in}{2.549592in}}%
\pgfpathclose%
\pgfusepath{fill}%
\end{pgfscope}%
\begin{pgfscope}%
\pgfpathrectangle{\pgfqpoint{0.564660in}{0.521603in}}{\pgfqpoint{3.720000in}{3.020000in}}%
\pgfusepath{clip}%
\pgfsetbuttcap%
\pgfsetroundjoin%
\definecolor{currentfill}{rgb}{0.903493,0.938235,0.938235}%
\pgfsetfillcolor{currentfill}%
\pgfsetlinewidth{0.000000pt}%
\definecolor{currentstroke}{rgb}{0.000000,0.000000,0.000000}%
\pgfsetstrokecolor{currentstroke}%
\pgfsetdash{}{0pt}%
\pgfpathmoveto{\pgfqpoint{3.068776in}{2.981122in}}%
\pgfpathlineto{\pgfqpoint{3.136577in}{2.972386in}}%
\pgfpathlineto{\pgfqpoint{3.204379in}{2.963651in}}%
\pgfpathlineto{\pgfqpoint{3.272180in}{2.954916in}}%
\pgfpathlineto{\pgfqpoint{3.317900in}{2.949026in}}%
\pgfpathlineto{\pgfqpoint{3.272180in}{2.985793in}}%
\pgfpathlineto{\pgfqpoint{3.204379in}{2.985793in}}%
\pgfpathlineto{\pgfqpoint{3.136577in}{2.985793in}}%
\pgfpathlineto{\pgfqpoint{3.068776in}{3.040318in}}%
\pgfpathlineto{\pgfqpoint{3.000974in}{3.040318in}}%
\pgfpathlineto{\pgfqpoint{2.933172in}{3.040318in}}%
\pgfpathlineto{\pgfqpoint{2.865371in}{3.094843in}}%
\pgfpathlineto{\pgfqpoint{2.797569in}{3.094843in}}%
\pgfpathlineto{\pgfqpoint{2.729768in}{3.094843in}}%
\pgfpathlineto{\pgfqpoint{2.661966in}{3.094843in}}%
\pgfpathlineto{\pgfqpoint{2.594164in}{3.149368in}}%
\pgfpathlineto{\pgfqpoint{2.526363in}{3.149368in}}%
\pgfpathlineto{\pgfqpoint{2.458561in}{3.149368in}}%
\pgfpathlineto{\pgfqpoint{2.390759in}{3.203894in}}%
\pgfpathlineto{\pgfqpoint{2.322958in}{3.203894in}}%
\pgfpathlineto{\pgfqpoint{2.255156in}{3.203894in}}%
\pgfpathlineto{\pgfqpoint{2.187355in}{3.203894in}}%
\pgfpathlineto{\pgfqpoint{2.119553in}{3.258419in}}%
\pgfpathlineto{\pgfqpoint{2.051751in}{3.258419in}}%
\pgfpathlineto{\pgfqpoint{1.983950in}{3.258419in}}%
\pgfpathlineto{\pgfqpoint{1.916148in}{3.312944in}}%
\pgfpathlineto{\pgfqpoint{1.848347in}{3.312944in}}%
\pgfpathlineto{\pgfqpoint{1.780545in}{3.312944in}}%
\pgfpathlineto{\pgfqpoint{1.712743in}{3.312944in}}%
\pgfpathlineto{\pgfqpoint{1.712743in}{3.258419in}}%
\pgfpathlineto{\pgfqpoint{1.669682in}{3.223789in}}%
\pgfpathlineto{\pgfqpoint{1.688645in}{3.203894in}}%
\pgfpathlineto{\pgfqpoint{1.712743in}{3.178610in}}%
\pgfpathlineto{\pgfqpoint{1.780545in}{3.153293in}}%
\pgfpathlineto{\pgfqpoint{1.799178in}{3.149368in}}%
\pgfpathlineto{\pgfqpoint{1.848347in}{3.138846in}}%
\pgfpathlineto{\pgfqpoint{1.916148in}{3.129618in}}%
\pgfpathlineto{\pgfqpoint{1.983950in}{3.120883in}}%
\pgfpathlineto{\pgfqpoint{2.051751in}{3.112148in}}%
\pgfpathlineto{\pgfqpoint{2.119553in}{3.103413in}}%
\pgfpathlineto{\pgfqpoint{2.186072in}{3.094843in}}%
\pgfpathlineto{\pgfqpoint{2.187355in}{3.094678in}}%
\pgfpathlineto{\pgfqpoint{2.255156in}{3.085943in}}%
\pgfpathlineto{\pgfqpoint{2.322958in}{3.077208in}}%
\pgfpathlineto{\pgfqpoint{2.390759in}{3.068473in}}%
\pgfpathlineto{\pgfqpoint{2.458561in}{3.059738in}}%
\pgfpathlineto{\pgfqpoint{2.526363in}{3.051002in}}%
\pgfpathlineto{\pgfqpoint{2.594164in}{3.042267in}}%
\pgfpathlineto{\pgfqpoint{2.609293in}{3.040318in}}%
\pgfpathlineto{\pgfqpoint{2.661966in}{3.033532in}}%
\pgfpathlineto{\pgfqpoint{2.729768in}{3.024797in}}%
\pgfpathlineto{\pgfqpoint{2.797569in}{3.016062in}}%
\pgfpathlineto{\pgfqpoint{2.865371in}{3.007327in}}%
\pgfpathlineto{\pgfqpoint{2.933172in}{2.998592in}}%
\pgfpathlineto{\pgfqpoint{3.000974in}{2.989857in}}%
\pgfpathlineto{\pgfqpoint{3.032515in}{2.985793in}}%
\pgfpathclose%
\pgfusepath{fill}%
\end{pgfscope}%
\begin{pgfscope}%
\pgfpathrectangle{\pgfqpoint{0.564660in}{0.521603in}}{\pgfqpoint{3.720000in}{3.020000in}}%
\pgfusepath{clip}%
\pgfsetbuttcap%
\pgfsetroundjoin%
\definecolor{currentfill}{rgb}{1.000000,1.000000,1.000000}%
\pgfsetfillcolor{currentfill}%
\pgfsetlinewidth{1.003750pt}%
\definecolor{currentstroke}{rgb}{0.000000,0.000000,0.000000}%
\pgfsetstrokecolor{currentstroke}%
\pgfsetdash{}{0pt}%
\pgfpathmoveto{\pgfqpoint{3.469480in}{2.911879in}}%
\pgfpathcurveto{\pgfqpoint{3.480530in}{2.911879in}}{\pgfqpoint{3.491129in}{2.916269in}}{\pgfqpoint{3.498943in}{2.924083in}}%
\pgfpathcurveto{\pgfqpoint{3.506756in}{2.931896in}}{\pgfqpoint{3.511147in}{2.942495in}}{\pgfqpoint{3.511147in}{2.953545in}}%
\pgfpathcurveto{\pgfqpoint{3.511147in}{2.964596in}}{\pgfqpoint{3.506756in}{2.975195in}}{\pgfqpoint{3.498943in}{2.983008in}}%
\pgfpathcurveto{\pgfqpoint{3.491129in}{2.990822in}}{\pgfqpoint{3.480530in}{2.995212in}}{\pgfqpoint{3.469480in}{2.995212in}}%
\pgfpathcurveto{\pgfqpoint{3.458430in}{2.995212in}}{\pgfqpoint{3.447831in}{2.990822in}}{\pgfqpoint{3.440017in}{2.983008in}}%
\pgfpathcurveto{\pgfqpoint{3.432204in}{2.975195in}}{\pgfqpoint{3.427813in}{2.964596in}}{\pgfqpoint{3.427813in}{2.953545in}}%
\pgfpathcurveto{\pgfqpoint{3.427813in}{2.942495in}}{\pgfqpoint{3.432204in}{2.931896in}}{\pgfqpoint{3.440017in}{2.924083in}}%
\pgfpathcurveto{\pgfqpoint{3.447831in}{2.916269in}}{\pgfqpoint{3.458430in}{2.911879in}}{\pgfqpoint{3.469480in}{2.911879in}}%
\pgfpathclose%
\pgfusepath{stroke,fill}%
\end{pgfscope}%
\begin{pgfscope}%
\pgfpathrectangle{\pgfqpoint{0.564660in}{0.521603in}}{\pgfqpoint{3.720000in}{3.020000in}}%
\pgfusepath{clip}%
\pgfsetbuttcap%
\pgfsetroundjoin%
\definecolor{currentfill}{rgb}{1.000000,1.000000,1.000000}%
\pgfsetfillcolor{currentfill}%
\pgfsetlinewidth{1.003750pt}%
\definecolor{currentstroke}{rgb}{0.000000,0.000000,0.000000}%
\pgfsetstrokecolor{currentstroke}%
\pgfsetdash{}{0pt}%
\pgfpathmoveto{\pgfqpoint{1.171593in}{1.744492in}}%
\pgfpathcurveto{\pgfqpoint{1.182643in}{1.744492in}}{\pgfqpoint{1.193242in}{1.748882in}}{\pgfqpoint{1.201056in}{1.756696in}}%
\pgfpathcurveto{\pgfqpoint{1.208869in}{1.764510in}}{\pgfqpoint{1.213260in}{1.775109in}}{\pgfqpoint{1.213260in}{1.786159in}}%
\pgfpathcurveto{\pgfqpoint{1.213260in}{1.797209in}}{\pgfqpoint{1.208869in}{1.807808in}}{\pgfqpoint{1.201056in}{1.815622in}}%
\pgfpathcurveto{\pgfqpoint{1.193242in}{1.823435in}}{\pgfqpoint{1.182643in}{1.827826in}}{\pgfqpoint{1.171593in}{1.827826in}}%
\pgfpathcurveto{\pgfqpoint{1.160543in}{1.827826in}}{\pgfqpoint{1.149944in}{1.823435in}}{\pgfqpoint{1.142130in}{1.815622in}}%
\pgfpathcurveto{\pgfqpoint{1.134317in}{1.807808in}}{\pgfqpoint{1.129926in}{1.797209in}}{\pgfqpoint{1.129926in}{1.786159in}}%
\pgfpathcurveto{\pgfqpoint{1.129926in}{1.775109in}}{\pgfqpoint{1.134317in}{1.764510in}}{\pgfqpoint{1.142130in}{1.756696in}}%
\pgfpathcurveto{\pgfqpoint{1.149944in}{1.748882in}}{\pgfqpoint{1.160543in}{1.744492in}}{\pgfqpoint{1.171593in}{1.744492in}}%
\pgfpathclose%
\pgfusepath{stroke,fill}%
\end{pgfscope}%
\begin{pgfscope}%
\pgfpathrectangle{\pgfqpoint{0.564660in}{0.521603in}}{\pgfqpoint{3.720000in}{3.020000in}}%
\pgfusepath{clip}%
\pgfsetbuttcap%
\pgfsetroundjoin%
\definecolor{currentfill}{rgb}{1.000000,1.000000,1.000000}%
\pgfsetfillcolor{currentfill}%
\pgfsetlinewidth{1.003750pt}%
\definecolor{currentstroke}{rgb}{0.000000,0.000000,0.000000}%
\pgfsetstrokecolor{currentstroke}%
\pgfsetdash{}{0pt}%
\pgfpathmoveto{\pgfqpoint{0.941940in}{1.749672in}}%
\pgfpathcurveto{\pgfqpoint{0.952990in}{1.749672in}}{\pgfqpoint{0.963589in}{1.754062in}}{\pgfqpoint{0.971403in}{1.761876in}}%
\pgfpathcurveto{\pgfqpoint{0.979216in}{1.769689in}}{\pgfqpoint{0.983606in}{1.780289in}}{\pgfqpoint{0.983606in}{1.791339in}}%
\pgfpathcurveto{\pgfqpoint{0.983606in}{1.802389in}}{\pgfqpoint{0.979216in}{1.812988in}}{\pgfqpoint{0.971403in}{1.820801in}}%
\pgfpathcurveto{\pgfqpoint{0.963589in}{1.828615in}}{\pgfqpoint{0.952990in}{1.833005in}}{\pgfqpoint{0.941940in}{1.833005in}}%
\pgfpathcurveto{\pgfqpoint{0.930890in}{1.833005in}}{\pgfqpoint{0.920291in}{1.828615in}}{\pgfqpoint{0.912477in}{1.820801in}}%
\pgfpathcurveto{\pgfqpoint{0.904663in}{1.812988in}}{\pgfqpoint{0.900273in}{1.802389in}}{\pgfqpoint{0.900273in}{1.791339in}}%
\pgfpathcurveto{\pgfqpoint{0.900273in}{1.780289in}}{\pgfqpoint{0.904663in}{1.769689in}}{\pgfqpoint{0.912477in}{1.761876in}}%
\pgfpathcurveto{\pgfqpoint{0.920291in}{1.754062in}}{\pgfqpoint{0.930890in}{1.749672in}}{\pgfqpoint{0.941940in}{1.749672in}}%
\pgfpathclose%
\pgfusepath{stroke,fill}%
\end{pgfscope}%
\begin{pgfscope}%
\pgfpathrectangle{\pgfqpoint{0.564660in}{0.521603in}}{\pgfqpoint{3.720000in}{3.020000in}}%
\pgfusepath{clip}%
\pgfsetbuttcap%
\pgfsetroundjoin%
\definecolor{currentfill}{rgb}{1.000000,1.000000,1.000000}%
\pgfsetfillcolor{currentfill}%
\pgfsetlinewidth{1.003750pt}%
\definecolor{currentstroke}{rgb}{0.000000,0.000000,0.000000}%
\pgfsetstrokecolor{currentstroke}%
\pgfsetdash{}{0pt}%
\pgfpathmoveto{\pgfqpoint{1.680850in}{0.972425in}}%
\pgfpathcurveto{\pgfqpoint{1.691900in}{0.972425in}}{\pgfqpoint{1.702499in}{0.976815in}}{\pgfqpoint{1.710313in}{0.984629in}}%
\pgfpathcurveto{\pgfqpoint{1.718127in}{0.992443in}}{\pgfqpoint{1.722517in}{1.003042in}}{\pgfqpoint{1.722517in}{1.014092in}}%
\pgfpathcurveto{\pgfqpoint{1.722517in}{1.025142in}}{\pgfqpoint{1.718127in}{1.035741in}}{\pgfqpoint{1.710313in}{1.043555in}}%
\pgfpathcurveto{\pgfqpoint{1.702499in}{1.051368in}}{\pgfqpoint{1.691900in}{1.055759in}}{\pgfqpoint{1.680850in}{1.055759in}}%
\pgfpathcurveto{\pgfqpoint{1.669800in}{1.055759in}}{\pgfqpoint{1.659201in}{1.051368in}}{\pgfqpoint{1.651387in}{1.043555in}}%
\pgfpathcurveto{\pgfqpoint{1.643574in}{1.035741in}}{\pgfqpoint{1.639183in}{1.025142in}}{\pgfqpoint{1.639183in}{1.014092in}}%
\pgfpathcurveto{\pgfqpoint{1.639183in}{1.003042in}}{\pgfqpoint{1.643574in}{0.992443in}}{\pgfqpoint{1.651387in}{0.984629in}}%
\pgfpathcurveto{\pgfqpoint{1.659201in}{0.976815in}}{\pgfqpoint{1.669800in}{0.972425in}}{\pgfqpoint{1.680850in}{0.972425in}}%
\pgfpathclose%
\pgfusepath{stroke,fill}%
\end{pgfscope}%
\begin{pgfscope}%
\pgfpathrectangle{\pgfqpoint{0.564660in}{0.521603in}}{\pgfqpoint{3.720000in}{3.020000in}}%
\pgfusepath{clip}%
\pgfsetbuttcap%
\pgfsetroundjoin%
\definecolor{currentfill}{rgb}{1.000000,1.000000,1.000000}%
\pgfsetfillcolor{currentfill}%
\pgfsetlinewidth{1.003750pt}%
\definecolor{currentstroke}{rgb}{0.000000,0.000000,0.000000}%
\pgfsetstrokecolor{currentstroke}%
\pgfsetdash{}{0pt}%
\pgfpathmoveto{\pgfqpoint{0.763521in}{0.982499in}}%
\pgfpathcurveto{\pgfqpoint{0.774571in}{0.982499in}}{\pgfqpoint{0.785170in}{0.986889in}}{\pgfqpoint{0.792983in}{0.994702in}}%
\pgfpathcurveto{\pgfqpoint{0.800797in}{1.002516in}}{\pgfqpoint{0.805187in}{1.013115in}}{\pgfqpoint{0.805187in}{1.024165in}}%
\pgfpathcurveto{\pgfqpoint{0.805187in}{1.035215in}}{\pgfqpoint{0.800797in}{1.045814in}}{\pgfqpoint{0.792983in}{1.053628in}}%
\pgfpathcurveto{\pgfqpoint{0.785170in}{1.061442in}}{\pgfqpoint{0.774571in}{1.065832in}}{\pgfqpoint{0.763521in}{1.065832in}}%
\pgfpathcurveto{\pgfqpoint{0.752471in}{1.065832in}}{\pgfqpoint{0.741872in}{1.061442in}}{\pgfqpoint{0.734058in}{1.053628in}}%
\pgfpathcurveto{\pgfqpoint{0.726244in}{1.045814in}}{\pgfqpoint{0.721854in}{1.035215in}}{\pgfqpoint{0.721854in}{1.024165in}}%
\pgfpathcurveto{\pgfqpoint{0.721854in}{1.013115in}}{\pgfqpoint{0.726244in}{1.002516in}}{\pgfqpoint{0.734058in}{0.994702in}}%
\pgfpathcurveto{\pgfqpoint{0.741872in}{0.986889in}}{\pgfqpoint{0.752471in}{0.982499in}}{\pgfqpoint{0.763521in}{0.982499in}}%
\pgfpathclose%
\pgfusepath{stroke,fill}%
\end{pgfscope}%
\begin{pgfscope}%
\pgfpathrectangle{\pgfqpoint{0.564660in}{0.521603in}}{\pgfqpoint{3.720000in}{3.020000in}}%
\pgfusepath{clip}%
\pgfsetbuttcap%
\pgfsetroundjoin%
\definecolor{currentfill}{rgb}{1.000000,1.000000,1.000000}%
\pgfsetfillcolor{currentfill}%
\pgfsetlinewidth{1.003750pt}%
\definecolor{currentstroke}{rgb}{0.000000,0.000000,0.000000}%
\pgfsetstrokecolor{currentstroke}%
\pgfsetdash{}{0pt}%
\pgfpathmoveto{\pgfqpoint{1.355680in}{0.737445in}}%
\pgfpathcurveto{\pgfqpoint{1.366730in}{0.737445in}}{\pgfqpoint{1.377329in}{0.741835in}}{\pgfqpoint{1.385143in}{0.749649in}}%
\pgfpathcurveto{\pgfqpoint{1.392956in}{0.757463in}}{\pgfqpoint{1.397346in}{0.768062in}}{\pgfqpoint{1.397346in}{0.779112in}}%
\pgfpathcurveto{\pgfqpoint{1.397346in}{0.790162in}}{\pgfqpoint{1.392956in}{0.800761in}}{\pgfqpoint{1.385143in}{0.808575in}}%
\pgfpathcurveto{\pgfqpoint{1.377329in}{0.816388in}}{\pgfqpoint{1.366730in}{0.820779in}}{\pgfqpoint{1.355680in}{0.820779in}}%
\pgfpathcurveto{\pgfqpoint{1.344630in}{0.820779in}}{\pgfqpoint{1.334031in}{0.816388in}}{\pgfqpoint{1.326217in}{0.808575in}}%
\pgfpathcurveto{\pgfqpoint{1.318403in}{0.800761in}}{\pgfqpoint{1.314013in}{0.790162in}}{\pgfqpoint{1.314013in}{0.779112in}}%
\pgfpathcurveto{\pgfqpoint{1.314013in}{0.768062in}}{\pgfqpoint{1.318403in}{0.757463in}}{\pgfqpoint{1.326217in}{0.749649in}}%
\pgfpathcurveto{\pgfqpoint{1.334031in}{0.741835in}}{\pgfqpoint{1.344630in}{0.737445in}}{\pgfqpoint{1.355680in}{0.737445in}}%
\pgfpathclose%
\pgfusepath{stroke,fill}%
\end{pgfscope}%
\begin{pgfscope}%
\pgfpathrectangle{\pgfqpoint{0.564660in}{0.521603in}}{\pgfqpoint{3.720000in}{3.020000in}}%
\pgfusepath{clip}%
\pgfsetbuttcap%
\pgfsetroundjoin%
\definecolor{currentfill}{rgb}{1.000000,1.000000,1.000000}%
\pgfsetfillcolor{currentfill}%
\pgfsetlinewidth{1.003750pt}%
\definecolor{currentstroke}{rgb}{0.000000,0.000000,0.000000}%
\pgfsetstrokecolor{currentstroke}%
\pgfsetdash{}{0pt}%
\pgfpathmoveto{\pgfqpoint{1.307448in}{0.654071in}}%
\pgfpathcurveto{\pgfqpoint{1.318498in}{0.654071in}}{\pgfqpoint{1.329097in}{0.658461in}}{\pgfqpoint{1.336910in}{0.666275in}}%
\pgfpathcurveto{\pgfqpoint{1.344724in}{0.674089in}}{\pgfqpoint{1.349114in}{0.684688in}}{\pgfqpoint{1.349114in}{0.695738in}}%
\pgfpathcurveto{\pgfqpoint{1.349114in}{0.706788in}}{\pgfqpoint{1.344724in}{0.717387in}}{\pgfqpoint{1.336910in}{0.725200in}}%
\pgfpathcurveto{\pgfqpoint{1.329097in}{0.733014in}}{\pgfqpoint{1.318498in}{0.737404in}}{\pgfqpoint{1.307448in}{0.737404in}}%
\pgfpathcurveto{\pgfqpoint{1.296398in}{0.737404in}}{\pgfqpoint{1.285799in}{0.733014in}}{\pgfqpoint{1.277985in}{0.725200in}}%
\pgfpathcurveto{\pgfqpoint{1.270171in}{0.717387in}}{\pgfqpoint{1.265781in}{0.706788in}}{\pgfqpoint{1.265781in}{0.695738in}}%
\pgfpathcurveto{\pgfqpoint{1.265781in}{0.684688in}}{\pgfqpoint{1.270171in}{0.674089in}}{\pgfqpoint{1.277985in}{0.666275in}}%
\pgfpathcurveto{\pgfqpoint{1.285799in}{0.658461in}}{\pgfqpoint{1.296398in}{0.654071in}}{\pgfqpoint{1.307448in}{0.654071in}}%
\pgfpathclose%
\pgfusepath{stroke,fill}%
\end{pgfscope}%
\begin{pgfscope}%
\pgfpathrectangle{\pgfqpoint{0.564660in}{0.521603in}}{\pgfqpoint{3.720000in}{3.020000in}}%
\pgfusepath{clip}%
\pgfsetbuttcap%
\pgfsetroundjoin%
\definecolor{currentfill}{rgb}{1.000000,1.000000,1.000000}%
\pgfsetfillcolor{currentfill}%
\pgfsetlinewidth{1.003750pt}%
\definecolor{currentstroke}{rgb}{0.000000,0.000000,0.000000}%
\pgfsetstrokecolor{currentstroke}%
\pgfsetdash{}{0pt}%
\pgfpathmoveto{\pgfqpoint{4.085800in}{2.313769in}}%
\pgfpathcurveto{\pgfqpoint{4.096850in}{2.313769in}}{\pgfqpoint{4.107449in}{2.318159in}}{\pgfqpoint{4.115263in}{2.325973in}}%
\pgfpathcurveto{\pgfqpoint{4.123076in}{2.333786in}}{\pgfqpoint{4.127467in}{2.344385in}}{\pgfqpoint{4.127467in}{2.355435in}}%
\pgfpathcurveto{\pgfqpoint{4.127467in}{2.366486in}}{\pgfqpoint{4.123076in}{2.377085in}}{\pgfqpoint{4.115263in}{2.384898in}}%
\pgfpathcurveto{\pgfqpoint{4.107449in}{2.392712in}}{\pgfqpoint{4.096850in}{2.397102in}}{\pgfqpoint{4.085800in}{2.397102in}}%
\pgfpathcurveto{\pgfqpoint{4.074750in}{2.397102in}}{\pgfqpoint{4.064151in}{2.392712in}}{\pgfqpoint{4.056337in}{2.384898in}}%
\pgfpathcurveto{\pgfqpoint{4.048523in}{2.377085in}}{\pgfqpoint{4.044133in}{2.366486in}}{\pgfqpoint{4.044133in}{2.355435in}}%
\pgfpathcurveto{\pgfqpoint{4.044133in}{2.344385in}}{\pgfqpoint{4.048523in}{2.333786in}}{\pgfqpoint{4.056337in}{2.325973in}}%
\pgfpathcurveto{\pgfqpoint{4.064151in}{2.318159in}}{\pgfqpoint{4.074750in}{2.313769in}}{\pgfqpoint{4.085800in}{2.313769in}}%
\pgfpathclose%
\pgfusepath{stroke,fill}%
\end{pgfscope}%
\begin{pgfscope}%
\pgfpathrectangle{\pgfqpoint{0.564660in}{0.521603in}}{\pgfqpoint{3.720000in}{3.020000in}}%
\pgfusepath{clip}%
\pgfsetbuttcap%
\pgfsetroundjoin%
\definecolor{currentfill}{rgb}{1.000000,1.000000,1.000000}%
\pgfsetfillcolor{currentfill}%
\pgfsetlinewidth{1.003750pt}%
\definecolor{currentstroke}{rgb}{0.000000,0.000000,0.000000}%
\pgfsetstrokecolor{currentstroke}%
\pgfsetdash{}{0pt}%
\pgfpathmoveto{\pgfqpoint{3.510562in}{1.703866in}}%
\pgfpathcurveto{\pgfqpoint{3.521612in}{1.703866in}}{\pgfqpoint{3.532211in}{1.708257in}}{\pgfqpoint{3.540025in}{1.716070in}}%
\pgfpathcurveto{\pgfqpoint{3.547839in}{1.723884in}}{\pgfqpoint{3.552229in}{1.734483in}}{\pgfqpoint{3.552229in}{1.745533in}}%
\pgfpathcurveto{\pgfqpoint{3.552229in}{1.756583in}}{\pgfqpoint{3.547839in}{1.767182in}}{\pgfqpoint{3.540025in}{1.774996in}}%
\pgfpathcurveto{\pgfqpoint{3.532211in}{1.782810in}}{\pgfqpoint{3.521612in}{1.787200in}}{\pgfqpoint{3.510562in}{1.787200in}}%
\pgfpathcurveto{\pgfqpoint{3.499512in}{1.787200in}}{\pgfqpoint{3.488913in}{1.782810in}}{\pgfqpoint{3.481099in}{1.774996in}}%
\pgfpathcurveto{\pgfqpoint{3.473286in}{1.767182in}}{\pgfqpoint{3.468895in}{1.756583in}}{\pgfqpoint{3.468895in}{1.745533in}}%
\pgfpathcurveto{\pgfqpoint{3.468895in}{1.734483in}}{\pgfqpoint{3.473286in}{1.723884in}}{\pgfqpoint{3.481099in}{1.716070in}}%
\pgfpathcurveto{\pgfqpoint{3.488913in}{1.708257in}}{\pgfqpoint{3.499512in}{1.703866in}}{\pgfqpoint{3.510562in}{1.703866in}}%
\pgfpathclose%
\pgfusepath{stroke,fill}%
\end{pgfscope}%
\begin{pgfscope}%
\pgfpathrectangle{\pgfqpoint{0.564660in}{0.521603in}}{\pgfqpoint{3.720000in}{3.020000in}}%
\pgfusepath{clip}%
\pgfsetbuttcap%
\pgfsetroundjoin%
\definecolor{currentfill}{rgb}{1.000000,1.000000,1.000000}%
\pgfsetfillcolor{currentfill}%
\pgfsetlinewidth{1.003750pt}%
\definecolor{currentstroke}{rgb}{0.000000,0.000000,0.000000}%
\pgfsetstrokecolor{currentstroke}%
\pgfsetdash{}{0pt}%
\pgfpathmoveto{\pgfqpoint{2.876111in}{1.338281in}}%
\pgfpathcurveto{\pgfqpoint{2.887161in}{1.338281in}}{\pgfqpoint{2.897760in}{1.342671in}}{\pgfqpoint{2.905574in}{1.350485in}}%
\pgfpathcurveto{\pgfqpoint{2.913387in}{1.358299in}}{\pgfqpoint{2.917777in}{1.368898in}}{\pgfqpoint{2.917777in}{1.379948in}}%
\pgfpathcurveto{\pgfqpoint{2.917777in}{1.390998in}}{\pgfqpoint{2.913387in}{1.401597in}}{\pgfqpoint{2.905574in}{1.409410in}}%
\pgfpathcurveto{\pgfqpoint{2.897760in}{1.417224in}}{\pgfqpoint{2.887161in}{1.421614in}}{\pgfqpoint{2.876111in}{1.421614in}}%
\pgfpathcurveto{\pgfqpoint{2.865061in}{1.421614in}}{\pgfqpoint{2.854462in}{1.417224in}}{\pgfqpoint{2.846648in}{1.409410in}}%
\pgfpathcurveto{\pgfqpoint{2.838834in}{1.401597in}}{\pgfqpoint{2.834444in}{1.390998in}}{\pgfqpoint{2.834444in}{1.379948in}}%
\pgfpathcurveto{\pgfqpoint{2.834444in}{1.368898in}}{\pgfqpoint{2.838834in}{1.358299in}}{\pgfqpoint{2.846648in}{1.350485in}}%
\pgfpathcurveto{\pgfqpoint{2.854462in}{1.342671in}}{\pgfqpoint{2.865061in}{1.338281in}}{\pgfqpoint{2.876111in}{1.338281in}}%
\pgfpathclose%
\pgfusepath{stroke,fill}%
\end{pgfscope}%
\begin{pgfscope}%
\pgfpathrectangle{\pgfqpoint{0.564660in}{0.521603in}}{\pgfqpoint{3.720000in}{3.020000in}}%
\pgfusepath{clip}%
\pgfsetbuttcap%
\pgfsetroundjoin%
\definecolor{currentfill}{rgb}{1.000000,1.000000,1.000000}%
\pgfsetfillcolor{currentfill}%
\pgfsetlinewidth{1.003750pt}%
\definecolor{currentstroke}{rgb}{0.000000,0.000000,0.000000}%
\pgfsetstrokecolor{currentstroke}%
\pgfsetdash{}{0pt}%
\pgfpathmoveto{\pgfqpoint{1.255941in}{2.308002in}}%
\pgfpathcurveto{\pgfqpoint{1.266991in}{2.308002in}}{\pgfqpoint{1.277590in}{2.312392in}}{\pgfqpoint{1.285404in}{2.320206in}}%
\pgfpathcurveto{\pgfqpoint{1.293218in}{2.328020in}}{\pgfqpoint{1.297608in}{2.338619in}}{\pgfqpoint{1.297608in}{2.349669in}}%
\pgfpathcurveto{\pgfqpoint{1.297608in}{2.360719in}}{\pgfqpoint{1.293218in}{2.371318in}}{\pgfqpoint{1.285404in}{2.379132in}}%
\pgfpathcurveto{\pgfqpoint{1.277590in}{2.386945in}}{\pgfqpoint{1.266991in}{2.391336in}}{\pgfqpoint{1.255941in}{2.391336in}}%
\pgfpathcurveto{\pgfqpoint{1.244891in}{2.391336in}}{\pgfqpoint{1.234292in}{2.386945in}}{\pgfqpoint{1.226478in}{2.379132in}}%
\pgfpathcurveto{\pgfqpoint{1.218665in}{2.371318in}}{\pgfqpoint{1.214275in}{2.360719in}}{\pgfqpoint{1.214275in}{2.349669in}}%
\pgfpathcurveto{\pgfqpoint{1.214275in}{2.338619in}}{\pgfqpoint{1.218665in}{2.328020in}}{\pgfqpoint{1.226478in}{2.320206in}}%
\pgfpathcurveto{\pgfqpoint{1.234292in}{2.312392in}}{\pgfqpoint{1.244891in}{2.308002in}}{\pgfqpoint{1.255941in}{2.308002in}}%
\pgfpathclose%
\pgfusepath{stroke,fill}%
\end{pgfscope}%
\begin{pgfscope}%
\pgfpathrectangle{\pgfqpoint{0.564660in}{0.521603in}}{\pgfqpoint{3.720000in}{3.020000in}}%
\pgfusepath{clip}%
\pgfsetbuttcap%
\pgfsetroundjoin%
\definecolor{currentfill}{rgb}{1.000000,1.000000,1.000000}%
\pgfsetfillcolor{currentfill}%
\pgfsetlinewidth{1.003750pt}%
\definecolor{currentstroke}{rgb}{0.000000,0.000000,0.000000}%
\pgfsetstrokecolor{currentstroke}%
\pgfsetdash{}{0pt}%
\pgfpathmoveto{\pgfqpoint{2.259339in}{0.939575in}}%
\pgfpathcurveto{\pgfqpoint{2.270389in}{0.939575in}}{\pgfqpoint{2.280988in}{0.943966in}}{\pgfqpoint{2.288801in}{0.951779in}}%
\pgfpathcurveto{\pgfqpoint{2.296615in}{0.959593in}}{\pgfqpoint{2.301005in}{0.970192in}}{\pgfqpoint{2.301005in}{0.981242in}}%
\pgfpathcurveto{\pgfqpoint{2.301005in}{0.992292in}}{\pgfqpoint{2.296615in}{1.002891in}}{\pgfqpoint{2.288801in}{1.010705in}}%
\pgfpathcurveto{\pgfqpoint{2.280988in}{1.018518in}}{\pgfqpoint{2.270389in}{1.022909in}}{\pgfqpoint{2.259339in}{1.022909in}}%
\pgfpathcurveto{\pgfqpoint{2.248289in}{1.022909in}}{\pgfqpoint{2.237690in}{1.018518in}}{\pgfqpoint{2.229876in}{1.010705in}}%
\pgfpathcurveto{\pgfqpoint{2.222062in}{1.002891in}}{\pgfqpoint{2.217672in}{0.992292in}}{\pgfqpoint{2.217672in}{0.981242in}}%
\pgfpathcurveto{\pgfqpoint{2.217672in}{0.970192in}}{\pgfqpoint{2.222062in}{0.959593in}}{\pgfqpoint{2.229876in}{0.951779in}}%
\pgfpathcurveto{\pgfqpoint{2.237690in}{0.943966in}}{\pgfqpoint{2.248289in}{0.939575in}}{\pgfqpoint{2.259339in}{0.939575in}}%
\pgfpathclose%
\pgfusepath{stroke,fill}%
\end{pgfscope}%
\begin{pgfscope}%
\pgfpathrectangle{\pgfqpoint{0.564660in}{0.521603in}}{\pgfqpoint{3.720000in}{3.020000in}}%
\pgfusepath{clip}%
\pgfsetbuttcap%
\pgfsetroundjoin%
\definecolor{currentfill}{rgb}{1.000000,1.000000,1.000000}%
\pgfsetfillcolor{currentfill}%
\pgfsetlinewidth{1.003750pt}%
\definecolor{currentstroke}{rgb}{0.000000,0.000000,0.000000}%
\pgfsetstrokecolor{currentstroke}%
\pgfsetdash{}{0pt}%
\pgfpathmoveto{\pgfqpoint{1.940911in}{1.013750in}}%
\pgfpathcurveto{\pgfqpoint{1.951961in}{1.013750in}}{\pgfqpoint{1.962560in}{1.018140in}}{\pgfqpoint{1.970374in}{1.025954in}}%
\pgfpathcurveto{\pgfqpoint{1.978187in}{1.033768in}}{\pgfqpoint{1.982578in}{1.044367in}}{\pgfqpoint{1.982578in}{1.055417in}}%
\pgfpathcurveto{\pgfqpoint{1.982578in}{1.066467in}}{\pgfqpoint{1.978187in}{1.077066in}}{\pgfqpoint{1.970374in}{1.084880in}}%
\pgfpathcurveto{\pgfqpoint{1.962560in}{1.092693in}}{\pgfqpoint{1.951961in}{1.097083in}}{\pgfqpoint{1.940911in}{1.097083in}}%
\pgfpathcurveto{\pgfqpoint{1.929861in}{1.097083in}}{\pgfqpoint{1.919262in}{1.092693in}}{\pgfqpoint{1.911448in}{1.084880in}}%
\pgfpathcurveto{\pgfqpoint{1.903635in}{1.077066in}}{\pgfqpoint{1.899244in}{1.066467in}}{\pgfqpoint{1.899244in}{1.055417in}}%
\pgfpathcurveto{\pgfqpoint{1.899244in}{1.044367in}}{\pgfqpoint{1.903635in}{1.033768in}}{\pgfqpoint{1.911448in}{1.025954in}}%
\pgfpathcurveto{\pgfqpoint{1.919262in}{1.018140in}}{\pgfqpoint{1.929861in}{1.013750in}}{\pgfqpoint{1.940911in}{1.013750in}}%
\pgfpathclose%
\pgfusepath{stroke,fill}%
\end{pgfscope}%
\begin{pgfscope}%
\pgfpathrectangle{\pgfqpoint{0.564660in}{0.521603in}}{\pgfqpoint{3.720000in}{3.020000in}}%
\pgfusepath{clip}%
\pgfsetbuttcap%
\pgfsetroundjoin%
\definecolor{currentfill}{rgb}{1.000000,1.000000,1.000000}%
\pgfsetfillcolor{currentfill}%
\pgfsetlinewidth{1.003750pt}%
\definecolor{currentstroke}{rgb}{0.000000,0.000000,0.000000}%
\pgfsetstrokecolor{currentstroke}%
\pgfsetdash{}{0pt}%
\pgfpathmoveto{\pgfqpoint{1.712153in}{3.325802in}}%
\pgfpathcurveto{\pgfqpoint{1.723203in}{3.325802in}}{\pgfqpoint{1.733802in}{3.330193in}}{\pgfqpoint{1.741616in}{3.338006in}}%
\pgfpathcurveto{\pgfqpoint{1.749429in}{3.345820in}}{\pgfqpoint{1.753820in}{3.356419in}}{\pgfqpoint{1.753820in}{3.367469in}}%
\pgfpathcurveto{\pgfqpoint{1.753820in}{3.378519in}}{\pgfqpoint{1.749429in}{3.389118in}}{\pgfqpoint{1.741616in}{3.396932in}}%
\pgfpathcurveto{\pgfqpoint{1.733802in}{3.404745in}}{\pgfqpoint{1.723203in}{3.409136in}}{\pgfqpoint{1.712153in}{3.409136in}}%
\pgfpathcurveto{\pgfqpoint{1.701103in}{3.409136in}}{\pgfqpoint{1.690504in}{3.404745in}}{\pgfqpoint{1.682690in}{3.396932in}}%
\pgfpathcurveto{\pgfqpoint{1.674876in}{3.389118in}}{\pgfqpoint{1.670486in}{3.378519in}}{\pgfqpoint{1.670486in}{3.367469in}}%
\pgfpathcurveto{\pgfqpoint{1.670486in}{3.356419in}}{\pgfqpoint{1.674876in}{3.345820in}}{\pgfqpoint{1.682690in}{3.338006in}}%
\pgfpathcurveto{\pgfqpoint{1.690504in}{3.330193in}}{\pgfqpoint{1.701103in}{3.325802in}}{\pgfqpoint{1.712153in}{3.325802in}}%
\pgfpathclose%
\pgfusepath{stroke,fill}%
\end{pgfscope}%
\begin{pgfscope}%
\pgfpathrectangle{\pgfqpoint{0.564660in}{0.521603in}}{\pgfqpoint{3.720000in}{3.020000in}}%
\pgfusepath{clip}%
\pgfsetbuttcap%
\pgfsetroundjoin%
\definecolor{currentfill}{rgb}{1.000000,1.000000,1.000000}%
\pgfsetfillcolor{currentfill}%
\pgfsetlinewidth{1.003750pt}%
\definecolor{currentstroke}{rgb}{0.000000,0.000000,0.000000}%
\pgfsetstrokecolor{currentstroke}%
\pgfsetdash{}{0pt}%
\pgfpathmoveto{\pgfqpoint{1.087553in}{1.563620in}}%
\pgfpathcurveto{\pgfqpoint{1.098603in}{1.563620in}}{\pgfqpoint{1.109202in}{1.568010in}}{\pgfqpoint{1.117016in}{1.575823in}}%
\pgfpathcurveto{\pgfqpoint{1.124830in}{1.583637in}}{\pgfqpoint{1.129220in}{1.594236in}}{\pgfqpoint{1.129220in}{1.605286in}}%
\pgfpathcurveto{\pgfqpoint{1.129220in}{1.616336in}}{\pgfqpoint{1.124830in}{1.626935in}}{\pgfqpoint{1.117016in}{1.634749in}}%
\pgfpathcurveto{\pgfqpoint{1.109202in}{1.642563in}}{\pgfqpoint{1.098603in}{1.646953in}}{\pgfqpoint{1.087553in}{1.646953in}}%
\pgfpathcurveto{\pgfqpoint{1.076503in}{1.646953in}}{\pgfqpoint{1.065904in}{1.642563in}}{\pgfqpoint{1.058090in}{1.634749in}}%
\pgfpathcurveto{\pgfqpoint{1.050277in}{1.626935in}}{\pgfqpoint{1.045886in}{1.616336in}}{\pgfqpoint{1.045886in}{1.605286in}}%
\pgfpathcurveto{\pgfqpoint{1.045886in}{1.594236in}}{\pgfqpoint{1.050277in}{1.583637in}}{\pgfqpoint{1.058090in}{1.575823in}}%
\pgfpathcurveto{\pgfqpoint{1.065904in}{1.568010in}}{\pgfqpoint{1.076503in}{1.563620in}}{\pgfqpoint{1.087553in}{1.563620in}}%
\pgfpathclose%
\pgfusepath{stroke,fill}%
\end{pgfscope}%
\begin{pgfscope}%
\pgfpathrectangle{\pgfqpoint{0.564660in}{0.521603in}}{\pgfqpoint{3.720000in}{3.020000in}}%
\pgfusepath{clip}%
\pgfsetbuttcap%
\pgfsetroundjoin%
\definecolor{currentfill}{rgb}{1.000000,1.000000,1.000000}%
\pgfsetfillcolor{currentfill}%
\pgfsetlinewidth{1.003750pt}%
\definecolor{currentstroke}{rgb}{0.000000,0.000000,0.000000}%
\pgfsetstrokecolor{currentstroke}%
\pgfsetdash{}{0pt}%
\pgfpathmoveto{\pgfqpoint{1.626658in}{0.731310in}}%
\pgfpathcurveto{\pgfqpoint{1.637708in}{0.731310in}}{\pgfqpoint{1.648307in}{0.735701in}}{\pgfqpoint{1.656121in}{0.743514in}}%
\pgfpathcurveto{\pgfqpoint{1.663934in}{0.751328in}}{\pgfqpoint{1.668325in}{0.761927in}}{\pgfqpoint{1.668325in}{0.772977in}}%
\pgfpathcurveto{\pgfqpoint{1.668325in}{0.784027in}}{\pgfqpoint{1.663934in}{0.794626in}}{\pgfqpoint{1.656121in}{0.802440in}}%
\pgfpathcurveto{\pgfqpoint{1.648307in}{0.810253in}}{\pgfqpoint{1.637708in}{0.814644in}}{\pgfqpoint{1.626658in}{0.814644in}}%
\pgfpathcurveto{\pgfqpoint{1.615608in}{0.814644in}}{\pgfqpoint{1.605009in}{0.810253in}}{\pgfqpoint{1.597195in}{0.802440in}}%
\pgfpathcurveto{\pgfqpoint{1.589382in}{0.794626in}}{\pgfqpoint{1.584991in}{0.784027in}}{\pgfqpoint{1.584991in}{0.772977in}}%
\pgfpathcurveto{\pgfqpoint{1.584991in}{0.761927in}}{\pgfqpoint{1.589382in}{0.751328in}}{\pgfqpoint{1.597195in}{0.743514in}}%
\pgfpathcurveto{\pgfqpoint{1.605009in}{0.735701in}}{\pgfqpoint{1.615608in}{0.731310in}}{\pgfqpoint{1.626658in}{0.731310in}}%
\pgfpathclose%
\pgfusepath{stroke,fill}%
\end{pgfscope}%
\begin{pgfscope}%
\pgfsetbuttcap%
\pgfsetroundjoin%
\definecolor{currentfill}{rgb}{0.000000,0.000000,0.000000}%
\pgfsetfillcolor{currentfill}%
\pgfsetlinewidth{0.803000pt}%
\definecolor{currentstroke}{rgb}{0.000000,0.000000,0.000000}%
\pgfsetstrokecolor{currentstroke}%
\pgfsetdash{}{0pt}%
\pgfsys@defobject{currentmarker}{\pgfqpoint{0.000000in}{-0.048611in}}{\pgfqpoint{0.000000in}{0.000000in}}{%
\pgfpathmoveto{\pgfqpoint{0.000000in}{0.000000in}}%
\pgfpathlineto{\pgfqpoint{0.000000in}{-0.048611in}}%
\pgfusepath{stroke,fill}%
}%
\begin{pgfscope}%
\pgfsys@transformshift{1.061822in}{0.521603in}%
\pgfsys@useobject{currentmarker}{}%
\end{pgfscope}%
\end{pgfscope}%
\begin{pgfscope}%
\definecolor{textcolor}{rgb}{0.000000,0.000000,0.000000}%
\pgfsetstrokecolor{textcolor}%
\pgfsetfillcolor{textcolor}%
\pgftext[x=1.061822in,y=0.424381in,,top]{\color{textcolor}\rmfamily\fontsize{10.000000}{12.000000}\selectfont \(\displaystyle 0.6\)}%
\end{pgfscope}%
\begin{pgfscope}%
\pgfsetbuttcap%
\pgfsetroundjoin%
\definecolor{currentfill}{rgb}{0.000000,0.000000,0.000000}%
\pgfsetfillcolor{currentfill}%
\pgfsetlinewidth{0.803000pt}%
\definecolor{currentstroke}{rgb}{0.000000,0.000000,0.000000}%
\pgfsetstrokecolor{currentstroke}%
\pgfsetdash{}{0pt}%
\pgfsys@defobject{currentmarker}{\pgfqpoint{0.000000in}{-0.048611in}}{\pgfqpoint{0.000000in}{0.000000in}}{%
\pgfpathmoveto{\pgfqpoint{0.000000in}{0.000000in}}%
\pgfpathlineto{\pgfqpoint{0.000000in}{-0.048611in}}%
\pgfusepath{stroke,fill}%
}%
\begin{pgfscope}%
\pgfsys@transformshift{1.572925in}{0.521603in}%
\pgfsys@useobject{currentmarker}{}%
\end{pgfscope}%
\end{pgfscope}%
\begin{pgfscope}%
\definecolor{textcolor}{rgb}{0.000000,0.000000,0.000000}%
\pgfsetstrokecolor{textcolor}%
\pgfsetfillcolor{textcolor}%
\pgftext[x=1.572925in,y=0.424381in,,top]{\color{textcolor}\rmfamily\fontsize{10.000000}{12.000000}\selectfont \(\displaystyle 0.8\)}%
\end{pgfscope}%
\begin{pgfscope}%
\pgfsetbuttcap%
\pgfsetroundjoin%
\definecolor{currentfill}{rgb}{0.000000,0.000000,0.000000}%
\pgfsetfillcolor{currentfill}%
\pgfsetlinewidth{0.803000pt}%
\definecolor{currentstroke}{rgb}{0.000000,0.000000,0.000000}%
\pgfsetstrokecolor{currentstroke}%
\pgfsetdash{}{0pt}%
\pgfsys@defobject{currentmarker}{\pgfqpoint{0.000000in}{-0.048611in}}{\pgfqpoint{0.000000in}{0.000000in}}{%
\pgfpathmoveto{\pgfqpoint{0.000000in}{0.000000in}}%
\pgfpathlineto{\pgfqpoint{0.000000in}{-0.048611in}}%
\pgfusepath{stroke,fill}%
}%
\begin{pgfscope}%
\pgfsys@transformshift{2.084027in}{0.521603in}%
\pgfsys@useobject{currentmarker}{}%
\end{pgfscope}%
\end{pgfscope}%
\begin{pgfscope}%
\definecolor{textcolor}{rgb}{0.000000,0.000000,0.000000}%
\pgfsetstrokecolor{textcolor}%
\pgfsetfillcolor{textcolor}%
\pgftext[x=2.084027in,y=0.424381in,,top]{\color{textcolor}\rmfamily\fontsize{10.000000}{12.000000}\selectfont \(\displaystyle 1.0\)}%
\end{pgfscope}%
\begin{pgfscope}%
\pgfsetbuttcap%
\pgfsetroundjoin%
\definecolor{currentfill}{rgb}{0.000000,0.000000,0.000000}%
\pgfsetfillcolor{currentfill}%
\pgfsetlinewidth{0.803000pt}%
\definecolor{currentstroke}{rgb}{0.000000,0.000000,0.000000}%
\pgfsetstrokecolor{currentstroke}%
\pgfsetdash{}{0pt}%
\pgfsys@defobject{currentmarker}{\pgfqpoint{0.000000in}{-0.048611in}}{\pgfqpoint{0.000000in}{0.000000in}}{%
\pgfpathmoveto{\pgfqpoint{0.000000in}{0.000000in}}%
\pgfpathlineto{\pgfqpoint{0.000000in}{-0.048611in}}%
\pgfusepath{stroke,fill}%
}%
\begin{pgfscope}%
\pgfsys@transformshift{2.595130in}{0.521603in}%
\pgfsys@useobject{currentmarker}{}%
\end{pgfscope}%
\end{pgfscope}%
\begin{pgfscope}%
\definecolor{textcolor}{rgb}{0.000000,0.000000,0.000000}%
\pgfsetstrokecolor{textcolor}%
\pgfsetfillcolor{textcolor}%
\pgftext[x=2.595130in,y=0.424381in,,top]{\color{textcolor}\rmfamily\fontsize{10.000000}{12.000000}\selectfont \(\displaystyle 1.2\)}%
\end{pgfscope}%
\begin{pgfscope}%
\pgfsetbuttcap%
\pgfsetroundjoin%
\definecolor{currentfill}{rgb}{0.000000,0.000000,0.000000}%
\pgfsetfillcolor{currentfill}%
\pgfsetlinewidth{0.803000pt}%
\definecolor{currentstroke}{rgb}{0.000000,0.000000,0.000000}%
\pgfsetstrokecolor{currentstroke}%
\pgfsetdash{}{0pt}%
\pgfsys@defobject{currentmarker}{\pgfqpoint{0.000000in}{-0.048611in}}{\pgfqpoint{0.000000in}{0.000000in}}{%
\pgfpathmoveto{\pgfqpoint{0.000000in}{0.000000in}}%
\pgfpathlineto{\pgfqpoint{0.000000in}{-0.048611in}}%
\pgfusepath{stroke,fill}%
}%
\begin{pgfscope}%
\pgfsys@transformshift{3.106233in}{0.521603in}%
\pgfsys@useobject{currentmarker}{}%
\end{pgfscope}%
\end{pgfscope}%
\begin{pgfscope}%
\definecolor{textcolor}{rgb}{0.000000,0.000000,0.000000}%
\pgfsetstrokecolor{textcolor}%
\pgfsetfillcolor{textcolor}%
\pgftext[x=3.106233in,y=0.424381in,,top]{\color{textcolor}\rmfamily\fontsize{10.000000}{12.000000}\selectfont \(\displaystyle 1.4\)}%
\end{pgfscope}%
\begin{pgfscope}%
\pgfsetbuttcap%
\pgfsetroundjoin%
\definecolor{currentfill}{rgb}{0.000000,0.000000,0.000000}%
\pgfsetfillcolor{currentfill}%
\pgfsetlinewidth{0.803000pt}%
\definecolor{currentstroke}{rgb}{0.000000,0.000000,0.000000}%
\pgfsetstrokecolor{currentstroke}%
\pgfsetdash{}{0pt}%
\pgfsys@defobject{currentmarker}{\pgfqpoint{0.000000in}{-0.048611in}}{\pgfqpoint{0.000000in}{0.000000in}}{%
\pgfpathmoveto{\pgfqpoint{0.000000in}{0.000000in}}%
\pgfpathlineto{\pgfqpoint{0.000000in}{-0.048611in}}%
\pgfusepath{stroke,fill}%
}%
\begin{pgfscope}%
\pgfsys@transformshift{3.617336in}{0.521603in}%
\pgfsys@useobject{currentmarker}{}%
\end{pgfscope}%
\end{pgfscope}%
\begin{pgfscope}%
\definecolor{textcolor}{rgb}{0.000000,0.000000,0.000000}%
\pgfsetstrokecolor{textcolor}%
\pgfsetfillcolor{textcolor}%
\pgftext[x=3.617336in,y=0.424381in,,top]{\color{textcolor}\rmfamily\fontsize{10.000000}{12.000000}\selectfont \(\displaystyle 1.6\)}%
\end{pgfscope}%
\begin{pgfscope}%
\pgfsetbuttcap%
\pgfsetroundjoin%
\definecolor{currentfill}{rgb}{0.000000,0.000000,0.000000}%
\pgfsetfillcolor{currentfill}%
\pgfsetlinewidth{0.803000pt}%
\definecolor{currentstroke}{rgb}{0.000000,0.000000,0.000000}%
\pgfsetstrokecolor{currentstroke}%
\pgfsetdash{}{0pt}%
\pgfsys@defobject{currentmarker}{\pgfqpoint{0.000000in}{-0.048611in}}{\pgfqpoint{0.000000in}{0.000000in}}{%
\pgfpathmoveto{\pgfqpoint{0.000000in}{0.000000in}}%
\pgfpathlineto{\pgfqpoint{0.000000in}{-0.048611in}}%
\pgfusepath{stroke,fill}%
}%
\begin{pgfscope}%
\pgfsys@transformshift{4.128439in}{0.521603in}%
\pgfsys@useobject{currentmarker}{}%
\end{pgfscope}%
\end{pgfscope}%
\begin{pgfscope}%
\definecolor{textcolor}{rgb}{0.000000,0.000000,0.000000}%
\pgfsetstrokecolor{textcolor}%
\pgfsetfillcolor{textcolor}%
\pgftext[x=4.128439in,y=0.424381in,,top]{\color{textcolor}\rmfamily\fontsize{10.000000}{12.000000}\selectfont \(\displaystyle 1.8\)}%
\end{pgfscope}%
\begin{pgfscope}%
\definecolor{textcolor}{rgb}{0.000000,0.000000,0.000000}%
\pgfsetstrokecolor{textcolor}%
\pgfsetfillcolor{textcolor}%
\pgftext[x=2.424660in,y=0.234413in,,top]{\color{textcolor}\rmfamily\fontsize{10.000000}{12.000000}\selectfont \(\displaystyle \varphi_s\) (kV)}%
\end{pgfscope}%
\begin{pgfscope}%
\pgfsetbuttcap%
\pgfsetroundjoin%
\definecolor{currentfill}{rgb}{0.000000,0.000000,0.000000}%
\pgfsetfillcolor{currentfill}%
\pgfsetlinewidth{0.803000pt}%
\definecolor{currentstroke}{rgb}{0.000000,0.000000,0.000000}%
\pgfsetstrokecolor{currentstroke}%
\pgfsetdash{}{0pt}%
\pgfsys@defobject{currentmarker}{\pgfqpoint{-0.048611in}{0.000000in}}{\pgfqpoint{0.000000in}{0.000000in}}{%
\pgfpathmoveto{\pgfqpoint{0.000000in}{0.000000in}}%
\pgfpathlineto{\pgfqpoint{-0.048611in}{0.000000in}}%
\pgfusepath{stroke,fill}%
}%
\begin{pgfscope}%
\pgfsys@transformshift{0.564660in}{0.523802in}%
\pgfsys@useobject{currentmarker}{}%
\end{pgfscope}%
\end{pgfscope}%
\begin{pgfscope}%
\definecolor{textcolor}{rgb}{0.000000,0.000000,0.000000}%
\pgfsetstrokecolor{textcolor}%
\pgfsetfillcolor{textcolor}%
\pgftext[x=0.289968in,y=0.471040in,left,base]{\color{textcolor}\rmfamily\fontsize{10.000000}{12.000000}\selectfont \(\displaystyle 0.0\)}%
\end{pgfscope}%
\begin{pgfscope}%
\pgfsetbuttcap%
\pgfsetroundjoin%
\definecolor{currentfill}{rgb}{0.000000,0.000000,0.000000}%
\pgfsetfillcolor{currentfill}%
\pgfsetlinewidth{0.803000pt}%
\definecolor{currentstroke}{rgb}{0.000000,0.000000,0.000000}%
\pgfsetstrokecolor{currentstroke}%
\pgfsetdash{}{0pt}%
\pgfsys@defobject{currentmarker}{\pgfqpoint{-0.048611in}{0.000000in}}{\pgfqpoint{0.000000in}{0.000000in}}{%
\pgfpathmoveto{\pgfqpoint{0.000000in}{0.000000in}}%
\pgfpathlineto{\pgfqpoint{-0.048611in}{0.000000in}}%
\pgfusepath{stroke,fill}%
}%
\begin{pgfscope}%
\pgfsys@transformshift{0.564660in}{1.096070in}%
\pgfsys@useobject{currentmarker}{}%
\end{pgfscope}%
\end{pgfscope}%
\begin{pgfscope}%
\definecolor{textcolor}{rgb}{0.000000,0.000000,0.000000}%
\pgfsetstrokecolor{textcolor}%
\pgfsetfillcolor{textcolor}%
\pgftext[x=0.289968in,y=1.043308in,left,base]{\color{textcolor}\rmfamily\fontsize{10.000000}{12.000000}\selectfont \(\displaystyle 0.1\)}%
\end{pgfscope}%
\begin{pgfscope}%
\pgfsetbuttcap%
\pgfsetroundjoin%
\definecolor{currentfill}{rgb}{0.000000,0.000000,0.000000}%
\pgfsetfillcolor{currentfill}%
\pgfsetlinewidth{0.803000pt}%
\definecolor{currentstroke}{rgb}{0.000000,0.000000,0.000000}%
\pgfsetstrokecolor{currentstroke}%
\pgfsetdash{}{0pt}%
\pgfsys@defobject{currentmarker}{\pgfqpoint{-0.048611in}{0.000000in}}{\pgfqpoint{0.000000in}{0.000000in}}{%
\pgfpathmoveto{\pgfqpoint{0.000000in}{0.000000in}}%
\pgfpathlineto{\pgfqpoint{-0.048611in}{0.000000in}}%
\pgfusepath{stroke,fill}%
}%
\begin{pgfscope}%
\pgfsys@transformshift{0.564660in}{1.668338in}%
\pgfsys@useobject{currentmarker}{}%
\end{pgfscope}%
\end{pgfscope}%
\begin{pgfscope}%
\definecolor{textcolor}{rgb}{0.000000,0.000000,0.000000}%
\pgfsetstrokecolor{textcolor}%
\pgfsetfillcolor{textcolor}%
\pgftext[x=0.289968in,y=1.615576in,left,base]{\color{textcolor}\rmfamily\fontsize{10.000000}{12.000000}\selectfont \(\displaystyle 0.2\)}%
\end{pgfscope}%
\begin{pgfscope}%
\pgfsetbuttcap%
\pgfsetroundjoin%
\definecolor{currentfill}{rgb}{0.000000,0.000000,0.000000}%
\pgfsetfillcolor{currentfill}%
\pgfsetlinewidth{0.803000pt}%
\definecolor{currentstroke}{rgb}{0.000000,0.000000,0.000000}%
\pgfsetstrokecolor{currentstroke}%
\pgfsetdash{}{0pt}%
\pgfsys@defobject{currentmarker}{\pgfqpoint{-0.048611in}{0.000000in}}{\pgfqpoint{0.000000in}{0.000000in}}{%
\pgfpathmoveto{\pgfqpoint{0.000000in}{0.000000in}}%
\pgfpathlineto{\pgfqpoint{-0.048611in}{0.000000in}}%
\pgfusepath{stroke,fill}%
}%
\begin{pgfscope}%
\pgfsys@transformshift{0.564660in}{2.240606in}%
\pgfsys@useobject{currentmarker}{}%
\end{pgfscope}%
\end{pgfscope}%
\begin{pgfscope}%
\definecolor{textcolor}{rgb}{0.000000,0.000000,0.000000}%
\pgfsetstrokecolor{textcolor}%
\pgfsetfillcolor{textcolor}%
\pgftext[x=0.289968in,y=2.187844in,left,base]{\color{textcolor}\rmfamily\fontsize{10.000000}{12.000000}\selectfont \(\displaystyle 0.3\)}%
\end{pgfscope}%
\begin{pgfscope}%
\pgfsetbuttcap%
\pgfsetroundjoin%
\definecolor{currentfill}{rgb}{0.000000,0.000000,0.000000}%
\pgfsetfillcolor{currentfill}%
\pgfsetlinewidth{0.803000pt}%
\definecolor{currentstroke}{rgb}{0.000000,0.000000,0.000000}%
\pgfsetstrokecolor{currentstroke}%
\pgfsetdash{}{0pt}%
\pgfsys@defobject{currentmarker}{\pgfqpoint{-0.048611in}{0.000000in}}{\pgfqpoint{0.000000in}{0.000000in}}{%
\pgfpathmoveto{\pgfqpoint{0.000000in}{0.000000in}}%
\pgfpathlineto{\pgfqpoint{-0.048611in}{0.000000in}}%
\pgfusepath{stroke,fill}%
}%
\begin{pgfscope}%
\pgfsys@transformshift{0.564660in}{2.812874in}%
\pgfsys@useobject{currentmarker}{}%
\end{pgfscope}%
\end{pgfscope}%
\begin{pgfscope}%
\definecolor{textcolor}{rgb}{0.000000,0.000000,0.000000}%
\pgfsetstrokecolor{textcolor}%
\pgfsetfillcolor{textcolor}%
\pgftext[x=0.289968in,y=2.760112in,left,base]{\color{textcolor}\rmfamily\fontsize{10.000000}{12.000000}\selectfont \(\displaystyle 0.4\)}%
\end{pgfscope}%
\begin{pgfscope}%
\pgfsetbuttcap%
\pgfsetroundjoin%
\definecolor{currentfill}{rgb}{0.000000,0.000000,0.000000}%
\pgfsetfillcolor{currentfill}%
\pgfsetlinewidth{0.803000pt}%
\definecolor{currentstroke}{rgb}{0.000000,0.000000,0.000000}%
\pgfsetstrokecolor{currentstroke}%
\pgfsetdash{}{0pt}%
\pgfsys@defobject{currentmarker}{\pgfqpoint{-0.048611in}{0.000000in}}{\pgfqpoint{0.000000in}{0.000000in}}{%
\pgfpathmoveto{\pgfqpoint{0.000000in}{0.000000in}}%
\pgfpathlineto{\pgfqpoint{-0.048611in}{0.000000in}}%
\pgfusepath{stroke,fill}%
}%
\begin{pgfscope}%
\pgfsys@transformshift{0.564660in}{3.385142in}%
\pgfsys@useobject{currentmarker}{}%
\end{pgfscope}%
\end{pgfscope}%
\begin{pgfscope}%
\definecolor{textcolor}{rgb}{0.000000,0.000000,0.000000}%
\pgfsetstrokecolor{textcolor}%
\pgfsetfillcolor{textcolor}%
\pgftext[x=0.289968in,y=3.332380in,left,base]{\color{textcolor}\rmfamily\fontsize{10.000000}{12.000000}\selectfont \(\displaystyle 0.5\)}%
\end{pgfscope}%
\begin{pgfscope}%
\definecolor{textcolor}{rgb}{0.000000,0.000000,0.000000}%
\pgfsetstrokecolor{textcolor}%
\pgfsetfillcolor{textcolor}%
\pgftext[x=0.234413in,y=2.031603in,,bottom,rotate=90.000000]{\color{textcolor}\rmfamily\fontsize{10.000000}{12.000000}\selectfont \(\displaystyle V_d\) (mL)}%
\end{pgfscope}%
\begin{pgfscope}%
\pgfsetrectcap%
\pgfsetmiterjoin%
\pgfsetlinewidth{0.803000pt}%
\definecolor{currentstroke}{rgb}{0.501961,0.501961,0.501961}%
\pgfsetstrokecolor{currentstroke}%
\pgfsetdash{}{0pt}%
\pgfpathmoveto{\pgfqpoint{0.564660in}{0.521603in}}%
\pgfpathlineto{\pgfqpoint{0.564660in}{3.541603in}}%
\pgfusepath{stroke}%
\end{pgfscope}%
\begin{pgfscope}%
\pgfsetrectcap%
\pgfsetmiterjoin%
\pgfsetlinewidth{0.803000pt}%
\definecolor{currentstroke}{rgb}{0.501961,0.501961,0.501961}%
\pgfsetstrokecolor{currentstroke}%
\pgfsetdash{}{0pt}%
\pgfpathmoveto{\pgfqpoint{4.284660in}{0.521603in}}%
\pgfpathlineto{\pgfqpoint{4.284660in}{3.541603in}}%
\pgfusepath{stroke}%
\end{pgfscope}%
\begin{pgfscope}%
\pgfsetrectcap%
\pgfsetmiterjoin%
\pgfsetlinewidth{0.803000pt}%
\definecolor{currentstroke}{rgb}{0.501961,0.501961,0.501961}%
\pgfsetstrokecolor{currentstroke}%
\pgfsetdash{}{0pt}%
\pgfpathmoveto{\pgfqpoint{0.564660in}{0.521603in}}%
\pgfpathlineto{\pgfqpoint{4.284660in}{0.521603in}}%
\pgfusepath{stroke}%
\end{pgfscope}%
\begin{pgfscope}%
\pgfsetrectcap%
\pgfsetmiterjoin%
\pgfsetlinewidth{0.803000pt}%
\definecolor{currentstroke}{rgb}{0.501961,0.501961,0.501961}%
\pgfsetstrokecolor{currentstroke}%
\pgfsetdash{}{0pt}%
\pgfpathmoveto{\pgfqpoint{0.564660in}{3.541603in}}%
\pgfpathlineto{\pgfqpoint{4.284660in}{3.541603in}}%
\pgfusepath{stroke}%
\end{pgfscope}%
\begin{pgfscope}%
\pgfpathrectangle{\pgfqpoint{4.517160in}{0.521603in}}{\pgfqpoint{0.151000in}{3.020000in}}%
\pgfusepath{clip}%
\pgfsetbuttcap%
\pgfsetmiterjoin%
\definecolor{currentfill}{rgb}{1.000000,1.000000,1.000000}%
\pgfsetfillcolor{currentfill}%
\pgfsetlinewidth{0.010037pt}%
\definecolor{currentstroke}{rgb}{1.000000,1.000000,1.000000}%
\pgfsetstrokecolor{currentstroke}%
\pgfsetdash{}{0pt}%
\pgfpathmoveto{\pgfqpoint{4.517160in}{0.521603in}}%
\pgfpathlineto{\pgfqpoint{4.517160in}{0.953032in}}%
\pgfpathlineto{\pgfqpoint{4.517160in}{3.110175in}}%
\pgfpathlineto{\pgfqpoint{4.517160in}{3.541603in}}%
\pgfpathlineto{\pgfqpoint{4.668160in}{3.541603in}}%
\pgfpathlineto{\pgfqpoint{4.668160in}{3.110175in}}%
\pgfpathlineto{\pgfqpoint{4.668160in}{0.953032in}}%
\pgfpathlineto{\pgfqpoint{4.668160in}{0.521603in}}%
\pgfpathclose%
\pgfusepath{stroke,fill}%
\end{pgfscope}%
\begin{pgfscope}%
\pgfpathrectangle{\pgfqpoint{4.517160in}{0.521603in}}{\pgfqpoint{0.151000in}{3.020000in}}%
\pgfusepath{clip}%
\pgfsetbuttcap%
\pgfsetroundjoin%
\definecolor{currentfill}{rgb}{0.061765,0.061765,0.085934}%
\pgfsetfillcolor{currentfill}%
\pgfsetlinewidth{0.000000pt}%
\definecolor{currentstroke}{rgb}{0.000000,0.000000,0.000000}%
\pgfsetstrokecolor{currentstroke}%
\pgfsetdash{}{0pt}%
\pgfpathmoveto{\pgfqpoint{4.517160in}{0.521603in}}%
\pgfpathlineto{\pgfqpoint{4.668160in}{0.521603in}}%
\pgfpathlineto{\pgfqpoint{4.668160in}{0.953032in}}%
\pgfpathlineto{\pgfqpoint{4.517160in}{0.953032in}}%
\pgfpathlineto{\pgfqpoint{4.517160in}{0.521603in}}%
\pgfusepath{fill}%
\end{pgfscope}%
\begin{pgfscope}%
\pgfpathrectangle{\pgfqpoint{4.517160in}{0.521603in}}{\pgfqpoint{0.151000in}{3.020000in}}%
\pgfusepath{clip}%
\pgfsetbuttcap%
\pgfsetroundjoin%
\definecolor{currentfill}{rgb}{0.185294,0.185294,0.257801}%
\pgfsetfillcolor{currentfill}%
\pgfsetlinewidth{0.000000pt}%
\definecolor{currentstroke}{rgb}{0.000000,0.000000,0.000000}%
\pgfsetstrokecolor{currentstroke}%
\pgfsetdash{}{0pt}%
\pgfpathmoveto{\pgfqpoint{4.517160in}{0.953032in}}%
\pgfpathlineto{\pgfqpoint{4.668160in}{0.953032in}}%
\pgfpathlineto{\pgfqpoint{4.668160in}{1.384460in}}%
\pgfpathlineto{\pgfqpoint{4.517160in}{1.384460in}}%
\pgfpathlineto{\pgfqpoint{4.517160in}{0.953032in}}%
\pgfusepath{fill}%
\end{pgfscope}%
\begin{pgfscope}%
\pgfpathrectangle{\pgfqpoint{4.517160in}{0.521603in}}{\pgfqpoint{0.151000in}{3.020000in}}%
\pgfusepath{clip}%
\pgfsetbuttcap%
\pgfsetroundjoin%
\definecolor{currentfill}{rgb}{0.312255,0.312255,0.434442}%
\pgfsetfillcolor{currentfill}%
\pgfsetlinewidth{0.000000pt}%
\definecolor{currentstroke}{rgb}{0.000000,0.000000,0.000000}%
\pgfsetstrokecolor{currentstroke}%
\pgfsetdash{}{0pt}%
\pgfpathmoveto{\pgfqpoint{4.517160in}{1.384460in}}%
\pgfpathlineto{\pgfqpoint{4.668160in}{1.384460in}}%
\pgfpathlineto{\pgfqpoint{4.668160in}{1.815889in}}%
\pgfpathlineto{\pgfqpoint{4.517160in}{1.815889in}}%
\pgfpathlineto{\pgfqpoint{4.517160in}{1.384460in}}%
\pgfusepath{fill}%
\end{pgfscope}%
\begin{pgfscope}%
\pgfpathrectangle{\pgfqpoint{4.517160in}{0.521603in}}{\pgfqpoint{0.151000in}{3.020000in}}%
\pgfusepath{clip}%
\pgfsetbuttcap%
\pgfsetroundjoin%
\definecolor{currentfill}{rgb}{0.439216,0.484130,0.564216}%
\pgfsetfillcolor{currentfill}%
\pgfsetlinewidth{0.000000pt}%
\definecolor{currentstroke}{rgb}{0.000000,0.000000,0.000000}%
\pgfsetstrokecolor{currentstroke}%
\pgfsetdash{}{0pt}%
\pgfpathmoveto{\pgfqpoint{4.517160in}{1.815889in}}%
\pgfpathlineto{\pgfqpoint{4.668160in}{1.815889in}}%
\pgfpathlineto{\pgfqpoint{4.668160in}{2.247318in}}%
\pgfpathlineto{\pgfqpoint{4.517160in}{2.247318in}}%
\pgfpathlineto{\pgfqpoint{4.517160in}{1.815889in}}%
\pgfusepath{fill}%
\end{pgfscope}%
\begin{pgfscope}%
\pgfpathrectangle{\pgfqpoint{4.517160in}{0.521603in}}{\pgfqpoint{0.151000in}{3.020000in}}%
\pgfusepath{clip}%
\pgfsetbuttcap%
\pgfsetroundjoin%
\definecolor{currentfill}{rgb}{0.562745,0.653983,0.687745}%
\pgfsetfillcolor{currentfill}%
\pgfsetlinewidth{0.000000pt}%
\definecolor{currentstroke}{rgb}{0.000000,0.000000,0.000000}%
\pgfsetstrokecolor{currentstroke}%
\pgfsetdash{}{0pt}%
\pgfpathmoveto{\pgfqpoint{4.517160in}{2.247318in}}%
\pgfpathlineto{\pgfqpoint{4.668160in}{2.247318in}}%
\pgfpathlineto{\pgfqpoint{4.668160in}{2.678746in}}%
\pgfpathlineto{\pgfqpoint{4.517160in}{2.678746in}}%
\pgfpathlineto{\pgfqpoint{4.517160in}{2.247318in}}%
\pgfusepath{fill}%
\end{pgfscope}%
\begin{pgfscope}%
\pgfpathrectangle{\pgfqpoint{4.517160in}{0.521603in}}{\pgfqpoint{0.151000in}{3.020000in}}%
\pgfusepath{clip}%
\pgfsetbuttcap%
\pgfsetroundjoin%
\definecolor{currentfill}{rgb}{0.710478,0.814706,0.814706}%
\pgfsetfillcolor{currentfill}%
\pgfsetlinewidth{0.000000pt}%
\definecolor{currentstroke}{rgb}{0.000000,0.000000,0.000000}%
\pgfsetstrokecolor{currentstroke}%
\pgfsetdash{}{0pt}%
\pgfpathmoveto{\pgfqpoint{4.517160in}{2.678746in}}%
\pgfpathlineto{\pgfqpoint{4.668160in}{2.678746in}}%
\pgfpathlineto{\pgfqpoint{4.668160in}{3.110175in}}%
\pgfpathlineto{\pgfqpoint{4.517160in}{3.110175in}}%
\pgfpathlineto{\pgfqpoint{4.517160in}{2.678746in}}%
\pgfusepath{fill}%
\end{pgfscope}%
\begin{pgfscope}%
\pgfpathrectangle{\pgfqpoint{4.517160in}{0.521603in}}{\pgfqpoint{0.151000in}{3.020000in}}%
\pgfusepath{clip}%
\pgfsetbuttcap%
\pgfsetroundjoin%
\definecolor{currentfill}{rgb}{0.903493,0.938235,0.938235}%
\pgfsetfillcolor{currentfill}%
\pgfsetlinewidth{0.000000pt}%
\definecolor{currentstroke}{rgb}{0.000000,0.000000,0.000000}%
\pgfsetstrokecolor{currentstroke}%
\pgfsetdash{}{0pt}%
\pgfpathmoveto{\pgfqpoint{4.517160in}{3.110175in}}%
\pgfpathlineto{\pgfqpoint{4.668160in}{3.110175in}}%
\pgfpathlineto{\pgfqpoint{4.668160in}{3.541603in}}%
\pgfpathlineto{\pgfqpoint{4.517160in}{3.541603in}}%
\pgfpathlineto{\pgfqpoint{4.517160in}{3.110175in}}%
\pgfusepath{fill}%
\end{pgfscope}%
\begin{pgfscope}%
\pgfsetbuttcap%
\pgfsetroundjoin%
\definecolor{currentfill}{rgb}{0.000000,0.000000,0.000000}%
\pgfsetfillcolor{currentfill}%
\pgfsetlinewidth{0.803000pt}%
\definecolor{currentstroke}{rgb}{0.000000,0.000000,0.000000}%
\pgfsetstrokecolor{currentstroke}%
\pgfsetdash{}{0pt}%
\pgfsys@defobject{currentmarker}{\pgfqpoint{0.000000in}{0.000000in}}{\pgfqpoint{0.048611in}{0.000000in}}{%
\pgfpathmoveto{\pgfqpoint{0.000000in}{0.000000in}}%
\pgfpathlineto{\pgfqpoint{0.048611in}{0.000000in}}%
\pgfusepath{stroke,fill}%
}%
\begin{pgfscope}%
\pgfsys@transformshift{4.668160in}{0.521603in}%
\pgfsys@useobject{currentmarker}{}%
\end{pgfscope}%
\end{pgfscope}%
\begin{pgfscope}%
\definecolor{textcolor}{rgb}{0.000000,0.000000,0.000000}%
\pgfsetstrokecolor{textcolor}%
\pgfsetfillcolor{textcolor}%
\pgftext[x=4.765383in,y=0.468842in,left,base]{\color{textcolor}\rmfamily\fontsize{10.000000}{12.000000}\selectfont \(\displaystyle 0.0\)}%
\end{pgfscope}%
\begin{pgfscope}%
\pgfsetbuttcap%
\pgfsetroundjoin%
\definecolor{currentfill}{rgb}{0.000000,0.000000,0.000000}%
\pgfsetfillcolor{currentfill}%
\pgfsetlinewidth{0.803000pt}%
\definecolor{currentstroke}{rgb}{0.000000,0.000000,0.000000}%
\pgfsetstrokecolor{currentstroke}%
\pgfsetdash{}{0pt}%
\pgfsys@defobject{currentmarker}{\pgfqpoint{0.000000in}{0.000000in}}{\pgfqpoint{0.048611in}{0.000000in}}{%
\pgfpathmoveto{\pgfqpoint{0.000000in}{0.000000in}}%
\pgfpathlineto{\pgfqpoint{0.048611in}{0.000000in}}%
\pgfusepath{stroke,fill}%
}%
\begin{pgfscope}%
\pgfsys@transformshift{4.668160in}{0.953032in}%
\pgfsys@useobject{currentmarker}{}%
\end{pgfscope}%
\end{pgfscope}%
\begin{pgfscope}%
\definecolor{textcolor}{rgb}{0.000000,0.000000,0.000000}%
\pgfsetstrokecolor{textcolor}%
\pgfsetfillcolor{textcolor}%
\pgftext[x=4.765383in,y=0.900270in,left,base]{\color{textcolor}\rmfamily\fontsize{10.000000}{12.000000}\selectfont \(\displaystyle 0.2\)}%
\end{pgfscope}%
\begin{pgfscope}%
\pgfsetbuttcap%
\pgfsetroundjoin%
\definecolor{currentfill}{rgb}{0.000000,0.000000,0.000000}%
\pgfsetfillcolor{currentfill}%
\pgfsetlinewidth{0.803000pt}%
\definecolor{currentstroke}{rgb}{0.000000,0.000000,0.000000}%
\pgfsetstrokecolor{currentstroke}%
\pgfsetdash{}{0pt}%
\pgfsys@defobject{currentmarker}{\pgfqpoint{0.000000in}{0.000000in}}{\pgfqpoint{0.048611in}{0.000000in}}{%
\pgfpathmoveto{\pgfqpoint{0.000000in}{0.000000in}}%
\pgfpathlineto{\pgfqpoint{0.048611in}{0.000000in}}%
\pgfusepath{stroke,fill}%
}%
\begin{pgfscope}%
\pgfsys@transformshift{4.668160in}{1.384460in}%
\pgfsys@useobject{currentmarker}{}%
\end{pgfscope}%
\end{pgfscope}%
\begin{pgfscope}%
\definecolor{textcolor}{rgb}{0.000000,0.000000,0.000000}%
\pgfsetstrokecolor{textcolor}%
\pgfsetfillcolor{textcolor}%
\pgftext[x=4.765383in,y=1.331699in,left,base]{\color{textcolor}\rmfamily\fontsize{10.000000}{12.000000}\selectfont \(\displaystyle 0.4\)}%
\end{pgfscope}%
\begin{pgfscope}%
\pgfsetbuttcap%
\pgfsetroundjoin%
\definecolor{currentfill}{rgb}{0.000000,0.000000,0.000000}%
\pgfsetfillcolor{currentfill}%
\pgfsetlinewidth{0.803000pt}%
\definecolor{currentstroke}{rgb}{0.000000,0.000000,0.000000}%
\pgfsetstrokecolor{currentstroke}%
\pgfsetdash{}{0pt}%
\pgfsys@defobject{currentmarker}{\pgfqpoint{0.000000in}{0.000000in}}{\pgfqpoint{0.048611in}{0.000000in}}{%
\pgfpathmoveto{\pgfqpoint{0.000000in}{0.000000in}}%
\pgfpathlineto{\pgfqpoint{0.048611in}{0.000000in}}%
\pgfusepath{stroke,fill}%
}%
\begin{pgfscope}%
\pgfsys@transformshift{4.668160in}{1.815889in}%
\pgfsys@useobject{currentmarker}{}%
\end{pgfscope}%
\end{pgfscope}%
\begin{pgfscope}%
\definecolor{textcolor}{rgb}{0.000000,0.000000,0.000000}%
\pgfsetstrokecolor{textcolor}%
\pgfsetfillcolor{textcolor}%
\pgftext[x=4.765383in,y=1.763128in,left,base]{\color{textcolor}\rmfamily\fontsize{10.000000}{12.000000}\selectfont \(\displaystyle 0.6\)}%
\end{pgfscope}%
\begin{pgfscope}%
\pgfsetbuttcap%
\pgfsetroundjoin%
\definecolor{currentfill}{rgb}{0.000000,0.000000,0.000000}%
\pgfsetfillcolor{currentfill}%
\pgfsetlinewidth{0.803000pt}%
\definecolor{currentstroke}{rgb}{0.000000,0.000000,0.000000}%
\pgfsetstrokecolor{currentstroke}%
\pgfsetdash{}{0pt}%
\pgfsys@defobject{currentmarker}{\pgfqpoint{0.000000in}{0.000000in}}{\pgfqpoint{0.048611in}{0.000000in}}{%
\pgfpathmoveto{\pgfqpoint{0.000000in}{0.000000in}}%
\pgfpathlineto{\pgfqpoint{0.048611in}{0.000000in}}%
\pgfusepath{stroke,fill}%
}%
\begin{pgfscope}%
\pgfsys@transformshift{4.668160in}{2.247318in}%
\pgfsys@useobject{currentmarker}{}%
\end{pgfscope}%
\end{pgfscope}%
\begin{pgfscope}%
\definecolor{textcolor}{rgb}{0.000000,0.000000,0.000000}%
\pgfsetstrokecolor{textcolor}%
\pgfsetfillcolor{textcolor}%
\pgftext[x=4.765383in,y=2.194556in,left,base]{\color{textcolor}\rmfamily\fontsize{10.000000}{12.000000}\selectfont \(\displaystyle 0.8\)}%
\end{pgfscope}%
\begin{pgfscope}%
\pgfsetbuttcap%
\pgfsetroundjoin%
\definecolor{currentfill}{rgb}{0.000000,0.000000,0.000000}%
\pgfsetfillcolor{currentfill}%
\pgfsetlinewidth{0.803000pt}%
\definecolor{currentstroke}{rgb}{0.000000,0.000000,0.000000}%
\pgfsetstrokecolor{currentstroke}%
\pgfsetdash{}{0pt}%
\pgfsys@defobject{currentmarker}{\pgfqpoint{0.000000in}{0.000000in}}{\pgfqpoint{0.048611in}{0.000000in}}{%
\pgfpathmoveto{\pgfqpoint{0.000000in}{0.000000in}}%
\pgfpathlineto{\pgfqpoint{0.048611in}{0.000000in}}%
\pgfusepath{stroke,fill}%
}%
\begin{pgfscope}%
\pgfsys@transformshift{4.668160in}{2.678746in}%
\pgfsys@useobject{currentmarker}{}%
\end{pgfscope}%
\end{pgfscope}%
\begin{pgfscope}%
\definecolor{textcolor}{rgb}{0.000000,0.000000,0.000000}%
\pgfsetstrokecolor{textcolor}%
\pgfsetfillcolor{textcolor}%
\pgftext[x=4.765383in,y=2.625985in,left,base]{\color{textcolor}\rmfamily\fontsize{10.000000}{12.000000}\selectfont \(\displaystyle 1.0\)}%
\end{pgfscope}%
\begin{pgfscope}%
\pgfsetbuttcap%
\pgfsetroundjoin%
\definecolor{currentfill}{rgb}{0.000000,0.000000,0.000000}%
\pgfsetfillcolor{currentfill}%
\pgfsetlinewidth{0.803000pt}%
\definecolor{currentstroke}{rgb}{0.000000,0.000000,0.000000}%
\pgfsetstrokecolor{currentstroke}%
\pgfsetdash{}{0pt}%
\pgfsys@defobject{currentmarker}{\pgfqpoint{0.000000in}{0.000000in}}{\pgfqpoint{0.048611in}{0.000000in}}{%
\pgfpathmoveto{\pgfqpoint{0.000000in}{0.000000in}}%
\pgfpathlineto{\pgfqpoint{0.048611in}{0.000000in}}%
\pgfusepath{stroke,fill}%
}%
\begin{pgfscope}%
\pgfsys@transformshift{4.668160in}{3.110175in}%
\pgfsys@useobject{currentmarker}{}%
\end{pgfscope}%
\end{pgfscope}%
\begin{pgfscope}%
\definecolor{textcolor}{rgb}{0.000000,0.000000,0.000000}%
\pgfsetstrokecolor{textcolor}%
\pgfsetfillcolor{textcolor}%
\pgftext[x=4.765383in,y=3.057413in,left,base]{\color{textcolor}\rmfamily\fontsize{10.000000}{12.000000}\selectfont \(\displaystyle 1.2\)}%
\end{pgfscope}%
\begin{pgfscope}%
\pgfsetbuttcap%
\pgfsetroundjoin%
\definecolor{currentfill}{rgb}{0.000000,0.000000,0.000000}%
\pgfsetfillcolor{currentfill}%
\pgfsetlinewidth{0.803000pt}%
\definecolor{currentstroke}{rgb}{0.000000,0.000000,0.000000}%
\pgfsetstrokecolor{currentstroke}%
\pgfsetdash{}{0pt}%
\pgfsys@defobject{currentmarker}{\pgfqpoint{0.000000in}{0.000000in}}{\pgfqpoint{0.048611in}{0.000000in}}{%
\pgfpathmoveto{\pgfqpoint{0.000000in}{0.000000in}}%
\pgfpathlineto{\pgfqpoint{0.048611in}{0.000000in}}%
\pgfusepath{stroke,fill}%
}%
\begin{pgfscope}%
\pgfsys@transformshift{4.668160in}{3.541603in}%
\pgfsys@useobject{currentmarker}{}%
\end{pgfscope}%
\end{pgfscope}%
\begin{pgfscope}%
\definecolor{textcolor}{rgb}{0.000000,0.000000,0.000000}%
\pgfsetstrokecolor{textcolor}%
\pgfsetfillcolor{textcolor}%
\pgftext[x=4.765383in,y=3.488842in,left,base]{\color{textcolor}\rmfamily\fontsize{10.000000}{12.000000}\selectfont \(\displaystyle 1.4\)}%
\end{pgfscope}%
\begin{pgfscope}%
\definecolor{textcolor}{rgb}{0.000000,0.000000,0.000000}%
\pgfsetstrokecolor{textcolor}%
\pgfsetfillcolor{textcolor}%
\pgftext[x=4.998408in,y=2.031603in,,top,rotate=90.000000]{\color{textcolor}\rmfamily\fontsize{10.000000}{12.000000}\selectfont \(\displaystyle q\) (C)}%
\end{pgfscope}%
\begin{pgfscope}%
\definecolor{textcolor}{rgb}{0.000000,0.000000,0.000000}%
\pgfsetstrokecolor{textcolor}%
\pgfsetfillcolor{textcolor}%
\pgftext[x=4.668160in,y=3.583270in,right,base]{\color{textcolor}\rmfamily\fontsize{10.000000}{12.000000}\selectfont \(\displaystyle \times10^{-9}\)}%
\end{pgfscope}%
\begin{pgfscope}%
\pgfsetbuttcap%
\pgfsetmiterjoin%
\pgfsetlinewidth{0.803000pt}%
\definecolor{currentstroke}{rgb}{0.501961,0.501961,0.501961}%
\pgfsetstrokecolor{currentstroke}%
\pgfsetdash{}{0pt}%
\pgfpathmoveto{\pgfqpoint{4.517160in}{0.521603in}}%
\pgfpathlineto{\pgfqpoint{4.517160in}{0.953032in}}%
\pgfpathlineto{\pgfqpoint{4.517160in}{3.110175in}}%
\pgfpathlineto{\pgfqpoint{4.517160in}{3.541603in}}%
\pgfpathlineto{\pgfqpoint{4.668160in}{3.541603in}}%
\pgfpathlineto{\pgfqpoint{4.668160in}{3.110175in}}%
\pgfpathlineto{\pgfqpoint{4.668160in}{0.953032in}}%
\pgfpathlineto{\pgfqpoint{4.668160in}{0.521603in}}%
\pgfpathclose%
\pgfusepath{stroke}%
\end{pgfscope}%
\end{pgfpicture}%
\makeatother%
\endgroup%

    \caption{A simple EMA plot.\label{fig:charge}}
\end{figure}

\begin{figure}[htb]
    \centering
    %% Creator: Matplotlib, PGF backend
%%
%% To include the figure in your LaTeX document, write
%%   \input{<filename>.pgf}
%%
%% Make sure the required packages are loaded in your preamble
%%   \usepackage{pgf}
%%
%% Figures using additional raster images can only be included by \input if
%% they are in the same directory as the main LaTeX file. For loading figures
%% from other directories you can use the `import` package
%%   \usepackage{import}
%% and then include the figures with
%%   \import{<path to file>}{<filename>.pgf}
%%
%% Matplotlib used the following preamble
%%   \usepackage[utf8x]{inputenc}
%%   \usepackage[T1]{fontenc}
%%
\begingroup%
\makeatletter%
\begin{pgfpicture}%
\pgfpathrectangle{\pgfpointorigin}{\pgfqpoint{5.770716in}{3.735201in}}%
\pgfusepath{use as bounding box, clip}%
\begin{pgfscope}%
\pgfsetbuttcap%
\pgfsetmiterjoin%
\definecolor{currentfill}{rgb}{1.000000,1.000000,1.000000}%
\pgfsetfillcolor{currentfill}%
\pgfsetlinewidth{0.000000pt}%
\definecolor{currentstroke}{rgb}{1.000000,1.000000,1.000000}%
\pgfsetstrokecolor{currentstroke}%
\pgfsetdash{}{0pt}%
\pgfpathmoveto{\pgfqpoint{0.000000in}{0.000000in}}%
\pgfpathlineto{\pgfqpoint{5.770716in}{0.000000in}}%
\pgfpathlineto{\pgfqpoint{5.770716in}{3.735201in}}%
\pgfpathlineto{\pgfqpoint{0.000000in}{3.735201in}}%
\pgfpathclose%
\pgfusepath{fill}%
\end{pgfscope}%
\begin{pgfscope}%
\pgfsetbuttcap%
\pgfsetmiterjoin%
\definecolor{currentfill}{rgb}{1.000000,1.000000,1.000000}%
\pgfsetfillcolor{currentfill}%
\pgfsetlinewidth{0.000000pt}%
\definecolor{currentstroke}{rgb}{0.000000,0.000000,0.000000}%
\pgfsetstrokecolor{currentstroke}%
\pgfsetstrokeopacity{0.000000}%
\pgfsetdash{}{0pt}%
\pgfpathmoveto{\pgfqpoint{0.501000in}{0.566590in}}%
\pgfpathlineto{\pgfqpoint{4.221000in}{0.566590in}}%
\pgfpathlineto{\pgfqpoint{4.221000in}{3.586590in}}%
\pgfpathlineto{\pgfqpoint{0.501000in}{3.586590in}}%
\pgfpathclose%
\pgfusepath{fill}%
\end{pgfscope}%
\begin{pgfscope}%
\pgfsetbuttcap%
\pgfsetroundjoin%
\definecolor{currentfill}{rgb}{0.000000,0.000000,0.000000}%
\pgfsetfillcolor{currentfill}%
\pgfsetlinewidth{0.803000pt}%
\definecolor{currentstroke}{rgb}{0.000000,0.000000,0.000000}%
\pgfsetstrokecolor{currentstroke}%
\pgfsetdash{}{0pt}%
\pgfsys@defobject{currentmarker}{\pgfqpoint{0.000000in}{-0.048611in}}{\pgfqpoint{0.000000in}{0.000000in}}{%
\pgfpathmoveto{\pgfqpoint{0.000000in}{0.000000in}}%
\pgfpathlineto{\pgfqpoint{0.000000in}{-0.048611in}}%
\pgfusepath{stroke,fill}%
}%
\begin{pgfscope}%
\pgfsys@transformshift{1.209572in}{0.566590in}%
\pgfsys@useobject{currentmarker}{}%
\end{pgfscope}%
\end{pgfscope}%
\begin{pgfscope}%
\pgftext[x=1.209572in,y=0.469368in,,top]{\rmfamily\fontsize{12.000000}{14.400000}\selectfont \(\displaystyle 0.5\)}%
\end{pgfscope}%
\begin{pgfscope}%
\pgfsetbuttcap%
\pgfsetroundjoin%
\definecolor{currentfill}{rgb}{0.000000,0.000000,0.000000}%
\pgfsetfillcolor{currentfill}%
\pgfsetlinewidth{0.803000pt}%
\definecolor{currentstroke}{rgb}{0.000000,0.000000,0.000000}%
\pgfsetstrokecolor{currentstroke}%
\pgfsetdash{}{0pt}%
\pgfsys@defobject{currentmarker}{\pgfqpoint{0.000000in}{-0.048611in}}{\pgfqpoint{0.000000in}{0.000000in}}{%
\pgfpathmoveto{\pgfqpoint{0.000000in}{0.000000in}}%
\pgfpathlineto{\pgfqpoint{0.000000in}{-0.048611in}}%
\pgfusepath{stroke,fill}%
}%
\begin{pgfscope}%
\pgfsys@transformshift{2.095286in}{0.566590in}%
\pgfsys@useobject{currentmarker}{}%
\end{pgfscope}%
\end{pgfscope}%
\begin{pgfscope}%
\pgftext[x=2.095286in,y=0.469368in,,top]{\rmfamily\fontsize{12.000000}{14.400000}\selectfont \(\displaystyle 1.0\)}%
\end{pgfscope}%
\begin{pgfscope}%
\pgfsetbuttcap%
\pgfsetroundjoin%
\definecolor{currentfill}{rgb}{0.000000,0.000000,0.000000}%
\pgfsetfillcolor{currentfill}%
\pgfsetlinewidth{0.803000pt}%
\definecolor{currentstroke}{rgb}{0.000000,0.000000,0.000000}%
\pgfsetstrokecolor{currentstroke}%
\pgfsetdash{}{0pt}%
\pgfsys@defobject{currentmarker}{\pgfqpoint{0.000000in}{-0.048611in}}{\pgfqpoint{0.000000in}{0.000000in}}{%
\pgfpathmoveto{\pgfqpoint{0.000000in}{0.000000in}}%
\pgfpathlineto{\pgfqpoint{0.000000in}{-0.048611in}}%
\pgfusepath{stroke,fill}%
}%
\begin{pgfscope}%
\pgfsys@transformshift{2.981000in}{0.566590in}%
\pgfsys@useobject{currentmarker}{}%
\end{pgfscope}%
\end{pgfscope}%
\begin{pgfscope}%
\pgftext[x=2.981000in,y=0.469368in,,top]{\rmfamily\fontsize{12.000000}{14.400000}\selectfont \(\displaystyle 1.5\)}%
\end{pgfscope}%
\begin{pgfscope}%
\pgfsetbuttcap%
\pgfsetroundjoin%
\definecolor{currentfill}{rgb}{0.000000,0.000000,0.000000}%
\pgfsetfillcolor{currentfill}%
\pgfsetlinewidth{0.803000pt}%
\definecolor{currentstroke}{rgb}{0.000000,0.000000,0.000000}%
\pgfsetstrokecolor{currentstroke}%
\pgfsetdash{}{0pt}%
\pgfsys@defobject{currentmarker}{\pgfqpoint{0.000000in}{-0.048611in}}{\pgfqpoint{0.000000in}{0.000000in}}{%
\pgfpathmoveto{\pgfqpoint{0.000000in}{0.000000in}}%
\pgfpathlineto{\pgfqpoint{0.000000in}{-0.048611in}}%
\pgfusepath{stroke,fill}%
}%
\begin{pgfscope}%
\pgfsys@transformshift{3.866714in}{0.566590in}%
\pgfsys@useobject{currentmarker}{}%
\end{pgfscope}%
\end{pgfscope}%
\begin{pgfscope}%
\pgftext[x=3.866714in,y=0.469368in,,top]{\rmfamily\fontsize{12.000000}{14.400000}\selectfont \(\displaystyle 2.0\)}%
\end{pgfscope}%
\begin{pgfscope}%
\pgftext[x=2.361000in,y=0.266626in,,top]{\rmfamily\fontsize{12.000000}{14.400000}\selectfont \(\displaystyle t\) (s)}%
\end{pgfscope}%
\begin{pgfscope}%
\pgfsetbuttcap%
\pgfsetroundjoin%
\definecolor{currentfill}{rgb}{0.000000,0.000000,0.000000}%
\pgfsetfillcolor{currentfill}%
\pgfsetlinewidth{0.803000pt}%
\definecolor{currentstroke}{rgb}{0.000000,0.000000,0.000000}%
\pgfsetstrokecolor{currentstroke}%
\pgfsetdash{}{0pt}%
\pgfsys@defobject{currentmarker}{\pgfqpoint{-0.048611in}{0.000000in}}{\pgfqpoint{0.000000in}{0.000000in}}{%
\pgfpathmoveto{\pgfqpoint{0.000000in}{0.000000in}}%
\pgfpathlineto{\pgfqpoint{-0.048611in}{0.000000in}}%
\pgfusepath{stroke,fill}%
}%
\begin{pgfscope}%
\pgfsys@transformshift{0.501000in}{0.642969in}%
\pgfsys@useobject{currentmarker}{}%
\end{pgfscope}%
\end{pgfscope}%
\begin{pgfscope}%
\pgftext[x=0.322182in,y=0.585576in,left,base]{\rmfamily\fontsize{12.000000}{14.400000}\selectfont \(\displaystyle 0\)}%
\end{pgfscope}%
\begin{pgfscope}%
\pgfsetbuttcap%
\pgfsetroundjoin%
\definecolor{currentfill}{rgb}{0.000000,0.000000,0.000000}%
\pgfsetfillcolor{currentfill}%
\pgfsetlinewidth{0.803000pt}%
\definecolor{currentstroke}{rgb}{0.000000,0.000000,0.000000}%
\pgfsetstrokecolor{currentstroke}%
\pgfsetdash{}{0pt}%
\pgfsys@defobject{currentmarker}{\pgfqpoint{-0.048611in}{0.000000in}}{\pgfqpoint{0.000000in}{0.000000in}}{%
\pgfpathmoveto{\pgfqpoint{0.000000in}{0.000000in}}%
\pgfpathlineto{\pgfqpoint{-0.048611in}{0.000000in}}%
\pgfusepath{stroke,fill}%
}%
\begin{pgfscope}%
\pgfsys@transformshift{0.501000in}{1.009698in}%
\pgfsys@useobject{currentmarker}{}%
\end{pgfscope}%
\end{pgfscope}%
\begin{pgfscope}%
\pgftext[x=0.322182in,y=0.952305in,left,base]{\rmfamily\fontsize{12.000000}{14.400000}\selectfont \(\displaystyle 1\)}%
\end{pgfscope}%
\begin{pgfscope}%
\pgfsetbuttcap%
\pgfsetroundjoin%
\definecolor{currentfill}{rgb}{0.000000,0.000000,0.000000}%
\pgfsetfillcolor{currentfill}%
\pgfsetlinewidth{0.803000pt}%
\definecolor{currentstroke}{rgb}{0.000000,0.000000,0.000000}%
\pgfsetstrokecolor{currentstroke}%
\pgfsetdash{}{0pt}%
\pgfsys@defobject{currentmarker}{\pgfqpoint{-0.048611in}{0.000000in}}{\pgfqpoint{0.000000in}{0.000000in}}{%
\pgfpathmoveto{\pgfqpoint{0.000000in}{0.000000in}}%
\pgfpathlineto{\pgfqpoint{-0.048611in}{0.000000in}}%
\pgfusepath{stroke,fill}%
}%
\begin{pgfscope}%
\pgfsys@transformshift{0.501000in}{1.376427in}%
\pgfsys@useobject{currentmarker}{}%
\end{pgfscope}%
\end{pgfscope}%
\begin{pgfscope}%
\pgftext[x=0.322182in,y=1.319033in,left,base]{\rmfamily\fontsize{12.000000}{14.400000}\selectfont \(\displaystyle 2\)}%
\end{pgfscope}%
\begin{pgfscope}%
\pgfsetbuttcap%
\pgfsetroundjoin%
\definecolor{currentfill}{rgb}{0.000000,0.000000,0.000000}%
\pgfsetfillcolor{currentfill}%
\pgfsetlinewidth{0.803000pt}%
\definecolor{currentstroke}{rgb}{0.000000,0.000000,0.000000}%
\pgfsetstrokecolor{currentstroke}%
\pgfsetdash{}{0pt}%
\pgfsys@defobject{currentmarker}{\pgfqpoint{-0.048611in}{0.000000in}}{\pgfqpoint{0.000000in}{0.000000in}}{%
\pgfpathmoveto{\pgfqpoint{0.000000in}{0.000000in}}%
\pgfpathlineto{\pgfqpoint{-0.048611in}{0.000000in}}%
\pgfusepath{stroke,fill}%
}%
\begin{pgfscope}%
\pgfsys@transformshift{0.501000in}{1.743156in}%
\pgfsys@useobject{currentmarker}{}%
\end{pgfscope}%
\end{pgfscope}%
\begin{pgfscope}%
\pgftext[x=0.322182in,y=1.685762in,left,base]{\rmfamily\fontsize{12.000000}{14.400000}\selectfont \(\displaystyle 3\)}%
\end{pgfscope}%
\begin{pgfscope}%
\pgfsetbuttcap%
\pgfsetroundjoin%
\definecolor{currentfill}{rgb}{0.000000,0.000000,0.000000}%
\pgfsetfillcolor{currentfill}%
\pgfsetlinewidth{0.803000pt}%
\definecolor{currentstroke}{rgb}{0.000000,0.000000,0.000000}%
\pgfsetstrokecolor{currentstroke}%
\pgfsetdash{}{0pt}%
\pgfsys@defobject{currentmarker}{\pgfqpoint{-0.048611in}{0.000000in}}{\pgfqpoint{0.000000in}{0.000000in}}{%
\pgfpathmoveto{\pgfqpoint{0.000000in}{0.000000in}}%
\pgfpathlineto{\pgfqpoint{-0.048611in}{0.000000in}}%
\pgfusepath{stroke,fill}%
}%
\begin{pgfscope}%
\pgfsys@transformshift{0.501000in}{2.109884in}%
\pgfsys@useobject{currentmarker}{}%
\end{pgfscope}%
\end{pgfscope}%
\begin{pgfscope}%
\pgftext[x=0.322182in,y=2.052491in,left,base]{\rmfamily\fontsize{12.000000}{14.400000}\selectfont \(\displaystyle 4\)}%
\end{pgfscope}%
\begin{pgfscope}%
\pgfsetbuttcap%
\pgfsetroundjoin%
\definecolor{currentfill}{rgb}{0.000000,0.000000,0.000000}%
\pgfsetfillcolor{currentfill}%
\pgfsetlinewidth{0.803000pt}%
\definecolor{currentstroke}{rgb}{0.000000,0.000000,0.000000}%
\pgfsetstrokecolor{currentstroke}%
\pgfsetdash{}{0pt}%
\pgfsys@defobject{currentmarker}{\pgfqpoint{-0.048611in}{0.000000in}}{\pgfqpoint{0.000000in}{0.000000in}}{%
\pgfpathmoveto{\pgfqpoint{0.000000in}{0.000000in}}%
\pgfpathlineto{\pgfqpoint{-0.048611in}{0.000000in}}%
\pgfusepath{stroke,fill}%
}%
\begin{pgfscope}%
\pgfsys@transformshift{0.501000in}{2.476613in}%
\pgfsys@useobject{currentmarker}{}%
\end{pgfscope}%
\end{pgfscope}%
\begin{pgfscope}%
\pgftext[x=0.322182in,y=2.419220in,left,base]{\rmfamily\fontsize{12.000000}{14.400000}\selectfont \(\displaystyle 5\)}%
\end{pgfscope}%
\begin{pgfscope}%
\pgfsetbuttcap%
\pgfsetroundjoin%
\definecolor{currentfill}{rgb}{0.000000,0.000000,0.000000}%
\pgfsetfillcolor{currentfill}%
\pgfsetlinewidth{0.803000pt}%
\definecolor{currentstroke}{rgb}{0.000000,0.000000,0.000000}%
\pgfsetstrokecolor{currentstroke}%
\pgfsetdash{}{0pt}%
\pgfsys@defobject{currentmarker}{\pgfqpoint{-0.048611in}{0.000000in}}{\pgfqpoint{0.000000in}{0.000000in}}{%
\pgfpathmoveto{\pgfqpoint{0.000000in}{0.000000in}}%
\pgfpathlineto{\pgfqpoint{-0.048611in}{0.000000in}}%
\pgfusepath{stroke,fill}%
}%
\begin{pgfscope}%
\pgfsys@transformshift{0.501000in}{2.843342in}%
\pgfsys@useobject{currentmarker}{}%
\end{pgfscope}%
\end{pgfscope}%
\begin{pgfscope}%
\pgftext[x=0.322182in,y=2.785949in,left,base]{\rmfamily\fontsize{12.000000}{14.400000}\selectfont \(\displaystyle 6\)}%
\end{pgfscope}%
\begin{pgfscope}%
\pgfsetbuttcap%
\pgfsetroundjoin%
\definecolor{currentfill}{rgb}{0.000000,0.000000,0.000000}%
\pgfsetfillcolor{currentfill}%
\pgfsetlinewidth{0.803000pt}%
\definecolor{currentstroke}{rgb}{0.000000,0.000000,0.000000}%
\pgfsetstrokecolor{currentstroke}%
\pgfsetdash{}{0pt}%
\pgfsys@defobject{currentmarker}{\pgfqpoint{-0.048611in}{0.000000in}}{\pgfqpoint{0.000000in}{0.000000in}}{%
\pgfpathmoveto{\pgfqpoint{0.000000in}{0.000000in}}%
\pgfpathlineto{\pgfqpoint{-0.048611in}{0.000000in}}%
\pgfusepath{stroke,fill}%
}%
\begin{pgfscope}%
\pgfsys@transformshift{0.501000in}{3.210071in}%
\pgfsys@useobject{currentmarker}{}%
\end{pgfscope}%
\end{pgfscope}%
\begin{pgfscope}%
\pgftext[x=0.322182in,y=3.152677in,left,base]{\rmfamily\fontsize{12.000000}{14.400000}\selectfont \(\displaystyle 7\)}%
\end{pgfscope}%
\begin{pgfscope}%
\pgfsetbuttcap%
\pgfsetroundjoin%
\definecolor{currentfill}{rgb}{0.000000,0.000000,0.000000}%
\pgfsetfillcolor{currentfill}%
\pgfsetlinewidth{0.803000pt}%
\definecolor{currentstroke}{rgb}{0.000000,0.000000,0.000000}%
\pgfsetstrokecolor{currentstroke}%
\pgfsetdash{}{0pt}%
\pgfsys@defobject{currentmarker}{\pgfqpoint{-0.048611in}{0.000000in}}{\pgfqpoint{0.000000in}{0.000000in}}{%
\pgfpathmoveto{\pgfqpoint{0.000000in}{0.000000in}}%
\pgfpathlineto{\pgfqpoint{-0.048611in}{0.000000in}}%
\pgfusepath{stroke,fill}%
}%
\begin{pgfscope}%
\pgfsys@transformshift{0.501000in}{3.576800in}%
\pgfsys@useobject{currentmarker}{}%
\end{pgfscope}%
\end{pgfscope}%
\begin{pgfscope}%
\pgftext[x=0.322182in,y=3.519406in,left,base]{\rmfamily\fontsize{12.000000}{14.400000}\selectfont \(\displaystyle 8\)}%
\end{pgfscope}%
\begin{pgfscope}%
\pgftext[x=0.266626in,y=2.076590in,,bottom,rotate=90.000000]{\rmfamily\fontsize{12.000000}{14.400000}\selectfont \(\displaystyle y\) (cm)}%
\end{pgfscope}%
\begin{pgfscope}%
\pgfpathrectangle{\pgfqpoint{0.501000in}{0.566590in}}{\pgfqpoint{3.720000in}{3.020000in}} %
\pgfusepath{clip}%
\pgfsetrectcap%
\pgfsetroundjoin%
\pgfsetlinewidth{1.505625pt}%
\definecolor{currentstroke}{rgb}{0.500000,0.000000,1.000000}%
\pgfsetstrokecolor{currentstroke}%
\pgfsetdash{}{0pt}%
\pgfpathmoveto{\pgfqpoint{0.487111in}{1.045226in}}%
\pgfpathlineto{\pgfqpoint{0.501000in}{1.074337in}}%
\pgfpathlineto{\pgfqpoint{0.515762in}{1.102645in}}%
\pgfpathlineto{\pgfqpoint{0.530524in}{1.126774in}}%
\pgfpathlineto{\pgfqpoint{0.545286in}{1.151928in}}%
\pgfpathlineto{\pgfqpoint{0.560048in}{1.174547in}}%
\pgfpathlineto{\pgfqpoint{0.574810in}{1.201417in}}%
\pgfpathlineto{\pgfqpoint{0.589572in}{1.230112in}}%
\pgfpathlineto{\pgfqpoint{0.604333in}{1.251935in}}%
\pgfpathlineto{\pgfqpoint{0.619095in}{1.275884in}}%
\pgfpathlineto{\pgfqpoint{0.633857in}{1.305137in}}%
\pgfpathlineto{\pgfqpoint{0.648619in}{1.331365in}}%
\pgfpathlineto{\pgfqpoint{0.663381in}{1.354408in}}%
\pgfpathlineto{\pgfqpoint{0.678143in}{1.374340in}}%
\pgfpathlineto{\pgfqpoint{0.692905in}{1.400486in}}%
\pgfpathlineto{\pgfqpoint{0.707667in}{1.427625in}}%
\pgfpathlineto{\pgfqpoint{0.722429in}{1.453041in}}%
\pgfpathlineto{\pgfqpoint{0.737191in}{1.474783in}}%
\pgfpathlineto{\pgfqpoint{0.751953in}{1.499082in}}%
\pgfpathlineto{\pgfqpoint{0.766714in}{1.525359in}}%
\pgfpathlineto{\pgfqpoint{0.781476in}{1.548829in}}%
\pgfpathlineto{\pgfqpoint{0.796238in}{1.569832in}}%
\pgfpathlineto{\pgfqpoint{0.811000in}{1.593525in}}%
\pgfpathlineto{\pgfqpoint{0.825762in}{1.618324in}}%
\pgfpathlineto{\pgfqpoint{0.840524in}{1.643381in}}%
\pgfpathlineto{\pgfqpoint{0.855286in}{1.663311in}}%
\pgfpathlineto{\pgfqpoint{0.870048in}{1.687126in}}%
\pgfpathlineto{\pgfqpoint{0.884810in}{1.711993in}}%
\pgfpathlineto{\pgfqpoint{0.899572in}{1.735587in}}%
\pgfpathlineto{\pgfqpoint{0.914333in}{1.756876in}}%
\pgfpathlineto{\pgfqpoint{0.929095in}{1.778567in}}%
\pgfpathlineto{\pgfqpoint{0.943857in}{1.802092in}}%
\pgfpathlineto{\pgfqpoint{0.958619in}{1.826311in}}%
\pgfpathlineto{\pgfqpoint{0.973381in}{1.845729in}}%
\pgfpathlineto{\pgfqpoint{0.988143in}{1.869787in}}%
\pgfpathlineto{\pgfqpoint{1.002905in}{1.892445in}}%
\pgfpathlineto{\pgfqpoint{1.017667in}{1.915958in}}%
\pgfpathlineto{\pgfqpoint{1.032429in}{1.937485in}}%
\pgfpathlineto{\pgfqpoint{1.047191in}{1.958490in}}%
\pgfpathlineto{\pgfqpoint{1.061953in}{1.978498in}}%
\pgfpathlineto{\pgfqpoint{1.076714in}{2.002561in}}%
\pgfpathlineto{\pgfqpoint{1.091476in}{2.023527in}}%
\pgfpathlineto{\pgfqpoint{1.106238in}{2.047054in}}%
\pgfpathlineto{\pgfqpoint{1.121000in}{2.069427in}}%
\pgfpathlineto{\pgfqpoint{1.135762in}{2.092499in}}%
\pgfpathlineto{\pgfqpoint{1.150524in}{2.114376in}}%
\pgfpathlineto{\pgfqpoint{1.165286in}{2.133825in}}%
\pgfpathlineto{\pgfqpoint{1.180048in}{2.153003in}}%
\pgfpathlineto{\pgfqpoint{1.194810in}{2.175907in}}%
\pgfpathlineto{\pgfqpoint{1.209572in}{2.197673in}}%
\pgfpathlineto{\pgfqpoint{1.224333in}{2.220081in}}%
\pgfpathlineto{\pgfqpoint{1.239095in}{2.242686in}}%
\pgfpathlineto{\pgfqpoint{1.253857in}{2.265069in}}%
\pgfpathlineto{\pgfqpoint{1.268619in}{2.287084in}}%
\pgfpathlineto{\pgfqpoint{1.283381in}{2.306296in}}%
\pgfpathlineto{\pgfqpoint{1.298143in}{2.324973in}}%
\pgfpathlineto{\pgfqpoint{1.312905in}{2.345597in}}%
\pgfpathlineto{\pgfqpoint{1.327667in}{2.368203in}}%
\pgfpathlineto{\pgfqpoint{1.342429in}{2.390363in}}%
\pgfpathlineto{\pgfqpoint{1.357191in}{2.413851in}}%
\pgfpathlineto{\pgfqpoint{1.371953in}{2.435551in}}%
\pgfpathlineto{\pgfqpoint{1.386714in}{2.457271in}}%
\pgfpathlineto{\pgfqpoint{1.401476in}{2.475980in}}%
\pgfpathlineto{\pgfqpoint{1.416238in}{2.494665in}}%
\pgfpathlineto{\pgfqpoint{1.431000in}{2.514310in}}%
\pgfpathlineto{\pgfqpoint{1.445762in}{2.535941in}}%
\pgfpathlineto{\pgfqpoint{1.460524in}{2.558888in}}%
\pgfpathlineto{\pgfqpoint{1.475286in}{2.582619in}}%
\pgfpathlineto{\pgfqpoint{1.490048in}{2.603774in}}%
\pgfpathlineto{\pgfqpoint{1.504810in}{2.624842in}}%
\pgfpathlineto{\pgfqpoint{1.519572in}{2.643625in}}%
\pgfpathlineto{\pgfqpoint{1.534333in}{2.661259in}}%
\pgfpathlineto{\pgfqpoint{1.549095in}{2.680627in}}%
\pgfpathlineto{\pgfqpoint{1.563857in}{2.701084in}}%
\pgfpathlineto{\pgfqpoint{1.578619in}{2.724323in}}%
\pgfpathlineto{\pgfqpoint{1.593381in}{2.747477in}}%
\pgfpathlineto{\pgfqpoint{1.608143in}{2.769253in}}%
\pgfpathlineto{\pgfqpoint{1.622905in}{2.790354in}}%
\pgfpathlineto{\pgfqpoint{1.637667in}{2.808290in}}%
\pgfpathlineto{\pgfqpoint{1.652429in}{2.825577in}}%
\pgfpathlineto{\pgfqpoint{1.667191in}{2.844883in}}%
\pgfpathlineto{\pgfqpoint{1.681953in}{2.865440in}}%
\pgfpathlineto{\pgfqpoint{1.696714in}{2.887808in}}%
\pgfpathlineto{\pgfqpoint{1.711476in}{2.911400in}}%
\pgfpathlineto{\pgfqpoint{1.726238in}{2.932485in}}%
\pgfpathlineto{\pgfqpoint{1.741000in}{2.952895in}}%
\pgfpathlineto{\pgfqpoint{1.755762in}{2.971551in}}%
\pgfpathlineto{\pgfqpoint{1.770524in}{2.987888in}}%
\pgfpathlineto{\pgfqpoint{1.785286in}{3.006680in}}%
\pgfpathlineto{\pgfqpoint{1.800048in}{3.026680in}}%
\pgfpathlineto{\pgfqpoint{1.814810in}{3.048746in}}%
\pgfpathlineto{\pgfqpoint{1.829572in}{3.072675in}}%
\pgfpathlineto{\pgfqpoint{1.844333in}{3.093598in}}%
\pgfpathlineto{\pgfqpoint{1.859095in}{3.113454in}}%
\pgfpathlineto{\pgfqpoint{1.873857in}{3.133018in}}%
\pgfpathlineto{\pgfqpoint{1.888619in}{3.148914in}}%
\pgfpathlineto{\pgfqpoint{1.903381in}{3.167103in}}%
\pgfpathlineto{\pgfqpoint{1.918143in}{3.187055in}}%
\pgfpathlineto{\pgfqpoint{1.932905in}{3.209492in}}%
\pgfpathlineto{\pgfqpoint{1.947667in}{3.232609in}}%
\pgfpathlineto{\pgfqpoint{1.962429in}{3.252922in}}%
\pgfpathlineto{\pgfqpoint{1.977191in}{3.272361in}}%
\pgfpathlineto{\pgfqpoint{1.991953in}{3.292171in}}%
\pgfpathlineto{\pgfqpoint{2.006714in}{3.308472in}}%
\pgfpathlineto{\pgfqpoint{2.021476in}{3.325817in}}%
\pgfpathlineto{\pgfqpoint{2.036238in}{3.345986in}}%
\pgfpathlineto{\pgfqpoint{2.051000in}{3.368580in}}%
\pgfpathlineto{\pgfqpoint{2.065762in}{3.391434in}}%
\pgfpathlineto{\pgfqpoint{2.080524in}{3.410321in}}%
\pgfpathlineto{\pgfqpoint{2.095286in}{3.431514in}}%
\pgfpathlineto{\pgfqpoint{2.110048in}{3.449317in}}%
\pgfusepath{stroke}%
\end{pgfscope}%
\begin{pgfscope}%
\pgfpathrectangle{\pgfqpoint{0.501000in}{0.566590in}}{\pgfqpoint{3.720000in}{3.020000in}} %
\pgfusepath{clip}%
\pgfsetrectcap%
\pgfsetroundjoin%
\pgfsetlinewidth{1.505625pt}%
\definecolor{currentstroke}{rgb}{0.013725,0.691698,0.927951}%
\pgfsetstrokecolor{currentstroke}%
\pgfsetdash{}{0pt}%
\pgfpathmoveto{\pgfqpoint{0.487111in}{0.979381in}}%
\pgfpathlineto{\pgfqpoint{0.501000in}{0.986006in}}%
\pgfpathlineto{\pgfqpoint{0.515762in}{1.018330in}}%
\pgfpathlineto{\pgfqpoint{0.530524in}{1.045355in}}%
\pgfpathlineto{\pgfqpoint{0.545286in}{1.060987in}}%
\pgfpathlineto{\pgfqpoint{0.560048in}{1.080810in}}%
\pgfpathlineto{\pgfqpoint{0.574810in}{1.105099in}}%
\pgfpathlineto{\pgfqpoint{0.589572in}{1.133063in}}%
\pgfpathlineto{\pgfqpoint{0.604333in}{1.152288in}}%
\pgfpathlineto{\pgfqpoint{0.619095in}{1.168475in}}%
\pgfpathlineto{\pgfqpoint{0.633857in}{1.190321in}}%
\pgfpathlineto{\pgfqpoint{0.648619in}{1.216028in}}%
\pgfpathlineto{\pgfqpoint{0.663381in}{1.235493in}}%
\pgfpathlineto{\pgfqpoint{0.692905in}{1.268933in}}%
\pgfpathlineto{\pgfqpoint{0.707667in}{1.290013in}}%
\pgfpathlineto{\pgfqpoint{0.722429in}{1.314282in}}%
\pgfpathlineto{\pgfqpoint{0.737191in}{1.332092in}}%
\pgfpathlineto{\pgfqpoint{0.751953in}{1.345020in}}%
\pgfpathlineto{\pgfqpoint{0.766714in}{1.364786in}}%
\pgfpathlineto{\pgfqpoint{0.781476in}{1.387358in}}%
\pgfpathlineto{\pgfqpoint{0.796238in}{1.405143in}}%
\pgfpathlineto{\pgfqpoint{0.825762in}{1.434087in}}%
\pgfpathlineto{\pgfqpoint{0.855286in}{1.474191in}}%
\pgfpathlineto{\pgfqpoint{0.870048in}{1.490116in}}%
\pgfpathlineto{\pgfqpoint{0.884810in}{1.501728in}}%
\pgfpathlineto{\pgfqpoint{0.929095in}{1.555884in}}%
\pgfpathlineto{\pgfqpoint{0.958619in}{1.581664in}}%
\pgfpathlineto{\pgfqpoint{0.973381in}{1.598595in}}%
\pgfpathlineto{\pgfqpoint{0.988143in}{1.618034in}}%
\pgfpathlineto{\pgfqpoint{1.032429in}{1.658006in}}%
\pgfpathlineto{\pgfqpoint{1.047191in}{1.676008in}}%
\pgfpathlineto{\pgfqpoint{1.061953in}{1.690595in}}%
\pgfpathlineto{\pgfqpoint{1.091476in}{1.713719in}}%
\pgfpathlineto{\pgfqpoint{1.135762in}{1.758477in}}%
\pgfpathlineto{\pgfqpoint{1.150524in}{1.768174in}}%
\pgfpathlineto{\pgfqpoint{1.194810in}{1.812710in}}%
\pgfpathlineto{\pgfqpoint{1.224333in}{1.833693in}}%
\pgfpathlineto{\pgfqpoint{1.253857in}{1.863761in}}%
\pgfpathlineto{\pgfqpoint{1.283381in}{1.883579in}}%
\pgfpathlineto{\pgfqpoint{1.327667in}{1.923974in}}%
\pgfpathlineto{\pgfqpoint{1.357191in}{1.943216in}}%
\pgfpathlineto{\pgfqpoint{1.386714in}{1.970883in}}%
\pgfpathlineto{\pgfqpoint{1.416238in}{1.989079in}}%
\pgfpathlineto{\pgfqpoint{1.460524in}{2.026049in}}%
\pgfpathlineto{\pgfqpoint{1.490048in}{2.043853in}}%
\pgfpathlineto{\pgfqpoint{1.519572in}{2.069316in}}%
\pgfpathlineto{\pgfqpoint{1.549095in}{2.086283in}}%
\pgfpathlineto{\pgfqpoint{1.593381in}{2.120465in}}%
\pgfpathlineto{\pgfqpoint{1.622905in}{2.136877in}}%
\pgfpathlineto{\pgfqpoint{1.652429in}{2.160377in}}%
\pgfpathlineto{\pgfqpoint{1.681953in}{2.176095in}}%
\pgfpathlineto{\pgfqpoint{1.726238in}{2.207610in}}%
\pgfpathlineto{\pgfqpoint{1.755762in}{2.222785in}}%
\pgfpathlineto{\pgfqpoint{1.785286in}{2.244347in}}%
\pgfpathlineto{\pgfqpoint{1.814810in}{2.259144in}}%
\pgfpathlineto{\pgfqpoint{1.859095in}{2.288288in}}%
\pgfpathlineto{\pgfqpoint{1.888619in}{2.302669in}}%
\pgfpathlineto{\pgfqpoint{1.918143in}{2.322160in}}%
\pgfpathlineto{\pgfqpoint{1.947667in}{2.336288in}}%
\pgfpathlineto{\pgfqpoint{1.991953in}{2.362964in}}%
\pgfpathlineto{\pgfqpoint{2.006714in}{2.369039in}}%
\pgfpathlineto{\pgfqpoint{2.036238in}{2.386228in}}%
\pgfpathlineto{\pgfqpoint{2.051000in}{2.394527in}}%
\pgfpathlineto{\pgfqpoint{2.080524in}{2.407860in}}%
\pgfpathlineto{\pgfqpoint{2.124810in}{2.432327in}}%
\pgfpathlineto{\pgfqpoint{2.154333in}{2.445373in}}%
\pgfpathlineto{\pgfqpoint{2.183857in}{2.461965in}}%
\pgfpathlineto{\pgfqpoint{2.213381in}{2.474317in}}%
\pgfpathlineto{\pgfqpoint{2.257667in}{2.497176in}}%
\pgfpathlineto{\pgfqpoint{2.287191in}{2.509391in}}%
\pgfpathlineto{\pgfqpoint{2.316714in}{2.524978in}}%
\pgfpathlineto{\pgfqpoint{2.346238in}{2.536699in}}%
\pgfpathlineto{\pgfqpoint{2.390524in}{2.557787in}}%
\pgfpathlineto{\pgfqpoint{2.420048in}{2.569541in}}%
\pgfpathlineto{\pgfqpoint{2.449572in}{2.584141in}}%
\pgfpathlineto{\pgfqpoint{2.479095in}{2.595001in}}%
\pgfpathlineto{\pgfqpoint{2.508619in}{2.609226in}}%
\pgfpathlineto{\pgfqpoint{2.552905in}{2.625835in}}%
\pgfpathlineto{\pgfqpoint{2.582429in}{2.639233in}}%
\pgfpathlineto{\pgfqpoint{2.611953in}{2.649410in}}%
\pgfpathlineto{\pgfqpoint{2.641476in}{2.662621in}}%
\pgfpathlineto{\pgfqpoint{2.685762in}{2.678389in}}%
\pgfpathlineto{\pgfqpoint{2.715286in}{2.690555in}}%
\pgfpathlineto{\pgfqpoint{2.744810in}{2.700127in}}%
\pgfpathlineto{\pgfqpoint{2.774333in}{2.712396in}}%
\pgfpathlineto{\pgfqpoint{2.803857in}{2.721179in}}%
\pgfpathlineto{\pgfqpoint{2.848143in}{2.738669in}}%
\pgfpathlineto{\pgfqpoint{2.877667in}{2.747785in}}%
\pgfpathlineto{\pgfqpoint{2.907191in}{2.758966in}}%
\pgfpathlineto{\pgfqpoint{2.936714in}{2.767464in}}%
\pgfpathlineto{\pgfqpoint{2.966238in}{2.779123in}}%
\pgfpathlineto{\pgfqpoint{3.069572in}{2.810585in}}%
\pgfpathlineto{\pgfqpoint{3.099095in}{2.821457in}}%
\pgfpathlineto{\pgfqpoint{3.158143in}{2.838513in}}%
\pgfpathlineto{\pgfqpoint{3.246714in}{2.864446in}}%
\pgfpathlineto{\pgfqpoint{3.276238in}{2.871767in}}%
\pgfpathlineto{\pgfqpoint{3.305762in}{2.880454in}}%
\pgfpathlineto{\pgfqpoint{3.335286in}{2.888133in}}%
\pgfpathlineto{\pgfqpoint{3.364810in}{2.897083in}}%
\pgfpathlineto{\pgfqpoint{3.468143in}{2.923006in}}%
\pgfpathlineto{\pgfqpoint{3.497667in}{2.931099in}}%
\pgfpathlineto{\pgfqpoint{3.541953in}{2.941220in}}%
\pgfpathlineto{\pgfqpoint{3.615762in}{2.959134in}}%
\pgfpathlineto{\pgfqpoint{3.704333in}{2.978463in}}%
\pgfpathlineto{\pgfqpoint{3.778143in}{2.993916in}}%
\pgfpathlineto{\pgfqpoint{3.778143in}{2.993916in}}%
\pgfusepath{stroke}%
\end{pgfscope}%
\begin{pgfscope}%
\pgfpathrectangle{\pgfqpoint{0.501000in}{0.566590in}}{\pgfqpoint{3.720000in}{3.020000in}} %
\pgfusepath{clip}%
\pgfsetrectcap%
\pgfsetroundjoin%
\pgfsetlinewidth{1.505625pt}%
\definecolor{currentstroke}{rgb}{0.374510,0.195845,0.995147}%
\pgfsetstrokecolor{currentstroke}%
\pgfsetdash{}{0pt}%
\pgfpathmoveto{\pgfqpoint{0.487111in}{0.982800in}}%
\pgfpathlineto{\pgfqpoint{0.501000in}{1.008468in}}%
\pgfpathlineto{\pgfqpoint{0.545286in}{1.072766in}}%
\pgfpathlineto{\pgfqpoint{0.574810in}{1.125558in}}%
\pgfpathlineto{\pgfqpoint{0.589572in}{1.143479in}}%
\pgfpathlineto{\pgfqpoint{0.604333in}{1.164436in}}%
\pgfpathlineto{\pgfqpoint{0.633857in}{1.215673in}}%
\pgfpathlineto{\pgfqpoint{0.663381in}{1.253527in}}%
\pgfpathlineto{\pgfqpoint{0.678143in}{1.275531in}}%
\pgfpathlineto{\pgfqpoint{0.692905in}{1.301573in}}%
\pgfpathlineto{\pgfqpoint{0.707667in}{1.323177in}}%
\pgfpathlineto{\pgfqpoint{0.722429in}{1.340508in}}%
\pgfpathlineto{\pgfqpoint{0.737191in}{1.361021in}}%
\pgfpathlineto{\pgfqpoint{0.766714in}{1.407964in}}%
\pgfpathlineto{\pgfqpoint{0.796238in}{1.444001in}}%
\pgfpathlineto{\pgfqpoint{0.840524in}{1.508850in}}%
\pgfpathlineto{\pgfqpoint{0.855286in}{1.525728in}}%
\pgfpathlineto{\pgfqpoint{0.914333in}{1.606001in}}%
\pgfpathlineto{\pgfqpoint{0.929095in}{1.624131in}}%
\pgfpathlineto{\pgfqpoint{0.958619in}{1.666897in}}%
\pgfpathlineto{\pgfqpoint{0.988143in}{1.701131in}}%
\pgfpathlineto{\pgfqpoint{1.032429in}{1.761374in}}%
\pgfpathlineto{\pgfqpoint{1.061953in}{1.795948in}}%
\pgfpathlineto{\pgfqpoint{1.091476in}{1.835967in}}%
\pgfpathlineto{\pgfqpoint{1.121000in}{1.869410in}}%
\pgfpathlineto{\pgfqpoint{1.150524in}{1.909393in}}%
\pgfpathlineto{\pgfqpoint{1.194810in}{1.960658in}}%
\pgfpathlineto{\pgfqpoint{1.224333in}{1.998255in}}%
\pgfpathlineto{\pgfqpoint{1.253857in}{2.031768in}}%
\pgfpathlineto{\pgfqpoint{1.283381in}{2.069315in}}%
\pgfpathlineto{\pgfqpoint{1.327667in}{2.119983in}}%
\pgfpathlineto{\pgfqpoint{1.357191in}{2.154947in}}%
\pgfpathlineto{\pgfqpoint{1.386714in}{2.188518in}}%
\pgfpathlineto{\pgfqpoint{1.416238in}{2.224273in}}%
\pgfpathlineto{\pgfqpoint{1.445762in}{2.256415in}}%
\pgfpathlineto{\pgfqpoint{1.475286in}{2.291984in}}%
\pgfpathlineto{\pgfqpoint{1.504810in}{2.323604in}}%
\pgfpathlineto{\pgfqpoint{1.549095in}{2.375305in}}%
\pgfpathlineto{\pgfqpoint{1.578619in}{2.407382in}}%
\pgfpathlineto{\pgfqpoint{1.608143in}{2.441462in}}%
\pgfpathlineto{\pgfqpoint{1.637667in}{2.472970in}}%
\pgfpathlineto{\pgfqpoint{1.681953in}{2.523057in}}%
\pgfpathlineto{\pgfqpoint{1.726238in}{2.572077in}}%
\pgfpathlineto{\pgfqpoint{2.124810in}{2.998282in}}%
\pgfpathlineto{\pgfqpoint{2.198619in}{3.075163in}}%
\pgfpathlineto{\pgfqpoint{2.464333in}{3.347645in}}%
\pgfpathlineto{\pgfqpoint{2.464333in}{3.347645in}}%
\pgfusepath{stroke}%
\end{pgfscope}%
\begin{pgfscope}%
\pgfpathrectangle{\pgfqpoint{0.501000in}{0.566590in}}{\pgfqpoint{3.720000in}{3.020000in}} %
\pgfusepath{clip}%
\pgfsetrectcap%
\pgfsetroundjoin%
\pgfsetlinewidth{1.505625pt}%
\definecolor{currentstroke}{rgb}{0.080392,0.790532,0.897892}%
\pgfsetstrokecolor{currentstroke}%
\pgfsetdash{}{0pt}%
\pgfpathmoveto{\pgfqpoint{0.487111in}{0.964344in}}%
\pgfpathlineto{\pgfqpoint{0.501000in}{0.983131in}}%
\pgfpathlineto{\pgfqpoint{0.515762in}{0.991148in}}%
\pgfpathlineto{\pgfqpoint{0.545286in}{1.045787in}}%
\pgfpathlineto{\pgfqpoint{0.560048in}{1.062268in}}%
\pgfpathlineto{\pgfqpoint{0.574810in}{1.076357in}}%
\pgfpathlineto{\pgfqpoint{0.604333in}{1.120632in}}%
\pgfpathlineto{\pgfqpoint{0.648619in}{1.169857in}}%
\pgfpathlineto{\pgfqpoint{0.663381in}{1.190919in}}%
\pgfpathlineto{\pgfqpoint{0.707667in}{1.239050in}}%
\pgfpathlineto{\pgfqpoint{0.722429in}{1.256579in}}%
\pgfpathlineto{\pgfqpoint{0.751953in}{1.283720in}}%
\pgfpathlineto{\pgfqpoint{0.781476in}{1.317965in}}%
\pgfpathlineto{\pgfqpoint{0.811000in}{1.343594in}}%
\pgfpathlineto{\pgfqpoint{0.840524in}{1.375086in}}%
\pgfpathlineto{\pgfqpoint{0.870048in}{1.398908in}}%
\pgfpathlineto{\pgfqpoint{0.899572in}{1.428454in}}%
\pgfpathlineto{\pgfqpoint{0.914333in}{1.438320in}}%
\pgfpathlineto{\pgfqpoint{0.958619in}{1.478013in}}%
\pgfpathlineto{\pgfqpoint{0.988143in}{1.499694in}}%
\pgfpathlineto{\pgfqpoint{1.002905in}{1.513732in}}%
\pgfpathlineto{\pgfqpoint{1.047191in}{1.545648in}}%
\pgfpathlineto{\pgfqpoint{1.061953in}{1.557969in}}%
\pgfpathlineto{\pgfqpoint{1.106238in}{1.587998in}}%
\pgfpathlineto{\pgfqpoint{1.121000in}{1.599418in}}%
\pgfpathlineto{\pgfqpoint{1.150524in}{1.616834in}}%
\pgfpathlineto{\pgfqpoint{1.180048in}{1.637659in}}%
\pgfpathlineto{\pgfqpoint{1.209572in}{1.653947in}}%
\pgfpathlineto{\pgfqpoint{1.239095in}{1.673428in}}%
\pgfpathlineto{\pgfqpoint{1.253857in}{1.680363in}}%
\pgfpathlineto{\pgfqpoint{1.298143in}{1.706686in}}%
\pgfpathlineto{\pgfqpoint{1.327667in}{1.721458in}}%
\pgfpathlineto{\pgfqpoint{1.342429in}{1.730556in}}%
\pgfpathlineto{\pgfqpoint{1.386714in}{1.751647in}}%
\pgfpathlineto{\pgfqpoint{1.416238in}{1.765744in}}%
\pgfpathlineto{\pgfqpoint{1.431000in}{1.771636in}}%
\pgfpathlineto{\pgfqpoint{1.460524in}{1.786412in}}%
\pgfpathlineto{\pgfqpoint{1.490048in}{1.797562in}}%
\pgfpathlineto{\pgfqpoint{1.519572in}{1.810903in}}%
\pgfpathlineto{\pgfqpoint{1.549095in}{1.821306in}}%
\pgfpathlineto{\pgfqpoint{1.578619in}{1.833181in}}%
\pgfpathlineto{\pgfqpoint{1.608143in}{1.843007in}}%
\pgfpathlineto{\pgfqpoint{1.637667in}{1.853648in}}%
\pgfpathlineto{\pgfqpoint{1.667191in}{1.862598in}}%
\pgfpathlineto{\pgfqpoint{1.696714in}{1.871652in}}%
\pgfpathlineto{\pgfqpoint{1.726238in}{1.880143in}}%
\pgfpathlineto{\pgfqpoint{1.755762in}{1.888206in}}%
\pgfpathlineto{\pgfqpoint{1.785286in}{1.895819in}}%
\pgfpathlineto{\pgfqpoint{1.844333in}{1.909287in}}%
\pgfpathlineto{\pgfqpoint{1.903381in}{1.920977in}}%
\pgfpathlineto{\pgfqpoint{1.977191in}{1.932716in}}%
\pgfpathlineto{\pgfqpoint{2.006714in}{1.936378in}}%
\pgfpathlineto{\pgfqpoint{2.036238in}{1.939832in}}%
\pgfpathlineto{\pgfqpoint{2.154333in}{1.948992in}}%
\pgfpathlineto{\pgfqpoint{2.242905in}{1.951525in}}%
\pgfpathlineto{\pgfqpoint{2.346238in}{1.948723in}}%
\pgfpathlineto{\pgfqpoint{2.390524in}{1.945470in}}%
\pgfpathlineto{\pgfqpoint{2.434810in}{1.941736in}}%
\pgfpathlineto{\pgfqpoint{2.538143in}{1.928816in}}%
\pgfpathlineto{\pgfqpoint{2.685762in}{1.899463in}}%
\pgfpathlineto{\pgfqpoint{2.744810in}{1.884189in}}%
\pgfpathlineto{\pgfqpoint{2.803857in}{1.866764in}}%
\pgfpathlineto{\pgfqpoint{2.877667in}{1.841532in}}%
\pgfpathlineto{\pgfqpoint{2.966238in}{1.806428in}}%
\pgfpathlineto{\pgfqpoint{3.025286in}{1.779864in}}%
\pgfpathlineto{\pgfqpoint{3.084333in}{1.750679in}}%
\pgfpathlineto{\pgfqpoint{3.158143in}{1.710212in}}%
\pgfpathlineto{\pgfqpoint{3.217191in}{1.674871in}}%
\pgfpathlineto{\pgfqpoint{3.305762in}{1.616529in}}%
\pgfpathlineto{\pgfqpoint{3.379572in}{1.562734in}}%
\pgfpathlineto{\pgfqpoint{3.453381in}{1.504184in}}%
\pgfpathlineto{\pgfqpoint{3.527191in}{1.440511in}}%
\pgfpathlineto{\pgfqpoint{3.586238in}{1.385666in}}%
\pgfpathlineto{\pgfqpoint{3.645286in}{1.327178in}}%
\pgfpathlineto{\pgfqpoint{3.704333in}{1.264811in}}%
\pgfpathlineto{\pgfqpoint{3.763381in}{1.197985in}}%
\pgfpathlineto{\pgfqpoint{3.763381in}{1.197985in}}%
\pgfusepath{stroke}%
\end{pgfscope}%
\begin{pgfscope}%
\pgfpathrectangle{\pgfqpoint{0.501000in}{0.566590in}}{\pgfqpoint{3.720000in}{3.020000in}} %
\pgfusepath{clip}%
\pgfsetrectcap%
\pgfsetroundjoin%
\pgfsetlinewidth{1.505625pt}%
\definecolor{currentstroke}{rgb}{0.323529,0.961826,0.798017}%
\pgfsetstrokecolor{currentstroke}%
\pgfsetdash{}{0pt}%
\pgfpathmoveto{\pgfqpoint{0.487111in}{0.992706in}}%
\pgfpathlineto{\pgfqpoint{0.501000in}{1.017999in}}%
\pgfpathlineto{\pgfqpoint{0.515762in}{1.036169in}}%
\pgfpathlineto{\pgfqpoint{0.530524in}{1.060186in}}%
\pgfpathlineto{\pgfqpoint{0.545286in}{1.077204in}}%
\pgfpathlineto{\pgfqpoint{0.560048in}{1.090859in}}%
\pgfpathlineto{\pgfqpoint{0.574810in}{1.109864in}}%
\pgfpathlineto{\pgfqpoint{0.589572in}{1.133990in}}%
\pgfpathlineto{\pgfqpoint{0.604333in}{1.153552in}}%
\pgfpathlineto{\pgfqpoint{0.633857in}{1.180233in}}%
\pgfpathlineto{\pgfqpoint{0.663381in}{1.219343in}}%
\pgfpathlineto{\pgfqpoint{0.678143in}{1.231969in}}%
\pgfpathlineto{\pgfqpoint{0.707667in}{1.253028in}}%
\pgfpathlineto{\pgfqpoint{0.737191in}{1.287429in}}%
\pgfpathlineto{\pgfqpoint{0.751953in}{1.297399in}}%
\pgfpathlineto{\pgfqpoint{0.766714in}{1.304472in}}%
\pgfpathlineto{\pgfqpoint{0.781476in}{1.316777in}}%
\pgfpathlineto{\pgfqpoint{0.796238in}{1.332119in}}%
\pgfpathlineto{\pgfqpoint{0.811000in}{1.343503in}}%
\pgfpathlineto{\pgfqpoint{0.840524in}{1.355887in}}%
\pgfpathlineto{\pgfqpoint{0.855286in}{1.366656in}}%
\pgfpathlineto{\pgfqpoint{0.870048in}{1.379668in}}%
\pgfpathlineto{\pgfqpoint{0.884810in}{1.388684in}}%
\pgfpathlineto{\pgfqpoint{0.899572in}{1.391548in}}%
\pgfpathlineto{\pgfqpoint{0.914333in}{1.398427in}}%
\pgfpathlineto{\pgfqpoint{0.929095in}{1.409215in}}%
\pgfpathlineto{\pgfqpoint{0.943857in}{1.417855in}}%
\pgfpathlineto{\pgfqpoint{0.958619in}{1.422258in}}%
\pgfpathlineto{\pgfqpoint{0.973381in}{1.424307in}}%
\pgfpathlineto{\pgfqpoint{0.988143in}{1.429401in}}%
\pgfpathlineto{\pgfqpoint{1.017667in}{1.444253in}}%
\pgfpathlineto{\pgfqpoint{1.047191in}{1.446493in}}%
\pgfpathlineto{\pgfqpoint{1.076714in}{1.458060in}}%
\pgfpathlineto{\pgfqpoint{1.091476in}{1.460291in}}%
\pgfpathlineto{\pgfqpoint{1.121000in}{1.459490in}}%
\pgfpathlineto{\pgfqpoint{1.150524in}{1.467128in}}%
\pgfpathlineto{\pgfqpoint{1.180048in}{1.463304in}}%
\pgfpathlineto{\pgfqpoint{1.224333in}{1.466421in}}%
\pgfpathlineto{\pgfqpoint{1.253857in}{1.459061in}}%
\pgfpathlineto{\pgfqpoint{1.268619in}{1.458002in}}%
\pgfpathlineto{\pgfqpoint{1.283381in}{1.458701in}}%
\pgfpathlineto{\pgfqpoint{1.298143in}{1.455001in}}%
\pgfpathlineto{\pgfqpoint{1.312905in}{1.448913in}}%
\pgfpathlineto{\pgfqpoint{1.327667in}{1.445356in}}%
\pgfpathlineto{\pgfqpoint{1.342429in}{1.443961in}}%
\pgfpathlineto{\pgfqpoint{1.357191in}{1.440947in}}%
\pgfpathlineto{\pgfqpoint{1.401476in}{1.421635in}}%
\pgfpathlineto{\pgfqpoint{1.416238in}{1.418394in}}%
\pgfpathlineto{\pgfqpoint{1.431000in}{1.412224in}}%
\pgfpathlineto{\pgfqpoint{1.460524in}{1.395183in}}%
\pgfpathlineto{\pgfqpoint{1.490048in}{1.383077in}}%
\pgfpathlineto{\pgfqpoint{1.519572in}{1.363001in}}%
\pgfpathlineto{\pgfqpoint{1.534333in}{1.352953in}}%
\pgfpathlineto{\pgfqpoint{1.549095in}{1.345628in}}%
\pgfpathlineto{\pgfqpoint{1.563857in}{1.336351in}}%
\pgfpathlineto{\pgfqpoint{1.593381in}{1.312045in}}%
\pgfpathlineto{\pgfqpoint{1.637667in}{1.278910in}}%
\pgfpathlineto{\pgfqpoint{1.667191in}{1.249516in}}%
\pgfpathlineto{\pgfqpoint{1.696714in}{1.223906in}}%
\pgfpathlineto{\pgfqpoint{1.785286in}{1.125098in}}%
\pgfpathlineto{\pgfqpoint{1.814810in}{1.087668in}}%
\pgfpathlineto{\pgfqpoint{1.829572in}{1.069469in}}%
\pgfpathlineto{\pgfqpoint{1.888619in}{0.984492in}}%
\pgfpathlineto{\pgfqpoint{1.918143in}{0.936806in}}%
\pgfpathlineto{\pgfqpoint{1.977191in}{0.832593in}}%
\pgfpathlineto{\pgfqpoint{1.991953in}{0.803748in}}%
\pgfpathlineto{\pgfqpoint{2.006714in}{0.784739in}}%
\pgfpathlineto{\pgfqpoint{2.036238in}{0.751317in}}%
\pgfpathlineto{\pgfqpoint{2.051000in}{0.745048in}}%
\pgfpathlineto{\pgfqpoint{2.065762in}{0.749573in}}%
\pgfpathlineto{\pgfqpoint{2.080524in}{0.758701in}}%
\pgfpathlineto{\pgfqpoint{2.095286in}{0.774875in}}%
\pgfpathlineto{\pgfqpoint{2.110048in}{0.797098in}}%
\pgfpathlineto{\pgfqpoint{2.124810in}{0.815800in}}%
\pgfpathlineto{\pgfqpoint{2.139572in}{0.845448in}}%
\pgfpathlineto{\pgfqpoint{2.169095in}{0.889823in}}%
\pgfpathlineto{\pgfqpoint{2.228143in}{0.956866in}}%
\pgfpathlineto{\pgfqpoint{2.257667in}{0.984631in}}%
\pgfpathlineto{\pgfqpoint{2.287191in}{1.011463in}}%
\pgfpathlineto{\pgfqpoint{2.301953in}{1.024877in}}%
\pgfpathlineto{\pgfqpoint{2.361000in}{1.063750in}}%
\pgfpathlineto{\pgfqpoint{2.375762in}{1.071895in}}%
\pgfpathlineto{\pgfqpoint{2.420048in}{1.091987in}}%
\pgfpathlineto{\pgfqpoint{2.434810in}{1.099063in}}%
\pgfpathlineto{\pgfqpoint{2.464333in}{1.106572in}}%
\pgfpathlineto{\pgfqpoint{2.479095in}{1.109338in}}%
\pgfpathlineto{\pgfqpoint{2.508619in}{1.116931in}}%
\pgfpathlineto{\pgfqpoint{2.582429in}{1.119872in}}%
\pgfpathlineto{\pgfqpoint{2.611953in}{1.114382in}}%
\pgfpathlineto{\pgfqpoint{2.641476in}{1.110337in}}%
\pgfpathlineto{\pgfqpoint{2.715286in}{1.084457in}}%
\pgfpathlineto{\pgfqpoint{2.774333in}{1.051216in}}%
\pgfpathlineto{\pgfqpoint{2.818619in}{1.017586in}}%
\pgfpathlineto{\pgfqpoint{2.848143in}{0.993503in}}%
\pgfpathlineto{\pgfqpoint{2.892429in}{0.947319in}}%
\pgfpathlineto{\pgfqpoint{2.921953in}{0.913050in}}%
\pgfpathlineto{\pgfqpoint{2.966238in}{0.851109in}}%
\pgfpathlineto{\pgfqpoint{2.995762in}{0.803736in}}%
\pgfpathlineto{\pgfqpoint{3.010524in}{0.782210in}}%
\pgfpathlineto{\pgfqpoint{3.025286in}{0.764813in}}%
\pgfpathlineto{\pgfqpoint{3.040048in}{0.751111in}}%
\pgfpathlineto{\pgfqpoint{3.054810in}{0.750314in}}%
\pgfpathlineto{\pgfqpoint{3.069572in}{0.758330in}}%
\pgfpathlineto{\pgfqpoint{3.084333in}{0.774908in}}%
\pgfpathlineto{\pgfqpoint{3.143381in}{0.857486in}}%
\pgfpathlineto{\pgfqpoint{3.158143in}{0.880507in}}%
\pgfpathlineto{\pgfqpoint{3.172905in}{0.899243in}}%
\pgfpathlineto{\pgfqpoint{3.202429in}{0.929419in}}%
\pgfpathlineto{\pgfqpoint{3.231953in}{0.954091in}}%
\pgfpathlineto{\pgfqpoint{3.261476in}{0.977379in}}%
\pgfpathlineto{\pgfqpoint{3.320524in}{1.011058in}}%
\pgfpathlineto{\pgfqpoint{3.335286in}{1.016483in}}%
\pgfpathlineto{\pgfqpoint{3.364810in}{1.024848in}}%
\pgfpathlineto{\pgfqpoint{3.379572in}{1.030297in}}%
\pgfpathlineto{\pgfqpoint{3.394333in}{1.033737in}}%
\pgfpathlineto{\pgfqpoint{3.453381in}{1.038750in}}%
\pgfpathlineto{\pgfqpoint{3.482905in}{1.034930in}}%
\pgfpathlineto{\pgfqpoint{3.541953in}{1.023060in}}%
\pgfpathlineto{\pgfqpoint{3.601000in}{0.997864in}}%
\pgfpathlineto{\pgfqpoint{3.645286in}{0.969388in}}%
\pgfpathlineto{\pgfqpoint{3.674810in}{0.947399in}}%
\pgfpathlineto{\pgfqpoint{3.719095in}{0.904803in}}%
\pgfpathlineto{\pgfqpoint{3.733857in}{0.889412in}}%
\pgfpathlineto{\pgfqpoint{3.733857in}{0.889412in}}%
\pgfusepath{stroke}%
\end{pgfscope}%
\begin{pgfscope}%
\pgfpathrectangle{\pgfqpoint{0.501000in}{0.566590in}}{\pgfqpoint{3.720000in}{3.020000in}} %
\pgfusepath{clip}%
\pgfsetrectcap%
\pgfsetroundjoin%
\pgfsetlinewidth{1.505625pt}%
\definecolor{currentstroke}{rgb}{1.000000,0.000000,0.000000}%
\pgfsetstrokecolor{currentstroke}%
\pgfsetdash{}{0pt}%
\pgfpathmoveto{\pgfqpoint{0.487111in}{0.800061in}}%
\pgfpathlineto{\pgfqpoint{0.501000in}{0.803724in}}%
\pgfpathlineto{\pgfqpoint{0.530524in}{0.813540in}}%
\pgfpathlineto{\pgfqpoint{0.560048in}{0.819223in}}%
\pgfpathlineto{\pgfqpoint{0.574810in}{0.823893in}}%
\pgfpathlineto{\pgfqpoint{0.589572in}{0.823453in}}%
\pgfpathlineto{\pgfqpoint{0.619095in}{0.827552in}}%
\pgfpathlineto{\pgfqpoint{0.648619in}{0.828042in}}%
\pgfpathlineto{\pgfqpoint{0.678143in}{0.826767in}}%
\pgfpathlineto{\pgfqpoint{0.707667in}{0.824241in}}%
\pgfpathlineto{\pgfqpoint{0.737191in}{0.819046in}}%
\pgfpathlineto{\pgfqpoint{0.781476in}{0.807804in}}%
\pgfpathlineto{\pgfqpoint{0.811000in}{0.798106in}}%
\pgfpathlineto{\pgfqpoint{0.855286in}{0.778974in}}%
\pgfpathlineto{\pgfqpoint{0.884810in}{0.761574in}}%
\pgfpathlineto{\pgfqpoint{0.899572in}{0.752119in}}%
\pgfpathlineto{\pgfqpoint{0.943857in}{0.712811in}}%
\pgfpathlineto{\pgfqpoint{0.958619in}{0.705759in}}%
\pgfpathlineto{\pgfqpoint{0.973381in}{0.708486in}}%
\pgfpathlineto{\pgfqpoint{1.002905in}{0.730583in}}%
\pgfpathlineto{\pgfqpoint{1.017667in}{0.741278in}}%
\pgfpathlineto{\pgfqpoint{1.032429in}{0.748566in}}%
\pgfpathlineto{\pgfqpoint{1.076714in}{0.763004in}}%
\pgfpathlineto{\pgfqpoint{1.106238in}{0.766303in}}%
\pgfpathlineto{\pgfqpoint{1.135762in}{0.764285in}}%
\pgfpathlineto{\pgfqpoint{1.165286in}{0.758076in}}%
\pgfpathlineto{\pgfqpoint{1.180048in}{0.753404in}}%
\pgfpathlineto{\pgfqpoint{1.209572in}{0.738365in}}%
\pgfpathlineto{\pgfqpoint{1.224333in}{0.728926in}}%
\pgfpathlineto{\pgfqpoint{1.253857in}{0.707436in}}%
\pgfpathlineto{\pgfqpoint{1.268619in}{0.704707in}}%
\pgfpathlineto{\pgfqpoint{1.283381in}{0.709282in}}%
\pgfpathlineto{\pgfqpoint{1.312905in}{0.727872in}}%
\pgfpathlineto{\pgfqpoint{1.327667in}{0.735487in}}%
\pgfpathlineto{\pgfqpoint{1.357191in}{0.743595in}}%
\pgfpathlineto{\pgfqpoint{1.371953in}{0.746996in}}%
\pgfpathlineto{\pgfqpoint{1.416238in}{0.744862in}}%
\pgfpathlineto{\pgfqpoint{1.445762in}{0.736147in}}%
\pgfpathlineto{\pgfqpoint{1.460524in}{0.729364in}}%
\pgfpathlineto{\pgfqpoint{1.490048in}{0.711687in}}%
\pgfpathlineto{\pgfqpoint{1.504810in}{0.705078in}}%
\pgfpathlineto{\pgfqpoint{1.519572in}{0.705329in}}%
\pgfpathlineto{\pgfqpoint{1.534333in}{0.710584in}}%
\pgfpathlineto{\pgfqpoint{1.549095in}{0.719492in}}%
\pgfpathlineto{\pgfqpoint{1.578619in}{0.730660in}}%
\pgfpathlineto{\pgfqpoint{1.593381in}{0.732710in}}%
\pgfpathlineto{\pgfqpoint{1.608143in}{0.732615in}}%
\pgfpathlineto{\pgfqpoint{1.622905in}{0.730844in}}%
\pgfpathlineto{\pgfqpoint{1.637667in}{0.727107in}}%
\pgfpathlineto{\pgfqpoint{1.667191in}{0.713321in}}%
\pgfpathlineto{\pgfqpoint{1.681953in}{0.706278in}}%
\pgfpathlineto{\pgfqpoint{1.696714in}{0.703863in}}%
\pgfpathlineto{\pgfqpoint{1.711476in}{0.707417in}}%
\pgfpathlineto{\pgfqpoint{1.741000in}{0.721095in}}%
\pgfpathlineto{\pgfqpoint{1.755762in}{0.724726in}}%
\pgfpathlineto{\pgfqpoint{1.770524in}{0.724968in}}%
\pgfpathlineto{\pgfqpoint{1.785286in}{0.722347in}}%
\pgfpathlineto{\pgfqpoint{1.829572in}{0.708157in}}%
\pgfpathlineto{\pgfqpoint{1.844333in}{0.707061in}}%
\pgfpathlineto{\pgfqpoint{1.859095in}{0.708903in}}%
\pgfpathlineto{\pgfqpoint{1.888619in}{0.718216in}}%
\pgfpathlineto{\pgfqpoint{1.903381in}{0.720836in}}%
\pgfpathlineto{\pgfqpoint{1.918143in}{0.720176in}}%
\pgfpathlineto{\pgfqpoint{1.962429in}{0.707044in}}%
\pgfpathlineto{\pgfqpoint{1.977191in}{0.706813in}}%
\pgfpathlineto{\pgfqpoint{2.006714in}{0.714311in}}%
\pgfpathlineto{\pgfqpoint{2.021476in}{0.717941in}}%
\pgfpathlineto{\pgfqpoint{2.036238in}{0.719282in}}%
\pgfpathlineto{\pgfqpoint{2.051000in}{0.717597in}}%
\pgfpathlineto{\pgfqpoint{2.095286in}{0.708797in}}%
\pgfpathlineto{\pgfqpoint{2.110048in}{0.709144in}}%
\pgfpathlineto{\pgfqpoint{2.169095in}{0.717378in}}%
\pgfpathlineto{\pgfqpoint{2.183857in}{0.713621in}}%
\pgfpathlineto{\pgfqpoint{2.228143in}{0.710358in}}%
\pgfpathlineto{\pgfqpoint{2.287191in}{0.716744in}}%
\pgfpathlineto{\pgfqpoint{2.316714in}{0.714408in}}%
\pgfpathlineto{\pgfqpoint{2.346238in}{0.711078in}}%
\pgfpathlineto{\pgfqpoint{2.375762in}{0.712004in}}%
\pgfpathlineto{\pgfqpoint{2.420048in}{0.716370in}}%
\pgfpathlineto{\pgfqpoint{2.449572in}{0.714887in}}%
\pgfpathlineto{\pgfqpoint{2.493857in}{0.711824in}}%
\pgfpathlineto{\pgfqpoint{2.567667in}{0.714722in}}%
\pgfpathlineto{\pgfqpoint{2.611953in}{0.711727in}}%
\pgfpathlineto{\pgfqpoint{2.656238in}{0.714342in}}%
\pgfpathlineto{\pgfqpoint{2.685762in}{0.714566in}}%
\pgfpathlineto{\pgfqpoint{2.730048in}{0.711484in}}%
\pgfpathlineto{\pgfqpoint{2.774333in}{0.713862in}}%
\pgfpathlineto{\pgfqpoint{2.803857in}{0.714902in}}%
\pgfpathlineto{\pgfqpoint{2.877667in}{0.712167in}}%
\pgfpathlineto{\pgfqpoint{2.936714in}{0.714558in}}%
\pgfpathlineto{\pgfqpoint{3.010524in}{0.712289in}}%
\pgfpathlineto{\pgfqpoint{3.069572in}{0.713467in}}%
\pgfpathlineto{\pgfqpoint{3.128619in}{0.712337in}}%
\pgfpathlineto{\pgfqpoint{3.187667in}{0.712530in}}%
\pgfpathlineto{\pgfqpoint{3.231953in}{0.711627in}}%
\pgfpathlineto{\pgfqpoint{3.305762in}{0.712417in}}%
\pgfpathlineto{\pgfqpoint{3.364810in}{0.712081in}}%
\pgfpathlineto{\pgfqpoint{3.423857in}{0.712607in}}%
\pgfpathlineto{\pgfqpoint{3.497667in}{0.712559in}}%
\pgfpathlineto{\pgfqpoint{3.541953in}{0.712352in}}%
\pgfpathlineto{\pgfqpoint{3.601000in}{0.712238in}}%
\pgfpathlineto{\pgfqpoint{3.660048in}{0.712759in}}%
\pgfpathlineto{\pgfqpoint{3.733857in}{0.712634in}}%
\pgfpathlineto{\pgfqpoint{3.748619in}{0.712974in}}%
\pgfpathlineto{\pgfqpoint{3.748619in}{0.712974in}}%
\pgfusepath{stroke}%
\end{pgfscope}%
\begin{pgfscope}%
\pgfpathrectangle{\pgfqpoint{0.501000in}{0.566590in}}{\pgfqpoint{3.720000in}{3.020000in}} %
\pgfusepath{clip}%
\pgfsetrectcap%
\pgfsetroundjoin%
\pgfsetlinewidth{1.505625pt}%
\definecolor{currentstroke}{rgb}{0.621569,0.981823,0.636474}%
\pgfsetstrokecolor{currentstroke}%
\pgfsetdash{}{0pt}%
\pgfpathmoveto{\pgfqpoint{0.487111in}{0.914199in}}%
\pgfpathlineto{\pgfqpoint{0.501000in}{0.934123in}}%
\pgfpathlineto{\pgfqpoint{0.515762in}{0.950220in}}%
\pgfpathlineto{\pgfqpoint{0.530524in}{0.961286in}}%
\pgfpathlineto{\pgfqpoint{0.560048in}{0.996134in}}%
\pgfpathlineto{\pgfqpoint{0.589572in}{1.016963in}}%
\pgfpathlineto{\pgfqpoint{0.604333in}{1.032217in}}%
\pgfpathlineto{\pgfqpoint{0.619095in}{1.045070in}}%
\pgfpathlineto{\pgfqpoint{0.648619in}{1.059524in}}%
\pgfpathlineto{\pgfqpoint{0.663381in}{1.071974in}}%
\pgfpathlineto{\pgfqpoint{0.678143in}{1.080393in}}%
\pgfpathlineto{\pgfqpoint{0.692905in}{1.084099in}}%
\pgfpathlineto{\pgfqpoint{0.722429in}{1.101309in}}%
\pgfpathlineto{\pgfqpoint{0.737191in}{1.105192in}}%
\pgfpathlineto{\pgfqpoint{0.751953in}{1.106314in}}%
\pgfpathlineto{\pgfqpoint{0.781476in}{1.118356in}}%
\pgfpathlineto{\pgfqpoint{0.811000in}{1.119126in}}%
\pgfpathlineto{\pgfqpoint{0.825762in}{1.123553in}}%
\pgfpathlineto{\pgfqpoint{0.840524in}{1.125305in}}%
\pgfpathlineto{\pgfqpoint{0.855286in}{1.122018in}}%
\pgfpathlineto{\pgfqpoint{0.870048in}{1.121040in}}%
\pgfpathlineto{\pgfqpoint{0.884810in}{1.122865in}}%
\pgfpathlineto{\pgfqpoint{0.899572in}{1.120118in}}%
\pgfpathlineto{\pgfqpoint{0.914333in}{1.114908in}}%
\pgfpathlineto{\pgfqpoint{0.943857in}{1.111905in}}%
\pgfpathlineto{\pgfqpoint{0.973381in}{1.097233in}}%
\pgfpathlineto{\pgfqpoint{0.988143in}{1.093441in}}%
\pgfpathlineto{\pgfqpoint{1.002905in}{1.088025in}}%
\pgfpathlineto{\pgfqpoint{1.017667in}{1.077559in}}%
\pgfpathlineto{\pgfqpoint{1.032429in}{1.069367in}}%
\pgfpathlineto{\pgfqpoint{1.047191in}{1.063378in}}%
\pgfpathlineto{\pgfqpoint{1.061953in}{1.053321in}}%
\pgfpathlineto{\pgfqpoint{1.076714in}{1.039748in}}%
\pgfpathlineto{\pgfqpoint{1.106238in}{1.019515in}}%
\pgfpathlineto{\pgfqpoint{1.135762in}{0.989636in}}%
\pgfpathlineto{\pgfqpoint{1.165286in}{0.962942in}}%
\pgfpathlineto{\pgfqpoint{1.194810in}{0.925443in}}%
\pgfpathlineto{\pgfqpoint{1.209572in}{0.909058in}}%
\pgfpathlineto{\pgfqpoint{1.224333in}{0.889448in}}%
\pgfpathlineto{\pgfqpoint{1.268619in}{0.820624in}}%
\pgfpathlineto{\pgfqpoint{1.283381in}{0.790239in}}%
\pgfpathlineto{\pgfqpoint{1.298143in}{0.764343in}}%
\pgfpathlineto{\pgfqpoint{1.327667in}{0.723945in}}%
\pgfpathlineto{\pgfqpoint{1.342429in}{0.724082in}}%
\pgfpathlineto{\pgfqpoint{1.357191in}{0.730953in}}%
\pgfpathlineto{\pgfqpoint{1.371953in}{0.753345in}}%
\pgfpathlineto{\pgfqpoint{1.401476in}{0.794568in}}%
\pgfpathlineto{\pgfqpoint{1.431000in}{0.845728in}}%
\pgfpathlineto{\pgfqpoint{1.445762in}{0.865581in}}%
\pgfpathlineto{\pgfqpoint{1.475286in}{0.895622in}}%
\pgfpathlineto{\pgfqpoint{1.504810in}{0.922543in}}%
\pgfpathlineto{\pgfqpoint{1.549095in}{0.951424in}}%
\pgfpathlineto{\pgfqpoint{1.563857in}{0.959091in}}%
\pgfpathlineto{\pgfqpoint{1.608143in}{0.975142in}}%
\pgfpathlineto{\pgfqpoint{1.637667in}{0.979750in}}%
\pgfpathlineto{\pgfqpoint{1.667191in}{0.981142in}}%
\pgfpathlineto{\pgfqpoint{1.681953in}{0.979913in}}%
\pgfpathlineto{\pgfqpoint{1.726238in}{0.970819in}}%
\pgfpathlineto{\pgfqpoint{1.755762in}{0.958935in}}%
\pgfpathlineto{\pgfqpoint{1.785286in}{0.943019in}}%
\pgfpathlineto{\pgfqpoint{1.829572in}{0.910762in}}%
\pgfpathlineto{\pgfqpoint{1.859095in}{0.881835in}}%
\pgfpathlineto{\pgfqpoint{1.888619in}{0.847853in}}%
\pgfpathlineto{\pgfqpoint{1.903381in}{0.827407in}}%
\pgfpathlineto{\pgfqpoint{1.932905in}{0.777341in}}%
\pgfpathlineto{\pgfqpoint{1.947667in}{0.760846in}}%
\pgfpathlineto{\pgfqpoint{1.962429in}{0.748553in}}%
\pgfpathlineto{\pgfqpoint{1.977191in}{0.734088in}}%
\pgfpathlineto{\pgfqpoint{1.991953in}{0.735792in}}%
\pgfpathlineto{\pgfqpoint{2.006714in}{0.743179in}}%
\pgfpathlineto{\pgfqpoint{2.021476in}{0.757666in}}%
\pgfpathlineto{\pgfqpoint{2.036238in}{0.774601in}}%
\pgfpathlineto{\pgfqpoint{2.051000in}{0.798233in}}%
\pgfpathlineto{\pgfqpoint{2.080524in}{0.830893in}}%
\pgfpathlineto{\pgfqpoint{2.095286in}{0.841194in}}%
\pgfpathlineto{\pgfqpoint{2.110048in}{0.856688in}}%
\pgfpathlineto{\pgfqpoint{2.124810in}{0.866363in}}%
\pgfpathlineto{\pgfqpoint{2.169095in}{0.882317in}}%
\pgfpathlineto{\pgfqpoint{2.183857in}{0.886215in}}%
\pgfpathlineto{\pgfqpoint{2.213381in}{0.883872in}}%
\pgfpathlineto{\pgfqpoint{2.228143in}{0.883819in}}%
\pgfpathlineto{\pgfqpoint{2.242905in}{0.881534in}}%
\pgfpathlineto{\pgfqpoint{2.272429in}{0.869678in}}%
\pgfpathlineto{\pgfqpoint{2.287191in}{0.864580in}}%
\pgfpathlineto{\pgfqpoint{2.301953in}{0.854391in}}%
\pgfpathlineto{\pgfqpoint{2.316714in}{0.837702in}}%
\pgfpathlineto{\pgfqpoint{2.331476in}{0.826722in}}%
\pgfpathlineto{\pgfqpoint{2.346238in}{0.813551in}}%
\pgfpathlineto{\pgfqpoint{2.361000in}{0.795621in}}%
\pgfpathlineto{\pgfqpoint{2.375762in}{0.774680in}}%
\pgfpathlineto{\pgfqpoint{2.390524in}{0.762129in}}%
\pgfpathlineto{\pgfqpoint{2.405286in}{0.754374in}}%
\pgfpathlineto{\pgfqpoint{2.420048in}{0.742729in}}%
\pgfpathlineto{\pgfqpoint{2.434810in}{0.743527in}}%
\pgfpathlineto{\pgfqpoint{2.449572in}{0.749403in}}%
\pgfpathlineto{\pgfqpoint{2.493857in}{0.787466in}}%
\pgfpathlineto{\pgfqpoint{2.508619in}{0.804290in}}%
\pgfpathlineto{\pgfqpoint{2.523381in}{0.818033in}}%
\pgfpathlineto{\pgfqpoint{2.552905in}{0.829263in}}%
\pgfpathlineto{\pgfqpoint{2.567667in}{0.838363in}}%
\pgfpathlineto{\pgfqpoint{2.582429in}{0.841739in}}%
\pgfpathlineto{\pgfqpoint{2.597191in}{0.838097in}}%
\pgfpathlineto{\pgfqpoint{2.611953in}{0.837335in}}%
\pgfpathlineto{\pgfqpoint{2.626714in}{0.838128in}}%
\pgfpathlineto{\pgfqpoint{2.641476in}{0.827012in}}%
\pgfpathlineto{\pgfqpoint{2.656238in}{0.819142in}}%
\pgfpathlineto{\pgfqpoint{2.671000in}{0.808035in}}%
\pgfpathlineto{\pgfqpoint{2.715286in}{0.763881in}}%
\pgfpathlineto{\pgfqpoint{2.744810in}{0.744656in}}%
\pgfpathlineto{\pgfqpoint{2.759572in}{0.744153in}}%
\pgfpathlineto{\pgfqpoint{2.774333in}{0.748366in}}%
\pgfpathlineto{\pgfqpoint{2.789095in}{0.758736in}}%
\pgfpathlineto{\pgfqpoint{2.803857in}{0.773705in}}%
\pgfpathlineto{\pgfqpoint{2.818619in}{0.783612in}}%
\pgfpathlineto{\pgfqpoint{2.848143in}{0.815167in}}%
\pgfpathlineto{\pgfqpoint{2.877667in}{0.829521in}}%
\pgfpathlineto{\pgfqpoint{2.907191in}{0.840587in}}%
\pgfpathlineto{\pgfqpoint{2.936714in}{0.839870in}}%
\pgfpathlineto{\pgfqpoint{2.951476in}{0.838734in}}%
\pgfpathlineto{\pgfqpoint{2.981000in}{0.829069in}}%
\pgfpathlineto{\pgfqpoint{3.010524in}{0.813523in}}%
\pgfpathlineto{\pgfqpoint{3.025286in}{0.795396in}}%
\pgfpathlineto{\pgfqpoint{3.084333in}{0.750204in}}%
\pgfpathlineto{\pgfqpoint{3.099095in}{0.747179in}}%
\pgfpathlineto{\pgfqpoint{3.113857in}{0.749263in}}%
\pgfpathlineto{\pgfqpoint{3.128619in}{0.756831in}}%
\pgfpathlineto{\pgfqpoint{3.158143in}{0.777634in}}%
\pgfpathlineto{\pgfqpoint{3.172905in}{0.790213in}}%
\pgfpathlineto{\pgfqpoint{3.217191in}{0.815558in}}%
\pgfpathlineto{\pgfqpoint{3.231953in}{0.822350in}}%
\pgfpathlineto{\pgfqpoint{3.261476in}{0.822541in}}%
\pgfpathlineto{\pgfqpoint{3.276238in}{0.821650in}}%
\pgfpathlineto{\pgfqpoint{3.291000in}{0.817666in}}%
\pgfpathlineto{\pgfqpoint{3.320524in}{0.798533in}}%
\pgfpathlineto{\pgfqpoint{3.350048in}{0.779205in}}%
\pgfpathlineto{\pgfqpoint{3.364810in}{0.767461in}}%
\pgfpathlineto{\pgfqpoint{3.394333in}{0.753356in}}%
\pgfpathlineto{\pgfqpoint{3.409095in}{0.751002in}}%
\pgfpathlineto{\pgfqpoint{3.423857in}{0.754328in}}%
\pgfpathlineto{\pgfqpoint{3.438619in}{0.762476in}}%
\pgfpathlineto{\pgfqpoint{3.482905in}{0.791435in}}%
\pgfpathlineto{\pgfqpoint{3.497667in}{0.800538in}}%
\pgfpathlineto{\pgfqpoint{3.512429in}{0.806803in}}%
\pgfpathlineto{\pgfqpoint{3.527191in}{0.818554in}}%
\pgfpathlineto{\pgfqpoint{3.541953in}{0.822553in}}%
\pgfpathlineto{\pgfqpoint{3.571476in}{0.824984in}}%
\pgfpathlineto{\pgfqpoint{3.601000in}{0.820773in}}%
\pgfpathlineto{\pgfqpoint{3.615762in}{0.817105in}}%
\pgfpathlineto{\pgfqpoint{3.630524in}{0.805668in}}%
\pgfpathlineto{\pgfqpoint{3.645286in}{0.798823in}}%
\pgfpathlineto{\pgfqpoint{3.674810in}{0.780285in}}%
\pgfpathlineto{\pgfqpoint{3.704333in}{0.762670in}}%
\pgfpathlineto{\pgfqpoint{3.719095in}{0.758724in}}%
\pgfpathlineto{\pgfqpoint{3.733857in}{0.756587in}}%
\pgfpathlineto{\pgfqpoint{3.748619in}{0.759980in}}%
\pgfpathlineto{\pgfqpoint{3.763381in}{0.766328in}}%
\pgfpathlineto{\pgfqpoint{3.763381in}{0.766328in}}%
\pgfusepath{stroke}%
\end{pgfscope}%
\begin{pgfscope}%
\pgfpathrectangle{\pgfqpoint{0.501000in}{0.566590in}}{\pgfqpoint{3.720000in}{3.020000in}} %
\pgfusepath{clip}%
\pgfsetrectcap%
\pgfsetroundjoin%
\pgfsetlinewidth{1.505625pt}%
\definecolor{currentstroke}{rgb}{0.982353,0.726434,0.395451}%
\pgfsetstrokecolor{currentstroke}%
\pgfsetdash{}{0pt}%
\pgfpathmoveto{\pgfqpoint{0.487111in}{0.878216in}}%
\pgfpathlineto{\pgfqpoint{0.515762in}{0.901328in}}%
\pgfpathlineto{\pgfqpoint{0.530524in}{0.906839in}}%
\pgfpathlineto{\pgfqpoint{0.545286in}{0.916274in}}%
\pgfpathlineto{\pgfqpoint{0.560048in}{0.922820in}}%
\pgfpathlineto{\pgfqpoint{0.574810in}{0.925842in}}%
\pgfpathlineto{\pgfqpoint{0.589572in}{0.932859in}}%
\pgfpathlineto{\pgfqpoint{0.604333in}{0.935130in}}%
\pgfpathlineto{\pgfqpoint{0.619095in}{0.934597in}}%
\pgfpathlineto{\pgfqpoint{0.633857in}{0.939641in}}%
\pgfpathlineto{\pgfqpoint{0.648619in}{0.938823in}}%
\pgfpathlineto{\pgfqpoint{0.663381in}{0.935896in}}%
\pgfpathlineto{\pgfqpoint{0.678143in}{0.937326in}}%
\pgfpathlineto{\pgfqpoint{0.707667in}{0.928179in}}%
\pgfpathlineto{\pgfqpoint{0.722429in}{0.925787in}}%
\pgfpathlineto{\pgfqpoint{0.766714in}{0.905013in}}%
\pgfpathlineto{\pgfqpoint{0.796238in}{0.884177in}}%
\pgfpathlineto{\pgfqpoint{0.811000in}{0.875039in}}%
\pgfpathlineto{\pgfqpoint{0.855286in}{0.833894in}}%
\pgfpathlineto{\pgfqpoint{0.899572in}{0.777702in}}%
\pgfpathlineto{\pgfqpoint{0.914333in}{0.755805in}}%
\pgfpathlineto{\pgfqpoint{0.929095in}{0.740769in}}%
\pgfpathlineto{\pgfqpoint{0.943857in}{0.722907in}}%
\pgfpathlineto{\pgfqpoint{0.958619in}{0.723059in}}%
\pgfpathlineto{\pgfqpoint{0.973381in}{0.729642in}}%
\pgfpathlineto{\pgfqpoint{0.988143in}{0.746215in}}%
\pgfpathlineto{\pgfqpoint{1.002905in}{0.765563in}}%
\pgfpathlineto{\pgfqpoint{1.017667in}{0.782313in}}%
\pgfpathlineto{\pgfqpoint{1.061953in}{0.818137in}}%
\pgfpathlineto{\pgfqpoint{1.076714in}{0.820530in}}%
\pgfpathlineto{\pgfqpoint{1.091476in}{0.831163in}}%
\pgfpathlineto{\pgfqpoint{1.106238in}{0.836108in}}%
\pgfpathlineto{\pgfqpoint{1.121000in}{0.834428in}}%
\pgfpathlineto{\pgfqpoint{1.135762in}{0.839080in}}%
\pgfpathlineto{\pgfqpoint{1.150524in}{0.841092in}}%
\pgfpathlineto{\pgfqpoint{1.165286in}{0.835601in}}%
\pgfpathlineto{\pgfqpoint{1.180048in}{0.836901in}}%
\pgfpathlineto{\pgfqpoint{1.194810in}{0.831763in}}%
\pgfpathlineto{\pgfqpoint{1.209572in}{0.823759in}}%
\pgfpathlineto{\pgfqpoint{1.224333in}{0.820443in}}%
\pgfpathlineto{\pgfqpoint{1.239095in}{0.810414in}}%
\pgfpathlineto{\pgfqpoint{1.253857in}{0.797493in}}%
\pgfpathlineto{\pgfqpoint{1.268619in}{0.788095in}}%
\pgfpathlineto{\pgfqpoint{1.312905in}{0.741689in}}%
\pgfpathlineto{\pgfqpoint{1.327667in}{0.729326in}}%
\pgfpathlineto{\pgfqpoint{1.342429in}{0.728399in}}%
\pgfpathlineto{\pgfqpoint{1.357191in}{0.732450in}}%
\pgfpathlineto{\pgfqpoint{1.386714in}{0.758013in}}%
\pgfpathlineto{\pgfqpoint{1.401476in}{0.773697in}}%
\pgfpathlineto{\pgfqpoint{1.416238in}{0.782803in}}%
\pgfpathlineto{\pgfqpoint{1.431000in}{0.794386in}}%
\pgfpathlineto{\pgfqpoint{1.445762in}{0.798669in}}%
\pgfpathlineto{\pgfqpoint{1.460524in}{0.798980in}}%
\pgfpathlineto{\pgfqpoint{1.475286in}{0.801807in}}%
\pgfpathlineto{\pgfqpoint{1.490048in}{0.799592in}}%
\pgfpathlineto{\pgfqpoint{1.519572in}{0.785545in}}%
\pgfpathlineto{\pgfqpoint{1.534333in}{0.774702in}}%
\pgfpathlineto{\pgfqpoint{1.549095in}{0.758738in}}%
\pgfpathlineto{\pgfqpoint{1.563857in}{0.749019in}}%
\pgfpathlineto{\pgfqpoint{1.578619in}{0.735811in}}%
\pgfpathlineto{\pgfqpoint{1.593381in}{0.731061in}}%
\pgfpathlineto{\pgfqpoint{1.608143in}{0.732893in}}%
\pgfpathlineto{\pgfqpoint{1.637667in}{0.752511in}}%
\pgfpathlineto{\pgfqpoint{1.652429in}{0.768229in}}%
\pgfpathlineto{\pgfqpoint{1.667191in}{0.778619in}}%
\pgfpathlineto{\pgfqpoint{1.696714in}{0.790177in}}%
\pgfpathlineto{\pgfqpoint{1.711476in}{0.791467in}}%
\pgfpathlineto{\pgfqpoint{1.741000in}{0.785647in}}%
\pgfpathlineto{\pgfqpoint{1.755762in}{0.779528in}}%
\pgfpathlineto{\pgfqpoint{1.770524in}{0.768542in}}%
\pgfpathlineto{\pgfqpoint{1.785286in}{0.760726in}}%
\pgfpathlineto{\pgfqpoint{1.800048in}{0.746996in}}%
\pgfpathlineto{\pgfqpoint{1.814810in}{0.740169in}}%
\pgfpathlineto{\pgfqpoint{1.829572in}{0.736057in}}%
\pgfpathlineto{\pgfqpoint{1.844333in}{0.739603in}}%
\pgfpathlineto{\pgfqpoint{1.859095in}{0.745862in}}%
\pgfpathlineto{\pgfqpoint{1.873857in}{0.758174in}}%
\pgfpathlineto{\pgfqpoint{1.903381in}{0.775764in}}%
\pgfpathlineto{\pgfqpoint{1.918143in}{0.781488in}}%
\pgfpathlineto{\pgfqpoint{1.932905in}{0.784017in}}%
\pgfpathlineto{\pgfqpoint{1.947667in}{0.782512in}}%
\pgfpathlineto{\pgfqpoint{1.962429in}{0.776859in}}%
\pgfpathlineto{\pgfqpoint{1.991953in}{0.758327in}}%
\pgfpathlineto{\pgfqpoint{2.006714in}{0.748162in}}%
\pgfpathlineto{\pgfqpoint{2.021476in}{0.740444in}}%
\pgfpathlineto{\pgfqpoint{2.036238in}{0.737955in}}%
\pgfpathlineto{\pgfqpoint{2.051000in}{0.740927in}}%
\pgfpathlineto{\pgfqpoint{2.110048in}{0.777457in}}%
\pgfpathlineto{\pgfqpoint{2.124810in}{0.782093in}}%
\pgfpathlineto{\pgfqpoint{2.139572in}{0.783904in}}%
\pgfpathlineto{\pgfqpoint{2.154333in}{0.781199in}}%
\pgfpathlineto{\pgfqpoint{2.169095in}{0.775340in}}%
\pgfpathlineto{\pgfqpoint{2.228143in}{0.742202in}}%
\pgfpathlineto{\pgfqpoint{2.242905in}{0.742182in}}%
\pgfpathlineto{\pgfqpoint{2.257667in}{0.747704in}}%
\pgfpathlineto{\pgfqpoint{2.301953in}{0.775485in}}%
\pgfpathlineto{\pgfqpoint{2.316714in}{0.781545in}}%
\pgfpathlineto{\pgfqpoint{2.331476in}{0.784297in}}%
\pgfpathlineto{\pgfqpoint{2.346238in}{0.783233in}}%
\pgfpathlineto{\pgfqpoint{2.361000in}{0.778546in}}%
\pgfpathlineto{\pgfqpoint{2.390524in}{0.762792in}}%
\pgfpathlineto{\pgfqpoint{2.420048in}{0.747305in}}%
\pgfpathlineto{\pgfqpoint{2.434810in}{0.746173in}}%
\pgfpathlineto{\pgfqpoint{2.449572in}{0.749239in}}%
\pgfpathlineto{\pgfqpoint{2.464333in}{0.756502in}}%
\pgfpathlineto{\pgfqpoint{2.493857in}{0.774199in}}%
\pgfpathlineto{\pgfqpoint{2.508619in}{0.781363in}}%
\pgfpathlineto{\pgfqpoint{2.523381in}{0.784729in}}%
\pgfpathlineto{\pgfqpoint{2.538143in}{0.785839in}}%
\pgfpathlineto{\pgfqpoint{2.552905in}{0.783171in}}%
\pgfpathlineto{\pgfqpoint{2.582429in}{0.770210in}}%
\pgfpathlineto{\pgfqpoint{2.611953in}{0.754544in}}%
\pgfpathlineto{\pgfqpoint{2.626714in}{0.749984in}}%
\pgfpathlineto{\pgfqpoint{2.641476in}{0.749784in}}%
\pgfpathlineto{\pgfqpoint{2.656238in}{0.754044in}}%
\pgfpathlineto{\pgfqpoint{2.700524in}{0.777326in}}%
\pgfpathlineto{\pgfqpoint{2.715286in}{0.783211in}}%
\pgfpathlineto{\pgfqpoint{2.730048in}{0.785746in}}%
\pgfpathlineto{\pgfqpoint{2.744810in}{0.785342in}}%
\pgfpathlineto{\pgfqpoint{2.759572in}{0.781906in}}%
\pgfpathlineto{\pgfqpoint{2.818619in}{0.754936in}}%
\pgfpathlineto{\pgfqpoint{2.833381in}{0.753060in}}%
\pgfpathlineto{\pgfqpoint{2.848143in}{0.754342in}}%
\pgfpathlineto{\pgfqpoint{2.877667in}{0.767891in}}%
\pgfpathlineto{\pgfqpoint{2.921953in}{0.786677in}}%
\pgfpathlineto{\pgfqpoint{2.936714in}{0.787218in}}%
\pgfpathlineto{\pgfqpoint{2.951476in}{0.786045in}}%
\pgfpathlineto{\pgfqpoint{2.966238in}{0.781405in}}%
\pgfpathlineto{\pgfqpoint{3.025286in}{0.756622in}}%
\pgfpathlineto{\pgfqpoint{3.040048in}{0.755457in}}%
\pgfpathlineto{\pgfqpoint{3.054810in}{0.757647in}}%
\pgfpathlineto{\pgfqpoint{3.084333in}{0.768612in}}%
\pgfpathlineto{\pgfqpoint{3.113857in}{0.780033in}}%
\pgfpathlineto{\pgfqpoint{3.128619in}{0.783640in}}%
\pgfpathlineto{\pgfqpoint{3.143381in}{0.784461in}}%
\pgfpathlineto{\pgfqpoint{3.172905in}{0.777679in}}%
\pgfpathlineto{\pgfqpoint{3.217191in}{0.761148in}}%
\pgfpathlineto{\pgfqpoint{3.231953in}{0.758195in}}%
\pgfpathlineto{\pgfqpoint{3.246714in}{0.757946in}}%
\pgfpathlineto{\pgfqpoint{3.276238in}{0.764718in}}%
\pgfpathlineto{\pgfqpoint{3.305762in}{0.775498in}}%
\pgfpathlineto{\pgfqpoint{3.320524in}{0.778805in}}%
\pgfpathlineto{\pgfqpoint{3.335286in}{0.780414in}}%
\pgfpathlineto{\pgfqpoint{3.350048in}{0.779933in}}%
\pgfpathlineto{\pgfqpoint{3.364810in}{0.776970in}}%
\pgfpathlineto{\pgfqpoint{3.409095in}{0.762863in}}%
\pgfpathlineto{\pgfqpoint{3.423857in}{0.760046in}}%
\pgfpathlineto{\pgfqpoint{3.438619in}{0.759337in}}%
\pgfpathlineto{\pgfqpoint{3.453381in}{0.760727in}}%
\pgfpathlineto{\pgfqpoint{3.527191in}{0.777859in}}%
\pgfpathlineto{\pgfqpoint{3.556714in}{0.774983in}}%
\pgfpathlineto{\pgfqpoint{3.615762in}{0.761264in}}%
\pgfpathlineto{\pgfqpoint{3.630524in}{0.760519in}}%
\pgfpathlineto{\pgfqpoint{3.660048in}{0.764770in}}%
\pgfpathlineto{\pgfqpoint{3.704333in}{0.774480in}}%
\pgfpathlineto{\pgfqpoint{3.733857in}{0.774946in}}%
\pgfpathlineto{\pgfqpoint{3.778143in}{0.766669in}}%
\pgfpathlineto{\pgfqpoint{3.778143in}{0.766669in}}%
\pgfusepath{stroke}%
\end{pgfscope}%
\begin{pgfscope}%
\pgfsetrectcap%
\pgfsetmiterjoin%
\pgfsetlinewidth{0.803000pt}%
\definecolor{currentstroke}{rgb}{0.000000,0.000000,0.000000}%
\pgfsetstrokecolor{currentstroke}%
\pgfsetdash{}{0pt}%
\pgfpathmoveto{\pgfqpoint{0.501000in}{0.566590in}}%
\pgfpathlineto{\pgfqpoint{0.501000in}{3.586590in}}%
\pgfusepath{stroke}%
\end{pgfscope}%
\begin{pgfscope}%
\pgfsetrectcap%
\pgfsetmiterjoin%
\pgfsetlinewidth{0.803000pt}%
\definecolor{currentstroke}{rgb}{0.000000,0.000000,0.000000}%
\pgfsetstrokecolor{currentstroke}%
\pgfsetdash{}{0pt}%
\pgfpathmoveto{\pgfqpoint{4.221000in}{0.566590in}}%
\pgfpathlineto{\pgfqpoint{4.221000in}{3.586590in}}%
\pgfusepath{stroke}%
\end{pgfscope}%
\begin{pgfscope}%
\pgfsetrectcap%
\pgfsetmiterjoin%
\pgfsetlinewidth{0.803000pt}%
\definecolor{currentstroke}{rgb}{0.000000,0.000000,0.000000}%
\pgfsetstrokecolor{currentstroke}%
\pgfsetdash{}{0pt}%
\pgfpathmoveto{\pgfqpoint{0.501000in}{0.566590in}}%
\pgfpathlineto{\pgfqpoint{4.221000in}{0.566590in}}%
\pgfusepath{stroke}%
\end{pgfscope}%
\begin{pgfscope}%
\pgfsetrectcap%
\pgfsetmiterjoin%
\pgfsetlinewidth{0.803000pt}%
\definecolor{currentstroke}{rgb}{0.000000,0.000000,0.000000}%
\pgfsetstrokecolor{currentstroke}%
\pgfsetdash{}{0pt}%
\pgfpathmoveto{\pgfqpoint{0.501000in}{3.586590in}}%
\pgfpathlineto{\pgfqpoint{4.221000in}{3.586590in}}%
\pgfusepath{stroke}%
\end{pgfscope}%
\begin{pgfscope}%
\pgfpathrectangle{\pgfqpoint{4.453500in}{0.566590in}}{\pgfqpoint{0.151000in}{3.020000in}} %
\pgfusepath{clip}%
\pgfsetbuttcap%
\pgfsetmiterjoin%
\definecolor{currentfill}{rgb}{1.000000,1.000000,1.000000}%
\pgfsetfillcolor{currentfill}%
\pgfsetlinewidth{0.010037pt}%
\definecolor{currentstroke}{rgb}{1.000000,1.000000,1.000000}%
\pgfsetstrokecolor{currentstroke}%
\pgfsetdash{}{0pt}%
\pgfpathmoveto{\pgfqpoint{4.453500in}{0.566590in}}%
\pgfpathlineto{\pgfqpoint{4.453500in}{0.578387in}}%
\pgfpathlineto{\pgfqpoint{4.453500in}{3.574793in}}%
\pgfpathlineto{\pgfqpoint{4.453500in}{3.586590in}}%
\pgfpathlineto{\pgfqpoint{4.604500in}{3.586590in}}%
\pgfpathlineto{\pgfqpoint{4.604500in}{3.574793in}}%
\pgfpathlineto{\pgfqpoint{4.604500in}{0.578387in}}%
\pgfpathlineto{\pgfqpoint{4.604500in}{0.566590in}}%
\pgfpathclose%
\pgfusepath{stroke,fill}%
\end{pgfscope}%
\begin{pgfscope}%
\pgfsys@transformshift{4.458333in}{0.582423in}%
\pgftext[left,bottom]{\pgfimage[interpolate=true,width=0.152778in,height=3.013889in]{series-img0.png}}%
\end{pgfscope}%
\begin{pgfscope}%
\pgfsetbuttcap%
\pgfsetroundjoin%
\definecolor{currentfill}{rgb}{0.000000,0.000000,0.000000}%
\pgfsetfillcolor{currentfill}%
\pgfsetlinewidth{0.803000pt}%
\definecolor{currentstroke}{rgb}{0.000000,0.000000,0.000000}%
\pgfsetstrokecolor{currentstroke}%
\pgfsetdash{}{0pt}%
\pgfsys@defobject{currentmarker}{\pgfqpoint{0.000000in}{0.000000in}}{\pgfqpoint{0.048611in}{0.000000in}}{%
\pgfpathmoveto{\pgfqpoint{0.000000in}{0.000000in}}%
\pgfpathlineto{\pgfqpoint{0.048611in}{0.000000in}}%
\pgfusepath{stroke,fill}%
}%
\begin{pgfscope}%
\pgfsys@transformshift{4.604500in}{0.601555in}%
\pgfsys@useobject{currentmarker}{}%
\end{pgfscope}%
\end{pgfscope}%
\begin{pgfscope}%
\pgfsetbuttcap%
\pgfsetroundjoin%
\definecolor{currentfill}{rgb}{0.000000,0.000000,0.000000}%
\pgfsetfillcolor{currentfill}%
\pgfsetlinewidth{0.803000pt}%
\definecolor{currentstroke}{rgb}{0.000000,0.000000,0.000000}%
\pgfsetstrokecolor{currentstroke}%
\pgfsetdash{}{0pt}%
\pgfsys@defobject{currentmarker}{\pgfqpoint{0.000000in}{0.000000in}}{\pgfqpoint{0.048611in}{0.000000in}}{%
\pgfpathmoveto{\pgfqpoint{0.000000in}{0.000000in}}%
\pgfpathlineto{\pgfqpoint{0.048611in}{0.000000in}}%
\pgfusepath{stroke,fill}%
}%
\begin{pgfscope}%
\pgfsys@transformshift{4.604500in}{0.683069in}%
\pgfsys@useobject{currentmarker}{}%
\end{pgfscope}%
\end{pgfscope}%
\begin{pgfscope}%
\pgftext[x=4.701722in,y=0.625676in,left,base]{\rmfamily\fontsize{12.000000}{14.400000}\selectfont \(\displaystyle 10^{-1}\)}%
\end{pgfscope}%
\begin{pgfscope}%
\pgfsetbuttcap%
\pgfsetroundjoin%
\definecolor{currentfill}{rgb}{0.000000,0.000000,0.000000}%
\pgfsetfillcolor{currentfill}%
\pgfsetlinewidth{0.803000pt}%
\definecolor{currentstroke}{rgb}{0.000000,0.000000,0.000000}%
\pgfsetstrokecolor{currentstroke}%
\pgfsetdash{}{0pt}%
\pgfsys@defobject{currentmarker}{\pgfqpoint{0.000000in}{0.000000in}}{\pgfqpoint{0.048611in}{0.000000in}}{%
\pgfpathmoveto{\pgfqpoint{0.000000in}{0.000000in}}%
\pgfpathlineto{\pgfqpoint{0.048611in}{0.000000in}}%
\pgfusepath{stroke,fill}%
}%
\begin{pgfscope}%
\pgfsys@transformshift{4.604500in}{1.219335in}%
\pgfsys@useobject{currentmarker}{}%
\end{pgfscope}%
\end{pgfscope}%
\begin{pgfscope}%
\pgfsetbuttcap%
\pgfsetroundjoin%
\definecolor{currentfill}{rgb}{0.000000,0.000000,0.000000}%
\pgfsetfillcolor{currentfill}%
\pgfsetlinewidth{0.803000pt}%
\definecolor{currentstroke}{rgb}{0.000000,0.000000,0.000000}%
\pgfsetstrokecolor{currentstroke}%
\pgfsetdash{}{0pt}%
\pgfsys@defobject{currentmarker}{\pgfqpoint{0.000000in}{0.000000in}}{\pgfqpoint{0.048611in}{0.000000in}}{%
\pgfpathmoveto{\pgfqpoint{0.000000in}{0.000000in}}%
\pgfpathlineto{\pgfqpoint{0.048611in}{0.000000in}}%
\pgfusepath{stroke,fill}%
}%
\begin{pgfscope}%
\pgfsys@transformshift{4.604500in}{1.533030in}%
\pgfsys@useobject{currentmarker}{}%
\end{pgfscope}%
\end{pgfscope}%
\begin{pgfscope}%
\pgfsetbuttcap%
\pgfsetroundjoin%
\definecolor{currentfill}{rgb}{0.000000,0.000000,0.000000}%
\pgfsetfillcolor{currentfill}%
\pgfsetlinewidth{0.803000pt}%
\definecolor{currentstroke}{rgb}{0.000000,0.000000,0.000000}%
\pgfsetstrokecolor{currentstroke}%
\pgfsetdash{}{0pt}%
\pgfsys@defobject{currentmarker}{\pgfqpoint{0.000000in}{0.000000in}}{\pgfqpoint{0.048611in}{0.000000in}}{%
\pgfpathmoveto{\pgfqpoint{0.000000in}{0.000000in}}%
\pgfpathlineto{\pgfqpoint{0.048611in}{0.000000in}}%
\pgfusepath{stroke,fill}%
}%
\begin{pgfscope}%
\pgfsys@transformshift{4.604500in}{1.755600in}%
\pgfsys@useobject{currentmarker}{}%
\end{pgfscope}%
\end{pgfscope}%
\begin{pgfscope}%
\pgfsetbuttcap%
\pgfsetroundjoin%
\definecolor{currentfill}{rgb}{0.000000,0.000000,0.000000}%
\pgfsetfillcolor{currentfill}%
\pgfsetlinewidth{0.803000pt}%
\definecolor{currentstroke}{rgb}{0.000000,0.000000,0.000000}%
\pgfsetstrokecolor{currentstroke}%
\pgfsetdash{}{0pt}%
\pgfsys@defobject{currentmarker}{\pgfqpoint{0.000000in}{0.000000in}}{\pgfqpoint{0.048611in}{0.000000in}}{%
\pgfpathmoveto{\pgfqpoint{0.000000in}{0.000000in}}%
\pgfpathlineto{\pgfqpoint{0.048611in}{0.000000in}}%
\pgfusepath{stroke,fill}%
}%
\begin{pgfscope}%
\pgfsys@transformshift{4.604500in}{1.928239in}%
\pgfsys@useobject{currentmarker}{}%
\end{pgfscope}%
\end{pgfscope}%
\begin{pgfscope}%
\pgfsetbuttcap%
\pgfsetroundjoin%
\definecolor{currentfill}{rgb}{0.000000,0.000000,0.000000}%
\pgfsetfillcolor{currentfill}%
\pgfsetlinewidth{0.803000pt}%
\definecolor{currentstroke}{rgb}{0.000000,0.000000,0.000000}%
\pgfsetstrokecolor{currentstroke}%
\pgfsetdash{}{0pt}%
\pgfsys@defobject{currentmarker}{\pgfqpoint{0.000000in}{0.000000in}}{\pgfqpoint{0.048611in}{0.000000in}}{%
\pgfpathmoveto{\pgfqpoint{0.000000in}{0.000000in}}%
\pgfpathlineto{\pgfqpoint{0.048611in}{0.000000in}}%
\pgfusepath{stroke,fill}%
}%
\begin{pgfscope}%
\pgfsys@transformshift{4.604500in}{2.069295in}%
\pgfsys@useobject{currentmarker}{}%
\end{pgfscope}%
\end{pgfscope}%
\begin{pgfscope}%
\pgfsetbuttcap%
\pgfsetroundjoin%
\definecolor{currentfill}{rgb}{0.000000,0.000000,0.000000}%
\pgfsetfillcolor{currentfill}%
\pgfsetlinewidth{0.803000pt}%
\definecolor{currentstroke}{rgb}{0.000000,0.000000,0.000000}%
\pgfsetstrokecolor{currentstroke}%
\pgfsetdash{}{0pt}%
\pgfsys@defobject{currentmarker}{\pgfqpoint{0.000000in}{0.000000in}}{\pgfqpoint{0.048611in}{0.000000in}}{%
\pgfpathmoveto{\pgfqpoint{0.000000in}{0.000000in}}%
\pgfpathlineto{\pgfqpoint{0.048611in}{0.000000in}}%
\pgfusepath{stroke,fill}%
}%
\begin{pgfscope}%
\pgfsys@transformshift{4.604500in}{2.188556in}%
\pgfsys@useobject{currentmarker}{}%
\end{pgfscope}%
\end{pgfscope}%
\begin{pgfscope}%
\pgfsetbuttcap%
\pgfsetroundjoin%
\definecolor{currentfill}{rgb}{0.000000,0.000000,0.000000}%
\pgfsetfillcolor{currentfill}%
\pgfsetlinewidth{0.803000pt}%
\definecolor{currentstroke}{rgb}{0.000000,0.000000,0.000000}%
\pgfsetstrokecolor{currentstroke}%
\pgfsetdash{}{0pt}%
\pgfsys@defobject{currentmarker}{\pgfqpoint{0.000000in}{0.000000in}}{\pgfqpoint{0.048611in}{0.000000in}}{%
\pgfpathmoveto{\pgfqpoint{0.000000in}{0.000000in}}%
\pgfpathlineto{\pgfqpoint{0.048611in}{0.000000in}}%
\pgfusepath{stroke,fill}%
}%
\begin{pgfscope}%
\pgfsys@transformshift{4.604500in}{2.291865in}%
\pgfsys@useobject{currentmarker}{}%
\end{pgfscope}%
\end{pgfscope}%
\begin{pgfscope}%
\pgfsetbuttcap%
\pgfsetroundjoin%
\definecolor{currentfill}{rgb}{0.000000,0.000000,0.000000}%
\pgfsetfillcolor{currentfill}%
\pgfsetlinewidth{0.803000pt}%
\definecolor{currentstroke}{rgb}{0.000000,0.000000,0.000000}%
\pgfsetstrokecolor{currentstroke}%
\pgfsetdash{}{0pt}%
\pgfsys@defobject{currentmarker}{\pgfqpoint{0.000000in}{0.000000in}}{\pgfqpoint{0.048611in}{0.000000in}}{%
\pgfpathmoveto{\pgfqpoint{0.000000in}{0.000000in}}%
\pgfpathlineto{\pgfqpoint{0.048611in}{0.000000in}}%
\pgfusepath{stroke,fill}%
}%
\begin{pgfscope}%
\pgfsys@transformshift{4.604500in}{2.382990in}%
\pgfsys@useobject{currentmarker}{}%
\end{pgfscope}%
\end{pgfscope}%
\begin{pgfscope}%
\pgfsetbuttcap%
\pgfsetroundjoin%
\definecolor{currentfill}{rgb}{0.000000,0.000000,0.000000}%
\pgfsetfillcolor{currentfill}%
\pgfsetlinewidth{0.803000pt}%
\definecolor{currentstroke}{rgb}{0.000000,0.000000,0.000000}%
\pgfsetstrokecolor{currentstroke}%
\pgfsetdash{}{0pt}%
\pgfsys@defobject{currentmarker}{\pgfqpoint{0.000000in}{0.000000in}}{\pgfqpoint{0.048611in}{0.000000in}}{%
\pgfpathmoveto{\pgfqpoint{0.000000in}{0.000000in}}%
\pgfpathlineto{\pgfqpoint{0.048611in}{0.000000in}}%
\pgfusepath{stroke,fill}%
}%
\begin{pgfscope}%
\pgfsys@transformshift{4.604500in}{2.464504in}%
\pgfsys@useobject{currentmarker}{}%
\end{pgfscope}%
\end{pgfscope}%
\begin{pgfscope}%
\pgftext[x=4.701722in,y=2.407111in,left,base]{\rmfamily\fontsize{12.000000}{14.400000}\selectfont \(\displaystyle 10^{0}\)}%
\end{pgfscope}%
\begin{pgfscope}%
\pgfsetbuttcap%
\pgfsetroundjoin%
\definecolor{currentfill}{rgb}{0.000000,0.000000,0.000000}%
\pgfsetfillcolor{currentfill}%
\pgfsetlinewidth{0.803000pt}%
\definecolor{currentstroke}{rgb}{0.000000,0.000000,0.000000}%
\pgfsetstrokecolor{currentstroke}%
\pgfsetdash{}{0pt}%
\pgfsys@defobject{currentmarker}{\pgfqpoint{0.000000in}{0.000000in}}{\pgfqpoint{0.048611in}{0.000000in}}{%
\pgfpathmoveto{\pgfqpoint{0.000000in}{0.000000in}}%
\pgfpathlineto{\pgfqpoint{0.048611in}{0.000000in}}%
\pgfusepath{stroke,fill}%
}%
\begin{pgfscope}%
\pgfsys@transformshift{4.604500in}{3.000770in}%
\pgfsys@useobject{currentmarker}{}%
\end{pgfscope}%
\end{pgfscope}%
\begin{pgfscope}%
\pgfsetbuttcap%
\pgfsetroundjoin%
\definecolor{currentfill}{rgb}{0.000000,0.000000,0.000000}%
\pgfsetfillcolor{currentfill}%
\pgfsetlinewidth{0.803000pt}%
\definecolor{currentstroke}{rgb}{0.000000,0.000000,0.000000}%
\pgfsetstrokecolor{currentstroke}%
\pgfsetdash{}{0pt}%
\pgfsys@defobject{currentmarker}{\pgfqpoint{0.000000in}{0.000000in}}{\pgfqpoint{0.048611in}{0.000000in}}{%
\pgfpathmoveto{\pgfqpoint{0.000000in}{0.000000in}}%
\pgfpathlineto{\pgfqpoint{0.048611in}{0.000000in}}%
\pgfusepath{stroke,fill}%
}%
\begin{pgfscope}%
\pgfsys@transformshift{4.604500in}{3.314465in}%
\pgfsys@useobject{currentmarker}{}%
\end{pgfscope}%
\end{pgfscope}%
\begin{pgfscope}%
\pgfsetbuttcap%
\pgfsetroundjoin%
\definecolor{currentfill}{rgb}{0.000000,0.000000,0.000000}%
\pgfsetfillcolor{currentfill}%
\pgfsetlinewidth{0.803000pt}%
\definecolor{currentstroke}{rgb}{0.000000,0.000000,0.000000}%
\pgfsetstrokecolor{currentstroke}%
\pgfsetdash{}{0pt}%
\pgfsys@defobject{currentmarker}{\pgfqpoint{0.000000in}{0.000000in}}{\pgfqpoint{0.048611in}{0.000000in}}{%
\pgfpathmoveto{\pgfqpoint{0.000000in}{0.000000in}}%
\pgfpathlineto{\pgfqpoint{0.048611in}{0.000000in}}%
\pgfusepath{stroke,fill}%
}%
\begin{pgfscope}%
\pgfsys@transformshift{4.604500in}{3.537035in}%
\pgfsys@useobject{currentmarker}{}%
\end{pgfscope}%
\end{pgfscope}%
\begin{pgfscope}%
\pgftext[x=5.217155in,y=2.076590in,,top]{\rmfamily\fontsize{12.000000}{14.400000}\selectfont \(\displaystyle {\mathbf{E} \mbox{u}}_e \equiv \frac{m U_0^2}{q E_0 L}\)}%
\end{pgfscope}%
\begin{pgfscope}%
\pgfsetbuttcap%
\pgfsetmiterjoin%
\pgfsetlinewidth{0.803000pt}%
\definecolor{currentstroke}{rgb}{0.000000,0.000000,0.000000}%
\pgfsetstrokecolor{currentstroke}%
\pgfsetdash{}{0pt}%
\pgfpathmoveto{\pgfqpoint{4.453500in}{0.566590in}}%
\pgfpathlineto{\pgfqpoint{4.453500in}{0.578387in}}%
\pgfpathlineto{\pgfqpoint{4.453500in}{3.574793in}}%
\pgfpathlineto{\pgfqpoint{4.453500in}{3.586590in}}%
\pgfpathlineto{\pgfqpoint{4.604500in}{3.586590in}}%
\pgfpathlineto{\pgfqpoint{4.604500in}{3.574793in}}%
\pgfpathlineto{\pgfqpoint{4.604500in}{0.578387in}}%
\pgfpathlineto{\pgfqpoint{4.604500in}{0.566590in}}%
\pgfpathclose%
\pgfusepath{stroke}%
\end{pgfscope}%
\end{pgfpicture}%
\makeatother%
\endgroup%

    \caption{A simple EMA plot.\label{fig:series}}
\end{figure}

\begin{figure}[htb]
    \centering
    \resizebox{14cm}{!}{%% Creator: Matplotlib, PGF backend
%%
%% To include the figure in your LaTeX document, write
%%   \input{<filename>.pgf}
%%
%% Make sure the required packages are loaded in your preamble
%%   \usepackage{pgf}
%%
%% Figures using additional raster images can only be included by \input if
%% they are in the same directory as the main LaTeX file. For loading figures
%% from other directories you can use the `import` package
%%   \usepackage{import}
%% and then include the figures with
%%   \import{<path to file>}{<filename>.pgf}
%%
%% Matplotlib used the following preamble
%%   \usepackage{fontspec}
%%   \setmainfont{DejaVu Serif}
%%   \setsansfont{DejaVu Sans}
%%   \setmonofont{DejaVu Sans Mono}
%%
\begingroup%
\makeatletter%
\begin{pgfpicture}%
\pgfpathrectangle{\pgfpointorigin}{\pgfqpoint{12.806532in}{8.493808in}}%
\pgfusepath{use as bounding box, clip}%
\begin{pgfscope}%
\pgfsetbuttcap%
\pgfsetmiterjoin%
\definecolor{currentfill}{rgb}{1.000000,1.000000,1.000000}%
\pgfsetfillcolor{currentfill}%
\pgfsetlinewidth{0.000000pt}%
\definecolor{currentstroke}{rgb}{1.000000,1.000000,1.000000}%
\pgfsetstrokecolor{currentstroke}%
\pgfsetdash{}{0pt}%
\pgfpathmoveto{\pgfqpoint{0.000000in}{0.000000in}}%
\pgfpathlineto{\pgfqpoint{12.806532in}{0.000000in}}%
\pgfpathlineto{\pgfqpoint{12.806532in}{8.493808in}}%
\pgfpathlineto{\pgfqpoint{0.000000in}{8.493808in}}%
\pgfpathclose%
\pgfusepath{fill}%
\end{pgfscope}%
\begin{pgfscope}%
\pgfsetbuttcap%
\pgfsetmiterjoin%
\definecolor{currentfill}{rgb}{1.000000,1.000000,1.000000}%
\pgfsetfillcolor{currentfill}%
\pgfsetlinewidth{0.000000pt}%
\definecolor{currentstroke}{rgb}{0.000000,0.000000,0.000000}%
\pgfsetstrokecolor{currentstroke}%
\pgfsetstrokeopacity{0.000000}%
\pgfsetdash{}{0pt}%
\pgfpathmoveto{\pgfqpoint{0.880000in}{6.967719in}}%
\pgfpathlineto{\pgfqpoint{2.777959in}{6.967719in}}%
\pgfpathlineto{\pgfqpoint{2.777959in}{8.340446in}}%
\pgfpathlineto{\pgfqpoint{0.880000in}{8.340446in}}%
\pgfpathclose%
\pgfusepath{fill}%
\end{pgfscope}%
\begin{pgfscope}%
\pgfsetbuttcap%
\pgfsetroundjoin%
\definecolor{currentfill}{rgb}{0.000000,0.000000,0.000000}%
\pgfsetfillcolor{currentfill}%
\pgfsetlinewidth{0.803000pt}%
\definecolor{currentstroke}{rgb}{0.000000,0.000000,0.000000}%
\pgfsetstrokecolor{currentstroke}%
\pgfsetdash{}{0pt}%
\pgfsys@defobject{currentmarker}{\pgfqpoint{0.000000in}{-0.048611in}}{\pgfqpoint{0.000000in}{0.000000in}}{%
\pgfpathmoveto{\pgfqpoint{0.000000in}{0.000000in}}%
\pgfpathlineto{\pgfqpoint{0.000000in}{-0.048611in}}%
\pgfusepath{stroke,fill}%
}%
\begin{pgfscope}%
\pgfsys@transformshift{0.966271in}{6.967719in}%
\pgfsys@useobject{currentmarker}{}%
\end{pgfscope}%
\end{pgfscope}%
\begin{pgfscope}%
\pgftext[x=0.966271in,y=6.870496in,,top]{\rmfamily\fontsize{10.000000}{12.000000}\selectfont \(\displaystyle 0.075\)}%
\end{pgfscope}%
\begin{pgfscope}%
\pgfsetbuttcap%
\pgfsetroundjoin%
\definecolor{currentfill}{rgb}{0.000000,0.000000,0.000000}%
\pgfsetfillcolor{currentfill}%
\pgfsetlinewidth{0.803000pt}%
\definecolor{currentstroke}{rgb}{0.000000,0.000000,0.000000}%
\pgfsetstrokecolor{currentstroke}%
\pgfsetdash{}{0pt}%
\pgfsys@defobject{currentmarker}{\pgfqpoint{0.000000in}{-0.048611in}}{\pgfqpoint{0.000000in}{0.000000in}}{%
\pgfpathmoveto{\pgfqpoint{0.000000in}{0.000000in}}%
\pgfpathlineto{\pgfqpoint{0.000000in}{-0.048611in}}%
\pgfusepath{stroke,fill}%
}%
\begin{pgfscope}%
\pgfsys@transformshift{1.541410in}{6.967719in}%
\pgfsys@useobject{currentmarker}{}%
\end{pgfscope}%
\end{pgfscope}%
\begin{pgfscope}%
\pgftext[x=1.541410in,y=6.870496in,,top]{\rmfamily\fontsize{10.000000}{12.000000}\selectfont \(\displaystyle 0.100\)}%
\end{pgfscope}%
\begin{pgfscope}%
\pgfsetbuttcap%
\pgfsetroundjoin%
\definecolor{currentfill}{rgb}{0.000000,0.000000,0.000000}%
\pgfsetfillcolor{currentfill}%
\pgfsetlinewidth{0.803000pt}%
\definecolor{currentstroke}{rgb}{0.000000,0.000000,0.000000}%
\pgfsetstrokecolor{currentstroke}%
\pgfsetdash{}{0pt}%
\pgfsys@defobject{currentmarker}{\pgfqpoint{0.000000in}{-0.048611in}}{\pgfqpoint{0.000000in}{0.000000in}}{%
\pgfpathmoveto{\pgfqpoint{0.000000in}{0.000000in}}%
\pgfpathlineto{\pgfqpoint{0.000000in}{-0.048611in}}%
\pgfusepath{stroke,fill}%
}%
\begin{pgfscope}%
\pgfsys@transformshift{2.116549in}{6.967719in}%
\pgfsys@useobject{currentmarker}{}%
\end{pgfscope}%
\end{pgfscope}%
\begin{pgfscope}%
\pgftext[x=2.116549in,y=6.870496in,,top]{\rmfamily\fontsize{10.000000}{12.000000}\selectfont \(\displaystyle 0.125\)}%
\end{pgfscope}%
\begin{pgfscope}%
\pgfsetbuttcap%
\pgfsetroundjoin%
\definecolor{currentfill}{rgb}{0.000000,0.000000,0.000000}%
\pgfsetfillcolor{currentfill}%
\pgfsetlinewidth{0.803000pt}%
\definecolor{currentstroke}{rgb}{0.000000,0.000000,0.000000}%
\pgfsetstrokecolor{currentstroke}%
\pgfsetdash{}{0pt}%
\pgfsys@defobject{currentmarker}{\pgfqpoint{0.000000in}{-0.048611in}}{\pgfqpoint{0.000000in}{0.000000in}}{%
\pgfpathmoveto{\pgfqpoint{0.000000in}{0.000000in}}%
\pgfpathlineto{\pgfqpoint{0.000000in}{-0.048611in}}%
\pgfusepath{stroke,fill}%
}%
\begin{pgfscope}%
\pgfsys@transformshift{2.691688in}{6.967719in}%
\pgfsys@useobject{currentmarker}{}%
\end{pgfscope}%
\end{pgfscope}%
\begin{pgfscope}%
\pgftext[x=2.691688in,y=6.870496in,,top]{\rmfamily\fontsize{10.000000}{12.000000}\selectfont \(\displaystyle 0.150\)}%
\end{pgfscope}%
\begin{pgfscope}%
\pgfsetbuttcap%
\pgfsetroundjoin%
\definecolor{currentfill}{rgb}{0.000000,0.000000,0.000000}%
\pgfsetfillcolor{currentfill}%
\pgfsetlinewidth{0.803000pt}%
\definecolor{currentstroke}{rgb}{0.000000,0.000000,0.000000}%
\pgfsetstrokecolor{currentstroke}%
\pgfsetdash{}{0pt}%
\pgfsys@defobject{currentmarker}{\pgfqpoint{-0.048611in}{0.000000in}}{\pgfqpoint{0.000000in}{0.000000in}}{%
\pgfpathmoveto{\pgfqpoint{0.000000in}{0.000000in}}%
\pgfpathlineto{\pgfqpoint{-0.048611in}{0.000000in}}%
\pgfusepath{stroke,fill}%
}%
\begin{pgfscope}%
\pgfsys@transformshift{0.880000in}{7.505067in}%
\pgfsys@useobject{currentmarker}{}%
\end{pgfscope}%
\end{pgfscope}%
\begin{pgfscope}%
\pgftext[x=0.494775in,y=7.452305in,left,base]{\rmfamily\fontsize{10.000000}{12.000000}\selectfont \(\displaystyle 10^{-7}\)}%
\end{pgfscope}%
\begin{pgfscope}%
\pgfsetbuttcap%
\pgfsetroundjoin%
\definecolor{currentfill}{rgb}{0.000000,0.000000,0.000000}%
\pgfsetfillcolor{currentfill}%
\pgfsetlinewidth{0.803000pt}%
\definecolor{currentstroke}{rgb}{0.000000,0.000000,0.000000}%
\pgfsetstrokecolor{currentstroke}%
\pgfsetdash{}{0pt}%
\pgfsys@defobject{currentmarker}{\pgfqpoint{-0.048611in}{0.000000in}}{\pgfqpoint{0.000000in}{0.000000in}}{%
\pgfpathmoveto{\pgfqpoint{0.000000in}{0.000000in}}%
\pgfpathlineto{\pgfqpoint{-0.048611in}{0.000000in}}%
\pgfusepath{stroke,fill}%
}%
\begin{pgfscope}%
\pgfsys@transformshift{0.880000in}{8.094310in}%
\pgfsys@useobject{currentmarker}{}%
\end{pgfscope}%
\end{pgfscope}%
\begin{pgfscope}%
\pgftext[x=0.494775in,y=8.041548in,left,base]{\rmfamily\fontsize{10.000000}{12.000000}\selectfont \(\displaystyle 10^{-6}\)}%
\end{pgfscope}%
\begin{pgfscope}%
\pgfsetbuttcap%
\pgfsetroundjoin%
\definecolor{currentfill}{rgb}{0.000000,0.000000,0.000000}%
\pgfsetfillcolor{currentfill}%
\pgfsetlinewidth{0.602250pt}%
\definecolor{currentstroke}{rgb}{0.000000,0.000000,0.000000}%
\pgfsetstrokecolor{currentstroke}%
\pgfsetdash{}{0pt}%
\pgfsys@defobject{currentmarker}{\pgfqpoint{-0.027778in}{0.000000in}}{\pgfqpoint{0.000000in}{0.000000in}}{%
\pgfpathmoveto{\pgfqpoint{0.000000in}{0.000000in}}%
\pgfpathlineto{\pgfqpoint{-0.027778in}{0.000000in}}%
\pgfusepath{stroke,fill}%
}%
\begin{pgfscope}%
\pgfsys@transformshift{0.880000in}{7.093203in}%
\pgfsys@useobject{currentmarker}{}%
\end{pgfscope}%
\end{pgfscope}%
\begin{pgfscope}%
\pgfsetbuttcap%
\pgfsetroundjoin%
\definecolor{currentfill}{rgb}{0.000000,0.000000,0.000000}%
\pgfsetfillcolor{currentfill}%
\pgfsetlinewidth{0.602250pt}%
\definecolor{currentstroke}{rgb}{0.000000,0.000000,0.000000}%
\pgfsetstrokecolor{currentstroke}%
\pgfsetdash{}{0pt}%
\pgfsys@defobject{currentmarker}{\pgfqpoint{-0.027778in}{0.000000in}}{\pgfqpoint{0.000000in}{0.000000in}}{%
\pgfpathmoveto{\pgfqpoint{0.000000in}{0.000000in}}%
\pgfpathlineto{\pgfqpoint{-0.027778in}{0.000000in}}%
\pgfusepath{stroke,fill}%
}%
\begin{pgfscope}%
\pgfsys@transformshift{0.880000in}{7.196964in}%
\pgfsys@useobject{currentmarker}{}%
\end{pgfscope}%
\end{pgfscope}%
\begin{pgfscope}%
\pgfsetbuttcap%
\pgfsetroundjoin%
\definecolor{currentfill}{rgb}{0.000000,0.000000,0.000000}%
\pgfsetfillcolor{currentfill}%
\pgfsetlinewidth{0.602250pt}%
\definecolor{currentstroke}{rgb}{0.000000,0.000000,0.000000}%
\pgfsetstrokecolor{currentstroke}%
\pgfsetdash{}{0pt}%
\pgfsys@defobject{currentmarker}{\pgfqpoint{-0.027778in}{0.000000in}}{\pgfqpoint{0.000000in}{0.000000in}}{%
\pgfpathmoveto{\pgfqpoint{0.000000in}{0.000000in}}%
\pgfpathlineto{\pgfqpoint{-0.027778in}{0.000000in}}%
\pgfusepath{stroke,fill}%
}%
\begin{pgfscope}%
\pgfsys@transformshift{0.880000in}{7.270583in}%
\pgfsys@useobject{currentmarker}{}%
\end{pgfscope}%
\end{pgfscope}%
\begin{pgfscope}%
\pgfsetbuttcap%
\pgfsetroundjoin%
\definecolor{currentfill}{rgb}{0.000000,0.000000,0.000000}%
\pgfsetfillcolor{currentfill}%
\pgfsetlinewidth{0.602250pt}%
\definecolor{currentstroke}{rgb}{0.000000,0.000000,0.000000}%
\pgfsetstrokecolor{currentstroke}%
\pgfsetdash{}{0pt}%
\pgfsys@defobject{currentmarker}{\pgfqpoint{-0.027778in}{0.000000in}}{\pgfqpoint{0.000000in}{0.000000in}}{%
\pgfpathmoveto{\pgfqpoint{0.000000in}{0.000000in}}%
\pgfpathlineto{\pgfqpoint{-0.027778in}{0.000000in}}%
\pgfusepath{stroke,fill}%
}%
\begin{pgfscope}%
\pgfsys@transformshift{0.880000in}{7.327687in}%
\pgfsys@useobject{currentmarker}{}%
\end{pgfscope}%
\end{pgfscope}%
\begin{pgfscope}%
\pgfsetbuttcap%
\pgfsetroundjoin%
\definecolor{currentfill}{rgb}{0.000000,0.000000,0.000000}%
\pgfsetfillcolor{currentfill}%
\pgfsetlinewidth{0.602250pt}%
\definecolor{currentstroke}{rgb}{0.000000,0.000000,0.000000}%
\pgfsetstrokecolor{currentstroke}%
\pgfsetdash{}{0pt}%
\pgfsys@defobject{currentmarker}{\pgfqpoint{-0.027778in}{0.000000in}}{\pgfqpoint{0.000000in}{0.000000in}}{%
\pgfpathmoveto{\pgfqpoint{0.000000in}{0.000000in}}%
\pgfpathlineto{\pgfqpoint{-0.027778in}{0.000000in}}%
\pgfusepath{stroke,fill}%
}%
\begin{pgfscope}%
\pgfsys@transformshift{0.880000in}{7.374344in}%
\pgfsys@useobject{currentmarker}{}%
\end{pgfscope}%
\end{pgfscope}%
\begin{pgfscope}%
\pgfsetbuttcap%
\pgfsetroundjoin%
\definecolor{currentfill}{rgb}{0.000000,0.000000,0.000000}%
\pgfsetfillcolor{currentfill}%
\pgfsetlinewidth{0.602250pt}%
\definecolor{currentstroke}{rgb}{0.000000,0.000000,0.000000}%
\pgfsetstrokecolor{currentstroke}%
\pgfsetdash{}{0pt}%
\pgfsys@defobject{currentmarker}{\pgfqpoint{-0.027778in}{0.000000in}}{\pgfqpoint{0.000000in}{0.000000in}}{%
\pgfpathmoveto{\pgfqpoint{0.000000in}{0.000000in}}%
\pgfpathlineto{\pgfqpoint{-0.027778in}{0.000000in}}%
\pgfusepath{stroke,fill}%
}%
\begin{pgfscope}%
\pgfsys@transformshift{0.880000in}{7.413792in}%
\pgfsys@useobject{currentmarker}{}%
\end{pgfscope}%
\end{pgfscope}%
\begin{pgfscope}%
\pgfsetbuttcap%
\pgfsetroundjoin%
\definecolor{currentfill}{rgb}{0.000000,0.000000,0.000000}%
\pgfsetfillcolor{currentfill}%
\pgfsetlinewidth{0.602250pt}%
\definecolor{currentstroke}{rgb}{0.000000,0.000000,0.000000}%
\pgfsetstrokecolor{currentstroke}%
\pgfsetdash{}{0pt}%
\pgfsys@defobject{currentmarker}{\pgfqpoint{-0.027778in}{0.000000in}}{\pgfqpoint{0.000000in}{0.000000in}}{%
\pgfpathmoveto{\pgfqpoint{0.000000in}{0.000000in}}%
\pgfpathlineto{\pgfqpoint{-0.027778in}{0.000000in}}%
\pgfusepath{stroke,fill}%
}%
\begin{pgfscope}%
\pgfsys@transformshift{0.880000in}{7.447963in}%
\pgfsys@useobject{currentmarker}{}%
\end{pgfscope}%
\end{pgfscope}%
\begin{pgfscope}%
\pgfsetbuttcap%
\pgfsetroundjoin%
\definecolor{currentfill}{rgb}{0.000000,0.000000,0.000000}%
\pgfsetfillcolor{currentfill}%
\pgfsetlinewidth{0.602250pt}%
\definecolor{currentstroke}{rgb}{0.000000,0.000000,0.000000}%
\pgfsetstrokecolor{currentstroke}%
\pgfsetdash{}{0pt}%
\pgfsys@defobject{currentmarker}{\pgfqpoint{-0.027778in}{0.000000in}}{\pgfqpoint{0.000000in}{0.000000in}}{%
\pgfpathmoveto{\pgfqpoint{0.000000in}{0.000000in}}%
\pgfpathlineto{\pgfqpoint{-0.027778in}{0.000000in}}%
\pgfusepath{stroke,fill}%
}%
\begin{pgfscope}%
\pgfsys@transformshift{0.880000in}{7.478104in}%
\pgfsys@useobject{currentmarker}{}%
\end{pgfscope}%
\end{pgfscope}%
\begin{pgfscope}%
\pgfsetbuttcap%
\pgfsetroundjoin%
\definecolor{currentfill}{rgb}{0.000000,0.000000,0.000000}%
\pgfsetfillcolor{currentfill}%
\pgfsetlinewidth{0.602250pt}%
\definecolor{currentstroke}{rgb}{0.000000,0.000000,0.000000}%
\pgfsetstrokecolor{currentstroke}%
\pgfsetdash{}{0pt}%
\pgfsys@defobject{currentmarker}{\pgfqpoint{-0.027778in}{0.000000in}}{\pgfqpoint{0.000000in}{0.000000in}}{%
\pgfpathmoveto{\pgfqpoint{0.000000in}{0.000000in}}%
\pgfpathlineto{\pgfqpoint{-0.027778in}{0.000000in}}%
\pgfusepath{stroke,fill}%
}%
\begin{pgfscope}%
\pgfsys@transformshift{0.880000in}{7.682446in}%
\pgfsys@useobject{currentmarker}{}%
\end{pgfscope}%
\end{pgfscope}%
\begin{pgfscope}%
\pgfsetbuttcap%
\pgfsetroundjoin%
\definecolor{currentfill}{rgb}{0.000000,0.000000,0.000000}%
\pgfsetfillcolor{currentfill}%
\pgfsetlinewidth{0.602250pt}%
\definecolor{currentstroke}{rgb}{0.000000,0.000000,0.000000}%
\pgfsetstrokecolor{currentstroke}%
\pgfsetdash{}{0pt}%
\pgfsys@defobject{currentmarker}{\pgfqpoint{-0.027778in}{0.000000in}}{\pgfqpoint{0.000000in}{0.000000in}}{%
\pgfpathmoveto{\pgfqpoint{0.000000in}{0.000000in}}%
\pgfpathlineto{\pgfqpoint{-0.027778in}{0.000000in}}%
\pgfusepath{stroke,fill}%
}%
\begin{pgfscope}%
\pgfsys@transformshift{0.880000in}{7.786207in}%
\pgfsys@useobject{currentmarker}{}%
\end{pgfscope}%
\end{pgfscope}%
\begin{pgfscope}%
\pgfsetbuttcap%
\pgfsetroundjoin%
\definecolor{currentfill}{rgb}{0.000000,0.000000,0.000000}%
\pgfsetfillcolor{currentfill}%
\pgfsetlinewidth{0.602250pt}%
\definecolor{currentstroke}{rgb}{0.000000,0.000000,0.000000}%
\pgfsetstrokecolor{currentstroke}%
\pgfsetdash{}{0pt}%
\pgfsys@defobject{currentmarker}{\pgfqpoint{-0.027778in}{0.000000in}}{\pgfqpoint{0.000000in}{0.000000in}}{%
\pgfpathmoveto{\pgfqpoint{0.000000in}{0.000000in}}%
\pgfpathlineto{\pgfqpoint{-0.027778in}{0.000000in}}%
\pgfusepath{stroke,fill}%
}%
\begin{pgfscope}%
\pgfsys@transformshift{0.880000in}{7.859826in}%
\pgfsys@useobject{currentmarker}{}%
\end{pgfscope}%
\end{pgfscope}%
\begin{pgfscope}%
\pgfsetbuttcap%
\pgfsetroundjoin%
\definecolor{currentfill}{rgb}{0.000000,0.000000,0.000000}%
\pgfsetfillcolor{currentfill}%
\pgfsetlinewidth{0.602250pt}%
\definecolor{currentstroke}{rgb}{0.000000,0.000000,0.000000}%
\pgfsetstrokecolor{currentstroke}%
\pgfsetdash{}{0pt}%
\pgfsys@defobject{currentmarker}{\pgfqpoint{-0.027778in}{0.000000in}}{\pgfqpoint{0.000000in}{0.000000in}}{%
\pgfpathmoveto{\pgfqpoint{0.000000in}{0.000000in}}%
\pgfpathlineto{\pgfqpoint{-0.027778in}{0.000000in}}%
\pgfusepath{stroke,fill}%
}%
\begin{pgfscope}%
\pgfsys@transformshift{0.880000in}{7.916930in}%
\pgfsys@useobject{currentmarker}{}%
\end{pgfscope}%
\end{pgfscope}%
\begin{pgfscope}%
\pgfsetbuttcap%
\pgfsetroundjoin%
\definecolor{currentfill}{rgb}{0.000000,0.000000,0.000000}%
\pgfsetfillcolor{currentfill}%
\pgfsetlinewidth{0.602250pt}%
\definecolor{currentstroke}{rgb}{0.000000,0.000000,0.000000}%
\pgfsetstrokecolor{currentstroke}%
\pgfsetdash{}{0pt}%
\pgfsys@defobject{currentmarker}{\pgfqpoint{-0.027778in}{0.000000in}}{\pgfqpoint{0.000000in}{0.000000in}}{%
\pgfpathmoveto{\pgfqpoint{0.000000in}{0.000000in}}%
\pgfpathlineto{\pgfqpoint{-0.027778in}{0.000000in}}%
\pgfusepath{stroke,fill}%
}%
\begin{pgfscope}%
\pgfsys@transformshift{0.880000in}{7.963587in}%
\pgfsys@useobject{currentmarker}{}%
\end{pgfscope}%
\end{pgfscope}%
\begin{pgfscope}%
\pgfsetbuttcap%
\pgfsetroundjoin%
\definecolor{currentfill}{rgb}{0.000000,0.000000,0.000000}%
\pgfsetfillcolor{currentfill}%
\pgfsetlinewidth{0.602250pt}%
\definecolor{currentstroke}{rgb}{0.000000,0.000000,0.000000}%
\pgfsetstrokecolor{currentstroke}%
\pgfsetdash{}{0pt}%
\pgfsys@defobject{currentmarker}{\pgfqpoint{-0.027778in}{0.000000in}}{\pgfqpoint{0.000000in}{0.000000in}}{%
\pgfpathmoveto{\pgfqpoint{0.000000in}{0.000000in}}%
\pgfpathlineto{\pgfqpoint{-0.027778in}{0.000000in}}%
\pgfusepath{stroke,fill}%
}%
\begin{pgfscope}%
\pgfsys@transformshift{0.880000in}{8.003035in}%
\pgfsys@useobject{currentmarker}{}%
\end{pgfscope}%
\end{pgfscope}%
\begin{pgfscope}%
\pgfsetbuttcap%
\pgfsetroundjoin%
\definecolor{currentfill}{rgb}{0.000000,0.000000,0.000000}%
\pgfsetfillcolor{currentfill}%
\pgfsetlinewidth{0.602250pt}%
\definecolor{currentstroke}{rgb}{0.000000,0.000000,0.000000}%
\pgfsetstrokecolor{currentstroke}%
\pgfsetdash{}{0pt}%
\pgfsys@defobject{currentmarker}{\pgfqpoint{-0.027778in}{0.000000in}}{\pgfqpoint{0.000000in}{0.000000in}}{%
\pgfpathmoveto{\pgfqpoint{0.000000in}{0.000000in}}%
\pgfpathlineto{\pgfqpoint{-0.027778in}{0.000000in}}%
\pgfusepath{stroke,fill}%
}%
\begin{pgfscope}%
\pgfsys@transformshift{0.880000in}{8.037206in}%
\pgfsys@useobject{currentmarker}{}%
\end{pgfscope}%
\end{pgfscope}%
\begin{pgfscope}%
\pgfsetbuttcap%
\pgfsetroundjoin%
\definecolor{currentfill}{rgb}{0.000000,0.000000,0.000000}%
\pgfsetfillcolor{currentfill}%
\pgfsetlinewidth{0.602250pt}%
\definecolor{currentstroke}{rgb}{0.000000,0.000000,0.000000}%
\pgfsetstrokecolor{currentstroke}%
\pgfsetdash{}{0pt}%
\pgfsys@defobject{currentmarker}{\pgfqpoint{-0.027778in}{0.000000in}}{\pgfqpoint{0.000000in}{0.000000in}}{%
\pgfpathmoveto{\pgfqpoint{0.000000in}{0.000000in}}%
\pgfpathlineto{\pgfqpoint{-0.027778in}{0.000000in}}%
\pgfusepath{stroke,fill}%
}%
\begin{pgfscope}%
\pgfsys@transformshift{0.880000in}{8.067347in}%
\pgfsys@useobject{currentmarker}{}%
\end{pgfscope}%
\end{pgfscope}%
\begin{pgfscope}%
\pgfsetbuttcap%
\pgfsetroundjoin%
\definecolor{currentfill}{rgb}{0.000000,0.000000,0.000000}%
\pgfsetfillcolor{currentfill}%
\pgfsetlinewidth{0.602250pt}%
\definecolor{currentstroke}{rgb}{0.000000,0.000000,0.000000}%
\pgfsetstrokecolor{currentstroke}%
\pgfsetdash{}{0pt}%
\pgfsys@defobject{currentmarker}{\pgfqpoint{-0.027778in}{0.000000in}}{\pgfqpoint{0.000000in}{0.000000in}}{%
\pgfpathmoveto{\pgfqpoint{0.000000in}{0.000000in}}%
\pgfpathlineto{\pgfqpoint{-0.027778in}{0.000000in}}%
\pgfusepath{stroke,fill}%
}%
\begin{pgfscope}%
\pgfsys@transformshift{0.880000in}{8.271690in}%
\pgfsys@useobject{currentmarker}{}%
\end{pgfscope}%
\end{pgfscope}%
\begin{pgfscope}%
\pgfpathrectangle{\pgfqpoint{0.880000in}{6.967719in}}{\pgfqpoint{1.897959in}{1.372727in}} %
\pgfusepath{clip}%
\pgfsetbuttcap%
\pgfsetroundjoin%
\pgfsetlinewidth{1.505625pt}%
\definecolor{currentstroke}{rgb}{1.000000,0.000000,0.000000}%
\pgfsetstrokecolor{currentstroke}%
\pgfsetdash{{5.550000pt}{2.400000pt}}{0.000000pt}%
\pgfpathmoveto{\pgfqpoint{0.966271in}{8.242165in}}%
\pgfpathlineto{\pgfqpoint{1.157984in}{8.240978in}}%
\pgfpathlineto{\pgfqpoint{1.349697in}{8.239882in}}%
\pgfpathlineto{\pgfqpoint{1.541410in}{8.238877in}}%
\pgfpathlineto{\pgfqpoint{1.733123in}{8.237961in}}%
\pgfpathlineto{\pgfqpoint{1.924836in}{8.237135in}}%
\pgfpathlineto{\pgfqpoint{2.116549in}{8.236398in}}%
\pgfpathlineto{\pgfqpoint{2.308262in}{8.235751in}}%
\pgfpathlineto{\pgfqpoint{2.499975in}{8.235193in}}%
\pgfpathlineto{\pgfqpoint{2.691688in}{8.234723in}}%
\pgfusepath{stroke}%
\end{pgfscope}%
\begin{pgfscope}%
\pgfpathrectangle{\pgfqpoint{0.880000in}{6.967719in}}{\pgfqpoint{1.897959in}{1.372727in}} %
\pgfusepath{clip}%
\pgfsetbuttcap%
\pgfsetmiterjoin%
\definecolor{currentfill}{rgb}{1.000000,0.000000,0.000000}%
\pgfsetfillcolor{currentfill}%
\pgfsetlinewidth{1.003750pt}%
\definecolor{currentstroke}{rgb}{1.000000,0.000000,0.000000}%
\pgfsetstrokecolor{currentstroke}%
\pgfsetdash{}{0pt}%
\pgfsys@defobject{currentmarker}{\pgfqpoint{-0.041667in}{-0.041667in}}{\pgfqpoint{0.041667in}{0.041667in}}{%
\pgfpathmoveto{\pgfqpoint{-0.041667in}{-0.041667in}}%
\pgfpathlineto{\pgfqpoint{0.041667in}{-0.041667in}}%
\pgfpathlineto{\pgfqpoint{0.041667in}{0.041667in}}%
\pgfpathlineto{\pgfqpoint{-0.041667in}{0.041667in}}%
\pgfpathclose%
\pgfusepath{stroke,fill}%
}%
\begin{pgfscope}%
\pgfsys@transformshift{0.966271in}{8.242165in}%
\pgfsys@useobject{currentmarker}{}%
\end{pgfscope}%
\begin{pgfscope}%
\pgfsys@transformshift{1.349697in}{8.239882in}%
\pgfsys@useobject{currentmarker}{}%
\end{pgfscope}%
\begin{pgfscope}%
\pgfsys@transformshift{1.733123in}{8.237961in}%
\pgfsys@useobject{currentmarker}{}%
\end{pgfscope}%
\begin{pgfscope}%
\pgfsys@transformshift{2.116549in}{8.236398in}%
\pgfsys@useobject{currentmarker}{}%
\end{pgfscope}%
\begin{pgfscope}%
\pgfsys@transformshift{2.499975in}{8.235193in}%
\pgfsys@useobject{currentmarker}{}%
\end{pgfscope}%
\end{pgfscope}%
\begin{pgfscope}%
\pgfpathrectangle{\pgfqpoint{0.880000in}{6.967719in}}{\pgfqpoint{1.897959in}{1.372727in}} %
\pgfusepath{clip}%
\pgfsetrectcap%
\pgfsetroundjoin%
\pgfsetlinewidth{1.505625pt}%
\definecolor{currentstroke}{rgb}{0.000000,0.000000,1.000000}%
\pgfsetstrokecolor{currentstroke}%
\pgfsetdash{}{0pt}%
\pgfpathmoveto{\pgfqpoint{0.966271in}{7.344411in}}%
\pgfpathlineto{\pgfqpoint{1.157984in}{7.320663in}}%
\pgfpathlineto{\pgfqpoint{1.349697in}{7.295335in}}%
\pgfpathlineto{\pgfqpoint{1.541410in}{7.268119in}}%
\pgfpathlineto{\pgfqpoint{1.733123in}{7.238624in}}%
\pgfpathlineto{\pgfqpoint{1.924836in}{7.206340in}}%
\pgfpathlineto{\pgfqpoint{2.116549in}{7.170577in}}%
\pgfpathlineto{\pgfqpoint{2.308262in}{7.130367in}}%
\pgfpathlineto{\pgfqpoint{2.499975in}{7.084288in}}%
\pgfpathlineto{\pgfqpoint{2.691688in}{7.030115in}}%
\pgfusepath{stroke}%
\end{pgfscope}%
\begin{pgfscope}%
\pgfpathrectangle{\pgfqpoint{0.880000in}{6.967719in}}{\pgfqpoint{1.897959in}{1.372727in}} %
\pgfusepath{clip}%
\pgfsetbuttcap%
\pgfsetroundjoin%
\definecolor{currentfill}{rgb}{0.000000,0.000000,1.000000}%
\pgfsetfillcolor{currentfill}%
\pgfsetlinewidth{1.003750pt}%
\definecolor{currentstroke}{rgb}{0.000000,0.000000,1.000000}%
\pgfsetstrokecolor{currentstroke}%
\pgfsetdash{}{0pt}%
\pgfsys@defobject{currentmarker}{\pgfqpoint{-0.041667in}{-0.041667in}}{\pgfqpoint{0.041667in}{0.041667in}}{%
\pgfpathmoveto{\pgfqpoint{0.000000in}{-0.041667in}}%
\pgfpathcurveto{\pgfqpoint{0.011050in}{-0.041667in}}{\pgfqpoint{0.021649in}{-0.037276in}}{\pgfqpoint{0.029463in}{-0.029463in}}%
\pgfpathcurveto{\pgfqpoint{0.037276in}{-0.021649in}}{\pgfqpoint{0.041667in}{-0.011050in}}{\pgfqpoint{0.041667in}{0.000000in}}%
\pgfpathcurveto{\pgfqpoint{0.041667in}{0.011050in}}{\pgfqpoint{0.037276in}{0.021649in}}{\pgfqpoint{0.029463in}{0.029463in}}%
\pgfpathcurveto{\pgfqpoint{0.021649in}{0.037276in}}{\pgfqpoint{0.011050in}{0.041667in}}{\pgfqpoint{0.000000in}{0.041667in}}%
\pgfpathcurveto{\pgfqpoint{-0.011050in}{0.041667in}}{\pgfqpoint{-0.021649in}{0.037276in}}{\pgfqpoint{-0.029463in}{0.029463in}}%
\pgfpathcurveto{\pgfqpoint{-0.037276in}{0.021649in}}{\pgfqpoint{-0.041667in}{0.011050in}}{\pgfqpoint{-0.041667in}{0.000000in}}%
\pgfpathcurveto{\pgfqpoint{-0.041667in}{-0.011050in}}{\pgfqpoint{-0.037276in}{-0.021649in}}{\pgfqpoint{-0.029463in}{-0.029463in}}%
\pgfpathcurveto{\pgfqpoint{-0.021649in}{-0.037276in}}{\pgfqpoint{-0.011050in}{-0.041667in}}{\pgfqpoint{0.000000in}{-0.041667in}}%
\pgfpathclose%
\pgfusepath{stroke,fill}%
}%
\begin{pgfscope}%
\pgfsys@transformshift{0.966271in}{7.344411in}%
\pgfsys@useobject{currentmarker}{}%
\end{pgfscope}%
\begin{pgfscope}%
\pgfsys@transformshift{1.349697in}{7.295335in}%
\pgfsys@useobject{currentmarker}{}%
\end{pgfscope}%
\begin{pgfscope}%
\pgfsys@transformshift{1.733123in}{7.238624in}%
\pgfsys@useobject{currentmarker}{}%
\end{pgfscope}%
\begin{pgfscope}%
\pgfsys@transformshift{2.116549in}{7.170577in}%
\pgfsys@useobject{currentmarker}{}%
\end{pgfscope}%
\begin{pgfscope}%
\pgfsys@transformshift{2.499975in}{7.084288in}%
\pgfsys@useobject{currentmarker}{}%
\end{pgfscope}%
\end{pgfscope}%
\begin{pgfscope}%
\pgfpathrectangle{\pgfqpoint{0.880000in}{6.967719in}}{\pgfqpoint{1.897959in}{1.372727in}} %
\pgfusepath{clip}%
\pgfsetbuttcap%
\pgfsetroundjoin%
\pgfsetlinewidth{1.505625pt}%
\definecolor{currentstroke}{rgb}{0.000000,0.750000,0.750000}%
\pgfsetstrokecolor{currentstroke}%
\pgfsetdash{{9.600000pt}{2.400000pt}{1.500000pt}{2.400000pt}}{0.000000pt}%
\pgfpathmoveto{\pgfqpoint{0.966271in}{7.702201in}}%
\pgfpathlineto{\pgfqpoint{1.157984in}{7.683138in}}%
\pgfpathlineto{\pgfqpoint{1.349697in}{7.666190in}}%
\pgfpathlineto{\pgfqpoint{1.541410in}{7.651157in}}%
\pgfpathlineto{\pgfqpoint{1.733123in}{7.637876in}}%
\pgfpathlineto{\pgfqpoint{1.924836in}{7.626211in}}%
\pgfpathlineto{\pgfqpoint{2.116549in}{7.616049in}}%
\pgfpathlineto{\pgfqpoint{2.308262in}{7.607297in}}%
\pgfpathlineto{\pgfqpoint{2.499975in}{7.599880in}}%
\pgfpathlineto{\pgfqpoint{2.691688in}{7.593734in}}%
\pgfusepath{stroke}%
\end{pgfscope}%
\begin{pgfscope}%
\pgfpathrectangle{\pgfqpoint{0.880000in}{6.967719in}}{\pgfqpoint{1.897959in}{1.372727in}} %
\pgfusepath{clip}%
\pgfsetbuttcap%
\pgfsetmiterjoin%
\definecolor{currentfill}{rgb}{0.000000,0.750000,0.750000}%
\pgfsetfillcolor{currentfill}%
\pgfsetlinewidth{1.003750pt}%
\definecolor{currentstroke}{rgb}{0.000000,0.750000,0.750000}%
\pgfsetstrokecolor{currentstroke}%
\pgfsetdash{}{0pt}%
\pgfsys@defobject{currentmarker}{\pgfqpoint{-0.041667in}{-0.041667in}}{\pgfqpoint{0.041667in}{0.041667in}}{%
\pgfpathmoveto{\pgfqpoint{-0.000000in}{-0.041667in}}%
\pgfpathlineto{\pgfqpoint{0.041667in}{0.041667in}}%
\pgfpathlineto{\pgfqpoint{-0.041667in}{0.041667in}}%
\pgfpathclose%
\pgfusepath{stroke,fill}%
}%
\begin{pgfscope}%
\pgfsys@transformshift{0.966271in}{7.702201in}%
\pgfsys@useobject{currentmarker}{}%
\end{pgfscope}%
\begin{pgfscope}%
\pgfsys@transformshift{1.349697in}{7.666190in}%
\pgfsys@useobject{currentmarker}{}%
\end{pgfscope}%
\begin{pgfscope}%
\pgfsys@transformshift{1.733123in}{7.637876in}%
\pgfsys@useobject{currentmarker}{}%
\end{pgfscope}%
\begin{pgfscope}%
\pgfsys@transformshift{2.116549in}{7.616049in}%
\pgfsys@useobject{currentmarker}{}%
\end{pgfscope}%
\begin{pgfscope}%
\pgfsys@transformshift{2.499975in}{7.599880in}%
\pgfsys@useobject{currentmarker}{}%
\end{pgfscope}%
\end{pgfscope}%
\begin{pgfscope}%
\pgfpathrectangle{\pgfqpoint{0.880000in}{6.967719in}}{\pgfqpoint{1.897959in}{1.372727in}} %
\pgfusepath{clip}%
\pgfsetbuttcap%
\pgfsetroundjoin%
\pgfsetlinewidth{1.505625pt}%
\definecolor{currentstroke}{rgb}{0.000000,0.000000,0.000000}%
\pgfsetstrokecolor{currentstroke}%
\pgfsetdash{{1.500000pt}{2.475000pt}}{0.000000pt}%
\pgfpathmoveto{\pgfqpoint{0.966271in}{8.278049in}}%
\pgfpathlineto{\pgfqpoint{1.157984in}{8.274690in}}%
\pgfpathlineto{\pgfqpoint{1.349697in}{8.271493in}}%
\pgfpathlineto{\pgfqpoint{1.541410in}{8.268643in}}%
\pgfpathlineto{\pgfqpoint{1.733123in}{8.266093in}}%
\pgfpathlineto{\pgfqpoint{1.924836in}{8.263813in}}%
\pgfpathlineto{\pgfqpoint{2.116549in}{8.261778in}}%
\pgfpathlineto{\pgfqpoint{2.308262in}{8.259969in}}%
\pgfpathlineto{\pgfqpoint{2.499975in}{8.258373in}}%
\pgfpathlineto{\pgfqpoint{2.691688in}{8.256980in}}%
\pgfusepath{stroke}%
\end{pgfscope}%
\begin{pgfscope}%
\pgfpathrectangle{\pgfqpoint{0.880000in}{6.967719in}}{\pgfqpoint{1.897959in}{1.372727in}} %
\pgfusepath{clip}%
\pgfsetbuttcap%
\pgfsetroundjoin%
\definecolor{currentfill}{rgb}{0.000000,0.000000,0.000000}%
\pgfsetfillcolor{currentfill}%
\pgfsetlinewidth{1.003750pt}%
\definecolor{currentstroke}{rgb}{0.000000,0.000000,0.000000}%
\pgfsetstrokecolor{currentstroke}%
\pgfsetdash{}{0pt}%
\pgfsys@defobject{currentmarker}{\pgfqpoint{-0.041667in}{-0.041667in}}{\pgfqpoint{0.041667in}{0.041667in}}{%
\pgfpathmoveto{\pgfqpoint{-0.041667in}{0.000000in}}%
\pgfpathlineto{\pgfqpoint{0.041667in}{0.000000in}}%
\pgfpathmoveto{\pgfqpoint{0.000000in}{-0.041667in}}%
\pgfpathlineto{\pgfqpoint{0.000000in}{0.041667in}}%
\pgfusepath{stroke,fill}%
}%
\begin{pgfscope}%
\pgfsys@transformshift{0.966271in}{8.278049in}%
\pgfsys@useobject{currentmarker}{}%
\end{pgfscope}%
\begin{pgfscope}%
\pgfsys@transformshift{1.349697in}{8.271493in}%
\pgfsys@useobject{currentmarker}{}%
\end{pgfscope}%
\begin{pgfscope}%
\pgfsys@transformshift{1.733123in}{8.266093in}%
\pgfsys@useobject{currentmarker}{}%
\end{pgfscope}%
\begin{pgfscope}%
\pgfsys@transformshift{2.116549in}{8.261778in}%
\pgfsys@useobject{currentmarker}{}%
\end{pgfscope}%
\begin{pgfscope}%
\pgfsys@transformshift{2.499975in}{8.258373in}%
\pgfsys@useobject{currentmarker}{}%
\end{pgfscope}%
\end{pgfscope}%
\begin{pgfscope}%
\pgfsetrectcap%
\pgfsetmiterjoin%
\pgfsetlinewidth{0.803000pt}%
\definecolor{currentstroke}{rgb}{0.000000,0.000000,0.000000}%
\pgfsetstrokecolor{currentstroke}%
\pgfsetdash{}{0pt}%
\pgfpathmoveto{\pgfqpoint{0.880000in}{6.967719in}}%
\pgfpathlineto{\pgfqpoint{0.880000in}{8.340446in}}%
\pgfusepath{stroke}%
\end{pgfscope}%
\begin{pgfscope}%
\pgfsetrectcap%
\pgfsetmiterjoin%
\pgfsetlinewidth{0.803000pt}%
\definecolor{currentstroke}{rgb}{0.000000,0.000000,0.000000}%
\pgfsetstrokecolor{currentstroke}%
\pgfsetdash{}{0pt}%
\pgfpathmoveto{\pgfqpoint{2.777959in}{6.967719in}}%
\pgfpathlineto{\pgfqpoint{2.777959in}{8.340446in}}%
\pgfusepath{stroke}%
\end{pgfscope}%
\begin{pgfscope}%
\pgfsetrectcap%
\pgfsetmiterjoin%
\pgfsetlinewidth{0.803000pt}%
\definecolor{currentstroke}{rgb}{0.000000,0.000000,0.000000}%
\pgfsetstrokecolor{currentstroke}%
\pgfsetdash{}{0pt}%
\pgfpathmoveto{\pgfqpoint{0.880000in}{6.967719in}}%
\pgfpathlineto{\pgfqpoint{2.777959in}{6.967719in}}%
\pgfusepath{stroke}%
\end{pgfscope}%
\begin{pgfscope}%
\pgfsetrectcap%
\pgfsetmiterjoin%
\pgfsetlinewidth{0.803000pt}%
\definecolor{currentstroke}{rgb}{0.000000,0.000000,0.000000}%
\pgfsetstrokecolor{currentstroke}%
\pgfsetdash{}{0pt}%
\pgfpathmoveto{\pgfqpoint{0.880000in}{8.340446in}}%
\pgfpathlineto{\pgfqpoint{2.777959in}{8.340446in}}%
\pgfusepath{stroke}%
\end{pgfscope}%
\begin{pgfscope}%
\pgfsetbuttcap%
\pgfsetmiterjoin%
\definecolor{currentfill}{rgb}{1.000000,1.000000,1.000000}%
\pgfsetfillcolor{currentfill}%
\pgfsetlinewidth{0.000000pt}%
\definecolor{currentstroke}{rgb}{0.000000,0.000000,0.000000}%
\pgfsetstrokecolor{currentstroke}%
\pgfsetstrokeopacity{0.000000}%
\pgfsetdash{}{0pt}%
\pgfpathmoveto{\pgfqpoint{3.347347in}{6.967719in}}%
\pgfpathlineto{\pgfqpoint{5.245306in}{6.967719in}}%
\pgfpathlineto{\pgfqpoint{5.245306in}{8.340446in}}%
\pgfpathlineto{\pgfqpoint{3.347347in}{8.340446in}}%
\pgfpathclose%
\pgfusepath{fill}%
\end{pgfscope}%
\begin{pgfscope}%
\pgfsetbuttcap%
\pgfsetroundjoin%
\definecolor{currentfill}{rgb}{0.000000,0.000000,0.000000}%
\pgfsetfillcolor{currentfill}%
\pgfsetlinewidth{0.803000pt}%
\definecolor{currentstroke}{rgb}{0.000000,0.000000,0.000000}%
\pgfsetstrokecolor{currentstroke}%
\pgfsetdash{}{0pt}%
\pgfsys@defobject{currentmarker}{\pgfqpoint{0.000000in}{-0.048611in}}{\pgfqpoint{0.000000in}{0.000000in}}{%
\pgfpathmoveto{\pgfqpoint{0.000000in}{0.000000in}}%
\pgfpathlineto{\pgfqpoint{0.000000in}{-0.048611in}}%
\pgfusepath{stroke,fill}%
}%
\begin{pgfscope}%
\pgfsys@transformshift{3.649295in}{6.967719in}%
\pgfsys@useobject{currentmarker}{}%
\end{pgfscope}%
\end{pgfscope}%
\begin{pgfscope}%
\pgftext[x=3.649295in,y=6.870496in,,top]{\rmfamily\fontsize{10.000000}{12.000000}\selectfont \(\displaystyle 0.10\)}%
\end{pgfscope}%
\begin{pgfscope}%
\pgfsetbuttcap%
\pgfsetroundjoin%
\definecolor{currentfill}{rgb}{0.000000,0.000000,0.000000}%
\pgfsetfillcolor{currentfill}%
\pgfsetlinewidth{0.803000pt}%
\definecolor{currentstroke}{rgb}{0.000000,0.000000,0.000000}%
\pgfsetstrokecolor{currentstroke}%
\pgfsetdash{}{0pt}%
\pgfsys@defobject{currentmarker}{\pgfqpoint{0.000000in}{-0.048611in}}{\pgfqpoint{0.000000in}{0.000000in}}{%
\pgfpathmoveto{\pgfqpoint{0.000000in}{0.000000in}}%
\pgfpathlineto{\pgfqpoint{0.000000in}{-0.048611in}}%
\pgfusepath{stroke,fill}%
}%
\begin{pgfscope}%
\pgfsys@transformshift{4.166920in}{6.967719in}%
\pgfsys@useobject{currentmarker}{}%
\end{pgfscope}%
\end{pgfscope}%
\begin{pgfscope}%
\pgftext[x=4.166920in,y=6.870496in,,top]{\rmfamily\fontsize{10.000000}{12.000000}\selectfont \(\displaystyle 0.12\)}%
\end{pgfscope}%
\begin{pgfscope}%
\pgfsetbuttcap%
\pgfsetroundjoin%
\definecolor{currentfill}{rgb}{0.000000,0.000000,0.000000}%
\pgfsetfillcolor{currentfill}%
\pgfsetlinewidth{0.803000pt}%
\definecolor{currentstroke}{rgb}{0.000000,0.000000,0.000000}%
\pgfsetstrokecolor{currentstroke}%
\pgfsetdash{}{0pt}%
\pgfsys@defobject{currentmarker}{\pgfqpoint{0.000000in}{-0.048611in}}{\pgfqpoint{0.000000in}{0.000000in}}{%
\pgfpathmoveto{\pgfqpoint{0.000000in}{0.000000in}}%
\pgfpathlineto{\pgfqpoint{0.000000in}{-0.048611in}}%
\pgfusepath{stroke,fill}%
}%
\begin{pgfscope}%
\pgfsys@transformshift{4.684545in}{6.967719in}%
\pgfsys@useobject{currentmarker}{}%
\end{pgfscope}%
\end{pgfscope}%
\begin{pgfscope}%
\pgftext[x=4.684545in,y=6.870496in,,top]{\rmfamily\fontsize{10.000000}{12.000000}\selectfont \(\displaystyle 0.14\)}%
\end{pgfscope}%
\begin{pgfscope}%
\pgfsetbuttcap%
\pgfsetroundjoin%
\definecolor{currentfill}{rgb}{0.000000,0.000000,0.000000}%
\pgfsetfillcolor{currentfill}%
\pgfsetlinewidth{0.803000pt}%
\definecolor{currentstroke}{rgb}{0.000000,0.000000,0.000000}%
\pgfsetstrokecolor{currentstroke}%
\pgfsetdash{}{0pt}%
\pgfsys@defobject{currentmarker}{\pgfqpoint{0.000000in}{-0.048611in}}{\pgfqpoint{0.000000in}{0.000000in}}{%
\pgfpathmoveto{\pgfqpoint{0.000000in}{0.000000in}}%
\pgfpathlineto{\pgfqpoint{0.000000in}{-0.048611in}}%
\pgfusepath{stroke,fill}%
}%
\begin{pgfscope}%
\pgfsys@transformshift{5.202171in}{6.967719in}%
\pgfsys@useobject{currentmarker}{}%
\end{pgfscope}%
\end{pgfscope}%
\begin{pgfscope}%
\pgftext[x=5.202171in,y=6.870496in,,top]{\rmfamily\fontsize{10.000000}{12.000000}\selectfont \(\displaystyle 0.16\)}%
\end{pgfscope}%
\begin{pgfscope}%
\pgfsetbuttcap%
\pgfsetroundjoin%
\definecolor{currentfill}{rgb}{0.000000,0.000000,0.000000}%
\pgfsetfillcolor{currentfill}%
\pgfsetlinewidth{0.803000pt}%
\definecolor{currentstroke}{rgb}{0.000000,0.000000,0.000000}%
\pgfsetstrokecolor{currentstroke}%
\pgfsetdash{}{0pt}%
\pgfsys@defobject{currentmarker}{\pgfqpoint{-0.048611in}{0.000000in}}{\pgfqpoint{0.000000in}{0.000000in}}{%
\pgfpathmoveto{\pgfqpoint{0.000000in}{0.000000in}}%
\pgfpathlineto{\pgfqpoint{-0.048611in}{0.000000in}}%
\pgfusepath{stroke,fill}%
}%
\begin{pgfscope}%
\pgfsys@transformshift{3.347347in}{7.221568in}%
\pgfsys@useobject{currentmarker}{}%
\end{pgfscope}%
\end{pgfscope}%
\begin{pgfscope}%
\pgftext[x=2.962122in,y=7.168806in,left,base]{\rmfamily\fontsize{10.000000}{12.000000}\selectfont \(\displaystyle 10^{-7}\)}%
\end{pgfscope}%
\begin{pgfscope}%
\pgfsetbuttcap%
\pgfsetroundjoin%
\definecolor{currentfill}{rgb}{0.000000,0.000000,0.000000}%
\pgfsetfillcolor{currentfill}%
\pgfsetlinewidth{0.803000pt}%
\definecolor{currentstroke}{rgb}{0.000000,0.000000,0.000000}%
\pgfsetstrokecolor{currentstroke}%
\pgfsetdash{}{0pt}%
\pgfsys@defobject{currentmarker}{\pgfqpoint{-0.048611in}{0.000000in}}{\pgfqpoint{0.000000in}{0.000000in}}{%
\pgfpathmoveto{\pgfqpoint{0.000000in}{0.000000in}}%
\pgfpathlineto{\pgfqpoint{-0.048611in}{0.000000in}}%
\pgfusepath{stroke,fill}%
}%
\begin{pgfscope}%
\pgfsys@transformshift{3.347347in}{7.687308in}%
\pgfsys@useobject{currentmarker}{}%
\end{pgfscope}%
\end{pgfscope}%
\begin{pgfscope}%
\pgftext[x=2.962122in,y=7.634546in,left,base]{\rmfamily\fontsize{10.000000}{12.000000}\selectfont \(\displaystyle 10^{-6}\)}%
\end{pgfscope}%
\begin{pgfscope}%
\pgfsetbuttcap%
\pgfsetroundjoin%
\definecolor{currentfill}{rgb}{0.000000,0.000000,0.000000}%
\pgfsetfillcolor{currentfill}%
\pgfsetlinewidth{0.803000pt}%
\definecolor{currentstroke}{rgb}{0.000000,0.000000,0.000000}%
\pgfsetstrokecolor{currentstroke}%
\pgfsetdash{}{0pt}%
\pgfsys@defobject{currentmarker}{\pgfqpoint{-0.048611in}{0.000000in}}{\pgfqpoint{0.000000in}{0.000000in}}{%
\pgfpathmoveto{\pgfqpoint{0.000000in}{0.000000in}}%
\pgfpathlineto{\pgfqpoint{-0.048611in}{0.000000in}}%
\pgfusepath{stroke,fill}%
}%
\begin{pgfscope}%
\pgfsys@transformshift{3.347347in}{8.153048in}%
\pgfsys@useobject{currentmarker}{}%
\end{pgfscope}%
\end{pgfscope}%
\begin{pgfscope}%
\pgftext[x=2.962122in,y=8.100286in,left,base]{\rmfamily\fontsize{10.000000}{12.000000}\selectfont \(\displaystyle 10^{-5}\)}%
\end{pgfscope}%
\begin{pgfscope}%
\pgfsetbuttcap%
\pgfsetroundjoin%
\definecolor{currentfill}{rgb}{0.000000,0.000000,0.000000}%
\pgfsetfillcolor{currentfill}%
\pgfsetlinewidth{0.602250pt}%
\definecolor{currentstroke}{rgb}{0.000000,0.000000,0.000000}%
\pgfsetstrokecolor{currentstroke}%
\pgfsetdash{}{0pt}%
\pgfsys@defobject{currentmarker}{\pgfqpoint{-0.027778in}{0.000000in}}{\pgfqpoint{0.000000in}{0.000000in}}{%
\pgfpathmoveto{\pgfqpoint{0.000000in}{0.000000in}}%
\pgfpathlineto{\pgfqpoint{-0.027778in}{0.000000in}}%
\pgfusepath{stroke,fill}%
}%
\begin{pgfscope}%
\pgfsys@transformshift{3.347347in}{6.978042in}%
\pgfsys@useobject{currentmarker}{}%
\end{pgfscope}%
\end{pgfscope}%
\begin{pgfscope}%
\pgfsetbuttcap%
\pgfsetroundjoin%
\definecolor{currentfill}{rgb}{0.000000,0.000000,0.000000}%
\pgfsetfillcolor{currentfill}%
\pgfsetlinewidth{0.602250pt}%
\definecolor{currentstroke}{rgb}{0.000000,0.000000,0.000000}%
\pgfsetstrokecolor{currentstroke}%
\pgfsetdash{}{0pt}%
\pgfsys@defobject{currentmarker}{\pgfqpoint{-0.027778in}{0.000000in}}{\pgfqpoint{0.000000in}{0.000000in}}{%
\pgfpathmoveto{\pgfqpoint{0.000000in}{0.000000in}}%
\pgfpathlineto{\pgfqpoint{-0.027778in}{0.000000in}}%
\pgfusepath{stroke,fill}%
}%
\begin{pgfscope}%
\pgfsys@transformshift{3.347347in}{7.036231in}%
\pgfsys@useobject{currentmarker}{}%
\end{pgfscope}%
\end{pgfscope}%
\begin{pgfscope}%
\pgfsetbuttcap%
\pgfsetroundjoin%
\definecolor{currentfill}{rgb}{0.000000,0.000000,0.000000}%
\pgfsetfillcolor{currentfill}%
\pgfsetlinewidth{0.602250pt}%
\definecolor{currentstroke}{rgb}{0.000000,0.000000,0.000000}%
\pgfsetstrokecolor{currentstroke}%
\pgfsetdash{}{0pt}%
\pgfsys@defobject{currentmarker}{\pgfqpoint{-0.027778in}{0.000000in}}{\pgfqpoint{0.000000in}{0.000000in}}{%
\pgfpathmoveto{\pgfqpoint{0.000000in}{0.000000in}}%
\pgfpathlineto{\pgfqpoint{-0.027778in}{0.000000in}}%
\pgfusepath{stroke,fill}%
}%
\begin{pgfscope}%
\pgfsys@transformshift{3.347347in}{7.081366in}%
\pgfsys@useobject{currentmarker}{}%
\end{pgfscope}%
\end{pgfscope}%
\begin{pgfscope}%
\pgfsetbuttcap%
\pgfsetroundjoin%
\definecolor{currentfill}{rgb}{0.000000,0.000000,0.000000}%
\pgfsetfillcolor{currentfill}%
\pgfsetlinewidth{0.602250pt}%
\definecolor{currentstroke}{rgb}{0.000000,0.000000,0.000000}%
\pgfsetstrokecolor{currentstroke}%
\pgfsetdash{}{0pt}%
\pgfsys@defobject{currentmarker}{\pgfqpoint{-0.027778in}{0.000000in}}{\pgfqpoint{0.000000in}{0.000000in}}{%
\pgfpathmoveto{\pgfqpoint{0.000000in}{0.000000in}}%
\pgfpathlineto{\pgfqpoint{-0.027778in}{0.000000in}}%
\pgfusepath{stroke,fill}%
}%
\begin{pgfscope}%
\pgfsys@transformshift{3.347347in}{7.118244in}%
\pgfsys@useobject{currentmarker}{}%
\end{pgfscope}%
\end{pgfscope}%
\begin{pgfscope}%
\pgfsetbuttcap%
\pgfsetroundjoin%
\definecolor{currentfill}{rgb}{0.000000,0.000000,0.000000}%
\pgfsetfillcolor{currentfill}%
\pgfsetlinewidth{0.602250pt}%
\definecolor{currentstroke}{rgb}{0.000000,0.000000,0.000000}%
\pgfsetstrokecolor{currentstroke}%
\pgfsetdash{}{0pt}%
\pgfsys@defobject{currentmarker}{\pgfqpoint{-0.027778in}{0.000000in}}{\pgfqpoint{0.000000in}{0.000000in}}{%
\pgfpathmoveto{\pgfqpoint{0.000000in}{0.000000in}}%
\pgfpathlineto{\pgfqpoint{-0.027778in}{0.000000in}}%
\pgfusepath{stroke,fill}%
}%
\begin{pgfscope}%
\pgfsys@transformshift{3.347347in}{7.149424in}%
\pgfsys@useobject{currentmarker}{}%
\end{pgfscope}%
\end{pgfscope}%
\begin{pgfscope}%
\pgfsetbuttcap%
\pgfsetroundjoin%
\definecolor{currentfill}{rgb}{0.000000,0.000000,0.000000}%
\pgfsetfillcolor{currentfill}%
\pgfsetlinewidth{0.602250pt}%
\definecolor{currentstroke}{rgb}{0.000000,0.000000,0.000000}%
\pgfsetstrokecolor{currentstroke}%
\pgfsetdash{}{0pt}%
\pgfsys@defobject{currentmarker}{\pgfqpoint{-0.027778in}{0.000000in}}{\pgfqpoint{0.000000in}{0.000000in}}{%
\pgfpathmoveto{\pgfqpoint{0.000000in}{0.000000in}}%
\pgfpathlineto{\pgfqpoint{-0.027778in}{0.000000in}}%
\pgfusepath{stroke,fill}%
}%
\begin{pgfscope}%
\pgfsys@transformshift{3.347347in}{7.176433in}%
\pgfsys@useobject{currentmarker}{}%
\end{pgfscope}%
\end{pgfscope}%
\begin{pgfscope}%
\pgfsetbuttcap%
\pgfsetroundjoin%
\definecolor{currentfill}{rgb}{0.000000,0.000000,0.000000}%
\pgfsetfillcolor{currentfill}%
\pgfsetlinewidth{0.602250pt}%
\definecolor{currentstroke}{rgb}{0.000000,0.000000,0.000000}%
\pgfsetstrokecolor{currentstroke}%
\pgfsetdash{}{0pt}%
\pgfsys@defobject{currentmarker}{\pgfqpoint{-0.027778in}{0.000000in}}{\pgfqpoint{0.000000in}{0.000000in}}{%
\pgfpathmoveto{\pgfqpoint{0.000000in}{0.000000in}}%
\pgfpathlineto{\pgfqpoint{-0.027778in}{0.000000in}}%
\pgfusepath{stroke,fill}%
}%
\begin{pgfscope}%
\pgfsys@transformshift{3.347347in}{7.200257in}%
\pgfsys@useobject{currentmarker}{}%
\end{pgfscope}%
\end{pgfscope}%
\begin{pgfscope}%
\pgfsetbuttcap%
\pgfsetroundjoin%
\definecolor{currentfill}{rgb}{0.000000,0.000000,0.000000}%
\pgfsetfillcolor{currentfill}%
\pgfsetlinewidth{0.602250pt}%
\definecolor{currentstroke}{rgb}{0.000000,0.000000,0.000000}%
\pgfsetstrokecolor{currentstroke}%
\pgfsetdash{}{0pt}%
\pgfsys@defobject{currentmarker}{\pgfqpoint{-0.027778in}{0.000000in}}{\pgfqpoint{0.000000in}{0.000000in}}{%
\pgfpathmoveto{\pgfqpoint{0.000000in}{0.000000in}}%
\pgfpathlineto{\pgfqpoint{-0.027778in}{0.000000in}}%
\pgfusepath{stroke,fill}%
}%
\begin{pgfscope}%
\pgfsys@transformshift{3.347347in}{7.361769in}%
\pgfsys@useobject{currentmarker}{}%
\end{pgfscope}%
\end{pgfscope}%
\begin{pgfscope}%
\pgfsetbuttcap%
\pgfsetroundjoin%
\definecolor{currentfill}{rgb}{0.000000,0.000000,0.000000}%
\pgfsetfillcolor{currentfill}%
\pgfsetlinewidth{0.602250pt}%
\definecolor{currentstroke}{rgb}{0.000000,0.000000,0.000000}%
\pgfsetstrokecolor{currentstroke}%
\pgfsetdash{}{0pt}%
\pgfsys@defobject{currentmarker}{\pgfqpoint{-0.027778in}{0.000000in}}{\pgfqpoint{0.000000in}{0.000000in}}{%
\pgfpathmoveto{\pgfqpoint{0.000000in}{0.000000in}}%
\pgfpathlineto{\pgfqpoint{-0.027778in}{0.000000in}}%
\pgfusepath{stroke,fill}%
}%
\begin{pgfscope}%
\pgfsys@transformshift{3.347347in}{7.443782in}%
\pgfsys@useobject{currentmarker}{}%
\end{pgfscope}%
\end{pgfscope}%
\begin{pgfscope}%
\pgfsetbuttcap%
\pgfsetroundjoin%
\definecolor{currentfill}{rgb}{0.000000,0.000000,0.000000}%
\pgfsetfillcolor{currentfill}%
\pgfsetlinewidth{0.602250pt}%
\definecolor{currentstroke}{rgb}{0.000000,0.000000,0.000000}%
\pgfsetstrokecolor{currentstroke}%
\pgfsetdash{}{0pt}%
\pgfsys@defobject{currentmarker}{\pgfqpoint{-0.027778in}{0.000000in}}{\pgfqpoint{0.000000in}{0.000000in}}{%
\pgfpathmoveto{\pgfqpoint{0.000000in}{0.000000in}}%
\pgfpathlineto{\pgfqpoint{-0.027778in}{0.000000in}}%
\pgfusepath{stroke,fill}%
}%
\begin{pgfscope}%
\pgfsys@transformshift{3.347347in}{7.501971in}%
\pgfsys@useobject{currentmarker}{}%
\end{pgfscope}%
\end{pgfscope}%
\begin{pgfscope}%
\pgfsetbuttcap%
\pgfsetroundjoin%
\definecolor{currentfill}{rgb}{0.000000,0.000000,0.000000}%
\pgfsetfillcolor{currentfill}%
\pgfsetlinewidth{0.602250pt}%
\definecolor{currentstroke}{rgb}{0.000000,0.000000,0.000000}%
\pgfsetstrokecolor{currentstroke}%
\pgfsetdash{}{0pt}%
\pgfsys@defobject{currentmarker}{\pgfqpoint{-0.027778in}{0.000000in}}{\pgfqpoint{0.000000in}{0.000000in}}{%
\pgfpathmoveto{\pgfqpoint{0.000000in}{0.000000in}}%
\pgfpathlineto{\pgfqpoint{-0.027778in}{0.000000in}}%
\pgfusepath{stroke,fill}%
}%
\begin{pgfscope}%
\pgfsys@transformshift{3.347347in}{7.547106in}%
\pgfsys@useobject{currentmarker}{}%
\end{pgfscope}%
\end{pgfscope}%
\begin{pgfscope}%
\pgfsetbuttcap%
\pgfsetroundjoin%
\definecolor{currentfill}{rgb}{0.000000,0.000000,0.000000}%
\pgfsetfillcolor{currentfill}%
\pgfsetlinewidth{0.602250pt}%
\definecolor{currentstroke}{rgb}{0.000000,0.000000,0.000000}%
\pgfsetstrokecolor{currentstroke}%
\pgfsetdash{}{0pt}%
\pgfsys@defobject{currentmarker}{\pgfqpoint{-0.027778in}{0.000000in}}{\pgfqpoint{0.000000in}{0.000000in}}{%
\pgfpathmoveto{\pgfqpoint{0.000000in}{0.000000in}}%
\pgfpathlineto{\pgfqpoint{-0.027778in}{0.000000in}}%
\pgfusepath{stroke,fill}%
}%
\begin{pgfscope}%
\pgfsys@transformshift{3.347347in}{7.583984in}%
\pgfsys@useobject{currentmarker}{}%
\end{pgfscope}%
\end{pgfscope}%
\begin{pgfscope}%
\pgfsetbuttcap%
\pgfsetroundjoin%
\definecolor{currentfill}{rgb}{0.000000,0.000000,0.000000}%
\pgfsetfillcolor{currentfill}%
\pgfsetlinewidth{0.602250pt}%
\definecolor{currentstroke}{rgb}{0.000000,0.000000,0.000000}%
\pgfsetstrokecolor{currentstroke}%
\pgfsetdash{}{0pt}%
\pgfsys@defobject{currentmarker}{\pgfqpoint{-0.027778in}{0.000000in}}{\pgfqpoint{0.000000in}{0.000000in}}{%
\pgfpathmoveto{\pgfqpoint{0.000000in}{0.000000in}}%
\pgfpathlineto{\pgfqpoint{-0.027778in}{0.000000in}}%
\pgfusepath{stroke,fill}%
}%
\begin{pgfscope}%
\pgfsys@transformshift{3.347347in}{7.615164in}%
\pgfsys@useobject{currentmarker}{}%
\end{pgfscope}%
\end{pgfscope}%
\begin{pgfscope}%
\pgfsetbuttcap%
\pgfsetroundjoin%
\definecolor{currentfill}{rgb}{0.000000,0.000000,0.000000}%
\pgfsetfillcolor{currentfill}%
\pgfsetlinewidth{0.602250pt}%
\definecolor{currentstroke}{rgb}{0.000000,0.000000,0.000000}%
\pgfsetstrokecolor{currentstroke}%
\pgfsetdash{}{0pt}%
\pgfsys@defobject{currentmarker}{\pgfqpoint{-0.027778in}{0.000000in}}{\pgfqpoint{0.000000in}{0.000000in}}{%
\pgfpathmoveto{\pgfqpoint{0.000000in}{0.000000in}}%
\pgfpathlineto{\pgfqpoint{-0.027778in}{0.000000in}}%
\pgfusepath{stroke,fill}%
}%
\begin{pgfscope}%
\pgfsys@transformshift{3.347347in}{7.642173in}%
\pgfsys@useobject{currentmarker}{}%
\end{pgfscope}%
\end{pgfscope}%
\begin{pgfscope}%
\pgfsetbuttcap%
\pgfsetroundjoin%
\definecolor{currentfill}{rgb}{0.000000,0.000000,0.000000}%
\pgfsetfillcolor{currentfill}%
\pgfsetlinewidth{0.602250pt}%
\definecolor{currentstroke}{rgb}{0.000000,0.000000,0.000000}%
\pgfsetstrokecolor{currentstroke}%
\pgfsetdash{}{0pt}%
\pgfsys@defobject{currentmarker}{\pgfqpoint{-0.027778in}{0.000000in}}{\pgfqpoint{0.000000in}{0.000000in}}{%
\pgfpathmoveto{\pgfqpoint{0.000000in}{0.000000in}}%
\pgfpathlineto{\pgfqpoint{-0.027778in}{0.000000in}}%
\pgfusepath{stroke,fill}%
}%
\begin{pgfscope}%
\pgfsys@transformshift{3.347347in}{7.665997in}%
\pgfsys@useobject{currentmarker}{}%
\end{pgfscope}%
\end{pgfscope}%
\begin{pgfscope}%
\pgfsetbuttcap%
\pgfsetroundjoin%
\definecolor{currentfill}{rgb}{0.000000,0.000000,0.000000}%
\pgfsetfillcolor{currentfill}%
\pgfsetlinewidth{0.602250pt}%
\definecolor{currentstroke}{rgb}{0.000000,0.000000,0.000000}%
\pgfsetstrokecolor{currentstroke}%
\pgfsetdash{}{0pt}%
\pgfsys@defobject{currentmarker}{\pgfqpoint{-0.027778in}{0.000000in}}{\pgfqpoint{0.000000in}{0.000000in}}{%
\pgfpathmoveto{\pgfqpoint{0.000000in}{0.000000in}}%
\pgfpathlineto{\pgfqpoint{-0.027778in}{0.000000in}}%
\pgfusepath{stroke,fill}%
}%
\begin{pgfscope}%
\pgfsys@transformshift{3.347347in}{7.827509in}%
\pgfsys@useobject{currentmarker}{}%
\end{pgfscope}%
\end{pgfscope}%
\begin{pgfscope}%
\pgfsetbuttcap%
\pgfsetroundjoin%
\definecolor{currentfill}{rgb}{0.000000,0.000000,0.000000}%
\pgfsetfillcolor{currentfill}%
\pgfsetlinewidth{0.602250pt}%
\definecolor{currentstroke}{rgb}{0.000000,0.000000,0.000000}%
\pgfsetstrokecolor{currentstroke}%
\pgfsetdash{}{0pt}%
\pgfsys@defobject{currentmarker}{\pgfqpoint{-0.027778in}{0.000000in}}{\pgfqpoint{0.000000in}{0.000000in}}{%
\pgfpathmoveto{\pgfqpoint{0.000000in}{0.000000in}}%
\pgfpathlineto{\pgfqpoint{-0.027778in}{0.000000in}}%
\pgfusepath{stroke,fill}%
}%
\begin{pgfscope}%
\pgfsys@transformshift{3.347347in}{7.909522in}%
\pgfsys@useobject{currentmarker}{}%
\end{pgfscope}%
\end{pgfscope}%
\begin{pgfscope}%
\pgfsetbuttcap%
\pgfsetroundjoin%
\definecolor{currentfill}{rgb}{0.000000,0.000000,0.000000}%
\pgfsetfillcolor{currentfill}%
\pgfsetlinewidth{0.602250pt}%
\definecolor{currentstroke}{rgb}{0.000000,0.000000,0.000000}%
\pgfsetstrokecolor{currentstroke}%
\pgfsetdash{}{0pt}%
\pgfsys@defobject{currentmarker}{\pgfqpoint{-0.027778in}{0.000000in}}{\pgfqpoint{0.000000in}{0.000000in}}{%
\pgfpathmoveto{\pgfqpoint{0.000000in}{0.000000in}}%
\pgfpathlineto{\pgfqpoint{-0.027778in}{0.000000in}}%
\pgfusepath{stroke,fill}%
}%
\begin{pgfscope}%
\pgfsys@transformshift{3.347347in}{7.967711in}%
\pgfsys@useobject{currentmarker}{}%
\end{pgfscope}%
\end{pgfscope}%
\begin{pgfscope}%
\pgfsetbuttcap%
\pgfsetroundjoin%
\definecolor{currentfill}{rgb}{0.000000,0.000000,0.000000}%
\pgfsetfillcolor{currentfill}%
\pgfsetlinewidth{0.602250pt}%
\definecolor{currentstroke}{rgb}{0.000000,0.000000,0.000000}%
\pgfsetstrokecolor{currentstroke}%
\pgfsetdash{}{0pt}%
\pgfsys@defobject{currentmarker}{\pgfqpoint{-0.027778in}{0.000000in}}{\pgfqpoint{0.000000in}{0.000000in}}{%
\pgfpathmoveto{\pgfqpoint{0.000000in}{0.000000in}}%
\pgfpathlineto{\pgfqpoint{-0.027778in}{0.000000in}}%
\pgfusepath{stroke,fill}%
}%
\begin{pgfscope}%
\pgfsys@transformshift{3.347347in}{8.012846in}%
\pgfsys@useobject{currentmarker}{}%
\end{pgfscope}%
\end{pgfscope}%
\begin{pgfscope}%
\pgfsetbuttcap%
\pgfsetroundjoin%
\definecolor{currentfill}{rgb}{0.000000,0.000000,0.000000}%
\pgfsetfillcolor{currentfill}%
\pgfsetlinewidth{0.602250pt}%
\definecolor{currentstroke}{rgb}{0.000000,0.000000,0.000000}%
\pgfsetstrokecolor{currentstroke}%
\pgfsetdash{}{0pt}%
\pgfsys@defobject{currentmarker}{\pgfqpoint{-0.027778in}{0.000000in}}{\pgfqpoint{0.000000in}{0.000000in}}{%
\pgfpathmoveto{\pgfqpoint{0.000000in}{0.000000in}}%
\pgfpathlineto{\pgfqpoint{-0.027778in}{0.000000in}}%
\pgfusepath{stroke,fill}%
}%
\begin{pgfscope}%
\pgfsys@transformshift{3.347347in}{8.049724in}%
\pgfsys@useobject{currentmarker}{}%
\end{pgfscope}%
\end{pgfscope}%
\begin{pgfscope}%
\pgfsetbuttcap%
\pgfsetroundjoin%
\definecolor{currentfill}{rgb}{0.000000,0.000000,0.000000}%
\pgfsetfillcolor{currentfill}%
\pgfsetlinewidth{0.602250pt}%
\definecolor{currentstroke}{rgb}{0.000000,0.000000,0.000000}%
\pgfsetstrokecolor{currentstroke}%
\pgfsetdash{}{0pt}%
\pgfsys@defobject{currentmarker}{\pgfqpoint{-0.027778in}{0.000000in}}{\pgfqpoint{0.000000in}{0.000000in}}{%
\pgfpathmoveto{\pgfqpoint{0.000000in}{0.000000in}}%
\pgfpathlineto{\pgfqpoint{-0.027778in}{0.000000in}}%
\pgfusepath{stroke,fill}%
}%
\begin{pgfscope}%
\pgfsys@transformshift{3.347347in}{8.080904in}%
\pgfsys@useobject{currentmarker}{}%
\end{pgfscope}%
\end{pgfscope}%
\begin{pgfscope}%
\pgfsetbuttcap%
\pgfsetroundjoin%
\definecolor{currentfill}{rgb}{0.000000,0.000000,0.000000}%
\pgfsetfillcolor{currentfill}%
\pgfsetlinewidth{0.602250pt}%
\definecolor{currentstroke}{rgb}{0.000000,0.000000,0.000000}%
\pgfsetstrokecolor{currentstroke}%
\pgfsetdash{}{0pt}%
\pgfsys@defobject{currentmarker}{\pgfqpoint{-0.027778in}{0.000000in}}{\pgfqpoint{0.000000in}{0.000000in}}{%
\pgfpathmoveto{\pgfqpoint{0.000000in}{0.000000in}}%
\pgfpathlineto{\pgfqpoint{-0.027778in}{0.000000in}}%
\pgfusepath{stroke,fill}%
}%
\begin{pgfscope}%
\pgfsys@transformshift{3.347347in}{8.107913in}%
\pgfsys@useobject{currentmarker}{}%
\end{pgfscope}%
\end{pgfscope}%
\begin{pgfscope}%
\pgfsetbuttcap%
\pgfsetroundjoin%
\definecolor{currentfill}{rgb}{0.000000,0.000000,0.000000}%
\pgfsetfillcolor{currentfill}%
\pgfsetlinewidth{0.602250pt}%
\definecolor{currentstroke}{rgb}{0.000000,0.000000,0.000000}%
\pgfsetstrokecolor{currentstroke}%
\pgfsetdash{}{0pt}%
\pgfsys@defobject{currentmarker}{\pgfqpoint{-0.027778in}{0.000000in}}{\pgfqpoint{0.000000in}{0.000000in}}{%
\pgfpathmoveto{\pgfqpoint{0.000000in}{0.000000in}}%
\pgfpathlineto{\pgfqpoint{-0.027778in}{0.000000in}}%
\pgfusepath{stroke,fill}%
}%
\begin{pgfscope}%
\pgfsys@transformshift{3.347347in}{8.131737in}%
\pgfsys@useobject{currentmarker}{}%
\end{pgfscope}%
\end{pgfscope}%
\begin{pgfscope}%
\pgfsetbuttcap%
\pgfsetroundjoin%
\definecolor{currentfill}{rgb}{0.000000,0.000000,0.000000}%
\pgfsetfillcolor{currentfill}%
\pgfsetlinewidth{0.602250pt}%
\definecolor{currentstroke}{rgb}{0.000000,0.000000,0.000000}%
\pgfsetstrokecolor{currentstroke}%
\pgfsetdash{}{0pt}%
\pgfsys@defobject{currentmarker}{\pgfqpoint{-0.027778in}{0.000000in}}{\pgfqpoint{0.000000in}{0.000000in}}{%
\pgfpathmoveto{\pgfqpoint{0.000000in}{0.000000in}}%
\pgfpathlineto{\pgfqpoint{-0.027778in}{0.000000in}}%
\pgfusepath{stroke,fill}%
}%
\begin{pgfscope}%
\pgfsys@transformshift{3.347347in}{8.293250in}%
\pgfsys@useobject{currentmarker}{}%
\end{pgfscope}%
\end{pgfscope}%
\begin{pgfscope}%
\pgfpathrectangle{\pgfqpoint{3.347347in}{6.967719in}}{\pgfqpoint{1.897959in}{1.372727in}} %
\pgfusepath{clip}%
\pgfsetbuttcap%
\pgfsetroundjoin%
\pgfsetlinewidth{1.505625pt}%
\definecolor{currentstroke}{rgb}{1.000000,0.000000,0.000000}%
\pgfsetstrokecolor{currentstroke}%
\pgfsetdash{{5.550000pt}{2.400000pt}}{0.000000pt}%
\pgfpathmoveto{\pgfqpoint{3.433618in}{8.151508in}}%
\pgfpathlineto{\pgfqpoint{3.649295in}{8.150283in}}%
\pgfpathlineto{\pgfqpoint{3.864972in}{8.149170in}}%
\pgfpathlineto{\pgfqpoint{4.080649in}{8.148166in}}%
\pgfpathlineto{\pgfqpoint{4.296327in}{8.147271in}}%
\pgfpathlineto{\pgfqpoint{4.512004in}{8.146482in}}%
\pgfpathlineto{\pgfqpoint{4.727681in}{8.145796in}}%
\pgfpathlineto{\pgfqpoint{4.943358in}{8.145214in}}%
\pgfpathlineto{\pgfqpoint{5.159035in}{8.144735in}}%
\pgfusepath{stroke}%
\end{pgfscope}%
\begin{pgfscope}%
\pgfpathrectangle{\pgfqpoint{3.347347in}{6.967719in}}{\pgfqpoint{1.897959in}{1.372727in}} %
\pgfusepath{clip}%
\pgfsetbuttcap%
\pgfsetmiterjoin%
\definecolor{currentfill}{rgb}{1.000000,0.000000,0.000000}%
\pgfsetfillcolor{currentfill}%
\pgfsetlinewidth{1.003750pt}%
\definecolor{currentstroke}{rgb}{1.000000,0.000000,0.000000}%
\pgfsetstrokecolor{currentstroke}%
\pgfsetdash{}{0pt}%
\pgfsys@defobject{currentmarker}{\pgfqpoint{-0.041667in}{-0.041667in}}{\pgfqpoint{0.041667in}{0.041667in}}{%
\pgfpathmoveto{\pgfqpoint{-0.041667in}{-0.041667in}}%
\pgfpathlineto{\pgfqpoint{0.041667in}{-0.041667in}}%
\pgfpathlineto{\pgfqpoint{0.041667in}{0.041667in}}%
\pgfpathlineto{\pgfqpoint{-0.041667in}{0.041667in}}%
\pgfpathclose%
\pgfusepath{stroke,fill}%
}%
\begin{pgfscope}%
\pgfsys@transformshift{3.433618in}{8.151508in}%
\pgfsys@useobject{currentmarker}{}%
\end{pgfscope}%
\begin{pgfscope}%
\pgfsys@transformshift{3.864972in}{8.149170in}%
\pgfsys@useobject{currentmarker}{}%
\end{pgfscope}%
\begin{pgfscope}%
\pgfsys@transformshift{4.296327in}{8.147271in}%
\pgfsys@useobject{currentmarker}{}%
\end{pgfscope}%
\begin{pgfscope}%
\pgfsys@transformshift{4.727681in}{8.145796in}%
\pgfsys@useobject{currentmarker}{}%
\end{pgfscope}%
\begin{pgfscope}%
\pgfsys@transformshift{5.159035in}{8.144735in}%
\pgfsys@useobject{currentmarker}{}%
\end{pgfscope}%
\end{pgfscope}%
\begin{pgfscope}%
\pgfpathrectangle{\pgfqpoint{3.347347in}{6.967719in}}{\pgfqpoint{1.897959in}{1.372727in}} %
\pgfusepath{clip}%
\pgfsetrectcap%
\pgfsetroundjoin%
\pgfsetlinewidth{1.505625pt}%
\definecolor{currentstroke}{rgb}{0.000000,0.000000,1.000000}%
\pgfsetstrokecolor{currentstroke}%
\pgfsetdash{}{0pt}%
\pgfpathmoveto{\pgfqpoint{3.433618in}{7.298527in}}%
\pgfpathlineto{\pgfqpoint{3.649295in}{7.274379in}}%
\pgfpathlineto{\pgfqpoint{3.864972in}{7.248840in}}%
\pgfpathlineto{\pgfqpoint{4.080649in}{7.221496in}}%
\pgfpathlineto{\pgfqpoint{4.296327in}{7.191822in}}%
\pgfpathlineto{\pgfqpoint{4.512004in}{7.159119in}}%
\pgfpathlineto{\pgfqpoint{4.727681in}{7.122400in}}%
\pgfpathlineto{\pgfqpoint{4.943358in}{7.080193in}}%
\pgfpathlineto{\pgfqpoint{5.159035in}{7.030115in}}%
\pgfusepath{stroke}%
\end{pgfscope}%
\begin{pgfscope}%
\pgfpathrectangle{\pgfqpoint{3.347347in}{6.967719in}}{\pgfqpoint{1.897959in}{1.372727in}} %
\pgfusepath{clip}%
\pgfsetbuttcap%
\pgfsetroundjoin%
\definecolor{currentfill}{rgb}{0.000000,0.000000,1.000000}%
\pgfsetfillcolor{currentfill}%
\pgfsetlinewidth{1.003750pt}%
\definecolor{currentstroke}{rgb}{0.000000,0.000000,1.000000}%
\pgfsetstrokecolor{currentstroke}%
\pgfsetdash{}{0pt}%
\pgfsys@defobject{currentmarker}{\pgfqpoint{-0.041667in}{-0.041667in}}{\pgfqpoint{0.041667in}{0.041667in}}{%
\pgfpathmoveto{\pgfqpoint{0.000000in}{-0.041667in}}%
\pgfpathcurveto{\pgfqpoint{0.011050in}{-0.041667in}}{\pgfqpoint{0.021649in}{-0.037276in}}{\pgfqpoint{0.029463in}{-0.029463in}}%
\pgfpathcurveto{\pgfqpoint{0.037276in}{-0.021649in}}{\pgfqpoint{0.041667in}{-0.011050in}}{\pgfqpoint{0.041667in}{0.000000in}}%
\pgfpathcurveto{\pgfqpoint{0.041667in}{0.011050in}}{\pgfqpoint{0.037276in}{0.021649in}}{\pgfqpoint{0.029463in}{0.029463in}}%
\pgfpathcurveto{\pgfqpoint{0.021649in}{0.037276in}}{\pgfqpoint{0.011050in}{0.041667in}}{\pgfqpoint{0.000000in}{0.041667in}}%
\pgfpathcurveto{\pgfqpoint{-0.011050in}{0.041667in}}{\pgfqpoint{-0.021649in}{0.037276in}}{\pgfqpoint{-0.029463in}{0.029463in}}%
\pgfpathcurveto{\pgfqpoint{-0.037276in}{0.021649in}}{\pgfqpoint{-0.041667in}{0.011050in}}{\pgfqpoint{-0.041667in}{0.000000in}}%
\pgfpathcurveto{\pgfqpoint{-0.041667in}{-0.011050in}}{\pgfqpoint{-0.037276in}{-0.021649in}}{\pgfqpoint{-0.029463in}{-0.029463in}}%
\pgfpathcurveto{\pgfqpoint{-0.021649in}{-0.037276in}}{\pgfqpoint{-0.011050in}{-0.041667in}}{\pgfqpoint{0.000000in}{-0.041667in}}%
\pgfpathclose%
\pgfusepath{stroke,fill}%
}%
\begin{pgfscope}%
\pgfsys@transformshift{3.433618in}{7.298527in}%
\pgfsys@useobject{currentmarker}{}%
\end{pgfscope}%
\begin{pgfscope}%
\pgfsys@transformshift{3.864972in}{7.248840in}%
\pgfsys@useobject{currentmarker}{}%
\end{pgfscope}%
\begin{pgfscope}%
\pgfsys@transformshift{4.296327in}{7.191822in}%
\pgfsys@useobject{currentmarker}{}%
\end{pgfscope}%
\begin{pgfscope}%
\pgfsys@transformshift{4.727681in}{7.122400in}%
\pgfsys@useobject{currentmarker}{}%
\end{pgfscope}%
\begin{pgfscope}%
\pgfsys@transformshift{5.159035in}{7.030115in}%
\pgfsys@useobject{currentmarker}{}%
\end{pgfscope}%
\end{pgfscope}%
\begin{pgfscope}%
\pgfpathrectangle{\pgfqpoint{3.347347in}{6.967719in}}{\pgfqpoint{1.897959in}{1.372727in}} %
\pgfusepath{clip}%
\pgfsetbuttcap%
\pgfsetroundjoin%
\pgfsetlinewidth{1.505625pt}%
\definecolor{currentstroke}{rgb}{0.000000,0.750000,0.750000}%
\pgfsetstrokecolor{currentstroke}%
\pgfsetdash{{9.600000pt}{2.400000pt}{1.500000pt}{2.400000pt}}{0.000000pt}%
\pgfpathmoveto{\pgfqpoint{3.433618in}{8.120839in}}%
\pgfpathlineto{\pgfqpoint{3.649295in}{8.104810in}}%
\pgfpathlineto{\pgfqpoint{3.864972in}{8.090833in}}%
\pgfpathlineto{\pgfqpoint{4.080649in}{8.078676in}}%
\pgfpathlineto{\pgfqpoint{4.296327in}{8.068154in}}%
\pgfpathlineto{\pgfqpoint{4.512004in}{8.059119in}}%
\pgfpathlineto{\pgfqpoint{4.727681in}{8.051456in}}%
\pgfpathlineto{\pgfqpoint{4.943358in}{8.045071in}}%
\pgfpathlineto{\pgfqpoint{5.159035in}{8.039889in}}%
\pgfusepath{stroke}%
\end{pgfscope}%
\begin{pgfscope}%
\pgfpathrectangle{\pgfqpoint{3.347347in}{6.967719in}}{\pgfqpoint{1.897959in}{1.372727in}} %
\pgfusepath{clip}%
\pgfsetbuttcap%
\pgfsetmiterjoin%
\definecolor{currentfill}{rgb}{0.000000,0.750000,0.750000}%
\pgfsetfillcolor{currentfill}%
\pgfsetlinewidth{1.003750pt}%
\definecolor{currentstroke}{rgb}{0.000000,0.750000,0.750000}%
\pgfsetstrokecolor{currentstroke}%
\pgfsetdash{}{0pt}%
\pgfsys@defobject{currentmarker}{\pgfqpoint{-0.041667in}{-0.041667in}}{\pgfqpoint{0.041667in}{0.041667in}}{%
\pgfpathmoveto{\pgfqpoint{-0.000000in}{-0.041667in}}%
\pgfpathlineto{\pgfqpoint{0.041667in}{0.041667in}}%
\pgfpathlineto{\pgfqpoint{-0.041667in}{0.041667in}}%
\pgfpathclose%
\pgfusepath{stroke,fill}%
}%
\begin{pgfscope}%
\pgfsys@transformshift{3.433618in}{8.120839in}%
\pgfsys@useobject{currentmarker}{}%
\end{pgfscope}%
\begin{pgfscope}%
\pgfsys@transformshift{3.864972in}{8.090833in}%
\pgfsys@useobject{currentmarker}{}%
\end{pgfscope}%
\begin{pgfscope}%
\pgfsys@transformshift{4.296327in}{8.068154in}%
\pgfsys@useobject{currentmarker}{}%
\end{pgfscope}%
\begin{pgfscope}%
\pgfsys@transformshift{4.727681in}{8.051456in}%
\pgfsys@useobject{currentmarker}{}%
\end{pgfscope}%
\begin{pgfscope}%
\pgfsys@transformshift{5.159035in}{8.039889in}%
\pgfsys@useobject{currentmarker}{}%
\end{pgfscope}%
\end{pgfscope}%
\begin{pgfscope}%
\pgfpathrectangle{\pgfqpoint{3.347347in}{6.967719in}}{\pgfqpoint{1.897959in}{1.372727in}} %
\pgfusepath{clip}%
\pgfsetbuttcap%
\pgfsetroundjoin%
\pgfsetlinewidth{1.505625pt}%
\definecolor{currentstroke}{rgb}{0.000000,0.000000,0.000000}%
\pgfsetstrokecolor{currentstroke}%
\pgfsetdash{{1.500000pt}{2.475000pt}}{0.000000pt}%
\pgfpathmoveto{\pgfqpoint{3.433618in}{8.278049in}}%
\pgfpathlineto{\pgfqpoint{3.649295in}{8.270743in}}%
\pgfpathlineto{\pgfqpoint{3.864972in}{8.263845in}}%
\pgfpathlineto{\pgfqpoint{4.080649in}{8.257970in}}%
\pgfpathlineto{\pgfqpoint{4.296327in}{8.252971in}}%
\pgfpathlineto{\pgfqpoint{4.512004in}{8.248735in}}%
\pgfpathlineto{\pgfqpoint{4.727681in}{8.245176in}}%
\pgfpathlineto{\pgfqpoint{4.943358in}{8.242226in}}%
\pgfpathlineto{\pgfqpoint{5.159035in}{8.239834in}}%
\pgfusepath{stroke}%
\end{pgfscope}%
\begin{pgfscope}%
\pgfpathrectangle{\pgfqpoint{3.347347in}{6.967719in}}{\pgfqpoint{1.897959in}{1.372727in}} %
\pgfusepath{clip}%
\pgfsetbuttcap%
\pgfsetroundjoin%
\definecolor{currentfill}{rgb}{0.000000,0.000000,0.000000}%
\pgfsetfillcolor{currentfill}%
\pgfsetlinewidth{1.003750pt}%
\definecolor{currentstroke}{rgb}{0.000000,0.000000,0.000000}%
\pgfsetstrokecolor{currentstroke}%
\pgfsetdash{}{0pt}%
\pgfsys@defobject{currentmarker}{\pgfqpoint{-0.041667in}{-0.041667in}}{\pgfqpoint{0.041667in}{0.041667in}}{%
\pgfpathmoveto{\pgfqpoint{-0.041667in}{0.000000in}}%
\pgfpathlineto{\pgfqpoint{0.041667in}{0.000000in}}%
\pgfpathmoveto{\pgfqpoint{0.000000in}{-0.041667in}}%
\pgfpathlineto{\pgfqpoint{0.000000in}{0.041667in}}%
\pgfusepath{stroke,fill}%
}%
\begin{pgfscope}%
\pgfsys@transformshift{3.433618in}{8.278049in}%
\pgfsys@useobject{currentmarker}{}%
\end{pgfscope}%
\begin{pgfscope}%
\pgfsys@transformshift{3.864972in}{8.263845in}%
\pgfsys@useobject{currentmarker}{}%
\end{pgfscope}%
\begin{pgfscope}%
\pgfsys@transformshift{4.296327in}{8.252971in}%
\pgfsys@useobject{currentmarker}{}%
\end{pgfscope}%
\begin{pgfscope}%
\pgfsys@transformshift{4.727681in}{8.245176in}%
\pgfsys@useobject{currentmarker}{}%
\end{pgfscope}%
\begin{pgfscope}%
\pgfsys@transformshift{5.159035in}{8.239834in}%
\pgfsys@useobject{currentmarker}{}%
\end{pgfscope}%
\end{pgfscope}%
\begin{pgfscope}%
\pgfsetrectcap%
\pgfsetmiterjoin%
\pgfsetlinewidth{0.803000pt}%
\definecolor{currentstroke}{rgb}{0.000000,0.000000,0.000000}%
\pgfsetstrokecolor{currentstroke}%
\pgfsetdash{}{0pt}%
\pgfpathmoveto{\pgfqpoint{3.347347in}{6.967719in}}%
\pgfpathlineto{\pgfqpoint{3.347347in}{8.340446in}}%
\pgfusepath{stroke}%
\end{pgfscope}%
\begin{pgfscope}%
\pgfsetrectcap%
\pgfsetmiterjoin%
\pgfsetlinewidth{0.803000pt}%
\definecolor{currentstroke}{rgb}{0.000000,0.000000,0.000000}%
\pgfsetstrokecolor{currentstroke}%
\pgfsetdash{}{0pt}%
\pgfpathmoveto{\pgfqpoint{5.245306in}{6.967719in}}%
\pgfpathlineto{\pgfqpoint{5.245306in}{8.340446in}}%
\pgfusepath{stroke}%
\end{pgfscope}%
\begin{pgfscope}%
\pgfsetrectcap%
\pgfsetmiterjoin%
\pgfsetlinewidth{0.803000pt}%
\definecolor{currentstroke}{rgb}{0.000000,0.000000,0.000000}%
\pgfsetstrokecolor{currentstroke}%
\pgfsetdash{}{0pt}%
\pgfpathmoveto{\pgfqpoint{3.347347in}{6.967719in}}%
\pgfpathlineto{\pgfqpoint{5.245306in}{6.967719in}}%
\pgfusepath{stroke}%
\end{pgfscope}%
\begin{pgfscope}%
\pgfsetrectcap%
\pgfsetmiterjoin%
\pgfsetlinewidth{0.803000pt}%
\definecolor{currentstroke}{rgb}{0.000000,0.000000,0.000000}%
\pgfsetstrokecolor{currentstroke}%
\pgfsetdash{}{0pt}%
\pgfpathmoveto{\pgfqpoint{3.347347in}{8.340446in}}%
\pgfpathlineto{\pgfqpoint{5.245306in}{8.340446in}}%
\pgfusepath{stroke}%
\end{pgfscope}%
\begin{pgfscope}%
\pgfsetbuttcap%
\pgfsetmiterjoin%
\definecolor{currentfill}{rgb}{1.000000,1.000000,1.000000}%
\pgfsetfillcolor{currentfill}%
\pgfsetlinewidth{0.000000pt}%
\definecolor{currentstroke}{rgb}{0.000000,0.000000,0.000000}%
\pgfsetstrokecolor{currentstroke}%
\pgfsetstrokeopacity{0.000000}%
\pgfsetdash{}{0pt}%
\pgfpathmoveto{\pgfqpoint{5.814694in}{6.967719in}}%
\pgfpathlineto{\pgfqpoint{7.712653in}{6.967719in}}%
\pgfpathlineto{\pgfqpoint{7.712653in}{8.340446in}}%
\pgfpathlineto{\pgfqpoint{5.814694in}{8.340446in}}%
\pgfpathclose%
\pgfusepath{fill}%
\end{pgfscope}%
\begin{pgfscope}%
\pgfsetbuttcap%
\pgfsetroundjoin%
\definecolor{currentfill}{rgb}{0.000000,0.000000,0.000000}%
\pgfsetfillcolor{currentfill}%
\pgfsetlinewidth{0.803000pt}%
\definecolor{currentstroke}{rgb}{0.000000,0.000000,0.000000}%
\pgfsetstrokecolor{currentstroke}%
\pgfsetdash{}{0pt}%
\pgfsys@defobject{currentmarker}{\pgfqpoint{0.000000in}{-0.048611in}}{\pgfqpoint{0.000000in}{0.000000in}}{%
\pgfpathmoveto{\pgfqpoint{0.000000in}{0.000000in}}%
\pgfpathlineto{\pgfqpoint{0.000000in}{-0.048611in}}%
\pgfusepath{stroke,fill}%
}%
\begin{pgfscope}%
\pgfsys@transformshift{6.073506in}{6.967719in}%
\pgfsys@useobject{currentmarker}{}%
\end{pgfscope}%
\end{pgfscope}%
\begin{pgfscope}%
\pgftext[x=6.073506in,y=6.870496in,,top]{\rmfamily\fontsize{10.000000}{12.000000}\selectfont \(\displaystyle 0.075\)}%
\end{pgfscope}%
\begin{pgfscope}%
\pgfsetbuttcap%
\pgfsetroundjoin%
\definecolor{currentfill}{rgb}{0.000000,0.000000,0.000000}%
\pgfsetfillcolor{currentfill}%
\pgfsetlinewidth{0.803000pt}%
\definecolor{currentstroke}{rgb}{0.000000,0.000000,0.000000}%
\pgfsetstrokecolor{currentstroke}%
\pgfsetdash{}{0pt}%
\pgfsys@defobject{currentmarker}{\pgfqpoint{0.000000in}{-0.048611in}}{\pgfqpoint{0.000000in}{0.000000in}}{%
\pgfpathmoveto{\pgfqpoint{0.000000in}{0.000000in}}%
\pgfpathlineto{\pgfqpoint{0.000000in}{-0.048611in}}%
\pgfusepath{stroke,fill}%
}%
\begin{pgfscope}%
\pgfsys@transformshift{6.591132in}{6.967719in}%
\pgfsys@useobject{currentmarker}{}%
\end{pgfscope}%
\end{pgfscope}%
\begin{pgfscope}%
\pgftext[x=6.591132in,y=6.870496in,,top]{\rmfamily\fontsize{10.000000}{12.000000}\selectfont \(\displaystyle 0.100\)}%
\end{pgfscope}%
\begin{pgfscope}%
\pgfsetbuttcap%
\pgfsetroundjoin%
\definecolor{currentfill}{rgb}{0.000000,0.000000,0.000000}%
\pgfsetfillcolor{currentfill}%
\pgfsetlinewidth{0.803000pt}%
\definecolor{currentstroke}{rgb}{0.000000,0.000000,0.000000}%
\pgfsetstrokecolor{currentstroke}%
\pgfsetdash{}{0pt}%
\pgfsys@defobject{currentmarker}{\pgfqpoint{0.000000in}{-0.048611in}}{\pgfqpoint{0.000000in}{0.000000in}}{%
\pgfpathmoveto{\pgfqpoint{0.000000in}{0.000000in}}%
\pgfpathlineto{\pgfqpoint{0.000000in}{-0.048611in}}%
\pgfusepath{stroke,fill}%
}%
\begin{pgfscope}%
\pgfsys@transformshift{7.108757in}{6.967719in}%
\pgfsys@useobject{currentmarker}{}%
\end{pgfscope}%
\end{pgfscope}%
\begin{pgfscope}%
\pgftext[x=7.108757in,y=6.870496in,,top]{\rmfamily\fontsize{10.000000}{12.000000}\selectfont \(\displaystyle 0.125\)}%
\end{pgfscope}%
\begin{pgfscope}%
\pgfsetbuttcap%
\pgfsetroundjoin%
\definecolor{currentfill}{rgb}{0.000000,0.000000,0.000000}%
\pgfsetfillcolor{currentfill}%
\pgfsetlinewidth{0.803000pt}%
\definecolor{currentstroke}{rgb}{0.000000,0.000000,0.000000}%
\pgfsetstrokecolor{currentstroke}%
\pgfsetdash{}{0pt}%
\pgfsys@defobject{currentmarker}{\pgfqpoint{0.000000in}{-0.048611in}}{\pgfqpoint{0.000000in}{0.000000in}}{%
\pgfpathmoveto{\pgfqpoint{0.000000in}{0.000000in}}%
\pgfpathlineto{\pgfqpoint{0.000000in}{-0.048611in}}%
\pgfusepath{stroke,fill}%
}%
\begin{pgfscope}%
\pgfsys@transformshift{7.626382in}{6.967719in}%
\pgfsys@useobject{currentmarker}{}%
\end{pgfscope}%
\end{pgfscope}%
\begin{pgfscope}%
\pgftext[x=7.626382in,y=6.870496in,,top]{\rmfamily\fontsize{10.000000}{12.000000}\selectfont \(\displaystyle 0.150\)}%
\end{pgfscope}%
\begin{pgfscope}%
\pgfsetbuttcap%
\pgfsetroundjoin%
\definecolor{currentfill}{rgb}{0.000000,0.000000,0.000000}%
\pgfsetfillcolor{currentfill}%
\pgfsetlinewidth{0.803000pt}%
\definecolor{currentstroke}{rgb}{0.000000,0.000000,0.000000}%
\pgfsetstrokecolor{currentstroke}%
\pgfsetdash{}{0pt}%
\pgfsys@defobject{currentmarker}{\pgfqpoint{-0.048611in}{0.000000in}}{\pgfqpoint{0.000000in}{0.000000in}}{%
\pgfpathmoveto{\pgfqpoint{0.000000in}{0.000000in}}%
\pgfpathlineto{\pgfqpoint{-0.048611in}{0.000000in}}%
\pgfusepath{stroke,fill}%
}%
\begin{pgfscope}%
\pgfsys@transformshift{5.814694in}{7.045423in}%
\pgfsys@useobject{currentmarker}{}%
\end{pgfscope}%
\end{pgfscope}%
\begin{pgfscope}%
\pgftext[x=5.429469in,y=6.992662in,left,base]{\rmfamily\fontsize{10.000000}{12.000000}\selectfont \(\displaystyle 10^{-7}\)}%
\end{pgfscope}%
\begin{pgfscope}%
\pgfsetbuttcap%
\pgfsetroundjoin%
\definecolor{currentfill}{rgb}{0.000000,0.000000,0.000000}%
\pgfsetfillcolor{currentfill}%
\pgfsetlinewidth{0.803000pt}%
\definecolor{currentstroke}{rgb}{0.000000,0.000000,0.000000}%
\pgfsetstrokecolor{currentstroke}%
\pgfsetdash{}{0pt}%
\pgfsys@defobject{currentmarker}{\pgfqpoint{-0.048611in}{0.000000in}}{\pgfqpoint{0.000000in}{0.000000in}}{%
\pgfpathmoveto{\pgfqpoint{0.000000in}{0.000000in}}%
\pgfpathlineto{\pgfqpoint{-0.048611in}{0.000000in}}%
\pgfusepath{stroke,fill}%
}%
\begin{pgfscope}%
\pgfsys@transformshift{5.814694in}{7.477298in}%
\pgfsys@useobject{currentmarker}{}%
\end{pgfscope}%
\end{pgfscope}%
\begin{pgfscope}%
\pgftext[x=5.429469in,y=7.424536in,left,base]{\rmfamily\fontsize{10.000000}{12.000000}\selectfont \(\displaystyle 10^{-6}\)}%
\end{pgfscope}%
\begin{pgfscope}%
\pgfsetbuttcap%
\pgfsetroundjoin%
\definecolor{currentfill}{rgb}{0.000000,0.000000,0.000000}%
\pgfsetfillcolor{currentfill}%
\pgfsetlinewidth{0.803000pt}%
\definecolor{currentstroke}{rgb}{0.000000,0.000000,0.000000}%
\pgfsetstrokecolor{currentstroke}%
\pgfsetdash{}{0pt}%
\pgfsys@defobject{currentmarker}{\pgfqpoint{-0.048611in}{0.000000in}}{\pgfqpoint{0.000000in}{0.000000in}}{%
\pgfpathmoveto{\pgfqpoint{0.000000in}{0.000000in}}%
\pgfpathlineto{\pgfqpoint{-0.048611in}{0.000000in}}%
\pgfusepath{stroke,fill}%
}%
\begin{pgfscope}%
\pgfsys@transformshift{5.814694in}{7.909172in}%
\pgfsys@useobject{currentmarker}{}%
\end{pgfscope}%
\end{pgfscope}%
\begin{pgfscope}%
\pgftext[x=5.429469in,y=7.856411in,left,base]{\rmfamily\fontsize{10.000000}{12.000000}\selectfont \(\displaystyle 10^{-5}\)}%
\end{pgfscope}%
\begin{pgfscope}%
\pgfsetbuttcap%
\pgfsetroundjoin%
\definecolor{currentfill}{rgb}{0.000000,0.000000,0.000000}%
\pgfsetfillcolor{currentfill}%
\pgfsetlinewidth{0.803000pt}%
\definecolor{currentstroke}{rgb}{0.000000,0.000000,0.000000}%
\pgfsetstrokecolor{currentstroke}%
\pgfsetdash{}{0pt}%
\pgfsys@defobject{currentmarker}{\pgfqpoint{-0.048611in}{0.000000in}}{\pgfqpoint{0.000000in}{0.000000in}}{%
\pgfpathmoveto{\pgfqpoint{0.000000in}{0.000000in}}%
\pgfpathlineto{\pgfqpoint{-0.048611in}{0.000000in}}%
\pgfusepath{stroke,fill}%
}%
\begin{pgfscope}%
\pgfsys@transformshift{5.814694in}{8.341047in}%
\pgfsys@useobject{currentmarker}{}%
\end{pgfscope}%
\end{pgfscope}%
\begin{pgfscope}%
\pgftext[x=5.429469in,y=8.288285in,left,base]{\rmfamily\fontsize{10.000000}{12.000000}\selectfont \(\displaystyle 10^{-4}\)}%
\end{pgfscope}%
\begin{pgfscope}%
\pgfsetbuttcap%
\pgfsetroundjoin%
\definecolor{currentfill}{rgb}{0.000000,0.000000,0.000000}%
\pgfsetfillcolor{currentfill}%
\pgfsetlinewidth{0.602250pt}%
\definecolor{currentstroke}{rgb}{0.000000,0.000000,0.000000}%
\pgfsetstrokecolor{currentstroke}%
\pgfsetdash{}{0pt}%
\pgfsys@defobject{currentmarker}{\pgfqpoint{-0.027778in}{0.000000in}}{\pgfqpoint{0.000000in}{0.000000in}}{%
\pgfpathmoveto{\pgfqpoint{0.000000in}{0.000000in}}%
\pgfpathlineto{\pgfqpoint{-0.027778in}{0.000000in}}%
\pgfusepath{stroke,fill}%
}%
\begin{pgfscope}%
\pgfsys@transformshift{5.814694in}{6.978525in}%
\pgfsys@useobject{currentmarker}{}%
\end{pgfscope}%
\end{pgfscope}%
\begin{pgfscope}%
\pgfsetbuttcap%
\pgfsetroundjoin%
\definecolor{currentfill}{rgb}{0.000000,0.000000,0.000000}%
\pgfsetfillcolor{currentfill}%
\pgfsetlinewidth{0.602250pt}%
\definecolor{currentstroke}{rgb}{0.000000,0.000000,0.000000}%
\pgfsetstrokecolor{currentstroke}%
\pgfsetdash{}{0pt}%
\pgfsys@defobject{currentmarker}{\pgfqpoint{-0.027778in}{0.000000in}}{\pgfqpoint{0.000000in}{0.000000in}}{%
\pgfpathmoveto{\pgfqpoint{0.000000in}{0.000000in}}%
\pgfpathlineto{\pgfqpoint{-0.027778in}{0.000000in}}%
\pgfusepath{stroke,fill}%
}%
\begin{pgfscope}%
\pgfsys@transformshift{5.814694in}{7.003570in}%
\pgfsys@useobject{currentmarker}{}%
\end{pgfscope}%
\end{pgfscope}%
\begin{pgfscope}%
\pgfsetbuttcap%
\pgfsetroundjoin%
\definecolor{currentfill}{rgb}{0.000000,0.000000,0.000000}%
\pgfsetfillcolor{currentfill}%
\pgfsetlinewidth{0.602250pt}%
\definecolor{currentstroke}{rgb}{0.000000,0.000000,0.000000}%
\pgfsetstrokecolor{currentstroke}%
\pgfsetdash{}{0pt}%
\pgfsys@defobject{currentmarker}{\pgfqpoint{-0.027778in}{0.000000in}}{\pgfqpoint{0.000000in}{0.000000in}}{%
\pgfpathmoveto{\pgfqpoint{0.000000in}{0.000000in}}%
\pgfpathlineto{\pgfqpoint{-0.027778in}{0.000000in}}%
\pgfusepath{stroke,fill}%
}%
\begin{pgfscope}%
\pgfsys@transformshift{5.814694in}{7.025662in}%
\pgfsys@useobject{currentmarker}{}%
\end{pgfscope}%
\end{pgfscope}%
\begin{pgfscope}%
\pgfsetbuttcap%
\pgfsetroundjoin%
\definecolor{currentfill}{rgb}{0.000000,0.000000,0.000000}%
\pgfsetfillcolor{currentfill}%
\pgfsetlinewidth{0.602250pt}%
\definecolor{currentstroke}{rgb}{0.000000,0.000000,0.000000}%
\pgfsetstrokecolor{currentstroke}%
\pgfsetdash{}{0pt}%
\pgfsys@defobject{currentmarker}{\pgfqpoint{-0.027778in}{0.000000in}}{\pgfqpoint{0.000000in}{0.000000in}}{%
\pgfpathmoveto{\pgfqpoint{0.000000in}{0.000000in}}%
\pgfpathlineto{\pgfqpoint{-0.027778in}{0.000000in}}%
\pgfusepath{stroke,fill}%
}%
\begin{pgfscope}%
\pgfsys@transformshift{5.814694in}{7.175431in}%
\pgfsys@useobject{currentmarker}{}%
\end{pgfscope}%
\end{pgfscope}%
\begin{pgfscope}%
\pgfsetbuttcap%
\pgfsetroundjoin%
\definecolor{currentfill}{rgb}{0.000000,0.000000,0.000000}%
\pgfsetfillcolor{currentfill}%
\pgfsetlinewidth{0.602250pt}%
\definecolor{currentstroke}{rgb}{0.000000,0.000000,0.000000}%
\pgfsetstrokecolor{currentstroke}%
\pgfsetdash{}{0pt}%
\pgfsys@defobject{currentmarker}{\pgfqpoint{-0.027778in}{0.000000in}}{\pgfqpoint{0.000000in}{0.000000in}}{%
\pgfpathmoveto{\pgfqpoint{0.000000in}{0.000000in}}%
\pgfpathlineto{\pgfqpoint{-0.027778in}{0.000000in}}%
\pgfusepath{stroke,fill}%
}%
\begin{pgfscope}%
\pgfsys@transformshift{5.814694in}{7.251480in}%
\pgfsys@useobject{currentmarker}{}%
\end{pgfscope}%
\end{pgfscope}%
\begin{pgfscope}%
\pgfsetbuttcap%
\pgfsetroundjoin%
\definecolor{currentfill}{rgb}{0.000000,0.000000,0.000000}%
\pgfsetfillcolor{currentfill}%
\pgfsetlinewidth{0.602250pt}%
\definecolor{currentstroke}{rgb}{0.000000,0.000000,0.000000}%
\pgfsetstrokecolor{currentstroke}%
\pgfsetdash{}{0pt}%
\pgfsys@defobject{currentmarker}{\pgfqpoint{-0.027778in}{0.000000in}}{\pgfqpoint{0.000000in}{0.000000in}}{%
\pgfpathmoveto{\pgfqpoint{0.000000in}{0.000000in}}%
\pgfpathlineto{\pgfqpoint{-0.027778in}{0.000000in}}%
\pgfusepath{stroke,fill}%
}%
\begin{pgfscope}%
\pgfsys@transformshift{5.814694in}{7.305438in}%
\pgfsys@useobject{currentmarker}{}%
\end{pgfscope}%
\end{pgfscope}%
\begin{pgfscope}%
\pgfsetbuttcap%
\pgfsetroundjoin%
\definecolor{currentfill}{rgb}{0.000000,0.000000,0.000000}%
\pgfsetfillcolor{currentfill}%
\pgfsetlinewidth{0.602250pt}%
\definecolor{currentstroke}{rgb}{0.000000,0.000000,0.000000}%
\pgfsetstrokecolor{currentstroke}%
\pgfsetdash{}{0pt}%
\pgfsys@defobject{currentmarker}{\pgfqpoint{-0.027778in}{0.000000in}}{\pgfqpoint{0.000000in}{0.000000in}}{%
\pgfpathmoveto{\pgfqpoint{0.000000in}{0.000000in}}%
\pgfpathlineto{\pgfqpoint{-0.027778in}{0.000000in}}%
\pgfusepath{stroke,fill}%
}%
\begin{pgfscope}%
\pgfsys@transformshift{5.814694in}{7.347291in}%
\pgfsys@useobject{currentmarker}{}%
\end{pgfscope}%
\end{pgfscope}%
\begin{pgfscope}%
\pgfsetbuttcap%
\pgfsetroundjoin%
\definecolor{currentfill}{rgb}{0.000000,0.000000,0.000000}%
\pgfsetfillcolor{currentfill}%
\pgfsetlinewidth{0.602250pt}%
\definecolor{currentstroke}{rgb}{0.000000,0.000000,0.000000}%
\pgfsetstrokecolor{currentstroke}%
\pgfsetdash{}{0pt}%
\pgfsys@defobject{currentmarker}{\pgfqpoint{-0.027778in}{0.000000in}}{\pgfqpoint{0.000000in}{0.000000in}}{%
\pgfpathmoveto{\pgfqpoint{0.000000in}{0.000000in}}%
\pgfpathlineto{\pgfqpoint{-0.027778in}{0.000000in}}%
\pgfusepath{stroke,fill}%
}%
\begin{pgfscope}%
\pgfsys@transformshift{5.814694in}{7.381487in}%
\pgfsys@useobject{currentmarker}{}%
\end{pgfscope}%
\end{pgfscope}%
\begin{pgfscope}%
\pgfsetbuttcap%
\pgfsetroundjoin%
\definecolor{currentfill}{rgb}{0.000000,0.000000,0.000000}%
\pgfsetfillcolor{currentfill}%
\pgfsetlinewidth{0.602250pt}%
\definecolor{currentstroke}{rgb}{0.000000,0.000000,0.000000}%
\pgfsetstrokecolor{currentstroke}%
\pgfsetdash{}{0pt}%
\pgfsys@defobject{currentmarker}{\pgfqpoint{-0.027778in}{0.000000in}}{\pgfqpoint{0.000000in}{0.000000in}}{%
\pgfpathmoveto{\pgfqpoint{0.000000in}{0.000000in}}%
\pgfpathlineto{\pgfqpoint{-0.027778in}{0.000000in}}%
\pgfusepath{stroke,fill}%
}%
\begin{pgfscope}%
\pgfsys@transformshift{5.814694in}{7.410400in}%
\pgfsys@useobject{currentmarker}{}%
\end{pgfscope}%
\end{pgfscope}%
\begin{pgfscope}%
\pgfsetbuttcap%
\pgfsetroundjoin%
\definecolor{currentfill}{rgb}{0.000000,0.000000,0.000000}%
\pgfsetfillcolor{currentfill}%
\pgfsetlinewidth{0.602250pt}%
\definecolor{currentstroke}{rgb}{0.000000,0.000000,0.000000}%
\pgfsetstrokecolor{currentstroke}%
\pgfsetdash{}{0pt}%
\pgfsys@defobject{currentmarker}{\pgfqpoint{-0.027778in}{0.000000in}}{\pgfqpoint{0.000000in}{0.000000in}}{%
\pgfpathmoveto{\pgfqpoint{0.000000in}{0.000000in}}%
\pgfpathlineto{\pgfqpoint{-0.027778in}{0.000000in}}%
\pgfusepath{stroke,fill}%
}%
\begin{pgfscope}%
\pgfsys@transformshift{5.814694in}{7.435445in}%
\pgfsys@useobject{currentmarker}{}%
\end{pgfscope}%
\end{pgfscope}%
\begin{pgfscope}%
\pgfsetbuttcap%
\pgfsetroundjoin%
\definecolor{currentfill}{rgb}{0.000000,0.000000,0.000000}%
\pgfsetfillcolor{currentfill}%
\pgfsetlinewidth{0.602250pt}%
\definecolor{currentstroke}{rgb}{0.000000,0.000000,0.000000}%
\pgfsetstrokecolor{currentstroke}%
\pgfsetdash{}{0pt}%
\pgfsys@defobject{currentmarker}{\pgfqpoint{-0.027778in}{0.000000in}}{\pgfqpoint{0.000000in}{0.000000in}}{%
\pgfpathmoveto{\pgfqpoint{0.000000in}{0.000000in}}%
\pgfpathlineto{\pgfqpoint{-0.027778in}{0.000000in}}%
\pgfusepath{stroke,fill}%
}%
\begin{pgfscope}%
\pgfsys@transformshift{5.814694in}{7.457536in}%
\pgfsys@useobject{currentmarker}{}%
\end{pgfscope}%
\end{pgfscope}%
\begin{pgfscope}%
\pgfsetbuttcap%
\pgfsetroundjoin%
\definecolor{currentfill}{rgb}{0.000000,0.000000,0.000000}%
\pgfsetfillcolor{currentfill}%
\pgfsetlinewidth{0.602250pt}%
\definecolor{currentstroke}{rgb}{0.000000,0.000000,0.000000}%
\pgfsetstrokecolor{currentstroke}%
\pgfsetdash{}{0pt}%
\pgfsys@defobject{currentmarker}{\pgfqpoint{-0.027778in}{0.000000in}}{\pgfqpoint{0.000000in}{0.000000in}}{%
\pgfpathmoveto{\pgfqpoint{0.000000in}{0.000000in}}%
\pgfpathlineto{\pgfqpoint{-0.027778in}{0.000000in}}%
\pgfusepath{stroke,fill}%
}%
\begin{pgfscope}%
\pgfsys@transformshift{5.814694in}{7.607305in}%
\pgfsys@useobject{currentmarker}{}%
\end{pgfscope}%
\end{pgfscope}%
\begin{pgfscope}%
\pgfsetbuttcap%
\pgfsetroundjoin%
\definecolor{currentfill}{rgb}{0.000000,0.000000,0.000000}%
\pgfsetfillcolor{currentfill}%
\pgfsetlinewidth{0.602250pt}%
\definecolor{currentstroke}{rgb}{0.000000,0.000000,0.000000}%
\pgfsetstrokecolor{currentstroke}%
\pgfsetdash{}{0pt}%
\pgfsys@defobject{currentmarker}{\pgfqpoint{-0.027778in}{0.000000in}}{\pgfqpoint{0.000000in}{0.000000in}}{%
\pgfpathmoveto{\pgfqpoint{0.000000in}{0.000000in}}%
\pgfpathlineto{\pgfqpoint{-0.027778in}{0.000000in}}%
\pgfusepath{stroke,fill}%
}%
\begin{pgfscope}%
\pgfsys@transformshift{5.814694in}{7.683354in}%
\pgfsys@useobject{currentmarker}{}%
\end{pgfscope}%
\end{pgfscope}%
\begin{pgfscope}%
\pgfsetbuttcap%
\pgfsetroundjoin%
\definecolor{currentfill}{rgb}{0.000000,0.000000,0.000000}%
\pgfsetfillcolor{currentfill}%
\pgfsetlinewidth{0.602250pt}%
\definecolor{currentstroke}{rgb}{0.000000,0.000000,0.000000}%
\pgfsetstrokecolor{currentstroke}%
\pgfsetdash{}{0pt}%
\pgfsys@defobject{currentmarker}{\pgfqpoint{-0.027778in}{0.000000in}}{\pgfqpoint{0.000000in}{0.000000in}}{%
\pgfpathmoveto{\pgfqpoint{0.000000in}{0.000000in}}%
\pgfpathlineto{\pgfqpoint{-0.027778in}{0.000000in}}%
\pgfusepath{stroke,fill}%
}%
\begin{pgfscope}%
\pgfsys@transformshift{5.814694in}{7.737312in}%
\pgfsys@useobject{currentmarker}{}%
\end{pgfscope}%
\end{pgfscope}%
\begin{pgfscope}%
\pgfsetbuttcap%
\pgfsetroundjoin%
\definecolor{currentfill}{rgb}{0.000000,0.000000,0.000000}%
\pgfsetfillcolor{currentfill}%
\pgfsetlinewidth{0.602250pt}%
\definecolor{currentstroke}{rgb}{0.000000,0.000000,0.000000}%
\pgfsetstrokecolor{currentstroke}%
\pgfsetdash{}{0pt}%
\pgfsys@defobject{currentmarker}{\pgfqpoint{-0.027778in}{0.000000in}}{\pgfqpoint{0.000000in}{0.000000in}}{%
\pgfpathmoveto{\pgfqpoint{0.000000in}{0.000000in}}%
\pgfpathlineto{\pgfqpoint{-0.027778in}{0.000000in}}%
\pgfusepath{stroke,fill}%
}%
\begin{pgfscope}%
\pgfsys@transformshift{5.814694in}{7.779165in}%
\pgfsys@useobject{currentmarker}{}%
\end{pgfscope}%
\end{pgfscope}%
\begin{pgfscope}%
\pgfsetbuttcap%
\pgfsetroundjoin%
\definecolor{currentfill}{rgb}{0.000000,0.000000,0.000000}%
\pgfsetfillcolor{currentfill}%
\pgfsetlinewidth{0.602250pt}%
\definecolor{currentstroke}{rgb}{0.000000,0.000000,0.000000}%
\pgfsetstrokecolor{currentstroke}%
\pgfsetdash{}{0pt}%
\pgfsys@defobject{currentmarker}{\pgfqpoint{-0.027778in}{0.000000in}}{\pgfqpoint{0.000000in}{0.000000in}}{%
\pgfpathmoveto{\pgfqpoint{0.000000in}{0.000000in}}%
\pgfpathlineto{\pgfqpoint{-0.027778in}{0.000000in}}%
\pgfusepath{stroke,fill}%
}%
\begin{pgfscope}%
\pgfsys@transformshift{5.814694in}{7.813362in}%
\pgfsys@useobject{currentmarker}{}%
\end{pgfscope}%
\end{pgfscope}%
\begin{pgfscope}%
\pgfsetbuttcap%
\pgfsetroundjoin%
\definecolor{currentfill}{rgb}{0.000000,0.000000,0.000000}%
\pgfsetfillcolor{currentfill}%
\pgfsetlinewidth{0.602250pt}%
\definecolor{currentstroke}{rgb}{0.000000,0.000000,0.000000}%
\pgfsetstrokecolor{currentstroke}%
\pgfsetdash{}{0pt}%
\pgfsys@defobject{currentmarker}{\pgfqpoint{-0.027778in}{0.000000in}}{\pgfqpoint{0.000000in}{0.000000in}}{%
\pgfpathmoveto{\pgfqpoint{0.000000in}{0.000000in}}%
\pgfpathlineto{\pgfqpoint{-0.027778in}{0.000000in}}%
\pgfusepath{stroke,fill}%
}%
\begin{pgfscope}%
\pgfsys@transformshift{5.814694in}{7.842274in}%
\pgfsys@useobject{currentmarker}{}%
\end{pgfscope}%
\end{pgfscope}%
\begin{pgfscope}%
\pgfsetbuttcap%
\pgfsetroundjoin%
\definecolor{currentfill}{rgb}{0.000000,0.000000,0.000000}%
\pgfsetfillcolor{currentfill}%
\pgfsetlinewidth{0.602250pt}%
\definecolor{currentstroke}{rgb}{0.000000,0.000000,0.000000}%
\pgfsetstrokecolor{currentstroke}%
\pgfsetdash{}{0pt}%
\pgfsys@defobject{currentmarker}{\pgfqpoint{-0.027778in}{0.000000in}}{\pgfqpoint{0.000000in}{0.000000in}}{%
\pgfpathmoveto{\pgfqpoint{0.000000in}{0.000000in}}%
\pgfpathlineto{\pgfqpoint{-0.027778in}{0.000000in}}%
\pgfusepath{stroke,fill}%
}%
\begin{pgfscope}%
\pgfsys@transformshift{5.814694in}{7.867319in}%
\pgfsys@useobject{currentmarker}{}%
\end{pgfscope}%
\end{pgfscope}%
\begin{pgfscope}%
\pgfsetbuttcap%
\pgfsetroundjoin%
\definecolor{currentfill}{rgb}{0.000000,0.000000,0.000000}%
\pgfsetfillcolor{currentfill}%
\pgfsetlinewidth{0.602250pt}%
\definecolor{currentstroke}{rgb}{0.000000,0.000000,0.000000}%
\pgfsetstrokecolor{currentstroke}%
\pgfsetdash{}{0pt}%
\pgfsys@defobject{currentmarker}{\pgfqpoint{-0.027778in}{0.000000in}}{\pgfqpoint{0.000000in}{0.000000in}}{%
\pgfpathmoveto{\pgfqpoint{0.000000in}{0.000000in}}%
\pgfpathlineto{\pgfqpoint{-0.027778in}{0.000000in}}%
\pgfusepath{stroke,fill}%
}%
\begin{pgfscope}%
\pgfsys@transformshift{5.814694in}{7.889411in}%
\pgfsys@useobject{currentmarker}{}%
\end{pgfscope}%
\end{pgfscope}%
\begin{pgfscope}%
\pgfsetbuttcap%
\pgfsetroundjoin%
\definecolor{currentfill}{rgb}{0.000000,0.000000,0.000000}%
\pgfsetfillcolor{currentfill}%
\pgfsetlinewidth{0.602250pt}%
\definecolor{currentstroke}{rgb}{0.000000,0.000000,0.000000}%
\pgfsetstrokecolor{currentstroke}%
\pgfsetdash{}{0pt}%
\pgfsys@defobject{currentmarker}{\pgfqpoint{-0.027778in}{0.000000in}}{\pgfqpoint{0.000000in}{0.000000in}}{%
\pgfpathmoveto{\pgfqpoint{0.000000in}{0.000000in}}%
\pgfpathlineto{\pgfqpoint{-0.027778in}{0.000000in}}%
\pgfusepath{stroke,fill}%
}%
\begin{pgfscope}%
\pgfsys@transformshift{5.814694in}{8.039180in}%
\pgfsys@useobject{currentmarker}{}%
\end{pgfscope}%
\end{pgfscope}%
\begin{pgfscope}%
\pgfsetbuttcap%
\pgfsetroundjoin%
\definecolor{currentfill}{rgb}{0.000000,0.000000,0.000000}%
\pgfsetfillcolor{currentfill}%
\pgfsetlinewidth{0.602250pt}%
\definecolor{currentstroke}{rgb}{0.000000,0.000000,0.000000}%
\pgfsetstrokecolor{currentstroke}%
\pgfsetdash{}{0pt}%
\pgfsys@defobject{currentmarker}{\pgfqpoint{-0.027778in}{0.000000in}}{\pgfqpoint{0.000000in}{0.000000in}}{%
\pgfpathmoveto{\pgfqpoint{0.000000in}{0.000000in}}%
\pgfpathlineto{\pgfqpoint{-0.027778in}{0.000000in}}%
\pgfusepath{stroke,fill}%
}%
\begin{pgfscope}%
\pgfsys@transformshift{5.814694in}{8.115229in}%
\pgfsys@useobject{currentmarker}{}%
\end{pgfscope}%
\end{pgfscope}%
\begin{pgfscope}%
\pgfsetbuttcap%
\pgfsetroundjoin%
\definecolor{currentfill}{rgb}{0.000000,0.000000,0.000000}%
\pgfsetfillcolor{currentfill}%
\pgfsetlinewidth{0.602250pt}%
\definecolor{currentstroke}{rgb}{0.000000,0.000000,0.000000}%
\pgfsetstrokecolor{currentstroke}%
\pgfsetdash{}{0pt}%
\pgfsys@defobject{currentmarker}{\pgfqpoint{-0.027778in}{0.000000in}}{\pgfqpoint{0.000000in}{0.000000in}}{%
\pgfpathmoveto{\pgfqpoint{0.000000in}{0.000000in}}%
\pgfpathlineto{\pgfqpoint{-0.027778in}{0.000000in}}%
\pgfusepath{stroke,fill}%
}%
\begin{pgfscope}%
\pgfsys@transformshift{5.814694in}{8.169187in}%
\pgfsys@useobject{currentmarker}{}%
\end{pgfscope}%
\end{pgfscope}%
\begin{pgfscope}%
\pgfsetbuttcap%
\pgfsetroundjoin%
\definecolor{currentfill}{rgb}{0.000000,0.000000,0.000000}%
\pgfsetfillcolor{currentfill}%
\pgfsetlinewidth{0.602250pt}%
\definecolor{currentstroke}{rgb}{0.000000,0.000000,0.000000}%
\pgfsetstrokecolor{currentstroke}%
\pgfsetdash{}{0pt}%
\pgfsys@defobject{currentmarker}{\pgfqpoint{-0.027778in}{0.000000in}}{\pgfqpoint{0.000000in}{0.000000in}}{%
\pgfpathmoveto{\pgfqpoint{0.000000in}{0.000000in}}%
\pgfpathlineto{\pgfqpoint{-0.027778in}{0.000000in}}%
\pgfusepath{stroke,fill}%
}%
\begin{pgfscope}%
\pgfsys@transformshift{5.814694in}{8.211040in}%
\pgfsys@useobject{currentmarker}{}%
\end{pgfscope}%
\end{pgfscope}%
\begin{pgfscope}%
\pgfsetbuttcap%
\pgfsetroundjoin%
\definecolor{currentfill}{rgb}{0.000000,0.000000,0.000000}%
\pgfsetfillcolor{currentfill}%
\pgfsetlinewidth{0.602250pt}%
\definecolor{currentstroke}{rgb}{0.000000,0.000000,0.000000}%
\pgfsetstrokecolor{currentstroke}%
\pgfsetdash{}{0pt}%
\pgfsys@defobject{currentmarker}{\pgfqpoint{-0.027778in}{0.000000in}}{\pgfqpoint{0.000000in}{0.000000in}}{%
\pgfpathmoveto{\pgfqpoint{0.000000in}{0.000000in}}%
\pgfpathlineto{\pgfqpoint{-0.027778in}{0.000000in}}%
\pgfusepath{stroke,fill}%
}%
\begin{pgfscope}%
\pgfsys@transformshift{5.814694in}{8.245236in}%
\pgfsys@useobject{currentmarker}{}%
\end{pgfscope}%
\end{pgfscope}%
\begin{pgfscope}%
\pgfsetbuttcap%
\pgfsetroundjoin%
\definecolor{currentfill}{rgb}{0.000000,0.000000,0.000000}%
\pgfsetfillcolor{currentfill}%
\pgfsetlinewidth{0.602250pt}%
\definecolor{currentstroke}{rgb}{0.000000,0.000000,0.000000}%
\pgfsetstrokecolor{currentstroke}%
\pgfsetdash{}{0pt}%
\pgfsys@defobject{currentmarker}{\pgfqpoint{-0.027778in}{0.000000in}}{\pgfqpoint{0.000000in}{0.000000in}}{%
\pgfpathmoveto{\pgfqpoint{0.000000in}{0.000000in}}%
\pgfpathlineto{\pgfqpoint{-0.027778in}{0.000000in}}%
\pgfusepath{stroke,fill}%
}%
\begin{pgfscope}%
\pgfsys@transformshift{5.814694in}{8.274149in}%
\pgfsys@useobject{currentmarker}{}%
\end{pgfscope}%
\end{pgfscope}%
\begin{pgfscope}%
\pgfsetbuttcap%
\pgfsetroundjoin%
\definecolor{currentfill}{rgb}{0.000000,0.000000,0.000000}%
\pgfsetfillcolor{currentfill}%
\pgfsetlinewidth{0.602250pt}%
\definecolor{currentstroke}{rgb}{0.000000,0.000000,0.000000}%
\pgfsetstrokecolor{currentstroke}%
\pgfsetdash{}{0pt}%
\pgfsys@defobject{currentmarker}{\pgfqpoint{-0.027778in}{0.000000in}}{\pgfqpoint{0.000000in}{0.000000in}}{%
\pgfpathmoveto{\pgfqpoint{0.000000in}{0.000000in}}%
\pgfpathlineto{\pgfqpoint{-0.027778in}{0.000000in}}%
\pgfusepath{stroke,fill}%
}%
\begin{pgfscope}%
\pgfsys@transformshift{5.814694in}{8.299194in}%
\pgfsys@useobject{currentmarker}{}%
\end{pgfscope}%
\end{pgfscope}%
\begin{pgfscope}%
\pgfsetbuttcap%
\pgfsetroundjoin%
\definecolor{currentfill}{rgb}{0.000000,0.000000,0.000000}%
\pgfsetfillcolor{currentfill}%
\pgfsetlinewidth{0.602250pt}%
\definecolor{currentstroke}{rgb}{0.000000,0.000000,0.000000}%
\pgfsetstrokecolor{currentstroke}%
\pgfsetdash{}{0pt}%
\pgfsys@defobject{currentmarker}{\pgfqpoint{-0.027778in}{0.000000in}}{\pgfqpoint{0.000000in}{0.000000in}}{%
\pgfpathmoveto{\pgfqpoint{0.000000in}{0.000000in}}%
\pgfpathlineto{\pgfqpoint{-0.027778in}{0.000000in}}%
\pgfusepath{stroke,fill}%
}%
\begin{pgfscope}%
\pgfsys@transformshift{5.814694in}{8.321285in}%
\pgfsys@useobject{currentmarker}{}%
\end{pgfscope}%
\end{pgfscope}%
\begin{pgfscope}%
\pgfpathrectangle{\pgfqpoint{5.814694in}{6.967719in}}{\pgfqpoint{1.897959in}{1.372727in}} %
\pgfusepath{clip}%
\pgfsetbuttcap%
\pgfsetroundjoin%
\pgfsetlinewidth{1.505625pt}%
\definecolor{currentstroke}{rgb}{1.000000,0.000000,0.000000}%
\pgfsetstrokecolor{currentstroke}%
\pgfsetdash{{5.550000pt}{2.400000pt}}{0.000000pt}%
\pgfpathmoveto{\pgfqpoint{5.900965in}{8.182145in}}%
\pgfpathlineto{\pgfqpoint{6.073506in}{8.179883in}}%
\pgfpathlineto{\pgfqpoint{6.246048in}{8.177803in}}%
\pgfpathlineto{\pgfqpoint{6.418590in}{8.175900in}}%
\pgfpathlineto{\pgfqpoint{6.591132in}{8.174169in}}%
\pgfpathlineto{\pgfqpoint{6.763673in}{8.172607in}}%
\pgfpathlineto{\pgfqpoint{6.936215in}{8.171211in}}%
\pgfpathlineto{\pgfqpoint{7.108757in}{8.169980in}}%
\pgfpathlineto{\pgfqpoint{7.281299in}{8.168910in}}%
\pgfpathlineto{\pgfqpoint{7.453840in}{8.168002in}}%
\pgfpathlineto{\pgfqpoint{7.626382in}{8.167253in}}%
\pgfusepath{stroke}%
\end{pgfscope}%
\begin{pgfscope}%
\pgfpathrectangle{\pgfqpoint{5.814694in}{6.967719in}}{\pgfqpoint{1.897959in}{1.372727in}} %
\pgfusepath{clip}%
\pgfsetbuttcap%
\pgfsetmiterjoin%
\definecolor{currentfill}{rgb}{1.000000,0.000000,0.000000}%
\pgfsetfillcolor{currentfill}%
\pgfsetlinewidth{1.003750pt}%
\definecolor{currentstroke}{rgb}{1.000000,0.000000,0.000000}%
\pgfsetstrokecolor{currentstroke}%
\pgfsetdash{}{0pt}%
\pgfsys@defobject{currentmarker}{\pgfqpoint{-0.041667in}{-0.041667in}}{\pgfqpoint{0.041667in}{0.041667in}}{%
\pgfpathmoveto{\pgfqpoint{-0.041667in}{-0.041667in}}%
\pgfpathlineto{\pgfqpoint{0.041667in}{-0.041667in}}%
\pgfpathlineto{\pgfqpoint{0.041667in}{0.041667in}}%
\pgfpathlineto{\pgfqpoint{-0.041667in}{0.041667in}}%
\pgfpathclose%
\pgfusepath{stroke,fill}%
}%
\begin{pgfscope}%
\pgfsys@transformshift{5.900965in}{8.182145in}%
\pgfsys@useobject{currentmarker}{}%
\end{pgfscope}%
\begin{pgfscope}%
\pgfsys@transformshift{6.246048in}{8.177803in}%
\pgfsys@useobject{currentmarker}{}%
\end{pgfscope}%
\begin{pgfscope}%
\pgfsys@transformshift{6.591132in}{8.174169in}%
\pgfsys@useobject{currentmarker}{}%
\end{pgfscope}%
\begin{pgfscope}%
\pgfsys@transformshift{6.936215in}{8.171211in}%
\pgfsys@useobject{currentmarker}{}%
\end{pgfscope}%
\begin{pgfscope}%
\pgfsys@transformshift{7.281299in}{8.168910in}%
\pgfsys@useobject{currentmarker}{}%
\end{pgfscope}%
\begin{pgfscope}%
\pgfsys@transformshift{7.626382in}{8.167253in}%
\pgfsys@useobject{currentmarker}{}%
\end{pgfscope}%
\end{pgfscope}%
\begin{pgfscope}%
\pgfpathrectangle{\pgfqpoint{5.814694in}{6.967719in}}{\pgfqpoint{1.897959in}{1.372727in}} %
\pgfusepath{clip}%
\pgfsetrectcap%
\pgfsetroundjoin%
\pgfsetlinewidth{1.505625pt}%
\definecolor{currentstroke}{rgb}{0.000000,0.000000,1.000000}%
\pgfsetstrokecolor{currentstroke}%
\pgfsetdash{}{0pt}%
\pgfpathmoveto{\pgfqpoint{5.900965in}{7.338250in}}%
\pgfpathlineto{\pgfqpoint{6.073506in}{7.316061in}}%
\pgfpathlineto{\pgfqpoint{6.246048in}{7.293235in}}%
\pgfpathlineto{\pgfqpoint{6.418590in}{7.269439in}}%
\pgfpathlineto{\pgfqpoint{6.591132in}{7.244315in}}%
\pgfpathlineto{\pgfqpoint{6.763673in}{7.217451in}}%
\pgfpathlineto{\pgfqpoint{6.936215in}{7.188329in}}%
\pgfpathlineto{\pgfqpoint{7.108757in}{7.156268in}}%
\pgfpathlineto{\pgfqpoint{7.281299in}{7.120311in}}%
\pgfpathlineto{\pgfqpoint{7.453840in}{7.079028in}}%
\pgfpathlineto{\pgfqpoint{7.626382in}{7.030115in}}%
\pgfusepath{stroke}%
\end{pgfscope}%
\begin{pgfscope}%
\pgfpathrectangle{\pgfqpoint{5.814694in}{6.967719in}}{\pgfqpoint{1.897959in}{1.372727in}} %
\pgfusepath{clip}%
\pgfsetbuttcap%
\pgfsetroundjoin%
\definecolor{currentfill}{rgb}{0.000000,0.000000,1.000000}%
\pgfsetfillcolor{currentfill}%
\pgfsetlinewidth{1.003750pt}%
\definecolor{currentstroke}{rgb}{0.000000,0.000000,1.000000}%
\pgfsetstrokecolor{currentstroke}%
\pgfsetdash{}{0pt}%
\pgfsys@defobject{currentmarker}{\pgfqpoint{-0.041667in}{-0.041667in}}{\pgfqpoint{0.041667in}{0.041667in}}{%
\pgfpathmoveto{\pgfqpoint{0.000000in}{-0.041667in}}%
\pgfpathcurveto{\pgfqpoint{0.011050in}{-0.041667in}}{\pgfqpoint{0.021649in}{-0.037276in}}{\pgfqpoint{0.029463in}{-0.029463in}}%
\pgfpathcurveto{\pgfqpoint{0.037276in}{-0.021649in}}{\pgfqpoint{0.041667in}{-0.011050in}}{\pgfqpoint{0.041667in}{0.000000in}}%
\pgfpathcurveto{\pgfqpoint{0.041667in}{0.011050in}}{\pgfqpoint{0.037276in}{0.021649in}}{\pgfqpoint{0.029463in}{0.029463in}}%
\pgfpathcurveto{\pgfqpoint{0.021649in}{0.037276in}}{\pgfqpoint{0.011050in}{0.041667in}}{\pgfqpoint{0.000000in}{0.041667in}}%
\pgfpathcurveto{\pgfqpoint{-0.011050in}{0.041667in}}{\pgfqpoint{-0.021649in}{0.037276in}}{\pgfqpoint{-0.029463in}{0.029463in}}%
\pgfpathcurveto{\pgfqpoint{-0.037276in}{0.021649in}}{\pgfqpoint{-0.041667in}{0.011050in}}{\pgfqpoint{-0.041667in}{0.000000in}}%
\pgfpathcurveto{\pgfqpoint{-0.041667in}{-0.011050in}}{\pgfqpoint{-0.037276in}{-0.021649in}}{\pgfqpoint{-0.029463in}{-0.029463in}}%
\pgfpathcurveto{\pgfqpoint{-0.021649in}{-0.037276in}}{\pgfqpoint{-0.011050in}{-0.041667in}}{\pgfqpoint{0.000000in}{-0.041667in}}%
\pgfpathclose%
\pgfusepath{stroke,fill}%
}%
\begin{pgfscope}%
\pgfsys@transformshift{5.900965in}{7.338250in}%
\pgfsys@useobject{currentmarker}{}%
\end{pgfscope}%
\begin{pgfscope}%
\pgfsys@transformshift{6.246048in}{7.293235in}%
\pgfsys@useobject{currentmarker}{}%
\end{pgfscope}%
\begin{pgfscope}%
\pgfsys@transformshift{6.591132in}{7.244315in}%
\pgfsys@useobject{currentmarker}{}%
\end{pgfscope}%
\begin{pgfscope}%
\pgfsys@transformshift{6.936215in}{7.188329in}%
\pgfsys@useobject{currentmarker}{}%
\end{pgfscope}%
\begin{pgfscope}%
\pgfsys@transformshift{7.281299in}{7.120311in}%
\pgfsys@useobject{currentmarker}{}%
\end{pgfscope}%
\begin{pgfscope}%
\pgfsys@transformshift{7.626382in}{7.030115in}%
\pgfsys@useobject{currentmarker}{}%
\end{pgfscope}%
\end{pgfscope}%
\begin{pgfscope}%
\pgfpathrectangle{\pgfqpoint{5.814694in}{6.967719in}}{\pgfqpoint{1.897959in}{1.372727in}} %
\pgfusepath{clip}%
\pgfsetbuttcap%
\pgfsetroundjoin%
\pgfsetlinewidth{1.505625pt}%
\definecolor{currentstroke}{rgb}{0.000000,0.750000,0.750000}%
\pgfsetstrokecolor{currentstroke}%
\pgfsetdash{{9.600000pt}{2.400000pt}{1.500000pt}{2.400000pt}}{0.000000pt}%
\pgfpathmoveto{\pgfqpoint{5.900965in}{8.105465in}}%
\pgfpathlineto{\pgfqpoint{6.073506in}{8.079770in}}%
\pgfpathlineto{\pgfqpoint{6.246048in}{8.057748in}}%
\pgfpathlineto{\pgfqpoint{6.418590in}{8.038780in}}%
\pgfpathlineto{\pgfqpoint{6.591132in}{8.022408in}}%
\pgfpathlineto{\pgfqpoint{6.763673in}{8.008288in}}%
\pgfpathlineto{\pgfqpoint{6.936215in}{7.996153in}}%
\pgfpathlineto{\pgfqpoint{7.108757in}{7.985797in}}%
\pgfpathlineto{\pgfqpoint{7.281299in}{7.977057in}}%
\pgfpathlineto{\pgfqpoint{7.453840in}{7.969807in}}%
\pgfpathlineto{\pgfqpoint{7.626382in}{7.963948in}}%
\pgfusepath{stroke}%
\end{pgfscope}%
\begin{pgfscope}%
\pgfpathrectangle{\pgfqpoint{5.814694in}{6.967719in}}{\pgfqpoint{1.897959in}{1.372727in}} %
\pgfusepath{clip}%
\pgfsetbuttcap%
\pgfsetmiterjoin%
\definecolor{currentfill}{rgb}{0.000000,0.750000,0.750000}%
\pgfsetfillcolor{currentfill}%
\pgfsetlinewidth{1.003750pt}%
\definecolor{currentstroke}{rgb}{0.000000,0.750000,0.750000}%
\pgfsetstrokecolor{currentstroke}%
\pgfsetdash{}{0pt}%
\pgfsys@defobject{currentmarker}{\pgfqpoint{-0.041667in}{-0.041667in}}{\pgfqpoint{0.041667in}{0.041667in}}{%
\pgfpathmoveto{\pgfqpoint{-0.000000in}{-0.041667in}}%
\pgfpathlineto{\pgfqpoint{0.041667in}{0.041667in}}%
\pgfpathlineto{\pgfqpoint{-0.041667in}{0.041667in}}%
\pgfpathclose%
\pgfusepath{stroke,fill}%
}%
\begin{pgfscope}%
\pgfsys@transformshift{5.900965in}{8.105465in}%
\pgfsys@useobject{currentmarker}{}%
\end{pgfscope}%
\begin{pgfscope}%
\pgfsys@transformshift{6.246048in}{8.057748in}%
\pgfsys@useobject{currentmarker}{}%
\end{pgfscope}%
\begin{pgfscope}%
\pgfsys@transformshift{6.591132in}{8.022408in}%
\pgfsys@useobject{currentmarker}{}%
\end{pgfscope}%
\begin{pgfscope}%
\pgfsys@transformshift{6.936215in}{7.996153in}%
\pgfsys@useobject{currentmarker}{}%
\end{pgfscope}%
\begin{pgfscope}%
\pgfsys@transformshift{7.281299in}{7.977057in}%
\pgfsys@useobject{currentmarker}{}%
\end{pgfscope}%
\begin{pgfscope}%
\pgfsys@transformshift{7.626382in}{7.963948in}%
\pgfsys@useobject{currentmarker}{}%
\end{pgfscope}%
\end{pgfscope}%
\begin{pgfscope}%
\pgfpathrectangle{\pgfqpoint{5.814694in}{6.967719in}}{\pgfqpoint{1.897959in}{1.372727in}} %
\pgfusepath{clip}%
\pgfsetbuttcap%
\pgfsetroundjoin%
\pgfsetlinewidth{1.505625pt}%
\definecolor{currentstroke}{rgb}{0.000000,0.000000,0.000000}%
\pgfsetstrokecolor{currentstroke}%
\pgfsetdash{{1.500000pt}{2.475000pt}}{0.000000pt}%
\pgfpathmoveto{\pgfqpoint{5.900965in}{8.278049in}}%
\pgfpathlineto{\pgfqpoint{6.073506in}{8.268048in}}%
\pgfpathlineto{\pgfqpoint{6.246048in}{8.258651in}}%
\pgfpathlineto{\pgfqpoint{6.418590in}{8.250866in}}%
\pgfpathlineto{\pgfqpoint{6.591132in}{8.244355in}}%
\pgfpathlineto{\pgfqpoint{6.763673in}{8.238877in}}%
\pgfpathlineto{\pgfqpoint{6.936215in}{8.234261in}}%
\pgfpathlineto{\pgfqpoint{7.108757in}{8.230379in}}%
\pgfpathlineto{\pgfqpoint{7.281299in}{8.227137in}}%
\pgfpathlineto{\pgfqpoint{7.453840in}{8.224465in}}%
\pgfpathlineto{\pgfqpoint{7.626382in}{8.222311in}}%
\pgfusepath{stroke}%
\end{pgfscope}%
\begin{pgfscope}%
\pgfpathrectangle{\pgfqpoint{5.814694in}{6.967719in}}{\pgfqpoint{1.897959in}{1.372727in}} %
\pgfusepath{clip}%
\pgfsetbuttcap%
\pgfsetroundjoin%
\definecolor{currentfill}{rgb}{0.000000,0.000000,0.000000}%
\pgfsetfillcolor{currentfill}%
\pgfsetlinewidth{1.003750pt}%
\definecolor{currentstroke}{rgb}{0.000000,0.000000,0.000000}%
\pgfsetstrokecolor{currentstroke}%
\pgfsetdash{}{0pt}%
\pgfsys@defobject{currentmarker}{\pgfqpoint{-0.041667in}{-0.041667in}}{\pgfqpoint{0.041667in}{0.041667in}}{%
\pgfpathmoveto{\pgfqpoint{-0.041667in}{0.000000in}}%
\pgfpathlineto{\pgfqpoint{0.041667in}{0.000000in}}%
\pgfpathmoveto{\pgfqpoint{0.000000in}{-0.041667in}}%
\pgfpathlineto{\pgfqpoint{0.000000in}{0.041667in}}%
\pgfusepath{stroke,fill}%
}%
\begin{pgfscope}%
\pgfsys@transformshift{5.900965in}{8.278049in}%
\pgfsys@useobject{currentmarker}{}%
\end{pgfscope}%
\begin{pgfscope}%
\pgfsys@transformshift{6.246048in}{8.258651in}%
\pgfsys@useobject{currentmarker}{}%
\end{pgfscope}%
\begin{pgfscope}%
\pgfsys@transformshift{6.591132in}{8.244355in}%
\pgfsys@useobject{currentmarker}{}%
\end{pgfscope}%
\begin{pgfscope}%
\pgfsys@transformshift{6.936215in}{8.234261in}%
\pgfsys@useobject{currentmarker}{}%
\end{pgfscope}%
\begin{pgfscope}%
\pgfsys@transformshift{7.281299in}{8.227137in}%
\pgfsys@useobject{currentmarker}{}%
\end{pgfscope}%
\begin{pgfscope}%
\pgfsys@transformshift{7.626382in}{8.222311in}%
\pgfsys@useobject{currentmarker}{}%
\end{pgfscope}%
\end{pgfscope}%
\begin{pgfscope}%
\pgfsetrectcap%
\pgfsetmiterjoin%
\pgfsetlinewidth{0.803000pt}%
\definecolor{currentstroke}{rgb}{0.000000,0.000000,0.000000}%
\pgfsetstrokecolor{currentstroke}%
\pgfsetdash{}{0pt}%
\pgfpathmoveto{\pgfqpoint{5.814694in}{6.967719in}}%
\pgfpathlineto{\pgfqpoint{5.814694in}{8.340446in}}%
\pgfusepath{stroke}%
\end{pgfscope}%
\begin{pgfscope}%
\pgfsetrectcap%
\pgfsetmiterjoin%
\pgfsetlinewidth{0.803000pt}%
\definecolor{currentstroke}{rgb}{0.000000,0.000000,0.000000}%
\pgfsetstrokecolor{currentstroke}%
\pgfsetdash{}{0pt}%
\pgfpathmoveto{\pgfqpoint{7.712653in}{6.967719in}}%
\pgfpathlineto{\pgfqpoint{7.712653in}{8.340446in}}%
\pgfusepath{stroke}%
\end{pgfscope}%
\begin{pgfscope}%
\pgfsetrectcap%
\pgfsetmiterjoin%
\pgfsetlinewidth{0.803000pt}%
\definecolor{currentstroke}{rgb}{0.000000,0.000000,0.000000}%
\pgfsetstrokecolor{currentstroke}%
\pgfsetdash{}{0pt}%
\pgfpathmoveto{\pgfqpoint{5.814694in}{6.967719in}}%
\pgfpathlineto{\pgfqpoint{7.712653in}{6.967719in}}%
\pgfusepath{stroke}%
\end{pgfscope}%
\begin{pgfscope}%
\pgfsetrectcap%
\pgfsetmiterjoin%
\pgfsetlinewidth{0.803000pt}%
\definecolor{currentstroke}{rgb}{0.000000,0.000000,0.000000}%
\pgfsetstrokecolor{currentstroke}%
\pgfsetdash{}{0pt}%
\pgfpathmoveto{\pgfqpoint{5.814694in}{8.340446in}}%
\pgfpathlineto{\pgfqpoint{7.712653in}{8.340446in}}%
\pgfusepath{stroke}%
\end{pgfscope}%
\begin{pgfscope}%
\pgfsetbuttcap%
\pgfsetmiterjoin%
\definecolor{currentfill}{rgb}{1.000000,1.000000,1.000000}%
\pgfsetfillcolor{currentfill}%
\pgfsetlinewidth{0.000000pt}%
\definecolor{currentstroke}{rgb}{0.000000,0.000000,0.000000}%
\pgfsetstrokecolor{currentstroke}%
\pgfsetstrokeopacity{0.000000}%
\pgfsetdash{}{0pt}%
\pgfpathmoveto{\pgfqpoint{8.282041in}{6.967719in}}%
\pgfpathlineto{\pgfqpoint{10.180000in}{6.967719in}}%
\pgfpathlineto{\pgfqpoint{10.180000in}{8.340446in}}%
\pgfpathlineto{\pgfqpoint{8.282041in}{8.340446in}}%
\pgfpathclose%
\pgfusepath{fill}%
\end{pgfscope}%
\begin{pgfscope}%
\pgfsetbuttcap%
\pgfsetroundjoin%
\definecolor{currentfill}{rgb}{0.000000,0.000000,0.000000}%
\pgfsetfillcolor{currentfill}%
\pgfsetlinewidth{0.803000pt}%
\definecolor{currentstroke}{rgb}{0.000000,0.000000,0.000000}%
\pgfsetstrokecolor{currentstroke}%
\pgfsetdash{}{0pt}%
\pgfsys@defobject{currentmarker}{\pgfqpoint{0.000000in}{-0.048611in}}{\pgfqpoint{0.000000in}{0.000000in}}{%
\pgfpathmoveto{\pgfqpoint{0.000000in}{0.000000in}}%
\pgfpathlineto{\pgfqpoint{0.000000in}{-0.048611in}}%
\pgfusepath{stroke,fill}%
}%
\begin{pgfscope}%
\pgfsys@transformshift{8.540853in}{6.967719in}%
\pgfsys@useobject{currentmarker}{}%
\end{pgfscope}%
\end{pgfscope}%
\begin{pgfscope}%
\pgftext[x=8.540853in,y=6.870496in,,top]{\rmfamily\fontsize{10.000000}{12.000000}\selectfont \(\displaystyle 0.100\)}%
\end{pgfscope}%
\begin{pgfscope}%
\pgfsetbuttcap%
\pgfsetroundjoin%
\definecolor{currentfill}{rgb}{0.000000,0.000000,0.000000}%
\pgfsetfillcolor{currentfill}%
\pgfsetlinewidth{0.803000pt}%
\definecolor{currentstroke}{rgb}{0.000000,0.000000,0.000000}%
\pgfsetstrokecolor{currentstroke}%
\pgfsetdash{}{0pt}%
\pgfsys@defobject{currentmarker}{\pgfqpoint{0.000000in}{-0.048611in}}{\pgfqpoint{0.000000in}{0.000000in}}{%
\pgfpathmoveto{\pgfqpoint{0.000000in}{0.000000in}}%
\pgfpathlineto{\pgfqpoint{0.000000in}{-0.048611in}}%
\pgfusepath{stroke,fill}%
}%
\begin{pgfscope}%
\pgfsys@transformshift{9.058479in}{6.967719in}%
\pgfsys@useobject{currentmarker}{}%
\end{pgfscope}%
\end{pgfscope}%
\begin{pgfscope}%
\pgftext[x=9.058479in,y=6.870496in,,top]{\rmfamily\fontsize{10.000000}{12.000000}\selectfont \(\displaystyle 0.125\)}%
\end{pgfscope}%
\begin{pgfscope}%
\pgfsetbuttcap%
\pgfsetroundjoin%
\definecolor{currentfill}{rgb}{0.000000,0.000000,0.000000}%
\pgfsetfillcolor{currentfill}%
\pgfsetlinewidth{0.803000pt}%
\definecolor{currentstroke}{rgb}{0.000000,0.000000,0.000000}%
\pgfsetstrokecolor{currentstroke}%
\pgfsetdash{}{0pt}%
\pgfsys@defobject{currentmarker}{\pgfqpoint{0.000000in}{-0.048611in}}{\pgfqpoint{0.000000in}{0.000000in}}{%
\pgfpathmoveto{\pgfqpoint{0.000000in}{0.000000in}}%
\pgfpathlineto{\pgfqpoint{0.000000in}{-0.048611in}}%
\pgfusepath{stroke,fill}%
}%
\begin{pgfscope}%
\pgfsys@transformshift{9.576104in}{6.967719in}%
\pgfsys@useobject{currentmarker}{}%
\end{pgfscope}%
\end{pgfscope}%
\begin{pgfscope}%
\pgftext[x=9.576104in,y=6.870496in,,top]{\rmfamily\fontsize{10.000000}{12.000000}\selectfont \(\displaystyle 0.150\)}%
\end{pgfscope}%
\begin{pgfscope}%
\pgfsetbuttcap%
\pgfsetroundjoin%
\definecolor{currentfill}{rgb}{0.000000,0.000000,0.000000}%
\pgfsetfillcolor{currentfill}%
\pgfsetlinewidth{0.803000pt}%
\definecolor{currentstroke}{rgb}{0.000000,0.000000,0.000000}%
\pgfsetstrokecolor{currentstroke}%
\pgfsetdash{}{0pt}%
\pgfsys@defobject{currentmarker}{\pgfqpoint{0.000000in}{-0.048611in}}{\pgfqpoint{0.000000in}{0.000000in}}{%
\pgfpathmoveto{\pgfqpoint{0.000000in}{0.000000in}}%
\pgfpathlineto{\pgfqpoint{0.000000in}{-0.048611in}}%
\pgfusepath{stroke,fill}%
}%
\begin{pgfscope}%
\pgfsys@transformshift{10.093729in}{6.967719in}%
\pgfsys@useobject{currentmarker}{}%
\end{pgfscope}%
\end{pgfscope}%
\begin{pgfscope}%
\pgftext[x=10.093729in,y=6.870496in,,top]{\rmfamily\fontsize{10.000000}{12.000000}\selectfont \(\displaystyle 0.175\)}%
\end{pgfscope}%
\begin{pgfscope}%
\pgfsetbuttcap%
\pgfsetroundjoin%
\definecolor{currentfill}{rgb}{0.000000,0.000000,0.000000}%
\pgfsetfillcolor{currentfill}%
\pgfsetlinewidth{0.803000pt}%
\definecolor{currentstroke}{rgb}{0.000000,0.000000,0.000000}%
\pgfsetstrokecolor{currentstroke}%
\pgfsetdash{}{0pt}%
\pgfsys@defobject{currentmarker}{\pgfqpoint{-0.048611in}{0.000000in}}{\pgfqpoint{0.000000in}{0.000000in}}{%
\pgfpathmoveto{\pgfqpoint{0.000000in}{0.000000in}}%
\pgfpathlineto{\pgfqpoint{-0.048611in}{0.000000in}}%
\pgfusepath{stroke,fill}%
}%
\begin{pgfscope}%
\pgfsys@transformshift{8.282041in}{6.987653in}%
\pgfsys@useobject{currentmarker}{}%
\end{pgfscope}%
\end{pgfscope}%
\begin{pgfscope}%
\pgftext[x=7.896816in,y=6.934892in,left,base]{\rmfamily\fontsize{10.000000}{12.000000}\selectfont \(\displaystyle 10^{-7}\)}%
\end{pgfscope}%
\begin{pgfscope}%
\pgfsetbuttcap%
\pgfsetroundjoin%
\definecolor{currentfill}{rgb}{0.000000,0.000000,0.000000}%
\pgfsetfillcolor{currentfill}%
\pgfsetlinewidth{0.803000pt}%
\definecolor{currentstroke}{rgb}{0.000000,0.000000,0.000000}%
\pgfsetstrokecolor{currentstroke}%
\pgfsetdash{}{0pt}%
\pgfsys@defobject{currentmarker}{\pgfqpoint{-0.048611in}{0.000000in}}{\pgfqpoint{0.000000in}{0.000000in}}{%
\pgfpathmoveto{\pgfqpoint{0.000000in}{0.000000in}}%
\pgfpathlineto{\pgfqpoint{-0.048611in}{0.000000in}}%
\pgfusepath{stroke,fill}%
}%
\begin{pgfscope}%
\pgfsys@transformshift{8.282041in}{7.380664in}%
\pgfsys@useobject{currentmarker}{}%
\end{pgfscope}%
\end{pgfscope}%
\begin{pgfscope}%
\pgftext[x=7.896816in,y=7.327903in,left,base]{\rmfamily\fontsize{10.000000}{12.000000}\selectfont \(\displaystyle 10^{-6}\)}%
\end{pgfscope}%
\begin{pgfscope}%
\pgfsetbuttcap%
\pgfsetroundjoin%
\definecolor{currentfill}{rgb}{0.000000,0.000000,0.000000}%
\pgfsetfillcolor{currentfill}%
\pgfsetlinewidth{0.803000pt}%
\definecolor{currentstroke}{rgb}{0.000000,0.000000,0.000000}%
\pgfsetstrokecolor{currentstroke}%
\pgfsetdash{}{0pt}%
\pgfsys@defobject{currentmarker}{\pgfqpoint{-0.048611in}{0.000000in}}{\pgfqpoint{0.000000in}{0.000000in}}{%
\pgfpathmoveto{\pgfqpoint{0.000000in}{0.000000in}}%
\pgfpathlineto{\pgfqpoint{-0.048611in}{0.000000in}}%
\pgfusepath{stroke,fill}%
}%
\begin{pgfscope}%
\pgfsys@transformshift{8.282041in}{7.773676in}%
\pgfsys@useobject{currentmarker}{}%
\end{pgfscope}%
\end{pgfscope}%
\begin{pgfscope}%
\pgftext[x=7.896816in,y=7.720914in,left,base]{\rmfamily\fontsize{10.000000}{12.000000}\selectfont \(\displaystyle 10^{-5}\)}%
\end{pgfscope}%
\begin{pgfscope}%
\pgfsetbuttcap%
\pgfsetroundjoin%
\definecolor{currentfill}{rgb}{0.000000,0.000000,0.000000}%
\pgfsetfillcolor{currentfill}%
\pgfsetlinewidth{0.803000pt}%
\definecolor{currentstroke}{rgb}{0.000000,0.000000,0.000000}%
\pgfsetstrokecolor{currentstroke}%
\pgfsetdash{}{0pt}%
\pgfsys@defobject{currentmarker}{\pgfqpoint{-0.048611in}{0.000000in}}{\pgfqpoint{0.000000in}{0.000000in}}{%
\pgfpathmoveto{\pgfqpoint{0.000000in}{0.000000in}}%
\pgfpathlineto{\pgfqpoint{-0.048611in}{0.000000in}}%
\pgfusepath{stroke,fill}%
}%
\begin{pgfscope}%
\pgfsys@transformshift{8.282041in}{8.166687in}%
\pgfsys@useobject{currentmarker}{}%
\end{pgfscope}%
\end{pgfscope}%
\begin{pgfscope}%
\pgftext[x=7.896816in,y=8.113926in,left,base]{\rmfamily\fontsize{10.000000}{12.000000}\selectfont \(\displaystyle 10^{-4}\)}%
\end{pgfscope}%
\begin{pgfscope}%
\pgfsetbuttcap%
\pgfsetroundjoin%
\definecolor{currentfill}{rgb}{0.000000,0.000000,0.000000}%
\pgfsetfillcolor{currentfill}%
\pgfsetlinewidth{0.602250pt}%
\definecolor{currentstroke}{rgb}{0.000000,0.000000,0.000000}%
\pgfsetstrokecolor{currentstroke}%
\pgfsetdash{}{0pt}%
\pgfsys@defobject{currentmarker}{\pgfqpoint{-0.027778in}{0.000000in}}{\pgfqpoint{0.000000in}{0.000000in}}{%
\pgfpathmoveto{\pgfqpoint{0.000000in}{0.000000in}}%
\pgfpathlineto{\pgfqpoint{-0.027778in}{0.000000in}}%
\pgfusepath{stroke,fill}%
}%
\begin{pgfscope}%
\pgfsys@transformshift{8.282041in}{6.969670in}%
\pgfsys@useobject{currentmarker}{}%
\end{pgfscope}%
\end{pgfscope}%
\begin{pgfscope}%
\pgfsetbuttcap%
\pgfsetroundjoin%
\definecolor{currentfill}{rgb}{0.000000,0.000000,0.000000}%
\pgfsetfillcolor{currentfill}%
\pgfsetlinewidth{0.602250pt}%
\definecolor{currentstroke}{rgb}{0.000000,0.000000,0.000000}%
\pgfsetstrokecolor{currentstroke}%
\pgfsetdash{}{0pt}%
\pgfsys@defobject{currentmarker}{\pgfqpoint{-0.027778in}{0.000000in}}{\pgfqpoint{0.000000in}{0.000000in}}{%
\pgfpathmoveto{\pgfqpoint{0.000000in}{0.000000in}}%
\pgfpathlineto{\pgfqpoint{-0.027778in}{0.000000in}}%
\pgfusepath{stroke,fill}%
}%
\begin{pgfscope}%
\pgfsys@transformshift{8.282041in}{7.105961in}%
\pgfsys@useobject{currentmarker}{}%
\end{pgfscope}%
\end{pgfscope}%
\begin{pgfscope}%
\pgfsetbuttcap%
\pgfsetroundjoin%
\definecolor{currentfill}{rgb}{0.000000,0.000000,0.000000}%
\pgfsetfillcolor{currentfill}%
\pgfsetlinewidth{0.602250pt}%
\definecolor{currentstroke}{rgb}{0.000000,0.000000,0.000000}%
\pgfsetstrokecolor{currentstroke}%
\pgfsetdash{}{0pt}%
\pgfsys@defobject{currentmarker}{\pgfqpoint{-0.027778in}{0.000000in}}{\pgfqpoint{0.000000in}{0.000000in}}{%
\pgfpathmoveto{\pgfqpoint{0.000000in}{0.000000in}}%
\pgfpathlineto{\pgfqpoint{-0.027778in}{0.000000in}}%
\pgfusepath{stroke,fill}%
}%
\begin{pgfscope}%
\pgfsys@transformshift{8.282041in}{7.175167in}%
\pgfsys@useobject{currentmarker}{}%
\end{pgfscope}%
\end{pgfscope}%
\begin{pgfscope}%
\pgfsetbuttcap%
\pgfsetroundjoin%
\definecolor{currentfill}{rgb}{0.000000,0.000000,0.000000}%
\pgfsetfillcolor{currentfill}%
\pgfsetlinewidth{0.602250pt}%
\definecolor{currentstroke}{rgb}{0.000000,0.000000,0.000000}%
\pgfsetstrokecolor{currentstroke}%
\pgfsetdash{}{0pt}%
\pgfsys@defobject{currentmarker}{\pgfqpoint{-0.027778in}{0.000000in}}{\pgfqpoint{0.000000in}{0.000000in}}{%
\pgfpathmoveto{\pgfqpoint{0.000000in}{0.000000in}}%
\pgfpathlineto{\pgfqpoint{-0.027778in}{0.000000in}}%
\pgfusepath{stroke,fill}%
}%
\begin{pgfscope}%
\pgfsys@transformshift{8.282041in}{7.224270in}%
\pgfsys@useobject{currentmarker}{}%
\end{pgfscope}%
\end{pgfscope}%
\begin{pgfscope}%
\pgfsetbuttcap%
\pgfsetroundjoin%
\definecolor{currentfill}{rgb}{0.000000,0.000000,0.000000}%
\pgfsetfillcolor{currentfill}%
\pgfsetlinewidth{0.602250pt}%
\definecolor{currentstroke}{rgb}{0.000000,0.000000,0.000000}%
\pgfsetstrokecolor{currentstroke}%
\pgfsetdash{}{0pt}%
\pgfsys@defobject{currentmarker}{\pgfqpoint{-0.027778in}{0.000000in}}{\pgfqpoint{0.000000in}{0.000000in}}{%
\pgfpathmoveto{\pgfqpoint{0.000000in}{0.000000in}}%
\pgfpathlineto{\pgfqpoint{-0.027778in}{0.000000in}}%
\pgfusepath{stroke,fill}%
}%
\begin{pgfscope}%
\pgfsys@transformshift{8.282041in}{7.262356in}%
\pgfsys@useobject{currentmarker}{}%
\end{pgfscope}%
\end{pgfscope}%
\begin{pgfscope}%
\pgfsetbuttcap%
\pgfsetroundjoin%
\definecolor{currentfill}{rgb}{0.000000,0.000000,0.000000}%
\pgfsetfillcolor{currentfill}%
\pgfsetlinewidth{0.602250pt}%
\definecolor{currentstroke}{rgb}{0.000000,0.000000,0.000000}%
\pgfsetstrokecolor{currentstroke}%
\pgfsetdash{}{0pt}%
\pgfsys@defobject{currentmarker}{\pgfqpoint{-0.027778in}{0.000000in}}{\pgfqpoint{0.000000in}{0.000000in}}{%
\pgfpathmoveto{\pgfqpoint{0.000000in}{0.000000in}}%
\pgfpathlineto{\pgfqpoint{-0.027778in}{0.000000in}}%
\pgfusepath{stroke,fill}%
}%
\begin{pgfscope}%
\pgfsys@transformshift{8.282041in}{7.293475in}%
\pgfsys@useobject{currentmarker}{}%
\end{pgfscope}%
\end{pgfscope}%
\begin{pgfscope}%
\pgfsetbuttcap%
\pgfsetroundjoin%
\definecolor{currentfill}{rgb}{0.000000,0.000000,0.000000}%
\pgfsetfillcolor{currentfill}%
\pgfsetlinewidth{0.602250pt}%
\definecolor{currentstroke}{rgb}{0.000000,0.000000,0.000000}%
\pgfsetstrokecolor{currentstroke}%
\pgfsetdash{}{0pt}%
\pgfsys@defobject{currentmarker}{\pgfqpoint{-0.027778in}{0.000000in}}{\pgfqpoint{0.000000in}{0.000000in}}{%
\pgfpathmoveto{\pgfqpoint{0.000000in}{0.000000in}}%
\pgfpathlineto{\pgfqpoint{-0.027778in}{0.000000in}}%
\pgfusepath{stroke,fill}%
}%
\begin{pgfscope}%
\pgfsys@transformshift{8.282041in}{7.319786in}%
\pgfsys@useobject{currentmarker}{}%
\end{pgfscope}%
\end{pgfscope}%
\begin{pgfscope}%
\pgfsetbuttcap%
\pgfsetroundjoin%
\definecolor{currentfill}{rgb}{0.000000,0.000000,0.000000}%
\pgfsetfillcolor{currentfill}%
\pgfsetlinewidth{0.602250pt}%
\definecolor{currentstroke}{rgb}{0.000000,0.000000,0.000000}%
\pgfsetstrokecolor{currentstroke}%
\pgfsetdash{}{0pt}%
\pgfsys@defobject{currentmarker}{\pgfqpoint{-0.027778in}{0.000000in}}{\pgfqpoint{0.000000in}{0.000000in}}{%
\pgfpathmoveto{\pgfqpoint{0.000000in}{0.000000in}}%
\pgfpathlineto{\pgfqpoint{-0.027778in}{0.000000in}}%
\pgfusepath{stroke,fill}%
}%
\begin{pgfscope}%
\pgfsys@transformshift{8.282041in}{7.342578in}%
\pgfsys@useobject{currentmarker}{}%
\end{pgfscope}%
\end{pgfscope}%
\begin{pgfscope}%
\pgfsetbuttcap%
\pgfsetroundjoin%
\definecolor{currentfill}{rgb}{0.000000,0.000000,0.000000}%
\pgfsetfillcolor{currentfill}%
\pgfsetlinewidth{0.602250pt}%
\definecolor{currentstroke}{rgb}{0.000000,0.000000,0.000000}%
\pgfsetstrokecolor{currentstroke}%
\pgfsetdash{}{0pt}%
\pgfsys@defobject{currentmarker}{\pgfqpoint{-0.027778in}{0.000000in}}{\pgfqpoint{0.000000in}{0.000000in}}{%
\pgfpathmoveto{\pgfqpoint{0.000000in}{0.000000in}}%
\pgfpathlineto{\pgfqpoint{-0.027778in}{0.000000in}}%
\pgfusepath{stroke,fill}%
}%
\begin{pgfscope}%
\pgfsys@transformshift{8.282041in}{7.362681in}%
\pgfsys@useobject{currentmarker}{}%
\end{pgfscope}%
\end{pgfscope}%
\begin{pgfscope}%
\pgfsetbuttcap%
\pgfsetroundjoin%
\definecolor{currentfill}{rgb}{0.000000,0.000000,0.000000}%
\pgfsetfillcolor{currentfill}%
\pgfsetlinewidth{0.602250pt}%
\definecolor{currentstroke}{rgb}{0.000000,0.000000,0.000000}%
\pgfsetstrokecolor{currentstroke}%
\pgfsetdash{}{0pt}%
\pgfsys@defobject{currentmarker}{\pgfqpoint{-0.027778in}{0.000000in}}{\pgfqpoint{0.000000in}{0.000000in}}{%
\pgfpathmoveto{\pgfqpoint{0.000000in}{0.000000in}}%
\pgfpathlineto{\pgfqpoint{-0.027778in}{0.000000in}}%
\pgfusepath{stroke,fill}%
}%
\begin{pgfscope}%
\pgfsys@transformshift{8.282041in}{7.498973in}%
\pgfsys@useobject{currentmarker}{}%
\end{pgfscope}%
\end{pgfscope}%
\begin{pgfscope}%
\pgfsetbuttcap%
\pgfsetroundjoin%
\definecolor{currentfill}{rgb}{0.000000,0.000000,0.000000}%
\pgfsetfillcolor{currentfill}%
\pgfsetlinewidth{0.602250pt}%
\definecolor{currentstroke}{rgb}{0.000000,0.000000,0.000000}%
\pgfsetstrokecolor{currentstroke}%
\pgfsetdash{}{0pt}%
\pgfsys@defobject{currentmarker}{\pgfqpoint{-0.027778in}{0.000000in}}{\pgfqpoint{0.000000in}{0.000000in}}{%
\pgfpathmoveto{\pgfqpoint{0.000000in}{0.000000in}}%
\pgfpathlineto{\pgfqpoint{-0.027778in}{0.000000in}}%
\pgfusepath{stroke,fill}%
}%
\begin{pgfscope}%
\pgfsys@transformshift{8.282041in}{7.568179in}%
\pgfsys@useobject{currentmarker}{}%
\end{pgfscope}%
\end{pgfscope}%
\begin{pgfscope}%
\pgfsetbuttcap%
\pgfsetroundjoin%
\definecolor{currentfill}{rgb}{0.000000,0.000000,0.000000}%
\pgfsetfillcolor{currentfill}%
\pgfsetlinewidth{0.602250pt}%
\definecolor{currentstroke}{rgb}{0.000000,0.000000,0.000000}%
\pgfsetstrokecolor{currentstroke}%
\pgfsetdash{}{0pt}%
\pgfsys@defobject{currentmarker}{\pgfqpoint{-0.027778in}{0.000000in}}{\pgfqpoint{0.000000in}{0.000000in}}{%
\pgfpathmoveto{\pgfqpoint{0.000000in}{0.000000in}}%
\pgfpathlineto{\pgfqpoint{-0.027778in}{0.000000in}}%
\pgfusepath{stroke,fill}%
}%
\begin{pgfscope}%
\pgfsys@transformshift{8.282041in}{7.617281in}%
\pgfsys@useobject{currentmarker}{}%
\end{pgfscope}%
\end{pgfscope}%
\begin{pgfscope}%
\pgfsetbuttcap%
\pgfsetroundjoin%
\definecolor{currentfill}{rgb}{0.000000,0.000000,0.000000}%
\pgfsetfillcolor{currentfill}%
\pgfsetlinewidth{0.602250pt}%
\definecolor{currentstroke}{rgb}{0.000000,0.000000,0.000000}%
\pgfsetstrokecolor{currentstroke}%
\pgfsetdash{}{0pt}%
\pgfsys@defobject{currentmarker}{\pgfqpoint{-0.027778in}{0.000000in}}{\pgfqpoint{0.000000in}{0.000000in}}{%
\pgfpathmoveto{\pgfqpoint{0.000000in}{0.000000in}}%
\pgfpathlineto{\pgfqpoint{-0.027778in}{0.000000in}}%
\pgfusepath{stroke,fill}%
}%
\begin{pgfscope}%
\pgfsys@transformshift{8.282041in}{7.655368in}%
\pgfsys@useobject{currentmarker}{}%
\end{pgfscope}%
\end{pgfscope}%
\begin{pgfscope}%
\pgfsetbuttcap%
\pgfsetroundjoin%
\definecolor{currentfill}{rgb}{0.000000,0.000000,0.000000}%
\pgfsetfillcolor{currentfill}%
\pgfsetlinewidth{0.602250pt}%
\definecolor{currentstroke}{rgb}{0.000000,0.000000,0.000000}%
\pgfsetstrokecolor{currentstroke}%
\pgfsetdash{}{0pt}%
\pgfsys@defobject{currentmarker}{\pgfqpoint{-0.027778in}{0.000000in}}{\pgfqpoint{0.000000in}{0.000000in}}{%
\pgfpathmoveto{\pgfqpoint{0.000000in}{0.000000in}}%
\pgfpathlineto{\pgfqpoint{-0.027778in}{0.000000in}}%
\pgfusepath{stroke,fill}%
}%
\begin{pgfscope}%
\pgfsys@transformshift{8.282041in}{7.686487in}%
\pgfsys@useobject{currentmarker}{}%
\end{pgfscope}%
\end{pgfscope}%
\begin{pgfscope}%
\pgfsetbuttcap%
\pgfsetroundjoin%
\definecolor{currentfill}{rgb}{0.000000,0.000000,0.000000}%
\pgfsetfillcolor{currentfill}%
\pgfsetlinewidth{0.602250pt}%
\definecolor{currentstroke}{rgb}{0.000000,0.000000,0.000000}%
\pgfsetstrokecolor{currentstroke}%
\pgfsetdash{}{0pt}%
\pgfsys@defobject{currentmarker}{\pgfqpoint{-0.027778in}{0.000000in}}{\pgfqpoint{0.000000in}{0.000000in}}{%
\pgfpathmoveto{\pgfqpoint{0.000000in}{0.000000in}}%
\pgfpathlineto{\pgfqpoint{-0.027778in}{0.000000in}}%
\pgfusepath{stroke,fill}%
}%
\begin{pgfscope}%
\pgfsys@transformshift{8.282041in}{7.712798in}%
\pgfsys@useobject{currentmarker}{}%
\end{pgfscope}%
\end{pgfscope}%
\begin{pgfscope}%
\pgfsetbuttcap%
\pgfsetroundjoin%
\definecolor{currentfill}{rgb}{0.000000,0.000000,0.000000}%
\pgfsetfillcolor{currentfill}%
\pgfsetlinewidth{0.602250pt}%
\definecolor{currentstroke}{rgb}{0.000000,0.000000,0.000000}%
\pgfsetstrokecolor{currentstroke}%
\pgfsetdash{}{0pt}%
\pgfsys@defobject{currentmarker}{\pgfqpoint{-0.027778in}{0.000000in}}{\pgfqpoint{0.000000in}{0.000000in}}{%
\pgfpathmoveto{\pgfqpoint{0.000000in}{0.000000in}}%
\pgfpathlineto{\pgfqpoint{-0.027778in}{0.000000in}}%
\pgfusepath{stroke,fill}%
}%
\begin{pgfscope}%
\pgfsys@transformshift{8.282041in}{7.735589in}%
\pgfsys@useobject{currentmarker}{}%
\end{pgfscope}%
\end{pgfscope}%
\begin{pgfscope}%
\pgfsetbuttcap%
\pgfsetroundjoin%
\definecolor{currentfill}{rgb}{0.000000,0.000000,0.000000}%
\pgfsetfillcolor{currentfill}%
\pgfsetlinewidth{0.602250pt}%
\definecolor{currentstroke}{rgb}{0.000000,0.000000,0.000000}%
\pgfsetstrokecolor{currentstroke}%
\pgfsetdash{}{0pt}%
\pgfsys@defobject{currentmarker}{\pgfqpoint{-0.027778in}{0.000000in}}{\pgfqpoint{0.000000in}{0.000000in}}{%
\pgfpathmoveto{\pgfqpoint{0.000000in}{0.000000in}}%
\pgfpathlineto{\pgfqpoint{-0.027778in}{0.000000in}}%
\pgfusepath{stroke,fill}%
}%
\begin{pgfscope}%
\pgfsys@transformshift{8.282041in}{7.755693in}%
\pgfsys@useobject{currentmarker}{}%
\end{pgfscope}%
\end{pgfscope}%
\begin{pgfscope}%
\pgfsetbuttcap%
\pgfsetroundjoin%
\definecolor{currentfill}{rgb}{0.000000,0.000000,0.000000}%
\pgfsetfillcolor{currentfill}%
\pgfsetlinewidth{0.602250pt}%
\definecolor{currentstroke}{rgb}{0.000000,0.000000,0.000000}%
\pgfsetstrokecolor{currentstroke}%
\pgfsetdash{}{0pt}%
\pgfsys@defobject{currentmarker}{\pgfqpoint{-0.027778in}{0.000000in}}{\pgfqpoint{0.000000in}{0.000000in}}{%
\pgfpathmoveto{\pgfqpoint{0.000000in}{0.000000in}}%
\pgfpathlineto{\pgfqpoint{-0.027778in}{0.000000in}}%
\pgfusepath{stroke,fill}%
}%
\begin{pgfscope}%
\pgfsys@transformshift{8.282041in}{7.891984in}%
\pgfsys@useobject{currentmarker}{}%
\end{pgfscope}%
\end{pgfscope}%
\begin{pgfscope}%
\pgfsetbuttcap%
\pgfsetroundjoin%
\definecolor{currentfill}{rgb}{0.000000,0.000000,0.000000}%
\pgfsetfillcolor{currentfill}%
\pgfsetlinewidth{0.602250pt}%
\definecolor{currentstroke}{rgb}{0.000000,0.000000,0.000000}%
\pgfsetstrokecolor{currentstroke}%
\pgfsetdash{}{0pt}%
\pgfsys@defobject{currentmarker}{\pgfqpoint{-0.027778in}{0.000000in}}{\pgfqpoint{0.000000in}{0.000000in}}{%
\pgfpathmoveto{\pgfqpoint{0.000000in}{0.000000in}}%
\pgfpathlineto{\pgfqpoint{-0.027778in}{0.000000in}}%
\pgfusepath{stroke,fill}%
}%
\begin{pgfscope}%
\pgfsys@transformshift{8.282041in}{7.961190in}%
\pgfsys@useobject{currentmarker}{}%
\end{pgfscope}%
\end{pgfscope}%
\begin{pgfscope}%
\pgfsetbuttcap%
\pgfsetroundjoin%
\definecolor{currentfill}{rgb}{0.000000,0.000000,0.000000}%
\pgfsetfillcolor{currentfill}%
\pgfsetlinewidth{0.602250pt}%
\definecolor{currentstroke}{rgb}{0.000000,0.000000,0.000000}%
\pgfsetstrokecolor{currentstroke}%
\pgfsetdash{}{0pt}%
\pgfsys@defobject{currentmarker}{\pgfqpoint{-0.027778in}{0.000000in}}{\pgfqpoint{0.000000in}{0.000000in}}{%
\pgfpathmoveto{\pgfqpoint{0.000000in}{0.000000in}}%
\pgfpathlineto{\pgfqpoint{-0.027778in}{0.000000in}}%
\pgfusepath{stroke,fill}%
}%
\begin{pgfscope}%
\pgfsys@transformshift{8.282041in}{8.010292in}%
\pgfsys@useobject{currentmarker}{}%
\end{pgfscope}%
\end{pgfscope}%
\begin{pgfscope}%
\pgfsetbuttcap%
\pgfsetroundjoin%
\definecolor{currentfill}{rgb}{0.000000,0.000000,0.000000}%
\pgfsetfillcolor{currentfill}%
\pgfsetlinewidth{0.602250pt}%
\definecolor{currentstroke}{rgb}{0.000000,0.000000,0.000000}%
\pgfsetstrokecolor{currentstroke}%
\pgfsetdash{}{0pt}%
\pgfsys@defobject{currentmarker}{\pgfqpoint{-0.027778in}{0.000000in}}{\pgfqpoint{0.000000in}{0.000000in}}{%
\pgfpathmoveto{\pgfqpoint{0.000000in}{0.000000in}}%
\pgfpathlineto{\pgfqpoint{-0.027778in}{0.000000in}}%
\pgfusepath{stroke,fill}%
}%
\begin{pgfscope}%
\pgfsys@transformshift{8.282041in}{8.048379in}%
\pgfsys@useobject{currentmarker}{}%
\end{pgfscope}%
\end{pgfscope}%
\begin{pgfscope}%
\pgfsetbuttcap%
\pgfsetroundjoin%
\definecolor{currentfill}{rgb}{0.000000,0.000000,0.000000}%
\pgfsetfillcolor{currentfill}%
\pgfsetlinewidth{0.602250pt}%
\definecolor{currentstroke}{rgb}{0.000000,0.000000,0.000000}%
\pgfsetstrokecolor{currentstroke}%
\pgfsetdash{}{0pt}%
\pgfsys@defobject{currentmarker}{\pgfqpoint{-0.027778in}{0.000000in}}{\pgfqpoint{0.000000in}{0.000000in}}{%
\pgfpathmoveto{\pgfqpoint{0.000000in}{0.000000in}}%
\pgfpathlineto{\pgfqpoint{-0.027778in}{0.000000in}}%
\pgfusepath{stroke,fill}%
}%
\begin{pgfscope}%
\pgfsys@transformshift{8.282041in}{8.079498in}%
\pgfsys@useobject{currentmarker}{}%
\end{pgfscope}%
\end{pgfscope}%
\begin{pgfscope}%
\pgfsetbuttcap%
\pgfsetroundjoin%
\definecolor{currentfill}{rgb}{0.000000,0.000000,0.000000}%
\pgfsetfillcolor{currentfill}%
\pgfsetlinewidth{0.602250pt}%
\definecolor{currentstroke}{rgb}{0.000000,0.000000,0.000000}%
\pgfsetstrokecolor{currentstroke}%
\pgfsetdash{}{0pt}%
\pgfsys@defobject{currentmarker}{\pgfqpoint{-0.027778in}{0.000000in}}{\pgfqpoint{0.000000in}{0.000000in}}{%
\pgfpathmoveto{\pgfqpoint{0.000000in}{0.000000in}}%
\pgfpathlineto{\pgfqpoint{-0.027778in}{0.000000in}}%
\pgfusepath{stroke,fill}%
}%
\begin{pgfscope}%
\pgfsys@transformshift{8.282041in}{8.105809in}%
\pgfsys@useobject{currentmarker}{}%
\end{pgfscope}%
\end{pgfscope}%
\begin{pgfscope}%
\pgfsetbuttcap%
\pgfsetroundjoin%
\definecolor{currentfill}{rgb}{0.000000,0.000000,0.000000}%
\pgfsetfillcolor{currentfill}%
\pgfsetlinewidth{0.602250pt}%
\definecolor{currentstroke}{rgb}{0.000000,0.000000,0.000000}%
\pgfsetstrokecolor{currentstroke}%
\pgfsetdash{}{0pt}%
\pgfsys@defobject{currentmarker}{\pgfqpoint{-0.027778in}{0.000000in}}{\pgfqpoint{0.000000in}{0.000000in}}{%
\pgfpathmoveto{\pgfqpoint{0.000000in}{0.000000in}}%
\pgfpathlineto{\pgfqpoint{-0.027778in}{0.000000in}}%
\pgfusepath{stroke,fill}%
}%
\begin{pgfscope}%
\pgfsys@transformshift{8.282041in}{8.128601in}%
\pgfsys@useobject{currentmarker}{}%
\end{pgfscope}%
\end{pgfscope}%
\begin{pgfscope}%
\pgfsetbuttcap%
\pgfsetroundjoin%
\definecolor{currentfill}{rgb}{0.000000,0.000000,0.000000}%
\pgfsetfillcolor{currentfill}%
\pgfsetlinewidth{0.602250pt}%
\definecolor{currentstroke}{rgb}{0.000000,0.000000,0.000000}%
\pgfsetstrokecolor{currentstroke}%
\pgfsetdash{}{0pt}%
\pgfsys@defobject{currentmarker}{\pgfqpoint{-0.027778in}{0.000000in}}{\pgfqpoint{0.000000in}{0.000000in}}{%
\pgfpathmoveto{\pgfqpoint{0.000000in}{0.000000in}}%
\pgfpathlineto{\pgfqpoint{-0.027778in}{0.000000in}}%
\pgfusepath{stroke,fill}%
}%
\begin{pgfscope}%
\pgfsys@transformshift{8.282041in}{8.148704in}%
\pgfsys@useobject{currentmarker}{}%
\end{pgfscope}%
\end{pgfscope}%
\begin{pgfscope}%
\pgfsetbuttcap%
\pgfsetroundjoin%
\definecolor{currentfill}{rgb}{0.000000,0.000000,0.000000}%
\pgfsetfillcolor{currentfill}%
\pgfsetlinewidth{0.602250pt}%
\definecolor{currentstroke}{rgb}{0.000000,0.000000,0.000000}%
\pgfsetstrokecolor{currentstroke}%
\pgfsetdash{}{0pt}%
\pgfsys@defobject{currentmarker}{\pgfqpoint{-0.027778in}{0.000000in}}{\pgfqpoint{0.000000in}{0.000000in}}{%
\pgfpathmoveto{\pgfqpoint{0.000000in}{0.000000in}}%
\pgfpathlineto{\pgfqpoint{-0.027778in}{0.000000in}}%
\pgfusepath{stroke,fill}%
}%
\begin{pgfscope}%
\pgfsys@transformshift{8.282041in}{8.284995in}%
\pgfsys@useobject{currentmarker}{}%
\end{pgfscope}%
\end{pgfscope}%
\begin{pgfscope}%
\pgfpathrectangle{\pgfqpoint{8.282041in}{6.967719in}}{\pgfqpoint{1.897959in}{1.372727in}} %
\pgfusepath{clip}%
\pgfsetbuttcap%
\pgfsetroundjoin%
\pgfsetlinewidth{1.505625pt}%
\definecolor{currentstroke}{rgb}{1.000000,0.000000,0.000000}%
\pgfsetstrokecolor{currentstroke}%
\pgfsetdash{{5.550000pt}{2.400000pt}}{0.000000pt}%
\pgfpathmoveto{\pgfqpoint{8.368312in}{8.179801in}}%
\pgfpathlineto{\pgfqpoint{8.540853in}{8.177814in}}%
\pgfpathlineto{\pgfqpoint{8.713395in}{8.175988in}}%
\pgfpathlineto{\pgfqpoint{8.885937in}{8.174318in}}%
\pgfpathlineto{\pgfqpoint{9.058479in}{8.172802in}}%
\pgfpathlineto{\pgfqpoint{9.231020in}{8.171435in}}%
\pgfpathlineto{\pgfqpoint{9.403562in}{8.170217in}}%
\pgfpathlineto{\pgfqpoint{9.576104in}{8.169145in}}%
\pgfpathlineto{\pgfqpoint{9.748646in}{8.168217in}}%
\pgfpathlineto{\pgfqpoint{9.921187in}{8.167433in}}%
\pgfpathlineto{\pgfqpoint{10.093729in}{8.166791in}}%
\pgfusepath{stroke}%
\end{pgfscope}%
\begin{pgfscope}%
\pgfpathrectangle{\pgfqpoint{8.282041in}{6.967719in}}{\pgfqpoint{1.897959in}{1.372727in}} %
\pgfusepath{clip}%
\pgfsetbuttcap%
\pgfsetmiterjoin%
\definecolor{currentfill}{rgb}{1.000000,0.000000,0.000000}%
\pgfsetfillcolor{currentfill}%
\pgfsetlinewidth{1.003750pt}%
\definecolor{currentstroke}{rgb}{1.000000,0.000000,0.000000}%
\pgfsetstrokecolor{currentstroke}%
\pgfsetdash{}{0pt}%
\pgfsys@defobject{currentmarker}{\pgfqpoint{-0.041667in}{-0.041667in}}{\pgfqpoint{0.041667in}{0.041667in}}{%
\pgfpathmoveto{\pgfqpoint{-0.041667in}{-0.041667in}}%
\pgfpathlineto{\pgfqpoint{0.041667in}{-0.041667in}}%
\pgfpathlineto{\pgfqpoint{0.041667in}{0.041667in}}%
\pgfpathlineto{\pgfqpoint{-0.041667in}{0.041667in}}%
\pgfpathclose%
\pgfusepath{stroke,fill}%
}%
\begin{pgfscope}%
\pgfsys@transformshift{8.368312in}{8.179801in}%
\pgfsys@useobject{currentmarker}{}%
\end{pgfscope}%
\begin{pgfscope}%
\pgfsys@transformshift{8.713395in}{8.175988in}%
\pgfsys@useobject{currentmarker}{}%
\end{pgfscope}%
\begin{pgfscope}%
\pgfsys@transformshift{9.058479in}{8.172802in}%
\pgfsys@useobject{currentmarker}{}%
\end{pgfscope}%
\begin{pgfscope}%
\pgfsys@transformshift{9.403562in}{8.170217in}%
\pgfsys@useobject{currentmarker}{}%
\end{pgfscope}%
\begin{pgfscope}%
\pgfsys@transformshift{9.748646in}{8.168217in}%
\pgfsys@useobject{currentmarker}{}%
\end{pgfscope}%
\begin{pgfscope}%
\pgfsys@transformshift{10.093729in}{8.166791in}%
\pgfsys@useobject{currentmarker}{}%
\end{pgfscope}%
\end{pgfscope}%
\begin{pgfscope}%
\pgfpathrectangle{\pgfqpoint{8.282041in}{6.967719in}}{\pgfqpoint{1.897959in}{1.372727in}} %
\pgfusepath{clip}%
\pgfsetrectcap%
\pgfsetroundjoin%
\pgfsetlinewidth{1.505625pt}%
\definecolor{currentstroke}{rgb}{0.000000,0.000000,1.000000}%
\pgfsetstrokecolor{currentstroke}%
\pgfsetdash{}{0pt}%
\pgfpathmoveto{\pgfqpoint{8.368312in}{7.320746in}}%
\pgfpathlineto{\pgfqpoint{8.540853in}{7.300476in}}%
\pgfpathlineto{\pgfqpoint{8.713395in}{7.279407in}}%
\pgfpathlineto{\pgfqpoint{8.885937in}{7.257276in}}%
\pgfpathlineto{\pgfqpoint{9.058479in}{7.233777in}}%
\pgfpathlineto{\pgfqpoint{9.231020in}{7.208534in}}%
\pgfpathlineto{\pgfqpoint{9.403562in}{7.181061in}}%
\pgfpathlineto{\pgfqpoint{9.576104in}{7.150703in}}%
\pgfpathlineto{\pgfqpoint{9.748646in}{7.116523in}}%
\pgfpathlineto{\pgfqpoint{9.921187in}{7.077102in}}%
\pgfpathlineto{\pgfqpoint{10.093729in}{7.030115in}}%
\pgfusepath{stroke}%
\end{pgfscope}%
\begin{pgfscope}%
\pgfpathrectangle{\pgfqpoint{8.282041in}{6.967719in}}{\pgfqpoint{1.897959in}{1.372727in}} %
\pgfusepath{clip}%
\pgfsetbuttcap%
\pgfsetroundjoin%
\definecolor{currentfill}{rgb}{0.000000,0.000000,1.000000}%
\pgfsetfillcolor{currentfill}%
\pgfsetlinewidth{1.003750pt}%
\definecolor{currentstroke}{rgb}{0.000000,0.000000,1.000000}%
\pgfsetstrokecolor{currentstroke}%
\pgfsetdash{}{0pt}%
\pgfsys@defobject{currentmarker}{\pgfqpoint{-0.041667in}{-0.041667in}}{\pgfqpoint{0.041667in}{0.041667in}}{%
\pgfpathmoveto{\pgfqpoint{0.000000in}{-0.041667in}}%
\pgfpathcurveto{\pgfqpoint{0.011050in}{-0.041667in}}{\pgfqpoint{0.021649in}{-0.037276in}}{\pgfqpoint{0.029463in}{-0.029463in}}%
\pgfpathcurveto{\pgfqpoint{0.037276in}{-0.021649in}}{\pgfqpoint{0.041667in}{-0.011050in}}{\pgfqpoint{0.041667in}{0.000000in}}%
\pgfpathcurveto{\pgfqpoint{0.041667in}{0.011050in}}{\pgfqpoint{0.037276in}{0.021649in}}{\pgfqpoint{0.029463in}{0.029463in}}%
\pgfpathcurveto{\pgfqpoint{0.021649in}{0.037276in}}{\pgfqpoint{0.011050in}{0.041667in}}{\pgfqpoint{0.000000in}{0.041667in}}%
\pgfpathcurveto{\pgfqpoint{-0.011050in}{0.041667in}}{\pgfqpoint{-0.021649in}{0.037276in}}{\pgfqpoint{-0.029463in}{0.029463in}}%
\pgfpathcurveto{\pgfqpoint{-0.037276in}{0.021649in}}{\pgfqpoint{-0.041667in}{0.011050in}}{\pgfqpoint{-0.041667in}{0.000000in}}%
\pgfpathcurveto{\pgfqpoint{-0.041667in}{-0.011050in}}{\pgfqpoint{-0.037276in}{-0.021649in}}{\pgfqpoint{-0.029463in}{-0.029463in}}%
\pgfpathcurveto{\pgfqpoint{-0.021649in}{-0.037276in}}{\pgfqpoint{-0.011050in}{-0.041667in}}{\pgfqpoint{0.000000in}{-0.041667in}}%
\pgfpathclose%
\pgfusepath{stroke,fill}%
}%
\begin{pgfscope}%
\pgfsys@transformshift{8.368312in}{7.320746in}%
\pgfsys@useobject{currentmarker}{}%
\end{pgfscope}%
\begin{pgfscope}%
\pgfsys@transformshift{8.713395in}{7.279407in}%
\pgfsys@useobject{currentmarker}{}%
\end{pgfscope}%
\begin{pgfscope}%
\pgfsys@transformshift{9.058479in}{7.233777in}%
\pgfsys@useobject{currentmarker}{}%
\end{pgfscope}%
\begin{pgfscope}%
\pgfsys@transformshift{9.403562in}{7.181061in}%
\pgfsys@useobject{currentmarker}{}%
\end{pgfscope}%
\begin{pgfscope}%
\pgfsys@transformshift{9.748646in}{7.116523in}%
\pgfsys@useobject{currentmarker}{}%
\end{pgfscope}%
\begin{pgfscope}%
\pgfsys@transformshift{10.093729in}{7.030115in}%
\pgfsys@useobject{currentmarker}{}%
\end{pgfscope}%
\end{pgfscope}%
\begin{pgfscope}%
\pgfpathrectangle{\pgfqpoint{8.282041in}{6.967719in}}{\pgfqpoint{1.897959in}{1.372727in}} %
\pgfusepath{clip}%
\pgfsetbuttcap%
\pgfsetroundjoin%
\pgfsetlinewidth{1.505625pt}%
\definecolor{currentstroke}{rgb}{0.000000,0.750000,0.750000}%
\pgfsetstrokecolor{currentstroke}%
\pgfsetdash{{9.600000pt}{2.400000pt}{1.500000pt}{2.400000pt}}{0.000000pt}%
\pgfpathmoveto{\pgfqpoint{8.368312in}{8.136715in}}%
\pgfpathlineto{\pgfqpoint{8.540853in}{8.120963in}}%
\pgfpathlineto{\pgfqpoint{8.713395in}{8.107166in}}%
\pgfpathlineto{\pgfqpoint{8.885937in}{8.095075in}}%
\pgfpathlineto{\pgfqpoint{9.058479in}{8.084493in}}%
\pgfpathlineto{\pgfqpoint{9.231020in}{8.075264in}}%
\pgfpathlineto{\pgfqpoint{9.403562in}{8.067265in}}%
\pgfpathlineto{\pgfqpoint{9.576104in}{8.060395in}}%
\pgfpathlineto{\pgfqpoint{9.748646in}{8.054574in}}%
\pgfpathlineto{\pgfqpoint{9.921187in}{8.049737in}}%
\pgfpathlineto{\pgfqpoint{10.093729in}{8.045832in}}%
\pgfusepath{stroke}%
\end{pgfscope}%
\begin{pgfscope}%
\pgfpathrectangle{\pgfqpoint{8.282041in}{6.967719in}}{\pgfqpoint{1.897959in}{1.372727in}} %
\pgfusepath{clip}%
\pgfsetbuttcap%
\pgfsetmiterjoin%
\definecolor{currentfill}{rgb}{0.000000,0.750000,0.750000}%
\pgfsetfillcolor{currentfill}%
\pgfsetlinewidth{1.003750pt}%
\definecolor{currentstroke}{rgb}{0.000000,0.750000,0.750000}%
\pgfsetstrokecolor{currentstroke}%
\pgfsetdash{}{0pt}%
\pgfsys@defobject{currentmarker}{\pgfqpoint{-0.041667in}{-0.041667in}}{\pgfqpoint{0.041667in}{0.041667in}}{%
\pgfpathmoveto{\pgfqpoint{-0.000000in}{-0.041667in}}%
\pgfpathlineto{\pgfqpoint{0.041667in}{0.041667in}}%
\pgfpathlineto{\pgfqpoint{-0.041667in}{0.041667in}}%
\pgfpathclose%
\pgfusepath{stroke,fill}%
}%
\begin{pgfscope}%
\pgfsys@transformshift{8.368312in}{8.136715in}%
\pgfsys@useobject{currentmarker}{}%
\end{pgfscope}%
\begin{pgfscope}%
\pgfsys@transformshift{8.713395in}{8.107166in}%
\pgfsys@useobject{currentmarker}{}%
\end{pgfscope}%
\begin{pgfscope}%
\pgfsys@transformshift{9.058479in}{8.084493in}%
\pgfsys@useobject{currentmarker}{}%
\end{pgfscope}%
\begin{pgfscope}%
\pgfsys@transformshift{9.403562in}{8.067265in}%
\pgfsys@useobject{currentmarker}{}%
\end{pgfscope}%
\begin{pgfscope}%
\pgfsys@transformshift{9.748646in}{8.054574in}%
\pgfsys@useobject{currentmarker}{}%
\end{pgfscope}%
\begin{pgfscope}%
\pgfsys@transformshift{10.093729in}{8.045832in}%
\pgfsys@useobject{currentmarker}{}%
\end{pgfscope}%
\end{pgfscope}%
\begin{pgfscope}%
\pgfpathrectangle{\pgfqpoint{8.282041in}{6.967719in}}{\pgfqpoint{1.897959in}{1.372727in}} %
\pgfusepath{clip}%
\pgfsetbuttcap%
\pgfsetroundjoin%
\pgfsetlinewidth{1.505625pt}%
\definecolor{currentstroke}{rgb}{0.000000,0.000000,0.000000}%
\pgfsetstrokecolor{currentstroke}%
\pgfsetdash{{1.500000pt}{2.475000pt}}{0.000000pt}%
\pgfpathmoveto{\pgfqpoint{8.368312in}{8.278049in}}%
\pgfpathlineto{\pgfqpoint{8.540853in}{8.270877in}}%
\pgfpathlineto{\pgfqpoint{8.713395in}{8.264063in}}%
\pgfpathlineto{\pgfqpoint{8.885937in}{8.258212in}}%
\pgfpathlineto{\pgfqpoint{9.058479in}{8.253177in}}%
\pgfpathlineto{\pgfqpoint{9.231020in}{8.248847in}}%
\pgfpathlineto{\pgfqpoint{9.403562in}{8.245134in}}%
\pgfpathlineto{\pgfqpoint{9.576104in}{8.241974in}}%
\pgfpathlineto{\pgfqpoint{9.748646in}{8.239312in}}%
\pgfpathlineto{\pgfqpoint{9.921187in}{8.237109in}}%
\pgfpathlineto{\pgfqpoint{10.093729in}{8.235333in}}%
\pgfusepath{stroke}%
\end{pgfscope}%
\begin{pgfscope}%
\pgfpathrectangle{\pgfqpoint{8.282041in}{6.967719in}}{\pgfqpoint{1.897959in}{1.372727in}} %
\pgfusepath{clip}%
\pgfsetbuttcap%
\pgfsetroundjoin%
\definecolor{currentfill}{rgb}{0.000000,0.000000,0.000000}%
\pgfsetfillcolor{currentfill}%
\pgfsetlinewidth{1.003750pt}%
\definecolor{currentstroke}{rgb}{0.000000,0.000000,0.000000}%
\pgfsetstrokecolor{currentstroke}%
\pgfsetdash{}{0pt}%
\pgfsys@defobject{currentmarker}{\pgfqpoint{-0.041667in}{-0.041667in}}{\pgfqpoint{0.041667in}{0.041667in}}{%
\pgfpathmoveto{\pgfqpoint{-0.041667in}{0.000000in}}%
\pgfpathlineto{\pgfqpoint{0.041667in}{0.000000in}}%
\pgfpathmoveto{\pgfqpoint{0.000000in}{-0.041667in}}%
\pgfpathlineto{\pgfqpoint{0.000000in}{0.041667in}}%
\pgfusepath{stroke,fill}%
}%
\begin{pgfscope}%
\pgfsys@transformshift{8.368312in}{8.278049in}%
\pgfsys@useobject{currentmarker}{}%
\end{pgfscope}%
\begin{pgfscope}%
\pgfsys@transformshift{8.713395in}{8.264063in}%
\pgfsys@useobject{currentmarker}{}%
\end{pgfscope}%
\begin{pgfscope}%
\pgfsys@transformshift{9.058479in}{8.253177in}%
\pgfsys@useobject{currentmarker}{}%
\end{pgfscope}%
\begin{pgfscope}%
\pgfsys@transformshift{9.403562in}{8.245134in}%
\pgfsys@useobject{currentmarker}{}%
\end{pgfscope}%
\begin{pgfscope}%
\pgfsys@transformshift{9.748646in}{8.239312in}%
\pgfsys@useobject{currentmarker}{}%
\end{pgfscope}%
\begin{pgfscope}%
\pgfsys@transformshift{10.093729in}{8.235333in}%
\pgfsys@useobject{currentmarker}{}%
\end{pgfscope}%
\end{pgfscope}%
\begin{pgfscope}%
\pgfsetrectcap%
\pgfsetmiterjoin%
\pgfsetlinewidth{0.803000pt}%
\definecolor{currentstroke}{rgb}{0.000000,0.000000,0.000000}%
\pgfsetstrokecolor{currentstroke}%
\pgfsetdash{}{0pt}%
\pgfpathmoveto{\pgfqpoint{8.282041in}{6.967719in}}%
\pgfpathlineto{\pgfqpoint{8.282041in}{8.340446in}}%
\pgfusepath{stroke}%
\end{pgfscope}%
\begin{pgfscope}%
\pgfsetrectcap%
\pgfsetmiterjoin%
\pgfsetlinewidth{0.803000pt}%
\definecolor{currentstroke}{rgb}{0.000000,0.000000,0.000000}%
\pgfsetstrokecolor{currentstroke}%
\pgfsetdash{}{0pt}%
\pgfpathmoveto{\pgfqpoint{10.180000in}{6.967719in}}%
\pgfpathlineto{\pgfqpoint{10.180000in}{8.340446in}}%
\pgfusepath{stroke}%
\end{pgfscope}%
\begin{pgfscope}%
\pgfsetrectcap%
\pgfsetmiterjoin%
\pgfsetlinewidth{0.803000pt}%
\definecolor{currentstroke}{rgb}{0.000000,0.000000,0.000000}%
\pgfsetstrokecolor{currentstroke}%
\pgfsetdash{}{0pt}%
\pgfpathmoveto{\pgfqpoint{8.282041in}{6.967719in}}%
\pgfpathlineto{\pgfqpoint{10.180000in}{6.967719in}}%
\pgfusepath{stroke}%
\end{pgfscope}%
\begin{pgfscope}%
\pgfsetrectcap%
\pgfsetmiterjoin%
\pgfsetlinewidth{0.803000pt}%
\definecolor{currentstroke}{rgb}{0.000000,0.000000,0.000000}%
\pgfsetstrokecolor{currentstroke}%
\pgfsetdash{}{0pt}%
\pgfpathmoveto{\pgfqpoint{8.282041in}{8.340446in}}%
\pgfpathlineto{\pgfqpoint{10.180000in}{8.340446in}}%
\pgfusepath{stroke}%
\end{pgfscope}%
\begin{pgfscope}%
\pgfsetbuttcap%
\pgfsetmiterjoin%
\definecolor{currentfill}{rgb}{1.000000,1.000000,1.000000}%
\pgfsetfillcolor{currentfill}%
\pgfsetlinewidth{0.000000pt}%
\definecolor{currentstroke}{rgb}{0.000000,0.000000,0.000000}%
\pgfsetstrokecolor{currentstroke}%
\pgfsetstrokeopacity{0.000000}%
\pgfsetdash{}{0pt}%
\pgfpathmoveto{\pgfqpoint{0.880000in}{4.908628in}}%
\pgfpathlineto{\pgfqpoint{2.777959in}{4.908628in}}%
\pgfpathlineto{\pgfqpoint{2.777959in}{6.281355in}}%
\pgfpathlineto{\pgfqpoint{0.880000in}{6.281355in}}%
\pgfpathclose%
\pgfusepath{fill}%
\end{pgfscope}%
\begin{pgfscope}%
\pgfsetbuttcap%
\pgfsetroundjoin%
\definecolor{currentfill}{rgb}{0.000000,0.000000,0.000000}%
\pgfsetfillcolor{currentfill}%
\pgfsetlinewidth{0.803000pt}%
\definecolor{currentstroke}{rgb}{0.000000,0.000000,0.000000}%
\pgfsetstrokecolor{currentstroke}%
\pgfsetdash{}{0pt}%
\pgfsys@defobject{currentmarker}{\pgfqpoint{0.000000in}{-0.048611in}}{\pgfqpoint{0.000000in}{0.000000in}}{%
\pgfpathmoveto{\pgfqpoint{0.000000in}{0.000000in}}%
\pgfpathlineto{\pgfqpoint{0.000000in}{-0.048611in}}%
\pgfusepath{stroke,fill}%
}%
\begin{pgfscope}%
\pgfsys@transformshift{1.138813in}{4.908628in}%
\pgfsys@useobject{currentmarker}{}%
\end{pgfscope}%
\end{pgfscope}%
\begin{pgfscope}%
\pgftext[x=1.138813in,y=4.811405in,,top]{\rmfamily\fontsize{10.000000}{12.000000}\selectfont \(\displaystyle 0.10\)}%
\end{pgfscope}%
\begin{pgfscope}%
\pgfsetbuttcap%
\pgfsetroundjoin%
\definecolor{currentfill}{rgb}{0.000000,0.000000,0.000000}%
\pgfsetfillcolor{currentfill}%
\pgfsetlinewidth{0.803000pt}%
\definecolor{currentstroke}{rgb}{0.000000,0.000000,0.000000}%
\pgfsetstrokecolor{currentstroke}%
\pgfsetdash{}{0pt}%
\pgfsys@defobject{currentmarker}{\pgfqpoint{0.000000in}{-0.048611in}}{\pgfqpoint{0.000000in}{0.000000in}}{%
\pgfpathmoveto{\pgfqpoint{0.000000in}{0.000000in}}%
\pgfpathlineto{\pgfqpoint{0.000000in}{-0.048611in}}%
\pgfusepath{stroke,fill}%
}%
\begin{pgfscope}%
\pgfsys@transformshift{1.656438in}{4.908628in}%
\pgfsys@useobject{currentmarker}{}%
\end{pgfscope}%
\end{pgfscope}%
\begin{pgfscope}%
\pgftext[x=1.656438in,y=4.811405in,,top]{\rmfamily\fontsize{10.000000}{12.000000}\selectfont \(\displaystyle 0.15\)}%
\end{pgfscope}%
\begin{pgfscope}%
\pgfsetbuttcap%
\pgfsetroundjoin%
\definecolor{currentfill}{rgb}{0.000000,0.000000,0.000000}%
\pgfsetfillcolor{currentfill}%
\pgfsetlinewidth{0.803000pt}%
\definecolor{currentstroke}{rgb}{0.000000,0.000000,0.000000}%
\pgfsetstrokecolor{currentstroke}%
\pgfsetdash{}{0pt}%
\pgfsys@defobject{currentmarker}{\pgfqpoint{0.000000in}{-0.048611in}}{\pgfqpoint{0.000000in}{0.000000in}}{%
\pgfpathmoveto{\pgfqpoint{0.000000in}{0.000000in}}%
\pgfpathlineto{\pgfqpoint{0.000000in}{-0.048611in}}%
\pgfusepath{stroke,fill}%
}%
\begin{pgfscope}%
\pgfsys@transformshift{2.174063in}{4.908628in}%
\pgfsys@useobject{currentmarker}{}%
\end{pgfscope}%
\end{pgfscope}%
\begin{pgfscope}%
\pgftext[x=2.174063in,y=4.811405in,,top]{\rmfamily\fontsize{10.000000}{12.000000}\selectfont \(\displaystyle 0.20\)}%
\end{pgfscope}%
\begin{pgfscope}%
\pgfsetbuttcap%
\pgfsetroundjoin%
\definecolor{currentfill}{rgb}{0.000000,0.000000,0.000000}%
\pgfsetfillcolor{currentfill}%
\pgfsetlinewidth{0.803000pt}%
\definecolor{currentstroke}{rgb}{0.000000,0.000000,0.000000}%
\pgfsetstrokecolor{currentstroke}%
\pgfsetdash{}{0pt}%
\pgfsys@defobject{currentmarker}{\pgfqpoint{0.000000in}{-0.048611in}}{\pgfqpoint{0.000000in}{0.000000in}}{%
\pgfpathmoveto{\pgfqpoint{0.000000in}{0.000000in}}%
\pgfpathlineto{\pgfqpoint{0.000000in}{-0.048611in}}%
\pgfusepath{stroke,fill}%
}%
\begin{pgfscope}%
\pgfsys@transformshift{2.691688in}{4.908628in}%
\pgfsys@useobject{currentmarker}{}%
\end{pgfscope}%
\end{pgfscope}%
\begin{pgfscope}%
\pgftext[x=2.691688in,y=4.811405in,,top]{\rmfamily\fontsize{10.000000}{12.000000}\selectfont \(\displaystyle 0.25\)}%
\end{pgfscope}%
\begin{pgfscope}%
\pgfsetbuttcap%
\pgfsetroundjoin%
\definecolor{currentfill}{rgb}{0.000000,0.000000,0.000000}%
\pgfsetfillcolor{currentfill}%
\pgfsetlinewidth{0.803000pt}%
\definecolor{currentstroke}{rgb}{0.000000,0.000000,0.000000}%
\pgfsetstrokecolor{currentstroke}%
\pgfsetdash{}{0pt}%
\pgfsys@defobject{currentmarker}{\pgfqpoint{-0.048611in}{0.000000in}}{\pgfqpoint{0.000000in}{0.000000in}}{%
\pgfpathmoveto{\pgfqpoint{0.000000in}{0.000000in}}%
\pgfpathlineto{\pgfqpoint{-0.048611in}{0.000000in}}%
\pgfusepath{stroke,fill}%
}%
\begin{pgfscope}%
\pgfsys@transformshift{0.880000in}{5.127687in}%
\pgfsys@useobject{currentmarker}{}%
\end{pgfscope}%
\end{pgfscope}%
\begin{pgfscope}%
\pgftext[x=0.494775in,y=5.074926in,left,base]{\rmfamily\fontsize{10.000000}{12.000000}\selectfont \(\displaystyle 10^{-7}\)}%
\end{pgfscope}%
\begin{pgfscope}%
\pgfsetbuttcap%
\pgfsetroundjoin%
\definecolor{currentfill}{rgb}{0.000000,0.000000,0.000000}%
\pgfsetfillcolor{currentfill}%
\pgfsetlinewidth{0.803000pt}%
\definecolor{currentstroke}{rgb}{0.000000,0.000000,0.000000}%
\pgfsetstrokecolor{currentstroke}%
\pgfsetdash{}{0pt}%
\pgfsys@defobject{currentmarker}{\pgfqpoint{-0.048611in}{0.000000in}}{\pgfqpoint{0.000000in}{0.000000in}}{%
\pgfpathmoveto{\pgfqpoint{0.000000in}{0.000000in}}%
\pgfpathlineto{\pgfqpoint{-0.048611in}{0.000000in}}%
\pgfusepath{stroke,fill}%
}%
\begin{pgfscope}%
\pgfsys@transformshift{0.880000in}{5.587098in}%
\pgfsys@useobject{currentmarker}{}%
\end{pgfscope}%
\end{pgfscope}%
\begin{pgfscope}%
\pgftext[x=0.494775in,y=5.534337in,left,base]{\rmfamily\fontsize{10.000000}{12.000000}\selectfont \(\displaystyle 10^{-6}\)}%
\end{pgfscope}%
\begin{pgfscope}%
\pgfsetbuttcap%
\pgfsetroundjoin%
\definecolor{currentfill}{rgb}{0.000000,0.000000,0.000000}%
\pgfsetfillcolor{currentfill}%
\pgfsetlinewidth{0.803000pt}%
\definecolor{currentstroke}{rgb}{0.000000,0.000000,0.000000}%
\pgfsetstrokecolor{currentstroke}%
\pgfsetdash{}{0pt}%
\pgfsys@defobject{currentmarker}{\pgfqpoint{-0.048611in}{0.000000in}}{\pgfqpoint{0.000000in}{0.000000in}}{%
\pgfpathmoveto{\pgfqpoint{0.000000in}{0.000000in}}%
\pgfpathlineto{\pgfqpoint{-0.048611in}{0.000000in}}%
\pgfusepath{stroke,fill}%
}%
\begin{pgfscope}%
\pgfsys@transformshift{0.880000in}{6.046509in}%
\pgfsys@useobject{currentmarker}{}%
\end{pgfscope}%
\end{pgfscope}%
\begin{pgfscope}%
\pgftext[x=0.494775in,y=5.993748in,left,base]{\rmfamily\fontsize{10.000000}{12.000000}\selectfont \(\displaystyle 10^{-5}\)}%
\end{pgfscope}%
\begin{pgfscope}%
\pgfsetbuttcap%
\pgfsetroundjoin%
\definecolor{currentfill}{rgb}{0.000000,0.000000,0.000000}%
\pgfsetfillcolor{currentfill}%
\pgfsetlinewidth{0.602250pt}%
\definecolor{currentstroke}{rgb}{0.000000,0.000000,0.000000}%
\pgfsetstrokecolor{currentstroke}%
\pgfsetdash{}{0pt}%
\pgfsys@defobject{currentmarker}{\pgfqpoint{-0.027778in}{0.000000in}}{\pgfqpoint{0.000000in}{0.000000in}}{%
\pgfpathmoveto{\pgfqpoint{0.000000in}{0.000000in}}%
\pgfpathlineto{\pgfqpoint{-0.027778in}{0.000000in}}%
\pgfusepath{stroke,fill}%
}%
\begin{pgfscope}%
\pgfsys@transformshift{0.880000in}{4.944869in}%
\pgfsys@useobject{currentmarker}{}%
\end{pgfscope}%
\end{pgfscope}%
\begin{pgfscope}%
\pgfsetbuttcap%
\pgfsetroundjoin%
\definecolor{currentfill}{rgb}{0.000000,0.000000,0.000000}%
\pgfsetfillcolor{currentfill}%
\pgfsetlinewidth{0.602250pt}%
\definecolor{currentstroke}{rgb}{0.000000,0.000000,0.000000}%
\pgfsetstrokecolor{currentstroke}%
\pgfsetdash{}{0pt}%
\pgfsys@defobject{currentmarker}{\pgfqpoint{-0.027778in}{0.000000in}}{\pgfqpoint{0.000000in}{0.000000in}}{%
\pgfpathmoveto{\pgfqpoint{0.000000in}{0.000000in}}%
\pgfpathlineto{\pgfqpoint{-0.027778in}{0.000000in}}%
\pgfusepath{stroke,fill}%
}%
\begin{pgfscope}%
\pgfsys@transformshift{0.880000in}{4.989391in}%
\pgfsys@useobject{currentmarker}{}%
\end{pgfscope}%
\end{pgfscope}%
\begin{pgfscope}%
\pgfsetbuttcap%
\pgfsetroundjoin%
\definecolor{currentfill}{rgb}{0.000000,0.000000,0.000000}%
\pgfsetfillcolor{currentfill}%
\pgfsetlinewidth{0.602250pt}%
\definecolor{currentstroke}{rgb}{0.000000,0.000000,0.000000}%
\pgfsetstrokecolor{currentstroke}%
\pgfsetdash{}{0pt}%
\pgfsys@defobject{currentmarker}{\pgfqpoint{-0.027778in}{0.000000in}}{\pgfqpoint{0.000000in}{0.000000in}}{%
\pgfpathmoveto{\pgfqpoint{0.000000in}{0.000000in}}%
\pgfpathlineto{\pgfqpoint{-0.027778in}{0.000000in}}%
\pgfusepath{stroke,fill}%
}%
\begin{pgfscope}%
\pgfsys@transformshift{0.880000in}{5.025767in}%
\pgfsys@useobject{currentmarker}{}%
\end{pgfscope}%
\end{pgfscope}%
\begin{pgfscope}%
\pgfsetbuttcap%
\pgfsetroundjoin%
\definecolor{currentfill}{rgb}{0.000000,0.000000,0.000000}%
\pgfsetfillcolor{currentfill}%
\pgfsetlinewidth{0.602250pt}%
\definecolor{currentstroke}{rgb}{0.000000,0.000000,0.000000}%
\pgfsetstrokecolor{currentstroke}%
\pgfsetdash{}{0pt}%
\pgfsys@defobject{currentmarker}{\pgfqpoint{-0.027778in}{0.000000in}}{\pgfqpoint{0.000000in}{0.000000in}}{%
\pgfpathmoveto{\pgfqpoint{0.000000in}{0.000000in}}%
\pgfpathlineto{\pgfqpoint{-0.027778in}{0.000000in}}%
\pgfusepath{stroke,fill}%
}%
\begin{pgfscope}%
\pgfsys@transformshift{0.880000in}{5.056523in}%
\pgfsys@useobject{currentmarker}{}%
\end{pgfscope}%
\end{pgfscope}%
\begin{pgfscope}%
\pgfsetbuttcap%
\pgfsetroundjoin%
\definecolor{currentfill}{rgb}{0.000000,0.000000,0.000000}%
\pgfsetfillcolor{currentfill}%
\pgfsetlinewidth{0.602250pt}%
\definecolor{currentstroke}{rgb}{0.000000,0.000000,0.000000}%
\pgfsetstrokecolor{currentstroke}%
\pgfsetdash{}{0pt}%
\pgfsys@defobject{currentmarker}{\pgfqpoint{-0.027778in}{0.000000in}}{\pgfqpoint{0.000000in}{0.000000in}}{%
\pgfpathmoveto{\pgfqpoint{0.000000in}{0.000000in}}%
\pgfpathlineto{\pgfqpoint{-0.027778in}{0.000000in}}%
\pgfusepath{stroke,fill}%
}%
\begin{pgfscope}%
\pgfsys@transformshift{0.880000in}{5.083166in}%
\pgfsys@useobject{currentmarker}{}%
\end{pgfscope}%
\end{pgfscope}%
\begin{pgfscope}%
\pgfsetbuttcap%
\pgfsetroundjoin%
\definecolor{currentfill}{rgb}{0.000000,0.000000,0.000000}%
\pgfsetfillcolor{currentfill}%
\pgfsetlinewidth{0.602250pt}%
\definecolor{currentstroke}{rgb}{0.000000,0.000000,0.000000}%
\pgfsetstrokecolor{currentstroke}%
\pgfsetdash{}{0pt}%
\pgfsys@defobject{currentmarker}{\pgfqpoint{-0.027778in}{0.000000in}}{\pgfqpoint{0.000000in}{0.000000in}}{%
\pgfpathmoveto{\pgfqpoint{0.000000in}{0.000000in}}%
\pgfpathlineto{\pgfqpoint{-0.027778in}{0.000000in}}%
\pgfusepath{stroke,fill}%
}%
\begin{pgfscope}%
\pgfsys@transformshift{0.880000in}{5.106666in}%
\pgfsys@useobject{currentmarker}{}%
\end{pgfscope}%
\end{pgfscope}%
\begin{pgfscope}%
\pgfsetbuttcap%
\pgfsetroundjoin%
\definecolor{currentfill}{rgb}{0.000000,0.000000,0.000000}%
\pgfsetfillcolor{currentfill}%
\pgfsetlinewidth{0.602250pt}%
\definecolor{currentstroke}{rgb}{0.000000,0.000000,0.000000}%
\pgfsetstrokecolor{currentstroke}%
\pgfsetdash{}{0pt}%
\pgfsys@defobject{currentmarker}{\pgfqpoint{-0.027778in}{0.000000in}}{\pgfqpoint{0.000000in}{0.000000in}}{%
\pgfpathmoveto{\pgfqpoint{0.000000in}{0.000000in}}%
\pgfpathlineto{\pgfqpoint{-0.027778in}{0.000000in}}%
\pgfusepath{stroke,fill}%
}%
\begin{pgfscope}%
\pgfsys@transformshift{0.880000in}{5.265984in}%
\pgfsys@useobject{currentmarker}{}%
\end{pgfscope}%
\end{pgfscope}%
\begin{pgfscope}%
\pgfsetbuttcap%
\pgfsetroundjoin%
\definecolor{currentfill}{rgb}{0.000000,0.000000,0.000000}%
\pgfsetfillcolor{currentfill}%
\pgfsetlinewidth{0.602250pt}%
\definecolor{currentstroke}{rgb}{0.000000,0.000000,0.000000}%
\pgfsetstrokecolor{currentstroke}%
\pgfsetdash{}{0pt}%
\pgfsys@defobject{currentmarker}{\pgfqpoint{-0.027778in}{0.000000in}}{\pgfqpoint{0.000000in}{0.000000in}}{%
\pgfpathmoveto{\pgfqpoint{0.000000in}{0.000000in}}%
\pgfpathlineto{\pgfqpoint{-0.027778in}{0.000000in}}%
\pgfusepath{stroke,fill}%
}%
\begin{pgfscope}%
\pgfsys@transformshift{0.880000in}{5.346882in}%
\pgfsys@useobject{currentmarker}{}%
\end{pgfscope}%
\end{pgfscope}%
\begin{pgfscope}%
\pgfsetbuttcap%
\pgfsetroundjoin%
\definecolor{currentfill}{rgb}{0.000000,0.000000,0.000000}%
\pgfsetfillcolor{currentfill}%
\pgfsetlinewidth{0.602250pt}%
\definecolor{currentstroke}{rgb}{0.000000,0.000000,0.000000}%
\pgfsetstrokecolor{currentstroke}%
\pgfsetdash{}{0pt}%
\pgfsys@defobject{currentmarker}{\pgfqpoint{-0.027778in}{0.000000in}}{\pgfqpoint{0.000000in}{0.000000in}}{%
\pgfpathmoveto{\pgfqpoint{0.000000in}{0.000000in}}%
\pgfpathlineto{\pgfqpoint{-0.027778in}{0.000000in}}%
\pgfusepath{stroke,fill}%
}%
\begin{pgfscope}%
\pgfsys@transformshift{0.880000in}{5.404280in}%
\pgfsys@useobject{currentmarker}{}%
\end{pgfscope}%
\end{pgfscope}%
\begin{pgfscope}%
\pgfsetbuttcap%
\pgfsetroundjoin%
\definecolor{currentfill}{rgb}{0.000000,0.000000,0.000000}%
\pgfsetfillcolor{currentfill}%
\pgfsetlinewidth{0.602250pt}%
\definecolor{currentstroke}{rgb}{0.000000,0.000000,0.000000}%
\pgfsetstrokecolor{currentstroke}%
\pgfsetdash{}{0pt}%
\pgfsys@defobject{currentmarker}{\pgfqpoint{-0.027778in}{0.000000in}}{\pgfqpoint{0.000000in}{0.000000in}}{%
\pgfpathmoveto{\pgfqpoint{0.000000in}{0.000000in}}%
\pgfpathlineto{\pgfqpoint{-0.027778in}{0.000000in}}%
\pgfusepath{stroke,fill}%
}%
\begin{pgfscope}%
\pgfsys@transformshift{0.880000in}{5.448802in}%
\pgfsys@useobject{currentmarker}{}%
\end{pgfscope}%
\end{pgfscope}%
\begin{pgfscope}%
\pgfsetbuttcap%
\pgfsetroundjoin%
\definecolor{currentfill}{rgb}{0.000000,0.000000,0.000000}%
\pgfsetfillcolor{currentfill}%
\pgfsetlinewidth{0.602250pt}%
\definecolor{currentstroke}{rgb}{0.000000,0.000000,0.000000}%
\pgfsetstrokecolor{currentstroke}%
\pgfsetdash{}{0pt}%
\pgfsys@defobject{currentmarker}{\pgfqpoint{-0.027778in}{0.000000in}}{\pgfqpoint{0.000000in}{0.000000in}}{%
\pgfpathmoveto{\pgfqpoint{0.000000in}{0.000000in}}%
\pgfpathlineto{\pgfqpoint{-0.027778in}{0.000000in}}%
\pgfusepath{stroke,fill}%
}%
\begin{pgfscope}%
\pgfsys@transformshift{0.880000in}{5.485178in}%
\pgfsys@useobject{currentmarker}{}%
\end{pgfscope}%
\end{pgfscope}%
\begin{pgfscope}%
\pgfsetbuttcap%
\pgfsetroundjoin%
\definecolor{currentfill}{rgb}{0.000000,0.000000,0.000000}%
\pgfsetfillcolor{currentfill}%
\pgfsetlinewidth{0.602250pt}%
\definecolor{currentstroke}{rgb}{0.000000,0.000000,0.000000}%
\pgfsetstrokecolor{currentstroke}%
\pgfsetdash{}{0pt}%
\pgfsys@defobject{currentmarker}{\pgfqpoint{-0.027778in}{0.000000in}}{\pgfqpoint{0.000000in}{0.000000in}}{%
\pgfpathmoveto{\pgfqpoint{0.000000in}{0.000000in}}%
\pgfpathlineto{\pgfqpoint{-0.027778in}{0.000000in}}%
\pgfusepath{stroke,fill}%
}%
\begin{pgfscope}%
\pgfsys@transformshift{0.880000in}{5.515934in}%
\pgfsys@useobject{currentmarker}{}%
\end{pgfscope}%
\end{pgfscope}%
\begin{pgfscope}%
\pgfsetbuttcap%
\pgfsetroundjoin%
\definecolor{currentfill}{rgb}{0.000000,0.000000,0.000000}%
\pgfsetfillcolor{currentfill}%
\pgfsetlinewidth{0.602250pt}%
\definecolor{currentstroke}{rgb}{0.000000,0.000000,0.000000}%
\pgfsetstrokecolor{currentstroke}%
\pgfsetdash{}{0pt}%
\pgfsys@defobject{currentmarker}{\pgfqpoint{-0.027778in}{0.000000in}}{\pgfqpoint{0.000000in}{0.000000in}}{%
\pgfpathmoveto{\pgfqpoint{0.000000in}{0.000000in}}%
\pgfpathlineto{\pgfqpoint{-0.027778in}{0.000000in}}%
\pgfusepath{stroke,fill}%
}%
\begin{pgfscope}%
\pgfsys@transformshift{0.880000in}{5.542577in}%
\pgfsys@useobject{currentmarker}{}%
\end{pgfscope}%
\end{pgfscope}%
\begin{pgfscope}%
\pgfsetbuttcap%
\pgfsetroundjoin%
\definecolor{currentfill}{rgb}{0.000000,0.000000,0.000000}%
\pgfsetfillcolor{currentfill}%
\pgfsetlinewidth{0.602250pt}%
\definecolor{currentstroke}{rgb}{0.000000,0.000000,0.000000}%
\pgfsetstrokecolor{currentstroke}%
\pgfsetdash{}{0pt}%
\pgfsys@defobject{currentmarker}{\pgfqpoint{-0.027778in}{0.000000in}}{\pgfqpoint{0.000000in}{0.000000in}}{%
\pgfpathmoveto{\pgfqpoint{0.000000in}{0.000000in}}%
\pgfpathlineto{\pgfqpoint{-0.027778in}{0.000000in}}%
\pgfusepath{stroke,fill}%
}%
\begin{pgfscope}%
\pgfsys@transformshift{0.880000in}{5.566077in}%
\pgfsys@useobject{currentmarker}{}%
\end{pgfscope}%
\end{pgfscope}%
\begin{pgfscope}%
\pgfsetbuttcap%
\pgfsetroundjoin%
\definecolor{currentfill}{rgb}{0.000000,0.000000,0.000000}%
\pgfsetfillcolor{currentfill}%
\pgfsetlinewidth{0.602250pt}%
\definecolor{currentstroke}{rgb}{0.000000,0.000000,0.000000}%
\pgfsetstrokecolor{currentstroke}%
\pgfsetdash{}{0pt}%
\pgfsys@defobject{currentmarker}{\pgfqpoint{-0.027778in}{0.000000in}}{\pgfqpoint{0.000000in}{0.000000in}}{%
\pgfpathmoveto{\pgfqpoint{0.000000in}{0.000000in}}%
\pgfpathlineto{\pgfqpoint{-0.027778in}{0.000000in}}%
\pgfusepath{stroke,fill}%
}%
\begin{pgfscope}%
\pgfsys@transformshift{0.880000in}{5.725395in}%
\pgfsys@useobject{currentmarker}{}%
\end{pgfscope}%
\end{pgfscope}%
\begin{pgfscope}%
\pgfsetbuttcap%
\pgfsetroundjoin%
\definecolor{currentfill}{rgb}{0.000000,0.000000,0.000000}%
\pgfsetfillcolor{currentfill}%
\pgfsetlinewidth{0.602250pt}%
\definecolor{currentstroke}{rgb}{0.000000,0.000000,0.000000}%
\pgfsetstrokecolor{currentstroke}%
\pgfsetdash{}{0pt}%
\pgfsys@defobject{currentmarker}{\pgfqpoint{-0.027778in}{0.000000in}}{\pgfqpoint{0.000000in}{0.000000in}}{%
\pgfpathmoveto{\pgfqpoint{0.000000in}{0.000000in}}%
\pgfpathlineto{\pgfqpoint{-0.027778in}{0.000000in}}%
\pgfusepath{stroke,fill}%
}%
\begin{pgfscope}%
\pgfsys@transformshift{0.880000in}{5.806293in}%
\pgfsys@useobject{currentmarker}{}%
\end{pgfscope}%
\end{pgfscope}%
\begin{pgfscope}%
\pgfsetbuttcap%
\pgfsetroundjoin%
\definecolor{currentfill}{rgb}{0.000000,0.000000,0.000000}%
\pgfsetfillcolor{currentfill}%
\pgfsetlinewidth{0.602250pt}%
\definecolor{currentstroke}{rgb}{0.000000,0.000000,0.000000}%
\pgfsetstrokecolor{currentstroke}%
\pgfsetdash{}{0pt}%
\pgfsys@defobject{currentmarker}{\pgfqpoint{-0.027778in}{0.000000in}}{\pgfqpoint{0.000000in}{0.000000in}}{%
\pgfpathmoveto{\pgfqpoint{0.000000in}{0.000000in}}%
\pgfpathlineto{\pgfqpoint{-0.027778in}{0.000000in}}%
\pgfusepath{stroke,fill}%
}%
\begin{pgfscope}%
\pgfsys@transformshift{0.880000in}{5.863691in}%
\pgfsys@useobject{currentmarker}{}%
\end{pgfscope}%
\end{pgfscope}%
\begin{pgfscope}%
\pgfsetbuttcap%
\pgfsetroundjoin%
\definecolor{currentfill}{rgb}{0.000000,0.000000,0.000000}%
\pgfsetfillcolor{currentfill}%
\pgfsetlinewidth{0.602250pt}%
\definecolor{currentstroke}{rgb}{0.000000,0.000000,0.000000}%
\pgfsetstrokecolor{currentstroke}%
\pgfsetdash{}{0pt}%
\pgfsys@defobject{currentmarker}{\pgfqpoint{-0.027778in}{0.000000in}}{\pgfqpoint{0.000000in}{0.000000in}}{%
\pgfpathmoveto{\pgfqpoint{0.000000in}{0.000000in}}%
\pgfpathlineto{\pgfqpoint{-0.027778in}{0.000000in}}%
\pgfusepath{stroke,fill}%
}%
\begin{pgfscope}%
\pgfsys@transformshift{0.880000in}{5.908213in}%
\pgfsys@useobject{currentmarker}{}%
\end{pgfscope}%
\end{pgfscope}%
\begin{pgfscope}%
\pgfsetbuttcap%
\pgfsetroundjoin%
\definecolor{currentfill}{rgb}{0.000000,0.000000,0.000000}%
\pgfsetfillcolor{currentfill}%
\pgfsetlinewidth{0.602250pt}%
\definecolor{currentstroke}{rgb}{0.000000,0.000000,0.000000}%
\pgfsetstrokecolor{currentstroke}%
\pgfsetdash{}{0pt}%
\pgfsys@defobject{currentmarker}{\pgfqpoint{-0.027778in}{0.000000in}}{\pgfqpoint{0.000000in}{0.000000in}}{%
\pgfpathmoveto{\pgfqpoint{0.000000in}{0.000000in}}%
\pgfpathlineto{\pgfqpoint{-0.027778in}{0.000000in}}%
\pgfusepath{stroke,fill}%
}%
\begin{pgfscope}%
\pgfsys@transformshift{0.880000in}{5.944589in}%
\pgfsys@useobject{currentmarker}{}%
\end{pgfscope}%
\end{pgfscope}%
\begin{pgfscope}%
\pgfsetbuttcap%
\pgfsetroundjoin%
\definecolor{currentfill}{rgb}{0.000000,0.000000,0.000000}%
\pgfsetfillcolor{currentfill}%
\pgfsetlinewidth{0.602250pt}%
\definecolor{currentstroke}{rgb}{0.000000,0.000000,0.000000}%
\pgfsetstrokecolor{currentstroke}%
\pgfsetdash{}{0pt}%
\pgfsys@defobject{currentmarker}{\pgfqpoint{-0.027778in}{0.000000in}}{\pgfqpoint{0.000000in}{0.000000in}}{%
\pgfpathmoveto{\pgfqpoint{0.000000in}{0.000000in}}%
\pgfpathlineto{\pgfqpoint{-0.027778in}{0.000000in}}%
\pgfusepath{stroke,fill}%
}%
\begin{pgfscope}%
\pgfsys@transformshift{0.880000in}{5.975346in}%
\pgfsys@useobject{currentmarker}{}%
\end{pgfscope}%
\end{pgfscope}%
\begin{pgfscope}%
\pgfsetbuttcap%
\pgfsetroundjoin%
\definecolor{currentfill}{rgb}{0.000000,0.000000,0.000000}%
\pgfsetfillcolor{currentfill}%
\pgfsetlinewidth{0.602250pt}%
\definecolor{currentstroke}{rgb}{0.000000,0.000000,0.000000}%
\pgfsetstrokecolor{currentstroke}%
\pgfsetdash{}{0pt}%
\pgfsys@defobject{currentmarker}{\pgfqpoint{-0.027778in}{0.000000in}}{\pgfqpoint{0.000000in}{0.000000in}}{%
\pgfpathmoveto{\pgfqpoint{0.000000in}{0.000000in}}%
\pgfpathlineto{\pgfqpoint{-0.027778in}{0.000000in}}%
\pgfusepath{stroke,fill}%
}%
\begin{pgfscope}%
\pgfsys@transformshift{0.880000in}{6.001988in}%
\pgfsys@useobject{currentmarker}{}%
\end{pgfscope}%
\end{pgfscope}%
\begin{pgfscope}%
\pgfsetbuttcap%
\pgfsetroundjoin%
\definecolor{currentfill}{rgb}{0.000000,0.000000,0.000000}%
\pgfsetfillcolor{currentfill}%
\pgfsetlinewidth{0.602250pt}%
\definecolor{currentstroke}{rgb}{0.000000,0.000000,0.000000}%
\pgfsetstrokecolor{currentstroke}%
\pgfsetdash{}{0pt}%
\pgfsys@defobject{currentmarker}{\pgfqpoint{-0.027778in}{0.000000in}}{\pgfqpoint{0.000000in}{0.000000in}}{%
\pgfpathmoveto{\pgfqpoint{0.000000in}{0.000000in}}%
\pgfpathlineto{\pgfqpoint{-0.027778in}{0.000000in}}%
\pgfusepath{stroke,fill}%
}%
\begin{pgfscope}%
\pgfsys@transformshift{0.880000in}{6.025488in}%
\pgfsys@useobject{currentmarker}{}%
\end{pgfscope}%
\end{pgfscope}%
\begin{pgfscope}%
\pgfsetbuttcap%
\pgfsetroundjoin%
\definecolor{currentfill}{rgb}{0.000000,0.000000,0.000000}%
\pgfsetfillcolor{currentfill}%
\pgfsetlinewidth{0.602250pt}%
\definecolor{currentstroke}{rgb}{0.000000,0.000000,0.000000}%
\pgfsetstrokecolor{currentstroke}%
\pgfsetdash{}{0pt}%
\pgfsys@defobject{currentmarker}{\pgfqpoint{-0.027778in}{0.000000in}}{\pgfqpoint{0.000000in}{0.000000in}}{%
\pgfpathmoveto{\pgfqpoint{0.000000in}{0.000000in}}%
\pgfpathlineto{\pgfqpoint{-0.027778in}{0.000000in}}%
\pgfusepath{stroke,fill}%
}%
\begin{pgfscope}%
\pgfsys@transformshift{0.880000in}{6.184806in}%
\pgfsys@useobject{currentmarker}{}%
\end{pgfscope}%
\end{pgfscope}%
\begin{pgfscope}%
\pgfsetbuttcap%
\pgfsetroundjoin%
\definecolor{currentfill}{rgb}{0.000000,0.000000,0.000000}%
\pgfsetfillcolor{currentfill}%
\pgfsetlinewidth{0.602250pt}%
\definecolor{currentstroke}{rgb}{0.000000,0.000000,0.000000}%
\pgfsetstrokecolor{currentstroke}%
\pgfsetdash{}{0pt}%
\pgfsys@defobject{currentmarker}{\pgfqpoint{-0.027778in}{0.000000in}}{\pgfqpoint{0.000000in}{0.000000in}}{%
\pgfpathmoveto{\pgfqpoint{0.000000in}{0.000000in}}%
\pgfpathlineto{\pgfqpoint{-0.027778in}{0.000000in}}%
\pgfusepath{stroke,fill}%
}%
\begin{pgfscope}%
\pgfsys@transformshift{0.880000in}{6.265704in}%
\pgfsys@useobject{currentmarker}{}%
\end{pgfscope}%
\end{pgfscope}%
\begin{pgfscope}%
\pgfpathrectangle{\pgfqpoint{0.880000in}{4.908628in}}{\pgfqpoint{1.897959in}{1.372727in}} %
\pgfusepath{clip}%
\pgfsetbuttcap%
\pgfsetroundjoin%
\pgfsetlinewidth{1.505625pt}%
\definecolor{currentstroke}{rgb}{1.000000,0.000000,0.000000}%
\pgfsetstrokecolor{currentstroke}%
\pgfsetdash{{5.550000pt}{2.400000pt}}{0.000000pt}%
\pgfpathmoveto{\pgfqpoint{0.966271in}{6.144224in}}%
\pgfpathlineto{\pgfqpoint{1.052542in}{6.141798in}}%
\pgfpathlineto{\pgfqpoint{1.138813in}{6.139481in}}%
\pgfpathlineto{\pgfqpoint{1.225083in}{6.137270in}}%
\pgfpathlineto{\pgfqpoint{1.311354in}{6.135164in}}%
\pgfpathlineto{\pgfqpoint{1.397625in}{6.133162in}}%
\pgfpathlineto{\pgfqpoint{1.483896in}{6.131261in}}%
\pgfpathlineto{\pgfqpoint{1.570167in}{6.129461in}}%
\pgfpathlineto{\pgfqpoint{1.656438in}{6.127762in}}%
\pgfpathlineto{\pgfqpoint{1.742709in}{6.126163in}}%
\pgfpathlineto{\pgfqpoint{1.828980in}{6.124662in}}%
\pgfpathlineto{\pgfqpoint{1.915250in}{6.123260in}}%
\pgfpathlineto{\pgfqpoint{2.001521in}{6.121956in}}%
\pgfpathlineto{\pgfqpoint{2.087792in}{6.120749in}}%
\pgfpathlineto{\pgfqpoint{2.174063in}{6.119639in}}%
\pgfpathlineto{\pgfqpoint{2.260334in}{6.118626in}}%
\pgfpathlineto{\pgfqpoint{2.346605in}{6.117709in}}%
\pgfpathlineto{\pgfqpoint{2.432876in}{6.116887in}}%
\pgfpathlineto{\pgfqpoint{2.519147in}{6.116162in}}%
\pgfpathlineto{\pgfqpoint{2.605417in}{6.115531in}}%
\pgfpathlineto{\pgfqpoint{2.691688in}{6.114996in}}%
\pgfusepath{stroke}%
\end{pgfscope}%
\begin{pgfscope}%
\pgfpathrectangle{\pgfqpoint{0.880000in}{4.908628in}}{\pgfqpoint{1.897959in}{1.372727in}} %
\pgfusepath{clip}%
\pgfsetbuttcap%
\pgfsetmiterjoin%
\definecolor{currentfill}{rgb}{1.000000,0.000000,0.000000}%
\pgfsetfillcolor{currentfill}%
\pgfsetlinewidth{1.003750pt}%
\definecolor{currentstroke}{rgb}{1.000000,0.000000,0.000000}%
\pgfsetstrokecolor{currentstroke}%
\pgfsetdash{}{0pt}%
\pgfsys@defobject{currentmarker}{\pgfqpoint{-0.041667in}{-0.041667in}}{\pgfqpoint{0.041667in}{0.041667in}}{%
\pgfpathmoveto{\pgfqpoint{-0.041667in}{-0.041667in}}%
\pgfpathlineto{\pgfqpoint{0.041667in}{-0.041667in}}%
\pgfpathlineto{\pgfqpoint{0.041667in}{0.041667in}}%
\pgfpathlineto{\pgfqpoint{-0.041667in}{0.041667in}}%
\pgfpathclose%
\pgfusepath{stroke,fill}%
}%
\begin{pgfscope}%
\pgfsys@transformshift{0.966271in}{6.144224in}%
\pgfsys@useobject{currentmarker}{}%
\end{pgfscope}%
\begin{pgfscope}%
\pgfsys@transformshift{1.311354in}{6.135164in}%
\pgfsys@useobject{currentmarker}{}%
\end{pgfscope}%
\begin{pgfscope}%
\pgfsys@transformshift{1.656438in}{6.127762in}%
\pgfsys@useobject{currentmarker}{}%
\end{pgfscope}%
\begin{pgfscope}%
\pgfsys@transformshift{2.001521in}{6.121956in}%
\pgfsys@useobject{currentmarker}{}%
\end{pgfscope}%
\begin{pgfscope}%
\pgfsys@transformshift{2.346605in}{6.117709in}%
\pgfsys@useobject{currentmarker}{}%
\end{pgfscope}%
\begin{pgfscope}%
\pgfsys@transformshift{2.691688in}{6.114996in}%
\pgfsys@useobject{currentmarker}{}%
\end{pgfscope}%
\end{pgfscope}%
\begin{pgfscope}%
\pgfpathrectangle{\pgfqpoint{0.880000in}{4.908628in}}{\pgfqpoint{1.897959in}{1.372727in}} %
\pgfusepath{clip}%
\pgfsetrectcap%
\pgfsetroundjoin%
\pgfsetlinewidth{1.505625pt}%
\definecolor{currentstroke}{rgb}{0.000000,0.000000,1.000000}%
\pgfsetstrokecolor{currentstroke}%
\pgfsetdash{}{0pt}%
\pgfpathmoveto{\pgfqpoint{0.966271in}{5.385505in}}%
\pgfpathlineto{\pgfqpoint{1.052542in}{5.371885in}}%
\pgfpathlineto{\pgfqpoint{1.138813in}{5.358203in}}%
\pgfpathlineto{\pgfqpoint{1.225083in}{5.344366in}}%
\pgfpathlineto{\pgfqpoint{1.311354in}{5.330293in}}%
\pgfpathlineto{\pgfqpoint{1.397625in}{5.315905in}}%
\pgfpathlineto{\pgfqpoint{1.483896in}{5.301127in}}%
\pgfpathlineto{\pgfqpoint{1.570167in}{5.285878in}}%
\pgfpathlineto{\pgfqpoint{1.656438in}{5.270073in}}%
\pgfpathlineto{\pgfqpoint{1.742709in}{5.253616in}}%
\pgfpathlineto{\pgfqpoint{1.828980in}{5.236400in}}%
\pgfpathlineto{\pgfqpoint{1.915250in}{5.218298in}}%
\pgfpathlineto{\pgfqpoint{2.001521in}{5.199162in}}%
\pgfpathlineto{\pgfqpoint{2.087792in}{5.178811in}}%
\pgfpathlineto{\pgfqpoint{2.174063in}{5.157019in}}%
\pgfpathlineto{\pgfqpoint{2.260334in}{5.133503in}}%
\pgfpathlineto{\pgfqpoint{2.346605in}{5.107892in}}%
\pgfpathlineto{\pgfqpoint{2.432876in}{5.079695in}}%
\pgfpathlineto{\pgfqpoint{2.519147in}{5.048227in}}%
\pgfpathlineto{\pgfqpoint{2.605417in}{5.012504in}}%
\pgfpathlineto{\pgfqpoint{2.691688in}{4.971024in}}%
\pgfusepath{stroke}%
\end{pgfscope}%
\begin{pgfscope}%
\pgfpathrectangle{\pgfqpoint{0.880000in}{4.908628in}}{\pgfqpoint{1.897959in}{1.372727in}} %
\pgfusepath{clip}%
\pgfsetbuttcap%
\pgfsetroundjoin%
\definecolor{currentfill}{rgb}{0.000000,0.000000,1.000000}%
\pgfsetfillcolor{currentfill}%
\pgfsetlinewidth{1.003750pt}%
\definecolor{currentstroke}{rgb}{0.000000,0.000000,1.000000}%
\pgfsetstrokecolor{currentstroke}%
\pgfsetdash{}{0pt}%
\pgfsys@defobject{currentmarker}{\pgfqpoint{-0.041667in}{-0.041667in}}{\pgfqpoint{0.041667in}{0.041667in}}{%
\pgfpathmoveto{\pgfqpoint{0.000000in}{-0.041667in}}%
\pgfpathcurveto{\pgfqpoint{0.011050in}{-0.041667in}}{\pgfqpoint{0.021649in}{-0.037276in}}{\pgfqpoint{0.029463in}{-0.029463in}}%
\pgfpathcurveto{\pgfqpoint{0.037276in}{-0.021649in}}{\pgfqpoint{0.041667in}{-0.011050in}}{\pgfqpoint{0.041667in}{0.000000in}}%
\pgfpathcurveto{\pgfqpoint{0.041667in}{0.011050in}}{\pgfqpoint{0.037276in}{0.021649in}}{\pgfqpoint{0.029463in}{0.029463in}}%
\pgfpathcurveto{\pgfqpoint{0.021649in}{0.037276in}}{\pgfqpoint{0.011050in}{0.041667in}}{\pgfqpoint{0.000000in}{0.041667in}}%
\pgfpathcurveto{\pgfqpoint{-0.011050in}{0.041667in}}{\pgfqpoint{-0.021649in}{0.037276in}}{\pgfqpoint{-0.029463in}{0.029463in}}%
\pgfpathcurveto{\pgfqpoint{-0.037276in}{0.021649in}}{\pgfqpoint{-0.041667in}{0.011050in}}{\pgfqpoint{-0.041667in}{0.000000in}}%
\pgfpathcurveto{\pgfqpoint{-0.041667in}{-0.011050in}}{\pgfqpoint{-0.037276in}{-0.021649in}}{\pgfqpoint{-0.029463in}{-0.029463in}}%
\pgfpathcurveto{\pgfqpoint{-0.021649in}{-0.037276in}}{\pgfqpoint{-0.011050in}{-0.041667in}}{\pgfqpoint{0.000000in}{-0.041667in}}%
\pgfpathclose%
\pgfusepath{stroke,fill}%
}%
\begin{pgfscope}%
\pgfsys@transformshift{0.966271in}{5.385505in}%
\pgfsys@useobject{currentmarker}{}%
\end{pgfscope}%
\begin{pgfscope}%
\pgfsys@transformshift{1.311354in}{5.330293in}%
\pgfsys@useobject{currentmarker}{}%
\end{pgfscope}%
\begin{pgfscope}%
\pgfsys@transformshift{1.656438in}{5.270073in}%
\pgfsys@useobject{currentmarker}{}%
\end{pgfscope}%
\begin{pgfscope}%
\pgfsys@transformshift{2.001521in}{5.199162in}%
\pgfsys@useobject{currentmarker}{}%
\end{pgfscope}%
\begin{pgfscope}%
\pgfsys@transformshift{2.346605in}{5.107892in}%
\pgfsys@useobject{currentmarker}{}%
\end{pgfscope}%
\begin{pgfscope}%
\pgfsys@transformshift{2.691688in}{4.971024in}%
\pgfsys@useobject{currentmarker}{}%
\end{pgfscope}%
\end{pgfscope}%
\begin{pgfscope}%
\pgfpathrectangle{\pgfqpoint{0.880000in}{4.908628in}}{\pgfqpoint{1.897959in}{1.372727in}} %
\pgfusepath{clip}%
\pgfsetbuttcap%
\pgfsetroundjoin%
\pgfsetlinewidth{1.505625pt}%
\definecolor{currentstroke}{rgb}{0.000000,0.750000,0.750000}%
\pgfsetstrokecolor{currentstroke}%
\pgfsetdash{{9.600000pt}{2.400000pt}{1.500000pt}{2.400000pt}}{0.000000pt}%
\pgfpathmoveto{\pgfqpoint{0.966271in}{5.978793in}}%
\pgfpathlineto{\pgfqpoint{1.052542in}{5.952688in}}%
\pgfpathlineto{\pgfqpoint{1.138813in}{5.929403in}}%
\pgfpathlineto{\pgfqpoint{1.225083in}{5.908512in}}%
\pgfpathlineto{\pgfqpoint{1.311354in}{5.889682in}}%
\pgfpathlineto{\pgfqpoint{1.397625in}{5.872650in}}%
\pgfpathlineto{\pgfqpoint{1.483896in}{5.857205in}}%
\pgfpathlineto{\pgfqpoint{1.570167in}{5.843174in}}%
\pgfpathlineto{\pgfqpoint{1.656438in}{5.830413in}}%
\pgfpathlineto{\pgfqpoint{1.742709in}{5.818804in}}%
\pgfpathlineto{\pgfqpoint{1.828980in}{5.808246in}}%
\pgfpathlineto{\pgfqpoint{1.915250in}{5.798656in}}%
\pgfpathlineto{\pgfqpoint{2.001521in}{5.789961in}}%
\pgfpathlineto{\pgfqpoint{2.087792in}{5.782101in}}%
\pgfpathlineto{\pgfqpoint{2.174063in}{5.775022in}}%
\pgfpathlineto{\pgfqpoint{2.260334in}{5.768679in}}%
\pgfpathlineto{\pgfqpoint{2.346605in}{5.763034in}}%
\pgfpathlineto{\pgfqpoint{2.432876in}{5.758053in}}%
\pgfpathlineto{\pgfqpoint{2.519147in}{5.753710in}}%
\pgfpathlineto{\pgfqpoint{2.605417in}{5.749978in}}%
\pgfpathlineto{\pgfqpoint{2.691688in}{5.746840in}}%
\pgfusepath{stroke}%
\end{pgfscope}%
\begin{pgfscope}%
\pgfpathrectangle{\pgfqpoint{0.880000in}{4.908628in}}{\pgfqpoint{1.897959in}{1.372727in}} %
\pgfusepath{clip}%
\pgfsetbuttcap%
\pgfsetmiterjoin%
\definecolor{currentfill}{rgb}{0.000000,0.750000,0.750000}%
\pgfsetfillcolor{currentfill}%
\pgfsetlinewidth{1.003750pt}%
\definecolor{currentstroke}{rgb}{0.000000,0.750000,0.750000}%
\pgfsetstrokecolor{currentstroke}%
\pgfsetdash{}{0pt}%
\pgfsys@defobject{currentmarker}{\pgfqpoint{-0.041667in}{-0.041667in}}{\pgfqpoint{0.041667in}{0.041667in}}{%
\pgfpathmoveto{\pgfqpoint{-0.000000in}{-0.041667in}}%
\pgfpathlineto{\pgfqpoint{0.041667in}{0.041667in}}%
\pgfpathlineto{\pgfqpoint{-0.041667in}{0.041667in}}%
\pgfpathclose%
\pgfusepath{stroke,fill}%
}%
\begin{pgfscope}%
\pgfsys@transformshift{0.966271in}{5.978793in}%
\pgfsys@useobject{currentmarker}{}%
\end{pgfscope}%
\begin{pgfscope}%
\pgfsys@transformshift{1.311354in}{5.889682in}%
\pgfsys@useobject{currentmarker}{}%
\end{pgfscope}%
\begin{pgfscope}%
\pgfsys@transformshift{1.656438in}{5.830413in}%
\pgfsys@useobject{currentmarker}{}%
\end{pgfscope}%
\begin{pgfscope}%
\pgfsys@transformshift{2.001521in}{5.789961in}%
\pgfsys@useobject{currentmarker}{}%
\end{pgfscope}%
\begin{pgfscope}%
\pgfsys@transformshift{2.346605in}{5.763034in}%
\pgfsys@useobject{currentmarker}{}%
\end{pgfscope}%
\begin{pgfscope}%
\pgfsys@transformshift{2.691688in}{5.746840in}%
\pgfsys@useobject{currentmarker}{}%
\end{pgfscope}%
\end{pgfscope}%
\begin{pgfscope}%
\pgfpathrectangle{\pgfqpoint{0.880000in}{4.908628in}}{\pgfqpoint{1.897959in}{1.372727in}} %
\pgfusepath{clip}%
\pgfsetbuttcap%
\pgfsetroundjoin%
\pgfsetlinewidth{1.505625pt}%
\definecolor{currentstroke}{rgb}{0.000000,0.000000,0.000000}%
\pgfsetstrokecolor{currentstroke}%
\pgfsetdash{{1.500000pt}{2.475000pt}}{0.000000pt}%
\pgfpathmoveto{\pgfqpoint{0.966271in}{6.218958in}}%
\pgfpathlineto{\pgfqpoint{1.052542in}{6.210451in}}%
\pgfpathlineto{\pgfqpoint{1.138813in}{6.202347in}}%
\pgfpathlineto{\pgfqpoint{1.225083in}{6.195324in}}%
\pgfpathlineto{\pgfqpoint{1.311354in}{6.189167in}}%
\pgfpathlineto{\pgfqpoint{1.397625in}{6.183719in}}%
\pgfpathlineto{\pgfqpoint{1.483896in}{6.178860in}}%
\pgfpathlineto{\pgfqpoint{1.570167in}{6.174503in}}%
\pgfpathlineto{\pgfqpoint{1.656438in}{6.170576in}}%
\pgfpathlineto{\pgfqpoint{1.742709in}{6.167025in}}%
\pgfpathlineto{\pgfqpoint{1.828980in}{6.163808in}}%
\pgfpathlineto{\pgfqpoint{1.915250in}{6.160890in}}%
\pgfpathlineto{\pgfqpoint{2.001521in}{6.158244in}}%
\pgfpathlineto{\pgfqpoint{2.087792in}{6.155847in}}%
\pgfpathlineto{\pgfqpoint{2.174063in}{6.153681in}}%
\pgfpathlineto{\pgfqpoint{2.260334in}{6.151730in}}%
\pgfpathlineto{\pgfqpoint{2.346605in}{6.149983in}}%
\pgfpathlineto{\pgfqpoint{2.432876in}{6.148428in}}%
\pgfpathlineto{\pgfqpoint{2.519147in}{6.147058in}}%
\pgfpathlineto{\pgfqpoint{2.605417in}{6.145867in}}%
\pgfpathlineto{\pgfqpoint{2.691688in}{6.144848in}}%
\pgfusepath{stroke}%
\end{pgfscope}%
\begin{pgfscope}%
\pgfpathrectangle{\pgfqpoint{0.880000in}{4.908628in}}{\pgfqpoint{1.897959in}{1.372727in}} %
\pgfusepath{clip}%
\pgfsetbuttcap%
\pgfsetroundjoin%
\definecolor{currentfill}{rgb}{0.000000,0.000000,0.000000}%
\pgfsetfillcolor{currentfill}%
\pgfsetlinewidth{1.003750pt}%
\definecolor{currentstroke}{rgb}{0.000000,0.000000,0.000000}%
\pgfsetstrokecolor{currentstroke}%
\pgfsetdash{}{0pt}%
\pgfsys@defobject{currentmarker}{\pgfqpoint{-0.041667in}{-0.041667in}}{\pgfqpoint{0.041667in}{0.041667in}}{%
\pgfpathmoveto{\pgfqpoint{-0.041667in}{0.000000in}}%
\pgfpathlineto{\pgfqpoint{0.041667in}{0.000000in}}%
\pgfpathmoveto{\pgfqpoint{0.000000in}{-0.041667in}}%
\pgfpathlineto{\pgfqpoint{0.000000in}{0.041667in}}%
\pgfusepath{stroke,fill}%
}%
\begin{pgfscope}%
\pgfsys@transformshift{0.966271in}{6.218958in}%
\pgfsys@useobject{currentmarker}{}%
\end{pgfscope}%
\begin{pgfscope}%
\pgfsys@transformshift{1.311354in}{6.189167in}%
\pgfsys@useobject{currentmarker}{}%
\end{pgfscope}%
\begin{pgfscope}%
\pgfsys@transformshift{1.656438in}{6.170576in}%
\pgfsys@useobject{currentmarker}{}%
\end{pgfscope}%
\begin{pgfscope}%
\pgfsys@transformshift{2.001521in}{6.158244in}%
\pgfsys@useobject{currentmarker}{}%
\end{pgfscope}%
\begin{pgfscope}%
\pgfsys@transformshift{2.346605in}{6.149983in}%
\pgfsys@useobject{currentmarker}{}%
\end{pgfscope}%
\begin{pgfscope}%
\pgfsys@transformshift{2.691688in}{6.144848in}%
\pgfsys@useobject{currentmarker}{}%
\end{pgfscope}%
\end{pgfscope}%
\begin{pgfscope}%
\pgfsetrectcap%
\pgfsetmiterjoin%
\pgfsetlinewidth{0.803000pt}%
\definecolor{currentstroke}{rgb}{0.000000,0.000000,0.000000}%
\pgfsetstrokecolor{currentstroke}%
\pgfsetdash{}{0pt}%
\pgfpathmoveto{\pgfqpoint{0.880000in}{4.908628in}}%
\pgfpathlineto{\pgfqpoint{0.880000in}{6.281355in}}%
\pgfusepath{stroke}%
\end{pgfscope}%
\begin{pgfscope}%
\pgfsetrectcap%
\pgfsetmiterjoin%
\pgfsetlinewidth{0.803000pt}%
\definecolor{currentstroke}{rgb}{0.000000,0.000000,0.000000}%
\pgfsetstrokecolor{currentstroke}%
\pgfsetdash{}{0pt}%
\pgfpathmoveto{\pgfqpoint{2.777959in}{4.908628in}}%
\pgfpathlineto{\pgfqpoint{2.777959in}{6.281355in}}%
\pgfusepath{stroke}%
\end{pgfscope}%
\begin{pgfscope}%
\pgfsetrectcap%
\pgfsetmiterjoin%
\pgfsetlinewidth{0.803000pt}%
\definecolor{currentstroke}{rgb}{0.000000,0.000000,0.000000}%
\pgfsetstrokecolor{currentstroke}%
\pgfsetdash{}{0pt}%
\pgfpathmoveto{\pgfqpoint{0.880000in}{4.908628in}}%
\pgfpathlineto{\pgfqpoint{2.777959in}{4.908628in}}%
\pgfusepath{stroke}%
\end{pgfscope}%
\begin{pgfscope}%
\pgfsetrectcap%
\pgfsetmiterjoin%
\pgfsetlinewidth{0.803000pt}%
\definecolor{currentstroke}{rgb}{0.000000,0.000000,0.000000}%
\pgfsetstrokecolor{currentstroke}%
\pgfsetdash{}{0pt}%
\pgfpathmoveto{\pgfqpoint{0.880000in}{6.281355in}}%
\pgfpathlineto{\pgfqpoint{2.777959in}{6.281355in}}%
\pgfusepath{stroke}%
\end{pgfscope}%
\begin{pgfscope}%
\pgfsetbuttcap%
\pgfsetmiterjoin%
\definecolor{currentfill}{rgb}{1.000000,1.000000,1.000000}%
\pgfsetfillcolor{currentfill}%
\pgfsetlinewidth{0.000000pt}%
\definecolor{currentstroke}{rgb}{0.000000,0.000000,0.000000}%
\pgfsetstrokecolor{currentstroke}%
\pgfsetstrokeopacity{0.000000}%
\pgfsetdash{}{0pt}%
\pgfpathmoveto{\pgfqpoint{3.347347in}{4.908628in}}%
\pgfpathlineto{\pgfqpoint{5.245306in}{4.908628in}}%
\pgfpathlineto{\pgfqpoint{5.245306in}{6.281355in}}%
\pgfpathlineto{\pgfqpoint{3.347347in}{6.281355in}}%
\pgfpathclose%
\pgfusepath{fill}%
\end{pgfscope}%
\begin{pgfscope}%
\pgfsetbuttcap%
\pgfsetroundjoin%
\definecolor{currentfill}{rgb}{0.000000,0.000000,0.000000}%
\pgfsetfillcolor{currentfill}%
\pgfsetlinewidth{0.803000pt}%
\definecolor{currentstroke}{rgb}{0.000000,0.000000,0.000000}%
\pgfsetstrokecolor{currentstroke}%
\pgfsetdash{}{0pt}%
\pgfsys@defobject{currentmarker}{\pgfqpoint{0.000000in}{-0.048611in}}{\pgfqpoint{0.000000in}{0.000000in}}{%
\pgfpathmoveto{\pgfqpoint{0.000000in}{0.000000in}}%
\pgfpathlineto{\pgfqpoint{0.000000in}{-0.048611in}}%
\pgfusepath{stroke,fill}%
}%
\begin{pgfscope}%
\pgfsys@transformshift{3.721187in}{4.908628in}%
\pgfsys@useobject{currentmarker}{}%
\end{pgfscope}%
\end{pgfscope}%
\begin{pgfscope}%
\pgftext[x=3.721187in,y=4.811405in,,top]{\rmfamily\fontsize{10.000000}{12.000000}\selectfont \(\displaystyle 0.1\)}%
\end{pgfscope}%
\begin{pgfscope}%
\pgfsetbuttcap%
\pgfsetroundjoin%
\definecolor{currentfill}{rgb}{0.000000,0.000000,0.000000}%
\pgfsetfillcolor{currentfill}%
\pgfsetlinewidth{0.803000pt}%
\definecolor{currentstroke}{rgb}{0.000000,0.000000,0.000000}%
\pgfsetstrokecolor{currentstroke}%
\pgfsetdash{}{0pt}%
\pgfsys@defobject{currentmarker}{\pgfqpoint{0.000000in}{-0.048611in}}{\pgfqpoint{0.000000in}{0.000000in}}{%
\pgfpathmoveto{\pgfqpoint{0.000000in}{0.000000in}}%
\pgfpathlineto{\pgfqpoint{0.000000in}{-0.048611in}}%
\pgfusepath{stroke,fill}%
}%
\begin{pgfscope}%
\pgfsys@transformshift{4.583896in}{4.908628in}%
\pgfsys@useobject{currentmarker}{}%
\end{pgfscope}%
\end{pgfscope}%
\begin{pgfscope}%
\pgftext[x=4.583896in,y=4.811405in,,top]{\rmfamily\fontsize{10.000000}{12.000000}\selectfont \(\displaystyle 0.2\)}%
\end{pgfscope}%
\begin{pgfscope}%
\pgfsetbuttcap%
\pgfsetroundjoin%
\definecolor{currentfill}{rgb}{0.000000,0.000000,0.000000}%
\pgfsetfillcolor{currentfill}%
\pgfsetlinewidth{0.803000pt}%
\definecolor{currentstroke}{rgb}{0.000000,0.000000,0.000000}%
\pgfsetstrokecolor{currentstroke}%
\pgfsetdash{}{0pt}%
\pgfsys@defobject{currentmarker}{\pgfqpoint{-0.048611in}{0.000000in}}{\pgfqpoint{0.000000in}{0.000000in}}{%
\pgfpathmoveto{\pgfqpoint{0.000000in}{0.000000in}}%
\pgfpathlineto{\pgfqpoint{-0.048611in}{0.000000in}}%
\pgfusepath{stroke,fill}%
}%
\begin{pgfscope}%
\pgfsys@transformshift{3.347347in}{5.277214in}%
\pgfsys@useobject{currentmarker}{}%
\end{pgfscope}%
\end{pgfscope}%
\begin{pgfscope}%
\pgftext[x=2.962122in,y=5.224452in,left,base]{\rmfamily\fontsize{10.000000}{12.000000}\selectfont \(\displaystyle 10^{-7}\)}%
\end{pgfscope}%
\begin{pgfscope}%
\pgfsetbuttcap%
\pgfsetroundjoin%
\definecolor{currentfill}{rgb}{0.000000,0.000000,0.000000}%
\pgfsetfillcolor{currentfill}%
\pgfsetlinewidth{0.803000pt}%
\definecolor{currentstroke}{rgb}{0.000000,0.000000,0.000000}%
\pgfsetstrokecolor{currentstroke}%
\pgfsetdash{}{0pt}%
\pgfsys@defobject{currentmarker}{\pgfqpoint{-0.048611in}{0.000000in}}{\pgfqpoint{0.000000in}{0.000000in}}{%
\pgfpathmoveto{\pgfqpoint{0.000000in}{0.000000in}}%
\pgfpathlineto{\pgfqpoint{-0.048611in}{0.000000in}}%
\pgfusepath{stroke,fill}%
}%
\begin{pgfscope}%
\pgfsys@transformshift{3.347347in}{5.783107in}%
\pgfsys@useobject{currentmarker}{}%
\end{pgfscope}%
\end{pgfscope}%
\begin{pgfscope}%
\pgftext[x=2.962122in,y=5.730345in,left,base]{\rmfamily\fontsize{10.000000}{12.000000}\selectfont \(\displaystyle 10^{-6}\)}%
\end{pgfscope}%
\begin{pgfscope}%
\pgfsetbuttcap%
\pgfsetroundjoin%
\definecolor{currentfill}{rgb}{0.000000,0.000000,0.000000}%
\pgfsetfillcolor{currentfill}%
\pgfsetlinewidth{0.602250pt}%
\definecolor{currentstroke}{rgb}{0.000000,0.000000,0.000000}%
\pgfsetstrokecolor{currentstroke}%
\pgfsetdash{}{0pt}%
\pgfsys@defobject{currentmarker}{\pgfqpoint{-0.027778in}{0.000000in}}{\pgfqpoint{0.000000in}{0.000000in}}{%
\pgfpathmoveto{\pgfqpoint{0.000000in}{0.000000in}}%
\pgfpathlineto{\pgfqpoint{-0.027778in}{0.000000in}}%
\pgfusepath{stroke,fill}%
}%
\begin{pgfscope}%
\pgfsys@transformshift{3.347347in}{4.923609in}%
\pgfsys@useobject{currentmarker}{}%
\end{pgfscope}%
\end{pgfscope}%
\begin{pgfscope}%
\pgfsetbuttcap%
\pgfsetroundjoin%
\definecolor{currentfill}{rgb}{0.000000,0.000000,0.000000}%
\pgfsetfillcolor{currentfill}%
\pgfsetlinewidth{0.602250pt}%
\definecolor{currentstroke}{rgb}{0.000000,0.000000,0.000000}%
\pgfsetstrokecolor{currentstroke}%
\pgfsetdash{}{0pt}%
\pgfsys@defobject{currentmarker}{\pgfqpoint{-0.027778in}{0.000000in}}{\pgfqpoint{0.000000in}{0.000000in}}{%
\pgfpathmoveto{\pgfqpoint{0.000000in}{0.000000in}}%
\pgfpathlineto{\pgfqpoint{-0.027778in}{0.000000in}}%
\pgfusepath{stroke,fill}%
}%
\begin{pgfscope}%
\pgfsys@transformshift{3.347347in}{5.012693in}%
\pgfsys@useobject{currentmarker}{}%
\end{pgfscope}%
\end{pgfscope}%
\begin{pgfscope}%
\pgfsetbuttcap%
\pgfsetroundjoin%
\definecolor{currentfill}{rgb}{0.000000,0.000000,0.000000}%
\pgfsetfillcolor{currentfill}%
\pgfsetlinewidth{0.602250pt}%
\definecolor{currentstroke}{rgb}{0.000000,0.000000,0.000000}%
\pgfsetstrokecolor{currentstroke}%
\pgfsetdash{}{0pt}%
\pgfsys@defobject{currentmarker}{\pgfqpoint{-0.027778in}{0.000000in}}{\pgfqpoint{0.000000in}{0.000000in}}{%
\pgfpathmoveto{\pgfqpoint{0.000000in}{0.000000in}}%
\pgfpathlineto{\pgfqpoint{-0.027778in}{0.000000in}}%
\pgfusepath{stroke,fill}%
}%
\begin{pgfscope}%
\pgfsys@transformshift{3.347347in}{5.075898in}%
\pgfsys@useobject{currentmarker}{}%
\end{pgfscope}%
\end{pgfscope}%
\begin{pgfscope}%
\pgfsetbuttcap%
\pgfsetroundjoin%
\definecolor{currentfill}{rgb}{0.000000,0.000000,0.000000}%
\pgfsetfillcolor{currentfill}%
\pgfsetlinewidth{0.602250pt}%
\definecolor{currentstroke}{rgb}{0.000000,0.000000,0.000000}%
\pgfsetstrokecolor{currentstroke}%
\pgfsetdash{}{0pt}%
\pgfsys@defobject{currentmarker}{\pgfqpoint{-0.027778in}{0.000000in}}{\pgfqpoint{0.000000in}{0.000000in}}{%
\pgfpathmoveto{\pgfqpoint{0.000000in}{0.000000in}}%
\pgfpathlineto{\pgfqpoint{-0.027778in}{0.000000in}}%
\pgfusepath{stroke,fill}%
}%
\begin{pgfscope}%
\pgfsys@transformshift{3.347347in}{5.124925in}%
\pgfsys@useobject{currentmarker}{}%
\end{pgfscope}%
\end{pgfscope}%
\begin{pgfscope}%
\pgfsetbuttcap%
\pgfsetroundjoin%
\definecolor{currentfill}{rgb}{0.000000,0.000000,0.000000}%
\pgfsetfillcolor{currentfill}%
\pgfsetlinewidth{0.602250pt}%
\definecolor{currentstroke}{rgb}{0.000000,0.000000,0.000000}%
\pgfsetstrokecolor{currentstroke}%
\pgfsetdash{}{0pt}%
\pgfsys@defobject{currentmarker}{\pgfqpoint{-0.027778in}{0.000000in}}{\pgfqpoint{0.000000in}{0.000000in}}{%
\pgfpathmoveto{\pgfqpoint{0.000000in}{0.000000in}}%
\pgfpathlineto{\pgfqpoint{-0.027778in}{0.000000in}}%
\pgfusepath{stroke,fill}%
}%
\begin{pgfscope}%
\pgfsys@transformshift{3.347347in}{5.164982in}%
\pgfsys@useobject{currentmarker}{}%
\end{pgfscope}%
\end{pgfscope}%
\begin{pgfscope}%
\pgfsetbuttcap%
\pgfsetroundjoin%
\definecolor{currentfill}{rgb}{0.000000,0.000000,0.000000}%
\pgfsetfillcolor{currentfill}%
\pgfsetlinewidth{0.602250pt}%
\definecolor{currentstroke}{rgb}{0.000000,0.000000,0.000000}%
\pgfsetstrokecolor{currentstroke}%
\pgfsetdash{}{0pt}%
\pgfsys@defobject{currentmarker}{\pgfqpoint{-0.027778in}{0.000000in}}{\pgfqpoint{0.000000in}{0.000000in}}{%
\pgfpathmoveto{\pgfqpoint{0.000000in}{0.000000in}}%
\pgfpathlineto{\pgfqpoint{-0.027778in}{0.000000in}}%
\pgfusepath{stroke,fill}%
}%
\begin{pgfscope}%
\pgfsys@transformshift{3.347347in}{5.198850in}%
\pgfsys@useobject{currentmarker}{}%
\end{pgfscope}%
\end{pgfscope}%
\begin{pgfscope}%
\pgfsetbuttcap%
\pgfsetroundjoin%
\definecolor{currentfill}{rgb}{0.000000,0.000000,0.000000}%
\pgfsetfillcolor{currentfill}%
\pgfsetlinewidth{0.602250pt}%
\definecolor{currentstroke}{rgb}{0.000000,0.000000,0.000000}%
\pgfsetstrokecolor{currentstroke}%
\pgfsetdash{}{0pt}%
\pgfsys@defobject{currentmarker}{\pgfqpoint{-0.027778in}{0.000000in}}{\pgfqpoint{0.000000in}{0.000000in}}{%
\pgfpathmoveto{\pgfqpoint{0.000000in}{0.000000in}}%
\pgfpathlineto{\pgfqpoint{-0.027778in}{0.000000in}}%
\pgfusepath{stroke,fill}%
}%
\begin{pgfscope}%
\pgfsys@transformshift{3.347347in}{5.228188in}%
\pgfsys@useobject{currentmarker}{}%
\end{pgfscope}%
\end{pgfscope}%
\begin{pgfscope}%
\pgfsetbuttcap%
\pgfsetroundjoin%
\definecolor{currentfill}{rgb}{0.000000,0.000000,0.000000}%
\pgfsetfillcolor{currentfill}%
\pgfsetlinewidth{0.602250pt}%
\definecolor{currentstroke}{rgb}{0.000000,0.000000,0.000000}%
\pgfsetstrokecolor{currentstroke}%
\pgfsetdash{}{0pt}%
\pgfsys@defobject{currentmarker}{\pgfqpoint{-0.027778in}{0.000000in}}{\pgfqpoint{0.000000in}{0.000000in}}{%
\pgfpathmoveto{\pgfqpoint{0.000000in}{0.000000in}}%
\pgfpathlineto{\pgfqpoint{-0.027778in}{0.000000in}}%
\pgfusepath{stroke,fill}%
}%
\begin{pgfscope}%
\pgfsys@transformshift{3.347347in}{5.254065in}%
\pgfsys@useobject{currentmarker}{}%
\end{pgfscope}%
\end{pgfscope}%
\begin{pgfscope}%
\pgfsetbuttcap%
\pgfsetroundjoin%
\definecolor{currentfill}{rgb}{0.000000,0.000000,0.000000}%
\pgfsetfillcolor{currentfill}%
\pgfsetlinewidth{0.602250pt}%
\definecolor{currentstroke}{rgb}{0.000000,0.000000,0.000000}%
\pgfsetstrokecolor{currentstroke}%
\pgfsetdash{}{0pt}%
\pgfsys@defobject{currentmarker}{\pgfqpoint{-0.027778in}{0.000000in}}{\pgfqpoint{0.000000in}{0.000000in}}{%
\pgfpathmoveto{\pgfqpoint{0.000000in}{0.000000in}}%
\pgfpathlineto{\pgfqpoint{-0.027778in}{0.000000in}}%
\pgfusepath{stroke,fill}%
}%
\begin{pgfscope}%
\pgfsys@transformshift{3.347347in}{5.429503in}%
\pgfsys@useobject{currentmarker}{}%
\end{pgfscope}%
\end{pgfscope}%
\begin{pgfscope}%
\pgfsetbuttcap%
\pgfsetroundjoin%
\definecolor{currentfill}{rgb}{0.000000,0.000000,0.000000}%
\pgfsetfillcolor{currentfill}%
\pgfsetlinewidth{0.602250pt}%
\definecolor{currentstroke}{rgb}{0.000000,0.000000,0.000000}%
\pgfsetstrokecolor{currentstroke}%
\pgfsetdash{}{0pt}%
\pgfsys@defobject{currentmarker}{\pgfqpoint{-0.027778in}{0.000000in}}{\pgfqpoint{0.000000in}{0.000000in}}{%
\pgfpathmoveto{\pgfqpoint{0.000000in}{0.000000in}}%
\pgfpathlineto{\pgfqpoint{-0.027778in}{0.000000in}}%
\pgfusepath{stroke,fill}%
}%
\begin{pgfscope}%
\pgfsys@transformshift{3.347347in}{5.518586in}%
\pgfsys@useobject{currentmarker}{}%
\end{pgfscope}%
\end{pgfscope}%
\begin{pgfscope}%
\pgfsetbuttcap%
\pgfsetroundjoin%
\definecolor{currentfill}{rgb}{0.000000,0.000000,0.000000}%
\pgfsetfillcolor{currentfill}%
\pgfsetlinewidth{0.602250pt}%
\definecolor{currentstroke}{rgb}{0.000000,0.000000,0.000000}%
\pgfsetstrokecolor{currentstroke}%
\pgfsetdash{}{0pt}%
\pgfsys@defobject{currentmarker}{\pgfqpoint{-0.027778in}{0.000000in}}{\pgfqpoint{0.000000in}{0.000000in}}{%
\pgfpathmoveto{\pgfqpoint{0.000000in}{0.000000in}}%
\pgfpathlineto{\pgfqpoint{-0.027778in}{0.000000in}}%
\pgfusepath{stroke,fill}%
}%
\begin{pgfscope}%
\pgfsys@transformshift{3.347347in}{5.581792in}%
\pgfsys@useobject{currentmarker}{}%
\end{pgfscope}%
\end{pgfscope}%
\begin{pgfscope}%
\pgfsetbuttcap%
\pgfsetroundjoin%
\definecolor{currentfill}{rgb}{0.000000,0.000000,0.000000}%
\pgfsetfillcolor{currentfill}%
\pgfsetlinewidth{0.602250pt}%
\definecolor{currentstroke}{rgb}{0.000000,0.000000,0.000000}%
\pgfsetstrokecolor{currentstroke}%
\pgfsetdash{}{0pt}%
\pgfsys@defobject{currentmarker}{\pgfqpoint{-0.027778in}{0.000000in}}{\pgfqpoint{0.000000in}{0.000000in}}{%
\pgfpathmoveto{\pgfqpoint{0.000000in}{0.000000in}}%
\pgfpathlineto{\pgfqpoint{-0.027778in}{0.000000in}}%
\pgfusepath{stroke,fill}%
}%
\begin{pgfscope}%
\pgfsys@transformshift{3.347347in}{5.630818in}%
\pgfsys@useobject{currentmarker}{}%
\end{pgfscope}%
\end{pgfscope}%
\begin{pgfscope}%
\pgfsetbuttcap%
\pgfsetroundjoin%
\definecolor{currentfill}{rgb}{0.000000,0.000000,0.000000}%
\pgfsetfillcolor{currentfill}%
\pgfsetlinewidth{0.602250pt}%
\definecolor{currentstroke}{rgb}{0.000000,0.000000,0.000000}%
\pgfsetstrokecolor{currentstroke}%
\pgfsetdash{}{0pt}%
\pgfsys@defobject{currentmarker}{\pgfqpoint{-0.027778in}{0.000000in}}{\pgfqpoint{0.000000in}{0.000000in}}{%
\pgfpathmoveto{\pgfqpoint{0.000000in}{0.000000in}}%
\pgfpathlineto{\pgfqpoint{-0.027778in}{0.000000in}}%
\pgfusepath{stroke,fill}%
}%
\begin{pgfscope}%
\pgfsys@transformshift{3.347347in}{5.670875in}%
\pgfsys@useobject{currentmarker}{}%
\end{pgfscope}%
\end{pgfscope}%
\begin{pgfscope}%
\pgfsetbuttcap%
\pgfsetroundjoin%
\definecolor{currentfill}{rgb}{0.000000,0.000000,0.000000}%
\pgfsetfillcolor{currentfill}%
\pgfsetlinewidth{0.602250pt}%
\definecolor{currentstroke}{rgb}{0.000000,0.000000,0.000000}%
\pgfsetstrokecolor{currentstroke}%
\pgfsetdash{}{0pt}%
\pgfsys@defobject{currentmarker}{\pgfqpoint{-0.027778in}{0.000000in}}{\pgfqpoint{0.000000in}{0.000000in}}{%
\pgfpathmoveto{\pgfqpoint{0.000000in}{0.000000in}}%
\pgfpathlineto{\pgfqpoint{-0.027778in}{0.000000in}}%
\pgfusepath{stroke,fill}%
}%
\begin{pgfscope}%
\pgfsys@transformshift{3.347347in}{5.704743in}%
\pgfsys@useobject{currentmarker}{}%
\end{pgfscope}%
\end{pgfscope}%
\begin{pgfscope}%
\pgfsetbuttcap%
\pgfsetroundjoin%
\definecolor{currentfill}{rgb}{0.000000,0.000000,0.000000}%
\pgfsetfillcolor{currentfill}%
\pgfsetlinewidth{0.602250pt}%
\definecolor{currentstroke}{rgb}{0.000000,0.000000,0.000000}%
\pgfsetstrokecolor{currentstroke}%
\pgfsetdash{}{0pt}%
\pgfsys@defobject{currentmarker}{\pgfqpoint{-0.027778in}{0.000000in}}{\pgfqpoint{0.000000in}{0.000000in}}{%
\pgfpathmoveto{\pgfqpoint{0.000000in}{0.000000in}}%
\pgfpathlineto{\pgfqpoint{-0.027778in}{0.000000in}}%
\pgfusepath{stroke,fill}%
}%
\begin{pgfscope}%
\pgfsys@transformshift{3.347347in}{5.734081in}%
\pgfsys@useobject{currentmarker}{}%
\end{pgfscope}%
\end{pgfscope}%
\begin{pgfscope}%
\pgfsetbuttcap%
\pgfsetroundjoin%
\definecolor{currentfill}{rgb}{0.000000,0.000000,0.000000}%
\pgfsetfillcolor{currentfill}%
\pgfsetlinewidth{0.602250pt}%
\definecolor{currentstroke}{rgb}{0.000000,0.000000,0.000000}%
\pgfsetstrokecolor{currentstroke}%
\pgfsetdash{}{0pt}%
\pgfsys@defobject{currentmarker}{\pgfqpoint{-0.027778in}{0.000000in}}{\pgfqpoint{0.000000in}{0.000000in}}{%
\pgfpathmoveto{\pgfqpoint{0.000000in}{0.000000in}}%
\pgfpathlineto{\pgfqpoint{-0.027778in}{0.000000in}}%
\pgfusepath{stroke,fill}%
}%
\begin{pgfscope}%
\pgfsys@transformshift{3.347347in}{5.759959in}%
\pgfsys@useobject{currentmarker}{}%
\end{pgfscope}%
\end{pgfscope}%
\begin{pgfscope}%
\pgfsetbuttcap%
\pgfsetroundjoin%
\definecolor{currentfill}{rgb}{0.000000,0.000000,0.000000}%
\pgfsetfillcolor{currentfill}%
\pgfsetlinewidth{0.602250pt}%
\definecolor{currentstroke}{rgb}{0.000000,0.000000,0.000000}%
\pgfsetstrokecolor{currentstroke}%
\pgfsetdash{}{0pt}%
\pgfsys@defobject{currentmarker}{\pgfqpoint{-0.027778in}{0.000000in}}{\pgfqpoint{0.000000in}{0.000000in}}{%
\pgfpathmoveto{\pgfqpoint{0.000000in}{0.000000in}}%
\pgfpathlineto{\pgfqpoint{-0.027778in}{0.000000in}}%
\pgfusepath{stroke,fill}%
}%
\begin{pgfscope}%
\pgfsys@transformshift{3.347347in}{5.935396in}%
\pgfsys@useobject{currentmarker}{}%
\end{pgfscope}%
\end{pgfscope}%
\begin{pgfscope}%
\pgfsetbuttcap%
\pgfsetroundjoin%
\definecolor{currentfill}{rgb}{0.000000,0.000000,0.000000}%
\pgfsetfillcolor{currentfill}%
\pgfsetlinewidth{0.602250pt}%
\definecolor{currentstroke}{rgb}{0.000000,0.000000,0.000000}%
\pgfsetstrokecolor{currentstroke}%
\pgfsetdash{}{0pt}%
\pgfsys@defobject{currentmarker}{\pgfqpoint{-0.027778in}{0.000000in}}{\pgfqpoint{0.000000in}{0.000000in}}{%
\pgfpathmoveto{\pgfqpoint{0.000000in}{0.000000in}}%
\pgfpathlineto{\pgfqpoint{-0.027778in}{0.000000in}}%
\pgfusepath{stroke,fill}%
}%
\begin{pgfscope}%
\pgfsys@transformshift{3.347347in}{6.024479in}%
\pgfsys@useobject{currentmarker}{}%
\end{pgfscope}%
\end{pgfscope}%
\begin{pgfscope}%
\pgfsetbuttcap%
\pgfsetroundjoin%
\definecolor{currentfill}{rgb}{0.000000,0.000000,0.000000}%
\pgfsetfillcolor{currentfill}%
\pgfsetlinewidth{0.602250pt}%
\definecolor{currentstroke}{rgb}{0.000000,0.000000,0.000000}%
\pgfsetstrokecolor{currentstroke}%
\pgfsetdash{}{0pt}%
\pgfsys@defobject{currentmarker}{\pgfqpoint{-0.027778in}{0.000000in}}{\pgfqpoint{0.000000in}{0.000000in}}{%
\pgfpathmoveto{\pgfqpoint{0.000000in}{0.000000in}}%
\pgfpathlineto{\pgfqpoint{-0.027778in}{0.000000in}}%
\pgfusepath{stroke,fill}%
}%
\begin{pgfscope}%
\pgfsys@transformshift{3.347347in}{6.087685in}%
\pgfsys@useobject{currentmarker}{}%
\end{pgfscope}%
\end{pgfscope}%
\begin{pgfscope}%
\pgfsetbuttcap%
\pgfsetroundjoin%
\definecolor{currentfill}{rgb}{0.000000,0.000000,0.000000}%
\pgfsetfillcolor{currentfill}%
\pgfsetlinewidth{0.602250pt}%
\definecolor{currentstroke}{rgb}{0.000000,0.000000,0.000000}%
\pgfsetstrokecolor{currentstroke}%
\pgfsetdash{}{0pt}%
\pgfsys@defobject{currentmarker}{\pgfqpoint{-0.027778in}{0.000000in}}{\pgfqpoint{0.000000in}{0.000000in}}{%
\pgfpathmoveto{\pgfqpoint{0.000000in}{0.000000in}}%
\pgfpathlineto{\pgfqpoint{-0.027778in}{0.000000in}}%
\pgfusepath{stroke,fill}%
}%
\begin{pgfscope}%
\pgfsys@transformshift{3.347347in}{6.136711in}%
\pgfsys@useobject{currentmarker}{}%
\end{pgfscope}%
\end{pgfscope}%
\begin{pgfscope}%
\pgfsetbuttcap%
\pgfsetroundjoin%
\definecolor{currentfill}{rgb}{0.000000,0.000000,0.000000}%
\pgfsetfillcolor{currentfill}%
\pgfsetlinewidth{0.602250pt}%
\definecolor{currentstroke}{rgb}{0.000000,0.000000,0.000000}%
\pgfsetstrokecolor{currentstroke}%
\pgfsetdash{}{0pt}%
\pgfsys@defobject{currentmarker}{\pgfqpoint{-0.027778in}{0.000000in}}{\pgfqpoint{0.000000in}{0.000000in}}{%
\pgfpathmoveto{\pgfqpoint{0.000000in}{0.000000in}}%
\pgfpathlineto{\pgfqpoint{-0.027778in}{0.000000in}}%
\pgfusepath{stroke,fill}%
}%
\begin{pgfscope}%
\pgfsys@transformshift{3.347347in}{6.176768in}%
\pgfsys@useobject{currentmarker}{}%
\end{pgfscope}%
\end{pgfscope}%
\begin{pgfscope}%
\pgfsetbuttcap%
\pgfsetroundjoin%
\definecolor{currentfill}{rgb}{0.000000,0.000000,0.000000}%
\pgfsetfillcolor{currentfill}%
\pgfsetlinewidth{0.602250pt}%
\definecolor{currentstroke}{rgb}{0.000000,0.000000,0.000000}%
\pgfsetstrokecolor{currentstroke}%
\pgfsetdash{}{0pt}%
\pgfsys@defobject{currentmarker}{\pgfqpoint{-0.027778in}{0.000000in}}{\pgfqpoint{0.000000in}{0.000000in}}{%
\pgfpathmoveto{\pgfqpoint{0.000000in}{0.000000in}}%
\pgfpathlineto{\pgfqpoint{-0.027778in}{0.000000in}}%
\pgfusepath{stroke,fill}%
}%
\begin{pgfscope}%
\pgfsys@transformshift{3.347347in}{6.210636in}%
\pgfsys@useobject{currentmarker}{}%
\end{pgfscope}%
\end{pgfscope}%
\begin{pgfscope}%
\pgfsetbuttcap%
\pgfsetroundjoin%
\definecolor{currentfill}{rgb}{0.000000,0.000000,0.000000}%
\pgfsetfillcolor{currentfill}%
\pgfsetlinewidth{0.602250pt}%
\definecolor{currentstroke}{rgb}{0.000000,0.000000,0.000000}%
\pgfsetstrokecolor{currentstroke}%
\pgfsetdash{}{0pt}%
\pgfsys@defobject{currentmarker}{\pgfqpoint{-0.027778in}{0.000000in}}{\pgfqpoint{0.000000in}{0.000000in}}{%
\pgfpathmoveto{\pgfqpoint{0.000000in}{0.000000in}}%
\pgfpathlineto{\pgfqpoint{-0.027778in}{0.000000in}}%
\pgfusepath{stroke,fill}%
}%
\begin{pgfscope}%
\pgfsys@transformshift{3.347347in}{6.239974in}%
\pgfsys@useobject{currentmarker}{}%
\end{pgfscope}%
\end{pgfscope}%
\begin{pgfscope}%
\pgfsetbuttcap%
\pgfsetroundjoin%
\definecolor{currentfill}{rgb}{0.000000,0.000000,0.000000}%
\pgfsetfillcolor{currentfill}%
\pgfsetlinewidth{0.602250pt}%
\definecolor{currentstroke}{rgb}{0.000000,0.000000,0.000000}%
\pgfsetstrokecolor{currentstroke}%
\pgfsetdash{}{0pt}%
\pgfsys@defobject{currentmarker}{\pgfqpoint{-0.027778in}{0.000000in}}{\pgfqpoint{0.000000in}{0.000000in}}{%
\pgfpathmoveto{\pgfqpoint{0.000000in}{0.000000in}}%
\pgfpathlineto{\pgfqpoint{-0.027778in}{0.000000in}}%
\pgfusepath{stroke,fill}%
}%
\begin{pgfscope}%
\pgfsys@transformshift{3.347347in}{6.265852in}%
\pgfsys@useobject{currentmarker}{}%
\end{pgfscope}%
\end{pgfscope}%
\begin{pgfscope}%
\pgfpathrectangle{\pgfqpoint{3.347347in}{4.908628in}}{\pgfqpoint{1.897959in}{1.372727in}} %
\pgfusepath{clip}%
\pgfsetbuttcap%
\pgfsetroundjoin%
\pgfsetlinewidth{1.505625pt}%
\definecolor{currentstroke}{rgb}{1.000000,0.000000,0.000000}%
\pgfsetstrokecolor{currentstroke}%
\pgfsetdash{{5.550000pt}{2.400000pt}}{0.000000pt}%
\pgfpathmoveto{\pgfqpoint{3.433618in}{6.146045in}}%
\pgfpathlineto{\pgfqpoint{3.505510in}{6.143051in}}%
\pgfpathlineto{\pgfqpoint{3.577403in}{6.140174in}}%
\pgfpathlineto{\pgfqpoint{3.649295in}{6.137411in}}%
\pgfpathlineto{\pgfqpoint{3.721187in}{6.134760in}}%
\pgfpathlineto{\pgfqpoint{3.793080in}{6.132221in}}%
\pgfpathlineto{\pgfqpoint{3.864972in}{6.129792in}}%
\pgfpathlineto{\pgfqpoint{3.936865in}{6.127472in}}%
\pgfpathlineto{\pgfqpoint{4.008757in}{6.125262in}}%
\pgfpathlineto{\pgfqpoint{4.080649in}{6.123160in}}%
\pgfpathlineto{\pgfqpoint{4.152542in}{6.121166in}}%
\pgfpathlineto{\pgfqpoint{4.224434in}{6.119279in}}%
\pgfpathlineto{\pgfqpoint{4.296327in}{6.117500in}}%
\pgfpathlineto{\pgfqpoint{4.368219in}{6.115827in}}%
\pgfpathlineto{\pgfqpoint{4.440111in}{6.114260in}}%
\pgfpathlineto{\pgfqpoint{4.512004in}{6.112798in}}%
\pgfpathlineto{\pgfqpoint{4.583896in}{6.111442in}}%
\pgfpathlineto{\pgfqpoint{4.655788in}{6.110191in}}%
\pgfpathlineto{\pgfqpoint{4.727681in}{6.109044in}}%
\pgfpathlineto{\pgfqpoint{4.799573in}{6.108002in}}%
\pgfpathlineto{\pgfqpoint{4.871466in}{6.107063in}}%
\pgfpathlineto{\pgfqpoint{4.943358in}{6.106227in}}%
\pgfpathlineto{\pgfqpoint{5.015250in}{6.105495in}}%
\pgfpathlineto{\pgfqpoint{5.087143in}{6.104865in}}%
\pgfpathlineto{\pgfqpoint{5.159035in}{6.104338in}}%
\pgfusepath{stroke}%
\end{pgfscope}%
\begin{pgfscope}%
\pgfpathrectangle{\pgfqpoint{3.347347in}{4.908628in}}{\pgfqpoint{1.897959in}{1.372727in}} %
\pgfusepath{clip}%
\pgfsetbuttcap%
\pgfsetmiterjoin%
\definecolor{currentfill}{rgb}{1.000000,0.000000,0.000000}%
\pgfsetfillcolor{currentfill}%
\pgfsetlinewidth{1.003750pt}%
\definecolor{currentstroke}{rgb}{1.000000,0.000000,0.000000}%
\pgfsetstrokecolor{currentstroke}%
\pgfsetdash{}{0pt}%
\pgfsys@defobject{currentmarker}{\pgfqpoint{-0.041667in}{-0.041667in}}{\pgfqpoint{0.041667in}{0.041667in}}{%
\pgfpathmoveto{\pgfqpoint{-0.041667in}{-0.041667in}}%
\pgfpathlineto{\pgfqpoint{0.041667in}{-0.041667in}}%
\pgfpathlineto{\pgfqpoint{0.041667in}{0.041667in}}%
\pgfpathlineto{\pgfqpoint{-0.041667in}{0.041667in}}%
\pgfpathclose%
\pgfusepath{stroke,fill}%
}%
\begin{pgfscope}%
\pgfsys@transformshift{3.433618in}{6.146045in}%
\pgfsys@useobject{currentmarker}{}%
\end{pgfscope}%
\begin{pgfscope}%
\pgfsys@transformshift{3.793080in}{6.132221in}%
\pgfsys@useobject{currentmarker}{}%
\end{pgfscope}%
\begin{pgfscope}%
\pgfsys@transformshift{4.152542in}{6.121166in}%
\pgfsys@useobject{currentmarker}{}%
\end{pgfscope}%
\begin{pgfscope}%
\pgfsys@transformshift{4.512004in}{6.112798in}%
\pgfsys@useobject{currentmarker}{}%
\end{pgfscope}%
\begin{pgfscope}%
\pgfsys@transformshift{4.871466in}{6.107063in}%
\pgfsys@useobject{currentmarker}{}%
\end{pgfscope}%
\end{pgfscope}%
\begin{pgfscope}%
\pgfpathrectangle{\pgfqpoint{3.347347in}{4.908628in}}{\pgfqpoint{1.897959in}{1.372727in}} %
\pgfusepath{clip}%
\pgfsetrectcap%
\pgfsetroundjoin%
\pgfsetlinewidth{1.505625pt}%
\definecolor{currentstroke}{rgb}{0.000000,0.000000,1.000000}%
\pgfsetstrokecolor{currentstroke}%
\pgfsetdash{}{0pt}%
\pgfpathmoveto{\pgfqpoint{3.433618in}{5.480609in}}%
\pgfpathlineto{\pgfqpoint{3.505510in}{5.467353in}}%
\pgfpathlineto{\pgfqpoint{3.577403in}{5.454121in}}%
\pgfpathlineto{\pgfqpoint{3.649295in}{5.440820in}}%
\pgfpathlineto{\pgfqpoint{3.721187in}{5.427376in}}%
\pgfpathlineto{\pgfqpoint{3.793080in}{5.413722in}}%
\pgfpathlineto{\pgfqpoint{3.864972in}{5.399797in}}%
\pgfpathlineto{\pgfqpoint{3.936865in}{5.385540in}}%
\pgfpathlineto{\pgfqpoint{4.008757in}{5.370889in}}%
\pgfpathlineto{\pgfqpoint{4.080649in}{5.355781in}}%
\pgfpathlineto{\pgfqpoint{4.152542in}{5.340144in}}%
\pgfpathlineto{\pgfqpoint{4.224434in}{5.323901in}}%
\pgfpathlineto{\pgfqpoint{4.296327in}{5.306963in}}%
\pgfpathlineto{\pgfqpoint{4.368219in}{5.289228in}}%
\pgfpathlineto{\pgfqpoint{4.440111in}{5.270577in}}%
\pgfpathlineto{\pgfqpoint{4.512004in}{5.250866in}}%
\pgfpathlineto{\pgfqpoint{4.583896in}{5.229922in}}%
\pgfpathlineto{\pgfqpoint{4.655788in}{5.207531in}}%
\pgfpathlineto{\pgfqpoint{4.727681in}{5.183423in}}%
\pgfpathlineto{\pgfqpoint{4.799573in}{5.157251in}}%
\pgfpathlineto{\pgfqpoint{4.871466in}{5.128556in}}%
\pgfpathlineto{\pgfqpoint{4.943358in}{5.096715in}}%
\pgfpathlineto{\pgfqpoint{5.015250in}{5.060844in}}%
\pgfpathlineto{\pgfqpoint{5.087143in}{5.019635in}}%
\pgfpathlineto{\pgfqpoint{5.159035in}{4.971024in}}%
\pgfusepath{stroke}%
\end{pgfscope}%
\begin{pgfscope}%
\pgfpathrectangle{\pgfqpoint{3.347347in}{4.908628in}}{\pgfqpoint{1.897959in}{1.372727in}} %
\pgfusepath{clip}%
\pgfsetbuttcap%
\pgfsetroundjoin%
\definecolor{currentfill}{rgb}{0.000000,0.000000,1.000000}%
\pgfsetfillcolor{currentfill}%
\pgfsetlinewidth{1.003750pt}%
\definecolor{currentstroke}{rgb}{0.000000,0.000000,1.000000}%
\pgfsetstrokecolor{currentstroke}%
\pgfsetdash{}{0pt}%
\pgfsys@defobject{currentmarker}{\pgfqpoint{-0.041667in}{-0.041667in}}{\pgfqpoint{0.041667in}{0.041667in}}{%
\pgfpathmoveto{\pgfqpoint{0.000000in}{-0.041667in}}%
\pgfpathcurveto{\pgfqpoint{0.011050in}{-0.041667in}}{\pgfqpoint{0.021649in}{-0.037276in}}{\pgfqpoint{0.029463in}{-0.029463in}}%
\pgfpathcurveto{\pgfqpoint{0.037276in}{-0.021649in}}{\pgfqpoint{0.041667in}{-0.011050in}}{\pgfqpoint{0.041667in}{0.000000in}}%
\pgfpathcurveto{\pgfqpoint{0.041667in}{0.011050in}}{\pgfqpoint{0.037276in}{0.021649in}}{\pgfqpoint{0.029463in}{0.029463in}}%
\pgfpathcurveto{\pgfqpoint{0.021649in}{0.037276in}}{\pgfqpoint{0.011050in}{0.041667in}}{\pgfqpoint{0.000000in}{0.041667in}}%
\pgfpathcurveto{\pgfqpoint{-0.011050in}{0.041667in}}{\pgfqpoint{-0.021649in}{0.037276in}}{\pgfqpoint{-0.029463in}{0.029463in}}%
\pgfpathcurveto{\pgfqpoint{-0.037276in}{0.021649in}}{\pgfqpoint{-0.041667in}{0.011050in}}{\pgfqpoint{-0.041667in}{0.000000in}}%
\pgfpathcurveto{\pgfqpoint{-0.041667in}{-0.011050in}}{\pgfqpoint{-0.037276in}{-0.021649in}}{\pgfqpoint{-0.029463in}{-0.029463in}}%
\pgfpathcurveto{\pgfqpoint{-0.021649in}{-0.037276in}}{\pgfqpoint{-0.011050in}{-0.041667in}}{\pgfqpoint{0.000000in}{-0.041667in}}%
\pgfpathclose%
\pgfusepath{stroke,fill}%
}%
\begin{pgfscope}%
\pgfsys@transformshift{3.433618in}{5.480609in}%
\pgfsys@useobject{currentmarker}{}%
\end{pgfscope}%
\begin{pgfscope}%
\pgfsys@transformshift{3.793080in}{5.413722in}%
\pgfsys@useobject{currentmarker}{}%
\end{pgfscope}%
\begin{pgfscope}%
\pgfsys@transformshift{4.152542in}{5.340144in}%
\pgfsys@useobject{currentmarker}{}%
\end{pgfscope}%
\begin{pgfscope}%
\pgfsys@transformshift{4.512004in}{5.250866in}%
\pgfsys@useobject{currentmarker}{}%
\end{pgfscope}%
\begin{pgfscope}%
\pgfsys@transformshift{4.871466in}{5.128556in}%
\pgfsys@useobject{currentmarker}{}%
\end{pgfscope}%
\end{pgfscope}%
\begin{pgfscope}%
\pgfpathrectangle{\pgfqpoint{3.347347in}{4.908628in}}{\pgfqpoint{1.897959in}{1.372727in}} %
\pgfusepath{clip}%
\pgfsetbuttcap%
\pgfsetroundjoin%
\pgfsetlinewidth{1.505625pt}%
\definecolor{currentstroke}{rgb}{0.000000,0.750000,0.750000}%
\pgfsetstrokecolor{currentstroke}%
\pgfsetdash{{9.600000pt}{2.400000pt}{1.500000pt}{2.400000pt}}{0.000000pt}%
\pgfpathmoveto{\pgfqpoint{3.433618in}{5.915312in}}%
\pgfpathlineto{\pgfqpoint{3.505510in}{5.880830in}}%
\pgfpathlineto{\pgfqpoint{3.577403in}{5.850286in}}%
\pgfpathlineto{\pgfqpoint{3.649295in}{5.823014in}}%
\pgfpathlineto{\pgfqpoint{3.721187in}{5.798510in}}%
\pgfpathlineto{\pgfqpoint{3.793080in}{5.776383in}}%
\pgfpathlineto{\pgfqpoint{3.864972in}{5.756321in}}%
\pgfpathlineto{\pgfqpoint{3.936865in}{5.738076in}}%
\pgfpathlineto{\pgfqpoint{4.008757in}{5.721444in}}%
\pgfpathlineto{\pgfqpoint{4.080649in}{5.706255in}}%
\pgfpathlineto{\pgfqpoint{4.152542in}{5.692370in}}%
\pgfpathlineto{\pgfqpoint{4.224434in}{5.679671in}}%
\pgfpathlineto{\pgfqpoint{4.296327in}{5.668058in}}%
\pgfpathlineto{\pgfqpoint{4.368219in}{5.657444in}}%
\pgfpathlineto{\pgfqpoint{4.440111in}{5.647757in}}%
\pgfpathlineto{\pgfqpoint{4.512004in}{5.638934in}}%
\pgfpathlineto{\pgfqpoint{4.583896in}{5.630920in}}%
\pgfpathlineto{\pgfqpoint{4.655788in}{5.623667in}}%
\pgfpathlineto{\pgfqpoint{4.727681in}{5.617136in}}%
\pgfpathlineto{\pgfqpoint{4.799573in}{5.611291in}}%
\pgfpathlineto{\pgfqpoint{4.871466in}{5.606100in}}%
\pgfpathlineto{\pgfqpoint{4.943358in}{5.601538in}}%
\pgfpathlineto{\pgfqpoint{5.015250in}{5.597582in}}%
\pgfpathlineto{\pgfqpoint{5.087143in}{5.594213in}}%
\pgfpathlineto{\pgfqpoint{5.159035in}{5.591415in}}%
\pgfusepath{stroke}%
\end{pgfscope}%
\begin{pgfscope}%
\pgfpathrectangle{\pgfqpoint{3.347347in}{4.908628in}}{\pgfqpoint{1.897959in}{1.372727in}} %
\pgfusepath{clip}%
\pgfsetbuttcap%
\pgfsetmiterjoin%
\definecolor{currentfill}{rgb}{0.000000,0.750000,0.750000}%
\pgfsetfillcolor{currentfill}%
\pgfsetlinewidth{1.003750pt}%
\definecolor{currentstroke}{rgb}{0.000000,0.750000,0.750000}%
\pgfsetstrokecolor{currentstroke}%
\pgfsetdash{}{0pt}%
\pgfsys@defobject{currentmarker}{\pgfqpoint{-0.041667in}{-0.041667in}}{\pgfqpoint{0.041667in}{0.041667in}}{%
\pgfpathmoveto{\pgfqpoint{-0.000000in}{-0.041667in}}%
\pgfpathlineto{\pgfqpoint{0.041667in}{0.041667in}}%
\pgfpathlineto{\pgfqpoint{-0.041667in}{0.041667in}}%
\pgfpathclose%
\pgfusepath{stroke,fill}%
}%
\begin{pgfscope}%
\pgfsys@transformshift{3.433618in}{5.915312in}%
\pgfsys@useobject{currentmarker}{}%
\end{pgfscope}%
\begin{pgfscope}%
\pgfsys@transformshift{3.793080in}{5.776383in}%
\pgfsys@useobject{currentmarker}{}%
\end{pgfscope}%
\begin{pgfscope}%
\pgfsys@transformshift{4.152542in}{5.692370in}%
\pgfsys@useobject{currentmarker}{}%
\end{pgfscope}%
\begin{pgfscope}%
\pgfsys@transformshift{4.512004in}{5.638934in}%
\pgfsys@useobject{currentmarker}{}%
\end{pgfscope}%
\begin{pgfscope}%
\pgfsys@transformshift{4.871466in}{5.606100in}%
\pgfsys@useobject{currentmarker}{}%
\end{pgfscope}%
\end{pgfscope}%
\begin{pgfscope}%
\pgfpathrectangle{\pgfqpoint{3.347347in}{4.908628in}}{\pgfqpoint{1.897959in}{1.372727in}} %
\pgfusepath{clip}%
\pgfsetbuttcap%
\pgfsetroundjoin%
\pgfsetlinewidth{1.505625pt}%
\definecolor{currentstroke}{rgb}{0.000000,0.000000,0.000000}%
\pgfsetstrokecolor{currentstroke}%
\pgfsetdash{{1.500000pt}{2.475000pt}}{0.000000pt}%
\pgfpathmoveto{\pgfqpoint{3.433618in}{6.218958in}}%
\pgfpathlineto{\pgfqpoint{3.505510in}{6.209228in}}%
\pgfpathlineto{\pgfqpoint{3.577403in}{6.199988in}}%
\pgfpathlineto{\pgfqpoint{3.649295in}{6.192033in}}%
\pgfpathlineto{\pgfqpoint{3.721187in}{6.185071in}}%
\pgfpathlineto{\pgfqpoint{3.793080in}{6.178898in}}%
\pgfpathlineto{\pgfqpoint{3.864972in}{6.173367in}}%
\pgfpathlineto{\pgfqpoint{3.936865in}{6.168368in}}%
\pgfpathlineto{\pgfqpoint{4.008757in}{6.163819in}}%
\pgfpathlineto{\pgfqpoint{4.080649in}{6.159658in}}%
\pgfpathlineto{\pgfqpoint{4.152542in}{6.155838in}}%
\pgfpathlineto{\pgfqpoint{4.224434in}{6.152319in}}%
\pgfpathlineto{\pgfqpoint{4.296327in}{6.149073in}}%
\pgfpathlineto{\pgfqpoint{4.368219in}{6.146075in}}%
\pgfpathlineto{\pgfqpoint{4.440111in}{6.143306in}}%
\pgfpathlineto{\pgfqpoint{4.512004in}{6.140750in}}%
\pgfpathlineto{\pgfqpoint{4.583896in}{6.138394in}}%
\pgfpathlineto{\pgfqpoint{4.655788in}{6.136229in}}%
\pgfpathlineto{\pgfqpoint{4.727681in}{6.134244in}}%
\pgfpathlineto{\pgfqpoint{4.799573in}{6.132434in}}%
\pgfpathlineto{\pgfqpoint{4.871466in}{6.130792in}}%
\pgfpathlineto{\pgfqpoint{4.943358in}{6.129313in}}%
\pgfpathlineto{\pgfqpoint{5.015250in}{6.127994in}}%
\pgfpathlineto{\pgfqpoint{5.087143in}{6.126832in}}%
\pgfpathlineto{\pgfqpoint{5.159035in}{6.125824in}}%
\pgfusepath{stroke}%
\end{pgfscope}%
\begin{pgfscope}%
\pgfpathrectangle{\pgfqpoint{3.347347in}{4.908628in}}{\pgfqpoint{1.897959in}{1.372727in}} %
\pgfusepath{clip}%
\pgfsetbuttcap%
\pgfsetroundjoin%
\definecolor{currentfill}{rgb}{0.000000,0.000000,0.000000}%
\pgfsetfillcolor{currentfill}%
\pgfsetlinewidth{1.003750pt}%
\definecolor{currentstroke}{rgb}{0.000000,0.000000,0.000000}%
\pgfsetstrokecolor{currentstroke}%
\pgfsetdash{}{0pt}%
\pgfsys@defobject{currentmarker}{\pgfqpoint{-0.041667in}{-0.041667in}}{\pgfqpoint{0.041667in}{0.041667in}}{%
\pgfpathmoveto{\pgfqpoint{-0.041667in}{0.000000in}}%
\pgfpathlineto{\pgfqpoint{0.041667in}{0.000000in}}%
\pgfpathmoveto{\pgfqpoint{0.000000in}{-0.041667in}}%
\pgfpathlineto{\pgfqpoint{0.000000in}{0.041667in}}%
\pgfusepath{stroke,fill}%
}%
\begin{pgfscope}%
\pgfsys@transformshift{3.433618in}{6.218958in}%
\pgfsys@useobject{currentmarker}{}%
\end{pgfscope}%
\begin{pgfscope}%
\pgfsys@transformshift{3.793080in}{6.178898in}%
\pgfsys@useobject{currentmarker}{}%
\end{pgfscope}%
\begin{pgfscope}%
\pgfsys@transformshift{4.152542in}{6.155838in}%
\pgfsys@useobject{currentmarker}{}%
\end{pgfscope}%
\begin{pgfscope}%
\pgfsys@transformshift{4.512004in}{6.140750in}%
\pgfsys@useobject{currentmarker}{}%
\end{pgfscope}%
\begin{pgfscope}%
\pgfsys@transformshift{4.871466in}{6.130792in}%
\pgfsys@useobject{currentmarker}{}%
\end{pgfscope}%
\end{pgfscope}%
\begin{pgfscope}%
\pgfsetrectcap%
\pgfsetmiterjoin%
\pgfsetlinewidth{0.803000pt}%
\definecolor{currentstroke}{rgb}{0.000000,0.000000,0.000000}%
\pgfsetstrokecolor{currentstroke}%
\pgfsetdash{}{0pt}%
\pgfpathmoveto{\pgfqpoint{3.347347in}{4.908628in}}%
\pgfpathlineto{\pgfqpoint{3.347347in}{6.281355in}}%
\pgfusepath{stroke}%
\end{pgfscope}%
\begin{pgfscope}%
\pgfsetrectcap%
\pgfsetmiterjoin%
\pgfsetlinewidth{0.803000pt}%
\definecolor{currentstroke}{rgb}{0.000000,0.000000,0.000000}%
\pgfsetstrokecolor{currentstroke}%
\pgfsetdash{}{0pt}%
\pgfpathmoveto{\pgfqpoint{5.245306in}{4.908628in}}%
\pgfpathlineto{\pgfqpoint{5.245306in}{6.281355in}}%
\pgfusepath{stroke}%
\end{pgfscope}%
\begin{pgfscope}%
\pgfsetrectcap%
\pgfsetmiterjoin%
\pgfsetlinewidth{0.803000pt}%
\definecolor{currentstroke}{rgb}{0.000000,0.000000,0.000000}%
\pgfsetstrokecolor{currentstroke}%
\pgfsetdash{}{0pt}%
\pgfpathmoveto{\pgfqpoint{3.347347in}{4.908628in}}%
\pgfpathlineto{\pgfqpoint{5.245306in}{4.908628in}}%
\pgfusepath{stroke}%
\end{pgfscope}%
\begin{pgfscope}%
\pgfsetrectcap%
\pgfsetmiterjoin%
\pgfsetlinewidth{0.803000pt}%
\definecolor{currentstroke}{rgb}{0.000000,0.000000,0.000000}%
\pgfsetstrokecolor{currentstroke}%
\pgfsetdash{}{0pt}%
\pgfpathmoveto{\pgfqpoint{3.347347in}{6.281355in}}%
\pgfpathlineto{\pgfqpoint{5.245306in}{6.281355in}}%
\pgfusepath{stroke}%
\end{pgfscope}%
\begin{pgfscope}%
\pgfsetbuttcap%
\pgfsetmiterjoin%
\definecolor{currentfill}{rgb}{1.000000,1.000000,1.000000}%
\pgfsetfillcolor{currentfill}%
\pgfsetlinewidth{0.000000pt}%
\definecolor{currentstroke}{rgb}{0.000000,0.000000,0.000000}%
\pgfsetstrokecolor{currentstroke}%
\pgfsetstrokeopacity{0.000000}%
\pgfsetdash{}{0pt}%
\pgfpathmoveto{\pgfqpoint{5.814694in}{4.908628in}}%
\pgfpathlineto{\pgfqpoint{7.712653in}{4.908628in}}%
\pgfpathlineto{\pgfqpoint{7.712653in}{6.281355in}}%
\pgfpathlineto{\pgfqpoint{5.814694in}{6.281355in}}%
\pgfpathclose%
\pgfusepath{fill}%
\end{pgfscope}%
\begin{pgfscope}%
\pgfsetbuttcap%
\pgfsetroundjoin%
\definecolor{currentfill}{rgb}{0.000000,0.000000,0.000000}%
\pgfsetfillcolor{currentfill}%
\pgfsetlinewidth{0.803000pt}%
\definecolor{currentstroke}{rgb}{0.000000,0.000000,0.000000}%
\pgfsetstrokecolor{currentstroke}%
\pgfsetdash{}{0pt}%
\pgfsys@defobject{currentmarker}{\pgfqpoint{0.000000in}{-0.048611in}}{\pgfqpoint{0.000000in}{0.000000in}}{%
\pgfpathmoveto{\pgfqpoint{0.000000in}{0.000000in}}%
\pgfpathlineto{\pgfqpoint{0.000000in}{-0.048611in}}%
\pgfusepath{stroke,fill}%
}%
\begin{pgfscope}%
\pgfsys@transformshift{6.126019in}{4.908628in}%
\pgfsys@useobject{currentmarker}{}%
\end{pgfscope}%
\end{pgfscope}%
\begin{pgfscope}%
\pgftext[x=6.126019in,y=4.811405in,,top]{\rmfamily\fontsize{10.000000}{12.000000}\selectfont \(\displaystyle 0.1\)}%
\end{pgfscope}%
\begin{pgfscope}%
\pgfsetbuttcap%
\pgfsetroundjoin%
\definecolor{currentfill}{rgb}{0.000000,0.000000,0.000000}%
\pgfsetfillcolor{currentfill}%
\pgfsetlinewidth{0.803000pt}%
\definecolor{currentstroke}{rgb}{0.000000,0.000000,0.000000}%
\pgfsetstrokecolor{currentstroke}%
\pgfsetdash{}{0pt}%
\pgfsys@defobject{currentmarker}{\pgfqpoint{0.000000in}{-0.048611in}}{\pgfqpoint{0.000000in}{0.000000in}}{%
\pgfpathmoveto{\pgfqpoint{0.000000in}{0.000000in}}%
\pgfpathlineto{\pgfqpoint{0.000000in}{-0.048611in}}%
\pgfusepath{stroke,fill}%
}%
\begin{pgfscope}%
\pgfsys@transformshift{7.026237in}{4.908628in}%
\pgfsys@useobject{currentmarker}{}%
\end{pgfscope}%
\end{pgfscope}%
\begin{pgfscope}%
\pgftext[x=7.026237in,y=4.811405in,,top]{\rmfamily\fontsize{10.000000}{12.000000}\selectfont \(\displaystyle 0.2\)}%
\end{pgfscope}%
\begin{pgfscope}%
\pgfsetbuttcap%
\pgfsetroundjoin%
\definecolor{currentfill}{rgb}{0.000000,0.000000,0.000000}%
\pgfsetfillcolor{currentfill}%
\pgfsetlinewidth{0.803000pt}%
\definecolor{currentstroke}{rgb}{0.000000,0.000000,0.000000}%
\pgfsetstrokecolor{currentstroke}%
\pgfsetdash{}{0pt}%
\pgfsys@defobject{currentmarker}{\pgfqpoint{-0.048611in}{0.000000in}}{\pgfqpoint{0.000000in}{0.000000in}}{%
\pgfpathmoveto{\pgfqpoint{0.000000in}{0.000000in}}%
\pgfpathlineto{\pgfqpoint{-0.048611in}{0.000000in}}%
\pgfusepath{stroke,fill}%
}%
\begin{pgfscope}%
\pgfsys@transformshift{5.814694in}{5.051496in}%
\pgfsys@useobject{currentmarker}{}%
\end{pgfscope}%
\end{pgfscope}%
\begin{pgfscope}%
\pgftext[x=5.429469in,y=4.998734in,left,base]{\rmfamily\fontsize{10.000000}{12.000000}\selectfont \(\displaystyle 10^{-7}\)}%
\end{pgfscope}%
\begin{pgfscope}%
\pgfsetbuttcap%
\pgfsetroundjoin%
\definecolor{currentfill}{rgb}{0.000000,0.000000,0.000000}%
\pgfsetfillcolor{currentfill}%
\pgfsetlinewidth{0.803000pt}%
\definecolor{currentstroke}{rgb}{0.000000,0.000000,0.000000}%
\pgfsetstrokecolor{currentstroke}%
\pgfsetdash{}{0pt}%
\pgfsys@defobject{currentmarker}{\pgfqpoint{-0.048611in}{0.000000in}}{\pgfqpoint{0.000000in}{0.000000in}}{%
\pgfpathmoveto{\pgfqpoint{0.000000in}{0.000000in}}%
\pgfpathlineto{\pgfqpoint{-0.048611in}{0.000000in}}%
\pgfusepath{stroke,fill}%
}%
\begin{pgfscope}%
\pgfsys@transformshift{5.814694in}{5.449072in}%
\pgfsys@useobject{currentmarker}{}%
\end{pgfscope}%
\end{pgfscope}%
\begin{pgfscope}%
\pgftext[x=5.429469in,y=5.396311in,left,base]{\rmfamily\fontsize{10.000000}{12.000000}\selectfont \(\displaystyle 10^{-6}\)}%
\end{pgfscope}%
\begin{pgfscope}%
\pgfsetbuttcap%
\pgfsetroundjoin%
\definecolor{currentfill}{rgb}{0.000000,0.000000,0.000000}%
\pgfsetfillcolor{currentfill}%
\pgfsetlinewidth{0.803000pt}%
\definecolor{currentstroke}{rgb}{0.000000,0.000000,0.000000}%
\pgfsetstrokecolor{currentstroke}%
\pgfsetdash{}{0pt}%
\pgfsys@defobject{currentmarker}{\pgfqpoint{-0.048611in}{0.000000in}}{\pgfqpoint{0.000000in}{0.000000in}}{%
\pgfpathmoveto{\pgfqpoint{0.000000in}{0.000000in}}%
\pgfpathlineto{\pgfqpoint{-0.048611in}{0.000000in}}%
\pgfusepath{stroke,fill}%
}%
\begin{pgfscope}%
\pgfsys@transformshift{5.814694in}{5.846649in}%
\pgfsys@useobject{currentmarker}{}%
\end{pgfscope}%
\end{pgfscope}%
\begin{pgfscope}%
\pgftext[x=5.429469in,y=5.793887in,left,base]{\rmfamily\fontsize{10.000000}{12.000000}\selectfont \(\displaystyle 10^{-5}\)}%
\end{pgfscope}%
\begin{pgfscope}%
\pgfsetbuttcap%
\pgfsetroundjoin%
\definecolor{currentfill}{rgb}{0.000000,0.000000,0.000000}%
\pgfsetfillcolor{currentfill}%
\pgfsetlinewidth{0.803000pt}%
\definecolor{currentstroke}{rgb}{0.000000,0.000000,0.000000}%
\pgfsetstrokecolor{currentstroke}%
\pgfsetdash{}{0pt}%
\pgfsys@defobject{currentmarker}{\pgfqpoint{-0.048611in}{0.000000in}}{\pgfqpoint{0.000000in}{0.000000in}}{%
\pgfpathmoveto{\pgfqpoint{0.000000in}{0.000000in}}%
\pgfpathlineto{\pgfqpoint{-0.048611in}{0.000000in}}%
\pgfusepath{stroke,fill}%
}%
\begin{pgfscope}%
\pgfsys@transformshift{5.814694in}{6.244225in}%
\pgfsys@useobject{currentmarker}{}%
\end{pgfscope}%
\end{pgfscope}%
\begin{pgfscope}%
\pgftext[x=5.429469in,y=6.191464in,left,base]{\rmfamily\fontsize{10.000000}{12.000000}\selectfont \(\displaystyle 10^{-4}\)}%
\end{pgfscope}%
\begin{pgfscope}%
\pgfsetbuttcap%
\pgfsetroundjoin%
\definecolor{currentfill}{rgb}{0.000000,0.000000,0.000000}%
\pgfsetfillcolor{currentfill}%
\pgfsetlinewidth{0.602250pt}%
\definecolor{currentstroke}{rgb}{0.000000,0.000000,0.000000}%
\pgfsetstrokecolor{currentstroke}%
\pgfsetdash{}{0pt}%
\pgfsys@defobject{currentmarker}{\pgfqpoint{-0.027778in}{0.000000in}}{\pgfqpoint{0.000000in}{0.000000in}}{%
\pgfpathmoveto{\pgfqpoint{0.000000in}{0.000000in}}%
\pgfpathlineto{\pgfqpoint{-0.027778in}{0.000000in}}%
\pgfusepath{stroke,fill}%
}%
\begin{pgfscope}%
\pgfsys@transformshift{5.814694in}{4.931814in}%
\pgfsys@useobject{currentmarker}{}%
\end{pgfscope}%
\end{pgfscope}%
\begin{pgfscope}%
\pgfsetbuttcap%
\pgfsetroundjoin%
\definecolor{currentfill}{rgb}{0.000000,0.000000,0.000000}%
\pgfsetfillcolor{currentfill}%
\pgfsetlinewidth{0.602250pt}%
\definecolor{currentstroke}{rgb}{0.000000,0.000000,0.000000}%
\pgfsetstrokecolor{currentstroke}%
\pgfsetdash{}{0pt}%
\pgfsys@defobject{currentmarker}{\pgfqpoint{-0.027778in}{0.000000in}}{\pgfqpoint{0.000000in}{0.000000in}}{%
\pgfpathmoveto{\pgfqpoint{0.000000in}{0.000000in}}%
\pgfpathlineto{\pgfqpoint{-0.027778in}{0.000000in}}%
\pgfusepath{stroke,fill}%
}%
\begin{pgfscope}%
\pgfsys@transformshift{5.814694in}{4.963294in}%
\pgfsys@useobject{currentmarker}{}%
\end{pgfscope}%
\end{pgfscope}%
\begin{pgfscope}%
\pgfsetbuttcap%
\pgfsetroundjoin%
\definecolor{currentfill}{rgb}{0.000000,0.000000,0.000000}%
\pgfsetfillcolor{currentfill}%
\pgfsetlinewidth{0.602250pt}%
\definecolor{currentstroke}{rgb}{0.000000,0.000000,0.000000}%
\pgfsetstrokecolor{currentstroke}%
\pgfsetdash{}{0pt}%
\pgfsys@defobject{currentmarker}{\pgfqpoint{-0.027778in}{0.000000in}}{\pgfqpoint{0.000000in}{0.000000in}}{%
\pgfpathmoveto{\pgfqpoint{0.000000in}{0.000000in}}%
\pgfpathlineto{\pgfqpoint{-0.027778in}{0.000000in}}%
\pgfusepath{stroke,fill}%
}%
\begin{pgfscope}%
\pgfsys@transformshift{5.814694in}{4.989911in}%
\pgfsys@useobject{currentmarker}{}%
\end{pgfscope}%
\end{pgfscope}%
\begin{pgfscope}%
\pgfsetbuttcap%
\pgfsetroundjoin%
\definecolor{currentfill}{rgb}{0.000000,0.000000,0.000000}%
\pgfsetfillcolor{currentfill}%
\pgfsetlinewidth{0.602250pt}%
\definecolor{currentstroke}{rgb}{0.000000,0.000000,0.000000}%
\pgfsetstrokecolor{currentstroke}%
\pgfsetdash{}{0pt}%
\pgfsys@defobject{currentmarker}{\pgfqpoint{-0.027778in}{0.000000in}}{\pgfqpoint{0.000000in}{0.000000in}}{%
\pgfpathmoveto{\pgfqpoint{0.000000in}{0.000000in}}%
\pgfpathlineto{\pgfqpoint{-0.027778in}{0.000000in}}%
\pgfusepath{stroke,fill}%
}%
\begin{pgfscope}%
\pgfsys@transformshift{5.814694in}{5.012967in}%
\pgfsys@useobject{currentmarker}{}%
\end{pgfscope}%
\end{pgfscope}%
\begin{pgfscope}%
\pgfsetbuttcap%
\pgfsetroundjoin%
\definecolor{currentfill}{rgb}{0.000000,0.000000,0.000000}%
\pgfsetfillcolor{currentfill}%
\pgfsetlinewidth{0.602250pt}%
\definecolor{currentstroke}{rgb}{0.000000,0.000000,0.000000}%
\pgfsetstrokecolor{currentstroke}%
\pgfsetdash{}{0pt}%
\pgfsys@defobject{currentmarker}{\pgfqpoint{-0.027778in}{0.000000in}}{\pgfqpoint{0.000000in}{0.000000in}}{%
\pgfpathmoveto{\pgfqpoint{0.000000in}{0.000000in}}%
\pgfpathlineto{\pgfqpoint{-0.027778in}{0.000000in}}%
\pgfusepath{stroke,fill}%
}%
\begin{pgfscope}%
\pgfsys@transformshift{5.814694in}{5.033304in}%
\pgfsys@useobject{currentmarker}{}%
\end{pgfscope}%
\end{pgfscope}%
\begin{pgfscope}%
\pgfsetbuttcap%
\pgfsetroundjoin%
\definecolor{currentfill}{rgb}{0.000000,0.000000,0.000000}%
\pgfsetfillcolor{currentfill}%
\pgfsetlinewidth{0.602250pt}%
\definecolor{currentstroke}{rgb}{0.000000,0.000000,0.000000}%
\pgfsetstrokecolor{currentstroke}%
\pgfsetdash{}{0pt}%
\pgfsys@defobject{currentmarker}{\pgfqpoint{-0.027778in}{0.000000in}}{\pgfqpoint{0.000000in}{0.000000in}}{%
\pgfpathmoveto{\pgfqpoint{0.000000in}{0.000000in}}%
\pgfpathlineto{\pgfqpoint{-0.027778in}{0.000000in}}%
\pgfusepath{stroke,fill}%
}%
\begin{pgfscope}%
\pgfsys@transformshift{5.814694in}{5.171178in}%
\pgfsys@useobject{currentmarker}{}%
\end{pgfscope}%
\end{pgfscope}%
\begin{pgfscope}%
\pgfsetbuttcap%
\pgfsetroundjoin%
\definecolor{currentfill}{rgb}{0.000000,0.000000,0.000000}%
\pgfsetfillcolor{currentfill}%
\pgfsetlinewidth{0.602250pt}%
\definecolor{currentstroke}{rgb}{0.000000,0.000000,0.000000}%
\pgfsetstrokecolor{currentstroke}%
\pgfsetdash{}{0pt}%
\pgfsys@defobject{currentmarker}{\pgfqpoint{-0.027778in}{0.000000in}}{\pgfqpoint{0.000000in}{0.000000in}}{%
\pgfpathmoveto{\pgfqpoint{0.000000in}{0.000000in}}%
\pgfpathlineto{\pgfqpoint{-0.027778in}{0.000000in}}%
\pgfusepath{stroke,fill}%
}%
\begin{pgfscope}%
\pgfsys@transformshift{5.814694in}{5.241188in}%
\pgfsys@useobject{currentmarker}{}%
\end{pgfscope}%
\end{pgfscope}%
\begin{pgfscope}%
\pgfsetbuttcap%
\pgfsetroundjoin%
\definecolor{currentfill}{rgb}{0.000000,0.000000,0.000000}%
\pgfsetfillcolor{currentfill}%
\pgfsetlinewidth{0.602250pt}%
\definecolor{currentstroke}{rgb}{0.000000,0.000000,0.000000}%
\pgfsetstrokecolor{currentstroke}%
\pgfsetdash{}{0pt}%
\pgfsys@defobject{currentmarker}{\pgfqpoint{-0.027778in}{0.000000in}}{\pgfqpoint{0.000000in}{0.000000in}}{%
\pgfpathmoveto{\pgfqpoint{0.000000in}{0.000000in}}%
\pgfpathlineto{\pgfqpoint{-0.027778in}{0.000000in}}%
\pgfusepath{stroke,fill}%
}%
\begin{pgfscope}%
\pgfsys@transformshift{5.814694in}{5.290861in}%
\pgfsys@useobject{currentmarker}{}%
\end{pgfscope}%
\end{pgfscope}%
\begin{pgfscope}%
\pgfsetbuttcap%
\pgfsetroundjoin%
\definecolor{currentfill}{rgb}{0.000000,0.000000,0.000000}%
\pgfsetfillcolor{currentfill}%
\pgfsetlinewidth{0.602250pt}%
\definecolor{currentstroke}{rgb}{0.000000,0.000000,0.000000}%
\pgfsetstrokecolor{currentstroke}%
\pgfsetdash{}{0pt}%
\pgfsys@defobject{currentmarker}{\pgfqpoint{-0.027778in}{0.000000in}}{\pgfqpoint{0.000000in}{0.000000in}}{%
\pgfpathmoveto{\pgfqpoint{0.000000in}{0.000000in}}%
\pgfpathlineto{\pgfqpoint{-0.027778in}{0.000000in}}%
\pgfusepath{stroke,fill}%
}%
\begin{pgfscope}%
\pgfsys@transformshift{5.814694in}{5.329390in}%
\pgfsys@useobject{currentmarker}{}%
\end{pgfscope}%
\end{pgfscope}%
\begin{pgfscope}%
\pgfsetbuttcap%
\pgfsetroundjoin%
\definecolor{currentfill}{rgb}{0.000000,0.000000,0.000000}%
\pgfsetfillcolor{currentfill}%
\pgfsetlinewidth{0.602250pt}%
\definecolor{currentstroke}{rgb}{0.000000,0.000000,0.000000}%
\pgfsetstrokecolor{currentstroke}%
\pgfsetdash{}{0pt}%
\pgfsys@defobject{currentmarker}{\pgfqpoint{-0.027778in}{0.000000in}}{\pgfqpoint{0.000000in}{0.000000in}}{%
\pgfpathmoveto{\pgfqpoint{0.000000in}{0.000000in}}%
\pgfpathlineto{\pgfqpoint{-0.027778in}{0.000000in}}%
\pgfusepath{stroke,fill}%
}%
\begin{pgfscope}%
\pgfsys@transformshift{5.814694in}{5.360871in}%
\pgfsys@useobject{currentmarker}{}%
\end{pgfscope}%
\end{pgfscope}%
\begin{pgfscope}%
\pgfsetbuttcap%
\pgfsetroundjoin%
\definecolor{currentfill}{rgb}{0.000000,0.000000,0.000000}%
\pgfsetfillcolor{currentfill}%
\pgfsetlinewidth{0.602250pt}%
\definecolor{currentstroke}{rgb}{0.000000,0.000000,0.000000}%
\pgfsetstrokecolor{currentstroke}%
\pgfsetdash{}{0pt}%
\pgfsys@defobject{currentmarker}{\pgfqpoint{-0.027778in}{0.000000in}}{\pgfqpoint{0.000000in}{0.000000in}}{%
\pgfpathmoveto{\pgfqpoint{0.000000in}{0.000000in}}%
\pgfpathlineto{\pgfqpoint{-0.027778in}{0.000000in}}%
\pgfusepath{stroke,fill}%
}%
\begin{pgfscope}%
\pgfsys@transformshift{5.814694in}{5.387487in}%
\pgfsys@useobject{currentmarker}{}%
\end{pgfscope}%
\end{pgfscope}%
\begin{pgfscope}%
\pgfsetbuttcap%
\pgfsetroundjoin%
\definecolor{currentfill}{rgb}{0.000000,0.000000,0.000000}%
\pgfsetfillcolor{currentfill}%
\pgfsetlinewidth{0.602250pt}%
\definecolor{currentstroke}{rgb}{0.000000,0.000000,0.000000}%
\pgfsetstrokecolor{currentstroke}%
\pgfsetdash{}{0pt}%
\pgfsys@defobject{currentmarker}{\pgfqpoint{-0.027778in}{0.000000in}}{\pgfqpoint{0.000000in}{0.000000in}}{%
\pgfpathmoveto{\pgfqpoint{0.000000in}{0.000000in}}%
\pgfpathlineto{\pgfqpoint{-0.027778in}{0.000000in}}%
\pgfusepath{stroke,fill}%
}%
\begin{pgfscope}%
\pgfsys@transformshift{5.814694in}{5.410543in}%
\pgfsys@useobject{currentmarker}{}%
\end{pgfscope}%
\end{pgfscope}%
\begin{pgfscope}%
\pgfsetbuttcap%
\pgfsetroundjoin%
\definecolor{currentfill}{rgb}{0.000000,0.000000,0.000000}%
\pgfsetfillcolor{currentfill}%
\pgfsetlinewidth{0.602250pt}%
\definecolor{currentstroke}{rgb}{0.000000,0.000000,0.000000}%
\pgfsetstrokecolor{currentstroke}%
\pgfsetdash{}{0pt}%
\pgfsys@defobject{currentmarker}{\pgfqpoint{-0.027778in}{0.000000in}}{\pgfqpoint{0.000000in}{0.000000in}}{%
\pgfpathmoveto{\pgfqpoint{0.000000in}{0.000000in}}%
\pgfpathlineto{\pgfqpoint{-0.027778in}{0.000000in}}%
\pgfusepath{stroke,fill}%
}%
\begin{pgfscope}%
\pgfsys@transformshift{5.814694in}{5.430880in}%
\pgfsys@useobject{currentmarker}{}%
\end{pgfscope}%
\end{pgfscope}%
\begin{pgfscope}%
\pgfsetbuttcap%
\pgfsetroundjoin%
\definecolor{currentfill}{rgb}{0.000000,0.000000,0.000000}%
\pgfsetfillcolor{currentfill}%
\pgfsetlinewidth{0.602250pt}%
\definecolor{currentstroke}{rgb}{0.000000,0.000000,0.000000}%
\pgfsetstrokecolor{currentstroke}%
\pgfsetdash{}{0pt}%
\pgfsys@defobject{currentmarker}{\pgfqpoint{-0.027778in}{0.000000in}}{\pgfqpoint{0.000000in}{0.000000in}}{%
\pgfpathmoveto{\pgfqpoint{0.000000in}{0.000000in}}%
\pgfpathlineto{\pgfqpoint{-0.027778in}{0.000000in}}%
\pgfusepath{stroke,fill}%
}%
\begin{pgfscope}%
\pgfsys@transformshift{5.814694in}{5.568755in}%
\pgfsys@useobject{currentmarker}{}%
\end{pgfscope}%
\end{pgfscope}%
\begin{pgfscope}%
\pgfsetbuttcap%
\pgfsetroundjoin%
\definecolor{currentfill}{rgb}{0.000000,0.000000,0.000000}%
\pgfsetfillcolor{currentfill}%
\pgfsetlinewidth{0.602250pt}%
\definecolor{currentstroke}{rgb}{0.000000,0.000000,0.000000}%
\pgfsetstrokecolor{currentstroke}%
\pgfsetdash{}{0pt}%
\pgfsys@defobject{currentmarker}{\pgfqpoint{-0.027778in}{0.000000in}}{\pgfqpoint{0.000000in}{0.000000in}}{%
\pgfpathmoveto{\pgfqpoint{0.000000in}{0.000000in}}%
\pgfpathlineto{\pgfqpoint{-0.027778in}{0.000000in}}%
\pgfusepath{stroke,fill}%
}%
\begin{pgfscope}%
\pgfsys@transformshift{5.814694in}{5.638765in}%
\pgfsys@useobject{currentmarker}{}%
\end{pgfscope}%
\end{pgfscope}%
\begin{pgfscope}%
\pgfsetbuttcap%
\pgfsetroundjoin%
\definecolor{currentfill}{rgb}{0.000000,0.000000,0.000000}%
\pgfsetfillcolor{currentfill}%
\pgfsetlinewidth{0.602250pt}%
\definecolor{currentstroke}{rgb}{0.000000,0.000000,0.000000}%
\pgfsetstrokecolor{currentstroke}%
\pgfsetdash{}{0pt}%
\pgfsys@defobject{currentmarker}{\pgfqpoint{-0.027778in}{0.000000in}}{\pgfqpoint{0.000000in}{0.000000in}}{%
\pgfpathmoveto{\pgfqpoint{0.000000in}{0.000000in}}%
\pgfpathlineto{\pgfqpoint{-0.027778in}{0.000000in}}%
\pgfusepath{stroke,fill}%
}%
\begin{pgfscope}%
\pgfsys@transformshift{5.814694in}{5.688437in}%
\pgfsys@useobject{currentmarker}{}%
\end{pgfscope}%
\end{pgfscope}%
\begin{pgfscope}%
\pgfsetbuttcap%
\pgfsetroundjoin%
\definecolor{currentfill}{rgb}{0.000000,0.000000,0.000000}%
\pgfsetfillcolor{currentfill}%
\pgfsetlinewidth{0.602250pt}%
\definecolor{currentstroke}{rgb}{0.000000,0.000000,0.000000}%
\pgfsetstrokecolor{currentstroke}%
\pgfsetdash{}{0pt}%
\pgfsys@defobject{currentmarker}{\pgfqpoint{-0.027778in}{0.000000in}}{\pgfqpoint{0.000000in}{0.000000in}}{%
\pgfpathmoveto{\pgfqpoint{0.000000in}{0.000000in}}%
\pgfpathlineto{\pgfqpoint{-0.027778in}{0.000000in}}%
\pgfusepath{stroke,fill}%
}%
\begin{pgfscope}%
\pgfsys@transformshift{5.814694in}{5.726966in}%
\pgfsys@useobject{currentmarker}{}%
\end{pgfscope}%
\end{pgfscope}%
\begin{pgfscope}%
\pgfsetbuttcap%
\pgfsetroundjoin%
\definecolor{currentfill}{rgb}{0.000000,0.000000,0.000000}%
\pgfsetfillcolor{currentfill}%
\pgfsetlinewidth{0.602250pt}%
\definecolor{currentstroke}{rgb}{0.000000,0.000000,0.000000}%
\pgfsetstrokecolor{currentstroke}%
\pgfsetdash{}{0pt}%
\pgfsys@defobject{currentmarker}{\pgfqpoint{-0.027778in}{0.000000in}}{\pgfqpoint{0.000000in}{0.000000in}}{%
\pgfpathmoveto{\pgfqpoint{0.000000in}{0.000000in}}%
\pgfpathlineto{\pgfqpoint{-0.027778in}{0.000000in}}%
\pgfusepath{stroke,fill}%
}%
\begin{pgfscope}%
\pgfsys@transformshift{5.814694in}{5.758447in}%
\pgfsys@useobject{currentmarker}{}%
\end{pgfscope}%
\end{pgfscope}%
\begin{pgfscope}%
\pgfsetbuttcap%
\pgfsetroundjoin%
\definecolor{currentfill}{rgb}{0.000000,0.000000,0.000000}%
\pgfsetfillcolor{currentfill}%
\pgfsetlinewidth{0.602250pt}%
\definecolor{currentstroke}{rgb}{0.000000,0.000000,0.000000}%
\pgfsetstrokecolor{currentstroke}%
\pgfsetdash{}{0pt}%
\pgfsys@defobject{currentmarker}{\pgfqpoint{-0.027778in}{0.000000in}}{\pgfqpoint{0.000000in}{0.000000in}}{%
\pgfpathmoveto{\pgfqpoint{0.000000in}{0.000000in}}%
\pgfpathlineto{\pgfqpoint{-0.027778in}{0.000000in}}%
\pgfusepath{stroke,fill}%
}%
\begin{pgfscope}%
\pgfsys@transformshift{5.814694in}{5.785063in}%
\pgfsys@useobject{currentmarker}{}%
\end{pgfscope}%
\end{pgfscope}%
\begin{pgfscope}%
\pgfsetbuttcap%
\pgfsetroundjoin%
\definecolor{currentfill}{rgb}{0.000000,0.000000,0.000000}%
\pgfsetfillcolor{currentfill}%
\pgfsetlinewidth{0.602250pt}%
\definecolor{currentstroke}{rgb}{0.000000,0.000000,0.000000}%
\pgfsetstrokecolor{currentstroke}%
\pgfsetdash{}{0pt}%
\pgfsys@defobject{currentmarker}{\pgfqpoint{-0.027778in}{0.000000in}}{\pgfqpoint{0.000000in}{0.000000in}}{%
\pgfpathmoveto{\pgfqpoint{0.000000in}{0.000000in}}%
\pgfpathlineto{\pgfqpoint{-0.027778in}{0.000000in}}%
\pgfusepath{stroke,fill}%
}%
\begin{pgfscope}%
\pgfsys@transformshift{5.814694in}{5.808120in}%
\pgfsys@useobject{currentmarker}{}%
\end{pgfscope}%
\end{pgfscope}%
\begin{pgfscope}%
\pgfsetbuttcap%
\pgfsetroundjoin%
\definecolor{currentfill}{rgb}{0.000000,0.000000,0.000000}%
\pgfsetfillcolor{currentfill}%
\pgfsetlinewidth{0.602250pt}%
\definecolor{currentstroke}{rgb}{0.000000,0.000000,0.000000}%
\pgfsetstrokecolor{currentstroke}%
\pgfsetdash{}{0pt}%
\pgfsys@defobject{currentmarker}{\pgfqpoint{-0.027778in}{0.000000in}}{\pgfqpoint{0.000000in}{0.000000in}}{%
\pgfpathmoveto{\pgfqpoint{0.000000in}{0.000000in}}%
\pgfpathlineto{\pgfqpoint{-0.027778in}{0.000000in}}%
\pgfusepath{stroke,fill}%
}%
\begin{pgfscope}%
\pgfsys@transformshift{5.814694in}{5.828457in}%
\pgfsys@useobject{currentmarker}{}%
\end{pgfscope}%
\end{pgfscope}%
\begin{pgfscope}%
\pgfsetbuttcap%
\pgfsetroundjoin%
\definecolor{currentfill}{rgb}{0.000000,0.000000,0.000000}%
\pgfsetfillcolor{currentfill}%
\pgfsetlinewidth{0.602250pt}%
\definecolor{currentstroke}{rgb}{0.000000,0.000000,0.000000}%
\pgfsetstrokecolor{currentstroke}%
\pgfsetdash{}{0pt}%
\pgfsys@defobject{currentmarker}{\pgfqpoint{-0.027778in}{0.000000in}}{\pgfqpoint{0.000000in}{0.000000in}}{%
\pgfpathmoveto{\pgfqpoint{0.000000in}{0.000000in}}%
\pgfpathlineto{\pgfqpoint{-0.027778in}{0.000000in}}%
\pgfusepath{stroke,fill}%
}%
\begin{pgfscope}%
\pgfsys@transformshift{5.814694in}{5.966331in}%
\pgfsys@useobject{currentmarker}{}%
\end{pgfscope}%
\end{pgfscope}%
\begin{pgfscope}%
\pgfsetbuttcap%
\pgfsetroundjoin%
\definecolor{currentfill}{rgb}{0.000000,0.000000,0.000000}%
\pgfsetfillcolor{currentfill}%
\pgfsetlinewidth{0.602250pt}%
\definecolor{currentstroke}{rgb}{0.000000,0.000000,0.000000}%
\pgfsetstrokecolor{currentstroke}%
\pgfsetdash{}{0pt}%
\pgfsys@defobject{currentmarker}{\pgfqpoint{-0.027778in}{0.000000in}}{\pgfqpoint{0.000000in}{0.000000in}}{%
\pgfpathmoveto{\pgfqpoint{0.000000in}{0.000000in}}%
\pgfpathlineto{\pgfqpoint{-0.027778in}{0.000000in}}%
\pgfusepath{stroke,fill}%
}%
\begin{pgfscope}%
\pgfsys@transformshift{5.814694in}{6.036341in}%
\pgfsys@useobject{currentmarker}{}%
\end{pgfscope}%
\end{pgfscope}%
\begin{pgfscope}%
\pgfsetbuttcap%
\pgfsetroundjoin%
\definecolor{currentfill}{rgb}{0.000000,0.000000,0.000000}%
\pgfsetfillcolor{currentfill}%
\pgfsetlinewidth{0.602250pt}%
\definecolor{currentstroke}{rgb}{0.000000,0.000000,0.000000}%
\pgfsetstrokecolor{currentstroke}%
\pgfsetdash{}{0pt}%
\pgfsys@defobject{currentmarker}{\pgfqpoint{-0.027778in}{0.000000in}}{\pgfqpoint{0.000000in}{0.000000in}}{%
\pgfpathmoveto{\pgfqpoint{0.000000in}{0.000000in}}%
\pgfpathlineto{\pgfqpoint{-0.027778in}{0.000000in}}%
\pgfusepath{stroke,fill}%
}%
\begin{pgfscope}%
\pgfsys@transformshift{5.814694in}{6.086014in}%
\pgfsys@useobject{currentmarker}{}%
\end{pgfscope}%
\end{pgfscope}%
\begin{pgfscope}%
\pgfsetbuttcap%
\pgfsetroundjoin%
\definecolor{currentfill}{rgb}{0.000000,0.000000,0.000000}%
\pgfsetfillcolor{currentfill}%
\pgfsetlinewidth{0.602250pt}%
\definecolor{currentstroke}{rgb}{0.000000,0.000000,0.000000}%
\pgfsetstrokecolor{currentstroke}%
\pgfsetdash{}{0pt}%
\pgfsys@defobject{currentmarker}{\pgfqpoint{-0.027778in}{0.000000in}}{\pgfqpoint{0.000000in}{0.000000in}}{%
\pgfpathmoveto{\pgfqpoint{0.000000in}{0.000000in}}%
\pgfpathlineto{\pgfqpoint{-0.027778in}{0.000000in}}%
\pgfusepath{stroke,fill}%
}%
\begin{pgfscope}%
\pgfsys@transformshift{5.814694in}{6.124543in}%
\pgfsys@useobject{currentmarker}{}%
\end{pgfscope}%
\end{pgfscope}%
\begin{pgfscope}%
\pgfsetbuttcap%
\pgfsetroundjoin%
\definecolor{currentfill}{rgb}{0.000000,0.000000,0.000000}%
\pgfsetfillcolor{currentfill}%
\pgfsetlinewidth{0.602250pt}%
\definecolor{currentstroke}{rgb}{0.000000,0.000000,0.000000}%
\pgfsetstrokecolor{currentstroke}%
\pgfsetdash{}{0pt}%
\pgfsys@defobject{currentmarker}{\pgfqpoint{-0.027778in}{0.000000in}}{\pgfqpoint{0.000000in}{0.000000in}}{%
\pgfpathmoveto{\pgfqpoint{0.000000in}{0.000000in}}%
\pgfpathlineto{\pgfqpoint{-0.027778in}{0.000000in}}%
\pgfusepath{stroke,fill}%
}%
\begin{pgfscope}%
\pgfsys@transformshift{5.814694in}{6.156023in}%
\pgfsys@useobject{currentmarker}{}%
\end{pgfscope}%
\end{pgfscope}%
\begin{pgfscope}%
\pgfsetbuttcap%
\pgfsetroundjoin%
\definecolor{currentfill}{rgb}{0.000000,0.000000,0.000000}%
\pgfsetfillcolor{currentfill}%
\pgfsetlinewidth{0.602250pt}%
\definecolor{currentstroke}{rgb}{0.000000,0.000000,0.000000}%
\pgfsetstrokecolor{currentstroke}%
\pgfsetdash{}{0pt}%
\pgfsys@defobject{currentmarker}{\pgfqpoint{-0.027778in}{0.000000in}}{\pgfqpoint{0.000000in}{0.000000in}}{%
\pgfpathmoveto{\pgfqpoint{0.000000in}{0.000000in}}%
\pgfpathlineto{\pgfqpoint{-0.027778in}{0.000000in}}%
\pgfusepath{stroke,fill}%
}%
\begin{pgfscope}%
\pgfsys@transformshift{5.814694in}{6.182640in}%
\pgfsys@useobject{currentmarker}{}%
\end{pgfscope}%
\end{pgfscope}%
\begin{pgfscope}%
\pgfsetbuttcap%
\pgfsetroundjoin%
\definecolor{currentfill}{rgb}{0.000000,0.000000,0.000000}%
\pgfsetfillcolor{currentfill}%
\pgfsetlinewidth{0.602250pt}%
\definecolor{currentstroke}{rgb}{0.000000,0.000000,0.000000}%
\pgfsetstrokecolor{currentstroke}%
\pgfsetdash{}{0pt}%
\pgfsys@defobject{currentmarker}{\pgfqpoint{-0.027778in}{0.000000in}}{\pgfqpoint{0.000000in}{0.000000in}}{%
\pgfpathmoveto{\pgfqpoint{0.000000in}{0.000000in}}%
\pgfpathlineto{\pgfqpoint{-0.027778in}{0.000000in}}%
\pgfusepath{stroke,fill}%
}%
\begin{pgfscope}%
\pgfsys@transformshift{5.814694in}{6.205696in}%
\pgfsys@useobject{currentmarker}{}%
\end{pgfscope}%
\end{pgfscope}%
\begin{pgfscope}%
\pgfsetbuttcap%
\pgfsetroundjoin%
\definecolor{currentfill}{rgb}{0.000000,0.000000,0.000000}%
\pgfsetfillcolor{currentfill}%
\pgfsetlinewidth{0.602250pt}%
\definecolor{currentstroke}{rgb}{0.000000,0.000000,0.000000}%
\pgfsetstrokecolor{currentstroke}%
\pgfsetdash{}{0pt}%
\pgfsys@defobject{currentmarker}{\pgfqpoint{-0.027778in}{0.000000in}}{\pgfqpoint{0.000000in}{0.000000in}}{%
\pgfpathmoveto{\pgfqpoint{0.000000in}{0.000000in}}%
\pgfpathlineto{\pgfqpoint{-0.027778in}{0.000000in}}%
\pgfusepath{stroke,fill}%
}%
\begin{pgfscope}%
\pgfsys@transformshift{5.814694in}{6.226033in}%
\pgfsys@useobject{currentmarker}{}%
\end{pgfscope}%
\end{pgfscope}%
\begin{pgfscope}%
\pgfpathrectangle{\pgfqpoint{5.814694in}{4.908628in}}{\pgfqpoint{1.897959in}{1.372727in}} %
\pgfusepath{clip}%
\pgfsetbuttcap%
\pgfsetroundjoin%
\pgfsetlinewidth{1.505625pt}%
\definecolor{currentstroke}{rgb}{1.000000,0.000000,0.000000}%
\pgfsetstrokecolor{currentstroke}%
\pgfsetdash{{5.550000pt}{2.400000pt}}{0.000000pt}%
\pgfpathmoveto{\pgfqpoint{5.900965in}{6.125484in}}%
\pgfpathlineto{\pgfqpoint{5.975983in}{6.122537in}}%
\pgfpathlineto{\pgfqpoint{6.051001in}{6.119723in}}%
\pgfpathlineto{\pgfqpoint{6.126019in}{6.117037in}}%
\pgfpathlineto{\pgfqpoint{6.201037in}{6.114477in}}%
\pgfpathlineto{\pgfqpoint{6.276055in}{6.112038in}}%
\pgfpathlineto{\pgfqpoint{6.351074in}{6.109718in}}%
\pgfpathlineto{\pgfqpoint{6.426092in}{6.107516in}}%
\pgfpathlineto{\pgfqpoint{6.501110in}{6.105429in}}%
\pgfpathlineto{\pgfqpoint{6.576128in}{6.103456in}}%
\pgfpathlineto{\pgfqpoint{6.651146in}{6.101596in}}%
\pgfpathlineto{\pgfqpoint{6.726164in}{6.099848in}}%
\pgfpathlineto{\pgfqpoint{6.801183in}{6.098210in}}%
\pgfpathlineto{\pgfqpoint{6.876201in}{6.096681in}}%
\pgfpathlineto{\pgfqpoint{6.951219in}{6.095261in}}%
\pgfpathlineto{\pgfqpoint{7.026237in}{6.093948in}}%
\pgfpathlineto{\pgfqpoint{7.101255in}{6.092743in}}%
\pgfpathlineto{\pgfqpoint{7.176273in}{6.091643in}}%
\pgfpathlineto{\pgfqpoint{7.251291in}{6.090649in}}%
\pgfpathlineto{\pgfqpoint{7.326310in}{6.089760in}}%
\pgfpathlineto{\pgfqpoint{7.401328in}{6.088975in}}%
\pgfpathlineto{\pgfqpoint{7.476346in}{6.088294in}}%
\pgfpathlineto{\pgfqpoint{7.551364in}{6.087717in}}%
\pgfpathlineto{\pgfqpoint{7.626382in}{6.087243in}}%
\pgfusepath{stroke}%
\end{pgfscope}%
\begin{pgfscope}%
\pgfpathrectangle{\pgfqpoint{5.814694in}{4.908628in}}{\pgfqpoint{1.897959in}{1.372727in}} %
\pgfusepath{clip}%
\pgfsetbuttcap%
\pgfsetmiterjoin%
\definecolor{currentfill}{rgb}{1.000000,0.000000,0.000000}%
\pgfsetfillcolor{currentfill}%
\pgfsetlinewidth{1.003750pt}%
\definecolor{currentstroke}{rgb}{1.000000,0.000000,0.000000}%
\pgfsetstrokecolor{currentstroke}%
\pgfsetdash{}{0pt}%
\pgfsys@defobject{currentmarker}{\pgfqpoint{-0.041667in}{-0.041667in}}{\pgfqpoint{0.041667in}{0.041667in}}{%
\pgfpathmoveto{\pgfqpoint{-0.041667in}{-0.041667in}}%
\pgfpathlineto{\pgfqpoint{0.041667in}{-0.041667in}}%
\pgfpathlineto{\pgfqpoint{0.041667in}{0.041667in}}%
\pgfpathlineto{\pgfqpoint{-0.041667in}{0.041667in}}%
\pgfpathclose%
\pgfusepath{stroke,fill}%
}%
\begin{pgfscope}%
\pgfsys@transformshift{5.900965in}{6.125484in}%
\pgfsys@useobject{currentmarker}{}%
\end{pgfscope}%
\begin{pgfscope}%
\pgfsys@transformshift{6.276055in}{6.112038in}%
\pgfsys@useobject{currentmarker}{}%
\end{pgfscope}%
\begin{pgfscope}%
\pgfsys@transformshift{6.651146in}{6.101596in}%
\pgfsys@useobject{currentmarker}{}%
\end{pgfscope}%
\begin{pgfscope}%
\pgfsys@transformshift{7.026237in}{6.093948in}%
\pgfsys@useobject{currentmarker}{}%
\end{pgfscope}%
\begin{pgfscope}%
\pgfsys@transformshift{7.401328in}{6.088975in}%
\pgfsys@useobject{currentmarker}{}%
\end{pgfscope}%
\end{pgfscope}%
\begin{pgfscope}%
\pgfpathrectangle{\pgfqpoint{5.814694in}{4.908628in}}{\pgfqpoint{1.897959in}{1.372727in}} %
\pgfusepath{clip}%
\pgfsetrectcap%
\pgfsetroundjoin%
\pgfsetlinewidth{1.505625pt}%
\definecolor{currentstroke}{rgb}{0.000000,0.000000,1.000000}%
\pgfsetstrokecolor{currentstroke}%
\pgfsetdash{}{0pt}%
\pgfpathmoveto{\pgfqpoint{5.900965in}{5.420144in}}%
\pgfpathlineto{\pgfqpoint{5.975983in}{5.407002in}}%
\pgfpathlineto{\pgfqpoint{6.051001in}{5.394089in}}%
\pgfpathlineto{\pgfqpoint{6.126019in}{5.381273in}}%
\pgfpathlineto{\pgfqpoint{6.201037in}{5.368448in}}%
\pgfpathlineto{\pgfqpoint{6.276055in}{5.355526in}}%
\pgfpathlineto{\pgfqpoint{6.351074in}{5.342430in}}%
\pgfpathlineto{\pgfqpoint{6.426092in}{5.329086in}}%
\pgfpathlineto{\pgfqpoint{6.501110in}{5.315421in}}%
\pgfpathlineto{\pgfqpoint{6.576128in}{5.301362in}}%
\pgfpathlineto{\pgfqpoint{6.651146in}{5.286829in}}%
\pgfpathlineto{\pgfqpoint{6.726164in}{5.271735in}}%
\pgfpathlineto{\pgfqpoint{6.801183in}{5.255981in}}%
\pgfpathlineto{\pgfqpoint{6.876201in}{5.239453in}}%
\pgfpathlineto{\pgfqpoint{6.951219in}{5.222016in}}%
\pgfpathlineto{\pgfqpoint{7.026237in}{5.203506in}}%
\pgfpathlineto{\pgfqpoint{7.101255in}{5.183721in}}%
\pgfpathlineto{\pgfqpoint{7.176273in}{5.162406in}}%
\pgfpathlineto{\pgfqpoint{7.251291in}{5.139227in}}%
\pgfpathlineto{\pgfqpoint{7.326310in}{5.113742in}}%
\pgfpathlineto{\pgfqpoint{7.401328in}{5.085338in}}%
\pgfpathlineto{\pgfqpoint{7.476346in}{5.053129in}}%
\pgfpathlineto{\pgfqpoint{7.551364in}{5.015764in}}%
\pgfpathlineto{\pgfqpoint{7.626382in}{4.971024in}}%
\pgfusepath{stroke}%
\end{pgfscope}%
\begin{pgfscope}%
\pgfpathrectangle{\pgfqpoint{5.814694in}{4.908628in}}{\pgfqpoint{1.897959in}{1.372727in}} %
\pgfusepath{clip}%
\pgfsetbuttcap%
\pgfsetroundjoin%
\definecolor{currentfill}{rgb}{0.000000,0.000000,1.000000}%
\pgfsetfillcolor{currentfill}%
\pgfsetlinewidth{1.003750pt}%
\definecolor{currentstroke}{rgb}{0.000000,0.000000,1.000000}%
\pgfsetstrokecolor{currentstroke}%
\pgfsetdash{}{0pt}%
\pgfsys@defobject{currentmarker}{\pgfqpoint{-0.041667in}{-0.041667in}}{\pgfqpoint{0.041667in}{0.041667in}}{%
\pgfpathmoveto{\pgfqpoint{0.000000in}{-0.041667in}}%
\pgfpathcurveto{\pgfqpoint{0.011050in}{-0.041667in}}{\pgfqpoint{0.021649in}{-0.037276in}}{\pgfqpoint{0.029463in}{-0.029463in}}%
\pgfpathcurveto{\pgfqpoint{0.037276in}{-0.021649in}}{\pgfqpoint{0.041667in}{-0.011050in}}{\pgfqpoint{0.041667in}{0.000000in}}%
\pgfpathcurveto{\pgfqpoint{0.041667in}{0.011050in}}{\pgfqpoint{0.037276in}{0.021649in}}{\pgfqpoint{0.029463in}{0.029463in}}%
\pgfpathcurveto{\pgfqpoint{0.021649in}{0.037276in}}{\pgfqpoint{0.011050in}{0.041667in}}{\pgfqpoint{0.000000in}{0.041667in}}%
\pgfpathcurveto{\pgfqpoint{-0.011050in}{0.041667in}}{\pgfqpoint{-0.021649in}{0.037276in}}{\pgfqpoint{-0.029463in}{0.029463in}}%
\pgfpathcurveto{\pgfqpoint{-0.037276in}{0.021649in}}{\pgfqpoint{-0.041667in}{0.011050in}}{\pgfqpoint{-0.041667in}{0.000000in}}%
\pgfpathcurveto{\pgfqpoint{-0.041667in}{-0.011050in}}{\pgfqpoint{-0.037276in}{-0.021649in}}{\pgfqpoint{-0.029463in}{-0.029463in}}%
\pgfpathcurveto{\pgfqpoint{-0.021649in}{-0.037276in}}{\pgfqpoint{-0.011050in}{-0.041667in}}{\pgfqpoint{0.000000in}{-0.041667in}}%
\pgfpathclose%
\pgfusepath{stroke,fill}%
}%
\begin{pgfscope}%
\pgfsys@transformshift{5.900965in}{5.420144in}%
\pgfsys@useobject{currentmarker}{}%
\end{pgfscope}%
\begin{pgfscope}%
\pgfsys@transformshift{6.276055in}{5.355526in}%
\pgfsys@useobject{currentmarker}{}%
\end{pgfscope}%
\begin{pgfscope}%
\pgfsys@transformshift{6.651146in}{5.286829in}%
\pgfsys@useobject{currentmarker}{}%
\end{pgfscope}%
\begin{pgfscope}%
\pgfsys@transformshift{7.026237in}{5.203506in}%
\pgfsys@useobject{currentmarker}{}%
\end{pgfscope}%
\begin{pgfscope}%
\pgfsys@transformshift{7.401328in}{5.085338in}%
\pgfsys@useobject{currentmarker}{}%
\end{pgfscope}%
\end{pgfscope}%
\begin{pgfscope}%
\pgfpathrectangle{\pgfqpoint{5.814694in}{4.908628in}}{\pgfqpoint{1.897959in}{1.372727in}} %
\pgfusepath{clip}%
\pgfsetbuttcap%
\pgfsetroundjoin%
\pgfsetlinewidth{1.505625pt}%
\definecolor{currentstroke}{rgb}{0.000000,0.750000,0.750000}%
\pgfsetstrokecolor{currentstroke}%
\pgfsetdash{{9.600000pt}{2.400000pt}{1.500000pt}{2.400000pt}}{0.000000pt}%
\pgfpathmoveto{\pgfqpoint{5.900965in}{6.066668in}}%
\pgfpathlineto{\pgfqpoint{5.975983in}{6.039105in}}%
\pgfpathlineto{\pgfqpoint{6.051001in}{6.014896in}}%
\pgfpathlineto{\pgfqpoint{6.126019in}{5.993435in}}%
\pgfpathlineto{\pgfqpoint{6.201037in}{5.974273in}}%
\pgfpathlineto{\pgfqpoint{6.276055in}{5.957067in}}%
\pgfpathlineto{\pgfqpoint{6.351074in}{5.941547in}}%
\pgfpathlineto{\pgfqpoint{6.426092in}{5.927502in}}%
\pgfpathlineto{\pgfqpoint{6.501110in}{5.914760in}}%
\pgfpathlineto{\pgfqpoint{6.576128in}{5.903179in}}%
\pgfpathlineto{\pgfqpoint{6.651146in}{5.892644in}}%
\pgfpathlineto{\pgfqpoint{6.726164in}{5.883058in}}%
\pgfpathlineto{\pgfqpoint{6.801183in}{5.874338in}}%
\pgfpathlineto{\pgfqpoint{6.876201in}{5.866417in}}%
\pgfpathlineto{\pgfqpoint{6.951219in}{5.859235in}}%
\pgfpathlineto{\pgfqpoint{7.026237in}{5.852741in}}%
\pgfpathlineto{\pgfqpoint{7.101255in}{5.846894in}}%
\pgfpathlineto{\pgfqpoint{7.176273in}{5.841655in}}%
\pgfpathlineto{\pgfqpoint{7.251291in}{5.836994in}}%
\pgfpathlineto{\pgfqpoint{7.326310in}{5.832883in}}%
\pgfpathlineto{\pgfqpoint{7.401328in}{5.829298in}}%
\pgfpathlineto{\pgfqpoint{7.476346in}{5.826221in}}%
\pgfpathlineto{\pgfqpoint{7.551364in}{5.823636in}}%
\pgfpathlineto{\pgfqpoint{7.626382in}{5.821528in}}%
\pgfusepath{stroke}%
\end{pgfscope}%
\begin{pgfscope}%
\pgfpathrectangle{\pgfqpoint{5.814694in}{4.908628in}}{\pgfqpoint{1.897959in}{1.372727in}} %
\pgfusepath{clip}%
\pgfsetbuttcap%
\pgfsetmiterjoin%
\definecolor{currentfill}{rgb}{0.000000,0.750000,0.750000}%
\pgfsetfillcolor{currentfill}%
\pgfsetlinewidth{1.003750pt}%
\definecolor{currentstroke}{rgb}{0.000000,0.750000,0.750000}%
\pgfsetstrokecolor{currentstroke}%
\pgfsetdash{}{0pt}%
\pgfsys@defobject{currentmarker}{\pgfqpoint{-0.041667in}{-0.041667in}}{\pgfqpoint{0.041667in}{0.041667in}}{%
\pgfpathmoveto{\pgfqpoint{-0.000000in}{-0.041667in}}%
\pgfpathlineto{\pgfqpoint{0.041667in}{0.041667in}}%
\pgfpathlineto{\pgfqpoint{-0.041667in}{0.041667in}}%
\pgfpathclose%
\pgfusepath{stroke,fill}%
}%
\begin{pgfscope}%
\pgfsys@transformshift{5.900965in}{6.066668in}%
\pgfsys@useobject{currentmarker}{}%
\end{pgfscope}%
\begin{pgfscope}%
\pgfsys@transformshift{6.276055in}{5.957067in}%
\pgfsys@useobject{currentmarker}{}%
\end{pgfscope}%
\begin{pgfscope}%
\pgfsys@transformshift{6.651146in}{5.892644in}%
\pgfsys@useobject{currentmarker}{}%
\end{pgfscope}%
\begin{pgfscope}%
\pgfsys@transformshift{7.026237in}{5.852741in}%
\pgfsys@useobject{currentmarker}{}%
\end{pgfscope}%
\begin{pgfscope}%
\pgfsys@transformshift{7.401328in}{5.829298in}%
\pgfsys@useobject{currentmarker}{}%
\end{pgfscope}%
\end{pgfscope}%
\begin{pgfscope}%
\pgfpathrectangle{\pgfqpoint{5.814694in}{4.908628in}}{\pgfqpoint{1.897959in}{1.372727in}} %
\pgfusepath{clip}%
\pgfsetbuttcap%
\pgfsetroundjoin%
\pgfsetlinewidth{1.505625pt}%
\definecolor{currentstroke}{rgb}{0.000000,0.000000,0.000000}%
\pgfsetstrokecolor{currentstroke}%
\pgfsetdash{{1.500000pt}{2.475000pt}}{0.000000pt}%
\pgfpathmoveto{\pgfqpoint{5.900965in}{6.218958in}}%
\pgfpathlineto{\pgfqpoint{5.975983in}{6.207656in}}%
\pgfpathlineto{\pgfqpoint{6.051001in}{6.196848in}}%
\pgfpathlineto{\pgfqpoint{6.126019in}{6.187653in}}%
\pgfpathlineto{\pgfqpoint{6.201037in}{6.179711in}}%
\pgfpathlineto{\pgfqpoint{6.276055in}{6.172768in}}%
\pgfpathlineto{\pgfqpoint{6.351074in}{6.166637in}}%
\pgfpathlineto{\pgfqpoint{6.426092in}{6.161179in}}%
\pgfpathlineto{\pgfqpoint{6.501110in}{6.156290in}}%
\pgfpathlineto{\pgfqpoint{6.576128in}{6.151888in}}%
\pgfpathlineto{\pgfqpoint{6.651146in}{6.147911in}}%
\pgfpathlineto{\pgfqpoint{6.726164in}{6.144308in}}%
\pgfpathlineto{\pgfqpoint{6.801183in}{6.141039in}}%
\pgfpathlineto{\pgfqpoint{6.876201in}{6.138074in}}%
\pgfpathlineto{\pgfqpoint{6.951219in}{6.135385in}}%
\pgfpathlineto{\pgfqpoint{7.026237in}{6.132950in}}%
\pgfpathlineto{\pgfqpoint{7.101255in}{6.130754in}}%
\pgfpathlineto{\pgfqpoint{7.176273in}{6.128779in}}%
\pgfpathlineto{\pgfqpoint{7.251291in}{6.127016in}}%
\pgfpathlineto{\pgfqpoint{7.326310in}{6.125453in}}%
\pgfpathlineto{\pgfqpoint{7.401328in}{6.124083in}}%
\pgfpathlineto{\pgfqpoint{7.476346in}{6.122898in}}%
\pgfpathlineto{\pgfqpoint{7.551364in}{6.121893in}}%
\pgfpathlineto{\pgfqpoint{7.626382in}{6.121064in}}%
\pgfusepath{stroke}%
\end{pgfscope}%
\begin{pgfscope}%
\pgfpathrectangle{\pgfqpoint{5.814694in}{4.908628in}}{\pgfqpoint{1.897959in}{1.372727in}} %
\pgfusepath{clip}%
\pgfsetbuttcap%
\pgfsetroundjoin%
\definecolor{currentfill}{rgb}{0.000000,0.000000,0.000000}%
\pgfsetfillcolor{currentfill}%
\pgfsetlinewidth{1.003750pt}%
\definecolor{currentstroke}{rgb}{0.000000,0.000000,0.000000}%
\pgfsetstrokecolor{currentstroke}%
\pgfsetdash{}{0pt}%
\pgfsys@defobject{currentmarker}{\pgfqpoint{-0.041667in}{-0.041667in}}{\pgfqpoint{0.041667in}{0.041667in}}{%
\pgfpathmoveto{\pgfqpoint{-0.041667in}{0.000000in}}%
\pgfpathlineto{\pgfqpoint{0.041667in}{0.000000in}}%
\pgfpathmoveto{\pgfqpoint{0.000000in}{-0.041667in}}%
\pgfpathlineto{\pgfqpoint{0.000000in}{0.041667in}}%
\pgfusepath{stroke,fill}%
}%
\begin{pgfscope}%
\pgfsys@transformshift{5.900965in}{6.218958in}%
\pgfsys@useobject{currentmarker}{}%
\end{pgfscope}%
\begin{pgfscope}%
\pgfsys@transformshift{6.276055in}{6.172768in}%
\pgfsys@useobject{currentmarker}{}%
\end{pgfscope}%
\begin{pgfscope}%
\pgfsys@transformshift{6.651146in}{6.147911in}%
\pgfsys@useobject{currentmarker}{}%
\end{pgfscope}%
\begin{pgfscope}%
\pgfsys@transformshift{7.026237in}{6.132950in}%
\pgfsys@useobject{currentmarker}{}%
\end{pgfscope}%
\begin{pgfscope}%
\pgfsys@transformshift{7.401328in}{6.124083in}%
\pgfsys@useobject{currentmarker}{}%
\end{pgfscope}%
\end{pgfscope}%
\begin{pgfscope}%
\pgfsetrectcap%
\pgfsetmiterjoin%
\pgfsetlinewidth{0.803000pt}%
\definecolor{currentstroke}{rgb}{0.000000,0.000000,0.000000}%
\pgfsetstrokecolor{currentstroke}%
\pgfsetdash{}{0pt}%
\pgfpathmoveto{\pgfqpoint{5.814694in}{4.908628in}}%
\pgfpathlineto{\pgfqpoint{5.814694in}{6.281355in}}%
\pgfusepath{stroke}%
\end{pgfscope}%
\begin{pgfscope}%
\pgfsetrectcap%
\pgfsetmiterjoin%
\pgfsetlinewidth{0.803000pt}%
\definecolor{currentstroke}{rgb}{0.000000,0.000000,0.000000}%
\pgfsetstrokecolor{currentstroke}%
\pgfsetdash{}{0pt}%
\pgfpathmoveto{\pgfqpoint{7.712653in}{4.908628in}}%
\pgfpathlineto{\pgfqpoint{7.712653in}{6.281355in}}%
\pgfusepath{stroke}%
\end{pgfscope}%
\begin{pgfscope}%
\pgfsetrectcap%
\pgfsetmiterjoin%
\pgfsetlinewidth{0.803000pt}%
\definecolor{currentstroke}{rgb}{0.000000,0.000000,0.000000}%
\pgfsetstrokecolor{currentstroke}%
\pgfsetdash{}{0pt}%
\pgfpathmoveto{\pgfqpoint{5.814694in}{4.908628in}}%
\pgfpathlineto{\pgfqpoint{7.712653in}{4.908628in}}%
\pgfusepath{stroke}%
\end{pgfscope}%
\begin{pgfscope}%
\pgfsetrectcap%
\pgfsetmiterjoin%
\pgfsetlinewidth{0.803000pt}%
\definecolor{currentstroke}{rgb}{0.000000,0.000000,0.000000}%
\pgfsetstrokecolor{currentstroke}%
\pgfsetdash{}{0pt}%
\pgfpathmoveto{\pgfqpoint{5.814694in}{6.281355in}}%
\pgfpathlineto{\pgfqpoint{7.712653in}{6.281355in}}%
\pgfusepath{stroke}%
\end{pgfscope}%
\begin{pgfscope}%
\pgfsetbuttcap%
\pgfsetmiterjoin%
\definecolor{currentfill}{rgb}{1.000000,1.000000,1.000000}%
\pgfsetfillcolor{currentfill}%
\pgfsetlinewidth{0.000000pt}%
\definecolor{currentstroke}{rgb}{0.000000,0.000000,0.000000}%
\pgfsetstrokecolor{currentstroke}%
\pgfsetstrokeopacity{0.000000}%
\pgfsetdash{}{0pt}%
\pgfpathmoveto{\pgfqpoint{8.282041in}{4.908628in}}%
\pgfpathlineto{\pgfqpoint{10.180000in}{4.908628in}}%
\pgfpathlineto{\pgfqpoint{10.180000in}{6.281355in}}%
\pgfpathlineto{\pgfqpoint{8.282041in}{6.281355in}}%
\pgfpathclose%
\pgfusepath{fill}%
\end{pgfscope}%
\begin{pgfscope}%
\pgfsetbuttcap%
\pgfsetroundjoin%
\definecolor{currentfill}{rgb}{0.000000,0.000000,0.000000}%
\pgfsetfillcolor{currentfill}%
\pgfsetlinewidth{0.803000pt}%
\definecolor{currentstroke}{rgb}{0.000000,0.000000,0.000000}%
\pgfsetstrokecolor{currentstroke}%
\pgfsetdash{}{0pt}%
\pgfsys@defobject{currentmarker}{\pgfqpoint{0.000000in}{-0.048611in}}{\pgfqpoint{0.000000in}{0.000000in}}{%
\pgfpathmoveto{\pgfqpoint{0.000000in}{0.000000in}}%
\pgfpathlineto{\pgfqpoint{0.000000in}{-0.048611in}}%
\pgfusepath{stroke,fill}%
}%
\begin{pgfscope}%
\pgfsys@transformshift{8.759406in}{4.908628in}%
\pgfsys@useobject{currentmarker}{}%
\end{pgfscope}%
\end{pgfscope}%
\begin{pgfscope}%
\pgftext[x=8.759406in,y=4.811405in,,top]{\rmfamily\fontsize{10.000000}{12.000000}\selectfont \(\displaystyle 0.2\)}%
\end{pgfscope}%
\begin{pgfscope}%
\pgfsetbuttcap%
\pgfsetroundjoin%
\definecolor{currentfill}{rgb}{0.000000,0.000000,0.000000}%
\pgfsetfillcolor{currentfill}%
\pgfsetlinewidth{0.803000pt}%
\definecolor{currentstroke}{rgb}{0.000000,0.000000,0.000000}%
\pgfsetstrokecolor{currentstroke}%
\pgfsetdash{}{0pt}%
\pgfsys@defobject{currentmarker}{\pgfqpoint{0.000000in}{-0.048611in}}{\pgfqpoint{0.000000in}{0.000000in}}{%
\pgfpathmoveto{\pgfqpoint{0.000000in}{0.000000in}}%
\pgfpathlineto{\pgfqpoint{0.000000in}{-0.048611in}}%
\pgfusepath{stroke,fill}%
}%
\begin{pgfscope}%
\pgfsys@transformshift{9.311540in}{4.908628in}%
\pgfsys@useobject{currentmarker}{}%
\end{pgfscope}%
\end{pgfscope}%
\begin{pgfscope}%
\pgftext[x=9.311540in,y=4.811405in,,top]{\rmfamily\fontsize{10.000000}{12.000000}\selectfont \(\displaystyle 0.4\)}%
\end{pgfscope}%
\begin{pgfscope}%
\pgfsetbuttcap%
\pgfsetroundjoin%
\definecolor{currentfill}{rgb}{0.000000,0.000000,0.000000}%
\pgfsetfillcolor{currentfill}%
\pgfsetlinewidth{0.803000pt}%
\definecolor{currentstroke}{rgb}{0.000000,0.000000,0.000000}%
\pgfsetstrokecolor{currentstroke}%
\pgfsetdash{}{0pt}%
\pgfsys@defobject{currentmarker}{\pgfqpoint{0.000000in}{-0.048611in}}{\pgfqpoint{0.000000in}{0.000000in}}{%
\pgfpathmoveto{\pgfqpoint{0.000000in}{0.000000in}}%
\pgfpathlineto{\pgfqpoint{0.000000in}{-0.048611in}}%
\pgfusepath{stroke,fill}%
}%
\begin{pgfscope}%
\pgfsys@transformshift{9.863673in}{4.908628in}%
\pgfsys@useobject{currentmarker}{}%
\end{pgfscope}%
\end{pgfscope}%
\begin{pgfscope}%
\pgftext[x=9.863673in,y=4.811405in,,top]{\rmfamily\fontsize{10.000000}{12.000000}\selectfont \(\displaystyle 0.6\)}%
\end{pgfscope}%
\begin{pgfscope}%
\pgfsetbuttcap%
\pgfsetroundjoin%
\definecolor{currentfill}{rgb}{0.000000,0.000000,0.000000}%
\pgfsetfillcolor{currentfill}%
\pgfsetlinewidth{0.803000pt}%
\definecolor{currentstroke}{rgb}{0.000000,0.000000,0.000000}%
\pgfsetstrokecolor{currentstroke}%
\pgfsetdash{}{0pt}%
\pgfsys@defobject{currentmarker}{\pgfqpoint{-0.048611in}{0.000000in}}{\pgfqpoint{0.000000in}{0.000000in}}{%
\pgfpathmoveto{\pgfqpoint{0.000000in}{0.000000in}}%
\pgfpathlineto{\pgfqpoint{-0.048611in}{0.000000in}}%
\pgfusepath{stroke,fill}%
}%
\begin{pgfscope}%
\pgfsys@transformshift{8.282041in}{4.975720in}%
\pgfsys@useobject{currentmarker}{}%
\end{pgfscope}%
\end{pgfscope}%
\begin{pgfscope}%
\pgftext[x=7.896816in,y=4.922959in,left,base]{\rmfamily\fontsize{10.000000}{12.000000}\selectfont \(\displaystyle 10^{-8}\)}%
\end{pgfscope}%
\begin{pgfscope}%
\pgfsetbuttcap%
\pgfsetroundjoin%
\definecolor{currentfill}{rgb}{0.000000,0.000000,0.000000}%
\pgfsetfillcolor{currentfill}%
\pgfsetlinewidth{0.803000pt}%
\definecolor{currentstroke}{rgb}{0.000000,0.000000,0.000000}%
\pgfsetstrokecolor{currentstroke}%
\pgfsetdash{}{0pt}%
\pgfsys@defobject{currentmarker}{\pgfqpoint{-0.048611in}{0.000000in}}{\pgfqpoint{0.000000in}{0.000000in}}{%
\pgfpathmoveto{\pgfqpoint{0.000000in}{0.000000in}}%
\pgfpathlineto{\pgfqpoint{-0.048611in}{0.000000in}}%
\pgfusepath{stroke,fill}%
}%
\begin{pgfscope}%
\pgfsys@transformshift{8.282041in}{5.435532in}%
\pgfsys@useobject{currentmarker}{}%
\end{pgfscope}%
\end{pgfscope}%
\begin{pgfscope}%
\pgftext[x=7.896816in,y=5.382770in,left,base]{\rmfamily\fontsize{10.000000}{12.000000}\selectfont \(\displaystyle 10^{-7}\)}%
\end{pgfscope}%
\begin{pgfscope}%
\pgfsetbuttcap%
\pgfsetroundjoin%
\definecolor{currentfill}{rgb}{0.000000,0.000000,0.000000}%
\pgfsetfillcolor{currentfill}%
\pgfsetlinewidth{0.803000pt}%
\definecolor{currentstroke}{rgb}{0.000000,0.000000,0.000000}%
\pgfsetstrokecolor{currentstroke}%
\pgfsetdash{}{0pt}%
\pgfsys@defobject{currentmarker}{\pgfqpoint{-0.048611in}{0.000000in}}{\pgfqpoint{0.000000in}{0.000000in}}{%
\pgfpathmoveto{\pgfqpoint{0.000000in}{0.000000in}}%
\pgfpathlineto{\pgfqpoint{-0.048611in}{0.000000in}}%
\pgfusepath{stroke,fill}%
}%
\begin{pgfscope}%
\pgfsys@transformshift{8.282041in}{5.895344in}%
\pgfsys@useobject{currentmarker}{}%
\end{pgfscope}%
\end{pgfscope}%
\begin{pgfscope}%
\pgftext[x=7.896816in,y=5.842582in,left,base]{\rmfamily\fontsize{10.000000}{12.000000}\selectfont \(\displaystyle 10^{-6}\)}%
\end{pgfscope}%
\begin{pgfscope}%
\pgfsetbuttcap%
\pgfsetroundjoin%
\definecolor{currentfill}{rgb}{0.000000,0.000000,0.000000}%
\pgfsetfillcolor{currentfill}%
\pgfsetlinewidth{0.602250pt}%
\definecolor{currentstroke}{rgb}{0.000000,0.000000,0.000000}%
\pgfsetstrokecolor{currentstroke}%
\pgfsetdash{}{0pt}%
\pgfsys@defobject{currentmarker}{\pgfqpoint{-0.027778in}{0.000000in}}{\pgfqpoint{0.000000in}{0.000000in}}{%
\pgfpathmoveto{\pgfqpoint{0.000000in}{0.000000in}}%
\pgfpathlineto{\pgfqpoint{-0.027778in}{0.000000in}}%
\pgfusepath{stroke,fill}%
}%
\begin{pgfscope}%
\pgfsys@transformshift{8.282041in}{4.904495in}%
\pgfsys@useobject{currentmarker}{}%
\end{pgfscope}%
\end{pgfscope}%
\begin{pgfscope}%
\pgfsetbuttcap%
\pgfsetroundjoin%
\definecolor{currentfill}{rgb}{0.000000,0.000000,0.000000}%
\pgfsetfillcolor{currentfill}%
\pgfsetlinewidth{0.602250pt}%
\definecolor{currentstroke}{rgb}{0.000000,0.000000,0.000000}%
\pgfsetstrokecolor{currentstroke}%
\pgfsetdash{}{0pt}%
\pgfsys@defobject{currentmarker}{\pgfqpoint{-0.027778in}{0.000000in}}{\pgfqpoint{0.000000in}{0.000000in}}{%
\pgfpathmoveto{\pgfqpoint{0.000000in}{0.000000in}}%
\pgfpathlineto{\pgfqpoint{-0.027778in}{0.000000in}}%
\pgfusepath{stroke,fill}%
}%
\begin{pgfscope}%
\pgfsys@transformshift{8.282041in}{4.931160in}%
\pgfsys@useobject{currentmarker}{}%
\end{pgfscope}%
\end{pgfscope}%
\begin{pgfscope}%
\pgfsetbuttcap%
\pgfsetroundjoin%
\definecolor{currentfill}{rgb}{0.000000,0.000000,0.000000}%
\pgfsetfillcolor{currentfill}%
\pgfsetlinewidth{0.602250pt}%
\definecolor{currentstroke}{rgb}{0.000000,0.000000,0.000000}%
\pgfsetstrokecolor{currentstroke}%
\pgfsetdash{}{0pt}%
\pgfsys@defobject{currentmarker}{\pgfqpoint{-0.027778in}{0.000000in}}{\pgfqpoint{0.000000in}{0.000000in}}{%
\pgfpathmoveto{\pgfqpoint{0.000000in}{0.000000in}}%
\pgfpathlineto{\pgfqpoint{-0.027778in}{0.000000in}}%
\pgfusepath{stroke,fill}%
}%
\begin{pgfscope}%
\pgfsys@transformshift{8.282041in}{4.954681in}%
\pgfsys@useobject{currentmarker}{}%
\end{pgfscope}%
\end{pgfscope}%
\begin{pgfscope}%
\pgfsetbuttcap%
\pgfsetroundjoin%
\definecolor{currentfill}{rgb}{0.000000,0.000000,0.000000}%
\pgfsetfillcolor{currentfill}%
\pgfsetlinewidth{0.602250pt}%
\definecolor{currentstroke}{rgb}{0.000000,0.000000,0.000000}%
\pgfsetstrokecolor{currentstroke}%
\pgfsetdash{}{0pt}%
\pgfsys@defobject{currentmarker}{\pgfqpoint{-0.027778in}{0.000000in}}{\pgfqpoint{0.000000in}{0.000000in}}{%
\pgfpathmoveto{\pgfqpoint{0.000000in}{0.000000in}}%
\pgfpathlineto{\pgfqpoint{-0.027778in}{0.000000in}}%
\pgfusepath{stroke,fill}%
}%
\begin{pgfscope}%
\pgfsys@transformshift{8.282041in}{5.114137in}%
\pgfsys@useobject{currentmarker}{}%
\end{pgfscope}%
\end{pgfscope}%
\begin{pgfscope}%
\pgfsetbuttcap%
\pgfsetroundjoin%
\definecolor{currentfill}{rgb}{0.000000,0.000000,0.000000}%
\pgfsetfillcolor{currentfill}%
\pgfsetlinewidth{0.602250pt}%
\definecolor{currentstroke}{rgb}{0.000000,0.000000,0.000000}%
\pgfsetstrokecolor{currentstroke}%
\pgfsetdash{}{0pt}%
\pgfsys@defobject{currentmarker}{\pgfqpoint{-0.027778in}{0.000000in}}{\pgfqpoint{0.000000in}{0.000000in}}{%
\pgfpathmoveto{\pgfqpoint{0.000000in}{0.000000in}}%
\pgfpathlineto{\pgfqpoint{-0.027778in}{0.000000in}}%
\pgfusepath{stroke,fill}%
}%
\begin{pgfscope}%
\pgfsys@transformshift{8.282041in}{5.195106in}%
\pgfsys@useobject{currentmarker}{}%
\end{pgfscope}%
\end{pgfscope}%
\begin{pgfscope}%
\pgfsetbuttcap%
\pgfsetroundjoin%
\definecolor{currentfill}{rgb}{0.000000,0.000000,0.000000}%
\pgfsetfillcolor{currentfill}%
\pgfsetlinewidth{0.602250pt}%
\definecolor{currentstroke}{rgb}{0.000000,0.000000,0.000000}%
\pgfsetstrokecolor{currentstroke}%
\pgfsetdash{}{0pt}%
\pgfsys@defobject{currentmarker}{\pgfqpoint{-0.027778in}{0.000000in}}{\pgfqpoint{0.000000in}{0.000000in}}{%
\pgfpathmoveto{\pgfqpoint{0.000000in}{0.000000in}}%
\pgfpathlineto{\pgfqpoint{-0.027778in}{0.000000in}}%
\pgfusepath{stroke,fill}%
}%
\begin{pgfscope}%
\pgfsys@transformshift{8.282041in}{5.252554in}%
\pgfsys@useobject{currentmarker}{}%
\end{pgfscope}%
\end{pgfscope}%
\begin{pgfscope}%
\pgfsetbuttcap%
\pgfsetroundjoin%
\definecolor{currentfill}{rgb}{0.000000,0.000000,0.000000}%
\pgfsetfillcolor{currentfill}%
\pgfsetlinewidth{0.602250pt}%
\definecolor{currentstroke}{rgb}{0.000000,0.000000,0.000000}%
\pgfsetstrokecolor{currentstroke}%
\pgfsetdash{}{0pt}%
\pgfsys@defobject{currentmarker}{\pgfqpoint{-0.027778in}{0.000000in}}{\pgfqpoint{0.000000in}{0.000000in}}{%
\pgfpathmoveto{\pgfqpoint{0.000000in}{0.000000in}}%
\pgfpathlineto{\pgfqpoint{-0.027778in}{0.000000in}}%
\pgfusepath{stroke,fill}%
}%
\begin{pgfscope}%
\pgfsys@transformshift{8.282041in}{5.297115in}%
\pgfsys@useobject{currentmarker}{}%
\end{pgfscope}%
\end{pgfscope}%
\begin{pgfscope}%
\pgfsetbuttcap%
\pgfsetroundjoin%
\definecolor{currentfill}{rgb}{0.000000,0.000000,0.000000}%
\pgfsetfillcolor{currentfill}%
\pgfsetlinewidth{0.602250pt}%
\definecolor{currentstroke}{rgb}{0.000000,0.000000,0.000000}%
\pgfsetstrokecolor{currentstroke}%
\pgfsetdash{}{0pt}%
\pgfsys@defobject{currentmarker}{\pgfqpoint{-0.027778in}{0.000000in}}{\pgfqpoint{0.000000in}{0.000000in}}{%
\pgfpathmoveto{\pgfqpoint{0.000000in}{0.000000in}}%
\pgfpathlineto{\pgfqpoint{-0.027778in}{0.000000in}}%
\pgfusepath{stroke,fill}%
}%
\begin{pgfscope}%
\pgfsys@transformshift{8.282041in}{5.333523in}%
\pgfsys@useobject{currentmarker}{}%
\end{pgfscope}%
\end{pgfscope}%
\begin{pgfscope}%
\pgfsetbuttcap%
\pgfsetroundjoin%
\definecolor{currentfill}{rgb}{0.000000,0.000000,0.000000}%
\pgfsetfillcolor{currentfill}%
\pgfsetlinewidth{0.602250pt}%
\definecolor{currentstroke}{rgb}{0.000000,0.000000,0.000000}%
\pgfsetstrokecolor{currentstroke}%
\pgfsetdash{}{0pt}%
\pgfsys@defobject{currentmarker}{\pgfqpoint{-0.027778in}{0.000000in}}{\pgfqpoint{0.000000in}{0.000000in}}{%
\pgfpathmoveto{\pgfqpoint{0.000000in}{0.000000in}}%
\pgfpathlineto{\pgfqpoint{-0.027778in}{0.000000in}}%
\pgfusepath{stroke,fill}%
}%
\begin{pgfscope}%
\pgfsys@transformshift{8.282041in}{5.364306in}%
\pgfsys@useobject{currentmarker}{}%
\end{pgfscope}%
\end{pgfscope}%
\begin{pgfscope}%
\pgfsetbuttcap%
\pgfsetroundjoin%
\definecolor{currentfill}{rgb}{0.000000,0.000000,0.000000}%
\pgfsetfillcolor{currentfill}%
\pgfsetlinewidth{0.602250pt}%
\definecolor{currentstroke}{rgb}{0.000000,0.000000,0.000000}%
\pgfsetstrokecolor{currentstroke}%
\pgfsetdash{}{0pt}%
\pgfsys@defobject{currentmarker}{\pgfqpoint{-0.027778in}{0.000000in}}{\pgfqpoint{0.000000in}{0.000000in}}{%
\pgfpathmoveto{\pgfqpoint{0.000000in}{0.000000in}}%
\pgfpathlineto{\pgfqpoint{-0.027778in}{0.000000in}}%
\pgfusepath{stroke,fill}%
}%
\begin{pgfscope}%
\pgfsys@transformshift{8.282041in}{5.390972in}%
\pgfsys@useobject{currentmarker}{}%
\end{pgfscope}%
\end{pgfscope}%
\begin{pgfscope}%
\pgfsetbuttcap%
\pgfsetroundjoin%
\definecolor{currentfill}{rgb}{0.000000,0.000000,0.000000}%
\pgfsetfillcolor{currentfill}%
\pgfsetlinewidth{0.602250pt}%
\definecolor{currentstroke}{rgb}{0.000000,0.000000,0.000000}%
\pgfsetstrokecolor{currentstroke}%
\pgfsetdash{}{0pt}%
\pgfsys@defobject{currentmarker}{\pgfqpoint{-0.027778in}{0.000000in}}{\pgfqpoint{0.000000in}{0.000000in}}{%
\pgfpathmoveto{\pgfqpoint{0.000000in}{0.000000in}}%
\pgfpathlineto{\pgfqpoint{-0.027778in}{0.000000in}}%
\pgfusepath{stroke,fill}%
}%
\begin{pgfscope}%
\pgfsys@transformshift{8.282041in}{5.414492in}%
\pgfsys@useobject{currentmarker}{}%
\end{pgfscope}%
\end{pgfscope}%
\begin{pgfscope}%
\pgfsetbuttcap%
\pgfsetroundjoin%
\definecolor{currentfill}{rgb}{0.000000,0.000000,0.000000}%
\pgfsetfillcolor{currentfill}%
\pgfsetlinewidth{0.602250pt}%
\definecolor{currentstroke}{rgb}{0.000000,0.000000,0.000000}%
\pgfsetstrokecolor{currentstroke}%
\pgfsetdash{}{0pt}%
\pgfsys@defobject{currentmarker}{\pgfqpoint{-0.027778in}{0.000000in}}{\pgfqpoint{0.000000in}{0.000000in}}{%
\pgfpathmoveto{\pgfqpoint{0.000000in}{0.000000in}}%
\pgfpathlineto{\pgfqpoint{-0.027778in}{0.000000in}}%
\pgfusepath{stroke,fill}%
}%
\begin{pgfscope}%
\pgfsys@transformshift{8.282041in}{5.573949in}%
\pgfsys@useobject{currentmarker}{}%
\end{pgfscope}%
\end{pgfscope}%
\begin{pgfscope}%
\pgfsetbuttcap%
\pgfsetroundjoin%
\definecolor{currentfill}{rgb}{0.000000,0.000000,0.000000}%
\pgfsetfillcolor{currentfill}%
\pgfsetlinewidth{0.602250pt}%
\definecolor{currentstroke}{rgb}{0.000000,0.000000,0.000000}%
\pgfsetstrokecolor{currentstroke}%
\pgfsetdash{}{0pt}%
\pgfsys@defobject{currentmarker}{\pgfqpoint{-0.027778in}{0.000000in}}{\pgfqpoint{0.000000in}{0.000000in}}{%
\pgfpathmoveto{\pgfqpoint{0.000000in}{0.000000in}}%
\pgfpathlineto{\pgfqpoint{-0.027778in}{0.000000in}}%
\pgfusepath{stroke,fill}%
}%
\begin{pgfscope}%
\pgfsys@transformshift{8.282041in}{5.654918in}%
\pgfsys@useobject{currentmarker}{}%
\end{pgfscope}%
\end{pgfscope}%
\begin{pgfscope}%
\pgfsetbuttcap%
\pgfsetroundjoin%
\definecolor{currentfill}{rgb}{0.000000,0.000000,0.000000}%
\pgfsetfillcolor{currentfill}%
\pgfsetlinewidth{0.602250pt}%
\definecolor{currentstroke}{rgb}{0.000000,0.000000,0.000000}%
\pgfsetstrokecolor{currentstroke}%
\pgfsetdash{}{0pt}%
\pgfsys@defobject{currentmarker}{\pgfqpoint{-0.027778in}{0.000000in}}{\pgfqpoint{0.000000in}{0.000000in}}{%
\pgfpathmoveto{\pgfqpoint{0.000000in}{0.000000in}}%
\pgfpathlineto{\pgfqpoint{-0.027778in}{0.000000in}}%
\pgfusepath{stroke,fill}%
}%
\begin{pgfscope}%
\pgfsys@transformshift{8.282041in}{5.712366in}%
\pgfsys@useobject{currentmarker}{}%
\end{pgfscope}%
\end{pgfscope}%
\begin{pgfscope}%
\pgfsetbuttcap%
\pgfsetroundjoin%
\definecolor{currentfill}{rgb}{0.000000,0.000000,0.000000}%
\pgfsetfillcolor{currentfill}%
\pgfsetlinewidth{0.602250pt}%
\definecolor{currentstroke}{rgb}{0.000000,0.000000,0.000000}%
\pgfsetstrokecolor{currentstroke}%
\pgfsetdash{}{0pt}%
\pgfsys@defobject{currentmarker}{\pgfqpoint{-0.027778in}{0.000000in}}{\pgfqpoint{0.000000in}{0.000000in}}{%
\pgfpathmoveto{\pgfqpoint{0.000000in}{0.000000in}}%
\pgfpathlineto{\pgfqpoint{-0.027778in}{0.000000in}}%
\pgfusepath{stroke,fill}%
}%
\begin{pgfscope}%
\pgfsys@transformshift{8.282041in}{5.756926in}%
\pgfsys@useobject{currentmarker}{}%
\end{pgfscope}%
\end{pgfscope}%
\begin{pgfscope}%
\pgfsetbuttcap%
\pgfsetroundjoin%
\definecolor{currentfill}{rgb}{0.000000,0.000000,0.000000}%
\pgfsetfillcolor{currentfill}%
\pgfsetlinewidth{0.602250pt}%
\definecolor{currentstroke}{rgb}{0.000000,0.000000,0.000000}%
\pgfsetstrokecolor{currentstroke}%
\pgfsetdash{}{0pt}%
\pgfsys@defobject{currentmarker}{\pgfqpoint{-0.027778in}{0.000000in}}{\pgfqpoint{0.000000in}{0.000000in}}{%
\pgfpathmoveto{\pgfqpoint{0.000000in}{0.000000in}}%
\pgfpathlineto{\pgfqpoint{-0.027778in}{0.000000in}}%
\pgfusepath{stroke,fill}%
}%
\begin{pgfscope}%
\pgfsys@transformshift{8.282041in}{5.793335in}%
\pgfsys@useobject{currentmarker}{}%
\end{pgfscope}%
\end{pgfscope}%
\begin{pgfscope}%
\pgfsetbuttcap%
\pgfsetroundjoin%
\definecolor{currentfill}{rgb}{0.000000,0.000000,0.000000}%
\pgfsetfillcolor{currentfill}%
\pgfsetlinewidth{0.602250pt}%
\definecolor{currentstroke}{rgb}{0.000000,0.000000,0.000000}%
\pgfsetstrokecolor{currentstroke}%
\pgfsetdash{}{0pt}%
\pgfsys@defobject{currentmarker}{\pgfqpoint{-0.027778in}{0.000000in}}{\pgfqpoint{0.000000in}{0.000000in}}{%
\pgfpathmoveto{\pgfqpoint{0.000000in}{0.000000in}}%
\pgfpathlineto{\pgfqpoint{-0.027778in}{0.000000in}}%
\pgfusepath{stroke,fill}%
}%
\begin{pgfscope}%
\pgfsys@transformshift{8.282041in}{5.824118in}%
\pgfsys@useobject{currentmarker}{}%
\end{pgfscope}%
\end{pgfscope}%
\begin{pgfscope}%
\pgfsetbuttcap%
\pgfsetroundjoin%
\definecolor{currentfill}{rgb}{0.000000,0.000000,0.000000}%
\pgfsetfillcolor{currentfill}%
\pgfsetlinewidth{0.602250pt}%
\definecolor{currentstroke}{rgb}{0.000000,0.000000,0.000000}%
\pgfsetstrokecolor{currentstroke}%
\pgfsetdash{}{0pt}%
\pgfsys@defobject{currentmarker}{\pgfqpoint{-0.027778in}{0.000000in}}{\pgfqpoint{0.000000in}{0.000000in}}{%
\pgfpathmoveto{\pgfqpoint{0.000000in}{0.000000in}}%
\pgfpathlineto{\pgfqpoint{-0.027778in}{0.000000in}}%
\pgfusepath{stroke,fill}%
}%
\begin{pgfscope}%
\pgfsys@transformshift{8.282041in}{5.850783in}%
\pgfsys@useobject{currentmarker}{}%
\end{pgfscope}%
\end{pgfscope}%
\begin{pgfscope}%
\pgfsetbuttcap%
\pgfsetroundjoin%
\definecolor{currentfill}{rgb}{0.000000,0.000000,0.000000}%
\pgfsetfillcolor{currentfill}%
\pgfsetlinewidth{0.602250pt}%
\definecolor{currentstroke}{rgb}{0.000000,0.000000,0.000000}%
\pgfsetstrokecolor{currentstroke}%
\pgfsetdash{}{0pt}%
\pgfsys@defobject{currentmarker}{\pgfqpoint{-0.027778in}{0.000000in}}{\pgfqpoint{0.000000in}{0.000000in}}{%
\pgfpathmoveto{\pgfqpoint{0.000000in}{0.000000in}}%
\pgfpathlineto{\pgfqpoint{-0.027778in}{0.000000in}}%
\pgfusepath{stroke,fill}%
}%
\begin{pgfscope}%
\pgfsys@transformshift{8.282041in}{5.874304in}%
\pgfsys@useobject{currentmarker}{}%
\end{pgfscope}%
\end{pgfscope}%
\begin{pgfscope}%
\pgfsetbuttcap%
\pgfsetroundjoin%
\definecolor{currentfill}{rgb}{0.000000,0.000000,0.000000}%
\pgfsetfillcolor{currentfill}%
\pgfsetlinewidth{0.602250pt}%
\definecolor{currentstroke}{rgb}{0.000000,0.000000,0.000000}%
\pgfsetstrokecolor{currentstroke}%
\pgfsetdash{}{0pt}%
\pgfsys@defobject{currentmarker}{\pgfqpoint{-0.027778in}{0.000000in}}{\pgfqpoint{0.000000in}{0.000000in}}{%
\pgfpathmoveto{\pgfqpoint{0.000000in}{0.000000in}}%
\pgfpathlineto{\pgfqpoint{-0.027778in}{0.000000in}}%
\pgfusepath{stroke,fill}%
}%
\begin{pgfscope}%
\pgfsys@transformshift{8.282041in}{6.033761in}%
\pgfsys@useobject{currentmarker}{}%
\end{pgfscope}%
\end{pgfscope}%
\begin{pgfscope}%
\pgfsetbuttcap%
\pgfsetroundjoin%
\definecolor{currentfill}{rgb}{0.000000,0.000000,0.000000}%
\pgfsetfillcolor{currentfill}%
\pgfsetlinewidth{0.602250pt}%
\definecolor{currentstroke}{rgb}{0.000000,0.000000,0.000000}%
\pgfsetstrokecolor{currentstroke}%
\pgfsetdash{}{0pt}%
\pgfsys@defobject{currentmarker}{\pgfqpoint{-0.027778in}{0.000000in}}{\pgfqpoint{0.000000in}{0.000000in}}{%
\pgfpathmoveto{\pgfqpoint{0.000000in}{0.000000in}}%
\pgfpathlineto{\pgfqpoint{-0.027778in}{0.000000in}}%
\pgfusepath{stroke,fill}%
}%
\begin{pgfscope}%
\pgfsys@transformshift{8.282041in}{6.114729in}%
\pgfsys@useobject{currentmarker}{}%
\end{pgfscope}%
\end{pgfscope}%
\begin{pgfscope}%
\pgfsetbuttcap%
\pgfsetroundjoin%
\definecolor{currentfill}{rgb}{0.000000,0.000000,0.000000}%
\pgfsetfillcolor{currentfill}%
\pgfsetlinewidth{0.602250pt}%
\definecolor{currentstroke}{rgb}{0.000000,0.000000,0.000000}%
\pgfsetstrokecolor{currentstroke}%
\pgfsetdash{}{0pt}%
\pgfsys@defobject{currentmarker}{\pgfqpoint{-0.027778in}{0.000000in}}{\pgfqpoint{0.000000in}{0.000000in}}{%
\pgfpathmoveto{\pgfqpoint{0.000000in}{0.000000in}}%
\pgfpathlineto{\pgfqpoint{-0.027778in}{0.000000in}}%
\pgfusepath{stroke,fill}%
}%
\begin{pgfscope}%
\pgfsys@transformshift{8.282041in}{6.172178in}%
\pgfsys@useobject{currentmarker}{}%
\end{pgfscope}%
\end{pgfscope}%
\begin{pgfscope}%
\pgfsetbuttcap%
\pgfsetroundjoin%
\definecolor{currentfill}{rgb}{0.000000,0.000000,0.000000}%
\pgfsetfillcolor{currentfill}%
\pgfsetlinewidth{0.602250pt}%
\definecolor{currentstroke}{rgb}{0.000000,0.000000,0.000000}%
\pgfsetstrokecolor{currentstroke}%
\pgfsetdash{}{0pt}%
\pgfsys@defobject{currentmarker}{\pgfqpoint{-0.027778in}{0.000000in}}{\pgfqpoint{0.000000in}{0.000000in}}{%
\pgfpathmoveto{\pgfqpoint{0.000000in}{0.000000in}}%
\pgfpathlineto{\pgfqpoint{-0.027778in}{0.000000in}}%
\pgfusepath{stroke,fill}%
}%
\begin{pgfscope}%
\pgfsys@transformshift{8.282041in}{6.216738in}%
\pgfsys@useobject{currentmarker}{}%
\end{pgfscope}%
\end{pgfscope}%
\begin{pgfscope}%
\pgfsetbuttcap%
\pgfsetroundjoin%
\definecolor{currentfill}{rgb}{0.000000,0.000000,0.000000}%
\pgfsetfillcolor{currentfill}%
\pgfsetlinewidth{0.602250pt}%
\definecolor{currentstroke}{rgb}{0.000000,0.000000,0.000000}%
\pgfsetstrokecolor{currentstroke}%
\pgfsetdash{}{0pt}%
\pgfsys@defobject{currentmarker}{\pgfqpoint{-0.027778in}{0.000000in}}{\pgfqpoint{0.000000in}{0.000000in}}{%
\pgfpathmoveto{\pgfqpoint{0.000000in}{0.000000in}}%
\pgfpathlineto{\pgfqpoint{-0.027778in}{0.000000in}}%
\pgfusepath{stroke,fill}%
}%
\begin{pgfscope}%
\pgfsys@transformshift{8.282041in}{6.253146in}%
\pgfsys@useobject{currentmarker}{}%
\end{pgfscope}%
\end{pgfscope}%
\begin{pgfscope}%
\pgfsetbuttcap%
\pgfsetroundjoin%
\definecolor{currentfill}{rgb}{0.000000,0.000000,0.000000}%
\pgfsetfillcolor{currentfill}%
\pgfsetlinewidth{0.602250pt}%
\definecolor{currentstroke}{rgb}{0.000000,0.000000,0.000000}%
\pgfsetstrokecolor{currentstroke}%
\pgfsetdash{}{0pt}%
\pgfsys@defobject{currentmarker}{\pgfqpoint{-0.027778in}{0.000000in}}{\pgfqpoint{0.000000in}{0.000000in}}{%
\pgfpathmoveto{\pgfqpoint{0.000000in}{0.000000in}}%
\pgfpathlineto{\pgfqpoint{-0.027778in}{0.000000in}}%
\pgfusepath{stroke,fill}%
}%
\begin{pgfscope}%
\pgfsys@transformshift{8.282041in}{6.283929in}%
\pgfsys@useobject{currentmarker}{}%
\end{pgfscope}%
\end{pgfscope}%
\begin{pgfscope}%
\pgfpathrectangle{\pgfqpoint{8.282041in}{4.908628in}}{\pgfqpoint{1.897959in}{1.372727in}} %
\pgfusepath{clip}%
\pgfsetbuttcap%
\pgfsetroundjoin%
\pgfsetlinewidth{1.505625pt}%
\definecolor{currentstroke}{rgb}{1.000000,0.000000,0.000000}%
\pgfsetstrokecolor{currentstroke}%
\pgfsetdash{{5.550000pt}{2.400000pt}}{0.000000pt}%
\pgfpathmoveto{\pgfqpoint{8.368312in}{6.127938in}}%
\pgfpathlineto{\pgfqpoint{8.391317in}{6.125102in}}%
\pgfpathlineto{\pgfqpoint{8.414323in}{6.122307in}}%
\pgfpathlineto{\pgfqpoint{8.437328in}{6.119551in}}%
\pgfpathlineto{\pgfqpoint{8.460334in}{6.116832in}}%
\pgfpathlineto{\pgfqpoint{8.483340in}{6.114152in}}%
\pgfpathlineto{\pgfqpoint{8.506345in}{6.111509in}}%
\pgfpathlineto{\pgfqpoint{8.529351in}{6.108903in}}%
\pgfpathlineto{\pgfqpoint{8.552356in}{6.106335in}}%
\pgfpathlineto{\pgfqpoint{8.575362in}{6.103804in}}%
\pgfpathlineto{\pgfqpoint{8.598367in}{6.101310in}}%
\pgfpathlineto{\pgfqpoint{8.621373in}{6.098855in}}%
\pgfpathlineto{\pgfqpoint{8.644378in}{6.096437in}}%
\pgfpathlineto{\pgfqpoint{8.667384in}{6.094058in}}%
\pgfpathlineto{\pgfqpoint{8.690390in}{6.091716in}}%
\pgfpathlineto{\pgfqpoint{8.713395in}{6.089413in}}%
\pgfpathlineto{\pgfqpoint{8.736401in}{6.087148in}}%
\pgfpathlineto{\pgfqpoint{8.759406in}{6.084922in}}%
\pgfpathlineto{\pgfqpoint{8.782412in}{6.082734in}}%
\pgfpathlineto{\pgfqpoint{8.805417in}{6.080585in}}%
\pgfpathlineto{\pgfqpoint{8.828423in}{6.078474in}}%
\pgfpathlineto{\pgfqpoint{8.851429in}{6.076402in}}%
\pgfpathlineto{\pgfqpoint{8.874434in}{6.074369in}}%
\pgfpathlineto{\pgfqpoint{8.897440in}{6.072374in}}%
\pgfpathlineto{\pgfqpoint{8.920445in}{6.070418in}}%
\pgfpathlineto{\pgfqpoint{8.943451in}{6.068500in}}%
\pgfpathlineto{\pgfqpoint{8.966456in}{6.066620in}}%
\pgfpathlineto{\pgfqpoint{8.989462in}{6.064779in}}%
\pgfpathlineto{\pgfqpoint{9.012468in}{6.062976in}}%
\pgfpathlineto{\pgfqpoint{9.035473in}{6.061211in}}%
\pgfpathlineto{\pgfqpoint{9.058479in}{6.059483in}}%
\pgfpathlineto{\pgfqpoint{9.081484in}{6.057794in}}%
\pgfpathlineto{\pgfqpoint{9.104490in}{6.056142in}}%
\pgfpathlineto{\pgfqpoint{9.127495in}{6.054527in}}%
\pgfpathlineto{\pgfqpoint{9.150501in}{6.052950in}}%
\pgfpathlineto{\pgfqpoint{9.173506in}{6.051410in}}%
\pgfpathlineto{\pgfqpoint{9.196512in}{6.049907in}}%
\pgfpathlineto{\pgfqpoint{9.219518in}{6.048441in}}%
\pgfpathlineto{\pgfqpoint{9.242523in}{6.047011in}}%
\pgfpathlineto{\pgfqpoint{9.265529in}{6.045618in}}%
\pgfpathlineto{\pgfqpoint{9.288534in}{6.044261in}}%
\pgfpathlineto{\pgfqpoint{9.311540in}{6.042940in}}%
\pgfpathlineto{\pgfqpoint{9.334545in}{6.041655in}}%
\pgfpathlineto{\pgfqpoint{9.357551in}{6.040405in}}%
\pgfpathlineto{\pgfqpoint{9.380557in}{6.039191in}}%
\pgfpathlineto{\pgfqpoint{9.403562in}{6.038012in}}%
\pgfpathlineto{\pgfqpoint{9.426568in}{6.036868in}}%
\pgfpathlineto{\pgfqpoint{9.449573in}{6.035759in}}%
\pgfpathlineto{\pgfqpoint{9.472579in}{6.034685in}}%
\pgfpathlineto{\pgfqpoint{9.495584in}{6.033645in}}%
\pgfpathlineto{\pgfqpoint{9.518590in}{6.032640in}}%
\pgfpathlineto{\pgfqpoint{9.541596in}{6.031669in}}%
\pgfpathlineto{\pgfqpoint{9.564601in}{6.030732in}}%
\pgfpathlineto{\pgfqpoint{9.587607in}{6.029828in}}%
\pgfpathlineto{\pgfqpoint{9.610612in}{6.028959in}}%
\pgfpathlineto{\pgfqpoint{9.633618in}{6.028122in}}%
\pgfpathlineto{\pgfqpoint{9.656623in}{6.027319in}}%
\pgfpathlineto{\pgfqpoint{9.679629in}{6.026550in}}%
\pgfpathlineto{\pgfqpoint{9.702635in}{6.025813in}}%
\pgfpathlineto{\pgfqpoint{9.725640in}{6.025109in}}%
\pgfpathlineto{\pgfqpoint{9.748646in}{6.024437in}}%
\pgfpathlineto{\pgfqpoint{9.771651in}{6.023799in}}%
\pgfpathlineto{\pgfqpoint{9.794657in}{6.023192in}}%
\pgfpathlineto{\pgfqpoint{9.817662in}{6.022619in}}%
\pgfpathlineto{\pgfqpoint{9.840668in}{6.022077in}}%
\pgfpathlineto{\pgfqpoint{9.863673in}{6.021567in}}%
\pgfpathlineto{\pgfqpoint{9.886679in}{6.021089in}}%
\pgfpathlineto{\pgfqpoint{9.909685in}{6.020643in}}%
\pgfpathlineto{\pgfqpoint{9.932690in}{6.020229in}}%
\pgfpathlineto{\pgfqpoint{9.955696in}{6.019846in}}%
\pgfpathlineto{\pgfqpoint{9.978701in}{6.019495in}}%
\pgfpathlineto{\pgfqpoint{10.001707in}{6.019175in}}%
\pgfpathlineto{\pgfqpoint{10.024712in}{6.018887in}}%
\pgfpathlineto{\pgfqpoint{10.047718in}{6.018630in}}%
\pgfpathlineto{\pgfqpoint{10.070724in}{6.018404in}}%
\pgfpathlineto{\pgfqpoint{10.093729in}{6.018209in}}%
\pgfusepath{stroke}%
\end{pgfscope}%
\begin{pgfscope}%
\pgfpathrectangle{\pgfqpoint{8.282041in}{4.908628in}}{\pgfqpoint{1.897959in}{1.372727in}} %
\pgfusepath{clip}%
\pgfsetbuttcap%
\pgfsetmiterjoin%
\definecolor{currentfill}{rgb}{1.000000,0.000000,0.000000}%
\pgfsetfillcolor{currentfill}%
\pgfsetlinewidth{1.003750pt}%
\definecolor{currentstroke}{rgb}{1.000000,0.000000,0.000000}%
\pgfsetstrokecolor{currentstroke}%
\pgfsetdash{}{0pt}%
\pgfsys@defobject{currentmarker}{\pgfqpoint{-0.041667in}{-0.041667in}}{\pgfqpoint{0.041667in}{0.041667in}}{%
\pgfpathmoveto{\pgfqpoint{-0.041667in}{-0.041667in}}%
\pgfpathlineto{\pgfqpoint{0.041667in}{-0.041667in}}%
\pgfpathlineto{\pgfqpoint{0.041667in}{0.041667in}}%
\pgfpathlineto{\pgfqpoint{-0.041667in}{0.041667in}}%
\pgfpathclose%
\pgfusepath{stroke,fill}%
}%
\begin{pgfscope}%
\pgfsys@transformshift{8.368312in}{6.127938in}%
\pgfsys@useobject{currentmarker}{}%
\end{pgfscope}%
\begin{pgfscope}%
\pgfsys@transformshift{8.713395in}{6.089413in}%
\pgfsys@useobject{currentmarker}{}%
\end{pgfscope}%
\begin{pgfscope}%
\pgfsys@transformshift{9.058479in}{6.059483in}%
\pgfsys@useobject{currentmarker}{}%
\end{pgfscope}%
\begin{pgfscope}%
\pgfsys@transformshift{9.403562in}{6.038012in}%
\pgfsys@useobject{currentmarker}{}%
\end{pgfscope}%
\begin{pgfscope}%
\pgfsys@transformshift{9.748646in}{6.024437in}%
\pgfsys@useobject{currentmarker}{}%
\end{pgfscope}%
\begin{pgfscope}%
\pgfsys@transformshift{10.093729in}{6.018209in}%
\pgfsys@useobject{currentmarker}{}%
\end{pgfscope}%
\end{pgfscope}%
\begin{pgfscope}%
\pgfpathrectangle{\pgfqpoint{8.282041in}{4.908628in}}{\pgfqpoint{1.897959in}{1.372727in}} %
\pgfusepath{clip}%
\pgfsetrectcap%
\pgfsetroundjoin%
\pgfsetlinewidth{1.505625pt}%
\definecolor{currentstroke}{rgb}{0.000000,0.000000,1.000000}%
\pgfsetstrokecolor{currentstroke}%
\pgfsetdash{}{0pt}%
\pgfpathmoveto{\pgfqpoint{8.368312in}{5.654281in}}%
\pgfpathlineto{\pgfqpoint{8.391317in}{5.648078in}}%
\pgfpathlineto{\pgfqpoint{8.414323in}{5.642141in}}%
\pgfpathlineto{\pgfqpoint{8.437328in}{5.636395in}}%
\pgfpathlineto{\pgfqpoint{8.460334in}{5.630792in}}%
\pgfpathlineto{\pgfqpoint{8.483340in}{5.625297in}}%
\pgfpathlineto{\pgfqpoint{8.506345in}{5.619884in}}%
\pgfpathlineto{\pgfqpoint{8.529351in}{5.614533in}}%
\pgfpathlineto{\pgfqpoint{8.552356in}{5.609229in}}%
\pgfpathlineto{\pgfqpoint{8.575362in}{5.603961in}}%
\pgfpathlineto{\pgfqpoint{8.598367in}{5.598717in}}%
\pgfpathlineto{\pgfqpoint{8.621373in}{5.593491in}}%
\pgfpathlineto{\pgfqpoint{8.644378in}{5.588274in}}%
\pgfpathlineto{\pgfqpoint{8.667384in}{5.583062in}}%
\pgfpathlineto{\pgfqpoint{8.690390in}{5.577848in}}%
\pgfpathlineto{\pgfqpoint{8.713395in}{5.572628in}}%
\pgfpathlineto{\pgfqpoint{8.736401in}{5.567398in}}%
\pgfpathlineto{\pgfqpoint{8.759406in}{5.562153in}}%
\pgfpathlineto{\pgfqpoint{8.782412in}{5.556889in}}%
\pgfpathlineto{\pgfqpoint{8.805417in}{5.551604in}}%
\pgfpathlineto{\pgfqpoint{8.828423in}{5.546294in}}%
\pgfpathlineto{\pgfqpoint{8.851429in}{5.540955in}}%
\pgfpathlineto{\pgfqpoint{8.874434in}{5.535584in}}%
\pgfpathlineto{\pgfqpoint{8.897440in}{5.530177in}}%
\pgfpathlineto{\pgfqpoint{8.920445in}{5.524732in}}%
\pgfpathlineto{\pgfqpoint{8.943451in}{5.519246in}}%
\pgfpathlineto{\pgfqpoint{8.966456in}{5.513714in}}%
\pgfpathlineto{\pgfqpoint{8.989462in}{5.508134in}}%
\pgfpathlineto{\pgfqpoint{9.012468in}{5.502501in}}%
\pgfpathlineto{\pgfqpoint{9.035473in}{5.496813in}}%
\pgfpathlineto{\pgfqpoint{9.058479in}{5.491066in}}%
\pgfpathlineto{\pgfqpoint{9.081484in}{5.485256in}}%
\pgfpathlineto{\pgfqpoint{9.104490in}{5.479378in}}%
\pgfpathlineto{\pgfqpoint{9.127495in}{5.473429in}}%
\pgfpathlineto{\pgfqpoint{9.150501in}{5.467405in}}%
\pgfpathlineto{\pgfqpoint{9.173506in}{5.461300in}}%
\pgfpathlineto{\pgfqpoint{9.196512in}{5.455111in}}%
\pgfpathlineto{\pgfqpoint{9.219518in}{5.448831in}}%
\pgfpathlineto{\pgfqpoint{9.242523in}{5.442456in}}%
\pgfpathlineto{\pgfqpoint{9.265529in}{5.435979in}}%
\pgfpathlineto{\pgfqpoint{9.288534in}{5.429395in}}%
\pgfpathlineto{\pgfqpoint{9.311540in}{5.422697in}}%
\pgfpathlineto{\pgfqpoint{9.334545in}{5.415878in}}%
\pgfpathlineto{\pgfqpoint{9.357551in}{5.408931in}}%
\pgfpathlineto{\pgfqpoint{9.380557in}{5.401848in}}%
\pgfpathlineto{\pgfqpoint{9.403562in}{5.394619in}}%
\pgfpathlineto{\pgfqpoint{9.426568in}{5.387237in}}%
\pgfpathlineto{\pgfqpoint{9.449573in}{5.379690in}}%
\pgfpathlineto{\pgfqpoint{9.472579in}{5.371967in}}%
\pgfpathlineto{\pgfqpoint{9.495584in}{5.364058in}}%
\pgfpathlineto{\pgfqpoint{9.518590in}{5.355948in}}%
\pgfpathlineto{\pgfqpoint{9.541596in}{5.347623in}}%
\pgfpathlineto{\pgfqpoint{9.564601in}{5.339068in}}%
\pgfpathlineto{\pgfqpoint{9.587607in}{5.330266in}}%
\pgfpathlineto{\pgfqpoint{9.610612in}{5.321197in}}%
\pgfpathlineto{\pgfqpoint{9.633618in}{5.311840in}}%
\pgfpathlineto{\pgfqpoint{9.656623in}{5.302171in}}%
\pgfpathlineto{\pgfqpoint{9.679629in}{5.292163in}}%
\pgfpathlineto{\pgfqpoint{9.702635in}{5.281787in}}%
\pgfpathlineto{\pgfqpoint{9.725640in}{5.271008in}}%
\pgfpathlineto{\pgfqpoint{9.748646in}{5.259789in}}%
\pgfpathlineto{\pgfqpoint{9.771651in}{5.248084in}}%
\pgfpathlineto{\pgfqpoint{9.794657in}{5.235842in}}%
\pgfpathlineto{\pgfqpoint{9.817662in}{5.223004in}}%
\pgfpathlineto{\pgfqpoint{9.840668in}{5.209501in}}%
\pgfpathlineto{\pgfqpoint{9.863673in}{5.195249in}}%
\pgfpathlineto{\pgfqpoint{9.886679in}{5.180152in}}%
\pgfpathlineto{\pgfqpoint{9.909685in}{5.164090in}}%
\pgfpathlineto{\pgfqpoint{9.932690in}{5.146918in}}%
\pgfpathlineto{\pgfqpoint{9.955696in}{5.128456in}}%
\pgfpathlineto{\pgfqpoint{9.978701in}{5.108476in}}%
\pgfpathlineto{\pgfqpoint{10.001707in}{5.086685in}}%
\pgfpathlineto{\pgfqpoint{10.024712in}{5.062693in}}%
\pgfpathlineto{\pgfqpoint{10.047718in}{5.035975in}}%
\pgfpathlineto{\pgfqpoint{10.070724in}{5.005784in}}%
\pgfpathlineto{\pgfqpoint{10.093729in}{4.971024in}}%
\pgfusepath{stroke}%
\end{pgfscope}%
\begin{pgfscope}%
\pgfpathrectangle{\pgfqpoint{8.282041in}{4.908628in}}{\pgfqpoint{1.897959in}{1.372727in}} %
\pgfusepath{clip}%
\pgfsetbuttcap%
\pgfsetroundjoin%
\definecolor{currentfill}{rgb}{0.000000,0.000000,1.000000}%
\pgfsetfillcolor{currentfill}%
\pgfsetlinewidth{1.003750pt}%
\definecolor{currentstroke}{rgb}{0.000000,0.000000,1.000000}%
\pgfsetstrokecolor{currentstroke}%
\pgfsetdash{}{0pt}%
\pgfsys@defobject{currentmarker}{\pgfqpoint{-0.041667in}{-0.041667in}}{\pgfqpoint{0.041667in}{0.041667in}}{%
\pgfpathmoveto{\pgfqpoint{0.000000in}{-0.041667in}}%
\pgfpathcurveto{\pgfqpoint{0.011050in}{-0.041667in}}{\pgfqpoint{0.021649in}{-0.037276in}}{\pgfqpoint{0.029463in}{-0.029463in}}%
\pgfpathcurveto{\pgfqpoint{0.037276in}{-0.021649in}}{\pgfqpoint{0.041667in}{-0.011050in}}{\pgfqpoint{0.041667in}{0.000000in}}%
\pgfpathcurveto{\pgfqpoint{0.041667in}{0.011050in}}{\pgfqpoint{0.037276in}{0.021649in}}{\pgfqpoint{0.029463in}{0.029463in}}%
\pgfpathcurveto{\pgfqpoint{0.021649in}{0.037276in}}{\pgfqpoint{0.011050in}{0.041667in}}{\pgfqpoint{0.000000in}{0.041667in}}%
\pgfpathcurveto{\pgfqpoint{-0.011050in}{0.041667in}}{\pgfqpoint{-0.021649in}{0.037276in}}{\pgfqpoint{-0.029463in}{0.029463in}}%
\pgfpathcurveto{\pgfqpoint{-0.037276in}{0.021649in}}{\pgfqpoint{-0.041667in}{0.011050in}}{\pgfqpoint{-0.041667in}{0.000000in}}%
\pgfpathcurveto{\pgfqpoint{-0.041667in}{-0.011050in}}{\pgfqpoint{-0.037276in}{-0.021649in}}{\pgfqpoint{-0.029463in}{-0.029463in}}%
\pgfpathcurveto{\pgfqpoint{-0.021649in}{-0.037276in}}{\pgfqpoint{-0.011050in}{-0.041667in}}{\pgfqpoint{0.000000in}{-0.041667in}}%
\pgfpathclose%
\pgfusepath{stroke,fill}%
}%
\begin{pgfscope}%
\pgfsys@transformshift{8.368312in}{5.654281in}%
\pgfsys@useobject{currentmarker}{}%
\end{pgfscope}%
\begin{pgfscope}%
\pgfsys@transformshift{8.713395in}{5.572628in}%
\pgfsys@useobject{currentmarker}{}%
\end{pgfscope}%
\begin{pgfscope}%
\pgfsys@transformshift{9.058479in}{5.491066in}%
\pgfsys@useobject{currentmarker}{}%
\end{pgfscope}%
\begin{pgfscope}%
\pgfsys@transformshift{9.403562in}{5.394619in}%
\pgfsys@useobject{currentmarker}{}%
\end{pgfscope}%
\begin{pgfscope}%
\pgfsys@transformshift{9.748646in}{5.259789in}%
\pgfsys@useobject{currentmarker}{}%
\end{pgfscope}%
\begin{pgfscope}%
\pgfsys@transformshift{10.093729in}{4.971024in}%
\pgfsys@useobject{currentmarker}{}%
\end{pgfscope}%
\end{pgfscope}%
\begin{pgfscope}%
\pgfpathrectangle{\pgfqpoint{8.282041in}{4.908628in}}{\pgfqpoint{1.897959in}{1.372727in}} %
\pgfusepath{clip}%
\pgfsetbuttcap%
\pgfsetroundjoin%
\pgfsetlinewidth{1.505625pt}%
\definecolor{currentstroke}{rgb}{0.000000,0.750000,0.750000}%
\pgfsetstrokecolor{currentstroke}%
\pgfsetdash{{9.600000pt}{2.400000pt}{1.500000pt}{2.400000pt}}{0.000000pt}%
\pgfpathmoveto{\pgfqpoint{8.368312in}{5.987072in}}%
\pgfpathlineto{\pgfqpoint{8.391317in}{5.945048in}}%
\pgfpathlineto{\pgfqpoint{8.414323in}{5.907873in}}%
\pgfpathlineto{\pgfqpoint{8.437328in}{5.874598in}}%
\pgfpathlineto{\pgfqpoint{8.460334in}{5.844531in}}%
\pgfpathlineto{\pgfqpoint{8.483340in}{5.817148in}}%
\pgfpathlineto{\pgfqpoint{8.506345in}{5.792048in}}%
\pgfpathlineto{\pgfqpoint{8.529351in}{5.768910in}}%
\pgfpathlineto{\pgfqpoint{8.552356in}{5.747481in}}%
\pgfpathlineto{\pgfqpoint{8.575362in}{5.727551in}}%
\pgfpathlineto{\pgfqpoint{8.598367in}{5.708949in}}%
\pgfpathlineto{\pgfqpoint{8.621373in}{5.691530in}}%
\pgfpathlineto{\pgfqpoint{8.644378in}{5.675173in}}%
\pgfpathlineto{\pgfqpoint{8.667384in}{5.659774in}}%
\pgfpathlineto{\pgfqpoint{8.690390in}{5.645245in}}%
\pgfpathlineto{\pgfqpoint{8.713395in}{5.631509in}}%
\pgfpathlineto{\pgfqpoint{8.736401in}{5.618498in}}%
\pgfpathlineto{\pgfqpoint{8.759406in}{5.606154in}}%
\pgfpathlineto{\pgfqpoint{8.782412in}{5.594425in}}%
\pgfpathlineto{\pgfqpoint{8.805417in}{5.583264in}}%
\pgfpathlineto{\pgfqpoint{8.828423in}{5.572632in}}%
\pgfpathlineto{\pgfqpoint{8.851429in}{5.562492in}}%
\pgfpathlineto{\pgfqpoint{8.874434in}{5.552811in}}%
\pgfpathlineto{\pgfqpoint{8.897440in}{5.543559in}}%
\pgfpathlineto{\pgfqpoint{8.920445in}{5.534710in}}%
\pgfpathlineto{\pgfqpoint{8.943451in}{5.526240in}}%
\pgfpathlineto{\pgfqpoint{8.966456in}{5.518127in}}%
\pgfpathlineto{\pgfqpoint{8.989462in}{5.510350in}}%
\pgfpathlineto{\pgfqpoint{9.012468in}{5.502892in}}%
\pgfpathlineto{\pgfqpoint{9.035473in}{5.495736in}}%
\pgfpathlineto{\pgfqpoint{9.058479in}{5.488865in}}%
\pgfpathlineto{\pgfqpoint{9.081484in}{5.482268in}}%
\pgfpathlineto{\pgfqpoint{9.104490in}{5.475929in}}%
\pgfpathlineto{\pgfqpoint{9.127495in}{5.469837in}}%
\pgfpathlineto{\pgfqpoint{9.150501in}{5.463981in}}%
\pgfpathlineto{\pgfqpoint{9.173506in}{5.458351in}}%
\pgfpathlineto{\pgfqpoint{9.196512in}{5.452937in}}%
\pgfpathlineto{\pgfqpoint{9.219518in}{5.447730in}}%
\pgfpathlineto{\pgfqpoint{9.242523in}{5.442721in}}%
\pgfpathlineto{\pgfqpoint{9.265529in}{5.437904in}}%
\pgfpathlineto{\pgfqpoint{9.288534in}{5.433270in}}%
\pgfpathlineto{\pgfqpoint{9.311540in}{5.428813in}}%
\pgfpathlineto{\pgfqpoint{9.334545in}{5.424527in}}%
\pgfpathlineto{\pgfqpoint{9.357551in}{5.420406in}}%
\pgfpathlineto{\pgfqpoint{9.380557in}{5.416444in}}%
\pgfpathlineto{\pgfqpoint{9.403562in}{5.412636in}}%
\pgfpathlineto{\pgfqpoint{9.426568in}{5.408978in}}%
\pgfpathlineto{\pgfqpoint{9.449573in}{5.405464in}}%
\pgfpathlineto{\pgfqpoint{9.472579in}{5.402091in}}%
\pgfpathlineto{\pgfqpoint{9.495584in}{5.398853in}}%
\pgfpathlineto{\pgfqpoint{9.518590in}{5.395749in}}%
\pgfpathlineto{\pgfqpoint{9.541596in}{5.392773in}}%
\pgfpathlineto{\pgfqpoint{9.564601in}{5.389923in}}%
\pgfpathlineto{\pgfqpoint{9.587607in}{5.387196in}}%
\pgfpathlineto{\pgfqpoint{9.610612in}{5.384588in}}%
\pgfpathlineto{\pgfqpoint{9.633618in}{5.382097in}}%
\pgfpathlineto{\pgfqpoint{9.656623in}{5.379720in}}%
\pgfpathlineto{\pgfqpoint{9.679629in}{5.377454in}}%
\pgfpathlineto{\pgfqpoint{9.702635in}{5.375298in}}%
\pgfpathlineto{\pgfqpoint{9.725640in}{5.373249in}}%
\pgfpathlineto{\pgfqpoint{9.748646in}{5.371305in}}%
\pgfpathlineto{\pgfqpoint{9.771651in}{5.369464in}}%
\pgfpathlineto{\pgfqpoint{9.794657in}{5.367725in}}%
\pgfpathlineto{\pgfqpoint{9.817662in}{5.366085in}}%
\pgfpathlineto{\pgfqpoint{9.840668in}{5.364544in}}%
\pgfpathlineto{\pgfqpoint{9.863673in}{5.363099in}}%
\pgfpathlineto{\pgfqpoint{9.886679in}{5.361749in}}%
\pgfpathlineto{\pgfqpoint{9.909685in}{5.360493in}}%
\pgfpathlineto{\pgfqpoint{9.932690in}{5.359331in}}%
\pgfpathlineto{\pgfqpoint{9.955696in}{5.358259in}}%
\pgfpathlineto{\pgfqpoint{9.978701in}{5.357279in}}%
\pgfpathlineto{\pgfqpoint{10.001707in}{5.356389in}}%
\pgfpathlineto{\pgfqpoint{10.024712in}{5.355587in}}%
\pgfpathlineto{\pgfqpoint{10.047718in}{5.354874in}}%
\pgfpathlineto{\pgfqpoint{10.070724in}{5.354248in}}%
\pgfpathlineto{\pgfqpoint{10.093729in}{5.353710in}}%
\pgfusepath{stroke}%
\end{pgfscope}%
\begin{pgfscope}%
\pgfpathrectangle{\pgfqpoint{8.282041in}{4.908628in}}{\pgfqpoint{1.897959in}{1.372727in}} %
\pgfusepath{clip}%
\pgfsetbuttcap%
\pgfsetmiterjoin%
\definecolor{currentfill}{rgb}{0.000000,0.750000,0.750000}%
\pgfsetfillcolor{currentfill}%
\pgfsetlinewidth{1.003750pt}%
\definecolor{currentstroke}{rgb}{0.000000,0.750000,0.750000}%
\pgfsetstrokecolor{currentstroke}%
\pgfsetdash{}{0pt}%
\pgfsys@defobject{currentmarker}{\pgfqpoint{-0.041667in}{-0.041667in}}{\pgfqpoint{0.041667in}{0.041667in}}{%
\pgfpathmoveto{\pgfqpoint{-0.000000in}{-0.041667in}}%
\pgfpathlineto{\pgfqpoint{0.041667in}{0.041667in}}%
\pgfpathlineto{\pgfqpoint{-0.041667in}{0.041667in}}%
\pgfpathclose%
\pgfusepath{stroke,fill}%
}%
\begin{pgfscope}%
\pgfsys@transformshift{8.368312in}{5.987072in}%
\pgfsys@useobject{currentmarker}{}%
\end{pgfscope}%
\begin{pgfscope}%
\pgfsys@transformshift{8.713395in}{5.631509in}%
\pgfsys@useobject{currentmarker}{}%
\end{pgfscope}%
\begin{pgfscope}%
\pgfsys@transformshift{9.058479in}{5.488865in}%
\pgfsys@useobject{currentmarker}{}%
\end{pgfscope}%
\begin{pgfscope}%
\pgfsys@transformshift{9.403562in}{5.412636in}%
\pgfsys@useobject{currentmarker}{}%
\end{pgfscope}%
\begin{pgfscope}%
\pgfsys@transformshift{9.748646in}{5.371305in}%
\pgfsys@useobject{currentmarker}{}%
\end{pgfscope}%
\begin{pgfscope}%
\pgfsys@transformshift{10.093729in}{5.353710in}%
\pgfsys@useobject{currentmarker}{}%
\end{pgfscope}%
\end{pgfscope}%
\begin{pgfscope}%
\pgfpathrectangle{\pgfqpoint{8.282041in}{4.908628in}}{\pgfqpoint{1.897959in}{1.372727in}} %
\pgfusepath{clip}%
\pgfsetbuttcap%
\pgfsetroundjoin%
\pgfsetlinewidth{1.505625pt}%
\definecolor{currentstroke}{rgb}{0.000000,0.000000,0.000000}%
\pgfsetstrokecolor{currentstroke}%
\pgfsetdash{{1.500000pt}{2.475000pt}}{0.000000pt}%
\pgfpathmoveto{\pgfqpoint{8.368312in}{6.218958in}}%
\pgfpathlineto{\pgfqpoint{8.391317in}{6.206345in}}%
\pgfpathlineto{\pgfqpoint{8.414323in}{6.194429in}}%
\pgfpathlineto{\pgfqpoint{8.437328in}{6.184498in}}%
\pgfpathlineto{\pgfqpoint{8.460334in}{6.176016in}}%
\pgfpathlineto{\pgfqpoint{8.483340in}{6.168623in}}%
\pgfpathlineto{\pgfqpoint{8.506345in}{6.162067in}}%
\pgfpathlineto{\pgfqpoint{8.529351in}{6.156166in}}%
\pgfpathlineto{\pgfqpoint{8.552356in}{6.150788in}}%
\pgfpathlineto{\pgfqpoint{8.575362in}{6.145834in}}%
\pgfpathlineto{\pgfqpoint{8.598367in}{6.141231in}}%
\pgfpathlineto{\pgfqpoint{8.621373in}{6.136921in}}%
\pgfpathlineto{\pgfqpoint{8.644378in}{6.132858in}}%
\pgfpathlineto{\pgfqpoint{8.667384in}{6.129009in}}%
\pgfpathlineto{\pgfqpoint{8.690390in}{6.125346in}}%
\pgfpathlineto{\pgfqpoint{8.713395in}{6.121844in}}%
\pgfpathlineto{\pgfqpoint{8.736401in}{6.118487in}}%
\pgfpathlineto{\pgfqpoint{8.759406in}{6.115259in}}%
\pgfpathlineto{\pgfqpoint{8.782412in}{6.112147in}}%
\pgfpathlineto{\pgfqpoint{8.805417in}{6.109142in}}%
\pgfpathlineto{\pgfqpoint{8.828423in}{6.106234in}}%
\pgfpathlineto{\pgfqpoint{8.851429in}{6.103417in}}%
\pgfpathlineto{\pgfqpoint{8.874434in}{6.100683in}}%
\pgfpathlineto{\pgfqpoint{8.897440in}{6.098028in}}%
\pgfpathlineto{\pgfqpoint{8.920445in}{6.095446in}}%
\pgfpathlineto{\pgfqpoint{8.943451in}{6.092934in}}%
\pgfpathlineto{\pgfqpoint{8.966456in}{6.090489in}}%
\pgfpathlineto{\pgfqpoint{8.989462in}{6.088107in}}%
\pgfpathlineto{\pgfqpoint{9.012468in}{6.085786in}}%
\pgfpathlineto{\pgfqpoint{9.035473in}{6.083523in}}%
\pgfpathlineto{\pgfqpoint{9.058479in}{6.081317in}}%
\pgfpathlineto{\pgfqpoint{9.081484in}{6.079165in}}%
\pgfpathlineto{\pgfqpoint{9.104490in}{6.077065in}}%
\pgfpathlineto{\pgfqpoint{9.127495in}{6.075016in}}%
\pgfpathlineto{\pgfqpoint{9.150501in}{6.073018in}}%
\pgfpathlineto{\pgfqpoint{9.173506in}{6.071068in}}%
\pgfpathlineto{\pgfqpoint{9.196512in}{6.069165in}}%
\pgfpathlineto{\pgfqpoint{9.219518in}{6.067309in}}%
\pgfpathlineto{\pgfqpoint{9.242523in}{6.065499in}}%
\pgfpathlineto{\pgfqpoint{9.265529in}{6.063733in}}%
\pgfpathlineto{\pgfqpoint{9.288534in}{6.062011in}}%
\pgfpathlineto{\pgfqpoint{9.311540in}{6.060333in}}%
\pgfpathlineto{\pgfqpoint{9.334545in}{6.058697in}}%
\pgfpathlineto{\pgfqpoint{9.357551in}{6.057103in}}%
\pgfpathlineto{\pgfqpoint{9.380557in}{6.055551in}}%
\pgfpathlineto{\pgfqpoint{9.403562in}{6.054039in}}%
\pgfpathlineto{\pgfqpoint{9.426568in}{6.052569in}}%
\pgfpathlineto{\pgfqpoint{9.449573in}{6.051138in}}%
\pgfpathlineto{\pgfqpoint{9.472579in}{6.049746in}}%
\pgfpathlineto{\pgfqpoint{9.495584in}{6.048394in}}%
\pgfpathlineto{\pgfqpoint{9.518590in}{6.047081in}}%
\pgfpathlineto{\pgfqpoint{9.541596in}{6.045806in}}%
\pgfpathlineto{\pgfqpoint{9.564601in}{6.044570in}}%
\pgfpathlineto{\pgfqpoint{9.587607in}{6.043371in}}%
\pgfpathlineto{\pgfqpoint{9.610612in}{6.042211in}}%
\pgfpathlineto{\pgfqpoint{9.633618in}{6.041087in}}%
\pgfpathlineto{\pgfqpoint{9.656623in}{6.040001in}}%
\pgfpathlineto{\pgfqpoint{9.679629in}{6.038952in}}%
\pgfpathlineto{\pgfqpoint{9.702635in}{6.037940in}}%
\pgfpathlineto{\pgfqpoint{9.725640in}{6.036964in}}%
\pgfpathlineto{\pgfqpoint{9.748646in}{6.036025in}}%
\pgfpathlineto{\pgfqpoint{9.771651in}{6.035123in}}%
\pgfpathlineto{\pgfqpoint{9.794657in}{6.034256in}}%
\pgfpathlineto{\pgfqpoint{9.817662in}{6.033426in}}%
\pgfpathlineto{\pgfqpoint{9.840668in}{6.032632in}}%
\pgfpathlineto{\pgfqpoint{9.863673in}{6.031874in}}%
\pgfpathlineto{\pgfqpoint{9.886679in}{6.031152in}}%
\pgfpathlineto{\pgfqpoint{9.909685in}{6.030465in}}%
\pgfpathlineto{\pgfqpoint{9.932690in}{6.029815in}}%
\pgfpathlineto{\pgfqpoint{9.955696in}{6.029200in}}%
\pgfpathlineto{\pgfqpoint{9.978701in}{6.028622in}}%
\pgfpathlineto{\pgfqpoint{10.001707in}{6.028079in}}%
\pgfpathlineto{\pgfqpoint{10.024712in}{6.027573in}}%
\pgfpathlineto{\pgfqpoint{10.047718in}{6.027102in}}%
\pgfpathlineto{\pgfqpoint{10.070724in}{6.026668in}}%
\pgfpathlineto{\pgfqpoint{10.093729in}{6.026270in}}%
\pgfusepath{stroke}%
\end{pgfscope}%
\begin{pgfscope}%
\pgfpathrectangle{\pgfqpoint{8.282041in}{4.908628in}}{\pgfqpoint{1.897959in}{1.372727in}} %
\pgfusepath{clip}%
\pgfsetbuttcap%
\pgfsetroundjoin%
\definecolor{currentfill}{rgb}{0.000000,0.000000,0.000000}%
\pgfsetfillcolor{currentfill}%
\pgfsetlinewidth{1.003750pt}%
\definecolor{currentstroke}{rgb}{0.000000,0.000000,0.000000}%
\pgfsetstrokecolor{currentstroke}%
\pgfsetdash{}{0pt}%
\pgfsys@defobject{currentmarker}{\pgfqpoint{-0.041667in}{-0.041667in}}{\pgfqpoint{0.041667in}{0.041667in}}{%
\pgfpathmoveto{\pgfqpoint{-0.041667in}{0.000000in}}%
\pgfpathlineto{\pgfqpoint{0.041667in}{0.000000in}}%
\pgfpathmoveto{\pgfqpoint{0.000000in}{-0.041667in}}%
\pgfpathlineto{\pgfqpoint{0.000000in}{0.041667in}}%
\pgfusepath{stroke,fill}%
}%
\begin{pgfscope}%
\pgfsys@transformshift{8.368312in}{6.218958in}%
\pgfsys@useobject{currentmarker}{}%
\end{pgfscope}%
\begin{pgfscope}%
\pgfsys@transformshift{8.713395in}{6.121844in}%
\pgfsys@useobject{currentmarker}{}%
\end{pgfscope}%
\begin{pgfscope}%
\pgfsys@transformshift{9.058479in}{6.081317in}%
\pgfsys@useobject{currentmarker}{}%
\end{pgfscope}%
\begin{pgfscope}%
\pgfsys@transformshift{9.403562in}{6.054039in}%
\pgfsys@useobject{currentmarker}{}%
\end{pgfscope}%
\begin{pgfscope}%
\pgfsys@transformshift{9.748646in}{6.036025in}%
\pgfsys@useobject{currentmarker}{}%
\end{pgfscope}%
\begin{pgfscope}%
\pgfsys@transformshift{10.093729in}{6.026270in}%
\pgfsys@useobject{currentmarker}{}%
\end{pgfscope}%
\end{pgfscope}%
\begin{pgfscope}%
\pgfsetrectcap%
\pgfsetmiterjoin%
\pgfsetlinewidth{0.803000pt}%
\definecolor{currentstroke}{rgb}{0.000000,0.000000,0.000000}%
\pgfsetstrokecolor{currentstroke}%
\pgfsetdash{}{0pt}%
\pgfpathmoveto{\pgfqpoint{8.282041in}{4.908628in}}%
\pgfpathlineto{\pgfqpoint{8.282041in}{6.281355in}}%
\pgfusepath{stroke}%
\end{pgfscope}%
\begin{pgfscope}%
\pgfsetrectcap%
\pgfsetmiterjoin%
\pgfsetlinewidth{0.803000pt}%
\definecolor{currentstroke}{rgb}{0.000000,0.000000,0.000000}%
\pgfsetstrokecolor{currentstroke}%
\pgfsetdash{}{0pt}%
\pgfpathmoveto{\pgfqpoint{10.180000in}{4.908628in}}%
\pgfpathlineto{\pgfqpoint{10.180000in}{6.281355in}}%
\pgfusepath{stroke}%
\end{pgfscope}%
\begin{pgfscope}%
\pgfsetrectcap%
\pgfsetmiterjoin%
\pgfsetlinewidth{0.803000pt}%
\definecolor{currentstroke}{rgb}{0.000000,0.000000,0.000000}%
\pgfsetstrokecolor{currentstroke}%
\pgfsetdash{}{0pt}%
\pgfpathmoveto{\pgfqpoint{8.282041in}{4.908628in}}%
\pgfpathlineto{\pgfqpoint{10.180000in}{4.908628in}}%
\pgfusepath{stroke}%
\end{pgfscope}%
\begin{pgfscope}%
\pgfsetrectcap%
\pgfsetmiterjoin%
\pgfsetlinewidth{0.803000pt}%
\definecolor{currentstroke}{rgb}{0.000000,0.000000,0.000000}%
\pgfsetstrokecolor{currentstroke}%
\pgfsetdash{}{0pt}%
\pgfpathmoveto{\pgfqpoint{8.282041in}{6.281355in}}%
\pgfpathlineto{\pgfqpoint{10.180000in}{6.281355in}}%
\pgfusepath{stroke}%
\end{pgfscope}%
\begin{pgfscope}%
\pgfsetbuttcap%
\pgfsetmiterjoin%
\definecolor{currentfill}{rgb}{1.000000,1.000000,1.000000}%
\pgfsetfillcolor{currentfill}%
\pgfsetlinewidth{0.000000pt}%
\definecolor{currentstroke}{rgb}{0.000000,0.000000,0.000000}%
\pgfsetstrokecolor{currentstroke}%
\pgfsetstrokeopacity{0.000000}%
\pgfsetdash{}{0pt}%
\pgfpathmoveto{\pgfqpoint{0.880000in}{2.849537in}}%
\pgfpathlineto{\pgfqpoint{2.777959in}{2.849537in}}%
\pgfpathlineto{\pgfqpoint{2.777959in}{4.222264in}}%
\pgfpathlineto{\pgfqpoint{0.880000in}{4.222264in}}%
\pgfpathclose%
\pgfusepath{fill}%
\end{pgfscope}%
\begin{pgfscope}%
\pgfsetbuttcap%
\pgfsetroundjoin%
\definecolor{currentfill}{rgb}{0.000000,0.000000,0.000000}%
\pgfsetfillcolor{currentfill}%
\pgfsetlinewidth{0.803000pt}%
\definecolor{currentstroke}{rgb}{0.000000,0.000000,0.000000}%
\pgfsetstrokecolor{currentstroke}%
\pgfsetdash{}{0pt}%
\pgfsys@defobject{currentmarker}{\pgfqpoint{0.000000in}{-0.048611in}}{\pgfqpoint{0.000000in}{0.000000in}}{%
\pgfpathmoveto{\pgfqpoint{0.000000in}{0.000000in}}%
\pgfpathlineto{\pgfqpoint{0.000000in}{-0.048611in}}%
\pgfusepath{stroke,fill}%
}%
\begin{pgfscope}%
\pgfsys@transformshift{1.360652in}{2.849537in}%
\pgfsys@useobject{currentmarker}{}%
\end{pgfscope}%
\end{pgfscope}%
\begin{pgfscope}%
\pgftext[x=1.360652in,y=2.752315in,,top]{\rmfamily\fontsize{10.000000}{12.000000}\selectfont \(\displaystyle 0.2\)}%
\end{pgfscope}%
\begin{pgfscope}%
\pgfsetbuttcap%
\pgfsetroundjoin%
\definecolor{currentfill}{rgb}{0.000000,0.000000,0.000000}%
\pgfsetfillcolor{currentfill}%
\pgfsetlinewidth{0.803000pt}%
\definecolor{currentstroke}{rgb}{0.000000,0.000000,0.000000}%
\pgfsetstrokecolor{currentstroke}%
\pgfsetdash{}{0pt}%
\pgfsys@defobject{currentmarker}{\pgfqpoint{0.000000in}{-0.048611in}}{\pgfqpoint{0.000000in}{0.000000in}}{%
\pgfpathmoveto{\pgfqpoint{0.000000in}{0.000000in}}%
\pgfpathlineto{\pgfqpoint{0.000000in}{-0.048611in}}%
\pgfusepath{stroke,fill}%
}%
\begin{pgfscope}%
\pgfsys@transformshift{1.952224in}{2.849537in}%
\pgfsys@useobject{currentmarker}{}%
\end{pgfscope}%
\end{pgfscope}%
\begin{pgfscope}%
\pgftext[x=1.952224in,y=2.752315in,,top]{\rmfamily\fontsize{10.000000}{12.000000}\selectfont \(\displaystyle 0.4\)}%
\end{pgfscope}%
\begin{pgfscope}%
\pgfsetbuttcap%
\pgfsetroundjoin%
\definecolor{currentfill}{rgb}{0.000000,0.000000,0.000000}%
\pgfsetfillcolor{currentfill}%
\pgfsetlinewidth{0.803000pt}%
\definecolor{currentstroke}{rgb}{0.000000,0.000000,0.000000}%
\pgfsetstrokecolor{currentstroke}%
\pgfsetdash{}{0pt}%
\pgfsys@defobject{currentmarker}{\pgfqpoint{0.000000in}{-0.048611in}}{\pgfqpoint{0.000000in}{0.000000in}}{%
\pgfpathmoveto{\pgfqpoint{0.000000in}{0.000000in}}%
\pgfpathlineto{\pgfqpoint{0.000000in}{-0.048611in}}%
\pgfusepath{stroke,fill}%
}%
\begin{pgfscope}%
\pgfsys@transformshift{2.543795in}{2.849537in}%
\pgfsys@useobject{currentmarker}{}%
\end{pgfscope}%
\end{pgfscope}%
\begin{pgfscope}%
\pgftext[x=2.543795in,y=2.752315in,,top]{\rmfamily\fontsize{10.000000}{12.000000}\selectfont \(\displaystyle 0.6\)}%
\end{pgfscope}%
\begin{pgfscope}%
\pgfsetbuttcap%
\pgfsetroundjoin%
\definecolor{currentfill}{rgb}{0.000000,0.000000,0.000000}%
\pgfsetfillcolor{currentfill}%
\pgfsetlinewidth{0.803000pt}%
\definecolor{currentstroke}{rgb}{0.000000,0.000000,0.000000}%
\pgfsetstrokecolor{currentstroke}%
\pgfsetdash{}{0pt}%
\pgfsys@defobject{currentmarker}{\pgfqpoint{-0.048611in}{0.000000in}}{\pgfqpoint{0.000000in}{0.000000in}}{%
\pgfpathmoveto{\pgfqpoint{0.000000in}{0.000000in}}%
\pgfpathlineto{\pgfqpoint{-0.048611in}{0.000000in}}%
\pgfusepath{stroke,fill}%
}%
\begin{pgfscope}%
\pgfsys@transformshift{0.880000in}{3.238450in}%
\pgfsys@useobject{currentmarker}{}%
\end{pgfscope}%
\end{pgfscope}%
\begin{pgfscope}%
\pgftext[x=0.494775in,y=3.185688in,left,base]{\rmfamily\fontsize{10.000000}{12.000000}\selectfont \(\displaystyle 10^{-7}\)}%
\end{pgfscope}%
\begin{pgfscope}%
\pgfsetbuttcap%
\pgfsetroundjoin%
\definecolor{currentfill}{rgb}{0.000000,0.000000,0.000000}%
\pgfsetfillcolor{currentfill}%
\pgfsetlinewidth{0.803000pt}%
\definecolor{currentstroke}{rgb}{0.000000,0.000000,0.000000}%
\pgfsetstrokecolor{currentstroke}%
\pgfsetdash{}{0pt}%
\pgfsys@defobject{currentmarker}{\pgfqpoint{-0.048611in}{0.000000in}}{\pgfqpoint{0.000000in}{0.000000in}}{%
\pgfpathmoveto{\pgfqpoint{0.000000in}{0.000000in}}%
\pgfpathlineto{\pgfqpoint{-0.048611in}{0.000000in}}%
\pgfusepath{stroke,fill}%
}%
\begin{pgfscope}%
\pgfsys@transformshift{0.880000in}{3.649957in}%
\pgfsys@useobject{currentmarker}{}%
\end{pgfscope}%
\end{pgfscope}%
\begin{pgfscope}%
\pgftext[x=0.494775in,y=3.597196in,left,base]{\rmfamily\fontsize{10.000000}{12.000000}\selectfont \(\displaystyle 10^{-6}\)}%
\end{pgfscope}%
\begin{pgfscope}%
\pgfsetbuttcap%
\pgfsetroundjoin%
\definecolor{currentfill}{rgb}{0.000000,0.000000,0.000000}%
\pgfsetfillcolor{currentfill}%
\pgfsetlinewidth{0.803000pt}%
\definecolor{currentstroke}{rgb}{0.000000,0.000000,0.000000}%
\pgfsetstrokecolor{currentstroke}%
\pgfsetdash{}{0pt}%
\pgfsys@defobject{currentmarker}{\pgfqpoint{-0.048611in}{0.000000in}}{\pgfqpoint{0.000000in}{0.000000in}}{%
\pgfpathmoveto{\pgfqpoint{0.000000in}{0.000000in}}%
\pgfpathlineto{\pgfqpoint{-0.048611in}{0.000000in}}%
\pgfusepath{stroke,fill}%
}%
\begin{pgfscope}%
\pgfsys@transformshift{0.880000in}{4.061464in}%
\pgfsys@useobject{currentmarker}{}%
\end{pgfscope}%
\end{pgfscope}%
\begin{pgfscope}%
\pgftext[x=0.494775in,y=4.008703in,left,base]{\rmfamily\fontsize{10.000000}{12.000000}\selectfont \(\displaystyle 10^{-5}\)}%
\end{pgfscope}%
\begin{pgfscope}%
\pgfsetbuttcap%
\pgfsetroundjoin%
\definecolor{currentfill}{rgb}{0.000000,0.000000,0.000000}%
\pgfsetfillcolor{currentfill}%
\pgfsetlinewidth{0.602250pt}%
\definecolor{currentstroke}{rgb}{0.000000,0.000000,0.000000}%
\pgfsetstrokecolor{currentstroke}%
\pgfsetdash{}{0pt}%
\pgfsys@defobject{currentmarker}{\pgfqpoint{-0.027778in}{0.000000in}}{\pgfqpoint{0.000000in}{0.000000in}}{%
\pgfpathmoveto{\pgfqpoint{0.000000in}{0.000000in}}%
\pgfpathlineto{\pgfqpoint{-0.027778in}{0.000000in}}%
\pgfusepath{stroke,fill}%
}%
\begin{pgfscope}%
\pgfsys@transformshift{0.880000in}{2.950819in}%
\pgfsys@useobject{currentmarker}{}%
\end{pgfscope}%
\end{pgfscope}%
\begin{pgfscope}%
\pgfsetbuttcap%
\pgfsetroundjoin%
\definecolor{currentfill}{rgb}{0.000000,0.000000,0.000000}%
\pgfsetfillcolor{currentfill}%
\pgfsetlinewidth{0.602250pt}%
\definecolor{currentstroke}{rgb}{0.000000,0.000000,0.000000}%
\pgfsetstrokecolor{currentstroke}%
\pgfsetdash{}{0pt}%
\pgfsys@defobject{currentmarker}{\pgfqpoint{-0.027778in}{0.000000in}}{\pgfqpoint{0.000000in}{0.000000in}}{%
\pgfpathmoveto{\pgfqpoint{0.000000in}{0.000000in}}%
\pgfpathlineto{\pgfqpoint{-0.027778in}{0.000000in}}%
\pgfusepath{stroke,fill}%
}%
\begin{pgfscope}%
\pgfsys@transformshift{0.880000in}{3.023282in}%
\pgfsys@useobject{currentmarker}{}%
\end{pgfscope}%
\end{pgfscope}%
\begin{pgfscope}%
\pgfsetbuttcap%
\pgfsetroundjoin%
\definecolor{currentfill}{rgb}{0.000000,0.000000,0.000000}%
\pgfsetfillcolor{currentfill}%
\pgfsetlinewidth{0.602250pt}%
\definecolor{currentstroke}{rgb}{0.000000,0.000000,0.000000}%
\pgfsetstrokecolor{currentstroke}%
\pgfsetdash{}{0pt}%
\pgfsys@defobject{currentmarker}{\pgfqpoint{-0.027778in}{0.000000in}}{\pgfqpoint{0.000000in}{0.000000in}}{%
\pgfpathmoveto{\pgfqpoint{0.000000in}{0.000000in}}%
\pgfpathlineto{\pgfqpoint{-0.027778in}{0.000000in}}%
\pgfusepath{stroke,fill}%
}%
\begin{pgfscope}%
\pgfsys@transformshift{0.880000in}{3.074695in}%
\pgfsys@useobject{currentmarker}{}%
\end{pgfscope}%
\end{pgfscope}%
\begin{pgfscope}%
\pgfsetbuttcap%
\pgfsetroundjoin%
\definecolor{currentfill}{rgb}{0.000000,0.000000,0.000000}%
\pgfsetfillcolor{currentfill}%
\pgfsetlinewidth{0.602250pt}%
\definecolor{currentstroke}{rgb}{0.000000,0.000000,0.000000}%
\pgfsetstrokecolor{currentstroke}%
\pgfsetdash{}{0pt}%
\pgfsys@defobject{currentmarker}{\pgfqpoint{-0.027778in}{0.000000in}}{\pgfqpoint{0.000000in}{0.000000in}}{%
\pgfpathmoveto{\pgfqpoint{0.000000in}{0.000000in}}%
\pgfpathlineto{\pgfqpoint{-0.027778in}{0.000000in}}%
\pgfusepath{stroke,fill}%
}%
\begin{pgfscope}%
\pgfsys@transformshift{0.880000in}{3.114574in}%
\pgfsys@useobject{currentmarker}{}%
\end{pgfscope}%
\end{pgfscope}%
\begin{pgfscope}%
\pgfsetbuttcap%
\pgfsetroundjoin%
\definecolor{currentfill}{rgb}{0.000000,0.000000,0.000000}%
\pgfsetfillcolor{currentfill}%
\pgfsetlinewidth{0.602250pt}%
\definecolor{currentstroke}{rgb}{0.000000,0.000000,0.000000}%
\pgfsetstrokecolor{currentstroke}%
\pgfsetdash{}{0pt}%
\pgfsys@defobject{currentmarker}{\pgfqpoint{-0.027778in}{0.000000in}}{\pgfqpoint{0.000000in}{0.000000in}}{%
\pgfpathmoveto{\pgfqpoint{0.000000in}{0.000000in}}%
\pgfpathlineto{\pgfqpoint{-0.027778in}{0.000000in}}%
\pgfusepath{stroke,fill}%
}%
\begin{pgfscope}%
\pgfsys@transformshift{0.880000in}{3.147158in}%
\pgfsys@useobject{currentmarker}{}%
\end{pgfscope}%
\end{pgfscope}%
\begin{pgfscope}%
\pgfsetbuttcap%
\pgfsetroundjoin%
\definecolor{currentfill}{rgb}{0.000000,0.000000,0.000000}%
\pgfsetfillcolor{currentfill}%
\pgfsetlinewidth{0.602250pt}%
\definecolor{currentstroke}{rgb}{0.000000,0.000000,0.000000}%
\pgfsetstrokecolor{currentstroke}%
\pgfsetdash{}{0pt}%
\pgfsys@defobject{currentmarker}{\pgfqpoint{-0.027778in}{0.000000in}}{\pgfqpoint{0.000000in}{0.000000in}}{%
\pgfpathmoveto{\pgfqpoint{0.000000in}{0.000000in}}%
\pgfpathlineto{\pgfqpoint{-0.027778in}{0.000000in}}%
\pgfusepath{stroke,fill}%
}%
\begin{pgfscope}%
\pgfsys@transformshift{0.880000in}{3.174707in}%
\pgfsys@useobject{currentmarker}{}%
\end{pgfscope}%
\end{pgfscope}%
\begin{pgfscope}%
\pgfsetbuttcap%
\pgfsetroundjoin%
\definecolor{currentfill}{rgb}{0.000000,0.000000,0.000000}%
\pgfsetfillcolor{currentfill}%
\pgfsetlinewidth{0.602250pt}%
\definecolor{currentstroke}{rgb}{0.000000,0.000000,0.000000}%
\pgfsetstrokecolor{currentstroke}%
\pgfsetdash{}{0pt}%
\pgfsys@defobject{currentmarker}{\pgfqpoint{-0.027778in}{0.000000in}}{\pgfqpoint{0.000000in}{0.000000in}}{%
\pgfpathmoveto{\pgfqpoint{0.000000in}{0.000000in}}%
\pgfpathlineto{\pgfqpoint{-0.027778in}{0.000000in}}%
\pgfusepath{stroke,fill}%
}%
\begin{pgfscope}%
\pgfsys@transformshift{0.880000in}{3.198571in}%
\pgfsys@useobject{currentmarker}{}%
\end{pgfscope}%
\end{pgfscope}%
\begin{pgfscope}%
\pgfsetbuttcap%
\pgfsetroundjoin%
\definecolor{currentfill}{rgb}{0.000000,0.000000,0.000000}%
\pgfsetfillcolor{currentfill}%
\pgfsetlinewidth{0.602250pt}%
\definecolor{currentstroke}{rgb}{0.000000,0.000000,0.000000}%
\pgfsetstrokecolor{currentstroke}%
\pgfsetdash{}{0pt}%
\pgfsys@defobject{currentmarker}{\pgfqpoint{-0.027778in}{0.000000in}}{\pgfqpoint{0.000000in}{0.000000in}}{%
\pgfpathmoveto{\pgfqpoint{0.000000in}{0.000000in}}%
\pgfpathlineto{\pgfqpoint{-0.027778in}{0.000000in}}%
\pgfusepath{stroke,fill}%
}%
\begin{pgfscope}%
\pgfsys@transformshift{0.880000in}{3.219620in}%
\pgfsys@useobject{currentmarker}{}%
\end{pgfscope}%
\end{pgfscope}%
\begin{pgfscope}%
\pgfsetbuttcap%
\pgfsetroundjoin%
\definecolor{currentfill}{rgb}{0.000000,0.000000,0.000000}%
\pgfsetfillcolor{currentfill}%
\pgfsetlinewidth{0.602250pt}%
\definecolor{currentstroke}{rgb}{0.000000,0.000000,0.000000}%
\pgfsetstrokecolor{currentstroke}%
\pgfsetdash{}{0pt}%
\pgfsys@defobject{currentmarker}{\pgfqpoint{-0.027778in}{0.000000in}}{\pgfqpoint{0.000000in}{0.000000in}}{%
\pgfpathmoveto{\pgfqpoint{0.000000in}{0.000000in}}%
\pgfpathlineto{\pgfqpoint{-0.027778in}{0.000000in}}%
\pgfusepath{stroke,fill}%
}%
\begin{pgfscope}%
\pgfsys@transformshift{0.880000in}{3.362326in}%
\pgfsys@useobject{currentmarker}{}%
\end{pgfscope}%
\end{pgfscope}%
\begin{pgfscope}%
\pgfsetbuttcap%
\pgfsetroundjoin%
\definecolor{currentfill}{rgb}{0.000000,0.000000,0.000000}%
\pgfsetfillcolor{currentfill}%
\pgfsetlinewidth{0.602250pt}%
\definecolor{currentstroke}{rgb}{0.000000,0.000000,0.000000}%
\pgfsetstrokecolor{currentstroke}%
\pgfsetdash{}{0pt}%
\pgfsys@defobject{currentmarker}{\pgfqpoint{-0.027778in}{0.000000in}}{\pgfqpoint{0.000000in}{0.000000in}}{%
\pgfpathmoveto{\pgfqpoint{0.000000in}{0.000000in}}%
\pgfpathlineto{\pgfqpoint{-0.027778in}{0.000000in}}%
\pgfusepath{stroke,fill}%
}%
\begin{pgfscope}%
\pgfsys@transformshift{0.880000in}{3.434789in}%
\pgfsys@useobject{currentmarker}{}%
\end{pgfscope}%
\end{pgfscope}%
\begin{pgfscope}%
\pgfsetbuttcap%
\pgfsetroundjoin%
\definecolor{currentfill}{rgb}{0.000000,0.000000,0.000000}%
\pgfsetfillcolor{currentfill}%
\pgfsetlinewidth{0.602250pt}%
\definecolor{currentstroke}{rgb}{0.000000,0.000000,0.000000}%
\pgfsetstrokecolor{currentstroke}%
\pgfsetdash{}{0pt}%
\pgfsys@defobject{currentmarker}{\pgfqpoint{-0.027778in}{0.000000in}}{\pgfqpoint{0.000000in}{0.000000in}}{%
\pgfpathmoveto{\pgfqpoint{0.000000in}{0.000000in}}%
\pgfpathlineto{\pgfqpoint{-0.027778in}{0.000000in}}%
\pgfusepath{stroke,fill}%
}%
\begin{pgfscope}%
\pgfsys@transformshift{0.880000in}{3.486202in}%
\pgfsys@useobject{currentmarker}{}%
\end{pgfscope}%
\end{pgfscope}%
\begin{pgfscope}%
\pgfsetbuttcap%
\pgfsetroundjoin%
\definecolor{currentfill}{rgb}{0.000000,0.000000,0.000000}%
\pgfsetfillcolor{currentfill}%
\pgfsetlinewidth{0.602250pt}%
\definecolor{currentstroke}{rgb}{0.000000,0.000000,0.000000}%
\pgfsetstrokecolor{currentstroke}%
\pgfsetdash{}{0pt}%
\pgfsys@defobject{currentmarker}{\pgfqpoint{-0.027778in}{0.000000in}}{\pgfqpoint{0.000000in}{0.000000in}}{%
\pgfpathmoveto{\pgfqpoint{0.000000in}{0.000000in}}%
\pgfpathlineto{\pgfqpoint{-0.027778in}{0.000000in}}%
\pgfusepath{stroke,fill}%
}%
\begin{pgfscope}%
\pgfsys@transformshift{0.880000in}{3.526081in}%
\pgfsys@useobject{currentmarker}{}%
\end{pgfscope}%
\end{pgfscope}%
\begin{pgfscope}%
\pgfsetbuttcap%
\pgfsetroundjoin%
\definecolor{currentfill}{rgb}{0.000000,0.000000,0.000000}%
\pgfsetfillcolor{currentfill}%
\pgfsetlinewidth{0.602250pt}%
\definecolor{currentstroke}{rgb}{0.000000,0.000000,0.000000}%
\pgfsetstrokecolor{currentstroke}%
\pgfsetdash{}{0pt}%
\pgfsys@defobject{currentmarker}{\pgfqpoint{-0.027778in}{0.000000in}}{\pgfqpoint{0.000000in}{0.000000in}}{%
\pgfpathmoveto{\pgfqpoint{0.000000in}{0.000000in}}%
\pgfpathlineto{\pgfqpoint{-0.027778in}{0.000000in}}%
\pgfusepath{stroke,fill}%
}%
\begin{pgfscope}%
\pgfsys@transformshift{0.880000in}{3.558665in}%
\pgfsys@useobject{currentmarker}{}%
\end{pgfscope}%
\end{pgfscope}%
\begin{pgfscope}%
\pgfsetbuttcap%
\pgfsetroundjoin%
\definecolor{currentfill}{rgb}{0.000000,0.000000,0.000000}%
\pgfsetfillcolor{currentfill}%
\pgfsetlinewidth{0.602250pt}%
\definecolor{currentstroke}{rgb}{0.000000,0.000000,0.000000}%
\pgfsetstrokecolor{currentstroke}%
\pgfsetdash{}{0pt}%
\pgfsys@defobject{currentmarker}{\pgfqpoint{-0.027778in}{0.000000in}}{\pgfqpoint{0.000000in}{0.000000in}}{%
\pgfpathmoveto{\pgfqpoint{0.000000in}{0.000000in}}%
\pgfpathlineto{\pgfqpoint{-0.027778in}{0.000000in}}%
\pgfusepath{stroke,fill}%
}%
\begin{pgfscope}%
\pgfsys@transformshift{0.880000in}{3.586214in}%
\pgfsys@useobject{currentmarker}{}%
\end{pgfscope}%
\end{pgfscope}%
\begin{pgfscope}%
\pgfsetbuttcap%
\pgfsetroundjoin%
\definecolor{currentfill}{rgb}{0.000000,0.000000,0.000000}%
\pgfsetfillcolor{currentfill}%
\pgfsetlinewidth{0.602250pt}%
\definecolor{currentstroke}{rgb}{0.000000,0.000000,0.000000}%
\pgfsetstrokecolor{currentstroke}%
\pgfsetdash{}{0pt}%
\pgfsys@defobject{currentmarker}{\pgfqpoint{-0.027778in}{0.000000in}}{\pgfqpoint{0.000000in}{0.000000in}}{%
\pgfpathmoveto{\pgfqpoint{0.000000in}{0.000000in}}%
\pgfpathlineto{\pgfqpoint{-0.027778in}{0.000000in}}%
\pgfusepath{stroke,fill}%
}%
\begin{pgfscope}%
\pgfsys@transformshift{0.880000in}{3.610078in}%
\pgfsys@useobject{currentmarker}{}%
\end{pgfscope}%
\end{pgfscope}%
\begin{pgfscope}%
\pgfsetbuttcap%
\pgfsetroundjoin%
\definecolor{currentfill}{rgb}{0.000000,0.000000,0.000000}%
\pgfsetfillcolor{currentfill}%
\pgfsetlinewidth{0.602250pt}%
\definecolor{currentstroke}{rgb}{0.000000,0.000000,0.000000}%
\pgfsetstrokecolor{currentstroke}%
\pgfsetdash{}{0pt}%
\pgfsys@defobject{currentmarker}{\pgfqpoint{-0.027778in}{0.000000in}}{\pgfqpoint{0.000000in}{0.000000in}}{%
\pgfpathmoveto{\pgfqpoint{0.000000in}{0.000000in}}%
\pgfpathlineto{\pgfqpoint{-0.027778in}{0.000000in}}%
\pgfusepath{stroke,fill}%
}%
\begin{pgfscope}%
\pgfsys@transformshift{0.880000in}{3.631128in}%
\pgfsys@useobject{currentmarker}{}%
\end{pgfscope}%
\end{pgfscope}%
\begin{pgfscope}%
\pgfsetbuttcap%
\pgfsetroundjoin%
\definecolor{currentfill}{rgb}{0.000000,0.000000,0.000000}%
\pgfsetfillcolor{currentfill}%
\pgfsetlinewidth{0.602250pt}%
\definecolor{currentstroke}{rgb}{0.000000,0.000000,0.000000}%
\pgfsetstrokecolor{currentstroke}%
\pgfsetdash{}{0pt}%
\pgfsys@defobject{currentmarker}{\pgfqpoint{-0.027778in}{0.000000in}}{\pgfqpoint{0.000000in}{0.000000in}}{%
\pgfpathmoveto{\pgfqpoint{0.000000in}{0.000000in}}%
\pgfpathlineto{\pgfqpoint{-0.027778in}{0.000000in}}%
\pgfusepath{stroke,fill}%
}%
\begin{pgfscope}%
\pgfsys@transformshift{0.880000in}{3.773833in}%
\pgfsys@useobject{currentmarker}{}%
\end{pgfscope}%
\end{pgfscope}%
\begin{pgfscope}%
\pgfsetbuttcap%
\pgfsetroundjoin%
\definecolor{currentfill}{rgb}{0.000000,0.000000,0.000000}%
\pgfsetfillcolor{currentfill}%
\pgfsetlinewidth{0.602250pt}%
\definecolor{currentstroke}{rgb}{0.000000,0.000000,0.000000}%
\pgfsetstrokecolor{currentstroke}%
\pgfsetdash{}{0pt}%
\pgfsys@defobject{currentmarker}{\pgfqpoint{-0.027778in}{0.000000in}}{\pgfqpoint{0.000000in}{0.000000in}}{%
\pgfpathmoveto{\pgfqpoint{0.000000in}{0.000000in}}%
\pgfpathlineto{\pgfqpoint{-0.027778in}{0.000000in}}%
\pgfusepath{stroke,fill}%
}%
\begin{pgfscope}%
\pgfsys@transformshift{0.880000in}{3.846296in}%
\pgfsys@useobject{currentmarker}{}%
\end{pgfscope}%
\end{pgfscope}%
\begin{pgfscope}%
\pgfsetbuttcap%
\pgfsetroundjoin%
\definecolor{currentfill}{rgb}{0.000000,0.000000,0.000000}%
\pgfsetfillcolor{currentfill}%
\pgfsetlinewidth{0.602250pt}%
\definecolor{currentstroke}{rgb}{0.000000,0.000000,0.000000}%
\pgfsetstrokecolor{currentstroke}%
\pgfsetdash{}{0pt}%
\pgfsys@defobject{currentmarker}{\pgfqpoint{-0.027778in}{0.000000in}}{\pgfqpoint{0.000000in}{0.000000in}}{%
\pgfpathmoveto{\pgfqpoint{0.000000in}{0.000000in}}%
\pgfpathlineto{\pgfqpoint{-0.027778in}{0.000000in}}%
\pgfusepath{stroke,fill}%
}%
\begin{pgfscope}%
\pgfsys@transformshift{0.880000in}{3.897709in}%
\pgfsys@useobject{currentmarker}{}%
\end{pgfscope}%
\end{pgfscope}%
\begin{pgfscope}%
\pgfsetbuttcap%
\pgfsetroundjoin%
\definecolor{currentfill}{rgb}{0.000000,0.000000,0.000000}%
\pgfsetfillcolor{currentfill}%
\pgfsetlinewidth{0.602250pt}%
\definecolor{currentstroke}{rgb}{0.000000,0.000000,0.000000}%
\pgfsetstrokecolor{currentstroke}%
\pgfsetdash{}{0pt}%
\pgfsys@defobject{currentmarker}{\pgfqpoint{-0.027778in}{0.000000in}}{\pgfqpoint{0.000000in}{0.000000in}}{%
\pgfpathmoveto{\pgfqpoint{0.000000in}{0.000000in}}%
\pgfpathlineto{\pgfqpoint{-0.027778in}{0.000000in}}%
\pgfusepath{stroke,fill}%
}%
\begin{pgfscope}%
\pgfsys@transformshift{0.880000in}{3.937588in}%
\pgfsys@useobject{currentmarker}{}%
\end{pgfscope}%
\end{pgfscope}%
\begin{pgfscope}%
\pgfsetbuttcap%
\pgfsetroundjoin%
\definecolor{currentfill}{rgb}{0.000000,0.000000,0.000000}%
\pgfsetfillcolor{currentfill}%
\pgfsetlinewidth{0.602250pt}%
\definecolor{currentstroke}{rgb}{0.000000,0.000000,0.000000}%
\pgfsetstrokecolor{currentstroke}%
\pgfsetdash{}{0pt}%
\pgfsys@defobject{currentmarker}{\pgfqpoint{-0.027778in}{0.000000in}}{\pgfqpoint{0.000000in}{0.000000in}}{%
\pgfpathmoveto{\pgfqpoint{0.000000in}{0.000000in}}%
\pgfpathlineto{\pgfqpoint{-0.027778in}{0.000000in}}%
\pgfusepath{stroke,fill}%
}%
\begin{pgfscope}%
\pgfsys@transformshift{0.880000in}{3.970172in}%
\pgfsys@useobject{currentmarker}{}%
\end{pgfscope}%
\end{pgfscope}%
\begin{pgfscope}%
\pgfsetbuttcap%
\pgfsetroundjoin%
\definecolor{currentfill}{rgb}{0.000000,0.000000,0.000000}%
\pgfsetfillcolor{currentfill}%
\pgfsetlinewidth{0.602250pt}%
\definecolor{currentstroke}{rgb}{0.000000,0.000000,0.000000}%
\pgfsetstrokecolor{currentstroke}%
\pgfsetdash{}{0pt}%
\pgfsys@defobject{currentmarker}{\pgfqpoint{-0.027778in}{0.000000in}}{\pgfqpoint{0.000000in}{0.000000in}}{%
\pgfpathmoveto{\pgfqpoint{0.000000in}{0.000000in}}%
\pgfpathlineto{\pgfqpoint{-0.027778in}{0.000000in}}%
\pgfusepath{stroke,fill}%
}%
\begin{pgfscope}%
\pgfsys@transformshift{0.880000in}{3.997721in}%
\pgfsys@useobject{currentmarker}{}%
\end{pgfscope}%
\end{pgfscope}%
\begin{pgfscope}%
\pgfsetbuttcap%
\pgfsetroundjoin%
\definecolor{currentfill}{rgb}{0.000000,0.000000,0.000000}%
\pgfsetfillcolor{currentfill}%
\pgfsetlinewidth{0.602250pt}%
\definecolor{currentstroke}{rgb}{0.000000,0.000000,0.000000}%
\pgfsetstrokecolor{currentstroke}%
\pgfsetdash{}{0pt}%
\pgfsys@defobject{currentmarker}{\pgfqpoint{-0.027778in}{0.000000in}}{\pgfqpoint{0.000000in}{0.000000in}}{%
\pgfpathmoveto{\pgfqpoint{0.000000in}{0.000000in}}%
\pgfpathlineto{\pgfqpoint{-0.027778in}{0.000000in}}%
\pgfusepath{stroke,fill}%
}%
\begin{pgfscope}%
\pgfsys@transformshift{0.880000in}{4.021585in}%
\pgfsys@useobject{currentmarker}{}%
\end{pgfscope}%
\end{pgfscope}%
\begin{pgfscope}%
\pgfsetbuttcap%
\pgfsetroundjoin%
\definecolor{currentfill}{rgb}{0.000000,0.000000,0.000000}%
\pgfsetfillcolor{currentfill}%
\pgfsetlinewidth{0.602250pt}%
\definecolor{currentstroke}{rgb}{0.000000,0.000000,0.000000}%
\pgfsetstrokecolor{currentstroke}%
\pgfsetdash{}{0pt}%
\pgfsys@defobject{currentmarker}{\pgfqpoint{-0.027778in}{0.000000in}}{\pgfqpoint{0.000000in}{0.000000in}}{%
\pgfpathmoveto{\pgfqpoint{0.000000in}{0.000000in}}%
\pgfpathlineto{\pgfqpoint{-0.027778in}{0.000000in}}%
\pgfusepath{stroke,fill}%
}%
\begin{pgfscope}%
\pgfsys@transformshift{0.880000in}{4.042635in}%
\pgfsys@useobject{currentmarker}{}%
\end{pgfscope}%
\end{pgfscope}%
\begin{pgfscope}%
\pgfsetbuttcap%
\pgfsetroundjoin%
\definecolor{currentfill}{rgb}{0.000000,0.000000,0.000000}%
\pgfsetfillcolor{currentfill}%
\pgfsetlinewidth{0.602250pt}%
\definecolor{currentstroke}{rgb}{0.000000,0.000000,0.000000}%
\pgfsetstrokecolor{currentstroke}%
\pgfsetdash{}{0pt}%
\pgfsys@defobject{currentmarker}{\pgfqpoint{-0.027778in}{0.000000in}}{\pgfqpoint{0.000000in}{0.000000in}}{%
\pgfpathmoveto{\pgfqpoint{0.000000in}{0.000000in}}%
\pgfpathlineto{\pgfqpoint{-0.027778in}{0.000000in}}%
\pgfusepath{stroke,fill}%
}%
\begin{pgfscope}%
\pgfsys@transformshift{0.880000in}{4.185341in}%
\pgfsys@useobject{currentmarker}{}%
\end{pgfscope}%
\end{pgfscope}%
\begin{pgfscope}%
\pgfpathrectangle{\pgfqpoint{0.880000in}{2.849537in}}{\pgfqpoint{1.897959in}{1.372727in}} %
\pgfusepath{clip}%
\pgfsetbuttcap%
\pgfsetroundjoin%
\pgfsetlinewidth{1.505625pt}%
\definecolor{currentstroke}{rgb}{1.000000,0.000000,0.000000}%
\pgfsetstrokecolor{currentstroke}%
\pgfsetdash{{5.550000pt}{2.400000pt}}{0.000000pt}%
\pgfpathmoveto{\pgfqpoint{0.966271in}{4.087056in}}%
\pgfpathlineto{\pgfqpoint{0.990920in}{4.084235in}}%
\pgfpathlineto{\pgfqpoint{1.015569in}{4.081457in}}%
\pgfpathlineto{\pgfqpoint{1.040217in}{4.078722in}}%
\pgfpathlineto{\pgfqpoint{1.064866in}{4.076028in}}%
\pgfpathlineto{\pgfqpoint{1.089515in}{4.073376in}}%
\pgfpathlineto{\pgfqpoint{1.114164in}{4.070765in}}%
\pgfpathlineto{\pgfqpoint{1.138813in}{4.068195in}}%
\pgfpathlineto{\pgfqpoint{1.163461in}{4.065666in}}%
\pgfpathlineto{\pgfqpoint{1.188110in}{4.063178in}}%
\pgfpathlineto{\pgfqpoint{1.212759in}{4.060732in}}%
\pgfpathlineto{\pgfqpoint{1.237408in}{4.058327in}}%
\pgfpathlineto{\pgfqpoint{1.262057in}{4.055963in}}%
\pgfpathlineto{\pgfqpoint{1.286706in}{4.053641in}}%
\pgfpathlineto{\pgfqpoint{1.311354in}{4.051361in}}%
\pgfpathlineto{\pgfqpoint{1.336003in}{4.049122in}}%
\pgfpathlineto{\pgfqpoint{1.360652in}{4.046925in}}%
\pgfpathlineto{\pgfqpoint{1.385301in}{4.044770in}}%
\pgfpathlineto{\pgfqpoint{1.409950in}{4.042656in}}%
\pgfpathlineto{\pgfqpoint{1.434598in}{4.040584in}}%
\pgfpathlineto{\pgfqpoint{1.459247in}{4.038554in}}%
\pgfpathlineto{\pgfqpoint{1.483896in}{4.036564in}}%
\pgfpathlineto{\pgfqpoint{1.508545in}{4.034616in}}%
\pgfpathlineto{\pgfqpoint{1.533194in}{4.032710in}}%
\pgfpathlineto{\pgfqpoint{1.557843in}{4.030844in}}%
\pgfpathlineto{\pgfqpoint{1.582491in}{4.029019in}}%
\pgfpathlineto{\pgfqpoint{1.607140in}{4.027235in}}%
\pgfpathlineto{\pgfqpoint{1.631789in}{4.025492in}}%
\pgfpathlineto{\pgfqpoint{1.656438in}{4.023788in}}%
\pgfpathlineto{\pgfqpoint{1.681087in}{4.022126in}}%
\pgfpathlineto{\pgfqpoint{1.705735in}{4.020503in}}%
\pgfpathlineto{\pgfqpoint{1.730384in}{4.018919in}}%
\pgfpathlineto{\pgfqpoint{1.755033in}{4.017376in}}%
\pgfpathlineto{\pgfqpoint{1.779682in}{4.015871in}}%
\pgfpathlineto{\pgfqpoint{1.804331in}{4.014406in}}%
\pgfpathlineto{\pgfqpoint{1.828980in}{4.012979in}}%
\pgfpathlineto{\pgfqpoint{1.853628in}{4.011591in}}%
\pgfpathlineto{\pgfqpoint{1.878277in}{4.010242in}}%
\pgfpathlineto{\pgfqpoint{1.902926in}{4.008931in}}%
\pgfpathlineto{\pgfqpoint{1.927575in}{4.007657in}}%
\pgfpathlineto{\pgfqpoint{1.952224in}{4.006422in}}%
\pgfpathlineto{\pgfqpoint{1.976873in}{4.005224in}}%
\pgfpathlineto{\pgfqpoint{2.001521in}{4.004063in}}%
\pgfpathlineto{\pgfqpoint{2.026170in}{4.002939in}}%
\pgfpathlineto{\pgfqpoint{2.050819in}{4.001852in}}%
\pgfpathlineto{\pgfqpoint{2.075468in}{4.000801in}}%
\pgfpathlineto{\pgfqpoint{2.100117in}{3.999787in}}%
\pgfpathlineto{\pgfqpoint{2.124765in}{3.998809in}}%
\pgfpathlineto{\pgfqpoint{2.149414in}{3.997867in}}%
\pgfpathlineto{\pgfqpoint{2.174063in}{3.996961in}}%
\pgfpathlineto{\pgfqpoint{2.198712in}{3.996090in}}%
\pgfpathlineto{\pgfqpoint{2.223361in}{3.995255in}}%
\pgfpathlineto{\pgfqpoint{2.248010in}{3.994454in}}%
\pgfpathlineto{\pgfqpoint{2.272658in}{3.993689in}}%
\pgfpathlineto{\pgfqpoint{2.297307in}{3.992959in}}%
\pgfpathlineto{\pgfqpoint{2.321956in}{3.992263in}}%
\pgfpathlineto{\pgfqpoint{2.346605in}{3.991602in}}%
\pgfpathlineto{\pgfqpoint{2.371254in}{3.990975in}}%
\pgfpathlineto{\pgfqpoint{2.395902in}{3.990382in}}%
\pgfpathlineto{\pgfqpoint{2.420551in}{3.989824in}}%
\pgfpathlineto{\pgfqpoint{2.445200in}{3.989299in}}%
\pgfpathlineto{\pgfqpoint{2.469849in}{3.988808in}}%
\pgfpathlineto{\pgfqpoint{2.494498in}{3.988351in}}%
\pgfpathlineto{\pgfqpoint{2.519147in}{3.987927in}}%
\pgfpathlineto{\pgfqpoint{2.543795in}{3.987536in}}%
\pgfpathlineto{\pgfqpoint{2.568444in}{3.987179in}}%
\pgfpathlineto{\pgfqpoint{2.593093in}{3.986856in}}%
\pgfpathlineto{\pgfqpoint{2.617742in}{3.986565in}}%
\pgfpathlineto{\pgfqpoint{2.642391in}{3.986308in}}%
\pgfpathlineto{\pgfqpoint{2.667039in}{3.986083in}}%
\pgfpathlineto{\pgfqpoint{2.691688in}{3.985892in}}%
\pgfusepath{stroke}%
\end{pgfscope}%
\begin{pgfscope}%
\pgfpathrectangle{\pgfqpoint{0.880000in}{2.849537in}}{\pgfqpoint{1.897959in}{1.372727in}} %
\pgfusepath{clip}%
\pgfsetbuttcap%
\pgfsetmiterjoin%
\definecolor{currentfill}{rgb}{1.000000,0.000000,0.000000}%
\pgfsetfillcolor{currentfill}%
\pgfsetlinewidth{1.003750pt}%
\definecolor{currentstroke}{rgb}{1.000000,0.000000,0.000000}%
\pgfsetstrokecolor{currentstroke}%
\pgfsetdash{}{0pt}%
\pgfsys@defobject{currentmarker}{\pgfqpoint{-0.041667in}{-0.041667in}}{\pgfqpoint{0.041667in}{0.041667in}}{%
\pgfpathmoveto{\pgfqpoint{-0.041667in}{-0.041667in}}%
\pgfpathlineto{\pgfqpoint{0.041667in}{-0.041667in}}%
\pgfpathlineto{\pgfqpoint{0.041667in}{0.041667in}}%
\pgfpathlineto{\pgfqpoint{-0.041667in}{0.041667in}}%
\pgfpathclose%
\pgfusepath{stroke,fill}%
}%
\begin{pgfscope}%
\pgfsys@transformshift{0.966271in}{4.087056in}%
\pgfsys@useobject{currentmarker}{}%
\end{pgfscope}%
\begin{pgfscope}%
\pgfsys@transformshift{1.311354in}{4.051361in}%
\pgfsys@useobject{currentmarker}{}%
\end{pgfscope}%
\begin{pgfscope}%
\pgfsys@transformshift{1.656438in}{4.023788in}%
\pgfsys@useobject{currentmarker}{}%
\end{pgfscope}%
\begin{pgfscope}%
\pgfsys@transformshift{2.001521in}{4.004063in}%
\pgfsys@useobject{currentmarker}{}%
\end{pgfscope}%
\begin{pgfscope}%
\pgfsys@transformshift{2.346605in}{3.991602in}%
\pgfsys@useobject{currentmarker}{}%
\end{pgfscope}%
\begin{pgfscope}%
\pgfsys@transformshift{2.691688in}{3.985892in}%
\pgfsys@useobject{currentmarker}{}%
\end{pgfscope}%
\end{pgfscope}%
\begin{pgfscope}%
\pgfpathrectangle{\pgfqpoint{0.880000in}{2.849537in}}{\pgfqpoint{1.897959in}{1.372727in}} %
\pgfusepath{clip}%
\pgfsetrectcap%
\pgfsetroundjoin%
\pgfsetlinewidth{1.505625pt}%
\definecolor{currentstroke}{rgb}{0.000000,0.000000,1.000000}%
\pgfsetstrokecolor{currentstroke}%
\pgfsetdash{}{0pt}%
\pgfpathmoveto{\pgfqpoint{0.966271in}{3.541127in}}%
\pgfpathlineto{\pgfqpoint{0.990920in}{3.535088in}}%
\pgfpathlineto{\pgfqpoint{1.015569in}{3.529271in}}%
\pgfpathlineto{\pgfqpoint{1.040217in}{3.523620in}}%
\pgfpathlineto{\pgfqpoint{1.064866in}{3.518096in}}%
\pgfpathlineto{\pgfqpoint{1.089515in}{3.512667in}}%
\pgfpathlineto{\pgfqpoint{1.114164in}{3.507313in}}%
\pgfpathlineto{\pgfqpoint{1.138813in}{3.502015in}}%
\pgfpathlineto{\pgfqpoint{1.163461in}{3.496761in}}%
\pgfpathlineto{\pgfqpoint{1.188110in}{3.491538in}}%
\pgfpathlineto{\pgfqpoint{1.212759in}{3.486338in}}%
\pgfpathlineto{\pgfqpoint{1.237408in}{3.481153in}}%
\pgfpathlineto{\pgfqpoint{1.262057in}{3.475975in}}%
\pgfpathlineto{\pgfqpoint{1.286706in}{3.470799in}}%
\pgfpathlineto{\pgfqpoint{1.311354in}{3.465619in}}%
\pgfpathlineto{\pgfqpoint{1.336003in}{3.460431in}}%
\pgfpathlineto{\pgfqpoint{1.360652in}{3.455229in}}%
\pgfpathlineto{\pgfqpoint{1.385301in}{3.450009in}}%
\pgfpathlineto{\pgfqpoint{1.409950in}{3.444768in}}%
\pgfpathlineto{\pgfqpoint{1.434598in}{3.439501in}}%
\pgfpathlineto{\pgfqpoint{1.459247in}{3.434204in}}%
\pgfpathlineto{\pgfqpoint{1.483896in}{3.428874in}}%
\pgfpathlineto{\pgfqpoint{1.508545in}{3.423506in}}%
\pgfpathlineto{\pgfqpoint{1.533194in}{3.418098in}}%
\pgfpathlineto{\pgfqpoint{1.557843in}{3.412644in}}%
\pgfpathlineto{\pgfqpoint{1.582491in}{3.407142in}}%
\pgfpathlineto{\pgfqpoint{1.607140in}{3.401587in}}%
\pgfpathlineto{\pgfqpoint{1.631789in}{3.395975in}}%
\pgfpathlineto{\pgfqpoint{1.656438in}{3.390302in}}%
\pgfpathlineto{\pgfqpoint{1.681087in}{3.384563in}}%
\pgfpathlineto{\pgfqpoint{1.705735in}{3.378755in}}%
\pgfpathlineto{\pgfqpoint{1.730384in}{3.372871in}}%
\pgfpathlineto{\pgfqpoint{1.755033in}{3.366907in}}%
\pgfpathlineto{\pgfqpoint{1.779682in}{3.360859in}}%
\pgfpathlineto{\pgfqpoint{1.804331in}{3.354719in}}%
\pgfpathlineto{\pgfqpoint{1.828980in}{3.348482in}}%
\pgfpathlineto{\pgfqpoint{1.853628in}{3.342142in}}%
\pgfpathlineto{\pgfqpoint{1.878277in}{3.335692in}}%
\pgfpathlineto{\pgfqpoint{1.902926in}{3.329126in}}%
\pgfpathlineto{\pgfqpoint{1.927575in}{3.322434in}}%
\pgfpathlineto{\pgfqpoint{1.952224in}{3.315609in}}%
\pgfpathlineto{\pgfqpoint{1.976873in}{3.308641in}}%
\pgfpathlineto{\pgfqpoint{2.001521in}{3.301522in}}%
\pgfpathlineto{\pgfqpoint{2.026170in}{3.294241in}}%
\pgfpathlineto{\pgfqpoint{2.050819in}{3.286785in}}%
\pgfpathlineto{\pgfqpoint{2.075468in}{3.279143in}}%
\pgfpathlineto{\pgfqpoint{2.100117in}{3.271301in}}%
\pgfpathlineto{\pgfqpoint{2.124765in}{3.263243in}}%
\pgfpathlineto{\pgfqpoint{2.149414in}{3.254954in}}%
\pgfpathlineto{\pgfqpoint{2.174063in}{3.246414in}}%
\pgfpathlineto{\pgfqpoint{2.198712in}{3.237603in}}%
\pgfpathlineto{\pgfqpoint{2.223361in}{3.228498in}}%
\pgfpathlineto{\pgfqpoint{2.248010in}{3.219074in}}%
\pgfpathlineto{\pgfqpoint{2.272658in}{3.209301in}}%
\pgfpathlineto{\pgfqpoint{2.297307in}{3.199146in}}%
\pgfpathlineto{\pgfqpoint{2.321956in}{3.188572in}}%
\pgfpathlineto{\pgfqpoint{2.346605in}{3.177536in}}%
\pgfpathlineto{\pgfqpoint{2.371254in}{3.165988in}}%
\pgfpathlineto{\pgfqpoint{2.395902in}{3.153869in}}%
\pgfpathlineto{\pgfqpoint{2.420551in}{3.141112in}}%
\pgfpathlineto{\pgfqpoint{2.445200in}{3.127635in}}%
\pgfpathlineto{\pgfqpoint{2.469849in}{3.113341in}}%
\pgfpathlineto{\pgfqpoint{2.494498in}{3.098112in}}%
\pgfpathlineto{\pgfqpoint{2.519147in}{3.081802in}}%
\pgfpathlineto{\pgfqpoint{2.543795in}{3.064230in}}%
\pgfpathlineto{\pgfqpoint{2.568444in}{3.045165in}}%
\pgfpathlineto{\pgfqpoint{2.593093in}{3.024304in}}%
\pgfpathlineto{\pgfqpoint{2.617742in}{3.001245in}}%
\pgfpathlineto{\pgfqpoint{2.642391in}{2.975431in}}%
\pgfpathlineto{\pgfqpoint{2.667039in}{2.946063in}}%
\pgfpathlineto{\pgfqpoint{2.691688in}{2.911933in}}%
\pgfusepath{stroke}%
\end{pgfscope}%
\begin{pgfscope}%
\pgfpathrectangle{\pgfqpoint{0.880000in}{2.849537in}}{\pgfqpoint{1.897959in}{1.372727in}} %
\pgfusepath{clip}%
\pgfsetbuttcap%
\pgfsetroundjoin%
\definecolor{currentfill}{rgb}{0.000000,0.000000,1.000000}%
\pgfsetfillcolor{currentfill}%
\pgfsetlinewidth{1.003750pt}%
\definecolor{currentstroke}{rgb}{0.000000,0.000000,1.000000}%
\pgfsetstrokecolor{currentstroke}%
\pgfsetdash{}{0pt}%
\pgfsys@defobject{currentmarker}{\pgfqpoint{-0.041667in}{-0.041667in}}{\pgfqpoint{0.041667in}{0.041667in}}{%
\pgfpathmoveto{\pgfqpoint{0.000000in}{-0.041667in}}%
\pgfpathcurveto{\pgfqpoint{0.011050in}{-0.041667in}}{\pgfqpoint{0.021649in}{-0.037276in}}{\pgfqpoint{0.029463in}{-0.029463in}}%
\pgfpathcurveto{\pgfqpoint{0.037276in}{-0.021649in}}{\pgfqpoint{0.041667in}{-0.011050in}}{\pgfqpoint{0.041667in}{0.000000in}}%
\pgfpathcurveto{\pgfqpoint{0.041667in}{0.011050in}}{\pgfqpoint{0.037276in}{0.021649in}}{\pgfqpoint{0.029463in}{0.029463in}}%
\pgfpathcurveto{\pgfqpoint{0.021649in}{0.037276in}}{\pgfqpoint{0.011050in}{0.041667in}}{\pgfqpoint{0.000000in}{0.041667in}}%
\pgfpathcurveto{\pgfqpoint{-0.011050in}{0.041667in}}{\pgfqpoint{-0.021649in}{0.037276in}}{\pgfqpoint{-0.029463in}{0.029463in}}%
\pgfpathcurveto{\pgfqpoint{-0.037276in}{0.021649in}}{\pgfqpoint{-0.041667in}{0.011050in}}{\pgfqpoint{-0.041667in}{0.000000in}}%
\pgfpathcurveto{\pgfqpoint{-0.041667in}{-0.011050in}}{\pgfqpoint{-0.037276in}{-0.021649in}}{\pgfqpoint{-0.029463in}{-0.029463in}}%
\pgfpathcurveto{\pgfqpoint{-0.021649in}{-0.037276in}}{\pgfqpoint{-0.011050in}{-0.041667in}}{\pgfqpoint{0.000000in}{-0.041667in}}%
\pgfpathclose%
\pgfusepath{stroke,fill}%
}%
\begin{pgfscope}%
\pgfsys@transformshift{0.966271in}{3.541127in}%
\pgfsys@useobject{currentmarker}{}%
\end{pgfscope}%
\begin{pgfscope}%
\pgfsys@transformshift{1.311354in}{3.465619in}%
\pgfsys@useobject{currentmarker}{}%
\end{pgfscope}%
\begin{pgfscope}%
\pgfsys@transformshift{1.656438in}{3.390302in}%
\pgfsys@useobject{currentmarker}{}%
\end{pgfscope}%
\begin{pgfscope}%
\pgfsys@transformshift{2.001521in}{3.301522in}%
\pgfsys@useobject{currentmarker}{}%
\end{pgfscope}%
\begin{pgfscope}%
\pgfsys@transformshift{2.346605in}{3.177536in}%
\pgfsys@useobject{currentmarker}{}%
\end{pgfscope}%
\begin{pgfscope}%
\pgfsys@transformshift{2.691688in}{2.911933in}%
\pgfsys@useobject{currentmarker}{}%
\end{pgfscope}%
\end{pgfscope}%
\begin{pgfscope}%
\pgfpathrectangle{\pgfqpoint{0.880000in}{2.849537in}}{\pgfqpoint{1.897959in}{1.372727in}} %
\pgfusepath{clip}%
\pgfsetbuttcap%
\pgfsetroundjoin%
\pgfsetlinewidth{1.505625pt}%
\definecolor{currentstroke}{rgb}{0.000000,0.750000,0.750000}%
\pgfsetstrokecolor{currentstroke}%
\pgfsetdash{{9.600000pt}{2.400000pt}{1.500000pt}{2.400000pt}}{0.000000pt}%
\pgfpathmoveto{\pgfqpoint{0.966271in}{3.949557in}}%
\pgfpathlineto{\pgfqpoint{0.990920in}{3.916184in}}%
\pgfpathlineto{\pgfqpoint{1.015569in}{3.886378in}}%
\pgfpathlineto{\pgfqpoint{1.040217in}{3.859496in}}%
\pgfpathlineto{\pgfqpoint{1.064866in}{3.835055in}}%
\pgfpathlineto{\pgfqpoint{1.089515in}{3.812682in}}%
\pgfpathlineto{\pgfqpoint{1.114164in}{3.792087in}}%
\pgfpathlineto{\pgfqpoint{1.138813in}{3.773036in}}%
\pgfpathlineto{\pgfqpoint{1.163461in}{3.755338in}}%
\pgfpathlineto{\pgfqpoint{1.188110in}{3.738837in}}%
\pgfpathlineto{\pgfqpoint{1.212759in}{3.723401in}}%
\pgfpathlineto{\pgfqpoint{1.237408in}{3.708921in}}%
\pgfpathlineto{\pgfqpoint{1.262057in}{3.695303in}}%
\pgfpathlineto{\pgfqpoint{1.286706in}{3.682467in}}%
\pgfpathlineto{\pgfqpoint{1.311354in}{3.670342in}}%
\pgfpathlineto{\pgfqpoint{1.336003in}{3.658868in}}%
\pgfpathlineto{\pgfqpoint{1.360652in}{3.647993in}}%
\pgfpathlineto{\pgfqpoint{1.385301in}{3.637669in}}%
\pgfpathlineto{\pgfqpoint{1.409950in}{3.627855in}}%
\pgfpathlineto{\pgfqpoint{1.434598in}{3.618515in}}%
\pgfpathlineto{\pgfqpoint{1.459247in}{3.609616in}}%
\pgfpathlineto{\pgfqpoint{1.483896in}{3.601129in}}%
\pgfpathlineto{\pgfqpoint{1.508545in}{3.593026in}}%
\pgfpathlineto{\pgfqpoint{1.533194in}{3.585284in}}%
\pgfpathlineto{\pgfqpoint{1.557843in}{3.577882in}}%
\pgfpathlineto{\pgfqpoint{1.582491in}{3.570799in}}%
\pgfpathlineto{\pgfqpoint{1.607140in}{3.564019in}}%
\pgfpathlineto{\pgfqpoint{1.631789in}{3.557524in}}%
\pgfpathlineto{\pgfqpoint{1.656438in}{3.551300in}}%
\pgfpathlineto{\pgfqpoint{1.681087in}{3.545332in}}%
\pgfpathlineto{\pgfqpoint{1.705735in}{3.539609in}}%
\pgfpathlineto{\pgfqpoint{1.730384in}{3.534119in}}%
\pgfpathlineto{\pgfqpoint{1.755033in}{3.528850in}}%
\pgfpathlineto{\pgfqpoint{1.779682in}{3.523794in}}%
\pgfpathlineto{\pgfqpoint{1.804331in}{3.518940in}}%
\pgfpathlineto{\pgfqpoint{1.828980in}{3.514281in}}%
\pgfpathlineto{\pgfqpoint{1.853628in}{3.509809in}}%
\pgfpathlineto{\pgfqpoint{1.878277in}{3.505515in}}%
\pgfpathlineto{\pgfqpoint{1.902926in}{3.501394in}}%
\pgfpathlineto{\pgfqpoint{1.927575in}{3.497438in}}%
\pgfpathlineto{\pgfqpoint{1.952224in}{3.493643in}}%
\pgfpathlineto{\pgfqpoint{1.976873in}{3.490001in}}%
\pgfpathlineto{\pgfqpoint{2.001521in}{3.486510in}}%
\pgfpathlineto{\pgfqpoint{2.026170in}{3.483162in}}%
\pgfpathlineto{\pgfqpoint{2.050819in}{3.479954in}}%
\pgfpathlineto{\pgfqpoint{2.075468in}{3.476882in}}%
\pgfpathlineto{\pgfqpoint{2.100117in}{3.473942in}}%
\pgfpathlineto{\pgfqpoint{2.124765in}{3.471129in}}%
\pgfpathlineto{\pgfqpoint{2.149414in}{3.468441in}}%
\pgfpathlineto{\pgfqpoint{2.174063in}{3.465874in}}%
\pgfpathlineto{\pgfqpoint{2.198712in}{3.463425in}}%
\pgfpathlineto{\pgfqpoint{2.223361in}{3.461091in}}%
\pgfpathlineto{\pgfqpoint{2.248010in}{3.458870in}}%
\pgfpathlineto{\pgfqpoint{2.272658in}{3.456759in}}%
\pgfpathlineto{\pgfqpoint{2.297307in}{3.454755in}}%
\pgfpathlineto{\pgfqpoint{2.321956in}{3.452857in}}%
\pgfpathlineto{\pgfqpoint{2.346605in}{3.451062in}}%
\pgfpathlineto{\pgfqpoint{2.371254in}{3.449368in}}%
\pgfpathlineto{\pgfqpoint{2.395902in}{3.447774in}}%
\pgfpathlineto{\pgfqpoint{2.420551in}{3.446278in}}%
\pgfpathlineto{\pgfqpoint{2.445200in}{3.444879in}}%
\pgfpathlineto{\pgfqpoint{2.469849in}{3.443574in}}%
\pgfpathlineto{\pgfqpoint{2.494498in}{3.442363in}}%
\pgfpathlineto{\pgfqpoint{2.519147in}{3.441245in}}%
\pgfpathlineto{\pgfqpoint{2.543795in}{3.440218in}}%
\pgfpathlineto{\pgfqpoint{2.568444in}{3.439281in}}%
\pgfpathlineto{\pgfqpoint{2.593093in}{3.438433in}}%
\pgfpathlineto{\pgfqpoint{2.617742in}{3.437674in}}%
\pgfpathlineto{\pgfqpoint{2.642391in}{3.437003in}}%
\pgfpathlineto{\pgfqpoint{2.667039in}{3.436418in}}%
\pgfpathlineto{\pgfqpoint{2.691688in}{3.435921in}}%
\pgfusepath{stroke}%
\end{pgfscope}%
\begin{pgfscope}%
\pgfpathrectangle{\pgfqpoint{0.880000in}{2.849537in}}{\pgfqpoint{1.897959in}{1.372727in}} %
\pgfusepath{clip}%
\pgfsetbuttcap%
\pgfsetmiterjoin%
\definecolor{currentfill}{rgb}{0.000000,0.750000,0.750000}%
\pgfsetfillcolor{currentfill}%
\pgfsetlinewidth{1.003750pt}%
\definecolor{currentstroke}{rgb}{0.000000,0.750000,0.750000}%
\pgfsetstrokecolor{currentstroke}%
\pgfsetdash{}{0pt}%
\pgfsys@defobject{currentmarker}{\pgfqpoint{-0.041667in}{-0.041667in}}{\pgfqpoint{0.041667in}{0.041667in}}{%
\pgfpathmoveto{\pgfqpoint{-0.000000in}{-0.041667in}}%
\pgfpathlineto{\pgfqpoint{0.041667in}{0.041667in}}%
\pgfpathlineto{\pgfqpoint{-0.041667in}{0.041667in}}%
\pgfpathclose%
\pgfusepath{stroke,fill}%
}%
\begin{pgfscope}%
\pgfsys@transformshift{0.966271in}{3.949557in}%
\pgfsys@useobject{currentmarker}{}%
\end{pgfscope}%
\begin{pgfscope}%
\pgfsys@transformshift{1.311354in}{3.670342in}%
\pgfsys@useobject{currentmarker}{}%
\end{pgfscope}%
\begin{pgfscope}%
\pgfsys@transformshift{1.656438in}{3.551300in}%
\pgfsys@useobject{currentmarker}{}%
\end{pgfscope}%
\begin{pgfscope}%
\pgfsys@transformshift{2.001521in}{3.486510in}%
\pgfsys@useobject{currentmarker}{}%
\end{pgfscope}%
\begin{pgfscope}%
\pgfsys@transformshift{2.346605in}{3.451062in}%
\pgfsys@useobject{currentmarker}{}%
\end{pgfscope}%
\begin{pgfscope}%
\pgfsys@transformshift{2.691688in}{3.435921in}%
\pgfsys@useobject{currentmarker}{}%
\end{pgfscope}%
\end{pgfscope}%
\begin{pgfscope}%
\pgfpathrectangle{\pgfqpoint{0.880000in}{2.849537in}}{\pgfqpoint{1.897959in}{1.372727in}} %
\pgfusepath{clip}%
\pgfsetbuttcap%
\pgfsetroundjoin%
\pgfsetlinewidth{1.505625pt}%
\definecolor{currentstroke}{rgb}{0.000000,0.000000,0.000000}%
\pgfsetstrokecolor{currentstroke}%
\pgfsetdash{{1.500000pt}{2.475000pt}}{0.000000pt}%
\pgfpathmoveto{\pgfqpoint{0.966271in}{4.159867in}}%
\pgfpathlineto{\pgfqpoint{0.990920in}{4.149431in}}%
\pgfpathlineto{\pgfqpoint{1.015569in}{4.139488in}}%
\pgfpathlineto{\pgfqpoint{1.040217in}{4.131002in}}%
\pgfpathlineto{\pgfqpoint{1.064866in}{4.123612in}}%
\pgfpathlineto{\pgfqpoint{1.089515in}{4.117068in}}%
\pgfpathlineto{\pgfqpoint{1.114164in}{4.111189in}}%
\pgfpathlineto{\pgfqpoint{1.138813in}{4.105843in}}%
\pgfpathlineto{\pgfqpoint{1.163461in}{4.100931in}}%
\pgfpathlineto{\pgfqpoint{1.188110in}{4.096377in}}%
\pgfpathlineto{\pgfqpoint{1.212759in}{4.092124in}}%
\pgfpathlineto{\pgfqpoint{1.237408in}{4.088127in}}%
\pgfpathlineto{\pgfqpoint{1.262057in}{4.084352in}}%
\pgfpathlineto{\pgfqpoint{1.286706in}{4.080768in}}%
\pgfpathlineto{\pgfqpoint{1.311354in}{4.077354in}}%
\pgfpathlineto{\pgfqpoint{1.336003in}{4.074090in}}%
\pgfpathlineto{\pgfqpoint{1.360652in}{4.070962in}}%
\pgfpathlineto{\pgfqpoint{1.385301in}{4.067956in}}%
\pgfpathlineto{\pgfqpoint{1.409950in}{4.065063in}}%
\pgfpathlineto{\pgfqpoint{1.434598in}{4.062272in}}%
\pgfpathlineto{\pgfqpoint{1.459247in}{4.059577in}}%
\pgfpathlineto{\pgfqpoint{1.483896in}{4.056970in}}%
\pgfpathlineto{\pgfqpoint{1.508545in}{4.054447in}}%
\pgfpathlineto{\pgfqpoint{1.533194in}{4.052003in}}%
\pgfpathlineto{\pgfqpoint{1.557843in}{4.049633in}}%
\pgfpathlineto{\pgfqpoint{1.582491in}{4.047334in}}%
\pgfpathlineto{\pgfqpoint{1.607140in}{4.045102in}}%
\pgfpathlineto{\pgfqpoint{1.631789in}{4.042935in}}%
\pgfpathlineto{\pgfqpoint{1.656438in}{4.040831in}}%
\pgfpathlineto{\pgfqpoint{1.681087in}{4.038786in}}%
\pgfpathlineto{\pgfqpoint{1.705735in}{4.036799in}}%
\pgfpathlineto{\pgfqpoint{1.730384in}{4.034869in}}%
\pgfpathlineto{\pgfqpoint{1.755033in}{4.032993in}}%
\pgfpathlineto{\pgfqpoint{1.779682in}{4.031171in}}%
\pgfpathlineto{\pgfqpoint{1.804331in}{4.029400in}}%
\pgfpathlineto{\pgfqpoint{1.828980in}{4.027680in}}%
\pgfpathlineto{\pgfqpoint{1.853628in}{4.026010in}}%
\pgfpathlineto{\pgfqpoint{1.878277in}{4.024388in}}%
\pgfpathlineto{\pgfqpoint{1.902926in}{4.022813in}}%
\pgfpathlineto{\pgfqpoint{1.927575in}{4.021285in}}%
\pgfpathlineto{\pgfqpoint{1.952224in}{4.019804in}}%
\pgfpathlineto{\pgfqpoint{1.976873in}{4.018367in}}%
\pgfpathlineto{\pgfqpoint{2.001521in}{4.016975in}}%
\pgfpathlineto{\pgfqpoint{2.026170in}{4.015626in}}%
\pgfpathlineto{\pgfqpoint{2.050819in}{4.014321in}}%
\pgfpathlineto{\pgfqpoint{2.075468in}{4.013059in}}%
\pgfpathlineto{\pgfqpoint{2.100117in}{4.011839in}}%
\pgfpathlineto{\pgfqpoint{2.124765in}{4.010660in}}%
\pgfpathlineto{\pgfqpoint{2.149414in}{4.009523in}}%
\pgfpathlineto{\pgfqpoint{2.174063in}{4.008427in}}%
\pgfpathlineto{\pgfqpoint{2.198712in}{4.007371in}}%
\pgfpathlineto{\pgfqpoint{2.223361in}{4.006356in}}%
\pgfpathlineto{\pgfqpoint{2.248010in}{4.005380in}}%
\pgfpathlineto{\pgfqpoint{2.272658in}{4.004444in}}%
\pgfpathlineto{\pgfqpoint{2.297307in}{4.003547in}}%
\pgfpathlineto{\pgfqpoint{2.321956in}{4.002689in}}%
\pgfpathlineto{\pgfqpoint{2.346605in}{4.001869in}}%
\pgfpathlineto{\pgfqpoint{2.371254in}{4.001088in}}%
\pgfpathlineto{\pgfqpoint{2.395902in}{4.000345in}}%
\pgfpathlineto{\pgfqpoint{2.420551in}{3.999641in}}%
\pgfpathlineto{\pgfqpoint{2.445200in}{3.998974in}}%
\pgfpathlineto{\pgfqpoint{2.469849in}{3.998345in}}%
\pgfpathlineto{\pgfqpoint{2.494498in}{3.997754in}}%
\pgfpathlineto{\pgfqpoint{2.519147in}{3.997200in}}%
\pgfpathlineto{\pgfqpoint{2.543795in}{3.996683in}}%
\pgfpathlineto{\pgfqpoint{2.568444in}{3.996204in}}%
\pgfpathlineto{\pgfqpoint{2.593093in}{3.995762in}}%
\pgfpathlineto{\pgfqpoint{2.617742in}{3.995358in}}%
\pgfpathlineto{\pgfqpoint{2.642391in}{3.994991in}}%
\pgfpathlineto{\pgfqpoint{2.667039in}{3.994661in}}%
\pgfpathlineto{\pgfqpoint{2.691688in}{3.994368in}}%
\pgfusepath{stroke}%
\end{pgfscope}%
\begin{pgfscope}%
\pgfpathrectangle{\pgfqpoint{0.880000in}{2.849537in}}{\pgfqpoint{1.897959in}{1.372727in}} %
\pgfusepath{clip}%
\pgfsetbuttcap%
\pgfsetroundjoin%
\definecolor{currentfill}{rgb}{0.000000,0.000000,0.000000}%
\pgfsetfillcolor{currentfill}%
\pgfsetlinewidth{1.003750pt}%
\definecolor{currentstroke}{rgb}{0.000000,0.000000,0.000000}%
\pgfsetstrokecolor{currentstroke}%
\pgfsetdash{}{0pt}%
\pgfsys@defobject{currentmarker}{\pgfqpoint{-0.041667in}{-0.041667in}}{\pgfqpoint{0.041667in}{0.041667in}}{%
\pgfpathmoveto{\pgfqpoint{-0.041667in}{0.000000in}}%
\pgfpathlineto{\pgfqpoint{0.041667in}{0.000000in}}%
\pgfpathmoveto{\pgfqpoint{0.000000in}{-0.041667in}}%
\pgfpathlineto{\pgfqpoint{0.000000in}{0.041667in}}%
\pgfusepath{stroke,fill}%
}%
\begin{pgfscope}%
\pgfsys@transformshift{0.966271in}{4.159867in}%
\pgfsys@useobject{currentmarker}{}%
\end{pgfscope}%
\begin{pgfscope}%
\pgfsys@transformshift{1.311354in}{4.077354in}%
\pgfsys@useobject{currentmarker}{}%
\end{pgfscope}%
\begin{pgfscope}%
\pgfsys@transformshift{1.656438in}{4.040831in}%
\pgfsys@useobject{currentmarker}{}%
\end{pgfscope}%
\begin{pgfscope}%
\pgfsys@transformshift{2.001521in}{4.016975in}%
\pgfsys@useobject{currentmarker}{}%
\end{pgfscope}%
\begin{pgfscope}%
\pgfsys@transformshift{2.346605in}{4.001869in}%
\pgfsys@useobject{currentmarker}{}%
\end{pgfscope}%
\begin{pgfscope}%
\pgfsys@transformshift{2.691688in}{3.994368in}%
\pgfsys@useobject{currentmarker}{}%
\end{pgfscope}%
\end{pgfscope}%
\begin{pgfscope}%
\pgfsetrectcap%
\pgfsetmiterjoin%
\pgfsetlinewidth{0.803000pt}%
\definecolor{currentstroke}{rgb}{0.000000,0.000000,0.000000}%
\pgfsetstrokecolor{currentstroke}%
\pgfsetdash{}{0pt}%
\pgfpathmoveto{\pgfqpoint{0.880000in}{2.849537in}}%
\pgfpathlineto{\pgfqpoint{0.880000in}{4.222264in}}%
\pgfusepath{stroke}%
\end{pgfscope}%
\begin{pgfscope}%
\pgfsetrectcap%
\pgfsetmiterjoin%
\pgfsetlinewidth{0.803000pt}%
\definecolor{currentstroke}{rgb}{0.000000,0.000000,0.000000}%
\pgfsetstrokecolor{currentstroke}%
\pgfsetdash{}{0pt}%
\pgfpathmoveto{\pgfqpoint{2.777959in}{2.849537in}}%
\pgfpathlineto{\pgfqpoint{2.777959in}{4.222264in}}%
\pgfusepath{stroke}%
\end{pgfscope}%
\begin{pgfscope}%
\pgfsetrectcap%
\pgfsetmiterjoin%
\pgfsetlinewidth{0.803000pt}%
\definecolor{currentstroke}{rgb}{0.000000,0.000000,0.000000}%
\pgfsetstrokecolor{currentstroke}%
\pgfsetdash{}{0pt}%
\pgfpathmoveto{\pgfqpoint{0.880000in}{2.849537in}}%
\pgfpathlineto{\pgfqpoint{2.777959in}{2.849537in}}%
\pgfusepath{stroke}%
\end{pgfscope}%
\begin{pgfscope}%
\pgfsetrectcap%
\pgfsetmiterjoin%
\pgfsetlinewidth{0.803000pt}%
\definecolor{currentstroke}{rgb}{0.000000,0.000000,0.000000}%
\pgfsetstrokecolor{currentstroke}%
\pgfsetdash{}{0pt}%
\pgfpathmoveto{\pgfqpoint{0.880000in}{4.222264in}}%
\pgfpathlineto{\pgfqpoint{2.777959in}{4.222264in}}%
\pgfusepath{stroke}%
\end{pgfscope}%
\begin{pgfscope}%
\pgfsetbuttcap%
\pgfsetmiterjoin%
\definecolor{currentfill}{rgb}{1.000000,1.000000,1.000000}%
\pgfsetfillcolor{currentfill}%
\pgfsetlinewidth{0.000000pt}%
\definecolor{currentstroke}{rgb}{0.000000,0.000000,0.000000}%
\pgfsetstrokecolor{currentstroke}%
\pgfsetstrokeopacity{0.000000}%
\pgfsetdash{}{0pt}%
\pgfpathmoveto{\pgfqpoint{3.347347in}{2.849537in}}%
\pgfpathlineto{\pgfqpoint{5.245306in}{2.849537in}}%
\pgfpathlineto{\pgfqpoint{5.245306in}{4.222264in}}%
\pgfpathlineto{\pgfqpoint{3.347347in}{4.222264in}}%
\pgfpathclose%
\pgfusepath{fill}%
\end{pgfscope}%
\begin{pgfscope}%
\pgfsetbuttcap%
\pgfsetroundjoin%
\definecolor{currentfill}{rgb}{0.000000,0.000000,0.000000}%
\pgfsetfillcolor{currentfill}%
\pgfsetlinewidth{0.803000pt}%
\definecolor{currentstroke}{rgb}{0.000000,0.000000,0.000000}%
\pgfsetstrokecolor{currentstroke}%
\pgfsetdash{}{0pt}%
\pgfsys@defobject{currentmarker}{\pgfqpoint{0.000000in}{-0.048611in}}{\pgfqpoint{0.000000in}{0.000000in}}{%
\pgfpathmoveto{\pgfqpoint{0.000000in}{0.000000in}}%
\pgfpathlineto{\pgfqpoint{0.000000in}{-0.048611in}}%
\pgfusepath{stroke,fill}%
}%
\begin{pgfscope}%
\pgfsys@transformshift{4.033763in}{2.849537in}%
\pgfsys@useobject{currentmarker}{}%
\end{pgfscope}%
\end{pgfscope}%
\begin{pgfscope}%
\pgftext[x=4.033763in,y=2.752315in,,top]{\rmfamily\fontsize{10.000000}{12.000000}\selectfont \(\displaystyle 0.2\)}%
\end{pgfscope}%
\begin{pgfscope}%
\pgfsetbuttcap%
\pgfsetroundjoin%
\definecolor{currentfill}{rgb}{0.000000,0.000000,0.000000}%
\pgfsetfillcolor{currentfill}%
\pgfsetlinewidth{0.803000pt}%
\definecolor{currentstroke}{rgb}{0.000000,0.000000,0.000000}%
\pgfsetstrokecolor{currentstroke}%
\pgfsetdash{}{0pt}%
\pgfsys@defobject{currentmarker}{\pgfqpoint{0.000000in}{-0.048611in}}{\pgfqpoint{0.000000in}{0.000000in}}{%
\pgfpathmoveto{\pgfqpoint{0.000000in}{0.000000in}}%
\pgfpathlineto{\pgfqpoint{0.000000in}{-0.048611in}}%
\pgfusepath{stroke,fill}%
}%
\begin{pgfscope}%
\pgfsys@transformshift{4.933981in}{2.849537in}%
\pgfsys@useobject{currentmarker}{}%
\end{pgfscope}%
\end{pgfscope}%
\begin{pgfscope}%
\pgftext[x=4.933981in,y=2.752315in,,top]{\rmfamily\fontsize{10.000000}{12.000000}\selectfont \(\displaystyle 0.4\)}%
\end{pgfscope}%
\begin{pgfscope}%
\pgfsetbuttcap%
\pgfsetroundjoin%
\definecolor{currentfill}{rgb}{0.000000,0.000000,0.000000}%
\pgfsetfillcolor{currentfill}%
\pgfsetlinewidth{0.803000pt}%
\definecolor{currentstroke}{rgb}{0.000000,0.000000,0.000000}%
\pgfsetstrokecolor{currentstroke}%
\pgfsetdash{}{0pt}%
\pgfsys@defobject{currentmarker}{\pgfqpoint{-0.048611in}{0.000000in}}{\pgfqpoint{0.000000in}{0.000000in}}{%
\pgfpathmoveto{\pgfqpoint{0.000000in}{0.000000in}}%
\pgfpathlineto{\pgfqpoint{-0.048611in}{0.000000in}}%
\pgfusepath{stroke,fill}%
}%
\begin{pgfscope}%
\pgfsys@transformshift{3.347347in}{3.020671in}%
\pgfsys@useobject{currentmarker}{}%
\end{pgfscope}%
\end{pgfscope}%
\begin{pgfscope}%
\pgftext[x=2.962122in,y=2.967909in,left,base]{\rmfamily\fontsize{10.000000}{12.000000}\selectfont \(\displaystyle 10^{-7}\)}%
\end{pgfscope}%
\begin{pgfscope}%
\pgfsetbuttcap%
\pgfsetroundjoin%
\definecolor{currentfill}{rgb}{0.000000,0.000000,0.000000}%
\pgfsetfillcolor{currentfill}%
\pgfsetlinewidth{0.803000pt}%
\definecolor{currentstroke}{rgb}{0.000000,0.000000,0.000000}%
\pgfsetstrokecolor{currentstroke}%
\pgfsetdash{}{0pt}%
\pgfsys@defobject{currentmarker}{\pgfqpoint{-0.048611in}{0.000000in}}{\pgfqpoint{0.000000in}{0.000000in}}{%
\pgfpathmoveto{\pgfqpoint{0.000000in}{0.000000in}}%
\pgfpathlineto{\pgfqpoint{-0.048611in}{0.000000in}}%
\pgfusepath{stroke,fill}%
}%
\begin{pgfscope}%
\pgfsys@transformshift{3.347347in}{3.392614in}%
\pgfsys@useobject{currentmarker}{}%
\end{pgfscope}%
\end{pgfscope}%
\begin{pgfscope}%
\pgftext[x=2.962122in,y=3.339853in,left,base]{\rmfamily\fontsize{10.000000}{12.000000}\selectfont \(\displaystyle 10^{-6}\)}%
\end{pgfscope}%
\begin{pgfscope}%
\pgfsetbuttcap%
\pgfsetroundjoin%
\definecolor{currentfill}{rgb}{0.000000,0.000000,0.000000}%
\pgfsetfillcolor{currentfill}%
\pgfsetlinewidth{0.803000pt}%
\definecolor{currentstroke}{rgb}{0.000000,0.000000,0.000000}%
\pgfsetstrokecolor{currentstroke}%
\pgfsetdash{}{0pt}%
\pgfsys@defobject{currentmarker}{\pgfqpoint{-0.048611in}{0.000000in}}{\pgfqpoint{0.000000in}{0.000000in}}{%
\pgfpathmoveto{\pgfqpoint{0.000000in}{0.000000in}}%
\pgfpathlineto{\pgfqpoint{-0.048611in}{0.000000in}}%
\pgfusepath{stroke,fill}%
}%
\begin{pgfscope}%
\pgfsys@transformshift{3.347347in}{3.764558in}%
\pgfsys@useobject{currentmarker}{}%
\end{pgfscope}%
\end{pgfscope}%
\begin{pgfscope}%
\pgftext[x=2.962122in,y=3.711796in,left,base]{\rmfamily\fontsize{10.000000}{12.000000}\selectfont \(\displaystyle 10^{-5}\)}%
\end{pgfscope}%
\begin{pgfscope}%
\pgfsetbuttcap%
\pgfsetroundjoin%
\definecolor{currentfill}{rgb}{0.000000,0.000000,0.000000}%
\pgfsetfillcolor{currentfill}%
\pgfsetlinewidth{0.803000pt}%
\definecolor{currentstroke}{rgb}{0.000000,0.000000,0.000000}%
\pgfsetstrokecolor{currentstroke}%
\pgfsetdash{}{0pt}%
\pgfsys@defobject{currentmarker}{\pgfqpoint{-0.048611in}{0.000000in}}{\pgfqpoint{0.000000in}{0.000000in}}{%
\pgfpathmoveto{\pgfqpoint{0.000000in}{0.000000in}}%
\pgfpathlineto{\pgfqpoint{-0.048611in}{0.000000in}}%
\pgfusepath{stroke,fill}%
}%
\begin{pgfscope}%
\pgfsys@transformshift{3.347347in}{4.136501in}%
\pgfsys@useobject{currentmarker}{}%
\end{pgfscope}%
\end{pgfscope}%
\begin{pgfscope}%
\pgftext[x=2.962122in,y=4.083740in,left,base]{\rmfamily\fontsize{10.000000}{12.000000}\selectfont \(\displaystyle 10^{-4}\)}%
\end{pgfscope}%
\begin{pgfscope}%
\pgfsetbuttcap%
\pgfsetroundjoin%
\definecolor{currentfill}{rgb}{0.000000,0.000000,0.000000}%
\pgfsetfillcolor{currentfill}%
\pgfsetlinewidth{0.602250pt}%
\definecolor{currentstroke}{rgb}{0.000000,0.000000,0.000000}%
\pgfsetstrokecolor{currentstroke}%
\pgfsetdash{}{0pt}%
\pgfsys@defobject{currentmarker}{\pgfqpoint{-0.027778in}{0.000000in}}{\pgfqpoint{0.000000in}{0.000000in}}{%
\pgfpathmoveto{\pgfqpoint{0.000000in}{0.000000in}}%
\pgfpathlineto{\pgfqpoint{-0.027778in}{0.000000in}}%
\pgfusepath{stroke,fill}%
}%
\begin{pgfscope}%
\pgfsys@transformshift{3.347347in}{2.872659in}%
\pgfsys@useobject{currentmarker}{}%
\end{pgfscope}%
\end{pgfscope}%
\begin{pgfscope}%
\pgfsetbuttcap%
\pgfsetroundjoin%
\definecolor{currentfill}{rgb}{0.000000,0.000000,0.000000}%
\pgfsetfillcolor{currentfill}%
\pgfsetlinewidth{0.602250pt}%
\definecolor{currentstroke}{rgb}{0.000000,0.000000,0.000000}%
\pgfsetstrokecolor{currentstroke}%
\pgfsetdash{}{0pt}%
\pgfsys@defobject{currentmarker}{\pgfqpoint{-0.027778in}{0.000000in}}{\pgfqpoint{0.000000in}{0.000000in}}{%
\pgfpathmoveto{\pgfqpoint{0.000000in}{0.000000in}}%
\pgfpathlineto{\pgfqpoint{-0.027778in}{0.000000in}}%
\pgfusepath{stroke,fill}%
}%
\begin{pgfscope}%
\pgfsys@transformshift{3.347347in}{2.908704in}%
\pgfsys@useobject{currentmarker}{}%
\end{pgfscope}%
\end{pgfscope}%
\begin{pgfscope}%
\pgfsetbuttcap%
\pgfsetroundjoin%
\definecolor{currentfill}{rgb}{0.000000,0.000000,0.000000}%
\pgfsetfillcolor{currentfill}%
\pgfsetlinewidth{0.602250pt}%
\definecolor{currentstroke}{rgb}{0.000000,0.000000,0.000000}%
\pgfsetstrokecolor{currentstroke}%
\pgfsetdash{}{0pt}%
\pgfsys@defobject{currentmarker}{\pgfqpoint{-0.027778in}{0.000000in}}{\pgfqpoint{0.000000in}{0.000000in}}{%
\pgfpathmoveto{\pgfqpoint{0.000000in}{0.000000in}}%
\pgfpathlineto{\pgfqpoint{-0.027778in}{0.000000in}}%
\pgfusepath{stroke,fill}%
}%
\begin{pgfscope}%
\pgfsys@transformshift{3.347347in}{2.938155in}%
\pgfsys@useobject{currentmarker}{}%
\end{pgfscope}%
\end{pgfscope}%
\begin{pgfscope}%
\pgfsetbuttcap%
\pgfsetroundjoin%
\definecolor{currentfill}{rgb}{0.000000,0.000000,0.000000}%
\pgfsetfillcolor{currentfill}%
\pgfsetlinewidth{0.602250pt}%
\definecolor{currentstroke}{rgb}{0.000000,0.000000,0.000000}%
\pgfsetstrokecolor{currentstroke}%
\pgfsetdash{}{0pt}%
\pgfsys@defobject{currentmarker}{\pgfqpoint{-0.027778in}{0.000000in}}{\pgfqpoint{0.000000in}{0.000000in}}{%
\pgfpathmoveto{\pgfqpoint{0.000000in}{0.000000in}}%
\pgfpathlineto{\pgfqpoint{-0.027778in}{0.000000in}}%
\pgfusepath{stroke,fill}%
}%
\begin{pgfscope}%
\pgfsys@transformshift{3.347347in}{2.963056in}%
\pgfsys@useobject{currentmarker}{}%
\end{pgfscope}%
\end{pgfscope}%
\begin{pgfscope}%
\pgfsetbuttcap%
\pgfsetroundjoin%
\definecolor{currentfill}{rgb}{0.000000,0.000000,0.000000}%
\pgfsetfillcolor{currentfill}%
\pgfsetlinewidth{0.602250pt}%
\definecolor{currentstroke}{rgb}{0.000000,0.000000,0.000000}%
\pgfsetstrokecolor{currentstroke}%
\pgfsetdash{}{0pt}%
\pgfsys@defobject{currentmarker}{\pgfqpoint{-0.027778in}{0.000000in}}{\pgfqpoint{0.000000in}{0.000000in}}{%
\pgfpathmoveto{\pgfqpoint{0.000000in}{0.000000in}}%
\pgfpathlineto{\pgfqpoint{-0.027778in}{0.000000in}}%
\pgfusepath{stroke,fill}%
}%
\begin{pgfscope}%
\pgfsys@transformshift{3.347347in}{2.984626in}%
\pgfsys@useobject{currentmarker}{}%
\end{pgfscope}%
\end{pgfscope}%
\begin{pgfscope}%
\pgfsetbuttcap%
\pgfsetroundjoin%
\definecolor{currentfill}{rgb}{0.000000,0.000000,0.000000}%
\pgfsetfillcolor{currentfill}%
\pgfsetlinewidth{0.602250pt}%
\definecolor{currentstroke}{rgb}{0.000000,0.000000,0.000000}%
\pgfsetstrokecolor{currentstroke}%
\pgfsetdash{}{0pt}%
\pgfsys@defobject{currentmarker}{\pgfqpoint{-0.027778in}{0.000000in}}{\pgfqpoint{0.000000in}{0.000000in}}{%
\pgfpathmoveto{\pgfqpoint{0.000000in}{0.000000in}}%
\pgfpathlineto{\pgfqpoint{-0.027778in}{0.000000in}}%
\pgfusepath{stroke,fill}%
}%
\begin{pgfscope}%
\pgfsys@transformshift{3.347347in}{3.003651in}%
\pgfsys@useobject{currentmarker}{}%
\end{pgfscope}%
\end{pgfscope}%
\begin{pgfscope}%
\pgfsetbuttcap%
\pgfsetroundjoin%
\definecolor{currentfill}{rgb}{0.000000,0.000000,0.000000}%
\pgfsetfillcolor{currentfill}%
\pgfsetlinewidth{0.602250pt}%
\definecolor{currentstroke}{rgb}{0.000000,0.000000,0.000000}%
\pgfsetstrokecolor{currentstroke}%
\pgfsetdash{}{0pt}%
\pgfsys@defobject{currentmarker}{\pgfqpoint{-0.027778in}{0.000000in}}{\pgfqpoint{0.000000in}{0.000000in}}{%
\pgfpathmoveto{\pgfqpoint{0.000000in}{0.000000in}}%
\pgfpathlineto{\pgfqpoint{-0.027778in}{0.000000in}}%
\pgfusepath{stroke,fill}%
}%
\begin{pgfscope}%
\pgfsys@transformshift{3.347347in}{3.132637in}%
\pgfsys@useobject{currentmarker}{}%
\end{pgfscope}%
\end{pgfscope}%
\begin{pgfscope}%
\pgfsetbuttcap%
\pgfsetroundjoin%
\definecolor{currentfill}{rgb}{0.000000,0.000000,0.000000}%
\pgfsetfillcolor{currentfill}%
\pgfsetlinewidth{0.602250pt}%
\definecolor{currentstroke}{rgb}{0.000000,0.000000,0.000000}%
\pgfsetstrokecolor{currentstroke}%
\pgfsetdash{}{0pt}%
\pgfsys@defobject{currentmarker}{\pgfqpoint{-0.027778in}{0.000000in}}{\pgfqpoint{0.000000in}{0.000000in}}{%
\pgfpathmoveto{\pgfqpoint{0.000000in}{0.000000in}}%
\pgfpathlineto{\pgfqpoint{-0.027778in}{0.000000in}}%
\pgfusepath{stroke,fill}%
}%
\begin{pgfscope}%
\pgfsys@transformshift{3.347347in}{3.198133in}%
\pgfsys@useobject{currentmarker}{}%
\end{pgfscope}%
\end{pgfscope}%
\begin{pgfscope}%
\pgfsetbuttcap%
\pgfsetroundjoin%
\definecolor{currentfill}{rgb}{0.000000,0.000000,0.000000}%
\pgfsetfillcolor{currentfill}%
\pgfsetlinewidth{0.602250pt}%
\definecolor{currentstroke}{rgb}{0.000000,0.000000,0.000000}%
\pgfsetstrokecolor{currentstroke}%
\pgfsetdash{}{0pt}%
\pgfsys@defobject{currentmarker}{\pgfqpoint{-0.027778in}{0.000000in}}{\pgfqpoint{0.000000in}{0.000000in}}{%
\pgfpathmoveto{\pgfqpoint{0.000000in}{0.000000in}}%
\pgfpathlineto{\pgfqpoint{-0.027778in}{0.000000in}}%
\pgfusepath{stroke,fill}%
}%
\begin{pgfscope}%
\pgfsys@transformshift{3.347347in}{3.244603in}%
\pgfsys@useobject{currentmarker}{}%
\end{pgfscope}%
\end{pgfscope}%
\begin{pgfscope}%
\pgfsetbuttcap%
\pgfsetroundjoin%
\definecolor{currentfill}{rgb}{0.000000,0.000000,0.000000}%
\pgfsetfillcolor{currentfill}%
\pgfsetlinewidth{0.602250pt}%
\definecolor{currentstroke}{rgb}{0.000000,0.000000,0.000000}%
\pgfsetstrokecolor{currentstroke}%
\pgfsetdash{}{0pt}%
\pgfsys@defobject{currentmarker}{\pgfqpoint{-0.027778in}{0.000000in}}{\pgfqpoint{0.000000in}{0.000000in}}{%
\pgfpathmoveto{\pgfqpoint{0.000000in}{0.000000in}}%
\pgfpathlineto{\pgfqpoint{-0.027778in}{0.000000in}}%
\pgfusepath{stroke,fill}%
}%
\begin{pgfscope}%
\pgfsys@transformshift{3.347347in}{3.280648in}%
\pgfsys@useobject{currentmarker}{}%
\end{pgfscope}%
\end{pgfscope}%
\begin{pgfscope}%
\pgfsetbuttcap%
\pgfsetroundjoin%
\definecolor{currentfill}{rgb}{0.000000,0.000000,0.000000}%
\pgfsetfillcolor{currentfill}%
\pgfsetlinewidth{0.602250pt}%
\definecolor{currentstroke}{rgb}{0.000000,0.000000,0.000000}%
\pgfsetstrokecolor{currentstroke}%
\pgfsetdash{}{0pt}%
\pgfsys@defobject{currentmarker}{\pgfqpoint{-0.027778in}{0.000000in}}{\pgfqpoint{0.000000in}{0.000000in}}{%
\pgfpathmoveto{\pgfqpoint{0.000000in}{0.000000in}}%
\pgfpathlineto{\pgfqpoint{-0.027778in}{0.000000in}}%
\pgfusepath{stroke,fill}%
}%
\begin{pgfscope}%
\pgfsys@transformshift{3.347347in}{3.310099in}%
\pgfsys@useobject{currentmarker}{}%
\end{pgfscope}%
\end{pgfscope}%
\begin{pgfscope}%
\pgfsetbuttcap%
\pgfsetroundjoin%
\definecolor{currentfill}{rgb}{0.000000,0.000000,0.000000}%
\pgfsetfillcolor{currentfill}%
\pgfsetlinewidth{0.602250pt}%
\definecolor{currentstroke}{rgb}{0.000000,0.000000,0.000000}%
\pgfsetstrokecolor{currentstroke}%
\pgfsetdash{}{0pt}%
\pgfsys@defobject{currentmarker}{\pgfqpoint{-0.027778in}{0.000000in}}{\pgfqpoint{0.000000in}{0.000000in}}{%
\pgfpathmoveto{\pgfqpoint{0.000000in}{0.000000in}}%
\pgfpathlineto{\pgfqpoint{-0.027778in}{0.000000in}}%
\pgfusepath{stroke,fill}%
}%
\begin{pgfscope}%
\pgfsys@transformshift{3.347347in}{3.334999in}%
\pgfsys@useobject{currentmarker}{}%
\end{pgfscope}%
\end{pgfscope}%
\begin{pgfscope}%
\pgfsetbuttcap%
\pgfsetroundjoin%
\definecolor{currentfill}{rgb}{0.000000,0.000000,0.000000}%
\pgfsetfillcolor{currentfill}%
\pgfsetlinewidth{0.602250pt}%
\definecolor{currentstroke}{rgb}{0.000000,0.000000,0.000000}%
\pgfsetstrokecolor{currentstroke}%
\pgfsetdash{}{0pt}%
\pgfsys@defobject{currentmarker}{\pgfqpoint{-0.027778in}{0.000000in}}{\pgfqpoint{0.000000in}{0.000000in}}{%
\pgfpathmoveto{\pgfqpoint{0.000000in}{0.000000in}}%
\pgfpathlineto{\pgfqpoint{-0.027778in}{0.000000in}}%
\pgfusepath{stroke,fill}%
}%
\begin{pgfscope}%
\pgfsys@transformshift{3.347347in}{3.356569in}%
\pgfsys@useobject{currentmarker}{}%
\end{pgfscope}%
\end{pgfscope}%
\begin{pgfscope}%
\pgfsetbuttcap%
\pgfsetroundjoin%
\definecolor{currentfill}{rgb}{0.000000,0.000000,0.000000}%
\pgfsetfillcolor{currentfill}%
\pgfsetlinewidth{0.602250pt}%
\definecolor{currentstroke}{rgb}{0.000000,0.000000,0.000000}%
\pgfsetstrokecolor{currentstroke}%
\pgfsetdash{}{0pt}%
\pgfsys@defobject{currentmarker}{\pgfqpoint{-0.027778in}{0.000000in}}{\pgfqpoint{0.000000in}{0.000000in}}{%
\pgfpathmoveto{\pgfqpoint{0.000000in}{0.000000in}}%
\pgfpathlineto{\pgfqpoint{-0.027778in}{0.000000in}}%
\pgfusepath{stroke,fill}%
}%
\begin{pgfscope}%
\pgfsys@transformshift{3.347347in}{3.375595in}%
\pgfsys@useobject{currentmarker}{}%
\end{pgfscope}%
\end{pgfscope}%
\begin{pgfscope}%
\pgfsetbuttcap%
\pgfsetroundjoin%
\definecolor{currentfill}{rgb}{0.000000,0.000000,0.000000}%
\pgfsetfillcolor{currentfill}%
\pgfsetlinewidth{0.602250pt}%
\definecolor{currentstroke}{rgb}{0.000000,0.000000,0.000000}%
\pgfsetstrokecolor{currentstroke}%
\pgfsetdash{}{0pt}%
\pgfsys@defobject{currentmarker}{\pgfqpoint{-0.027778in}{0.000000in}}{\pgfqpoint{0.000000in}{0.000000in}}{%
\pgfpathmoveto{\pgfqpoint{0.000000in}{0.000000in}}%
\pgfpathlineto{\pgfqpoint{-0.027778in}{0.000000in}}%
\pgfusepath{stroke,fill}%
}%
\begin{pgfscope}%
\pgfsys@transformshift{3.347347in}{3.504580in}%
\pgfsys@useobject{currentmarker}{}%
\end{pgfscope}%
\end{pgfscope}%
\begin{pgfscope}%
\pgfsetbuttcap%
\pgfsetroundjoin%
\definecolor{currentfill}{rgb}{0.000000,0.000000,0.000000}%
\pgfsetfillcolor{currentfill}%
\pgfsetlinewidth{0.602250pt}%
\definecolor{currentstroke}{rgb}{0.000000,0.000000,0.000000}%
\pgfsetstrokecolor{currentstroke}%
\pgfsetdash{}{0pt}%
\pgfsys@defobject{currentmarker}{\pgfqpoint{-0.027778in}{0.000000in}}{\pgfqpoint{0.000000in}{0.000000in}}{%
\pgfpathmoveto{\pgfqpoint{0.000000in}{0.000000in}}%
\pgfpathlineto{\pgfqpoint{-0.027778in}{0.000000in}}%
\pgfusepath{stroke,fill}%
}%
\begin{pgfscope}%
\pgfsys@transformshift{3.347347in}{3.570076in}%
\pgfsys@useobject{currentmarker}{}%
\end{pgfscope}%
\end{pgfscope}%
\begin{pgfscope}%
\pgfsetbuttcap%
\pgfsetroundjoin%
\definecolor{currentfill}{rgb}{0.000000,0.000000,0.000000}%
\pgfsetfillcolor{currentfill}%
\pgfsetlinewidth{0.602250pt}%
\definecolor{currentstroke}{rgb}{0.000000,0.000000,0.000000}%
\pgfsetstrokecolor{currentstroke}%
\pgfsetdash{}{0pt}%
\pgfsys@defobject{currentmarker}{\pgfqpoint{-0.027778in}{0.000000in}}{\pgfqpoint{0.000000in}{0.000000in}}{%
\pgfpathmoveto{\pgfqpoint{0.000000in}{0.000000in}}%
\pgfpathlineto{\pgfqpoint{-0.027778in}{0.000000in}}%
\pgfusepath{stroke,fill}%
}%
\begin{pgfscope}%
\pgfsys@transformshift{3.347347in}{3.616546in}%
\pgfsys@useobject{currentmarker}{}%
\end{pgfscope}%
\end{pgfscope}%
\begin{pgfscope}%
\pgfsetbuttcap%
\pgfsetroundjoin%
\definecolor{currentfill}{rgb}{0.000000,0.000000,0.000000}%
\pgfsetfillcolor{currentfill}%
\pgfsetlinewidth{0.602250pt}%
\definecolor{currentstroke}{rgb}{0.000000,0.000000,0.000000}%
\pgfsetstrokecolor{currentstroke}%
\pgfsetdash{}{0pt}%
\pgfsys@defobject{currentmarker}{\pgfqpoint{-0.027778in}{0.000000in}}{\pgfqpoint{0.000000in}{0.000000in}}{%
\pgfpathmoveto{\pgfqpoint{0.000000in}{0.000000in}}%
\pgfpathlineto{\pgfqpoint{-0.027778in}{0.000000in}}%
\pgfusepath{stroke,fill}%
}%
\begin{pgfscope}%
\pgfsys@transformshift{3.347347in}{3.652591in}%
\pgfsys@useobject{currentmarker}{}%
\end{pgfscope}%
\end{pgfscope}%
\begin{pgfscope}%
\pgfsetbuttcap%
\pgfsetroundjoin%
\definecolor{currentfill}{rgb}{0.000000,0.000000,0.000000}%
\pgfsetfillcolor{currentfill}%
\pgfsetlinewidth{0.602250pt}%
\definecolor{currentstroke}{rgb}{0.000000,0.000000,0.000000}%
\pgfsetstrokecolor{currentstroke}%
\pgfsetdash{}{0pt}%
\pgfsys@defobject{currentmarker}{\pgfqpoint{-0.027778in}{0.000000in}}{\pgfqpoint{0.000000in}{0.000000in}}{%
\pgfpathmoveto{\pgfqpoint{0.000000in}{0.000000in}}%
\pgfpathlineto{\pgfqpoint{-0.027778in}{0.000000in}}%
\pgfusepath{stroke,fill}%
}%
\begin{pgfscope}%
\pgfsys@transformshift{3.347347in}{3.682042in}%
\pgfsys@useobject{currentmarker}{}%
\end{pgfscope}%
\end{pgfscope}%
\begin{pgfscope}%
\pgfsetbuttcap%
\pgfsetroundjoin%
\definecolor{currentfill}{rgb}{0.000000,0.000000,0.000000}%
\pgfsetfillcolor{currentfill}%
\pgfsetlinewidth{0.602250pt}%
\definecolor{currentstroke}{rgb}{0.000000,0.000000,0.000000}%
\pgfsetstrokecolor{currentstroke}%
\pgfsetdash{}{0pt}%
\pgfsys@defobject{currentmarker}{\pgfqpoint{-0.027778in}{0.000000in}}{\pgfqpoint{0.000000in}{0.000000in}}{%
\pgfpathmoveto{\pgfqpoint{0.000000in}{0.000000in}}%
\pgfpathlineto{\pgfqpoint{-0.027778in}{0.000000in}}%
\pgfusepath{stroke,fill}%
}%
\begin{pgfscope}%
\pgfsys@transformshift{3.347347in}{3.706943in}%
\pgfsys@useobject{currentmarker}{}%
\end{pgfscope}%
\end{pgfscope}%
\begin{pgfscope}%
\pgfsetbuttcap%
\pgfsetroundjoin%
\definecolor{currentfill}{rgb}{0.000000,0.000000,0.000000}%
\pgfsetfillcolor{currentfill}%
\pgfsetlinewidth{0.602250pt}%
\definecolor{currentstroke}{rgb}{0.000000,0.000000,0.000000}%
\pgfsetstrokecolor{currentstroke}%
\pgfsetdash{}{0pt}%
\pgfsys@defobject{currentmarker}{\pgfqpoint{-0.027778in}{0.000000in}}{\pgfqpoint{0.000000in}{0.000000in}}{%
\pgfpathmoveto{\pgfqpoint{0.000000in}{0.000000in}}%
\pgfpathlineto{\pgfqpoint{-0.027778in}{0.000000in}}%
\pgfusepath{stroke,fill}%
}%
\begin{pgfscope}%
\pgfsys@transformshift{3.347347in}{3.728513in}%
\pgfsys@useobject{currentmarker}{}%
\end{pgfscope}%
\end{pgfscope}%
\begin{pgfscope}%
\pgfsetbuttcap%
\pgfsetroundjoin%
\definecolor{currentfill}{rgb}{0.000000,0.000000,0.000000}%
\pgfsetfillcolor{currentfill}%
\pgfsetlinewidth{0.602250pt}%
\definecolor{currentstroke}{rgb}{0.000000,0.000000,0.000000}%
\pgfsetstrokecolor{currentstroke}%
\pgfsetdash{}{0pt}%
\pgfsys@defobject{currentmarker}{\pgfqpoint{-0.027778in}{0.000000in}}{\pgfqpoint{0.000000in}{0.000000in}}{%
\pgfpathmoveto{\pgfqpoint{0.000000in}{0.000000in}}%
\pgfpathlineto{\pgfqpoint{-0.027778in}{0.000000in}}%
\pgfusepath{stroke,fill}%
}%
\begin{pgfscope}%
\pgfsys@transformshift{3.347347in}{3.747538in}%
\pgfsys@useobject{currentmarker}{}%
\end{pgfscope}%
\end{pgfscope}%
\begin{pgfscope}%
\pgfsetbuttcap%
\pgfsetroundjoin%
\definecolor{currentfill}{rgb}{0.000000,0.000000,0.000000}%
\pgfsetfillcolor{currentfill}%
\pgfsetlinewidth{0.602250pt}%
\definecolor{currentstroke}{rgb}{0.000000,0.000000,0.000000}%
\pgfsetstrokecolor{currentstroke}%
\pgfsetdash{}{0pt}%
\pgfsys@defobject{currentmarker}{\pgfqpoint{-0.027778in}{0.000000in}}{\pgfqpoint{0.000000in}{0.000000in}}{%
\pgfpathmoveto{\pgfqpoint{0.000000in}{0.000000in}}%
\pgfpathlineto{\pgfqpoint{-0.027778in}{0.000000in}}%
\pgfusepath{stroke,fill}%
}%
\begin{pgfscope}%
\pgfsys@transformshift{3.347347in}{3.876524in}%
\pgfsys@useobject{currentmarker}{}%
\end{pgfscope}%
\end{pgfscope}%
\begin{pgfscope}%
\pgfsetbuttcap%
\pgfsetroundjoin%
\definecolor{currentfill}{rgb}{0.000000,0.000000,0.000000}%
\pgfsetfillcolor{currentfill}%
\pgfsetlinewidth{0.602250pt}%
\definecolor{currentstroke}{rgb}{0.000000,0.000000,0.000000}%
\pgfsetstrokecolor{currentstroke}%
\pgfsetdash{}{0pt}%
\pgfsys@defobject{currentmarker}{\pgfqpoint{-0.027778in}{0.000000in}}{\pgfqpoint{0.000000in}{0.000000in}}{%
\pgfpathmoveto{\pgfqpoint{0.000000in}{0.000000in}}%
\pgfpathlineto{\pgfqpoint{-0.027778in}{0.000000in}}%
\pgfusepath{stroke,fill}%
}%
\begin{pgfscope}%
\pgfsys@transformshift{3.347347in}{3.942020in}%
\pgfsys@useobject{currentmarker}{}%
\end{pgfscope}%
\end{pgfscope}%
\begin{pgfscope}%
\pgfsetbuttcap%
\pgfsetroundjoin%
\definecolor{currentfill}{rgb}{0.000000,0.000000,0.000000}%
\pgfsetfillcolor{currentfill}%
\pgfsetlinewidth{0.602250pt}%
\definecolor{currentstroke}{rgb}{0.000000,0.000000,0.000000}%
\pgfsetstrokecolor{currentstroke}%
\pgfsetdash{}{0pt}%
\pgfsys@defobject{currentmarker}{\pgfqpoint{-0.027778in}{0.000000in}}{\pgfqpoint{0.000000in}{0.000000in}}{%
\pgfpathmoveto{\pgfqpoint{0.000000in}{0.000000in}}%
\pgfpathlineto{\pgfqpoint{-0.027778in}{0.000000in}}%
\pgfusepath{stroke,fill}%
}%
\begin{pgfscope}%
\pgfsys@transformshift{3.347347in}{3.988490in}%
\pgfsys@useobject{currentmarker}{}%
\end{pgfscope}%
\end{pgfscope}%
\begin{pgfscope}%
\pgfsetbuttcap%
\pgfsetroundjoin%
\definecolor{currentfill}{rgb}{0.000000,0.000000,0.000000}%
\pgfsetfillcolor{currentfill}%
\pgfsetlinewidth{0.602250pt}%
\definecolor{currentstroke}{rgb}{0.000000,0.000000,0.000000}%
\pgfsetstrokecolor{currentstroke}%
\pgfsetdash{}{0pt}%
\pgfsys@defobject{currentmarker}{\pgfqpoint{-0.027778in}{0.000000in}}{\pgfqpoint{0.000000in}{0.000000in}}{%
\pgfpathmoveto{\pgfqpoint{0.000000in}{0.000000in}}%
\pgfpathlineto{\pgfqpoint{-0.027778in}{0.000000in}}%
\pgfusepath{stroke,fill}%
}%
\begin{pgfscope}%
\pgfsys@transformshift{3.347347in}{4.024535in}%
\pgfsys@useobject{currentmarker}{}%
\end{pgfscope}%
\end{pgfscope}%
\begin{pgfscope}%
\pgfsetbuttcap%
\pgfsetroundjoin%
\definecolor{currentfill}{rgb}{0.000000,0.000000,0.000000}%
\pgfsetfillcolor{currentfill}%
\pgfsetlinewidth{0.602250pt}%
\definecolor{currentstroke}{rgb}{0.000000,0.000000,0.000000}%
\pgfsetstrokecolor{currentstroke}%
\pgfsetdash{}{0pt}%
\pgfsys@defobject{currentmarker}{\pgfqpoint{-0.027778in}{0.000000in}}{\pgfqpoint{0.000000in}{0.000000in}}{%
\pgfpathmoveto{\pgfqpoint{0.000000in}{0.000000in}}%
\pgfpathlineto{\pgfqpoint{-0.027778in}{0.000000in}}%
\pgfusepath{stroke,fill}%
}%
\begin{pgfscope}%
\pgfsys@transformshift{3.347347in}{4.053986in}%
\pgfsys@useobject{currentmarker}{}%
\end{pgfscope}%
\end{pgfscope}%
\begin{pgfscope}%
\pgfsetbuttcap%
\pgfsetroundjoin%
\definecolor{currentfill}{rgb}{0.000000,0.000000,0.000000}%
\pgfsetfillcolor{currentfill}%
\pgfsetlinewidth{0.602250pt}%
\definecolor{currentstroke}{rgb}{0.000000,0.000000,0.000000}%
\pgfsetstrokecolor{currentstroke}%
\pgfsetdash{}{0pt}%
\pgfsys@defobject{currentmarker}{\pgfqpoint{-0.027778in}{0.000000in}}{\pgfqpoint{0.000000in}{0.000000in}}{%
\pgfpathmoveto{\pgfqpoint{0.000000in}{0.000000in}}%
\pgfpathlineto{\pgfqpoint{-0.027778in}{0.000000in}}%
\pgfusepath{stroke,fill}%
}%
\begin{pgfscope}%
\pgfsys@transformshift{3.347347in}{4.078886in}%
\pgfsys@useobject{currentmarker}{}%
\end{pgfscope}%
\end{pgfscope}%
\begin{pgfscope}%
\pgfsetbuttcap%
\pgfsetroundjoin%
\definecolor{currentfill}{rgb}{0.000000,0.000000,0.000000}%
\pgfsetfillcolor{currentfill}%
\pgfsetlinewidth{0.602250pt}%
\definecolor{currentstroke}{rgb}{0.000000,0.000000,0.000000}%
\pgfsetstrokecolor{currentstroke}%
\pgfsetdash{}{0pt}%
\pgfsys@defobject{currentmarker}{\pgfqpoint{-0.027778in}{0.000000in}}{\pgfqpoint{0.000000in}{0.000000in}}{%
\pgfpathmoveto{\pgfqpoint{0.000000in}{0.000000in}}%
\pgfpathlineto{\pgfqpoint{-0.027778in}{0.000000in}}%
\pgfusepath{stroke,fill}%
}%
\begin{pgfscope}%
\pgfsys@transformshift{3.347347in}{4.100456in}%
\pgfsys@useobject{currentmarker}{}%
\end{pgfscope}%
\end{pgfscope}%
\begin{pgfscope}%
\pgfsetbuttcap%
\pgfsetroundjoin%
\definecolor{currentfill}{rgb}{0.000000,0.000000,0.000000}%
\pgfsetfillcolor{currentfill}%
\pgfsetlinewidth{0.602250pt}%
\definecolor{currentstroke}{rgb}{0.000000,0.000000,0.000000}%
\pgfsetstrokecolor{currentstroke}%
\pgfsetdash{}{0pt}%
\pgfsys@defobject{currentmarker}{\pgfqpoint{-0.027778in}{0.000000in}}{\pgfqpoint{0.000000in}{0.000000in}}{%
\pgfpathmoveto{\pgfqpoint{0.000000in}{0.000000in}}%
\pgfpathlineto{\pgfqpoint{-0.027778in}{0.000000in}}%
\pgfusepath{stroke,fill}%
}%
\begin{pgfscope}%
\pgfsys@transformshift{3.347347in}{4.119482in}%
\pgfsys@useobject{currentmarker}{}%
\end{pgfscope}%
\end{pgfscope}%
\begin{pgfscope}%
\pgfpathrectangle{\pgfqpoint{3.347347in}{2.849537in}}{\pgfqpoint{1.897959in}{1.372727in}} %
\pgfusepath{clip}%
\pgfsetbuttcap%
\pgfsetroundjoin%
\pgfsetlinewidth{1.505625pt}%
\definecolor{currentstroke}{rgb}{1.000000,0.000000,0.000000}%
\pgfsetstrokecolor{currentstroke}%
\pgfsetdash{{5.550000pt}{2.400000pt}}{0.000000pt}%
\pgfpathmoveto{\pgfqpoint{3.433618in}{4.069185in}}%
\pgfpathlineto{\pgfqpoint{3.471127in}{4.065666in}}%
\pgfpathlineto{\pgfqpoint{3.508636in}{4.062243in}}%
\pgfpathlineto{\pgfqpoint{3.546145in}{4.058912in}}%
\pgfpathlineto{\pgfqpoint{3.583654in}{4.055670in}}%
\pgfpathlineto{\pgfqpoint{3.621163in}{4.052515in}}%
\pgfpathlineto{\pgfqpoint{3.658672in}{4.049446in}}%
\pgfpathlineto{\pgfqpoint{3.696181in}{4.046462in}}%
\pgfpathlineto{\pgfqpoint{3.733690in}{4.043561in}}%
\pgfpathlineto{\pgfqpoint{3.771199in}{4.040742in}}%
\pgfpathlineto{\pgfqpoint{3.808709in}{4.038004in}}%
\pgfpathlineto{\pgfqpoint{3.846218in}{4.035346in}}%
\pgfpathlineto{\pgfqpoint{3.883727in}{4.032768in}}%
\pgfpathlineto{\pgfqpoint{3.921236in}{4.030268in}}%
\pgfpathlineto{\pgfqpoint{3.958745in}{4.027845in}}%
\pgfpathlineto{\pgfqpoint{3.996254in}{4.025499in}}%
\pgfpathlineto{\pgfqpoint{4.033763in}{4.023228in}}%
\pgfpathlineto{\pgfqpoint{4.071272in}{4.021033in}}%
\pgfpathlineto{\pgfqpoint{4.108781in}{4.018911in}}%
\pgfpathlineto{\pgfqpoint{4.146290in}{4.016862in}}%
\pgfpathlineto{\pgfqpoint{4.183799in}{4.014886in}}%
\pgfpathlineto{\pgfqpoint{4.221308in}{4.012981in}}%
\pgfpathlineto{\pgfqpoint{4.258817in}{4.011146in}}%
\pgfpathlineto{\pgfqpoint{4.296327in}{4.009382in}}%
\pgfpathlineto{\pgfqpoint{4.333836in}{4.007687in}}%
\pgfpathlineto{\pgfqpoint{4.371345in}{4.006060in}}%
\pgfpathlineto{\pgfqpoint{4.408854in}{4.004500in}}%
\pgfpathlineto{\pgfqpoint{4.446363in}{4.003008in}}%
\pgfpathlineto{\pgfqpoint{4.483872in}{4.001582in}}%
\pgfpathlineto{\pgfqpoint{4.521381in}{4.000221in}}%
\pgfpathlineto{\pgfqpoint{4.558890in}{3.998925in}}%
\pgfpathlineto{\pgfqpoint{4.596399in}{3.997694in}}%
\pgfpathlineto{\pgfqpoint{4.633908in}{3.996526in}}%
\pgfpathlineto{\pgfqpoint{4.671417in}{3.995422in}}%
\pgfpathlineto{\pgfqpoint{4.708926in}{3.994380in}}%
\pgfpathlineto{\pgfqpoint{4.746435in}{3.993401in}}%
\pgfpathlineto{\pgfqpoint{4.783945in}{3.992483in}}%
\pgfpathlineto{\pgfqpoint{4.821454in}{3.991626in}}%
\pgfpathlineto{\pgfqpoint{4.858963in}{3.990830in}}%
\pgfpathlineto{\pgfqpoint{4.896472in}{3.990095in}}%
\pgfpathlineto{\pgfqpoint{4.933981in}{3.989419in}}%
\pgfpathlineto{\pgfqpoint{4.971490in}{3.988804in}}%
\pgfpathlineto{\pgfqpoint{5.008999in}{3.988247in}}%
\pgfpathlineto{\pgfqpoint{5.046508in}{3.987750in}}%
\pgfpathlineto{\pgfqpoint{5.084017in}{3.987312in}}%
\pgfpathlineto{\pgfqpoint{5.121526in}{3.986932in}}%
\pgfpathlineto{\pgfqpoint{5.159035in}{3.986611in}}%
\pgfusepath{stroke}%
\end{pgfscope}%
\begin{pgfscope}%
\pgfpathrectangle{\pgfqpoint{3.347347in}{2.849537in}}{\pgfqpoint{1.897959in}{1.372727in}} %
\pgfusepath{clip}%
\pgfsetbuttcap%
\pgfsetmiterjoin%
\definecolor{currentfill}{rgb}{1.000000,0.000000,0.000000}%
\pgfsetfillcolor{currentfill}%
\pgfsetlinewidth{1.003750pt}%
\definecolor{currentstroke}{rgb}{1.000000,0.000000,0.000000}%
\pgfsetstrokecolor{currentstroke}%
\pgfsetdash{}{0pt}%
\pgfsys@defobject{currentmarker}{\pgfqpoint{-0.041667in}{-0.041667in}}{\pgfqpoint{0.041667in}{0.041667in}}{%
\pgfpathmoveto{\pgfqpoint{-0.041667in}{-0.041667in}}%
\pgfpathlineto{\pgfqpoint{0.041667in}{-0.041667in}}%
\pgfpathlineto{\pgfqpoint{0.041667in}{0.041667in}}%
\pgfpathlineto{\pgfqpoint{-0.041667in}{0.041667in}}%
\pgfpathclose%
\pgfusepath{stroke,fill}%
}%
\begin{pgfscope}%
\pgfsys@transformshift{3.433618in}{4.069185in}%
\pgfsys@useobject{currentmarker}{}%
\end{pgfscope}%
\begin{pgfscope}%
\pgfsys@transformshift{3.808709in}{4.038004in}%
\pgfsys@useobject{currentmarker}{}%
\end{pgfscope}%
\begin{pgfscope}%
\pgfsys@transformshift{4.183799in}{4.014886in}%
\pgfsys@useobject{currentmarker}{}%
\end{pgfscope}%
\begin{pgfscope}%
\pgfsys@transformshift{4.558890in}{3.998925in}%
\pgfsys@useobject{currentmarker}{}%
\end{pgfscope}%
\begin{pgfscope}%
\pgfsys@transformshift{4.933981in}{3.989419in}%
\pgfsys@useobject{currentmarker}{}%
\end{pgfscope}%
\end{pgfscope}%
\begin{pgfscope}%
\pgfpathrectangle{\pgfqpoint{3.347347in}{2.849537in}}{\pgfqpoint{1.897959in}{1.372727in}} %
\pgfusepath{clip}%
\pgfsetrectcap%
\pgfsetroundjoin%
\pgfsetlinewidth{1.505625pt}%
\definecolor{currentstroke}{rgb}{0.000000,0.000000,1.000000}%
\pgfsetstrokecolor{currentstroke}%
\pgfsetdash{}{0pt}%
\pgfpathmoveto{\pgfqpoint{3.433618in}{3.448135in}}%
\pgfpathlineto{\pgfqpoint{3.471127in}{3.439500in}}%
\pgfpathlineto{\pgfqpoint{3.508636in}{3.431208in}}%
\pgfpathlineto{\pgfqpoint{3.546145in}{3.423172in}}%
\pgfpathlineto{\pgfqpoint{3.583654in}{3.415329in}}%
\pgfpathlineto{\pgfqpoint{3.621163in}{3.407630in}}%
\pgfpathlineto{\pgfqpoint{3.658672in}{3.400038in}}%
\pgfpathlineto{\pgfqpoint{3.696181in}{3.392522in}}%
\pgfpathlineto{\pgfqpoint{3.733690in}{3.385057in}}%
\pgfpathlineto{\pgfqpoint{3.771199in}{3.377622in}}%
\pgfpathlineto{\pgfqpoint{3.808709in}{3.370198in}}%
\pgfpathlineto{\pgfqpoint{3.846218in}{3.362768in}}%
\pgfpathlineto{\pgfqpoint{3.883727in}{3.355317in}}%
\pgfpathlineto{\pgfqpoint{3.921236in}{3.347830in}}%
\pgfpathlineto{\pgfqpoint{3.958745in}{3.340294in}}%
\pgfpathlineto{\pgfqpoint{3.996254in}{3.332696in}}%
\pgfpathlineto{\pgfqpoint{4.033763in}{3.325023in}}%
\pgfpathlineto{\pgfqpoint{4.071272in}{3.317262in}}%
\pgfpathlineto{\pgfqpoint{4.108781in}{3.309398in}}%
\pgfpathlineto{\pgfqpoint{4.146290in}{3.301420in}}%
\pgfpathlineto{\pgfqpoint{4.183799in}{3.293313in}}%
\pgfpathlineto{\pgfqpoint{4.221308in}{3.285061in}}%
\pgfpathlineto{\pgfqpoint{4.258817in}{3.276649in}}%
\pgfpathlineto{\pgfqpoint{4.296327in}{3.268060in}}%
\pgfpathlineto{\pgfqpoint{4.333836in}{3.259275in}}%
\pgfpathlineto{\pgfqpoint{4.371345in}{3.250276in}}%
\pgfpathlineto{\pgfqpoint{4.408854in}{3.241039in}}%
\pgfpathlineto{\pgfqpoint{4.446363in}{3.231541in}}%
\pgfpathlineto{\pgfqpoint{4.483872in}{3.221755in}}%
\pgfpathlineto{\pgfqpoint{4.521381in}{3.211651in}}%
\pgfpathlineto{\pgfqpoint{4.558890in}{3.201196in}}%
\pgfpathlineto{\pgfqpoint{4.596399in}{3.190350in}}%
\pgfpathlineto{\pgfqpoint{4.633908in}{3.179071in}}%
\pgfpathlineto{\pgfqpoint{4.671417in}{3.167308in}}%
\pgfpathlineto{\pgfqpoint{4.708926in}{3.155002in}}%
\pgfpathlineto{\pgfqpoint{4.746435in}{3.142084in}}%
\pgfpathlineto{\pgfqpoint{4.783945in}{3.128472in}}%
\pgfpathlineto{\pgfqpoint{4.821454in}{3.114069in}}%
\pgfpathlineto{\pgfqpoint{4.858963in}{3.098754in}}%
\pgfpathlineto{\pgfqpoint{4.896472in}{3.082382in}}%
\pgfpathlineto{\pgfqpoint{4.933981in}{3.064767in}}%
\pgfpathlineto{\pgfqpoint{4.971490in}{3.045673in}}%
\pgfpathlineto{\pgfqpoint{5.008999in}{3.024791in}}%
\pgfpathlineto{\pgfqpoint{5.046508in}{3.001706in}}%
\pgfpathlineto{\pgfqpoint{5.084017in}{2.975838in}}%
\pgfpathlineto{\pgfqpoint{5.121526in}{2.946346in}}%
\pgfpathlineto{\pgfqpoint{5.159035in}{2.911933in}}%
\pgfusepath{stroke}%
\end{pgfscope}%
\begin{pgfscope}%
\pgfpathrectangle{\pgfqpoint{3.347347in}{2.849537in}}{\pgfqpoint{1.897959in}{1.372727in}} %
\pgfusepath{clip}%
\pgfsetbuttcap%
\pgfsetroundjoin%
\definecolor{currentfill}{rgb}{0.000000,0.000000,1.000000}%
\pgfsetfillcolor{currentfill}%
\pgfsetlinewidth{1.003750pt}%
\definecolor{currentstroke}{rgb}{0.000000,0.000000,1.000000}%
\pgfsetstrokecolor{currentstroke}%
\pgfsetdash{}{0pt}%
\pgfsys@defobject{currentmarker}{\pgfqpoint{-0.041667in}{-0.041667in}}{\pgfqpoint{0.041667in}{0.041667in}}{%
\pgfpathmoveto{\pgfqpoint{0.000000in}{-0.041667in}}%
\pgfpathcurveto{\pgfqpoint{0.011050in}{-0.041667in}}{\pgfqpoint{0.021649in}{-0.037276in}}{\pgfqpoint{0.029463in}{-0.029463in}}%
\pgfpathcurveto{\pgfqpoint{0.037276in}{-0.021649in}}{\pgfqpoint{0.041667in}{-0.011050in}}{\pgfqpoint{0.041667in}{0.000000in}}%
\pgfpathcurveto{\pgfqpoint{0.041667in}{0.011050in}}{\pgfqpoint{0.037276in}{0.021649in}}{\pgfqpoint{0.029463in}{0.029463in}}%
\pgfpathcurveto{\pgfqpoint{0.021649in}{0.037276in}}{\pgfqpoint{0.011050in}{0.041667in}}{\pgfqpoint{0.000000in}{0.041667in}}%
\pgfpathcurveto{\pgfqpoint{-0.011050in}{0.041667in}}{\pgfqpoint{-0.021649in}{0.037276in}}{\pgfqpoint{-0.029463in}{0.029463in}}%
\pgfpathcurveto{\pgfqpoint{-0.037276in}{0.021649in}}{\pgfqpoint{-0.041667in}{0.011050in}}{\pgfqpoint{-0.041667in}{0.000000in}}%
\pgfpathcurveto{\pgfqpoint{-0.041667in}{-0.011050in}}{\pgfqpoint{-0.037276in}{-0.021649in}}{\pgfqpoint{-0.029463in}{-0.029463in}}%
\pgfpathcurveto{\pgfqpoint{-0.021649in}{-0.037276in}}{\pgfqpoint{-0.011050in}{-0.041667in}}{\pgfqpoint{0.000000in}{-0.041667in}}%
\pgfpathclose%
\pgfusepath{stroke,fill}%
}%
\begin{pgfscope}%
\pgfsys@transformshift{3.433618in}{3.448135in}%
\pgfsys@useobject{currentmarker}{}%
\end{pgfscope}%
\begin{pgfscope}%
\pgfsys@transformshift{3.808709in}{3.370198in}%
\pgfsys@useobject{currentmarker}{}%
\end{pgfscope}%
\begin{pgfscope}%
\pgfsys@transformshift{4.183799in}{3.293313in}%
\pgfsys@useobject{currentmarker}{}%
\end{pgfscope}%
\begin{pgfscope}%
\pgfsys@transformshift{4.558890in}{3.201196in}%
\pgfsys@useobject{currentmarker}{}%
\end{pgfscope}%
\begin{pgfscope}%
\pgfsys@transformshift{4.933981in}{3.064767in}%
\pgfsys@useobject{currentmarker}{}%
\end{pgfscope}%
\end{pgfscope}%
\begin{pgfscope}%
\pgfpathrectangle{\pgfqpoint{3.347347in}{2.849537in}}{\pgfqpoint{1.897959in}{1.372727in}} %
\pgfusepath{clip}%
\pgfsetbuttcap%
\pgfsetroundjoin%
\pgfsetlinewidth{1.505625pt}%
\definecolor{currentstroke}{rgb}{0.000000,0.750000,0.750000}%
\pgfsetstrokecolor{currentstroke}%
\pgfsetdash{{9.600000pt}{2.400000pt}{1.500000pt}{2.400000pt}}{0.000000pt}%
\pgfpathmoveto{\pgfqpoint{3.433618in}{4.020969in}}%
\pgfpathlineto{\pgfqpoint{3.471127in}{3.994255in}}%
\pgfpathlineto{\pgfqpoint{3.508636in}{3.970443in}}%
\pgfpathlineto{\pgfqpoint{3.546145in}{3.949017in}}%
\pgfpathlineto{\pgfqpoint{3.583654in}{3.929591in}}%
\pgfpathlineto{\pgfqpoint{3.621163in}{3.911865in}}%
\pgfpathlineto{\pgfqpoint{3.658672in}{3.895605in}}%
\pgfpathlineto{\pgfqpoint{3.696181in}{3.880620in}}%
\pgfpathlineto{\pgfqpoint{3.733690in}{3.866758in}}%
\pgfpathlineto{\pgfqpoint{3.771199in}{3.853891in}}%
\pgfpathlineto{\pgfqpoint{3.808709in}{3.841913in}}%
\pgfpathlineto{\pgfqpoint{3.846218in}{3.830734in}}%
\pgfpathlineto{\pgfqpoint{3.883727in}{3.820277in}}%
\pgfpathlineto{\pgfqpoint{3.921236in}{3.810479in}}%
\pgfpathlineto{\pgfqpoint{3.958745in}{3.801281in}}%
\pgfpathlineto{\pgfqpoint{3.996254in}{3.792636in}}%
\pgfpathlineto{\pgfqpoint{4.033763in}{3.784498in}}%
\pgfpathlineto{\pgfqpoint{4.071272in}{3.776832in}}%
\pgfpathlineto{\pgfqpoint{4.108781in}{3.769603in}}%
\pgfpathlineto{\pgfqpoint{4.146290in}{3.762782in}}%
\pgfpathlineto{\pgfqpoint{4.183799in}{3.756342in}}%
\pgfpathlineto{\pgfqpoint{4.221308in}{3.750259in}}%
\pgfpathlineto{\pgfqpoint{4.258817in}{3.744512in}}%
\pgfpathlineto{\pgfqpoint{4.296327in}{3.739083in}}%
\pgfpathlineto{\pgfqpoint{4.333836in}{3.733953in}}%
\pgfpathlineto{\pgfqpoint{4.371345in}{3.729108in}}%
\pgfpathlineto{\pgfqpoint{4.408854in}{3.724533in}}%
\pgfpathlineto{\pgfqpoint{4.446363in}{3.720215in}}%
\pgfpathlineto{\pgfqpoint{4.483872in}{3.716142in}}%
\pgfpathlineto{\pgfqpoint{4.521381in}{3.712305in}}%
\pgfpathlineto{\pgfqpoint{4.558890in}{3.708692in}}%
\pgfpathlineto{\pgfqpoint{4.596399in}{3.705297in}}%
\pgfpathlineto{\pgfqpoint{4.633908in}{3.702109in}}%
\pgfpathlineto{\pgfqpoint{4.671417in}{3.699123in}}%
\pgfpathlineto{\pgfqpoint{4.708926in}{3.696331in}}%
\pgfpathlineto{\pgfqpoint{4.746435in}{3.693727in}}%
\pgfpathlineto{\pgfqpoint{4.783945in}{3.691305in}}%
\pgfpathlineto{\pgfqpoint{4.821454in}{3.689061in}}%
\pgfpathlineto{\pgfqpoint{4.858963in}{3.686990in}}%
\pgfpathlineto{\pgfqpoint{4.896472in}{3.685088in}}%
\pgfpathlineto{\pgfqpoint{4.933981in}{3.683350in}}%
\pgfpathlineto{\pgfqpoint{4.971490in}{3.681774in}}%
\pgfpathlineto{\pgfqpoint{5.008999in}{3.680356in}}%
\pgfpathlineto{\pgfqpoint{5.046508in}{3.679094in}}%
\pgfpathlineto{\pgfqpoint{5.084017in}{3.677986in}}%
\pgfpathlineto{\pgfqpoint{5.121526in}{3.677028in}}%
\pgfpathlineto{\pgfqpoint{5.159035in}{3.676220in}}%
\pgfusepath{stroke}%
\end{pgfscope}%
\begin{pgfscope}%
\pgfpathrectangle{\pgfqpoint{3.347347in}{2.849537in}}{\pgfqpoint{1.897959in}{1.372727in}} %
\pgfusepath{clip}%
\pgfsetbuttcap%
\pgfsetmiterjoin%
\definecolor{currentfill}{rgb}{0.000000,0.750000,0.750000}%
\pgfsetfillcolor{currentfill}%
\pgfsetlinewidth{1.003750pt}%
\definecolor{currentstroke}{rgb}{0.000000,0.750000,0.750000}%
\pgfsetstrokecolor{currentstroke}%
\pgfsetdash{}{0pt}%
\pgfsys@defobject{currentmarker}{\pgfqpoint{-0.041667in}{-0.041667in}}{\pgfqpoint{0.041667in}{0.041667in}}{%
\pgfpathmoveto{\pgfqpoint{-0.000000in}{-0.041667in}}%
\pgfpathlineto{\pgfqpoint{0.041667in}{0.041667in}}%
\pgfpathlineto{\pgfqpoint{-0.041667in}{0.041667in}}%
\pgfpathclose%
\pgfusepath{stroke,fill}%
}%
\begin{pgfscope}%
\pgfsys@transformshift{3.433618in}{4.020969in}%
\pgfsys@useobject{currentmarker}{}%
\end{pgfscope}%
\begin{pgfscope}%
\pgfsys@transformshift{3.808709in}{3.841913in}%
\pgfsys@useobject{currentmarker}{}%
\end{pgfscope}%
\begin{pgfscope}%
\pgfsys@transformshift{4.183799in}{3.756342in}%
\pgfsys@useobject{currentmarker}{}%
\end{pgfscope}%
\begin{pgfscope}%
\pgfsys@transformshift{4.558890in}{3.708692in}%
\pgfsys@useobject{currentmarker}{}%
\end{pgfscope}%
\begin{pgfscope}%
\pgfsys@transformshift{4.933981in}{3.683350in}%
\pgfsys@useobject{currentmarker}{}%
\end{pgfscope}%
\end{pgfscope}%
\begin{pgfscope}%
\pgfpathrectangle{\pgfqpoint{3.347347in}{2.849537in}}{\pgfqpoint{1.897959in}{1.372727in}} %
\pgfusepath{clip}%
\pgfsetbuttcap%
\pgfsetroundjoin%
\pgfsetlinewidth{1.505625pt}%
\definecolor{currentstroke}{rgb}{0.000000,0.000000,0.000000}%
\pgfsetstrokecolor{currentstroke}%
\pgfsetdash{{1.500000pt}{2.475000pt}}{0.000000pt}%
\pgfpathmoveto{\pgfqpoint{3.433618in}{4.159867in}}%
\pgfpathlineto{\pgfqpoint{3.471127in}{4.148327in}}%
\pgfpathlineto{\pgfqpoint{3.508636in}{4.137154in}}%
\pgfpathlineto{\pgfqpoint{3.546145in}{4.127454in}}%
\pgfpathlineto{\pgfqpoint{3.583654in}{4.118903in}}%
\pgfpathlineto{\pgfqpoint{3.621163in}{4.111268in}}%
\pgfpathlineto{\pgfqpoint{3.658672in}{4.104377in}}%
\pgfpathlineto{\pgfqpoint{3.696181in}{4.098101in}}%
\pgfpathlineto{\pgfqpoint{3.733690in}{4.092340in}}%
\pgfpathlineto{\pgfqpoint{3.771199in}{4.087018in}}%
\pgfpathlineto{\pgfqpoint{3.808709in}{4.082075in}}%
\pgfpathlineto{\pgfqpoint{3.846218in}{4.077461in}}%
\pgfpathlineto{\pgfqpoint{3.883727in}{4.073138in}}%
\pgfpathlineto{\pgfqpoint{3.921236in}{4.069074in}}%
\pgfpathlineto{\pgfqpoint{3.958745in}{4.065244in}}%
\pgfpathlineto{\pgfqpoint{3.996254in}{4.061624in}}%
\pgfpathlineto{\pgfqpoint{4.033763in}{4.058197in}}%
\pgfpathlineto{\pgfqpoint{4.071272in}{4.054948in}}%
\pgfpathlineto{\pgfqpoint{4.108781in}{4.051863in}}%
\pgfpathlineto{\pgfqpoint{4.146290in}{4.048931in}}%
\pgfpathlineto{\pgfqpoint{4.183799in}{4.046142in}}%
\pgfpathlineto{\pgfqpoint{4.221308in}{4.043489in}}%
\pgfpathlineto{\pgfqpoint{4.258817in}{4.040963in}}%
\pgfpathlineto{\pgfqpoint{4.296327in}{4.038559in}}%
\pgfpathlineto{\pgfqpoint{4.333836in}{4.036270in}}%
\pgfpathlineto{\pgfqpoint{4.371345in}{4.034091in}}%
\pgfpathlineto{\pgfqpoint{4.408854in}{4.032019in}}%
\pgfpathlineto{\pgfqpoint{4.446363in}{4.030049in}}%
\pgfpathlineto{\pgfqpoint{4.483872in}{4.028178in}}%
\pgfpathlineto{\pgfqpoint{4.521381in}{4.026402in}}%
\pgfpathlineto{\pgfqpoint{4.558890in}{4.024718in}}%
\pgfpathlineto{\pgfqpoint{4.596399in}{4.023125in}}%
\pgfpathlineto{\pgfqpoint{4.633908in}{4.021619in}}%
\pgfpathlineto{\pgfqpoint{4.671417in}{4.020199in}}%
\pgfpathlineto{\pgfqpoint{4.708926in}{4.018863in}}%
\pgfpathlineto{\pgfqpoint{4.746435in}{4.017608in}}%
\pgfpathlineto{\pgfqpoint{4.783945in}{4.016434in}}%
\pgfpathlineto{\pgfqpoint{4.821454in}{4.015339in}}%
\pgfpathlineto{\pgfqpoint{4.858963in}{4.014322in}}%
\pgfpathlineto{\pgfqpoint{4.896472in}{4.013381in}}%
\pgfpathlineto{\pgfqpoint{4.933981in}{4.012516in}}%
\pgfpathlineto{\pgfqpoint{4.971490in}{4.011726in}}%
\pgfpathlineto{\pgfqpoint{5.008999in}{4.011009in}}%
\pgfpathlineto{\pgfqpoint{5.046508in}{4.010366in}}%
\pgfpathlineto{\pgfqpoint{5.084017in}{4.009796in}}%
\pgfpathlineto{\pgfqpoint{5.121526in}{4.009298in}}%
\pgfpathlineto{\pgfqpoint{5.159035in}{4.008872in}}%
\pgfusepath{stroke}%
\end{pgfscope}%
\begin{pgfscope}%
\pgfpathrectangle{\pgfqpoint{3.347347in}{2.849537in}}{\pgfqpoint{1.897959in}{1.372727in}} %
\pgfusepath{clip}%
\pgfsetbuttcap%
\pgfsetroundjoin%
\definecolor{currentfill}{rgb}{0.000000,0.000000,0.000000}%
\pgfsetfillcolor{currentfill}%
\pgfsetlinewidth{1.003750pt}%
\definecolor{currentstroke}{rgb}{0.000000,0.000000,0.000000}%
\pgfsetstrokecolor{currentstroke}%
\pgfsetdash{}{0pt}%
\pgfsys@defobject{currentmarker}{\pgfqpoint{-0.041667in}{-0.041667in}}{\pgfqpoint{0.041667in}{0.041667in}}{%
\pgfpathmoveto{\pgfqpoint{-0.041667in}{0.000000in}}%
\pgfpathlineto{\pgfqpoint{0.041667in}{0.000000in}}%
\pgfpathmoveto{\pgfqpoint{0.000000in}{-0.041667in}}%
\pgfpathlineto{\pgfqpoint{0.000000in}{0.041667in}}%
\pgfusepath{stroke,fill}%
}%
\begin{pgfscope}%
\pgfsys@transformshift{3.433618in}{4.159867in}%
\pgfsys@useobject{currentmarker}{}%
\end{pgfscope}%
\begin{pgfscope}%
\pgfsys@transformshift{3.808709in}{4.082075in}%
\pgfsys@useobject{currentmarker}{}%
\end{pgfscope}%
\begin{pgfscope}%
\pgfsys@transformshift{4.183799in}{4.046142in}%
\pgfsys@useobject{currentmarker}{}%
\end{pgfscope}%
\begin{pgfscope}%
\pgfsys@transformshift{4.558890in}{4.024718in}%
\pgfsys@useobject{currentmarker}{}%
\end{pgfscope}%
\begin{pgfscope}%
\pgfsys@transformshift{4.933981in}{4.012516in}%
\pgfsys@useobject{currentmarker}{}%
\end{pgfscope}%
\end{pgfscope}%
\begin{pgfscope}%
\pgfsetrectcap%
\pgfsetmiterjoin%
\pgfsetlinewidth{0.803000pt}%
\definecolor{currentstroke}{rgb}{0.000000,0.000000,0.000000}%
\pgfsetstrokecolor{currentstroke}%
\pgfsetdash{}{0pt}%
\pgfpathmoveto{\pgfqpoint{3.347347in}{2.849537in}}%
\pgfpathlineto{\pgfqpoint{3.347347in}{4.222264in}}%
\pgfusepath{stroke}%
\end{pgfscope}%
\begin{pgfscope}%
\pgfsetrectcap%
\pgfsetmiterjoin%
\pgfsetlinewidth{0.803000pt}%
\definecolor{currentstroke}{rgb}{0.000000,0.000000,0.000000}%
\pgfsetstrokecolor{currentstroke}%
\pgfsetdash{}{0pt}%
\pgfpathmoveto{\pgfqpoint{5.245306in}{2.849537in}}%
\pgfpathlineto{\pgfqpoint{5.245306in}{4.222264in}}%
\pgfusepath{stroke}%
\end{pgfscope}%
\begin{pgfscope}%
\pgfsetrectcap%
\pgfsetmiterjoin%
\pgfsetlinewidth{0.803000pt}%
\definecolor{currentstroke}{rgb}{0.000000,0.000000,0.000000}%
\pgfsetstrokecolor{currentstroke}%
\pgfsetdash{}{0pt}%
\pgfpathmoveto{\pgfqpoint{3.347347in}{2.849537in}}%
\pgfpathlineto{\pgfqpoint{5.245306in}{2.849537in}}%
\pgfusepath{stroke}%
\end{pgfscope}%
\begin{pgfscope}%
\pgfsetrectcap%
\pgfsetmiterjoin%
\pgfsetlinewidth{0.803000pt}%
\definecolor{currentstroke}{rgb}{0.000000,0.000000,0.000000}%
\pgfsetstrokecolor{currentstroke}%
\pgfsetdash{}{0pt}%
\pgfpathmoveto{\pgfqpoint{3.347347in}{4.222264in}}%
\pgfpathlineto{\pgfqpoint{5.245306in}{4.222264in}}%
\pgfusepath{stroke}%
\end{pgfscope}%
\begin{pgfscope}%
\pgfsetbuttcap%
\pgfsetmiterjoin%
\definecolor{currentfill}{rgb}{1.000000,1.000000,1.000000}%
\pgfsetfillcolor{currentfill}%
\pgfsetlinewidth{0.000000pt}%
\definecolor{currentstroke}{rgb}{0.000000,0.000000,0.000000}%
\pgfsetstrokecolor{currentstroke}%
\pgfsetstrokeopacity{0.000000}%
\pgfsetdash{}{0pt}%
\pgfpathmoveto{\pgfqpoint{5.814694in}{2.849537in}}%
\pgfpathlineto{\pgfqpoint{7.712653in}{2.849537in}}%
\pgfpathlineto{\pgfqpoint{7.712653in}{4.222264in}}%
\pgfpathlineto{\pgfqpoint{5.814694in}{4.222264in}}%
\pgfpathclose%
\pgfusepath{fill}%
\end{pgfscope}%
\begin{pgfscope}%
\pgfsetbuttcap%
\pgfsetroundjoin%
\definecolor{currentfill}{rgb}{0.000000,0.000000,0.000000}%
\pgfsetfillcolor{currentfill}%
\pgfsetlinewidth{0.803000pt}%
\definecolor{currentstroke}{rgb}{0.000000,0.000000,0.000000}%
\pgfsetstrokecolor{currentstroke}%
\pgfsetdash{}{0pt}%
\pgfsys@defobject{currentmarker}{\pgfqpoint{0.000000in}{-0.048611in}}{\pgfqpoint{0.000000in}{0.000000in}}{%
\pgfpathmoveto{\pgfqpoint{0.000000in}{0.000000in}}%
\pgfpathlineto{\pgfqpoint{0.000000in}{-0.048611in}}%
\pgfusepath{stroke,fill}%
}%
\begin{pgfscope}%
\pgfsys@transformshift{6.340162in}{2.849537in}%
\pgfsys@useobject{currentmarker}{}%
\end{pgfscope}%
\end{pgfscope}%
\begin{pgfscope}%
\pgftext[x=6.340162in,y=2.752315in,,top]{\rmfamily\fontsize{10.000000}{12.000000}\selectfont \(\displaystyle 0.2\)}%
\end{pgfscope}%
\begin{pgfscope}%
\pgfsetbuttcap%
\pgfsetroundjoin%
\definecolor{currentfill}{rgb}{0.000000,0.000000,0.000000}%
\pgfsetfillcolor{currentfill}%
\pgfsetlinewidth{0.803000pt}%
\definecolor{currentstroke}{rgb}{0.000000,0.000000,0.000000}%
\pgfsetstrokecolor{currentstroke}%
\pgfsetdash{}{0pt}%
\pgfsys@defobject{currentmarker}{\pgfqpoint{0.000000in}{-0.048611in}}{\pgfqpoint{0.000000in}{0.000000in}}{%
\pgfpathmoveto{\pgfqpoint{0.000000in}{0.000000in}}%
\pgfpathlineto{\pgfqpoint{0.000000in}{-0.048611in}}%
\pgfusepath{stroke,fill}%
}%
\begin{pgfscope}%
\pgfsys@transformshift{7.093071in}{2.849537in}%
\pgfsys@useobject{currentmarker}{}%
\end{pgfscope}%
\end{pgfscope}%
\begin{pgfscope}%
\pgftext[x=7.093071in,y=2.752315in,,top]{\rmfamily\fontsize{10.000000}{12.000000}\selectfont \(\displaystyle 0.4\)}%
\end{pgfscope}%
\begin{pgfscope}%
\pgfsetbuttcap%
\pgfsetroundjoin%
\definecolor{currentfill}{rgb}{0.000000,0.000000,0.000000}%
\pgfsetfillcolor{currentfill}%
\pgfsetlinewidth{0.803000pt}%
\definecolor{currentstroke}{rgb}{0.000000,0.000000,0.000000}%
\pgfsetstrokecolor{currentstroke}%
\pgfsetdash{}{0pt}%
\pgfsys@defobject{currentmarker}{\pgfqpoint{-0.048611in}{0.000000in}}{\pgfqpoint{0.000000in}{0.000000in}}{%
\pgfpathmoveto{\pgfqpoint{0.000000in}{0.000000in}}%
\pgfpathlineto{\pgfqpoint{-0.048611in}{0.000000in}}%
\pgfusepath{stroke,fill}%
}%
\begin{pgfscope}%
\pgfsys@transformshift{5.814694in}{2.977681in}%
\pgfsys@useobject{currentmarker}{}%
\end{pgfscope}%
\end{pgfscope}%
\begin{pgfscope}%
\pgftext[x=5.429469in,y=2.924920in,left,base]{\rmfamily\fontsize{10.000000}{12.000000}\selectfont \(\displaystyle 10^{-7}\)}%
\end{pgfscope}%
\begin{pgfscope}%
\pgfsetbuttcap%
\pgfsetroundjoin%
\definecolor{currentfill}{rgb}{0.000000,0.000000,0.000000}%
\pgfsetfillcolor{currentfill}%
\pgfsetlinewidth{0.803000pt}%
\definecolor{currentstroke}{rgb}{0.000000,0.000000,0.000000}%
\pgfsetstrokecolor{currentstroke}%
\pgfsetdash{}{0pt}%
\pgfsys@defobject{currentmarker}{\pgfqpoint{-0.048611in}{0.000000in}}{\pgfqpoint{0.000000in}{0.000000in}}{%
\pgfpathmoveto{\pgfqpoint{0.000000in}{0.000000in}}%
\pgfpathlineto{\pgfqpoint{-0.048611in}{0.000000in}}%
\pgfusepath{stroke,fill}%
}%
\begin{pgfscope}%
\pgfsys@transformshift{5.814694in}{3.332507in}%
\pgfsys@useobject{currentmarker}{}%
\end{pgfscope}%
\end{pgfscope}%
\begin{pgfscope}%
\pgftext[x=5.429469in,y=3.279746in,left,base]{\rmfamily\fontsize{10.000000}{12.000000}\selectfont \(\displaystyle 10^{-6}\)}%
\end{pgfscope}%
\begin{pgfscope}%
\pgfsetbuttcap%
\pgfsetroundjoin%
\definecolor{currentfill}{rgb}{0.000000,0.000000,0.000000}%
\pgfsetfillcolor{currentfill}%
\pgfsetlinewidth{0.803000pt}%
\definecolor{currentstroke}{rgb}{0.000000,0.000000,0.000000}%
\pgfsetstrokecolor{currentstroke}%
\pgfsetdash{}{0pt}%
\pgfsys@defobject{currentmarker}{\pgfqpoint{-0.048611in}{0.000000in}}{\pgfqpoint{0.000000in}{0.000000in}}{%
\pgfpathmoveto{\pgfqpoint{0.000000in}{0.000000in}}%
\pgfpathlineto{\pgfqpoint{-0.048611in}{0.000000in}}%
\pgfusepath{stroke,fill}%
}%
\begin{pgfscope}%
\pgfsys@transformshift{5.814694in}{3.687334in}%
\pgfsys@useobject{currentmarker}{}%
\end{pgfscope}%
\end{pgfscope}%
\begin{pgfscope}%
\pgftext[x=5.429469in,y=3.634572in,left,base]{\rmfamily\fontsize{10.000000}{12.000000}\selectfont \(\displaystyle 10^{-5}\)}%
\end{pgfscope}%
\begin{pgfscope}%
\pgfsetbuttcap%
\pgfsetroundjoin%
\definecolor{currentfill}{rgb}{0.000000,0.000000,0.000000}%
\pgfsetfillcolor{currentfill}%
\pgfsetlinewidth{0.803000pt}%
\definecolor{currentstroke}{rgb}{0.000000,0.000000,0.000000}%
\pgfsetstrokecolor{currentstroke}%
\pgfsetdash{}{0pt}%
\pgfsys@defobject{currentmarker}{\pgfqpoint{-0.048611in}{0.000000in}}{\pgfqpoint{0.000000in}{0.000000in}}{%
\pgfpathmoveto{\pgfqpoint{0.000000in}{0.000000in}}%
\pgfpathlineto{\pgfqpoint{-0.048611in}{0.000000in}}%
\pgfusepath{stroke,fill}%
}%
\begin{pgfscope}%
\pgfsys@transformshift{5.814694in}{4.042160in}%
\pgfsys@useobject{currentmarker}{}%
\end{pgfscope}%
\end{pgfscope}%
\begin{pgfscope}%
\pgftext[x=5.429469in,y=3.989399in,left,base]{\rmfamily\fontsize{10.000000}{12.000000}\selectfont \(\displaystyle 10^{-4}\)}%
\end{pgfscope}%
\begin{pgfscope}%
\pgfsetbuttcap%
\pgfsetroundjoin%
\definecolor{currentfill}{rgb}{0.000000,0.000000,0.000000}%
\pgfsetfillcolor{currentfill}%
\pgfsetlinewidth{0.602250pt}%
\definecolor{currentstroke}{rgb}{0.000000,0.000000,0.000000}%
\pgfsetstrokecolor{currentstroke}%
\pgfsetdash{}{0pt}%
\pgfsys@defobject{currentmarker}{\pgfqpoint{-0.027778in}{0.000000in}}{\pgfqpoint{0.000000in}{0.000000in}}{%
\pgfpathmoveto{\pgfqpoint{0.000000in}{0.000000in}}%
\pgfpathlineto{\pgfqpoint{-0.027778in}{0.000000in}}%
\pgfusepath{stroke,fill}%
}%
\begin{pgfscope}%
\pgfsys@transformshift{5.814694in}{2.870868in}%
\pgfsys@useobject{currentmarker}{}%
\end{pgfscope}%
\end{pgfscope}%
\begin{pgfscope}%
\pgfsetbuttcap%
\pgfsetroundjoin%
\definecolor{currentfill}{rgb}{0.000000,0.000000,0.000000}%
\pgfsetfillcolor{currentfill}%
\pgfsetlinewidth{0.602250pt}%
\definecolor{currentstroke}{rgb}{0.000000,0.000000,0.000000}%
\pgfsetstrokecolor{currentstroke}%
\pgfsetdash{}{0pt}%
\pgfsys@defobject{currentmarker}{\pgfqpoint{-0.027778in}{0.000000in}}{\pgfqpoint{0.000000in}{0.000000in}}{%
\pgfpathmoveto{\pgfqpoint{0.000000in}{0.000000in}}%
\pgfpathlineto{\pgfqpoint{-0.027778in}{0.000000in}}%
\pgfusepath{stroke,fill}%
}%
\begin{pgfscope}%
\pgfsys@transformshift{5.814694in}{2.898963in}%
\pgfsys@useobject{currentmarker}{}%
\end{pgfscope}%
\end{pgfscope}%
\begin{pgfscope}%
\pgfsetbuttcap%
\pgfsetroundjoin%
\definecolor{currentfill}{rgb}{0.000000,0.000000,0.000000}%
\pgfsetfillcolor{currentfill}%
\pgfsetlinewidth{0.602250pt}%
\definecolor{currentstroke}{rgb}{0.000000,0.000000,0.000000}%
\pgfsetstrokecolor{currentstroke}%
\pgfsetdash{}{0pt}%
\pgfsys@defobject{currentmarker}{\pgfqpoint{-0.027778in}{0.000000in}}{\pgfqpoint{0.000000in}{0.000000in}}{%
\pgfpathmoveto{\pgfqpoint{0.000000in}{0.000000in}}%
\pgfpathlineto{\pgfqpoint{-0.027778in}{0.000000in}}%
\pgfusepath{stroke,fill}%
}%
\begin{pgfscope}%
\pgfsys@transformshift{5.814694in}{2.922718in}%
\pgfsys@useobject{currentmarker}{}%
\end{pgfscope}%
\end{pgfscope}%
\begin{pgfscope}%
\pgfsetbuttcap%
\pgfsetroundjoin%
\definecolor{currentfill}{rgb}{0.000000,0.000000,0.000000}%
\pgfsetfillcolor{currentfill}%
\pgfsetlinewidth{0.602250pt}%
\definecolor{currentstroke}{rgb}{0.000000,0.000000,0.000000}%
\pgfsetstrokecolor{currentstroke}%
\pgfsetdash{}{0pt}%
\pgfsys@defobject{currentmarker}{\pgfqpoint{-0.027778in}{0.000000in}}{\pgfqpoint{0.000000in}{0.000000in}}{%
\pgfpathmoveto{\pgfqpoint{0.000000in}{0.000000in}}%
\pgfpathlineto{\pgfqpoint{-0.027778in}{0.000000in}}%
\pgfusepath{stroke,fill}%
}%
\begin{pgfscope}%
\pgfsys@transformshift{5.814694in}{2.943295in}%
\pgfsys@useobject{currentmarker}{}%
\end{pgfscope}%
\end{pgfscope}%
\begin{pgfscope}%
\pgfsetbuttcap%
\pgfsetroundjoin%
\definecolor{currentfill}{rgb}{0.000000,0.000000,0.000000}%
\pgfsetfillcolor{currentfill}%
\pgfsetlinewidth{0.602250pt}%
\definecolor{currentstroke}{rgb}{0.000000,0.000000,0.000000}%
\pgfsetstrokecolor{currentstroke}%
\pgfsetdash{}{0pt}%
\pgfsys@defobject{currentmarker}{\pgfqpoint{-0.027778in}{0.000000in}}{\pgfqpoint{0.000000in}{0.000000in}}{%
\pgfpathmoveto{\pgfqpoint{0.000000in}{0.000000in}}%
\pgfpathlineto{\pgfqpoint{-0.027778in}{0.000000in}}%
\pgfusepath{stroke,fill}%
}%
\begin{pgfscope}%
\pgfsys@transformshift{5.814694in}{2.961445in}%
\pgfsys@useobject{currentmarker}{}%
\end{pgfscope}%
\end{pgfscope}%
\begin{pgfscope}%
\pgfsetbuttcap%
\pgfsetroundjoin%
\definecolor{currentfill}{rgb}{0.000000,0.000000,0.000000}%
\pgfsetfillcolor{currentfill}%
\pgfsetlinewidth{0.602250pt}%
\definecolor{currentstroke}{rgb}{0.000000,0.000000,0.000000}%
\pgfsetstrokecolor{currentstroke}%
\pgfsetdash{}{0pt}%
\pgfsys@defobject{currentmarker}{\pgfqpoint{-0.027778in}{0.000000in}}{\pgfqpoint{0.000000in}{0.000000in}}{%
\pgfpathmoveto{\pgfqpoint{0.000000in}{0.000000in}}%
\pgfpathlineto{\pgfqpoint{-0.027778in}{0.000000in}}%
\pgfusepath{stroke,fill}%
}%
\begin{pgfscope}%
\pgfsys@transformshift{5.814694in}{3.084494in}%
\pgfsys@useobject{currentmarker}{}%
\end{pgfscope}%
\end{pgfscope}%
\begin{pgfscope}%
\pgfsetbuttcap%
\pgfsetroundjoin%
\definecolor{currentfill}{rgb}{0.000000,0.000000,0.000000}%
\pgfsetfillcolor{currentfill}%
\pgfsetlinewidth{0.602250pt}%
\definecolor{currentstroke}{rgb}{0.000000,0.000000,0.000000}%
\pgfsetstrokecolor{currentstroke}%
\pgfsetdash{}{0pt}%
\pgfsys@defobject{currentmarker}{\pgfqpoint{-0.027778in}{0.000000in}}{\pgfqpoint{0.000000in}{0.000000in}}{%
\pgfpathmoveto{\pgfqpoint{0.000000in}{0.000000in}}%
\pgfpathlineto{\pgfqpoint{-0.027778in}{0.000000in}}%
\pgfusepath{stroke,fill}%
}%
\begin{pgfscope}%
\pgfsys@transformshift{5.814694in}{3.146976in}%
\pgfsys@useobject{currentmarker}{}%
\end{pgfscope}%
\end{pgfscope}%
\begin{pgfscope}%
\pgfsetbuttcap%
\pgfsetroundjoin%
\definecolor{currentfill}{rgb}{0.000000,0.000000,0.000000}%
\pgfsetfillcolor{currentfill}%
\pgfsetlinewidth{0.602250pt}%
\definecolor{currentstroke}{rgb}{0.000000,0.000000,0.000000}%
\pgfsetstrokecolor{currentstroke}%
\pgfsetdash{}{0pt}%
\pgfsys@defobject{currentmarker}{\pgfqpoint{-0.027778in}{0.000000in}}{\pgfqpoint{0.000000in}{0.000000in}}{%
\pgfpathmoveto{\pgfqpoint{0.000000in}{0.000000in}}%
\pgfpathlineto{\pgfqpoint{-0.027778in}{0.000000in}}%
\pgfusepath{stroke,fill}%
}%
\begin{pgfscope}%
\pgfsys@transformshift{5.814694in}{3.191308in}%
\pgfsys@useobject{currentmarker}{}%
\end{pgfscope}%
\end{pgfscope}%
\begin{pgfscope}%
\pgfsetbuttcap%
\pgfsetroundjoin%
\definecolor{currentfill}{rgb}{0.000000,0.000000,0.000000}%
\pgfsetfillcolor{currentfill}%
\pgfsetlinewidth{0.602250pt}%
\definecolor{currentstroke}{rgb}{0.000000,0.000000,0.000000}%
\pgfsetstrokecolor{currentstroke}%
\pgfsetdash{}{0pt}%
\pgfsys@defobject{currentmarker}{\pgfqpoint{-0.027778in}{0.000000in}}{\pgfqpoint{0.000000in}{0.000000in}}{%
\pgfpathmoveto{\pgfqpoint{0.000000in}{0.000000in}}%
\pgfpathlineto{\pgfqpoint{-0.027778in}{0.000000in}}%
\pgfusepath{stroke,fill}%
}%
\begin{pgfscope}%
\pgfsys@transformshift{5.814694in}{3.225694in}%
\pgfsys@useobject{currentmarker}{}%
\end{pgfscope}%
\end{pgfscope}%
\begin{pgfscope}%
\pgfsetbuttcap%
\pgfsetroundjoin%
\definecolor{currentfill}{rgb}{0.000000,0.000000,0.000000}%
\pgfsetfillcolor{currentfill}%
\pgfsetlinewidth{0.602250pt}%
\definecolor{currentstroke}{rgb}{0.000000,0.000000,0.000000}%
\pgfsetstrokecolor{currentstroke}%
\pgfsetdash{}{0pt}%
\pgfsys@defobject{currentmarker}{\pgfqpoint{-0.027778in}{0.000000in}}{\pgfqpoint{0.000000in}{0.000000in}}{%
\pgfpathmoveto{\pgfqpoint{0.000000in}{0.000000in}}%
\pgfpathlineto{\pgfqpoint{-0.027778in}{0.000000in}}%
\pgfusepath{stroke,fill}%
}%
\begin{pgfscope}%
\pgfsys@transformshift{5.814694in}{3.253790in}%
\pgfsys@useobject{currentmarker}{}%
\end{pgfscope}%
\end{pgfscope}%
\begin{pgfscope}%
\pgfsetbuttcap%
\pgfsetroundjoin%
\definecolor{currentfill}{rgb}{0.000000,0.000000,0.000000}%
\pgfsetfillcolor{currentfill}%
\pgfsetlinewidth{0.602250pt}%
\definecolor{currentstroke}{rgb}{0.000000,0.000000,0.000000}%
\pgfsetstrokecolor{currentstroke}%
\pgfsetdash{}{0pt}%
\pgfsys@defobject{currentmarker}{\pgfqpoint{-0.027778in}{0.000000in}}{\pgfqpoint{0.000000in}{0.000000in}}{%
\pgfpathmoveto{\pgfqpoint{0.000000in}{0.000000in}}%
\pgfpathlineto{\pgfqpoint{-0.027778in}{0.000000in}}%
\pgfusepath{stroke,fill}%
}%
\begin{pgfscope}%
\pgfsys@transformshift{5.814694in}{3.277544in}%
\pgfsys@useobject{currentmarker}{}%
\end{pgfscope}%
\end{pgfscope}%
\begin{pgfscope}%
\pgfsetbuttcap%
\pgfsetroundjoin%
\definecolor{currentfill}{rgb}{0.000000,0.000000,0.000000}%
\pgfsetfillcolor{currentfill}%
\pgfsetlinewidth{0.602250pt}%
\definecolor{currentstroke}{rgb}{0.000000,0.000000,0.000000}%
\pgfsetstrokecolor{currentstroke}%
\pgfsetdash{}{0pt}%
\pgfsys@defobject{currentmarker}{\pgfqpoint{-0.027778in}{0.000000in}}{\pgfqpoint{0.000000in}{0.000000in}}{%
\pgfpathmoveto{\pgfqpoint{0.000000in}{0.000000in}}%
\pgfpathlineto{\pgfqpoint{-0.027778in}{0.000000in}}%
\pgfusepath{stroke,fill}%
}%
\begin{pgfscope}%
\pgfsys@transformshift{5.814694in}{3.298121in}%
\pgfsys@useobject{currentmarker}{}%
\end{pgfscope}%
\end{pgfscope}%
\begin{pgfscope}%
\pgfsetbuttcap%
\pgfsetroundjoin%
\definecolor{currentfill}{rgb}{0.000000,0.000000,0.000000}%
\pgfsetfillcolor{currentfill}%
\pgfsetlinewidth{0.602250pt}%
\definecolor{currentstroke}{rgb}{0.000000,0.000000,0.000000}%
\pgfsetstrokecolor{currentstroke}%
\pgfsetdash{}{0pt}%
\pgfsys@defobject{currentmarker}{\pgfqpoint{-0.027778in}{0.000000in}}{\pgfqpoint{0.000000in}{0.000000in}}{%
\pgfpathmoveto{\pgfqpoint{0.000000in}{0.000000in}}%
\pgfpathlineto{\pgfqpoint{-0.027778in}{0.000000in}}%
\pgfusepath{stroke,fill}%
}%
\begin{pgfscope}%
\pgfsys@transformshift{5.814694in}{3.316271in}%
\pgfsys@useobject{currentmarker}{}%
\end{pgfscope}%
\end{pgfscope}%
\begin{pgfscope}%
\pgfsetbuttcap%
\pgfsetroundjoin%
\definecolor{currentfill}{rgb}{0.000000,0.000000,0.000000}%
\pgfsetfillcolor{currentfill}%
\pgfsetlinewidth{0.602250pt}%
\definecolor{currentstroke}{rgb}{0.000000,0.000000,0.000000}%
\pgfsetstrokecolor{currentstroke}%
\pgfsetdash{}{0pt}%
\pgfsys@defobject{currentmarker}{\pgfqpoint{-0.027778in}{0.000000in}}{\pgfqpoint{0.000000in}{0.000000in}}{%
\pgfpathmoveto{\pgfqpoint{0.000000in}{0.000000in}}%
\pgfpathlineto{\pgfqpoint{-0.027778in}{0.000000in}}%
\pgfusepath{stroke,fill}%
}%
\begin{pgfscope}%
\pgfsys@transformshift{5.814694in}{3.439321in}%
\pgfsys@useobject{currentmarker}{}%
\end{pgfscope}%
\end{pgfscope}%
\begin{pgfscope}%
\pgfsetbuttcap%
\pgfsetroundjoin%
\definecolor{currentfill}{rgb}{0.000000,0.000000,0.000000}%
\pgfsetfillcolor{currentfill}%
\pgfsetlinewidth{0.602250pt}%
\definecolor{currentstroke}{rgb}{0.000000,0.000000,0.000000}%
\pgfsetstrokecolor{currentstroke}%
\pgfsetdash{}{0pt}%
\pgfsys@defobject{currentmarker}{\pgfqpoint{-0.027778in}{0.000000in}}{\pgfqpoint{0.000000in}{0.000000in}}{%
\pgfpathmoveto{\pgfqpoint{0.000000in}{0.000000in}}%
\pgfpathlineto{\pgfqpoint{-0.027778in}{0.000000in}}%
\pgfusepath{stroke,fill}%
}%
\begin{pgfscope}%
\pgfsys@transformshift{5.814694in}{3.501803in}%
\pgfsys@useobject{currentmarker}{}%
\end{pgfscope}%
\end{pgfscope}%
\begin{pgfscope}%
\pgfsetbuttcap%
\pgfsetroundjoin%
\definecolor{currentfill}{rgb}{0.000000,0.000000,0.000000}%
\pgfsetfillcolor{currentfill}%
\pgfsetlinewidth{0.602250pt}%
\definecolor{currentstroke}{rgb}{0.000000,0.000000,0.000000}%
\pgfsetstrokecolor{currentstroke}%
\pgfsetdash{}{0pt}%
\pgfsys@defobject{currentmarker}{\pgfqpoint{-0.027778in}{0.000000in}}{\pgfqpoint{0.000000in}{0.000000in}}{%
\pgfpathmoveto{\pgfqpoint{0.000000in}{0.000000in}}%
\pgfpathlineto{\pgfqpoint{-0.027778in}{0.000000in}}%
\pgfusepath{stroke,fill}%
}%
\begin{pgfscope}%
\pgfsys@transformshift{5.814694in}{3.546134in}%
\pgfsys@useobject{currentmarker}{}%
\end{pgfscope}%
\end{pgfscope}%
\begin{pgfscope}%
\pgfsetbuttcap%
\pgfsetroundjoin%
\definecolor{currentfill}{rgb}{0.000000,0.000000,0.000000}%
\pgfsetfillcolor{currentfill}%
\pgfsetlinewidth{0.602250pt}%
\definecolor{currentstroke}{rgb}{0.000000,0.000000,0.000000}%
\pgfsetstrokecolor{currentstroke}%
\pgfsetdash{}{0pt}%
\pgfsys@defobject{currentmarker}{\pgfqpoint{-0.027778in}{0.000000in}}{\pgfqpoint{0.000000in}{0.000000in}}{%
\pgfpathmoveto{\pgfqpoint{0.000000in}{0.000000in}}%
\pgfpathlineto{\pgfqpoint{-0.027778in}{0.000000in}}%
\pgfusepath{stroke,fill}%
}%
\begin{pgfscope}%
\pgfsys@transformshift{5.814694in}{3.580520in}%
\pgfsys@useobject{currentmarker}{}%
\end{pgfscope}%
\end{pgfscope}%
\begin{pgfscope}%
\pgfsetbuttcap%
\pgfsetroundjoin%
\definecolor{currentfill}{rgb}{0.000000,0.000000,0.000000}%
\pgfsetfillcolor{currentfill}%
\pgfsetlinewidth{0.602250pt}%
\definecolor{currentstroke}{rgb}{0.000000,0.000000,0.000000}%
\pgfsetstrokecolor{currentstroke}%
\pgfsetdash{}{0pt}%
\pgfsys@defobject{currentmarker}{\pgfqpoint{-0.027778in}{0.000000in}}{\pgfqpoint{0.000000in}{0.000000in}}{%
\pgfpathmoveto{\pgfqpoint{0.000000in}{0.000000in}}%
\pgfpathlineto{\pgfqpoint{-0.027778in}{0.000000in}}%
\pgfusepath{stroke,fill}%
}%
\begin{pgfscope}%
\pgfsys@transformshift{5.814694in}{3.608616in}%
\pgfsys@useobject{currentmarker}{}%
\end{pgfscope}%
\end{pgfscope}%
\begin{pgfscope}%
\pgfsetbuttcap%
\pgfsetroundjoin%
\definecolor{currentfill}{rgb}{0.000000,0.000000,0.000000}%
\pgfsetfillcolor{currentfill}%
\pgfsetlinewidth{0.602250pt}%
\definecolor{currentstroke}{rgb}{0.000000,0.000000,0.000000}%
\pgfsetstrokecolor{currentstroke}%
\pgfsetdash{}{0pt}%
\pgfsys@defobject{currentmarker}{\pgfqpoint{-0.027778in}{0.000000in}}{\pgfqpoint{0.000000in}{0.000000in}}{%
\pgfpathmoveto{\pgfqpoint{0.000000in}{0.000000in}}%
\pgfpathlineto{\pgfqpoint{-0.027778in}{0.000000in}}%
\pgfusepath{stroke,fill}%
}%
\begin{pgfscope}%
\pgfsys@transformshift{5.814694in}{3.632371in}%
\pgfsys@useobject{currentmarker}{}%
\end{pgfscope}%
\end{pgfscope}%
\begin{pgfscope}%
\pgfsetbuttcap%
\pgfsetroundjoin%
\definecolor{currentfill}{rgb}{0.000000,0.000000,0.000000}%
\pgfsetfillcolor{currentfill}%
\pgfsetlinewidth{0.602250pt}%
\definecolor{currentstroke}{rgb}{0.000000,0.000000,0.000000}%
\pgfsetstrokecolor{currentstroke}%
\pgfsetdash{}{0pt}%
\pgfsys@defobject{currentmarker}{\pgfqpoint{-0.027778in}{0.000000in}}{\pgfqpoint{0.000000in}{0.000000in}}{%
\pgfpathmoveto{\pgfqpoint{0.000000in}{0.000000in}}%
\pgfpathlineto{\pgfqpoint{-0.027778in}{0.000000in}}%
\pgfusepath{stroke,fill}%
}%
\begin{pgfscope}%
\pgfsys@transformshift{5.814694in}{3.652948in}%
\pgfsys@useobject{currentmarker}{}%
\end{pgfscope}%
\end{pgfscope}%
\begin{pgfscope}%
\pgfsetbuttcap%
\pgfsetroundjoin%
\definecolor{currentfill}{rgb}{0.000000,0.000000,0.000000}%
\pgfsetfillcolor{currentfill}%
\pgfsetlinewidth{0.602250pt}%
\definecolor{currentstroke}{rgb}{0.000000,0.000000,0.000000}%
\pgfsetstrokecolor{currentstroke}%
\pgfsetdash{}{0pt}%
\pgfsys@defobject{currentmarker}{\pgfqpoint{-0.027778in}{0.000000in}}{\pgfqpoint{0.000000in}{0.000000in}}{%
\pgfpathmoveto{\pgfqpoint{0.000000in}{0.000000in}}%
\pgfpathlineto{\pgfqpoint{-0.027778in}{0.000000in}}%
\pgfusepath{stroke,fill}%
}%
\begin{pgfscope}%
\pgfsys@transformshift{5.814694in}{3.671098in}%
\pgfsys@useobject{currentmarker}{}%
\end{pgfscope}%
\end{pgfscope}%
\begin{pgfscope}%
\pgfsetbuttcap%
\pgfsetroundjoin%
\definecolor{currentfill}{rgb}{0.000000,0.000000,0.000000}%
\pgfsetfillcolor{currentfill}%
\pgfsetlinewidth{0.602250pt}%
\definecolor{currentstroke}{rgb}{0.000000,0.000000,0.000000}%
\pgfsetstrokecolor{currentstroke}%
\pgfsetdash{}{0pt}%
\pgfsys@defobject{currentmarker}{\pgfqpoint{-0.027778in}{0.000000in}}{\pgfqpoint{0.000000in}{0.000000in}}{%
\pgfpathmoveto{\pgfqpoint{0.000000in}{0.000000in}}%
\pgfpathlineto{\pgfqpoint{-0.027778in}{0.000000in}}%
\pgfusepath{stroke,fill}%
}%
\begin{pgfscope}%
\pgfsys@transformshift{5.814694in}{3.794147in}%
\pgfsys@useobject{currentmarker}{}%
\end{pgfscope}%
\end{pgfscope}%
\begin{pgfscope}%
\pgfsetbuttcap%
\pgfsetroundjoin%
\definecolor{currentfill}{rgb}{0.000000,0.000000,0.000000}%
\pgfsetfillcolor{currentfill}%
\pgfsetlinewidth{0.602250pt}%
\definecolor{currentstroke}{rgb}{0.000000,0.000000,0.000000}%
\pgfsetstrokecolor{currentstroke}%
\pgfsetdash{}{0pt}%
\pgfsys@defobject{currentmarker}{\pgfqpoint{-0.027778in}{0.000000in}}{\pgfqpoint{0.000000in}{0.000000in}}{%
\pgfpathmoveto{\pgfqpoint{0.000000in}{0.000000in}}%
\pgfpathlineto{\pgfqpoint{-0.027778in}{0.000000in}}%
\pgfusepath{stroke,fill}%
}%
\begin{pgfscope}%
\pgfsys@transformshift{5.814694in}{3.856629in}%
\pgfsys@useobject{currentmarker}{}%
\end{pgfscope}%
\end{pgfscope}%
\begin{pgfscope}%
\pgfsetbuttcap%
\pgfsetroundjoin%
\definecolor{currentfill}{rgb}{0.000000,0.000000,0.000000}%
\pgfsetfillcolor{currentfill}%
\pgfsetlinewidth{0.602250pt}%
\definecolor{currentstroke}{rgb}{0.000000,0.000000,0.000000}%
\pgfsetstrokecolor{currentstroke}%
\pgfsetdash{}{0pt}%
\pgfsys@defobject{currentmarker}{\pgfqpoint{-0.027778in}{0.000000in}}{\pgfqpoint{0.000000in}{0.000000in}}{%
\pgfpathmoveto{\pgfqpoint{0.000000in}{0.000000in}}%
\pgfpathlineto{\pgfqpoint{-0.027778in}{0.000000in}}%
\pgfusepath{stroke,fill}%
}%
\begin{pgfscope}%
\pgfsys@transformshift{5.814694in}{3.900961in}%
\pgfsys@useobject{currentmarker}{}%
\end{pgfscope}%
\end{pgfscope}%
\begin{pgfscope}%
\pgfsetbuttcap%
\pgfsetroundjoin%
\definecolor{currentfill}{rgb}{0.000000,0.000000,0.000000}%
\pgfsetfillcolor{currentfill}%
\pgfsetlinewidth{0.602250pt}%
\definecolor{currentstroke}{rgb}{0.000000,0.000000,0.000000}%
\pgfsetstrokecolor{currentstroke}%
\pgfsetdash{}{0pt}%
\pgfsys@defobject{currentmarker}{\pgfqpoint{-0.027778in}{0.000000in}}{\pgfqpoint{0.000000in}{0.000000in}}{%
\pgfpathmoveto{\pgfqpoint{0.000000in}{0.000000in}}%
\pgfpathlineto{\pgfqpoint{-0.027778in}{0.000000in}}%
\pgfusepath{stroke,fill}%
}%
\begin{pgfscope}%
\pgfsys@transformshift{5.814694in}{3.935347in}%
\pgfsys@useobject{currentmarker}{}%
\end{pgfscope}%
\end{pgfscope}%
\begin{pgfscope}%
\pgfsetbuttcap%
\pgfsetroundjoin%
\definecolor{currentfill}{rgb}{0.000000,0.000000,0.000000}%
\pgfsetfillcolor{currentfill}%
\pgfsetlinewidth{0.602250pt}%
\definecolor{currentstroke}{rgb}{0.000000,0.000000,0.000000}%
\pgfsetstrokecolor{currentstroke}%
\pgfsetdash{}{0pt}%
\pgfsys@defobject{currentmarker}{\pgfqpoint{-0.027778in}{0.000000in}}{\pgfqpoint{0.000000in}{0.000000in}}{%
\pgfpathmoveto{\pgfqpoint{0.000000in}{0.000000in}}%
\pgfpathlineto{\pgfqpoint{-0.027778in}{0.000000in}}%
\pgfusepath{stroke,fill}%
}%
\begin{pgfscope}%
\pgfsys@transformshift{5.814694in}{3.963443in}%
\pgfsys@useobject{currentmarker}{}%
\end{pgfscope}%
\end{pgfscope}%
\begin{pgfscope}%
\pgfsetbuttcap%
\pgfsetroundjoin%
\definecolor{currentfill}{rgb}{0.000000,0.000000,0.000000}%
\pgfsetfillcolor{currentfill}%
\pgfsetlinewidth{0.602250pt}%
\definecolor{currentstroke}{rgb}{0.000000,0.000000,0.000000}%
\pgfsetstrokecolor{currentstroke}%
\pgfsetdash{}{0pt}%
\pgfsys@defobject{currentmarker}{\pgfqpoint{-0.027778in}{0.000000in}}{\pgfqpoint{0.000000in}{0.000000in}}{%
\pgfpathmoveto{\pgfqpoint{0.000000in}{0.000000in}}%
\pgfpathlineto{\pgfqpoint{-0.027778in}{0.000000in}}%
\pgfusepath{stroke,fill}%
}%
\begin{pgfscope}%
\pgfsys@transformshift{5.814694in}{3.987197in}%
\pgfsys@useobject{currentmarker}{}%
\end{pgfscope}%
\end{pgfscope}%
\begin{pgfscope}%
\pgfsetbuttcap%
\pgfsetroundjoin%
\definecolor{currentfill}{rgb}{0.000000,0.000000,0.000000}%
\pgfsetfillcolor{currentfill}%
\pgfsetlinewidth{0.602250pt}%
\definecolor{currentstroke}{rgb}{0.000000,0.000000,0.000000}%
\pgfsetstrokecolor{currentstroke}%
\pgfsetdash{}{0pt}%
\pgfsys@defobject{currentmarker}{\pgfqpoint{-0.027778in}{0.000000in}}{\pgfqpoint{0.000000in}{0.000000in}}{%
\pgfpathmoveto{\pgfqpoint{0.000000in}{0.000000in}}%
\pgfpathlineto{\pgfqpoint{-0.027778in}{0.000000in}}%
\pgfusepath{stroke,fill}%
}%
\begin{pgfscope}%
\pgfsys@transformshift{5.814694in}{4.007774in}%
\pgfsys@useobject{currentmarker}{}%
\end{pgfscope}%
\end{pgfscope}%
\begin{pgfscope}%
\pgfsetbuttcap%
\pgfsetroundjoin%
\definecolor{currentfill}{rgb}{0.000000,0.000000,0.000000}%
\pgfsetfillcolor{currentfill}%
\pgfsetlinewidth{0.602250pt}%
\definecolor{currentstroke}{rgb}{0.000000,0.000000,0.000000}%
\pgfsetstrokecolor{currentstroke}%
\pgfsetdash{}{0pt}%
\pgfsys@defobject{currentmarker}{\pgfqpoint{-0.027778in}{0.000000in}}{\pgfqpoint{0.000000in}{0.000000in}}{%
\pgfpathmoveto{\pgfqpoint{0.000000in}{0.000000in}}%
\pgfpathlineto{\pgfqpoint{-0.027778in}{0.000000in}}%
\pgfusepath{stroke,fill}%
}%
\begin{pgfscope}%
\pgfsys@transformshift{5.814694in}{4.025924in}%
\pgfsys@useobject{currentmarker}{}%
\end{pgfscope}%
\end{pgfscope}%
\begin{pgfscope}%
\pgfsetbuttcap%
\pgfsetroundjoin%
\definecolor{currentfill}{rgb}{0.000000,0.000000,0.000000}%
\pgfsetfillcolor{currentfill}%
\pgfsetlinewidth{0.602250pt}%
\definecolor{currentstroke}{rgb}{0.000000,0.000000,0.000000}%
\pgfsetstrokecolor{currentstroke}%
\pgfsetdash{}{0pt}%
\pgfsys@defobject{currentmarker}{\pgfqpoint{-0.027778in}{0.000000in}}{\pgfqpoint{0.000000in}{0.000000in}}{%
\pgfpathmoveto{\pgfqpoint{0.000000in}{0.000000in}}%
\pgfpathlineto{\pgfqpoint{-0.027778in}{0.000000in}}%
\pgfusepath{stroke,fill}%
}%
\begin{pgfscope}%
\pgfsys@transformshift{5.814694in}{4.148974in}%
\pgfsys@useobject{currentmarker}{}%
\end{pgfscope}%
\end{pgfscope}%
\begin{pgfscope}%
\pgfsetbuttcap%
\pgfsetroundjoin%
\definecolor{currentfill}{rgb}{0.000000,0.000000,0.000000}%
\pgfsetfillcolor{currentfill}%
\pgfsetlinewidth{0.602250pt}%
\definecolor{currentstroke}{rgb}{0.000000,0.000000,0.000000}%
\pgfsetstrokecolor{currentstroke}%
\pgfsetdash{}{0pt}%
\pgfsys@defobject{currentmarker}{\pgfqpoint{-0.027778in}{0.000000in}}{\pgfqpoint{0.000000in}{0.000000in}}{%
\pgfpathmoveto{\pgfqpoint{0.000000in}{0.000000in}}%
\pgfpathlineto{\pgfqpoint{-0.027778in}{0.000000in}}%
\pgfusepath{stroke,fill}%
}%
\begin{pgfscope}%
\pgfsys@transformshift{5.814694in}{4.211456in}%
\pgfsys@useobject{currentmarker}{}%
\end{pgfscope}%
\end{pgfscope}%
\begin{pgfscope}%
\pgfpathrectangle{\pgfqpoint{5.814694in}{2.849537in}}{\pgfqpoint{1.897959in}{1.372727in}} %
\pgfusepath{clip}%
\pgfsetbuttcap%
\pgfsetroundjoin%
\pgfsetlinewidth{1.505625pt}%
\definecolor{currentstroke}{rgb}{1.000000,0.000000,0.000000}%
\pgfsetstrokecolor{currentstroke}%
\pgfsetdash{{5.550000pt}{2.400000pt}}{0.000000pt}%
\pgfpathmoveto{\pgfqpoint{5.900965in}{4.025621in}}%
\pgfpathlineto{\pgfqpoint{5.932336in}{4.022311in}}%
\pgfpathlineto{\pgfqpoint{5.963707in}{4.019092in}}%
\pgfpathlineto{\pgfqpoint{5.995078in}{4.015961in}}%
\pgfpathlineto{\pgfqpoint{6.026450in}{4.012913in}}%
\pgfpathlineto{\pgfqpoint{6.057821in}{4.009946in}}%
\pgfpathlineto{\pgfqpoint{6.089192in}{4.007056in}}%
\pgfpathlineto{\pgfqpoint{6.120563in}{4.004242in}}%
\pgfpathlineto{\pgfqpoint{6.151935in}{4.001502in}}%
\pgfpathlineto{\pgfqpoint{6.183306in}{3.998834in}}%
\pgfpathlineto{\pgfqpoint{6.214677in}{3.996237in}}%
\pgfpathlineto{\pgfqpoint{6.246048in}{3.993708in}}%
\pgfpathlineto{\pgfqpoint{6.277419in}{3.991247in}}%
\pgfpathlineto{\pgfqpoint{6.308791in}{3.988852in}}%
\pgfpathlineto{\pgfqpoint{6.340162in}{3.986522in}}%
\pgfpathlineto{\pgfqpoint{6.371533in}{3.984257in}}%
\pgfpathlineto{\pgfqpoint{6.402904in}{3.982054in}}%
\pgfpathlineto{\pgfqpoint{6.434276in}{3.979913in}}%
\pgfpathlineto{\pgfqpoint{6.465647in}{3.977833in}}%
\pgfpathlineto{\pgfqpoint{6.497018in}{3.975812in}}%
\pgfpathlineto{\pgfqpoint{6.528389in}{3.973851in}}%
\pgfpathlineto{\pgfqpoint{6.559760in}{3.971947in}}%
\pgfpathlineto{\pgfqpoint{6.591132in}{3.970101in}}%
\pgfpathlineto{\pgfqpoint{6.622503in}{3.968310in}}%
\pgfpathlineto{\pgfqpoint{6.653874in}{3.966576in}}%
\pgfpathlineto{\pgfqpoint{6.685245in}{3.964896in}}%
\pgfpathlineto{\pgfqpoint{6.716617in}{3.963269in}}%
\pgfpathlineto{\pgfqpoint{6.747988in}{3.961696in}}%
\pgfpathlineto{\pgfqpoint{6.779359in}{3.960176in}}%
\pgfpathlineto{\pgfqpoint{6.810730in}{3.958707in}}%
\pgfpathlineto{\pgfqpoint{6.842102in}{3.957289in}}%
\pgfpathlineto{\pgfqpoint{6.873473in}{3.955921in}}%
\pgfpathlineto{\pgfqpoint{6.904844in}{3.954604in}}%
\pgfpathlineto{\pgfqpoint{6.936215in}{3.953335in}}%
\pgfpathlineto{\pgfqpoint{6.967586in}{3.952115in}}%
\pgfpathlineto{\pgfqpoint{6.998958in}{3.950944in}}%
\pgfpathlineto{\pgfqpoint{7.030329in}{3.949819in}}%
\pgfpathlineto{\pgfqpoint{7.061700in}{3.948742in}}%
\pgfpathlineto{\pgfqpoint{7.093071in}{3.947712in}}%
\pgfpathlineto{\pgfqpoint{7.124443in}{3.946727in}}%
\pgfpathlineto{\pgfqpoint{7.155814in}{3.945788in}}%
\pgfpathlineto{\pgfqpoint{7.187185in}{3.944895in}}%
\pgfpathlineto{\pgfqpoint{7.218556in}{3.944046in}}%
\pgfpathlineto{\pgfqpoint{7.249927in}{3.943241in}}%
\pgfpathlineto{\pgfqpoint{7.281299in}{3.942481in}}%
\pgfpathlineto{\pgfqpoint{7.312670in}{3.941764in}}%
\pgfpathlineto{\pgfqpoint{7.344041in}{3.941091in}}%
\pgfpathlineto{\pgfqpoint{7.375412in}{3.940461in}}%
\pgfpathlineto{\pgfqpoint{7.406784in}{3.939874in}}%
\pgfpathlineto{\pgfqpoint{7.438155in}{3.939330in}}%
\pgfpathlineto{\pgfqpoint{7.469526in}{3.938827in}}%
\pgfpathlineto{\pgfqpoint{7.500897in}{3.938367in}}%
\pgfpathlineto{\pgfqpoint{7.532269in}{3.937949in}}%
\pgfpathlineto{\pgfqpoint{7.563640in}{3.937572in}}%
\pgfpathlineto{\pgfqpoint{7.595011in}{3.937237in}}%
\pgfpathlineto{\pgfqpoint{7.626382in}{3.936944in}}%
\pgfusepath{stroke}%
\end{pgfscope}%
\begin{pgfscope}%
\pgfpathrectangle{\pgfqpoint{5.814694in}{2.849537in}}{\pgfqpoint{1.897959in}{1.372727in}} %
\pgfusepath{clip}%
\pgfsetbuttcap%
\pgfsetmiterjoin%
\definecolor{currentfill}{rgb}{1.000000,0.000000,0.000000}%
\pgfsetfillcolor{currentfill}%
\pgfsetlinewidth{1.003750pt}%
\definecolor{currentstroke}{rgb}{1.000000,0.000000,0.000000}%
\pgfsetstrokecolor{currentstroke}%
\pgfsetdash{}{0pt}%
\pgfsys@defobject{currentmarker}{\pgfqpoint{-0.041667in}{-0.041667in}}{\pgfqpoint{0.041667in}{0.041667in}}{%
\pgfpathmoveto{\pgfqpoint{-0.041667in}{-0.041667in}}%
\pgfpathlineto{\pgfqpoint{0.041667in}{-0.041667in}}%
\pgfpathlineto{\pgfqpoint{0.041667in}{0.041667in}}%
\pgfpathlineto{\pgfqpoint{-0.041667in}{0.041667in}}%
\pgfpathclose%
\pgfusepath{stroke,fill}%
}%
\begin{pgfscope}%
\pgfsys@transformshift{5.900965in}{4.025621in}%
\pgfsys@useobject{currentmarker}{}%
\end{pgfscope}%
\begin{pgfscope}%
\pgfsys@transformshift{6.246048in}{3.993708in}%
\pgfsys@useobject{currentmarker}{}%
\end{pgfscope}%
\begin{pgfscope}%
\pgfsys@transformshift{6.591132in}{3.970101in}%
\pgfsys@useobject{currentmarker}{}%
\end{pgfscope}%
\begin{pgfscope}%
\pgfsys@transformshift{6.936215in}{3.953335in}%
\pgfsys@useobject{currentmarker}{}%
\end{pgfscope}%
\begin{pgfscope}%
\pgfsys@transformshift{7.281299in}{3.942481in}%
\pgfsys@useobject{currentmarker}{}%
\end{pgfscope}%
\begin{pgfscope}%
\pgfsys@transformshift{7.626382in}{3.936944in}%
\pgfsys@useobject{currentmarker}{}%
\end{pgfscope}%
\end{pgfscope}%
\begin{pgfscope}%
\pgfpathrectangle{\pgfqpoint{5.814694in}{2.849537in}}{\pgfqpoint{1.897959in}{1.372727in}} %
\pgfusepath{clip}%
\pgfsetrectcap%
\pgfsetroundjoin%
\pgfsetlinewidth{1.505625pt}%
\definecolor{currentstroke}{rgb}{0.000000,0.000000,1.000000}%
\pgfsetstrokecolor{currentstroke}%
\pgfsetdash{}{0pt}%
\pgfpathmoveto{\pgfqpoint{5.900965in}{3.427663in}}%
\pgfpathlineto{\pgfqpoint{5.932336in}{3.419454in}}%
\pgfpathlineto{\pgfqpoint{5.963707in}{3.411636in}}%
\pgfpathlineto{\pgfqpoint{5.995078in}{3.404129in}}%
\pgfpathlineto{\pgfqpoint{6.026450in}{3.396873in}}%
\pgfpathlineto{\pgfqpoint{6.057821in}{3.389822in}}%
\pgfpathlineto{\pgfqpoint{6.089192in}{3.382937in}}%
\pgfpathlineto{\pgfqpoint{6.120563in}{3.376187in}}%
\pgfpathlineto{\pgfqpoint{6.151935in}{3.369549in}}%
\pgfpathlineto{\pgfqpoint{6.183306in}{3.363000in}}%
\pgfpathlineto{\pgfqpoint{6.214677in}{3.356522in}}%
\pgfpathlineto{\pgfqpoint{6.246048in}{3.350099in}}%
\pgfpathlineto{\pgfqpoint{6.277419in}{3.343718in}}%
\pgfpathlineto{\pgfqpoint{6.308791in}{3.337365in}}%
\pgfpathlineto{\pgfqpoint{6.340162in}{3.331029in}}%
\pgfpathlineto{\pgfqpoint{6.371533in}{3.324700in}}%
\pgfpathlineto{\pgfqpoint{6.402904in}{3.318367in}}%
\pgfpathlineto{\pgfqpoint{6.434276in}{3.312022in}}%
\pgfpathlineto{\pgfqpoint{6.465647in}{3.305656in}}%
\pgfpathlineto{\pgfqpoint{6.497018in}{3.299259in}}%
\pgfpathlineto{\pgfqpoint{6.528389in}{3.292824in}}%
\pgfpathlineto{\pgfqpoint{6.559760in}{3.286341in}}%
\pgfpathlineto{\pgfqpoint{6.591132in}{3.279803in}}%
\pgfpathlineto{\pgfqpoint{6.622503in}{3.273201in}}%
\pgfpathlineto{\pgfqpoint{6.653874in}{3.266526in}}%
\pgfpathlineto{\pgfqpoint{6.685245in}{3.259770in}}%
\pgfpathlineto{\pgfqpoint{6.716617in}{3.252922in}}%
\pgfpathlineto{\pgfqpoint{6.747988in}{3.245974in}}%
\pgfpathlineto{\pgfqpoint{6.779359in}{3.238915in}}%
\pgfpathlineto{\pgfqpoint{6.810730in}{3.231735in}}%
\pgfpathlineto{\pgfqpoint{6.842102in}{3.224421in}}%
\pgfpathlineto{\pgfqpoint{6.873473in}{3.216963in}}%
\pgfpathlineto{\pgfqpoint{6.904844in}{3.209345in}}%
\pgfpathlineto{\pgfqpoint{6.936215in}{3.201555in}}%
\pgfpathlineto{\pgfqpoint{6.967586in}{3.193576in}}%
\pgfpathlineto{\pgfqpoint{6.998958in}{3.185392in}}%
\pgfpathlineto{\pgfqpoint{7.030329in}{3.176984in}}%
\pgfpathlineto{\pgfqpoint{7.061700in}{3.168330in}}%
\pgfpathlineto{\pgfqpoint{7.093071in}{3.159408in}}%
\pgfpathlineto{\pgfqpoint{7.124443in}{3.150191in}}%
\pgfpathlineto{\pgfqpoint{7.155814in}{3.140651in}}%
\pgfpathlineto{\pgfqpoint{7.187185in}{3.130754in}}%
\pgfpathlineto{\pgfqpoint{7.218556in}{3.120462in}}%
\pgfpathlineto{\pgfqpoint{7.249927in}{3.109732in}}%
\pgfpathlineto{\pgfqpoint{7.281299in}{3.098514in}}%
\pgfpathlineto{\pgfqpoint{7.312670in}{3.086749in}}%
\pgfpathlineto{\pgfqpoint{7.344041in}{3.074366in}}%
\pgfpathlineto{\pgfqpoint{7.375412in}{3.061284in}}%
\pgfpathlineto{\pgfqpoint{7.406784in}{3.047402in}}%
\pgfpathlineto{\pgfqpoint{7.438155in}{3.032597in}}%
\pgfpathlineto{\pgfqpoint{7.469526in}{3.016719in}}%
\pgfpathlineto{\pgfqpoint{7.500897in}{2.999575in}}%
\pgfpathlineto{\pgfqpoint{7.532269in}{2.980918in}}%
\pgfpathlineto{\pgfqpoint{7.563640in}{2.960420in}}%
\pgfpathlineto{\pgfqpoint{7.595011in}{2.937636in}}%
\pgfpathlineto{\pgfqpoint{7.626382in}{2.911933in}}%
\pgfusepath{stroke}%
\end{pgfscope}%
\begin{pgfscope}%
\pgfpathrectangle{\pgfqpoint{5.814694in}{2.849537in}}{\pgfqpoint{1.897959in}{1.372727in}} %
\pgfusepath{clip}%
\pgfsetbuttcap%
\pgfsetroundjoin%
\definecolor{currentfill}{rgb}{0.000000,0.000000,1.000000}%
\pgfsetfillcolor{currentfill}%
\pgfsetlinewidth{1.003750pt}%
\definecolor{currentstroke}{rgb}{0.000000,0.000000,1.000000}%
\pgfsetstrokecolor{currentstroke}%
\pgfsetdash{}{0pt}%
\pgfsys@defobject{currentmarker}{\pgfqpoint{-0.041667in}{-0.041667in}}{\pgfqpoint{0.041667in}{0.041667in}}{%
\pgfpathmoveto{\pgfqpoint{0.000000in}{-0.041667in}}%
\pgfpathcurveto{\pgfqpoint{0.011050in}{-0.041667in}}{\pgfqpoint{0.021649in}{-0.037276in}}{\pgfqpoint{0.029463in}{-0.029463in}}%
\pgfpathcurveto{\pgfqpoint{0.037276in}{-0.021649in}}{\pgfqpoint{0.041667in}{-0.011050in}}{\pgfqpoint{0.041667in}{0.000000in}}%
\pgfpathcurveto{\pgfqpoint{0.041667in}{0.011050in}}{\pgfqpoint{0.037276in}{0.021649in}}{\pgfqpoint{0.029463in}{0.029463in}}%
\pgfpathcurveto{\pgfqpoint{0.021649in}{0.037276in}}{\pgfqpoint{0.011050in}{0.041667in}}{\pgfqpoint{0.000000in}{0.041667in}}%
\pgfpathcurveto{\pgfqpoint{-0.011050in}{0.041667in}}{\pgfqpoint{-0.021649in}{0.037276in}}{\pgfqpoint{-0.029463in}{0.029463in}}%
\pgfpathcurveto{\pgfqpoint{-0.037276in}{0.021649in}}{\pgfqpoint{-0.041667in}{0.011050in}}{\pgfqpoint{-0.041667in}{0.000000in}}%
\pgfpathcurveto{\pgfqpoint{-0.041667in}{-0.011050in}}{\pgfqpoint{-0.037276in}{-0.021649in}}{\pgfqpoint{-0.029463in}{-0.029463in}}%
\pgfpathcurveto{\pgfqpoint{-0.021649in}{-0.037276in}}{\pgfqpoint{-0.011050in}{-0.041667in}}{\pgfqpoint{0.000000in}{-0.041667in}}%
\pgfpathclose%
\pgfusepath{stroke,fill}%
}%
\begin{pgfscope}%
\pgfsys@transformshift{5.900965in}{3.427663in}%
\pgfsys@useobject{currentmarker}{}%
\end{pgfscope}%
\begin{pgfscope}%
\pgfsys@transformshift{6.246048in}{3.350099in}%
\pgfsys@useobject{currentmarker}{}%
\end{pgfscope}%
\begin{pgfscope}%
\pgfsys@transformshift{6.591132in}{3.279803in}%
\pgfsys@useobject{currentmarker}{}%
\end{pgfscope}%
\begin{pgfscope}%
\pgfsys@transformshift{6.936215in}{3.201555in}%
\pgfsys@useobject{currentmarker}{}%
\end{pgfscope}%
\begin{pgfscope}%
\pgfsys@transformshift{7.281299in}{3.098514in}%
\pgfsys@useobject{currentmarker}{}%
\end{pgfscope}%
\begin{pgfscope}%
\pgfsys@transformshift{7.626382in}{2.911933in}%
\pgfsys@useobject{currentmarker}{}%
\end{pgfscope}%
\end{pgfscope}%
\begin{pgfscope}%
\pgfpathrectangle{\pgfqpoint{5.814694in}{2.849537in}}{\pgfqpoint{1.897959in}{1.372727in}} %
\pgfusepath{clip}%
\pgfsetbuttcap%
\pgfsetroundjoin%
\pgfsetlinewidth{1.505625pt}%
\definecolor{currentstroke}{rgb}{0.000000,0.750000,0.750000}%
\pgfsetstrokecolor{currentstroke}%
\pgfsetdash{{9.600000pt}{2.400000pt}{1.500000pt}{2.400000pt}}{0.000000pt}%
\pgfpathmoveto{\pgfqpoint{5.900965in}{4.075564in}}%
\pgfpathlineto{\pgfqpoint{5.932336in}{4.054350in}}%
\pgfpathlineto{\pgfqpoint{5.963707in}{4.035181in}}%
\pgfpathlineto{\pgfqpoint{5.995078in}{4.017729in}}%
\pgfpathlineto{\pgfqpoint{6.026450in}{4.001740in}}%
\pgfpathlineto{\pgfqpoint{6.057821in}{3.987013in}}%
\pgfpathlineto{\pgfqpoint{6.089192in}{3.973386in}}%
\pgfpathlineto{\pgfqpoint{6.120563in}{3.960726in}}%
\pgfpathlineto{\pgfqpoint{6.151935in}{3.948925in}}%
\pgfpathlineto{\pgfqpoint{6.183306in}{3.937892in}}%
\pgfpathlineto{\pgfqpoint{6.214677in}{3.927548in}}%
\pgfpathlineto{\pgfqpoint{6.246048in}{3.917828in}}%
\pgfpathlineto{\pgfqpoint{6.277419in}{3.908676in}}%
\pgfpathlineto{\pgfqpoint{6.308791in}{3.900042in}}%
\pgfpathlineto{\pgfqpoint{6.340162in}{3.891884in}}%
\pgfpathlineto{\pgfqpoint{6.371533in}{3.884164in}}%
\pgfpathlineto{\pgfqpoint{6.402904in}{3.876849in}}%
\pgfpathlineto{\pgfqpoint{6.434276in}{3.869910in}}%
\pgfpathlineto{\pgfqpoint{6.465647in}{3.863321in}}%
\pgfpathlineto{\pgfqpoint{6.497018in}{3.857059in}}%
\pgfpathlineto{\pgfqpoint{6.528389in}{3.851102in}}%
\pgfpathlineto{\pgfqpoint{6.559760in}{3.845433in}}%
\pgfpathlineto{\pgfqpoint{6.591132in}{3.840033in}}%
\pgfpathlineto{\pgfqpoint{6.622503in}{3.834888in}}%
\pgfpathlineto{\pgfqpoint{6.653874in}{3.829984in}}%
\pgfpathlineto{\pgfqpoint{6.685245in}{3.825308in}}%
\pgfpathlineto{\pgfqpoint{6.716617in}{3.820848in}}%
\pgfpathlineto{\pgfqpoint{6.747988in}{3.816594in}}%
\pgfpathlineto{\pgfqpoint{6.779359in}{3.812537in}}%
\pgfpathlineto{\pgfqpoint{6.810730in}{3.808667in}}%
\pgfpathlineto{\pgfqpoint{6.842102in}{3.804976in}}%
\pgfpathlineto{\pgfqpoint{6.873473in}{3.801456in}}%
\pgfpathlineto{\pgfqpoint{6.904844in}{3.798101in}}%
\pgfpathlineto{\pgfqpoint{6.936215in}{3.794905in}}%
\pgfpathlineto{\pgfqpoint{6.967586in}{3.791861in}}%
\pgfpathlineto{\pgfqpoint{6.998958in}{3.788963in}}%
\pgfpathlineto{\pgfqpoint{7.030329in}{3.786208in}}%
\pgfpathlineto{\pgfqpoint{7.061700in}{3.783589in}}%
\pgfpathlineto{\pgfqpoint{7.093071in}{3.781103in}}%
\pgfpathlineto{\pgfqpoint{7.124443in}{3.778746in}}%
\pgfpathlineto{\pgfqpoint{7.155814in}{3.776514in}}%
\pgfpathlineto{\pgfqpoint{7.187185in}{3.774403in}}%
\pgfpathlineto{\pgfqpoint{7.218556in}{3.772411in}}%
\pgfpathlineto{\pgfqpoint{7.249927in}{3.770533in}}%
\pgfpathlineto{\pgfqpoint{7.281299in}{3.768769in}}%
\pgfpathlineto{\pgfqpoint{7.312670in}{3.767114in}}%
\pgfpathlineto{\pgfqpoint{7.344041in}{3.765567in}}%
\pgfpathlineto{\pgfqpoint{7.375412in}{3.764126in}}%
\pgfpathlineto{\pgfqpoint{7.406784in}{3.762788in}}%
\pgfpathlineto{\pgfqpoint{7.438155in}{3.761552in}}%
\pgfpathlineto{\pgfqpoint{7.469526in}{3.760416in}}%
\pgfpathlineto{\pgfqpoint{7.500897in}{3.759379in}}%
\pgfpathlineto{\pgfqpoint{7.532269in}{3.758439in}}%
\pgfpathlineto{\pgfqpoint{7.563640in}{3.757594in}}%
\pgfpathlineto{\pgfqpoint{7.595011in}{3.756845in}}%
\pgfpathlineto{\pgfqpoint{7.626382in}{3.756190in}}%
\pgfusepath{stroke}%
\end{pgfscope}%
\begin{pgfscope}%
\pgfpathrectangle{\pgfqpoint{5.814694in}{2.849537in}}{\pgfqpoint{1.897959in}{1.372727in}} %
\pgfusepath{clip}%
\pgfsetbuttcap%
\pgfsetmiterjoin%
\definecolor{currentfill}{rgb}{0.000000,0.750000,0.750000}%
\pgfsetfillcolor{currentfill}%
\pgfsetlinewidth{1.003750pt}%
\definecolor{currentstroke}{rgb}{0.000000,0.750000,0.750000}%
\pgfsetstrokecolor{currentstroke}%
\pgfsetdash{}{0pt}%
\pgfsys@defobject{currentmarker}{\pgfqpoint{-0.041667in}{-0.041667in}}{\pgfqpoint{0.041667in}{0.041667in}}{%
\pgfpathmoveto{\pgfqpoint{-0.000000in}{-0.041667in}}%
\pgfpathlineto{\pgfqpoint{0.041667in}{0.041667in}}%
\pgfpathlineto{\pgfqpoint{-0.041667in}{0.041667in}}%
\pgfpathclose%
\pgfusepath{stroke,fill}%
}%
\begin{pgfscope}%
\pgfsys@transformshift{5.900965in}{4.075564in}%
\pgfsys@useobject{currentmarker}{}%
\end{pgfscope}%
\begin{pgfscope}%
\pgfsys@transformshift{6.246048in}{3.917828in}%
\pgfsys@useobject{currentmarker}{}%
\end{pgfscope}%
\begin{pgfscope}%
\pgfsys@transformshift{6.591132in}{3.840033in}%
\pgfsys@useobject{currentmarker}{}%
\end{pgfscope}%
\begin{pgfscope}%
\pgfsys@transformshift{6.936215in}{3.794905in}%
\pgfsys@useobject{currentmarker}{}%
\end{pgfscope}%
\begin{pgfscope}%
\pgfsys@transformshift{7.281299in}{3.768769in}%
\pgfsys@useobject{currentmarker}{}%
\end{pgfscope}%
\begin{pgfscope}%
\pgfsys@transformshift{7.626382in}{3.756190in}%
\pgfsys@useobject{currentmarker}{}%
\end{pgfscope}%
\end{pgfscope}%
\begin{pgfscope}%
\pgfpathrectangle{\pgfqpoint{5.814694in}{2.849537in}}{\pgfqpoint{1.897959in}{1.372727in}} %
\pgfusepath{clip}%
\pgfsetbuttcap%
\pgfsetroundjoin%
\pgfsetlinewidth{1.505625pt}%
\definecolor{currentstroke}{rgb}{0.000000,0.000000,0.000000}%
\pgfsetstrokecolor{currentstroke}%
\pgfsetdash{{1.500000pt}{2.475000pt}}{0.000000pt}%
\pgfpathmoveto{\pgfqpoint{5.900965in}{4.159867in}}%
\pgfpathlineto{\pgfqpoint{5.932336in}{4.147790in}}%
\pgfpathlineto{\pgfqpoint{5.963707in}{4.135922in}}%
\pgfpathlineto{\pgfqpoint{5.995078in}{4.125385in}}%
\pgfpathlineto{\pgfqpoint{6.026450in}{4.115935in}}%
\pgfpathlineto{\pgfqpoint{6.057821in}{4.107388in}}%
\pgfpathlineto{\pgfqpoint{6.089192in}{4.099599in}}%
\pgfpathlineto{\pgfqpoint{6.120563in}{4.092456in}}%
\pgfpathlineto{\pgfqpoint{6.151935in}{4.085866in}}%
\pgfpathlineto{\pgfqpoint{6.183306in}{4.079756in}}%
\pgfpathlineto{\pgfqpoint{6.214677in}{4.074068in}}%
\pgfpathlineto{\pgfqpoint{6.246048in}{4.068750in}}%
\pgfpathlineto{\pgfqpoint{6.277419in}{4.063763in}}%
\pgfpathlineto{\pgfqpoint{6.308791in}{4.059071in}}%
\pgfpathlineto{\pgfqpoint{6.340162in}{4.054645in}}%
\pgfpathlineto{\pgfqpoint{6.371533in}{4.050461in}}%
\pgfpathlineto{\pgfqpoint{6.402904in}{4.046497in}}%
\pgfpathlineto{\pgfqpoint{6.434276in}{4.042735in}}%
\pgfpathlineto{\pgfqpoint{6.465647in}{4.039159in}}%
\pgfpathlineto{\pgfqpoint{6.497018in}{4.035755in}}%
\pgfpathlineto{\pgfqpoint{6.528389in}{4.032511in}}%
\pgfpathlineto{\pgfqpoint{6.559760in}{4.029417in}}%
\pgfpathlineto{\pgfqpoint{6.591132in}{4.026462in}}%
\pgfpathlineto{\pgfqpoint{6.622503in}{4.023639in}}%
\pgfpathlineto{\pgfqpoint{6.653874in}{4.020940in}}%
\pgfpathlineto{\pgfqpoint{6.685245in}{4.018359in}}%
\pgfpathlineto{\pgfqpoint{6.716617in}{4.015890in}}%
\pgfpathlineto{\pgfqpoint{6.747988in}{4.013526in}}%
\pgfpathlineto{\pgfqpoint{6.779359in}{4.011264in}}%
\pgfpathlineto{\pgfqpoint{6.810730in}{4.009099in}}%
\pgfpathlineto{\pgfqpoint{6.842102in}{4.007027in}}%
\pgfpathlineto{\pgfqpoint{6.873473in}{4.005044in}}%
\pgfpathlineto{\pgfqpoint{6.904844in}{4.003147in}}%
\pgfpathlineto{\pgfqpoint{6.936215in}{4.001333in}}%
\pgfpathlineto{\pgfqpoint{6.967586in}{3.999600in}}%
\pgfpathlineto{\pgfqpoint{6.998958in}{3.997944in}}%
\pgfpathlineto{\pgfqpoint{7.030329in}{3.996364in}}%
\pgfpathlineto{\pgfqpoint{7.061700in}{3.994857in}}%
\pgfpathlineto{\pgfqpoint{7.093071in}{3.993422in}}%
\pgfpathlineto{\pgfqpoint{7.124443in}{3.992056in}}%
\pgfpathlineto{\pgfqpoint{7.155814in}{3.990758in}}%
\pgfpathlineto{\pgfqpoint{7.187185in}{3.989527in}}%
\pgfpathlineto{\pgfqpoint{7.218556in}{3.988360in}}%
\pgfpathlineto{\pgfqpoint{7.249927in}{3.987258in}}%
\pgfpathlineto{\pgfqpoint{7.281299in}{3.986218in}}%
\pgfpathlineto{\pgfqpoint{7.312670in}{3.985239in}}%
\pgfpathlineto{\pgfqpoint{7.344041in}{3.984321in}}%
\pgfpathlineto{\pgfqpoint{7.375412in}{3.983462in}}%
\pgfpathlineto{\pgfqpoint{7.406784in}{3.982662in}}%
\pgfpathlineto{\pgfqpoint{7.438155in}{3.981920in}}%
\pgfpathlineto{\pgfqpoint{7.469526in}{3.981235in}}%
\pgfpathlineto{\pgfqpoint{7.500897in}{3.980607in}}%
\pgfpathlineto{\pgfqpoint{7.532269in}{3.980035in}}%
\pgfpathlineto{\pgfqpoint{7.563640in}{3.979518in}}%
\pgfpathlineto{\pgfqpoint{7.595011in}{3.979057in}}%
\pgfpathlineto{\pgfqpoint{7.626382in}{3.978651in}}%
\pgfusepath{stroke}%
\end{pgfscope}%
\begin{pgfscope}%
\pgfpathrectangle{\pgfqpoint{5.814694in}{2.849537in}}{\pgfqpoint{1.897959in}{1.372727in}} %
\pgfusepath{clip}%
\pgfsetbuttcap%
\pgfsetroundjoin%
\definecolor{currentfill}{rgb}{0.000000,0.000000,0.000000}%
\pgfsetfillcolor{currentfill}%
\pgfsetlinewidth{1.003750pt}%
\definecolor{currentstroke}{rgb}{0.000000,0.000000,0.000000}%
\pgfsetstrokecolor{currentstroke}%
\pgfsetdash{}{0pt}%
\pgfsys@defobject{currentmarker}{\pgfqpoint{-0.041667in}{-0.041667in}}{\pgfqpoint{0.041667in}{0.041667in}}{%
\pgfpathmoveto{\pgfqpoint{-0.041667in}{0.000000in}}%
\pgfpathlineto{\pgfqpoint{0.041667in}{0.000000in}}%
\pgfpathmoveto{\pgfqpoint{0.000000in}{-0.041667in}}%
\pgfpathlineto{\pgfqpoint{0.000000in}{0.041667in}}%
\pgfusepath{stroke,fill}%
}%
\begin{pgfscope}%
\pgfsys@transformshift{5.900965in}{4.159867in}%
\pgfsys@useobject{currentmarker}{}%
\end{pgfscope}%
\begin{pgfscope}%
\pgfsys@transformshift{6.246048in}{4.068750in}%
\pgfsys@useobject{currentmarker}{}%
\end{pgfscope}%
\begin{pgfscope}%
\pgfsys@transformshift{6.591132in}{4.026462in}%
\pgfsys@useobject{currentmarker}{}%
\end{pgfscope}%
\begin{pgfscope}%
\pgfsys@transformshift{6.936215in}{4.001333in}%
\pgfsys@useobject{currentmarker}{}%
\end{pgfscope}%
\begin{pgfscope}%
\pgfsys@transformshift{7.281299in}{3.986218in}%
\pgfsys@useobject{currentmarker}{}%
\end{pgfscope}%
\begin{pgfscope}%
\pgfsys@transformshift{7.626382in}{3.978651in}%
\pgfsys@useobject{currentmarker}{}%
\end{pgfscope}%
\end{pgfscope}%
\begin{pgfscope}%
\pgfsetrectcap%
\pgfsetmiterjoin%
\pgfsetlinewidth{0.803000pt}%
\definecolor{currentstroke}{rgb}{0.000000,0.000000,0.000000}%
\pgfsetstrokecolor{currentstroke}%
\pgfsetdash{}{0pt}%
\pgfpathmoveto{\pgfqpoint{5.814694in}{2.849537in}}%
\pgfpathlineto{\pgfqpoint{5.814694in}{4.222264in}}%
\pgfusepath{stroke}%
\end{pgfscope}%
\begin{pgfscope}%
\pgfsetrectcap%
\pgfsetmiterjoin%
\pgfsetlinewidth{0.803000pt}%
\definecolor{currentstroke}{rgb}{0.000000,0.000000,0.000000}%
\pgfsetstrokecolor{currentstroke}%
\pgfsetdash{}{0pt}%
\pgfpathmoveto{\pgfqpoint{7.712653in}{2.849537in}}%
\pgfpathlineto{\pgfqpoint{7.712653in}{4.222264in}}%
\pgfusepath{stroke}%
\end{pgfscope}%
\begin{pgfscope}%
\pgfsetrectcap%
\pgfsetmiterjoin%
\pgfsetlinewidth{0.803000pt}%
\definecolor{currentstroke}{rgb}{0.000000,0.000000,0.000000}%
\pgfsetstrokecolor{currentstroke}%
\pgfsetdash{}{0pt}%
\pgfpathmoveto{\pgfqpoint{5.814694in}{2.849537in}}%
\pgfpathlineto{\pgfqpoint{7.712653in}{2.849537in}}%
\pgfusepath{stroke}%
\end{pgfscope}%
\begin{pgfscope}%
\pgfsetrectcap%
\pgfsetmiterjoin%
\pgfsetlinewidth{0.803000pt}%
\definecolor{currentstroke}{rgb}{0.000000,0.000000,0.000000}%
\pgfsetstrokecolor{currentstroke}%
\pgfsetdash{}{0pt}%
\pgfpathmoveto{\pgfqpoint{5.814694in}{4.222264in}}%
\pgfpathlineto{\pgfqpoint{7.712653in}{4.222264in}}%
\pgfusepath{stroke}%
\end{pgfscope}%
\begin{pgfscope}%
\pgfsetbuttcap%
\pgfsetmiterjoin%
\definecolor{currentfill}{rgb}{1.000000,1.000000,1.000000}%
\pgfsetfillcolor{currentfill}%
\pgfsetlinewidth{0.000000pt}%
\definecolor{currentstroke}{rgb}{0.000000,0.000000,0.000000}%
\pgfsetstrokecolor{currentstroke}%
\pgfsetstrokeopacity{0.000000}%
\pgfsetdash{}{0pt}%
\pgfpathmoveto{\pgfqpoint{8.282041in}{2.849537in}}%
\pgfpathlineto{\pgfqpoint{10.180000in}{2.849537in}}%
\pgfpathlineto{\pgfqpoint{10.180000in}{4.222264in}}%
\pgfpathlineto{\pgfqpoint{8.282041in}{4.222264in}}%
\pgfpathclose%
\pgfusepath{fill}%
\end{pgfscope}%
\begin{pgfscope}%
\pgfsetbuttcap%
\pgfsetroundjoin%
\definecolor{currentfill}{rgb}{0.000000,0.000000,0.000000}%
\pgfsetfillcolor{currentfill}%
\pgfsetlinewidth{0.803000pt}%
\definecolor{currentstroke}{rgb}{0.000000,0.000000,0.000000}%
\pgfsetstrokecolor{currentstroke}%
\pgfsetdash{}{0pt}%
\pgfsys@defobject{currentmarker}{\pgfqpoint{0.000000in}{-0.048611in}}{\pgfqpoint{0.000000in}{0.000000in}}{%
\pgfpathmoveto{\pgfqpoint{0.000000in}{0.000000in}}%
\pgfpathlineto{\pgfqpoint{0.000000in}{-0.048611in}}%
\pgfusepath{stroke,fill}%
}%
\begin{pgfscope}%
\pgfsys@transformshift{9.128665in}{2.849537in}%
\pgfsys@useobject{currentmarker}{}%
\end{pgfscope}%
\end{pgfscope}%
\begin{pgfscope}%
\pgftext[x=9.128665in,y=2.752315in,,top]{\rmfamily\fontsize{10.000000}{12.000000}\selectfont \(\displaystyle 0.5\)}%
\end{pgfscope}%
\begin{pgfscope}%
\pgfsetbuttcap%
\pgfsetroundjoin%
\definecolor{currentfill}{rgb}{0.000000,0.000000,0.000000}%
\pgfsetfillcolor{currentfill}%
\pgfsetlinewidth{0.803000pt}%
\definecolor{currentstroke}{rgb}{0.000000,0.000000,0.000000}%
\pgfsetstrokecolor{currentstroke}%
\pgfsetdash{}{0pt}%
\pgfsys@defobject{currentmarker}{\pgfqpoint{0.000000in}{-0.048611in}}{\pgfqpoint{0.000000in}{0.000000in}}{%
\pgfpathmoveto{\pgfqpoint{0.000000in}{0.000000in}}%
\pgfpathlineto{\pgfqpoint{0.000000in}{-0.048611in}}%
\pgfusepath{stroke,fill}%
}%
\begin{pgfscope}%
\pgfsys@transformshift{10.005996in}{2.849537in}%
\pgfsys@useobject{currentmarker}{}%
\end{pgfscope}%
\end{pgfscope}%
\begin{pgfscope}%
\pgftext[x=10.005996in,y=2.752315in,,top]{\rmfamily\fontsize{10.000000}{12.000000}\selectfont \(\displaystyle 1.0\)}%
\end{pgfscope}%
\begin{pgfscope}%
\pgfsetbuttcap%
\pgfsetroundjoin%
\definecolor{currentfill}{rgb}{0.000000,0.000000,0.000000}%
\pgfsetfillcolor{currentfill}%
\pgfsetlinewidth{0.803000pt}%
\definecolor{currentstroke}{rgb}{0.000000,0.000000,0.000000}%
\pgfsetstrokecolor{currentstroke}%
\pgfsetdash{}{0pt}%
\pgfsys@defobject{currentmarker}{\pgfqpoint{-0.048611in}{0.000000in}}{\pgfqpoint{0.000000in}{0.000000in}}{%
\pgfpathmoveto{\pgfqpoint{0.000000in}{0.000000in}}%
\pgfpathlineto{\pgfqpoint{-0.048611in}{0.000000in}}%
\pgfusepath{stroke,fill}%
}%
\begin{pgfscope}%
\pgfsys@transformshift{8.282041in}{3.062033in}%
\pgfsys@useobject{currentmarker}{}%
\end{pgfscope}%
\end{pgfscope}%
\begin{pgfscope}%
\pgftext[x=7.896816in,y=3.009271in,left,base]{\rmfamily\fontsize{10.000000}{12.000000}\selectfont \(\displaystyle 10^{-8}\)}%
\end{pgfscope}%
\begin{pgfscope}%
\pgfsetbuttcap%
\pgfsetroundjoin%
\definecolor{currentfill}{rgb}{0.000000,0.000000,0.000000}%
\pgfsetfillcolor{currentfill}%
\pgfsetlinewidth{0.803000pt}%
\definecolor{currentstroke}{rgb}{0.000000,0.000000,0.000000}%
\pgfsetstrokecolor{currentstroke}%
\pgfsetdash{}{0pt}%
\pgfsys@defobject{currentmarker}{\pgfqpoint{-0.048611in}{0.000000in}}{\pgfqpoint{0.000000in}{0.000000in}}{%
\pgfpathmoveto{\pgfqpoint{0.000000in}{0.000000in}}%
\pgfpathlineto{\pgfqpoint{-0.048611in}{0.000000in}}%
\pgfusepath{stroke,fill}%
}%
\begin{pgfscope}%
\pgfsys@transformshift{8.282041in}{3.359739in}%
\pgfsys@useobject{currentmarker}{}%
\end{pgfscope}%
\end{pgfscope}%
\begin{pgfscope}%
\pgftext[x=7.896816in,y=3.306977in,left,base]{\rmfamily\fontsize{10.000000}{12.000000}\selectfont \(\displaystyle 10^{-7}\)}%
\end{pgfscope}%
\begin{pgfscope}%
\pgfsetbuttcap%
\pgfsetroundjoin%
\definecolor{currentfill}{rgb}{0.000000,0.000000,0.000000}%
\pgfsetfillcolor{currentfill}%
\pgfsetlinewidth{0.803000pt}%
\definecolor{currentstroke}{rgb}{0.000000,0.000000,0.000000}%
\pgfsetstrokecolor{currentstroke}%
\pgfsetdash{}{0pt}%
\pgfsys@defobject{currentmarker}{\pgfqpoint{-0.048611in}{0.000000in}}{\pgfqpoint{0.000000in}{0.000000in}}{%
\pgfpathmoveto{\pgfqpoint{0.000000in}{0.000000in}}%
\pgfpathlineto{\pgfqpoint{-0.048611in}{0.000000in}}%
\pgfusepath{stroke,fill}%
}%
\begin{pgfscope}%
\pgfsys@transformshift{8.282041in}{3.657445in}%
\pgfsys@useobject{currentmarker}{}%
\end{pgfscope}%
\end{pgfscope}%
\begin{pgfscope}%
\pgftext[x=7.896816in,y=3.604683in,left,base]{\rmfamily\fontsize{10.000000}{12.000000}\selectfont \(\displaystyle 10^{-6}\)}%
\end{pgfscope}%
\begin{pgfscope}%
\pgfsetbuttcap%
\pgfsetroundjoin%
\definecolor{currentfill}{rgb}{0.000000,0.000000,0.000000}%
\pgfsetfillcolor{currentfill}%
\pgfsetlinewidth{0.803000pt}%
\definecolor{currentstroke}{rgb}{0.000000,0.000000,0.000000}%
\pgfsetstrokecolor{currentstroke}%
\pgfsetdash{}{0pt}%
\pgfsys@defobject{currentmarker}{\pgfqpoint{-0.048611in}{0.000000in}}{\pgfqpoint{0.000000in}{0.000000in}}{%
\pgfpathmoveto{\pgfqpoint{0.000000in}{0.000000in}}%
\pgfpathlineto{\pgfqpoint{-0.048611in}{0.000000in}}%
\pgfusepath{stroke,fill}%
}%
\begin{pgfscope}%
\pgfsys@transformshift{8.282041in}{3.955151in}%
\pgfsys@useobject{currentmarker}{}%
\end{pgfscope}%
\end{pgfscope}%
\begin{pgfscope}%
\pgftext[x=7.896816in,y=3.902390in,left,base]{\rmfamily\fontsize{10.000000}{12.000000}\selectfont \(\displaystyle 10^{-5}\)}%
\end{pgfscope}%
\begin{pgfscope}%
\pgfsetbuttcap%
\pgfsetroundjoin%
\definecolor{currentfill}{rgb}{0.000000,0.000000,0.000000}%
\pgfsetfillcolor{currentfill}%
\pgfsetlinewidth{0.602250pt}%
\definecolor{currentstroke}{rgb}{0.000000,0.000000,0.000000}%
\pgfsetstrokecolor{currentstroke}%
\pgfsetdash{}{0pt}%
\pgfsys@defobject{currentmarker}{\pgfqpoint{-0.027778in}{0.000000in}}{\pgfqpoint{0.000000in}{0.000000in}}{%
\pgfpathmoveto{\pgfqpoint{0.000000in}{0.000000in}}%
\pgfpathlineto{\pgfqpoint{-0.027778in}{0.000000in}}%
\pgfusepath{stroke,fill}%
}%
\begin{pgfscope}%
\pgfsys@transformshift{8.282041in}{2.853945in}%
\pgfsys@useobject{currentmarker}{}%
\end{pgfscope}%
\end{pgfscope}%
\begin{pgfscope}%
\pgfsetbuttcap%
\pgfsetroundjoin%
\definecolor{currentfill}{rgb}{0.000000,0.000000,0.000000}%
\pgfsetfillcolor{currentfill}%
\pgfsetlinewidth{0.602250pt}%
\definecolor{currentstroke}{rgb}{0.000000,0.000000,0.000000}%
\pgfsetstrokecolor{currentstroke}%
\pgfsetdash{}{0pt}%
\pgfsys@defobject{currentmarker}{\pgfqpoint{-0.027778in}{0.000000in}}{\pgfqpoint{0.000000in}{0.000000in}}{%
\pgfpathmoveto{\pgfqpoint{0.000000in}{0.000000in}}%
\pgfpathlineto{\pgfqpoint{-0.027778in}{0.000000in}}%
\pgfusepath{stroke,fill}%
}%
\begin{pgfscope}%
\pgfsys@transformshift{8.282041in}{2.906368in}%
\pgfsys@useobject{currentmarker}{}%
\end{pgfscope}%
\end{pgfscope}%
\begin{pgfscope}%
\pgfsetbuttcap%
\pgfsetroundjoin%
\definecolor{currentfill}{rgb}{0.000000,0.000000,0.000000}%
\pgfsetfillcolor{currentfill}%
\pgfsetlinewidth{0.602250pt}%
\definecolor{currentstroke}{rgb}{0.000000,0.000000,0.000000}%
\pgfsetstrokecolor{currentstroke}%
\pgfsetdash{}{0pt}%
\pgfsys@defobject{currentmarker}{\pgfqpoint{-0.027778in}{0.000000in}}{\pgfqpoint{0.000000in}{0.000000in}}{%
\pgfpathmoveto{\pgfqpoint{0.000000in}{0.000000in}}%
\pgfpathlineto{\pgfqpoint{-0.027778in}{0.000000in}}%
\pgfusepath{stroke,fill}%
}%
\begin{pgfscope}%
\pgfsys@transformshift{8.282041in}{2.943563in}%
\pgfsys@useobject{currentmarker}{}%
\end{pgfscope}%
\end{pgfscope}%
\begin{pgfscope}%
\pgfsetbuttcap%
\pgfsetroundjoin%
\definecolor{currentfill}{rgb}{0.000000,0.000000,0.000000}%
\pgfsetfillcolor{currentfill}%
\pgfsetlinewidth{0.602250pt}%
\definecolor{currentstroke}{rgb}{0.000000,0.000000,0.000000}%
\pgfsetstrokecolor{currentstroke}%
\pgfsetdash{}{0pt}%
\pgfsys@defobject{currentmarker}{\pgfqpoint{-0.027778in}{0.000000in}}{\pgfqpoint{0.000000in}{0.000000in}}{%
\pgfpathmoveto{\pgfqpoint{0.000000in}{0.000000in}}%
\pgfpathlineto{\pgfqpoint{-0.027778in}{0.000000in}}%
\pgfusepath{stroke,fill}%
}%
\begin{pgfscope}%
\pgfsys@transformshift{8.282041in}{2.972414in}%
\pgfsys@useobject{currentmarker}{}%
\end{pgfscope}%
\end{pgfscope}%
\begin{pgfscope}%
\pgfsetbuttcap%
\pgfsetroundjoin%
\definecolor{currentfill}{rgb}{0.000000,0.000000,0.000000}%
\pgfsetfillcolor{currentfill}%
\pgfsetlinewidth{0.602250pt}%
\definecolor{currentstroke}{rgb}{0.000000,0.000000,0.000000}%
\pgfsetstrokecolor{currentstroke}%
\pgfsetdash{}{0pt}%
\pgfsys@defobject{currentmarker}{\pgfqpoint{-0.027778in}{0.000000in}}{\pgfqpoint{0.000000in}{0.000000in}}{%
\pgfpathmoveto{\pgfqpoint{0.000000in}{0.000000in}}%
\pgfpathlineto{\pgfqpoint{-0.027778in}{0.000000in}}%
\pgfusepath{stroke,fill}%
}%
\begin{pgfscope}%
\pgfsys@transformshift{8.282041in}{2.995987in}%
\pgfsys@useobject{currentmarker}{}%
\end{pgfscope}%
\end{pgfscope}%
\begin{pgfscope}%
\pgfsetbuttcap%
\pgfsetroundjoin%
\definecolor{currentfill}{rgb}{0.000000,0.000000,0.000000}%
\pgfsetfillcolor{currentfill}%
\pgfsetlinewidth{0.602250pt}%
\definecolor{currentstroke}{rgb}{0.000000,0.000000,0.000000}%
\pgfsetstrokecolor{currentstroke}%
\pgfsetdash{}{0pt}%
\pgfsys@defobject{currentmarker}{\pgfqpoint{-0.027778in}{0.000000in}}{\pgfqpoint{0.000000in}{0.000000in}}{%
\pgfpathmoveto{\pgfqpoint{0.000000in}{0.000000in}}%
\pgfpathlineto{\pgfqpoint{-0.027778in}{0.000000in}}%
\pgfusepath{stroke,fill}%
}%
\begin{pgfscope}%
\pgfsys@transformshift{8.282041in}{3.015917in}%
\pgfsys@useobject{currentmarker}{}%
\end{pgfscope}%
\end{pgfscope}%
\begin{pgfscope}%
\pgfsetbuttcap%
\pgfsetroundjoin%
\definecolor{currentfill}{rgb}{0.000000,0.000000,0.000000}%
\pgfsetfillcolor{currentfill}%
\pgfsetlinewidth{0.602250pt}%
\definecolor{currentstroke}{rgb}{0.000000,0.000000,0.000000}%
\pgfsetstrokecolor{currentstroke}%
\pgfsetdash{}{0pt}%
\pgfsys@defobject{currentmarker}{\pgfqpoint{-0.027778in}{0.000000in}}{\pgfqpoint{0.000000in}{0.000000in}}{%
\pgfpathmoveto{\pgfqpoint{0.000000in}{0.000000in}}%
\pgfpathlineto{\pgfqpoint{-0.027778in}{0.000000in}}%
\pgfusepath{stroke,fill}%
}%
\begin{pgfscope}%
\pgfsys@transformshift{8.282041in}{3.033182in}%
\pgfsys@useobject{currentmarker}{}%
\end{pgfscope}%
\end{pgfscope}%
\begin{pgfscope}%
\pgfsetbuttcap%
\pgfsetroundjoin%
\definecolor{currentfill}{rgb}{0.000000,0.000000,0.000000}%
\pgfsetfillcolor{currentfill}%
\pgfsetlinewidth{0.602250pt}%
\definecolor{currentstroke}{rgb}{0.000000,0.000000,0.000000}%
\pgfsetstrokecolor{currentstroke}%
\pgfsetdash{}{0pt}%
\pgfsys@defobject{currentmarker}{\pgfqpoint{-0.027778in}{0.000000in}}{\pgfqpoint{0.000000in}{0.000000in}}{%
\pgfpathmoveto{\pgfqpoint{0.000000in}{0.000000in}}%
\pgfpathlineto{\pgfqpoint{-0.027778in}{0.000000in}}%
\pgfusepath{stroke,fill}%
}%
\begin{pgfscope}%
\pgfsys@transformshift{8.282041in}{3.048410in}%
\pgfsys@useobject{currentmarker}{}%
\end{pgfscope}%
\end{pgfscope}%
\begin{pgfscope}%
\pgfsetbuttcap%
\pgfsetroundjoin%
\definecolor{currentfill}{rgb}{0.000000,0.000000,0.000000}%
\pgfsetfillcolor{currentfill}%
\pgfsetlinewidth{0.602250pt}%
\definecolor{currentstroke}{rgb}{0.000000,0.000000,0.000000}%
\pgfsetstrokecolor{currentstroke}%
\pgfsetdash{}{0pt}%
\pgfsys@defobject{currentmarker}{\pgfqpoint{-0.027778in}{0.000000in}}{\pgfqpoint{0.000000in}{0.000000in}}{%
\pgfpathmoveto{\pgfqpoint{0.000000in}{0.000000in}}%
\pgfpathlineto{\pgfqpoint{-0.027778in}{0.000000in}}%
\pgfusepath{stroke,fill}%
}%
\begin{pgfscope}%
\pgfsys@transformshift{8.282041in}{3.151651in}%
\pgfsys@useobject{currentmarker}{}%
\end{pgfscope}%
\end{pgfscope}%
\begin{pgfscope}%
\pgfsetbuttcap%
\pgfsetroundjoin%
\definecolor{currentfill}{rgb}{0.000000,0.000000,0.000000}%
\pgfsetfillcolor{currentfill}%
\pgfsetlinewidth{0.602250pt}%
\definecolor{currentstroke}{rgb}{0.000000,0.000000,0.000000}%
\pgfsetstrokecolor{currentstroke}%
\pgfsetdash{}{0pt}%
\pgfsys@defobject{currentmarker}{\pgfqpoint{-0.027778in}{0.000000in}}{\pgfqpoint{0.000000in}{0.000000in}}{%
\pgfpathmoveto{\pgfqpoint{0.000000in}{0.000000in}}%
\pgfpathlineto{\pgfqpoint{-0.027778in}{0.000000in}}%
\pgfusepath{stroke,fill}%
}%
\begin{pgfscope}%
\pgfsys@transformshift{8.282041in}{3.204074in}%
\pgfsys@useobject{currentmarker}{}%
\end{pgfscope}%
\end{pgfscope}%
\begin{pgfscope}%
\pgfsetbuttcap%
\pgfsetroundjoin%
\definecolor{currentfill}{rgb}{0.000000,0.000000,0.000000}%
\pgfsetfillcolor{currentfill}%
\pgfsetlinewidth{0.602250pt}%
\definecolor{currentstroke}{rgb}{0.000000,0.000000,0.000000}%
\pgfsetstrokecolor{currentstroke}%
\pgfsetdash{}{0pt}%
\pgfsys@defobject{currentmarker}{\pgfqpoint{-0.027778in}{0.000000in}}{\pgfqpoint{0.000000in}{0.000000in}}{%
\pgfpathmoveto{\pgfqpoint{0.000000in}{0.000000in}}%
\pgfpathlineto{\pgfqpoint{-0.027778in}{0.000000in}}%
\pgfusepath{stroke,fill}%
}%
\begin{pgfscope}%
\pgfsys@transformshift{8.282041in}{3.241269in}%
\pgfsys@useobject{currentmarker}{}%
\end{pgfscope}%
\end{pgfscope}%
\begin{pgfscope}%
\pgfsetbuttcap%
\pgfsetroundjoin%
\definecolor{currentfill}{rgb}{0.000000,0.000000,0.000000}%
\pgfsetfillcolor{currentfill}%
\pgfsetlinewidth{0.602250pt}%
\definecolor{currentstroke}{rgb}{0.000000,0.000000,0.000000}%
\pgfsetstrokecolor{currentstroke}%
\pgfsetdash{}{0pt}%
\pgfsys@defobject{currentmarker}{\pgfqpoint{-0.027778in}{0.000000in}}{\pgfqpoint{0.000000in}{0.000000in}}{%
\pgfpathmoveto{\pgfqpoint{0.000000in}{0.000000in}}%
\pgfpathlineto{\pgfqpoint{-0.027778in}{0.000000in}}%
\pgfusepath{stroke,fill}%
}%
\begin{pgfscope}%
\pgfsys@transformshift{8.282041in}{3.270120in}%
\pgfsys@useobject{currentmarker}{}%
\end{pgfscope}%
\end{pgfscope}%
\begin{pgfscope}%
\pgfsetbuttcap%
\pgfsetroundjoin%
\definecolor{currentfill}{rgb}{0.000000,0.000000,0.000000}%
\pgfsetfillcolor{currentfill}%
\pgfsetlinewidth{0.602250pt}%
\definecolor{currentstroke}{rgb}{0.000000,0.000000,0.000000}%
\pgfsetstrokecolor{currentstroke}%
\pgfsetdash{}{0pt}%
\pgfsys@defobject{currentmarker}{\pgfqpoint{-0.027778in}{0.000000in}}{\pgfqpoint{0.000000in}{0.000000in}}{%
\pgfpathmoveto{\pgfqpoint{0.000000in}{0.000000in}}%
\pgfpathlineto{\pgfqpoint{-0.027778in}{0.000000in}}%
\pgfusepath{stroke,fill}%
}%
\begin{pgfscope}%
\pgfsys@transformshift{8.282041in}{3.293693in}%
\pgfsys@useobject{currentmarker}{}%
\end{pgfscope}%
\end{pgfscope}%
\begin{pgfscope}%
\pgfsetbuttcap%
\pgfsetroundjoin%
\definecolor{currentfill}{rgb}{0.000000,0.000000,0.000000}%
\pgfsetfillcolor{currentfill}%
\pgfsetlinewidth{0.602250pt}%
\definecolor{currentstroke}{rgb}{0.000000,0.000000,0.000000}%
\pgfsetstrokecolor{currentstroke}%
\pgfsetdash{}{0pt}%
\pgfsys@defobject{currentmarker}{\pgfqpoint{-0.027778in}{0.000000in}}{\pgfqpoint{0.000000in}{0.000000in}}{%
\pgfpathmoveto{\pgfqpoint{0.000000in}{0.000000in}}%
\pgfpathlineto{\pgfqpoint{-0.027778in}{0.000000in}}%
\pgfusepath{stroke,fill}%
}%
\begin{pgfscope}%
\pgfsys@transformshift{8.282041in}{3.313623in}%
\pgfsys@useobject{currentmarker}{}%
\end{pgfscope}%
\end{pgfscope}%
\begin{pgfscope}%
\pgfsetbuttcap%
\pgfsetroundjoin%
\definecolor{currentfill}{rgb}{0.000000,0.000000,0.000000}%
\pgfsetfillcolor{currentfill}%
\pgfsetlinewidth{0.602250pt}%
\definecolor{currentstroke}{rgb}{0.000000,0.000000,0.000000}%
\pgfsetstrokecolor{currentstroke}%
\pgfsetdash{}{0pt}%
\pgfsys@defobject{currentmarker}{\pgfqpoint{-0.027778in}{0.000000in}}{\pgfqpoint{0.000000in}{0.000000in}}{%
\pgfpathmoveto{\pgfqpoint{0.000000in}{0.000000in}}%
\pgfpathlineto{\pgfqpoint{-0.027778in}{0.000000in}}%
\pgfusepath{stroke,fill}%
}%
\begin{pgfscope}%
\pgfsys@transformshift{8.282041in}{3.330888in}%
\pgfsys@useobject{currentmarker}{}%
\end{pgfscope}%
\end{pgfscope}%
\begin{pgfscope}%
\pgfsetbuttcap%
\pgfsetroundjoin%
\definecolor{currentfill}{rgb}{0.000000,0.000000,0.000000}%
\pgfsetfillcolor{currentfill}%
\pgfsetlinewidth{0.602250pt}%
\definecolor{currentstroke}{rgb}{0.000000,0.000000,0.000000}%
\pgfsetstrokecolor{currentstroke}%
\pgfsetdash{}{0pt}%
\pgfsys@defobject{currentmarker}{\pgfqpoint{-0.027778in}{0.000000in}}{\pgfqpoint{0.000000in}{0.000000in}}{%
\pgfpathmoveto{\pgfqpoint{0.000000in}{0.000000in}}%
\pgfpathlineto{\pgfqpoint{-0.027778in}{0.000000in}}%
\pgfusepath{stroke,fill}%
}%
\begin{pgfscope}%
\pgfsys@transformshift{8.282041in}{3.346116in}%
\pgfsys@useobject{currentmarker}{}%
\end{pgfscope}%
\end{pgfscope}%
\begin{pgfscope}%
\pgfsetbuttcap%
\pgfsetroundjoin%
\definecolor{currentfill}{rgb}{0.000000,0.000000,0.000000}%
\pgfsetfillcolor{currentfill}%
\pgfsetlinewidth{0.602250pt}%
\definecolor{currentstroke}{rgb}{0.000000,0.000000,0.000000}%
\pgfsetstrokecolor{currentstroke}%
\pgfsetdash{}{0pt}%
\pgfsys@defobject{currentmarker}{\pgfqpoint{-0.027778in}{0.000000in}}{\pgfqpoint{0.000000in}{0.000000in}}{%
\pgfpathmoveto{\pgfqpoint{0.000000in}{0.000000in}}%
\pgfpathlineto{\pgfqpoint{-0.027778in}{0.000000in}}%
\pgfusepath{stroke,fill}%
}%
\begin{pgfscope}%
\pgfsys@transformshift{8.282041in}{3.449357in}%
\pgfsys@useobject{currentmarker}{}%
\end{pgfscope}%
\end{pgfscope}%
\begin{pgfscope}%
\pgfsetbuttcap%
\pgfsetroundjoin%
\definecolor{currentfill}{rgb}{0.000000,0.000000,0.000000}%
\pgfsetfillcolor{currentfill}%
\pgfsetlinewidth{0.602250pt}%
\definecolor{currentstroke}{rgb}{0.000000,0.000000,0.000000}%
\pgfsetstrokecolor{currentstroke}%
\pgfsetdash{}{0pt}%
\pgfsys@defobject{currentmarker}{\pgfqpoint{-0.027778in}{0.000000in}}{\pgfqpoint{0.000000in}{0.000000in}}{%
\pgfpathmoveto{\pgfqpoint{0.000000in}{0.000000in}}%
\pgfpathlineto{\pgfqpoint{-0.027778in}{0.000000in}}%
\pgfusepath{stroke,fill}%
}%
\begin{pgfscope}%
\pgfsys@transformshift{8.282041in}{3.501781in}%
\pgfsys@useobject{currentmarker}{}%
\end{pgfscope}%
\end{pgfscope}%
\begin{pgfscope}%
\pgfsetbuttcap%
\pgfsetroundjoin%
\definecolor{currentfill}{rgb}{0.000000,0.000000,0.000000}%
\pgfsetfillcolor{currentfill}%
\pgfsetlinewidth{0.602250pt}%
\definecolor{currentstroke}{rgb}{0.000000,0.000000,0.000000}%
\pgfsetstrokecolor{currentstroke}%
\pgfsetdash{}{0pt}%
\pgfsys@defobject{currentmarker}{\pgfqpoint{-0.027778in}{0.000000in}}{\pgfqpoint{0.000000in}{0.000000in}}{%
\pgfpathmoveto{\pgfqpoint{0.000000in}{0.000000in}}%
\pgfpathlineto{\pgfqpoint{-0.027778in}{0.000000in}}%
\pgfusepath{stroke,fill}%
}%
\begin{pgfscope}%
\pgfsys@transformshift{8.282041in}{3.538976in}%
\pgfsys@useobject{currentmarker}{}%
\end{pgfscope}%
\end{pgfscope}%
\begin{pgfscope}%
\pgfsetbuttcap%
\pgfsetroundjoin%
\definecolor{currentfill}{rgb}{0.000000,0.000000,0.000000}%
\pgfsetfillcolor{currentfill}%
\pgfsetlinewidth{0.602250pt}%
\definecolor{currentstroke}{rgb}{0.000000,0.000000,0.000000}%
\pgfsetstrokecolor{currentstroke}%
\pgfsetdash{}{0pt}%
\pgfsys@defobject{currentmarker}{\pgfqpoint{-0.027778in}{0.000000in}}{\pgfqpoint{0.000000in}{0.000000in}}{%
\pgfpathmoveto{\pgfqpoint{0.000000in}{0.000000in}}%
\pgfpathlineto{\pgfqpoint{-0.027778in}{0.000000in}}%
\pgfusepath{stroke,fill}%
}%
\begin{pgfscope}%
\pgfsys@transformshift{8.282041in}{3.567826in}%
\pgfsys@useobject{currentmarker}{}%
\end{pgfscope}%
\end{pgfscope}%
\begin{pgfscope}%
\pgfsetbuttcap%
\pgfsetroundjoin%
\definecolor{currentfill}{rgb}{0.000000,0.000000,0.000000}%
\pgfsetfillcolor{currentfill}%
\pgfsetlinewidth{0.602250pt}%
\definecolor{currentstroke}{rgb}{0.000000,0.000000,0.000000}%
\pgfsetstrokecolor{currentstroke}%
\pgfsetdash{}{0pt}%
\pgfsys@defobject{currentmarker}{\pgfqpoint{-0.027778in}{0.000000in}}{\pgfqpoint{0.000000in}{0.000000in}}{%
\pgfpathmoveto{\pgfqpoint{0.000000in}{0.000000in}}%
\pgfpathlineto{\pgfqpoint{-0.027778in}{0.000000in}}%
\pgfusepath{stroke,fill}%
}%
\begin{pgfscope}%
\pgfsys@transformshift{8.282041in}{3.591399in}%
\pgfsys@useobject{currentmarker}{}%
\end{pgfscope}%
\end{pgfscope}%
\begin{pgfscope}%
\pgfsetbuttcap%
\pgfsetroundjoin%
\definecolor{currentfill}{rgb}{0.000000,0.000000,0.000000}%
\pgfsetfillcolor{currentfill}%
\pgfsetlinewidth{0.602250pt}%
\definecolor{currentstroke}{rgb}{0.000000,0.000000,0.000000}%
\pgfsetstrokecolor{currentstroke}%
\pgfsetdash{}{0pt}%
\pgfsys@defobject{currentmarker}{\pgfqpoint{-0.027778in}{0.000000in}}{\pgfqpoint{0.000000in}{0.000000in}}{%
\pgfpathmoveto{\pgfqpoint{0.000000in}{0.000000in}}%
\pgfpathlineto{\pgfqpoint{-0.027778in}{0.000000in}}%
\pgfusepath{stroke,fill}%
}%
\begin{pgfscope}%
\pgfsys@transformshift{8.282041in}{3.611330in}%
\pgfsys@useobject{currentmarker}{}%
\end{pgfscope}%
\end{pgfscope}%
\begin{pgfscope}%
\pgfsetbuttcap%
\pgfsetroundjoin%
\definecolor{currentfill}{rgb}{0.000000,0.000000,0.000000}%
\pgfsetfillcolor{currentfill}%
\pgfsetlinewidth{0.602250pt}%
\definecolor{currentstroke}{rgb}{0.000000,0.000000,0.000000}%
\pgfsetstrokecolor{currentstroke}%
\pgfsetdash{}{0pt}%
\pgfsys@defobject{currentmarker}{\pgfqpoint{-0.027778in}{0.000000in}}{\pgfqpoint{0.000000in}{0.000000in}}{%
\pgfpathmoveto{\pgfqpoint{0.000000in}{0.000000in}}%
\pgfpathlineto{\pgfqpoint{-0.027778in}{0.000000in}}%
\pgfusepath{stroke,fill}%
}%
\begin{pgfscope}%
\pgfsys@transformshift{8.282041in}{3.628594in}%
\pgfsys@useobject{currentmarker}{}%
\end{pgfscope}%
\end{pgfscope}%
\begin{pgfscope}%
\pgfsetbuttcap%
\pgfsetroundjoin%
\definecolor{currentfill}{rgb}{0.000000,0.000000,0.000000}%
\pgfsetfillcolor{currentfill}%
\pgfsetlinewidth{0.602250pt}%
\definecolor{currentstroke}{rgb}{0.000000,0.000000,0.000000}%
\pgfsetstrokecolor{currentstroke}%
\pgfsetdash{}{0pt}%
\pgfsys@defobject{currentmarker}{\pgfqpoint{-0.027778in}{0.000000in}}{\pgfqpoint{0.000000in}{0.000000in}}{%
\pgfpathmoveto{\pgfqpoint{0.000000in}{0.000000in}}%
\pgfpathlineto{\pgfqpoint{-0.027778in}{0.000000in}}%
\pgfusepath{stroke,fill}%
}%
\begin{pgfscope}%
\pgfsys@transformshift{8.282041in}{3.643823in}%
\pgfsys@useobject{currentmarker}{}%
\end{pgfscope}%
\end{pgfscope}%
\begin{pgfscope}%
\pgfsetbuttcap%
\pgfsetroundjoin%
\definecolor{currentfill}{rgb}{0.000000,0.000000,0.000000}%
\pgfsetfillcolor{currentfill}%
\pgfsetlinewidth{0.602250pt}%
\definecolor{currentstroke}{rgb}{0.000000,0.000000,0.000000}%
\pgfsetstrokecolor{currentstroke}%
\pgfsetdash{}{0pt}%
\pgfsys@defobject{currentmarker}{\pgfqpoint{-0.027778in}{0.000000in}}{\pgfqpoint{0.000000in}{0.000000in}}{%
\pgfpathmoveto{\pgfqpoint{0.000000in}{0.000000in}}%
\pgfpathlineto{\pgfqpoint{-0.027778in}{0.000000in}}%
\pgfusepath{stroke,fill}%
}%
\begin{pgfscope}%
\pgfsys@transformshift{8.282041in}{3.747063in}%
\pgfsys@useobject{currentmarker}{}%
\end{pgfscope}%
\end{pgfscope}%
\begin{pgfscope}%
\pgfsetbuttcap%
\pgfsetroundjoin%
\definecolor{currentfill}{rgb}{0.000000,0.000000,0.000000}%
\pgfsetfillcolor{currentfill}%
\pgfsetlinewidth{0.602250pt}%
\definecolor{currentstroke}{rgb}{0.000000,0.000000,0.000000}%
\pgfsetstrokecolor{currentstroke}%
\pgfsetdash{}{0pt}%
\pgfsys@defobject{currentmarker}{\pgfqpoint{-0.027778in}{0.000000in}}{\pgfqpoint{0.000000in}{0.000000in}}{%
\pgfpathmoveto{\pgfqpoint{0.000000in}{0.000000in}}%
\pgfpathlineto{\pgfqpoint{-0.027778in}{0.000000in}}%
\pgfusepath{stroke,fill}%
}%
\begin{pgfscope}%
\pgfsys@transformshift{8.282041in}{3.799487in}%
\pgfsys@useobject{currentmarker}{}%
\end{pgfscope}%
\end{pgfscope}%
\begin{pgfscope}%
\pgfsetbuttcap%
\pgfsetroundjoin%
\definecolor{currentfill}{rgb}{0.000000,0.000000,0.000000}%
\pgfsetfillcolor{currentfill}%
\pgfsetlinewidth{0.602250pt}%
\definecolor{currentstroke}{rgb}{0.000000,0.000000,0.000000}%
\pgfsetstrokecolor{currentstroke}%
\pgfsetdash{}{0pt}%
\pgfsys@defobject{currentmarker}{\pgfqpoint{-0.027778in}{0.000000in}}{\pgfqpoint{0.000000in}{0.000000in}}{%
\pgfpathmoveto{\pgfqpoint{0.000000in}{0.000000in}}%
\pgfpathlineto{\pgfqpoint{-0.027778in}{0.000000in}}%
\pgfusepath{stroke,fill}%
}%
\begin{pgfscope}%
\pgfsys@transformshift{8.282041in}{3.836682in}%
\pgfsys@useobject{currentmarker}{}%
\end{pgfscope}%
\end{pgfscope}%
\begin{pgfscope}%
\pgfsetbuttcap%
\pgfsetroundjoin%
\definecolor{currentfill}{rgb}{0.000000,0.000000,0.000000}%
\pgfsetfillcolor{currentfill}%
\pgfsetlinewidth{0.602250pt}%
\definecolor{currentstroke}{rgb}{0.000000,0.000000,0.000000}%
\pgfsetstrokecolor{currentstroke}%
\pgfsetdash{}{0pt}%
\pgfsys@defobject{currentmarker}{\pgfqpoint{-0.027778in}{0.000000in}}{\pgfqpoint{0.000000in}{0.000000in}}{%
\pgfpathmoveto{\pgfqpoint{0.000000in}{0.000000in}}%
\pgfpathlineto{\pgfqpoint{-0.027778in}{0.000000in}}%
\pgfusepath{stroke,fill}%
}%
\begin{pgfscope}%
\pgfsys@transformshift{8.282041in}{3.865533in}%
\pgfsys@useobject{currentmarker}{}%
\end{pgfscope}%
\end{pgfscope}%
\begin{pgfscope}%
\pgfsetbuttcap%
\pgfsetroundjoin%
\definecolor{currentfill}{rgb}{0.000000,0.000000,0.000000}%
\pgfsetfillcolor{currentfill}%
\pgfsetlinewidth{0.602250pt}%
\definecolor{currentstroke}{rgb}{0.000000,0.000000,0.000000}%
\pgfsetstrokecolor{currentstroke}%
\pgfsetdash{}{0pt}%
\pgfsys@defobject{currentmarker}{\pgfqpoint{-0.027778in}{0.000000in}}{\pgfqpoint{0.000000in}{0.000000in}}{%
\pgfpathmoveto{\pgfqpoint{0.000000in}{0.000000in}}%
\pgfpathlineto{\pgfqpoint{-0.027778in}{0.000000in}}%
\pgfusepath{stroke,fill}%
}%
\begin{pgfscope}%
\pgfsys@transformshift{8.282041in}{3.889105in}%
\pgfsys@useobject{currentmarker}{}%
\end{pgfscope}%
\end{pgfscope}%
\begin{pgfscope}%
\pgfsetbuttcap%
\pgfsetroundjoin%
\definecolor{currentfill}{rgb}{0.000000,0.000000,0.000000}%
\pgfsetfillcolor{currentfill}%
\pgfsetlinewidth{0.602250pt}%
\definecolor{currentstroke}{rgb}{0.000000,0.000000,0.000000}%
\pgfsetstrokecolor{currentstroke}%
\pgfsetdash{}{0pt}%
\pgfsys@defobject{currentmarker}{\pgfqpoint{-0.027778in}{0.000000in}}{\pgfqpoint{0.000000in}{0.000000in}}{%
\pgfpathmoveto{\pgfqpoint{0.000000in}{0.000000in}}%
\pgfpathlineto{\pgfqpoint{-0.027778in}{0.000000in}}%
\pgfusepath{stroke,fill}%
}%
\begin{pgfscope}%
\pgfsys@transformshift{8.282041in}{3.909036in}%
\pgfsys@useobject{currentmarker}{}%
\end{pgfscope}%
\end{pgfscope}%
\begin{pgfscope}%
\pgfsetbuttcap%
\pgfsetroundjoin%
\definecolor{currentfill}{rgb}{0.000000,0.000000,0.000000}%
\pgfsetfillcolor{currentfill}%
\pgfsetlinewidth{0.602250pt}%
\definecolor{currentstroke}{rgb}{0.000000,0.000000,0.000000}%
\pgfsetstrokecolor{currentstroke}%
\pgfsetdash{}{0pt}%
\pgfsys@defobject{currentmarker}{\pgfqpoint{-0.027778in}{0.000000in}}{\pgfqpoint{0.000000in}{0.000000in}}{%
\pgfpathmoveto{\pgfqpoint{0.000000in}{0.000000in}}%
\pgfpathlineto{\pgfqpoint{-0.027778in}{0.000000in}}%
\pgfusepath{stroke,fill}%
}%
\begin{pgfscope}%
\pgfsys@transformshift{8.282041in}{3.926300in}%
\pgfsys@useobject{currentmarker}{}%
\end{pgfscope}%
\end{pgfscope}%
\begin{pgfscope}%
\pgfsetbuttcap%
\pgfsetroundjoin%
\definecolor{currentfill}{rgb}{0.000000,0.000000,0.000000}%
\pgfsetfillcolor{currentfill}%
\pgfsetlinewidth{0.602250pt}%
\definecolor{currentstroke}{rgb}{0.000000,0.000000,0.000000}%
\pgfsetstrokecolor{currentstroke}%
\pgfsetdash{}{0pt}%
\pgfsys@defobject{currentmarker}{\pgfqpoint{-0.027778in}{0.000000in}}{\pgfqpoint{0.000000in}{0.000000in}}{%
\pgfpathmoveto{\pgfqpoint{0.000000in}{0.000000in}}%
\pgfpathlineto{\pgfqpoint{-0.027778in}{0.000000in}}%
\pgfusepath{stroke,fill}%
}%
\begin{pgfscope}%
\pgfsys@transformshift{8.282041in}{3.941529in}%
\pgfsys@useobject{currentmarker}{}%
\end{pgfscope}%
\end{pgfscope}%
\begin{pgfscope}%
\pgfsetbuttcap%
\pgfsetroundjoin%
\definecolor{currentfill}{rgb}{0.000000,0.000000,0.000000}%
\pgfsetfillcolor{currentfill}%
\pgfsetlinewidth{0.602250pt}%
\definecolor{currentstroke}{rgb}{0.000000,0.000000,0.000000}%
\pgfsetstrokecolor{currentstroke}%
\pgfsetdash{}{0pt}%
\pgfsys@defobject{currentmarker}{\pgfqpoint{-0.027778in}{0.000000in}}{\pgfqpoint{0.000000in}{0.000000in}}{%
\pgfpathmoveto{\pgfqpoint{0.000000in}{0.000000in}}%
\pgfpathlineto{\pgfqpoint{-0.027778in}{0.000000in}}%
\pgfusepath{stroke,fill}%
}%
\begin{pgfscope}%
\pgfsys@transformshift{8.282041in}{4.044770in}%
\pgfsys@useobject{currentmarker}{}%
\end{pgfscope}%
\end{pgfscope}%
\begin{pgfscope}%
\pgfsetbuttcap%
\pgfsetroundjoin%
\definecolor{currentfill}{rgb}{0.000000,0.000000,0.000000}%
\pgfsetfillcolor{currentfill}%
\pgfsetlinewidth{0.602250pt}%
\definecolor{currentstroke}{rgb}{0.000000,0.000000,0.000000}%
\pgfsetstrokecolor{currentstroke}%
\pgfsetdash{}{0pt}%
\pgfsys@defobject{currentmarker}{\pgfqpoint{-0.027778in}{0.000000in}}{\pgfqpoint{0.000000in}{0.000000in}}{%
\pgfpathmoveto{\pgfqpoint{0.000000in}{0.000000in}}%
\pgfpathlineto{\pgfqpoint{-0.027778in}{0.000000in}}%
\pgfusepath{stroke,fill}%
}%
\begin{pgfscope}%
\pgfsys@transformshift{8.282041in}{4.097193in}%
\pgfsys@useobject{currentmarker}{}%
\end{pgfscope}%
\end{pgfscope}%
\begin{pgfscope}%
\pgfsetbuttcap%
\pgfsetroundjoin%
\definecolor{currentfill}{rgb}{0.000000,0.000000,0.000000}%
\pgfsetfillcolor{currentfill}%
\pgfsetlinewidth{0.602250pt}%
\definecolor{currentstroke}{rgb}{0.000000,0.000000,0.000000}%
\pgfsetstrokecolor{currentstroke}%
\pgfsetdash{}{0pt}%
\pgfsys@defobject{currentmarker}{\pgfqpoint{-0.027778in}{0.000000in}}{\pgfqpoint{0.000000in}{0.000000in}}{%
\pgfpathmoveto{\pgfqpoint{0.000000in}{0.000000in}}%
\pgfpathlineto{\pgfqpoint{-0.027778in}{0.000000in}}%
\pgfusepath{stroke,fill}%
}%
\begin{pgfscope}%
\pgfsys@transformshift{8.282041in}{4.134388in}%
\pgfsys@useobject{currentmarker}{}%
\end{pgfscope}%
\end{pgfscope}%
\begin{pgfscope}%
\pgfsetbuttcap%
\pgfsetroundjoin%
\definecolor{currentfill}{rgb}{0.000000,0.000000,0.000000}%
\pgfsetfillcolor{currentfill}%
\pgfsetlinewidth{0.602250pt}%
\definecolor{currentstroke}{rgb}{0.000000,0.000000,0.000000}%
\pgfsetstrokecolor{currentstroke}%
\pgfsetdash{}{0pt}%
\pgfsys@defobject{currentmarker}{\pgfqpoint{-0.027778in}{0.000000in}}{\pgfqpoint{0.000000in}{0.000000in}}{%
\pgfpathmoveto{\pgfqpoint{0.000000in}{0.000000in}}%
\pgfpathlineto{\pgfqpoint{-0.027778in}{0.000000in}}%
\pgfusepath{stroke,fill}%
}%
\begin{pgfscope}%
\pgfsys@transformshift{8.282041in}{4.163239in}%
\pgfsys@useobject{currentmarker}{}%
\end{pgfscope}%
\end{pgfscope}%
\begin{pgfscope}%
\pgfsetbuttcap%
\pgfsetroundjoin%
\definecolor{currentfill}{rgb}{0.000000,0.000000,0.000000}%
\pgfsetfillcolor{currentfill}%
\pgfsetlinewidth{0.602250pt}%
\definecolor{currentstroke}{rgb}{0.000000,0.000000,0.000000}%
\pgfsetstrokecolor{currentstroke}%
\pgfsetdash{}{0pt}%
\pgfsys@defobject{currentmarker}{\pgfqpoint{-0.027778in}{0.000000in}}{\pgfqpoint{0.000000in}{0.000000in}}{%
\pgfpathmoveto{\pgfqpoint{0.000000in}{0.000000in}}%
\pgfpathlineto{\pgfqpoint{-0.027778in}{0.000000in}}%
\pgfusepath{stroke,fill}%
}%
\begin{pgfscope}%
\pgfsys@transformshift{8.282041in}{4.186812in}%
\pgfsys@useobject{currentmarker}{}%
\end{pgfscope}%
\end{pgfscope}%
\begin{pgfscope}%
\pgfsetbuttcap%
\pgfsetroundjoin%
\definecolor{currentfill}{rgb}{0.000000,0.000000,0.000000}%
\pgfsetfillcolor{currentfill}%
\pgfsetlinewidth{0.602250pt}%
\definecolor{currentstroke}{rgb}{0.000000,0.000000,0.000000}%
\pgfsetstrokecolor{currentstroke}%
\pgfsetdash{}{0pt}%
\pgfsys@defobject{currentmarker}{\pgfqpoint{-0.027778in}{0.000000in}}{\pgfqpoint{0.000000in}{0.000000in}}{%
\pgfpathmoveto{\pgfqpoint{0.000000in}{0.000000in}}%
\pgfpathlineto{\pgfqpoint{-0.027778in}{0.000000in}}%
\pgfusepath{stroke,fill}%
}%
\begin{pgfscope}%
\pgfsys@transformshift{8.282041in}{4.206742in}%
\pgfsys@useobject{currentmarker}{}%
\end{pgfscope}%
\end{pgfscope}%
\begin{pgfscope}%
\pgfsetbuttcap%
\pgfsetroundjoin%
\definecolor{currentfill}{rgb}{0.000000,0.000000,0.000000}%
\pgfsetfillcolor{currentfill}%
\pgfsetlinewidth{0.602250pt}%
\definecolor{currentstroke}{rgb}{0.000000,0.000000,0.000000}%
\pgfsetstrokecolor{currentstroke}%
\pgfsetdash{}{0pt}%
\pgfsys@defobject{currentmarker}{\pgfqpoint{-0.027778in}{0.000000in}}{\pgfqpoint{0.000000in}{0.000000in}}{%
\pgfpathmoveto{\pgfqpoint{0.000000in}{0.000000in}}%
\pgfpathlineto{\pgfqpoint{-0.027778in}{0.000000in}}%
\pgfusepath{stroke,fill}%
}%
\begin{pgfscope}%
\pgfsys@transformshift{8.282041in}{4.224007in}%
\pgfsys@useobject{currentmarker}{}%
\end{pgfscope}%
\end{pgfscope}%
\begin{pgfscope}%
\pgfpathrectangle{\pgfqpoint{8.282041in}{2.849537in}}{\pgfqpoint{1.897959in}{1.372727in}} %
\pgfusepath{clip}%
\pgfsetbuttcap%
\pgfsetroundjoin%
\pgfsetlinewidth{1.505625pt}%
\definecolor{currentstroke}{rgb}{1.000000,0.000000,0.000000}%
\pgfsetstrokecolor{currentstroke}%
\pgfsetdash{{5.550000pt}{2.400000pt}}{0.000000pt}%
\pgfpathmoveto{\pgfqpoint{8.368312in}{4.078457in}}%
\pgfpathlineto{\pgfqpoint{8.382934in}{4.075878in}}%
\pgfpathlineto{\pgfqpoint{8.397556in}{4.073335in}}%
\pgfpathlineto{\pgfqpoint{8.412178in}{4.070827in}}%
\pgfpathlineto{\pgfqpoint{8.426800in}{4.068351in}}%
\pgfpathlineto{\pgfqpoint{8.441423in}{4.065908in}}%
\pgfpathlineto{\pgfqpoint{8.456045in}{4.063497in}}%
\pgfpathlineto{\pgfqpoint{8.470667in}{4.061118in}}%
\pgfpathlineto{\pgfqpoint{8.485289in}{4.058771in}}%
\pgfpathlineto{\pgfqpoint{8.499911in}{4.056455in}}%
\pgfpathlineto{\pgfqpoint{8.514534in}{4.054170in}}%
\pgfpathlineto{\pgfqpoint{8.529156in}{4.051917in}}%
\pgfpathlineto{\pgfqpoint{8.543778in}{4.049695in}}%
\pgfpathlineto{\pgfqpoint{8.558400in}{4.047504in}}%
\pgfpathlineto{\pgfqpoint{8.573022in}{4.045344in}}%
\pgfpathlineto{\pgfqpoint{8.587644in}{4.043216in}}%
\pgfpathlineto{\pgfqpoint{8.602267in}{4.041118in}}%
\pgfpathlineto{\pgfqpoint{8.616889in}{4.039051in}}%
\pgfpathlineto{\pgfqpoint{8.631511in}{4.037014in}}%
\pgfpathlineto{\pgfqpoint{8.646133in}{4.035008in}}%
\pgfpathlineto{\pgfqpoint{8.660755in}{4.033033in}}%
\pgfpathlineto{\pgfqpoint{8.675378in}{4.031088in}}%
\pgfpathlineto{\pgfqpoint{8.690000in}{4.029173in}}%
\pgfpathlineto{\pgfqpoint{8.704622in}{4.027287in}}%
\pgfpathlineto{\pgfqpoint{8.719244in}{4.025431in}}%
\pgfpathlineto{\pgfqpoint{8.733866in}{4.023605in}}%
\pgfpathlineto{\pgfqpoint{8.748488in}{4.021808in}}%
\pgfpathlineto{\pgfqpoint{8.763111in}{4.020039in}}%
\pgfpathlineto{\pgfqpoint{8.777733in}{4.018300in}}%
\pgfpathlineto{\pgfqpoint{8.792355in}{4.016588in}}%
\pgfpathlineto{\pgfqpoint{8.806977in}{4.014905in}}%
\pgfpathlineto{\pgfqpoint{8.821599in}{4.013250in}}%
\pgfpathlineto{\pgfqpoint{8.836222in}{4.011622in}}%
\pgfpathlineto{\pgfqpoint{8.850844in}{4.010022in}}%
\pgfpathlineto{\pgfqpoint{8.865466in}{4.008448in}}%
\pgfpathlineto{\pgfqpoint{8.880088in}{4.006902in}}%
\pgfpathlineto{\pgfqpoint{8.894710in}{4.005382in}}%
\pgfpathlineto{\pgfqpoint{8.909332in}{4.003888in}}%
\pgfpathlineto{\pgfqpoint{8.923955in}{4.002420in}}%
\pgfpathlineto{\pgfqpoint{8.938577in}{4.000977in}}%
\pgfpathlineto{\pgfqpoint{8.953199in}{3.999560in}}%
\pgfpathlineto{\pgfqpoint{8.967821in}{3.998168in}}%
\pgfpathlineto{\pgfqpoint{8.982443in}{3.996801in}}%
\pgfpathlineto{\pgfqpoint{8.997066in}{3.995458in}}%
\pgfpathlineto{\pgfqpoint{9.011688in}{3.994140in}}%
\pgfpathlineto{\pgfqpoint{9.026310in}{3.992845in}}%
\pgfpathlineto{\pgfqpoint{9.040932in}{3.991574in}}%
\pgfpathlineto{\pgfqpoint{9.055554in}{3.990327in}}%
\pgfpathlineto{\pgfqpoint{9.070176in}{3.989102in}}%
\pgfpathlineto{\pgfqpoint{9.084799in}{3.987901in}}%
\pgfpathlineto{\pgfqpoint{9.099421in}{3.986722in}}%
\pgfpathlineto{\pgfqpoint{9.114043in}{3.985566in}}%
\pgfpathlineto{\pgfqpoint{9.128665in}{3.984431in}}%
\pgfpathlineto{\pgfqpoint{9.143287in}{3.983319in}}%
\pgfpathlineto{\pgfqpoint{9.157909in}{3.982228in}}%
\pgfpathlineto{\pgfqpoint{9.172532in}{3.981158in}}%
\pgfpathlineto{\pgfqpoint{9.187154in}{3.980110in}}%
\pgfpathlineto{\pgfqpoint{9.201776in}{3.979083in}}%
\pgfpathlineto{\pgfqpoint{9.216398in}{3.978076in}}%
\pgfpathlineto{\pgfqpoint{9.231020in}{3.977090in}}%
\pgfpathlineto{\pgfqpoint{9.245643in}{3.976124in}}%
\pgfpathlineto{\pgfqpoint{9.260265in}{3.975178in}}%
\pgfpathlineto{\pgfqpoint{9.274887in}{3.974252in}}%
\pgfpathlineto{\pgfqpoint{9.289509in}{3.973345in}}%
\pgfpathlineto{\pgfqpoint{9.304131in}{3.972458in}}%
\pgfpathlineto{\pgfqpoint{9.318753in}{3.971591in}}%
\pgfpathlineto{\pgfqpoint{9.333376in}{3.970742in}}%
\pgfpathlineto{\pgfqpoint{9.347998in}{3.969912in}}%
\pgfpathlineto{\pgfqpoint{9.362620in}{3.969101in}}%
\pgfpathlineto{\pgfqpoint{9.377242in}{3.968308in}}%
\pgfpathlineto{\pgfqpoint{9.391864in}{3.967534in}}%
\pgfpathlineto{\pgfqpoint{9.406487in}{3.966777in}}%
\pgfpathlineto{\pgfqpoint{9.421109in}{3.966039in}}%
\pgfpathlineto{\pgfqpoint{9.435731in}{3.965319in}}%
\pgfpathlineto{\pgfqpoint{9.450353in}{3.964616in}}%
\pgfpathlineto{\pgfqpoint{9.464975in}{3.963931in}}%
\pgfpathlineto{\pgfqpoint{9.479597in}{3.963263in}}%
\pgfpathlineto{\pgfqpoint{9.494220in}{3.962612in}}%
\pgfpathlineto{\pgfqpoint{9.508842in}{3.961978in}}%
\pgfpathlineto{\pgfqpoint{9.523464in}{3.961362in}}%
\pgfpathlineto{\pgfqpoint{9.538086in}{3.960762in}}%
\pgfpathlineto{\pgfqpoint{9.552708in}{3.960179in}}%
\pgfpathlineto{\pgfqpoint{9.567331in}{3.959612in}}%
\pgfpathlineto{\pgfqpoint{9.581953in}{3.959062in}}%
\pgfpathlineto{\pgfqpoint{9.596575in}{3.958528in}}%
\pgfpathlineto{\pgfqpoint{9.611197in}{3.958010in}}%
\pgfpathlineto{\pgfqpoint{9.625819in}{3.957508in}}%
\pgfpathlineto{\pgfqpoint{9.640441in}{3.957023in}}%
\pgfpathlineto{\pgfqpoint{9.655064in}{3.956553in}}%
\pgfpathlineto{\pgfqpoint{9.669686in}{3.956099in}}%
\pgfpathlineto{\pgfqpoint{9.684308in}{3.955660in}}%
\pgfpathlineto{\pgfqpoint{9.698930in}{3.955237in}}%
\pgfpathlineto{\pgfqpoint{9.713552in}{3.954830in}}%
\pgfpathlineto{\pgfqpoint{9.728175in}{3.954438in}}%
\pgfpathlineto{\pgfqpoint{9.742797in}{3.954061in}}%
\pgfpathlineto{\pgfqpoint{9.757419in}{3.953700in}}%
\pgfpathlineto{\pgfqpoint{9.772041in}{3.953354in}}%
\pgfpathlineto{\pgfqpoint{9.786663in}{3.953023in}}%
\pgfpathlineto{\pgfqpoint{9.801285in}{3.952706in}}%
\pgfpathlineto{\pgfqpoint{9.815908in}{3.952405in}}%
\pgfpathlineto{\pgfqpoint{9.830530in}{3.952119in}}%
\pgfpathlineto{\pgfqpoint{9.845152in}{3.951847in}}%
\pgfpathlineto{\pgfqpoint{9.859774in}{3.951590in}}%
\pgfpathlineto{\pgfqpoint{9.874396in}{3.951348in}}%
\pgfpathlineto{\pgfqpoint{9.889019in}{3.951120in}}%
\pgfpathlineto{\pgfqpoint{9.903641in}{3.950907in}}%
\pgfpathlineto{\pgfqpoint{9.918263in}{3.950709in}}%
\pgfpathlineto{\pgfqpoint{9.932885in}{3.950525in}}%
\pgfpathlineto{\pgfqpoint{9.947507in}{3.950356in}}%
\pgfpathlineto{\pgfqpoint{9.962129in}{3.950201in}}%
\pgfpathlineto{\pgfqpoint{9.976752in}{3.950060in}}%
\pgfpathlineto{\pgfqpoint{9.991374in}{3.949934in}}%
\pgfpathlineto{\pgfqpoint{10.005996in}{3.949822in}}%
\pgfpathlineto{\pgfqpoint{10.020618in}{3.949724in}}%
\pgfpathlineto{\pgfqpoint{10.035240in}{3.949641in}}%
\pgfpathlineto{\pgfqpoint{10.049863in}{3.949572in}}%
\pgfpathlineto{\pgfqpoint{10.064485in}{3.949517in}}%
\pgfpathlineto{\pgfqpoint{10.079107in}{3.949476in}}%
\pgfpathlineto{\pgfqpoint{10.093729in}{3.949450in}}%
\pgfusepath{stroke}%
\end{pgfscope}%
\begin{pgfscope}%
\pgfpathrectangle{\pgfqpoint{8.282041in}{2.849537in}}{\pgfqpoint{1.897959in}{1.372727in}} %
\pgfusepath{clip}%
\pgfsetbuttcap%
\pgfsetmiterjoin%
\definecolor{currentfill}{rgb}{1.000000,0.000000,0.000000}%
\pgfsetfillcolor{currentfill}%
\pgfsetlinewidth{1.003750pt}%
\definecolor{currentstroke}{rgb}{1.000000,0.000000,0.000000}%
\pgfsetstrokecolor{currentstroke}%
\pgfsetdash{}{0pt}%
\pgfsys@defobject{currentmarker}{\pgfqpoint{-0.041667in}{-0.041667in}}{\pgfqpoint{0.041667in}{0.041667in}}{%
\pgfpathmoveto{\pgfqpoint{-0.041667in}{-0.041667in}}%
\pgfpathlineto{\pgfqpoint{0.041667in}{-0.041667in}}%
\pgfpathlineto{\pgfqpoint{0.041667in}{0.041667in}}%
\pgfpathlineto{\pgfqpoint{-0.041667in}{0.041667in}}%
\pgfpathclose%
\pgfusepath{stroke,fill}%
}%
\begin{pgfscope}%
\pgfsys@transformshift{8.368312in}{4.078457in}%
\pgfsys@useobject{currentmarker}{}%
\end{pgfscope}%
\begin{pgfscope}%
\pgfsys@transformshift{8.719244in}{4.025431in}%
\pgfsys@useobject{currentmarker}{}%
\end{pgfscope}%
\begin{pgfscope}%
\pgfsys@transformshift{9.070176in}{3.989102in}%
\pgfsys@useobject{currentmarker}{}%
\end{pgfscope}%
\begin{pgfscope}%
\pgfsys@transformshift{9.421109in}{3.966039in}%
\pgfsys@useobject{currentmarker}{}%
\end{pgfscope}%
\begin{pgfscope}%
\pgfsys@transformshift{9.772041in}{3.953354in}%
\pgfsys@useobject{currentmarker}{}%
\end{pgfscope}%
\end{pgfscope}%
\begin{pgfscope}%
\pgfpathrectangle{\pgfqpoint{8.282041in}{2.849537in}}{\pgfqpoint{1.897959in}{1.372727in}} %
\pgfusepath{clip}%
\pgfsetrectcap%
\pgfsetroundjoin%
\pgfsetlinewidth{1.505625pt}%
\definecolor{currentstroke}{rgb}{0.000000,0.000000,1.000000}%
\pgfsetstrokecolor{currentstroke}%
\pgfsetdash{}{0pt}%
\pgfpathmoveto{\pgfqpoint{8.368312in}{3.665392in}}%
\pgfpathlineto{\pgfqpoint{8.382934in}{3.660990in}}%
\pgfpathlineto{\pgfqpoint{8.397556in}{3.656839in}}%
\pgfpathlineto{\pgfqpoint{8.412178in}{3.652885in}}%
\pgfpathlineto{\pgfqpoint{8.426800in}{3.649090in}}%
\pgfpathlineto{\pgfqpoint{8.441423in}{3.645424in}}%
\pgfpathlineto{\pgfqpoint{8.456045in}{3.641868in}}%
\pgfpathlineto{\pgfqpoint{8.470667in}{3.638402in}}%
\pgfpathlineto{\pgfqpoint{8.485289in}{3.635015in}}%
\pgfpathlineto{\pgfqpoint{8.499911in}{3.631695in}}%
\pgfpathlineto{\pgfqpoint{8.514534in}{3.628434in}}%
\pgfpathlineto{\pgfqpoint{8.529156in}{3.625225in}}%
\pgfpathlineto{\pgfqpoint{8.543778in}{3.622061in}}%
\pgfpathlineto{\pgfqpoint{8.558400in}{3.618937in}}%
\pgfpathlineto{\pgfqpoint{8.573022in}{3.615848in}}%
\pgfpathlineto{\pgfqpoint{8.587644in}{3.612792in}}%
\pgfpathlineto{\pgfqpoint{8.602267in}{3.609764in}}%
\pgfpathlineto{\pgfqpoint{8.616889in}{3.606762in}}%
\pgfpathlineto{\pgfqpoint{8.631511in}{3.603782in}}%
\pgfpathlineto{\pgfqpoint{8.646133in}{3.600823in}}%
\pgfpathlineto{\pgfqpoint{8.660755in}{3.597882in}}%
\pgfpathlineto{\pgfqpoint{8.675378in}{3.594957in}}%
\pgfpathlineto{\pgfqpoint{8.690000in}{3.592047in}}%
\pgfpathlineto{\pgfqpoint{8.704622in}{3.589149in}}%
\pgfpathlineto{\pgfqpoint{8.719244in}{3.586263in}}%
\pgfpathlineto{\pgfqpoint{8.733866in}{3.583386in}}%
\pgfpathlineto{\pgfqpoint{8.748488in}{3.580517in}}%
\pgfpathlineto{\pgfqpoint{8.763111in}{3.577655in}}%
\pgfpathlineto{\pgfqpoint{8.777733in}{3.574799in}}%
\pgfpathlineto{\pgfqpoint{8.792355in}{3.571948in}}%
\pgfpathlineto{\pgfqpoint{8.806977in}{3.569099in}}%
\pgfpathlineto{\pgfqpoint{8.821599in}{3.566253in}}%
\pgfpathlineto{\pgfqpoint{8.836222in}{3.563408in}}%
\pgfpathlineto{\pgfqpoint{8.850844in}{3.560563in}}%
\pgfpathlineto{\pgfqpoint{8.865466in}{3.557716in}}%
\pgfpathlineto{\pgfqpoint{8.880088in}{3.554868in}}%
\pgfpathlineto{\pgfqpoint{8.894710in}{3.552016in}}%
\pgfpathlineto{\pgfqpoint{8.909332in}{3.549161in}}%
\pgfpathlineto{\pgfqpoint{8.923955in}{3.546300in}}%
\pgfpathlineto{\pgfqpoint{8.938577in}{3.543433in}}%
\pgfpathlineto{\pgfqpoint{8.953199in}{3.540559in}}%
\pgfpathlineto{\pgfqpoint{8.967821in}{3.537677in}}%
\pgfpathlineto{\pgfqpoint{8.982443in}{3.534786in}}%
\pgfpathlineto{\pgfqpoint{8.997066in}{3.531884in}}%
\pgfpathlineto{\pgfqpoint{9.011688in}{3.528972in}}%
\pgfpathlineto{\pgfqpoint{9.026310in}{3.526048in}}%
\pgfpathlineto{\pgfqpoint{9.040932in}{3.523110in}}%
\pgfpathlineto{\pgfqpoint{9.055554in}{3.520159in}}%
\pgfpathlineto{\pgfqpoint{9.070176in}{3.517192in}}%
\pgfpathlineto{\pgfqpoint{9.084799in}{3.514209in}}%
\pgfpathlineto{\pgfqpoint{9.099421in}{3.511208in}}%
\pgfpathlineto{\pgfqpoint{9.114043in}{3.508189in}}%
\pgfpathlineto{\pgfqpoint{9.128665in}{3.505151in}}%
\pgfpathlineto{\pgfqpoint{9.143287in}{3.502091in}}%
\pgfpathlineto{\pgfqpoint{9.157909in}{3.499010in}}%
\pgfpathlineto{\pgfqpoint{9.172532in}{3.495905in}}%
\pgfpathlineto{\pgfqpoint{9.187154in}{3.492775in}}%
\pgfpathlineto{\pgfqpoint{9.201776in}{3.489620in}}%
\pgfpathlineto{\pgfqpoint{9.216398in}{3.486437in}}%
\pgfpathlineto{\pgfqpoint{9.231020in}{3.483226in}}%
\pgfpathlineto{\pgfqpoint{9.245643in}{3.479984in}}%
\pgfpathlineto{\pgfqpoint{9.260265in}{3.476711in}}%
\pgfpathlineto{\pgfqpoint{9.274887in}{3.473404in}}%
\pgfpathlineto{\pgfqpoint{9.289509in}{3.470062in}}%
\pgfpathlineto{\pgfqpoint{9.304131in}{3.466683in}}%
\pgfpathlineto{\pgfqpoint{9.318753in}{3.463266in}}%
\pgfpathlineto{\pgfqpoint{9.333376in}{3.459808in}}%
\pgfpathlineto{\pgfqpoint{9.347998in}{3.456307in}}%
\pgfpathlineto{\pgfqpoint{9.362620in}{3.452762in}}%
\pgfpathlineto{\pgfqpoint{9.377242in}{3.449170in}}%
\pgfpathlineto{\pgfqpoint{9.391864in}{3.445529in}}%
\pgfpathlineto{\pgfqpoint{9.406487in}{3.441836in}}%
\pgfpathlineto{\pgfqpoint{9.421109in}{3.438089in}}%
\pgfpathlineto{\pgfqpoint{9.435731in}{3.434285in}}%
\pgfpathlineto{\pgfqpoint{9.450353in}{3.430421in}}%
\pgfpathlineto{\pgfqpoint{9.464975in}{3.426493in}}%
\pgfpathlineto{\pgfqpoint{9.479597in}{3.422500in}}%
\pgfpathlineto{\pgfqpoint{9.494220in}{3.418436in}}%
\pgfpathlineto{\pgfqpoint{9.508842in}{3.414299in}}%
\pgfpathlineto{\pgfqpoint{9.523464in}{3.410084in}}%
\pgfpathlineto{\pgfqpoint{9.538086in}{3.405787in}}%
\pgfpathlineto{\pgfqpoint{9.552708in}{3.401403in}}%
\pgfpathlineto{\pgfqpoint{9.567331in}{3.396928in}}%
\pgfpathlineto{\pgfqpoint{9.581953in}{3.392356in}}%
\pgfpathlineto{\pgfqpoint{9.596575in}{3.387681in}}%
\pgfpathlineto{\pgfqpoint{9.611197in}{3.382898in}}%
\pgfpathlineto{\pgfqpoint{9.625819in}{3.377998in}}%
\pgfpathlineto{\pgfqpoint{9.640441in}{3.372975in}}%
\pgfpathlineto{\pgfqpoint{9.655064in}{3.367821in}}%
\pgfpathlineto{\pgfqpoint{9.669686in}{3.362527in}}%
\pgfpathlineto{\pgfqpoint{9.684308in}{3.357083in}}%
\pgfpathlineto{\pgfqpoint{9.698930in}{3.351479in}}%
\pgfpathlineto{\pgfqpoint{9.713552in}{3.345701in}}%
\pgfpathlineto{\pgfqpoint{9.728175in}{3.339739in}}%
\pgfpathlineto{\pgfqpoint{9.742797in}{3.333576in}}%
\pgfpathlineto{\pgfqpoint{9.757419in}{3.327197in}}%
\pgfpathlineto{\pgfqpoint{9.772041in}{3.320583in}}%
\pgfpathlineto{\pgfqpoint{9.786663in}{3.313713in}}%
\pgfpathlineto{\pgfqpoint{9.801285in}{3.306564in}}%
\pgfpathlineto{\pgfqpoint{9.815908in}{3.299109in}}%
\pgfpathlineto{\pgfqpoint{9.830530in}{3.291317in}}%
\pgfpathlineto{\pgfqpoint{9.845152in}{3.283153in}}%
\pgfpathlineto{\pgfqpoint{9.859774in}{3.274574in}}%
\pgfpathlineto{\pgfqpoint{9.874396in}{3.265533in}}%
\pgfpathlineto{\pgfqpoint{9.889019in}{3.255969in}}%
\pgfpathlineto{\pgfqpoint{9.903641in}{3.245815in}}%
\pgfpathlineto{\pgfqpoint{9.918263in}{3.234986in}}%
\pgfpathlineto{\pgfqpoint{9.932885in}{3.223377in}}%
\pgfpathlineto{\pgfqpoint{9.947507in}{3.210857in}}%
\pgfpathlineto{\pgfqpoint{9.962129in}{3.197262in}}%
\pgfpathlineto{\pgfqpoint{9.976752in}{3.182375in}}%
\pgfpathlineto{\pgfqpoint{9.991374in}{3.165908in}}%
\pgfpathlineto{\pgfqpoint{10.005996in}{3.147462in}}%
\pgfpathlineto{\pgfqpoint{10.020618in}{3.126466in}}%
\pgfpathlineto{\pgfqpoint{10.035240in}{3.102057in}}%
\pgfpathlineto{\pgfqpoint{10.049863in}{3.072835in}}%
\pgfpathlineto{\pgfqpoint{10.064485in}{3.036310in}}%
\pgfpathlineto{\pgfqpoint{10.079107in}{2.987320in}}%
\pgfpathlineto{\pgfqpoint{10.093729in}{2.911933in}}%
\pgfusepath{stroke}%
\end{pgfscope}%
\begin{pgfscope}%
\pgfpathrectangle{\pgfqpoint{8.282041in}{2.849537in}}{\pgfqpoint{1.897959in}{1.372727in}} %
\pgfusepath{clip}%
\pgfsetbuttcap%
\pgfsetroundjoin%
\definecolor{currentfill}{rgb}{0.000000,0.000000,1.000000}%
\pgfsetfillcolor{currentfill}%
\pgfsetlinewidth{1.003750pt}%
\definecolor{currentstroke}{rgb}{0.000000,0.000000,1.000000}%
\pgfsetstrokecolor{currentstroke}%
\pgfsetdash{}{0pt}%
\pgfsys@defobject{currentmarker}{\pgfqpoint{-0.041667in}{-0.041667in}}{\pgfqpoint{0.041667in}{0.041667in}}{%
\pgfpathmoveto{\pgfqpoint{0.000000in}{-0.041667in}}%
\pgfpathcurveto{\pgfqpoint{0.011050in}{-0.041667in}}{\pgfqpoint{0.021649in}{-0.037276in}}{\pgfqpoint{0.029463in}{-0.029463in}}%
\pgfpathcurveto{\pgfqpoint{0.037276in}{-0.021649in}}{\pgfqpoint{0.041667in}{-0.011050in}}{\pgfqpoint{0.041667in}{0.000000in}}%
\pgfpathcurveto{\pgfqpoint{0.041667in}{0.011050in}}{\pgfqpoint{0.037276in}{0.021649in}}{\pgfqpoint{0.029463in}{0.029463in}}%
\pgfpathcurveto{\pgfqpoint{0.021649in}{0.037276in}}{\pgfqpoint{0.011050in}{0.041667in}}{\pgfqpoint{0.000000in}{0.041667in}}%
\pgfpathcurveto{\pgfqpoint{-0.011050in}{0.041667in}}{\pgfqpoint{-0.021649in}{0.037276in}}{\pgfqpoint{-0.029463in}{0.029463in}}%
\pgfpathcurveto{\pgfqpoint{-0.037276in}{0.021649in}}{\pgfqpoint{-0.041667in}{0.011050in}}{\pgfqpoint{-0.041667in}{0.000000in}}%
\pgfpathcurveto{\pgfqpoint{-0.041667in}{-0.011050in}}{\pgfqpoint{-0.037276in}{-0.021649in}}{\pgfqpoint{-0.029463in}{-0.029463in}}%
\pgfpathcurveto{\pgfqpoint{-0.021649in}{-0.037276in}}{\pgfqpoint{-0.011050in}{-0.041667in}}{\pgfqpoint{0.000000in}{-0.041667in}}%
\pgfpathclose%
\pgfusepath{stroke,fill}%
}%
\begin{pgfscope}%
\pgfsys@transformshift{8.368312in}{3.665392in}%
\pgfsys@useobject{currentmarker}{}%
\end{pgfscope}%
\begin{pgfscope}%
\pgfsys@transformshift{8.719244in}{3.586263in}%
\pgfsys@useobject{currentmarker}{}%
\end{pgfscope}%
\begin{pgfscope}%
\pgfsys@transformshift{9.070176in}{3.517192in}%
\pgfsys@useobject{currentmarker}{}%
\end{pgfscope}%
\begin{pgfscope}%
\pgfsys@transformshift{9.421109in}{3.438089in}%
\pgfsys@useobject{currentmarker}{}%
\end{pgfscope}%
\begin{pgfscope}%
\pgfsys@transformshift{9.772041in}{3.320583in}%
\pgfsys@useobject{currentmarker}{}%
\end{pgfscope}%
\end{pgfscope}%
\begin{pgfscope}%
\pgfpathrectangle{\pgfqpoint{8.282041in}{2.849537in}}{\pgfqpoint{1.897959in}{1.372727in}} %
\pgfusepath{clip}%
\pgfsetbuttcap%
\pgfsetroundjoin%
\pgfsetlinewidth{1.505625pt}%
\definecolor{currentstroke}{rgb}{0.000000,0.750000,0.750000}%
\pgfsetstrokecolor{currentstroke}%
\pgfsetdash{{9.600000pt}{2.400000pt}{1.500000pt}{2.400000pt}}{0.000000pt}%
\pgfpathmoveto{\pgfqpoint{8.368312in}{4.057155in}}%
\pgfpathlineto{\pgfqpoint{8.382934in}{4.034604in}}%
\pgfpathlineto{\pgfqpoint{8.397556in}{4.014328in}}%
\pgfpathlineto{\pgfqpoint{8.412178in}{3.995930in}}%
\pgfpathlineto{\pgfqpoint{8.426800in}{3.979108in}}%
\pgfpathlineto{\pgfqpoint{8.441423in}{3.963629in}}%
\pgfpathlineto{\pgfqpoint{8.456045in}{3.949307in}}%
\pgfpathlineto{\pgfqpoint{8.470667in}{3.935993in}}%
\pgfpathlineto{\pgfqpoint{8.485289in}{3.923566in}}%
\pgfpathlineto{\pgfqpoint{8.499911in}{3.911925in}}%
\pgfpathlineto{\pgfqpoint{8.514534in}{3.900985in}}%
\pgfpathlineto{\pgfqpoint{8.529156in}{3.890674in}}%
\pgfpathlineto{\pgfqpoint{8.543778in}{3.880932in}}%
\pgfpathlineto{\pgfqpoint{8.558400in}{3.871706in}}%
\pgfpathlineto{\pgfqpoint{8.573022in}{3.862949in}}%
\pgfpathlineto{\pgfqpoint{8.587644in}{3.854624in}}%
\pgfpathlineto{\pgfqpoint{8.602267in}{3.846693in}}%
\pgfpathlineto{\pgfqpoint{8.616889in}{3.839127in}}%
\pgfpathlineto{\pgfqpoint{8.631511in}{3.831899in}}%
\pgfpathlineto{\pgfqpoint{8.646133in}{3.824983in}}%
\pgfpathlineto{\pgfqpoint{8.660755in}{3.818358in}}%
\pgfpathlineto{\pgfqpoint{8.675378in}{3.812005in}}%
\pgfpathlineto{\pgfqpoint{8.690000in}{3.805905in}}%
\pgfpathlineto{\pgfqpoint{8.704622in}{3.800043in}}%
\pgfpathlineto{\pgfqpoint{8.719244in}{3.794404in}}%
\pgfpathlineto{\pgfqpoint{8.733866in}{3.788975in}}%
\pgfpathlineto{\pgfqpoint{8.748488in}{3.783744in}}%
\pgfpathlineto{\pgfqpoint{8.763111in}{3.778699in}}%
\pgfpathlineto{\pgfqpoint{8.777733in}{3.773832in}}%
\pgfpathlineto{\pgfqpoint{8.792355in}{3.769131in}}%
\pgfpathlineto{\pgfqpoint{8.806977in}{3.764589in}}%
\pgfpathlineto{\pgfqpoint{8.821599in}{3.760198in}}%
\pgfpathlineto{\pgfqpoint{8.836222in}{3.755951in}}%
\pgfpathlineto{\pgfqpoint{8.850844in}{3.751840in}}%
\pgfpathlineto{\pgfqpoint{8.865466in}{3.747859in}}%
\pgfpathlineto{\pgfqpoint{8.880088in}{3.744002in}}%
\pgfpathlineto{\pgfqpoint{8.894710in}{3.740265in}}%
\pgfpathlineto{\pgfqpoint{8.909332in}{3.736641in}}%
\pgfpathlineto{\pgfqpoint{8.923955in}{3.733126in}}%
\pgfpathlineto{\pgfqpoint{8.938577in}{3.729715in}}%
\pgfpathlineto{\pgfqpoint{8.953199in}{3.726405in}}%
\pgfpathlineto{\pgfqpoint{8.967821in}{3.723191in}}%
\pgfpathlineto{\pgfqpoint{8.982443in}{3.720069in}}%
\pgfpathlineto{\pgfqpoint{8.997066in}{3.717037in}}%
\pgfpathlineto{\pgfqpoint{9.011688in}{3.714090in}}%
\pgfpathlineto{\pgfqpoint{9.026310in}{3.711226in}}%
\pgfpathlineto{\pgfqpoint{9.040932in}{3.708442in}}%
\pgfpathlineto{\pgfqpoint{9.055554in}{3.705735in}}%
\pgfpathlineto{\pgfqpoint{9.070176in}{3.703102in}}%
\pgfpathlineto{\pgfqpoint{9.084799in}{3.700541in}}%
\pgfpathlineto{\pgfqpoint{9.099421in}{3.698050in}}%
\pgfpathlineto{\pgfqpoint{9.114043in}{3.695626in}}%
\pgfpathlineto{\pgfqpoint{9.128665in}{3.693268in}}%
\pgfpathlineto{\pgfqpoint{9.143287in}{3.690973in}}%
\pgfpathlineto{\pgfqpoint{9.157909in}{3.688739in}}%
\pgfpathlineto{\pgfqpoint{9.172532in}{3.686565in}}%
\pgfpathlineto{\pgfqpoint{9.187154in}{3.684449in}}%
\pgfpathlineto{\pgfqpoint{9.201776in}{3.682389in}}%
\pgfpathlineto{\pgfqpoint{9.216398in}{3.680384in}}%
\pgfpathlineto{\pgfqpoint{9.231020in}{3.678432in}}%
\pgfpathlineto{\pgfqpoint{9.245643in}{3.676532in}}%
\pgfpathlineto{\pgfqpoint{9.260265in}{3.674683in}}%
\pgfpathlineto{\pgfqpoint{9.274887in}{3.672883in}}%
\pgfpathlineto{\pgfqpoint{9.289509in}{3.671131in}}%
\pgfpathlineto{\pgfqpoint{9.304131in}{3.669426in}}%
\pgfpathlineto{\pgfqpoint{9.318753in}{3.667766in}}%
\pgfpathlineto{\pgfqpoint{9.333376in}{3.666151in}}%
\pgfpathlineto{\pgfqpoint{9.347998in}{3.664580in}}%
\pgfpathlineto{\pgfqpoint{9.362620in}{3.663052in}}%
\pgfpathlineto{\pgfqpoint{9.377242in}{3.661565in}}%
\pgfpathlineto{\pgfqpoint{9.391864in}{3.660120in}}%
\pgfpathlineto{\pgfqpoint{9.406487in}{3.658714in}}%
\pgfpathlineto{\pgfqpoint{9.421109in}{3.657347in}}%
\pgfpathlineto{\pgfqpoint{9.435731in}{3.656019in}}%
\pgfpathlineto{\pgfqpoint{9.450353in}{3.654729in}}%
\pgfpathlineto{\pgfqpoint{9.464975in}{3.653475in}}%
\pgfpathlineto{\pgfqpoint{9.479597in}{3.652258in}}%
\pgfpathlineto{\pgfqpoint{9.494220in}{3.651076in}}%
\pgfpathlineto{\pgfqpoint{9.508842in}{3.649930in}}%
\pgfpathlineto{\pgfqpoint{9.523464in}{3.648817in}}%
\pgfpathlineto{\pgfqpoint{9.538086in}{3.647739in}}%
\pgfpathlineto{\pgfqpoint{9.552708in}{3.646694in}}%
\pgfpathlineto{\pgfqpoint{9.567331in}{3.645681in}}%
\pgfpathlineto{\pgfqpoint{9.581953in}{3.644701in}}%
\pgfpathlineto{\pgfqpoint{9.596575in}{3.643752in}}%
\pgfpathlineto{\pgfqpoint{9.611197in}{3.642835in}}%
\pgfpathlineto{\pgfqpoint{9.625819in}{3.641949in}}%
\pgfpathlineto{\pgfqpoint{9.640441in}{3.641092in}}%
\pgfpathlineto{\pgfqpoint{9.655064in}{3.640266in}}%
\pgfpathlineto{\pgfqpoint{9.669686in}{3.639470in}}%
\pgfpathlineto{\pgfqpoint{9.684308in}{3.638702in}}%
\pgfpathlineto{\pgfqpoint{9.698930in}{3.637964in}}%
\pgfpathlineto{\pgfqpoint{9.713552in}{3.637254in}}%
\pgfpathlineto{\pgfqpoint{9.728175in}{3.636572in}}%
\pgfpathlineto{\pgfqpoint{9.742797in}{3.635918in}}%
\pgfpathlineto{\pgfqpoint{9.757419in}{3.635291in}}%
\pgfpathlineto{\pgfqpoint{9.772041in}{3.634692in}}%
\pgfpathlineto{\pgfqpoint{9.786663in}{3.634120in}}%
\pgfpathlineto{\pgfqpoint{9.801285in}{3.633575in}}%
\pgfpathlineto{\pgfqpoint{9.815908in}{3.633056in}}%
\pgfpathlineto{\pgfqpoint{9.830530in}{3.632564in}}%
\pgfpathlineto{\pgfqpoint{9.845152in}{3.632097in}}%
\pgfpathlineto{\pgfqpoint{9.859774in}{3.631657in}}%
\pgfpathlineto{\pgfqpoint{9.874396in}{3.631242in}}%
\pgfpathlineto{\pgfqpoint{9.889019in}{3.630853in}}%
\pgfpathlineto{\pgfqpoint{9.903641in}{3.630489in}}%
\pgfpathlineto{\pgfqpoint{9.918263in}{3.630150in}}%
\pgfpathlineto{\pgfqpoint{9.932885in}{3.629836in}}%
\pgfpathlineto{\pgfqpoint{9.947507in}{3.629547in}}%
\pgfpathlineto{\pgfqpoint{9.962129in}{3.629283in}}%
\pgfpathlineto{\pgfqpoint{9.976752in}{3.629044in}}%
\pgfpathlineto{\pgfqpoint{9.991374in}{3.628829in}}%
\pgfpathlineto{\pgfqpoint{10.005996in}{3.628639in}}%
\pgfpathlineto{\pgfqpoint{10.020618in}{3.628473in}}%
\pgfpathlineto{\pgfqpoint{10.035240in}{3.628331in}}%
\pgfpathlineto{\pgfqpoint{10.049863in}{3.628214in}}%
\pgfpathlineto{\pgfqpoint{10.064485in}{3.628121in}}%
\pgfpathlineto{\pgfqpoint{10.079107in}{3.628052in}}%
\pgfpathlineto{\pgfqpoint{10.093729in}{3.628007in}}%
\pgfusepath{stroke}%
\end{pgfscope}%
\begin{pgfscope}%
\pgfpathrectangle{\pgfqpoint{8.282041in}{2.849537in}}{\pgfqpoint{1.897959in}{1.372727in}} %
\pgfusepath{clip}%
\pgfsetbuttcap%
\pgfsetmiterjoin%
\definecolor{currentfill}{rgb}{0.000000,0.750000,0.750000}%
\pgfsetfillcolor{currentfill}%
\pgfsetlinewidth{1.003750pt}%
\definecolor{currentstroke}{rgb}{0.000000,0.750000,0.750000}%
\pgfsetstrokecolor{currentstroke}%
\pgfsetdash{}{0pt}%
\pgfsys@defobject{currentmarker}{\pgfqpoint{-0.041667in}{-0.041667in}}{\pgfqpoint{0.041667in}{0.041667in}}{%
\pgfpathmoveto{\pgfqpoint{-0.000000in}{-0.041667in}}%
\pgfpathlineto{\pgfqpoint{0.041667in}{0.041667in}}%
\pgfpathlineto{\pgfqpoint{-0.041667in}{0.041667in}}%
\pgfpathclose%
\pgfusepath{stroke,fill}%
}%
\begin{pgfscope}%
\pgfsys@transformshift{8.368312in}{4.057155in}%
\pgfsys@useobject{currentmarker}{}%
\end{pgfscope}%
\begin{pgfscope}%
\pgfsys@transformshift{8.719244in}{3.794404in}%
\pgfsys@useobject{currentmarker}{}%
\end{pgfscope}%
\begin{pgfscope}%
\pgfsys@transformshift{9.070176in}{3.703102in}%
\pgfsys@useobject{currentmarker}{}%
\end{pgfscope}%
\begin{pgfscope}%
\pgfsys@transformshift{9.421109in}{3.657347in}%
\pgfsys@useobject{currentmarker}{}%
\end{pgfscope}%
\begin{pgfscope}%
\pgfsys@transformshift{9.772041in}{3.634692in}%
\pgfsys@useobject{currentmarker}{}%
\end{pgfscope}%
\end{pgfscope}%
\begin{pgfscope}%
\pgfpathrectangle{\pgfqpoint{8.282041in}{2.849537in}}{\pgfqpoint{1.897959in}{1.372727in}} %
\pgfusepath{clip}%
\pgfsetbuttcap%
\pgfsetroundjoin%
\pgfsetlinewidth{1.505625pt}%
\definecolor{currentstroke}{rgb}{0.000000,0.000000,0.000000}%
\pgfsetstrokecolor{currentstroke}%
\pgfsetdash{{1.500000pt}{2.475000pt}}{0.000000pt}%
\pgfpathmoveto{\pgfqpoint{8.368312in}{4.159867in}}%
\pgfpathlineto{\pgfqpoint{8.382934in}{4.149899in}}%
\pgfpathlineto{\pgfqpoint{8.397556in}{4.140219in}}%
\pgfpathlineto{\pgfqpoint{8.412178in}{4.131817in}}%
\pgfpathlineto{\pgfqpoint{8.426800in}{4.124412in}}%
\pgfpathlineto{\pgfqpoint{8.441423in}{4.117799in}}%
\pgfpathlineto{\pgfqpoint{8.456045in}{4.111826in}}%
\pgfpathlineto{\pgfqpoint{8.470667in}{4.106376in}}%
\pgfpathlineto{\pgfqpoint{8.485289in}{4.101359in}}%
\pgfpathlineto{\pgfqpoint{8.499911in}{4.096706in}}%
\pgfpathlineto{\pgfqpoint{8.514534in}{4.092361in}}%
\pgfpathlineto{\pgfqpoint{8.529156in}{4.088282in}}%
\pgfpathlineto{\pgfqpoint{8.543778in}{4.084432in}}%
\pgfpathlineto{\pgfqpoint{8.558400in}{4.080781in}}%
\pgfpathlineto{\pgfqpoint{8.573022in}{4.077307in}}%
\pgfpathlineto{\pgfqpoint{8.587644in}{4.073990in}}%
\pgfpathlineto{\pgfqpoint{8.602267in}{4.070812in}}%
\pgfpathlineto{\pgfqpoint{8.616889in}{4.067761in}}%
\pgfpathlineto{\pgfqpoint{8.631511in}{4.064824in}}%
\pgfpathlineto{\pgfqpoint{8.646133in}{4.061991in}}%
\pgfpathlineto{\pgfqpoint{8.660755in}{4.059254in}}%
\pgfpathlineto{\pgfqpoint{8.675378in}{4.056605in}}%
\pgfpathlineto{\pgfqpoint{8.690000in}{4.054038in}}%
\pgfpathlineto{\pgfqpoint{8.704622in}{4.051547in}}%
\pgfpathlineto{\pgfqpoint{8.719244in}{4.049126in}}%
\pgfpathlineto{\pgfqpoint{8.733866in}{4.046773in}}%
\pgfpathlineto{\pgfqpoint{8.748488in}{4.044483in}}%
\pgfpathlineto{\pgfqpoint{8.763111in}{4.042252in}}%
\pgfpathlineto{\pgfqpoint{8.777733in}{4.040078in}}%
\pgfpathlineto{\pgfqpoint{8.792355in}{4.037958in}}%
\pgfpathlineto{\pgfqpoint{8.806977in}{4.035888in}}%
\pgfpathlineto{\pgfqpoint{8.821599in}{4.033868in}}%
\pgfpathlineto{\pgfqpoint{8.836222in}{4.031894in}}%
\pgfpathlineto{\pgfqpoint{8.850844in}{4.029965in}}%
\pgfpathlineto{\pgfqpoint{8.865466in}{4.028080in}}%
\pgfpathlineto{\pgfqpoint{8.880088in}{4.026236in}}%
\pgfpathlineto{\pgfqpoint{8.894710in}{4.024433in}}%
\pgfpathlineto{\pgfqpoint{8.909332in}{4.022669in}}%
\pgfpathlineto{\pgfqpoint{8.923955in}{4.020942in}}%
\pgfpathlineto{\pgfqpoint{8.938577in}{4.019252in}}%
\pgfpathlineto{\pgfqpoint{8.953199in}{4.017598in}}%
\pgfpathlineto{\pgfqpoint{8.967821in}{4.015978in}}%
\pgfpathlineto{\pgfqpoint{8.982443in}{4.014392in}}%
\pgfpathlineto{\pgfqpoint{8.997066in}{4.012839in}}%
\pgfpathlineto{\pgfqpoint{9.011688in}{4.011317in}}%
\pgfpathlineto{\pgfqpoint{9.026310in}{4.009827in}}%
\pgfpathlineto{\pgfqpoint{9.040932in}{4.008367in}}%
\pgfpathlineto{\pgfqpoint{9.055554in}{4.006937in}}%
\pgfpathlineto{\pgfqpoint{9.070176in}{4.005536in}}%
\pgfpathlineto{\pgfqpoint{9.084799in}{4.004163in}}%
\pgfpathlineto{\pgfqpoint{9.099421in}{4.002818in}}%
\pgfpathlineto{\pgfqpoint{9.114043in}{4.001501in}}%
\pgfpathlineto{\pgfqpoint{9.128665in}{4.000210in}}%
\pgfpathlineto{\pgfqpoint{9.143287in}{3.998946in}}%
\pgfpathlineto{\pgfqpoint{9.157909in}{3.997707in}}%
\pgfpathlineto{\pgfqpoint{9.172532in}{3.996494in}}%
\pgfpathlineto{\pgfqpoint{9.187154in}{3.995305in}}%
\pgfpathlineto{\pgfqpoint{9.201776in}{3.994142in}}%
\pgfpathlineto{\pgfqpoint{9.216398in}{3.993002in}}%
\pgfpathlineto{\pgfqpoint{9.231020in}{3.991886in}}%
\pgfpathlineto{\pgfqpoint{9.245643in}{3.990793in}}%
\pgfpathlineto{\pgfqpoint{9.260265in}{3.989724in}}%
\pgfpathlineto{\pgfqpoint{9.274887in}{3.988677in}}%
\pgfpathlineto{\pgfqpoint{9.289509in}{3.987652in}}%
\pgfpathlineto{\pgfqpoint{9.304131in}{3.986649in}}%
\pgfpathlineto{\pgfqpoint{9.318753in}{3.985668in}}%
\pgfpathlineto{\pgfqpoint{9.333376in}{3.984708in}}%
\pgfpathlineto{\pgfqpoint{9.347998in}{3.983770in}}%
\pgfpathlineto{\pgfqpoint{9.362620in}{3.982852in}}%
\pgfpathlineto{\pgfqpoint{9.377242in}{3.981955in}}%
\pgfpathlineto{\pgfqpoint{9.391864in}{3.981078in}}%
\pgfpathlineto{\pgfqpoint{9.406487in}{3.980222in}}%
\pgfpathlineto{\pgfqpoint{9.421109in}{3.979385in}}%
\pgfpathlineto{\pgfqpoint{9.435731in}{3.978568in}}%
\pgfpathlineto{\pgfqpoint{9.450353in}{3.977770in}}%
\pgfpathlineto{\pgfqpoint{9.464975in}{3.976992in}}%
\pgfpathlineto{\pgfqpoint{9.479597in}{3.976233in}}%
\pgfpathlineto{\pgfqpoint{9.494220in}{3.975492in}}%
\pgfpathlineto{\pgfqpoint{9.508842in}{3.974770in}}%
\pgfpathlineto{\pgfqpoint{9.523464in}{3.974067in}}%
\pgfpathlineto{\pgfqpoint{9.538086in}{3.973382in}}%
\pgfpathlineto{\pgfqpoint{9.552708in}{3.972715in}}%
\pgfpathlineto{\pgfqpoint{9.567331in}{3.972066in}}%
\pgfpathlineto{\pgfqpoint{9.581953in}{3.971435in}}%
\pgfpathlineto{\pgfqpoint{9.596575in}{3.970821in}}%
\pgfpathlineto{\pgfqpoint{9.611197in}{3.970226in}}%
\pgfpathlineto{\pgfqpoint{9.625819in}{3.969648in}}%
\pgfpathlineto{\pgfqpoint{9.640441in}{3.969087in}}%
\pgfpathlineto{\pgfqpoint{9.655064in}{3.968543in}}%
\pgfpathlineto{\pgfqpoint{9.669686in}{3.968016in}}%
\pgfpathlineto{\pgfqpoint{9.684308in}{3.967506in}}%
\pgfpathlineto{\pgfqpoint{9.698930in}{3.967013in}}%
\pgfpathlineto{\pgfqpoint{9.713552in}{3.966537in}}%
\pgfpathlineto{\pgfqpoint{9.728175in}{3.966078in}}%
\pgfpathlineto{\pgfqpoint{9.742797in}{3.965635in}}%
\pgfpathlineto{\pgfqpoint{9.757419in}{3.965208in}}%
\pgfpathlineto{\pgfqpoint{9.772041in}{3.964798in}}%
\pgfpathlineto{\pgfqpoint{9.786663in}{3.964405in}}%
\pgfpathlineto{\pgfqpoint{9.801285in}{3.964027in}}%
\pgfpathlineto{\pgfqpoint{9.815908in}{3.963666in}}%
\pgfpathlineto{\pgfqpoint{9.830530in}{3.963321in}}%
\pgfpathlineto{\pgfqpoint{9.845152in}{3.962992in}}%
\pgfpathlineto{\pgfqpoint{9.859774in}{3.962679in}}%
\pgfpathlineto{\pgfqpoint{9.874396in}{3.962382in}}%
\pgfpathlineto{\pgfqpoint{9.889019in}{3.962100in}}%
\pgfpathlineto{\pgfqpoint{9.903641in}{3.961835in}}%
\pgfpathlineto{\pgfqpoint{9.918263in}{3.961586in}}%
\pgfpathlineto{\pgfqpoint{9.932885in}{3.961352in}}%
\pgfpathlineto{\pgfqpoint{9.947507in}{3.961134in}}%
\pgfpathlineto{\pgfqpoint{9.962129in}{3.960932in}}%
\pgfpathlineto{\pgfqpoint{9.976752in}{3.960746in}}%
\pgfpathlineto{\pgfqpoint{9.991374in}{3.960576in}}%
\pgfpathlineto{\pgfqpoint{10.005996in}{3.960421in}}%
\pgfpathlineto{\pgfqpoint{10.020618in}{3.960283in}}%
\pgfpathlineto{\pgfqpoint{10.035240in}{3.960160in}}%
\pgfpathlineto{\pgfqpoint{10.049863in}{3.960053in}}%
\pgfpathlineto{\pgfqpoint{10.064485in}{3.959962in}}%
\pgfpathlineto{\pgfqpoint{10.079107in}{3.959887in}}%
\pgfpathlineto{\pgfqpoint{10.093729in}{3.959828in}}%
\pgfusepath{stroke}%
\end{pgfscope}%
\begin{pgfscope}%
\pgfpathrectangle{\pgfqpoint{8.282041in}{2.849537in}}{\pgfqpoint{1.897959in}{1.372727in}} %
\pgfusepath{clip}%
\pgfsetbuttcap%
\pgfsetroundjoin%
\definecolor{currentfill}{rgb}{0.000000,0.000000,0.000000}%
\pgfsetfillcolor{currentfill}%
\pgfsetlinewidth{1.003750pt}%
\definecolor{currentstroke}{rgb}{0.000000,0.000000,0.000000}%
\pgfsetstrokecolor{currentstroke}%
\pgfsetdash{}{0pt}%
\pgfsys@defobject{currentmarker}{\pgfqpoint{-0.041667in}{-0.041667in}}{\pgfqpoint{0.041667in}{0.041667in}}{%
\pgfpathmoveto{\pgfqpoint{-0.041667in}{0.000000in}}%
\pgfpathlineto{\pgfqpoint{0.041667in}{0.000000in}}%
\pgfpathmoveto{\pgfqpoint{0.000000in}{-0.041667in}}%
\pgfpathlineto{\pgfqpoint{0.000000in}{0.041667in}}%
\pgfusepath{stroke,fill}%
}%
\begin{pgfscope}%
\pgfsys@transformshift{8.368312in}{4.159867in}%
\pgfsys@useobject{currentmarker}{}%
\end{pgfscope}%
\begin{pgfscope}%
\pgfsys@transformshift{8.719244in}{4.049126in}%
\pgfsys@useobject{currentmarker}{}%
\end{pgfscope}%
\begin{pgfscope}%
\pgfsys@transformshift{9.070176in}{4.005536in}%
\pgfsys@useobject{currentmarker}{}%
\end{pgfscope}%
\begin{pgfscope}%
\pgfsys@transformshift{9.421109in}{3.979385in}%
\pgfsys@useobject{currentmarker}{}%
\end{pgfscope}%
\begin{pgfscope}%
\pgfsys@transformshift{9.772041in}{3.964798in}%
\pgfsys@useobject{currentmarker}{}%
\end{pgfscope}%
\end{pgfscope}%
\begin{pgfscope}%
\pgfsetrectcap%
\pgfsetmiterjoin%
\pgfsetlinewidth{0.803000pt}%
\definecolor{currentstroke}{rgb}{0.000000,0.000000,0.000000}%
\pgfsetstrokecolor{currentstroke}%
\pgfsetdash{}{0pt}%
\pgfpathmoveto{\pgfqpoint{8.282041in}{2.849537in}}%
\pgfpathlineto{\pgfqpoint{8.282041in}{4.222264in}}%
\pgfusepath{stroke}%
\end{pgfscope}%
\begin{pgfscope}%
\pgfsetrectcap%
\pgfsetmiterjoin%
\pgfsetlinewidth{0.803000pt}%
\definecolor{currentstroke}{rgb}{0.000000,0.000000,0.000000}%
\pgfsetstrokecolor{currentstroke}%
\pgfsetdash{}{0pt}%
\pgfpathmoveto{\pgfqpoint{10.180000in}{2.849537in}}%
\pgfpathlineto{\pgfqpoint{10.180000in}{4.222264in}}%
\pgfusepath{stroke}%
\end{pgfscope}%
\begin{pgfscope}%
\pgfsetrectcap%
\pgfsetmiterjoin%
\pgfsetlinewidth{0.803000pt}%
\definecolor{currentstroke}{rgb}{0.000000,0.000000,0.000000}%
\pgfsetstrokecolor{currentstroke}%
\pgfsetdash{}{0pt}%
\pgfpathmoveto{\pgfqpoint{8.282041in}{2.849537in}}%
\pgfpathlineto{\pgfqpoint{10.180000in}{2.849537in}}%
\pgfusepath{stroke}%
\end{pgfscope}%
\begin{pgfscope}%
\pgfsetrectcap%
\pgfsetmiterjoin%
\pgfsetlinewidth{0.803000pt}%
\definecolor{currentstroke}{rgb}{0.000000,0.000000,0.000000}%
\pgfsetstrokecolor{currentstroke}%
\pgfsetdash{}{0pt}%
\pgfpathmoveto{\pgfqpoint{8.282041in}{4.222264in}}%
\pgfpathlineto{\pgfqpoint{10.180000in}{4.222264in}}%
\pgfusepath{stroke}%
\end{pgfscope}%
\begin{pgfscope}%
\pgfsetbuttcap%
\pgfsetmiterjoin%
\definecolor{currentfill}{rgb}{1.000000,1.000000,1.000000}%
\pgfsetfillcolor{currentfill}%
\pgfsetlinewidth{0.000000pt}%
\definecolor{currentstroke}{rgb}{0.000000,0.000000,0.000000}%
\pgfsetstrokecolor{currentstroke}%
\pgfsetstrokeopacity{0.000000}%
\pgfsetdash{}{0pt}%
\pgfpathmoveto{\pgfqpoint{0.880000in}{0.790446in}}%
\pgfpathlineto{\pgfqpoint{2.777959in}{0.790446in}}%
\pgfpathlineto{\pgfqpoint{2.777959in}{2.163173in}}%
\pgfpathlineto{\pgfqpoint{0.880000in}{2.163173in}}%
\pgfpathclose%
\pgfusepath{fill}%
\end{pgfscope}%
\begin{pgfscope}%
\pgfsetbuttcap%
\pgfsetroundjoin%
\definecolor{currentfill}{rgb}{0.000000,0.000000,0.000000}%
\pgfsetfillcolor{currentfill}%
\pgfsetlinewidth{0.803000pt}%
\definecolor{currentstroke}{rgb}{0.000000,0.000000,0.000000}%
\pgfsetstrokecolor{currentstroke}%
\pgfsetdash{}{0pt}%
\pgfsys@defobject{currentmarker}{\pgfqpoint{0.000000in}{-0.048611in}}{\pgfqpoint{0.000000in}{0.000000in}}{%
\pgfpathmoveto{\pgfqpoint{0.000000in}{0.000000in}}%
\pgfpathlineto{\pgfqpoint{0.000000in}{-0.048611in}}%
\pgfusepath{stroke,fill}%
}%
\begin{pgfscope}%
\pgfsys@transformshift{0.896456in}{0.790446in}%
\pgfsys@useobject{currentmarker}{}%
\end{pgfscope}%
\end{pgfscope}%
\begin{pgfscope}%
\pgftext[x=0.896456in,y=0.693224in,,top]{\rmfamily\fontsize{10.000000}{12.000000}\selectfont \(\displaystyle 0.0\)}%
\end{pgfscope}%
\begin{pgfscope}%
\pgfsetbuttcap%
\pgfsetroundjoin%
\definecolor{currentfill}{rgb}{0.000000,0.000000,0.000000}%
\pgfsetfillcolor{currentfill}%
\pgfsetlinewidth{0.803000pt}%
\definecolor{currentstroke}{rgb}{0.000000,0.000000,0.000000}%
\pgfsetstrokecolor{currentstroke}%
\pgfsetdash{}{0pt}%
\pgfsys@defobject{currentmarker}{\pgfqpoint{0.000000in}{-0.048611in}}{\pgfqpoint{0.000000in}{0.000000in}}{%
\pgfpathmoveto{\pgfqpoint{0.000000in}{0.000000in}}%
\pgfpathlineto{\pgfqpoint{0.000000in}{-0.048611in}}%
\pgfusepath{stroke,fill}%
}%
\begin{pgfscope}%
\pgfsys@transformshift{1.494867in}{0.790446in}%
\pgfsys@useobject{currentmarker}{}%
\end{pgfscope}%
\end{pgfscope}%
\begin{pgfscope}%
\pgftext[x=1.494867in,y=0.693224in,,top]{\rmfamily\fontsize{10.000000}{12.000000}\selectfont \(\displaystyle 0.5\)}%
\end{pgfscope}%
\begin{pgfscope}%
\pgfsetbuttcap%
\pgfsetroundjoin%
\definecolor{currentfill}{rgb}{0.000000,0.000000,0.000000}%
\pgfsetfillcolor{currentfill}%
\pgfsetlinewidth{0.803000pt}%
\definecolor{currentstroke}{rgb}{0.000000,0.000000,0.000000}%
\pgfsetstrokecolor{currentstroke}%
\pgfsetdash{}{0pt}%
\pgfsys@defobject{currentmarker}{\pgfqpoint{0.000000in}{-0.048611in}}{\pgfqpoint{0.000000in}{0.000000in}}{%
\pgfpathmoveto{\pgfqpoint{0.000000in}{0.000000in}}%
\pgfpathlineto{\pgfqpoint{0.000000in}{-0.048611in}}%
\pgfusepath{stroke,fill}%
}%
\begin{pgfscope}%
\pgfsys@transformshift{2.093278in}{0.790446in}%
\pgfsys@useobject{currentmarker}{}%
\end{pgfscope}%
\end{pgfscope}%
\begin{pgfscope}%
\pgftext[x=2.093278in,y=0.693224in,,top]{\rmfamily\fontsize{10.000000}{12.000000}\selectfont \(\displaystyle 1.0\)}%
\end{pgfscope}%
\begin{pgfscope}%
\pgfsetbuttcap%
\pgfsetroundjoin%
\definecolor{currentfill}{rgb}{0.000000,0.000000,0.000000}%
\pgfsetfillcolor{currentfill}%
\pgfsetlinewidth{0.803000pt}%
\definecolor{currentstroke}{rgb}{0.000000,0.000000,0.000000}%
\pgfsetstrokecolor{currentstroke}%
\pgfsetdash{}{0pt}%
\pgfsys@defobject{currentmarker}{\pgfqpoint{0.000000in}{-0.048611in}}{\pgfqpoint{0.000000in}{0.000000in}}{%
\pgfpathmoveto{\pgfqpoint{0.000000in}{0.000000in}}%
\pgfpathlineto{\pgfqpoint{0.000000in}{-0.048611in}}%
\pgfusepath{stroke,fill}%
}%
\begin{pgfscope}%
\pgfsys@transformshift{2.691688in}{0.790446in}%
\pgfsys@useobject{currentmarker}{}%
\end{pgfscope}%
\end{pgfscope}%
\begin{pgfscope}%
\pgftext[x=2.691688in,y=0.693224in,,top]{\rmfamily\fontsize{10.000000}{12.000000}\selectfont \(\displaystyle 1.5\)}%
\end{pgfscope}%
\begin{pgfscope}%
\pgfsetbuttcap%
\pgfsetroundjoin%
\definecolor{currentfill}{rgb}{0.000000,0.000000,0.000000}%
\pgfsetfillcolor{currentfill}%
\pgfsetlinewidth{0.803000pt}%
\definecolor{currentstroke}{rgb}{0.000000,0.000000,0.000000}%
\pgfsetstrokecolor{currentstroke}%
\pgfsetdash{}{0pt}%
\pgfsys@defobject{currentmarker}{\pgfqpoint{-0.048611in}{0.000000in}}{\pgfqpoint{0.000000in}{0.000000in}}{%
\pgfpathmoveto{\pgfqpoint{0.000000in}{0.000000in}}%
\pgfpathlineto{\pgfqpoint{-0.048611in}{0.000000in}}%
\pgfusepath{stroke,fill}%
}%
\begin{pgfscope}%
\pgfsys@transformshift{0.880000in}{1.236240in}%
\pgfsys@useobject{currentmarker}{}%
\end{pgfscope}%
\end{pgfscope}%
\begin{pgfscope}%
\pgftext[x=0.494775in,y=1.183479in,left,base]{\rmfamily\fontsize{10.000000}{12.000000}\selectfont \(\displaystyle 10^{-8}\)}%
\end{pgfscope}%
\begin{pgfscope}%
\pgfsetbuttcap%
\pgfsetroundjoin%
\definecolor{currentfill}{rgb}{0.000000,0.000000,0.000000}%
\pgfsetfillcolor{currentfill}%
\pgfsetlinewidth{0.803000pt}%
\definecolor{currentstroke}{rgb}{0.000000,0.000000,0.000000}%
\pgfsetstrokecolor{currentstroke}%
\pgfsetdash{}{0pt}%
\pgfsys@defobject{currentmarker}{\pgfqpoint{-0.048611in}{0.000000in}}{\pgfqpoint{0.000000in}{0.000000in}}{%
\pgfpathmoveto{\pgfqpoint{0.000000in}{0.000000in}}%
\pgfpathlineto{\pgfqpoint{-0.048611in}{0.000000in}}%
\pgfusepath{stroke,fill}%
}%
\begin{pgfscope}%
\pgfsys@transformshift{0.880000in}{1.814659in}%
\pgfsys@useobject{currentmarker}{}%
\end{pgfscope}%
\end{pgfscope}%
\begin{pgfscope}%
\pgftext[x=0.494775in,y=1.761897in,left,base]{\rmfamily\fontsize{10.000000}{12.000000}\selectfont \(\displaystyle 10^{-6}\)}%
\end{pgfscope}%
\begin{pgfscope}%
\pgfpathrectangle{\pgfqpoint{0.880000in}{0.790446in}}{\pgfqpoint{1.897959in}{1.372727in}} %
\pgfusepath{clip}%
\pgfsetbuttcap%
\pgfsetroundjoin%
\pgfsetlinewidth{1.505625pt}%
\definecolor{currentstroke}{rgb}{1.000000,0.000000,0.000000}%
\pgfsetstrokecolor{currentstroke}%
\pgfsetdash{{5.550000pt}{2.400000pt}}{0.000000pt}%
\pgfpathmoveto{\pgfqpoint{0.966271in}{2.043665in}}%
\pgfpathlineto{\pgfqpoint{1.095927in}{2.017278in}}%
\pgfpathlineto{\pgfqpoint{1.205635in}{1.997116in}}%
\pgfpathlineto{\pgfqpoint{1.315344in}{1.979177in}}%
\pgfpathlineto{\pgfqpoint{1.425052in}{1.963437in}}%
\pgfpathlineto{\pgfqpoint{1.544735in}{1.948600in}}%
\pgfpathlineto{\pgfqpoint{1.664417in}{1.935959in}}%
\pgfpathlineto{\pgfqpoint{1.794072in}{1.924464in}}%
\pgfpathlineto{\pgfqpoint{1.933701in}{1.914344in}}%
\pgfpathlineto{\pgfqpoint{2.083304in}{1.905798in}}%
\pgfpathlineto{\pgfqpoint{2.242880in}{1.899013in}}%
\pgfpathlineto{\pgfqpoint{2.402456in}{1.894406in}}%
\pgfpathlineto{\pgfqpoint{2.572006in}{1.891722in}}%
\pgfpathlineto{\pgfqpoint{2.691688in}{1.891138in}}%
\pgfpathlineto{\pgfqpoint{2.691688in}{1.891138in}}%
\pgfusepath{stroke}%
\end{pgfscope}%
\begin{pgfscope}%
\pgfpathrectangle{\pgfqpoint{0.880000in}{0.790446in}}{\pgfqpoint{1.897959in}{1.372727in}} %
\pgfusepath{clip}%
\pgfsetbuttcap%
\pgfsetmiterjoin%
\definecolor{currentfill}{rgb}{1.000000,0.000000,0.000000}%
\pgfsetfillcolor{currentfill}%
\pgfsetlinewidth{1.003750pt}%
\definecolor{currentstroke}{rgb}{1.000000,0.000000,0.000000}%
\pgfsetstrokecolor{currentstroke}%
\pgfsetdash{}{0pt}%
\pgfsys@defobject{currentmarker}{\pgfqpoint{-0.041667in}{-0.041667in}}{\pgfqpoint{0.041667in}{0.041667in}}{%
\pgfpathmoveto{\pgfqpoint{-0.041667in}{-0.041667in}}%
\pgfpathlineto{\pgfqpoint{0.041667in}{-0.041667in}}%
\pgfpathlineto{\pgfqpoint{0.041667in}{0.041667in}}%
\pgfpathlineto{\pgfqpoint{-0.041667in}{0.041667in}}%
\pgfpathclose%
\pgfusepath{stroke,fill}%
}%
\begin{pgfscope}%
\pgfsys@transformshift{0.966271in}{2.043665in}%
\pgfsys@useobject{currentmarker}{}%
\end{pgfscope}%
\begin{pgfscope}%
\pgfsys@transformshift{1.315344in}{1.979177in}%
\pgfsys@useobject{currentmarker}{}%
\end{pgfscope}%
\begin{pgfscope}%
\pgfsys@transformshift{1.664417in}{1.935959in}%
\pgfsys@useobject{currentmarker}{}%
\end{pgfscope}%
\begin{pgfscope}%
\pgfsys@transformshift{2.013490in}{1.909508in}%
\pgfsys@useobject{currentmarker}{}%
\end{pgfscope}%
\begin{pgfscope}%
\pgfsys@transformshift{2.362562in}{1.895363in}%
\pgfsys@useobject{currentmarker}{}%
\end{pgfscope}%
\end{pgfscope}%
\begin{pgfscope}%
\pgfpathrectangle{\pgfqpoint{0.880000in}{0.790446in}}{\pgfqpoint{1.897959in}{1.372727in}} %
\pgfusepath{clip}%
\pgfsetrectcap%
\pgfsetroundjoin%
\pgfsetlinewidth{1.505625pt}%
\definecolor{currentstroke}{rgb}{0.000000,0.000000,1.000000}%
\pgfsetstrokecolor{currentstroke}%
\pgfsetdash{}{0pt}%
\pgfpathmoveto{\pgfqpoint{0.966271in}{1.734005in}}%
\pgfpathlineto{\pgfqpoint{1.016138in}{1.720666in}}%
\pgfpathlineto{\pgfqpoint{1.095927in}{1.702318in}}%
\pgfpathlineto{\pgfqpoint{1.215609in}{1.677315in}}%
\pgfpathlineto{\pgfqpoint{1.415079in}{1.638292in}}%
\pgfpathlineto{\pgfqpoint{1.734231in}{1.575956in}}%
\pgfpathlineto{\pgfqpoint{1.873860in}{1.546401in}}%
\pgfpathlineto{\pgfqpoint{1.983569in}{1.521106in}}%
\pgfpathlineto{\pgfqpoint{2.073331in}{1.498399in}}%
\pgfpathlineto{\pgfqpoint{2.153119in}{1.476099in}}%
\pgfpathlineto{\pgfqpoint{2.222933in}{1.454369in}}%
\pgfpathlineto{\pgfqpoint{2.282774in}{1.433536in}}%
\pgfpathlineto{\pgfqpoint{2.332642in}{1.414117in}}%
\pgfpathlineto{\pgfqpoint{2.382509in}{1.392219in}}%
\pgfpathlineto{\pgfqpoint{2.422404in}{1.372344in}}%
\pgfpathlineto{\pgfqpoint{2.462298in}{1.349657in}}%
\pgfpathlineto{\pgfqpoint{2.492218in}{1.330185in}}%
\pgfpathlineto{\pgfqpoint{2.522139in}{1.307870in}}%
\pgfpathlineto{\pgfqpoint{2.542086in}{1.290914in}}%
\pgfpathlineto{\pgfqpoint{2.562033in}{1.271765in}}%
\pgfpathlineto{\pgfqpoint{2.581980in}{1.249727in}}%
\pgfpathlineto{\pgfqpoint{2.601927in}{1.223709in}}%
\pgfpathlineto{\pgfqpoint{2.621874in}{1.191850in}}%
\pgfpathlineto{\pgfqpoint{2.631847in}{1.172730in}}%
\pgfpathlineto{\pgfqpoint{2.641821in}{1.150563in}}%
\pgfpathlineto{\pgfqpoint{2.651794in}{1.124145in}}%
\pgfpathlineto{\pgfqpoint{2.661768in}{1.091381in}}%
\pgfpathlineto{\pgfqpoint{2.671741in}{1.048081in}}%
\pgfpathlineto{\pgfqpoint{2.681715in}{0.983699in}}%
\pgfpathlineto{\pgfqpoint{2.691688in}{0.852843in}}%
\pgfpathlineto{\pgfqpoint{2.691688in}{0.852843in}}%
\pgfusepath{stroke}%
\end{pgfscope}%
\begin{pgfscope}%
\pgfpathrectangle{\pgfqpoint{0.880000in}{0.790446in}}{\pgfqpoint{1.897959in}{1.372727in}} %
\pgfusepath{clip}%
\pgfsetbuttcap%
\pgfsetroundjoin%
\definecolor{currentfill}{rgb}{0.000000,0.000000,1.000000}%
\pgfsetfillcolor{currentfill}%
\pgfsetlinewidth{1.003750pt}%
\definecolor{currentstroke}{rgb}{0.000000,0.000000,1.000000}%
\pgfsetstrokecolor{currentstroke}%
\pgfsetdash{}{0pt}%
\pgfsys@defobject{currentmarker}{\pgfqpoint{-0.041667in}{-0.041667in}}{\pgfqpoint{0.041667in}{0.041667in}}{%
\pgfpathmoveto{\pgfqpoint{0.000000in}{-0.041667in}}%
\pgfpathcurveto{\pgfqpoint{0.011050in}{-0.041667in}}{\pgfqpoint{0.021649in}{-0.037276in}}{\pgfqpoint{0.029463in}{-0.029463in}}%
\pgfpathcurveto{\pgfqpoint{0.037276in}{-0.021649in}}{\pgfqpoint{0.041667in}{-0.011050in}}{\pgfqpoint{0.041667in}{0.000000in}}%
\pgfpathcurveto{\pgfqpoint{0.041667in}{0.011050in}}{\pgfqpoint{0.037276in}{0.021649in}}{\pgfqpoint{0.029463in}{0.029463in}}%
\pgfpathcurveto{\pgfqpoint{0.021649in}{0.037276in}}{\pgfqpoint{0.011050in}{0.041667in}}{\pgfqpoint{0.000000in}{0.041667in}}%
\pgfpathcurveto{\pgfqpoint{-0.011050in}{0.041667in}}{\pgfqpoint{-0.021649in}{0.037276in}}{\pgfqpoint{-0.029463in}{0.029463in}}%
\pgfpathcurveto{\pgfqpoint{-0.037276in}{0.021649in}}{\pgfqpoint{-0.041667in}{0.011050in}}{\pgfqpoint{-0.041667in}{0.000000in}}%
\pgfpathcurveto{\pgfqpoint{-0.041667in}{-0.011050in}}{\pgfqpoint{-0.037276in}{-0.021649in}}{\pgfqpoint{-0.029463in}{-0.029463in}}%
\pgfpathcurveto{\pgfqpoint{-0.021649in}{-0.037276in}}{\pgfqpoint{-0.011050in}{-0.041667in}}{\pgfqpoint{0.000000in}{-0.041667in}}%
\pgfpathclose%
\pgfusepath{stroke,fill}%
}%
\begin{pgfscope}%
\pgfsys@transformshift{0.966271in}{1.734005in}%
\pgfsys@useobject{currentmarker}{}%
\end{pgfscope}%
\begin{pgfscope}%
\pgfsys@transformshift{1.315344in}{1.657568in}%
\pgfsys@useobject{currentmarker}{}%
\end{pgfscope}%
\begin{pgfscope}%
\pgfsys@transformshift{1.664417in}{1.589995in}%
\pgfsys@useobject{currentmarker}{}%
\end{pgfscope}%
\begin{pgfscope}%
\pgfsys@transformshift{2.013490in}{1.513772in}%
\pgfsys@useobject{currentmarker}{}%
\end{pgfscope}%
\begin{pgfscope}%
\pgfsys@transformshift{2.362562in}{1.401323in}%
\pgfsys@useobject{currentmarker}{}%
\end{pgfscope}%
\end{pgfscope}%
\begin{pgfscope}%
\pgfpathrectangle{\pgfqpoint{0.880000in}{0.790446in}}{\pgfqpoint{1.897959in}{1.372727in}} %
\pgfusepath{clip}%
\pgfsetbuttcap%
\pgfsetroundjoin%
\pgfsetlinewidth{1.505625pt}%
\definecolor{currentstroke}{rgb}{0.000000,0.750000,0.750000}%
\pgfsetstrokecolor{currentstroke}%
\pgfsetdash{{9.600000pt}{2.400000pt}{1.500000pt}{2.400000pt}}{0.000000pt}%
\pgfpathmoveto{\pgfqpoint{0.966271in}{1.956604in}}%
\pgfpathlineto{\pgfqpoint{0.986218in}{1.907845in}}%
\pgfpathlineto{\pgfqpoint{1.006165in}{1.868227in}}%
\pgfpathlineto{\pgfqpoint{1.026112in}{1.834967in}}%
\pgfpathlineto{\pgfqpoint{1.046059in}{1.806388in}}%
\pgfpathlineto{\pgfqpoint{1.066006in}{1.781402in}}%
\pgfpathlineto{\pgfqpoint{1.085953in}{1.759260in}}%
\pgfpathlineto{\pgfqpoint{1.115874in}{1.730245in}}%
\pgfpathlineto{\pgfqpoint{1.145794in}{1.705181in}}%
\pgfpathlineto{\pgfqpoint{1.175715in}{1.683209in}}%
\pgfpathlineto{\pgfqpoint{1.205635in}{1.663721in}}%
\pgfpathlineto{\pgfqpoint{1.245529in}{1.640850in}}%
\pgfpathlineto{\pgfqpoint{1.285423in}{1.620847in}}%
\pgfpathlineto{\pgfqpoint{1.325317in}{1.603165in}}%
\pgfpathlineto{\pgfqpoint{1.375185in}{1.583715in}}%
\pgfpathlineto{\pgfqpoint{1.425052in}{1.566678in}}%
\pgfpathlineto{\pgfqpoint{1.484893in}{1.548818in}}%
\pgfpathlineto{\pgfqpoint{1.544735in}{1.533271in}}%
\pgfpathlineto{\pgfqpoint{1.614549in}{1.517524in}}%
\pgfpathlineto{\pgfqpoint{1.694337in}{1.502114in}}%
\pgfpathlineto{\pgfqpoint{1.774125in}{1.488976in}}%
\pgfpathlineto{\pgfqpoint{1.863887in}{1.476445in}}%
\pgfpathlineto{\pgfqpoint{1.963622in}{1.464860in}}%
\pgfpathlineto{\pgfqpoint{2.073331in}{1.454509in}}%
\pgfpathlineto{\pgfqpoint{2.193013in}{1.445650in}}%
\pgfpathlineto{\pgfqpoint{2.312695in}{1.438994in}}%
\pgfpathlineto{\pgfqpoint{2.442351in}{1.433993in}}%
\pgfpathlineto{\pgfqpoint{2.581980in}{1.430956in}}%
\pgfpathlineto{\pgfqpoint{2.691688in}{1.430187in}}%
\pgfpathlineto{\pgfqpoint{2.691688in}{1.430187in}}%
\pgfusepath{stroke}%
\end{pgfscope}%
\begin{pgfscope}%
\pgfpathrectangle{\pgfqpoint{0.880000in}{0.790446in}}{\pgfqpoint{1.897959in}{1.372727in}} %
\pgfusepath{clip}%
\pgfsetbuttcap%
\pgfsetmiterjoin%
\definecolor{currentfill}{rgb}{0.000000,0.750000,0.750000}%
\pgfsetfillcolor{currentfill}%
\pgfsetlinewidth{1.003750pt}%
\definecolor{currentstroke}{rgb}{0.000000,0.750000,0.750000}%
\pgfsetstrokecolor{currentstroke}%
\pgfsetdash{}{0pt}%
\pgfsys@defobject{currentmarker}{\pgfqpoint{-0.041667in}{-0.041667in}}{\pgfqpoint{0.041667in}{0.041667in}}{%
\pgfpathmoveto{\pgfqpoint{-0.000000in}{-0.041667in}}%
\pgfpathlineto{\pgfqpoint{0.041667in}{0.041667in}}%
\pgfpathlineto{\pgfqpoint{-0.041667in}{0.041667in}}%
\pgfpathclose%
\pgfusepath{stroke,fill}%
}%
\begin{pgfscope}%
\pgfsys@transformshift{0.966271in}{1.956604in}%
\pgfsys@useobject{currentmarker}{}%
\end{pgfscope}%
\begin{pgfscope}%
\pgfsys@transformshift{1.315344in}{1.607392in}%
\pgfsys@useobject{currentmarker}{}%
\end{pgfscope}%
\begin{pgfscope}%
\pgfsys@transformshift{1.664417in}{1.507602in}%
\pgfsys@useobject{currentmarker}{}%
\end{pgfscope}%
\begin{pgfscope}%
\pgfsys@transformshift{2.013490in}{1.459868in}%
\pgfsys@useobject{currentmarker}{}%
\end{pgfscope}%
\begin{pgfscope}%
\pgfsys@transformshift{2.362562in}{1.436810in}%
\pgfsys@useobject{currentmarker}{}%
\end{pgfscope}%
\end{pgfscope}%
\begin{pgfscope}%
\pgfpathrectangle{\pgfqpoint{0.880000in}{0.790446in}}{\pgfqpoint{1.897959in}{1.372727in}} %
\pgfusepath{clip}%
\pgfsetbuttcap%
\pgfsetroundjoin%
\pgfsetlinewidth{1.505625pt}%
\definecolor{currentstroke}{rgb}{0.000000,0.000000,0.000000}%
\pgfsetstrokecolor{currentstroke}%
\pgfsetdash{{1.500000pt}{2.475000pt}}{0.000000pt}%
\pgfpathmoveto{\pgfqpoint{0.966271in}{2.100776in}}%
\pgfpathlineto{\pgfqpoint{0.986218in}{2.085041in}}%
\pgfpathlineto{\pgfqpoint{1.006165in}{2.072915in}}%
\pgfpathlineto{\pgfqpoint{1.026112in}{2.063490in}}%
\pgfpathlineto{\pgfqpoint{1.056032in}{2.052227in}}%
\pgfpathlineto{\pgfqpoint{1.095927in}{2.040172in}}%
\pgfpathlineto{\pgfqpoint{1.145794in}{2.027637in}}%
\pgfpathlineto{\pgfqpoint{1.215609in}{2.012678in}}%
\pgfpathlineto{\pgfqpoint{1.295397in}{1.997853in}}%
\pgfpathlineto{\pgfqpoint{1.395132in}{1.981701in}}%
\pgfpathlineto{\pgfqpoint{1.504840in}{1.966318in}}%
\pgfpathlineto{\pgfqpoint{1.624523in}{1.951885in}}%
\pgfpathlineto{\pgfqpoint{1.754178in}{1.938595in}}%
\pgfpathlineto{\pgfqpoint{1.893807in}{1.926634in}}%
\pgfpathlineto{\pgfqpoint{2.043410in}{1.916182in}}%
\pgfpathlineto{\pgfqpoint{2.193013in}{1.907915in}}%
\pgfpathlineto{\pgfqpoint{2.352589in}{1.901300in}}%
\pgfpathlineto{\pgfqpoint{2.512165in}{1.896831in}}%
\pgfpathlineto{\pgfqpoint{2.671741in}{1.894473in}}%
\pgfpathlineto{\pgfqpoint{2.691688in}{1.894345in}}%
\pgfpathlineto{\pgfqpoint{2.691688in}{1.894345in}}%
\pgfusepath{stroke}%
\end{pgfscope}%
\begin{pgfscope}%
\pgfpathrectangle{\pgfqpoint{0.880000in}{0.790446in}}{\pgfqpoint{1.897959in}{1.372727in}} %
\pgfusepath{clip}%
\pgfsetbuttcap%
\pgfsetroundjoin%
\definecolor{currentfill}{rgb}{0.000000,0.000000,0.000000}%
\pgfsetfillcolor{currentfill}%
\pgfsetlinewidth{1.003750pt}%
\definecolor{currentstroke}{rgb}{0.000000,0.000000,0.000000}%
\pgfsetstrokecolor{currentstroke}%
\pgfsetdash{}{0pt}%
\pgfsys@defobject{currentmarker}{\pgfqpoint{-0.041667in}{-0.041667in}}{\pgfqpoint{0.041667in}{0.041667in}}{%
\pgfpathmoveto{\pgfqpoint{-0.041667in}{0.000000in}}%
\pgfpathlineto{\pgfqpoint{0.041667in}{0.000000in}}%
\pgfpathmoveto{\pgfqpoint{0.000000in}{-0.041667in}}%
\pgfpathlineto{\pgfqpoint{0.000000in}{0.041667in}}%
\pgfusepath{stroke,fill}%
}%
\begin{pgfscope}%
\pgfsys@transformshift{0.966271in}{2.100776in}%
\pgfsys@useobject{currentmarker}{}%
\end{pgfscope}%
\begin{pgfscope}%
\pgfsys@transformshift{1.315344in}{1.994433in}%
\pgfsys@useobject{currentmarker}{}%
\end{pgfscope}%
\begin{pgfscope}%
\pgfsys@transformshift{1.664417in}{1.947553in}%
\pgfsys@useobject{currentmarker}{}%
\end{pgfscope}%
\begin{pgfscope}%
\pgfsys@transformshift{2.013490in}{1.918091in}%
\pgfsys@useobject{currentmarker}{}%
\end{pgfscope}%
\begin{pgfscope}%
\pgfsys@transformshift{2.362562in}{1.900959in}%
\pgfsys@useobject{currentmarker}{}%
\end{pgfscope}%
\end{pgfscope}%
\begin{pgfscope}%
\pgfsetrectcap%
\pgfsetmiterjoin%
\pgfsetlinewidth{0.803000pt}%
\definecolor{currentstroke}{rgb}{0.000000,0.000000,0.000000}%
\pgfsetstrokecolor{currentstroke}%
\pgfsetdash{}{0pt}%
\pgfpathmoveto{\pgfqpoint{0.880000in}{0.790446in}}%
\pgfpathlineto{\pgfqpoint{0.880000in}{2.163173in}}%
\pgfusepath{stroke}%
\end{pgfscope}%
\begin{pgfscope}%
\pgfsetrectcap%
\pgfsetmiterjoin%
\pgfsetlinewidth{0.803000pt}%
\definecolor{currentstroke}{rgb}{0.000000,0.000000,0.000000}%
\pgfsetstrokecolor{currentstroke}%
\pgfsetdash{}{0pt}%
\pgfpathmoveto{\pgfqpoint{2.777959in}{0.790446in}}%
\pgfpathlineto{\pgfqpoint{2.777959in}{2.163173in}}%
\pgfusepath{stroke}%
\end{pgfscope}%
\begin{pgfscope}%
\pgfsetrectcap%
\pgfsetmiterjoin%
\pgfsetlinewidth{0.803000pt}%
\definecolor{currentstroke}{rgb}{0.000000,0.000000,0.000000}%
\pgfsetstrokecolor{currentstroke}%
\pgfsetdash{}{0pt}%
\pgfpathmoveto{\pgfqpoint{0.880000in}{0.790446in}}%
\pgfpathlineto{\pgfqpoint{2.777959in}{0.790446in}}%
\pgfusepath{stroke}%
\end{pgfscope}%
\begin{pgfscope}%
\pgfsetrectcap%
\pgfsetmiterjoin%
\pgfsetlinewidth{0.803000pt}%
\definecolor{currentstroke}{rgb}{0.000000,0.000000,0.000000}%
\pgfsetstrokecolor{currentstroke}%
\pgfsetdash{}{0pt}%
\pgfpathmoveto{\pgfqpoint{0.880000in}{2.163173in}}%
\pgfpathlineto{\pgfqpoint{2.777959in}{2.163173in}}%
\pgfusepath{stroke}%
\end{pgfscope}%
\begin{pgfscope}%
\pgfsetbuttcap%
\pgfsetmiterjoin%
\definecolor{currentfill}{rgb}{1.000000,1.000000,1.000000}%
\pgfsetfillcolor{currentfill}%
\pgfsetlinewidth{0.000000pt}%
\definecolor{currentstroke}{rgb}{0.000000,0.000000,0.000000}%
\pgfsetstrokecolor{currentstroke}%
\pgfsetstrokeopacity{0.000000}%
\pgfsetdash{}{0pt}%
\pgfpathmoveto{\pgfqpoint{3.347347in}{0.790446in}}%
\pgfpathlineto{\pgfqpoint{5.245306in}{0.790446in}}%
\pgfpathlineto{\pgfqpoint{5.245306in}{2.163173in}}%
\pgfpathlineto{\pgfqpoint{3.347347in}{2.163173in}}%
\pgfpathclose%
\pgfusepath{fill}%
\end{pgfscope}%
\begin{pgfscope}%
\pgfsetbuttcap%
\pgfsetroundjoin%
\definecolor{currentfill}{rgb}{0.000000,0.000000,0.000000}%
\pgfsetfillcolor{currentfill}%
\pgfsetlinewidth{0.803000pt}%
\definecolor{currentstroke}{rgb}{0.000000,0.000000,0.000000}%
\pgfsetstrokecolor{currentstroke}%
\pgfsetdash{}{0pt}%
\pgfsys@defobject{currentmarker}{\pgfqpoint{0.000000in}{-0.048611in}}{\pgfqpoint{0.000000in}{0.000000in}}{%
\pgfpathmoveto{\pgfqpoint{0.000000in}{0.000000in}}%
\pgfpathlineto{\pgfqpoint{0.000000in}{-0.048611in}}%
\pgfusepath{stroke,fill}%
}%
\begin{pgfscope}%
\pgfsys@transformshift{3.346953in}{0.790446in}%
\pgfsys@useobject{currentmarker}{}%
\end{pgfscope}%
\end{pgfscope}%
\begin{pgfscope}%
\pgftext[x=3.346953in,y=0.693224in,,top]{\rmfamily\fontsize{10.000000}{12.000000}\selectfont \(\displaystyle 0\)}%
\end{pgfscope}%
\begin{pgfscope}%
\pgfsetbuttcap%
\pgfsetroundjoin%
\definecolor{currentfill}{rgb}{0.000000,0.000000,0.000000}%
\pgfsetfillcolor{currentfill}%
\pgfsetlinewidth{0.803000pt}%
\definecolor{currentstroke}{rgb}{0.000000,0.000000,0.000000}%
\pgfsetstrokecolor{currentstroke}%
\pgfsetdash{}{0pt}%
\pgfsys@defobject{currentmarker}{\pgfqpoint{0.000000in}{-0.048611in}}{\pgfqpoint{0.000000in}{0.000000in}}{%
\pgfpathmoveto{\pgfqpoint{0.000000in}{0.000000in}}%
\pgfpathlineto{\pgfqpoint{0.000000in}{-0.048611in}}%
\pgfusepath{stroke,fill}%
}%
\begin{pgfscope}%
\pgfsys@transformshift{4.292387in}{0.790446in}%
\pgfsys@useobject{currentmarker}{}%
\end{pgfscope}%
\end{pgfscope}%
\begin{pgfscope}%
\pgftext[x=4.292387in,y=0.693224in,,top]{\rmfamily\fontsize{10.000000}{12.000000}\selectfont \(\displaystyle 1\)}%
\end{pgfscope}%
\begin{pgfscope}%
\pgfsetbuttcap%
\pgfsetroundjoin%
\definecolor{currentfill}{rgb}{0.000000,0.000000,0.000000}%
\pgfsetfillcolor{currentfill}%
\pgfsetlinewidth{0.803000pt}%
\definecolor{currentstroke}{rgb}{0.000000,0.000000,0.000000}%
\pgfsetstrokecolor{currentstroke}%
\pgfsetdash{}{0pt}%
\pgfsys@defobject{currentmarker}{\pgfqpoint{0.000000in}{-0.048611in}}{\pgfqpoint{0.000000in}{0.000000in}}{%
\pgfpathmoveto{\pgfqpoint{0.000000in}{0.000000in}}%
\pgfpathlineto{\pgfqpoint{0.000000in}{-0.048611in}}%
\pgfusepath{stroke,fill}%
}%
\begin{pgfscope}%
\pgfsys@transformshift{5.237821in}{0.790446in}%
\pgfsys@useobject{currentmarker}{}%
\end{pgfscope}%
\end{pgfscope}%
\begin{pgfscope}%
\pgftext[x=5.237821in,y=0.693224in,,top]{\rmfamily\fontsize{10.000000}{12.000000}\selectfont \(\displaystyle 2\)}%
\end{pgfscope}%
\begin{pgfscope}%
\pgfsetbuttcap%
\pgfsetroundjoin%
\definecolor{currentfill}{rgb}{0.000000,0.000000,0.000000}%
\pgfsetfillcolor{currentfill}%
\pgfsetlinewidth{0.803000pt}%
\definecolor{currentstroke}{rgb}{0.000000,0.000000,0.000000}%
\pgfsetstrokecolor{currentstroke}%
\pgfsetdash{}{0pt}%
\pgfsys@defobject{currentmarker}{\pgfqpoint{-0.048611in}{0.000000in}}{\pgfqpoint{0.000000in}{0.000000in}}{%
\pgfpathmoveto{\pgfqpoint{0.000000in}{0.000000in}}%
\pgfpathlineto{\pgfqpoint{-0.048611in}{0.000000in}}%
\pgfusepath{stroke,fill}%
}%
\begin{pgfscope}%
\pgfsys@transformshift{3.347347in}{0.954399in}%
\pgfsys@useobject{currentmarker}{}%
\end{pgfscope}%
\end{pgfscope}%
\begin{pgfscope}%
\pgftext[x=2.962122in,y=0.901638in,left,base]{\rmfamily\fontsize{10.000000}{12.000000}\selectfont \(\displaystyle 10^{-7}\)}%
\end{pgfscope}%
\begin{pgfscope}%
\pgfsetbuttcap%
\pgfsetroundjoin%
\definecolor{currentfill}{rgb}{0.000000,0.000000,0.000000}%
\pgfsetfillcolor{currentfill}%
\pgfsetlinewidth{0.803000pt}%
\definecolor{currentstroke}{rgb}{0.000000,0.000000,0.000000}%
\pgfsetstrokecolor{currentstroke}%
\pgfsetdash{}{0pt}%
\pgfsys@defobject{currentmarker}{\pgfqpoint{-0.048611in}{0.000000in}}{\pgfqpoint{0.000000in}{0.000000in}}{%
\pgfpathmoveto{\pgfqpoint{0.000000in}{0.000000in}}%
\pgfpathlineto{\pgfqpoint{-0.048611in}{0.000000in}}%
\pgfusepath{stroke,fill}%
}%
\begin{pgfscope}%
\pgfsys@transformshift{3.347347in}{1.380173in}%
\pgfsys@useobject{currentmarker}{}%
\end{pgfscope}%
\end{pgfscope}%
\begin{pgfscope}%
\pgftext[x=2.962122in,y=1.327412in,left,base]{\rmfamily\fontsize{10.000000}{12.000000}\selectfont \(\displaystyle 10^{-6}\)}%
\end{pgfscope}%
\begin{pgfscope}%
\pgfsetbuttcap%
\pgfsetroundjoin%
\definecolor{currentfill}{rgb}{0.000000,0.000000,0.000000}%
\pgfsetfillcolor{currentfill}%
\pgfsetlinewidth{0.803000pt}%
\definecolor{currentstroke}{rgb}{0.000000,0.000000,0.000000}%
\pgfsetstrokecolor{currentstroke}%
\pgfsetdash{}{0pt}%
\pgfsys@defobject{currentmarker}{\pgfqpoint{-0.048611in}{0.000000in}}{\pgfqpoint{0.000000in}{0.000000in}}{%
\pgfpathmoveto{\pgfqpoint{0.000000in}{0.000000in}}%
\pgfpathlineto{\pgfqpoint{-0.048611in}{0.000000in}}%
\pgfusepath{stroke,fill}%
}%
\begin{pgfscope}%
\pgfsys@transformshift{3.347347in}{1.805947in}%
\pgfsys@useobject{currentmarker}{}%
\end{pgfscope}%
\end{pgfscope}%
\begin{pgfscope}%
\pgftext[x=2.962122in,y=1.753185in,left,base]{\rmfamily\fontsize{10.000000}{12.000000}\selectfont \(\displaystyle 10^{-5}\)}%
\end{pgfscope}%
\begin{pgfscope}%
\pgfsetbuttcap%
\pgfsetroundjoin%
\definecolor{currentfill}{rgb}{0.000000,0.000000,0.000000}%
\pgfsetfillcolor{currentfill}%
\pgfsetlinewidth{0.602250pt}%
\definecolor{currentstroke}{rgb}{0.000000,0.000000,0.000000}%
\pgfsetstrokecolor{currentstroke}%
\pgfsetdash{}{0pt}%
\pgfsys@defobject{currentmarker}{\pgfqpoint{-0.027778in}{0.000000in}}{\pgfqpoint{0.000000in}{0.000000in}}{%
\pgfpathmoveto{\pgfqpoint{0.000000in}{0.000000in}}%
\pgfpathlineto{\pgfqpoint{-0.027778in}{0.000000in}}%
\pgfusepath{stroke,fill}%
}%
\begin{pgfscope}%
\pgfsys@transformshift{3.347347in}{0.826229in}%
\pgfsys@useobject{currentmarker}{}%
\end{pgfscope}%
\end{pgfscope}%
\begin{pgfscope}%
\pgfsetbuttcap%
\pgfsetroundjoin%
\definecolor{currentfill}{rgb}{0.000000,0.000000,0.000000}%
\pgfsetfillcolor{currentfill}%
\pgfsetlinewidth{0.602250pt}%
\definecolor{currentstroke}{rgb}{0.000000,0.000000,0.000000}%
\pgfsetstrokecolor{currentstroke}%
\pgfsetdash{}{0pt}%
\pgfsys@defobject{currentmarker}{\pgfqpoint{-0.027778in}{0.000000in}}{\pgfqpoint{0.000000in}{0.000000in}}{%
\pgfpathmoveto{\pgfqpoint{0.000000in}{0.000000in}}%
\pgfpathlineto{\pgfqpoint{-0.027778in}{0.000000in}}%
\pgfusepath{stroke,fill}%
}%
\begin{pgfscope}%
\pgfsys@transformshift{3.347347in}{0.859942in}%
\pgfsys@useobject{currentmarker}{}%
\end{pgfscope}%
\end{pgfscope}%
\begin{pgfscope}%
\pgfsetbuttcap%
\pgfsetroundjoin%
\definecolor{currentfill}{rgb}{0.000000,0.000000,0.000000}%
\pgfsetfillcolor{currentfill}%
\pgfsetlinewidth{0.602250pt}%
\definecolor{currentstroke}{rgb}{0.000000,0.000000,0.000000}%
\pgfsetstrokecolor{currentstroke}%
\pgfsetdash{}{0pt}%
\pgfsys@defobject{currentmarker}{\pgfqpoint{-0.027778in}{0.000000in}}{\pgfqpoint{0.000000in}{0.000000in}}{%
\pgfpathmoveto{\pgfqpoint{0.000000in}{0.000000in}}%
\pgfpathlineto{\pgfqpoint{-0.027778in}{0.000000in}}%
\pgfusepath{stroke,fill}%
}%
\begin{pgfscope}%
\pgfsys@transformshift{3.347347in}{0.888446in}%
\pgfsys@useobject{currentmarker}{}%
\end{pgfscope}%
\end{pgfscope}%
\begin{pgfscope}%
\pgfsetbuttcap%
\pgfsetroundjoin%
\definecolor{currentfill}{rgb}{0.000000,0.000000,0.000000}%
\pgfsetfillcolor{currentfill}%
\pgfsetlinewidth{0.602250pt}%
\definecolor{currentstroke}{rgb}{0.000000,0.000000,0.000000}%
\pgfsetstrokecolor{currentstroke}%
\pgfsetdash{}{0pt}%
\pgfsys@defobject{currentmarker}{\pgfqpoint{-0.027778in}{0.000000in}}{\pgfqpoint{0.000000in}{0.000000in}}{%
\pgfpathmoveto{\pgfqpoint{0.000000in}{0.000000in}}%
\pgfpathlineto{\pgfqpoint{-0.027778in}{0.000000in}}%
\pgfusepath{stroke,fill}%
}%
\begin{pgfscope}%
\pgfsys@transformshift{3.347347in}{0.913138in}%
\pgfsys@useobject{currentmarker}{}%
\end{pgfscope}%
\end{pgfscope}%
\begin{pgfscope}%
\pgfsetbuttcap%
\pgfsetroundjoin%
\definecolor{currentfill}{rgb}{0.000000,0.000000,0.000000}%
\pgfsetfillcolor{currentfill}%
\pgfsetlinewidth{0.602250pt}%
\definecolor{currentstroke}{rgb}{0.000000,0.000000,0.000000}%
\pgfsetstrokecolor{currentstroke}%
\pgfsetdash{}{0pt}%
\pgfsys@defobject{currentmarker}{\pgfqpoint{-0.027778in}{0.000000in}}{\pgfqpoint{0.000000in}{0.000000in}}{%
\pgfpathmoveto{\pgfqpoint{0.000000in}{0.000000in}}%
\pgfpathlineto{\pgfqpoint{-0.027778in}{0.000000in}}%
\pgfusepath{stroke,fill}%
}%
\begin{pgfscope}%
\pgfsys@transformshift{3.347347in}{0.934917in}%
\pgfsys@useobject{currentmarker}{}%
\end{pgfscope}%
\end{pgfscope}%
\begin{pgfscope}%
\pgfsetbuttcap%
\pgfsetroundjoin%
\definecolor{currentfill}{rgb}{0.000000,0.000000,0.000000}%
\pgfsetfillcolor{currentfill}%
\pgfsetlinewidth{0.602250pt}%
\definecolor{currentstroke}{rgb}{0.000000,0.000000,0.000000}%
\pgfsetstrokecolor{currentstroke}%
\pgfsetdash{}{0pt}%
\pgfsys@defobject{currentmarker}{\pgfqpoint{-0.027778in}{0.000000in}}{\pgfqpoint{0.000000in}{0.000000in}}{%
\pgfpathmoveto{\pgfqpoint{0.000000in}{0.000000in}}%
\pgfpathlineto{\pgfqpoint{-0.027778in}{0.000000in}}%
\pgfusepath{stroke,fill}%
}%
\begin{pgfscope}%
\pgfsys@transformshift{3.347347in}{1.082570in}%
\pgfsys@useobject{currentmarker}{}%
\end{pgfscope}%
\end{pgfscope}%
\begin{pgfscope}%
\pgfsetbuttcap%
\pgfsetroundjoin%
\definecolor{currentfill}{rgb}{0.000000,0.000000,0.000000}%
\pgfsetfillcolor{currentfill}%
\pgfsetlinewidth{0.602250pt}%
\definecolor{currentstroke}{rgb}{0.000000,0.000000,0.000000}%
\pgfsetstrokecolor{currentstroke}%
\pgfsetdash{}{0pt}%
\pgfsys@defobject{currentmarker}{\pgfqpoint{-0.027778in}{0.000000in}}{\pgfqpoint{0.000000in}{0.000000in}}{%
\pgfpathmoveto{\pgfqpoint{0.000000in}{0.000000in}}%
\pgfpathlineto{\pgfqpoint{-0.027778in}{0.000000in}}%
\pgfusepath{stroke,fill}%
}%
\begin{pgfscope}%
\pgfsys@transformshift{3.347347in}{1.157545in}%
\pgfsys@useobject{currentmarker}{}%
\end{pgfscope}%
\end{pgfscope}%
\begin{pgfscope}%
\pgfsetbuttcap%
\pgfsetroundjoin%
\definecolor{currentfill}{rgb}{0.000000,0.000000,0.000000}%
\pgfsetfillcolor{currentfill}%
\pgfsetlinewidth{0.602250pt}%
\definecolor{currentstroke}{rgb}{0.000000,0.000000,0.000000}%
\pgfsetstrokecolor{currentstroke}%
\pgfsetdash{}{0pt}%
\pgfsys@defobject{currentmarker}{\pgfqpoint{-0.027778in}{0.000000in}}{\pgfqpoint{0.000000in}{0.000000in}}{%
\pgfpathmoveto{\pgfqpoint{0.000000in}{0.000000in}}%
\pgfpathlineto{\pgfqpoint{-0.027778in}{0.000000in}}%
\pgfusepath{stroke,fill}%
}%
\begin{pgfscope}%
\pgfsys@transformshift{3.347347in}{1.210741in}%
\pgfsys@useobject{currentmarker}{}%
\end{pgfscope}%
\end{pgfscope}%
\begin{pgfscope}%
\pgfsetbuttcap%
\pgfsetroundjoin%
\definecolor{currentfill}{rgb}{0.000000,0.000000,0.000000}%
\pgfsetfillcolor{currentfill}%
\pgfsetlinewidth{0.602250pt}%
\definecolor{currentstroke}{rgb}{0.000000,0.000000,0.000000}%
\pgfsetstrokecolor{currentstroke}%
\pgfsetdash{}{0pt}%
\pgfsys@defobject{currentmarker}{\pgfqpoint{-0.027778in}{0.000000in}}{\pgfqpoint{0.000000in}{0.000000in}}{%
\pgfpathmoveto{\pgfqpoint{0.000000in}{0.000000in}}%
\pgfpathlineto{\pgfqpoint{-0.027778in}{0.000000in}}%
\pgfusepath{stroke,fill}%
}%
\begin{pgfscope}%
\pgfsys@transformshift{3.347347in}{1.252002in}%
\pgfsys@useobject{currentmarker}{}%
\end{pgfscope}%
\end{pgfscope}%
\begin{pgfscope}%
\pgfsetbuttcap%
\pgfsetroundjoin%
\definecolor{currentfill}{rgb}{0.000000,0.000000,0.000000}%
\pgfsetfillcolor{currentfill}%
\pgfsetlinewidth{0.602250pt}%
\definecolor{currentstroke}{rgb}{0.000000,0.000000,0.000000}%
\pgfsetstrokecolor{currentstroke}%
\pgfsetdash{}{0pt}%
\pgfsys@defobject{currentmarker}{\pgfqpoint{-0.027778in}{0.000000in}}{\pgfqpoint{0.000000in}{0.000000in}}{%
\pgfpathmoveto{\pgfqpoint{0.000000in}{0.000000in}}%
\pgfpathlineto{\pgfqpoint{-0.027778in}{0.000000in}}%
\pgfusepath{stroke,fill}%
}%
\begin{pgfscope}%
\pgfsys@transformshift{3.347347in}{1.285716in}%
\pgfsys@useobject{currentmarker}{}%
\end{pgfscope}%
\end{pgfscope}%
\begin{pgfscope}%
\pgfsetbuttcap%
\pgfsetroundjoin%
\definecolor{currentfill}{rgb}{0.000000,0.000000,0.000000}%
\pgfsetfillcolor{currentfill}%
\pgfsetlinewidth{0.602250pt}%
\definecolor{currentstroke}{rgb}{0.000000,0.000000,0.000000}%
\pgfsetstrokecolor{currentstroke}%
\pgfsetdash{}{0pt}%
\pgfsys@defobject{currentmarker}{\pgfqpoint{-0.027778in}{0.000000in}}{\pgfqpoint{0.000000in}{0.000000in}}{%
\pgfpathmoveto{\pgfqpoint{0.000000in}{0.000000in}}%
\pgfpathlineto{\pgfqpoint{-0.027778in}{0.000000in}}%
\pgfusepath{stroke,fill}%
}%
\begin{pgfscope}%
\pgfsys@transformshift{3.347347in}{1.314220in}%
\pgfsys@useobject{currentmarker}{}%
\end{pgfscope}%
\end{pgfscope}%
\begin{pgfscope}%
\pgfsetbuttcap%
\pgfsetroundjoin%
\definecolor{currentfill}{rgb}{0.000000,0.000000,0.000000}%
\pgfsetfillcolor{currentfill}%
\pgfsetlinewidth{0.602250pt}%
\definecolor{currentstroke}{rgb}{0.000000,0.000000,0.000000}%
\pgfsetstrokecolor{currentstroke}%
\pgfsetdash{}{0pt}%
\pgfsys@defobject{currentmarker}{\pgfqpoint{-0.027778in}{0.000000in}}{\pgfqpoint{0.000000in}{0.000000in}}{%
\pgfpathmoveto{\pgfqpoint{0.000000in}{0.000000in}}%
\pgfpathlineto{\pgfqpoint{-0.027778in}{0.000000in}}%
\pgfusepath{stroke,fill}%
}%
\begin{pgfscope}%
\pgfsys@transformshift{3.347347in}{1.338911in}%
\pgfsys@useobject{currentmarker}{}%
\end{pgfscope}%
\end{pgfscope}%
\begin{pgfscope}%
\pgfsetbuttcap%
\pgfsetroundjoin%
\definecolor{currentfill}{rgb}{0.000000,0.000000,0.000000}%
\pgfsetfillcolor{currentfill}%
\pgfsetlinewidth{0.602250pt}%
\definecolor{currentstroke}{rgb}{0.000000,0.000000,0.000000}%
\pgfsetstrokecolor{currentstroke}%
\pgfsetdash{}{0pt}%
\pgfsys@defobject{currentmarker}{\pgfqpoint{-0.027778in}{0.000000in}}{\pgfqpoint{0.000000in}{0.000000in}}{%
\pgfpathmoveto{\pgfqpoint{0.000000in}{0.000000in}}%
\pgfpathlineto{\pgfqpoint{-0.027778in}{0.000000in}}%
\pgfusepath{stroke,fill}%
}%
\begin{pgfscope}%
\pgfsys@transformshift{3.347347in}{1.360691in}%
\pgfsys@useobject{currentmarker}{}%
\end{pgfscope}%
\end{pgfscope}%
\begin{pgfscope}%
\pgfsetbuttcap%
\pgfsetroundjoin%
\definecolor{currentfill}{rgb}{0.000000,0.000000,0.000000}%
\pgfsetfillcolor{currentfill}%
\pgfsetlinewidth{0.602250pt}%
\definecolor{currentstroke}{rgb}{0.000000,0.000000,0.000000}%
\pgfsetstrokecolor{currentstroke}%
\pgfsetdash{}{0pt}%
\pgfsys@defobject{currentmarker}{\pgfqpoint{-0.027778in}{0.000000in}}{\pgfqpoint{0.000000in}{0.000000in}}{%
\pgfpathmoveto{\pgfqpoint{0.000000in}{0.000000in}}%
\pgfpathlineto{\pgfqpoint{-0.027778in}{0.000000in}}%
\pgfusepath{stroke,fill}%
}%
\begin{pgfscope}%
\pgfsys@transformshift{3.347347in}{1.508344in}%
\pgfsys@useobject{currentmarker}{}%
\end{pgfscope}%
\end{pgfscope}%
\begin{pgfscope}%
\pgfsetbuttcap%
\pgfsetroundjoin%
\definecolor{currentfill}{rgb}{0.000000,0.000000,0.000000}%
\pgfsetfillcolor{currentfill}%
\pgfsetlinewidth{0.602250pt}%
\definecolor{currentstroke}{rgb}{0.000000,0.000000,0.000000}%
\pgfsetstrokecolor{currentstroke}%
\pgfsetdash{}{0pt}%
\pgfsys@defobject{currentmarker}{\pgfqpoint{-0.027778in}{0.000000in}}{\pgfqpoint{0.000000in}{0.000000in}}{%
\pgfpathmoveto{\pgfqpoint{0.000000in}{0.000000in}}%
\pgfpathlineto{\pgfqpoint{-0.027778in}{0.000000in}}%
\pgfusepath{stroke,fill}%
}%
\begin{pgfscope}%
\pgfsys@transformshift{3.347347in}{1.583319in}%
\pgfsys@useobject{currentmarker}{}%
\end{pgfscope}%
\end{pgfscope}%
\begin{pgfscope}%
\pgfsetbuttcap%
\pgfsetroundjoin%
\definecolor{currentfill}{rgb}{0.000000,0.000000,0.000000}%
\pgfsetfillcolor{currentfill}%
\pgfsetlinewidth{0.602250pt}%
\definecolor{currentstroke}{rgb}{0.000000,0.000000,0.000000}%
\pgfsetstrokecolor{currentstroke}%
\pgfsetdash{}{0pt}%
\pgfsys@defobject{currentmarker}{\pgfqpoint{-0.027778in}{0.000000in}}{\pgfqpoint{0.000000in}{0.000000in}}{%
\pgfpathmoveto{\pgfqpoint{0.000000in}{0.000000in}}%
\pgfpathlineto{\pgfqpoint{-0.027778in}{0.000000in}}%
\pgfusepath{stroke,fill}%
}%
\begin{pgfscope}%
\pgfsys@transformshift{3.347347in}{1.636514in}%
\pgfsys@useobject{currentmarker}{}%
\end{pgfscope}%
\end{pgfscope}%
\begin{pgfscope}%
\pgfsetbuttcap%
\pgfsetroundjoin%
\definecolor{currentfill}{rgb}{0.000000,0.000000,0.000000}%
\pgfsetfillcolor{currentfill}%
\pgfsetlinewidth{0.602250pt}%
\definecolor{currentstroke}{rgb}{0.000000,0.000000,0.000000}%
\pgfsetstrokecolor{currentstroke}%
\pgfsetdash{}{0pt}%
\pgfsys@defobject{currentmarker}{\pgfqpoint{-0.027778in}{0.000000in}}{\pgfqpoint{0.000000in}{0.000000in}}{%
\pgfpathmoveto{\pgfqpoint{0.000000in}{0.000000in}}%
\pgfpathlineto{\pgfqpoint{-0.027778in}{0.000000in}}%
\pgfusepath{stroke,fill}%
}%
\begin{pgfscope}%
\pgfsys@transformshift{3.347347in}{1.677776in}%
\pgfsys@useobject{currentmarker}{}%
\end{pgfscope}%
\end{pgfscope}%
\begin{pgfscope}%
\pgfsetbuttcap%
\pgfsetroundjoin%
\definecolor{currentfill}{rgb}{0.000000,0.000000,0.000000}%
\pgfsetfillcolor{currentfill}%
\pgfsetlinewidth{0.602250pt}%
\definecolor{currentstroke}{rgb}{0.000000,0.000000,0.000000}%
\pgfsetstrokecolor{currentstroke}%
\pgfsetdash{}{0pt}%
\pgfsys@defobject{currentmarker}{\pgfqpoint{-0.027778in}{0.000000in}}{\pgfqpoint{0.000000in}{0.000000in}}{%
\pgfpathmoveto{\pgfqpoint{0.000000in}{0.000000in}}%
\pgfpathlineto{\pgfqpoint{-0.027778in}{0.000000in}}%
\pgfusepath{stroke,fill}%
}%
\begin{pgfscope}%
\pgfsys@transformshift{3.347347in}{1.711490in}%
\pgfsys@useobject{currentmarker}{}%
\end{pgfscope}%
\end{pgfscope}%
\begin{pgfscope}%
\pgfsetbuttcap%
\pgfsetroundjoin%
\definecolor{currentfill}{rgb}{0.000000,0.000000,0.000000}%
\pgfsetfillcolor{currentfill}%
\pgfsetlinewidth{0.602250pt}%
\definecolor{currentstroke}{rgb}{0.000000,0.000000,0.000000}%
\pgfsetstrokecolor{currentstroke}%
\pgfsetdash{}{0pt}%
\pgfsys@defobject{currentmarker}{\pgfqpoint{-0.027778in}{0.000000in}}{\pgfqpoint{0.000000in}{0.000000in}}{%
\pgfpathmoveto{\pgfqpoint{0.000000in}{0.000000in}}%
\pgfpathlineto{\pgfqpoint{-0.027778in}{0.000000in}}%
\pgfusepath{stroke,fill}%
}%
\begin{pgfscope}%
\pgfsys@transformshift{3.347347in}{1.739994in}%
\pgfsys@useobject{currentmarker}{}%
\end{pgfscope}%
\end{pgfscope}%
\begin{pgfscope}%
\pgfsetbuttcap%
\pgfsetroundjoin%
\definecolor{currentfill}{rgb}{0.000000,0.000000,0.000000}%
\pgfsetfillcolor{currentfill}%
\pgfsetlinewidth{0.602250pt}%
\definecolor{currentstroke}{rgb}{0.000000,0.000000,0.000000}%
\pgfsetstrokecolor{currentstroke}%
\pgfsetdash{}{0pt}%
\pgfsys@defobject{currentmarker}{\pgfqpoint{-0.027778in}{0.000000in}}{\pgfqpoint{0.000000in}{0.000000in}}{%
\pgfpathmoveto{\pgfqpoint{0.000000in}{0.000000in}}%
\pgfpathlineto{\pgfqpoint{-0.027778in}{0.000000in}}%
\pgfusepath{stroke,fill}%
}%
\begin{pgfscope}%
\pgfsys@transformshift{3.347347in}{1.764685in}%
\pgfsys@useobject{currentmarker}{}%
\end{pgfscope}%
\end{pgfscope}%
\begin{pgfscope}%
\pgfsetbuttcap%
\pgfsetroundjoin%
\definecolor{currentfill}{rgb}{0.000000,0.000000,0.000000}%
\pgfsetfillcolor{currentfill}%
\pgfsetlinewidth{0.602250pt}%
\definecolor{currentstroke}{rgb}{0.000000,0.000000,0.000000}%
\pgfsetstrokecolor{currentstroke}%
\pgfsetdash{}{0pt}%
\pgfsys@defobject{currentmarker}{\pgfqpoint{-0.027778in}{0.000000in}}{\pgfqpoint{0.000000in}{0.000000in}}{%
\pgfpathmoveto{\pgfqpoint{0.000000in}{0.000000in}}%
\pgfpathlineto{\pgfqpoint{-0.027778in}{0.000000in}}%
\pgfusepath{stroke,fill}%
}%
\begin{pgfscope}%
\pgfsys@transformshift{3.347347in}{1.786465in}%
\pgfsys@useobject{currentmarker}{}%
\end{pgfscope}%
\end{pgfscope}%
\begin{pgfscope}%
\pgfsetbuttcap%
\pgfsetroundjoin%
\definecolor{currentfill}{rgb}{0.000000,0.000000,0.000000}%
\pgfsetfillcolor{currentfill}%
\pgfsetlinewidth{0.602250pt}%
\definecolor{currentstroke}{rgb}{0.000000,0.000000,0.000000}%
\pgfsetstrokecolor{currentstroke}%
\pgfsetdash{}{0pt}%
\pgfsys@defobject{currentmarker}{\pgfqpoint{-0.027778in}{0.000000in}}{\pgfqpoint{0.000000in}{0.000000in}}{%
\pgfpathmoveto{\pgfqpoint{0.000000in}{0.000000in}}%
\pgfpathlineto{\pgfqpoint{-0.027778in}{0.000000in}}%
\pgfusepath{stroke,fill}%
}%
\begin{pgfscope}%
\pgfsys@transformshift{3.347347in}{1.934118in}%
\pgfsys@useobject{currentmarker}{}%
\end{pgfscope}%
\end{pgfscope}%
\begin{pgfscope}%
\pgfsetbuttcap%
\pgfsetroundjoin%
\definecolor{currentfill}{rgb}{0.000000,0.000000,0.000000}%
\pgfsetfillcolor{currentfill}%
\pgfsetlinewidth{0.602250pt}%
\definecolor{currentstroke}{rgb}{0.000000,0.000000,0.000000}%
\pgfsetstrokecolor{currentstroke}%
\pgfsetdash{}{0pt}%
\pgfsys@defobject{currentmarker}{\pgfqpoint{-0.027778in}{0.000000in}}{\pgfqpoint{0.000000in}{0.000000in}}{%
\pgfpathmoveto{\pgfqpoint{0.000000in}{0.000000in}}%
\pgfpathlineto{\pgfqpoint{-0.027778in}{0.000000in}}%
\pgfusepath{stroke,fill}%
}%
\begin{pgfscope}%
\pgfsys@transformshift{3.347347in}{2.009093in}%
\pgfsys@useobject{currentmarker}{}%
\end{pgfscope}%
\end{pgfscope}%
\begin{pgfscope}%
\pgfsetbuttcap%
\pgfsetroundjoin%
\definecolor{currentfill}{rgb}{0.000000,0.000000,0.000000}%
\pgfsetfillcolor{currentfill}%
\pgfsetlinewidth{0.602250pt}%
\definecolor{currentstroke}{rgb}{0.000000,0.000000,0.000000}%
\pgfsetstrokecolor{currentstroke}%
\pgfsetdash{}{0pt}%
\pgfsys@defobject{currentmarker}{\pgfqpoint{-0.027778in}{0.000000in}}{\pgfqpoint{0.000000in}{0.000000in}}{%
\pgfpathmoveto{\pgfqpoint{0.000000in}{0.000000in}}%
\pgfpathlineto{\pgfqpoint{-0.027778in}{0.000000in}}%
\pgfusepath{stroke,fill}%
}%
\begin{pgfscope}%
\pgfsys@transformshift{3.347347in}{2.062288in}%
\pgfsys@useobject{currentmarker}{}%
\end{pgfscope}%
\end{pgfscope}%
\begin{pgfscope}%
\pgfsetbuttcap%
\pgfsetroundjoin%
\definecolor{currentfill}{rgb}{0.000000,0.000000,0.000000}%
\pgfsetfillcolor{currentfill}%
\pgfsetlinewidth{0.602250pt}%
\definecolor{currentstroke}{rgb}{0.000000,0.000000,0.000000}%
\pgfsetstrokecolor{currentstroke}%
\pgfsetdash{}{0pt}%
\pgfsys@defobject{currentmarker}{\pgfqpoint{-0.027778in}{0.000000in}}{\pgfqpoint{0.000000in}{0.000000in}}{%
\pgfpathmoveto{\pgfqpoint{0.000000in}{0.000000in}}%
\pgfpathlineto{\pgfqpoint{-0.027778in}{0.000000in}}%
\pgfusepath{stroke,fill}%
}%
\begin{pgfscope}%
\pgfsys@transformshift{3.347347in}{2.103550in}%
\pgfsys@useobject{currentmarker}{}%
\end{pgfscope}%
\end{pgfscope}%
\begin{pgfscope}%
\pgfsetbuttcap%
\pgfsetroundjoin%
\definecolor{currentfill}{rgb}{0.000000,0.000000,0.000000}%
\pgfsetfillcolor{currentfill}%
\pgfsetlinewidth{0.602250pt}%
\definecolor{currentstroke}{rgb}{0.000000,0.000000,0.000000}%
\pgfsetstrokecolor{currentstroke}%
\pgfsetdash{}{0pt}%
\pgfsys@defobject{currentmarker}{\pgfqpoint{-0.027778in}{0.000000in}}{\pgfqpoint{0.000000in}{0.000000in}}{%
\pgfpathmoveto{\pgfqpoint{0.000000in}{0.000000in}}%
\pgfpathlineto{\pgfqpoint{-0.027778in}{0.000000in}}%
\pgfusepath{stroke,fill}%
}%
\begin{pgfscope}%
\pgfsys@transformshift{3.347347in}{2.137263in}%
\pgfsys@useobject{currentmarker}{}%
\end{pgfscope}%
\end{pgfscope}%
\begin{pgfscope}%
\pgfsetbuttcap%
\pgfsetroundjoin%
\definecolor{currentfill}{rgb}{0.000000,0.000000,0.000000}%
\pgfsetfillcolor{currentfill}%
\pgfsetlinewidth{0.602250pt}%
\definecolor{currentstroke}{rgb}{0.000000,0.000000,0.000000}%
\pgfsetstrokecolor{currentstroke}%
\pgfsetdash{}{0pt}%
\pgfsys@defobject{currentmarker}{\pgfqpoint{-0.027778in}{0.000000in}}{\pgfqpoint{0.000000in}{0.000000in}}{%
\pgfpathmoveto{\pgfqpoint{0.000000in}{0.000000in}}%
\pgfpathlineto{\pgfqpoint{-0.027778in}{0.000000in}}%
\pgfusepath{stroke,fill}%
}%
\begin{pgfscope}%
\pgfsys@transformshift{3.347347in}{2.165768in}%
\pgfsys@useobject{currentmarker}{}%
\end{pgfscope}%
\end{pgfscope}%
\begin{pgfscope}%
\pgfpathrectangle{\pgfqpoint{3.347347in}{0.790446in}}{\pgfqpoint{1.897959in}{1.372727in}} %
\pgfusepath{clip}%
\pgfsetbuttcap%
\pgfsetroundjoin%
\pgfsetlinewidth{1.505625pt}%
\definecolor{currentstroke}{rgb}{1.000000,0.000000,0.000000}%
\pgfsetstrokecolor{currentstroke}%
\pgfsetdash{{5.550000pt}{2.400000pt}}{0.000000pt}%
\pgfpathmoveto{\pgfqpoint{3.433618in}{2.022449in}}%
\pgfpathlineto{\pgfqpoint{3.504525in}{1.989368in}}%
\pgfpathlineto{\pgfqpoint{3.575433in}{1.958899in}}%
\pgfpathlineto{\pgfqpoint{3.638462in}{1.934124in}}%
\pgfpathlineto{\pgfqpoint{3.709369in}{1.908785in}}%
\pgfpathlineto{\pgfqpoint{3.780277in}{1.885944in}}%
\pgfpathlineto{\pgfqpoint{3.851185in}{1.865367in}}%
\pgfpathlineto{\pgfqpoint{3.929971in}{1.844868in}}%
\pgfpathlineto{\pgfqpoint{4.008757in}{1.826566in}}%
\pgfpathlineto{\pgfqpoint{4.095422in}{1.808656in}}%
\pgfpathlineto{\pgfqpoint{4.189965in}{1.791434in}}%
\pgfpathlineto{\pgfqpoint{4.292387in}{1.775141in}}%
\pgfpathlineto{\pgfqpoint{4.402688in}{1.759980in}}%
\pgfpathlineto{\pgfqpoint{4.520867in}{1.746116in}}%
\pgfpathlineto{\pgfqpoint{4.646925in}{1.733695in}}%
\pgfpathlineto{\pgfqpoint{4.780862in}{1.722851in}}%
\pgfpathlineto{\pgfqpoint{4.922677in}{1.713718in}}%
\pgfpathlineto{\pgfqpoint{5.072370in}{1.706438in}}%
\pgfpathlineto{\pgfqpoint{5.159035in}{1.703252in}}%
\pgfpathlineto{\pgfqpoint{5.159035in}{1.703252in}}%
\pgfusepath{stroke}%
\end{pgfscope}%
\begin{pgfscope}%
\pgfpathrectangle{\pgfqpoint{3.347347in}{0.790446in}}{\pgfqpoint{1.897959in}{1.372727in}} %
\pgfusepath{clip}%
\pgfsetbuttcap%
\pgfsetmiterjoin%
\definecolor{currentfill}{rgb}{1.000000,0.000000,0.000000}%
\pgfsetfillcolor{currentfill}%
\pgfsetlinewidth{1.003750pt}%
\definecolor{currentstroke}{rgb}{1.000000,0.000000,0.000000}%
\pgfsetstrokecolor{currentstroke}%
\pgfsetdash{}{0pt}%
\pgfsys@defobject{currentmarker}{\pgfqpoint{-0.041667in}{-0.041667in}}{\pgfqpoint{0.041667in}{0.041667in}}{%
\pgfpathmoveto{\pgfqpoint{-0.041667in}{-0.041667in}}%
\pgfpathlineto{\pgfqpoint{0.041667in}{-0.041667in}}%
\pgfpathlineto{\pgfqpoint{0.041667in}{0.041667in}}%
\pgfpathlineto{\pgfqpoint{-0.041667in}{0.041667in}}%
\pgfpathclose%
\pgfusepath{stroke,fill}%
}%
\begin{pgfscope}%
\pgfsys@transformshift{3.433618in}{2.022449in}%
\pgfsys@useobject{currentmarker}{}%
\end{pgfscope}%
\begin{pgfscope}%
\pgfsys@transformshift{3.780277in}{1.885944in}%
\pgfsys@useobject{currentmarker}{}%
\end{pgfscope}%
\begin{pgfscope}%
\pgfsys@transformshift{4.126936in}{1.802663in}%
\pgfsys@useobject{currentmarker}{}%
\end{pgfscope}%
\begin{pgfscope}%
\pgfsys@transformshift{4.473595in}{1.751387in}%
\pgfsys@useobject{currentmarker}{}%
\end{pgfscope}%
\begin{pgfscope}%
\pgfsys@transformshift{4.820255in}{1.720083in}%
\pgfsys@useobject{currentmarker}{}%
\end{pgfscope}%
\end{pgfscope}%
\begin{pgfscope}%
\pgfpathrectangle{\pgfqpoint{3.347347in}{0.790446in}}{\pgfqpoint{1.897959in}{1.372727in}} %
\pgfusepath{clip}%
\pgfsetrectcap%
\pgfsetroundjoin%
\pgfsetlinewidth{1.505625pt}%
\definecolor{currentstroke}{rgb}{0.000000,0.000000,1.000000}%
\pgfsetstrokecolor{currentstroke}%
\pgfsetdash{}{0pt}%
\pgfpathmoveto{\pgfqpoint{3.433618in}{1.430064in}}%
\pgfpathlineto{\pgfqpoint{3.465132in}{1.414454in}}%
\pgfpathlineto{\pgfqpoint{3.512404in}{1.393935in}}%
\pgfpathlineto{\pgfqpoint{3.567554in}{1.372539in}}%
\pgfpathlineto{\pgfqpoint{3.638462in}{1.347570in}}%
\pgfpathlineto{\pgfqpoint{3.725127in}{1.319669in}}%
\pgfpathlineto{\pgfqpoint{3.827549in}{1.289210in}}%
\pgfpathlineto{\pgfqpoint{3.953607in}{1.254120in}}%
\pgfpathlineto{\pgfqpoint{4.150572in}{1.201950in}}%
\pgfpathlineto{\pgfqpoint{4.434202in}{1.126673in}}%
\pgfpathlineto{\pgfqpoint{4.568139in}{1.088819in}}%
\pgfpathlineto{\pgfqpoint{4.678440in}{1.055432in}}%
\pgfpathlineto{\pgfqpoint{4.772983in}{1.024523in}}%
\pgfpathlineto{\pgfqpoint{4.851769in}{0.996587in}}%
\pgfpathlineto{\pgfqpoint{4.922677in}{0.969233in}}%
\pgfpathlineto{\pgfqpoint{4.985706in}{0.942663in}}%
\pgfpathlineto{\pgfqpoint{5.040856in}{0.917205in}}%
\pgfpathlineto{\pgfqpoint{5.096006in}{0.889128in}}%
\pgfpathlineto{\pgfqpoint{5.143278in}{0.862409in}}%
\pgfpathlineto{\pgfqpoint{5.159035in}{0.852843in}}%
\pgfpathlineto{\pgfqpoint{5.159035in}{0.852843in}}%
\pgfusepath{stroke}%
\end{pgfscope}%
\begin{pgfscope}%
\pgfpathrectangle{\pgfqpoint{3.347347in}{0.790446in}}{\pgfqpoint{1.897959in}{1.372727in}} %
\pgfusepath{clip}%
\pgfsetbuttcap%
\pgfsetroundjoin%
\definecolor{currentfill}{rgb}{0.000000,0.000000,1.000000}%
\pgfsetfillcolor{currentfill}%
\pgfsetlinewidth{1.003750pt}%
\definecolor{currentstroke}{rgb}{0.000000,0.000000,1.000000}%
\pgfsetstrokecolor{currentstroke}%
\pgfsetdash{}{0pt}%
\pgfsys@defobject{currentmarker}{\pgfqpoint{-0.041667in}{-0.041667in}}{\pgfqpoint{0.041667in}{0.041667in}}{%
\pgfpathmoveto{\pgfqpoint{0.000000in}{-0.041667in}}%
\pgfpathcurveto{\pgfqpoint{0.011050in}{-0.041667in}}{\pgfqpoint{0.021649in}{-0.037276in}}{\pgfqpoint{0.029463in}{-0.029463in}}%
\pgfpathcurveto{\pgfqpoint{0.037276in}{-0.021649in}}{\pgfqpoint{0.041667in}{-0.011050in}}{\pgfqpoint{0.041667in}{0.000000in}}%
\pgfpathcurveto{\pgfqpoint{0.041667in}{0.011050in}}{\pgfqpoint{0.037276in}{0.021649in}}{\pgfqpoint{0.029463in}{0.029463in}}%
\pgfpathcurveto{\pgfqpoint{0.021649in}{0.037276in}}{\pgfqpoint{0.011050in}{0.041667in}}{\pgfqpoint{0.000000in}{0.041667in}}%
\pgfpathcurveto{\pgfqpoint{-0.011050in}{0.041667in}}{\pgfqpoint{-0.021649in}{0.037276in}}{\pgfqpoint{-0.029463in}{0.029463in}}%
\pgfpathcurveto{\pgfqpoint{-0.037276in}{0.021649in}}{\pgfqpoint{-0.041667in}{0.011050in}}{\pgfqpoint{-0.041667in}{0.000000in}}%
\pgfpathcurveto{\pgfqpoint{-0.041667in}{-0.011050in}}{\pgfqpoint{-0.037276in}{-0.021649in}}{\pgfqpoint{-0.029463in}{-0.029463in}}%
\pgfpathcurveto{\pgfqpoint{-0.021649in}{-0.037276in}}{\pgfqpoint{-0.011050in}{-0.041667in}}{\pgfqpoint{0.000000in}{-0.041667in}}%
\pgfpathclose%
\pgfusepath{stroke,fill}%
}%
\begin{pgfscope}%
\pgfsys@transformshift{3.433618in}{1.430064in}%
\pgfsys@useobject{currentmarker}{}%
\end{pgfscope}%
\begin{pgfscope}%
\pgfsys@transformshift{3.780277in}{1.302994in}%
\pgfsys@useobject{currentmarker}{}%
\end{pgfscope}%
\begin{pgfscope}%
\pgfsys@transformshift{4.126936in}{1.208134in}%
\pgfsys@useobject{currentmarker}{}%
\end{pgfscope}%
\begin{pgfscope}%
\pgfsys@transformshift{4.473595in}{1.115778in}%
\pgfsys@useobject{currentmarker}{}%
\end{pgfscope}%
\begin{pgfscope}%
\pgfsys@transformshift{4.820255in}{1.008037in}%
\pgfsys@useobject{currentmarker}{}%
\end{pgfscope}%
\end{pgfscope}%
\begin{pgfscope}%
\pgfpathrectangle{\pgfqpoint{3.347347in}{0.790446in}}{\pgfqpoint{1.897959in}{1.372727in}} %
\pgfusepath{clip}%
\pgfsetbuttcap%
\pgfsetroundjoin%
\pgfsetlinewidth{1.505625pt}%
\definecolor{currentstroke}{rgb}{0.000000,0.750000,0.750000}%
\pgfsetstrokecolor{currentstroke}%
\pgfsetdash{{9.600000pt}{2.400000pt}{1.500000pt}{2.400000pt}}{0.000000pt}%
\pgfpathmoveto{\pgfqpoint{3.433618in}{1.890956in}}%
\pgfpathlineto{\pgfqpoint{3.449375in}{1.843858in}}%
\pgfpathlineto{\pgfqpoint{3.465132in}{1.803246in}}%
\pgfpathlineto{\pgfqpoint{3.480890in}{1.767621in}}%
\pgfpathlineto{\pgfqpoint{3.496647in}{1.735952in}}%
\pgfpathlineto{\pgfqpoint{3.520283in}{1.694292in}}%
\pgfpathlineto{\pgfqpoint{3.543918in}{1.658140in}}%
\pgfpathlineto{\pgfqpoint{3.567554in}{1.626302in}}%
\pgfpathlineto{\pgfqpoint{3.591190in}{1.597931in}}%
\pgfpathlineto{\pgfqpoint{3.614826in}{1.572407in}}%
\pgfpathlineto{\pgfqpoint{3.646341in}{1.542007in}}%
\pgfpathlineto{\pgfqpoint{3.677855in}{1.515012in}}%
\pgfpathlineto{\pgfqpoint{3.709369in}{1.490810in}}%
\pgfpathlineto{\pgfqpoint{3.748763in}{1.463791in}}%
\pgfpathlineto{\pgfqpoint{3.788156in}{1.439741in}}%
\pgfpathlineto{\pgfqpoint{3.827549in}{1.418151in}}%
\pgfpathlineto{\pgfqpoint{3.874820in}{1.394948in}}%
\pgfpathlineto{\pgfqpoint{3.922092in}{1.374215in}}%
\pgfpathlineto{\pgfqpoint{3.977242in}{1.352629in}}%
\pgfpathlineto{\pgfqpoint{4.032393in}{1.333395in}}%
\pgfpathlineto{\pgfqpoint{4.095422in}{1.313828in}}%
\pgfpathlineto{\pgfqpoint{4.166329in}{1.294397in}}%
\pgfpathlineto{\pgfqpoint{4.237237in}{1.277265in}}%
\pgfpathlineto{\pgfqpoint{4.316023in}{1.260507in}}%
\pgfpathlineto{\pgfqpoint{4.402688in}{1.244421in}}%
\pgfpathlineto{\pgfqpoint{4.497231in}{1.229258in}}%
\pgfpathlineto{\pgfqpoint{4.599653in}{1.215228in}}%
\pgfpathlineto{\pgfqpoint{4.709954in}{1.202515in}}%
\pgfpathlineto{\pgfqpoint{4.828133in}{1.191283in}}%
\pgfpathlineto{\pgfqpoint{4.954191in}{1.181696in}}%
\pgfpathlineto{\pgfqpoint{5.080249in}{1.174314in}}%
\pgfpathlineto{\pgfqpoint{5.159035in}{1.170731in}}%
\pgfpathlineto{\pgfqpoint{5.159035in}{1.170731in}}%
\pgfusepath{stroke}%
\end{pgfscope}%
\begin{pgfscope}%
\pgfpathrectangle{\pgfqpoint{3.347347in}{0.790446in}}{\pgfqpoint{1.897959in}{1.372727in}} %
\pgfusepath{clip}%
\pgfsetbuttcap%
\pgfsetmiterjoin%
\definecolor{currentfill}{rgb}{0.000000,0.750000,0.750000}%
\pgfsetfillcolor{currentfill}%
\pgfsetlinewidth{1.003750pt}%
\definecolor{currentstroke}{rgb}{0.000000,0.750000,0.750000}%
\pgfsetstrokecolor{currentstroke}%
\pgfsetdash{}{0pt}%
\pgfsys@defobject{currentmarker}{\pgfqpoint{-0.041667in}{-0.041667in}}{\pgfqpoint{0.041667in}{0.041667in}}{%
\pgfpathmoveto{\pgfqpoint{-0.000000in}{-0.041667in}}%
\pgfpathlineto{\pgfqpoint{0.041667in}{0.041667in}}%
\pgfpathlineto{\pgfqpoint{-0.041667in}{0.041667in}}%
\pgfpathclose%
\pgfusepath{stroke,fill}%
}%
\begin{pgfscope}%
\pgfsys@transformshift{3.433618in}{1.890956in}%
\pgfsys@useobject{currentmarker}{}%
\end{pgfscope}%
\begin{pgfscope}%
\pgfsys@transformshift{3.780277in}{1.444340in}%
\pgfsys@useobject{currentmarker}{}%
\end{pgfscope}%
\begin{pgfscope}%
\pgfsys@transformshift{4.126936in}{1.304881in}%
\pgfsys@useobject{currentmarker}{}%
\end{pgfscope}%
\begin{pgfscope}%
\pgfsys@transformshift{4.473595in}{1.232836in}%
\pgfsys@useobject{currentmarker}{}%
\end{pgfscope}%
\begin{pgfscope}%
\pgfsys@transformshift{4.820255in}{1.191961in}%
\pgfsys@useobject{currentmarker}{}%
\end{pgfscope}%
\end{pgfscope}%
\begin{pgfscope}%
\pgfpathrectangle{\pgfqpoint{3.347347in}{0.790446in}}{\pgfqpoint{1.897959in}{1.372727in}} %
\pgfusepath{clip}%
\pgfsetbuttcap%
\pgfsetroundjoin%
\pgfsetlinewidth{1.505625pt}%
\definecolor{currentstroke}{rgb}{0.000000,0.000000,0.000000}%
\pgfsetstrokecolor{currentstroke}%
\pgfsetdash{{1.500000pt}{2.475000pt}}{0.000000pt}%
\pgfpathmoveto{\pgfqpoint{3.433618in}{2.100776in}}%
\pgfpathlineto{\pgfqpoint{3.457254in}{2.073759in}}%
\pgfpathlineto{\pgfqpoint{3.473011in}{2.058916in}}%
\pgfpathlineto{\pgfqpoint{3.496647in}{2.040049in}}%
\pgfpathlineto{\pgfqpoint{3.520283in}{2.023882in}}%
\pgfpathlineto{\pgfqpoint{3.551797in}{2.005056in}}%
\pgfpathlineto{\pgfqpoint{3.591190in}{1.984453in}}%
\pgfpathlineto{\pgfqpoint{3.638462in}{1.962651in}}%
\pgfpathlineto{\pgfqpoint{3.693612in}{1.940103in}}%
\pgfpathlineto{\pgfqpoint{3.748763in}{1.919934in}}%
\pgfpathlineto{\pgfqpoint{3.811791in}{1.899225in}}%
\pgfpathlineto{\pgfqpoint{3.882699in}{1.878395in}}%
\pgfpathlineto{\pgfqpoint{3.961485in}{1.857813in}}%
\pgfpathlineto{\pgfqpoint{4.040271in}{1.839509in}}%
\pgfpathlineto{\pgfqpoint{4.126936in}{1.821613in}}%
\pgfpathlineto{\pgfqpoint{4.221480in}{1.804381in}}%
\pgfpathlineto{\pgfqpoint{4.323902in}{1.788025in}}%
\pgfpathlineto{\pgfqpoint{4.434202in}{1.772726in}}%
\pgfpathlineto{\pgfqpoint{4.552382in}{1.758641in}}%
\pgfpathlineto{\pgfqpoint{4.678440in}{1.745917in}}%
\pgfpathlineto{\pgfqpoint{4.812376in}{1.734694in}}%
\pgfpathlineto{\pgfqpoint{4.954191in}{1.725126in}}%
\pgfpathlineto{\pgfqpoint{5.096006in}{1.717733in}}%
\pgfpathlineto{\pgfqpoint{5.159035in}{1.715106in}}%
\pgfpathlineto{\pgfqpoint{5.159035in}{1.715106in}}%
\pgfusepath{stroke}%
\end{pgfscope}%
\begin{pgfscope}%
\pgfpathrectangle{\pgfqpoint{3.347347in}{0.790446in}}{\pgfqpoint{1.897959in}{1.372727in}} %
\pgfusepath{clip}%
\pgfsetbuttcap%
\pgfsetroundjoin%
\definecolor{currentfill}{rgb}{0.000000,0.000000,0.000000}%
\pgfsetfillcolor{currentfill}%
\pgfsetlinewidth{1.003750pt}%
\definecolor{currentstroke}{rgb}{0.000000,0.000000,0.000000}%
\pgfsetstrokecolor{currentstroke}%
\pgfsetdash{}{0pt}%
\pgfsys@defobject{currentmarker}{\pgfqpoint{-0.041667in}{-0.041667in}}{\pgfqpoint{0.041667in}{0.041667in}}{%
\pgfpathmoveto{\pgfqpoint{-0.041667in}{0.000000in}}%
\pgfpathlineto{\pgfqpoint{0.041667in}{0.000000in}}%
\pgfpathmoveto{\pgfqpoint{0.000000in}{-0.041667in}}%
\pgfpathlineto{\pgfqpoint{0.000000in}{0.041667in}}%
\pgfusepath{stroke,fill}%
}%
\begin{pgfscope}%
\pgfsys@transformshift{3.433618in}{2.100776in}%
\pgfsys@useobject{currentmarker}{}%
\end{pgfscope}%
\begin{pgfscope}%
\pgfsys@transformshift{3.780277in}{1.909296in}%
\pgfsys@useobject{currentmarker}{}%
\end{pgfscope}%
\begin{pgfscope}%
\pgfsys@transformshift{4.126936in}{1.821613in}%
\pgfsys@useobject{currentmarker}{}%
\end{pgfscope}%
\begin{pgfscope}%
\pgfsys@transformshift{4.473595in}{1.767781in}%
\pgfsys@useobject{currentmarker}{}%
\end{pgfscope}%
\begin{pgfscope}%
\pgfsys@transformshift{4.820255in}{1.734103in}%
\pgfsys@useobject{currentmarker}{}%
\end{pgfscope}%
\end{pgfscope}%
\begin{pgfscope}%
\pgfsetrectcap%
\pgfsetmiterjoin%
\pgfsetlinewidth{0.803000pt}%
\definecolor{currentstroke}{rgb}{0.000000,0.000000,0.000000}%
\pgfsetstrokecolor{currentstroke}%
\pgfsetdash{}{0pt}%
\pgfpathmoveto{\pgfqpoint{3.347347in}{0.790446in}}%
\pgfpathlineto{\pgfqpoint{3.347347in}{2.163173in}}%
\pgfusepath{stroke}%
\end{pgfscope}%
\begin{pgfscope}%
\pgfsetrectcap%
\pgfsetmiterjoin%
\pgfsetlinewidth{0.803000pt}%
\definecolor{currentstroke}{rgb}{0.000000,0.000000,0.000000}%
\pgfsetstrokecolor{currentstroke}%
\pgfsetdash{}{0pt}%
\pgfpathmoveto{\pgfqpoint{5.245306in}{0.790446in}}%
\pgfpathlineto{\pgfqpoint{5.245306in}{2.163173in}}%
\pgfusepath{stroke}%
\end{pgfscope}%
\begin{pgfscope}%
\pgfsetrectcap%
\pgfsetmiterjoin%
\pgfsetlinewidth{0.803000pt}%
\definecolor{currentstroke}{rgb}{0.000000,0.000000,0.000000}%
\pgfsetstrokecolor{currentstroke}%
\pgfsetdash{}{0pt}%
\pgfpathmoveto{\pgfqpoint{3.347347in}{0.790446in}}%
\pgfpathlineto{\pgfqpoint{5.245306in}{0.790446in}}%
\pgfusepath{stroke}%
\end{pgfscope}%
\begin{pgfscope}%
\pgfsetrectcap%
\pgfsetmiterjoin%
\pgfsetlinewidth{0.803000pt}%
\definecolor{currentstroke}{rgb}{0.000000,0.000000,0.000000}%
\pgfsetstrokecolor{currentstroke}%
\pgfsetdash{}{0pt}%
\pgfpathmoveto{\pgfqpoint{3.347347in}{2.163173in}}%
\pgfpathlineto{\pgfqpoint{5.245306in}{2.163173in}}%
\pgfusepath{stroke}%
\end{pgfscope}%
\begin{pgfscope}%
\pgfsetbuttcap%
\pgfsetmiterjoin%
\definecolor{currentfill}{rgb}{1.000000,1.000000,1.000000}%
\pgfsetfillcolor{currentfill}%
\pgfsetlinewidth{0.000000pt}%
\definecolor{currentstroke}{rgb}{0.000000,0.000000,0.000000}%
\pgfsetstrokecolor{currentstroke}%
\pgfsetstrokeopacity{0.000000}%
\pgfsetdash{}{0pt}%
\pgfpathmoveto{\pgfqpoint{5.814694in}{0.790446in}}%
\pgfpathlineto{\pgfqpoint{7.712653in}{0.790446in}}%
\pgfpathlineto{\pgfqpoint{7.712653in}{2.163173in}}%
\pgfpathlineto{\pgfqpoint{5.814694in}{2.163173in}}%
\pgfpathclose%
\pgfusepath{fill}%
\end{pgfscope}%
\begin{pgfscope}%
\pgfsetbuttcap%
\pgfsetroundjoin%
\definecolor{currentfill}{rgb}{0.000000,0.000000,0.000000}%
\pgfsetfillcolor{currentfill}%
\pgfsetlinewidth{0.803000pt}%
\definecolor{currentstroke}{rgb}{0.000000,0.000000,0.000000}%
\pgfsetstrokecolor{currentstroke}%
\pgfsetdash{}{0pt}%
\pgfsys@defobject{currentmarker}{\pgfqpoint{0.000000in}{-0.048611in}}{\pgfqpoint{0.000000in}{0.000000in}}{%
\pgfpathmoveto{\pgfqpoint{0.000000in}{0.000000in}}%
\pgfpathlineto{\pgfqpoint{0.000000in}{-0.048611in}}%
\pgfusepath{stroke,fill}%
}%
\begin{pgfscope}%
\pgfsys@transformshift{5.824280in}{0.790446in}%
\pgfsys@useobject{currentmarker}{}%
\end{pgfscope}%
\end{pgfscope}%
\begin{pgfscope}%
\pgftext[x=5.824280in,y=0.693224in,,top]{\rmfamily\fontsize{10.000000}{12.000000}\selectfont \(\displaystyle 0.0\)}%
\end{pgfscope}%
\begin{pgfscope}%
\pgfsetbuttcap%
\pgfsetroundjoin%
\definecolor{currentfill}{rgb}{0.000000,0.000000,0.000000}%
\pgfsetfillcolor{currentfill}%
\pgfsetlinewidth{0.803000pt}%
\definecolor{currentstroke}{rgb}{0.000000,0.000000,0.000000}%
\pgfsetstrokecolor{currentstroke}%
\pgfsetdash{}{0pt}%
\pgfsys@defobject{currentmarker}{\pgfqpoint{0.000000in}{-0.048611in}}{\pgfqpoint{0.000000in}{0.000000in}}{%
\pgfpathmoveto{\pgfqpoint{0.000000in}{0.000000in}}%
\pgfpathlineto{\pgfqpoint{0.000000in}{-0.048611in}}%
\pgfusepath{stroke,fill}%
}%
\begin{pgfscope}%
\pgfsys@transformshift{6.591132in}{0.790446in}%
\pgfsys@useobject{currentmarker}{}%
\end{pgfscope}%
\end{pgfscope}%
\begin{pgfscope}%
\pgftext[x=6.591132in,y=0.693224in,,top]{\rmfamily\fontsize{10.000000}{12.000000}\selectfont \(\displaystyle 0.5\)}%
\end{pgfscope}%
\begin{pgfscope}%
\pgfsetbuttcap%
\pgfsetroundjoin%
\definecolor{currentfill}{rgb}{0.000000,0.000000,0.000000}%
\pgfsetfillcolor{currentfill}%
\pgfsetlinewidth{0.803000pt}%
\definecolor{currentstroke}{rgb}{0.000000,0.000000,0.000000}%
\pgfsetstrokecolor{currentstroke}%
\pgfsetdash{}{0pt}%
\pgfsys@defobject{currentmarker}{\pgfqpoint{0.000000in}{-0.048611in}}{\pgfqpoint{0.000000in}{0.000000in}}{%
\pgfpathmoveto{\pgfqpoint{0.000000in}{0.000000in}}%
\pgfpathlineto{\pgfqpoint{0.000000in}{-0.048611in}}%
\pgfusepath{stroke,fill}%
}%
\begin{pgfscope}%
\pgfsys@transformshift{7.357984in}{0.790446in}%
\pgfsys@useobject{currentmarker}{}%
\end{pgfscope}%
\end{pgfscope}%
\begin{pgfscope}%
\pgftext[x=7.357984in,y=0.693224in,,top]{\rmfamily\fontsize{10.000000}{12.000000}\selectfont \(\displaystyle 1.0\)}%
\end{pgfscope}%
\begin{pgfscope}%
\pgfsetbuttcap%
\pgfsetroundjoin%
\definecolor{currentfill}{rgb}{0.000000,0.000000,0.000000}%
\pgfsetfillcolor{currentfill}%
\pgfsetlinewidth{0.803000pt}%
\definecolor{currentstroke}{rgb}{0.000000,0.000000,0.000000}%
\pgfsetstrokecolor{currentstroke}%
\pgfsetdash{}{0pt}%
\pgfsys@defobject{currentmarker}{\pgfqpoint{-0.048611in}{0.000000in}}{\pgfqpoint{0.000000in}{0.000000in}}{%
\pgfpathmoveto{\pgfqpoint{0.000000in}{0.000000in}}%
\pgfpathlineto{\pgfqpoint{-0.048611in}{0.000000in}}%
\pgfusepath{stroke,fill}%
}%
\begin{pgfscope}%
\pgfsys@transformshift{5.814694in}{0.844602in}%
\pgfsys@useobject{currentmarker}{}%
\end{pgfscope}%
\end{pgfscope}%
\begin{pgfscope}%
\pgftext[x=5.429469in,y=0.791841in,left,base]{\rmfamily\fontsize{10.000000}{12.000000}\selectfont \(\displaystyle 10^{-7}\)}%
\end{pgfscope}%
\begin{pgfscope}%
\pgfsetbuttcap%
\pgfsetroundjoin%
\definecolor{currentfill}{rgb}{0.000000,0.000000,0.000000}%
\pgfsetfillcolor{currentfill}%
\pgfsetlinewidth{0.803000pt}%
\definecolor{currentstroke}{rgb}{0.000000,0.000000,0.000000}%
\pgfsetstrokecolor{currentstroke}%
\pgfsetdash{}{0pt}%
\pgfsys@defobject{currentmarker}{\pgfqpoint{-0.048611in}{0.000000in}}{\pgfqpoint{0.000000in}{0.000000in}}{%
\pgfpathmoveto{\pgfqpoint{0.000000in}{0.000000in}}%
\pgfpathlineto{\pgfqpoint{-0.048611in}{0.000000in}}%
\pgfusepath{stroke,fill}%
}%
\begin{pgfscope}%
\pgfsys@transformshift{5.814694in}{1.352175in}%
\pgfsys@useobject{currentmarker}{}%
\end{pgfscope}%
\end{pgfscope}%
\begin{pgfscope}%
\pgftext[x=5.429469in,y=1.299414in,left,base]{\rmfamily\fontsize{10.000000}{12.000000}\selectfont \(\displaystyle 10^{-6}\)}%
\end{pgfscope}%
\begin{pgfscope}%
\pgfsetbuttcap%
\pgfsetroundjoin%
\definecolor{currentfill}{rgb}{0.000000,0.000000,0.000000}%
\pgfsetfillcolor{currentfill}%
\pgfsetlinewidth{0.803000pt}%
\definecolor{currentstroke}{rgb}{0.000000,0.000000,0.000000}%
\pgfsetstrokecolor{currentstroke}%
\pgfsetdash{}{0pt}%
\pgfsys@defobject{currentmarker}{\pgfqpoint{-0.048611in}{0.000000in}}{\pgfqpoint{0.000000in}{0.000000in}}{%
\pgfpathmoveto{\pgfqpoint{0.000000in}{0.000000in}}%
\pgfpathlineto{\pgfqpoint{-0.048611in}{0.000000in}}%
\pgfusepath{stroke,fill}%
}%
\begin{pgfscope}%
\pgfsys@transformshift{5.814694in}{1.859748in}%
\pgfsys@useobject{currentmarker}{}%
\end{pgfscope}%
\end{pgfscope}%
\begin{pgfscope}%
\pgftext[x=5.429469in,y=1.806987in,left,base]{\rmfamily\fontsize{10.000000}{12.000000}\selectfont \(\displaystyle 10^{-5}\)}%
\end{pgfscope}%
\begin{pgfscope}%
\pgfsetbuttcap%
\pgfsetroundjoin%
\definecolor{currentfill}{rgb}{0.000000,0.000000,0.000000}%
\pgfsetfillcolor{currentfill}%
\pgfsetlinewidth{0.602250pt}%
\definecolor{currentstroke}{rgb}{0.000000,0.000000,0.000000}%
\pgfsetstrokecolor{currentstroke}%
\pgfsetdash{}{0pt}%
\pgfsys@defobject{currentmarker}{\pgfqpoint{-0.027778in}{0.000000in}}{\pgfqpoint{0.000000in}{0.000000in}}{%
\pgfpathmoveto{\pgfqpoint{0.000000in}{0.000000in}}%
\pgfpathlineto{\pgfqpoint{-0.027778in}{0.000000in}}%
\pgfusepath{stroke,fill}%
}%
\begin{pgfscope}%
\pgfsys@transformshift{5.814694in}{0.795413in}%
\pgfsys@useobject{currentmarker}{}%
\end{pgfscope}%
\end{pgfscope}%
\begin{pgfscope}%
\pgfsetbuttcap%
\pgfsetroundjoin%
\definecolor{currentfill}{rgb}{0.000000,0.000000,0.000000}%
\pgfsetfillcolor{currentfill}%
\pgfsetlinewidth{0.602250pt}%
\definecolor{currentstroke}{rgb}{0.000000,0.000000,0.000000}%
\pgfsetstrokecolor{currentstroke}%
\pgfsetdash{}{0pt}%
\pgfsys@defobject{currentmarker}{\pgfqpoint{-0.027778in}{0.000000in}}{\pgfqpoint{0.000000in}{0.000000in}}{%
\pgfpathmoveto{\pgfqpoint{0.000000in}{0.000000in}}%
\pgfpathlineto{\pgfqpoint{-0.027778in}{0.000000in}}%
\pgfusepath{stroke,fill}%
}%
\begin{pgfscope}%
\pgfsys@transformshift{5.814694in}{0.821377in}%
\pgfsys@useobject{currentmarker}{}%
\end{pgfscope}%
\end{pgfscope}%
\begin{pgfscope}%
\pgfsetbuttcap%
\pgfsetroundjoin%
\definecolor{currentfill}{rgb}{0.000000,0.000000,0.000000}%
\pgfsetfillcolor{currentfill}%
\pgfsetlinewidth{0.602250pt}%
\definecolor{currentstroke}{rgb}{0.000000,0.000000,0.000000}%
\pgfsetstrokecolor{currentstroke}%
\pgfsetdash{}{0pt}%
\pgfsys@defobject{currentmarker}{\pgfqpoint{-0.027778in}{0.000000in}}{\pgfqpoint{0.000000in}{0.000000in}}{%
\pgfpathmoveto{\pgfqpoint{0.000000in}{0.000000in}}%
\pgfpathlineto{\pgfqpoint{-0.027778in}{0.000000in}}%
\pgfusepath{stroke,fill}%
}%
\begin{pgfscope}%
\pgfsys@transformshift{5.814694in}{0.997397in}%
\pgfsys@useobject{currentmarker}{}%
\end{pgfscope}%
\end{pgfscope}%
\begin{pgfscope}%
\pgfsetbuttcap%
\pgfsetroundjoin%
\definecolor{currentfill}{rgb}{0.000000,0.000000,0.000000}%
\pgfsetfillcolor{currentfill}%
\pgfsetlinewidth{0.602250pt}%
\definecolor{currentstroke}{rgb}{0.000000,0.000000,0.000000}%
\pgfsetstrokecolor{currentstroke}%
\pgfsetdash{}{0pt}%
\pgfsys@defobject{currentmarker}{\pgfqpoint{-0.027778in}{0.000000in}}{\pgfqpoint{0.000000in}{0.000000in}}{%
\pgfpathmoveto{\pgfqpoint{0.000000in}{0.000000in}}%
\pgfpathlineto{\pgfqpoint{-0.027778in}{0.000000in}}%
\pgfusepath{stroke,fill}%
}%
\begin{pgfscope}%
\pgfsys@transformshift{5.814694in}{1.086776in}%
\pgfsys@useobject{currentmarker}{}%
\end{pgfscope}%
\end{pgfscope}%
\begin{pgfscope}%
\pgfsetbuttcap%
\pgfsetroundjoin%
\definecolor{currentfill}{rgb}{0.000000,0.000000,0.000000}%
\pgfsetfillcolor{currentfill}%
\pgfsetlinewidth{0.602250pt}%
\definecolor{currentstroke}{rgb}{0.000000,0.000000,0.000000}%
\pgfsetstrokecolor{currentstroke}%
\pgfsetdash{}{0pt}%
\pgfsys@defobject{currentmarker}{\pgfqpoint{-0.027778in}{0.000000in}}{\pgfqpoint{0.000000in}{0.000000in}}{%
\pgfpathmoveto{\pgfqpoint{0.000000in}{0.000000in}}%
\pgfpathlineto{\pgfqpoint{-0.027778in}{0.000000in}}%
\pgfusepath{stroke,fill}%
}%
\begin{pgfscope}%
\pgfsys@transformshift{5.814694in}{1.150192in}%
\pgfsys@useobject{currentmarker}{}%
\end{pgfscope}%
\end{pgfscope}%
\begin{pgfscope}%
\pgfsetbuttcap%
\pgfsetroundjoin%
\definecolor{currentfill}{rgb}{0.000000,0.000000,0.000000}%
\pgfsetfillcolor{currentfill}%
\pgfsetlinewidth{0.602250pt}%
\definecolor{currentstroke}{rgb}{0.000000,0.000000,0.000000}%
\pgfsetstrokecolor{currentstroke}%
\pgfsetdash{}{0pt}%
\pgfsys@defobject{currentmarker}{\pgfqpoint{-0.027778in}{0.000000in}}{\pgfqpoint{0.000000in}{0.000000in}}{%
\pgfpathmoveto{\pgfqpoint{0.000000in}{0.000000in}}%
\pgfpathlineto{\pgfqpoint{-0.027778in}{0.000000in}}%
\pgfusepath{stroke,fill}%
}%
\begin{pgfscope}%
\pgfsys@transformshift{5.814694in}{1.199381in}%
\pgfsys@useobject{currentmarker}{}%
\end{pgfscope}%
\end{pgfscope}%
\begin{pgfscope}%
\pgfsetbuttcap%
\pgfsetroundjoin%
\definecolor{currentfill}{rgb}{0.000000,0.000000,0.000000}%
\pgfsetfillcolor{currentfill}%
\pgfsetlinewidth{0.602250pt}%
\definecolor{currentstroke}{rgb}{0.000000,0.000000,0.000000}%
\pgfsetstrokecolor{currentstroke}%
\pgfsetdash{}{0pt}%
\pgfsys@defobject{currentmarker}{\pgfqpoint{-0.027778in}{0.000000in}}{\pgfqpoint{0.000000in}{0.000000in}}{%
\pgfpathmoveto{\pgfqpoint{0.000000in}{0.000000in}}%
\pgfpathlineto{\pgfqpoint{-0.027778in}{0.000000in}}%
\pgfusepath{stroke,fill}%
}%
\begin{pgfscope}%
\pgfsys@transformshift{5.814694in}{1.239571in}%
\pgfsys@useobject{currentmarker}{}%
\end{pgfscope}%
\end{pgfscope}%
\begin{pgfscope}%
\pgfsetbuttcap%
\pgfsetroundjoin%
\definecolor{currentfill}{rgb}{0.000000,0.000000,0.000000}%
\pgfsetfillcolor{currentfill}%
\pgfsetlinewidth{0.602250pt}%
\definecolor{currentstroke}{rgb}{0.000000,0.000000,0.000000}%
\pgfsetstrokecolor{currentstroke}%
\pgfsetdash{}{0pt}%
\pgfsys@defobject{currentmarker}{\pgfqpoint{-0.027778in}{0.000000in}}{\pgfqpoint{0.000000in}{0.000000in}}{%
\pgfpathmoveto{\pgfqpoint{0.000000in}{0.000000in}}%
\pgfpathlineto{\pgfqpoint{-0.027778in}{0.000000in}}%
\pgfusepath{stroke,fill}%
}%
\begin{pgfscope}%
\pgfsys@transformshift{5.814694in}{1.273551in}%
\pgfsys@useobject{currentmarker}{}%
\end{pgfscope}%
\end{pgfscope}%
\begin{pgfscope}%
\pgfsetbuttcap%
\pgfsetroundjoin%
\definecolor{currentfill}{rgb}{0.000000,0.000000,0.000000}%
\pgfsetfillcolor{currentfill}%
\pgfsetlinewidth{0.602250pt}%
\definecolor{currentstroke}{rgb}{0.000000,0.000000,0.000000}%
\pgfsetstrokecolor{currentstroke}%
\pgfsetdash{}{0pt}%
\pgfsys@defobject{currentmarker}{\pgfqpoint{-0.027778in}{0.000000in}}{\pgfqpoint{0.000000in}{0.000000in}}{%
\pgfpathmoveto{\pgfqpoint{0.000000in}{0.000000in}}%
\pgfpathlineto{\pgfqpoint{-0.027778in}{0.000000in}}%
\pgfusepath{stroke,fill}%
}%
\begin{pgfscope}%
\pgfsys@transformshift{5.814694in}{1.302986in}%
\pgfsys@useobject{currentmarker}{}%
\end{pgfscope}%
\end{pgfscope}%
\begin{pgfscope}%
\pgfsetbuttcap%
\pgfsetroundjoin%
\definecolor{currentfill}{rgb}{0.000000,0.000000,0.000000}%
\pgfsetfillcolor{currentfill}%
\pgfsetlinewidth{0.602250pt}%
\definecolor{currentstroke}{rgb}{0.000000,0.000000,0.000000}%
\pgfsetstrokecolor{currentstroke}%
\pgfsetdash{}{0pt}%
\pgfsys@defobject{currentmarker}{\pgfqpoint{-0.027778in}{0.000000in}}{\pgfqpoint{0.000000in}{0.000000in}}{%
\pgfpathmoveto{\pgfqpoint{0.000000in}{0.000000in}}%
\pgfpathlineto{\pgfqpoint{-0.027778in}{0.000000in}}%
\pgfusepath{stroke,fill}%
}%
\begin{pgfscope}%
\pgfsys@transformshift{5.814694in}{1.328950in}%
\pgfsys@useobject{currentmarker}{}%
\end{pgfscope}%
\end{pgfscope}%
\begin{pgfscope}%
\pgfsetbuttcap%
\pgfsetroundjoin%
\definecolor{currentfill}{rgb}{0.000000,0.000000,0.000000}%
\pgfsetfillcolor{currentfill}%
\pgfsetlinewidth{0.602250pt}%
\definecolor{currentstroke}{rgb}{0.000000,0.000000,0.000000}%
\pgfsetstrokecolor{currentstroke}%
\pgfsetdash{}{0pt}%
\pgfsys@defobject{currentmarker}{\pgfqpoint{-0.027778in}{0.000000in}}{\pgfqpoint{0.000000in}{0.000000in}}{%
\pgfpathmoveto{\pgfqpoint{0.000000in}{0.000000in}}%
\pgfpathlineto{\pgfqpoint{-0.027778in}{0.000000in}}%
\pgfusepath{stroke,fill}%
}%
\begin{pgfscope}%
\pgfsys@transformshift{5.814694in}{1.504970in}%
\pgfsys@useobject{currentmarker}{}%
\end{pgfscope}%
\end{pgfscope}%
\begin{pgfscope}%
\pgfsetbuttcap%
\pgfsetroundjoin%
\definecolor{currentfill}{rgb}{0.000000,0.000000,0.000000}%
\pgfsetfillcolor{currentfill}%
\pgfsetlinewidth{0.602250pt}%
\definecolor{currentstroke}{rgb}{0.000000,0.000000,0.000000}%
\pgfsetstrokecolor{currentstroke}%
\pgfsetdash{}{0pt}%
\pgfsys@defobject{currentmarker}{\pgfqpoint{-0.027778in}{0.000000in}}{\pgfqpoint{0.000000in}{0.000000in}}{%
\pgfpathmoveto{\pgfqpoint{0.000000in}{0.000000in}}%
\pgfpathlineto{\pgfqpoint{-0.027778in}{0.000000in}}%
\pgfusepath{stroke,fill}%
}%
\begin{pgfscope}%
\pgfsys@transformshift{5.814694in}{1.594349in}%
\pgfsys@useobject{currentmarker}{}%
\end{pgfscope}%
\end{pgfscope}%
\begin{pgfscope}%
\pgfsetbuttcap%
\pgfsetroundjoin%
\definecolor{currentfill}{rgb}{0.000000,0.000000,0.000000}%
\pgfsetfillcolor{currentfill}%
\pgfsetlinewidth{0.602250pt}%
\definecolor{currentstroke}{rgb}{0.000000,0.000000,0.000000}%
\pgfsetstrokecolor{currentstroke}%
\pgfsetdash{}{0pt}%
\pgfsys@defobject{currentmarker}{\pgfqpoint{-0.027778in}{0.000000in}}{\pgfqpoint{0.000000in}{0.000000in}}{%
\pgfpathmoveto{\pgfqpoint{0.000000in}{0.000000in}}%
\pgfpathlineto{\pgfqpoint{-0.027778in}{0.000000in}}%
\pgfusepath{stroke,fill}%
}%
\begin{pgfscope}%
\pgfsys@transformshift{5.814694in}{1.657765in}%
\pgfsys@useobject{currentmarker}{}%
\end{pgfscope}%
\end{pgfscope}%
\begin{pgfscope}%
\pgfsetbuttcap%
\pgfsetroundjoin%
\definecolor{currentfill}{rgb}{0.000000,0.000000,0.000000}%
\pgfsetfillcolor{currentfill}%
\pgfsetlinewidth{0.602250pt}%
\definecolor{currentstroke}{rgb}{0.000000,0.000000,0.000000}%
\pgfsetstrokecolor{currentstroke}%
\pgfsetdash{}{0pt}%
\pgfsys@defobject{currentmarker}{\pgfqpoint{-0.027778in}{0.000000in}}{\pgfqpoint{0.000000in}{0.000000in}}{%
\pgfpathmoveto{\pgfqpoint{0.000000in}{0.000000in}}%
\pgfpathlineto{\pgfqpoint{-0.027778in}{0.000000in}}%
\pgfusepath{stroke,fill}%
}%
\begin{pgfscope}%
\pgfsys@transformshift{5.814694in}{1.706954in}%
\pgfsys@useobject{currentmarker}{}%
\end{pgfscope}%
\end{pgfscope}%
\begin{pgfscope}%
\pgfsetbuttcap%
\pgfsetroundjoin%
\definecolor{currentfill}{rgb}{0.000000,0.000000,0.000000}%
\pgfsetfillcolor{currentfill}%
\pgfsetlinewidth{0.602250pt}%
\definecolor{currentstroke}{rgb}{0.000000,0.000000,0.000000}%
\pgfsetstrokecolor{currentstroke}%
\pgfsetdash{}{0pt}%
\pgfsys@defobject{currentmarker}{\pgfqpoint{-0.027778in}{0.000000in}}{\pgfqpoint{0.000000in}{0.000000in}}{%
\pgfpathmoveto{\pgfqpoint{0.000000in}{0.000000in}}%
\pgfpathlineto{\pgfqpoint{-0.027778in}{0.000000in}}%
\pgfusepath{stroke,fill}%
}%
\begin{pgfscope}%
\pgfsys@transformshift{5.814694in}{1.747144in}%
\pgfsys@useobject{currentmarker}{}%
\end{pgfscope}%
\end{pgfscope}%
\begin{pgfscope}%
\pgfsetbuttcap%
\pgfsetroundjoin%
\definecolor{currentfill}{rgb}{0.000000,0.000000,0.000000}%
\pgfsetfillcolor{currentfill}%
\pgfsetlinewidth{0.602250pt}%
\definecolor{currentstroke}{rgb}{0.000000,0.000000,0.000000}%
\pgfsetstrokecolor{currentstroke}%
\pgfsetdash{}{0pt}%
\pgfsys@defobject{currentmarker}{\pgfqpoint{-0.027778in}{0.000000in}}{\pgfqpoint{0.000000in}{0.000000in}}{%
\pgfpathmoveto{\pgfqpoint{0.000000in}{0.000000in}}%
\pgfpathlineto{\pgfqpoint{-0.027778in}{0.000000in}}%
\pgfusepath{stroke,fill}%
}%
\begin{pgfscope}%
\pgfsys@transformshift{5.814694in}{1.781124in}%
\pgfsys@useobject{currentmarker}{}%
\end{pgfscope}%
\end{pgfscope}%
\begin{pgfscope}%
\pgfsetbuttcap%
\pgfsetroundjoin%
\definecolor{currentfill}{rgb}{0.000000,0.000000,0.000000}%
\pgfsetfillcolor{currentfill}%
\pgfsetlinewidth{0.602250pt}%
\definecolor{currentstroke}{rgb}{0.000000,0.000000,0.000000}%
\pgfsetstrokecolor{currentstroke}%
\pgfsetdash{}{0pt}%
\pgfsys@defobject{currentmarker}{\pgfqpoint{-0.027778in}{0.000000in}}{\pgfqpoint{0.000000in}{0.000000in}}{%
\pgfpathmoveto{\pgfqpoint{0.000000in}{0.000000in}}%
\pgfpathlineto{\pgfqpoint{-0.027778in}{0.000000in}}%
\pgfusepath{stroke,fill}%
}%
\begin{pgfscope}%
\pgfsys@transformshift{5.814694in}{1.810559in}%
\pgfsys@useobject{currentmarker}{}%
\end{pgfscope}%
\end{pgfscope}%
\begin{pgfscope}%
\pgfsetbuttcap%
\pgfsetroundjoin%
\definecolor{currentfill}{rgb}{0.000000,0.000000,0.000000}%
\pgfsetfillcolor{currentfill}%
\pgfsetlinewidth{0.602250pt}%
\definecolor{currentstroke}{rgb}{0.000000,0.000000,0.000000}%
\pgfsetstrokecolor{currentstroke}%
\pgfsetdash{}{0pt}%
\pgfsys@defobject{currentmarker}{\pgfqpoint{-0.027778in}{0.000000in}}{\pgfqpoint{0.000000in}{0.000000in}}{%
\pgfpathmoveto{\pgfqpoint{0.000000in}{0.000000in}}%
\pgfpathlineto{\pgfqpoint{-0.027778in}{0.000000in}}%
\pgfusepath{stroke,fill}%
}%
\begin{pgfscope}%
\pgfsys@transformshift{5.814694in}{1.836523in}%
\pgfsys@useobject{currentmarker}{}%
\end{pgfscope}%
\end{pgfscope}%
\begin{pgfscope}%
\pgfsetbuttcap%
\pgfsetroundjoin%
\definecolor{currentfill}{rgb}{0.000000,0.000000,0.000000}%
\pgfsetfillcolor{currentfill}%
\pgfsetlinewidth{0.602250pt}%
\definecolor{currentstroke}{rgb}{0.000000,0.000000,0.000000}%
\pgfsetstrokecolor{currentstroke}%
\pgfsetdash{}{0pt}%
\pgfsys@defobject{currentmarker}{\pgfqpoint{-0.027778in}{0.000000in}}{\pgfqpoint{0.000000in}{0.000000in}}{%
\pgfpathmoveto{\pgfqpoint{0.000000in}{0.000000in}}%
\pgfpathlineto{\pgfqpoint{-0.027778in}{0.000000in}}%
\pgfusepath{stroke,fill}%
}%
\begin{pgfscope}%
\pgfsys@transformshift{5.814694in}{2.012543in}%
\pgfsys@useobject{currentmarker}{}%
\end{pgfscope}%
\end{pgfscope}%
\begin{pgfscope}%
\pgfsetbuttcap%
\pgfsetroundjoin%
\definecolor{currentfill}{rgb}{0.000000,0.000000,0.000000}%
\pgfsetfillcolor{currentfill}%
\pgfsetlinewidth{0.602250pt}%
\definecolor{currentstroke}{rgb}{0.000000,0.000000,0.000000}%
\pgfsetstrokecolor{currentstroke}%
\pgfsetdash{}{0pt}%
\pgfsys@defobject{currentmarker}{\pgfqpoint{-0.027778in}{0.000000in}}{\pgfqpoint{0.000000in}{0.000000in}}{%
\pgfpathmoveto{\pgfqpoint{0.000000in}{0.000000in}}%
\pgfpathlineto{\pgfqpoint{-0.027778in}{0.000000in}}%
\pgfusepath{stroke,fill}%
}%
\begin{pgfscope}%
\pgfsys@transformshift{5.814694in}{2.101922in}%
\pgfsys@useobject{currentmarker}{}%
\end{pgfscope}%
\end{pgfscope}%
\begin{pgfscope}%
\pgfsetbuttcap%
\pgfsetroundjoin%
\definecolor{currentfill}{rgb}{0.000000,0.000000,0.000000}%
\pgfsetfillcolor{currentfill}%
\pgfsetlinewidth{0.602250pt}%
\definecolor{currentstroke}{rgb}{0.000000,0.000000,0.000000}%
\pgfsetstrokecolor{currentstroke}%
\pgfsetdash{}{0pt}%
\pgfsys@defobject{currentmarker}{\pgfqpoint{-0.027778in}{0.000000in}}{\pgfqpoint{0.000000in}{0.000000in}}{%
\pgfpathmoveto{\pgfqpoint{0.000000in}{0.000000in}}%
\pgfpathlineto{\pgfqpoint{-0.027778in}{0.000000in}}%
\pgfusepath{stroke,fill}%
}%
\begin{pgfscope}%
\pgfsys@transformshift{5.814694in}{2.165338in}%
\pgfsys@useobject{currentmarker}{}%
\end{pgfscope}%
\end{pgfscope}%
\begin{pgfscope}%
\pgfpathrectangle{\pgfqpoint{5.814694in}{0.790446in}}{\pgfqpoint{1.897959in}{1.372727in}} %
\pgfusepath{clip}%
\pgfsetbuttcap%
\pgfsetroundjoin%
\pgfsetlinewidth{1.505625pt}%
\definecolor{currentstroke}{rgb}{1.000000,0.000000,0.000000}%
\pgfsetstrokecolor{currentstroke}%
\pgfsetdash{{5.550000pt}{2.400000pt}}{0.000000pt}%
\pgfpathmoveto{\pgfqpoint{5.900965in}{1.923560in}}%
\pgfpathlineto{\pgfqpoint{6.118240in}{1.842845in}}%
\pgfpathlineto{\pgfqpoint{6.246048in}{1.797788in}}%
\pgfpathlineto{\pgfqpoint{6.361076in}{1.759623in}}%
\pgfpathlineto{\pgfqpoint{6.476104in}{1.723905in}}%
\pgfpathlineto{\pgfqpoint{6.591132in}{1.690602in}}%
\pgfpathlineto{\pgfqpoint{6.706160in}{1.659579in}}%
\pgfpathlineto{\pgfqpoint{6.833968in}{1.627568in}}%
\pgfpathlineto{\pgfqpoint{6.961777in}{1.597895in}}%
\pgfpathlineto{\pgfqpoint{7.102367in}{1.567661in}}%
\pgfpathlineto{\pgfqpoint{7.242956in}{1.539660in}}%
\pgfpathlineto{\pgfqpoint{7.396327in}{1.511356in}}%
\pgfpathlineto{\pgfqpoint{7.562478in}{1.483004in}}%
\pgfpathlineto{\pgfqpoint{7.626382in}{1.472675in}}%
\pgfpathlineto{\pgfqpoint{7.626382in}{1.472675in}}%
\pgfusepath{stroke}%
\end{pgfscope}%
\begin{pgfscope}%
\pgfpathrectangle{\pgfqpoint{5.814694in}{0.790446in}}{\pgfqpoint{1.897959in}{1.372727in}} %
\pgfusepath{clip}%
\pgfsetbuttcap%
\pgfsetmiterjoin%
\definecolor{currentfill}{rgb}{1.000000,0.000000,0.000000}%
\pgfsetfillcolor{currentfill}%
\pgfsetlinewidth{1.003750pt}%
\definecolor{currentstroke}{rgb}{1.000000,0.000000,0.000000}%
\pgfsetstrokecolor{currentstroke}%
\pgfsetdash{}{0pt}%
\pgfsys@defobject{currentmarker}{\pgfqpoint{-0.041667in}{-0.041667in}}{\pgfqpoint{0.041667in}{0.041667in}}{%
\pgfpathmoveto{\pgfqpoint{-0.041667in}{-0.041667in}}%
\pgfpathlineto{\pgfqpoint{0.041667in}{-0.041667in}}%
\pgfpathlineto{\pgfqpoint{0.041667in}{0.041667in}}%
\pgfpathlineto{\pgfqpoint{-0.041667in}{0.041667in}}%
\pgfpathclose%
\pgfusepath{stroke,fill}%
}%
\begin{pgfscope}%
\pgfsys@transformshift{5.900965in}{1.923560in}%
\pgfsys@useobject{currentmarker}{}%
\end{pgfscope}%
\begin{pgfscope}%
\pgfsys@transformshift{6.246048in}{1.797788in}%
\pgfsys@useobject{currentmarker}{}%
\end{pgfscope}%
\begin{pgfscope}%
\pgfsys@transformshift{6.591132in}{1.690602in}%
\pgfsys@useobject{currentmarker}{}%
\end{pgfscope}%
\begin{pgfscope}%
\pgfsys@transformshift{6.936215in}{1.603654in}%
\pgfsys@useobject{currentmarker}{}%
\end{pgfscope}%
\begin{pgfscope}%
\pgfsys@transformshift{7.281299in}{1.532376in}%
\pgfsys@useobject{currentmarker}{}%
\end{pgfscope}%
\begin{pgfscope}%
\pgfsys@transformshift{7.626382in}{1.472675in}%
\pgfsys@useobject{currentmarker}{}%
\end{pgfscope}%
\end{pgfscope}%
\begin{pgfscope}%
\pgfpathrectangle{\pgfqpoint{5.814694in}{0.790446in}}{\pgfqpoint{1.897959in}{1.372727in}} %
\pgfusepath{clip}%
\pgfsetrectcap%
\pgfsetroundjoin%
\pgfsetlinewidth{1.505625pt}%
\definecolor{currentstroke}{rgb}{0.000000,0.000000,1.000000}%
\pgfsetstrokecolor{currentstroke}%
\pgfsetdash{}{0pt}%
\pgfpathmoveto{\pgfqpoint{5.900965in}{1.394954in}}%
\pgfpathlineto{\pgfqpoint{5.939307in}{1.383451in}}%
\pgfpathlineto{\pgfqpoint{5.990431in}{1.371355in}}%
\pgfpathlineto{\pgfqpoint{6.054335in}{1.358792in}}%
\pgfpathlineto{\pgfqpoint{6.143801in}{1.343871in}}%
\pgfpathlineto{\pgfqpoint{6.246048in}{1.329266in}}%
\pgfpathlineto{\pgfqpoint{6.361076in}{1.315050in}}%
\pgfpathlineto{\pgfqpoint{6.501666in}{1.300052in}}%
\pgfpathlineto{\pgfqpoint{6.655036in}{1.285954in}}%
\pgfpathlineto{\pgfqpoint{6.833968in}{1.271767in}}%
\pgfpathlineto{\pgfqpoint{7.038462in}{1.257796in}}%
\pgfpathlineto{\pgfqpoint{7.281299in}{1.243515in}}%
\pgfpathlineto{\pgfqpoint{7.562478in}{1.229281in}}%
\pgfpathlineto{\pgfqpoint{7.626382in}{1.226319in}}%
\pgfpathlineto{\pgfqpoint{7.626382in}{1.226319in}}%
\pgfusepath{stroke}%
\end{pgfscope}%
\begin{pgfscope}%
\pgfpathrectangle{\pgfqpoint{5.814694in}{0.790446in}}{\pgfqpoint{1.897959in}{1.372727in}} %
\pgfusepath{clip}%
\pgfsetbuttcap%
\pgfsetroundjoin%
\definecolor{currentfill}{rgb}{0.000000,0.000000,1.000000}%
\pgfsetfillcolor{currentfill}%
\pgfsetlinewidth{1.003750pt}%
\definecolor{currentstroke}{rgb}{0.000000,0.000000,1.000000}%
\pgfsetstrokecolor{currentstroke}%
\pgfsetdash{}{0pt}%
\pgfsys@defobject{currentmarker}{\pgfqpoint{-0.041667in}{-0.041667in}}{\pgfqpoint{0.041667in}{0.041667in}}{%
\pgfpathmoveto{\pgfqpoint{0.000000in}{-0.041667in}}%
\pgfpathcurveto{\pgfqpoint{0.011050in}{-0.041667in}}{\pgfqpoint{0.021649in}{-0.037276in}}{\pgfqpoint{0.029463in}{-0.029463in}}%
\pgfpathcurveto{\pgfqpoint{0.037276in}{-0.021649in}}{\pgfqpoint{0.041667in}{-0.011050in}}{\pgfqpoint{0.041667in}{0.000000in}}%
\pgfpathcurveto{\pgfqpoint{0.041667in}{0.011050in}}{\pgfqpoint{0.037276in}{0.021649in}}{\pgfqpoint{0.029463in}{0.029463in}}%
\pgfpathcurveto{\pgfqpoint{0.021649in}{0.037276in}}{\pgfqpoint{0.011050in}{0.041667in}}{\pgfqpoint{0.000000in}{0.041667in}}%
\pgfpathcurveto{\pgfqpoint{-0.011050in}{0.041667in}}{\pgfqpoint{-0.021649in}{0.037276in}}{\pgfqpoint{-0.029463in}{0.029463in}}%
\pgfpathcurveto{\pgfqpoint{-0.037276in}{0.021649in}}{\pgfqpoint{-0.041667in}{0.011050in}}{\pgfqpoint{-0.041667in}{0.000000in}}%
\pgfpathcurveto{\pgfqpoint{-0.041667in}{-0.011050in}}{\pgfqpoint{-0.037276in}{-0.021649in}}{\pgfqpoint{-0.029463in}{-0.029463in}}%
\pgfpathcurveto{\pgfqpoint{-0.021649in}{-0.037276in}}{\pgfqpoint{-0.011050in}{-0.041667in}}{\pgfqpoint{0.000000in}{-0.041667in}}%
\pgfpathclose%
\pgfusepath{stroke,fill}%
}%
\begin{pgfscope}%
\pgfsys@transformshift{5.900965in}{1.394954in}%
\pgfsys@useobject{currentmarker}{}%
\end{pgfscope}%
\begin{pgfscope}%
\pgfsys@transformshift{6.246048in}{1.329266in}%
\pgfsys@useobject{currentmarker}{}%
\end{pgfscope}%
\begin{pgfscope}%
\pgfsys@transformshift{6.591132in}{1.291580in}%
\pgfsys@useobject{currentmarker}{}%
\end{pgfscope}%
\begin{pgfscope}%
\pgfsys@transformshift{6.936215in}{1.264523in}%
\pgfsys@useobject{currentmarker}{}%
\end{pgfscope}%
\begin{pgfscope}%
\pgfsys@transformshift{7.281299in}{1.243515in}%
\pgfsys@useobject{currentmarker}{}%
\end{pgfscope}%
\begin{pgfscope}%
\pgfsys@transformshift{7.626382in}{1.226319in}%
\pgfsys@useobject{currentmarker}{}%
\end{pgfscope}%
\end{pgfscope}%
\begin{pgfscope}%
\pgfpathrectangle{\pgfqpoint{5.814694in}{0.790446in}}{\pgfqpoint{1.897959in}{1.372727in}} %
\pgfusepath{clip}%
\pgfsetbuttcap%
\pgfsetroundjoin%
\pgfsetlinewidth{1.505625pt}%
\definecolor{currentstroke}{rgb}{0.000000,0.750000,0.750000}%
\pgfsetstrokecolor{currentstroke}%
\pgfsetdash{{9.600000pt}{2.400000pt}{1.500000pt}{2.400000pt}}{0.000000pt}%
\pgfpathmoveto{\pgfqpoint{5.900965in}{1.957717in}}%
\pgfpathlineto{\pgfqpoint{5.913746in}{1.909026in}}%
\pgfpathlineto{\pgfqpoint{5.926526in}{1.865714in}}%
\pgfpathlineto{\pgfqpoint{5.952088in}{1.791269in}}%
\pgfpathlineto{\pgfqpoint{5.977650in}{1.728805in}}%
\pgfpathlineto{\pgfqpoint{6.003212in}{1.675049in}}%
\pgfpathlineto{\pgfqpoint{6.028773in}{1.627911in}}%
\pgfpathlineto{\pgfqpoint{6.054335in}{1.585973in}}%
\pgfpathlineto{\pgfqpoint{6.079897in}{1.548229in}}%
\pgfpathlineto{\pgfqpoint{6.105459in}{1.513939in}}%
\pgfpathlineto{\pgfqpoint{6.131020in}{1.482541in}}%
\pgfpathlineto{\pgfqpoint{6.169363in}{1.439942in}}%
\pgfpathlineto{\pgfqpoint{6.207706in}{1.401781in}}%
\pgfpathlineto{\pgfqpoint{6.246048in}{1.367249in}}%
\pgfpathlineto{\pgfqpoint{6.284391in}{1.335737in}}%
\pgfpathlineto{\pgfqpoint{6.322733in}{1.306777in}}%
\pgfpathlineto{\pgfqpoint{6.373857in}{1.271506in}}%
\pgfpathlineto{\pgfqpoint{6.424980in}{1.239443in}}%
\pgfpathlineto{\pgfqpoint{6.476104in}{1.210073in}}%
\pgfpathlineto{\pgfqpoint{6.527227in}{1.182991in}}%
\pgfpathlineto{\pgfqpoint{6.591132in}{1.151875in}}%
\pgfpathlineto{\pgfqpoint{6.655036in}{1.123346in}}%
\pgfpathlineto{\pgfqpoint{6.718940in}{1.097018in}}%
\pgfpathlineto{\pgfqpoint{6.795626in}{1.067903in}}%
\pgfpathlineto{\pgfqpoint{6.872311in}{1.041101in}}%
\pgfpathlineto{\pgfqpoint{6.961777in}{1.012316in}}%
\pgfpathlineto{\pgfqpoint{7.051243in}{0.985815in}}%
\pgfpathlineto{\pgfqpoint{7.153490in}{0.957907in}}%
\pgfpathlineto{\pgfqpoint{7.268518in}{0.929091in}}%
\pgfpathlineto{\pgfqpoint{7.383546in}{0.902595in}}%
\pgfpathlineto{\pgfqpoint{7.511354in}{0.875467in}}%
\pgfpathlineto{\pgfqpoint{7.626382in}{0.852843in}}%
\pgfpathlineto{\pgfqpoint{7.626382in}{0.852843in}}%
\pgfusepath{stroke}%
\end{pgfscope}%
\begin{pgfscope}%
\pgfpathrectangle{\pgfqpoint{5.814694in}{0.790446in}}{\pgfqpoint{1.897959in}{1.372727in}} %
\pgfusepath{clip}%
\pgfsetbuttcap%
\pgfsetmiterjoin%
\definecolor{currentfill}{rgb}{0.000000,0.750000,0.750000}%
\pgfsetfillcolor{currentfill}%
\pgfsetlinewidth{1.003750pt}%
\definecolor{currentstroke}{rgb}{0.000000,0.750000,0.750000}%
\pgfsetstrokecolor{currentstroke}%
\pgfsetdash{}{0pt}%
\pgfsys@defobject{currentmarker}{\pgfqpoint{-0.041667in}{-0.041667in}}{\pgfqpoint{0.041667in}{0.041667in}}{%
\pgfpathmoveto{\pgfqpoint{-0.000000in}{-0.041667in}}%
\pgfpathlineto{\pgfqpoint{0.041667in}{0.041667in}}%
\pgfpathlineto{\pgfqpoint{-0.041667in}{0.041667in}}%
\pgfpathclose%
\pgfusepath{stroke,fill}%
}%
\begin{pgfscope}%
\pgfsys@transformshift{5.900965in}{1.957717in}%
\pgfsys@useobject{currentmarker}{}%
\end{pgfscope}%
\begin{pgfscope}%
\pgfsys@transformshift{6.246048in}{1.367249in}%
\pgfsys@useobject{currentmarker}{}%
\end{pgfscope}%
\begin{pgfscope}%
\pgfsys@transformshift{6.591132in}{1.151875in}%
\pgfsys@useobject{currentmarker}{}%
\end{pgfscope}%
\begin{pgfscope}%
\pgfsys@transformshift{6.936215in}{1.020291in}%
\pgfsys@useobject{currentmarker}{}%
\end{pgfscope}%
\begin{pgfscope}%
\pgfsys@transformshift{7.281299in}{0.926039in}%
\pgfsys@useobject{currentmarker}{}%
\end{pgfscope}%
\begin{pgfscope}%
\pgfsys@transformshift{7.626382in}{0.852843in}%
\pgfsys@useobject{currentmarker}{}%
\end{pgfscope}%
\end{pgfscope}%
\begin{pgfscope}%
\pgfpathrectangle{\pgfqpoint{5.814694in}{0.790446in}}{\pgfqpoint{1.897959in}{1.372727in}} %
\pgfusepath{clip}%
\pgfsetbuttcap%
\pgfsetroundjoin%
\pgfsetlinewidth{1.505625pt}%
\definecolor{currentstroke}{rgb}{0.000000,0.000000,0.000000}%
\pgfsetstrokecolor{currentstroke}%
\pgfsetdash{{1.500000pt}{2.475000pt}}{0.000000pt}%
\pgfpathmoveto{\pgfqpoint{5.900965in}{2.100776in}}%
\pgfpathlineto{\pgfqpoint{5.926526in}{2.055718in}}%
\pgfpathlineto{\pgfqpoint{5.939307in}{2.036912in}}%
\pgfpathlineto{\pgfqpoint{5.952088in}{2.020698in}}%
\pgfpathlineto{\pgfqpoint{5.977650in}{1.993840in}}%
\pgfpathlineto{\pgfqpoint{6.003212in}{1.972049in}}%
\pgfpathlineto{\pgfqpoint{6.028773in}{1.953587in}}%
\pgfpathlineto{\pgfqpoint{6.054335in}{1.937414in}}%
\pgfpathlineto{\pgfqpoint{6.092678in}{1.916092in}}%
\pgfpathlineto{\pgfqpoint{6.131020in}{1.897190in}}%
\pgfpathlineto{\pgfqpoint{6.182144in}{1.874519in}}%
\pgfpathlineto{\pgfqpoint{6.246048in}{1.848977in}}%
\pgfpathlineto{\pgfqpoint{6.322733in}{1.821205in}}%
\pgfpathlineto{\pgfqpoint{6.399419in}{1.795782in}}%
\pgfpathlineto{\pgfqpoint{6.488885in}{1.768487in}}%
\pgfpathlineto{\pgfqpoint{6.591132in}{1.739885in}}%
\pgfpathlineto{\pgfqpoint{6.693379in}{1.713639in}}%
\pgfpathlineto{\pgfqpoint{6.808407in}{1.686542in}}%
\pgfpathlineto{\pgfqpoint{6.923434in}{1.661682in}}%
\pgfpathlineto{\pgfqpoint{7.051243in}{1.636344in}}%
\pgfpathlineto{\pgfqpoint{7.191833in}{1.610884in}}%
\pgfpathlineto{\pgfqpoint{7.345203in}{1.585596in}}%
\pgfpathlineto{\pgfqpoint{7.498573in}{1.562542in}}%
\pgfpathlineto{\pgfqpoint{7.626382in}{1.544818in}}%
\pgfpathlineto{\pgfqpoint{7.626382in}{1.544818in}}%
\pgfusepath{stroke}%
\end{pgfscope}%
\begin{pgfscope}%
\pgfpathrectangle{\pgfqpoint{5.814694in}{0.790446in}}{\pgfqpoint{1.897959in}{1.372727in}} %
\pgfusepath{clip}%
\pgfsetbuttcap%
\pgfsetroundjoin%
\definecolor{currentfill}{rgb}{0.000000,0.000000,0.000000}%
\pgfsetfillcolor{currentfill}%
\pgfsetlinewidth{1.003750pt}%
\definecolor{currentstroke}{rgb}{0.000000,0.000000,0.000000}%
\pgfsetstrokecolor{currentstroke}%
\pgfsetdash{}{0pt}%
\pgfsys@defobject{currentmarker}{\pgfqpoint{-0.041667in}{-0.041667in}}{\pgfqpoint{0.041667in}{0.041667in}}{%
\pgfpathmoveto{\pgfqpoint{-0.041667in}{0.000000in}}%
\pgfpathlineto{\pgfqpoint{0.041667in}{0.000000in}}%
\pgfpathmoveto{\pgfqpoint{0.000000in}{-0.041667in}}%
\pgfpathlineto{\pgfqpoint{0.000000in}{0.041667in}}%
\pgfusepath{stroke,fill}%
}%
\begin{pgfscope}%
\pgfsys@transformshift{5.900965in}{2.100776in}%
\pgfsys@useobject{currentmarker}{}%
\end{pgfscope}%
\begin{pgfscope}%
\pgfsys@transformshift{6.246048in}{1.848977in}%
\pgfsys@useobject{currentmarker}{}%
\end{pgfscope}%
\begin{pgfscope}%
\pgfsys@transformshift{6.591132in}{1.739885in}%
\pgfsys@useobject{currentmarker}{}%
\end{pgfscope}%
\begin{pgfscope}%
\pgfsys@transformshift{6.936215in}{1.659046in}%
\pgfsys@useobject{currentmarker}{}%
\end{pgfscope}%
\begin{pgfscope}%
\pgfsys@transformshift{7.281299in}{1.595842in}%
\pgfsys@useobject{currentmarker}{}%
\end{pgfscope}%
\begin{pgfscope}%
\pgfsys@transformshift{7.626382in}{1.544818in}%
\pgfsys@useobject{currentmarker}{}%
\end{pgfscope}%
\end{pgfscope}%
\begin{pgfscope}%
\pgfsetrectcap%
\pgfsetmiterjoin%
\pgfsetlinewidth{0.803000pt}%
\definecolor{currentstroke}{rgb}{0.000000,0.000000,0.000000}%
\pgfsetstrokecolor{currentstroke}%
\pgfsetdash{}{0pt}%
\pgfpathmoveto{\pgfqpoint{5.814694in}{0.790446in}}%
\pgfpathlineto{\pgfqpoint{5.814694in}{2.163173in}}%
\pgfusepath{stroke}%
\end{pgfscope}%
\begin{pgfscope}%
\pgfsetrectcap%
\pgfsetmiterjoin%
\pgfsetlinewidth{0.803000pt}%
\definecolor{currentstroke}{rgb}{0.000000,0.000000,0.000000}%
\pgfsetstrokecolor{currentstroke}%
\pgfsetdash{}{0pt}%
\pgfpathmoveto{\pgfqpoint{7.712653in}{0.790446in}}%
\pgfpathlineto{\pgfqpoint{7.712653in}{2.163173in}}%
\pgfusepath{stroke}%
\end{pgfscope}%
\begin{pgfscope}%
\pgfsetrectcap%
\pgfsetmiterjoin%
\pgfsetlinewidth{0.803000pt}%
\definecolor{currentstroke}{rgb}{0.000000,0.000000,0.000000}%
\pgfsetstrokecolor{currentstroke}%
\pgfsetdash{}{0pt}%
\pgfpathmoveto{\pgfqpoint{5.814694in}{0.790446in}}%
\pgfpathlineto{\pgfqpoint{7.712653in}{0.790446in}}%
\pgfusepath{stroke}%
\end{pgfscope}%
\begin{pgfscope}%
\pgfsetrectcap%
\pgfsetmiterjoin%
\pgfsetlinewidth{0.803000pt}%
\definecolor{currentstroke}{rgb}{0.000000,0.000000,0.000000}%
\pgfsetstrokecolor{currentstroke}%
\pgfsetdash{}{0pt}%
\pgfpathmoveto{\pgfqpoint{5.814694in}{2.163173in}}%
\pgfpathlineto{\pgfqpoint{7.712653in}{2.163173in}}%
\pgfusepath{stroke}%
\end{pgfscope}%
\begin{pgfscope}%
\pgfsetbuttcap%
\pgfsetmiterjoin%
\definecolor{currentfill}{rgb}{1.000000,1.000000,1.000000}%
\pgfsetfillcolor{currentfill}%
\pgfsetlinewidth{0.000000pt}%
\definecolor{currentstroke}{rgb}{0.000000,0.000000,0.000000}%
\pgfsetstrokecolor{currentstroke}%
\pgfsetstrokeopacity{0.000000}%
\pgfsetdash{}{0pt}%
\pgfpathmoveto{\pgfqpoint{8.282041in}{0.790446in}}%
\pgfpathlineto{\pgfqpoint{10.180000in}{0.790446in}}%
\pgfpathlineto{\pgfqpoint{10.180000in}{2.163173in}}%
\pgfpathlineto{\pgfqpoint{8.282041in}{2.163173in}}%
\pgfpathclose%
\pgfusepath{fill}%
\end{pgfscope}%
\begin{pgfscope}%
\pgfsetbuttcap%
\pgfsetroundjoin%
\definecolor{currentfill}{rgb}{0.000000,0.000000,0.000000}%
\pgfsetfillcolor{currentfill}%
\pgfsetlinewidth{0.803000pt}%
\definecolor{currentstroke}{rgb}{0.000000,0.000000,0.000000}%
\pgfsetstrokecolor{currentstroke}%
\pgfsetdash{}{0pt}%
\pgfsys@defobject{currentmarker}{\pgfqpoint{0.000000in}{-0.048611in}}{\pgfqpoint{0.000000in}{0.000000in}}{%
\pgfpathmoveto{\pgfqpoint{0.000000in}{0.000000in}}%
\pgfpathlineto{\pgfqpoint{0.000000in}{-0.048611in}}%
\pgfusepath{stroke,fill}%
}%
\begin{pgfscope}%
\pgfsys@transformshift{8.703810in}{0.790446in}%
\pgfsys@useobject{currentmarker}{}%
\end{pgfscope}%
\end{pgfscope}%
\begin{pgfscope}%
\pgftext[x=8.703810in,y=0.693224in,,top]{\rmfamily\fontsize{10.000000}{12.000000}\selectfont \(\displaystyle 0.25\)}%
\end{pgfscope}%
\begin{pgfscope}%
\pgfsetbuttcap%
\pgfsetroundjoin%
\definecolor{currentfill}{rgb}{0.000000,0.000000,0.000000}%
\pgfsetfillcolor{currentfill}%
\pgfsetlinewidth{0.803000pt}%
\definecolor{currentstroke}{rgb}{0.000000,0.000000,0.000000}%
\pgfsetstrokecolor{currentstroke}%
\pgfsetdash{}{0pt}%
\pgfsys@defobject{currentmarker}{\pgfqpoint{0.000000in}{-0.048611in}}{\pgfqpoint{0.000000in}{0.000000in}}{%
\pgfpathmoveto{\pgfqpoint{0.000000in}{0.000000in}}%
\pgfpathlineto{\pgfqpoint{0.000000in}{-0.048611in}}%
\pgfusepath{stroke,fill}%
}%
\begin{pgfscope}%
\pgfsys@transformshift{9.183092in}{0.790446in}%
\pgfsys@useobject{currentmarker}{}%
\end{pgfscope}%
\end{pgfscope}%
\begin{pgfscope}%
\pgftext[x=9.183092in,y=0.693224in,,top]{\rmfamily\fontsize{10.000000}{12.000000}\selectfont \(\displaystyle 0.50\)}%
\end{pgfscope}%
\begin{pgfscope}%
\pgfsetbuttcap%
\pgfsetroundjoin%
\definecolor{currentfill}{rgb}{0.000000,0.000000,0.000000}%
\pgfsetfillcolor{currentfill}%
\pgfsetlinewidth{0.803000pt}%
\definecolor{currentstroke}{rgb}{0.000000,0.000000,0.000000}%
\pgfsetstrokecolor{currentstroke}%
\pgfsetdash{}{0pt}%
\pgfsys@defobject{currentmarker}{\pgfqpoint{0.000000in}{-0.048611in}}{\pgfqpoint{0.000000in}{0.000000in}}{%
\pgfpathmoveto{\pgfqpoint{0.000000in}{0.000000in}}%
\pgfpathlineto{\pgfqpoint{0.000000in}{-0.048611in}}%
\pgfusepath{stroke,fill}%
}%
\begin{pgfscope}%
\pgfsys@transformshift{9.662375in}{0.790446in}%
\pgfsys@useobject{currentmarker}{}%
\end{pgfscope}%
\end{pgfscope}%
\begin{pgfscope}%
\pgftext[x=9.662375in,y=0.693224in,,top]{\rmfamily\fontsize{10.000000}{12.000000}\selectfont \(\displaystyle 0.75\)}%
\end{pgfscope}%
\begin{pgfscope}%
\pgfsetbuttcap%
\pgfsetroundjoin%
\definecolor{currentfill}{rgb}{0.000000,0.000000,0.000000}%
\pgfsetfillcolor{currentfill}%
\pgfsetlinewidth{0.803000pt}%
\definecolor{currentstroke}{rgb}{0.000000,0.000000,0.000000}%
\pgfsetstrokecolor{currentstroke}%
\pgfsetdash{}{0pt}%
\pgfsys@defobject{currentmarker}{\pgfqpoint{0.000000in}{-0.048611in}}{\pgfqpoint{0.000000in}{0.000000in}}{%
\pgfpathmoveto{\pgfqpoint{0.000000in}{0.000000in}}%
\pgfpathlineto{\pgfqpoint{0.000000in}{-0.048611in}}%
\pgfusepath{stroke,fill}%
}%
\begin{pgfscope}%
\pgfsys@transformshift{10.141657in}{0.790446in}%
\pgfsys@useobject{currentmarker}{}%
\end{pgfscope}%
\end{pgfscope}%
\begin{pgfscope}%
\pgftext[x=10.141657in,y=0.693224in,,top]{\rmfamily\fontsize{10.000000}{12.000000}\selectfont \(\displaystyle 1.00\)}%
\end{pgfscope}%
\begin{pgfscope}%
\pgfsetbuttcap%
\pgfsetroundjoin%
\definecolor{currentfill}{rgb}{0.000000,0.000000,0.000000}%
\pgfsetfillcolor{currentfill}%
\pgfsetlinewidth{0.803000pt}%
\definecolor{currentstroke}{rgb}{0.000000,0.000000,0.000000}%
\pgfsetstrokecolor{currentstroke}%
\pgfsetdash{}{0pt}%
\pgfsys@defobject{currentmarker}{\pgfqpoint{-0.048611in}{0.000000in}}{\pgfqpoint{0.000000in}{0.000000in}}{%
\pgfpathmoveto{\pgfqpoint{0.000000in}{0.000000in}}%
\pgfpathlineto{\pgfqpoint{-0.048611in}{0.000000in}}%
\pgfusepath{stroke,fill}%
}%
\begin{pgfscope}%
\pgfsys@transformshift{8.282041in}{0.944172in}%
\pgfsys@useobject{currentmarker}{}%
\end{pgfscope}%
\end{pgfscope}%
\begin{pgfscope}%
\pgftext[x=7.896816in,y=0.891410in,left,base]{\rmfamily\fontsize{10.000000}{12.000000}\selectfont \(\displaystyle 10^{-7}\)}%
\end{pgfscope}%
\begin{pgfscope}%
\pgfsetbuttcap%
\pgfsetroundjoin%
\definecolor{currentfill}{rgb}{0.000000,0.000000,0.000000}%
\pgfsetfillcolor{currentfill}%
\pgfsetlinewidth{0.803000pt}%
\definecolor{currentstroke}{rgb}{0.000000,0.000000,0.000000}%
\pgfsetstrokecolor{currentstroke}%
\pgfsetdash{}{0pt}%
\pgfsys@defobject{currentmarker}{\pgfqpoint{-0.048611in}{0.000000in}}{\pgfqpoint{0.000000in}{0.000000in}}{%
\pgfpathmoveto{\pgfqpoint{0.000000in}{0.000000in}}%
\pgfpathlineto{\pgfqpoint{-0.048611in}{0.000000in}}%
\pgfusepath{stroke,fill}%
}%
\begin{pgfscope}%
\pgfsys@transformshift{8.282041in}{1.470685in}%
\pgfsys@useobject{currentmarker}{}%
\end{pgfscope}%
\end{pgfscope}%
\begin{pgfscope}%
\pgftext[x=7.896816in,y=1.417923in,left,base]{\rmfamily\fontsize{10.000000}{12.000000}\selectfont \(\displaystyle 10^{-6}\)}%
\end{pgfscope}%
\begin{pgfscope}%
\pgfsetbuttcap%
\pgfsetroundjoin%
\definecolor{currentfill}{rgb}{0.000000,0.000000,0.000000}%
\pgfsetfillcolor{currentfill}%
\pgfsetlinewidth{0.803000pt}%
\definecolor{currentstroke}{rgb}{0.000000,0.000000,0.000000}%
\pgfsetstrokecolor{currentstroke}%
\pgfsetdash{}{0pt}%
\pgfsys@defobject{currentmarker}{\pgfqpoint{-0.048611in}{0.000000in}}{\pgfqpoint{0.000000in}{0.000000in}}{%
\pgfpathmoveto{\pgfqpoint{0.000000in}{0.000000in}}%
\pgfpathlineto{\pgfqpoint{-0.048611in}{0.000000in}}%
\pgfusepath{stroke,fill}%
}%
\begin{pgfscope}%
\pgfsys@transformshift{8.282041in}{1.997198in}%
\pgfsys@useobject{currentmarker}{}%
\end{pgfscope}%
\end{pgfscope}%
\begin{pgfscope}%
\pgftext[x=7.896816in,y=1.944436in,left,base]{\rmfamily\fontsize{10.000000}{12.000000}\selectfont \(\displaystyle 10^{-5}\)}%
\end{pgfscope}%
\begin{pgfscope}%
\pgfsetbuttcap%
\pgfsetroundjoin%
\definecolor{currentfill}{rgb}{0.000000,0.000000,0.000000}%
\pgfsetfillcolor{currentfill}%
\pgfsetlinewidth{0.602250pt}%
\definecolor{currentstroke}{rgb}{0.000000,0.000000,0.000000}%
\pgfsetstrokecolor{currentstroke}%
\pgfsetdash{}{0pt}%
\pgfsys@defobject{currentmarker}{\pgfqpoint{-0.027778in}{0.000000in}}{\pgfqpoint{0.000000in}{0.000000in}}{%
\pgfpathmoveto{\pgfqpoint{0.000000in}{0.000000in}}%
\pgfpathlineto{\pgfqpoint{-0.027778in}{0.000000in}}%
\pgfusepath{stroke,fill}%
}%
\begin{pgfscope}%
\pgfsys@transformshift{8.282041in}{0.785675in}%
\pgfsys@useobject{currentmarker}{}%
\end{pgfscope}%
\end{pgfscope}%
\begin{pgfscope}%
\pgfsetbuttcap%
\pgfsetroundjoin%
\definecolor{currentfill}{rgb}{0.000000,0.000000,0.000000}%
\pgfsetfillcolor{currentfill}%
\pgfsetlinewidth{0.602250pt}%
\definecolor{currentstroke}{rgb}{0.000000,0.000000,0.000000}%
\pgfsetstrokecolor{currentstroke}%
\pgfsetdash{}{0pt}%
\pgfsys@defobject{currentmarker}{\pgfqpoint{-0.027778in}{0.000000in}}{\pgfqpoint{0.000000in}{0.000000in}}{%
\pgfpathmoveto{\pgfqpoint{0.000000in}{0.000000in}}%
\pgfpathlineto{\pgfqpoint{-0.027778in}{0.000000in}}%
\pgfusepath{stroke,fill}%
}%
\begin{pgfscope}%
\pgfsys@transformshift{8.282041in}{0.827365in}%
\pgfsys@useobject{currentmarker}{}%
\end{pgfscope}%
\end{pgfscope}%
\begin{pgfscope}%
\pgfsetbuttcap%
\pgfsetroundjoin%
\definecolor{currentfill}{rgb}{0.000000,0.000000,0.000000}%
\pgfsetfillcolor{currentfill}%
\pgfsetlinewidth{0.602250pt}%
\definecolor{currentstroke}{rgb}{0.000000,0.000000,0.000000}%
\pgfsetstrokecolor{currentstroke}%
\pgfsetdash{}{0pt}%
\pgfsys@defobject{currentmarker}{\pgfqpoint{-0.027778in}{0.000000in}}{\pgfqpoint{0.000000in}{0.000000in}}{%
\pgfpathmoveto{\pgfqpoint{0.000000in}{0.000000in}}%
\pgfpathlineto{\pgfqpoint{-0.027778in}{0.000000in}}%
\pgfusepath{stroke,fill}%
}%
\begin{pgfscope}%
\pgfsys@transformshift{8.282041in}{0.862614in}%
\pgfsys@useobject{currentmarker}{}%
\end{pgfscope}%
\end{pgfscope}%
\begin{pgfscope}%
\pgfsetbuttcap%
\pgfsetroundjoin%
\definecolor{currentfill}{rgb}{0.000000,0.000000,0.000000}%
\pgfsetfillcolor{currentfill}%
\pgfsetlinewidth{0.602250pt}%
\definecolor{currentstroke}{rgb}{0.000000,0.000000,0.000000}%
\pgfsetstrokecolor{currentstroke}%
\pgfsetdash{}{0pt}%
\pgfsys@defobject{currentmarker}{\pgfqpoint{-0.027778in}{0.000000in}}{\pgfqpoint{0.000000in}{0.000000in}}{%
\pgfpathmoveto{\pgfqpoint{0.000000in}{0.000000in}}%
\pgfpathlineto{\pgfqpoint{-0.027778in}{0.000000in}}%
\pgfusepath{stroke,fill}%
}%
\begin{pgfscope}%
\pgfsys@transformshift{8.282041in}{0.893147in}%
\pgfsys@useobject{currentmarker}{}%
\end{pgfscope}%
\end{pgfscope}%
\begin{pgfscope}%
\pgfsetbuttcap%
\pgfsetroundjoin%
\definecolor{currentfill}{rgb}{0.000000,0.000000,0.000000}%
\pgfsetfillcolor{currentfill}%
\pgfsetlinewidth{0.602250pt}%
\definecolor{currentstroke}{rgb}{0.000000,0.000000,0.000000}%
\pgfsetstrokecolor{currentstroke}%
\pgfsetdash{}{0pt}%
\pgfsys@defobject{currentmarker}{\pgfqpoint{-0.027778in}{0.000000in}}{\pgfqpoint{0.000000in}{0.000000in}}{%
\pgfpathmoveto{\pgfqpoint{0.000000in}{0.000000in}}%
\pgfpathlineto{\pgfqpoint{-0.027778in}{0.000000in}}%
\pgfusepath{stroke,fill}%
}%
\begin{pgfscope}%
\pgfsys@transformshift{8.282041in}{0.920080in}%
\pgfsys@useobject{currentmarker}{}%
\end{pgfscope}%
\end{pgfscope}%
\begin{pgfscope}%
\pgfsetbuttcap%
\pgfsetroundjoin%
\definecolor{currentfill}{rgb}{0.000000,0.000000,0.000000}%
\pgfsetfillcolor{currentfill}%
\pgfsetlinewidth{0.602250pt}%
\definecolor{currentstroke}{rgb}{0.000000,0.000000,0.000000}%
\pgfsetstrokecolor{currentstroke}%
\pgfsetdash{}{0pt}%
\pgfsys@defobject{currentmarker}{\pgfqpoint{-0.027778in}{0.000000in}}{\pgfqpoint{0.000000in}{0.000000in}}{%
\pgfpathmoveto{\pgfqpoint{0.000000in}{0.000000in}}%
\pgfpathlineto{\pgfqpoint{-0.027778in}{0.000000in}}%
\pgfusepath{stroke,fill}%
}%
\begin{pgfscope}%
\pgfsys@transformshift{8.282041in}{1.102668in}%
\pgfsys@useobject{currentmarker}{}%
\end{pgfscope}%
\end{pgfscope}%
\begin{pgfscope}%
\pgfsetbuttcap%
\pgfsetroundjoin%
\definecolor{currentfill}{rgb}{0.000000,0.000000,0.000000}%
\pgfsetfillcolor{currentfill}%
\pgfsetlinewidth{0.602250pt}%
\definecolor{currentstroke}{rgb}{0.000000,0.000000,0.000000}%
\pgfsetstrokecolor{currentstroke}%
\pgfsetdash{}{0pt}%
\pgfsys@defobject{currentmarker}{\pgfqpoint{-0.027778in}{0.000000in}}{\pgfqpoint{0.000000in}{0.000000in}}{%
\pgfpathmoveto{\pgfqpoint{0.000000in}{0.000000in}}%
\pgfpathlineto{\pgfqpoint{-0.027778in}{0.000000in}}%
\pgfusepath{stroke,fill}%
}%
\begin{pgfscope}%
\pgfsys@transformshift{8.282041in}{1.195382in}%
\pgfsys@useobject{currentmarker}{}%
\end{pgfscope}%
\end{pgfscope}%
\begin{pgfscope}%
\pgfsetbuttcap%
\pgfsetroundjoin%
\definecolor{currentfill}{rgb}{0.000000,0.000000,0.000000}%
\pgfsetfillcolor{currentfill}%
\pgfsetlinewidth{0.602250pt}%
\definecolor{currentstroke}{rgb}{0.000000,0.000000,0.000000}%
\pgfsetstrokecolor{currentstroke}%
\pgfsetdash{}{0pt}%
\pgfsys@defobject{currentmarker}{\pgfqpoint{-0.027778in}{0.000000in}}{\pgfqpoint{0.000000in}{0.000000in}}{%
\pgfpathmoveto{\pgfqpoint{0.000000in}{0.000000in}}%
\pgfpathlineto{\pgfqpoint{-0.027778in}{0.000000in}}%
\pgfusepath{stroke,fill}%
}%
\begin{pgfscope}%
\pgfsys@transformshift{8.282041in}{1.261164in}%
\pgfsys@useobject{currentmarker}{}%
\end{pgfscope}%
\end{pgfscope}%
\begin{pgfscope}%
\pgfsetbuttcap%
\pgfsetroundjoin%
\definecolor{currentfill}{rgb}{0.000000,0.000000,0.000000}%
\pgfsetfillcolor{currentfill}%
\pgfsetlinewidth{0.602250pt}%
\definecolor{currentstroke}{rgb}{0.000000,0.000000,0.000000}%
\pgfsetstrokecolor{currentstroke}%
\pgfsetdash{}{0pt}%
\pgfsys@defobject{currentmarker}{\pgfqpoint{-0.027778in}{0.000000in}}{\pgfqpoint{0.000000in}{0.000000in}}{%
\pgfpathmoveto{\pgfqpoint{0.000000in}{0.000000in}}%
\pgfpathlineto{\pgfqpoint{-0.027778in}{0.000000in}}%
\pgfusepath{stroke,fill}%
}%
\begin{pgfscope}%
\pgfsys@transformshift{8.282041in}{1.312189in}%
\pgfsys@useobject{currentmarker}{}%
\end{pgfscope}%
\end{pgfscope}%
\begin{pgfscope}%
\pgfsetbuttcap%
\pgfsetroundjoin%
\definecolor{currentfill}{rgb}{0.000000,0.000000,0.000000}%
\pgfsetfillcolor{currentfill}%
\pgfsetlinewidth{0.602250pt}%
\definecolor{currentstroke}{rgb}{0.000000,0.000000,0.000000}%
\pgfsetstrokecolor{currentstroke}%
\pgfsetdash{}{0pt}%
\pgfsys@defobject{currentmarker}{\pgfqpoint{-0.027778in}{0.000000in}}{\pgfqpoint{0.000000in}{0.000000in}}{%
\pgfpathmoveto{\pgfqpoint{0.000000in}{0.000000in}}%
\pgfpathlineto{\pgfqpoint{-0.027778in}{0.000000in}}%
\pgfusepath{stroke,fill}%
}%
\begin{pgfscope}%
\pgfsys@transformshift{8.282041in}{1.353879in}%
\pgfsys@useobject{currentmarker}{}%
\end{pgfscope}%
\end{pgfscope}%
\begin{pgfscope}%
\pgfsetbuttcap%
\pgfsetroundjoin%
\definecolor{currentfill}{rgb}{0.000000,0.000000,0.000000}%
\pgfsetfillcolor{currentfill}%
\pgfsetlinewidth{0.602250pt}%
\definecolor{currentstroke}{rgb}{0.000000,0.000000,0.000000}%
\pgfsetstrokecolor{currentstroke}%
\pgfsetdash{}{0pt}%
\pgfsys@defobject{currentmarker}{\pgfqpoint{-0.027778in}{0.000000in}}{\pgfqpoint{0.000000in}{0.000000in}}{%
\pgfpathmoveto{\pgfqpoint{0.000000in}{0.000000in}}%
\pgfpathlineto{\pgfqpoint{-0.027778in}{0.000000in}}%
\pgfusepath{stroke,fill}%
}%
\begin{pgfscope}%
\pgfsys@transformshift{8.282041in}{1.389127in}%
\pgfsys@useobject{currentmarker}{}%
\end{pgfscope}%
\end{pgfscope}%
\begin{pgfscope}%
\pgfsetbuttcap%
\pgfsetroundjoin%
\definecolor{currentfill}{rgb}{0.000000,0.000000,0.000000}%
\pgfsetfillcolor{currentfill}%
\pgfsetlinewidth{0.602250pt}%
\definecolor{currentstroke}{rgb}{0.000000,0.000000,0.000000}%
\pgfsetstrokecolor{currentstroke}%
\pgfsetdash{}{0pt}%
\pgfsys@defobject{currentmarker}{\pgfqpoint{-0.027778in}{0.000000in}}{\pgfqpoint{0.000000in}{0.000000in}}{%
\pgfpathmoveto{\pgfqpoint{0.000000in}{0.000000in}}%
\pgfpathlineto{\pgfqpoint{-0.027778in}{0.000000in}}%
\pgfusepath{stroke,fill}%
}%
\begin{pgfscope}%
\pgfsys@transformshift{8.282041in}{1.419660in}%
\pgfsys@useobject{currentmarker}{}%
\end{pgfscope}%
\end{pgfscope}%
\begin{pgfscope}%
\pgfsetbuttcap%
\pgfsetroundjoin%
\definecolor{currentfill}{rgb}{0.000000,0.000000,0.000000}%
\pgfsetfillcolor{currentfill}%
\pgfsetlinewidth{0.602250pt}%
\definecolor{currentstroke}{rgb}{0.000000,0.000000,0.000000}%
\pgfsetstrokecolor{currentstroke}%
\pgfsetdash{}{0pt}%
\pgfsys@defobject{currentmarker}{\pgfqpoint{-0.027778in}{0.000000in}}{\pgfqpoint{0.000000in}{0.000000in}}{%
\pgfpathmoveto{\pgfqpoint{0.000000in}{0.000000in}}%
\pgfpathlineto{\pgfqpoint{-0.027778in}{0.000000in}}%
\pgfusepath{stroke,fill}%
}%
\begin{pgfscope}%
\pgfsys@transformshift{8.282041in}{1.446593in}%
\pgfsys@useobject{currentmarker}{}%
\end{pgfscope}%
\end{pgfscope}%
\begin{pgfscope}%
\pgfsetbuttcap%
\pgfsetroundjoin%
\definecolor{currentfill}{rgb}{0.000000,0.000000,0.000000}%
\pgfsetfillcolor{currentfill}%
\pgfsetlinewidth{0.602250pt}%
\definecolor{currentstroke}{rgb}{0.000000,0.000000,0.000000}%
\pgfsetstrokecolor{currentstroke}%
\pgfsetdash{}{0pt}%
\pgfsys@defobject{currentmarker}{\pgfqpoint{-0.027778in}{0.000000in}}{\pgfqpoint{0.000000in}{0.000000in}}{%
\pgfpathmoveto{\pgfqpoint{0.000000in}{0.000000in}}%
\pgfpathlineto{\pgfqpoint{-0.027778in}{0.000000in}}%
\pgfusepath{stroke,fill}%
}%
\begin{pgfscope}%
\pgfsys@transformshift{8.282041in}{1.629181in}%
\pgfsys@useobject{currentmarker}{}%
\end{pgfscope}%
\end{pgfscope}%
\begin{pgfscope}%
\pgfsetbuttcap%
\pgfsetroundjoin%
\definecolor{currentfill}{rgb}{0.000000,0.000000,0.000000}%
\pgfsetfillcolor{currentfill}%
\pgfsetlinewidth{0.602250pt}%
\definecolor{currentstroke}{rgb}{0.000000,0.000000,0.000000}%
\pgfsetstrokecolor{currentstroke}%
\pgfsetdash{}{0pt}%
\pgfsys@defobject{currentmarker}{\pgfqpoint{-0.027778in}{0.000000in}}{\pgfqpoint{0.000000in}{0.000000in}}{%
\pgfpathmoveto{\pgfqpoint{0.000000in}{0.000000in}}%
\pgfpathlineto{\pgfqpoint{-0.027778in}{0.000000in}}%
\pgfusepath{stroke,fill}%
}%
\begin{pgfscope}%
\pgfsys@transformshift{8.282041in}{1.721895in}%
\pgfsys@useobject{currentmarker}{}%
\end{pgfscope}%
\end{pgfscope}%
\begin{pgfscope}%
\pgfsetbuttcap%
\pgfsetroundjoin%
\definecolor{currentfill}{rgb}{0.000000,0.000000,0.000000}%
\pgfsetfillcolor{currentfill}%
\pgfsetlinewidth{0.602250pt}%
\definecolor{currentstroke}{rgb}{0.000000,0.000000,0.000000}%
\pgfsetstrokecolor{currentstroke}%
\pgfsetdash{}{0pt}%
\pgfsys@defobject{currentmarker}{\pgfqpoint{-0.027778in}{0.000000in}}{\pgfqpoint{0.000000in}{0.000000in}}{%
\pgfpathmoveto{\pgfqpoint{0.000000in}{0.000000in}}%
\pgfpathlineto{\pgfqpoint{-0.027778in}{0.000000in}}%
\pgfusepath{stroke,fill}%
}%
\begin{pgfscope}%
\pgfsys@transformshift{8.282041in}{1.787677in}%
\pgfsys@useobject{currentmarker}{}%
\end{pgfscope}%
\end{pgfscope}%
\begin{pgfscope}%
\pgfsetbuttcap%
\pgfsetroundjoin%
\definecolor{currentfill}{rgb}{0.000000,0.000000,0.000000}%
\pgfsetfillcolor{currentfill}%
\pgfsetlinewidth{0.602250pt}%
\definecolor{currentstroke}{rgb}{0.000000,0.000000,0.000000}%
\pgfsetstrokecolor{currentstroke}%
\pgfsetdash{}{0pt}%
\pgfsys@defobject{currentmarker}{\pgfqpoint{-0.027778in}{0.000000in}}{\pgfqpoint{0.000000in}{0.000000in}}{%
\pgfpathmoveto{\pgfqpoint{0.000000in}{0.000000in}}%
\pgfpathlineto{\pgfqpoint{-0.027778in}{0.000000in}}%
\pgfusepath{stroke,fill}%
}%
\begin{pgfscope}%
\pgfsys@transformshift{8.282041in}{1.838702in}%
\pgfsys@useobject{currentmarker}{}%
\end{pgfscope}%
\end{pgfscope}%
\begin{pgfscope}%
\pgfsetbuttcap%
\pgfsetroundjoin%
\definecolor{currentfill}{rgb}{0.000000,0.000000,0.000000}%
\pgfsetfillcolor{currentfill}%
\pgfsetlinewidth{0.602250pt}%
\definecolor{currentstroke}{rgb}{0.000000,0.000000,0.000000}%
\pgfsetstrokecolor{currentstroke}%
\pgfsetdash{}{0pt}%
\pgfsys@defobject{currentmarker}{\pgfqpoint{-0.027778in}{0.000000in}}{\pgfqpoint{0.000000in}{0.000000in}}{%
\pgfpathmoveto{\pgfqpoint{0.000000in}{0.000000in}}%
\pgfpathlineto{\pgfqpoint{-0.027778in}{0.000000in}}%
\pgfusepath{stroke,fill}%
}%
\begin{pgfscope}%
\pgfsys@transformshift{8.282041in}{1.880392in}%
\pgfsys@useobject{currentmarker}{}%
\end{pgfscope}%
\end{pgfscope}%
\begin{pgfscope}%
\pgfsetbuttcap%
\pgfsetroundjoin%
\definecolor{currentfill}{rgb}{0.000000,0.000000,0.000000}%
\pgfsetfillcolor{currentfill}%
\pgfsetlinewidth{0.602250pt}%
\definecolor{currentstroke}{rgb}{0.000000,0.000000,0.000000}%
\pgfsetstrokecolor{currentstroke}%
\pgfsetdash{}{0pt}%
\pgfsys@defobject{currentmarker}{\pgfqpoint{-0.027778in}{0.000000in}}{\pgfqpoint{0.000000in}{0.000000in}}{%
\pgfpathmoveto{\pgfqpoint{0.000000in}{0.000000in}}%
\pgfpathlineto{\pgfqpoint{-0.027778in}{0.000000in}}%
\pgfusepath{stroke,fill}%
}%
\begin{pgfscope}%
\pgfsys@transformshift{8.282041in}{1.915640in}%
\pgfsys@useobject{currentmarker}{}%
\end{pgfscope}%
\end{pgfscope}%
\begin{pgfscope}%
\pgfsetbuttcap%
\pgfsetroundjoin%
\definecolor{currentfill}{rgb}{0.000000,0.000000,0.000000}%
\pgfsetfillcolor{currentfill}%
\pgfsetlinewidth{0.602250pt}%
\definecolor{currentstroke}{rgb}{0.000000,0.000000,0.000000}%
\pgfsetstrokecolor{currentstroke}%
\pgfsetdash{}{0pt}%
\pgfsys@defobject{currentmarker}{\pgfqpoint{-0.027778in}{0.000000in}}{\pgfqpoint{0.000000in}{0.000000in}}{%
\pgfpathmoveto{\pgfqpoint{0.000000in}{0.000000in}}%
\pgfpathlineto{\pgfqpoint{-0.027778in}{0.000000in}}%
\pgfusepath{stroke,fill}%
}%
\begin{pgfscope}%
\pgfsys@transformshift{8.282041in}{1.946174in}%
\pgfsys@useobject{currentmarker}{}%
\end{pgfscope}%
\end{pgfscope}%
\begin{pgfscope}%
\pgfsetbuttcap%
\pgfsetroundjoin%
\definecolor{currentfill}{rgb}{0.000000,0.000000,0.000000}%
\pgfsetfillcolor{currentfill}%
\pgfsetlinewidth{0.602250pt}%
\definecolor{currentstroke}{rgb}{0.000000,0.000000,0.000000}%
\pgfsetstrokecolor{currentstroke}%
\pgfsetdash{}{0pt}%
\pgfsys@defobject{currentmarker}{\pgfqpoint{-0.027778in}{0.000000in}}{\pgfqpoint{0.000000in}{0.000000in}}{%
\pgfpathmoveto{\pgfqpoint{0.000000in}{0.000000in}}%
\pgfpathlineto{\pgfqpoint{-0.027778in}{0.000000in}}%
\pgfusepath{stroke,fill}%
}%
\begin{pgfscope}%
\pgfsys@transformshift{8.282041in}{1.973106in}%
\pgfsys@useobject{currentmarker}{}%
\end{pgfscope}%
\end{pgfscope}%
\begin{pgfscope}%
\pgfsetbuttcap%
\pgfsetroundjoin%
\definecolor{currentfill}{rgb}{0.000000,0.000000,0.000000}%
\pgfsetfillcolor{currentfill}%
\pgfsetlinewidth{0.602250pt}%
\definecolor{currentstroke}{rgb}{0.000000,0.000000,0.000000}%
\pgfsetstrokecolor{currentstroke}%
\pgfsetdash{}{0pt}%
\pgfsys@defobject{currentmarker}{\pgfqpoint{-0.027778in}{0.000000in}}{\pgfqpoint{0.000000in}{0.000000in}}{%
\pgfpathmoveto{\pgfqpoint{0.000000in}{0.000000in}}%
\pgfpathlineto{\pgfqpoint{-0.027778in}{0.000000in}}%
\pgfusepath{stroke,fill}%
}%
\begin{pgfscope}%
\pgfsys@transformshift{8.282041in}{2.155694in}%
\pgfsys@useobject{currentmarker}{}%
\end{pgfscope}%
\end{pgfscope}%
\begin{pgfscope}%
\pgfpathrectangle{\pgfqpoint{8.282041in}{0.790446in}}{\pgfqpoint{1.897959in}{1.372727in}} %
\pgfusepath{clip}%
\pgfsetbuttcap%
\pgfsetroundjoin%
\pgfsetlinewidth{1.505625pt}%
\definecolor{currentstroke}{rgb}{1.000000,0.000000,0.000000}%
\pgfsetstrokecolor{currentstroke}%
\pgfsetdash{{5.550000pt}{2.400000pt}}{0.000000pt}%
\pgfpathmoveto{\pgfqpoint{8.368312in}{2.007879in}}%
\pgfpathlineto{\pgfqpoint{8.384288in}{2.002324in}}%
\pgfpathlineto{\pgfqpoint{8.400264in}{1.996760in}}%
\pgfpathlineto{\pgfqpoint{8.416240in}{1.991191in}}%
\pgfpathlineto{\pgfqpoint{8.432216in}{1.985620in}}%
\pgfpathlineto{\pgfqpoint{8.448192in}{1.980048in}}%
\pgfpathlineto{\pgfqpoint{8.464168in}{1.974479in}}%
\pgfpathlineto{\pgfqpoint{8.480144in}{1.968915in}}%
\pgfpathlineto{\pgfqpoint{8.496120in}{1.963359in}}%
\pgfpathlineto{\pgfqpoint{8.512096in}{1.957814in}}%
\pgfpathlineto{\pgfqpoint{8.528073in}{1.952281in}}%
\pgfpathlineto{\pgfqpoint{8.544049in}{1.946763in}}%
\pgfpathlineto{\pgfqpoint{8.560025in}{1.941261in}}%
\pgfpathlineto{\pgfqpoint{8.576001in}{1.935778in}}%
\pgfpathlineto{\pgfqpoint{8.591977in}{1.930316in}}%
\pgfpathlineto{\pgfqpoint{8.607953in}{1.924876in}}%
\pgfpathlineto{\pgfqpoint{8.623929in}{1.919459in}}%
\pgfpathlineto{\pgfqpoint{8.639905in}{1.914068in}}%
\pgfpathlineto{\pgfqpoint{8.655881in}{1.908703in}}%
\pgfpathlineto{\pgfqpoint{8.671857in}{1.903366in}}%
\pgfpathlineto{\pgfqpoint{8.687833in}{1.898057in}}%
\pgfpathlineto{\pgfqpoint{8.703810in}{1.892779in}}%
\pgfpathlineto{\pgfqpoint{8.719786in}{1.887531in}}%
\pgfpathlineto{\pgfqpoint{8.735762in}{1.882315in}}%
\pgfpathlineto{\pgfqpoint{8.751738in}{1.877131in}}%
\pgfpathlineto{\pgfqpoint{8.767714in}{1.871980in}}%
\pgfpathlineto{\pgfqpoint{8.783690in}{1.866863in}}%
\pgfpathlineto{\pgfqpoint{8.799666in}{1.861780in}}%
\pgfpathlineto{\pgfqpoint{8.815642in}{1.856731in}}%
\pgfpathlineto{\pgfqpoint{8.831618in}{1.851718in}}%
\pgfpathlineto{\pgfqpoint{8.847594in}{1.846739in}}%
\pgfpathlineto{\pgfqpoint{8.863570in}{1.841797in}}%
\pgfpathlineto{\pgfqpoint{8.879546in}{1.836890in}}%
\pgfpathlineto{\pgfqpoint{8.895523in}{1.832019in}}%
\pgfpathlineto{\pgfqpoint{8.911499in}{1.827184in}}%
\pgfpathlineto{\pgfqpoint{8.927475in}{1.822386in}}%
\pgfpathlineto{\pgfqpoint{8.943451in}{1.817623in}}%
\pgfpathlineto{\pgfqpoint{8.959427in}{1.812897in}}%
\pgfpathlineto{\pgfqpoint{8.975403in}{1.808208in}}%
\pgfpathlineto{\pgfqpoint{8.991379in}{1.803554in}}%
\pgfpathlineto{\pgfqpoint{9.007355in}{1.798937in}}%
\pgfpathlineto{\pgfqpoint{9.023331in}{1.794356in}}%
\pgfpathlineto{\pgfqpoint{9.039307in}{1.789810in}}%
\pgfpathlineto{\pgfqpoint{9.055283in}{1.785301in}}%
\pgfpathlineto{\pgfqpoint{9.071260in}{1.780827in}}%
\pgfpathlineto{\pgfqpoint{9.087236in}{1.776388in}}%
\pgfpathlineto{\pgfqpoint{9.103212in}{1.771985in}}%
\pgfpathlineto{\pgfqpoint{9.119188in}{1.767617in}}%
\pgfpathlineto{\pgfqpoint{9.135164in}{1.763283in}}%
\pgfpathlineto{\pgfqpoint{9.151140in}{1.758984in}}%
\pgfpathlineto{\pgfqpoint{9.167116in}{1.754720in}}%
\pgfpathlineto{\pgfqpoint{9.183092in}{1.750489in}}%
\pgfpathlineto{\pgfqpoint{9.199068in}{1.746292in}}%
\pgfpathlineto{\pgfqpoint{9.215044in}{1.742128in}}%
\pgfpathlineto{\pgfqpoint{9.231020in}{1.737998in}}%
\pgfpathlineto{\pgfqpoint{9.246996in}{1.733901in}}%
\pgfpathlineto{\pgfqpoint{9.262973in}{1.729836in}}%
\pgfpathlineto{\pgfqpoint{9.278949in}{1.725803in}}%
\pgfpathlineto{\pgfqpoint{9.294925in}{1.721803in}}%
\pgfpathlineto{\pgfqpoint{9.310901in}{1.717834in}}%
\pgfpathlineto{\pgfqpoint{9.326877in}{1.713896in}}%
\pgfpathlineto{\pgfqpoint{9.342853in}{1.709990in}}%
\pgfpathlineto{\pgfqpoint{9.358829in}{1.706114in}}%
\pgfpathlineto{\pgfqpoint{9.374805in}{1.702268in}}%
\pgfpathlineto{\pgfqpoint{9.390781in}{1.698453in}}%
\pgfpathlineto{\pgfqpoint{9.406757in}{1.694667in}}%
\pgfpathlineto{\pgfqpoint{9.422733in}{1.690911in}}%
\pgfpathlineto{\pgfqpoint{9.438710in}{1.687184in}}%
\pgfpathlineto{\pgfqpoint{9.454686in}{1.683486in}}%
\pgfpathlineto{\pgfqpoint{9.470662in}{1.679816in}}%
\pgfpathlineto{\pgfqpoint{9.486638in}{1.676175in}}%
\pgfpathlineto{\pgfqpoint{9.502614in}{1.672561in}}%
\pgfpathlineto{\pgfqpoint{9.518590in}{1.668975in}}%
\pgfpathlineto{\pgfqpoint{9.534566in}{1.665416in}}%
\pgfpathlineto{\pgfqpoint{9.550542in}{1.661884in}}%
\pgfpathlineto{\pgfqpoint{9.566518in}{1.658379in}}%
\pgfpathlineto{\pgfqpoint{9.582494in}{1.654900in}}%
\pgfpathlineto{\pgfqpoint{9.598470in}{1.651447in}}%
\pgfpathlineto{\pgfqpoint{9.614447in}{1.648020in}}%
\pgfpathlineto{\pgfqpoint{9.630423in}{1.644618in}}%
\pgfpathlineto{\pgfqpoint{9.646399in}{1.641242in}}%
\pgfpathlineto{\pgfqpoint{9.662375in}{1.637890in}}%
\pgfpathlineto{\pgfqpoint{9.678351in}{1.634563in}}%
\pgfpathlineto{\pgfqpoint{9.694327in}{1.631260in}}%
\pgfpathlineto{\pgfqpoint{9.710303in}{1.627981in}}%
\pgfpathlineto{\pgfqpoint{9.726279in}{1.624726in}}%
\pgfpathlineto{\pgfqpoint{9.742255in}{1.621494in}}%
\pgfpathlineto{\pgfqpoint{9.758231in}{1.618285in}}%
\pgfpathlineto{\pgfqpoint{9.774207in}{1.615100in}}%
\pgfpathlineto{\pgfqpoint{9.790183in}{1.611937in}}%
\pgfpathlineto{\pgfqpoint{9.806160in}{1.608796in}}%
\pgfpathlineto{\pgfqpoint{9.822136in}{1.605677in}}%
\pgfpathlineto{\pgfqpoint{9.838112in}{1.602581in}}%
\pgfpathlineto{\pgfqpoint{9.854088in}{1.599506in}}%
\pgfpathlineto{\pgfqpoint{9.870064in}{1.596452in}}%
\pgfpathlineto{\pgfqpoint{9.886040in}{1.593419in}}%
\pgfpathlineto{\pgfqpoint{9.902016in}{1.590408in}}%
\pgfpathlineto{\pgfqpoint{9.917992in}{1.587417in}}%
\pgfpathlineto{\pgfqpoint{9.933968in}{1.584446in}}%
\pgfpathlineto{\pgfqpoint{9.949944in}{1.581496in}}%
\pgfpathlineto{\pgfqpoint{9.965920in}{1.578565in}}%
\pgfpathlineto{\pgfqpoint{9.981897in}{1.575654in}}%
\pgfpathlineto{\pgfqpoint{9.997873in}{1.572763in}}%
\pgfpathlineto{\pgfqpoint{10.013849in}{1.569891in}}%
\pgfpathlineto{\pgfqpoint{10.029825in}{1.567037in}}%
\pgfpathlineto{\pgfqpoint{10.045801in}{1.564203in}}%
\pgfpathlineto{\pgfqpoint{10.061777in}{1.561388in}}%
\pgfpathlineto{\pgfqpoint{10.077753in}{1.558590in}}%
\pgfpathlineto{\pgfqpoint{10.093729in}{1.555811in}}%
\pgfusepath{stroke}%
\end{pgfscope}%
\begin{pgfscope}%
\pgfpathrectangle{\pgfqpoint{8.282041in}{0.790446in}}{\pgfqpoint{1.897959in}{1.372727in}} %
\pgfusepath{clip}%
\pgfsetbuttcap%
\pgfsetmiterjoin%
\definecolor{currentfill}{rgb}{1.000000,0.000000,0.000000}%
\pgfsetfillcolor{currentfill}%
\pgfsetlinewidth{1.003750pt}%
\definecolor{currentstroke}{rgb}{1.000000,0.000000,0.000000}%
\pgfsetstrokecolor{currentstroke}%
\pgfsetdash{}{0pt}%
\pgfsys@defobject{currentmarker}{\pgfqpoint{-0.041667in}{-0.041667in}}{\pgfqpoint{0.041667in}{0.041667in}}{%
\pgfpathmoveto{\pgfqpoint{-0.041667in}{-0.041667in}}%
\pgfpathlineto{\pgfqpoint{0.041667in}{-0.041667in}}%
\pgfpathlineto{\pgfqpoint{0.041667in}{0.041667in}}%
\pgfpathlineto{\pgfqpoint{-0.041667in}{0.041667in}}%
\pgfpathclose%
\pgfusepath{stroke,fill}%
}%
\begin{pgfscope}%
\pgfsys@transformshift{8.368312in}{2.007879in}%
\pgfsys@useobject{currentmarker}{}%
\end{pgfscope}%
\begin{pgfscope}%
\pgfsys@transformshift{8.719786in}{1.887531in}%
\pgfsys@useobject{currentmarker}{}%
\end{pgfscope}%
\begin{pgfscope}%
\pgfsys@transformshift{9.071260in}{1.780827in}%
\pgfsys@useobject{currentmarker}{}%
\end{pgfscope}%
\begin{pgfscope}%
\pgfsys@transformshift{9.422733in}{1.690911in}%
\pgfsys@useobject{currentmarker}{}%
\end{pgfscope}%
\begin{pgfscope}%
\pgfsys@transformshift{9.774207in}{1.615100in}%
\pgfsys@useobject{currentmarker}{}%
\end{pgfscope}%
\end{pgfscope}%
\begin{pgfscope}%
\pgfpathrectangle{\pgfqpoint{8.282041in}{0.790446in}}{\pgfqpoint{1.897959in}{1.372727in}} %
\pgfusepath{clip}%
\pgfsetrectcap%
\pgfsetroundjoin%
\pgfsetlinewidth{1.505625pt}%
\definecolor{currentstroke}{rgb}{0.000000,0.000000,1.000000}%
\pgfsetstrokecolor{currentstroke}%
\pgfsetdash{}{0pt}%
\pgfpathmoveto{\pgfqpoint{8.368312in}{1.531881in}}%
\pgfpathlineto{\pgfqpoint{8.384288in}{1.529662in}}%
\pgfpathlineto{\pgfqpoint{8.400264in}{1.527538in}}%
\pgfpathlineto{\pgfqpoint{8.416240in}{1.525496in}}%
\pgfpathlineto{\pgfqpoint{8.432216in}{1.523526in}}%
\pgfpathlineto{\pgfqpoint{8.448192in}{1.521620in}}%
\pgfpathlineto{\pgfqpoint{8.464168in}{1.519774in}}%
\pgfpathlineto{\pgfqpoint{8.480144in}{1.517981in}}%
\pgfpathlineto{\pgfqpoint{8.496120in}{1.516237in}}%
\pgfpathlineto{\pgfqpoint{8.512096in}{1.514540in}}%
\pgfpathlineto{\pgfqpoint{8.528073in}{1.512886in}}%
\pgfpathlineto{\pgfqpoint{8.544049in}{1.511272in}}%
\pgfpathlineto{\pgfqpoint{8.560025in}{1.509697in}}%
\pgfpathlineto{\pgfqpoint{8.576001in}{1.508158in}}%
\pgfpathlineto{\pgfqpoint{8.591977in}{1.506654in}}%
\pgfpathlineto{\pgfqpoint{8.607953in}{1.505182in}}%
\pgfpathlineto{\pgfqpoint{8.623929in}{1.503742in}}%
\pgfpathlineto{\pgfqpoint{8.639905in}{1.502333in}}%
\pgfpathlineto{\pgfqpoint{8.655881in}{1.500952in}}%
\pgfpathlineto{\pgfqpoint{8.671857in}{1.499600in}}%
\pgfpathlineto{\pgfqpoint{8.687833in}{1.498274in}}%
\pgfpathlineto{\pgfqpoint{8.703810in}{1.496975in}}%
\pgfpathlineto{\pgfqpoint{8.719786in}{1.495700in}}%
\pgfpathlineto{\pgfqpoint{8.735762in}{1.494449in}}%
\pgfpathlineto{\pgfqpoint{8.751738in}{1.493222in}}%
\pgfpathlineto{\pgfqpoint{8.767714in}{1.492018in}}%
\pgfpathlineto{\pgfqpoint{8.783690in}{1.490835in}}%
\pgfpathlineto{\pgfqpoint{8.799666in}{1.489674in}}%
\pgfpathlineto{\pgfqpoint{8.815642in}{1.488533in}}%
\pgfpathlineto{\pgfqpoint{8.831618in}{1.487412in}}%
\pgfpathlineto{\pgfqpoint{8.847594in}{1.486310in}}%
\pgfpathlineto{\pgfqpoint{8.863570in}{1.485227in}}%
\pgfpathlineto{\pgfqpoint{8.879546in}{1.484163in}}%
\pgfpathlineto{\pgfqpoint{8.895523in}{1.483116in}}%
\pgfpathlineto{\pgfqpoint{8.911499in}{1.482086in}}%
\pgfpathlineto{\pgfqpoint{8.927475in}{1.481072in}}%
\pgfpathlineto{\pgfqpoint{8.943451in}{1.480076in}}%
\pgfpathlineto{\pgfqpoint{8.959427in}{1.479094in}}%
\pgfpathlineto{\pgfqpoint{8.975403in}{1.478129in}}%
\pgfpathlineto{\pgfqpoint{8.991379in}{1.477178in}}%
\pgfpathlineto{\pgfqpoint{9.007355in}{1.476241in}}%
\pgfpathlineto{\pgfqpoint{9.023331in}{1.475319in}}%
\pgfpathlineto{\pgfqpoint{9.039307in}{1.474410in}}%
\pgfpathlineto{\pgfqpoint{9.055283in}{1.473515in}}%
\pgfpathlineto{\pgfqpoint{9.071260in}{1.472633in}}%
\pgfpathlineto{\pgfqpoint{9.087236in}{1.471764in}}%
\pgfpathlineto{\pgfqpoint{9.103212in}{1.470907in}}%
\pgfpathlineto{\pgfqpoint{9.119188in}{1.470062in}}%
\pgfpathlineto{\pgfqpoint{9.135164in}{1.469229in}}%
\pgfpathlineto{\pgfqpoint{9.151140in}{1.468407in}}%
\pgfpathlineto{\pgfqpoint{9.167116in}{1.467596in}}%
\pgfpathlineto{\pgfqpoint{9.183092in}{1.466797in}}%
\pgfpathlineto{\pgfqpoint{9.199068in}{1.466007in}}%
\pgfpathlineto{\pgfqpoint{9.215044in}{1.465229in}}%
\pgfpathlineto{\pgfqpoint{9.231020in}{1.464460in}}%
\pgfpathlineto{\pgfqpoint{9.246996in}{1.463701in}}%
\pgfpathlineto{\pgfqpoint{9.262973in}{1.462952in}}%
\pgfpathlineto{\pgfqpoint{9.278949in}{1.462213in}}%
\pgfpathlineto{\pgfqpoint{9.294925in}{1.461482in}}%
\pgfpathlineto{\pgfqpoint{9.310901in}{1.460761in}}%
\pgfpathlineto{\pgfqpoint{9.326877in}{1.460048in}}%
\pgfpathlineto{\pgfqpoint{9.342853in}{1.459344in}}%
\pgfpathlineto{\pgfqpoint{9.358829in}{1.458648in}}%
\pgfpathlineto{\pgfqpoint{9.374805in}{1.457960in}}%
\pgfpathlineto{\pgfqpoint{9.390781in}{1.457280in}}%
\pgfpathlineto{\pgfqpoint{9.406757in}{1.456608in}}%
\pgfpathlineto{\pgfqpoint{9.422733in}{1.455944in}}%
\pgfpathlineto{\pgfqpoint{9.438710in}{1.455287in}}%
\pgfpathlineto{\pgfqpoint{9.454686in}{1.454637in}}%
\pgfpathlineto{\pgfqpoint{9.470662in}{1.453995in}}%
\pgfpathlineto{\pgfqpoint{9.486638in}{1.453359in}}%
\pgfpathlineto{\pgfqpoint{9.502614in}{1.452731in}}%
\pgfpathlineto{\pgfqpoint{9.518590in}{1.452109in}}%
\pgfpathlineto{\pgfqpoint{9.534566in}{1.451493in}}%
\pgfpathlineto{\pgfqpoint{9.550542in}{1.450884in}}%
\pgfpathlineto{\pgfqpoint{9.566518in}{1.450281in}}%
\pgfpathlineto{\pgfqpoint{9.582494in}{1.449684in}}%
\pgfpathlineto{\pgfqpoint{9.598470in}{1.449094in}}%
\pgfpathlineto{\pgfqpoint{9.614447in}{1.448509in}}%
\pgfpathlineto{\pgfqpoint{9.630423in}{1.447930in}}%
\pgfpathlineto{\pgfqpoint{9.646399in}{1.447356in}}%
\pgfpathlineto{\pgfqpoint{9.662375in}{1.446788in}}%
\pgfpathlineto{\pgfqpoint{9.678351in}{1.446226in}}%
\pgfpathlineto{\pgfqpoint{9.694327in}{1.445669in}}%
\pgfpathlineto{\pgfqpoint{9.710303in}{1.445117in}}%
\pgfpathlineto{\pgfqpoint{9.726279in}{1.444570in}}%
\pgfpathlineto{\pgfqpoint{9.742255in}{1.444028in}}%
\pgfpathlineto{\pgfqpoint{9.758231in}{1.443491in}}%
\pgfpathlineto{\pgfqpoint{9.774207in}{1.442958in}}%
\pgfpathlineto{\pgfqpoint{9.790183in}{1.442431in}}%
\pgfpathlineto{\pgfqpoint{9.806160in}{1.441908in}}%
\pgfpathlineto{\pgfqpoint{9.822136in}{1.441390in}}%
\pgfpathlineto{\pgfqpoint{9.838112in}{1.440876in}}%
\pgfpathlineto{\pgfqpoint{9.854088in}{1.440366in}}%
\pgfpathlineto{\pgfqpoint{9.870064in}{1.439861in}}%
\pgfpathlineto{\pgfqpoint{9.886040in}{1.439360in}}%
\pgfpathlineto{\pgfqpoint{9.902016in}{1.438863in}}%
\pgfpathlineto{\pgfqpoint{9.917992in}{1.438370in}}%
\pgfpathlineto{\pgfqpoint{9.933968in}{1.437881in}}%
\pgfpathlineto{\pgfqpoint{9.949944in}{1.437396in}}%
\pgfpathlineto{\pgfqpoint{9.965920in}{1.436914in}}%
\pgfpathlineto{\pgfqpoint{9.981897in}{1.436437in}}%
\pgfpathlineto{\pgfqpoint{9.997873in}{1.435963in}}%
\pgfpathlineto{\pgfqpoint{10.013849in}{1.435493in}}%
\pgfpathlineto{\pgfqpoint{10.029825in}{1.435026in}}%
\pgfpathlineto{\pgfqpoint{10.045801in}{1.434563in}}%
\pgfpathlineto{\pgfqpoint{10.061777in}{1.434103in}}%
\pgfpathlineto{\pgfqpoint{10.077753in}{1.433647in}}%
\pgfpathlineto{\pgfqpoint{10.093729in}{1.433194in}}%
\pgfusepath{stroke}%
\end{pgfscope}%
\begin{pgfscope}%
\pgfpathrectangle{\pgfqpoint{8.282041in}{0.790446in}}{\pgfqpoint{1.897959in}{1.372727in}} %
\pgfusepath{clip}%
\pgfsetbuttcap%
\pgfsetroundjoin%
\definecolor{currentfill}{rgb}{0.000000,0.000000,1.000000}%
\pgfsetfillcolor{currentfill}%
\pgfsetlinewidth{1.003750pt}%
\definecolor{currentstroke}{rgb}{0.000000,0.000000,1.000000}%
\pgfsetstrokecolor{currentstroke}%
\pgfsetdash{}{0pt}%
\pgfsys@defobject{currentmarker}{\pgfqpoint{-0.041667in}{-0.041667in}}{\pgfqpoint{0.041667in}{0.041667in}}{%
\pgfpathmoveto{\pgfqpoint{0.000000in}{-0.041667in}}%
\pgfpathcurveto{\pgfqpoint{0.011050in}{-0.041667in}}{\pgfqpoint{0.021649in}{-0.037276in}}{\pgfqpoint{0.029463in}{-0.029463in}}%
\pgfpathcurveto{\pgfqpoint{0.037276in}{-0.021649in}}{\pgfqpoint{0.041667in}{-0.011050in}}{\pgfqpoint{0.041667in}{0.000000in}}%
\pgfpathcurveto{\pgfqpoint{0.041667in}{0.011050in}}{\pgfqpoint{0.037276in}{0.021649in}}{\pgfqpoint{0.029463in}{0.029463in}}%
\pgfpathcurveto{\pgfqpoint{0.021649in}{0.037276in}}{\pgfqpoint{0.011050in}{0.041667in}}{\pgfqpoint{0.000000in}{0.041667in}}%
\pgfpathcurveto{\pgfqpoint{-0.011050in}{0.041667in}}{\pgfqpoint{-0.021649in}{0.037276in}}{\pgfqpoint{-0.029463in}{0.029463in}}%
\pgfpathcurveto{\pgfqpoint{-0.037276in}{0.021649in}}{\pgfqpoint{-0.041667in}{0.011050in}}{\pgfqpoint{-0.041667in}{0.000000in}}%
\pgfpathcurveto{\pgfqpoint{-0.041667in}{-0.011050in}}{\pgfqpoint{-0.037276in}{-0.021649in}}{\pgfqpoint{-0.029463in}{-0.029463in}}%
\pgfpathcurveto{\pgfqpoint{-0.021649in}{-0.037276in}}{\pgfqpoint{-0.011050in}{-0.041667in}}{\pgfqpoint{0.000000in}{-0.041667in}}%
\pgfpathclose%
\pgfusepath{stroke,fill}%
}%
\begin{pgfscope}%
\pgfsys@transformshift{8.368312in}{1.531881in}%
\pgfsys@useobject{currentmarker}{}%
\end{pgfscope}%
\begin{pgfscope}%
\pgfsys@transformshift{8.719786in}{1.495700in}%
\pgfsys@useobject{currentmarker}{}%
\end{pgfscope}%
\begin{pgfscope}%
\pgfsys@transformshift{9.071260in}{1.472633in}%
\pgfsys@useobject{currentmarker}{}%
\end{pgfscope}%
\begin{pgfscope}%
\pgfsys@transformshift{9.422733in}{1.455944in}%
\pgfsys@useobject{currentmarker}{}%
\end{pgfscope}%
\begin{pgfscope}%
\pgfsys@transformshift{9.774207in}{1.442958in}%
\pgfsys@useobject{currentmarker}{}%
\end{pgfscope}%
\end{pgfscope}%
\begin{pgfscope}%
\pgfpathrectangle{\pgfqpoint{8.282041in}{0.790446in}}{\pgfqpoint{1.897959in}{1.372727in}} %
\pgfusepath{clip}%
\pgfsetbuttcap%
\pgfsetroundjoin%
\pgfsetlinewidth{1.505625pt}%
\definecolor{currentstroke}{rgb}{0.000000,0.750000,0.750000}%
\pgfsetstrokecolor{currentstroke}%
\pgfsetdash{{9.600000pt}{2.400000pt}{1.500000pt}{2.400000pt}}{0.000000pt}%
\pgfpathmoveto{\pgfqpoint{8.368312in}{1.786740in}}%
\pgfpathlineto{\pgfqpoint{8.384288in}{1.753759in}}%
\pgfpathlineto{\pgfqpoint{8.400264in}{1.723196in}}%
\pgfpathlineto{\pgfqpoint{8.416240in}{1.694725in}}%
\pgfpathlineto{\pgfqpoint{8.432216in}{1.668085in}}%
\pgfpathlineto{\pgfqpoint{8.448192in}{1.643057in}}%
\pgfpathlineto{\pgfqpoint{8.464168in}{1.619461in}}%
\pgfpathlineto{\pgfqpoint{8.480144in}{1.597147in}}%
\pgfpathlineto{\pgfqpoint{8.496120in}{1.575984in}}%
\pgfpathlineto{\pgfqpoint{8.512096in}{1.555862in}}%
\pgfpathlineto{\pgfqpoint{8.528073in}{1.536686in}}%
\pgfpathlineto{\pgfqpoint{8.544049in}{1.518373in}}%
\pgfpathlineto{\pgfqpoint{8.560025in}{1.500850in}}%
\pgfpathlineto{\pgfqpoint{8.576001in}{1.484054in}}%
\pgfpathlineto{\pgfqpoint{8.591977in}{1.467928in}}%
\pgfpathlineto{\pgfqpoint{8.607953in}{1.452421in}}%
\pgfpathlineto{\pgfqpoint{8.623929in}{1.437490in}}%
\pgfpathlineto{\pgfqpoint{8.639905in}{1.423093in}}%
\pgfpathlineto{\pgfqpoint{8.655881in}{1.409196in}}%
\pgfpathlineto{\pgfqpoint{8.671857in}{1.395764in}}%
\pgfpathlineto{\pgfqpoint{8.687833in}{1.382770in}}%
\pgfpathlineto{\pgfqpoint{8.703810in}{1.370185in}}%
\pgfpathlineto{\pgfqpoint{8.719786in}{1.357985in}}%
\pgfpathlineto{\pgfqpoint{8.735762in}{1.346149in}}%
\pgfpathlineto{\pgfqpoint{8.751738in}{1.334654in}}%
\pgfpathlineto{\pgfqpoint{8.767714in}{1.323484in}}%
\pgfpathlineto{\pgfqpoint{8.783690in}{1.312620in}}%
\pgfpathlineto{\pgfqpoint{8.799666in}{1.302046in}}%
\pgfpathlineto{\pgfqpoint{8.815642in}{1.291748in}}%
\pgfpathlineto{\pgfqpoint{8.831618in}{1.281711in}}%
\pgfpathlineto{\pgfqpoint{8.847594in}{1.271924in}}%
\pgfpathlineto{\pgfqpoint{8.863570in}{1.262374in}}%
\pgfpathlineto{\pgfqpoint{8.879546in}{1.253051in}}%
\pgfpathlineto{\pgfqpoint{8.895523in}{1.243943in}}%
\pgfpathlineto{\pgfqpoint{8.911499in}{1.235042in}}%
\pgfpathlineto{\pgfqpoint{8.927475in}{1.226339in}}%
\pgfpathlineto{\pgfqpoint{8.943451in}{1.217824in}}%
\pgfpathlineto{\pgfqpoint{8.959427in}{1.209491in}}%
\pgfpathlineto{\pgfqpoint{8.975403in}{1.201331in}}%
\pgfpathlineto{\pgfqpoint{8.991379in}{1.193339in}}%
\pgfpathlineto{\pgfqpoint{9.007355in}{1.185506in}}%
\pgfpathlineto{\pgfqpoint{9.023331in}{1.177828in}}%
\pgfpathlineto{\pgfqpoint{9.039307in}{1.170298in}}%
\pgfpathlineto{\pgfqpoint{9.055283in}{1.162910in}}%
\pgfpathlineto{\pgfqpoint{9.071260in}{1.155660in}}%
\pgfpathlineto{\pgfqpoint{9.087236in}{1.148543in}}%
\pgfpathlineto{\pgfqpoint{9.103212in}{1.141553in}}%
\pgfpathlineto{\pgfqpoint{9.119188in}{1.134687in}}%
\pgfpathlineto{\pgfqpoint{9.135164in}{1.127940in}}%
\pgfpathlineto{\pgfqpoint{9.151140in}{1.121308in}}%
\pgfpathlineto{\pgfqpoint{9.167116in}{1.114787in}}%
\pgfpathlineto{\pgfqpoint{9.183092in}{1.108375in}}%
\pgfpathlineto{\pgfqpoint{9.199068in}{1.102066in}}%
\pgfpathlineto{\pgfqpoint{9.215044in}{1.095859in}}%
\pgfpathlineto{\pgfqpoint{9.231020in}{1.089749in}}%
\pgfpathlineto{\pgfqpoint{9.246996in}{1.083735in}}%
\pgfpathlineto{\pgfqpoint{9.262973in}{1.077812in}}%
\pgfpathlineto{\pgfqpoint{9.278949in}{1.071979in}}%
\pgfpathlineto{\pgfqpoint{9.294925in}{1.066232in}}%
\pgfpathlineto{\pgfqpoint{9.310901in}{1.060570in}}%
\pgfpathlineto{\pgfqpoint{9.326877in}{1.054989in}}%
\pgfpathlineto{\pgfqpoint{9.342853in}{1.049488in}}%
\pgfpathlineto{\pgfqpoint{9.358829in}{1.044064in}}%
\pgfpathlineto{\pgfqpoint{9.374805in}{1.038715in}}%
\pgfpathlineto{\pgfqpoint{9.390781in}{1.033439in}}%
\pgfpathlineto{\pgfqpoint{9.406757in}{1.028235in}}%
\pgfpathlineto{\pgfqpoint{9.422733in}{1.023099in}}%
\pgfpathlineto{\pgfqpoint{9.438710in}{1.018032in}}%
\pgfpathlineto{\pgfqpoint{9.454686in}{1.013030in}}%
\pgfpathlineto{\pgfqpoint{9.470662in}{1.008092in}}%
\pgfpathlineto{\pgfqpoint{9.486638in}{1.003217in}}%
\pgfpathlineto{\pgfqpoint{9.502614in}{0.998402in}}%
\pgfpathlineto{\pgfqpoint{9.518590in}{0.993647in}}%
\pgfpathlineto{\pgfqpoint{9.534566in}{0.988950in}}%
\pgfpathlineto{\pgfqpoint{9.550542in}{0.984310in}}%
\pgfpathlineto{\pgfqpoint{9.566518in}{0.979725in}}%
\pgfpathlineto{\pgfqpoint{9.582494in}{0.975195in}}%
\pgfpathlineto{\pgfqpoint{9.598470in}{0.970717in}}%
\pgfpathlineto{\pgfqpoint{9.614447in}{0.966290in}}%
\pgfpathlineto{\pgfqpoint{9.630423in}{0.961914in}}%
\pgfpathlineto{\pgfqpoint{9.646399in}{0.957587in}}%
\pgfpathlineto{\pgfqpoint{9.662375in}{0.953309in}}%
\pgfpathlineto{\pgfqpoint{9.678351in}{0.949078in}}%
\pgfpathlineto{\pgfqpoint{9.694327in}{0.944892in}}%
\pgfpathlineto{\pgfqpoint{9.710303in}{0.940752in}}%
\pgfpathlineto{\pgfqpoint{9.726279in}{0.936657in}}%
\pgfpathlineto{\pgfqpoint{9.742255in}{0.932604in}}%
\pgfpathlineto{\pgfqpoint{9.758231in}{0.928594in}}%
\pgfpathlineto{\pgfqpoint{9.774207in}{0.924626in}}%
\pgfpathlineto{\pgfqpoint{9.790183in}{0.920698in}}%
\pgfpathlineto{\pgfqpoint{9.806160in}{0.916810in}}%
\pgfpathlineto{\pgfqpoint{9.822136in}{0.912961in}}%
\pgfpathlineto{\pgfqpoint{9.838112in}{0.909150in}}%
\pgfpathlineto{\pgfqpoint{9.854088in}{0.905377in}}%
\pgfpathlineto{\pgfqpoint{9.870064in}{0.901641in}}%
\pgfpathlineto{\pgfqpoint{9.886040in}{0.897941in}}%
\pgfpathlineto{\pgfqpoint{9.902016in}{0.894277in}}%
\pgfpathlineto{\pgfqpoint{9.917992in}{0.890647in}}%
\pgfpathlineto{\pgfqpoint{9.933968in}{0.887051in}}%
\pgfpathlineto{\pgfqpoint{9.949944in}{0.883489in}}%
\pgfpathlineto{\pgfqpoint{9.965920in}{0.879960in}}%
\pgfpathlineto{\pgfqpoint{9.981897in}{0.876463in}}%
\pgfpathlineto{\pgfqpoint{9.997873in}{0.872998in}}%
\pgfpathlineto{\pgfqpoint{10.013849in}{0.869564in}}%
\pgfpathlineto{\pgfqpoint{10.029825in}{0.866160in}}%
\pgfpathlineto{\pgfqpoint{10.045801in}{0.862787in}}%
\pgfpathlineto{\pgfqpoint{10.061777in}{0.859443in}}%
\pgfpathlineto{\pgfqpoint{10.077753in}{0.856129in}}%
\pgfpathlineto{\pgfqpoint{10.093729in}{0.852843in}}%
\pgfusepath{stroke}%
\end{pgfscope}%
\begin{pgfscope}%
\pgfpathrectangle{\pgfqpoint{8.282041in}{0.790446in}}{\pgfqpoint{1.897959in}{1.372727in}} %
\pgfusepath{clip}%
\pgfsetbuttcap%
\pgfsetmiterjoin%
\definecolor{currentfill}{rgb}{0.000000,0.750000,0.750000}%
\pgfsetfillcolor{currentfill}%
\pgfsetlinewidth{1.003750pt}%
\definecolor{currentstroke}{rgb}{0.000000,0.750000,0.750000}%
\pgfsetstrokecolor{currentstroke}%
\pgfsetdash{}{0pt}%
\pgfsys@defobject{currentmarker}{\pgfqpoint{-0.041667in}{-0.041667in}}{\pgfqpoint{0.041667in}{0.041667in}}{%
\pgfpathmoveto{\pgfqpoint{-0.000000in}{-0.041667in}}%
\pgfpathlineto{\pgfqpoint{0.041667in}{0.041667in}}%
\pgfpathlineto{\pgfqpoint{-0.041667in}{0.041667in}}%
\pgfpathclose%
\pgfusepath{stroke,fill}%
}%
\begin{pgfscope}%
\pgfsys@transformshift{8.368312in}{1.786740in}%
\pgfsys@useobject{currentmarker}{}%
\end{pgfscope}%
\begin{pgfscope}%
\pgfsys@transformshift{8.719786in}{1.357985in}%
\pgfsys@useobject{currentmarker}{}%
\end{pgfscope}%
\begin{pgfscope}%
\pgfsys@transformshift{9.071260in}{1.155660in}%
\pgfsys@useobject{currentmarker}{}%
\end{pgfscope}%
\begin{pgfscope}%
\pgfsys@transformshift{9.422733in}{1.023099in}%
\pgfsys@useobject{currentmarker}{}%
\end{pgfscope}%
\begin{pgfscope}%
\pgfsys@transformshift{9.774207in}{0.924626in}%
\pgfsys@useobject{currentmarker}{}%
\end{pgfscope}%
\end{pgfscope}%
\begin{pgfscope}%
\pgfpathrectangle{\pgfqpoint{8.282041in}{0.790446in}}{\pgfqpoint{1.897959in}{1.372727in}} %
\pgfusepath{clip}%
\pgfsetbuttcap%
\pgfsetroundjoin%
\pgfsetlinewidth{1.505625pt}%
\definecolor{currentstroke}{rgb}{0.000000,0.000000,0.000000}%
\pgfsetstrokecolor{currentstroke}%
\pgfsetdash{{1.500000pt}{2.475000pt}}{0.000000pt}%
\pgfpathmoveto{\pgfqpoint{8.368312in}{2.100776in}}%
\pgfpathlineto{\pgfqpoint{8.384288in}{2.089725in}}%
\pgfpathlineto{\pgfqpoint{8.400264in}{2.078890in}}%
\pgfpathlineto{\pgfqpoint{8.416240in}{2.068959in}}%
\pgfpathlineto{\pgfqpoint{8.432216in}{2.059746in}}%
\pgfpathlineto{\pgfqpoint{8.448192in}{2.051119in}}%
\pgfpathlineto{\pgfqpoint{8.464168in}{2.042976in}}%
\pgfpathlineto{\pgfqpoint{8.480144in}{2.035239in}}%
\pgfpathlineto{\pgfqpoint{8.496120in}{2.027843in}}%
\pgfpathlineto{\pgfqpoint{8.512096in}{2.020739in}}%
\pgfpathlineto{\pgfqpoint{8.528073in}{2.013887in}}%
\pgfpathlineto{\pgfqpoint{8.544049in}{2.007256in}}%
\pgfpathlineto{\pgfqpoint{8.560025in}{2.000819in}}%
\pgfpathlineto{\pgfqpoint{8.576001in}{1.994555in}}%
\pgfpathlineto{\pgfqpoint{8.591977in}{1.988447in}}%
\pgfpathlineto{\pgfqpoint{8.607953in}{1.982479in}}%
\pgfpathlineto{\pgfqpoint{8.623929in}{1.976638in}}%
\pgfpathlineto{\pgfqpoint{8.639905in}{1.970917in}}%
\pgfpathlineto{\pgfqpoint{8.655881in}{1.965304in}}%
\pgfpathlineto{\pgfqpoint{8.671857in}{1.959794in}}%
\pgfpathlineto{\pgfqpoint{8.687833in}{1.954379in}}%
\pgfpathlineto{\pgfqpoint{8.703810in}{1.949055in}}%
\pgfpathlineto{\pgfqpoint{8.719786in}{1.943816in}}%
\pgfpathlineto{\pgfqpoint{8.735762in}{1.938660in}}%
\pgfpathlineto{\pgfqpoint{8.751738in}{1.933580in}}%
\pgfpathlineto{\pgfqpoint{8.767714in}{1.928575in}}%
\pgfpathlineto{\pgfqpoint{8.783690in}{1.923643in}}%
\pgfpathlineto{\pgfqpoint{8.799666in}{1.918779in}}%
\pgfpathlineto{\pgfqpoint{8.815642in}{1.913984in}}%
\pgfpathlineto{\pgfqpoint{8.831618in}{1.909253in}}%
\pgfpathlineto{\pgfqpoint{8.847594in}{1.904586in}}%
\pgfpathlineto{\pgfqpoint{8.863570in}{1.899980in}}%
\pgfpathlineto{\pgfqpoint{8.879546in}{1.895435in}}%
\pgfpathlineto{\pgfqpoint{8.895523in}{1.890949in}}%
\pgfpathlineto{\pgfqpoint{8.911499in}{1.886521in}}%
\pgfpathlineto{\pgfqpoint{8.927475in}{1.882149in}}%
\pgfpathlineto{\pgfqpoint{8.943451in}{1.877831in}}%
\pgfpathlineto{\pgfqpoint{8.959427in}{1.873567in}}%
\pgfpathlineto{\pgfqpoint{8.975403in}{1.869357in}}%
\pgfpathlineto{\pgfqpoint{8.991379in}{1.865198in}}%
\pgfpathlineto{\pgfqpoint{9.007355in}{1.861090in}}%
\pgfpathlineto{\pgfqpoint{9.023331in}{1.857033in}}%
\pgfpathlineto{\pgfqpoint{9.039307in}{1.853026in}}%
\pgfpathlineto{\pgfqpoint{9.055283in}{1.849066in}}%
\pgfpathlineto{\pgfqpoint{9.071260in}{1.845154in}}%
\pgfpathlineto{\pgfqpoint{9.087236in}{1.841290in}}%
\pgfpathlineto{\pgfqpoint{9.103212in}{1.837470in}}%
\pgfpathlineto{\pgfqpoint{9.119188in}{1.833697in}}%
\pgfpathlineto{\pgfqpoint{9.135164in}{1.829969in}}%
\pgfpathlineto{\pgfqpoint{9.151140in}{1.826285in}}%
\pgfpathlineto{\pgfqpoint{9.167116in}{1.822643in}}%
\pgfpathlineto{\pgfqpoint{9.183092in}{1.819045in}}%
\pgfpathlineto{\pgfqpoint{9.199068in}{1.815488in}}%
\pgfpathlineto{\pgfqpoint{9.215044in}{1.811973in}}%
\pgfpathlineto{\pgfqpoint{9.231020in}{1.808499in}}%
\pgfpathlineto{\pgfqpoint{9.246996in}{1.805064in}}%
\pgfpathlineto{\pgfqpoint{9.262973in}{1.801668in}}%
\pgfpathlineto{\pgfqpoint{9.278949in}{1.798313in}}%
\pgfpathlineto{\pgfqpoint{9.294925in}{1.794997in}}%
\pgfpathlineto{\pgfqpoint{9.310901in}{1.791719in}}%
\pgfpathlineto{\pgfqpoint{9.326877in}{1.788476in}}%
\pgfpathlineto{\pgfqpoint{9.342853in}{1.785271in}}%
\pgfpathlineto{\pgfqpoint{9.358829in}{1.782102in}}%
\pgfpathlineto{\pgfqpoint{9.374805in}{1.778969in}}%
\pgfpathlineto{\pgfqpoint{9.390781in}{1.775871in}}%
\pgfpathlineto{\pgfqpoint{9.406757in}{1.772807in}}%
\pgfpathlineto{\pgfqpoint{9.422733in}{1.769777in}}%
\pgfpathlineto{\pgfqpoint{9.438710in}{1.766781in}}%
\pgfpathlineto{\pgfqpoint{9.454686in}{1.763819in}}%
\pgfpathlineto{\pgfqpoint{9.470662in}{1.760888in}}%
\pgfpathlineto{\pgfqpoint{9.486638in}{1.757990in}}%
\pgfpathlineto{\pgfqpoint{9.502614in}{1.755124in}}%
\pgfpathlineto{\pgfqpoint{9.518590in}{1.752289in}}%
\pgfpathlineto{\pgfqpoint{9.534566in}{1.749485in}}%
\pgfpathlineto{\pgfqpoint{9.550542in}{1.746711in}}%
\pgfpathlineto{\pgfqpoint{9.566518in}{1.743967in}}%
\pgfpathlineto{\pgfqpoint{9.582494in}{1.741252in}}%
\pgfpathlineto{\pgfqpoint{9.598470in}{1.738566in}}%
\pgfpathlineto{\pgfqpoint{9.614447in}{1.735909in}}%
\pgfpathlineto{\pgfqpoint{9.630423in}{1.733280in}}%
\pgfpathlineto{\pgfqpoint{9.646399in}{1.730679in}}%
\pgfpathlineto{\pgfqpoint{9.662375in}{1.728106in}}%
\pgfpathlineto{\pgfqpoint{9.678351in}{1.725560in}}%
\pgfpathlineto{\pgfqpoint{9.694327in}{1.723041in}}%
\pgfpathlineto{\pgfqpoint{9.710303in}{1.720547in}}%
\pgfpathlineto{\pgfqpoint{9.726279in}{1.718079in}}%
\pgfpathlineto{\pgfqpoint{9.742255in}{1.715637in}}%
\pgfpathlineto{\pgfqpoint{9.758231in}{1.713220in}}%
\pgfpathlineto{\pgfqpoint{9.774207in}{1.710828in}}%
\pgfpathlineto{\pgfqpoint{9.790183in}{1.708461in}}%
\pgfpathlineto{\pgfqpoint{9.806160in}{1.706118in}}%
\pgfpathlineto{\pgfqpoint{9.822136in}{1.703798in}}%
\pgfpathlineto{\pgfqpoint{9.838112in}{1.701502in}}%
\pgfpathlineto{\pgfqpoint{9.854088in}{1.699228in}}%
\pgfpathlineto{\pgfqpoint{9.870064in}{1.696978in}}%
\pgfpathlineto{\pgfqpoint{9.886040in}{1.694750in}}%
\pgfpathlineto{\pgfqpoint{9.902016in}{1.692545in}}%
\pgfpathlineto{\pgfqpoint{9.917992in}{1.690362in}}%
\pgfpathlineto{\pgfqpoint{9.933968in}{1.688199in}}%
\pgfpathlineto{\pgfqpoint{9.949944in}{1.686058in}}%
\pgfpathlineto{\pgfqpoint{9.965920in}{1.683938in}}%
\pgfpathlineto{\pgfqpoint{9.981897in}{1.681839in}}%
\pgfpathlineto{\pgfqpoint{9.997873in}{1.679760in}}%
\pgfpathlineto{\pgfqpoint{10.013849in}{1.677701in}}%
\pgfpathlineto{\pgfqpoint{10.029825in}{1.675663in}}%
\pgfpathlineto{\pgfqpoint{10.045801in}{1.673643in}}%
\pgfpathlineto{\pgfqpoint{10.061777in}{1.671643in}}%
\pgfpathlineto{\pgfqpoint{10.077753in}{1.669662in}}%
\pgfpathlineto{\pgfqpoint{10.093729in}{1.667700in}}%
\pgfusepath{stroke}%
\end{pgfscope}%
\begin{pgfscope}%
\pgfpathrectangle{\pgfqpoint{8.282041in}{0.790446in}}{\pgfqpoint{1.897959in}{1.372727in}} %
\pgfusepath{clip}%
\pgfsetbuttcap%
\pgfsetroundjoin%
\definecolor{currentfill}{rgb}{0.000000,0.000000,0.000000}%
\pgfsetfillcolor{currentfill}%
\pgfsetlinewidth{1.003750pt}%
\definecolor{currentstroke}{rgb}{0.000000,0.000000,0.000000}%
\pgfsetstrokecolor{currentstroke}%
\pgfsetdash{}{0pt}%
\pgfsys@defobject{currentmarker}{\pgfqpoint{-0.041667in}{-0.041667in}}{\pgfqpoint{0.041667in}{0.041667in}}{%
\pgfpathmoveto{\pgfqpoint{-0.041667in}{0.000000in}}%
\pgfpathlineto{\pgfqpoint{0.041667in}{0.000000in}}%
\pgfpathmoveto{\pgfqpoint{0.000000in}{-0.041667in}}%
\pgfpathlineto{\pgfqpoint{0.000000in}{0.041667in}}%
\pgfusepath{stroke,fill}%
}%
\begin{pgfscope}%
\pgfsys@transformshift{8.368312in}{2.100776in}%
\pgfsys@useobject{currentmarker}{}%
\end{pgfscope}%
\begin{pgfscope}%
\pgfsys@transformshift{8.719786in}{1.943816in}%
\pgfsys@useobject{currentmarker}{}%
\end{pgfscope}%
\begin{pgfscope}%
\pgfsys@transformshift{9.071260in}{1.845154in}%
\pgfsys@useobject{currentmarker}{}%
\end{pgfscope}%
\begin{pgfscope}%
\pgfsys@transformshift{9.422733in}{1.769777in}%
\pgfsys@useobject{currentmarker}{}%
\end{pgfscope}%
\begin{pgfscope}%
\pgfsys@transformshift{9.774207in}{1.710828in}%
\pgfsys@useobject{currentmarker}{}%
\end{pgfscope}%
\end{pgfscope}%
\begin{pgfscope}%
\pgfsetrectcap%
\pgfsetmiterjoin%
\pgfsetlinewidth{0.803000pt}%
\definecolor{currentstroke}{rgb}{0.000000,0.000000,0.000000}%
\pgfsetstrokecolor{currentstroke}%
\pgfsetdash{}{0pt}%
\pgfpathmoveto{\pgfqpoint{8.282041in}{0.790446in}}%
\pgfpathlineto{\pgfqpoint{8.282041in}{2.163173in}}%
\pgfusepath{stroke}%
\end{pgfscope}%
\begin{pgfscope}%
\pgfsetrectcap%
\pgfsetmiterjoin%
\pgfsetlinewidth{0.803000pt}%
\definecolor{currentstroke}{rgb}{0.000000,0.000000,0.000000}%
\pgfsetstrokecolor{currentstroke}%
\pgfsetdash{}{0pt}%
\pgfpathmoveto{\pgfqpoint{10.180000in}{0.790446in}}%
\pgfpathlineto{\pgfqpoint{10.180000in}{2.163173in}}%
\pgfusepath{stroke}%
\end{pgfscope}%
\begin{pgfscope}%
\pgfsetrectcap%
\pgfsetmiterjoin%
\pgfsetlinewidth{0.803000pt}%
\definecolor{currentstroke}{rgb}{0.000000,0.000000,0.000000}%
\pgfsetstrokecolor{currentstroke}%
\pgfsetdash{}{0pt}%
\pgfpathmoveto{\pgfqpoint{8.282041in}{0.790446in}}%
\pgfpathlineto{\pgfqpoint{10.180000in}{0.790446in}}%
\pgfusepath{stroke}%
\end{pgfscope}%
\begin{pgfscope}%
\pgfsetrectcap%
\pgfsetmiterjoin%
\pgfsetlinewidth{0.803000pt}%
\definecolor{currentstroke}{rgb}{0.000000,0.000000,0.000000}%
\pgfsetstrokecolor{currentstroke}%
\pgfsetdash{}{0pt}%
\pgfpathmoveto{\pgfqpoint{8.282041in}{2.163173in}}%
\pgfpathlineto{\pgfqpoint{10.180000in}{2.163173in}}%
\pgfusepath{stroke}%
\end{pgfscope}%
\begin{pgfscope}%
\pgfsetbuttcap%
\pgfsetmiterjoin%
\definecolor{currentfill}{rgb}{1.000000,1.000000,1.000000}%
\pgfsetfillcolor{currentfill}%
\pgfsetfillopacity{0.800000}%
\pgfsetlinewidth{1.003750pt}%
\definecolor{currentstroke}{rgb}{0.800000,0.800000,0.800000}%
\pgfsetstrokecolor{currentstroke}%
\pgfsetstrokeopacity{0.800000}%
\pgfsetdash{}{0pt}%
\pgfpathmoveto{\pgfqpoint{10.598481in}{4.153628in}}%
\pgfpathlineto{\pgfqpoint{12.667643in}{4.153628in}}%
\pgfpathquadraticcurveto{\pgfqpoint{12.706532in}{4.153628in}}{\pgfqpoint{12.706532in}{4.192517in}}%
\pgfpathlineto{\pgfqpoint{12.706532in}{5.320180in}}%
\pgfpathquadraticcurveto{\pgfqpoint{12.706532in}{5.359069in}}{\pgfqpoint{12.667643in}{5.359069in}}%
\pgfpathlineto{\pgfqpoint{10.598481in}{5.359069in}}%
\pgfpathquadraticcurveto{\pgfqpoint{10.559592in}{5.359069in}}{\pgfqpoint{10.559592in}{5.320180in}}%
\pgfpathlineto{\pgfqpoint{10.559592in}{4.192517in}}%
\pgfpathquadraticcurveto{\pgfqpoint{10.559592in}{4.153628in}}{\pgfqpoint{10.598481in}{4.153628in}}%
\pgfpathclose%
\pgfusepath{stroke,fill}%
\end{pgfscope}%
\begin{pgfscope}%
\pgfsetbuttcap%
\pgfsetroundjoin%
\pgfsetlinewidth{1.505625pt}%
\definecolor{currentstroke}{rgb}{1.000000,0.000000,0.000000}%
\pgfsetstrokecolor{currentstroke}%
\pgfsetdash{{5.550000pt}{2.400000pt}}{0.000000pt}%
\pgfpathmoveto{\pgfqpoint{10.637370in}{5.201614in}}%
\pgfpathlineto{\pgfqpoint{11.026259in}{5.201614in}}%
\pgfusepath{stroke}%
\end{pgfscope}%
\begin{pgfscope}%
\pgfsetbuttcap%
\pgfsetmiterjoin%
\definecolor{currentfill}{rgb}{1.000000,0.000000,0.000000}%
\pgfsetfillcolor{currentfill}%
\pgfsetlinewidth{1.003750pt}%
\definecolor{currentstroke}{rgb}{1.000000,0.000000,0.000000}%
\pgfsetstrokecolor{currentstroke}%
\pgfsetdash{}{0pt}%
\pgfsys@defobject{currentmarker}{\pgfqpoint{-0.041667in}{-0.041667in}}{\pgfqpoint{0.041667in}{0.041667in}}{%
\pgfpathmoveto{\pgfqpoint{-0.041667in}{-0.041667in}}%
\pgfpathlineto{\pgfqpoint{0.041667in}{-0.041667in}}%
\pgfpathlineto{\pgfqpoint{0.041667in}{0.041667in}}%
\pgfpathlineto{\pgfqpoint{-0.041667in}{0.041667in}}%
\pgfpathclose%
\pgfusepath{stroke,fill}%
}%
\begin{pgfscope}%
\pgfsys@transformshift{10.831814in}{5.201614in}%
\pgfsys@useobject{currentmarker}{}%
\end{pgfscope}%
\end{pgfscope}%
\begin{pgfscope}%
\pgftext[x=11.181814in,y=5.133559in,left,base]{\rmfamily\fontsize{14.000000}{16.800000}\selectfont Coulomb force}%
\end{pgfscope}%
\begin{pgfscope}%
\pgfsetrectcap%
\pgfsetroundjoin%
\pgfsetlinewidth{1.505625pt}%
\definecolor{currentstroke}{rgb}{0.000000,0.000000,1.000000}%
\pgfsetstrokecolor{currentstroke}%
\pgfsetdash{}{0pt}%
\pgfpathmoveto{\pgfqpoint{10.637370in}{4.916214in}}%
\pgfpathlineto{\pgfqpoint{11.026259in}{4.916214in}}%
\pgfusepath{stroke}%
\end{pgfscope}%
\begin{pgfscope}%
\pgfsetbuttcap%
\pgfsetroundjoin%
\definecolor{currentfill}{rgb}{0.000000,0.000000,1.000000}%
\pgfsetfillcolor{currentfill}%
\pgfsetlinewidth{1.003750pt}%
\definecolor{currentstroke}{rgb}{0.000000,0.000000,1.000000}%
\pgfsetstrokecolor{currentstroke}%
\pgfsetdash{}{0pt}%
\pgfsys@defobject{currentmarker}{\pgfqpoint{-0.041667in}{-0.041667in}}{\pgfqpoint{0.041667in}{0.041667in}}{%
\pgfpathmoveto{\pgfqpoint{0.000000in}{-0.041667in}}%
\pgfpathcurveto{\pgfqpoint{0.011050in}{-0.041667in}}{\pgfqpoint{0.021649in}{-0.037276in}}{\pgfqpoint{0.029463in}{-0.029463in}}%
\pgfpathcurveto{\pgfqpoint{0.037276in}{-0.021649in}}{\pgfqpoint{0.041667in}{-0.011050in}}{\pgfqpoint{0.041667in}{0.000000in}}%
\pgfpathcurveto{\pgfqpoint{0.041667in}{0.011050in}}{\pgfqpoint{0.037276in}{0.021649in}}{\pgfqpoint{0.029463in}{0.029463in}}%
\pgfpathcurveto{\pgfqpoint{0.021649in}{0.037276in}}{\pgfqpoint{0.011050in}{0.041667in}}{\pgfqpoint{0.000000in}{0.041667in}}%
\pgfpathcurveto{\pgfqpoint{-0.011050in}{0.041667in}}{\pgfqpoint{-0.021649in}{0.037276in}}{\pgfqpoint{-0.029463in}{0.029463in}}%
\pgfpathcurveto{\pgfqpoint{-0.037276in}{0.021649in}}{\pgfqpoint{-0.041667in}{0.011050in}}{\pgfqpoint{-0.041667in}{0.000000in}}%
\pgfpathcurveto{\pgfqpoint{-0.041667in}{-0.011050in}}{\pgfqpoint{-0.037276in}{-0.021649in}}{\pgfqpoint{-0.029463in}{-0.029463in}}%
\pgfpathcurveto{\pgfqpoint{-0.021649in}{-0.037276in}}{\pgfqpoint{-0.011050in}{-0.041667in}}{\pgfqpoint{0.000000in}{-0.041667in}}%
\pgfpathclose%
\pgfusepath{stroke,fill}%
}%
\begin{pgfscope}%
\pgfsys@transformshift{10.831814in}{4.916214in}%
\pgfsys@useobject{currentmarker}{}%
\end{pgfscope}%
\end{pgfscope}%
\begin{pgfscope}%
\pgftext[x=11.181814in,y=4.848158in,left,base]{\rmfamily\fontsize{14.000000}{16.800000}\selectfont Drag force}%
\end{pgfscope}%
\begin{pgfscope}%
\pgfsetbuttcap%
\pgfsetroundjoin%
\pgfsetlinewidth{1.505625pt}%
\definecolor{currentstroke}{rgb}{0.000000,0.750000,0.750000}%
\pgfsetstrokecolor{currentstroke}%
\pgfsetdash{{9.600000pt}{2.400000pt}{1.500000pt}{2.400000pt}}{0.000000pt}%
\pgfpathmoveto{\pgfqpoint{10.637370in}{4.628060in}}%
\pgfpathlineto{\pgfqpoint{11.026259in}{4.628060in}}%
\pgfusepath{stroke}%
\end{pgfscope}%
\begin{pgfscope}%
\pgfsetbuttcap%
\pgfsetmiterjoin%
\definecolor{currentfill}{rgb}{0.000000,0.750000,0.750000}%
\pgfsetfillcolor{currentfill}%
\pgfsetlinewidth{1.003750pt}%
\definecolor{currentstroke}{rgb}{0.000000,0.750000,0.750000}%
\pgfsetstrokecolor{currentstroke}%
\pgfsetdash{}{0pt}%
\pgfsys@defobject{currentmarker}{\pgfqpoint{-0.041667in}{-0.041667in}}{\pgfqpoint{0.041667in}{0.041667in}}{%
\pgfpathmoveto{\pgfqpoint{-0.000000in}{-0.041667in}}%
\pgfpathlineto{\pgfqpoint{0.041667in}{0.041667in}}%
\pgfpathlineto{\pgfqpoint{-0.041667in}{0.041667in}}%
\pgfpathclose%
\pgfusepath{stroke,fill}%
}%
\begin{pgfscope}%
\pgfsys@transformshift{10.831814in}{4.628060in}%
\pgfsys@useobject{currentmarker}{}%
\end{pgfscope}%
\end{pgfscope}%
\begin{pgfscope}%
\pgftext[x=11.181814in,y=4.560005in,left,base]{\rmfamily\fontsize{14.000000}{16.800000}\selectfont Image force}%
\end{pgfscope}%
\begin{pgfscope}%
\pgfsetbuttcap%
\pgfsetroundjoin%
\pgfsetlinewidth{1.505625pt}%
\definecolor{currentstroke}{rgb}{0.000000,0.000000,0.000000}%
\pgfsetstrokecolor{currentstroke}%
\pgfsetdash{{1.500000pt}{2.475000pt}}{0.000000pt}%
\pgfpathmoveto{\pgfqpoint{10.637370in}{4.339907in}}%
\pgfpathlineto{\pgfqpoint{11.026259in}{4.339907in}}%
\pgfusepath{stroke}%
\end{pgfscope}%
\begin{pgfscope}%
\pgfsetbuttcap%
\pgfsetroundjoin%
\definecolor{currentfill}{rgb}{0.000000,0.000000,0.000000}%
\pgfsetfillcolor{currentfill}%
\pgfsetlinewidth{1.003750pt}%
\definecolor{currentstroke}{rgb}{0.000000,0.000000,0.000000}%
\pgfsetstrokecolor{currentstroke}%
\pgfsetdash{}{0pt}%
\pgfsys@defobject{currentmarker}{\pgfqpoint{-0.041667in}{-0.041667in}}{\pgfqpoint{0.041667in}{0.041667in}}{%
\pgfpathmoveto{\pgfqpoint{-0.041667in}{0.000000in}}%
\pgfpathlineto{\pgfqpoint{0.041667in}{0.000000in}}%
\pgfpathmoveto{\pgfqpoint{0.000000in}{-0.041667in}}%
\pgfpathlineto{\pgfqpoint{0.000000in}{0.041667in}}%
\pgfusepath{stroke,fill}%
}%
\begin{pgfscope}%
\pgfsys@transformshift{10.831814in}{4.339907in}%
\pgfsys@useobject{currentmarker}{}%
\end{pgfscope}%
\end{pgfscope}%
\begin{pgfscope}%
\pgftext[x=11.181814in,y=4.271851in,left,base]{\rmfamily\fontsize{14.000000}{16.800000}\selectfont Inertia}%
\end{pgfscope}%
\begin{pgfscope}%
\pgftext[x=5.380000in,y=0.140446in,,base]{\rmfamily\fontsize{14.000000}{16.800000}\selectfont \(\displaystyle t\) (s)}%
\end{pgfscope}%
\begin{pgfscope}%
\pgftext[x=0.247732in,y=4.095635in,left,base,rotate=90.000000]{\rmfamily\fontsize{14.000000}{16.800000}\selectfont force (N)}%
\end{pgfscope}%
\end{pgfpicture}%
\makeatother%
\endgroup%
}
    \caption{A simple EMA plot.\label{fig:forces}}
\end{figure}

\begin{figure}[htb]
    \centering
    %% Creator: Matplotlib, PGF backend
%%
%% To include the figure in your LaTeX document, write
%%   \input{<filename>.pgf}
%%
%% Make sure the required packages are loaded in your preamble
%%   \usepackage{pgf}
%%
%% Figures using additional raster images can only be included by \input if
%% they are in the same directory as the main LaTeX file. For loading figures
%% from other directories you can use the `import` package
%%   \usepackage{import}
%% and then include the figures with
%%   \import{<path to file>}{<filename>.pgf}
%%
%% Matplotlib used the following preamble
%%   \usepackage{fontspec}
%%   \setmainfont{DejaVu Serif}
%%   \setsansfont{DejaVu Sans}
%%   \setmonofont{DejaVu Sans Mono}
%%
\begingroup%
\makeatletter%
\begin{pgfpicture}%
\pgfpathrectangle{\pgfpointorigin}{\pgfqpoint{5.329166in}{3.676603in}}%
\pgfusepath{use as bounding box, clip}%
\begin{pgfscope}%
\pgfsetbuttcap%
\pgfsetmiterjoin%
\definecolor{currentfill}{rgb}{1.000000,1.000000,1.000000}%
\pgfsetfillcolor{currentfill}%
\pgfsetlinewidth{0.000000pt}%
\definecolor{currentstroke}{rgb}{1.000000,1.000000,1.000000}%
\pgfsetstrokecolor{currentstroke}%
\pgfsetdash{}{0pt}%
\pgfpathmoveto{\pgfqpoint{-0.000000in}{0.000000in}}%
\pgfpathlineto{\pgfqpoint{5.329166in}{0.000000in}}%
\pgfpathlineto{\pgfqpoint{5.329166in}{3.676603in}}%
\pgfpathlineto{\pgfqpoint{-0.000000in}{3.676603in}}%
\pgfpathclose%
\pgfusepath{fill}%
\end{pgfscope}%
\begin{pgfscope}%
\pgfsetbuttcap%
\pgfsetmiterjoin%
\definecolor{currentfill}{rgb}{1.000000,1.000000,1.000000}%
\pgfsetfillcolor{currentfill}%
\pgfsetlinewidth{0.000000pt}%
\definecolor{currentstroke}{rgb}{0.000000,0.000000,0.000000}%
\pgfsetstrokecolor{currentstroke}%
\pgfsetstrokeopacity{0.000000}%
\pgfsetdash{}{0pt}%
\pgfpathmoveto{\pgfqpoint{0.466126in}{0.521603in}}%
\pgfpathlineto{\pgfqpoint{4.186126in}{0.521603in}}%
\pgfpathlineto{\pgfqpoint{4.186126in}{3.541603in}}%
\pgfpathlineto{\pgfqpoint{0.466126in}{3.541603in}}%
\pgfpathclose%
\pgfusepath{fill}%
\end{pgfscope}%
\begin{pgfscope}%
\pgfpathrectangle{\pgfqpoint{0.466126in}{0.521603in}}{\pgfqpoint{3.720000in}{3.020000in}} %
\pgfusepath{clip}%
\pgfsetbuttcap%
\pgfsetroundjoin%
\definecolor{currentfill}{rgb}{1.000000,0.255843,0.128999}%
\pgfsetfillcolor{currentfill}%
\pgfsetlinewidth{1.003750pt}%
\definecolor{currentstroke}{rgb}{1.000000,0.255843,0.128999}%
\pgfsetstrokecolor{currentstroke}%
\pgfsetdash{}{0pt}%
\pgfpathmoveto{\pgfqpoint{2.577334in}{1.175315in}}%
\pgfpathcurveto{\pgfqpoint{2.588384in}{1.175315in}}{\pgfqpoint{2.598983in}{1.179706in}}{\pgfqpoint{2.606797in}{1.187519in}}%
\pgfpathcurveto{\pgfqpoint{2.614611in}{1.195333in}}{\pgfqpoint{2.619001in}{1.205932in}}{\pgfqpoint{2.619001in}{1.216982in}}%
\pgfpathcurveto{\pgfqpoint{2.619001in}{1.228032in}}{\pgfqpoint{2.614611in}{1.238631in}}{\pgfqpoint{2.606797in}{1.246445in}}%
\pgfpathcurveto{\pgfqpoint{2.598983in}{1.254259in}}{\pgfqpoint{2.588384in}{1.258649in}}{\pgfqpoint{2.577334in}{1.258649in}}%
\pgfpathcurveto{\pgfqpoint{2.566284in}{1.258649in}}{\pgfqpoint{2.555685in}{1.254259in}}{\pgfqpoint{2.547872in}{1.246445in}}%
\pgfpathcurveto{\pgfqpoint{2.540058in}{1.238631in}}{\pgfqpoint{2.535668in}{1.228032in}}{\pgfqpoint{2.535668in}{1.216982in}}%
\pgfpathcurveto{\pgfqpoint{2.535668in}{1.205932in}}{\pgfqpoint{2.540058in}{1.195333in}}{\pgfqpoint{2.547872in}{1.187519in}}%
\pgfpathcurveto{\pgfqpoint{2.555685in}{1.179706in}}{\pgfqpoint{2.566284in}{1.175315in}}{\pgfqpoint{2.577334in}{1.175315in}}%
\pgfpathclose%
\pgfusepath{stroke,fill}%
\end{pgfscope}%
\begin{pgfscope}%
\pgfpathrectangle{\pgfqpoint{0.466126in}{0.521603in}}{\pgfqpoint{3.720000in}{3.020000in}} %
\pgfusepath{clip}%
\pgfsetbuttcap%
\pgfsetroundjoin%
\definecolor{currentfill}{rgb}{0.319608,0.279583,0.989980}%
\pgfsetfillcolor{currentfill}%
\pgfsetlinewidth{1.003750pt}%
\definecolor{currentstroke}{rgb}{0.319608,0.279583,0.989980}%
\pgfsetstrokecolor{currentstroke}%
\pgfsetdash{}{0pt}%
\pgfpathmoveto{\pgfqpoint{2.622359in}{1.378029in}}%
\pgfpathcurveto{\pgfqpoint{2.633409in}{1.378029in}}{\pgfqpoint{2.644008in}{1.382419in}}{\pgfqpoint{2.651822in}{1.390233in}}%
\pgfpathcurveto{\pgfqpoint{2.659636in}{1.398047in}}{\pgfqpoint{2.664026in}{1.408646in}}{\pgfqpoint{2.664026in}{1.419696in}}%
\pgfpathcurveto{\pgfqpoint{2.664026in}{1.430746in}}{\pgfqpoint{2.659636in}{1.441345in}}{\pgfqpoint{2.651822in}{1.449158in}}%
\pgfpathcurveto{\pgfqpoint{2.644008in}{1.456972in}}{\pgfqpoint{2.633409in}{1.461362in}}{\pgfqpoint{2.622359in}{1.461362in}}%
\pgfpathcurveto{\pgfqpoint{2.611309in}{1.461362in}}{\pgfqpoint{2.600710in}{1.456972in}}{\pgfqpoint{2.592896in}{1.449158in}}%
\pgfpathcurveto{\pgfqpoint{2.585083in}{1.441345in}}{\pgfqpoint{2.580692in}{1.430746in}}{\pgfqpoint{2.580692in}{1.419696in}}%
\pgfpathcurveto{\pgfqpoint{2.580692in}{1.408646in}}{\pgfqpoint{2.585083in}{1.398047in}}{\pgfqpoint{2.592896in}{1.390233in}}%
\pgfpathcurveto{\pgfqpoint{2.600710in}{1.382419in}}{\pgfqpoint{2.611309in}{1.378029in}}{\pgfqpoint{2.622359in}{1.378029in}}%
\pgfpathclose%
\pgfusepath{stroke,fill}%
\end{pgfscope}%
\begin{pgfscope}%
\pgfpathrectangle{\pgfqpoint{0.466126in}{0.521603in}}{\pgfqpoint{3.720000in}{3.020000in}} %
\pgfusepath{clip}%
\pgfsetbuttcap%
\pgfsetroundjoin%
\definecolor{currentfill}{rgb}{0.460784,0.061561,0.999526}%
\pgfsetfillcolor{currentfill}%
\pgfsetlinewidth{1.003750pt}%
\definecolor{currentstroke}{rgb}{0.460784,0.061561,0.999526}%
\pgfsetstrokecolor{currentstroke}%
\pgfsetdash{}{0pt}%
\pgfpathmoveto{\pgfqpoint{1.993095in}{1.931016in}}%
\pgfpathcurveto{\pgfqpoint{2.004145in}{1.931016in}}{\pgfqpoint{2.014744in}{1.935406in}}{\pgfqpoint{2.022557in}{1.943220in}}%
\pgfpathcurveto{\pgfqpoint{2.030371in}{1.951033in}}{\pgfqpoint{2.034761in}{1.961632in}}{\pgfqpoint{2.034761in}{1.972682in}}%
\pgfpathcurveto{\pgfqpoint{2.034761in}{1.983733in}}{\pgfqpoint{2.030371in}{1.994332in}}{\pgfqpoint{2.022557in}{2.002145in}}%
\pgfpathcurveto{\pgfqpoint{2.014744in}{2.009959in}}{\pgfqpoint{2.004145in}{2.014349in}}{\pgfqpoint{1.993095in}{2.014349in}}%
\pgfpathcurveto{\pgfqpoint{1.982045in}{2.014349in}}{\pgfqpoint{1.971446in}{2.009959in}}{\pgfqpoint{1.963632in}{2.002145in}}%
\pgfpathcurveto{\pgfqpoint{1.955818in}{1.994332in}}{\pgfqpoint{1.951428in}{1.983733in}}{\pgfqpoint{1.951428in}{1.972682in}}%
\pgfpathcurveto{\pgfqpoint{1.951428in}{1.961632in}}{\pgfqpoint{1.955818in}{1.951033in}}{\pgfqpoint{1.963632in}{1.943220in}}%
\pgfpathcurveto{\pgfqpoint{1.971446in}{1.935406in}}{\pgfqpoint{1.982045in}{1.931016in}}{\pgfqpoint{1.993095in}{1.931016in}}%
\pgfpathclose%
\pgfusepath{stroke,fill}%
\end{pgfscope}%
\begin{pgfscope}%
\pgfpathrectangle{\pgfqpoint{0.466126in}{0.521603in}}{\pgfqpoint{3.720000in}{3.020000in}} %
\pgfusepath{clip}%
\pgfsetbuttcap%
\pgfsetroundjoin%
\definecolor{currentfill}{rgb}{0.500000,0.000000,1.000000}%
\pgfsetfillcolor{currentfill}%
\pgfsetlinewidth{1.003750pt}%
\definecolor{currentstroke}{rgb}{0.500000,0.000000,1.000000}%
\pgfsetstrokecolor{currentstroke}%
\pgfsetdash{}{0pt}%
\pgfpathmoveto{\pgfqpoint{0.696445in}{1.941612in}}%
\pgfpathcurveto{\pgfqpoint{0.707495in}{1.941612in}}{\pgfqpoint{0.718094in}{1.946002in}}{\pgfqpoint{0.725908in}{1.953816in}}%
\pgfpathcurveto{\pgfqpoint{0.733721in}{1.961629in}}{\pgfqpoint{0.738112in}{1.972228in}}{\pgfqpoint{0.738112in}{1.983278in}}%
\pgfpathcurveto{\pgfqpoint{0.738112in}{1.994329in}}{\pgfqpoint{0.733721in}{2.004928in}}{\pgfqpoint{0.725908in}{2.012741in}}%
\pgfpathcurveto{\pgfqpoint{0.718094in}{2.020555in}}{\pgfqpoint{0.707495in}{2.024945in}}{\pgfqpoint{0.696445in}{2.024945in}}%
\pgfpathcurveto{\pgfqpoint{0.685395in}{2.024945in}}{\pgfqpoint{0.674796in}{2.020555in}}{\pgfqpoint{0.666982in}{2.012741in}}%
\pgfpathcurveto{\pgfqpoint{0.659168in}{2.004928in}}{\pgfqpoint{0.654778in}{1.994329in}}{\pgfqpoint{0.654778in}{1.983278in}}%
\pgfpathcurveto{\pgfqpoint{0.654778in}{1.972228in}}{\pgfqpoint{0.659168in}{1.961629in}}{\pgfqpoint{0.666982in}{1.953816in}}%
\pgfpathcurveto{\pgfqpoint{0.674796in}{1.946002in}}{\pgfqpoint{0.685395in}{1.941612in}}{\pgfqpoint{0.696445in}{1.941612in}}%
\pgfpathclose%
\pgfusepath{stroke,fill}%
\end{pgfscope}%
\begin{pgfscope}%
\pgfpathrectangle{\pgfqpoint{0.466126in}{0.521603in}}{\pgfqpoint{3.720000in}{3.020000in}} %
\pgfusepath{clip}%
\pgfsetbuttcap%
\pgfsetroundjoin%
\definecolor{currentfill}{rgb}{1.000000,0.000000,0.000000}%
\pgfsetfillcolor{currentfill}%
\pgfsetlinewidth{1.003750pt}%
\definecolor{currentstroke}{rgb}{1.000000,0.000000,0.000000}%
\pgfsetstrokecolor{currentstroke}%
\pgfsetdash{}{0pt}%
\pgfpathmoveto{\pgfqpoint{3.840259in}{0.755334in}}%
\pgfpathcurveto{\pgfqpoint{3.851309in}{0.755334in}}{\pgfqpoint{3.861909in}{0.759724in}}{\pgfqpoint{3.869722in}{0.767538in}}%
\pgfpathcurveto{\pgfqpoint{3.877536in}{0.775352in}}{\pgfqpoint{3.881926in}{0.785951in}}{\pgfqpoint{3.881926in}{0.797001in}}%
\pgfpathcurveto{\pgfqpoint{3.881926in}{0.808051in}}{\pgfqpoint{3.877536in}{0.818650in}}{\pgfqpoint{3.869722in}{0.826464in}}%
\pgfpathcurveto{\pgfqpoint{3.861909in}{0.834277in}}{\pgfqpoint{3.851309in}{0.838668in}}{\pgfqpoint{3.840259in}{0.838668in}}%
\pgfpathcurveto{\pgfqpoint{3.829209in}{0.838668in}}{\pgfqpoint{3.818610in}{0.834277in}}{\pgfqpoint{3.810797in}{0.826464in}}%
\pgfpathcurveto{\pgfqpoint{3.802983in}{0.818650in}}{\pgfqpoint{3.798593in}{0.808051in}}{\pgfqpoint{3.798593in}{0.797001in}}%
\pgfpathcurveto{\pgfqpoint{3.798593in}{0.785951in}}{\pgfqpoint{3.802983in}{0.775352in}}{\pgfqpoint{3.810797in}{0.767538in}}%
\pgfpathcurveto{\pgfqpoint{3.818610in}{0.759724in}}{\pgfqpoint{3.829209in}{0.755334in}}{\pgfqpoint{3.840259in}{0.755334in}}%
\pgfpathclose%
\pgfusepath{stroke,fill}%
\end{pgfscope}%
\begin{pgfscope}%
\pgfpathrectangle{\pgfqpoint{0.466126in}{0.521603in}}{\pgfqpoint{3.720000in}{3.020000in}} %
\pgfusepath{clip}%
\pgfsetbuttcap%
\pgfsetroundjoin%
\definecolor{currentfill}{rgb}{1.000000,0.000000,0.000000}%
\pgfsetfillcolor{currentfill}%
\pgfsetlinewidth{1.003750pt}%
\definecolor{currentstroke}{rgb}{1.000000,0.000000,0.000000}%
\pgfsetstrokecolor{currentstroke}%
\pgfsetdash{}{0pt}%
\pgfpathmoveto{\pgfqpoint{3.268783in}{0.755334in}}%
\pgfpathcurveto{\pgfqpoint{3.279833in}{0.755334in}}{\pgfqpoint{3.290432in}{0.759724in}}{\pgfqpoint{3.298246in}{0.767538in}}%
\pgfpathcurveto{\pgfqpoint{3.306059in}{0.775352in}}{\pgfqpoint{3.310450in}{0.785951in}}{\pgfqpoint{3.310450in}{0.797001in}}%
\pgfpathcurveto{\pgfqpoint{3.310450in}{0.808051in}}{\pgfqpoint{3.306059in}{0.818650in}}{\pgfqpoint{3.298246in}{0.826464in}}%
\pgfpathcurveto{\pgfqpoint{3.290432in}{0.834277in}}{\pgfqpoint{3.279833in}{0.838668in}}{\pgfqpoint{3.268783in}{0.838668in}}%
\pgfpathcurveto{\pgfqpoint{3.257733in}{0.838668in}}{\pgfqpoint{3.247134in}{0.834277in}}{\pgfqpoint{3.239320in}{0.826464in}}%
\pgfpathcurveto{\pgfqpoint{3.231507in}{0.818650in}}{\pgfqpoint{3.227116in}{0.808051in}}{\pgfqpoint{3.227116in}{0.797001in}}%
\pgfpathcurveto{\pgfqpoint{3.227116in}{0.785951in}}{\pgfqpoint{3.231507in}{0.775352in}}{\pgfqpoint{3.239320in}{0.767538in}}%
\pgfpathcurveto{\pgfqpoint{3.247134in}{0.759724in}}{\pgfqpoint{3.257733in}{0.755334in}}{\pgfqpoint{3.268783in}{0.755334in}}%
\pgfpathclose%
\pgfusepath{stroke,fill}%
\end{pgfscope}%
\begin{pgfscope}%
\pgfpathrectangle{\pgfqpoint{0.466126in}{0.521603in}}{\pgfqpoint{3.720000in}{3.020000in}} %
\pgfusepath{clip}%
\pgfsetbuttcap%
\pgfsetroundjoin%
\definecolor{currentfill}{rgb}{1.000000,0.171626,0.086133}%
\pgfsetfillcolor{currentfill}%
\pgfsetlinewidth{1.003750pt}%
\definecolor{currentstroke}{rgb}{1.000000,0.171626,0.086133}%
\pgfsetstrokecolor{currentstroke}%
\pgfsetdash{}{0pt}%
\pgfpathmoveto{\pgfqpoint{3.194710in}{1.070975in}}%
\pgfpathcurveto{\pgfqpoint{3.205760in}{1.070975in}}{\pgfqpoint{3.216359in}{1.075365in}}{\pgfqpoint{3.224173in}{1.083179in}}%
\pgfpathcurveto{\pgfqpoint{3.231986in}{1.090993in}}{\pgfqpoint{3.236377in}{1.101592in}}{\pgfqpoint{3.236377in}{1.112642in}}%
\pgfpathcurveto{\pgfqpoint{3.236377in}{1.123692in}}{\pgfqpoint{3.231986in}{1.134291in}}{\pgfqpoint{3.224173in}{1.142104in}}%
\pgfpathcurveto{\pgfqpoint{3.216359in}{1.149918in}}{\pgfqpoint{3.205760in}{1.154308in}}{\pgfqpoint{3.194710in}{1.154308in}}%
\pgfpathcurveto{\pgfqpoint{3.183660in}{1.154308in}}{\pgfqpoint{3.173061in}{1.149918in}}{\pgfqpoint{3.165247in}{1.142104in}}%
\pgfpathcurveto{\pgfqpoint{3.157434in}{1.134291in}}{\pgfqpoint{3.153043in}{1.123692in}}{\pgfqpoint{3.153043in}{1.112642in}}%
\pgfpathcurveto{\pgfqpoint{3.153043in}{1.101592in}}{\pgfqpoint{3.157434in}{1.090993in}}{\pgfqpoint{3.165247in}{1.083179in}}%
\pgfpathcurveto{\pgfqpoint{3.173061in}{1.075365in}}{\pgfqpoint{3.183660in}{1.070975in}}{\pgfqpoint{3.194710in}{1.070975in}}%
\pgfpathclose%
\pgfusepath{stroke,fill}%
\end{pgfscope}%
\begin{pgfscope}%
\pgfpathrectangle{\pgfqpoint{0.466126in}{0.521603in}}{\pgfqpoint{3.720000in}{3.020000in}} %
\pgfusepath{clip}%
\pgfsetbuttcap%
\pgfsetroundjoin%
\definecolor{currentfill}{rgb}{1.000000,0.171626,0.086133}%
\pgfsetfillcolor{currentfill}%
\pgfsetlinewidth{1.003750pt}%
\definecolor{currentstroke}{rgb}{1.000000,0.171626,0.086133}%
\pgfsetstrokecolor{currentstroke}%
\pgfsetdash{}{0pt}%
\pgfpathmoveto{\pgfqpoint{2.826142in}{1.357655in}}%
\pgfpathcurveto{\pgfqpoint{2.837193in}{1.357655in}}{\pgfqpoint{2.847792in}{1.362046in}}{\pgfqpoint{2.855605in}{1.369859in}}%
\pgfpathcurveto{\pgfqpoint{2.863419in}{1.377673in}}{\pgfqpoint{2.867809in}{1.388272in}}{\pgfqpoint{2.867809in}{1.399322in}}%
\pgfpathcurveto{\pgfqpoint{2.867809in}{1.410372in}}{\pgfqpoint{2.863419in}{1.420971in}}{\pgfqpoint{2.855605in}{1.428785in}}%
\pgfpathcurveto{\pgfqpoint{2.847792in}{1.436598in}}{\pgfqpoint{2.837193in}{1.440989in}}{\pgfqpoint{2.826142in}{1.440989in}}%
\pgfpathcurveto{\pgfqpoint{2.815092in}{1.440989in}}{\pgfqpoint{2.804493in}{1.436598in}}{\pgfqpoint{2.796680in}{1.428785in}}%
\pgfpathcurveto{\pgfqpoint{2.788866in}{1.420971in}}{\pgfqpoint{2.784476in}{1.410372in}}{\pgfqpoint{2.784476in}{1.399322in}}%
\pgfpathcurveto{\pgfqpoint{2.784476in}{1.388272in}}{\pgfqpoint{2.788866in}{1.377673in}}{\pgfqpoint{2.796680in}{1.369859in}}%
\pgfpathcurveto{\pgfqpoint{2.804493in}{1.362046in}}{\pgfqpoint{2.815092in}{1.357655in}}{\pgfqpoint{2.826142in}{1.357655in}}%
\pgfpathclose%
\pgfusepath{stroke,fill}%
\end{pgfscope}%
\begin{pgfscope}%
\pgfpathrectangle{\pgfqpoint{0.466126in}{0.521603in}}{\pgfqpoint{3.720000in}{3.020000in}} %
\pgfusepath{clip}%
\pgfsetbuttcap%
\pgfsetroundjoin%
\definecolor{currentfill}{rgb}{0.660784,0.968276,0.612420}%
\pgfsetfillcolor{currentfill}%
\pgfsetlinewidth{1.003750pt}%
\definecolor{currentstroke}{rgb}{0.660784,0.968276,0.612420}%
\pgfsetstrokecolor{currentstroke}%
\pgfsetdash{}{0pt}%
\pgfpathmoveto{\pgfqpoint{2.517392in}{1.171558in}}%
\pgfpathcurveto{\pgfqpoint{2.528442in}{1.171558in}}{\pgfqpoint{2.539041in}{1.175948in}}{\pgfqpoint{2.546854in}{1.183762in}}%
\pgfpathcurveto{\pgfqpoint{2.554668in}{1.191575in}}{\pgfqpoint{2.559058in}{1.202174in}}{\pgfqpoint{2.559058in}{1.213224in}}%
\pgfpathcurveto{\pgfqpoint{2.559058in}{1.224274in}}{\pgfqpoint{2.554668in}{1.234873in}}{\pgfqpoint{2.546854in}{1.242687in}}%
\pgfpathcurveto{\pgfqpoint{2.539041in}{1.250501in}}{\pgfqpoint{2.528442in}{1.254891in}}{\pgfqpoint{2.517392in}{1.254891in}}%
\pgfpathcurveto{\pgfqpoint{2.506341in}{1.254891in}}{\pgfqpoint{2.495742in}{1.250501in}}{\pgfqpoint{2.487929in}{1.242687in}}%
\pgfpathcurveto{\pgfqpoint{2.480115in}{1.234873in}}{\pgfqpoint{2.475725in}{1.224274in}}{\pgfqpoint{2.475725in}{1.213224in}}%
\pgfpathcurveto{\pgfqpoint{2.475725in}{1.202174in}}{\pgfqpoint{2.480115in}{1.191575in}}{\pgfqpoint{2.487929in}{1.183762in}}%
\pgfpathcurveto{\pgfqpoint{2.495742in}{1.175948in}}{\pgfqpoint{2.506341in}{1.171558in}}{\pgfqpoint{2.517392in}{1.171558in}}%
\pgfpathclose%
\pgfusepath{stroke,fill}%
\end{pgfscope}%
\begin{pgfscope}%
\pgfpathrectangle{\pgfqpoint{0.466126in}{0.521603in}}{\pgfqpoint{3.720000in}{3.020000in}} %
\pgfusepath{clip}%
\pgfsetbuttcap%
\pgfsetroundjoin%
\definecolor{currentfill}{rgb}{0.103922,0.812622,0.889604}%
\pgfsetfillcolor{currentfill}%
\pgfsetlinewidth{1.003750pt}%
\definecolor{currentstroke}{rgb}{0.103922,0.812622,0.889604}%
\pgfsetstrokecolor{currentstroke}%
\pgfsetdash{}{0pt}%
\pgfpathmoveto{\pgfqpoint{2.450118in}{1.378029in}}%
\pgfpathcurveto{\pgfqpoint{2.461168in}{1.378029in}}{\pgfqpoint{2.471768in}{1.382419in}}{\pgfqpoint{2.479581in}{1.390233in}}%
\pgfpathcurveto{\pgfqpoint{2.487395in}{1.398047in}}{\pgfqpoint{2.491785in}{1.408646in}}{\pgfqpoint{2.491785in}{1.419696in}}%
\pgfpathcurveto{\pgfqpoint{2.491785in}{1.430746in}}{\pgfqpoint{2.487395in}{1.441345in}}{\pgfqpoint{2.479581in}{1.449158in}}%
\pgfpathcurveto{\pgfqpoint{2.471768in}{1.456972in}}{\pgfqpoint{2.461168in}{1.461362in}}{\pgfqpoint{2.450118in}{1.461362in}}%
\pgfpathcurveto{\pgfqpoint{2.439068in}{1.461362in}}{\pgfqpoint{2.428469in}{1.456972in}}{\pgfqpoint{2.420656in}{1.449158in}}%
\pgfpathcurveto{\pgfqpoint{2.412842in}{1.441345in}}{\pgfqpoint{2.408452in}{1.430746in}}{\pgfqpoint{2.408452in}{1.419696in}}%
\pgfpathcurveto{\pgfqpoint{2.408452in}{1.408646in}}{\pgfqpoint{2.412842in}{1.398047in}}{\pgfqpoint{2.420656in}{1.390233in}}%
\pgfpathcurveto{\pgfqpoint{2.428469in}{1.382419in}}{\pgfqpoint{2.439068in}{1.378029in}}{\pgfqpoint{2.450118in}{1.378029in}}%
\pgfpathclose%
\pgfusepath{stroke,fill}%
\end{pgfscope}%
\begin{pgfscope}%
\pgfpathrectangle{\pgfqpoint{0.466126in}{0.521603in}}{\pgfqpoint{3.720000in}{3.020000in}} %
\pgfusepath{clip}%
\pgfsetbuttcap%
\pgfsetroundjoin%
\definecolor{currentfill}{rgb}{0.005882,0.700543,0.925638}%
\pgfsetfillcolor{currentfill}%
\pgfsetlinewidth{1.003750pt}%
\definecolor{currentstroke}{rgb}{0.005882,0.700543,0.925638}%
\pgfsetstrokecolor{currentstroke}%
\pgfsetdash{}{0pt}%
\pgfpathmoveto{\pgfqpoint{1.717732in}{0.972602in}}%
\pgfpathcurveto{\pgfqpoint{1.728782in}{0.972602in}}{\pgfqpoint{1.739381in}{0.976992in}}{\pgfqpoint{1.747195in}{0.984806in}}%
\pgfpathcurveto{\pgfqpoint{1.755008in}{0.992619in}}{\pgfqpoint{1.759398in}{1.003218in}}{\pgfqpoint{1.759398in}{1.014269in}}%
\pgfpathcurveto{\pgfqpoint{1.759398in}{1.025319in}}{\pgfqpoint{1.755008in}{1.035918in}}{\pgfqpoint{1.747195in}{1.043731in}}%
\pgfpathcurveto{\pgfqpoint{1.739381in}{1.051545in}}{\pgfqpoint{1.728782in}{1.055935in}}{\pgfqpoint{1.717732in}{1.055935in}}%
\pgfpathcurveto{\pgfqpoint{1.706682in}{1.055935in}}{\pgfqpoint{1.696083in}{1.051545in}}{\pgfqpoint{1.688269in}{1.043731in}}%
\pgfpathcurveto{\pgfqpoint{1.680455in}{1.035918in}}{\pgfqpoint{1.676065in}{1.025319in}}{\pgfqpoint{1.676065in}{1.014269in}}%
\pgfpathcurveto{\pgfqpoint{1.676065in}{1.003218in}}{\pgfqpoint{1.680455in}{0.992619in}}{\pgfqpoint{1.688269in}{0.984806in}}%
\pgfpathcurveto{\pgfqpoint{1.696083in}{0.976992in}}{\pgfqpoint{1.706682in}{0.972602in}}{\pgfqpoint{1.717732in}{0.972602in}}%
\pgfpathclose%
\pgfusepath{stroke,fill}%
\end{pgfscope}%
\begin{pgfscope}%
\pgfpathrectangle{\pgfqpoint{0.466126in}{0.521603in}}{\pgfqpoint{3.720000in}{3.020000in}} %
\pgfusepath{clip}%
\pgfsetbuttcap%
\pgfsetroundjoin%
\definecolor{currentfill}{rgb}{0.005882,0.700543,0.925638}%
\pgfsetfillcolor{currentfill}%
\pgfsetlinewidth{1.003750pt}%
\definecolor{currentstroke}{rgb}{0.005882,0.700543,0.925638}%
\pgfsetstrokecolor{currentstroke}%
\pgfsetdash{}{0pt}%
\pgfpathmoveto{\pgfqpoint{0.977083in}{0.769888in}}%
\pgfpathcurveto{\pgfqpoint{0.988133in}{0.769888in}}{\pgfqpoint{0.998732in}{0.774279in}}{\pgfqpoint{1.006545in}{0.782092in}}%
\pgfpathcurveto{\pgfqpoint{1.014359in}{0.789906in}}{\pgfqpoint{1.018749in}{0.800505in}}{\pgfqpoint{1.018749in}{0.811555in}}%
\pgfpathcurveto{\pgfqpoint{1.018749in}{0.822605in}}{\pgfqpoint{1.014359in}{0.833204in}}{\pgfqpoint{1.006545in}{0.841018in}}%
\pgfpathcurveto{\pgfqpoint{0.998732in}{0.848831in}}{\pgfqpoint{0.988133in}{0.853222in}}{\pgfqpoint{0.977083in}{0.853222in}}%
\pgfpathcurveto{\pgfqpoint{0.966032in}{0.853222in}}{\pgfqpoint{0.955433in}{0.848831in}}{\pgfqpoint{0.947620in}{0.841018in}}%
\pgfpathcurveto{\pgfqpoint{0.939806in}{0.833204in}}{\pgfqpoint{0.935416in}{0.822605in}}{\pgfqpoint{0.935416in}{0.811555in}}%
\pgfpathcurveto{\pgfqpoint{0.935416in}{0.800505in}}{\pgfqpoint{0.939806in}{0.789906in}}{\pgfqpoint{0.947620in}{0.782092in}}%
\pgfpathcurveto{\pgfqpoint{0.955433in}{0.774279in}}{\pgfqpoint{0.966032in}{0.769888in}}{\pgfqpoint{0.977083in}{0.769888in}}%
\pgfpathclose%
\pgfusepath{stroke,fill}%
\end{pgfscope}%
\begin{pgfscope}%
\pgfpathrectangle{\pgfqpoint{0.466126in}{0.521603in}}{\pgfqpoint{3.720000in}{3.020000in}} %
\pgfusepath{clip}%
\pgfsetbuttcap%
\pgfsetroundjoin%
\definecolor{currentfill}{rgb}{0.139216,0.536867,0.960122}%
\pgfsetfillcolor{currentfill}%
\pgfsetlinewidth{1.003750pt}%
\definecolor{currentstroke}{rgb}{0.139216,0.536867,0.960122}%
\pgfsetstrokecolor{currentstroke}%
\pgfsetdash{}{0pt}%
\pgfpathmoveto{\pgfqpoint{2.309822in}{1.357655in}}%
\pgfpathcurveto{\pgfqpoint{2.320872in}{1.357655in}}{\pgfqpoint{2.331471in}{1.362046in}}{\pgfqpoint{2.339284in}{1.369859in}}%
\pgfpathcurveto{\pgfqpoint{2.347098in}{1.377673in}}{\pgfqpoint{2.351488in}{1.388272in}}{\pgfqpoint{2.351488in}{1.399322in}}%
\pgfpathcurveto{\pgfqpoint{2.351488in}{1.410372in}}{\pgfqpoint{2.347098in}{1.420971in}}{\pgfqpoint{2.339284in}{1.428785in}}%
\pgfpathcurveto{\pgfqpoint{2.331471in}{1.436598in}}{\pgfqpoint{2.320872in}{1.440989in}}{\pgfqpoint{2.309822in}{1.440989in}}%
\pgfpathcurveto{\pgfqpoint{2.298771in}{1.440989in}}{\pgfqpoint{2.288172in}{1.436598in}}{\pgfqpoint{2.280359in}{1.428785in}}%
\pgfpathcurveto{\pgfqpoint{2.272545in}{1.420971in}}{\pgfqpoint{2.268155in}{1.410372in}}{\pgfqpoint{2.268155in}{1.399322in}}%
\pgfpathcurveto{\pgfqpoint{2.268155in}{1.388272in}}{\pgfqpoint{2.272545in}{1.377673in}}{\pgfqpoint{2.280359in}{1.369859in}}%
\pgfpathcurveto{\pgfqpoint{2.288172in}{1.362046in}}{\pgfqpoint{2.298771in}{1.357655in}}{\pgfqpoint{2.309822in}{1.357655in}}%
\pgfpathclose%
\pgfusepath{stroke,fill}%
\end{pgfscope}%
\begin{pgfscope}%
\pgfpathrectangle{\pgfqpoint{0.466126in}{0.521603in}}{\pgfqpoint{3.720000in}{3.020000in}} %
\pgfusepath{clip}%
\pgfsetbuttcap%
\pgfsetroundjoin%
\definecolor{currentfill}{rgb}{0.139216,0.536867,0.960122}%
\pgfsetfillcolor{currentfill}%
\pgfsetlinewidth{1.003750pt}%
\definecolor{currentstroke}{rgb}{0.139216,0.536867,0.960122}%
\pgfsetstrokecolor{currentstroke}%
\pgfsetdash{}{0pt}%
\pgfpathmoveto{\pgfqpoint{1.871755in}{1.357655in}}%
\pgfpathcurveto{\pgfqpoint{1.882806in}{1.357655in}}{\pgfqpoint{1.893405in}{1.362046in}}{\pgfqpoint{1.901218in}{1.369859in}}%
\pgfpathcurveto{\pgfqpoint{1.909032in}{1.377673in}}{\pgfqpoint{1.913422in}{1.388272in}}{\pgfqpoint{1.913422in}{1.399322in}}%
\pgfpathcurveto{\pgfqpoint{1.913422in}{1.410372in}}{\pgfqpoint{1.909032in}{1.420971in}}{\pgfqpoint{1.901218in}{1.428785in}}%
\pgfpathcurveto{\pgfqpoint{1.893405in}{1.436598in}}{\pgfqpoint{1.882806in}{1.440989in}}{\pgfqpoint{1.871755in}{1.440989in}}%
\pgfpathcurveto{\pgfqpoint{1.860705in}{1.440989in}}{\pgfqpoint{1.850106in}{1.436598in}}{\pgfqpoint{1.842293in}{1.428785in}}%
\pgfpathcurveto{\pgfqpoint{1.834479in}{1.420971in}}{\pgfqpoint{1.830089in}{1.410372in}}{\pgfqpoint{1.830089in}{1.399322in}}%
\pgfpathcurveto{\pgfqpoint{1.830089in}{1.388272in}}{\pgfqpoint{1.834479in}{1.377673in}}{\pgfqpoint{1.842293in}{1.369859in}}%
\pgfpathcurveto{\pgfqpoint{1.850106in}{1.362046in}}{\pgfqpoint{1.860705in}{1.357655in}}{\pgfqpoint{1.871755in}{1.357655in}}%
\pgfpathclose%
\pgfusepath{stroke,fill}%
\end{pgfscope}%
\begin{pgfscope}%
\pgfpathrectangle{\pgfqpoint{0.466126in}{0.521603in}}{\pgfqpoint{3.720000in}{3.020000in}} %
\pgfusepath{clip}%
\pgfsetbuttcap%
\pgfsetroundjoin%
\definecolor{currentfill}{rgb}{0.139216,0.536867,0.960122}%
\pgfsetfillcolor{currentfill}%
\pgfsetlinewidth{1.003750pt}%
\definecolor{currentstroke}{rgb}{0.139216,0.536867,0.960122}%
\pgfsetstrokecolor{currentstroke}%
\pgfsetdash{}{0pt}%
\pgfpathmoveto{\pgfqpoint{1.613588in}{1.357655in}}%
\pgfpathcurveto{\pgfqpoint{1.624638in}{1.357655in}}{\pgfqpoint{1.635237in}{1.362046in}}{\pgfqpoint{1.643051in}{1.369859in}}%
\pgfpathcurveto{\pgfqpoint{1.650865in}{1.377673in}}{\pgfqpoint{1.655255in}{1.388272in}}{\pgfqpoint{1.655255in}{1.399322in}}%
\pgfpathcurveto{\pgfqpoint{1.655255in}{1.410372in}}{\pgfqpoint{1.650865in}{1.420971in}}{\pgfqpoint{1.643051in}{1.428785in}}%
\pgfpathcurveto{\pgfqpoint{1.635237in}{1.436598in}}{\pgfqpoint{1.624638in}{1.440989in}}{\pgfqpoint{1.613588in}{1.440989in}}%
\pgfpathcurveto{\pgfqpoint{1.602538in}{1.440989in}}{\pgfqpoint{1.591939in}{1.436598in}}{\pgfqpoint{1.584125in}{1.428785in}}%
\pgfpathcurveto{\pgfqpoint{1.576312in}{1.420971in}}{\pgfqpoint{1.571922in}{1.410372in}}{\pgfqpoint{1.571922in}{1.399322in}}%
\pgfpathcurveto{\pgfqpoint{1.571922in}{1.388272in}}{\pgfqpoint{1.576312in}{1.377673in}}{\pgfqpoint{1.584125in}{1.369859in}}%
\pgfpathcurveto{\pgfqpoint{1.591939in}{1.362046in}}{\pgfqpoint{1.602538in}{1.357655in}}{\pgfqpoint{1.613588in}{1.357655in}}%
\pgfpathclose%
\pgfusepath{stroke,fill}%
\end{pgfscope}%
\begin{pgfscope}%
\pgfpathrectangle{\pgfqpoint{0.466126in}{0.521603in}}{\pgfqpoint{3.720000in}{3.020000in}} %
\pgfusepath{clip}%
\pgfsetbuttcap%
\pgfsetroundjoin%
\definecolor{currentfill}{rgb}{0.139216,0.536867,0.960122}%
\pgfsetfillcolor{currentfill}%
\pgfsetlinewidth{1.003750pt}%
\definecolor{currentstroke}{rgb}{0.139216,0.536867,0.960122}%
\pgfsetstrokecolor{currentstroke}%
\pgfsetdash{}{0pt}%
\pgfpathmoveto{\pgfqpoint{1.372829in}{1.357655in}}%
\pgfpathcurveto{\pgfqpoint{1.383879in}{1.357655in}}{\pgfqpoint{1.394478in}{1.362046in}}{\pgfqpoint{1.402291in}{1.369859in}}%
\pgfpathcurveto{\pgfqpoint{1.410105in}{1.377673in}}{\pgfqpoint{1.414495in}{1.388272in}}{\pgfqpoint{1.414495in}{1.399322in}}%
\pgfpathcurveto{\pgfqpoint{1.414495in}{1.410372in}}{\pgfqpoint{1.410105in}{1.420971in}}{\pgfqpoint{1.402291in}{1.428785in}}%
\pgfpathcurveto{\pgfqpoint{1.394478in}{1.436598in}}{\pgfqpoint{1.383879in}{1.440989in}}{\pgfqpoint{1.372829in}{1.440989in}}%
\pgfpathcurveto{\pgfqpoint{1.361778in}{1.440989in}}{\pgfqpoint{1.351179in}{1.436598in}}{\pgfqpoint{1.343366in}{1.428785in}}%
\pgfpathcurveto{\pgfqpoint{1.335552in}{1.420971in}}{\pgfqpoint{1.331162in}{1.410372in}}{\pgfqpoint{1.331162in}{1.399322in}}%
\pgfpathcurveto{\pgfqpoint{1.331162in}{1.388272in}}{\pgfqpoint{1.335552in}{1.377673in}}{\pgfqpoint{1.343366in}{1.369859in}}%
\pgfpathcurveto{\pgfqpoint{1.351179in}{1.362046in}}{\pgfqpoint{1.361778in}{1.357655in}}{\pgfqpoint{1.372829in}{1.357655in}}%
\pgfpathclose%
\pgfusepath{stroke,fill}%
\end{pgfscope}%
\begin{pgfscope}%
\pgfpathrectangle{\pgfqpoint{0.466126in}{0.521603in}}{\pgfqpoint{3.720000in}{3.020000in}} %
\pgfusepath{clip}%
\pgfsetbuttcap%
\pgfsetroundjoin%
\definecolor{currentfill}{rgb}{0.139216,0.536867,0.960122}%
\pgfsetfillcolor{currentfill}%
\pgfsetlinewidth{1.003750pt}%
\definecolor{currentstroke}{rgb}{0.139216,0.536867,0.960122}%
\pgfsetstrokecolor{currentstroke}%
\pgfsetdash{}{0pt}%
\pgfpathmoveto{\pgfqpoint{0.707998in}{1.070975in}}%
\pgfpathcurveto{\pgfqpoint{0.719048in}{1.070975in}}{\pgfqpoint{0.729647in}{1.075365in}}{\pgfqpoint{0.737460in}{1.083179in}}%
\pgfpathcurveto{\pgfqpoint{0.745274in}{1.090993in}}{\pgfqpoint{0.749664in}{1.101592in}}{\pgfqpoint{0.749664in}{1.112642in}}%
\pgfpathcurveto{\pgfqpoint{0.749664in}{1.123692in}}{\pgfqpoint{0.745274in}{1.134291in}}{\pgfqpoint{0.737460in}{1.142104in}}%
\pgfpathcurveto{\pgfqpoint{0.729647in}{1.149918in}}{\pgfqpoint{0.719048in}{1.154308in}}{\pgfqpoint{0.707998in}{1.154308in}}%
\pgfpathcurveto{\pgfqpoint{0.696947in}{1.154308in}}{\pgfqpoint{0.686348in}{1.149918in}}{\pgfqpoint{0.678535in}{1.142104in}}%
\pgfpathcurveto{\pgfqpoint{0.670721in}{1.134291in}}{\pgfqpoint{0.666331in}{1.123692in}}{\pgfqpoint{0.666331in}{1.112642in}}%
\pgfpathcurveto{\pgfqpoint{0.666331in}{1.101592in}}{\pgfqpoint{0.670721in}{1.090993in}}{\pgfqpoint{0.678535in}{1.083179in}}%
\pgfpathcurveto{\pgfqpoint{0.686348in}{1.075365in}}{\pgfqpoint{0.696947in}{1.070975in}}{\pgfqpoint{0.707998in}{1.070975in}}%
\pgfpathclose%
\pgfusepath{stroke,fill}%
\end{pgfscope}%
\begin{pgfscope}%
\pgfsetbuttcap%
\pgfsetroundjoin%
\definecolor{currentfill}{rgb}{0.000000,0.000000,0.000000}%
\pgfsetfillcolor{currentfill}%
\pgfsetlinewidth{0.803000pt}%
\definecolor{currentstroke}{rgb}{0.000000,0.000000,0.000000}%
\pgfsetstrokecolor{currentstroke}%
\pgfsetdash{}{0pt}%
\pgfsys@defobject{currentmarker}{\pgfqpoint{0.000000in}{-0.048611in}}{\pgfqpoint{0.000000in}{0.000000in}}{%
\pgfpathmoveto{\pgfqpoint{0.000000in}{0.000000in}}%
\pgfpathlineto{\pgfqpoint{0.000000in}{-0.048611in}}%
\pgfusepath{stroke,fill}%
}%
\begin{pgfscope}%
\pgfsys@transformshift{1.268041in}{0.521603in}%
\pgfsys@useobject{currentmarker}{}%
\end{pgfscope}%
\end{pgfscope}%
\begin{pgfscope}%
\pgftext[x=1.268041in,y=0.424381in,,top]{\rmfamily\fontsize{10.000000}{12.000000}\selectfont \(\displaystyle 10^{-1}\)}%
\end{pgfscope}%
\begin{pgfscope}%
\pgfsetbuttcap%
\pgfsetroundjoin%
\definecolor{currentfill}{rgb}{0.000000,0.000000,0.000000}%
\pgfsetfillcolor{currentfill}%
\pgfsetlinewidth{0.803000pt}%
\definecolor{currentstroke}{rgb}{0.000000,0.000000,0.000000}%
\pgfsetstrokecolor{currentstroke}%
\pgfsetdash{}{0pt}%
\pgfsys@defobject{currentmarker}{\pgfqpoint{0.000000in}{-0.048611in}}{\pgfqpoint{0.000000in}{0.000000in}}{%
\pgfpathmoveto{\pgfqpoint{0.000000in}{0.000000in}}%
\pgfpathlineto{\pgfqpoint{0.000000in}{-0.048611in}}%
\pgfusepath{stroke,fill}%
}%
\begin{pgfscope}%
\pgfsys@transformshift{3.931945in}{0.521603in}%
\pgfsys@useobject{currentmarker}{}%
\end{pgfscope}%
\end{pgfscope}%
\begin{pgfscope}%
\pgftext[x=3.931945in,y=0.424381in,,top]{\rmfamily\fontsize{10.000000}{12.000000}\selectfont \(\displaystyle 10^{0}\)}%
\end{pgfscope}%
\begin{pgfscope}%
\pgfsetbuttcap%
\pgfsetroundjoin%
\definecolor{currentfill}{rgb}{0.000000,0.000000,0.000000}%
\pgfsetfillcolor{currentfill}%
\pgfsetlinewidth{0.602250pt}%
\definecolor{currentstroke}{rgb}{0.000000,0.000000,0.000000}%
\pgfsetstrokecolor{currentstroke}%
\pgfsetdash{}{0pt}%
\pgfsys@defobject{currentmarker}{\pgfqpoint{0.000000in}{-0.027778in}}{\pgfqpoint{0.000000in}{0.000000in}}{%
\pgfpathmoveto{\pgfqpoint{0.000000in}{0.000000in}}%
\pgfpathlineto{\pgfqpoint{0.000000in}{-0.027778in}}%
\pgfusepath{stroke,fill}%
}%
\begin{pgfscope}%
\pgfsys@transformshift{0.466126in}{0.521603in}%
\pgfsys@useobject{currentmarker}{}%
\end{pgfscope}%
\end{pgfscope}%
\begin{pgfscope}%
\pgfsetbuttcap%
\pgfsetroundjoin%
\definecolor{currentfill}{rgb}{0.000000,0.000000,0.000000}%
\pgfsetfillcolor{currentfill}%
\pgfsetlinewidth{0.602250pt}%
\definecolor{currentstroke}{rgb}{0.000000,0.000000,0.000000}%
\pgfsetstrokecolor{currentstroke}%
\pgfsetdash{}{0pt}%
\pgfsys@defobject{currentmarker}{\pgfqpoint{0.000000in}{-0.027778in}}{\pgfqpoint{0.000000in}{0.000000in}}{%
\pgfpathmoveto{\pgfqpoint{0.000000in}{0.000000in}}%
\pgfpathlineto{\pgfqpoint{0.000000in}{-0.027778in}}%
\pgfusepath{stroke,fill}%
}%
\begin{pgfscope}%
\pgfsys@transformshift{0.677058in}{0.521603in}%
\pgfsys@useobject{currentmarker}{}%
\end{pgfscope}%
\end{pgfscope}%
\begin{pgfscope}%
\pgfsetbuttcap%
\pgfsetroundjoin%
\definecolor{currentfill}{rgb}{0.000000,0.000000,0.000000}%
\pgfsetfillcolor{currentfill}%
\pgfsetlinewidth{0.602250pt}%
\definecolor{currentstroke}{rgb}{0.000000,0.000000,0.000000}%
\pgfsetstrokecolor{currentstroke}%
\pgfsetdash{}{0pt}%
\pgfsys@defobject{currentmarker}{\pgfqpoint{0.000000in}{-0.027778in}}{\pgfqpoint{0.000000in}{0.000000in}}{%
\pgfpathmoveto{\pgfqpoint{0.000000in}{0.000000in}}%
\pgfpathlineto{\pgfqpoint{0.000000in}{-0.027778in}}%
\pgfusepath{stroke,fill}%
}%
\begin{pgfscope}%
\pgfsys@transformshift{0.855397in}{0.521603in}%
\pgfsys@useobject{currentmarker}{}%
\end{pgfscope}%
\end{pgfscope}%
\begin{pgfscope}%
\pgfsetbuttcap%
\pgfsetroundjoin%
\definecolor{currentfill}{rgb}{0.000000,0.000000,0.000000}%
\pgfsetfillcolor{currentfill}%
\pgfsetlinewidth{0.602250pt}%
\definecolor{currentstroke}{rgb}{0.000000,0.000000,0.000000}%
\pgfsetstrokecolor{currentstroke}%
\pgfsetdash{}{0pt}%
\pgfsys@defobject{currentmarker}{\pgfqpoint{0.000000in}{-0.027778in}}{\pgfqpoint{0.000000in}{0.000000in}}{%
\pgfpathmoveto{\pgfqpoint{0.000000in}{0.000000in}}%
\pgfpathlineto{\pgfqpoint{0.000000in}{-0.027778in}}%
\pgfusepath{stroke,fill}%
}%
\begin{pgfscope}%
\pgfsys@transformshift{1.009882in}{0.521603in}%
\pgfsys@useobject{currentmarker}{}%
\end{pgfscope}%
\end{pgfscope}%
\begin{pgfscope}%
\pgfsetbuttcap%
\pgfsetroundjoin%
\definecolor{currentfill}{rgb}{0.000000,0.000000,0.000000}%
\pgfsetfillcolor{currentfill}%
\pgfsetlinewidth{0.602250pt}%
\definecolor{currentstroke}{rgb}{0.000000,0.000000,0.000000}%
\pgfsetstrokecolor{currentstroke}%
\pgfsetdash{}{0pt}%
\pgfsys@defobject{currentmarker}{\pgfqpoint{0.000000in}{-0.027778in}}{\pgfqpoint{0.000000in}{0.000000in}}{%
\pgfpathmoveto{\pgfqpoint{0.000000in}{0.000000in}}%
\pgfpathlineto{\pgfqpoint{0.000000in}{-0.027778in}}%
\pgfusepath{stroke,fill}%
}%
\begin{pgfscope}%
\pgfsys@transformshift{1.146148in}{0.521603in}%
\pgfsys@useobject{currentmarker}{}%
\end{pgfscope}%
\end{pgfscope}%
\begin{pgfscope}%
\pgfsetbuttcap%
\pgfsetroundjoin%
\definecolor{currentfill}{rgb}{0.000000,0.000000,0.000000}%
\pgfsetfillcolor{currentfill}%
\pgfsetlinewidth{0.602250pt}%
\definecolor{currentstroke}{rgb}{0.000000,0.000000,0.000000}%
\pgfsetstrokecolor{currentstroke}%
\pgfsetdash{}{0pt}%
\pgfsys@defobject{currentmarker}{\pgfqpoint{0.000000in}{-0.027778in}}{\pgfqpoint{0.000000in}{0.000000in}}{%
\pgfpathmoveto{\pgfqpoint{0.000000in}{0.000000in}}%
\pgfpathlineto{\pgfqpoint{0.000000in}{-0.027778in}}%
\pgfusepath{stroke,fill}%
}%
\begin{pgfscope}%
\pgfsys@transformshift{2.069956in}{0.521603in}%
\pgfsys@useobject{currentmarker}{}%
\end{pgfscope}%
\end{pgfscope}%
\begin{pgfscope}%
\pgfsetbuttcap%
\pgfsetroundjoin%
\definecolor{currentfill}{rgb}{0.000000,0.000000,0.000000}%
\pgfsetfillcolor{currentfill}%
\pgfsetlinewidth{0.602250pt}%
\definecolor{currentstroke}{rgb}{0.000000,0.000000,0.000000}%
\pgfsetstrokecolor{currentstroke}%
\pgfsetdash{}{0pt}%
\pgfsys@defobject{currentmarker}{\pgfqpoint{0.000000in}{-0.027778in}}{\pgfqpoint{0.000000in}{0.000000in}}{%
\pgfpathmoveto{\pgfqpoint{0.000000in}{0.000000in}}%
\pgfpathlineto{\pgfqpoint{0.000000in}{-0.027778in}}%
\pgfusepath{stroke,fill}%
}%
\begin{pgfscope}%
\pgfsys@transformshift{2.539046in}{0.521603in}%
\pgfsys@useobject{currentmarker}{}%
\end{pgfscope}%
\end{pgfscope}%
\begin{pgfscope}%
\pgfsetbuttcap%
\pgfsetroundjoin%
\definecolor{currentfill}{rgb}{0.000000,0.000000,0.000000}%
\pgfsetfillcolor{currentfill}%
\pgfsetlinewidth{0.602250pt}%
\definecolor{currentstroke}{rgb}{0.000000,0.000000,0.000000}%
\pgfsetstrokecolor{currentstroke}%
\pgfsetdash{}{0pt}%
\pgfsys@defobject{currentmarker}{\pgfqpoint{0.000000in}{-0.027778in}}{\pgfqpoint{0.000000in}{0.000000in}}{%
\pgfpathmoveto{\pgfqpoint{0.000000in}{0.000000in}}%
\pgfpathlineto{\pgfqpoint{0.000000in}{-0.027778in}}%
\pgfusepath{stroke,fill}%
}%
\begin{pgfscope}%
\pgfsys@transformshift{2.871871in}{0.521603in}%
\pgfsys@useobject{currentmarker}{}%
\end{pgfscope}%
\end{pgfscope}%
\begin{pgfscope}%
\pgfsetbuttcap%
\pgfsetroundjoin%
\definecolor{currentfill}{rgb}{0.000000,0.000000,0.000000}%
\pgfsetfillcolor{currentfill}%
\pgfsetlinewidth{0.602250pt}%
\definecolor{currentstroke}{rgb}{0.000000,0.000000,0.000000}%
\pgfsetstrokecolor{currentstroke}%
\pgfsetdash{}{0pt}%
\pgfsys@defobject{currentmarker}{\pgfqpoint{0.000000in}{-0.027778in}}{\pgfqpoint{0.000000in}{0.000000in}}{%
\pgfpathmoveto{\pgfqpoint{0.000000in}{0.000000in}}%
\pgfpathlineto{\pgfqpoint{0.000000in}{-0.027778in}}%
\pgfusepath{stroke,fill}%
}%
\begin{pgfscope}%
\pgfsys@transformshift{3.130030in}{0.521603in}%
\pgfsys@useobject{currentmarker}{}%
\end{pgfscope}%
\end{pgfscope}%
\begin{pgfscope}%
\pgfsetbuttcap%
\pgfsetroundjoin%
\definecolor{currentfill}{rgb}{0.000000,0.000000,0.000000}%
\pgfsetfillcolor{currentfill}%
\pgfsetlinewidth{0.602250pt}%
\definecolor{currentstroke}{rgb}{0.000000,0.000000,0.000000}%
\pgfsetstrokecolor{currentstroke}%
\pgfsetdash{}{0pt}%
\pgfsys@defobject{currentmarker}{\pgfqpoint{0.000000in}{-0.027778in}}{\pgfqpoint{0.000000in}{0.000000in}}{%
\pgfpathmoveto{\pgfqpoint{0.000000in}{0.000000in}}%
\pgfpathlineto{\pgfqpoint{0.000000in}{-0.027778in}}%
\pgfusepath{stroke,fill}%
}%
\begin{pgfscope}%
\pgfsys@transformshift{3.340961in}{0.521603in}%
\pgfsys@useobject{currentmarker}{}%
\end{pgfscope}%
\end{pgfscope}%
\begin{pgfscope}%
\pgfsetbuttcap%
\pgfsetroundjoin%
\definecolor{currentfill}{rgb}{0.000000,0.000000,0.000000}%
\pgfsetfillcolor{currentfill}%
\pgfsetlinewidth{0.602250pt}%
\definecolor{currentstroke}{rgb}{0.000000,0.000000,0.000000}%
\pgfsetstrokecolor{currentstroke}%
\pgfsetdash{}{0pt}%
\pgfsys@defobject{currentmarker}{\pgfqpoint{0.000000in}{-0.027778in}}{\pgfqpoint{0.000000in}{0.000000in}}{%
\pgfpathmoveto{\pgfqpoint{0.000000in}{0.000000in}}%
\pgfpathlineto{\pgfqpoint{0.000000in}{-0.027778in}}%
\pgfusepath{stroke,fill}%
}%
\begin{pgfscope}%
\pgfsys@transformshift{3.519301in}{0.521603in}%
\pgfsys@useobject{currentmarker}{}%
\end{pgfscope}%
\end{pgfscope}%
\begin{pgfscope}%
\pgfsetbuttcap%
\pgfsetroundjoin%
\definecolor{currentfill}{rgb}{0.000000,0.000000,0.000000}%
\pgfsetfillcolor{currentfill}%
\pgfsetlinewidth{0.602250pt}%
\definecolor{currentstroke}{rgb}{0.000000,0.000000,0.000000}%
\pgfsetstrokecolor{currentstroke}%
\pgfsetdash{}{0pt}%
\pgfsys@defobject{currentmarker}{\pgfqpoint{0.000000in}{-0.027778in}}{\pgfqpoint{0.000000in}{0.000000in}}{%
\pgfpathmoveto{\pgfqpoint{0.000000in}{0.000000in}}%
\pgfpathlineto{\pgfqpoint{0.000000in}{-0.027778in}}%
\pgfusepath{stroke,fill}%
}%
\begin{pgfscope}%
\pgfsys@transformshift{3.673786in}{0.521603in}%
\pgfsys@useobject{currentmarker}{}%
\end{pgfscope}%
\end{pgfscope}%
\begin{pgfscope}%
\pgfsetbuttcap%
\pgfsetroundjoin%
\definecolor{currentfill}{rgb}{0.000000,0.000000,0.000000}%
\pgfsetfillcolor{currentfill}%
\pgfsetlinewidth{0.602250pt}%
\definecolor{currentstroke}{rgb}{0.000000,0.000000,0.000000}%
\pgfsetstrokecolor{currentstroke}%
\pgfsetdash{}{0pt}%
\pgfsys@defobject{currentmarker}{\pgfqpoint{0.000000in}{-0.027778in}}{\pgfqpoint{0.000000in}{0.000000in}}{%
\pgfpathmoveto{\pgfqpoint{0.000000in}{0.000000in}}%
\pgfpathlineto{\pgfqpoint{0.000000in}{-0.027778in}}%
\pgfusepath{stroke,fill}%
}%
\begin{pgfscope}%
\pgfsys@transformshift{3.810051in}{0.521603in}%
\pgfsys@useobject{currentmarker}{}%
\end{pgfscope}%
\end{pgfscope}%
\begin{pgfscope}%
\pgftext[x=2.326126in,y=0.234413in,,top]{\rmfamily\fontsize{10.000000}{12.000000}\selectfont \(\displaystyle \mathbf{W}\mbox{e}\)}%
\end{pgfscope}%
\begin{pgfscope}%
\pgfsetbuttcap%
\pgfsetroundjoin%
\definecolor{currentfill}{rgb}{0.000000,0.000000,0.000000}%
\pgfsetfillcolor{currentfill}%
\pgfsetlinewidth{0.803000pt}%
\definecolor{currentstroke}{rgb}{0.000000,0.000000,0.000000}%
\pgfsetstrokecolor{currentstroke}%
\pgfsetdash{}{0pt}%
\pgfsys@defobject{currentmarker}{\pgfqpoint{-0.048611in}{0.000000in}}{\pgfqpoint{0.000000in}{0.000000in}}{%
\pgfpathmoveto{\pgfqpoint{0.000000in}{0.000000in}}%
\pgfpathlineto{\pgfqpoint{-0.048611in}{0.000000in}}%
\pgfusepath{stroke,fill}%
}%
\begin{pgfscope}%
\pgfsys@transformshift{0.466126in}{0.521603in}%
\pgfsys@useobject{currentmarker}{}%
\end{pgfscope}%
\end{pgfscope}%
\begin{pgfscope}%
\pgftext[x=0.299459in,y=0.468842in,left,base]{\rmfamily\fontsize{10.000000}{12.000000}\selectfont \(\displaystyle 2\)}%
\end{pgfscope}%
\begin{pgfscope}%
\pgfsetbuttcap%
\pgfsetroundjoin%
\definecolor{currentfill}{rgb}{0.000000,0.000000,0.000000}%
\pgfsetfillcolor{currentfill}%
\pgfsetlinewidth{0.803000pt}%
\definecolor{currentstroke}{rgb}{0.000000,0.000000,0.000000}%
\pgfsetstrokecolor{currentstroke}%
\pgfsetdash{}{0pt}%
\pgfsys@defobject{currentmarker}{\pgfqpoint{-0.048611in}{0.000000in}}{\pgfqpoint{0.000000in}{0.000000in}}{%
\pgfpathmoveto{\pgfqpoint{0.000000in}{0.000000in}}%
\pgfpathlineto{\pgfqpoint{-0.048611in}{0.000000in}}%
\pgfusepath{stroke,fill}%
}%
\begin{pgfscope}%
\pgfsys@transformshift{0.466126in}{1.086125in}%
\pgfsys@useobject{currentmarker}{}%
\end{pgfscope}%
\end{pgfscope}%
\begin{pgfscope}%
\pgftext[x=0.299459in,y=1.033363in,left,base]{\rmfamily\fontsize{10.000000}{12.000000}\selectfont \(\displaystyle 3\)}%
\end{pgfscope}%
\begin{pgfscope}%
\pgfsetbuttcap%
\pgfsetroundjoin%
\definecolor{currentfill}{rgb}{0.000000,0.000000,0.000000}%
\pgfsetfillcolor{currentfill}%
\pgfsetlinewidth{0.803000pt}%
\definecolor{currentstroke}{rgb}{0.000000,0.000000,0.000000}%
\pgfsetstrokecolor{currentstroke}%
\pgfsetdash{}{0pt}%
\pgfsys@defobject{currentmarker}{\pgfqpoint{-0.048611in}{0.000000in}}{\pgfqpoint{0.000000in}{0.000000in}}{%
\pgfpathmoveto{\pgfqpoint{0.000000in}{0.000000in}}%
\pgfpathlineto{\pgfqpoint{-0.048611in}{0.000000in}}%
\pgfusepath{stroke,fill}%
}%
\begin{pgfscope}%
\pgfsys@transformshift{0.466126in}{1.650646in}%
\pgfsys@useobject{currentmarker}{}%
\end{pgfscope}%
\end{pgfscope}%
\begin{pgfscope}%
\pgftext[x=0.299459in,y=1.597885in,left,base]{\rmfamily\fontsize{10.000000}{12.000000}\selectfont \(\displaystyle 4\)}%
\end{pgfscope}%
\begin{pgfscope}%
\pgfsetbuttcap%
\pgfsetroundjoin%
\definecolor{currentfill}{rgb}{0.000000,0.000000,0.000000}%
\pgfsetfillcolor{currentfill}%
\pgfsetlinewidth{0.803000pt}%
\definecolor{currentstroke}{rgb}{0.000000,0.000000,0.000000}%
\pgfsetstrokecolor{currentstroke}%
\pgfsetdash{}{0pt}%
\pgfsys@defobject{currentmarker}{\pgfqpoint{-0.048611in}{0.000000in}}{\pgfqpoint{0.000000in}{0.000000in}}{%
\pgfpathmoveto{\pgfqpoint{0.000000in}{0.000000in}}%
\pgfpathlineto{\pgfqpoint{-0.048611in}{0.000000in}}%
\pgfusepath{stroke,fill}%
}%
\begin{pgfscope}%
\pgfsys@transformshift{0.466126in}{2.215168in}%
\pgfsys@useobject{currentmarker}{}%
\end{pgfscope}%
\end{pgfscope}%
\begin{pgfscope}%
\pgftext[x=0.299459in,y=2.162406in,left,base]{\rmfamily\fontsize{10.000000}{12.000000}\selectfont \(\displaystyle 5\)}%
\end{pgfscope}%
\begin{pgfscope}%
\pgfsetbuttcap%
\pgfsetroundjoin%
\definecolor{currentfill}{rgb}{0.000000,0.000000,0.000000}%
\pgfsetfillcolor{currentfill}%
\pgfsetlinewidth{0.803000pt}%
\definecolor{currentstroke}{rgb}{0.000000,0.000000,0.000000}%
\pgfsetstrokecolor{currentstroke}%
\pgfsetdash{}{0pt}%
\pgfsys@defobject{currentmarker}{\pgfqpoint{-0.048611in}{0.000000in}}{\pgfqpoint{0.000000in}{0.000000in}}{%
\pgfpathmoveto{\pgfqpoint{0.000000in}{0.000000in}}%
\pgfpathlineto{\pgfqpoint{-0.048611in}{0.000000in}}%
\pgfusepath{stroke,fill}%
}%
\begin{pgfscope}%
\pgfsys@transformshift{0.466126in}{2.779690in}%
\pgfsys@useobject{currentmarker}{}%
\end{pgfscope}%
\end{pgfscope}%
\begin{pgfscope}%
\pgftext[x=0.299459in,y=2.726928in,left,base]{\rmfamily\fontsize{10.000000}{12.000000}\selectfont \(\displaystyle 6\)}%
\end{pgfscope}%
\begin{pgfscope}%
\pgfsetbuttcap%
\pgfsetroundjoin%
\definecolor{currentfill}{rgb}{0.000000,0.000000,0.000000}%
\pgfsetfillcolor{currentfill}%
\pgfsetlinewidth{0.803000pt}%
\definecolor{currentstroke}{rgb}{0.000000,0.000000,0.000000}%
\pgfsetstrokecolor{currentstroke}%
\pgfsetdash{}{0pt}%
\pgfsys@defobject{currentmarker}{\pgfqpoint{-0.048611in}{0.000000in}}{\pgfqpoint{0.000000in}{0.000000in}}{%
\pgfpathmoveto{\pgfqpoint{0.000000in}{0.000000in}}%
\pgfpathlineto{\pgfqpoint{-0.048611in}{0.000000in}}%
\pgfusepath{stroke,fill}%
}%
\begin{pgfscope}%
\pgfsys@transformshift{0.466126in}{3.344211in}%
\pgfsys@useobject{currentmarker}{}%
\end{pgfscope}%
\end{pgfscope}%
\begin{pgfscope}%
\pgftext[x=0.299459in,y=3.291450in,left,base]{\rmfamily\fontsize{10.000000}{12.000000}\selectfont \(\displaystyle 7\)}%
\end{pgfscope}%
\begin{pgfscope}%
\pgftext[x=0.243904in,y=2.031603in,,bottom,rotate=90.000000]{\rmfamily\fontsize{10.000000}{12.000000}\selectfont \(\displaystyle t_j/ \tau\)}%
\end{pgfscope}%
\begin{pgfscope}%
\pgfpathrectangle{\pgfqpoint{0.466126in}{0.521603in}}{\pgfqpoint{3.720000in}{3.020000in}} %
\pgfusepath{clip}%
\pgfsetbuttcap%
\pgfsetroundjoin%
\pgfsetlinewidth{1.505625pt}%
\definecolor{currentstroke}{rgb}{0.000000,0.000000,0.000000}%
\pgfsetstrokecolor{currentstroke}%
\pgfsetdash{{5.550000pt}{2.400000pt}}{0.000000pt}%
\pgfpathmoveto{\pgfqpoint{0.456126in}{0.634508in}}%
\pgfpathlineto{\pgfqpoint{0.456740in}{0.634508in}}%
\pgfpathlineto{\pgfqpoint{0.667278in}{0.634508in}}%
\pgfpathlineto{\pgfqpoint{0.845337in}{0.634508in}}%
\pgfpathlineto{\pgfqpoint{0.999612in}{0.634508in}}%
\pgfpathlineto{\pgfqpoint{1.135713in}{0.634508in}}%
\pgfpathlineto{\pgfqpoint{1.257476in}{0.634508in}}%
\pgfpathlineto{\pgfqpoint{1.367635in}{0.634508in}}%
\pgfpathlineto{\pgfqpoint{1.468210in}{0.634508in}}%
\pgfpathlineto{\pgfqpoint{1.560738in}{0.634508in}}%
\pgfpathlineto{\pgfqpoint{1.646410in}{0.634508in}}%
\pgfpathlineto{\pgfqpoint{1.726173in}{0.634508in}}%
\pgfpathlineto{\pgfqpoint{1.800789in}{0.634508in}}%
\pgfpathlineto{\pgfqpoint{1.870884in}{0.634508in}}%
\pgfpathlineto{\pgfqpoint{1.936973in}{0.634508in}}%
\pgfpathlineto{\pgfqpoint{1.999490in}{0.634508in}}%
\pgfpathlineto{\pgfqpoint{2.058801in}{0.634508in}}%
\pgfpathlineto{\pgfqpoint{2.115219in}{0.634508in}}%
\pgfpathlineto{\pgfqpoint{2.169013in}{0.634508in}}%
\pgfpathlineto{\pgfqpoint{2.220417in}{0.634508in}}%
\pgfpathlineto{\pgfqpoint{2.269634in}{0.634508in}}%
\pgfpathlineto{\pgfqpoint{2.316842in}{0.634508in}}%
\pgfpathlineto{\pgfqpoint{2.362199in}{0.634508in}}%
\pgfpathlineto{\pgfqpoint{2.405844in}{0.634508in}}%
\pgfpathlineto{\pgfqpoint{2.447903in}{0.634508in}}%
\pgfpathlineto{\pgfqpoint{2.488486in}{0.634508in}}%
\pgfpathlineto{\pgfqpoint{2.527694in}{0.634508in}}%
\pgfpathlineto{\pgfqpoint{2.565617in}{0.634508in}}%
\pgfpathlineto{\pgfqpoint{2.602335in}{0.634508in}}%
\pgfpathlineto{\pgfqpoint{2.637925in}{0.634508in}}%
\pgfpathlineto{\pgfqpoint{2.672452in}{0.634508in}}%
\pgfpathlineto{\pgfqpoint{2.705978in}{0.634508in}}%
\pgfpathlineto{\pgfqpoint{2.738560in}{0.634508in}}%
\pgfpathlineto{\pgfqpoint{2.770249in}{0.634508in}}%
\pgfpathlineto{\pgfqpoint{2.801094in}{0.634508in}}%
\pgfpathlineto{\pgfqpoint{2.831138in}{0.634508in}}%
\pgfpathlineto{\pgfqpoint{2.860421in}{0.634508in}}%
\pgfpathlineto{\pgfqpoint{2.888981in}{0.634508in}}%
\pgfpathlineto{\pgfqpoint{2.916853in}{0.634508in}}%
\pgfpathlineto{\pgfqpoint{2.944069in}{0.634508in}}%
\pgfpathlineto{\pgfqpoint{2.970660in}{0.634508in}}%
\pgfpathlineto{\pgfqpoint{2.996653in}{0.634508in}}%
\pgfpathlineto{\pgfqpoint{3.022075in}{0.634508in}}%
\pgfpathlineto{\pgfqpoint{3.046951in}{0.634508in}}%
\pgfpathlineto{\pgfqpoint{3.071303in}{0.634508in}}%
\pgfpathlineto{\pgfqpoint{3.095152in}{0.634508in}}%
\pgfpathlineto{\pgfqpoint{3.118521in}{0.634508in}}%
\pgfpathlineto{\pgfqpoint{3.141426in}{0.634508in}}%
\pgfpathlineto{\pgfqpoint{3.163887in}{0.634508in}}%
\pgfpathlineto{\pgfqpoint{3.185920in}{0.634508in}}%
\pgfpathlineto{\pgfqpoint{3.207541in}{0.634508in}}%
\pgfpathlineto{\pgfqpoint{3.228765in}{0.634508in}}%
\pgfpathlineto{\pgfqpoint{3.249607in}{0.634508in}}%
\pgfpathlineto{\pgfqpoint{3.270081in}{0.634508in}}%
\pgfpathlineto{\pgfqpoint{3.290198in}{0.634508in}}%
\pgfpathlineto{\pgfqpoint{3.309971in}{0.634508in}}%
\pgfpathlineto{\pgfqpoint{3.329412in}{0.634508in}}%
\pgfpathlineto{\pgfqpoint{3.348532in}{0.634508in}}%
\pgfpathlineto{\pgfqpoint{3.367341in}{0.634508in}}%
\pgfpathlineto{\pgfqpoint{3.385849in}{0.634508in}}%
\pgfpathlineto{\pgfqpoint{3.404066in}{0.634508in}}%
\pgfpathlineto{\pgfqpoint{3.422000in}{0.634508in}}%
\pgfpathlineto{\pgfqpoint{3.439661in}{0.634508in}}%
\pgfpathlineto{\pgfqpoint{3.457056in}{0.634508in}}%
\pgfpathlineto{\pgfqpoint{3.474193in}{0.634508in}}%
\pgfpathlineto{\pgfqpoint{3.491080in}{0.634508in}}%
\pgfpathlineto{\pgfqpoint{3.507724in}{0.634508in}}%
\pgfpathlineto{\pgfqpoint{3.524132in}{0.634508in}}%
\pgfpathlineto{\pgfqpoint{3.540311in}{0.634508in}}%
\pgfpathlineto{\pgfqpoint{3.556266in}{0.634508in}}%
\pgfpathlineto{\pgfqpoint{3.572005in}{0.634508in}}%
\pgfpathlineto{\pgfqpoint{3.587532in}{0.634508in}}%
\pgfpathlineto{\pgfqpoint{3.602854in}{0.634508in}}%
\pgfpathlineto{\pgfqpoint{3.617975in}{0.634508in}}%
\pgfpathlineto{\pgfqpoint{3.632901in}{0.634508in}}%
\pgfpathlineto{\pgfqpoint{3.647637in}{0.634508in}}%
\pgfpathlineto{\pgfqpoint{3.662188in}{0.634508in}}%
\pgfpathlineto{\pgfqpoint{3.676558in}{0.634508in}}%
\pgfpathlineto{\pgfqpoint{3.690752in}{0.634508in}}%
\pgfpathlineto{\pgfqpoint{3.704773in}{0.634508in}}%
\pgfpathlineto{\pgfqpoint{3.718627in}{0.634508in}}%
\pgfpathlineto{\pgfqpoint{3.732317in}{0.634508in}}%
\pgfpathlineto{\pgfqpoint{3.745847in}{0.634508in}}%
\pgfpathlineto{\pgfqpoint{3.759220in}{0.634508in}}%
\pgfpathlineto{\pgfqpoint{3.772441in}{0.634508in}}%
\pgfpathlineto{\pgfqpoint{3.785512in}{0.634508in}}%
\pgfpathlineto{\pgfqpoint{3.798437in}{0.634508in}}%
\pgfpathlineto{\pgfqpoint{3.811219in}{0.634508in}}%
\pgfpathlineto{\pgfqpoint{3.823862in}{0.634508in}}%
\pgfpathlineto{\pgfqpoint{3.836368in}{0.634508in}}%
\pgfpathlineto{\pgfqpoint{3.848740in}{0.634508in}}%
\pgfpathlineto{\pgfqpoint{3.860981in}{0.634508in}}%
\pgfpathlineto{\pgfqpoint{3.873095in}{0.634508in}}%
\pgfpathlineto{\pgfqpoint{3.885082in}{0.634508in}}%
\pgfpathlineto{\pgfqpoint{3.896947in}{0.634508in}}%
\pgfpathlineto{\pgfqpoint{3.908691in}{0.634508in}}%
\pgfpathlineto{\pgfqpoint{3.920317in}{0.634508in}}%
\pgfusepath{stroke}%
\end{pgfscope}%
\begin{pgfscope}%
\pgfpathrectangle{\pgfqpoint{0.466126in}{0.521603in}}{\pgfqpoint{3.720000in}{3.020000in}} %
\pgfusepath{clip}%
\pgfsetbuttcap%
\pgfsetroundjoin%
\pgfsetlinewidth{1.505625pt}%
\definecolor{currentstroke}{rgb}{0.501961,0.501961,0.501961}%
\pgfsetstrokecolor{currentstroke}%
\pgfsetdash{{9.600000pt}{2.400000pt}{1.500000pt}{2.400000pt}}{0.000000pt}%
\pgfpathmoveto{\pgfqpoint{0.456126in}{2.392640in}}%
\pgfpathlineto{\pgfqpoint{0.456740in}{2.392341in}}%
\pgfpathlineto{\pgfqpoint{0.667278in}{2.289608in}}%
\pgfpathlineto{\pgfqpoint{0.845337in}{2.202724in}}%
\pgfpathlineto{\pgfqpoint{0.999612in}{2.127445in}}%
\pgfpathlineto{\pgfqpoint{1.135713in}{2.061034in}}%
\pgfpathlineto{\pgfqpoint{1.257476in}{2.001619in}}%
\pgfpathlineto{\pgfqpoint{1.367635in}{1.947867in}}%
\pgfpathlineto{\pgfqpoint{1.468210in}{1.898791in}}%
\pgfpathlineto{\pgfqpoint{1.560738in}{1.853642in}}%
\pgfpathlineto{\pgfqpoint{1.646410in}{1.811838in}}%
\pgfpathlineto{\pgfqpoint{1.726173in}{1.772917in}}%
\pgfpathlineto{\pgfqpoint{1.800789in}{1.736508in}}%
\pgfpathlineto{\pgfqpoint{1.870884in}{1.702305in}}%
\pgfpathlineto{\pgfqpoint{1.936973in}{1.670057in}}%
\pgfpathlineto{\pgfqpoint{1.999490in}{1.639552in}}%
\pgfpathlineto{\pgfqpoint{2.058801in}{1.610611in}}%
\pgfpathlineto{\pgfqpoint{2.115219in}{1.583081in}}%
\pgfpathlineto{\pgfqpoint{2.169013in}{1.556832in}}%
\pgfpathlineto{\pgfqpoint{2.220417in}{1.531750in}}%
\pgfpathlineto{\pgfqpoint{2.269634in}{1.507734in}}%
\pgfpathlineto{\pgfqpoint{2.316842in}{1.484699in}}%
\pgfpathlineto{\pgfqpoint{2.362199in}{1.462567in}}%
\pgfpathlineto{\pgfqpoint{2.405844in}{1.441270in}}%
\pgfpathlineto{\pgfqpoint{2.447903in}{1.420747in}}%
\pgfpathlineto{\pgfqpoint{2.488486in}{1.400944in}}%
\pgfpathlineto{\pgfqpoint{2.527694in}{1.381813in}}%
\pgfpathlineto{\pgfqpoint{2.565617in}{1.363308in}}%
\pgfpathlineto{\pgfqpoint{2.602335in}{1.345391in}}%
\pgfpathlineto{\pgfqpoint{2.637925in}{1.328026in}}%
\pgfpathlineto{\pgfqpoint{2.672452in}{1.311178in}}%
\pgfpathlineto{\pgfqpoint{2.705978in}{1.294819in}}%
\pgfpathlineto{\pgfqpoint{2.738560in}{1.278920in}}%
\pgfpathlineto{\pgfqpoint{2.770249in}{1.263457in}}%
\pgfpathlineto{\pgfqpoint{2.801094in}{1.248407in}}%
\pgfpathlineto{\pgfqpoint{2.831138in}{1.233747in}}%
\pgfpathlineto{\pgfqpoint{2.860421in}{1.219458in}}%
\pgfpathlineto{\pgfqpoint{2.888981in}{1.205522in}}%
\pgfpathlineto{\pgfqpoint{2.916853in}{1.191922in}}%
\pgfpathlineto{\pgfqpoint{2.944069in}{1.178642in}}%
\pgfpathlineto{\pgfqpoint{2.970660in}{1.165667in}}%
\pgfpathlineto{\pgfqpoint{2.996653in}{1.152983in}}%
\pgfpathlineto{\pgfqpoint{3.022075in}{1.140578in}}%
\pgfpathlineto{\pgfqpoint{3.046951in}{1.128440in}}%
\pgfpathlineto{\pgfqpoint{3.071303in}{1.116558in}}%
\pgfpathlineto{\pgfqpoint{3.095152in}{1.104920in}}%
\pgfpathlineto{\pgfqpoint{3.118521in}{1.093518in}}%
\pgfpathlineto{\pgfqpoint{3.141426in}{1.082341in}}%
\pgfpathlineto{\pgfqpoint{3.163887in}{1.071381in}}%
\pgfpathlineto{\pgfqpoint{3.185920in}{1.060630in}}%
\pgfpathlineto{\pgfqpoint{3.207541in}{1.050080in}}%
\pgfpathlineto{\pgfqpoint{3.228765in}{1.039723in}}%
\pgfpathlineto{\pgfqpoint{3.249607in}{1.029553in}}%
\pgfpathlineto{\pgfqpoint{3.270081in}{1.019563in}}%
\pgfpathlineto{\pgfqpoint{3.290198in}{1.009747in}}%
\pgfpathlineto{\pgfqpoint{3.309971in}{1.000099in}}%
\pgfpathlineto{\pgfqpoint{3.329412in}{0.990612in}}%
\pgfpathlineto{\pgfqpoint{3.348532in}{0.981283in}}%
\pgfpathlineto{\pgfqpoint{3.367341in}{0.972105in}}%
\pgfpathlineto{\pgfqpoint{3.385849in}{0.963074in}}%
\pgfpathlineto{\pgfqpoint{3.404066in}{0.954185in}}%
\pgfpathlineto{\pgfqpoint{3.422000in}{0.945434in}}%
\pgfpathlineto{\pgfqpoint{3.439661in}{0.936816in}}%
\pgfpathlineto{\pgfqpoint{3.457056in}{0.928328in}}%
\pgfpathlineto{\pgfqpoint{3.474193in}{0.919966in}}%
\pgfpathlineto{\pgfqpoint{3.491080in}{0.911726in}}%
\pgfpathlineto{\pgfqpoint{3.507724in}{0.903605in}}%
\pgfpathlineto{\pgfqpoint{3.524132in}{0.895598in}}%
\pgfpathlineto{\pgfqpoint{3.540311in}{0.887704in}}%
\pgfpathlineto{\pgfqpoint{3.556266in}{0.879918in}}%
\pgfpathlineto{\pgfqpoint{3.572005in}{0.872239in}}%
\pgfpathlineto{\pgfqpoint{3.587532in}{0.864662in}}%
\pgfpathlineto{\pgfqpoint{3.602854in}{0.857186in}}%
\pgfpathlineto{\pgfqpoint{3.617975in}{0.849807in}}%
\pgfpathlineto{\pgfqpoint{3.632901in}{0.842524in}}%
\pgfpathlineto{\pgfqpoint{3.647637in}{0.835334in}}%
\pgfpathlineto{\pgfqpoint{3.662188in}{0.828234in}}%
\pgfpathlineto{\pgfqpoint{3.676558in}{0.821222in}}%
\pgfpathlineto{\pgfqpoint{3.690752in}{0.814296in}}%
\pgfpathlineto{\pgfqpoint{3.704773in}{0.807454in}}%
\pgfpathlineto{\pgfqpoint{3.718627in}{0.800694in}}%
\pgfpathlineto{\pgfqpoint{3.732317in}{0.794014in}}%
\pgfpathlineto{\pgfqpoint{3.745847in}{0.787412in}}%
\pgfpathlineto{\pgfqpoint{3.759220in}{0.780886in}}%
\pgfpathlineto{\pgfqpoint{3.772441in}{0.774435in}}%
\pgfpathlineto{\pgfqpoint{3.785512in}{0.768057in}}%
\pgfpathlineto{\pgfqpoint{3.798437in}{0.761751in}}%
\pgfpathlineto{\pgfqpoint{3.811219in}{0.755513in}}%
\pgfpathlineto{\pgfqpoint{3.823862in}{0.749344in}}%
\pgfpathlineto{\pgfqpoint{3.836368in}{0.743242in}}%
\pgfpathlineto{\pgfqpoint{3.848740in}{0.737205in}}%
\pgfpathlineto{\pgfqpoint{3.860981in}{0.731232in}}%
\pgfpathlineto{\pgfqpoint{3.873095in}{0.725321in}}%
\pgfpathlineto{\pgfqpoint{3.885082in}{0.719472in}}%
\pgfpathlineto{\pgfqpoint{3.896947in}{0.713682in}}%
\pgfpathlineto{\pgfqpoint{3.908691in}{0.707952in}}%
\pgfpathlineto{\pgfqpoint{3.920317in}{0.702279in}}%
\pgfusepath{stroke}%
\end{pgfscope}%
\begin{pgfscope}%
\pgfpathrectangle{\pgfqpoint{0.466126in}{0.521603in}}{\pgfqpoint{3.720000in}{3.020000in}} %
\pgfusepath{clip}%
\pgfsetrectcap%
\pgfsetroundjoin%
\pgfsetlinewidth{1.505625pt}%
\definecolor{currentstroke}{rgb}{0.174510,0.872120,0.862929}%
\pgfsetstrokecolor{currentstroke}%
\pgfsetdash{}{0pt}%
\pgfpathmoveto{\pgfqpoint{0.456126in}{1.360815in}}%
\pgfpathlineto{\pgfqpoint{0.456740in}{1.360702in}}%
\pgfpathlineto{\pgfqpoint{0.667278in}{1.323228in}}%
\pgfpathlineto{\pgfqpoint{0.845337in}{1.292751in}}%
\pgfpathlineto{\pgfqpoint{0.999612in}{1.267322in}}%
\pgfpathlineto{\pgfqpoint{1.135713in}{1.245679in}}%
\pgfpathlineto{\pgfqpoint{1.257476in}{1.226964in}}%
\pgfpathlineto{\pgfqpoint{1.367635in}{1.210570in}}%
\pgfpathlineto{\pgfqpoint{1.468210in}{1.196051in}}%
\pgfpathlineto{\pgfqpoint{1.560738in}{1.183076in}}%
\pgfpathlineto{\pgfqpoint{1.646410in}{1.171387in}}%
\pgfpathlineto{\pgfqpoint{1.726173in}{1.160786in}}%
\pgfpathlineto{\pgfqpoint{1.800789in}{1.151113in}}%
\pgfpathlineto{\pgfqpoint{1.870884in}{1.142240in}}%
\pgfpathlineto{\pgfqpoint{1.936973in}{1.134062in}}%
\pgfpathlineto{\pgfqpoint{1.999490in}{1.126493in}}%
\pgfpathlineto{\pgfqpoint{2.058801in}{1.119462in}}%
\pgfpathlineto{\pgfqpoint{2.115219in}{1.112906in}}%
\pgfpathlineto{\pgfqpoint{2.169013in}{1.106775in}}%
\pgfpathlineto{\pgfqpoint{2.220417in}{1.101025in}}%
\pgfpathlineto{\pgfqpoint{2.269634in}{1.095617in}}%
\pgfpathlineto{\pgfqpoint{2.316842in}{1.090520in}}%
\pgfpathlineto{\pgfqpoint{2.362199in}{1.085704in}}%
\pgfpathlineto{\pgfqpoint{2.405844in}{1.081145in}}%
\pgfpathlineto{\pgfqpoint{2.447903in}{1.076820in}}%
\pgfpathlineto{\pgfqpoint{2.488486in}{1.072710in}}%
\pgfpathlineto{\pgfqpoint{2.527694in}{1.068798in}}%
\pgfpathlineto{\pgfqpoint{2.565617in}{1.065069in}}%
\pgfpathlineto{\pgfqpoint{2.602335in}{1.061508in}}%
\pgfpathlineto{\pgfqpoint{2.637925in}{1.058103in}}%
\pgfpathlineto{\pgfqpoint{2.672452in}{1.054844in}}%
\pgfpathlineto{\pgfqpoint{2.705978in}{1.051719in}}%
\pgfpathlineto{\pgfqpoint{2.738560in}{1.048721in}}%
\pgfpathlineto{\pgfqpoint{2.770249in}{1.045841in}}%
\pgfpathlineto{\pgfqpoint{2.801094in}{1.043071in}}%
\pgfpathlineto{\pgfqpoint{2.831138in}{1.040404in}}%
\pgfpathlineto{\pgfqpoint{2.860421in}{1.037835in}}%
\pgfpathlineto{\pgfqpoint{2.888981in}{1.035357in}}%
\pgfpathlineto{\pgfqpoint{2.916853in}{1.032965in}}%
\pgfpathlineto{\pgfqpoint{2.944069in}{1.030654in}}%
\pgfpathlineto{\pgfqpoint{2.970660in}{1.028420in}}%
\pgfpathlineto{\pgfqpoint{2.996653in}{1.026258in}}%
\pgfpathlineto{\pgfqpoint{3.022075in}{1.024165in}}%
\pgfpathlineto{\pgfqpoint{3.046951in}{1.022138in}}%
\pgfpathlineto{\pgfqpoint{3.071303in}{1.020172in}}%
\pgfpathlineto{\pgfqpoint{3.095152in}{1.018265in}}%
\pgfpathlineto{\pgfqpoint{3.118521in}{1.016414in}}%
\pgfpathlineto{\pgfqpoint{3.141426in}{1.014616in}}%
\pgfpathlineto{\pgfqpoint{3.163887in}{1.012869in}}%
\pgfpathlineto{\pgfqpoint{3.185920in}{1.011170in}}%
\pgfpathlineto{\pgfqpoint{3.207541in}{1.009517in}}%
\pgfpathlineto{\pgfqpoint{3.228765in}{1.007908in}}%
\pgfpathlineto{\pgfqpoint{3.249607in}{1.006342in}}%
\pgfpathlineto{\pgfqpoint{3.270081in}{1.004816in}}%
\pgfpathlineto{\pgfqpoint{3.290198in}{1.003329in}}%
\pgfpathlineto{\pgfqpoint{3.309971in}{1.001879in}}%
\pgfpathlineto{\pgfqpoint{3.329412in}{1.000464in}}%
\pgfpathlineto{\pgfqpoint{3.348532in}{0.999083in}}%
\pgfpathlineto{\pgfqpoint{3.367341in}{0.997736in}}%
\pgfpathlineto{\pgfqpoint{3.385849in}{0.996419in}}%
\pgfpathlineto{\pgfqpoint{3.404066in}{0.995133in}}%
\pgfpathlineto{\pgfqpoint{3.422000in}{0.993876in}}%
\pgfpathlineto{\pgfqpoint{3.439661in}{0.992648in}}%
\pgfpathlineto{\pgfqpoint{3.457056in}{0.991446in}}%
\pgfpathlineto{\pgfqpoint{3.474193in}{0.990270in}}%
\pgfpathlineto{\pgfqpoint{3.491080in}{0.989119in}}%
\pgfpathlineto{\pgfqpoint{3.507724in}{0.987993in}}%
\pgfpathlineto{\pgfqpoint{3.524132in}{0.986890in}}%
\pgfpathlineto{\pgfqpoint{3.540311in}{0.985810in}}%
\pgfpathlineto{\pgfqpoint{3.556266in}{0.984751in}}%
\pgfpathlineto{\pgfqpoint{3.572005in}{0.983714in}}%
\pgfpathlineto{\pgfqpoint{3.587532in}{0.982697in}}%
\pgfpathlineto{\pgfqpoint{3.602854in}{0.981700in}}%
\pgfpathlineto{\pgfqpoint{3.617975in}{0.980722in}}%
\pgfpathlineto{\pgfqpoint{3.632901in}{0.979762in}}%
\pgfpathlineto{\pgfqpoint{3.647637in}{0.978821in}}%
\pgfpathlineto{\pgfqpoint{3.662188in}{0.977896in}}%
\pgfpathlineto{\pgfqpoint{3.676558in}{0.976989in}}%
\pgfpathlineto{\pgfqpoint{3.690752in}{0.976098in}}%
\pgfpathlineto{\pgfqpoint{3.704773in}{0.975223in}}%
\pgfpathlineto{\pgfqpoint{3.718627in}{0.974363in}}%
\pgfpathlineto{\pgfqpoint{3.732317in}{0.973519in}}%
\pgfpathlineto{\pgfqpoint{3.745847in}{0.972688in}}%
\pgfpathlineto{\pgfqpoint{3.759220in}{0.971872in}}%
\pgfpathlineto{\pgfqpoint{3.772441in}{0.971070in}}%
\pgfpathlineto{\pgfqpoint{3.785512in}{0.970281in}}%
\pgfpathlineto{\pgfqpoint{3.798437in}{0.969505in}}%
\pgfpathlineto{\pgfqpoint{3.811219in}{0.968742in}}%
\pgfpathlineto{\pgfqpoint{3.823862in}{0.967991in}}%
\pgfpathlineto{\pgfqpoint{3.836368in}{0.967251in}}%
\pgfpathlineto{\pgfqpoint{3.848740in}{0.966524in}}%
\pgfpathlineto{\pgfqpoint{3.860981in}{0.965808in}}%
\pgfpathlineto{\pgfqpoint{3.873095in}{0.965103in}}%
\pgfpathlineto{\pgfqpoint{3.885082in}{0.964408in}}%
\pgfpathlineto{\pgfqpoint{3.896947in}{0.963724in}}%
\pgfpathlineto{\pgfqpoint{3.908691in}{0.963051in}}%
\pgfpathlineto{\pgfqpoint{3.920317in}{0.962387in}}%
\pgfusepath{stroke}%
\end{pgfscope}%
\begin{pgfscope}%
\pgfpathrectangle{\pgfqpoint{0.466126in}{0.521603in}}{\pgfqpoint{3.720000in}{3.020000in}} %
\pgfusepath{clip}%
\pgfsetrectcap%
\pgfsetroundjoin%
\pgfsetlinewidth{1.505625pt}%
\definecolor{currentstroke}{rgb}{0.174510,0.872120,0.862929}%
\pgfsetstrokecolor{currentstroke}%
\pgfsetdash{}{0pt}%
\pgfpathmoveto{\pgfqpoint{0.456126in}{1.298707in}}%
\pgfpathlineto{\pgfqpoint{0.456740in}{1.298599in}}%
\pgfpathlineto{\pgfqpoint{0.667278in}{1.261802in}}%
\pgfpathlineto{\pgfqpoint{0.845337in}{1.230924in}}%
\pgfpathlineto{\pgfqpoint{0.999612in}{1.204496in}}%
\pgfpathlineto{\pgfqpoint{1.135713in}{1.181522in}}%
\pgfpathlineto{\pgfqpoint{1.257476in}{1.161297in}}%
\pgfpathlineto{\pgfqpoint{1.367635in}{1.143307in}}%
\pgfpathlineto{\pgfqpoint{1.468210in}{1.127164in}}%
\pgfpathlineto{\pgfqpoint{1.560738in}{1.112568in}}%
\pgfpathlineto{\pgfqpoint{1.646410in}{1.099285in}}%
\pgfpathlineto{\pgfqpoint{1.726173in}{1.087128in}}%
\pgfpathlineto{\pgfqpoint{1.800789in}{1.075945in}}%
\pgfpathlineto{\pgfqpoint{1.870884in}{1.065612in}}%
\pgfpathlineto{\pgfqpoint{1.936973in}{1.056026in}}%
\pgfpathlineto{\pgfqpoint{1.999490in}{1.047100in}}%
\pgfpathlineto{\pgfqpoint{2.058801in}{1.038762in}}%
\pgfpathlineto{\pgfqpoint{2.115219in}{1.030949in}}%
\pgfpathlineto{\pgfqpoint{2.169013in}{1.023609in}}%
\pgfpathlineto{\pgfqpoint{2.220417in}{1.016695in}}%
\pgfpathlineto{\pgfqpoint{2.269634in}{1.010168in}}%
\pgfpathlineto{\pgfqpoint{2.316842in}{1.003993in}}%
\pgfpathlineto{\pgfqpoint{2.362199in}{0.998138in}}%
\pgfpathlineto{\pgfqpoint{2.405844in}{0.992578in}}%
\pgfpathlineto{\pgfqpoint{2.447903in}{0.987288in}}%
\pgfpathlineto{\pgfqpoint{2.488486in}{0.982247in}}%
\pgfpathlineto{\pgfqpoint{2.527694in}{0.977437in}}%
\pgfpathlineto{\pgfqpoint{2.565617in}{0.972839in}}%
\pgfpathlineto{\pgfqpoint{2.602335in}{0.968439in}}%
\pgfpathlineto{\pgfqpoint{2.637925in}{0.964222in}}%
\pgfpathlineto{\pgfqpoint{2.672452in}{0.960178in}}%
\pgfpathlineto{\pgfqpoint{2.705978in}{0.956293in}}%
\pgfpathlineto{\pgfqpoint{2.738560in}{0.952558in}}%
\pgfpathlineto{\pgfqpoint{2.770249in}{0.948964in}}%
\pgfpathlineto{\pgfqpoint{2.801094in}{0.945501in}}%
\pgfpathlineto{\pgfqpoint{2.831138in}{0.942162in}}%
\pgfpathlineto{\pgfqpoint{2.860421in}{0.938940in}}%
\pgfpathlineto{\pgfqpoint{2.888981in}{0.935828in}}%
\pgfpathlineto{\pgfqpoint{2.916853in}{0.932820in}}%
\pgfpathlineto{\pgfqpoint{2.944069in}{0.929910in}}%
\pgfpathlineto{\pgfqpoint{2.970660in}{0.927093in}}%
\pgfpathlineto{\pgfqpoint{2.996653in}{0.924365in}}%
\pgfpathlineto{\pgfqpoint{3.022075in}{0.921720in}}%
\pgfpathlineto{\pgfqpoint{3.046951in}{0.919154in}}%
\pgfpathlineto{\pgfqpoint{3.071303in}{0.916664in}}%
\pgfpathlineto{\pgfqpoint{3.095152in}{0.914246in}}%
\pgfpathlineto{\pgfqpoint{3.118521in}{0.911896in}}%
\pgfpathlineto{\pgfqpoint{3.141426in}{0.909612in}}%
\pgfpathlineto{\pgfqpoint{3.163887in}{0.907390in}}%
\pgfpathlineto{\pgfqpoint{3.185920in}{0.905228in}}%
\pgfpathlineto{\pgfqpoint{3.207541in}{0.903122in}}%
\pgfpathlineto{\pgfqpoint{3.228765in}{0.901072in}}%
\pgfpathlineto{\pgfqpoint{3.249607in}{0.899073in}}%
\pgfpathlineto{\pgfqpoint{3.270081in}{0.897124in}}%
\pgfpathlineto{\pgfqpoint{3.290198in}{0.895223in}}%
\pgfpathlineto{\pgfqpoint{3.309971in}{0.893369in}}%
\pgfpathlineto{\pgfqpoint{3.329412in}{0.891558in}}%
\pgfpathlineto{\pgfqpoint{3.348532in}{0.889790in}}%
\pgfpathlineto{\pgfqpoint{3.367341in}{0.888063in}}%
\pgfpathlineto{\pgfqpoint{3.385849in}{0.886375in}}%
\pgfpathlineto{\pgfqpoint{3.404066in}{0.884725in}}%
\pgfpathlineto{\pgfqpoint{3.422000in}{0.883111in}}%
\pgfpathlineto{\pgfqpoint{3.439661in}{0.881532in}}%
\pgfpathlineto{\pgfqpoint{3.457056in}{0.879987in}}%
\pgfpathlineto{\pgfqpoint{3.474193in}{0.878475in}}%
\pgfpathlineto{\pgfqpoint{3.491080in}{0.876994in}}%
\pgfpathlineto{\pgfqpoint{3.507724in}{0.875544in}}%
\pgfpathlineto{\pgfqpoint{3.524132in}{0.874123in}}%
\pgfpathlineto{\pgfqpoint{3.540311in}{0.872731in}}%
\pgfpathlineto{\pgfqpoint{3.556266in}{0.871366in}}%
\pgfpathlineto{\pgfqpoint{3.572005in}{0.870028in}}%
\pgfpathlineto{\pgfqpoint{3.587532in}{0.868715in}}%
\pgfpathlineto{\pgfqpoint{3.602854in}{0.867428in}}%
\pgfpathlineto{\pgfqpoint{3.617975in}{0.866164in}}%
\pgfpathlineto{\pgfqpoint{3.632901in}{0.864924in}}%
\pgfpathlineto{\pgfqpoint{3.647637in}{0.863707in}}%
\pgfpathlineto{\pgfqpoint{3.662188in}{0.862511in}}%
\pgfpathlineto{\pgfqpoint{3.676558in}{0.861337in}}%
\pgfpathlineto{\pgfqpoint{3.690752in}{0.860184in}}%
\pgfpathlineto{\pgfqpoint{3.704773in}{0.859051in}}%
\pgfpathlineto{\pgfqpoint{3.718627in}{0.857937in}}%
\pgfpathlineto{\pgfqpoint{3.732317in}{0.856842in}}%
\pgfpathlineto{\pgfqpoint{3.745847in}{0.855766in}}%
\pgfpathlineto{\pgfqpoint{3.759220in}{0.854708in}}%
\pgfpathlineto{\pgfqpoint{3.772441in}{0.853667in}}%
\pgfpathlineto{\pgfqpoint{3.785512in}{0.852643in}}%
\pgfpathlineto{\pgfqpoint{3.798437in}{0.851636in}}%
\pgfpathlineto{\pgfqpoint{3.811219in}{0.850645in}}%
\pgfpathlineto{\pgfqpoint{3.823862in}{0.849669in}}%
\pgfpathlineto{\pgfqpoint{3.836368in}{0.848709in}}%
\pgfpathlineto{\pgfqpoint{3.848740in}{0.847763in}}%
\pgfpathlineto{\pgfqpoint{3.860981in}{0.846832in}}%
\pgfpathlineto{\pgfqpoint{3.873095in}{0.845915in}}%
\pgfpathlineto{\pgfqpoint{3.885082in}{0.845012in}}%
\pgfpathlineto{\pgfqpoint{3.896947in}{0.844122in}}%
\pgfpathlineto{\pgfqpoint{3.908691in}{0.843246in}}%
\pgfpathlineto{\pgfqpoint{3.920317in}{0.842382in}}%
\pgfusepath{stroke}%
\end{pgfscope}%
\begin{pgfscope}%
\pgfpathrectangle{\pgfqpoint{0.466126in}{0.521603in}}{\pgfqpoint{3.720000in}{3.020000in}} %
\pgfusepath{clip}%
\pgfsetrectcap%
\pgfsetroundjoin%
\pgfsetlinewidth{1.505625pt}%
\definecolor{currentstroke}{rgb}{0.582353,0.991645,0.659925}%
\pgfsetstrokecolor{currentstroke}%
\pgfsetdash{}{0pt}%
\pgfpathmoveto{\pgfqpoint{0.456126in}{1.402729in}}%
\pgfpathlineto{\pgfqpoint{0.456740in}{1.402620in}}%
\pgfpathlineto{\pgfqpoint{0.667278in}{1.364742in}}%
\pgfpathlineto{\pgfqpoint{0.845337in}{1.332654in}}%
\pgfpathlineto{\pgfqpoint{0.999612in}{1.304970in}}%
\pgfpathlineto{\pgfqpoint{1.135713in}{1.280737in}}%
\pgfpathlineto{\pgfqpoint{1.257476in}{1.259278in}}%
\pgfpathlineto{\pgfqpoint{1.367635in}{1.240090in}}%
\pgfpathlineto{\pgfqpoint{1.468210in}{1.222792in}}%
\pgfpathlineto{\pgfqpoint{1.560738in}{1.207089in}}%
\pgfpathlineto{\pgfqpoint{1.646410in}{1.192746in}}%
\pgfpathlineto{\pgfqpoint{1.726173in}{1.179575in}}%
\pgfpathlineto{\pgfqpoint{1.800789in}{1.167424in}}%
\pgfpathlineto{\pgfqpoint{1.870884in}{1.156165in}}%
\pgfpathlineto{\pgfqpoint{1.936973in}{1.145695in}}%
\pgfpathlineto{\pgfqpoint{1.999490in}{1.135923in}}%
\pgfpathlineto{\pgfqpoint{2.058801in}{1.126776in}}%
\pgfpathlineto{\pgfqpoint{2.115219in}{1.118189in}}%
\pgfpathlineto{\pgfqpoint{2.169013in}{1.110107in}}%
\pgfpathlineto{\pgfqpoint{2.220417in}{1.102482in}}%
\pgfpathlineto{\pgfqpoint{2.269634in}{1.095272in}}%
\pgfpathlineto{\pgfqpoint{2.316842in}{1.088441in}}%
\pgfpathlineto{\pgfqpoint{2.362199in}{1.081956in}}%
\pgfpathlineto{\pgfqpoint{2.405844in}{1.075789in}}%
\pgfpathlineto{\pgfqpoint{2.447903in}{1.069916in}}%
\pgfpathlineto{\pgfqpoint{2.488486in}{1.064312in}}%
\pgfpathlineto{\pgfqpoint{2.527694in}{1.058959in}}%
\pgfpathlineto{\pgfqpoint{2.565617in}{1.053837in}}%
\pgfpathlineto{\pgfqpoint{2.602335in}{1.048931in}}%
\pgfpathlineto{\pgfqpoint{2.637925in}{1.044226in}}%
\pgfpathlineto{\pgfqpoint{2.672452in}{1.039709in}}%
\pgfpathlineto{\pgfqpoint{2.705978in}{1.035366in}}%
\pgfpathlineto{\pgfqpoint{2.738560in}{1.031188in}}%
\pgfpathlineto{\pgfqpoint{2.770249in}{1.027164in}}%
\pgfpathlineto{\pgfqpoint{2.801094in}{1.023285in}}%
\pgfpathlineto{\pgfqpoint{2.831138in}{1.019543in}}%
\pgfpathlineto{\pgfqpoint{2.860421in}{1.015929in}}%
\pgfpathlineto{\pgfqpoint{2.888981in}{1.012436in}}%
\pgfpathlineto{\pgfqpoint{2.916853in}{1.009058in}}%
\pgfpathlineto{\pgfqpoint{2.944069in}{1.005788in}}%
\pgfpathlineto{\pgfqpoint{2.970660in}{1.002621in}}%
\pgfpathlineto{\pgfqpoint{2.996653in}{0.999552in}}%
\pgfpathlineto{\pgfqpoint{3.022075in}{0.996575in}}%
\pgfpathlineto{\pgfqpoint{3.046951in}{0.993687in}}%
\pgfpathlineto{\pgfqpoint{3.071303in}{0.990882in}}%
\pgfpathlineto{\pgfqpoint{3.095152in}{0.988157in}}%
\pgfpathlineto{\pgfqpoint{3.118521in}{0.985508in}}%
\pgfpathlineto{\pgfqpoint{3.141426in}{0.982932in}}%
\pgfpathlineto{\pgfqpoint{3.163887in}{0.980425in}}%
\pgfpathlineto{\pgfqpoint{3.185920in}{0.977984in}}%
\pgfpathlineto{\pgfqpoint{3.207541in}{0.975607in}}%
\pgfpathlineto{\pgfqpoint{3.228765in}{0.973291in}}%
\pgfpathlineto{\pgfqpoint{3.249607in}{0.971032in}}%
\pgfpathlineto{\pgfqpoint{3.270081in}{0.968830in}}%
\pgfpathlineto{\pgfqpoint{3.290198in}{0.966681in}}%
\pgfpathlineto{\pgfqpoint{3.309971in}{0.964583in}}%
\pgfpathlineto{\pgfqpoint{3.329412in}{0.962535in}}%
\pgfpathlineto{\pgfqpoint{3.348532in}{0.960534in}}%
\pgfpathlineto{\pgfqpoint{3.367341in}{0.958579in}}%
\pgfpathlineto{\pgfqpoint{3.385849in}{0.956667in}}%
\pgfpathlineto{\pgfqpoint{3.404066in}{0.954798in}}%
\pgfpathlineto{\pgfqpoint{3.422000in}{0.952970in}}%
\pgfpathlineto{\pgfqpoint{3.439661in}{0.951181in}}%
\pgfpathlineto{\pgfqpoint{3.457056in}{0.949430in}}%
\pgfpathlineto{\pgfqpoint{3.474193in}{0.947716in}}%
\pgfpathlineto{\pgfqpoint{3.491080in}{0.946037in}}%
\pgfpathlineto{\pgfqpoint{3.507724in}{0.944392in}}%
\pgfpathlineto{\pgfqpoint{3.524132in}{0.942780in}}%
\pgfpathlineto{\pgfqpoint{3.540311in}{0.941200in}}%
\pgfpathlineto{\pgfqpoint{3.556266in}{0.939651in}}%
\pgfpathlineto{\pgfqpoint{3.572005in}{0.938132in}}%
\pgfpathlineto{\pgfqpoint{3.587532in}{0.936641in}}%
\pgfpathlineto{\pgfqpoint{3.602854in}{0.935179in}}%
\pgfpathlineto{\pgfqpoint{3.617975in}{0.933744in}}%
\pgfpathlineto{\pgfqpoint{3.632901in}{0.932335in}}%
\pgfpathlineto{\pgfqpoint{3.647637in}{0.930952in}}%
\pgfpathlineto{\pgfqpoint{3.662188in}{0.929593in}}%
\pgfpathlineto{\pgfqpoint{3.676558in}{0.928259in}}%
\pgfpathlineto{\pgfqpoint{3.690752in}{0.926948in}}%
\pgfpathlineto{\pgfqpoint{3.704773in}{0.925659in}}%
\pgfpathlineto{\pgfqpoint{3.718627in}{0.924393in}}%
\pgfpathlineto{\pgfqpoint{3.732317in}{0.923147in}}%
\pgfpathlineto{\pgfqpoint{3.745847in}{0.921923in}}%
\pgfpathlineto{\pgfqpoint{3.759220in}{0.920719in}}%
\pgfpathlineto{\pgfqpoint{3.772441in}{0.919535in}}%
\pgfpathlineto{\pgfqpoint{3.785512in}{0.918370in}}%
\pgfpathlineto{\pgfqpoint{3.798437in}{0.917223in}}%
\pgfpathlineto{\pgfqpoint{3.811219in}{0.916094in}}%
\pgfpathlineto{\pgfqpoint{3.823862in}{0.914984in}}%
\pgfpathlineto{\pgfqpoint{3.836368in}{0.913890in}}%
\pgfpathlineto{\pgfqpoint{3.848740in}{0.912813in}}%
\pgfpathlineto{\pgfqpoint{3.860981in}{0.911753in}}%
\pgfpathlineto{\pgfqpoint{3.873095in}{0.910708in}}%
\pgfpathlineto{\pgfqpoint{3.885082in}{0.909679in}}%
\pgfpathlineto{\pgfqpoint{3.896947in}{0.908666in}}%
\pgfpathlineto{\pgfqpoint{3.908691in}{0.907667in}}%
\pgfpathlineto{\pgfqpoint{3.920317in}{0.906682in}}%
\pgfusepath{stroke}%
\end{pgfscope}%
\begin{pgfscope}%
\pgfpathrectangle{\pgfqpoint{0.466126in}{0.521603in}}{\pgfqpoint{3.720000in}{3.020000in}} %
\pgfusepath{clip}%
\pgfsetrectcap%
\pgfsetroundjoin%
\pgfsetlinewidth{1.505625pt}%
\definecolor{currentstroke}{rgb}{0.990196,0.717912,0.389786}%
\pgfsetstrokecolor{currentstroke}%
\pgfsetdash{}{0pt}%
\pgfpathmoveto{\pgfqpoint{0.456126in}{1.953597in}}%
\pgfpathlineto{\pgfqpoint{0.456740in}{1.953452in}}%
\pgfpathlineto{\pgfqpoint{0.667278in}{1.903913in}}%
\pgfpathlineto{\pgfqpoint{0.845337in}{1.862388in}}%
\pgfpathlineto{\pgfqpoint{0.999612in}{1.826879in}}%
\pgfpathlineto{\pgfqpoint{1.135713in}{1.796035in}}%
\pgfpathlineto{\pgfqpoint{1.257476in}{1.768901in}}%
\pgfpathlineto{\pgfqpoint{1.367635in}{1.744779in}}%
\pgfpathlineto{\pgfqpoint{1.468210in}{1.723143in}}%
\pgfpathlineto{\pgfqpoint{1.560738in}{1.703591in}}%
\pgfpathlineto{\pgfqpoint{1.646410in}{1.685805in}}%
\pgfpathlineto{\pgfqpoint{1.726173in}{1.669532in}}%
\pgfpathlineto{\pgfqpoint{1.800789in}{1.654567in}}%
\pgfpathlineto{\pgfqpoint{1.870884in}{1.640744in}}%
\pgfpathlineto{\pgfqpoint{1.936973in}{1.627924in}}%
\pgfpathlineto{\pgfqpoint{1.999490in}{1.615989in}}%
\pgfpathlineto{\pgfqpoint{2.058801in}{1.604843in}}%
\pgfpathlineto{\pgfqpoint{2.115219in}{1.594402in}}%
\pgfpathlineto{\pgfqpoint{2.169013in}{1.584594in}}%
\pgfpathlineto{\pgfqpoint{2.220417in}{1.575357in}}%
\pgfpathlineto{\pgfqpoint{2.269634in}{1.566638in}}%
\pgfpathlineto{\pgfqpoint{2.316842in}{1.558390in}}%
\pgfpathlineto{\pgfqpoint{2.362199in}{1.550573in}}%
\pgfpathlineto{\pgfqpoint{2.405844in}{1.543149in}}%
\pgfpathlineto{\pgfqpoint{2.447903in}{1.536087in}}%
\pgfpathlineto{\pgfqpoint{2.488486in}{1.529358in}}%
\pgfpathlineto{\pgfqpoint{2.527694in}{1.522937in}}%
\pgfpathlineto{\pgfqpoint{2.565617in}{1.516801in}}%
\pgfpathlineto{\pgfqpoint{2.602335in}{1.510929in}}%
\pgfpathlineto{\pgfqpoint{2.637925in}{1.505303in}}%
\pgfpathlineto{\pgfqpoint{2.672452in}{1.499907in}}%
\pgfpathlineto{\pgfqpoint{2.705978in}{1.494724in}}%
\pgfpathlineto{\pgfqpoint{2.738560in}{1.489742in}}%
\pgfpathlineto{\pgfqpoint{2.770249in}{1.484947in}}%
\pgfpathlineto{\pgfqpoint{2.801094in}{1.480329in}}%
\pgfpathlineto{\pgfqpoint{2.831138in}{1.475876in}}%
\pgfpathlineto{\pgfqpoint{2.860421in}{1.471579in}}%
\pgfpathlineto{\pgfqpoint{2.888981in}{1.467429in}}%
\pgfpathlineto{\pgfqpoint{2.916853in}{1.463418in}}%
\pgfpathlineto{\pgfqpoint{2.944069in}{1.459538in}}%
\pgfpathlineto{\pgfqpoint{2.970660in}{1.455782in}}%
\pgfpathlineto{\pgfqpoint{2.996653in}{1.452144in}}%
\pgfpathlineto{\pgfqpoint{3.022075in}{1.448618in}}%
\pgfpathlineto{\pgfqpoint{3.046951in}{1.445198in}}%
\pgfpathlineto{\pgfqpoint{3.071303in}{1.441879in}}%
\pgfpathlineto{\pgfqpoint{3.095152in}{1.438655in}}%
\pgfpathlineto{\pgfqpoint{3.118521in}{1.435524in}}%
\pgfpathlineto{\pgfqpoint{3.141426in}{1.432479in}}%
\pgfpathlineto{\pgfqpoint{3.163887in}{1.429518in}}%
\pgfpathlineto{\pgfqpoint{3.185920in}{1.426636in}}%
\pgfpathlineto{\pgfqpoint{3.207541in}{1.423830in}}%
\pgfpathlineto{\pgfqpoint{3.228765in}{1.421097in}}%
\pgfpathlineto{\pgfqpoint{3.249607in}{1.418434in}}%
\pgfpathlineto{\pgfqpoint{3.270081in}{1.415837in}}%
\pgfpathlineto{\pgfqpoint{3.290198in}{1.413304in}}%
\pgfpathlineto{\pgfqpoint{3.309971in}{1.410833in}}%
\pgfpathlineto{\pgfqpoint{3.329412in}{1.408420in}}%
\pgfpathlineto{\pgfqpoint{3.348532in}{1.406064in}}%
\pgfpathlineto{\pgfqpoint{3.367341in}{1.403763in}}%
\pgfpathlineto{\pgfqpoint{3.385849in}{1.401514in}}%
\pgfpathlineto{\pgfqpoint{3.404066in}{1.399316in}}%
\pgfpathlineto{\pgfqpoint{3.422000in}{1.397166in}}%
\pgfpathlineto{\pgfqpoint{3.439661in}{1.395063in}}%
\pgfpathlineto{\pgfqpoint{3.457056in}{1.393005in}}%
\pgfpathlineto{\pgfqpoint{3.474193in}{1.390991in}}%
\pgfpathlineto{\pgfqpoint{3.491080in}{1.389018in}}%
\pgfpathlineto{\pgfqpoint{3.507724in}{1.387086in}}%
\pgfpathlineto{\pgfqpoint{3.524132in}{1.385194in}}%
\pgfpathlineto{\pgfqpoint{3.540311in}{1.383339in}}%
\pgfpathlineto{\pgfqpoint{3.556266in}{1.381521in}}%
\pgfpathlineto{\pgfqpoint{3.572005in}{1.379739in}}%
\pgfpathlineto{\pgfqpoint{3.587532in}{1.377990in}}%
\pgfpathlineto{\pgfqpoint{3.602854in}{1.376275in}}%
\pgfpathlineto{\pgfqpoint{3.617975in}{1.374593in}}%
\pgfpathlineto{\pgfqpoint{3.632901in}{1.372941in}}%
\pgfpathlineto{\pgfqpoint{3.647637in}{1.371320in}}%
\pgfpathlineto{\pgfqpoint{3.662188in}{1.369728in}}%
\pgfpathlineto{\pgfqpoint{3.676558in}{1.368164in}}%
\pgfpathlineto{\pgfqpoint{3.690752in}{1.366628in}}%
\pgfpathlineto{\pgfqpoint{3.704773in}{1.365119in}}%
\pgfpathlineto{\pgfqpoint{3.718627in}{1.363636in}}%
\pgfpathlineto{\pgfqpoint{3.732317in}{1.362178in}}%
\pgfpathlineto{\pgfqpoint{3.745847in}{1.360745in}}%
\pgfpathlineto{\pgfqpoint{3.759220in}{1.359336in}}%
\pgfpathlineto{\pgfqpoint{3.772441in}{1.357950in}}%
\pgfpathlineto{\pgfqpoint{3.785512in}{1.356587in}}%
\pgfpathlineto{\pgfqpoint{3.798437in}{1.355245in}}%
\pgfpathlineto{\pgfqpoint{3.811219in}{1.353925in}}%
\pgfpathlineto{\pgfqpoint{3.823862in}{1.352626in}}%
\pgfpathlineto{\pgfqpoint{3.836368in}{1.351348in}}%
\pgfpathlineto{\pgfqpoint{3.848740in}{1.350089in}}%
\pgfpathlineto{\pgfqpoint{3.860981in}{1.348849in}}%
\pgfpathlineto{\pgfqpoint{3.873095in}{1.347628in}}%
\pgfpathlineto{\pgfqpoint{3.885082in}{1.346426in}}%
\pgfpathlineto{\pgfqpoint{3.896947in}{1.345241in}}%
\pgfpathlineto{\pgfqpoint{3.908691in}{1.344074in}}%
\pgfpathlineto{\pgfqpoint{3.920317in}{1.342924in}}%
\pgfusepath{stroke}%
\end{pgfscope}%
\begin{pgfscope}%
\pgfpathrectangle{\pgfqpoint{0.466126in}{0.521603in}}{\pgfqpoint{3.720000in}{3.020000in}} %
\pgfusepath{clip}%
\pgfsetrectcap%
\pgfsetroundjoin%
\pgfsetlinewidth{1.505625pt}%
\definecolor{currentstroke}{rgb}{1.000000,0.462204,0.237935}%
\pgfsetstrokecolor{currentstroke}%
\pgfsetdash{}{0pt}%
\pgfpathmoveto{\pgfqpoint{0.456126in}{2.807218in}}%
\pgfpathlineto{\pgfqpoint{0.456740in}{2.807019in}}%
\pgfpathlineto{\pgfqpoint{0.667278in}{2.740540in}}%
\pgfpathlineto{\pgfqpoint{0.845337in}{2.685938in}}%
\pgfpathlineto{\pgfqpoint{0.999612in}{2.640020in}}%
\pgfpathlineto{\pgfqpoint{1.135713in}{2.600687in}}%
\pgfpathlineto{\pgfqpoint{1.257476in}{2.566493in}}%
\pgfpathlineto{\pgfqpoint{1.367635in}{2.536403in}}%
\pgfpathlineto{\pgfqpoint{1.468210in}{2.509652in}}%
\pgfpathlineto{\pgfqpoint{1.560738in}{2.485664in}}%
\pgfpathlineto{\pgfqpoint{1.646410in}{2.463992in}}%
\pgfpathlineto{\pgfqpoint{1.726173in}{2.444285in}}%
\pgfpathlineto{\pgfqpoint{1.800789in}{2.426261in}}%
\pgfpathlineto{\pgfqpoint{1.870884in}{2.409694in}}%
\pgfpathlineto{\pgfqpoint{1.936973in}{2.394397in}}%
\pgfpathlineto{\pgfqpoint{1.999490in}{2.380215in}}%
\pgfpathlineto{\pgfqpoint{2.058801in}{2.367019in}}%
\pgfpathlineto{\pgfqpoint{2.115219in}{2.354699in}}%
\pgfpathlineto{\pgfqpoint{2.169013in}{2.343163in}}%
\pgfpathlineto{\pgfqpoint{2.220417in}{2.332331in}}%
\pgfpathlineto{\pgfqpoint{2.269634in}{2.322133in}}%
\pgfpathlineto{\pgfqpoint{2.316842in}{2.312510in}}%
\pgfpathlineto{\pgfqpoint{2.362199in}{2.303411in}}%
\pgfpathlineto{\pgfqpoint{2.405844in}{2.294788in}}%
\pgfpathlineto{\pgfqpoint{2.447903in}{2.286603in}}%
\pgfpathlineto{\pgfqpoint{2.488486in}{2.278818in}}%
\pgfpathlineto{\pgfqpoint{2.527694in}{2.271403in}}%
\pgfpathlineto{\pgfqpoint{2.565617in}{2.264329in}}%
\pgfpathlineto{\pgfqpoint{2.602335in}{2.257571in}}%
\pgfpathlineto{\pgfqpoint{2.637925in}{2.251105in}}%
\pgfpathlineto{\pgfqpoint{2.672452in}{2.244912in}}%
\pgfpathlineto{\pgfqpoint{2.705978in}{2.238972in}}%
\pgfpathlineto{\pgfqpoint{2.738560in}{2.233270in}}%
\pgfpathlineto{\pgfqpoint{2.770249in}{2.227788in}}%
\pgfpathlineto{\pgfqpoint{2.801094in}{2.222515in}}%
\pgfpathlineto{\pgfqpoint{2.831138in}{2.217436in}}%
\pgfpathlineto{\pgfqpoint{2.860421in}{2.212539in}}%
\pgfpathlineto{\pgfqpoint{2.888981in}{2.207816in}}%
\pgfpathlineto{\pgfqpoint{2.916853in}{2.203254in}}%
\pgfpathlineto{\pgfqpoint{2.944069in}{2.198846in}}%
\pgfpathlineto{\pgfqpoint{2.970660in}{2.194583in}}%
\pgfpathlineto{\pgfqpoint{2.996653in}{2.190457in}}%
\pgfpathlineto{\pgfqpoint{3.022075in}{2.186461in}}%
\pgfpathlineto{\pgfqpoint{3.046951in}{2.182588in}}%
\pgfpathlineto{\pgfqpoint{3.071303in}{2.178832in}}%
\pgfpathlineto{\pgfqpoint{3.095152in}{2.175188in}}%
\pgfpathlineto{\pgfqpoint{3.118521in}{2.171649in}}%
\pgfpathlineto{\pgfqpoint{3.141426in}{2.168211in}}%
\pgfpathlineto{\pgfqpoint{3.163887in}{2.164869in}}%
\pgfpathlineto{\pgfqpoint{3.185920in}{2.161619in}}%
\pgfpathlineto{\pgfqpoint{3.207541in}{2.158457in}}%
\pgfpathlineto{\pgfqpoint{3.228765in}{2.155379in}}%
\pgfpathlineto{\pgfqpoint{3.249607in}{2.152380in}}%
\pgfpathlineto{\pgfqpoint{3.270081in}{2.149458in}}%
\pgfpathlineto{\pgfqpoint{3.290198in}{2.146610in}}%
\pgfpathlineto{\pgfqpoint{3.309971in}{2.143832in}}%
\pgfpathlineto{\pgfqpoint{3.329412in}{2.141122in}}%
\pgfpathlineto{\pgfqpoint{3.348532in}{2.138476in}}%
\pgfpathlineto{\pgfqpoint{3.367341in}{2.135893in}}%
\pgfpathlineto{\pgfqpoint{3.385849in}{2.133370in}}%
\pgfpathlineto{\pgfqpoint{3.404066in}{2.130905in}}%
\pgfpathlineto{\pgfqpoint{3.422000in}{2.128495in}}%
\pgfpathlineto{\pgfqpoint{3.439661in}{2.126138in}}%
\pgfpathlineto{\pgfqpoint{3.457056in}{2.123833in}}%
\pgfpathlineto{\pgfqpoint{3.474193in}{2.121578in}}%
\pgfpathlineto{\pgfqpoint{3.491080in}{2.119371in}}%
\pgfpathlineto{\pgfqpoint{3.507724in}{2.117209in}}%
\pgfpathlineto{\pgfqpoint{3.524132in}{2.115093in}}%
\pgfpathlineto{\pgfqpoint{3.540311in}{2.113019in}}%
\pgfpathlineto{\pgfqpoint{3.556266in}{2.110987in}}%
\pgfpathlineto{\pgfqpoint{3.572005in}{2.108996in}}%
\pgfpathlineto{\pgfqpoint{3.587532in}{2.107043in}}%
\pgfpathlineto{\pgfqpoint{3.602854in}{2.105128in}}%
\pgfpathlineto{\pgfqpoint{3.617975in}{2.103250in}}%
\pgfpathlineto{\pgfqpoint{3.632901in}{2.101407in}}%
\pgfpathlineto{\pgfqpoint{3.647637in}{2.099598in}}%
\pgfpathlineto{\pgfqpoint{3.662188in}{2.097823in}}%
\pgfpathlineto{\pgfqpoint{3.676558in}{2.096079in}}%
\pgfpathlineto{\pgfqpoint{3.690752in}{2.094367in}}%
\pgfpathlineto{\pgfqpoint{3.704773in}{2.092686in}}%
\pgfpathlineto{\pgfqpoint{3.718627in}{2.091033in}}%
\pgfpathlineto{\pgfqpoint{3.732317in}{2.089410in}}%
\pgfpathlineto{\pgfqpoint{3.745847in}{2.087814in}}%
\pgfpathlineto{\pgfqpoint{3.759220in}{2.086245in}}%
\pgfpathlineto{\pgfqpoint{3.772441in}{2.084702in}}%
\pgfpathlineto{\pgfqpoint{3.785512in}{2.083185in}}%
\pgfpathlineto{\pgfqpoint{3.798437in}{2.081693in}}%
\pgfpathlineto{\pgfqpoint{3.811219in}{2.080225in}}%
\pgfpathlineto{\pgfqpoint{3.823862in}{2.078781in}}%
\pgfpathlineto{\pgfqpoint{3.836368in}{2.077359in}}%
\pgfpathlineto{\pgfqpoint{3.848740in}{2.075959in}}%
\pgfpathlineto{\pgfqpoint{3.860981in}{2.074582in}}%
\pgfpathlineto{\pgfqpoint{3.873095in}{2.073225in}}%
\pgfpathlineto{\pgfqpoint{3.885082in}{2.071889in}}%
\pgfpathlineto{\pgfqpoint{3.896947in}{2.070574in}}%
\pgfpathlineto{\pgfqpoint{3.908691in}{2.069277in}}%
\pgfpathlineto{\pgfqpoint{3.920317in}{2.068000in}}%
\pgfusepath{stroke}%
\end{pgfscope}%
\begin{pgfscope}%
\pgfsetrectcap%
\pgfsetmiterjoin%
\pgfsetlinewidth{0.803000pt}%
\definecolor{currentstroke}{rgb}{0.000000,0.000000,0.000000}%
\pgfsetstrokecolor{currentstroke}%
\pgfsetdash{}{0pt}%
\pgfpathmoveto{\pgfqpoint{0.466126in}{0.521603in}}%
\pgfpathlineto{\pgfqpoint{0.466126in}{3.541603in}}%
\pgfusepath{stroke}%
\end{pgfscope}%
\begin{pgfscope}%
\pgfsetrectcap%
\pgfsetmiterjoin%
\pgfsetlinewidth{0.803000pt}%
\definecolor{currentstroke}{rgb}{0.000000,0.000000,0.000000}%
\pgfsetstrokecolor{currentstroke}%
\pgfsetdash{}{0pt}%
\pgfpathmoveto{\pgfqpoint{4.186126in}{0.521603in}}%
\pgfpathlineto{\pgfqpoint{4.186126in}{3.541603in}}%
\pgfusepath{stroke}%
\end{pgfscope}%
\begin{pgfscope}%
\pgfsetrectcap%
\pgfsetmiterjoin%
\pgfsetlinewidth{0.803000pt}%
\definecolor{currentstroke}{rgb}{0.000000,0.000000,0.000000}%
\pgfsetstrokecolor{currentstroke}%
\pgfsetdash{}{0pt}%
\pgfpathmoveto{\pgfqpoint{0.466126in}{0.521603in}}%
\pgfpathlineto{\pgfqpoint{4.186126in}{0.521603in}}%
\pgfusepath{stroke}%
\end{pgfscope}%
\begin{pgfscope}%
\pgfsetrectcap%
\pgfsetmiterjoin%
\pgfsetlinewidth{0.803000pt}%
\definecolor{currentstroke}{rgb}{0.000000,0.000000,0.000000}%
\pgfsetstrokecolor{currentstroke}%
\pgfsetdash{}{0pt}%
\pgfpathmoveto{\pgfqpoint{0.466126in}{3.541603in}}%
\pgfpathlineto{\pgfqpoint{4.186126in}{3.541603in}}%
\pgfusepath{stroke}%
\end{pgfscope}%
\begin{pgfscope}%
\pgfsetbuttcap%
\pgfsetmiterjoin%
\definecolor{currentfill}{rgb}{1.000000,1.000000,1.000000}%
\pgfsetfillcolor{currentfill}%
\pgfsetfillopacity{0.800000}%
\pgfsetlinewidth{1.003750pt}%
\definecolor{currentstroke}{rgb}{0.800000,0.800000,0.800000}%
\pgfsetstrokecolor{currentstroke}%
\pgfsetstrokeopacity{0.800000}%
\pgfsetdash{}{0pt}%
\pgfpathmoveto{\pgfqpoint{0.563349in}{2.003492in}}%
\pgfpathlineto{\pgfqpoint{3.101215in}{2.003492in}}%
\pgfpathquadraticcurveto{\pgfqpoint{3.128993in}{2.003492in}}{\pgfqpoint{3.128993in}{2.031269in}}%
\pgfpathlineto{\pgfqpoint{3.128993in}{3.444381in}}%
\pgfpathquadraticcurveto{\pgfqpoint{3.128993in}{3.472159in}}{\pgfqpoint{3.101215in}{3.472159in}}%
\pgfpathlineto{\pgfqpoint{0.563349in}{3.472159in}}%
\pgfpathquadraticcurveto{\pgfqpoint{0.535571in}{3.472159in}}{\pgfqpoint{0.535571in}{3.444381in}}%
\pgfpathlineto{\pgfqpoint{0.535571in}{2.031269in}}%
\pgfpathquadraticcurveto{\pgfqpoint{0.535571in}{2.003492in}}{\pgfqpoint{0.563349in}{2.003492in}}%
\pgfpathclose%
\pgfusepath{stroke,fill}%
\end{pgfscope}%
\begin{pgfscope}%
\pgfsetbuttcap%
\pgfsetroundjoin%
\pgfsetlinewidth{1.505625pt}%
\definecolor{currentstroke}{rgb}{0.000000,0.000000,0.000000}%
\pgfsetstrokecolor{currentstroke}%
\pgfsetdash{{5.550000pt}{2.400000pt}}{0.000000pt}%
\pgfpathmoveto{\pgfqpoint{0.591126in}{3.359691in}}%
\pgfpathlineto{\pgfqpoint{0.868904in}{3.359691in}}%
\pgfusepath{stroke}%
\end{pgfscope}%
\begin{pgfscope}%
\pgftext[x=0.980015in,y=3.311080in,left,base]{\rmfamily\fontsize{10.000000}{12.000000}\selectfont Richards, 2001}%
\end{pgfscope}%
\begin{pgfscope}%
\pgfsetbuttcap%
\pgfsetroundjoin%
\pgfsetlinewidth{1.505625pt}%
\definecolor{currentstroke}{rgb}{0.501961,0.501961,0.501961}%
\pgfsetstrokecolor{currentstroke}%
\pgfsetdash{{9.600000pt}{2.400000pt}{1.500000pt}{2.400000pt}}{0.000000pt}%
\pgfpathmoveto{\pgfqpoint{0.591126in}{3.155834in}}%
\pgfpathlineto{\pgfqpoint{0.868904in}{3.155834in}}%
\pgfusepath{stroke}%
\end{pgfscope}%
\begin{pgfscope}%
\pgftext[x=0.980015in,y=3.107223in,left,base]{\rmfamily\fontsize{10.000000}{12.000000}\selectfont Gopinath et al., 2002}%
\end{pgfscope}%
\begin{pgfscope}%
\pgfsetrectcap%
\pgfsetroundjoin%
\pgfsetlinewidth{1.505625pt}%
\definecolor{currentstroke}{rgb}{0.174510,0.872120,0.862929}%
\pgfsetstrokecolor{currentstroke}%
\pgfsetdash{}{0pt}%
\pgfpathmoveto{\pgfqpoint{0.591126in}{2.951977in}}%
\pgfpathlineto{\pgfqpoint{0.868904in}{2.951977in}}%
\pgfusepath{stroke}%
\end{pgfscope}%
\begin{pgfscope}%
\pgftext[x=0.980015in,y=2.903366in,left,base]{\rmfamily\fontsize{10.000000}{12.000000}\selectfont Molacek et al., 2012, \(\displaystyle \mathbf{B}\mbox{o} = \)0.2}%
\end{pgfscope}%
\begin{pgfscope}%
\pgfsetrectcap%
\pgfsetroundjoin%
\pgfsetlinewidth{1.505625pt}%
\definecolor{currentstroke}{rgb}{0.174510,0.872120,0.862929}%
\pgfsetstrokecolor{currentstroke}%
\pgfsetdash{}{0pt}%
\pgfpathmoveto{\pgfqpoint{0.591126in}{2.748120in}}%
\pgfpathlineto{\pgfqpoint{0.868904in}{2.748120in}}%
\pgfusepath{stroke}%
\end{pgfscope}%
\begin{pgfscope}%
\pgftext[x=0.980015in,y=2.699509in,left,base]{\rmfamily\fontsize{10.000000}{12.000000}\selectfont Molacek et al., 2012, \(\displaystyle \mathbf{B}\mbox{o} = \)0.4}%
\end{pgfscope}%
\begin{pgfscope}%
\pgfsetrectcap%
\pgfsetroundjoin%
\pgfsetlinewidth{1.505625pt}%
\definecolor{currentstroke}{rgb}{0.582353,0.991645,0.659925}%
\pgfsetstrokecolor{currentstroke}%
\pgfsetdash{}{0pt}%
\pgfpathmoveto{\pgfqpoint{0.591126in}{2.544262in}}%
\pgfpathlineto{\pgfqpoint{0.868904in}{2.544262in}}%
\pgfusepath{stroke}%
\end{pgfscope}%
\begin{pgfscope}%
\pgftext[x=0.980015in,y=2.495651in,left,base]{\rmfamily\fontsize{10.000000}{12.000000}\selectfont Molacek et al., 2012, \(\displaystyle \mathbf{B}\mbox{o} = \)0.6}%
\end{pgfscope}%
\begin{pgfscope}%
\pgfsetrectcap%
\pgfsetroundjoin%
\pgfsetlinewidth{1.505625pt}%
\definecolor{currentstroke}{rgb}{0.990196,0.717912,0.389786}%
\pgfsetstrokecolor{currentstroke}%
\pgfsetdash{}{0pt}%
\pgfpathmoveto{\pgfqpoint{0.591126in}{2.340405in}}%
\pgfpathlineto{\pgfqpoint{0.868904in}{2.340405in}}%
\pgfusepath{stroke}%
\end{pgfscope}%
\begin{pgfscope}%
\pgftext[x=0.980015in,y=2.291794in,left,base]{\rmfamily\fontsize{10.000000}{12.000000}\selectfont Molacek et al., 2012, \(\displaystyle \mathbf{B}\mbox{o} = \)0.8}%
\end{pgfscope}%
\begin{pgfscope}%
\pgfsetrectcap%
\pgfsetroundjoin%
\pgfsetlinewidth{1.505625pt}%
\definecolor{currentstroke}{rgb}{1.000000,0.462204,0.237935}%
\pgfsetstrokecolor{currentstroke}%
\pgfsetdash{}{0pt}%
\pgfpathmoveto{\pgfqpoint{0.591126in}{2.136548in}}%
\pgfpathlineto{\pgfqpoint{0.868904in}{2.136548in}}%
\pgfusepath{stroke}%
\end{pgfscope}%
\begin{pgfscope}%
\pgftext[x=0.980015in,y=2.087937in,left,base]{\rmfamily\fontsize{10.000000}{12.000000}\selectfont Molacek et al., 2012, \(\displaystyle \mathbf{B}\mbox{o} = \)0.9}%
\end{pgfscope}%
\begin{pgfscope}%
\pgfpathrectangle{\pgfqpoint{4.418626in}{0.521603in}}{\pgfqpoint{0.151000in}{3.020000in}} %
\pgfusepath{clip}%
\pgfsetbuttcap%
\pgfsetmiterjoin%
\definecolor{currentfill}{rgb}{1.000000,1.000000,1.000000}%
\pgfsetfillcolor{currentfill}%
\pgfsetlinewidth{0.010037pt}%
\definecolor{currentstroke}{rgb}{1.000000,1.000000,1.000000}%
\pgfsetstrokecolor{currentstroke}%
\pgfsetdash{}{0pt}%
\pgfpathmoveto{\pgfqpoint{4.418626in}{0.521603in}}%
\pgfpathlineto{\pgfqpoint{4.418626in}{0.533400in}}%
\pgfpathlineto{\pgfqpoint{4.418626in}{3.529806in}}%
\pgfpathlineto{\pgfqpoint{4.418626in}{3.541603in}}%
\pgfpathlineto{\pgfqpoint{4.569626in}{3.541603in}}%
\pgfpathlineto{\pgfqpoint{4.569626in}{3.529806in}}%
\pgfpathlineto{\pgfqpoint{4.569626in}{0.533400in}}%
\pgfpathlineto{\pgfqpoint{4.569626in}{0.521603in}}%
\pgfpathclose%
\pgfusepath{stroke,fill}%
\end{pgfscope}%
\begin{pgfscope}%
\pgfsys@transformshift{4.420000in}{0.526603in}%
\pgftext[left,bottom]{\pgfimage[interpolate=true,width=0.150000in,height=3.020000in]{contact-img0.png}}%
\end{pgfscope}%
\begin{pgfscope}%
\pgfsetbuttcap%
\pgfsetroundjoin%
\definecolor{currentfill}{rgb}{0.000000,0.000000,0.000000}%
\pgfsetfillcolor{currentfill}%
\pgfsetlinewidth{0.803000pt}%
\definecolor{currentstroke}{rgb}{0.000000,0.000000,0.000000}%
\pgfsetstrokecolor{currentstroke}%
\pgfsetdash{}{0pt}%
\pgfsys@defobject{currentmarker}{\pgfqpoint{0.000000in}{0.000000in}}{\pgfqpoint{0.048611in}{0.000000in}}{%
\pgfpathmoveto{\pgfqpoint{0.000000in}{0.000000in}}%
\pgfpathlineto{\pgfqpoint{0.048611in}{0.000000in}}%
\pgfusepath{stroke,fill}%
}%
\begin{pgfscope}%
\pgfsys@transformshift{4.569626in}{0.925593in}%
\pgfsys@useobject{currentmarker}{}%
\end{pgfscope}%
\end{pgfscope}%
\begin{pgfscope}%
\pgftext[x=4.666849in,y=0.872831in,left,base]{\rmfamily\fontsize{10.000000}{12.000000}\selectfont \(\displaystyle 0.2\)}%
\end{pgfscope}%
\begin{pgfscope}%
\pgfsetbuttcap%
\pgfsetroundjoin%
\definecolor{currentfill}{rgb}{0.000000,0.000000,0.000000}%
\pgfsetfillcolor{currentfill}%
\pgfsetlinewidth{0.803000pt}%
\definecolor{currentstroke}{rgb}{0.000000,0.000000,0.000000}%
\pgfsetstrokecolor{currentstroke}%
\pgfsetdash{}{0pt}%
\pgfsys@defobject{currentmarker}{\pgfqpoint{0.000000in}{0.000000in}}{\pgfqpoint{0.048611in}{0.000000in}}{%
\pgfpathmoveto{\pgfqpoint{0.000000in}{0.000000in}}%
\pgfpathlineto{\pgfqpoint{0.048611in}{0.000000in}}%
\pgfusepath{stroke,fill}%
}%
\begin{pgfscope}%
\pgfsys@transformshift{4.569626in}{1.540182in}%
\pgfsys@useobject{currentmarker}{}%
\end{pgfscope}%
\end{pgfscope}%
\begin{pgfscope}%
\pgftext[x=4.666849in,y=1.487421in,left,base]{\rmfamily\fontsize{10.000000}{12.000000}\selectfont \(\displaystyle 0.4\)}%
\end{pgfscope}%
\begin{pgfscope}%
\pgfsetbuttcap%
\pgfsetroundjoin%
\definecolor{currentfill}{rgb}{0.000000,0.000000,0.000000}%
\pgfsetfillcolor{currentfill}%
\pgfsetlinewidth{0.803000pt}%
\definecolor{currentstroke}{rgb}{0.000000,0.000000,0.000000}%
\pgfsetstrokecolor{currentstroke}%
\pgfsetdash{}{0pt}%
\pgfsys@defobject{currentmarker}{\pgfqpoint{0.000000in}{0.000000in}}{\pgfqpoint{0.048611in}{0.000000in}}{%
\pgfpathmoveto{\pgfqpoint{0.000000in}{0.000000in}}%
\pgfpathlineto{\pgfqpoint{0.048611in}{0.000000in}}%
\pgfusepath{stroke,fill}%
}%
\begin{pgfscope}%
\pgfsys@transformshift{4.569626in}{2.154772in}%
\pgfsys@useobject{currentmarker}{}%
\end{pgfscope}%
\end{pgfscope}%
\begin{pgfscope}%
\pgftext[x=4.666849in,y=2.102010in,left,base]{\rmfamily\fontsize{10.000000}{12.000000}\selectfont \(\displaystyle 0.6\)}%
\end{pgfscope}%
\begin{pgfscope}%
\pgfsetbuttcap%
\pgfsetroundjoin%
\definecolor{currentfill}{rgb}{0.000000,0.000000,0.000000}%
\pgfsetfillcolor{currentfill}%
\pgfsetlinewidth{0.803000pt}%
\definecolor{currentstroke}{rgb}{0.000000,0.000000,0.000000}%
\pgfsetstrokecolor{currentstroke}%
\pgfsetdash{}{0pt}%
\pgfsys@defobject{currentmarker}{\pgfqpoint{0.000000in}{0.000000in}}{\pgfqpoint{0.048611in}{0.000000in}}{%
\pgfpathmoveto{\pgfqpoint{0.000000in}{0.000000in}}%
\pgfpathlineto{\pgfqpoint{0.048611in}{0.000000in}}%
\pgfusepath{stroke,fill}%
}%
\begin{pgfscope}%
\pgfsys@transformshift{4.569626in}{2.769361in}%
\pgfsys@useobject{currentmarker}{}%
\end{pgfscope}%
\end{pgfscope}%
\begin{pgfscope}%
\pgftext[x=4.666849in,y=2.716599in,left,base]{\rmfamily\fontsize{10.000000}{12.000000}\selectfont \(\displaystyle 0.8\)}%
\end{pgfscope}%
\begin{pgfscope}%
\pgfsetbuttcap%
\pgfsetroundjoin%
\definecolor{currentfill}{rgb}{0.000000,0.000000,0.000000}%
\pgfsetfillcolor{currentfill}%
\pgfsetlinewidth{0.803000pt}%
\definecolor{currentstroke}{rgb}{0.000000,0.000000,0.000000}%
\pgfsetstrokecolor{currentstroke}%
\pgfsetdash{}{0pt}%
\pgfsys@defobject{currentmarker}{\pgfqpoint{0.000000in}{0.000000in}}{\pgfqpoint{0.048611in}{0.000000in}}{%
\pgfpathmoveto{\pgfqpoint{0.000000in}{0.000000in}}%
\pgfpathlineto{\pgfqpoint{0.048611in}{0.000000in}}%
\pgfusepath{stroke,fill}%
}%
\begin{pgfscope}%
\pgfsys@transformshift{4.569626in}{3.383950in}%
\pgfsys@useobject{currentmarker}{}%
\end{pgfscope}%
\end{pgfscope}%
\begin{pgfscope}%
\pgftext[x=4.666849in,y=3.331189in,left,base]{\rmfamily\fontsize{10.000000}{12.000000}\selectfont \(\displaystyle 1.0\)}%
\end{pgfscope}%
\begin{pgfscope}%
\pgftext[x=4.899874in,y=2.031603in,,top,rotate=90.000000]{\rmfamily\fontsize{10.000000}{12.000000}\selectfont \(\displaystyle \mathrm{\mathit{Bo_e}} \equiv \frac{\epsilon E_0^2 R_0}{\gamma}\)}%
\end{pgfscope}%
\begin{pgfscope}%
\pgfsetbuttcap%
\pgfsetmiterjoin%
\pgfsetlinewidth{0.803000pt}%
\definecolor{currentstroke}{rgb}{0.000000,0.000000,0.000000}%
\pgfsetstrokecolor{currentstroke}%
\pgfsetdash{}{0pt}%
\pgfpathmoveto{\pgfqpoint{4.418626in}{0.521603in}}%
\pgfpathlineto{\pgfqpoint{4.418626in}{0.533400in}}%
\pgfpathlineto{\pgfqpoint{4.418626in}{3.529806in}}%
\pgfpathlineto{\pgfqpoint{4.418626in}{3.541603in}}%
\pgfpathlineto{\pgfqpoint{4.569626in}{3.541603in}}%
\pgfpathlineto{\pgfqpoint{4.569626in}{3.529806in}}%
\pgfpathlineto{\pgfqpoint{4.569626in}{0.533400in}}%
\pgfpathlineto{\pgfqpoint{4.569626in}{0.521603in}}%
\pgfpathclose%
\pgfusepath{stroke}%
\end{pgfscope}%
\end{pgfpicture}%
\makeatother%
\endgroup%

    \caption{A simple EMA plot.\label{fig:contact}}
\end{figure}

\begin{figure}[htb]
    \centering
    %% Creator: Matplotlib, PGF backend
%%
%% To include the figure in your LaTeX document, write
%%   \input{<filename>.pgf}
%%
%% Make sure the required packages are loaded in your preamble
%%   \usepackage{pgf}
%%
%% Figures using additional raster images can only be included by \input if
%% they are in the same directory as the main LaTeX file. For loading figures
%% from other directories you can use the `import` package
%%   \usepackage{import}
%% and then include the figures with
%%   \import{<path to file>}{<filename>.pgf}
%%
%% Matplotlib used the following preamble
%%   \usepackage{fontspec}
%%   \setmainfont{DejaVu Serif}
%%   \setsansfont{DejaVu Sans}
%%   \setmonofont{DejaVu Sans Mono}
%%
\begingroup%
\makeatletter%
\begin{pgfpicture}%
\pgfpathrectangle{\pgfpointorigin}{\pgfqpoint{5.427700in}{3.676603in}}%
\pgfusepath{use as bounding box, clip}%
\begin{pgfscope}%
\pgfsetbuttcap%
\pgfsetmiterjoin%
\definecolor{currentfill}{rgb}{1.000000,1.000000,1.000000}%
\pgfsetfillcolor{currentfill}%
\pgfsetlinewidth{0.000000pt}%
\definecolor{currentstroke}{rgb}{1.000000,1.000000,1.000000}%
\pgfsetstrokecolor{currentstroke}%
\pgfsetdash{}{0pt}%
\pgfpathmoveto{\pgfqpoint{0.000000in}{0.000000in}}%
\pgfpathlineto{\pgfqpoint{5.427700in}{0.000000in}}%
\pgfpathlineto{\pgfqpoint{5.427700in}{3.676603in}}%
\pgfpathlineto{\pgfqpoint{0.000000in}{3.676603in}}%
\pgfpathclose%
\pgfusepath{fill}%
\end{pgfscope}%
\begin{pgfscope}%
\pgfsetbuttcap%
\pgfsetmiterjoin%
\definecolor{currentfill}{rgb}{1.000000,1.000000,1.000000}%
\pgfsetfillcolor{currentfill}%
\pgfsetlinewidth{0.000000pt}%
\definecolor{currentstroke}{rgb}{0.000000,0.000000,0.000000}%
\pgfsetstrokecolor{currentstroke}%
\pgfsetstrokeopacity{0.000000}%
\pgfsetdash{}{0pt}%
\pgfpathmoveto{\pgfqpoint{0.564660in}{0.521603in}}%
\pgfpathlineto{\pgfqpoint{4.284660in}{0.521603in}}%
\pgfpathlineto{\pgfqpoint{4.284660in}{3.541603in}}%
\pgfpathlineto{\pgfqpoint{0.564660in}{3.541603in}}%
\pgfpathclose%
\pgfusepath{fill}%
\end{pgfscope}%
\begin{pgfscope}%
\pgfpathrectangle{\pgfqpoint{0.564660in}{0.521603in}}{\pgfqpoint{3.720000in}{3.020000in}} %
\pgfusepath{clip}%
\pgfsetbuttcap%
\pgfsetroundjoin%
\definecolor{currentfill}{rgb}{1.000000,0.255843,0.128999}%
\pgfsetfillcolor{currentfill}%
\pgfsetlinewidth{1.003750pt}%
\definecolor{currentstroke}{rgb}{1.000000,0.255843,0.128999}%
\pgfsetstrokecolor{currentstroke}%
\pgfsetdash{}{0pt}%
\pgfpathmoveto{\pgfqpoint{3.297585in}{2.468980in}}%
\pgfpathcurveto{\pgfqpoint{3.308636in}{2.468980in}}{\pgfqpoint{3.319235in}{2.473370in}}{\pgfqpoint{3.327048in}{2.481184in}}%
\pgfpathcurveto{\pgfqpoint{3.334862in}{2.488998in}}{\pgfqpoint{3.339252in}{2.499597in}}{\pgfqpoint{3.339252in}{2.510647in}}%
\pgfpathcurveto{\pgfqpoint{3.339252in}{2.521697in}}{\pgfqpoint{3.334862in}{2.532296in}}{\pgfqpoint{3.327048in}{2.540110in}}%
\pgfpathcurveto{\pgfqpoint{3.319235in}{2.547923in}}{\pgfqpoint{3.308636in}{2.552314in}}{\pgfqpoint{3.297585in}{2.552314in}}%
\pgfpathcurveto{\pgfqpoint{3.286535in}{2.552314in}}{\pgfqpoint{3.275936in}{2.547923in}}{\pgfqpoint{3.268123in}{2.540110in}}%
\pgfpathcurveto{\pgfqpoint{3.260309in}{2.532296in}}{\pgfqpoint{3.255919in}{2.521697in}}{\pgfqpoint{3.255919in}{2.510647in}}%
\pgfpathcurveto{\pgfqpoint{3.255919in}{2.499597in}}{\pgfqpoint{3.260309in}{2.488998in}}{\pgfqpoint{3.268123in}{2.481184in}}%
\pgfpathcurveto{\pgfqpoint{3.275936in}{2.473370in}}{\pgfqpoint{3.286535in}{2.468980in}}{\pgfqpoint{3.297585in}{2.468980in}}%
\pgfpathclose%
\pgfusepath{stroke,fill}%
\end{pgfscope}%
\begin{pgfscope}%
\pgfpathrectangle{\pgfqpoint{0.564660in}{0.521603in}}{\pgfqpoint{3.720000in}{3.020000in}} %
\pgfusepath{clip}%
\pgfsetbuttcap%
\pgfsetroundjoin%
\definecolor{currentfill}{rgb}{0.319608,0.279583,0.989980}%
\pgfsetfillcolor{currentfill}%
\pgfsetlinewidth{1.003750pt}%
\definecolor{currentstroke}{rgb}{0.319608,0.279583,0.989980}%
\pgfsetstrokecolor{currentstroke}%
\pgfsetdash{}{0pt}%
\pgfpathmoveto{\pgfqpoint{3.328555in}{1.873234in}}%
\pgfpathcurveto{\pgfqpoint{3.339606in}{1.873234in}}{\pgfqpoint{3.350205in}{1.877624in}}{\pgfqpoint{3.358018in}{1.885438in}}%
\pgfpathcurveto{\pgfqpoint{3.365832in}{1.893251in}}{\pgfqpoint{3.370222in}{1.903850in}}{\pgfqpoint{3.370222in}{1.914900in}}%
\pgfpathcurveto{\pgfqpoint{3.370222in}{1.925951in}}{\pgfqpoint{3.365832in}{1.936550in}}{\pgfqpoint{3.358018in}{1.944363in}}%
\pgfpathcurveto{\pgfqpoint{3.350205in}{1.952177in}}{\pgfqpoint{3.339606in}{1.956567in}}{\pgfqpoint{3.328555in}{1.956567in}}%
\pgfpathcurveto{\pgfqpoint{3.317505in}{1.956567in}}{\pgfqpoint{3.306906in}{1.952177in}}{\pgfqpoint{3.299093in}{1.944363in}}%
\pgfpathcurveto{\pgfqpoint{3.291279in}{1.936550in}}{\pgfqpoint{3.286889in}{1.925951in}}{\pgfqpoint{3.286889in}{1.914900in}}%
\pgfpathcurveto{\pgfqpoint{3.286889in}{1.903850in}}{\pgfqpoint{3.291279in}{1.893251in}}{\pgfqpoint{3.299093in}{1.885438in}}%
\pgfpathcurveto{\pgfqpoint{3.306906in}{1.877624in}}{\pgfqpoint{3.317505in}{1.873234in}}{\pgfqpoint{3.328555in}{1.873234in}}%
\pgfpathclose%
\pgfusepath{stroke,fill}%
\end{pgfscope}%
\begin{pgfscope}%
\pgfpathrectangle{\pgfqpoint{0.564660in}{0.521603in}}{\pgfqpoint{3.720000in}{3.020000in}} %
\pgfusepath{clip}%
\pgfsetbuttcap%
\pgfsetroundjoin%
\definecolor{currentfill}{rgb}{0.460784,0.061561,0.999526}%
\pgfsetfillcolor{currentfill}%
\pgfsetlinewidth{1.003750pt}%
\definecolor{currentstroke}{rgb}{0.460784,0.061561,0.999526}%
\pgfsetstrokecolor{currentstroke}%
\pgfsetdash{}{0pt}%
\pgfpathmoveto{\pgfqpoint{2.895722in}{1.799203in}}%
\pgfpathcurveto{\pgfqpoint{2.906772in}{1.799203in}}{\pgfqpoint{2.917371in}{1.803593in}}{\pgfqpoint{2.925185in}{1.811407in}}%
\pgfpathcurveto{\pgfqpoint{2.932998in}{1.819220in}}{\pgfqpoint{2.937389in}{1.829819in}}{\pgfqpoint{2.937389in}{1.840870in}}%
\pgfpathcurveto{\pgfqpoint{2.937389in}{1.851920in}}{\pgfqpoint{2.932998in}{1.862519in}}{\pgfqpoint{2.925185in}{1.870332in}}%
\pgfpathcurveto{\pgfqpoint{2.917371in}{1.878146in}}{\pgfqpoint{2.906772in}{1.882536in}}{\pgfqpoint{2.895722in}{1.882536in}}%
\pgfpathcurveto{\pgfqpoint{2.884672in}{1.882536in}}{\pgfqpoint{2.874073in}{1.878146in}}{\pgfqpoint{2.866259in}{1.870332in}}%
\pgfpathcurveto{\pgfqpoint{2.858445in}{1.862519in}}{\pgfqpoint{2.854055in}{1.851920in}}{\pgfqpoint{2.854055in}{1.840870in}}%
\pgfpathcurveto{\pgfqpoint{2.854055in}{1.829819in}}{\pgfqpoint{2.858445in}{1.819220in}}{\pgfqpoint{2.866259in}{1.811407in}}%
\pgfpathcurveto{\pgfqpoint{2.874073in}{1.803593in}}{\pgfqpoint{2.884672in}{1.799203in}}{\pgfqpoint{2.895722in}{1.799203in}}%
\pgfpathclose%
\pgfusepath{stroke,fill}%
\end{pgfscope}%
\begin{pgfscope}%
\pgfpathrectangle{\pgfqpoint{0.564660in}{0.521603in}}{\pgfqpoint{3.720000in}{3.020000in}} %
\pgfusepath{clip}%
\pgfsetbuttcap%
\pgfsetroundjoin%
\definecolor{currentfill}{rgb}{0.500000,0.000000,1.000000}%
\pgfsetfillcolor{currentfill}%
\pgfsetlinewidth{1.003750pt}%
\definecolor{currentstroke}{rgb}{0.500000,0.000000,1.000000}%
\pgfsetstrokecolor{currentstroke}%
\pgfsetdash{}{0pt}%
\pgfpathmoveto{\pgfqpoint{2.003834in}{2.169242in}}%
\pgfpathcurveto{\pgfqpoint{2.014884in}{2.169242in}}{\pgfqpoint{2.025483in}{2.173632in}}{\pgfqpoint{2.033297in}{2.181446in}}%
\pgfpathcurveto{\pgfqpoint{2.041110in}{2.189259in}}{\pgfqpoint{2.045500in}{2.199858in}}{\pgfqpoint{2.045500in}{2.210908in}}%
\pgfpathcurveto{\pgfqpoint{2.045500in}{2.221959in}}{\pgfqpoint{2.041110in}{2.232558in}}{\pgfqpoint{2.033297in}{2.240371in}}%
\pgfpathcurveto{\pgfqpoint{2.025483in}{2.248185in}}{\pgfqpoint{2.014884in}{2.252575in}}{\pgfqpoint{2.003834in}{2.252575in}}%
\pgfpathcurveto{\pgfqpoint{1.992784in}{2.252575in}}{\pgfqpoint{1.982185in}{2.248185in}}{\pgfqpoint{1.974371in}{2.240371in}}%
\pgfpathcurveto{\pgfqpoint{1.966557in}{2.232558in}}{\pgfqpoint{1.962167in}{2.221959in}}{\pgfqpoint{1.962167in}{2.210908in}}%
\pgfpathcurveto{\pgfqpoint{1.962167in}{2.199858in}}{\pgfqpoint{1.966557in}{2.189259in}}{\pgfqpoint{1.974371in}{2.181446in}}%
\pgfpathcurveto{\pgfqpoint{1.982185in}{2.173632in}}{\pgfqpoint{1.992784in}{2.169242in}}{\pgfqpoint{2.003834in}{2.169242in}}%
\pgfpathclose%
\pgfusepath{stroke,fill}%
\end{pgfscope}%
\begin{pgfscope}%
\pgfpathrectangle{\pgfqpoint{0.564660in}{0.521603in}}{\pgfqpoint{3.720000in}{3.020000in}} %
\pgfusepath{clip}%
\pgfsetbuttcap%
\pgfsetroundjoin%
\definecolor{currentfill}{rgb}{1.000000,0.000000,0.000000}%
\pgfsetfillcolor{currentfill}%
\pgfsetlinewidth{1.003750pt}%
\definecolor{currentstroke}{rgb}{1.000000,0.000000,0.000000}%
\pgfsetstrokecolor{currentstroke}%
\pgfsetdash{}{0pt}%
\pgfpathmoveto{\pgfqpoint{4.166276in}{1.932500in}}%
\pgfpathcurveto{\pgfqpoint{4.177326in}{1.932500in}}{\pgfqpoint{4.187926in}{1.936890in}}{\pgfqpoint{4.195739in}{1.944704in}}%
\pgfpathcurveto{\pgfqpoint{4.203553in}{1.952518in}}{\pgfqpoint{4.207943in}{1.963117in}}{\pgfqpoint{4.207943in}{1.974167in}}%
\pgfpathcurveto{\pgfqpoint{4.207943in}{1.985217in}}{\pgfqpoint{4.203553in}{1.995816in}}{\pgfqpoint{4.195739in}{2.003630in}}%
\pgfpathcurveto{\pgfqpoint{4.187926in}{2.011443in}}{\pgfqpoint{4.177326in}{2.015834in}}{\pgfqpoint{4.166276in}{2.015834in}}%
\pgfpathcurveto{\pgfqpoint{4.155226in}{2.015834in}}{\pgfqpoint{4.144627in}{2.011443in}}{\pgfqpoint{4.136814in}{2.003630in}}%
\pgfpathcurveto{\pgfqpoint{4.129000in}{1.995816in}}{\pgfqpoint{4.124610in}{1.985217in}}{\pgfqpoint{4.124610in}{1.974167in}}%
\pgfpathcurveto{\pgfqpoint{4.124610in}{1.963117in}}{\pgfqpoint{4.129000in}{1.952518in}}{\pgfqpoint{4.136814in}{1.944704in}}%
\pgfpathcurveto{\pgfqpoint{4.144627in}{1.936890in}}{\pgfqpoint{4.155226in}{1.932500in}}{\pgfqpoint{4.166276in}{1.932500in}}%
\pgfpathclose%
\pgfusepath{stroke,fill}%
\end{pgfscope}%
\begin{pgfscope}%
\pgfpathrectangle{\pgfqpoint{0.564660in}{0.521603in}}{\pgfqpoint{3.720000in}{3.020000in}} %
\pgfusepath{clip}%
\pgfsetbuttcap%
\pgfsetroundjoin%
\definecolor{currentfill}{rgb}{1.000000,0.000000,0.000000}%
\pgfsetfillcolor{currentfill}%
\pgfsetlinewidth{1.003750pt}%
\definecolor{currentstroke}{rgb}{1.000000,0.000000,0.000000}%
\pgfsetstrokecolor{currentstroke}%
\pgfsetdash{}{0pt}%
\pgfpathmoveto{\pgfqpoint{3.773192in}{2.692753in}}%
\pgfpathcurveto{\pgfqpoint{3.784242in}{2.692753in}}{\pgfqpoint{3.794841in}{2.697143in}}{\pgfqpoint{3.802655in}{2.704957in}}%
\pgfpathcurveto{\pgfqpoint{3.810468in}{2.712770in}}{\pgfqpoint{3.814859in}{2.723369in}}{\pgfqpoint{3.814859in}{2.734420in}}%
\pgfpathcurveto{\pgfqpoint{3.814859in}{2.745470in}}{\pgfqpoint{3.810468in}{2.756069in}}{\pgfqpoint{3.802655in}{2.763882in}}%
\pgfpathcurveto{\pgfqpoint{3.794841in}{2.771696in}}{\pgfqpoint{3.784242in}{2.776086in}}{\pgfqpoint{3.773192in}{2.776086in}}%
\pgfpathcurveto{\pgfqpoint{3.762142in}{2.776086in}}{\pgfqpoint{3.751543in}{2.771696in}}{\pgfqpoint{3.743729in}{2.763882in}}%
\pgfpathcurveto{\pgfqpoint{3.735915in}{2.756069in}}{\pgfqpoint{3.731525in}{2.745470in}}{\pgfqpoint{3.731525in}{2.734420in}}%
\pgfpathcurveto{\pgfqpoint{3.731525in}{2.723369in}}{\pgfqpoint{3.735915in}{2.712770in}}{\pgfqpoint{3.743729in}{2.704957in}}%
\pgfpathcurveto{\pgfqpoint{3.751543in}{2.697143in}}{\pgfqpoint{3.762142in}{2.692753in}}{\pgfqpoint{3.773192in}{2.692753in}}%
\pgfpathclose%
\pgfusepath{stroke,fill}%
\end{pgfscope}%
\begin{pgfscope}%
\pgfpathrectangle{\pgfqpoint{0.564660in}{0.521603in}}{\pgfqpoint{3.720000in}{3.020000in}} %
\pgfusepath{clip}%
\pgfsetbuttcap%
\pgfsetroundjoin%
\definecolor{currentfill}{rgb}{1.000000,0.171626,0.086133}%
\pgfsetfillcolor{currentfill}%
\pgfsetlinewidth{1.003750pt}%
\definecolor{currentstroke}{rgb}{1.000000,0.171626,0.086133}%
\pgfsetstrokecolor{currentstroke}%
\pgfsetdash{}{0pt}%
\pgfpathmoveto{\pgfqpoint{3.722241in}{2.302709in}}%
\pgfpathcurveto{\pgfqpoint{3.733292in}{2.302709in}}{\pgfqpoint{3.743891in}{2.307099in}}{\pgfqpoint{3.751704in}{2.314913in}}%
\pgfpathcurveto{\pgfqpoint{3.759518in}{2.322726in}}{\pgfqpoint{3.763908in}{2.333325in}}{\pgfqpoint{3.763908in}{2.344375in}}%
\pgfpathcurveto{\pgfqpoint{3.763908in}{2.355425in}}{\pgfqpoint{3.759518in}{2.366025in}}{\pgfqpoint{3.751704in}{2.373838in}}%
\pgfpathcurveto{\pgfqpoint{3.743891in}{2.381652in}}{\pgfqpoint{3.733292in}{2.386042in}}{\pgfqpoint{3.722241in}{2.386042in}}%
\pgfpathcurveto{\pgfqpoint{3.711191in}{2.386042in}}{\pgfqpoint{3.700592in}{2.381652in}}{\pgfqpoint{3.692779in}{2.373838in}}%
\pgfpathcurveto{\pgfqpoint{3.684965in}{2.366025in}}{\pgfqpoint{3.680575in}{2.355425in}}{\pgfqpoint{3.680575in}{2.344375in}}%
\pgfpathcurveto{\pgfqpoint{3.680575in}{2.333325in}}{\pgfqpoint{3.684965in}{2.322726in}}{\pgfqpoint{3.692779in}{2.314913in}}%
\pgfpathcurveto{\pgfqpoint{3.700592in}{2.307099in}}{\pgfqpoint{3.711191in}{2.302709in}}{\pgfqpoint{3.722241in}{2.302709in}}%
\pgfpathclose%
\pgfusepath{stroke,fill}%
\end{pgfscope}%
\begin{pgfscope}%
\pgfpathrectangle{\pgfqpoint{0.564660in}{0.521603in}}{\pgfqpoint{3.720000in}{3.020000in}} %
\pgfusepath{clip}%
\pgfsetbuttcap%
\pgfsetroundjoin%
\definecolor{currentfill}{rgb}{1.000000,0.171626,0.086133}%
\pgfsetfillcolor{currentfill}%
\pgfsetlinewidth{1.003750pt}%
\definecolor{currentstroke}{rgb}{1.000000,0.171626,0.086133}%
\pgfsetstrokecolor{currentstroke}%
\pgfsetdash{}{0pt}%
\pgfpathmoveto{\pgfqpoint{3.468726in}{0.678635in}}%
\pgfpathcurveto{\pgfqpoint{3.479776in}{0.678635in}}{\pgfqpoint{3.490375in}{0.683025in}}{\pgfqpoint{3.498189in}{0.690839in}}%
\pgfpathcurveto{\pgfqpoint{3.506002in}{0.698653in}}{\pgfqpoint{3.510392in}{0.709252in}}{\pgfqpoint{3.510392in}{0.720302in}}%
\pgfpathcurveto{\pgfqpoint{3.510392in}{0.731352in}}{\pgfqpoint{3.506002in}{0.741951in}}{\pgfqpoint{3.498189in}{0.749765in}}%
\pgfpathcurveto{\pgfqpoint{3.490375in}{0.757578in}}{\pgfqpoint{3.479776in}{0.761968in}}{\pgfqpoint{3.468726in}{0.761968in}}%
\pgfpathcurveto{\pgfqpoint{3.457676in}{0.761968in}}{\pgfqpoint{3.447077in}{0.757578in}}{\pgfqpoint{3.439263in}{0.749765in}}%
\pgfpathcurveto{\pgfqpoint{3.431449in}{0.741951in}}{\pgfqpoint{3.427059in}{0.731352in}}{\pgfqpoint{3.427059in}{0.720302in}}%
\pgfpathcurveto{\pgfqpoint{3.427059in}{0.709252in}}{\pgfqpoint{3.431449in}{0.698653in}}{\pgfqpoint{3.439263in}{0.690839in}}%
\pgfpathcurveto{\pgfqpoint{3.447077in}{0.683025in}}{\pgfqpoint{3.457676in}{0.678635in}}{\pgfqpoint{3.468726in}{0.678635in}}%
\pgfpathclose%
\pgfusepath{stroke,fill}%
\end{pgfscope}%
\begin{pgfscope}%
\pgfpathrectangle{\pgfqpoint{0.564660in}{0.521603in}}{\pgfqpoint{3.720000in}{3.020000in}} %
\pgfusepath{clip}%
\pgfsetbuttcap%
\pgfsetroundjoin%
\definecolor{currentfill}{rgb}{0.660784,0.968276,0.612420}%
\pgfsetfillcolor{currentfill}%
\pgfsetlinewidth{1.003750pt}%
\definecolor{currentstroke}{rgb}{0.660784,0.968276,0.612420}%
\pgfsetstrokecolor{currentstroke}%
\pgfsetdash{}{0pt}%
\pgfpathmoveto{\pgfqpoint{3.256354in}{0.943869in}}%
\pgfpathcurveto{\pgfqpoint{3.267405in}{0.943869in}}{\pgfqpoint{3.278004in}{0.948259in}}{\pgfqpoint{3.285817in}{0.956072in}}%
\pgfpathcurveto{\pgfqpoint{3.293631in}{0.963886in}}{\pgfqpoint{3.298021in}{0.974485in}}{\pgfqpoint{3.298021in}{0.985535in}}%
\pgfpathcurveto{\pgfqpoint{3.298021in}{0.996585in}}{\pgfqpoint{3.293631in}{1.007184in}}{\pgfqpoint{3.285817in}{1.014998in}}%
\pgfpathcurveto{\pgfqpoint{3.278004in}{1.022812in}}{\pgfqpoint{3.267405in}{1.027202in}}{\pgfqpoint{3.256354in}{1.027202in}}%
\pgfpathcurveto{\pgfqpoint{3.245304in}{1.027202in}}{\pgfqpoint{3.234705in}{1.022812in}}{\pgfqpoint{3.226892in}{1.014998in}}%
\pgfpathcurveto{\pgfqpoint{3.219078in}{1.007184in}}{\pgfqpoint{3.214688in}{0.996585in}}{\pgfqpoint{3.214688in}{0.985535in}}%
\pgfpathcurveto{\pgfqpoint{3.214688in}{0.974485in}}{\pgfqpoint{3.219078in}{0.963886in}}{\pgfqpoint{3.226892in}{0.956072in}}%
\pgfpathcurveto{\pgfqpoint{3.234705in}{0.948259in}}{\pgfqpoint{3.245304in}{0.943869in}}{\pgfqpoint{3.256354in}{0.943869in}}%
\pgfpathclose%
\pgfusepath{stroke,fill}%
\end{pgfscope}%
\begin{pgfscope}%
\pgfpathrectangle{\pgfqpoint{0.564660in}{0.521603in}}{\pgfqpoint{3.720000in}{3.020000in}} %
\pgfusepath{clip}%
\pgfsetbuttcap%
\pgfsetroundjoin%
\definecolor{currentfill}{rgb}{0.103922,0.812622,0.889604}%
\pgfsetfillcolor{currentfill}%
\pgfsetlinewidth{1.003750pt}%
\definecolor{currentstroke}{rgb}{0.103922,0.812622,0.889604}%
\pgfsetstrokecolor{currentstroke}%
\pgfsetdash{}{0pt}%
\pgfpathmoveto{\pgfqpoint{3.210081in}{1.002082in}}%
\pgfpathcurveto{\pgfqpoint{3.221131in}{1.002082in}}{\pgfqpoint{3.231730in}{1.006472in}}{\pgfqpoint{3.239544in}{1.014286in}}%
\pgfpathcurveto{\pgfqpoint{3.247358in}{1.022099in}}{\pgfqpoint{3.251748in}{1.032698in}}{\pgfqpoint{3.251748in}{1.043748in}}%
\pgfpathcurveto{\pgfqpoint{3.251748in}{1.054799in}}{\pgfqpoint{3.247358in}{1.065398in}}{\pgfqpoint{3.239544in}{1.073211in}}%
\pgfpathcurveto{\pgfqpoint{3.231730in}{1.081025in}}{\pgfqpoint{3.221131in}{1.085415in}}{\pgfqpoint{3.210081in}{1.085415in}}%
\pgfpathcurveto{\pgfqpoint{3.199031in}{1.085415in}}{\pgfqpoint{3.188432in}{1.081025in}}{\pgfqpoint{3.180618in}{1.073211in}}%
\pgfpathcurveto{\pgfqpoint{3.172805in}{1.065398in}}{\pgfqpoint{3.168415in}{1.054799in}}{\pgfqpoint{3.168415in}{1.043748in}}%
\pgfpathcurveto{\pgfqpoint{3.168415in}{1.032698in}}{\pgfqpoint{3.172805in}{1.022099in}}{\pgfqpoint{3.180618in}{1.014286in}}%
\pgfpathcurveto{\pgfqpoint{3.188432in}{1.006472in}}{\pgfqpoint{3.199031in}{1.002082in}}{\pgfqpoint{3.210081in}{1.002082in}}%
\pgfpathclose%
\pgfusepath{stroke,fill}%
\end{pgfscope}%
\begin{pgfscope}%
\pgfpathrectangle{\pgfqpoint{0.564660in}{0.521603in}}{\pgfqpoint{3.720000in}{3.020000in}} %
\pgfusepath{clip}%
\pgfsetbuttcap%
\pgfsetroundjoin%
\definecolor{currentfill}{rgb}{0.005882,0.700543,0.925638}%
\pgfsetfillcolor{currentfill}%
\pgfsetlinewidth{1.003750pt}%
\definecolor{currentstroke}{rgb}{0.005882,0.700543,0.925638}%
\pgfsetstrokecolor{currentstroke}%
\pgfsetdash{}{0pt}%
\pgfpathmoveto{\pgfqpoint{2.706316in}{1.513387in}}%
\pgfpathcurveto{\pgfqpoint{2.717366in}{1.513387in}}{\pgfqpoint{2.727965in}{1.517777in}}{\pgfqpoint{2.735779in}{1.525591in}}%
\pgfpathcurveto{\pgfqpoint{2.743593in}{1.533405in}}{\pgfqpoint{2.747983in}{1.544004in}}{\pgfqpoint{2.747983in}{1.555054in}}%
\pgfpathcurveto{\pgfqpoint{2.747983in}{1.566104in}}{\pgfqpoint{2.743593in}{1.576703in}}{\pgfqpoint{2.735779in}{1.584517in}}%
\pgfpathcurveto{\pgfqpoint{2.727965in}{1.592330in}}{\pgfqpoint{2.717366in}{1.596720in}}{\pgfqpoint{2.706316in}{1.596720in}}%
\pgfpathcurveto{\pgfqpoint{2.695266in}{1.596720in}}{\pgfqpoint{2.684667in}{1.592330in}}{\pgfqpoint{2.676853in}{1.584517in}}%
\pgfpathcurveto{\pgfqpoint{2.669040in}{1.576703in}}{\pgfqpoint{2.664649in}{1.566104in}}{\pgfqpoint{2.664649in}{1.555054in}}%
\pgfpathcurveto{\pgfqpoint{2.664649in}{1.544004in}}{\pgfqpoint{2.669040in}{1.533405in}}{\pgfqpoint{2.676853in}{1.525591in}}%
\pgfpathcurveto{\pgfqpoint{2.684667in}{1.517777in}}{\pgfqpoint{2.695266in}{1.513387in}}{\pgfqpoint{2.706316in}{1.513387in}}%
\pgfpathclose%
\pgfusepath{stroke,fill}%
\end{pgfscope}%
\begin{pgfscope}%
\pgfpathrectangle{\pgfqpoint{0.564660in}{0.521603in}}{\pgfqpoint{3.720000in}{3.020000in}} %
\pgfusepath{clip}%
\pgfsetbuttcap%
\pgfsetroundjoin%
\definecolor{currentfill}{rgb}{0.005882,0.700543,0.925638}%
\pgfsetfillcolor{currentfill}%
\pgfsetlinewidth{1.003750pt}%
\definecolor{currentstroke}{rgb}{0.005882,0.700543,0.925638}%
\pgfsetstrokecolor{currentstroke}%
\pgfsetdash{}{0pt}%
\pgfpathmoveto{\pgfqpoint{2.196868in}{2.803645in}}%
\pgfpathcurveto{\pgfqpoint{2.207918in}{2.803645in}}{\pgfqpoint{2.218517in}{2.808035in}}{\pgfqpoint{2.226330in}{2.815849in}}%
\pgfpathcurveto{\pgfqpoint{2.234144in}{2.823663in}}{\pgfqpoint{2.238534in}{2.834262in}}{\pgfqpoint{2.238534in}{2.845312in}}%
\pgfpathcurveto{\pgfqpoint{2.238534in}{2.856362in}}{\pgfqpoint{2.234144in}{2.866961in}}{\pgfqpoint{2.226330in}{2.874775in}}%
\pgfpathcurveto{\pgfqpoint{2.218517in}{2.882588in}}{\pgfqpoint{2.207918in}{2.886978in}}{\pgfqpoint{2.196868in}{2.886978in}}%
\pgfpathcurveto{\pgfqpoint{2.185818in}{2.886978in}}{\pgfqpoint{2.175219in}{2.882588in}}{\pgfqpoint{2.167405in}{2.874775in}}%
\pgfpathcurveto{\pgfqpoint{2.159591in}{2.866961in}}{\pgfqpoint{2.155201in}{2.856362in}}{\pgfqpoint{2.155201in}{2.845312in}}%
\pgfpathcurveto{\pgfqpoint{2.155201in}{2.834262in}}{\pgfqpoint{2.159591in}{2.823663in}}{\pgfqpoint{2.167405in}{2.815849in}}%
\pgfpathcurveto{\pgfqpoint{2.175219in}{2.808035in}}{\pgfqpoint{2.185818in}{2.803645in}}{\pgfqpoint{2.196868in}{2.803645in}}%
\pgfpathclose%
\pgfusepath{stroke,fill}%
\end{pgfscope}%
\begin{pgfscope}%
\pgfpathrectangle{\pgfqpoint{0.564660in}{0.521603in}}{\pgfqpoint{3.720000in}{3.020000in}} %
\pgfusepath{clip}%
\pgfsetbuttcap%
\pgfsetroundjoin%
\definecolor{currentfill}{rgb}{0.139216,0.536867,0.960122}%
\pgfsetfillcolor{currentfill}%
\pgfsetlinewidth{1.003750pt}%
\definecolor{currentstroke}{rgb}{0.139216,0.536867,0.960122}%
\pgfsetstrokecolor{currentstroke}%
\pgfsetdash{}{0pt}%
\pgfpathmoveto{\pgfqpoint{3.113579in}{2.722935in}}%
\pgfpathcurveto{\pgfqpoint{3.124629in}{2.722935in}}{\pgfqpoint{3.135229in}{2.727325in}}{\pgfqpoint{3.143042in}{2.735139in}}%
\pgfpathcurveto{\pgfqpoint{3.150856in}{2.742952in}}{\pgfqpoint{3.155246in}{2.753551in}}{\pgfqpoint{3.155246in}{2.764601in}}%
\pgfpathcurveto{\pgfqpoint{3.155246in}{2.775651in}}{\pgfqpoint{3.150856in}{2.786250in}}{\pgfqpoint{3.143042in}{2.794064in}}%
\pgfpathcurveto{\pgfqpoint{3.135229in}{2.801878in}}{\pgfqpoint{3.124629in}{2.806268in}}{\pgfqpoint{3.113579in}{2.806268in}}%
\pgfpathcurveto{\pgfqpoint{3.102529in}{2.806268in}}{\pgfqpoint{3.091930in}{2.801878in}}{\pgfqpoint{3.084117in}{2.794064in}}%
\pgfpathcurveto{\pgfqpoint{3.076303in}{2.786250in}}{\pgfqpoint{3.071913in}{2.775651in}}{\pgfqpoint{3.071913in}{2.764601in}}%
\pgfpathcurveto{\pgfqpoint{3.071913in}{2.753551in}}{\pgfqpoint{3.076303in}{2.742952in}}{\pgfqpoint{3.084117in}{2.735139in}}%
\pgfpathcurveto{\pgfqpoint{3.091930in}{2.727325in}}{\pgfqpoint{3.102529in}{2.722935in}}{\pgfqpoint{3.113579in}{2.722935in}}%
\pgfpathclose%
\pgfusepath{stroke,fill}%
\end{pgfscope}%
\begin{pgfscope}%
\pgfpathrectangle{\pgfqpoint{0.564660in}{0.521603in}}{\pgfqpoint{3.720000in}{3.020000in}} %
\pgfusepath{clip}%
\pgfsetbuttcap%
\pgfsetroundjoin%
\definecolor{currentfill}{rgb}{0.139216,0.536867,0.960122}%
\pgfsetfillcolor{currentfill}%
\pgfsetlinewidth{1.003750pt}%
\definecolor{currentstroke}{rgb}{0.139216,0.536867,0.960122}%
\pgfsetstrokecolor{currentstroke}%
\pgfsetdash{}{0pt}%
\pgfpathmoveto{\pgfqpoint{2.812260in}{2.289241in}}%
\pgfpathcurveto{\pgfqpoint{2.823310in}{2.289241in}}{\pgfqpoint{2.833909in}{2.293632in}}{\pgfqpoint{2.841723in}{2.301445in}}%
\pgfpathcurveto{\pgfqpoint{2.849536in}{2.309259in}}{\pgfqpoint{2.853926in}{2.319858in}}{\pgfqpoint{2.853926in}{2.330908in}}%
\pgfpathcurveto{\pgfqpoint{2.853926in}{2.341958in}}{\pgfqpoint{2.849536in}{2.352557in}}{\pgfqpoint{2.841723in}{2.360371in}}%
\pgfpathcurveto{\pgfqpoint{2.833909in}{2.368185in}}{\pgfqpoint{2.823310in}{2.372575in}}{\pgfqpoint{2.812260in}{2.372575in}}%
\pgfpathcurveto{\pgfqpoint{2.801210in}{2.372575in}}{\pgfqpoint{2.790611in}{2.368185in}}{\pgfqpoint{2.782797in}{2.360371in}}%
\pgfpathcurveto{\pgfqpoint{2.774983in}{2.352557in}}{\pgfqpoint{2.770593in}{2.341958in}}{\pgfqpoint{2.770593in}{2.330908in}}%
\pgfpathcurveto{\pgfqpoint{2.770593in}{2.319858in}}{\pgfqpoint{2.774983in}{2.309259in}}{\pgfqpoint{2.782797in}{2.301445in}}%
\pgfpathcurveto{\pgfqpoint{2.790611in}{2.293632in}}{\pgfqpoint{2.801210in}{2.289241in}}{\pgfqpoint{2.812260in}{2.289241in}}%
\pgfpathclose%
\pgfusepath{stroke,fill}%
\end{pgfscope}%
\begin{pgfscope}%
\pgfpathrectangle{\pgfqpoint{0.564660in}{0.521603in}}{\pgfqpoint{3.720000in}{3.020000in}} %
\pgfusepath{clip}%
\pgfsetbuttcap%
\pgfsetroundjoin%
\definecolor{currentfill}{rgb}{0.139216,0.536867,0.960122}%
\pgfsetfillcolor{currentfill}%
\pgfsetlinewidth{1.003750pt}%
\definecolor{currentstroke}{rgb}{0.139216,0.536867,0.960122}%
\pgfsetstrokecolor{currentstroke}%
\pgfsetdash{}{0pt}%
\pgfpathmoveto{\pgfqpoint{2.634682in}{2.190627in}}%
\pgfpathcurveto{\pgfqpoint{2.645732in}{2.190627in}}{\pgfqpoint{2.656331in}{2.195017in}}{\pgfqpoint{2.664145in}{2.202831in}}%
\pgfpathcurveto{\pgfqpoint{2.671958in}{2.210644in}}{\pgfqpoint{2.676349in}{2.221243in}}{\pgfqpoint{2.676349in}{2.232294in}}%
\pgfpathcurveto{\pgfqpoint{2.676349in}{2.243344in}}{\pgfqpoint{2.671958in}{2.253943in}}{\pgfqpoint{2.664145in}{2.261756in}}%
\pgfpathcurveto{\pgfqpoint{2.656331in}{2.269570in}}{\pgfqpoint{2.645732in}{2.273960in}}{\pgfqpoint{2.634682in}{2.273960in}}%
\pgfpathcurveto{\pgfqpoint{2.623632in}{2.273960in}}{\pgfqpoint{2.613033in}{2.269570in}}{\pgfqpoint{2.605219in}{2.261756in}}%
\pgfpathcurveto{\pgfqpoint{2.597406in}{2.253943in}}{\pgfqpoint{2.593015in}{2.243344in}}{\pgfqpoint{2.593015in}{2.232294in}}%
\pgfpathcurveto{\pgfqpoint{2.593015in}{2.221243in}}{\pgfqpoint{2.597406in}{2.210644in}}{\pgfqpoint{2.605219in}{2.202831in}}%
\pgfpathcurveto{\pgfqpoint{2.613033in}{2.195017in}}{\pgfqpoint{2.623632in}{2.190627in}}{\pgfqpoint{2.634682in}{2.190627in}}%
\pgfpathclose%
\pgfusepath{stroke,fill}%
\end{pgfscope}%
\begin{pgfscope}%
\pgfpathrectangle{\pgfqpoint{0.564660in}{0.521603in}}{\pgfqpoint{3.720000in}{3.020000in}} %
\pgfusepath{clip}%
\pgfsetbuttcap%
\pgfsetroundjoin%
\definecolor{currentfill}{rgb}{0.139216,0.536867,0.960122}%
\pgfsetfillcolor{currentfill}%
\pgfsetlinewidth{1.003750pt}%
\definecolor{currentstroke}{rgb}{0.139216,0.536867,0.960122}%
\pgfsetstrokecolor{currentstroke}%
\pgfsetdash{}{0pt}%
\pgfpathmoveto{\pgfqpoint{2.469078in}{1.382766in}}%
\pgfpathcurveto{\pgfqpoint{2.480128in}{1.382766in}}{\pgfqpoint{2.490727in}{1.387157in}}{\pgfqpoint{2.498541in}{1.394970in}}%
\pgfpathcurveto{\pgfqpoint{2.506354in}{1.402784in}}{\pgfqpoint{2.510744in}{1.413383in}}{\pgfqpoint{2.510744in}{1.424433in}}%
\pgfpathcurveto{\pgfqpoint{2.510744in}{1.435483in}}{\pgfqpoint{2.506354in}{1.446082in}}{\pgfqpoint{2.498541in}{1.453896in}}%
\pgfpathcurveto{\pgfqpoint{2.490727in}{1.461710in}}{\pgfqpoint{2.480128in}{1.466100in}}{\pgfqpoint{2.469078in}{1.466100in}}%
\pgfpathcurveto{\pgfqpoint{2.458028in}{1.466100in}}{\pgfqpoint{2.447429in}{1.461710in}}{\pgfqpoint{2.439615in}{1.453896in}}%
\pgfpathcurveto{\pgfqpoint{2.431801in}{1.446082in}}{\pgfqpoint{2.427411in}{1.435483in}}{\pgfqpoint{2.427411in}{1.424433in}}%
\pgfpathcurveto{\pgfqpoint{2.427411in}{1.413383in}}{\pgfqpoint{2.431801in}{1.402784in}}{\pgfqpoint{2.439615in}{1.394970in}}%
\pgfpathcurveto{\pgfqpoint{2.447429in}{1.387157in}}{\pgfqpoint{2.458028in}{1.382766in}}{\pgfqpoint{2.469078in}{1.382766in}}%
\pgfpathclose%
\pgfusepath{stroke,fill}%
\end{pgfscope}%
\begin{pgfscope}%
\pgfpathrectangle{\pgfqpoint{0.564660in}{0.521603in}}{\pgfqpoint{3.720000in}{3.020000in}} %
\pgfusepath{clip}%
\pgfsetbuttcap%
\pgfsetroundjoin%
\definecolor{currentfill}{rgb}{0.139216,0.536867,0.960122}%
\pgfsetfillcolor{currentfill}%
\pgfsetlinewidth{1.003750pt}%
\definecolor{currentstroke}{rgb}{0.139216,0.536867,0.960122}%
\pgfsetstrokecolor{currentstroke}%
\pgfsetdash{}{0pt}%
\pgfpathmoveto{\pgfqpoint{2.011780in}{3.301238in}}%
\pgfpathcurveto{\pgfqpoint{2.022830in}{3.301238in}}{\pgfqpoint{2.033429in}{3.305628in}}{\pgfqpoint{2.041243in}{3.313442in}}%
\pgfpathcurveto{\pgfqpoint{2.049057in}{3.321256in}}{\pgfqpoint{2.053447in}{3.331855in}}{\pgfqpoint{2.053447in}{3.342905in}}%
\pgfpathcurveto{\pgfqpoint{2.053447in}{3.353955in}}{\pgfqpoint{2.049057in}{3.364554in}}{\pgfqpoint{2.041243in}{3.372368in}}%
\pgfpathcurveto{\pgfqpoint{2.033429in}{3.380181in}}{\pgfqpoint{2.022830in}{3.384572in}}{\pgfqpoint{2.011780in}{3.384572in}}%
\pgfpathcurveto{\pgfqpoint{2.000730in}{3.384572in}}{\pgfqpoint{1.990131in}{3.380181in}}{\pgfqpoint{1.982317in}{3.372368in}}%
\pgfpathcurveto{\pgfqpoint{1.974504in}{3.364554in}}{\pgfqpoint{1.970113in}{3.353955in}}{\pgfqpoint{1.970113in}{3.342905in}}%
\pgfpathcurveto{\pgfqpoint{1.970113in}{3.331855in}}{\pgfqpoint{1.974504in}{3.321256in}}{\pgfqpoint{1.982317in}{3.313442in}}%
\pgfpathcurveto{\pgfqpoint{1.990131in}{3.305628in}}{\pgfqpoint{2.000730in}{3.301238in}}{\pgfqpoint{2.011780in}{3.301238in}}%
\pgfpathclose%
\pgfusepath{stroke,fill}%
\end{pgfscope}%
\begin{pgfscope}%
\pgfsetbuttcap%
\pgfsetroundjoin%
\definecolor{currentfill}{rgb}{0.000000,0.000000,0.000000}%
\pgfsetfillcolor{currentfill}%
\pgfsetlinewidth{0.803000pt}%
\definecolor{currentstroke}{rgb}{0.000000,0.000000,0.000000}%
\pgfsetstrokecolor{currentstroke}%
\pgfsetdash{}{0pt}%
\pgfsys@defobject{currentmarker}{\pgfqpoint{0.000000in}{-0.048611in}}{\pgfqpoint{0.000000in}{0.000000in}}{%
\pgfpathmoveto{\pgfqpoint{0.000000in}{0.000000in}}%
\pgfpathlineto{\pgfqpoint{0.000000in}{-0.048611in}}%
\pgfusepath{stroke,fill}%
}%
\begin{pgfscope}%
\pgfsys@transformshift{0.564660in}{0.521603in}%
\pgfsys@useobject{currentmarker}{}%
\end{pgfscope}%
\end{pgfscope}%
\begin{pgfscope}%
\pgftext[x=0.564660in,y=0.424381in,,top]{\rmfamily\fontsize{10.000000}{12.000000}\selectfont \(\displaystyle 10^{-2}\)}%
\end{pgfscope}%
\begin{pgfscope}%
\pgfsetbuttcap%
\pgfsetroundjoin%
\definecolor{currentfill}{rgb}{0.000000,0.000000,0.000000}%
\pgfsetfillcolor{currentfill}%
\pgfsetlinewidth{0.803000pt}%
\definecolor{currentstroke}{rgb}{0.000000,0.000000,0.000000}%
\pgfsetstrokecolor{currentstroke}%
\pgfsetdash{}{0pt}%
\pgfsys@defobject{currentmarker}{\pgfqpoint{0.000000in}{-0.048611in}}{\pgfqpoint{0.000000in}{0.000000in}}{%
\pgfpathmoveto{\pgfqpoint{0.000000in}{0.000000in}}%
\pgfpathlineto{\pgfqpoint{0.000000in}{-0.048611in}}%
\pgfusepath{stroke,fill}%
}%
\begin{pgfscope}%
\pgfsys@transformshift{2.397001in}{0.521603in}%
\pgfsys@useobject{currentmarker}{}%
\end{pgfscope}%
\end{pgfscope}%
\begin{pgfscope}%
\pgftext[x=2.397001in,y=0.424381in,,top]{\rmfamily\fontsize{10.000000}{12.000000}\selectfont \(\displaystyle 10^{-1}\)}%
\end{pgfscope}%
\begin{pgfscope}%
\pgfsetbuttcap%
\pgfsetroundjoin%
\definecolor{currentfill}{rgb}{0.000000,0.000000,0.000000}%
\pgfsetfillcolor{currentfill}%
\pgfsetlinewidth{0.803000pt}%
\definecolor{currentstroke}{rgb}{0.000000,0.000000,0.000000}%
\pgfsetstrokecolor{currentstroke}%
\pgfsetdash{}{0pt}%
\pgfsys@defobject{currentmarker}{\pgfqpoint{0.000000in}{-0.048611in}}{\pgfqpoint{0.000000in}{0.000000in}}{%
\pgfpathmoveto{\pgfqpoint{0.000000in}{0.000000in}}%
\pgfpathlineto{\pgfqpoint{0.000000in}{-0.048611in}}%
\pgfusepath{stroke,fill}%
}%
\begin{pgfscope}%
\pgfsys@transformshift{4.229341in}{0.521603in}%
\pgfsys@useobject{currentmarker}{}%
\end{pgfscope}%
\end{pgfscope}%
\begin{pgfscope}%
\pgftext[x=4.229341in,y=0.424381in,,top]{\rmfamily\fontsize{10.000000}{12.000000}\selectfont \(\displaystyle 10^{0}\)}%
\end{pgfscope}%
\begin{pgfscope}%
\pgfsetbuttcap%
\pgfsetroundjoin%
\definecolor{currentfill}{rgb}{0.000000,0.000000,0.000000}%
\pgfsetfillcolor{currentfill}%
\pgfsetlinewidth{0.602250pt}%
\definecolor{currentstroke}{rgb}{0.000000,0.000000,0.000000}%
\pgfsetstrokecolor{currentstroke}%
\pgfsetdash{}{0pt}%
\pgfsys@defobject{currentmarker}{\pgfqpoint{0.000000in}{-0.027778in}}{\pgfqpoint{0.000000in}{0.000000in}}{%
\pgfpathmoveto{\pgfqpoint{0.000000in}{0.000000in}}%
\pgfpathlineto{\pgfqpoint{0.000000in}{-0.027778in}}%
\pgfusepath{stroke,fill}%
}%
\begin{pgfscope}%
\pgfsys@transformshift{1.116250in}{0.521603in}%
\pgfsys@useobject{currentmarker}{}%
\end{pgfscope}%
\end{pgfscope}%
\begin{pgfscope}%
\pgfsetbuttcap%
\pgfsetroundjoin%
\definecolor{currentfill}{rgb}{0.000000,0.000000,0.000000}%
\pgfsetfillcolor{currentfill}%
\pgfsetlinewidth{0.602250pt}%
\definecolor{currentstroke}{rgb}{0.000000,0.000000,0.000000}%
\pgfsetstrokecolor{currentstroke}%
\pgfsetdash{}{0pt}%
\pgfsys@defobject{currentmarker}{\pgfqpoint{0.000000in}{-0.027778in}}{\pgfqpoint{0.000000in}{0.000000in}}{%
\pgfpathmoveto{\pgfqpoint{0.000000in}{0.000000in}}%
\pgfpathlineto{\pgfqpoint{0.000000in}{-0.027778in}}%
\pgfusepath{stroke,fill}%
}%
\begin{pgfscope}%
\pgfsys@transformshift{1.438909in}{0.521603in}%
\pgfsys@useobject{currentmarker}{}%
\end{pgfscope}%
\end{pgfscope}%
\begin{pgfscope}%
\pgfsetbuttcap%
\pgfsetroundjoin%
\definecolor{currentfill}{rgb}{0.000000,0.000000,0.000000}%
\pgfsetfillcolor{currentfill}%
\pgfsetlinewidth{0.602250pt}%
\definecolor{currentstroke}{rgb}{0.000000,0.000000,0.000000}%
\pgfsetstrokecolor{currentstroke}%
\pgfsetdash{}{0pt}%
\pgfsys@defobject{currentmarker}{\pgfqpoint{0.000000in}{-0.027778in}}{\pgfqpoint{0.000000in}{0.000000in}}{%
\pgfpathmoveto{\pgfqpoint{0.000000in}{0.000000in}}%
\pgfpathlineto{\pgfqpoint{0.000000in}{-0.027778in}}%
\pgfusepath{stroke,fill}%
}%
\begin{pgfscope}%
\pgfsys@transformshift{1.667839in}{0.521603in}%
\pgfsys@useobject{currentmarker}{}%
\end{pgfscope}%
\end{pgfscope}%
\begin{pgfscope}%
\pgfsetbuttcap%
\pgfsetroundjoin%
\definecolor{currentfill}{rgb}{0.000000,0.000000,0.000000}%
\pgfsetfillcolor{currentfill}%
\pgfsetlinewidth{0.602250pt}%
\definecolor{currentstroke}{rgb}{0.000000,0.000000,0.000000}%
\pgfsetstrokecolor{currentstroke}%
\pgfsetdash{}{0pt}%
\pgfsys@defobject{currentmarker}{\pgfqpoint{0.000000in}{-0.027778in}}{\pgfqpoint{0.000000in}{0.000000in}}{%
\pgfpathmoveto{\pgfqpoint{0.000000in}{0.000000in}}%
\pgfpathlineto{\pgfqpoint{0.000000in}{-0.027778in}}%
\pgfusepath{stroke,fill}%
}%
\begin{pgfscope}%
\pgfsys@transformshift{1.845411in}{0.521603in}%
\pgfsys@useobject{currentmarker}{}%
\end{pgfscope}%
\end{pgfscope}%
\begin{pgfscope}%
\pgfsetbuttcap%
\pgfsetroundjoin%
\definecolor{currentfill}{rgb}{0.000000,0.000000,0.000000}%
\pgfsetfillcolor{currentfill}%
\pgfsetlinewidth{0.602250pt}%
\definecolor{currentstroke}{rgb}{0.000000,0.000000,0.000000}%
\pgfsetstrokecolor{currentstroke}%
\pgfsetdash{}{0pt}%
\pgfsys@defobject{currentmarker}{\pgfqpoint{0.000000in}{-0.027778in}}{\pgfqpoint{0.000000in}{0.000000in}}{%
\pgfpathmoveto{\pgfqpoint{0.000000in}{0.000000in}}%
\pgfpathlineto{\pgfqpoint{0.000000in}{-0.027778in}}%
\pgfusepath{stroke,fill}%
}%
\begin{pgfscope}%
\pgfsys@transformshift{1.990498in}{0.521603in}%
\pgfsys@useobject{currentmarker}{}%
\end{pgfscope}%
\end{pgfscope}%
\begin{pgfscope}%
\pgfsetbuttcap%
\pgfsetroundjoin%
\definecolor{currentfill}{rgb}{0.000000,0.000000,0.000000}%
\pgfsetfillcolor{currentfill}%
\pgfsetlinewidth{0.602250pt}%
\definecolor{currentstroke}{rgb}{0.000000,0.000000,0.000000}%
\pgfsetstrokecolor{currentstroke}%
\pgfsetdash{}{0pt}%
\pgfsys@defobject{currentmarker}{\pgfqpoint{0.000000in}{-0.027778in}}{\pgfqpoint{0.000000in}{0.000000in}}{%
\pgfpathmoveto{\pgfqpoint{0.000000in}{0.000000in}}%
\pgfpathlineto{\pgfqpoint{0.000000in}{-0.027778in}}%
\pgfusepath{stroke,fill}%
}%
\begin{pgfscope}%
\pgfsys@transformshift{2.113168in}{0.521603in}%
\pgfsys@useobject{currentmarker}{}%
\end{pgfscope}%
\end{pgfscope}%
\begin{pgfscope}%
\pgfsetbuttcap%
\pgfsetroundjoin%
\definecolor{currentfill}{rgb}{0.000000,0.000000,0.000000}%
\pgfsetfillcolor{currentfill}%
\pgfsetlinewidth{0.602250pt}%
\definecolor{currentstroke}{rgb}{0.000000,0.000000,0.000000}%
\pgfsetstrokecolor{currentstroke}%
\pgfsetdash{}{0pt}%
\pgfsys@defobject{currentmarker}{\pgfqpoint{0.000000in}{-0.027778in}}{\pgfqpoint{0.000000in}{0.000000in}}{%
\pgfpathmoveto{\pgfqpoint{0.000000in}{0.000000in}}%
\pgfpathlineto{\pgfqpoint{0.000000in}{-0.027778in}}%
\pgfusepath{stroke,fill}%
}%
\begin{pgfscope}%
\pgfsys@transformshift{2.219429in}{0.521603in}%
\pgfsys@useobject{currentmarker}{}%
\end{pgfscope}%
\end{pgfscope}%
\begin{pgfscope}%
\pgfsetbuttcap%
\pgfsetroundjoin%
\definecolor{currentfill}{rgb}{0.000000,0.000000,0.000000}%
\pgfsetfillcolor{currentfill}%
\pgfsetlinewidth{0.602250pt}%
\definecolor{currentstroke}{rgb}{0.000000,0.000000,0.000000}%
\pgfsetstrokecolor{currentstroke}%
\pgfsetdash{}{0pt}%
\pgfsys@defobject{currentmarker}{\pgfqpoint{0.000000in}{-0.027778in}}{\pgfqpoint{0.000000in}{0.000000in}}{%
\pgfpathmoveto{\pgfqpoint{0.000000in}{0.000000in}}%
\pgfpathlineto{\pgfqpoint{0.000000in}{-0.027778in}}%
\pgfusepath{stroke,fill}%
}%
\begin{pgfscope}%
\pgfsys@transformshift{2.313157in}{0.521603in}%
\pgfsys@useobject{currentmarker}{}%
\end{pgfscope}%
\end{pgfscope}%
\begin{pgfscope}%
\pgfsetbuttcap%
\pgfsetroundjoin%
\definecolor{currentfill}{rgb}{0.000000,0.000000,0.000000}%
\pgfsetfillcolor{currentfill}%
\pgfsetlinewidth{0.602250pt}%
\definecolor{currentstroke}{rgb}{0.000000,0.000000,0.000000}%
\pgfsetstrokecolor{currentstroke}%
\pgfsetdash{}{0pt}%
\pgfsys@defobject{currentmarker}{\pgfqpoint{0.000000in}{-0.027778in}}{\pgfqpoint{0.000000in}{0.000000in}}{%
\pgfpathmoveto{\pgfqpoint{0.000000in}{0.000000in}}%
\pgfpathlineto{\pgfqpoint{0.000000in}{-0.027778in}}%
\pgfusepath{stroke,fill}%
}%
\begin{pgfscope}%
\pgfsys@transformshift{2.948590in}{0.521603in}%
\pgfsys@useobject{currentmarker}{}%
\end{pgfscope}%
\end{pgfscope}%
\begin{pgfscope}%
\pgfsetbuttcap%
\pgfsetroundjoin%
\definecolor{currentfill}{rgb}{0.000000,0.000000,0.000000}%
\pgfsetfillcolor{currentfill}%
\pgfsetlinewidth{0.602250pt}%
\definecolor{currentstroke}{rgb}{0.000000,0.000000,0.000000}%
\pgfsetstrokecolor{currentstroke}%
\pgfsetdash{}{0pt}%
\pgfsys@defobject{currentmarker}{\pgfqpoint{0.000000in}{-0.027778in}}{\pgfqpoint{0.000000in}{0.000000in}}{%
\pgfpathmoveto{\pgfqpoint{0.000000in}{0.000000in}}%
\pgfpathlineto{\pgfqpoint{0.000000in}{-0.027778in}}%
\pgfusepath{stroke,fill}%
}%
\begin{pgfscope}%
\pgfsys@transformshift{3.271249in}{0.521603in}%
\pgfsys@useobject{currentmarker}{}%
\end{pgfscope}%
\end{pgfscope}%
\begin{pgfscope}%
\pgfsetbuttcap%
\pgfsetroundjoin%
\definecolor{currentfill}{rgb}{0.000000,0.000000,0.000000}%
\pgfsetfillcolor{currentfill}%
\pgfsetlinewidth{0.602250pt}%
\definecolor{currentstroke}{rgb}{0.000000,0.000000,0.000000}%
\pgfsetstrokecolor{currentstroke}%
\pgfsetdash{}{0pt}%
\pgfsys@defobject{currentmarker}{\pgfqpoint{0.000000in}{-0.027778in}}{\pgfqpoint{0.000000in}{0.000000in}}{%
\pgfpathmoveto{\pgfqpoint{0.000000in}{0.000000in}}%
\pgfpathlineto{\pgfqpoint{0.000000in}{-0.027778in}}%
\pgfusepath{stroke,fill}%
}%
\begin{pgfscope}%
\pgfsys@transformshift{3.500180in}{0.521603in}%
\pgfsys@useobject{currentmarker}{}%
\end{pgfscope}%
\end{pgfscope}%
\begin{pgfscope}%
\pgfsetbuttcap%
\pgfsetroundjoin%
\definecolor{currentfill}{rgb}{0.000000,0.000000,0.000000}%
\pgfsetfillcolor{currentfill}%
\pgfsetlinewidth{0.602250pt}%
\definecolor{currentstroke}{rgb}{0.000000,0.000000,0.000000}%
\pgfsetstrokecolor{currentstroke}%
\pgfsetdash{}{0pt}%
\pgfsys@defobject{currentmarker}{\pgfqpoint{0.000000in}{-0.027778in}}{\pgfqpoint{0.000000in}{0.000000in}}{%
\pgfpathmoveto{\pgfqpoint{0.000000in}{0.000000in}}%
\pgfpathlineto{\pgfqpoint{0.000000in}{-0.027778in}}%
\pgfusepath{stroke,fill}%
}%
\begin{pgfscope}%
\pgfsys@transformshift{3.677752in}{0.521603in}%
\pgfsys@useobject{currentmarker}{}%
\end{pgfscope}%
\end{pgfscope}%
\begin{pgfscope}%
\pgfsetbuttcap%
\pgfsetroundjoin%
\definecolor{currentfill}{rgb}{0.000000,0.000000,0.000000}%
\pgfsetfillcolor{currentfill}%
\pgfsetlinewidth{0.602250pt}%
\definecolor{currentstroke}{rgb}{0.000000,0.000000,0.000000}%
\pgfsetstrokecolor{currentstroke}%
\pgfsetdash{}{0pt}%
\pgfsys@defobject{currentmarker}{\pgfqpoint{0.000000in}{-0.027778in}}{\pgfqpoint{0.000000in}{0.000000in}}{%
\pgfpathmoveto{\pgfqpoint{0.000000in}{0.000000in}}%
\pgfpathlineto{\pgfqpoint{0.000000in}{-0.027778in}}%
\pgfusepath{stroke,fill}%
}%
\begin{pgfscope}%
\pgfsys@transformshift{3.822839in}{0.521603in}%
\pgfsys@useobject{currentmarker}{}%
\end{pgfscope}%
\end{pgfscope}%
\begin{pgfscope}%
\pgfsetbuttcap%
\pgfsetroundjoin%
\definecolor{currentfill}{rgb}{0.000000,0.000000,0.000000}%
\pgfsetfillcolor{currentfill}%
\pgfsetlinewidth{0.602250pt}%
\definecolor{currentstroke}{rgb}{0.000000,0.000000,0.000000}%
\pgfsetstrokecolor{currentstroke}%
\pgfsetdash{}{0pt}%
\pgfsys@defobject{currentmarker}{\pgfqpoint{0.000000in}{-0.027778in}}{\pgfqpoint{0.000000in}{0.000000in}}{%
\pgfpathmoveto{\pgfqpoint{0.000000in}{0.000000in}}%
\pgfpathlineto{\pgfqpoint{0.000000in}{-0.027778in}}%
\pgfusepath{stroke,fill}%
}%
\begin{pgfscope}%
\pgfsys@transformshift{3.945508in}{0.521603in}%
\pgfsys@useobject{currentmarker}{}%
\end{pgfscope}%
\end{pgfscope}%
\begin{pgfscope}%
\pgfsetbuttcap%
\pgfsetroundjoin%
\definecolor{currentfill}{rgb}{0.000000,0.000000,0.000000}%
\pgfsetfillcolor{currentfill}%
\pgfsetlinewidth{0.602250pt}%
\definecolor{currentstroke}{rgb}{0.000000,0.000000,0.000000}%
\pgfsetstrokecolor{currentstroke}%
\pgfsetdash{}{0pt}%
\pgfsys@defobject{currentmarker}{\pgfqpoint{0.000000in}{-0.027778in}}{\pgfqpoint{0.000000in}{0.000000in}}{%
\pgfpathmoveto{\pgfqpoint{0.000000in}{0.000000in}}%
\pgfpathlineto{\pgfqpoint{0.000000in}{-0.027778in}}%
\pgfusepath{stroke,fill}%
}%
\begin{pgfscope}%
\pgfsys@transformshift{4.051769in}{0.521603in}%
\pgfsys@useobject{currentmarker}{}%
\end{pgfscope}%
\end{pgfscope}%
\begin{pgfscope}%
\pgfsetbuttcap%
\pgfsetroundjoin%
\definecolor{currentfill}{rgb}{0.000000,0.000000,0.000000}%
\pgfsetfillcolor{currentfill}%
\pgfsetlinewidth{0.602250pt}%
\definecolor{currentstroke}{rgb}{0.000000,0.000000,0.000000}%
\pgfsetstrokecolor{currentstroke}%
\pgfsetdash{}{0pt}%
\pgfsys@defobject{currentmarker}{\pgfqpoint{0.000000in}{-0.027778in}}{\pgfqpoint{0.000000in}{0.000000in}}{%
\pgfpathmoveto{\pgfqpoint{0.000000in}{0.000000in}}%
\pgfpathlineto{\pgfqpoint{0.000000in}{-0.027778in}}%
\pgfusepath{stroke,fill}%
}%
\begin{pgfscope}%
\pgfsys@transformshift{4.145498in}{0.521603in}%
\pgfsys@useobject{currentmarker}{}%
\end{pgfscope}%
\end{pgfscope}%
\begin{pgfscope}%
\pgftext[x=2.424660in,y=0.234413in,,top]{\rmfamily\fontsize{10.000000}{12.000000}\selectfont \(\displaystyle \mathbf{W}\mbox{e}\)}%
\end{pgfscope}%
\begin{pgfscope}%
\pgfsetbuttcap%
\pgfsetroundjoin%
\definecolor{currentfill}{rgb}{0.000000,0.000000,0.000000}%
\pgfsetfillcolor{currentfill}%
\pgfsetlinewidth{0.803000pt}%
\definecolor{currentstroke}{rgb}{0.000000,0.000000,0.000000}%
\pgfsetstrokecolor{currentstroke}%
\pgfsetdash{}{0pt}%
\pgfsys@defobject{currentmarker}{\pgfqpoint{-0.048611in}{0.000000in}}{\pgfqpoint{0.000000in}{0.000000in}}{%
\pgfpathmoveto{\pgfqpoint{0.000000in}{0.000000in}}%
\pgfpathlineto{\pgfqpoint{-0.048611in}{0.000000in}}%
\pgfusepath{stroke,fill}%
}%
\begin{pgfscope}%
\pgfsys@transformshift{0.564660in}{1.024740in}%
\pgfsys@useobject{currentmarker}{}%
\end{pgfscope}%
\end{pgfscope}%
\begin{pgfscope}%
\pgftext[x=0.289968in,y=0.971978in,left,base]{\rmfamily\fontsize{10.000000}{12.000000}\selectfont \(\displaystyle 0.4\)}%
\end{pgfscope}%
\begin{pgfscope}%
\pgfsetbuttcap%
\pgfsetroundjoin%
\definecolor{currentfill}{rgb}{0.000000,0.000000,0.000000}%
\pgfsetfillcolor{currentfill}%
\pgfsetlinewidth{0.803000pt}%
\definecolor{currentstroke}{rgb}{0.000000,0.000000,0.000000}%
\pgfsetstrokecolor{currentstroke}%
\pgfsetdash{}{0pt}%
\pgfsys@defobject{currentmarker}{\pgfqpoint{-0.048611in}{0.000000in}}{\pgfqpoint{0.000000in}{0.000000in}}{%
\pgfpathmoveto{\pgfqpoint{0.000000in}{0.000000in}}%
\pgfpathlineto{\pgfqpoint{-0.048611in}{0.000000in}}%
\pgfusepath{stroke,fill}%
}%
\begin{pgfscope}%
\pgfsys@transformshift{0.564660in}{1.643093in}%
\pgfsys@useobject{currentmarker}{}%
\end{pgfscope}%
\end{pgfscope}%
\begin{pgfscope}%
\pgftext[x=0.289968in,y=1.590331in,left,base]{\rmfamily\fontsize{10.000000}{12.000000}\selectfont \(\displaystyle 0.5\)}%
\end{pgfscope}%
\begin{pgfscope}%
\pgfsetbuttcap%
\pgfsetroundjoin%
\definecolor{currentfill}{rgb}{0.000000,0.000000,0.000000}%
\pgfsetfillcolor{currentfill}%
\pgfsetlinewidth{0.803000pt}%
\definecolor{currentstroke}{rgb}{0.000000,0.000000,0.000000}%
\pgfsetstrokecolor{currentstroke}%
\pgfsetdash{}{0pt}%
\pgfsys@defobject{currentmarker}{\pgfqpoint{-0.048611in}{0.000000in}}{\pgfqpoint{0.000000in}{0.000000in}}{%
\pgfpathmoveto{\pgfqpoint{0.000000in}{0.000000in}}%
\pgfpathlineto{\pgfqpoint{-0.048611in}{0.000000in}}%
\pgfusepath{stroke,fill}%
}%
\begin{pgfscope}%
\pgfsys@transformshift{0.564660in}{2.261445in}%
\pgfsys@useobject{currentmarker}{}%
\end{pgfscope}%
\end{pgfscope}%
\begin{pgfscope}%
\pgftext[x=0.289968in,y=2.208684in,left,base]{\rmfamily\fontsize{10.000000}{12.000000}\selectfont \(\displaystyle 0.6\)}%
\end{pgfscope}%
\begin{pgfscope}%
\pgfsetbuttcap%
\pgfsetroundjoin%
\definecolor{currentfill}{rgb}{0.000000,0.000000,0.000000}%
\pgfsetfillcolor{currentfill}%
\pgfsetlinewidth{0.803000pt}%
\definecolor{currentstroke}{rgb}{0.000000,0.000000,0.000000}%
\pgfsetstrokecolor{currentstroke}%
\pgfsetdash{}{0pt}%
\pgfsys@defobject{currentmarker}{\pgfqpoint{-0.048611in}{0.000000in}}{\pgfqpoint{0.000000in}{0.000000in}}{%
\pgfpathmoveto{\pgfqpoint{0.000000in}{0.000000in}}%
\pgfpathlineto{\pgfqpoint{-0.048611in}{0.000000in}}%
\pgfusepath{stroke,fill}%
}%
\begin{pgfscope}%
\pgfsys@transformshift{0.564660in}{2.879798in}%
\pgfsys@useobject{currentmarker}{}%
\end{pgfscope}%
\end{pgfscope}%
\begin{pgfscope}%
\pgftext[x=0.289968in,y=2.827036in,left,base]{\rmfamily\fontsize{10.000000}{12.000000}\selectfont \(\displaystyle 0.7\)}%
\end{pgfscope}%
\begin{pgfscope}%
\pgfsetbuttcap%
\pgfsetroundjoin%
\definecolor{currentfill}{rgb}{0.000000,0.000000,0.000000}%
\pgfsetfillcolor{currentfill}%
\pgfsetlinewidth{0.803000pt}%
\definecolor{currentstroke}{rgb}{0.000000,0.000000,0.000000}%
\pgfsetstrokecolor{currentstroke}%
\pgfsetdash{}{0pt}%
\pgfsys@defobject{currentmarker}{\pgfqpoint{-0.048611in}{0.000000in}}{\pgfqpoint{0.000000in}{0.000000in}}{%
\pgfpathmoveto{\pgfqpoint{0.000000in}{0.000000in}}%
\pgfpathlineto{\pgfqpoint{-0.048611in}{0.000000in}}%
\pgfusepath{stroke,fill}%
}%
\begin{pgfscope}%
\pgfsys@transformshift{0.564660in}{3.498150in}%
\pgfsys@useobject{currentmarker}{}%
\end{pgfscope}%
\end{pgfscope}%
\begin{pgfscope}%
\pgftext[x=0.289968in,y=3.445389in,left,base]{\rmfamily\fontsize{10.000000}{12.000000}\selectfont \(\displaystyle 0.8\)}%
\end{pgfscope}%
\begin{pgfscope}%
\pgftext[x=0.234413in,y=2.031603in,,bottom,rotate=90.000000]{\rmfamily\fontsize{10.000000}{12.000000}\selectfont \(\displaystyle C_r\)}%
\end{pgfscope}%
\begin{pgfscope}%
\pgfsetrectcap%
\pgfsetmiterjoin%
\pgfsetlinewidth{0.803000pt}%
\definecolor{currentstroke}{rgb}{0.000000,0.000000,0.000000}%
\pgfsetstrokecolor{currentstroke}%
\pgfsetdash{}{0pt}%
\pgfpathmoveto{\pgfqpoint{0.564660in}{0.521603in}}%
\pgfpathlineto{\pgfqpoint{0.564660in}{3.541603in}}%
\pgfusepath{stroke}%
\end{pgfscope}%
\begin{pgfscope}%
\pgfsetrectcap%
\pgfsetmiterjoin%
\pgfsetlinewidth{0.803000pt}%
\definecolor{currentstroke}{rgb}{0.000000,0.000000,0.000000}%
\pgfsetstrokecolor{currentstroke}%
\pgfsetdash{}{0pt}%
\pgfpathmoveto{\pgfqpoint{4.284660in}{0.521603in}}%
\pgfpathlineto{\pgfqpoint{4.284660in}{3.541603in}}%
\pgfusepath{stroke}%
\end{pgfscope}%
\begin{pgfscope}%
\pgfsetrectcap%
\pgfsetmiterjoin%
\pgfsetlinewidth{0.803000pt}%
\definecolor{currentstroke}{rgb}{0.000000,0.000000,0.000000}%
\pgfsetstrokecolor{currentstroke}%
\pgfsetdash{}{0pt}%
\pgfpathmoveto{\pgfqpoint{0.564660in}{0.521603in}}%
\pgfpathlineto{\pgfqpoint{4.284660in}{0.521603in}}%
\pgfusepath{stroke}%
\end{pgfscope}%
\begin{pgfscope}%
\pgfsetrectcap%
\pgfsetmiterjoin%
\pgfsetlinewidth{0.803000pt}%
\definecolor{currentstroke}{rgb}{0.000000,0.000000,0.000000}%
\pgfsetstrokecolor{currentstroke}%
\pgfsetdash{}{0pt}%
\pgfpathmoveto{\pgfqpoint{0.564660in}{3.541603in}}%
\pgfpathlineto{\pgfqpoint{4.284660in}{3.541603in}}%
\pgfusepath{stroke}%
\end{pgfscope}%
\begin{pgfscope}%
\pgfsetbuttcap%
\pgfsetmiterjoin%
\definecolor{currentfill}{rgb}{1.000000,1.000000,1.000000}%
\pgfsetfillcolor{currentfill}%
\pgfsetfillopacity{0.700000}%
\pgfsetlinewidth{1.003750pt}%
\definecolor{currentstroke}{rgb}{0.500000,0.500000,0.500000}%
\pgfsetstrokecolor{currentstroke}%
\pgfsetstrokeopacity{0.700000}%
\pgfsetdash{}{0pt}%
\pgfpathmoveto{\pgfqpoint{0.887319in}{3.107570in}}%
\pgfpathlineto{\pgfqpoint{1.534885in}{3.107570in}}%
\pgfpathquadraticcurveto{\pgfqpoint{1.576552in}{3.107570in}}{\pgfqpoint{1.576552in}{3.149237in}}%
\pgfpathlineto{\pgfqpoint{1.576552in}{3.294497in}}%
\pgfpathquadraticcurveto{\pgfqpoint{1.576552in}{3.336164in}}{\pgfqpoint{1.534885in}{3.336164in}}%
\pgfpathlineto{\pgfqpoint{0.887319in}{3.336164in}}%
\pgfpathquadraticcurveto{\pgfqpoint{0.845653in}{3.336164in}}{\pgfqpoint{0.845653in}{3.294497in}}%
\pgfpathlineto{\pgfqpoint{0.845653in}{3.149237in}}%
\pgfpathquadraticcurveto{\pgfqpoint{0.845653in}{3.107570in}}{\pgfqpoint{0.887319in}{3.107570in}}%
\pgfpathclose%
\pgfusepath{stroke,fill}%
\end{pgfscope}%
\begin{pgfscope}%
\pgftext[x=0.887319in,y=3.188974in,left,base]{\rmfamily\fontsize{10.000000}{12.000000}\selectfont \(\displaystyle \mathbf{O}\mbox{h}_{\mu} = 2.2\)}%
\end{pgfscope}%
\begin{pgfscope}%
\pgfpathrectangle{\pgfqpoint{4.517160in}{0.521603in}}{\pgfqpoint{0.151000in}{3.020000in}} %
\pgfusepath{clip}%
\pgfsetbuttcap%
\pgfsetmiterjoin%
\definecolor{currentfill}{rgb}{1.000000,1.000000,1.000000}%
\pgfsetfillcolor{currentfill}%
\pgfsetlinewidth{0.010037pt}%
\definecolor{currentstroke}{rgb}{1.000000,1.000000,1.000000}%
\pgfsetstrokecolor{currentstroke}%
\pgfsetdash{}{0pt}%
\pgfpathmoveto{\pgfqpoint{4.517160in}{0.521603in}}%
\pgfpathlineto{\pgfqpoint{4.517160in}{0.533400in}}%
\pgfpathlineto{\pgfqpoint{4.517160in}{3.529806in}}%
\pgfpathlineto{\pgfqpoint{4.517160in}{3.541603in}}%
\pgfpathlineto{\pgfqpoint{4.668160in}{3.541603in}}%
\pgfpathlineto{\pgfqpoint{4.668160in}{3.529806in}}%
\pgfpathlineto{\pgfqpoint{4.668160in}{0.533400in}}%
\pgfpathlineto{\pgfqpoint{4.668160in}{0.521603in}}%
\pgfpathclose%
\pgfusepath{stroke,fill}%
\end{pgfscope}%
\begin{pgfscope}%
\pgfsys@transformshift{4.520000in}{0.526603in}%
\pgftext[left,bottom]{\pgfimage[interpolate=true,width=0.150000in,height=3.020000in]{restitution-img0.png}}%
\end{pgfscope}%
\begin{pgfscope}%
\pgfsetbuttcap%
\pgfsetroundjoin%
\definecolor{currentfill}{rgb}{0.000000,0.000000,0.000000}%
\pgfsetfillcolor{currentfill}%
\pgfsetlinewidth{0.803000pt}%
\definecolor{currentstroke}{rgb}{0.000000,0.000000,0.000000}%
\pgfsetstrokecolor{currentstroke}%
\pgfsetdash{}{0pt}%
\pgfsys@defobject{currentmarker}{\pgfqpoint{0.000000in}{0.000000in}}{\pgfqpoint{0.048611in}{0.000000in}}{%
\pgfpathmoveto{\pgfqpoint{0.000000in}{0.000000in}}%
\pgfpathlineto{\pgfqpoint{0.048611in}{0.000000in}}%
\pgfusepath{stroke,fill}%
}%
\begin{pgfscope}%
\pgfsys@transformshift{4.668160in}{0.925593in}%
\pgfsys@useobject{currentmarker}{}%
\end{pgfscope}%
\end{pgfscope}%
\begin{pgfscope}%
\pgftext[x=4.765383in,y=0.872831in,left,base]{\rmfamily\fontsize{10.000000}{12.000000}\selectfont \(\displaystyle 0.2\)}%
\end{pgfscope}%
\begin{pgfscope}%
\pgfsetbuttcap%
\pgfsetroundjoin%
\definecolor{currentfill}{rgb}{0.000000,0.000000,0.000000}%
\pgfsetfillcolor{currentfill}%
\pgfsetlinewidth{0.803000pt}%
\definecolor{currentstroke}{rgb}{0.000000,0.000000,0.000000}%
\pgfsetstrokecolor{currentstroke}%
\pgfsetdash{}{0pt}%
\pgfsys@defobject{currentmarker}{\pgfqpoint{0.000000in}{0.000000in}}{\pgfqpoint{0.048611in}{0.000000in}}{%
\pgfpathmoveto{\pgfqpoint{0.000000in}{0.000000in}}%
\pgfpathlineto{\pgfqpoint{0.048611in}{0.000000in}}%
\pgfusepath{stroke,fill}%
}%
\begin{pgfscope}%
\pgfsys@transformshift{4.668160in}{1.540182in}%
\pgfsys@useobject{currentmarker}{}%
\end{pgfscope}%
\end{pgfscope}%
\begin{pgfscope}%
\pgftext[x=4.765383in,y=1.487421in,left,base]{\rmfamily\fontsize{10.000000}{12.000000}\selectfont \(\displaystyle 0.4\)}%
\end{pgfscope}%
\begin{pgfscope}%
\pgfsetbuttcap%
\pgfsetroundjoin%
\definecolor{currentfill}{rgb}{0.000000,0.000000,0.000000}%
\pgfsetfillcolor{currentfill}%
\pgfsetlinewidth{0.803000pt}%
\definecolor{currentstroke}{rgb}{0.000000,0.000000,0.000000}%
\pgfsetstrokecolor{currentstroke}%
\pgfsetdash{}{0pt}%
\pgfsys@defobject{currentmarker}{\pgfqpoint{0.000000in}{0.000000in}}{\pgfqpoint{0.048611in}{0.000000in}}{%
\pgfpathmoveto{\pgfqpoint{0.000000in}{0.000000in}}%
\pgfpathlineto{\pgfqpoint{0.048611in}{0.000000in}}%
\pgfusepath{stroke,fill}%
}%
\begin{pgfscope}%
\pgfsys@transformshift{4.668160in}{2.154772in}%
\pgfsys@useobject{currentmarker}{}%
\end{pgfscope}%
\end{pgfscope}%
\begin{pgfscope}%
\pgftext[x=4.765383in,y=2.102010in,left,base]{\rmfamily\fontsize{10.000000}{12.000000}\selectfont \(\displaystyle 0.6\)}%
\end{pgfscope}%
\begin{pgfscope}%
\pgfsetbuttcap%
\pgfsetroundjoin%
\definecolor{currentfill}{rgb}{0.000000,0.000000,0.000000}%
\pgfsetfillcolor{currentfill}%
\pgfsetlinewidth{0.803000pt}%
\definecolor{currentstroke}{rgb}{0.000000,0.000000,0.000000}%
\pgfsetstrokecolor{currentstroke}%
\pgfsetdash{}{0pt}%
\pgfsys@defobject{currentmarker}{\pgfqpoint{0.000000in}{0.000000in}}{\pgfqpoint{0.048611in}{0.000000in}}{%
\pgfpathmoveto{\pgfqpoint{0.000000in}{0.000000in}}%
\pgfpathlineto{\pgfqpoint{0.048611in}{0.000000in}}%
\pgfusepath{stroke,fill}%
}%
\begin{pgfscope}%
\pgfsys@transformshift{4.668160in}{2.769361in}%
\pgfsys@useobject{currentmarker}{}%
\end{pgfscope}%
\end{pgfscope}%
\begin{pgfscope}%
\pgftext[x=4.765383in,y=2.716599in,left,base]{\rmfamily\fontsize{10.000000}{12.000000}\selectfont \(\displaystyle 0.8\)}%
\end{pgfscope}%
\begin{pgfscope}%
\pgfsetbuttcap%
\pgfsetroundjoin%
\definecolor{currentfill}{rgb}{0.000000,0.000000,0.000000}%
\pgfsetfillcolor{currentfill}%
\pgfsetlinewidth{0.803000pt}%
\definecolor{currentstroke}{rgb}{0.000000,0.000000,0.000000}%
\pgfsetstrokecolor{currentstroke}%
\pgfsetdash{}{0pt}%
\pgfsys@defobject{currentmarker}{\pgfqpoint{0.000000in}{0.000000in}}{\pgfqpoint{0.048611in}{0.000000in}}{%
\pgfpathmoveto{\pgfqpoint{0.000000in}{0.000000in}}%
\pgfpathlineto{\pgfqpoint{0.048611in}{0.000000in}}%
\pgfusepath{stroke,fill}%
}%
\begin{pgfscope}%
\pgfsys@transformshift{4.668160in}{3.383950in}%
\pgfsys@useobject{currentmarker}{}%
\end{pgfscope}%
\end{pgfscope}%
\begin{pgfscope}%
\pgftext[x=4.765383in,y=3.331189in,left,base]{\rmfamily\fontsize{10.000000}{12.000000}\selectfont \(\displaystyle 1.0\)}%
\end{pgfscope}%
\begin{pgfscope}%
\pgftext[x=4.998408in,y=2.031603in,,top,rotate=90.000000]{\rmfamily\fontsize{10.000000}{12.000000}\selectfont \(\displaystyle \mathrm{\mathit{Bo_e}} \equiv \frac{\epsilon E_0^2 R_0}{\gamma}\)}%
\end{pgfscope}%
\begin{pgfscope}%
\pgfsetbuttcap%
\pgfsetmiterjoin%
\pgfsetlinewidth{0.803000pt}%
\definecolor{currentstroke}{rgb}{0.000000,0.000000,0.000000}%
\pgfsetstrokecolor{currentstroke}%
\pgfsetdash{}{0pt}%
\pgfpathmoveto{\pgfqpoint{4.517160in}{0.521603in}}%
\pgfpathlineto{\pgfqpoint{4.517160in}{0.533400in}}%
\pgfpathlineto{\pgfqpoint{4.517160in}{3.529806in}}%
\pgfpathlineto{\pgfqpoint{4.517160in}{3.541603in}}%
\pgfpathlineto{\pgfqpoint{4.668160in}{3.541603in}}%
\pgfpathlineto{\pgfqpoint{4.668160in}{3.529806in}}%
\pgfpathlineto{\pgfqpoint{4.668160in}{0.533400in}}%
\pgfpathlineto{\pgfqpoint{4.668160in}{0.521603in}}%
\pgfpathclose%
\pgfusepath{stroke}%
\end{pgfscope}%
\end{pgfpicture}%
\makeatother%
\endgroup%

    \caption{A simple EMA plot.\label{fig:restitution}}
\end{figure}

\newpage
\appendix

\section{\\Droplet Charge} \label{sec.drop_charge}
\subsection*{Parallel Plate Method}
Since, by the earlier scaling, we presuppose the source of the droplet bouncing behavior to be primarily Coulombic in origin (as opposed to dielectrophoretic), the droplet must have some free charge in addition to the charge induced by the electric field. To measure this charge concurrent methodologies were used. The first method of determining the droplet free charge is observation of the deflection of the droplets in the region of a known uniform field in a fashion inspired by Millikan's famous experiment to determine the fundamental charge of the electron [ref].

Droplets were jumped in freefall from a sandpaper superhydrophobic surface placed between the plates of a parallel plate capacitor of known uniform electric field. The surface was first deionized using a balanced \emph{Ptec} IN5120 DC air-ionizer to remove the effect of surface charge of the superhydrophobic surface from perturbing the otherwise uniform field set up between the parallel plates. A schematic of the parallel plate drop apparatus is shown in Figure \ref{fig:millikan}. Since the droplet initial velocity $U_0$ is parallel to the electric field, the droplets are inertial in the direction of the electric force, and neglecting the effect of image charges mirrored across the conductors, we can determine the magnitude of the droplet charge by a balance of Coulombic force and inertia given by the equation of motion

\[ \frac{d^2y}{dt^2} = q\mathbf{E}. \]

Since the drag is negligible in the inertial limit we can find the charge $q$ by fitting a second-order least squares polynomial to the measured droplet positions, equating the $t^2$ term to the constant acceleration, and dividing by the known, constant magnitude of the electric field. From a survey of literature we suppose the droplet charge, if they are indeed charged by contact with PTFE, to be some function of the droplet volume and the residence time on the superhydrophobic surface. However, sweeping though droplet volumes over a series of drop tower experiments we find little correlation between droplet volume  and free droplet charge. This is shown in figure \ref{fig:drop_charge}.

A 200-880 VAC source with a full wave bridge rectifier circuit was prototyped on perf-board for initial experiments to measure droplet charge. The circuit was analyzed on an laboratory oscilloscope to verify that the AC component of the signal was appropriately small (13 mV at 35 kHz). Current was determined to be a relatively low 80 $\mu$A. The high-voltage source terminals were led to two parallel polished 150x150 mm aluminum plate electrodes. The electrodes were mounted on an insulated 80/20 extruded aluminum rail for ease of adjustment. All droplet charge experiments were conducted with an electrode spacing of 28.30 mm. With this spacing the calibrated electric field between the plates was $\mathbf{E} \approx 35$kV/m. The electrodes were electrically isolated from the drop rig by two alternating layers of 4 mm thick laser cut acrylic sheet and Kapton tape. Potential across the plates was measured periodically with a load-impedance corrected multimeter to account for battery depletion. The typical capacitor rise time of the plates was measured to be 1.4 s, thus to make the most economical use of the brief window a micro-gravity a weighted switch was set by hand prior to the drop to close the high-voltage circuit, but which passively safed the system at the resumption of 1-g conditions in the tower. The drop apparatus is shown in Fig.


%\begin{figure}
%  \centerline{\includegraphics[height=7cm,width=13cm]{modes.eps}}
 % \centerline{\includegraphics[height=7cm,width=7cm]{e88_1_thumb.jpg}}
  %\caption{Parallel plate experimental apparatus.}
%\label{fig:millikan}
%\end{figure} 

%\begin{figure}
%  \centerline{\includegraphics[height=7cm,width=13cm]{modes.eps}}
 % \centerline{\includegraphics[height=7cm,width=7cm]{e88_1_thumb.jpg}}
  %\caption{Parallel plates: Droplet free charge vs volume.}
%\label{fig:drop_charge}
%\end{figure}

A brief screening experiment was conducted which alternated the polarity of the field by switching the positive and negative terminal leads between plates. Qualitative observations of droplet electrode preference seem to indicate that the assumption of small polarization stress was well founded. Following this a orthogonal array $3^2$ factorial design experiment with two replicates was conducted to test the effect of varying droplet volume and surface stay time on free charge at the time of jumping. It was hypothesized in accordance with previous studies \cite{Choi:2013dg}, that free charge would increase for levels of both factors. The results of the factorial experiment are presented in Fig. \ref{fig:chart1}. ANOVA analysis in \emph{R} of the linear multiple regression model for the data set indicates that neither droplet volume ($p=0.105$), nor surface stay time ($p=0.358$) is significant at the 95\% confidence level. The overall model F-statistics (2.177 in 2 and 13 degrees of freedom), and coefficients of determination ($r^2 = 0.2509$) indicate that the linear model neither fits the data particularly well, nor does it offer an improvement over the mean model. The mean charge was determined to be positive $2.3 \cdot 10^{-11}$ C, with a standard deviation of $1.8 \cdot 10^{-11}$ C.

\end{document}

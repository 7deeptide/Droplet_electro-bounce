\documentclass[10pt,a4paper]{article}
\usepackage[utf8]{inputenc}
\usepackage[english]{babel}
\usepackage{amsmath}
\usepackage{amsfonts}
\usepackage{amssymb}
\usepackage{graphicx}
\usepackage{color,soul}
\usepackage{listings}
\author{Erin Schmidt}
\begin{document}
\section{Electrets}
\begin{itemize}
\item The method for determination of the evolution of the surface charge density for low conductivity polymers is described in \hl{[Davies, 1967]}. Surface charge in injected into the surface of the dielectric mounted on top of an grounded electrode. The charge decay is measured with a calibrated probe at periodic intervals.
\item Additive stacking of electrets has been used in electret based vibrational energy harvesters \hl{[Stacking Electrets for Electrostatic Vibration Energy Harvesters, Wada et al. 2012]} and in water desalinization \hl{[ref Application of electret technology to low cost desalination, R.C. Amme, 2002]}
\item There are several charge decay mechanisms: internal ones, such as Ohmic resistence, and external ones such as compensation by envonmental ionic species. The relative magnitudes of these charge transport mechanisms, and therfore the stability of the electret varies drastically depending on its initial surface potential, material properties, environment, and charging method. In the case of unshielded electrets compensation by atmospheric ions is significant \hl{[E.W. Anderson et al, 1973]}. Because envornmental convection will tend to  maintain a gradient of these ions, sealing an electret in a container from the atmosphere will effectively halt this decay mechanism. Atmospheric humidity and water droplet condensation also significantly increase charge decay (presumably by reducing the surface resistence) \hl{[Haenen, 1975]}.
\item According to \hl{[Sessler 1987]} an electret is a dielectric material with ``quasi-permanent'' electric charge in the sense that the characteristic decay period of the electret is much greater than a given period of interest.
\item Electret charge may be `true' charge in the form of surface or space charges, or polarization charges (such as bound charges). If the electret is not screened by a conductor then it produces an external electric field if the polarization and real charges do not uniformly compates each other throughout the volume of the electret. For this reason electrets are though of as electrostatic analouges to permanent magnets (and the name \emph{electret} itself is a portmantau to the effect conjured by Heaviside in 1892 \hl{[ref O. Heaviside, 1982]}). Typical commerical electrets are Teflon type polytetraflouroethylene (PTFE) polymer films on the order of 10-50 $\mu$m thick with the charge being primarily real surface charge. Electrets have a plethora of applications, but most germanely they have been used in Electro-Wetting On Dielectric (EWOD) devices for low-voltage manipulation of small droplets \hl{[Wu, 2010]}. 
\item Real charge electrets can be produced by contact electrification, injection or deposition) of charge carriers by discharge or electron beam, ionizing radiation, or by frictional triboelectrification. Dipolar electrets by contrast are produced by a combination of polarization at elevated temperatures in a strong external electric field, followed by an annealing process.
\item Effective surface charge densities are limited to the material dielectric strengths due to internal dielectric breakdown phenomenon (this typically occurs before external breakdown or Paschen breakdown).
\item We use an isothermal electret formation process using a variation of the widely applied corona-charging technique. The typical corona-charging technique uses strong inhomogenous DC electric fields to produce discharge in air at ambient conditions; the dielectric substrate is atop a grounded electrode, and there is a screening potential electrode intermediately positioned to control (the the surface potential of the dielectric will tend to saturate at this grid potential if the material is not space-charge current limited). The corona field is usually produced by pin-shaped electrode. In air the most common charge carriers thus produced are $\mbox{CO}_3^-$ ions. This approach is known to generally produce samples with fairly uniform surface charge densities.
\item Some work \hl{[ref J. van Turnhout, 1975]} showed using a Thermally Stimulated Current (TCS) measurements that in 4.8 mm thick polymethyl methacrylate (PMMA) polarization of the dielectric is non-uniform, due to real space-charge mostly ($\sim$90\%) residing in a thin (0.1-0.2 mm) layer near the free-surface of the sample.
\item \hl{Notes on our version of the corona-discharge process, figure, plot of charge decay}.
\item We plainly see a cross-over effect in the decay of the surface potential in our electret samples, whereby the samples charged to higher initial surface potentials decayed faster and reached the lower overall final potentials. This is a well known effect in polyethylenes charged by corona \hl{[Ferreira et al. 1992]}
\end{itemize}
\section{Surface charge density}
Since the laser-etched PMMA superhydrophobic surface is a dielectric material there are some additional considerations for determining the surface charge density. Firstly, for an insulated conductor $A$ with charge $q$, the charge distribution in equilibrium will be such that the field on the interior of the conductor is zero, with field lines being normal to the surface and the integral of the field strength $\mathbf{E}$ from any point $P$ in or on the conductor to a ground point $G$ is a constant given by
\[ V = \int^G_P \mathbf{E} \cdot da, \]
where $V$ is the voltage or potential of the conductor (or alternatively, the surface voltage potential). The voltage $V$ and the charge $q$ are proportional and $q$ is usually given by
\[q = CV, \]
where $C$ is the capacitance of the insulated conductor and is determined by the conductor's size and shape, and its placement relative to the conductors and ground. However, the case of a charged insulator departs significantly from that of the insulated conductor. Here we have the possibility of the density and net polarity of the charge varying as a function of position about the surface. The field of the interior may not be zero, and the field lines are not necessarily normal to the the surface. The integral of the field strength from a point on or in the insulator to ground is usually different from point to point. In general the surface voltages will vary from point to point on an insulator (including point on the interior of the insulator. Therefore it is not possible to uniquely define the voltage of a charged insulator, except in certain contrived cases (notably the the potential of a uniformly charged sphere positioned infinitely far away from any conductors). It follows that there is no definition for the capacitance of an insulator as well. Though insulative sheets may store charge and discharge in the manner of capacitors, this discharge is never particularly complete.

There are two cases where we can make meaningful quantitative measurements of state variables for charged insulators: we can measure the electric field arising from the charge, and possible the total charge for uniformly charged free insulative sheets, and uniformly charged insulative sheets backed by a grounded conductor. To do this we can use field meters and non-contacting voltmeters. If the field strength indicated on the meter is $\mathbf{E}$, the charge density $\sigma$ on the part of the insulator in the front should be $\sigma = \epsilon_0 \mathbf{E}$. If a non-contacting voltmeter is placed a at a distance $d$ from the sheet, then the surface voltage $V_s$ indicated on the meter is given by 
\[V_s = d \mathbf{E} = d \frac{\sigma}{\epsilon_0}. \]


An ideal approach to determining surface charge on a dielectric surface is to screen perturbing effects of external electric fields. This is partly accomplished by grounding the fieldmeter, and by placing the dielectric sample on a grounded conductive plate backing. In this case the surface charge density is determined from

\[ \sigma = \frac{V_s \epsilon}{l}, \]

where $l$ and $\epsilon$ are the absolute permittivity and thickness of the dielectric surface respectively. The measured surface voltage is a function of position away from the charged dielectric. In most cases this function is relatively constant at a distance about 1-2 cm away from the surface (there is some measurement error in surface voltage due to small mispositioning of the electrostatic fieldmeter, say by $\pm$1 mm) and add it to the precision error of measurement in $V_s$. The measured surface voltage can also be compared to the analytically derived surface voltage determined from the analytical expression for the electric field from above, and to those determined from finite element solutions to poisson's equation on the domain in question. [Notes about \emph{Agros2D}] The surface voltage is shown schematically as a function of the measurement distance in Figure \ref{fig:surface_voltage} for a sample charge density $\sigma = $.

A further consideration is the possibility of the change in total charge during a typical experimental timescale. If we consider the drop rig to be a ground (which seems reasonable given that the rig is isolated from true ground, but is at some reference voltage with respect to the surface charges on the dielectric, it also has an abundance of free charge carriers, that it, it is conductive), then there will be both bulk and surface decay of the charge on the dielectric. The evolution of the charge can be approximated by
\[ \sigma = \sigma_0 e^{\frac{-t}{\epsilon \rho}}, \]
where $\sigma_0$ is the initial surface charge density, and $\rho$ is the bulk resistivity (which can also be reframed in terms of conductivity by $\rho = 1/\gamma$, where $\gamma$ is the conductivity). For an example case of a surface with an initial surface charge density $\sigma = 2.4 \cdot (10^{-6})$ $C/m^2$, relative permittivity $\epsilon = 3.5$ and bulk resistivity $\rho = 1.6 \cdot (10^{16})$ $\Omega/cm$ such as with the case of 1/4'' continuous cast acrylic sheet, then the time constant $\tau = \epsilon \rho$ is approximately 5000 s, which is a great deal longer than the typical time period for of a drop tower experiment. \hl{Add specific notes on charge measurement procedure}.
\end{document}
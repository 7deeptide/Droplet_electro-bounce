\documentclass[10pt,a4paper]{article}
\usepackage[utf8]{inputenc}
%\usepackage{fontspec} % This line only for XeLaTeX and LuaLaTeX
\usepackage{pgfplots}
\usepackage{pgf}
\usepackage[english]{babel}
\usepackage{amsmath}
\usepackage{amsfonts}
\usepackage{amssymb}
\usepackage{graphicx}
\graphicspath{ {../figures/} }
%\usepackage{svg}
\usepackage{verbatim}
\usepackage{color,soul}
\usepackage{listings}
\usepackage{setspace}
\author{Erin Schmidt}

\newlength\figureheight
\newlength\figurewidth
\setlength\figureheight{7cm}
\setlength\figurewidth{10cm}

\let\pgfimageWithoutPath\pgfimage 
\renewcommand{\pgfimage}[2][]{\pgfimageWithoutPath[#1]{../figures/#2}}

\usepackage[
backend=biber,
style=phys,
sorting=none
]{biblatex}
\addbibresource{thesis.bib}

\begin{document}

\doublespacing
\section{Charge Estimates}
We found the distribution of mostly likely experimental net charges for a population of the droplets jumped in low-gravity. A covariance plot of the model variables is shown in Figure \ref{fig:scatter}. The multicollinear dependence of charge on droplet surface area, $A$, and the characteristic electric field, $E_0$, is evident. Assuming the main effect is the interaction between charge and electric field, a Robust Least Squares model fit $q \sim kAE_0$ (using the Python \verb|statsmodels.formula.api.RLM| function), with the non-linear transformation $A = V_d^{2/3}$, found that $k=5.01 \times 10^{-11} \pm  2.85 \times 10^{-11}$ with $R^2 = 0.946$. This model uses Huber's T norm, median absolute scaling, and H1 covariance estimation. A contour plot showing the estimated droplet free charge as a function of $V_d$ and $\varphi_s$ is shown in Figure \ref{fig:charge}.
\begin{figure}[htb]
    \centering
    \resizebox{12cm}{!}{%% Creator: Matplotlib, PGF backend
%%
%% To include the figure in your LaTeX document, write
%%   \input{<filename>.pgf}
%%
%% Make sure the required packages are loaded in your preamble
%%   \usepackage{pgf}
%%
%% Figures using additional raster images can only be included by \input if
%% they are in the same directory as the main LaTeX file. For loading figures
%% from other directories you can use the `import` package
%%   \usepackage{import}
%% and then include the figures with
%%   \import{<path to file>}{<filename>.pgf}
%%
%% Matplotlib used the following preamble
%%   \usepackage{fontspec}
%%   \setmainfont{DejaVu Serif}
%%   \setsansfont{DejaVu Sans}
%%   \setmonofont{DejaVu Sans Mono}
%%
\begingroup%
\makeatletter%
\begin{pgfpicture}%
\pgfpathrectangle{\pgfpointorigin}{\pgfqpoint{5.674225in}{4.068832in}}%
\pgfusepath{use as bounding box, clip}%
\begin{pgfscope}%
\pgfsetbuttcap%
\pgfsetmiterjoin%
\definecolor{currentfill}{rgb}{1.000000,1.000000,1.000000}%
\pgfsetfillcolor{currentfill}%
\pgfsetlinewidth{0.000000pt}%
\definecolor{currentstroke}{rgb}{1.000000,1.000000,1.000000}%
\pgfsetstrokecolor{currentstroke}%
\pgfsetdash{}{0pt}%
\pgfpathmoveto{\pgfqpoint{0.000000in}{0.000000in}}%
\pgfpathlineto{\pgfqpoint{5.674225in}{0.000000in}}%
\pgfpathlineto{\pgfqpoint{5.674225in}{4.068832in}}%
\pgfpathlineto{\pgfqpoint{0.000000in}{4.068832in}}%
\pgfpathclose%
\pgfusepath{fill}%
\end{pgfscope}%
\begin{pgfscope}%
\pgfsetbuttcap%
\pgfsetmiterjoin%
\definecolor{currentfill}{rgb}{1.000000,1.000000,1.000000}%
\pgfsetfillcolor{currentfill}%
\pgfsetlinewidth{0.000000pt}%
\definecolor{currentstroke}{rgb}{0.000000,0.000000,0.000000}%
\pgfsetstrokecolor{currentstroke}%
\pgfsetstrokeopacity{0.000000}%
\pgfsetdash{}{0pt}%
\pgfpathmoveto{\pgfqpoint{0.889225in}{3.178832in}}%
\pgfpathlineto{\pgfqpoint{2.051725in}{3.178832in}}%
\pgfpathlineto{\pgfqpoint{2.051725in}{3.933832in}}%
\pgfpathlineto{\pgfqpoint{0.889225in}{3.933832in}}%
\pgfpathclose%
\pgfusepath{fill}%
\end{pgfscope}%
\begin{pgfscope}%
\pgfsetbuttcap%
\pgfsetroundjoin%
\definecolor{currentfill}{rgb}{0.000000,0.000000,0.000000}%
\pgfsetfillcolor{currentfill}%
\pgfsetlinewidth{0.803000pt}%
\definecolor{currentstroke}{rgb}{0.000000,0.000000,0.000000}%
\pgfsetstrokecolor{currentstroke}%
\pgfsetdash{}{0pt}%
\pgfsys@defobject{currentmarker}{\pgfqpoint{-0.048611in}{0.000000in}}{\pgfqpoint{0.000000in}{0.000000in}}{%
\pgfpathmoveto{\pgfqpoint{0.000000in}{0.000000in}}%
\pgfpathlineto{\pgfqpoint{-0.048611in}{0.000000in}}%
\pgfusepath{stroke,fill}%
}%
\begin{pgfscope}%
\pgfsys@transformshift{0.889225in}{3.404583in}%
\pgfsys@useobject{currentmarker}{}%
\end{pgfscope}%
\end{pgfscope}%
\begin{pgfscope}%
\pgftext[x=0.370616in,y=3.362373in,left,base]{\rmfamily\fontsize{8.000000}{9.600000}\selectfont 2.5e-05}%
\end{pgfscope}%
\begin{pgfscope}%
\pgfsetbuttcap%
\pgfsetroundjoin%
\definecolor{currentfill}{rgb}{0.000000,0.000000,0.000000}%
\pgfsetfillcolor{currentfill}%
\pgfsetlinewidth{0.803000pt}%
\definecolor{currentstroke}{rgb}{0.000000,0.000000,0.000000}%
\pgfsetstrokecolor{currentstroke}%
\pgfsetdash{}{0pt}%
\pgfsys@defobject{currentmarker}{\pgfqpoint{-0.048611in}{0.000000in}}{\pgfqpoint{0.000000in}{0.000000in}}{%
\pgfpathmoveto{\pgfqpoint{0.000000in}{0.000000in}}%
\pgfpathlineto{\pgfqpoint{-0.048611in}{0.000000in}}%
\pgfusepath{stroke,fill}%
}%
\begin{pgfscope}%
\pgfsys@transformshift{0.889225in}{3.743297in}%
\pgfsys@useobject{currentmarker}{}%
\end{pgfscope}%
\end{pgfscope}%
\begin{pgfscope}%
\pgftext[x=0.476627in,y=3.701087in,left,base]{\rmfamily\fontsize{8.000000}{9.600000}\selectfont 5e-05}%
\end{pgfscope}%
\begin{pgfscope}%
\pgftext[x=0.315061in,y=3.556332in,,bottom,rotate=90.000000]{\rmfamily\fontsize{16.000000}{19.200000}\selectfont area}%
\end{pgfscope}%
\begin{pgfscope}%
\pgfpathrectangle{\pgfqpoint{0.889225in}{3.178832in}}{\pgfqpoint{1.162500in}{0.755000in}}%
\pgfusepath{clip}%
\pgfsetrectcap%
\pgfsetroundjoin%
\pgfsetlinewidth{1.505625pt}%
\definecolor{currentstroke}{rgb}{0.121569,0.466667,0.705882}%
\pgfsetstrokecolor{currentstroke}%
\pgfsetdash{}{0pt}%
\pgfpathmoveto{\pgfqpoint{0.916904in}{3.605762in}}%
\pgfpathlineto{\pgfqpoint{0.947935in}{3.674214in}}%
\pgfpathlineto{\pgfqpoint{0.973424in}{3.725304in}}%
\pgfpathlineto{\pgfqpoint{0.995589in}{3.765054in}}%
\pgfpathlineto{\pgfqpoint{1.016646in}{3.798265in}}%
\pgfpathlineto{\pgfqpoint{1.035486in}{3.823954in}}%
\pgfpathlineto{\pgfqpoint{1.053218in}{3.844547in}}%
\pgfpathlineto{\pgfqpoint{1.069842in}{3.860681in}}%
\pgfpathlineto{\pgfqpoint{1.086466in}{3.873795in}}%
\pgfpathlineto{\pgfqpoint{1.103090in}{3.883997in}}%
\pgfpathlineto{\pgfqpoint{1.119714in}{3.891448in}}%
\pgfpathlineto{\pgfqpoint{1.136337in}{3.896349in}}%
\pgfpathlineto{\pgfqpoint{1.152961in}{3.898936in}}%
\pgfpathlineto{\pgfqpoint{1.170693in}{3.899439in}}%
\pgfpathlineto{\pgfqpoint{1.189533in}{3.897784in}}%
\pgfpathlineto{\pgfqpoint{1.211698in}{3.893479in}}%
\pgfpathlineto{\pgfqpoint{1.237188in}{3.886125in}}%
\pgfpathlineto{\pgfqpoint{1.269327in}{3.874363in}}%
\pgfpathlineto{\pgfqpoint{1.312549in}{3.855975in}}%
\pgfpathlineto{\pgfqpoint{1.362421in}{3.832368in}}%
\pgfpathlineto{\pgfqpoint{1.402318in}{3.811182in}}%
\pgfpathlineto{\pgfqpoint{1.435565in}{3.791115in}}%
\pgfpathlineto{\pgfqpoint{1.466596in}{3.769769in}}%
\pgfpathlineto{\pgfqpoint{1.496519in}{3.746420in}}%
\pgfpathlineto{\pgfqpoint{1.526442in}{3.720247in}}%
\pgfpathlineto{\pgfqpoint{1.558581in}{3.689172in}}%
\pgfpathlineto{\pgfqpoint{1.596262in}{3.649514in}}%
\pgfpathlineto{\pgfqpoint{1.649458in}{3.589904in}}%
\pgfpathlineto{\pgfqpoint{1.740334in}{3.487986in}}%
\pgfpathlineto{\pgfqpoint{1.790205in}{3.435782in}}%
\pgfpathlineto{\pgfqpoint{1.842293in}{3.384616in}}%
\pgfpathlineto{\pgfqpoint{1.920979in}{3.310990in}}%
\pgfpathlineto{\pgfqpoint{2.010747in}{3.226056in}}%
\pgfpathlineto{\pgfqpoint{2.024046in}{3.213151in}}%
\pgfpathlineto{\pgfqpoint{2.024046in}{3.213151in}}%
\pgfusepath{stroke}%
\end{pgfscope}%
\begin{pgfscope}%
\pgfsetrectcap%
\pgfsetmiterjoin%
\pgfsetlinewidth{0.803000pt}%
\definecolor{currentstroke}{rgb}{0.501961,0.501961,0.501961}%
\pgfsetstrokecolor{currentstroke}%
\pgfsetdash{}{0pt}%
\pgfpathmoveto{\pgfqpoint{0.889225in}{3.178832in}}%
\pgfpathlineto{\pgfqpoint{0.889225in}{3.933832in}}%
\pgfusepath{stroke}%
\end{pgfscope}%
\begin{pgfscope}%
\pgfsetrectcap%
\pgfsetmiterjoin%
\pgfsetlinewidth{0.803000pt}%
\definecolor{currentstroke}{rgb}{0.501961,0.501961,0.501961}%
\pgfsetstrokecolor{currentstroke}%
\pgfsetdash{}{0pt}%
\pgfpathmoveto{\pgfqpoint{2.051725in}{3.178832in}}%
\pgfpathlineto{\pgfqpoint{2.051725in}{3.933832in}}%
\pgfusepath{stroke}%
\end{pgfscope}%
\begin{pgfscope}%
\pgfsetrectcap%
\pgfsetmiterjoin%
\pgfsetlinewidth{0.803000pt}%
\definecolor{currentstroke}{rgb}{0.501961,0.501961,0.501961}%
\pgfsetstrokecolor{currentstroke}%
\pgfsetdash{}{0pt}%
\pgfpathmoveto{\pgfqpoint{0.889225in}{3.178832in}}%
\pgfpathlineto{\pgfqpoint{2.051725in}{3.178832in}}%
\pgfusepath{stroke}%
\end{pgfscope}%
\begin{pgfscope}%
\pgfsetrectcap%
\pgfsetmiterjoin%
\pgfsetlinewidth{0.803000pt}%
\definecolor{currentstroke}{rgb}{0.501961,0.501961,0.501961}%
\pgfsetstrokecolor{currentstroke}%
\pgfsetdash{}{0pt}%
\pgfpathmoveto{\pgfqpoint{0.889225in}{3.933832in}}%
\pgfpathlineto{\pgfqpoint{2.051725in}{3.933832in}}%
\pgfusepath{stroke}%
\end{pgfscope}%
\begin{pgfscope}%
\pgfsetbuttcap%
\pgfsetmiterjoin%
\definecolor{currentfill}{rgb}{1.000000,1.000000,1.000000}%
\pgfsetfillcolor{currentfill}%
\pgfsetlinewidth{0.000000pt}%
\definecolor{currentstroke}{rgb}{0.000000,0.000000,0.000000}%
\pgfsetstrokecolor{currentstroke}%
\pgfsetstrokeopacity{0.000000}%
\pgfsetdash{}{0pt}%
\pgfpathmoveto{\pgfqpoint{2.051725in}{3.178832in}}%
\pgfpathlineto{\pgfqpoint{3.214225in}{3.178832in}}%
\pgfpathlineto{\pgfqpoint{3.214225in}{3.933832in}}%
\pgfpathlineto{\pgfqpoint{2.051725in}{3.933832in}}%
\pgfpathclose%
\pgfusepath{fill}%
\end{pgfscope}%
\begin{pgfscope}%
\pgfpathrectangle{\pgfqpoint{2.051725in}{3.178832in}}{\pgfqpoint{1.162500in}{0.755000in}}%
\pgfusepath{clip}%
\pgfsetbuttcap%
\pgfsetroundjoin%
\definecolor{currentfill}{rgb}{0.000000,0.000000,0.000000}%
\pgfsetfillcolor{currentfill}%
\pgfsetfillopacity{0.500000}%
\pgfsetlinewidth{0.000000pt}%
\definecolor{currentstroke}{rgb}{0.000000,0.000000,0.000000}%
\pgfsetstrokecolor{currentstroke}%
\pgfsetdash{}{0pt}%
\pgfpathmoveto{\pgfqpoint{3.107569in}{3.810398in}}%
\pgfpathcurveto{\pgfqpoint{3.113094in}{3.810398in}}{\pgfqpoint{3.118394in}{3.812593in}}{\pgfqpoint{3.122301in}{3.816499in}}%
\pgfpathcurveto{\pgfqpoint{3.126207in}{3.820406in}}{\pgfqpoint{3.128403in}{3.825706in}}{\pgfqpoint{3.128403in}{3.831231in}}%
\pgfpathcurveto{\pgfqpoint{3.128403in}{3.836756in}}{\pgfqpoint{3.126207in}{3.842055in}}{\pgfqpoint{3.122301in}{3.845962in}}%
\pgfpathcurveto{\pgfqpoint{3.118394in}{3.849869in}}{\pgfqpoint{3.113094in}{3.852064in}}{\pgfqpoint{3.107569in}{3.852064in}}%
\pgfpathcurveto{\pgfqpoint{3.102044in}{3.852064in}}{\pgfqpoint{3.096745in}{3.849869in}}{\pgfqpoint{3.092838in}{3.845962in}}%
\pgfpathcurveto{\pgfqpoint{3.088931in}{3.842055in}}{\pgfqpoint{3.086736in}{3.836756in}}{\pgfqpoint{3.086736in}{3.831231in}}%
\pgfpathcurveto{\pgfqpoint{3.086736in}{3.825706in}}{\pgfqpoint{3.088931in}{3.820406in}}{\pgfqpoint{3.092838in}{3.816499in}}%
\pgfpathcurveto{\pgfqpoint{3.096745in}{3.812593in}}{\pgfqpoint{3.102044in}{3.810398in}}{\pgfqpoint{3.107569in}{3.810398in}}%
\pgfpathclose%
\pgfusepath{fill}%
\end{pgfscope}%
\begin{pgfscope}%
\pgfpathrectangle{\pgfqpoint{2.051725in}{3.178832in}}{\pgfqpoint{1.162500in}{0.755000in}}%
\pgfusepath{clip}%
\pgfsetbuttcap%
\pgfsetroundjoin%
\definecolor{currentfill}{rgb}{0.000000,0.000000,0.000000}%
\pgfsetfillcolor{currentfill}%
\pgfsetfillopacity{0.500000}%
\pgfsetlinewidth{0.000000pt}%
\definecolor{currentstroke}{rgb}{0.000000,0.000000,0.000000}%
\pgfsetstrokecolor{currentstroke}%
\pgfsetdash{}{0pt}%
\pgfpathmoveto{\pgfqpoint{2.270173in}{3.539667in}}%
\pgfpathcurveto{\pgfqpoint{2.275698in}{3.539667in}}{\pgfqpoint{2.280997in}{3.541862in}}{\pgfqpoint{2.284904in}{3.545769in}}%
\pgfpathcurveto{\pgfqpoint{2.288811in}{3.549675in}}{\pgfqpoint{2.291006in}{3.554975in}}{\pgfqpoint{2.291006in}{3.560500in}}%
\pgfpathcurveto{\pgfqpoint{2.291006in}{3.566025in}}{\pgfqpoint{2.288811in}{3.571325in}}{\pgfqpoint{2.284904in}{3.575231in}}%
\pgfpathcurveto{\pgfqpoint{2.280997in}{3.579138in}}{\pgfqpoint{2.275698in}{3.581333in}}{\pgfqpoint{2.270173in}{3.581333in}}%
\pgfpathcurveto{\pgfqpoint{2.264648in}{3.581333in}}{\pgfqpoint{2.259348in}{3.579138in}}{\pgfqpoint{2.255441in}{3.575231in}}%
\pgfpathcurveto{\pgfqpoint{2.251535in}{3.571325in}}{\pgfqpoint{2.249339in}{3.566025in}}{\pgfqpoint{2.249339in}{3.560500in}}%
\pgfpathcurveto{\pgfqpoint{2.249339in}{3.554975in}}{\pgfqpoint{2.251535in}{3.549675in}}{\pgfqpoint{2.255441in}{3.545769in}}%
\pgfpathcurveto{\pgfqpoint{2.259348in}{3.541862in}}{\pgfqpoint{2.264648in}{3.539667in}}{\pgfqpoint{2.270173in}{3.539667in}}%
\pgfpathclose%
\pgfusepath{fill}%
\end{pgfscope}%
\begin{pgfscope}%
\pgfpathrectangle{\pgfqpoint{2.051725in}{3.178832in}}{\pgfqpoint{1.162500in}{0.755000in}}%
\pgfusepath{clip}%
\pgfsetbuttcap%
\pgfsetroundjoin%
\definecolor{currentfill}{rgb}{0.000000,0.000000,0.000000}%
\pgfsetfillcolor{currentfill}%
\pgfsetfillopacity{0.500000}%
\pgfsetlinewidth{0.000000pt}%
\definecolor{currentstroke}{rgb}{0.000000,0.000000,0.000000}%
\pgfsetstrokecolor{currentstroke}%
\pgfsetdash{}{0pt}%
\pgfpathmoveto{\pgfqpoint{2.264867in}{3.541019in}}%
\pgfpathcurveto{\pgfqpoint{2.270392in}{3.541019in}}{\pgfqpoint{2.275692in}{3.543214in}}{\pgfqpoint{2.279599in}{3.547121in}}%
\pgfpathcurveto{\pgfqpoint{2.283506in}{3.551028in}}{\pgfqpoint{2.285701in}{3.556327in}}{\pgfqpoint{2.285701in}{3.561852in}}%
\pgfpathcurveto{\pgfqpoint{2.285701in}{3.567377in}}{\pgfqpoint{2.283506in}{3.572677in}}{\pgfqpoint{2.279599in}{3.576584in}}%
\pgfpathcurveto{\pgfqpoint{2.275692in}{3.580490in}}{\pgfqpoint{2.270392in}{3.582685in}}{\pgfqpoint{2.264867in}{3.582685in}}%
\pgfpathcurveto{\pgfqpoint{2.259342in}{3.582685in}}{\pgfqpoint{2.254043in}{3.580490in}}{\pgfqpoint{2.250136in}{3.576584in}}%
\pgfpathcurveto{\pgfqpoint{2.246229in}{3.572677in}}{\pgfqpoint{2.244034in}{3.567377in}}{\pgfqpoint{2.244034in}{3.561852in}}%
\pgfpathcurveto{\pgfqpoint{2.244034in}{3.556327in}}{\pgfqpoint{2.246229in}{3.551028in}}{\pgfqpoint{2.250136in}{3.547121in}}%
\pgfpathcurveto{\pgfqpoint{2.254043in}{3.543214in}}{\pgfqpoint{2.259342in}{3.541019in}}{\pgfqpoint{2.264867in}{3.541019in}}%
\pgfpathclose%
\pgfusepath{fill}%
\end{pgfscope}%
\begin{pgfscope}%
\pgfpathrectangle{\pgfqpoint{2.051725in}{3.178832in}}{\pgfqpoint{1.162500in}{0.755000in}}%
\pgfusepath{clip}%
\pgfsetbuttcap%
\pgfsetroundjoin%
\definecolor{currentfill}{rgb}{0.000000,0.000000,0.000000}%
\pgfsetfillcolor{currentfill}%
\pgfsetfillopacity{0.500000}%
\pgfsetlinewidth{0.000000pt}%
\definecolor{currentstroke}{rgb}{0.000000,0.000000,0.000000}%
\pgfsetstrokecolor{currentstroke}%
\pgfsetdash{}{0pt}%
\pgfpathmoveto{\pgfqpoint{2.154193in}{3.308342in}}%
\pgfpathcurveto{\pgfqpoint{2.159718in}{3.308342in}}{\pgfqpoint{2.165018in}{3.310537in}}{\pgfqpoint{2.168924in}{3.314444in}}%
\pgfpathcurveto{\pgfqpoint{2.172831in}{3.318351in}}{\pgfqpoint{2.175026in}{3.323651in}}{\pgfqpoint{2.175026in}{3.329176in}}%
\pgfpathcurveto{\pgfqpoint{2.175026in}{3.334701in}}{\pgfqpoint{2.172831in}{3.340000in}}{\pgfqpoint{2.168924in}{3.343907in}}%
\pgfpathcurveto{\pgfqpoint{2.165018in}{3.347814in}}{\pgfqpoint{2.159718in}{3.350009in}}{\pgfqpoint{2.154193in}{3.350009in}}%
\pgfpathcurveto{\pgfqpoint{2.148668in}{3.350009in}}{\pgfqpoint{2.143368in}{3.347814in}}{\pgfqpoint{2.139462in}{3.343907in}}%
\pgfpathcurveto{\pgfqpoint{2.135555in}{3.340000in}}{\pgfqpoint{2.133360in}{3.334701in}}{\pgfqpoint{2.133360in}{3.329176in}}%
\pgfpathcurveto{\pgfqpoint{2.133360in}{3.323651in}}{\pgfqpoint{2.135555in}{3.318351in}}{\pgfqpoint{2.139462in}{3.314444in}}%
\pgfpathcurveto{\pgfqpoint{2.143368in}{3.310537in}}{\pgfqpoint{2.148668in}{3.308342in}}{\pgfqpoint{2.154193in}{3.308342in}}%
\pgfpathclose%
\pgfusepath{fill}%
\end{pgfscope}%
\begin{pgfscope}%
\pgfpathrectangle{\pgfqpoint{2.051725in}{3.178832in}}{\pgfqpoint{1.162500in}{0.755000in}}%
\pgfusepath{clip}%
\pgfsetbuttcap%
\pgfsetroundjoin%
\definecolor{currentfill}{rgb}{0.000000,0.000000,0.000000}%
\pgfsetfillcolor{currentfill}%
\pgfsetfillopacity{0.500000}%
\pgfsetlinewidth{0.000000pt}%
\definecolor{currentstroke}{rgb}{0.000000,0.000000,0.000000}%
\pgfsetstrokecolor{currentstroke}%
\pgfsetdash{}{0pt}%
\pgfpathmoveto{\pgfqpoint{2.183344in}{3.311937in}}%
\pgfpathcurveto{\pgfqpoint{2.188869in}{3.311937in}}{\pgfqpoint{2.194169in}{3.314132in}}{\pgfqpoint{2.198076in}{3.318039in}}%
\pgfpathcurveto{\pgfqpoint{2.201982in}{3.321945in}}{\pgfqpoint{2.204178in}{3.327245in}}{\pgfqpoint{2.204178in}{3.332770in}}%
\pgfpathcurveto{\pgfqpoint{2.204178in}{3.338295in}}{\pgfqpoint{2.201982in}{3.343595in}}{\pgfqpoint{2.198076in}{3.347501in}}%
\pgfpathcurveto{\pgfqpoint{2.194169in}{3.351408in}}{\pgfqpoint{2.188869in}{3.353603in}}{\pgfqpoint{2.183344in}{3.353603in}}%
\pgfpathcurveto{\pgfqpoint{2.177819in}{3.353603in}}{\pgfqpoint{2.172520in}{3.351408in}}{\pgfqpoint{2.168613in}{3.347501in}}%
\pgfpathcurveto{\pgfqpoint{2.164706in}{3.343595in}}{\pgfqpoint{2.162511in}{3.338295in}}{\pgfqpoint{2.162511in}{3.332770in}}%
\pgfpathcurveto{\pgfqpoint{2.162511in}{3.327245in}}{\pgfqpoint{2.164706in}{3.321945in}}{\pgfqpoint{2.168613in}{3.318039in}}%
\pgfpathcurveto{\pgfqpoint{2.172520in}{3.314132in}}{\pgfqpoint{2.177819in}{3.311937in}}{\pgfqpoint{2.183344in}{3.311937in}}%
\pgfpathclose%
\pgfusepath{fill}%
\end{pgfscope}%
\begin{pgfscope}%
\pgfpathrectangle{\pgfqpoint{2.051725in}{3.178832in}}{\pgfqpoint{1.162500in}{0.755000in}}%
\pgfusepath{clip}%
\pgfsetbuttcap%
\pgfsetroundjoin%
\definecolor{currentfill}{rgb}{0.000000,0.000000,0.000000}%
\pgfsetfillcolor{currentfill}%
\pgfsetfillopacity{0.500000}%
\pgfsetlinewidth{0.000000pt}%
\definecolor{currentstroke}{rgb}{0.000000,0.000000,0.000000}%
\pgfsetstrokecolor{currentstroke}%
\pgfsetdash{}{0pt}%
\pgfpathmoveto{\pgfqpoint{2.082232in}{3.215463in}}%
\pgfpathcurveto{\pgfqpoint{2.087757in}{3.215463in}}{\pgfqpoint{2.093056in}{3.217658in}}{\pgfqpoint{2.096963in}{3.221565in}}%
\pgfpathcurveto{\pgfqpoint{2.100870in}{3.225471in}}{\pgfqpoint{2.103065in}{3.230771in}}{\pgfqpoint{2.103065in}{3.236296in}}%
\pgfpathcurveto{\pgfqpoint{2.103065in}{3.241821in}}{\pgfqpoint{2.100870in}{3.247121in}}{\pgfqpoint{2.096963in}{3.251027in}}%
\pgfpathcurveto{\pgfqpoint{2.093056in}{3.254934in}}{\pgfqpoint{2.087757in}{3.257129in}}{\pgfqpoint{2.082232in}{3.257129in}}%
\pgfpathcurveto{\pgfqpoint{2.076707in}{3.257129in}}{\pgfqpoint{2.071407in}{3.254934in}}{\pgfqpoint{2.067500in}{3.251027in}}%
\pgfpathcurveto{\pgfqpoint{2.063594in}{3.247121in}}{\pgfqpoint{2.061398in}{3.241821in}}{\pgfqpoint{2.061398in}{3.236296in}}%
\pgfpathcurveto{\pgfqpoint{2.061398in}{3.230771in}}{\pgfqpoint{2.063594in}{3.225471in}}{\pgfqpoint{2.067500in}{3.221565in}}%
\pgfpathcurveto{\pgfqpoint{2.071407in}{3.217658in}}{\pgfqpoint{2.076707in}{3.215463in}}{\pgfqpoint{2.082232in}{3.215463in}}%
\pgfpathclose%
\pgfusepath{fill}%
\end{pgfscope}%
\begin{pgfscope}%
\pgfpathrectangle{\pgfqpoint{2.051725in}{3.178832in}}{\pgfqpoint{1.162500in}{0.755000in}}%
\pgfusepath{clip}%
\pgfsetbuttcap%
\pgfsetroundjoin%
\definecolor{currentfill}{rgb}{0.000000,0.000000,0.000000}%
\pgfsetfillcolor{currentfill}%
\pgfsetfillopacity{0.500000}%
\pgfsetlinewidth{0.000000pt}%
\definecolor{currentstroke}{rgb}{0.000000,0.000000,0.000000}%
\pgfsetstrokecolor{currentstroke}%
\pgfsetdash{}{0pt}%
\pgfpathmoveto{\pgfqpoint{2.079404in}{3.175975in}}%
\pgfpathcurveto{\pgfqpoint{2.084929in}{3.175975in}}{\pgfqpoint{2.090228in}{3.178170in}}{\pgfqpoint{2.094135in}{3.182077in}}%
\pgfpathcurveto{\pgfqpoint{2.098042in}{3.185984in}}{\pgfqpoint{2.100237in}{3.191283in}}{\pgfqpoint{2.100237in}{3.196809in}}%
\pgfpathcurveto{\pgfqpoint{2.100237in}{3.202334in}}{\pgfqpoint{2.098042in}{3.207633in}}{\pgfqpoint{2.094135in}{3.211540in}}%
\pgfpathcurveto{\pgfqpoint{2.090228in}{3.215447in}}{\pgfqpoint{2.084929in}{3.217642in}}{\pgfqpoint{2.079404in}{3.217642in}}%
\pgfpathcurveto{\pgfqpoint{2.073879in}{3.217642in}}{\pgfqpoint{2.068579in}{3.215447in}}{\pgfqpoint{2.064672in}{3.211540in}}%
\pgfpathcurveto{\pgfqpoint{2.060765in}{3.207633in}}{\pgfqpoint{2.058570in}{3.202334in}}{\pgfqpoint{2.058570in}{3.196809in}}%
\pgfpathcurveto{\pgfqpoint{2.058570in}{3.191283in}}{\pgfqpoint{2.060765in}{3.185984in}}{\pgfqpoint{2.064672in}{3.182077in}}%
\pgfpathcurveto{\pgfqpoint{2.068579in}{3.178170in}}{\pgfqpoint{2.073879in}{3.175975in}}{\pgfqpoint{2.079404in}{3.175975in}}%
\pgfpathclose%
\pgfusepath{fill}%
\end{pgfscope}%
\begin{pgfscope}%
\pgfpathrectangle{\pgfqpoint{2.051725in}{3.178832in}}{\pgfqpoint{1.162500in}{0.755000in}}%
\pgfusepath{clip}%
\pgfsetbuttcap%
\pgfsetroundjoin%
\definecolor{currentfill}{rgb}{0.000000,0.000000,0.000000}%
\pgfsetfillcolor{currentfill}%
\pgfsetfillopacity{0.500000}%
\pgfsetlinewidth{0.000000pt}%
\definecolor{currentstroke}{rgb}{0.000000,0.000000,0.000000}%
\pgfsetstrokecolor{currentstroke}%
\pgfsetdash{}{0pt}%
\pgfpathmoveto{\pgfqpoint{2.536683in}{3.678982in}}%
\pgfpathcurveto{\pgfqpoint{2.542208in}{3.678982in}}{\pgfqpoint{2.547508in}{3.681177in}}{\pgfqpoint{2.551415in}{3.685084in}}%
\pgfpathcurveto{\pgfqpoint{2.555322in}{3.688990in}}{\pgfqpoint{2.557517in}{3.694290in}}{\pgfqpoint{2.557517in}{3.699815in}}%
\pgfpathcurveto{\pgfqpoint{2.557517in}{3.705340in}}{\pgfqpoint{2.555322in}{3.710640in}}{\pgfqpoint{2.551415in}{3.714546in}}%
\pgfpathcurveto{\pgfqpoint{2.547508in}{3.718453in}}{\pgfqpoint{2.542208in}{3.720648in}}{\pgfqpoint{2.536683in}{3.720648in}}%
\pgfpathcurveto{\pgfqpoint{2.531158in}{3.720648in}}{\pgfqpoint{2.525859in}{3.718453in}}{\pgfqpoint{2.521952in}{3.714546in}}%
\pgfpathcurveto{\pgfqpoint{2.518045in}{3.710640in}}{\pgfqpoint{2.515850in}{3.705340in}}{\pgfqpoint{2.515850in}{3.699815in}}%
\pgfpathcurveto{\pgfqpoint{2.515850in}{3.694290in}}{\pgfqpoint{2.518045in}{3.688990in}}{\pgfqpoint{2.521952in}{3.685084in}}%
\pgfpathcurveto{\pgfqpoint{2.525859in}{3.681177in}}{\pgfqpoint{2.531158in}{3.678982in}}{\pgfqpoint{2.536683in}{3.678982in}}%
\pgfpathclose%
\pgfusepath{fill}%
\end{pgfscope}%
\begin{pgfscope}%
\pgfpathrectangle{\pgfqpoint{2.051725in}{3.178832in}}{\pgfqpoint{1.162500in}{0.755000in}}%
\pgfusepath{clip}%
\pgfsetbuttcap%
\pgfsetroundjoin%
\definecolor{currentfill}{rgb}{0.000000,0.000000,0.000000}%
\pgfsetfillcolor{currentfill}%
\pgfsetfillopacity{0.500000}%
\pgfsetlinewidth{0.000000pt}%
\definecolor{currentstroke}{rgb}{0.000000,0.000000,0.000000}%
\pgfsetstrokecolor{currentstroke}%
\pgfsetdash{}{0pt}%
\pgfpathmoveto{\pgfqpoint{2.417381in}{3.528997in}}%
\pgfpathcurveto{\pgfqpoint{2.422906in}{3.528997in}}{\pgfqpoint{2.428205in}{3.531192in}}{\pgfqpoint{2.432112in}{3.535099in}}%
\pgfpathcurveto{\pgfqpoint{2.436019in}{3.539005in}}{\pgfqpoint{2.438214in}{3.544305in}}{\pgfqpoint{2.438214in}{3.549830in}}%
\pgfpathcurveto{\pgfqpoint{2.438214in}{3.555355in}}{\pgfqpoint{2.436019in}{3.560655in}}{\pgfqpoint{2.432112in}{3.564561in}}%
\pgfpathcurveto{\pgfqpoint{2.428205in}{3.568468in}}{\pgfqpoint{2.422906in}{3.570663in}}{\pgfqpoint{2.417381in}{3.570663in}}%
\pgfpathcurveto{\pgfqpoint{2.411856in}{3.570663in}}{\pgfqpoint{2.406556in}{3.568468in}}{\pgfqpoint{2.402649in}{3.564561in}}%
\pgfpathcurveto{\pgfqpoint{2.398743in}{3.560655in}}{\pgfqpoint{2.396547in}{3.555355in}}{\pgfqpoint{2.396547in}{3.549830in}}%
\pgfpathcurveto{\pgfqpoint{2.396547in}{3.544305in}}{\pgfqpoint{2.398743in}{3.539005in}}{\pgfqpoint{2.402649in}{3.535099in}}%
\pgfpathcurveto{\pgfqpoint{2.406556in}{3.531192in}}{\pgfqpoint{2.411856in}{3.528997in}}{\pgfqpoint{2.417381in}{3.528997in}}%
\pgfpathclose%
\pgfusepath{fill}%
\end{pgfscope}%
\begin{pgfscope}%
\pgfpathrectangle{\pgfqpoint{2.051725in}{3.178832in}}{\pgfqpoint{1.162500in}{0.755000in}}%
\pgfusepath{clip}%
\pgfsetbuttcap%
\pgfsetroundjoin%
\definecolor{currentfill}{rgb}{0.000000,0.000000,0.000000}%
\pgfsetfillcolor{currentfill}%
\pgfsetfillopacity{0.500000}%
\pgfsetlinewidth{0.000000pt}%
\definecolor{currentstroke}{rgb}{0.000000,0.000000,0.000000}%
\pgfsetstrokecolor{currentstroke}%
\pgfsetdash{}{0pt}%
\pgfpathmoveto{\pgfqpoint{2.412630in}{3.426855in}}%
\pgfpathcurveto{\pgfqpoint{2.418155in}{3.426855in}}{\pgfqpoint{2.423454in}{3.429051in}}{\pgfqpoint{2.427361in}{3.432957in}}%
\pgfpathcurveto{\pgfqpoint{2.431268in}{3.436864in}}{\pgfqpoint{2.433463in}{3.442164in}}{\pgfqpoint{2.433463in}{3.447689in}}%
\pgfpathcurveto{\pgfqpoint{2.433463in}{3.453214in}}{\pgfqpoint{2.431268in}{3.458513in}}{\pgfqpoint{2.427361in}{3.462420in}}%
\pgfpathcurveto{\pgfqpoint{2.423454in}{3.466327in}}{\pgfqpoint{2.418155in}{3.468522in}}{\pgfqpoint{2.412630in}{3.468522in}}%
\pgfpathcurveto{\pgfqpoint{2.407105in}{3.468522in}}{\pgfqpoint{2.401805in}{3.466327in}}{\pgfqpoint{2.397898in}{3.462420in}}%
\pgfpathcurveto{\pgfqpoint{2.393991in}{3.458513in}}{\pgfqpoint{2.391796in}{3.453214in}}{\pgfqpoint{2.391796in}{3.447689in}}%
\pgfpathcurveto{\pgfqpoint{2.391796in}{3.442164in}}{\pgfqpoint{2.393991in}{3.436864in}}{\pgfqpoint{2.397898in}{3.432957in}}%
\pgfpathcurveto{\pgfqpoint{2.401805in}{3.429051in}}{\pgfqpoint{2.407105in}{3.426855in}}{\pgfqpoint{2.412630in}{3.426855in}}%
\pgfpathclose%
\pgfusepath{fill}%
\end{pgfscope}%
\begin{pgfscope}%
\pgfpathrectangle{\pgfqpoint{2.051725in}{3.178832in}}{\pgfqpoint{1.162500in}{0.755000in}}%
\pgfusepath{clip}%
\pgfsetbuttcap%
\pgfsetroundjoin%
\definecolor{currentfill}{rgb}{0.000000,0.000000,0.000000}%
\pgfsetfillcolor{currentfill}%
\pgfsetfillopacity{0.500000}%
\pgfsetlinewidth{0.000000pt}%
\definecolor{currentstroke}{rgb}{0.000000,0.000000,0.000000}%
\pgfsetstrokecolor{currentstroke}%
\pgfsetdash{}{0pt}%
\pgfpathmoveto{\pgfqpoint{2.478188in}{3.677650in}}%
\pgfpathcurveto{\pgfqpoint{2.483713in}{3.677650in}}{\pgfqpoint{2.489012in}{3.679845in}}{\pgfqpoint{2.492919in}{3.683752in}}%
\pgfpathcurveto{\pgfqpoint{2.496826in}{3.687659in}}{\pgfqpoint{2.499021in}{3.692959in}}{\pgfqpoint{2.499021in}{3.698484in}}%
\pgfpathcurveto{\pgfqpoint{2.499021in}{3.704009in}}{\pgfqpoint{2.496826in}{3.709308in}}{\pgfqpoint{2.492919in}{3.713215in}}%
\pgfpathcurveto{\pgfqpoint{2.489012in}{3.717122in}}{\pgfqpoint{2.483713in}{3.719317in}}{\pgfqpoint{2.478188in}{3.719317in}}%
\pgfpathcurveto{\pgfqpoint{2.472663in}{3.719317in}}{\pgfqpoint{2.467363in}{3.717122in}}{\pgfqpoint{2.463456in}{3.713215in}}%
\pgfpathcurveto{\pgfqpoint{2.459549in}{3.709308in}}{\pgfqpoint{2.457354in}{3.704009in}}{\pgfqpoint{2.457354in}{3.698484in}}%
\pgfpathcurveto{\pgfqpoint{2.457354in}{3.692959in}}{\pgfqpoint{2.459549in}{3.687659in}}{\pgfqpoint{2.463456in}{3.683752in}}%
\pgfpathcurveto{\pgfqpoint{2.467363in}{3.679845in}}{\pgfqpoint{2.472663in}{3.677650in}}{\pgfqpoint{2.478188in}{3.677650in}}%
\pgfpathclose%
\pgfusepath{fill}%
\end{pgfscope}%
\begin{pgfscope}%
\pgfpathrectangle{\pgfqpoint{2.051725in}{3.178832in}}{\pgfqpoint{1.162500in}{0.755000in}}%
\pgfusepath{clip}%
\pgfsetbuttcap%
\pgfsetroundjoin%
\definecolor{currentfill}{rgb}{0.000000,0.000000,0.000000}%
\pgfsetfillcolor{currentfill}%
\pgfsetfillopacity{0.500000}%
\pgfsetlinewidth{0.000000pt}%
\definecolor{currentstroke}{rgb}{0.000000,0.000000,0.000000}%
\pgfsetstrokecolor{currentstroke}%
\pgfsetdash{}{0pt}%
\pgfpathmoveto{\pgfqpoint{2.172291in}{3.296446in}}%
\pgfpathcurveto{\pgfqpoint{2.177816in}{3.296446in}}{\pgfqpoint{2.183116in}{3.298641in}}{\pgfqpoint{2.187022in}{3.302548in}}%
\pgfpathcurveto{\pgfqpoint{2.190929in}{3.306455in}}{\pgfqpoint{2.193124in}{3.311754in}}{\pgfqpoint{2.193124in}{3.317279in}}%
\pgfpathcurveto{\pgfqpoint{2.193124in}{3.322804in}}{\pgfqpoint{2.190929in}{3.328104in}}{\pgfqpoint{2.187022in}{3.332011in}}%
\pgfpathcurveto{\pgfqpoint{2.183116in}{3.335917in}}{\pgfqpoint{2.177816in}{3.338112in}}{\pgfqpoint{2.172291in}{3.338112in}}%
\pgfpathcurveto{\pgfqpoint{2.166766in}{3.338112in}}{\pgfqpoint{2.161467in}{3.335917in}}{\pgfqpoint{2.157560in}{3.332011in}}%
\pgfpathcurveto{\pgfqpoint{2.153653in}{3.328104in}}{\pgfqpoint{2.151458in}{3.322804in}}{\pgfqpoint{2.151458in}{3.317279in}}%
\pgfpathcurveto{\pgfqpoint{2.151458in}{3.311754in}}{\pgfqpoint{2.153653in}{3.306455in}}{\pgfqpoint{2.157560in}{3.302548in}}%
\pgfpathcurveto{\pgfqpoint{2.161467in}{3.298641in}}{\pgfqpoint{2.166766in}{3.296446in}}{\pgfqpoint{2.172291in}{3.296446in}}%
\pgfpathclose%
\pgfusepath{fill}%
\end{pgfscope}%
\begin{pgfscope}%
\pgfpathrectangle{\pgfqpoint{2.051725in}{3.178832in}}{\pgfqpoint{1.162500in}{0.755000in}}%
\pgfusepath{clip}%
\pgfsetbuttcap%
\pgfsetroundjoin%
\definecolor{currentfill}{rgb}{0.000000,0.000000,0.000000}%
\pgfsetfillcolor{currentfill}%
\pgfsetfillopacity{0.500000}%
\pgfsetlinewidth{0.000000pt}%
\definecolor{currentstroke}{rgb}{0.000000,0.000000,0.000000}%
\pgfsetstrokecolor{currentstroke}%
\pgfsetdash{}{0pt}%
\pgfpathmoveto{\pgfqpoint{2.205718in}{3.322937in}}%
\pgfpathcurveto{\pgfqpoint{2.211243in}{3.322937in}}{\pgfqpoint{2.216543in}{3.325133in}}{\pgfqpoint{2.220450in}{3.329039in}}%
\pgfpathcurveto{\pgfqpoint{2.224357in}{3.332946in}}{\pgfqpoint{2.226552in}{3.338246in}}{\pgfqpoint{2.226552in}{3.343771in}}%
\pgfpathcurveto{\pgfqpoint{2.226552in}{3.349296in}}{\pgfqpoint{2.224357in}{3.354595in}}{\pgfqpoint{2.220450in}{3.358502in}}%
\pgfpathcurveto{\pgfqpoint{2.216543in}{3.362409in}}{\pgfqpoint{2.211243in}{3.364604in}}{\pgfqpoint{2.205718in}{3.364604in}}%
\pgfpathcurveto{\pgfqpoint{2.200193in}{3.364604in}}{\pgfqpoint{2.194894in}{3.362409in}}{\pgfqpoint{2.190987in}{3.358502in}}%
\pgfpathcurveto{\pgfqpoint{2.187080in}{3.354595in}}{\pgfqpoint{2.184885in}{3.349296in}}{\pgfqpoint{2.184885in}{3.343771in}}%
\pgfpathcurveto{\pgfqpoint{2.184885in}{3.338246in}}{\pgfqpoint{2.187080in}{3.332946in}}{\pgfqpoint{2.190987in}{3.329039in}}%
\pgfpathcurveto{\pgfqpoint{2.194894in}{3.325133in}}{\pgfqpoint{2.200193in}{3.322937in}}{\pgfqpoint{2.205718in}{3.322937in}}%
\pgfpathclose%
\pgfusepath{fill}%
\end{pgfscope}%
\begin{pgfscope}%
\pgfpathrectangle{\pgfqpoint{2.051725in}{3.178832in}}{\pgfqpoint{1.162500in}{0.755000in}}%
\pgfusepath{clip}%
\pgfsetbuttcap%
\pgfsetroundjoin%
\definecolor{currentfill}{rgb}{0.000000,0.000000,0.000000}%
\pgfsetfillcolor{currentfill}%
\pgfsetfillopacity{0.500000}%
\pgfsetlinewidth{0.000000pt}%
\definecolor{currentstroke}{rgb}{0.000000,0.000000,0.000000}%
\pgfsetstrokecolor{currentstroke}%
\pgfsetdash{}{0pt}%
\pgfpathmoveto{\pgfqpoint{3.186546in}{3.895023in}}%
\pgfpathcurveto{\pgfqpoint{3.192071in}{3.895023in}}{\pgfqpoint{3.197371in}{3.897218in}}{\pgfqpoint{3.201278in}{3.901125in}}%
\pgfpathcurveto{\pgfqpoint{3.205185in}{3.905032in}}{\pgfqpoint{3.207380in}{3.910331in}}{\pgfqpoint{3.207380in}{3.915856in}}%
\pgfpathcurveto{\pgfqpoint{3.207380in}{3.921381in}}{\pgfqpoint{3.205185in}{3.926681in}}{\pgfqpoint{3.201278in}{3.930588in}}%
\pgfpathcurveto{\pgfqpoint{3.197371in}{3.934494in}}{\pgfqpoint{3.192071in}{3.936690in}}{\pgfqpoint{3.186546in}{3.936690in}}%
\pgfpathcurveto{\pgfqpoint{3.181021in}{3.936690in}}{\pgfqpoint{3.175722in}{3.934494in}}{\pgfqpoint{3.171815in}{3.930588in}}%
\pgfpathcurveto{\pgfqpoint{3.167908in}{3.926681in}}{\pgfqpoint{3.165713in}{3.921381in}}{\pgfqpoint{3.165713in}{3.915856in}}%
\pgfpathcurveto{\pgfqpoint{3.165713in}{3.910331in}}{\pgfqpoint{3.167908in}{3.905032in}}{\pgfqpoint{3.171815in}{3.901125in}}%
\pgfpathcurveto{\pgfqpoint{3.175722in}{3.897218in}}{\pgfqpoint{3.181021in}{3.895023in}}{\pgfqpoint{3.186546in}{3.895023in}}%
\pgfpathclose%
\pgfusepath{fill}%
\end{pgfscope}%
\begin{pgfscope}%
\pgfpathrectangle{\pgfqpoint{2.051725in}{3.178832in}}{\pgfqpoint{1.162500in}{0.755000in}}%
\pgfusepath{clip}%
\pgfsetbuttcap%
\pgfsetroundjoin%
\definecolor{currentfill}{rgb}{0.000000,0.000000,0.000000}%
\pgfsetfillcolor{currentfill}%
\pgfsetfillopacity{0.500000}%
\pgfsetlinewidth{0.000000pt}%
\definecolor{currentstroke}{rgb}{0.000000,0.000000,0.000000}%
\pgfsetstrokecolor{currentstroke}%
\pgfsetdash{}{0pt}%
\pgfpathmoveto{\pgfqpoint{2.394110in}{3.491212in}}%
\pgfpathcurveto{\pgfqpoint{2.399635in}{3.491212in}}{\pgfqpoint{2.404935in}{3.493407in}}{\pgfqpoint{2.408842in}{3.497314in}}%
\pgfpathcurveto{\pgfqpoint{2.412749in}{3.501221in}}{\pgfqpoint{2.414944in}{3.506520in}}{\pgfqpoint{2.414944in}{3.512045in}}%
\pgfpathcurveto{\pgfqpoint{2.414944in}{3.517571in}}{\pgfqpoint{2.412749in}{3.522870in}}{\pgfqpoint{2.408842in}{3.526777in}}%
\pgfpathcurveto{\pgfqpoint{2.404935in}{3.530684in}}{\pgfqpoint{2.399635in}{3.532879in}}{\pgfqpoint{2.394110in}{3.532879in}}%
\pgfpathcurveto{\pgfqpoint{2.388585in}{3.532879in}}{\pgfqpoint{2.383286in}{3.530684in}}{\pgfqpoint{2.379379in}{3.526777in}}%
\pgfpathcurveto{\pgfqpoint{2.375472in}{3.522870in}}{\pgfqpoint{2.373277in}{3.517571in}}{\pgfqpoint{2.373277in}{3.512045in}}%
\pgfpathcurveto{\pgfqpoint{2.373277in}{3.506520in}}{\pgfqpoint{2.375472in}{3.501221in}}{\pgfqpoint{2.379379in}{3.497314in}}%
\pgfpathcurveto{\pgfqpoint{2.383286in}{3.493407in}}{\pgfqpoint{2.388585in}{3.491212in}}{\pgfqpoint{2.394110in}{3.491212in}}%
\pgfpathclose%
\pgfusepath{fill}%
\end{pgfscope}%
\begin{pgfscope}%
\pgfpathrectangle{\pgfqpoint{2.051725in}{3.178832in}}{\pgfqpoint{1.162500in}{0.755000in}}%
\pgfusepath{clip}%
\pgfsetbuttcap%
\pgfsetroundjoin%
\definecolor{currentfill}{rgb}{0.000000,0.000000,0.000000}%
\pgfsetfillcolor{currentfill}%
\pgfsetfillopacity{0.500000}%
\pgfsetlinewidth{0.000000pt}%
\definecolor{currentstroke}{rgb}{0.000000,0.000000,0.000000}%
\pgfsetstrokecolor{currentstroke}%
\pgfsetdash{}{0pt}%
\pgfpathmoveto{\pgfqpoint{2.125339in}{3.212721in}}%
\pgfpathcurveto{\pgfqpoint{2.130865in}{3.212721in}}{\pgfqpoint{2.136164in}{3.214917in}}{\pgfqpoint{2.140071in}{3.218823in}}%
\pgfpathcurveto{\pgfqpoint{2.143978in}{3.222730in}}{\pgfqpoint{2.146173in}{3.228030in}}{\pgfqpoint{2.146173in}{3.233555in}}%
\pgfpathcurveto{\pgfqpoint{2.146173in}{3.239080in}}{\pgfqpoint{2.143978in}{3.244379in}}{\pgfqpoint{2.140071in}{3.248286in}}%
\pgfpathcurveto{\pgfqpoint{2.136164in}{3.252193in}}{\pgfqpoint{2.130865in}{3.254388in}}{\pgfqpoint{2.125339in}{3.254388in}}%
\pgfpathcurveto{\pgfqpoint{2.119814in}{3.254388in}}{\pgfqpoint{2.114515in}{3.252193in}}{\pgfqpoint{2.110608in}{3.248286in}}%
\pgfpathcurveto{\pgfqpoint{2.106701in}{3.244379in}}{\pgfqpoint{2.104506in}{3.239080in}}{\pgfqpoint{2.104506in}{3.233555in}}%
\pgfpathcurveto{\pgfqpoint{2.104506in}{3.228030in}}{\pgfqpoint{2.106701in}{3.222730in}}{\pgfqpoint{2.110608in}{3.218823in}}%
\pgfpathcurveto{\pgfqpoint{2.114515in}{3.214917in}}{\pgfqpoint{2.119814in}{3.212721in}}{\pgfqpoint{2.125339in}{3.212721in}}%
\pgfpathclose%
\pgfusepath{fill}%
\end{pgfscope}%
\begin{pgfscope}%
\pgfsetrectcap%
\pgfsetmiterjoin%
\pgfsetlinewidth{0.803000pt}%
\definecolor{currentstroke}{rgb}{0.501961,0.501961,0.501961}%
\pgfsetstrokecolor{currentstroke}%
\pgfsetdash{}{0pt}%
\pgfpathmoveto{\pgfqpoint{2.051725in}{3.178832in}}%
\pgfpathlineto{\pgfqpoint{2.051725in}{3.933832in}}%
\pgfusepath{stroke}%
\end{pgfscope}%
\begin{pgfscope}%
\pgfsetrectcap%
\pgfsetmiterjoin%
\pgfsetlinewidth{0.803000pt}%
\definecolor{currentstroke}{rgb}{0.501961,0.501961,0.501961}%
\pgfsetstrokecolor{currentstroke}%
\pgfsetdash{}{0pt}%
\pgfpathmoveto{\pgfqpoint{3.214225in}{3.178832in}}%
\pgfpathlineto{\pgfqpoint{3.214225in}{3.933832in}}%
\pgfusepath{stroke}%
\end{pgfscope}%
\begin{pgfscope}%
\pgfsetrectcap%
\pgfsetmiterjoin%
\pgfsetlinewidth{0.803000pt}%
\definecolor{currentstroke}{rgb}{0.501961,0.501961,0.501961}%
\pgfsetstrokecolor{currentstroke}%
\pgfsetdash{}{0pt}%
\pgfpathmoveto{\pgfqpoint{2.051725in}{3.178832in}}%
\pgfpathlineto{\pgfqpoint{3.214225in}{3.178832in}}%
\pgfusepath{stroke}%
\end{pgfscope}%
\begin{pgfscope}%
\pgfsetrectcap%
\pgfsetmiterjoin%
\pgfsetlinewidth{0.803000pt}%
\definecolor{currentstroke}{rgb}{0.501961,0.501961,0.501961}%
\pgfsetstrokecolor{currentstroke}%
\pgfsetdash{}{0pt}%
\pgfpathmoveto{\pgfqpoint{2.051725in}{3.933832in}}%
\pgfpathlineto{\pgfqpoint{3.214225in}{3.933832in}}%
\pgfusepath{stroke}%
\end{pgfscope}%
\begin{pgfscope}%
\pgfsetbuttcap%
\pgfsetmiterjoin%
\definecolor{currentfill}{rgb}{1.000000,1.000000,1.000000}%
\pgfsetfillcolor{currentfill}%
\pgfsetlinewidth{0.000000pt}%
\definecolor{currentstroke}{rgb}{0.000000,0.000000,0.000000}%
\pgfsetstrokecolor{currentstroke}%
\pgfsetstrokeopacity{0.000000}%
\pgfsetdash{}{0pt}%
\pgfpathmoveto{\pgfqpoint{3.214225in}{3.178832in}}%
\pgfpathlineto{\pgfqpoint{4.376725in}{3.178832in}}%
\pgfpathlineto{\pgfqpoint{4.376725in}{3.933832in}}%
\pgfpathlineto{\pgfqpoint{3.214225in}{3.933832in}}%
\pgfpathclose%
\pgfusepath{fill}%
\end{pgfscope}%
\begin{pgfscope}%
\pgfpathrectangle{\pgfqpoint{3.214225in}{3.178832in}}{\pgfqpoint{1.162500in}{0.755000in}}%
\pgfusepath{clip}%
\pgfsetbuttcap%
\pgfsetroundjoin%
\definecolor{currentfill}{rgb}{0.000000,0.000000,0.000000}%
\pgfsetfillcolor{currentfill}%
\pgfsetfillopacity{0.500000}%
\pgfsetlinewidth{0.000000pt}%
\definecolor{currentstroke}{rgb}{0.000000,0.000000,0.000000}%
\pgfsetstrokecolor{currentstroke}%
\pgfsetdash{}{0pt}%
\pgfpathmoveto{\pgfqpoint{3.656364in}{3.810398in}}%
\pgfpathcurveto{\pgfqpoint{3.661889in}{3.810398in}}{\pgfqpoint{3.667189in}{3.812593in}}{\pgfqpoint{3.671096in}{3.816499in}}%
\pgfpathcurveto{\pgfqpoint{3.675002in}{3.820406in}}{\pgfqpoint{3.677198in}{3.825706in}}{\pgfqpoint{3.677198in}{3.831231in}}%
\pgfpathcurveto{\pgfqpoint{3.677198in}{3.836756in}}{\pgfqpoint{3.675002in}{3.842055in}}{\pgfqpoint{3.671096in}{3.845962in}}%
\pgfpathcurveto{\pgfqpoint{3.667189in}{3.849869in}}{\pgfqpoint{3.661889in}{3.852064in}}{\pgfqpoint{3.656364in}{3.852064in}}%
\pgfpathcurveto{\pgfqpoint{3.650839in}{3.852064in}}{\pgfqpoint{3.645540in}{3.849869in}}{\pgfqpoint{3.641633in}{3.845962in}}%
\pgfpathcurveto{\pgfqpoint{3.637726in}{3.842055in}}{\pgfqpoint{3.635531in}{3.836756in}}{\pgfqpoint{3.635531in}{3.831231in}}%
\pgfpathcurveto{\pgfqpoint{3.635531in}{3.825706in}}{\pgfqpoint{3.637726in}{3.820406in}}{\pgfqpoint{3.641633in}{3.816499in}}%
\pgfpathcurveto{\pgfqpoint{3.645540in}{3.812593in}}{\pgfqpoint{3.650839in}{3.810398in}}{\pgfqpoint{3.656364in}{3.810398in}}%
\pgfpathclose%
\pgfusepath{fill}%
\end{pgfscope}%
\begin{pgfscope}%
\pgfpathrectangle{\pgfqpoint{3.214225in}{3.178832in}}{\pgfqpoint{1.162500in}{0.755000in}}%
\pgfusepath{clip}%
\pgfsetbuttcap%
\pgfsetroundjoin%
\definecolor{currentfill}{rgb}{0.000000,0.000000,0.000000}%
\pgfsetfillcolor{currentfill}%
\pgfsetfillopacity{0.500000}%
\pgfsetlinewidth{0.000000pt}%
\definecolor{currentstroke}{rgb}{0.000000,0.000000,0.000000}%
\pgfsetstrokecolor{currentstroke}%
\pgfsetdash{}{0pt}%
\pgfpathmoveto{\pgfqpoint{4.259629in}{3.539667in}}%
\pgfpathcurveto{\pgfqpoint{4.265154in}{3.539667in}}{\pgfqpoint{4.270454in}{3.541862in}}{\pgfqpoint{4.274361in}{3.545769in}}%
\pgfpathcurveto{\pgfqpoint{4.278268in}{3.549675in}}{\pgfqpoint{4.280463in}{3.554975in}}{\pgfqpoint{4.280463in}{3.560500in}}%
\pgfpathcurveto{\pgfqpoint{4.280463in}{3.566025in}}{\pgfqpoint{4.278268in}{3.571325in}}{\pgfqpoint{4.274361in}{3.575231in}}%
\pgfpathcurveto{\pgfqpoint{4.270454in}{3.579138in}}{\pgfqpoint{4.265154in}{3.581333in}}{\pgfqpoint{4.259629in}{3.581333in}}%
\pgfpathcurveto{\pgfqpoint{4.254104in}{3.581333in}}{\pgfqpoint{4.248805in}{3.579138in}}{\pgfqpoint{4.244898in}{3.575231in}}%
\pgfpathcurveto{\pgfqpoint{4.240991in}{3.571325in}}{\pgfqpoint{4.238796in}{3.566025in}}{\pgfqpoint{4.238796in}{3.560500in}}%
\pgfpathcurveto{\pgfqpoint{4.238796in}{3.554975in}}{\pgfqpoint{4.240991in}{3.549675in}}{\pgfqpoint{4.244898in}{3.545769in}}%
\pgfpathcurveto{\pgfqpoint{4.248805in}{3.541862in}}{\pgfqpoint{4.254104in}{3.539667in}}{\pgfqpoint{4.259629in}{3.539667in}}%
\pgfpathclose%
\pgfusepath{fill}%
\end{pgfscope}%
\begin{pgfscope}%
\pgfpathrectangle{\pgfqpoint{3.214225in}{3.178832in}}{\pgfqpoint{1.162500in}{0.755000in}}%
\pgfusepath{clip}%
\pgfsetbuttcap%
\pgfsetroundjoin%
\definecolor{currentfill}{rgb}{0.000000,0.000000,0.000000}%
\pgfsetfillcolor{currentfill}%
\pgfsetfillopacity{0.500000}%
\pgfsetlinewidth{0.000000pt}%
\definecolor{currentstroke}{rgb}{0.000000,0.000000,0.000000}%
\pgfsetstrokecolor{currentstroke}%
\pgfsetdash{}{0pt}%
\pgfpathmoveto{\pgfqpoint{4.349046in}{3.541019in}}%
\pgfpathcurveto{\pgfqpoint{4.354571in}{3.541019in}}{\pgfqpoint{4.359871in}{3.543214in}}{\pgfqpoint{4.363778in}{3.547121in}}%
\pgfpathcurveto{\pgfqpoint{4.367685in}{3.551028in}}{\pgfqpoint{4.369880in}{3.556327in}}{\pgfqpoint{4.369880in}{3.561852in}}%
\pgfpathcurveto{\pgfqpoint{4.369880in}{3.567377in}}{\pgfqpoint{4.367685in}{3.572677in}}{\pgfqpoint{4.363778in}{3.576584in}}%
\pgfpathcurveto{\pgfqpoint{4.359871in}{3.580490in}}{\pgfqpoint{4.354571in}{3.582685in}}{\pgfqpoint{4.349046in}{3.582685in}}%
\pgfpathcurveto{\pgfqpoint{4.343521in}{3.582685in}}{\pgfqpoint{4.338222in}{3.580490in}}{\pgfqpoint{4.334315in}{3.576584in}}%
\pgfpathcurveto{\pgfqpoint{4.330408in}{3.572677in}}{\pgfqpoint{4.328213in}{3.567377in}}{\pgfqpoint{4.328213in}{3.561852in}}%
\pgfpathcurveto{\pgfqpoint{4.328213in}{3.556327in}}{\pgfqpoint{4.330408in}{3.551028in}}{\pgfqpoint{4.334315in}{3.547121in}}%
\pgfpathcurveto{\pgfqpoint{4.338222in}{3.543214in}}{\pgfqpoint{4.343521in}{3.541019in}}{\pgfqpoint{4.349046in}{3.541019in}}%
\pgfpathclose%
\pgfusepath{fill}%
\end{pgfscope}%
\begin{pgfscope}%
\pgfpathrectangle{\pgfqpoint{3.214225in}{3.178832in}}{\pgfqpoint{1.162500in}{0.755000in}}%
\pgfusepath{clip}%
\pgfsetbuttcap%
\pgfsetroundjoin%
\definecolor{currentfill}{rgb}{0.000000,0.000000,0.000000}%
\pgfsetfillcolor{currentfill}%
\pgfsetfillopacity{0.500000}%
\pgfsetlinewidth{0.000000pt}%
\definecolor{currentstroke}{rgb}{0.000000,0.000000,0.000000}%
\pgfsetstrokecolor{currentstroke}%
\pgfsetdash{}{0pt}%
\pgfpathmoveto{\pgfqpoint{3.915823in}{3.308342in}}%
\pgfpathcurveto{\pgfqpoint{3.921348in}{3.308342in}}{\pgfqpoint{3.926648in}{3.310537in}}{\pgfqpoint{3.930554in}{3.314444in}}%
\pgfpathcurveto{\pgfqpoint{3.934461in}{3.318351in}}{\pgfqpoint{3.936656in}{3.323651in}}{\pgfqpoint{3.936656in}{3.329176in}}%
\pgfpathcurveto{\pgfqpoint{3.936656in}{3.334701in}}{\pgfqpoint{3.934461in}{3.340000in}}{\pgfqpoint{3.930554in}{3.343907in}}%
\pgfpathcurveto{\pgfqpoint{3.926648in}{3.347814in}}{\pgfqpoint{3.921348in}{3.350009in}}{\pgfqpoint{3.915823in}{3.350009in}}%
\pgfpathcurveto{\pgfqpoint{3.910298in}{3.350009in}}{\pgfqpoint{3.904998in}{3.347814in}}{\pgfqpoint{3.901092in}{3.343907in}}%
\pgfpathcurveto{\pgfqpoint{3.897185in}{3.340000in}}{\pgfqpoint{3.894990in}{3.334701in}}{\pgfqpoint{3.894990in}{3.329176in}}%
\pgfpathcurveto{\pgfqpoint{3.894990in}{3.323651in}}{\pgfqpoint{3.897185in}{3.318351in}}{\pgfqpoint{3.901092in}{3.314444in}}%
\pgfpathcurveto{\pgfqpoint{3.904998in}{3.310537in}}{\pgfqpoint{3.910298in}{3.308342in}}{\pgfqpoint{3.915823in}{3.308342in}}%
\pgfpathclose%
\pgfusepath{fill}%
\end{pgfscope}%
\begin{pgfscope}%
\pgfpathrectangle{\pgfqpoint{3.214225in}{3.178832in}}{\pgfqpoint{1.162500in}{0.755000in}}%
\pgfusepath{clip}%
\pgfsetbuttcap%
\pgfsetroundjoin%
\definecolor{currentfill}{rgb}{0.000000,0.000000,0.000000}%
\pgfsetfillcolor{currentfill}%
\pgfsetfillopacity{0.500000}%
\pgfsetlinewidth{0.000000pt}%
\definecolor{currentstroke}{rgb}{0.000000,0.000000,0.000000}%
\pgfsetstrokecolor{currentstroke}%
\pgfsetdash{}{0pt}%
\pgfpathmoveto{\pgfqpoint{3.986481in}{3.311937in}}%
\pgfpathcurveto{\pgfqpoint{3.992006in}{3.311937in}}{\pgfqpoint{3.997306in}{3.314132in}}{\pgfqpoint{4.001213in}{3.318039in}}%
\pgfpathcurveto{\pgfqpoint{4.005120in}{3.321945in}}{\pgfqpoint{4.007315in}{3.327245in}}{\pgfqpoint{4.007315in}{3.332770in}}%
\pgfpathcurveto{\pgfqpoint{4.007315in}{3.338295in}}{\pgfqpoint{4.005120in}{3.343595in}}{\pgfqpoint{4.001213in}{3.347501in}}%
\pgfpathcurveto{\pgfqpoint{3.997306in}{3.351408in}}{\pgfqpoint{3.992006in}{3.353603in}}{\pgfqpoint{3.986481in}{3.353603in}}%
\pgfpathcurveto{\pgfqpoint{3.980956in}{3.353603in}}{\pgfqpoint{3.975657in}{3.351408in}}{\pgfqpoint{3.971750in}{3.347501in}}%
\pgfpathcurveto{\pgfqpoint{3.967843in}{3.343595in}}{\pgfqpoint{3.965648in}{3.338295in}}{\pgfqpoint{3.965648in}{3.332770in}}%
\pgfpathcurveto{\pgfqpoint{3.965648in}{3.327245in}}{\pgfqpoint{3.967843in}{3.321945in}}{\pgfqpoint{3.971750in}{3.318039in}}%
\pgfpathcurveto{\pgfqpoint{3.975657in}{3.314132in}}{\pgfqpoint{3.980956in}{3.311937in}}{\pgfqpoint{3.986481in}{3.311937in}}%
\pgfpathclose%
\pgfusepath{fill}%
\end{pgfscope}%
\begin{pgfscope}%
\pgfpathrectangle{\pgfqpoint{3.214225in}{3.178832in}}{\pgfqpoint{1.162500in}{0.755000in}}%
\pgfusepath{clip}%
\pgfsetbuttcap%
\pgfsetroundjoin%
\definecolor{currentfill}{rgb}{0.000000,0.000000,0.000000}%
\pgfsetfillcolor{currentfill}%
\pgfsetfillopacity{0.500000}%
\pgfsetlinewidth{0.000000pt}%
\definecolor{currentstroke}{rgb}{0.000000,0.000000,0.000000}%
\pgfsetstrokecolor{currentstroke}%
\pgfsetdash{}{0pt}%
\pgfpathmoveto{\pgfqpoint{3.831974in}{3.215463in}}%
\pgfpathcurveto{\pgfqpoint{3.837499in}{3.215463in}}{\pgfqpoint{3.842798in}{3.217658in}}{\pgfqpoint{3.846705in}{3.221565in}}%
\pgfpathcurveto{\pgfqpoint{3.850612in}{3.225471in}}{\pgfqpoint{3.852807in}{3.230771in}}{\pgfqpoint{3.852807in}{3.236296in}}%
\pgfpathcurveto{\pgfqpoint{3.852807in}{3.241821in}}{\pgfqpoint{3.850612in}{3.247121in}}{\pgfqpoint{3.846705in}{3.251027in}}%
\pgfpathcurveto{\pgfqpoint{3.842798in}{3.254934in}}{\pgfqpoint{3.837499in}{3.257129in}}{\pgfqpoint{3.831974in}{3.257129in}}%
\pgfpathcurveto{\pgfqpoint{3.826449in}{3.257129in}}{\pgfqpoint{3.821149in}{3.254934in}}{\pgfqpoint{3.817242in}{3.251027in}}%
\pgfpathcurveto{\pgfqpoint{3.813336in}{3.247121in}}{\pgfqpoint{3.811140in}{3.241821in}}{\pgfqpoint{3.811140in}{3.236296in}}%
\pgfpathcurveto{\pgfqpoint{3.811140in}{3.230771in}}{\pgfqpoint{3.813336in}{3.225471in}}{\pgfqpoint{3.817242in}{3.221565in}}%
\pgfpathcurveto{\pgfqpoint{3.821149in}{3.217658in}}{\pgfqpoint{3.826449in}{3.215463in}}{\pgfqpoint{3.831974in}{3.215463in}}%
\pgfpathclose%
\pgfusepath{fill}%
\end{pgfscope}%
\begin{pgfscope}%
\pgfpathrectangle{\pgfqpoint{3.214225in}{3.178832in}}{\pgfqpoint{1.162500in}{0.755000in}}%
\pgfusepath{clip}%
\pgfsetbuttcap%
\pgfsetroundjoin%
\definecolor{currentfill}{rgb}{0.000000,0.000000,0.000000}%
\pgfsetfillcolor{currentfill}%
\pgfsetfillopacity{0.500000}%
\pgfsetlinewidth{0.000000pt}%
\definecolor{currentstroke}{rgb}{0.000000,0.000000,0.000000}%
\pgfsetstrokecolor{currentstroke}%
\pgfsetdash{}{0pt}%
\pgfpathmoveto{\pgfqpoint{3.241904in}{3.175975in}}%
\pgfpathcurveto{\pgfqpoint{3.247429in}{3.175975in}}{\pgfqpoint{3.252728in}{3.178170in}}{\pgfqpoint{3.256635in}{3.182077in}}%
\pgfpathcurveto{\pgfqpoint{3.260542in}{3.185984in}}{\pgfqpoint{3.262737in}{3.191283in}}{\pgfqpoint{3.262737in}{3.196809in}}%
\pgfpathcurveto{\pgfqpoint{3.262737in}{3.202334in}}{\pgfqpoint{3.260542in}{3.207633in}}{\pgfqpoint{3.256635in}{3.211540in}}%
\pgfpathcurveto{\pgfqpoint{3.252728in}{3.215447in}}{\pgfqpoint{3.247429in}{3.217642in}}{\pgfqpoint{3.241904in}{3.217642in}}%
\pgfpathcurveto{\pgfqpoint{3.236379in}{3.217642in}}{\pgfqpoint{3.231079in}{3.215447in}}{\pgfqpoint{3.227172in}{3.211540in}}%
\pgfpathcurveto{\pgfqpoint{3.223265in}{3.207633in}}{\pgfqpoint{3.221070in}{3.202334in}}{\pgfqpoint{3.221070in}{3.196809in}}%
\pgfpathcurveto{\pgfqpoint{3.221070in}{3.191283in}}{\pgfqpoint{3.223265in}{3.185984in}}{\pgfqpoint{3.227172in}{3.182077in}}%
\pgfpathcurveto{\pgfqpoint{3.231079in}{3.178170in}}{\pgfqpoint{3.236379in}{3.175975in}}{\pgfqpoint{3.241904in}{3.175975in}}%
\pgfpathclose%
\pgfusepath{fill}%
\end{pgfscope}%
\begin{pgfscope}%
\pgfpathrectangle{\pgfqpoint{3.214225in}{3.178832in}}{\pgfqpoint{1.162500in}{0.755000in}}%
\pgfusepath{clip}%
\pgfsetbuttcap%
\pgfsetroundjoin%
\definecolor{currentfill}{rgb}{0.000000,0.000000,0.000000}%
\pgfsetfillcolor{currentfill}%
\pgfsetfillopacity{0.500000}%
\pgfsetlinewidth{0.000000pt}%
\definecolor{currentstroke}{rgb}{0.000000,0.000000,0.000000}%
\pgfsetstrokecolor{currentstroke}%
\pgfsetdash{}{0pt}%
\pgfpathmoveto{\pgfqpoint{4.220724in}{3.678982in}}%
\pgfpathcurveto{\pgfqpoint{4.226249in}{3.678982in}}{\pgfqpoint{4.231548in}{3.681177in}}{\pgfqpoint{4.235455in}{3.685084in}}%
\pgfpathcurveto{\pgfqpoint{4.239362in}{3.688990in}}{\pgfqpoint{4.241557in}{3.694290in}}{\pgfqpoint{4.241557in}{3.699815in}}%
\pgfpathcurveto{\pgfqpoint{4.241557in}{3.705340in}}{\pgfqpoint{4.239362in}{3.710640in}}{\pgfqpoint{4.235455in}{3.714546in}}%
\pgfpathcurveto{\pgfqpoint{4.231548in}{3.718453in}}{\pgfqpoint{4.226249in}{3.720648in}}{\pgfqpoint{4.220724in}{3.720648in}}%
\pgfpathcurveto{\pgfqpoint{4.215199in}{3.720648in}}{\pgfqpoint{4.209899in}{3.718453in}}{\pgfqpoint{4.205992in}{3.714546in}}%
\pgfpathcurveto{\pgfqpoint{4.202086in}{3.710640in}}{\pgfqpoint{4.199890in}{3.705340in}}{\pgfqpoint{4.199890in}{3.699815in}}%
\pgfpathcurveto{\pgfqpoint{4.199890in}{3.694290in}}{\pgfqpoint{4.202086in}{3.688990in}}{\pgfqpoint{4.205992in}{3.685084in}}%
\pgfpathcurveto{\pgfqpoint{4.209899in}{3.681177in}}{\pgfqpoint{4.215199in}{3.678982in}}{\pgfqpoint{4.220724in}{3.678982in}}%
\pgfpathclose%
\pgfusepath{fill}%
\end{pgfscope}%
\begin{pgfscope}%
\pgfpathrectangle{\pgfqpoint{3.214225in}{3.178832in}}{\pgfqpoint{1.162500in}{0.755000in}}%
\pgfusepath{clip}%
\pgfsetbuttcap%
\pgfsetroundjoin%
\definecolor{currentfill}{rgb}{0.000000,0.000000,0.000000}%
\pgfsetfillcolor{currentfill}%
\pgfsetfillopacity{0.500000}%
\pgfsetlinewidth{0.000000pt}%
\definecolor{currentstroke}{rgb}{0.000000,0.000000,0.000000}%
\pgfsetstrokecolor{currentstroke}%
\pgfsetdash{}{0pt}%
\pgfpathmoveto{\pgfqpoint{3.980838in}{3.528997in}}%
\pgfpathcurveto{\pgfqpoint{3.986364in}{3.528997in}}{\pgfqpoint{3.991663in}{3.531192in}}{\pgfqpoint{3.995570in}{3.535099in}}%
\pgfpathcurveto{\pgfqpoint{3.999477in}{3.539005in}}{\pgfqpoint{4.001672in}{3.544305in}}{\pgfqpoint{4.001672in}{3.549830in}}%
\pgfpathcurveto{\pgfqpoint{4.001672in}{3.555355in}}{\pgfqpoint{3.999477in}{3.560655in}}{\pgfqpoint{3.995570in}{3.564561in}}%
\pgfpathcurveto{\pgfqpoint{3.991663in}{3.568468in}}{\pgfqpoint{3.986364in}{3.570663in}}{\pgfqpoint{3.980838in}{3.570663in}}%
\pgfpathcurveto{\pgfqpoint{3.975313in}{3.570663in}}{\pgfqpoint{3.970014in}{3.568468in}}{\pgfqpoint{3.966107in}{3.564561in}}%
\pgfpathcurveto{\pgfqpoint{3.962200in}{3.560655in}}{\pgfqpoint{3.960005in}{3.555355in}}{\pgfqpoint{3.960005in}{3.549830in}}%
\pgfpathcurveto{\pgfqpoint{3.960005in}{3.544305in}}{\pgfqpoint{3.962200in}{3.539005in}}{\pgfqpoint{3.966107in}{3.535099in}}%
\pgfpathcurveto{\pgfqpoint{3.970014in}{3.531192in}}{\pgfqpoint{3.975313in}{3.528997in}}{\pgfqpoint{3.980838in}{3.528997in}}%
\pgfpathclose%
\pgfusepath{fill}%
\end{pgfscope}%
\begin{pgfscope}%
\pgfpathrectangle{\pgfqpoint{3.214225in}{3.178832in}}{\pgfqpoint{1.162500in}{0.755000in}}%
\pgfusepath{clip}%
\pgfsetbuttcap%
\pgfsetroundjoin%
\definecolor{currentfill}{rgb}{0.000000,0.000000,0.000000}%
\pgfsetfillcolor{currentfill}%
\pgfsetfillopacity{0.500000}%
\pgfsetlinewidth{0.000000pt}%
\definecolor{currentstroke}{rgb}{0.000000,0.000000,0.000000}%
\pgfsetstrokecolor{currentstroke}%
\pgfsetdash{}{0pt}%
\pgfpathmoveto{\pgfqpoint{3.708643in}{3.426855in}}%
\pgfpathcurveto{\pgfqpoint{3.714168in}{3.426855in}}{\pgfqpoint{3.719468in}{3.429051in}}{\pgfqpoint{3.723375in}{3.432957in}}%
\pgfpathcurveto{\pgfqpoint{3.727282in}{3.436864in}}{\pgfqpoint{3.729477in}{3.442164in}}{\pgfqpoint{3.729477in}{3.447689in}}%
\pgfpathcurveto{\pgfqpoint{3.729477in}{3.453214in}}{\pgfqpoint{3.727282in}{3.458513in}}{\pgfqpoint{3.723375in}{3.462420in}}%
\pgfpathcurveto{\pgfqpoint{3.719468in}{3.466327in}}{\pgfqpoint{3.714168in}{3.468522in}}{\pgfqpoint{3.708643in}{3.468522in}}%
\pgfpathcurveto{\pgfqpoint{3.703118in}{3.468522in}}{\pgfqpoint{3.697819in}{3.466327in}}{\pgfqpoint{3.693912in}{3.462420in}}%
\pgfpathcurveto{\pgfqpoint{3.690005in}{3.458513in}}{\pgfqpoint{3.687810in}{3.453214in}}{\pgfqpoint{3.687810in}{3.447689in}}%
\pgfpathcurveto{\pgfqpoint{3.687810in}{3.442164in}}{\pgfqpoint{3.690005in}{3.436864in}}{\pgfqpoint{3.693912in}{3.432957in}}%
\pgfpathcurveto{\pgfqpoint{3.697819in}{3.429051in}}{\pgfqpoint{3.703118in}{3.426855in}}{\pgfqpoint{3.708643in}{3.426855in}}%
\pgfpathclose%
\pgfusepath{fill}%
\end{pgfscope}%
\begin{pgfscope}%
\pgfpathrectangle{\pgfqpoint{3.214225in}{3.178832in}}{\pgfqpoint{1.162500in}{0.755000in}}%
\pgfusepath{clip}%
\pgfsetbuttcap%
\pgfsetroundjoin%
\definecolor{currentfill}{rgb}{0.000000,0.000000,0.000000}%
\pgfsetfillcolor{currentfill}%
\pgfsetfillopacity{0.500000}%
\pgfsetlinewidth{0.000000pt}%
\definecolor{currentstroke}{rgb}{0.000000,0.000000,0.000000}%
\pgfsetstrokecolor{currentstroke}%
\pgfsetdash{}{0pt}%
\pgfpathmoveto{\pgfqpoint{4.124371in}{3.677650in}}%
\pgfpathcurveto{\pgfqpoint{4.129896in}{3.677650in}}{\pgfqpoint{4.135195in}{3.679845in}}{\pgfqpoint{4.139102in}{3.683752in}}%
\pgfpathcurveto{\pgfqpoint{4.143009in}{3.687659in}}{\pgfqpoint{4.145204in}{3.692959in}}{\pgfqpoint{4.145204in}{3.698484in}}%
\pgfpathcurveto{\pgfqpoint{4.145204in}{3.704009in}}{\pgfqpoint{4.143009in}{3.709308in}}{\pgfqpoint{4.139102in}{3.713215in}}%
\pgfpathcurveto{\pgfqpoint{4.135195in}{3.717122in}}{\pgfqpoint{4.129896in}{3.719317in}}{\pgfqpoint{4.124371in}{3.719317in}}%
\pgfpathcurveto{\pgfqpoint{4.118846in}{3.719317in}}{\pgfqpoint{4.113546in}{3.717122in}}{\pgfqpoint{4.109639in}{3.713215in}}%
\pgfpathcurveto{\pgfqpoint{4.105732in}{3.709308in}}{\pgfqpoint{4.103537in}{3.704009in}}{\pgfqpoint{4.103537in}{3.698484in}}%
\pgfpathcurveto{\pgfqpoint{4.103537in}{3.692959in}}{\pgfqpoint{4.105732in}{3.687659in}}{\pgfqpoint{4.109639in}{3.683752in}}%
\pgfpathcurveto{\pgfqpoint{4.113546in}{3.679845in}}{\pgfqpoint{4.118846in}{3.677650in}}{\pgfqpoint{4.124371in}{3.677650in}}%
\pgfpathclose%
\pgfusepath{fill}%
\end{pgfscope}%
\begin{pgfscope}%
\pgfpathrectangle{\pgfqpoint{3.214225in}{3.178832in}}{\pgfqpoint{1.162500in}{0.755000in}}%
\pgfusepath{clip}%
\pgfsetbuttcap%
\pgfsetroundjoin%
\definecolor{currentfill}{rgb}{0.000000,0.000000,0.000000}%
\pgfsetfillcolor{currentfill}%
\pgfsetfillopacity{0.500000}%
\pgfsetlinewidth{0.000000pt}%
\definecolor{currentstroke}{rgb}{0.000000,0.000000,0.000000}%
\pgfsetstrokecolor{currentstroke}%
\pgfsetdash{}{0pt}%
\pgfpathmoveto{\pgfqpoint{3.695649in}{3.296446in}}%
\pgfpathcurveto{\pgfqpoint{3.701174in}{3.296446in}}{\pgfqpoint{3.706474in}{3.298641in}}{\pgfqpoint{3.710381in}{3.302548in}}%
\pgfpathcurveto{\pgfqpoint{3.714288in}{3.306455in}}{\pgfqpoint{3.716483in}{3.311754in}}{\pgfqpoint{3.716483in}{3.317279in}}%
\pgfpathcurveto{\pgfqpoint{3.716483in}{3.322804in}}{\pgfqpoint{3.714288in}{3.328104in}}{\pgfqpoint{3.710381in}{3.332011in}}%
\pgfpathcurveto{\pgfqpoint{3.706474in}{3.335917in}}{\pgfqpoint{3.701174in}{3.338112in}}{\pgfqpoint{3.695649in}{3.338112in}}%
\pgfpathcurveto{\pgfqpoint{3.690124in}{3.338112in}}{\pgfqpoint{3.684825in}{3.335917in}}{\pgfqpoint{3.680918in}{3.332011in}}%
\pgfpathcurveto{\pgfqpoint{3.677011in}{3.328104in}}{\pgfqpoint{3.674816in}{3.322804in}}{\pgfqpoint{3.674816in}{3.317279in}}%
\pgfpathcurveto{\pgfqpoint{3.674816in}{3.311754in}}{\pgfqpoint{3.677011in}{3.306455in}}{\pgfqpoint{3.680918in}{3.302548in}}%
\pgfpathcurveto{\pgfqpoint{3.684825in}{3.298641in}}{\pgfqpoint{3.690124in}{3.296446in}}{\pgfqpoint{3.695649in}{3.296446in}}%
\pgfpathclose%
\pgfusepath{fill}%
\end{pgfscope}%
\begin{pgfscope}%
\pgfpathrectangle{\pgfqpoint{3.214225in}{3.178832in}}{\pgfqpoint{1.162500in}{0.755000in}}%
\pgfusepath{clip}%
\pgfsetbuttcap%
\pgfsetroundjoin%
\definecolor{currentfill}{rgb}{0.000000,0.000000,0.000000}%
\pgfsetfillcolor{currentfill}%
\pgfsetfillopacity{0.500000}%
\pgfsetlinewidth{0.000000pt}%
\definecolor{currentstroke}{rgb}{0.000000,0.000000,0.000000}%
\pgfsetstrokecolor{currentstroke}%
\pgfsetdash{}{0pt}%
\pgfpathmoveto{\pgfqpoint{3.329643in}{3.322937in}}%
\pgfpathcurveto{\pgfqpoint{3.335168in}{3.322937in}}{\pgfqpoint{3.340468in}{3.325133in}}{\pgfqpoint{3.344375in}{3.329039in}}%
\pgfpathcurveto{\pgfqpoint{3.348281in}{3.332946in}}{\pgfqpoint{3.350476in}{3.338246in}}{\pgfqpoint{3.350476in}{3.343771in}}%
\pgfpathcurveto{\pgfqpoint{3.350476in}{3.349296in}}{\pgfqpoint{3.348281in}{3.354595in}}{\pgfqpoint{3.344375in}{3.358502in}}%
\pgfpathcurveto{\pgfqpoint{3.340468in}{3.362409in}}{\pgfqpoint{3.335168in}{3.364604in}}{\pgfqpoint{3.329643in}{3.364604in}}%
\pgfpathcurveto{\pgfqpoint{3.324118in}{3.364604in}}{\pgfqpoint{3.318819in}{3.362409in}}{\pgfqpoint{3.314912in}{3.358502in}}%
\pgfpathcurveto{\pgfqpoint{3.311005in}{3.354595in}}{\pgfqpoint{3.308810in}{3.349296in}}{\pgfqpoint{3.308810in}{3.343771in}}%
\pgfpathcurveto{\pgfqpoint{3.308810in}{3.338246in}}{\pgfqpoint{3.311005in}{3.332946in}}{\pgfqpoint{3.314912in}{3.329039in}}%
\pgfpathcurveto{\pgfqpoint{3.318819in}{3.325133in}}{\pgfqpoint{3.324118in}{3.322937in}}{\pgfqpoint{3.329643in}{3.322937in}}%
\pgfpathclose%
\pgfusepath{fill}%
\end{pgfscope}%
\begin{pgfscope}%
\pgfpathrectangle{\pgfqpoint{3.214225in}{3.178832in}}{\pgfqpoint{1.162500in}{0.755000in}}%
\pgfusepath{clip}%
\pgfsetbuttcap%
\pgfsetroundjoin%
\definecolor{currentfill}{rgb}{0.000000,0.000000,0.000000}%
\pgfsetfillcolor{currentfill}%
\pgfsetfillopacity{0.500000}%
\pgfsetlinewidth{0.000000pt}%
\definecolor{currentstroke}{rgb}{0.000000,0.000000,0.000000}%
\pgfsetstrokecolor{currentstroke}%
\pgfsetdash{}{0pt}%
\pgfpathmoveto{\pgfqpoint{4.167862in}{3.895023in}}%
\pgfpathcurveto{\pgfqpoint{4.173387in}{3.895023in}}{\pgfqpoint{4.178687in}{3.897218in}}{\pgfqpoint{4.182593in}{3.901125in}}%
\pgfpathcurveto{\pgfqpoint{4.186500in}{3.905032in}}{\pgfqpoint{4.188695in}{3.910331in}}{\pgfqpoint{4.188695in}{3.915856in}}%
\pgfpathcurveto{\pgfqpoint{4.188695in}{3.921381in}}{\pgfqpoint{4.186500in}{3.926681in}}{\pgfqpoint{4.182593in}{3.930588in}}%
\pgfpathcurveto{\pgfqpoint{4.178687in}{3.934494in}}{\pgfqpoint{4.173387in}{3.936690in}}{\pgfqpoint{4.167862in}{3.936690in}}%
\pgfpathcurveto{\pgfqpoint{4.162337in}{3.936690in}}{\pgfqpoint{4.157037in}{3.934494in}}{\pgfqpoint{4.153131in}{3.930588in}}%
\pgfpathcurveto{\pgfqpoint{4.149224in}{3.926681in}}{\pgfqpoint{4.147029in}{3.921381in}}{\pgfqpoint{4.147029in}{3.915856in}}%
\pgfpathcurveto{\pgfqpoint{4.147029in}{3.910331in}}{\pgfqpoint{4.149224in}{3.905032in}}{\pgfqpoint{4.153131in}{3.901125in}}%
\pgfpathcurveto{\pgfqpoint{4.157037in}{3.897218in}}{\pgfqpoint{4.162337in}{3.895023in}}{\pgfqpoint{4.167862in}{3.895023in}}%
\pgfpathclose%
\pgfusepath{fill}%
\end{pgfscope}%
\begin{pgfscope}%
\pgfpathrectangle{\pgfqpoint{3.214225in}{3.178832in}}{\pgfqpoint{1.162500in}{0.755000in}}%
\pgfusepath{clip}%
\pgfsetbuttcap%
\pgfsetroundjoin%
\definecolor{currentfill}{rgb}{0.000000,0.000000,0.000000}%
\pgfsetfillcolor{currentfill}%
\pgfsetfillopacity{0.500000}%
\pgfsetlinewidth{0.000000pt}%
\definecolor{currentstroke}{rgb}{0.000000,0.000000,0.000000}%
\pgfsetstrokecolor{currentstroke}%
\pgfsetdash{}{0pt}%
\pgfpathmoveto{\pgfqpoint{4.120816in}{3.491212in}}%
\pgfpathcurveto{\pgfqpoint{4.126341in}{3.491212in}}{\pgfqpoint{4.131641in}{3.493407in}}{\pgfqpoint{4.135548in}{3.497314in}}%
\pgfpathcurveto{\pgfqpoint{4.139455in}{3.501221in}}{\pgfqpoint{4.141650in}{3.506520in}}{\pgfqpoint{4.141650in}{3.512045in}}%
\pgfpathcurveto{\pgfqpoint{4.141650in}{3.517571in}}{\pgfqpoint{4.139455in}{3.522870in}}{\pgfqpoint{4.135548in}{3.526777in}}%
\pgfpathcurveto{\pgfqpoint{4.131641in}{3.530684in}}{\pgfqpoint{4.126341in}{3.532879in}}{\pgfqpoint{4.120816in}{3.532879in}}%
\pgfpathcurveto{\pgfqpoint{4.115291in}{3.532879in}}{\pgfqpoint{4.109992in}{3.530684in}}{\pgfqpoint{4.106085in}{3.526777in}}%
\pgfpathcurveto{\pgfqpoint{4.102178in}{3.522870in}}{\pgfqpoint{4.099983in}{3.517571in}}{\pgfqpoint{4.099983in}{3.512045in}}%
\pgfpathcurveto{\pgfqpoint{4.099983in}{3.506520in}}{\pgfqpoint{4.102178in}{3.501221in}}{\pgfqpoint{4.106085in}{3.497314in}}%
\pgfpathcurveto{\pgfqpoint{4.109992in}{3.493407in}}{\pgfqpoint{4.115291in}{3.491212in}}{\pgfqpoint{4.120816in}{3.491212in}}%
\pgfpathclose%
\pgfusepath{fill}%
\end{pgfscope}%
\begin{pgfscope}%
\pgfpathrectangle{\pgfqpoint{3.214225in}{3.178832in}}{\pgfqpoint{1.162500in}{0.755000in}}%
\pgfusepath{clip}%
\pgfsetbuttcap%
\pgfsetroundjoin%
\definecolor{currentfill}{rgb}{0.000000,0.000000,0.000000}%
\pgfsetfillcolor{currentfill}%
\pgfsetfillopacity{0.500000}%
\pgfsetlinewidth{0.000000pt}%
\definecolor{currentstroke}{rgb}{0.000000,0.000000,0.000000}%
\pgfsetstrokecolor{currentstroke}%
\pgfsetdash{}{0pt}%
\pgfpathmoveto{\pgfqpoint{3.792736in}{3.212721in}}%
\pgfpathcurveto{\pgfqpoint{3.798261in}{3.212721in}}{\pgfqpoint{3.803560in}{3.214917in}}{\pgfqpoint{3.807467in}{3.218823in}}%
\pgfpathcurveto{\pgfqpoint{3.811374in}{3.222730in}}{\pgfqpoint{3.813569in}{3.228030in}}{\pgfqpoint{3.813569in}{3.233555in}}%
\pgfpathcurveto{\pgfqpoint{3.813569in}{3.239080in}}{\pgfqpoint{3.811374in}{3.244379in}}{\pgfqpoint{3.807467in}{3.248286in}}%
\pgfpathcurveto{\pgfqpoint{3.803560in}{3.252193in}}{\pgfqpoint{3.798261in}{3.254388in}}{\pgfqpoint{3.792736in}{3.254388in}}%
\pgfpathcurveto{\pgfqpoint{3.787211in}{3.254388in}}{\pgfqpoint{3.781911in}{3.252193in}}{\pgfqpoint{3.778004in}{3.248286in}}%
\pgfpathcurveto{\pgfqpoint{3.774097in}{3.244379in}}{\pgfqpoint{3.771902in}{3.239080in}}{\pgfqpoint{3.771902in}{3.233555in}}%
\pgfpathcurveto{\pgfqpoint{3.771902in}{3.228030in}}{\pgfqpoint{3.774097in}{3.222730in}}{\pgfqpoint{3.778004in}{3.218823in}}%
\pgfpathcurveto{\pgfqpoint{3.781911in}{3.214917in}}{\pgfqpoint{3.787211in}{3.212721in}}{\pgfqpoint{3.792736in}{3.212721in}}%
\pgfpathclose%
\pgfusepath{fill}%
\end{pgfscope}%
\begin{pgfscope}%
\pgfsetrectcap%
\pgfsetmiterjoin%
\pgfsetlinewidth{0.803000pt}%
\definecolor{currentstroke}{rgb}{0.501961,0.501961,0.501961}%
\pgfsetstrokecolor{currentstroke}%
\pgfsetdash{}{0pt}%
\pgfpathmoveto{\pgfqpoint{3.214225in}{3.178832in}}%
\pgfpathlineto{\pgfqpoint{3.214225in}{3.933832in}}%
\pgfusepath{stroke}%
\end{pgfscope}%
\begin{pgfscope}%
\pgfsetrectcap%
\pgfsetmiterjoin%
\pgfsetlinewidth{0.803000pt}%
\definecolor{currentstroke}{rgb}{0.501961,0.501961,0.501961}%
\pgfsetstrokecolor{currentstroke}%
\pgfsetdash{}{0pt}%
\pgfpathmoveto{\pgfqpoint{4.376725in}{3.178832in}}%
\pgfpathlineto{\pgfqpoint{4.376725in}{3.933832in}}%
\pgfusepath{stroke}%
\end{pgfscope}%
\begin{pgfscope}%
\pgfsetrectcap%
\pgfsetmiterjoin%
\pgfsetlinewidth{0.803000pt}%
\definecolor{currentstroke}{rgb}{0.501961,0.501961,0.501961}%
\pgfsetstrokecolor{currentstroke}%
\pgfsetdash{}{0pt}%
\pgfpathmoveto{\pgfqpoint{3.214225in}{3.178832in}}%
\pgfpathlineto{\pgfqpoint{4.376725in}{3.178832in}}%
\pgfusepath{stroke}%
\end{pgfscope}%
\begin{pgfscope}%
\pgfsetrectcap%
\pgfsetmiterjoin%
\pgfsetlinewidth{0.803000pt}%
\definecolor{currentstroke}{rgb}{0.501961,0.501961,0.501961}%
\pgfsetstrokecolor{currentstroke}%
\pgfsetdash{}{0pt}%
\pgfpathmoveto{\pgfqpoint{3.214225in}{3.933832in}}%
\pgfpathlineto{\pgfqpoint{4.376725in}{3.933832in}}%
\pgfusepath{stroke}%
\end{pgfscope}%
\begin{pgfscope}%
\pgfsetbuttcap%
\pgfsetmiterjoin%
\definecolor{currentfill}{rgb}{1.000000,1.000000,1.000000}%
\pgfsetfillcolor{currentfill}%
\pgfsetlinewidth{0.000000pt}%
\definecolor{currentstroke}{rgb}{0.000000,0.000000,0.000000}%
\pgfsetstrokecolor{currentstroke}%
\pgfsetstrokeopacity{0.000000}%
\pgfsetdash{}{0pt}%
\pgfpathmoveto{\pgfqpoint{4.376725in}{3.178832in}}%
\pgfpathlineto{\pgfqpoint{5.539225in}{3.178832in}}%
\pgfpathlineto{\pgfqpoint{5.539225in}{3.933832in}}%
\pgfpathlineto{\pgfqpoint{4.376725in}{3.933832in}}%
\pgfpathclose%
\pgfusepath{fill}%
\end{pgfscope}%
\begin{pgfscope}%
\pgfpathrectangle{\pgfqpoint{4.376725in}{3.178832in}}{\pgfqpoint{1.162500in}{0.755000in}}%
\pgfusepath{clip}%
\pgfsetbuttcap%
\pgfsetroundjoin%
\definecolor{currentfill}{rgb}{0.000000,0.000000,0.000000}%
\pgfsetfillcolor{currentfill}%
\pgfsetfillopacity{0.500000}%
\pgfsetlinewidth{0.000000pt}%
\definecolor{currentstroke}{rgb}{0.000000,0.000000,0.000000}%
\pgfsetstrokecolor{currentstroke}%
\pgfsetdash{}{0pt}%
\pgfpathmoveto{\pgfqpoint{5.214960in}{3.810398in}}%
\pgfpathcurveto{\pgfqpoint{5.220485in}{3.810398in}}{\pgfqpoint{5.225785in}{3.812593in}}{\pgfqpoint{5.229692in}{3.816499in}}%
\pgfpathcurveto{\pgfqpoint{5.233598in}{3.820406in}}{\pgfqpoint{5.235794in}{3.825706in}}{\pgfqpoint{5.235794in}{3.831231in}}%
\pgfpathcurveto{\pgfqpoint{5.235794in}{3.836756in}}{\pgfqpoint{5.233598in}{3.842055in}}{\pgfqpoint{5.229692in}{3.845962in}}%
\pgfpathcurveto{\pgfqpoint{5.225785in}{3.849869in}}{\pgfqpoint{5.220485in}{3.852064in}}{\pgfqpoint{5.214960in}{3.852064in}}%
\pgfpathcurveto{\pgfqpoint{5.209435in}{3.852064in}}{\pgfqpoint{5.204136in}{3.849869in}}{\pgfqpoint{5.200229in}{3.845962in}}%
\pgfpathcurveto{\pgfqpoint{5.196322in}{3.842055in}}{\pgfqpoint{5.194127in}{3.836756in}}{\pgfqpoint{5.194127in}{3.831231in}}%
\pgfpathcurveto{\pgfqpoint{5.194127in}{3.825706in}}{\pgfqpoint{5.196322in}{3.820406in}}{\pgfqpoint{5.200229in}{3.816499in}}%
\pgfpathcurveto{\pgfqpoint{5.204136in}{3.812593in}}{\pgfqpoint{5.209435in}{3.810398in}}{\pgfqpoint{5.214960in}{3.810398in}}%
\pgfpathclose%
\pgfusepath{fill}%
\end{pgfscope}%
\begin{pgfscope}%
\pgfpathrectangle{\pgfqpoint{4.376725in}{3.178832in}}{\pgfqpoint{1.162500in}{0.755000in}}%
\pgfusepath{clip}%
\pgfsetbuttcap%
\pgfsetroundjoin%
\definecolor{currentfill}{rgb}{0.000000,0.000000,0.000000}%
\pgfsetfillcolor{currentfill}%
\pgfsetfillopacity{0.500000}%
\pgfsetlinewidth{0.000000pt}%
\definecolor{currentstroke}{rgb}{0.000000,0.000000,0.000000}%
\pgfsetstrokecolor{currentstroke}%
\pgfsetdash{}{0pt}%
\pgfpathmoveto{\pgfqpoint{4.521739in}{3.539667in}}%
\pgfpathcurveto{\pgfqpoint{4.527264in}{3.539667in}}{\pgfqpoint{4.532563in}{3.541862in}}{\pgfqpoint{4.536470in}{3.545769in}}%
\pgfpathcurveto{\pgfqpoint{4.540377in}{3.549675in}}{\pgfqpoint{4.542572in}{3.554975in}}{\pgfqpoint{4.542572in}{3.560500in}}%
\pgfpathcurveto{\pgfqpoint{4.542572in}{3.566025in}}{\pgfqpoint{4.540377in}{3.571325in}}{\pgfqpoint{4.536470in}{3.575231in}}%
\pgfpathcurveto{\pgfqpoint{4.532563in}{3.579138in}}{\pgfqpoint{4.527264in}{3.581333in}}{\pgfqpoint{4.521739in}{3.581333in}}%
\pgfpathcurveto{\pgfqpoint{4.516214in}{3.581333in}}{\pgfqpoint{4.510914in}{3.579138in}}{\pgfqpoint{4.507007in}{3.575231in}}%
\pgfpathcurveto{\pgfqpoint{4.503101in}{3.571325in}}{\pgfqpoint{4.500905in}{3.566025in}}{\pgfqpoint{4.500905in}{3.560500in}}%
\pgfpathcurveto{\pgfqpoint{4.500905in}{3.554975in}}{\pgfqpoint{4.503101in}{3.549675in}}{\pgfqpoint{4.507007in}{3.545769in}}%
\pgfpathcurveto{\pgfqpoint{4.510914in}{3.541862in}}{\pgfqpoint{4.516214in}{3.539667in}}{\pgfqpoint{4.521739in}{3.539667in}}%
\pgfpathclose%
\pgfusepath{fill}%
\end{pgfscope}%
\begin{pgfscope}%
\pgfpathrectangle{\pgfqpoint{4.376725in}{3.178832in}}{\pgfqpoint{1.162500in}{0.755000in}}%
\pgfusepath{clip}%
\pgfsetbuttcap%
\pgfsetroundjoin%
\definecolor{currentfill}{rgb}{0.000000,0.000000,0.000000}%
\pgfsetfillcolor{currentfill}%
\pgfsetfillopacity{0.500000}%
\pgfsetlinewidth{0.000000pt}%
\definecolor{currentstroke}{rgb}{0.000000,0.000000,0.000000}%
\pgfsetstrokecolor{currentstroke}%
\pgfsetdash{}{0pt}%
\pgfpathmoveto{\pgfqpoint{4.448919in}{3.541019in}}%
\pgfpathcurveto{\pgfqpoint{4.454444in}{3.541019in}}{\pgfqpoint{4.459744in}{3.543214in}}{\pgfqpoint{4.463650in}{3.547121in}}%
\pgfpathcurveto{\pgfqpoint{4.467557in}{3.551028in}}{\pgfqpoint{4.469752in}{3.556327in}}{\pgfqpoint{4.469752in}{3.561852in}}%
\pgfpathcurveto{\pgfqpoint{4.469752in}{3.567377in}}{\pgfqpoint{4.467557in}{3.572677in}}{\pgfqpoint{4.463650in}{3.576584in}}%
\pgfpathcurveto{\pgfqpoint{4.459744in}{3.580490in}}{\pgfqpoint{4.454444in}{3.582685in}}{\pgfqpoint{4.448919in}{3.582685in}}%
\pgfpathcurveto{\pgfqpoint{4.443394in}{3.582685in}}{\pgfqpoint{4.438094in}{3.580490in}}{\pgfqpoint{4.434188in}{3.576584in}}%
\pgfpathcurveto{\pgfqpoint{4.430281in}{3.572677in}}{\pgfqpoint{4.428086in}{3.567377in}}{\pgfqpoint{4.428086in}{3.561852in}}%
\pgfpathcurveto{\pgfqpoint{4.428086in}{3.556327in}}{\pgfqpoint{4.430281in}{3.551028in}}{\pgfqpoint{4.434188in}{3.547121in}}%
\pgfpathcurveto{\pgfqpoint{4.438094in}{3.543214in}}{\pgfqpoint{4.443394in}{3.541019in}}{\pgfqpoint{4.448919in}{3.541019in}}%
\pgfpathclose%
\pgfusepath{fill}%
\end{pgfscope}%
\begin{pgfscope}%
\pgfpathrectangle{\pgfqpoint{4.376725in}{3.178832in}}{\pgfqpoint{1.162500in}{0.755000in}}%
\pgfusepath{clip}%
\pgfsetbuttcap%
\pgfsetroundjoin%
\definecolor{currentfill}{rgb}{0.000000,0.000000,0.000000}%
\pgfsetfillcolor{currentfill}%
\pgfsetfillopacity{0.500000}%
\pgfsetlinewidth{0.000000pt}%
\definecolor{currentstroke}{rgb}{0.000000,0.000000,0.000000}%
\pgfsetstrokecolor{currentstroke}%
\pgfsetdash{}{0pt}%
\pgfpathmoveto{\pgfqpoint{4.704251in}{3.308342in}}%
\pgfpathcurveto{\pgfqpoint{4.709776in}{3.308342in}}{\pgfqpoint{4.715075in}{3.310537in}}{\pgfqpoint{4.718982in}{3.314444in}}%
\pgfpathcurveto{\pgfqpoint{4.722889in}{3.318351in}}{\pgfqpoint{4.725084in}{3.323651in}}{\pgfqpoint{4.725084in}{3.329176in}}%
\pgfpathcurveto{\pgfqpoint{4.725084in}{3.334701in}}{\pgfqpoint{4.722889in}{3.340000in}}{\pgfqpoint{4.718982in}{3.343907in}}%
\pgfpathcurveto{\pgfqpoint{4.715075in}{3.347814in}}{\pgfqpoint{4.709776in}{3.350009in}}{\pgfqpoint{4.704251in}{3.350009in}}%
\pgfpathcurveto{\pgfqpoint{4.698726in}{3.350009in}}{\pgfqpoint{4.693426in}{3.347814in}}{\pgfqpoint{4.689519in}{3.343907in}}%
\pgfpathcurveto{\pgfqpoint{4.685612in}{3.340000in}}{\pgfqpoint{4.683417in}{3.334701in}}{\pgfqpoint{4.683417in}{3.329176in}}%
\pgfpathcurveto{\pgfqpoint{4.683417in}{3.323651in}}{\pgfqpoint{4.685612in}{3.318351in}}{\pgfqpoint{4.689519in}{3.314444in}}%
\pgfpathcurveto{\pgfqpoint{4.693426in}{3.310537in}}{\pgfqpoint{4.698726in}{3.308342in}}{\pgfqpoint{4.704251in}{3.308342in}}%
\pgfpathclose%
\pgfusepath{fill}%
\end{pgfscope}%
\begin{pgfscope}%
\pgfpathrectangle{\pgfqpoint{4.376725in}{3.178832in}}{\pgfqpoint{1.162500in}{0.755000in}}%
\pgfusepath{clip}%
\pgfsetbuttcap%
\pgfsetroundjoin%
\definecolor{currentfill}{rgb}{0.000000,0.000000,0.000000}%
\pgfsetfillcolor{currentfill}%
\pgfsetfillopacity{0.500000}%
\pgfsetlinewidth{0.000000pt}%
\definecolor{currentstroke}{rgb}{0.000000,0.000000,0.000000}%
\pgfsetstrokecolor{currentstroke}%
\pgfsetdash{}{0pt}%
\pgfpathmoveto{\pgfqpoint{4.404404in}{3.311937in}}%
\pgfpathcurveto{\pgfqpoint{4.409929in}{3.311937in}}{\pgfqpoint{4.415228in}{3.314132in}}{\pgfqpoint{4.419135in}{3.318039in}}%
\pgfpathcurveto{\pgfqpoint{4.423042in}{3.321945in}}{\pgfqpoint{4.425237in}{3.327245in}}{\pgfqpoint{4.425237in}{3.332770in}}%
\pgfpathcurveto{\pgfqpoint{4.425237in}{3.338295in}}{\pgfqpoint{4.423042in}{3.343595in}}{\pgfqpoint{4.419135in}{3.347501in}}%
\pgfpathcurveto{\pgfqpoint{4.415228in}{3.351408in}}{\pgfqpoint{4.409929in}{3.353603in}}{\pgfqpoint{4.404404in}{3.353603in}}%
\pgfpathcurveto{\pgfqpoint{4.398879in}{3.353603in}}{\pgfqpoint{4.393579in}{3.351408in}}{\pgfqpoint{4.389672in}{3.347501in}}%
\pgfpathcurveto{\pgfqpoint{4.385765in}{3.343595in}}{\pgfqpoint{4.383570in}{3.338295in}}{\pgfqpoint{4.383570in}{3.332770in}}%
\pgfpathcurveto{\pgfqpoint{4.383570in}{3.327245in}}{\pgfqpoint{4.385765in}{3.321945in}}{\pgfqpoint{4.389672in}{3.318039in}}%
\pgfpathcurveto{\pgfqpoint{4.393579in}{3.314132in}}{\pgfqpoint{4.398879in}{3.311937in}}{\pgfqpoint{4.404404in}{3.311937in}}%
\pgfpathclose%
\pgfusepath{fill}%
\end{pgfscope}%
\begin{pgfscope}%
\pgfpathrectangle{\pgfqpoint{4.376725in}{3.178832in}}{\pgfqpoint{1.162500in}{0.755000in}}%
\pgfusepath{clip}%
\pgfsetbuttcap%
\pgfsetroundjoin%
\definecolor{currentfill}{rgb}{0.000000,0.000000,0.000000}%
\pgfsetfillcolor{currentfill}%
\pgfsetfillopacity{0.500000}%
\pgfsetlinewidth{0.000000pt}%
\definecolor{currentstroke}{rgb}{0.000000,0.000000,0.000000}%
\pgfsetstrokecolor{currentstroke}%
\pgfsetdash{}{0pt}%
\pgfpathmoveto{\pgfqpoint{4.607580in}{3.215463in}}%
\pgfpathcurveto{\pgfqpoint{4.613105in}{3.215463in}}{\pgfqpoint{4.618405in}{3.217658in}}{\pgfqpoint{4.622312in}{3.221565in}}%
\pgfpathcurveto{\pgfqpoint{4.626218in}{3.225471in}}{\pgfqpoint{4.628414in}{3.230771in}}{\pgfqpoint{4.628414in}{3.236296in}}%
\pgfpathcurveto{\pgfqpoint{4.628414in}{3.241821in}}{\pgfqpoint{4.626218in}{3.247121in}}{\pgfqpoint{4.622312in}{3.251027in}}%
\pgfpathcurveto{\pgfqpoint{4.618405in}{3.254934in}}{\pgfqpoint{4.613105in}{3.257129in}}{\pgfqpoint{4.607580in}{3.257129in}}%
\pgfpathcurveto{\pgfqpoint{4.602055in}{3.257129in}}{\pgfqpoint{4.596756in}{3.254934in}}{\pgfqpoint{4.592849in}{3.251027in}}%
\pgfpathcurveto{\pgfqpoint{4.588942in}{3.247121in}}{\pgfqpoint{4.586747in}{3.241821in}}{\pgfqpoint{4.586747in}{3.236296in}}%
\pgfpathcurveto{\pgfqpoint{4.586747in}{3.230771in}}{\pgfqpoint{4.588942in}{3.225471in}}{\pgfqpoint{4.592849in}{3.221565in}}%
\pgfpathcurveto{\pgfqpoint{4.596756in}{3.217658in}}{\pgfqpoint{4.602055in}{3.215463in}}{\pgfqpoint{4.607580in}{3.215463in}}%
\pgfpathclose%
\pgfusepath{fill}%
\end{pgfscope}%
\begin{pgfscope}%
\pgfpathrectangle{\pgfqpoint{4.376725in}{3.178832in}}{\pgfqpoint{1.162500in}{0.755000in}}%
\pgfusepath{clip}%
\pgfsetbuttcap%
\pgfsetroundjoin%
\definecolor{currentfill}{rgb}{0.000000,0.000000,0.000000}%
\pgfsetfillcolor{currentfill}%
\pgfsetfillopacity{0.500000}%
\pgfsetlinewidth{0.000000pt}%
\definecolor{currentstroke}{rgb}{0.000000,0.000000,0.000000}%
\pgfsetstrokecolor{currentstroke}%
\pgfsetdash{}{0pt}%
\pgfpathmoveto{\pgfqpoint{4.596304in}{3.175975in}}%
\pgfpathcurveto{\pgfqpoint{4.601829in}{3.175975in}}{\pgfqpoint{4.607129in}{3.178170in}}{\pgfqpoint{4.611036in}{3.182077in}}%
\pgfpathcurveto{\pgfqpoint{4.614942in}{3.185984in}}{\pgfqpoint{4.617137in}{3.191283in}}{\pgfqpoint{4.617137in}{3.196809in}}%
\pgfpathcurveto{\pgfqpoint{4.617137in}{3.202334in}}{\pgfqpoint{4.614942in}{3.207633in}}{\pgfqpoint{4.611036in}{3.211540in}}%
\pgfpathcurveto{\pgfqpoint{4.607129in}{3.215447in}}{\pgfqpoint{4.601829in}{3.217642in}}{\pgfqpoint{4.596304in}{3.217642in}}%
\pgfpathcurveto{\pgfqpoint{4.590779in}{3.217642in}}{\pgfqpoint{4.585480in}{3.215447in}}{\pgfqpoint{4.581573in}{3.211540in}}%
\pgfpathcurveto{\pgfqpoint{4.577666in}{3.207633in}}{\pgfqpoint{4.575471in}{3.202334in}}{\pgfqpoint{4.575471in}{3.196809in}}%
\pgfpathcurveto{\pgfqpoint{4.575471in}{3.191283in}}{\pgfqpoint{4.577666in}{3.185984in}}{\pgfqpoint{4.581573in}{3.182077in}}%
\pgfpathcurveto{\pgfqpoint{4.585480in}{3.178170in}}{\pgfqpoint{4.590779in}{3.175975in}}{\pgfqpoint{4.596304in}{3.175975in}}%
\pgfpathclose%
\pgfusepath{fill}%
\end{pgfscope}%
\begin{pgfscope}%
\pgfpathrectangle{\pgfqpoint{4.376725in}{3.178832in}}{\pgfqpoint{1.162500in}{0.755000in}}%
\pgfusepath{clip}%
\pgfsetbuttcap%
\pgfsetroundjoin%
\definecolor{currentfill}{rgb}{0.000000,0.000000,0.000000}%
\pgfsetfillcolor{currentfill}%
\pgfsetfillopacity{0.500000}%
\pgfsetlinewidth{0.000000pt}%
\definecolor{currentstroke}{rgb}{0.000000,0.000000,0.000000}%
\pgfsetstrokecolor{currentstroke}%
\pgfsetdash{}{0pt}%
\pgfpathmoveto{\pgfqpoint{5.423123in}{3.678982in}}%
\pgfpathcurveto{\pgfqpoint{5.428649in}{3.678982in}}{\pgfqpoint{5.433948in}{3.681177in}}{\pgfqpoint{5.437855in}{3.685084in}}%
\pgfpathcurveto{\pgfqpoint{5.441762in}{3.688990in}}{\pgfqpoint{5.443957in}{3.694290in}}{\pgfqpoint{5.443957in}{3.699815in}}%
\pgfpathcurveto{\pgfqpoint{5.443957in}{3.705340in}}{\pgfqpoint{5.441762in}{3.710640in}}{\pgfqpoint{5.437855in}{3.714546in}}%
\pgfpathcurveto{\pgfqpoint{5.433948in}{3.718453in}}{\pgfqpoint{5.428649in}{3.720648in}}{\pgfqpoint{5.423123in}{3.720648in}}%
\pgfpathcurveto{\pgfqpoint{5.417598in}{3.720648in}}{\pgfqpoint{5.412299in}{3.718453in}}{\pgfqpoint{5.408392in}{3.714546in}}%
\pgfpathcurveto{\pgfqpoint{5.404485in}{3.710640in}}{\pgfqpoint{5.402290in}{3.705340in}}{\pgfqpoint{5.402290in}{3.699815in}}%
\pgfpathcurveto{\pgfqpoint{5.402290in}{3.694290in}}{\pgfqpoint{5.404485in}{3.688990in}}{\pgfqpoint{5.408392in}{3.685084in}}%
\pgfpathcurveto{\pgfqpoint{5.412299in}{3.681177in}}{\pgfqpoint{5.417598in}{3.678982in}}{\pgfqpoint{5.423123in}{3.678982in}}%
\pgfpathclose%
\pgfusepath{fill}%
\end{pgfscope}%
\begin{pgfscope}%
\pgfpathrectangle{\pgfqpoint{4.376725in}{3.178832in}}{\pgfqpoint{1.162500in}{0.755000in}}%
\pgfusepath{clip}%
\pgfsetbuttcap%
\pgfsetroundjoin%
\definecolor{currentfill}{rgb}{0.000000,0.000000,0.000000}%
\pgfsetfillcolor{currentfill}%
\pgfsetfillopacity{0.500000}%
\pgfsetlinewidth{0.000000pt}%
\definecolor{currentstroke}{rgb}{0.000000,0.000000,0.000000}%
\pgfsetstrokecolor{currentstroke}%
\pgfsetdash{}{0pt}%
\pgfpathmoveto{\pgfqpoint{5.511546in}{3.528997in}}%
\pgfpathcurveto{\pgfqpoint{5.517071in}{3.528997in}}{\pgfqpoint{5.522371in}{3.531192in}}{\pgfqpoint{5.526278in}{3.535099in}}%
\pgfpathcurveto{\pgfqpoint{5.530185in}{3.539005in}}{\pgfqpoint{5.532380in}{3.544305in}}{\pgfqpoint{5.532380in}{3.549830in}}%
\pgfpathcurveto{\pgfqpoint{5.532380in}{3.555355in}}{\pgfqpoint{5.530185in}{3.560655in}}{\pgfqpoint{5.526278in}{3.564561in}}%
\pgfpathcurveto{\pgfqpoint{5.522371in}{3.568468in}}{\pgfqpoint{5.517071in}{3.570663in}}{\pgfqpoint{5.511546in}{3.570663in}}%
\pgfpathcurveto{\pgfqpoint{5.506021in}{3.570663in}}{\pgfqpoint{5.500722in}{3.568468in}}{\pgfqpoint{5.496815in}{3.564561in}}%
\pgfpathcurveto{\pgfqpoint{5.492908in}{3.560655in}}{\pgfqpoint{5.490713in}{3.555355in}}{\pgfqpoint{5.490713in}{3.549830in}}%
\pgfpathcurveto{\pgfqpoint{5.490713in}{3.544305in}}{\pgfqpoint{5.492908in}{3.539005in}}{\pgfqpoint{5.496815in}{3.535099in}}%
\pgfpathcurveto{\pgfqpoint{5.500722in}{3.531192in}}{\pgfqpoint{5.506021in}{3.528997in}}{\pgfqpoint{5.511546in}{3.528997in}}%
\pgfpathclose%
\pgfusepath{fill}%
\end{pgfscope}%
\begin{pgfscope}%
\pgfpathrectangle{\pgfqpoint{4.376725in}{3.178832in}}{\pgfqpoint{1.162500in}{0.755000in}}%
\pgfusepath{clip}%
\pgfsetbuttcap%
\pgfsetroundjoin%
\definecolor{currentfill}{rgb}{0.000000,0.000000,0.000000}%
\pgfsetfillcolor{currentfill}%
\pgfsetfillopacity{0.500000}%
\pgfsetlinewidth{0.000000pt}%
\definecolor{currentstroke}{rgb}{0.000000,0.000000,0.000000}%
\pgfsetstrokecolor{currentstroke}%
\pgfsetdash{}{0pt}%
\pgfpathmoveto{\pgfqpoint{5.286995in}{3.426855in}}%
\pgfpathcurveto{\pgfqpoint{5.292520in}{3.426855in}}{\pgfqpoint{5.297820in}{3.429051in}}{\pgfqpoint{5.301726in}{3.432957in}}%
\pgfpathcurveto{\pgfqpoint{5.305633in}{3.436864in}}{\pgfqpoint{5.307828in}{3.442164in}}{\pgfqpoint{5.307828in}{3.447689in}}%
\pgfpathcurveto{\pgfqpoint{5.307828in}{3.453214in}}{\pgfqpoint{5.305633in}{3.458513in}}{\pgfqpoint{5.301726in}{3.462420in}}%
\pgfpathcurveto{\pgfqpoint{5.297820in}{3.466327in}}{\pgfqpoint{5.292520in}{3.468522in}}{\pgfqpoint{5.286995in}{3.468522in}}%
\pgfpathcurveto{\pgfqpoint{5.281470in}{3.468522in}}{\pgfqpoint{5.276170in}{3.466327in}}{\pgfqpoint{5.272264in}{3.462420in}}%
\pgfpathcurveto{\pgfqpoint{5.268357in}{3.458513in}}{\pgfqpoint{5.266162in}{3.453214in}}{\pgfqpoint{5.266162in}{3.447689in}}%
\pgfpathcurveto{\pgfqpoint{5.266162in}{3.442164in}}{\pgfqpoint{5.268357in}{3.436864in}}{\pgfqpoint{5.272264in}{3.432957in}}%
\pgfpathcurveto{\pgfqpoint{5.276170in}{3.429051in}}{\pgfqpoint{5.281470in}{3.426855in}}{\pgfqpoint{5.286995in}{3.426855in}}%
\pgfpathclose%
\pgfusepath{fill}%
\end{pgfscope}%
\begin{pgfscope}%
\pgfpathrectangle{\pgfqpoint{4.376725in}{3.178832in}}{\pgfqpoint{1.162500in}{0.755000in}}%
\pgfusepath{clip}%
\pgfsetbuttcap%
\pgfsetroundjoin%
\definecolor{currentfill}{rgb}{0.000000,0.000000,0.000000}%
\pgfsetfillcolor{currentfill}%
\pgfsetfillopacity{0.500000}%
\pgfsetlinewidth{0.000000pt}%
\definecolor{currentstroke}{rgb}{0.000000,0.000000,0.000000}%
\pgfsetstrokecolor{currentstroke}%
\pgfsetdash{}{0pt}%
\pgfpathmoveto{\pgfqpoint{4.645822in}{3.677650in}}%
\pgfpathcurveto{\pgfqpoint{4.651348in}{3.677650in}}{\pgfqpoint{4.656647in}{3.679845in}}{\pgfqpoint{4.660554in}{3.683752in}}%
\pgfpathcurveto{\pgfqpoint{4.664461in}{3.687659in}}{\pgfqpoint{4.666656in}{3.692959in}}{\pgfqpoint{4.666656in}{3.698484in}}%
\pgfpathcurveto{\pgfqpoint{4.666656in}{3.704009in}}{\pgfqpoint{4.664461in}{3.709308in}}{\pgfqpoint{4.660554in}{3.713215in}}%
\pgfpathcurveto{\pgfqpoint{4.656647in}{3.717122in}}{\pgfqpoint{4.651348in}{3.719317in}}{\pgfqpoint{4.645822in}{3.719317in}}%
\pgfpathcurveto{\pgfqpoint{4.640297in}{3.719317in}}{\pgfqpoint{4.634998in}{3.717122in}}{\pgfqpoint{4.631091in}{3.713215in}}%
\pgfpathcurveto{\pgfqpoint{4.627184in}{3.709308in}}{\pgfqpoint{4.624989in}{3.704009in}}{\pgfqpoint{4.624989in}{3.698484in}}%
\pgfpathcurveto{\pgfqpoint{4.624989in}{3.692959in}}{\pgfqpoint{4.627184in}{3.687659in}}{\pgfqpoint{4.631091in}{3.683752in}}%
\pgfpathcurveto{\pgfqpoint{4.634998in}{3.679845in}}{\pgfqpoint{4.640297in}{3.677650in}}{\pgfqpoint{4.645822in}{3.677650in}}%
\pgfpathclose%
\pgfusepath{fill}%
\end{pgfscope}%
\begin{pgfscope}%
\pgfpathrectangle{\pgfqpoint{4.376725in}{3.178832in}}{\pgfqpoint{1.162500in}{0.755000in}}%
\pgfusepath{clip}%
\pgfsetbuttcap%
\pgfsetroundjoin%
\definecolor{currentfill}{rgb}{0.000000,0.000000,0.000000}%
\pgfsetfillcolor{currentfill}%
\pgfsetfillopacity{0.500000}%
\pgfsetlinewidth{0.000000pt}%
\definecolor{currentstroke}{rgb}{0.000000,0.000000,0.000000}%
\pgfsetstrokecolor{currentstroke}%
\pgfsetdash{}{0pt}%
\pgfpathmoveto{\pgfqpoint{5.069839in}{3.296446in}}%
\pgfpathcurveto{\pgfqpoint{5.075364in}{3.296446in}}{\pgfqpoint{5.080663in}{3.298641in}}{\pgfqpoint{5.084570in}{3.302548in}}%
\pgfpathcurveto{\pgfqpoint{5.088477in}{3.306455in}}{\pgfqpoint{5.090672in}{3.311754in}}{\pgfqpoint{5.090672in}{3.317279in}}%
\pgfpathcurveto{\pgfqpoint{5.090672in}{3.322804in}}{\pgfqpoint{5.088477in}{3.328104in}}{\pgfqpoint{5.084570in}{3.332011in}}%
\pgfpathcurveto{\pgfqpoint{5.080663in}{3.335917in}}{\pgfqpoint{5.075364in}{3.338112in}}{\pgfqpoint{5.069839in}{3.338112in}}%
\pgfpathcurveto{\pgfqpoint{5.064313in}{3.338112in}}{\pgfqpoint{5.059014in}{3.335917in}}{\pgfqpoint{5.055107in}{3.332011in}}%
\pgfpathcurveto{\pgfqpoint{5.051200in}{3.328104in}}{\pgfqpoint{5.049005in}{3.322804in}}{\pgfqpoint{5.049005in}{3.317279in}}%
\pgfpathcurveto{\pgfqpoint{5.049005in}{3.311754in}}{\pgfqpoint{5.051200in}{3.306455in}}{\pgfqpoint{5.055107in}{3.302548in}}%
\pgfpathcurveto{\pgfqpoint{5.059014in}{3.298641in}}{\pgfqpoint{5.064313in}{3.296446in}}{\pgfqpoint{5.069839in}{3.296446in}}%
\pgfpathclose%
\pgfusepath{fill}%
\end{pgfscope}%
\begin{pgfscope}%
\pgfpathrectangle{\pgfqpoint{4.376725in}{3.178832in}}{\pgfqpoint{1.162500in}{0.755000in}}%
\pgfusepath{clip}%
\pgfsetbuttcap%
\pgfsetroundjoin%
\definecolor{currentfill}{rgb}{0.000000,0.000000,0.000000}%
\pgfsetfillcolor{currentfill}%
\pgfsetfillopacity{0.500000}%
\pgfsetlinewidth{0.000000pt}%
\definecolor{currentstroke}{rgb}{0.000000,0.000000,0.000000}%
\pgfsetstrokecolor{currentstroke}%
\pgfsetdash{}{0pt}%
\pgfpathmoveto{\pgfqpoint{4.941375in}{3.322937in}}%
\pgfpathcurveto{\pgfqpoint{4.946900in}{3.322937in}}{\pgfqpoint{4.952199in}{3.325133in}}{\pgfqpoint{4.956106in}{3.329039in}}%
\pgfpathcurveto{\pgfqpoint{4.960013in}{3.332946in}}{\pgfqpoint{4.962208in}{3.338246in}}{\pgfqpoint{4.962208in}{3.343771in}}%
\pgfpathcurveto{\pgfqpoint{4.962208in}{3.349296in}}{\pgfqpoint{4.960013in}{3.354595in}}{\pgfqpoint{4.956106in}{3.358502in}}%
\pgfpathcurveto{\pgfqpoint{4.952199in}{3.362409in}}{\pgfqpoint{4.946900in}{3.364604in}}{\pgfqpoint{4.941375in}{3.364604in}}%
\pgfpathcurveto{\pgfqpoint{4.935850in}{3.364604in}}{\pgfqpoint{4.930550in}{3.362409in}}{\pgfqpoint{4.926643in}{3.358502in}}%
\pgfpathcurveto{\pgfqpoint{4.922736in}{3.354595in}}{\pgfqpoint{4.920541in}{3.349296in}}{\pgfqpoint{4.920541in}{3.343771in}}%
\pgfpathcurveto{\pgfqpoint{4.920541in}{3.338246in}}{\pgfqpoint{4.922736in}{3.332946in}}{\pgfqpoint{4.926643in}{3.329039in}}%
\pgfpathcurveto{\pgfqpoint{4.930550in}{3.325133in}}{\pgfqpoint{4.935850in}{3.322937in}}{\pgfqpoint{4.941375in}{3.322937in}}%
\pgfpathclose%
\pgfusepath{fill}%
\end{pgfscope}%
\begin{pgfscope}%
\pgfpathrectangle{\pgfqpoint{4.376725in}{3.178832in}}{\pgfqpoint{1.162500in}{0.755000in}}%
\pgfusepath{clip}%
\pgfsetbuttcap%
\pgfsetroundjoin%
\definecolor{currentfill}{rgb}{0.000000,0.000000,0.000000}%
\pgfsetfillcolor{currentfill}%
\pgfsetfillopacity{0.500000}%
\pgfsetlinewidth{0.000000pt}%
\definecolor{currentstroke}{rgb}{0.000000,0.000000,0.000000}%
\pgfsetstrokecolor{currentstroke}%
\pgfsetdash{}{0pt}%
\pgfpathmoveto{\pgfqpoint{4.799153in}{3.895023in}}%
\pgfpathcurveto{\pgfqpoint{4.804678in}{3.895023in}}{\pgfqpoint{4.809977in}{3.897218in}}{\pgfqpoint{4.813884in}{3.901125in}}%
\pgfpathcurveto{\pgfqpoint{4.817791in}{3.905032in}}{\pgfqpoint{4.819986in}{3.910331in}}{\pgfqpoint{4.819986in}{3.915856in}}%
\pgfpathcurveto{\pgfqpoint{4.819986in}{3.921381in}}{\pgfqpoint{4.817791in}{3.926681in}}{\pgfqpoint{4.813884in}{3.930588in}}%
\pgfpathcurveto{\pgfqpoint{4.809977in}{3.934494in}}{\pgfqpoint{4.804678in}{3.936690in}}{\pgfqpoint{4.799153in}{3.936690in}}%
\pgfpathcurveto{\pgfqpoint{4.793628in}{3.936690in}}{\pgfqpoint{4.788328in}{3.934494in}}{\pgfqpoint{4.784421in}{3.930588in}}%
\pgfpathcurveto{\pgfqpoint{4.780515in}{3.926681in}}{\pgfqpoint{4.778320in}{3.921381in}}{\pgfqpoint{4.778320in}{3.915856in}}%
\pgfpathcurveto{\pgfqpoint{4.778320in}{3.910331in}}{\pgfqpoint{4.780515in}{3.905032in}}{\pgfqpoint{4.784421in}{3.901125in}}%
\pgfpathcurveto{\pgfqpoint{4.788328in}{3.897218in}}{\pgfqpoint{4.793628in}{3.895023in}}{\pgfqpoint{4.799153in}{3.895023in}}%
\pgfpathclose%
\pgfusepath{fill}%
\end{pgfscope}%
\begin{pgfscope}%
\pgfpathrectangle{\pgfqpoint{4.376725in}{3.178832in}}{\pgfqpoint{1.162500in}{0.755000in}}%
\pgfusepath{clip}%
\pgfsetbuttcap%
\pgfsetroundjoin%
\definecolor{currentfill}{rgb}{0.000000,0.000000,0.000000}%
\pgfsetfillcolor{currentfill}%
\pgfsetfillopacity{0.500000}%
\pgfsetlinewidth{0.000000pt}%
\definecolor{currentstroke}{rgb}{0.000000,0.000000,0.000000}%
\pgfsetstrokecolor{currentstroke}%
\pgfsetdash{}{0pt}%
\pgfpathmoveto{\pgfqpoint{4.595218in}{3.491212in}}%
\pgfpathcurveto{\pgfqpoint{4.600743in}{3.491212in}}{\pgfqpoint{4.606043in}{3.493407in}}{\pgfqpoint{4.609950in}{3.497314in}}%
\pgfpathcurveto{\pgfqpoint{4.613857in}{3.501221in}}{\pgfqpoint{4.616052in}{3.506520in}}{\pgfqpoint{4.616052in}{3.512045in}}%
\pgfpathcurveto{\pgfqpoint{4.616052in}{3.517571in}}{\pgfqpoint{4.613857in}{3.522870in}}{\pgfqpoint{4.609950in}{3.526777in}}%
\pgfpathcurveto{\pgfqpoint{4.606043in}{3.530684in}}{\pgfqpoint{4.600743in}{3.532879in}}{\pgfqpoint{4.595218in}{3.532879in}}%
\pgfpathcurveto{\pgfqpoint{4.589693in}{3.532879in}}{\pgfqpoint{4.584394in}{3.530684in}}{\pgfqpoint{4.580487in}{3.526777in}}%
\pgfpathcurveto{\pgfqpoint{4.576580in}{3.522870in}}{\pgfqpoint{4.574385in}{3.517571in}}{\pgfqpoint{4.574385in}{3.512045in}}%
\pgfpathcurveto{\pgfqpoint{4.574385in}{3.506520in}}{\pgfqpoint{4.576580in}{3.501221in}}{\pgfqpoint{4.580487in}{3.497314in}}%
\pgfpathcurveto{\pgfqpoint{4.584394in}{3.493407in}}{\pgfqpoint{4.589693in}{3.491212in}}{\pgfqpoint{4.595218in}{3.491212in}}%
\pgfpathclose%
\pgfusepath{fill}%
\end{pgfscope}%
\begin{pgfscope}%
\pgfpathrectangle{\pgfqpoint{4.376725in}{3.178832in}}{\pgfqpoint{1.162500in}{0.755000in}}%
\pgfusepath{clip}%
\pgfsetbuttcap%
\pgfsetroundjoin%
\definecolor{currentfill}{rgb}{0.000000,0.000000,0.000000}%
\pgfsetfillcolor{currentfill}%
\pgfsetfillopacity{0.500000}%
\pgfsetlinewidth{0.000000pt}%
\definecolor{currentstroke}{rgb}{0.000000,0.000000,0.000000}%
\pgfsetstrokecolor{currentstroke}%
\pgfsetdash{}{0pt}%
\pgfpathmoveto{\pgfqpoint{4.834270in}{3.212721in}}%
\pgfpathcurveto{\pgfqpoint{4.839795in}{3.212721in}}{\pgfqpoint{4.845094in}{3.214917in}}{\pgfqpoint{4.849001in}{3.218823in}}%
\pgfpathcurveto{\pgfqpoint{4.852908in}{3.222730in}}{\pgfqpoint{4.855103in}{3.228030in}}{\pgfqpoint{4.855103in}{3.233555in}}%
\pgfpathcurveto{\pgfqpoint{4.855103in}{3.239080in}}{\pgfqpoint{4.852908in}{3.244379in}}{\pgfqpoint{4.849001in}{3.248286in}}%
\pgfpathcurveto{\pgfqpoint{4.845094in}{3.252193in}}{\pgfqpoint{4.839795in}{3.254388in}}{\pgfqpoint{4.834270in}{3.254388in}}%
\pgfpathcurveto{\pgfqpoint{4.828745in}{3.254388in}}{\pgfqpoint{4.823445in}{3.252193in}}{\pgfqpoint{4.819538in}{3.248286in}}%
\pgfpathcurveto{\pgfqpoint{4.815632in}{3.244379in}}{\pgfqpoint{4.813437in}{3.239080in}}{\pgfqpoint{4.813437in}{3.233555in}}%
\pgfpathcurveto{\pgfqpoint{4.813437in}{3.228030in}}{\pgfqpoint{4.815632in}{3.222730in}}{\pgfqpoint{4.819538in}{3.218823in}}%
\pgfpathcurveto{\pgfqpoint{4.823445in}{3.214917in}}{\pgfqpoint{4.828745in}{3.212721in}}{\pgfqpoint{4.834270in}{3.212721in}}%
\pgfpathclose%
\pgfusepath{fill}%
\end{pgfscope}%
\begin{pgfscope}%
\pgfsetrectcap%
\pgfsetmiterjoin%
\pgfsetlinewidth{0.803000pt}%
\definecolor{currentstroke}{rgb}{0.501961,0.501961,0.501961}%
\pgfsetstrokecolor{currentstroke}%
\pgfsetdash{}{0pt}%
\pgfpathmoveto{\pgfqpoint{4.376725in}{3.178832in}}%
\pgfpathlineto{\pgfqpoint{4.376725in}{3.933832in}}%
\pgfusepath{stroke}%
\end{pgfscope}%
\begin{pgfscope}%
\pgfsetrectcap%
\pgfsetmiterjoin%
\pgfsetlinewidth{0.803000pt}%
\definecolor{currentstroke}{rgb}{0.501961,0.501961,0.501961}%
\pgfsetstrokecolor{currentstroke}%
\pgfsetdash{}{0pt}%
\pgfpathmoveto{\pgfqpoint{5.539225in}{3.178832in}}%
\pgfpathlineto{\pgfqpoint{5.539225in}{3.933832in}}%
\pgfusepath{stroke}%
\end{pgfscope}%
\begin{pgfscope}%
\pgfsetrectcap%
\pgfsetmiterjoin%
\pgfsetlinewidth{0.803000pt}%
\definecolor{currentstroke}{rgb}{0.501961,0.501961,0.501961}%
\pgfsetstrokecolor{currentstroke}%
\pgfsetdash{}{0pt}%
\pgfpathmoveto{\pgfqpoint{4.376725in}{3.178832in}}%
\pgfpathlineto{\pgfqpoint{5.539225in}{3.178832in}}%
\pgfusepath{stroke}%
\end{pgfscope}%
\begin{pgfscope}%
\pgfsetrectcap%
\pgfsetmiterjoin%
\pgfsetlinewidth{0.803000pt}%
\definecolor{currentstroke}{rgb}{0.501961,0.501961,0.501961}%
\pgfsetstrokecolor{currentstroke}%
\pgfsetdash{}{0pt}%
\pgfpathmoveto{\pgfqpoint{4.376725in}{3.933832in}}%
\pgfpathlineto{\pgfqpoint{5.539225in}{3.933832in}}%
\pgfusepath{stroke}%
\end{pgfscope}%
\begin{pgfscope}%
\pgfsetbuttcap%
\pgfsetmiterjoin%
\definecolor{currentfill}{rgb}{1.000000,1.000000,1.000000}%
\pgfsetfillcolor{currentfill}%
\pgfsetlinewidth{0.000000pt}%
\definecolor{currentstroke}{rgb}{0.000000,0.000000,0.000000}%
\pgfsetstrokecolor{currentstroke}%
\pgfsetstrokeopacity{0.000000}%
\pgfsetdash{}{0pt}%
\pgfpathmoveto{\pgfqpoint{0.889225in}{2.423832in}}%
\pgfpathlineto{\pgfqpoint{2.051725in}{2.423832in}}%
\pgfpathlineto{\pgfqpoint{2.051725in}{3.178832in}}%
\pgfpathlineto{\pgfqpoint{0.889225in}{3.178832in}}%
\pgfpathclose%
\pgfusepath{fill}%
\end{pgfscope}%
\begin{pgfscope}%
\pgfpathrectangle{\pgfqpoint{0.889225in}{2.423832in}}{\pgfqpoint{1.162500in}{0.755000in}}%
\pgfusepath{clip}%
\pgfsetbuttcap%
\pgfsetroundjoin%
\definecolor{currentfill}{rgb}{0.000000,0.000000,0.000000}%
\pgfsetfillcolor{currentfill}%
\pgfsetfillopacity{0.500000}%
\pgfsetlinewidth{0.000000pt}%
\definecolor{currentstroke}{rgb}{0.000000,0.000000,0.000000}%
\pgfsetstrokecolor{currentstroke}%
\pgfsetdash{}{0pt}%
\pgfpathmoveto{\pgfqpoint{1.893746in}{3.088730in}}%
\pgfpathcurveto{\pgfqpoint{1.899271in}{3.088730in}}{\pgfqpoint{1.904570in}{3.090925in}}{\pgfqpoint{1.908477in}{3.094832in}}%
\pgfpathcurveto{\pgfqpoint{1.912384in}{3.098739in}}{\pgfqpoint{1.914579in}{3.104038in}}{\pgfqpoint{1.914579in}{3.109563in}}%
\pgfpathcurveto{\pgfqpoint{1.914579in}{3.115089in}}{\pgfqpoint{1.912384in}{3.120388in}}{\pgfqpoint{1.908477in}{3.124295in}}%
\pgfpathcurveto{\pgfqpoint{1.904570in}{3.128202in}}{\pgfqpoint{1.899271in}{3.130397in}}{\pgfqpoint{1.893746in}{3.130397in}}%
\pgfpathcurveto{\pgfqpoint{1.888221in}{3.130397in}}{\pgfqpoint{1.882921in}{3.128202in}}{\pgfqpoint{1.879015in}{3.124295in}}%
\pgfpathcurveto{\pgfqpoint{1.875108in}{3.120388in}}{\pgfqpoint{1.872913in}{3.115089in}}{\pgfqpoint{1.872913in}{3.109563in}}%
\pgfpathcurveto{\pgfqpoint{1.872913in}{3.104038in}}{\pgfqpoint{1.875108in}{3.098739in}}{\pgfqpoint{1.879015in}{3.094832in}}%
\pgfpathcurveto{\pgfqpoint{1.882921in}{3.090925in}}{\pgfqpoint{1.888221in}{3.088730in}}{\pgfqpoint{1.893746in}{3.088730in}}%
\pgfpathclose%
\pgfusepath{fill}%
\end{pgfscope}%
\begin{pgfscope}%
\pgfpathrectangle{\pgfqpoint{0.889225in}{2.423832in}}{\pgfqpoint{1.162500in}{0.755000in}}%
\pgfusepath{clip}%
\pgfsetbuttcap%
\pgfsetroundjoin%
\definecolor{currentfill}{rgb}{0.000000,0.000000,0.000000}%
\pgfsetfillcolor{currentfill}%
\pgfsetfillopacity{0.500000}%
\pgfsetlinewidth{0.000000pt}%
\definecolor{currentstroke}{rgb}{0.000000,0.000000,0.000000}%
\pgfsetstrokecolor{currentstroke}%
\pgfsetdash{}{0pt}%
\pgfpathmoveto{\pgfqpoint{1.476892in}{2.544873in}}%
\pgfpathcurveto{\pgfqpoint{1.482417in}{2.544873in}}{\pgfqpoint{1.487717in}{2.547068in}}{\pgfqpoint{1.491623in}{2.550975in}}%
\pgfpathcurveto{\pgfqpoint{1.495530in}{2.554881in}}{\pgfqpoint{1.497725in}{2.560181in}}{\pgfqpoint{1.497725in}{2.565706in}}%
\pgfpathcurveto{\pgfqpoint{1.497725in}{2.571231in}}{\pgfqpoint{1.495530in}{2.576531in}}{\pgfqpoint{1.491623in}{2.580437in}}%
\pgfpathcurveto{\pgfqpoint{1.487717in}{2.584344in}}{\pgfqpoint{1.482417in}{2.586539in}}{\pgfqpoint{1.476892in}{2.586539in}}%
\pgfpathcurveto{\pgfqpoint{1.471367in}{2.586539in}}{\pgfqpoint{1.466067in}{2.584344in}}{\pgfqpoint{1.462161in}{2.580437in}}%
\pgfpathcurveto{\pgfqpoint{1.458254in}{2.576531in}}{\pgfqpoint{1.456059in}{2.571231in}}{\pgfqpoint{1.456059in}{2.565706in}}%
\pgfpathcurveto{\pgfqpoint{1.456059in}{2.560181in}}{\pgfqpoint{1.458254in}{2.554881in}}{\pgfqpoint{1.462161in}{2.550975in}}%
\pgfpathcurveto{\pgfqpoint{1.466067in}{2.547068in}}{\pgfqpoint{1.471367in}{2.544873in}}{\pgfqpoint{1.476892in}{2.544873in}}%
\pgfpathclose%
\pgfusepath{fill}%
\end{pgfscope}%
\begin{pgfscope}%
\pgfpathrectangle{\pgfqpoint{0.889225in}{2.423832in}}{\pgfqpoint{1.162500in}{0.755000in}}%
\pgfusepath{clip}%
\pgfsetbuttcap%
\pgfsetroundjoin%
\definecolor{currentfill}{rgb}{0.000000,0.000000,0.000000}%
\pgfsetfillcolor{currentfill}%
\pgfsetfillopacity{0.500000}%
\pgfsetlinewidth{0.000000pt}%
\definecolor{currentstroke}{rgb}{0.000000,0.000000,0.000000}%
\pgfsetstrokecolor{currentstroke}%
\pgfsetdash{}{0pt}%
\pgfpathmoveto{\pgfqpoint{1.478974in}{2.541427in}}%
\pgfpathcurveto{\pgfqpoint{1.484499in}{2.541427in}}{\pgfqpoint{1.489799in}{2.543622in}}{\pgfqpoint{1.493705in}{2.547529in}}%
\pgfpathcurveto{\pgfqpoint{1.497612in}{2.551436in}}{\pgfqpoint{1.499807in}{2.556735in}}{\pgfqpoint{1.499807in}{2.562260in}}%
\pgfpathcurveto{\pgfqpoint{1.499807in}{2.567785in}}{\pgfqpoint{1.497612in}{2.573085in}}{\pgfqpoint{1.493705in}{2.576992in}}%
\pgfpathcurveto{\pgfqpoint{1.489799in}{2.580899in}}{\pgfqpoint{1.484499in}{2.583094in}}{\pgfqpoint{1.478974in}{2.583094in}}%
\pgfpathcurveto{\pgfqpoint{1.473449in}{2.583094in}}{\pgfqpoint{1.468149in}{2.580899in}}{\pgfqpoint{1.464243in}{2.576992in}}%
\pgfpathcurveto{\pgfqpoint{1.460336in}{2.573085in}}{\pgfqpoint{1.458141in}{2.567785in}}{\pgfqpoint{1.458141in}{2.562260in}}%
\pgfpathcurveto{\pgfqpoint{1.458141in}{2.556735in}}{\pgfqpoint{1.460336in}{2.551436in}}{\pgfqpoint{1.464243in}{2.547529in}}%
\pgfpathcurveto{\pgfqpoint{1.468149in}{2.543622in}}{\pgfqpoint{1.473449in}{2.541427in}}{\pgfqpoint{1.478974in}{2.541427in}}%
\pgfpathclose%
\pgfusepath{fill}%
\end{pgfscope}%
\begin{pgfscope}%
\pgfpathrectangle{\pgfqpoint{0.889225in}{2.423832in}}{\pgfqpoint{1.162500in}{0.755000in}}%
\pgfusepath{clip}%
\pgfsetbuttcap%
\pgfsetroundjoin%
\definecolor{currentfill}{rgb}{0.000000,0.000000,0.000000}%
\pgfsetfillcolor{currentfill}%
\pgfsetfillopacity{0.500000}%
\pgfsetlinewidth{0.000000pt}%
\definecolor{currentstroke}{rgb}{0.000000,0.000000,0.000000}%
\pgfsetstrokecolor{currentstroke}%
\pgfsetdash{}{0pt}%
\pgfpathmoveto{\pgfqpoint{1.120714in}{2.469548in}}%
\pgfpathcurveto{\pgfqpoint{1.126239in}{2.469548in}}{\pgfqpoint{1.131538in}{2.471743in}}{\pgfqpoint{1.135445in}{2.475650in}}%
\pgfpathcurveto{\pgfqpoint{1.139352in}{2.479557in}}{\pgfqpoint{1.141547in}{2.484856in}}{\pgfqpoint{1.141547in}{2.490381in}}%
\pgfpathcurveto{\pgfqpoint{1.141547in}{2.495907in}}{\pgfqpoint{1.139352in}{2.501206in}}{\pgfqpoint{1.135445in}{2.505113in}}%
\pgfpathcurveto{\pgfqpoint{1.131538in}{2.509020in}}{\pgfqpoint{1.126239in}{2.511215in}}{\pgfqpoint{1.120714in}{2.511215in}}%
\pgfpathcurveto{\pgfqpoint{1.115189in}{2.511215in}}{\pgfqpoint{1.109889in}{2.509020in}}{\pgfqpoint{1.105982in}{2.505113in}}%
\pgfpathcurveto{\pgfqpoint{1.102076in}{2.501206in}}{\pgfqpoint{1.099881in}{2.495907in}}{\pgfqpoint{1.099881in}{2.490381in}}%
\pgfpathcurveto{\pgfqpoint{1.099881in}{2.484856in}}{\pgfqpoint{1.102076in}{2.479557in}}{\pgfqpoint{1.105982in}{2.475650in}}%
\pgfpathcurveto{\pgfqpoint{1.109889in}{2.471743in}}{\pgfqpoint{1.115189in}{2.469548in}}{\pgfqpoint{1.120714in}{2.469548in}}%
\pgfpathclose%
\pgfusepath{fill}%
\end{pgfscope}%
\begin{pgfscope}%
\pgfpathrectangle{\pgfqpoint{0.889225in}{2.423832in}}{\pgfqpoint{1.162500in}{0.755000in}}%
\pgfusepath{clip}%
\pgfsetbuttcap%
\pgfsetroundjoin%
\definecolor{currentfill}{rgb}{0.000000,0.000000,0.000000}%
\pgfsetfillcolor{currentfill}%
\pgfsetfillopacity{0.500000}%
\pgfsetlinewidth{0.000000pt}%
\definecolor{currentstroke}{rgb}{0.000000,0.000000,0.000000}%
\pgfsetstrokecolor{currentstroke}%
\pgfsetdash{}{0pt}%
\pgfpathmoveto{\pgfqpoint{1.126248in}{2.488481in}}%
\pgfpathcurveto{\pgfqpoint{1.131773in}{2.488481in}}{\pgfqpoint{1.137073in}{2.490676in}}{\pgfqpoint{1.140980in}{2.494583in}}%
\pgfpathcurveto{\pgfqpoint{1.144886in}{2.498489in}}{\pgfqpoint{1.147082in}{2.503789in}}{\pgfqpoint{1.147082in}{2.509314in}}%
\pgfpathcurveto{\pgfqpoint{1.147082in}{2.514839in}}{\pgfqpoint{1.144886in}{2.520139in}}{\pgfqpoint{1.140980in}{2.524045in}}%
\pgfpathcurveto{\pgfqpoint{1.137073in}{2.527952in}}{\pgfqpoint{1.131773in}{2.530147in}}{\pgfqpoint{1.126248in}{2.530147in}}%
\pgfpathcurveto{\pgfqpoint{1.120723in}{2.530147in}}{\pgfqpoint{1.115424in}{2.527952in}}{\pgfqpoint{1.111517in}{2.524045in}}%
\pgfpathcurveto{\pgfqpoint{1.107610in}{2.520139in}}{\pgfqpoint{1.105415in}{2.514839in}}{\pgfqpoint{1.105415in}{2.509314in}}%
\pgfpathcurveto{\pgfqpoint{1.105415in}{2.503789in}}{\pgfqpoint{1.107610in}{2.498489in}}{\pgfqpoint{1.111517in}{2.494583in}}%
\pgfpathcurveto{\pgfqpoint{1.115424in}{2.490676in}}{\pgfqpoint{1.120723in}{2.488481in}}{\pgfqpoint{1.126248in}{2.488481in}}%
\pgfpathclose%
\pgfusepath{fill}%
\end{pgfscope}%
\begin{pgfscope}%
\pgfpathrectangle{\pgfqpoint{0.889225in}{2.423832in}}{\pgfqpoint{1.162500in}{0.755000in}}%
\pgfusepath{clip}%
\pgfsetbuttcap%
\pgfsetroundjoin%
\definecolor{currentfill}{rgb}{0.000000,0.000000,0.000000}%
\pgfsetfillcolor{currentfill}%
\pgfsetfillopacity{0.500000}%
\pgfsetlinewidth{0.000000pt}%
\definecolor{currentstroke}{rgb}{0.000000,0.000000,0.000000}%
\pgfsetstrokecolor{currentstroke}%
\pgfsetdash{}{0pt}%
\pgfpathmoveto{\pgfqpoint{0.977704in}{2.422812in}}%
\pgfpathcurveto{\pgfqpoint{0.983229in}{2.422812in}}{\pgfqpoint{0.988528in}{2.425007in}}{\pgfqpoint{0.992435in}{2.428914in}}%
\pgfpathcurveto{\pgfqpoint{0.996342in}{2.432821in}}{\pgfqpoint{0.998537in}{2.438120in}}{\pgfqpoint{0.998537in}{2.443645in}}%
\pgfpathcurveto{\pgfqpoint{0.998537in}{2.449170in}}{\pgfqpoint{0.996342in}{2.454470in}}{\pgfqpoint{0.992435in}{2.458377in}}%
\pgfpathcurveto{\pgfqpoint{0.988528in}{2.462284in}}{\pgfqpoint{0.983229in}{2.464479in}}{\pgfqpoint{0.977704in}{2.464479in}}%
\pgfpathcurveto{\pgfqpoint{0.972179in}{2.464479in}}{\pgfqpoint{0.966879in}{2.462284in}}{\pgfqpoint{0.962972in}{2.458377in}}%
\pgfpathcurveto{\pgfqpoint{0.959066in}{2.454470in}}{\pgfqpoint{0.956870in}{2.449170in}}{\pgfqpoint{0.956870in}{2.443645in}}%
\pgfpathcurveto{\pgfqpoint{0.956870in}{2.438120in}}{\pgfqpoint{0.959066in}{2.432821in}}{\pgfqpoint{0.962972in}{2.428914in}}%
\pgfpathcurveto{\pgfqpoint{0.966879in}{2.425007in}}{\pgfqpoint{0.972179in}{2.422812in}}{\pgfqpoint{0.977704in}{2.422812in}}%
\pgfpathclose%
\pgfusepath{fill}%
\end{pgfscope}%
\begin{pgfscope}%
\pgfpathrectangle{\pgfqpoint{0.889225in}{2.423832in}}{\pgfqpoint{1.162500in}{0.755000in}}%
\pgfusepath{clip}%
\pgfsetbuttcap%
\pgfsetroundjoin%
\definecolor{currentfill}{rgb}{0.000000,0.000000,0.000000}%
\pgfsetfillcolor{currentfill}%
\pgfsetfillopacity{0.500000}%
\pgfsetlinewidth{0.000000pt}%
\definecolor{currentstroke}{rgb}{0.000000,0.000000,0.000000}%
\pgfsetstrokecolor{currentstroke}%
\pgfsetdash{}{0pt}%
\pgfpathmoveto{\pgfqpoint{0.916904in}{2.420975in}}%
\pgfpathcurveto{\pgfqpoint{0.922429in}{2.420975in}}{\pgfqpoint{0.927728in}{2.423170in}}{\pgfqpoint{0.931635in}{2.427077in}}%
\pgfpathcurveto{\pgfqpoint{0.935542in}{2.430984in}}{\pgfqpoint{0.937737in}{2.436283in}}{\pgfqpoint{0.937737in}{2.441809in}}%
\pgfpathcurveto{\pgfqpoint{0.937737in}{2.447334in}}{\pgfqpoint{0.935542in}{2.452633in}}{\pgfqpoint{0.931635in}{2.456540in}}%
\pgfpathcurveto{\pgfqpoint{0.927728in}{2.460447in}}{\pgfqpoint{0.922429in}{2.462642in}}{\pgfqpoint{0.916904in}{2.462642in}}%
\pgfpathcurveto{\pgfqpoint{0.911379in}{2.462642in}}{\pgfqpoint{0.906079in}{2.460447in}}{\pgfqpoint{0.902172in}{2.456540in}}%
\pgfpathcurveto{\pgfqpoint{0.898265in}{2.452633in}}{\pgfqpoint{0.896070in}{2.447334in}}{\pgfqpoint{0.896070in}{2.441809in}}%
\pgfpathcurveto{\pgfqpoint{0.896070in}{2.436283in}}{\pgfqpoint{0.898265in}{2.430984in}}{\pgfqpoint{0.902172in}{2.427077in}}%
\pgfpathcurveto{\pgfqpoint{0.906079in}{2.423170in}}{\pgfqpoint{0.911379in}{2.420975in}}{\pgfqpoint{0.916904in}{2.420975in}}%
\pgfpathclose%
\pgfusepath{fill}%
\end{pgfscope}%
\begin{pgfscope}%
\pgfpathrectangle{\pgfqpoint{0.889225in}{2.423832in}}{\pgfqpoint{1.162500in}{0.755000in}}%
\pgfusepath{clip}%
\pgfsetbuttcap%
\pgfsetroundjoin%
\definecolor{currentfill}{rgb}{0.000000,0.000000,0.000000}%
\pgfsetfillcolor{currentfill}%
\pgfsetfillopacity{0.500000}%
\pgfsetlinewidth{0.000000pt}%
\definecolor{currentstroke}{rgb}{0.000000,0.000000,0.000000}%
\pgfsetstrokecolor{currentstroke}%
\pgfsetdash{}{0pt}%
\pgfpathmoveto{\pgfqpoint{1.691400in}{2.717961in}}%
\pgfpathcurveto{\pgfqpoint{1.696925in}{2.717961in}}{\pgfqpoint{1.702225in}{2.720156in}}{\pgfqpoint{1.706132in}{2.724063in}}%
\pgfpathcurveto{\pgfqpoint{1.710038in}{2.727970in}}{\pgfqpoint{1.712234in}{2.733269in}}{\pgfqpoint{1.712234in}{2.738795in}}%
\pgfpathcurveto{\pgfqpoint{1.712234in}{2.744320in}}{\pgfqpoint{1.710038in}{2.749619in}}{\pgfqpoint{1.706132in}{2.753526in}}%
\pgfpathcurveto{\pgfqpoint{1.702225in}{2.757433in}}{\pgfqpoint{1.696925in}{2.759628in}}{\pgfqpoint{1.691400in}{2.759628in}}%
\pgfpathcurveto{\pgfqpoint{1.685875in}{2.759628in}}{\pgfqpoint{1.680576in}{2.757433in}}{\pgfqpoint{1.676669in}{2.753526in}}%
\pgfpathcurveto{\pgfqpoint{1.672762in}{2.749619in}}{\pgfqpoint{1.670567in}{2.744320in}}{\pgfqpoint{1.670567in}{2.738795in}}%
\pgfpathcurveto{\pgfqpoint{1.670567in}{2.733269in}}{\pgfqpoint{1.672762in}{2.727970in}}{\pgfqpoint{1.676669in}{2.724063in}}%
\pgfpathcurveto{\pgfqpoint{1.680576in}{2.720156in}}{\pgfqpoint{1.685875in}{2.717961in}}{\pgfqpoint{1.691400in}{2.717961in}}%
\pgfpathclose%
\pgfusepath{fill}%
\end{pgfscope}%
\begin{pgfscope}%
\pgfpathrectangle{\pgfqpoint{0.889225in}{2.423832in}}{\pgfqpoint{1.162500in}{0.755000in}}%
\pgfusepath{clip}%
\pgfsetbuttcap%
\pgfsetroundjoin%
\definecolor{currentfill}{rgb}{0.000000,0.000000,0.000000}%
\pgfsetfillcolor{currentfill}%
\pgfsetfillopacity{0.500000}%
\pgfsetlinewidth{0.000000pt}%
\definecolor{currentstroke}{rgb}{0.000000,0.000000,0.000000}%
\pgfsetstrokecolor{currentstroke}%
\pgfsetdash{}{0pt}%
\pgfpathmoveto{\pgfqpoint{1.460463in}{2.640479in}}%
\pgfpathcurveto{\pgfqpoint{1.465988in}{2.640479in}}{\pgfqpoint{1.471288in}{2.642674in}}{\pgfqpoint{1.475194in}{2.646581in}}%
\pgfpathcurveto{\pgfqpoint{1.479101in}{2.650487in}}{\pgfqpoint{1.481296in}{2.655787in}}{\pgfqpoint{1.481296in}{2.661312in}}%
\pgfpathcurveto{\pgfqpoint{1.481296in}{2.666837in}}{\pgfqpoint{1.479101in}{2.672137in}}{\pgfqpoint{1.475194in}{2.676043in}}%
\pgfpathcurveto{\pgfqpoint{1.471288in}{2.679950in}}{\pgfqpoint{1.465988in}{2.682145in}}{\pgfqpoint{1.460463in}{2.682145in}}%
\pgfpathcurveto{\pgfqpoint{1.454938in}{2.682145in}}{\pgfqpoint{1.449638in}{2.679950in}}{\pgfqpoint{1.445732in}{2.676043in}}%
\pgfpathcurveto{\pgfqpoint{1.441825in}{2.672137in}}{\pgfqpoint{1.439630in}{2.666837in}}{\pgfqpoint{1.439630in}{2.661312in}}%
\pgfpathcurveto{\pgfqpoint{1.439630in}{2.655787in}}{\pgfqpoint{1.441825in}{2.650487in}}{\pgfqpoint{1.445732in}{2.646581in}}%
\pgfpathcurveto{\pgfqpoint{1.449638in}{2.642674in}}{\pgfqpoint{1.454938in}{2.640479in}}{\pgfqpoint{1.460463in}{2.640479in}}%
\pgfpathclose%
\pgfusepath{fill}%
\end{pgfscope}%
\begin{pgfscope}%
\pgfpathrectangle{\pgfqpoint{0.889225in}{2.423832in}}{\pgfqpoint{1.162500in}{0.755000in}}%
\pgfusepath{clip}%
\pgfsetbuttcap%
\pgfsetroundjoin%
\definecolor{currentfill}{rgb}{0.000000,0.000000,0.000000}%
\pgfsetfillcolor{currentfill}%
\pgfsetfillopacity{0.500000}%
\pgfsetlinewidth{0.000000pt}%
\definecolor{currentstroke}{rgb}{0.000000,0.000000,0.000000}%
\pgfsetstrokecolor{currentstroke}%
\pgfsetdash{}{0pt}%
\pgfpathmoveto{\pgfqpoint{1.303193in}{2.637393in}}%
\pgfpathcurveto{\pgfqpoint{1.308718in}{2.637393in}}{\pgfqpoint{1.314017in}{2.639588in}}{\pgfqpoint{1.317924in}{2.643495in}}%
\pgfpathcurveto{\pgfqpoint{1.321831in}{2.647402in}}{\pgfqpoint{1.324026in}{2.652701in}}{\pgfqpoint{1.324026in}{2.658226in}}%
\pgfpathcurveto{\pgfqpoint{1.324026in}{2.663751in}}{\pgfqpoint{1.321831in}{2.669051in}}{\pgfqpoint{1.317924in}{2.672958in}}%
\pgfpathcurveto{\pgfqpoint{1.314017in}{2.676865in}}{\pgfqpoint{1.308718in}{2.679060in}}{\pgfqpoint{1.303193in}{2.679060in}}%
\pgfpathcurveto{\pgfqpoint{1.297667in}{2.679060in}}{\pgfqpoint{1.292368in}{2.676865in}}{\pgfqpoint{1.288461in}{2.672958in}}%
\pgfpathcurveto{\pgfqpoint{1.284554in}{2.669051in}}{\pgfqpoint{1.282359in}{2.663751in}}{\pgfqpoint{1.282359in}{2.658226in}}%
\pgfpathcurveto{\pgfqpoint{1.282359in}{2.652701in}}{\pgfqpoint{1.284554in}{2.647402in}}{\pgfqpoint{1.288461in}{2.643495in}}%
\pgfpathcurveto{\pgfqpoint{1.292368in}{2.639588in}}{\pgfqpoint{1.297667in}{2.637393in}}{\pgfqpoint{1.303193in}{2.637393in}}%
\pgfpathclose%
\pgfusepath{fill}%
\end{pgfscope}%
\begin{pgfscope}%
\pgfpathrectangle{\pgfqpoint{0.889225in}{2.423832in}}{\pgfqpoint{1.162500in}{0.755000in}}%
\pgfusepath{clip}%
\pgfsetbuttcap%
\pgfsetroundjoin%
\definecolor{currentfill}{rgb}{0.000000,0.000000,0.000000}%
\pgfsetfillcolor{currentfill}%
\pgfsetfillopacity{0.500000}%
\pgfsetlinewidth{0.000000pt}%
\definecolor{currentstroke}{rgb}{0.000000,0.000000,0.000000}%
\pgfsetstrokecolor{currentstroke}%
\pgfsetdash{}{0pt}%
\pgfpathmoveto{\pgfqpoint{1.689350in}{2.679970in}}%
\pgfpathcurveto{\pgfqpoint{1.694875in}{2.679970in}}{\pgfqpoint{1.700175in}{2.682166in}}{\pgfqpoint{1.704082in}{2.686072in}}%
\pgfpathcurveto{\pgfqpoint{1.707989in}{2.689979in}}{\pgfqpoint{1.710184in}{2.695279in}}{\pgfqpoint{1.710184in}{2.700804in}}%
\pgfpathcurveto{\pgfqpoint{1.710184in}{2.706329in}}{\pgfqpoint{1.707989in}{2.711628in}}{\pgfqpoint{1.704082in}{2.715535in}}%
\pgfpathcurveto{\pgfqpoint{1.700175in}{2.719442in}}{\pgfqpoint{1.694875in}{2.721637in}}{\pgfqpoint{1.689350in}{2.721637in}}%
\pgfpathcurveto{\pgfqpoint{1.683825in}{2.721637in}}{\pgfqpoint{1.678526in}{2.719442in}}{\pgfqpoint{1.674619in}{2.715535in}}%
\pgfpathcurveto{\pgfqpoint{1.670712in}{2.711628in}}{\pgfqpoint{1.668517in}{2.706329in}}{\pgfqpoint{1.668517in}{2.700804in}}%
\pgfpathcurveto{\pgfqpoint{1.668517in}{2.695279in}}{\pgfqpoint{1.670712in}{2.689979in}}{\pgfqpoint{1.674619in}{2.686072in}}%
\pgfpathcurveto{\pgfqpoint{1.678526in}{2.682166in}}{\pgfqpoint{1.683825in}{2.679970in}}{\pgfqpoint{1.689350in}{2.679970in}}%
\pgfpathclose%
\pgfusepath{fill}%
\end{pgfscope}%
\begin{pgfscope}%
\pgfpathrectangle{\pgfqpoint{0.889225in}{2.423832in}}{\pgfqpoint{1.162500in}{0.755000in}}%
\pgfusepath{clip}%
\pgfsetbuttcap%
\pgfsetroundjoin%
\definecolor{currentfill}{rgb}{0.000000,0.000000,0.000000}%
\pgfsetfillcolor{currentfill}%
\pgfsetfillopacity{0.500000}%
\pgfsetlinewidth{0.000000pt}%
\definecolor{currentstroke}{rgb}{0.000000,0.000000,0.000000}%
\pgfsetstrokecolor{currentstroke}%
\pgfsetdash{}{0pt}%
\pgfpathmoveto{\pgfqpoint{1.102396in}{2.481302in}}%
\pgfpathcurveto{\pgfqpoint{1.107921in}{2.481302in}}{\pgfqpoint{1.113221in}{2.483497in}}{\pgfqpoint{1.117128in}{2.487404in}}%
\pgfpathcurveto{\pgfqpoint{1.121035in}{2.491311in}}{\pgfqpoint{1.123230in}{2.496610in}}{\pgfqpoint{1.123230in}{2.502135in}}%
\pgfpathcurveto{\pgfqpoint{1.123230in}{2.507661in}}{\pgfqpoint{1.121035in}{2.512960in}}{\pgfqpoint{1.117128in}{2.516867in}}%
\pgfpathcurveto{\pgfqpoint{1.113221in}{2.520774in}}{\pgfqpoint{1.107921in}{2.522969in}}{\pgfqpoint{1.102396in}{2.522969in}}%
\pgfpathcurveto{\pgfqpoint{1.096871in}{2.522969in}}{\pgfqpoint{1.091572in}{2.520774in}}{\pgfqpoint{1.087665in}{2.516867in}}%
\pgfpathcurveto{\pgfqpoint{1.083758in}{2.512960in}}{\pgfqpoint{1.081563in}{2.507661in}}{\pgfqpoint{1.081563in}{2.502135in}}%
\pgfpathcurveto{\pgfqpoint{1.081563in}{2.496610in}}{\pgfqpoint{1.083758in}{2.491311in}}{\pgfqpoint{1.087665in}{2.487404in}}%
\pgfpathcurveto{\pgfqpoint{1.091572in}{2.483497in}}{\pgfqpoint{1.096871in}{2.481302in}}{\pgfqpoint{1.102396in}{2.481302in}}%
\pgfpathclose%
\pgfusepath{fill}%
\end{pgfscope}%
\begin{pgfscope}%
\pgfpathrectangle{\pgfqpoint{0.889225in}{2.423832in}}{\pgfqpoint{1.162500in}{0.755000in}}%
\pgfusepath{clip}%
\pgfsetbuttcap%
\pgfsetroundjoin%
\definecolor{currentfill}{rgb}{0.000000,0.000000,0.000000}%
\pgfsetfillcolor{currentfill}%
\pgfsetfillopacity{0.500000}%
\pgfsetlinewidth{0.000000pt}%
\definecolor{currentstroke}{rgb}{0.000000,0.000000,0.000000}%
\pgfsetstrokecolor{currentstroke}%
\pgfsetdash{}{0pt}%
\pgfpathmoveto{\pgfqpoint{1.143186in}{2.503012in}}%
\pgfpathcurveto{\pgfqpoint{1.148711in}{2.503012in}}{\pgfqpoint{1.154011in}{2.505207in}}{\pgfqpoint{1.157918in}{2.509114in}}%
\pgfpathcurveto{\pgfqpoint{1.161825in}{2.513021in}}{\pgfqpoint{1.164020in}{2.518320in}}{\pgfqpoint{1.164020in}{2.523845in}}%
\pgfpathcurveto{\pgfqpoint{1.164020in}{2.529370in}}{\pgfqpoint{1.161825in}{2.534670in}}{\pgfqpoint{1.157918in}{2.538577in}}%
\pgfpathcurveto{\pgfqpoint{1.154011in}{2.542483in}}{\pgfqpoint{1.148711in}{2.544679in}}{\pgfqpoint{1.143186in}{2.544679in}}%
\pgfpathcurveto{\pgfqpoint{1.137661in}{2.544679in}}{\pgfqpoint{1.132362in}{2.542483in}}{\pgfqpoint{1.128455in}{2.538577in}}%
\pgfpathcurveto{\pgfqpoint{1.124548in}{2.534670in}}{\pgfqpoint{1.122353in}{2.529370in}}{\pgfqpoint{1.122353in}{2.523845in}}%
\pgfpathcurveto{\pgfqpoint{1.122353in}{2.518320in}}{\pgfqpoint{1.124548in}{2.513021in}}{\pgfqpoint{1.128455in}{2.509114in}}%
\pgfpathcurveto{\pgfqpoint{1.132362in}{2.505207in}}{\pgfqpoint{1.137661in}{2.503012in}}{\pgfqpoint{1.143186in}{2.503012in}}%
\pgfpathclose%
\pgfusepath{fill}%
\end{pgfscope}%
\begin{pgfscope}%
\pgfpathrectangle{\pgfqpoint{0.889225in}{2.423832in}}{\pgfqpoint{1.162500in}{0.755000in}}%
\pgfusepath{clip}%
\pgfsetbuttcap%
\pgfsetroundjoin%
\definecolor{currentfill}{rgb}{0.000000,0.000000,0.000000}%
\pgfsetfillcolor{currentfill}%
\pgfsetfillopacity{0.500000}%
\pgfsetlinewidth{0.000000pt}%
\definecolor{currentstroke}{rgb}{0.000000,0.000000,0.000000}%
\pgfsetstrokecolor{currentstroke}%
\pgfsetdash{}{0pt}%
\pgfpathmoveto{\pgfqpoint{2.024046in}{3.140023in}}%
\pgfpathcurveto{\pgfqpoint{2.029571in}{3.140023in}}{\pgfqpoint{2.034871in}{3.142218in}}{\pgfqpoint{2.038778in}{3.146125in}}%
\pgfpathcurveto{\pgfqpoint{2.042685in}{3.150032in}}{\pgfqpoint{2.044880in}{3.155331in}}{\pgfqpoint{2.044880in}{3.160856in}}%
\pgfpathcurveto{\pgfqpoint{2.044880in}{3.166381in}}{\pgfqpoint{2.042685in}{3.171681in}}{\pgfqpoint{2.038778in}{3.175588in}}%
\pgfpathcurveto{\pgfqpoint{2.034871in}{3.179494in}}{\pgfqpoint{2.029571in}{3.181690in}}{\pgfqpoint{2.024046in}{3.181690in}}%
\pgfpathcurveto{\pgfqpoint{2.018521in}{3.181690in}}{\pgfqpoint{2.013222in}{3.179494in}}{\pgfqpoint{2.009315in}{3.175588in}}%
\pgfpathcurveto{\pgfqpoint{2.005408in}{3.171681in}}{\pgfqpoint{2.003213in}{3.166381in}}{\pgfqpoint{2.003213in}{3.160856in}}%
\pgfpathcurveto{\pgfqpoint{2.003213in}{3.155331in}}{\pgfqpoint{2.005408in}{3.150032in}}{\pgfqpoint{2.009315in}{3.146125in}}%
\pgfpathcurveto{\pgfqpoint{2.013222in}{3.142218in}}{\pgfqpoint{2.018521in}{3.140023in}}{\pgfqpoint{2.024046in}{3.140023in}}%
\pgfpathclose%
\pgfusepath{fill}%
\end{pgfscope}%
\begin{pgfscope}%
\pgfpathrectangle{\pgfqpoint{0.889225in}{2.423832in}}{\pgfqpoint{1.162500in}{0.755000in}}%
\pgfusepath{clip}%
\pgfsetbuttcap%
\pgfsetroundjoin%
\definecolor{currentfill}{rgb}{0.000000,0.000000,0.000000}%
\pgfsetfillcolor{currentfill}%
\pgfsetfillopacity{0.500000}%
\pgfsetlinewidth{0.000000pt}%
\definecolor{currentstroke}{rgb}{0.000000,0.000000,0.000000}%
\pgfsetstrokecolor{currentstroke}%
\pgfsetdash{}{0pt}%
\pgfpathmoveto{\pgfqpoint{1.402285in}{2.625365in}}%
\pgfpathcurveto{\pgfqpoint{1.407810in}{2.625365in}}{\pgfqpoint{1.413110in}{2.627561in}}{\pgfqpoint{1.417016in}{2.631467in}}%
\pgfpathcurveto{\pgfqpoint{1.420923in}{2.635374in}}{\pgfqpoint{1.423118in}{2.640674in}}{\pgfqpoint{1.423118in}{2.646199in}}%
\pgfpathcurveto{\pgfqpoint{1.423118in}{2.651724in}}{\pgfqpoint{1.420923in}{2.657023in}}{\pgfqpoint{1.417016in}{2.660930in}}%
\pgfpathcurveto{\pgfqpoint{1.413110in}{2.664837in}}{\pgfqpoint{1.407810in}{2.667032in}}{\pgfqpoint{1.402285in}{2.667032in}}%
\pgfpathcurveto{\pgfqpoint{1.396760in}{2.667032in}}{\pgfqpoint{1.391460in}{2.664837in}}{\pgfqpoint{1.387554in}{2.660930in}}%
\pgfpathcurveto{\pgfqpoint{1.383647in}{2.657023in}}{\pgfqpoint{1.381452in}{2.651724in}}{\pgfqpoint{1.381452in}{2.646199in}}%
\pgfpathcurveto{\pgfqpoint{1.381452in}{2.640674in}}{\pgfqpoint{1.383647in}{2.635374in}}{\pgfqpoint{1.387554in}{2.631467in}}%
\pgfpathcurveto{\pgfqpoint{1.391460in}{2.627561in}}{\pgfqpoint{1.396760in}{2.625365in}}{\pgfqpoint{1.402285in}{2.625365in}}%
\pgfpathclose%
\pgfusepath{fill}%
\end{pgfscope}%
\begin{pgfscope}%
\pgfpathrectangle{\pgfqpoint{0.889225in}{2.423832in}}{\pgfqpoint{1.162500in}{0.755000in}}%
\pgfusepath{clip}%
\pgfsetbuttcap%
\pgfsetroundjoin%
\definecolor{currentfill}{rgb}{0.000000,0.000000,0.000000}%
\pgfsetfillcolor{currentfill}%
\pgfsetfillopacity{0.500000}%
\pgfsetlinewidth{0.000000pt}%
\definecolor{currentstroke}{rgb}{0.000000,0.000000,0.000000}%
\pgfsetstrokecolor{currentstroke}%
\pgfsetdash{}{0pt}%
\pgfpathmoveto{\pgfqpoint{0.973483in}{2.450809in}}%
\pgfpathcurveto{\pgfqpoint{0.979008in}{2.450809in}}{\pgfqpoint{0.984308in}{2.453004in}}{\pgfqpoint{0.988214in}{2.456911in}}%
\pgfpathcurveto{\pgfqpoint{0.992121in}{2.460818in}}{\pgfqpoint{0.994316in}{2.466117in}}{\pgfqpoint{0.994316in}{2.471642in}}%
\pgfpathcurveto{\pgfqpoint{0.994316in}{2.477167in}}{\pgfqpoint{0.992121in}{2.482467in}}{\pgfqpoint{0.988214in}{2.486374in}}%
\pgfpathcurveto{\pgfqpoint{0.984308in}{2.490280in}}{\pgfqpoint{0.979008in}{2.492476in}}{\pgfqpoint{0.973483in}{2.492476in}}%
\pgfpathcurveto{\pgfqpoint{0.967958in}{2.492476in}}{\pgfqpoint{0.962658in}{2.490280in}}{\pgfqpoint{0.958752in}{2.486374in}}%
\pgfpathcurveto{\pgfqpoint{0.954845in}{2.482467in}}{\pgfqpoint{0.952650in}{2.477167in}}{\pgfqpoint{0.952650in}{2.471642in}}%
\pgfpathcurveto{\pgfqpoint{0.952650in}{2.466117in}}{\pgfqpoint{0.954845in}{2.460818in}}{\pgfqpoint{0.958752in}{2.456911in}}%
\pgfpathcurveto{\pgfqpoint{0.962658in}{2.453004in}}{\pgfqpoint{0.967958in}{2.450809in}}{\pgfqpoint{0.973483in}{2.450809in}}%
\pgfpathclose%
\pgfusepath{fill}%
\end{pgfscope}%
\begin{pgfscope}%
\pgfsetbuttcap%
\pgfsetroundjoin%
\definecolor{currentfill}{rgb}{0.000000,0.000000,0.000000}%
\pgfsetfillcolor{currentfill}%
\pgfsetlinewidth{0.803000pt}%
\definecolor{currentstroke}{rgb}{0.000000,0.000000,0.000000}%
\pgfsetstrokecolor{currentstroke}%
\pgfsetdash{}{0pt}%
\pgfsys@defobject{currentmarker}{\pgfqpoint{-0.048611in}{0.000000in}}{\pgfqpoint{0.000000in}{0.000000in}}{%
\pgfpathmoveto{\pgfqpoint{0.000000in}{0.000000in}}%
\pgfpathlineto{\pgfqpoint{-0.048611in}{0.000000in}}%
\pgfusepath{stroke,fill}%
}%
\begin{pgfscope}%
\pgfsys@transformshift{0.889225in}{2.691122in}%
\pgfsys@useobject{currentmarker}{}%
\end{pgfscope}%
\end{pgfscope}%
\begin{pgfscope}%
\pgftext[x=0.641152in,y=2.648913in,left,base]{\rmfamily\fontsize{8.000000}{9.600000}\selectfont \(\displaystyle 0.5\)}%
\end{pgfscope}%
\begin{pgfscope}%
\pgfsetbuttcap%
\pgfsetroundjoin%
\definecolor{currentfill}{rgb}{0.000000,0.000000,0.000000}%
\pgfsetfillcolor{currentfill}%
\pgfsetlinewidth{0.803000pt}%
\definecolor{currentstroke}{rgb}{0.000000,0.000000,0.000000}%
\pgfsetstrokecolor{currentstroke}%
\pgfsetdash{}{0pt}%
\pgfsys@defobject{currentmarker}{\pgfqpoint{-0.048611in}{0.000000in}}{\pgfqpoint{0.000000in}{0.000000in}}{%
\pgfpathmoveto{\pgfqpoint{0.000000in}{0.000000in}}%
\pgfpathlineto{\pgfqpoint{-0.048611in}{0.000000in}}%
\pgfusepath{stroke,fill}%
}%
\begin{pgfscope}%
\pgfsys@transformshift{0.889225in}{2.985310in}%
\pgfsys@useobject{currentmarker}{}%
\end{pgfscope}%
\end{pgfscope}%
\begin{pgfscope}%
\pgftext[x=0.641152in,y=2.943101in,left,base]{\rmfamily\fontsize{8.000000}{9.600000}\selectfont \(\displaystyle 1.0\)}%
\end{pgfscope}%
\begin{pgfscope}%
\pgftext[x=0.585596in,y=2.801332in,,bottom,rotate=90.000000]{\rmfamily\fontsize{16.000000}{19.200000}\selectfont charge}%
\end{pgfscope}%
\begin{pgfscope}%
\pgftext[x=0.889225in,y=3.220499in,left,base]{\rmfamily\fontsize{16.000000}{19.200000}\selectfont \(\displaystyle \times10^{-9}\)}%
\end{pgfscope}%
\begin{pgfscope}%
\pgfsetrectcap%
\pgfsetmiterjoin%
\pgfsetlinewidth{0.803000pt}%
\definecolor{currentstroke}{rgb}{0.501961,0.501961,0.501961}%
\pgfsetstrokecolor{currentstroke}%
\pgfsetdash{}{0pt}%
\pgfpathmoveto{\pgfqpoint{0.889225in}{2.423832in}}%
\pgfpathlineto{\pgfqpoint{0.889225in}{3.178832in}}%
\pgfusepath{stroke}%
\end{pgfscope}%
\begin{pgfscope}%
\pgfsetrectcap%
\pgfsetmiterjoin%
\pgfsetlinewidth{0.803000pt}%
\definecolor{currentstroke}{rgb}{0.501961,0.501961,0.501961}%
\pgfsetstrokecolor{currentstroke}%
\pgfsetdash{}{0pt}%
\pgfpathmoveto{\pgfqpoint{2.051725in}{2.423832in}}%
\pgfpathlineto{\pgfqpoint{2.051725in}{3.178832in}}%
\pgfusepath{stroke}%
\end{pgfscope}%
\begin{pgfscope}%
\pgfsetrectcap%
\pgfsetmiterjoin%
\pgfsetlinewidth{0.803000pt}%
\definecolor{currentstroke}{rgb}{0.501961,0.501961,0.501961}%
\pgfsetstrokecolor{currentstroke}%
\pgfsetdash{}{0pt}%
\pgfpathmoveto{\pgfqpoint{0.889225in}{2.423832in}}%
\pgfpathlineto{\pgfqpoint{2.051725in}{2.423832in}}%
\pgfusepath{stroke}%
\end{pgfscope}%
\begin{pgfscope}%
\pgfsetrectcap%
\pgfsetmiterjoin%
\pgfsetlinewidth{0.803000pt}%
\definecolor{currentstroke}{rgb}{0.501961,0.501961,0.501961}%
\pgfsetstrokecolor{currentstroke}%
\pgfsetdash{}{0pt}%
\pgfpathmoveto{\pgfqpoint{0.889225in}{3.178832in}}%
\pgfpathlineto{\pgfqpoint{2.051725in}{3.178832in}}%
\pgfusepath{stroke}%
\end{pgfscope}%
\begin{pgfscope}%
\pgfsetbuttcap%
\pgfsetmiterjoin%
\definecolor{currentfill}{rgb}{1.000000,1.000000,1.000000}%
\pgfsetfillcolor{currentfill}%
\pgfsetlinewidth{0.000000pt}%
\definecolor{currentstroke}{rgb}{0.000000,0.000000,0.000000}%
\pgfsetstrokecolor{currentstroke}%
\pgfsetstrokeopacity{0.000000}%
\pgfsetdash{}{0pt}%
\pgfpathmoveto{\pgfqpoint{2.051725in}{2.423832in}}%
\pgfpathlineto{\pgfqpoint{3.214225in}{2.423832in}}%
\pgfpathlineto{\pgfqpoint{3.214225in}{3.178832in}}%
\pgfpathlineto{\pgfqpoint{2.051725in}{3.178832in}}%
\pgfpathclose%
\pgfusepath{fill}%
\end{pgfscope}%
\begin{pgfscope}%
\pgfpathrectangle{\pgfqpoint{2.051725in}{2.423832in}}{\pgfqpoint{1.162500in}{0.755000in}}%
\pgfusepath{clip}%
\pgfsetrectcap%
\pgfsetroundjoin%
\pgfsetlinewidth{1.505625pt}%
\definecolor{currentstroke}{rgb}{0.121569,0.466667,0.705882}%
\pgfsetstrokecolor{currentstroke}%
\pgfsetdash{}{0pt}%
\pgfpathmoveto{\pgfqpoint{2.079404in}{2.981458in}}%
\pgfpathlineto{\pgfqpoint{2.103785in}{3.023270in}}%
\pgfpathlineto{\pgfqpoint{2.124842in}{3.055299in}}%
\pgfpathlineto{\pgfqpoint{2.143682in}{3.080259in}}%
\pgfpathlineto{\pgfqpoint{2.161414in}{3.100257in}}%
\pgfpathlineto{\pgfqpoint{2.178038in}{3.115741in}}%
\pgfpathlineto{\pgfqpoint{2.193553in}{3.127237in}}%
\pgfpathlineto{\pgfqpoint{2.207961in}{3.135306in}}%
\pgfpathlineto{\pgfqpoint{2.222368in}{3.140849in}}%
\pgfpathlineto{\pgfqpoint{2.235667in}{3.143732in}}%
\pgfpathlineto{\pgfqpoint{2.248966in}{3.144496in}}%
\pgfpathlineto{\pgfqpoint{2.262265in}{3.143183in}}%
\pgfpathlineto{\pgfqpoint{2.276672in}{3.139480in}}%
\pgfpathlineto{\pgfqpoint{2.291080in}{3.133489in}}%
\pgfpathlineto{\pgfqpoint{2.306595in}{3.124594in}}%
\pgfpathlineto{\pgfqpoint{2.323219in}{3.112404in}}%
\pgfpathlineto{\pgfqpoint{2.340951in}{3.096574in}}%
\pgfpathlineto{\pgfqpoint{2.360899in}{3.075567in}}%
\pgfpathlineto{\pgfqpoint{2.381956in}{3.050096in}}%
\pgfpathlineto{\pgfqpoint{2.406338in}{3.016922in}}%
\pgfpathlineto{\pgfqpoint{2.434044in}{2.975246in}}%
\pgfpathlineto{\pgfqpoint{2.467291in}{2.920995in}}%
\pgfpathlineto{\pgfqpoint{2.514946in}{2.838468in}}%
\pgfpathlineto{\pgfqpoint{2.590307in}{2.707949in}}%
\pgfpathlineto{\pgfqpoint{2.625771in}{2.651279in}}%
\pgfpathlineto{\pgfqpoint{2.654586in}{2.609241in}}%
\pgfpathlineto{\pgfqpoint{2.680076in}{2.575738in}}%
\pgfpathlineto{\pgfqpoint{2.704457in}{2.547348in}}%
\pgfpathlineto{\pgfqpoint{2.726622in}{2.524875in}}%
\pgfpathlineto{\pgfqpoint{2.747679in}{2.506592in}}%
\pgfpathlineto{\pgfqpoint{2.767628in}{2.492072in}}%
\pgfpathlineto{\pgfqpoint{2.787576in}{2.480266in}}%
\pgfpathlineto{\pgfqpoint{2.807525in}{2.471117in}}%
\pgfpathlineto{\pgfqpoint{2.827473in}{2.464520in}}%
\pgfpathlineto{\pgfqpoint{2.847422in}{2.460331in}}%
\pgfpathlineto{\pgfqpoint{2.867370in}{2.458366in}}%
\pgfpathlineto{\pgfqpoint{2.888427in}{2.458462in}}%
\pgfpathlineto{\pgfqpoint{2.911700in}{2.460809in}}%
\pgfpathlineto{\pgfqpoint{2.938298in}{2.465819in}}%
\pgfpathlineto{\pgfqpoint{2.971546in}{2.474517in}}%
\pgfpathlineto{\pgfqpoint{3.079046in}{2.504560in}}%
\pgfpathlineto{\pgfqpoint{3.107861in}{2.509223in}}%
\pgfpathlineto{\pgfqpoint{3.134459in}{2.511245in}}%
\pgfpathlineto{\pgfqpoint{3.159948in}{2.510915in}}%
\pgfpathlineto{\pgfqpoint{3.185438in}{2.508314in}}%
\pgfpathlineto{\pgfqpoint{3.186546in}{2.508151in}}%
\pgfpathlineto{\pgfqpoint{3.186546in}{2.508151in}}%
\pgfusepath{stroke}%
\end{pgfscope}%
\begin{pgfscope}%
\pgfsetrectcap%
\pgfsetmiterjoin%
\pgfsetlinewidth{0.803000pt}%
\definecolor{currentstroke}{rgb}{0.501961,0.501961,0.501961}%
\pgfsetstrokecolor{currentstroke}%
\pgfsetdash{}{0pt}%
\pgfpathmoveto{\pgfqpoint{2.051725in}{2.423832in}}%
\pgfpathlineto{\pgfqpoint{2.051725in}{3.178832in}}%
\pgfusepath{stroke}%
\end{pgfscope}%
\begin{pgfscope}%
\pgfsetrectcap%
\pgfsetmiterjoin%
\pgfsetlinewidth{0.803000pt}%
\definecolor{currentstroke}{rgb}{0.501961,0.501961,0.501961}%
\pgfsetstrokecolor{currentstroke}%
\pgfsetdash{}{0pt}%
\pgfpathmoveto{\pgfqpoint{3.214225in}{2.423832in}}%
\pgfpathlineto{\pgfqpoint{3.214225in}{3.178832in}}%
\pgfusepath{stroke}%
\end{pgfscope}%
\begin{pgfscope}%
\pgfsetrectcap%
\pgfsetmiterjoin%
\pgfsetlinewidth{0.803000pt}%
\definecolor{currentstroke}{rgb}{0.501961,0.501961,0.501961}%
\pgfsetstrokecolor{currentstroke}%
\pgfsetdash{}{0pt}%
\pgfpathmoveto{\pgfqpoint{2.051725in}{2.423832in}}%
\pgfpathlineto{\pgfqpoint{3.214225in}{2.423832in}}%
\pgfusepath{stroke}%
\end{pgfscope}%
\begin{pgfscope}%
\pgfsetrectcap%
\pgfsetmiterjoin%
\pgfsetlinewidth{0.803000pt}%
\definecolor{currentstroke}{rgb}{0.501961,0.501961,0.501961}%
\pgfsetstrokecolor{currentstroke}%
\pgfsetdash{}{0pt}%
\pgfpathmoveto{\pgfqpoint{2.051725in}{3.178832in}}%
\pgfpathlineto{\pgfqpoint{3.214225in}{3.178832in}}%
\pgfusepath{stroke}%
\end{pgfscope}%
\begin{pgfscope}%
\pgfsetbuttcap%
\pgfsetmiterjoin%
\definecolor{currentfill}{rgb}{1.000000,1.000000,1.000000}%
\pgfsetfillcolor{currentfill}%
\pgfsetlinewidth{0.000000pt}%
\definecolor{currentstroke}{rgb}{0.000000,0.000000,0.000000}%
\pgfsetstrokecolor{currentstroke}%
\pgfsetstrokeopacity{0.000000}%
\pgfsetdash{}{0pt}%
\pgfpathmoveto{\pgfqpoint{3.214225in}{2.423832in}}%
\pgfpathlineto{\pgfqpoint{4.376725in}{2.423832in}}%
\pgfpathlineto{\pgfqpoint{4.376725in}{3.178832in}}%
\pgfpathlineto{\pgfqpoint{3.214225in}{3.178832in}}%
\pgfpathclose%
\pgfusepath{fill}%
\end{pgfscope}%
\begin{pgfscope}%
\pgfpathrectangle{\pgfqpoint{3.214225in}{2.423832in}}{\pgfqpoint{1.162500in}{0.755000in}}%
\pgfusepath{clip}%
\pgfsetbuttcap%
\pgfsetroundjoin%
\definecolor{currentfill}{rgb}{0.000000,0.000000,0.000000}%
\pgfsetfillcolor{currentfill}%
\pgfsetfillopacity{0.500000}%
\pgfsetlinewidth{0.000000pt}%
\definecolor{currentstroke}{rgb}{0.000000,0.000000,0.000000}%
\pgfsetstrokecolor{currentstroke}%
\pgfsetdash{}{0pt}%
\pgfpathmoveto{\pgfqpoint{3.656364in}{3.088730in}}%
\pgfpathcurveto{\pgfqpoint{3.661889in}{3.088730in}}{\pgfqpoint{3.667189in}{3.090925in}}{\pgfqpoint{3.671096in}{3.094832in}}%
\pgfpathcurveto{\pgfqpoint{3.675002in}{3.098739in}}{\pgfqpoint{3.677198in}{3.104038in}}{\pgfqpoint{3.677198in}{3.109563in}}%
\pgfpathcurveto{\pgfqpoint{3.677198in}{3.115089in}}{\pgfqpoint{3.675002in}{3.120388in}}{\pgfqpoint{3.671096in}{3.124295in}}%
\pgfpathcurveto{\pgfqpoint{3.667189in}{3.128202in}}{\pgfqpoint{3.661889in}{3.130397in}}{\pgfqpoint{3.656364in}{3.130397in}}%
\pgfpathcurveto{\pgfqpoint{3.650839in}{3.130397in}}{\pgfqpoint{3.645540in}{3.128202in}}{\pgfqpoint{3.641633in}{3.124295in}}%
\pgfpathcurveto{\pgfqpoint{3.637726in}{3.120388in}}{\pgfqpoint{3.635531in}{3.115089in}}{\pgfqpoint{3.635531in}{3.109563in}}%
\pgfpathcurveto{\pgfqpoint{3.635531in}{3.104038in}}{\pgfqpoint{3.637726in}{3.098739in}}{\pgfqpoint{3.641633in}{3.094832in}}%
\pgfpathcurveto{\pgfqpoint{3.645540in}{3.090925in}}{\pgfqpoint{3.650839in}{3.088730in}}{\pgfqpoint{3.656364in}{3.088730in}}%
\pgfpathclose%
\pgfusepath{fill}%
\end{pgfscope}%
\begin{pgfscope}%
\pgfpathrectangle{\pgfqpoint{3.214225in}{2.423832in}}{\pgfqpoint{1.162500in}{0.755000in}}%
\pgfusepath{clip}%
\pgfsetbuttcap%
\pgfsetroundjoin%
\definecolor{currentfill}{rgb}{0.000000,0.000000,0.000000}%
\pgfsetfillcolor{currentfill}%
\pgfsetfillopacity{0.500000}%
\pgfsetlinewidth{0.000000pt}%
\definecolor{currentstroke}{rgb}{0.000000,0.000000,0.000000}%
\pgfsetstrokecolor{currentstroke}%
\pgfsetdash{}{0pt}%
\pgfpathmoveto{\pgfqpoint{4.259629in}{2.544873in}}%
\pgfpathcurveto{\pgfqpoint{4.265154in}{2.544873in}}{\pgfqpoint{4.270454in}{2.547068in}}{\pgfqpoint{4.274361in}{2.550975in}}%
\pgfpathcurveto{\pgfqpoint{4.278268in}{2.554881in}}{\pgfqpoint{4.280463in}{2.560181in}}{\pgfqpoint{4.280463in}{2.565706in}}%
\pgfpathcurveto{\pgfqpoint{4.280463in}{2.571231in}}{\pgfqpoint{4.278268in}{2.576531in}}{\pgfqpoint{4.274361in}{2.580437in}}%
\pgfpathcurveto{\pgfqpoint{4.270454in}{2.584344in}}{\pgfqpoint{4.265154in}{2.586539in}}{\pgfqpoint{4.259629in}{2.586539in}}%
\pgfpathcurveto{\pgfqpoint{4.254104in}{2.586539in}}{\pgfqpoint{4.248805in}{2.584344in}}{\pgfqpoint{4.244898in}{2.580437in}}%
\pgfpathcurveto{\pgfqpoint{4.240991in}{2.576531in}}{\pgfqpoint{4.238796in}{2.571231in}}{\pgfqpoint{4.238796in}{2.565706in}}%
\pgfpathcurveto{\pgfqpoint{4.238796in}{2.560181in}}{\pgfqpoint{4.240991in}{2.554881in}}{\pgfqpoint{4.244898in}{2.550975in}}%
\pgfpathcurveto{\pgfqpoint{4.248805in}{2.547068in}}{\pgfqpoint{4.254104in}{2.544873in}}{\pgfqpoint{4.259629in}{2.544873in}}%
\pgfpathclose%
\pgfusepath{fill}%
\end{pgfscope}%
\begin{pgfscope}%
\pgfpathrectangle{\pgfqpoint{3.214225in}{2.423832in}}{\pgfqpoint{1.162500in}{0.755000in}}%
\pgfusepath{clip}%
\pgfsetbuttcap%
\pgfsetroundjoin%
\definecolor{currentfill}{rgb}{0.000000,0.000000,0.000000}%
\pgfsetfillcolor{currentfill}%
\pgfsetfillopacity{0.500000}%
\pgfsetlinewidth{0.000000pt}%
\definecolor{currentstroke}{rgb}{0.000000,0.000000,0.000000}%
\pgfsetstrokecolor{currentstroke}%
\pgfsetdash{}{0pt}%
\pgfpathmoveto{\pgfqpoint{4.349046in}{2.541427in}}%
\pgfpathcurveto{\pgfqpoint{4.354571in}{2.541427in}}{\pgfqpoint{4.359871in}{2.543622in}}{\pgfqpoint{4.363778in}{2.547529in}}%
\pgfpathcurveto{\pgfqpoint{4.367685in}{2.551436in}}{\pgfqpoint{4.369880in}{2.556735in}}{\pgfqpoint{4.369880in}{2.562260in}}%
\pgfpathcurveto{\pgfqpoint{4.369880in}{2.567785in}}{\pgfqpoint{4.367685in}{2.573085in}}{\pgfqpoint{4.363778in}{2.576992in}}%
\pgfpathcurveto{\pgfqpoint{4.359871in}{2.580899in}}{\pgfqpoint{4.354571in}{2.583094in}}{\pgfqpoint{4.349046in}{2.583094in}}%
\pgfpathcurveto{\pgfqpoint{4.343521in}{2.583094in}}{\pgfqpoint{4.338222in}{2.580899in}}{\pgfqpoint{4.334315in}{2.576992in}}%
\pgfpathcurveto{\pgfqpoint{4.330408in}{2.573085in}}{\pgfqpoint{4.328213in}{2.567785in}}{\pgfqpoint{4.328213in}{2.562260in}}%
\pgfpathcurveto{\pgfqpoint{4.328213in}{2.556735in}}{\pgfqpoint{4.330408in}{2.551436in}}{\pgfqpoint{4.334315in}{2.547529in}}%
\pgfpathcurveto{\pgfqpoint{4.338222in}{2.543622in}}{\pgfqpoint{4.343521in}{2.541427in}}{\pgfqpoint{4.349046in}{2.541427in}}%
\pgfpathclose%
\pgfusepath{fill}%
\end{pgfscope}%
\begin{pgfscope}%
\pgfpathrectangle{\pgfqpoint{3.214225in}{2.423832in}}{\pgfqpoint{1.162500in}{0.755000in}}%
\pgfusepath{clip}%
\pgfsetbuttcap%
\pgfsetroundjoin%
\definecolor{currentfill}{rgb}{0.000000,0.000000,0.000000}%
\pgfsetfillcolor{currentfill}%
\pgfsetfillopacity{0.500000}%
\pgfsetlinewidth{0.000000pt}%
\definecolor{currentstroke}{rgb}{0.000000,0.000000,0.000000}%
\pgfsetstrokecolor{currentstroke}%
\pgfsetdash{}{0pt}%
\pgfpathmoveto{\pgfqpoint{3.915823in}{2.469548in}}%
\pgfpathcurveto{\pgfqpoint{3.921348in}{2.469548in}}{\pgfqpoint{3.926648in}{2.471743in}}{\pgfqpoint{3.930554in}{2.475650in}}%
\pgfpathcurveto{\pgfqpoint{3.934461in}{2.479557in}}{\pgfqpoint{3.936656in}{2.484856in}}{\pgfqpoint{3.936656in}{2.490381in}}%
\pgfpathcurveto{\pgfqpoint{3.936656in}{2.495907in}}{\pgfqpoint{3.934461in}{2.501206in}}{\pgfqpoint{3.930554in}{2.505113in}}%
\pgfpathcurveto{\pgfqpoint{3.926648in}{2.509020in}}{\pgfqpoint{3.921348in}{2.511215in}}{\pgfqpoint{3.915823in}{2.511215in}}%
\pgfpathcurveto{\pgfqpoint{3.910298in}{2.511215in}}{\pgfqpoint{3.904998in}{2.509020in}}{\pgfqpoint{3.901092in}{2.505113in}}%
\pgfpathcurveto{\pgfqpoint{3.897185in}{2.501206in}}{\pgfqpoint{3.894990in}{2.495907in}}{\pgfqpoint{3.894990in}{2.490381in}}%
\pgfpathcurveto{\pgfqpoint{3.894990in}{2.484856in}}{\pgfqpoint{3.897185in}{2.479557in}}{\pgfqpoint{3.901092in}{2.475650in}}%
\pgfpathcurveto{\pgfqpoint{3.904998in}{2.471743in}}{\pgfqpoint{3.910298in}{2.469548in}}{\pgfqpoint{3.915823in}{2.469548in}}%
\pgfpathclose%
\pgfusepath{fill}%
\end{pgfscope}%
\begin{pgfscope}%
\pgfpathrectangle{\pgfqpoint{3.214225in}{2.423832in}}{\pgfqpoint{1.162500in}{0.755000in}}%
\pgfusepath{clip}%
\pgfsetbuttcap%
\pgfsetroundjoin%
\definecolor{currentfill}{rgb}{0.000000,0.000000,0.000000}%
\pgfsetfillcolor{currentfill}%
\pgfsetfillopacity{0.500000}%
\pgfsetlinewidth{0.000000pt}%
\definecolor{currentstroke}{rgb}{0.000000,0.000000,0.000000}%
\pgfsetstrokecolor{currentstroke}%
\pgfsetdash{}{0pt}%
\pgfpathmoveto{\pgfqpoint{3.986481in}{2.488481in}}%
\pgfpathcurveto{\pgfqpoint{3.992006in}{2.488481in}}{\pgfqpoint{3.997306in}{2.490676in}}{\pgfqpoint{4.001213in}{2.494583in}}%
\pgfpathcurveto{\pgfqpoint{4.005120in}{2.498489in}}{\pgfqpoint{4.007315in}{2.503789in}}{\pgfqpoint{4.007315in}{2.509314in}}%
\pgfpathcurveto{\pgfqpoint{4.007315in}{2.514839in}}{\pgfqpoint{4.005120in}{2.520139in}}{\pgfqpoint{4.001213in}{2.524045in}}%
\pgfpathcurveto{\pgfqpoint{3.997306in}{2.527952in}}{\pgfqpoint{3.992006in}{2.530147in}}{\pgfqpoint{3.986481in}{2.530147in}}%
\pgfpathcurveto{\pgfqpoint{3.980956in}{2.530147in}}{\pgfqpoint{3.975657in}{2.527952in}}{\pgfqpoint{3.971750in}{2.524045in}}%
\pgfpathcurveto{\pgfqpoint{3.967843in}{2.520139in}}{\pgfqpoint{3.965648in}{2.514839in}}{\pgfqpoint{3.965648in}{2.509314in}}%
\pgfpathcurveto{\pgfqpoint{3.965648in}{2.503789in}}{\pgfqpoint{3.967843in}{2.498489in}}{\pgfqpoint{3.971750in}{2.494583in}}%
\pgfpathcurveto{\pgfqpoint{3.975657in}{2.490676in}}{\pgfqpoint{3.980956in}{2.488481in}}{\pgfqpoint{3.986481in}{2.488481in}}%
\pgfpathclose%
\pgfusepath{fill}%
\end{pgfscope}%
\begin{pgfscope}%
\pgfpathrectangle{\pgfqpoint{3.214225in}{2.423832in}}{\pgfqpoint{1.162500in}{0.755000in}}%
\pgfusepath{clip}%
\pgfsetbuttcap%
\pgfsetroundjoin%
\definecolor{currentfill}{rgb}{0.000000,0.000000,0.000000}%
\pgfsetfillcolor{currentfill}%
\pgfsetfillopacity{0.500000}%
\pgfsetlinewidth{0.000000pt}%
\definecolor{currentstroke}{rgb}{0.000000,0.000000,0.000000}%
\pgfsetstrokecolor{currentstroke}%
\pgfsetdash{}{0pt}%
\pgfpathmoveto{\pgfqpoint{3.831974in}{2.422812in}}%
\pgfpathcurveto{\pgfqpoint{3.837499in}{2.422812in}}{\pgfqpoint{3.842798in}{2.425007in}}{\pgfqpoint{3.846705in}{2.428914in}}%
\pgfpathcurveto{\pgfqpoint{3.850612in}{2.432821in}}{\pgfqpoint{3.852807in}{2.438120in}}{\pgfqpoint{3.852807in}{2.443645in}}%
\pgfpathcurveto{\pgfqpoint{3.852807in}{2.449170in}}{\pgfqpoint{3.850612in}{2.454470in}}{\pgfqpoint{3.846705in}{2.458377in}}%
\pgfpathcurveto{\pgfqpoint{3.842798in}{2.462284in}}{\pgfqpoint{3.837499in}{2.464479in}}{\pgfqpoint{3.831974in}{2.464479in}}%
\pgfpathcurveto{\pgfqpoint{3.826449in}{2.464479in}}{\pgfqpoint{3.821149in}{2.462284in}}{\pgfqpoint{3.817242in}{2.458377in}}%
\pgfpathcurveto{\pgfqpoint{3.813336in}{2.454470in}}{\pgfqpoint{3.811140in}{2.449170in}}{\pgfqpoint{3.811140in}{2.443645in}}%
\pgfpathcurveto{\pgfqpoint{3.811140in}{2.438120in}}{\pgfqpoint{3.813336in}{2.432821in}}{\pgfqpoint{3.817242in}{2.428914in}}%
\pgfpathcurveto{\pgfqpoint{3.821149in}{2.425007in}}{\pgfqpoint{3.826449in}{2.422812in}}{\pgfqpoint{3.831974in}{2.422812in}}%
\pgfpathclose%
\pgfusepath{fill}%
\end{pgfscope}%
\begin{pgfscope}%
\pgfpathrectangle{\pgfqpoint{3.214225in}{2.423832in}}{\pgfqpoint{1.162500in}{0.755000in}}%
\pgfusepath{clip}%
\pgfsetbuttcap%
\pgfsetroundjoin%
\definecolor{currentfill}{rgb}{0.000000,0.000000,0.000000}%
\pgfsetfillcolor{currentfill}%
\pgfsetfillopacity{0.500000}%
\pgfsetlinewidth{0.000000pt}%
\definecolor{currentstroke}{rgb}{0.000000,0.000000,0.000000}%
\pgfsetstrokecolor{currentstroke}%
\pgfsetdash{}{0pt}%
\pgfpathmoveto{\pgfqpoint{3.241904in}{2.420975in}}%
\pgfpathcurveto{\pgfqpoint{3.247429in}{2.420975in}}{\pgfqpoint{3.252728in}{2.423170in}}{\pgfqpoint{3.256635in}{2.427077in}}%
\pgfpathcurveto{\pgfqpoint{3.260542in}{2.430984in}}{\pgfqpoint{3.262737in}{2.436283in}}{\pgfqpoint{3.262737in}{2.441809in}}%
\pgfpathcurveto{\pgfqpoint{3.262737in}{2.447334in}}{\pgfqpoint{3.260542in}{2.452633in}}{\pgfqpoint{3.256635in}{2.456540in}}%
\pgfpathcurveto{\pgfqpoint{3.252728in}{2.460447in}}{\pgfqpoint{3.247429in}{2.462642in}}{\pgfqpoint{3.241904in}{2.462642in}}%
\pgfpathcurveto{\pgfqpoint{3.236379in}{2.462642in}}{\pgfqpoint{3.231079in}{2.460447in}}{\pgfqpoint{3.227172in}{2.456540in}}%
\pgfpathcurveto{\pgfqpoint{3.223265in}{2.452633in}}{\pgfqpoint{3.221070in}{2.447334in}}{\pgfqpoint{3.221070in}{2.441809in}}%
\pgfpathcurveto{\pgfqpoint{3.221070in}{2.436283in}}{\pgfqpoint{3.223265in}{2.430984in}}{\pgfqpoint{3.227172in}{2.427077in}}%
\pgfpathcurveto{\pgfqpoint{3.231079in}{2.423170in}}{\pgfqpoint{3.236379in}{2.420975in}}{\pgfqpoint{3.241904in}{2.420975in}}%
\pgfpathclose%
\pgfusepath{fill}%
\end{pgfscope}%
\begin{pgfscope}%
\pgfpathrectangle{\pgfqpoint{3.214225in}{2.423832in}}{\pgfqpoint{1.162500in}{0.755000in}}%
\pgfusepath{clip}%
\pgfsetbuttcap%
\pgfsetroundjoin%
\definecolor{currentfill}{rgb}{0.000000,0.000000,0.000000}%
\pgfsetfillcolor{currentfill}%
\pgfsetfillopacity{0.500000}%
\pgfsetlinewidth{0.000000pt}%
\definecolor{currentstroke}{rgb}{0.000000,0.000000,0.000000}%
\pgfsetstrokecolor{currentstroke}%
\pgfsetdash{}{0pt}%
\pgfpathmoveto{\pgfqpoint{4.220724in}{2.717961in}}%
\pgfpathcurveto{\pgfqpoint{4.226249in}{2.717961in}}{\pgfqpoint{4.231548in}{2.720156in}}{\pgfqpoint{4.235455in}{2.724063in}}%
\pgfpathcurveto{\pgfqpoint{4.239362in}{2.727970in}}{\pgfqpoint{4.241557in}{2.733269in}}{\pgfqpoint{4.241557in}{2.738795in}}%
\pgfpathcurveto{\pgfqpoint{4.241557in}{2.744320in}}{\pgfqpoint{4.239362in}{2.749619in}}{\pgfqpoint{4.235455in}{2.753526in}}%
\pgfpathcurveto{\pgfqpoint{4.231548in}{2.757433in}}{\pgfqpoint{4.226249in}{2.759628in}}{\pgfqpoint{4.220724in}{2.759628in}}%
\pgfpathcurveto{\pgfqpoint{4.215199in}{2.759628in}}{\pgfqpoint{4.209899in}{2.757433in}}{\pgfqpoint{4.205992in}{2.753526in}}%
\pgfpathcurveto{\pgfqpoint{4.202086in}{2.749619in}}{\pgfqpoint{4.199890in}{2.744320in}}{\pgfqpoint{4.199890in}{2.738795in}}%
\pgfpathcurveto{\pgfqpoint{4.199890in}{2.733269in}}{\pgfqpoint{4.202086in}{2.727970in}}{\pgfqpoint{4.205992in}{2.724063in}}%
\pgfpathcurveto{\pgfqpoint{4.209899in}{2.720156in}}{\pgfqpoint{4.215199in}{2.717961in}}{\pgfqpoint{4.220724in}{2.717961in}}%
\pgfpathclose%
\pgfusepath{fill}%
\end{pgfscope}%
\begin{pgfscope}%
\pgfpathrectangle{\pgfqpoint{3.214225in}{2.423832in}}{\pgfqpoint{1.162500in}{0.755000in}}%
\pgfusepath{clip}%
\pgfsetbuttcap%
\pgfsetroundjoin%
\definecolor{currentfill}{rgb}{0.000000,0.000000,0.000000}%
\pgfsetfillcolor{currentfill}%
\pgfsetfillopacity{0.500000}%
\pgfsetlinewidth{0.000000pt}%
\definecolor{currentstroke}{rgb}{0.000000,0.000000,0.000000}%
\pgfsetstrokecolor{currentstroke}%
\pgfsetdash{}{0pt}%
\pgfpathmoveto{\pgfqpoint{3.980838in}{2.640479in}}%
\pgfpathcurveto{\pgfqpoint{3.986364in}{2.640479in}}{\pgfqpoint{3.991663in}{2.642674in}}{\pgfqpoint{3.995570in}{2.646581in}}%
\pgfpathcurveto{\pgfqpoint{3.999477in}{2.650487in}}{\pgfqpoint{4.001672in}{2.655787in}}{\pgfqpoint{4.001672in}{2.661312in}}%
\pgfpathcurveto{\pgfqpoint{4.001672in}{2.666837in}}{\pgfqpoint{3.999477in}{2.672137in}}{\pgfqpoint{3.995570in}{2.676043in}}%
\pgfpathcurveto{\pgfqpoint{3.991663in}{2.679950in}}{\pgfqpoint{3.986364in}{2.682145in}}{\pgfqpoint{3.980838in}{2.682145in}}%
\pgfpathcurveto{\pgfqpoint{3.975313in}{2.682145in}}{\pgfqpoint{3.970014in}{2.679950in}}{\pgfqpoint{3.966107in}{2.676043in}}%
\pgfpathcurveto{\pgfqpoint{3.962200in}{2.672137in}}{\pgfqpoint{3.960005in}{2.666837in}}{\pgfqpoint{3.960005in}{2.661312in}}%
\pgfpathcurveto{\pgfqpoint{3.960005in}{2.655787in}}{\pgfqpoint{3.962200in}{2.650487in}}{\pgfqpoint{3.966107in}{2.646581in}}%
\pgfpathcurveto{\pgfqpoint{3.970014in}{2.642674in}}{\pgfqpoint{3.975313in}{2.640479in}}{\pgfqpoint{3.980838in}{2.640479in}}%
\pgfpathclose%
\pgfusepath{fill}%
\end{pgfscope}%
\begin{pgfscope}%
\pgfpathrectangle{\pgfqpoint{3.214225in}{2.423832in}}{\pgfqpoint{1.162500in}{0.755000in}}%
\pgfusepath{clip}%
\pgfsetbuttcap%
\pgfsetroundjoin%
\definecolor{currentfill}{rgb}{0.000000,0.000000,0.000000}%
\pgfsetfillcolor{currentfill}%
\pgfsetfillopacity{0.500000}%
\pgfsetlinewidth{0.000000pt}%
\definecolor{currentstroke}{rgb}{0.000000,0.000000,0.000000}%
\pgfsetstrokecolor{currentstroke}%
\pgfsetdash{}{0pt}%
\pgfpathmoveto{\pgfqpoint{3.708643in}{2.637393in}}%
\pgfpathcurveto{\pgfqpoint{3.714168in}{2.637393in}}{\pgfqpoint{3.719468in}{2.639588in}}{\pgfqpoint{3.723375in}{2.643495in}}%
\pgfpathcurveto{\pgfqpoint{3.727282in}{2.647402in}}{\pgfqpoint{3.729477in}{2.652701in}}{\pgfqpoint{3.729477in}{2.658226in}}%
\pgfpathcurveto{\pgfqpoint{3.729477in}{2.663751in}}{\pgfqpoint{3.727282in}{2.669051in}}{\pgfqpoint{3.723375in}{2.672958in}}%
\pgfpathcurveto{\pgfqpoint{3.719468in}{2.676865in}}{\pgfqpoint{3.714168in}{2.679060in}}{\pgfqpoint{3.708643in}{2.679060in}}%
\pgfpathcurveto{\pgfqpoint{3.703118in}{2.679060in}}{\pgfqpoint{3.697819in}{2.676865in}}{\pgfqpoint{3.693912in}{2.672958in}}%
\pgfpathcurveto{\pgfqpoint{3.690005in}{2.669051in}}{\pgfqpoint{3.687810in}{2.663751in}}{\pgfqpoint{3.687810in}{2.658226in}}%
\pgfpathcurveto{\pgfqpoint{3.687810in}{2.652701in}}{\pgfqpoint{3.690005in}{2.647402in}}{\pgfqpoint{3.693912in}{2.643495in}}%
\pgfpathcurveto{\pgfqpoint{3.697819in}{2.639588in}}{\pgfqpoint{3.703118in}{2.637393in}}{\pgfqpoint{3.708643in}{2.637393in}}%
\pgfpathclose%
\pgfusepath{fill}%
\end{pgfscope}%
\begin{pgfscope}%
\pgfpathrectangle{\pgfqpoint{3.214225in}{2.423832in}}{\pgfqpoint{1.162500in}{0.755000in}}%
\pgfusepath{clip}%
\pgfsetbuttcap%
\pgfsetroundjoin%
\definecolor{currentfill}{rgb}{0.000000,0.000000,0.000000}%
\pgfsetfillcolor{currentfill}%
\pgfsetfillopacity{0.500000}%
\pgfsetlinewidth{0.000000pt}%
\definecolor{currentstroke}{rgb}{0.000000,0.000000,0.000000}%
\pgfsetstrokecolor{currentstroke}%
\pgfsetdash{}{0pt}%
\pgfpathmoveto{\pgfqpoint{4.124371in}{2.679970in}}%
\pgfpathcurveto{\pgfqpoint{4.129896in}{2.679970in}}{\pgfqpoint{4.135195in}{2.682166in}}{\pgfqpoint{4.139102in}{2.686072in}}%
\pgfpathcurveto{\pgfqpoint{4.143009in}{2.689979in}}{\pgfqpoint{4.145204in}{2.695279in}}{\pgfqpoint{4.145204in}{2.700804in}}%
\pgfpathcurveto{\pgfqpoint{4.145204in}{2.706329in}}{\pgfqpoint{4.143009in}{2.711628in}}{\pgfqpoint{4.139102in}{2.715535in}}%
\pgfpathcurveto{\pgfqpoint{4.135195in}{2.719442in}}{\pgfqpoint{4.129896in}{2.721637in}}{\pgfqpoint{4.124371in}{2.721637in}}%
\pgfpathcurveto{\pgfqpoint{4.118846in}{2.721637in}}{\pgfqpoint{4.113546in}{2.719442in}}{\pgfqpoint{4.109639in}{2.715535in}}%
\pgfpathcurveto{\pgfqpoint{4.105732in}{2.711628in}}{\pgfqpoint{4.103537in}{2.706329in}}{\pgfqpoint{4.103537in}{2.700804in}}%
\pgfpathcurveto{\pgfqpoint{4.103537in}{2.695279in}}{\pgfqpoint{4.105732in}{2.689979in}}{\pgfqpoint{4.109639in}{2.686072in}}%
\pgfpathcurveto{\pgfqpoint{4.113546in}{2.682166in}}{\pgfqpoint{4.118846in}{2.679970in}}{\pgfqpoint{4.124371in}{2.679970in}}%
\pgfpathclose%
\pgfusepath{fill}%
\end{pgfscope}%
\begin{pgfscope}%
\pgfpathrectangle{\pgfqpoint{3.214225in}{2.423832in}}{\pgfqpoint{1.162500in}{0.755000in}}%
\pgfusepath{clip}%
\pgfsetbuttcap%
\pgfsetroundjoin%
\definecolor{currentfill}{rgb}{0.000000,0.000000,0.000000}%
\pgfsetfillcolor{currentfill}%
\pgfsetfillopacity{0.500000}%
\pgfsetlinewidth{0.000000pt}%
\definecolor{currentstroke}{rgb}{0.000000,0.000000,0.000000}%
\pgfsetstrokecolor{currentstroke}%
\pgfsetdash{}{0pt}%
\pgfpathmoveto{\pgfqpoint{3.695649in}{2.481302in}}%
\pgfpathcurveto{\pgfqpoint{3.701174in}{2.481302in}}{\pgfqpoint{3.706474in}{2.483497in}}{\pgfqpoint{3.710381in}{2.487404in}}%
\pgfpathcurveto{\pgfqpoint{3.714288in}{2.491311in}}{\pgfqpoint{3.716483in}{2.496610in}}{\pgfqpoint{3.716483in}{2.502135in}}%
\pgfpathcurveto{\pgfqpoint{3.716483in}{2.507661in}}{\pgfqpoint{3.714288in}{2.512960in}}{\pgfqpoint{3.710381in}{2.516867in}}%
\pgfpathcurveto{\pgfqpoint{3.706474in}{2.520774in}}{\pgfqpoint{3.701174in}{2.522969in}}{\pgfqpoint{3.695649in}{2.522969in}}%
\pgfpathcurveto{\pgfqpoint{3.690124in}{2.522969in}}{\pgfqpoint{3.684825in}{2.520774in}}{\pgfqpoint{3.680918in}{2.516867in}}%
\pgfpathcurveto{\pgfqpoint{3.677011in}{2.512960in}}{\pgfqpoint{3.674816in}{2.507661in}}{\pgfqpoint{3.674816in}{2.502135in}}%
\pgfpathcurveto{\pgfqpoint{3.674816in}{2.496610in}}{\pgfqpoint{3.677011in}{2.491311in}}{\pgfqpoint{3.680918in}{2.487404in}}%
\pgfpathcurveto{\pgfqpoint{3.684825in}{2.483497in}}{\pgfqpoint{3.690124in}{2.481302in}}{\pgfqpoint{3.695649in}{2.481302in}}%
\pgfpathclose%
\pgfusepath{fill}%
\end{pgfscope}%
\begin{pgfscope}%
\pgfpathrectangle{\pgfqpoint{3.214225in}{2.423832in}}{\pgfqpoint{1.162500in}{0.755000in}}%
\pgfusepath{clip}%
\pgfsetbuttcap%
\pgfsetroundjoin%
\definecolor{currentfill}{rgb}{0.000000,0.000000,0.000000}%
\pgfsetfillcolor{currentfill}%
\pgfsetfillopacity{0.500000}%
\pgfsetlinewidth{0.000000pt}%
\definecolor{currentstroke}{rgb}{0.000000,0.000000,0.000000}%
\pgfsetstrokecolor{currentstroke}%
\pgfsetdash{}{0pt}%
\pgfpathmoveto{\pgfqpoint{3.329643in}{2.503012in}}%
\pgfpathcurveto{\pgfqpoint{3.335168in}{2.503012in}}{\pgfqpoint{3.340468in}{2.505207in}}{\pgfqpoint{3.344375in}{2.509114in}}%
\pgfpathcurveto{\pgfqpoint{3.348281in}{2.513021in}}{\pgfqpoint{3.350476in}{2.518320in}}{\pgfqpoint{3.350476in}{2.523845in}}%
\pgfpathcurveto{\pgfqpoint{3.350476in}{2.529370in}}{\pgfqpoint{3.348281in}{2.534670in}}{\pgfqpoint{3.344375in}{2.538577in}}%
\pgfpathcurveto{\pgfqpoint{3.340468in}{2.542483in}}{\pgfqpoint{3.335168in}{2.544679in}}{\pgfqpoint{3.329643in}{2.544679in}}%
\pgfpathcurveto{\pgfqpoint{3.324118in}{2.544679in}}{\pgfqpoint{3.318819in}{2.542483in}}{\pgfqpoint{3.314912in}{2.538577in}}%
\pgfpathcurveto{\pgfqpoint{3.311005in}{2.534670in}}{\pgfqpoint{3.308810in}{2.529370in}}{\pgfqpoint{3.308810in}{2.523845in}}%
\pgfpathcurveto{\pgfqpoint{3.308810in}{2.518320in}}{\pgfqpoint{3.311005in}{2.513021in}}{\pgfqpoint{3.314912in}{2.509114in}}%
\pgfpathcurveto{\pgfqpoint{3.318819in}{2.505207in}}{\pgfqpoint{3.324118in}{2.503012in}}{\pgfqpoint{3.329643in}{2.503012in}}%
\pgfpathclose%
\pgfusepath{fill}%
\end{pgfscope}%
\begin{pgfscope}%
\pgfpathrectangle{\pgfqpoint{3.214225in}{2.423832in}}{\pgfqpoint{1.162500in}{0.755000in}}%
\pgfusepath{clip}%
\pgfsetbuttcap%
\pgfsetroundjoin%
\definecolor{currentfill}{rgb}{0.000000,0.000000,0.000000}%
\pgfsetfillcolor{currentfill}%
\pgfsetfillopacity{0.500000}%
\pgfsetlinewidth{0.000000pt}%
\definecolor{currentstroke}{rgb}{0.000000,0.000000,0.000000}%
\pgfsetstrokecolor{currentstroke}%
\pgfsetdash{}{0pt}%
\pgfpathmoveto{\pgfqpoint{4.167862in}{3.140023in}}%
\pgfpathcurveto{\pgfqpoint{4.173387in}{3.140023in}}{\pgfqpoint{4.178687in}{3.142218in}}{\pgfqpoint{4.182593in}{3.146125in}}%
\pgfpathcurveto{\pgfqpoint{4.186500in}{3.150032in}}{\pgfqpoint{4.188695in}{3.155331in}}{\pgfqpoint{4.188695in}{3.160856in}}%
\pgfpathcurveto{\pgfqpoint{4.188695in}{3.166381in}}{\pgfqpoint{4.186500in}{3.171681in}}{\pgfqpoint{4.182593in}{3.175588in}}%
\pgfpathcurveto{\pgfqpoint{4.178687in}{3.179494in}}{\pgfqpoint{4.173387in}{3.181690in}}{\pgfqpoint{4.167862in}{3.181690in}}%
\pgfpathcurveto{\pgfqpoint{4.162337in}{3.181690in}}{\pgfqpoint{4.157037in}{3.179494in}}{\pgfqpoint{4.153131in}{3.175588in}}%
\pgfpathcurveto{\pgfqpoint{4.149224in}{3.171681in}}{\pgfqpoint{4.147029in}{3.166381in}}{\pgfqpoint{4.147029in}{3.160856in}}%
\pgfpathcurveto{\pgfqpoint{4.147029in}{3.155331in}}{\pgfqpoint{4.149224in}{3.150032in}}{\pgfqpoint{4.153131in}{3.146125in}}%
\pgfpathcurveto{\pgfqpoint{4.157037in}{3.142218in}}{\pgfqpoint{4.162337in}{3.140023in}}{\pgfqpoint{4.167862in}{3.140023in}}%
\pgfpathclose%
\pgfusepath{fill}%
\end{pgfscope}%
\begin{pgfscope}%
\pgfpathrectangle{\pgfqpoint{3.214225in}{2.423832in}}{\pgfqpoint{1.162500in}{0.755000in}}%
\pgfusepath{clip}%
\pgfsetbuttcap%
\pgfsetroundjoin%
\definecolor{currentfill}{rgb}{0.000000,0.000000,0.000000}%
\pgfsetfillcolor{currentfill}%
\pgfsetfillopacity{0.500000}%
\pgfsetlinewidth{0.000000pt}%
\definecolor{currentstroke}{rgb}{0.000000,0.000000,0.000000}%
\pgfsetstrokecolor{currentstroke}%
\pgfsetdash{}{0pt}%
\pgfpathmoveto{\pgfqpoint{4.120816in}{2.625365in}}%
\pgfpathcurveto{\pgfqpoint{4.126341in}{2.625365in}}{\pgfqpoint{4.131641in}{2.627561in}}{\pgfqpoint{4.135548in}{2.631467in}}%
\pgfpathcurveto{\pgfqpoint{4.139455in}{2.635374in}}{\pgfqpoint{4.141650in}{2.640674in}}{\pgfqpoint{4.141650in}{2.646199in}}%
\pgfpathcurveto{\pgfqpoint{4.141650in}{2.651724in}}{\pgfqpoint{4.139455in}{2.657023in}}{\pgfqpoint{4.135548in}{2.660930in}}%
\pgfpathcurveto{\pgfqpoint{4.131641in}{2.664837in}}{\pgfqpoint{4.126341in}{2.667032in}}{\pgfqpoint{4.120816in}{2.667032in}}%
\pgfpathcurveto{\pgfqpoint{4.115291in}{2.667032in}}{\pgfqpoint{4.109992in}{2.664837in}}{\pgfqpoint{4.106085in}{2.660930in}}%
\pgfpathcurveto{\pgfqpoint{4.102178in}{2.657023in}}{\pgfqpoint{4.099983in}{2.651724in}}{\pgfqpoint{4.099983in}{2.646199in}}%
\pgfpathcurveto{\pgfqpoint{4.099983in}{2.640674in}}{\pgfqpoint{4.102178in}{2.635374in}}{\pgfqpoint{4.106085in}{2.631467in}}%
\pgfpathcurveto{\pgfqpoint{4.109992in}{2.627561in}}{\pgfqpoint{4.115291in}{2.625365in}}{\pgfqpoint{4.120816in}{2.625365in}}%
\pgfpathclose%
\pgfusepath{fill}%
\end{pgfscope}%
\begin{pgfscope}%
\pgfpathrectangle{\pgfqpoint{3.214225in}{2.423832in}}{\pgfqpoint{1.162500in}{0.755000in}}%
\pgfusepath{clip}%
\pgfsetbuttcap%
\pgfsetroundjoin%
\definecolor{currentfill}{rgb}{0.000000,0.000000,0.000000}%
\pgfsetfillcolor{currentfill}%
\pgfsetfillopacity{0.500000}%
\pgfsetlinewidth{0.000000pt}%
\definecolor{currentstroke}{rgb}{0.000000,0.000000,0.000000}%
\pgfsetstrokecolor{currentstroke}%
\pgfsetdash{}{0pt}%
\pgfpathmoveto{\pgfqpoint{3.792736in}{2.450809in}}%
\pgfpathcurveto{\pgfqpoint{3.798261in}{2.450809in}}{\pgfqpoint{3.803560in}{2.453004in}}{\pgfqpoint{3.807467in}{2.456911in}}%
\pgfpathcurveto{\pgfqpoint{3.811374in}{2.460818in}}{\pgfqpoint{3.813569in}{2.466117in}}{\pgfqpoint{3.813569in}{2.471642in}}%
\pgfpathcurveto{\pgfqpoint{3.813569in}{2.477167in}}{\pgfqpoint{3.811374in}{2.482467in}}{\pgfqpoint{3.807467in}{2.486374in}}%
\pgfpathcurveto{\pgfqpoint{3.803560in}{2.490280in}}{\pgfqpoint{3.798261in}{2.492476in}}{\pgfqpoint{3.792736in}{2.492476in}}%
\pgfpathcurveto{\pgfqpoint{3.787211in}{2.492476in}}{\pgfqpoint{3.781911in}{2.490280in}}{\pgfqpoint{3.778004in}{2.486374in}}%
\pgfpathcurveto{\pgfqpoint{3.774097in}{2.482467in}}{\pgfqpoint{3.771902in}{2.477167in}}{\pgfqpoint{3.771902in}{2.471642in}}%
\pgfpathcurveto{\pgfqpoint{3.771902in}{2.466117in}}{\pgfqpoint{3.774097in}{2.460818in}}{\pgfqpoint{3.778004in}{2.456911in}}%
\pgfpathcurveto{\pgfqpoint{3.781911in}{2.453004in}}{\pgfqpoint{3.787211in}{2.450809in}}{\pgfqpoint{3.792736in}{2.450809in}}%
\pgfpathclose%
\pgfusepath{fill}%
\end{pgfscope}%
\begin{pgfscope}%
\pgfsetrectcap%
\pgfsetmiterjoin%
\pgfsetlinewidth{0.803000pt}%
\definecolor{currentstroke}{rgb}{0.501961,0.501961,0.501961}%
\pgfsetstrokecolor{currentstroke}%
\pgfsetdash{}{0pt}%
\pgfpathmoveto{\pgfqpoint{3.214225in}{2.423832in}}%
\pgfpathlineto{\pgfqpoint{3.214225in}{3.178832in}}%
\pgfusepath{stroke}%
\end{pgfscope}%
\begin{pgfscope}%
\pgfsetrectcap%
\pgfsetmiterjoin%
\pgfsetlinewidth{0.803000pt}%
\definecolor{currentstroke}{rgb}{0.501961,0.501961,0.501961}%
\pgfsetstrokecolor{currentstroke}%
\pgfsetdash{}{0pt}%
\pgfpathmoveto{\pgfqpoint{4.376725in}{2.423832in}}%
\pgfpathlineto{\pgfqpoint{4.376725in}{3.178832in}}%
\pgfusepath{stroke}%
\end{pgfscope}%
\begin{pgfscope}%
\pgfsetrectcap%
\pgfsetmiterjoin%
\pgfsetlinewidth{0.803000pt}%
\definecolor{currentstroke}{rgb}{0.501961,0.501961,0.501961}%
\pgfsetstrokecolor{currentstroke}%
\pgfsetdash{}{0pt}%
\pgfpathmoveto{\pgfqpoint{3.214225in}{2.423832in}}%
\pgfpathlineto{\pgfqpoint{4.376725in}{2.423832in}}%
\pgfusepath{stroke}%
\end{pgfscope}%
\begin{pgfscope}%
\pgfsetrectcap%
\pgfsetmiterjoin%
\pgfsetlinewidth{0.803000pt}%
\definecolor{currentstroke}{rgb}{0.501961,0.501961,0.501961}%
\pgfsetstrokecolor{currentstroke}%
\pgfsetdash{}{0pt}%
\pgfpathmoveto{\pgfqpoint{3.214225in}{3.178832in}}%
\pgfpathlineto{\pgfqpoint{4.376725in}{3.178832in}}%
\pgfusepath{stroke}%
\end{pgfscope}%
\begin{pgfscope}%
\pgfsetbuttcap%
\pgfsetmiterjoin%
\definecolor{currentfill}{rgb}{1.000000,1.000000,1.000000}%
\pgfsetfillcolor{currentfill}%
\pgfsetlinewidth{0.000000pt}%
\definecolor{currentstroke}{rgb}{0.000000,0.000000,0.000000}%
\pgfsetstrokecolor{currentstroke}%
\pgfsetstrokeopacity{0.000000}%
\pgfsetdash{}{0pt}%
\pgfpathmoveto{\pgfqpoint{4.376725in}{2.423832in}}%
\pgfpathlineto{\pgfqpoint{5.539225in}{2.423832in}}%
\pgfpathlineto{\pgfqpoint{5.539225in}{3.178832in}}%
\pgfpathlineto{\pgfqpoint{4.376725in}{3.178832in}}%
\pgfpathclose%
\pgfusepath{fill}%
\end{pgfscope}%
\begin{pgfscope}%
\pgfpathrectangle{\pgfqpoint{4.376725in}{2.423832in}}{\pgfqpoint{1.162500in}{0.755000in}}%
\pgfusepath{clip}%
\pgfsetbuttcap%
\pgfsetroundjoin%
\definecolor{currentfill}{rgb}{0.000000,0.000000,0.000000}%
\pgfsetfillcolor{currentfill}%
\pgfsetfillopacity{0.500000}%
\pgfsetlinewidth{0.000000pt}%
\definecolor{currentstroke}{rgb}{0.000000,0.000000,0.000000}%
\pgfsetstrokecolor{currentstroke}%
\pgfsetdash{}{0pt}%
\pgfpathmoveto{\pgfqpoint{5.214960in}{3.088730in}}%
\pgfpathcurveto{\pgfqpoint{5.220485in}{3.088730in}}{\pgfqpoint{5.225785in}{3.090925in}}{\pgfqpoint{5.229692in}{3.094832in}}%
\pgfpathcurveto{\pgfqpoint{5.233598in}{3.098739in}}{\pgfqpoint{5.235794in}{3.104038in}}{\pgfqpoint{5.235794in}{3.109563in}}%
\pgfpathcurveto{\pgfqpoint{5.235794in}{3.115089in}}{\pgfqpoint{5.233598in}{3.120388in}}{\pgfqpoint{5.229692in}{3.124295in}}%
\pgfpathcurveto{\pgfqpoint{5.225785in}{3.128202in}}{\pgfqpoint{5.220485in}{3.130397in}}{\pgfqpoint{5.214960in}{3.130397in}}%
\pgfpathcurveto{\pgfqpoint{5.209435in}{3.130397in}}{\pgfqpoint{5.204136in}{3.128202in}}{\pgfqpoint{5.200229in}{3.124295in}}%
\pgfpathcurveto{\pgfqpoint{5.196322in}{3.120388in}}{\pgfqpoint{5.194127in}{3.115089in}}{\pgfqpoint{5.194127in}{3.109563in}}%
\pgfpathcurveto{\pgfqpoint{5.194127in}{3.104038in}}{\pgfqpoint{5.196322in}{3.098739in}}{\pgfqpoint{5.200229in}{3.094832in}}%
\pgfpathcurveto{\pgfqpoint{5.204136in}{3.090925in}}{\pgfqpoint{5.209435in}{3.088730in}}{\pgfqpoint{5.214960in}{3.088730in}}%
\pgfpathclose%
\pgfusepath{fill}%
\end{pgfscope}%
\begin{pgfscope}%
\pgfpathrectangle{\pgfqpoint{4.376725in}{2.423832in}}{\pgfqpoint{1.162500in}{0.755000in}}%
\pgfusepath{clip}%
\pgfsetbuttcap%
\pgfsetroundjoin%
\definecolor{currentfill}{rgb}{0.000000,0.000000,0.000000}%
\pgfsetfillcolor{currentfill}%
\pgfsetfillopacity{0.500000}%
\pgfsetlinewidth{0.000000pt}%
\definecolor{currentstroke}{rgb}{0.000000,0.000000,0.000000}%
\pgfsetstrokecolor{currentstroke}%
\pgfsetdash{}{0pt}%
\pgfpathmoveto{\pgfqpoint{4.521739in}{2.544873in}}%
\pgfpathcurveto{\pgfqpoint{4.527264in}{2.544873in}}{\pgfqpoint{4.532563in}{2.547068in}}{\pgfqpoint{4.536470in}{2.550975in}}%
\pgfpathcurveto{\pgfqpoint{4.540377in}{2.554881in}}{\pgfqpoint{4.542572in}{2.560181in}}{\pgfqpoint{4.542572in}{2.565706in}}%
\pgfpathcurveto{\pgfqpoint{4.542572in}{2.571231in}}{\pgfqpoint{4.540377in}{2.576531in}}{\pgfqpoint{4.536470in}{2.580437in}}%
\pgfpathcurveto{\pgfqpoint{4.532563in}{2.584344in}}{\pgfqpoint{4.527264in}{2.586539in}}{\pgfqpoint{4.521739in}{2.586539in}}%
\pgfpathcurveto{\pgfqpoint{4.516214in}{2.586539in}}{\pgfqpoint{4.510914in}{2.584344in}}{\pgfqpoint{4.507007in}{2.580437in}}%
\pgfpathcurveto{\pgfqpoint{4.503101in}{2.576531in}}{\pgfqpoint{4.500905in}{2.571231in}}{\pgfqpoint{4.500905in}{2.565706in}}%
\pgfpathcurveto{\pgfqpoint{4.500905in}{2.560181in}}{\pgfqpoint{4.503101in}{2.554881in}}{\pgfqpoint{4.507007in}{2.550975in}}%
\pgfpathcurveto{\pgfqpoint{4.510914in}{2.547068in}}{\pgfqpoint{4.516214in}{2.544873in}}{\pgfqpoint{4.521739in}{2.544873in}}%
\pgfpathclose%
\pgfusepath{fill}%
\end{pgfscope}%
\begin{pgfscope}%
\pgfpathrectangle{\pgfqpoint{4.376725in}{2.423832in}}{\pgfqpoint{1.162500in}{0.755000in}}%
\pgfusepath{clip}%
\pgfsetbuttcap%
\pgfsetroundjoin%
\definecolor{currentfill}{rgb}{0.000000,0.000000,0.000000}%
\pgfsetfillcolor{currentfill}%
\pgfsetfillopacity{0.500000}%
\pgfsetlinewidth{0.000000pt}%
\definecolor{currentstroke}{rgb}{0.000000,0.000000,0.000000}%
\pgfsetstrokecolor{currentstroke}%
\pgfsetdash{}{0pt}%
\pgfpathmoveto{\pgfqpoint{4.448919in}{2.541427in}}%
\pgfpathcurveto{\pgfqpoint{4.454444in}{2.541427in}}{\pgfqpoint{4.459744in}{2.543622in}}{\pgfqpoint{4.463650in}{2.547529in}}%
\pgfpathcurveto{\pgfqpoint{4.467557in}{2.551436in}}{\pgfqpoint{4.469752in}{2.556735in}}{\pgfqpoint{4.469752in}{2.562260in}}%
\pgfpathcurveto{\pgfqpoint{4.469752in}{2.567785in}}{\pgfqpoint{4.467557in}{2.573085in}}{\pgfqpoint{4.463650in}{2.576992in}}%
\pgfpathcurveto{\pgfqpoint{4.459744in}{2.580899in}}{\pgfqpoint{4.454444in}{2.583094in}}{\pgfqpoint{4.448919in}{2.583094in}}%
\pgfpathcurveto{\pgfqpoint{4.443394in}{2.583094in}}{\pgfqpoint{4.438094in}{2.580899in}}{\pgfqpoint{4.434188in}{2.576992in}}%
\pgfpathcurveto{\pgfqpoint{4.430281in}{2.573085in}}{\pgfqpoint{4.428086in}{2.567785in}}{\pgfqpoint{4.428086in}{2.562260in}}%
\pgfpathcurveto{\pgfqpoint{4.428086in}{2.556735in}}{\pgfqpoint{4.430281in}{2.551436in}}{\pgfqpoint{4.434188in}{2.547529in}}%
\pgfpathcurveto{\pgfqpoint{4.438094in}{2.543622in}}{\pgfqpoint{4.443394in}{2.541427in}}{\pgfqpoint{4.448919in}{2.541427in}}%
\pgfpathclose%
\pgfusepath{fill}%
\end{pgfscope}%
\begin{pgfscope}%
\pgfpathrectangle{\pgfqpoint{4.376725in}{2.423832in}}{\pgfqpoint{1.162500in}{0.755000in}}%
\pgfusepath{clip}%
\pgfsetbuttcap%
\pgfsetroundjoin%
\definecolor{currentfill}{rgb}{0.000000,0.000000,0.000000}%
\pgfsetfillcolor{currentfill}%
\pgfsetfillopacity{0.500000}%
\pgfsetlinewidth{0.000000pt}%
\definecolor{currentstroke}{rgb}{0.000000,0.000000,0.000000}%
\pgfsetstrokecolor{currentstroke}%
\pgfsetdash{}{0pt}%
\pgfpathmoveto{\pgfqpoint{4.704251in}{2.469548in}}%
\pgfpathcurveto{\pgfqpoint{4.709776in}{2.469548in}}{\pgfqpoint{4.715075in}{2.471743in}}{\pgfqpoint{4.718982in}{2.475650in}}%
\pgfpathcurveto{\pgfqpoint{4.722889in}{2.479557in}}{\pgfqpoint{4.725084in}{2.484856in}}{\pgfqpoint{4.725084in}{2.490381in}}%
\pgfpathcurveto{\pgfqpoint{4.725084in}{2.495907in}}{\pgfqpoint{4.722889in}{2.501206in}}{\pgfqpoint{4.718982in}{2.505113in}}%
\pgfpathcurveto{\pgfqpoint{4.715075in}{2.509020in}}{\pgfqpoint{4.709776in}{2.511215in}}{\pgfqpoint{4.704251in}{2.511215in}}%
\pgfpathcurveto{\pgfqpoint{4.698726in}{2.511215in}}{\pgfqpoint{4.693426in}{2.509020in}}{\pgfqpoint{4.689519in}{2.505113in}}%
\pgfpathcurveto{\pgfqpoint{4.685612in}{2.501206in}}{\pgfqpoint{4.683417in}{2.495907in}}{\pgfqpoint{4.683417in}{2.490381in}}%
\pgfpathcurveto{\pgfqpoint{4.683417in}{2.484856in}}{\pgfqpoint{4.685612in}{2.479557in}}{\pgfqpoint{4.689519in}{2.475650in}}%
\pgfpathcurveto{\pgfqpoint{4.693426in}{2.471743in}}{\pgfqpoint{4.698726in}{2.469548in}}{\pgfqpoint{4.704251in}{2.469548in}}%
\pgfpathclose%
\pgfusepath{fill}%
\end{pgfscope}%
\begin{pgfscope}%
\pgfpathrectangle{\pgfqpoint{4.376725in}{2.423832in}}{\pgfqpoint{1.162500in}{0.755000in}}%
\pgfusepath{clip}%
\pgfsetbuttcap%
\pgfsetroundjoin%
\definecolor{currentfill}{rgb}{0.000000,0.000000,0.000000}%
\pgfsetfillcolor{currentfill}%
\pgfsetfillopacity{0.500000}%
\pgfsetlinewidth{0.000000pt}%
\definecolor{currentstroke}{rgb}{0.000000,0.000000,0.000000}%
\pgfsetstrokecolor{currentstroke}%
\pgfsetdash{}{0pt}%
\pgfpathmoveto{\pgfqpoint{4.404404in}{2.488481in}}%
\pgfpathcurveto{\pgfqpoint{4.409929in}{2.488481in}}{\pgfqpoint{4.415228in}{2.490676in}}{\pgfqpoint{4.419135in}{2.494583in}}%
\pgfpathcurveto{\pgfqpoint{4.423042in}{2.498489in}}{\pgfqpoint{4.425237in}{2.503789in}}{\pgfqpoint{4.425237in}{2.509314in}}%
\pgfpathcurveto{\pgfqpoint{4.425237in}{2.514839in}}{\pgfqpoint{4.423042in}{2.520139in}}{\pgfqpoint{4.419135in}{2.524045in}}%
\pgfpathcurveto{\pgfqpoint{4.415228in}{2.527952in}}{\pgfqpoint{4.409929in}{2.530147in}}{\pgfqpoint{4.404404in}{2.530147in}}%
\pgfpathcurveto{\pgfqpoint{4.398879in}{2.530147in}}{\pgfqpoint{4.393579in}{2.527952in}}{\pgfqpoint{4.389672in}{2.524045in}}%
\pgfpathcurveto{\pgfqpoint{4.385765in}{2.520139in}}{\pgfqpoint{4.383570in}{2.514839in}}{\pgfqpoint{4.383570in}{2.509314in}}%
\pgfpathcurveto{\pgfqpoint{4.383570in}{2.503789in}}{\pgfqpoint{4.385765in}{2.498489in}}{\pgfqpoint{4.389672in}{2.494583in}}%
\pgfpathcurveto{\pgfqpoint{4.393579in}{2.490676in}}{\pgfqpoint{4.398879in}{2.488481in}}{\pgfqpoint{4.404404in}{2.488481in}}%
\pgfpathclose%
\pgfusepath{fill}%
\end{pgfscope}%
\begin{pgfscope}%
\pgfpathrectangle{\pgfqpoint{4.376725in}{2.423832in}}{\pgfqpoint{1.162500in}{0.755000in}}%
\pgfusepath{clip}%
\pgfsetbuttcap%
\pgfsetroundjoin%
\definecolor{currentfill}{rgb}{0.000000,0.000000,0.000000}%
\pgfsetfillcolor{currentfill}%
\pgfsetfillopacity{0.500000}%
\pgfsetlinewidth{0.000000pt}%
\definecolor{currentstroke}{rgb}{0.000000,0.000000,0.000000}%
\pgfsetstrokecolor{currentstroke}%
\pgfsetdash{}{0pt}%
\pgfpathmoveto{\pgfqpoint{4.607580in}{2.422812in}}%
\pgfpathcurveto{\pgfqpoint{4.613105in}{2.422812in}}{\pgfqpoint{4.618405in}{2.425007in}}{\pgfqpoint{4.622312in}{2.428914in}}%
\pgfpathcurveto{\pgfqpoint{4.626218in}{2.432821in}}{\pgfqpoint{4.628414in}{2.438120in}}{\pgfqpoint{4.628414in}{2.443645in}}%
\pgfpathcurveto{\pgfqpoint{4.628414in}{2.449170in}}{\pgfqpoint{4.626218in}{2.454470in}}{\pgfqpoint{4.622312in}{2.458377in}}%
\pgfpathcurveto{\pgfqpoint{4.618405in}{2.462284in}}{\pgfqpoint{4.613105in}{2.464479in}}{\pgfqpoint{4.607580in}{2.464479in}}%
\pgfpathcurveto{\pgfqpoint{4.602055in}{2.464479in}}{\pgfqpoint{4.596756in}{2.462284in}}{\pgfqpoint{4.592849in}{2.458377in}}%
\pgfpathcurveto{\pgfqpoint{4.588942in}{2.454470in}}{\pgfqpoint{4.586747in}{2.449170in}}{\pgfqpoint{4.586747in}{2.443645in}}%
\pgfpathcurveto{\pgfqpoint{4.586747in}{2.438120in}}{\pgfqpoint{4.588942in}{2.432821in}}{\pgfqpoint{4.592849in}{2.428914in}}%
\pgfpathcurveto{\pgfqpoint{4.596756in}{2.425007in}}{\pgfqpoint{4.602055in}{2.422812in}}{\pgfqpoint{4.607580in}{2.422812in}}%
\pgfpathclose%
\pgfusepath{fill}%
\end{pgfscope}%
\begin{pgfscope}%
\pgfpathrectangle{\pgfqpoint{4.376725in}{2.423832in}}{\pgfqpoint{1.162500in}{0.755000in}}%
\pgfusepath{clip}%
\pgfsetbuttcap%
\pgfsetroundjoin%
\definecolor{currentfill}{rgb}{0.000000,0.000000,0.000000}%
\pgfsetfillcolor{currentfill}%
\pgfsetfillopacity{0.500000}%
\pgfsetlinewidth{0.000000pt}%
\definecolor{currentstroke}{rgb}{0.000000,0.000000,0.000000}%
\pgfsetstrokecolor{currentstroke}%
\pgfsetdash{}{0pt}%
\pgfpathmoveto{\pgfqpoint{4.596304in}{2.420975in}}%
\pgfpathcurveto{\pgfqpoint{4.601829in}{2.420975in}}{\pgfqpoint{4.607129in}{2.423170in}}{\pgfqpoint{4.611036in}{2.427077in}}%
\pgfpathcurveto{\pgfqpoint{4.614942in}{2.430984in}}{\pgfqpoint{4.617137in}{2.436283in}}{\pgfqpoint{4.617137in}{2.441809in}}%
\pgfpathcurveto{\pgfqpoint{4.617137in}{2.447334in}}{\pgfqpoint{4.614942in}{2.452633in}}{\pgfqpoint{4.611036in}{2.456540in}}%
\pgfpathcurveto{\pgfqpoint{4.607129in}{2.460447in}}{\pgfqpoint{4.601829in}{2.462642in}}{\pgfqpoint{4.596304in}{2.462642in}}%
\pgfpathcurveto{\pgfqpoint{4.590779in}{2.462642in}}{\pgfqpoint{4.585480in}{2.460447in}}{\pgfqpoint{4.581573in}{2.456540in}}%
\pgfpathcurveto{\pgfqpoint{4.577666in}{2.452633in}}{\pgfqpoint{4.575471in}{2.447334in}}{\pgfqpoint{4.575471in}{2.441809in}}%
\pgfpathcurveto{\pgfqpoint{4.575471in}{2.436283in}}{\pgfqpoint{4.577666in}{2.430984in}}{\pgfqpoint{4.581573in}{2.427077in}}%
\pgfpathcurveto{\pgfqpoint{4.585480in}{2.423170in}}{\pgfqpoint{4.590779in}{2.420975in}}{\pgfqpoint{4.596304in}{2.420975in}}%
\pgfpathclose%
\pgfusepath{fill}%
\end{pgfscope}%
\begin{pgfscope}%
\pgfpathrectangle{\pgfqpoint{4.376725in}{2.423832in}}{\pgfqpoint{1.162500in}{0.755000in}}%
\pgfusepath{clip}%
\pgfsetbuttcap%
\pgfsetroundjoin%
\definecolor{currentfill}{rgb}{0.000000,0.000000,0.000000}%
\pgfsetfillcolor{currentfill}%
\pgfsetfillopacity{0.500000}%
\pgfsetlinewidth{0.000000pt}%
\definecolor{currentstroke}{rgb}{0.000000,0.000000,0.000000}%
\pgfsetstrokecolor{currentstroke}%
\pgfsetdash{}{0pt}%
\pgfpathmoveto{\pgfqpoint{5.423123in}{2.717961in}}%
\pgfpathcurveto{\pgfqpoint{5.428649in}{2.717961in}}{\pgfqpoint{5.433948in}{2.720156in}}{\pgfqpoint{5.437855in}{2.724063in}}%
\pgfpathcurveto{\pgfqpoint{5.441762in}{2.727970in}}{\pgfqpoint{5.443957in}{2.733269in}}{\pgfqpoint{5.443957in}{2.738795in}}%
\pgfpathcurveto{\pgfqpoint{5.443957in}{2.744320in}}{\pgfqpoint{5.441762in}{2.749619in}}{\pgfqpoint{5.437855in}{2.753526in}}%
\pgfpathcurveto{\pgfqpoint{5.433948in}{2.757433in}}{\pgfqpoint{5.428649in}{2.759628in}}{\pgfqpoint{5.423123in}{2.759628in}}%
\pgfpathcurveto{\pgfqpoint{5.417598in}{2.759628in}}{\pgfqpoint{5.412299in}{2.757433in}}{\pgfqpoint{5.408392in}{2.753526in}}%
\pgfpathcurveto{\pgfqpoint{5.404485in}{2.749619in}}{\pgfqpoint{5.402290in}{2.744320in}}{\pgfqpoint{5.402290in}{2.738795in}}%
\pgfpathcurveto{\pgfqpoint{5.402290in}{2.733269in}}{\pgfqpoint{5.404485in}{2.727970in}}{\pgfqpoint{5.408392in}{2.724063in}}%
\pgfpathcurveto{\pgfqpoint{5.412299in}{2.720156in}}{\pgfqpoint{5.417598in}{2.717961in}}{\pgfqpoint{5.423123in}{2.717961in}}%
\pgfpathclose%
\pgfusepath{fill}%
\end{pgfscope}%
\begin{pgfscope}%
\pgfpathrectangle{\pgfqpoint{4.376725in}{2.423832in}}{\pgfqpoint{1.162500in}{0.755000in}}%
\pgfusepath{clip}%
\pgfsetbuttcap%
\pgfsetroundjoin%
\definecolor{currentfill}{rgb}{0.000000,0.000000,0.000000}%
\pgfsetfillcolor{currentfill}%
\pgfsetfillopacity{0.500000}%
\pgfsetlinewidth{0.000000pt}%
\definecolor{currentstroke}{rgb}{0.000000,0.000000,0.000000}%
\pgfsetstrokecolor{currentstroke}%
\pgfsetdash{}{0pt}%
\pgfpathmoveto{\pgfqpoint{5.511546in}{2.640479in}}%
\pgfpathcurveto{\pgfqpoint{5.517071in}{2.640479in}}{\pgfqpoint{5.522371in}{2.642674in}}{\pgfqpoint{5.526278in}{2.646581in}}%
\pgfpathcurveto{\pgfqpoint{5.530185in}{2.650487in}}{\pgfqpoint{5.532380in}{2.655787in}}{\pgfqpoint{5.532380in}{2.661312in}}%
\pgfpathcurveto{\pgfqpoint{5.532380in}{2.666837in}}{\pgfqpoint{5.530185in}{2.672137in}}{\pgfqpoint{5.526278in}{2.676043in}}%
\pgfpathcurveto{\pgfqpoint{5.522371in}{2.679950in}}{\pgfqpoint{5.517071in}{2.682145in}}{\pgfqpoint{5.511546in}{2.682145in}}%
\pgfpathcurveto{\pgfqpoint{5.506021in}{2.682145in}}{\pgfqpoint{5.500722in}{2.679950in}}{\pgfqpoint{5.496815in}{2.676043in}}%
\pgfpathcurveto{\pgfqpoint{5.492908in}{2.672137in}}{\pgfqpoint{5.490713in}{2.666837in}}{\pgfqpoint{5.490713in}{2.661312in}}%
\pgfpathcurveto{\pgfqpoint{5.490713in}{2.655787in}}{\pgfqpoint{5.492908in}{2.650487in}}{\pgfqpoint{5.496815in}{2.646581in}}%
\pgfpathcurveto{\pgfqpoint{5.500722in}{2.642674in}}{\pgfqpoint{5.506021in}{2.640479in}}{\pgfqpoint{5.511546in}{2.640479in}}%
\pgfpathclose%
\pgfusepath{fill}%
\end{pgfscope}%
\begin{pgfscope}%
\pgfpathrectangle{\pgfqpoint{4.376725in}{2.423832in}}{\pgfqpoint{1.162500in}{0.755000in}}%
\pgfusepath{clip}%
\pgfsetbuttcap%
\pgfsetroundjoin%
\definecolor{currentfill}{rgb}{0.000000,0.000000,0.000000}%
\pgfsetfillcolor{currentfill}%
\pgfsetfillopacity{0.500000}%
\pgfsetlinewidth{0.000000pt}%
\definecolor{currentstroke}{rgb}{0.000000,0.000000,0.000000}%
\pgfsetstrokecolor{currentstroke}%
\pgfsetdash{}{0pt}%
\pgfpathmoveto{\pgfqpoint{5.286995in}{2.637393in}}%
\pgfpathcurveto{\pgfqpoint{5.292520in}{2.637393in}}{\pgfqpoint{5.297820in}{2.639588in}}{\pgfqpoint{5.301726in}{2.643495in}}%
\pgfpathcurveto{\pgfqpoint{5.305633in}{2.647402in}}{\pgfqpoint{5.307828in}{2.652701in}}{\pgfqpoint{5.307828in}{2.658226in}}%
\pgfpathcurveto{\pgfqpoint{5.307828in}{2.663751in}}{\pgfqpoint{5.305633in}{2.669051in}}{\pgfqpoint{5.301726in}{2.672958in}}%
\pgfpathcurveto{\pgfqpoint{5.297820in}{2.676865in}}{\pgfqpoint{5.292520in}{2.679060in}}{\pgfqpoint{5.286995in}{2.679060in}}%
\pgfpathcurveto{\pgfqpoint{5.281470in}{2.679060in}}{\pgfqpoint{5.276170in}{2.676865in}}{\pgfqpoint{5.272264in}{2.672958in}}%
\pgfpathcurveto{\pgfqpoint{5.268357in}{2.669051in}}{\pgfqpoint{5.266162in}{2.663751in}}{\pgfqpoint{5.266162in}{2.658226in}}%
\pgfpathcurveto{\pgfqpoint{5.266162in}{2.652701in}}{\pgfqpoint{5.268357in}{2.647402in}}{\pgfqpoint{5.272264in}{2.643495in}}%
\pgfpathcurveto{\pgfqpoint{5.276170in}{2.639588in}}{\pgfqpoint{5.281470in}{2.637393in}}{\pgfqpoint{5.286995in}{2.637393in}}%
\pgfpathclose%
\pgfusepath{fill}%
\end{pgfscope}%
\begin{pgfscope}%
\pgfpathrectangle{\pgfqpoint{4.376725in}{2.423832in}}{\pgfqpoint{1.162500in}{0.755000in}}%
\pgfusepath{clip}%
\pgfsetbuttcap%
\pgfsetroundjoin%
\definecolor{currentfill}{rgb}{0.000000,0.000000,0.000000}%
\pgfsetfillcolor{currentfill}%
\pgfsetfillopacity{0.500000}%
\pgfsetlinewidth{0.000000pt}%
\definecolor{currentstroke}{rgb}{0.000000,0.000000,0.000000}%
\pgfsetstrokecolor{currentstroke}%
\pgfsetdash{}{0pt}%
\pgfpathmoveto{\pgfqpoint{4.645822in}{2.679970in}}%
\pgfpathcurveto{\pgfqpoint{4.651348in}{2.679970in}}{\pgfqpoint{4.656647in}{2.682166in}}{\pgfqpoint{4.660554in}{2.686072in}}%
\pgfpathcurveto{\pgfqpoint{4.664461in}{2.689979in}}{\pgfqpoint{4.666656in}{2.695279in}}{\pgfqpoint{4.666656in}{2.700804in}}%
\pgfpathcurveto{\pgfqpoint{4.666656in}{2.706329in}}{\pgfqpoint{4.664461in}{2.711628in}}{\pgfqpoint{4.660554in}{2.715535in}}%
\pgfpathcurveto{\pgfqpoint{4.656647in}{2.719442in}}{\pgfqpoint{4.651348in}{2.721637in}}{\pgfqpoint{4.645822in}{2.721637in}}%
\pgfpathcurveto{\pgfqpoint{4.640297in}{2.721637in}}{\pgfqpoint{4.634998in}{2.719442in}}{\pgfqpoint{4.631091in}{2.715535in}}%
\pgfpathcurveto{\pgfqpoint{4.627184in}{2.711628in}}{\pgfqpoint{4.624989in}{2.706329in}}{\pgfqpoint{4.624989in}{2.700804in}}%
\pgfpathcurveto{\pgfqpoint{4.624989in}{2.695279in}}{\pgfqpoint{4.627184in}{2.689979in}}{\pgfqpoint{4.631091in}{2.686072in}}%
\pgfpathcurveto{\pgfqpoint{4.634998in}{2.682166in}}{\pgfqpoint{4.640297in}{2.679970in}}{\pgfqpoint{4.645822in}{2.679970in}}%
\pgfpathclose%
\pgfusepath{fill}%
\end{pgfscope}%
\begin{pgfscope}%
\pgfpathrectangle{\pgfqpoint{4.376725in}{2.423832in}}{\pgfqpoint{1.162500in}{0.755000in}}%
\pgfusepath{clip}%
\pgfsetbuttcap%
\pgfsetroundjoin%
\definecolor{currentfill}{rgb}{0.000000,0.000000,0.000000}%
\pgfsetfillcolor{currentfill}%
\pgfsetfillopacity{0.500000}%
\pgfsetlinewidth{0.000000pt}%
\definecolor{currentstroke}{rgb}{0.000000,0.000000,0.000000}%
\pgfsetstrokecolor{currentstroke}%
\pgfsetdash{}{0pt}%
\pgfpathmoveto{\pgfqpoint{5.069839in}{2.481302in}}%
\pgfpathcurveto{\pgfqpoint{5.075364in}{2.481302in}}{\pgfqpoint{5.080663in}{2.483497in}}{\pgfqpoint{5.084570in}{2.487404in}}%
\pgfpathcurveto{\pgfqpoint{5.088477in}{2.491311in}}{\pgfqpoint{5.090672in}{2.496610in}}{\pgfqpoint{5.090672in}{2.502135in}}%
\pgfpathcurveto{\pgfqpoint{5.090672in}{2.507661in}}{\pgfqpoint{5.088477in}{2.512960in}}{\pgfqpoint{5.084570in}{2.516867in}}%
\pgfpathcurveto{\pgfqpoint{5.080663in}{2.520774in}}{\pgfqpoint{5.075364in}{2.522969in}}{\pgfqpoint{5.069839in}{2.522969in}}%
\pgfpathcurveto{\pgfqpoint{5.064313in}{2.522969in}}{\pgfqpoint{5.059014in}{2.520774in}}{\pgfqpoint{5.055107in}{2.516867in}}%
\pgfpathcurveto{\pgfqpoint{5.051200in}{2.512960in}}{\pgfqpoint{5.049005in}{2.507661in}}{\pgfqpoint{5.049005in}{2.502135in}}%
\pgfpathcurveto{\pgfqpoint{5.049005in}{2.496610in}}{\pgfqpoint{5.051200in}{2.491311in}}{\pgfqpoint{5.055107in}{2.487404in}}%
\pgfpathcurveto{\pgfqpoint{5.059014in}{2.483497in}}{\pgfqpoint{5.064313in}{2.481302in}}{\pgfqpoint{5.069839in}{2.481302in}}%
\pgfpathclose%
\pgfusepath{fill}%
\end{pgfscope}%
\begin{pgfscope}%
\pgfpathrectangle{\pgfqpoint{4.376725in}{2.423832in}}{\pgfqpoint{1.162500in}{0.755000in}}%
\pgfusepath{clip}%
\pgfsetbuttcap%
\pgfsetroundjoin%
\definecolor{currentfill}{rgb}{0.000000,0.000000,0.000000}%
\pgfsetfillcolor{currentfill}%
\pgfsetfillopacity{0.500000}%
\pgfsetlinewidth{0.000000pt}%
\definecolor{currentstroke}{rgb}{0.000000,0.000000,0.000000}%
\pgfsetstrokecolor{currentstroke}%
\pgfsetdash{}{0pt}%
\pgfpathmoveto{\pgfqpoint{4.941375in}{2.503012in}}%
\pgfpathcurveto{\pgfqpoint{4.946900in}{2.503012in}}{\pgfqpoint{4.952199in}{2.505207in}}{\pgfqpoint{4.956106in}{2.509114in}}%
\pgfpathcurveto{\pgfqpoint{4.960013in}{2.513021in}}{\pgfqpoint{4.962208in}{2.518320in}}{\pgfqpoint{4.962208in}{2.523845in}}%
\pgfpathcurveto{\pgfqpoint{4.962208in}{2.529370in}}{\pgfqpoint{4.960013in}{2.534670in}}{\pgfqpoint{4.956106in}{2.538577in}}%
\pgfpathcurveto{\pgfqpoint{4.952199in}{2.542483in}}{\pgfqpoint{4.946900in}{2.544679in}}{\pgfqpoint{4.941375in}{2.544679in}}%
\pgfpathcurveto{\pgfqpoint{4.935850in}{2.544679in}}{\pgfqpoint{4.930550in}{2.542483in}}{\pgfqpoint{4.926643in}{2.538577in}}%
\pgfpathcurveto{\pgfqpoint{4.922736in}{2.534670in}}{\pgfqpoint{4.920541in}{2.529370in}}{\pgfqpoint{4.920541in}{2.523845in}}%
\pgfpathcurveto{\pgfqpoint{4.920541in}{2.518320in}}{\pgfqpoint{4.922736in}{2.513021in}}{\pgfqpoint{4.926643in}{2.509114in}}%
\pgfpathcurveto{\pgfqpoint{4.930550in}{2.505207in}}{\pgfqpoint{4.935850in}{2.503012in}}{\pgfqpoint{4.941375in}{2.503012in}}%
\pgfpathclose%
\pgfusepath{fill}%
\end{pgfscope}%
\begin{pgfscope}%
\pgfpathrectangle{\pgfqpoint{4.376725in}{2.423832in}}{\pgfqpoint{1.162500in}{0.755000in}}%
\pgfusepath{clip}%
\pgfsetbuttcap%
\pgfsetroundjoin%
\definecolor{currentfill}{rgb}{0.000000,0.000000,0.000000}%
\pgfsetfillcolor{currentfill}%
\pgfsetfillopacity{0.500000}%
\pgfsetlinewidth{0.000000pt}%
\definecolor{currentstroke}{rgb}{0.000000,0.000000,0.000000}%
\pgfsetstrokecolor{currentstroke}%
\pgfsetdash{}{0pt}%
\pgfpathmoveto{\pgfqpoint{4.799153in}{3.140023in}}%
\pgfpathcurveto{\pgfqpoint{4.804678in}{3.140023in}}{\pgfqpoint{4.809977in}{3.142218in}}{\pgfqpoint{4.813884in}{3.146125in}}%
\pgfpathcurveto{\pgfqpoint{4.817791in}{3.150032in}}{\pgfqpoint{4.819986in}{3.155331in}}{\pgfqpoint{4.819986in}{3.160856in}}%
\pgfpathcurveto{\pgfqpoint{4.819986in}{3.166381in}}{\pgfqpoint{4.817791in}{3.171681in}}{\pgfqpoint{4.813884in}{3.175588in}}%
\pgfpathcurveto{\pgfqpoint{4.809977in}{3.179494in}}{\pgfqpoint{4.804678in}{3.181690in}}{\pgfqpoint{4.799153in}{3.181690in}}%
\pgfpathcurveto{\pgfqpoint{4.793628in}{3.181690in}}{\pgfqpoint{4.788328in}{3.179494in}}{\pgfqpoint{4.784421in}{3.175588in}}%
\pgfpathcurveto{\pgfqpoint{4.780515in}{3.171681in}}{\pgfqpoint{4.778320in}{3.166381in}}{\pgfqpoint{4.778320in}{3.160856in}}%
\pgfpathcurveto{\pgfqpoint{4.778320in}{3.155331in}}{\pgfqpoint{4.780515in}{3.150032in}}{\pgfqpoint{4.784421in}{3.146125in}}%
\pgfpathcurveto{\pgfqpoint{4.788328in}{3.142218in}}{\pgfqpoint{4.793628in}{3.140023in}}{\pgfqpoint{4.799153in}{3.140023in}}%
\pgfpathclose%
\pgfusepath{fill}%
\end{pgfscope}%
\begin{pgfscope}%
\pgfpathrectangle{\pgfqpoint{4.376725in}{2.423832in}}{\pgfqpoint{1.162500in}{0.755000in}}%
\pgfusepath{clip}%
\pgfsetbuttcap%
\pgfsetroundjoin%
\definecolor{currentfill}{rgb}{0.000000,0.000000,0.000000}%
\pgfsetfillcolor{currentfill}%
\pgfsetfillopacity{0.500000}%
\pgfsetlinewidth{0.000000pt}%
\definecolor{currentstroke}{rgb}{0.000000,0.000000,0.000000}%
\pgfsetstrokecolor{currentstroke}%
\pgfsetdash{}{0pt}%
\pgfpathmoveto{\pgfqpoint{4.595218in}{2.625365in}}%
\pgfpathcurveto{\pgfqpoint{4.600743in}{2.625365in}}{\pgfqpoint{4.606043in}{2.627561in}}{\pgfqpoint{4.609950in}{2.631467in}}%
\pgfpathcurveto{\pgfqpoint{4.613857in}{2.635374in}}{\pgfqpoint{4.616052in}{2.640674in}}{\pgfqpoint{4.616052in}{2.646199in}}%
\pgfpathcurveto{\pgfqpoint{4.616052in}{2.651724in}}{\pgfqpoint{4.613857in}{2.657023in}}{\pgfqpoint{4.609950in}{2.660930in}}%
\pgfpathcurveto{\pgfqpoint{4.606043in}{2.664837in}}{\pgfqpoint{4.600743in}{2.667032in}}{\pgfqpoint{4.595218in}{2.667032in}}%
\pgfpathcurveto{\pgfqpoint{4.589693in}{2.667032in}}{\pgfqpoint{4.584394in}{2.664837in}}{\pgfqpoint{4.580487in}{2.660930in}}%
\pgfpathcurveto{\pgfqpoint{4.576580in}{2.657023in}}{\pgfqpoint{4.574385in}{2.651724in}}{\pgfqpoint{4.574385in}{2.646199in}}%
\pgfpathcurveto{\pgfqpoint{4.574385in}{2.640674in}}{\pgfqpoint{4.576580in}{2.635374in}}{\pgfqpoint{4.580487in}{2.631467in}}%
\pgfpathcurveto{\pgfqpoint{4.584394in}{2.627561in}}{\pgfqpoint{4.589693in}{2.625365in}}{\pgfqpoint{4.595218in}{2.625365in}}%
\pgfpathclose%
\pgfusepath{fill}%
\end{pgfscope}%
\begin{pgfscope}%
\pgfpathrectangle{\pgfqpoint{4.376725in}{2.423832in}}{\pgfqpoint{1.162500in}{0.755000in}}%
\pgfusepath{clip}%
\pgfsetbuttcap%
\pgfsetroundjoin%
\definecolor{currentfill}{rgb}{0.000000,0.000000,0.000000}%
\pgfsetfillcolor{currentfill}%
\pgfsetfillopacity{0.500000}%
\pgfsetlinewidth{0.000000pt}%
\definecolor{currentstroke}{rgb}{0.000000,0.000000,0.000000}%
\pgfsetstrokecolor{currentstroke}%
\pgfsetdash{}{0pt}%
\pgfpathmoveto{\pgfqpoint{4.834270in}{2.450809in}}%
\pgfpathcurveto{\pgfqpoint{4.839795in}{2.450809in}}{\pgfqpoint{4.845094in}{2.453004in}}{\pgfqpoint{4.849001in}{2.456911in}}%
\pgfpathcurveto{\pgfqpoint{4.852908in}{2.460818in}}{\pgfqpoint{4.855103in}{2.466117in}}{\pgfqpoint{4.855103in}{2.471642in}}%
\pgfpathcurveto{\pgfqpoint{4.855103in}{2.477167in}}{\pgfqpoint{4.852908in}{2.482467in}}{\pgfqpoint{4.849001in}{2.486374in}}%
\pgfpathcurveto{\pgfqpoint{4.845094in}{2.490280in}}{\pgfqpoint{4.839795in}{2.492476in}}{\pgfqpoint{4.834270in}{2.492476in}}%
\pgfpathcurveto{\pgfqpoint{4.828745in}{2.492476in}}{\pgfqpoint{4.823445in}{2.490280in}}{\pgfqpoint{4.819538in}{2.486374in}}%
\pgfpathcurveto{\pgfqpoint{4.815632in}{2.482467in}}{\pgfqpoint{4.813437in}{2.477167in}}{\pgfqpoint{4.813437in}{2.471642in}}%
\pgfpathcurveto{\pgfqpoint{4.813437in}{2.466117in}}{\pgfqpoint{4.815632in}{2.460818in}}{\pgfqpoint{4.819538in}{2.456911in}}%
\pgfpathcurveto{\pgfqpoint{4.823445in}{2.453004in}}{\pgfqpoint{4.828745in}{2.450809in}}{\pgfqpoint{4.834270in}{2.450809in}}%
\pgfpathclose%
\pgfusepath{fill}%
\end{pgfscope}%
\begin{pgfscope}%
\pgfsetrectcap%
\pgfsetmiterjoin%
\pgfsetlinewidth{0.803000pt}%
\definecolor{currentstroke}{rgb}{0.501961,0.501961,0.501961}%
\pgfsetstrokecolor{currentstroke}%
\pgfsetdash{}{0pt}%
\pgfpathmoveto{\pgfqpoint{4.376725in}{2.423832in}}%
\pgfpathlineto{\pgfqpoint{4.376725in}{3.178832in}}%
\pgfusepath{stroke}%
\end{pgfscope}%
\begin{pgfscope}%
\pgfsetrectcap%
\pgfsetmiterjoin%
\pgfsetlinewidth{0.803000pt}%
\definecolor{currentstroke}{rgb}{0.501961,0.501961,0.501961}%
\pgfsetstrokecolor{currentstroke}%
\pgfsetdash{}{0pt}%
\pgfpathmoveto{\pgfqpoint{5.539225in}{2.423832in}}%
\pgfpathlineto{\pgfqpoint{5.539225in}{3.178832in}}%
\pgfusepath{stroke}%
\end{pgfscope}%
\begin{pgfscope}%
\pgfsetrectcap%
\pgfsetmiterjoin%
\pgfsetlinewidth{0.803000pt}%
\definecolor{currentstroke}{rgb}{0.501961,0.501961,0.501961}%
\pgfsetstrokecolor{currentstroke}%
\pgfsetdash{}{0pt}%
\pgfpathmoveto{\pgfqpoint{4.376725in}{2.423832in}}%
\pgfpathlineto{\pgfqpoint{5.539225in}{2.423832in}}%
\pgfusepath{stroke}%
\end{pgfscope}%
\begin{pgfscope}%
\pgfsetrectcap%
\pgfsetmiterjoin%
\pgfsetlinewidth{0.803000pt}%
\definecolor{currentstroke}{rgb}{0.501961,0.501961,0.501961}%
\pgfsetstrokecolor{currentstroke}%
\pgfsetdash{}{0pt}%
\pgfpathmoveto{\pgfqpoint{4.376725in}{3.178832in}}%
\pgfpathlineto{\pgfqpoint{5.539225in}{3.178832in}}%
\pgfusepath{stroke}%
\end{pgfscope}%
\begin{pgfscope}%
\pgfsetbuttcap%
\pgfsetmiterjoin%
\definecolor{currentfill}{rgb}{1.000000,1.000000,1.000000}%
\pgfsetfillcolor{currentfill}%
\pgfsetlinewidth{0.000000pt}%
\definecolor{currentstroke}{rgb}{0.000000,0.000000,0.000000}%
\pgfsetstrokecolor{currentstroke}%
\pgfsetstrokeopacity{0.000000}%
\pgfsetdash{}{0pt}%
\pgfpathmoveto{\pgfqpoint{0.889225in}{1.668832in}}%
\pgfpathlineto{\pgfqpoint{2.051725in}{1.668832in}}%
\pgfpathlineto{\pgfqpoint{2.051725in}{2.423832in}}%
\pgfpathlineto{\pgfqpoint{0.889225in}{2.423832in}}%
\pgfpathclose%
\pgfusepath{fill}%
\end{pgfscope}%
\begin{pgfscope}%
\pgfpathrectangle{\pgfqpoint{0.889225in}{1.668832in}}{\pgfqpoint{1.162500in}{0.755000in}}%
\pgfusepath{clip}%
\pgfsetbuttcap%
\pgfsetroundjoin%
\definecolor{currentfill}{rgb}{0.000000,0.000000,0.000000}%
\pgfsetfillcolor{currentfill}%
\pgfsetfillopacity{0.500000}%
\pgfsetlinewidth{0.000000pt}%
\definecolor{currentstroke}{rgb}{0.000000,0.000000,0.000000}%
\pgfsetstrokecolor{currentstroke}%
\pgfsetdash{}{0pt}%
\pgfpathmoveto{\pgfqpoint{1.893746in}{1.935152in}}%
\pgfpathcurveto{\pgfqpoint{1.899271in}{1.935152in}}{\pgfqpoint{1.904570in}{1.937347in}}{\pgfqpoint{1.908477in}{1.941254in}}%
\pgfpathcurveto{\pgfqpoint{1.912384in}{1.945161in}}{\pgfqpoint{1.914579in}{1.950460in}}{\pgfqpoint{1.914579in}{1.955985in}}%
\pgfpathcurveto{\pgfqpoint{1.914579in}{1.961510in}}{\pgfqpoint{1.912384in}{1.966810in}}{\pgfqpoint{1.908477in}{1.970717in}}%
\pgfpathcurveto{\pgfqpoint{1.904570in}{1.974623in}}{\pgfqpoint{1.899271in}{1.976819in}}{\pgfqpoint{1.893746in}{1.976819in}}%
\pgfpathcurveto{\pgfqpoint{1.888221in}{1.976819in}}{\pgfqpoint{1.882921in}{1.974623in}}{\pgfqpoint{1.879015in}{1.970717in}}%
\pgfpathcurveto{\pgfqpoint{1.875108in}{1.966810in}}{\pgfqpoint{1.872913in}{1.961510in}}{\pgfqpoint{1.872913in}{1.955985in}}%
\pgfpathcurveto{\pgfqpoint{1.872913in}{1.950460in}}{\pgfqpoint{1.875108in}{1.945161in}}{\pgfqpoint{1.879015in}{1.941254in}}%
\pgfpathcurveto{\pgfqpoint{1.882921in}{1.937347in}}{\pgfqpoint{1.888221in}{1.935152in}}{\pgfqpoint{1.893746in}{1.935152in}}%
\pgfpathclose%
\pgfusepath{fill}%
\end{pgfscope}%
\begin{pgfscope}%
\pgfpathrectangle{\pgfqpoint{0.889225in}{1.668832in}}{\pgfqpoint{1.162500in}{0.755000in}}%
\pgfusepath{clip}%
\pgfsetbuttcap%
\pgfsetroundjoin%
\definecolor{currentfill}{rgb}{0.000000,0.000000,0.000000}%
\pgfsetfillcolor{currentfill}%
\pgfsetfillopacity{0.500000}%
\pgfsetlinewidth{0.000000pt}%
\definecolor{currentstroke}{rgb}{0.000000,0.000000,0.000000}%
\pgfsetstrokecolor{currentstroke}%
\pgfsetdash{}{0pt}%
\pgfpathmoveto{\pgfqpoint{1.476892in}{2.326950in}}%
\pgfpathcurveto{\pgfqpoint{1.482417in}{2.326950in}}{\pgfqpoint{1.487717in}{2.329145in}}{\pgfqpoint{1.491623in}{2.333052in}}%
\pgfpathcurveto{\pgfqpoint{1.495530in}{2.336959in}}{\pgfqpoint{1.497725in}{2.342258in}}{\pgfqpoint{1.497725in}{2.347783in}}%
\pgfpathcurveto{\pgfqpoint{1.497725in}{2.353308in}}{\pgfqpoint{1.495530in}{2.358608in}}{\pgfqpoint{1.491623in}{2.362515in}}%
\pgfpathcurveto{\pgfqpoint{1.487717in}{2.366421in}}{\pgfqpoint{1.482417in}{2.368616in}}{\pgfqpoint{1.476892in}{2.368616in}}%
\pgfpathcurveto{\pgfqpoint{1.471367in}{2.368616in}}{\pgfqpoint{1.466067in}{2.366421in}}{\pgfqpoint{1.462161in}{2.362515in}}%
\pgfpathcurveto{\pgfqpoint{1.458254in}{2.358608in}}{\pgfqpoint{1.456059in}{2.353308in}}{\pgfqpoint{1.456059in}{2.347783in}}%
\pgfpathcurveto{\pgfqpoint{1.456059in}{2.342258in}}{\pgfqpoint{1.458254in}{2.336959in}}{\pgfqpoint{1.462161in}{2.333052in}}%
\pgfpathcurveto{\pgfqpoint{1.466067in}{2.329145in}}{\pgfqpoint{1.471367in}{2.326950in}}{\pgfqpoint{1.476892in}{2.326950in}}%
\pgfpathclose%
\pgfusepath{fill}%
\end{pgfscope}%
\begin{pgfscope}%
\pgfpathrectangle{\pgfqpoint{0.889225in}{1.668832in}}{\pgfqpoint{1.162500in}{0.755000in}}%
\pgfusepath{clip}%
\pgfsetbuttcap%
\pgfsetroundjoin%
\definecolor{currentfill}{rgb}{0.000000,0.000000,0.000000}%
\pgfsetfillcolor{currentfill}%
\pgfsetfillopacity{0.500000}%
\pgfsetlinewidth{0.000000pt}%
\definecolor{currentstroke}{rgb}{0.000000,0.000000,0.000000}%
\pgfsetstrokecolor{currentstroke}%
\pgfsetdash{}{0pt}%
\pgfpathmoveto{\pgfqpoint{1.478974in}{2.385023in}}%
\pgfpathcurveto{\pgfqpoint{1.484499in}{2.385023in}}{\pgfqpoint{1.489799in}{2.387218in}}{\pgfqpoint{1.493705in}{2.391125in}}%
\pgfpathcurveto{\pgfqpoint{1.497612in}{2.395032in}}{\pgfqpoint{1.499807in}{2.400331in}}{\pgfqpoint{1.499807in}{2.405856in}}%
\pgfpathcurveto{\pgfqpoint{1.499807in}{2.411381in}}{\pgfqpoint{1.497612in}{2.416681in}}{\pgfqpoint{1.493705in}{2.420588in}}%
\pgfpathcurveto{\pgfqpoint{1.489799in}{2.424494in}}{\pgfqpoint{1.484499in}{2.426690in}}{\pgfqpoint{1.478974in}{2.426690in}}%
\pgfpathcurveto{\pgfqpoint{1.473449in}{2.426690in}}{\pgfqpoint{1.468149in}{2.424494in}}{\pgfqpoint{1.464243in}{2.420588in}}%
\pgfpathcurveto{\pgfqpoint{1.460336in}{2.416681in}}{\pgfqpoint{1.458141in}{2.411381in}}{\pgfqpoint{1.458141in}{2.405856in}}%
\pgfpathcurveto{\pgfqpoint{1.458141in}{2.400331in}}{\pgfqpoint{1.460336in}{2.395032in}}{\pgfqpoint{1.464243in}{2.391125in}}%
\pgfpathcurveto{\pgfqpoint{1.468149in}{2.387218in}}{\pgfqpoint{1.473449in}{2.385023in}}{\pgfqpoint{1.478974in}{2.385023in}}%
\pgfpathclose%
\pgfusepath{fill}%
\end{pgfscope}%
\begin{pgfscope}%
\pgfpathrectangle{\pgfqpoint{0.889225in}{1.668832in}}{\pgfqpoint{1.162500in}{0.755000in}}%
\pgfusepath{clip}%
\pgfsetbuttcap%
\pgfsetroundjoin%
\definecolor{currentfill}{rgb}{0.000000,0.000000,0.000000}%
\pgfsetfillcolor{currentfill}%
\pgfsetfillopacity{0.500000}%
\pgfsetlinewidth{0.000000pt}%
\definecolor{currentstroke}{rgb}{0.000000,0.000000,0.000000}%
\pgfsetstrokecolor{currentstroke}%
\pgfsetdash{}{0pt}%
\pgfpathmoveto{\pgfqpoint{1.120714in}{2.103661in}}%
\pgfpathcurveto{\pgfqpoint{1.126239in}{2.103661in}}{\pgfqpoint{1.131538in}{2.105856in}}{\pgfqpoint{1.135445in}{2.109762in}}%
\pgfpathcurveto{\pgfqpoint{1.139352in}{2.113669in}}{\pgfqpoint{1.141547in}{2.118969in}}{\pgfqpoint{1.141547in}{2.124494in}}%
\pgfpathcurveto{\pgfqpoint{1.141547in}{2.130019in}}{\pgfqpoint{1.139352in}{2.135318in}}{\pgfqpoint{1.135445in}{2.139225in}}%
\pgfpathcurveto{\pgfqpoint{1.131538in}{2.143132in}}{\pgfqpoint{1.126239in}{2.145327in}}{\pgfqpoint{1.120714in}{2.145327in}}%
\pgfpathcurveto{\pgfqpoint{1.115189in}{2.145327in}}{\pgfqpoint{1.109889in}{2.143132in}}{\pgfqpoint{1.105982in}{2.139225in}}%
\pgfpathcurveto{\pgfqpoint{1.102076in}{2.135318in}}{\pgfqpoint{1.099881in}{2.130019in}}{\pgfqpoint{1.099881in}{2.124494in}}%
\pgfpathcurveto{\pgfqpoint{1.099881in}{2.118969in}}{\pgfqpoint{1.102076in}{2.113669in}}{\pgfqpoint{1.105982in}{2.109762in}}%
\pgfpathcurveto{\pgfqpoint{1.109889in}{2.105856in}}{\pgfqpoint{1.115189in}{2.103661in}}{\pgfqpoint{1.120714in}{2.103661in}}%
\pgfpathclose%
\pgfusepath{fill}%
\end{pgfscope}%
\begin{pgfscope}%
\pgfpathrectangle{\pgfqpoint{0.889225in}{1.668832in}}{\pgfqpoint{1.162500in}{0.755000in}}%
\pgfusepath{clip}%
\pgfsetbuttcap%
\pgfsetroundjoin%
\definecolor{currentfill}{rgb}{0.000000,0.000000,0.000000}%
\pgfsetfillcolor{currentfill}%
\pgfsetfillopacity{0.500000}%
\pgfsetlinewidth{0.000000pt}%
\definecolor{currentstroke}{rgb}{0.000000,0.000000,0.000000}%
\pgfsetstrokecolor{currentstroke}%
\pgfsetdash{}{0pt}%
\pgfpathmoveto{\pgfqpoint{1.126248in}{2.149550in}}%
\pgfpathcurveto{\pgfqpoint{1.131773in}{2.149550in}}{\pgfqpoint{1.137073in}{2.151746in}}{\pgfqpoint{1.140980in}{2.155652in}}%
\pgfpathcurveto{\pgfqpoint{1.144886in}{2.159559in}}{\pgfqpoint{1.147082in}{2.164859in}}{\pgfqpoint{1.147082in}{2.170384in}}%
\pgfpathcurveto{\pgfqpoint{1.147082in}{2.175909in}}{\pgfqpoint{1.144886in}{2.181208in}}{\pgfqpoint{1.140980in}{2.185115in}}%
\pgfpathcurveto{\pgfqpoint{1.137073in}{2.189022in}}{\pgfqpoint{1.131773in}{2.191217in}}{\pgfqpoint{1.126248in}{2.191217in}}%
\pgfpathcurveto{\pgfqpoint{1.120723in}{2.191217in}}{\pgfqpoint{1.115424in}{2.189022in}}{\pgfqpoint{1.111517in}{2.185115in}}%
\pgfpathcurveto{\pgfqpoint{1.107610in}{2.181208in}}{\pgfqpoint{1.105415in}{2.175909in}}{\pgfqpoint{1.105415in}{2.170384in}}%
\pgfpathcurveto{\pgfqpoint{1.105415in}{2.164859in}}{\pgfqpoint{1.107610in}{2.159559in}}{\pgfqpoint{1.111517in}{2.155652in}}%
\pgfpathcurveto{\pgfqpoint{1.115424in}{2.151746in}}{\pgfqpoint{1.120723in}{2.149550in}}{\pgfqpoint{1.126248in}{2.149550in}}%
\pgfpathclose%
\pgfusepath{fill}%
\end{pgfscope}%
\begin{pgfscope}%
\pgfpathrectangle{\pgfqpoint{0.889225in}{1.668832in}}{\pgfqpoint{1.162500in}{0.755000in}}%
\pgfusepath{clip}%
\pgfsetbuttcap%
\pgfsetroundjoin%
\definecolor{currentfill}{rgb}{0.000000,0.000000,0.000000}%
\pgfsetfillcolor{currentfill}%
\pgfsetfillopacity{0.500000}%
\pgfsetlinewidth{0.000000pt}%
\definecolor{currentstroke}{rgb}{0.000000,0.000000,0.000000}%
\pgfsetstrokecolor{currentstroke}%
\pgfsetdash{}{0pt}%
\pgfpathmoveto{\pgfqpoint{0.977704in}{2.049204in}}%
\pgfpathcurveto{\pgfqpoint{0.983229in}{2.049204in}}{\pgfqpoint{0.988528in}{2.051399in}}{\pgfqpoint{0.992435in}{2.055306in}}%
\pgfpathcurveto{\pgfqpoint{0.996342in}{2.059212in}}{\pgfqpoint{0.998537in}{2.064512in}}{\pgfqpoint{0.998537in}{2.070037in}}%
\pgfpathcurveto{\pgfqpoint{0.998537in}{2.075562in}}{\pgfqpoint{0.996342in}{2.080862in}}{\pgfqpoint{0.992435in}{2.084768in}}%
\pgfpathcurveto{\pgfqpoint{0.988528in}{2.088675in}}{\pgfqpoint{0.983229in}{2.090870in}}{\pgfqpoint{0.977704in}{2.090870in}}%
\pgfpathcurveto{\pgfqpoint{0.972179in}{2.090870in}}{\pgfqpoint{0.966879in}{2.088675in}}{\pgfqpoint{0.962972in}{2.084768in}}%
\pgfpathcurveto{\pgfqpoint{0.959066in}{2.080862in}}{\pgfqpoint{0.956870in}{2.075562in}}{\pgfqpoint{0.956870in}{2.070037in}}%
\pgfpathcurveto{\pgfqpoint{0.956870in}{2.064512in}}{\pgfqpoint{0.959066in}{2.059212in}}{\pgfqpoint{0.962972in}{2.055306in}}%
\pgfpathcurveto{\pgfqpoint{0.966879in}{2.051399in}}{\pgfqpoint{0.972179in}{2.049204in}}{\pgfqpoint{0.977704in}{2.049204in}}%
\pgfpathclose%
\pgfusepath{fill}%
\end{pgfscope}%
\begin{pgfscope}%
\pgfpathrectangle{\pgfqpoint{0.889225in}{1.668832in}}{\pgfqpoint{1.162500in}{0.755000in}}%
\pgfusepath{clip}%
\pgfsetbuttcap%
\pgfsetroundjoin%
\definecolor{currentfill}{rgb}{0.000000,0.000000,0.000000}%
\pgfsetfillcolor{currentfill}%
\pgfsetfillopacity{0.500000}%
\pgfsetlinewidth{0.000000pt}%
\definecolor{currentstroke}{rgb}{0.000000,0.000000,0.000000}%
\pgfsetstrokecolor{currentstroke}%
\pgfsetdash{}{0pt}%
\pgfpathmoveto{\pgfqpoint{0.916904in}{1.665975in}}%
\pgfpathcurveto{\pgfqpoint{0.922429in}{1.665975in}}{\pgfqpoint{0.927728in}{1.668170in}}{\pgfqpoint{0.931635in}{1.672077in}}%
\pgfpathcurveto{\pgfqpoint{0.935542in}{1.675984in}}{\pgfqpoint{0.937737in}{1.681283in}}{\pgfqpoint{0.937737in}{1.686809in}}%
\pgfpathcurveto{\pgfqpoint{0.937737in}{1.692334in}}{\pgfqpoint{0.935542in}{1.697633in}}{\pgfqpoint{0.931635in}{1.701540in}}%
\pgfpathcurveto{\pgfqpoint{0.927728in}{1.705447in}}{\pgfqpoint{0.922429in}{1.707642in}}{\pgfqpoint{0.916904in}{1.707642in}}%
\pgfpathcurveto{\pgfqpoint{0.911379in}{1.707642in}}{\pgfqpoint{0.906079in}{1.705447in}}{\pgfqpoint{0.902172in}{1.701540in}}%
\pgfpathcurveto{\pgfqpoint{0.898265in}{1.697633in}}{\pgfqpoint{0.896070in}{1.692334in}}{\pgfqpoint{0.896070in}{1.686809in}}%
\pgfpathcurveto{\pgfqpoint{0.896070in}{1.681283in}}{\pgfqpoint{0.898265in}{1.675984in}}{\pgfqpoint{0.902172in}{1.672077in}}%
\pgfpathcurveto{\pgfqpoint{0.906079in}{1.668170in}}{\pgfqpoint{0.911379in}{1.665975in}}{\pgfqpoint{0.916904in}{1.665975in}}%
\pgfpathclose%
\pgfusepath{fill}%
\end{pgfscope}%
\begin{pgfscope}%
\pgfpathrectangle{\pgfqpoint{0.889225in}{1.668832in}}{\pgfqpoint{1.162500in}{0.755000in}}%
\pgfusepath{clip}%
\pgfsetbuttcap%
\pgfsetroundjoin%
\definecolor{currentfill}{rgb}{0.000000,0.000000,0.000000}%
\pgfsetfillcolor{currentfill}%
\pgfsetfillopacity{0.500000}%
\pgfsetlinewidth{0.000000pt}%
\definecolor{currentstroke}{rgb}{0.000000,0.000000,0.000000}%
\pgfsetstrokecolor{currentstroke}%
\pgfsetdash{}{0pt}%
\pgfpathmoveto{\pgfqpoint{1.691400in}{2.301682in}}%
\pgfpathcurveto{\pgfqpoint{1.696925in}{2.301682in}}{\pgfqpoint{1.702225in}{2.303877in}}{\pgfqpoint{1.706132in}{2.307784in}}%
\pgfpathcurveto{\pgfqpoint{1.710038in}{2.311691in}}{\pgfqpoint{1.712234in}{2.316990in}}{\pgfqpoint{1.712234in}{2.322515in}}%
\pgfpathcurveto{\pgfqpoint{1.712234in}{2.328041in}}{\pgfqpoint{1.710038in}{2.333340in}}{\pgfqpoint{1.706132in}{2.337247in}}%
\pgfpathcurveto{\pgfqpoint{1.702225in}{2.341154in}}{\pgfqpoint{1.696925in}{2.343349in}}{\pgfqpoint{1.691400in}{2.343349in}}%
\pgfpathcurveto{\pgfqpoint{1.685875in}{2.343349in}}{\pgfqpoint{1.680576in}{2.341154in}}{\pgfqpoint{1.676669in}{2.337247in}}%
\pgfpathcurveto{\pgfqpoint{1.672762in}{2.333340in}}{\pgfqpoint{1.670567in}{2.328041in}}{\pgfqpoint{1.670567in}{2.322515in}}%
\pgfpathcurveto{\pgfqpoint{1.670567in}{2.316990in}}{\pgfqpoint{1.672762in}{2.311691in}}{\pgfqpoint{1.676669in}{2.307784in}}%
\pgfpathcurveto{\pgfqpoint{1.680576in}{2.303877in}}{\pgfqpoint{1.685875in}{2.301682in}}{\pgfqpoint{1.691400in}{2.301682in}}%
\pgfpathclose%
\pgfusepath{fill}%
\end{pgfscope}%
\begin{pgfscope}%
\pgfpathrectangle{\pgfqpoint{0.889225in}{1.668832in}}{\pgfqpoint{1.162500in}{0.755000in}}%
\pgfusepath{clip}%
\pgfsetbuttcap%
\pgfsetroundjoin%
\definecolor{currentfill}{rgb}{0.000000,0.000000,0.000000}%
\pgfsetfillcolor{currentfill}%
\pgfsetfillopacity{0.500000}%
\pgfsetlinewidth{0.000000pt}%
\definecolor{currentstroke}{rgb}{0.000000,0.000000,0.000000}%
\pgfsetstrokecolor{currentstroke}%
\pgfsetdash{}{0pt}%
\pgfpathmoveto{\pgfqpoint{1.460463in}{2.145886in}}%
\pgfpathcurveto{\pgfqpoint{1.465988in}{2.145886in}}{\pgfqpoint{1.471288in}{2.148081in}}{\pgfqpoint{1.475194in}{2.151988in}}%
\pgfpathcurveto{\pgfqpoint{1.479101in}{2.155894in}}{\pgfqpoint{1.481296in}{2.161194in}}{\pgfqpoint{1.481296in}{2.166719in}}%
\pgfpathcurveto{\pgfqpoint{1.481296in}{2.172244in}}{\pgfqpoint{1.479101in}{2.177544in}}{\pgfqpoint{1.475194in}{2.181450in}}%
\pgfpathcurveto{\pgfqpoint{1.471288in}{2.185357in}}{\pgfqpoint{1.465988in}{2.187552in}}{\pgfqpoint{1.460463in}{2.187552in}}%
\pgfpathcurveto{\pgfqpoint{1.454938in}{2.187552in}}{\pgfqpoint{1.449638in}{2.185357in}}{\pgfqpoint{1.445732in}{2.181450in}}%
\pgfpathcurveto{\pgfqpoint{1.441825in}{2.177544in}}{\pgfqpoint{1.439630in}{2.172244in}}{\pgfqpoint{1.439630in}{2.166719in}}%
\pgfpathcurveto{\pgfqpoint{1.439630in}{2.161194in}}{\pgfqpoint{1.441825in}{2.155894in}}{\pgfqpoint{1.445732in}{2.151988in}}%
\pgfpathcurveto{\pgfqpoint{1.449638in}{2.148081in}}{\pgfqpoint{1.454938in}{2.145886in}}{\pgfqpoint{1.460463in}{2.145886in}}%
\pgfpathclose%
\pgfusepath{fill}%
\end{pgfscope}%
\begin{pgfscope}%
\pgfpathrectangle{\pgfqpoint{0.889225in}{1.668832in}}{\pgfqpoint{1.162500in}{0.755000in}}%
\pgfusepath{clip}%
\pgfsetbuttcap%
\pgfsetroundjoin%
\definecolor{currentfill}{rgb}{0.000000,0.000000,0.000000}%
\pgfsetfillcolor{currentfill}%
\pgfsetfillopacity{0.500000}%
\pgfsetlinewidth{0.000000pt}%
\definecolor{currentstroke}{rgb}{0.000000,0.000000,0.000000}%
\pgfsetstrokecolor{currentstroke}%
\pgfsetdash{}{0pt}%
\pgfpathmoveto{\pgfqpoint{1.303193in}{1.969105in}}%
\pgfpathcurveto{\pgfqpoint{1.308718in}{1.969105in}}{\pgfqpoint{1.314017in}{1.971300in}}{\pgfqpoint{1.317924in}{1.975207in}}%
\pgfpathcurveto{\pgfqpoint{1.321831in}{1.979114in}}{\pgfqpoint{1.324026in}{1.984413in}}{\pgfqpoint{1.324026in}{1.989939in}}%
\pgfpathcurveto{\pgfqpoint{1.324026in}{1.995464in}}{\pgfqpoint{1.321831in}{2.000763in}}{\pgfqpoint{1.317924in}{2.004670in}}%
\pgfpathcurveto{\pgfqpoint{1.314017in}{2.008577in}}{\pgfqpoint{1.308718in}{2.010772in}}{\pgfqpoint{1.303193in}{2.010772in}}%
\pgfpathcurveto{\pgfqpoint{1.297667in}{2.010772in}}{\pgfqpoint{1.292368in}{2.008577in}}{\pgfqpoint{1.288461in}{2.004670in}}%
\pgfpathcurveto{\pgfqpoint{1.284554in}{2.000763in}}{\pgfqpoint{1.282359in}{1.995464in}}{\pgfqpoint{1.282359in}{1.989939in}}%
\pgfpathcurveto{\pgfqpoint{1.282359in}{1.984413in}}{\pgfqpoint{1.284554in}{1.979114in}}{\pgfqpoint{1.288461in}{1.975207in}}%
\pgfpathcurveto{\pgfqpoint{1.292368in}{1.971300in}}{\pgfqpoint{1.297667in}{1.969105in}}{\pgfqpoint{1.303193in}{1.969105in}}%
\pgfpathclose%
\pgfusepath{fill}%
\end{pgfscope}%
\begin{pgfscope}%
\pgfpathrectangle{\pgfqpoint{0.889225in}{1.668832in}}{\pgfqpoint{1.162500in}{0.755000in}}%
\pgfusepath{clip}%
\pgfsetbuttcap%
\pgfsetroundjoin%
\definecolor{currentfill}{rgb}{0.000000,0.000000,0.000000}%
\pgfsetfillcolor{currentfill}%
\pgfsetfillopacity{0.500000}%
\pgfsetlinewidth{0.000000pt}%
\definecolor{currentstroke}{rgb}{0.000000,0.000000,0.000000}%
\pgfsetstrokecolor{currentstroke}%
\pgfsetdash{}{0pt}%
\pgfpathmoveto{\pgfqpoint{1.689350in}{2.239104in}}%
\pgfpathcurveto{\pgfqpoint{1.694875in}{2.239104in}}{\pgfqpoint{1.700175in}{2.241300in}}{\pgfqpoint{1.704082in}{2.245206in}}%
\pgfpathcurveto{\pgfqpoint{1.707989in}{2.249113in}}{\pgfqpoint{1.710184in}{2.254413in}}{\pgfqpoint{1.710184in}{2.259938in}}%
\pgfpathcurveto{\pgfqpoint{1.710184in}{2.265463in}}{\pgfqpoint{1.707989in}{2.270762in}}{\pgfqpoint{1.704082in}{2.274669in}}%
\pgfpathcurveto{\pgfqpoint{1.700175in}{2.278576in}}{\pgfqpoint{1.694875in}{2.280771in}}{\pgfqpoint{1.689350in}{2.280771in}}%
\pgfpathcurveto{\pgfqpoint{1.683825in}{2.280771in}}{\pgfqpoint{1.678526in}{2.278576in}}{\pgfqpoint{1.674619in}{2.274669in}}%
\pgfpathcurveto{\pgfqpoint{1.670712in}{2.270762in}}{\pgfqpoint{1.668517in}{2.265463in}}{\pgfqpoint{1.668517in}{2.259938in}}%
\pgfpathcurveto{\pgfqpoint{1.668517in}{2.254413in}}{\pgfqpoint{1.670712in}{2.249113in}}{\pgfqpoint{1.674619in}{2.245206in}}%
\pgfpathcurveto{\pgfqpoint{1.678526in}{2.241300in}}{\pgfqpoint{1.683825in}{2.239104in}}{\pgfqpoint{1.689350in}{2.239104in}}%
\pgfpathclose%
\pgfusepath{fill}%
\end{pgfscope}%
\begin{pgfscope}%
\pgfpathrectangle{\pgfqpoint{0.889225in}{1.668832in}}{\pgfqpoint{1.162500in}{0.755000in}}%
\pgfusepath{clip}%
\pgfsetbuttcap%
\pgfsetroundjoin%
\definecolor{currentfill}{rgb}{0.000000,0.000000,0.000000}%
\pgfsetfillcolor{currentfill}%
\pgfsetfillopacity{0.500000}%
\pgfsetlinewidth{0.000000pt}%
\definecolor{currentstroke}{rgb}{0.000000,0.000000,0.000000}%
\pgfsetstrokecolor{currentstroke}%
\pgfsetdash{}{0pt}%
\pgfpathmoveto{\pgfqpoint{1.102396in}{1.960666in}}%
\pgfpathcurveto{\pgfqpoint{1.107921in}{1.960666in}}{\pgfqpoint{1.113221in}{1.962861in}}{\pgfqpoint{1.117128in}{1.966768in}}%
\pgfpathcurveto{\pgfqpoint{1.121035in}{1.970675in}}{\pgfqpoint{1.123230in}{1.975974in}}{\pgfqpoint{1.123230in}{1.981499in}}%
\pgfpathcurveto{\pgfqpoint{1.123230in}{1.987024in}}{\pgfqpoint{1.121035in}{1.992324in}}{\pgfqpoint{1.117128in}{1.996231in}}%
\pgfpathcurveto{\pgfqpoint{1.113221in}{2.000138in}}{\pgfqpoint{1.107921in}{2.002333in}}{\pgfqpoint{1.102396in}{2.002333in}}%
\pgfpathcurveto{\pgfqpoint{1.096871in}{2.002333in}}{\pgfqpoint{1.091572in}{2.000138in}}{\pgfqpoint{1.087665in}{1.996231in}}%
\pgfpathcurveto{\pgfqpoint{1.083758in}{1.992324in}}{\pgfqpoint{1.081563in}{1.987024in}}{\pgfqpoint{1.081563in}{1.981499in}}%
\pgfpathcurveto{\pgfqpoint{1.081563in}{1.975974in}}{\pgfqpoint{1.083758in}{1.970675in}}{\pgfqpoint{1.087665in}{1.966768in}}%
\pgfpathcurveto{\pgfqpoint{1.091572in}{1.962861in}}{\pgfqpoint{1.096871in}{1.960666in}}{\pgfqpoint{1.102396in}{1.960666in}}%
\pgfpathclose%
\pgfusepath{fill}%
\end{pgfscope}%
\begin{pgfscope}%
\pgfpathrectangle{\pgfqpoint{0.889225in}{1.668832in}}{\pgfqpoint{1.162500in}{0.755000in}}%
\pgfusepath{clip}%
\pgfsetbuttcap%
\pgfsetroundjoin%
\definecolor{currentfill}{rgb}{0.000000,0.000000,0.000000}%
\pgfsetfillcolor{currentfill}%
\pgfsetfillopacity{0.500000}%
\pgfsetlinewidth{0.000000pt}%
\definecolor{currentstroke}{rgb}{0.000000,0.000000,0.000000}%
\pgfsetstrokecolor{currentstroke}%
\pgfsetdash{}{0pt}%
\pgfpathmoveto{\pgfqpoint{1.143186in}{1.722959in}}%
\pgfpathcurveto{\pgfqpoint{1.148711in}{1.722959in}}{\pgfqpoint{1.154011in}{1.725154in}}{\pgfqpoint{1.157918in}{1.729061in}}%
\pgfpathcurveto{\pgfqpoint{1.161825in}{1.732968in}}{\pgfqpoint{1.164020in}{1.738267in}}{\pgfqpoint{1.164020in}{1.743792in}}%
\pgfpathcurveto{\pgfqpoint{1.164020in}{1.749317in}}{\pgfqpoint{1.161825in}{1.754617in}}{\pgfqpoint{1.157918in}{1.758523in}}%
\pgfpathcurveto{\pgfqpoint{1.154011in}{1.762430in}}{\pgfqpoint{1.148711in}{1.764625in}}{\pgfqpoint{1.143186in}{1.764625in}}%
\pgfpathcurveto{\pgfqpoint{1.137661in}{1.764625in}}{\pgfqpoint{1.132362in}{1.762430in}}{\pgfqpoint{1.128455in}{1.758523in}}%
\pgfpathcurveto{\pgfqpoint{1.124548in}{1.754617in}}{\pgfqpoint{1.122353in}{1.749317in}}{\pgfqpoint{1.122353in}{1.743792in}}%
\pgfpathcurveto{\pgfqpoint{1.122353in}{1.738267in}}{\pgfqpoint{1.124548in}{1.732968in}}{\pgfqpoint{1.128455in}{1.729061in}}%
\pgfpathcurveto{\pgfqpoint{1.132362in}{1.725154in}}{\pgfqpoint{1.137661in}{1.722959in}}{\pgfqpoint{1.143186in}{1.722959in}}%
\pgfpathclose%
\pgfusepath{fill}%
\end{pgfscope}%
\begin{pgfscope}%
\pgfpathrectangle{\pgfqpoint{0.889225in}{1.668832in}}{\pgfqpoint{1.162500in}{0.755000in}}%
\pgfusepath{clip}%
\pgfsetbuttcap%
\pgfsetroundjoin%
\definecolor{currentfill}{rgb}{0.000000,0.000000,0.000000}%
\pgfsetfillcolor{currentfill}%
\pgfsetfillopacity{0.500000}%
\pgfsetlinewidth{0.000000pt}%
\definecolor{currentstroke}{rgb}{0.000000,0.000000,0.000000}%
\pgfsetstrokecolor{currentstroke}%
\pgfsetdash{}{0pt}%
\pgfpathmoveto{\pgfqpoint{2.024046in}{2.267350in}}%
\pgfpathcurveto{\pgfqpoint{2.029571in}{2.267350in}}{\pgfqpoint{2.034871in}{2.269545in}}{\pgfqpoint{2.038778in}{2.273452in}}%
\pgfpathcurveto{\pgfqpoint{2.042685in}{2.277359in}}{\pgfqpoint{2.044880in}{2.282659in}}{\pgfqpoint{2.044880in}{2.288184in}}%
\pgfpathcurveto{\pgfqpoint{2.044880in}{2.293709in}}{\pgfqpoint{2.042685in}{2.299008in}}{\pgfqpoint{2.038778in}{2.302915in}}%
\pgfpathcurveto{\pgfqpoint{2.034871in}{2.306822in}}{\pgfqpoint{2.029571in}{2.309017in}}{\pgfqpoint{2.024046in}{2.309017in}}%
\pgfpathcurveto{\pgfqpoint{2.018521in}{2.309017in}}{\pgfqpoint{2.013222in}{2.306822in}}{\pgfqpoint{2.009315in}{2.302915in}}%
\pgfpathcurveto{\pgfqpoint{2.005408in}{2.299008in}}{\pgfqpoint{2.003213in}{2.293709in}}{\pgfqpoint{2.003213in}{2.288184in}}%
\pgfpathcurveto{\pgfqpoint{2.003213in}{2.282659in}}{\pgfqpoint{2.005408in}{2.277359in}}{\pgfqpoint{2.009315in}{2.273452in}}%
\pgfpathcurveto{\pgfqpoint{2.013222in}{2.269545in}}{\pgfqpoint{2.018521in}{2.267350in}}{\pgfqpoint{2.024046in}{2.267350in}}%
\pgfpathclose%
\pgfusepath{fill}%
\end{pgfscope}%
\begin{pgfscope}%
\pgfpathrectangle{\pgfqpoint{0.889225in}{1.668832in}}{\pgfqpoint{1.162500in}{0.755000in}}%
\pgfusepath{clip}%
\pgfsetbuttcap%
\pgfsetroundjoin%
\definecolor{currentfill}{rgb}{0.000000,0.000000,0.000000}%
\pgfsetfillcolor{currentfill}%
\pgfsetfillopacity{0.500000}%
\pgfsetlinewidth{0.000000pt}%
\definecolor{currentstroke}{rgb}{0.000000,0.000000,0.000000}%
\pgfsetstrokecolor{currentstroke}%
\pgfsetdash{}{0pt}%
\pgfpathmoveto{\pgfqpoint{1.402285in}{2.236796in}}%
\pgfpathcurveto{\pgfqpoint{1.407810in}{2.236796in}}{\pgfqpoint{1.413110in}{2.238991in}}{\pgfqpoint{1.417016in}{2.242898in}}%
\pgfpathcurveto{\pgfqpoint{1.420923in}{2.246805in}}{\pgfqpoint{1.423118in}{2.252104in}}{\pgfqpoint{1.423118in}{2.257629in}}%
\pgfpathcurveto{\pgfqpoint{1.423118in}{2.263154in}}{\pgfqpoint{1.420923in}{2.268454in}}{\pgfqpoint{1.417016in}{2.272361in}}%
\pgfpathcurveto{\pgfqpoint{1.413110in}{2.276267in}}{\pgfqpoint{1.407810in}{2.278463in}}{\pgfqpoint{1.402285in}{2.278463in}}%
\pgfpathcurveto{\pgfqpoint{1.396760in}{2.278463in}}{\pgfqpoint{1.391460in}{2.276267in}}{\pgfqpoint{1.387554in}{2.272361in}}%
\pgfpathcurveto{\pgfqpoint{1.383647in}{2.268454in}}{\pgfqpoint{1.381452in}{2.263154in}}{\pgfqpoint{1.381452in}{2.257629in}}%
\pgfpathcurveto{\pgfqpoint{1.381452in}{2.252104in}}{\pgfqpoint{1.383647in}{2.246805in}}{\pgfqpoint{1.387554in}{2.242898in}}%
\pgfpathcurveto{\pgfqpoint{1.391460in}{2.238991in}}{\pgfqpoint{1.396760in}{2.236796in}}{\pgfqpoint{1.402285in}{2.236796in}}%
\pgfpathclose%
\pgfusepath{fill}%
\end{pgfscope}%
\begin{pgfscope}%
\pgfpathrectangle{\pgfqpoint{0.889225in}{1.668832in}}{\pgfqpoint{1.162500in}{0.755000in}}%
\pgfusepath{clip}%
\pgfsetbuttcap%
\pgfsetroundjoin%
\definecolor{currentfill}{rgb}{0.000000,0.000000,0.000000}%
\pgfsetfillcolor{currentfill}%
\pgfsetfillopacity{0.500000}%
\pgfsetlinewidth{0.000000pt}%
\definecolor{currentstroke}{rgb}{0.000000,0.000000,0.000000}%
\pgfsetstrokecolor{currentstroke}%
\pgfsetdash{}{0pt}%
\pgfpathmoveto{\pgfqpoint{0.973483in}{2.023720in}}%
\pgfpathcurveto{\pgfqpoint{0.979008in}{2.023720in}}{\pgfqpoint{0.984308in}{2.025915in}}{\pgfqpoint{0.988214in}{2.029822in}}%
\pgfpathcurveto{\pgfqpoint{0.992121in}{2.033729in}}{\pgfqpoint{0.994316in}{2.039028in}}{\pgfqpoint{0.994316in}{2.044553in}}%
\pgfpathcurveto{\pgfqpoint{0.994316in}{2.050078in}}{\pgfqpoint{0.992121in}{2.055378in}}{\pgfqpoint{0.988214in}{2.059285in}}%
\pgfpathcurveto{\pgfqpoint{0.984308in}{2.063191in}}{\pgfqpoint{0.979008in}{2.065387in}}{\pgfqpoint{0.973483in}{2.065387in}}%
\pgfpathcurveto{\pgfqpoint{0.967958in}{2.065387in}}{\pgfqpoint{0.962658in}{2.063191in}}{\pgfqpoint{0.958752in}{2.059285in}}%
\pgfpathcurveto{\pgfqpoint{0.954845in}{2.055378in}}{\pgfqpoint{0.952650in}{2.050078in}}{\pgfqpoint{0.952650in}{2.044553in}}%
\pgfpathcurveto{\pgfqpoint{0.952650in}{2.039028in}}{\pgfqpoint{0.954845in}{2.033729in}}{\pgfqpoint{0.958752in}{2.029822in}}%
\pgfpathcurveto{\pgfqpoint{0.962658in}{2.025915in}}{\pgfqpoint{0.967958in}{2.023720in}}{\pgfqpoint{0.973483in}{2.023720in}}%
\pgfpathclose%
\pgfusepath{fill}%
\end{pgfscope}%
\begin{pgfscope}%
\pgfsetbuttcap%
\pgfsetroundjoin%
\definecolor{currentfill}{rgb}{0.000000,0.000000,0.000000}%
\pgfsetfillcolor{currentfill}%
\pgfsetlinewidth{0.803000pt}%
\definecolor{currentstroke}{rgb}{0.000000,0.000000,0.000000}%
\pgfsetstrokecolor{currentstroke}%
\pgfsetdash{}{0pt}%
\pgfsys@defobject{currentmarker}{\pgfqpoint{-0.048611in}{0.000000in}}{\pgfqpoint{0.000000in}{0.000000in}}{%
\pgfpathmoveto{\pgfqpoint{0.000000in}{0.000000in}}%
\pgfpathlineto{\pgfqpoint{-0.048611in}{0.000000in}}%
\pgfusepath{stroke,fill}%
}%
\begin{pgfscope}%
\pgfsys@transformshift{0.889225in}{1.751846in}%
\pgfsys@useobject{currentmarker}{}%
\end{pgfscope}%
\end{pgfscope}%
\begin{pgfscope}%
\pgftext[x=0.523095in,y=1.709636in,left,base]{\rmfamily\fontsize{8.000000}{9.600000}\selectfont \(\displaystyle 0.025\)}%
\end{pgfscope}%
\begin{pgfscope}%
\pgfsetbuttcap%
\pgfsetroundjoin%
\definecolor{currentfill}{rgb}{0.000000,0.000000,0.000000}%
\pgfsetfillcolor{currentfill}%
\pgfsetlinewidth{0.803000pt}%
\definecolor{currentstroke}{rgb}{0.000000,0.000000,0.000000}%
\pgfsetstrokecolor{currentstroke}%
\pgfsetdash{}{0pt}%
\pgfsys@defobject{currentmarker}{\pgfqpoint{-0.048611in}{0.000000in}}{\pgfqpoint{0.000000in}{0.000000in}}{%
\pgfpathmoveto{\pgfqpoint{0.000000in}{0.000000in}}%
\pgfpathlineto{\pgfqpoint{-0.048611in}{0.000000in}}%
\pgfusepath{stroke,fill}%
}%
\begin{pgfscope}%
\pgfsys@transformshift{0.889225in}{2.018356in}%
\pgfsys@useobject{currentmarker}{}%
\end{pgfscope}%
\end{pgfscope}%
\begin{pgfscope}%
\pgftext[x=0.523095in,y=1.976147in,left,base]{\rmfamily\fontsize{8.000000}{9.600000}\selectfont \(\displaystyle 0.050\)}%
\end{pgfscope}%
\begin{pgfscope}%
\pgfsetbuttcap%
\pgfsetroundjoin%
\definecolor{currentfill}{rgb}{0.000000,0.000000,0.000000}%
\pgfsetfillcolor{currentfill}%
\pgfsetlinewidth{0.803000pt}%
\definecolor{currentstroke}{rgb}{0.000000,0.000000,0.000000}%
\pgfsetstrokecolor{currentstroke}%
\pgfsetdash{}{0pt}%
\pgfsys@defobject{currentmarker}{\pgfqpoint{-0.048611in}{0.000000in}}{\pgfqpoint{0.000000in}{0.000000in}}{%
\pgfpathmoveto{\pgfqpoint{0.000000in}{0.000000in}}%
\pgfpathlineto{\pgfqpoint{-0.048611in}{0.000000in}}%
\pgfusepath{stroke,fill}%
}%
\begin{pgfscope}%
\pgfsys@transformshift{0.889225in}{2.284867in}%
\pgfsys@useobject{currentmarker}{}%
\end{pgfscope}%
\end{pgfscope}%
\begin{pgfscope}%
\pgftext[x=0.523095in,y=2.242658in,left,base]{\rmfamily\fontsize{8.000000}{9.600000}\selectfont \(\displaystyle 0.075\)}%
\end{pgfscope}%
\begin{pgfscope}%
\pgftext[x=0.467539in,y=2.046332in,,bottom,rotate=90.000000]{\rmfamily\fontsize{16.000000}{19.200000}\selectfont u0}%
\end{pgfscope}%
\begin{pgfscope}%
\pgfsetrectcap%
\pgfsetmiterjoin%
\pgfsetlinewidth{0.803000pt}%
\definecolor{currentstroke}{rgb}{0.501961,0.501961,0.501961}%
\pgfsetstrokecolor{currentstroke}%
\pgfsetdash{}{0pt}%
\pgfpathmoveto{\pgfqpoint{0.889225in}{1.668832in}}%
\pgfpathlineto{\pgfqpoint{0.889225in}{2.423832in}}%
\pgfusepath{stroke}%
\end{pgfscope}%
\begin{pgfscope}%
\pgfsetrectcap%
\pgfsetmiterjoin%
\pgfsetlinewidth{0.803000pt}%
\definecolor{currentstroke}{rgb}{0.501961,0.501961,0.501961}%
\pgfsetstrokecolor{currentstroke}%
\pgfsetdash{}{0pt}%
\pgfpathmoveto{\pgfqpoint{2.051725in}{1.668832in}}%
\pgfpathlineto{\pgfqpoint{2.051725in}{2.423832in}}%
\pgfusepath{stroke}%
\end{pgfscope}%
\begin{pgfscope}%
\pgfsetrectcap%
\pgfsetmiterjoin%
\pgfsetlinewidth{0.803000pt}%
\definecolor{currentstroke}{rgb}{0.501961,0.501961,0.501961}%
\pgfsetstrokecolor{currentstroke}%
\pgfsetdash{}{0pt}%
\pgfpathmoveto{\pgfqpoint{0.889225in}{1.668832in}}%
\pgfpathlineto{\pgfqpoint{2.051725in}{1.668832in}}%
\pgfusepath{stroke}%
\end{pgfscope}%
\begin{pgfscope}%
\pgfsetrectcap%
\pgfsetmiterjoin%
\pgfsetlinewidth{0.803000pt}%
\definecolor{currentstroke}{rgb}{0.501961,0.501961,0.501961}%
\pgfsetstrokecolor{currentstroke}%
\pgfsetdash{}{0pt}%
\pgfpathmoveto{\pgfqpoint{0.889225in}{2.423832in}}%
\pgfpathlineto{\pgfqpoint{2.051725in}{2.423832in}}%
\pgfusepath{stroke}%
\end{pgfscope}%
\begin{pgfscope}%
\pgfsetbuttcap%
\pgfsetmiterjoin%
\definecolor{currentfill}{rgb}{1.000000,1.000000,1.000000}%
\pgfsetfillcolor{currentfill}%
\pgfsetlinewidth{0.000000pt}%
\definecolor{currentstroke}{rgb}{0.000000,0.000000,0.000000}%
\pgfsetstrokecolor{currentstroke}%
\pgfsetstrokeopacity{0.000000}%
\pgfsetdash{}{0pt}%
\pgfpathmoveto{\pgfqpoint{2.051725in}{1.668832in}}%
\pgfpathlineto{\pgfqpoint{3.214225in}{1.668832in}}%
\pgfpathlineto{\pgfqpoint{3.214225in}{2.423832in}}%
\pgfpathlineto{\pgfqpoint{2.051725in}{2.423832in}}%
\pgfpathclose%
\pgfusepath{fill}%
\end{pgfscope}%
\begin{pgfscope}%
\pgfpathrectangle{\pgfqpoint{2.051725in}{1.668832in}}{\pgfqpoint{1.162500in}{0.755000in}}%
\pgfusepath{clip}%
\pgfsetbuttcap%
\pgfsetroundjoin%
\definecolor{currentfill}{rgb}{0.000000,0.000000,0.000000}%
\pgfsetfillcolor{currentfill}%
\pgfsetfillopacity{0.500000}%
\pgfsetlinewidth{0.000000pt}%
\definecolor{currentstroke}{rgb}{0.000000,0.000000,0.000000}%
\pgfsetstrokecolor{currentstroke}%
\pgfsetdash{}{0pt}%
\pgfpathmoveto{\pgfqpoint{3.107569in}{1.935152in}}%
\pgfpathcurveto{\pgfqpoint{3.113094in}{1.935152in}}{\pgfqpoint{3.118394in}{1.937347in}}{\pgfqpoint{3.122301in}{1.941254in}}%
\pgfpathcurveto{\pgfqpoint{3.126207in}{1.945161in}}{\pgfqpoint{3.128403in}{1.950460in}}{\pgfqpoint{3.128403in}{1.955985in}}%
\pgfpathcurveto{\pgfqpoint{3.128403in}{1.961510in}}{\pgfqpoint{3.126207in}{1.966810in}}{\pgfqpoint{3.122301in}{1.970717in}}%
\pgfpathcurveto{\pgfqpoint{3.118394in}{1.974623in}}{\pgfqpoint{3.113094in}{1.976819in}}{\pgfqpoint{3.107569in}{1.976819in}}%
\pgfpathcurveto{\pgfqpoint{3.102044in}{1.976819in}}{\pgfqpoint{3.096745in}{1.974623in}}{\pgfqpoint{3.092838in}{1.970717in}}%
\pgfpathcurveto{\pgfqpoint{3.088931in}{1.966810in}}{\pgfqpoint{3.086736in}{1.961510in}}{\pgfqpoint{3.086736in}{1.955985in}}%
\pgfpathcurveto{\pgfqpoint{3.086736in}{1.950460in}}{\pgfqpoint{3.088931in}{1.945161in}}{\pgfqpoint{3.092838in}{1.941254in}}%
\pgfpathcurveto{\pgfqpoint{3.096745in}{1.937347in}}{\pgfqpoint{3.102044in}{1.935152in}}{\pgfqpoint{3.107569in}{1.935152in}}%
\pgfpathclose%
\pgfusepath{fill}%
\end{pgfscope}%
\begin{pgfscope}%
\pgfpathrectangle{\pgfqpoint{2.051725in}{1.668832in}}{\pgfqpoint{1.162500in}{0.755000in}}%
\pgfusepath{clip}%
\pgfsetbuttcap%
\pgfsetroundjoin%
\definecolor{currentfill}{rgb}{0.000000,0.000000,0.000000}%
\pgfsetfillcolor{currentfill}%
\pgfsetfillopacity{0.500000}%
\pgfsetlinewidth{0.000000pt}%
\definecolor{currentstroke}{rgb}{0.000000,0.000000,0.000000}%
\pgfsetstrokecolor{currentstroke}%
\pgfsetdash{}{0pt}%
\pgfpathmoveto{\pgfqpoint{2.270173in}{2.326950in}}%
\pgfpathcurveto{\pgfqpoint{2.275698in}{2.326950in}}{\pgfqpoint{2.280997in}{2.329145in}}{\pgfqpoint{2.284904in}{2.333052in}}%
\pgfpathcurveto{\pgfqpoint{2.288811in}{2.336959in}}{\pgfqpoint{2.291006in}{2.342258in}}{\pgfqpoint{2.291006in}{2.347783in}}%
\pgfpathcurveto{\pgfqpoint{2.291006in}{2.353308in}}{\pgfqpoint{2.288811in}{2.358608in}}{\pgfqpoint{2.284904in}{2.362515in}}%
\pgfpathcurveto{\pgfqpoint{2.280997in}{2.366421in}}{\pgfqpoint{2.275698in}{2.368616in}}{\pgfqpoint{2.270173in}{2.368616in}}%
\pgfpathcurveto{\pgfqpoint{2.264648in}{2.368616in}}{\pgfqpoint{2.259348in}{2.366421in}}{\pgfqpoint{2.255441in}{2.362515in}}%
\pgfpathcurveto{\pgfqpoint{2.251535in}{2.358608in}}{\pgfqpoint{2.249339in}{2.353308in}}{\pgfqpoint{2.249339in}{2.347783in}}%
\pgfpathcurveto{\pgfqpoint{2.249339in}{2.342258in}}{\pgfqpoint{2.251535in}{2.336959in}}{\pgfqpoint{2.255441in}{2.333052in}}%
\pgfpathcurveto{\pgfqpoint{2.259348in}{2.329145in}}{\pgfqpoint{2.264648in}{2.326950in}}{\pgfqpoint{2.270173in}{2.326950in}}%
\pgfpathclose%
\pgfusepath{fill}%
\end{pgfscope}%
\begin{pgfscope}%
\pgfpathrectangle{\pgfqpoint{2.051725in}{1.668832in}}{\pgfqpoint{1.162500in}{0.755000in}}%
\pgfusepath{clip}%
\pgfsetbuttcap%
\pgfsetroundjoin%
\definecolor{currentfill}{rgb}{0.000000,0.000000,0.000000}%
\pgfsetfillcolor{currentfill}%
\pgfsetfillopacity{0.500000}%
\pgfsetlinewidth{0.000000pt}%
\definecolor{currentstroke}{rgb}{0.000000,0.000000,0.000000}%
\pgfsetstrokecolor{currentstroke}%
\pgfsetdash{}{0pt}%
\pgfpathmoveto{\pgfqpoint{2.264867in}{2.385023in}}%
\pgfpathcurveto{\pgfqpoint{2.270392in}{2.385023in}}{\pgfqpoint{2.275692in}{2.387218in}}{\pgfqpoint{2.279599in}{2.391125in}}%
\pgfpathcurveto{\pgfqpoint{2.283506in}{2.395032in}}{\pgfqpoint{2.285701in}{2.400331in}}{\pgfqpoint{2.285701in}{2.405856in}}%
\pgfpathcurveto{\pgfqpoint{2.285701in}{2.411381in}}{\pgfqpoint{2.283506in}{2.416681in}}{\pgfqpoint{2.279599in}{2.420588in}}%
\pgfpathcurveto{\pgfqpoint{2.275692in}{2.424494in}}{\pgfqpoint{2.270392in}{2.426690in}}{\pgfqpoint{2.264867in}{2.426690in}}%
\pgfpathcurveto{\pgfqpoint{2.259342in}{2.426690in}}{\pgfqpoint{2.254043in}{2.424494in}}{\pgfqpoint{2.250136in}{2.420588in}}%
\pgfpathcurveto{\pgfqpoint{2.246229in}{2.416681in}}{\pgfqpoint{2.244034in}{2.411381in}}{\pgfqpoint{2.244034in}{2.405856in}}%
\pgfpathcurveto{\pgfqpoint{2.244034in}{2.400331in}}{\pgfqpoint{2.246229in}{2.395032in}}{\pgfqpoint{2.250136in}{2.391125in}}%
\pgfpathcurveto{\pgfqpoint{2.254043in}{2.387218in}}{\pgfqpoint{2.259342in}{2.385023in}}{\pgfqpoint{2.264867in}{2.385023in}}%
\pgfpathclose%
\pgfusepath{fill}%
\end{pgfscope}%
\begin{pgfscope}%
\pgfpathrectangle{\pgfqpoint{2.051725in}{1.668832in}}{\pgfqpoint{1.162500in}{0.755000in}}%
\pgfusepath{clip}%
\pgfsetbuttcap%
\pgfsetroundjoin%
\definecolor{currentfill}{rgb}{0.000000,0.000000,0.000000}%
\pgfsetfillcolor{currentfill}%
\pgfsetfillopacity{0.500000}%
\pgfsetlinewidth{0.000000pt}%
\definecolor{currentstroke}{rgb}{0.000000,0.000000,0.000000}%
\pgfsetstrokecolor{currentstroke}%
\pgfsetdash{}{0pt}%
\pgfpathmoveto{\pgfqpoint{2.154193in}{2.103661in}}%
\pgfpathcurveto{\pgfqpoint{2.159718in}{2.103661in}}{\pgfqpoint{2.165018in}{2.105856in}}{\pgfqpoint{2.168924in}{2.109762in}}%
\pgfpathcurveto{\pgfqpoint{2.172831in}{2.113669in}}{\pgfqpoint{2.175026in}{2.118969in}}{\pgfqpoint{2.175026in}{2.124494in}}%
\pgfpathcurveto{\pgfqpoint{2.175026in}{2.130019in}}{\pgfqpoint{2.172831in}{2.135318in}}{\pgfqpoint{2.168924in}{2.139225in}}%
\pgfpathcurveto{\pgfqpoint{2.165018in}{2.143132in}}{\pgfqpoint{2.159718in}{2.145327in}}{\pgfqpoint{2.154193in}{2.145327in}}%
\pgfpathcurveto{\pgfqpoint{2.148668in}{2.145327in}}{\pgfqpoint{2.143368in}{2.143132in}}{\pgfqpoint{2.139462in}{2.139225in}}%
\pgfpathcurveto{\pgfqpoint{2.135555in}{2.135318in}}{\pgfqpoint{2.133360in}{2.130019in}}{\pgfqpoint{2.133360in}{2.124494in}}%
\pgfpathcurveto{\pgfqpoint{2.133360in}{2.118969in}}{\pgfqpoint{2.135555in}{2.113669in}}{\pgfqpoint{2.139462in}{2.109762in}}%
\pgfpathcurveto{\pgfqpoint{2.143368in}{2.105856in}}{\pgfqpoint{2.148668in}{2.103661in}}{\pgfqpoint{2.154193in}{2.103661in}}%
\pgfpathclose%
\pgfusepath{fill}%
\end{pgfscope}%
\begin{pgfscope}%
\pgfpathrectangle{\pgfqpoint{2.051725in}{1.668832in}}{\pgfqpoint{1.162500in}{0.755000in}}%
\pgfusepath{clip}%
\pgfsetbuttcap%
\pgfsetroundjoin%
\definecolor{currentfill}{rgb}{0.000000,0.000000,0.000000}%
\pgfsetfillcolor{currentfill}%
\pgfsetfillopacity{0.500000}%
\pgfsetlinewidth{0.000000pt}%
\definecolor{currentstroke}{rgb}{0.000000,0.000000,0.000000}%
\pgfsetstrokecolor{currentstroke}%
\pgfsetdash{}{0pt}%
\pgfpathmoveto{\pgfqpoint{2.183344in}{2.149550in}}%
\pgfpathcurveto{\pgfqpoint{2.188869in}{2.149550in}}{\pgfqpoint{2.194169in}{2.151746in}}{\pgfqpoint{2.198076in}{2.155652in}}%
\pgfpathcurveto{\pgfqpoint{2.201982in}{2.159559in}}{\pgfqpoint{2.204178in}{2.164859in}}{\pgfqpoint{2.204178in}{2.170384in}}%
\pgfpathcurveto{\pgfqpoint{2.204178in}{2.175909in}}{\pgfqpoint{2.201982in}{2.181208in}}{\pgfqpoint{2.198076in}{2.185115in}}%
\pgfpathcurveto{\pgfqpoint{2.194169in}{2.189022in}}{\pgfqpoint{2.188869in}{2.191217in}}{\pgfqpoint{2.183344in}{2.191217in}}%
\pgfpathcurveto{\pgfqpoint{2.177819in}{2.191217in}}{\pgfqpoint{2.172520in}{2.189022in}}{\pgfqpoint{2.168613in}{2.185115in}}%
\pgfpathcurveto{\pgfqpoint{2.164706in}{2.181208in}}{\pgfqpoint{2.162511in}{2.175909in}}{\pgfqpoint{2.162511in}{2.170384in}}%
\pgfpathcurveto{\pgfqpoint{2.162511in}{2.164859in}}{\pgfqpoint{2.164706in}{2.159559in}}{\pgfqpoint{2.168613in}{2.155652in}}%
\pgfpathcurveto{\pgfqpoint{2.172520in}{2.151746in}}{\pgfqpoint{2.177819in}{2.149550in}}{\pgfqpoint{2.183344in}{2.149550in}}%
\pgfpathclose%
\pgfusepath{fill}%
\end{pgfscope}%
\begin{pgfscope}%
\pgfpathrectangle{\pgfqpoint{2.051725in}{1.668832in}}{\pgfqpoint{1.162500in}{0.755000in}}%
\pgfusepath{clip}%
\pgfsetbuttcap%
\pgfsetroundjoin%
\definecolor{currentfill}{rgb}{0.000000,0.000000,0.000000}%
\pgfsetfillcolor{currentfill}%
\pgfsetfillopacity{0.500000}%
\pgfsetlinewidth{0.000000pt}%
\definecolor{currentstroke}{rgb}{0.000000,0.000000,0.000000}%
\pgfsetstrokecolor{currentstroke}%
\pgfsetdash{}{0pt}%
\pgfpathmoveto{\pgfqpoint{2.082232in}{2.049204in}}%
\pgfpathcurveto{\pgfqpoint{2.087757in}{2.049204in}}{\pgfqpoint{2.093056in}{2.051399in}}{\pgfqpoint{2.096963in}{2.055306in}}%
\pgfpathcurveto{\pgfqpoint{2.100870in}{2.059212in}}{\pgfqpoint{2.103065in}{2.064512in}}{\pgfqpoint{2.103065in}{2.070037in}}%
\pgfpathcurveto{\pgfqpoint{2.103065in}{2.075562in}}{\pgfqpoint{2.100870in}{2.080862in}}{\pgfqpoint{2.096963in}{2.084768in}}%
\pgfpathcurveto{\pgfqpoint{2.093056in}{2.088675in}}{\pgfqpoint{2.087757in}{2.090870in}}{\pgfqpoint{2.082232in}{2.090870in}}%
\pgfpathcurveto{\pgfqpoint{2.076707in}{2.090870in}}{\pgfqpoint{2.071407in}{2.088675in}}{\pgfqpoint{2.067500in}{2.084768in}}%
\pgfpathcurveto{\pgfqpoint{2.063594in}{2.080862in}}{\pgfqpoint{2.061398in}{2.075562in}}{\pgfqpoint{2.061398in}{2.070037in}}%
\pgfpathcurveto{\pgfqpoint{2.061398in}{2.064512in}}{\pgfqpoint{2.063594in}{2.059212in}}{\pgfqpoint{2.067500in}{2.055306in}}%
\pgfpathcurveto{\pgfqpoint{2.071407in}{2.051399in}}{\pgfqpoint{2.076707in}{2.049204in}}{\pgfqpoint{2.082232in}{2.049204in}}%
\pgfpathclose%
\pgfusepath{fill}%
\end{pgfscope}%
\begin{pgfscope}%
\pgfpathrectangle{\pgfqpoint{2.051725in}{1.668832in}}{\pgfqpoint{1.162500in}{0.755000in}}%
\pgfusepath{clip}%
\pgfsetbuttcap%
\pgfsetroundjoin%
\definecolor{currentfill}{rgb}{0.000000,0.000000,0.000000}%
\pgfsetfillcolor{currentfill}%
\pgfsetfillopacity{0.500000}%
\pgfsetlinewidth{0.000000pt}%
\definecolor{currentstroke}{rgb}{0.000000,0.000000,0.000000}%
\pgfsetstrokecolor{currentstroke}%
\pgfsetdash{}{0pt}%
\pgfpathmoveto{\pgfqpoint{2.079404in}{1.665975in}}%
\pgfpathcurveto{\pgfqpoint{2.084929in}{1.665975in}}{\pgfqpoint{2.090228in}{1.668170in}}{\pgfqpoint{2.094135in}{1.672077in}}%
\pgfpathcurveto{\pgfqpoint{2.098042in}{1.675984in}}{\pgfqpoint{2.100237in}{1.681283in}}{\pgfqpoint{2.100237in}{1.686809in}}%
\pgfpathcurveto{\pgfqpoint{2.100237in}{1.692334in}}{\pgfqpoint{2.098042in}{1.697633in}}{\pgfqpoint{2.094135in}{1.701540in}}%
\pgfpathcurveto{\pgfqpoint{2.090228in}{1.705447in}}{\pgfqpoint{2.084929in}{1.707642in}}{\pgfqpoint{2.079404in}{1.707642in}}%
\pgfpathcurveto{\pgfqpoint{2.073879in}{1.707642in}}{\pgfqpoint{2.068579in}{1.705447in}}{\pgfqpoint{2.064672in}{1.701540in}}%
\pgfpathcurveto{\pgfqpoint{2.060765in}{1.697633in}}{\pgfqpoint{2.058570in}{1.692334in}}{\pgfqpoint{2.058570in}{1.686809in}}%
\pgfpathcurveto{\pgfqpoint{2.058570in}{1.681283in}}{\pgfqpoint{2.060765in}{1.675984in}}{\pgfqpoint{2.064672in}{1.672077in}}%
\pgfpathcurveto{\pgfqpoint{2.068579in}{1.668170in}}{\pgfqpoint{2.073879in}{1.665975in}}{\pgfqpoint{2.079404in}{1.665975in}}%
\pgfpathclose%
\pgfusepath{fill}%
\end{pgfscope}%
\begin{pgfscope}%
\pgfpathrectangle{\pgfqpoint{2.051725in}{1.668832in}}{\pgfqpoint{1.162500in}{0.755000in}}%
\pgfusepath{clip}%
\pgfsetbuttcap%
\pgfsetroundjoin%
\definecolor{currentfill}{rgb}{0.000000,0.000000,0.000000}%
\pgfsetfillcolor{currentfill}%
\pgfsetfillopacity{0.500000}%
\pgfsetlinewidth{0.000000pt}%
\definecolor{currentstroke}{rgb}{0.000000,0.000000,0.000000}%
\pgfsetstrokecolor{currentstroke}%
\pgfsetdash{}{0pt}%
\pgfpathmoveto{\pgfqpoint{2.536683in}{2.301682in}}%
\pgfpathcurveto{\pgfqpoint{2.542208in}{2.301682in}}{\pgfqpoint{2.547508in}{2.303877in}}{\pgfqpoint{2.551415in}{2.307784in}}%
\pgfpathcurveto{\pgfqpoint{2.555322in}{2.311691in}}{\pgfqpoint{2.557517in}{2.316990in}}{\pgfqpoint{2.557517in}{2.322515in}}%
\pgfpathcurveto{\pgfqpoint{2.557517in}{2.328041in}}{\pgfqpoint{2.555322in}{2.333340in}}{\pgfqpoint{2.551415in}{2.337247in}}%
\pgfpathcurveto{\pgfqpoint{2.547508in}{2.341154in}}{\pgfqpoint{2.542208in}{2.343349in}}{\pgfqpoint{2.536683in}{2.343349in}}%
\pgfpathcurveto{\pgfqpoint{2.531158in}{2.343349in}}{\pgfqpoint{2.525859in}{2.341154in}}{\pgfqpoint{2.521952in}{2.337247in}}%
\pgfpathcurveto{\pgfqpoint{2.518045in}{2.333340in}}{\pgfqpoint{2.515850in}{2.328041in}}{\pgfqpoint{2.515850in}{2.322515in}}%
\pgfpathcurveto{\pgfqpoint{2.515850in}{2.316990in}}{\pgfqpoint{2.518045in}{2.311691in}}{\pgfqpoint{2.521952in}{2.307784in}}%
\pgfpathcurveto{\pgfqpoint{2.525859in}{2.303877in}}{\pgfqpoint{2.531158in}{2.301682in}}{\pgfqpoint{2.536683in}{2.301682in}}%
\pgfpathclose%
\pgfusepath{fill}%
\end{pgfscope}%
\begin{pgfscope}%
\pgfpathrectangle{\pgfqpoint{2.051725in}{1.668832in}}{\pgfqpoint{1.162500in}{0.755000in}}%
\pgfusepath{clip}%
\pgfsetbuttcap%
\pgfsetroundjoin%
\definecolor{currentfill}{rgb}{0.000000,0.000000,0.000000}%
\pgfsetfillcolor{currentfill}%
\pgfsetfillopacity{0.500000}%
\pgfsetlinewidth{0.000000pt}%
\definecolor{currentstroke}{rgb}{0.000000,0.000000,0.000000}%
\pgfsetstrokecolor{currentstroke}%
\pgfsetdash{}{0pt}%
\pgfpathmoveto{\pgfqpoint{2.417381in}{2.145886in}}%
\pgfpathcurveto{\pgfqpoint{2.422906in}{2.145886in}}{\pgfqpoint{2.428205in}{2.148081in}}{\pgfqpoint{2.432112in}{2.151988in}}%
\pgfpathcurveto{\pgfqpoint{2.436019in}{2.155894in}}{\pgfqpoint{2.438214in}{2.161194in}}{\pgfqpoint{2.438214in}{2.166719in}}%
\pgfpathcurveto{\pgfqpoint{2.438214in}{2.172244in}}{\pgfqpoint{2.436019in}{2.177544in}}{\pgfqpoint{2.432112in}{2.181450in}}%
\pgfpathcurveto{\pgfqpoint{2.428205in}{2.185357in}}{\pgfqpoint{2.422906in}{2.187552in}}{\pgfqpoint{2.417381in}{2.187552in}}%
\pgfpathcurveto{\pgfqpoint{2.411856in}{2.187552in}}{\pgfqpoint{2.406556in}{2.185357in}}{\pgfqpoint{2.402649in}{2.181450in}}%
\pgfpathcurveto{\pgfqpoint{2.398743in}{2.177544in}}{\pgfqpoint{2.396547in}{2.172244in}}{\pgfqpoint{2.396547in}{2.166719in}}%
\pgfpathcurveto{\pgfqpoint{2.396547in}{2.161194in}}{\pgfqpoint{2.398743in}{2.155894in}}{\pgfqpoint{2.402649in}{2.151988in}}%
\pgfpathcurveto{\pgfqpoint{2.406556in}{2.148081in}}{\pgfqpoint{2.411856in}{2.145886in}}{\pgfqpoint{2.417381in}{2.145886in}}%
\pgfpathclose%
\pgfusepath{fill}%
\end{pgfscope}%
\begin{pgfscope}%
\pgfpathrectangle{\pgfqpoint{2.051725in}{1.668832in}}{\pgfqpoint{1.162500in}{0.755000in}}%
\pgfusepath{clip}%
\pgfsetbuttcap%
\pgfsetroundjoin%
\definecolor{currentfill}{rgb}{0.000000,0.000000,0.000000}%
\pgfsetfillcolor{currentfill}%
\pgfsetfillopacity{0.500000}%
\pgfsetlinewidth{0.000000pt}%
\definecolor{currentstroke}{rgb}{0.000000,0.000000,0.000000}%
\pgfsetstrokecolor{currentstroke}%
\pgfsetdash{}{0pt}%
\pgfpathmoveto{\pgfqpoint{2.412630in}{1.969105in}}%
\pgfpathcurveto{\pgfqpoint{2.418155in}{1.969105in}}{\pgfqpoint{2.423454in}{1.971300in}}{\pgfqpoint{2.427361in}{1.975207in}}%
\pgfpathcurveto{\pgfqpoint{2.431268in}{1.979114in}}{\pgfqpoint{2.433463in}{1.984413in}}{\pgfqpoint{2.433463in}{1.989939in}}%
\pgfpathcurveto{\pgfqpoint{2.433463in}{1.995464in}}{\pgfqpoint{2.431268in}{2.000763in}}{\pgfqpoint{2.427361in}{2.004670in}}%
\pgfpathcurveto{\pgfqpoint{2.423454in}{2.008577in}}{\pgfqpoint{2.418155in}{2.010772in}}{\pgfqpoint{2.412630in}{2.010772in}}%
\pgfpathcurveto{\pgfqpoint{2.407105in}{2.010772in}}{\pgfqpoint{2.401805in}{2.008577in}}{\pgfqpoint{2.397898in}{2.004670in}}%
\pgfpathcurveto{\pgfqpoint{2.393991in}{2.000763in}}{\pgfqpoint{2.391796in}{1.995464in}}{\pgfqpoint{2.391796in}{1.989939in}}%
\pgfpathcurveto{\pgfqpoint{2.391796in}{1.984413in}}{\pgfqpoint{2.393991in}{1.979114in}}{\pgfqpoint{2.397898in}{1.975207in}}%
\pgfpathcurveto{\pgfqpoint{2.401805in}{1.971300in}}{\pgfqpoint{2.407105in}{1.969105in}}{\pgfqpoint{2.412630in}{1.969105in}}%
\pgfpathclose%
\pgfusepath{fill}%
\end{pgfscope}%
\begin{pgfscope}%
\pgfpathrectangle{\pgfqpoint{2.051725in}{1.668832in}}{\pgfqpoint{1.162500in}{0.755000in}}%
\pgfusepath{clip}%
\pgfsetbuttcap%
\pgfsetroundjoin%
\definecolor{currentfill}{rgb}{0.000000,0.000000,0.000000}%
\pgfsetfillcolor{currentfill}%
\pgfsetfillopacity{0.500000}%
\pgfsetlinewidth{0.000000pt}%
\definecolor{currentstroke}{rgb}{0.000000,0.000000,0.000000}%
\pgfsetstrokecolor{currentstroke}%
\pgfsetdash{}{0pt}%
\pgfpathmoveto{\pgfqpoint{2.478188in}{2.239104in}}%
\pgfpathcurveto{\pgfqpoint{2.483713in}{2.239104in}}{\pgfqpoint{2.489012in}{2.241300in}}{\pgfqpoint{2.492919in}{2.245206in}}%
\pgfpathcurveto{\pgfqpoint{2.496826in}{2.249113in}}{\pgfqpoint{2.499021in}{2.254413in}}{\pgfqpoint{2.499021in}{2.259938in}}%
\pgfpathcurveto{\pgfqpoint{2.499021in}{2.265463in}}{\pgfqpoint{2.496826in}{2.270762in}}{\pgfqpoint{2.492919in}{2.274669in}}%
\pgfpathcurveto{\pgfqpoint{2.489012in}{2.278576in}}{\pgfqpoint{2.483713in}{2.280771in}}{\pgfqpoint{2.478188in}{2.280771in}}%
\pgfpathcurveto{\pgfqpoint{2.472663in}{2.280771in}}{\pgfqpoint{2.467363in}{2.278576in}}{\pgfqpoint{2.463456in}{2.274669in}}%
\pgfpathcurveto{\pgfqpoint{2.459549in}{2.270762in}}{\pgfqpoint{2.457354in}{2.265463in}}{\pgfqpoint{2.457354in}{2.259938in}}%
\pgfpathcurveto{\pgfqpoint{2.457354in}{2.254413in}}{\pgfqpoint{2.459549in}{2.249113in}}{\pgfqpoint{2.463456in}{2.245206in}}%
\pgfpathcurveto{\pgfqpoint{2.467363in}{2.241300in}}{\pgfqpoint{2.472663in}{2.239104in}}{\pgfqpoint{2.478188in}{2.239104in}}%
\pgfpathclose%
\pgfusepath{fill}%
\end{pgfscope}%
\begin{pgfscope}%
\pgfpathrectangle{\pgfqpoint{2.051725in}{1.668832in}}{\pgfqpoint{1.162500in}{0.755000in}}%
\pgfusepath{clip}%
\pgfsetbuttcap%
\pgfsetroundjoin%
\definecolor{currentfill}{rgb}{0.000000,0.000000,0.000000}%
\pgfsetfillcolor{currentfill}%
\pgfsetfillopacity{0.500000}%
\pgfsetlinewidth{0.000000pt}%
\definecolor{currentstroke}{rgb}{0.000000,0.000000,0.000000}%
\pgfsetstrokecolor{currentstroke}%
\pgfsetdash{}{0pt}%
\pgfpathmoveto{\pgfqpoint{2.172291in}{1.960666in}}%
\pgfpathcurveto{\pgfqpoint{2.177816in}{1.960666in}}{\pgfqpoint{2.183116in}{1.962861in}}{\pgfqpoint{2.187022in}{1.966768in}}%
\pgfpathcurveto{\pgfqpoint{2.190929in}{1.970675in}}{\pgfqpoint{2.193124in}{1.975974in}}{\pgfqpoint{2.193124in}{1.981499in}}%
\pgfpathcurveto{\pgfqpoint{2.193124in}{1.987024in}}{\pgfqpoint{2.190929in}{1.992324in}}{\pgfqpoint{2.187022in}{1.996231in}}%
\pgfpathcurveto{\pgfqpoint{2.183116in}{2.000138in}}{\pgfqpoint{2.177816in}{2.002333in}}{\pgfqpoint{2.172291in}{2.002333in}}%
\pgfpathcurveto{\pgfqpoint{2.166766in}{2.002333in}}{\pgfqpoint{2.161467in}{2.000138in}}{\pgfqpoint{2.157560in}{1.996231in}}%
\pgfpathcurveto{\pgfqpoint{2.153653in}{1.992324in}}{\pgfqpoint{2.151458in}{1.987024in}}{\pgfqpoint{2.151458in}{1.981499in}}%
\pgfpathcurveto{\pgfqpoint{2.151458in}{1.975974in}}{\pgfqpoint{2.153653in}{1.970675in}}{\pgfqpoint{2.157560in}{1.966768in}}%
\pgfpathcurveto{\pgfqpoint{2.161467in}{1.962861in}}{\pgfqpoint{2.166766in}{1.960666in}}{\pgfqpoint{2.172291in}{1.960666in}}%
\pgfpathclose%
\pgfusepath{fill}%
\end{pgfscope}%
\begin{pgfscope}%
\pgfpathrectangle{\pgfqpoint{2.051725in}{1.668832in}}{\pgfqpoint{1.162500in}{0.755000in}}%
\pgfusepath{clip}%
\pgfsetbuttcap%
\pgfsetroundjoin%
\definecolor{currentfill}{rgb}{0.000000,0.000000,0.000000}%
\pgfsetfillcolor{currentfill}%
\pgfsetfillopacity{0.500000}%
\pgfsetlinewidth{0.000000pt}%
\definecolor{currentstroke}{rgb}{0.000000,0.000000,0.000000}%
\pgfsetstrokecolor{currentstroke}%
\pgfsetdash{}{0pt}%
\pgfpathmoveto{\pgfqpoint{2.205718in}{1.722959in}}%
\pgfpathcurveto{\pgfqpoint{2.211243in}{1.722959in}}{\pgfqpoint{2.216543in}{1.725154in}}{\pgfqpoint{2.220450in}{1.729061in}}%
\pgfpathcurveto{\pgfqpoint{2.224357in}{1.732968in}}{\pgfqpoint{2.226552in}{1.738267in}}{\pgfqpoint{2.226552in}{1.743792in}}%
\pgfpathcurveto{\pgfqpoint{2.226552in}{1.749317in}}{\pgfqpoint{2.224357in}{1.754617in}}{\pgfqpoint{2.220450in}{1.758523in}}%
\pgfpathcurveto{\pgfqpoint{2.216543in}{1.762430in}}{\pgfqpoint{2.211243in}{1.764625in}}{\pgfqpoint{2.205718in}{1.764625in}}%
\pgfpathcurveto{\pgfqpoint{2.200193in}{1.764625in}}{\pgfqpoint{2.194894in}{1.762430in}}{\pgfqpoint{2.190987in}{1.758523in}}%
\pgfpathcurveto{\pgfqpoint{2.187080in}{1.754617in}}{\pgfqpoint{2.184885in}{1.749317in}}{\pgfqpoint{2.184885in}{1.743792in}}%
\pgfpathcurveto{\pgfqpoint{2.184885in}{1.738267in}}{\pgfqpoint{2.187080in}{1.732968in}}{\pgfqpoint{2.190987in}{1.729061in}}%
\pgfpathcurveto{\pgfqpoint{2.194894in}{1.725154in}}{\pgfqpoint{2.200193in}{1.722959in}}{\pgfqpoint{2.205718in}{1.722959in}}%
\pgfpathclose%
\pgfusepath{fill}%
\end{pgfscope}%
\begin{pgfscope}%
\pgfpathrectangle{\pgfqpoint{2.051725in}{1.668832in}}{\pgfqpoint{1.162500in}{0.755000in}}%
\pgfusepath{clip}%
\pgfsetbuttcap%
\pgfsetroundjoin%
\definecolor{currentfill}{rgb}{0.000000,0.000000,0.000000}%
\pgfsetfillcolor{currentfill}%
\pgfsetfillopacity{0.500000}%
\pgfsetlinewidth{0.000000pt}%
\definecolor{currentstroke}{rgb}{0.000000,0.000000,0.000000}%
\pgfsetstrokecolor{currentstroke}%
\pgfsetdash{}{0pt}%
\pgfpathmoveto{\pgfqpoint{3.186546in}{2.267350in}}%
\pgfpathcurveto{\pgfqpoint{3.192071in}{2.267350in}}{\pgfqpoint{3.197371in}{2.269545in}}{\pgfqpoint{3.201278in}{2.273452in}}%
\pgfpathcurveto{\pgfqpoint{3.205185in}{2.277359in}}{\pgfqpoint{3.207380in}{2.282659in}}{\pgfqpoint{3.207380in}{2.288184in}}%
\pgfpathcurveto{\pgfqpoint{3.207380in}{2.293709in}}{\pgfqpoint{3.205185in}{2.299008in}}{\pgfqpoint{3.201278in}{2.302915in}}%
\pgfpathcurveto{\pgfqpoint{3.197371in}{2.306822in}}{\pgfqpoint{3.192071in}{2.309017in}}{\pgfqpoint{3.186546in}{2.309017in}}%
\pgfpathcurveto{\pgfqpoint{3.181021in}{2.309017in}}{\pgfqpoint{3.175722in}{2.306822in}}{\pgfqpoint{3.171815in}{2.302915in}}%
\pgfpathcurveto{\pgfqpoint{3.167908in}{2.299008in}}{\pgfqpoint{3.165713in}{2.293709in}}{\pgfqpoint{3.165713in}{2.288184in}}%
\pgfpathcurveto{\pgfqpoint{3.165713in}{2.282659in}}{\pgfqpoint{3.167908in}{2.277359in}}{\pgfqpoint{3.171815in}{2.273452in}}%
\pgfpathcurveto{\pgfqpoint{3.175722in}{2.269545in}}{\pgfqpoint{3.181021in}{2.267350in}}{\pgfqpoint{3.186546in}{2.267350in}}%
\pgfpathclose%
\pgfusepath{fill}%
\end{pgfscope}%
\begin{pgfscope}%
\pgfpathrectangle{\pgfqpoint{2.051725in}{1.668832in}}{\pgfqpoint{1.162500in}{0.755000in}}%
\pgfusepath{clip}%
\pgfsetbuttcap%
\pgfsetroundjoin%
\definecolor{currentfill}{rgb}{0.000000,0.000000,0.000000}%
\pgfsetfillcolor{currentfill}%
\pgfsetfillopacity{0.500000}%
\pgfsetlinewidth{0.000000pt}%
\definecolor{currentstroke}{rgb}{0.000000,0.000000,0.000000}%
\pgfsetstrokecolor{currentstroke}%
\pgfsetdash{}{0pt}%
\pgfpathmoveto{\pgfqpoint{2.394110in}{2.236796in}}%
\pgfpathcurveto{\pgfqpoint{2.399635in}{2.236796in}}{\pgfqpoint{2.404935in}{2.238991in}}{\pgfqpoint{2.408842in}{2.242898in}}%
\pgfpathcurveto{\pgfqpoint{2.412749in}{2.246805in}}{\pgfqpoint{2.414944in}{2.252104in}}{\pgfqpoint{2.414944in}{2.257629in}}%
\pgfpathcurveto{\pgfqpoint{2.414944in}{2.263154in}}{\pgfqpoint{2.412749in}{2.268454in}}{\pgfqpoint{2.408842in}{2.272361in}}%
\pgfpathcurveto{\pgfqpoint{2.404935in}{2.276267in}}{\pgfqpoint{2.399635in}{2.278463in}}{\pgfqpoint{2.394110in}{2.278463in}}%
\pgfpathcurveto{\pgfqpoint{2.388585in}{2.278463in}}{\pgfqpoint{2.383286in}{2.276267in}}{\pgfqpoint{2.379379in}{2.272361in}}%
\pgfpathcurveto{\pgfqpoint{2.375472in}{2.268454in}}{\pgfqpoint{2.373277in}{2.263154in}}{\pgfqpoint{2.373277in}{2.257629in}}%
\pgfpathcurveto{\pgfqpoint{2.373277in}{2.252104in}}{\pgfqpoint{2.375472in}{2.246805in}}{\pgfqpoint{2.379379in}{2.242898in}}%
\pgfpathcurveto{\pgfqpoint{2.383286in}{2.238991in}}{\pgfqpoint{2.388585in}{2.236796in}}{\pgfqpoint{2.394110in}{2.236796in}}%
\pgfpathclose%
\pgfusepath{fill}%
\end{pgfscope}%
\begin{pgfscope}%
\pgfpathrectangle{\pgfqpoint{2.051725in}{1.668832in}}{\pgfqpoint{1.162500in}{0.755000in}}%
\pgfusepath{clip}%
\pgfsetbuttcap%
\pgfsetroundjoin%
\definecolor{currentfill}{rgb}{0.000000,0.000000,0.000000}%
\pgfsetfillcolor{currentfill}%
\pgfsetfillopacity{0.500000}%
\pgfsetlinewidth{0.000000pt}%
\definecolor{currentstroke}{rgb}{0.000000,0.000000,0.000000}%
\pgfsetstrokecolor{currentstroke}%
\pgfsetdash{}{0pt}%
\pgfpathmoveto{\pgfqpoint{2.125339in}{2.023720in}}%
\pgfpathcurveto{\pgfqpoint{2.130865in}{2.023720in}}{\pgfqpoint{2.136164in}{2.025915in}}{\pgfqpoint{2.140071in}{2.029822in}}%
\pgfpathcurveto{\pgfqpoint{2.143978in}{2.033729in}}{\pgfqpoint{2.146173in}{2.039028in}}{\pgfqpoint{2.146173in}{2.044553in}}%
\pgfpathcurveto{\pgfqpoint{2.146173in}{2.050078in}}{\pgfqpoint{2.143978in}{2.055378in}}{\pgfqpoint{2.140071in}{2.059285in}}%
\pgfpathcurveto{\pgfqpoint{2.136164in}{2.063191in}}{\pgfqpoint{2.130865in}{2.065387in}}{\pgfqpoint{2.125339in}{2.065387in}}%
\pgfpathcurveto{\pgfqpoint{2.119814in}{2.065387in}}{\pgfqpoint{2.114515in}{2.063191in}}{\pgfqpoint{2.110608in}{2.059285in}}%
\pgfpathcurveto{\pgfqpoint{2.106701in}{2.055378in}}{\pgfqpoint{2.104506in}{2.050078in}}{\pgfqpoint{2.104506in}{2.044553in}}%
\pgfpathcurveto{\pgfqpoint{2.104506in}{2.039028in}}{\pgfqpoint{2.106701in}{2.033729in}}{\pgfqpoint{2.110608in}{2.029822in}}%
\pgfpathcurveto{\pgfqpoint{2.114515in}{2.025915in}}{\pgfqpoint{2.119814in}{2.023720in}}{\pgfqpoint{2.125339in}{2.023720in}}%
\pgfpathclose%
\pgfusepath{fill}%
\end{pgfscope}%
\begin{pgfscope}%
\pgfsetrectcap%
\pgfsetmiterjoin%
\pgfsetlinewidth{0.803000pt}%
\definecolor{currentstroke}{rgb}{0.501961,0.501961,0.501961}%
\pgfsetstrokecolor{currentstroke}%
\pgfsetdash{}{0pt}%
\pgfpathmoveto{\pgfqpoint{2.051725in}{1.668832in}}%
\pgfpathlineto{\pgfqpoint{2.051725in}{2.423832in}}%
\pgfusepath{stroke}%
\end{pgfscope}%
\begin{pgfscope}%
\pgfsetrectcap%
\pgfsetmiterjoin%
\pgfsetlinewidth{0.803000pt}%
\definecolor{currentstroke}{rgb}{0.501961,0.501961,0.501961}%
\pgfsetstrokecolor{currentstroke}%
\pgfsetdash{}{0pt}%
\pgfpathmoveto{\pgfqpoint{3.214225in}{1.668832in}}%
\pgfpathlineto{\pgfqpoint{3.214225in}{2.423832in}}%
\pgfusepath{stroke}%
\end{pgfscope}%
\begin{pgfscope}%
\pgfsetrectcap%
\pgfsetmiterjoin%
\pgfsetlinewidth{0.803000pt}%
\definecolor{currentstroke}{rgb}{0.501961,0.501961,0.501961}%
\pgfsetstrokecolor{currentstroke}%
\pgfsetdash{}{0pt}%
\pgfpathmoveto{\pgfqpoint{2.051725in}{1.668832in}}%
\pgfpathlineto{\pgfqpoint{3.214225in}{1.668832in}}%
\pgfusepath{stroke}%
\end{pgfscope}%
\begin{pgfscope}%
\pgfsetrectcap%
\pgfsetmiterjoin%
\pgfsetlinewidth{0.803000pt}%
\definecolor{currentstroke}{rgb}{0.501961,0.501961,0.501961}%
\pgfsetstrokecolor{currentstroke}%
\pgfsetdash{}{0pt}%
\pgfpathmoveto{\pgfqpoint{2.051725in}{2.423832in}}%
\pgfpathlineto{\pgfqpoint{3.214225in}{2.423832in}}%
\pgfusepath{stroke}%
\end{pgfscope}%
\begin{pgfscope}%
\pgfsetbuttcap%
\pgfsetmiterjoin%
\definecolor{currentfill}{rgb}{1.000000,1.000000,1.000000}%
\pgfsetfillcolor{currentfill}%
\pgfsetlinewidth{0.000000pt}%
\definecolor{currentstroke}{rgb}{0.000000,0.000000,0.000000}%
\pgfsetstrokecolor{currentstroke}%
\pgfsetstrokeopacity{0.000000}%
\pgfsetdash{}{0pt}%
\pgfpathmoveto{\pgfqpoint{3.214225in}{1.668832in}}%
\pgfpathlineto{\pgfqpoint{4.376725in}{1.668832in}}%
\pgfpathlineto{\pgfqpoint{4.376725in}{2.423832in}}%
\pgfpathlineto{\pgfqpoint{3.214225in}{2.423832in}}%
\pgfpathclose%
\pgfusepath{fill}%
\end{pgfscope}%
\begin{pgfscope}%
\pgfpathrectangle{\pgfqpoint{3.214225in}{1.668832in}}{\pgfqpoint{1.162500in}{0.755000in}}%
\pgfusepath{clip}%
\pgfsetrectcap%
\pgfsetroundjoin%
\pgfsetlinewidth{1.505625pt}%
\definecolor{currentstroke}{rgb}{0.121569,0.466667,0.705882}%
\pgfsetstrokecolor{currentstroke}%
\pgfsetdash{}{0pt}%
\pgfpathmoveto{\pgfqpoint{3.241904in}{1.703151in}}%
\pgfpathlineto{\pgfqpoint{3.274043in}{1.719497in}}%
\pgfpathlineto{\pgfqpoint{3.312832in}{1.736554in}}%
\pgfpathlineto{\pgfqpoint{3.382651in}{1.766730in}}%
\pgfpathlineto{\pgfqpoint{3.409249in}{1.781089in}}%
\pgfpathlineto{\pgfqpoint{3.432523in}{1.796220in}}%
\pgfpathlineto{\pgfqpoint{3.454688in}{1.813410in}}%
\pgfpathlineto{\pgfqpoint{3.475745in}{1.832588in}}%
\pgfpathlineto{\pgfqpoint{3.496801in}{1.854715in}}%
\pgfpathlineto{\pgfqpoint{3.518966in}{1.881209in}}%
\pgfpathlineto{\pgfqpoint{3.543348in}{1.913943in}}%
\pgfpathlineto{\pgfqpoint{3.571054in}{1.955094in}}%
\pgfpathlineto{\pgfqpoint{3.605410in}{2.010375in}}%
\pgfpathlineto{\pgfqpoint{3.705153in}{2.173572in}}%
\pgfpathlineto{\pgfqpoint{3.733967in}{2.215073in}}%
\pgfpathlineto{\pgfqpoint{3.759457in}{2.248062in}}%
\pgfpathlineto{\pgfqpoint{3.783838in}{2.276027in}}%
\pgfpathlineto{\pgfqpoint{3.807112in}{2.299374in}}%
\pgfpathlineto{\pgfqpoint{3.830385in}{2.319532in}}%
\pgfpathlineto{\pgfqpoint{3.853658in}{2.336684in}}%
\pgfpathlineto{\pgfqpoint{3.876931in}{2.351073in}}%
\pgfpathlineto{\pgfqpoint{3.901313in}{2.363461in}}%
\pgfpathlineto{\pgfqpoint{3.925695in}{2.373347in}}%
\pgfpathlineto{\pgfqpoint{3.950076in}{2.380899in}}%
\pgfpathlineto{\pgfqpoint{3.974458in}{2.386158in}}%
\pgfpathlineto{\pgfqpoint{3.997731in}{2.388954in}}%
\pgfpathlineto{\pgfqpoint{4.019896in}{2.389411in}}%
\pgfpathlineto{\pgfqpoint{4.040953in}{2.387609in}}%
\pgfpathlineto{\pgfqpoint{4.060901in}{2.383634in}}%
\pgfpathlineto{\pgfqpoint{4.079741in}{2.377611in}}%
\pgfpathlineto{\pgfqpoint{4.098582in}{2.369147in}}%
\pgfpathlineto{\pgfqpoint{4.116314in}{2.358752in}}%
\pgfpathlineto{\pgfqpoint{4.134046in}{2.345834in}}%
\pgfpathlineto{\pgfqpoint{4.151778in}{2.330258in}}%
\pgfpathlineto{\pgfqpoint{4.170618in}{2.310693in}}%
\pgfpathlineto{\pgfqpoint{4.190567in}{2.286536in}}%
\pgfpathlineto{\pgfqpoint{4.210515in}{2.258869in}}%
\pgfpathlineto{\pgfqpoint{4.232680in}{2.224164in}}%
\pgfpathlineto{\pgfqpoint{4.257062in}{2.181540in}}%
\pgfpathlineto{\pgfqpoint{4.283660in}{2.130435in}}%
\pgfpathlineto{\pgfqpoint{4.316907in}{2.061369in}}%
\pgfpathlineto{\pgfqpoint{4.349046in}{1.991285in}}%
\pgfpathlineto{\pgfqpoint{4.349046in}{1.991285in}}%
\pgfusepath{stroke}%
\end{pgfscope}%
\begin{pgfscope}%
\pgfsetrectcap%
\pgfsetmiterjoin%
\pgfsetlinewidth{0.803000pt}%
\definecolor{currentstroke}{rgb}{0.501961,0.501961,0.501961}%
\pgfsetstrokecolor{currentstroke}%
\pgfsetdash{}{0pt}%
\pgfpathmoveto{\pgfqpoint{3.214225in}{1.668832in}}%
\pgfpathlineto{\pgfqpoint{3.214225in}{2.423832in}}%
\pgfusepath{stroke}%
\end{pgfscope}%
\begin{pgfscope}%
\pgfsetrectcap%
\pgfsetmiterjoin%
\pgfsetlinewidth{0.803000pt}%
\definecolor{currentstroke}{rgb}{0.501961,0.501961,0.501961}%
\pgfsetstrokecolor{currentstroke}%
\pgfsetdash{}{0pt}%
\pgfpathmoveto{\pgfqpoint{4.376725in}{1.668832in}}%
\pgfpathlineto{\pgfqpoint{4.376725in}{2.423832in}}%
\pgfusepath{stroke}%
\end{pgfscope}%
\begin{pgfscope}%
\pgfsetrectcap%
\pgfsetmiterjoin%
\pgfsetlinewidth{0.803000pt}%
\definecolor{currentstroke}{rgb}{0.501961,0.501961,0.501961}%
\pgfsetstrokecolor{currentstroke}%
\pgfsetdash{}{0pt}%
\pgfpathmoveto{\pgfqpoint{3.214225in}{1.668832in}}%
\pgfpathlineto{\pgfqpoint{4.376725in}{1.668832in}}%
\pgfusepath{stroke}%
\end{pgfscope}%
\begin{pgfscope}%
\pgfsetrectcap%
\pgfsetmiterjoin%
\pgfsetlinewidth{0.803000pt}%
\definecolor{currentstroke}{rgb}{0.501961,0.501961,0.501961}%
\pgfsetstrokecolor{currentstroke}%
\pgfsetdash{}{0pt}%
\pgfpathmoveto{\pgfqpoint{3.214225in}{2.423832in}}%
\pgfpathlineto{\pgfqpoint{4.376725in}{2.423832in}}%
\pgfusepath{stroke}%
\end{pgfscope}%
\begin{pgfscope}%
\pgfsetbuttcap%
\pgfsetmiterjoin%
\definecolor{currentfill}{rgb}{1.000000,1.000000,1.000000}%
\pgfsetfillcolor{currentfill}%
\pgfsetlinewidth{0.000000pt}%
\definecolor{currentstroke}{rgb}{0.000000,0.000000,0.000000}%
\pgfsetstrokecolor{currentstroke}%
\pgfsetstrokeopacity{0.000000}%
\pgfsetdash{}{0pt}%
\pgfpathmoveto{\pgfqpoint{4.376725in}{1.668832in}}%
\pgfpathlineto{\pgfqpoint{5.539225in}{1.668832in}}%
\pgfpathlineto{\pgfqpoint{5.539225in}{2.423832in}}%
\pgfpathlineto{\pgfqpoint{4.376725in}{2.423832in}}%
\pgfpathclose%
\pgfusepath{fill}%
\end{pgfscope}%
\begin{pgfscope}%
\pgfpathrectangle{\pgfqpoint{4.376725in}{1.668832in}}{\pgfqpoint{1.162500in}{0.755000in}}%
\pgfusepath{clip}%
\pgfsetbuttcap%
\pgfsetroundjoin%
\definecolor{currentfill}{rgb}{0.000000,0.000000,0.000000}%
\pgfsetfillcolor{currentfill}%
\pgfsetfillopacity{0.500000}%
\pgfsetlinewidth{0.000000pt}%
\definecolor{currentstroke}{rgb}{0.000000,0.000000,0.000000}%
\pgfsetstrokecolor{currentstroke}%
\pgfsetdash{}{0pt}%
\pgfpathmoveto{\pgfqpoint{5.214960in}{1.935152in}}%
\pgfpathcurveto{\pgfqpoint{5.220485in}{1.935152in}}{\pgfqpoint{5.225785in}{1.937347in}}{\pgfqpoint{5.229692in}{1.941254in}}%
\pgfpathcurveto{\pgfqpoint{5.233598in}{1.945161in}}{\pgfqpoint{5.235794in}{1.950460in}}{\pgfqpoint{5.235794in}{1.955985in}}%
\pgfpathcurveto{\pgfqpoint{5.235794in}{1.961510in}}{\pgfqpoint{5.233598in}{1.966810in}}{\pgfqpoint{5.229692in}{1.970717in}}%
\pgfpathcurveto{\pgfqpoint{5.225785in}{1.974623in}}{\pgfqpoint{5.220485in}{1.976819in}}{\pgfqpoint{5.214960in}{1.976819in}}%
\pgfpathcurveto{\pgfqpoint{5.209435in}{1.976819in}}{\pgfqpoint{5.204136in}{1.974623in}}{\pgfqpoint{5.200229in}{1.970717in}}%
\pgfpathcurveto{\pgfqpoint{5.196322in}{1.966810in}}{\pgfqpoint{5.194127in}{1.961510in}}{\pgfqpoint{5.194127in}{1.955985in}}%
\pgfpathcurveto{\pgfqpoint{5.194127in}{1.950460in}}{\pgfqpoint{5.196322in}{1.945161in}}{\pgfqpoint{5.200229in}{1.941254in}}%
\pgfpathcurveto{\pgfqpoint{5.204136in}{1.937347in}}{\pgfqpoint{5.209435in}{1.935152in}}{\pgfqpoint{5.214960in}{1.935152in}}%
\pgfpathclose%
\pgfusepath{fill}%
\end{pgfscope}%
\begin{pgfscope}%
\pgfpathrectangle{\pgfqpoint{4.376725in}{1.668832in}}{\pgfqpoint{1.162500in}{0.755000in}}%
\pgfusepath{clip}%
\pgfsetbuttcap%
\pgfsetroundjoin%
\definecolor{currentfill}{rgb}{0.000000,0.000000,0.000000}%
\pgfsetfillcolor{currentfill}%
\pgfsetfillopacity{0.500000}%
\pgfsetlinewidth{0.000000pt}%
\definecolor{currentstroke}{rgb}{0.000000,0.000000,0.000000}%
\pgfsetstrokecolor{currentstroke}%
\pgfsetdash{}{0pt}%
\pgfpathmoveto{\pgfqpoint{4.521739in}{2.326950in}}%
\pgfpathcurveto{\pgfqpoint{4.527264in}{2.326950in}}{\pgfqpoint{4.532563in}{2.329145in}}{\pgfqpoint{4.536470in}{2.333052in}}%
\pgfpathcurveto{\pgfqpoint{4.540377in}{2.336959in}}{\pgfqpoint{4.542572in}{2.342258in}}{\pgfqpoint{4.542572in}{2.347783in}}%
\pgfpathcurveto{\pgfqpoint{4.542572in}{2.353308in}}{\pgfqpoint{4.540377in}{2.358608in}}{\pgfqpoint{4.536470in}{2.362515in}}%
\pgfpathcurveto{\pgfqpoint{4.532563in}{2.366421in}}{\pgfqpoint{4.527264in}{2.368616in}}{\pgfqpoint{4.521739in}{2.368616in}}%
\pgfpathcurveto{\pgfqpoint{4.516214in}{2.368616in}}{\pgfqpoint{4.510914in}{2.366421in}}{\pgfqpoint{4.507007in}{2.362515in}}%
\pgfpathcurveto{\pgfqpoint{4.503101in}{2.358608in}}{\pgfqpoint{4.500905in}{2.353308in}}{\pgfqpoint{4.500905in}{2.347783in}}%
\pgfpathcurveto{\pgfqpoint{4.500905in}{2.342258in}}{\pgfqpoint{4.503101in}{2.336959in}}{\pgfqpoint{4.507007in}{2.333052in}}%
\pgfpathcurveto{\pgfqpoint{4.510914in}{2.329145in}}{\pgfqpoint{4.516214in}{2.326950in}}{\pgfqpoint{4.521739in}{2.326950in}}%
\pgfpathclose%
\pgfusepath{fill}%
\end{pgfscope}%
\begin{pgfscope}%
\pgfpathrectangle{\pgfqpoint{4.376725in}{1.668832in}}{\pgfqpoint{1.162500in}{0.755000in}}%
\pgfusepath{clip}%
\pgfsetbuttcap%
\pgfsetroundjoin%
\definecolor{currentfill}{rgb}{0.000000,0.000000,0.000000}%
\pgfsetfillcolor{currentfill}%
\pgfsetfillopacity{0.500000}%
\pgfsetlinewidth{0.000000pt}%
\definecolor{currentstroke}{rgb}{0.000000,0.000000,0.000000}%
\pgfsetstrokecolor{currentstroke}%
\pgfsetdash{}{0pt}%
\pgfpathmoveto{\pgfqpoint{4.448919in}{2.385023in}}%
\pgfpathcurveto{\pgfqpoint{4.454444in}{2.385023in}}{\pgfqpoint{4.459744in}{2.387218in}}{\pgfqpoint{4.463650in}{2.391125in}}%
\pgfpathcurveto{\pgfqpoint{4.467557in}{2.395032in}}{\pgfqpoint{4.469752in}{2.400331in}}{\pgfqpoint{4.469752in}{2.405856in}}%
\pgfpathcurveto{\pgfqpoint{4.469752in}{2.411381in}}{\pgfqpoint{4.467557in}{2.416681in}}{\pgfqpoint{4.463650in}{2.420588in}}%
\pgfpathcurveto{\pgfqpoint{4.459744in}{2.424494in}}{\pgfqpoint{4.454444in}{2.426690in}}{\pgfqpoint{4.448919in}{2.426690in}}%
\pgfpathcurveto{\pgfqpoint{4.443394in}{2.426690in}}{\pgfqpoint{4.438094in}{2.424494in}}{\pgfqpoint{4.434188in}{2.420588in}}%
\pgfpathcurveto{\pgfqpoint{4.430281in}{2.416681in}}{\pgfqpoint{4.428086in}{2.411381in}}{\pgfqpoint{4.428086in}{2.405856in}}%
\pgfpathcurveto{\pgfqpoint{4.428086in}{2.400331in}}{\pgfqpoint{4.430281in}{2.395032in}}{\pgfqpoint{4.434188in}{2.391125in}}%
\pgfpathcurveto{\pgfqpoint{4.438094in}{2.387218in}}{\pgfqpoint{4.443394in}{2.385023in}}{\pgfqpoint{4.448919in}{2.385023in}}%
\pgfpathclose%
\pgfusepath{fill}%
\end{pgfscope}%
\begin{pgfscope}%
\pgfpathrectangle{\pgfqpoint{4.376725in}{1.668832in}}{\pgfqpoint{1.162500in}{0.755000in}}%
\pgfusepath{clip}%
\pgfsetbuttcap%
\pgfsetroundjoin%
\definecolor{currentfill}{rgb}{0.000000,0.000000,0.000000}%
\pgfsetfillcolor{currentfill}%
\pgfsetfillopacity{0.500000}%
\pgfsetlinewidth{0.000000pt}%
\definecolor{currentstroke}{rgb}{0.000000,0.000000,0.000000}%
\pgfsetstrokecolor{currentstroke}%
\pgfsetdash{}{0pt}%
\pgfpathmoveto{\pgfqpoint{4.704251in}{2.103661in}}%
\pgfpathcurveto{\pgfqpoint{4.709776in}{2.103661in}}{\pgfqpoint{4.715075in}{2.105856in}}{\pgfqpoint{4.718982in}{2.109762in}}%
\pgfpathcurveto{\pgfqpoint{4.722889in}{2.113669in}}{\pgfqpoint{4.725084in}{2.118969in}}{\pgfqpoint{4.725084in}{2.124494in}}%
\pgfpathcurveto{\pgfqpoint{4.725084in}{2.130019in}}{\pgfqpoint{4.722889in}{2.135318in}}{\pgfqpoint{4.718982in}{2.139225in}}%
\pgfpathcurveto{\pgfqpoint{4.715075in}{2.143132in}}{\pgfqpoint{4.709776in}{2.145327in}}{\pgfqpoint{4.704251in}{2.145327in}}%
\pgfpathcurveto{\pgfqpoint{4.698726in}{2.145327in}}{\pgfqpoint{4.693426in}{2.143132in}}{\pgfqpoint{4.689519in}{2.139225in}}%
\pgfpathcurveto{\pgfqpoint{4.685612in}{2.135318in}}{\pgfqpoint{4.683417in}{2.130019in}}{\pgfqpoint{4.683417in}{2.124494in}}%
\pgfpathcurveto{\pgfqpoint{4.683417in}{2.118969in}}{\pgfqpoint{4.685612in}{2.113669in}}{\pgfqpoint{4.689519in}{2.109762in}}%
\pgfpathcurveto{\pgfqpoint{4.693426in}{2.105856in}}{\pgfqpoint{4.698726in}{2.103661in}}{\pgfqpoint{4.704251in}{2.103661in}}%
\pgfpathclose%
\pgfusepath{fill}%
\end{pgfscope}%
\begin{pgfscope}%
\pgfpathrectangle{\pgfqpoint{4.376725in}{1.668832in}}{\pgfqpoint{1.162500in}{0.755000in}}%
\pgfusepath{clip}%
\pgfsetbuttcap%
\pgfsetroundjoin%
\definecolor{currentfill}{rgb}{0.000000,0.000000,0.000000}%
\pgfsetfillcolor{currentfill}%
\pgfsetfillopacity{0.500000}%
\pgfsetlinewidth{0.000000pt}%
\definecolor{currentstroke}{rgb}{0.000000,0.000000,0.000000}%
\pgfsetstrokecolor{currentstroke}%
\pgfsetdash{}{0pt}%
\pgfpathmoveto{\pgfqpoint{4.404404in}{2.149550in}}%
\pgfpathcurveto{\pgfqpoint{4.409929in}{2.149550in}}{\pgfqpoint{4.415228in}{2.151746in}}{\pgfqpoint{4.419135in}{2.155652in}}%
\pgfpathcurveto{\pgfqpoint{4.423042in}{2.159559in}}{\pgfqpoint{4.425237in}{2.164859in}}{\pgfqpoint{4.425237in}{2.170384in}}%
\pgfpathcurveto{\pgfqpoint{4.425237in}{2.175909in}}{\pgfqpoint{4.423042in}{2.181208in}}{\pgfqpoint{4.419135in}{2.185115in}}%
\pgfpathcurveto{\pgfqpoint{4.415228in}{2.189022in}}{\pgfqpoint{4.409929in}{2.191217in}}{\pgfqpoint{4.404404in}{2.191217in}}%
\pgfpathcurveto{\pgfqpoint{4.398879in}{2.191217in}}{\pgfqpoint{4.393579in}{2.189022in}}{\pgfqpoint{4.389672in}{2.185115in}}%
\pgfpathcurveto{\pgfqpoint{4.385765in}{2.181208in}}{\pgfqpoint{4.383570in}{2.175909in}}{\pgfqpoint{4.383570in}{2.170384in}}%
\pgfpathcurveto{\pgfqpoint{4.383570in}{2.164859in}}{\pgfqpoint{4.385765in}{2.159559in}}{\pgfqpoint{4.389672in}{2.155652in}}%
\pgfpathcurveto{\pgfqpoint{4.393579in}{2.151746in}}{\pgfqpoint{4.398879in}{2.149550in}}{\pgfqpoint{4.404404in}{2.149550in}}%
\pgfpathclose%
\pgfusepath{fill}%
\end{pgfscope}%
\begin{pgfscope}%
\pgfpathrectangle{\pgfqpoint{4.376725in}{1.668832in}}{\pgfqpoint{1.162500in}{0.755000in}}%
\pgfusepath{clip}%
\pgfsetbuttcap%
\pgfsetroundjoin%
\definecolor{currentfill}{rgb}{0.000000,0.000000,0.000000}%
\pgfsetfillcolor{currentfill}%
\pgfsetfillopacity{0.500000}%
\pgfsetlinewidth{0.000000pt}%
\definecolor{currentstroke}{rgb}{0.000000,0.000000,0.000000}%
\pgfsetstrokecolor{currentstroke}%
\pgfsetdash{}{0pt}%
\pgfpathmoveto{\pgfqpoint{4.607580in}{2.049204in}}%
\pgfpathcurveto{\pgfqpoint{4.613105in}{2.049204in}}{\pgfqpoint{4.618405in}{2.051399in}}{\pgfqpoint{4.622312in}{2.055306in}}%
\pgfpathcurveto{\pgfqpoint{4.626218in}{2.059212in}}{\pgfqpoint{4.628414in}{2.064512in}}{\pgfqpoint{4.628414in}{2.070037in}}%
\pgfpathcurveto{\pgfqpoint{4.628414in}{2.075562in}}{\pgfqpoint{4.626218in}{2.080862in}}{\pgfqpoint{4.622312in}{2.084768in}}%
\pgfpathcurveto{\pgfqpoint{4.618405in}{2.088675in}}{\pgfqpoint{4.613105in}{2.090870in}}{\pgfqpoint{4.607580in}{2.090870in}}%
\pgfpathcurveto{\pgfqpoint{4.602055in}{2.090870in}}{\pgfqpoint{4.596756in}{2.088675in}}{\pgfqpoint{4.592849in}{2.084768in}}%
\pgfpathcurveto{\pgfqpoint{4.588942in}{2.080862in}}{\pgfqpoint{4.586747in}{2.075562in}}{\pgfqpoint{4.586747in}{2.070037in}}%
\pgfpathcurveto{\pgfqpoint{4.586747in}{2.064512in}}{\pgfqpoint{4.588942in}{2.059212in}}{\pgfqpoint{4.592849in}{2.055306in}}%
\pgfpathcurveto{\pgfqpoint{4.596756in}{2.051399in}}{\pgfqpoint{4.602055in}{2.049204in}}{\pgfqpoint{4.607580in}{2.049204in}}%
\pgfpathclose%
\pgfusepath{fill}%
\end{pgfscope}%
\begin{pgfscope}%
\pgfpathrectangle{\pgfqpoint{4.376725in}{1.668832in}}{\pgfqpoint{1.162500in}{0.755000in}}%
\pgfusepath{clip}%
\pgfsetbuttcap%
\pgfsetroundjoin%
\definecolor{currentfill}{rgb}{0.000000,0.000000,0.000000}%
\pgfsetfillcolor{currentfill}%
\pgfsetfillopacity{0.500000}%
\pgfsetlinewidth{0.000000pt}%
\definecolor{currentstroke}{rgb}{0.000000,0.000000,0.000000}%
\pgfsetstrokecolor{currentstroke}%
\pgfsetdash{}{0pt}%
\pgfpathmoveto{\pgfqpoint{4.596304in}{1.665975in}}%
\pgfpathcurveto{\pgfqpoint{4.601829in}{1.665975in}}{\pgfqpoint{4.607129in}{1.668170in}}{\pgfqpoint{4.611036in}{1.672077in}}%
\pgfpathcurveto{\pgfqpoint{4.614942in}{1.675984in}}{\pgfqpoint{4.617137in}{1.681283in}}{\pgfqpoint{4.617137in}{1.686809in}}%
\pgfpathcurveto{\pgfqpoint{4.617137in}{1.692334in}}{\pgfqpoint{4.614942in}{1.697633in}}{\pgfqpoint{4.611036in}{1.701540in}}%
\pgfpathcurveto{\pgfqpoint{4.607129in}{1.705447in}}{\pgfqpoint{4.601829in}{1.707642in}}{\pgfqpoint{4.596304in}{1.707642in}}%
\pgfpathcurveto{\pgfqpoint{4.590779in}{1.707642in}}{\pgfqpoint{4.585480in}{1.705447in}}{\pgfqpoint{4.581573in}{1.701540in}}%
\pgfpathcurveto{\pgfqpoint{4.577666in}{1.697633in}}{\pgfqpoint{4.575471in}{1.692334in}}{\pgfqpoint{4.575471in}{1.686809in}}%
\pgfpathcurveto{\pgfqpoint{4.575471in}{1.681283in}}{\pgfqpoint{4.577666in}{1.675984in}}{\pgfqpoint{4.581573in}{1.672077in}}%
\pgfpathcurveto{\pgfqpoint{4.585480in}{1.668170in}}{\pgfqpoint{4.590779in}{1.665975in}}{\pgfqpoint{4.596304in}{1.665975in}}%
\pgfpathclose%
\pgfusepath{fill}%
\end{pgfscope}%
\begin{pgfscope}%
\pgfpathrectangle{\pgfqpoint{4.376725in}{1.668832in}}{\pgfqpoint{1.162500in}{0.755000in}}%
\pgfusepath{clip}%
\pgfsetbuttcap%
\pgfsetroundjoin%
\definecolor{currentfill}{rgb}{0.000000,0.000000,0.000000}%
\pgfsetfillcolor{currentfill}%
\pgfsetfillopacity{0.500000}%
\pgfsetlinewidth{0.000000pt}%
\definecolor{currentstroke}{rgb}{0.000000,0.000000,0.000000}%
\pgfsetstrokecolor{currentstroke}%
\pgfsetdash{}{0pt}%
\pgfpathmoveto{\pgfqpoint{5.423123in}{2.301682in}}%
\pgfpathcurveto{\pgfqpoint{5.428649in}{2.301682in}}{\pgfqpoint{5.433948in}{2.303877in}}{\pgfqpoint{5.437855in}{2.307784in}}%
\pgfpathcurveto{\pgfqpoint{5.441762in}{2.311691in}}{\pgfqpoint{5.443957in}{2.316990in}}{\pgfqpoint{5.443957in}{2.322515in}}%
\pgfpathcurveto{\pgfqpoint{5.443957in}{2.328041in}}{\pgfqpoint{5.441762in}{2.333340in}}{\pgfqpoint{5.437855in}{2.337247in}}%
\pgfpathcurveto{\pgfqpoint{5.433948in}{2.341154in}}{\pgfqpoint{5.428649in}{2.343349in}}{\pgfqpoint{5.423123in}{2.343349in}}%
\pgfpathcurveto{\pgfqpoint{5.417598in}{2.343349in}}{\pgfqpoint{5.412299in}{2.341154in}}{\pgfqpoint{5.408392in}{2.337247in}}%
\pgfpathcurveto{\pgfqpoint{5.404485in}{2.333340in}}{\pgfqpoint{5.402290in}{2.328041in}}{\pgfqpoint{5.402290in}{2.322515in}}%
\pgfpathcurveto{\pgfqpoint{5.402290in}{2.316990in}}{\pgfqpoint{5.404485in}{2.311691in}}{\pgfqpoint{5.408392in}{2.307784in}}%
\pgfpathcurveto{\pgfqpoint{5.412299in}{2.303877in}}{\pgfqpoint{5.417598in}{2.301682in}}{\pgfqpoint{5.423123in}{2.301682in}}%
\pgfpathclose%
\pgfusepath{fill}%
\end{pgfscope}%
\begin{pgfscope}%
\pgfpathrectangle{\pgfqpoint{4.376725in}{1.668832in}}{\pgfqpoint{1.162500in}{0.755000in}}%
\pgfusepath{clip}%
\pgfsetbuttcap%
\pgfsetroundjoin%
\definecolor{currentfill}{rgb}{0.000000,0.000000,0.000000}%
\pgfsetfillcolor{currentfill}%
\pgfsetfillopacity{0.500000}%
\pgfsetlinewidth{0.000000pt}%
\definecolor{currentstroke}{rgb}{0.000000,0.000000,0.000000}%
\pgfsetstrokecolor{currentstroke}%
\pgfsetdash{}{0pt}%
\pgfpathmoveto{\pgfqpoint{5.511546in}{2.145886in}}%
\pgfpathcurveto{\pgfqpoint{5.517071in}{2.145886in}}{\pgfqpoint{5.522371in}{2.148081in}}{\pgfqpoint{5.526278in}{2.151988in}}%
\pgfpathcurveto{\pgfqpoint{5.530185in}{2.155894in}}{\pgfqpoint{5.532380in}{2.161194in}}{\pgfqpoint{5.532380in}{2.166719in}}%
\pgfpathcurveto{\pgfqpoint{5.532380in}{2.172244in}}{\pgfqpoint{5.530185in}{2.177544in}}{\pgfqpoint{5.526278in}{2.181450in}}%
\pgfpathcurveto{\pgfqpoint{5.522371in}{2.185357in}}{\pgfqpoint{5.517071in}{2.187552in}}{\pgfqpoint{5.511546in}{2.187552in}}%
\pgfpathcurveto{\pgfqpoint{5.506021in}{2.187552in}}{\pgfqpoint{5.500722in}{2.185357in}}{\pgfqpoint{5.496815in}{2.181450in}}%
\pgfpathcurveto{\pgfqpoint{5.492908in}{2.177544in}}{\pgfqpoint{5.490713in}{2.172244in}}{\pgfqpoint{5.490713in}{2.166719in}}%
\pgfpathcurveto{\pgfqpoint{5.490713in}{2.161194in}}{\pgfqpoint{5.492908in}{2.155894in}}{\pgfqpoint{5.496815in}{2.151988in}}%
\pgfpathcurveto{\pgfqpoint{5.500722in}{2.148081in}}{\pgfqpoint{5.506021in}{2.145886in}}{\pgfqpoint{5.511546in}{2.145886in}}%
\pgfpathclose%
\pgfusepath{fill}%
\end{pgfscope}%
\begin{pgfscope}%
\pgfpathrectangle{\pgfqpoint{4.376725in}{1.668832in}}{\pgfqpoint{1.162500in}{0.755000in}}%
\pgfusepath{clip}%
\pgfsetbuttcap%
\pgfsetroundjoin%
\definecolor{currentfill}{rgb}{0.000000,0.000000,0.000000}%
\pgfsetfillcolor{currentfill}%
\pgfsetfillopacity{0.500000}%
\pgfsetlinewidth{0.000000pt}%
\definecolor{currentstroke}{rgb}{0.000000,0.000000,0.000000}%
\pgfsetstrokecolor{currentstroke}%
\pgfsetdash{}{0pt}%
\pgfpathmoveto{\pgfqpoint{5.286995in}{1.969105in}}%
\pgfpathcurveto{\pgfqpoint{5.292520in}{1.969105in}}{\pgfqpoint{5.297820in}{1.971300in}}{\pgfqpoint{5.301726in}{1.975207in}}%
\pgfpathcurveto{\pgfqpoint{5.305633in}{1.979114in}}{\pgfqpoint{5.307828in}{1.984413in}}{\pgfqpoint{5.307828in}{1.989939in}}%
\pgfpathcurveto{\pgfqpoint{5.307828in}{1.995464in}}{\pgfqpoint{5.305633in}{2.000763in}}{\pgfqpoint{5.301726in}{2.004670in}}%
\pgfpathcurveto{\pgfqpoint{5.297820in}{2.008577in}}{\pgfqpoint{5.292520in}{2.010772in}}{\pgfqpoint{5.286995in}{2.010772in}}%
\pgfpathcurveto{\pgfqpoint{5.281470in}{2.010772in}}{\pgfqpoint{5.276170in}{2.008577in}}{\pgfqpoint{5.272264in}{2.004670in}}%
\pgfpathcurveto{\pgfqpoint{5.268357in}{2.000763in}}{\pgfqpoint{5.266162in}{1.995464in}}{\pgfqpoint{5.266162in}{1.989939in}}%
\pgfpathcurveto{\pgfqpoint{5.266162in}{1.984413in}}{\pgfqpoint{5.268357in}{1.979114in}}{\pgfqpoint{5.272264in}{1.975207in}}%
\pgfpathcurveto{\pgfqpoint{5.276170in}{1.971300in}}{\pgfqpoint{5.281470in}{1.969105in}}{\pgfqpoint{5.286995in}{1.969105in}}%
\pgfpathclose%
\pgfusepath{fill}%
\end{pgfscope}%
\begin{pgfscope}%
\pgfpathrectangle{\pgfqpoint{4.376725in}{1.668832in}}{\pgfqpoint{1.162500in}{0.755000in}}%
\pgfusepath{clip}%
\pgfsetbuttcap%
\pgfsetroundjoin%
\definecolor{currentfill}{rgb}{0.000000,0.000000,0.000000}%
\pgfsetfillcolor{currentfill}%
\pgfsetfillopacity{0.500000}%
\pgfsetlinewidth{0.000000pt}%
\definecolor{currentstroke}{rgb}{0.000000,0.000000,0.000000}%
\pgfsetstrokecolor{currentstroke}%
\pgfsetdash{}{0pt}%
\pgfpathmoveto{\pgfqpoint{4.645822in}{2.239104in}}%
\pgfpathcurveto{\pgfqpoint{4.651348in}{2.239104in}}{\pgfqpoint{4.656647in}{2.241300in}}{\pgfqpoint{4.660554in}{2.245206in}}%
\pgfpathcurveto{\pgfqpoint{4.664461in}{2.249113in}}{\pgfqpoint{4.666656in}{2.254413in}}{\pgfqpoint{4.666656in}{2.259938in}}%
\pgfpathcurveto{\pgfqpoint{4.666656in}{2.265463in}}{\pgfqpoint{4.664461in}{2.270762in}}{\pgfqpoint{4.660554in}{2.274669in}}%
\pgfpathcurveto{\pgfqpoint{4.656647in}{2.278576in}}{\pgfqpoint{4.651348in}{2.280771in}}{\pgfqpoint{4.645822in}{2.280771in}}%
\pgfpathcurveto{\pgfqpoint{4.640297in}{2.280771in}}{\pgfqpoint{4.634998in}{2.278576in}}{\pgfqpoint{4.631091in}{2.274669in}}%
\pgfpathcurveto{\pgfqpoint{4.627184in}{2.270762in}}{\pgfqpoint{4.624989in}{2.265463in}}{\pgfqpoint{4.624989in}{2.259938in}}%
\pgfpathcurveto{\pgfqpoint{4.624989in}{2.254413in}}{\pgfqpoint{4.627184in}{2.249113in}}{\pgfqpoint{4.631091in}{2.245206in}}%
\pgfpathcurveto{\pgfqpoint{4.634998in}{2.241300in}}{\pgfqpoint{4.640297in}{2.239104in}}{\pgfqpoint{4.645822in}{2.239104in}}%
\pgfpathclose%
\pgfusepath{fill}%
\end{pgfscope}%
\begin{pgfscope}%
\pgfpathrectangle{\pgfqpoint{4.376725in}{1.668832in}}{\pgfqpoint{1.162500in}{0.755000in}}%
\pgfusepath{clip}%
\pgfsetbuttcap%
\pgfsetroundjoin%
\definecolor{currentfill}{rgb}{0.000000,0.000000,0.000000}%
\pgfsetfillcolor{currentfill}%
\pgfsetfillopacity{0.500000}%
\pgfsetlinewidth{0.000000pt}%
\definecolor{currentstroke}{rgb}{0.000000,0.000000,0.000000}%
\pgfsetstrokecolor{currentstroke}%
\pgfsetdash{}{0pt}%
\pgfpathmoveto{\pgfqpoint{5.069839in}{1.960666in}}%
\pgfpathcurveto{\pgfqpoint{5.075364in}{1.960666in}}{\pgfqpoint{5.080663in}{1.962861in}}{\pgfqpoint{5.084570in}{1.966768in}}%
\pgfpathcurveto{\pgfqpoint{5.088477in}{1.970675in}}{\pgfqpoint{5.090672in}{1.975974in}}{\pgfqpoint{5.090672in}{1.981499in}}%
\pgfpathcurveto{\pgfqpoint{5.090672in}{1.987024in}}{\pgfqpoint{5.088477in}{1.992324in}}{\pgfqpoint{5.084570in}{1.996231in}}%
\pgfpathcurveto{\pgfqpoint{5.080663in}{2.000138in}}{\pgfqpoint{5.075364in}{2.002333in}}{\pgfqpoint{5.069839in}{2.002333in}}%
\pgfpathcurveto{\pgfqpoint{5.064313in}{2.002333in}}{\pgfqpoint{5.059014in}{2.000138in}}{\pgfqpoint{5.055107in}{1.996231in}}%
\pgfpathcurveto{\pgfqpoint{5.051200in}{1.992324in}}{\pgfqpoint{5.049005in}{1.987024in}}{\pgfqpoint{5.049005in}{1.981499in}}%
\pgfpathcurveto{\pgfqpoint{5.049005in}{1.975974in}}{\pgfqpoint{5.051200in}{1.970675in}}{\pgfqpoint{5.055107in}{1.966768in}}%
\pgfpathcurveto{\pgfqpoint{5.059014in}{1.962861in}}{\pgfqpoint{5.064313in}{1.960666in}}{\pgfqpoint{5.069839in}{1.960666in}}%
\pgfpathclose%
\pgfusepath{fill}%
\end{pgfscope}%
\begin{pgfscope}%
\pgfpathrectangle{\pgfqpoint{4.376725in}{1.668832in}}{\pgfqpoint{1.162500in}{0.755000in}}%
\pgfusepath{clip}%
\pgfsetbuttcap%
\pgfsetroundjoin%
\definecolor{currentfill}{rgb}{0.000000,0.000000,0.000000}%
\pgfsetfillcolor{currentfill}%
\pgfsetfillopacity{0.500000}%
\pgfsetlinewidth{0.000000pt}%
\definecolor{currentstroke}{rgb}{0.000000,0.000000,0.000000}%
\pgfsetstrokecolor{currentstroke}%
\pgfsetdash{}{0pt}%
\pgfpathmoveto{\pgfqpoint{4.941375in}{1.722959in}}%
\pgfpathcurveto{\pgfqpoint{4.946900in}{1.722959in}}{\pgfqpoint{4.952199in}{1.725154in}}{\pgfqpoint{4.956106in}{1.729061in}}%
\pgfpathcurveto{\pgfqpoint{4.960013in}{1.732968in}}{\pgfqpoint{4.962208in}{1.738267in}}{\pgfqpoint{4.962208in}{1.743792in}}%
\pgfpathcurveto{\pgfqpoint{4.962208in}{1.749317in}}{\pgfqpoint{4.960013in}{1.754617in}}{\pgfqpoint{4.956106in}{1.758523in}}%
\pgfpathcurveto{\pgfqpoint{4.952199in}{1.762430in}}{\pgfqpoint{4.946900in}{1.764625in}}{\pgfqpoint{4.941375in}{1.764625in}}%
\pgfpathcurveto{\pgfqpoint{4.935850in}{1.764625in}}{\pgfqpoint{4.930550in}{1.762430in}}{\pgfqpoint{4.926643in}{1.758523in}}%
\pgfpathcurveto{\pgfqpoint{4.922736in}{1.754617in}}{\pgfqpoint{4.920541in}{1.749317in}}{\pgfqpoint{4.920541in}{1.743792in}}%
\pgfpathcurveto{\pgfqpoint{4.920541in}{1.738267in}}{\pgfqpoint{4.922736in}{1.732968in}}{\pgfqpoint{4.926643in}{1.729061in}}%
\pgfpathcurveto{\pgfqpoint{4.930550in}{1.725154in}}{\pgfqpoint{4.935850in}{1.722959in}}{\pgfqpoint{4.941375in}{1.722959in}}%
\pgfpathclose%
\pgfusepath{fill}%
\end{pgfscope}%
\begin{pgfscope}%
\pgfpathrectangle{\pgfqpoint{4.376725in}{1.668832in}}{\pgfqpoint{1.162500in}{0.755000in}}%
\pgfusepath{clip}%
\pgfsetbuttcap%
\pgfsetroundjoin%
\definecolor{currentfill}{rgb}{0.000000,0.000000,0.000000}%
\pgfsetfillcolor{currentfill}%
\pgfsetfillopacity{0.500000}%
\pgfsetlinewidth{0.000000pt}%
\definecolor{currentstroke}{rgb}{0.000000,0.000000,0.000000}%
\pgfsetstrokecolor{currentstroke}%
\pgfsetdash{}{0pt}%
\pgfpathmoveto{\pgfqpoint{4.799153in}{2.267350in}}%
\pgfpathcurveto{\pgfqpoint{4.804678in}{2.267350in}}{\pgfqpoint{4.809977in}{2.269545in}}{\pgfqpoint{4.813884in}{2.273452in}}%
\pgfpathcurveto{\pgfqpoint{4.817791in}{2.277359in}}{\pgfqpoint{4.819986in}{2.282659in}}{\pgfqpoint{4.819986in}{2.288184in}}%
\pgfpathcurveto{\pgfqpoint{4.819986in}{2.293709in}}{\pgfqpoint{4.817791in}{2.299008in}}{\pgfqpoint{4.813884in}{2.302915in}}%
\pgfpathcurveto{\pgfqpoint{4.809977in}{2.306822in}}{\pgfqpoint{4.804678in}{2.309017in}}{\pgfqpoint{4.799153in}{2.309017in}}%
\pgfpathcurveto{\pgfqpoint{4.793628in}{2.309017in}}{\pgfqpoint{4.788328in}{2.306822in}}{\pgfqpoint{4.784421in}{2.302915in}}%
\pgfpathcurveto{\pgfqpoint{4.780515in}{2.299008in}}{\pgfqpoint{4.778320in}{2.293709in}}{\pgfqpoint{4.778320in}{2.288184in}}%
\pgfpathcurveto{\pgfqpoint{4.778320in}{2.282659in}}{\pgfqpoint{4.780515in}{2.277359in}}{\pgfqpoint{4.784421in}{2.273452in}}%
\pgfpathcurveto{\pgfqpoint{4.788328in}{2.269545in}}{\pgfqpoint{4.793628in}{2.267350in}}{\pgfqpoint{4.799153in}{2.267350in}}%
\pgfpathclose%
\pgfusepath{fill}%
\end{pgfscope}%
\begin{pgfscope}%
\pgfpathrectangle{\pgfqpoint{4.376725in}{1.668832in}}{\pgfqpoint{1.162500in}{0.755000in}}%
\pgfusepath{clip}%
\pgfsetbuttcap%
\pgfsetroundjoin%
\definecolor{currentfill}{rgb}{0.000000,0.000000,0.000000}%
\pgfsetfillcolor{currentfill}%
\pgfsetfillopacity{0.500000}%
\pgfsetlinewidth{0.000000pt}%
\definecolor{currentstroke}{rgb}{0.000000,0.000000,0.000000}%
\pgfsetstrokecolor{currentstroke}%
\pgfsetdash{}{0pt}%
\pgfpathmoveto{\pgfqpoint{4.595218in}{2.236796in}}%
\pgfpathcurveto{\pgfqpoint{4.600743in}{2.236796in}}{\pgfqpoint{4.606043in}{2.238991in}}{\pgfqpoint{4.609950in}{2.242898in}}%
\pgfpathcurveto{\pgfqpoint{4.613857in}{2.246805in}}{\pgfqpoint{4.616052in}{2.252104in}}{\pgfqpoint{4.616052in}{2.257629in}}%
\pgfpathcurveto{\pgfqpoint{4.616052in}{2.263154in}}{\pgfqpoint{4.613857in}{2.268454in}}{\pgfqpoint{4.609950in}{2.272361in}}%
\pgfpathcurveto{\pgfqpoint{4.606043in}{2.276267in}}{\pgfqpoint{4.600743in}{2.278463in}}{\pgfqpoint{4.595218in}{2.278463in}}%
\pgfpathcurveto{\pgfqpoint{4.589693in}{2.278463in}}{\pgfqpoint{4.584394in}{2.276267in}}{\pgfqpoint{4.580487in}{2.272361in}}%
\pgfpathcurveto{\pgfqpoint{4.576580in}{2.268454in}}{\pgfqpoint{4.574385in}{2.263154in}}{\pgfqpoint{4.574385in}{2.257629in}}%
\pgfpathcurveto{\pgfqpoint{4.574385in}{2.252104in}}{\pgfqpoint{4.576580in}{2.246805in}}{\pgfqpoint{4.580487in}{2.242898in}}%
\pgfpathcurveto{\pgfqpoint{4.584394in}{2.238991in}}{\pgfqpoint{4.589693in}{2.236796in}}{\pgfqpoint{4.595218in}{2.236796in}}%
\pgfpathclose%
\pgfusepath{fill}%
\end{pgfscope}%
\begin{pgfscope}%
\pgfpathrectangle{\pgfqpoint{4.376725in}{1.668832in}}{\pgfqpoint{1.162500in}{0.755000in}}%
\pgfusepath{clip}%
\pgfsetbuttcap%
\pgfsetroundjoin%
\definecolor{currentfill}{rgb}{0.000000,0.000000,0.000000}%
\pgfsetfillcolor{currentfill}%
\pgfsetfillopacity{0.500000}%
\pgfsetlinewidth{0.000000pt}%
\definecolor{currentstroke}{rgb}{0.000000,0.000000,0.000000}%
\pgfsetstrokecolor{currentstroke}%
\pgfsetdash{}{0pt}%
\pgfpathmoveto{\pgfqpoint{4.834270in}{2.023720in}}%
\pgfpathcurveto{\pgfqpoint{4.839795in}{2.023720in}}{\pgfqpoint{4.845094in}{2.025915in}}{\pgfqpoint{4.849001in}{2.029822in}}%
\pgfpathcurveto{\pgfqpoint{4.852908in}{2.033729in}}{\pgfqpoint{4.855103in}{2.039028in}}{\pgfqpoint{4.855103in}{2.044553in}}%
\pgfpathcurveto{\pgfqpoint{4.855103in}{2.050078in}}{\pgfqpoint{4.852908in}{2.055378in}}{\pgfqpoint{4.849001in}{2.059285in}}%
\pgfpathcurveto{\pgfqpoint{4.845094in}{2.063191in}}{\pgfqpoint{4.839795in}{2.065387in}}{\pgfqpoint{4.834270in}{2.065387in}}%
\pgfpathcurveto{\pgfqpoint{4.828745in}{2.065387in}}{\pgfqpoint{4.823445in}{2.063191in}}{\pgfqpoint{4.819538in}{2.059285in}}%
\pgfpathcurveto{\pgfqpoint{4.815632in}{2.055378in}}{\pgfqpoint{4.813437in}{2.050078in}}{\pgfqpoint{4.813437in}{2.044553in}}%
\pgfpathcurveto{\pgfqpoint{4.813437in}{2.039028in}}{\pgfqpoint{4.815632in}{2.033729in}}{\pgfqpoint{4.819538in}{2.029822in}}%
\pgfpathcurveto{\pgfqpoint{4.823445in}{2.025915in}}{\pgfqpoint{4.828745in}{2.023720in}}{\pgfqpoint{4.834270in}{2.023720in}}%
\pgfpathclose%
\pgfusepath{fill}%
\end{pgfscope}%
\begin{pgfscope}%
\pgfsetrectcap%
\pgfsetmiterjoin%
\pgfsetlinewidth{0.803000pt}%
\definecolor{currentstroke}{rgb}{0.501961,0.501961,0.501961}%
\pgfsetstrokecolor{currentstroke}%
\pgfsetdash{}{0pt}%
\pgfpathmoveto{\pgfqpoint{4.376725in}{1.668832in}}%
\pgfpathlineto{\pgfqpoint{4.376725in}{2.423832in}}%
\pgfusepath{stroke}%
\end{pgfscope}%
\begin{pgfscope}%
\pgfsetrectcap%
\pgfsetmiterjoin%
\pgfsetlinewidth{0.803000pt}%
\definecolor{currentstroke}{rgb}{0.501961,0.501961,0.501961}%
\pgfsetstrokecolor{currentstroke}%
\pgfsetdash{}{0pt}%
\pgfpathmoveto{\pgfqpoint{5.539225in}{1.668832in}}%
\pgfpathlineto{\pgfqpoint{5.539225in}{2.423832in}}%
\pgfusepath{stroke}%
\end{pgfscope}%
\begin{pgfscope}%
\pgfsetrectcap%
\pgfsetmiterjoin%
\pgfsetlinewidth{0.803000pt}%
\definecolor{currentstroke}{rgb}{0.501961,0.501961,0.501961}%
\pgfsetstrokecolor{currentstroke}%
\pgfsetdash{}{0pt}%
\pgfpathmoveto{\pgfqpoint{4.376725in}{1.668832in}}%
\pgfpathlineto{\pgfqpoint{5.539225in}{1.668832in}}%
\pgfusepath{stroke}%
\end{pgfscope}%
\begin{pgfscope}%
\pgfsetrectcap%
\pgfsetmiterjoin%
\pgfsetlinewidth{0.803000pt}%
\definecolor{currentstroke}{rgb}{0.501961,0.501961,0.501961}%
\pgfsetstrokecolor{currentstroke}%
\pgfsetdash{}{0pt}%
\pgfpathmoveto{\pgfqpoint{4.376725in}{2.423832in}}%
\pgfpathlineto{\pgfqpoint{5.539225in}{2.423832in}}%
\pgfusepath{stroke}%
\end{pgfscope}%
\begin{pgfscope}%
\pgfsetbuttcap%
\pgfsetmiterjoin%
\definecolor{currentfill}{rgb}{1.000000,1.000000,1.000000}%
\pgfsetfillcolor{currentfill}%
\pgfsetlinewidth{0.000000pt}%
\definecolor{currentstroke}{rgb}{0.000000,0.000000,0.000000}%
\pgfsetstrokecolor{currentstroke}%
\pgfsetstrokeopacity{0.000000}%
\pgfsetdash{}{0pt}%
\pgfpathmoveto{\pgfqpoint{0.889225in}{0.913832in}}%
\pgfpathlineto{\pgfqpoint{2.051725in}{0.913832in}}%
\pgfpathlineto{\pgfqpoint{2.051725in}{1.668832in}}%
\pgfpathlineto{\pgfqpoint{0.889225in}{1.668832in}}%
\pgfpathclose%
\pgfusepath{fill}%
\end{pgfscope}%
\begin{pgfscope}%
\pgfpathrectangle{\pgfqpoint{0.889225in}{0.913832in}}{\pgfqpoint{1.162500in}{0.755000in}}%
\pgfusepath{clip}%
\pgfsetbuttcap%
\pgfsetroundjoin%
\definecolor{currentfill}{rgb}{0.000000,0.000000,0.000000}%
\pgfsetfillcolor{currentfill}%
\pgfsetfillopacity{0.500000}%
\pgfsetlinewidth{0.000000pt}%
\definecolor{currentstroke}{rgb}{0.000000,0.000000,0.000000}%
\pgfsetstrokecolor{currentstroke}%
\pgfsetdash{}{0pt}%
\pgfpathmoveto{\pgfqpoint{1.893746in}{1.437401in}}%
\pgfpathcurveto{\pgfqpoint{1.899271in}{1.437401in}}{\pgfqpoint{1.904570in}{1.439596in}}{\pgfqpoint{1.908477in}{1.443503in}}%
\pgfpathcurveto{\pgfqpoint{1.912384in}{1.447410in}}{\pgfqpoint{1.914579in}{1.452710in}}{\pgfqpoint{1.914579in}{1.458235in}}%
\pgfpathcurveto{\pgfqpoint{1.914579in}{1.463760in}}{\pgfqpoint{1.912384in}{1.469059in}}{\pgfqpoint{1.908477in}{1.472966in}}%
\pgfpathcurveto{\pgfqpoint{1.904570in}{1.476873in}}{\pgfqpoint{1.899271in}{1.479068in}}{\pgfqpoint{1.893746in}{1.479068in}}%
\pgfpathcurveto{\pgfqpoint{1.888221in}{1.479068in}}{\pgfqpoint{1.882921in}{1.476873in}}{\pgfqpoint{1.879015in}{1.472966in}}%
\pgfpathcurveto{\pgfqpoint{1.875108in}{1.469059in}}{\pgfqpoint{1.872913in}{1.463760in}}{\pgfqpoint{1.872913in}{1.458235in}}%
\pgfpathcurveto{\pgfqpoint{1.872913in}{1.452710in}}{\pgfqpoint{1.875108in}{1.447410in}}{\pgfqpoint{1.879015in}{1.443503in}}%
\pgfpathcurveto{\pgfqpoint{1.882921in}{1.439596in}}{\pgfqpoint{1.888221in}{1.437401in}}{\pgfqpoint{1.893746in}{1.437401in}}%
\pgfpathclose%
\pgfusepath{fill}%
\end{pgfscope}%
\begin{pgfscope}%
\pgfpathrectangle{\pgfqpoint{0.889225in}{0.913832in}}{\pgfqpoint{1.162500in}{0.755000in}}%
\pgfusepath{clip}%
\pgfsetbuttcap%
\pgfsetroundjoin%
\definecolor{currentfill}{rgb}{0.000000,0.000000,0.000000}%
\pgfsetfillcolor{currentfill}%
\pgfsetfillopacity{0.500000}%
\pgfsetlinewidth{0.000000pt}%
\definecolor{currentstroke}{rgb}{0.000000,0.000000,0.000000}%
\pgfsetstrokecolor{currentstroke}%
\pgfsetdash{}{0pt}%
\pgfpathmoveto{\pgfqpoint{1.476892in}{0.987180in}}%
\pgfpathcurveto{\pgfqpoint{1.482417in}{0.987180in}}{\pgfqpoint{1.487717in}{0.989375in}}{\pgfqpoint{1.491623in}{0.993282in}}%
\pgfpathcurveto{\pgfqpoint{1.495530in}{0.997189in}}{\pgfqpoint{1.497725in}{1.002488in}}{\pgfqpoint{1.497725in}{1.008013in}}%
\pgfpathcurveto{\pgfqpoint{1.497725in}{1.013538in}}{\pgfqpoint{1.495530in}{1.018838in}}{\pgfqpoint{1.491623in}{1.022745in}}%
\pgfpathcurveto{\pgfqpoint{1.487717in}{1.026652in}}{\pgfqpoint{1.482417in}{1.028847in}}{\pgfqpoint{1.476892in}{1.028847in}}%
\pgfpathcurveto{\pgfqpoint{1.471367in}{1.028847in}}{\pgfqpoint{1.466067in}{1.026652in}}{\pgfqpoint{1.462161in}{1.022745in}}%
\pgfpathcurveto{\pgfqpoint{1.458254in}{1.018838in}}{\pgfqpoint{1.456059in}{1.013538in}}{\pgfqpoint{1.456059in}{1.008013in}}%
\pgfpathcurveto{\pgfqpoint{1.456059in}{1.002488in}}{\pgfqpoint{1.458254in}{0.997189in}}{\pgfqpoint{1.462161in}{0.993282in}}%
\pgfpathcurveto{\pgfqpoint{1.466067in}{0.989375in}}{\pgfqpoint{1.471367in}{0.987180in}}{\pgfqpoint{1.476892in}{0.987180in}}%
\pgfpathclose%
\pgfusepath{fill}%
\end{pgfscope}%
\begin{pgfscope}%
\pgfpathrectangle{\pgfqpoint{0.889225in}{0.913832in}}{\pgfqpoint{1.162500in}{0.755000in}}%
\pgfusepath{clip}%
\pgfsetbuttcap%
\pgfsetroundjoin%
\definecolor{currentfill}{rgb}{0.000000,0.000000,0.000000}%
\pgfsetfillcolor{currentfill}%
\pgfsetfillopacity{0.500000}%
\pgfsetlinewidth{0.000000pt}%
\definecolor{currentstroke}{rgb}{0.000000,0.000000,0.000000}%
\pgfsetstrokecolor{currentstroke}%
\pgfsetdash{}{0pt}%
\pgfpathmoveto{\pgfqpoint{1.478974in}{0.939886in}}%
\pgfpathcurveto{\pgfqpoint{1.484499in}{0.939886in}}{\pgfqpoint{1.489799in}{0.942081in}}{\pgfqpoint{1.493705in}{0.945988in}}%
\pgfpathcurveto{\pgfqpoint{1.497612in}{0.949895in}}{\pgfqpoint{1.499807in}{0.955195in}}{\pgfqpoint{1.499807in}{0.960720in}}%
\pgfpathcurveto{\pgfqpoint{1.499807in}{0.966245in}}{\pgfqpoint{1.497612in}{0.971544in}}{\pgfqpoint{1.493705in}{0.975451in}}%
\pgfpathcurveto{\pgfqpoint{1.489799in}{0.979358in}}{\pgfqpoint{1.484499in}{0.981553in}}{\pgfqpoint{1.478974in}{0.981553in}}%
\pgfpathcurveto{\pgfqpoint{1.473449in}{0.981553in}}{\pgfqpoint{1.468149in}{0.979358in}}{\pgfqpoint{1.464243in}{0.975451in}}%
\pgfpathcurveto{\pgfqpoint{1.460336in}{0.971544in}}{\pgfqpoint{1.458141in}{0.966245in}}{\pgfqpoint{1.458141in}{0.960720in}}%
\pgfpathcurveto{\pgfqpoint{1.458141in}{0.955195in}}{\pgfqpoint{1.460336in}{0.949895in}}{\pgfqpoint{1.464243in}{0.945988in}}%
\pgfpathcurveto{\pgfqpoint{1.468149in}{0.942081in}}{\pgfqpoint{1.473449in}{0.939886in}}{\pgfqpoint{1.478974in}{0.939886in}}%
\pgfpathclose%
\pgfusepath{fill}%
\end{pgfscope}%
\begin{pgfscope}%
\pgfpathrectangle{\pgfqpoint{0.889225in}{0.913832in}}{\pgfqpoint{1.162500in}{0.755000in}}%
\pgfusepath{clip}%
\pgfsetbuttcap%
\pgfsetroundjoin%
\definecolor{currentfill}{rgb}{0.000000,0.000000,0.000000}%
\pgfsetfillcolor{currentfill}%
\pgfsetfillopacity{0.500000}%
\pgfsetlinewidth{0.000000pt}%
\definecolor{currentstroke}{rgb}{0.000000,0.000000,0.000000}%
\pgfsetstrokecolor{currentstroke}%
\pgfsetdash{}{0pt}%
\pgfpathmoveto{\pgfqpoint{1.120714in}{1.105715in}}%
\pgfpathcurveto{\pgfqpoint{1.126239in}{1.105715in}}{\pgfqpoint{1.131538in}{1.107910in}}{\pgfqpoint{1.135445in}{1.111817in}}%
\pgfpathcurveto{\pgfqpoint{1.139352in}{1.115723in}}{\pgfqpoint{1.141547in}{1.121023in}}{\pgfqpoint{1.141547in}{1.126548in}}%
\pgfpathcurveto{\pgfqpoint{1.141547in}{1.132073in}}{\pgfqpoint{1.139352in}{1.137372in}}{\pgfqpoint{1.135445in}{1.141279in}}%
\pgfpathcurveto{\pgfqpoint{1.131538in}{1.145186in}}{\pgfqpoint{1.126239in}{1.147381in}}{\pgfqpoint{1.120714in}{1.147381in}}%
\pgfpathcurveto{\pgfqpoint{1.115189in}{1.147381in}}{\pgfqpoint{1.109889in}{1.145186in}}{\pgfqpoint{1.105982in}{1.141279in}}%
\pgfpathcurveto{\pgfqpoint{1.102076in}{1.137372in}}{\pgfqpoint{1.099881in}{1.132073in}}{\pgfqpoint{1.099881in}{1.126548in}}%
\pgfpathcurveto{\pgfqpoint{1.099881in}{1.121023in}}{\pgfqpoint{1.102076in}{1.115723in}}{\pgfqpoint{1.105982in}{1.111817in}}%
\pgfpathcurveto{\pgfqpoint{1.109889in}{1.107910in}}{\pgfqpoint{1.115189in}{1.105715in}}{\pgfqpoint{1.120714in}{1.105715in}}%
\pgfpathclose%
\pgfusepath{fill}%
\end{pgfscope}%
\begin{pgfscope}%
\pgfpathrectangle{\pgfqpoint{0.889225in}{0.913832in}}{\pgfqpoint{1.162500in}{0.755000in}}%
\pgfusepath{clip}%
\pgfsetbuttcap%
\pgfsetroundjoin%
\definecolor{currentfill}{rgb}{0.000000,0.000000,0.000000}%
\pgfsetfillcolor{currentfill}%
\pgfsetfillopacity{0.500000}%
\pgfsetlinewidth{0.000000pt}%
\definecolor{currentstroke}{rgb}{0.000000,0.000000,0.000000}%
\pgfsetstrokecolor{currentstroke}%
\pgfsetdash{}{0pt}%
\pgfpathmoveto{\pgfqpoint{1.126248in}{0.910975in}}%
\pgfpathcurveto{\pgfqpoint{1.131773in}{0.910975in}}{\pgfqpoint{1.137073in}{0.913170in}}{\pgfqpoint{1.140980in}{0.917077in}}%
\pgfpathcurveto{\pgfqpoint{1.144886in}{0.920984in}}{\pgfqpoint{1.147082in}{0.926283in}}{\pgfqpoint{1.147082in}{0.931809in}}%
\pgfpathcurveto{\pgfqpoint{1.147082in}{0.937334in}}{\pgfqpoint{1.144886in}{0.942633in}}{\pgfqpoint{1.140980in}{0.946540in}}%
\pgfpathcurveto{\pgfqpoint{1.137073in}{0.950447in}}{\pgfqpoint{1.131773in}{0.952642in}}{\pgfqpoint{1.126248in}{0.952642in}}%
\pgfpathcurveto{\pgfqpoint{1.120723in}{0.952642in}}{\pgfqpoint{1.115424in}{0.950447in}}{\pgfqpoint{1.111517in}{0.946540in}}%
\pgfpathcurveto{\pgfqpoint{1.107610in}{0.942633in}}{\pgfqpoint{1.105415in}{0.937334in}}{\pgfqpoint{1.105415in}{0.931809in}}%
\pgfpathcurveto{\pgfqpoint{1.105415in}{0.926283in}}{\pgfqpoint{1.107610in}{0.920984in}}{\pgfqpoint{1.111517in}{0.917077in}}%
\pgfpathcurveto{\pgfqpoint{1.115424in}{0.913170in}}{\pgfqpoint{1.120723in}{0.910975in}}{\pgfqpoint{1.126248in}{0.910975in}}%
\pgfpathclose%
\pgfusepath{fill}%
\end{pgfscope}%
\begin{pgfscope}%
\pgfpathrectangle{\pgfqpoint{0.889225in}{0.913832in}}{\pgfqpoint{1.162500in}{0.755000in}}%
\pgfusepath{clip}%
\pgfsetbuttcap%
\pgfsetroundjoin%
\definecolor{currentfill}{rgb}{0.000000,0.000000,0.000000}%
\pgfsetfillcolor{currentfill}%
\pgfsetfillopacity{0.500000}%
\pgfsetlinewidth{0.000000pt}%
\definecolor{currentstroke}{rgb}{0.000000,0.000000,0.000000}%
\pgfsetstrokecolor{currentstroke}%
\pgfsetdash{}{0pt}%
\pgfpathmoveto{\pgfqpoint{0.977704in}{1.042931in}}%
\pgfpathcurveto{\pgfqpoint{0.983229in}{1.042931in}}{\pgfqpoint{0.988528in}{1.045126in}}{\pgfqpoint{0.992435in}{1.049033in}}%
\pgfpathcurveto{\pgfqpoint{0.996342in}{1.052940in}}{\pgfqpoint{0.998537in}{1.058239in}}{\pgfqpoint{0.998537in}{1.063764in}}%
\pgfpathcurveto{\pgfqpoint{0.998537in}{1.069289in}}{\pgfqpoint{0.996342in}{1.074589in}}{\pgfqpoint{0.992435in}{1.078495in}}%
\pgfpathcurveto{\pgfqpoint{0.988528in}{1.082402in}}{\pgfqpoint{0.983229in}{1.084597in}}{\pgfqpoint{0.977704in}{1.084597in}}%
\pgfpathcurveto{\pgfqpoint{0.972179in}{1.084597in}}{\pgfqpoint{0.966879in}{1.082402in}}{\pgfqpoint{0.962972in}{1.078495in}}%
\pgfpathcurveto{\pgfqpoint{0.959066in}{1.074589in}}{\pgfqpoint{0.956870in}{1.069289in}}{\pgfqpoint{0.956870in}{1.063764in}}%
\pgfpathcurveto{\pgfqpoint{0.956870in}{1.058239in}}{\pgfqpoint{0.959066in}{1.052940in}}{\pgfqpoint{0.962972in}{1.049033in}}%
\pgfpathcurveto{\pgfqpoint{0.966879in}{1.045126in}}{\pgfqpoint{0.972179in}{1.042931in}}{\pgfqpoint{0.977704in}{1.042931in}}%
\pgfpathclose%
\pgfusepath{fill}%
\end{pgfscope}%
\begin{pgfscope}%
\pgfpathrectangle{\pgfqpoint{0.889225in}{0.913832in}}{\pgfqpoint{1.162500in}{0.755000in}}%
\pgfusepath{clip}%
\pgfsetbuttcap%
\pgfsetroundjoin%
\definecolor{currentfill}{rgb}{0.000000,0.000000,0.000000}%
\pgfsetfillcolor{currentfill}%
\pgfsetfillopacity{0.500000}%
\pgfsetlinewidth{0.000000pt}%
\definecolor{currentstroke}{rgb}{0.000000,0.000000,0.000000}%
\pgfsetstrokecolor{currentstroke}%
\pgfsetdash{}{0pt}%
\pgfpathmoveto{\pgfqpoint{0.916904in}{1.035607in}}%
\pgfpathcurveto{\pgfqpoint{0.922429in}{1.035607in}}{\pgfqpoint{0.927728in}{1.037803in}}{\pgfqpoint{0.931635in}{1.041709in}}%
\pgfpathcurveto{\pgfqpoint{0.935542in}{1.045616in}}{\pgfqpoint{0.937737in}{1.050916in}}{\pgfqpoint{0.937737in}{1.056441in}}%
\pgfpathcurveto{\pgfqpoint{0.937737in}{1.061966in}}{\pgfqpoint{0.935542in}{1.067265in}}{\pgfqpoint{0.931635in}{1.071172in}}%
\pgfpathcurveto{\pgfqpoint{0.927728in}{1.075079in}}{\pgfqpoint{0.922429in}{1.077274in}}{\pgfqpoint{0.916904in}{1.077274in}}%
\pgfpathcurveto{\pgfqpoint{0.911379in}{1.077274in}}{\pgfqpoint{0.906079in}{1.075079in}}{\pgfqpoint{0.902172in}{1.071172in}}%
\pgfpathcurveto{\pgfqpoint{0.898265in}{1.067265in}}{\pgfqpoint{0.896070in}{1.061966in}}{\pgfqpoint{0.896070in}{1.056441in}}%
\pgfpathcurveto{\pgfqpoint{0.896070in}{1.050916in}}{\pgfqpoint{0.898265in}{1.045616in}}{\pgfqpoint{0.902172in}{1.041709in}}%
\pgfpathcurveto{\pgfqpoint{0.906079in}{1.037803in}}{\pgfqpoint{0.911379in}{1.035607in}}{\pgfqpoint{0.916904in}{1.035607in}}%
\pgfpathclose%
\pgfusepath{fill}%
\end{pgfscope}%
\begin{pgfscope}%
\pgfpathrectangle{\pgfqpoint{0.889225in}{0.913832in}}{\pgfqpoint{1.162500in}{0.755000in}}%
\pgfusepath{clip}%
\pgfsetbuttcap%
\pgfsetroundjoin%
\definecolor{currentfill}{rgb}{0.000000,0.000000,0.000000}%
\pgfsetfillcolor{currentfill}%
\pgfsetfillopacity{0.500000}%
\pgfsetlinewidth{0.000000pt}%
\definecolor{currentstroke}{rgb}{0.000000,0.000000,0.000000}%
\pgfsetstrokecolor{currentstroke}%
\pgfsetdash{}{0pt}%
\pgfpathmoveto{\pgfqpoint{1.691400in}{1.572595in}}%
\pgfpathcurveto{\pgfqpoint{1.696925in}{1.572595in}}{\pgfqpoint{1.702225in}{1.574791in}}{\pgfqpoint{1.706132in}{1.578697in}}%
\pgfpathcurveto{\pgfqpoint{1.710038in}{1.582604in}}{\pgfqpoint{1.712234in}{1.587904in}}{\pgfqpoint{1.712234in}{1.593429in}}%
\pgfpathcurveto{\pgfqpoint{1.712234in}{1.598954in}}{\pgfqpoint{1.710038in}{1.604253in}}{\pgfqpoint{1.706132in}{1.608160in}}%
\pgfpathcurveto{\pgfqpoint{1.702225in}{1.612067in}}{\pgfqpoint{1.696925in}{1.614262in}}{\pgfqpoint{1.691400in}{1.614262in}}%
\pgfpathcurveto{\pgfqpoint{1.685875in}{1.614262in}}{\pgfqpoint{1.680576in}{1.612067in}}{\pgfqpoint{1.676669in}{1.608160in}}%
\pgfpathcurveto{\pgfqpoint{1.672762in}{1.604253in}}{\pgfqpoint{1.670567in}{1.598954in}}{\pgfqpoint{1.670567in}{1.593429in}}%
\pgfpathcurveto{\pgfqpoint{1.670567in}{1.587904in}}{\pgfqpoint{1.672762in}{1.582604in}}{\pgfqpoint{1.676669in}{1.578697in}}%
\pgfpathcurveto{\pgfqpoint{1.680576in}{1.574791in}}{\pgfqpoint{1.685875in}{1.572595in}}{\pgfqpoint{1.691400in}{1.572595in}}%
\pgfpathclose%
\pgfusepath{fill}%
\end{pgfscope}%
\begin{pgfscope}%
\pgfpathrectangle{\pgfqpoint{0.889225in}{0.913832in}}{\pgfqpoint{1.162500in}{0.755000in}}%
\pgfusepath{clip}%
\pgfsetbuttcap%
\pgfsetroundjoin%
\definecolor{currentfill}{rgb}{0.000000,0.000000,0.000000}%
\pgfsetfillcolor{currentfill}%
\pgfsetfillopacity{0.500000}%
\pgfsetlinewidth{0.000000pt}%
\definecolor{currentstroke}{rgb}{0.000000,0.000000,0.000000}%
\pgfsetstrokecolor{currentstroke}%
\pgfsetdash{}{0pt}%
\pgfpathmoveto{\pgfqpoint{1.460463in}{1.630023in}}%
\pgfpathcurveto{\pgfqpoint{1.465988in}{1.630023in}}{\pgfqpoint{1.471288in}{1.632218in}}{\pgfqpoint{1.475194in}{1.636125in}}%
\pgfpathcurveto{\pgfqpoint{1.479101in}{1.640032in}}{\pgfqpoint{1.481296in}{1.645331in}}{\pgfqpoint{1.481296in}{1.650856in}}%
\pgfpathcurveto{\pgfqpoint{1.481296in}{1.656381in}}{\pgfqpoint{1.479101in}{1.661681in}}{\pgfqpoint{1.475194in}{1.665588in}}%
\pgfpathcurveto{\pgfqpoint{1.471288in}{1.669494in}}{\pgfqpoint{1.465988in}{1.671690in}}{\pgfqpoint{1.460463in}{1.671690in}}%
\pgfpathcurveto{\pgfqpoint{1.454938in}{1.671690in}}{\pgfqpoint{1.449638in}{1.669494in}}{\pgfqpoint{1.445732in}{1.665588in}}%
\pgfpathcurveto{\pgfqpoint{1.441825in}{1.661681in}}{\pgfqpoint{1.439630in}{1.656381in}}{\pgfqpoint{1.439630in}{1.650856in}}%
\pgfpathcurveto{\pgfqpoint{1.439630in}{1.645331in}}{\pgfqpoint{1.441825in}{1.640032in}}{\pgfqpoint{1.445732in}{1.636125in}}%
\pgfpathcurveto{\pgfqpoint{1.449638in}{1.632218in}}{\pgfqpoint{1.454938in}{1.630023in}}{\pgfqpoint{1.460463in}{1.630023in}}%
\pgfpathclose%
\pgfusepath{fill}%
\end{pgfscope}%
\begin{pgfscope}%
\pgfpathrectangle{\pgfqpoint{0.889225in}{0.913832in}}{\pgfqpoint{1.162500in}{0.755000in}}%
\pgfusepath{clip}%
\pgfsetbuttcap%
\pgfsetroundjoin%
\definecolor{currentfill}{rgb}{0.000000,0.000000,0.000000}%
\pgfsetfillcolor{currentfill}%
\pgfsetfillopacity{0.500000}%
\pgfsetlinewidth{0.000000pt}%
\definecolor{currentstroke}{rgb}{0.000000,0.000000,0.000000}%
\pgfsetstrokecolor{currentstroke}%
\pgfsetdash{}{0pt}%
\pgfpathmoveto{\pgfqpoint{1.303193in}{1.484185in}}%
\pgfpathcurveto{\pgfqpoint{1.308718in}{1.484185in}}{\pgfqpoint{1.314017in}{1.486380in}}{\pgfqpoint{1.317924in}{1.490287in}}%
\pgfpathcurveto{\pgfqpoint{1.321831in}{1.494194in}}{\pgfqpoint{1.324026in}{1.499493in}}{\pgfqpoint{1.324026in}{1.505018in}}%
\pgfpathcurveto{\pgfqpoint{1.324026in}{1.510544in}}{\pgfqpoint{1.321831in}{1.515843in}}{\pgfqpoint{1.317924in}{1.519750in}}%
\pgfpathcurveto{\pgfqpoint{1.314017in}{1.523657in}}{\pgfqpoint{1.308718in}{1.525852in}}{\pgfqpoint{1.303193in}{1.525852in}}%
\pgfpathcurveto{\pgfqpoint{1.297667in}{1.525852in}}{\pgfqpoint{1.292368in}{1.523657in}}{\pgfqpoint{1.288461in}{1.519750in}}%
\pgfpathcurveto{\pgfqpoint{1.284554in}{1.515843in}}{\pgfqpoint{1.282359in}{1.510544in}}{\pgfqpoint{1.282359in}{1.505018in}}%
\pgfpathcurveto{\pgfqpoint{1.282359in}{1.499493in}}{\pgfqpoint{1.284554in}{1.494194in}}{\pgfqpoint{1.288461in}{1.490287in}}%
\pgfpathcurveto{\pgfqpoint{1.292368in}{1.486380in}}{\pgfqpoint{1.297667in}{1.484185in}}{\pgfqpoint{1.303193in}{1.484185in}}%
\pgfpathclose%
\pgfusepath{fill}%
\end{pgfscope}%
\begin{pgfscope}%
\pgfpathrectangle{\pgfqpoint{0.889225in}{0.913832in}}{\pgfqpoint{1.162500in}{0.755000in}}%
\pgfusepath{clip}%
\pgfsetbuttcap%
\pgfsetroundjoin%
\definecolor{currentfill}{rgb}{0.000000,0.000000,0.000000}%
\pgfsetfillcolor{currentfill}%
\pgfsetfillopacity{0.500000}%
\pgfsetlinewidth{0.000000pt}%
\definecolor{currentstroke}{rgb}{0.000000,0.000000,0.000000}%
\pgfsetstrokecolor{currentstroke}%
\pgfsetdash{}{0pt}%
\pgfpathmoveto{\pgfqpoint{1.689350in}{1.067768in}}%
\pgfpathcurveto{\pgfqpoint{1.694875in}{1.067768in}}{\pgfqpoint{1.700175in}{1.069963in}}{\pgfqpoint{1.704082in}{1.073870in}}%
\pgfpathcurveto{\pgfqpoint{1.707989in}{1.077776in}}{\pgfqpoint{1.710184in}{1.083076in}}{\pgfqpoint{1.710184in}{1.088601in}}%
\pgfpathcurveto{\pgfqpoint{1.710184in}{1.094126in}}{\pgfqpoint{1.707989in}{1.099426in}}{\pgfqpoint{1.704082in}{1.103332in}}%
\pgfpathcurveto{\pgfqpoint{1.700175in}{1.107239in}}{\pgfqpoint{1.694875in}{1.109434in}}{\pgfqpoint{1.689350in}{1.109434in}}%
\pgfpathcurveto{\pgfqpoint{1.683825in}{1.109434in}}{\pgfqpoint{1.678526in}{1.107239in}}{\pgfqpoint{1.674619in}{1.103332in}}%
\pgfpathcurveto{\pgfqpoint{1.670712in}{1.099426in}}{\pgfqpoint{1.668517in}{1.094126in}}{\pgfqpoint{1.668517in}{1.088601in}}%
\pgfpathcurveto{\pgfqpoint{1.668517in}{1.083076in}}{\pgfqpoint{1.670712in}{1.077776in}}{\pgfqpoint{1.674619in}{1.073870in}}%
\pgfpathcurveto{\pgfqpoint{1.678526in}{1.069963in}}{\pgfqpoint{1.683825in}{1.067768in}}{\pgfqpoint{1.689350in}{1.067768in}}%
\pgfpathclose%
\pgfusepath{fill}%
\end{pgfscope}%
\begin{pgfscope}%
\pgfpathrectangle{\pgfqpoint{0.889225in}{0.913832in}}{\pgfqpoint{1.162500in}{0.755000in}}%
\pgfusepath{clip}%
\pgfsetbuttcap%
\pgfsetroundjoin%
\definecolor{currentfill}{rgb}{0.000000,0.000000,0.000000}%
\pgfsetfillcolor{currentfill}%
\pgfsetfillopacity{0.500000}%
\pgfsetlinewidth{0.000000pt}%
\definecolor{currentstroke}{rgb}{0.000000,0.000000,0.000000}%
\pgfsetstrokecolor{currentstroke}%
\pgfsetdash{}{0pt}%
\pgfpathmoveto{\pgfqpoint{1.102396in}{1.343150in}}%
\pgfpathcurveto{\pgfqpoint{1.107921in}{1.343150in}}{\pgfqpoint{1.113221in}{1.345345in}}{\pgfqpoint{1.117128in}{1.349252in}}%
\pgfpathcurveto{\pgfqpoint{1.121035in}{1.353159in}}{\pgfqpoint{1.123230in}{1.358458in}}{\pgfqpoint{1.123230in}{1.363984in}}%
\pgfpathcurveto{\pgfqpoint{1.123230in}{1.369509in}}{\pgfqpoint{1.121035in}{1.374808in}}{\pgfqpoint{1.117128in}{1.378715in}}%
\pgfpathcurveto{\pgfqpoint{1.113221in}{1.382622in}}{\pgfqpoint{1.107921in}{1.384817in}}{\pgfqpoint{1.102396in}{1.384817in}}%
\pgfpathcurveto{\pgfqpoint{1.096871in}{1.384817in}}{\pgfqpoint{1.091572in}{1.382622in}}{\pgfqpoint{1.087665in}{1.378715in}}%
\pgfpathcurveto{\pgfqpoint{1.083758in}{1.374808in}}{\pgfqpoint{1.081563in}{1.369509in}}{\pgfqpoint{1.081563in}{1.363984in}}%
\pgfpathcurveto{\pgfqpoint{1.081563in}{1.358458in}}{\pgfqpoint{1.083758in}{1.353159in}}{\pgfqpoint{1.087665in}{1.349252in}}%
\pgfpathcurveto{\pgfqpoint{1.091572in}{1.345345in}}{\pgfqpoint{1.096871in}{1.343150in}}{\pgfqpoint{1.102396in}{1.343150in}}%
\pgfpathclose%
\pgfusepath{fill}%
\end{pgfscope}%
\begin{pgfscope}%
\pgfpathrectangle{\pgfqpoint{0.889225in}{0.913832in}}{\pgfqpoint{1.162500in}{0.755000in}}%
\pgfusepath{clip}%
\pgfsetbuttcap%
\pgfsetroundjoin%
\definecolor{currentfill}{rgb}{0.000000,0.000000,0.000000}%
\pgfsetfillcolor{currentfill}%
\pgfsetfillopacity{0.500000}%
\pgfsetlinewidth{0.000000pt}%
\definecolor{currentstroke}{rgb}{0.000000,0.000000,0.000000}%
\pgfsetstrokecolor{currentstroke}%
\pgfsetdash{}{0pt}%
\pgfpathmoveto{\pgfqpoint{1.143186in}{1.259718in}}%
\pgfpathcurveto{\pgfqpoint{1.148711in}{1.259718in}}{\pgfqpoint{1.154011in}{1.261913in}}{\pgfqpoint{1.157918in}{1.265820in}}%
\pgfpathcurveto{\pgfqpoint{1.161825in}{1.269726in}}{\pgfqpoint{1.164020in}{1.275026in}}{\pgfqpoint{1.164020in}{1.280551in}}%
\pgfpathcurveto{\pgfqpoint{1.164020in}{1.286076in}}{\pgfqpoint{1.161825in}{1.291376in}}{\pgfqpoint{1.157918in}{1.295282in}}%
\pgfpathcurveto{\pgfqpoint{1.154011in}{1.299189in}}{\pgfqpoint{1.148711in}{1.301384in}}{\pgfqpoint{1.143186in}{1.301384in}}%
\pgfpathcurveto{\pgfqpoint{1.137661in}{1.301384in}}{\pgfqpoint{1.132362in}{1.299189in}}{\pgfqpoint{1.128455in}{1.295282in}}%
\pgfpathcurveto{\pgfqpoint{1.124548in}{1.291376in}}{\pgfqpoint{1.122353in}{1.286076in}}{\pgfqpoint{1.122353in}{1.280551in}}%
\pgfpathcurveto{\pgfqpoint{1.122353in}{1.275026in}}{\pgfqpoint{1.124548in}{1.269726in}}{\pgfqpoint{1.128455in}{1.265820in}}%
\pgfpathcurveto{\pgfqpoint{1.132362in}{1.261913in}}{\pgfqpoint{1.137661in}{1.259718in}}{\pgfqpoint{1.143186in}{1.259718in}}%
\pgfpathclose%
\pgfusepath{fill}%
\end{pgfscope}%
\begin{pgfscope}%
\pgfpathrectangle{\pgfqpoint{0.889225in}{0.913832in}}{\pgfqpoint{1.162500in}{0.755000in}}%
\pgfusepath{clip}%
\pgfsetbuttcap%
\pgfsetroundjoin%
\definecolor{currentfill}{rgb}{0.000000,0.000000,0.000000}%
\pgfsetfillcolor{currentfill}%
\pgfsetfillopacity{0.500000}%
\pgfsetlinewidth{0.000000pt}%
\definecolor{currentstroke}{rgb}{0.000000,0.000000,0.000000}%
\pgfsetstrokecolor{currentstroke}%
\pgfsetdash{}{0pt}%
\pgfpathmoveto{\pgfqpoint{2.024046in}{1.167350in}}%
\pgfpathcurveto{\pgfqpoint{2.029571in}{1.167350in}}{\pgfqpoint{2.034871in}{1.169545in}}{\pgfqpoint{2.038778in}{1.173452in}}%
\pgfpathcurveto{\pgfqpoint{2.042685in}{1.177359in}}{\pgfqpoint{2.044880in}{1.182658in}}{\pgfqpoint{2.044880in}{1.188183in}}%
\pgfpathcurveto{\pgfqpoint{2.044880in}{1.193708in}}{\pgfqpoint{2.042685in}{1.199008in}}{\pgfqpoint{2.038778in}{1.202915in}}%
\pgfpathcurveto{\pgfqpoint{2.034871in}{1.206822in}}{\pgfqpoint{2.029571in}{1.209017in}}{\pgfqpoint{2.024046in}{1.209017in}}%
\pgfpathcurveto{\pgfqpoint{2.018521in}{1.209017in}}{\pgfqpoint{2.013222in}{1.206822in}}{\pgfqpoint{2.009315in}{1.202915in}}%
\pgfpathcurveto{\pgfqpoint{2.005408in}{1.199008in}}{\pgfqpoint{2.003213in}{1.193708in}}{\pgfqpoint{2.003213in}{1.188183in}}%
\pgfpathcurveto{\pgfqpoint{2.003213in}{1.182658in}}{\pgfqpoint{2.005408in}{1.177359in}}{\pgfqpoint{2.009315in}{1.173452in}}%
\pgfpathcurveto{\pgfqpoint{2.013222in}{1.169545in}}{\pgfqpoint{2.018521in}{1.167350in}}{\pgfqpoint{2.024046in}{1.167350in}}%
\pgfpathclose%
\pgfusepath{fill}%
\end{pgfscope}%
\begin{pgfscope}%
\pgfpathrectangle{\pgfqpoint{0.889225in}{0.913832in}}{\pgfqpoint{1.162500in}{0.755000in}}%
\pgfusepath{clip}%
\pgfsetbuttcap%
\pgfsetroundjoin%
\definecolor{currentfill}{rgb}{0.000000,0.000000,0.000000}%
\pgfsetfillcolor{currentfill}%
\pgfsetfillopacity{0.500000}%
\pgfsetlinewidth{0.000000pt}%
\definecolor{currentstroke}{rgb}{0.000000,0.000000,0.000000}%
\pgfsetstrokecolor{currentstroke}%
\pgfsetdash{}{0pt}%
\pgfpathmoveto{\pgfqpoint{1.402285in}{1.034902in}}%
\pgfpathcurveto{\pgfqpoint{1.407810in}{1.034902in}}{\pgfqpoint{1.413110in}{1.037097in}}{\pgfqpoint{1.417016in}{1.041004in}}%
\pgfpathcurveto{\pgfqpoint{1.420923in}{1.044911in}}{\pgfqpoint{1.423118in}{1.050211in}}{\pgfqpoint{1.423118in}{1.055736in}}%
\pgfpathcurveto{\pgfqpoint{1.423118in}{1.061261in}}{\pgfqpoint{1.420923in}{1.066560in}}{\pgfqpoint{1.417016in}{1.070467in}}%
\pgfpathcurveto{\pgfqpoint{1.413110in}{1.074374in}}{\pgfqpoint{1.407810in}{1.076569in}}{\pgfqpoint{1.402285in}{1.076569in}}%
\pgfpathcurveto{\pgfqpoint{1.396760in}{1.076569in}}{\pgfqpoint{1.391460in}{1.074374in}}{\pgfqpoint{1.387554in}{1.070467in}}%
\pgfpathcurveto{\pgfqpoint{1.383647in}{1.066560in}}{\pgfqpoint{1.381452in}{1.061261in}}{\pgfqpoint{1.381452in}{1.055736in}}%
\pgfpathcurveto{\pgfqpoint{1.381452in}{1.050211in}}{\pgfqpoint{1.383647in}{1.044911in}}{\pgfqpoint{1.387554in}{1.041004in}}%
\pgfpathcurveto{\pgfqpoint{1.391460in}{1.037097in}}{\pgfqpoint{1.396760in}{1.034902in}}{\pgfqpoint{1.402285in}{1.034902in}}%
\pgfpathclose%
\pgfusepath{fill}%
\end{pgfscope}%
\begin{pgfscope}%
\pgfpathrectangle{\pgfqpoint{0.889225in}{0.913832in}}{\pgfqpoint{1.162500in}{0.755000in}}%
\pgfusepath{clip}%
\pgfsetbuttcap%
\pgfsetroundjoin%
\definecolor{currentfill}{rgb}{0.000000,0.000000,0.000000}%
\pgfsetfillcolor{currentfill}%
\pgfsetfillopacity{0.500000}%
\pgfsetlinewidth{0.000000pt}%
\definecolor{currentstroke}{rgb}{0.000000,0.000000,0.000000}%
\pgfsetstrokecolor{currentstroke}%
\pgfsetdash{}{0pt}%
\pgfpathmoveto{\pgfqpoint{0.973483in}{1.190157in}}%
\pgfpathcurveto{\pgfqpoint{0.979008in}{1.190157in}}{\pgfqpoint{0.984308in}{1.192352in}}{\pgfqpoint{0.988214in}{1.196259in}}%
\pgfpathcurveto{\pgfqpoint{0.992121in}{1.200166in}}{\pgfqpoint{0.994316in}{1.205465in}}{\pgfqpoint{0.994316in}{1.210991in}}%
\pgfpathcurveto{\pgfqpoint{0.994316in}{1.216516in}}{\pgfqpoint{0.992121in}{1.221815in}}{\pgfqpoint{0.988214in}{1.225722in}}%
\pgfpathcurveto{\pgfqpoint{0.984308in}{1.229629in}}{\pgfqpoint{0.979008in}{1.231824in}}{\pgfqpoint{0.973483in}{1.231824in}}%
\pgfpathcurveto{\pgfqpoint{0.967958in}{1.231824in}}{\pgfqpoint{0.962658in}{1.229629in}}{\pgfqpoint{0.958752in}{1.225722in}}%
\pgfpathcurveto{\pgfqpoint{0.954845in}{1.221815in}}{\pgfqpoint{0.952650in}{1.216516in}}{\pgfqpoint{0.952650in}{1.210991in}}%
\pgfpathcurveto{\pgfqpoint{0.952650in}{1.205465in}}{\pgfqpoint{0.954845in}{1.200166in}}{\pgfqpoint{0.958752in}{1.196259in}}%
\pgfpathcurveto{\pgfqpoint{0.962658in}{1.192352in}}{\pgfqpoint{0.967958in}{1.190157in}}{\pgfqpoint{0.973483in}{1.190157in}}%
\pgfpathclose%
\pgfusepath{fill}%
\end{pgfscope}%
\begin{pgfscope}%
\pgfsetbuttcap%
\pgfsetroundjoin%
\definecolor{currentfill}{rgb}{0.000000,0.000000,0.000000}%
\pgfsetfillcolor{currentfill}%
\pgfsetlinewidth{0.803000pt}%
\definecolor{currentstroke}{rgb}{0.000000,0.000000,0.000000}%
\pgfsetstrokecolor{currentstroke}%
\pgfsetdash{}{0pt}%
\pgfsys@defobject{currentmarker}{\pgfqpoint{0.000000in}{-0.048611in}}{\pgfqpoint{0.000000in}{0.000000in}}{%
\pgfpathmoveto{\pgfqpoint{0.000000in}{0.000000in}}%
\pgfpathlineto{\pgfqpoint{0.000000in}{-0.048611in}}%
\pgfusepath{stroke,fill}%
}%
\begin{pgfscope}%
\pgfsys@transformshift{1.236821in}{0.913832in}%
\pgfsys@useobject{currentmarker}{}%
\end{pgfscope}%
\end{pgfscope}%
\begin{pgfscope}%
\pgftext[x=1.267474in,y=0.370616in,left,base,rotate=90.000000]{\rmfamily\fontsize{8.000000}{9.600000}\selectfont \(\displaystyle 0.000025\)}%
\end{pgfscope}%
\begin{pgfscope}%
\pgfsetbuttcap%
\pgfsetroundjoin%
\definecolor{currentfill}{rgb}{0.000000,0.000000,0.000000}%
\pgfsetfillcolor{currentfill}%
\pgfsetlinewidth{0.803000pt}%
\definecolor{currentstroke}{rgb}{0.000000,0.000000,0.000000}%
\pgfsetstrokecolor{currentstroke}%
\pgfsetdash{}{0pt}%
\pgfsys@defobject{currentmarker}{\pgfqpoint{0.000000in}{-0.048611in}}{\pgfqpoint{0.000000in}{0.000000in}}{%
\pgfpathmoveto{\pgfqpoint{0.000000in}{0.000000in}}%
\pgfpathlineto{\pgfqpoint{0.000000in}{-0.048611in}}%
\pgfusepath{stroke,fill}%
}%
\begin{pgfscope}%
\pgfsys@transformshift{1.758350in}{0.913832in}%
\pgfsys@useobject{currentmarker}{}%
\end{pgfscope}%
\end{pgfscope}%
\begin{pgfscope}%
\pgftext[x=1.789004in,y=0.370616in,left,base,rotate=90.000000]{\rmfamily\fontsize{8.000000}{9.600000}\selectfont \(\displaystyle 0.000050\)}%
\end{pgfscope}%
\begin{pgfscope}%
\pgftext[x=1.470475in,y=0.315061in,,top]{\rmfamily\fontsize{16.000000}{19.200000}\selectfont area}%
\end{pgfscope}%
\begin{pgfscope}%
\pgfsetbuttcap%
\pgfsetroundjoin%
\definecolor{currentfill}{rgb}{0.000000,0.000000,0.000000}%
\pgfsetfillcolor{currentfill}%
\pgfsetlinewidth{0.803000pt}%
\definecolor{currentstroke}{rgb}{0.000000,0.000000,0.000000}%
\pgfsetstrokecolor{currentstroke}%
\pgfsetdash{}{0pt}%
\pgfsys@defobject{currentmarker}{\pgfqpoint{-0.048611in}{0.000000in}}{\pgfqpoint{0.000000in}{0.000000in}}{%
\pgfpathmoveto{\pgfqpoint{0.000000in}{0.000000in}}%
\pgfpathlineto{\pgfqpoint{-0.048611in}{0.000000in}}%
\pgfusepath{stroke,fill}%
}%
\begin{pgfscope}%
\pgfsys@transformshift{0.889225in}{0.965005in}%
\pgfsys@useobject{currentmarker}{}%
\end{pgfscope}%
\end{pgfscope}%
\begin{pgfscope}%
\pgftext[x=0.641152in,y=0.922796in,left,base]{\rmfamily\fontsize{8.000000}{9.600000}\selectfont \(\displaystyle 0.5\)}%
\end{pgfscope}%
\begin{pgfscope}%
\pgfsetbuttcap%
\pgfsetroundjoin%
\definecolor{currentfill}{rgb}{0.000000,0.000000,0.000000}%
\pgfsetfillcolor{currentfill}%
\pgfsetlinewidth{0.803000pt}%
\definecolor{currentstroke}{rgb}{0.000000,0.000000,0.000000}%
\pgfsetstrokecolor{currentstroke}%
\pgfsetdash{}{0pt}%
\pgfsys@defobject{currentmarker}{\pgfqpoint{-0.048611in}{0.000000in}}{\pgfqpoint{0.000000in}{0.000000in}}{%
\pgfpathmoveto{\pgfqpoint{0.000000in}{0.000000in}}%
\pgfpathlineto{\pgfqpoint{-0.048611in}{0.000000in}}%
\pgfusepath{stroke,fill}%
}%
\begin{pgfscope}%
\pgfsys@transformshift{0.889225in}{1.260040in}%
\pgfsys@useobject{currentmarker}{}%
\end{pgfscope}%
\end{pgfscope}%
\begin{pgfscope}%
\pgftext[x=0.641152in,y=1.217831in,left,base]{\rmfamily\fontsize{8.000000}{9.600000}\selectfont \(\displaystyle 1.0\)}%
\end{pgfscope}%
\begin{pgfscope}%
\pgfsetbuttcap%
\pgfsetroundjoin%
\definecolor{currentfill}{rgb}{0.000000,0.000000,0.000000}%
\pgfsetfillcolor{currentfill}%
\pgfsetlinewidth{0.803000pt}%
\definecolor{currentstroke}{rgb}{0.000000,0.000000,0.000000}%
\pgfsetstrokecolor{currentstroke}%
\pgfsetdash{}{0pt}%
\pgfsys@defobject{currentmarker}{\pgfqpoint{-0.048611in}{0.000000in}}{\pgfqpoint{0.000000in}{0.000000in}}{%
\pgfpathmoveto{\pgfqpoint{0.000000in}{0.000000in}}%
\pgfpathlineto{\pgfqpoint{-0.048611in}{0.000000in}}%
\pgfusepath{stroke,fill}%
}%
\begin{pgfscope}%
\pgfsys@transformshift{0.889225in}{1.555075in}%
\pgfsys@useobject{currentmarker}{}%
\end{pgfscope}%
\end{pgfscope}%
\begin{pgfscope}%
\pgftext[x=0.641152in,y=1.512866in,left,base]{\rmfamily\fontsize{8.000000}{9.600000}\selectfont \(\displaystyle 1.5\)}%
\end{pgfscope}%
\begin{pgfscope}%
\pgftext[x=0.585596in,y=1.291332in,,bottom,rotate=90.000000]{\rmfamily\fontsize{16.000000}{19.200000}\selectfont Ef0}%
\end{pgfscope}%
\begin{pgfscope}%
\pgfsetrectcap%
\pgfsetmiterjoin%
\pgfsetlinewidth{0.803000pt}%
\definecolor{currentstroke}{rgb}{0.501961,0.501961,0.501961}%
\pgfsetstrokecolor{currentstroke}%
\pgfsetdash{}{0pt}%
\pgfpathmoveto{\pgfqpoint{0.889225in}{0.913832in}}%
\pgfpathlineto{\pgfqpoint{0.889225in}{1.668832in}}%
\pgfusepath{stroke}%
\end{pgfscope}%
\begin{pgfscope}%
\pgfsetrectcap%
\pgfsetmiterjoin%
\pgfsetlinewidth{0.803000pt}%
\definecolor{currentstroke}{rgb}{0.501961,0.501961,0.501961}%
\pgfsetstrokecolor{currentstroke}%
\pgfsetdash{}{0pt}%
\pgfpathmoveto{\pgfqpoint{2.051725in}{0.913832in}}%
\pgfpathlineto{\pgfqpoint{2.051725in}{1.668832in}}%
\pgfusepath{stroke}%
\end{pgfscope}%
\begin{pgfscope}%
\pgfsetrectcap%
\pgfsetmiterjoin%
\pgfsetlinewidth{0.803000pt}%
\definecolor{currentstroke}{rgb}{0.501961,0.501961,0.501961}%
\pgfsetstrokecolor{currentstroke}%
\pgfsetdash{}{0pt}%
\pgfpathmoveto{\pgfqpoint{0.889225in}{0.913832in}}%
\pgfpathlineto{\pgfqpoint{2.051725in}{0.913832in}}%
\pgfusepath{stroke}%
\end{pgfscope}%
\begin{pgfscope}%
\pgfsetrectcap%
\pgfsetmiterjoin%
\pgfsetlinewidth{0.803000pt}%
\definecolor{currentstroke}{rgb}{0.501961,0.501961,0.501961}%
\pgfsetstrokecolor{currentstroke}%
\pgfsetdash{}{0pt}%
\pgfpathmoveto{\pgfqpoint{0.889225in}{1.668832in}}%
\pgfpathlineto{\pgfqpoint{2.051725in}{1.668832in}}%
\pgfusepath{stroke}%
\end{pgfscope}%
\begin{pgfscope}%
\pgfsetbuttcap%
\pgfsetmiterjoin%
\definecolor{currentfill}{rgb}{1.000000,1.000000,1.000000}%
\pgfsetfillcolor{currentfill}%
\pgfsetlinewidth{0.000000pt}%
\definecolor{currentstroke}{rgb}{0.000000,0.000000,0.000000}%
\pgfsetstrokecolor{currentstroke}%
\pgfsetstrokeopacity{0.000000}%
\pgfsetdash{}{0pt}%
\pgfpathmoveto{\pgfqpoint{2.051725in}{0.913832in}}%
\pgfpathlineto{\pgfqpoint{3.214225in}{0.913832in}}%
\pgfpathlineto{\pgfqpoint{3.214225in}{1.668832in}}%
\pgfpathlineto{\pgfqpoint{2.051725in}{1.668832in}}%
\pgfpathclose%
\pgfusepath{fill}%
\end{pgfscope}%
\begin{pgfscope}%
\pgfpathrectangle{\pgfqpoint{2.051725in}{0.913832in}}{\pgfqpoint{1.162500in}{0.755000in}}%
\pgfusepath{clip}%
\pgfsetbuttcap%
\pgfsetroundjoin%
\definecolor{currentfill}{rgb}{0.000000,0.000000,0.000000}%
\pgfsetfillcolor{currentfill}%
\pgfsetfillopacity{0.500000}%
\pgfsetlinewidth{0.000000pt}%
\definecolor{currentstroke}{rgb}{0.000000,0.000000,0.000000}%
\pgfsetstrokecolor{currentstroke}%
\pgfsetdash{}{0pt}%
\pgfpathmoveto{\pgfqpoint{3.107569in}{1.437401in}}%
\pgfpathcurveto{\pgfqpoint{3.113094in}{1.437401in}}{\pgfqpoint{3.118394in}{1.439596in}}{\pgfqpoint{3.122301in}{1.443503in}}%
\pgfpathcurveto{\pgfqpoint{3.126207in}{1.447410in}}{\pgfqpoint{3.128403in}{1.452710in}}{\pgfqpoint{3.128403in}{1.458235in}}%
\pgfpathcurveto{\pgfqpoint{3.128403in}{1.463760in}}{\pgfqpoint{3.126207in}{1.469059in}}{\pgfqpoint{3.122301in}{1.472966in}}%
\pgfpathcurveto{\pgfqpoint{3.118394in}{1.476873in}}{\pgfqpoint{3.113094in}{1.479068in}}{\pgfqpoint{3.107569in}{1.479068in}}%
\pgfpathcurveto{\pgfqpoint{3.102044in}{1.479068in}}{\pgfqpoint{3.096745in}{1.476873in}}{\pgfqpoint{3.092838in}{1.472966in}}%
\pgfpathcurveto{\pgfqpoint{3.088931in}{1.469059in}}{\pgfqpoint{3.086736in}{1.463760in}}{\pgfqpoint{3.086736in}{1.458235in}}%
\pgfpathcurveto{\pgfqpoint{3.086736in}{1.452710in}}{\pgfqpoint{3.088931in}{1.447410in}}{\pgfqpoint{3.092838in}{1.443503in}}%
\pgfpathcurveto{\pgfqpoint{3.096745in}{1.439596in}}{\pgfqpoint{3.102044in}{1.437401in}}{\pgfqpoint{3.107569in}{1.437401in}}%
\pgfpathclose%
\pgfusepath{fill}%
\end{pgfscope}%
\begin{pgfscope}%
\pgfpathrectangle{\pgfqpoint{2.051725in}{0.913832in}}{\pgfqpoint{1.162500in}{0.755000in}}%
\pgfusepath{clip}%
\pgfsetbuttcap%
\pgfsetroundjoin%
\definecolor{currentfill}{rgb}{0.000000,0.000000,0.000000}%
\pgfsetfillcolor{currentfill}%
\pgfsetfillopacity{0.500000}%
\pgfsetlinewidth{0.000000pt}%
\definecolor{currentstroke}{rgb}{0.000000,0.000000,0.000000}%
\pgfsetstrokecolor{currentstroke}%
\pgfsetdash{}{0pt}%
\pgfpathmoveto{\pgfqpoint{2.270173in}{0.987180in}}%
\pgfpathcurveto{\pgfqpoint{2.275698in}{0.987180in}}{\pgfqpoint{2.280997in}{0.989375in}}{\pgfqpoint{2.284904in}{0.993282in}}%
\pgfpathcurveto{\pgfqpoint{2.288811in}{0.997189in}}{\pgfqpoint{2.291006in}{1.002488in}}{\pgfqpoint{2.291006in}{1.008013in}}%
\pgfpathcurveto{\pgfqpoint{2.291006in}{1.013538in}}{\pgfqpoint{2.288811in}{1.018838in}}{\pgfqpoint{2.284904in}{1.022745in}}%
\pgfpathcurveto{\pgfqpoint{2.280997in}{1.026652in}}{\pgfqpoint{2.275698in}{1.028847in}}{\pgfqpoint{2.270173in}{1.028847in}}%
\pgfpathcurveto{\pgfqpoint{2.264648in}{1.028847in}}{\pgfqpoint{2.259348in}{1.026652in}}{\pgfqpoint{2.255441in}{1.022745in}}%
\pgfpathcurveto{\pgfqpoint{2.251535in}{1.018838in}}{\pgfqpoint{2.249339in}{1.013538in}}{\pgfqpoint{2.249339in}{1.008013in}}%
\pgfpathcurveto{\pgfqpoint{2.249339in}{1.002488in}}{\pgfqpoint{2.251535in}{0.997189in}}{\pgfqpoint{2.255441in}{0.993282in}}%
\pgfpathcurveto{\pgfqpoint{2.259348in}{0.989375in}}{\pgfqpoint{2.264648in}{0.987180in}}{\pgfqpoint{2.270173in}{0.987180in}}%
\pgfpathclose%
\pgfusepath{fill}%
\end{pgfscope}%
\begin{pgfscope}%
\pgfpathrectangle{\pgfqpoint{2.051725in}{0.913832in}}{\pgfqpoint{1.162500in}{0.755000in}}%
\pgfusepath{clip}%
\pgfsetbuttcap%
\pgfsetroundjoin%
\definecolor{currentfill}{rgb}{0.000000,0.000000,0.000000}%
\pgfsetfillcolor{currentfill}%
\pgfsetfillopacity{0.500000}%
\pgfsetlinewidth{0.000000pt}%
\definecolor{currentstroke}{rgb}{0.000000,0.000000,0.000000}%
\pgfsetstrokecolor{currentstroke}%
\pgfsetdash{}{0pt}%
\pgfpathmoveto{\pgfqpoint{2.264867in}{0.939886in}}%
\pgfpathcurveto{\pgfqpoint{2.270392in}{0.939886in}}{\pgfqpoint{2.275692in}{0.942081in}}{\pgfqpoint{2.279599in}{0.945988in}}%
\pgfpathcurveto{\pgfqpoint{2.283506in}{0.949895in}}{\pgfqpoint{2.285701in}{0.955195in}}{\pgfqpoint{2.285701in}{0.960720in}}%
\pgfpathcurveto{\pgfqpoint{2.285701in}{0.966245in}}{\pgfqpoint{2.283506in}{0.971544in}}{\pgfqpoint{2.279599in}{0.975451in}}%
\pgfpathcurveto{\pgfqpoint{2.275692in}{0.979358in}}{\pgfqpoint{2.270392in}{0.981553in}}{\pgfqpoint{2.264867in}{0.981553in}}%
\pgfpathcurveto{\pgfqpoint{2.259342in}{0.981553in}}{\pgfqpoint{2.254043in}{0.979358in}}{\pgfqpoint{2.250136in}{0.975451in}}%
\pgfpathcurveto{\pgfqpoint{2.246229in}{0.971544in}}{\pgfqpoint{2.244034in}{0.966245in}}{\pgfqpoint{2.244034in}{0.960720in}}%
\pgfpathcurveto{\pgfqpoint{2.244034in}{0.955195in}}{\pgfqpoint{2.246229in}{0.949895in}}{\pgfqpoint{2.250136in}{0.945988in}}%
\pgfpathcurveto{\pgfqpoint{2.254043in}{0.942081in}}{\pgfqpoint{2.259342in}{0.939886in}}{\pgfqpoint{2.264867in}{0.939886in}}%
\pgfpathclose%
\pgfusepath{fill}%
\end{pgfscope}%
\begin{pgfscope}%
\pgfpathrectangle{\pgfqpoint{2.051725in}{0.913832in}}{\pgfqpoint{1.162500in}{0.755000in}}%
\pgfusepath{clip}%
\pgfsetbuttcap%
\pgfsetroundjoin%
\definecolor{currentfill}{rgb}{0.000000,0.000000,0.000000}%
\pgfsetfillcolor{currentfill}%
\pgfsetfillopacity{0.500000}%
\pgfsetlinewidth{0.000000pt}%
\definecolor{currentstroke}{rgb}{0.000000,0.000000,0.000000}%
\pgfsetstrokecolor{currentstroke}%
\pgfsetdash{}{0pt}%
\pgfpathmoveto{\pgfqpoint{2.154193in}{1.105715in}}%
\pgfpathcurveto{\pgfqpoint{2.159718in}{1.105715in}}{\pgfqpoint{2.165018in}{1.107910in}}{\pgfqpoint{2.168924in}{1.111817in}}%
\pgfpathcurveto{\pgfqpoint{2.172831in}{1.115723in}}{\pgfqpoint{2.175026in}{1.121023in}}{\pgfqpoint{2.175026in}{1.126548in}}%
\pgfpathcurveto{\pgfqpoint{2.175026in}{1.132073in}}{\pgfqpoint{2.172831in}{1.137372in}}{\pgfqpoint{2.168924in}{1.141279in}}%
\pgfpathcurveto{\pgfqpoint{2.165018in}{1.145186in}}{\pgfqpoint{2.159718in}{1.147381in}}{\pgfqpoint{2.154193in}{1.147381in}}%
\pgfpathcurveto{\pgfqpoint{2.148668in}{1.147381in}}{\pgfqpoint{2.143368in}{1.145186in}}{\pgfqpoint{2.139462in}{1.141279in}}%
\pgfpathcurveto{\pgfqpoint{2.135555in}{1.137372in}}{\pgfqpoint{2.133360in}{1.132073in}}{\pgfqpoint{2.133360in}{1.126548in}}%
\pgfpathcurveto{\pgfqpoint{2.133360in}{1.121023in}}{\pgfqpoint{2.135555in}{1.115723in}}{\pgfqpoint{2.139462in}{1.111817in}}%
\pgfpathcurveto{\pgfqpoint{2.143368in}{1.107910in}}{\pgfqpoint{2.148668in}{1.105715in}}{\pgfqpoint{2.154193in}{1.105715in}}%
\pgfpathclose%
\pgfusepath{fill}%
\end{pgfscope}%
\begin{pgfscope}%
\pgfpathrectangle{\pgfqpoint{2.051725in}{0.913832in}}{\pgfqpoint{1.162500in}{0.755000in}}%
\pgfusepath{clip}%
\pgfsetbuttcap%
\pgfsetroundjoin%
\definecolor{currentfill}{rgb}{0.000000,0.000000,0.000000}%
\pgfsetfillcolor{currentfill}%
\pgfsetfillopacity{0.500000}%
\pgfsetlinewidth{0.000000pt}%
\definecolor{currentstroke}{rgb}{0.000000,0.000000,0.000000}%
\pgfsetstrokecolor{currentstroke}%
\pgfsetdash{}{0pt}%
\pgfpathmoveto{\pgfqpoint{2.183344in}{0.910975in}}%
\pgfpathcurveto{\pgfqpoint{2.188869in}{0.910975in}}{\pgfqpoint{2.194169in}{0.913170in}}{\pgfqpoint{2.198076in}{0.917077in}}%
\pgfpathcurveto{\pgfqpoint{2.201982in}{0.920984in}}{\pgfqpoint{2.204178in}{0.926283in}}{\pgfqpoint{2.204178in}{0.931809in}}%
\pgfpathcurveto{\pgfqpoint{2.204178in}{0.937334in}}{\pgfqpoint{2.201982in}{0.942633in}}{\pgfqpoint{2.198076in}{0.946540in}}%
\pgfpathcurveto{\pgfqpoint{2.194169in}{0.950447in}}{\pgfqpoint{2.188869in}{0.952642in}}{\pgfqpoint{2.183344in}{0.952642in}}%
\pgfpathcurveto{\pgfqpoint{2.177819in}{0.952642in}}{\pgfqpoint{2.172520in}{0.950447in}}{\pgfqpoint{2.168613in}{0.946540in}}%
\pgfpathcurveto{\pgfqpoint{2.164706in}{0.942633in}}{\pgfqpoint{2.162511in}{0.937334in}}{\pgfqpoint{2.162511in}{0.931809in}}%
\pgfpathcurveto{\pgfqpoint{2.162511in}{0.926283in}}{\pgfqpoint{2.164706in}{0.920984in}}{\pgfqpoint{2.168613in}{0.917077in}}%
\pgfpathcurveto{\pgfqpoint{2.172520in}{0.913170in}}{\pgfqpoint{2.177819in}{0.910975in}}{\pgfqpoint{2.183344in}{0.910975in}}%
\pgfpathclose%
\pgfusepath{fill}%
\end{pgfscope}%
\begin{pgfscope}%
\pgfpathrectangle{\pgfqpoint{2.051725in}{0.913832in}}{\pgfqpoint{1.162500in}{0.755000in}}%
\pgfusepath{clip}%
\pgfsetbuttcap%
\pgfsetroundjoin%
\definecolor{currentfill}{rgb}{0.000000,0.000000,0.000000}%
\pgfsetfillcolor{currentfill}%
\pgfsetfillopacity{0.500000}%
\pgfsetlinewidth{0.000000pt}%
\definecolor{currentstroke}{rgb}{0.000000,0.000000,0.000000}%
\pgfsetstrokecolor{currentstroke}%
\pgfsetdash{}{0pt}%
\pgfpathmoveto{\pgfqpoint{2.082232in}{1.042931in}}%
\pgfpathcurveto{\pgfqpoint{2.087757in}{1.042931in}}{\pgfqpoint{2.093056in}{1.045126in}}{\pgfqpoint{2.096963in}{1.049033in}}%
\pgfpathcurveto{\pgfqpoint{2.100870in}{1.052940in}}{\pgfqpoint{2.103065in}{1.058239in}}{\pgfqpoint{2.103065in}{1.063764in}}%
\pgfpathcurveto{\pgfqpoint{2.103065in}{1.069289in}}{\pgfqpoint{2.100870in}{1.074589in}}{\pgfqpoint{2.096963in}{1.078495in}}%
\pgfpathcurveto{\pgfqpoint{2.093056in}{1.082402in}}{\pgfqpoint{2.087757in}{1.084597in}}{\pgfqpoint{2.082232in}{1.084597in}}%
\pgfpathcurveto{\pgfqpoint{2.076707in}{1.084597in}}{\pgfqpoint{2.071407in}{1.082402in}}{\pgfqpoint{2.067500in}{1.078495in}}%
\pgfpathcurveto{\pgfqpoint{2.063594in}{1.074589in}}{\pgfqpoint{2.061398in}{1.069289in}}{\pgfqpoint{2.061398in}{1.063764in}}%
\pgfpathcurveto{\pgfqpoint{2.061398in}{1.058239in}}{\pgfqpoint{2.063594in}{1.052940in}}{\pgfqpoint{2.067500in}{1.049033in}}%
\pgfpathcurveto{\pgfqpoint{2.071407in}{1.045126in}}{\pgfqpoint{2.076707in}{1.042931in}}{\pgfqpoint{2.082232in}{1.042931in}}%
\pgfpathclose%
\pgfusepath{fill}%
\end{pgfscope}%
\begin{pgfscope}%
\pgfpathrectangle{\pgfqpoint{2.051725in}{0.913832in}}{\pgfqpoint{1.162500in}{0.755000in}}%
\pgfusepath{clip}%
\pgfsetbuttcap%
\pgfsetroundjoin%
\definecolor{currentfill}{rgb}{0.000000,0.000000,0.000000}%
\pgfsetfillcolor{currentfill}%
\pgfsetfillopacity{0.500000}%
\pgfsetlinewidth{0.000000pt}%
\definecolor{currentstroke}{rgb}{0.000000,0.000000,0.000000}%
\pgfsetstrokecolor{currentstroke}%
\pgfsetdash{}{0pt}%
\pgfpathmoveto{\pgfqpoint{2.079404in}{1.035607in}}%
\pgfpathcurveto{\pgfqpoint{2.084929in}{1.035607in}}{\pgfqpoint{2.090228in}{1.037803in}}{\pgfqpoint{2.094135in}{1.041709in}}%
\pgfpathcurveto{\pgfqpoint{2.098042in}{1.045616in}}{\pgfqpoint{2.100237in}{1.050916in}}{\pgfqpoint{2.100237in}{1.056441in}}%
\pgfpathcurveto{\pgfqpoint{2.100237in}{1.061966in}}{\pgfqpoint{2.098042in}{1.067265in}}{\pgfqpoint{2.094135in}{1.071172in}}%
\pgfpathcurveto{\pgfqpoint{2.090228in}{1.075079in}}{\pgfqpoint{2.084929in}{1.077274in}}{\pgfqpoint{2.079404in}{1.077274in}}%
\pgfpathcurveto{\pgfqpoint{2.073879in}{1.077274in}}{\pgfqpoint{2.068579in}{1.075079in}}{\pgfqpoint{2.064672in}{1.071172in}}%
\pgfpathcurveto{\pgfqpoint{2.060765in}{1.067265in}}{\pgfqpoint{2.058570in}{1.061966in}}{\pgfqpoint{2.058570in}{1.056441in}}%
\pgfpathcurveto{\pgfqpoint{2.058570in}{1.050916in}}{\pgfqpoint{2.060765in}{1.045616in}}{\pgfqpoint{2.064672in}{1.041709in}}%
\pgfpathcurveto{\pgfqpoint{2.068579in}{1.037803in}}{\pgfqpoint{2.073879in}{1.035607in}}{\pgfqpoint{2.079404in}{1.035607in}}%
\pgfpathclose%
\pgfusepath{fill}%
\end{pgfscope}%
\begin{pgfscope}%
\pgfpathrectangle{\pgfqpoint{2.051725in}{0.913832in}}{\pgfqpoint{1.162500in}{0.755000in}}%
\pgfusepath{clip}%
\pgfsetbuttcap%
\pgfsetroundjoin%
\definecolor{currentfill}{rgb}{0.000000,0.000000,0.000000}%
\pgfsetfillcolor{currentfill}%
\pgfsetfillopacity{0.500000}%
\pgfsetlinewidth{0.000000pt}%
\definecolor{currentstroke}{rgb}{0.000000,0.000000,0.000000}%
\pgfsetstrokecolor{currentstroke}%
\pgfsetdash{}{0pt}%
\pgfpathmoveto{\pgfqpoint{2.536683in}{1.572595in}}%
\pgfpathcurveto{\pgfqpoint{2.542208in}{1.572595in}}{\pgfqpoint{2.547508in}{1.574791in}}{\pgfqpoint{2.551415in}{1.578697in}}%
\pgfpathcurveto{\pgfqpoint{2.555322in}{1.582604in}}{\pgfqpoint{2.557517in}{1.587904in}}{\pgfqpoint{2.557517in}{1.593429in}}%
\pgfpathcurveto{\pgfqpoint{2.557517in}{1.598954in}}{\pgfqpoint{2.555322in}{1.604253in}}{\pgfqpoint{2.551415in}{1.608160in}}%
\pgfpathcurveto{\pgfqpoint{2.547508in}{1.612067in}}{\pgfqpoint{2.542208in}{1.614262in}}{\pgfqpoint{2.536683in}{1.614262in}}%
\pgfpathcurveto{\pgfqpoint{2.531158in}{1.614262in}}{\pgfqpoint{2.525859in}{1.612067in}}{\pgfqpoint{2.521952in}{1.608160in}}%
\pgfpathcurveto{\pgfqpoint{2.518045in}{1.604253in}}{\pgfqpoint{2.515850in}{1.598954in}}{\pgfqpoint{2.515850in}{1.593429in}}%
\pgfpathcurveto{\pgfqpoint{2.515850in}{1.587904in}}{\pgfqpoint{2.518045in}{1.582604in}}{\pgfqpoint{2.521952in}{1.578697in}}%
\pgfpathcurveto{\pgfqpoint{2.525859in}{1.574791in}}{\pgfqpoint{2.531158in}{1.572595in}}{\pgfqpoint{2.536683in}{1.572595in}}%
\pgfpathclose%
\pgfusepath{fill}%
\end{pgfscope}%
\begin{pgfscope}%
\pgfpathrectangle{\pgfqpoint{2.051725in}{0.913832in}}{\pgfqpoint{1.162500in}{0.755000in}}%
\pgfusepath{clip}%
\pgfsetbuttcap%
\pgfsetroundjoin%
\definecolor{currentfill}{rgb}{0.000000,0.000000,0.000000}%
\pgfsetfillcolor{currentfill}%
\pgfsetfillopacity{0.500000}%
\pgfsetlinewidth{0.000000pt}%
\definecolor{currentstroke}{rgb}{0.000000,0.000000,0.000000}%
\pgfsetstrokecolor{currentstroke}%
\pgfsetdash{}{0pt}%
\pgfpathmoveto{\pgfqpoint{2.417381in}{1.630023in}}%
\pgfpathcurveto{\pgfqpoint{2.422906in}{1.630023in}}{\pgfqpoint{2.428205in}{1.632218in}}{\pgfqpoint{2.432112in}{1.636125in}}%
\pgfpathcurveto{\pgfqpoint{2.436019in}{1.640032in}}{\pgfqpoint{2.438214in}{1.645331in}}{\pgfqpoint{2.438214in}{1.650856in}}%
\pgfpathcurveto{\pgfqpoint{2.438214in}{1.656381in}}{\pgfqpoint{2.436019in}{1.661681in}}{\pgfqpoint{2.432112in}{1.665588in}}%
\pgfpathcurveto{\pgfqpoint{2.428205in}{1.669494in}}{\pgfqpoint{2.422906in}{1.671690in}}{\pgfqpoint{2.417381in}{1.671690in}}%
\pgfpathcurveto{\pgfqpoint{2.411856in}{1.671690in}}{\pgfqpoint{2.406556in}{1.669494in}}{\pgfqpoint{2.402649in}{1.665588in}}%
\pgfpathcurveto{\pgfqpoint{2.398743in}{1.661681in}}{\pgfqpoint{2.396547in}{1.656381in}}{\pgfqpoint{2.396547in}{1.650856in}}%
\pgfpathcurveto{\pgfqpoint{2.396547in}{1.645331in}}{\pgfqpoint{2.398743in}{1.640032in}}{\pgfqpoint{2.402649in}{1.636125in}}%
\pgfpathcurveto{\pgfqpoint{2.406556in}{1.632218in}}{\pgfqpoint{2.411856in}{1.630023in}}{\pgfqpoint{2.417381in}{1.630023in}}%
\pgfpathclose%
\pgfusepath{fill}%
\end{pgfscope}%
\begin{pgfscope}%
\pgfpathrectangle{\pgfqpoint{2.051725in}{0.913832in}}{\pgfqpoint{1.162500in}{0.755000in}}%
\pgfusepath{clip}%
\pgfsetbuttcap%
\pgfsetroundjoin%
\definecolor{currentfill}{rgb}{0.000000,0.000000,0.000000}%
\pgfsetfillcolor{currentfill}%
\pgfsetfillopacity{0.500000}%
\pgfsetlinewidth{0.000000pt}%
\definecolor{currentstroke}{rgb}{0.000000,0.000000,0.000000}%
\pgfsetstrokecolor{currentstroke}%
\pgfsetdash{}{0pt}%
\pgfpathmoveto{\pgfqpoint{2.412630in}{1.484185in}}%
\pgfpathcurveto{\pgfqpoint{2.418155in}{1.484185in}}{\pgfqpoint{2.423454in}{1.486380in}}{\pgfqpoint{2.427361in}{1.490287in}}%
\pgfpathcurveto{\pgfqpoint{2.431268in}{1.494194in}}{\pgfqpoint{2.433463in}{1.499493in}}{\pgfqpoint{2.433463in}{1.505018in}}%
\pgfpathcurveto{\pgfqpoint{2.433463in}{1.510544in}}{\pgfqpoint{2.431268in}{1.515843in}}{\pgfqpoint{2.427361in}{1.519750in}}%
\pgfpathcurveto{\pgfqpoint{2.423454in}{1.523657in}}{\pgfqpoint{2.418155in}{1.525852in}}{\pgfqpoint{2.412630in}{1.525852in}}%
\pgfpathcurveto{\pgfqpoint{2.407105in}{1.525852in}}{\pgfqpoint{2.401805in}{1.523657in}}{\pgfqpoint{2.397898in}{1.519750in}}%
\pgfpathcurveto{\pgfqpoint{2.393991in}{1.515843in}}{\pgfqpoint{2.391796in}{1.510544in}}{\pgfqpoint{2.391796in}{1.505018in}}%
\pgfpathcurveto{\pgfqpoint{2.391796in}{1.499493in}}{\pgfqpoint{2.393991in}{1.494194in}}{\pgfqpoint{2.397898in}{1.490287in}}%
\pgfpathcurveto{\pgfqpoint{2.401805in}{1.486380in}}{\pgfqpoint{2.407105in}{1.484185in}}{\pgfqpoint{2.412630in}{1.484185in}}%
\pgfpathclose%
\pgfusepath{fill}%
\end{pgfscope}%
\begin{pgfscope}%
\pgfpathrectangle{\pgfqpoint{2.051725in}{0.913832in}}{\pgfqpoint{1.162500in}{0.755000in}}%
\pgfusepath{clip}%
\pgfsetbuttcap%
\pgfsetroundjoin%
\definecolor{currentfill}{rgb}{0.000000,0.000000,0.000000}%
\pgfsetfillcolor{currentfill}%
\pgfsetfillopacity{0.500000}%
\pgfsetlinewidth{0.000000pt}%
\definecolor{currentstroke}{rgb}{0.000000,0.000000,0.000000}%
\pgfsetstrokecolor{currentstroke}%
\pgfsetdash{}{0pt}%
\pgfpathmoveto{\pgfqpoint{2.478188in}{1.067768in}}%
\pgfpathcurveto{\pgfqpoint{2.483713in}{1.067768in}}{\pgfqpoint{2.489012in}{1.069963in}}{\pgfqpoint{2.492919in}{1.073870in}}%
\pgfpathcurveto{\pgfqpoint{2.496826in}{1.077776in}}{\pgfqpoint{2.499021in}{1.083076in}}{\pgfqpoint{2.499021in}{1.088601in}}%
\pgfpathcurveto{\pgfqpoint{2.499021in}{1.094126in}}{\pgfqpoint{2.496826in}{1.099426in}}{\pgfqpoint{2.492919in}{1.103332in}}%
\pgfpathcurveto{\pgfqpoint{2.489012in}{1.107239in}}{\pgfqpoint{2.483713in}{1.109434in}}{\pgfqpoint{2.478188in}{1.109434in}}%
\pgfpathcurveto{\pgfqpoint{2.472663in}{1.109434in}}{\pgfqpoint{2.467363in}{1.107239in}}{\pgfqpoint{2.463456in}{1.103332in}}%
\pgfpathcurveto{\pgfqpoint{2.459549in}{1.099426in}}{\pgfqpoint{2.457354in}{1.094126in}}{\pgfqpoint{2.457354in}{1.088601in}}%
\pgfpathcurveto{\pgfqpoint{2.457354in}{1.083076in}}{\pgfqpoint{2.459549in}{1.077776in}}{\pgfqpoint{2.463456in}{1.073870in}}%
\pgfpathcurveto{\pgfqpoint{2.467363in}{1.069963in}}{\pgfqpoint{2.472663in}{1.067768in}}{\pgfqpoint{2.478188in}{1.067768in}}%
\pgfpathclose%
\pgfusepath{fill}%
\end{pgfscope}%
\begin{pgfscope}%
\pgfpathrectangle{\pgfqpoint{2.051725in}{0.913832in}}{\pgfqpoint{1.162500in}{0.755000in}}%
\pgfusepath{clip}%
\pgfsetbuttcap%
\pgfsetroundjoin%
\definecolor{currentfill}{rgb}{0.000000,0.000000,0.000000}%
\pgfsetfillcolor{currentfill}%
\pgfsetfillopacity{0.500000}%
\pgfsetlinewidth{0.000000pt}%
\definecolor{currentstroke}{rgb}{0.000000,0.000000,0.000000}%
\pgfsetstrokecolor{currentstroke}%
\pgfsetdash{}{0pt}%
\pgfpathmoveto{\pgfqpoint{2.172291in}{1.343150in}}%
\pgfpathcurveto{\pgfqpoint{2.177816in}{1.343150in}}{\pgfqpoint{2.183116in}{1.345345in}}{\pgfqpoint{2.187022in}{1.349252in}}%
\pgfpathcurveto{\pgfqpoint{2.190929in}{1.353159in}}{\pgfqpoint{2.193124in}{1.358458in}}{\pgfqpoint{2.193124in}{1.363984in}}%
\pgfpathcurveto{\pgfqpoint{2.193124in}{1.369509in}}{\pgfqpoint{2.190929in}{1.374808in}}{\pgfqpoint{2.187022in}{1.378715in}}%
\pgfpathcurveto{\pgfqpoint{2.183116in}{1.382622in}}{\pgfqpoint{2.177816in}{1.384817in}}{\pgfqpoint{2.172291in}{1.384817in}}%
\pgfpathcurveto{\pgfqpoint{2.166766in}{1.384817in}}{\pgfqpoint{2.161467in}{1.382622in}}{\pgfqpoint{2.157560in}{1.378715in}}%
\pgfpathcurveto{\pgfqpoint{2.153653in}{1.374808in}}{\pgfqpoint{2.151458in}{1.369509in}}{\pgfqpoint{2.151458in}{1.363984in}}%
\pgfpathcurveto{\pgfqpoint{2.151458in}{1.358458in}}{\pgfqpoint{2.153653in}{1.353159in}}{\pgfqpoint{2.157560in}{1.349252in}}%
\pgfpathcurveto{\pgfqpoint{2.161467in}{1.345345in}}{\pgfqpoint{2.166766in}{1.343150in}}{\pgfqpoint{2.172291in}{1.343150in}}%
\pgfpathclose%
\pgfusepath{fill}%
\end{pgfscope}%
\begin{pgfscope}%
\pgfpathrectangle{\pgfqpoint{2.051725in}{0.913832in}}{\pgfqpoint{1.162500in}{0.755000in}}%
\pgfusepath{clip}%
\pgfsetbuttcap%
\pgfsetroundjoin%
\definecolor{currentfill}{rgb}{0.000000,0.000000,0.000000}%
\pgfsetfillcolor{currentfill}%
\pgfsetfillopacity{0.500000}%
\pgfsetlinewidth{0.000000pt}%
\definecolor{currentstroke}{rgb}{0.000000,0.000000,0.000000}%
\pgfsetstrokecolor{currentstroke}%
\pgfsetdash{}{0pt}%
\pgfpathmoveto{\pgfqpoint{2.205718in}{1.259718in}}%
\pgfpathcurveto{\pgfqpoint{2.211243in}{1.259718in}}{\pgfqpoint{2.216543in}{1.261913in}}{\pgfqpoint{2.220450in}{1.265820in}}%
\pgfpathcurveto{\pgfqpoint{2.224357in}{1.269726in}}{\pgfqpoint{2.226552in}{1.275026in}}{\pgfqpoint{2.226552in}{1.280551in}}%
\pgfpathcurveto{\pgfqpoint{2.226552in}{1.286076in}}{\pgfqpoint{2.224357in}{1.291376in}}{\pgfqpoint{2.220450in}{1.295282in}}%
\pgfpathcurveto{\pgfqpoint{2.216543in}{1.299189in}}{\pgfqpoint{2.211243in}{1.301384in}}{\pgfqpoint{2.205718in}{1.301384in}}%
\pgfpathcurveto{\pgfqpoint{2.200193in}{1.301384in}}{\pgfqpoint{2.194894in}{1.299189in}}{\pgfqpoint{2.190987in}{1.295282in}}%
\pgfpathcurveto{\pgfqpoint{2.187080in}{1.291376in}}{\pgfqpoint{2.184885in}{1.286076in}}{\pgfqpoint{2.184885in}{1.280551in}}%
\pgfpathcurveto{\pgfqpoint{2.184885in}{1.275026in}}{\pgfqpoint{2.187080in}{1.269726in}}{\pgfqpoint{2.190987in}{1.265820in}}%
\pgfpathcurveto{\pgfqpoint{2.194894in}{1.261913in}}{\pgfqpoint{2.200193in}{1.259718in}}{\pgfqpoint{2.205718in}{1.259718in}}%
\pgfpathclose%
\pgfusepath{fill}%
\end{pgfscope}%
\begin{pgfscope}%
\pgfpathrectangle{\pgfqpoint{2.051725in}{0.913832in}}{\pgfqpoint{1.162500in}{0.755000in}}%
\pgfusepath{clip}%
\pgfsetbuttcap%
\pgfsetroundjoin%
\definecolor{currentfill}{rgb}{0.000000,0.000000,0.000000}%
\pgfsetfillcolor{currentfill}%
\pgfsetfillopacity{0.500000}%
\pgfsetlinewidth{0.000000pt}%
\definecolor{currentstroke}{rgb}{0.000000,0.000000,0.000000}%
\pgfsetstrokecolor{currentstroke}%
\pgfsetdash{}{0pt}%
\pgfpathmoveto{\pgfqpoint{3.186546in}{1.167350in}}%
\pgfpathcurveto{\pgfqpoint{3.192071in}{1.167350in}}{\pgfqpoint{3.197371in}{1.169545in}}{\pgfqpoint{3.201278in}{1.173452in}}%
\pgfpathcurveto{\pgfqpoint{3.205185in}{1.177359in}}{\pgfqpoint{3.207380in}{1.182658in}}{\pgfqpoint{3.207380in}{1.188183in}}%
\pgfpathcurveto{\pgfqpoint{3.207380in}{1.193708in}}{\pgfqpoint{3.205185in}{1.199008in}}{\pgfqpoint{3.201278in}{1.202915in}}%
\pgfpathcurveto{\pgfqpoint{3.197371in}{1.206822in}}{\pgfqpoint{3.192071in}{1.209017in}}{\pgfqpoint{3.186546in}{1.209017in}}%
\pgfpathcurveto{\pgfqpoint{3.181021in}{1.209017in}}{\pgfqpoint{3.175722in}{1.206822in}}{\pgfqpoint{3.171815in}{1.202915in}}%
\pgfpathcurveto{\pgfqpoint{3.167908in}{1.199008in}}{\pgfqpoint{3.165713in}{1.193708in}}{\pgfqpoint{3.165713in}{1.188183in}}%
\pgfpathcurveto{\pgfqpoint{3.165713in}{1.182658in}}{\pgfqpoint{3.167908in}{1.177359in}}{\pgfqpoint{3.171815in}{1.173452in}}%
\pgfpathcurveto{\pgfqpoint{3.175722in}{1.169545in}}{\pgfqpoint{3.181021in}{1.167350in}}{\pgfqpoint{3.186546in}{1.167350in}}%
\pgfpathclose%
\pgfusepath{fill}%
\end{pgfscope}%
\begin{pgfscope}%
\pgfpathrectangle{\pgfqpoint{2.051725in}{0.913832in}}{\pgfqpoint{1.162500in}{0.755000in}}%
\pgfusepath{clip}%
\pgfsetbuttcap%
\pgfsetroundjoin%
\definecolor{currentfill}{rgb}{0.000000,0.000000,0.000000}%
\pgfsetfillcolor{currentfill}%
\pgfsetfillopacity{0.500000}%
\pgfsetlinewidth{0.000000pt}%
\definecolor{currentstroke}{rgb}{0.000000,0.000000,0.000000}%
\pgfsetstrokecolor{currentstroke}%
\pgfsetdash{}{0pt}%
\pgfpathmoveto{\pgfqpoint{2.394110in}{1.034902in}}%
\pgfpathcurveto{\pgfqpoint{2.399635in}{1.034902in}}{\pgfqpoint{2.404935in}{1.037097in}}{\pgfqpoint{2.408842in}{1.041004in}}%
\pgfpathcurveto{\pgfqpoint{2.412749in}{1.044911in}}{\pgfqpoint{2.414944in}{1.050211in}}{\pgfqpoint{2.414944in}{1.055736in}}%
\pgfpathcurveto{\pgfqpoint{2.414944in}{1.061261in}}{\pgfqpoint{2.412749in}{1.066560in}}{\pgfqpoint{2.408842in}{1.070467in}}%
\pgfpathcurveto{\pgfqpoint{2.404935in}{1.074374in}}{\pgfqpoint{2.399635in}{1.076569in}}{\pgfqpoint{2.394110in}{1.076569in}}%
\pgfpathcurveto{\pgfqpoint{2.388585in}{1.076569in}}{\pgfqpoint{2.383286in}{1.074374in}}{\pgfqpoint{2.379379in}{1.070467in}}%
\pgfpathcurveto{\pgfqpoint{2.375472in}{1.066560in}}{\pgfqpoint{2.373277in}{1.061261in}}{\pgfqpoint{2.373277in}{1.055736in}}%
\pgfpathcurveto{\pgfqpoint{2.373277in}{1.050211in}}{\pgfqpoint{2.375472in}{1.044911in}}{\pgfqpoint{2.379379in}{1.041004in}}%
\pgfpathcurveto{\pgfqpoint{2.383286in}{1.037097in}}{\pgfqpoint{2.388585in}{1.034902in}}{\pgfqpoint{2.394110in}{1.034902in}}%
\pgfpathclose%
\pgfusepath{fill}%
\end{pgfscope}%
\begin{pgfscope}%
\pgfpathrectangle{\pgfqpoint{2.051725in}{0.913832in}}{\pgfqpoint{1.162500in}{0.755000in}}%
\pgfusepath{clip}%
\pgfsetbuttcap%
\pgfsetroundjoin%
\definecolor{currentfill}{rgb}{0.000000,0.000000,0.000000}%
\pgfsetfillcolor{currentfill}%
\pgfsetfillopacity{0.500000}%
\pgfsetlinewidth{0.000000pt}%
\definecolor{currentstroke}{rgb}{0.000000,0.000000,0.000000}%
\pgfsetstrokecolor{currentstroke}%
\pgfsetdash{}{0pt}%
\pgfpathmoveto{\pgfqpoint{2.125339in}{1.190157in}}%
\pgfpathcurveto{\pgfqpoint{2.130865in}{1.190157in}}{\pgfqpoint{2.136164in}{1.192352in}}{\pgfqpoint{2.140071in}{1.196259in}}%
\pgfpathcurveto{\pgfqpoint{2.143978in}{1.200166in}}{\pgfqpoint{2.146173in}{1.205465in}}{\pgfqpoint{2.146173in}{1.210991in}}%
\pgfpathcurveto{\pgfqpoint{2.146173in}{1.216516in}}{\pgfqpoint{2.143978in}{1.221815in}}{\pgfqpoint{2.140071in}{1.225722in}}%
\pgfpathcurveto{\pgfqpoint{2.136164in}{1.229629in}}{\pgfqpoint{2.130865in}{1.231824in}}{\pgfqpoint{2.125339in}{1.231824in}}%
\pgfpathcurveto{\pgfqpoint{2.119814in}{1.231824in}}{\pgfqpoint{2.114515in}{1.229629in}}{\pgfqpoint{2.110608in}{1.225722in}}%
\pgfpathcurveto{\pgfqpoint{2.106701in}{1.221815in}}{\pgfqpoint{2.104506in}{1.216516in}}{\pgfqpoint{2.104506in}{1.210991in}}%
\pgfpathcurveto{\pgfqpoint{2.104506in}{1.205465in}}{\pgfqpoint{2.106701in}{1.200166in}}{\pgfqpoint{2.110608in}{1.196259in}}%
\pgfpathcurveto{\pgfqpoint{2.114515in}{1.192352in}}{\pgfqpoint{2.119814in}{1.190157in}}{\pgfqpoint{2.125339in}{1.190157in}}%
\pgfpathclose%
\pgfusepath{fill}%
\end{pgfscope}%
\begin{pgfscope}%
\pgfsetbuttcap%
\pgfsetroundjoin%
\definecolor{currentfill}{rgb}{0.000000,0.000000,0.000000}%
\pgfsetfillcolor{currentfill}%
\pgfsetlinewidth{0.803000pt}%
\definecolor{currentstroke}{rgb}{0.000000,0.000000,0.000000}%
\pgfsetstrokecolor{currentstroke}%
\pgfsetdash{}{0pt}%
\pgfsys@defobject{currentmarker}{\pgfqpoint{0.000000in}{-0.048611in}}{\pgfqpoint{0.000000in}{0.000000in}}{%
\pgfpathmoveto{\pgfqpoint{0.000000in}{0.000000in}}%
\pgfpathlineto{\pgfqpoint{0.000000in}{-0.048611in}}%
\pgfusepath{stroke,fill}%
}%
\begin{pgfscope}%
\pgfsys@transformshift{2.463280in}{0.913832in}%
\pgfsys@useobject{currentmarker}{}%
\end{pgfscope}%
\end{pgfscope}%
\begin{pgfscope}%
\pgftext[x=2.493933in,y=0.665759in,left,base,rotate=90.000000]{\rmfamily\fontsize{8.000000}{9.600000}\selectfont \(\displaystyle 0.5\)}%
\end{pgfscope}%
\begin{pgfscope}%
\pgfsetbuttcap%
\pgfsetroundjoin%
\definecolor{currentfill}{rgb}{0.000000,0.000000,0.000000}%
\pgfsetfillcolor{currentfill}%
\pgfsetlinewidth{0.803000pt}%
\definecolor{currentstroke}{rgb}{0.000000,0.000000,0.000000}%
\pgfsetstrokecolor{currentstroke}%
\pgfsetdash{}{0pt}%
\pgfsys@defobject{currentmarker}{\pgfqpoint{0.000000in}{-0.048611in}}{\pgfqpoint{0.000000in}{0.000000in}}{%
\pgfpathmoveto{\pgfqpoint{0.000000in}{0.000000in}}%
\pgfpathlineto{\pgfqpoint{0.000000in}{-0.048611in}}%
\pgfusepath{stroke,fill}%
}%
\begin{pgfscope}%
\pgfsys@transformshift{2.916252in}{0.913832in}%
\pgfsys@useobject{currentmarker}{}%
\end{pgfscope}%
\end{pgfscope}%
\begin{pgfscope}%
\pgftext[x=2.946906in,y=0.665759in,left,base,rotate=90.000000]{\rmfamily\fontsize{8.000000}{9.600000}\selectfont \(\displaystyle 1.0\)}%
\end{pgfscope}%
\begin{pgfscope}%
\pgftext[x=2.632975in,y=0.610204in,,top]{\rmfamily\fontsize{16.000000}{19.200000}\selectfont charge}%
\end{pgfscope}%
\begin{pgfscope}%
\pgftext[x=3.214225in,y=0.624092in,right,top]{\rmfamily\fontsize{16.000000}{19.200000}\selectfont \(\displaystyle \times10^{-9}\)}%
\end{pgfscope}%
\begin{pgfscope}%
\pgfsetrectcap%
\pgfsetmiterjoin%
\pgfsetlinewidth{0.803000pt}%
\definecolor{currentstroke}{rgb}{0.501961,0.501961,0.501961}%
\pgfsetstrokecolor{currentstroke}%
\pgfsetdash{}{0pt}%
\pgfpathmoveto{\pgfqpoint{2.051725in}{0.913832in}}%
\pgfpathlineto{\pgfqpoint{2.051725in}{1.668832in}}%
\pgfusepath{stroke}%
\end{pgfscope}%
\begin{pgfscope}%
\pgfsetrectcap%
\pgfsetmiterjoin%
\pgfsetlinewidth{0.803000pt}%
\definecolor{currentstroke}{rgb}{0.501961,0.501961,0.501961}%
\pgfsetstrokecolor{currentstroke}%
\pgfsetdash{}{0pt}%
\pgfpathmoveto{\pgfqpoint{3.214225in}{0.913832in}}%
\pgfpathlineto{\pgfqpoint{3.214225in}{1.668832in}}%
\pgfusepath{stroke}%
\end{pgfscope}%
\begin{pgfscope}%
\pgfsetrectcap%
\pgfsetmiterjoin%
\pgfsetlinewidth{0.803000pt}%
\definecolor{currentstroke}{rgb}{0.501961,0.501961,0.501961}%
\pgfsetstrokecolor{currentstroke}%
\pgfsetdash{}{0pt}%
\pgfpathmoveto{\pgfqpoint{2.051725in}{0.913832in}}%
\pgfpathlineto{\pgfqpoint{3.214225in}{0.913832in}}%
\pgfusepath{stroke}%
\end{pgfscope}%
\begin{pgfscope}%
\pgfsetrectcap%
\pgfsetmiterjoin%
\pgfsetlinewidth{0.803000pt}%
\definecolor{currentstroke}{rgb}{0.501961,0.501961,0.501961}%
\pgfsetstrokecolor{currentstroke}%
\pgfsetdash{}{0pt}%
\pgfpathmoveto{\pgfqpoint{2.051725in}{1.668832in}}%
\pgfpathlineto{\pgfqpoint{3.214225in}{1.668832in}}%
\pgfusepath{stroke}%
\end{pgfscope}%
\begin{pgfscope}%
\pgfsetbuttcap%
\pgfsetmiterjoin%
\definecolor{currentfill}{rgb}{1.000000,1.000000,1.000000}%
\pgfsetfillcolor{currentfill}%
\pgfsetlinewidth{0.000000pt}%
\definecolor{currentstroke}{rgb}{0.000000,0.000000,0.000000}%
\pgfsetstrokecolor{currentstroke}%
\pgfsetstrokeopacity{0.000000}%
\pgfsetdash{}{0pt}%
\pgfpathmoveto{\pgfqpoint{3.214225in}{0.913832in}}%
\pgfpathlineto{\pgfqpoint{4.376725in}{0.913832in}}%
\pgfpathlineto{\pgfqpoint{4.376725in}{1.668832in}}%
\pgfpathlineto{\pgfqpoint{3.214225in}{1.668832in}}%
\pgfpathclose%
\pgfusepath{fill}%
\end{pgfscope}%
\begin{pgfscope}%
\pgfpathrectangle{\pgfqpoint{3.214225in}{0.913832in}}{\pgfqpoint{1.162500in}{0.755000in}}%
\pgfusepath{clip}%
\pgfsetbuttcap%
\pgfsetroundjoin%
\definecolor{currentfill}{rgb}{0.000000,0.000000,0.000000}%
\pgfsetfillcolor{currentfill}%
\pgfsetfillopacity{0.500000}%
\pgfsetlinewidth{0.000000pt}%
\definecolor{currentstroke}{rgb}{0.000000,0.000000,0.000000}%
\pgfsetstrokecolor{currentstroke}%
\pgfsetdash{}{0pt}%
\pgfpathmoveto{\pgfqpoint{3.656364in}{1.437401in}}%
\pgfpathcurveto{\pgfqpoint{3.661889in}{1.437401in}}{\pgfqpoint{3.667189in}{1.439596in}}{\pgfqpoint{3.671096in}{1.443503in}}%
\pgfpathcurveto{\pgfqpoint{3.675002in}{1.447410in}}{\pgfqpoint{3.677198in}{1.452710in}}{\pgfqpoint{3.677198in}{1.458235in}}%
\pgfpathcurveto{\pgfqpoint{3.677198in}{1.463760in}}{\pgfqpoint{3.675002in}{1.469059in}}{\pgfqpoint{3.671096in}{1.472966in}}%
\pgfpathcurveto{\pgfqpoint{3.667189in}{1.476873in}}{\pgfqpoint{3.661889in}{1.479068in}}{\pgfqpoint{3.656364in}{1.479068in}}%
\pgfpathcurveto{\pgfqpoint{3.650839in}{1.479068in}}{\pgfqpoint{3.645540in}{1.476873in}}{\pgfqpoint{3.641633in}{1.472966in}}%
\pgfpathcurveto{\pgfqpoint{3.637726in}{1.469059in}}{\pgfqpoint{3.635531in}{1.463760in}}{\pgfqpoint{3.635531in}{1.458235in}}%
\pgfpathcurveto{\pgfqpoint{3.635531in}{1.452710in}}{\pgfqpoint{3.637726in}{1.447410in}}{\pgfqpoint{3.641633in}{1.443503in}}%
\pgfpathcurveto{\pgfqpoint{3.645540in}{1.439596in}}{\pgfqpoint{3.650839in}{1.437401in}}{\pgfqpoint{3.656364in}{1.437401in}}%
\pgfpathclose%
\pgfusepath{fill}%
\end{pgfscope}%
\begin{pgfscope}%
\pgfpathrectangle{\pgfqpoint{3.214225in}{0.913832in}}{\pgfqpoint{1.162500in}{0.755000in}}%
\pgfusepath{clip}%
\pgfsetbuttcap%
\pgfsetroundjoin%
\definecolor{currentfill}{rgb}{0.000000,0.000000,0.000000}%
\pgfsetfillcolor{currentfill}%
\pgfsetfillopacity{0.500000}%
\pgfsetlinewidth{0.000000pt}%
\definecolor{currentstroke}{rgb}{0.000000,0.000000,0.000000}%
\pgfsetstrokecolor{currentstroke}%
\pgfsetdash{}{0pt}%
\pgfpathmoveto{\pgfqpoint{4.259629in}{0.987180in}}%
\pgfpathcurveto{\pgfqpoint{4.265154in}{0.987180in}}{\pgfqpoint{4.270454in}{0.989375in}}{\pgfqpoint{4.274361in}{0.993282in}}%
\pgfpathcurveto{\pgfqpoint{4.278268in}{0.997189in}}{\pgfqpoint{4.280463in}{1.002488in}}{\pgfqpoint{4.280463in}{1.008013in}}%
\pgfpathcurveto{\pgfqpoint{4.280463in}{1.013538in}}{\pgfqpoint{4.278268in}{1.018838in}}{\pgfqpoint{4.274361in}{1.022745in}}%
\pgfpathcurveto{\pgfqpoint{4.270454in}{1.026652in}}{\pgfqpoint{4.265154in}{1.028847in}}{\pgfqpoint{4.259629in}{1.028847in}}%
\pgfpathcurveto{\pgfqpoint{4.254104in}{1.028847in}}{\pgfqpoint{4.248805in}{1.026652in}}{\pgfqpoint{4.244898in}{1.022745in}}%
\pgfpathcurveto{\pgfqpoint{4.240991in}{1.018838in}}{\pgfqpoint{4.238796in}{1.013538in}}{\pgfqpoint{4.238796in}{1.008013in}}%
\pgfpathcurveto{\pgfqpoint{4.238796in}{1.002488in}}{\pgfqpoint{4.240991in}{0.997189in}}{\pgfqpoint{4.244898in}{0.993282in}}%
\pgfpathcurveto{\pgfqpoint{4.248805in}{0.989375in}}{\pgfqpoint{4.254104in}{0.987180in}}{\pgfqpoint{4.259629in}{0.987180in}}%
\pgfpathclose%
\pgfusepath{fill}%
\end{pgfscope}%
\begin{pgfscope}%
\pgfpathrectangle{\pgfqpoint{3.214225in}{0.913832in}}{\pgfqpoint{1.162500in}{0.755000in}}%
\pgfusepath{clip}%
\pgfsetbuttcap%
\pgfsetroundjoin%
\definecolor{currentfill}{rgb}{0.000000,0.000000,0.000000}%
\pgfsetfillcolor{currentfill}%
\pgfsetfillopacity{0.500000}%
\pgfsetlinewidth{0.000000pt}%
\definecolor{currentstroke}{rgb}{0.000000,0.000000,0.000000}%
\pgfsetstrokecolor{currentstroke}%
\pgfsetdash{}{0pt}%
\pgfpathmoveto{\pgfqpoint{4.349046in}{0.939886in}}%
\pgfpathcurveto{\pgfqpoint{4.354571in}{0.939886in}}{\pgfqpoint{4.359871in}{0.942081in}}{\pgfqpoint{4.363778in}{0.945988in}}%
\pgfpathcurveto{\pgfqpoint{4.367685in}{0.949895in}}{\pgfqpoint{4.369880in}{0.955195in}}{\pgfqpoint{4.369880in}{0.960720in}}%
\pgfpathcurveto{\pgfqpoint{4.369880in}{0.966245in}}{\pgfqpoint{4.367685in}{0.971544in}}{\pgfqpoint{4.363778in}{0.975451in}}%
\pgfpathcurveto{\pgfqpoint{4.359871in}{0.979358in}}{\pgfqpoint{4.354571in}{0.981553in}}{\pgfqpoint{4.349046in}{0.981553in}}%
\pgfpathcurveto{\pgfqpoint{4.343521in}{0.981553in}}{\pgfqpoint{4.338222in}{0.979358in}}{\pgfqpoint{4.334315in}{0.975451in}}%
\pgfpathcurveto{\pgfqpoint{4.330408in}{0.971544in}}{\pgfqpoint{4.328213in}{0.966245in}}{\pgfqpoint{4.328213in}{0.960720in}}%
\pgfpathcurveto{\pgfqpoint{4.328213in}{0.955195in}}{\pgfqpoint{4.330408in}{0.949895in}}{\pgfqpoint{4.334315in}{0.945988in}}%
\pgfpathcurveto{\pgfqpoint{4.338222in}{0.942081in}}{\pgfqpoint{4.343521in}{0.939886in}}{\pgfqpoint{4.349046in}{0.939886in}}%
\pgfpathclose%
\pgfusepath{fill}%
\end{pgfscope}%
\begin{pgfscope}%
\pgfpathrectangle{\pgfqpoint{3.214225in}{0.913832in}}{\pgfqpoint{1.162500in}{0.755000in}}%
\pgfusepath{clip}%
\pgfsetbuttcap%
\pgfsetroundjoin%
\definecolor{currentfill}{rgb}{0.000000,0.000000,0.000000}%
\pgfsetfillcolor{currentfill}%
\pgfsetfillopacity{0.500000}%
\pgfsetlinewidth{0.000000pt}%
\definecolor{currentstroke}{rgb}{0.000000,0.000000,0.000000}%
\pgfsetstrokecolor{currentstroke}%
\pgfsetdash{}{0pt}%
\pgfpathmoveto{\pgfqpoint{3.915823in}{1.105715in}}%
\pgfpathcurveto{\pgfqpoint{3.921348in}{1.105715in}}{\pgfqpoint{3.926648in}{1.107910in}}{\pgfqpoint{3.930554in}{1.111817in}}%
\pgfpathcurveto{\pgfqpoint{3.934461in}{1.115723in}}{\pgfqpoint{3.936656in}{1.121023in}}{\pgfqpoint{3.936656in}{1.126548in}}%
\pgfpathcurveto{\pgfqpoint{3.936656in}{1.132073in}}{\pgfqpoint{3.934461in}{1.137372in}}{\pgfqpoint{3.930554in}{1.141279in}}%
\pgfpathcurveto{\pgfqpoint{3.926648in}{1.145186in}}{\pgfqpoint{3.921348in}{1.147381in}}{\pgfqpoint{3.915823in}{1.147381in}}%
\pgfpathcurveto{\pgfqpoint{3.910298in}{1.147381in}}{\pgfqpoint{3.904998in}{1.145186in}}{\pgfqpoint{3.901092in}{1.141279in}}%
\pgfpathcurveto{\pgfqpoint{3.897185in}{1.137372in}}{\pgfqpoint{3.894990in}{1.132073in}}{\pgfqpoint{3.894990in}{1.126548in}}%
\pgfpathcurveto{\pgfqpoint{3.894990in}{1.121023in}}{\pgfqpoint{3.897185in}{1.115723in}}{\pgfqpoint{3.901092in}{1.111817in}}%
\pgfpathcurveto{\pgfqpoint{3.904998in}{1.107910in}}{\pgfqpoint{3.910298in}{1.105715in}}{\pgfqpoint{3.915823in}{1.105715in}}%
\pgfpathclose%
\pgfusepath{fill}%
\end{pgfscope}%
\begin{pgfscope}%
\pgfpathrectangle{\pgfqpoint{3.214225in}{0.913832in}}{\pgfqpoint{1.162500in}{0.755000in}}%
\pgfusepath{clip}%
\pgfsetbuttcap%
\pgfsetroundjoin%
\definecolor{currentfill}{rgb}{0.000000,0.000000,0.000000}%
\pgfsetfillcolor{currentfill}%
\pgfsetfillopacity{0.500000}%
\pgfsetlinewidth{0.000000pt}%
\definecolor{currentstroke}{rgb}{0.000000,0.000000,0.000000}%
\pgfsetstrokecolor{currentstroke}%
\pgfsetdash{}{0pt}%
\pgfpathmoveto{\pgfqpoint{3.986481in}{0.910975in}}%
\pgfpathcurveto{\pgfqpoint{3.992006in}{0.910975in}}{\pgfqpoint{3.997306in}{0.913170in}}{\pgfqpoint{4.001213in}{0.917077in}}%
\pgfpathcurveto{\pgfqpoint{4.005120in}{0.920984in}}{\pgfqpoint{4.007315in}{0.926283in}}{\pgfqpoint{4.007315in}{0.931809in}}%
\pgfpathcurveto{\pgfqpoint{4.007315in}{0.937334in}}{\pgfqpoint{4.005120in}{0.942633in}}{\pgfqpoint{4.001213in}{0.946540in}}%
\pgfpathcurveto{\pgfqpoint{3.997306in}{0.950447in}}{\pgfqpoint{3.992006in}{0.952642in}}{\pgfqpoint{3.986481in}{0.952642in}}%
\pgfpathcurveto{\pgfqpoint{3.980956in}{0.952642in}}{\pgfqpoint{3.975657in}{0.950447in}}{\pgfqpoint{3.971750in}{0.946540in}}%
\pgfpathcurveto{\pgfqpoint{3.967843in}{0.942633in}}{\pgfqpoint{3.965648in}{0.937334in}}{\pgfqpoint{3.965648in}{0.931809in}}%
\pgfpathcurveto{\pgfqpoint{3.965648in}{0.926283in}}{\pgfqpoint{3.967843in}{0.920984in}}{\pgfqpoint{3.971750in}{0.917077in}}%
\pgfpathcurveto{\pgfqpoint{3.975657in}{0.913170in}}{\pgfqpoint{3.980956in}{0.910975in}}{\pgfqpoint{3.986481in}{0.910975in}}%
\pgfpathclose%
\pgfusepath{fill}%
\end{pgfscope}%
\begin{pgfscope}%
\pgfpathrectangle{\pgfqpoint{3.214225in}{0.913832in}}{\pgfqpoint{1.162500in}{0.755000in}}%
\pgfusepath{clip}%
\pgfsetbuttcap%
\pgfsetroundjoin%
\definecolor{currentfill}{rgb}{0.000000,0.000000,0.000000}%
\pgfsetfillcolor{currentfill}%
\pgfsetfillopacity{0.500000}%
\pgfsetlinewidth{0.000000pt}%
\definecolor{currentstroke}{rgb}{0.000000,0.000000,0.000000}%
\pgfsetstrokecolor{currentstroke}%
\pgfsetdash{}{0pt}%
\pgfpathmoveto{\pgfqpoint{3.831974in}{1.042931in}}%
\pgfpathcurveto{\pgfqpoint{3.837499in}{1.042931in}}{\pgfqpoint{3.842798in}{1.045126in}}{\pgfqpoint{3.846705in}{1.049033in}}%
\pgfpathcurveto{\pgfqpoint{3.850612in}{1.052940in}}{\pgfqpoint{3.852807in}{1.058239in}}{\pgfqpoint{3.852807in}{1.063764in}}%
\pgfpathcurveto{\pgfqpoint{3.852807in}{1.069289in}}{\pgfqpoint{3.850612in}{1.074589in}}{\pgfqpoint{3.846705in}{1.078495in}}%
\pgfpathcurveto{\pgfqpoint{3.842798in}{1.082402in}}{\pgfqpoint{3.837499in}{1.084597in}}{\pgfqpoint{3.831974in}{1.084597in}}%
\pgfpathcurveto{\pgfqpoint{3.826449in}{1.084597in}}{\pgfqpoint{3.821149in}{1.082402in}}{\pgfqpoint{3.817242in}{1.078495in}}%
\pgfpathcurveto{\pgfqpoint{3.813336in}{1.074589in}}{\pgfqpoint{3.811140in}{1.069289in}}{\pgfqpoint{3.811140in}{1.063764in}}%
\pgfpathcurveto{\pgfqpoint{3.811140in}{1.058239in}}{\pgfqpoint{3.813336in}{1.052940in}}{\pgfqpoint{3.817242in}{1.049033in}}%
\pgfpathcurveto{\pgfqpoint{3.821149in}{1.045126in}}{\pgfqpoint{3.826449in}{1.042931in}}{\pgfqpoint{3.831974in}{1.042931in}}%
\pgfpathclose%
\pgfusepath{fill}%
\end{pgfscope}%
\begin{pgfscope}%
\pgfpathrectangle{\pgfqpoint{3.214225in}{0.913832in}}{\pgfqpoint{1.162500in}{0.755000in}}%
\pgfusepath{clip}%
\pgfsetbuttcap%
\pgfsetroundjoin%
\definecolor{currentfill}{rgb}{0.000000,0.000000,0.000000}%
\pgfsetfillcolor{currentfill}%
\pgfsetfillopacity{0.500000}%
\pgfsetlinewidth{0.000000pt}%
\definecolor{currentstroke}{rgb}{0.000000,0.000000,0.000000}%
\pgfsetstrokecolor{currentstroke}%
\pgfsetdash{}{0pt}%
\pgfpathmoveto{\pgfqpoint{3.241904in}{1.035607in}}%
\pgfpathcurveto{\pgfqpoint{3.247429in}{1.035607in}}{\pgfqpoint{3.252728in}{1.037803in}}{\pgfqpoint{3.256635in}{1.041709in}}%
\pgfpathcurveto{\pgfqpoint{3.260542in}{1.045616in}}{\pgfqpoint{3.262737in}{1.050916in}}{\pgfqpoint{3.262737in}{1.056441in}}%
\pgfpathcurveto{\pgfqpoint{3.262737in}{1.061966in}}{\pgfqpoint{3.260542in}{1.067265in}}{\pgfqpoint{3.256635in}{1.071172in}}%
\pgfpathcurveto{\pgfqpoint{3.252728in}{1.075079in}}{\pgfqpoint{3.247429in}{1.077274in}}{\pgfqpoint{3.241904in}{1.077274in}}%
\pgfpathcurveto{\pgfqpoint{3.236379in}{1.077274in}}{\pgfqpoint{3.231079in}{1.075079in}}{\pgfqpoint{3.227172in}{1.071172in}}%
\pgfpathcurveto{\pgfqpoint{3.223265in}{1.067265in}}{\pgfqpoint{3.221070in}{1.061966in}}{\pgfqpoint{3.221070in}{1.056441in}}%
\pgfpathcurveto{\pgfqpoint{3.221070in}{1.050916in}}{\pgfqpoint{3.223265in}{1.045616in}}{\pgfqpoint{3.227172in}{1.041709in}}%
\pgfpathcurveto{\pgfqpoint{3.231079in}{1.037803in}}{\pgfqpoint{3.236379in}{1.035607in}}{\pgfqpoint{3.241904in}{1.035607in}}%
\pgfpathclose%
\pgfusepath{fill}%
\end{pgfscope}%
\begin{pgfscope}%
\pgfpathrectangle{\pgfqpoint{3.214225in}{0.913832in}}{\pgfqpoint{1.162500in}{0.755000in}}%
\pgfusepath{clip}%
\pgfsetbuttcap%
\pgfsetroundjoin%
\definecolor{currentfill}{rgb}{0.000000,0.000000,0.000000}%
\pgfsetfillcolor{currentfill}%
\pgfsetfillopacity{0.500000}%
\pgfsetlinewidth{0.000000pt}%
\definecolor{currentstroke}{rgb}{0.000000,0.000000,0.000000}%
\pgfsetstrokecolor{currentstroke}%
\pgfsetdash{}{0pt}%
\pgfpathmoveto{\pgfqpoint{4.220724in}{1.572595in}}%
\pgfpathcurveto{\pgfqpoint{4.226249in}{1.572595in}}{\pgfqpoint{4.231548in}{1.574791in}}{\pgfqpoint{4.235455in}{1.578697in}}%
\pgfpathcurveto{\pgfqpoint{4.239362in}{1.582604in}}{\pgfqpoint{4.241557in}{1.587904in}}{\pgfqpoint{4.241557in}{1.593429in}}%
\pgfpathcurveto{\pgfqpoint{4.241557in}{1.598954in}}{\pgfqpoint{4.239362in}{1.604253in}}{\pgfqpoint{4.235455in}{1.608160in}}%
\pgfpathcurveto{\pgfqpoint{4.231548in}{1.612067in}}{\pgfqpoint{4.226249in}{1.614262in}}{\pgfqpoint{4.220724in}{1.614262in}}%
\pgfpathcurveto{\pgfqpoint{4.215199in}{1.614262in}}{\pgfqpoint{4.209899in}{1.612067in}}{\pgfqpoint{4.205992in}{1.608160in}}%
\pgfpathcurveto{\pgfqpoint{4.202086in}{1.604253in}}{\pgfqpoint{4.199890in}{1.598954in}}{\pgfqpoint{4.199890in}{1.593429in}}%
\pgfpathcurveto{\pgfqpoint{4.199890in}{1.587904in}}{\pgfqpoint{4.202086in}{1.582604in}}{\pgfqpoint{4.205992in}{1.578697in}}%
\pgfpathcurveto{\pgfqpoint{4.209899in}{1.574791in}}{\pgfqpoint{4.215199in}{1.572595in}}{\pgfqpoint{4.220724in}{1.572595in}}%
\pgfpathclose%
\pgfusepath{fill}%
\end{pgfscope}%
\begin{pgfscope}%
\pgfpathrectangle{\pgfqpoint{3.214225in}{0.913832in}}{\pgfqpoint{1.162500in}{0.755000in}}%
\pgfusepath{clip}%
\pgfsetbuttcap%
\pgfsetroundjoin%
\definecolor{currentfill}{rgb}{0.000000,0.000000,0.000000}%
\pgfsetfillcolor{currentfill}%
\pgfsetfillopacity{0.500000}%
\pgfsetlinewidth{0.000000pt}%
\definecolor{currentstroke}{rgb}{0.000000,0.000000,0.000000}%
\pgfsetstrokecolor{currentstroke}%
\pgfsetdash{}{0pt}%
\pgfpathmoveto{\pgfqpoint{3.980838in}{1.630023in}}%
\pgfpathcurveto{\pgfqpoint{3.986364in}{1.630023in}}{\pgfqpoint{3.991663in}{1.632218in}}{\pgfqpoint{3.995570in}{1.636125in}}%
\pgfpathcurveto{\pgfqpoint{3.999477in}{1.640032in}}{\pgfqpoint{4.001672in}{1.645331in}}{\pgfqpoint{4.001672in}{1.650856in}}%
\pgfpathcurveto{\pgfqpoint{4.001672in}{1.656381in}}{\pgfqpoint{3.999477in}{1.661681in}}{\pgfqpoint{3.995570in}{1.665588in}}%
\pgfpathcurveto{\pgfqpoint{3.991663in}{1.669494in}}{\pgfqpoint{3.986364in}{1.671690in}}{\pgfqpoint{3.980838in}{1.671690in}}%
\pgfpathcurveto{\pgfqpoint{3.975313in}{1.671690in}}{\pgfqpoint{3.970014in}{1.669494in}}{\pgfqpoint{3.966107in}{1.665588in}}%
\pgfpathcurveto{\pgfqpoint{3.962200in}{1.661681in}}{\pgfqpoint{3.960005in}{1.656381in}}{\pgfqpoint{3.960005in}{1.650856in}}%
\pgfpathcurveto{\pgfqpoint{3.960005in}{1.645331in}}{\pgfqpoint{3.962200in}{1.640032in}}{\pgfqpoint{3.966107in}{1.636125in}}%
\pgfpathcurveto{\pgfqpoint{3.970014in}{1.632218in}}{\pgfqpoint{3.975313in}{1.630023in}}{\pgfqpoint{3.980838in}{1.630023in}}%
\pgfpathclose%
\pgfusepath{fill}%
\end{pgfscope}%
\begin{pgfscope}%
\pgfpathrectangle{\pgfqpoint{3.214225in}{0.913832in}}{\pgfqpoint{1.162500in}{0.755000in}}%
\pgfusepath{clip}%
\pgfsetbuttcap%
\pgfsetroundjoin%
\definecolor{currentfill}{rgb}{0.000000,0.000000,0.000000}%
\pgfsetfillcolor{currentfill}%
\pgfsetfillopacity{0.500000}%
\pgfsetlinewidth{0.000000pt}%
\definecolor{currentstroke}{rgb}{0.000000,0.000000,0.000000}%
\pgfsetstrokecolor{currentstroke}%
\pgfsetdash{}{0pt}%
\pgfpathmoveto{\pgfqpoint{3.708643in}{1.484185in}}%
\pgfpathcurveto{\pgfqpoint{3.714168in}{1.484185in}}{\pgfqpoint{3.719468in}{1.486380in}}{\pgfqpoint{3.723375in}{1.490287in}}%
\pgfpathcurveto{\pgfqpoint{3.727282in}{1.494194in}}{\pgfqpoint{3.729477in}{1.499493in}}{\pgfqpoint{3.729477in}{1.505018in}}%
\pgfpathcurveto{\pgfqpoint{3.729477in}{1.510544in}}{\pgfqpoint{3.727282in}{1.515843in}}{\pgfqpoint{3.723375in}{1.519750in}}%
\pgfpathcurveto{\pgfqpoint{3.719468in}{1.523657in}}{\pgfqpoint{3.714168in}{1.525852in}}{\pgfqpoint{3.708643in}{1.525852in}}%
\pgfpathcurveto{\pgfqpoint{3.703118in}{1.525852in}}{\pgfqpoint{3.697819in}{1.523657in}}{\pgfqpoint{3.693912in}{1.519750in}}%
\pgfpathcurveto{\pgfqpoint{3.690005in}{1.515843in}}{\pgfqpoint{3.687810in}{1.510544in}}{\pgfqpoint{3.687810in}{1.505018in}}%
\pgfpathcurveto{\pgfqpoint{3.687810in}{1.499493in}}{\pgfqpoint{3.690005in}{1.494194in}}{\pgfqpoint{3.693912in}{1.490287in}}%
\pgfpathcurveto{\pgfqpoint{3.697819in}{1.486380in}}{\pgfqpoint{3.703118in}{1.484185in}}{\pgfqpoint{3.708643in}{1.484185in}}%
\pgfpathclose%
\pgfusepath{fill}%
\end{pgfscope}%
\begin{pgfscope}%
\pgfpathrectangle{\pgfqpoint{3.214225in}{0.913832in}}{\pgfqpoint{1.162500in}{0.755000in}}%
\pgfusepath{clip}%
\pgfsetbuttcap%
\pgfsetroundjoin%
\definecolor{currentfill}{rgb}{0.000000,0.000000,0.000000}%
\pgfsetfillcolor{currentfill}%
\pgfsetfillopacity{0.500000}%
\pgfsetlinewidth{0.000000pt}%
\definecolor{currentstroke}{rgb}{0.000000,0.000000,0.000000}%
\pgfsetstrokecolor{currentstroke}%
\pgfsetdash{}{0pt}%
\pgfpathmoveto{\pgfqpoint{4.124371in}{1.067768in}}%
\pgfpathcurveto{\pgfqpoint{4.129896in}{1.067768in}}{\pgfqpoint{4.135195in}{1.069963in}}{\pgfqpoint{4.139102in}{1.073870in}}%
\pgfpathcurveto{\pgfqpoint{4.143009in}{1.077776in}}{\pgfqpoint{4.145204in}{1.083076in}}{\pgfqpoint{4.145204in}{1.088601in}}%
\pgfpathcurveto{\pgfqpoint{4.145204in}{1.094126in}}{\pgfqpoint{4.143009in}{1.099426in}}{\pgfqpoint{4.139102in}{1.103332in}}%
\pgfpathcurveto{\pgfqpoint{4.135195in}{1.107239in}}{\pgfqpoint{4.129896in}{1.109434in}}{\pgfqpoint{4.124371in}{1.109434in}}%
\pgfpathcurveto{\pgfqpoint{4.118846in}{1.109434in}}{\pgfqpoint{4.113546in}{1.107239in}}{\pgfqpoint{4.109639in}{1.103332in}}%
\pgfpathcurveto{\pgfqpoint{4.105732in}{1.099426in}}{\pgfqpoint{4.103537in}{1.094126in}}{\pgfqpoint{4.103537in}{1.088601in}}%
\pgfpathcurveto{\pgfqpoint{4.103537in}{1.083076in}}{\pgfqpoint{4.105732in}{1.077776in}}{\pgfqpoint{4.109639in}{1.073870in}}%
\pgfpathcurveto{\pgfqpoint{4.113546in}{1.069963in}}{\pgfqpoint{4.118846in}{1.067768in}}{\pgfqpoint{4.124371in}{1.067768in}}%
\pgfpathclose%
\pgfusepath{fill}%
\end{pgfscope}%
\begin{pgfscope}%
\pgfpathrectangle{\pgfqpoint{3.214225in}{0.913832in}}{\pgfqpoint{1.162500in}{0.755000in}}%
\pgfusepath{clip}%
\pgfsetbuttcap%
\pgfsetroundjoin%
\definecolor{currentfill}{rgb}{0.000000,0.000000,0.000000}%
\pgfsetfillcolor{currentfill}%
\pgfsetfillopacity{0.500000}%
\pgfsetlinewidth{0.000000pt}%
\definecolor{currentstroke}{rgb}{0.000000,0.000000,0.000000}%
\pgfsetstrokecolor{currentstroke}%
\pgfsetdash{}{0pt}%
\pgfpathmoveto{\pgfqpoint{3.695649in}{1.343150in}}%
\pgfpathcurveto{\pgfqpoint{3.701174in}{1.343150in}}{\pgfqpoint{3.706474in}{1.345345in}}{\pgfqpoint{3.710381in}{1.349252in}}%
\pgfpathcurveto{\pgfqpoint{3.714288in}{1.353159in}}{\pgfqpoint{3.716483in}{1.358458in}}{\pgfqpoint{3.716483in}{1.363984in}}%
\pgfpathcurveto{\pgfqpoint{3.716483in}{1.369509in}}{\pgfqpoint{3.714288in}{1.374808in}}{\pgfqpoint{3.710381in}{1.378715in}}%
\pgfpathcurveto{\pgfqpoint{3.706474in}{1.382622in}}{\pgfqpoint{3.701174in}{1.384817in}}{\pgfqpoint{3.695649in}{1.384817in}}%
\pgfpathcurveto{\pgfqpoint{3.690124in}{1.384817in}}{\pgfqpoint{3.684825in}{1.382622in}}{\pgfqpoint{3.680918in}{1.378715in}}%
\pgfpathcurveto{\pgfqpoint{3.677011in}{1.374808in}}{\pgfqpoint{3.674816in}{1.369509in}}{\pgfqpoint{3.674816in}{1.363984in}}%
\pgfpathcurveto{\pgfqpoint{3.674816in}{1.358458in}}{\pgfqpoint{3.677011in}{1.353159in}}{\pgfqpoint{3.680918in}{1.349252in}}%
\pgfpathcurveto{\pgfqpoint{3.684825in}{1.345345in}}{\pgfqpoint{3.690124in}{1.343150in}}{\pgfqpoint{3.695649in}{1.343150in}}%
\pgfpathclose%
\pgfusepath{fill}%
\end{pgfscope}%
\begin{pgfscope}%
\pgfpathrectangle{\pgfqpoint{3.214225in}{0.913832in}}{\pgfqpoint{1.162500in}{0.755000in}}%
\pgfusepath{clip}%
\pgfsetbuttcap%
\pgfsetroundjoin%
\definecolor{currentfill}{rgb}{0.000000,0.000000,0.000000}%
\pgfsetfillcolor{currentfill}%
\pgfsetfillopacity{0.500000}%
\pgfsetlinewidth{0.000000pt}%
\definecolor{currentstroke}{rgb}{0.000000,0.000000,0.000000}%
\pgfsetstrokecolor{currentstroke}%
\pgfsetdash{}{0pt}%
\pgfpathmoveto{\pgfqpoint{3.329643in}{1.259718in}}%
\pgfpathcurveto{\pgfqpoint{3.335168in}{1.259718in}}{\pgfqpoint{3.340468in}{1.261913in}}{\pgfqpoint{3.344375in}{1.265820in}}%
\pgfpathcurveto{\pgfqpoint{3.348281in}{1.269726in}}{\pgfqpoint{3.350476in}{1.275026in}}{\pgfqpoint{3.350476in}{1.280551in}}%
\pgfpathcurveto{\pgfqpoint{3.350476in}{1.286076in}}{\pgfqpoint{3.348281in}{1.291376in}}{\pgfqpoint{3.344375in}{1.295282in}}%
\pgfpathcurveto{\pgfqpoint{3.340468in}{1.299189in}}{\pgfqpoint{3.335168in}{1.301384in}}{\pgfqpoint{3.329643in}{1.301384in}}%
\pgfpathcurveto{\pgfqpoint{3.324118in}{1.301384in}}{\pgfqpoint{3.318819in}{1.299189in}}{\pgfqpoint{3.314912in}{1.295282in}}%
\pgfpathcurveto{\pgfqpoint{3.311005in}{1.291376in}}{\pgfqpoint{3.308810in}{1.286076in}}{\pgfqpoint{3.308810in}{1.280551in}}%
\pgfpathcurveto{\pgfqpoint{3.308810in}{1.275026in}}{\pgfqpoint{3.311005in}{1.269726in}}{\pgfqpoint{3.314912in}{1.265820in}}%
\pgfpathcurveto{\pgfqpoint{3.318819in}{1.261913in}}{\pgfqpoint{3.324118in}{1.259718in}}{\pgfqpoint{3.329643in}{1.259718in}}%
\pgfpathclose%
\pgfusepath{fill}%
\end{pgfscope}%
\begin{pgfscope}%
\pgfpathrectangle{\pgfqpoint{3.214225in}{0.913832in}}{\pgfqpoint{1.162500in}{0.755000in}}%
\pgfusepath{clip}%
\pgfsetbuttcap%
\pgfsetroundjoin%
\definecolor{currentfill}{rgb}{0.000000,0.000000,0.000000}%
\pgfsetfillcolor{currentfill}%
\pgfsetfillopacity{0.500000}%
\pgfsetlinewidth{0.000000pt}%
\definecolor{currentstroke}{rgb}{0.000000,0.000000,0.000000}%
\pgfsetstrokecolor{currentstroke}%
\pgfsetdash{}{0pt}%
\pgfpathmoveto{\pgfqpoint{4.167862in}{1.167350in}}%
\pgfpathcurveto{\pgfqpoint{4.173387in}{1.167350in}}{\pgfqpoint{4.178687in}{1.169545in}}{\pgfqpoint{4.182593in}{1.173452in}}%
\pgfpathcurveto{\pgfqpoint{4.186500in}{1.177359in}}{\pgfqpoint{4.188695in}{1.182658in}}{\pgfqpoint{4.188695in}{1.188183in}}%
\pgfpathcurveto{\pgfqpoint{4.188695in}{1.193708in}}{\pgfqpoint{4.186500in}{1.199008in}}{\pgfqpoint{4.182593in}{1.202915in}}%
\pgfpathcurveto{\pgfqpoint{4.178687in}{1.206822in}}{\pgfqpoint{4.173387in}{1.209017in}}{\pgfqpoint{4.167862in}{1.209017in}}%
\pgfpathcurveto{\pgfqpoint{4.162337in}{1.209017in}}{\pgfqpoint{4.157037in}{1.206822in}}{\pgfqpoint{4.153131in}{1.202915in}}%
\pgfpathcurveto{\pgfqpoint{4.149224in}{1.199008in}}{\pgfqpoint{4.147029in}{1.193708in}}{\pgfqpoint{4.147029in}{1.188183in}}%
\pgfpathcurveto{\pgfqpoint{4.147029in}{1.182658in}}{\pgfqpoint{4.149224in}{1.177359in}}{\pgfqpoint{4.153131in}{1.173452in}}%
\pgfpathcurveto{\pgfqpoint{4.157037in}{1.169545in}}{\pgfqpoint{4.162337in}{1.167350in}}{\pgfqpoint{4.167862in}{1.167350in}}%
\pgfpathclose%
\pgfusepath{fill}%
\end{pgfscope}%
\begin{pgfscope}%
\pgfpathrectangle{\pgfqpoint{3.214225in}{0.913832in}}{\pgfqpoint{1.162500in}{0.755000in}}%
\pgfusepath{clip}%
\pgfsetbuttcap%
\pgfsetroundjoin%
\definecolor{currentfill}{rgb}{0.000000,0.000000,0.000000}%
\pgfsetfillcolor{currentfill}%
\pgfsetfillopacity{0.500000}%
\pgfsetlinewidth{0.000000pt}%
\definecolor{currentstroke}{rgb}{0.000000,0.000000,0.000000}%
\pgfsetstrokecolor{currentstroke}%
\pgfsetdash{}{0pt}%
\pgfpathmoveto{\pgfqpoint{4.120816in}{1.034902in}}%
\pgfpathcurveto{\pgfqpoint{4.126341in}{1.034902in}}{\pgfqpoint{4.131641in}{1.037097in}}{\pgfqpoint{4.135548in}{1.041004in}}%
\pgfpathcurveto{\pgfqpoint{4.139455in}{1.044911in}}{\pgfqpoint{4.141650in}{1.050211in}}{\pgfqpoint{4.141650in}{1.055736in}}%
\pgfpathcurveto{\pgfqpoint{4.141650in}{1.061261in}}{\pgfqpoint{4.139455in}{1.066560in}}{\pgfqpoint{4.135548in}{1.070467in}}%
\pgfpathcurveto{\pgfqpoint{4.131641in}{1.074374in}}{\pgfqpoint{4.126341in}{1.076569in}}{\pgfqpoint{4.120816in}{1.076569in}}%
\pgfpathcurveto{\pgfqpoint{4.115291in}{1.076569in}}{\pgfqpoint{4.109992in}{1.074374in}}{\pgfqpoint{4.106085in}{1.070467in}}%
\pgfpathcurveto{\pgfqpoint{4.102178in}{1.066560in}}{\pgfqpoint{4.099983in}{1.061261in}}{\pgfqpoint{4.099983in}{1.055736in}}%
\pgfpathcurveto{\pgfqpoint{4.099983in}{1.050211in}}{\pgfqpoint{4.102178in}{1.044911in}}{\pgfqpoint{4.106085in}{1.041004in}}%
\pgfpathcurveto{\pgfqpoint{4.109992in}{1.037097in}}{\pgfqpoint{4.115291in}{1.034902in}}{\pgfqpoint{4.120816in}{1.034902in}}%
\pgfpathclose%
\pgfusepath{fill}%
\end{pgfscope}%
\begin{pgfscope}%
\pgfpathrectangle{\pgfqpoint{3.214225in}{0.913832in}}{\pgfqpoint{1.162500in}{0.755000in}}%
\pgfusepath{clip}%
\pgfsetbuttcap%
\pgfsetroundjoin%
\definecolor{currentfill}{rgb}{0.000000,0.000000,0.000000}%
\pgfsetfillcolor{currentfill}%
\pgfsetfillopacity{0.500000}%
\pgfsetlinewidth{0.000000pt}%
\definecolor{currentstroke}{rgb}{0.000000,0.000000,0.000000}%
\pgfsetstrokecolor{currentstroke}%
\pgfsetdash{}{0pt}%
\pgfpathmoveto{\pgfqpoint{3.792736in}{1.190157in}}%
\pgfpathcurveto{\pgfqpoint{3.798261in}{1.190157in}}{\pgfqpoint{3.803560in}{1.192352in}}{\pgfqpoint{3.807467in}{1.196259in}}%
\pgfpathcurveto{\pgfqpoint{3.811374in}{1.200166in}}{\pgfqpoint{3.813569in}{1.205465in}}{\pgfqpoint{3.813569in}{1.210991in}}%
\pgfpathcurveto{\pgfqpoint{3.813569in}{1.216516in}}{\pgfqpoint{3.811374in}{1.221815in}}{\pgfqpoint{3.807467in}{1.225722in}}%
\pgfpathcurveto{\pgfqpoint{3.803560in}{1.229629in}}{\pgfqpoint{3.798261in}{1.231824in}}{\pgfqpoint{3.792736in}{1.231824in}}%
\pgfpathcurveto{\pgfqpoint{3.787211in}{1.231824in}}{\pgfqpoint{3.781911in}{1.229629in}}{\pgfqpoint{3.778004in}{1.225722in}}%
\pgfpathcurveto{\pgfqpoint{3.774097in}{1.221815in}}{\pgfqpoint{3.771902in}{1.216516in}}{\pgfqpoint{3.771902in}{1.210991in}}%
\pgfpathcurveto{\pgfqpoint{3.771902in}{1.205465in}}{\pgfqpoint{3.774097in}{1.200166in}}{\pgfqpoint{3.778004in}{1.196259in}}%
\pgfpathcurveto{\pgfqpoint{3.781911in}{1.192352in}}{\pgfqpoint{3.787211in}{1.190157in}}{\pgfqpoint{3.792736in}{1.190157in}}%
\pgfpathclose%
\pgfusepath{fill}%
\end{pgfscope}%
\begin{pgfscope}%
\pgfsetbuttcap%
\pgfsetroundjoin%
\definecolor{currentfill}{rgb}{0.000000,0.000000,0.000000}%
\pgfsetfillcolor{currentfill}%
\pgfsetlinewidth{0.803000pt}%
\definecolor{currentstroke}{rgb}{0.000000,0.000000,0.000000}%
\pgfsetstrokecolor{currentstroke}%
\pgfsetdash{}{0pt}%
\pgfsys@defobject{currentmarker}{\pgfqpoint{0.000000in}{-0.048611in}}{\pgfqpoint{0.000000in}{0.000000in}}{%
\pgfpathmoveto{\pgfqpoint{0.000000in}{0.000000in}}%
\pgfpathlineto{\pgfqpoint{0.000000in}{-0.048611in}}%
\pgfusepath{stroke,fill}%
}%
\begin{pgfscope}%
\pgfsys@transformshift{3.342043in}{0.913832in}%
\pgfsys@useobject{currentmarker}{}%
\end{pgfscope}%
\end{pgfscope}%
\begin{pgfscope}%
\pgftext[x=3.372697in,y=0.547702in,left,base,rotate=90.000000]{\rmfamily\fontsize{8.000000}{9.600000}\selectfont \(\displaystyle 0.025\)}%
\end{pgfscope}%
\begin{pgfscope}%
\pgfsetbuttcap%
\pgfsetroundjoin%
\definecolor{currentfill}{rgb}{0.000000,0.000000,0.000000}%
\pgfsetfillcolor{currentfill}%
\pgfsetlinewidth{0.803000pt}%
\definecolor{currentstroke}{rgb}{0.000000,0.000000,0.000000}%
\pgfsetstrokecolor{currentstroke}%
\pgfsetdash{}{0pt}%
\pgfsys@defobject{currentmarker}{\pgfqpoint{0.000000in}{-0.048611in}}{\pgfqpoint{0.000000in}{0.000000in}}{%
\pgfpathmoveto{\pgfqpoint{0.000000in}{0.000000in}}%
\pgfpathlineto{\pgfqpoint{0.000000in}{-0.048611in}}%
\pgfusepath{stroke,fill}%
}%
\begin{pgfscope}%
\pgfsys@transformshift{3.752399in}{0.913832in}%
\pgfsys@useobject{currentmarker}{}%
\end{pgfscope}%
\end{pgfscope}%
\begin{pgfscope}%
\pgftext[x=3.783053in,y=0.547702in,left,base,rotate=90.000000]{\rmfamily\fontsize{8.000000}{9.600000}\selectfont \(\displaystyle 0.050\)}%
\end{pgfscope}%
\begin{pgfscope}%
\pgfsetbuttcap%
\pgfsetroundjoin%
\definecolor{currentfill}{rgb}{0.000000,0.000000,0.000000}%
\pgfsetfillcolor{currentfill}%
\pgfsetlinewidth{0.803000pt}%
\definecolor{currentstroke}{rgb}{0.000000,0.000000,0.000000}%
\pgfsetstrokecolor{currentstroke}%
\pgfsetdash{}{0pt}%
\pgfsys@defobject{currentmarker}{\pgfqpoint{0.000000in}{-0.048611in}}{\pgfqpoint{0.000000in}{0.000000in}}{%
\pgfpathmoveto{\pgfqpoint{0.000000in}{0.000000in}}%
\pgfpathlineto{\pgfqpoint{0.000000in}{-0.048611in}}%
\pgfusepath{stroke,fill}%
}%
\begin{pgfscope}%
\pgfsys@transformshift{4.162756in}{0.913832in}%
\pgfsys@useobject{currentmarker}{}%
\end{pgfscope}%
\end{pgfscope}%
\begin{pgfscope}%
\pgftext[x=4.193409in,y=0.547702in,left,base,rotate=90.000000]{\rmfamily\fontsize{8.000000}{9.600000}\selectfont \(\displaystyle 0.075\)}%
\end{pgfscope}%
\begin{pgfscope}%
\pgftext[x=3.795475in,y=0.492146in,,top]{\rmfamily\fontsize{16.000000}{19.200000}\selectfont u0}%
\end{pgfscope}%
\begin{pgfscope}%
\pgfsetrectcap%
\pgfsetmiterjoin%
\pgfsetlinewidth{0.803000pt}%
\definecolor{currentstroke}{rgb}{0.501961,0.501961,0.501961}%
\pgfsetstrokecolor{currentstroke}%
\pgfsetdash{}{0pt}%
\pgfpathmoveto{\pgfqpoint{3.214225in}{0.913832in}}%
\pgfpathlineto{\pgfqpoint{3.214225in}{1.668832in}}%
\pgfusepath{stroke}%
\end{pgfscope}%
\begin{pgfscope}%
\pgfsetrectcap%
\pgfsetmiterjoin%
\pgfsetlinewidth{0.803000pt}%
\definecolor{currentstroke}{rgb}{0.501961,0.501961,0.501961}%
\pgfsetstrokecolor{currentstroke}%
\pgfsetdash{}{0pt}%
\pgfpathmoveto{\pgfqpoint{4.376725in}{0.913832in}}%
\pgfpathlineto{\pgfqpoint{4.376725in}{1.668832in}}%
\pgfusepath{stroke}%
\end{pgfscope}%
\begin{pgfscope}%
\pgfsetrectcap%
\pgfsetmiterjoin%
\pgfsetlinewidth{0.803000pt}%
\definecolor{currentstroke}{rgb}{0.501961,0.501961,0.501961}%
\pgfsetstrokecolor{currentstroke}%
\pgfsetdash{}{0pt}%
\pgfpathmoveto{\pgfqpoint{3.214225in}{0.913832in}}%
\pgfpathlineto{\pgfqpoint{4.376725in}{0.913832in}}%
\pgfusepath{stroke}%
\end{pgfscope}%
\begin{pgfscope}%
\pgfsetrectcap%
\pgfsetmiterjoin%
\pgfsetlinewidth{0.803000pt}%
\definecolor{currentstroke}{rgb}{0.501961,0.501961,0.501961}%
\pgfsetstrokecolor{currentstroke}%
\pgfsetdash{}{0pt}%
\pgfpathmoveto{\pgfqpoint{3.214225in}{1.668832in}}%
\pgfpathlineto{\pgfqpoint{4.376725in}{1.668832in}}%
\pgfusepath{stroke}%
\end{pgfscope}%
\begin{pgfscope}%
\pgfsetbuttcap%
\pgfsetmiterjoin%
\definecolor{currentfill}{rgb}{1.000000,1.000000,1.000000}%
\pgfsetfillcolor{currentfill}%
\pgfsetlinewidth{0.000000pt}%
\definecolor{currentstroke}{rgb}{0.000000,0.000000,0.000000}%
\pgfsetstrokecolor{currentstroke}%
\pgfsetstrokeopacity{0.000000}%
\pgfsetdash{}{0pt}%
\pgfpathmoveto{\pgfqpoint{4.376725in}{0.913832in}}%
\pgfpathlineto{\pgfqpoint{5.539225in}{0.913832in}}%
\pgfpathlineto{\pgfqpoint{5.539225in}{1.668832in}}%
\pgfpathlineto{\pgfqpoint{4.376725in}{1.668832in}}%
\pgfpathclose%
\pgfusepath{fill}%
\end{pgfscope}%
\begin{pgfscope}%
\pgfsetbuttcap%
\pgfsetroundjoin%
\definecolor{currentfill}{rgb}{0.000000,0.000000,0.000000}%
\pgfsetfillcolor{currentfill}%
\pgfsetlinewidth{0.803000pt}%
\definecolor{currentstroke}{rgb}{0.000000,0.000000,0.000000}%
\pgfsetstrokecolor{currentstroke}%
\pgfsetdash{}{0pt}%
\pgfsys@defobject{currentmarker}{\pgfqpoint{0.000000in}{-0.048611in}}{\pgfqpoint{0.000000in}{0.000000in}}{%
\pgfpathmoveto{\pgfqpoint{0.000000in}{0.000000in}}%
\pgfpathlineto{\pgfqpoint{0.000000in}{-0.048611in}}%
\pgfusepath{stroke,fill}%
}%
\begin{pgfscope}%
\pgfsys@transformshift{4.455518in}{0.913832in}%
\pgfsys@useobject{currentmarker}{}%
\end{pgfscope}%
\end{pgfscope}%
\begin{pgfscope}%
\pgftext[x=4.486171in,y=0.665759in,left,base,rotate=90.000000]{\rmfamily\fontsize{8.000000}{9.600000}\selectfont \(\displaystyle 0.5\)}%
\end{pgfscope}%
\begin{pgfscope}%
\pgfsetbuttcap%
\pgfsetroundjoin%
\definecolor{currentfill}{rgb}{0.000000,0.000000,0.000000}%
\pgfsetfillcolor{currentfill}%
\pgfsetlinewidth{0.803000pt}%
\definecolor{currentstroke}{rgb}{0.000000,0.000000,0.000000}%
\pgfsetstrokecolor{currentstroke}%
\pgfsetdash{}{0pt}%
\pgfsys@defobject{currentmarker}{\pgfqpoint{0.000000in}{-0.048611in}}{\pgfqpoint{0.000000in}{0.000000in}}{%
\pgfpathmoveto{\pgfqpoint{0.000000in}{0.000000in}}%
\pgfpathlineto{\pgfqpoint{0.000000in}{-0.048611in}}%
\pgfusepath{stroke,fill}%
}%
\begin{pgfscope}%
\pgfsys@transformshift{4.909793in}{0.913832in}%
\pgfsys@useobject{currentmarker}{}%
\end{pgfscope}%
\end{pgfscope}%
\begin{pgfscope}%
\pgftext[x=4.940447in,y=0.665759in,left,base,rotate=90.000000]{\rmfamily\fontsize{8.000000}{9.600000}\selectfont \(\displaystyle 1.0\)}%
\end{pgfscope}%
\begin{pgfscope}%
\pgfsetbuttcap%
\pgfsetroundjoin%
\definecolor{currentfill}{rgb}{0.000000,0.000000,0.000000}%
\pgfsetfillcolor{currentfill}%
\pgfsetlinewidth{0.803000pt}%
\definecolor{currentstroke}{rgb}{0.000000,0.000000,0.000000}%
\pgfsetstrokecolor{currentstroke}%
\pgfsetdash{}{0pt}%
\pgfsys@defobject{currentmarker}{\pgfqpoint{0.000000in}{-0.048611in}}{\pgfqpoint{0.000000in}{0.000000in}}{%
\pgfpathmoveto{\pgfqpoint{0.000000in}{0.000000in}}%
\pgfpathlineto{\pgfqpoint{0.000000in}{-0.048611in}}%
\pgfusepath{stroke,fill}%
}%
\begin{pgfscope}%
\pgfsys@transformshift{5.364069in}{0.913832in}%
\pgfsys@useobject{currentmarker}{}%
\end{pgfscope}%
\end{pgfscope}%
\begin{pgfscope}%
\pgftext[x=5.394722in,y=0.665759in,left,base,rotate=90.000000]{\rmfamily\fontsize{8.000000}{9.600000}\selectfont \(\displaystyle 1.5\)}%
\end{pgfscope}%
\begin{pgfscope}%
\pgftext[x=4.957975in,y=0.610204in,,top]{\rmfamily\fontsize{16.000000}{19.200000}\selectfont Ef0}%
\end{pgfscope}%
\begin{pgfscope}%
\pgfpathrectangle{\pgfqpoint{4.376725in}{0.913832in}}{\pgfqpoint{1.162500in}{0.755000in}}%
\pgfusepath{clip}%
\pgfsetrectcap%
\pgfsetroundjoin%
\pgfsetlinewidth{1.505625pt}%
\definecolor{currentstroke}{rgb}{0.121569,0.466667,0.705882}%
\pgfsetstrokecolor{currentstroke}%
\pgfsetdash{}{0pt}%
\pgfpathmoveto{\pgfqpoint{4.404404in}{1.292854in}}%
\pgfpathlineto{\pgfqpoint{4.443192in}{1.385258in}}%
\pgfpathlineto{\pgfqpoint{4.470899in}{1.445863in}}%
\pgfpathlineto{\pgfqpoint{4.495280in}{1.493940in}}%
\pgfpathlineto{\pgfqpoint{4.516337in}{1.530674in}}%
\pgfpathlineto{\pgfqpoint{4.535177in}{1.559293in}}%
\pgfpathlineto{\pgfqpoint{4.552909in}{1.582261in}}%
\pgfpathlineto{\pgfqpoint{4.569533in}{1.600110in}}%
\pgfpathlineto{\pgfqpoint{4.585049in}{1.613447in}}%
\pgfpathlineto{\pgfqpoint{4.599456in}{1.622913in}}%
\pgfpathlineto{\pgfqpoint{4.612755in}{1.629148in}}%
\pgfpathlineto{\pgfqpoint{4.626054in}{1.632994in}}%
\pgfpathlineto{\pgfqpoint{4.639353in}{1.634490in}}%
\pgfpathlineto{\pgfqpoint{4.652652in}{1.633693in}}%
\pgfpathlineto{\pgfqpoint{4.665951in}{1.630682in}}%
\pgfpathlineto{\pgfqpoint{4.680358in}{1.625032in}}%
\pgfpathlineto{\pgfqpoint{4.695874in}{1.616332in}}%
\pgfpathlineto{\pgfqpoint{4.712497in}{1.604229in}}%
\pgfpathlineto{\pgfqpoint{4.730229in}{1.588458in}}%
\pgfpathlineto{\pgfqpoint{4.750178in}{1.567637in}}%
\pgfpathlineto{\pgfqpoint{4.773451in}{1.539943in}}%
\pgfpathlineto{\pgfqpoint{4.802266in}{1.501880in}}%
\pgfpathlineto{\pgfqpoint{4.844379in}{1.442067in}}%
\pgfpathlineto{\pgfqpoint{4.908658in}{1.351028in}}%
\pgfpathlineto{\pgfqpoint{4.941905in}{1.307935in}}%
\pgfpathlineto{\pgfqpoint{4.970720in}{1.274145in}}%
\pgfpathlineto{\pgfqpoint{4.997318in}{1.246340in}}%
\pgfpathlineto{\pgfqpoint{5.022808in}{1.222922in}}%
\pgfpathlineto{\pgfqpoint{5.047189in}{1.203484in}}%
\pgfpathlineto{\pgfqpoint{5.071571in}{1.186825in}}%
\pgfpathlineto{\pgfqpoint{5.097061in}{1.172140in}}%
\pgfpathlineto{\pgfqpoint{5.123659in}{1.159413in}}%
\pgfpathlineto{\pgfqpoint{5.153581in}{1.147648in}}%
\pgfpathlineto{\pgfqpoint{5.190154in}{1.135777in}}%
\pgfpathlineto{\pgfqpoint{5.293221in}{1.103752in}}%
\pgfpathlineto{\pgfqpoint{5.324252in}{1.090901in}}%
\pgfpathlineto{\pgfqpoint{5.353067in}{1.076511in}}%
\pgfpathlineto{\pgfqpoint{5.380773in}{1.060111in}}%
\pgfpathlineto{\pgfqpoint{5.408479in}{1.041034in}}%
\pgfpathlineto{\pgfqpoint{5.437294in}{1.018351in}}%
\pgfpathlineto{\pgfqpoint{5.467216in}{0.991904in}}%
\pgfpathlineto{\pgfqpoint{5.500464in}{0.959524in}}%
\pgfpathlineto{\pgfqpoint{5.511546in}{0.948151in}}%
\pgfpathlineto{\pgfqpoint{5.511546in}{0.948151in}}%
\pgfusepath{stroke}%
\end{pgfscope}%
\begin{pgfscope}%
\pgfsetrectcap%
\pgfsetmiterjoin%
\pgfsetlinewidth{0.803000pt}%
\definecolor{currentstroke}{rgb}{0.501961,0.501961,0.501961}%
\pgfsetstrokecolor{currentstroke}%
\pgfsetdash{}{0pt}%
\pgfpathmoveto{\pgfqpoint{4.376725in}{0.913832in}}%
\pgfpathlineto{\pgfqpoint{4.376725in}{1.668832in}}%
\pgfusepath{stroke}%
\end{pgfscope}%
\begin{pgfscope}%
\pgfsetrectcap%
\pgfsetmiterjoin%
\pgfsetlinewidth{0.803000pt}%
\definecolor{currentstroke}{rgb}{0.501961,0.501961,0.501961}%
\pgfsetstrokecolor{currentstroke}%
\pgfsetdash{}{0pt}%
\pgfpathmoveto{\pgfqpoint{5.539225in}{0.913832in}}%
\pgfpathlineto{\pgfqpoint{5.539225in}{1.668832in}}%
\pgfusepath{stroke}%
\end{pgfscope}%
\begin{pgfscope}%
\pgfsetrectcap%
\pgfsetmiterjoin%
\pgfsetlinewidth{0.803000pt}%
\definecolor{currentstroke}{rgb}{0.501961,0.501961,0.501961}%
\pgfsetstrokecolor{currentstroke}%
\pgfsetdash{}{0pt}%
\pgfpathmoveto{\pgfqpoint{4.376725in}{0.913832in}}%
\pgfpathlineto{\pgfqpoint{5.539225in}{0.913832in}}%
\pgfusepath{stroke}%
\end{pgfscope}%
\begin{pgfscope}%
\pgfsetrectcap%
\pgfsetmiterjoin%
\pgfsetlinewidth{0.803000pt}%
\definecolor{currentstroke}{rgb}{0.501961,0.501961,0.501961}%
\pgfsetstrokecolor{currentstroke}%
\pgfsetdash{}{0pt}%
\pgfpathmoveto{\pgfqpoint{4.376725in}{1.668832in}}%
\pgfpathlineto{\pgfqpoint{5.539225in}{1.668832in}}%
\pgfusepath{stroke}%
\end{pgfscope}%
\end{pgfpicture}%
\makeatother%
\endgroup%
}
    \caption{A simple EMA plot.\label{fig:scatter}}
\end{figure}
\begin{figure}[htb]
    \centering
    %% Creator: Matplotlib, PGF backend
%%
%% To include the figure in your LaTeX document, write
%%   \input{<filename>.pgf}
%%
%% Make sure the required packages are loaded in your preamble
%%   \usepackage{pgf}
%%
%% Figures using additional raster images can only be included by \input if
%% they are in the same directory as the main LaTeX file. For loading figures
%% from other directories you can use the `import` package
%%   \usepackage{import}
%% and then include the figures with
%%   \import{<path to file>}{<filename>.pgf}
%%
%% Matplotlib used the following preamble
%%   \usepackage{fontspec}
%%   \setmainfont{DejaVuSerif.ttf}[Path=/home/erin/anaconda3/lib/python3.6/site-packages/matplotlib/mpl-data/fonts/ttf/]
%%   \setsansfont{DejaVuSans.ttf}[Path=/home/erin/anaconda3/lib/python3.6/site-packages/matplotlib/mpl-data/fonts/ttf/]
%%   \setmonofont{DejaVuSansMono.ttf}[Path=/home/erin/anaconda3/lib/python3.6/site-packages/matplotlib/mpl-data/fonts/ttf/]
%%
\begingroup%
\makeatletter%
\begin{pgfpicture}%
\pgfpathrectangle{\pgfpointorigin}{\pgfqpoint{5.232821in}{3.788793in}}%
\pgfusepath{use as bounding box, clip}%
\begin{pgfscope}%
\pgfsetbuttcap%
\pgfsetmiterjoin%
\definecolor{currentfill}{rgb}{1.000000,1.000000,1.000000}%
\pgfsetfillcolor{currentfill}%
\pgfsetlinewidth{0.000000pt}%
\definecolor{currentstroke}{rgb}{1.000000,1.000000,1.000000}%
\pgfsetstrokecolor{currentstroke}%
\pgfsetdash{}{0pt}%
\pgfpathmoveto{\pgfqpoint{0.000000in}{0.000000in}}%
\pgfpathlineto{\pgfqpoint{5.232821in}{0.000000in}}%
\pgfpathlineto{\pgfqpoint{5.232821in}{3.788793in}}%
\pgfpathlineto{\pgfqpoint{0.000000in}{3.788793in}}%
\pgfpathclose%
\pgfusepath{fill}%
\end{pgfscope}%
\begin{pgfscope}%
\pgfsetbuttcap%
\pgfsetmiterjoin%
\definecolor{currentfill}{rgb}{1.000000,1.000000,1.000000}%
\pgfsetfillcolor{currentfill}%
\pgfsetlinewidth{0.000000pt}%
\definecolor{currentstroke}{rgb}{0.000000,0.000000,0.000000}%
\pgfsetstrokecolor{currentstroke}%
\pgfsetstrokeopacity{0.000000}%
\pgfsetdash{}{0pt}%
\pgfpathmoveto{\pgfqpoint{0.564660in}{0.521603in}}%
\pgfpathlineto{\pgfqpoint{4.284660in}{0.521603in}}%
\pgfpathlineto{\pgfqpoint{4.284660in}{3.541603in}}%
\pgfpathlineto{\pgfqpoint{0.564660in}{3.541603in}}%
\pgfpathclose%
\pgfusepath{fill}%
\end{pgfscope}%
\begin{pgfscope}%
\pgfpathrectangle{\pgfqpoint{0.564660in}{0.521603in}}{\pgfqpoint{3.720000in}{3.020000in}}%
\pgfusepath{clip}%
\pgfsetbuttcap%
\pgfsetroundjoin%
\definecolor{currentfill}{rgb}{0.061765,0.061765,0.085934}%
\pgfsetfillcolor{currentfill}%
\pgfsetlinewidth{0.000000pt}%
\definecolor{currentstroke}{rgb}{0.000000,0.000000,0.000000}%
\pgfsetstrokecolor{currentstroke}%
\pgfsetdash{}{0pt}%
\pgfpathmoveto{\pgfqpoint{1.238132in}{0.750263in}}%
\pgfpathlineto{\pgfqpoint{1.305934in}{0.750263in}}%
\pgfpathlineto{\pgfqpoint{1.373735in}{0.750263in}}%
\pgfpathlineto{\pgfqpoint{1.441537in}{0.750263in}}%
\pgfpathlineto{\pgfqpoint{1.509338in}{0.750263in}}%
\pgfpathlineto{\pgfqpoint{1.577140in}{0.804788in}}%
\pgfpathlineto{\pgfqpoint{1.644942in}{0.804788in}}%
\pgfpathlineto{\pgfqpoint{1.712743in}{0.804788in}}%
\pgfpathlineto{\pgfqpoint{1.780545in}{0.859313in}}%
\pgfpathlineto{\pgfqpoint{1.848347in}{0.859313in}}%
\pgfpathlineto{\pgfqpoint{1.916148in}{0.913838in}}%
\pgfpathlineto{\pgfqpoint{1.983950in}{0.913838in}}%
\pgfpathlineto{\pgfqpoint{2.051751in}{0.913838in}}%
\pgfpathlineto{\pgfqpoint{2.119553in}{0.968363in}}%
\pgfpathlineto{\pgfqpoint{2.187355in}{0.968363in}}%
\pgfpathlineto{\pgfqpoint{2.255156in}{1.022888in}}%
\pgfpathlineto{\pgfqpoint{2.264819in}{1.022888in}}%
\pgfpathlineto{\pgfqpoint{2.255156in}{1.025047in}}%
\pgfpathlineto{\pgfqpoint{2.187355in}{1.025510in}}%
\pgfpathlineto{\pgfqpoint{2.119553in}{1.024502in}}%
\pgfpathlineto{\pgfqpoint{2.078575in}{1.022888in}}%
\pgfpathlineto{\pgfqpoint{2.051751in}{1.021485in}}%
\pgfpathlineto{\pgfqpoint{1.983950in}{1.014951in}}%
\pgfpathlineto{\pgfqpoint{1.916148in}{1.008416in}}%
\pgfpathlineto{\pgfqpoint{1.885240in}{1.022888in}}%
\pgfpathlineto{\pgfqpoint{1.848347in}{1.047771in}}%
\pgfpathlineto{\pgfqpoint{1.805159in}{1.077414in}}%
\pgfpathlineto{\pgfqpoint{1.780545in}{1.090266in}}%
\pgfpathlineto{\pgfqpoint{1.712743in}{1.125668in}}%
\pgfpathlineto{\pgfqpoint{1.700735in}{1.131939in}}%
\pgfpathlineto{\pgfqpoint{1.644942in}{1.161071in}}%
\pgfpathlineto{\pgfqpoint{1.613824in}{1.186464in}}%
\pgfpathlineto{\pgfqpoint{1.577140in}{1.212962in}}%
\pgfpathlineto{\pgfqpoint{1.532843in}{1.240989in}}%
\pgfpathlineto{\pgfqpoint{1.509338in}{1.260474in}}%
\pgfpathlineto{\pgfqpoint{1.441537in}{1.291166in}}%
\pgfpathlineto{\pgfqpoint{1.430114in}{1.295514in}}%
\pgfpathlineto{\pgfqpoint{1.373735in}{1.316975in}}%
\pgfpathlineto{\pgfqpoint{1.305934in}{1.342785in}}%
\pgfpathlineto{\pgfqpoint{1.282780in}{1.295514in}}%
\pgfpathlineto{\pgfqpoint{1.272931in}{1.240989in}}%
\pgfpathlineto{\pgfqpoint{1.268059in}{1.186464in}}%
\pgfpathlineto{\pgfqpoint{1.263374in}{1.131939in}}%
\pgfpathlineto{\pgfqpoint{1.249947in}{1.077414in}}%
\pgfpathlineto{\pgfqpoint{1.238132in}{1.033194in}}%
\pgfpathlineto{\pgfqpoint{1.170330in}{1.023200in}}%
\pgfpathlineto{\pgfqpoint{1.102529in}{1.027231in}}%
\pgfpathlineto{\pgfqpoint{1.034727in}{1.031263in}}%
\pgfpathlineto{\pgfqpoint{0.966926in}{1.035294in}}%
\pgfpathlineto{\pgfqpoint{0.899124in}{1.039326in}}%
\pgfpathlineto{\pgfqpoint{0.831322in}{1.043357in}}%
\pgfpathlineto{\pgfqpoint{0.831322in}{1.022888in}}%
\pgfpathlineto{\pgfqpoint{0.899124in}{0.968363in}}%
\pgfpathlineto{\pgfqpoint{0.966926in}{0.913838in}}%
\pgfpathlineto{\pgfqpoint{1.034727in}{0.913838in}}%
\pgfpathlineto{\pgfqpoint{1.102529in}{0.859313in}}%
\pgfpathlineto{\pgfqpoint{1.170330in}{0.804788in}}%
\pgfpathclose%
\pgfusepath{fill}%
\end{pgfscope}%
\begin{pgfscope}%
\pgfpathrectangle{\pgfqpoint{0.564660in}{0.521603in}}{\pgfqpoint{3.720000in}{3.020000in}}%
\pgfusepath{clip}%
\pgfsetbuttcap%
\pgfsetroundjoin%
\definecolor{currentfill}{rgb}{0.185294,0.185294,0.257801}%
\pgfsetfillcolor{currentfill}%
\pgfsetlinewidth{0.000000pt}%
\definecolor{currentstroke}{rgb}{0.000000,0.000000,0.000000}%
\pgfsetstrokecolor{currentstroke}%
\pgfsetdash{}{0pt}%
\pgfpathmoveto{\pgfqpoint{1.916148in}{1.008416in}}%
\pgfpathlineto{\pgfqpoint{1.983950in}{1.014951in}}%
\pgfpathlineto{\pgfqpoint{2.051751in}{1.021485in}}%
\pgfpathlineto{\pgfqpoint{2.078575in}{1.022888in}}%
\pgfpathlineto{\pgfqpoint{2.119553in}{1.024502in}}%
\pgfpathlineto{\pgfqpoint{2.187355in}{1.025510in}}%
\pgfpathlineto{\pgfqpoint{2.255156in}{1.025047in}}%
\pgfpathlineto{\pgfqpoint{2.264819in}{1.022888in}}%
\pgfpathlineto{\pgfqpoint{2.322958in}{1.022888in}}%
\pgfpathlineto{\pgfqpoint{2.390759in}{1.077414in}}%
\pgfpathlineto{\pgfqpoint{2.458561in}{1.131939in}}%
\pgfpathlineto{\pgfqpoint{2.526363in}{1.186464in}}%
\pgfpathlineto{\pgfqpoint{2.594164in}{1.186464in}}%
\pgfpathlineto{\pgfqpoint{2.661966in}{1.240989in}}%
\pgfpathlineto{\pgfqpoint{2.729768in}{1.295514in}}%
\pgfpathlineto{\pgfqpoint{2.756979in}{1.317397in}}%
\pgfpathlineto{\pgfqpoint{2.729768in}{1.323475in}}%
\pgfpathlineto{\pgfqpoint{2.661966in}{1.338621in}}%
\pgfpathlineto{\pgfqpoint{2.610848in}{1.350039in}}%
\pgfpathlineto{\pgfqpoint{2.594164in}{1.353766in}}%
\pgfpathlineto{\pgfqpoint{2.526363in}{1.368911in}}%
\pgfpathlineto{\pgfqpoint{2.458561in}{1.384057in}}%
\pgfpathlineto{\pgfqpoint{2.390759in}{1.399202in}}%
\pgfpathlineto{\pgfqpoint{2.366754in}{1.404564in}}%
\pgfpathlineto{\pgfqpoint{2.322958in}{1.414347in}}%
\pgfpathlineto{\pgfqpoint{2.255156in}{1.429493in}}%
\pgfpathlineto{\pgfqpoint{2.187355in}{1.444638in}}%
\pgfpathlineto{\pgfqpoint{2.119553in}{1.452237in}}%
\pgfpathlineto{\pgfqpoint{2.051751in}{1.451228in}}%
\pgfpathlineto{\pgfqpoint{1.983950in}{1.450219in}}%
\pgfpathlineto{\pgfqpoint{1.916148in}{1.449211in}}%
\pgfpathlineto{\pgfqpoint{1.905252in}{1.459089in}}%
\pgfpathlineto{\pgfqpoint{1.848347in}{1.505259in}}%
\pgfpathlineto{\pgfqpoint{1.840382in}{1.513615in}}%
\pgfpathlineto{\pgfqpoint{1.788412in}{1.568140in}}%
\pgfpathlineto{\pgfqpoint{1.780545in}{1.576394in}}%
\pgfpathlineto{\pgfqpoint{1.736442in}{1.622665in}}%
\pgfpathlineto{\pgfqpoint{1.712743in}{1.647529in}}%
\pgfpathlineto{\pgfqpoint{1.684473in}{1.677190in}}%
\pgfpathlineto{\pgfqpoint{1.644942in}{1.718664in}}%
\pgfpathlineto{\pgfqpoint{1.631711in}{1.731715in}}%
\pgfpathlineto{\pgfqpoint{1.580533in}{1.786240in}}%
\pgfpathlineto{\pgfqpoint{1.577140in}{1.789800in}}%
\pgfpathlineto{\pgfqpoint{1.519093in}{1.840765in}}%
\pgfpathlineto{\pgfqpoint{1.509338in}{1.851646in}}%
\pgfpathlineto{\pgfqpoint{1.441537in}{1.890104in}}%
\pgfpathlineto{\pgfqpoint{1.433944in}{1.895291in}}%
\pgfpathlineto{\pgfqpoint{1.373735in}{1.931697in}}%
\pgfpathlineto{\pgfqpoint{1.351722in}{1.949816in}}%
\pgfpathlineto{\pgfqpoint{1.305934in}{1.987504in}}%
\pgfpathlineto{\pgfqpoint{1.285478in}{2.004341in}}%
\pgfpathlineto{\pgfqpoint{1.238132in}{2.043312in}}%
\pgfpathlineto{\pgfqpoint{1.196249in}{2.058866in}}%
\pgfpathlineto{\pgfqpoint{1.170330in}{2.067546in}}%
\pgfpathlineto{\pgfqpoint{1.102529in}{2.062611in}}%
\pgfpathlineto{\pgfqpoint{1.102529in}{2.058866in}}%
\pgfpathlineto{\pgfqpoint{1.102529in}{2.004341in}}%
\pgfpathlineto{\pgfqpoint{1.034727in}{1.949816in}}%
\pgfpathlineto{\pgfqpoint{1.034727in}{1.895291in}}%
\pgfpathlineto{\pgfqpoint{0.966926in}{1.840765in}}%
\pgfpathlineto{\pgfqpoint{0.966926in}{1.786240in}}%
\pgfpathlineto{\pgfqpoint{0.966926in}{1.731715in}}%
\pgfpathlineto{\pgfqpoint{0.966926in}{1.677190in}}%
\pgfpathlineto{\pgfqpoint{0.966926in}{1.622665in}}%
\pgfpathlineto{\pgfqpoint{0.899124in}{1.568140in}}%
\pgfpathlineto{\pgfqpoint{0.899124in}{1.513615in}}%
\pgfpathlineto{\pgfqpoint{0.899124in}{1.459089in}}%
\pgfpathlineto{\pgfqpoint{0.899124in}{1.404564in}}%
\pgfpathlineto{\pgfqpoint{0.899124in}{1.350039in}}%
\pgfpathlineto{\pgfqpoint{0.831322in}{1.295514in}}%
\pgfpathlineto{\pgfqpoint{0.831322in}{1.240989in}}%
\pgfpathlineto{\pgfqpoint{0.831322in}{1.186464in}}%
\pgfpathlineto{\pgfqpoint{0.831322in}{1.131939in}}%
\pgfpathlineto{\pgfqpoint{0.831322in}{1.077414in}}%
\pgfpathlineto{\pgfqpoint{0.831322in}{1.043357in}}%
\pgfpathlineto{\pgfqpoint{0.899124in}{1.039326in}}%
\pgfpathlineto{\pgfqpoint{0.966926in}{1.035294in}}%
\pgfpathlineto{\pgfqpoint{1.034727in}{1.031263in}}%
\pgfpathlineto{\pgfqpoint{1.102529in}{1.027231in}}%
\pgfpathlineto{\pgfqpoint{1.170330in}{1.023200in}}%
\pgfpathlineto{\pgfqpoint{1.238132in}{1.033194in}}%
\pgfpathlineto{\pgfqpoint{1.249947in}{1.077414in}}%
\pgfpathlineto{\pgfqpoint{1.263374in}{1.131939in}}%
\pgfpathlineto{\pgfqpoint{1.268059in}{1.186464in}}%
\pgfpathlineto{\pgfqpoint{1.272931in}{1.240989in}}%
\pgfpathlineto{\pgfqpoint{1.282780in}{1.295514in}}%
\pgfpathlineto{\pgfqpoint{1.305934in}{1.342785in}}%
\pgfpathlineto{\pgfqpoint{1.373735in}{1.316975in}}%
\pgfpathlineto{\pgfqpoint{1.430114in}{1.295514in}}%
\pgfpathlineto{\pgfqpoint{1.441537in}{1.291166in}}%
\pgfpathlineto{\pgfqpoint{1.509338in}{1.260474in}}%
\pgfpathlineto{\pgfqpoint{1.532843in}{1.240989in}}%
\pgfpathlineto{\pgfqpoint{1.577140in}{1.212962in}}%
\pgfpathlineto{\pgfqpoint{1.613824in}{1.186464in}}%
\pgfpathlineto{\pgfqpoint{1.644942in}{1.161071in}}%
\pgfpathlineto{\pgfqpoint{1.700735in}{1.131939in}}%
\pgfpathlineto{\pgfqpoint{1.712743in}{1.125668in}}%
\pgfpathlineto{\pgfqpoint{1.780545in}{1.090266in}}%
\pgfpathlineto{\pgfqpoint{1.805159in}{1.077414in}}%
\pgfpathlineto{\pgfqpoint{1.848347in}{1.047771in}}%
\pgfpathlineto{\pgfqpoint{1.885240in}{1.022888in}}%
\pgfpathclose%
\pgfpathmoveto{\pgfqpoint{1.091247in}{1.568140in}}%
\pgfpathlineto{\pgfqpoint{1.101834in}{1.622665in}}%
\pgfpathlineto{\pgfqpoint{1.102529in}{1.623097in}}%
\pgfpathlineto{\pgfqpoint{1.102695in}{1.622665in}}%
\pgfpathlineto{\pgfqpoint{1.105105in}{1.568140in}}%
\pgfpathlineto{\pgfqpoint{1.102529in}{1.554543in}}%
\pgfpathclose%
\pgfusepath{fill}%
\end{pgfscope}%
\begin{pgfscope}%
\pgfpathrectangle{\pgfqpoint{0.564660in}{0.521603in}}{\pgfqpoint{3.720000in}{3.020000in}}%
\pgfusepath{clip}%
\pgfsetbuttcap%
\pgfsetroundjoin%
\definecolor{currentfill}{rgb}{0.185294,0.185294,0.257801}%
\pgfsetfillcolor{currentfill}%
\pgfsetlinewidth{0.000000pt}%
\definecolor{currentstroke}{rgb}{0.000000,0.000000,0.000000}%
\pgfsetstrokecolor{currentstroke}%
\pgfsetdash{}{0pt}%
\pgfpathmoveto{\pgfqpoint{2.933172in}{1.404564in}}%
\pgfpathlineto{\pgfqpoint{2.939480in}{1.409637in}}%
\pgfpathlineto{\pgfqpoint{2.933172in}{1.408107in}}%
\pgfpathlineto{\pgfqpoint{2.927424in}{1.404564in}}%
\pgfpathclose%
\pgfusepath{fill}%
\end{pgfscope}%
\begin{pgfscope}%
\pgfpathrectangle{\pgfqpoint{0.564660in}{0.521603in}}{\pgfqpoint{3.720000in}{3.020000in}}%
\pgfusepath{clip}%
\pgfsetbuttcap%
\pgfsetroundjoin%
\definecolor{currentfill}{rgb}{0.312255,0.312255,0.434442}%
\pgfsetfillcolor{currentfill}%
\pgfsetlinewidth{0.000000pt}%
\definecolor{currentstroke}{rgb}{0.000000,0.000000,0.000000}%
\pgfsetstrokecolor{currentstroke}%
\pgfsetdash{}{0pt}%
\pgfpathmoveto{\pgfqpoint{2.661966in}{1.338621in}}%
\pgfpathlineto{\pgfqpoint{2.729768in}{1.323475in}}%
\pgfpathlineto{\pgfqpoint{2.756979in}{1.317397in}}%
\pgfpathlineto{\pgfqpoint{2.797569in}{1.350039in}}%
\pgfpathlineto{\pgfqpoint{2.865371in}{1.404564in}}%
\pgfpathlineto{\pgfqpoint{2.927424in}{1.404564in}}%
\pgfpathlineto{\pgfqpoint{2.933172in}{1.408107in}}%
\pgfpathlineto{\pgfqpoint{2.939480in}{1.409637in}}%
\pgfpathlineto{\pgfqpoint{3.000974in}{1.459089in}}%
\pgfpathlineto{\pgfqpoint{3.068776in}{1.513615in}}%
\pgfpathlineto{\pgfqpoint{3.136577in}{1.568140in}}%
\pgfpathlineto{\pgfqpoint{3.204379in}{1.568140in}}%
\pgfpathlineto{\pgfqpoint{3.272180in}{1.622665in}}%
\pgfpathlineto{\pgfqpoint{3.339982in}{1.677190in}}%
\pgfpathlineto{\pgfqpoint{3.407784in}{1.731715in}}%
\pgfpathlineto{\pgfqpoint{3.475585in}{1.731715in}}%
\pgfpathlineto{\pgfqpoint{3.543387in}{1.786240in}}%
\pgfpathlineto{\pgfqpoint{3.543387in}{1.840765in}}%
\pgfpathlineto{\pgfqpoint{3.611189in}{1.895291in}}%
\pgfpathlineto{\pgfqpoint{3.678990in}{1.949816in}}%
\pgfpathlineto{\pgfqpoint{3.746792in}{2.004341in}}%
\pgfpathlineto{\pgfqpoint{3.746792in}{2.058866in}}%
\pgfpathlineto{\pgfqpoint{3.814593in}{2.113391in}}%
\pgfpathlineto{\pgfqpoint{3.882395in}{2.167916in}}%
\pgfpathlineto{\pgfqpoint{3.950197in}{2.222441in}}%
\pgfpathlineto{\pgfqpoint{3.950197in}{2.276966in}}%
\pgfpathlineto{\pgfqpoint{4.017998in}{2.331492in}}%
\pgfpathlineto{\pgfqpoint{4.017998in}{2.339530in}}%
\pgfpathlineto{\pgfqpoint{4.006327in}{2.331492in}}%
\pgfpathlineto{\pgfqpoint{3.950197in}{2.292831in}}%
\pgfpathlineto{\pgfqpoint{3.927163in}{2.276966in}}%
\pgfpathlineto{\pgfqpoint{3.882395in}{2.246132in}}%
\pgfpathlineto{\pgfqpoint{3.847998in}{2.222441in}}%
\pgfpathlineto{\pgfqpoint{3.814593in}{2.199433in}}%
\pgfpathlineto{\pgfqpoint{3.768834in}{2.167916in}}%
\pgfpathlineto{\pgfqpoint{3.746792in}{2.152734in}}%
\pgfpathlineto{\pgfqpoint{3.689670in}{2.113391in}}%
\pgfpathlineto{\pgfqpoint{3.678990in}{2.106035in}}%
\pgfpathlineto{\pgfqpoint{3.611189in}{2.059336in}}%
\pgfpathlineto{\pgfqpoint{3.610505in}{2.058866in}}%
\pgfpathlineto{\pgfqpoint{3.543387in}{2.012638in}}%
\pgfpathlineto{\pgfqpoint{3.530445in}{2.004341in}}%
\pgfpathlineto{\pgfqpoint{3.475585in}{1.969319in}}%
\pgfpathlineto{\pgfqpoint{3.440935in}{1.949816in}}%
\pgfpathlineto{\pgfqpoint{3.407784in}{1.931156in}}%
\pgfpathlineto{\pgfqpoint{3.344064in}{1.895291in}}%
\pgfpathlineto{\pgfqpoint{3.339982in}{1.892993in}}%
\pgfpathlineto{\pgfqpoint{3.272180in}{1.854830in}}%
\pgfpathlineto{\pgfqpoint{3.247194in}{1.840765in}}%
\pgfpathlineto{\pgfqpoint{3.204379in}{1.816666in}}%
\pgfpathlineto{\pgfqpoint{3.150323in}{1.786240in}}%
\pgfpathlineto{\pgfqpoint{3.136577in}{1.778503in}}%
\pgfpathlineto{\pgfqpoint{3.068776in}{1.740340in}}%
\pgfpathlineto{\pgfqpoint{3.051071in}{1.731715in}}%
\pgfpathlineto{\pgfqpoint{3.000974in}{1.704773in}}%
\pgfpathlineto{\pgfqpoint{2.933172in}{1.707964in}}%
\pgfpathlineto{\pgfqpoint{2.865371in}{1.716699in}}%
\pgfpathlineto{\pgfqpoint{2.797569in}{1.725434in}}%
\pgfpathlineto{\pgfqpoint{2.748816in}{1.731715in}}%
\pgfpathlineto{\pgfqpoint{2.729768in}{1.734169in}}%
\pgfpathlineto{\pgfqpoint{2.661966in}{1.743597in}}%
\pgfpathlineto{\pgfqpoint{2.594164in}{1.758212in}}%
\pgfpathlineto{\pgfqpoint{2.526363in}{1.773357in}}%
\pgfpathlineto{\pgfqpoint{2.468688in}{1.786240in}}%
\pgfpathlineto{\pgfqpoint{2.458561in}{1.788502in}}%
\pgfpathlineto{\pgfqpoint{2.390759in}{1.803648in}}%
\pgfpathlineto{\pgfqpoint{2.322958in}{1.818793in}}%
\pgfpathlineto{\pgfqpoint{2.255156in}{1.833938in}}%
\pgfpathlineto{\pgfqpoint{2.224593in}{1.840765in}}%
\pgfpathlineto{\pgfqpoint{2.187355in}{1.849084in}}%
\pgfpathlineto{\pgfqpoint{2.119553in}{1.864229in}}%
\pgfpathlineto{\pgfqpoint{2.051751in}{1.878612in}}%
\pgfpathlineto{\pgfqpoint{1.983950in}{1.877955in}}%
\pgfpathlineto{\pgfqpoint{1.916148in}{1.876946in}}%
\pgfpathlineto{\pgfqpoint{1.848347in}{1.888029in}}%
\pgfpathlineto{\pgfqpoint{1.841425in}{1.895291in}}%
\pgfpathlineto{\pgfqpoint{1.789455in}{1.949816in}}%
\pgfpathlineto{\pgfqpoint{1.780545in}{1.959164in}}%
\pgfpathlineto{\pgfqpoint{1.737486in}{2.004341in}}%
\pgfpathlineto{\pgfqpoint{1.712743in}{2.030299in}}%
\pgfpathlineto{\pgfqpoint{1.685516in}{2.058866in}}%
\pgfpathlineto{\pgfqpoint{1.644942in}{2.101435in}}%
\pgfpathlineto{\pgfqpoint{1.633546in}{2.113391in}}%
\pgfpathlineto{\pgfqpoint{1.581576in}{2.167916in}}%
\pgfpathlineto{\pgfqpoint{1.577140in}{2.172570in}}%
\pgfpathlineto{\pgfqpoint{1.529606in}{2.222441in}}%
\pgfpathlineto{\pgfqpoint{1.509338in}{2.243705in}}%
\pgfpathlineto{\pgfqpoint{1.477636in}{2.276966in}}%
\pgfpathlineto{\pgfqpoint{1.441537in}{2.314840in}}%
\pgfpathlineto{\pgfqpoint{1.425666in}{2.331492in}}%
\pgfpathlineto{\pgfqpoint{1.373735in}{2.385975in}}%
\pgfpathlineto{\pgfqpoint{1.373696in}{2.386017in}}%
\pgfpathlineto{\pgfqpoint{1.321726in}{2.440542in}}%
\pgfpathlineto{\pgfqpoint{1.305934in}{2.446237in}}%
\pgfpathlineto{\pgfqpoint{1.305934in}{2.440542in}}%
\pgfpathlineto{\pgfqpoint{1.295552in}{2.432193in}}%
\pgfpathlineto{\pgfqpoint{1.276312in}{2.386017in}}%
\pgfpathlineto{\pgfqpoint{1.238132in}{2.363523in}}%
\pgfpathlineto{\pgfqpoint{1.238132in}{2.331492in}}%
\pgfpathlineto{\pgfqpoint{1.238132in}{2.276966in}}%
\pgfpathlineto{\pgfqpoint{1.170330in}{2.222441in}}%
\pgfpathlineto{\pgfqpoint{1.170330in}{2.167916in}}%
\pgfpathlineto{\pgfqpoint{1.102529in}{2.113391in}}%
\pgfpathlineto{\pgfqpoint{1.102529in}{2.062611in}}%
\pgfpathlineto{\pgfqpoint{1.170330in}{2.067546in}}%
\pgfpathlineto{\pgfqpoint{1.196249in}{2.058866in}}%
\pgfpathlineto{\pgfqpoint{1.238132in}{2.043312in}}%
\pgfpathlineto{\pgfqpoint{1.285478in}{2.004341in}}%
\pgfpathlineto{\pgfqpoint{1.305934in}{1.987504in}}%
\pgfpathlineto{\pgfqpoint{1.351722in}{1.949816in}}%
\pgfpathlineto{\pgfqpoint{1.373735in}{1.931697in}}%
\pgfpathlineto{\pgfqpoint{1.433944in}{1.895291in}}%
\pgfpathlineto{\pgfqpoint{1.441537in}{1.890104in}}%
\pgfpathlineto{\pgfqpoint{1.509338in}{1.851646in}}%
\pgfpathlineto{\pgfqpoint{1.519093in}{1.840765in}}%
\pgfpathlineto{\pgfqpoint{1.577140in}{1.789800in}}%
\pgfpathlineto{\pgfqpoint{1.580533in}{1.786240in}}%
\pgfpathlineto{\pgfqpoint{1.631711in}{1.731715in}}%
\pgfpathlineto{\pgfqpoint{1.644942in}{1.718664in}}%
\pgfpathlineto{\pgfqpoint{1.684473in}{1.677190in}}%
\pgfpathlineto{\pgfqpoint{1.712743in}{1.647529in}}%
\pgfpathlineto{\pgfqpoint{1.736442in}{1.622665in}}%
\pgfpathlineto{\pgfqpoint{1.780545in}{1.576394in}}%
\pgfpathlineto{\pgfqpoint{1.788412in}{1.568140in}}%
\pgfpathlineto{\pgfqpoint{1.840382in}{1.513615in}}%
\pgfpathlineto{\pgfqpoint{1.848347in}{1.505259in}}%
\pgfpathlineto{\pgfqpoint{1.905252in}{1.459089in}}%
\pgfpathlineto{\pgfqpoint{1.916148in}{1.449211in}}%
\pgfpathlineto{\pgfqpoint{1.983950in}{1.450219in}}%
\pgfpathlineto{\pgfqpoint{2.051751in}{1.451228in}}%
\pgfpathlineto{\pgfqpoint{2.119553in}{1.452237in}}%
\pgfpathlineto{\pgfqpoint{2.187355in}{1.444638in}}%
\pgfpathlineto{\pgfqpoint{2.255156in}{1.429493in}}%
\pgfpathlineto{\pgfqpoint{2.322958in}{1.414347in}}%
\pgfpathlineto{\pgfqpoint{2.366754in}{1.404564in}}%
\pgfpathlineto{\pgfqpoint{2.390759in}{1.399202in}}%
\pgfpathlineto{\pgfqpoint{2.458561in}{1.384057in}}%
\pgfpathlineto{\pgfqpoint{2.526363in}{1.368911in}}%
\pgfpathlineto{\pgfqpoint{2.594164in}{1.353766in}}%
\pgfpathlineto{\pgfqpoint{2.610848in}{1.350039in}}%
\pgfpathclose%
\pgfusepath{fill}%
\end{pgfscope}%
\begin{pgfscope}%
\pgfpathrectangle{\pgfqpoint{0.564660in}{0.521603in}}{\pgfqpoint{3.720000in}{3.020000in}}%
\pgfusepath{clip}%
\pgfsetbuttcap%
\pgfsetroundjoin%
\definecolor{currentfill}{rgb}{0.312255,0.312255,0.434442}%
\pgfsetfillcolor{currentfill}%
\pgfsetlinewidth{0.000000pt}%
\definecolor{currentstroke}{rgb}{0.000000,0.000000,0.000000}%
\pgfsetstrokecolor{currentstroke}%
\pgfsetdash{}{0pt}%
\pgfpathmoveto{\pgfqpoint{1.102529in}{1.554543in}}%
\pgfpathlineto{\pgfqpoint{1.105105in}{1.568140in}}%
\pgfpathlineto{\pgfqpoint{1.102695in}{1.622665in}}%
\pgfpathlineto{\pgfqpoint{1.102529in}{1.623097in}}%
\pgfpathlineto{\pgfqpoint{1.101834in}{1.622665in}}%
\pgfpathlineto{\pgfqpoint{1.091247in}{1.568140in}}%
\pgfpathclose%
\pgfusepath{fill}%
\end{pgfscope}%
\begin{pgfscope}%
\pgfpathrectangle{\pgfqpoint{0.564660in}{0.521603in}}{\pgfqpoint{3.720000in}{3.020000in}}%
\pgfusepath{clip}%
\pgfsetbuttcap%
\pgfsetroundjoin%
\definecolor{currentfill}{rgb}{0.439216,0.484130,0.564216}%
\pgfsetfillcolor{currentfill}%
\pgfsetlinewidth{0.000000pt}%
\definecolor{currentstroke}{rgb}{0.000000,0.000000,0.000000}%
\pgfsetstrokecolor{currentstroke}%
\pgfsetdash{}{0pt}%
\pgfpathmoveto{\pgfqpoint{2.797569in}{1.725434in}}%
\pgfpathlineto{\pgfqpoint{2.865371in}{1.716699in}}%
\pgfpathlineto{\pgfqpoint{2.933172in}{1.707964in}}%
\pgfpathlineto{\pgfqpoint{3.000974in}{1.704773in}}%
\pgfpathlineto{\pgfqpoint{3.051071in}{1.731715in}}%
\pgfpathlineto{\pgfqpoint{3.068776in}{1.740340in}}%
\pgfpathlineto{\pgfqpoint{3.136577in}{1.778503in}}%
\pgfpathlineto{\pgfqpoint{3.150323in}{1.786240in}}%
\pgfpathlineto{\pgfqpoint{3.204379in}{1.816666in}}%
\pgfpathlineto{\pgfqpoint{3.247194in}{1.840765in}}%
\pgfpathlineto{\pgfqpoint{3.272180in}{1.854830in}}%
\pgfpathlineto{\pgfqpoint{3.339982in}{1.892993in}}%
\pgfpathlineto{\pgfqpoint{3.344064in}{1.895291in}}%
\pgfpathlineto{\pgfqpoint{3.407784in}{1.931156in}}%
\pgfpathlineto{\pgfqpoint{3.440935in}{1.949816in}}%
\pgfpathlineto{\pgfqpoint{3.475585in}{1.969319in}}%
\pgfpathlineto{\pgfqpoint{3.530445in}{2.004341in}}%
\pgfpathlineto{\pgfqpoint{3.543387in}{2.012638in}}%
\pgfpathlineto{\pgfqpoint{3.610505in}{2.058866in}}%
\pgfpathlineto{\pgfqpoint{3.611189in}{2.059336in}}%
\pgfpathlineto{\pgfqpoint{3.678990in}{2.106035in}}%
\pgfpathlineto{\pgfqpoint{3.689670in}{2.113391in}}%
\pgfpathlineto{\pgfqpoint{3.746792in}{2.152734in}}%
\pgfpathlineto{\pgfqpoint{3.768834in}{2.167916in}}%
\pgfpathlineto{\pgfqpoint{3.814593in}{2.199433in}}%
\pgfpathlineto{\pgfqpoint{3.847998in}{2.222441in}}%
\pgfpathlineto{\pgfqpoint{3.882395in}{2.246132in}}%
\pgfpathlineto{\pgfqpoint{3.927163in}{2.276966in}}%
\pgfpathlineto{\pgfqpoint{3.950197in}{2.292831in}}%
\pgfpathlineto{\pgfqpoint{4.006327in}{2.331492in}}%
\pgfpathlineto{\pgfqpoint{4.017998in}{2.339530in}}%
\pgfpathlineto{\pgfqpoint{4.017998in}{2.386017in}}%
\pgfpathlineto{\pgfqpoint{3.950197in}{2.440542in}}%
\pgfpathlineto{\pgfqpoint{3.882395in}{2.495067in}}%
\pgfpathlineto{\pgfqpoint{3.882395in}{2.549592in}}%
\pgfpathlineto{\pgfqpoint{3.868264in}{2.560956in}}%
\pgfpathlineto{\pgfqpoint{3.851765in}{2.549592in}}%
\pgfpathlineto{\pgfqpoint{3.814593in}{2.523990in}}%
\pgfpathlineto{\pgfqpoint{3.772600in}{2.495067in}}%
\pgfpathlineto{\pgfqpoint{3.746792in}{2.477291in}}%
\pgfpathlineto{\pgfqpoint{3.693436in}{2.440542in}}%
\pgfpathlineto{\pgfqpoint{3.678990in}{2.430592in}}%
\pgfpathlineto{\pgfqpoint{3.614272in}{2.386017in}}%
\pgfpathlineto{\pgfqpoint{3.611189in}{2.383893in}}%
\pgfpathlineto{\pgfqpoint{3.543387in}{2.337194in}}%
\pgfpathlineto{\pgfqpoint{3.534764in}{2.331492in}}%
\pgfpathlineto{\pgfqpoint{3.475585in}{2.292518in}}%
\pgfpathlineto{\pgfqpoint{3.447956in}{2.276966in}}%
\pgfpathlineto{\pgfqpoint{3.407784in}{2.254355in}}%
\pgfpathlineto{\pgfqpoint{3.351085in}{2.222441in}}%
\pgfpathlineto{\pgfqpoint{3.339982in}{2.216192in}}%
\pgfpathlineto{\pgfqpoint{3.272180in}{2.178028in}}%
\pgfpathlineto{\pgfqpoint{3.254215in}{2.167916in}}%
\pgfpathlineto{\pgfqpoint{3.204379in}{2.139865in}}%
\pgfpathlineto{\pgfqpoint{3.139209in}{2.113391in}}%
\pgfpathlineto{\pgfqpoint{3.136577in}{2.112063in}}%
\pgfpathlineto{\pgfqpoint{3.125530in}{2.113391in}}%
\pgfpathlineto{\pgfqpoint{3.068776in}{2.120703in}}%
\pgfpathlineto{\pgfqpoint{3.000974in}{2.129438in}}%
\pgfpathlineto{\pgfqpoint{2.933172in}{2.138173in}}%
\pgfpathlineto{\pgfqpoint{2.865371in}{2.146908in}}%
\pgfpathlineto{\pgfqpoint{2.797569in}{2.155643in}}%
\pgfpathlineto{\pgfqpoint{2.729768in}{2.164378in}}%
\pgfpathlineto{\pgfqpoint{2.702308in}{2.167916in}}%
\pgfpathlineto{\pgfqpoint{2.661966in}{2.173114in}}%
\pgfpathlineto{\pgfqpoint{2.594164in}{2.181849in}}%
\pgfpathlineto{\pgfqpoint{2.526363in}{2.190584in}}%
\pgfpathlineto{\pgfqpoint{2.458561in}{2.199319in}}%
\pgfpathlineto{\pgfqpoint{2.390759in}{2.208709in}}%
\pgfpathlineto{\pgfqpoint{2.326733in}{2.222441in}}%
\pgfpathlineto{\pgfqpoint{2.322958in}{2.223239in}}%
\pgfpathlineto{\pgfqpoint{2.255156in}{2.238384in}}%
\pgfpathlineto{\pgfqpoint{2.187355in}{2.253529in}}%
\pgfpathlineto{\pgfqpoint{2.119553in}{2.268675in}}%
\pgfpathlineto{\pgfqpoint{2.082433in}{2.276966in}}%
\pgfpathlineto{\pgfqpoint{2.051751in}{2.283820in}}%
\pgfpathlineto{\pgfqpoint{1.983950in}{2.298965in}}%
\pgfpathlineto{\pgfqpoint{1.916148in}{2.304681in}}%
\pgfpathlineto{\pgfqpoint{1.848347in}{2.303672in}}%
\pgfpathlineto{\pgfqpoint{1.800582in}{2.331492in}}%
\pgfpathlineto{\pgfqpoint{1.780545in}{2.341934in}}%
\pgfpathlineto{\pgfqpoint{1.738529in}{2.386017in}}%
\pgfpathlineto{\pgfqpoint{1.712743in}{2.413070in}}%
\pgfpathlineto{\pgfqpoint{1.686559in}{2.440542in}}%
\pgfpathlineto{\pgfqpoint{1.644942in}{2.484205in}}%
\pgfpathlineto{\pgfqpoint{1.634589in}{2.495067in}}%
\pgfpathlineto{\pgfqpoint{1.582619in}{2.549592in}}%
\pgfpathlineto{\pgfqpoint{1.577140in}{2.555340in}}%
\pgfpathlineto{\pgfqpoint{1.530649in}{2.604117in}}%
\pgfpathlineto{\pgfqpoint{1.509338in}{2.626475in}}%
\pgfpathlineto{\pgfqpoint{1.478679in}{2.658642in}}%
\pgfpathlineto{\pgfqpoint{1.441537in}{2.697610in}}%
\pgfpathlineto{\pgfqpoint{1.404225in}{2.658642in}}%
\pgfpathlineto{\pgfqpoint{1.373735in}{2.649805in}}%
\pgfpathlineto{\pgfqpoint{1.373735in}{2.604117in}}%
\pgfpathlineto{\pgfqpoint{1.373735in}{2.549592in}}%
\pgfpathlineto{\pgfqpoint{1.305934in}{2.495067in}}%
\pgfpathlineto{\pgfqpoint{1.305934in}{2.446237in}}%
\pgfpathlineto{\pgfqpoint{1.321726in}{2.440542in}}%
\pgfpathlineto{\pgfqpoint{1.373696in}{2.386017in}}%
\pgfpathlineto{\pgfqpoint{1.373735in}{2.385975in}}%
\pgfpathlineto{\pgfqpoint{1.425666in}{2.331492in}}%
\pgfpathlineto{\pgfqpoint{1.441537in}{2.314840in}}%
\pgfpathlineto{\pgfqpoint{1.477636in}{2.276966in}}%
\pgfpathlineto{\pgfqpoint{1.509338in}{2.243705in}}%
\pgfpathlineto{\pgfqpoint{1.529606in}{2.222441in}}%
\pgfpathlineto{\pgfqpoint{1.577140in}{2.172570in}}%
\pgfpathlineto{\pgfqpoint{1.581576in}{2.167916in}}%
\pgfpathlineto{\pgfqpoint{1.633546in}{2.113391in}}%
\pgfpathlineto{\pgfqpoint{1.644942in}{2.101435in}}%
\pgfpathlineto{\pgfqpoint{1.685516in}{2.058866in}}%
\pgfpathlineto{\pgfqpoint{1.712743in}{2.030299in}}%
\pgfpathlineto{\pgfqpoint{1.737486in}{2.004341in}}%
\pgfpathlineto{\pgfqpoint{1.780545in}{1.959164in}}%
\pgfpathlineto{\pgfqpoint{1.789455in}{1.949816in}}%
\pgfpathlineto{\pgfqpoint{1.841425in}{1.895291in}}%
\pgfpathlineto{\pgfqpoint{1.848347in}{1.888029in}}%
\pgfpathlineto{\pgfqpoint{1.916148in}{1.876946in}}%
\pgfpathlineto{\pgfqpoint{1.983950in}{1.877955in}}%
\pgfpathlineto{\pgfqpoint{2.051751in}{1.878612in}}%
\pgfpathlineto{\pgfqpoint{2.119553in}{1.864229in}}%
\pgfpathlineto{\pgfqpoint{2.187355in}{1.849084in}}%
\pgfpathlineto{\pgfqpoint{2.224593in}{1.840765in}}%
\pgfpathlineto{\pgfqpoint{2.255156in}{1.833938in}}%
\pgfpathlineto{\pgfqpoint{2.322958in}{1.818793in}}%
\pgfpathlineto{\pgfqpoint{2.390759in}{1.803648in}}%
\pgfpathlineto{\pgfqpoint{2.458561in}{1.788502in}}%
\pgfpathlineto{\pgfqpoint{2.468688in}{1.786240in}}%
\pgfpathlineto{\pgfqpoint{2.526363in}{1.773357in}}%
\pgfpathlineto{\pgfqpoint{2.594164in}{1.758212in}}%
\pgfpathlineto{\pgfqpoint{2.661966in}{1.743597in}}%
\pgfpathlineto{\pgfqpoint{2.729768in}{1.734169in}}%
\pgfpathlineto{\pgfqpoint{2.748816in}{1.731715in}}%
\pgfpathclose%
\pgfusepath{fill}%
\end{pgfscope}%
\begin{pgfscope}%
\pgfpathrectangle{\pgfqpoint{0.564660in}{0.521603in}}{\pgfqpoint{3.720000in}{3.020000in}}%
\pgfusepath{clip}%
\pgfsetbuttcap%
\pgfsetroundjoin%
\definecolor{currentfill}{rgb}{0.439216,0.484130,0.564216}%
\pgfsetfillcolor{currentfill}%
\pgfsetlinewidth{0.000000pt}%
\definecolor{currentstroke}{rgb}{0.000000,0.000000,0.000000}%
\pgfsetstrokecolor{currentstroke}%
\pgfsetdash{}{0pt}%
\pgfpathmoveto{\pgfqpoint{1.276312in}{2.386017in}}%
\pgfpathlineto{\pgfqpoint{1.295552in}{2.432193in}}%
\pgfpathlineto{\pgfqpoint{1.238132in}{2.386017in}}%
\pgfpathlineto{\pgfqpoint{1.238132in}{2.363523in}}%
\pgfpathclose%
\pgfusepath{fill}%
\end{pgfscope}%
\begin{pgfscope}%
\pgfpathrectangle{\pgfqpoint{0.564660in}{0.521603in}}{\pgfqpoint{3.720000in}{3.020000in}}%
\pgfusepath{clip}%
\pgfsetbuttcap%
\pgfsetroundjoin%
\definecolor{currentfill}{rgb}{0.562745,0.653983,0.687745}%
\pgfsetfillcolor{currentfill}%
\pgfsetlinewidth{0.000000pt}%
\definecolor{currentstroke}{rgb}{0.000000,0.000000,0.000000}%
\pgfsetstrokecolor{currentstroke}%
\pgfsetdash{}{0pt}%
\pgfpathmoveto{\pgfqpoint{3.136577in}{2.112063in}}%
\pgfpathlineto{\pgfqpoint{3.139209in}{2.113391in}}%
\pgfpathlineto{\pgfqpoint{3.204379in}{2.139865in}}%
\pgfpathlineto{\pgfqpoint{3.254215in}{2.167916in}}%
\pgfpathlineto{\pgfqpoint{3.272180in}{2.178028in}}%
\pgfpathlineto{\pgfqpoint{3.339982in}{2.216192in}}%
\pgfpathlineto{\pgfqpoint{3.351085in}{2.222441in}}%
\pgfpathlineto{\pgfqpoint{3.407784in}{2.254355in}}%
\pgfpathlineto{\pgfqpoint{3.447956in}{2.276966in}}%
\pgfpathlineto{\pgfqpoint{3.475585in}{2.292518in}}%
\pgfpathlineto{\pgfqpoint{3.534764in}{2.331492in}}%
\pgfpathlineto{\pgfqpoint{3.543387in}{2.337194in}}%
\pgfpathlineto{\pgfqpoint{3.611189in}{2.383893in}}%
\pgfpathlineto{\pgfqpoint{3.614272in}{2.386017in}}%
\pgfpathlineto{\pgfqpoint{3.678990in}{2.430592in}}%
\pgfpathlineto{\pgfqpoint{3.693436in}{2.440542in}}%
\pgfpathlineto{\pgfqpoint{3.746792in}{2.477291in}}%
\pgfpathlineto{\pgfqpoint{3.772600in}{2.495067in}}%
\pgfpathlineto{\pgfqpoint{3.814593in}{2.523990in}}%
\pgfpathlineto{\pgfqpoint{3.851765in}{2.549592in}}%
\pgfpathlineto{\pgfqpoint{3.868264in}{2.560956in}}%
\pgfpathlineto{\pgfqpoint{3.814593in}{2.604117in}}%
\pgfpathlineto{\pgfqpoint{3.746792in}{2.658642in}}%
\pgfpathlineto{\pgfqpoint{3.678990in}{2.713167in}}%
\pgfpathlineto{\pgfqpoint{3.650870in}{2.735781in}}%
\pgfpathlineto{\pgfqpoint{3.618038in}{2.713167in}}%
\pgfpathlineto{\pgfqpoint{3.611189in}{2.708450in}}%
\pgfpathlineto{\pgfqpoint{3.543387in}{2.661751in}}%
\pgfpathlineto{\pgfqpoint{3.538827in}{2.658642in}}%
\pgfpathlineto{\pgfqpoint{3.475585in}{2.615717in}}%
\pgfpathlineto{\pgfqpoint{3.454977in}{2.604117in}}%
\pgfpathlineto{\pgfqpoint{3.407784in}{2.577554in}}%
\pgfpathlineto{\pgfqpoint{3.358106in}{2.549592in}}%
\pgfpathlineto{\pgfqpoint{3.339982in}{2.539390in}}%
\pgfpathlineto{\pgfqpoint{3.272180in}{2.524707in}}%
\pgfpathlineto{\pgfqpoint{3.204379in}{2.533442in}}%
\pgfpathlineto{\pgfqpoint{3.136577in}{2.542177in}}%
\pgfpathlineto{\pgfqpoint{3.079023in}{2.549592in}}%
\pgfpathlineto{\pgfqpoint{3.068776in}{2.550912in}}%
\pgfpathlineto{\pgfqpoint{3.000974in}{2.559647in}}%
\pgfpathlineto{\pgfqpoint{2.933172in}{2.568382in}}%
\pgfpathlineto{\pgfqpoint{2.865371in}{2.577118in}}%
\pgfpathlineto{\pgfqpoint{2.797569in}{2.585853in}}%
\pgfpathlineto{\pgfqpoint{2.729768in}{2.594588in}}%
\pgfpathlineto{\pgfqpoint{2.661966in}{2.603323in}}%
\pgfpathlineto{\pgfqpoint{2.655801in}{2.604117in}}%
\pgfpathlineto{\pgfqpoint{2.594164in}{2.612058in}}%
\pgfpathlineto{\pgfqpoint{2.526363in}{2.620793in}}%
\pgfpathlineto{\pgfqpoint{2.458561in}{2.629528in}}%
\pgfpathlineto{\pgfqpoint{2.390759in}{2.638263in}}%
\pgfpathlineto{\pgfqpoint{2.322958in}{2.646998in}}%
\pgfpathlineto{\pgfqpoint{2.255156in}{2.655734in}}%
\pgfpathlineto{\pgfqpoint{2.232579in}{2.658642in}}%
\pgfpathlineto{\pgfqpoint{2.187355in}{2.664469in}}%
\pgfpathlineto{\pgfqpoint{2.119553in}{2.673809in}}%
\pgfpathlineto{\pgfqpoint{2.051751in}{2.688266in}}%
\pgfpathlineto{\pgfqpoint{1.983950in}{2.703411in}}%
\pgfpathlineto{\pgfqpoint{1.940273in}{2.713167in}}%
\pgfpathlineto{\pgfqpoint{1.916148in}{2.718556in}}%
\pgfpathlineto{\pgfqpoint{1.848347in}{2.731408in}}%
\pgfpathlineto{\pgfqpoint{1.780545in}{2.730399in}}%
\pgfpathlineto{\pgfqpoint{1.743764in}{2.767693in}}%
\pgfpathlineto{\pgfqpoint{1.712743in}{2.795840in}}%
\pgfpathlineto{\pgfqpoint{1.687602in}{2.822218in}}%
\pgfpathlineto{\pgfqpoint{1.644942in}{2.866975in}}%
\pgfpathlineto{\pgfqpoint{1.635632in}{2.876743in}}%
\pgfpathlineto{\pgfqpoint{1.583662in}{2.931268in}}%
\pgfpathlineto{\pgfqpoint{1.577140in}{2.938110in}}%
\pgfpathlineto{\pgfqpoint{1.534395in}{2.951418in}}%
\pgfpathlineto{\pgfqpoint{1.509338in}{2.931268in}}%
\pgfpathlineto{\pgfqpoint{1.509338in}{2.876743in}}%
\pgfpathlineto{\pgfqpoint{1.509338in}{2.822218in}}%
\pgfpathlineto{\pgfqpoint{1.441537in}{2.767693in}}%
\pgfpathlineto{\pgfqpoint{1.441537in}{2.713167in}}%
\pgfpathlineto{\pgfqpoint{1.373735in}{2.658642in}}%
\pgfpathlineto{\pgfqpoint{1.373735in}{2.649805in}}%
\pgfpathlineto{\pgfqpoint{1.404225in}{2.658642in}}%
\pgfpathlineto{\pgfqpoint{1.441537in}{2.697610in}}%
\pgfpathlineto{\pgfqpoint{1.478679in}{2.658642in}}%
\pgfpathlineto{\pgfqpoint{1.509338in}{2.626475in}}%
\pgfpathlineto{\pgfqpoint{1.530649in}{2.604117in}}%
\pgfpathlineto{\pgfqpoint{1.577140in}{2.555340in}}%
\pgfpathlineto{\pgfqpoint{1.582619in}{2.549592in}}%
\pgfpathlineto{\pgfqpoint{1.634589in}{2.495067in}}%
\pgfpathlineto{\pgfqpoint{1.644942in}{2.484205in}}%
\pgfpathlineto{\pgfqpoint{1.686559in}{2.440542in}}%
\pgfpathlineto{\pgfqpoint{1.712743in}{2.413070in}}%
\pgfpathlineto{\pgfqpoint{1.738529in}{2.386017in}}%
\pgfpathlineto{\pgfqpoint{1.780545in}{2.341934in}}%
\pgfpathlineto{\pgfqpoint{1.800582in}{2.331492in}}%
\pgfpathlineto{\pgfqpoint{1.848347in}{2.303672in}}%
\pgfpathlineto{\pgfqpoint{1.916148in}{2.304681in}}%
\pgfpathlineto{\pgfqpoint{1.983950in}{2.298965in}}%
\pgfpathlineto{\pgfqpoint{2.051751in}{2.283820in}}%
\pgfpathlineto{\pgfqpoint{2.082433in}{2.276966in}}%
\pgfpathlineto{\pgfqpoint{2.119553in}{2.268675in}}%
\pgfpathlineto{\pgfqpoint{2.187355in}{2.253529in}}%
\pgfpathlineto{\pgfqpoint{2.255156in}{2.238384in}}%
\pgfpathlineto{\pgfqpoint{2.322958in}{2.223239in}}%
\pgfpathlineto{\pgfqpoint{2.326733in}{2.222441in}}%
\pgfpathlineto{\pgfqpoint{2.390759in}{2.208709in}}%
\pgfpathlineto{\pgfqpoint{2.458561in}{2.199319in}}%
\pgfpathlineto{\pgfqpoint{2.526363in}{2.190584in}}%
\pgfpathlineto{\pgfqpoint{2.594164in}{2.181849in}}%
\pgfpathlineto{\pgfqpoint{2.661966in}{2.173114in}}%
\pgfpathlineto{\pgfqpoint{2.702308in}{2.167916in}}%
\pgfpathlineto{\pgfqpoint{2.729768in}{2.164378in}}%
\pgfpathlineto{\pgfqpoint{2.797569in}{2.155643in}}%
\pgfpathlineto{\pgfqpoint{2.865371in}{2.146908in}}%
\pgfpathlineto{\pgfqpoint{2.933172in}{2.138173in}}%
\pgfpathlineto{\pgfqpoint{3.000974in}{2.129438in}}%
\pgfpathlineto{\pgfqpoint{3.068776in}{2.120703in}}%
\pgfpathlineto{\pgfqpoint{3.125530in}{2.113391in}}%
\pgfpathclose%
\pgfusepath{fill}%
\end{pgfscope}%
\begin{pgfscope}%
\pgfpathrectangle{\pgfqpoint{0.564660in}{0.521603in}}{\pgfqpoint{3.720000in}{3.020000in}}%
\pgfusepath{clip}%
\pgfsetbuttcap%
\pgfsetroundjoin%
\definecolor{currentfill}{rgb}{0.710478,0.814706,0.814706}%
\pgfsetfillcolor{currentfill}%
\pgfsetlinewidth{0.000000pt}%
\definecolor{currentstroke}{rgb}{0.000000,0.000000,0.000000}%
\pgfsetstrokecolor{currentstroke}%
\pgfsetdash{}{0pt}%
\pgfpathmoveto{\pgfqpoint{3.136577in}{2.542177in}}%
\pgfpathlineto{\pgfqpoint{3.204379in}{2.533442in}}%
\pgfpathlineto{\pgfqpoint{3.272180in}{2.524707in}}%
\pgfpathlineto{\pgfqpoint{3.339982in}{2.539390in}}%
\pgfpathlineto{\pgfqpoint{3.358106in}{2.549592in}}%
\pgfpathlineto{\pgfqpoint{3.407784in}{2.577554in}}%
\pgfpathlineto{\pgfqpoint{3.454977in}{2.604117in}}%
\pgfpathlineto{\pgfqpoint{3.475585in}{2.615717in}}%
\pgfpathlineto{\pgfqpoint{3.538827in}{2.658642in}}%
\pgfpathlineto{\pgfqpoint{3.543387in}{2.661751in}}%
\pgfpathlineto{\pgfqpoint{3.611189in}{2.708450in}}%
\pgfpathlineto{\pgfqpoint{3.618038in}{2.713167in}}%
\pgfpathlineto{\pgfqpoint{3.650870in}{2.735781in}}%
\pgfpathlineto{\pgfqpoint{3.611189in}{2.767693in}}%
\pgfpathlineto{\pgfqpoint{3.543387in}{2.822218in}}%
\pgfpathlineto{\pgfqpoint{3.543387in}{2.876743in}}%
\pgfpathlineto{\pgfqpoint{3.475585in}{2.931268in}}%
\pgfpathlineto{\pgfqpoint{3.407784in}{2.931268in}}%
\pgfpathlineto{\pgfqpoint{3.339982in}{2.931268in}}%
\pgfpathlineto{\pgfqpoint{3.317900in}{2.949026in}}%
\pgfpathlineto{\pgfqpoint{3.272180in}{2.954916in}}%
\pgfpathlineto{\pgfqpoint{3.204379in}{2.963651in}}%
\pgfpathlineto{\pgfqpoint{3.136577in}{2.972386in}}%
\pgfpathlineto{\pgfqpoint{3.068776in}{2.981122in}}%
\pgfpathlineto{\pgfqpoint{3.032515in}{2.985793in}}%
\pgfpathlineto{\pgfqpoint{3.000974in}{2.989857in}}%
\pgfpathlineto{\pgfqpoint{2.933172in}{2.998592in}}%
\pgfpathlineto{\pgfqpoint{2.865371in}{3.007327in}}%
\pgfpathlineto{\pgfqpoint{2.797569in}{3.016062in}}%
\pgfpathlineto{\pgfqpoint{2.729768in}{3.024797in}}%
\pgfpathlineto{\pgfqpoint{2.661966in}{3.033532in}}%
\pgfpathlineto{\pgfqpoint{2.609293in}{3.040318in}}%
\pgfpathlineto{\pgfqpoint{2.594164in}{3.042267in}}%
\pgfpathlineto{\pgfqpoint{2.526363in}{3.051002in}}%
\pgfpathlineto{\pgfqpoint{2.458561in}{3.059738in}}%
\pgfpathlineto{\pgfqpoint{2.390759in}{3.068473in}}%
\pgfpathlineto{\pgfqpoint{2.322958in}{3.077208in}}%
\pgfpathlineto{\pgfqpoint{2.255156in}{3.085943in}}%
\pgfpathlineto{\pgfqpoint{2.187355in}{3.094678in}}%
\pgfpathlineto{\pgfqpoint{2.186072in}{3.094843in}}%
\pgfpathlineto{\pgfqpoint{2.119553in}{3.103413in}}%
\pgfpathlineto{\pgfqpoint{2.051751in}{3.112148in}}%
\pgfpathlineto{\pgfqpoint{1.983950in}{3.120883in}}%
\pgfpathlineto{\pgfqpoint{1.916148in}{3.129618in}}%
\pgfpathlineto{\pgfqpoint{1.848347in}{3.138846in}}%
\pgfpathlineto{\pgfqpoint{1.799178in}{3.149368in}}%
\pgfpathlineto{\pgfqpoint{1.780545in}{3.153293in}}%
\pgfpathlineto{\pgfqpoint{1.712743in}{3.178610in}}%
\pgfpathlineto{\pgfqpoint{1.688645in}{3.203894in}}%
\pgfpathlineto{\pgfqpoint{1.669682in}{3.223789in}}%
\pgfpathlineto{\pgfqpoint{1.644942in}{3.203894in}}%
\pgfpathlineto{\pgfqpoint{1.644942in}{3.149368in}}%
\pgfpathlineto{\pgfqpoint{1.644942in}{3.094843in}}%
\pgfpathlineto{\pgfqpoint{1.577140in}{3.040318in}}%
\pgfpathlineto{\pgfqpoint{1.577140in}{2.985793in}}%
\pgfpathlineto{\pgfqpoint{1.534395in}{2.951418in}}%
\pgfpathlineto{\pgfqpoint{1.577140in}{2.938110in}}%
\pgfpathlineto{\pgfqpoint{1.583662in}{2.931268in}}%
\pgfpathlineto{\pgfqpoint{1.635632in}{2.876743in}}%
\pgfpathlineto{\pgfqpoint{1.644942in}{2.866975in}}%
\pgfpathlineto{\pgfqpoint{1.687602in}{2.822218in}}%
\pgfpathlineto{\pgfqpoint{1.712743in}{2.795840in}}%
\pgfpathlineto{\pgfqpoint{1.743764in}{2.767693in}}%
\pgfpathlineto{\pgfqpoint{1.780545in}{2.730399in}}%
\pgfpathlineto{\pgfqpoint{1.848347in}{2.731408in}}%
\pgfpathlineto{\pgfqpoint{1.916148in}{2.718556in}}%
\pgfpathlineto{\pgfqpoint{1.940273in}{2.713167in}}%
\pgfpathlineto{\pgfqpoint{1.983950in}{2.703411in}}%
\pgfpathlineto{\pgfqpoint{2.051751in}{2.688266in}}%
\pgfpathlineto{\pgfqpoint{2.119553in}{2.673809in}}%
\pgfpathlineto{\pgfqpoint{2.187355in}{2.664469in}}%
\pgfpathlineto{\pgfqpoint{2.232579in}{2.658642in}}%
\pgfpathlineto{\pgfqpoint{2.255156in}{2.655734in}}%
\pgfpathlineto{\pgfqpoint{2.322958in}{2.646998in}}%
\pgfpathlineto{\pgfqpoint{2.390759in}{2.638263in}}%
\pgfpathlineto{\pgfqpoint{2.458561in}{2.629528in}}%
\pgfpathlineto{\pgfqpoint{2.526363in}{2.620793in}}%
\pgfpathlineto{\pgfqpoint{2.594164in}{2.612058in}}%
\pgfpathlineto{\pgfqpoint{2.655801in}{2.604117in}}%
\pgfpathlineto{\pgfqpoint{2.661966in}{2.603323in}}%
\pgfpathlineto{\pgfqpoint{2.729768in}{2.594588in}}%
\pgfpathlineto{\pgfqpoint{2.797569in}{2.585853in}}%
\pgfpathlineto{\pgfqpoint{2.865371in}{2.577118in}}%
\pgfpathlineto{\pgfqpoint{2.933172in}{2.568382in}}%
\pgfpathlineto{\pgfqpoint{3.000974in}{2.559647in}}%
\pgfpathlineto{\pgfqpoint{3.068776in}{2.550912in}}%
\pgfpathlineto{\pgfqpoint{3.079023in}{2.549592in}}%
\pgfpathclose%
\pgfusepath{fill}%
\end{pgfscope}%
\begin{pgfscope}%
\pgfpathrectangle{\pgfqpoint{0.564660in}{0.521603in}}{\pgfqpoint{3.720000in}{3.020000in}}%
\pgfusepath{clip}%
\pgfsetbuttcap%
\pgfsetroundjoin%
\definecolor{currentfill}{rgb}{0.903493,0.938235,0.938235}%
\pgfsetfillcolor{currentfill}%
\pgfsetlinewidth{0.000000pt}%
\definecolor{currentstroke}{rgb}{0.000000,0.000000,0.000000}%
\pgfsetstrokecolor{currentstroke}%
\pgfsetdash{}{0pt}%
\pgfpathmoveto{\pgfqpoint{3.068776in}{2.981122in}}%
\pgfpathlineto{\pgfqpoint{3.136577in}{2.972386in}}%
\pgfpathlineto{\pgfqpoint{3.204379in}{2.963651in}}%
\pgfpathlineto{\pgfqpoint{3.272180in}{2.954916in}}%
\pgfpathlineto{\pgfqpoint{3.317900in}{2.949026in}}%
\pgfpathlineto{\pgfqpoint{3.272180in}{2.985793in}}%
\pgfpathlineto{\pgfqpoint{3.204379in}{2.985793in}}%
\pgfpathlineto{\pgfqpoint{3.136577in}{2.985793in}}%
\pgfpathlineto{\pgfqpoint{3.068776in}{3.040318in}}%
\pgfpathlineto{\pgfqpoint{3.000974in}{3.040318in}}%
\pgfpathlineto{\pgfqpoint{2.933172in}{3.040318in}}%
\pgfpathlineto{\pgfqpoint{2.865371in}{3.094843in}}%
\pgfpathlineto{\pgfqpoint{2.797569in}{3.094843in}}%
\pgfpathlineto{\pgfqpoint{2.729768in}{3.094843in}}%
\pgfpathlineto{\pgfqpoint{2.661966in}{3.094843in}}%
\pgfpathlineto{\pgfqpoint{2.594164in}{3.149368in}}%
\pgfpathlineto{\pgfqpoint{2.526363in}{3.149368in}}%
\pgfpathlineto{\pgfqpoint{2.458561in}{3.149368in}}%
\pgfpathlineto{\pgfqpoint{2.390759in}{3.203894in}}%
\pgfpathlineto{\pgfqpoint{2.322958in}{3.203894in}}%
\pgfpathlineto{\pgfqpoint{2.255156in}{3.203894in}}%
\pgfpathlineto{\pgfqpoint{2.187355in}{3.203894in}}%
\pgfpathlineto{\pgfqpoint{2.119553in}{3.258419in}}%
\pgfpathlineto{\pgfqpoint{2.051751in}{3.258419in}}%
\pgfpathlineto{\pgfqpoint{1.983950in}{3.258419in}}%
\pgfpathlineto{\pgfqpoint{1.916148in}{3.312944in}}%
\pgfpathlineto{\pgfqpoint{1.848347in}{3.312944in}}%
\pgfpathlineto{\pgfqpoint{1.780545in}{3.312944in}}%
\pgfpathlineto{\pgfqpoint{1.712743in}{3.312944in}}%
\pgfpathlineto{\pgfqpoint{1.712743in}{3.258419in}}%
\pgfpathlineto{\pgfqpoint{1.669682in}{3.223789in}}%
\pgfpathlineto{\pgfqpoint{1.688645in}{3.203894in}}%
\pgfpathlineto{\pgfqpoint{1.712743in}{3.178610in}}%
\pgfpathlineto{\pgfqpoint{1.780545in}{3.153293in}}%
\pgfpathlineto{\pgfqpoint{1.799178in}{3.149368in}}%
\pgfpathlineto{\pgfqpoint{1.848347in}{3.138846in}}%
\pgfpathlineto{\pgfqpoint{1.916148in}{3.129618in}}%
\pgfpathlineto{\pgfqpoint{1.983950in}{3.120883in}}%
\pgfpathlineto{\pgfqpoint{2.051751in}{3.112148in}}%
\pgfpathlineto{\pgfqpoint{2.119553in}{3.103413in}}%
\pgfpathlineto{\pgfqpoint{2.186072in}{3.094843in}}%
\pgfpathlineto{\pgfqpoint{2.187355in}{3.094678in}}%
\pgfpathlineto{\pgfqpoint{2.255156in}{3.085943in}}%
\pgfpathlineto{\pgfqpoint{2.322958in}{3.077208in}}%
\pgfpathlineto{\pgfqpoint{2.390759in}{3.068473in}}%
\pgfpathlineto{\pgfqpoint{2.458561in}{3.059738in}}%
\pgfpathlineto{\pgfqpoint{2.526363in}{3.051002in}}%
\pgfpathlineto{\pgfqpoint{2.594164in}{3.042267in}}%
\pgfpathlineto{\pgfqpoint{2.609293in}{3.040318in}}%
\pgfpathlineto{\pgfqpoint{2.661966in}{3.033532in}}%
\pgfpathlineto{\pgfqpoint{2.729768in}{3.024797in}}%
\pgfpathlineto{\pgfqpoint{2.797569in}{3.016062in}}%
\pgfpathlineto{\pgfqpoint{2.865371in}{3.007327in}}%
\pgfpathlineto{\pgfqpoint{2.933172in}{2.998592in}}%
\pgfpathlineto{\pgfqpoint{3.000974in}{2.989857in}}%
\pgfpathlineto{\pgfqpoint{3.032515in}{2.985793in}}%
\pgfpathclose%
\pgfusepath{fill}%
\end{pgfscope}%
\begin{pgfscope}%
\pgfpathrectangle{\pgfqpoint{0.564660in}{0.521603in}}{\pgfqpoint{3.720000in}{3.020000in}}%
\pgfusepath{clip}%
\pgfsetbuttcap%
\pgfsetroundjoin%
\definecolor{currentfill}{rgb}{1.000000,1.000000,1.000000}%
\pgfsetfillcolor{currentfill}%
\pgfsetlinewidth{1.003750pt}%
\definecolor{currentstroke}{rgb}{0.000000,0.000000,0.000000}%
\pgfsetstrokecolor{currentstroke}%
\pgfsetdash{}{0pt}%
\pgfpathmoveto{\pgfqpoint{3.469480in}{2.911879in}}%
\pgfpathcurveto{\pgfqpoint{3.480530in}{2.911879in}}{\pgfqpoint{3.491129in}{2.916269in}}{\pgfqpoint{3.498943in}{2.924083in}}%
\pgfpathcurveto{\pgfqpoint{3.506756in}{2.931896in}}{\pgfqpoint{3.511147in}{2.942495in}}{\pgfqpoint{3.511147in}{2.953545in}}%
\pgfpathcurveto{\pgfqpoint{3.511147in}{2.964596in}}{\pgfqpoint{3.506756in}{2.975195in}}{\pgfqpoint{3.498943in}{2.983008in}}%
\pgfpathcurveto{\pgfqpoint{3.491129in}{2.990822in}}{\pgfqpoint{3.480530in}{2.995212in}}{\pgfqpoint{3.469480in}{2.995212in}}%
\pgfpathcurveto{\pgfqpoint{3.458430in}{2.995212in}}{\pgfqpoint{3.447831in}{2.990822in}}{\pgfqpoint{3.440017in}{2.983008in}}%
\pgfpathcurveto{\pgfqpoint{3.432204in}{2.975195in}}{\pgfqpoint{3.427813in}{2.964596in}}{\pgfqpoint{3.427813in}{2.953545in}}%
\pgfpathcurveto{\pgfqpoint{3.427813in}{2.942495in}}{\pgfqpoint{3.432204in}{2.931896in}}{\pgfqpoint{3.440017in}{2.924083in}}%
\pgfpathcurveto{\pgfqpoint{3.447831in}{2.916269in}}{\pgfqpoint{3.458430in}{2.911879in}}{\pgfqpoint{3.469480in}{2.911879in}}%
\pgfpathclose%
\pgfusepath{stroke,fill}%
\end{pgfscope}%
\begin{pgfscope}%
\pgfpathrectangle{\pgfqpoint{0.564660in}{0.521603in}}{\pgfqpoint{3.720000in}{3.020000in}}%
\pgfusepath{clip}%
\pgfsetbuttcap%
\pgfsetroundjoin%
\definecolor{currentfill}{rgb}{1.000000,1.000000,1.000000}%
\pgfsetfillcolor{currentfill}%
\pgfsetlinewidth{1.003750pt}%
\definecolor{currentstroke}{rgb}{0.000000,0.000000,0.000000}%
\pgfsetstrokecolor{currentstroke}%
\pgfsetdash{}{0pt}%
\pgfpathmoveto{\pgfqpoint{1.171593in}{1.744492in}}%
\pgfpathcurveto{\pgfqpoint{1.182643in}{1.744492in}}{\pgfqpoint{1.193242in}{1.748882in}}{\pgfqpoint{1.201056in}{1.756696in}}%
\pgfpathcurveto{\pgfqpoint{1.208869in}{1.764510in}}{\pgfqpoint{1.213260in}{1.775109in}}{\pgfqpoint{1.213260in}{1.786159in}}%
\pgfpathcurveto{\pgfqpoint{1.213260in}{1.797209in}}{\pgfqpoint{1.208869in}{1.807808in}}{\pgfqpoint{1.201056in}{1.815622in}}%
\pgfpathcurveto{\pgfqpoint{1.193242in}{1.823435in}}{\pgfqpoint{1.182643in}{1.827826in}}{\pgfqpoint{1.171593in}{1.827826in}}%
\pgfpathcurveto{\pgfqpoint{1.160543in}{1.827826in}}{\pgfqpoint{1.149944in}{1.823435in}}{\pgfqpoint{1.142130in}{1.815622in}}%
\pgfpathcurveto{\pgfqpoint{1.134317in}{1.807808in}}{\pgfqpoint{1.129926in}{1.797209in}}{\pgfqpoint{1.129926in}{1.786159in}}%
\pgfpathcurveto{\pgfqpoint{1.129926in}{1.775109in}}{\pgfqpoint{1.134317in}{1.764510in}}{\pgfqpoint{1.142130in}{1.756696in}}%
\pgfpathcurveto{\pgfqpoint{1.149944in}{1.748882in}}{\pgfqpoint{1.160543in}{1.744492in}}{\pgfqpoint{1.171593in}{1.744492in}}%
\pgfpathclose%
\pgfusepath{stroke,fill}%
\end{pgfscope}%
\begin{pgfscope}%
\pgfpathrectangle{\pgfqpoint{0.564660in}{0.521603in}}{\pgfqpoint{3.720000in}{3.020000in}}%
\pgfusepath{clip}%
\pgfsetbuttcap%
\pgfsetroundjoin%
\definecolor{currentfill}{rgb}{1.000000,1.000000,1.000000}%
\pgfsetfillcolor{currentfill}%
\pgfsetlinewidth{1.003750pt}%
\definecolor{currentstroke}{rgb}{0.000000,0.000000,0.000000}%
\pgfsetstrokecolor{currentstroke}%
\pgfsetdash{}{0pt}%
\pgfpathmoveto{\pgfqpoint{0.941940in}{1.749672in}}%
\pgfpathcurveto{\pgfqpoint{0.952990in}{1.749672in}}{\pgfqpoint{0.963589in}{1.754062in}}{\pgfqpoint{0.971403in}{1.761876in}}%
\pgfpathcurveto{\pgfqpoint{0.979216in}{1.769689in}}{\pgfqpoint{0.983606in}{1.780289in}}{\pgfqpoint{0.983606in}{1.791339in}}%
\pgfpathcurveto{\pgfqpoint{0.983606in}{1.802389in}}{\pgfqpoint{0.979216in}{1.812988in}}{\pgfqpoint{0.971403in}{1.820801in}}%
\pgfpathcurveto{\pgfqpoint{0.963589in}{1.828615in}}{\pgfqpoint{0.952990in}{1.833005in}}{\pgfqpoint{0.941940in}{1.833005in}}%
\pgfpathcurveto{\pgfqpoint{0.930890in}{1.833005in}}{\pgfqpoint{0.920291in}{1.828615in}}{\pgfqpoint{0.912477in}{1.820801in}}%
\pgfpathcurveto{\pgfqpoint{0.904663in}{1.812988in}}{\pgfqpoint{0.900273in}{1.802389in}}{\pgfqpoint{0.900273in}{1.791339in}}%
\pgfpathcurveto{\pgfqpoint{0.900273in}{1.780289in}}{\pgfqpoint{0.904663in}{1.769689in}}{\pgfqpoint{0.912477in}{1.761876in}}%
\pgfpathcurveto{\pgfqpoint{0.920291in}{1.754062in}}{\pgfqpoint{0.930890in}{1.749672in}}{\pgfqpoint{0.941940in}{1.749672in}}%
\pgfpathclose%
\pgfusepath{stroke,fill}%
\end{pgfscope}%
\begin{pgfscope}%
\pgfpathrectangle{\pgfqpoint{0.564660in}{0.521603in}}{\pgfqpoint{3.720000in}{3.020000in}}%
\pgfusepath{clip}%
\pgfsetbuttcap%
\pgfsetroundjoin%
\definecolor{currentfill}{rgb}{1.000000,1.000000,1.000000}%
\pgfsetfillcolor{currentfill}%
\pgfsetlinewidth{1.003750pt}%
\definecolor{currentstroke}{rgb}{0.000000,0.000000,0.000000}%
\pgfsetstrokecolor{currentstroke}%
\pgfsetdash{}{0pt}%
\pgfpathmoveto{\pgfqpoint{1.680850in}{0.972425in}}%
\pgfpathcurveto{\pgfqpoint{1.691900in}{0.972425in}}{\pgfqpoint{1.702499in}{0.976815in}}{\pgfqpoint{1.710313in}{0.984629in}}%
\pgfpathcurveto{\pgfqpoint{1.718127in}{0.992443in}}{\pgfqpoint{1.722517in}{1.003042in}}{\pgfqpoint{1.722517in}{1.014092in}}%
\pgfpathcurveto{\pgfqpoint{1.722517in}{1.025142in}}{\pgfqpoint{1.718127in}{1.035741in}}{\pgfqpoint{1.710313in}{1.043555in}}%
\pgfpathcurveto{\pgfqpoint{1.702499in}{1.051368in}}{\pgfqpoint{1.691900in}{1.055759in}}{\pgfqpoint{1.680850in}{1.055759in}}%
\pgfpathcurveto{\pgfqpoint{1.669800in}{1.055759in}}{\pgfqpoint{1.659201in}{1.051368in}}{\pgfqpoint{1.651387in}{1.043555in}}%
\pgfpathcurveto{\pgfqpoint{1.643574in}{1.035741in}}{\pgfqpoint{1.639183in}{1.025142in}}{\pgfqpoint{1.639183in}{1.014092in}}%
\pgfpathcurveto{\pgfqpoint{1.639183in}{1.003042in}}{\pgfqpoint{1.643574in}{0.992443in}}{\pgfqpoint{1.651387in}{0.984629in}}%
\pgfpathcurveto{\pgfqpoint{1.659201in}{0.976815in}}{\pgfqpoint{1.669800in}{0.972425in}}{\pgfqpoint{1.680850in}{0.972425in}}%
\pgfpathclose%
\pgfusepath{stroke,fill}%
\end{pgfscope}%
\begin{pgfscope}%
\pgfpathrectangle{\pgfqpoint{0.564660in}{0.521603in}}{\pgfqpoint{3.720000in}{3.020000in}}%
\pgfusepath{clip}%
\pgfsetbuttcap%
\pgfsetroundjoin%
\definecolor{currentfill}{rgb}{1.000000,1.000000,1.000000}%
\pgfsetfillcolor{currentfill}%
\pgfsetlinewidth{1.003750pt}%
\definecolor{currentstroke}{rgb}{0.000000,0.000000,0.000000}%
\pgfsetstrokecolor{currentstroke}%
\pgfsetdash{}{0pt}%
\pgfpathmoveto{\pgfqpoint{0.763521in}{0.982499in}}%
\pgfpathcurveto{\pgfqpoint{0.774571in}{0.982499in}}{\pgfqpoint{0.785170in}{0.986889in}}{\pgfqpoint{0.792983in}{0.994702in}}%
\pgfpathcurveto{\pgfqpoint{0.800797in}{1.002516in}}{\pgfqpoint{0.805187in}{1.013115in}}{\pgfqpoint{0.805187in}{1.024165in}}%
\pgfpathcurveto{\pgfqpoint{0.805187in}{1.035215in}}{\pgfqpoint{0.800797in}{1.045814in}}{\pgfqpoint{0.792983in}{1.053628in}}%
\pgfpathcurveto{\pgfqpoint{0.785170in}{1.061442in}}{\pgfqpoint{0.774571in}{1.065832in}}{\pgfqpoint{0.763521in}{1.065832in}}%
\pgfpathcurveto{\pgfqpoint{0.752471in}{1.065832in}}{\pgfqpoint{0.741872in}{1.061442in}}{\pgfqpoint{0.734058in}{1.053628in}}%
\pgfpathcurveto{\pgfqpoint{0.726244in}{1.045814in}}{\pgfqpoint{0.721854in}{1.035215in}}{\pgfqpoint{0.721854in}{1.024165in}}%
\pgfpathcurveto{\pgfqpoint{0.721854in}{1.013115in}}{\pgfqpoint{0.726244in}{1.002516in}}{\pgfqpoint{0.734058in}{0.994702in}}%
\pgfpathcurveto{\pgfqpoint{0.741872in}{0.986889in}}{\pgfqpoint{0.752471in}{0.982499in}}{\pgfqpoint{0.763521in}{0.982499in}}%
\pgfpathclose%
\pgfusepath{stroke,fill}%
\end{pgfscope}%
\begin{pgfscope}%
\pgfpathrectangle{\pgfqpoint{0.564660in}{0.521603in}}{\pgfqpoint{3.720000in}{3.020000in}}%
\pgfusepath{clip}%
\pgfsetbuttcap%
\pgfsetroundjoin%
\definecolor{currentfill}{rgb}{1.000000,1.000000,1.000000}%
\pgfsetfillcolor{currentfill}%
\pgfsetlinewidth{1.003750pt}%
\definecolor{currentstroke}{rgb}{0.000000,0.000000,0.000000}%
\pgfsetstrokecolor{currentstroke}%
\pgfsetdash{}{0pt}%
\pgfpathmoveto{\pgfqpoint{1.355680in}{0.737445in}}%
\pgfpathcurveto{\pgfqpoint{1.366730in}{0.737445in}}{\pgfqpoint{1.377329in}{0.741835in}}{\pgfqpoint{1.385143in}{0.749649in}}%
\pgfpathcurveto{\pgfqpoint{1.392956in}{0.757463in}}{\pgfqpoint{1.397346in}{0.768062in}}{\pgfqpoint{1.397346in}{0.779112in}}%
\pgfpathcurveto{\pgfqpoint{1.397346in}{0.790162in}}{\pgfqpoint{1.392956in}{0.800761in}}{\pgfqpoint{1.385143in}{0.808575in}}%
\pgfpathcurveto{\pgfqpoint{1.377329in}{0.816388in}}{\pgfqpoint{1.366730in}{0.820779in}}{\pgfqpoint{1.355680in}{0.820779in}}%
\pgfpathcurveto{\pgfqpoint{1.344630in}{0.820779in}}{\pgfqpoint{1.334031in}{0.816388in}}{\pgfqpoint{1.326217in}{0.808575in}}%
\pgfpathcurveto{\pgfqpoint{1.318403in}{0.800761in}}{\pgfqpoint{1.314013in}{0.790162in}}{\pgfqpoint{1.314013in}{0.779112in}}%
\pgfpathcurveto{\pgfqpoint{1.314013in}{0.768062in}}{\pgfqpoint{1.318403in}{0.757463in}}{\pgfqpoint{1.326217in}{0.749649in}}%
\pgfpathcurveto{\pgfqpoint{1.334031in}{0.741835in}}{\pgfqpoint{1.344630in}{0.737445in}}{\pgfqpoint{1.355680in}{0.737445in}}%
\pgfpathclose%
\pgfusepath{stroke,fill}%
\end{pgfscope}%
\begin{pgfscope}%
\pgfpathrectangle{\pgfqpoint{0.564660in}{0.521603in}}{\pgfqpoint{3.720000in}{3.020000in}}%
\pgfusepath{clip}%
\pgfsetbuttcap%
\pgfsetroundjoin%
\definecolor{currentfill}{rgb}{1.000000,1.000000,1.000000}%
\pgfsetfillcolor{currentfill}%
\pgfsetlinewidth{1.003750pt}%
\definecolor{currentstroke}{rgb}{0.000000,0.000000,0.000000}%
\pgfsetstrokecolor{currentstroke}%
\pgfsetdash{}{0pt}%
\pgfpathmoveto{\pgfqpoint{1.307448in}{0.654071in}}%
\pgfpathcurveto{\pgfqpoint{1.318498in}{0.654071in}}{\pgfqpoint{1.329097in}{0.658461in}}{\pgfqpoint{1.336910in}{0.666275in}}%
\pgfpathcurveto{\pgfqpoint{1.344724in}{0.674089in}}{\pgfqpoint{1.349114in}{0.684688in}}{\pgfqpoint{1.349114in}{0.695738in}}%
\pgfpathcurveto{\pgfqpoint{1.349114in}{0.706788in}}{\pgfqpoint{1.344724in}{0.717387in}}{\pgfqpoint{1.336910in}{0.725200in}}%
\pgfpathcurveto{\pgfqpoint{1.329097in}{0.733014in}}{\pgfqpoint{1.318498in}{0.737404in}}{\pgfqpoint{1.307448in}{0.737404in}}%
\pgfpathcurveto{\pgfqpoint{1.296398in}{0.737404in}}{\pgfqpoint{1.285799in}{0.733014in}}{\pgfqpoint{1.277985in}{0.725200in}}%
\pgfpathcurveto{\pgfqpoint{1.270171in}{0.717387in}}{\pgfqpoint{1.265781in}{0.706788in}}{\pgfqpoint{1.265781in}{0.695738in}}%
\pgfpathcurveto{\pgfqpoint{1.265781in}{0.684688in}}{\pgfqpoint{1.270171in}{0.674089in}}{\pgfqpoint{1.277985in}{0.666275in}}%
\pgfpathcurveto{\pgfqpoint{1.285799in}{0.658461in}}{\pgfqpoint{1.296398in}{0.654071in}}{\pgfqpoint{1.307448in}{0.654071in}}%
\pgfpathclose%
\pgfusepath{stroke,fill}%
\end{pgfscope}%
\begin{pgfscope}%
\pgfpathrectangle{\pgfqpoint{0.564660in}{0.521603in}}{\pgfqpoint{3.720000in}{3.020000in}}%
\pgfusepath{clip}%
\pgfsetbuttcap%
\pgfsetroundjoin%
\definecolor{currentfill}{rgb}{1.000000,1.000000,1.000000}%
\pgfsetfillcolor{currentfill}%
\pgfsetlinewidth{1.003750pt}%
\definecolor{currentstroke}{rgb}{0.000000,0.000000,0.000000}%
\pgfsetstrokecolor{currentstroke}%
\pgfsetdash{}{0pt}%
\pgfpathmoveto{\pgfqpoint{4.085800in}{2.313769in}}%
\pgfpathcurveto{\pgfqpoint{4.096850in}{2.313769in}}{\pgfqpoint{4.107449in}{2.318159in}}{\pgfqpoint{4.115263in}{2.325973in}}%
\pgfpathcurveto{\pgfqpoint{4.123076in}{2.333786in}}{\pgfqpoint{4.127467in}{2.344385in}}{\pgfqpoint{4.127467in}{2.355435in}}%
\pgfpathcurveto{\pgfqpoint{4.127467in}{2.366486in}}{\pgfqpoint{4.123076in}{2.377085in}}{\pgfqpoint{4.115263in}{2.384898in}}%
\pgfpathcurveto{\pgfqpoint{4.107449in}{2.392712in}}{\pgfqpoint{4.096850in}{2.397102in}}{\pgfqpoint{4.085800in}{2.397102in}}%
\pgfpathcurveto{\pgfqpoint{4.074750in}{2.397102in}}{\pgfqpoint{4.064151in}{2.392712in}}{\pgfqpoint{4.056337in}{2.384898in}}%
\pgfpathcurveto{\pgfqpoint{4.048523in}{2.377085in}}{\pgfqpoint{4.044133in}{2.366486in}}{\pgfqpoint{4.044133in}{2.355435in}}%
\pgfpathcurveto{\pgfqpoint{4.044133in}{2.344385in}}{\pgfqpoint{4.048523in}{2.333786in}}{\pgfqpoint{4.056337in}{2.325973in}}%
\pgfpathcurveto{\pgfqpoint{4.064151in}{2.318159in}}{\pgfqpoint{4.074750in}{2.313769in}}{\pgfqpoint{4.085800in}{2.313769in}}%
\pgfpathclose%
\pgfusepath{stroke,fill}%
\end{pgfscope}%
\begin{pgfscope}%
\pgfpathrectangle{\pgfqpoint{0.564660in}{0.521603in}}{\pgfqpoint{3.720000in}{3.020000in}}%
\pgfusepath{clip}%
\pgfsetbuttcap%
\pgfsetroundjoin%
\definecolor{currentfill}{rgb}{1.000000,1.000000,1.000000}%
\pgfsetfillcolor{currentfill}%
\pgfsetlinewidth{1.003750pt}%
\definecolor{currentstroke}{rgb}{0.000000,0.000000,0.000000}%
\pgfsetstrokecolor{currentstroke}%
\pgfsetdash{}{0pt}%
\pgfpathmoveto{\pgfqpoint{3.510562in}{1.703866in}}%
\pgfpathcurveto{\pgfqpoint{3.521612in}{1.703866in}}{\pgfqpoint{3.532211in}{1.708257in}}{\pgfqpoint{3.540025in}{1.716070in}}%
\pgfpathcurveto{\pgfqpoint{3.547839in}{1.723884in}}{\pgfqpoint{3.552229in}{1.734483in}}{\pgfqpoint{3.552229in}{1.745533in}}%
\pgfpathcurveto{\pgfqpoint{3.552229in}{1.756583in}}{\pgfqpoint{3.547839in}{1.767182in}}{\pgfqpoint{3.540025in}{1.774996in}}%
\pgfpathcurveto{\pgfqpoint{3.532211in}{1.782810in}}{\pgfqpoint{3.521612in}{1.787200in}}{\pgfqpoint{3.510562in}{1.787200in}}%
\pgfpathcurveto{\pgfqpoint{3.499512in}{1.787200in}}{\pgfqpoint{3.488913in}{1.782810in}}{\pgfqpoint{3.481099in}{1.774996in}}%
\pgfpathcurveto{\pgfqpoint{3.473286in}{1.767182in}}{\pgfqpoint{3.468895in}{1.756583in}}{\pgfqpoint{3.468895in}{1.745533in}}%
\pgfpathcurveto{\pgfqpoint{3.468895in}{1.734483in}}{\pgfqpoint{3.473286in}{1.723884in}}{\pgfqpoint{3.481099in}{1.716070in}}%
\pgfpathcurveto{\pgfqpoint{3.488913in}{1.708257in}}{\pgfqpoint{3.499512in}{1.703866in}}{\pgfqpoint{3.510562in}{1.703866in}}%
\pgfpathclose%
\pgfusepath{stroke,fill}%
\end{pgfscope}%
\begin{pgfscope}%
\pgfpathrectangle{\pgfqpoint{0.564660in}{0.521603in}}{\pgfqpoint{3.720000in}{3.020000in}}%
\pgfusepath{clip}%
\pgfsetbuttcap%
\pgfsetroundjoin%
\definecolor{currentfill}{rgb}{1.000000,1.000000,1.000000}%
\pgfsetfillcolor{currentfill}%
\pgfsetlinewidth{1.003750pt}%
\definecolor{currentstroke}{rgb}{0.000000,0.000000,0.000000}%
\pgfsetstrokecolor{currentstroke}%
\pgfsetdash{}{0pt}%
\pgfpathmoveto{\pgfqpoint{2.876111in}{1.338281in}}%
\pgfpathcurveto{\pgfqpoint{2.887161in}{1.338281in}}{\pgfqpoint{2.897760in}{1.342671in}}{\pgfqpoint{2.905574in}{1.350485in}}%
\pgfpathcurveto{\pgfqpoint{2.913387in}{1.358299in}}{\pgfqpoint{2.917777in}{1.368898in}}{\pgfqpoint{2.917777in}{1.379948in}}%
\pgfpathcurveto{\pgfqpoint{2.917777in}{1.390998in}}{\pgfqpoint{2.913387in}{1.401597in}}{\pgfqpoint{2.905574in}{1.409410in}}%
\pgfpathcurveto{\pgfqpoint{2.897760in}{1.417224in}}{\pgfqpoint{2.887161in}{1.421614in}}{\pgfqpoint{2.876111in}{1.421614in}}%
\pgfpathcurveto{\pgfqpoint{2.865061in}{1.421614in}}{\pgfqpoint{2.854462in}{1.417224in}}{\pgfqpoint{2.846648in}{1.409410in}}%
\pgfpathcurveto{\pgfqpoint{2.838834in}{1.401597in}}{\pgfqpoint{2.834444in}{1.390998in}}{\pgfqpoint{2.834444in}{1.379948in}}%
\pgfpathcurveto{\pgfqpoint{2.834444in}{1.368898in}}{\pgfqpoint{2.838834in}{1.358299in}}{\pgfqpoint{2.846648in}{1.350485in}}%
\pgfpathcurveto{\pgfqpoint{2.854462in}{1.342671in}}{\pgfqpoint{2.865061in}{1.338281in}}{\pgfqpoint{2.876111in}{1.338281in}}%
\pgfpathclose%
\pgfusepath{stroke,fill}%
\end{pgfscope}%
\begin{pgfscope}%
\pgfpathrectangle{\pgfqpoint{0.564660in}{0.521603in}}{\pgfqpoint{3.720000in}{3.020000in}}%
\pgfusepath{clip}%
\pgfsetbuttcap%
\pgfsetroundjoin%
\definecolor{currentfill}{rgb}{1.000000,1.000000,1.000000}%
\pgfsetfillcolor{currentfill}%
\pgfsetlinewidth{1.003750pt}%
\definecolor{currentstroke}{rgb}{0.000000,0.000000,0.000000}%
\pgfsetstrokecolor{currentstroke}%
\pgfsetdash{}{0pt}%
\pgfpathmoveto{\pgfqpoint{1.255941in}{2.308002in}}%
\pgfpathcurveto{\pgfqpoint{1.266991in}{2.308002in}}{\pgfqpoint{1.277590in}{2.312392in}}{\pgfqpoint{1.285404in}{2.320206in}}%
\pgfpathcurveto{\pgfqpoint{1.293218in}{2.328020in}}{\pgfqpoint{1.297608in}{2.338619in}}{\pgfqpoint{1.297608in}{2.349669in}}%
\pgfpathcurveto{\pgfqpoint{1.297608in}{2.360719in}}{\pgfqpoint{1.293218in}{2.371318in}}{\pgfqpoint{1.285404in}{2.379132in}}%
\pgfpathcurveto{\pgfqpoint{1.277590in}{2.386945in}}{\pgfqpoint{1.266991in}{2.391336in}}{\pgfqpoint{1.255941in}{2.391336in}}%
\pgfpathcurveto{\pgfqpoint{1.244891in}{2.391336in}}{\pgfqpoint{1.234292in}{2.386945in}}{\pgfqpoint{1.226478in}{2.379132in}}%
\pgfpathcurveto{\pgfqpoint{1.218665in}{2.371318in}}{\pgfqpoint{1.214275in}{2.360719in}}{\pgfqpoint{1.214275in}{2.349669in}}%
\pgfpathcurveto{\pgfqpoint{1.214275in}{2.338619in}}{\pgfqpoint{1.218665in}{2.328020in}}{\pgfqpoint{1.226478in}{2.320206in}}%
\pgfpathcurveto{\pgfqpoint{1.234292in}{2.312392in}}{\pgfqpoint{1.244891in}{2.308002in}}{\pgfqpoint{1.255941in}{2.308002in}}%
\pgfpathclose%
\pgfusepath{stroke,fill}%
\end{pgfscope}%
\begin{pgfscope}%
\pgfpathrectangle{\pgfqpoint{0.564660in}{0.521603in}}{\pgfqpoint{3.720000in}{3.020000in}}%
\pgfusepath{clip}%
\pgfsetbuttcap%
\pgfsetroundjoin%
\definecolor{currentfill}{rgb}{1.000000,1.000000,1.000000}%
\pgfsetfillcolor{currentfill}%
\pgfsetlinewidth{1.003750pt}%
\definecolor{currentstroke}{rgb}{0.000000,0.000000,0.000000}%
\pgfsetstrokecolor{currentstroke}%
\pgfsetdash{}{0pt}%
\pgfpathmoveto{\pgfqpoint{2.259339in}{0.939575in}}%
\pgfpathcurveto{\pgfqpoint{2.270389in}{0.939575in}}{\pgfqpoint{2.280988in}{0.943966in}}{\pgfqpoint{2.288801in}{0.951779in}}%
\pgfpathcurveto{\pgfqpoint{2.296615in}{0.959593in}}{\pgfqpoint{2.301005in}{0.970192in}}{\pgfqpoint{2.301005in}{0.981242in}}%
\pgfpathcurveto{\pgfqpoint{2.301005in}{0.992292in}}{\pgfqpoint{2.296615in}{1.002891in}}{\pgfqpoint{2.288801in}{1.010705in}}%
\pgfpathcurveto{\pgfqpoint{2.280988in}{1.018518in}}{\pgfqpoint{2.270389in}{1.022909in}}{\pgfqpoint{2.259339in}{1.022909in}}%
\pgfpathcurveto{\pgfqpoint{2.248289in}{1.022909in}}{\pgfqpoint{2.237690in}{1.018518in}}{\pgfqpoint{2.229876in}{1.010705in}}%
\pgfpathcurveto{\pgfqpoint{2.222062in}{1.002891in}}{\pgfqpoint{2.217672in}{0.992292in}}{\pgfqpoint{2.217672in}{0.981242in}}%
\pgfpathcurveto{\pgfqpoint{2.217672in}{0.970192in}}{\pgfqpoint{2.222062in}{0.959593in}}{\pgfqpoint{2.229876in}{0.951779in}}%
\pgfpathcurveto{\pgfqpoint{2.237690in}{0.943966in}}{\pgfqpoint{2.248289in}{0.939575in}}{\pgfqpoint{2.259339in}{0.939575in}}%
\pgfpathclose%
\pgfusepath{stroke,fill}%
\end{pgfscope}%
\begin{pgfscope}%
\pgfpathrectangle{\pgfqpoint{0.564660in}{0.521603in}}{\pgfqpoint{3.720000in}{3.020000in}}%
\pgfusepath{clip}%
\pgfsetbuttcap%
\pgfsetroundjoin%
\definecolor{currentfill}{rgb}{1.000000,1.000000,1.000000}%
\pgfsetfillcolor{currentfill}%
\pgfsetlinewidth{1.003750pt}%
\definecolor{currentstroke}{rgb}{0.000000,0.000000,0.000000}%
\pgfsetstrokecolor{currentstroke}%
\pgfsetdash{}{0pt}%
\pgfpathmoveto{\pgfqpoint{1.940911in}{1.013750in}}%
\pgfpathcurveto{\pgfqpoint{1.951961in}{1.013750in}}{\pgfqpoint{1.962560in}{1.018140in}}{\pgfqpoint{1.970374in}{1.025954in}}%
\pgfpathcurveto{\pgfqpoint{1.978187in}{1.033768in}}{\pgfqpoint{1.982578in}{1.044367in}}{\pgfqpoint{1.982578in}{1.055417in}}%
\pgfpathcurveto{\pgfqpoint{1.982578in}{1.066467in}}{\pgfqpoint{1.978187in}{1.077066in}}{\pgfqpoint{1.970374in}{1.084880in}}%
\pgfpathcurveto{\pgfqpoint{1.962560in}{1.092693in}}{\pgfqpoint{1.951961in}{1.097083in}}{\pgfqpoint{1.940911in}{1.097083in}}%
\pgfpathcurveto{\pgfqpoint{1.929861in}{1.097083in}}{\pgfqpoint{1.919262in}{1.092693in}}{\pgfqpoint{1.911448in}{1.084880in}}%
\pgfpathcurveto{\pgfqpoint{1.903635in}{1.077066in}}{\pgfqpoint{1.899244in}{1.066467in}}{\pgfqpoint{1.899244in}{1.055417in}}%
\pgfpathcurveto{\pgfqpoint{1.899244in}{1.044367in}}{\pgfqpoint{1.903635in}{1.033768in}}{\pgfqpoint{1.911448in}{1.025954in}}%
\pgfpathcurveto{\pgfqpoint{1.919262in}{1.018140in}}{\pgfqpoint{1.929861in}{1.013750in}}{\pgfqpoint{1.940911in}{1.013750in}}%
\pgfpathclose%
\pgfusepath{stroke,fill}%
\end{pgfscope}%
\begin{pgfscope}%
\pgfpathrectangle{\pgfqpoint{0.564660in}{0.521603in}}{\pgfqpoint{3.720000in}{3.020000in}}%
\pgfusepath{clip}%
\pgfsetbuttcap%
\pgfsetroundjoin%
\definecolor{currentfill}{rgb}{1.000000,1.000000,1.000000}%
\pgfsetfillcolor{currentfill}%
\pgfsetlinewidth{1.003750pt}%
\definecolor{currentstroke}{rgb}{0.000000,0.000000,0.000000}%
\pgfsetstrokecolor{currentstroke}%
\pgfsetdash{}{0pt}%
\pgfpathmoveto{\pgfqpoint{1.712153in}{3.325802in}}%
\pgfpathcurveto{\pgfqpoint{1.723203in}{3.325802in}}{\pgfqpoint{1.733802in}{3.330193in}}{\pgfqpoint{1.741616in}{3.338006in}}%
\pgfpathcurveto{\pgfqpoint{1.749429in}{3.345820in}}{\pgfqpoint{1.753820in}{3.356419in}}{\pgfqpoint{1.753820in}{3.367469in}}%
\pgfpathcurveto{\pgfqpoint{1.753820in}{3.378519in}}{\pgfqpoint{1.749429in}{3.389118in}}{\pgfqpoint{1.741616in}{3.396932in}}%
\pgfpathcurveto{\pgfqpoint{1.733802in}{3.404745in}}{\pgfqpoint{1.723203in}{3.409136in}}{\pgfqpoint{1.712153in}{3.409136in}}%
\pgfpathcurveto{\pgfqpoint{1.701103in}{3.409136in}}{\pgfqpoint{1.690504in}{3.404745in}}{\pgfqpoint{1.682690in}{3.396932in}}%
\pgfpathcurveto{\pgfqpoint{1.674876in}{3.389118in}}{\pgfqpoint{1.670486in}{3.378519in}}{\pgfqpoint{1.670486in}{3.367469in}}%
\pgfpathcurveto{\pgfqpoint{1.670486in}{3.356419in}}{\pgfqpoint{1.674876in}{3.345820in}}{\pgfqpoint{1.682690in}{3.338006in}}%
\pgfpathcurveto{\pgfqpoint{1.690504in}{3.330193in}}{\pgfqpoint{1.701103in}{3.325802in}}{\pgfqpoint{1.712153in}{3.325802in}}%
\pgfpathclose%
\pgfusepath{stroke,fill}%
\end{pgfscope}%
\begin{pgfscope}%
\pgfpathrectangle{\pgfqpoint{0.564660in}{0.521603in}}{\pgfqpoint{3.720000in}{3.020000in}}%
\pgfusepath{clip}%
\pgfsetbuttcap%
\pgfsetroundjoin%
\definecolor{currentfill}{rgb}{1.000000,1.000000,1.000000}%
\pgfsetfillcolor{currentfill}%
\pgfsetlinewidth{1.003750pt}%
\definecolor{currentstroke}{rgb}{0.000000,0.000000,0.000000}%
\pgfsetstrokecolor{currentstroke}%
\pgfsetdash{}{0pt}%
\pgfpathmoveto{\pgfqpoint{1.087553in}{1.563620in}}%
\pgfpathcurveto{\pgfqpoint{1.098603in}{1.563620in}}{\pgfqpoint{1.109202in}{1.568010in}}{\pgfqpoint{1.117016in}{1.575823in}}%
\pgfpathcurveto{\pgfqpoint{1.124830in}{1.583637in}}{\pgfqpoint{1.129220in}{1.594236in}}{\pgfqpoint{1.129220in}{1.605286in}}%
\pgfpathcurveto{\pgfqpoint{1.129220in}{1.616336in}}{\pgfqpoint{1.124830in}{1.626935in}}{\pgfqpoint{1.117016in}{1.634749in}}%
\pgfpathcurveto{\pgfqpoint{1.109202in}{1.642563in}}{\pgfqpoint{1.098603in}{1.646953in}}{\pgfqpoint{1.087553in}{1.646953in}}%
\pgfpathcurveto{\pgfqpoint{1.076503in}{1.646953in}}{\pgfqpoint{1.065904in}{1.642563in}}{\pgfqpoint{1.058090in}{1.634749in}}%
\pgfpathcurveto{\pgfqpoint{1.050277in}{1.626935in}}{\pgfqpoint{1.045886in}{1.616336in}}{\pgfqpoint{1.045886in}{1.605286in}}%
\pgfpathcurveto{\pgfqpoint{1.045886in}{1.594236in}}{\pgfqpoint{1.050277in}{1.583637in}}{\pgfqpoint{1.058090in}{1.575823in}}%
\pgfpathcurveto{\pgfqpoint{1.065904in}{1.568010in}}{\pgfqpoint{1.076503in}{1.563620in}}{\pgfqpoint{1.087553in}{1.563620in}}%
\pgfpathclose%
\pgfusepath{stroke,fill}%
\end{pgfscope}%
\begin{pgfscope}%
\pgfpathrectangle{\pgfqpoint{0.564660in}{0.521603in}}{\pgfqpoint{3.720000in}{3.020000in}}%
\pgfusepath{clip}%
\pgfsetbuttcap%
\pgfsetroundjoin%
\definecolor{currentfill}{rgb}{1.000000,1.000000,1.000000}%
\pgfsetfillcolor{currentfill}%
\pgfsetlinewidth{1.003750pt}%
\definecolor{currentstroke}{rgb}{0.000000,0.000000,0.000000}%
\pgfsetstrokecolor{currentstroke}%
\pgfsetdash{}{0pt}%
\pgfpathmoveto{\pgfqpoint{1.626658in}{0.731310in}}%
\pgfpathcurveto{\pgfqpoint{1.637708in}{0.731310in}}{\pgfqpoint{1.648307in}{0.735701in}}{\pgfqpoint{1.656121in}{0.743514in}}%
\pgfpathcurveto{\pgfqpoint{1.663934in}{0.751328in}}{\pgfqpoint{1.668325in}{0.761927in}}{\pgfqpoint{1.668325in}{0.772977in}}%
\pgfpathcurveto{\pgfqpoint{1.668325in}{0.784027in}}{\pgfqpoint{1.663934in}{0.794626in}}{\pgfqpoint{1.656121in}{0.802440in}}%
\pgfpathcurveto{\pgfqpoint{1.648307in}{0.810253in}}{\pgfqpoint{1.637708in}{0.814644in}}{\pgfqpoint{1.626658in}{0.814644in}}%
\pgfpathcurveto{\pgfqpoint{1.615608in}{0.814644in}}{\pgfqpoint{1.605009in}{0.810253in}}{\pgfqpoint{1.597195in}{0.802440in}}%
\pgfpathcurveto{\pgfqpoint{1.589382in}{0.794626in}}{\pgfqpoint{1.584991in}{0.784027in}}{\pgfqpoint{1.584991in}{0.772977in}}%
\pgfpathcurveto{\pgfqpoint{1.584991in}{0.761927in}}{\pgfqpoint{1.589382in}{0.751328in}}{\pgfqpoint{1.597195in}{0.743514in}}%
\pgfpathcurveto{\pgfqpoint{1.605009in}{0.735701in}}{\pgfqpoint{1.615608in}{0.731310in}}{\pgfqpoint{1.626658in}{0.731310in}}%
\pgfpathclose%
\pgfusepath{stroke,fill}%
\end{pgfscope}%
\begin{pgfscope}%
\pgfsetbuttcap%
\pgfsetroundjoin%
\definecolor{currentfill}{rgb}{0.000000,0.000000,0.000000}%
\pgfsetfillcolor{currentfill}%
\pgfsetlinewidth{0.803000pt}%
\definecolor{currentstroke}{rgb}{0.000000,0.000000,0.000000}%
\pgfsetstrokecolor{currentstroke}%
\pgfsetdash{}{0pt}%
\pgfsys@defobject{currentmarker}{\pgfqpoint{0.000000in}{-0.048611in}}{\pgfqpoint{0.000000in}{0.000000in}}{%
\pgfpathmoveto{\pgfqpoint{0.000000in}{0.000000in}}%
\pgfpathlineto{\pgfqpoint{0.000000in}{-0.048611in}}%
\pgfusepath{stroke,fill}%
}%
\begin{pgfscope}%
\pgfsys@transformshift{1.061822in}{0.521603in}%
\pgfsys@useobject{currentmarker}{}%
\end{pgfscope}%
\end{pgfscope}%
\begin{pgfscope}%
\definecolor{textcolor}{rgb}{0.000000,0.000000,0.000000}%
\pgfsetstrokecolor{textcolor}%
\pgfsetfillcolor{textcolor}%
\pgftext[x=1.061822in,y=0.424381in,,top]{\color{textcolor}\rmfamily\fontsize{10.000000}{12.000000}\selectfont \(\displaystyle 0.6\)}%
\end{pgfscope}%
\begin{pgfscope}%
\pgfsetbuttcap%
\pgfsetroundjoin%
\definecolor{currentfill}{rgb}{0.000000,0.000000,0.000000}%
\pgfsetfillcolor{currentfill}%
\pgfsetlinewidth{0.803000pt}%
\definecolor{currentstroke}{rgb}{0.000000,0.000000,0.000000}%
\pgfsetstrokecolor{currentstroke}%
\pgfsetdash{}{0pt}%
\pgfsys@defobject{currentmarker}{\pgfqpoint{0.000000in}{-0.048611in}}{\pgfqpoint{0.000000in}{0.000000in}}{%
\pgfpathmoveto{\pgfqpoint{0.000000in}{0.000000in}}%
\pgfpathlineto{\pgfqpoint{0.000000in}{-0.048611in}}%
\pgfusepath{stroke,fill}%
}%
\begin{pgfscope}%
\pgfsys@transformshift{1.572925in}{0.521603in}%
\pgfsys@useobject{currentmarker}{}%
\end{pgfscope}%
\end{pgfscope}%
\begin{pgfscope}%
\definecolor{textcolor}{rgb}{0.000000,0.000000,0.000000}%
\pgfsetstrokecolor{textcolor}%
\pgfsetfillcolor{textcolor}%
\pgftext[x=1.572925in,y=0.424381in,,top]{\color{textcolor}\rmfamily\fontsize{10.000000}{12.000000}\selectfont \(\displaystyle 0.8\)}%
\end{pgfscope}%
\begin{pgfscope}%
\pgfsetbuttcap%
\pgfsetroundjoin%
\definecolor{currentfill}{rgb}{0.000000,0.000000,0.000000}%
\pgfsetfillcolor{currentfill}%
\pgfsetlinewidth{0.803000pt}%
\definecolor{currentstroke}{rgb}{0.000000,0.000000,0.000000}%
\pgfsetstrokecolor{currentstroke}%
\pgfsetdash{}{0pt}%
\pgfsys@defobject{currentmarker}{\pgfqpoint{0.000000in}{-0.048611in}}{\pgfqpoint{0.000000in}{0.000000in}}{%
\pgfpathmoveto{\pgfqpoint{0.000000in}{0.000000in}}%
\pgfpathlineto{\pgfqpoint{0.000000in}{-0.048611in}}%
\pgfusepath{stroke,fill}%
}%
\begin{pgfscope}%
\pgfsys@transformshift{2.084027in}{0.521603in}%
\pgfsys@useobject{currentmarker}{}%
\end{pgfscope}%
\end{pgfscope}%
\begin{pgfscope}%
\definecolor{textcolor}{rgb}{0.000000,0.000000,0.000000}%
\pgfsetstrokecolor{textcolor}%
\pgfsetfillcolor{textcolor}%
\pgftext[x=2.084027in,y=0.424381in,,top]{\color{textcolor}\rmfamily\fontsize{10.000000}{12.000000}\selectfont \(\displaystyle 1.0\)}%
\end{pgfscope}%
\begin{pgfscope}%
\pgfsetbuttcap%
\pgfsetroundjoin%
\definecolor{currentfill}{rgb}{0.000000,0.000000,0.000000}%
\pgfsetfillcolor{currentfill}%
\pgfsetlinewidth{0.803000pt}%
\definecolor{currentstroke}{rgb}{0.000000,0.000000,0.000000}%
\pgfsetstrokecolor{currentstroke}%
\pgfsetdash{}{0pt}%
\pgfsys@defobject{currentmarker}{\pgfqpoint{0.000000in}{-0.048611in}}{\pgfqpoint{0.000000in}{0.000000in}}{%
\pgfpathmoveto{\pgfqpoint{0.000000in}{0.000000in}}%
\pgfpathlineto{\pgfqpoint{0.000000in}{-0.048611in}}%
\pgfusepath{stroke,fill}%
}%
\begin{pgfscope}%
\pgfsys@transformshift{2.595130in}{0.521603in}%
\pgfsys@useobject{currentmarker}{}%
\end{pgfscope}%
\end{pgfscope}%
\begin{pgfscope}%
\definecolor{textcolor}{rgb}{0.000000,0.000000,0.000000}%
\pgfsetstrokecolor{textcolor}%
\pgfsetfillcolor{textcolor}%
\pgftext[x=2.595130in,y=0.424381in,,top]{\color{textcolor}\rmfamily\fontsize{10.000000}{12.000000}\selectfont \(\displaystyle 1.2\)}%
\end{pgfscope}%
\begin{pgfscope}%
\pgfsetbuttcap%
\pgfsetroundjoin%
\definecolor{currentfill}{rgb}{0.000000,0.000000,0.000000}%
\pgfsetfillcolor{currentfill}%
\pgfsetlinewidth{0.803000pt}%
\definecolor{currentstroke}{rgb}{0.000000,0.000000,0.000000}%
\pgfsetstrokecolor{currentstroke}%
\pgfsetdash{}{0pt}%
\pgfsys@defobject{currentmarker}{\pgfqpoint{0.000000in}{-0.048611in}}{\pgfqpoint{0.000000in}{0.000000in}}{%
\pgfpathmoveto{\pgfqpoint{0.000000in}{0.000000in}}%
\pgfpathlineto{\pgfqpoint{0.000000in}{-0.048611in}}%
\pgfusepath{stroke,fill}%
}%
\begin{pgfscope}%
\pgfsys@transformshift{3.106233in}{0.521603in}%
\pgfsys@useobject{currentmarker}{}%
\end{pgfscope}%
\end{pgfscope}%
\begin{pgfscope}%
\definecolor{textcolor}{rgb}{0.000000,0.000000,0.000000}%
\pgfsetstrokecolor{textcolor}%
\pgfsetfillcolor{textcolor}%
\pgftext[x=3.106233in,y=0.424381in,,top]{\color{textcolor}\rmfamily\fontsize{10.000000}{12.000000}\selectfont \(\displaystyle 1.4\)}%
\end{pgfscope}%
\begin{pgfscope}%
\pgfsetbuttcap%
\pgfsetroundjoin%
\definecolor{currentfill}{rgb}{0.000000,0.000000,0.000000}%
\pgfsetfillcolor{currentfill}%
\pgfsetlinewidth{0.803000pt}%
\definecolor{currentstroke}{rgb}{0.000000,0.000000,0.000000}%
\pgfsetstrokecolor{currentstroke}%
\pgfsetdash{}{0pt}%
\pgfsys@defobject{currentmarker}{\pgfqpoint{0.000000in}{-0.048611in}}{\pgfqpoint{0.000000in}{0.000000in}}{%
\pgfpathmoveto{\pgfqpoint{0.000000in}{0.000000in}}%
\pgfpathlineto{\pgfqpoint{0.000000in}{-0.048611in}}%
\pgfusepath{stroke,fill}%
}%
\begin{pgfscope}%
\pgfsys@transformshift{3.617336in}{0.521603in}%
\pgfsys@useobject{currentmarker}{}%
\end{pgfscope}%
\end{pgfscope}%
\begin{pgfscope}%
\definecolor{textcolor}{rgb}{0.000000,0.000000,0.000000}%
\pgfsetstrokecolor{textcolor}%
\pgfsetfillcolor{textcolor}%
\pgftext[x=3.617336in,y=0.424381in,,top]{\color{textcolor}\rmfamily\fontsize{10.000000}{12.000000}\selectfont \(\displaystyle 1.6\)}%
\end{pgfscope}%
\begin{pgfscope}%
\pgfsetbuttcap%
\pgfsetroundjoin%
\definecolor{currentfill}{rgb}{0.000000,0.000000,0.000000}%
\pgfsetfillcolor{currentfill}%
\pgfsetlinewidth{0.803000pt}%
\definecolor{currentstroke}{rgb}{0.000000,0.000000,0.000000}%
\pgfsetstrokecolor{currentstroke}%
\pgfsetdash{}{0pt}%
\pgfsys@defobject{currentmarker}{\pgfqpoint{0.000000in}{-0.048611in}}{\pgfqpoint{0.000000in}{0.000000in}}{%
\pgfpathmoveto{\pgfqpoint{0.000000in}{0.000000in}}%
\pgfpathlineto{\pgfqpoint{0.000000in}{-0.048611in}}%
\pgfusepath{stroke,fill}%
}%
\begin{pgfscope}%
\pgfsys@transformshift{4.128439in}{0.521603in}%
\pgfsys@useobject{currentmarker}{}%
\end{pgfscope}%
\end{pgfscope}%
\begin{pgfscope}%
\definecolor{textcolor}{rgb}{0.000000,0.000000,0.000000}%
\pgfsetstrokecolor{textcolor}%
\pgfsetfillcolor{textcolor}%
\pgftext[x=4.128439in,y=0.424381in,,top]{\color{textcolor}\rmfamily\fontsize{10.000000}{12.000000}\selectfont \(\displaystyle 1.8\)}%
\end{pgfscope}%
\begin{pgfscope}%
\definecolor{textcolor}{rgb}{0.000000,0.000000,0.000000}%
\pgfsetstrokecolor{textcolor}%
\pgfsetfillcolor{textcolor}%
\pgftext[x=2.424660in,y=0.234413in,,top]{\color{textcolor}\rmfamily\fontsize{10.000000}{12.000000}\selectfont \(\displaystyle \varphi_s\) (kV)}%
\end{pgfscope}%
\begin{pgfscope}%
\pgfsetbuttcap%
\pgfsetroundjoin%
\definecolor{currentfill}{rgb}{0.000000,0.000000,0.000000}%
\pgfsetfillcolor{currentfill}%
\pgfsetlinewidth{0.803000pt}%
\definecolor{currentstroke}{rgb}{0.000000,0.000000,0.000000}%
\pgfsetstrokecolor{currentstroke}%
\pgfsetdash{}{0pt}%
\pgfsys@defobject{currentmarker}{\pgfqpoint{-0.048611in}{0.000000in}}{\pgfqpoint{0.000000in}{0.000000in}}{%
\pgfpathmoveto{\pgfqpoint{0.000000in}{0.000000in}}%
\pgfpathlineto{\pgfqpoint{-0.048611in}{0.000000in}}%
\pgfusepath{stroke,fill}%
}%
\begin{pgfscope}%
\pgfsys@transformshift{0.564660in}{0.523802in}%
\pgfsys@useobject{currentmarker}{}%
\end{pgfscope}%
\end{pgfscope}%
\begin{pgfscope}%
\definecolor{textcolor}{rgb}{0.000000,0.000000,0.000000}%
\pgfsetstrokecolor{textcolor}%
\pgfsetfillcolor{textcolor}%
\pgftext[x=0.289968in,y=0.471040in,left,base]{\color{textcolor}\rmfamily\fontsize{10.000000}{12.000000}\selectfont \(\displaystyle 0.0\)}%
\end{pgfscope}%
\begin{pgfscope}%
\pgfsetbuttcap%
\pgfsetroundjoin%
\definecolor{currentfill}{rgb}{0.000000,0.000000,0.000000}%
\pgfsetfillcolor{currentfill}%
\pgfsetlinewidth{0.803000pt}%
\definecolor{currentstroke}{rgb}{0.000000,0.000000,0.000000}%
\pgfsetstrokecolor{currentstroke}%
\pgfsetdash{}{0pt}%
\pgfsys@defobject{currentmarker}{\pgfqpoint{-0.048611in}{0.000000in}}{\pgfqpoint{0.000000in}{0.000000in}}{%
\pgfpathmoveto{\pgfqpoint{0.000000in}{0.000000in}}%
\pgfpathlineto{\pgfqpoint{-0.048611in}{0.000000in}}%
\pgfusepath{stroke,fill}%
}%
\begin{pgfscope}%
\pgfsys@transformshift{0.564660in}{1.096070in}%
\pgfsys@useobject{currentmarker}{}%
\end{pgfscope}%
\end{pgfscope}%
\begin{pgfscope}%
\definecolor{textcolor}{rgb}{0.000000,0.000000,0.000000}%
\pgfsetstrokecolor{textcolor}%
\pgfsetfillcolor{textcolor}%
\pgftext[x=0.289968in,y=1.043308in,left,base]{\color{textcolor}\rmfamily\fontsize{10.000000}{12.000000}\selectfont \(\displaystyle 0.1\)}%
\end{pgfscope}%
\begin{pgfscope}%
\pgfsetbuttcap%
\pgfsetroundjoin%
\definecolor{currentfill}{rgb}{0.000000,0.000000,0.000000}%
\pgfsetfillcolor{currentfill}%
\pgfsetlinewidth{0.803000pt}%
\definecolor{currentstroke}{rgb}{0.000000,0.000000,0.000000}%
\pgfsetstrokecolor{currentstroke}%
\pgfsetdash{}{0pt}%
\pgfsys@defobject{currentmarker}{\pgfqpoint{-0.048611in}{0.000000in}}{\pgfqpoint{0.000000in}{0.000000in}}{%
\pgfpathmoveto{\pgfqpoint{0.000000in}{0.000000in}}%
\pgfpathlineto{\pgfqpoint{-0.048611in}{0.000000in}}%
\pgfusepath{stroke,fill}%
}%
\begin{pgfscope}%
\pgfsys@transformshift{0.564660in}{1.668338in}%
\pgfsys@useobject{currentmarker}{}%
\end{pgfscope}%
\end{pgfscope}%
\begin{pgfscope}%
\definecolor{textcolor}{rgb}{0.000000,0.000000,0.000000}%
\pgfsetstrokecolor{textcolor}%
\pgfsetfillcolor{textcolor}%
\pgftext[x=0.289968in,y=1.615576in,left,base]{\color{textcolor}\rmfamily\fontsize{10.000000}{12.000000}\selectfont \(\displaystyle 0.2\)}%
\end{pgfscope}%
\begin{pgfscope}%
\pgfsetbuttcap%
\pgfsetroundjoin%
\definecolor{currentfill}{rgb}{0.000000,0.000000,0.000000}%
\pgfsetfillcolor{currentfill}%
\pgfsetlinewidth{0.803000pt}%
\definecolor{currentstroke}{rgb}{0.000000,0.000000,0.000000}%
\pgfsetstrokecolor{currentstroke}%
\pgfsetdash{}{0pt}%
\pgfsys@defobject{currentmarker}{\pgfqpoint{-0.048611in}{0.000000in}}{\pgfqpoint{0.000000in}{0.000000in}}{%
\pgfpathmoveto{\pgfqpoint{0.000000in}{0.000000in}}%
\pgfpathlineto{\pgfqpoint{-0.048611in}{0.000000in}}%
\pgfusepath{stroke,fill}%
}%
\begin{pgfscope}%
\pgfsys@transformshift{0.564660in}{2.240606in}%
\pgfsys@useobject{currentmarker}{}%
\end{pgfscope}%
\end{pgfscope}%
\begin{pgfscope}%
\definecolor{textcolor}{rgb}{0.000000,0.000000,0.000000}%
\pgfsetstrokecolor{textcolor}%
\pgfsetfillcolor{textcolor}%
\pgftext[x=0.289968in,y=2.187844in,left,base]{\color{textcolor}\rmfamily\fontsize{10.000000}{12.000000}\selectfont \(\displaystyle 0.3\)}%
\end{pgfscope}%
\begin{pgfscope}%
\pgfsetbuttcap%
\pgfsetroundjoin%
\definecolor{currentfill}{rgb}{0.000000,0.000000,0.000000}%
\pgfsetfillcolor{currentfill}%
\pgfsetlinewidth{0.803000pt}%
\definecolor{currentstroke}{rgb}{0.000000,0.000000,0.000000}%
\pgfsetstrokecolor{currentstroke}%
\pgfsetdash{}{0pt}%
\pgfsys@defobject{currentmarker}{\pgfqpoint{-0.048611in}{0.000000in}}{\pgfqpoint{0.000000in}{0.000000in}}{%
\pgfpathmoveto{\pgfqpoint{0.000000in}{0.000000in}}%
\pgfpathlineto{\pgfqpoint{-0.048611in}{0.000000in}}%
\pgfusepath{stroke,fill}%
}%
\begin{pgfscope}%
\pgfsys@transformshift{0.564660in}{2.812874in}%
\pgfsys@useobject{currentmarker}{}%
\end{pgfscope}%
\end{pgfscope}%
\begin{pgfscope}%
\definecolor{textcolor}{rgb}{0.000000,0.000000,0.000000}%
\pgfsetstrokecolor{textcolor}%
\pgfsetfillcolor{textcolor}%
\pgftext[x=0.289968in,y=2.760112in,left,base]{\color{textcolor}\rmfamily\fontsize{10.000000}{12.000000}\selectfont \(\displaystyle 0.4\)}%
\end{pgfscope}%
\begin{pgfscope}%
\pgfsetbuttcap%
\pgfsetroundjoin%
\definecolor{currentfill}{rgb}{0.000000,0.000000,0.000000}%
\pgfsetfillcolor{currentfill}%
\pgfsetlinewidth{0.803000pt}%
\definecolor{currentstroke}{rgb}{0.000000,0.000000,0.000000}%
\pgfsetstrokecolor{currentstroke}%
\pgfsetdash{}{0pt}%
\pgfsys@defobject{currentmarker}{\pgfqpoint{-0.048611in}{0.000000in}}{\pgfqpoint{0.000000in}{0.000000in}}{%
\pgfpathmoveto{\pgfqpoint{0.000000in}{0.000000in}}%
\pgfpathlineto{\pgfqpoint{-0.048611in}{0.000000in}}%
\pgfusepath{stroke,fill}%
}%
\begin{pgfscope}%
\pgfsys@transformshift{0.564660in}{3.385142in}%
\pgfsys@useobject{currentmarker}{}%
\end{pgfscope}%
\end{pgfscope}%
\begin{pgfscope}%
\definecolor{textcolor}{rgb}{0.000000,0.000000,0.000000}%
\pgfsetstrokecolor{textcolor}%
\pgfsetfillcolor{textcolor}%
\pgftext[x=0.289968in,y=3.332380in,left,base]{\color{textcolor}\rmfamily\fontsize{10.000000}{12.000000}\selectfont \(\displaystyle 0.5\)}%
\end{pgfscope}%
\begin{pgfscope}%
\definecolor{textcolor}{rgb}{0.000000,0.000000,0.000000}%
\pgfsetstrokecolor{textcolor}%
\pgfsetfillcolor{textcolor}%
\pgftext[x=0.234413in,y=2.031603in,,bottom,rotate=90.000000]{\color{textcolor}\rmfamily\fontsize{10.000000}{12.000000}\selectfont \(\displaystyle V_d\) (mL)}%
\end{pgfscope}%
\begin{pgfscope}%
\pgfsetrectcap%
\pgfsetmiterjoin%
\pgfsetlinewidth{0.803000pt}%
\definecolor{currentstroke}{rgb}{0.501961,0.501961,0.501961}%
\pgfsetstrokecolor{currentstroke}%
\pgfsetdash{}{0pt}%
\pgfpathmoveto{\pgfqpoint{0.564660in}{0.521603in}}%
\pgfpathlineto{\pgfqpoint{0.564660in}{3.541603in}}%
\pgfusepath{stroke}%
\end{pgfscope}%
\begin{pgfscope}%
\pgfsetrectcap%
\pgfsetmiterjoin%
\pgfsetlinewidth{0.803000pt}%
\definecolor{currentstroke}{rgb}{0.501961,0.501961,0.501961}%
\pgfsetstrokecolor{currentstroke}%
\pgfsetdash{}{0pt}%
\pgfpathmoveto{\pgfqpoint{4.284660in}{0.521603in}}%
\pgfpathlineto{\pgfqpoint{4.284660in}{3.541603in}}%
\pgfusepath{stroke}%
\end{pgfscope}%
\begin{pgfscope}%
\pgfsetrectcap%
\pgfsetmiterjoin%
\pgfsetlinewidth{0.803000pt}%
\definecolor{currentstroke}{rgb}{0.501961,0.501961,0.501961}%
\pgfsetstrokecolor{currentstroke}%
\pgfsetdash{}{0pt}%
\pgfpathmoveto{\pgfqpoint{0.564660in}{0.521603in}}%
\pgfpathlineto{\pgfqpoint{4.284660in}{0.521603in}}%
\pgfusepath{stroke}%
\end{pgfscope}%
\begin{pgfscope}%
\pgfsetrectcap%
\pgfsetmiterjoin%
\pgfsetlinewidth{0.803000pt}%
\definecolor{currentstroke}{rgb}{0.501961,0.501961,0.501961}%
\pgfsetstrokecolor{currentstroke}%
\pgfsetdash{}{0pt}%
\pgfpathmoveto{\pgfqpoint{0.564660in}{3.541603in}}%
\pgfpathlineto{\pgfqpoint{4.284660in}{3.541603in}}%
\pgfusepath{stroke}%
\end{pgfscope}%
\begin{pgfscope}%
\pgfpathrectangle{\pgfqpoint{4.517160in}{0.521603in}}{\pgfqpoint{0.151000in}{3.020000in}}%
\pgfusepath{clip}%
\pgfsetbuttcap%
\pgfsetmiterjoin%
\definecolor{currentfill}{rgb}{1.000000,1.000000,1.000000}%
\pgfsetfillcolor{currentfill}%
\pgfsetlinewidth{0.010037pt}%
\definecolor{currentstroke}{rgb}{1.000000,1.000000,1.000000}%
\pgfsetstrokecolor{currentstroke}%
\pgfsetdash{}{0pt}%
\pgfpathmoveto{\pgfqpoint{4.517160in}{0.521603in}}%
\pgfpathlineto{\pgfqpoint{4.517160in}{0.953032in}}%
\pgfpathlineto{\pgfqpoint{4.517160in}{3.110175in}}%
\pgfpathlineto{\pgfqpoint{4.517160in}{3.541603in}}%
\pgfpathlineto{\pgfqpoint{4.668160in}{3.541603in}}%
\pgfpathlineto{\pgfqpoint{4.668160in}{3.110175in}}%
\pgfpathlineto{\pgfqpoint{4.668160in}{0.953032in}}%
\pgfpathlineto{\pgfqpoint{4.668160in}{0.521603in}}%
\pgfpathclose%
\pgfusepath{stroke,fill}%
\end{pgfscope}%
\begin{pgfscope}%
\pgfpathrectangle{\pgfqpoint{4.517160in}{0.521603in}}{\pgfqpoint{0.151000in}{3.020000in}}%
\pgfusepath{clip}%
\pgfsetbuttcap%
\pgfsetroundjoin%
\definecolor{currentfill}{rgb}{0.061765,0.061765,0.085934}%
\pgfsetfillcolor{currentfill}%
\pgfsetlinewidth{0.000000pt}%
\definecolor{currentstroke}{rgb}{0.000000,0.000000,0.000000}%
\pgfsetstrokecolor{currentstroke}%
\pgfsetdash{}{0pt}%
\pgfpathmoveto{\pgfqpoint{4.517160in}{0.521603in}}%
\pgfpathlineto{\pgfqpoint{4.668160in}{0.521603in}}%
\pgfpathlineto{\pgfqpoint{4.668160in}{0.953032in}}%
\pgfpathlineto{\pgfqpoint{4.517160in}{0.953032in}}%
\pgfpathlineto{\pgfqpoint{4.517160in}{0.521603in}}%
\pgfusepath{fill}%
\end{pgfscope}%
\begin{pgfscope}%
\pgfpathrectangle{\pgfqpoint{4.517160in}{0.521603in}}{\pgfqpoint{0.151000in}{3.020000in}}%
\pgfusepath{clip}%
\pgfsetbuttcap%
\pgfsetroundjoin%
\definecolor{currentfill}{rgb}{0.185294,0.185294,0.257801}%
\pgfsetfillcolor{currentfill}%
\pgfsetlinewidth{0.000000pt}%
\definecolor{currentstroke}{rgb}{0.000000,0.000000,0.000000}%
\pgfsetstrokecolor{currentstroke}%
\pgfsetdash{}{0pt}%
\pgfpathmoveto{\pgfqpoint{4.517160in}{0.953032in}}%
\pgfpathlineto{\pgfqpoint{4.668160in}{0.953032in}}%
\pgfpathlineto{\pgfqpoint{4.668160in}{1.384460in}}%
\pgfpathlineto{\pgfqpoint{4.517160in}{1.384460in}}%
\pgfpathlineto{\pgfqpoint{4.517160in}{0.953032in}}%
\pgfusepath{fill}%
\end{pgfscope}%
\begin{pgfscope}%
\pgfpathrectangle{\pgfqpoint{4.517160in}{0.521603in}}{\pgfqpoint{0.151000in}{3.020000in}}%
\pgfusepath{clip}%
\pgfsetbuttcap%
\pgfsetroundjoin%
\definecolor{currentfill}{rgb}{0.312255,0.312255,0.434442}%
\pgfsetfillcolor{currentfill}%
\pgfsetlinewidth{0.000000pt}%
\definecolor{currentstroke}{rgb}{0.000000,0.000000,0.000000}%
\pgfsetstrokecolor{currentstroke}%
\pgfsetdash{}{0pt}%
\pgfpathmoveto{\pgfqpoint{4.517160in}{1.384460in}}%
\pgfpathlineto{\pgfqpoint{4.668160in}{1.384460in}}%
\pgfpathlineto{\pgfqpoint{4.668160in}{1.815889in}}%
\pgfpathlineto{\pgfqpoint{4.517160in}{1.815889in}}%
\pgfpathlineto{\pgfqpoint{4.517160in}{1.384460in}}%
\pgfusepath{fill}%
\end{pgfscope}%
\begin{pgfscope}%
\pgfpathrectangle{\pgfqpoint{4.517160in}{0.521603in}}{\pgfqpoint{0.151000in}{3.020000in}}%
\pgfusepath{clip}%
\pgfsetbuttcap%
\pgfsetroundjoin%
\definecolor{currentfill}{rgb}{0.439216,0.484130,0.564216}%
\pgfsetfillcolor{currentfill}%
\pgfsetlinewidth{0.000000pt}%
\definecolor{currentstroke}{rgb}{0.000000,0.000000,0.000000}%
\pgfsetstrokecolor{currentstroke}%
\pgfsetdash{}{0pt}%
\pgfpathmoveto{\pgfqpoint{4.517160in}{1.815889in}}%
\pgfpathlineto{\pgfqpoint{4.668160in}{1.815889in}}%
\pgfpathlineto{\pgfqpoint{4.668160in}{2.247318in}}%
\pgfpathlineto{\pgfqpoint{4.517160in}{2.247318in}}%
\pgfpathlineto{\pgfqpoint{4.517160in}{1.815889in}}%
\pgfusepath{fill}%
\end{pgfscope}%
\begin{pgfscope}%
\pgfpathrectangle{\pgfqpoint{4.517160in}{0.521603in}}{\pgfqpoint{0.151000in}{3.020000in}}%
\pgfusepath{clip}%
\pgfsetbuttcap%
\pgfsetroundjoin%
\definecolor{currentfill}{rgb}{0.562745,0.653983,0.687745}%
\pgfsetfillcolor{currentfill}%
\pgfsetlinewidth{0.000000pt}%
\definecolor{currentstroke}{rgb}{0.000000,0.000000,0.000000}%
\pgfsetstrokecolor{currentstroke}%
\pgfsetdash{}{0pt}%
\pgfpathmoveto{\pgfqpoint{4.517160in}{2.247318in}}%
\pgfpathlineto{\pgfqpoint{4.668160in}{2.247318in}}%
\pgfpathlineto{\pgfqpoint{4.668160in}{2.678746in}}%
\pgfpathlineto{\pgfqpoint{4.517160in}{2.678746in}}%
\pgfpathlineto{\pgfqpoint{4.517160in}{2.247318in}}%
\pgfusepath{fill}%
\end{pgfscope}%
\begin{pgfscope}%
\pgfpathrectangle{\pgfqpoint{4.517160in}{0.521603in}}{\pgfqpoint{0.151000in}{3.020000in}}%
\pgfusepath{clip}%
\pgfsetbuttcap%
\pgfsetroundjoin%
\definecolor{currentfill}{rgb}{0.710478,0.814706,0.814706}%
\pgfsetfillcolor{currentfill}%
\pgfsetlinewidth{0.000000pt}%
\definecolor{currentstroke}{rgb}{0.000000,0.000000,0.000000}%
\pgfsetstrokecolor{currentstroke}%
\pgfsetdash{}{0pt}%
\pgfpathmoveto{\pgfqpoint{4.517160in}{2.678746in}}%
\pgfpathlineto{\pgfqpoint{4.668160in}{2.678746in}}%
\pgfpathlineto{\pgfqpoint{4.668160in}{3.110175in}}%
\pgfpathlineto{\pgfqpoint{4.517160in}{3.110175in}}%
\pgfpathlineto{\pgfqpoint{4.517160in}{2.678746in}}%
\pgfusepath{fill}%
\end{pgfscope}%
\begin{pgfscope}%
\pgfpathrectangle{\pgfqpoint{4.517160in}{0.521603in}}{\pgfqpoint{0.151000in}{3.020000in}}%
\pgfusepath{clip}%
\pgfsetbuttcap%
\pgfsetroundjoin%
\definecolor{currentfill}{rgb}{0.903493,0.938235,0.938235}%
\pgfsetfillcolor{currentfill}%
\pgfsetlinewidth{0.000000pt}%
\definecolor{currentstroke}{rgb}{0.000000,0.000000,0.000000}%
\pgfsetstrokecolor{currentstroke}%
\pgfsetdash{}{0pt}%
\pgfpathmoveto{\pgfqpoint{4.517160in}{3.110175in}}%
\pgfpathlineto{\pgfqpoint{4.668160in}{3.110175in}}%
\pgfpathlineto{\pgfqpoint{4.668160in}{3.541603in}}%
\pgfpathlineto{\pgfqpoint{4.517160in}{3.541603in}}%
\pgfpathlineto{\pgfqpoint{4.517160in}{3.110175in}}%
\pgfusepath{fill}%
\end{pgfscope}%
\begin{pgfscope}%
\pgfsetbuttcap%
\pgfsetroundjoin%
\definecolor{currentfill}{rgb}{0.000000,0.000000,0.000000}%
\pgfsetfillcolor{currentfill}%
\pgfsetlinewidth{0.803000pt}%
\definecolor{currentstroke}{rgb}{0.000000,0.000000,0.000000}%
\pgfsetstrokecolor{currentstroke}%
\pgfsetdash{}{0pt}%
\pgfsys@defobject{currentmarker}{\pgfqpoint{0.000000in}{0.000000in}}{\pgfqpoint{0.048611in}{0.000000in}}{%
\pgfpathmoveto{\pgfqpoint{0.000000in}{0.000000in}}%
\pgfpathlineto{\pgfqpoint{0.048611in}{0.000000in}}%
\pgfusepath{stroke,fill}%
}%
\begin{pgfscope}%
\pgfsys@transformshift{4.668160in}{0.521603in}%
\pgfsys@useobject{currentmarker}{}%
\end{pgfscope}%
\end{pgfscope}%
\begin{pgfscope}%
\definecolor{textcolor}{rgb}{0.000000,0.000000,0.000000}%
\pgfsetstrokecolor{textcolor}%
\pgfsetfillcolor{textcolor}%
\pgftext[x=4.765383in,y=0.468842in,left,base]{\color{textcolor}\rmfamily\fontsize{10.000000}{12.000000}\selectfont \(\displaystyle 0.0\)}%
\end{pgfscope}%
\begin{pgfscope}%
\pgfsetbuttcap%
\pgfsetroundjoin%
\definecolor{currentfill}{rgb}{0.000000,0.000000,0.000000}%
\pgfsetfillcolor{currentfill}%
\pgfsetlinewidth{0.803000pt}%
\definecolor{currentstroke}{rgb}{0.000000,0.000000,0.000000}%
\pgfsetstrokecolor{currentstroke}%
\pgfsetdash{}{0pt}%
\pgfsys@defobject{currentmarker}{\pgfqpoint{0.000000in}{0.000000in}}{\pgfqpoint{0.048611in}{0.000000in}}{%
\pgfpathmoveto{\pgfqpoint{0.000000in}{0.000000in}}%
\pgfpathlineto{\pgfqpoint{0.048611in}{0.000000in}}%
\pgfusepath{stroke,fill}%
}%
\begin{pgfscope}%
\pgfsys@transformshift{4.668160in}{0.953032in}%
\pgfsys@useobject{currentmarker}{}%
\end{pgfscope}%
\end{pgfscope}%
\begin{pgfscope}%
\definecolor{textcolor}{rgb}{0.000000,0.000000,0.000000}%
\pgfsetstrokecolor{textcolor}%
\pgfsetfillcolor{textcolor}%
\pgftext[x=4.765383in,y=0.900270in,left,base]{\color{textcolor}\rmfamily\fontsize{10.000000}{12.000000}\selectfont \(\displaystyle 0.2\)}%
\end{pgfscope}%
\begin{pgfscope}%
\pgfsetbuttcap%
\pgfsetroundjoin%
\definecolor{currentfill}{rgb}{0.000000,0.000000,0.000000}%
\pgfsetfillcolor{currentfill}%
\pgfsetlinewidth{0.803000pt}%
\definecolor{currentstroke}{rgb}{0.000000,0.000000,0.000000}%
\pgfsetstrokecolor{currentstroke}%
\pgfsetdash{}{0pt}%
\pgfsys@defobject{currentmarker}{\pgfqpoint{0.000000in}{0.000000in}}{\pgfqpoint{0.048611in}{0.000000in}}{%
\pgfpathmoveto{\pgfqpoint{0.000000in}{0.000000in}}%
\pgfpathlineto{\pgfqpoint{0.048611in}{0.000000in}}%
\pgfusepath{stroke,fill}%
}%
\begin{pgfscope}%
\pgfsys@transformshift{4.668160in}{1.384460in}%
\pgfsys@useobject{currentmarker}{}%
\end{pgfscope}%
\end{pgfscope}%
\begin{pgfscope}%
\definecolor{textcolor}{rgb}{0.000000,0.000000,0.000000}%
\pgfsetstrokecolor{textcolor}%
\pgfsetfillcolor{textcolor}%
\pgftext[x=4.765383in,y=1.331699in,left,base]{\color{textcolor}\rmfamily\fontsize{10.000000}{12.000000}\selectfont \(\displaystyle 0.4\)}%
\end{pgfscope}%
\begin{pgfscope}%
\pgfsetbuttcap%
\pgfsetroundjoin%
\definecolor{currentfill}{rgb}{0.000000,0.000000,0.000000}%
\pgfsetfillcolor{currentfill}%
\pgfsetlinewidth{0.803000pt}%
\definecolor{currentstroke}{rgb}{0.000000,0.000000,0.000000}%
\pgfsetstrokecolor{currentstroke}%
\pgfsetdash{}{0pt}%
\pgfsys@defobject{currentmarker}{\pgfqpoint{0.000000in}{0.000000in}}{\pgfqpoint{0.048611in}{0.000000in}}{%
\pgfpathmoveto{\pgfqpoint{0.000000in}{0.000000in}}%
\pgfpathlineto{\pgfqpoint{0.048611in}{0.000000in}}%
\pgfusepath{stroke,fill}%
}%
\begin{pgfscope}%
\pgfsys@transformshift{4.668160in}{1.815889in}%
\pgfsys@useobject{currentmarker}{}%
\end{pgfscope}%
\end{pgfscope}%
\begin{pgfscope}%
\definecolor{textcolor}{rgb}{0.000000,0.000000,0.000000}%
\pgfsetstrokecolor{textcolor}%
\pgfsetfillcolor{textcolor}%
\pgftext[x=4.765383in,y=1.763128in,left,base]{\color{textcolor}\rmfamily\fontsize{10.000000}{12.000000}\selectfont \(\displaystyle 0.6\)}%
\end{pgfscope}%
\begin{pgfscope}%
\pgfsetbuttcap%
\pgfsetroundjoin%
\definecolor{currentfill}{rgb}{0.000000,0.000000,0.000000}%
\pgfsetfillcolor{currentfill}%
\pgfsetlinewidth{0.803000pt}%
\definecolor{currentstroke}{rgb}{0.000000,0.000000,0.000000}%
\pgfsetstrokecolor{currentstroke}%
\pgfsetdash{}{0pt}%
\pgfsys@defobject{currentmarker}{\pgfqpoint{0.000000in}{0.000000in}}{\pgfqpoint{0.048611in}{0.000000in}}{%
\pgfpathmoveto{\pgfqpoint{0.000000in}{0.000000in}}%
\pgfpathlineto{\pgfqpoint{0.048611in}{0.000000in}}%
\pgfusepath{stroke,fill}%
}%
\begin{pgfscope}%
\pgfsys@transformshift{4.668160in}{2.247318in}%
\pgfsys@useobject{currentmarker}{}%
\end{pgfscope}%
\end{pgfscope}%
\begin{pgfscope}%
\definecolor{textcolor}{rgb}{0.000000,0.000000,0.000000}%
\pgfsetstrokecolor{textcolor}%
\pgfsetfillcolor{textcolor}%
\pgftext[x=4.765383in,y=2.194556in,left,base]{\color{textcolor}\rmfamily\fontsize{10.000000}{12.000000}\selectfont \(\displaystyle 0.8\)}%
\end{pgfscope}%
\begin{pgfscope}%
\pgfsetbuttcap%
\pgfsetroundjoin%
\definecolor{currentfill}{rgb}{0.000000,0.000000,0.000000}%
\pgfsetfillcolor{currentfill}%
\pgfsetlinewidth{0.803000pt}%
\definecolor{currentstroke}{rgb}{0.000000,0.000000,0.000000}%
\pgfsetstrokecolor{currentstroke}%
\pgfsetdash{}{0pt}%
\pgfsys@defobject{currentmarker}{\pgfqpoint{0.000000in}{0.000000in}}{\pgfqpoint{0.048611in}{0.000000in}}{%
\pgfpathmoveto{\pgfqpoint{0.000000in}{0.000000in}}%
\pgfpathlineto{\pgfqpoint{0.048611in}{0.000000in}}%
\pgfusepath{stroke,fill}%
}%
\begin{pgfscope}%
\pgfsys@transformshift{4.668160in}{2.678746in}%
\pgfsys@useobject{currentmarker}{}%
\end{pgfscope}%
\end{pgfscope}%
\begin{pgfscope}%
\definecolor{textcolor}{rgb}{0.000000,0.000000,0.000000}%
\pgfsetstrokecolor{textcolor}%
\pgfsetfillcolor{textcolor}%
\pgftext[x=4.765383in,y=2.625985in,left,base]{\color{textcolor}\rmfamily\fontsize{10.000000}{12.000000}\selectfont \(\displaystyle 1.0\)}%
\end{pgfscope}%
\begin{pgfscope}%
\pgfsetbuttcap%
\pgfsetroundjoin%
\definecolor{currentfill}{rgb}{0.000000,0.000000,0.000000}%
\pgfsetfillcolor{currentfill}%
\pgfsetlinewidth{0.803000pt}%
\definecolor{currentstroke}{rgb}{0.000000,0.000000,0.000000}%
\pgfsetstrokecolor{currentstroke}%
\pgfsetdash{}{0pt}%
\pgfsys@defobject{currentmarker}{\pgfqpoint{0.000000in}{0.000000in}}{\pgfqpoint{0.048611in}{0.000000in}}{%
\pgfpathmoveto{\pgfqpoint{0.000000in}{0.000000in}}%
\pgfpathlineto{\pgfqpoint{0.048611in}{0.000000in}}%
\pgfusepath{stroke,fill}%
}%
\begin{pgfscope}%
\pgfsys@transformshift{4.668160in}{3.110175in}%
\pgfsys@useobject{currentmarker}{}%
\end{pgfscope}%
\end{pgfscope}%
\begin{pgfscope}%
\definecolor{textcolor}{rgb}{0.000000,0.000000,0.000000}%
\pgfsetstrokecolor{textcolor}%
\pgfsetfillcolor{textcolor}%
\pgftext[x=4.765383in,y=3.057413in,left,base]{\color{textcolor}\rmfamily\fontsize{10.000000}{12.000000}\selectfont \(\displaystyle 1.2\)}%
\end{pgfscope}%
\begin{pgfscope}%
\pgfsetbuttcap%
\pgfsetroundjoin%
\definecolor{currentfill}{rgb}{0.000000,0.000000,0.000000}%
\pgfsetfillcolor{currentfill}%
\pgfsetlinewidth{0.803000pt}%
\definecolor{currentstroke}{rgb}{0.000000,0.000000,0.000000}%
\pgfsetstrokecolor{currentstroke}%
\pgfsetdash{}{0pt}%
\pgfsys@defobject{currentmarker}{\pgfqpoint{0.000000in}{0.000000in}}{\pgfqpoint{0.048611in}{0.000000in}}{%
\pgfpathmoveto{\pgfqpoint{0.000000in}{0.000000in}}%
\pgfpathlineto{\pgfqpoint{0.048611in}{0.000000in}}%
\pgfusepath{stroke,fill}%
}%
\begin{pgfscope}%
\pgfsys@transformshift{4.668160in}{3.541603in}%
\pgfsys@useobject{currentmarker}{}%
\end{pgfscope}%
\end{pgfscope}%
\begin{pgfscope}%
\definecolor{textcolor}{rgb}{0.000000,0.000000,0.000000}%
\pgfsetstrokecolor{textcolor}%
\pgfsetfillcolor{textcolor}%
\pgftext[x=4.765383in,y=3.488842in,left,base]{\color{textcolor}\rmfamily\fontsize{10.000000}{12.000000}\selectfont \(\displaystyle 1.4\)}%
\end{pgfscope}%
\begin{pgfscope}%
\definecolor{textcolor}{rgb}{0.000000,0.000000,0.000000}%
\pgfsetstrokecolor{textcolor}%
\pgfsetfillcolor{textcolor}%
\pgftext[x=4.998408in,y=2.031603in,,top,rotate=90.000000]{\color{textcolor}\rmfamily\fontsize{10.000000}{12.000000}\selectfont \(\displaystyle q\) (C)}%
\end{pgfscope}%
\begin{pgfscope}%
\definecolor{textcolor}{rgb}{0.000000,0.000000,0.000000}%
\pgfsetstrokecolor{textcolor}%
\pgfsetfillcolor{textcolor}%
\pgftext[x=4.668160in,y=3.583270in,right,base]{\color{textcolor}\rmfamily\fontsize{10.000000}{12.000000}\selectfont \(\displaystyle \times10^{-9}\)}%
\end{pgfscope}%
\begin{pgfscope}%
\pgfsetbuttcap%
\pgfsetmiterjoin%
\pgfsetlinewidth{0.803000pt}%
\definecolor{currentstroke}{rgb}{0.501961,0.501961,0.501961}%
\pgfsetstrokecolor{currentstroke}%
\pgfsetdash{}{0pt}%
\pgfpathmoveto{\pgfqpoint{4.517160in}{0.521603in}}%
\pgfpathlineto{\pgfqpoint{4.517160in}{0.953032in}}%
\pgfpathlineto{\pgfqpoint{4.517160in}{3.110175in}}%
\pgfpathlineto{\pgfqpoint{4.517160in}{3.541603in}}%
\pgfpathlineto{\pgfqpoint{4.668160in}{3.541603in}}%
\pgfpathlineto{\pgfqpoint{4.668160in}{3.110175in}}%
\pgfpathlineto{\pgfqpoint{4.668160in}{0.953032in}}%
\pgfpathlineto{\pgfqpoint{4.668160in}{0.521603in}}%
\pgfpathclose%
\pgfusepath{stroke}%
\end{pgfscope}%
\end{pgfpicture}%
\makeatother%
\endgroup%

    \caption{A simple EMA plot.\label{fig:charge}}
\end{figure}

A two-ways T-test comparison of charge distributions between the droplet bounce experiment and a corollary experiment with zero electric field at the time of droplet deposition on the superhydrophobic surface suggests that the droplet charge is induced by the electric field, rather than through contact charging on the PTFE layer ($t = 5.11, p = 0.0002$). The T-test informs us that the charge distribution  are about 5 times more different from each other as they are within each other, and there is a 0.02$\%$ probability that this result happened by chance. This corollary experiment is documented in Appendix \ref{sec.drop_charge}.

The model $q \sim kAE_0$ is incidentally very similar to the classical solution for the surface charge density of a half-spherical conductor with a field induced dipole \cite{david_j._griffiths_introduction_1999}
\begin{eqnarray*}
q &=& 3 \epsilon_0 E_0 \int_A \cos \theta dA \\
&=& 3 \pi^{1/3} 6 \left(6 V_d \right)^{2/3} \epsilon_0 E_0 \int^{4 \pi/2}_{\pi / 2} \cos \theta d\theta \\
&=& k E_0 V_d^{2/3}
\end{eqnarray*}
with $k \approx 1.3 \times 10^{-10}$. This is also of a similar form to the charge found by Takamatsu and coauthors for droplets falling from a grounded nozzle in an external electric field \cite{takamatsu_theoretical_1981}
\[q = 4 \pi \epsilon_0 \beta E_0 R_d^2 \]
with $\beta \approx 2.63$.

\section{Scale Quantities}
The dielectrophoretic force plays a very small role when droplets have net charge in a DC field; the condition to neglect the DEP force was satisfied for all experiments in the dataset. Dimensional droplet apoapses scale closely with $\mathbb{E}\mbox{u}$ as seen in Figure \ref{fig:series_s_eu}. The relative magnitudes of the simulated forces felt by the droplets is shown in Figure \ref{fig:forces}. Forces acting on the drops vary in magnitude between $\mathcal{O}(10^{-6})-\mathcal{O}(10^{-4})$ N. We see that, of the drops in the experimental dataset only the two with the largest $\mathbb{E}\mbox{u}$, $\mathbb{E}\mbox{u} \sim \mathcal{O}(1)$ could appropriately be said to be in the inertial electro-viscous regime. In all other cases image forces are much stronger than drag. For these drops $\mathbb{E}\mbox{u} \gg 1/8 \pi$, and are likely on escape trajectories. The image forces themselves rapidly become small compared to Coulomb forces for drops with apoapses $\mbox{max}\left( y\right) \gtrapprox L$, thus it is reasonable to claim that for intermediate drops Coulomb force scales as inertia, and we can neglect the effects of drag and image forces.

\begin{figure}[htb]
    \centering
    \input{../figures/series_s_eu.pgf}
    \caption{Droplet trajectories as a function of $\mathbb{E}\mbox{u}$.\label{fig:series_s_eu}}
\end{figure}
\begin{figure}[htb]
    \centering
    \resizebox{14cm}{!}{%% Creator: Matplotlib, PGF backend
%%
%% To include the figure in your LaTeX document, write
%%   \input{<filename>.pgf}
%%
%% Make sure the required packages are loaded in your preamble
%%   \usepackage{pgf}
%%
%% Figures using additional raster images can only be included by \input if
%% they are in the same directory as the main LaTeX file. For loading figures
%% from other directories you can use the `import` package
%%   \usepackage{import}
%% and then include the figures with
%%   \import{<path to file>}{<filename>.pgf}
%%
%% Matplotlib used the following preamble
%%   \usepackage{fontspec}
%%   \setmainfont{DejaVu Serif}
%%   \setsansfont{DejaVu Sans}
%%   \setmonofont{DejaVu Sans Mono}
%%
\begingroup%
\makeatletter%
\begin{pgfpicture}%
\pgfpathrectangle{\pgfpointorigin}{\pgfqpoint{12.806532in}{8.493808in}}%
\pgfusepath{use as bounding box, clip}%
\begin{pgfscope}%
\pgfsetbuttcap%
\pgfsetmiterjoin%
\definecolor{currentfill}{rgb}{1.000000,1.000000,1.000000}%
\pgfsetfillcolor{currentfill}%
\pgfsetlinewidth{0.000000pt}%
\definecolor{currentstroke}{rgb}{1.000000,1.000000,1.000000}%
\pgfsetstrokecolor{currentstroke}%
\pgfsetdash{}{0pt}%
\pgfpathmoveto{\pgfqpoint{0.000000in}{0.000000in}}%
\pgfpathlineto{\pgfqpoint{12.806532in}{0.000000in}}%
\pgfpathlineto{\pgfqpoint{12.806532in}{8.493808in}}%
\pgfpathlineto{\pgfqpoint{0.000000in}{8.493808in}}%
\pgfpathclose%
\pgfusepath{fill}%
\end{pgfscope}%
\begin{pgfscope}%
\pgfsetbuttcap%
\pgfsetmiterjoin%
\definecolor{currentfill}{rgb}{1.000000,1.000000,1.000000}%
\pgfsetfillcolor{currentfill}%
\pgfsetlinewidth{0.000000pt}%
\definecolor{currentstroke}{rgb}{0.000000,0.000000,0.000000}%
\pgfsetstrokecolor{currentstroke}%
\pgfsetstrokeopacity{0.000000}%
\pgfsetdash{}{0pt}%
\pgfpathmoveto{\pgfqpoint{0.880000in}{6.967719in}}%
\pgfpathlineto{\pgfqpoint{2.777959in}{6.967719in}}%
\pgfpathlineto{\pgfqpoint{2.777959in}{8.340446in}}%
\pgfpathlineto{\pgfqpoint{0.880000in}{8.340446in}}%
\pgfpathclose%
\pgfusepath{fill}%
\end{pgfscope}%
\begin{pgfscope}%
\pgfsetbuttcap%
\pgfsetroundjoin%
\definecolor{currentfill}{rgb}{0.000000,0.000000,0.000000}%
\pgfsetfillcolor{currentfill}%
\pgfsetlinewidth{0.803000pt}%
\definecolor{currentstroke}{rgb}{0.000000,0.000000,0.000000}%
\pgfsetstrokecolor{currentstroke}%
\pgfsetdash{}{0pt}%
\pgfsys@defobject{currentmarker}{\pgfqpoint{0.000000in}{-0.048611in}}{\pgfqpoint{0.000000in}{0.000000in}}{%
\pgfpathmoveto{\pgfqpoint{0.000000in}{0.000000in}}%
\pgfpathlineto{\pgfqpoint{0.000000in}{-0.048611in}}%
\pgfusepath{stroke,fill}%
}%
\begin{pgfscope}%
\pgfsys@transformshift{0.966271in}{6.967719in}%
\pgfsys@useobject{currentmarker}{}%
\end{pgfscope}%
\end{pgfscope}%
\begin{pgfscope}%
\pgftext[x=0.966271in,y=6.870496in,,top]{\rmfamily\fontsize{10.000000}{12.000000}\selectfont \(\displaystyle 0.075\)}%
\end{pgfscope}%
\begin{pgfscope}%
\pgfsetbuttcap%
\pgfsetroundjoin%
\definecolor{currentfill}{rgb}{0.000000,0.000000,0.000000}%
\pgfsetfillcolor{currentfill}%
\pgfsetlinewidth{0.803000pt}%
\definecolor{currentstroke}{rgb}{0.000000,0.000000,0.000000}%
\pgfsetstrokecolor{currentstroke}%
\pgfsetdash{}{0pt}%
\pgfsys@defobject{currentmarker}{\pgfqpoint{0.000000in}{-0.048611in}}{\pgfqpoint{0.000000in}{0.000000in}}{%
\pgfpathmoveto{\pgfqpoint{0.000000in}{0.000000in}}%
\pgfpathlineto{\pgfqpoint{0.000000in}{-0.048611in}}%
\pgfusepath{stroke,fill}%
}%
\begin{pgfscope}%
\pgfsys@transformshift{1.541410in}{6.967719in}%
\pgfsys@useobject{currentmarker}{}%
\end{pgfscope}%
\end{pgfscope}%
\begin{pgfscope}%
\pgftext[x=1.541410in,y=6.870496in,,top]{\rmfamily\fontsize{10.000000}{12.000000}\selectfont \(\displaystyle 0.100\)}%
\end{pgfscope}%
\begin{pgfscope}%
\pgfsetbuttcap%
\pgfsetroundjoin%
\definecolor{currentfill}{rgb}{0.000000,0.000000,0.000000}%
\pgfsetfillcolor{currentfill}%
\pgfsetlinewidth{0.803000pt}%
\definecolor{currentstroke}{rgb}{0.000000,0.000000,0.000000}%
\pgfsetstrokecolor{currentstroke}%
\pgfsetdash{}{0pt}%
\pgfsys@defobject{currentmarker}{\pgfqpoint{0.000000in}{-0.048611in}}{\pgfqpoint{0.000000in}{0.000000in}}{%
\pgfpathmoveto{\pgfqpoint{0.000000in}{0.000000in}}%
\pgfpathlineto{\pgfqpoint{0.000000in}{-0.048611in}}%
\pgfusepath{stroke,fill}%
}%
\begin{pgfscope}%
\pgfsys@transformshift{2.116549in}{6.967719in}%
\pgfsys@useobject{currentmarker}{}%
\end{pgfscope}%
\end{pgfscope}%
\begin{pgfscope}%
\pgftext[x=2.116549in,y=6.870496in,,top]{\rmfamily\fontsize{10.000000}{12.000000}\selectfont \(\displaystyle 0.125\)}%
\end{pgfscope}%
\begin{pgfscope}%
\pgfsetbuttcap%
\pgfsetroundjoin%
\definecolor{currentfill}{rgb}{0.000000,0.000000,0.000000}%
\pgfsetfillcolor{currentfill}%
\pgfsetlinewidth{0.803000pt}%
\definecolor{currentstroke}{rgb}{0.000000,0.000000,0.000000}%
\pgfsetstrokecolor{currentstroke}%
\pgfsetdash{}{0pt}%
\pgfsys@defobject{currentmarker}{\pgfqpoint{0.000000in}{-0.048611in}}{\pgfqpoint{0.000000in}{0.000000in}}{%
\pgfpathmoveto{\pgfqpoint{0.000000in}{0.000000in}}%
\pgfpathlineto{\pgfqpoint{0.000000in}{-0.048611in}}%
\pgfusepath{stroke,fill}%
}%
\begin{pgfscope}%
\pgfsys@transformshift{2.691688in}{6.967719in}%
\pgfsys@useobject{currentmarker}{}%
\end{pgfscope}%
\end{pgfscope}%
\begin{pgfscope}%
\pgftext[x=2.691688in,y=6.870496in,,top]{\rmfamily\fontsize{10.000000}{12.000000}\selectfont \(\displaystyle 0.150\)}%
\end{pgfscope}%
\begin{pgfscope}%
\pgfsetbuttcap%
\pgfsetroundjoin%
\definecolor{currentfill}{rgb}{0.000000,0.000000,0.000000}%
\pgfsetfillcolor{currentfill}%
\pgfsetlinewidth{0.803000pt}%
\definecolor{currentstroke}{rgb}{0.000000,0.000000,0.000000}%
\pgfsetstrokecolor{currentstroke}%
\pgfsetdash{}{0pt}%
\pgfsys@defobject{currentmarker}{\pgfqpoint{-0.048611in}{0.000000in}}{\pgfqpoint{0.000000in}{0.000000in}}{%
\pgfpathmoveto{\pgfqpoint{0.000000in}{0.000000in}}%
\pgfpathlineto{\pgfqpoint{-0.048611in}{0.000000in}}%
\pgfusepath{stroke,fill}%
}%
\begin{pgfscope}%
\pgfsys@transformshift{0.880000in}{7.505067in}%
\pgfsys@useobject{currentmarker}{}%
\end{pgfscope}%
\end{pgfscope}%
\begin{pgfscope}%
\pgftext[x=0.494775in,y=7.452305in,left,base]{\rmfamily\fontsize{10.000000}{12.000000}\selectfont \(\displaystyle 10^{-7}\)}%
\end{pgfscope}%
\begin{pgfscope}%
\pgfsetbuttcap%
\pgfsetroundjoin%
\definecolor{currentfill}{rgb}{0.000000,0.000000,0.000000}%
\pgfsetfillcolor{currentfill}%
\pgfsetlinewidth{0.803000pt}%
\definecolor{currentstroke}{rgb}{0.000000,0.000000,0.000000}%
\pgfsetstrokecolor{currentstroke}%
\pgfsetdash{}{0pt}%
\pgfsys@defobject{currentmarker}{\pgfqpoint{-0.048611in}{0.000000in}}{\pgfqpoint{0.000000in}{0.000000in}}{%
\pgfpathmoveto{\pgfqpoint{0.000000in}{0.000000in}}%
\pgfpathlineto{\pgfqpoint{-0.048611in}{0.000000in}}%
\pgfusepath{stroke,fill}%
}%
\begin{pgfscope}%
\pgfsys@transformshift{0.880000in}{8.094310in}%
\pgfsys@useobject{currentmarker}{}%
\end{pgfscope}%
\end{pgfscope}%
\begin{pgfscope}%
\pgftext[x=0.494775in,y=8.041548in,left,base]{\rmfamily\fontsize{10.000000}{12.000000}\selectfont \(\displaystyle 10^{-6}\)}%
\end{pgfscope}%
\begin{pgfscope}%
\pgfsetbuttcap%
\pgfsetroundjoin%
\definecolor{currentfill}{rgb}{0.000000,0.000000,0.000000}%
\pgfsetfillcolor{currentfill}%
\pgfsetlinewidth{0.602250pt}%
\definecolor{currentstroke}{rgb}{0.000000,0.000000,0.000000}%
\pgfsetstrokecolor{currentstroke}%
\pgfsetdash{}{0pt}%
\pgfsys@defobject{currentmarker}{\pgfqpoint{-0.027778in}{0.000000in}}{\pgfqpoint{0.000000in}{0.000000in}}{%
\pgfpathmoveto{\pgfqpoint{0.000000in}{0.000000in}}%
\pgfpathlineto{\pgfqpoint{-0.027778in}{0.000000in}}%
\pgfusepath{stroke,fill}%
}%
\begin{pgfscope}%
\pgfsys@transformshift{0.880000in}{7.093203in}%
\pgfsys@useobject{currentmarker}{}%
\end{pgfscope}%
\end{pgfscope}%
\begin{pgfscope}%
\pgfsetbuttcap%
\pgfsetroundjoin%
\definecolor{currentfill}{rgb}{0.000000,0.000000,0.000000}%
\pgfsetfillcolor{currentfill}%
\pgfsetlinewidth{0.602250pt}%
\definecolor{currentstroke}{rgb}{0.000000,0.000000,0.000000}%
\pgfsetstrokecolor{currentstroke}%
\pgfsetdash{}{0pt}%
\pgfsys@defobject{currentmarker}{\pgfqpoint{-0.027778in}{0.000000in}}{\pgfqpoint{0.000000in}{0.000000in}}{%
\pgfpathmoveto{\pgfqpoint{0.000000in}{0.000000in}}%
\pgfpathlineto{\pgfqpoint{-0.027778in}{0.000000in}}%
\pgfusepath{stroke,fill}%
}%
\begin{pgfscope}%
\pgfsys@transformshift{0.880000in}{7.196964in}%
\pgfsys@useobject{currentmarker}{}%
\end{pgfscope}%
\end{pgfscope}%
\begin{pgfscope}%
\pgfsetbuttcap%
\pgfsetroundjoin%
\definecolor{currentfill}{rgb}{0.000000,0.000000,0.000000}%
\pgfsetfillcolor{currentfill}%
\pgfsetlinewidth{0.602250pt}%
\definecolor{currentstroke}{rgb}{0.000000,0.000000,0.000000}%
\pgfsetstrokecolor{currentstroke}%
\pgfsetdash{}{0pt}%
\pgfsys@defobject{currentmarker}{\pgfqpoint{-0.027778in}{0.000000in}}{\pgfqpoint{0.000000in}{0.000000in}}{%
\pgfpathmoveto{\pgfqpoint{0.000000in}{0.000000in}}%
\pgfpathlineto{\pgfqpoint{-0.027778in}{0.000000in}}%
\pgfusepath{stroke,fill}%
}%
\begin{pgfscope}%
\pgfsys@transformshift{0.880000in}{7.270583in}%
\pgfsys@useobject{currentmarker}{}%
\end{pgfscope}%
\end{pgfscope}%
\begin{pgfscope}%
\pgfsetbuttcap%
\pgfsetroundjoin%
\definecolor{currentfill}{rgb}{0.000000,0.000000,0.000000}%
\pgfsetfillcolor{currentfill}%
\pgfsetlinewidth{0.602250pt}%
\definecolor{currentstroke}{rgb}{0.000000,0.000000,0.000000}%
\pgfsetstrokecolor{currentstroke}%
\pgfsetdash{}{0pt}%
\pgfsys@defobject{currentmarker}{\pgfqpoint{-0.027778in}{0.000000in}}{\pgfqpoint{0.000000in}{0.000000in}}{%
\pgfpathmoveto{\pgfqpoint{0.000000in}{0.000000in}}%
\pgfpathlineto{\pgfqpoint{-0.027778in}{0.000000in}}%
\pgfusepath{stroke,fill}%
}%
\begin{pgfscope}%
\pgfsys@transformshift{0.880000in}{7.327687in}%
\pgfsys@useobject{currentmarker}{}%
\end{pgfscope}%
\end{pgfscope}%
\begin{pgfscope}%
\pgfsetbuttcap%
\pgfsetroundjoin%
\definecolor{currentfill}{rgb}{0.000000,0.000000,0.000000}%
\pgfsetfillcolor{currentfill}%
\pgfsetlinewidth{0.602250pt}%
\definecolor{currentstroke}{rgb}{0.000000,0.000000,0.000000}%
\pgfsetstrokecolor{currentstroke}%
\pgfsetdash{}{0pt}%
\pgfsys@defobject{currentmarker}{\pgfqpoint{-0.027778in}{0.000000in}}{\pgfqpoint{0.000000in}{0.000000in}}{%
\pgfpathmoveto{\pgfqpoint{0.000000in}{0.000000in}}%
\pgfpathlineto{\pgfqpoint{-0.027778in}{0.000000in}}%
\pgfusepath{stroke,fill}%
}%
\begin{pgfscope}%
\pgfsys@transformshift{0.880000in}{7.374344in}%
\pgfsys@useobject{currentmarker}{}%
\end{pgfscope}%
\end{pgfscope}%
\begin{pgfscope}%
\pgfsetbuttcap%
\pgfsetroundjoin%
\definecolor{currentfill}{rgb}{0.000000,0.000000,0.000000}%
\pgfsetfillcolor{currentfill}%
\pgfsetlinewidth{0.602250pt}%
\definecolor{currentstroke}{rgb}{0.000000,0.000000,0.000000}%
\pgfsetstrokecolor{currentstroke}%
\pgfsetdash{}{0pt}%
\pgfsys@defobject{currentmarker}{\pgfqpoint{-0.027778in}{0.000000in}}{\pgfqpoint{0.000000in}{0.000000in}}{%
\pgfpathmoveto{\pgfqpoint{0.000000in}{0.000000in}}%
\pgfpathlineto{\pgfqpoint{-0.027778in}{0.000000in}}%
\pgfusepath{stroke,fill}%
}%
\begin{pgfscope}%
\pgfsys@transformshift{0.880000in}{7.413792in}%
\pgfsys@useobject{currentmarker}{}%
\end{pgfscope}%
\end{pgfscope}%
\begin{pgfscope}%
\pgfsetbuttcap%
\pgfsetroundjoin%
\definecolor{currentfill}{rgb}{0.000000,0.000000,0.000000}%
\pgfsetfillcolor{currentfill}%
\pgfsetlinewidth{0.602250pt}%
\definecolor{currentstroke}{rgb}{0.000000,0.000000,0.000000}%
\pgfsetstrokecolor{currentstroke}%
\pgfsetdash{}{0pt}%
\pgfsys@defobject{currentmarker}{\pgfqpoint{-0.027778in}{0.000000in}}{\pgfqpoint{0.000000in}{0.000000in}}{%
\pgfpathmoveto{\pgfqpoint{0.000000in}{0.000000in}}%
\pgfpathlineto{\pgfqpoint{-0.027778in}{0.000000in}}%
\pgfusepath{stroke,fill}%
}%
\begin{pgfscope}%
\pgfsys@transformshift{0.880000in}{7.447963in}%
\pgfsys@useobject{currentmarker}{}%
\end{pgfscope}%
\end{pgfscope}%
\begin{pgfscope}%
\pgfsetbuttcap%
\pgfsetroundjoin%
\definecolor{currentfill}{rgb}{0.000000,0.000000,0.000000}%
\pgfsetfillcolor{currentfill}%
\pgfsetlinewidth{0.602250pt}%
\definecolor{currentstroke}{rgb}{0.000000,0.000000,0.000000}%
\pgfsetstrokecolor{currentstroke}%
\pgfsetdash{}{0pt}%
\pgfsys@defobject{currentmarker}{\pgfqpoint{-0.027778in}{0.000000in}}{\pgfqpoint{0.000000in}{0.000000in}}{%
\pgfpathmoveto{\pgfqpoint{0.000000in}{0.000000in}}%
\pgfpathlineto{\pgfqpoint{-0.027778in}{0.000000in}}%
\pgfusepath{stroke,fill}%
}%
\begin{pgfscope}%
\pgfsys@transformshift{0.880000in}{7.478104in}%
\pgfsys@useobject{currentmarker}{}%
\end{pgfscope}%
\end{pgfscope}%
\begin{pgfscope}%
\pgfsetbuttcap%
\pgfsetroundjoin%
\definecolor{currentfill}{rgb}{0.000000,0.000000,0.000000}%
\pgfsetfillcolor{currentfill}%
\pgfsetlinewidth{0.602250pt}%
\definecolor{currentstroke}{rgb}{0.000000,0.000000,0.000000}%
\pgfsetstrokecolor{currentstroke}%
\pgfsetdash{}{0pt}%
\pgfsys@defobject{currentmarker}{\pgfqpoint{-0.027778in}{0.000000in}}{\pgfqpoint{0.000000in}{0.000000in}}{%
\pgfpathmoveto{\pgfqpoint{0.000000in}{0.000000in}}%
\pgfpathlineto{\pgfqpoint{-0.027778in}{0.000000in}}%
\pgfusepath{stroke,fill}%
}%
\begin{pgfscope}%
\pgfsys@transformshift{0.880000in}{7.682446in}%
\pgfsys@useobject{currentmarker}{}%
\end{pgfscope}%
\end{pgfscope}%
\begin{pgfscope}%
\pgfsetbuttcap%
\pgfsetroundjoin%
\definecolor{currentfill}{rgb}{0.000000,0.000000,0.000000}%
\pgfsetfillcolor{currentfill}%
\pgfsetlinewidth{0.602250pt}%
\definecolor{currentstroke}{rgb}{0.000000,0.000000,0.000000}%
\pgfsetstrokecolor{currentstroke}%
\pgfsetdash{}{0pt}%
\pgfsys@defobject{currentmarker}{\pgfqpoint{-0.027778in}{0.000000in}}{\pgfqpoint{0.000000in}{0.000000in}}{%
\pgfpathmoveto{\pgfqpoint{0.000000in}{0.000000in}}%
\pgfpathlineto{\pgfqpoint{-0.027778in}{0.000000in}}%
\pgfusepath{stroke,fill}%
}%
\begin{pgfscope}%
\pgfsys@transformshift{0.880000in}{7.786207in}%
\pgfsys@useobject{currentmarker}{}%
\end{pgfscope}%
\end{pgfscope}%
\begin{pgfscope}%
\pgfsetbuttcap%
\pgfsetroundjoin%
\definecolor{currentfill}{rgb}{0.000000,0.000000,0.000000}%
\pgfsetfillcolor{currentfill}%
\pgfsetlinewidth{0.602250pt}%
\definecolor{currentstroke}{rgb}{0.000000,0.000000,0.000000}%
\pgfsetstrokecolor{currentstroke}%
\pgfsetdash{}{0pt}%
\pgfsys@defobject{currentmarker}{\pgfqpoint{-0.027778in}{0.000000in}}{\pgfqpoint{0.000000in}{0.000000in}}{%
\pgfpathmoveto{\pgfqpoint{0.000000in}{0.000000in}}%
\pgfpathlineto{\pgfqpoint{-0.027778in}{0.000000in}}%
\pgfusepath{stroke,fill}%
}%
\begin{pgfscope}%
\pgfsys@transformshift{0.880000in}{7.859826in}%
\pgfsys@useobject{currentmarker}{}%
\end{pgfscope}%
\end{pgfscope}%
\begin{pgfscope}%
\pgfsetbuttcap%
\pgfsetroundjoin%
\definecolor{currentfill}{rgb}{0.000000,0.000000,0.000000}%
\pgfsetfillcolor{currentfill}%
\pgfsetlinewidth{0.602250pt}%
\definecolor{currentstroke}{rgb}{0.000000,0.000000,0.000000}%
\pgfsetstrokecolor{currentstroke}%
\pgfsetdash{}{0pt}%
\pgfsys@defobject{currentmarker}{\pgfqpoint{-0.027778in}{0.000000in}}{\pgfqpoint{0.000000in}{0.000000in}}{%
\pgfpathmoveto{\pgfqpoint{0.000000in}{0.000000in}}%
\pgfpathlineto{\pgfqpoint{-0.027778in}{0.000000in}}%
\pgfusepath{stroke,fill}%
}%
\begin{pgfscope}%
\pgfsys@transformshift{0.880000in}{7.916930in}%
\pgfsys@useobject{currentmarker}{}%
\end{pgfscope}%
\end{pgfscope}%
\begin{pgfscope}%
\pgfsetbuttcap%
\pgfsetroundjoin%
\definecolor{currentfill}{rgb}{0.000000,0.000000,0.000000}%
\pgfsetfillcolor{currentfill}%
\pgfsetlinewidth{0.602250pt}%
\definecolor{currentstroke}{rgb}{0.000000,0.000000,0.000000}%
\pgfsetstrokecolor{currentstroke}%
\pgfsetdash{}{0pt}%
\pgfsys@defobject{currentmarker}{\pgfqpoint{-0.027778in}{0.000000in}}{\pgfqpoint{0.000000in}{0.000000in}}{%
\pgfpathmoveto{\pgfqpoint{0.000000in}{0.000000in}}%
\pgfpathlineto{\pgfqpoint{-0.027778in}{0.000000in}}%
\pgfusepath{stroke,fill}%
}%
\begin{pgfscope}%
\pgfsys@transformshift{0.880000in}{7.963587in}%
\pgfsys@useobject{currentmarker}{}%
\end{pgfscope}%
\end{pgfscope}%
\begin{pgfscope}%
\pgfsetbuttcap%
\pgfsetroundjoin%
\definecolor{currentfill}{rgb}{0.000000,0.000000,0.000000}%
\pgfsetfillcolor{currentfill}%
\pgfsetlinewidth{0.602250pt}%
\definecolor{currentstroke}{rgb}{0.000000,0.000000,0.000000}%
\pgfsetstrokecolor{currentstroke}%
\pgfsetdash{}{0pt}%
\pgfsys@defobject{currentmarker}{\pgfqpoint{-0.027778in}{0.000000in}}{\pgfqpoint{0.000000in}{0.000000in}}{%
\pgfpathmoveto{\pgfqpoint{0.000000in}{0.000000in}}%
\pgfpathlineto{\pgfqpoint{-0.027778in}{0.000000in}}%
\pgfusepath{stroke,fill}%
}%
\begin{pgfscope}%
\pgfsys@transformshift{0.880000in}{8.003035in}%
\pgfsys@useobject{currentmarker}{}%
\end{pgfscope}%
\end{pgfscope}%
\begin{pgfscope}%
\pgfsetbuttcap%
\pgfsetroundjoin%
\definecolor{currentfill}{rgb}{0.000000,0.000000,0.000000}%
\pgfsetfillcolor{currentfill}%
\pgfsetlinewidth{0.602250pt}%
\definecolor{currentstroke}{rgb}{0.000000,0.000000,0.000000}%
\pgfsetstrokecolor{currentstroke}%
\pgfsetdash{}{0pt}%
\pgfsys@defobject{currentmarker}{\pgfqpoint{-0.027778in}{0.000000in}}{\pgfqpoint{0.000000in}{0.000000in}}{%
\pgfpathmoveto{\pgfqpoint{0.000000in}{0.000000in}}%
\pgfpathlineto{\pgfqpoint{-0.027778in}{0.000000in}}%
\pgfusepath{stroke,fill}%
}%
\begin{pgfscope}%
\pgfsys@transformshift{0.880000in}{8.037206in}%
\pgfsys@useobject{currentmarker}{}%
\end{pgfscope}%
\end{pgfscope}%
\begin{pgfscope}%
\pgfsetbuttcap%
\pgfsetroundjoin%
\definecolor{currentfill}{rgb}{0.000000,0.000000,0.000000}%
\pgfsetfillcolor{currentfill}%
\pgfsetlinewidth{0.602250pt}%
\definecolor{currentstroke}{rgb}{0.000000,0.000000,0.000000}%
\pgfsetstrokecolor{currentstroke}%
\pgfsetdash{}{0pt}%
\pgfsys@defobject{currentmarker}{\pgfqpoint{-0.027778in}{0.000000in}}{\pgfqpoint{0.000000in}{0.000000in}}{%
\pgfpathmoveto{\pgfqpoint{0.000000in}{0.000000in}}%
\pgfpathlineto{\pgfqpoint{-0.027778in}{0.000000in}}%
\pgfusepath{stroke,fill}%
}%
\begin{pgfscope}%
\pgfsys@transformshift{0.880000in}{8.067347in}%
\pgfsys@useobject{currentmarker}{}%
\end{pgfscope}%
\end{pgfscope}%
\begin{pgfscope}%
\pgfsetbuttcap%
\pgfsetroundjoin%
\definecolor{currentfill}{rgb}{0.000000,0.000000,0.000000}%
\pgfsetfillcolor{currentfill}%
\pgfsetlinewidth{0.602250pt}%
\definecolor{currentstroke}{rgb}{0.000000,0.000000,0.000000}%
\pgfsetstrokecolor{currentstroke}%
\pgfsetdash{}{0pt}%
\pgfsys@defobject{currentmarker}{\pgfqpoint{-0.027778in}{0.000000in}}{\pgfqpoint{0.000000in}{0.000000in}}{%
\pgfpathmoveto{\pgfqpoint{0.000000in}{0.000000in}}%
\pgfpathlineto{\pgfqpoint{-0.027778in}{0.000000in}}%
\pgfusepath{stroke,fill}%
}%
\begin{pgfscope}%
\pgfsys@transformshift{0.880000in}{8.271690in}%
\pgfsys@useobject{currentmarker}{}%
\end{pgfscope}%
\end{pgfscope}%
\begin{pgfscope}%
\pgfpathrectangle{\pgfqpoint{0.880000in}{6.967719in}}{\pgfqpoint{1.897959in}{1.372727in}} %
\pgfusepath{clip}%
\pgfsetbuttcap%
\pgfsetroundjoin%
\pgfsetlinewidth{1.505625pt}%
\definecolor{currentstroke}{rgb}{1.000000,0.000000,0.000000}%
\pgfsetstrokecolor{currentstroke}%
\pgfsetdash{{5.550000pt}{2.400000pt}}{0.000000pt}%
\pgfpathmoveto{\pgfqpoint{0.966271in}{8.242165in}}%
\pgfpathlineto{\pgfqpoint{1.157984in}{8.240978in}}%
\pgfpathlineto{\pgfqpoint{1.349697in}{8.239882in}}%
\pgfpathlineto{\pgfqpoint{1.541410in}{8.238877in}}%
\pgfpathlineto{\pgfqpoint{1.733123in}{8.237961in}}%
\pgfpathlineto{\pgfqpoint{1.924836in}{8.237135in}}%
\pgfpathlineto{\pgfqpoint{2.116549in}{8.236398in}}%
\pgfpathlineto{\pgfqpoint{2.308262in}{8.235751in}}%
\pgfpathlineto{\pgfqpoint{2.499975in}{8.235193in}}%
\pgfpathlineto{\pgfqpoint{2.691688in}{8.234723in}}%
\pgfusepath{stroke}%
\end{pgfscope}%
\begin{pgfscope}%
\pgfpathrectangle{\pgfqpoint{0.880000in}{6.967719in}}{\pgfqpoint{1.897959in}{1.372727in}} %
\pgfusepath{clip}%
\pgfsetbuttcap%
\pgfsetmiterjoin%
\definecolor{currentfill}{rgb}{1.000000,0.000000,0.000000}%
\pgfsetfillcolor{currentfill}%
\pgfsetlinewidth{1.003750pt}%
\definecolor{currentstroke}{rgb}{1.000000,0.000000,0.000000}%
\pgfsetstrokecolor{currentstroke}%
\pgfsetdash{}{0pt}%
\pgfsys@defobject{currentmarker}{\pgfqpoint{-0.041667in}{-0.041667in}}{\pgfqpoint{0.041667in}{0.041667in}}{%
\pgfpathmoveto{\pgfqpoint{-0.041667in}{-0.041667in}}%
\pgfpathlineto{\pgfqpoint{0.041667in}{-0.041667in}}%
\pgfpathlineto{\pgfqpoint{0.041667in}{0.041667in}}%
\pgfpathlineto{\pgfqpoint{-0.041667in}{0.041667in}}%
\pgfpathclose%
\pgfusepath{stroke,fill}%
}%
\begin{pgfscope}%
\pgfsys@transformshift{0.966271in}{8.242165in}%
\pgfsys@useobject{currentmarker}{}%
\end{pgfscope}%
\begin{pgfscope}%
\pgfsys@transformshift{1.349697in}{8.239882in}%
\pgfsys@useobject{currentmarker}{}%
\end{pgfscope}%
\begin{pgfscope}%
\pgfsys@transformshift{1.733123in}{8.237961in}%
\pgfsys@useobject{currentmarker}{}%
\end{pgfscope}%
\begin{pgfscope}%
\pgfsys@transformshift{2.116549in}{8.236398in}%
\pgfsys@useobject{currentmarker}{}%
\end{pgfscope}%
\begin{pgfscope}%
\pgfsys@transformshift{2.499975in}{8.235193in}%
\pgfsys@useobject{currentmarker}{}%
\end{pgfscope}%
\end{pgfscope}%
\begin{pgfscope}%
\pgfpathrectangle{\pgfqpoint{0.880000in}{6.967719in}}{\pgfqpoint{1.897959in}{1.372727in}} %
\pgfusepath{clip}%
\pgfsetrectcap%
\pgfsetroundjoin%
\pgfsetlinewidth{1.505625pt}%
\definecolor{currentstroke}{rgb}{0.000000,0.000000,1.000000}%
\pgfsetstrokecolor{currentstroke}%
\pgfsetdash{}{0pt}%
\pgfpathmoveto{\pgfqpoint{0.966271in}{7.344411in}}%
\pgfpathlineto{\pgfqpoint{1.157984in}{7.320663in}}%
\pgfpathlineto{\pgfqpoint{1.349697in}{7.295335in}}%
\pgfpathlineto{\pgfqpoint{1.541410in}{7.268119in}}%
\pgfpathlineto{\pgfqpoint{1.733123in}{7.238624in}}%
\pgfpathlineto{\pgfqpoint{1.924836in}{7.206340in}}%
\pgfpathlineto{\pgfqpoint{2.116549in}{7.170577in}}%
\pgfpathlineto{\pgfqpoint{2.308262in}{7.130367in}}%
\pgfpathlineto{\pgfqpoint{2.499975in}{7.084288in}}%
\pgfpathlineto{\pgfqpoint{2.691688in}{7.030115in}}%
\pgfusepath{stroke}%
\end{pgfscope}%
\begin{pgfscope}%
\pgfpathrectangle{\pgfqpoint{0.880000in}{6.967719in}}{\pgfqpoint{1.897959in}{1.372727in}} %
\pgfusepath{clip}%
\pgfsetbuttcap%
\pgfsetroundjoin%
\definecolor{currentfill}{rgb}{0.000000,0.000000,1.000000}%
\pgfsetfillcolor{currentfill}%
\pgfsetlinewidth{1.003750pt}%
\definecolor{currentstroke}{rgb}{0.000000,0.000000,1.000000}%
\pgfsetstrokecolor{currentstroke}%
\pgfsetdash{}{0pt}%
\pgfsys@defobject{currentmarker}{\pgfqpoint{-0.041667in}{-0.041667in}}{\pgfqpoint{0.041667in}{0.041667in}}{%
\pgfpathmoveto{\pgfqpoint{0.000000in}{-0.041667in}}%
\pgfpathcurveto{\pgfqpoint{0.011050in}{-0.041667in}}{\pgfqpoint{0.021649in}{-0.037276in}}{\pgfqpoint{0.029463in}{-0.029463in}}%
\pgfpathcurveto{\pgfqpoint{0.037276in}{-0.021649in}}{\pgfqpoint{0.041667in}{-0.011050in}}{\pgfqpoint{0.041667in}{0.000000in}}%
\pgfpathcurveto{\pgfqpoint{0.041667in}{0.011050in}}{\pgfqpoint{0.037276in}{0.021649in}}{\pgfqpoint{0.029463in}{0.029463in}}%
\pgfpathcurveto{\pgfqpoint{0.021649in}{0.037276in}}{\pgfqpoint{0.011050in}{0.041667in}}{\pgfqpoint{0.000000in}{0.041667in}}%
\pgfpathcurveto{\pgfqpoint{-0.011050in}{0.041667in}}{\pgfqpoint{-0.021649in}{0.037276in}}{\pgfqpoint{-0.029463in}{0.029463in}}%
\pgfpathcurveto{\pgfqpoint{-0.037276in}{0.021649in}}{\pgfqpoint{-0.041667in}{0.011050in}}{\pgfqpoint{-0.041667in}{0.000000in}}%
\pgfpathcurveto{\pgfqpoint{-0.041667in}{-0.011050in}}{\pgfqpoint{-0.037276in}{-0.021649in}}{\pgfqpoint{-0.029463in}{-0.029463in}}%
\pgfpathcurveto{\pgfqpoint{-0.021649in}{-0.037276in}}{\pgfqpoint{-0.011050in}{-0.041667in}}{\pgfqpoint{0.000000in}{-0.041667in}}%
\pgfpathclose%
\pgfusepath{stroke,fill}%
}%
\begin{pgfscope}%
\pgfsys@transformshift{0.966271in}{7.344411in}%
\pgfsys@useobject{currentmarker}{}%
\end{pgfscope}%
\begin{pgfscope}%
\pgfsys@transformshift{1.349697in}{7.295335in}%
\pgfsys@useobject{currentmarker}{}%
\end{pgfscope}%
\begin{pgfscope}%
\pgfsys@transformshift{1.733123in}{7.238624in}%
\pgfsys@useobject{currentmarker}{}%
\end{pgfscope}%
\begin{pgfscope}%
\pgfsys@transformshift{2.116549in}{7.170577in}%
\pgfsys@useobject{currentmarker}{}%
\end{pgfscope}%
\begin{pgfscope}%
\pgfsys@transformshift{2.499975in}{7.084288in}%
\pgfsys@useobject{currentmarker}{}%
\end{pgfscope}%
\end{pgfscope}%
\begin{pgfscope}%
\pgfpathrectangle{\pgfqpoint{0.880000in}{6.967719in}}{\pgfqpoint{1.897959in}{1.372727in}} %
\pgfusepath{clip}%
\pgfsetbuttcap%
\pgfsetroundjoin%
\pgfsetlinewidth{1.505625pt}%
\definecolor{currentstroke}{rgb}{0.000000,0.750000,0.750000}%
\pgfsetstrokecolor{currentstroke}%
\pgfsetdash{{9.600000pt}{2.400000pt}{1.500000pt}{2.400000pt}}{0.000000pt}%
\pgfpathmoveto{\pgfqpoint{0.966271in}{7.702201in}}%
\pgfpathlineto{\pgfqpoint{1.157984in}{7.683138in}}%
\pgfpathlineto{\pgfqpoint{1.349697in}{7.666190in}}%
\pgfpathlineto{\pgfqpoint{1.541410in}{7.651157in}}%
\pgfpathlineto{\pgfqpoint{1.733123in}{7.637876in}}%
\pgfpathlineto{\pgfqpoint{1.924836in}{7.626211in}}%
\pgfpathlineto{\pgfqpoint{2.116549in}{7.616049in}}%
\pgfpathlineto{\pgfqpoint{2.308262in}{7.607297in}}%
\pgfpathlineto{\pgfqpoint{2.499975in}{7.599880in}}%
\pgfpathlineto{\pgfqpoint{2.691688in}{7.593734in}}%
\pgfusepath{stroke}%
\end{pgfscope}%
\begin{pgfscope}%
\pgfpathrectangle{\pgfqpoint{0.880000in}{6.967719in}}{\pgfqpoint{1.897959in}{1.372727in}} %
\pgfusepath{clip}%
\pgfsetbuttcap%
\pgfsetmiterjoin%
\definecolor{currentfill}{rgb}{0.000000,0.750000,0.750000}%
\pgfsetfillcolor{currentfill}%
\pgfsetlinewidth{1.003750pt}%
\definecolor{currentstroke}{rgb}{0.000000,0.750000,0.750000}%
\pgfsetstrokecolor{currentstroke}%
\pgfsetdash{}{0pt}%
\pgfsys@defobject{currentmarker}{\pgfqpoint{-0.041667in}{-0.041667in}}{\pgfqpoint{0.041667in}{0.041667in}}{%
\pgfpathmoveto{\pgfqpoint{-0.000000in}{-0.041667in}}%
\pgfpathlineto{\pgfqpoint{0.041667in}{0.041667in}}%
\pgfpathlineto{\pgfqpoint{-0.041667in}{0.041667in}}%
\pgfpathclose%
\pgfusepath{stroke,fill}%
}%
\begin{pgfscope}%
\pgfsys@transformshift{0.966271in}{7.702201in}%
\pgfsys@useobject{currentmarker}{}%
\end{pgfscope}%
\begin{pgfscope}%
\pgfsys@transformshift{1.349697in}{7.666190in}%
\pgfsys@useobject{currentmarker}{}%
\end{pgfscope}%
\begin{pgfscope}%
\pgfsys@transformshift{1.733123in}{7.637876in}%
\pgfsys@useobject{currentmarker}{}%
\end{pgfscope}%
\begin{pgfscope}%
\pgfsys@transformshift{2.116549in}{7.616049in}%
\pgfsys@useobject{currentmarker}{}%
\end{pgfscope}%
\begin{pgfscope}%
\pgfsys@transformshift{2.499975in}{7.599880in}%
\pgfsys@useobject{currentmarker}{}%
\end{pgfscope}%
\end{pgfscope}%
\begin{pgfscope}%
\pgfpathrectangle{\pgfqpoint{0.880000in}{6.967719in}}{\pgfqpoint{1.897959in}{1.372727in}} %
\pgfusepath{clip}%
\pgfsetbuttcap%
\pgfsetroundjoin%
\pgfsetlinewidth{1.505625pt}%
\definecolor{currentstroke}{rgb}{0.000000,0.000000,0.000000}%
\pgfsetstrokecolor{currentstroke}%
\pgfsetdash{{1.500000pt}{2.475000pt}}{0.000000pt}%
\pgfpathmoveto{\pgfqpoint{0.966271in}{8.278049in}}%
\pgfpathlineto{\pgfqpoint{1.157984in}{8.274690in}}%
\pgfpathlineto{\pgfqpoint{1.349697in}{8.271493in}}%
\pgfpathlineto{\pgfqpoint{1.541410in}{8.268643in}}%
\pgfpathlineto{\pgfqpoint{1.733123in}{8.266093in}}%
\pgfpathlineto{\pgfqpoint{1.924836in}{8.263813in}}%
\pgfpathlineto{\pgfqpoint{2.116549in}{8.261778in}}%
\pgfpathlineto{\pgfqpoint{2.308262in}{8.259969in}}%
\pgfpathlineto{\pgfqpoint{2.499975in}{8.258373in}}%
\pgfpathlineto{\pgfqpoint{2.691688in}{8.256980in}}%
\pgfusepath{stroke}%
\end{pgfscope}%
\begin{pgfscope}%
\pgfpathrectangle{\pgfqpoint{0.880000in}{6.967719in}}{\pgfqpoint{1.897959in}{1.372727in}} %
\pgfusepath{clip}%
\pgfsetbuttcap%
\pgfsetroundjoin%
\definecolor{currentfill}{rgb}{0.000000,0.000000,0.000000}%
\pgfsetfillcolor{currentfill}%
\pgfsetlinewidth{1.003750pt}%
\definecolor{currentstroke}{rgb}{0.000000,0.000000,0.000000}%
\pgfsetstrokecolor{currentstroke}%
\pgfsetdash{}{0pt}%
\pgfsys@defobject{currentmarker}{\pgfqpoint{-0.041667in}{-0.041667in}}{\pgfqpoint{0.041667in}{0.041667in}}{%
\pgfpathmoveto{\pgfqpoint{-0.041667in}{0.000000in}}%
\pgfpathlineto{\pgfqpoint{0.041667in}{0.000000in}}%
\pgfpathmoveto{\pgfqpoint{0.000000in}{-0.041667in}}%
\pgfpathlineto{\pgfqpoint{0.000000in}{0.041667in}}%
\pgfusepath{stroke,fill}%
}%
\begin{pgfscope}%
\pgfsys@transformshift{0.966271in}{8.278049in}%
\pgfsys@useobject{currentmarker}{}%
\end{pgfscope}%
\begin{pgfscope}%
\pgfsys@transformshift{1.349697in}{8.271493in}%
\pgfsys@useobject{currentmarker}{}%
\end{pgfscope}%
\begin{pgfscope}%
\pgfsys@transformshift{1.733123in}{8.266093in}%
\pgfsys@useobject{currentmarker}{}%
\end{pgfscope}%
\begin{pgfscope}%
\pgfsys@transformshift{2.116549in}{8.261778in}%
\pgfsys@useobject{currentmarker}{}%
\end{pgfscope}%
\begin{pgfscope}%
\pgfsys@transformshift{2.499975in}{8.258373in}%
\pgfsys@useobject{currentmarker}{}%
\end{pgfscope}%
\end{pgfscope}%
\begin{pgfscope}%
\pgfsetrectcap%
\pgfsetmiterjoin%
\pgfsetlinewidth{0.803000pt}%
\definecolor{currentstroke}{rgb}{0.000000,0.000000,0.000000}%
\pgfsetstrokecolor{currentstroke}%
\pgfsetdash{}{0pt}%
\pgfpathmoveto{\pgfqpoint{0.880000in}{6.967719in}}%
\pgfpathlineto{\pgfqpoint{0.880000in}{8.340446in}}%
\pgfusepath{stroke}%
\end{pgfscope}%
\begin{pgfscope}%
\pgfsetrectcap%
\pgfsetmiterjoin%
\pgfsetlinewidth{0.803000pt}%
\definecolor{currentstroke}{rgb}{0.000000,0.000000,0.000000}%
\pgfsetstrokecolor{currentstroke}%
\pgfsetdash{}{0pt}%
\pgfpathmoveto{\pgfqpoint{2.777959in}{6.967719in}}%
\pgfpathlineto{\pgfqpoint{2.777959in}{8.340446in}}%
\pgfusepath{stroke}%
\end{pgfscope}%
\begin{pgfscope}%
\pgfsetrectcap%
\pgfsetmiterjoin%
\pgfsetlinewidth{0.803000pt}%
\definecolor{currentstroke}{rgb}{0.000000,0.000000,0.000000}%
\pgfsetstrokecolor{currentstroke}%
\pgfsetdash{}{0pt}%
\pgfpathmoveto{\pgfqpoint{0.880000in}{6.967719in}}%
\pgfpathlineto{\pgfqpoint{2.777959in}{6.967719in}}%
\pgfusepath{stroke}%
\end{pgfscope}%
\begin{pgfscope}%
\pgfsetrectcap%
\pgfsetmiterjoin%
\pgfsetlinewidth{0.803000pt}%
\definecolor{currentstroke}{rgb}{0.000000,0.000000,0.000000}%
\pgfsetstrokecolor{currentstroke}%
\pgfsetdash{}{0pt}%
\pgfpathmoveto{\pgfqpoint{0.880000in}{8.340446in}}%
\pgfpathlineto{\pgfqpoint{2.777959in}{8.340446in}}%
\pgfusepath{stroke}%
\end{pgfscope}%
\begin{pgfscope}%
\pgfsetbuttcap%
\pgfsetmiterjoin%
\definecolor{currentfill}{rgb}{1.000000,1.000000,1.000000}%
\pgfsetfillcolor{currentfill}%
\pgfsetlinewidth{0.000000pt}%
\definecolor{currentstroke}{rgb}{0.000000,0.000000,0.000000}%
\pgfsetstrokecolor{currentstroke}%
\pgfsetstrokeopacity{0.000000}%
\pgfsetdash{}{0pt}%
\pgfpathmoveto{\pgfqpoint{3.347347in}{6.967719in}}%
\pgfpathlineto{\pgfqpoint{5.245306in}{6.967719in}}%
\pgfpathlineto{\pgfqpoint{5.245306in}{8.340446in}}%
\pgfpathlineto{\pgfqpoint{3.347347in}{8.340446in}}%
\pgfpathclose%
\pgfusepath{fill}%
\end{pgfscope}%
\begin{pgfscope}%
\pgfsetbuttcap%
\pgfsetroundjoin%
\definecolor{currentfill}{rgb}{0.000000,0.000000,0.000000}%
\pgfsetfillcolor{currentfill}%
\pgfsetlinewidth{0.803000pt}%
\definecolor{currentstroke}{rgb}{0.000000,0.000000,0.000000}%
\pgfsetstrokecolor{currentstroke}%
\pgfsetdash{}{0pt}%
\pgfsys@defobject{currentmarker}{\pgfqpoint{0.000000in}{-0.048611in}}{\pgfqpoint{0.000000in}{0.000000in}}{%
\pgfpathmoveto{\pgfqpoint{0.000000in}{0.000000in}}%
\pgfpathlineto{\pgfqpoint{0.000000in}{-0.048611in}}%
\pgfusepath{stroke,fill}%
}%
\begin{pgfscope}%
\pgfsys@transformshift{3.649295in}{6.967719in}%
\pgfsys@useobject{currentmarker}{}%
\end{pgfscope}%
\end{pgfscope}%
\begin{pgfscope}%
\pgftext[x=3.649295in,y=6.870496in,,top]{\rmfamily\fontsize{10.000000}{12.000000}\selectfont \(\displaystyle 0.10\)}%
\end{pgfscope}%
\begin{pgfscope}%
\pgfsetbuttcap%
\pgfsetroundjoin%
\definecolor{currentfill}{rgb}{0.000000,0.000000,0.000000}%
\pgfsetfillcolor{currentfill}%
\pgfsetlinewidth{0.803000pt}%
\definecolor{currentstroke}{rgb}{0.000000,0.000000,0.000000}%
\pgfsetstrokecolor{currentstroke}%
\pgfsetdash{}{0pt}%
\pgfsys@defobject{currentmarker}{\pgfqpoint{0.000000in}{-0.048611in}}{\pgfqpoint{0.000000in}{0.000000in}}{%
\pgfpathmoveto{\pgfqpoint{0.000000in}{0.000000in}}%
\pgfpathlineto{\pgfqpoint{0.000000in}{-0.048611in}}%
\pgfusepath{stroke,fill}%
}%
\begin{pgfscope}%
\pgfsys@transformshift{4.166920in}{6.967719in}%
\pgfsys@useobject{currentmarker}{}%
\end{pgfscope}%
\end{pgfscope}%
\begin{pgfscope}%
\pgftext[x=4.166920in,y=6.870496in,,top]{\rmfamily\fontsize{10.000000}{12.000000}\selectfont \(\displaystyle 0.12\)}%
\end{pgfscope}%
\begin{pgfscope}%
\pgfsetbuttcap%
\pgfsetroundjoin%
\definecolor{currentfill}{rgb}{0.000000,0.000000,0.000000}%
\pgfsetfillcolor{currentfill}%
\pgfsetlinewidth{0.803000pt}%
\definecolor{currentstroke}{rgb}{0.000000,0.000000,0.000000}%
\pgfsetstrokecolor{currentstroke}%
\pgfsetdash{}{0pt}%
\pgfsys@defobject{currentmarker}{\pgfqpoint{0.000000in}{-0.048611in}}{\pgfqpoint{0.000000in}{0.000000in}}{%
\pgfpathmoveto{\pgfqpoint{0.000000in}{0.000000in}}%
\pgfpathlineto{\pgfqpoint{0.000000in}{-0.048611in}}%
\pgfusepath{stroke,fill}%
}%
\begin{pgfscope}%
\pgfsys@transformshift{4.684545in}{6.967719in}%
\pgfsys@useobject{currentmarker}{}%
\end{pgfscope}%
\end{pgfscope}%
\begin{pgfscope}%
\pgftext[x=4.684545in,y=6.870496in,,top]{\rmfamily\fontsize{10.000000}{12.000000}\selectfont \(\displaystyle 0.14\)}%
\end{pgfscope}%
\begin{pgfscope}%
\pgfsetbuttcap%
\pgfsetroundjoin%
\definecolor{currentfill}{rgb}{0.000000,0.000000,0.000000}%
\pgfsetfillcolor{currentfill}%
\pgfsetlinewidth{0.803000pt}%
\definecolor{currentstroke}{rgb}{0.000000,0.000000,0.000000}%
\pgfsetstrokecolor{currentstroke}%
\pgfsetdash{}{0pt}%
\pgfsys@defobject{currentmarker}{\pgfqpoint{0.000000in}{-0.048611in}}{\pgfqpoint{0.000000in}{0.000000in}}{%
\pgfpathmoveto{\pgfqpoint{0.000000in}{0.000000in}}%
\pgfpathlineto{\pgfqpoint{0.000000in}{-0.048611in}}%
\pgfusepath{stroke,fill}%
}%
\begin{pgfscope}%
\pgfsys@transformshift{5.202171in}{6.967719in}%
\pgfsys@useobject{currentmarker}{}%
\end{pgfscope}%
\end{pgfscope}%
\begin{pgfscope}%
\pgftext[x=5.202171in,y=6.870496in,,top]{\rmfamily\fontsize{10.000000}{12.000000}\selectfont \(\displaystyle 0.16\)}%
\end{pgfscope}%
\begin{pgfscope}%
\pgfsetbuttcap%
\pgfsetroundjoin%
\definecolor{currentfill}{rgb}{0.000000,0.000000,0.000000}%
\pgfsetfillcolor{currentfill}%
\pgfsetlinewidth{0.803000pt}%
\definecolor{currentstroke}{rgb}{0.000000,0.000000,0.000000}%
\pgfsetstrokecolor{currentstroke}%
\pgfsetdash{}{0pt}%
\pgfsys@defobject{currentmarker}{\pgfqpoint{-0.048611in}{0.000000in}}{\pgfqpoint{0.000000in}{0.000000in}}{%
\pgfpathmoveto{\pgfqpoint{0.000000in}{0.000000in}}%
\pgfpathlineto{\pgfqpoint{-0.048611in}{0.000000in}}%
\pgfusepath{stroke,fill}%
}%
\begin{pgfscope}%
\pgfsys@transformshift{3.347347in}{7.221568in}%
\pgfsys@useobject{currentmarker}{}%
\end{pgfscope}%
\end{pgfscope}%
\begin{pgfscope}%
\pgftext[x=2.962122in,y=7.168806in,left,base]{\rmfamily\fontsize{10.000000}{12.000000}\selectfont \(\displaystyle 10^{-7}\)}%
\end{pgfscope}%
\begin{pgfscope}%
\pgfsetbuttcap%
\pgfsetroundjoin%
\definecolor{currentfill}{rgb}{0.000000,0.000000,0.000000}%
\pgfsetfillcolor{currentfill}%
\pgfsetlinewidth{0.803000pt}%
\definecolor{currentstroke}{rgb}{0.000000,0.000000,0.000000}%
\pgfsetstrokecolor{currentstroke}%
\pgfsetdash{}{0pt}%
\pgfsys@defobject{currentmarker}{\pgfqpoint{-0.048611in}{0.000000in}}{\pgfqpoint{0.000000in}{0.000000in}}{%
\pgfpathmoveto{\pgfqpoint{0.000000in}{0.000000in}}%
\pgfpathlineto{\pgfqpoint{-0.048611in}{0.000000in}}%
\pgfusepath{stroke,fill}%
}%
\begin{pgfscope}%
\pgfsys@transformshift{3.347347in}{7.687308in}%
\pgfsys@useobject{currentmarker}{}%
\end{pgfscope}%
\end{pgfscope}%
\begin{pgfscope}%
\pgftext[x=2.962122in,y=7.634546in,left,base]{\rmfamily\fontsize{10.000000}{12.000000}\selectfont \(\displaystyle 10^{-6}\)}%
\end{pgfscope}%
\begin{pgfscope}%
\pgfsetbuttcap%
\pgfsetroundjoin%
\definecolor{currentfill}{rgb}{0.000000,0.000000,0.000000}%
\pgfsetfillcolor{currentfill}%
\pgfsetlinewidth{0.803000pt}%
\definecolor{currentstroke}{rgb}{0.000000,0.000000,0.000000}%
\pgfsetstrokecolor{currentstroke}%
\pgfsetdash{}{0pt}%
\pgfsys@defobject{currentmarker}{\pgfqpoint{-0.048611in}{0.000000in}}{\pgfqpoint{0.000000in}{0.000000in}}{%
\pgfpathmoveto{\pgfqpoint{0.000000in}{0.000000in}}%
\pgfpathlineto{\pgfqpoint{-0.048611in}{0.000000in}}%
\pgfusepath{stroke,fill}%
}%
\begin{pgfscope}%
\pgfsys@transformshift{3.347347in}{8.153048in}%
\pgfsys@useobject{currentmarker}{}%
\end{pgfscope}%
\end{pgfscope}%
\begin{pgfscope}%
\pgftext[x=2.962122in,y=8.100286in,left,base]{\rmfamily\fontsize{10.000000}{12.000000}\selectfont \(\displaystyle 10^{-5}\)}%
\end{pgfscope}%
\begin{pgfscope}%
\pgfsetbuttcap%
\pgfsetroundjoin%
\definecolor{currentfill}{rgb}{0.000000,0.000000,0.000000}%
\pgfsetfillcolor{currentfill}%
\pgfsetlinewidth{0.602250pt}%
\definecolor{currentstroke}{rgb}{0.000000,0.000000,0.000000}%
\pgfsetstrokecolor{currentstroke}%
\pgfsetdash{}{0pt}%
\pgfsys@defobject{currentmarker}{\pgfqpoint{-0.027778in}{0.000000in}}{\pgfqpoint{0.000000in}{0.000000in}}{%
\pgfpathmoveto{\pgfqpoint{0.000000in}{0.000000in}}%
\pgfpathlineto{\pgfqpoint{-0.027778in}{0.000000in}}%
\pgfusepath{stroke,fill}%
}%
\begin{pgfscope}%
\pgfsys@transformshift{3.347347in}{6.978042in}%
\pgfsys@useobject{currentmarker}{}%
\end{pgfscope}%
\end{pgfscope}%
\begin{pgfscope}%
\pgfsetbuttcap%
\pgfsetroundjoin%
\definecolor{currentfill}{rgb}{0.000000,0.000000,0.000000}%
\pgfsetfillcolor{currentfill}%
\pgfsetlinewidth{0.602250pt}%
\definecolor{currentstroke}{rgb}{0.000000,0.000000,0.000000}%
\pgfsetstrokecolor{currentstroke}%
\pgfsetdash{}{0pt}%
\pgfsys@defobject{currentmarker}{\pgfqpoint{-0.027778in}{0.000000in}}{\pgfqpoint{0.000000in}{0.000000in}}{%
\pgfpathmoveto{\pgfqpoint{0.000000in}{0.000000in}}%
\pgfpathlineto{\pgfqpoint{-0.027778in}{0.000000in}}%
\pgfusepath{stroke,fill}%
}%
\begin{pgfscope}%
\pgfsys@transformshift{3.347347in}{7.036231in}%
\pgfsys@useobject{currentmarker}{}%
\end{pgfscope}%
\end{pgfscope}%
\begin{pgfscope}%
\pgfsetbuttcap%
\pgfsetroundjoin%
\definecolor{currentfill}{rgb}{0.000000,0.000000,0.000000}%
\pgfsetfillcolor{currentfill}%
\pgfsetlinewidth{0.602250pt}%
\definecolor{currentstroke}{rgb}{0.000000,0.000000,0.000000}%
\pgfsetstrokecolor{currentstroke}%
\pgfsetdash{}{0pt}%
\pgfsys@defobject{currentmarker}{\pgfqpoint{-0.027778in}{0.000000in}}{\pgfqpoint{0.000000in}{0.000000in}}{%
\pgfpathmoveto{\pgfqpoint{0.000000in}{0.000000in}}%
\pgfpathlineto{\pgfqpoint{-0.027778in}{0.000000in}}%
\pgfusepath{stroke,fill}%
}%
\begin{pgfscope}%
\pgfsys@transformshift{3.347347in}{7.081366in}%
\pgfsys@useobject{currentmarker}{}%
\end{pgfscope}%
\end{pgfscope}%
\begin{pgfscope}%
\pgfsetbuttcap%
\pgfsetroundjoin%
\definecolor{currentfill}{rgb}{0.000000,0.000000,0.000000}%
\pgfsetfillcolor{currentfill}%
\pgfsetlinewidth{0.602250pt}%
\definecolor{currentstroke}{rgb}{0.000000,0.000000,0.000000}%
\pgfsetstrokecolor{currentstroke}%
\pgfsetdash{}{0pt}%
\pgfsys@defobject{currentmarker}{\pgfqpoint{-0.027778in}{0.000000in}}{\pgfqpoint{0.000000in}{0.000000in}}{%
\pgfpathmoveto{\pgfqpoint{0.000000in}{0.000000in}}%
\pgfpathlineto{\pgfqpoint{-0.027778in}{0.000000in}}%
\pgfusepath{stroke,fill}%
}%
\begin{pgfscope}%
\pgfsys@transformshift{3.347347in}{7.118244in}%
\pgfsys@useobject{currentmarker}{}%
\end{pgfscope}%
\end{pgfscope}%
\begin{pgfscope}%
\pgfsetbuttcap%
\pgfsetroundjoin%
\definecolor{currentfill}{rgb}{0.000000,0.000000,0.000000}%
\pgfsetfillcolor{currentfill}%
\pgfsetlinewidth{0.602250pt}%
\definecolor{currentstroke}{rgb}{0.000000,0.000000,0.000000}%
\pgfsetstrokecolor{currentstroke}%
\pgfsetdash{}{0pt}%
\pgfsys@defobject{currentmarker}{\pgfqpoint{-0.027778in}{0.000000in}}{\pgfqpoint{0.000000in}{0.000000in}}{%
\pgfpathmoveto{\pgfqpoint{0.000000in}{0.000000in}}%
\pgfpathlineto{\pgfqpoint{-0.027778in}{0.000000in}}%
\pgfusepath{stroke,fill}%
}%
\begin{pgfscope}%
\pgfsys@transformshift{3.347347in}{7.149424in}%
\pgfsys@useobject{currentmarker}{}%
\end{pgfscope}%
\end{pgfscope}%
\begin{pgfscope}%
\pgfsetbuttcap%
\pgfsetroundjoin%
\definecolor{currentfill}{rgb}{0.000000,0.000000,0.000000}%
\pgfsetfillcolor{currentfill}%
\pgfsetlinewidth{0.602250pt}%
\definecolor{currentstroke}{rgb}{0.000000,0.000000,0.000000}%
\pgfsetstrokecolor{currentstroke}%
\pgfsetdash{}{0pt}%
\pgfsys@defobject{currentmarker}{\pgfqpoint{-0.027778in}{0.000000in}}{\pgfqpoint{0.000000in}{0.000000in}}{%
\pgfpathmoveto{\pgfqpoint{0.000000in}{0.000000in}}%
\pgfpathlineto{\pgfqpoint{-0.027778in}{0.000000in}}%
\pgfusepath{stroke,fill}%
}%
\begin{pgfscope}%
\pgfsys@transformshift{3.347347in}{7.176433in}%
\pgfsys@useobject{currentmarker}{}%
\end{pgfscope}%
\end{pgfscope}%
\begin{pgfscope}%
\pgfsetbuttcap%
\pgfsetroundjoin%
\definecolor{currentfill}{rgb}{0.000000,0.000000,0.000000}%
\pgfsetfillcolor{currentfill}%
\pgfsetlinewidth{0.602250pt}%
\definecolor{currentstroke}{rgb}{0.000000,0.000000,0.000000}%
\pgfsetstrokecolor{currentstroke}%
\pgfsetdash{}{0pt}%
\pgfsys@defobject{currentmarker}{\pgfqpoint{-0.027778in}{0.000000in}}{\pgfqpoint{0.000000in}{0.000000in}}{%
\pgfpathmoveto{\pgfqpoint{0.000000in}{0.000000in}}%
\pgfpathlineto{\pgfqpoint{-0.027778in}{0.000000in}}%
\pgfusepath{stroke,fill}%
}%
\begin{pgfscope}%
\pgfsys@transformshift{3.347347in}{7.200257in}%
\pgfsys@useobject{currentmarker}{}%
\end{pgfscope}%
\end{pgfscope}%
\begin{pgfscope}%
\pgfsetbuttcap%
\pgfsetroundjoin%
\definecolor{currentfill}{rgb}{0.000000,0.000000,0.000000}%
\pgfsetfillcolor{currentfill}%
\pgfsetlinewidth{0.602250pt}%
\definecolor{currentstroke}{rgb}{0.000000,0.000000,0.000000}%
\pgfsetstrokecolor{currentstroke}%
\pgfsetdash{}{0pt}%
\pgfsys@defobject{currentmarker}{\pgfqpoint{-0.027778in}{0.000000in}}{\pgfqpoint{0.000000in}{0.000000in}}{%
\pgfpathmoveto{\pgfqpoint{0.000000in}{0.000000in}}%
\pgfpathlineto{\pgfqpoint{-0.027778in}{0.000000in}}%
\pgfusepath{stroke,fill}%
}%
\begin{pgfscope}%
\pgfsys@transformshift{3.347347in}{7.361769in}%
\pgfsys@useobject{currentmarker}{}%
\end{pgfscope}%
\end{pgfscope}%
\begin{pgfscope}%
\pgfsetbuttcap%
\pgfsetroundjoin%
\definecolor{currentfill}{rgb}{0.000000,0.000000,0.000000}%
\pgfsetfillcolor{currentfill}%
\pgfsetlinewidth{0.602250pt}%
\definecolor{currentstroke}{rgb}{0.000000,0.000000,0.000000}%
\pgfsetstrokecolor{currentstroke}%
\pgfsetdash{}{0pt}%
\pgfsys@defobject{currentmarker}{\pgfqpoint{-0.027778in}{0.000000in}}{\pgfqpoint{0.000000in}{0.000000in}}{%
\pgfpathmoveto{\pgfqpoint{0.000000in}{0.000000in}}%
\pgfpathlineto{\pgfqpoint{-0.027778in}{0.000000in}}%
\pgfusepath{stroke,fill}%
}%
\begin{pgfscope}%
\pgfsys@transformshift{3.347347in}{7.443782in}%
\pgfsys@useobject{currentmarker}{}%
\end{pgfscope}%
\end{pgfscope}%
\begin{pgfscope}%
\pgfsetbuttcap%
\pgfsetroundjoin%
\definecolor{currentfill}{rgb}{0.000000,0.000000,0.000000}%
\pgfsetfillcolor{currentfill}%
\pgfsetlinewidth{0.602250pt}%
\definecolor{currentstroke}{rgb}{0.000000,0.000000,0.000000}%
\pgfsetstrokecolor{currentstroke}%
\pgfsetdash{}{0pt}%
\pgfsys@defobject{currentmarker}{\pgfqpoint{-0.027778in}{0.000000in}}{\pgfqpoint{0.000000in}{0.000000in}}{%
\pgfpathmoveto{\pgfqpoint{0.000000in}{0.000000in}}%
\pgfpathlineto{\pgfqpoint{-0.027778in}{0.000000in}}%
\pgfusepath{stroke,fill}%
}%
\begin{pgfscope}%
\pgfsys@transformshift{3.347347in}{7.501971in}%
\pgfsys@useobject{currentmarker}{}%
\end{pgfscope}%
\end{pgfscope}%
\begin{pgfscope}%
\pgfsetbuttcap%
\pgfsetroundjoin%
\definecolor{currentfill}{rgb}{0.000000,0.000000,0.000000}%
\pgfsetfillcolor{currentfill}%
\pgfsetlinewidth{0.602250pt}%
\definecolor{currentstroke}{rgb}{0.000000,0.000000,0.000000}%
\pgfsetstrokecolor{currentstroke}%
\pgfsetdash{}{0pt}%
\pgfsys@defobject{currentmarker}{\pgfqpoint{-0.027778in}{0.000000in}}{\pgfqpoint{0.000000in}{0.000000in}}{%
\pgfpathmoveto{\pgfqpoint{0.000000in}{0.000000in}}%
\pgfpathlineto{\pgfqpoint{-0.027778in}{0.000000in}}%
\pgfusepath{stroke,fill}%
}%
\begin{pgfscope}%
\pgfsys@transformshift{3.347347in}{7.547106in}%
\pgfsys@useobject{currentmarker}{}%
\end{pgfscope}%
\end{pgfscope}%
\begin{pgfscope}%
\pgfsetbuttcap%
\pgfsetroundjoin%
\definecolor{currentfill}{rgb}{0.000000,0.000000,0.000000}%
\pgfsetfillcolor{currentfill}%
\pgfsetlinewidth{0.602250pt}%
\definecolor{currentstroke}{rgb}{0.000000,0.000000,0.000000}%
\pgfsetstrokecolor{currentstroke}%
\pgfsetdash{}{0pt}%
\pgfsys@defobject{currentmarker}{\pgfqpoint{-0.027778in}{0.000000in}}{\pgfqpoint{0.000000in}{0.000000in}}{%
\pgfpathmoveto{\pgfqpoint{0.000000in}{0.000000in}}%
\pgfpathlineto{\pgfqpoint{-0.027778in}{0.000000in}}%
\pgfusepath{stroke,fill}%
}%
\begin{pgfscope}%
\pgfsys@transformshift{3.347347in}{7.583984in}%
\pgfsys@useobject{currentmarker}{}%
\end{pgfscope}%
\end{pgfscope}%
\begin{pgfscope}%
\pgfsetbuttcap%
\pgfsetroundjoin%
\definecolor{currentfill}{rgb}{0.000000,0.000000,0.000000}%
\pgfsetfillcolor{currentfill}%
\pgfsetlinewidth{0.602250pt}%
\definecolor{currentstroke}{rgb}{0.000000,0.000000,0.000000}%
\pgfsetstrokecolor{currentstroke}%
\pgfsetdash{}{0pt}%
\pgfsys@defobject{currentmarker}{\pgfqpoint{-0.027778in}{0.000000in}}{\pgfqpoint{0.000000in}{0.000000in}}{%
\pgfpathmoveto{\pgfqpoint{0.000000in}{0.000000in}}%
\pgfpathlineto{\pgfqpoint{-0.027778in}{0.000000in}}%
\pgfusepath{stroke,fill}%
}%
\begin{pgfscope}%
\pgfsys@transformshift{3.347347in}{7.615164in}%
\pgfsys@useobject{currentmarker}{}%
\end{pgfscope}%
\end{pgfscope}%
\begin{pgfscope}%
\pgfsetbuttcap%
\pgfsetroundjoin%
\definecolor{currentfill}{rgb}{0.000000,0.000000,0.000000}%
\pgfsetfillcolor{currentfill}%
\pgfsetlinewidth{0.602250pt}%
\definecolor{currentstroke}{rgb}{0.000000,0.000000,0.000000}%
\pgfsetstrokecolor{currentstroke}%
\pgfsetdash{}{0pt}%
\pgfsys@defobject{currentmarker}{\pgfqpoint{-0.027778in}{0.000000in}}{\pgfqpoint{0.000000in}{0.000000in}}{%
\pgfpathmoveto{\pgfqpoint{0.000000in}{0.000000in}}%
\pgfpathlineto{\pgfqpoint{-0.027778in}{0.000000in}}%
\pgfusepath{stroke,fill}%
}%
\begin{pgfscope}%
\pgfsys@transformshift{3.347347in}{7.642173in}%
\pgfsys@useobject{currentmarker}{}%
\end{pgfscope}%
\end{pgfscope}%
\begin{pgfscope}%
\pgfsetbuttcap%
\pgfsetroundjoin%
\definecolor{currentfill}{rgb}{0.000000,0.000000,0.000000}%
\pgfsetfillcolor{currentfill}%
\pgfsetlinewidth{0.602250pt}%
\definecolor{currentstroke}{rgb}{0.000000,0.000000,0.000000}%
\pgfsetstrokecolor{currentstroke}%
\pgfsetdash{}{0pt}%
\pgfsys@defobject{currentmarker}{\pgfqpoint{-0.027778in}{0.000000in}}{\pgfqpoint{0.000000in}{0.000000in}}{%
\pgfpathmoveto{\pgfqpoint{0.000000in}{0.000000in}}%
\pgfpathlineto{\pgfqpoint{-0.027778in}{0.000000in}}%
\pgfusepath{stroke,fill}%
}%
\begin{pgfscope}%
\pgfsys@transformshift{3.347347in}{7.665997in}%
\pgfsys@useobject{currentmarker}{}%
\end{pgfscope}%
\end{pgfscope}%
\begin{pgfscope}%
\pgfsetbuttcap%
\pgfsetroundjoin%
\definecolor{currentfill}{rgb}{0.000000,0.000000,0.000000}%
\pgfsetfillcolor{currentfill}%
\pgfsetlinewidth{0.602250pt}%
\definecolor{currentstroke}{rgb}{0.000000,0.000000,0.000000}%
\pgfsetstrokecolor{currentstroke}%
\pgfsetdash{}{0pt}%
\pgfsys@defobject{currentmarker}{\pgfqpoint{-0.027778in}{0.000000in}}{\pgfqpoint{0.000000in}{0.000000in}}{%
\pgfpathmoveto{\pgfqpoint{0.000000in}{0.000000in}}%
\pgfpathlineto{\pgfqpoint{-0.027778in}{0.000000in}}%
\pgfusepath{stroke,fill}%
}%
\begin{pgfscope}%
\pgfsys@transformshift{3.347347in}{7.827509in}%
\pgfsys@useobject{currentmarker}{}%
\end{pgfscope}%
\end{pgfscope}%
\begin{pgfscope}%
\pgfsetbuttcap%
\pgfsetroundjoin%
\definecolor{currentfill}{rgb}{0.000000,0.000000,0.000000}%
\pgfsetfillcolor{currentfill}%
\pgfsetlinewidth{0.602250pt}%
\definecolor{currentstroke}{rgb}{0.000000,0.000000,0.000000}%
\pgfsetstrokecolor{currentstroke}%
\pgfsetdash{}{0pt}%
\pgfsys@defobject{currentmarker}{\pgfqpoint{-0.027778in}{0.000000in}}{\pgfqpoint{0.000000in}{0.000000in}}{%
\pgfpathmoveto{\pgfqpoint{0.000000in}{0.000000in}}%
\pgfpathlineto{\pgfqpoint{-0.027778in}{0.000000in}}%
\pgfusepath{stroke,fill}%
}%
\begin{pgfscope}%
\pgfsys@transformshift{3.347347in}{7.909522in}%
\pgfsys@useobject{currentmarker}{}%
\end{pgfscope}%
\end{pgfscope}%
\begin{pgfscope}%
\pgfsetbuttcap%
\pgfsetroundjoin%
\definecolor{currentfill}{rgb}{0.000000,0.000000,0.000000}%
\pgfsetfillcolor{currentfill}%
\pgfsetlinewidth{0.602250pt}%
\definecolor{currentstroke}{rgb}{0.000000,0.000000,0.000000}%
\pgfsetstrokecolor{currentstroke}%
\pgfsetdash{}{0pt}%
\pgfsys@defobject{currentmarker}{\pgfqpoint{-0.027778in}{0.000000in}}{\pgfqpoint{0.000000in}{0.000000in}}{%
\pgfpathmoveto{\pgfqpoint{0.000000in}{0.000000in}}%
\pgfpathlineto{\pgfqpoint{-0.027778in}{0.000000in}}%
\pgfusepath{stroke,fill}%
}%
\begin{pgfscope}%
\pgfsys@transformshift{3.347347in}{7.967711in}%
\pgfsys@useobject{currentmarker}{}%
\end{pgfscope}%
\end{pgfscope}%
\begin{pgfscope}%
\pgfsetbuttcap%
\pgfsetroundjoin%
\definecolor{currentfill}{rgb}{0.000000,0.000000,0.000000}%
\pgfsetfillcolor{currentfill}%
\pgfsetlinewidth{0.602250pt}%
\definecolor{currentstroke}{rgb}{0.000000,0.000000,0.000000}%
\pgfsetstrokecolor{currentstroke}%
\pgfsetdash{}{0pt}%
\pgfsys@defobject{currentmarker}{\pgfqpoint{-0.027778in}{0.000000in}}{\pgfqpoint{0.000000in}{0.000000in}}{%
\pgfpathmoveto{\pgfqpoint{0.000000in}{0.000000in}}%
\pgfpathlineto{\pgfqpoint{-0.027778in}{0.000000in}}%
\pgfusepath{stroke,fill}%
}%
\begin{pgfscope}%
\pgfsys@transformshift{3.347347in}{8.012846in}%
\pgfsys@useobject{currentmarker}{}%
\end{pgfscope}%
\end{pgfscope}%
\begin{pgfscope}%
\pgfsetbuttcap%
\pgfsetroundjoin%
\definecolor{currentfill}{rgb}{0.000000,0.000000,0.000000}%
\pgfsetfillcolor{currentfill}%
\pgfsetlinewidth{0.602250pt}%
\definecolor{currentstroke}{rgb}{0.000000,0.000000,0.000000}%
\pgfsetstrokecolor{currentstroke}%
\pgfsetdash{}{0pt}%
\pgfsys@defobject{currentmarker}{\pgfqpoint{-0.027778in}{0.000000in}}{\pgfqpoint{0.000000in}{0.000000in}}{%
\pgfpathmoveto{\pgfqpoint{0.000000in}{0.000000in}}%
\pgfpathlineto{\pgfqpoint{-0.027778in}{0.000000in}}%
\pgfusepath{stroke,fill}%
}%
\begin{pgfscope}%
\pgfsys@transformshift{3.347347in}{8.049724in}%
\pgfsys@useobject{currentmarker}{}%
\end{pgfscope}%
\end{pgfscope}%
\begin{pgfscope}%
\pgfsetbuttcap%
\pgfsetroundjoin%
\definecolor{currentfill}{rgb}{0.000000,0.000000,0.000000}%
\pgfsetfillcolor{currentfill}%
\pgfsetlinewidth{0.602250pt}%
\definecolor{currentstroke}{rgb}{0.000000,0.000000,0.000000}%
\pgfsetstrokecolor{currentstroke}%
\pgfsetdash{}{0pt}%
\pgfsys@defobject{currentmarker}{\pgfqpoint{-0.027778in}{0.000000in}}{\pgfqpoint{0.000000in}{0.000000in}}{%
\pgfpathmoveto{\pgfqpoint{0.000000in}{0.000000in}}%
\pgfpathlineto{\pgfqpoint{-0.027778in}{0.000000in}}%
\pgfusepath{stroke,fill}%
}%
\begin{pgfscope}%
\pgfsys@transformshift{3.347347in}{8.080904in}%
\pgfsys@useobject{currentmarker}{}%
\end{pgfscope}%
\end{pgfscope}%
\begin{pgfscope}%
\pgfsetbuttcap%
\pgfsetroundjoin%
\definecolor{currentfill}{rgb}{0.000000,0.000000,0.000000}%
\pgfsetfillcolor{currentfill}%
\pgfsetlinewidth{0.602250pt}%
\definecolor{currentstroke}{rgb}{0.000000,0.000000,0.000000}%
\pgfsetstrokecolor{currentstroke}%
\pgfsetdash{}{0pt}%
\pgfsys@defobject{currentmarker}{\pgfqpoint{-0.027778in}{0.000000in}}{\pgfqpoint{0.000000in}{0.000000in}}{%
\pgfpathmoveto{\pgfqpoint{0.000000in}{0.000000in}}%
\pgfpathlineto{\pgfqpoint{-0.027778in}{0.000000in}}%
\pgfusepath{stroke,fill}%
}%
\begin{pgfscope}%
\pgfsys@transformshift{3.347347in}{8.107913in}%
\pgfsys@useobject{currentmarker}{}%
\end{pgfscope}%
\end{pgfscope}%
\begin{pgfscope}%
\pgfsetbuttcap%
\pgfsetroundjoin%
\definecolor{currentfill}{rgb}{0.000000,0.000000,0.000000}%
\pgfsetfillcolor{currentfill}%
\pgfsetlinewidth{0.602250pt}%
\definecolor{currentstroke}{rgb}{0.000000,0.000000,0.000000}%
\pgfsetstrokecolor{currentstroke}%
\pgfsetdash{}{0pt}%
\pgfsys@defobject{currentmarker}{\pgfqpoint{-0.027778in}{0.000000in}}{\pgfqpoint{0.000000in}{0.000000in}}{%
\pgfpathmoveto{\pgfqpoint{0.000000in}{0.000000in}}%
\pgfpathlineto{\pgfqpoint{-0.027778in}{0.000000in}}%
\pgfusepath{stroke,fill}%
}%
\begin{pgfscope}%
\pgfsys@transformshift{3.347347in}{8.131737in}%
\pgfsys@useobject{currentmarker}{}%
\end{pgfscope}%
\end{pgfscope}%
\begin{pgfscope}%
\pgfsetbuttcap%
\pgfsetroundjoin%
\definecolor{currentfill}{rgb}{0.000000,0.000000,0.000000}%
\pgfsetfillcolor{currentfill}%
\pgfsetlinewidth{0.602250pt}%
\definecolor{currentstroke}{rgb}{0.000000,0.000000,0.000000}%
\pgfsetstrokecolor{currentstroke}%
\pgfsetdash{}{0pt}%
\pgfsys@defobject{currentmarker}{\pgfqpoint{-0.027778in}{0.000000in}}{\pgfqpoint{0.000000in}{0.000000in}}{%
\pgfpathmoveto{\pgfqpoint{0.000000in}{0.000000in}}%
\pgfpathlineto{\pgfqpoint{-0.027778in}{0.000000in}}%
\pgfusepath{stroke,fill}%
}%
\begin{pgfscope}%
\pgfsys@transformshift{3.347347in}{8.293250in}%
\pgfsys@useobject{currentmarker}{}%
\end{pgfscope}%
\end{pgfscope}%
\begin{pgfscope}%
\pgfpathrectangle{\pgfqpoint{3.347347in}{6.967719in}}{\pgfqpoint{1.897959in}{1.372727in}} %
\pgfusepath{clip}%
\pgfsetbuttcap%
\pgfsetroundjoin%
\pgfsetlinewidth{1.505625pt}%
\definecolor{currentstroke}{rgb}{1.000000,0.000000,0.000000}%
\pgfsetstrokecolor{currentstroke}%
\pgfsetdash{{5.550000pt}{2.400000pt}}{0.000000pt}%
\pgfpathmoveto{\pgfqpoint{3.433618in}{8.151508in}}%
\pgfpathlineto{\pgfqpoint{3.649295in}{8.150283in}}%
\pgfpathlineto{\pgfqpoint{3.864972in}{8.149170in}}%
\pgfpathlineto{\pgfqpoint{4.080649in}{8.148166in}}%
\pgfpathlineto{\pgfqpoint{4.296327in}{8.147271in}}%
\pgfpathlineto{\pgfqpoint{4.512004in}{8.146482in}}%
\pgfpathlineto{\pgfqpoint{4.727681in}{8.145796in}}%
\pgfpathlineto{\pgfqpoint{4.943358in}{8.145214in}}%
\pgfpathlineto{\pgfqpoint{5.159035in}{8.144735in}}%
\pgfusepath{stroke}%
\end{pgfscope}%
\begin{pgfscope}%
\pgfpathrectangle{\pgfqpoint{3.347347in}{6.967719in}}{\pgfqpoint{1.897959in}{1.372727in}} %
\pgfusepath{clip}%
\pgfsetbuttcap%
\pgfsetmiterjoin%
\definecolor{currentfill}{rgb}{1.000000,0.000000,0.000000}%
\pgfsetfillcolor{currentfill}%
\pgfsetlinewidth{1.003750pt}%
\definecolor{currentstroke}{rgb}{1.000000,0.000000,0.000000}%
\pgfsetstrokecolor{currentstroke}%
\pgfsetdash{}{0pt}%
\pgfsys@defobject{currentmarker}{\pgfqpoint{-0.041667in}{-0.041667in}}{\pgfqpoint{0.041667in}{0.041667in}}{%
\pgfpathmoveto{\pgfqpoint{-0.041667in}{-0.041667in}}%
\pgfpathlineto{\pgfqpoint{0.041667in}{-0.041667in}}%
\pgfpathlineto{\pgfqpoint{0.041667in}{0.041667in}}%
\pgfpathlineto{\pgfqpoint{-0.041667in}{0.041667in}}%
\pgfpathclose%
\pgfusepath{stroke,fill}%
}%
\begin{pgfscope}%
\pgfsys@transformshift{3.433618in}{8.151508in}%
\pgfsys@useobject{currentmarker}{}%
\end{pgfscope}%
\begin{pgfscope}%
\pgfsys@transformshift{3.864972in}{8.149170in}%
\pgfsys@useobject{currentmarker}{}%
\end{pgfscope}%
\begin{pgfscope}%
\pgfsys@transformshift{4.296327in}{8.147271in}%
\pgfsys@useobject{currentmarker}{}%
\end{pgfscope}%
\begin{pgfscope}%
\pgfsys@transformshift{4.727681in}{8.145796in}%
\pgfsys@useobject{currentmarker}{}%
\end{pgfscope}%
\begin{pgfscope}%
\pgfsys@transformshift{5.159035in}{8.144735in}%
\pgfsys@useobject{currentmarker}{}%
\end{pgfscope}%
\end{pgfscope}%
\begin{pgfscope}%
\pgfpathrectangle{\pgfqpoint{3.347347in}{6.967719in}}{\pgfqpoint{1.897959in}{1.372727in}} %
\pgfusepath{clip}%
\pgfsetrectcap%
\pgfsetroundjoin%
\pgfsetlinewidth{1.505625pt}%
\definecolor{currentstroke}{rgb}{0.000000,0.000000,1.000000}%
\pgfsetstrokecolor{currentstroke}%
\pgfsetdash{}{0pt}%
\pgfpathmoveto{\pgfqpoint{3.433618in}{7.298527in}}%
\pgfpathlineto{\pgfqpoint{3.649295in}{7.274379in}}%
\pgfpathlineto{\pgfqpoint{3.864972in}{7.248840in}}%
\pgfpathlineto{\pgfqpoint{4.080649in}{7.221496in}}%
\pgfpathlineto{\pgfqpoint{4.296327in}{7.191822in}}%
\pgfpathlineto{\pgfqpoint{4.512004in}{7.159119in}}%
\pgfpathlineto{\pgfqpoint{4.727681in}{7.122400in}}%
\pgfpathlineto{\pgfqpoint{4.943358in}{7.080193in}}%
\pgfpathlineto{\pgfqpoint{5.159035in}{7.030115in}}%
\pgfusepath{stroke}%
\end{pgfscope}%
\begin{pgfscope}%
\pgfpathrectangle{\pgfqpoint{3.347347in}{6.967719in}}{\pgfqpoint{1.897959in}{1.372727in}} %
\pgfusepath{clip}%
\pgfsetbuttcap%
\pgfsetroundjoin%
\definecolor{currentfill}{rgb}{0.000000,0.000000,1.000000}%
\pgfsetfillcolor{currentfill}%
\pgfsetlinewidth{1.003750pt}%
\definecolor{currentstroke}{rgb}{0.000000,0.000000,1.000000}%
\pgfsetstrokecolor{currentstroke}%
\pgfsetdash{}{0pt}%
\pgfsys@defobject{currentmarker}{\pgfqpoint{-0.041667in}{-0.041667in}}{\pgfqpoint{0.041667in}{0.041667in}}{%
\pgfpathmoveto{\pgfqpoint{0.000000in}{-0.041667in}}%
\pgfpathcurveto{\pgfqpoint{0.011050in}{-0.041667in}}{\pgfqpoint{0.021649in}{-0.037276in}}{\pgfqpoint{0.029463in}{-0.029463in}}%
\pgfpathcurveto{\pgfqpoint{0.037276in}{-0.021649in}}{\pgfqpoint{0.041667in}{-0.011050in}}{\pgfqpoint{0.041667in}{0.000000in}}%
\pgfpathcurveto{\pgfqpoint{0.041667in}{0.011050in}}{\pgfqpoint{0.037276in}{0.021649in}}{\pgfqpoint{0.029463in}{0.029463in}}%
\pgfpathcurveto{\pgfqpoint{0.021649in}{0.037276in}}{\pgfqpoint{0.011050in}{0.041667in}}{\pgfqpoint{0.000000in}{0.041667in}}%
\pgfpathcurveto{\pgfqpoint{-0.011050in}{0.041667in}}{\pgfqpoint{-0.021649in}{0.037276in}}{\pgfqpoint{-0.029463in}{0.029463in}}%
\pgfpathcurveto{\pgfqpoint{-0.037276in}{0.021649in}}{\pgfqpoint{-0.041667in}{0.011050in}}{\pgfqpoint{-0.041667in}{0.000000in}}%
\pgfpathcurveto{\pgfqpoint{-0.041667in}{-0.011050in}}{\pgfqpoint{-0.037276in}{-0.021649in}}{\pgfqpoint{-0.029463in}{-0.029463in}}%
\pgfpathcurveto{\pgfqpoint{-0.021649in}{-0.037276in}}{\pgfqpoint{-0.011050in}{-0.041667in}}{\pgfqpoint{0.000000in}{-0.041667in}}%
\pgfpathclose%
\pgfusepath{stroke,fill}%
}%
\begin{pgfscope}%
\pgfsys@transformshift{3.433618in}{7.298527in}%
\pgfsys@useobject{currentmarker}{}%
\end{pgfscope}%
\begin{pgfscope}%
\pgfsys@transformshift{3.864972in}{7.248840in}%
\pgfsys@useobject{currentmarker}{}%
\end{pgfscope}%
\begin{pgfscope}%
\pgfsys@transformshift{4.296327in}{7.191822in}%
\pgfsys@useobject{currentmarker}{}%
\end{pgfscope}%
\begin{pgfscope}%
\pgfsys@transformshift{4.727681in}{7.122400in}%
\pgfsys@useobject{currentmarker}{}%
\end{pgfscope}%
\begin{pgfscope}%
\pgfsys@transformshift{5.159035in}{7.030115in}%
\pgfsys@useobject{currentmarker}{}%
\end{pgfscope}%
\end{pgfscope}%
\begin{pgfscope}%
\pgfpathrectangle{\pgfqpoint{3.347347in}{6.967719in}}{\pgfqpoint{1.897959in}{1.372727in}} %
\pgfusepath{clip}%
\pgfsetbuttcap%
\pgfsetroundjoin%
\pgfsetlinewidth{1.505625pt}%
\definecolor{currentstroke}{rgb}{0.000000,0.750000,0.750000}%
\pgfsetstrokecolor{currentstroke}%
\pgfsetdash{{9.600000pt}{2.400000pt}{1.500000pt}{2.400000pt}}{0.000000pt}%
\pgfpathmoveto{\pgfqpoint{3.433618in}{8.120839in}}%
\pgfpathlineto{\pgfqpoint{3.649295in}{8.104810in}}%
\pgfpathlineto{\pgfqpoint{3.864972in}{8.090833in}}%
\pgfpathlineto{\pgfqpoint{4.080649in}{8.078676in}}%
\pgfpathlineto{\pgfqpoint{4.296327in}{8.068154in}}%
\pgfpathlineto{\pgfqpoint{4.512004in}{8.059119in}}%
\pgfpathlineto{\pgfqpoint{4.727681in}{8.051456in}}%
\pgfpathlineto{\pgfqpoint{4.943358in}{8.045071in}}%
\pgfpathlineto{\pgfqpoint{5.159035in}{8.039889in}}%
\pgfusepath{stroke}%
\end{pgfscope}%
\begin{pgfscope}%
\pgfpathrectangle{\pgfqpoint{3.347347in}{6.967719in}}{\pgfqpoint{1.897959in}{1.372727in}} %
\pgfusepath{clip}%
\pgfsetbuttcap%
\pgfsetmiterjoin%
\definecolor{currentfill}{rgb}{0.000000,0.750000,0.750000}%
\pgfsetfillcolor{currentfill}%
\pgfsetlinewidth{1.003750pt}%
\definecolor{currentstroke}{rgb}{0.000000,0.750000,0.750000}%
\pgfsetstrokecolor{currentstroke}%
\pgfsetdash{}{0pt}%
\pgfsys@defobject{currentmarker}{\pgfqpoint{-0.041667in}{-0.041667in}}{\pgfqpoint{0.041667in}{0.041667in}}{%
\pgfpathmoveto{\pgfqpoint{-0.000000in}{-0.041667in}}%
\pgfpathlineto{\pgfqpoint{0.041667in}{0.041667in}}%
\pgfpathlineto{\pgfqpoint{-0.041667in}{0.041667in}}%
\pgfpathclose%
\pgfusepath{stroke,fill}%
}%
\begin{pgfscope}%
\pgfsys@transformshift{3.433618in}{8.120839in}%
\pgfsys@useobject{currentmarker}{}%
\end{pgfscope}%
\begin{pgfscope}%
\pgfsys@transformshift{3.864972in}{8.090833in}%
\pgfsys@useobject{currentmarker}{}%
\end{pgfscope}%
\begin{pgfscope}%
\pgfsys@transformshift{4.296327in}{8.068154in}%
\pgfsys@useobject{currentmarker}{}%
\end{pgfscope}%
\begin{pgfscope}%
\pgfsys@transformshift{4.727681in}{8.051456in}%
\pgfsys@useobject{currentmarker}{}%
\end{pgfscope}%
\begin{pgfscope}%
\pgfsys@transformshift{5.159035in}{8.039889in}%
\pgfsys@useobject{currentmarker}{}%
\end{pgfscope}%
\end{pgfscope}%
\begin{pgfscope}%
\pgfpathrectangle{\pgfqpoint{3.347347in}{6.967719in}}{\pgfqpoint{1.897959in}{1.372727in}} %
\pgfusepath{clip}%
\pgfsetbuttcap%
\pgfsetroundjoin%
\pgfsetlinewidth{1.505625pt}%
\definecolor{currentstroke}{rgb}{0.000000,0.000000,0.000000}%
\pgfsetstrokecolor{currentstroke}%
\pgfsetdash{{1.500000pt}{2.475000pt}}{0.000000pt}%
\pgfpathmoveto{\pgfqpoint{3.433618in}{8.278049in}}%
\pgfpathlineto{\pgfqpoint{3.649295in}{8.270743in}}%
\pgfpathlineto{\pgfqpoint{3.864972in}{8.263845in}}%
\pgfpathlineto{\pgfqpoint{4.080649in}{8.257970in}}%
\pgfpathlineto{\pgfqpoint{4.296327in}{8.252971in}}%
\pgfpathlineto{\pgfqpoint{4.512004in}{8.248735in}}%
\pgfpathlineto{\pgfqpoint{4.727681in}{8.245176in}}%
\pgfpathlineto{\pgfqpoint{4.943358in}{8.242226in}}%
\pgfpathlineto{\pgfqpoint{5.159035in}{8.239834in}}%
\pgfusepath{stroke}%
\end{pgfscope}%
\begin{pgfscope}%
\pgfpathrectangle{\pgfqpoint{3.347347in}{6.967719in}}{\pgfqpoint{1.897959in}{1.372727in}} %
\pgfusepath{clip}%
\pgfsetbuttcap%
\pgfsetroundjoin%
\definecolor{currentfill}{rgb}{0.000000,0.000000,0.000000}%
\pgfsetfillcolor{currentfill}%
\pgfsetlinewidth{1.003750pt}%
\definecolor{currentstroke}{rgb}{0.000000,0.000000,0.000000}%
\pgfsetstrokecolor{currentstroke}%
\pgfsetdash{}{0pt}%
\pgfsys@defobject{currentmarker}{\pgfqpoint{-0.041667in}{-0.041667in}}{\pgfqpoint{0.041667in}{0.041667in}}{%
\pgfpathmoveto{\pgfqpoint{-0.041667in}{0.000000in}}%
\pgfpathlineto{\pgfqpoint{0.041667in}{0.000000in}}%
\pgfpathmoveto{\pgfqpoint{0.000000in}{-0.041667in}}%
\pgfpathlineto{\pgfqpoint{0.000000in}{0.041667in}}%
\pgfusepath{stroke,fill}%
}%
\begin{pgfscope}%
\pgfsys@transformshift{3.433618in}{8.278049in}%
\pgfsys@useobject{currentmarker}{}%
\end{pgfscope}%
\begin{pgfscope}%
\pgfsys@transformshift{3.864972in}{8.263845in}%
\pgfsys@useobject{currentmarker}{}%
\end{pgfscope}%
\begin{pgfscope}%
\pgfsys@transformshift{4.296327in}{8.252971in}%
\pgfsys@useobject{currentmarker}{}%
\end{pgfscope}%
\begin{pgfscope}%
\pgfsys@transformshift{4.727681in}{8.245176in}%
\pgfsys@useobject{currentmarker}{}%
\end{pgfscope}%
\begin{pgfscope}%
\pgfsys@transformshift{5.159035in}{8.239834in}%
\pgfsys@useobject{currentmarker}{}%
\end{pgfscope}%
\end{pgfscope}%
\begin{pgfscope}%
\pgfsetrectcap%
\pgfsetmiterjoin%
\pgfsetlinewidth{0.803000pt}%
\definecolor{currentstroke}{rgb}{0.000000,0.000000,0.000000}%
\pgfsetstrokecolor{currentstroke}%
\pgfsetdash{}{0pt}%
\pgfpathmoveto{\pgfqpoint{3.347347in}{6.967719in}}%
\pgfpathlineto{\pgfqpoint{3.347347in}{8.340446in}}%
\pgfusepath{stroke}%
\end{pgfscope}%
\begin{pgfscope}%
\pgfsetrectcap%
\pgfsetmiterjoin%
\pgfsetlinewidth{0.803000pt}%
\definecolor{currentstroke}{rgb}{0.000000,0.000000,0.000000}%
\pgfsetstrokecolor{currentstroke}%
\pgfsetdash{}{0pt}%
\pgfpathmoveto{\pgfqpoint{5.245306in}{6.967719in}}%
\pgfpathlineto{\pgfqpoint{5.245306in}{8.340446in}}%
\pgfusepath{stroke}%
\end{pgfscope}%
\begin{pgfscope}%
\pgfsetrectcap%
\pgfsetmiterjoin%
\pgfsetlinewidth{0.803000pt}%
\definecolor{currentstroke}{rgb}{0.000000,0.000000,0.000000}%
\pgfsetstrokecolor{currentstroke}%
\pgfsetdash{}{0pt}%
\pgfpathmoveto{\pgfqpoint{3.347347in}{6.967719in}}%
\pgfpathlineto{\pgfqpoint{5.245306in}{6.967719in}}%
\pgfusepath{stroke}%
\end{pgfscope}%
\begin{pgfscope}%
\pgfsetrectcap%
\pgfsetmiterjoin%
\pgfsetlinewidth{0.803000pt}%
\definecolor{currentstroke}{rgb}{0.000000,0.000000,0.000000}%
\pgfsetstrokecolor{currentstroke}%
\pgfsetdash{}{0pt}%
\pgfpathmoveto{\pgfqpoint{3.347347in}{8.340446in}}%
\pgfpathlineto{\pgfqpoint{5.245306in}{8.340446in}}%
\pgfusepath{stroke}%
\end{pgfscope}%
\begin{pgfscope}%
\pgfsetbuttcap%
\pgfsetmiterjoin%
\definecolor{currentfill}{rgb}{1.000000,1.000000,1.000000}%
\pgfsetfillcolor{currentfill}%
\pgfsetlinewidth{0.000000pt}%
\definecolor{currentstroke}{rgb}{0.000000,0.000000,0.000000}%
\pgfsetstrokecolor{currentstroke}%
\pgfsetstrokeopacity{0.000000}%
\pgfsetdash{}{0pt}%
\pgfpathmoveto{\pgfqpoint{5.814694in}{6.967719in}}%
\pgfpathlineto{\pgfqpoint{7.712653in}{6.967719in}}%
\pgfpathlineto{\pgfqpoint{7.712653in}{8.340446in}}%
\pgfpathlineto{\pgfqpoint{5.814694in}{8.340446in}}%
\pgfpathclose%
\pgfusepath{fill}%
\end{pgfscope}%
\begin{pgfscope}%
\pgfsetbuttcap%
\pgfsetroundjoin%
\definecolor{currentfill}{rgb}{0.000000,0.000000,0.000000}%
\pgfsetfillcolor{currentfill}%
\pgfsetlinewidth{0.803000pt}%
\definecolor{currentstroke}{rgb}{0.000000,0.000000,0.000000}%
\pgfsetstrokecolor{currentstroke}%
\pgfsetdash{}{0pt}%
\pgfsys@defobject{currentmarker}{\pgfqpoint{0.000000in}{-0.048611in}}{\pgfqpoint{0.000000in}{0.000000in}}{%
\pgfpathmoveto{\pgfqpoint{0.000000in}{0.000000in}}%
\pgfpathlineto{\pgfqpoint{0.000000in}{-0.048611in}}%
\pgfusepath{stroke,fill}%
}%
\begin{pgfscope}%
\pgfsys@transformshift{6.073506in}{6.967719in}%
\pgfsys@useobject{currentmarker}{}%
\end{pgfscope}%
\end{pgfscope}%
\begin{pgfscope}%
\pgftext[x=6.073506in,y=6.870496in,,top]{\rmfamily\fontsize{10.000000}{12.000000}\selectfont \(\displaystyle 0.075\)}%
\end{pgfscope}%
\begin{pgfscope}%
\pgfsetbuttcap%
\pgfsetroundjoin%
\definecolor{currentfill}{rgb}{0.000000,0.000000,0.000000}%
\pgfsetfillcolor{currentfill}%
\pgfsetlinewidth{0.803000pt}%
\definecolor{currentstroke}{rgb}{0.000000,0.000000,0.000000}%
\pgfsetstrokecolor{currentstroke}%
\pgfsetdash{}{0pt}%
\pgfsys@defobject{currentmarker}{\pgfqpoint{0.000000in}{-0.048611in}}{\pgfqpoint{0.000000in}{0.000000in}}{%
\pgfpathmoveto{\pgfqpoint{0.000000in}{0.000000in}}%
\pgfpathlineto{\pgfqpoint{0.000000in}{-0.048611in}}%
\pgfusepath{stroke,fill}%
}%
\begin{pgfscope}%
\pgfsys@transformshift{6.591132in}{6.967719in}%
\pgfsys@useobject{currentmarker}{}%
\end{pgfscope}%
\end{pgfscope}%
\begin{pgfscope}%
\pgftext[x=6.591132in,y=6.870496in,,top]{\rmfamily\fontsize{10.000000}{12.000000}\selectfont \(\displaystyle 0.100\)}%
\end{pgfscope}%
\begin{pgfscope}%
\pgfsetbuttcap%
\pgfsetroundjoin%
\definecolor{currentfill}{rgb}{0.000000,0.000000,0.000000}%
\pgfsetfillcolor{currentfill}%
\pgfsetlinewidth{0.803000pt}%
\definecolor{currentstroke}{rgb}{0.000000,0.000000,0.000000}%
\pgfsetstrokecolor{currentstroke}%
\pgfsetdash{}{0pt}%
\pgfsys@defobject{currentmarker}{\pgfqpoint{0.000000in}{-0.048611in}}{\pgfqpoint{0.000000in}{0.000000in}}{%
\pgfpathmoveto{\pgfqpoint{0.000000in}{0.000000in}}%
\pgfpathlineto{\pgfqpoint{0.000000in}{-0.048611in}}%
\pgfusepath{stroke,fill}%
}%
\begin{pgfscope}%
\pgfsys@transformshift{7.108757in}{6.967719in}%
\pgfsys@useobject{currentmarker}{}%
\end{pgfscope}%
\end{pgfscope}%
\begin{pgfscope}%
\pgftext[x=7.108757in,y=6.870496in,,top]{\rmfamily\fontsize{10.000000}{12.000000}\selectfont \(\displaystyle 0.125\)}%
\end{pgfscope}%
\begin{pgfscope}%
\pgfsetbuttcap%
\pgfsetroundjoin%
\definecolor{currentfill}{rgb}{0.000000,0.000000,0.000000}%
\pgfsetfillcolor{currentfill}%
\pgfsetlinewidth{0.803000pt}%
\definecolor{currentstroke}{rgb}{0.000000,0.000000,0.000000}%
\pgfsetstrokecolor{currentstroke}%
\pgfsetdash{}{0pt}%
\pgfsys@defobject{currentmarker}{\pgfqpoint{0.000000in}{-0.048611in}}{\pgfqpoint{0.000000in}{0.000000in}}{%
\pgfpathmoveto{\pgfqpoint{0.000000in}{0.000000in}}%
\pgfpathlineto{\pgfqpoint{0.000000in}{-0.048611in}}%
\pgfusepath{stroke,fill}%
}%
\begin{pgfscope}%
\pgfsys@transformshift{7.626382in}{6.967719in}%
\pgfsys@useobject{currentmarker}{}%
\end{pgfscope}%
\end{pgfscope}%
\begin{pgfscope}%
\pgftext[x=7.626382in,y=6.870496in,,top]{\rmfamily\fontsize{10.000000}{12.000000}\selectfont \(\displaystyle 0.150\)}%
\end{pgfscope}%
\begin{pgfscope}%
\pgfsetbuttcap%
\pgfsetroundjoin%
\definecolor{currentfill}{rgb}{0.000000,0.000000,0.000000}%
\pgfsetfillcolor{currentfill}%
\pgfsetlinewidth{0.803000pt}%
\definecolor{currentstroke}{rgb}{0.000000,0.000000,0.000000}%
\pgfsetstrokecolor{currentstroke}%
\pgfsetdash{}{0pt}%
\pgfsys@defobject{currentmarker}{\pgfqpoint{-0.048611in}{0.000000in}}{\pgfqpoint{0.000000in}{0.000000in}}{%
\pgfpathmoveto{\pgfqpoint{0.000000in}{0.000000in}}%
\pgfpathlineto{\pgfqpoint{-0.048611in}{0.000000in}}%
\pgfusepath{stroke,fill}%
}%
\begin{pgfscope}%
\pgfsys@transformshift{5.814694in}{7.045423in}%
\pgfsys@useobject{currentmarker}{}%
\end{pgfscope}%
\end{pgfscope}%
\begin{pgfscope}%
\pgftext[x=5.429469in,y=6.992662in,left,base]{\rmfamily\fontsize{10.000000}{12.000000}\selectfont \(\displaystyle 10^{-7}\)}%
\end{pgfscope}%
\begin{pgfscope}%
\pgfsetbuttcap%
\pgfsetroundjoin%
\definecolor{currentfill}{rgb}{0.000000,0.000000,0.000000}%
\pgfsetfillcolor{currentfill}%
\pgfsetlinewidth{0.803000pt}%
\definecolor{currentstroke}{rgb}{0.000000,0.000000,0.000000}%
\pgfsetstrokecolor{currentstroke}%
\pgfsetdash{}{0pt}%
\pgfsys@defobject{currentmarker}{\pgfqpoint{-0.048611in}{0.000000in}}{\pgfqpoint{0.000000in}{0.000000in}}{%
\pgfpathmoveto{\pgfqpoint{0.000000in}{0.000000in}}%
\pgfpathlineto{\pgfqpoint{-0.048611in}{0.000000in}}%
\pgfusepath{stroke,fill}%
}%
\begin{pgfscope}%
\pgfsys@transformshift{5.814694in}{7.477298in}%
\pgfsys@useobject{currentmarker}{}%
\end{pgfscope}%
\end{pgfscope}%
\begin{pgfscope}%
\pgftext[x=5.429469in,y=7.424536in,left,base]{\rmfamily\fontsize{10.000000}{12.000000}\selectfont \(\displaystyle 10^{-6}\)}%
\end{pgfscope}%
\begin{pgfscope}%
\pgfsetbuttcap%
\pgfsetroundjoin%
\definecolor{currentfill}{rgb}{0.000000,0.000000,0.000000}%
\pgfsetfillcolor{currentfill}%
\pgfsetlinewidth{0.803000pt}%
\definecolor{currentstroke}{rgb}{0.000000,0.000000,0.000000}%
\pgfsetstrokecolor{currentstroke}%
\pgfsetdash{}{0pt}%
\pgfsys@defobject{currentmarker}{\pgfqpoint{-0.048611in}{0.000000in}}{\pgfqpoint{0.000000in}{0.000000in}}{%
\pgfpathmoveto{\pgfqpoint{0.000000in}{0.000000in}}%
\pgfpathlineto{\pgfqpoint{-0.048611in}{0.000000in}}%
\pgfusepath{stroke,fill}%
}%
\begin{pgfscope}%
\pgfsys@transformshift{5.814694in}{7.909172in}%
\pgfsys@useobject{currentmarker}{}%
\end{pgfscope}%
\end{pgfscope}%
\begin{pgfscope}%
\pgftext[x=5.429469in,y=7.856411in,left,base]{\rmfamily\fontsize{10.000000}{12.000000}\selectfont \(\displaystyle 10^{-5}\)}%
\end{pgfscope}%
\begin{pgfscope}%
\pgfsetbuttcap%
\pgfsetroundjoin%
\definecolor{currentfill}{rgb}{0.000000,0.000000,0.000000}%
\pgfsetfillcolor{currentfill}%
\pgfsetlinewidth{0.803000pt}%
\definecolor{currentstroke}{rgb}{0.000000,0.000000,0.000000}%
\pgfsetstrokecolor{currentstroke}%
\pgfsetdash{}{0pt}%
\pgfsys@defobject{currentmarker}{\pgfqpoint{-0.048611in}{0.000000in}}{\pgfqpoint{0.000000in}{0.000000in}}{%
\pgfpathmoveto{\pgfqpoint{0.000000in}{0.000000in}}%
\pgfpathlineto{\pgfqpoint{-0.048611in}{0.000000in}}%
\pgfusepath{stroke,fill}%
}%
\begin{pgfscope}%
\pgfsys@transformshift{5.814694in}{8.341047in}%
\pgfsys@useobject{currentmarker}{}%
\end{pgfscope}%
\end{pgfscope}%
\begin{pgfscope}%
\pgftext[x=5.429469in,y=8.288285in,left,base]{\rmfamily\fontsize{10.000000}{12.000000}\selectfont \(\displaystyle 10^{-4}\)}%
\end{pgfscope}%
\begin{pgfscope}%
\pgfsetbuttcap%
\pgfsetroundjoin%
\definecolor{currentfill}{rgb}{0.000000,0.000000,0.000000}%
\pgfsetfillcolor{currentfill}%
\pgfsetlinewidth{0.602250pt}%
\definecolor{currentstroke}{rgb}{0.000000,0.000000,0.000000}%
\pgfsetstrokecolor{currentstroke}%
\pgfsetdash{}{0pt}%
\pgfsys@defobject{currentmarker}{\pgfqpoint{-0.027778in}{0.000000in}}{\pgfqpoint{0.000000in}{0.000000in}}{%
\pgfpathmoveto{\pgfqpoint{0.000000in}{0.000000in}}%
\pgfpathlineto{\pgfqpoint{-0.027778in}{0.000000in}}%
\pgfusepath{stroke,fill}%
}%
\begin{pgfscope}%
\pgfsys@transformshift{5.814694in}{6.978525in}%
\pgfsys@useobject{currentmarker}{}%
\end{pgfscope}%
\end{pgfscope}%
\begin{pgfscope}%
\pgfsetbuttcap%
\pgfsetroundjoin%
\definecolor{currentfill}{rgb}{0.000000,0.000000,0.000000}%
\pgfsetfillcolor{currentfill}%
\pgfsetlinewidth{0.602250pt}%
\definecolor{currentstroke}{rgb}{0.000000,0.000000,0.000000}%
\pgfsetstrokecolor{currentstroke}%
\pgfsetdash{}{0pt}%
\pgfsys@defobject{currentmarker}{\pgfqpoint{-0.027778in}{0.000000in}}{\pgfqpoint{0.000000in}{0.000000in}}{%
\pgfpathmoveto{\pgfqpoint{0.000000in}{0.000000in}}%
\pgfpathlineto{\pgfqpoint{-0.027778in}{0.000000in}}%
\pgfusepath{stroke,fill}%
}%
\begin{pgfscope}%
\pgfsys@transformshift{5.814694in}{7.003570in}%
\pgfsys@useobject{currentmarker}{}%
\end{pgfscope}%
\end{pgfscope}%
\begin{pgfscope}%
\pgfsetbuttcap%
\pgfsetroundjoin%
\definecolor{currentfill}{rgb}{0.000000,0.000000,0.000000}%
\pgfsetfillcolor{currentfill}%
\pgfsetlinewidth{0.602250pt}%
\definecolor{currentstroke}{rgb}{0.000000,0.000000,0.000000}%
\pgfsetstrokecolor{currentstroke}%
\pgfsetdash{}{0pt}%
\pgfsys@defobject{currentmarker}{\pgfqpoint{-0.027778in}{0.000000in}}{\pgfqpoint{0.000000in}{0.000000in}}{%
\pgfpathmoveto{\pgfqpoint{0.000000in}{0.000000in}}%
\pgfpathlineto{\pgfqpoint{-0.027778in}{0.000000in}}%
\pgfusepath{stroke,fill}%
}%
\begin{pgfscope}%
\pgfsys@transformshift{5.814694in}{7.025662in}%
\pgfsys@useobject{currentmarker}{}%
\end{pgfscope}%
\end{pgfscope}%
\begin{pgfscope}%
\pgfsetbuttcap%
\pgfsetroundjoin%
\definecolor{currentfill}{rgb}{0.000000,0.000000,0.000000}%
\pgfsetfillcolor{currentfill}%
\pgfsetlinewidth{0.602250pt}%
\definecolor{currentstroke}{rgb}{0.000000,0.000000,0.000000}%
\pgfsetstrokecolor{currentstroke}%
\pgfsetdash{}{0pt}%
\pgfsys@defobject{currentmarker}{\pgfqpoint{-0.027778in}{0.000000in}}{\pgfqpoint{0.000000in}{0.000000in}}{%
\pgfpathmoveto{\pgfqpoint{0.000000in}{0.000000in}}%
\pgfpathlineto{\pgfqpoint{-0.027778in}{0.000000in}}%
\pgfusepath{stroke,fill}%
}%
\begin{pgfscope}%
\pgfsys@transformshift{5.814694in}{7.175431in}%
\pgfsys@useobject{currentmarker}{}%
\end{pgfscope}%
\end{pgfscope}%
\begin{pgfscope}%
\pgfsetbuttcap%
\pgfsetroundjoin%
\definecolor{currentfill}{rgb}{0.000000,0.000000,0.000000}%
\pgfsetfillcolor{currentfill}%
\pgfsetlinewidth{0.602250pt}%
\definecolor{currentstroke}{rgb}{0.000000,0.000000,0.000000}%
\pgfsetstrokecolor{currentstroke}%
\pgfsetdash{}{0pt}%
\pgfsys@defobject{currentmarker}{\pgfqpoint{-0.027778in}{0.000000in}}{\pgfqpoint{0.000000in}{0.000000in}}{%
\pgfpathmoveto{\pgfqpoint{0.000000in}{0.000000in}}%
\pgfpathlineto{\pgfqpoint{-0.027778in}{0.000000in}}%
\pgfusepath{stroke,fill}%
}%
\begin{pgfscope}%
\pgfsys@transformshift{5.814694in}{7.251480in}%
\pgfsys@useobject{currentmarker}{}%
\end{pgfscope}%
\end{pgfscope}%
\begin{pgfscope}%
\pgfsetbuttcap%
\pgfsetroundjoin%
\definecolor{currentfill}{rgb}{0.000000,0.000000,0.000000}%
\pgfsetfillcolor{currentfill}%
\pgfsetlinewidth{0.602250pt}%
\definecolor{currentstroke}{rgb}{0.000000,0.000000,0.000000}%
\pgfsetstrokecolor{currentstroke}%
\pgfsetdash{}{0pt}%
\pgfsys@defobject{currentmarker}{\pgfqpoint{-0.027778in}{0.000000in}}{\pgfqpoint{0.000000in}{0.000000in}}{%
\pgfpathmoveto{\pgfqpoint{0.000000in}{0.000000in}}%
\pgfpathlineto{\pgfqpoint{-0.027778in}{0.000000in}}%
\pgfusepath{stroke,fill}%
}%
\begin{pgfscope}%
\pgfsys@transformshift{5.814694in}{7.305438in}%
\pgfsys@useobject{currentmarker}{}%
\end{pgfscope}%
\end{pgfscope}%
\begin{pgfscope}%
\pgfsetbuttcap%
\pgfsetroundjoin%
\definecolor{currentfill}{rgb}{0.000000,0.000000,0.000000}%
\pgfsetfillcolor{currentfill}%
\pgfsetlinewidth{0.602250pt}%
\definecolor{currentstroke}{rgb}{0.000000,0.000000,0.000000}%
\pgfsetstrokecolor{currentstroke}%
\pgfsetdash{}{0pt}%
\pgfsys@defobject{currentmarker}{\pgfqpoint{-0.027778in}{0.000000in}}{\pgfqpoint{0.000000in}{0.000000in}}{%
\pgfpathmoveto{\pgfqpoint{0.000000in}{0.000000in}}%
\pgfpathlineto{\pgfqpoint{-0.027778in}{0.000000in}}%
\pgfusepath{stroke,fill}%
}%
\begin{pgfscope}%
\pgfsys@transformshift{5.814694in}{7.347291in}%
\pgfsys@useobject{currentmarker}{}%
\end{pgfscope}%
\end{pgfscope}%
\begin{pgfscope}%
\pgfsetbuttcap%
\pgfsetroundjoin%
\definecolor{currentfill}{rgb}{0.000000,0.000000,0.000000}%
\pgfsetfillcolor{currentfill}%
\pgfsetlinewidth{0.602250pt}%
\definecolor{currentstroke}{rgb}{0.000000,0.000000,0.000000}%
\pgfsetstrokecolor{currentstroke}%
\pgfsetdash{}{0pt}%
\pgfsys@defobject{currentmarker}{\pgfqpoint{-0.027778in}{0.000000in}}{\pgfqpoint{0.000000in}{0.000000in}}{%
\pgfpathmoveto{\pgfqpoint{0.000000in}{0.000000in}}%
\pgfpathlineto{\pgfqpoint{-0.027778in}{0.000000in}}%
\pgfusepath{stroke,fill}%
}%
\begin{pgfscope}%
\pgfsys@transformshift{5.814694in}{7.381487in}%
\pgfsys@useobject{currentmarker}{}%
\end{pgfscope}%
\end{pgfscope}%
\begin{pgfscope}%
\pgfsetbuttcap%
\pgfsetroundjoin%
\definecolor{currentfill}{rgb}{0.000000,0.000000,0.000000}%
\pgfsetfillcolor{currentfill}%
\pgfsetlinewidth{0.602250pt}%
\definecolor{currentstroke}{rgb}{0.000000,0.000000,0.000000}%
\pgfsetstrokecolor{currentstroke}%
\pgfsetdash{}{0pt}%
\pgfsys@defobject{currentmarker}{\pgfqpoint{-0.027778in}{0.000000in}}{\pgfqpoint{0.000000in}{0.000000in}}{%
\pgfpathmoveto{\pgfqpoint{0.000000in}{0.000000in}}%
\pgfpathlineto{\pgfqpoint{-0.027778in}{0.000000in}}%
\pgfusepath{stroke,fill}%
}%
\begin{pgfscope}%
\pgfsys@transformshift{5.814694in}{7.410400in}%
\pgfsys@useobject{currentmarker}{}%
\end{pgfscope}%
\end{pgfscope}%
\begin{pgfscope}%
\pgfsetbuttcap%
\pgfsetroundjoin%
\definecolor{currentfill}{rgb}{0.000000,0.000000,0.000000}%
\pgfsetfillcolor{currentfill}%
\pgfsetlinewidth{0.602250pt}%
\definecolor{currentstroke}{rgb}{0.000000,0.000000,0.000000}%
\pgfsetstrokecolor{currentstroke}%
\pgfsetdash{}{0pt}%
\pgfsys@defobject{currentmarker}{\pgfqpoint{-0.027778in}{0.000000in}}{\pgfqpoint{0.000000in}{0.000000in}}{%
\pgfpathmoveto{\pgfqpoint{0.000000in}{0.000000in}}%
\pgfpathlineto{\pgfqpoint{-0.027778in}{0.000000in}}%
\pgfusepath{stroke,fill}%
}%
\begin{pgfscope}%
\pgfsys@transformshift{5.814694in}{7.435445in}%
\pgfsys@useobject{currentmarker}{}%
\end{pgfscope}%
\end{pgfscope}%
\begin{pgfscope}%
\pgfsetbuttcap%
\pgfsetroundjoin%
\definecolor{currentfill}{rgb}{0.000000,0.000000,0.000000}%
\pgfsetfillcolor{currentfill}%
\pgfsetlinewidth{0.602250pt}%
\definecolor{currentstroke}{rgb}{0.000000,0.000000,0.000000}%
\pgfsetstrokecolor{currentstroke}%
\pgfsetdash{}{0pt}%
\pgfsys@defobject{currentmarker}{\pgfqpoint{-0.027778in}{0.000000in}}{\pgfqpoint{0.000000in}{0.000000in}}{%
\pgfpathmoveto{\pgfqpoint{0.000000in}{0.000000in}}%
\pgfpathlineto{\pgfqpoint{-0.027778in}{0.000000in}}%
\pgfusepath{stroke,fill}%
}%
\begin{pgfscope}%
\pgfsys@transformshift{5.814694in}{7.457536in}%
\pgfsys@useobject{currentmarker}{}%
\end{pgfscope}%
\end{pgfscope}%
\begin{pgfscope}%
\pgfsetbuttcap%
\pgfsetroundjoin%
\definecolor{currentfill}{rgb}{0.000000,0.000000,0.000000}%
\pgfsetfillcolor{currentfill}%
\pgfsetlinewidth{0.602250pt}%
\definecolor{currentstroke}{rgb}{0.000000,0.000000,0.000000}%
\pgfsetstrokecolor{currentstroke}%
\pgfsetdash{}{0pt}%
\pgfsys@defobject{currentmarker}{\pgfqpoint{-0.027778in}{0.000000in}}{\pgfqpoint{0.000000in}{0.000000in}}{%
\pgfpathmoveto{\pgfqpoint{0.000000in}{0.000000in}}%
\pgfpathlineto{\pgfqpoint{-0.027778in}{0.000000in}}%
\pgfusepath{stroke,fill}%
}%
\begin{pgfscope}%
\pgfsys@transformshift{5.814694in}{7.607305in}%
\pgfsys@useobject{currentmarker}{}%
\end{pgfscope}%
\end{pgfscope}%
\begin{pgfscope}%
\pgfsetbuttcap%
\pgfsetroundjoin%
\definecolor{currentfill}{rgb}{0.000000,0.000000,0.000000}%
\pgfsetfillcolor{currentfill}%
\pgfsetlinewidth{0.602250pt}%
\definecolor{currentstroke}{rgb}{0.000000,0.000000,0.000000}%
\pgfsetstrokecolor{currentstroke}%
\pgfsetdash{}{0pt}%
\pgfsys@defobject{currentmarker}{\pgfqpoint{-0.027778in}{0.000000in}}{\pgfqpoint{0.000000in}{0.000000in}}{%
\pgfpathmoveto{\pgfqpoint{0.000000in}{0.000000in}}%
\pgfpathlineto{\pgfqpoint{-0.027778in}{0.000000in}}%
\pgfusepath{stroke,fill}%
}%
\begin{pgfscope}%
\pgfsys@transformshift{5.814694in}{7.683354in}%
\pgfsys@useobject{currentmarker}{}%
\end{pgfscope}%
\end{pgfscope}%
\begin{pgfscope}%
\pgfsetbuttcap%
\pgfsetroundjoin%
\definecolor{currentfill}{rgb}{0.000000,0.000000,0.000000}%
\pgfsetfillcolor{currentfill}%
\pgfsetlinewidth{0.602250pt}%
\definecolor{currentstroke}{rgb}{0.000000,0.000000,0.000000}%
\pgfsetstrokecolor{currentstroke}%
\pgfsetdash{}{0pt}%
\pgfsys@defobject{currentmarker}{\pgfqpoint{-0.027778in}{0.000000in}}{\pgfqpoint{0.000000in}{0.000000in}}{%
\pgfpathmoveto{\pgfqpoint{0.000000in}{0.000000in}}%
\pgfpathlineto{\pgfqpoint{-0.027778in}{0.000000in}}%
\pgfusepath{stroke,fill}%
}%
\begin{pgfscope}%
\pgfsys@transformshift{5.814694in}{7.737312in}%
\pgfsys@useobject{currentmarker}{}%
\end{pgfscope}%
\end{pgfscope}%
\begin{pgfscope}%
\pgfsetbuttcap%
\pgfsetroundjoin%
\definecolor{currentfill}{rgb}{0.000000,0.000000,0.000000}%
\pgfsetfillcolor{currentfill}%
\pgfsetlinewidth{0.602250pt}%
\definecolor{currentstroke}{rgb}{0.000000,0.000000,0.000000}%
\pgfsetstrokecolor{currentstroke}%
\pgfsetdash{}{0pt}%
\pgfsys@defobject{currentmarker}{\pgfqpoint{-0.027778in}{0.000000in}}{\pgfqpoint{0.000000in}{0.000000in}}{%
\pgfpathmoveto{\pgfqpoint{0.000000in}{0.000000in}}%
\pgfpathlineto{\pgfqpoint{-0.027778in}{0.000000in}}%
\pgfusepath{stroke,fill}%
}%
\begin{pgfscope}%
\pgfsys@transformshift{5.814694in}{7.779165in}%
\pgfsys@useobject{currentmarker}{}%
\end{pgfscope}%
\end{pgfscope}%
\begin{pgfscope}%
\pgfsetbuttcap%
\pgfsetroundjoin%
\definecolor{currentfill}{rgb}{0.000000,0.000000,0.000000}%
\pgfsetfillcolor{currentfill}%
\pgfsetlinewidth{0.602250pt}%
\definecolor{currentstroke}{rgb}{0.000000,0.000000,0.000000}%
\pgfsetstrokecolor{currentstroke}%
\pgfsetdash{}{0pt}%
\pgfsys@defobject{currentmarker}{\pgfqpoint{-0.027778in}{0.000000in}}{\pgfqpoint{0.000000in}{0.000000in}}{%
\pgfpathmoveto{\pgfqpoint{0.000000in}{0.000000in}}%
\pgfpathlineto{\pgfqpoint{-0.027778in}{0.000000in}}%
\pgfusepath{stroke,fill}%
}%
\begin{pgfscope}%
\pgfsys@transformshift{5.814694in}{7.813362in}%
\pgfsys@useobject{currentmarker}{}%
\end{pgfscope}%
\end{pgfscope}%
\begin{pgfscope}%
\pgfsetbuttcap%
\pgfsetroundjoin%
\definecolor{currentfill}{rgb}{0.000000,0.000000,0.000000}%
\pgfsetfillcolor{currentfill}%
\pgfsetlinewidth{0.602250pt}%
\definecolor{currentstroke}{rgb}{0.000000,0.000000,0.000000}%
\pgfsetstrokecolor{currentstroke}%
\pgfsetdash{}{0pt}%
\pgfsys@defobject{currentmarker}{\pgfqpoint{-0.027778in}{0.000000in}}{\pgfqpoint{0.000000in}{0.000000in}}{%
\pgfpathmoveto{\pgfqpoint{0.000000in}{0.000000in}}%
\pgfpathlineto{\pgfqpoint{-0.027778in}{0.000000in}}%
\pgfusepath{stroke,fill}%
}%
\begin{pgfscope}%
\pgfsys@transformshift{5.814694in}{7.842274in}%
\pgfsys@useobject{currentmarker}{}%
\end{pgfscope}%
\end{pgfscope}%
\begin{pgfscope}%
\pgfsetbuttcap%
\pgfsetroundjoin%
\definecolor{currentfill}{rgb}{0.000000,0.000000,0.000000}%
\pgfsetfillcolor{currentfill}%
\pgfsetlinewidth{0.602250pt}%
\definecolor{currentstroke}{rgb}{0.000000,0.000000,0.000000}%
\pgfsetstrokecolor{currentstroke}%
\pgfsetdash{}{0pt}%
\pgfsys@defobject{currentmarker}{\pgfqpoint{-0.027778in}{0.000000in}}{\pgfqpoint{0.000000in}{0.000000in}}{%
\pgfpathmoveto{\pgfqpoint{0.000000in}{0.000000in}}%
\pgfpathlineto{\pgfqpoint{-0.027778in}{0.000000in}}%
\pgfusepath{stroke,fill}%
}%
\begin{pgfscope}%
\pgfsys@transformshift{5.814694in}{7.867319in}%
\pgfsys@useobject{currentmarker}{}%
\end{pgfscope}%
\end{pgfscope}%
\begin{pgfscope}%
\pgfsetbuttcap%
\pgfsetroundjoin%
\definecolor{currentfill}{rgb}{0.000000,0.000000,0.000000}%
\pgfsetfillcolor{currentfill}%
\pgfsetlinewidth{0.602250pt}%
\definecolor{currentstroke}{rgb}{0.000000,0.000000,0.000000}%
\pgfsetstrokecolor{currentstroke}%
\pgfsetdash{}{0pt}%
\pgfsys@defobject{currentmarker}{\pgfqpoint{-0.027778in}{0.000000in}}{\pgfqpoint{0.000000in}{0.000000in}}{%
\pgfpathmoveto{\pgfqpoint{0.000000in}{0.000000in}}%
\pgfpathlineto{\pgfqpoint{-0.027778in}{0.000000in}}%
\pgfusepath{stroke,fill}%
}%
\begin{pgfscope}%
\pgfsys@transformshift{5.814694in}{7.889411in}%
\pgfsys@useobject{currentmarker}{}%
\end{pgfscope}%
\end{pgfscope}%
\begin{pgfscope}%
\pgfsetbuttcap%
\pgfsetroundjoin%
\definecolor{currentfill}{rgb}{0.000000,0.000000,0.000000}%
\pgfsetfillcolor{currentfill}%
\pgfsetlinewidth{0.602250pt}%
\definecolor{currentstroke}{rgb}{0.000000,0.000000,0.000000}%
\pgfsetstrokecolor{currentstroke}%
\pgfsetdash{}{0pt}%
\pgfsys@defobject{currentmarker}{\pgfqpoint{-0.027778in}{0.000000in}}{\pgfqpoint{0.000000in}{0.000000in}}{%
\pgfpathmoveto{\pgfqpoint{0.000000in}{0.000000in}}%
\pgfpathlineto{\pgfqpoint{-0.027778in}{0.000000in}}%
\pgfusepath{stroke,fill}%
}%
\begin{pgfscope}%
\pgfsys@transformshift{5.814694in}{8.039180in}%
\pgfsys@useobject{currentmarker}{}%
\end{pgfscope}%
\end{pgfscope}%
\begin{pgfscope}%
\pgfsetbuttcap%
\pgfsetroundjoin%
\definecolor{currentfill}{rgb}{0.000000,0.000000,0.000000}%
\pgfsetfillcolor{currentfill}%
\pgfsetlinewidth{0.602250pt}%
\definecolor{currentstroke}{rgb}{0.000000,0.000000,0.000000}%
\pgfsetstrokecolor{currentstroke}%
\pgfsetdash{}{0pt}%
\pgfsys@defobject{currentmarker}{\pgfqpoint{-0.027778in}{0.000000in}}{\pgfqpoint{0.000000in}{0.000000in}}{%
\pgfpathmoveto{\pgfqpoint{0.000000in}{0.000000in}}%
\pgfpathlineto{\pgfqpoint{-0.027778in}{0.000000in}}%
\pgfusepath{stroke,fill}%
}%
\begin{pgfscope}%
\pgfsys@transformshift{5.814694in}{8.115229in}%
\pgfsys@useobject{currentmarker}{}%
\end{pgfscope}%
\end{pgfscope}%
\begin{pgfscope}%
\pgfsetbuttcap%
\pgfsetroundjoin%
\definecolor{currentfill}{rgb}{0.000000,0.000000,0.000000}%
\pgfsetfillcolor{currentfill}%
\pgfsetlinewidth{0.602250pt}%
\definecolor{currentstroke}{rgb}{0.000000,0.000000,0.000000}%
\pgfsetstrokecolor{currentstroke}%
\pgfsetdash{}{0pt}%
\pgfsys@defobject{currentmarker}{\pgfqpoint{-0.027778in}{0.000000in}}{\pgfqpoint{0.000000in}{0.000000in}}{%
\pgfpathmoveto{\pgfqpoint{0.000000in}{0.000000in}}%
\pgfpathlineto{\pgfqpoint{-0.027778in}{0.000000in}}%
\pgfusepath{stroke,fill}%
}%
\begin{pgfscope}%
\pgfsys@transformshift{5.814694in}{8.169187in}%
\pgfsys@useobject{currentmarker}{}%
\end{pgfscope}%
\end{pgfscope}%
\begin{pgfscope}%
\pgfsetbuttcap%
\pgfsetroundjoin%
\definecolor{currentfill}{rgb}{0.000000,0.000000,0.000000}%
\pgfsetfillcolor{currentfill}%
\pgfsetlinewidth{0.602250pt}%
\definecolor{currentstroke}{rgb}{0.000000,0.000000,0.000000}%
\pgfsetstrokecolor{currentstroke}%
\pgfsetdash{}{0pt}%
\pgfsys@defobject{currentmarker}{\pgfqpoint{-0.027778in}{0.000000in}}{\pgfqpoint{0.000000in}{0.000000in}}{%
\pgfpathmoveto{\pgfqpoint{0.000000in}{0.000000in}}%
\pgfpathlineto{\pgfqpoint{-0.027778in}{0.000000in}}%
\pgfusepath{stroke,fill}%
}%
\begin{pgfscope}%
\pgfsys@transformshift{5.814694in}{8.211040in}%
\pgfsys@useobject{currentmarker}{}%
\end{pgfscope}%
\end{pgfscope}%
\begin{pgfscope}%
\pgfsetbuttcap%
\pgfsetroundjoin%
\definecolor{currentfill}{rgb}{0.000000,0.000000,0.000000}%
\pgfsetfillcolor{currentfill}%
\pgfsetlinewidth{0.602250pt}%
\definecolor{currentstroke}{rgb}{0.000000,0.000000,0.000000}%
\pgfsetstrokecolor{currentstroke}%
\pgfsetdash{}{0pt}%
\pgfsys@defobject{currentmarker}{\pgfqpoint{-0.027778in}{0.000000in}}{\pgfqpoint{0.000000in}{0.000000in}}{%
\pgfpathmoveto{\pgfqpoint{0.000000in}{0.000000in}}%
\pgfpathlineto{\pgfqpoint{-0.027778in}{0.000000in}}%
\pgfusepath{stroke,fill}%
}%
\begin{pgfscope}%
\pgfsys@transformshift{5.814694in}{8.245236in}%
\pgfsys@useobject{currentmarker}{}%
\end{pgfscope}%
\end{pgfscope}%
\begin{pgfscope}%
\pgfsetbuttcap%
\pgfsetroundjoin%
\definecolor{currentfill}{rgb}{0.000000,0.000000,0.000000}%
\pgfsetfillcolor{currentfill}%
\pgfsetlinewidth{0.602250pt}%
\definecolor{currentstroke}{rgb}{0.000000,0.000000,0.000000}%
\pgfsetstrokecolor{currentstroke}%
\pgfsetdash{}{0pt}%
\pgfsys@defobject{currentmarker}{\pgfqpoint{-0.027778in}{0.000000in}}{\pgfqpoint{0.000000in}{0.000000in}}{%
\pgfpathmoveto{\pgfqpoint{0.000000in}{0.000000in}}%
\pgfpathlineto{\pgfqpoint{-0.027778in}{0.000000in}}%
\pgfusepath{stroke,fill}%
}%
\begin{pgfscope}%
\pgfsys@transformshift{5.814694in}{8.274149in}%
\pgfsys@useobject{currentmarker}{}%
\end{pgfscope}%
\end{pgfscope}%
\begin{pgfscope}%
\pgfsetbuttcap%
\pgfsetroundjoin%
\definecolor{currentfill}{rgb}{0.000000,0.000000,0.000000}%
\pgfsetfillcolor{currentfill}%
\pgfsetlinewidth{0.602250pt}%
\definecolor{currentstroke}{rgb}{0.000000,0.000000,0.000000}%
\pgfsetstrokecolor{currentstroke}%
\pgfsetdash{}{0pt}%
\pgfsys@defobject{currentmarker}{\pgfqpoint{-0.027778in}{0.000000in}}{\pgfqpoint{0.000000in}{0.000000in}}{%
\pgfpathmoveto{\pgfqpoint{0.000000in}{0.000000in}}%
\pgfpathlineto{\pgfqpoint{-0.027778in}{0.000000in}}%
\pgfusepath{stroke,fill}%
}%
\begin{pgfscope}%
\pgfsys@transformshift{5.814694in}{8.299194in}%
\pgfsys@useobject{currentmarker}{}%
\end{pgfscope}%
\end{pgfscope}%
\begin{pgfscope}%
\pgfsetbuttcap%
\pgfsetroundjoin%
\definecolor{currentfill}{rgb}{0.000000,0.000000,0.000000}%
\pgfsetfillcolor{currentfill}%
\pgfsetlinewidth{0.602250pt}%
\definecolor{currentstroke}{rgb}{0.000000,0.000000,0.000000}%
\pgfsetstrokecolor{currentstroke}%
\pgfsetdash{}{0pt}%
\pgfsys@defobject{currentmarker}{\pgfqpoint{-0.027778in}{0.000000in}}{\pgfqpoint{0.000000in}{0.000000in}}{%
\pgfpathmoveto{\pgfqpoint{0.000000in}{0.000000in}}%
\pgfpathlineto{\pgfqpoint{-0.027778in}{0.000000in}}%
\pgfusepath{stroke,fill}%
}%
\begin{pgfscope}%
\pgfsys@transformshift{5.814694in}{8.321285in}%
\pgfsys@useobject{currentmarker}{}%
\end{pgfscope}%
\end{pgfscope}%
\begin{pgfscope}%
\pgfpathrectangle{\pgfqpoint{5.814694in}{6.967719in}}{\pgfqpoint{1.897959in}{1.372727in}} %
\pgfusepath{clip}%
\pgfsetbuttcap%
\pgfsetroundjoin%
\pgfsetlinewidth{1.505625pt}%
\definecolor{currentstroke}{rgb}{1.000000,0.000000,0.000000}%
\pgfsetstrokecolor{currentstroke}%
\pgfsetdash{{5.550000pt}{2.400000pt}}{0.000000pt}%
\pgfpathmoveto{\pgfqpoint{5.900965in}{8.182145in}}%
\pgfpathlineto{\pgfqpoint{6.073506in}{8.179883in}}%
\pgfpathlineto{\pgfqpoint{6.246048in}{8.177803in}}%
\pgfpathlineto{\pgfqpoint{6.418590in}{8.175900in}}%
\pgfpathlineto{\pgfqpoint{6.591132in}{8.174169in}}%
\pgfpathlineto{\pgfqpoint{6.763673in}{8.172607in}}%
\pgfpathlineto{\pgfqpoint{6.936215in}{8.171211in}}%
\pgfpathlineto{\pgfqpoint{7.108757in}{8.169980in}}%
\pgfpathlineto{\pgfqpoint{7.281299in}{8.168910in}}%
\pgfpathlineto{\pgfqpoint{7.453840in}{8.168002in}}%
\pgfpathlineto{\pgfqpoint{7.626382in}{8.167253in}}%
\pgfusepath{stroke}%
\end{pgfscope}%
\begin{pgfscope}%
\pgfpathrectangle{\pgfqpoint{5.814694in}{6.967719in}}{\pgfqpoint{1.897959in}{1.372727in}} %
\pgfusepath{clip}%
\pgfsetbuttcap%
\pgfsetmiterjoin%
\definecolor{currentfill}{rgb}{1.000000,0.000000,0.000000}%
\pgfsetfillcolor{currentfill}%
\pgfsetlinewidth{1.003750pt}%
\definecolor{currentstroke}{rgb}{1.000000,0.000000,0.000000}%
\pgfsetstrokecolor{currentstroke}%
\pgfsetdash{}{0pt}%
\pgfsys@defobject{currentmarker}{\pgfqpoint{-0.041667in}{-0.041667in}}{\pgfqpoint{0.041667in}{0.041667in}}{%
\pgfpathmoveto{\pgfqpoint{-0.041667in}{-0.041667in}}%
\pgfpathlineto{\pgfqpoint{0.041667in}{-0.041667in}}%
\pgfpathlineto{\pgfqpoint{0.041667in}{0.041667in}}%
\pgfpathlineto{\pgfqpoint{-0.041667in}{0.041667in}}%
\pgfpathclose%
\pgfusepath{stroke,fill}%
}%
\begin{pgfscope}%
\pgfsys@transformshift{5.900965in}{8.182145in}%
\pgfsys@useobject{currentmarker}{}%
\end{pgfscope}%
\begin{pgfscope}%
\pgfsys@transformshift{6.246048in}{8.177803in}%
\pgfsys@useobject{currentmarker}{}%
\end{pgfscope}%
\begin{pgfscope}%
\pgfsys@transformshift{6.591132in}{8.174169in}%
\pgfsys@useobject{currentmarker}{}%
\end{pgfscope}%
\begin{pgfscope}%
\pgfsys@transformshift{6.936215in}{8.171211in}%
\pgfsys@useobject{currentmarker}{}%
\end{pgfscope}%
\begin{pgfscope}%
\pgfsys@transformshift{7.281299in}{8.168910in}%
\pgfsys@useobject{currentmarker}{}%
\end{pgfscope}%
\begin{pgfscope}%
\pgfsys@transformshift{7.626382in}{8.167253in}%
\pgfsys@useobject{currentmarker}{}%
\end{pgfscope}%
\end{pgfscope}%
\begin{pgfscope}%
\pgfpathrectangle{\pgfqpoint{5.814694in}{6.967719in}}{\pgfqpoint{1.897959in}{1.372727in}} %
\pgfusepath{clip}%
\pgfsetrectcap%
\pgfsetroundjoin%
\pgfsetlinewidth{1.505625pt}%
\definecolor{currentstroke}{rgb}{0.000000,0.000000,1.000000}%
\pgfsetstrokecolor{currentstroke}%
\pgfsetdash{}{0pt}%
\pgfpathmoveto{\pgfqpoint{5.900965in}{7.338250in}}%
\pgfpathlineto{\pgfqpoint{6.073506in}{7.316061in}}%
\pgfpathlineto{\pgfqpoint{6.246048in}{7.293235in}}%
\pgfpathlineto{\pgfqpoint{6.418590in}{7.269439in}}%
\pgfpathlineto{\pgfqpoint{6.591132in}{7.244315in}}%
\pgfpathlineto{\pgfqpoint{6.763673in}{7.217451in}}%
\pgfpathlineto{\pgfqpoint{6.936215in}{7.188329in}}%
\pgfpathlineto{\pgfqpoint{7.108757in}{7.156268in}}%
\pgfpathlineto{\pgfqpoint{7.281299in}{7.120311in}}%
\pgfpathlineto{\pgfqpoint{7.453840in}{7.079028in}}%
\pgfpathlineto{\pgfqpoint{7.626382in}{7.030115in}}%
\pgfusepath{stroke}%
\end{pgfscope}%
\begin{pgfscope}%
\pgfpathrectangle{\pgfqpoint{5.814694in}{6.967719in}}{\pgfqpoint{1.897959in}{1.372727in}} %
\pgfusepath{clip}%
\pgfsetbuttcap%
\pgfsetroundjoin%
\definecolor{currentfill}{rgb}{0.000000,0.000000,1.000000}%
\pgfsetfillcolor{currentfill}%
\pgfsetlinewidth{1.003750pt}%
\definecolor{currentstroke}{rgb}{0.000000,0.000000,1.000000}%
\pgfsetstrokecolor{currentstroke}%
\pgfsetdash{}{0pt}%
\pgfsys@defobject{currentmarker}{\pgfqpoint{-0.041667in}{-0.041667in}}{\pgfqpoint{0.041667in}{0.041667in}}{%
\pgfpathmoveto{\pgfqpoint{0.000000in}{-0.041667in}}%
\pgfpathcurveto{\pgfqpoint{0.011050in}{-0.041667in}}{\pgfqpoint{0.021649in}{-0.037276in}}{\pgfqpoint{0.029463in}{-0.029463in}}%
\pgfpathcurveto{\pgfqpoint{0.037276in}{-0.021649in}}{\pgfqpoint{0.041667in}{-0.011050in}}{\pgfqpoint{0.041667in}{0.000000in}}%
\pgfpathcurveto{\pgfqpoint{0.041667in}{0.011050in}}{\pgfqpoint{0.037276in}{0.021649in}}{\pgfqpoint{0.029463in}{0.029463in}}%
\pgfpathcurveto{\pgfqpoint{0.021649in}{0.037276in}}{\pgfqpoint{0.011050in}{0.041667in}}{\pgfqpoint{0.000000in}{0.041667in}}%
\pgfpathcurveto{\pgfqpoint{-0.011050in}{0.041667in}}{\pgfqpoint{-0.021649in}{0.037276in}}{\pgfqpoint{-0.029463in}{0.029463in}}%
\pgfpathcurveto{\pgfqpoint{-0.037276in}{0.021649in}}{\pgfqpoint{-0.041667in}{0.011050in}}{\pgfqpoint{-0.041667in}{0.000000in}}%
\pgfpathcurveto{\pgfqpoint{-0.041667in}{-0.011050in}}{\pgfqpoint{-0.037276in}{-0.021649in}}{\pgfqpoint{-0.029463in}{-0.029463in}}%
\pgfpathcurveto{\pgfqpoint{-0.021649in}{-0.037276in}}{\pgfqpoint{-0.011050in}{-0.041667in}}{\pgfqpoint{0.000000in}{-0.041667in}}%
\pgfpathclose%
\pgfusepath{stroke,fill}%
}%
\begin{pgfscope}%
\pgfsys@transformshift{5.900965in}{7.338250in}%
\pgfsys@useobject{currentmarker}{}%
\end{pgfscope}%
\begin{pgfscope}%
\pgfsys@transformshift{6.246048in}{7.293235in}%
\pgfsys@useobject{currentmarker}{}%
\end{pgfscope}%
\begin{pgfscope}%
\pgfsys@transformshift{6.591132in}{7.244315in}%
\pgfsys@useobject{currentmarker}{}%
\end{pgfscope}%
\begin{pgfscope}%
\pgfsys@transformshift{6.936215in}{7.188329in}%
\pgfsys@useobject{currentmarker}{}%
\end{pgfscope}%
\begin{pgfscope}%
\pgfsys@transformshift{7.281299in}{7.120311in}%
\pgfsys@useobject{currentmarker}{}%
\end{pgfscope}%
\begin{pgfscope}%
\pgfsys@transformshift{7.626382in}{7.030115in}%
\pgfsys@useobject{currentmarker}{}%
\end{pgfscope}%
\end{pgfscope}%
\begin{pgfscope}%
\pgfpathrectangle{\pgfqpoint{5.814694in}{6.967719in}}{\pgfqpoint{1.897959in}{1.372727in}} %
\pgfusepath{clip}%
\pgfsetbuttcap%
\pgfsetroundjoin%
\pgfsetlinewidth{1.505625pt}%
\definecolor{currentstroke}{rgb}{0.000000,0.750000,0.750000}%
\pgfsetstrokecolor{currentstroke}%
\pgfsetdash{{9.600000pt}{2.400000pt}{1.500000pt}{2.400000pt}}{0.000000pt}%
\pgfpathmoveto{\pgfqpoint{5.900965in}{8.105465in}}%
\pgfpathlineto{\pgfqpoint{6.073506in}{8.079770in}}%
\pgfpathlineto{\pgfqpoint{6.246048in}{8.057748in}}%
\pgfpathlineto{\pgfqpoint{6.418590in}{8.038780in}}%
\pgfpathlineto{\pgfqpoint{6.591132in}{8.022408in}}%
\pgfpathlineto{\pgfqpoint{6.763673in}{8.008288in}}%
\pgfpathlineto{\pgfqpoint{6.936215in}{7.996153in}}%
\pgfpathlineto{\pgfqpoint{7.108757in}{7.985797in}}%
\pgfpathlineto{\pgfqpoint{7.281299in}{7.977057in}}%
\pgfpathlineto{\pgfqpoint{7.453840in}{7.969807in}}%
\pgfpathlineto{\pgfqpoint{7.626382in}{7.963948in}}%
\pgfusepath{stroke}%
\end{pgfscope}%
\begin{pgfscope}%
\pgfpathrectangle{\pgfqpoint{5.814694in}{6.967719in}}{\pgfqpoint{1.897959in}{1.372727in}} %
\pgfusepath{clip}%
\pgfsetbuttcap%
\pgfsetmiterjoin%
\definecolor{currentfill}{rgb}{0.000000,0.750000,0.750000}%
\pgfsetfillcolor{currentfill}%
\pgfsetlinewidth{1.003750pt}%
\definecolor{currentstroke}{rgb}{0.000000,0.750000,0.750000}%
\pgfsetstrokecolor{currentstroke}%
\pgfsetdash{}{0pt}%
\pgfsys@defobject{currentmarker}{\pgfqpoint{-0.041667in}{-0.041667in}}{\pgfqpoint{0.041667in}{0.041667in}}{%
\pgfpathmoveto{\pgfqpoint{-0.000000in}{-0.041667in}}%
\pgfpathlineto{\pgfqpoint{0.041667in}{0.041667in}}%
\pgfpathlineto{\pgfqpoint{-0.041667in}{0.041667in}}%
\pgfpathclose%
\pgfusepath{stroke,fill}%
}%
\begin{pgfscope}%
\pgfsys@transformshift{5.900965in}{8.105465in}%
\pgfsys@useobject{currentmarker}{}%
\end{pgfscope}%
\begin{pgfscope}%
\pgfsys@transformshift{6.246048in}{8.057748in}%
\pgfsys@useobject{currentmarker}{}%
\end{pgfscope}%
\begin{pgfscope}%
\pgfsys@transformshift{6.591132in}{8.022408in}%
\pgfsys@useobject{currentmarker}{}%
\end{pgfscope}%
\begin{pgfscope}%
\pgfsys@transformshift{6.936215in}{7.996153in}%
\pgfsys@useobject{currentmarker}{}%
\end{pgfscope}%
\begin{pgfscope}%
\pgfsys@transformshift{7.281299in}{7.977057in}%
\pgfsys@useobject{currentmarker}{}%
\end{pgfscope}%
\begin{pgfscope}%
\pgfsys@transformshift{7.626382in}{7.963948in}%
\pgfsys@useobject{currentmarker}{}%
\end{pgfscope}%
\end{pgfscope}%
\begin{pgfscope}%
\pgfpathrectangle{\pgfqpoint{5.814694in}{6.967719in}}{\pgfqpoint{1.897959in}{1.372727in}} %
\pgfusepath{clip}%
\pgfsetbuttcap%
\pgfsetroundjoin%
\pgfsetlinewidth{1.505625pt}%
\definecolor{currentstroke}{rgb}{0.000000,0.000000,0.000000}%
\pgfsetstrokecolor{currentstroke}%
\pgfsetdash{{1.500000pt}{2.475000pt}}{0.000000pt}%
\pgfpathmoveto{\pgfqpoint{5.900965in}{8.278049in}}%
\pgfpathlineto{\pgfqpoint{6.073506in}{8.268048in}}%
\pgfpathlineto{\pgfqpoint{6.246048in}{8.258651in}}%
\pgfpathlineto{\pgfqpoint{6.418590in}{8.250866in}}%
\pgfpathlineto{\pgfqpoint{6.591132in}{8.244355in}}%
\pgfpathlineto{\pgfqpoint{6.763673in}{8.238877in}}%
\pgfpathlineto{\pgfqpoint{6.936215in}{8.234261in}}%
\pgfpathlineto{\pgfqpoint{7.108757in}{8.230379in}}%
\pgfpathlineto{\pgfqpoint{7.281299in}{8.227137in}}%
\pgfpathlineto{\pgfqpoint{7.453840in}{8.224465in}}%
\pgfpathlineto{\pgfqpoint{7.626382in}{8.222311in}}%
\pgfusepath{stroke}%
\end{pgfscope}%
\begin{pgfscope}%
\pgfpathrectangle{\pgfqpoint{5.814694in}{6.967719in}}{\pgfqpoint{1.897959in}{1.372727in}} %
\pgfusepath{clip}%
\pgfsetbuttcap%
\pgfsetroundjoin%
\definecolor{currentfill}{rgb}{0.000000,0.000000,0.000000}%
\pgfsetfillcolor{currentfill}%
\pgfsetlinewidth{1.003750pt}%
\definecolor{currentstroke}{rgb}{0.000000,0.000000,0.000000}%
\pgfsetstrokecolor{currentstroke}%
\pgfsetdash{}{0pt}%
\pgfsys@defobject{currentmarker}{\pgfqpoint{-0.041667in}{-0.041667in}}{\pgfqpoint{0.041667in}{0.041667in}}{%
\pgfpathmoveto{\pgfqpoint{-0.041667in}{0.000000in}}%
\pgfpathlineto{\pgfqpoint{0.041667in}{0.000000in}}%
\pgfpathmoveto{\pgfqpoint{0.000000in}{-0.041667in}}%
\pgfpathlineto{\pgfqpoint{0.000000in}{0.041667in}}%
\pgfusepath{stroke,fill}%
}%
\begin{pgfscope}%
\pgfsys@transformshift{5.900965in}{8.278049in}%
\pgfsys@useobject{currentmarker}{}%
\end{pgfscope}%
\begin{pgfscope}%
\pgfsys@transformshift{6.246048in}{8.258651in}%
\pgfsys@useobject{currentmarker}{}%
\end{pgfscope}%
\begin{pgfscope}%
\pgfsys@transformshift{6.591132in}{8.244355in}%
\pgfsys@useobject{currentmarker}{}%
\end{pgfscope}%
\begin{pgfscope}%
\pgfsys@transformshift{6.936215in}{8.234261in}%
\pgfsys@useobject{currentmarker}{}%
\end{pgfscope}%
\begin{pgfscope}%
\pgfsys@transformshift{7.281299in}{8.227137in}%
\pgfsys@useobject{currentmarker}{}%
\end{pgfscope}%
\begin{pgfscope}%
\pgfsys@transformshift{7.626382in}{8.222311in}%
\pgfsys@useobject{currentmarker}{}%
\end{pgfscope}%
\end{pgfscope}%
\begin{pgfscope}%
\pgfsetrectcap%
\pgfsetmiterjoin%
\pgfsetlinewidth{0.803000pt}%
\definecolor{currentstroke}{rgb}{0.000000,0.000000,0.000000}%
\pgfsetstrokecolor{currentstroke}%
\pgfsetdash{}{0pt}%
\pgfpathmoveto{\pgfqpoint{5.814694in}{6.967719in}}%
\pgfpathlineto{\pgfqpoint{5.814694in}{8.340446in}}%
\pgfusepath{stroke}%
\end{pgfscope}%
\begin{pgfscope}%
\pgfsetrectcap%
\pgfsetmiterjoin%
\pgfsetlinewidth{0.803000pt}%
\definecolor{currentstroke}{rgb}{0.000000,0.000000,0.000000}%
\pgfsetstrokecolor{currentstroke}%
\pgfsetdash{}{0pt}%
\pgfpathmoveto{\pgfqpoint{7.712653in}{6.967719in}}%
\pgfpathlineto{\pgfqpoint{7.712653in}{8.340446in}}%
\pgfusepath{stroke}%
\end{pgfscope}%
\begin{pgfscope}%
\pgfsetrectcap%
\pgfsetmiterjoin%
\pgfsetlinewidth{0.803000pt}%
\definecolor{currentstroke}{rgb}{0.000000,0.000000,0.000000}%
\pgfsetstrokecolor{currentstroke}%
\pgfsetdash{}{0pt}%
\pgfpathmoveto{\pgfqpoint{5.814694in}{6.967719in}}%
\pgfpathlineto{\pgfqpoint{7.712653in}{6.967719in}}%
\pgfusepath{stroke}%
\end{pgfscope}%
\begin{pgfscope}%
\pgfsetrectcap%
\pgfsetmiterjoin%
\pgfsetlinewidth{0.803000pt}%
\definecolor{currentstroke}{rgb}{0.000000,0.000000,0.000000}%
\pgfsetstrokecolor{currentstroke}%
\pgfsetdash{}{0pt}%
\pgfpathmoveto{\pgfqpoint{5.814694in}{8.340446in}}%
\pgfpathlineto{\pgfqpoint{7.712653in}{8.340446in}}%
\pgfusepath{stroke}%
\end{pgfscope}%
\begin{pgfscope}%
\pgfsetbuttcap%
\pgfsetmiterjoin%
\definecolor{currentfill}{rgb}{1.000000,1.000000,1.000000}%
\pgfsetfillcolor{currentfill}%
\pgfsetlinewidth{0.000000pt}%
\definecolor{currentstroke}{rgb}{0.000000,0.000000,0.000000}%
\pgfsetstrokecolor{currentstroke}%
\pgfsetstrokeopacity{0.000000}%
\pgfsetdash{}{0pt}%
\pgfpathmoveto{\pgfqpoint{8.282041in}{6.967719in}}%
\pgfpathlineto{\pgfqpoint{10.180000in}{6.967719in}}%
\pgfpathlineto{\pgfqpoint{10.180000in}{8.340446in}}%
\pgfpathlineto{\pgfqpoint{8.282041in}{8.340446in}}%
\pgfpathclose%
\pgfusepath{fill}%
\end{pgfscope}%
\begin{pgfscope}%
\pgfsetbuttcap%
\pgfsetroundjoin%
\definecolor{currentfill}{rgb}{0.000000,0.000000,0.000000}%
\pgfsetfillcolor{currentfill}%
\pgfsetlinewidth{0.803000pt}%
\definecolor{currentstroke}{rgb}{0.000000,0.000000,0.000000}%
\pgfsetstrokecolor{currentstroke}%
\pgfsetdash{}{0pt}%
\pgfsys@defobject{currentmarker}{\pgfqpoint{0.000000in}{-0.048611in}}{\pgfqpoint{0.000000in}{0.000000in}}{%
\pgfpathmoveto{\pgfqpoint{0.000000in}{0.000000in}}%
\pgfpathlineto{\pgfqpoint{0.000000in}{-0.048611in}}%
\pgfusepath{stroke,fill}%
}%
\begin{pgfscope}%
\pgfsys@transformshift{8.540853in}{6.967719in}%
\pgfsys@useobject{currentmarker}{}%
\end{pgfscope}%
\end{pgfscope}%
\begin{pgfscope}%
\pgftext[x=8.540853in,y=6.870496in,,top]{\rmfamily\fontsize{10.000000}{12.000000}\selectfont \(\displaystyle 0.100\)}%
\end{pgfscope}%
\begin{pgfscope}%
\pgfsetbuttcap%
\pgfsetroundjoin%
\definecolor{currentfill}{rgb}{0.000000,0.000000,0.000000}%
\pgfsetfillcolor{currentfill}%
\pgfsetlinewidth{0.803000pt}%
\definecolor{currentstroke}{rgb}{0.000000,0.000000,0.000000}%
\pgfsetstrokecolor{currentstroke}%
\pgfsetdash{}{0pt}%
\pgfsys@defobject{currentmarker}{\pgfqpoint{0.000000in}{-0.048611in}}{\pgfqpoint{0.000000in}{0.000000in}}{%
\pgfpathmoveto{\pgfqpoint{0.000000in}{0.000000in}}%
\pgfpathlineto{\pgfqpoint{0.000000in}{-0.048611in}}%
\pgfusepath{stroke,fill}%
}%
\begin{pgfscope}%
\pgfsys@transformshift{9.058479in}{6.967719in}%
\pgfsys@useobject{currentmarker}{}%
\end{pgfscope}%
\end{pgfscope}%
\begin{pgfscope}%
\pgftext[x=9.058479in,y=6.870496in,,top]{\rmfamily\fontsize{10.000000}{12.000000}\selectfont \(\displaystyle 0.125\)}%
\end{pgfscope}%
\begin{pgfscope}%
\pgfsetbuttcap%
\pgfsetroundjoin%
\definecolor{currentfill}{rgb}{0.000000,0.000000,0.000000}%
\pgfsetfillcolor{currentfill}%
\pgfsetlinewidth{0.803000pt}%
\definecolor{currentstroke}{rgb}{0.000000,0.000000,0.000000}%
\pgfsetstrokecolor{currentstroke}%
\pgfsetdash{}{0pt}%
\pgfsys@defobject{currentmarker}{\pgfqpoint{0.000000in}{-0.048611in}}{\pgfqpoint{0.000000in}{0.000000in}}{%
\pgfpathmoveto{\pgfqpoint{0.000000in}{0.000000in}}%
\pgfpathlineto{\pgfqpoint{0.000000in}{-0.048611in}}%
\pgfusepath{stroke,fill}%
}%
\begin{pgfscope}%
\pgfsys@transformshift{9.576104in}{6.967719in}%
\pgfsys@useobject{currentmarker}{}%
\end{pgfscope}%
\end{pgfscope}%
\begin{pgfscope}%
\pgftext[x=9.576104in,y=6.870496in,,top]{\rmfamily\fontsize{10.000000}{12.000000}\selectfont \(\displaystyle 0.150\)}%
\end{pgfscope}%
\begin{pgfscope}%
\pgfsetbuttcap%
\pgfsetroundjoin%
\definecolor{currentfill}{rgb}{0.000000,0.000000,0.000000}%
\pgfsetfillcolor{currentfill}%
\pgfsetlinewidth{0.803000pt}%
\definecolor{currentstroke}{rgb}{0.000000,0.000000,0.000000}%
\pgfsetstrokecolor{currentstroke}%
\pgfsetdash{}{0pt}%
\pgfsys@defobject{currentmarker}{\pgfqpoint{0.000000in}{-0.048611in}}{\pgfqpoint{0.000000in}{0.000000in}}{%
\pgfpathmoveto{\pgfqpoint{0.000000in}{0.000000in}}%
\pgfpathlineto{\pgfqpoint{0.000000in}{-0.048611in}}%
\pgfusepath{stroke,fill}%
}%
\begin{pgfscope}%
\pgfsys@transformshift{10.093729in}{6.967719in}%
\pgfsys@useobject{currentmarker}{}%
\end{pgfscope}%
\end{pgfscope}%
\begin{pgfscope}%
\pgftext[x=10.093729in,y=6.870496in,,top]{\rmfamily\fontsize{10.000000}{12.000000}\selectfont \(\displaystyle 0.175\)}%
\end{pgfscope}%
\begin{pgfscope}%
\pgfsetbuttcap%
\pgfsetroundjoin%
\definecolor{currentfill}{rgb}{0.000000,0.000000,0.000000}%
\pgfsetfillcolor{currentfill}%
\pgfsetlinewidth{0.803000pt}%
\definecolor{currentstroke}{rgb}{0.000000,0.000000,0.000000}%
\pgfsetstrokecolor{currentstroke}%
\pgfsetdash{}{0pt}%
\pgfsys@defobject{currentmarker}{\pgfqpoint{-0.048611in}{0.000000in}}{\pgfqpoint{0.000000in}{0.000000in}}{%
\pgfpathmoveto{\pgfqpoint{0.000000in}{0.000000in}}%
\pgfpathlineto{\pgfqpoint{-0.048611in}{0.000000in}}%
\pgfusepath{stroke,fill}%
}%
\begin{pgfscope}%
\pgfsys@transformshift{8.282041in}{6.987653in}%
\pgfsys@useobject{currentmarker}{}%
\end{pgfscope}%
\end{pgfscope}%
\begin{pgfscope}%
\pgftext[x=7.896816in,y=6.934892in,left,base]{\rmfamily\fontsize{10.000000}{12.000000}\selectfont \(\displaystyle 10^{-7}\)}%
\end{pgfscope}%
\begin{pgfscope}%
\pgfsetbuttcap%
\pgfsetroundjoin%
\definecolor{currentfill}{rgb}{0.000000,0.000000,0.000000}%
\pgfsetfillcolor{currentfill}%
\pgfsetlinewidth{0.803000pt}%
\definecolor{currentstroke}{rgb}{0.000000,0.000000,0.000000}%
\pgfsetstrokecolor{currentstroke}%
\pgfsetdash{}{0pt}%
\pgfsys@defobject{currentmarker}{\pgfqpoint{-0.048611in}{0.000000in}}{\pgfqpoint{0.000000in}{0.000000in}}{%
\pgfpathmoveto{\pgfqpoint{0.000000in}{0.000000in}}%
\pgfpathlineto{\pgfqpoint{-0.048611in}{0.000000in}}%
\pgfusepath{stroke,fill}%
}%
\begin{pgfscope}%
\pgfsys@transformshift{8.282041in}{7.380664in}%
\pgfsys@useobject{currentmarker}{}%
\end{pgfscope}%
\end{pgfscope}%
\begin{pgfscope}%
\pgftext[x=7.896816in,y=7.327903in,left,base]{\rmfamily\fontsize{10.000000}{12.000000}\selectfont \(\displaystyle 10^{-6}\)}%
\end{pgfscope}%
\begin{pgfscope}%
\pgfsetbuttcap%
\pgfsetroundjoin%
\definecolor{currentfill}{rgb}{0.000000,0.000000,0.000000}%
\pgfsetfillcolor{currentfill}%
\pgfsetlinewidth{0.803000pt}%
\definecolor{currentstroke}{rgb}{0.000000,0.000000,0.000000}%
\pgfsetstrokecolor{currentstroke}%
\pgfsetdash{}{0pt}%
\pgfsys@defobject{currentmarker}{\pgfqpoint{-0.048611in}{0.000000in}}{\pgfqpoint{0.000000in}{0.000000in}}{%
\pgfpathmoveto{\pgfqpoint{0.000000in}{0.000000in}}%
\pgfpathlineto{\pgfqpoint{-0.048611in}{0.000000in}}%
\pgfusepath{stroke,fill}%
}%
\begin{pgfscope}%
\pgfsys@transformshift{8.282041in}{7.773676in}%
\pgfsys@useobject{currentmarker}{}%
\end{pgfscope}%
\end{pgfscope}%
\begin{pgfscope}%
\pgftext[x=7.896816in,y=7.720914in,left,base]{\rmfamily\fontsize{10.000000}{12.000000}\selectfont \(\displaystyle 10^{-5}\)}%
\end{pgfscope}%
\begin{pgfscope}%
\pgfsetbuttcap%
\pgfsetroundjoin%
\definecolor{currentfill}{rgb}{0.000000,0.000000,0.000000}%
\pgfsetfillcolor{currentfill}%
\pgfsetlinewidth{0.803000pt}%
\definecolor{currentstroke}{rgb}{0.000000,0.000000,0.000000}%
\pgfsetstrokecolor{currentstroke}%
\pgfsetdash{}{0pt}%
\pgfsys@defobject{currentmarker}{\pgfqpoint{-0.048611in}{0.000000in}}{\pgfqpoint{0.000000in}{0.000000in}}{%
\pgfpathmoveto{\pgfqpoint{0.000000in}{0.000000in}}%
\pgfpathlineto{\pgfqpoint{-0.048611in}{0.000000in}}%
\pgfusepath{stroke,fill}%
}%
\begin{pgfscope}%
\pgfsys@transformshift{8.282041in}{8.166687in}%
\pgfsys@useobject{currentmarker}{}%
\end{pgfscope}%
\end{pgfscope}%
\begin{pgfscope}%
\pgftext[x=7.896816in,y=8.113926in,left,base]{\rmfamily\fontsize{10.000000}{12.000000}\selectfont \(\displaystyle 10^{-4}\)}%
\end{pgfscope}%
\begin{pgfscope}%
\pgfsetbuttcap%
\pgfsetroundjoin%
\definecolor{currentfill}{rgb}{0.000000,0.000000,0.000000}%
\pgfsetfillcolor{currentfill}%
\pgfsetlinewidth{0.602250pt}%
\definecolor{currentstroke}{rgb}{0.000000,0.000000,0.000000}%
\pgfsetstrokecolor{currentstroke}%
\pgfsetdash{}{0pt}%
\pgfsys@defobject{currentmarker}{\pgfqpoint{-0.027778in}{0.000000in}}{\pgfqpoint{0.000000in}{0.000000in}}{%
\pgfpathmoveto{\pgfqpoint{0.000000in}{0.000000in}}%
\pgfpathlineto{\pgfqpoint{-0.027778in}{0.000000in}}%
\pgfusepath{stroke,fill}%
}%
\begin{pgfscope}%
\pgfsys@transformshift{8.282041in}{6.969670in}%
\pgfsys@useobject{currentmarker}{}%
\end{pgfscope}%
\end{pgfscope}%
\begin{pgfscope}%
\pgfsetbuttcap%
\pgfsetroundjoin%
\definecolor{currentfill}{rgb}{0.000000,0.000000,0.000000}%
\pgfsetfillcolor{currentfill}%
\pgfsetlinewidth{0.602250pt}%
\definecolor{currentstroke}{rgb}{0.000000,0.000000,0.000000}%
\pgfsetstrokecolor{currentstroke}%
\pgfsetdash{}{0pt}%
\pgfsys@defobject{currentmarker}{\pgfqpoint{-0.027778in}{0.000000in}}{\pgfqpoint{0.000000in}{0.000000in}}{%
\pgfpathmoveto{\pgfqpoint{0.000000in}{0.000000in}}%
\pgfpathlineto{\pgfqpoint{-0.027778in}{0.000000in}}%
\pgfusepath{stroke,fill}%
}%
\begin{pgfscope}%
\pgfsys@transformshift{8.282041in}{7.105961in}%
\pgfsys@useobject{currentmarker}{}%
\end{pgfscope}%
\end{pgfscope}%
\begin{pgfscope}%
\pgfsetbuttcap%
\pgfsetroundjoin%
\definecolor{currentfill}{rgb}{0.000000,0.000000,0.000000}%
\pgfsetfillcolor{currentfill}%
\pgfsetlinewidth{0.602250pt}%
\definecolor{currentstroke}{rgb}{0.000000,0.000000,0.000000}%
\pgfsetstrokecolor{currentstroke}%
\pgfsetdash{}{0pt}%
\pgfsys@defobject{currentmarker}{\pgfqpoint{-0.027778in}{0.000000in}}{\pgfqpoint{0.000000in}{0.000000in}}{%
\pgfpathmoveto{\pgfqpoint{0.000000in}{0.000000in}}%
\pgfpathlineto{\pgfqpoint{-0.027778in}{0.000000in}}%
\pgfusepath{stroke,fill}%
}%
\begin{pgfscope}%
\pgfsys@transformshift{8.282041in}{7.175167in}%
\pgfsys@useobject{currentmarker}{}%
\end{pgfscope}%
\end{pgfscope}%
\begin{pgfscope}%
\pgfsetbuttcap%
\pgfsetroundjoin%
\definecolor{currentfill}{rgb}{0.000000,0.000000,0.000000}%
\pgfsetfillcolor{currentfill}%
\pgfsetlinewidth{0.602250pt}%
\definecolor{currentstroke}{rgb}{0.000000,0.000000,0.000000}%
\pgfsetstrokecolor{currentstroke}%
\pgfsetdash{}{0pt}%
\pgfsys@defobject{currentmarker}{\pgfqpoint{-0.027778in}{0.000000in}}{\pgfqpoint{0.000000in}{0.000000in}}{%
\pgfpathmoveto{\pgfqpoint{0.000000in}{0.000000in}}%
\pgfpathlineto{\pgfqpoint{-0.027778in}{0.000000in}}%
\pgfusepath{stroke,fill}%
}%
\begin{pgfscope}%
\pgfsys@transformshift{8.282041in}{7.224270in}%
\pgfsys@useobject{currentmarker}{}%
\end{pgfscope}%
\end{pgfscope}%
\begin{pgfscope}%
\pgfsetbuttcap%
\pgfsetroundjoin%
\definecolor{currentfill}{rgb}{0.000000,0.000000,0.000000}%
\pgfsetfillcolor{currentfill}%
\pgfsetlinewidth{0.602250pt}%
\definecolor{currentstroke}{rgb}{0.000000,0.000000,0.000000}%
\pgfsetstrokecolor{currentstroke}%
\pgfsetdash{}{0pt}%
\pgfsys@defobject{currentmarker}{\pgfqpoint{-0.027778in}{0.000000in}}{\pgfqpoint{0.000000in}{0.000000in}}{%
\pgfpathmoveto{\pgfqpoint{0.000000in}{0.000000in}}%
\pgfpathlineto{\pgfqpoint{-0.027778in}{0.000000in}}%
\pgfusepath{stroke,fill}%
}%
\begin{pgfscope}%
\pgfsys@transformshift{8.282041in}{7.262356in}%
\pgfsys@useobject{currentmarker}{}%
\end{pgfscope}%
\end{pgfscope}%
\begin{pgfscope}%
\pgfsetbuttcap%
\pgfsetroundjoin%
\definecolor{currentfill}{rgb}{0.000000,0.000000,0.000000}%
\pgfsetfillcolor{currentfill}%
\pgfsetlinewidth{0.602250pt}%
\definecolor{currentstroke}{rgb}{0.000000,0.000000,0.000000}%
\pgfsetstrokecolor{currentstroke}%
\pgfsetdash{}{0pt}%
\pgfsys@defobject{currentmarker}{\pgfqpoint{-0.027778in}{0.000000in}}{\pgfqpoint{0.000000in}{0.000000in}}{%
\pgfpathmoveto{\pgfqpoint{0.000000in}{0.000000in}}%
\pgfpathlineto{\pgfqpoint{-0.027778in}{0.000000in}}%
\pgfusepath{stroke,fill}%
}%
\begin{pgfscope}%
\pgfsys@transformshift{8.282041in}{7.293475in}%
\pgfsys@useobject{currentmarker}{}%
\end{pgfscope}%
\end{pgfscope}%
\begin{pgfscope}%
\pgfsetbuttcap%
\pgfsetroundjoin%
\definecolor{currentfill}{rgb}{0.000000,0.000000,0.000000}%
\pgfsetfillcolor{currentfill}%
\pgfsetlinewidth{0.602250pt}%
\definecolor{currentstroke}{rgb}{0.000000,0.000000,0.000000}%
\pgfsetstrokecolor{currentstroke}%
\pgfsetdash{}{0pt}%
\pgfsys@defobject{currentmarker}{\pgfqpoint{-0.027778in}{0.000000in}}{\pgfqpoint{0.000000in}{0.000000in}}{%
\pgfpathmoveto{\pgfqpoint{0.000000in}{0.000000in}}%
\pgfpathlineto{\pgfqpoint{-0.027778in}{0.000000in}}%
\pgfusepath{stroke,fill}%
}%
\begin{pgfscope}%
\pgfsys@transformshift{8.282041in}{7.319786in}%
\pgfsys@useobject{currentmarker}{}%
\end{pgfscope}%
\end{pgfscope}%
\begin{pgfscope}%
\pgfsetbuttcap%
\pgfsetroundjoin%
\definecolor{currentfill}{rgb}{0.000000,0.000000,0.000000}%
\pgfsetfillcolor{currentfill}%
\pgfsetlinewidth{0.602250pt}%
\definecolor{currentstroke}{rgb}{0.000000,0.000000,0.000000}%
\pgfsetstrokecolor{currentstroke}%
\pgfsetdash{}{0pt}%
\pgfsys@defobject{currentmarker}{\pgfqpoint{-0.027778in}{0.000000in}}{\pgfqpoint{0.000000in}{0.000000in}}{%
\pgfpathmoveto{\pgfqpoint{0.000000in}{0.000000in}}%
\pgfpathlineto{\pgfqpoint{-0.027778in}{0.000000in}}%
\pgfusepath{stroke,fill}%
}%
\begin{pgfscope}%
\pgfsys@transformshift{8.282041in}{7.342578in}%
\pgfsys@useobject{currentmarker}{}%
\end{pgfscope}%
\end{pgfscope}%
\begin{pgfscope}%
\pgfsetbuttcap%
\pgfsetroundjoin%
\definecolor{currentfill}{rgb}{0.000000,0.000000,0.000000}%
\pgfsetfillcolor{currentfill}%
\pgfsetlinewidth{0.602250pt}%
\definecolor{currentstroke}{rgb}{0.000000,0.000000,0.000000}%
\pgfsetstrokecolor{currentstroke}%
\pgfsetdash{}{0pt}%
\pgfsys@defobject{currentmarker}{\pgfqpoint{-0.027778in}{0.000000in}}{\pgfqpoint{0.000000in}{0.000000in}}{%
\pgfpathmoveto{\pgfqpoint{0.000000in}{0.000000in}}%
\pgfpathlineto{\pgfqpoint{-0.027778in}{0.000000in}}%
\pgfusepath{stroke,fill}%
}%
\begin{pgfscope}%
\pgfsys@transformshift{8.282041in}{7.362681in}%
\pgfsys@useobject{currentmarker}{}%
\end{pgfscope}%
\end{pgfscope}%
\begin{pgfscope}%
\pgfsetbuttcap%
\pgfsetroundjoin%
\definecolor{currentfill}{rgb}{0.000000,0.000000,0.000000}%
\pgfsetfillcolor{currentfill}%
\pgfsetlinewidth{0.602250pt}%
\definecolor{currentstroke}{rgb}{0.000000,0.000000,0.000000}%
\pgfsetstrokecolor{currentstroke}%
\pgfsetdash{}{0pt}%
\pgfsys@defobject{currentmarker}{\pgfqpoint{-0.027778in}{0.000000in}}{\pgfqpoint{0.000000in}{0.000000in}}{%
\pgfpathmoveto{\pgfqpoint{0.000000in}{0.000000in}}%
\pgfpathlineto{\pgfqpoint{-0.027778in}{0.000000in}}%
\pgfusepath{stroke,fill}%
}%
\begin{pgfscope}%
\pgfsys@transformshift{8.282041in}{7.498973in}%
\pgfsys@useobject{currentmarker}{}%
\end{pgfscope}%
\end{pgfscope}%
\begin{pgfscope}%
\pgfsetbuttcap%
\pgfsetroundjoin%
\definecolor{currentfill}{rgb}{0.000000,0.000000,0.000000}%
\pgfsetfillcolor{currentfill}%
\pgfsetlinewidth{0.602250pt}%
\definecolor{currentstroke}{rgb}{0.000000,0.000000,0.000000}%
\pgfsetstrokecolor{currentstroke}%
\pgfsetdash{}{0pt}%
\pgfsys@defobject{currentmarker}{\pgfqpoint{-0.027778in}{0.000000in}}{\pgfqpoint{0.000000in}{0.000000in}}{%
\pgfpathmoveto{\pgfqpoint{0.000000in}{0.000000in}}%
\pgfpathlineto{\pgfqpoint{-0.027778in}{0.000000in}}%
\pgfusepath{stroke,fill}%
}%
\begin{pgfscope}%
\pgfsys@transformshift{8.282041in}{7.568179in}%
\pgfsys@useobject{currentmarker}{}%
\end{pgfscope}%
\end{pgfscope}%
\begin{pgfscope}%
\pgfsetbuttcap%
\pgfsetroundjoin%
\definecolor{currentfill}{rgb}{0.000000,0.000000,0.000000}%
\pgfsetfillcolor{currentfill}%
\pgfsetlinewidth{0.602250pt}%
\definecolor{currentstroke}{rgb}{0.000000,0.000000,0.000000}%
\pgfsetstrokecolor{currentstroke}%
\pgfsetdash{}{0pt}%
\pgfsys@defobject{currentmarker}{\pgfqpoint{-0.027778in}{0.000000in}}{\pgfqpoint{0.000000in}{0.000000in}}{%
\pgfpathmoveto{\pgfqpoint{0.000000in}{0.000000in}}%
\pgfpathlineto{\pgfqpoint{-0.027778in}{0.000000in}}%
\pgfusepath{stroke,fill}%
}%
\begin{pgfscope}%
\pgfsys@transformshift{8.282041in}{7.617281in}%
\pgfsys@useobject{currentmarker}{}%
\end{pgfscope}%
\end{pgfscope}%
\begin{pgfscope}%
\pgfsetbuttcap%
\pgfsetroundjoin%
\definecolor{currentfill}{rgb}{0.000000,0.000000,0.000000}%
\pgfsetfillcolor{currentfill}%
\pgfsetlinewidth{0.602250pt}%
\definecolor{currentstroke}{rgb}{0.000000,0.000000,0.000000}%
\pgfsetstrokecolor{currentstroke}%
\pgfsetdash{}{0pt}%
\pgfsys@defobject{currentmarker}{\pgfqpoint{-0.027778in}{0.000000in}}{\pgfqpoint{0.000000in}{0.000000in}}{%
\pgfpathmoveto{\pgfqpoint{0.000000in}{0.000000in}}%
\pgfpathlineto{\pgfqpoint{-0.027778in}{0.000000in}}%
\pgfusepath{stroke,fill}%
}%
\begin{pgfscope}%
\pgfsys@transformshift{8.282041in}{7.655368in}%
\pgfsys@useobject{currentmarker}{}%
\end{pgfscope}%
\end{pgfscope}%
\begin{pgfscope}%
\pgfsetbuttcap%
\pgfsetroundjoin%
\definecolor{currentfill}{rgb}{0.000000,0.000000,0.000000}%
\pgfsetfillcolor{currentfill}%
\pgfsetlinewidth{0.602250pt}%
\definecolor{currentstroke}{rgb}{0.000000,0.000000,0.000000}%
\pgfsetstrokecolor{currentstroke}%
\pgfsetdash{}{0pt}%
\pgfsys@defobject{currentmarker}{\pgfqpoint{-0.027778in}{0.000000in}}{\pgfqpoint{0.000000in}{0.000000in}}{%
\pgfpathmoveto{\pgfqpoint{0.000000in}{0.000000in}}%
\pgfpathlineto{\pgfqpoint{-0.027778in}{0.000000in}}%
\pgfusepath{stroke,fill}%
}%
\begin{pgfscope}%
\pgfsys@transformshift{8.282041in}{7.686487in}%
\pgfsys@useobject{currentmarker}{}%
\end{pgfscope}%
\end{pgfscope}%
\begin{pgfscope}%
\pgfsetbuttcap%
\pgfsetroundjoin%
\definecolor{currentfill}{rgb}{0.000000,0.000000,0.000000}%
\pgfsetfillcolor{currentfill}%
\pgfsetlinewidth{0.602250pt}%
\definecolor{currentstroke}{rgb}{0.000000,0.000000,0.000000}%
\pgfsetstrokecolor{currentstroke}%
\pgfsetdash{}{0pt}%
\pgfsys@defobject{currentmarker}{\pgfqpoint{-0.027778in}{0.000000in}}{\pgfqpoint{0.000000in}{0.000000in}}{%
\pgfpathmoveto{\pgfqpoint{0.000000in}{0.000000in}}%
\pgfpathlineto{\pgfqpoint{-0.027778in}{0.000000in}}%
\pgfusepath{stroke,fill}%
}%
\begin{pgfscope}%
\pgfsys@transformshift{8.282041in}{7.712798in}%
\pgfsys@useobject{currentmarker}{}%
\end{pgfscope}%
\end{pgfscope}%
\begin{pgfscope}%
\pgfsetbuttcap%
\pgfsetroundjoin%
\definecolor{currentfill}{rgb}{0.000000,0.000000,0.000000}%
\pgfsetfillcolor{currentfill}%
\pgfsetlinewidth{0.602250pt}%
\definecolor{currentstroke}{rgb}{0.000000,0.000000,0.000000}%
\pgfsetstrokecolor{currentstroke}%
\pgfsetdash{}{0pt}%
\pgfsys@defobject{currentmarker}{\pgfqpoint{-0.027778in}{0.000000in}}{\pgfqpoint{0.000000in}{0.000000in}}{%
\pgfpathmoveto{\pgfqpoint{0.000000in}{0.000000in}}%
\pgfpathlineto{\pgfqpoint{-0.027778in}{0.000000in}}%
\pgfusepath{stroke,fill}%
}%
\begin{pgfscope}%
\pgfsys@transformshift{8.282041in}{7.735589in}%
\pgfsys@useobject{currentmarker}{}%
\end{pgfscope}%
\end{pgfscope}%
\begin{pgfscope}%
\pgfsetbuttcap%
\pgfsetroundjoin%
\definecolor{currentfill}{rgb}{0.000000,0.000000,0.000000}%
\pgfsetfillcolor{currentfill}%
\pgfsetlinewidth{0.602250pt}%
\definecolor{currentstroke}{rgb}{0.000000,0.000000,0.000000}%
\pgfsetstrokecolor{currentstroke}%
\pgfsetdash{}{0pt}%
\pgfsys@defobject{currentmarker}{\pgfqpoint{-0.027778in}{0.000000in}}{\pgfqpoint{0.000000in}{0.000000in}}{%
\pgfpathmoveto{\pgfqpoint{0.000000in}{0.000000in}}%
\pgfpathlineto{\pgfqpoint{-0.027778in}{0.000000in}}%
\pgfusepath{stroke,fill}%
}%
\begin{pgfscope}%
\pgfsys@transformshift{8.282041in}{7.755693in}%
\pgfsys@useobject{currentmarker}{}%
\end{pgfscope}%
\end{pgfscope}%
\begin{pgfscope}%
\pgfsetbuttcap%
\pgfsetroundjoin%
\definecolor{currentfill}{rgb}{0.000000,0.000000,0.000000}%
\pgfsetfillcolor{currentfill}%
\pgfsetlinewidth{0.602250pt}%
\definecolor{currentstroke}{rgb}{0.000000,0.000000,0.000000}%
\pgfsetstrokecolor{currentstroke}%
\pgfsetdash{}{0pt}%
\pgfsys@defobject{currentmarker}{\pgfqpoint{-0.027778in}{0.000000in}}{\pgfqpoint{0.000000in}{0.000000in}}{%
\pgfpathmoveto{\pgfqpoint{0.000000in}{0.000000in}}%
\pgfpathlineto{\pgfqpoint{-0.027778in}{0.000000in}}%
\pgfusepath{stroke,fill}%
}%
\begin{pgfscope}%
\pgfsys@transformshift{8.282041in}{7.891984in}%
\pgfsys@useobject{currentmarker}{}%
\end{pgfscope}%
\end{pgfscope}%
\begin{pgfscope}%
\pgfsetbuttcap%
\pgfsetroundjoin%
\definecolor{currentfill}{rgb}{0.000000,0.000000,0.000000}%
\pgfsetfillcolor{currentfill}%
\pgfsetlinewidth{0.602250pt}%
\definecolor{currentstroke}{rgb}{0.000000,0.000000,0.000000}%
\pgfsetstrokecolor{currentstroke}%
\pgfsetdash{}{0pt}%
\pgfsys@defobject{currentmarker}{\pgfqpoint{-0.027778in}{0.000000in}}{\pgfqpoint{0.000000in}{0.000000in}}{%
\pgfpathmoveto{\pgfqpoint{0.000000in}{0.000000in}}%
\pgfpathlineto{\pgfqpoint{-0.027778in}{0.000000in}}%
\pgfusepath{stroke,fill}%
}%
\begin{pgfscope}%
\pgfsys@transformshift{8.282041in}{7.961190in}%
\pgfsys@useobject{currentmarker}{}%
\end{pgfscope}%
\end{pgfscope}%
\begin{pgfscope}%
\pgfsetbuttcap%
\pgfsetroundjoin%
\definecolor{currentfill}{rgb}{0.000000,0.000000,0.000000}%
\pgfsetfillcolor{currentfill}%
\pgfsetlinewidth{0.602250pt}%
\definecolor{currentstroke}{rgb}{0.000000,0.000000,0.000000}%
\pgfsetstrokecolor{currentstroke}%
\pgfsetdash{}{0pt}%
\pgfsys@defobject{currentmarker}{\pgfqpoint{-0.027778in}{0.000000in}}{\pgfqpoint{0.000000in}{0.000000in}}{%
\pgfpathmoveto{\pgfqpoint{0.000000in}{0.000000in}}%
\pgfpathlineto{\pgfqpoint{-0.027778in}{0.000000in}}%
\pgfusepath{stroke,fill}%
}%
\begin{pgfscope}%
\pgfsys@transformshift{8.282041in}{8.010292in}%
\pgfsys@useobject{currentmarker}{}%
\end{pgfscope}%
\end{pgfscope}%
\begin{pgfscope}%
\pgfsetbuttcap%
\pgfsetroundjoin%
\definecolor{currentfill}{rgb}{0.000000,0.000000,0.000000}%
\pgfsetfillcolor{currentfill}%
\pgfsetlinewidth{0.602250pt}%
\definecolor{currentstroke}{rgb}{0.000000,0.000000,0.000000}%
\pgfsetstrokecolor{currentstroke}%
\pgfsetdash{}{0pt}%
\pgfsys@defobject{currentmarker}{\pgfqpoint{-0.027778in}{0.000000in}}{\pgfqpoint{0.000000in}{0.000000in}}{%
\pgfpathmoveto{\pgfqpoint{0.000000in}{0.000000in}}%
\pgfpathlineto{\pgfqpoint{-0.027778in}{0.000000in}}%
\pgfusepath{stroke,fill}%
}%
\begin{pgfscope}%
\pgfsys@transformshift{8.282041in}{8.048379in}%
\pgfsys@useobject{currentmarker}{}%
\end{pgfscope}%
\end{pgfscope}%
\begin{pgfscope}%
\pgfsetbuttcap%
\pgfsetroundjoin%
\definecolor{currentfill}{rgb}{0.000000,0.000000,0.000000}%
\pgfsetfillcolor{currentfill}%
\pgfsetlinewidth{0.602250pt}%
\definecolor{currentstroke}{rgb}{0.000000,0.000000,0.000000}%
\pgfsetstrokecolor{currentstroke}%
\pgfsetdash{}{0pt}%
\pgfsys@defobject{currentmarker}{\pgfqpoint{-0.027778in}{0.000000in}}{\pgfqpoint{0.000000in}{0.000000in}}{%
\pgfpathmoveto{\pgfqpoint{0.000000in}{0.000000in}}%
\pgfpathlineto{\pgfqpoint{-0.027778in}{0.000000in}}%
\pgfusepath{stroke,fill}%
}%
\begin{pgfscope}%
\pgfsys@transformshift{8.282041in}{8.079498in}%
\pgfsys@useobject{currentmarker}{}%
\end{pgfscope}%
\end{pgfscope}%
\begin{pgfscope}%
\pgfsetbuttcap%
\pgfsetroundjoin%
\definecolor{currentfill}{rgb}{0.000000,0.000000,0.000000}%
\pgfsetfillcolor{currentfill}%
\pgfsetlinewidth{0.602250pt}%
\definecolor{currentstroke}{rgb}{0.000000,0.000000,0.000000}%
\pgfsetstrokecolor{currentstroke}%
\pgfsetdash{}{0pt}%
\pgfsys@defobject{currentmarker}{\pgfqpoint{-0.027778in}{0.000000in}}{\pgfqpoint{0.000000in}{0.000000in}}{%
\pgfpathmoveto{\pgfqpoint{0.000000in}{0.000000in}}%
\pgfpathlineto{\pgfqpoint{-0.027778in}{0.000000in}}%
\pgfusepath{stroke,fill}%
}%
\begin{pgfscope}%
\pgfsys@transformshift{8.282041in}{8.105809in}%
\pgfsys@useobject{currentmarker}{}%
\end{pgfscope}%
\end{pgfscope}%
\begin{pgfscope}%
\pgfsetbuttcap%
\pgfsetroundjoin%
\definecolor{currentfill}{rgb}{0.000000,0.000000,0.000000}%
\pgfsetfillcolor{currentfill}%
\pgfsetlinewidth{0.602250pt}%
\definecolor{currentstroke}{rgb}{0.000000,0.000000,0.000000}%
\pgfsetstrokecolor{currentstroke}%
\pgfsetdash{}{0pt}%
\pgfsys@defobject{currentmarker}{\pgfqpoint{-0.027778in}{0.000000in}}{\pgfqpoint{0.000000in}{0.000000in}}{%
\pgfpathmoveto{\pgfqpoint{0.000000in}{0.000000in}}%
\pgfpathlineto{\pgfqpoint{-0.027778in}{0.000000in}}%
\pgfusepath{stroke,fill}%
}%
\begin{pgfscope}%
\pgfsys@transformshift{8.282041in}{8.128601in}%
\pgfsys@useobject{currentmarker}{}%
\end{pgfscope}%
\end{pgfscope}%
\begin{pgfscope}%
\pgfsetbuttcap%
\pgfsetroundjoin%
\definecolor{currentfill}{rgb}{0.000000,0.000000,0.000000}%
\pgfsetfillcolor{currentfill}%
\pgfsetlinewidth{0.602250pt}%
\definecolor{currentstroke}{rgb}{0.000000,0.000000,0.000000}%
\pgfsetstrokecolor{currentstroke}%
\pgfsetdash{}{0pt}%
\pgfsys@defobject{currentmarker}{\pgfqpoint{-0.027778in}{0.000000in}}{\pgfqpoint{0.000000in}{0.000000in}}{%
\pgfpathmoveto{\pgfqpoint{0.000000in}{0.000000in}}%
\pgfpathlineto{\pgfqpoint{-0.027778in}{0.000000in}}%
\pgfusepath{stroke,fill}%
}%
\begin{pgfscope}%
\pgfsys@transformshift{8.282041in}{8.148704in}%
\pgfsys@useobject{currentmarker}{}%
\end{pgfscope}%
\end{pgfscope}%
\begin{pgfscope}%
\pgfsetbuttcap%
\pgfsetroundjoin%
\definecolor{currentfill}{rgb}{0.000000,0.000000,0.000000}%
\pgfsetfillcolor{currentfill}%
\pgfsetlinewidth{0.602250pt}%
\definecolor{currentstroke}{rgb}{0.000000,0.000000,0.000000}%
\pgfsetstrokecolor{currentstroke}%
\pgfsetdash{}{0pt}%
\pgfsys@defobject{currentmarker}{\pgfqpoint{-0.027778in}{0.000000in}}{\pgfqpoint{0.000000in}{0.000000in}}{%
\pgfpathmoveto{\pgfqpoint{0.000000in}{0.000000in}}%
\pgfpathlineto{\pgfqpoint{-0.027778in}{0.000000in}}%
\pgfusepath{stroke,fill}%
}%
\begin{pgfscope}%
\pgfsys@transformshift{8.282041in}{8.284995in}%
\pgfsys@useobject{currentmarker}{}%
\end{pgfscope}%
\end{pgfscope}%
\begin{pgfscope}%
\pgfpathrectangle{\pgfqpoint{8.282041in}{6.967719in}}{\pgfqpoint{1.897959in}{1.372727in}} %
\pgfusepath{clip}%
\pgfsetbuttcap%
\pgfsetroundjoin%
\pgfsetlinewidth{1.505625pt}%
\definecolor{currentstroke}{rgb}{1.000000,0.000000,0.000000}%
\pgfsetstrokecolor{currentstroke}%
\pgfsetdash{{5.550000pt}{2.400000pt}}{0.000000pt}%
\pgfpathmoveto{\pgfqpoint{8.368312in}{8.179801in}}%
\pgfpathlineto{\pgfqpoint{8.540853in}{8.177814in}}%
\pgfpathlineto{\pgfqpoint{8.713395in}{8.175988in}}%
\pgfpathlineto{\pgfqpoint{8.885937in}{8.174318in}}%
\pgfpathlineto{\pgfqpoint{9.058479in}{8.172802in}}%
\pgfpathlineto{\pgfqpoint{9.231020in}{8.171435in}}%
\pgfpathlineto{\pgfqpoint{9.403562in}{8.170217in}}%
\pgfpathlineto{\pgfqpoint{9.576104in}{8.169145in}}%
\pgfpathlineto{\pgfqpoint{9.748646in}{8.168217in}}%
\pgfpathlineto{\pgfqpoint{9.921187in}{8.167433in}}%
\pgfpathlineto{\pgfqpoint{10.093729in}{8.166791in}}%
\pgfusepath{stroke}%
\end{pgfscope}%
\begin{pgfscope}%
\pgfpathrectangle{\pgfqpoint{8.282041in}{6.967719in}}{\pgfqpoint{1.897959in}{1.372727in}} %
\pgfusepath{clip}%
\pgfsetbuttcap%
\pgfsetmiterjoin%
\definecolor{currentfill}{rgb}{1.000000,0.000000,0.000000}%
\pgfsetfillcolor{currentfill}%
\pgfsetlinewidth{1.003750pt}%
\definecolor{currentstroke}{rgb}{1.000000,0.000000,0.000000}%
\pgfsetstrokecolor{currentstroke}%
\pgfsetdash{}{0pt}%
\pgfsys@defobject{currentmarker}{\pgfqpoint{-0.041667in}{-0.041667in}}{\pgfqpoint{0.041667in}{0.041667in}}{%
\pgfpathmoveto{\pgfqpoint{-0.041667in}{-0.041667in}}%
\pgfpathlineto{\pgfqpoint{0.041667in}{-0.041667in}}%
\pgfpathlineto{\pgfqpoint{0.041667in}{0.041667in}}%
\pgfpathlineto{\pgfqpoint{-0.041667in}{0.041667in}}%
\pgfpathclose%
\pgfusepath{stroke,fill}%
}%
\begin{pgfscope}%
\pgfsys@transformshift{8.368312in}{8.179801in}%
\pgfsys@useobject{currentmarker}{}%
\end{pgfscope}%
\begin{pgfscope}%
\pgfsys@transformshift{8.713395in}{8.175988in}%
\pgfsys@useobject{currentmarker}{}%
\end{pgfscope}%
\begin{pgfscope}%
\pgfsys@transformshift{9.058479in}{8.172802in}%
\pgfsys@useobject{currentmarker}{}%
\end{pgfscope}%
\begin{pgfscope}%
\pgfsys@transformshift{9.403562in}{8.170217in}%
\pgfsys@useobject{currentmarker}{}%
\end{pgfscope}%
\begin{pgfscope}%
\pgfsys@transformshift{9.748646in}{8.168217in}%
\pgfsys@useobject{currentmarker}{}%
\end{pgfscope}%
\begin{pgfscope}%
\pgfsys@transformshift{10.093729in}{8.166791in}%
\pgfsys@useobject{currentmarker}{}%
\end{pgfscope}%
\end{pgfscope}%
\begin{pgfscope}%
\pgfpathrectangle{\pgfqpoint{8.282041in}{6.967719in}}{\pgfqpoint{1.897959in}{1.372727in}} %
\pgfusepath{clip}%
\pgfsetrectcap%
\pgfsetroundjoin%
\pgfsetlinewidth{1.505625pt}%
\definecolor{currentstroke}{rgb}{0.000000,0.000000,1.000000}%
\pgfsetstrokecolor{currentstroke}%
\pgfsetdash{}{0pt}%
\pgfpathmoveto{\pgfqpoint{8.368312in}{7.320746in}}%
\pgfpathlineto{\pgfqpoint{8.540853in}{7.300476in}}%
\pgfpathlineto{\pgfqpoint{8.713395in}{7.279407in}}%
\pgfpathlineto{\pgfqpoint{8.885937in}{7.257276in}}%
\pgfpathlineto{\pgfqpoint{9.058479in}{7.233777in}}%
\pgfpathlineto{\pgfqpoint{9.231020in}{7.208534in}}%
\pgfpathlineto{\pgfqpoint{9.403562in}{7.181061in}}%
\pgfpathlineto{\pgfqpoint{9.576104in}{7.150703in}}%
\pgfpathlineto{\pgfqpoint{9.748646in}{7.116523in}}%
\pgfpathlineto{\pgfqpoint{9.921187in}{7.077102in}}%
\pgfpathlineto{\pgfqpoint{10.093729in}{7.030115in}}%
\pgfusepath{stroke}%
\end{pgfscope}%
\begin{pgfscope}%
\pgfpathrectangle{\pgfqpoint{8.282041in}{6.967719in}}{\pgfqpoint{1.897959in}{1.372727in}} %
\pgfusepath{clip}%
\pgfsetbuttcap%
\pgfsetroundjoin%
\definecolor{currentfill}{rgb}{0.000000,0.000000,1.000000}%
\pgfsetfillcolor{currentfill}%
\pgfsetlinewidth{1.003750pt}%
\definecolor{currentstroke}{rgb}{0.000000,0.000000,1.000000}%
\pgfsetstrokecolor{currentstroke}%
\pgfsetdash{}{0pt}%
\pgfsys@defobject{currentmarker}{\pgfqpoint{-0.041667in}{-0.041667in}}{\pgfqpoint{0.041667in}{0.041667in}}{%
\pgfpathmoveto{\pgfqpoint{0.000000in}{-0.041667in}}%
\pgfpathcurveto{\pgfqpoint{0.011050in}{-0.041667in}}{\pgfqpoint{0.021649in}{-0.037276in}}{\pgfqpoint{0.029463in}{-0.029463in}}%
\pgfpathcurveto{\pgfqpoint{0.037276in}{-0.021649in}}{\pgfqpoint{0.041667in}{-0.011050in}}{\pgfqpoint{0.041667in}{0.000000in}}%
\pgfpathcurveto{\pgfqpoint{0.041667in}{0.011050in}}{\pgfqpoint{0.037276in}{0.021649in}}{\pgfqpoint{0.029463in}{0.029463in}}%
\pgfpathcurveto{\pgfqpoint{0.021649in}{0.037276in}}{\pgfqpoint{0.011050in}{0.041667in}}{\pgfqpoint{0.000000in}{0.041667in}}%
\pgfpathcurveto{\pgfqpoint{-0.011050in}{0.041667in}}{\pgfqpoint{-0.021649in}{0.037276in}}{\pgfqpoint{-0.029463in}{0.029463in}}%
\pgfpathcurveto{\pgfqpoint{-0.037276in}{0.021649in}}{\pgfqpoint{-0.041667in}{0.011050in}}{\pgfqpoint{-0.041667in}{0.000000in}}%
\pgfpathcurveto{\pgfqpoint{-0.041667in}{-0.011050in}}{\pgfqpoint{-0.037276in}{-0.021649in}}{\pgfqpoint{-0.029463in}{-0.029463in}}%
\pgfpathcurveto{\pgfqpoint{-0.021649in}{-0.037276in}}{\pgfqpoint{-0.011050in}{-0.041667in}}{\pgfqpoint{0.000000in}{-0.041667in}}%
\pgfpathclose%
\pgfusepath{stroke,fill}%
}%
\begin{pgfscope}%
\pgfsys@transformshift{8.368312in}{7.320746in}%
\pgfsys@useobject{currentmarker}{}%
\end{pgfscope}%
\begin{pgfscope}%
\pgfsys@transformshift{8.713395in}{7.279407in}%
\pgfsys@useobject{currentmarker}{}%
\end{pgfscope}%
\begin{pgfscope}%
\pgfsys@transformshift{9.058479in}{7.233777in}%
\pgfsys@useobject{currentmarker}{}%
\end{pgfscope}%
\begin{pgfscope}%
\pgfsys@transformshift{9.403562in}{7.181061in}%
\pgfsys@useobject{currentmarker}{}%
\end{pgfscope}%
\begin{pgfscope}%
\pgfsys@transformshift{9.748646in}{7.116523in}%
\pgfsys@useobject{currentmarker}{}%
\end{pgfscope}%
\begin{pgfscope}%
\pgfsys@transformshift{10.093729in}{7.030115in}%
\pgfsys@useobject{currentmarker}{}%
\end{pgfscope}%
\end{pgfscope}%
\begin{pgfscope}%
\pgfpathrectangle{\pgfqpoint{8.282041in}{6.967719in}}{\pgfqpoint{1.897959in}{1.372727in}} %
\pgfusepath{clip}%
\pgfsetbuttcap%
\pgfsetroundjoin%
\pgfsetlinewidth{1.505625pt}%
\definecolor{currentstroke}{rgb}{0.000000,0.750000,0.750000}%
\pgfsetstrokecolor{currentstroke}%
\pgfsetdash{{9.600000pt}{2.400000pt}{1.500000pt}{2.400000pt}}{0.000000pt}%
\pgfpathmoveto{\pgfqpoint{8.368312in}{8.136715in}}%
\pgfpathlineto{\pgfqpoint{8.540853in}{8.120963in}}%
\pgfpathlineto{\pgfqpoint{8.713395in}{8.107166in}}%
\pgfpathlineto{\pgfqpoint{8.885937in}{8.095075in}}%
\pgfpathlineto{\pgfqpoint{9.058479in}{8.084493in}}%
\pgfpathlineto{\pgfqpoint{9.231020in}{8.075264in}}%
\pgfpathlineto{\pgfqpoint{9.403562in}{8.067265in}}%
\pgfpathlineto{\pgfqpoint{9.576104in}{8.060395in}}%
\pgfpathlineto{\pgfqpoint{9.748646in}{8.054574in}}%
\pgfpathlineto{\pgfqpoint{9.921187in}{8.049737in}}%
\pgfpathlineto{\pgfqpoint{10.093729in}{8.045832in}}%
\pgfusepath{stroke}%
\end{pgfscope}%
\begin{pgfscope}%
\pgfpathrectangle{\pgfqpoint{8.282041in}{6.967719in}}{\pgfqpoint{1.897959in}{1.372727in}} %
\pgfusepath{clip}%
\pgfsetbuttcap%
\pgfsetmiterjoin%
\definecolor{currentfill}{rgb}{0.000000,0.750000,0.750000}%
\pgfsetfillcolor{currentfill}%
\pgfsetlinewidth{1.003750pt}%
\definecolor{currentstroke}{rgb}{0.000000,0.750000,0.750000}%
\pgfsetstrokecolor{currentstroke}%
\pgfsetdash{}{0pt}%
\pgfsys@defobject{currentmarker}{\pgfqpoint{-0.041667in}{-0.041667in}}{\pgfqpoint{0.041667in}{0.041667in}}{%
\pgfpathmoveto{\pgfqpoint{-0.000000in}{-0.041667in}}%
\pgfpathlineto{\pgfqpoint{0.041667in}{0.041667in}}%
\pgfpathlineto{\pgfqpoint{-0.041667in}{0.041667in}}%
\pgfpathclose%
\pgfusepath{stroke,fill}%
}%
\begin{pgfscope}%
\pgfsys@transformshift{8.368312in}{8.136715in}%
\pgfsys@useobject{currentmarker}{}%
\end{pgfscope}%
\begin{pgfscope}%
\pgfsys@transformshift{8.713395in}{8.107166in}%
\pgfsys@useobject{currentmarker}{}%
\end{pgfscope}%
\begin{pgfscope}%
\pgfsys@transformshift{9.058479in}{8.084493in}%
\pgfsys@useobject{currentmarker}{}%
\end{pgfscope}%
\begin{pgfscope}%
\pgfsys@transformshift{9.403562in}{8.067265in}%
\pgfsys@useobject{currentmarker}{}%
\end{pgfscope}%
\begin{pgfscope}%
\pgfsys@transformshift{9.748646in}{8.054574in}%
\pgfsys@useobject{currentmarker}{}%
\end{pgfscope}%
\begin{pgfscope}%
\pgfsys@transformshift{10.093729in}{8.045832in}%
\pgfsys@useobject{currentmarker}{}%
\end{pgfscope}%
\end{pgfscope}%
\begin{pgfscope}%
\pgfpathrectangle{\pgfqpoint{8.282041in}{6.967719in}}{\pgfqpoint{1.897959in}{1.372727in}} %
\pgfusepath{clip}%
\pgfsetbuttcap%
\pgfsetroundjoin%
\pgfsetlinewidth{1.505625pt}%
\definecolor{currentstroke}{rgb}{0.000000,0.000000,0.000000}%
\pgfsetstrokecolor{currentstroke}%
\pgfsetdash{{1.500000pt}{2.475000pt}}{0.000000pt}%
\pgfpathmoveto{\pgfqpoint{8.368312in}{8.278049in}}%
\pgfpathlineto{\pgfqpoint{8.540853in}{8.270877in}}%
\pgfpathlineto{\pgfqpoint{8.713395in}{8.264063in}}%
\pgfpathlineto{\pgfqpoint{8.885937in}{8.258212in}}%
\pgfpathlineto{\pgfqpoint{9.058479in}{8.253177in}}%
\pgfpathlineto{\pgfqpoint{9.231020in}{8.248847in}}%
\pgfpathlineto{\pgfqpoint{9.403562in}{8.245134in}}%
\pgfpathlineto{\pgfqpoint{9.576104in}{8.241974in}}%
\pgfpathlineto{\pgfqpoint{9.748646in}{8.239312in}}%
\pgfpathlineto{\pgfqpoint{9.921187in}{8.237109in}}%
\pgfpathlineto{\pgfqpoint{10.093729in}{8.235333in}}%
\pgfusepath{stroke}%
\end{pgfscope}%
\begin{pgfscope}%
\pgfpathrectangle{\pgfqpoint{8.282041in}{6.967719in}}{\pgfqpoint{1.897959in}{1.372727in}} %
\pgfusepath{clip}%
\pgfsetbuttcap%
\pgfsetroundjoin%
\definecolor{currentfill}{rgb}{0.000000,0.000000,0.000000}%
\pgfsetfillcolor{currentfill}%
\pgfsetlinewidth{1.003750pt}%
\definecolor{currentstroke}{rgb}{0.000000,0.000000,0.000000}%
\pgfsetstrokecolor{currentstroke}%
\pgfsetdash{}{0pt}%
\pgfsys@defobject{currentmarker}{\pgfqpoint{-0.041667in}{-0.041667in}}{\pgfqpoint{0.041667in}{0.041667in}}{%
\pgfpathmoveto{\pgfqpoint{-0.041667in}{0.000000in}}%
\pgfpathlineto{\pgfqpoint{0.041667in}{0.000000in}}%
\pgfpathmoveto{\pgfqpoint{0.000000in}{-0.041667in}}%
\pgfpathlineto{\pgfqpoint{0.000000in}{0.041667in}}%
\pgfusepath{stroke,fill}%
}%
\begin{pgfscope}%
\pgfsys@transformshift{8.368312in}{8.278049in}%
\pgfsys@useobject{currentmarker}{}%
\end{pgfscope}%
\begin{pgfscope}%
\pgfsys@transformshift{8.713395in}{8.264063in}%
\pgfsys@useobject{currentmarker}{}%
\end{pgfscope}%
\begin{pgfscope}%
\pgfsys@transformshift{9.058479in}{8.253177in}%
\pgfsys@useobject{currentmarker}{}%
\end{pgfscope}%
\begin{pgfscope}%
\pgfsys@transformshift{9.403562in}{8.245134in}%
\pgfsys@useobject{currentmarker}{}%
\end{pgfscope}%
\begin{pgfscope}%
\pgfsys@transformshift{9.748646in}{8.239312in}%
\pgfsys@useobject{currentmarker}{}%
\end{pgfscope}%
\begin{pgfscope}%
\pgfsys@transformshift{10.093729in}{8.235333in}%
\pgfsys@useobject{currentmarker}{}%
\end{pgfscope}%
\end{pgfscope}%
\begin{pgfscope}%
\pgfsetrectcap%
\pgfsetmiterjoin%
\pgfsetlinewidth{0.803000pt}%
\definecolor{currentstroke}{rgb}{0.000000,0.000000,0.000000}%
\pgfsetstrokecolor{currentstroke}%
\pgfsetdash{}{0pt}%
\pgfpathmoveto{\pgfqpoint{8.282041in}{6.967719in}}%
\pgfpathlineto{\pgfqpoint{8.282041in}{8.340446in}}%
\pgfusepath{stroke}%
\end{pgfscope}%
\begin{pgfscope}%
\pgfsetrectcap%
\pgfsetmiterjoin%
\pgfsetlinewidth{0.803000pt}%
\definecolor{currentstroke}{rgb}{0.000000,0.000000,0.000000}%
\pgfsetstrokecolor{currentstroke}%
\pgfsetdash{}{0pt}%
\pgfpathmoveto{\pgfqpoint{10.180000in}{6.967719in}}%
\pgfpathlineto{\pgfqpoint{10.180000in}{8.340446in}}%
\pgfusepath{stroke}%
\end{pgfscope}%
\begin{pgfscope}%
\pgfsetrectcap%
\pgfsetmiterjoin%
\pgfsetlinewidth{0.803000pt}%
\definecolor{currentstroke}{rgb}{0.000000,0.000000,0.000000}%
\pgfsetstrokecolor{currentstroke}%
\pgfsetdash{}{0pt}%
\pgfpathmoveto{\pgfqpoint{8.282041in}{6.967719in}}%
\pgfpathlineto{\pgfqpoint{10.180000in}{6.967719in}}%
\pgfusepath{stroke}%
\end{pgfscope}%
\begin{pgfscope}%
\pgfsetrectcap%
\pgfsetmiterjoin%
\pgfsetlinewidth{0.803000pt}%
\definecolor{currentstroke}{rgb}{0.000000,0.000000,0.000000}%
\pgfsetstrokecolor{currentstroke}%
\pgfsetdash{}{0pt}%
\pgfpathmoveto{\pgfqpoint{8.282041in}{8.340446in}}%
\pgfpathlineto{\pgfqpoint{10.180000in}{8.340446in}}%
\pgfusepath{stroke}%
\end{pgfscope}%
\begin{pgfscope}%
\pgfsetbuttcap%
\pgfsetmiterjoin%
\definecolor{currentfill}{rgb}{1.000000,1.000000,1.000000}%
\pgfsetfillcolor{currentfill}%
\pgfsetlinewidth{0.000000pt}%
\definecolor{currentstroke}{rgb}{0.000000,0.000000,0.000000}%
\pgfsetstrokecolor{currentstroke}%
\pgfsetstrokeopacity{0.000000}%
\pgfsetdash{}{0pt}%
\pgfpathmoveto{\pgfqpoint{0.880000in}{4.908628in}}%
\pgfpathlineto{\pgfqpoint{2.777959in}{4.908628in}}%
\pgfpathlineto{\pgfqpoint{2.777959in}{6.281355in}}%
\pgfpathlineto{\pgfqpoint{0.880000in}{6.281355in}}%
\pgfpathclose%
\pgfusepath{fill}%
\end{pgfscope}%
\begin{pgfscope}%
\pgfsetbuttcap%
\pgfsetroundjoin%
\definecolor{currentfill}{rgb}{0.000000,0.000000,0.000000}%
\pgfsetfillcolor{currentfill}%
\pgfsetlinewidth{0.803000pt}%
\definecolor{currentstroke}{rgb}{0.000000,0.000000,0.000000}%
\pgfsetstrokecolor{currentstroke}%
\pgfsetdash{}{0pt}%
\pgfsys@defobject{currentmarker}{\pgfqpoint{0.000000in}{-0.048611in}}{\pgfqpoint{0.000000in}{0.000000in}}{%
\pgfpathmoveto{\pgfqpoint{0.000000in}{0.000000in}}%
\pgfpathlineto{\pgfqpoint{0.000000in}{-0.048611in}}%
\pgfusepath{stroke,fill}%
}%
\begin{pgfscope}%
\pgfsys@transformshift{1.138813in}{4.908628in}%
\pgfsys@useobject{currentmarker}{}%
\end{pgfscope}%
\end{pgfscope}%
\begin{pgfscope}%
\pgftext[x=1.138813in,y=4.811405in,,top]{\rmfamily\fontsize{10.000000}{12.000000}\selectfont \(\displaystyle 0.10\)}%
\end{pgfscope}%
\begin{pgfscope}%
\pgfsetbuttcap%
\pgfsetroundjoin%
\definecolor{currentfill}{rgb}{0.000000,0.000000,0.000000}%
\pgfsetfillcolor{currentfill}%
\pgfsetlinewidth{0.803000pt}%
\definecolor{currentstroke}{rgb}{0.000000,0.000000,0.000000}%
\pgfsetstrokecolor{currentstroke}%
\pgfsetdash{}{0pt}%
\pgfsys@defobject{currentmarker}{\pgfqpoint{0.000000in}{-0.048611in}}{\pgfqpoint{0.000000in}{0.000000in}}{%
\pgfpathmoveto{\pgfqpoint{0.000000in}{0.000000in}}%
\pgfpathlineto{\pgfqpoint{0.000000in}{-0.048611in}}%
\pgfusepath{stroke,fill}%
}%
\begin{pgfscope}%
\pgfsys@transformshift{1.656438in}{4.908628in}%
\pgfsys@useobject{currentmarker}{}%
\end{pgfscope}%
\end{pgfscope}%
\begin{pgfscope}%
\pgftext[x=1.656438in,y=4.811405in,,top]{\rmfamily\fontsize{10.000000}{12.000000}\selectfont \(\displaystyle 0.15\)}%
\end{pgfscope}%
\begin{pgfscope}%
\pgfsetbuttcap%
\pgfsetroundjoin%
\definecolor{currentfill}{rgb}{0.000000,0.000000,0.000000}%
\pgfsetfillcolor{currentfill}%
\pgfsetlinewidth{0.803000pt}%
\definecolor{currentstroke}{rgb}{0.000000,0.000000,0.000000}%
\pgfsetstrokecolor{currentstroke}%
\pgfsetdash{}{0pt}%
\pgfsys@defobject{currentmarker}{\pgfqpoint{0.000000in}{-0.048611in}}{\pgfqpoint{0.000000in}{0.000000in}}{%
\pgfpathmoveto{\pgfqpoint{0.000000in}{0.000000in}}%
\pgfpathlineto{\pgfqpoint{0.000000in}{-0.048611in}}%
\pgfusepath{stroke,fill}%
}%
\begin{pgfscope}%
\pgfsys@transformshift{2.174063in}{4.908628in}%
\pgfsys@useobject{currentmarker}{}%
\end{pgfscope}%
\end{pgfscope}%
\begin{pgfscope}%
\pgftext[x=2.174063in,y=4.811405in,,top]{\rmfamily\fontsize{10.000000}{12.000000}\selectfont \(\displaystyle 0.20\)}%
\end{pgfscope}%
\begin{pgfscope}%
\pgfsetbuttcap%
\pgfsetroundjoin%
\definecolor{currentfill}{rgb}{0.000000,0.000000,0.000000}%
\pgfsetfillcolor{currentfill}%
\pgfsetlinewidth{0.803000pt}%
\definecolor{currentstroke}{rgb}{0.000000,0.000000,0.000000}%
\pgfsetstrokecolor{currentstroke}%
\pgfsetdash{}{0pt}%
\pgfsys@defobject{currentmarker}{\pgfqpoint{0.000000in}{-0.048611in}}{\pgfqpoint{0.000000in}{0.000000in}}{%
\pgfpathmoveto{\pgfqpoint{0.000000in}{0.000000in}}%
\pgfpathlineto{\pgfqpoint{0.000000in}{-0.048611in}}%
\pgfusepath{stroke,fill}%
}%
\begin{pgfscope}%
\pgfsys@transformshift{2.691688in}{4.908628in}%
\pgfsys@useobject{currentmarker}{}%
\end{pgfscope}%
\end{pgfscope}%
\begin{pgfscope}%
\pgftext[x=2.691688in,y=4.811405in,,top]{\rmfamily\fontsize{10.000000}{12.000000}\selectfont \(\displaystyle 0.25\)}%
\end{pgfscope}%
\begin{pgfscope}%
\pgfsetbuttcap%
\pgfsetroundjoin%
\definecolor{currentfill}{rgb}{0.000000,0.000000,0.000000}%
\pgfsetfillcolor{currentfill}%
\pgfsetlinewidth{0.803000pt}%
\definecolor{currentstroke}{rgb}{0.000000,0.000000,0.000000}%
\pgfsetstrokecolor{currentstroke}%
\pgfsetdash{}{0pt}%
\pgfsys@defobject{currentmarker}{\pgfqpoint{-0.048611in}{0.000000in}}{\pgfqpoint{0.000000in}{0.000000in}}{%
\pgfpathmoveto{\pgfqpoint{0.000000in}{0.000000in}}%
\pgfpathlineto{\pgfqpoint{-0.048611in}{0.000000in}}%
\pgfusepath{stroke,fill}%
}%
\begin{pgfscope}%
\pgfsys@transformshift{0.880000in}{5.127687in}%
\pgfsys@useobject{currentmarker}{}%
\end{pgfscope}%
\end{pgfscope}%
\begin{pgfscope}%
\pgftext[x=0.494775in,y=5.074926in,left,base]{\rmfamily\fontsize{10.000000}{12.000000}\selectfont \(\displaystyle 10^{-7}\)}%
\end{pgfscope}%
\begin{pgfscope}%
\pgfsetbuttcap%
\pgfsetroundjoin%
\definecolor{currentfill}{rgb}{0.000000,0.000000,0.000000}%
\pgfsetfillcolor{currentfill}%
\pgfsetlinewidth{0.803000pt}%
\definecolor{currentstroke}{rgb}{0.000000,0.000000,0.000000}%
\pgfsetstrokecolor{currentstroke}%
\pgfsetdash{}{0pt}%
\pgfsys@defobject{currentmarker}{\pgfqpoint{-0.048611in}{0.000000in}}{\pgfqpoint{0.000000in}{0.000000in}}{%
\pgfpathmoveto{\pgfqpoint{0.000000in}{0.000000in}}%
\pgfpathlineto{\pgfqpoint{-0.048611in}{0.000000in}}%
\pgfusepath{stroke,fill}%
}%
\begin{pgfscope}%
\pgfsys@transformshift{0.880000in}{5.587098in}%
\pgfsys@useobject{currentmarker}{}%
\end{pgfscope}%
\end{pgfscope}%
\begin{pgfscope}%
\pgftext[x=0.494775in,y=5.534337in,left,base]{\rmfamily\fontsize{10.000000}{12.000000}\selectfont \(\displaystyle 10^{-6}\)}%
\end{pgfscope}%
\begin{pgfscope}%
\pgfsetbuttcap%
\pgfsetroundjoin%
\definecolor{currentfill}{rgb}{0.000000,0.000000,0.000000}%
\pgfsetfillcolor{currentfill}%
\pgfsetlinewidth{0.803000pt}%
\definecolor{currentstroke}{rgb}{0.000000,0.000000,0.000000}%
\pgfsetstrokecolor{currentstroke}%
\pgfsetdash{}{0pt}%
\pgfsys@defobject{currentmarker}{\pgfqpoint{-0.048611in}{0.000000in}}{\pgfqpoint{0.000000in}{0.000000in}}{%
\pgfpathmoveto{\pgfqpoint{0.000000in}{0.000000in}}%
\pgfpathlineto{\pgfqpoint{-0.048611in}{0.000000in}}%
\pgfusepath{stroke,fill}%
}%
\begin{pgfscope}%
\pgfsys@transformshift{0.880000in}{6.046509in}%
\pgfsys@useobject{currentmarker}{}%
\end{pgfscope}%
\end{pgfscope}%
\begin{pgfscope}%
\pgftext[x=0.494775in,y=5.993748in,left,base]{\rmfamily\fontsize{10.000000}{12.000000}\selectfont \(\displaystyle 10^{-5}\)}%
\end{pgfscope}%
\begin{pgfscope}%
\pgfsetbuttcap%
\pgfsetroundjoin%
\definecolor{currentfill}{rgb}{0.000000,0.000000,0.000000}%
\pgfsetfillcolor{currentfill}%
\pgfsetlinewidth{0.602250pt}%
\definecolor{currentstroke}{rgb}{0.000000,0.000000,0.000000}%
\pgfsetstrokecolor{currentstroke}%
\pgfsetdash{}{0pt}%
\pgfsys@defobject{currentmarker}{\pgfqpoint{-0.027778in}{0.000000in}}{\pgfqpoint{0.000000in}{0.000000in}}{%
\pgfpathmoveto{\pgfqpoint{0.000000in}{0.000000in}}%
\pgfpathlineto{\pgfqpoint{-0.027778in}{0.000000in}}%
\pgfusepath{stroke,fill}%
}%
\begin{pgfscope}%
\pgfsys@transformshift{0.880000in}{4.944869in}%
\pgfsys@useobject{currentmarker}{}%
\end{pgfscope}%
\end{pgfscope}%
\begin{pgfscope}%
\pgfsetbuttcap%
\pgfsetroundjoin%
\definecolor{currentfill}{rgb}{0.000000,0.000000,0.000000}%
\pgfsetfillcolor{currentfill}%
\pgfsetlinewidth{0.602250pt}%
\definecolor{currentstroke}{rgb}{0.000000,0.000000,0.000000}%
\pgfsetstrokecolor{currentstroke}%
\pgfsetdash{}{0pt}%
\pgfsys@defobject{currentmarker}{\pgfqpoint{-0.027778in}{0.000000in}}{\pgfqpoint{0.000000in}{0.000000in}}{%
\pgfpathmoveto{\pgfqpoint{0.000000in}{0.000000in}}%
\pgfpathlineto{\pgfqpoint{-0.027778in}{0.000000in}}%
\pgfusepath{stroke,fill}%
}%
\begin{pgfscope}%
\pgfsys@transformshift{0.880000in}{4.989391in}%
\pgfsys@useobject{currentmarker}{}%
\end{pgfscope}%
\end{pgfscope}%
\begin{pgfscope}%
\pgfsetbuttcap%
\pgfsetroundjoin%
\definecolor{currentfill}{rgb}{0.000000,0.000000,0.000000}%
\pgfsetfillcolor{currentfill}%
\pgfsetlinewidth{0.602250pt}%
\definecolor{currentstroke}{rgb}{0.000000,0.000000,0.000000}%
\pgfsetstrokecolor{currentstroke}%
\pgfsetdash{}{0pt}%
\pgfsys@defobject{currentmarker}{\pgfqpoint{-0.027778in}{0.000000in}}{\pgfqpoint{0.000000in}{0.000000in}}{%
\pgfpathmoveto{\pgfqpoint{0.000000in}{0.000000in}}%
\pgfpathlineto{\pgfqpoint{-0.027778in}{0.000000in}}%
\pgfusepath{stroke,fill}%
}%
\begin{pgfscope}%
\pgfsys@transformshift{0.880000in}{5.025767in}%
\pgfsys@useobject{currentmarker}{}%
\end{pgfscope}%
\end{pgfscope}%
\begin{pgfscope}%
\pgfsetbuttcap%
\pgfsetroundjoin%
\definecolor{currentfill}{rgb}{0.000000,0.000000,0.000000}%
\pgfsetfillcolor{currentfill}%
\pgfsetlinewidth{0.602250pt}%
\definecolor{currentstroke}{rgb}{0.000000,0.000000,0.000000}%
\pgfsetstrokecolor{currentstroke}%
\pgfsetdash{}{0pt}%
\pgfsys@defobject{currentmarker}{\pgfqpoint{-0.027778in}{0.000000in}}{\pgfqpoint{0.000000in}{0.000000in}}{%
\pgfpathmoveto{\pgfqpoint{0.000000in}{0.000000in}}%
\pgfpathlineto{\pgfqpoint{-0.027778in}{0.000000in}}%
\pgfusepath{stroke,fill}%
}%
\begin{pgfscope}%
\pgfsys@transformshift{0.880000in}{5.056523in}%
\pgfsys@useobject{currentmarker}{}%
\end{pgfscope}%
\end{pgfscope}%
\begin{pgfscope}%
\pgfsetbuttcap%
\pgfsetroundjoin%
\definecolor{currentfill}{rgb}{0.000000,0.000000,0.000000}%
\pgfsetfillcolor{currentfill}%
\pgfsetlinewidth{0.602250pt}%
\definecolor{currentstroke}{rgb}{0.000000,0.000000,0.000000}%
\pgfsetstrokecolor{currentstroke}%
\pgfsetdash{}{0pt}%
\pgfsys@defobject{currentmarker}{\pgfqpoint{-0.027778in}{0.000000in}}{\pgfqpoint{0.000000in}{0.000000in}}{%
\pgfpathmoveto{\pgfqpoint{0.000000in}{0.000000in}}%
\pgfpathlineto{\pgfqpoint{-0.027778in}{0.000000in}}%
\pgfusepath{stroke,fill}%
}%
\begin{pgfscope}%
\pgfsys@transformshift{0.880000in}{5.083166in}%
\pgfsys@useobject{currentmarker}{}%
\end{pgfscope}%
\end{pgfscope}%
\begin{pgfscope}%
\pgfsetbuttcap%
\pgfsetroundjoin%
\definecolor{currentfill}{rgb}{0.000000,0.000000,0.000000}%
\pgfsetfillcolor{currentfill}%
\pgfsetlinewidth{0.602250pt}%
\definecolor{currentstroke}{rgb}{0.000000,0.000000,0.000000}%
\pgfsetstrokecolor{currentstroke}%
\pgfsetdash{}{0pt}%
\pgfsys@defobject{currentmarker}{\pgfqpoint{-0.027778in}{0.000000in}}{\pgfqpoint{0.000000in}{0.000000in}}{%
\pgfpathmoveto{\pgfqpoint{0.000000in}{0.000000in}}%
\pgfpathlineto{\pgfqpoint{-0.027778in}{0.000000in}}%
\pgfusepath{stroke,fill}%
}%
\begin{pgfscope}%
\pgfsys@transformshift{0.880000in}{5.106666in}%
\pgfsys@useobject{currentmarker}{}%
\end{pgfscope}%
\end{pgfscope}%
\begin{pgfscope}%
\pgfsetbuttcap%
\pgfsetroundjoin%
\definecolor{currentfill}{rgb}{0.000000,0.000000,0.000000}%
\pgfsetfillcolor{currentfill}%
\pgfsetlinewidth{0.602250pt}%
\definecolor{currentstroke}{rgb}{0.000000,0.000000,0.000000}%
\pgfsetstrokecolor{currentstroke}%
\pgfsetdash{}{0pt}%
\pgfsys@defobject{currentmarker}{\pgfqpoint{-0.027778in}{0.000000in}}{\pgfqpoint{0.000000in}{0.000000in}}{%
\pgfpathmoveto{\pgfqpoint{0.000000in}{0.000000in}}%
\pgfpathlineto{\pgfqpoint{-0.027778in}{0.000000in}}%
\pgfusepath{stroke,fill}%
}%
\begin{pgfscope}%
\pgfsys@transformshift{0.880000in}{5.265984in}%
\pgfsys@useobject{currentmarker}{}%
\end{pgfscope}%
\end{pgfscope}%
\begin{pgfscope}%
\pgfsetbuttcap%
\pgfsetroundjoin%
\definecolor{currentfill}{rgb}{0.000000,0.000000,0.000000}%
\pgfsetfillcolor{currentfill}%
\pgfsetlinewidth{0.602250pt}%
\definecolor{currentstroke}{rgb}{0.000000,0.000000,0.000000}%
\pgfsetstrokecolor{currentstroke}%
\pgfsetdash{}{0pt}%
\pgfsys@defobject{currentmarker}{\pgfqpoint{-0.027778in}{0.000000in}}{\pgfqpoint{0.000000in}{0.000000in}}{%
\pgfpathmoveto{\pgfqpoint{0.000000in}{0.000000in}}%
\pgfpathlineto{\pgfqpoint{-0.027778in}{0.000000in}}%
\pgfusepath{stroke,fill}%
}%
\begin{pgfscope}%
\pgfsys@transformshift{0.880000in}{5.346882in}%
\pgfsys@useobject{currentmarker}{}%
\end{pgfscope}%
\end{pgfscope}%
\begin{pgfscope}%
\pgfsetbuttcap%
\pgfsetroundjoin%
\definecolor{currentfill}{rgb}{0.000000,0.000000,0.000000}%
\pgfsetfillcolor{currentfill}%
\pgfsetlinewidth{0.602250pt}%
\definecolor{currentstroke}{rgb}{0.000000,0.000000,0.000000}%
\pgfsetstrokecolor{currentstroke}%
\pgfsetdash{}{0pt}%
\pgfsys@defobject{currentmarker}{\pgfqpoint{-0.027778in}{0.000000in}}{\pgfqpoint{0.000000in}{0.000000in}}{%
\pgfpathmoveto{\pgfqpoint{0.000000in}{0.000000in}}%
\pgfpathlineto{\pgfqpoint{-0.027778in}{0.000000in}}%
\pgfusepath{stroke,fill}%
}%
\begin{pgfscope}%
\pgfsys@transformshift{0.880000in}{5.404280in}%
\pgfsys@useobject{currentmarker}{}%
\end{pgfscope}%
\end{pgfscope}%
\begin{pgfscope}%
\pgfsetbuttcap%
\pgfsetroundjoin%
\definecolor{currentfill}{rgb}{0.000000,0.000000,0.000000}%
\pgfsetfillcolor{currentfill}%
\pgfsetlinewidth{0.602250pt}%
\definecolor{currentstroke}{rgb}{0.000000,0.000000,0.000000}%
\pgfsetstrokecolor{currentstroke}%
\pgfsetdash{}{0pt}%
\pgfsys@defobject{currentmarker}{\pgfqpoint{-0.027778in}{0.000000in}}{\pgfqpoint{0.000000in}{0.000000in}}{%
\pgfpathmoveto{\pgfqpoint{0.000000in}{0.000000in}}%
\pgfpathlineto{\pgfqpoint{-0.027778in}{0.000000in}}%
\pgfusepath{stroke,fill}%
}%
\begin{pgfscope}%
\pgfsys@transformshift{0.880000in}{5.448802in}%
\pgfsys@useobject{currentmarker}{}%
\end{pgfscope}%
\end{pgfscope}%
\begin{pgfscope}%
\pgfsetbuttcap%
\pgfsetroundjoin%
\definecolor{currentfill}{rgb}{0.000000,0.000000,0.000000}%
\pgfsetfillcolor{currentfill}%
\pgfsetlinewidth{0.602250pt}%
\definecolor{currentstroke}{rgb}{0.000000,0.000000,0.000000}%
\pgfsetstrokecolor{currentstroke}%
\pgfsetdash{}{0pt}%
\pgfsys@defobject{currentmarker}{\pgfqpoint{-0.027778in}{0.000000in}}{\pgfqpoint{0.000000in}{0.000000in}}{%
\pgfpathmoveto{\pgfqpoint{0.000000in}{0.000000in}}%
\pgfpathlineto{\pgfqpoint{-0.027778in}{0.000000in}}%
\pgfusepath{stroke,fill}%
}%
\begin{pgfscope}%
\pgfsys@transformshift{0.880000in}{5.485178in}%
\pgfsys@useobject{currentmarker}{}%
\end{pgfscope}%
\end{pgfscope}%
\begin{pgfscope}%
\pgfsetbuttcap%
\pgfsetroundjoin%
\definecolor{currentfill}{rgb}{0.000000,0.000000,0.000000}%
\pgfsetfillcolor{currentfill}%
\pgfsetlinewidth{0.602250pt}%
\definecolor{currentstroke}{rgb}{0.000000,0.000000,0.000000}%
\pgfsetstrokecolor{currentstroke}%
\pgfsetdash{}{0pt}%
\pgfsys@defobject{currentmarker}{\pgfqpoint{-0.027778in}{0.000000in}}{\pgfqpoint{0.000000in}{0.000000in}}{%
\pgfpathmoveto{\pgfqpoint{0.000000in}{0.000000in}}%
\pgfpathlineto{\pgfqpoint{-0.027778in}{0.000000in}}%
\pgfusepath{stroke,fill}%
}%
\begin{pgfscope}%
\pgfsys@transformshift{0.880000in}{5.515934in}%
\pgfsys@useobject{currentmarker}{}%
\end{pgfscope}%
\end{pgfscope}%
\begin{pgfscope}%
\pgfsetbuttcap%
\pgfsetroundjoin%
\definecolor{currentfill}{rgb}{0.000000,0.000000,0.000000}%
\pgfsetfillcolor{currentfill}%
\pgfsetlinewidth{0.602250pt}%
\definecolor{currentstroke}{rgb}{0.000000,0.000000,0.000000}%
\pgfsetstrokecolor{currentstroke}%
\pgfsetdash{}{0pt}%
\pgfsys@defobject{currentmarker}{\pgfqpoint{-0.027778in}{0.000000in}}{\pgfqpoint{0.000000in}{0.000000in}}{%
\pgfpathmoveto{\pgfqpoint{0.000000in}{0.000000in}}%
\pgfpathlineto{\pgfqpoint{-0.027778in}{0.000000in}}%
\pgfusepath{stroke,fill}%
}%
\begin{pgfscope}%
\pgfsys@transformshift{0.880000in}{5.542577in}%
\pgfsys@useobject{currentmarker}{}%
\end{pgfscope}%
\end{pgfscope}%
\begin{pgfscope}%
\pgfsetbuttcap%
\pgfsetroundjoin%
\definecolor{currentfill}{rgb}{0.000000,0.000000,0.000000}%
\pgfsetfillcolor{currentfill}%
\pgfsetlinewidth{0.602250pt}%
\definecolor{currentstroke}{rgb}{0.000000,0.000000,0.000000}%
\pgfsetstrokecolor{currentstroke}%
\pgfsetdash{}{0pt}%
\pgfsys@defobject{currentmarker}{\pgfqpoint{-0.027778in}{0.000000in}}{\pgfqpoint{0.000000in}{0.000000in}}{%
\pgfpathmoveto{\pgfqpoint{0.000000in}{0.000000in}}%
\pgfpathlineto{\pgfqpoint{-0.027778in}{0.000000in}}%
\pgfusepath{stroke,fill}%
}%
\begin{pgfscope}%
\pgfsys@transformshift{0.880000in}{5.566077in}%
\pgfsys@useobject{currentmarker}{}%
\end{pgfscope}%
\end{pgfscope}%
\begin{pgfscope}%
\pgfsetbuttcap%
\pgfsetroundjoin%
\definecolor{currentfill}{rgb}{0.000000,0.000000,0.000000}%
\pgfsetfillcolor{currentfill}%
\pgfsetlinewidth{0.602250pt}%
\definecolor{currentstroke}{rgb}{0.000000,0.000000,0.000000}%
\pgfsetstrokecolor{currentstroke}%
\pgfsetdash{}{0pt}%
\pgfsys@defobject{currentmarker}{\pgfqpoint{-0.027778in}{0.000000in}}{\pgfqpoint{0.000000in}{0.000000in}}{%
\pgfpathmoveto{\pgfqpoint{0.000000in}{0.000000in}}%
\pgfpathlineto{\pgfqpoint{-0.027778in}{0.000000in}}%
\pgfusepath{stroke,fill}%
}%
\begin{pgfscope}%
\pgfsys@transformshift{0.880000in}{5.725395in}%
\pgfsys@useobject{currentmarker}{}%
\end{pgfscope}%
\end{pgfscope}%
\begin{pgfscope}%
\pgfsetbuttcap%
\pgfsetroundjoin%
\definecolor{currentfill}{rgb}{0.000000,0.000000,0.000000}%
\pgfsetfillcolor{currentfill}%
\pgfsetlinewidth{0.602250pt}%
\definecolor{currentstroke}{rgb}{0.000000,0.000000,0.000000}%
\pgfsetstrokecolor{currentstroke}%
\pgfsetdash{}{0pt}%
\pgfsys@defobject{currentmarker}{\pgfqpoint{-0.027778in}{0.000000in}}{\pgfqpoint{0.000000in}{0.000000in}}{%
\pgfpathmoveto{\pgfqpoint{0.000000in}{0.000000in}}%
\pgfpathlineto{\pgfqpoint{-0.027778in}{0.000000in}}%
\pgfusepath{stroke,fill}%
}%
\begin{pgfscope}%
\pgfsys@transformshift{0.880000in}{5.806293in}%
\pgfsys@useobject{currentmarker}{}%
\end{pgfscope}%
\end{pgfscope}%
\begin{pgfscope}%
\pgfsetbuttcap%
\pgfsetroundjoin%
\definecolor{currentfill}{rgb}{0.000000,0.000000,0.000000}%
\pgfsetfillcolor{currentfill}%
\pgfsetlinewidth{0.602250pt}%
\definecolor{currentstroke}{rgb}{0.000000,0.000000,0.000000}%
\pgfsetstrokecolor{currentstroke}%
\pgfsetdash{}{0pt}%
\pgfsys@defobject{currentmarker}{\pgfqpoint{-0.027778in}{0.000000in}}{\pgfqpoint{0.000000in}{0.000000in}}{%
\pgfpathmoveto{\pgfqpoint{0.000000in}{0.000000in}}%
\pgfpathlineto{\pgfqpoint{-0.027778in}{0.000000in}}%
\pgfusepath{stroke,fill}%
}%
\begin{pgfscope}%
\pgfsys@transformshift{0.880000in}{5.863691in}%
\pgfsys@useobject{currentmarker}{}%
\end{pgfscope}%
\end{pgfscope}%
\begin{pgfscope}%
\pgfsetbuttcap%
\pgfsetroundjoin%
\definecolor{currentfill}{rgb}{0.000000,0.000000,0.000000}%
\pgfsetfillcolor{currentfill}%
\pgfsetlinewidth{0.602250pt}%
\definecolor{currentstroke}{rgb}{0.000000,0.000000,0.000000}%
\pgfsetstrokecolor{currentstroke}%
\pgfsetdash{}{0pt}%
\pgfsys@defobject{currentmarker}{\pgfqpoint{-0.027778in}{0.000000in}}{\pgfqpoint{0.000000in}{0.000000in}}{%
\pgfpathmoveto{\pgfqpoint{0.000000in}{0.000000in}}%
\pgfpathlineto{\pgfqpoint{-0.027778in}{0.000000in}}%
\pgfusepath{stroke,fill}%
}%
\begin{pgfscope}%
\pgfsys@transformshift{0.880000in}{5.908213in}%
\pgfsys@useobject{currentmarker}{}%
\end{pgfscope}%
\end{pgfscope}%
\begin{pgfscope}%
\pgfsetbuttcap%
\pgfsetroundjoin%
\definecolor{currentfill}{rgb}{0.000000,0.000000,0.000000}%
\pgfsetfillcolor{currentfill}%
\pgfsetlinewidth{0.602250pt}%
\definecolor{currentstroke}{rgb}{0.000000,0.000000,0.000000}%
\pgfsetstrokecolor{currentstroke}%
\pgfsetdash{}{0pt}%
\pgfsys@defobject{currentmarker}{\pgfqpoint{-0.027778in}{0.000000in}}{\pgfqpoint{0.000000in}{0.000000in}}{%
\pgfpathmoveto{\pgfqpoint{0.000000in}{0.000000in}}%
\pgfpathlineto{\pgfqpoint{-0.027778in}{0.000000in}}%
\pgfusepath{stroke,fill}%
}%
\begin{pgfscope}%
\pgfsys@transformshift{0.880000in}{5.944589in}%
\pgfsys@useobject{currentmarker}{}%
\end{pgfscope}%
\end{pgfscope}%
\begin{pgfscope}%
\pgfsetbuttcap%
\pgfsetroundjoin%
\definecolor{currentfill}{rgb}{0.000000,0.000000,0.000000}%
\pgfsetfillcolor{currentfill}%
\pgfsetlinewidth{0.602250pt}%
\definecolor{currentstroke}{rgb}{0.000000,0.000000,0.000000}%
\pgfsetstrokecolor{currentstroke}%
\pgfsetdash{}{0pt}%
\pgfsys@defobject{currentmarker}{\pgfqpoint{-0.027778in}{0.000000in}}{\pgfqpoint{0.000000in}{0.000000in}}{%
\pgfpathmoveto{\pgfqpoint{0.000000in}{0.000000in}}%
\pgfpathlineto{\pgfqpoint{-0.027778in}{0.000000in}}%
\pgfusepath{stroke,fill}%
}%
\begin{pgfscope}%
\pgfsys@transformshift{0.880000in}{5.975346in}%
\pgfsys@useobject{currentmarker}{}%
\end{pgfscope}%
\end{pgfscope}%
\begin{pgfscope}%
\pgfsetbuttcap%
\pgfsetroundjoin%
\definecolor{currentfill}{rgb}{0.000000,0.000000,0.000000}%
\pgfsetfillcolor{currentfill}%
\pgfsetlinewidth{0.602250pt}%
\definecolor{currentstroke}{rgb}{0.000000,0.000000,0.000000}%
\pgfsetstrokecolor{currentstroke}%
\pgfsetdash{}{0pt}%
\pgfsys@defobject{currentmarker}{\pgfqpoint{-0.027778in}{0.000000in}}{\pgfqpoint{0.000000in}{0.000000in}}{%
\pgfpathmoveto{\pgfqpoint{0.000000in}{0.000000in}}%
\pgfpathlineto{\pgfqpoint{-0.027778in}{0.000000in}}%
\pgfusepath{stroke,fill}%
}%
\begin{pgfscope}%
\pgfsys@transformshift{0.880000in}{6.001988in}%
\pgfsys@useobject{currentmarker}{}%
\end{pgfscope}%
\end{pgfscope}%
\begin{pgfscope}%
\pgfsetbuttcap%
\pgfsetroundjoin%
\definecolor{currentfill}{rgb}{0.000000,0.000000,0.000000}%
\pgfsetfillcolor{currentfill}%
\pgfsetlinewidth{0.602250pt}%
\definecolor{currentstroke}{rgb}{0.000000,0.000000,0.000000}%
\pgfsetstrokecolor{currentstroke}%
\pgfsetdash{}{0pt}%
\pgfsys@defobject{currentmarker}{\pgfqpoint{-0.027778in}{0.000000in}}{\pgfqpoint{0.000000in}{0.000000in}}{%
\pgfpathmoveto{\pgfqpoint{0.000000in}{0.000000in}}%
\pgfpathlineto{\pgfqpoint{-0.027778in}{0.000000in}}%
\pgfusepath{stroke,fill}%
}%
\begin{pgfscope}%
\pgfsys@transformshift{0.880000in}{6.025488in}%
\pgfsys@useobject{currentmarker}{}%
\end{pgfscope}%
\end{pgfscope}%
\begin{pgfscope}%
\pgfsetbuttcap%
\pgfsetroundjoin%
\definecolor{currentfill}{rgb}{0.000000,0.000000,0.000000}%
\pgfsetfillcolor{currentfill}%
\pgfsetlinewidth{0.602250pt}%
\definecolor{currentstroke}{rgb}{0.000000,0.000000,0.000000}%
\pgfsetstrokecolor{currentstroke}%
\pgfsetdash{}{0pt}%
\pgfsys@defobject{currentmarker}{\pgfqpoint{-0.027778in}{0.000000in}}{\pgfqpoint{0.000000in}{0.000000in}}{%
\pgfpathmoveto{\pgfqpoint{0.000000in}{0.000000in}}%
\pgfpathlineto{\pgfqpoint{-0.027778in}{0.000000in}}%
\pgfusepath{stroke,fill}%
}%
\begin{pgfscope}%
\pgfsys@transformshift{0.880000in}{6.184806in}%
\pgfsys@useobject{currentmarker}{}%
\end{pgfscope}%
\end{pgfscope}%
\begin{pgfscope}%
\pgfsetbuttcap%
\pgfsetroundjoin%
\definecolor{currentfill}{rgb}{0.000000,0.000000,0.000000}%
\pgfsetfillcolor{currentfill}%
\pgfsetlinewidth{0.602250pt}%
\definecolor{currentstroke}{rgb}{0.000000,0.000000,0.000000}%
\pgfsetstrokecolor{currentstroke}%
\pgfsetdash{}{0pt}%
\pgfsys@defobject{currentmarker}{\pgfqpoint{-0.027778in}{0.000000in}}{\pgfqpoint{0.000000in}{0.000000in}}{%
\pgfpathmoveto{\pgfqpoint{0.000000in}{0.000000in}}%
\pgfpathlineto{\pgfqpoint{-0.027778in}{0.000000in}}%
\pgfusepath{stroke,fill}%
}%
\begin{pgfscope}%
\pgfsys@transformshift{0.880000in}{6.265704in}%
\pgfsys@useobject{currentmarker}{}%
\end{pgfscope}%
\end{pgfscope}%
\begin{pgfscope}%
\pgfpathrectangle{\pgfqpoint{0.880000in}{4.908628in}}{\pgfqpoint{1.897959in}{1.372727in}} %
\pgfusepath{clip}%
\pgfsetbuttcap%
\pgfsetroundjoin%
\pgfsetlinewidth{1.505625pt}%
\definecolor{currentstroke}{rgb}{1.000000,0.000000,0.000000}%
\pgfsetstrokecolor{currentstroke}%
\pgfsetdash{{5.550000pt}{2.400000pt}}{0.000000pt}%
\pgfpathmoveto{\pgfqpoint{0.966271in}{6.144224in}}%
\pgfpathlineto{\pgfqpoint{1.052542in}{6.141798in}}%
\pgfpathlineto{\pgfqpoint{1.138813in}{6.139481in}}%
\pgfpathlineto{\pgfqpoint{1.225083in}{6.137270in}}%
\pgfpathlineto{\pgfqpoint{1.311354in}{6.135164in}}%
\pgfpathlineto{\pgfqpoint{1.397625in}{6.133162in}}%
\pgfpathlineto{\pgfqpoint{1.483896in}{6.131261in}}%
\pgfpathlineto{\pgfqpoint{1.570167in}{6.129461in}}%
\pgfpathlineto{\pgfqpoint{1.656438in}{6.127762in}}%
\pgfpathlineto{\pgfqpoint{1.742709in}{6.126163in}}%
\pgfpathlineto{\pgfqpoint{1.828980in}{6.124662in}}%
\pgfpathlineto{\pgfqpoint{1.915250in}{6.123260in}}%
\pgfpathlineto{\pgfqpoint{2.001521in}{6.121956in}}%
\pgfpathlineto{\pgfqpoint{2.087792in}{6.120749in}}%
\pgfpathlineto{\pgfqpoint{2.174063in}{6.119639in}}%
\pgfpathlineto{\pgfqpoint{2.260334in}{6.118626in}}%
\pgfpathlineto{\pgfqpoint{2.346605in}{6.117709in}}%
\pgfpathlineto{\pgfqpoint{2.432876in}{6.116887in}}%
\pgfpathlineto{\pgfqpoint{2.519147in}{6.116162in}}%
\pgfpathlineto{\pgfqpoint{2.605417in}{6.115531in}}%
\pgfpathlineto{\pgfqpoint{2.691688in}{6.114996in}}%
\pgfusepath{stroke}%
\end{pgfscope}%
\begin{pgfscope}%
\pgfpathrectangle{\pgfqpoint{0.880000in}{4.908628in}}{\pgfqpoint{1.897959in}{1.372727in}} %
\pgfusepath{clip}%
\pgfsetbuttcap%
\pgfsetmiterjoin%
\definecolor{currentfill}{rgb}{1.000000,0.000000,0.000000}%
\pgfsetfillcolor{currentfill}%
\pgfsetlinewidth{1.003750pt}%
\definecolor{currentstroke}{rgb}{1.000000,0.000000,0.000000}%
\pgfsetstrokecolor{currentstroke}%
\pgfsetdash{}{0pt}%
\pgfsys@defobject{currentmarker}{\pgfqpoint{-0.041667in}{-0.041667in}}{\pgfqpoint{0.041667in}{0.041667in}}{%
\pgfpathmoveto{\pgfqpoint{-0.041667in}{-0.041667in}}%
\pgfpathlineto{\pgfqpoint{0.041667in}{-0.041667in}}%
\pgfpathlineto{\pgfqpoint{0.041667in}{0.041667in}}%
\pgfpathlineto{\pgfqpoint{-0.041667in}{0.041667in}}%
\pgfpathclose%
\pgfusepath{stroke,fill}%
}%
\begin{pgfscope}%
\pgfsys@transformshift{0.966271in}{6.144224in}%
\pgfsys@useobject{currentmarker}{}%
\end{pgfscope}%
\begin{pgfscope}%
\pgfsys@transformshift{1.311354in}{6.135164in}%
\pgfsys@useobject{currentmarker}{}%
\end{pgfscope}%
\begin{pgfscope}%
\pgfsys@transformshift{1.656438in}{6.127762in}%
\pgfsys@useobject{currentmarker}{}%
\end{pgfscope}%
\begin{pgfscope}%
\pgfsys@transformshift{2.001521in}{6.121956in}%
\pgfsys@useobject{currentmarker}{}%
\end{pgfscope}%
\begin{pgfscope}%
\pgfsys@transformshift{2.346605in}{6.117709in}%
\pgfsys@useobject{currentmarker}{}%
\end{pgfscope}%
\begin{pgfscope}%
\pgfsys@transformshift{2.691688in}{6.114996in}%
\pgfsys@useobject{currentmarker}{}%
\end{pgfscope}%
\end{pgfscope}%
\begin{pgfscope}%
\pgfpathrectangle{\pgfqpoint{0.880000in}{4.908628in}}{\pgfqpoint{1.897959in}{1.372727in}} %
\pgfusepath{clip}%
\pgfsetrectcap%
\pgfsetroundjoin%
\pgfsetlinewidth{1.505625pt}%
\definecolor{currentstroke}{rgb}{0.000000,0.000000,1.000000}%
\pgfsetstrokecolor{currentstroke}%
\pgfsetdash{}{0pt}%
\pgfpathmoveto{\pgfqpoint{0.966271in}{5.385505in}}%
\pgfpathlineto{\pgfqpoint{1.052542in}{5.371885in}}%
\pgfpathlineto{\pgfqpoint{1.138813in}{5.358203in}}%
\pgfpathlineto{\pgfqpoint{1.225083in}{5.344366in}}%
\pgfpathlineto{\pgfqpoint{1.311354in}{5.330293in}}%
\pgfpathlineto{\pgfqpoint{1.397625in}{5.315905in}}%
\pgfpathlineto{\pgfqpoint{1.483896in}{5.301127in}}%
\pgfpathlineto{\pgfqpoint{1.570167in}{5.285878in}}%
\pgfpathlineto{\pgfqpoint{1.656438in}{5.270073in}}%
\pgfpathlineto{\pgfqpoint{1.742709in}{5.253616in}}%
\pgfpathlineto{\pgfqpoint{1.828980in}{5.236400in}}%
\pgfpathlineto{\pgfqpoint{1.915250in}{5.218298in}}%
\pgfpathlineto{\pgfqpoint{2.001521in}{5.199162in}}%
\pgfpathlineto{\pgfqpoint{2.087792in}{5.178811in}}%
\pgfpathlineto{\pgfqpoint{2.174063in}{5.157019in}}%
\pgfpathlineto{\pgfqpoint{2.260334in}{5.133503in}}%
\pgfpathlineto{\pgfqpoint{2.346605in}{5.107892in}}%
\pgfpathlineto{\pgfqpoint{2.432876in}{5.079695in}}%
\pgfpathlineto{\pgfqpoint{2.519147in}{5.048227in}}%
\pgfpathlineto{\pgfqpoint{2.605417in}{5.012504in}}%
\pgfpathlineto{\pgfqpoint{2.691688in}{4.971024in}}%
\pgfusepath{stroke}%
\end{pgfscope}%
\begin{pgfscope}%
\pgfpathrectangle{\pgfqpoint{0.880000in}{4.908628in}}{\pgfqpoint{1.897959in}{1.372727in}} %
\pgfusepath{clip}%
\pgfsetbuttcap%
\pgfsetroundjoin%
\definecolor{currentfill}{rgb}{0.000000,0.000000,1.000000}%
\pgfsetfillcolor{currentfill}%
\pgfsetlinewidth{1.003750pt}%
\definecolor{currentstroke}{rgb}{0.000000,0.000000,1.000000}%
\pgfsetstrokecolor{currentstroke}%
\pgfsetdash{}{0pt}%
\pgfsys@defobject{currentmarker}{\pgfqpoint{-0.041667in}{-0.041667in}}{\pgfqpoint{0.041667in}{0.041667in}}{%
\pgfpathmoveto{\pgfqpoint{0.000000in}{-0.041667in}}%
\pgfpathcurveto{\pgfqpoint{0.011050in}{-0.041667in}}{\pgfqpoint{0.021649in}{-0.037276in}}{\pgfqpoint{0.029463in}{-0.029463in}}%
\pgfpathcurveto{\pgfqpoint{0.037276in}{-0.021649in}}{\pgfqpoint{0.041667in}{-0.011050in}}{\pgfqpoint{0.041667in}{0.000000in}}%
\pgfpathcurveto{\pgfqpoint{0.041667in}{0.011050in}}{\pgfqpoint{0.037276in}{0.021649in}}{\pgfqpoint{0.029463in}{0.029463in}}%
\pgfpathcurveto{\pgfqpoint{0.021649in}{0.037276in}}{\pgfqpoint{0.011050in}{0.041667in}}{\pgfqpoint{0.000000in}{0.041667in}}%
\pgfpathcurveto{\pgfqpoint{-0.011050in}{0.041667in}}{\pgfqpoint{-0.021649in}{0.037276in}}{\pgfqpoint{-0.029463in}{0.029463in}}%
\pgfpathcurveto{\pgfqpoint{-0.037276in}{0.021649in}}{\pgfqpoint{-0.041667in}{0.011050in}}{\pgfqpoint{-0.041667in}{0.000000in}}%
\pgfpathcurveto{\pgfqpoint{-0.041667in}{-0.011050in}}{\pgfqpoint{-0.037276in}{-0.021649in}}{\pgfqpoint{-0.029463in}{-0.029463in}}%
\pgfpathcurveto{\pgfqpoint{-0.021649in}{-0.037276in}}{\pgfqpoint{-0.011050in}{-0.041667in}}{\pgfqpoint{0.000000in}{-0.041667in}}%
\pgfpathclose%
\pgfusepath{stroke,fill}%
}%
\begin{pgfscope}%
\pgfsys@transformshift{0.966271in}{5.385505in}%
\pgfsys@useobject{currentmarker}{}%
\end{pgfscope}%
\begin{pgfscope}%
\pgfsys@transformshift{1.311354in}{5.330293in}%
\pgfsys@useobject{currentmarker}{}%
\end{pgfscope}%
\begin{pgfscope}%
\pgfsys@transformshift{1.656438in}{5.270073in}%
\pgfsys@useobject{currentmarker}{}%
\end{pgfscope}%
\begin{pgfscope}%
\pgfsys@transformshift{2.001521in}{5.199162in}%
\pgfsys@useobject{currentmarker}{}%
\end{pgfscope}%
\begin{pgfscope}%
\pgfsys@transformshift{2.346605in}{5.107892in}%
\pgfsys@useobject{currentmarker}{}%
\end{pgfscope}%
\begin{pgfscope}%
\pgfsys@transformshift{2.691688in}{4.971024in}%
\pgfsys@useobject{currentmarker}{}%
\end{pgfscope}%
\end{pgfscope}%
\begin{pgfscope}%
\pgfpathrectangle{\pgfqpoint{0.880000in}{4.908628in}}{\pgfqpoint{1.897959in}{1.372727in}} %
\pgfusepath{clip}%
\pgfsetbuttcap%
\pgfsetroundjoin%
\pgfsetlinewidth{1.505625pt}%
\definecolor{currentstroke}{rgb}{0.000000,0.750000,0.750000}%
\pgfsetstrokecolor{currentstroke}%
\pgfsetdash{{9.600000pt}{2.400000pt}{1.500000pt}{2.400000pt}}{0.000000pt}%
\pgfpathmoveto{\pgfqpoint{0.966271in}{5.978793in}}%
\pgfpathlineto{\pgfqpoint{1.052542in}{5.952688in}}%
\pgfpathlineto{\pgfqpoint{1.138813in}{5.929403in}}%
\pgfpathlineto{\pgfqpoint{1.225083in}{5.908512in}}%
\pgfpathlineto{\pgfqpoint{1.311354in}{5.889682in}}%
\pgfpathlineto{\pgfqpoint{1.397625in}{5.872650in}}%
\pgfpathlineto{\pgfqpoint{1.483896in}{5.857205in}}%
\pgfpathlineto{\pgfqpoint{1.570167in}{5.843174in}}%
\pgfpathlineto{\pgfqpoint{1.656438in}{5.830413in}}%
\pgfpathlineto{\pgfqpoint{1.742709in}{5.818804in}}%
\pgfpathlineto{\pgfqpoint{1.828980in}{5.808246in}}%
\pgfpathlineto{\pgfqpoint{1.915250in}{5.798656in}}%
\pgfpathlineto{\pgfqpoint{2.001521in}{5.789961in}}%
\pgfpathlineto{\pgfqpoint{2.087792in}{5.782101in}}%
\pgfpathlineto{\pgfqpoint{2.174063in}{5.775022in}}%
\pgfpathlineto{\pgfqpoint{2.260334in}{5.768679in}}%
\pgfpathlineto{\pgfqpoint{2.346605in}{5.763034in}}%
\pgfpathlineto{\pgfqpoint{2.432876in}{5.758053in}}%
\pgfpathlineto{\pgfqpoint{2.519147in}{5.753710in}}%
\pgfpathlineto{\pgfqpoint{2.605417in}{5.749978in}}%
\pgfpathlineto{\pgfqpoint{2.691688in}{5.746840in}}%
\pgfusepath{stroke}%
\end{pgfscope}%
\begin{pgfscope}%
\pgfpathrectangle{\pgfqpoint{0.880000in}{4.908628in}}{\pgfqpoint{1.897959in}{1.372727in}} %
\pgfusepath{clip}%
\pgfsetbuttcap%
\pgfsetmiterjoin%
\definecolor{currentfill}{rgb}{0.000000,0.750000,0.750000}%
\pgfsetfillcolor{currentfill}%
\pgfsetlinewidth{1.003750pt}%
\definecolor{currentstroke}{rgb}{0.000000,0.750000,0.750000}%
\pgfsetstrokecolor{currentstroke}%
\pgfsetdash{}{0pt}%
\pgfsys@defobject{currentmarker}{\pgfqpoint{-0.041667in}{-0.041667in}}{\pgfqpoint{0.041667in}{0.041667in}}{%
\pgfpathmoveto{\pgfqpoint{-0.000000in}{-0.041667in}}%
\pgfpathlineto{\pgfqpoint{0.041667in}{0.041667in}}%
\pgfpathlineto{\pgfqpoint{-0.041667in}{0.041667in}}%
\pgfpathclose%
\pgfusepath{stroke,fill}%
}%
\begin{pgfscope}%
\pgfsys@transformshift{0.966271in}{5.978793in}%
\pgfsys@useobject{currentmarker}{}%
\end{pgfscope}%
\begin{pgfscope}%
\pgfsys@transformshift{1.311354in}{5.889682in}%
\pgfsys@useobject{currentmarker}{}%
\end{pgfscope}%
\begin{pgfscope}%
\pgfsys@transformshift{1.656438in}{5.830413in}%
\pgfsys@useobject{currentmarker}{}%
\end{pgfscope}%
\begin{pgfscope}%
\pgfsys@transformshift{2.001521in}{5.789961in}%
\pgfsys@useobject{currentmarker}{}%
\end{pgfscope}%
\begin{pgfscope}%
\pgfsys@transformshift{2.346605in}{5.763034in}%
\pgfsys@useobject{currentmarker}{}%
\end{pgfscope}%
\begin{pgfscope}%
\pgfsys@transformshift{2.691688in}{5.746840in}%
\pgfsys@useobject{currentmarker}{}%
\end{pgfscope}%
\end{pgfscope}%
\begin{pgfscope}%
\pgfpathrectangle{\pgfqpoint{0.880000in}{4.908628in}}{\pgfqpoint{1.897959in}{1.372727in}} %
\pgfusepath{clip}%
\pgfsetbuttcap%
\pgfsetroundjoin%
\pgfsetlinewidth{1.505625pt}%
\definecolor{currentstroke}{rgb}{0.000000,0.000000,0.000000}%
\pgfsetstrokecolor{currentstroke}%
\pgfsetdash{{1.500000pt}{2.475000pt}}{0.000000pt}%
\pgfpathmoveto{\pgfqpoint{0.966271in}{6.218958in}}%
\pgfpathlineto{\pgfqpoint{1.052542in}{6.210451in}}%
\pgfpathlineto{\pgfqpoint{1.138813in}{6.202347in}}%
\pgfpathlineto{\pgfqpoint{1.225083in}{6.195324in}}%
\pgfpathlineto{\pgfqpoint{1.311354in}{6.189167in}}%
\pgfpathlineto{\pgfqpoint{1.397625in}{6.183719in}}%
\pgfpathlineto{\pgfqpoint{1.483896in}{6.178860in}}%
\pgfpathlineto{\pgfqpoint{1.570167in}{6.174503in}}%
\pgfpathlineto{\pgfqpoint{1.656438in}{6.170576in}}%
\pgfpathlineto{\pgfqpoint{1.742709in}{6.167025in}}%
\pgfpathlineto{\pgfqpoint{1.828980in}{6.163808in}}%
\pgfpathlineto{\pgfqpoint{1.915250in}{6.160890in}}%
\pgfpathlineto{\pgfqpoint{2.001521in}{6.158244in}}%
\pgfpathlineto{\pgfqpoint{2.087792in}{6.155847in}}%
\pgfpathlineto{\pgfqpoint{2.174063in}{6.153681in}}%
\pgfpathlineto{\pgfqpoint{2.260334in}{6.151730in}}%
\pgfpathlineto{\pgfqpoint{2.346605in}{6.149983in}}%
\pgfpathlineto{\pgfqpoint{2.432876in}{6.148428in}}%
\pgfpathlineto{\pgfqpoint{2.519147in}{6.147058in}}%
\pgfpathlineto{\pgfqpoint{2.605417in}{6.145867in}}%
\pgfpathlineto{\pgfqpoint{2.691688in}{6.144848in}}%
\pgfusepath{stroke}%
\end{pgfscope}%
\begin{pgfscope}%
\pgfpathrectangle{\pgfqpoint{0.880000in}{4.908628in}}{\pgfqpoint{1.897959in}{1.372727in}} %
\pgfusepath{clip}%
\pgfsetbuttcap%
\pgfsetroundjoin%
\definecolor{currentfill}{rgb}{0.000000,0.000000,0.000000}%
\pgfsetfillcolor{currentfill}%
\pgfsetlinewidth{1.003750pt}%
\definecolor{currentstroke}{rgb}{0.000000,0.000000,0.000000}%
\pgfsetstrokecolor{currentstroke}%
\pgfsetdash{}{0pt}%
\pgfsys@defobject{currentmarker}{\pgfqpoint{-0.041667in}{-0.041667in}}{\pgfqpoint{0.041667in}{0.041667in}}{%
\pgfpathmoveto{\pgfqpoint{-0.041667in}{0.000000in}}%
\pgfpathlineto{\pgfqpoint{0.041667in}{0.000000in}}%
\pgfpathmoveto{\pgfqpoint{0.000000in}{-0.041667in}}%
\pgfpathlineto{\pgfqpoint{0.000000in}{0.041667in}}%
\pgfusepath{stroke,fill}%
}%
\begin{pgfscope}%
\pgfsys@transformshift{0.966271in}{6.218958in}%
\pgfsys@useobject{currentmarker}{}%
\end{pgfscope}%
\begin{pgfscope}%
\pgfsys@transformshift{1.311354in}{6.189167in}%
\pgfsys@useobject{currentmarker}{}%
\end{pgfscope}%
\begin{pgfscope}%
\pgfsys@transformshift{1.656438in}{6.170576in}%
\pgfsys@useobject{currentmarker}{}%
\end{pgfscope}%
\begin{pgfscope}%
\pgfsys@transformshift{2.001521in}{6.158244in}%
\pgfsys@useobject{currentmarker}{}%
\end{pgfscope}%
\begin{pgfscope}%
\pgfsys@transformshift{2.346605in}{6.149983in}%
\pgfsys@useobject{currentmarker}{}%
\end{pgfscope}%
\begin{pgfscope}%
\pgfsys@transformshift{2.691688in}{6.144848in}%
\pgfsys@useobject{currentmarker}{}%
\end{pgfscope}%
\end{pgfscope}%
\begin{pgfscope}%
\pgfsetrectcap%
\pgfsetmiterjoin%
\pgfsetlinewidth{0.803000pt}%
\definecolor{currentstroke}{rgb}{0.000000,0.000000,0.000000}%
\pgfsetstrokecolor{currentstroke}%
\pgfsetdash{}{0pt}%
\pgfpathmoveto{\pgfqpoint{0.880000in}{4.908628in}}%
\pgfpathlineto{\pgfqpoint{0.880000in}{6.281355in}}%
\pgfusepath{stroke}%
\end{pgfscope}%
\begin{pgfscope}%
\pgfsetrectcap%
\pgfsetmiterjoin%
\pgfsetlinewidth{0.803000pt}%
\definecolor{currentstroke}{rgb}{0.000000,0.000000,0.000000}%
\pgfsetstrokecolor{currentstroke}%
\pgfsetdash{}{0pt}%
\pgfpathmoveto{\pgfqpoint{2.777959in}{4.908628in}}%
\pgfpathlineto{\pgfqpoint{2.777959in}{6.281355in}}%
\pgfusepath{stroke}%
\end{pgfscope}%
\begin{pgfscope}%
\pgfsetrectcap%
\pgfsetmiterjoin%
\pgfsetlinewidth{0.803000pt}%
\definecolor{currentstroke}{rgb}{0.000000,0.000000,0.000000}%
\pgfsetstrokecolor{currentstroke}%
\pgfsetdash{}{0pt}%
\pgfpathmoveto{\pgfqpoint{0.880000in}{4.908628in}}%
\pgfpathlineto{\pgfqpoint{2.777959in}{4.908628in}}%
\pgfusepath{stroke}%
\end{pgfscope}%
\begin{pgfscope}%
\pgfsetrectcap%
\pgfsetmiterjoin%
\pgfsetlinewidth{0.803000pt}%
\definecolor{currentstroke}{rgb}{0.000000,0.000000,0.000000}%
\pgfsetstrokecolor{currentstroke}%
\pgfsetdash{}{0pt}%
\pgfpathmoveto{\pgfqpoint{0.880000in}{6.281355in}}%
\pgfpathlineto{\pgfqpoint{2.777959in}{6.281355in}}%
\pgfusepath{stroke}%
\end{pgfscope}%
\begin{pgfscope}%
\pgfsetbuttcap%
\pgfsetmiterjoin%
\definecolor{currentfill}{rgb}{1.000000,1.000000,1.000000}%
\pgfsetfillcolor{currentfill}%
\pgfsetlinewidth{0.000000pt}%
\definecolor{currentstroke}{rgb}{0.000000,0.000000,0.000000}%
\pgfsetstrokecolor{currentstroke}%
\pgfsetstrokeopacity{0.000000}%
\pgfsetdash{}{0pt}%
\pgfpathmoveto{\pgfqpoint{3.347347in}{4.908628in}}%
\pgfpathlineto{\pgfqpoint{5.245306in}{4.908628in}}%
\pgfpathlineto{\pgfqpoint{5.245306in}{6.281355in}}%
\pgfpathlineto{\pgfqpoint{3.347347in}{6.281355in}}%
\pgfpathclose%
\pgfusepath{fill}%
\end{pgfscope}%
\begin{pgfscope}%
\pgfsetbuttcap%
\pgfsetroundjoin%
\definecolor{currentfill}{rgb}{0.000000,0.000000,0.000000}%
\pgfsetfillcolor{currentfill}%
\pgfsetlinewidth{0.803000pt}%
\definecolor{currentstroke}{rgb}{0.000000,0.000000,0.000000}%
\pgfsetstrokecolor{currentstroke}%
\pgfsetdash{}{0pt}%
\pgfsys@defobject{currentmarker}{\pgfqpoint{0.000000in}{-0.048611in}}{\pgfqpoint{0.000000in}{0.000000in}}{%
\pgfpathmoveto{\pgfqpoint{0.000000in}{0.000000in}}%
\pgfpathlineto{\pgfqpoint{0.000000in}{-0.048611in}}%
\pgfusepath{stroke,fill}%
}%
\begin{pgfscope}%
\pgfsys@transformshift{3.721187in}{4.908628in}%
\pgfsys@useobject{currentmarker}{}%
\end{pgfscope}%
\end{pgfscope}%
\begin{pgfscope}%
\pgftext[x=3.721187in,y=4.811405in,,top]{\rmfamily\fontsize{10.000000}{12.000000}\selectfont \(\displaystyle 0.1\)}%
\end{pgfscope}%
\begin{pgfscope}%
\pgfsetbuttcap%
\pgfsetroundjoin%
\definecolor{currentfill}{rgb}{0.000000,0.000000,0.000000}%
\pgfsetfillcolor{currentfill}%
\pgfsetlinewidth{0.803000pt}%
\definecolor{currentstroke}{rgb}{0.000000,0.000000,0.000000}%
\pgfsetstrokecolor{currentstroke}%
\pgfsetdash{}{0pt}%
\pgfsys@defobject{currentmarker}{\pgfqpoint{0.000000in}{-0.048611in}}{\pgfqpoint{0.000000in}{0.000000in}}{%
\pgfpathmoveto{\pgfqpoint{0.000000in}{0.000000in}}%
\pgfpathlineto{\pgfqpoint{0.000000in}{-0.048611in}}%
\pgfusepath{stroke,fill}%
}%
\begin{pgfscope}%
\pgfsys@transformshift{4.583896in}{4.908628in}%
\pgfsys@useobject{currentmarker}{}%
\end{pgfscope}%
\end{pgfscope}%
\begin{pgfscope}%
\pgftext[x=4.583896in,y=4.811405in,,top]{\rmfamily\fontsize{10.000000}{12.000000}\selectfont \(\displaystyle 0.2\)}%
\end{pgfscope}%
\begin{pgfscope}%
\pgfsetbuttcap%
\pgfsetroundjoin%
\definecolor{currentfill}{rgb}{0.000000,0.000000,0.000000}%
\pgfsetfillcolor{currentfill}%
\pgfsetlinewidth{0.803000pt}%
\definecolor{currentstroke}{rgb}{0.000000,0.000000,0.000000}%
\pgfsetstrokecolor{currentstroke}%
\pgfsetdash{}{0pt}%
\pgfsys@defobject{currentmarker}{\pgfqpoint{-0.048611in}{0.000000in}}{\pgfqpoint{0.000000in}{0.000000in}}{%
\pgfpathmoveto{\pgfqpoint{0.000000in}{0.000000in}}%
\pgfpathlineto{\pgfqpoint{-0.048611in}{0.000000in}}%
\pgfusepath{stroke,fill}%
}%
\begin{pgfscope}%
\pgfsys@transformshift{3.347347in}{5.277214in}%
\pgfsys@useobject{currentmarker}{}%
\end{pgfscope}%
\end{pgfscope}%
\begin{pgfscope}%
\pgftext[x=2.962122in,y=5.224452in,left,base]{\rmfamily\fontsize{10.000000}{12.000000}\selectfont \(\displaystyle 10^{-7}\)}%
\end{pgfscope}%
\begin{pgfscope}%
\pgfsetbuttcap%
\pgfsetroundjoin%
\definecolor{currentfill}{rgb}{0.000000,0.000000,0.000000}%
\pgfsetfillcolor{currentfill}%
\pgfsetlinewidth{0.803000pt}%
\definecolor{currentstroke}{rgb}{0.000000,0.000000,0.000000}%
\pgfsetstrokecolor{currentstroke}%
\pgfsetdash{}{0pt}%
\pgfsys@defobject{currentmarker}{\pgfqpoint{-0.048611in}{0.000000in}}{\pgfqpoint{0.000000in}{0.000000in}}{%
\pgfpathmoveto{\pgfqpoint{0.000000in}{0.000000in}}%
\pgfpathlineto{\pgfqpoint{-0.048611in}{0.000000in}}%
\pgfusepath{stroke,fill}%
}%
\begin{pgfscope}%
\pgfsys@transformshift{3.347347in}{5.783107in}%
\pgfsys@useobject{currentmarker}{}%
\end{pgfscope}%
\end{pgfscope}%
\begin{pgfscope}%
\pgftext[x=2.962122in,y=5.730345in,left,base]{\rmfamily\fontsize{10.000000}{12.000000}\selectfont \(\displaystyle 10^{-6}\)}%
\end{pgfscope}%
\begin{pgfscope}%
\pgfsetbuttcap%
\pgfsetroundjoin%
\definecolor{currentfill}{rgb}{0.000000,0.000000,0.000000}%
\pgfsetfillcolor{currentfill}%
\pgfsetlinewidth{0.602250pt}%
\definecolor{currentstroke}{rgb}{0.000000,0.000000,0.000000}%
\pgfsetstrokecolor{currentstroke}%
\pgfsetdash{}{0pt}%
\pgfsys@defobject{currentmarker}{\pgfqpoint{-0.027778in}{0.000000in}}{\pgfqpoint{0.000000in}{0.000000in}}{%
\pgfpathmoveto{\pgfqpoint{0.000000in}{0.000000in}}%
\pgfpathlineto{\pgfqpoint{-0.027778in}{0.000000in}}%
\pgfusepath{stroke,fill}%
}%
\begin{pgfscope}%
\pgfsys@transformshift{3.347347in}{4.923609in}%
\pgfsys@useobject{currentmarker}{}%
\end{pgfscope}%
\end{pgfscope}%
\begin{pgfscope}%
\pgfsetbuttcap%
\pgfsetroundjoin%
\definecolor{currentfill}{rgb}{0.000000,0.000000,0.000000}%
\pgfsetfillcolor{currentfill}%
\pgfsetlinewidth{0.602250pt}%
\definecolor{currentstroke}{rgb}{0.000000,0.000000,0.000000}%
\pgfsetstrokecolor{currentstroke}%
\pgfsetdash{}{0pt}%
\pgfsys@defobject{currentmarker}{\pgfqpoint{-0.027778in}{0.000000in}}{\pgfqpoint{0.000000in}{0.000000in}}{%
\pgfpathmoveto{\pgfqpoint{0.000000in}{0.000000in}}%
\pgfpathlineto{\pgfqpoint{-0.027778in}{0.000000in}}%
\pgfusepath{stroke,fill}%
}%
\begin{pgfscope}%
\pgfsys@transformshift{3.347347in}{5.012693in}%
\pgfsys@useobject{currentmarker}{}%
\end{pgfscope}%
\end{pgfscope}%
\begin{pgfscope}%
\pgfsetbuttcap%
\pgfsetroundjoin%
\definecolor{currentfill}{rgb}{0.000000,0.000000,0.000000}%
\pgfsetfillcolor{currentfill}%
\pgfsetlinewidth{0.602250pt}%
\definecolor{currentstroke}{rgb}{0.000000,0.000000,0.000000}%
\pgfsetstrokecolor{currentstroke}%
\pgfsetdash{}{0pt}%
\pgfsys@defobject{currentmarker}{\pgfqpoint{-0.027778in}{0.000000in}}{\pgfqpoint{0.000000in}{0.000000in}}{%
\pgfpathmoveto{\pgfqpoint{0.000000in}{0.000000in}}%
\pgfpathlineto{\pgfqpoint{-0.027778in}{0.000000in}}%
\pgfusepath{stroke,fill}%
}%
\begin{pgfscope}%
\pgfsys@transformshift{3.347347in}{5.075898in}%
\pgfsys@useobject{currentmarker}{}%
\end{pgfscope}%
\end{pgfscope}%
\begin{pgfscope}%
\pgfsetbuttcap%
\pgfsetroundjoin%
\definecolor{currentfill}{rgb}{0.000000,0.000000,0.000000}%
\pgfsetfillcolor{currentfill}%
\pgfsetlinewidth{0.602250pt}%
\definecolor{currentstroke}{rgb}{0.000000,0.000000,0.000000}%
\pgfsetstrokecolor{currentstroke}%
\pgfsetdash{}{0pt}%
\pgfsys@defobject{currentmarker}{\pgfqpoint{-0.027778in}{0.000000in}}{\pgfqpoint{0.000000in}{0.000000in}}{%
\pgfpathmoveto{\pgfqpoint{0.000000in}{0.000000in}}%
\pgfpathlineto{\pgfqpoint{-0.027778in}{0.000000in}}%
\pgfusepath{stroke,fill}%
}%
\begin{pgfscope}%
\pgfsys@transformshift{3.347347in}{5.124925in}%
\pgfsys@useobject{currentmarker}{}%
\end{pgfscope}%
\end{pgfscope}%
\begin{pgfscope}%
\pgfsetbuttcap%
\pgfsetroundjoin%
\definecolor{currentfill}{rgb}{0.000000,0.000000,0.000000}%
\pgfsetfillcolor{currentfill}%
\pgfsetlinewidth{0.602250pt}%
\definecolor{currentstroke}{rgb}{0.000000,0.000000,0.000000}%
\pgfsetstrokecolor{currentstroke}%
\pgfsetdash{}{0pt}%
\pgfsys@defobject{currentmarker}{\pgfqpoint{-0.027778in}{0.000000in}}{\pgfqpoint{0.000000in}{0.000000in}}{%
\pgfpathmoveto{\pgfqpoint{0.000000in}{0.000000in}}%
\pgfpathlineto{\pgfqpoint{-0.027778in}{0.000000in}}%
\pgfusepath{stroke,fill}%
}%
\begin{pgfscope}%
\pgfsys@transformshift{3.347347in}{5.164982in}%
\pgfsys@useobject{currentmarker}{}%
\end{pgfscope}%
\end{pgfscope}%
\begin{pgfscope}%
\pgfsetbuttcap%
\pgfsetroundjoin%
\definecolor{currentfill}{rgb}{0.000000,0.000000,0.000000}%
\pgfsetfillcolor{currentfill}%
\pgfsetlinewidth{0.602250pt}%
\definecolor{currentstroke}{rgb}{0.000000,0.000000,0.000000}%
\pgfsetstrokecolor{currentstroke}%
\pgfsetdash{}{0pt}%
\pgfsys@defobject{currentmarker}{\pgfqpoint{-0.027778in}{0.000000in}}{\pgfqpoint{0.000000in}{0.000000in}}{%
\pgfpathmoveto{\pgfqpoint{0.000000in}{0.000000in}}%
\pgfpathlineto{\pgfqpoint{-0.027778in}{0.000000in}}%
\pgfusepath{stroke,fill}%
}%
\begin{pgfscope}%
\pgfsys@transformshift{3.347347in}{5.198850in}%
\pgfsys@useobject{currentmarker}{}%
\end{pgfscope}%
\end{pgfscope}%
\begin{pgfscope}%
\pgfsetbuttcap%
\pgfsetroundjoin%
\definecolor{currentfill}{rgb}{0.000000,0.000000,0.000000}%
\pgfsetfillcolor{currentfill}%
\pgfsetlinewidth{0.602250pt}%
\definecolor{currentstroke}{rgb}{0.000000,0.000000,0.000000}%
\pgfsetstrokecolor{currentstroke}%
\pgfsetdash{}{0pt}%
\pgfsys@defobject{currentmarker}{\pgfqpoint{-0.027778in}{0.000000in}}{\pgfqpoint{0.000000in}{0.000000in}}{%
\pgfpathmoveto{\pgfqpoint{0.000000in}{0.000000in}}%
\pgfpathlineto{\pgfqpoint{-0.027778in}{0.000000in}}%
\pgfusepath{stroke,fill}%
}%
\begin{pgfscope}%
\pgfsys@transformshift{3.347347in}{5.228188in}%
\pgfsys@useobject{currentmarker}{}%
\end{pgfscope}%
\end{pgfscope}%
\begin{pgfscope}%
\pgfsetbuttcap%
\pgfsetroundjoin%
\definecolor{currentfill}{rgb}{0.000000,0.000000,0.000000}%
\pgfsetfillcolor{currentfill}%
\pgfsetlinewidth{0.602250pt}%
\definecolor{currentstroke}{rgb}{0.000000,0.000000,0.000000}%
\pgfsetstrokecolor{currentstroke}%
\pgfsetdash{}{0pt}%
\pgfsys@defobject{currentmarker}{\pgfqpoint{-0.027778in}{0.000000in}}{\pgfqpoint{0.000000in}{0.000000in}}{%
\pgfpathmoveto{\pgfqpoint{0.000000in}{0.000000in}}%
\pgfpathlineto{\pgfqpoint{-0.027778in}{0.000000in}}%
\pgfusepath{stroke,fill}%
}%
\begin{pgfscope}%
\pgfsys@transformshift{3.347347in}{5.254065in}%
\pgfsys@useobject{currentmarker}{}%
\end{pgfscope}%
\end{pgfscope}%
\begin{pgfscope}%
\pgfsetbuttcap%
\pgfsetroundjoin%
\definecolor{currentfill}{rgb}{0.000000,0.000000,0.000000}%
\pgfsetfillcolor{currentfill}%
\pgfsetlinewidth{0.602250pt}%
\definecolor{currentstroke}{rgb}{0.000000,0.000000,0.000000}%
\pgfsetstrokecolor{currentstroke}%
\pgfsetdash{}{0pt}%
\pgfsys@defobject{currentmarker}{\pgfqpoint{-0.027778in}{0.000000in}}{\pgfqpoint{0.000000in}{0.000000in}}{%
\pgfpathmoveto{\pgfqpoint{0.000000in}{0.000000in}}%
\pgfpathlineto{\pgfqpoint{-0.027778in}{0.000000in}}%
\pgfusepath{stroke,fill}%
}%
\begin{pgfscope}%
\pgfsys@transformshift{3.347347in}{5.429503in}%
\pgfsys@useobject{currentmarker}{}%
\end{pgfscope}%
\end{pgfscope}%
\begin{pgfscope}%
\pgfsetbuttcap%
\pgfsetroundjoin%
\definecolor{currentfill}{rgb}{0.000000,0.000000,0.000000}%
\pgfsetfillcolor{currentfill}%
\pgfsetlinewidth{0.602250pt}%
\definecolor{currentstroke}{rgb}{0.000000,0.000000,0.000000}%
\pgfsetstrokecolor{currentstroke}%
\pgfsetdash{}{0pt}%
\pgfsys@defobject{currentmarker}{\pgfqpoint{-0.027778in}{0.000000in}}{\pgfqpoint{0.000000in}{0.000000in}}{%
\pgfpathmoveto{\pgfqpoint{0.000000in}{0.000000in}}%
\pgfpathlineto{\pgfqpoint{-0.027778in}{0.000000in}}%
\pgfusepath{stroke,fill}%
}%
\begin{pgfscope}%
\pgfsys@transformshift{3.347347in}{5.518586in}%
\pgfsys@useobject{currentmarker}{}%
\end{pgfscope}%
\end{pgfscope}%
\begin{pgfscope}%
\pgfsetbuttcap%
\pgfsetroundjoin%
\definecolor{currentfill}{rgb}{0.000000,0.000000,0.000000}%
\pgfsetfillcolor{currentfill}%
\pgfsetlinewidth{0.602250pt}%
\definecolor{currentstroke}{rgb}{0.000000,0.000000,0.000000}%
\pgfsetstrokecolor{currentstroke}%
\pgfsetdash{}{0pt}%
\pgfsys@defobject{currentmarker}{\pgfqpoint{-0.027778in}{0.000000in}}{\pgfqpoint{0.000000in}{0.000000in}}{%
\pgfpathmoveto{\pgfqpoint{0.000000in}{0.000000in}}%
\pgfpathlineto{\pgfqpoint{-0.027778in}{0.000000in}}%
\pgfusepath{stroke,fill}%
}%
\begin{pgfscope}%
\pgfsys@transformshift{3.347347in}{5.581792in}%
\pgfsys@useobject{currentmarker}{}%
\end{pgfscope}%
\end{pgfscope}%
\begin{pgfscope}%
\pgfsetbuttcap%
\pgfsetroundjoin%
\definecolor{currentfill}{rgb}{0.000000,0.000000,0.000000}%
\pgfsetfillcolor{currentfill}%
\pgfsetlinewidth{0.602250pt}%
\definecolor{currentstroke}{rgb}{0.000000,0.000000,0.000000}%
\pgfsetstrokecolor{currentstroke}%
\pgfsetdash{}{0pt}%
\pgfsys@defobject{currentmarker}{\pgfqpoint{-0.027778in}{0.000000in}}{\pgfqpoint{0.000000in}{0.000000in}}{%
\pgfpathmoveto{\pgfqpoint{0.000000in}{0.000000in}}%
\pgfpathlineto{\pgfqpoint{-0.027778in}{0.000000in}}%
\pgfusepath{stroke,fill}%
}%
\begin{pgfscope}%
\pgfsys@transformshift{3.347347in}{5.630818in}%
\pgfsys@useobject{currentmarker}{}%
\end{pgfscope}%
\end{pgfscope}%
\begin{pgfscope}%
\pgfsetbuttcap%
\pgfsetroundjoin%
\definecolor{currentfill}{rgb}{0.000000,0.000000,0.000000}%
\pgfsetfillcolor{currentfill}%
\pgfsetlinewidth{0.602250pt}%
\definecolor{currentstroke}{rgb}{0.000000,0.000000,0.000000}%
\pgfsetstrokecolor{currentstroke}%
\pgfsetdash{}{0pt}%
\pgfsys@defobject{currentmarker}{\pgfqpoint{-0.027778in}{0.000000in}}{\pgfqpoint{0.000000in}{0.000000in}}{%
\pgfpathmoveto{\pgfqpoint{0.000000in}{0.000000in}}%
\pgfpathlineto{\pgfqpoint{-0.027778in}{0.000000in}}%
\pgfusepath{stroke,fill}%
}%
\begin{pgfscope}%
\pgfsys@transformshift{3.347347in}{5.670875in}%
\pgfsys@useobject{currentmarker}{}%
\end{pgfscope}%
\end{pgfscope}%
\begin{pgfscope}%
\pgfsetbuttcap%
\pgfsetroundjoin%
\definecolor{currentfill}{rgb}{0.000000,0.000000,0.000000}%
\pgfsetfillcolor{currentfill}%
\pgfsetlinewidth{0.602250pt}%
\definecolor{currentstroke}{rgb}{0.000000,0.000000,0.000000}%
\pgfsetstrokecolor{currentstroke}%
\pgfsetdash{}{0pt}%
\pgfsys@defobject{currentmarker}{\pgfqpoint{-0.027778in}{0.000000in}}{\pgfqpoint{0.000000in}{0.000000in}}{%
\pgfpathmoveto{\pgfqpoint{0.000000in}{0.000000in}}%
\pgfpathlineto{\pgfqpoint{-0.027778in}{0.000000in}}%
\pgfusepath{stroke,fill}%
}%
\begin{pgfscope}%
\pgfsys@transformshift{3.347347in}{5.704743in}%
\pgfsys@useobject{currentmarker}{}%
\end{pgfscope}%
\end{pgfscope}%
\begin{pgfscope}%
\pgfsetbuttcap%
\pgfsetroundjoin%
\definecolor{currentfill}{rgb}{0.000000,0.000000,0.000000}%
\pgfsetfillcolor{currentfill}%
\pgfsetlinewidth{0.602250pt}%
\definecolor{currentstroke}{rgb}{0.000000,0.000000,0.000000}%
\pgfsetstrokecolor{currentstroke}%
\pgfsetdash{}{0pt}%
\pgfsys@defobject{currentmarker}{\pgfqpoint{-0.027778in}{0.000000in}}{\pgfqpoint{0.000000in}{0.000000in}}{%
\pgfpathmoveto{\pgfqpoint{0.000000in}{0.000000in}}%
\pgfpathlineto{\pgfqpoint{-0.027778in}{0.000000in}}%
\pgfusepath{stroke,fill}%
}%
\begin{pgfscope}%
\pgfsys@transformshift{3.347347in}{5.734081in}%
\pgfsys@useobject{currentmarker}{}%
\end{pgfscope}%
\end{pgfscope}%
\begin{pgfscope}%
\pgfsetbuttcap%
\pgfsetroundjoin%
\definecolor{currentfill}{rgb}{0.000000,0.000000,0.000000}%
\pgfsetfillcolor{currentfill}%
\pgfsetlinewidth{0.602250pt}%
\definecolor{currentstroke}{rgb}{0.000000,0.000000,0.000000}%
\pgfsetstrokecolor{currentstroke}%
\pgfsetdash{}{0pt}%
\pgfsys@defobject{currentmarker}{\pgfqpoint{-0.027778in}{0.000000in}}{\pgfqpoint{0.000000in}{0.000000in}}{%
\pgfpathmoveto{\pgfqpoint{0.000000in}{0.000000in}}%
\pgfpathlineto{\pgfqpoint{-0.027778in}{0.000000in}}%
\pgfusepath{stroke,fill}%
}%
\begin{pgfscope}%
\pgfsys@transformshift{3.347347in}{5.759959in}%
\pgfsys@useobject{currentmarker}{}%
\end{pgfscope}%
\end{pgfscope}%
\begin{pgfscope}%
\pgfsetbuttcap%
\pgfsetroundjoin%
\definecolor{currentfill}{rgb}{0.000000,0.000000,0.000000}%
\pgfsetfillcolor{currentfill}%
\pgfsetlinewidth{0.602250pt}%
\definecolor{currentstroke}{rgb}{0.000000,0.000000,0.000000}%
\pgfsetstrokecolor{currentstroke}%
\pgfsetdash{}{0pt}%
\pgfsys@defobject{currentmarker}{\pgfqpoint{-0.027778in}{0.000000in}}{\pgfqpoint{0.000000in}{0.000000in}}{%
\pgfpathmoveto{\pgfqpoint{0.000000in}{0.000000in}}%
\pgfpathlineto{\pgfqpoint{-0.027778in}{0.000000in}}%
\pgfusepath{stroke,fill}%
}%
\begin{pgfscope}%
\pgfsys@transformshift{3.347347in}{5.935396in}%
\pgfsys@useobject{currentmarker}{}%
\end{pgfscope}%
\end{pgfscope}%
\begin{pgfscope}%
\pgfsetbuttcap%
\pgfsetroundjoin%
\definecolor{currentfill}{rgb}{0.000000,0.000000,0.000000}%
\pgfsetfillcolor{currentfill}%
\pgfsetlinewidth{0.602250pt}%
\definecolor{currentstroke}{rgb}{0.000000,0.000000,0.000000}%
\pgfsetstrokecolor{currentstroke}%
\pgfsetdash{}{0pt}%
\pgfsys@defobject{currentmarker}{\pgfqpoint{-0.027778in}{0.000000in}}{\pgfqpoint{0.000000in}{0.000000in}}{%
\pgfpathmoveto{\pgfqpoint{0.000000in}{0.000000in}}%
\pgfpathlineto{\pgfqpoint{-0.027778in}{0.000000in}}%
\pgfusepath{stroke,fill}%
}%
\begin{pgfscope}%
\pgfsys@transformshift{3.347347in}{6.024479in}%
\pgfsys@useobject{currentmarker}{}%
\end{pgfscope}%
\end{pgfscope}%
\begin{pgfscope}%
\pgfsetbuttcap%
\pgfsetroundjoin%
\definecolor{currentfill}{rgb}{0.000000,0.000000,0.000000}%
\pgfsetfillcolor{currentfill}%
\pgfsetlinewidth{0.602250pt}%
\definecolor{currentstroke}{rgb}{0.000000,0.000000,0.000000}%
\pgfsetstrokecolor{currentstroke}%
\pgfsetdash{}{0pt}%
\pgfsys@defobject{currentmarker}{\pgfqpoint{-0.027778in}{0.000000in}}{\pgfqpoint{0.000000in}{0.000000in}}{%
\pgfpathmoveto{\pgfqpoint{0.000000in}{0.000000in}}%
\pgfpathlineto{\pgfqpoint{-0.027778in}{0.000000in}}%
\pgfusepath{stroke,fill}%
}%
\begin{pgfscope}%
\pgfsys@transformshift{3.347347in}{6.087685in}%
\pgfsys@useobject{currentmarker}{}%
\end{pgfscope}%
\end{pgfscope}%
\begin{pgfscope}%
\pgfsetbuttcap%
\pgfsetroundjoin%
\definecolor{currentfill}{rgb}{0.000000,0.000000,0.000000}%
\pgfsetfillcolor{currentfill}%
\pgfsetlinewidth{0.602250pt}%
\definecolor{currentstroke}{rgb}{0.000000,0.000000,0.000000}%
\pgfsetstrokecolor{currentstroke}%
\pgfsetdash{}{0pt}%
\pgfsys@defobject{currentmarker}{\pgfqpoint{-0.027778in}{0.000000in}}{\pgfqpoint{0.000000in}{0.000000in}}{%
\pgfpathmoveto{\pgfqpoint{0.000000in}{0.000000in}}%
\pgfpathlineto{\pgfqpoint{-0.027778in}{0.000000in}}%
\pgfusepath{stroke,fill}%
}%
\begin{pgfscope}%
\pgfsys@transformshift{3.347347in}{6.136711in}%
\pgfsys@useobject{currentmarker}{}%
\end{pgfscope}%
\end{pgfscope}%
\begin{pgfscope}%
\pgfsetbuttcap%
\pgfsetroundjoin%
\definecolor{currentfill}{rgb}{0.000000,0.000000,0.000000}%
\pgfsetfillcolor{currentfill}%
\pgfsetlinewidth{0.602250pt}%
\definecolor{currentstroke}{rgb}{0.000000,0.000000,0.000000}%
\pgfsetstrokecolor{currentstroke}%
\pgfsetdash{}{0pt}%
\pgfsys@defobject{currentmarker}{\pgfqpoint{-0.027778in}{0.000000in}}{\pgfqpoint{0.000000in}{0.000000in}}{%
\pgfpathmoveto{\pgfqpoint{0.000000in}{0.000000in}}%
\pgfpathlineto{\pgfqpoint{-0.027778in}{0.000000in}}%
\pgfusepath{stroke,fill}%
}%
\begin{pgfscope}%
\pgfsys@transformshift{3.347347in}{6.176768in}%
\pgfsys@useobject{currentmarker}{}%
\end{pgfscope}%
\end{pgfscope}%
\begin{pgfscope}%
\pgfsetbuttcap%
\pgfsetroundjoin%
\definecolor{currentfill}{rgb}{0.000000,0.000000,0.000000}%
\pgfsetfillcolor{currentfill}%
\pgfsetlinewidth{0.602250pt}%
\definecolor{currentstroke}{rgb}{0.000000,0.000000,0.000000}%
\pgfsetstrokecolor{currentstroke}%
\pgfsetdash{}{0pt}%
\pgfsys@defobject{currentmarker}{\pgfqpoint{-0.027778in}{0.000000in}}{\pgfqpoint{0.000000in}{0.000000in}}{%
\pgfpathmoveto{\pgfqpoint{0.000000in}{0.000000in}}%
\pgfpathlineto{\pgfqpoint{-0.027778in}{0.000000in}}%
\pgfusepath{stroke,fill}%
}%
\begin{pgfscope}%
\pgfsys@transformshift{3.347347in}{6.210636in}%
\pgfsys@useobject{currentmarker}{}%
\end{pgfscope}%
\end{pgfscope}%
\begin{pgfscope}%
\pgfsetbuttcap%
\pgfsetroundjoin%
\definecolor{currentfill}{rgb}{0.000000,0.000000,0.000000}%
\pgfsetfillcolor{currentfill}%
\pgfsetlinewidth{0.602250pt}%
\definecolor{currentstroke}{rgb}{0.000000,0.000000,0.000000}%
\pgfsetstrokecolor{currentstroke}%
\pgfsetdash{}{0pt}%
\pgfsys@defobject{currentmarker}{\pgfqpoint{-0.027778in}{0.000000in}}{\pgfqpoint{0.000000in}{0.000000in}}{%
\pgfpathmoveto{\pgfqpoint{0.000000in}{0.000000in}}%
\pgfpathlineto{\pgfqpoint{-0.027778in}{0.000000in}}%
\pgfusepath{stroke,fill}%
}%
\begin{pgfscope}%
\pgfsys@transformshift{3.347347in}{6.239974in}%
\pgfsys@useobject{currentmarker}{}%
\end{pgfscope}%
\end{pgfscope}%
\begin{pgfscope}%
\pgfsetbuttcap%
\pgfsetroundjoin%
\definecolor{currentfill}{rgb}{0.000000,0.000000,0.000000}%
\pgfsetfillcolor{currentfill}%
\pgfsetlinewidth{0.602250pt}%
\definecolor{currentstroke}{rgb}{0.000000,0.000000,0.000000}%
\pgfsetstrokecolor{currentstroke}%
\pgfsetdash{}{0pt}%
\pgfsys@defobject{currentmarker}{\pgfqpoint{-0.027778in}{0.000000in}}{\pgfqpoint{0.000000in}{0.000000in}}{%
\pgfpathmoveto{\pgfqpoint{0.000000in}{0.000000in}}%
\pgfpathlineto{\pgfqpoint{-0.027778in}{0.000000in}}%
\pgfusepath{stroke,fill}%
}%
\begin{pgfscope}%
\pgfsys@transformshift{3.347347in}{6.265852in}%
\pgfsys@useobject{currentmarker}{}%
\end{pgfscope}%
\end{pgfscope}%
\begin{pgfscope}%
\pgfpathrectangle{\pgfqpoint{3.347347in}{4.908628in}}{\pgfqpoint{1.897959in}{1.372727in}} %
\pgfusepath{clip}%
\pgfsetbuttcap%
\pgfsetroundjoin%
\pgfsetlinewidth{1.505625pt}%
\definecolor{currentstroke}{rgb}{1.000000,0.000000,0.000000}%
\pgfsetstrokecolor{currentstroke}%
\pgfsetdash{{5.550000pt}{2.400000pt}}{0.000000pt}%
\pgfpathmoveto{\pgfqpoint{3.433618in}{6.146045in}}%
\pgfpathlineto{\pgfqpoint{3.505510in}{6.143051in}}%
\pgfpathlineto{\pgfqpoint{3.577403in}{6.140174in}}%
\pgfpathlineto{\pgfqpoint{3.649295in}{6.137411in}}%
\pgfpathlineto{\pgfqpoint{3.721187in}{6.134760in}}%
\pgfpathlineto{\pgfqpoint{3.793080in}{6.132221in}}%
\pgfpathlineto{\pgfqpoint{3.864972in}{6.129792in}}%
\pgfpathlineto{\pgfqpoint{3.936865in}{6.127472in}}%
\pgfpathlineto{\pgfqpoint{4.008757in}{6.125262in}}%
\pgfpathlineto{\pgfqpoint{4.080649in}{6.123160in}}%
\pgfpathlineto{\pgfqpoint{4.152542in}{6.121166in}}%
\pgfpathlineto{\pgfqpoint{4.224434in}{6.119279in}}%
\pgfpathlineto{\pgfqpoint{4.296327in}{6.117500in}}%
\pgfpathlineto{\pgfqpoint{4.368219in}{6.115827in}}%
\pgfpathlineto{\pgfqpoint{4.440111in}{6.114260in}}%
\pgfpathlineto{\pgfqpoint{4.512004in}{6.112798in}}%
\pgfpathlineto{\pgfqpoint{4.583896in}{6.111442in}}%
\pgfpathlineto{\pgfqpoint{4.655788in}{6.110191in}}%
\pgfpathlineto{\pgfqpoint{4.727681in}{6.109044in}}%
\pgfpathlineto{\pgfqpoint{4.799573in}{6.108002in}}%
\pgfpathlineto{\pgfqpoint{4.871466in}{6.107063in}}%
\pgfpathlineto{\pgfqpoint{4.943358in}{6.106227in}}%
\pgfpathlineto{\pgfqpoint{5.015250in}{6.105495in}}%
\pgfpathlineto{\pgfqpoint{5.087143in}{6.104865in}}%
\pgfpathlineto{\pgfqpoint{5.159035in}{6.104338in}}%
\pgfusepath{stroke}%
\end{pgfscope}%
\begin{pgfscope}%
\pgfpathrectangle{\pgfqpoint{3.347347in}{4.908628in}}{\pgfqpoint{1.897959in}{1.372727in}} %
\pgfusepath{clip}%
\pgfsetbuttcap%
\pgfsetmiterjoin%
\definecolor{currentfill}{rgb}{1.000000,0.000000,0.000000}%
\pgfsetfillcolor{currentfill}%
\pgfsetlinewidth{1.003750pt}%
\definecolor{currentstroke}{rgb}{1.000000,0.000000,0.000000}%
\pgfsetstrokecolor{currentstroke}%
\pgfsetdash{}{0pt}%
\pgfsys@defobject{currentmarker}{\pgfqpoint{-0.041667in}{-0.041667in}}{\pgfqpoint{0.041667in}{0.041667in}}{%
\pgfpathmoveto{\pgfqpoint{-0.041667in}{-0.041667in}}%
\pgfpathlineto{\pgfqpoint{0.041667in}{-0.041667in}}%
\pgfpathlineto{\pgfqpoint{0.041667in}{0.041667in}}%
\pgfpathlineto{\pgfqpoint{-0.041667in}{0.041667in}}%
\pgfpathclose%
\pgfusepath{stroke,fill}%
}%
\begin{pgfscope}%
\pgfsys@transformshift{3.433618in}{6.146045in}%
\pgfsys@useobject{currentmarker}{}%
\end{pgfscope}%
\begin{pgfscope}%
\pgfsys@transformshift{3.793080in}{6.132221in}%
\pgfsys@useobject{currentmarker}{}%
\end{pgfscope}%
\begin{pgfscope}%
\pgfsys@transformshift{4.152542in}{6.121166in}%
\pgfsys@useobject{currentmarker}{}%
\end{pgfscope}%
\begin{pgfscope}%
\pgfsys@transformshift{4.512004in}{6.112798in}%
\pgfsys@useobject{currentmarker}{}%
\end{pgfscope}%
\begin{pgfscope}%
\pgfsys@transformshift{4.871466in}{6.107063in}%
\pgfsys@useobject{currentmarker}{}%
\end{pgfscope}%
\end{pgfscope}%
\begin{pgfscope}%
\pgfpathrectangle{\pgfqpoint{3.347347in}{4.908628in}}{\pgfqpoint{1.897959in}{1.372727in}} %
\pgfusepath{clip}%
\pgfsetrectcap%
\pgfsetroundjoin%
\pgfsetlinewidth{1.505625pt}%
\definecolor{currentstroke}{rgb}{0.000000,0.000000,1.000000}%
\pgfsetstrokecolor{currentstroke}%
\pgfsetdash{}{0pt}%
\pgfpathmoveto{\pgfqpoint{3.433618in}{5.480609in}}%
\pgfpathlineto{\pgfqpoint{3.505510in}{5.467353in}}%
\pgfpathlineto{\pgfqpoint{3.577403in}{5.454121in}}%
\pgfpathlineto{\pgfqpoint{3.649295in}{5.440820in}}%
\pgfpathlineto{\pgfqpoint{3.721187in}{5.427376in}}%
\pgfpathlineto{\pgfqpoint{3.793080in}{5.413722in}}%
\pgfpathlineto{\pgfqpoint{3.864972in}{5.399797in}}%
\pgfpathlineto{\pgfqpoint{3.936865in}{5.385540in}}%
\pgfpathlineto{\pgfqpoint{4.008757in}{5.370889in}}%
\pgfpathlineto{\pgfqpoint{4.080649in}{5.355781in}}%
\pgfpathlineto{\pgfqpoint{4.152542in}{5.340144in}}%
\pgfpathlineto{\pgfqpoint{4.224434in}{5.323901in}}%
\pgfpathlineto{\pgfqpoint{4.296327in}{5.306963in}}%
\pgfpathlineto{\pgfqpoint{4.368219in}{5.289228in}}%
\pgfpathlineto{\pgfqpoint{4.440111in}{5.270577in}}%
\pgfpathlineto{\pgfqpoint{4.512004in}{5.250866in}}%
\pgfpathlineto{\pgfqpoint{4.583896in}{5.229922in}}%
\pgfpathlineto{\pgfqpoint{4.655788in}{5.207531in}}%
\pgfpathlineto{\pgfqpoint{4.727681in}{5.183423in}}%
\pgfpathlineto{\pgfqpoint{4.799573in}{5.157251in}}%
\pgfpathlineto{\pgfqpoint{4.871466in}{5.128556in}}%
\pgfpathlineto{\pgfqpoint{4.943358in}{5.096715in}}%
\pgfpathlineto{\pgfqpoint{5.015250in}{5.060844in}}%
\pgfpathlineto{\pgfqpoint{5.087143in}{5.019635in}}%
\pgfpathlineto{\pgfqpoint{5.159035in}{4.971024in}}%
\pgfusepath{stroke}%
\end{pgfscope}%
\begin{pgfscope}%
\pgfpathrectangle{\pgfqpoint{3.347347in}{4.908628in}}{\pgfqpoint{1.897959in}{1.372727in}} %
\pgfusepath{clip}%
\pgfsetbuttcap%
\pgfsetroundjoin%
\definecolor{currentfill}{rgb}{0.000000,0.000000,1.000000}%
\pgfsetfillcolor{currentfill}%
\pgfsetlinewidth{1.003750pt}%
\definecolor{currentstroke}{rgb}{0.000000,0.000000,1.000000}%
\pgfsetstrokecolor{currentstroke}%
\pgfsetdash{}{0pt}%
\pgfsys@defobject{currentmarker}{\pgfqpoint{-0.041667in}{-0.041667in}}{\pgfqpoint{0.041667in}{0.041667in}}{%
\pgfpathmoveto{\pgfqpoint{0.000000in}{-0.041667in}}%
\pgfpathcurveto{\pgfqpoint{0.011050in}{-0.041667in}}{\pgfqpoint{0.021649in}{-0.037276in}}{\pgfqpoint{0.029463in}{-0.029463in}}%
\pgfpathcurveto{\pgfqpoint{0.037276in}{-0.021649in}}{\pgfqpoint{0.041667in}{-0.011050in}}{\pgfqpoint{0.041667in}{0.000000in}}%
\pgfpathcurveto{\pgfqpoint{0.041667in}{0.011050in}}{\pgfqpoint{0.037276in}{0.021649in}}{\pgfqpoint{0.029463in}{0.029463in}}%
\pgfpathcurveto{\pgfqpoint{0.021649in}{0.037276in}}{\pgfqpoint{0.011050in}{0.041667in}}{\pgfqpoint{0.000000in}{0.041667in}}%
\pgfpathcurveto{\pgfqpoint{-0.011050in}{0.041667in}}{\pgfqpoint{-0.021649in}{0.037276in}}{\pgfqpoint{-0.029463in}{0.029463in}}%
\pgfpathcurveto{\pgfqpoint{-0.037276in}{0.021649in}}{\pgfqpoint{-0.041667in}{0.011050in}}{\pgfqpoint{-0.041667in}{0.000000in}}%
\pgfpathcurveto{\pgfqpoint{-0.041667in}{-0.011050in}}{\pgfqpoint{-0.037276in}{-0.021649in}}{\pgfqpoint{-0.029463in}{-0.029463in}}%
\pgfpathcurveto{\pgfqpoint{-0.021649in}{-0.037276in}}{\pgfqpoint{-0.011050in}{-0.041667in}}{\pgfqpoint{0.000000in}{-0.041667in}}%
\pgfpathclose%
\pgfusepath{stroke,fill}%
}%
\begin{pgfscope}%
\pgfsys@transformshift{3.433618in}{5.480609in}%
\pgfsys@useobject{currentmarker}{}%
\end{pgfscope}%
\begin{pgfscope}%
\pgfsys@transformshift{3.793080in}{5.413722in}%
\pgfsys@useobject{currentmarker}{}%
\end{pgfscope}%
\begin{pgfscope}%
\pgfsys@transformshift{4.152542in}{5.340144in}%
\pgfsys@useobject{currentmarker}{}%
\end{pgfscope}%
\begin{pgfscope}%
\pgfsys@transformshift{4.512004in}{5.250866in}%
\pgfsys@useobject{currentmarker}{}%
\end{pgfscope}%
\begin{pgfscope}%
\pgfsys@transformshift{4.871466in}{5.128556in}%
\pgfsys@useobject{currentmarker}{}%
\end{pgfscope}%
\end{pgfscope}%
\begin{pgfscope}%
\pgfpathrectangle{\pgfqpoint{3.347347in}{4.908628in}}{\pgfqpoint{1.897959in}{1.372727in}} %
\pgfusepath{clip}%
\pgfsetbuttcap%
\pgfsetroundjoin%
\pgfsetlinewidth{1.505625pt}%
\definecolor{currentstroke}{rgb}{0.000000,0.750000,0.750000}%
\pgfsetstrokecolor{currentstroke}%
\pgfsetdash{{9.600000pt}{2.400000pt}{1.500000pt}{2.400000pt}}{0.000000pt}%
\pgfpathmoveto{\pgfqpoint{3.433618in}{5.915312in}}%
\pgfpathlineto{\pgfqpoint{3.505510in}{5.880830in}}%
\pgfpathlineto{\pgfqpoint{3.577403in}{5.850286in}}%
\pgfpathlineto{\pgfqpoint{3.649295in}{5.823014in}}%
\pgfpathlineto{\pgfqpoint{3.721187in}{5.798510in}}%
\pgfpathlineto{\pgfqpoint{3.793080in}{5.776383in}}%
\pgfpathlineto{\pgfqpoint{3.864972in}{5.756321in}}%
\pgfpathlineto{\pgfqpoint{3.936865in}{5.738076in}}%
\pgfpathlineto{\pgfqpoint{4.008757in}{5.721444in}}%
\pgfpathlineto{\pgfqpoint{4.080649in}{5.706255in}}%
\pgfpathlineto{\pgfqpoint{4.152542in}{5.692370in}}%
\pgfpathlineto{\pgfqpoint{4.224434in}{5.679671in}}%
\pgfpathlineto{\pgfqpoint{4.296327in}{5.668058in}}%
\pgfpathlineto{\pgfqpoint{4.368219in}{5.657444in}}%
\pgfpathlineto{\pgfqpoint{4.440111in}{5.647757in}}%
\pgfpathlineto{\pgfqpoint{4.512004in}{5.638934in}}%
\pgfpathlineto{\pgfqpoint{4.583896in}{5.630920in}}%
\pgfpathlineto{\pgfqpoint{4.655788in}{5.623667in}}%
\pgfpathlineto{\pgfqpoint{4.727681in}{5.617136in}}%
\pgfpathlineto{\pgfqpoint{4.799573in}{5.611291in}}%
\pgfpathlineto{\pgfqpoint{4.871466in}{5.606100in}}%
\pgfpathlineto{\pgfqpoint{4.943358in}{5.601538in}}%
\pgfpathlineto{\pgfqpoint{5.015250in}{5.597582in}}%
\pgfpathlineto{\pgfqpoint{5.087143in}{5.594213in}}%
\pgfpathlineto{\pgfqpoint{5.159035in}{5.591415in}}%
\pgfusepath{stroke}%
\end{pgfscope}%
\begin{pgfscope}%
\pgfpathrectangle{\pgfqpoint{3.347347in}{4.908628in}}{\pgfqpoint{1.897959in}{1.372727in}} %
\pgfusepath{clip}%
\pgfsetbuttcap%
\pgfsetmiterjoin%
\definecolor{currentfill}{rgb}{0.000000,0.750000,0.750000}%
\pgfsetfillcolor{currentfill}%
\pgfsetlinewidth{1.003750pt}%
\definecolor{currentstroke}{rgb}{0.000000,0.750000,0.750000}%
\pgfsetstrokecolor{currentstroke}%
\pgfsetdash{}{0pt}%
\pgfsys@defobject{currentmarker}{\pgfqpoint{-0.041667in}{-0.041667in}}{\pgfqpoint{0.041667in}{0.041667in}}{%
\pgfpathmoveto{\pgfqpoint{-0.000000in}{-0.041667in}}%
\pgfpathlineto{\pgfqpoint{0.041667in}{0.041667in}}%
\pgfpathlineto{\pgfqpoint{-0.041667in}{0.041667in}}%
\pgfpathclose%
\pgfusepath{stroke,fill}%
}%
\begin{pgfscope}%
\pgfsys@transformshift{3.433618in}{5.915312in}%
\pgfsys@useobject{currentmarker}{}%
\end{pgfscope}%
\begin{pgfscope}%
\pgfsys@transformshift{3.793080in}{5.776383in}%
\pgfsys@useobject{currentmarker}{}%
\end{pgfscope}%
\begin{pgfscope}%
\pgfsys@transformshift{4.152542in}{5.692370in}%
\pgfsys@useobject{currentmarker}{}%
\end{pgfscope}%
\begin{pgfscope}%
\pgfsys@transformshift{4.512004in}{5.638934in}%
\pgfsys@useobject{currentmarker}{}%
\end{pgfscope}%
\begin{pgfscope}%
\pgfsys@transformshift{4.871466in}{5.606100in}%
\pgfsys@useobject{currentmarker}{}%
\end{pgfscope}%
\end{pgfscope}%
\begin{pgfscope}%
\pgfpathrectangle{\pgfqpoint{3.347347in}{4.908628in}}{\pgfqpoint{1.897959in}{1.372727in}} %
\pgfusepath{clip}%
\pgfsetbuttcap%
\pgfsetroundjoin%
\pgfsetlinewidth{1.505625pt}%
\definecolor{currentstroke}{rgb}{0.000000,0.000000,0.000000}%
\pgfsetstrokecolor{currentstroke}%
\pgfsetdash{{1.500000pt}{2.475000pt}}{0.000000pt}%
\pgfpathmoveto{\pgfqpoint{3.433618in}{6.218958in}}%
\pgfpathlineto{\pgfqpoint{3.505510in}{6.209228in}}%
\pgfpathlineto{\pgfqpoint{3.577403in}{6.199988in}}%
\pgfpathlineto{\pgfqpoint{3.649295in}{6.192033in}}%
\pgfpathlineto{\pgfqpoint{3.721187in}{6.185071in}}%
\pgfpathlineto{\pgfqpoint{3.793080in}{6.178898in}}%
\pgfpathlineto{\pgfqpoint{3.864972in}{6.173367in}}%
\pgfpathlineto{\pgfqpoint{3.936865in}{6.168368in}}%
\pgfpathlineto{\pgfqpoint{4.008757in}{6.163819in}}%
\pgfpathlineto{\pgfqpoint{4.080649in}{6.159658in}}%
\pgfpathlineto{\pgfqpoint{4.152542in}{6.155838in}}%
\pgfpathlineto{\pgfqpoint{4.224434in}{6.152319in}}%
\pgfpathlineto{\pgfqpoint{4.296327in}{6.149073in}}%
\pgfpathlineto{\pgfqpoint{4.368219in}{6.146075in}}%
\pgfpathlineto{\pgfqpoint{4.440111in}{6.143306in}}%
\pgfpathlineto{\pgfqpoint{4.512004in}{6.140750in}}%
\pgfpathlineto{\pgfqpoint{4.583896in}{6.138394in}}%
\pgfpathlineto{\pgfqpoint{4.655788in}{6.136229in}}%
\pgfpathlineto{\pgfqpoint{4.727681in}{6.134244in}}%
\pgfpathlineto{\pgfqpoint{4.799573in}{6.132434in}}%
\pgfpathlineto{\pgfqpoint{4.871466in}{6.130792in}}%
\pgfpathlineto{\pgfqpoint{4.943358in}{6.129313in}}%
\pgfpathlineto{\pgfqpoint{5.015250in}{6.127994in}}%
\pgfpathlineto{\pgfqpoint{5.087143in}{6.126832in}}%
\pgfpathlineto{\pgfqpoint{5.159035in}{6.125824in}}%
\pgfusepath{stroke}%
\end{pgfscope}%
\begin{pgfscope}%
\pgfpathrectangle{\pgfqpoint{3.347347in}{4.908628in}}{\pgfqpoint{1.897959in}{1.372727in}} %
\pgfusepath{clip}%
\pgfsetbuttcap%
\pgfsetroundjoin%
\definecolor{currentfill}{rgb}{0.000000,0.000000,0.000000}%
\pgfsetfillcolor{currentfill}%
\pgfsetlinewidth{1.003750pt}%
\definecolor{currentstroke}{rgb}{0.000000,0.000000,0.000000}%
\pgfsetstrokecolor{currentstroke}%
\pgfsetdash{}{0pt}%
\pgfsys@defobject{currentmarker}{\pgfqpoint{-0.041667in}{-0.041667in}}{\pgfqpoint{0.041667in}{0.041667in}}{%
\pgfpathmoveto{\pgfqpoint{-0.041667in}{0.000000in}}%
\pgfpathlineto{\pgfqpoint{0.041667in}{0.000000in}}%
\pgfpathmoveto{\pgfqpoint{0.000000in}{-0.041667in}}%
\pgfpathlineto{\pgfqpoint{0.000000in}{0.041667in}}%
\pgfusepath{stroke,fill}%
}%
\begin{pgfscope}%
\pgfsys@transformshift{3.433618in}{6.218958in}%
\pgfsys@useobject{currentmarker}{}%
\end{pgfscope}%
\begin{pgfscope}%
\pgfsys@transformshift{3.793080in}{6.178898in}%
\pgfsys@useobject{currentmarker}{}%
\end{pgfscope}%
\begin{pgfscope}%
\pgfsys@transformshift{4.152542in}{6.155838in}%
\pgfsys@useobject{currentmarker}{}%
\end{pgfscope}%
\begin{pgfscope}%
\pgfsys@transformshift{4.512004in}{6.140750in}%
\pgfsys@useobject{currentmarker}{}%
\end{pgfscope}%
\begin{pgfscope}%
\pgfsys@transformshift{4.871466in}{6.130792in}%
\pgfsys@useobject{currentmarker}{}%
\end{pgfscope}%
\end{pgfscope}%
\begin{pgfscope}%
\pgfsetrectcap%
\pgfsetmiterjoin%
\pgfsetlinewidth{0.803000pt}%
\definecolor{currentstroke}{rgb}{0.000000,0.000000,0.000000}%
\pgfsetstrokecolor{currentstroke}%
\pgfsetdash{}{0pt}%
\pgfpathmoveto{\pgfqpoint{3.347347in}{4.908628in}}%
\pgfpathlineto{\pgfqpoint{3.347347in}{6.281355in}}%
\pgfusepath{stroke}%
\end{pgfscope}%
\begin{pgfscope}%
\pgfsetrectcap%
\pgfsetmiterjoin%
\pgfsetlinewidth{0.803000pt}%
\definecolor{currentstroke}{rgb}{0.000000,0.000000,0.000000}%
\pgfsetstrokecolor{currentstroke}%
\pgfsetdash{}{0pt}%
\pgfpathmoveto{\pgfqpoint{5.245306in}{4.908628in}}%
\pgfpathlineto{\pgfqpoint{5.245306in}{6.281355in}}%
\pgfusepath{stroke}%
\end{pgfscope}%
\begin{pgfscope}%
\pgfsetrectcap%
\pgfsetmiterjoin%
\pgfsetlinewidth{0.803000pt}%
\definecolor{currentstroke}{rgb}{0.000000,0.000000,0.000000}%
\pgfsetstrokecolor{currentstroke}%
\pgfsetdash{}{0pt}%
\pgfpathmoveto{\pgfqpoint{3.347347in}{4.908628in}}%
\pgfpathlineto{\pgfqpoint{5.245306in}{4.908628in}}%
\pgfusepath{stroke}%
\end{pgfscope}%
\begin{pgfscope}%
\pgfsetrectcap%
\pgfsetmiterjoin%
\pgfsetlinewidth{0.803000pt}%
\definecolor{currentstroke}{rgb}{0.000000,0.000000,0.000000}%
\pgfsetstrokecolor{currentstroke}%
\pgfsetdash{}{0pt}%
\pgfpathmoveto{\pgfqpoint{3.347347in}{6.281355in}}%
\pgfpathlineto{\pgfqpoint{5.245306in}{6.281355in}}%
\pgfusepath{stroke}%
\end{pgfscope}%
\begin{pgfscope}%
\pgfsetbuttcap%
\pgfsetmiterjoin%
\definecolor{currentfill}{rgb}{1.000000,1.000000,1.000000}%
\pgfsetfillcolor{currentfill}%
\pgfsetlinewidth{0.000000pt}%
\definecolor{currentstroke}{rgb}{0.000000,0.000000,0.000000}%
\pgfsetstrokecolor{currentstroke}%
\pgfsetstrokeopacity{0.000000}%
\pgfsetdash{}{0pt}%
\pgfpathmoveto{\pgfqpoint{5.814694in}{4.908628in}}%
\pgfpathlineto{\pgfqpoint{7.712653in}{4.908628in}}%
\pgfpathlineto{\pgfqpoint{7.712653in}{6.281355in}}%
\pgfpathlineto{\pgfqpoint{5.814694in}{6.281355in}}%
\pgfpathclose%
\pgfusepath{fill}%
\end{pgfscope}%
\begin{pgfscope}%
\pgfsetbuttcap%
\pgfsetroundjoin%
\definecolor{currentfill}{rgb}{0.000000,0.000000,0.000000}%
\pgfsetfillcolor{currentfill}%
\pgfsetlinewidth{0.803000pt}%
\definecolor{currentstroke}{rgb}{0.000000,0.000000,0.000000}%
\pgfsetstrokecolor{currentstroke}%
\pgfsetdash{}{0pt}%
\pgfsys@defobject{currentmarker}{\pgfqpoint{0.000000in}{-0.048611in}}{\pgfqpoint{0.000000in}{0.000000in}}{%
\pgfpathmoveto{\pgfqpoint{0.000000in}{0.000000in}}%
\pgfpathlineto{\pgfqpoint{0.000000in}{-0.048611in}}%
\pgfusepath{stroke,fill}%
}%
\begin{pgfscope}%
\pgfsys@transformshift{6.126019in}{4.908628in}%
\pgfsys@useobject{currentmarker}{}%
\end{pgfscope}%
\end{pgfscope}%
\begin{pgfscope}%
\pgftext[x=6.126019in,y=4.811405in,,top]{\rmfamily\fontsize{10.000000}{12.000000}\selectfont \(\displaystyle 0.1\)}%
\end{pgfscope}%
\begin{pgfscope}%
\pgfsetbuttcap%
\pgfsetroundjoin%
\definecolor{currentfill}{rgb}{0.000000,0.000000,0.000000}%
\pgfsetfillcolor{currentfill}%
\pgfsetlinewidth{0.803000pt}%
\definecolor{currentstroke}{rgb}{0.000000,0.000000,0.000000}%
\pgfsetstrokecolor{currentstroke}%
\pgfsetdash{}{0pt}%
\pgfsys@defobject{currentmarker}{\pgfqpoint{0.000000in}{-0.048611in}}{\pgfqpoint{0.000000in}{0.000000in}}{%
\pgfpathmoveto{\pgfqpoint{0.000000in}{0.000000in}}%
\pgfpathlineto{\pgfqpoint{0.000000in}{-0.048611in}}%
\pgfusepath{stroke,fill}%
}%
\begin{pgfscope}%
\pgfsys@transformshift{7.026237in}{4.908628in}%
\pgfsys@useobject{currentmarker}{}%
\end{pgfscope}%
\end{pgfscope}%
\begin{pgfscope}%
\pgftext[x=7.026237in,y=4.811405in,,top]{\rmfamily\fontsize{10.000000}{12.000000}\selectfont \(\displaystyle 0.2\)}%
\end{pgfscope}%
\begin{pgfscope}%
\pgfsetbuttcap%
\pgfsetroundjoin%
\definecolor{currentfill}{rgb}{0.000000,0.000000,0.000000}%
\pgfsetfillcolor{currentfill}%
\pgfsetlinewidth{0.803000pt}%
\definecolor{currentstroke}{rgb}{0.000000,0.000000,0.000000}%
\pgfsetstrokecolor{currentstroke}%
\pgfsetdash{}{0pt}%
\pgfsys@defobject{currentmarker}{\pgfqpoint{-0.048611in}{0.000000in}}{\pgfqpoint{0.000000in}{0.000000in}}{%
\pgfpathmoveto{\pgfqpoint{0.000000in}{0.000000in}}%
\pgfpathlineto{\pgfqpoint{-0.048611in}{0.000000in}}%
\pgfusepath{stroke,fill}%
}%
\begin{pgfscope}%
\pgfsys@transformshift{5.814694in}{5.051496in}%
\pgfsys@useobject{currentmarker}{}%
\end{pgfscope}%
\end{pgfscope}%
\begin{pgfscope}%
\pgftext[x=5.429469in,y=4.998734in,left,base]{\rmfamily\fontsize{10.000000}{12.000000}\selectfont \(\displaystyle 10^{-7}\)}%
\end{pgfscope}%
\begin{pgfscope}%
\pgfsetbuttcap%
\pgfsetroundjoin%
\definecolor{currentfill}{rgb}{0.000000,0.000000,0.000000}%
\pgfsetfillcolor{currentfill}%
\pgfsetlinewidth{0.803000pt}%
\definecolor{currentstroke}{rgb}{0.000000,0.000000,0.000000}%
\pgfsetstrokecolor{currentstroke}%
\pgfsetdash{}{0pt}%
\pgfsys@defobject{currentmarker}{\pgfqpoint{-0.048611in}{0.000000in}}{\pgfqpoint{0.000000in}{0.000000in}}{%
\pgfpathmoveto{\pgfqpoint{0.000000in}{0.000000in}}%
\pgfpathlineto{\pgfqpoint{-0.048611in}{0.000000in}}%
\pgfusepath{stroke,fill}%
}%
\begin{pgfscope}%
\pgfsys@transformshift{5.814694in}{5.449072in}%
\pgfsys@useobject{currentmarker}{}%
\end{pgfscope}%
\end{pgfscope}%
\begin{pgfscope}%
\pgftext[x=5.429469in,y=5.396311in,left,base]{\rmfamily\fontsize{10.000000}{12.000000}\selectfont \(\displaystyle 10^{-6}\)}%
\end{pgfscope}%
\begin{pgfscope}%
\pgfsetbuttcap%
\pgfsetroundjoin%
\definecolor{currentfill}{rgb}{0.000000,0.000000,0.000000}%
\pgfsetfillcolor{currentfill}%
\pgfsetlinewidth{0.803000pt}%
\definecolor{currentstroke}{rgb}{0.000000,0.000000,0.000000}%
\pgfsetstrokecolor{currentstroke}%
\pgfsetdash{}{0pt}%
\pgfsys@defobject{currentmarker}{\pgfqpoint{-0.048611in}{0.000000in}}{\pgfqpoint{0.000000in}{0.000000in}}{%
\pgfpathmoveto{\pgfqpoint{0.000000in}{0.000000in}}%
\pgfpathlineto{\pgfqpoint{-0.048611in}{0.000000in}}%
\pgfusepath{stroke,fill}%
}%
\begin{pgfscope}%
\pgfsys@transformshift{5.814694in}{5.846649in}%
\pgfsys@useobject{currentmarker}{}%
\end{pgfscope}%
\end{pgfscope}%
\begin{pgfscope}%
\pgftext[x=5.429469in,y=5.793887in,left,base]{\rmfamily\fontsize{10.000000}{12.000000}\selectfont \(\displaystyle 10^{-5}\)}%
\end{pgfscope}%
\begin{pgfscope}%
\pgfsetbuttcap%
\pgfsetroundjoin%
\definecolor{currentfill}{rgb}{0.000000,0.000000,0.000000}%
\pgfsetfillcolor{currentfill}%
\pgfsetlinewidth{0.803000pt}%
\definecolor{currentstroke}{rgb}{0.000000,0.000000,0.000000}%
\pgfsetstrokecolor{currentstroke}%
\pgfsetdash{}{0pt}%
\pgfsys@defobject{currentmarker}{\pgfqpoint{-0.048611in}{0.000000in}}{\pgfqpoint{0.000000in}{0.000000in}}{%
\pgfpathmoveto{\pgfqpoint{0.000000in}{0.000000in}}%
\pgfpathlineto{\pgfqpoint{-0.048611in}{0.000000in}}%
\pgfusepath{stroke,fill}%
}%
\begin{pgfscope}%
\pgfsys@transformshift{5.814694in}{6.244225in}%
\pgfsys@useobject{currentmarker}{}%
\end{pgfscope}%
\end{pgfscope}%
\begin{pgfscope}%
\pgftext[x=5.429469in,y=6.191464in,left,base]{\rmfamily\fontsize{10.000000}{12.000000}\selectfont \(\displaystyle 10^{-4}\)}%
\end{pgfscope}%
\begin{pgfscope}%
\pgfsetbuttcap%
\pgfsetroundjoin%
\definecolor{currentfill}{rgb}{0.000000,0.000000,0.000000}%
\pgfsetfillcolor{currentfill}%
\pgfsetlinewidth{0.602250pt}%
\definecolor{currentstroke}{rgb}{0.000000,0.000000,0.000000}%
\pgfsetstrokecolor{currentstroke}%
\pgfsetdash{}{0pt}%
\pgfsys@defobject{currentmarker}{\pgfqpoint{-0.027778in}{0.000000in}}{\pgfqpoint{0.000000in}{0.000000in}}{%
\pgfpathmoveto{\pgfqpoint{0.000000in}{0.000000in}}%
\pgfpathlineto{\pgfqpoint{-0.027778in}{0.000000in}}%
\pgfusepath{stroke,fill}%
}%
\begin{pgfscope}%
\pgfsys@transformshift{5.814694in}{4.931814in}%
\pgfsys@useobject{currentmarker}{}%
\end{pgfscope}%
\end{pgfscope}%
\begin{pgfscope}%
\pgfsetbuttcap%
\pgfsetroundjoin%
\definecolor{currentfill}{rgb}{0.000000,0.000000,0.000000}%
\pgfsetfillcolor{currentfill}%
\pgfsetlinewidth{0.602250pt}%
\definecolor{currentstroke}{rgb}{0.000000,0.000000,0.000000}%
\pgfsetstrokecolor{currentstroke}%
\pgfsetdash{}{0pt}%
\pgfsys@defobject{currentmarker}{\pgfqpoint{-0.027778in}{0.000000in}}{\pgfqpoint{0.000000in}{0.000000in}}{%
\pgfpathmoveto{\pgfqpoint{0.000000in}{0.000000in}}%
\pgfpathlineto{\pgfqpoint{-0.027778in}{0.000000in}}%
\pgfusepath{stroke,fill}%
}%
\begin{pgfscope}%
\pgfsys@transformshift{5.814694in}{4.963294in}%
\pgfsys@useobject{currentmarker}{}%
\end{pgfscope}%
\end{pgfscope}%
\begin{pgfscope}%
\pgfsetbuttcap%
\pgfsetroundjoin%
\definecolor{currentfill}{rgb}{0.000000,0.000000,0.000000}%
\pgfsetfillcolor{currentfill}%
\pgfsetlinewidth{0.602250pt}%
\definecolor{currentstroke}{rgb}{0.000000,0.000000,0.000000}%
\pgfsetstrokecolor{currentstroke}%
\pgfsetdash{}{0pt}%
\pgfsys@defobject{currentmarker}{\pgfqpoint{-0.027778in}{0.000000in}}{\pgfqpoint{0.000000in}{0.000000in}}{%
\pgfpathmoveto{\pgfqpoint{0.000000in}{0.000000in}}%
\pgfpathlineto{\pgfqpoint{-0.027778in}{0.000000in}}%
\pgfusepath{stroke,fill}%
}%
\begin{pgfscope}%
\pgfsys@transformshift{5.814694in}{4.989911in}%
\pgfsys@useobject{currentmarker}{}%
\end{pgfscope}%
\end{pgfscope}%
\begin{pgfscope}%
\pgfsetbuttcap%
\pgfsetroundjoin%
\definecolor{currentfill}{rgb}{0.000000,0.000000,0.000000}%
\pgfsetfillcolor{currentfill}%
\pgfsetlinewidth{0.602250pt}%
\definecolor{currentstroke}{rgb}{0.000000,0.000000,0.000000}%
\pgfsetstrokecolor{currentstroke}%
\pgfsetdash{}{0pt}%
\pgfsys@defobject{currentmarker}{\pgfqpoint{-0.027778in}{0.000000in}}{\pgfqpoint{0.000000in}{0.000000in}}{%
\pgfpathmoveto{\pgfqpoint{0.000000in}{0.000000in}}%
\pgfpathlineto{\pgfqpoint{-0.027778in}{0.000000in}}%
\pgfusepath{stroke,fill}%
}%
\begin{pgfscope}%
\pgfsys@transformshift{5.814694in}{5.012967in}%
\pgfsys@useobject{currentmarker}{}%
\end{pgfscope}%
\end{pgfscope}%
\begin{pgfscope}%
\pgfsetbuttcap%
\pgfsetroundjoin%
\definecolor{currentfill}{rgb}{0.000000,0.000000,0.000000}%
\pgfsetfillcolor{currentfill}%
\pgfsetlinewidth{0.602250pt}%
\definecolor{currentstroke}{rgb}{0.000000,0.000000,0.000000}%
\pgfsetstrokecolor{currentstroke}%
\pgfsetdash{}{0pt}%
\pgfsys@defobject{currentmarker}{\pgfqpoint{-0.027778in}{0.000000in}}{\pgfqpoint{0.000000in}{0.000000in}}{%
\pgfpathmoveto{\pgfqpoint{0.000000in}{0.000000in}}%
\pgfpathlineto{\pgfqpoint{-0.027778in}{0.000000in}}%
\pgfusepath{stroke,fill}%
}%
\begin{pgfscope}%
\pgfsys@transformshift{5.814694in}{5.033304in}%
\pgfsys@useobject{currentmarker}{}%
\end{pgfscope}%
\end{pgfscope}%
\begin{pgfscope}%
\pgfsetbuttcap%
\pgfsetroundjoin%
\definecolor{currentfill}{rgb}{0.000000,0.000000,0.000000}%
\pgfsetfillcolor{currentfill}%
\pgfsetlinewidth{0.602250pt}%
\definecolor{currentstroke}{rgb}{0.000000,0.000000,0.000000}%
\pgfsetstrokecolor{currentstroke}%
\pgfsetdash{}{0pt}%
\pgfsys@defobject{currentmarker}{\pgfqpoint{-0.027778in}{0.000000in}}{\pgfqpoint{0.000000in}{0.000000in}}{%
\pgfpathmoveto{\pgfqpoint{0.000000in}{0.000000in}}%
\pgfpathlineto{\pgfqpoint{-0.027778in}{0.000000in}}%
\pgfusepath{stroke,fill}%
}%
\begin{pgfscope}%
\pgfsys@transformshift{5.814694in}{5.171178in}%
\pgfsys@useobject{currentmarker}{}%
\end{pgfscope}%
\end{pgfscope}%
\begin{pgfscope}%
\pgfsetbuttcap%
\pgfsetroundjoin%
\definecolor{currentfill}{rgb}{0.000000,0.000000,0.000000}%
\pgfsetfillcolor{currentfill}%
\pgfsetlinewidth{0.602250pt}%
\definecolor{currentstroke}{rgb}{0.000000,0.000000,0.000000}%
\pgfsetstrokecolor{currentstroke}%
\pgfsetdash{}{0pt}%
\pgfsys@defobject{currentmarker}{\pgfqpoint{-0.027778in}{0.000000in}}{\pgfqpoint{0.000000in}{0.000000in}}{%
\pgfpathmoveto{\pgfqpoint{0.000000in}{0.000000in}}%
\pgfpathlineto{\pgfqpoint{-0.027778in}{0.000000in}}%
\pgfusepath{stroke,fill}%
}%
\begin{pgfscope}%
\pgfsys@transformshift{5.814694in}{5.241188in}%
\pgfsys@useobject{currentmarker}{}%
\end{pgfscope}%
\end{pgfscope}%
\begin{pgfscope}%
\pgfsetbuttcap%
\pgfsetroundjoin%
\definecolor{currentfill}{rgb}{0.000000,0.000000,0.000000}%
\pgfsetfillcolor{currentfill}%
\pgfsetlinewidth{0.602250pt}%
\definecolor{currentstroke}{rgb}{0.000000,0.000000,0.000000}%
\pgfsetstrokecolor{currentstroke}%
\pgfsetdash{}{0pt}%
\pgfsys@defobject{currentmarker}{\pgfqpoint{-0.027778in}{0.000000in}}{\pgfqpoint{0.000000in}{0.000000in}}{%
\pgfpathmoveto{\pgfqpoint{0.000000in}{0.000000in}}%
\pgfpathlineto{\pgfqpoint{-0.027778in}{0.000000in}}%
\pgfusepath{stroke,fill}%
}%
\begin{pgfscope}%
\pgfsys@transformshift{5.814694in}{5.290861in}%
\pgfsys@useobject{currentmarker}{}%
\end{pgfscope}%
\end{pgfscope}%
\begin{pgfscope}%
\pgfsetbuttcap%
\pgfsetroundjoin%
\definecolor{currentfill}{rgb}{0.000000,0.000000,0.000000}%
\pgfsetfillcolor{currentfill}%
\pgfsetlinewidth{0.602250pt}%
\definecolor{currentstroke}{rgb}{0.000000,0.000000,0.000000}%
\pgfsetstrokecolor{currentstroke}%
\pgfsetdash{}{0pt}%
\pgfsys@defobject{currentmarker}{\pgfqpoint{-0.027778in}{0.000000in}}{\pgfqpoint{0.000000in}{0.000000in}}{%
\pgfpathmoveto{\pgfqpoint{0.000000in}{0.000000in}}%
\pgfpathlineto{\pgfqpoint{-0.027778in}{0.000000in}}%
\pgfusepath{stroke,fill}%
}%
\begin{pgfscope}%
\pgfsys@transformshift{5.814694in}{5.329390in}%
\pgfsys@useobject{currentmarker}{}%
\end{pgfscope}%
\end{pgfscope}%
\begin{pgfscope}%
\pgfsetbuttcap%
\pgfsetroundjoin%
\definecolor{currentfill}{rgb}{0.000000,0.000000,0.000000}%
\pgfsetfillcolor{currentfill}%
\pgfsetlinewidth{0.602250pt}%
\definecolor{currentstroke}{rgb}{0.000000,0.000000,0.000000}%
\pgfsetstrokecolor{currentstroke}%
\pgfsetdash{}{0pt}%
\pgfsys@defobject{currentmarker}{\pgfqpoint{-0.027778in}{0.000000in}}{\pgfqpoint{0.000000in}{0.000000in}}{%
\pgfpathmoveto{\pgfqpoint{0.000000in}{0.000000in}}%
\pgfpathlineto{\pgfqpoint{-0.027778in}{0.000000in}}%
\pgfusepath{stroke,fill}%
}%
\begin{pgfscope}%
\pgfsys@transformshift{5.814694in}{5.360871in}%
\pgfsys@useobject{currentmarker}{}%
\end{pgfscope}%
\end{pgfscope}%
\begin{pgfscope}%
\pgfsetbuttcap%
\pgfsetroundjoin%
\definecolor{currentfill}{rgb}{0.000000,0.000000,0.000000}%
\pgfsetfillcolor{currentfill}%
\pgfsetlinewidth{0.602250pt}%
\definecolor{currentstroke}{rgb}{0.000000,0.000000,0.000000}%
\pgfsetstrokecolor{currentstroke}%
\pgfsetdash{}{0pt}%
\pgfsys@defobject{currentmarker}{\pgfqpoint{-0.027778in}{0.000000in}}{\pgfqpoint{0.000000in}{0.000000in}}{%
\pgfpathmoveto{\pgfqpoint{0.000000in}{0.000000in}}%
\pgfpathlineto{\pgfqpoint{-0.027778in}{0.000000in}}%
\pgfusepath{stroke,fill}%
}%
\begin{pgfscope}%
\pgfsys@transformshift{5.814694in}{5.387487in}%
\pgfsys@useobject{currentmarker}{}%
\end{pgfscope}%
\end{pgfscope}%
\begin{pgfscope}%
\pgfsetbuttcap%
\pgfsetroundjoin%
\definecolor{currentfill}{rgb}{0.000000,0.000000,0.000000}%
\pgfsetfillcolor{currentfill}%
\pgfsetlinewidth{0.602250pt}%
\definecolor{currentstroke}{rgb}{0.000000,0.000000,0.000000}%
\pgfsetstrokecolor{currentstroke}%
\pgfsetdash{}{0pt}%
\pgfsys@defobject{currentmarker}{\pgfqpoint{-0.027778in}{0.000000in}}{\pgfqpoint{0.000000in}{0.000000in}}{%
\pgfpathmoveto{\pgfqpoint{0.000000in}{0.000000in}}%
\pgfpathlineto{\pgfqpoint{-0.027778in}{0.000000in}}%
\pgfusepath{stroke,fill}%
}%
\begin{pgfscope}%
\pgfsys@transformshift{5.814694in}{5.410543in}%
\pgfsys@useobject{currentmarker}{}%
\end{pgfscope}%
\end{pgfscope}%
\begin{pgfscope}%
\pgfsetbuttcap%
\pgfsetroundjoin%
\definecolor{currentfill}{rgb}{0.000000,0.000000,0.000000}%
\pgfsetfillcolor{currentfill}%
\pgfsetlinewidth{0.602250pt}%
\definecolor{currentstroke}{rgb}{0.000000,0.000000,0.000000}%
\pgfsetstrokecolor{currentstroke}%
\pgfsetdash{}{0pt}%
\pgfsys@defobject{currentmarker}{\pgfqpoint{-0.027778in}{0.000000in}}{\pgfqpoint{0.000000in}{0.000000in}}{%
\pgfpathmoveto{\pgfqpoint{0.000000in}{0.000000in}}%
\pgfpathlineto{\pgfqpoint{-0.027778in}{0.000000in}}%
\pgfusepath{stroke,fill}%
}%
\begin{pgfscope}%
\pgfsys@transformshift{5.814694in}{5.430880in}%
\pgfsys@useobject{currentmarker}{}%
\end{pgfscope}%
\end{pgfscope}%
\begin{pgfscope}%
\pgfsetbuttcap%
\pgfsetroundjoin%
\definecolor{currentfill}{rgb}{0.000000,0.000000,0.000000}%
\pgfsetfillcolor{currentfill}%
\pgfsetlinewidth{0.602250pt}%
\definecolor{currentstroke}{rgb}{0.000000,0.000000,0.000000}%
\pgfsetstrokecolor{currentstroke}%
\pgfsetdash{}{0pt}%
\pgfsys@defobject{currentmarker}{\pgfqpoint{-0.027778in}{0.000000in}}{\pgfqpoint{0.000000in}{0.000000in}}{%
\pgfpathmoveto{\pgfqpoint{0.000000in}{0.000000in}}%
\pgfpathlineto{\pgfqpoint{-0.027778in}{0.000000in}}%
\pgfusepath{stroke,fill}%
}%
\begin{pgfscope}%
\pgfsys@transformshift{5.814694in}{5.568755in}%
\pgfsys@useobject{currentmarker}{}%
\end{pgfscope}%
\end{pgfscope}%
\begin{pgfscope}%
\pgfsetbuttcap%
\pgfsetroundjoin%
\definecolor{currentfill}{rgb}{0.000000,0.000000,0.000000}%
\pgfsetfillcolor{currentfill}%
\pgfsetlinewidth{0.602250pt}%
\definecolor{currentstroke}{rgb}{0.000000,0.000000,0.000000}%
\pgfsetstrokecolor{currentstroke}%
\pgfsetdash{}{0pt}%
\pgfsys@defobject{currentmarker}{\pgfqpoint{-0.027778in}{0.000000in}}{\pgfqpoint{0.000000in}{0.000000in}}{%
\pgfpathmoveto{\pgfqpoint{0.000000in}{0.000000in}}%
\pgfpathlineto{\pgfqpoint{-0.027778in}{0.000000in}}%
\pgfusepath{stroke,fill}%
}%
\begin{pgfscope}%
\pgfsys@transformshift{5.814694in}{5.638765in}%
\pgfsys@useobject{currentmarker}{}%
\end{pgfscope}%
\end{pgfscope}%
\begin{pgfscope}%
\pgfsetbuttcap%
\pgfsetroundjoin%
\definecolor{currentfill}{rgb}{0.000000,0.000000,0.000000}%
\pgfsetfillcolor{currentfill}%
\pgfsetlinewidth{0.602250pt}%
\definecolor{currentstroke}{rgb}{0.000000,0.000000,0.000000}%
\pgfsetstrokecolor{currentstroke}%
\pgfsetdash{}{0pt}%
\pgfsys@defobject{currentmarker}{\pgfqpoint{-0.027778in}{0.000000in}}{\pgfqpoint{0.000000in}{0.000000in}}{%
\pgfpathmoveto{\pgfqpoint{0.000000in}{0.000000in}}%
\pgfpathlineto{\pgfqpoint{-0.027778in}{0.000000in}}%
\pgfusepath{stroke,fill}%
}%
\begin{pgfscope}%
\pgfsys@transformshift{5.814694in}{5.688437in}%
\pgfsys@useobject{currentmarker}{}%
\end{pgfscope}%
\end{pgfscope}%
\begin{pgfscope}%
\pgfsetbuttcap%
\pgfsetroundjoin%
\definecolor{currentfill}{rgb}{0.000000,0.000000,0.000000}%
\pgfsetfillcolor{currentfill}%
\pgfsetlinewidth{0.602250pt}%
\definecolor{currentstroke}{rgb}{0.000000,0.000000,0.000000}%
\pgfsetstrokecolor{currentstroke}%
\pgfsetdash{}{0pt}%
\pgfsys@defobject{currentmarker}{\pgfqpoint{-0.027778in}{0.000000in}}{\pgfqpoint{0.000000in}{0.000000in}}{%
\pgfpathmoveto{\pgfqpoint{0.000000in}{0.000000in}}%
\pgfpathlineto{\pgfqpoint{-0.027778in}{0.000000in}}%
\pgfusepath{stroke,fill}%
}%
\begin{pgfscope}%
\pgfsys@transformshift{5.814694in}{5.726966in}%
\pgfsys@useobject{currentmarker}{}%
\end{pgfscope}%
\end{pgfscope}%
\begin{pgfscope}%
\pgfsetbuttcap%
\pgfsetroundjoin%
\definecolor{currentfill}{rgb}{0.000000,0.000000,0.000000}%
\pgfsetfillcolor{currentfill}%
\pgfsetlinewidth{0.602250pt}%
\definecolor{currentstroke}{rgb}{0.000000,0.000000,0.000000}%
\pgfsetstrokecolor{currentstroke}%
\pgfsetdash{}{0pt}%
\pgfsys@defobject{currentmarker}{\pgfqpoint{-0.027778in}{0.000000in}}{\pgfqpoint{0.000000in}{0.000000in}}{%
\pgfpathmoveto{\pgfqpoint{0.000000in}{0.000000in}}%
\pgfpathlineto{\pgfqpoint{-0.027778in}{0.000000in}}%
\pgfusepath{stroke,fill}%
}%
\begin{pgfscope}%
\pgfsys@transformshift{5.814694in}{5.758447in}%
\pgfsys@useobject{currentmarker}{}%
\end{pgfscope}%
\end{pgfscope}%
\begin{pgfscope}%
\pgfsetbuttcap%
\pgfsetroundjoin%
\definecolor{currentfill}{rgb}{0.000000,0.000000,0.000000}%
\pgfsetfillcolor{currentfill}%
\pgfsetlinewidth{0.602250pt}%
\definecolor{currentstroke}{rgb}{0.000000,0.000000,0.000000}%
\pgfsetstrokecolor{currentstroke}%
\pgfsetdash{}{0pt}%
\pgfsys@defobject{currentmarker}{\pgfqpoint{-0.027778in}{0.000000in}}{\pgfqpoint{0.000000in}{0.000000in}}{%
\pgfpathmoveto{\pgfqpoint{0.000000in}{0.000000in}}%
\pgfpathlineto{\pgfqpoint{-0.027778in}{0.000000in}}%
\pgfusepath{stroke,fill}%
}%
\begin{pgfscope}%
\pgfsys@transformshift{5.814694in}{5.785063in}%
\pgfsys@useobject{currentmarker}{}%
\end{pgfscope}%
\end{pgfscope}%
\begin{pgfscope}%
\pgfsetbuttcap%
\pgfsetroundjoin%
\definecolor{currentfill}{rgb}{0.000000,0.000000,0.000000}%
\pgfsetfillcolor{currentfill}%
\pgfsetlinewidth{0.602250pt}%
\definecolor{currentstroke}{rgb}{0.000000,0.000000,0.000000}%
\pgfsetstrokecolor{currentstroke}%
\pgfsetdash{}{0pt}%
\pgfsys@defobject{currentmarker}{\pgfqpoint{-0.027778in}{0.000000in}}{\pgfqpoint{0.000000in}{0.000000in}}{%
\pgfpathmoveto{\pgfqpoint{0.000000in}{0.000000in}}%
\pgfpathlineto{\pgfqpoint{-0.027778in}{0.000000in}}%
\pgfusepath{stroke,fill}%
}%
\begin{pgfscope}%
\pgfsys@transformshift{5.814694in}{5.808120in}%
\pgfsys@useobject{currentmarker}{}%
\end{pgfscope}%
\end{pgfscope}%
\begin{pgfscope}%
\pgfsetbuttcap%
\pgfsetroundjoin%
\definecolor{currentfill}{rgb}{0.000000,0.000000,0.000000}%
\pgfsetfillcolor{currentfill}%
\pgfsetlinewidth{0.602250pt}%
\definecolor{currentstroke}{rgb}{0.000000,0.000000,0.000000}%
\pgfsetstrokecolor{currentstroke}%
\pgfsetdash{}{0pt}%
\pgfsys@defobject{currentmarker}{\pgfqpoint{-0.027778in}{0.000000in}}{\pgfqpoint{0.000000in}{0.000000in}}{%
\pgfpathmoveto{\pgfqpoint{0.000000in}{0.000000in}}%
\pgfpathlineto{\pgfqpoint{-0.027778in}{0.000000in}}%
\pgfusepath{stroke,fill}%
}%
\begin{pgfscope}%
\pgfsys@transformshift{5.814694in}{5.828457in}%
\pgfsys@useobject{currentmarker}{}%
\end{pgfscope}%
\end{pgfscope}%
\begin{pgfscope}%
\pgfsetbuttcap%
\pgfsetroundjoin%
\definecolor{currentfill}{rgb}{0.000000,0.000000,0.000000}%
\pgfsetfillcolor{currentfill}%
\pgfsetlinewidth{0.602250pt}%
\definecolor{currentstroke}{rgb}{0.000000,0.000000,0.000000}%
\pgfsetstrokecolor{currentstroke}%
\pgfsetdash{}{0pt}%
\pgfsys@defobject{currentmarker}{\pgfqpoint{-0.027778in}{0.000000in}}{\pgfqpoint{0.000000in}{0.000000in}}{%
\pgfpathmoveto{\pgfqpoint{0.000000in}{0.000000in}}%
\pgfpathlineto{\pgfqpoint{-0.027778in}{0.000000in}}%
\pgfusepath{stroke,fill}%
}%
\begin{pgfscope}%
\pgfsys@transformshift{5.814694in}{5.966331in}%
\pgfsys@useobject{currentmarker}{}%
\end{pgfscope}%
\end{pgfscope}%
\begin{pgfscope}%
\pgfsetbuttcap%
\pgfsetroundjoin%
\definecolor{currentfill}{rgb}{0.000000,0.000000,0.000000}%
\pgfsetfillcolor{currentfill}%
\pgfsetlinewidth{0.602250pt}%
\definecolor{currentstroke}{rgb}{0.000000,0.000000,0.000000}%
\pgfsetstrokecolor{currentstroke}%
\pgfsetdash{}{0pt}%
\pgfsys@defobject{currentmarker}{\pgfqpoint{-0.027778in}{0.000000in}}{\pgfqpoint{0.000000in}{0.000000in}}{%
\pgfpathmoveto{\pgfqpoint{0.000000in}{0.000000in}}%
\pgfpathlineto{\pgfqpoint{-0.027778in}{0.000000in}}%
\pgfusepath{stroke,fill}%
}%
\begin{pgfscope}%
\pgfsys@transformshift{5.814694in}{6.036341in}%
\pgfsys@useobject{currentmarker}{}%
\end{pgfscope}%
\end{pgfscope}%
\begin{pgfscope}%
\pgfsetbuttcap%
\pgfsetroundjoin%
\definecolor{currentfill}{rgb}{0.000000,0.000000,0.000000}%
\pgfsetfillcolor{currentfill}%
\pgfsetlinewidth{0.602250pt}%
\definecolor{currentstroke}{rgb}{0.000000,0.000000,0.000000}%
\pgfsetstrokecolor{currentstroke}%
\pgfsetdash{}{0pt}%
\pgfsys@defobject{currentmarker}{\pgfqpoint{-0.027778in}{0.000000in}}{\pgfqpoint{0.000000in}{0.000000in}}{%
\pgfpathmoveto{\pgfqpoint{0.000000in}{0.000000in}}%
\pgfpathlineto{\pgfqpoint{-0.027778in}{0.000000in}}%
\pgfusepath{stroke,fill}%
}%
\begin{pgfscope}%
\pgfsys@transformshift{5.814694in}{6.086014in}%
\pgfsys@useobject{currentmarker}{}%
\end{pgfscope}%
\end{pgfscope}%
\begin{pgfscope}%
\pgfsetbuttcap%
\pgfsetroundjoin%
\definecolor{currentfill}{rgb}{0.000000,0.000000,0.000000}%
\pgfsetfillcolor{currentfill}%
\pgfsetlinewidth{0.602250pt}%
\definecolor{currentstroke}{rgb}{0.000000,0.000000,0.000000}%
\pgfsetstrokecolor{currentstroke}%
\pgfsetdash{}{0pt}%
\pgfsys@defobject{currentmarker}{\pgfqpoint{-0.027778in}{0.000000in}}{\pgfqpoint{0.000000in}{0.000000in}}{%
\pgfpathmoveto{\pgfqpoint{0.000000in}{0.000000in}}%
\pgfpathlineto{\pgfqpoint{-0.027778in}{0.000000in}}%
\pgfusepath{stroke,fill}%
}%
\begin{pgfscope}%
\pgfsys@transformshift{5.814694in}{6.124543in}%
\pgfsys@useobject{currentmarker}{}%
\end{pgfscope}%
\end{pgfscope}%
\begin{pgfscope}%
\pgfsetbuttcap%
\pgfsetroundjoin%
\definecolor{currentfill}{rgb}{0.000000,0.000000,0.000000}%
\pgfsetfillcolor{currentfill}%
\pgfsetlinewidth{0.602250pt}%
\definecolor{currentstroke}{rgb}{0.000000,0.000000,0.000000}%
\pgfsetstrokecolor{currentstroke}%
\pgfsetdash{}{0pt}%
\pgfsys@defobject{currentmarker}{\pgfqpoint{-0.027778in}{0.000000in}}{\pgfqpoint{0.000000in}{0.000000in}}{%
\pgfpathmoveto{\pgfqpoint{0.000000in}{0.000000in}}%
\pgfpathlineto{\pgfqpoint{-0.027778in}{0.000000in}}%
\pgfusepath{stroke,fill}%
}%
\begin{pgfscope}%
\pgfsys@transformshift{5.814694in}{6.156023in}%
\pgfsys@useobject{currentmarker}{}%
\end{pgfscope}%
\end{pgfscope}%
\begin{pgfscope}%
\pgfsetbuttcap%
\pgfsetroundjoin%
\definecolor{currentfill}{rgb}{0.000000,0.000000,0.000000}%
\pgfsetfillcolor{currentfill}%
\pgfsetlinewidth{0.602250pt}%
\definecolor{currentstroke}{rgb}{0.000000,0.000000,0.000000}%
\pgfsetstrokecolor{currentstroke}%
\pgfsetdash{}{0pt}%
\pgfsys@defobject{currentmarker}{\pgfqpoint{-0.027778in}{0.000000in}}{\pgfqpoint{0.000000in}{0.000000in}}{%
\pgfpathmoveto{\pgfqpoint{0.000000in}{0.000000in}}%
\pgfpathlineto{\pgfqpoint{-0.027778in}{0.000000in}}%
\pgfusepath{stroke,fill}%
}%
\begin{pgfscope}%
\pgfsys@transformshift{5.814694in}{6.182640in}%
\pgfsys@useobject{currentmarker}{}%
\end{pgfscope}%
\end{pgfscope}%
\begin{pgfscope}%
\pgfsetbuttcap%
\pgfsetroundjoin%
\definecolor{currentfill}{rgb}{0.000000,0.000000,0.000000}%
\pgfsetfillcolor{currentfill}%
\pgfsetlinewidth{0.602250pt}%
\definecolor{currentstroke}{rgb}{0.000000,0.000000,0.000000}%
\pgfsetstrokecolor{currentstroke}%
\pgfsetdash{}{0pt}%
\pgfsys@defobject{currentmarker}{\pgfqpoint{-0.027778in}{0.000000in}}{\pgfqpoint{0.000000in}{0.000000in}}{%
\pgfpathmoveto{\pgfqpoint{0.000000in}{0.000000in}}%
\pgfpathlineto{\pgfqpoint{-0.027778in}{0.000000in}}%
\pgfusepath{stroke,fill}%
}%
\begin{pgfscope}%
\pgfsys@transformshift{5.814694in}{6.205696in}%
\pgfsys@useobject{currentmarker}{}%
\end{pgfscope}%
\end{pgfscope}%
\begin{pgfscope}%
\pgfsetbuttcap%
\pgfsetroundjoin%
\definecolor{currentfill}{rgb}{0.000000,0.000000,0.000000}%
\pgfsetfillcolor{currentfill}%
\pgfsetlinewidth{0.602250pt}%
\definecolor{currentstroke}{rgb}{0.000000,0.000000,0.000000}%
\pgfsetstrokecolor{currentstroke}%
\pgfsetdash{}{0pt}%
\pgfsys@defobject{currentmarker}{\pgfqpoint{-0.027778in}{0.000000in}}{\pgfqpoint{0.000000in}{0.000000in}}{%
\pgfpathmoveto{\pgfqpoint{0.000000in}{0.000000in}}%
\pgfpathlineto{\pgfqpoint{-0.027778in}{0.000000in}}%
\pgfusepath{stroke,fill}%
}%
\begin{pgfscope}%
\pgfsys@transformshift{5.814694in}{6.226033in}%
\pgfsys@useobject{currentmarker}{}%
\end{pgfscope}%
\end{pgfscope}%
\begin{pgfscope}%
\pgfpathrectangle{\pgfqpoint{5.814694in}{4.908628in}}{\pgfqpoint{1.897959in}{1.372727in}} %
\pgfusepath{clip}%
\pgfsetbuttcap%
\pgfsetroundjoin%
\pgfsetlinewidth{1.505625pt}%
\definecolor{currentstroke}{rgb}{1.000000,0.000000,0.000000}%
\pgfsetstrokecolor{currentstroke}%
\pgfsetdash{{5.550000pt}{2.400000pt}}{0.000000pt}%
\pgfpathmoveto{\pgfqpoint{5.900965in}{6.125484in}}%
\pgfpathlineto{\pgfqpoint{5.975983in}{6.122537in}}%
\pgfpathlineto{\pgfqpoint{6.051001in}{6.119723in}}%
\pgfpathlineto{\pgfqpoint{6.126019in}{6.117037in}}%
\pgfpathlineto{\pgfqpoint{6.201037in}{6.114477in}}%
\pgfpathlineto{\pgfqpoint{6.276055in}{6.112038in}}%
\pgfpathlineto{\pgfqpoint{6.351074in}{6.109718in}}%
\pgfpathlineto{\pgfqpoint{6.426092in}{6.107516in}}%
\pgfpathlineto{\pgfqpoint{6.501110in}{6.105429in}}%
\pgfpathlineto{\pgfqpoint{6.576128in}{6.103456in}}%
\pgfpathlineto{\pgfqpoint{6.651146in}{6.101596in}}%
\pgfpathlineto{\pgfqpoint{6.726164in}{6.099848in}}%
\pgfpathlineto{\pgfqpoint{6.801183in}{6.098210in}}%
\pgfpathlineto{\pgfqpoint{6.876201in}{6.096681in}}%
\pgfpathlineto{\pgfqpoint{6.951219in}{6.095261in}}%
\pgfpathlineto{\pgfqpoint{7.026237in}{6.093948in}}%
\pgfpathlineto{\pgfqpoint{7.101255in}{6.092743in}}%
\pgfpathlineto{\pgfqpoint{7.176273in}{6.091643in}}%
\pgfpathlineto{\pgfqpoint{7.251291in}{6.090649in}}%
\pgfpathlineto{\pgfqpoint{7.326310in}{6.089760in}}%
\pgfpathlineto{\pgfqpoint{7.401328in}{6.088975in}}%
\pgfpathlineto{\pgfqpoint{7.476346in}{6.088294in}}%
\pgfpathlineto{\pgfqpoint{7.551364in}{6.087717in}}%
\pgfpathlineto{\pgfqpoint{7.626382in}{6.087243in}}%
\pgfusepath{stroke}%
\end{pgfscope}%
\begin{pgfscope}%
\pgfpathrectangle{\pgfqpoint{5.814694in}{4.908628in}}{\pgfqpoint{1.897959in}{1.372727in}} %
\pgfusepath{clip}%
\pgfsetbuttcap%
\pgfsetmiterjoin%
\definecolor{currentfill}{rgb}{1.000000,0.000000,0.000000}%
\pgfsetfillcolor{currentfill}%
\pgfsetlinewidth{1.003750pt}%
\definecolor{currentstroke}{rgb}{1.000000,0.000000,0.000000}%
\pgfsetstrokecolor{currentstroke}%
\pgfsetdash{}{0pt}%
\pgfsys@defobject{currentmarker}{\pgfqpoint{-0.041667in}{-0.041667in}}{\pgfqpoint{0.041667in}{0.041667in}}{%
\pgfpathmoveto{\pgfqpoint{-0.041667in}{-0.041667in}}%
\pgfpathlineto{\pgfqpoint{0.041667in}{-0.041667in}}%
\pgfpathlineto{\pgfqpoint{0.041667in}{0.041667in}}%
\pgfpathlineto{\pgfqpoint{-0.041667in}{0.041667in}}%
\pgfpathclose%
\pgfusepath{stroke,fill}%
}%
\begin{pgfscope}%
\pgfsys@transformshift{5.900965in}{6.125484in}%
\pgfsys@useobject{currentmarker}{}%
\end{pgfscope}%
\begin{pgfscope}%
\pgfsys@transformshift{6.276055in}{6.112038in}%
\pgfsys@useobject{currentmarker}{}%
\end{pgfscope}%
\begin{pgfscope}%
\pgfsys@transformshift{6.651146in}{6.101596in}%
\pgfsys@useobject{currentmarker}{}%
\end{pgfscope}%
\begin{pgfscope}%
\pgfsys@transformshift{7.026237in}{6.093948in}%
\pgfsys@useobject{currentmarker}{}%
\end{pgfscope}%
\begin{pgfscope}%
\pgfsys@transformshift{7.401328in}{6.088975in}%
\pgfsys@useobject{currentmarker}{}%
\end{pgfscope}%
\end{pgfscope}%
\begin{pgfscope}%
\pgfpathrectangle{\pgfqpoint{5.814694in}{4.908628in}}{\pgfqpoint{1.897959in}{1.372727in}} %
\pgfusepath{clip}%
\pgfsetrectcap%
\pgfsetroundjoin%
\pgfsetlinewidth{1.505625pt}%
\definecolor{currentstroke}{rgb}{0.000000,0.000000,1.000000}%
\pgfsetstrokecolor{currentstroke}%
\pgfsetdash{}{0pt}%
\pgfpathmoveto{\pgfqpoint{5.900965in}{5.420144in}}%
\pgfpathlineto{\pgfqpoint{5.975983in}{5.407002in}}%
\pgfpathlineto{\pgfqpoint{6.051001in}{5.394089in}}%
\pgfpathlineto{\pgfqpoint{6.126019in}{5.381273in}}%
\pgfpathlineto{\pgfqpoint{6.201037in}{5.368448in}}%
\pgfpathlineto{\pgfqpoint{6.276055in}{5.355526in}}%
\pgfpathlineto{\pgfqpoint{6.351074in}{5.342430in}}%
\pgfpathlineto{\pgfqpoint{6.426092in}{5.329086in}}%
\pgfpathlineto{\pgfqpoint{6.501110in}{5.315421in}}%
\pgfpathlineto{\pgfqpoint{6.576128in}{5.301362in}}%
\pgfpathlineto{\pgfqpoint{6.651146in}{5.286829in}}%
\pgfpathlineto{\pgfqpoint{6.726164in}{5.271735in}}%
\pgfpathlineto{\pgfqpoint{6.801183in}{5.255981in}}%
\pgfpathlineto{\pgfqpoint{6.876201in}{5.239453in}}%
\pgfpathlineto{\pgfqpoint{6.951219in}{5.222016in}}%
\pgfpathlineto{\pgfqpoint{7.026237in}{5.203506in}}%
\pgfpathlineto{\pgfqpoint{7.101255in}{5.183721in}}%
\pgfpathlineto{\pgfqpoint{7.176273in}{5.162406in}}%
\pgfpathlineto{\pgfqpoint{7.251291in}{5.139227in}}%
\pgfpathlineto{\pgfqpoint{7.326310in}{5.113742in}}%
\pgfpathlineto{\pgfqpoint{7.401328in}{5.085338in}}%
\pgfpathlineto{\pgfqpoint{7.476346in}{5.053129in}}%
\pgfpathlineto{\pgfqpoint{7.551364in}{5.015764in}}%
\pgfpathlineto{\pgfqpoint{7.626382in}{4.971024in}}%
\pgfusepath{stroke}%
\end{pgfscope}%
\begin{pgfscope}%
\pgfpathrectangle{\pgfqpoint{5.814694in}{4.908628in}}{\pgfqpoint{1.897959in}{1.372727in}} %
\pgfusepath{clip}%
\pgfsetbuttcap%
\pgfsetroundjoin%
\definecolor{currentfill}{rgb}{0.000000,0.000000,1.000000}%
\pgfsetfillcolor{currentfill}%
\pgfsetlinewidth{1.003750pt}%
\definecolor{currentstroke}{rgb}{0.000000,0.000000,1.000000}%
\pgfsetstrokecolor{currentstroke}%
\pgfsetdash{}{0pt}%
\pgfsys@defobject{currentmarker}{\pgfqpoint{-0.041667in}{-0.041667in}}{\pgfqpoint{0.041667in}{0.041667in}}{%
\pgfpathmoveto{\pgfqpoint{0.000000in}{-0.041667in}}%
\pgfpathcurveto{\pgfqpoint{0.011050in}{-0.041667in}}{\pgfqpoint{0.021649in}{-0.037276in}}{\pgfqpoint{0.029463in}{-0.029463in}}%
\pgfpathcurveto{\pgfqpoint{0.037276in}{-0.021649in}}{\pgfqpoint{0.041667in}{-0.011050in}}{\pgfqpoint{0.041667in}{0.000000in}}%
\pgfpathcurveto{\pgfqpoint{0.041667in}{0.011050in}}{\pgfqpoint{0.037276in}{0.021649in}}{\pgfqpoint{0.029463in}{0.029463in}}%
\pgfpathcurveto{\pgfqpoint{0.021649in}{0.037276in}}{\pgfqpoint{0.011050in}{0.041667in}}{\pgfqpoint{0.000000in}{0.041667in}}%
\pgfpathcurveto{\pgfqpoint{-0.011050in}{0.041667in}}{\pgfqpoint{-0.021649in}{0.037276in}}{\pgfqpoint{-0.029463in}{0.029463in}}%
\pgfpathcurveto{\pgfqpoint{-0.037276in}{0.021649in}}{\pgfqpoint{-0.041667in}{0.011050in}}{\pgfqpoint{-0.041667in}{0.000000in}}%
\pgfpathcurveto{\pgfqpoint{-0.041667in}{-0.011050in}}{\pgfqpoint{-0.037276in}{-0.021649in}}{\pgfqpoint{-0.029463in}{-0.029463in}}%
\pgfpathcurveto{\pgfqpoint{-0.021649in}{-0.037276in}}{\pgfqpoint{-0.011050in}{-0.041667in}}{\pgfqpoint{0.000000in}{-0.041667in}}%
\pgfpathclose%
\pgfusepath{stroke,fill}%
}%
\begin{pgfscope}%
\pgfsys@transformshift{5.900965in}{5.420144in}%
\pgfsys@useobject{currentmarker}{}%
\end{pgfscope}%
\begin{pgfscope}%
\pgfsys@transformshift{6.276055in}{5.355526in}%
\pgfsys@useobject{currentmarker}{}%
\end{pgfscope}%
\begin{pgfscope}%
\pgfsys@transformshift{6.651146in}{5.286829in}%
\pgfsys@useobject{currentmarker}{}%
\end{pgfscope}%
\begin{pgfscope}%
\pgfsys@transformshift{7.026237in}{5.203506in}%
\pgfsys@useobject{currentmarker}{}%
\end{pgfscope}%
\begin{pgfscope}%
\pgfsys@transformshift{7.401328in}{5.085338in}%
\pgfsys@useobject{currentmarker}{}%
\end{pgfscope}%
\end{pgfscope}%
\begin{pgfscope}%
\pgfpathrectangle{\pgfqpoint{5.814694in}{4.908628in}}{\pgfqpoint{1.897959in}{1.372727in}} %
\pgfusepath{clip}%
\pgfsetbuttcap%
\pgfsetroundjoin%
\pgfsetlinewidth{1.505625pt}%
\definecolor{currentstroke}{rgb}{0.000000,0.750000,0.750000}%
\pgfsetstrokecolor{currentstroke}%
\pgfsetdash{{9.600000pt}{2.400000pt}{1.500000pt}{2.400000pt}}{0.000000pt}%
\pgfpathmoveto{\pgfqpoint{5.900965in}{6.066668in}}%
\pgfpathlineto{\pgfqpoint{5.975983in}{6.039105in}}%
\pgfpathlineto{\pgfqpoint{6.051001in}{6.014896in}}%
\pgfpathlineto{\pgfqpoint{6.126019in}{5.993435in}}%
\pgfpathlineto{\pgfqpoint{6.201037in}{5.974273in}}%
\pgfpathlineto{\pgfqpoint{6.276055in}{5.957067in}}%
\pgfpathlineto{\pgfqpoint{6.351074in}{5.941547in}}%
\pgfpathlineto{\pgfqpoint{6.426092in}{5.927502in}}%
\pgfpathlineto{\pgfqpoint{6.501110in}{5.914760in}}%
\pgfpathlineto{\pgfqpoint{6.576128in}{5.903179in}}%
\pgfpathlineto{\pgfqpoint{6.651146in}{5.892644in}}%
\pgfpathlineto{\pgfqpoint{6.726164in}{5.883058in}}%
\pgfpathlineto{\pgfqpoint{6.801183in}{5.874338in}}%
\pgfpathlineto{\pgfqpoint{6.876201in}{5.866417in}}%
\pgfpathlineto{\pgfqpoint{6.951219in}{5.859235in}}%
\pgfpathlineto{\pgfqpoint{7.026237in}{5.852741in}}%
\pgfpathlineto{\pgfqpoint{7.101255in}{5.846894in}}%
\pgfpathlineto{\pgfqpoint{7.176273in}{5.841655in}}%
\pgfpathlineto{\pgfqpoint{7.251291in}{5.836994in}}%
\pgfpathlineto{\pgfqpoint{7.326310in}{5.832883in}}%
\pgfpathlineto{\pgfqpoint{7.401328in}{5.829298in}}%
\pgfpathlineto{\pgfqpoint{7.476346in}{5.826221in}}%
\pgfpathlineto{\pgfqpoint{7.551364in}{5.823636in}}%
\pgfpathlineto{\pgfqpoint{7.626382in}{5.821528in}}%
\pgfusepath{stroke}%
\end{pgfscope}%
\begin{pgfscope}%
\pgfpathrectangle{\pgfqpoint{5.814694in}{4.908628in}}{\pgfqpoint{1.897959in}{1.372727in}} %
\pgfusepath{clip}%
\pgfsetbuttcap%
\pgfsetmiterjoin%
\definecolor{currentfill}{rgb}{0.000000,0.750000,0.750000}%
\pgfsetfillcolor{currentfill}%
\pgfsetlinewidth{1.003750pt}%
\definecolor{currentstroke}{rgb}{0.000000,0.750000,0.750000}%
\pgfsetstrokecolor{currentstroke}%
\pgfsetdash{}{0pt}%
\pgfsys@defobject{currentmarker}{\pgfqpoint{-0.041667in}{-0.041667in}}{\pgfqpoint{0.041667in}{0.041667in}}{%
\pgfpathmoveto{\pgfqpoint{-0.000000in}{-0.041667in}}%
\pgfpathlineto{\pgfqpoint{0.041667in}{0.041667in}}%
\pgfpathlineto{\pgfqpoint{-0.041667in}{0.041667in}}%
\pgfpathclose%
\pgfusepath{stroke,fill}%
}%
\begin{pgfscope}%
\pgfsys@transformshift{5.900965in}{6.066668in}%
\pgfsys@useobject{currentmarker}{}%
\end{pgfscope}%
\begin{pgfscope}%
\pgfsys@transformshift{6.276055in}{5.957067in}%
\pgfsys@useobject{currentmarker}{}%
\end{pgfscope}%
\begin{pgfscope}%
\pgfsys@transformshift{6.651146in}{5.892644in}%
\pgfsys@useobject{currentmarker}{}%
\end{pgfscope}%
\begin{pgfscope}%
\pgfsys@transformshift{7.026237in}{5.852741in}%
\pgfsys@useobject{currentmarker}{}%
\end{pgfscope}%
\begin{pgfscope}%
\pgfsys@transformshift{7.401328in}{5.829298in}%
\pgfsys@useobject{currentmarker}{}%
\end{pgfscope}%
\end{pgfscope}%
\begin{pgfscope}%
\pgfpathrectangle{\pgfqpoint{5.814694in}{4.908628in}}{\pgfqpoint{1.897959in}{1.372727in}} %
\pgfusepath{clip}%
\pgfsetbuttcap%
\pgfsetroundjoin%
\pgfsetlinewidth{1.505625pt}%
\definecolor{currentstroke}{rgb}{0.000000,0.000000,0.000000}%
\pgfsetstrokecolor{currentstroke}%
\pgfsetdash{{1.500000pt}{2.475000pt}}{0.000000pt}%
\pgfpathmoveto{\pgfqpoint{5.900965in}{6.218958in}}%
\pgfpathlineto{\pgfqpoint{5.975983in}{6.207656in}}%
\pgfpathlineto{\pgfqpoint{6.051001in}{6.196848in}}%
\pgfpathlineto{\pgfqpoint{6.126019in}{6.187653in}}%
\pgfpathlineto{\pgfqpoint{6.201037in}{6.179711in}}%
\pgfpathlineto{\pgfqpoint{6.276055in}{6.172768in}}%
\pgfpathlineto{\pgfqpoint{6.351074in}{6.166637in}}%
\pgfpathlineto{\pgfqpoint{6.426092in}{6.161179in}}%
\pgfpathlineto{\pgfqpoint{6.501110in}{6.156290in}}%
\pgfpathlineto{\pgfqpoint{6.576128in}{6.151888in}}%
\pgfpathlineto{\pgfqpoint{6.651146in}{6.147911in}}%
\pgfpathlineto{\pgfqpoint{6.726164in}{6.144308in}}%
\pgfpathlineto{\pgfqpoint{6.801183in}{6.141039in}}%
\pgfpathlineto{\pgfqpoint{6.876201in}{6.138074in}}%
\pgfpathlineto{\pgfqpoint{6.951219in}{6.135385in}}%
\pgfpathlineto{\pgfqpoint{7.026237in}{6.132950in}}%
\pgfpathlineto{\pgfqpoint{7.101255in}{6.130754in}}%
\pgfpathlineto{\pgfqpoint{7.176273in}{6.128779in}}%
\pgfpathlineto{\pgfqpoint{7.251291in}{6.127016in}}%
\pgfpathlineto{\pgfqpoint{7.326310in}{6.125453in}}%
\pgfpathlineto{\pgfqpoint{7.401328in}{6.124083in}}%
\pgfpathlineto{\pgfqpoint{7.476346in}{6.122898in}}%
\pgfpathlineto{\pgfqpoint{7.551364in}{6.121893in}}%
\pgfpathlineto{\pgfqpoint{7.626382in}{6.121064in}}%
\pgfusepath{stroke}%
\end{pgfscope}%
\begin{pgfscope}%
\pgfpathrectangle{\pgfqpoint{5.814694in}{4.908628in}}{\pgfqpoint{1.897959in}{1.372727in}} %
\pgfusepath{clip}%
\pgfsetbuttcap%
\pgfsetroundjoin%
\definecolor{currentfill}{rgb}{0.000000,0.000000,0.000000}%
\pgfsetfillcolor{currentfill}%
\pgfsetlinewidth{1.003750pt}%
\definecolor{currentstroke}{rgb}{0.000000,0.000000,0.000000}%
\pgfsetstrokecolor{currentstroke}%
\pgfsetdash{}{0pt}%
\pgfsys@defobject{currentmarker}{\pgfqpoint{-0.041667in}{-0.041667in}}{\pgfqpoint{0.041667in}{0.041667in}}{%
\pgfpathmoveto{\pgfqpoint{-0.041667in}{0.000000in}}%
\pgfpathlineto{\pgfqpoint{0.041667in}{0.000000in}}%
\pgfpathmoveto{\pgfqpoint{0.000000in}{-0.041667in}}%
\pgfpathlineto{\pgfqpoint{0.000000in}{0.041667in}}%
\pgfusepath{stroke,fill}%
}%
\begin{pgfscope}%
\pgfsys@transformshift{5.900965in}{6.218958in}%
\pgfsys@useobject{currentmarker}{}%
\end{pgfscope}%
\begin{pgfscope}%
\pgfsys@transformshift{6.276055in}{6.172768in}%
\pgfsys@useobject{currentmarker}{}%
\end{pgfscope}%
\begin{pgfscope}%
\pgfsys@transformshift{6.651146in}{6.147911in}%
\pgfsys@useobject{currentmarker}{}%
\end{pgfscope}%
\begin{pgfscope}%
\pgfsys@transformshift{7.026237in}{6.132950in}%
\pgfsys@useobject{currentmarker}{}%
\end{pgfscope}%
\begin{pgfscope}%
\pgfsys@transformshift{7.401328in}{6.124083in}%
\pgfsys@useobject{currentmarker}{}%
\end{pgfscope}%
\end{pgfscope}%
\begin{pgfscope}%
\pgfsetrectcap%
\pgfsetmiterjoin%
\pgfsetlinewidth{0.803000pt}%
\definecolor{currentstroke}{rgb}{0.000000,0.000000,0.000000}%
\pgfsetstrokecolor{currentstroke}%
\pgfsetdash{}{0pt}%
\pgfpathmoveto{\pgfqpoint{5.814694in}{4.908628in}}%
\pgfpathlineto{\pgfqpoint{5.814694in}{6.281355in}}%
\pgfusepath{stroke}%
\end{pgfscope}%
\begin{pgfscope}%
\pgfsetrectcap%
\pgfsetmiterjoin%
\pgfsetlinewidth{0.803000pt}%
\definecolor{currentstroke}{rgb}{0.000000,0.000000,0.000000}%
\pgfsetstrokecolor{currentstroke}%
\pgfsetdash{}{0pt}%
\pgfpathmoveto{\pgfqpoint{7.712653in}{4.908628in}}%
\pgfpathlineto{\pgfqpoint{7.712653in}{6.281355in}}%
\pgfusepath{stroke}%
\end{pgfscope}%
\begin{pgfscope}%
\pgfsetrectcap%
\pgfsetmiterjoin%
\pgfsetlinewidth{0.803000pt}%
\definecolor{currentstroke}{rgb}{0.000000,0.000000,0.000000}%
\pgfsetstrokecolor{currentstroke}%
\pgfsetdash{}{0pt}%
\pgfpathmoveto{\pgfqpoint{5.814694in}{4.908628in}}%
\pgfpathlineto{\pgfqpoint{7.712653in}{4.908628in}}%
\pgfusepath{stroke}%
\end{pgfscope}%
\begin{pgfscope}%
\pgfsetrectcap%
\pgfsetmiterjoin%
\pgfsetlinewidth{0.803000pt}%
\definecolor{currentstroke}{rgb}{0.000000,0.000000,0.000000}%
\pgfsetstrokecolor{currentstroke}%
\pgfsetdash{}{0pt}%
\pgfpathmoveto{\pgfqpoint{5.814694in}{6.281355in}}%
\pgfpathlineto{\pgfqpoint{7.712653in}{6.281355in}}%
\pgfusepath{stroke}%
\end{pgfscope}%
\begin{pgfscope}%
\pgfsetbuttcap%
\pgfsetmiterjoin%
\definecolor{currentfill}{rgb}{1.000000,1.000000,1.000000}%
\pgfsetfillcolor{currentfill}%
\pgfsetlinewidth{0.000000pt}%
\definecolor{currentstroke}{rgb}{0.000000,0.000000,0.000000}%
\pgfsetstrokecolor{currentstroke}%
\pgfsetstrokeopacity{0.000000}%
\pgfsetdash{}{0pt}%
\pgfpathmoveto{\pgfqpoint{8.282041in}{4.908628in}}%
\pgfpathlineto{\pgfqpoint{10.180000in}{4.908628in}}%
\pgfpathlineto{\pgfqpoint{10.180000in}{6.281355in}}%
\pgfpathlineto{\pgfqpoint{8.282041in}{6.281355in}}%
\pgfpathclose%
\pgfusepath{fill}%
\end{pgfscope}%
\begin{pgfscope}%
\pgfsetbuttcap%
\pgfsetroundjoin%
\definecolor{currentfill}{rgb}{0.000000,0.000000,0.000000}%
\pgfsetfillcolor{currentfill}%
\pgfsetlinewidth{0.803000pt}%
\definecolor{currentstroke}{rgb}{0.000000,0.000000,0.000000}%
\pgfsetstrokecolor{currentstroke}%
\pgfsetdash{}{0pt}%
\pgfsys@defobject{currentmarker}{\pgfqpoint{0.000000in}{-0.048611in}}{\pgfqpoint{0.000000in}{0.000000in}}{%
\pgfpathmoveto{\pgfqpoint{0.000000in}{0.000000in}}%
\pgfpathlineto{\pgfqpoint{0.000000in}{-0.048611in}}%
\pgfusepath{stroke,fill}%
}%
\begin{pgfscope}%
\pgfsys@transformshift{8.759406in}{4.908628in}%
\pgfsys@useobject{currentmarker}{}%
\end{pgfscope}%
\end{pgfscope}%
\begin{pgfscope}%
\pgftext[x=8.759406in,y=4.811405in,,top]{\rmfamily\fontsize{10.000000}{12.000000}\selectfont \(\displaystyle 0.2\)}%
\end{pgfscope}%
\begin{pgfscope}%
\pgfsetbuttcap%
\pgfsetroundjoin%
\definecolor{currentfill}{rgb}{0.000000,0.000000,0.000000}%
\pgfsetfillcolor{currentfill}%
\pgfsetlinewidth{0.803000pt}%
\definecolor{currentstroke}{rgb}{0.000000,0.000000,0.000000}%
\pgfsetstrokecolor{currentstroke}%
\pgfsetdash{}{0pt}%
\pgfsys@defobject{currentmarker}{\pgfqpoint{0.000000in}{-0.048611in}}{\pgfqpoint{0.000000in}{0.000000in}}{%
\pgfpathmoveto{\pgfqpoint{0.000000in}{0.000000in}}%
\pgfpathlineto{\pgfqpoint{0.000000in}{-0.048611in}}%
\pgfusepath{stroke,fill}%
}%
\begin{pgfscope}%
\pgfsys@transformshift{9.311540in}{4.908628in}%
\pgfsys@useobject{currentmarker}{}%
\end{pgfscope}%
\end{pgfscope}%
\begin{pgfscope}%
\pgftext[x=9.311540in,y=4.811405in,,top]{\rmfamily\fontsize{10.000000}{12.000000}\selectfont \(\displaystyle 0.4\)}%
\end{pgfscope}%
\begin{pgfscope}%
\pgfsetbuttcap%
\pgfsetroundjoin%
\definecolor{currentfill}{rgb}{0.000000,0.000000,0.000000}%
\pgfsetfillcolor{currentfill}%
\pgfsetlinewidth{0.803000pt}%
\definecolor{currentstroke}{rgb}{0.000000,0.000000,0.000000}%
\pgfsetstrokecolor{currentstroke}%
\pgfsetdash{}{0pt}%
\pgfsys@defobject{currentmarker}{\pgfqpoint{0.000000in}{-0.048611in}}{\pgfqpoint{0.000000in}{0.000000in}}{%
\pgfpathmoveto{\pgfqpoint{0.000000in}{0.000000in}}%
\pgfpathlineto{\pgfqpoint{0.000000in}{-0.048611in}}%
\pgfusepath{stroke,fill}%
}%
\begin{pgfscope}%
\pgfsys@transformshift{9.863673in}{4.908628in}%
\pgfsys@useobject{currentmarker}{}%
\end{pgfscope}%
\end{pgfscope}%
\begin{pgfscope}%
\pgftext[x=9.863673in,y=4.811405in,,top]{\rmfamily\fontsize{10.000000}{12.000000}\selectfont \(\displaystyle 0.6\)}%
\end{pgfscope}%
\begin{pgfscope}%
\pgfsetbuttcap%
\pgfsetroundjoin%
\definecolor{currentfill}{rgb}{0.000000,0.000000,0.000000}%
\pgfsetfillcolor{currentfill}%
\pgfsetlinewidth{0.803000pt}%
\definecolor{currentstroke}{rgb}{0.000000,0.000000,0.000000}%
\pgfsetstrokecolor{currentstroke}%
\pgfsetdash{}{0pt}%
\pgfsys@defobject{currentmarker}{\pgfqpoint{-0.048611in}{0.000000in}}{\pgfqpoint{0.000000in}{0.000000in}}{%
\pgfpathmoveto{\pgfqpoint{0.000000in}{0.000000in}}%
\pgfpathlineto{\pgfqpoint{-0.048611in}{0.000000in}}%
\pgfusepath{stroke,fill}%
}%
\begin{pgfscope}%
\pgfsys@transformshift{8.282041in}{4.975720in}%
\pgfsys@useobject{currentmarker}{}%
\end{pgfscope}%
\end{pgfscope}%
\begin{pgfscope}%
\pgftext[x=7.896816in,y=4.922959in,left,base]{\rmfamily\fontsize{10.000000}{12.000000}\selectfont \(\displaystyle 10^{-8}\)}%
\end{pgfscope}%
\begin{pgfscope}%
\pgfsetbuttcap%
\pgfsetroundjoin%
\definecolor{currentfill}{rgb}{0.000000,0.000000,0.000000}%
\pgfsetfillcolor{currentfill}%
\pgfsetlinewidth{0.803000pt}%
\definecolor{currentstroke}{rgb}{0.000000,0.000000,0.000000}%
\pgfsetstrokecolor{currentstroke}%
\pgfsetdash{}{0pt}%
\pgfsys@defobject{currentmarker}{\pgfqpoint{-0.048611in}{0.000000in}}{\pgfqpoint{0.000000in}{0.000000in}}{%
\pgfpathmoveto{\pgfqpoint{0.000000in}{0.000000in}}%
\pgfpathlineto{\pgfqpoint{-0.048611in}{0.000000in}}%
\pgfusepath{stroke,fill}%
}%
\begin{pgfscope}%
\pgfsys@transformshift{8.282041in}{5.435532in}%
\pgfsys@useobject{currentmarker}{}%
\end{pgfscope}%
\end{pgfscope}%
\begin{pgfscope}%
\pgftext[x=7.896816in,y=5.382770in,left,base]{\rmfamily\fontsize{10.000000}{12.000000}\selectfont \(\displaystyle 10^{-7}\)}%
\end{pgfscope}%
\begin{pgfscope}%
\pgfsetbuttcap%
\pgfsetroundjoin%
\definecolor{currentfill}{rgb}{0.000000,0.000000,0.000000}%
\pgfsetfillcolor{currentfill}%
\pgfsetlinewidth{0.803000pt}%
\definecolor{currentstroke}{rgb}{0.000000,0.000000,0.000000}%
\pgfsetstrokecolor{currentstroke}%
\pgfsetdash{}{0pt}%
\pgfsys@defobject{currentmarker}{\pgfqpoint{-0.048611in}{0.000000in}}{\pgfqpoint{0.000000in}{0.000000in}}{%
\pgfpathmoveto{\pgfqpoint{0.000000in}{0.000000in}}%
\pgfpathlineto{\pgfqpoint{-0.048611in}{0.000000in}}%
\pgfusepath{stroke,fill}%
}%
\begin{pgfscope}%
\pgfsys@transformshift{8.282041in}{5.895344in}%
\pgfsys@useobject{currentmarker}{}%
\end{pgfscope}%
\end{pgfscope}%
\begin{pgfscope}%
\pgftext[x=7.896816in,y=5.842582in,left,base]{\rmfamily\fontsize{10.000000}{12.000000}\selectfont \(\displaystyle 10^{-6}\)}%
\end{pgfscope}%
\begin{pgfscope}%
\pgfsetbuttcap%
\pgfsetroundjoin%
\definecolor{currentfill}{rgb}{0.000000,0.000000,0.000000}%
\pgfsetfillcolor{currentfill}%
\pgfsetlinewidth{0.602250pt}%
\definecolor{currentstroke}{rgb}{0.000000,0.000000,0.000000}%
\pgfsetstrokecolor{currentstroke}%
\pgfsetdash{}{0pt}%
\pgfsys@defobject{currentmarker}{\pgfqpoint{-0.027778in}{0.000000in}}{\pgfqpoint{0.000000in}{0.000000in}}{%
\pgfpathmoveto{\pgfqpoint{0.000000in}{0.000000in}}%
\pgfpathlineto{\pgfqpoint{-0.027778in}{0.000000in}}%
\pgfusepath{stroke,fill}%
}%
\begin{pgfscope}%
\pgfsys@transformshift{8.282041in}{4.904495in}%
\pgfsys@useobject{currentmarker}{}%
\end{pgfscope}%
\end{pgfscope}%
\begin{pgfscope}%
\pgfsetbuttcap%
\pgfsetroundjoin%
\definecolor{currentfill}{rgb}{0.000000,0.000000,0.000000}%
\pgfsetfillcolor{currentfill}%
\pgfsetlinewidth{0.602250pt}%
\definecolor{currentstroke}{rgb}{0.000000,0.000000,0.000000}%
\pgfsetstrokecolor{currentstroke}%
\pgfsetdash{}{0pt}%
\pgfsys@defobject{currentmarker}{\pgfqpoint{-0.027778in}{0.000000in}}{\pgfqpoint{0.000000in}{0.000000in}}{%
\pgfpathmoveto{\pgfqpoint{0.000000in}{0.000000in}}%
\pgfpathlineto{\pgfqpoint{-0.027778in}{0.000000in}}%
\pgfusepath{stroke,fill}%
}%
\begin{pgfscope}%
\pgfsys@transformshift{8.282041in}{4.931160in}%
\pgfsys@useobject{currentmarker}{}%
\end{pgfscope}%
\end{pgfscope}%
\begin{pgfscope}%
\pgfsetbuttcap%
\pgfsetroundjoin%
\definecolor{currentfill}{rgb}{0.000000,0.000000,0.000000}%
\pgfsetfillcolor{currentfill}%
\pgfsetlinewidth{0.602250pt}%
\definecolor{currentstroke}{rgb}{0.000000,0.000000,0.000000}%
\pgfsetstrokecolor{currentstroke}%
\pgfsetdash{}{0pt}%
\pgfsys@defobject{currentmarker}{\pgfqpoint{-0.027778in}{0.000000in}}{\pgfqpoint{0.000000in}{0.000000in}}{%
\pgfpathmoveto{\pgfqpoint{0.000000in}{0.000000in}}%
\pgfpathlineto{\pgfqpoint{-0.027778in}{0.000000in}}%
\pgfusepath{stroke,fill}%
}%
\begin{pgfscope}%
\pgfsys@transformshift{8.282041in}{4.954681in}%
\pgfsys@useobject{currentmarker}{}%
\end{pgfscope}%
\end{pgfscope}%
\begin{pgfscope}%
\pgfsetbuttcap%
\pgfsetroundjoin%
\definecolor{currentfill}{rgb}{0.000000,0.000000,0.000000}%
\pgfsetfillcolor{currentfill}%
\pgfsetlinewidth{0.602250pt}%
\definecolor{currentstroke}{rgb}{0.000000,0.000000,0.000000}%
\pgfsetstrokecolor{currentstroke}%
\pgfsetdash{}{0pt}%
\pgfsys@defobject{currentmarker}{\pgfqpoint{-0.027778in}{0.000000in}}{\pgfqpoint{0.000000in}{0.000000in}}{%
\pgfpathmoveto{\pgfqpoint{0.000000in}{0.000000in}}%
\pgfpathlineto{\pgfqpoint{-0.027778in}{0.000000in}}%
\pgfusepath{stroke,fill}%
}%
\begin{pgfscope}%
\pgfsys@transformshift{8.282041in}{5.114137in}%
\pgfsys@useobject{currentmarker}{}%
\end{pgfscope}%
\end{pgfscope}%
\begin{pgfscope}%
\pgfsetbuttcap%
\pgfsetroundjoin%
\definecolor{currentfill}{rgb}{0.000000,0.000000,0.000000}%
\pgfsetfillcolor{currentfill}%
\pgfsetlinewidth{0.602250pt}%
\definecolor{currentstroke}{rgb}{0.000000,0.000000,0.000000}%
\pgfsetstrokecolor{currentstroke}%
\pgfsetdash{}{0pt}%
\pgfsys@defobject{currentmarker}{\pgfqpoint{-0.027778in}{0.000000in}}{\pgfqpoint{0.000000in}{0.000000in}}{%
\pgfpathmoveto{\pgfqpoint{0.000000in}{0.000000in}}%
\pgfpathlineto{\pgfqpoint{-0.027778in}{0.000000in}}%
\pgfusepath{stroke,fill}%
}%
\begin{pgfscope}%
\pgfsys@transformshift{8.282041in}{5.195106in}%
\pgfsys@useobject{currentmarker}{}%
\end{pgfscope}%
\end{pgfscope}%
\begin{pgfscope}%
\pgfsetbuttcap%
\pgfsetroundjoin%
\definecolor{currentfill}{rgb}{0.000000,0.000000,0.000000}%
\pgfsetfillcolor{currentfill}%
\pgfsetlinewidth{0.602250pt}%
\definecolor{currentstroke}{rgb}{0.000000,0.000000,0.000000}%
\pgfsetstrokecolor{currentstroke}%
\pgfsetdash{}{0pt}%
\pgfsys@defobject{currentmarker}{\pgfqpoint{-0.027778in}{0.000000in}}{\pgfqpoint{0.000000in}{0.000000in}}{%
\pgfpathmoveto{\pgfqpoint{0.000000in}{0.000000in}}%
\pgfpathlineto{\pgfqpoint{-0.027778in}{0.000000in}}%
\pgfusepath{stroke,fill}%
}%
\begin{pgfscope}%
\pgfsys@transformshift{8.282041in}{5.252554in}%
\pgfsys@useobject{currentmarker}{}%
\end{pgfscope}%
\end{pgfscope}%
\begin{pgfscope}%
\pgfsetbuttcap%
\pgfsetroundjoin%
\definecolor{currentfill}{rgb}{0.000000,0.000000,0.000000}%
\pgfsetfillcolor{currentfill}%
\pgfsetlinewidth{0.602250pt}%
\definecolor{currentstroke}{rgb}{0.000000,0.000000,0.000000}%
\pgfsetstrokecolor{currentstroke}%
\pgfsetdash{}{0pt}%
\pgfsys@defobject{currentmarker}{\pgfqpoint{-0.027778in}{0.000000in}}{\pgfqpoint{0.000000in}{0.000000in}}{%
\pgfpathmoveto{\pgfqpoint{0.000000in}{0.000000in}}%
\pgfpathlineto{\pgfqpoint{-0.027778in}{0.000000in}}%
\pgfusepath{stroke,fill}%
}%
\begin{pgfscope}%
\pgfsys@transformshift{8.282041in}{5.297115in}%
\pgfsys@useobject{currentmarker}{}%
\end{pgfscope}%
\end{pgfscope}%
\begin{pgfscope}%
\pgfsetbuttcap%
\pgfsetroundjoin%
\definecolor{currentfill}{rgb}{0.000000,0.000000,0.000000}%
\pgfsetfillcolor{currentfill}%
\pgfsetlinewidth{0.602250pt}%
\definecolor{currentstroke}{rgb}{0.000000,0.000000,0.000000}%
\pgfsetstrokecolor{currentstroke}%
\pgfsetdash{}{0pt}%
\pgfsys@defobject{currentmarker}{\pgfqpoint{-0.027778in}{0.000000in}}{\pgfqpoint{0.000000in}{0.000000in}}{%
\pgfpathmoveto{\pgfqpoint{0.000000in}{0.000000in}}%
\pgfpathlineto{\pgfqpoint{-0.027778in}{0.000000in}}%
\pgfusepath{stroke,fill}%
}%
\begin{pgfscope}%
\pgfsys@transformshift{8.282041in}{5.333523in}%
\pgfsys@useobject{currentmarker}{}%
\end{pgfscope}%
\end{pgfscope}%
\begin{pgfscope}%
\pgfsetbuttcap%
\pgfsetroundjoin%
\definecolor{currentfill}{rgb}{0.000000,0.000000,0.000000}%
\pgfsetfillcolor{currentfill}%
\pgfsetlinewidth{0.602250pt}%
\definecolor{currentstroke}{rgb}{0.000000,0.000000,0.000000}%
\pgfsetstrokecolor{currentstroke}%
\pgfsetdash{}{0pt}%
\pgfsys@defobject{currentmarker}{\pgfqpoint{-0.027778in}{0.000000in}}{\pgfqpoint{0.000000in}{0.000000in}}{%
\pgfpathmoveto{\pgfqpoint{0.000000in}{0.000000in}}%
\pgfpathlineto{\pgfqpoint{-0.027778in}{0.000000in}}%
\pgfusepath{stroke,fill}%
}%
\begin{pgfscope}%
\pgfsys@transformshift{8.282041in}{5.364306in}%
\pgfsys@useobject{currentmarker}{}%
\end{pgfscope}%
\end{pgfscope}%
\begin{pgfscope}%
\pgfsetbuttcap%
\pgfsetroundjoin%
\definecolor{currentfill}{rgb}{0.000000,0.000000,0.000000}%
\pgfsetfillcolor{currentfill}%
\pgfsetlinewidth{0.602250pt}%
\definecolor{currentstroke}{rgb}{0.000000,0.000000,0.000000}%
\pgfsetstrokecolor{currentstroke}%
\pgfsetdash{}{0pt}%
\pgfsys@defobject{currentmarker}{\pgfqpoint{-0.027778in}{0.000000in}}{\pgfqpoint{0.000000in}{0.000000in}}{%
\pgfpathmoveto{\pgfqpoint{0.000000in}{0.000000in}}%
\pgfpathlineto{\pgfqpoint{-0.027778in}{0.000000in}}%
\pgfusepath{stroke,fill}%
}%
\begin{pgfscope}%
\pgfsys@transformshift{8.282041in}{5.390972in}%
\pgfsys@useobject{currentmarker}{}%
\end{pgfscope}%
\end{pgfscope}%
\begin{pgfscope}%
\pgfsetbuttcap%
\pgfsetroundjoin%
\definecolor{currentfill}{rgb}{0.000000,0.000000,0.000000}%
\pgfsetfillcolor{currentfill}%
\pgfsetlinewidth{0.602250pt}%
\definecolor{currentstroke}{rgb}{0.000000,0.000000,0.000000}%
\pgfsetstrokecolor{currentstroke}%
\pgfsetdash{}{0pt}%
\pgfsys@defobject{currentmarker}{\pgfqpoint{-0.027778in}{0.000000in}}{\pgfqpoint{0.000000in}{0.000000in}}{%
\pgfpathmoveto{\pgfqpoint{0.000000in}{0.000000in}}%
\pgfpathlineto{\pgfqpoint{-0.027778in}{0.000000in}}%
\pgfusepath{stroke,fill}%
}%
\begin{pgfscope}%
\pgfsys@transformshift{8.282041in}{5.414492in}%
\pgfsys@useobject{currentmarker}{}%
\end{pgfscope}%
\end{pgfscope}%
\begin{pgfscope}%
\pgfsetbuttcap%
\pgfsetroundjoin%
\definecolor{currentfill}{rgb}{0.000000,0.000000,0.000000}%
\pgfsetfillcolor{currentfill}%
\pgfsetlinewidth{0.602250pt}%
\definecolor{currentstroke}{rgb}{0.000000,0.000000,0.000000}%
\pgfsetstrokecolor{currentstroke}%
\pgfsetdash{}{0pt}%
\pgfsys@defobject{currentmarker}{\pgfqpoint{-0.027778in}{0.000000in}}{\pgfqpoint{0.000000in}{0.000000in}}{%
\pgfpathmoveto{\pgfqpoint{0.000000in}{0.000000in}}%
\pgfpathlineto{\pgfqpoint{-0.027778in}{0.000000in}}%
\pgfusepath{stroke,fill}%
}%
\begin{pgfscope}%
\pgfsys@transformshift{8.282041in}{5.573949in}%
\pgfsys@useobject{currentmarker}{}%
\end{pgfscope}%
\end{pgfscope}%
\begin{pgfscope}%
\pgfsetbuttcap%
\pgfsetroundjoin%
\definecolor{currentfill}{rgb}{0.000000,0.000000,0.000000}%
\pgfsetfillcolor{currentfill}%
\pgfsetlinewidth{0.602250pt}%
\definecolor{currentstroke}{rgb}{0.000000,0.000000,0.000000}%
\pgfsetstrokecolor{currentstroke}%
\pgfsetdash{}{0pt}%
\pgfsys@defobject{currentmarker}{\pgfqpoint{-0.027778in}{0.000000in}}{\pgfqpoint{0.000000in}{0.000000in}}{%
\pgfpathmoveto{\pgfqpoint{0.000000in}{0.000000in}}%
\pgfpathlineto{\pgfqpoint{-0.027778in}{0.000000in}}%
\pgfusepath{stroke,fill}%
}%
\begin{pgfscope}%
\pgfsys@transformshift{8.282041in}{5.654918in}%
\pgfsys@useobject{currentmarker}{}%
\end{pgfscope}%
\end{pgfscope}%
\begin{pgfscope}%
\pgfsetbuttcap%
\pgfsetroundjoin%
\definecolor{currentfill}{rgb}{0.000000,0.000000,0.000000}%
\pgfsetfillcolor{currentfill}%
\pgfsetlinewidth{0.602250pt}%
\definecolor{currentstroke}{rgb}{0.000000,0.000000,0.000000}%
\pgfsetstrokecolor{currentstroke}%
\pgfsetdash{}{0pt}%
\pgfsys@defobject{currentmarker}{\pgfqpoint{-0.027778in}{0.000000in}}{\pgfqpoint{0.000000in}{0.000000in}}{%
\pgfpathmoveto{\pgfqpoint{0.000000in}{0.000000in}}%
\pgfpathlineto{\pgfqpoint{-0.027778in}{0.000000in}}%
\pgfusepath{stroke,fill}%
}%
\begin{pgfscope}%
\pgfsys@transformshift{8.282041in}{5.712366in}%
\pgfsys@useobject{currentmarker}{}%
\end{pgfscope}%
\end{pgfscope}%
\begin{pgfscope}%
\pgfsetbuttcap%
\pgfsetroundjoin%
\definecolor{currentfill}{rgb}{0.000000,0.000000,0.000000}%
\pgfsetfillcolor{currentfill}%
\pgfsetlinewidth{0.602250pt}%
\definecolor{currentstroke}{rgb}{0.000000,0.000000,0.000000}%
\pgfsetstrokecolor{currentstroke}%
\pgfsetdash{}{0pt}%
\pgfsys@defobject{currentmarker}{\pgfqpoint{-0.027778in}{0.000000in}}{\pgfqpoint{0.000000in}{0.000000in}}{%
\pgfpathmoveto{\pgfqpoint{0.000000in}{0.000000in}}%
\pgfpathlineto{\pgfqpoint{-0.027778in}{0.000000in}}%
\pgfusepath{stroke,fill}%
}%
\begin{pgfscope}%
\pgfsys@transformshift{8.282041in}{5.756926in}%
\pgfsys@useobject{currentmarker}{}%
\end{pgfscope}%
\end{pgfscope}%
\begin{pgfscope}%
\pgfsetbuttcap%
\pgfsetroundjoin%
\definecolor{currentfill}{rgb}{0.000000,0.000000,0.000000}%
\pgfsetfillcolor{currentfill}%
\pgfsetlinewidth{0.602250pt}%
\definecolor{currentstroke}{rgb}{0.000000,0.000000,0.000000}%
\pgfsetstrokecolor{currentstroke}%
\pgfsetdash{}{0pt}%
\pgfsys@defobject{currentmarker}{\pgfqpoint{-0.027778in}{0.000000in}}{\pgfqpoint{0.000000in}{0.000000in}}{%
\pgfpathmoveto{\pgfqpoint{0.000000in}{0.000000in}}%
\pgfpathlineto{\pgfqpoint{-0.027778in}{0.000000in}}%
\pgfusepath{stroke,fill}%
}%
\begin{pgfscope}%
\pgfsys@transformshift{8.282041in}{5.793335in}%
\pgfsys@useobject{currentmarker}{}%
\end{pgfscope}%
\end{pgfscope}%
\begin{pgfscope}%
\pgfsetbuttcap%
\pgfsetroundjoin%
\definecolor{currentfill}{rgb}{0.000000,0.000000,0.000000}%
\pgfsetfillcolor{currentfill}%
\pgfsetlinewidth{0.602250pt}%
\definecolor{currentstroke}{rgb}{0.000000,0.000000,0.000000}%
\pgfsetstrokecolor{currentstroke}%
\pgfsetdash{}{0pt}%
\pgfsys@defobject{currentmarker}{\pgfqpoint{-0.027778in}{0.000000in}}{\pgfqpoint{0.000000in}{0.000000in}}{%
\pgfpathmoveto{\pgfqpoint{0.000000in}{0.000000in}}%
\pgfpathlineto{\pgfqpoint{-0.027778in}{0.000000in}}%
\pgfusepath{stroke,fill}%
}%
\begin{pgfscope}%
\pgfsys@transformshift{8.282041in}{5.824118in}%
\pgfsys@useobject{currentmarker}{}%
\end{pgfscope}%
\end{pgfscope}%
\begin{pgfscope}%
\pgfsetbuttcap%
\pgfsetroundjoin%
\definecolor{currentfill}{rgb}{0.000000,0.000000,0.000000}%
\pgfsetfillcolor{currentfill}%
\pgfsetlinewidth{0.602250pt}%
\definecolor{currentstroke}{rgb}{0.000000,0.000000,0.000000}%
\pgfsetstrokecolor{currentstroke}%
\pgfsetdash{}{0pt}%
\pgfsys@defobject{currentmarker}{\pgfqpoint{-0.027778in}{0.000000in}}{\pgfqpoint{0.000000in}{0.000000in}}{%
\pgfpathmoveto{\pgfqpoint{0.000000in}{0.000000in}}%
\pgfpathlineto{\pgfqpoint{-0.027778in}{0.000000in}}%
\pgfusepath{stroke,fill}%
}%
\begin{pgfscope}%
\pgfsys@transformshift{8.282041in}{5.850783in}%
\pgfsys@useobject{currentmarker}{}%
\end{pgfscope}%
\end{pgfscope}%
\begin{pgfscope}%
\pgfsetbuttcap%
\pgfsetroundjoin%
\definecolor{currentfill}{rgb}{0.000000,0.000000,0.000000}%
\pgfsetfillcolor{currentfill}%
\pgfsetlinewidth{0.602250pt}%
\definecolor{currentstroke}{rgb}{0.000000,0.000000,0.000000}%
\pgfsetstrokecolor{currentstroke}%
\pgfsetdash{}{0pt}%
\pgfsys@defobject{currentmarker}{\pgfqpoint{-0.027778in}{0.000000in}}{\pgfqpoint{0.000000in}{0.000000in}}{%
\pgfpathmoveto{\pgfqpoint{0.000000in}{0.000000in}}%
\pgfpathlineto{\pgfqpoint{-0.027778in}{0.000000in}}%
\pgfusepath{stroke,fill}%
}%
\begin{pgfscope}%
\pgfsys@transformshift{8.282041in}{5.874304in}%
\pgfsys@useobject{currentmarker}{}%
\end{pgfscope}%
\end{pgfscope}%
\begin{pgfscope}%
\pgfsetbuttcap%
\pgfsetroundjoin%
\definecolor{currentfill}{rgb}{0.000000,0.000000,0.000000}%
\pgfsetfillcolor{currentfill}%
\pgfsetlinewidth{0.602250pt}%
\definecolor{currentstroke}{rgb}{0.000000,0.000000,0.000000}%
\pgfsetstrokecolor{currentstroke}%
\pgfsetdash{}{0pt}%
\pgfsys@defobject{currentmarker}{\pgfqpoint{-0.027778in}{0.000000in}}{\pgfqpoint{0.000000in}{0.000000in}}{%
\pgfpathmoveto{\pgfqpoint{0.000000in}{0.000000in}}%
\pgfpathlineto{\pgfqpoint{-0.027778in}{0.000000in}}%
\pgfusepath{stroke,fill}%
}%
\begin{pgfscope}%
\pgfsys@transformshift{8.282041in}{6.033761in}%
\pgfsys@useobject{currentmarker}{}%
\end{pgfscope}%
\end{pgfscope}%
\begin{pgfscope}%
\pgfsetbuttcap%
\pgfsetroundjoin%
\definecolor{currentfill}{rgb}{0.000000,0.000000,0.000000}%
\pgfsetfillcolor{currentfill}%
\pgfsetlinewidth{0.602250pt}%
\definecolor{currentstroke}{rgb}{0.000000,0.000000,0.000000}%
\pgfsetstrokecolor{currentstroke}%
\pgfsetdash{}{0pt}%
\pgfsys@defobject{currentmarker}{\pgfqpoint{-0.027778in}{0.000000in}}{\pgfqpoint{0.000000in}{0.000000in}}{%
\pgfpathmoveto{\pgfqpoint{0.000000in}{0.000000in}}%
\pgfpathlineto{\pgfqpoint{-0.027778in}{0.000000in}}%
\pgfusepath{stroke,fill}%
}%
\begin{pgfscope}%
\pgfsys@transformshift{8.282041in}{6.114729in}%
\pgfsys@useobject{currentmarker}{}%
\end{pgfscope}%
\end{pgfscope}%
\begin{pgfscope}%
\pgfsetbuttcap%
\pgfsetroundjoin%
\definecolor{currentfill}{rgb}{0.000000,0.000000,0.000000}%
\pgfsetfillcolor{currentfill}%
\pgfsetlinewidth{0.602250pt}%
\definecolor{currentstroke}{rgb}{0.000000,0.000000,0.000000}%
\pgfsetstrokecolor{currentstroke}%
\pgfsetdash{}{0pt}%
\pgfsys@defobject{currentmarker}{\pgfqpoint{-0.027778in}{0.000000in}}{\pgfqpoint{0.000000in}{0.000000in}}{%
\pgfpathmoveto{\pgfqpoint{0.000000in}{0.000000in}}%
\pgfpathlineto{\pgfqpoint{-0.027778in}{0.000000in}}%
\pgfusepath{stroke,fill}%
}%
\begin{pgfscope}%
\pgfsys@transformshift{8.282041in}{6.172178in}%
\pgfsys@useobject{currentmarker}{}%
\end{pgfscope}%
\end{pgfscope}%
\begin{pgfscope}%
\pgfsetbuttcap%
\pgfsetroundjoin%
\definecolor{currentfill}{rgb}{0.000000,0.000000,0.000000}%
\pgfsetfillcolor{currentfill}%
\pgfsetlinewidth{0.602250pt}%
\definecolor{currentstroke}{rgb}{0.000000,0.000000,0.000000}%
\pgfsetstrokecolor{currentstroke}%
\pgfsetdash{}{0pt}%
\pgfsys@defobject{currentmarker}{\pgfqpoint{-0.027778in}{0.000000in}}{\pgfqpoint{0.000000in}{0.000000in}}{%
\pgfpathmoveto{\pgfqpoint{0.000000in}{0.000000in}}%
\pgfpathlineto{\pgfqpoint{-0.027778in}{0.000000in}}%
\pgfusepath{stroke,fill}%
}%
\begin{pgfscope}%
\pgfsys@transformshift{8.282041in}{6.216738in}%
\pgfsys@useobject{currentmarker}{}%
\end{pgfscope}%
\end{pgfscope}%
\begin{pgfscope}%
\pgfsetbuttcap%
\pgfsetroundjoin%
\definecolor{currentfill}{rgb}{0.000000,0.000000,0.000000}%
\pgfsetfillcolor{currentfill}%
\pgfsetlinewidth{0.602250pt}%
\definecolor{currentstroke}{rgb}{0.000000,0.000000,0.000000}%
\pgfsetstrokecolor{currentstroke}%
\pgfsetdash{}{0pt}%
\pgfsys@defobject{currentmarker}{\pgfqpoint{-0.027778in}{0.000000in}}{\pgfqpoint{0.000000in}{0.000000in}}{%
\pgfpathmoveto{\pgfqpoint{0.000000in}{0.000000in}}%
\pgfpathlineto{\pgfqpoint{-0.027778in}{0.000000in}}%
\pgfusepath{stroke,fill}%
}%
\begin{pgfscope}%
\pgfsys@transformshift{8.282041in}{6.253146in}%
\pgfsys@useobject{currentmarker}{}%
\end{pgfscope}%
\end{pgfscope}%
\begin{pgfscope}%
\pgfsetbuttcap%
\pgfsetroundjoin%
\definecolor{currentfill}{rgb}{0.000000,0.000000,0.000000}%
\pgfsetfillcolor{currentfill}%
\pgfsetlinewidth{0.602250pt}%
\definecolor{currentstroke}{rgb}{0.000000,0.000000,0.000000}%
\pgfsetstrokecolor{currentstroke}%
\pgfsetdash{}{0pt}%
\pgfsys@defobject{currentmarker}{\pgfqpoint{-0.027778in}{0.000000in}}{\pgfqpoint{0.000000in}{0.000000in}}{%
\pgfpathmoveto{\pgfqpoint{0.000000in}{0.000000in}}%
\pgfpathlineto{\pgfqpoint{-0.027778in}{0.000000in}}%
\pgfusepath{stroke,fill}%
}%
\begin{pgfscope}%
\pgfsys@transformshift{8.282041in}{6.283929in}%
\pgfsys@useobject{currentmarker}{}%
\end{pgfscope}%
\end{pgfscope}%
\begin{pgfscope}%
\pgfpathrectangle{\pgfqpoint{8.282041in}{4.908628in}}{\pgfqpoint{1.897959in}{1.372727in}} %
\pgfusepath{clip}%
\pgfsetbuttcap%
\pgfsetroundjoin%
\pgfsetlinewidth{1.505625pt}%
\definecolor{currentstroke}{rgb}{1.000000,0.000000,0.000000}%
\pgfsetstrokecolor{currentstroke}%
\pgfsetdash{{5.550000pt}{2.400000pt}}{0.000000pt}%
\pgfpathmoveto{\pgfqpoint{8.368312in}{6.127938in}}%
\pgfpathlineto{\pgfqpoint{8.391317in}{6.125102in}}%
\pgfpathlineto{\pgfqpoint{8.414323in}{6.122307in}}%
\pgfpathlineto{\pgfqpoint{8.437328in}{6.119551in}}%
\pgfpathlineto{\pgfqpoint{8.460334in}{6.116832in}}%
\pgfpathlineto{\pgfqpoint{8.483340in}{6.114152in}}%
\pgfpathlineto{\pgfqpoint{8.506345in}{6.111509in}}%
\pgfpathlineto{\pgfqpoint{8.529351in}{6.108903in}}%
\pgfpathlineto{\pgfqpoint{8.552356in}{6.106335in}}%
\pgfpathlineto{\pgfqpoint{8.575362in}{6.103804in}}%
\pgfpathlineto{\pgfqpoint{8.598367in}{6.101310in}}%
\pgfpathlineto{\pgfqpoint{8.621373in}{6.098855in}}%
\pgfpathlineto{\pgfqpoint{8.644378in}{6.096437in}}%
\pgfpathlineto{\pgfqpoint{8.667384in}{6.094058in}}%
\pgfpathlineto{\pgfqpoint{8.690390in}{6.091716in}}%
\pgfpathlineto{\pgfqpoint{8.713395in}{6.089413in}}%
\pgfpathlineto{\pgfqpoint{8.736401in}{6.087148in}}%
\pgfpathlineto{\pgfqpoint{8.759406in}{6.084922in}}%
\pgfpathlineto{\pgfqpoint{8.782412in}{6.082734in}}%
\pgfpathlineto{\pgfqpoint{8.805417in}{6.080585in}}%
\pgfpathlineto{\pgfqpoint{8.828423in}{6.078474in}}%
\pgfpathlineto{\pgfqpoint{8.851429in}{6.076402in}}%
\pgfpathlineto{\pgfqpoint{8.874434in}{6.074369in}}%
\pgfpathlineto{\pgfqpoint{8.897440in}{6.072374in}}%
\pgfpathlineto{\pgfqpoint{8.920445in}{6.070418in}}%
\pgfpathlineto{\pgfqpoint{8.943451in}{6.068500in}}%
\pgfpathlineto{\pgfqpoint{8.966456in}{6.066620in}}%
\pgfpathlineto{\pgfqpoint{8.989462in}{6.064779in}}%
\pgfpathlineto{\pgfqpoint{9.012468in}{6.062976in}}%
\pgfpathlineto{\pgfqpoint{9.035473in}{6.061211in}}%
\pgfpathlineto{\pgfqpoint{9.058479in}{6.059483in}}%
\pgfpathlineto{\pgfqpoint{9.081484in}{6.057794in}}%
\pgfpathlineto{\pgfqpoint{9.104490in}{6.056142in}}%
\pgfpathlineto{\pgfqpoint{9.127495in}{6.054527in}}%
\pgfpathlineto{\pgfqpoint{9.150501in}{6.052950in}}%
\pgfpathlineto{\pgfqpoint{9.173506in}{6.051410in}}%
\pgfpathlineto{\pgfqpoint{9.196512in}{6.049907in}}%
\pgfpathlineto{\pgfqpoint{9.219518in}{6.048441in}}%
\pgfpathlineto{\pgfqpoint{9.242523in}{6.047011in}}%
\pgfpathlineto{\pgfqpoint{9.265529in}{6.045618in}}%
\pgfpathlineto{\pgfqpoint{9.288534in}{6.044261in}}%
\pgfpathlineto{\pgfqpoint{9.311540in}{6.042940in}}%
\pgfpathlineto{\pgfqpoint{9.334545in}{6.041655in}}%
\pgfpathlineto{\pgfqpoint{9.357551in}{6.040405in}}%
\pgfpathlineto{\pgfqpoint{9.380557in}{6.039191in}}%
\pgfpathlineto{\pgfqpoint{9.403562in}{6.038012in}}%
\pgfpathlineto{\pgfqpoint{9.426568in}{6.036868in}}%
\pgfpathlineto{\pgfqpoint{9.449573in}{6.035759in}}%
\pgfpathlineto{\pgfqpoint{9.472579in}{6.034685in}}%
\pgfpathlineto{\pgfqpoint{9.495584in}{6.033645in}}%
\pgfpathlineto{\pgfqpoint{9.518590in}{6.032640in}}%
\pgfpathlineto{\pgfqpoint{9.541596in}{6.031669in}}%
\pgfpathlineto{\pgfqpoint{9.564601in}{6.030732in}}%
\pgfpathlineto{\pgfqpoint{9.587607in}{6.029828in}}%
\pgfpathlineto{\pgfqpoint{9.610612in}{6.028959in}}%
\pgfpathlineto{\pgfqpoint{9.633618in}{6.028122in}}%
\pgfpathlineto{\pgfqpoint{9.656623in}{6.027319in}}%
\pgfpathlineto{\pgfqpoint{9.679629in}{6.026550in}}%
\pgfpathlineto{\pgfqpoint{9.702635in}{6.025813in}}%
\pgfpathlineto{\pgfqpoint{9.725640in}{6.025109in}}%
\pgfpathlineto{\pgfqpoint{9.748646in}{6.024437in}}%
\pgfpathlineto{\pgfqpoint{9.771651in}{6.023799in}}%
\pgfpathlineto{\pgfqpoint{9.794657in}{6.023192in}}%
\pgfpathlineto{\pgfqpoint{9.817662in}{6.022619in}}%
\pgfpathlineto{\pgfqpoint{9.840668in}{6.022077in}}%
\pgfpathlineto{\pgfqpoint{9.863673in}{6.021567in}}%
\pgfpathlineto{\pgfqpoint{9.886679in}{6.021089in}}%
\pgfpathlineto{\pgfqpoint{9.909685in}{6.020643in}}%
\pgfpathlineto{\pgfqpoint{9.932690in}{6.020229in}}%
\pgfpathlineto{\pgfqpoint{9.955696in}{6.019846in}}%
\pgfpathlineto{\pgfqpoint{9.978701in}{6.019495in}}%
\pgfpathlineto{\pgfqpoint{10.001707in}{6.019175in}}%
\pgfpathlineto{\pgfqpoint{10.024712in}{6.018887in}}%
\pgfpathlineto{\pgfqpoint{10.047718in}{6.018630in}}%
\pgfpathlineto{\pgfqpoint{10.070724in}{6.018404in}}%
\pgfpathlineto{\pgfqpoint{10.093729in}{6.018209in}}%
\pgfusepath{stroke}%
\end{pgfscope}%
\begin{pgfscope}%
\pgfpathrectangle{\pgfqpoint{8.282041in}{4.908628in}}{\pgfqpoint{1.897959in}{1.372727in}} %
\pgfusepath{clip}%
\pgfsetbuttcap%
\pgfsetmiterjoin%
\definecolor{currentfill}{rgb}{1.000000,0.000000,0.000000}%
\pgfsetfillcolor{currentfill}%
\pgfsetlinewidth{1.003750pt}%
\definecolor{currentstroke}{rgb}{1.000000,0.000000,0.000000}%
\pgfsetstrokecolor{currentstroke}%
\pgfsetdash{}{0pt}%
\pgfsys@defobject{currentmarker}{\pgfqpoint{-0.041667in}{-0.041667in}}{\pgfqpoint{0.041667in}{0.041667in}}{%
\pgfpathmoveto{\pgfqpoint{-0.041667in}{-0.041667in}}%
\pgfpathlineto{\pgfqpoint{0.041667in}{-0.041667in}}%
\pgfpathlineto{\pgfqpoint{0.041667in}{0.041667in}}%
\pgfpathlineto{\pgfqpoint{-0.041667in}{0.041667in}}%
\pgfpathclose%
\pgfusepath{stroke,fill}%
}%
\begin{pgfscope}%
\pgfsys@transformshift{8.368312in}{6.127938in}%
\pgfsys@useobject{currentmarker}{}%
\end{pgfscope}%
\begin{pgfscope}%
\pgfsys@transformshift{8.713395in}{6.089413in}%
\pgfsys@useobject{currentmarker}{}%
\end{pgfscope}%
\begin{pgfscope}%
\pgfsys@transformshift{9.058479in}{6.059483in}%
\pgfsys@useobject{currentmarker}{}%
\end{pgfscope}%
\begin{pgfscope}%
\pgfsys@transformshift{9.403562in}{6.038012in}%
\pgfsys@useobject{currentmarker}{}%
\end{pgfscope}%
\begin{pgfscope}%
\pgfsys@transformshift{9.748646in}{6.024437in}%
\pgfsys@useobject{currentmarker}{}%
\end{pgfscope}%
\begin{pgfscope}%
\pgfsys@transformshift{10.093729in}{6.018209in}%
\pgfsys@useobject{currentmarker}{}%
\end{pgfscope}%
\end{pgfscope}%
\begin{pgfscope}%
\pgfpathrectangle{\pgfqpoint{8.282041in}{4.908628in}}{\pgfqpoint{1.897959in}{1.372727in}} %
\pgfusepath{clip}%
\pgfsetrectcap%
\pgfsetroundjoin%
\pgfsetlinewidth{1.505625pt}%
\definecolor{currentstroke}{rgb}{0.000000,0.000000,1.000000}%
\pgfsetstrokecolor{currentstroke}%
\pgfsetdash{}{0pt}%
\pgfpathmoveto{\pgfqpoint{8.368312in}{5.654281in}}%
\pgfpathlineto{\pgfqpoint{8.391317in}{5.648078in}}%
\pgfpathlineto{\pgfqpoint{8.414323in}{5.642141in}}%
\pgfpathlineto{\pgfqpoint{8.437328in}{5.636395in}}%
\pgfpathlineto{\pgfqpoint{8.460334in}{5.630792in}}%
\pgfpathlineto{\pgfqpoint{8.483340in}{5.625297in}}%
\pgfpathlineto{\pgfqpoint{8.506345in}{5.619884in}}%
\pgfpathlineto{\pgfqpoint{8.529351in}{5.614533in}}%
\pgfpathlineto{\pgfqpoint{8.552356in}{5.609229in}}%
\pgfpathlineto{\pgfqpoint{8.575362in}{5.603961in}}%
\pgfpathlineto{\pgfqpoint{8.598367in}{5.598717in}}%
\pgfpathlineto{\pgfqpoint{8.621373in}{5.593491in}}%
\pgfpathlineto{\pgfqpoint{8.644378in}{5.588274in}}%
\pgfpathlineto{\pgfqpoint{8.667384in}{5.583062in}}%
\pgfpathlineto{\pgfqpoint{8.690390in}{5.577848in}}%
\pgfpathlineto{\pgfqpoint{8.713395in}{5.572628in}}%
\pgfpathlineto{\pgfqpoint{8.736401in}{5.567398in}}%
\pgfpathlineto{\pgfqpoint{8.759406in}{5.562153in}}%
\pgfpathlineto{\pgfqpoint{8.782412in}{5.556889in}}%
\pgfpathlineto{\pgfqpoint{8.805417in}{5.551604in}}%
\pgfpathlineto{\pgfqpoint{8.828423in}{5.546294in}}%
\pgfpathlineto{\pgfqpoint{8.851429in}{5.540955in}}%
\pgfpathlineto{\pgfqpoint{8.874434in}{5.535584in}}%
\pgfpathlineto{\pgfqpoint{8.897440in}{5.530177in}}%
\pgfpathlineto{\pgfqpoint{8.920445in}{5.524732in}}%
\pgfpathlineto{\pgfqpoint{8.943451in}{5.519246in}}%
\pgfpathlineto{\pgfqpoint{8.966456in}{5.513714in}}%
\pgfpathlineto{\pgfqpoint{8.989462in}{5.508134in}}%
\pgfpathlineto{\pgfqpoint{9.012468in}{5.502501in}}%
\pgfpathlineto{\pgfqpoint{9.035473in}{5.496813in}}%
\pgfpathlineto{\pgfqpoint{9.058479in}{5.491066in}}%
\pgfpathlineto{\pgfqpoint{9.081484in}{5.485256in}}%
\pgfpathlineto{\pgfqpoint{9.104490in}{5.479378in}}%
\pgfpathlineto{\pgfqpoint{9.127495in}{5.473429in}}%
\pgfpathlineto{\pgfqpoint{9.150501in}{5.467405in}}%
\pgfpathlineto{\pgfqpoint{9.173506in}{5.461300in}}%
\pgfpathlineto{\pgfqpoint{9.196512in}{5.455111in}}%
\pgfpathlineto{\pgfqpoint{9.219518in}{5.448831in}}%
\pgfpathlineto{\pgfqpoint{9.242523in}{5.442456in}}%
\pgfpathlineto{\pgfqpoint{9.265529in}{5.435979in}}%
\pgfpathlineto{\pgfqpoint{9.288534in}{5.429395in}}%
\pgfpathlineto{\pgfqpoint{9.311540in}{5.422697in}}%
\pgfpathlineto{\pgfqpoint{9.334545in}{5.415878in}}%
\pgfpathlineto{\pgfqpoint{9.357551in}{5.408931in}}%
\pgfpathlineto{\pgfqpoint{9.380557in}{5.401848in}}%
\pgfpathlineto{\pgfqpoint{9.403562in}{5.394619in}}%
\pgfpathlineto{\pgfqpoint{9.426568in}{5.387237in}}%
\pgfpathlineto{\pgfqpoint{9.449573in}{5.379690in}}%
\pgfpathlineto{\pgfqpoint{9.472579in}{5.371967in}}%
\pgfpathlineto{\pgfqpoint{9.495584in}{5.364058in}}%
\pgfpathlineto{\pgfqpoint{9.518590in}{5.355948in}}%
\pgfpathlineto{\pgfqpoint{9.541596in}{5.347623in}}%
\pgfpathlineto{\pgfqpoint{9.564601in}{5.339068in}}%
\pgfpathlineto{\pgfqpoint{9.587607in}{5.330266in}}%
\pgfpathlineto{\pgfqpoint{9.610612in}{5.321197in}}%
\pgfpathlineto{\pgfqpoint{9.633618in}{5.311840in}}%
\pgfpathlineto{\pgfqpoint{9.656623in}{5.302171in}}%
\pgfpathlineto{\pgfqpoint{9.679629in}{5.292163in}}%
\pgfpathlineto{\pgfqpoint{9.702635in}{5.281787in}}%
\pgfpathlineto{\pgfqpoint{9.725640in}{5.271008in}}%
\pgfpathlineto{\pgfqpoint{9.748646in}{5.259789in}}%
\pgfpathlineto{\pgfqpoint{9.771651in}{5.248084in}}%
\pgfpathlineto{\pgfqpoint{9.794657in}{5.235842in}}%
\pgfpathlineto{\pgfqpoint{9.817662in}{5.223004in}}%
\pgfpathlineto{\pgfqpoint{9.840668in}{5.209501in}}%
\pgfpathlineto{\pgfqpoint{9.863673in}{5.195249in}}%
\pgfpathlineto{\pgfqpoint{9.886679in}{5.180152in}}%
\pgfpathlineto{\pgfqpoint{9.909685in}{5.164090in}}%
\pgfpathlineto{\pgfqpoint{9.932690in}{5.146918in}}%
\pgfpathlineto{\pgfqpoint{9.955696in}{5.128456in}}%
\pgfpathlineto{\pgfqpoint{9.978701in}{5.108476in}}%
\pgfpathlineto{\pgfqpoint{10.001707in}{5.086685in}}%
\pgfpathlineto{\pgfqpoint{10.024712in}{5.062693in}}%
\pgfpathlineto{\pgfqpoint{10.047718in}{5.035975in}}%
\pgfpathlineto{\pgfqpoint{10.070724in}{5.005784in}}%
\pgfpathlineto{\pgfqpoint{10.093729in}{4.971024in}}%
\pgfusepath{stroke}%
\end{pgfscope}%
\begin{pgfscope}%
\pgfpathrectangle{\pgfqpoint{8.282041in}{4.908628in}}{\pgfqpoint{1.897959in}{1.372727in}} %
\pgfusepath{clip}%
\pgfsetbuttcap%
\pgfsetroundjoin%
\definecolor{currentfill}{rgb}{0.000000,0.000000,1.000000}%
\pgfsetfillcolor{currentfill}%
\pgfsetlinewidth{1.003750pt}%
\definecolor{currentstroke}{rgb}{0.000000,0.000000,1.000000}%
\pgfsetstrokecolor{currentstroke}%
\pgfsetdash{}{0pt}%
\pgfsys@defobject{currentmarker}{\pgfqpoint{-0.041667in}{-0.041667in}}{\pgfqpoint{0.041667in}{0.041667in}}{%
\pgfpathmoveto{\pgfqpoint{0.000000in}{-0.041667in}}%
\pgfpathcurveto{\pgfqpoint{0.011050in}{-0.041667in}}{\pgfqpoint{0.021649in}{-0.037276in}}{\pgfqpoint{0.029463in}{-0.029463in}}%
\pgfpathcurveto{\pgfqpoint{0.037276in}{-0.021649in}}{\pgfqpoint{0.041667in}{-0.011050in}}{\pgfqpoint{0.041667in}{0.000000in}}%
\pgfpathcurveto{\pgfqpoint{0.041667in}{0.011050in}}{\pgfqpoint{0.037276in}{0.021649in}}{\pgfqpoint{0.029463in}{0.029463in}}%
\pgfpathcurveto{\pgfqpoint{0.021649in}{0.037276in}}{\pgfqpoint{0.011050in}{0.041667in}}{\pgfqpoint{0.000000in}{0.041667in}}%
\pgfpathcurveto{\pgfqpoint{-0.011050in}{0.041667in}}{\pgfqpoint{-0.021649in}{0.037276in}}{\pgfqpoint{-0.029463in}{0.029463in}}%
\pgfpathcurveto{\pgfqpoint{-0.037276in}{0.021649in}}{\pgfqpoint{-0.041667in}{0.011050in}}{\pgfqpoint{-0.041667in}{0.000000in}}%
\pgfpathcurveto{\pgfqpoint{-0.041667in}{-0.011050in}}{\pgfqpoint{-0.037276in}{-0.021649in}}{\pgfqpoint{-0.029463in}{-0.029463in}}%
\pgfpathcurveto{\pgfqpoint{-0.021649in}{-0.037276in}}{\pgfqpoint{-0.011050in}{-0.041667in}}{\pgfqpoint{0.000000in}{-0.041667in}}%
\pgfpathclose%
\pgfusepath{stroke,fill}%
}%
\begin{pgfscope}%
\pgfsys@transformshift{8.368312in}{5.654281in}%
\pgfsys@useobject{currentmarker}{}%
\end{pgfscope}%
\begin{pgfscope}%
\pgfsys@transformshift{8.713395in}{5.572628in}%
\pgfsys@useobject{currentmarker}{}%
\end{pgfscope}%
\begin{pgfscope}%
\pgfsys@transformshift{9.058479in}{5.491066in}%
\pgfsys@useobject{currentmarker}{}%
\end{pgfscope}%
\begin{pgfscope}%
\pgfsys@transformshift{9.403562in}{5.394619in}%
\pgfsys@useobject{currentmarker}{}%
\end{pgfscope}%
\begin{pgfscope}%
\pgfsys@transformshift{9.748646in}{5.259789in}%
\pgfsys@useobject{currentmarker}{}%
\end{pgfscope}%
\begin{pgfscope}%
\pgfsys@transformshift{10.093729in}{4.971024in}%
\pgfsys@useobject{currentmarker}{}%
\end{pgfscope}%
\end{pgfscope}%
\begin{pgfscope}%
\pgfpathrectangle{\pgfqpoint{8.282041in}{4.908628in}}{\pgfqpoint{1.897959in}{1.372727in}} %
\pgfusepath{clip}%
\pgfsetbuttcap%
\pgfsetroundjoin%
\pgfsetlinewidth{1.505625pt}%
\definecolor{currentstroke}{rgb}{0.000000,0.750000,0.750000}%
\pgfsetstrokecolor{currentstroke}%
\pgfsetdash{{9.600000pt}{2.400000pt}{1.500000pt}{2.400000pt}}{0.000000pt}%
\pgfpathmoveto{\pgfqpoint{8.368312in}{5.987072in}}%
\pgfpathlineto{\pgfqpoint{8.391317in}{5.945048in}}%
\pgfpathlineto{\pgfqpoint{8.414323in}{5.907873in}}%
\pgfpathlineto{\pgfqpoint{8.437328in}{5.874598in}}%
\pgfpathlineto{\pgfqpoint{8.460334in}{5.844531in}}%
\pgfpathlineto{\pgfqpoint{8.483340in}{5.817148in}}%
\pgfpathlineto{\pgfqpoint{8.506345in}{5.792048in}}%
\pgfpathlineto{\pgfqpoint{8.529351in}{5.768910in}}%
\pgfpathlineto{\pgfqpoint{8.552356in}{5.747481in}}%
\pgfpathlineto{\pgfqpoint{8.575362in}{5.727551in}}%
\pgfpathlineto{\pgfqpoint{8.598367in}{5.708949in}}%
\pgfpathlineto{\pgfqpoint{8.621373in}{5.691530in}}%
\pgfpathlineto{\pgfqpoint{8.644378in}{5.675173in}}%
\pgfpathlineto{\pgfqpoint{8.667384in}{5.659774in}}%
\pgfpathlineto{\pgfqpoint{8.690390in}{5.645245in}}%
\pgfpathlineto{\pgfqpoint{8.713395in}{5.631509in}}%
\pgfpathlineto{\pgfqpoint{8.736401in}{5.618498in}}%
\pgfpathlineto{\pgfqpoint{8.759406in}{5.606154in}}%
\pgfpathlineto{\pgfqpoint{8.782412in}{5.594425in}}%
\pgfpathlineto{\pgfqpoint{8.805417in}{5.583264in}}%
\pgfpathlineto{\pgfqpoint{8.828423in}{5.572632in}}%
\pgfpathlineto{\pgfqpoint{8.851429in}{5.562492in}}%
\pgfpathlineto{\pgfqpoint{8.874434in}{5.552811in}}%
\pgfpathlineto{\pgfqpoint{8.897440in}{5.543559in}}%
\pgfpathlineto{\pgfqpoint{8.920445in}{5.534710in}}%
\pgfpathlineto{\pgfqpoint{8.943451in}{5.526240in}}%
\pgfpathlineto{\pgfqpoint{8.966456in}{5.518127in}}%
\pgfpathlineto{\pgfqpoint{8.989462in}{5.510350in}}%
\pgfpathlineto{\pgfqpoint{9.012468in}{5.502892in}}%
\pgfpathlineto{\pgfqpoint{9.035473in}{5.495736in}}%
\pgfpathlineto{\pgfqpoint{9.058479in}{5.488865in}}%
\pgfpathlineto{\pgfqpoint{9.081484in}{5.482268in}}%
\pgfpathlineto{\pgfqpoint{9.104490in}{5.475929in}}%
\pgfpathlineto{\pgfqpoint{9.127495in}{5.469837in}}%
\pgfpathlineto{\pgfqpoint{9.150501in}{5.463981in}}%
\pgfpathlineto{\pgfqpoint{9.173506in}{5.458351in}}%
\pgfpathlineto{\pgfqpoint{9.196512in}{5.452937in}}%
\pgfpathlineto{\pgfqpoint{9.219518in}{5.447730in}}%
\pgfpathlineto{\pgfqpoint{9.242523in}{5.442721in}}%
\pgfpathlineto{\pgfqpoint{9.265529in}{5.437904in}}%
\pgfpathlineto{\pgfqpoint{9.288534in}{5.433270in}}%
\pgfpathlineto{\pgfqpoint{9.311540in}{5.428813in}}%
\pgfpathlineto{\pgfqpoint{9.334545in}{5.424527in}}%
\pgfpathlineto{\pgfqpoint{9.357551in}{5.420406in}}%
\pgfpathlineto{\pgfqpoint{9.380557in}{5.416444in}}%
\pgfpathlineto{\pgfqpoint{9.403562in}{5.412636in}}%
\pgfpathlineto{\pgfqpoint{9.426568in}{5.408978in}}%
\pgfpathlineto{\pgfqpoint{9.449573in}{5.405464in}}%
\pgfpathlineto{\pgfqpoint{9.472579in}{5.402091in}}%
\pgfpathlineto{\pgfqpoint{9.495584in}{5.398853in}}%
\pgfpathlineto{\pgfqpoint{9.518590in}{5.395749in}}%
\pgfpathlineto{\pgfqpoint{9.541596in}{5.392773in}}%
\pgfpathlineto{\pgfqpoint{9.564601in}{5.389923in}}%
\pgfpathlineto{\pgfqpoint{9.587607in}{5.387196in}}%
\pgfpathlineto{\pgfqpoint{9.610612in}{5.384588in}}%
\pgfpathlineto{\pgfqpoint{9.633618in}{5.382097in}}%
\pgfpathlineto{\pgfqpoint{9.656623in}{5.379720in}}%
\pgfpathlineto{\pgfqpoint{9.679629in}{5.377454in}}%
\pgfpathlineto{\pgfqpoint{9.702635in}{5.375298in}}%
\pgfpathlineto{\pgfqpoint{9.725640in}{5.373249in}}%
\pgfpathlineto{\pgfqpoint{9.748646in}{5.371305in}}%
\pgfpathlineto{\pgfqpoint{9.771651in}{5.369464in}}%
\pgfpathlineto{\pgfqpoint{9.794657in}{5.367725in}}%
\pgfpathlineto{\pgfqpoint{9.817662in}{5.366085in}}%
\pgfpathlineto{\pgfqpoint{9.840668in}{5.364544in}}%
\pgfpathlineto{\pgfqpoint{9.863673in}{5.363099in}}%
\pgfpathlineto{\pgfqpoint{9.886679in}{5.361749in}}%
\pgfpathlineto{\pgfqpoint{9.909685in}{5.360493in}}%
\pgfpathlineto{\pgfqpoint{9.932690in}{5.359331in}}%
\pgfpathlineto{\pgfqpoint{9.955696in}{5.358259in}}%
\pgfpathlineto{\pgfqpoint{9.978701in}{5.357279in}}%
\pgfpathlineto{\pgfqpoint{10.001707in}{5.356389in}}%
\pgfpathlineto{\pgfqpoint{10.024712in}{5.355587in}}%
\pgfpathlineto{\pgfqpoint{10.047718in}{5.354874in}}%
\pgfpathlineto{\pgfqpoint{10.070724in}{5.354248in}}%
\pgfpathlineto{\pgfqpoint{10.093729in}{5.353710in}}%
\pgfusepath{stroke}%
\end{pgfscope}%
\begin{pgfscope}%
\pgfpathrectangle{\pgfqpoint{8.282041in}{4.908628in}}{\pgfqpoint{1.897959in}{1.372727in}} %
\pgfusepath{clip}%
\pgfsetbuttcap%
\pgfsetmiterjoin%
\definecolor{currentfill}{rgb}{0.000000,0.750000,0.750000}%
\pgfsetfillcolor{currentfill}%
\pgfsetlinewidth{1.003750pt}%
\definecolor{currentstroke}{rgb}{0.000000,0.750000,0.750000}%
\pgfsetstrokecolor{currentstroke}%
\pgfsetdash{}{0pt}%
\pgfsys@defobject{currentmarker}{\pgfqpoint{-0.041667in}{-0.041667in}}{\pgfqpoint{0.041667in}{0.041667in}}{%
\pgfpathmoveto{\pgfqpoint{-0.000000in}{-0.041667in}}%
\pgfpathlineto{\pgfqpoint{0.041667in}{0.041667in}}%
\pgfpathlineto{\pgfqpoint{-0.041667in}{0.041667in}}%
\pgfpathclose%
\pgfusepath{stroke,fill}%
}%
\begin{pgfscope}%
\pgfsys@transformshift{8.368312in}{5.987072in}%
\pgfsys@useobject{currentmarker}{}%
\end{pgfscope}%
\begin{pgfscope}%
\pgfsys@transformshift{8.713395in}{5.631509in}%
\pgfsys@useobject{currentmarker}{}%
\end{pgfscope}%
\begin{pgfscope}%
\pgfsys@transformshift{9.058479in}{5.488865in}%
\pgfsys@useobject{currentmarker}{}%
\end{pgfscope}%
\begin{pgfscope}%
\pgfsys@transformshift{9.403562in}{5.412636in}%
\pgfsys@useobject{currentmarker}{}%
\end{pgfscope}%
\begin{pgfscope}%
\pgfsys@transformshift{9.748646in}{5.371305in}%
\pgfsys@useobject{currentmarker}{}%
\end{pgfscope}%
\begin{pgfscope}%
\pgfsys@transformshift{10.093729in}{5.353710in}%
\pgfsys@useobject{currentmarker}{}%
\end{pgfscope}%
\end{pgfscope}%
\begin{pgfscope}%
\pgfpathrectangle{\pgfqpoint{8.282041in}{4.908628in}}{\pgfqpoint{1.897959in}{1.372727in}} %
\pgfusepath{clip}%
\pgfsetbuttcap%
\pgfsetroundjoin%
\pgfsetlinewidth{1.505625pt}%
\definecolor{currentstroke}{rgb}{0.000000,0.000000,0.000000}%
\pgfsetstrokecolor{currentstroke}%
\pgfsetdash{{1.500000pt}{2.475000pt}}{0.000000pt}%
\pgfpathmoveto{\pgfqpoint{8.368312in}{6.218958in}}%
\pgfpathlineto{\pgfqpoint{8.391317in}{6.206345in}}%
\pgfpathlineto{\pgfqpoint{8.414323in}{6.194429in}}%
\pgfpathlineto{\pgfqpoint{8.437328in}{6.184498in}}%
\pgfpathlineto{\pgfqpoint{8.460334in}{6.176016in}}%
\pgfpathlineto{\pgfqpoint{8.483340in}{6.168623in}}%
\pgfpathlineto{\pgfqpoint{8.506345in}{6.162067in}}%
\pgfpathlineto{\pgfqpoint{8.529351in}{6.156166in}}%
\pgfpathlineto{\pgfqpoint{8.552356in}{6.150788in}}%
\pgfpathlineto{\pgfqpoint{8.575362in}{6.145834in}}%
\pgfpathlineto{\pgfqpoint{8.598367in}{6.141231in}}%
\pgfpathlineto{\pgfqpoint{8.621373in}{6.136921in}}%
\pgfpathlineto{\pgfqpoint{8.644378in}{6.132858in}}%
\pgfpathlineto{\pgfqpoint{8.667384in}{6.129009in}}%
\pgfpathlineto{\pgfqpoint{8.690390in}{6.125346in}}%
\pgfpathlineto{\pgfqpoint{8.713395in}{6.121844in}}%
\pgfpathlineto{\pgfqpoint{8.736401in}{6.118487in}}%
\pgfpathlineto{\pgfqpoint{8.759406in}{6.115259in}}%
\pgfpathlineto{\pgfqpoint{8.782412in}{6.112147in}}%
\pgfpathlineto{\pgfqpoint{8.805417in}{6.109142in}}%
\pgfpathlineto{\pgfqpoint{8.828423in}{6.106234in}}%
\pgfpathlineto{\pgfqpoint{8.851429in}{6.103417in}}%
\pgfpathlineto{\pgfqpoint{8.874434in}{6.100683in}}%
\pgfpathlineto{\pgfqpoint{8.897440in}{6.098028in}}%
\pgfpathlineto{\pgfqpoint{8.920445in}{6.095446in}}%
\pgfpathlineto{\pgfqpoint{8.943451in}{6.092934in}}%
\pgfpathlineto{\pgfqpoint{8.966456in}{6.090489in}}%
\pgfpathlineto{\pgfqpoint{8.989462in}{6.088107in}}%
\pgfpathlineto{\pgfqpoint{9.012468in}{6.085786in}}%
\pgfpathlineto{\pgfqpoint{9.035473in}{6.083523in}}%
\pgfpathlineto{\pgfqpoint{9.058479in}{6.081317in}}%
\pgfpathlineto{\pgfqpoint{9.081484in}{6.079165in}}%
\pgfpathlineto{\pgfqpoint{9.104490in}{6.077065in}}%
\pgfpathlineto{\pgfqpoint{9.127495in}{6.075016in}}%
\pgfpathlineto{\pgfqpoint{9.150501in}{6.073018in}}%
\pgfpathlineto{\pgfqpoint{9.173506in}{6.071068in}}%
\pgfpathlineto{\pgfqpoint{9.196512in}{6.069165in}}%
\pgfpathlineto{\pgfqpoint{9.219518in}{6.067309in}}%
\pgfpathlineto{\pgfqpoint{9.242523in}{6.065499in}}%
\pgfpathlineto{\pgfqpoint{9.265529in}{6.063733in}}%
\pgfpathlineto{\pgfqpoint{9.288534in}{6.062011in}}%
\pgfpathlineto{\pgfqpoint{9.311540in}{6.060333in}}%
\pgfpathlineto{\pgfqpoint{9.334545in}{6.058697in}}%
\pgfpathlineto{\pgfqpoint{9.357551in}{6.057103in}}%
\pgfpathlineto{\pgfqpoint{9.380557in}{6.055551in}}%
\pgfpathlineto{\pgfqpoint{9.403562in}{6.054039in}}%
\pgfpathlineto{\pgfqpoint{9.426568in}{6.052569in}}%
\pgfpathlineto{\pgfqpoint{9.449573in}{6.051138in}}%
\pgfpathlineto{\pgfqpoint{9.472579in}{6.049746in}}%
\pgfpathlineto{\pgfqpoint{9.495584in}{6.048394in}}%
\pgfpathlineto{\pgfqpoint{9.518590in}{6.047081in}}%
\pgfpathlineto{\pgfqpoint{9.541596in}{6.045806in}}%
\pgfpathlineto{\pgfqpoint{9.564601in}{6.044570in}}%
\pgfpathlineto{\pgfqpoint{9.587607in}{6.043371in}}%
\pgfpathlineto{\pgfqpoint{9.610612in}{6.042211in}}%
\pgfpathlineto{\pgfqpoint{9.633618in}{6.041087in}}%
\pgfpathlineto{\pgfqpoint{9.656623in}{6.040001in}}%
\pgfpathlineto{\pgfqpoint{9.679629in}{6.038952in}}%
\pgfpathlineto{\pgfqpoint{9.702635in}{6.037940in}}%
\pgfpathlineto{\pgfqpoint{9.725640in}{6.036964in}}%
\pgfpathlineto{\pgfqpoint{9.748646in}{6.036025in}}%
\pgfpathlineto{\pgfqpoint{9.771651in}{6.035123in}}%
\pgfpathlineto{\pgfqpoint{9.794657in}{6.034256in}}%
\pgfpathlineto{\pgfqpoint{9.817662in}{6.033426in}}%
\pgfpathlineto{\pgfqpoint{9.840668in}{6.032632in}}%
\pgfpathlineto{\pgfqpoint{9.863673in}{6.031874in}}%
\pgfpathlineto{\pgfqpoint{9.886679in}{6.031152in}}%
\pgfpathlineto{\pgfqpoint{9.909685in}{6.030465in}}%
\pgfpathlineto{\pgfqpoint{9.932690in}{6.029815in}}%
\pgfpathlineto{\pgfqpoint{9.955696in}{6.029200in}}%
\pgfpathlineto{\pgfqpoint{9.978701in}{6.028622in}}%
\pgfpathlineto{\pgfqpoint{10.001707in}{6.028079in}}%
\pgfpathlineto{\pgfqpoint{10.024712in}{6.027573in}}%
\pgfpathlineto{\pgfqpoint{10.047718in}{6.027102in}}%
\pgfpathlineto{\pgfqpoint{10.070724in}{6.026668in}}%
\pgfpathlineto{\pgfqpoint{10.093729in}{6.026270in}}%
\pgfusepath{stroke}%
\end{pgfscope}%
\begin{pgfscope}%
\pgfpathrectangle{\pgfqpoint{8.282041in}{4.908628in}}{\pgfqpoint{1.897959in}{1.372727in}} %
\pgfusepath{clip}%
\pgfsetbuttcap%
\pgfsetroundjoin%
\definecolor{currentfill}{rgb}{0.000000,0.000000,0.000000}%
\pgfsetfillcolor{currentfill}%
\pgfsetlinewidth{1.003750pt}%
\definecolor{currentstroke}{rgb}{0.000000,0.000000,0.000000}%
\pgfsetstrokecolor{currentstroke}%
\pgfsetdash{}{0pt}%
\pgfsys@defobject{currentmarker}{\pgfqpoint{-0.041667in}{-0.041667in}}{\pgfqpoint{0.041667in}{0.041667in}}{%
\pgfpathmoveto{\pgfqpoint{-0.041667in}{0.000000in}}%
\pgfpathlineto{\pgfqpoint{0.041667in}{0.000000in}}%
\pgfpathmoveto{\pgfqpoint{0.000000in}{-0.041667in}}%
\pgfpathlineto{\pgfqpoint{0.000000in}{0.041667in}}%
\pgfusepath{stroke,fill}%
}%
\begin{pgfscope}%
\pgfsys@transformshift{8.368312in}{6.218958in}%
\pgfsys@useobject{currentmarker}{}%
\end{pgfscope}%
\begin{pgfscope}%
\pgfsys@transformshift{8.713395in}{6.121844in}%
\pgfsys@useobject{currentmarker}{}%
\end{pgfscope}%
\begin{pgfscope}%
\pgfsys@transformshift{9.058479in}{6.081317in}%
\pgfsys@useobject{currentmarker}{}%
\end{pgfscope}%
\begin{pgfscope}%
\pgfsys@transformshift{9.403562in}{6.054039in}%
\pgfsys@useobject{currentmarker}{}%
\end{pgfscope}%
\begin{pgfscope}%
\pgfsys@transformshift{9.748646in}{6.036025in}%
\pgfsys@useobject{currentmarker}{}%
\end{pgfscope}%
\begin{pgfscope}%
\pgfsys@transformshift{10.093729in}{6.026270in}%
\pgfsys@useobject{currentmarker}{}%
\end{pgfscope}%
\end{pgfscope}%
\begin{pgfscope}%
\pgfsetrectcap%
\pgfsetmiterjoin%
\pgfsetlinewidth{0.803000pt}%
\definecolor{currentstroke}{rgb}{0.000000,0.000000,0.000000}%
\pgfsetstrokecolor{currentstroke}%
\pgfsetdash{}{0pt}%
\pgfpathmoveto{\pgfqpoint{8.282041in}{4.908628in}}%
\pgfpathlineto{\pgfqpoint{8.282041in}{6.281355in}}%
\pgfusepath{stroke}%
\end{pgfscope}%
\begin{pgfscope}%
\pgfsetrectcap%
\pgfsetmiterjoin%
\pgfsetlinewidth{0.803000pt}%
\definecolor{currentstroke}{rgb}{0.000000,0.000000,0.000000}%
\pgfsetstrokecolor{currentstroke}%
\pgfsetdash{}{0pt}%
\pgfpathmoveto{\pgfqpoint{10.180000in}{4.908628in}}%
\pgfpathlineto{\pgfqpoint{10.180000in}{6.281355in}}%
\pgfusepath{stroke}%
\end{pgfscope}%
\begin{pgfscope}%
\pgfsetrectcap%
\pgfsetmiterjoin%
\pgfsetlinewidth{0.803000pt}%
\definecolor{currentstroke}{rgb}{0.000000,0.000000,0.000000}%
\pgfsetstrokecolor{currentstroke}%
\pgfsetdash{}{0pt}%
\pgfpathmoveto{\pgfqpoint{8.282041in}{4.908628in}}%
\pgfpathlineto{\pgfqpoint{10.180000in}{4.908628in}}%
\pgfusepath{stroke}%
\end{pgfscope}%
\begin{pgfscope}%
\pgfsetrectcap%
\pgfsetmiterjoin%
\pgfsetlinewidth{0.803000pt}%
\definecolor{currentstroke}{rgb}{0.000000,0.000000,0.000000}%
\pgfsetstrokecolor{currentstroke}%
\pgfsetdash{}{0pt}%
\pgfpathmoveto{\pgfqpoint{8.282041in}{6.281355in}}%
\pgfpathlineto{\pgfqpoint{10.180000in}{6.281355in}}%
\pgfusepath{stroke}%
\end{pgfscope}%
\begin{pgfscope}%
\pgfsetbuttcap%
\pgfsetmiterjoin%
\definecolor{currentfill}{rgb}{1.000000,1.000000,1.000000}%
\pgfsetfillcolor{currentfill}%
\pgfsetlinewidth{0.000000pt}%
\definecolor{currentstroke}{rgb}{0.000000,0.000000,0.000000}%
\pgfsetstrokecolor{currentstroke}%
\pgfsetstrokeopacity{0.000000}%
\pgfsetdash{}{0pt}%
\pgfpathmoveto{\pgfqpoint{0.880000in}{2.849537in}}%
\pgfpathlineto{\pgfqpoint{2.777959in}{2.849537in}}%
\pgfpathlineto{\pgfqpoint{2.777959in}{4.222264in}}%
\pgfpathlineto{\pgfqpoint{0.880000in}{4.222264in}}%
\pgfpathclose%
\pgfusepath{fill}%
\end{pgfscope}%
\begin{pgfscope}%
\pgfsetbuttcap%
\pgfsetroundjoin%
\definecolor{currentfill}{rgb}{0.000000,0.000000,0.000000}%
\pgfsetfillcolor{currentfill}%
\pgfsetlinewidth{0.803000pt}%
\definecolor{currentstroke}{rgb}{0.000000,0.000000,0.000000}%
\pgfsetstrokecolor{currentstroke}%
\pgfsetdash{}{0pt}%
\pgfsys@defobject{currentmarker}{\pgfqpoint{0.000000in}{-0.048611in}}{\pgfqpoint{0.000000in}{0.000000in}}{%
\pgfpathmoveto{\pgfqpoint{0.000000in}{0.000000in}}%
\pgfpathlineto{\pgfqpoint{0.000000in}{-0.048611in}}%
\pgfusepath{stroke,fill}%
}%
\begin{pgfscope}%
\pgfsys@transformshift{1.360652in}{2.849537in}%
\pgfsys@useobject{currentmarker}{}%
\end{pgfscope}%
\end{pgfscope}%
\begin{pgfscope}%
\pgftext[x=1.360652in,y=2.752315in,,top]{\rmfamily\fontsize{10.000000}{12.000000}\selectfont \(\displaystyle 0.2\)}%
\end{pgfscope}%
\begin{pgfscope}%
\pgfsetbuttcap%
\pgfsetroundjoin%
\definecolor{currentfill}{rgb}{0.000000,0.000000,0.000000}%
\pgfsetfillcolor{currentfill}%
\pgfsetlinewidth{0.803000pt}%
\definecolor{currentstroke}{rgb}{0.000000,0.000000,0.000000}%
\pgfsetstrokecolor{currentstroke}%
\pgfsetdash{}{0pt}%
\pgfsys@defobject{currentmarker}{\pgfqpoint{0.000000in}{-0.048611in}}{\pgfqpoint{0.000000in}{0.000000in}}{%
\pgfpathmoveto{\pgfqpoint{0.000000in}{0.000000in}}%
\pgfpathlineto{\pgfqpoint{0.000000in}{-0.048611in}}%
\pgfusepath{stroke,fill}%
}%
\begin{pgfscope}%
\pgfsys@transformshift{1.952224in}{2.849537in}%
\pgfsys@useobject{currentmarker}{}%
\end{pgfscope}%
\end{pgfscope}%
\begin{pgfscope}%
\pgftext[x=1.952224in,y=2.752315in,,top]{\rmfamily\fontsize{10.000000}{12.000000}\selectfont \(\displaystyle 0.4\)}%
\end{pgfscope}%
\begin{pgfscope}%
\pgfsetbuttcap%
\pgfsetroundjoin%
\definecolor{currentfill}{rgb}{0.000000,0.000000,0.000000}%
\pgfsetfillcolor{currentfill}%
\pgfsetlinewidth{0.803000pt}%
\definecolor{currentstroke}{rgb}{0.000000,0.000000,0.000000}%
\pgfsetstrokecolor{currentstroke}%
\pgfsetdash{}{0pt}%
\pgfsys@defobject{currentmarker}{\pgfqpoint{0.000000in}{-0.048611in}}{\pgfqpoint{0.000000in}{0.000000in}}{%
\pgfpathmoveto{\pgfqpoint{0.000000in}{0.000000in}}%
\pgfpathlineto{\pgfqpoint{0.000000in}{-0.048611in}}%
\pgfusepath{stroke,fill}%
}%
\begin{pgfscope}%
\pgfsys@transformshift{2.543795in}{2.849537in}%
\pgfsys@useobject{currentmarker}{}%
\end{pgfscope}%
\end{pgfscope}%
\begin{pgfscope}%
\pgftext[x=2.543795in,y=2.752315in,,top]{\rmfamily\fontsize{10.000000}{12.000000}\selectfont \(\displaystyle 0.6\)}%
\end{pgfscope}%
\begin{pgfscope}%
\pgfsetbuttcap%
\pgfsetroundjoin%
\definecolor{currentfill}{rgb}{0.000000,0.000000,0.000000}%
\pgfsetfillcolor{currentfill}%
\pgfsetlinewidth{0.803000pt}%
\definecolor{currentstroke}{rgb}{0.000000,0.000000,0.000000}%
\pgfsetstrokecolor{currentstroke}%
\pgfsetdash{}{0pt}%
\pgfsys@defobject{currentmarker}{\pgfqpoint{-0.048611in}{0.000000in}}{\pgfqpoint{0.000000in}{0.000000in}}{%
\pgfpathmoveto{\pgfqpoint{0.000000in}{0.000000in}}%
\pgfpathlineto{\pgfqpoint{-0.048611in}{0.000000in}}%
\pgfusepath{stroke,fill}%
}%
\begin{pgfscope}%
\pgfsys@transformshift{0.880000in}{3.238450in}%
\pgfsys@useobject{currentmarker}{}%
\end{pgfscope}%
\end{pgfscope}%
\begin{pgfscope}%
\pgftext[x=0.494775in,y=3.185688in,left,base]{\rmfamily\fontsize{10.000000}{12.000000}\selectfont \(\displaystyle 10^{-7}\)}%
\end{pgfscope}%
\begin{pgfscope}%
\pgfsetbuttcap%
\pgfsetroundjoin%
\definecolor{currentfill}{rgb}{0.000000,0.000000,0.000000}%
\pgfsetfillcolor{currentfill}%
\pgfsetlinewidth{0.803000pt}%
\definecolor{currentstroke}{rgb}{0.000000,0.000000,0.000000}%
\pgfsetstrokecolor{currentstroke}%
\pgfsetdash{}{0pt}%
\pgfsys@defobject{currentmarker}{\pgfqpoint{-0.048611in}{0.000000in}}{\pgfqpoint{0.000000in}{0.000000in}}{%
\pgfpathmoveto{\pgfqpoint{0.000000in}{0.000000in}}%
\pgfpathlineto{\pgfqpoint{-0.048611in}{0.000000in}}%
\pgfusepath{stroke,fill}%
}%
\begin{pgfscope}%
\pgfsys@transformshift{0.880000in}{3.649957in}%
\pgfsys@useobject{currentmarker}{}%
\end{pgfscope}%
\end{pgfscope}%
\begin{pgfscope}%
\pgftext[x=0.494775in,y=3.597196in,left,base]{\rmfamily\fontsize{10.000000}{12.000000}\selectfont \(\displaystyle 10^{-6}\)}%
\end{pgfscope}%
\begin{pgfscope}%
\pgfsetbuttcap%
\pgfsetroundjoin%
\definecolor{currentfill}{rgb}{0.000000,0.000000,0.000000}%
\pgfsetfillcolor{currentfill}%
\pgfsetlinewidth{0.803000pt}%
\definecolor{currentstroke}{rgb}{0.000000,0.000000,0.000000}%
\pgfsetstrokecolor{currentstroke}%
\pgfsetdash{}{0pt}%
\pgfsys@defobject{currentmarker}{\pgfqpoint{-0.048611in}{0.000000in}}{\pgfqpoint{0.000000in}{0.000000in}}{%
\pgfpathmoveto{\pgfqpoint{0.000000in}{0.000000in}}%
\pgfpathlineto{\pgfqpoint{-0.048611in}{0.000000in}}%
\pgfusepath{stroke,fill}%
}%
\begin{pgfscope}%
\pgfsys@transformshift{0.880000in}{4.061464in}%
\pgfsys@useobject{currentmarker}{}%
\end{pgfscope}%
\end{pgfscope}%
\begin{pgfscope}%
\pgftext[x=0.494775in,y=4.008703in,left,base]{\rmfamily\fontsize{10.000000}{12.000000}\selectfont \(\displaystyle 10^{-5}\)}%
\end{pgfscope}%
\begin{pgfscope}%
\pgfsetbuttcap%
\pgfsetroundjoin%
\definecolor{currentfill}{rgb}{0.000000,0.000000,0.000000}%
\pgfsetfillcolor{currentfill}%
\pgfsetlinewidth{0.602250pt}%
\definecolor{currentstroke}{rgb}{0.000000,0.000000,0.000000}%
\pgfsetstrokecolor{currentstroke}%
\pgfsetdash{}{0pt}%
\pgfsys@defobject{currentmarker}{\pgfqpoint{-0.027778in}{0.000000in}}{\pgfqpoint{0.000000in}{0.000000in}}{%
\pgfpathmoveto{\pgfqpoint{0.000000in}{0.000000in}}%
\pgfpathlineto{\pgfqpoint{-0.027778in}{0.000000in}}%
\pgfusepath{stroke,fill}%
}%
\begin{pgfscope}%
\pgfsys@transformshift{0.880000in}{2.950819in}%
\pgfsys@useobject{currentmarker}{}%
\end{pgfscope}%
\end{pgfscope}%
\begin{pgfscope}%
\pgfsetbuttcap%
\pgfsetroundjoin%
\definecolor{currentfill}{rgb}{0.000000,0.000000,0.000000}%
\pgfsetfillcolor{currentfill}%
\pgfsetlinewidth{0.602250pt}%
\definecolor{currentstroke}{rgb}{0.000000,0.000000,0.000000}%
\pgfsetstrokecolor{currentstroke}%
\pgfsetdash{}{0pt}%
\pgfsys@defobject{currentmarker}{\pgfqpoint{-0.027778in}{0.000000in}}{\pgfqpoint{0.000000in}{0.000000in}}{%
\pgfpathmoveto{\pgfqpoint{0.000000in}{0.000000in}}%
\pgfpathlineto{\pgfqpoint{-0.027778in}{0.000000in}}%
\pgfusepath{stroke,fill}%
}%
\begin{pgfscope}%
\pgfsys@transformshift{0.880000in}{3.023282in}%
\pgfsys@useobject{currentmarker}{}%
\end{pgfscope}%
\end{pgfscope}%
\begin{pgfscope}%
\pgfsetbuttcap%
\pgfsetroundjoin%
\definecolor{currentfill}{rgb}{0.000000,0.000000,0.000000}%
\pgfsetfillcolor{currentfill}%
\pgfsetlinewidth{0.602250pt}%
\definecolor{currentstroke}{rgb}{0.000000,0.000000,0.000000}%
\pgfsetstrokecolor{currentstroke}%
\pgfsetdash{}{0pt}%
\pgfsys@defobject{currentmarker}{\pgfqpoint{-0.027778in}{0.000000in}}{\pgfqpoint{0.000000in}{0.000000in}}{%
\pgfpathmoveto{\pgfqpoint{0.000000in}{0.000000in}}%
\pgfpathlineto{\pgfqpoint{-0.027778in}{0.000000in}}%
\pgfusepath{stroke,fill}%
}%
\begin{pgfscope}%
\pgfsys@transformshift{0.880000in}{3.074695in}%
\pgfsys@useobject{currentmarker}{}%
\end{pgfscope}%
\end{pgfscope}%
\begin{pgfscope}%
\pgfsetbuttcap%
\pgfsetroundjoin%
\definecolor{currentfill}{rgb}{0.000000,0.000000,0.000000}%
\pgfsetfillcolor{currentfill}%
\pgfsetlinewidth{0.602250pt}%
\definecolor{currentstroke}{rgb}{0.000000,0.000000,0.000000}%
\pgfsetstrokecolor{currentstroke}%
\pgfsetdash{}{0pt}%
\pgfsys@defobject{currentmarker}{\pgfqpoint{-0.027778in}{0.000000in}}{\pgfqpoint{0.000000in}{0.000000in}}{%
\pgfpathmoveto{\pgfqpoint{0.000000in}{0.000000in}}%
\pgfpathlineto{\pgfqpoint{-0.027778in}{0.000000in}}%
\pgfusepath{stroke,fill}%
}%
\begin{pgfscope}%
\pgfsys@transformshift{0.880000in}{3.114574in}%
\pgfsys@useobject{currentmarker}{}%
\end{pgfscope}%
\end{pgfscope}%
\begin{pgfscope}%
\pgfsetbuttcap%
\pgfsetroundjoin%
\definecolor{currentfill}{rgb}{0.000000,0.000000,0.000000}%
\pgfsetfillcolor{currentfill}%
\pgfsetlinewidth{0.602250pt}%
\definecolor{currentstroke}{rgb}{0.000000,0.000000,0.000000}%
\pgfsetstrokecolor{currentstroke}%
\pgfsetdash{}{0pt}%
\pgfsys@defobject{currentmarker}{\pgfqpoint{-0.027778in}{0.000000in}}{\pgfqpoint{0.000000in}{0.000000in}}{%
\pgfpathmoveto{\pgfqpoint{0.000000in}{0.000000in}}%
\pgfpathlineto{\pgfqpoint{-0.027778in}{0.000000in}}%
\pgfusepath{stroke,fill}%
}%
\begin{pgfscope}%
\pgfsys@transformshift{0.880000in}{3.147158in}%
\pgfsys@useobject{currentmarker}{}%
\end{pgfscope}%
\end{pgfscope}%
\begin{pgfscope}%
\pgfsetbuttcap%
\pgfsetroundjoin%
\definecolor{currentfill}{rgb}{0.000000,0.000000,0.000000}%
\pgfsetfillcolor{currentfill}%
\pgfsetlinewidth{0.602250pt}%
\definecolor{currentstroke}{rgb}{0.000000,0.000000,0.000000}%
\pgfsetstrokecolor{currentstroke}%
\pgfsetdash{}{0pt}%
\pgfsys@defobject{currentmarker}{\pgfqpoint{-0.027778in}{0.000000in}}{\pgfqpoint{0.000000in}{0.000000in}}{%
\pgfpathmoveto{\pgfqpoint{0.000000in}{0.000000in}}%
\pgfpathlineto{\pgfqpoint{-0.027778in}{0.000000in}}%
\pgfusepath{stroke,fill}%
}%
\begin{pgfscope}%
\pgfsys@transformshift{0.880000in}{3.174707in}%
\pgfsys@useobject{currentmarker}{}%
\end{pgfscope}%
\end{pgfscope}%
\begin{pgfscope}%
\pgfsetbuttcap%
\pgfsetroundjoin%
\definecolor{currentfill}{rgb}{0.000000,0.000000,0.000000}%
\pgfsetfillcolor{currentfill}%
\pgfsetlinewidth{0.602250pt}%
\definecolor{currentstroke}{rgb}{0.000000,0.000000,0.000000}%
\pgfsetstrokecolor{currentstroke}%
\pgfsetdash{}{0pt}%
\pgfsys@defobject{currentmarker}{\pgfqpoint{-0.027778in}{0.000000in}}{\pgfqpoint{0.000000in}{0.000000in}}{%
\pgfpathmoveto{\pgfqpoint{0.000000in}{0.000000in}}%
\pgfpathlineto{\pgfqpoint{-0.027778in}{0.000000in}}%
\pgfusepath{stroke,fill}%
}%
\begin{pgfscope}%
\pgfsys@transformshift{0.880000in}{3.198571in}%
\pgfsys@useobject{currentmarker}{}%
\end{pgfscope}%
\end{pgfscope}%
\begin{pgfscope}%
\pgfsetbuttcap%
\pgfsetroundjoin%
\definecolor{currentfill}{rgb}{0.000000,0.000000,0.000000}%
\pgfsetfillcolor{currentfill}%
\pgfsetlinewidth{0.602250pt}%
\definecolor{currentstroke}{rgb}{0.000000,0.000000,0.000000}%
\pgfsetstrokecolor{currentstroke}%
\pgfsetdash{}{0pt}%
\pgfsys@defobject{currentmarker}{\pgfqpoint{-0.027778in}{0.000000in}}{\pgfqpoint{0.000000in}{0.000000in}}{%
\pgfpathmoveto{\pgfqpoint{0.000000in}{0.000000in}}%
\pgfpathlineto{\pgfqpoint{-0.027778in}{0.000000in}}%
\pgfusepath{stroke,fill}%
}%
\begin{pgfscope}%
\pgfsys@transformshift{0.880000in}{3.219620in}%
\pgfsys@useobject{currentmarker}{}%
\end{pgfscope}%
\end{pgfscope}%
\begin{pgfscope}%
\pgfsetbuttcap%
\pgfsetroundjoin%
\definecolor{currentfill}{rgb}{0.000000,0.000000,0.000000}%
\pgfsetfillcolor{currentfill}%
\pgfsetlinewidth{0.602250pt}%
\definecolor{currentstroke}{rgb}{0.000000,0.000000,0.000000}%
\pgfsetstrokecolor{currentstroke}%
\pgfsetdash{}{0pt}%
\pgfsys@defobject{currentmarker}{\pgfqpoint{-0.027778in}{0.000000in}}{\pgfqpoint{0.000000in}{0.000000in}}{%
\pgfpathmoveto{\pgfqpoint{0.000000in}{0.000000in}}%
\pgfpathlineto{\pgfqpoint{-0.027778in}{0.000000in}}%
\pgfusepath{stroke,fill}%
}%
\begin{pgfscope}%
\pgfsys@transformshift{0.880000in}{3.362326in}%
\pgfsys@useobject{currentmarker}{}%
\end{pgfscope}%
\end{pgfscope}%
\begin{pgfscope}%
\pgfsetbuttcap%
\pgfsetroundjoin%
\definecolor{currentfill}{rgb}{0.000000,0.000000,0.000000}%
\pgfsetfillcolor{currentfill}%
\pgfsetlinewidth{0.602250pt}%
\definecolor{currentstroke}{rgb}{0.000000,0.000000,0.000000}%
\pgfsetstrokecolor{currentstroke}%
\pgfsetdash{}{0pt}%
\pgfsys@defobject{currentmarker}{\pgfqpoint{-0.027778in}{0.000000in}}{\pgfqpoint{0.000000in}{0.000000in}}{%
\pgfpathmoveto{\pgfqpoint{0.000000in}{0.000000in}}%
\pgfpathlineto{\pgfqpoint{-0.027778in}{0.000000in}}%
\pgfusepath{stroke,fill}%
}%
\begin{pgfscope}%
\pgfsys@transformshift{0.880000in}{3.434789in}%
\pgfsys@useobject{currentmarker}{}%
\end{pgfscope}%
\end{pgfscope}%
\begin{pgfscope}%
\pgfsetbuttcap%
\pgfsetroundjoin%
\definecolor{currentfill}{rgb}{0.000000,0.000000,0.000000}%
\pgfsetfillcolor{currentfill}%
\pgfsetlinewidth{0.602250pt}%
\definecolor{currentstroke}{rgb}{0.000000,0.000000,0.000000}%
\pgfsetstrokecolor{currentstroke}%
\pgfsetdash{}{0pt}%
\pgfsys@defobject{currentmarker}{\pgfqpoint{-0.027778in}{0.000000in}}{\pgfqpoint{0.000000in}{0.000000in}}{%
\pgfpathmoveto{\pgfqpoint{0.000000in}{0.000000in}}%
\pgfpathlineto{\pgfqpoint{-0.027778in}{0.000000in}}%
\pgfusepath{stroke,fill}%
}%
\begin{pgfscope}%
\pgfsys@transformshift{0.880000in}{3.486202in}%
\pgfsys@useobject{currentmarker}{}%
\end{pgfscope}%
\end{pgfscope}%
\begin{pgfscope}%
\pgfsetbuttcap%
\pgfsetroundjoin%
\definecolor{currentfill}{rgb}{0.000000,0.000000,0.000000}%
\pgfsetfillcolor{currentfill}%
\pgfsetlinewidth{0.602250pt}%
\definecolor{currentstroke}{rgb}{0.000000,0.000000,0.000000}%
\pgfsetstrokecolor{currentstroke}%
\pgfsetdash{}{0pt}%
\pgfsys@defobject{currentmarker}{\pgfqpoint{-0.027778in}{0.000000in}}{\pgfqpoint{0.000000in}{0.000000in}}{%
\pgfpathmoveto{\pgfqpoint{0.000000in}{0.000000in}}%
\pgfpathlineto{\pgfqpoint{-0.027778in}{0.000000in}}%
\pgfusepath{stroke,fill}%
}%
\begin{pgfscope}%
\pgfsys@transformshift{0.880000in}{3.526081in}%
\pgfsys@useobject{currentmarker}{}%
\end{pgfscope}%
\end{pgfscope}%
\begin{pgfscope}%
\pgfsetbuttcap%
\pgfsetroundjoin%
\definecolor{currentfill}{rgb}{0.000000,0.000000,0.000000}%
\pgfsetfillcolor{currentfill}%
\pgfsetlinewidth{0.602250pt}%
\definecolor{currentstroke}{rgb}{0.000000,0.000000,0.000000}%
\pgfsetstrokecolor{currentstroke}%
\pgfsetdash{}{0pt}%
\pgfsys@defobject{currentmarker}{\pgfqpoint{-0.027778in}{0.000000in}}{\pgfqpoint{0.000000in}{0.000000in}}{%
\pgfpathmoveto{\pgfqpoint{0.000000in}{0.000000in}}%
\pgfpathlineto{\pgfqpoint{-0.027778in}{0.000000in}}%
\pgfusepath{stroke,fill}%
}%
\begin{pgfscope}%
\pgfsys@transformshift{0.880000in}{3.558665in}%
\pgfsys@useobject{currentmarker}{}%
\end{pgfscope}%
\end{pgfscope}%
\begin{pgfscope}%
\pgfsetbuttcap%
\pgfsetroundjoin%
\definecolor{currentfill}{rgb}{0.000000,0.000000,0.000000}%
\pgfsetfillcolor{currentfill}%
\pgfsetlinewidth{0.602250pt}%
\definecolor{currentstroke}{rgb}{0.000000,0.000000,0.000000}%
\pgfsetstrokecolor{currentstroke}%
\pgfsetdash{}{0pt}%
\pgfsys@defobject{currentmarker}{\pgfqpoint{-0.027778in}{0.000000in}}{\pgfqpoint{0.000000in}{0.000000in}}{%
\pgfpathmoveto{\pgfqpoint{0.000000in}{0.000000in}}%
\pgfpathlineto{\pgfqpoint{-0.027778in}{0.000000in}}%
\pgfusepath{stroke,fill}%
}%
\begin{pgfscope}%
\pgfsys@transformshift{0.880000in}{3.586214in}%
\pgfsys@useobject{currentmarker}{}%
\end{pgfscope}%
\end{pgfscope}%
\begin{pgfscope}%
\pgfsetbuttcap%
\pgfsetroundjoin%
\definecolor{currentfill}{rgb}{0.000000,0.000000,0.000000}%
\pgfsetfillcolor{currentfill}%
\pgfsetlinewidth{0.602250pt}%
\definecolor{currentstroke}{rgb}{0.000000,0.000000,0.000000}%
\pgfsetstrokecolor{currentstroke}%
\pgfsetdash{}{0pt}%
\pgfsys@defobject{currentmarker}{\pgfqpoint{-0.027778in}{0.000000in}}{\pgfqpoint{0.000000in}{0.000000in}}{%
\pgfpathmoveto{\pgfqpoint{0.000000in}{0.000000in}}%
\pgfpathlineto{\pgfqpoint{-0.027778in}{0.000000in}}%
\pgfusepath{stroke,fill}%
}%
\begin{pgfscope}%
\pgfsys@transformshift{0.880000in}{3.610078in}%
\pgfsys@useobject{currentmarker}{}%
\end{pgfscope}%
\end{pgfscope}%
\begin{pgfscope}%
\pgfsetbuttcap%
\pgfsetroundjoin%
\definecolor{currentfill}{rgb}{0.000000,0.000000,0.000000}%
\pgfsetfillcolor{currentfill}%
\pgfsetlinewidth{0.602250pt}%
\definecolor{currentstroke}{rgb}{0.000000,0.000000,0.000000}%
\pgfsetstrokecolor{currentstroke}%
\pgfsetdash{}{0pt}%
\pgfsys@defobject{currentmarker}{\pgfqpoint{-0.027778in}{0.000000in}}{\pgfqpoint{0.000000in}{0.000000in}}{%
\pgfpathmoveto{\pgfqpoint{0.000000in}{0.000000in}}%
\pgfpathlineto{\pgfqpoint{-0.027778in}{0.000000in}}%
\pgfusepath{stroke,fill}%
}%
\begin{pgfscope}%
\pgfsys@transformshift{0.880000in}{3.631128in}%
\pgfsys@useobject{currentmarker}{}%
\end{pgfscope}%
\end{pgfscope}%
\begin{pgfscope}%
\pgfsetbuttcap%
\pgfsetroundjoin%
\definecolor{currentfill}{rgb}{0.000000,0.000000,0.000000}%
\pgfsetfillcolor{currentfill}%
\pgfsetlinewidth{0.602250pt}%
\definecolor{currentstroke}{rgb}{0.000000,0.000000,0.000000}%
\pgfsetstrokecolor{currentstroke}%
\pgfsetdash{}{0pt}%
\pgfsys@defobject{currentmarker}{\pgfqpoint{-0.027778in}{0.000000in}}{\pgfqpoint{0.000000in}{0.000000in}}{%
\pgfpathmoveto{\pgfqpoint{0.000000in}{0.000000in}}%
\pgfpathlineto{\pgfqpoint{-0.027778in}{0.000000in}}%
\pgfusepath{stroke,fill}%
}%
\begin{pgfscope}%
\pgfsys@transformshift{0.880000in}{3.773833in}%
\pgfsys@useobject{currentmarker}{}%
\end{pgfscope}%
\end{pgfscope}%
\begin{pgfscope}%
\pgfsetbuttcap%
\pgfsetroundjoin%
\definecolor{currentfill}{rgb}{0.000000,0.000000,0.000000}%
\pgfsetfillcolor{currentfill}%
\pgfsetlinewidth{0.602250pt}%
\definecolor{currentstroke}{rgb}{0.000000,0.000000,0.000000}%
\pgfsetstrokecolor{currentstroke}%
\pgfsetdash{}{0pt}%
\pgfsys@defobject{currentmarker}{\pgfqpoint{-0.027778in}{0.000000in}}{\pgfqpoint{0.000000in}{0.000000in}}{%
\pgfpathmoveto{\pgfqpoint{0.000000in}{0.000000in}}%
\pgfpathlineto{\pgfqpoint{-0.027778in}{0.000000in}}%
\pgfusepath{stroke,fill}%
}%
\begin{pgfscope}%
\pgfsys@transformshift{0.880000in}{3.846296in}%
\pgfsys@useobject{currentmarker}{}%
\end{pgfscope}%
\end{pgfscope}%
\begin{pgfscope}%
\pgfsetbuttcap%
\pgfsetroundjoin%
\definecolor{currentfill}{rgb}{0.000000,0.000000,0.000000}%
\pgfsetfillcolor{currentfill}%
\pgfsetlinewidth{0.602250pt}%
\definecolor{currentstroke}{rgb}{0.000000,0.000000,0.000000}%
\pgfsetstrokecolor{currentstroke}%
\pgfsetdash{}{0pt}%
\pgfsys@defobject{currentmarker}{\pgfqpoint{-0.027778in}{0.000000in}}{\pgfqpoint{0.000000in}{0.000000in}}{%
\pgfpathmoveto{\pgfqpoint{0.000000in}{0.000000in}}%
\pgfpathlineto{\pgfqpoint{-0.027778in}{0.000000in}}%
\pgfusepath{stroke,fill}%
}%
\begin{pgfscope}%
\pgfsys@transformshift{0.880000in}{3.897709in}%
\pgfsys@useobject{currentmarker}{}%
\end{pgfscope}%
\end{pgfscope}%
\begin{pgfscope}%
\pgfsetbuttcap%
\pgfsetroundjoin%
\definecolor{currentfill}{rgb}{0.000000,0.000000,0.000000}%
\pgfsetfillcolor{currentfill}%
\pgfsetlinewidth{0.602250pt}%
\definecolor{currentstroke}{rgb}{0.000000,0.000000,0.000000}%
\pgfsetstrokecolor{currentstroke}%
\pgfsetdash{}{0pt}%
\pgfsys@defobject{currentmarker}{\pgfqpoint{-0.027778in}{0.000000in}}{\pgfqpoint{0.000000in}{0.000000in}}{%
\pgfpathmoveto{\pgfqpoint{0.000000in}{0.000000in}}%
\pgfpathlineto{\pgfqpoint{-0.027778in}{0.000000in}}%
\pgfusepath{stroke,fill}%
}%
\begin{pgfscope}%
\pgfsys@transformshift{0.880000in}{3.937588in}%
\pgfsys@useobject{currentmarker}{}%
\end{pgfscope}%
\end{pgfscope}%
\begin{pgfscope}%
\pgfsetbuttcap%
\pgfsetroundjoin%
\definecolor{currentfill}{rgb}{0.000000,0.000000,0.000000}%
\pgfsetfillcolor{currentfill}%
\pgfsetlinewidth{0.602250pt}%
\definecolor{currentstroke}{rgb}{0.000000,0.000000,0.000000}%
\pgfsetstrokecolor{currentstroke}%
\pgfsetdash{}{0pt}%
\pgfsys@defobject{currentmarker}{\pgfqpoint{-0.027778in}{0.000000in}}{\pgfqpoint{0.000000in}{0.000000in}}{%
\pgfpathmoveto{\pgfqpoint{0.000000in}{0.000000in}}%
\pgfpathlineto{\pgfqpoint{-0.027778in}{0.000000in}}%
\pgfusepath{stroke,fill}%
}%
\begin{pgfscope}%
\pgfsys@transformshift{0.880000in}{3.970172in}%
\pgfsys@useobject{currentmarker}{}%
\end{pgfscope}%
\end{pgfscope}%
\begin{pgfscope}%
\pgfsetbuttcap%
\pgfsetroundjoin%
\definecolor{currentfill}{rgb}{0.000000,0.000000,0.000000}%
\pgfsetfillcolor{currentfill}%
\pgfsetlinewidth{0.602250pt}%
\definecolor{currentstroke}{rgb}{0.000000,0.000000,0.000000}%
\pgfsetstrokecolor{currentstroke}%
\pgfsetdash{}{0pt}%
\pgfsys@defobject{currentmarker}{\pgfqpoint{-0.027778in}{0.000000in}}{\pgfqpoint{0.000000in}{0.000000in}}{%
\pgfpathmoveto{\pgfqpoint{0.000000in}{0.000000in}}%
\pgfpathlineto{\pgfqpoint{-0.027778in}{0.000000in}}%
\pgfusepath{stroke,fill}%
}%
\begin{pgfscope}%
\pgfsys@transformshift{0.880000in}{3.997721in}%
\pgfsys@useobject{currentmarker}{}%
\end{pgfscope}%
\end{pgfscope}%
\begin{pgfscope}%
\pgfsetbuttcap%
\pgfsetroundjoin%
\definecolor{currentfill}{rgb}{0.000000,0.000000,0.000000}%
\pgfsetfillcolor{currentfill}%
\pgfsetlinewidth{0.602250pt}%
\definecolor{currentstroke}{rgb}{0.000000,0.000000,0.000000}%
\pgfsetstrokecolor{currentstroke}%
\pgfsetdash{}{0pt}%
\pgfsys@defobject{currentmarker}{\pgfqpoint{-0.027778in}{0.000000in}}{\pgfqpoint{0.000000in}{0.000000in}}{%
\pgfpathmoveto{\pgfqpoint{0.000000in}{0.000000in}}%
\pgfpathlineto{\pgfqpoint{-0.027778in}{0.000000in}}%
\pgfusepath{stroke,fill}%
}%
\begin{pgfscope}%
\pgfsys@transformshift{0.880000in}{4.021585in}%
\pgfsys@useobject{currentmarker}{}%
\end{pgfscope}%
\end{pgfscope}%
\begin{pgfscope}%
\pgfsetbuttcap%
\pgfsetroundjoin%
\definecolor{currentfill}{rgb}{0.000000,0.000000,0.000000}%
\pgfsetfillcolor{currentfill}%
\pgfsetlinewidth{0.602250pt}%
\definecolor{currentstroke}{rgb}{0.000000,0.000000,0.000000}%
\pgfsetstrokecolor{currentstroke}%
\pgfsetdash{}{0pt}%
\pgfsys@defobject{currentmarker}{\pgfqpoint{-0.027778in}{0.000000in}}{\pgfqpoint{0.000000in}{0.000000in}}{%
\pgfpathmoveto{\pgfqpoint{0.000000in}{0.000000in}}%
\pgfpathlineto{\pgfqpoint{-0.027778in}{0.000000in}}%
\pgfusepath{stroke,fill}%
}%
\begin{pgfscope}%
\pgfsys@transformshift{0.880000in}{4.042635in}%
\pgfsys@useobject{currentmarker}{}%
\end{pgfscope}%
\end{pgfscope}%
\begin{pgfscope}%
\pgfsetbuttcap%
\pgfsetroundjoin%
\definecolor{currentfill}{rgb}{0.000000,0.000000,0.000000}%
\pgfsetfillcolor{currentfill}%
\pgfsetlinewidth{0.602250pt}%
\definecolor{currentstroke}{rgb}{0.000000,0.000000,0.000000}%
\pgfsetstrokecolor{currentstroke}%
\pgfsetdash{}{0pt}%
\pgfsys@defobject{currentmarker}{\pgfqpoint{-0.027778in}{0.000000in}}{\pgfqpoint{0.000000in}{0.000000in}}{%
\pgfpathmoveto{\pgfqpoint{0.000000in}{0.000000in}}%
\pgfpathlineto{\pgfqpoint{-0.027778in}{0.000000in}}%
\pgfusepath{stroke,fill}%
}%
\begin{pgfscope}%
\pgfsys@transformshift{0.880000in}{4.185341in}%
\pgfsys@useobject{currentmarker}{}%
\end{pgfscope}%
\end{pgfscope}%
\begin{pgfscope}%
\pgfpathrectangle{\pgfqpoint{0.880000in}{2.849537in}}{\pgfqpoint{1.897959in}{1.372727in}} %
\pgfusepath{clip}%
\pgfsetbuttcap%
\pgfsetroundjoin%
\pgfsetlinewidth{1.505625pt}%
\definecolor{currentstroke}{rgb}{1.000000,0.000000,0.000000}%
\pgfsetstrokecolor{currentstroke}%
\pgfsetdash{{5.550000pt}{2.400000pt}}{0.000000pt}%
\pgfpathmoveto{\pgfqpoint{0.966271in}{4.087056in}}%
\pgfpathlineto{\pgfqpoint{0.990920in}{4.084235in}}%
\pgfpathlineto{\pgfqpoint{1.015569in}{4.081457in}}%
\pgfpathlineto{\pgfqpoint{1.040217in}{4.078722in}}%
\pgfpathlineto{\pgfqpoint{1.064866in}{4.076028in}}%
\pgfpathlineto{\pgfqpoint{1.089515in}{4.073376in}}%
\pgfpathlineto{\pgfqpoint{1.114164in}{4.070765in}}%
\pgfpathlineto{\pgfqpoint{1.138813in}{4.068195in}}%
\pgfpathlineto{\pgfqpoint{1.163461in}{4.065666in}}%
\pgfpathlineto{\pgfqpoint{1.188110in}{4.063178in}}%
\pgfpathlineto{\pgfqpoint{1.212759in}{4.060732in}}%
\pgfpathlineto{\pgfqpoint{1.237408in}{4.058327in}}%
\pgfpathlineto{\pgfqpoint{1.262057in}{4.055963in}}%
\pgfpathlineto{\pgfqpoint{1.286706in}{4.053641in}}%
\pgfpathlineto{\pgfqpoint{1.311354in}{4.051361in}}%
\pgfpathlineto{\pgfqpoint{1.336003in}{4.049122in}}%
\pgfpathlineto{\pgfqpoint{1.360652in}{4.046925in}}%
\pgfpathlineto{\pgfqpoint{1.385301in}{4.044770in}}%
\pgfpathlineto{\pgfqpoint{1.409950in}{4.042656in}}%
\pgfpathlineto{\pgfqpoint{1.434598in}{4.040584in}}%
\pgfpathlineto{\pgfqpoint{1.459247in}{4.038554in}}%
\pgfpathlineto{\pgfqpoint{1.483896in}{4.036564in}}%
\pgfpathlineto{\pgfqpoint{1.508545in}{4.034616in}}%
\pgfpathlineto{\pgfqpoint{1.533194in}{4.032710in}}%
\pgfpathlineto{\pgfqpoint{1.557843in}{4.030844in}}%
\pgfpathlineto{\pgfqpoint{1.582491in}{4.029019in}}%
\pgfpathlineto{\pgfqpoint{1.607140in}{4.027235in}}%
\pgfpathlineto{\pgfqpoint{1.631789in}{4.025492in}}%
\pgfpathlineto{\pgfqpoint{1.656438in}{4.023788in}}%
\pgfpathlineto{\pgfqpoint{1.681087in}{4.022126in}}%
\pgfpathlineto{\pgfqpoint{1.705735in}{4.020503in}}%
\pgfpathlineto{\pgfqpoint{1.730384in}{4.018919in}}%
\pgfpathlineto{\pgfqpoint{1.755033in}{4.017376in}}%
\pgfpathlineto{\pgfqpoint{1.779682in}{4.015871in}}%
\pgfpathlineto{\pgfqpoint{1.804331in}{4.014406in}}%
\pgfpathlineto{\pgfqpoint{1.828980in}{4.012979in}}%
\pgfpathlineto{\pgfqpoint{1.853628in}{4.011591in}}%
\pgfpathlineto{\pgfqpoint{1.878277in}{4.010242in}}%
\pgfpathlineto{\pgfqpoint{1.902926in}{4.008931in}}%
\pgfpathlineto{\pgfqpoint{1.927575in}{4.007657in}}%
\pgfpathlineto{\pgfqpoint{1.952224in}{4.006422in}}%
\pgfpathlineto{\pgfqpoint{1.976873in}{4.005224in}}%
\pgfpathlineto{\pgfqpoint{2.001521in}{4.004063in}}%
\pgfpathlineto{\pgfqpoint{2.026170in}{4.002939in}}%
\pgfpathlineto{\pgfqpoint{2.050819in}{4.001852in}}%
\pgfpathlineto{\pgfqpoint{2.075468in}{4.000801in}}%
\pgfpathlineto{\pgfqpoint{2.100117in}{3.999787in}}%
\pgfpathlineto{\pgfqpoint{2.124765in}{3.998809in}}%
\pgfpathlineto{\pgfqpoint{2.149414in}{3.997867in}}%
\pgfpathlineto{\pgfqpoint{2.174063in}{3.996961in}}%
\pgfpathlineto{\pgfqpoint{2.198712in}{3.996090in}}%
\pgfpathlineto{\pgfqpoint{2.223361in}{3.995255in}}%
\pgfpathlineto{\pgfqpoint{2.248010in}{3.994454in}}%
\pgfpathlineto{\pgfqpoint{2.272658in}{3.993689in}}%
\pgfpathlineto{\pgfqpoint{2.297307in}{3.992959in}}%
\pgfpathlineto{\pgfqpoint{2.321956in}{3.992263in}}%
\pgfpathlineto{\pgfqpoint{2.346605in}{3.991602in}}%
\pgfpathlineto{\pgfqpoint{2.371254in}{3.990975in}}%
\pgfpathlineto{\pgfqpoint{2.395902in}{3.990382in}}%
\pgfpathlineto{\pgfqpoint{2.420551in}{3.989824in}}%
\pgfpathlineto{\pgfqpoint{2.445200in}{3.989299in}}%
\pgfpathlineto{\pgfqpoint{2.469849in}{3.988808in}}%
\pgfpathlineto{\pgfqpoint{2.494498in}{3.988351in}}%
\pgfpathlineto{\pgfqpoint{2.519147in}{3.987927in}}%
\pgfpathlineto{\pgfqpoint{2.543795in}{3.987536in}}%
\pgfpathlineto{\pgfqpoint{2.568444in}{3.987179in}}%
\pgfpathlineto{\pgfqpoint{2.593093in}{3.986856in}}%
\pgfpathlineto{\pgfqpoint{2.617742in}{3.986565in}}%
\pgfpathlineto{\pgfqpoint{2.642391in}{3.986308in}}%
\pgfpathlineto{\pgfqpoint{2.667039in}{3.986083in}}%
\pgfpathlineto{\pgfqpoint{2.691688in}{3.985892in}}%
\pgfusepath{stroke}%
\end{pgfscope}%
\begin{pgfscope}%
\pgfpathrectangle{\pgfqpoint{0.880000in}{2.849537in}}{\pgfqpoint{1.897959in}{1.372727in}} %
\pgfusepath{clip}%
\pgfsetbuttcap%
\pgfsetmiterjoin%
\definecolor{currentfill}{rgb}{1.000000,0.000000,0.000000}%
\pgfsetfillcolor{currentfill}%
\pgfsetlinewidth{1.003750pt}%
\definecolor{currentstroke}{rgb}{1.000000,0.000000,0.000000}%
\pgfsetstrokecolor{currentstroke}%
\pgfsetdash{}{0pt}%
\pgfsys@defobject{currentmarker}{\pgfqpoint{-0.041667in}{-0.041667in}}{\pgfqpoint{0.041667in}{0.041667in}}{%
\pgfpathmoveto{\pgfqpoint{-0.041667in}{-0.041667in}}%
\pgfpathlineto{\pgfqpoint{0.041667in}{-0.041667in}}%
\pgfpathlineto{\pgfqpoint{0.041667in}{0.041667in}}%
\pgfpathlineto{\pgfqpoint{-0.041667in}{0.041667in}}%
\pgfpathclose%
\pgfusepath{stroke,fill}%
}%
\begin{pgfscope}%
\pgfsys@transformshift{0.966271in}{4.087056in}%
\pgfsys@useobject{currentmarker}{}%
\end{pgfscope}%
\begin{pgfscope}%
\pgfsys@transformshift{1.311354in}{4.051361in}%
\pgfsys@useobject{currentmarker}{}%
\end{pgfscope}%
\begin{pgfscope}%
\pgfsys@transformshift{1.656438in}{4.023788in}%
\pgfsys@useobject{currentmarker}{}%
\end{pgfscope}%
\begin{pgfscope}%
\pgfsys@transformshift{2.001521in}{4.004063in}%
\pgfsys@useobject{currentmarker}{}%
\end{pgfscope}%
\begin{pgfscope}%
\pgfsys@transformshift{2.346605in}{3.991602in}%
\pgfsys@useobject{currentmarker}{}%
\end{pgfscope}%
\begin{pgfscope}%
\pgfsys@transformshift{2.691688in}{3.985892in}%
\pgfsys@useobject{currentmarker}{}%
\end{pgfscope}%
\end{pgfscope}%
\begin{pgfscope}%
\pgfpathrectangle{\pgfqpoint{0.880000in}{2.849537in}}{\pgfqpoint{1.897959in}{1.372727in}} %
\pgfusepath{clip}%
\pgfsetrectcap%
\pgfsetroundjoin%
\pgfsetlinewidth{1.505625pt}%
\definecolor{currentstroke}{rgb}{0.000000,0.000000,1.000000}%
\pgfsetstrokecolor{currentstroke}%
\pgfsetdash{}{0pt}%
\pgfpathmoveto{\pgfqpoint{0.966271in}{3.541127in}}%
\pgfpathlineto{\pgfqpoint{0.990920in}{3.535088in}}%
\pgfpathlineto{\pgfqpoint{1.015569in}{3.529271in}}%
\pgfpathlineto{\pgfqpoint{1.040217in}{3.523620in}}%
\pgfpathlineto{\pgfqpoint{1.064866in}{3.518096in}}%
\pgfpathlineto{\pgfqpoint{1.089515in}{3.512667in}}%
\pgfpathlineto{\pgfqpoint{1.114164in}{3.507313in}}%
\pgfpathlineto{\pgfqpoint{1.138813in}{3.502015in}}%
\pgfpathlineto{\pgfqpoint{1.163461in}{3.496761in}}%
\pgfpathlineto{\pgfqpoint{1.188110in}{3.491538in}}%
\pgfpathlineto{\pgfqpoint{1.212759in}{3.486338in}}%
\pgfpathlineto{\pgfqpoint{1.237408in}{3.481153in}}%
\pgfpathlineto{\pgfqpoint{1.262057in}{3.475975in}}%
\pgfpathlineto{\pgfqpoint{1.286706in}{3.470799in}}%
\pgfpathlineto{\pgfqpoint{1.311354in}{3.465619in}}%
\pgfpathlineto{\pgfqpoint{1.336003in}{3.460431in}}%
\pgfpathlineto{\pgfqpoint{1.360652in}{3.455229in}}%
\pgfpathlineto{\pgfqpoint{1.385301in}{3.450009in}}%
\pgfpathlineto{\pgfqpoint{1.409950in}{3.444768in}}%
\pgfpathlineto{\pgfqpoint{1.434598in}{3.439501in}}%
\pgfpathlineto{\pgfqpoint{1.459247in}{3.434204in}}%
\pgfpathlineto{\pgfqpoint{1.483896in}{3.428874in}}%
\pgfpathlineto{\pgfqpoint{1.508545in}{3.423506in}}%
\pgfpathlineto{\pgfqpoint{1.533194in}{3.418098in}}%
\pgfpathlineto{\pgfqpoint{1.557843in}{3.412644in}}%
\pgfpathlineto{\pgfqpoint{1.582491in}{3.407142in}}%
\pgfpathlineto{\pgfqpoint{1.607140in}{3.401587in}}%
\pgfpathlineto{\pgfqpoint{1.631789in}{3.395975in}}%
\pgfpathlineto{\pgfqpoint{1.656438in}{3.390302in}}%
\pgfpathlineto{\pgfqpoint{1.681087in}{3.384563in}}%
\pgfpathlineto{\pgfqpoint{1.705735in}{3.378755in}}%
\pgfpathlineto{\pgfqpoint{1.730384in}{3.372871in}}%
\pgfpathlineto{\pgfqpoint{1.755033in}{3.366907in}}%
\pgfpathlineto{\pgfqpoint{1.779682in}{3.360859in}}%
\pgfpathlineto{\pgfqpoint{1.804331in}{3.354719in}}%
\pgfpathlineto{\pgfqpoint{1.828980in}{3.348482in}}%
\pgfpathlineto{\pgfqpoint{1.853628in}{3.342142in}}%
\pgfpathlineto{\pgfqpoint{1.878277in}{3.335692in}}%
\pgfpathlineto{\pgfqpoint{1.902926in}{3.329126in}}%
\pgfpathlineto{\pgfqpoint{1.927575in}{3.322434in}}%
\pgfpathlineto{\pgfqpoint{1.952224in}{3.315609in}}%
\pgfpathlineto{\pgfqpoint{1.976873in}{3.308641in}}%
\pgfpathlineto{\pgfqpoint{2.001521in}{3.301522in}}%
\pgfpathlineto{\pgfqpoint{2.026170in}{3.294241in}}%
\pgfpathlineto{\pgfqpoint{2.050819in}{3.286785in}}%
\pgfpathlineto{\pgfqpoint{2.075468in}{3.279143in}}%
\pgfpathlineto{\pgfqpoint{2.100117in}{3.271301in}}%
\pgfpathlineto{\pgfqpoint{2.124765in}{3.263243in}}%
\pgfpathlineto{\pgfqpoint{2.149414in}{3.254954in}}%
\pgfpathlineto{\pgfqpoint{2.174063in}{3.246414in}}%
\pgfpathlineto{\pgfqpoint{2.198712in}{3.237603in}}%
\pgfpathlineto{\pgfqpoint{2.223361in}{3.228498in}}%
\pgfpathlineto{\pgfqpoint{2.248010in}{3.219074in}}%
\pgfpathlineto{\pgfqpoint{2.272658in}{3.209301in}}%
\pgfpathlineto{\pgfqpoint{2.297307in}{3.199146in}}%
\pgfpathlineto{\pgfqpoint{2.321956in}{3.188572in}}%
\pgfpathlineto{\pgfqpoint{2.346605in}{3.177536in}}%
\pgfpathlineto{\pgfqpoint{2.371254in}{3.165988in}}%
\pgfpathlineto{\pgfqpoint{2.395902in}{3.153869in}}%
\pgfpathlineto{\pgfqpoint{2.420551in}{3.141112in}}%
\pgfpathlineto{\pgfqpoint{2.445200in}{3.127635in}}%
\pgfpathlineto{\pgfqpoint{2.469849in}{3.113341in}}%
\pgfpathlineto{\pgfqpoint{2.494498in}{3.098112in}}%
\pgfpathlineto{\pgfqpoint{2.519147in}{3.081802in}}%
\pgfpathlineto{\pgfqpoint{2.543795in}{3.064230in}}%
\pgfpathlineto{\pgfqpoint{2.568444in}{3.045165in}}%
\pgfpathlineto{\pgfqpoint{2.593093in}{3.024304in}}%
\pgfpathlineto{\pgfqpoint{2.617742in}{3.001245in}}%
\pgfpathlineto{\pgfqpoint{2.642391in}{2.975431in}}%
\pgfpathlineto{\pgfqpoint{2.667039in}{2.946063in}}%
\pgfpathlineto{\pgfqpoint{2.691688in}{2.911933in}}%
\pgfusepath{stroke}%
\end{pgfscope}%
\begin{pgfscope}%
\pgfpathrectangle{\pgfqpoint{0.880000in}{2.849537in}}{\pgfqpoint{1.897959in}{1.372727in}} %
\pgfusepath{clip}%
\pgfsetbuttcap%
\pgfsetroundjoin%
\definecolor{currentfill}{rgb}{0.000000,0.000000,1.000000}%
\pgfsetfillcolor{currentfill}%
\pgfsetlinewidth{1.003750pt}%
\definecolor{currentstroke}{rgb}{0.000000,0.000000,1.000000}%
\pgfsetstrokecolor{currentstroke}%
\pgfsetdash{}{0pt}%
\pgfsys@defobject{currentmarker}{\pgfqpoint{-0.041667in}{-0.041667in}}{\pgfqpoint{0.041667in}{0.041667in}}{%
\pgfpathmoveto{\pgfqpoint{0.000000in}{-0.041667in}}%
\pgfpathcurveto{\pgfqpoint{0.011050in}{-0.041667in}}{\pgfqpoint{0.021649in}{-0.037276in}}{\pgfqpoint{0.029463in}{-0.029463in}}%
\pgfpathcurveto{\pgfqpoint{0.037276in}{-0.021649in}}{\pgfqpoint{0.041667in}{-0.011050in}}{\pgfqpoint{0.041667in}{0.000000in}}%
\pgfpathcurveto{\pgfqpoint{0.041667in}{0.011050in}}{\pgfqpoint{0.037276in}{0.021649in}}{\pgfqpoint{0.029463in}{0.029463in}}%
\pgfpathcurveto{\pgfqpoint{0.021649in}{0.037276in}}{\pgfqpoint{0.011050in}{0.041667in}}{\pgfqpoint{0.000000in}{0.041667in}}%
\pgfpathcurveto{\pgfqpoint{-0.011050in}{0.041667in}}{\pgfqpoint{-0.021649in}{0.037276in}}{\pgfqpoint{-0.029463in}{0.029463in}}%
\pgfpathcurveto{\pgfqpoint{-0.037276in}{0.021649in}}{\pgfqpoint{-0.041667in}{0.011050in}}{\pgfqpoint{-0.041667in}{0.000000in}}%
\pgfpathcurveto{\pgfqpoint{-0.041667in}{-0.011050in}}{\pgfqpoint{-0.037276in}{-0.021649in}}{\pgfqpoint{-0.029463in}{-0.029463in}}%
\pgfpathcurveto{\pgfqpoint{-0.021649in}{-0.037276in}}{\pgfqpoint{-0.011050in}{-0.041667in}}{\pgfqpoint{0.000000in}{-0.041667in}}%
\pgfpathclose%
\pgfusepath{stroke,fill}%
}%
\begin{pgfscope}%
\pgfsys@transformshift{0.966271in}{3.541127in}%
\pgfsys@useobject{currentmarker}{}%
\end{pgfscope}%
\begin{pgfscope}%
\pgfsys@transformshift{1.311354in}{3.465619in}%
\pgfsys@useobject{currentmarker}{}%
\end{pgfscope}%
\begin{pgfscope}%
\pgfsys@transformshift{1.656438in}{3.390302in}%
\pgfsys@useobject{currentmarker}{}%
\end{pgfscope}%
\begin{pgfscope}%
\pgfsys@transformshift{2.001521in}{3.301522in}%
\pgfsys@useobject{currentmarker}{}%
\end{pgfscope}%
\begin{pgfscope}%
\pgfsys@transformshift{2.346605in}{3.177536in}%
\pgfsys@useobject{currentmarker}{}%
\end{pgfscope}%
\begin{pgfscope}%
\pgfsys@transformshift{2.691688in}{2.911933in}%
\pgfsys@useobject{currentmarker}{}%
\end{pgfscope}%
\end{pgfscope}%
\begin{pgfscope}%
\pgfpathrectangle{\pgfqpoint{0.880000in}{2.849537in}}{\pgfqpoint{1.897959in}{1.372727in}} %
\pgfusepath{clip}%
\pgfsetbuttcap%
\pgfsetroundjoin%
\pgfsetlinewidth{1.505625pt}%
\definecolor{currentstroke}{rgb}{0.000000,0.750000,0.750000}%
\pgfsetstrokecolor{currentstroke}%
\pgfsetdash{{9.600000pt}{2.400000pt}{1.500000pt}{2.400000pt}}{0.000000pt}%
\pgfpathmoveto{\pgfqpoint{0.966271in}{3.949557in}}%
\pgfpathlineto{\pgfqpoint{0.990920in}{3.916184in}}%
\pgfpathlineto{\pgfqpoint{1.015569in}{3.886378in}}%
\pgfpathlineto{\pgfqpoint{1.040217in}{3.859496in}}%
\pgfpathlineto{\pgfqpoint{1.064866in}{3.835055in}}%
\pgfpathlineto{\pgfqpoint{1.089515in}{3.812682in}}%
\pgfpathlineto{\pgfqpoint{1.114164in}{3.792087in}}%
\pgfpathlineto{\pgfqpoint{1.138813in}{3.773036in}}%
\pgfpathlineto{\pgfqpoint{1.163461in}{3.755338in}}%
\pgfpathlineto{\pgfqpoint{1.188110in}{3.738837in}}%
\pgfpathlineto{\pgfqpoint{1.212759in}{3.723401in}}%
\pgfpathlineto{\pgfqpoint{1.237408in}{3.708921in}}%
\pgfpathlineto{\pgfqpoint{1.262057in}{3.695303in}}%
\pgfpathlineto{\pgfqpoint{1.286706in}{3.682467in}}%
\pgfpathlineto{\pgfqpoint{1.311354in}{3.670342in}}%
\pgfpathlineto{\pgfqpoint{1.336003in}{3.658868in}}%
\pgfpathlineto{\pgfqpoint{1.360652in}{3.647993in}}%
\pgfpathlineto{\pgfqpoint{1.385301in}{3.637669in}}%
\pgfpathlineto{\pgfqpoint{1.409950in}{3.627855in}}%
\pgfpathlineto{\pgfqpoint{1.434598in}{3.618515in}}%
\pgfpathlineto{\pgfqpoint{1.459247in}{3.609616in}}%
\pgfpathlineto{\pgfqpoint{1.483896in}{3.601129in}}%
\pgfpathlineto{\pgfqpoint{1.508545in}{3.593026in}}%
\pgfpathlineto{\pgfqpoint{1.533194in}{3.585284in}}%
\pgfpathlineto{\pgfqpoint{1.557843in}{3.577882in}}%
\pgfpathlineto{\pgfqpoint{1.582491in}{3.570799in}}%
\pgfpathlineto{\pgfqpoint{1.607140in}{3.564019in}}%
\pgfpathlineto{\pgfqpoint{1.631789in}{3.557524in}}%
\pgfpathlineto{\pgfqpoint{1.656438in}{3.551300in}}%
\pgfpathlineto{\pgfqpoint{1.681087in}{3.545332in}}%
\pgfpathlineto{\pgfqpoint{1.705735in}{3.539609in}}%
\pgfpathlineto{\pgfqpoint{1.730384in}{3.534119in}}%
\pgfpathlineto{\pgfqpoint{1.755033in}{3.528850in}}%
\pgfpathlineto{\pgfqpoint{1.779682in}{3.523794in}}%
\pgfpathlineto{\pgfqpoint{1.804331in}{3.518940in}}%
\pgfpathlineto{\pgfqpoint{1.828980in}{3.514281in}}%
\pgfpathlineto{\pgfqpoint{1.853628in}{3.509809in}}%
\pgfpathlineto{\pgfqpoint{1.878277in}{3.505515in}}%
\pgfpathlineto{\pgfqpoint{1.902926in}{3.501394in}}%
\pgfpathlineto{\pgfqpoint{1.927575in}{3.497438in}}%
\pgfpathlineto{\pgfqpoint{1.952224in}{3.493643in}}%
\pgfpathlineto{\pgfqpoint{1.976873in}{3.490001in}}%
\pgfpathlineto{\pgfqpoint{2.001521in}{3.486510in}}%
\pgfpathlineto{\pgfqpoint{2.026170in}{3.483162in}}%
\pgfpathlineto{\pgfqpoint{2.050819in}{3.479954in}}%
\pgfpathlineto{\pgfqpoint{2.075468in}{3.476882in}}%
\pgfpathlineto{\pgfqpoint{2.100117in}{3.473942in}}%
\pgfpathlineto{\pgfqpoint{2.124765in}{3.471129in}}%
\pgfpathlineto{\pgfqpoint{2.149414in}{3.468441in}}%
\pgfpathlineto{\pgfqpoint{2.174063in}{3.465874in}}%
\pgfpathlineto{\pgfqpoint{2.198712in}{3.463425in}}%
\pgfpathlineto{\pgfqpoint{2.223361in}{3.461091in}}%
\pgfpathlineto{\pgfqpoint{2.248010in}{3.458870in}}%
\pgfpathlineto{\pgfqpoint{2.272658in}{3.456759in}}%
\pgfpathlineto{\pgfqpoint{2.297307in}{3.454755in}}%
\pgfpathlineto{\pgfqpoint{2.321956in}{3.452857in}}%
\pgfpathlineto{\pgfqpoint{2.346605in}{3.451062in}}%
\pgfpathlineto{\pgfqpoint{2.371254in}{3.449368in}}%
\pgfpathlineto{\pgfqpoint{2.395902in}{3.447774in}}%
\pgfpathlineto{\pgfqpoint{2.420551in}{3.446278in}}%
\pgfpathlineto{\pgfqpoint{2.445200in}{3.444879in}}%
\pgfpathlineto{\pgfqpoint{2.469849in}{3.443574in}}%
\pgfpathlineto{\pgfqpoint{2.494498in}{3.442363in}}%
\pgfpathlineto{\pgfqpoint{2.519147in}{3.441245in}}%
\pgfpathlineto{\pgfqpoint{2.543795in}{3.440218in}}%
\pgfpathlineto{\pgfqpoint{2.568444in}{3.439281in}}%
\pgfpathlineto{\pgfqpoint{2.593093in}{3.438433in}}%
\pgfpathlineto{\pgfqpoint{2.617742in}{3.437674in}}%
\pgfpathlineto{\pgfqpoint{2.642391in}{3.437003in}}%
\pgfpathlineto{\pgfqpoint{2.667039in}{3.436418in}}%
\pgfpathlineto{\pgfqpoint{2.691688in}{3.435921in}}%
\pgfusepath{stroke}%
\end{pgfscope}%
\begin{pgfscope}%
\pgfpathrectangle{\pgfqpoint{0.880000in}{2.849537in}}{\pgfqpoint{1.897959in}{1.372727in}} %
\pgfusepath{clip}%
\pgfsetbuttcap%
\pgfsetmiterjoin%
\definecolor{currentfill}{rgb}{0.000000,0.750000,0.750000}%
\pgfsetfillcolor{currentfill}%
\pgfsetlinewidth{1.003750pt}%
\definecolor{currentstroke}{rgb}{0.000000,0.750000,0.750000}%
\pgfsetstrokecolor{currentstroke}%
\pgfsetdash{}{0pt}%
\pgfsys@defobject{currentmarker}{\pgfqpoint{-0.041667in}{-0.041667in}}{\pgfqpoint{0.041667in}{0.041667in}}{%
\pgfpathmoveto{\pgfqpoint{-0.000000in}{-0.041667in}}%
\pgfpathlineto{\pgfqpoint{0.041667in}{0.041667in}}%
\pgfpathlineto{\pgfqpoint{-0.041667in}{0.041667in}}%
\pgfpathclose%
\pgfusepath{stroke,fill}%
}%
\begin{pgfscope}%
\pgfsys@transformshift{0.966271in}{3.949557in}%
\pgfsys@useobject{currentmarker}{}%
\end{pgfscope}%
\begin{pgfscope}%
\pgfsys@transformshift{1.311354in}{3.670342in}%
\pgfsys@useobject{currentmarker}{}%
\end{pgfscope}%
\begin{pgfscope}%
\pgfsys@transformshift{1.656438in}{3.551300in}%
\pgfsys@useobject{currentmarker}{}%
\end{pgfscope}%
\begin{pgfscope}%
\pgfsys@transformshift{2.001521in}{3.486510in}%
\pgfsys@useobject{currentmarker}{}%
\end{pgfscope}%
\begin{pgfscope}%
\pgfsys@transformshift{2.346605in}{3.451062in}%
\pgfsys@useobject{currentmarker}{}%
\end{pgfscope}%
\begin{pgfscope}%
\pgfsys@transformshift{2.691688in}{3.435921in}%
\pgfsys@useobject{currentmarker}{}%
\end{pgfscope}%
\end{pgfscope}%
\begin{pgfscope}%
\pgfpathrectangle{\pgfqpoint{0.880000in}{2.849537in}}{\pgfqpoint{1.897959in}{1.372727in}} %
\pgfusepath{clip}%
\pgfsetbuttcap%
\pgfsetroundjoin%
\pgfsetlinewidth{1.505625pt}%
\definecolor{currentstroke}{rgb}{0.000000,0.000000,0.000000}%
\pgfsetstrokecolor{currentstroke}%
\pgfsetdash{{1.500000pt}{2.475000pt}}{0.000000pt}%
\pgfpathmoveto{\pgfqpoint{0.966271in}{4.159867in}}%
\pgfpathlineto{\pgfqpoint{0.990920in}{4.149431in}}%
\pgfpathlineto{\pgfqpoint{1.015569in}{4.139488in}}%
\pgfpathlineto{\pgfqpoint{1.040217in}{4.131002in}}%
\pgfpathlineto{\pgfqpoint{1.064866in}{4.123612in}}%
\pgfpathlineto{\pgfqpoint{1.089515in}{4.117068in}}%
\pgfpathlineto{\pgfqpoint{1.114164in}{4.111189in}}%
\pgfpathlineto{\pgfqpoint{1.138813in}{4.105843in}}%
\pgfpathlineto{\pgfqpoint{1.163461in}{4.100931in}}%
\pgfpathlineto{\pgfqpoint{1.188110in}{4.096377in}}%
\pgfpathlineto{\pgfqpoint{1.212759in}{4.092124in}}%
\pgfpathlineto{\pgfqpoint{1.237408in}{4.088127in}}%
\pgfpathlineto{\pgfqpoint{1.262057in}{4.084352in}}%
\pgfpathlineto{\pgfqpoint{1.286706in}{4.080768in}}%
\pgfpathlineto{\pgfqpoint{1.311354in}{4.077354in}}%
\pgfpathlineto{\pgfqpoint{1.336003in}{4.074090in}}%
\pgfpathlineto{\pgfqpoint{1.360652in}{4.070962in}}%
\pgfpathlineto{\pgfqpoint{1.385301in}{4.067956in}}%
\pgfpathlineto{\pgfqpoint{1.409950in}{4.065063in}}%
\pgfpathlineto{\pgfqpoint{1.434598in}{4.062272in}}%
\pgfpathlineto{\pgfqpoint{1.459247in}{4.059577in}}%
\pgfpathlineto{\pgfqpoint{1.483896in}{4.056970in}}%
\pgfpathlineto{\pgfqpoint{1.508545in}{4.054447in}}%
\pgfpathlineto{\pgfqpoint{1.533194in}{4.052003in}}%
\pgfpathlineto{\pgfqpoint{1.557843in}{4.049633in}}%
\pgfpathlineto{\pgfqpoint{1.582491in}{4.047334in}}%
\pgfpathlineto{\pgfqpoint{1.607140in}{4.045102in}}%
\pgfpathlineto{\pgfqpoint{1.631789in}{4.042935in}}%
\pgfpathlineto{\pgfqpoint{1.656438in}{4.040831in}}%
\pgfpathlineto{\pgfqpoint{1.681087in}{4.038786in}}%
\pgfpathlineto{\pgfqpoint{1.705735in}{4.036799in}}%
\pgfpathlineto{\pgfqpoint{1.730384in}{4.034869in}}%
\pgfpathlineto{\pgfqpoint{1.755033in}{4.032993in}}%
\pgfpathlineto{\pgfqpoint{1.779682in}{4.031171in}}%
\pgfpathlineto{\pgfqpoint{1.804331in}{4.029400in}}%
\pgfpathlineto{\pgfqpoint{1.828980in}{4.027680in}}%
\pgfpathlineto{\pgfqpoint{1.853628in}{4.026010in}}%
\pgfpathlineto{\pgfqpoint{1.878277in}{4.024388in}}%
\pgfpathlineto{\pgfqpoint{1.902926in}{4.022813in}}%
\pgfpathlineto{\pgfqpoint{1.927575in}{4.021285in}}%
\pgfpathlineto{\pgfqpoint{1.952224in}{4.019804in}}%
\pgfpathlineto{\pgfqpoint{1.976873in}{4.018367in}}%
\pgfpathlineto{\pgfqpoint{2.001521in}{4.016975in}}%
\pgfpathlineto{\pgfqpoint{2.026170in}{4.015626in}}%
\pgfpathlineto{\pgfqpoint{2.050819in}{4.014321in}}%
\pgfpathlineto{\pgfqpoint{2.075468in}{4.013059in}}%
\pgfpathlineto{\pgfqpoint{2.100117in}{4.011839in}}%
\pgfpathlineto{\pgfqpoint{2.124765in}{4.010660in}}%
\pgfpathlineto{\pgfqpoint{2.149414in}{4.009523in}}%
\pgfpathlineto{\pgfqpoint{2.174063in}{4.008427in}}%
\pgfpathlineto{\pgfqpoint{2.198712in}{4.007371in}}%
\pgfpathlineto{\pgfqpoint{2.223361in}{4.006356in}}%
\pgfpathlineto{\pgfqpoint{2.248010in}{4.005380in}}%
\pgfpathlineto{\pgfqpoint{2.272658in}{4.004444in}}%
\pgfpathlineto{\pgfqpoint{2.297307in}{4.003547in}}%
\pgfpathlineto{\pgfqpoint{2.321956in}{4.002689in}}%
\pgfpathlineto{\pgfqpoint{2.346605in}{4.001869in}}%
\pgfpathlineto{\pgfqpoint{2.371254in}{4.001088in}}%
\pgfpathlineto{\pgfqpoint{2.395902in}{4.000345in}}%
\pgfpathlineto{\pgfqpoint{2.420551in}{3.999641in}}%
\pgfpathlineto{\pgfqpoint{2.445200in}{3.998974in}}%
\pgfpathlineto{\pgfqpoint{2.469849in}{3.998345in}}%
\pgfpathlineto{\pgfqpoint{2.494498in}{3.997754in}}%
\pgfpathlineto{\pgfqpoint{2.519147in}{3.997200in}}%
\pgfpathlineto{\pgfqpoint{2.543795in}{3.996683in}}%
\pgfpathlineto{\pgfqpoint{2.568444in}{3.996204in}}%
\pgfpathlineto{\pgfqpoint{2.593093in}{3.995762in}}%
\pgfpathlineto{\pgfqpoint{2.617742in}{3.995358in}}%
\pgfpathlineto{\pgfqpoint{2.642391in}{3.994991in}}%
\pgfpathlineto{\pgfqpoint{2.667039in}{3.994661in}}%
\pgfpathlineto{\pgfqpoint{2.691688in}{3.994368in}}%
\pgfusepath{stroke}%
\end{pgfscope}%
\begin{pgfscope}%
\pgfpathrectangle{\pgfqpoint{0.880000in}{2.849537in}}{\pgfqpoint{1.897959in}{1.372727in}} %
\pgfusepath{clip}%
\pgfsetbuttcap%
\pgfsetroundjoin%
\definecolor{currentfill}{rgb}{0.000000,0.000000,0.000000}%
\pgfsetfillcolor{currentfill}%
\pgfsetlinewidth{1.003750pt}%
\definecolor{currentstroke}{rgb}{0.000000,0.000000,0.000000}%
\pgfsetstrokecolor{currentstroke}%
\pgfsetdash{}{0pt}%
\pgfsys@defobject{currentmarker}{\pgfqpoint{-0.041667in}{-0.041667in}}{\pgfqpoint{0.041667in}{0.041667in}}{%
\pgfpathmoveto{\pgfqpoint{-0.041667in}{0.000000in}}%
\pgfpathlineto{\pgfqpoint{0.041667in}{0.000000in}}%
\pgfpathmoveto{\pgfqpoint{0.000000in}{-0.041667in}}%
\pgfpathlineto{\pgfqpoint{0.000000in}{0.041667in}}%
\pgfusepath{stroke,fill}%
}%
\begin{pgfscope}%
\pgfsys@transformshift{0.966271in}{4.159867in}%
\pgfsys@useobject{currentmarker}{}%
\end{pgfscope}%
\begin{pgfscope}%
\pgfsys@transformshift{1.311354in}{4.077354in}%
\pgfsys@useobject{currentmarker}{}%
\end{pgfscope}%
\begin{pgfscope}%
\pgfsys@transformshift{1.656438in}{4.040831in}%
\pgfsys@useobject{currentmarker}{}%
\end{pgfscope}%
\begin{pgfscope}%
\pgfsys@transformshift{2.001521in}{4.016975in}%
\pgfsys@useobject{currentmarker}{}%
\end{pgfscope}%
\begin{pgfscope}%
\pgfsys@transformshift{2.346605in}{4.001869in}%
\pgfsys@useobject{currentmarker}{}%
\end{pgfscope}%
\begin{pgfscope}%
\pgfsys@transformshift{2.691688in}{3.994368in}%
\pgfsys@useobject{currentmarker}{}%
\end{pgfscope}%
\end{pgfscope}%
\begin{pgfscope}%
\pgfsetrectcap%
\pgfsetmiterjoin%
\pgfsetlinewidth{0.803000pt}%
\definecolor{currentstroke}{rgb}{0.000000,0.000000,0.000000}%
\pgfsetstrokecolor{currentstroke}%
\pgfsetdash{}{0pt}%
\pgfpathmoveto{\pgfqpoint{0.880000in}{2.849537in}}%
\pgfpathlineto{\pgfqpoint{0.880000in}{4.222264in}}%
\pgfusepath{stroke}%
\end{pgfscope}%
\begin{pgfscope}%
\pgfsetrectcap%
\pgfsetmiterjoin%
\pgfsetlinewidth{0.803000pt}%
\definecolor{currentstroke}{rgb}{0.000000,0.000000,0.000000}%
\pgfsetstrokecolor{currentstroke}%
\pgfsetdash{}{0pt}%
\pgfpathmoveto{\pgfqpoint{2.777959in}{2.849537in}}%
\pgfpathlineto{\pgfqpoint{2.777959in}{4.222264in}}%
\pgfusepath{stroke}%
\end{pgfscope}%
\begin{pgfscope}%
\pgfsetrectcap%
\pgfsetmiterjoin%
\pgfsetlinewidth{0.803000pt}%
\definecolor{currentstroke}{rgb}{0.000000,0.000000,0.000000}%
\pgfsetstrokecolor{currentstroke}%
\pgfsetdash{}{0pt}%
\pgfpathmoveto{\pgfqpoint{0.880000in}{2.849537in}}%
\pgfpathlineto{\pgfqpoint{2.777959in}{2.849537in}}%
\pgfusepath{stroke}%
\end{pgfscope}%
\begin{pgfscope}%
\pgfsetrectcap%
\pgfsetmiterjoin%
\pgfsetlinewidth{0.803000pt}%
\definecolor{currentstroke}{rgb}{0.000000,0.000000,0.000000}%
\pgfsetstrokecolor{currentstroke}%
\pgfsetdash{}{0pt}%
\pgfpathmoveto{\pgfqpoint{0.880000in}{4.222264in}}%
\pgfpathlineto{\pgfqpoint{2.777959in}{4.222264in}}%
\pgfusepath{stroke}%
\end{pgfscope}%
\begin{pgfscope}%
\pgfsetbuttcap%
\pgfsetmiterjoin%
\definecolor{currentfill}{rgb}{1.000000,1.000000,1.000000}%
\pgfsetfillcolor{currentfill}%
\pgfsetlinewidth{0.000000pt}%
\definecolor{currentstroke}{rgb}{0.000000,0.000000,0.000000}%
\pgfsetstrokecolor{currentstroke}%
\pgfsetstrokeopacity{0.000000}%
\pgfsetdash{}{0pt}%
\pgfpathmoveto{\pgfqpoint{3.347347in}{2.849537in}}%
\pgfpathlineto{\pgfqpoint{5.245306in}{2.849537in}}%
\pgfpathlineto{\pgfqpoint{5.245306in}{4.222264in}}%
\pgfpathlineto{\pgfqpoint{3.347347in}{4.222264in}}%
\pgfpathclose%
\pgfusepath{fill}%
\end{pgfscope}%
\begin{pgfscope}%
\pgfsetbuttcap%
\pgfsetroundjoin%
\definecolor{currentfill}{rgb}{0.000000,0.000000,0.000000}%
\pgfsetfillcolor{currentfill}%
\pgfsetlinewidth{0.803000pt}%
\definecolor{currentstroke}{rgb}{0.000000,0.000000,0.000000}%
\pgfsetstrokecolor{currentstroke}%
\pgfsetdash{}{0pt}%
\pgfsys@defobject{currentmarker}{\pgfqpoint{0.000000in}{-0.048611in}}{\pgfqpoint{0.000000in}{0.000000in}}{%
\pgfpathmoveto{\pgfqpoint{0.000000in}{0.000000in}}%
\pgfpathlineto{\pgfqpoint{0.000000in}{-0.048611in}}%
\pgfusepath{stroke,fill}%
}%
\begin{pgfscope}%
\pgfsys@transformshift{4.033763in}{2.849537in}%
\pgfsys@useobject{currentmarker}{}%
\end{pgfscope}%
\end{pgfscope}%
\begin{pgfscope}%
\pgftext[x=4.033763in,y=2.752315in,,top]{\rmfamily\fontsize{10.000000}{12.000000}\selectfont \(\displaystyle 0.2\)}%
\end{pgfscope}%
\begin{pgfscope}%
\pgfsetbuttcap%
\pgfsetroundjoin%
\definecolor{currentfill}{rgb}{0.000000,0.000000,0.000000}%
\pgfsetfillcolor{currentfill}%
\pgfsetlinewidth{0.803000pt}%
\definecolor{currentstroke}{rgb}{0.000000,0.000000,0.000000}%
\pgfsetstrokecolor{currentstroke}%
\pgfsetdash{}{0pt}%
\pgfsys@defobject{currentmarker}{\pgfqpoint{0.000000in}{-0.048611in}}{\pgfqpoint{0.000000in}{0.000000in}}{%
\pgfpathmoveto{\pgfqpoint{0.000000in}{0.000000in}}%
\pgfpathlineto{\pgfqpoint{0.000000in}{-0.048611in}}%
\pgfusepath{stroke,fill}%
}%
\begin{pgfscope}%
\pgfsys@transformshift{4.933981in}{2.849537in}%
\pgfsys@useobject{currentmarker}{}%
\end{pgfscope}%
\end{pgfscope}%
\begin{pgfscope}%
\pgftext[x=4.933981in,y=2.752315in,,top]{\rmfamily\fontsize{10.000000}{12.000000}\selectfont \(\displaystyle 0.4\)}%
\end{pgfscope}%
\begin{pgfscope}%
\pgfsetbuttcap%
\pgfsetroundjoin%
\definecolor{currentfill}{rgb}{0.000000,0.000000,0.000000}%
\pgfsetfillcolor{currentfill}%
\pgfsetlinewidth{0.803000pt}%
\definecolor{currentstroke}{rgb}{0.000000,0.000000,0.000000}%
\pgfsetstrokecolor{currentstroke}%
\pgfsetdash{}{0pt}%
\pgfsys@defobject{currentmarker}{\pgfqpoint{-0.048611in}{0.000000in}}{\pgfqpoint{0.000000in}{0.000000in}}{%
\pgfpathmoveto{\pgfqpoint{0.000000in}{0.000000in}}%
\pgfpathlineto{\pgfqpoint{-0.048611in}{0.000000in}}%
\pgfusepath{stroke,fill}%
}%
\begin{pgfscope}%
\pgfsys@transformshift{3.347347in}{3.020671in}%
\pgfsys@useobject{currentmarker}{}%
\end{pgfscope}%
\end{pgfscope}%
\begin{pgfscope}%
\pgftext[x=2.962122in,y=2.967909in,left,base]{\rmfamily\fontsize{10.000000}{12.000000}\selectfont \(\displaystyle 10^{-7}\)}%
\end{pgfscope}%
\begin{pgfscope}%
\pgfsetbuttcap%
\pgfsetroundjoin%
\definecolor{currentfill}{rgb}{0.000000,0.000000,0.000000}%
\pgfsetfillcolor{currentfill}%
\pgfsetlinewidth{0.803000pt}%
\definecolor{currentstroke}{rgb}{0.000000,0.000000,0.000000}%
\pgfsetstrokecolor{currentstroke}%
\pgfsetdash{}{0pt}%
\pgfsys@defobject{currentmarker}{\pgfqpoint{-0.048611in}{0.000000in}}{\pgfqpoint{0.000000in}{0.000000in}}{%
\pgfpathmoveto{\pgfqpoint{0.000000in}{0.000000in}}%
\pgfpathlineto{\pgfqpoint{-0.048611in}{0.000000in}}%
\pgfusepath{stroke,fill}%
}%
\begin{pgfscope}%
\pgfsys@transformshift{3.347347in}{3.392614in}%
\pgfsys@useobject{currentmarker}{}%
\end{pgfscope}%
\end{pgfscope}%
\begin{pgfscope}%
\pgftext[x=2.962122in,y=3.339853in,left,base]{\rmfamily\fontsize{10.000000}{12.000000}\selectfont \(\displaystyle 10^{-6}\)}%
\end{pgfscope}%
\begin{pgfscope}%
\pgfsetbuttcap%
\pgfsetroundjoin%
\definecolor{currentfill}{rgb}{0.000000,0.000000,0.000000}%
\pgfsetfillcolor{currentfill}%
\pgfsetlinewidth{0.803000pt}%
\definecolor{currentstroke}{rgb}{0.000000,0.000000,0.000000}%
\pgfsetstrokecolor{currentstroke}%
\pgfsetdash{}{0pt}%
\pgfsys@defobject{currentmarker}{\pgfqpoint{-0.048611in}{0.000000in}}{\pgfqpoint{0.000000in}{0.000000in}}{%
\pgfpathmoveto{\pgfqpoint{0.000000in}{0.000000in}}%
\pgfpathlineto{\pgfqpoint{-0.048611in}{0.000000in}}%
\pgfusepath{stroke,fill}%
}%
\begin{pgfscope}%
\pgfsys@transformshift{3.347347in}{3.764558in}%
\pgfsys@useobject{currentmarker}{}%
\end{pgfscope}%
\end{pgfscope}%
\begin{pgfscope}%
\pgftext[x=2.962122in,y=3.711796in,left,base]{\rmfamily\fontsize{10.000000}{12.000000}\selectfont \(\displaystyle 10^{-5}\)}%
\end{pgfscope}%
\begin{pgfscope}%
\pgfsetbuttcap%
\pgfsetroundjoin%
\definecolor{currentfill}{rgb}{0.000000,0.000000,0.000000}%
\pgfsetfillcolor{currentfill}%
\pgfsetlinewidth{0.803000pt}%
\definecolor{currentstroke}{rgb}{0.000000,0.000000,0.000000}%
\pgfsetstrokecolor{currentstroke}%
\pgfsetdash{}{0pt}%
\pgfsys@defobject{currentmarker}{\pgfqpoint{-0.048611in}{0.000000in}}{\pgfqpoint{0.000000in}{0.000000in}}{%
\pgfpathmoveto{\pgfqpoint{0.000000in}{0.000000in}}%
\pgfpathlineto{\pgfqpoint{-0.048611in}{0.000000in}}%
\pgfusepath{stroke,fill}%
}%
\begin{pgfscope}%
\pgfsys@transformshift{3.347347in}{4.136501in}%
\pgfsys@useobject{currentmarker}{}%
\end{pgfscope}%
\end{pgfscope}%
\begin{pgfscope}%
\pgftext[x=2.962122in,y=4.083740in,left,base]{\rmfamily\fontsize{10.000000}{12.000000}\selectfont \(\displaystyle 10^{-4}\)}%
\end{pgfscope}%
\begin{pgfscope}%
\pgfsetbuttcap%
\pgfsetroundjoin%
\definecolor{currentfill}{rgb}{0.000000,0.000000,0.000000}%
\pgfsetfillcolor{currentfill}%
\pgfsetlinewidth{0.602250pt}%
\definecolor{currentstroke}{rgb}{0.000000,0.000000,0.000000}%
\pgfsetstrokecolor{currentstroke}%
\pgfsetdash{}{0pt}%
\pgfsys@defobject{currentmarker}{\pgfqpoint{-0.027778in}{0.000000in}}{\pgfqpoint{0.000000in}{0.000000in}}{%
\pgfpathmoveto{\pgfqpoint{0.000000in}{0.000000in}}%
\pgfpathlineto{\pgfqpoint{-0.027778in}{0.000000in}}%
\pgfusepath{stroke,fill}%
}%
\begin{pgfscope}%
\pgfsys@transformshift{3.347347in}{2.872659in}%
\pgfsys@useobject{currentmarker}{}%
\end{pgfscope}%
\end{pgfscope}%
\begin{pgfscope}%
\pgfsetbuttcap%
\pgfsetroundjoin%
\definecolor{currentfill}{rgb}{0.000000,0.000000,0.000000}%
\pgfsetfillcolor{currentfill}%
\pgfsetlinewidth{0.602250pt}%
\definecolor{currentstroke}{rgb}{0.000000,0.000000,0.000000}%
\pgfsetstrokecolor{currentstroke}%
\pgfsetdash{}{0pt}%
\pgfsys@defobject{currentmarker}{\pgfqpoint{-0.027778in}{0.000000in}}{\pgfqpoint{0.000000in}{0.000000in}}{%
\pgfpathmoveto{\pgfqpoint{0.000000in}{0.000000in}}%
\pgfpathlineto{\pgfqpoint{-0.027778in}{0.000000in}}%
\pgfusepath{stroke,fill}%
}%
\begin{pgfscope}%
\pgfsys@transformshift{3.347347in}{2.908704in}%
\pgfsys@useobject{currentmarker}{}%
\end{pgfscope}%
\end{pgfscope}%
\begin{pgfscope}%
\pgfsetbuttcap%
\pgfsetroundjoin%
\definecolor{currentfill}{rgb}{0.000000,0.000000,0.000000}%
\pgfsetfillcolor{currentfill}%
\pgfsetlinewidth{0.602250pt}%
\definecolor{currentstroke}{rgb}{0.000000,0.000000,0.000000}%
\pgfsetstrokecolor{currentstroke}%
\pgfsetdash{}{0pt}%
\pgfsys@defobject{currentmarker}{\pgfqpoint{-0.027778in}{0.000000in}}{\pgfqpoint{0.000000in}{0.000000in}}{%
\pgfpathmoveto{\pgfqpoint{0.000000in}{0.000000in}}%
\pgfpathlineto{\pgfqpoint{-0.027778in}{0.000000in}}%
\pgfusepath{stroke,fill}%
}%
\begin{pgfscope}%
\pgfsys@transformshift{3.347347in}{2.938155in}%
\pgfsys@useobject{currentmarker}{}%
\end{pgfscope}%
\end{pgfscope}%
\begin{pgfscope}%
\pgfsetbuttcap%
\pgfsetroundjoin%
\definecolor{currentfill}{rgb}{0.000000,0.000000,0.000000}%
\pgfsetfillcolor{currentfill}%
\pgfsetlinewidth{0.602250pt}%
\definecolor{currentstroke}{rgb}{0.000000,0.000000,0.000000}%
\pgfsetstrokecolor{currentstroke}%
\pgfsetdash{}{0pt}%
\pgfsys@defobject{currentmarker}{\pgfqpoint{-0.027778in}{0.000000in}}{\pgfqpoint{0.000000in}{0.000000in}}{%
\pgfpathmoveto{\pgfqpoint{0.000000in}{0.000000in}}%
\pgfpathlineto{\pgfqpoint{-0.027778in}{0.000000in}}%
\pgfusepath{stroke,fill}%
}%
\begin{pgfscope}%
\pgfsys@transformshift{3.347347in}{2.963056in}%
\pgfsys@useobject{currentmarker}{}%
\end{pgfscope}%
\end{pgfscope}%
\begin{pgfscope}%
\pgfsetbuttcap%
\pgfsetroundjoin%
\definecolor{currentfill}{rgb}{0.000000,0.000000,0.000000}%
\pgfsetfillcolor{currentfill}%
\pgfsetlinewidth{0.602250pt}%
\definecolor{currentstroke}{rgb}{0.000000,0.000000,0.000000}%
\pgfsetstrokecolor{currentstroke}%
\pgfsetdash{}{0pt}%
\pgfsys@defobject{currentmarker}{\pgfqpoint{-0.027778in}{0.000000in}}{\pgfqpoint{0.000000in}{0.000000in}}{%
\pgfpathmoveto{\pgfqpoint{0.000000in}{0.000000in}}%
\pgfpathlineto{\pgfqpoint{-0.027778in}{0.000000in}}%
\pgfusepath{stroke,fill}%
}%
\begin{pgfscope}%
\pgfsys@transformshift{3.347347in}{2.984626in}%
\pgfsys@useobject{currentmarker}{}%
\end{pgfscope}%
\end{pgfscope}%
\begin{pgfscope}%
\pgfsetbuttcap%
\pgfsetroundjoin%
\definecolor{currentfill}{rgb}{0.000000,0.000000,0.000000}%
\pgfsetfillcolor{currentfill}%
\pgfsetlinewidth{0.602250pt}%
\definecolor{currentstroke}{rgb}{0.000000,0.000000,0.000000}%
\pgfsetstrokecolor{currentstroke}%
\pgfsetdash{}{0pt}%
\pgfsys@defobject{currentmarker}{\pgfqpoint{-0.027778in}{0.000000in}}{\pgfqpoint{0.000000in}{0.000000in}}{%
\pgfpathmoveto{\pgfqpoint{0.000000in}{0.000000in}}%
\pgfpathlineto{\pgfqpoint{-0.027778in}{0.000000in}}%
\pgfusepath{stroke,fill}%
}%
\begin{pgfscope}%
\pgfsys@transformshift{3.347347in}{3.003651in}%
\pgfsys@useobject{currentmarker}{}%
\end{pgfscope}%
\end{pgfscope}%
\begin{pgfscope}%
\pgfsetbuttcap%
\pgfsetroundjoin%
\definecolor{currentfill}{rgb}{0.000000,0.000000,0.000000}%
\pgfsetfillcolor{currentfill}%
\pgfsetlinewidth{0.602250pt}%
\definecolor{currentstroke}{rgb}{0.000000,0.000000,0.000000}%
\pgfsetstrokecolor{currentstroke}%
\pgfsetdash{}{0pt}%
\pgfsys@defobject{currentmarker}{\pgfqpoint{-0.027778in}{0.000000in}}{\pgfqpoint{0.000000in}{0.000000in}}{%
\pgfpathmoveto{\pgfqpoint{0.000000in}{0.000000in}}%
\pgfpathlineto{\pgfqpoint{-0.027778in}{0.000000in}}%
\pgfusepath{stroke,fill}%
}%
\begin{pgfscope}%
\pgfsys@transformshift{3.347347in}{3.132637in}%
\pgfsys@useobject{currentmarker}{}%
\end{pgfscope}%
\end{pgfscope}%
\begin{pgfscope}%
\pgfsetbuttcap%
\pgfsetroundjoin%
\definecolor{currentfill}{rgb}{0.000000,0.000000,0.000000}%
\pgfsetfillcolor{currentfill}%
\pgfsetlinewidth{0.602250pt}%
\definecolor{currentstroke}{rgb}{0.000000,0.000000,0.000000}%
\pgfsetstrokecolor{currentstroke}%
\pgfsetdash{}{0pt}%
\pgfsys@defobject{currentmarker}{\pgfqpoint{-0.027778in}{0.000000in}}{\pgfqpoint{0.000000in}{0.000000in}}{%
\pgfpathmoveto{\pgfqpoint{0.000000in}{0.000000in}}%
\pgfpathlineto{\pgfqpoint{-0.027778in}{0.000000in}}%
\pgfusepath{stroke,fill}%
}%
\begin{pgfscope}%
\pgfsys@transformshift{3.347347in}{3.198133in}%
\pgfsys@useobject{currentmarker}{}%
\end{pgfscope}%
\end{pgfscope}%
\begin{pgfscope}%
\pgfsetbuttcap%
\pgfsetroundjoin%
\definecolor{currentfill}{rgb}{0.000000,0.000000,0.000000}%
\pgfsetfillcolor{currentfill}%
\pgfsetlinewidth{0.602250pt}%
\definecolor{currentstroke}{rgb}{0.000000,0.000000,0.000000}%
\pgfsetstrokecolor{currentstroke}%
\pgfsetdash{}{0pt}%
\pgfsys@defobject{currentmarker}{\pgfqpoint{-0.027778in}{0.000000in}}{\pgfqpoint{0.000000in}{0.000000in}}{%
\pgfpathmoveto{\pgfqpoint{0.000000in}{0.000000in}}%
\pgfpathlineto{\pgfqpoint{-0.027778in}{0.000000in}}%
\pgfusepath{stroke,fill}%
}%
\begin{pgfscope}%
\pgfsys@transformshift{3.347347in}{3.244603in}%
\pgfsys@useobject{currentmarker}{}%
\end{pgfscope}%
\end{pgfscope}%
\begin{pgfscope}%
\pgfsetbuttcap%
\pgfsetroundjoin%
\definecolor{currentfill}{rgb}{0.000000,0.000000,0.000000}%
\pgfsetfillcolor{currentfill}%
\pgfsetlinewidth{0.602250pt}%
\definecolor{currentstroke}{rgb}{0.000000,0.000000,0.000000}%
\pgfsetstrokecolor{currentstroke}%
\pgfsetdash{}{0pt}%
\pgfsys@defobject{currentmarker}{\pgfqpoint{-0.027778in}{0.000000in}}{\pgfqpoint{0.000000in}{0.000000in}}{%
\pgfpathmoveto{\pgfqpoint{0.000000in}{0.000000in}}%
\pgfpathlineto{\pgfqpoint{-0.027778in}{0.000000in}}%
\pgfusepath{stroke,fill}%
}%
\begin{pgfscope}%
\pgfsys@transformshift{3.347347in}{3.280648in}%
\pgfsys@useobject{currentmarker}{}%
\end{pgfscope}%
\end{pgfscope}%
\begin{pgfscope}%
\pgfsetbuttcap%
\pgfsetroundjoin%
\definecolor{currentfill}{rgb}{0.000000,0.000000,0.000000}%
\pgfsetfillcolor{currentfill}%
\pgfsetlinewidth{0.602250pt}%
\definecolor{currentstroke}{rgb}{0.000000,0.000000,0.000000}%
\pgfsetstrokecolor{currentstroke}%
\pgfsetdash{}{0pt}%
\pgfsys@defobject{currentmarker}{\pgfqpoint{-0.027778in}{0.000000in}}{\pgfqpoint{0.000000in}{0.000000in}}{%
\pgfpathmoveto{\pgfqpoint{0.000000in}{0.000000in}}%
\pgfpathlineto{\pgfqpoint{-0.027778in}{0.000000in}}%
\pgfusepath{stroke,fill}%
}%
\begin{pgfscope}%
\pgfsys@transformshift{3.347347in}{3.310099in}%
\pgfsys@useobject{currentmarker}{}%
\end{pgfscope}%
\end{pgfscope}%
\begin{pgfscope}%
\pgfsetbuttcap%
\pgfsetroundjoin%
\definecolor{currentfill}{rgb}{0.000000,0.000000,0.000000}%
\pgfsetfillcolor{currentfill}%
\pgfsetlinewidth{0.602250pt}%
\definecolor{currentstroke}{rgb}{0.000000,0.000000,0.000000}%
\pgfsetstrokecolor{currentstroke}%
\pgfsetdash{}{0pt}%
\pgfsys@defobject{currentmarker}{\pgfqpoint{-0.027778in}{0.000000in}}{\pgfqpoint{0.000000in}{0.000000in}}{%
\pgfpathmoveto{\pgfqpoint{0.000000in}{0.000000in}}%
\pgfpathlineto{\pgfqpoint{-0.027778in}{0.000000in}}%
\pgfusepath{stroke,fill}%
}%
\begin{pgfscope}%
\pgfsys@transformshift{3.347347in}{3.334999in}%
\pgfsys@useobject{currentmarker}{}%
\end{pgfscope}%
\end{pgfscope}%
\begin{pgfscope}%
\pgfsetbuttcap%
\pgfsetroundjoin%
\definecolor{currentfill}{rgb}{0.000000,0.000000,0.000000}%
\pgfsetfillcolor{currentfill}%
\pgfsetlinewidth{0.602250pt}%
\definecolor{currentstroke}{rgb}{0.000000,0.000000,0.000000}%
\pgfsetstrokecolor{currentstroke}%
\pgfsetdash{}{0pt}%
\pgfsys@defobject{currentmarker}{\pgfqpoint{-0.027778in}{0.000000in}}{\pgfqpoint{0.000000in}{0.000000in}}{%
\pgfpathmoveto{\pgfqpoint{0.000000in}{0.000000in}}%
\pgfpathlineto{\pgfqpoint{-0.027778in}{0.000000in}}%
\pgfusepath{stroke,fill}%
}%
\begin{pgfscope}%
\pgfsys@transformshift{3.347347in}{3.356569in}%
\pgfsys@useobject{currentmarker}{}%
\end{pgfscope}%
\end{pgfscope}%
\begin{pgfscope}%
\pgfsetbuttcap%
\pgfsetroundjoin%
\definecolor{currentfill}{rgb}{0.000000,0.000000,0.000000}%
\pgfsetfillcolor{currentfill}%
\pgfsetlinewidth{0.602250pt}%
\definecolor{currentstroke}{rgb}{0.000000,0.000000,0.000000}%
\pgfsetstrokecolor{currentstroke}%
\pgfsetdash{}{0pt}%
\pgfsys@defobject{currentmarker}{\pgfqpoint{-0.027778in}{0.000000in}}{\pgfqpoint{0.000000in}{0.000000in}}{%
\pgfpathmoveto{\pgfqpoint{0.000000in}{0.000000in}}%
\pgfpathlineto{\pgfqpoint{-0.027778in}{0.000000in}}%
\pgfusepath{stroke,fill}%
}%
\begin{pgfscope}%
\pgfsys@transformshift{3.347347in}{3.375595in}%
\pgfsys@useobject{currentmarker}{}%
\end{pgfscope}%
\end{pgfscope}%
\begin{pgfscope}%
\pgfsetbuttcap%
\pgfsetroundjoin%
\definecolor{currentfill}{rgb}{0.000000,0.000000,0.000000}%
\pgfsetfillcolor{currentfill}%
\pgfsetlinewidth{0.602250pt}%
\definecolor{currentstroke}{rgb}{0.000000,0.000000,0.000000}%
\pgfsetstrokecolor{currentstroke}%
\pgfsetdash{}{0pt}%
\pgfsys@defobject{currentmarker}{\pgfqpoint{-0.027778in}{0.000000in}}{\pgfqpoint{0.000000in}{0.000000in}}{%
\pgfpathmoveto{\pgfqpoint{0.000000in}{0.000000in}}%
\pgfpathlineto{\pgfqpoint{-0.027778in}{0.000000in}}%
\pgfusepath{stroke,fill}%
}%
\begin{pgfscope}%
\pgfsys@transformshift{3.347347in}{3.504580in}%
\pgfsys@useobject{currentmarker}{}%
\end{pgfscope}%
\end{pgfscope}%
\begin{pgfscope}%
\pgfsetbuttcap%
\pgfsetroundjoin%
\definecolor{currentfill}{rgb}{0.000000,0.000000,0.000000}%
\pgfsetfillcolor{currentfill}%
\pgfsetlinewidth{0.602250pt}%
\definecolor{currentstroke}{rgb}{0.000000,0.000000,0.000000}%
\pgfsetstrokecolor{currentstroke}%
\pgfsetdash{}{0pt}%
\pgfsys@defobject{currentmarker}{\pgfqpoint{-0.027778in}{0.000000in}}{\pgfqpoint{0.000000in}{0.000000in}}{%
\pgfpathmoveto{\pgfqpoint{0.000000in}{0.000000in}}%
\pgfpathlineto{\pgfqpoint{-0.027778in}{0.000000in}}%
\pgfusepath{stroke,fill}%
}%
\begin{pgfscope}%
\pgfsys@transformshift{3.347347in}{3.570076in}%
\pgfsys@useobject{currentmarker}{}%
\end{pgfscope}%
\end{pgfscope}%
\begin{pgfscope}%
\pgfsetbuttcap%
\pgfsetroundjoin%
\definecolor{currentfill}{rgb}{0.000000,0.000000,0.000000}%
\pgfsetfillcolor{currentfill}%
\pgfsetlinewidth{0.602250pt}%
\definecolor{currentstroke}{rgb}{0.000000,0.000000,0.000000}%
\pgfsetstrokecolor{currentstroke}%
\pgfsetdash{}{0pt}%
\pgfsys@defobject{currentmarker}{\pgfqpoint{-0.027778in}{0.000000in}}{\pgfqpoint{0.000000in}{0.000000in}}{%
\pgfpathmoveto{\pgfqpoint{0.000000in}{0.000000in}}%
\pgfpathlineto{\pgfqpoint{-0.027778in}{0.000000in}}%
\pgfusepath{stroke,fill}%
}%
\begin{pgfscope}%
\pgfsys@transformshift{3.347347in}{3.616546in}%
\pgfsys@useobject{currentmarker}{}%
\end{pgfscope}%
\end{pgfscope}%
\begin{pgfscope}%
\pgfsetbuttcap%
\pgfsetroundjoin%
\definecolor{currentfill}{rgb}{0.000000,0.000000,0.000000}%
\pgfsetfillcolor{currentfill}%
\pgfsetlinewidth{0.602250pt}%
\definecolor{currentstroke}{rgb}{0.000000,0.000000,0.000000}%
\pgfsetstrokecolor{currentstroke}%
\pgfsetdash{}{0pt}%
\pgfsys@defobject{currentmarker}{\pgfqpoint{-0.027778in}{0.000000in}}{\pgfqpoint{0.000000in}{0.000000in}}{%
\pgfpathmoveto{\pgfqpoint{0.000000in}{0.000000in}}%
\pgfpathlineto{\pgfqpoint{-0.027778in}{0.000000in}}%
\pgfusepath{stroke,fill}%
}%
\begin{pgfscope}%
\pgfsys@transformshift{3.347347in}{3.652591in}%
\pgfsys@useobject{currentmarker}{}%
\end{pgfscope}%
\end{pgfscope}%
\begin{pgfscope}%
\pgfsetbuttcap%
\pgfsetroundjoin%
\definecolor{currentfill}{rgb}{0.000000,0.000000,0.000000}%
\pgfsetfillcolor{currentfill}%
\pgfsetlinewidth{0.602250pt}%
\definecolor{currentstroke}{rgb}{0.000000,0.000000,0.000000}%
\pgfsetstrokecolor{currentstroke}%
\pgfsetdash{}{0pt}%
\pgfsys@defobject{currentmarker}{\pgfqpoint{-0.027778in}{0.000000in}}{\pgfqpoint{0.000000in}{0.000000in}}{%
\pgfpathmoveto{\pgfqpoint{0.000000in}{0.000000in}}%
\pgfpathlineto{\pgfqpoint{-0.027778in}{0.000000in}}%
\pgfusepath{stroke,fill}%
}%
\begin{pgfscope}%
\pgfsys@transformshift{3.347347in}{3.682042in}%
\pgfsys@useobject{currentmarker}{}%
\end{pgfscope}%
\end{pgfscope}%
\begin{pgfscope}%
\pgfsetbuttcap%
\pgfsetroundjoin%
\definecolor{currentfill}{rgb}{0.000000,0.000000,0.000000}%
\pgfsetfillcolor{currentfill}%
\pgfsetlinewidth{0.602250pt}%
\definecolor{currentstroke}{rgb}{0.000000,0.000000,0.000000}%
\pgfsetstrokecolor{currentstroke}%
\pgfsetdash{}{0pt}%
\pgfsys@defobject{currentmarker}{\pgfqpoint{-0.027778in}{0.000000in}}{\pgfqpoint{0.000000in}{0.000000in}}{%
\pgfpathmoveto{\pgfqpoint{0.000000in}{0.000000in}}%
\pgfpathlineto{\pgfqpoint{-0.027778in}{0.000000in}}%
\pgfusepath{stroke,fill}%
}%
\begin{pgfscope}%
\pgfsys@transformshift{3.347347in}{3.706943in}%
\pgfsys@useobject{currentmarker}{}%
\end{pgfscope}%
\end{pgfscope}%
\begin{pgfscope}%
\pgfsetbuttcap%
\pgfsetroundjoin%
\definecolor{currentfill}{rgb}{0.000000,0.000000,0.000000}%
\pgfsetfillcolor{currentfill}%
\pgfsetlinewidth{0.602250pt}%
\definecolor{currentstroke}{rgb}{0.000000,0.000000,0.000000}%
\pgfsetstrokecolor{currentstroke}%
\pgfsetdash{}{0pt}%
\pgfsys@defobject{currentmarker}{\pgfqpoint{-0.027778in}{0.000000in}}{\pgfqpoint{0.000000in}{0.000000in}}{%
\pgfpathmoveto{\pgfqpoint{0.000000in}{0.000000in}}%
\pgfpathlineto{\pgfqpoint{-0.027778in}{0.000000in}}%
\pgfusepath{stroke,fill}%
}%
\begin{pgfscope}%
\pgfsys@transformshift{3.347347in}{3.728513in}%
\pgfsys@useobject{currentmarker}{}%
\end{pgfscope}%
\end{pgfscope}%
\begin{pgfscope}%
\pgfsetbuttcap%
\pgfsetroundjoin%
\definecolor{currentfill}{rgb}{0.000000,0.000000,0.000000}%
\pgfsetfillcolor{currentfill}%
\pgfsetlinewidth{0.602250pt}%
\definecolor{currentstroke}{rgb}{0.000000,0.000000,0.000000}%
\pgfsetstrokecolor{currentstroke}%
\pgfsetdash{}{0pt}%
\pgfsys@defobject{currentmarker}{\pgfqpoint{-0.027778in}{0.000000in}}{\pgfqpoint{0.000000in}{0.000000in}}{%
\pgfpathmoveto{\pgfqpoint{0.000000in}{0.000000in}}%
\pgfpathlineto{\pgfqpoint{-0.027778in}{0.000000in}}%
\pgfusepath{stroke,fill}%
}%
\begin{pgfscope}%
\pgfsys@transformshift{3.347347in}{3.747538in}%
\pgfsys@useobject{currentmarker}{}%
\end{pgfscope}%
\end{pgfscope}%
\begin{pgfscope}%
\pgfsetbuttcap%
\pgfsetroundjoin%
\definecolor{currentfill}{rgb}{0.000000,0.000000,0.000000}%
\pgfsetfillcolor{currentfill}%
\pgfsetlinewidth{0.602250pt}%
\definecolor{currentstroke}{rgb}{0.000000,0.000000,0.000000}%
\pgfsetstrokecolor{currentstroke}%
\pgfsetdash{}{0pt}%
\pgfsys@defobject{currentmarker}{\pgfqpoint{-0.027778in}{0.000000in}}{\pgfqpoint{0.000000in}{0.000000in}}{%
\pgfpathmoveto{\pgfqpoint{0.000000in}{0.000000in}}%
\pgfpathlineto{\pgfqpoint{-0.027778in}{0.000000in}}%
\pgfusepath{stroke,fill}%
}%
\begin{pgfscope}%
\pgfsys@transformshift{3.347347in}{3.876524in}%
\pgfsys@useobject{currentmarker}{}%
\end{pgfscope}%
\end{pgfscope}%
\begin{pgfscope}%
\pgfsetbuttcap%
\pgfsetroundjoin%
\definecolor{currentfill}{rgb}{0.000000,0.000000,0.000000}%
\pgfsetfillcolor{currentfill}%
\pgfsetlinewidth{0.602250pt}%
\definecolor{currentstroke}{rgb}{0.000000,0.000000,0.000000}%
\pgfsetstrokecolor{currentstroke}%
\pgfsetdash{}{0pt}%
\pgfsys@defobject{currentmarker}{\pgfqpoint{-0.027778in}{0.000000in}}{\pgfqpoint{0.000000in}{0.000000in}}{%
\pgfpathmoveto{\pgfqpoint{0.000000in}{0.000000in}}%
\pgfpathlineto{\pgfqpoint{-0.027778in}{0.000000in}}%
\pgfusepath{stroke,fill}%
}%
\begin{pgfscope}%
\pgfsys@transformshift{3.347347in}{3.942020in}%
\pgfsys@useobject{currentmarker}{}%
\end{pgfscope}%
\end{pgfscope}%
\begin{pgfscope}%
\pgfsetbuttcap%
\pgfsetroundjoin%
\definecolor{currentfill}{rgb}{0.000000,0.000000,0.000000}%
\pgfsetfillcolor{currentfill}%
\pgfsetlinewidth{0.602250pt}%
\definecolor{currentstroke}{rgb}{0.000000,0.000000,0.000000}%
\pgfsetstrokecolor{currentstroke}%
\pgfsetdash{}{0pt}%
\pgfsys@defobject{currentmarker}{\pgfqpoint{-0.027778in}{0.000000in}}{\pgfqpoint{0.000000in}{0.000000in}}{%
\pgfpathmoveto{\pgfqpoint{0.000000in}{0.000000in}}%
\pgfpathlineto{\pgfqpoint{-0.027778in}{0.000000in}}%
\pgfusepath{stroke,fill}%
}%
\begin{pgfscope}%
\pgfsys@transformshift{3.347347in}{3.988490in}%
\pgfsys@useobject{currentmarker}{}%
\end{pgfscope}%
\end{pgfscope}%
\begin{pgfscope}%
\pgfsetbuttcap%
\pgfsetroundjoin%
\definecolor{currentfill}{rgb}{0.000000,0.000000,0.000000}%
\pgfsetfillcolor{currentfill}%
\pgfsetlinewidth{0.602250pt}%
\definecolor{currentstroke}{rgb}{0.000000,0.000000,0.000000}%
\pgfsetstrokecolor{currentstroke}%
\pgfsetdash{}{0pt}%
\pgfsys@defobject{currentmarker}{\pgfqpoint{-0.027778in}{0.000000in}}{\pgfqpoint{0.000000in}{0.000000in}}{%
\pgfpathmoveto{\pgfqpoint{0.000000in}{0.000000in}}%
\pgfpathlineto{\pgfqpoint{-0.027778in}{0.000000in}}%
\pgfusepath{stroke,fill}%
}%
\begin{pgfscope}%
\pgfsys@transformshift{3.347347in}{4.024535in}%
\pgfsys@useobject{currentmarker}{}%
\end{pgfscope}%
\end{pgfscope}%
\begin{pgfscope}%
\pgfsetbuttcap%
\pgfsetroundjoin%
\definecolor{currentfill}{rgb}{0.000000,0.000000,0.000000}%
\pgfsetfillcolor{currentfill}%
\pgfsetlinewidth{0.602250pt}%
\definecolor{currentstroke}{rgb}{0.000000,0.000000,0.000000}%
\pgfsetstrokecolor{currentstroke}%
\pgfsetdash{}{0pt}%
\pgfsys@defobject{currentmarker}{\pgfqpoint{-0.027778in}{0.000000in}}{\pgfqpoint{0.000000in}{0.000000in}}{%
\pgfpathmoveto{\pgfqpoint{0.000000in}{0.000000in}}%
\pgfpathlineto{\pgfqpoint{-0.027778in}{0.000000in}}%
\pgfusepath{stroke,fill}%
}%
\begin{pgfscope}%
\pgfsys@transformshift{3.347347in}{4.053986in}%
\pgfsys@useobject{currentmarker}{}%
\end{pgfscope}%
\end{pgfscope}%
\begin{pgfscope}%
\pgfsetbuttcap%
\pgfsetroundjoin%
\definecolor{currentfill}{rgb}{0.000000,0.000000,0.000000}%
\pgfsetfillcolor{currentfill}%
\pgfsetlinewidth{0.602250pt}%
\definecolor{currentstroke}{rgb}{0.000000,0.000000,0.000000}%
\pgfsetstrokecolor{currentstroke}%
\pgfsetdash{}{0pt}%
\pgfsys@defobject{currentmarker}{\pgfqpoint{-0.027778in}{0.000000in}}{\pgfqpoint{0.000000in}{0.000000in}}{%
\pgfpathmoveto{\pgfqpoint{0.000000in}{0.000000in}}%
\pgfpathlineto{\pgfqpoint{-0.027778in}{0.000000in}}%
\pgfusepath{stroke,fill}%
}%
\begin{pgfscope}%
\pgfsys@transformshift{3.347347in}{4.078886in}%
\pgfsys@useobject{currentmarker}{}%
\end{pgfscope}%
\end{pgfscope}%
\begin{pgfscope}%
\pgfsetbuttcap%
\pgfsetroundjoin%
\definecolor{currentfill}{rgb}{0.000000,0.000000,0.000000}%
\pgfsetfillcolor{currentfill}%
\pgfsetlinewidth{0.602250pt}%
\definecolor{currentstroke}{rgb}{0.000000,0.000000,0.000000}%
\pgfsetstrokecolor{currentstroke}%
\pgfsetdash{}{0pt}%
\pgfsys@defobject{currentmarker}{\pgfqpoint{-0.027778in}{0.000000in}}{\pgfqpoint{0.000000in}{0.000000in}}{%
\pgfpathmoveto{\pgfqpoint{0.000000in}{0.000000in}}%
\pgfpathlineto{\pgfqpoint{-0.027778in}{0.000000in}}%
\pgfusepath{stroke,fill}%
}%
\begin{pgfscope}%
\pgfsys@transformshift{3.347347in}{4.100456in}%
\pgfsys@useobject{currentmarker}{}%
\end{pgfscope}%
\end{pgfscope}%
\begin{pgfscope}%
\pgfsetbuttcap%
\pgfsetroundjoin%
\definecolor{currentfill}{rgb}{0.000000,0.000000,0.000000}%
\pgfsetfillcolor{currentfill}%
\pgfsetlinewidth{0.602250pt}%
\definecolor{currentstroke}{rgb}{0.000000,0.000000,0.000000}%
\pgfsetstrokecolor{currentstroke}%
\pgfsetdash{}{0pt}%
\pgfsys@defobject{currentmarker}{\pgfqpoint{-0.027778in}{0.000000in}}{\pgfqpoint{0.000000in}{0.000000in}}{%
\pgfpathmoveto{\pgfqpoint{0.000000in}{0.000000in}}%
\pgfpathlineto{\pgfqpoint{-0.027778in}{0.000000in}}%
\pgfusepath{stroke,fill}%
}%
\begin{pgfscope}%
\pgfsys@transformshift{3.347347in}{4.119482in}%
\pgfsys@useobject{currentmarker}{}%
\end{pgfscope}%
\end{pgfscope}%
\begin{pgfscope}%
\pgfpathrectangle{\pgfqpoint{3.347347in}{2.849537in}}{\pgfqpoint{1.897959in}{1.372727in}} %
\pgfusepath{clip}%
\pgfsetbuttcap%
\pgfsetroundjoin%
\pgfsetlinewidth{1.505625pt}%
\definecolor{currentstroke}{rgb}{1.000000,0.000000,0.000000}%
\pgfsetstrokecolor{currentstroke}%
\pgfsetdash{{5.550000pt}{2.400000pt}}{0.000000pt}%
\pgfpathmoveto{\pgfqpoint{3.433618in}{4.069185in}}%
\pgfpathlineto{\pgfqpoint{3.471127in}{4.065666in}}%
\pgfpathlineto{\pgfqpoint{3.508636in}{4.062243in}}%
\pgfpathlineto{\pgfqpoint{3.546145in}{4.058912in}}%
\pgfpathlineto{\pgfqpoint{3.583654in}{4.055670in}}%
\pgfpathlineto{\pgfqpoint{3.621163in}{4.052515in}}%
\pgfpathlineto{\pgfqpoint{3.658672in}{4.049446in}}%
\pgfpathlineto{\pgfqpoint{3.696181in}{4.046462in}}%
\pgfpathlineto{\pgfqpoint{3.733690in}{4.043561in}}%
\pgfpathlineto{\pgfqpoint{3.771199in}{4.040742in}}%
\pgfpathlineto{\pgfqpoint{3.808709in}{4.038004in}}%
\pgfpathlineto{\pgfqpoint{3.846218in}{4.035346in}}%
\pgfpathlineto{\pgfqpoint{3.883727in}{4.032768in}}%
\pgfpathlineto{\pgfqpoint{3.921236in}{4.030268in}}%
\pgfpathlineto{\pgfqpoint{3.958745in}{4.027845in}}%
\pgfpathlineto{\pgfqpoint{3.996254in}{4.025499in}}%
\pgfpathlineto{\pgfqpoint{4.033763in}{4.023228in}}%
\pgfpathlineto{\pgfqpoint{4.071272in}{4.021033in}}%
\pgfpathlineto{\pgfqpoint{4.108781in}{4.018911in}}%
\pgfpathlineto{\pgfqpoint{4.146290in}{4.016862in}}%
\pgfpathlineto{\pgfqpoint{4.183799in}{4.014886in}}%
\pgfpathlineto{\pgfqpoint{4.221308in}{4.012981in}}%
\pgfpathlineto{\pgfqpoint{4.258817in}{4.011146in}}%
\pgfpathlineto{\pgfqpoint{4.296327in}{4.009382in}}%
\pgfpathlineto{\pgfqpoint{4.333836in}{4.007687in}}%
\pgfpathlineto{\pgfqpoint{4.371345in}{4.006060in}}%
\pgfpathlineto{\pgfqpoint{4.408854in}{4.004500in}}%
\pgfpathlineto{\pgfqpoint{4.446363in}{4.003008in}}%
\pgfpathlineto{\pgfqpoint{4.483872in}{4.001582in}}%
\pgfpathlineto{\pgfqpoint{4.521381in}{4.000221in}}%
\pgfpathlineto{\pgfqpoint{4.558890in}{3.998925in}}%
\pgfpathlineto{\pgfqpoint{4.596399in}{3.997694in}}%
\pgfpathlineto{\pgfqpoint{4.633908in}{3.996526in}}%
\pgfpathlineto{\pgfqpoint{4.671417in}{3.995422in}}%
\pgfpathlineto{\pgfqpoint{4.708926in}{3.994380in}}%
\pgfpathlineto{\pgfqpoint{4.746435in}{3.993401in}}%
\pgfpathlineto{\pgfqpoint{4.783945in}{3.992483in}}%
\pgfpathlineto{\pgfqpoint{4.821454in}{3.991626in}}%
\pgfpathlineto{\pgfqpoint{4.858963in}{3.990830in}}%
\pgfpathlineto{\pgfqpoint{4.896472in}{3.990095in}}%
\pgfpathlineto{\pgfqpoint{4.933981in}{3.989419in}}%
\pgfpathlineto{\pgfqpoint{4.971490in}{3.988804in}}%
\pgfpathlineto{\pgfqpoint{5.008999in}{3.988247in}}%
\pgfpathlineto{\pgfqpoint{5.046508in}{3.987750in}}%
\pgfpathlineto{\pgfqpoint{5.084017in}{3.987312in}}%
\pgfpathlineto{\pgfqpoint{5.121526in}{3.986932in}}%
\pgfpathlineto{\pgfqpoint{5.159035in}{3.986611in}}%
\pgfusepath{stroke}%
\end{pgfscope}%
\begin{pgfscope}%
\pgfpathrectangle{\pgfqpoint{3.347347in}{2.849537in}}{\pgfqpoint{1.897959in}{1.372727in}} %
\pgfusepath{clip}%
\pgfsetbuttcap%
\pgfsetmiterjoin%
\definecolor{currentfill}{rgb}{1.000000,0.000000,0.000000}%
\pgfsetfillcolor{currentfill}%
\pgfsetlinewidth{1.003750pt}%
\definecolor{currentstroke}{rgb}{1.000000,0.000000,0.000000}%
\pgfsetstrokecolor{currentstroke}%
\pgfsetdash{}{0pt}%
\pgfsys@defobject{currentmarker}{\pgfqpoint{-0.041667in}{-0.041667in}}{\pgfqpoint{0.041667in}{0.041667in}}{%
\pgfpathmoveto{\pgfqpoint{-0.041667in}{-0.041667in}}%
\pgfpathlineto{\pgfqpoint{0.041667in}{-0.041667in}}%
\pgfpathlineto{\pgfqpoint{0.041667in}{0.041667in}}%
\pgfpathlineto{\pgfqpoint{-0.041667in}{0.041667in}}%
\pgfpathclose%
\pgfusepath{stroke,fill}%
}%
\begin{pgfscope}%
\pgfsys@transformshift{3.433618in}{4.069185in}%
\pgfsys@useobject{currentmarker}{}%
\end{pgfscope}%
\begin{pgfscope}%
\pgfsys@transformshift{3.808709in}{4.038004in}%
\pgfsys@useobject{currentmarker}{}%
\end{pgfscope}%
\begin{pgfscope}%
\pgfsys@transformshift{4.183799in}{4.014886in}%
\pgfsys@useobject{currentmarker}{}%
\end{pgfscope}%
\begin{pgfscope}%
\pgfsys@transformshift{4.558890in}{3.998925in}%
\pgfsys@useobject{currentmarker}{}%
\end{pgfscope}%
\begin{pgfscope}%
\pgfsys@transformshift{4.933981in}{3.989419in}%
\pgfsys@useobject{currentmarker}{}%
\end{pgfscope}%
\end{pgfscope}%
\begin{pgfscope}%
\pgfpathrectangle{\pgfqpoint{3.347347in}{2.849537in}}{\pgfqpoint{1.897959in}{1.372727in}} %
\pgfusepath{clip}%
\pgfsetrectcap%
\pgfsetroundjoin%
\pgfsetlinewidth{1.505625pt}%
\definecolor{currentstroke}{rgb}{0.000000,0.000000,1.000000}%
\pgfsetstrokecolor{currentstroke}%
\pgfsetdash{}{0pt}%
\pgfpathmoveto{\pgfqpoint{3.433618in}{3.448135in}}%
\pgfpathlineto{\pgfqpoint{3.471127in}{3.439500in}}%
\pgfpathlineto{\pgfqpoint{3.508636in}{3.431208in}}%
\pgfpathlineto{\pgfqpoint{3.546145in}{3.423172in}}%
\pgfpathlineto{\pgfqpoint{3.583654in}{3.415329in}}%
\pgfpathlineto{\pgfqpoint{3.621163in}{3.407630in}}%
\pgfpathlineto{\pgfqpoint{3.658672in}{3.400038in}}%
\pgfpathlineto{\pgfqpoint{3.696181in}{3.392522in}}%
\pgfpathlineto{\pgfqpoint{3.733690in}{3.385057in}}%
\pgfpathlineto{\pgfqpoint{3.771199in}{3.377622in}}%
\pgfpathlineto{\pgfqpoint{3.808709in}{3.370198in}}%
\pgfpathlineto{\pgfqpoint{3.846218in}{3.362768in}}%
\pgfpathlineto{\pgfqpoint{3.883727in}{3.355317in}}%
\pgfpathlineto{\pgfqpoint{3.921236in}{3.347830in}}%
\pgfpathlineto{\pgfqpoint{3.958745in}{3.340294in}}%
\pgfpathlineto{\pgfqpoint{3.996254in}{3.332696in}}%
\pgfpathlineto{\pgfqpoint{4.033763in}{3.325023in}}%
\pgfpathlineto{\pgfqpoint{4.071272in}{3.317262in}}%
\pgfpathlineto{\pgfqpoint{4.108781in}{3.309398in}}%
\pgfpathlineto{\pgfqpoint{4.146290in}{3.301420in}}%
\pgfpathlineto{\pgfqpoint{4.183799in}{3.293313in}}%
\pgfpathlineto{\pgfqpoint{4.221308in}{3.285061in}}%
\pgfpathlineto{\pgfqpoint{4.258817in}{3.276649in}}%
\pgfpathlineto{\pgfqpoint{4.296327in}{3.268060in}}%
\pgfpathlineto{\pgfqpoint{4.333836in}{3.259275in}}%
\pgfpathlineto{\pgfqpoint{4.371345in}{3.250276in}}%
\pgfpathlineto{\pgfqpoint{4.408854in}{3.241039in}}%
\pgfpathlineto{\pgfqpoint{4.446363in}{3.231541in}}%
\pgfpathlineto{\pgfqpoint{4.483872in}{3.221755in}}%
\pgfpathlineto{\pgfqpoint{4.521381in}{3.211651in}}%
\pgfpathlineto{\pgfqpoint{4.558890in}{3.201196in}}%
\pgfpathlineto{\pgfqpoint{4.596399in}{3.190350in}}%
\pgfpathlineto{\pgfqpoint{4.633908in}{3.179071in}}%
\pgfpathlineto{\pgfqpoint{4.671417in}{3.167308in}}%
\pgfpathlineto{\pgfqpoint{4.708926in}{3.155002in}}%
\pgfpathlineto{\pgfqpoint{4.746435in}{3.142084in}}%
\pgfpathlineto{\pgfqpoint{4.783945in}{3.128472in}}%
\pgfpathlineto{\pgfqpoint{4.821454in}{3.114069in}}%
\pgfpathlineto{\pgfqpoint{4.858963in}{3.098754in}}%
\pgfpathlineto{\pgfqpoint{4.896472in}{3.082382in}}%
\pgfpathlineto{\pgfqpoint{4.933981in}{3.064767in}}%
\pgfpathlineto{\pgfqpoint{4.971490in}{3.045673in}}%
\pgfpathlineto{\pgfqpoint{5.008999in}{3.024791in}}%
\pgfpathlineto{\pgfqpoint{5.046508in}{3.001706in}}%
\pgfpathlineto{\pgfqpoint{5.084017in}{2.975838in}}%
\pgfpathlineto{\pgfqpoint{5.121526in}{2.946346in}}%
\pgfpathlineto{\pgfqpoint{5.159035in}{2.911933in}}%
\pgfusepath{stroke}%
\end{pgfscope}%
\begin{pgfscope}%
\pgfpathrectangle{\pgfqpoint{3.347347in}{2.849537in}}{\pgfqpoint{1.897959in}{1.372727in}} %
\pgfusepath{clip}%
\pgfsetbuttcap%
\pgfsetroundjoin%
\definecolor{currentfill}{rgb}{0.000000,0.000000,1.000000}%
\pgfsetfillcolor{currentfill}%
\pgfsetlinewidth{1.003750pt}%
\definecolor{currentstroke}{rgb}{0.000000,0.000000,1.000000}%
\pgfsetstrokecolor{currentstroke}%
\pgfsetdash{}{0pt}%
\pgfsys@defobject{currentmarker}{\pgfqpoint{-0.041667in}{-0.041667in}}{\pgfqpoint{0.041667in}{0.041667in}}{%
\pgfpathmoveto{\pgfqpoint{0.000000in}{-0.041667in}}%
\pgfpathcurveto{\pgfqpoint{0.011050in}{-0.041667in}}{\pgfqpoint{0.021649in}{-0.037276in}}{\pgfqpoint{0.029463in}{-0.029463in}}%
\pgfpathcurveto{\pgfqpoint{0.037276in}{-0.021649in}}{\pgfqpoint{0.041667in}{-0.011050in}}{\pgfqpoint{0.041667in}{0.000000in}}%
\pgfpathcurveto{\pgfqpoint{0.041667in}{0.011050in}}{\pgfqpoint{0.037276in}{0.021649in}}{\pgfqpoint{0.029463in}{0.029463in}}%
\pgfpathcurveto{\pgfqpoint{0.021649in}{0.037276in}}{\pgfqpoint{0.011050in}{0.041667in}}{\pgfqpoint{0.000000in}{0.041667in}}%
\pgfpathcurveto{\pgfqpoint{-0.011050in}{0.041667in}}{\pgfqpoint{-0.021649in}{0.037276in}}{\pgfqpoint{-0.029463in}{0.029463in}}%
\pgfpathcurveto{\pgfqpoint{-0.037276in}{0.021649in}}{\pgfqpoint{-0.041667in}{0.011050in}}{\pgfqpoint{-0.041667in}{0.000000in}}%
\pgfpathcurveto{\pgfqpoint{-0.041667in}{-0.011050in}}{\pgfqpoint{-0.037276in}{-0.021649in}}{\pgfqpoint{-0.029463in}{-0.029463in}}%
\pgfpathcurveto{\pgfqpoint{-0.021649in}{-0.037276in}}{\pgfqpoint{-0.011050in}{-0.041667in}}{\pgfqpoint{0.000000in}{-0.041667in}}%
\pgfpathclose%
\pgfusepath{stroke,fill}%
}%
\begin{pgfscope}%
\pgfsys@transformshift{3.433618in}{3.448135in}%
\pgfsys@useobject{currentmarker}{}%
\end{pgfscope}%
\begin{pgfscope}%
\pgfsys@transformshift{3.808709in}{3.370198in}%
\pgfsys@useobject{currentmarker}{}%
\end{pgfscope}%
\begin{pgfscope}%
\pgfsys@transformshift{4.183799in}{3.293313in}%
\pgfsys@useobject{currentmarker}{}%
\end{pgfscope}%
\begin{pgfscope}%
\pgfsys@transformshift{4.558890in}{3.201196in}%
\pgfsys@useobject{currentmarker}{}%
\end{pgfscope}%
\begin{pgfscope}%
\pgfsys@transformshift{4.933981in}{3.064767in}%
\pgfsys@useobject{currentmarker}{}%
\end{pgfscope}%
\end{pgfscope}%
\begin{pgfscope}%
\pgfpathrectangle{\pgfqpoint{3.347347in}{2.849537in}}{\pgfqpoint{1.897959in}{1.372727in}} %
\pgfusepath{clip}%
\pgfsetbuttcap%
\pgfsetroundjoin%
\pgfsetlinewidth{1.505625pt}%
\definecolor{currentstroke}{rgb}{0.000000,0.750000,0.750000}%
\pgfsetstrokecolor{currentstroke}%
\pgfsetdash{{9.600000pt}{2.400000pt}{1.500000pt}{2.400000pt}}{0.000000pt}%
\pgfpathmoveto{\pgfqpoint{3.433618in}{4.020969in}}%
\pgfpathlineto{\pgfqpoint{3.471127in}{3.994255in}}%
\pgfpathlineto{\pgfqpoint{3.508636in}{3.970443in}}%
\pgfpathlineto{\pgfqpoint{3.546145in}{3.949017in}}%
\pgfpathlineto{\pgfqpoint{3.583654in}{3.929591in}}%
\pgfpathlineto{\pgfqpoint{3.621163in}{3.911865in}}%
\pgfpathlineto{\pgfqpoint{3.658672in}{3.895605in}}%
\pgfpathlineto{\pgfqpoint{3.696181in}{3.880620in}}%
\pgfpathlineto{\pgfqpoint{3.733690in}{3.866758in}}%
\pgfpathlineto{\pgfqpoint{3.771199in}{3.853891in}}%
\pgfpathlineto{\pgfqpoint{3.808709in}{3.841913in}}%
\pgfpathlineto{\pgfqpoint{3.846218in}{3.830734in}}%
\pgfpathlineto{\pgfqpoint{3.883727in}{3.820277in}}%
\pgfpathlineto{\pgfqpoint{3.921236in}{3.810479in}}%
\pgfpathlineto{\pgfqpoint{3.958745in}{3.801281in}}%
\pgfpathlineto{\pgfqpoint{3.996254in}{3.792636in}}%
\pgfpathlineto{\pgfqpoint{4.033763in}{3.784498in}}%
\pgfpathlineto{\pgfqpoint{4.071272in}{3.776832in}}%
\pgfpathlineto{\pgfqpoint{4.108781in}{3.769603in}}%
\pgfpathlineto{\pgfqpoint{4.146290in}{3.762782in}}%
\pgfpathlineto{\pgfqpoint{4.183799in}{3.756342in}}%
\pgfpathlineto{\pgfqpoint{4.221308in}{3.750259in}}%
\pgfpathlineto{\pgfqpoint{4.258817in}{3.744512in}}%
\pgfpathlineto{\pgfqpoint{4.296327in}{3.739083in}}%
\pgfpathlineto{\pgfqpoint{4.333836in}{3.733953in}}%
\pgfpathlineto{\pgfqpoint{4.371345in}{3.729108in}}%
\pgfpathlineto{\pgfqpoint{4.408854in}{3.724533in}}%
\pgfpathlineto{\pgfqpoint{4.446363in}{3.720215in}}%
\pgfpathlineto{\pgfqpoint{4.483872in}{3.716142in}}%
\pgfpathlineto{\pgfqpoint{4.521381in}{3.712305in}}%
\pgfpathlineto{\pgfqpoint{4.558890in}{3.708692in}}%
\pgfpathlineto{\pgfqpoint{4.596399in}{3.705297in}}%
\pgfpathlineto{\pgfqpoint{4.633908in}{3.702109in}}%
\pgfpathlineto{\pgfqpoint{4.671417in}{3.699123in}}%
\pgfpathlineto{\pgfqpoint{4.708926in}{3.696331in}}%
\pgfpathlineto{\pgfqpoint{4.746435in}{3.693727in}}%
\pgfpathlineto{\pgfqpoint{4.783945in}{3.691305in}}%
\pgfpathlineto{\pgfqpoint{4.821454in}{3.689061in}}%
\pgfpathlineto{\pgfqpoint{4.858963in}{3.686990in}}%
\pgfpathlineto{\pgfqpoint{4.896472in}{3.685088in}}%
\pgfpathlineto{\pgfqpoint{4.933981in}{3.683350in}}%
\pgfpathlineto{\pgfqpoint{4.971490in}{3.681774in}}%
\pgfpathlineto{\pgfqpoint{5.008999in}{3.680356in}}%
\pgfpathlineto{\pgfqpoint{5.046508in}{3.679094in}}%
\pgfpathlineto{\pgfqpoint{5.084017in}{3.677986in}}%
\pgfpathlineto{\pgfqpoint{5.121526in}{3.677028in}}%
\pgfpathlineto{\pgfqpoint{5.159035in}{3.676220in}}%
\pgfusepath{stroke}%
\end{pgfscope}%
\begin{pgfscope}%
\pgfpathrectangle{\pgfqpoint{3.347347in}{2.849537in}}{\pgfqpoint{1.897959in}{1.372727in}} %
\pgfusepath{clip}%
\pgfsetbuttcap%
\pgfsetmiterjoin%
\definecolor{currentfill}{rgb}{0.000000,0.750000,0.750000}%
\pgfsetfillcolor{currentfill}%
\pgfsetlinewidth{1.003750pt}%
\definecolor{currentstroke}{rgb}{0.000000,0.750000,0.750000}%
\pgfsetstrokecolor{currentstroke}%
\pgfsetdash{}{0pt}%
\pgfsys@defobject{currentmarker}{\pgfqpoint{-0.041667in}{-0.041667in}}{\pgfqpoint{0.041667in}{0.041667in}}{%
\pgfpathmoveto{\pgfqpoint{-0.000000in}{-0.041667in}}%
\pgfpathlineto{\pgfqpoint{0.041667in}{0.041667in}}%
\pgfpathlineto{\pgfqpoint{-0.041667in}{0.041667in}}%
\pgfpathclose%
\pgfusepath{stroke,fill}%
}%
\begin{pgfscope}%
\pgfsys@transformshift{3.433618in}{4.020969in}%
\pgfsys@useobject{currentmarker}{}%
\end{pgfscope}%
\begin{pgfscope}%
\pgfsys@transformshift{3.808709in}{3.841913in}%
\pgfsys@useobject{currentmarker}{}%
\end{pgfscope}%
\begin{pgfscope}%
\pgfsys@transformshift{4.183799in}{3.756342in}%
\pgfsys@useobject{currentmarker}{}%
\end{pgfscope}%
\begin{pgfscope}%
\pgfsys@transformshift{4.558890in}{3.708692in}%
\pgfsys@useobject{currentmarker}{}%
\end{pgfscope}%
\begin{pgfscope}%
\pgfsys@transformshift{4.933981in}{3.683350in}%
\pgfsys@useobject{currentmarker}{}%
\end{pgfscope}%
\end{pgfscope}%
\begin{pgfscope}%
\pgfpathrectangle{\pgfqpoint{3.347347in}{2.849537in}}{\pgfqpoint{1.897959in}{1.372727in}} %
\pgfusepath{clip}%
\pgfsetbuttcap%
\pgfsetroundjoin%
\pgfsetlinewidth{1.505625pt}%
\definecolor{currentstroke}{rgb}{0.000000,0.000000,0.000000}%
\pgfsetstrokecolor{currentstroke}%
\pgfsetdash{{1.500000pt}{2.475000pt}}{0.000000pt}%
\pgfpathmoveto{\pgfqpoint{3.433618in}{4.159867in}}%
\pgfpathlineto{\pgfqpoint{3.471127in}{4.148327in}}%
\pgfpathlineto{\pgfqpoint{3.508636in}{4.137154in}}%
\pgfpathlineto{\pgfqpoint{3.546145in}{4.127454in}}%
\pgfpathlineto{\pgfqpoint{3.583654in}{4.118903in}}%
\pgfpathlineto{\pgfqpoint{3.621163in}{4.111268in}}%
\pgfpathlineto{\pgfqpoint{3.658672in}{4.104377in}}%
\pgfpathlineto{\pgfqpoint{3.696181in}{4.098101in}}%
\pgfpathlineto{\pgfqpoint{3.733690in}{4.092340in}}%
\pgfpathlineto{\pgfqpoint{3.771199in}{4.087018in}}%
\pgfpathlineto{\pgfqpoint{3.808709in}{4.082075in}}%
\pgfpathlineto{\pgfqpoint{3.846218in}{4.077461in}}%
\pgfpathlineto{\pgfqpoint{3.883727in}{4.073138in}}%
\pgfpathlineto{\pgfqpoint{3.921236in}{4.069074in}}%
\pgfpathlineto{\pgfqpoint{3.958745in}{4.065244in}}%
\pgfpathlineto{\pgfqpoint{3.996254in}{4.061624in}}%
\pgfpathlineto{\pgfqpoint{4.033763in}{4.058197in}}%
\pgfpathlineto{\pgfqpoint{4.071272in}{4.054948in}}%
\pgfpathlineto{\pgfqpoint{4.108781in}{4.051863in}}%
\pgfpathlineto{\pgfqpoint{4.146290in}{4.048931in}}%
\pgfpathlineto{\pgfqpoint{4.183799in}{4.046142in}}%
\pgfpathlineto{\pgfqpoint{4.221308in}{4.043489in}}%
\pgfpathlineto{\pgfqpoint{4.258817in}{4.040963in}}%
\pgfpathlineto{\pgfqpoint{4.296327in}{4.038559in}}%
\pgfpathlineto{\pgfqpoint{4.333836in}{4.036270in}}%
\pgfpathlineto{\pgfqpoint{4.371345in}{4.034091in}}%
\pgfpathlineto{\pgfqpoint{4.408854in}{4.032019in}}%
\pgfpathlineto{\pgfqpoint{4.446363in}{4.030049in}}%
\pgfpathlineto{\pgfqpoint{4.483872in}{4.028178in}}%
\pgfpathlineto{\pgfqpoint{4.521381in}{4.026402in}}%
\pgfpathlineto{\pgfqpoint{4.558890in}{4.024718in}}%
\pgfpathlineto{\pgfqpoint{4.596399in}{4.023125in}}%
\pgfpathlineto{\pgfqpoint{4.633908in}{4.021619in}}%
\pgfpathlineto{\pgfqpoint{4.671417in}{4.020199in}}%
\pgfpathlineto{\pgfqpoint{4.708926in}{4.018863in}}%
\pgfpathlineto{\pgfqpoint{4.746435in}{4.017608in}}%
\pgfpathlineto{\pgfqpoint{4.783945in}{4.016434in}}%
\pgfpathlineto{\pgfqpoint{4.821454in}{4.015339in}}%
\pgfpathlineto{\pgfqpoint{4.858963in}{4.014322in}}%
\pgfpathlineto{\pgfqpoint{4.896472in}{4.013381in}}%
\pgfpathlineto{\pgfqpoint{4.933981in}{4.012516in}}%
\pgfpathlineto{\pgfqpoint{4.971490in}{4.011726in}}%
\pgfpathlineto{\pgfqpoint{5.008999in}{4.011009in}}%
\pgfpathlineto{\pgfqpoint{5.046508in}{4.010366in}}%
\pgfpathlineto{\pgfqpoint{5.084017in}{4.009796in}}%
\pgfpathlineto{\pgfqpoint{5.121526in}{4.009298in}}%
\pgfpathlineto{\pgfqpoint{5.159035in}{4.008872in}}%
\pgfusepath{stroke}%
\end{pgfscope}%
\begin{pgfscope}%
\pgfpathrectangle{\pgfqpoint{3.347347in}{2.849537in}}{\pgfqpoint{1.897959in}{1.372727in}} %
\pgfusepath{clip}%
\pgfsetbuttcap%
\pgfsetroundjoin%
\definecolor{currentfill}{rgb}{0.000000,0.000000,0.000000}%
\pgfsetfillcolor{currentfill}%
\pgfsetlinewidth{1.003750pt}%
\definecolor{currentstroke}{rgb}{0.000000,0.000000,0.000000}%
\pgfsetstrokecolor{currentstroke}%
\pgfsetdash{}{0pt}%
\pgfsys@defobject{currentmarker}{\pgfqpoint{-0.041667in}{-0.041667in}}{\pgfqpoint{0.041667in}{0.041667in}}{%
\pgfpathmoveto{\pgfqpoint{-0.041667in}{0.000000in}}%
\pgfpathlineto{\pgfqpoint{0.041667in}{0.000000in}}%
\pgfpathmoveto{\pgfqpoint{0.000000in}{-0.041667in}}%
\pgfpathlineto{\pgfqpoint{0.000000in}{0.041667in}}%
\pgfusepath{stroke,fill}%
}%
\begin{pgfscope}%
\pgfsys@transformshift{3.433618in}{4.159867in}%
\pgfsys@useobject{currentmarker}{}%
\end{pgfscope}%
\begin{pgfscope}%
\pgfsys@transformshift{3.808709in}{4.082075in}%
\pgfsys@useobject{currentmarker}{}%
\end{pgfscope}%
\begin{pgfscope}%
\pgfsys@transformshift{4.183799in}{4.046142in}%
\pgfsys@useobject{currentmarker}{}%
\end{pgfscope}%
\begin{pgfscope}%
\pgfsys@transformshift{4.558890in}{4.024718in}%
\pgfsys@useobject{currentmarker}{}%
\end{pgfscope}%
\begin{pgfscope}%
\pgfsys@transformshift{4.933981in}{4.012516in}%
\pgfsys@useobject{currentmarker}{}%
\end{pgfscope}%
\end{pgfscope}%
\begin{pgfscope}%
\pgfsetrectcap%
\pgfsetmiterjoin%
\pgfsetlinewidth{0.803000pt}%
\definecolor{currentstroke}{rgb}{0.000000,0.000000,0.000000}%
\pgfsetstrokecolor{currentstroke}%
\pgfsetdash{}{0pt}%
\pgfpathmoveto{\pgfqpoint{3.347347in}{2.849537in}}%
\pgfpathlineto{\pgfqpoint{3.347347in}{4.222264in}}%
\pgfusepath{stroke}%
\end{pgfscope}%
\begin{pgfscope}%
\pgfsetrectcap%
\pgfsetmiterjoin%
\pgfsetlinewidth{0.803000pt}%
\definecolor{currentstroke}{rgb}{0.000000,0.000000,0.000000}%
\pgfsetstrokecolor{currentstroke}%
\pgfsetdash{}{0pt}%
\pgfpathmoveto{\pgfqpoint{5.245306in}{2.849537in}}%
\pgfpathlineto{\pgfqpoint{5.245306in}{4.222264in}}%
\pgfusepath{stroke}%
\end{pgfscope}%
\begin{pgfscope}%
\pgfsetrectcap%
\pgfsetmiterjoin%
\pgfsetlinewidth{0.803000pt}%
\definecolor{currentstroke}{rgb}{0.000000,0.000000,0.000000}%
\pgfsetstrokecolor{currentstroke}%
\pgfsetdash{}{0pt}%
\pgfpathmoveto{\pgfqpoint{3.347347in}{2.849537in}}%
\pgfpathlineto{\pgfqpoint{5.245306in}{2.849537in}}%
\pgfusepath{stroke}%
\end{pgfscope}%
\begin{pgfscope}%
\pgfsetrectcap%
\pgfsetmiterjoin%
\pgfsetlinewidth{0.803000pt}%
\definecolor{currentstroke}{rgb}{0.000000,0.000000,0.000000}%
\pgfsetstrokecolor{currentstroke}%
\pgfsetdash{}{0pt}%
\pgfpathmoveto{\pgfqpoint{3.347347in}{4.222264in}}%
\pgfpathlineto{\pgfqpoint{5.245306in}{4.222264in}}%
\pgfusepath{stroke}%
\end{pgfscope}%
\begin{pgfscope}%
\pgfsetbuttcap%
\pgfsetmiterjoin%
\definecolor{currentfill}{rgb}{1.000000,1.000000,1.000000}%
\pgfsetfillcolor{currentfill}%
\pgfsetlinewidth{0.000000pt}%
\definecolor{currentstroke}{rgb}{0.000000,0.000000,0.000000}%
\pgfsetstrokecolor{currentstroke}%
\pgfsetstrokeopacity{0.000000}%
\pgfsetdash{}{0pt}%
\pgfpathmoveto{\pgfqpoint{5.814694in}{2.849537in}}%
\pgfpathlineto{\pgfqpoint{7.712653in}{2.849537in}}%
\pgfpathlineto{\pgfqpoint{7.712653in}{4.222264in}}%
\pgfpathlineto{\pgfqpoint{5.814694in}{4.222264in}}%
\pgfpathclose%
\pgfusepath{fill}%
\end{pgfscope}%
\begin{pgfscope}%
\pgfsetbuttcap%
\pgfsetroundjoin%
\definecolor{currentfill}{rgb}{0.000000,0.000000,0.000000}%
\pgfsetfillcolor{currentfill}%
\pgfsetlinewidth{0.803000pt}%
\definecolor{currentstroke}{rgb}{0.000000,0.000000,0.000000}%
\pgfsetstrokecolor{currentstroke}%
\pgfsetdash{}{0pt}%
\pgfsys@defobject{currentmarker}{\pgfqpoint{0.000000in}{-0.048611in}}{\pgfqpoint{0.000000in}{0.000000in}}{%
\pgfpathmoveto{\pgfqpoint{0.000000in}{0.000000in}}%
\pgfpathlineto{\pgfqpoint{0.000000in}{-0.048611in}}%
\pgfusepath{stroke,fill}%
}%
\begin{pgfscope}%
\pgfsys@transformshift{6.340162in}{2.849537in}%
\pgfsys@useobject{currentmarker}{}%
\end{pgfscope}%
\end{pgfscope}%
\begin{pgfscope}%
\pgftext[x=6.340162in,y=2.752315in,,top]{\rmfamily\fontsize{10.000000}{12.000000}\selectfont \(\displaystyle 0.2\)}%
\end{pgfscope}%
\begin{pgfscope}%
\pgfsetbuttcap%
\pgfsetroundjoin%
\definecolor{currentfill}{rgb}{0.000000,0.000000,0.000000}%
\pgfsetfillcolor{currentfill}%
\pgfsetlinewidth{0.803000pt}%
\definecolor{currentstroke}{rgb}{0.000000,0.000000,0.000000}%
\pgfsetstrokecolor{currentstroke}%
\pgfsetdash{}{0pt}%
\pgfsys@defobject{currentmarker}{\pgfqpoint{0.000000in}{-0.048611in}}{\pgfqpoint{0.000000in}{0.000000in}}{%
\pgfpathmoveto{\pgfqpoint{0.000000in}{0.000000in}}%
\pgfpathlineto{\pgfqpoint{0.000000in}{-0.048611in}}%
\pgfusepath{stroke,fill}%
}%
\begin{pgfscope}%
\pgfsys@transformshift{7.093071in}{2.849537in}%
\pgfsys@useobject{currentmarker}{}%
\end{pgfscope}%
\end{pgfscope}%
\begin{pgfscope}%
\pgftext[x=7.093071in,y=2.752315in,,top]{\rmfamily\fontsize{10.000000}{12.000000}\selectfont \(\displaystyle 0.4\)}%
\end{pgfscope}%
\begin{pgfscope}%
\pgfsetbuttcap%
\pgfsetroundjoin%
\definecolor{currentfill}{rgb}{0.000000,0.000000,0.000000}%
\pgfsetfillcolor{currentfill}%
\pgfsetlinewidth{0.803000pt}%
\definecolor{currentstroke}{rgb}{0.000000,0.000000,0.000000}%
\pgfsetstrokecolor{currentstroke}%
\pgfsetdash{}{0pt}%
\pgfsys@defobject{currentmarker}{\pgfqpoint{-0.048611in}{0.000000in}}{\pgfqpoint{0.000000in}{0.000000in}}{%
\pgfpathmoveto{\pgfqpoint{0.000000in}{0.000000in}}%
\pgfpathlineto{\pgfqpoint{-0.048611in}{0.000000in}}%
\pgfusepath{stroke,fill}%
}%
\begin{pgfscope}%
\pgfsys@transformshift{5.814694in}{2.977681in}%
\pgfsys@useobject{currentmarker}{}%
\end{pgfscope}%
\end{pgfscope}%
\begin{pgfscope}%
\pgftext[x=5.429469in,y=2.924920in,left,base]{\rmfamily\fontsize{10.000000}{12.000000}\selectfont \(\displaystyle 10^{-7}\)}%
\end{pgfscope}%
\begin{pgfscope}%
\pgfsetbuttcap%
\pgfsetroundjoin%
\definecolor{currentfill}{rgb}{0.000000,0.000000,0.000000}%
\pgfsetfillcolor{currentfill}%
\pgfsetlinewidth{0.803000pt}%
\definecolor{currentstroke}{rgb}{0.000000,0.000000,0.000000}%
\pgfsetstrokecolor{currentstroke}%
\pgfsetdash{}{0pt}%
\pgfsys@defobject{currentmarker}{\pgfqpoint{-0.048611in}{0.000000in}}{\pgfqpoint{0.000000in}{0.000000in}}{%
\pgfpathmoveto{\pgfqpoint{0.000000in}{0.000000in}}%
\pgfpathlineto{\pgfqpoint{-0.048611in}{0.000000in}}%
\pgfusepath{stroke,fill}%
}%
\begin{pgfscope}%
\pgfsys@transformshift{5.814694in}{3.332507in}%
\pgfsys@useobject{currentmarker}{}%
\end{pgfscope}%
\end{pgfscope}%
\begin{pgfscope}%
\pgftext[x=5.429469in,y=3.279746in,left,base]{\rmfamily\fontsize{10.000000}{12.000000}\selectfont \(\displaystyle 10^{-6}\)}%
\end{pgfscope}%
\begin{pgfscope}%
\pgfsetbuttcap%
\pgfsetroundjoin%
\definecolor{currentfill}{rgb}{0.000000,0.000000,0.000000}%
\pgfsetfillcolor{currentfill}%
\pgfsetlinewidth{0.803000pt}%
\definecolor{currentstroke}{rgb}{0.000000,0.000000,0.000000}%
\pgfsetstrokecolor{currentstroke}%
\pgfsetdash{}{0pt}%
\pgfsys@defobject{currentmarker}{\pgfqpoint{-0.048611in}{0.000000in}}{\pgfqpoint{0.000000in}{0.000000in}}{%
\pgfpathmoveto{\pgfqpoint{0.000000in}{0.000000in}}%
\pgfpathlineto{\pgfqpoint{-0.048611in}{0.000000in}}%
\pgfusepath{stroke,fill}%
}%
\begin{pgfscope}%
\pgfsys@transformshift{5.814694in}{3.687334in}%
\pgfsys@useobject{currentmarker}{}%
\end{pgfscope}%
\end{pgfscope}%
\begin{pgfscope}%
\pgftext[x=5.429469in,y=3.634572in,left,base]{\rmfamily\fontsize{10.000000}{12.000000}\selectfont \(\displaystyle 10^{-5}\)}%
\end{pgfscope}%
\begin{pgfscope}%
\pgfsetbuttcap%
\pgfsetroundjoin%
\definecolor{currentfill}{rgb}{0.000000,0.000000,0.000000}%
\pgfsetfillcolor{currentfill}%
\pgfsetlinewidth{0.803000pt}%
\definecolor{currentstroke}{rgb}{0.000000,0.000000,0.000000}%
\pgfsetstrokecolor{currentstroke}%
\pgfsetdash{}{0pt}%
\pgfsys@defobject{currentmarker}{\pgfqpoint{-0.048611in}{0.000000in}}{\pgfqpoint{0.000000in}{0.000000in}}{%
\pgfpathmoveto{\pgfqpoint{0.000000in}{0.000000in}}%
\pgfpathlineto{\pgfqpoint{-0.048611in}{0.000000in}}%
\pgfusepath{stroke,fill}%
}%
\begin{pgfscope}%
\pgfsys@transformshift{5.814694in}{4.042160in}%
\pgfsys@useobject{currentmarker}{}%
\end{pgfscope}%
\end{pgfscope}%
\begin{pgfscope}%
\pgftext[x=5.429469in,y=3.989399in,left,base]{\rmfamily\fontsize{10.000000}{12.000000}\selectfont \(\displaystyle 10^{-4}\)}%
\end{pgfscope}%
\begin{pgfscope}%
\pgfsetbuttcap%
\pgfsetroundjoin%
\definecolor{currentfill}{rgb}{0.000000,0.000000,0.000000}%
\pgfsetfillcolor{currentfill}%
\pgfsetlinewidth{0.602250pt}%
\definecolor{currentstroke}{rgb}{0.000000,0.000000,0.000000}%
\pgfsetstrokecolor{currentstroke}%
\pgfsetdash{}{0pt}%
\pgfsys@defobject{currentmarker}{\pgfqpoint{-0.027778in}{0.000000in}}{\pgfqpoint{0.000000in}{0.000000in}}{%
\pgfpathmoveto{\pgfqpoint{0.000000in}{0.000000in}}%
\pgfpathlineto{\pgfqpoint{-0.027778in}{0.000000in}}%
\pgfusepath{stroke,fill}%
}%
\begin{pgfscope}%
\pgfsys@transformshift{5.814694in}{2.870868in}%
\pgfsys@useobject{currentmarker}{}%
\end{pgfscope}%
\end{pgfscope}%
\begin{pgfscope}%
\pgfsetbuttcap%
\pgfsetroundjoin%
\definecolor{currentfill}{rgb}{0.000000,0.000000,0.000000}%
\pgfsetfillcolor{currentfill}%
\pgfsetlinewidth{0.602250pt}%
\definecolor{currentstroke}{rgb}{0.000000,0.000000,0.000000}%
\pgfsetstrokecolor{currentstroke}%
\pgfsetdash{}{0pt}%
\pgfsys@defobject{currentmarker}{\pgfqpoint{-0.027778in}{0.000000in}}{\pgfqpoint{0.000000in}{0.000000in}}{%
\pgfpathmoveto{\pgfqpoint{0.000000in}{0.000000in}}%
\pgfpathlineto{\pgfqpoint{-0.027778in}{0.000000in}}%
\pgfusepath{stroke,fill}%
}%
\begin{pgfscope}%
\pgfsys@transformshift{5.814694in}{2.898963in}%
\pgfsys@useobject{currentmarker}{}%
\end{pgfscope}%
\end{pgfscope}%
\begin{pgfscope}%
\pgfsetbuttcap%
\pgfsetroundjoin%
\definecolor{currentfill}{rgb}{0.000000,0.000000,0.000000}%
\pgfsetfillcolor{currentfill}%
\pgfsetlinewidth{0.602250pt}%
\definecolor{currentstroke}{rgb}{0.000000,0.000000,0.000000}%
\pgfsetstrokecolor{currentstroke}%
\pgfsetdash{}{0pt}%
\pgfsys@defobject{currentmarker}{\pgfqpoint{-0.027778in}{0.000000in}}{\pgfqpoint{0.000000in}{0.000000in}}{%
\pgfpathmoveto{\pgfqpoint{0.000000in}{0.000000in}}%
\pgfpathlineto{\pgfqpoint{-0.027778in}{0.000000in}}%
\pgfusepath{stroke,fill}%
}%
\begin{pgfscope}%
\pgfsys@transformshift{5.814694in}{2.922718in}%
\pgfsys@useobject{currentmarker}{}%
\end{pgfscope}%
\end{pgfscope}%
\begin{pgfscope}%
\pgfsetbuttcap%
\pgfsetroundjoin%
\definecolor{currentfill}{rgb}{0.000000,0.000000,0.000000}%
\pgfsetfillcolor{currentfill}%
\pgfsetlinewidth{0.602250pt}%
\definecolor{currentstroke}{rgb}{0.000000,0.000000,0.000000}%
\pgfsetstrokecolor{currentstroke}%
\pgfsetdash{}{0pt}%
\pgfsys@defobject{currentmarker}{\pgfqpoint{-0.027778in}{0.000000in}}{\pgfqpoint{0.000000in}{0.000000in}}{%
\pgfpathmoveto{\pgfqpoint{0.000000in}{0.000000in}}%
\pgfpathlineto{\pgfqpoint{-0.027778in}{0.000000in}}%
\pgfusepath{stroke,fill}%
}%
\begin{pgfscope}%
\pgfsys@transformshift{5.814694in}{2.943295in}%
\pgfsys@useobject{currentmarker}{}%
\end{pgfscope}%
\end{pgfscope}%
\begin{pgfscope}%
\pgfsetbuttcap%
\pgfsetroundjoin%
\definecolor{currentfill}{rgb}{0.000000,0.000000,0.000000}%
\pgfsetfillcolor{currentfill}%
\pgfsetlinewidth{0.602250pt}%
\definecolor{currentstroke}{rgb}{0.000000,0.000000,0.000000}%
\pgfsetstrokecolor{currentstroke}%
\pgfsetdash{}{0pt}%
\pgfsys@defobject{currentmarker}{\pgfqpoint{-0.027778in}{0.000000in}}{\pgfqpoint{0.000000in}{0.000000in}}{%
\pgfpathmoveto{\pgfqpoint{0.000000in}{0.000000in}}%
\pgfpathlineto{\pgfqpoint{-0.027778in}{0.000000in}}%
\pgfusepath{stroke,fill}%
}%
\begin{pgfscope}%
\pgfsys@transformshift{5.814694in}{2.961445in}%
\pgfsys@useobject{currentmarker}{}%
\end{pgfscope}%
\end{pgfscope}%
\begin{pgfscope}%
\pgfsetbuttcap%
\pgfsetroundjoin%
\definecolor{currentfill}{rgb}{0.000000,0.000000,0.000000}%
\pgfsetfillcolor{currentfill}%
\pgfsetlinewidth{0.602250pt}%
\definecolor{currentstroke}{rgb}{0.000000,0.000000,0.000000}%
\pgfsetstrokecolor{currentstroke}%
\pgfsetdash{}{0pt}%
\pgfsys@defobject{currentmarker}{\pgfqpoint{-0.027778in}{0.000000in}}{\pgfqpoint{0.000000in}{0.000000in}}{%
\pgfpathmoveto{\pgfqpoint{0.000000in}{0.000000in}}%
\pgfpathlineto{\pgfqpoint{-0.027778in}{0.000000in}}%
\pgfusepath{stroke,fill}%
}%
\begin{pgfscope}%
\pgfsys@transformshift{5.814694in}{3.084494in}%
\pgfsys@useobject{currentmarker}{}%
\end{pgfscope}%
\end{pgfscope}%
\begin{pgfscope}%
\pgfsetbuttcap%
\pgfsetroundjoin%
\definecolor{currentfill}{rgb}{0.000000,0.000000,0.000000}%
\pgfsetfillcolor{currentfill}%
\pgfsetlinewidth{0.602250pt}%
\definecolor{currentstroke}{rgb}{0.000000,0.000000,0.000000}%
\pgfsetstrokecolor{currentstroke}%
\pgfsetdash{}{0pt}%
\pgfsys@defobject{currentmarker}{\pgfqpoint{-0.027778in}{0.000000in}}{\pgfqpoint{0.000000in}{0.000000in}}{%
\pgfpathmoveto{\pgfqpoint{0.000000in}{0.000000in}}%
\pgfpathlineto{\pgfqpoint{-0.027778in}{0.000000in}}%
\pgfusepath{stroke,fill}%
}%
\begin{pgfscope}%
\pgfsys@transformshift{5.814694in}{3.146976in}%
\pgfsys@useobject{currentmarker}{}%
\end{pgfscope}%
\end{pgfscope}%
\begin{pgfscope}%
\pgfsetbuttcap%
\pgfsetroundjoin%
\definecolor{currentfill}{rgb}{0.000000,0.000000,0.000000}%
\pgfsetfillcolor{currentfill}%
\pgfsetlinewidth{0.602250pt}%
\definecolor{currentstroke}{rgb}{0.000000,0.000000,0.000000}%
\pgfsetstrokecolor{currentstroke}%
\pgfsetdash{}{0pt}%
\pgfsys@defobject{currentmarker}{\pgfqpoint{-0.027778in}{0.000000in}}{\pgfqpoint{0.000000in}{0.000000in}}{%
\pgfpathmoveto{\pgfqpoint{0.000000in}{0.000000in}}%
\pgfpathlineto{\pgfqpoint{-0.027778in}{0.000000in}}%
\pgfusepath{stroke,fill}%
}%
\begin{pgfscope}%
\pgfsys@transformshift{5.814694in}{3.191308in}%
\pgfsys@useobject{currentmarker}{}%
\end{pgfscope}%
\end{pgfscope}%
\begin{pgfscope}%
\pgfsetbuttcap%
\pgfsetroundjoin%
\definecolor{currentfill}{rgb}{0.000000,0.000000,0.000000}%
\pgfsetfillcolor{currentfill}%
\pgfsetlinewidth{0.602250pt}%
\definecolor{currentstroke}{rgb}{0.000000,0.000000,0.000000}%
\pgfsetstrokecolor{currentstroke}%
\pgfsetdash{}{0pt}%
\pgfsys@defobject{currentmarker}{\pgfqpoint{-0.027778in}{0.000000in}}{\pgfqpoint{0.000000in}{0.000000in}}{%
\pgfpathmoveto{\pgfqpoint{0.000000in}{0.000000in}}%
\pgfpathlineto{\pgfqpoint{-0.027778in}{0.000000in}}%
\pgfusepath{stroke,fill}%
}%
\begin{pgfscope}%
\pgfsys@transformshift{5.814694in}{3.225694in}%
\pgfsys@useobject{currentmarker}{}%
\end{pgfscope}%
\end{pgfscope}%
\begin{pgfscope}%
\pgfsetbuttcap%
\pgfsetroundjoin%
\definecolor{currentfill}{rgb}{0.000000,0.000000,0.000000}%
\pgfsetfillcolor{currentfill}%
\pgfsetlinewidth{0.602250pt}%
\definecolor{currentstroke}{rgb}{0.000000,0.000000,0.000000}%
\pgfsetstrokecolor{currentstroke}%
\pgfsetdash{}{0pt}%
\pgfsys@defobject{currentmarker}{\pgfqpoint{-0.027778in}{0.000000in}}{\pgfqpoint{0.000000in}{0.000000in}}{%
\pgfpathmoveto{\pgfqpoint{0.000000in}{0.000000in}}%
\pgfpathlineto{\pgfqpoint{-0.027778in}{0.000000in}}%
\pgfusepath{stroke,fill}%
}%
\begin{pgfscope}%
\pgfsys@transformshift{5.814694in}{3.253790in}%
\pgfsys@useobject{currentmarker}{}%
\end{pgfscope}%
\end{pgfscope}%
\begin{pgfscope}%
\pgfsetbuttcap%
\pgfsetroundjoin%
\definecolor{currentfill}{rgb}{0.000000,0.000000,0.000000}%
\pgfsetfillcolor{currentfill}%
\pgfsetlinewidth{0.602250pt}%
\definecolor{currentstroke}{rgb}{0.000000,0.000000,0.000000}%
\pgfsetstrokecolor{currentstroke}%
\pgfsetdash{}{0pt}%
\pgfsys@defobject{currentmarker}{\pgfqpoint{-0.027778in}{0.000000in}}{\pgfqpoint{0.000000in}{0.000000in}}{%
\pgfpathmoveto{\pgfqpoint{0.000000in}{0.000000in}}%
\pgfpathlineto{\pgfqpoint{-0.027778in}{0.000000in}}%
\pgfusepath{stroke,fill}%
}%
\begin{pgfscope}%
\pgfsys@transformshift{5.814694in}{3.277544in}%
\pgfsys@useobject{currentmarker}{}%
\end{pgfscope}%
\end{pgfscope}%
\begin{pgfscope}%
\pgfsetbuttcap%
\pgfsetroundjoin%
\definecolor{currentfill}{rgb}{0.000000,0.000000,0.000000}%
\pgfsetfillcolor{currentfill}%
\pgfsetlinewidth{0.602250pt}%
\definecolor{currentstroke}{rgb}{0.000000,0.000000,0.000000}%
\pgfsetstrokecolor{currentstroke}%
\pgfsetdash{}{0pt}%
\pgfsys@defobject{currentmarker}{\pgfqpoint{-0.027778in}{0.000000in}}{\pgfqpoint{0.000000in}{0.000000in}}{%
\pgfpathmoveto{\pgfqpoint{0.000000in}{0.000000in}}%
\pgfpathlineto{\pgfqpoint{-0.027778in}{0.000000in}}%
\pgfusepath{stroke,fill}%
}%
\begin{pgfscope}%
\pgfsys@transformshift{5.814694in}{3.298121in}%
\pgfsys@useobject{currentmarker}{}%
\end{pgfscope}%
\end{pgfscope}%
\begin{pgfscope}%
\pgfsetbuttcap%
\pgfsetroundjoin%
\definecolor{currentfill}{rgb}{0.000000,0.000000,0.000000}%
\pgfsetfillcolor{currentfill}%
\pgfsetlinewidth{0.602250pt}%
\definecolor{currentstroke}{rgb}{0.000000,0.000000,0.000000}%
\pgfsetstrokecolor{currentstroke}%
\pgfsetdash{}{0pt}%
\pgfsys@defobject{currentmarker}{\pgfqpoint{-0.027778in}{0.000000in}}{\pgfqpoint{0.000000in}{0.000000in}}{%
\pgfpathmoveto{\pgfqpoint{0.000000in}{0.000000in}}%
\pgfpathlineto{\pgfqpoint{-0.027778in}{0.000000in}}%
\pgfusepath{stroke,fill}%
}%
\begin{pgfscope}%
\pgfsys@transformshift{5.814694in}{3.316271in}%
\pgfsys@useobject{currentmarker}{}%
\end{pgfscope}%
\end{pgfscope}%
\begin{pgfscope}%
\pgfsetbuttcap%
\pgfsetroundjoin%
\definecolor{currentfill}{rgb}{0.000000,0.000000,0.000000}%
\pgfsetfillcolor{currentfill}%
\pgfsetlinewidth{0.602250pt}%
\definecolor{currentstroke}{rgb}{0.000000,0.000000,0.000000}%
\pgfsetstrokecolor{currentstroke}%
\pgfsetdash{}{0pt}%
\pgfsys@defobject{currentmarker}{\pgfqpoint{-0.027778in}{0.000000in}}{\pgfqpoint{0.000000in}{0.000000in}}{%
\pgfpathmoveto{\pgfqpoint{0.000000in}{0.000000in}}%
\pgfpathlineto{\pgfqpoint{-0.027778in}{0.000000in}}%
\pgfusepath{stroke,fill}%
}%
\begin{pgfscope}%
\pgfsys@transformshift{5.814694in}{3.439321in}%
\pgfsys@useobject{currentmarker}{}%
\end{pgfscope}%
\end{pgfscope}%
\begin{pgfscope}%
\pgfsetbuttcap%
\pgfsetroundjoin%
\definecolor{currentfill}{rgb}{0.000000,0.000000,0.000000}%
\pgfsetfillcolor{currentfill}%
\pgfsetlinewidth{0.602250pt}%
\definecolor{currentstroke}{rgb}{0.000000,0.000000,0.000000}%
\pgfsetstrokecolor{currentstroke}%
\pgfsetdash{}{0pt}%
\pgfsys@defobject{currentmarker}{\pgfqpoint{-0.027778in}{0.000000in}}{\pgfqpoint{0.000000in}{0.000000in}}{%
\pgfpathmoveto{\pgfqpoint{0.000000in}{0.000000in}}%
\pgfpathlineto{\pgfqpoint{-0.027778in}{0.000000in}}%
\pgfusepath{stroke,fill}%
}%
\begin{pgfscope}%
\pgfsys@transformshift{5.814694in}{3.501803in}%
\pgfsys@useobject{currentmarker}{}%
\end{pgfscope}%
\end{pgfscope}%
\begin{pgfscope}%
\pgfsetbuttcap%
\pgfsetroundjoin%
\definecolor{currentfill}{rgb}{0.000000,0.000000,0.000000}%
\pgfsetfillcolor{currentfill}%
\pgfsetlinewidth{0.602250pt}%
\definecolor{currentstroke}{rgb}{0.000000,0.000000,0.000000}%
\pgfsetstrokecolor{currentstroke}%
\pgfsetdash{}{0pt}%
\pgfsys@defobject{currentmarker}{\pgfqpoint{-0.027778in}{0.000000in}}{\pgfqpoint{0.000000in}{0.000000in}}{%
\pgfpathmoveto{\pgfqpoint{0.000000in}{0.000000in}}%
\pgfpathlineto{\pgfqpoint{-0.027778in}{0.000000in}}%
\pgfusepath{stroke,fill}%
}%
\begin{pgfscope}%
\pgfsys@transformshift{5.814694in}{3.546134in}%
\pgfsys@useobject{currentmarker}{}%
\end{pgfscope}%
\end{pgfscope}%
\begin{pgfscope}%
\pgfsetbuttcap%
\pgfsetroundjoin%
\definecolor{currentfill}{rgb}{0.000000,0.000000,0.000000}%
\pgfsetfillcolor{currentfill}%
\pgfsetlinewidth{0.602250pt}%
\definecolor{currentstroke}{rgb}{0.000000,0.000000,0.000000}%
\pgfsetstrokecolor{currentstroke}%
\pgfsetdash{}{0pt}%
\pgfsys@defobject{currentmarker}{\pgfqpoint{-0.027778in}{0.000000in}}{\pgfqpoint{0.000000in}{0.000000in}}{%
\pgfpathmoveto{\pgfqpoint{0.000000in}{0.000000in}}%
\pgfpathlineto{\pgfqpoint{-0.027778in}{0.000000in}}%
\pgfusepath{stroke,fill}%
}%
\begin{pgfscope}%
\pgfsys@transformshift{5.814694in}{3.580520in}%
\pgfsys@useobject{currentmarker}{}%
\end{pgfscope}%
\end{pgfscope}%
\begin{pgfscope}%
\pgfsetbuttcap%
\pgfsetroundjoin%
\definecolor{currentfill}{rgb}{0.000000,0.000000,0.000000}%
\pgfsetfillcolor{currentfill}%
\pgfsetlinewidth{0.602250pt}%
\definecolor{currentstroke}{rgb}{0.000000,0.000000,0.000000}%
\pgfsetstrokecolor{currentstroke}%
\pgfsetdash{}{0pt}%
\pgfsys@defobject{currentmarker}{\pgfqpoint{-0.027778in}{0.000000in}}{\pgfqpoint{0.000000in}{0.000000in}}{%
\pgfpathmoveto{\pgfqpoint{0.000000in}{0.000000in}}%
\pgfpathlineto{\pgfqpoint{-0.027778in}{0.000000in}}%
\pgfusepath{stroke,fill}%
}%
\begin{pgfscope}%
\pgfsys@transformshift{5.814694in}{3.608616in}%
\pgfsys@useobject{currentmarker}{}%
\end{pgfscope}%
\end{pgfscope}%
\begin{pgfscope}%
\pgfsetbuttcap%
\pgfsetroundjoin%
\definecolor{currentfill}{rgb}{0.000000,0.000000,0.000000}%
\pgfsetfillcolor{currentfill}%
\pgfsetlinewidth{0.602250pt}%
\definecolor{currentstroke}{rgb}{0.000000,0.000000,0.000000}%
\pgfsetstrokecolor{currentstroke}%
\pgfsetdash{}{0pt}%
\pgfsys@defobject{currentmarker}{\pgfqpoint{-0.027778in}{0.000000in}}{\pgfqpoint{0.000000in}{0.000000in}}{%
\pgfpathmoveto{\pgfqpoint{0.000000in}{0.000000in}}%
\pgfpathlineto{\pgfqpoint{-0.027778in}{0.000000in}}%
\pgfusepath{stroke,fill}%
}%
\begin{pgfscope}%
\pgfsys@transformshift{5.814694in}{3.632371in}%
\pgfsys@useobject{currentmarker}{}%
\end{pgfscope}%
\end{pgfscope}%
\begin{pgfscope}%
\pgfsetbuttcap%
\pgfsetroundjoin%
\definecolor{currentfill}{rgb}{0.000000,0.000000,0.000000}%
\pgfsetfillcolor{currentfill}%
\pgfsetlinewidth{0.602250pt}%
\definecolor{currentstroke}{rgb}{0.000000,0.000000,0.000000}%
\pgfsetstrokecolor{currentstroke}%
\pgfsetdash{}{0pt}%
\pgfsys@defobject{currentmarker}{\pgfqpoint{-0.027778in}{0.000000in}}{\pgfqpoint{0.000000in}{0.000000in}}{%
\pgfpathmoveto{\pgfqpoint{0.000000in}{0.000000in}}%
\pgfpathlineto{\pgfqpoint{-0.027778in}{0.000000in}}%
\pgfusepath{stroke,fill}%
}%
\begin{pgfscope}%
\pgfsys@transformshift{5.814694in}{3.652948in}%
\pgfsys@useobject{currentmarker}{}%
\end{pgfscope}%
\end{pgfscope}%
\begin{pgfscope}%
\pgfsetbuttcap%
\pgfsetroundjoin%
\definecolor{currentfill}{rgb}{0.000000,0.000000,0.000000}%
\pgfsetfillcolor{currentfill}%
\pgfsetlinewidth{0.602250pt}%
\definecolor{currentstroke}{rgb}{0.000000,0.000000,0.000000}%
\pgfsetstrokecolor{currentstroke}%
\pgfsetdash{}{0pt}%
\pgfsys@defobject{currentmarker}{\pgfqpoint{-0.027778in}{0.000000in}}{\pgfqpoint{0.000000in}{0.000000in}}{%
\pgfpathmoveto{\pgfqpoint{0.000000in}{0.000000in}}%
\pgfpathlineto{\pgfqpoint{-0.027778in}{0.000000in}}%
\pgfusepath{stroke,fill}%
}%
\begin{pgfscope}%
\pgfsys@transformshift{5.814694in}{3.671098in}%
\pgfsys@useobject{currentmarker}{}%
\end{pgfscope}%
\end{pgfscope}%
\begin{pgfscope}%
\pgfsetbuttcap%
\pgfsetroundjoin%
\definecolor{currentfill}{rgb}{0.000000,0.000000,0.000000}%
\pgfsetfillcolor{currentfill}%
\pgfsetlinewidth{0.602250pt}%
\definecolor{currentstroke}{rgb}{0.000000,0.000000,0.000000}%
\pgfsetstrokecolor{currentstroke}%
\pgfsetdash{}{0pt}%
\pgfsys@defobject{currentmarker}{\pgfqpoint{-0.027778in}{0.000000in}}{\pgfqpoint{0.000000in}{0.000000in}}{%
\pgfpathmoveto{\pgfqpoint{0.000000in}{0.000000in}}%
\pgfpathlineto{\pgfqpoint{-0.027778in}{0.000000in}}%
\pgfusepath{stroke,fill}%
}%
\begin{pgfscope}%
\pgfsys@transformshift{5.814694in}{3.794147in}%
\pgfsys@useobject{currentmarker}{}%
\end{pgfscope}%
\end{pgfscope}%
\begin{pgfscope}%
\pgfsetbuttcap%
\pgfsetroundjoin%
\definecolor{currentfill}{rgb}{0.000000,0.000000,0.000000}%
\pgfsetfillcolor{currentfill}%
\pgfsetlinewidth{0.602250pt}%
\definecolor{currentstroke}{rgb}{0.000000,0.000000,0.000000}%
\pgfsetstrokecolor{currentstroke}%
\pgfsetdash{}{0pt}%
\pgfsys@defobject{currentmarker}{\pgfqpoint{-0.027778in}{0.000000in}}{\pgfqpoint{0.000000in}{0.000000in}}{%
\pgfpathmoveto{\pgfqpoint{0.000000in}{0.000000in}}%
\pgfpathlineto{\pgfqpoint{-0.027778in}{0.000000in}}%
\pgfusepath{stroke,fill}%
}%
\begin{pgfscope}%
\pgfsys@transformshift{5.814694in}{3.856629in}%
\pgfsys@useobject{currentmarker}{}%
\end{pgfscope}%
\end{pgfscope}%
\begin{pgfscope}%
\pgfsetbuttcap%
\pgfsetroundjoin%
\definecolor{currentfill}{rgb}{0.000000,0.000000,0.000000}%
\pgfsetfillcolor{currentfill}%
\pgfsetlinewidth{0.602250pt}%
\definecolor{currentstroke}{rgb}{0.000000,0.000000,0.000000}%
\pgfsetstrokecolor{currentstroke}%
\pgfsetdash{}{0pt}%
\pgfsys@defobject{currentmarker}{\pgfqpoint{-0.027778in}{0.000000in}}{\pgfqpoint{0.000000in}{0.000000in}}{%
\pgfpathmoveto{\pgfqpoint{0.000000in}{0.000000in}}%
\pgfpathlineto{\pgfqpoint{-0.027778in}{0.000000in}}%
\pgfusepath{stroke,fill}%
}%
\begin{pgfscope}%
\pgfsys@transformshift{5.814694in}{3.900961in}%
\pgfsys@useobject{currentmarker}{}%
\end{pgfscope}%
\end{pgfscope}%
\begin{pgfscope}%
\pgfsetbuttcap%
\pgfsetroundjoin%
\definecolor{currentfill}{rgb}{0.000000,0.000000,0.000000}%
\pgfsetfillcolor{currentfill}%
\pgfsetlinewidth{0.602250pt}%
\definecolor{currentstroke}{rgb}{0.000000,0.000000,0.000000}%
\pgfsetstrokecolor{currentstroke}%
\pgfsetdash{}{0pt}%
\pgfsys@defobject{currentmarker}{\pgfqpoint{-0.027778in}{0.000000in}}{\pgfqpoint{0.000000in}{0.000000in}}{%
\pgfpathmoveto{\pgfqpoint{0.000000in}{0.000000in}}%
\pgfpathlineto{\pgfqpoint{-0.027778in}{0.000000in}}%
\pgfusepath{stroke,fill}%
}%
\begin{pgfscope}%
\pgfsys@transformshift{5.814694in}{3.935347in}%
\pgfsys@useobject{currentmarker}{}%
\end{pgfscope}%
\end{pgfscope}%
\begin{pgfscope}%
\pgfsetbuttcap%
\pgfsetroundjoin%
\definecolor{currentfill}{rgb}{0.000000,0.000000,0.000000}%
\pgfsetfillcolor{currentfill}%
\pgfsetlinewidth{0.602250pt}%
\definecolor{currentstroke}{rgb}{0.000000,0.000000,0.000000}%
\pgfsetstrokecolor{currentstroke}%
\pgfsetdash{}{0pt}%
\pgfsys@defobject{currentmarker}{\pgfqpoint{-0.027778in}{0.000000in}}{\pgfqpoint{0.000000in}{0.000000in}}{%
\pgfpathmoveto{\pgfqpoint{0.000000in}{0.000000in}}%
\pgfpathlineto{\pgfqpoint{-0.027778in}{0.000000in}}%
\pgfusepath{stroke,fill}%
}%
\begin{pgfscope}%
\pgfsys@transformshift{5.814694in}{3.963443in}%
\pgfsys@useobject{currentmarker}{}%
\end{pgfscope}%
\end{pgfscope}%
\begin{pgfscope}%
\pgfsetbuttcap%
\pgfsetroundjoin%
\definecolor{currentfill}{rgb}{0.000000,0.000000,0.000000}%
\pgfsetfillcolor{currentfill}%
\pgfsetlinewidth{0.602250pt}%
\definecolor{currentstroke}{rgb}{0.000000,0.000000,0.000000}%
\pgfsetstrokecolor{currentstroke}%
\pgfsetdash{}{0pt}%
\pgfsys@defobject{currentmarker}{\pgfqpoint{-0.027778in}{0.000000in}}{\pgfqpoint{0.000000in}{0.000000in}}{%
\pgfpathmoveto{\pgfqpoint{0.000000in}{0.000000in}}%
\pgfpathlineto{\pgfqpoint{-0.027778in}{0.000000in}}%
\pgfusepath{stroke,fill}%
}%
\begin{pgfscope}%
\pgfsys@transformshift{5.814694in}{3.987197in}%
\pgfsys@useobject{currentmarker}{}%
\end{pgfscope}%
\end{pgfscope}%
\begin{pgfscope}%
\pgfsetbuttcap%
\pgfsetroundjoin%
\definecolor{currentfill}{rgb}{0.000000,0.000000,0.000000}%
\pgfsetfillcolor{currentfill}%
\pgfsetlinewidth{0.602250pt}%
\definecolor{currentstroke}{rgb}{0.000000,0.000000,0.000000}%
\pgfsetstrokecolor{currentstroke}%
\pgfsetdash{}{0pt}%
\pgfsys@defobject{currentmarker}{\pgfqpoint{-0.027778in}{0.000000in}}{\pgfqpoint{0.000000in}{0.000000in}}{%
\pgfpathmoveto{\pgfqpoint{0.000000in}{0.000000in}}%
\pgfpathlineto{\pgfqpoint{-0.027778in}{0.000000in}}%
\pgfusepath{stroke,fill}%
}%
\begin{pgfscope}%
\pgfsys@transformshift{5.814694in}{4.007774in}%
\pgfsys@useobject{currentmarker}{}%
\end{pgfscope}%
\end{pgfscope}%
\begin{pgfscope}%
\pgfsetbuttcap%
\pgfsetroundjoin%
\definecolor{currentfill}{rgb}{0.000000,0.000000,0.000000}%
\pgfsetfillcolor{currentfill}%
\pgfsetlinewidth{0.602250pt}%
\definecolor{currentstroke}{rgb}{0.000000,0.000000,0.000000}%
\pgfsetstrokecolor{currentstroke}%
\pgfsetdash{}{0pt}%
\pgfsys@defobject{currentmarker}{\pgfqpoint{-0.027778in}{0.000000in}}{\pgfqpoint{0.000000in}{0.000000in}}{%
\pgfpathmoveto{\pgfqpoint{0.000000in}{0.000000in}}%
\pgfpathlineto{\pgfqpoint{-0.027778in}{0.000000in}}%
\pgfusepath{stroke,fill}%
}%
\begin{pgfscope}%
\pgfsys@transformshift{5.814694in}{4.025924in}%
\pgfsys@useobject{currentmarker}{}%
\end{pgfscope}%
\end{pgfscope}%
\begin{pgfscope}%
\pgfsetbuttcap%
\pgfsetroundjoin%
\definecolor{currentfill}{rgb}{0.000000,0.000000,0.000000}%
\pgfsetfillcolor{currentfill}%
\pgfsetlinewidth{0.602250pt}%
\definecolor{currentstroke}{rgb}{0.000000,0.000000,0.000000}%
\pgfsetstrokecolor{currentstroke}%
\pgfsetdash{}{0pt}%
\pgfsys@defobject{currentmarker}{\pgfqpoint{-0.027778in}{0.000000in}}{\pgfqpoint{0.000000in}{0.000000in}}{%
\pgfpathmoveto{\pgfqpoint{0.000000in}{0.000000in}}%
\pgfpathlineto{\pgfqpoint{-0.027778in}{0.000000in}}%
\pgfusepath{stroke,fill}%
}%
\begin{pgfscope}%
\pgfsys@transformshift{5.814694in}{4.148974in}%
\pgfsys@useobject{currentmarker}{}%
\end{pgfscope}%
\end{pgfscope}%
\begin{pgfscope}%
\pgfsetbuttcap%
\pgfsetroundjoin%
\definecolor{currentfill}{rgb}{0.000000,0.000000,0.000000}%
\pgfsetfillcolor{currentfill}%
\pgfsetlinewidth{0.602250pt}%
\definecolor{currentstroke}{rgb}{0.000000,0.000000,0.000000}%
\pgfsetstrokecolor{currentstroke}%
\pgfsetdash{}{0pt}%
\pgfsys@defobject{currentmarker}{\pgfqpoint{-0.027778in}{0.000000in}}{\pgfqpoint{0.000000in}{0.000000in}}{%
\pgfpathmoveto{\pgfqpoint{0.000000in}{0.000000in}}%
\pgfpathlineto{\pgfqpoint{-0.027778in}{0.000000in}}%
\pgfusepath{stroke,fill}%
}%
\begin{pgfscope}%
\pgfsys@transformshift{5.814694in}{4.211456in}%
\pgfsys@useobject{currentmarker}{}%
\end{pgfscope}%
\end{pgfscope}%
\begin{pgfscope}%
\pgfpathrectangle{\pgfqpoint{5.814694in}{2.849537in}}{\pgfqpoint{1.897959in}{1.372727in}} %
\pgfusepath{clip}%
\pgfsetbuttcap%
\pgfsetroundjoin%
\pgfsetlinewidth{1.505625pt}%
\definecolor{currentstroke}{rgb}{1.000000,0.000000,0.000000}%
\pgfsetstrokecolor{currentstroke}%
\pgfsetdash{{5.550000pt}{2.400000pt}}{0.000000pt}%
\pgfpathmoveto{\pgfqpoint{5.900965in}{4.025621in}}%
\pgfpathlineto{\pgfqpoint{5.932336in}{4.022311in}}%
\pgfpathlineto{\pgfqpoint{5.963707in}{4.019092in}}%
\pgfpathlineto{\pgfqpoint{5.995078in}{4.015961in}}%
\pgfpathlineto{\pgfqpoint{6.026450in}{4.012913in}}%
\pgfpathlineto{\pgfqpoint{6.057821in}{4.009946in}}%
\pgfpathlineto{\pgfqpoint{6.089192in}{4.007056in}}%
\pgfpathlineto{\pgfqpoint{6.120563in}{4.004242in}}%
\pgfpathlineto{\pgfqpoint{6.151935in}{4.001502in}}%
\pgfpathlineto{\pgfqpoint{6.183306in}{3.998834in}}%
\pgfpathlineto{\pgfqpoint{6.214677in}{3.996237in}}%
\pgfpathlineto{\pgfqpoint{6.246048in}{3.993708in}}%
\pgfpathlineto{\pgfqpoint{6.277419in}{3.991247in}}%
\pgfpathlineto{\pgfqpoint{6.308791in}{3.988852in}}%
\pgfpathlineto{\pgfqpoint{6.340162in}{3.986522in}}%
\pgfpathlineto{\pgfqpoint{6.371533in}{3.984257in}}%
\pgfpathlineto{\pgfqpoint{6.402904in}{3.982054in}}%
\pgfpathlineto{\pgfqpoint{6.434276in}{3.979913in}}%
\pgfpathlineto{\pgfqpoint{6.465647in}{3.977833in}}%
\pgfpathlineto{\pgfqpoint{6.497018in}{3.975812in}}%
\pgfpathlineto{\pgfqpoint{6.528389in}{3.973851in}}%
\pgfpathlineto{\pgfqpoint{6.559760in}{3.971947in}}%
\pgfpathlineto{\pgfqpoint{6.591132in}{3.970101in}}%
\pgfpathlineto{\pgfqpoint{6.622503in}{3.968310in}}%
\pgfpathlineto{\pgfqpoint{6.653874in}{3.966576in}}%
\pgfpathlineto{\pgfqpoint{6.685245in}{3.964896in}}%
\pgfpathlineto{\pgfqpoint{6.716617in}{3.963269in}}%
\pgfpathlineto{\pgfqpoint{6.747988in}{3.961696in}}%
\pgfpathlineto{\pgfqpoint{6.779359in}{3.960176in}}%
\pgfpathlineto{\pgfqpoint{6.810730in}{3.958707in}}%
\pgfpathlineto{\pgfqpoint{6.842102in}{3.957289in}}%
\pgfpathlineto{\pgfqpoint{6.873473in}{3.955921in}}%
\pgfpathlineto{\pgfqpoint{6.904844in}{3.954604in}}%
\pgfpathlineto{\pgfqpoint{6.936215in}{3.953335in}}%
\pgfpathlineto{\pgfqpoint{6.967586in}{3.952115in}}%
\pgfpathlineto{\pgfqpoint{6.998958in}{3.950944in}}%
\pgfpathlineto{\pgfqpoint{7.030329in}{3.949819in}}%
\pgfpathlineto{\pgfqpoint{7.061700in}{3.948742in}}%
\pgfpathlineto{\pgfqpoint{7.093071in}{3.947712in}}%
\pgfpathlineto{\pgfqpoint{7.124443in}{3.946727in}}%
\pgfpathlineto{\pgfqpoint{7.155814in}{3.945788in}}%
\pgfpathlineto{\pgfqpoint{7.187185in}{3.944895in}}%
\pgfpathlineto{\pgfqpoint{7.218556in}{3.944046in}}%
\pgfpathlineto{\pgfqpoint{7.249927in}{3.943241in}}%
\pgfpathlineto{\pgfqpoint{7.281299in}{3.942481in}}%
\pgfpathlineto{\pgfqpoint{7.312670in}{3.941764in}}%
\pgfpathlineto{\pgfqpoint{7.344041in}{3.941091in}}%
\pgfpathlineto{\pgfqpoint{7.375412in}{3.940461in}}%
\pgfpathlineto{\pgfqpoint{7.406784in}{3.939874in}}%
\pgfpathlineto{\pgfqpoint{7.438155in}{3.939330in}}%
\pgfpathlineto{\pgfqpoint{7.469526in}{3.938827in}}%
\pgfpathlineto{\pgfqpoint{7.500897in}{3.938367in}}%
\pgfpathlineto{\pgfqpoint{7.532269in}{3.937949in}}%
\pgfpathlineto{\pgfqpoint{7.563640in}{3.937572in}}%
\pgfpathlineto{\pgfqpoint{7.595011in}{3.937237in}}%
\pgfpathlineto{\pgfqpoint{7.626382in}{3.936944in}}%
\pgfusepath{stroke}%
\end{pgfscope}%
\begin{pgfscope}%
\pgfpathrectangle{\pgfqpoint{5.814694in}{2.849537in}}{\pgfqpoint{1.897959in}{1.372727in}} %
\pgfusepath{clip}%
\pgfsetbuttcap%
\pgfsetmiterjoin%
\definecolor{currentfill}{rgb}{1.000000,0.000000,0.000000}%
\pgfsetfillcolor{currentfill}%
\pgfsetlinewidth{1.003750pt}%
\definecolor{currentstroke}{rgb}{1.000000,0.000000,0.000000}%
\pgfsetstrokecolor{currentstroke}%
\pgfsetdash{}{0pt}%
\pgfsys@defobject{currentmarker}{\pgfqpoint{-0.041667in}{-0.041667in}}{\pgfqpoint{0.041667in}{0.041667in}}{%
\pgfpathmoveto{\pgfqpoint{-0.041667in}{-0.041667in}}%
\pgfpathlineto{\pgfqpoint{0.041667in}{-0.041667in}}%
\pgfpathlineto{\pgfqpoint{0.041667in}{0.041667in}}%
\pgfpathlineto{\pgfqpoint{-0.041667in}{0.041667in}}%
\pgfpathclose%
\pgfusepath{stroke,fill}%
}%
\begin{pgfscope}%
\pgfsys@transformshift{5.900965in}{4.025621in}%
\pgfsys@useobject{currentmarker}{}%
\end{pgfscope}%
\begin{pgfscope}%
\pgfsys@transformshift{6.246048in}{3.993708in}%
\pgfsys@useobject{currentmarker}{}%
\end{pgfscope}%
\begin{pgfscope}%
\pgfsys@transformshift{6.591132in}{3.970101in}%
\pgfsys@useobject{currentmarker}{}%
\end{pgfscope}%
\begin{pgfscope}%
\pgfsys@transformshift{6.936215in}{3.953335in}%
\pgfsys@useobject{currentmarker}{}%
\end{pgfscope}%
\begin{pgfscope}%
\pgfsys@transformshift{7.281299in}{3.942481in}%
\pgfsys@useobject{currentmarker}{}%
\end{pgfscope}%
\begin{pgfscope}%
\pgfsys@transformshift{7.626382in}{3.936944in}%
\pgfsys@useobject{currentmarker}{}%
\end{pgfscope}%
\end{pgfscope}%
\begin{pgfscope}%
\pgfpathrectangle{\pgfqpoint{5.814694in}{2.849537in}}{\pgfqpoint{1.897959in}{1.372727in}} %
\pgfusepath{clip}%
\pgfsetrectcap%
\pgfsetroundjoin%
\pgfsetlinewidth{1.505625pt}%
\definecolor{currentstroke}{rgb}{0.000000,0.000000,1.000000}%
\pgfsetstrokecolor{currentstroke}%
\pgfsetdash{}{0pt}%
\pgfpathmoveto{\pgfqpoint{5.900965in}{3.427663in}}%
\pgfpathlineto{\pgfqpoint{5.932336in}{3.419454in}}%
\pgfpathlineto{\pgfqpoint{5.963707in}{3.411636in}}%
\pgfpathlineto{\pgfqpoint{5.995078in}{3.404129in}}%
\pgfpathlineto{\pgfqpoint{6.026450in}{3.396873in}}%
\pgfpathlineto{\pgfqpoint{6.057821in}{3.389822in}}%
\pgfpathlineto{\pgfqpoint{6.089192in}{3.382937in}}%
\pgfpathlineto{\pgfqpoint{6.120563in}{3.376187in}}%
\pgfpathlineto{\pgfqpoint{6.151935in}{3.369549in}}%
\pgfpathlineto{\pgfqpoint{6.183306in}{3.363000in}}%
\pgfpathlineto{\pgfqpoint{6.214677in}{3.356522in}}%
\pgfpathlineto{\pgfqpoint{6.246048in}{3.350099in}}%
\pgfpathlineto{\pgfqpoint{6.277419in}{3.343718in}}%
\pgfpathlineto{\pgfqpoint{6.308791in}{3.337365in}}%
\pgfpathlineto{\pgfqpoint{6.340162in}{3.331029in}}%
\pgfpathlineto{\pgfqpoint{6.371533in}{3.324700in}}%
\pgfpathlineto{\pgfqpoint{6.402904in}{3.318367in}}%
\pgfpathlineto{\pgfqpoint{6.434276in}{3.312022in}}%
\pgfpathlineto{\pgfqpoint{6.465647in}{3.305656in}}%
\pgfpathlineto{\pgfqpoint{6.497018in}{3.299259in}}%
\pgfpathlineto{\pgfqpoint{6.528389in}{3.292824in}}%
\pgfpathlineto{\pgfqpoint{6.559760in}{3.286341in}}%
\pgfpathlineto{\pgfqpoint{6.591132in}{3.279803in}}%
\pgfpathlineto{\pgfqpoint{6.622503in}{3.273201in}}%
\pgfpathlineto{\pgfqpoint{6.653874in}{3.266526in}}%
\pgfpathlineto{\pgfqpoint{6.685245in}{3.259770in}}%
\pgfpathlineto{\pgfqpoint{6.716617in}{3.252922in}}%
\pgfpathlineto{\pgfqpoint{6.747988in}{3.245974in}}%
\pgfpathlineto{\pgfqpoint{6.779359in}{3.238915in}}%
\pgfpathlineto{\pgfqpoint{6.810730in}{3.231735in}}%
\pgfpathlineto{\pgfqpoint{6.842102in}{3.224421in}}%
\pgfpathlineto{\pgfqpoint{6.873473in}{3.216963in}}%
\pgfpathlineto{\pgfqpoint{6.904844in}{3.209345in}}%
\pgfpathlineto{\pgfqpoint{6.936215in}{3.201555in}}%
\pgfpathlineto{\pgfqpoint{6.967586in}{3.193576in}}%
\pgfpathlineto{\pgfqpoint{6.998958in}{3.185392in}}%
\pgfpathlineto{\pgfqpoint{7.030329in}{3.176984in}}%
\pgfpathlineto{\pgfqpoint{7.061700in}{3.168330in}}%
\pgfpathlineto{\pgfqpoint{7.093071in}{3.159408in}}%
\pgfpathlineto{\pgfqpoint{7.124443in}{3.150191in}}%
\pgfpathlineto{\pgfqpoint{7.155814in}{3.140651in}}%
\pgfpathlineto{\pgfqpoint{7.187185in}{3.130754in}}%
\pgfpathlineto{\pgfqpoint{7.218556in}{3.120462in}}%
\pgfpathlineto{\pgfqpoint{7.249927in}{3.109732in}}%
\pgfpathlineto{\pgfqpoint{7.281299in}{3.098514in}}%
\pgfpathlineto{\pgfqpoint{7.312670in}{3.086749in}}%
\pgfpathlineto{\pgfqpoint{7.344041in}{3.074366in}}%
\pgfpathlineto{\pgfqpoint{7.375412in}{3.061284in}}%
\pgfpathlineto{\pgfqpoint{7.406784in}{3.047402in}}%
\pgfpathlineto{\pgfqpoint{7.438155in}{3.032597in}}%
\pgfpathlineto{\pgfqpoint{7.469526in}{3.016719in}}%
\pgfpathlineto{\pgfqpoint{7.500897in}{2.999575in}}%
\pgfpathlineto{\pgfqpoint{7.532269in}{2.980918in}}%
\pgfpathlineto{\pgfqpoint{7.563640in}{2.960420in}}%
\pgfpathlineto{\pgfqpoint{7.595011in}{2.937636in}}%
\pgfpathlineto{\pgfqpoint{7.626382in}{2.911933in}}%
\pgfusepath{stroke}%
\end{pgfscope}%
\begin{pgfscope}%
\pgfpathrectangle{\pgfqpoint{5.814694in}{2.849537in}}{\pgfqpoint{1.897959in}{1.372727in}} %
\pgfusepath{clip}%
\pgfsetbuttcap%
\pgfsetroundjoin%
\definecolor{currentfill}{rgb}{0.000000,0.000000,1.000000}%
\pgfsetfillcolor{currentfill}%
\pgfsetlinewidth{1.003750pt}%
\definecolor{currentstroke}{rgb}{0.000000,0.000000,1.000000}%
\pgfsetstrokecolor{currentstroke}%
\pgfsetdash{}{0pt}%
\pgfsys@defobject{currentmarker}{\pgfqpoint{-0.041667in}{-0.041667in}}{\pgfqpoint{0.041667in}{0.041667in}}{%
\pgfpathmoveto{\pgfqpoint{0.000000in}{-0.041667in}}%
\pgfpathcurveto{\pgfqpoint{0.011050in}{-0.041667in}}{\pgfqpoint{0.021649in}{-0.037276in}}{\pgfqpoint{0.029463in}{-0.029463in}}%
\pgfpathcurveto{\pgfqpoint{0.037276in}{-0.021649in}}{\pgfqpoint{0.041667in}{-0.011050in}}{\pgfqpoint{0.041667in}{0.000000in}}%
\pgfpathcurveto{\pgfqpoint{0.041667in}{0.011050in}}{\pgfqpoint{0.037276in}{0.021649in}}{\pgfqpoint{0.029463in}{0.029463in}}%
\pgfpathcurveto{\pgfqpoint{0.021649in}{0.037276in}}{\pgfqpoint{0.011050in}{0.041667in}}{\pgfqpoint{0.000000in}{0.041667in}}%
\pgfpathcurveto{\pgfqpoint{-0.011050in}{0.041667in}}{\pgfqpoint{-0.021649in}{0.037276in}}{\pgfqpoint{-0.029463in}{0.029463in}}%
\pgfpathcurveto{\pgfqpoint{-0.037276in}{0.021649in}}{\pgfqpoint{-0.041667in}{0.011050in}}{\pgfqpoint{-0.041667in}{0.000000in}}%
\pgfpathcurveto{\pgfqpoint{-0.041667in}{-0.011050in}}{\pgfqpoint{-0.037276in}{-0.021649in}}{\pgfqpoint{-0.029463in}{-0.029463in}}%
\pgfpathcurveto{\pgfqpoint{-0.021649in}{-0.037276in}}{\pgfqpoint{-0.011050in}{-0.041667in}}{\pgfqpoint{0.000000in}{-0.041667in}}%
\pgfpathclose%
\pgfusepath{stroke,fill}%
}%
\begin{pgfscope}%
\pgfsys@transformshift{5.900965in}{3.427663in}%
\pgfsys@useobject{currentmarker}{}%
\end{pgfscope}%
\begin{pgfscope}%
\pgfsys@transformshift{6.246048in}{3.350099in}%
\pgfsys@useobject{currentmarker}{}%
\end{pgfscope}%
\begin{pgfscope}%
\pgfsys@transformshift{6.591132in}{3.279803in}%
\pgfsys@useobject{currentmarker}{}%
\end{pgfscope}%
\begin{pgfscope}%
\pgfsys@transformshift{6.936215in}{3.201555in}%
\pgfsys@useobject{currentmarker}{}%
\end{pgfscope}%
\begin{pgfscope}%
\pgfsys@transformshift{7.281299in}{3.098514in}%
\pgfsys@useobject{currentmarker}{}%
\end{pgfscope}%
\begin{pgfscope}%
\pgfsys@transformshift{7.626382in}{2.911933in}%
\pgfsys@useobject{currentmarker}{}%
\end{pgfscope}%
\end{pgfscope}%
\begin{pgfscope}%
\pgfpathrectangle{\pgfqpoint{5.814694in}{2.849537in}}{\pgfqpoint{1.897959in}{1.372727in}} %
\pgfusepath{clip}%
\pgfsetbuttcap%
\pgfsetroundjoin%
\pgfsetlinewidth{1.505625pt}%
\definecolor{currentstroke}{rgb}{0.000000,0.750000,0.750000}%
\pgfsetstrokecolor{currentstroke}%
\pgfsetdash{{9.600000pt}{2.400000pt}{1.500000pt}{2.400000pt}}{0.000000pt}%
\pgfpathmoveto{\pgfqpoint{5.900965in}{4.075564in}}%
\pgfpathlineto{\pgfqpoint{5.932336in}{4.054350in}}%
\pgfpathlineto{\pgfqpoint{5.963707in}{4.035181in}}%
\pgfpathlineto{\pgfqpoint{5.995078in}{4.017729in}}%
\pgfpathlineto{\pgfqpoint{6.026450in}{4.001740in}}%
\pgfpathlineto{\pgfqpoint{6.057821in}{3.987013in}}%
\pgfpathlineto{\pgfqpoint{6.089192in}{3.973386in}}%
\pgfpathlineto{\pgfqpoint{6.120563in}{3.960726in}}%
\pgfpathlineto{\pgfqpoint{6.151935in}{3.948925in}}%
\pgfpathlineto{\pgfqpoint{6.183306in}{3.937892in}}%
\pgfpathlineto{\pgfqpoint{6.214677in}{3.927548in}}%
\pgfpathlineto{\pgfqpoint{6.246048in}{3.917828in}}%
\pgfpathlineto{\pgfqpoint{6.277419in}{3.908676in}}%
\pgfpathlineto{\pgfqpoint{6.308791in}{3.900042in}}%
\pgfpathlineto{\pgfqpoint{6.340162in}{3.891884in}}%
\pgfpathlineto{\pgfqpoint{6.371533in}{3.884164in}}%
\pgfpathlineto{\pgfqpoint{6.402904in}{3.876849in}}%
\pgfpathlineto{\pgfqpoint{6.434276in}{3.869910in}}%
\pgfpathlineto{\pgfqpoint{6.465647in}{3.863321in}}%
\pgfpathlineto{\pgfqpoint{6.497018in}{3.857059in}}%
\pgfpathlineto{\pgfqpoint{6.528389in}{3.851102in}}%
\pgfpathlineto{\pgfqpoint{6.559760in}{3.845433in}}%
\pgfpathlineto{\pgfqpoint{6.591132in}{3.840033in}}%
\pgfpathlineto{\pgfqpoint{6.622503in}{3.834888in}}%
\pgfpathlineto{\pgfqpoint{6.653874in}{3.829984in}}%
\pgfpathlineto{\pgfqpoint{6.685245in}{3.825308in}}%
\pgfpathlineto{\pgfqpoint{6.716617in}{3.820848in}}%
\pgfpathlineto{\pgfqpoint{6.747988in}{3.816594in}}%
\pgfpathlineto{\pgfqpoint{6.779359in}{3.812537in}}%
\pgfpathlineto{\pgfqpoint{6.810730in}{3.808667in}}%
\pgfpathlineto{\pgfqpoint{6.842102in}{3.804976in}}%
\pgfpathlineto{\pgfqpoint{6.873473in}{3.801456in}}%
\pgfpathlineto{\pgfqpoint{6.904844in}{3.798101in}}%
\pgfpathlineto{\pgfqpoint{6.936215in}{3.794905in}}%
\pgfpathlineto{\pgfqpoint{6.967586in}{3.791861in}}%
\pgfpathlineto{\pgfqpoint{6.998958in}{3.788963in}}%
\pgfpathlineto{\pgfqpoint{7.030329in}{3.786208in}}%
\pgfpathlineto{\pgfqpoint{7.061700in}{3.783589in}}%
\pgfpathlineto{\pgfqpoint{7.093071in}{3.781103in}}%
\pgfpathlineto{\pgfqpoint{7.124443in}{3.778746in}}%
\pgfpathlineto{\pgfqpoint{7.155814in}{3.776514in}}%
\pgfpathlineto{\pgfqpoint{7.187185in}{3.774403in}}%
\pgfpathlineto{\pgfqpoint{7.218556in}{3.772411in}}%
\pgfpathlineto{\pgfqpoint{7.249927in}{3.770533in}}%
\pgfpathlineto{\pgfqpoint{7.281299in}{3.768769in}}%
\pgfpathlineto{\pgfqpoint{7.312670in}{3.767114in}}%
\pgfpathlineto{\pgfqpoint{7.344041in}{3.765567in}}%
\pgfpathlineto{\pgfqpoint{7.375412in}{3.764126in}}%
\pgfpathlineto{\pgfqpoint{7.406784in}{3.762788in}}%
\pgfpathlineto{\pgfqpoint{7.438155in}{3.761552in}}%
\pgfpathlineto{\pgfqpoint{7.469526in}{3.760416in}}%
\pgfpathlineto{\pgfqpoint{7.500897in}{3.759379in}}%
\pgfpathlineto{\pgfqpoint{7.532269in}{3.758439in}}%
\pgfpathlineto{\pgfqpoint{7.563640in}{3.757594in}}%
\pgfpathlineto{\pgfqpoint{7.595011in}{3.756845in}}%
\pgfpathlineto{\pgfqpoint{7.626382in}{3.756190in}}%
\pgfusepath{stroke}%
\end{pgfscope}%
\begin{pgfscope}%
\pgfpathrectangle{\pgfqpoint{5.814694in}{2.849537in}}{\pgfqpoint{1.897959in}{1.372727in}} %
\pgfusepath{clip}%
\pgfsetbuttcap%
\pgfsetmiterjoin%
\definecolor{currentfill}{rgb}{0.000000,0.750000,0.750000}%
\pgfsetfillcolor{currentfill}%
\pgfsetlinewidth{1.003750pt}%
\definecolor{currentstroke}{rgb}{0.000000,0.750000,0.750000}%
\pgfsetstrokecolor{currentstroke}%
\pgfsetdash{}{0pt}%
\pgfsys@defobject{currentmarker}{\pgfqpoint{-0.041667in}{-0.041667in}}{\pgfqpoint{0.041667in}{0.041667in}}{%
\pgfpathmoveto{\pgfqpoint{-0.000000in}{-0.041667in}}%
\pgfpathlineto{\pgfqpoint{0.041667in}{0.041667in}}%
\pgfpathlineto{\pgfqpoint{-0.041667in}{0.041667in}}%
\pgfpathclose%
\pgfusepath{stroke,fill}%
}%
\begin{pgfscope}%
\pgfsys@transformshift{5.900965in}{4.075564in}%
\pgfsys@useobject{currentmarker}{}%
\end{pgfscope}%
\begin{pgfscope}%
\pgfsys@transformshift{6.246048in}{3.917828in}%
\pgfsys@useobject{currentmarker}{}%
\end{pgfscope}%
\begin{pgfscope}%
\pgfsys@transformshift{6.591132in}{3.840033in}%
\pgfsys@useobject{currentmarker}{}%
\end{pgfscope}%
\begin{pgfscope}%
\pgfsys@transformshift{6.936215in}{3.794905in}%
\pgfsys@useobject{currentmarker}{}%
\end{pgfscope}%
\begin{pgfscope}%
\pgfsys@transformshift{7.281299in}{3.768769in}%
\pgfsys@useobject{currentmarker}{}%
\end{pgfscope}%
\begin{pgfscope}%
\pgfsys@transformshift{7.626382in}{3.756190in}%
\pgfsys@useobject{currentmarker}{}%
\end{pgfscope}%
\end{pgfscope}%
\begin{pgfscope}%
\pgfpathrectangle{\pgfqpoint{5.814694in}{2.849537in}}{\pgfqpoint{1.897959in}{1.372727in}} %
\pgfusepath{clip}%
\pgfsetbuttcap%
\pgfsetroundjoin%
\pgfsetlinewidth{1.505625pt}%
\definecolor{currentstroke}{rgb}{0.000000,0.000000,0.000000}%
\pgfsetstrokecolor{currentstroke}%
\pgfsetdash{{1.500000pt}{2.475000pt}}{0.000000pt}%
\pgfpathmoveto{\pgfqpoint{5.900965in}{4.159867in}}%
\pgfpathlineto{\pgfqpoint{5.932336in}{4.147790in}}%
\pgfpathlineto{\pgfqpoint{5.963707in}{4.135922in}}%
\pgfpathlineto{\pgfqpoint{5.995078in}{4.125385in}}%
\pgfpathlineto{\pgfqpoint{6.026450in}{4.115935in}}%
\pgfpathlineto{\pgfqpoint{6.057821in}{4.107388in}}%
\pgfpathlineto{\pgfqpoint{6.089192in}{4.099599in}}%
\pgfpathlineto{\pgfqpoint{6.120563in}{4.092456in}}%
\pgfpathlineto{\pgfqpoint{6.151935in}{4.085866in}}%
\pgfpathlineto{\pgfqpoint{6.183306in}{4.079756in}}%
\pgfpathlineto{\pgfqpoint{6.214677in}{4.074068in}}%
\pgfpathlineto{\pgfqpoint{6.246048in}{4.068750in}}%
\pgfpathlineto{\pgfqpoint{6.277419in}{4.063763in}}%
\pgfpathlineto{\pgfqpoint{6.308791in}{4.059071in}}%
\pgfpathlineto{\pgfqpoint{6.340162in}{4.054645in}}%
\pgfpathlineto{\pgfqpoint{6.371533in}{4.050461in}}%
\pgfpathlineto{\pgfqpoint{6.402904in}{4.046497in}}%
\pgfpathlineto{\pgfqpoint{6.434276in}{4.042735in}}%
\pgfpathlineto{\pgfqpoint{6.465647in}{4.039159in}}%
\pgfpathlineto{\pgfqpoint{6.497018in}{4.035755in}}%
\pgfpathlineto{\pgfqpoint{6.528389in}{4.032511in}}%
\pgfpathlineto{\pgfqpoint{6.559760in}{4.029417in}}%
\pgfpathlineto{\pgfqpoint{6.591132in}{4.026462in}}%
\pgfpathlineto{\pgfqpoint{6.622503in}{4.023639in}}%
\pgfpathlineto{\pgfqpoint{6.653874in}{4.020940in}}%
\pgfpathlineto{\pgfqpoint{6.685245in}{4.018359in}}%
\pgfpathlineto{\pgfqpoint{6.716617in}{4.015890in}}%
\pgfpathlineto{\pgfqpoint{6.747988in}{4.013526in}}%
\pgfpathlineto{\pgfqpoint{6.779359in}{4.011264in}}%
\pgfpathlineto{\pgfqpoint{6.810730in}{4.009099in}}%
\pgfpathlineto{\pgfqpoint{6.842102in}{4.007027in}}%
\pgfpathlineto{\pgfqpoint{6.873473in}{4.005044in}}%
\pgfpathlineto{\pgfqpoint{6.904844in}{4.003147in}}%
\pgfpathlineto{\pgfqpoint{6.936215in}{4.001333in}}%
\pgfpathlineto{\pgfqpoint{6.967586in}{3.999600in}}%
\pgfpathlineto{\pgfqpoint{6.998958in}{3.997944in}}%
\pgfpathlineto{\pgfqpoint{7.030329in}{3.996364in}}%
\pgfpathlineto{\pgfqpoint{7.061700in}{3.994857in}}%
\pgfpathlineto{\pgfqpoint{7.093071in}{3.993422in}}%
\pgfpathlineto{\pgfqpoint{7.124443in}{3.992056in}}%
\pgfpathlineto{\pgfqpoint{7.155814in}{3.990758in}}%
\pgfpathlineto{\pgfqpoint{7.187185in}{3.989527in}}%
\pgfpathlineto{\pgfqpoint{7.218556in}{3.988360in}}%
\pgfpathlineto{\pgfqpoint{7.249927in}{3.987258in}}%
\pgfpathlineto{\pgfqpoint{7.281299in}{3.986218in}}%
\pgfpathlineto{\pgfqpoint{7.312670in}{3.985239in}}%
\pgfpathlineto{\pgfqpoint{7.344041in}{3.984321in}}%
\pgfpathlineto{\pgfqpoint{7.375412in}{3.983462in}}%
\pgfpathlineto{\pgfqpoint{7.406784in}{3.982662in}}%
\pgfpathlineto{\pgfqpoint{7.438155in}{3.981920in}}%
\pgfpathlineto{\pgfqpoint{7.469526in}{3.981235in}}%
\pgfpathlineto{\pgfqpoint{7.500897in}{3.980607in}}%
\pgfpathlineto{\pgfqpoint{7.532269in}{3.980035in}}%
\pgfpathlineto{\pgfqpoint{7.563640in}{3.979518in}}%
\pgfpathlineto{\pgfqpoint{7.595011in}{3.979057in}}%
\pgfpathlineto{\pgfqpoint{7.626382in}{3.978651in}}%
\pgfusepath{stroke}%
\end{pgfscope}%
\begin{pgfscope}%
\pgfpathrectangle{\pgfqpoint{5.814694in}{2.849537in}}{\pgfqpoint{1.897959in}{1.372727in}} %
\pgfusepath{clip}%
\pgfsetbuttcap%
\pgfsetroundjoin%
\definecolor{currentfill}{rgb}{0.000000,0.000000,0.000000}%
\pgfsetfillcolor{currentfill}%
\pgfsetlinewidth{1.003750pt}%
\definecolor{currentstroke}{rgb}{0.000000,0.000000,0.000000}%
\pgfsetstrokecolor{currentstroke}%
\pgfsetdash{}{0pt}%
\pgfsys@defobject{currentmarker}{\pgfqpoint{-0.041667in}{-0.041667in}}{\pgfqpoint{0.041667in}{0.041667in}}{%
\pgfpathmoveto{\pgfqpoint{-0.041667in}{0.000000in}}%
\pgfpathlineto{\pgfqpoint{0.041667in}{0.000000in}}%
\pgfpathmoveto{\pgfqpoint{0.000000in}{-0.041667in}}%
\pgfpathlineto{\pgfqpoint{0.000000in}{0.041667in}}%
\pgfusepath{stroke,fill}%
}%
\begin{pgfscope}%
\pgfsys@transformshift{5.900965in}{4.159867in}%
\pgfsys@useobject{currentmarker}{}%
\end{pgfscope}%
\begin{pgfscope}%
\pgfsys@transformshift{6.246048in}{4.068750in}%
\pgfsys@useobject{currentmarker}{}%
\end{pgfscope}%
\begin{pgfscope}%
\pgfsys@transformshift{6.591132in}{4.026462in}%
\pgfsys@useobject{currentmarker}{}%
\end{pgfscope}%
\begin{pgfscope}%
\pgfsys@transformshift{6.936215in}{4.001333in}%
\pgfsys@useobject{currentmarker}{}%
\end{pgfscope}%
\begin{pgfscope}%
\pgfsys@transformshift{7.281299in}{3.986218in}%
\pgfsys@useobject{currentmarker}{}%
\end{pgfscope}%
\begin{pgfscope}%
\pgfsys@transformshift{7.626382in}{3.978651in}%
\pgfsys@useobject{currentmarker}{}%
\end{pgfscope}%
\end{pgfscope}%
\begin{pgfscope}%
\pgfsetrectcap%
\pgfsetmiterjoin%
\pgfsetlinewidth{0.803000pt}%
\definecolor{currentstroke}{rgb}{0.000000,0.000000,0.000000}%
\pgfsetstrokecolor{currentstroke}%
\pgfsetdash{}{0pt}%
\pgfpathmoveto{\pgfqpoint{5.814694in}{2.849537in}}%
\pgfpathlineto{\pgfqpoint{5.814694in}{4.222264in}}%
\pgfusepath{stroke}%
\end{pgfscope}%
\begin{pgfscope}%
\pgfsetrectcap%
\pgfsetmiterjoin%
\pgfsetlinewidth{0.803000pt}%
\definecolor{currentstroke}{rgb}{0.000000,0.000000,0.000000}%
\pgfsetstrokecolor{currentstroke}%
\pgfsetdash{}{0pt}%
\pgfpathmoveto{\pgfqpoint{7.712653in}{2.849537in}}%
\pgfpathlineto{\pgfqpoint{7.712653in}{4.222264in}}%
\pgfusepath{stroke}%
\end{pgfscope}%
\begin{pgfscope}%
\pgfsetrectcap%
\pgfsetmiterjoin%
\pgfsetlinewidth{0.803000pt}%
\definecolor{currentstroke}{rgb}{0.000000,0.000000,0.000000}%
\pgfsetstrokecolor{currentstroke}%
\pgfsetdash{}{0pt}%
\pgfpathmoveto{\pgfqpoint{5.814694in}{2.849537in}}%
\pgfpathlineto{\pgfqpoint{7.712653in}{2.849537in}}%
\pgfusepath{stroke}%
\end{pgfscope}%
\begin{pgfscope}%
\pgfsetrectcap%
\pgfsetmiterjoin%
\pgfsetlinewidth{0.803000pt}%
\definecolor{currentstroke}{rgb}{0.000000,0.000000,0.000000}%
\pgfsetstrokecolor{currentstroke}%
\pgfsetdash{}{0pt}%
\pgfpathmoveto{\pgfqpoint{5.814694in}{4.222264in}}%
\pgfpathlineto{\pgfqpoint{7.712653in}{4.222264in}}%
\pgfusepath{stroke}%
\end{pgfscope}%
\begin{pgfscope}%
\pgfsetbuttcap%
\pgfsetmiterjoin%
\definecolor{currentfill}{rgb}{1.000000,1.000000,1.000000}%
\pgfsetfillcolor{currentfill}%
\pgfsetlinewidth{0.000000pt}%
\definecolor{currentstroke}{rgb}{0.000000,0.000000,0.000000}%
\pgfsetstrokecolor{currentstroke}%
\pgfsetstrokeopacity{0.000000}%
\pgfsetdash{}{0pt}%
\pgfpathmoveto{\pgfqpoint{8.282041in}{2.849537in}}%
\pgfpathlineto{\pgfqpoint{10.180000in}{2.849537in}}%
\pgfpathlineto{\pgfqpoint{10.180000in}{4.222264in}}%
\pgfpathlineto{\pgfqpoint{8.282041in}{4.222264in}}%
\pgfpathclose%
\pgfusepath{fill}%
\end{pgfscope}%
\begin{pgfscope}%
\pgfsetbuttcap%
\pgfsetroundjoin%
\definecolor{currentfill}{rgb}{0.000000,0.000000,0.000000}%
\pgfsetfillcolor{currentfill}%
\pgfsetlinewidth{0.803000pt}%
\definecolor{currentstroke}{rgb}{0.000000,0.000000,0.000000}%
\pgfsetstrokecolor{currentstroke}%
\pgfsetdash{}{0pt}%
\pgfsys@defobject{currentmarker}{\pgfqpoint{0.000000in}{-0.048611in}}{\pgfqpoint{0.000000in}{0.000000in}}{%
\pgfpathmoveto{\pgfqpoint{0.000000in}{0.000000in}}%
\pgfpathlineto{\pgfqpoint{0.000000in}{-0.048611in}}%
\pgfusepath{stroke,fill}%
}%
\begin{pgfscope}%
\pgfsys@transformshift{9.128665in}{2.849537in}%
\pgfsys@useobject{currentmarker}{}%
\end{pgfscope}%
\end{pgfscope}%
\begin{pgfscope}%
\pgftext[x=9.128665in,y=2.752315in,,top]{\rmfamily\fontsize{10.000000}{12.000000}\selectfont \(\displaystyle 0.5\)}%
\end{pgfscope}%
\begin{pgfscope}%
\pgfsetbuttcap%
\pgfsetroundjoin%
\definecolor{currentfill}{rgb}{0.000000,0.000000,0.000000}%
\pgfsetfillcolor{currentfill}%
\pgfsetlinewidth{0.803000pt}%
\definecolor{currentstroke}{rgb}{0.000000,0.000000,0.000000}%
\pgfsetstrokecolor{currentstroke}%
\pgfsetdash{}{0pt}%
\pgfsys@defobject{currentmarker}{\pgfqpoint{0.000000in}{-0.048611in}}{\pgfqpoint{0.000000in}{0.000000in}}{%
\pgfpathmoveto{\pgfqpoint{0.000000in}{0.000000in}}%
\pgfpathlineto{\pgfqpoint{0.000000in}{-0.048611in}}%
\pgfusepath{stroke,fill}%
}%
\begin{pgfscope}%
\pgfsys@transformshift{10.005996in}{2.849537in}%
\pgfsys@useobject{currentmarker}{}%
\end{pgfscope}%
\end{pgfscope}%
\begin{pgfscope}%
\pgftext[x=10.005996in,y=2.752315in,,top]{\rmfamily\fontsize{10.000000}{12.000000}\selectfont \(\displaystyle 1.0\)}%
\end{pgfscope}%
\begin{pgfscope}%
\pgfsetbuttcap%
\pgfsetroundjoin%
\definecolor{currentfill}{rgb}{0.000000,0.000000,0.000000}%
\pgfsetfillcolor{currentfill}%
\pgfsetlinewidth{0.803000pt}%
\definecolor{currentstroke}{rgb}{0.000000,0.000000,0.000000}%
\pgfsetstrokecolor{currentstroke}%
\pgfsetdash{}{0pt}%
\pgfsys@defobject{currentmarker}{\pgfqpoint{-0.048611in}{0.000000in}}{\pgfqpoint{0.000000in}{0.000000in}}{%
\pgfpathmoveto{\pgfqpoint{0.000000in}{0.000000in}}%
\pgfpathlineto{\pgfqpoint{-0.048611in}{0.000000in}}%
\pgfusepath{stroke,fill}%
}%
\begin{pgfscope}%
\pgfsys@transformshift{8.282041in}{3.062033in}%
\pgfsys@useobject{currentmarker}{}%
\end{pgfscope}%
\end{pgfscope}%
\begin{pgfscope}%
\pgftext[x=7.896816in,y=3.009271in,left,base]{\rmfamily\fontsize{10.000000}{12.000000}\selectfont \(\displaystyle 10^{-8}\)}%
\end{pgfscope}%
\begin{pgfscope}%
\pgfsetbuttcap%
\pgfsetroundjoin%
\definecolor{currentfill}{rgb}{0.000000,0.000000,0.000000}%
\pgfsetfillcolor{currentfill}%
\pgfsetlinewidth{0.803000pt}%
\definecolor{currentstroke}{rgb}{0.000000,0.000000,0.000000}%
\pgfsetstrokecolor{currentstroke}%
\pgfsetdash{}{0pt}%
\pgfsys@defobject{currentmarker}{\pgfqpoint{-0.048611in}{0.000000in}}{\pgfqpoint{0.000000in}{0.000000in}}{%
\pgfpathmoveto{\pgfqpoint{0.000000in}{0.000000in}}%
\pgfpathlineto{\pgfqpoint{-0.048611in}{0.000000in}}%
\pgfusepath{stroke,fill}%
}%
\begin{pgfscope}%
\pgfsys@transformshift{8.282041in}{3.359739in}%
\pgfsys@useobject{currentmarker}{}%
\end{pgfscope}%
\end{pgfscope}%
\begin{pgfscope}%
\pgftext[x=7.896816in,y=3.306977in,left,base]{\rmfamily\fontsize{10.000000}{12.000000}\selectfont \(\displaystyle 10^{-7}\)}%
\end{pgfscope}%
\begin{pgfscope}%
\pgfsetbuttcap%
\pgfsetroundjoin%
\definecolor{currentfill}{rgb}{0.000000,0.000000,0.000000}%
\pgfsetfillcolor{currentfill}%
\pgfsetlinewidth{0.803000pt}%
\definecolor{currentstroke}{rgb}{0.000000,0.000000,0.000000}%
\pgfsetstrokecolor{currentstroke}%
\pgfsetdash{}{0pt}%
\pgfsys@defobject{currentmarker}{\pgfqpoint{-0.048611in}{0.000000in}}{\pgfqpoint{0.000000in}{0.000000in}}{%
\pgfpathmoveto{\pgfqpoint{0.000000in}{0.000000in}}%
\pgfpathlineto{\pgfqpoint{-0.048611in}{0.000000in}}%
\pgfusepath{stroke,fill}%
}%
\begin{pgfscope}%
\pgfsys@transformshift{8.282041in}{3.657445in}%
\pgfsys@useobject{currentmarker}{}%
\end{pgfscope}%
\end{pgfscope}%
\begin{pgfscope}%
\pgftext[x=7.896816in,y=3.604683in,left,base]{\rmfamily\fontsize{10.000000}{12.000000}\selectfont \(\displaystyle 10^{-6}\)}%
\end{pgfscope}%
\begin{pgfscope}%
\pgfsetbuttcap%
\pgfsetroundjoin%
\definecolor{currentfill}{rgb}{0.000000,0.000000,0.000000}%
\pgfsetfillcolor{currentfill}%
\pgfsetlinewidth{0.803000pt}%
\definecolor{currentstroke}{rgb}{0.000000,0.000000,0.000000}%
\pgfsetstrokecolor{currentstroke}%
\pgfsetdash{}{0pt}%
\pgfsys@defobject{currentmarker}{\pgfqpoint{-0.048611in}{0.000000in}}{\pgfqpoint{0.000000in}{0.000000in}}{%
\pgfpathmoveto{\pgfqpoint{0.000000in}{0.000000in}}%
\pgfpathlineto{\pgfqpoint{-0.048611in}{0.000000in}}%
\pgfusepath{stroke,fill}%
}%
\begin{pgfscope}%
\pgfsys@transformshift{8.282041in}{3.955151in}%
\pgfsys@useobject{currentmarker}{}%
\end{pgfscope}%
\end{pgfscope}%
\begin{pgfscope}%
\pgftext[x=7.896816in,y=3.902390in,left,base]{\rmfamily\fontsize{10.000000}{12.000000}\selectfont \(\displaystyle 10^{-5}\)}%
\end{pgfscope}%
\begin{pgfscope}%
\pgfsetbuttcap%
\pgfsetroundjoin%
\definecolor{currentfill}{rgb}{0.000000,0.000000,0.000000}%
\pgfsetfillcolor{currentfill}%
\pgfsetlinewidth{0.602250pt}%
\definecolor{currentstroke}{rgb}{0.000000,0.000000,0.000000}%
\pgfsetstrokecolor{currentstroke}%
\pgfsetdash{}{0pt}%
\pgfsys@defobject{currentmarker}{\pgfqpoint{-0.027778in}{0.000000in}}{\pgfqpoint{0.000000in}{0.000000in}}{%
\pgfpathmoveto{\pgfqpoint{0.000000in}{0.000000in}}%
\pgfpathlineto{\pgfqpoint{-0.027778in}{0.000000in}}%
\pgfusepath{stroke,fill}%
}%
\begin{pgfscope}%
\pgfsys@transformshift{8.282041in}{2.853945in}%
\pgfsys@useobject{currentmarker}{}%
\end{pgfscope}%
\end{pgfscope}%
\begin{pgfscope}%
\pgfsetbuttcap%
\pgfsetroundjoin%
\definecolor{currentfill}{rgb}{0.000000,0.000000,0.000000}%
\pgfsetfillcolor{currentfill}%
\pgfsetlinewidth{0.602250pt}%
\definecolor{currentstroke}{rgb}{0.000000,0.000000,0.000000}%
\pgfsetstrokecolor{currentstroke}%
\pgfsetdash{}{0pt}%
\pgfsys@defobject{currentmarker}{\pgfqpoint{-0.027778in}{0.000000in}}{\pgfqpoint{0.000000in}{0.000000in}}{%
\pgfpathmoveto{\pgfqpoint{0.000000in}{0.000000in}}%
\pgfpathlineto{\pgfqpoint{-0.027778in}{0.000000in}}%
\pgfusepath{stroke,fill}%
}%
\begin{pgfscope}%
\pgfsys@transformshift{8.282041in}{2.906368in}%
\pgfsys@useobject{currentmarker}{}%
\end{pgfscope}%
\end{pgfscope}%
\begin{pgfscope}%
\pgfsetbuttcap%
\pgfsetroundjoin%
\definecolor{currentfill}{rgb}{0.000000,0.000000,0.000000}%
\pgfsetfillcolor{currentfill}%
\pgfsetlinewidth{0.602250pt}%
\definecolor{currentstroke}{rgb}{0.000000,0.000000,0.000000}%
\pgfsetstrokecolor{currentstroke}%
\pgfsetdash{}{0pt}%
\pgfsys@defobject{currentmarker}{\pgfqpoint{-0.027778in}{0.000000in}}{\pgfqpoint{0.000000in}{0.000000in}}{%
\pgfpathmoveto{\pgfqpoint{0.000000in}{0.000000in}}%
\pgfpathlineto{\pgfqpoint{-0.027778in}{0.000000in}}%
\pgfusepath{stroke,fill}%
}%
\begin{pgfscope}%
\pgfsys@transformshift{8.282041in}{2.943563in}%
\pgfsys@useobject{currentmarker}{}%
\end{pgfscope}%
\end{pgfscope}%
\begin{pgfscope}%
\pgfsetbuttcap%
\pgfsetroundjoin%
\definecolor{currentfill}{rgb}{0.000000,0.000000,0.000000}%
\pgfsetfillcolor{currentfill}%
\pgfsetlinewidth{0.602250pt}%
\definecolor{currentstroke}{rgb}{0.000000,0.000000,0.000000}%
\pgfsetstrokecolor{currentstroke}%
\pgfsetdash{}{0pt}%
\pgfsys@defobject{currentmarker}{\pgfqpoint{-0.027778in}{0.000000in}}{\pgfqpoint{0.000000in}{0.000000in}}{%
\pgfpathmoveto{\pgfqpoint{0.000000in}{0.000000in}}%
\pgfpathlineto{\pgfqpoint{-0.027778in}{0.000000in}}%
\pgfusepath{stroke,fill}%
}%
\begin{pgfscope}%
\pgfsys@transformshift{8.282041in}{2.972414in}%
\pgfsys@useobject{currentmarker}{}%
\end{pgfscope}%
\end{pgfscope}%
\begin{pgfscope}%
\pgfsetbuttcap%
\pgfsetroundjoin%
\definecolor{currentfill}{rgb}{0.000000,0.000000,0.000000}%
\pgfsetfillcolor{currentfill}%
\pgfsetlinewidth{0.602250pt}%
\definecolor{currentstroke}{rgb}{0.000000,0.000000,0.000000}%
\pgfsetstrokecolor{currentstroke}%
\pgfsetdash{}{0pt}%
\pgfsys@defobject{currentmarker}{\pgfqpoint{-0.027778in}{0.000000in}}{\pgfqpoint{0.000000in}{0.000000in}}{%
\pgfpathmoveto{\pgfqpoint{0.000000in}{0.000000in}}%
\pgfpathlineto{\pgfqpoint{-0.027778in}{0.000000in}}%
\pgfusepath{stroke,fill}%
}%
\begin{pgfscope}%
\pgfsys@transformshift{8.282041in}{2.995987in}%
\pgfsys@useobject{currentmarker}{}%
\end{pgfscope}%
\end{pgfscope}%
\begin{pgfscope}%
\pgfsetbuttcap%
\pgfsetroundjoin%
\definecolor{currentfill}{rgb}{0.000000,0.000000,0.000000}%
\pgfsetfillcolor{currentfill}%
\pgfsetlinewidth{0.602250pt}%
\definecolor{currentstroke}{rgb}{0.000000,0.000000,0.000000}%
\pgfsetstrokecolor{currentstroke}%
\pgfsetdash{}{0pt}%
\pgfsys@defobject{currentmarker}{\pgfqpoint{-0.027778in}{0.000000in}}{\pgfqpoint{0.000000in}{0.000000in}}{%
\pgfpathmoveto{\pgfqpoint{0.000000in}{0.000000in}}%
\pgfpathlineto{\pgfqpoint{-0.027778in}{0.000000in}}%
\pgfusepath{stroke,fill}%
}%
\begin{pgfscope}%
\pgfsys@transformshift{8.282041in}{3.015917in}%
\pgfsys@useobject{currentmarker}{}%
\end{pgfscope}%
\end{pgfscope}%
\begin{pgfscope}%
\pgfsetbuttcap%
\pgfsetroundjoin%
\definecolor{currentfill}{rgb}{0.000000,0.000000,0.000000}%
\pgfsetfillcolor{currentfill}%
\pgfsetlinewidth{0.602250pt}%
\definecolor{currentstroke}{rgb}{0.000000,0.000000,0.000000}%
\pgfsetstrokecolor{currentstroke}%
\pgfsetdash{}{0pt}%
\pgfsys@defobject{currentmarker}{\pgfqpoint{-0.027778in}{0.000000in}}{\pgfqpoint{0.000000in}{0.000000in}}{%
\pgfpathmoveto{\pgfqpoint{0.000000in}{0.000000in}}%
\pgfpathlineto{\pgfqpoint{-0.027778in}{0.000000in}}%
\pgfusepath{stroke,fill}%
}%
\begin{pgfscope}%
\pgfsys@transformshift{8.282041in}{3.033182in}%
\pgfsys@useobject{currentmarker}{}%
\end{pgfscope}%
\end{pgfscope}%
\begin{pgfscope}%
\pgfsetbuttcap%
\pgfsetroundjoin%
\definecolor{currentfill}{rgb}{0.000000,0.000000,0.000000}%
\pgfsetfillcolor{currentfill}%
\pgfsetlinewidth{0.602250pt}%
\definecolor{currentstroke}{rgb}{0.000000,0.000000,0.000000}%
\pgfsetstrokecolor{currentstroke}%
\pgfsetdash{}{0pt}%
\pgfsys@defobject{currentmarker}{\pgfqpoint{-0.027778in}{0.000000in}}{\pgfqpoint{0.000000in}{0.000000in}}{%
\pgfpathmoveto{\pgfqpoint{0.000000in}{0.000000in}}%
\pgfpathlineto{\pgfqpoint{-0.027778in}{0.000000in}}%
\pgfusepath{stroke,fill}%
}%
\begin{pgfscope}%
\pgfsys@transformshift{8.282041in}{3.048410in}%
\pgfsys@useobject{currentmarker}{}%
\end{pgfscope}%
\end{pgfscope}%
\begin{pgfscope}%
\pgfsetbuttcap%
\pgfsetroundjoin%
\definecolor{currentfill}{rgb}{0.000000,0.000000,0.000000}%
\pgfsetfillcolor{currentfill}%
\pgfsetlinewidth{0.602250pt}%
\definecolor{currentstroke}{rgb}{0.000000,0.000000,0.000000}%
\pgfsetstrokecolor{currentstroke}%
\pgfsetdash{}{0pt}%
\pgfsys@defobject{currentmarker}{\pgfqpoint{-0.027778in}{0.000000in}}{\pgfqpoint{0.000000in}{0.000000in}}{%
\pgfpathmoveto{\pgfqpoint{0.000000in}{0.000000in}}%
\pgfpathlineto{\pgfqpoint{-0.027778in}{0.000000in}}%
\pgfusepath{stroke,fill}%
}%
\begin{pgfscope}%
\pgfsys@transformshift{8.282041in}{3.151651in}%
\pgfsys@useobject{currentmarker}{}%
\end{pgfscope}%
\end{pgfscope}%
\begin{pgfscope}%
\pgfsetbuttcap%
\pgfsetroundjoin%
\definecolor{currentfill}{rgb}{0.000000,0.000000,0.000000}%
\pgfsetfillcolor{currentfill}%
\pgfsetlinewidth{0.602250pt}%
\definecolor{currentstroke}{rgb}{0.000000,0.000000,0.000000}%
\pgfsetstrokecolor{currentstroke}%
\pgfsetdash{}{0pt}%
\pgfsys@defobject{currentmarker}{\pgfqpoint{-0.027778in}{0.000000in}}{\pgfqpoint{0.000000in}{0.000000in}}{%
\pgfpathmoveto{\pgfqpoint{0.000000in}{0.000000in}}%
\pgfpathlineto{\pgfqpoint{-0.027778in}{0.000000in}}%
\pgfusepath{stroke,fill}%
}%
\begin{pgfscope}%
\pgfsys@transformshift{8.282041in}{3.204074in}%
\pgfsys@useobject{currentmarker}{}%
\end{pgfscope}%
\end{pgfscope}%
\begin{pgfscope}%
\pgfsetbuttcap%
\pgfsetroundjoin%
\definecolor{currentfill}{rgb}{0.000000,0.000000,0.000000}%
\pgfsetfillcolor{currentfill}%
\pgfsetlinewidth{0.602250pt}%
\definecolor{currentstroke}{rgb}{0.000000,0.000000,0.000000}%
\pgfsetstrokecolor{currentstroke}%
\pgfsetdash{}{0pt}%
\pgfsys@defobject{currentmarker}{\pgfqpoint{-0.027778in}{0.000000in}}{\pgfqpoint{0.000000in}{0.000000in}}{%
\pgfpathmoveto{\pgfqpoint{0.000000in}{0.000000in}}%
\pgfpathlineto{\pgfqpoint{-0.027778in}{0.000000in}}%
\pgfusepath{stroke,fill}%
}%
\begin{pgfscope}%
\pgfsys@transformshift{8.282041in}{3.241269in}%
\pgfsys@useobject{currentmarker}{}%
\end{pgfscope}%
\end{pgfscope}%
\begin{pgfscope}%
\pgfsetbuttcap%
\pgfsetroundjoin%
\definecolor{currentfill}{rgb}{0.000000,0.000000,0.000000}%
\pgfsetfillcolor{currentfill}%
\pgfsetlinewidth{0.602250pt}%
\definecolor{currentstroke}{rgb}{0.000000,0.000000,0.000000}%
\pgfsetstrokecolor{currentstroke}%
\pgfsetdash{}{0pt}%
\pgfsys@defobject{currentmarker}{\pgfqpoint{-0.027778in}{0.000000in}}{\pgfqpoint{0.000000in}{0.000000in}}{%
\pgfpathmoveto{\pgfqpoint{0.000000in}{0.000000in}}%
\pgfpathlineto{\pgfqpoint{-0.027778in}{0.000000in}}%
\pgfusepath{stroke,fill}%
}%
\begin{pgfscope}%
\pgfsys@transformshift{8.282041in}{3.270120in}%
\pgfsys@useobject{currentmarker}{}%
\end{pgfscope}%
\end{pgfscope}%
\begin{pgfscope}%
\pgfsetbuttcap%
\pgfsetroundjoin%
\definecolor{currentfill}{rgb}{0.000000,0.000000,0.000000}%
\pgfsetfillcolor{currentfill}%
\pgfsetlinewidth{0.602250pt}%
\definecolor{currentstroke}{rgb}{0.000000,0.000000,0.000000}%
\pgfsetstrokecolor{currentstroke}%
\pgfsetdash{}{0pt}%
\pgfsys@defobject{currentmarker}{\pgfqpoint{-0.027778in}{0.000000in}}{\pgfqpoint{0.000000in}{0.000000in}}{%
\pgfpathmoveto{\pgfqpoint{0.000000in}{0.000000in}}%
\pgfpathlineto{\pgfqpoint{-0.027778in}{0.000000in}}%
\pgfusepath{stroke,fill}%
}%
\begin{pgfscope}%
\pgfsys@transformshift{8.282041in}{3.293693in}%
\pgfsys@useobject{currentmarker}{}%
\end{pgfscope}%
\end{pgfscope}%
\begin{pgfscope}%
\pgfsetbuttcap%
\pgfsetroundjoin%
\definecolor{currentfill}{rgb}{0.000000,0.000000,0.000000}%
\pgfsetfillcolor{currentfill}%
\pgfsetlinewidth{0.602250pt}%
\definecolor{currentstroke}{rgb}{0.000000,0.000000,0.000000}%
\pgfsetstrokecolor{currentstroke}%
\pgfsetdash{}{0pt}%
\pgfsys@defobject{currentmarker}{\pgfqpoint{-0.027778in}{0.000000in}}{\pgfqpoint{0.000000in}{0.000000in}}{%
\pgfpathmoveto{\pgfqpoint{0.000000in}{0.000000in}}%
\pgfpathlineto{\pgfqpoint{-0.027778in}{0.000000in}}%
\pgfusepath{stroke,fill}%
}%
\begin{pgfscope}%
\pgfsys@transformshift{8.282041in}{3.313623in}%
\pgfsys@useobject{currentmarker}{}%
\end{pgfscope}%
\end{pgfscope}%
\begin{pgfscope}%
\pgfsetbuttcap%
\pgfsetroundjoin%
\definecolor{currentfill}{rgb}{0.000000,0.000000,0.000000}%
\pgfsetfillcolor{currentfill}%
\pgfsetlinewidth{0.602250pt}%
\definecolor{currentstroke}{rgb}{0.000000,0.000000,0.000000}%
\pgfsetstrokecolor{currentstroke}%
\pgfsetdash{}{0pt}%
\pgfsys@defobject{currentmarker}{\pgfqpoint{-0.027778in}{0.000000in}}{\pgfqpoint{0.000000in}{0.000000in}}{%
\pgfpathmoveto{\pgfqpoint{0.000000in}{0.000000in}}%
\pgfpathlineto{\pgfqpoint{-0.027778in}{0.000000in}}%
\pgfusepath{stroke,fill}%
}%
\begin{pgfscope}%
\pgfsys@transformshift{8.282041in}{3.330888in}%
\pgfsys@useobject{currentmarker}{}%
\end{pgfscope}%
\end{pgfscope}%
\begin{pgfscope}%
\pgfsetbuttcap%
\pgfsetroundjoin%
\definecolor{currentfill}{rgb}{0.000000,0.000000,0.000000}%
\pgfsetfillcolor{currentfill}%
\pgfsetlinewidth{0.602250pt}%
\definecolor{currentstroke}{rgb}{0.000000,0.000000,0.000000}%
\pgfsetstrokecolor{currentstroke}%
\pgfsetdash{}{0pt}%
\pgfsys@defobject{currentmarker}{\pgfqpoint{-0.027778in}{0.000000in}}{\pgfqpoint{0.000000in}{0.000000in}}{%
\pgfpathmoveto{\pgfqpoint{0.000000in}{0.000000in}}%
\pgfpathlineto{\pgfqpoint{-0.027778in}{0.000000in}}%
\pgfusepath{stroke,fill}%
}%
\begin{pgfscope}%
\pgfsys@transformshift{8.282041in}{3.346116in}%
\pgfsys@useobject{currentmarker}{}%
\end{pgfscope}%
\end{pgfscope}%
\begin{pgfscope}%
\pgfsetbuttcap%
\pgfsetroundjoin%
\definecolor{currentfill}{rgb}{0.000000,0.000000,0.000000}%
\pgfsetfillcolor{currentfill}%
\pgfsetlinewidth{0.602250pt}%
\definecolor{currentstroke}{rgb}{0.000000,0.000000,0.000000}%
\pgfsetstrokecolor{currentstroke}%
\pgfsetdash{}{0pt}%
\pgfsys@defobject{currentmarker}{\pgfqpoint{-0.027778in}{0.000000in}}{\pgfqpoint{0.000000in}{0.000000in}}{%
\pgfpathmoveto{\pgfqpoint{0.000000in}{0.000000in}}%
\pgfpathlineto{\pgfqpoint{-0.027778in}{0.000000in}}%
\pgfusepath{stroke,fill}%
}%
\begin{pgfscope}%
\pgfsys@transformshift{8.282041in}{3.449357in}%
\pgfsys@useobject{currentmarker}{}%
\end{pgfscope}%
\end{pgfscope}%
\begin{pgfscope}%
\pgfsetbuttcap%
\pgfsetroundjoin%
\definecolor{currentfill}{rgb}{0.000000,0.000000,0.000000}%
\pgfsetfillcolor{currentfill}%
\pgfsetlinewidth{0.602250pt}%
\definecolor{currentstroke}{rgb}{0.000000,0.000000,0.000000}%
\pgfsetstrokecolor{currentstroke}%
\pgfsetdash{}{0pt}%
\pgfsys@defobject{currentmarker}{\pgfqpoint{-0.027778in}{0.000000in}}{\pgfqpoint{0.000000in}{0.000000in}}{%
\pgfpathmoveto{\pgfqpoint{0.000000in}{0.000000in}}%
\pgfpathlineto{\pgfqpoint{-0.027778in}{0.000000in}}%
\pgfusepath{stroke,fill}%
}%
\begin{pgfscope}%
\pgfsys@transformshift{8.282041in}{3.501781in}%
\pgfsys@useobject{currentmarker}{}%
\end{pgfscope}%
\end{pgfscope}%
\begin{pgfscope}%
\pgfsetbuttcap%
\pgfsetroundjoin%
\definecolor{currentfill}{rgb}{0.000000,0.000000,0.000000}%
\pgfsetfillcolor{currentfill}%
\pgfsetlinewidth{0.602250pt}%
\definecolor{currentstroke}{rgb}{0.000000,0.000000,0.000000}%
\pgfsetstrokecolor{currentstroke}%
\pgfsetdash{}{0pt}%
\pgfsys@defobject{currentmarker}{\pgfqpoint{-0.027778in}{0.000000in}}{\pgfqpoint{0.000000in}{0.000000in}}{%
\pgfpathmoveto{\pgfqpoint{0.000000in}{0.000000in}}%
\pgfpathlineto{\pgfqpoint{-0.027778in}{0.000000in}}%
\pgfusepath{stroke,fill}%
}%
\begin{pgfscope}%
\pgfsys@transformshift{8.282041in}{3.538976in}%
\pgfsys@useobject{currentmarker}{}%
\end{pgfscope}%
\end{pgfscope}%
\begin{pgfscope}%
\pgfsetbuttcap%
\pgfsetroundjoin%
\definecolor{currentfill}{rgb}{0.000000,0.000000,0.000000}%
\pgfsetfillcolor{currentfill}%
\pgfsetlinewidth{0.602250pt}%
\definecolor{currentstroke}{rgb}{0.000000,0.000000,0.000000}%
\pgfsetstrokecolor{currentstroke}%
\pgfsetdash{}{0pt}%
\pgfsys@defobject{currentmarker}{\pgfqpoint{-0.027778in}{0.000000in}}{\pgfqpoint{0.000000in}{0.000000in}}{%
\pgfpathmoveto{\pgfqpoint{0.000000in}{0.000000in}}%
\pgfpathlineto{\pgfqpoint{-0.027778in}{0.000000in}}%
\pgfusepath{stroke,fill}%
}%
\begin{pgfscope}%
\pgfsys@transformshift{8.282041in}{3.567826in}%
\pgfsys@useobject{currentmarker}{}%
\end{pgfscope}%
\end{pgfscope}%
\begin{pgfscope}%
\pgfsetbuttcap%
\pgfsetroundjoin%
\definecolor{currentfill}{rgb}{0.000000,0.000000,0.000000}%
\pgfsetfillcolor{currentfill}%
\pgfsetlinewidth{0.602250pt}%
\definecolor{currentstroke}{rgb}{0.000000,0.000000,0.000000}%
\pgfsetstrokecolor{currentstroke}%
\pgfsetdash{}{0pt}%
\pgfsys@defobject{currentmarker}{\pgfqpoint{-0.027778in}{0.000000in}}{\pgfqpoint{0.000000in}{0.000000in}}{%
\pgfpathmoveto{\pgfqpoint{0.000000in}{0.000000in}}%
\pgfpathlineto{\pgfqpoint{-0.027778in}{0.000000in}}%
\pgfusepath{stroke,fill}%
}%
\begin{pgfscope}%
\pgfsys@transformshift{8.282041in}{3.591399in}%
\pgfsys@useobject{currentmarker}{}%
\end{pgfscope}%
\end{pgfscope}%
\begin{pgfscope}%
\pgfsetbuttcap%
\pgfsetroundjoin%
\definecolor{currentfill}{rgb}{0.000000,0.000000,0.000000}%
\pgfsetfillcolor{currentfill}%
\pgfsetlinewidth{0.602250pt}%
\definecolor{currentstroke}{rgb}{0.000000,0.000000,0.000000}%
\pgfsetstrokecolor{currentstroke}%
\pgfsetdash{}{0pt}%
\pgfsys@defobject{currentmarker}{\pgfqpoint{-0.027778in}{0.000000in}}{\pgfqpoint{0.000000in}{0.000000in}}{%
\pgfpathmoveto{\pgfqpoint{0.000000in}{0.000000in}}%
\pgfpathlineto{\pgfqpoint{-0.027778in}{0.000000in}}%
\pgfusepath{stroke,fill}%
}%
\begin{pgfscope}%
\pgfsys@transformshift{8.282041in}{3.611330in}%
\pgfsys@useobject{currentmarker}{}%
\end{pgfscope}%
\end{pgfscope}%
\begin{pgfscope}%
\pgfsetbuttcap%
\pgfsetroundjoin%
\definecolor{currentfill}{rgb}{0.000000,0.000000,0.000000}%
\pgfsetfillcolor{currentfill}%
\pgfsetlinewidth{0.602250pt}%
\definecolor{currentstroke}{rgb}{0.000000,0.000000,0.000000}%
\pgfsetstrokecolor{currentstroke}%
\pgfsetdash{}{0pt}%
\pgfsys@defobject{currentmarker}{\pgfqpoint{-0.027778in}{0.000000in}}{\pgfqpoint{0.000000in}{0.000000in}}{%
\pgfpathmoveto{\pgfqpoint{0.000000in}{0.000000in}}%
\pgfpathlineto{\pgfqpoint{-0.027778in}{0.000000in}}%
\pgfusepath{stroke,fill}%
}%
\begin{pgfscope}%
\pgfsys@transformshift{8.282041in}{3.628594in}%
\pgfsys@useobject{currentmarker}{}%
\end{pgfscope}%
\end{pgfscope}%
\begin{pgfscope}%
\pgfsetbuttcap%
\pgfsetroundjoin%
\definecolor{currentfill}{rgb}{0.000000,0.000000,0.000000}%
\pgfsetfillcolor{currentfill}%
\pgfsetlinewidth{0.602250pt}%
\definecolor{currentstroke}{rgb}{0.000000,0.000000,0.000000}%
\pgfsetstrokecolor{currentstroke}%
\pgfsetdash{}{0pt}%
\pgfsys@defobject{currentmarker}{\pgfqpoint{-0.027778in}{0.000000in}}{\pgfqpoint{0.000000in}{0.000000in}}{%
\pgfpathmoveto{\pgfqpoint{0.000000in}{0.000000in}}%
\pgfpathlineto{\pgfqpoint{-0.027778in}{0.000000in}}%
\pgfusepath{stroke,fill}%
}%
\begin{pgfscope}%
\pgfsys@transformshift{8.282041in}{3.643823in}%
\pgfsys@useobject{currentmarker}{}%
\end{pgfscope}%
\end{pgfscope}%
\begin{pgfscope}%
\pgfsetbuttcap%
\pgfsetroundjoin%
\definecolor{currentfill}{rgb}{0.000000,0.000000,0.000000}%
\pgfsetfillcolor{currentfill}%
\pgfsetlinewidth{0.602250pt}%
\definecolor{currentstroke}{rgb}{0.000000,0.000000,0.000000}%
\pgfsetstrokecolor{currentstroke}%
\pgfsetdash{}{0pt}%
\pgfsys@defobject{currentmarker}{\pgfqpoint{-0.027778in}{0.000000in}}{\pgfqpoint{0.000000in}{0.000000in}}{%
\pgfpathmoveto{\pgfqpoint{0.000000in}{0.000000in}}%
\pgfpathlineto{\pgfqpoint{-0.027778in}{0.000000in}}%
\pgfusepath{stroke,fill}%
}%
\begin{pgfscope}%
\pgfsys@transformshift{8.282041in}{3.747063in}%
\pgfsys@useobject{currentmarker}{}%
\end{pgfscope}%
\end{pgfscope}%
\begin{pgfscope}%
\pgfsetbuttcap%
\pgfsetroundjoin%
\definecolor{currentfill}{rgb}{0.000000,0.000000,0.000000}%
\pgfsetfillcolor{currentfill}%
\pgfsetlinewidth{0.602250pt}%
\definecolor{currentstroke}{rgb}{0.000000,0.000000,0.000000}%
\pgfsetstrokecolor{currentstroke}%
\pgfsetdash{}{0pt}%
\pgfsys@defobject{currentmarker}{\pgfqpoint{-0.027778in}{0.000000in}}{\pgfqpoint{0.000000in}{0.000000in}}{%
\pgfpathmoveto{\pgfqpoint{0.000000in}{0.000000in}}%
\pgfpathlineto{\pgfqpoint{-0.027778in}{0.000000in}}%
\pgfusepath{stroke,fill}%
}%
\begin{pgfscope}%
\pgfsys@transformshift{8.282041in}{3.799487in}%
\pgfsys@useobject{currentmarker}{}%
\end{pgfscope}%
\end{pgfscope}%
\begin{pgfscope}%
\pgfsetbuttcap%
\pgfsetroundjoin%
\definecolor{currentfill}{rgb}{0.000000,0.000000,0.000000}%
\pgfsetfillcolor{currentfill}%
\pgfsetlinewidth{0.602250pt}%
\definecolor{currentstroke}{rgb}{0.000000,0.000000,0.000000}%
\pgfsetstrokecolor{currentstroke}%
\pgfsetdash{}{0pt}%
\pgfsys@defobject{currentmarker}{\pgfqpoint{-0.027778in}{0.000000in}}{\pgfqpoint{0.000000in}{0.000000in}}{%
\pgfpathmoveto{\pgfqpoint{0.000000in}{0.000000in}}%
\pgfpathlineto{\pgfqpoint{-0.027778in}{0.000000in}}%
\pgfusepath{stroke,fill}%
}%
\begin{pgfscope}%
\pgfsys@transformshift{8.282041in}{3.836682in}%
\pgfsys@useobject{currentmarker}{}%
\end{pgfscope}%
\end{pgfscope}%
\begin{pgfscope}%
\pgfsetbuttcap%
\pgfsetroundjoin%
\definecolor{currentfill}{rgb}{0.000000,0.000000,0.000000}%
\pgfsetfillcolor{currentfill}%
\pgfsetlinewidth{0.602250pt}%
\definecolor{currentstroke}{rgb}{0.000000,0.000000,0.000000}%
\pgfsetstrokecolor{currentstroke}%
\pgfsetdash{}{0pt}%
\pgfsys@defobject{currentmarker}{\pgfqpoint{-0.027778in}{0.000000in}}{\pgfqpoint{0.000000in}{0.000000in}}{%
\pgfpathmoveto{\pgfqpoint{0.000000in}{0.000000in}}%
\pgfpathlineto{\pgfqpoint{-0.027778in}{0.000000in}}%
\pgfusepath{stroke,fill}%
}%
\begin{pgfscope}%
\pgfsys@transformshift{8.282041in}{3.865533in}%
\pgfsys@useobject{currentmarker}{}%
\end{pgfscope}%
\end{pgfscope}%
\begin{pgfscope}%
\pgfsetbuttcap%
\pgfsetroundjoin%
\definecolor{currentfill}{rgb}{0.000000,0.000000,0.000000}%
\pgfsetfillcolor{currentfill}%
\pgfsetlinewidth{0.602250pt}%
\definecolor{currentstroke}{rgb}{0.000000,0.000000,0.000000}%
\pgfsetstrokecolor{currentstroke}%
\pgfsetdash{}{0pt}%
\pgfsys@defobject{currentmarker}{\pgfqpoint{-0.027778in}{0.000000in}}{\pgfqpoint{0.000000in}{0.000000in}}{%
\pgfpathmoveto{\pgfqpoint{0.000000in}{0.000000in}}%
\pgfpathlineto{\pgfqpoint{-0.027778in}{0.000000in}}%
\pgfusepath{stroke,fill}%
}%
\begin{pgfscope}%
\pgfsys@transformshift{8.282041in}{3.889105in}%
\pgfsys@useobject{currentmarker}{}%
\end{pgfscope}%
\end{pgfscope}%
\begin{pgfscope}%
\pgfsetbuttcap%
\pgfsetroundjoin%
\definecolor{currentfill}{rgb}{0.000000,0.000000,0.000000}%
\pgfsetfillcolor{currentfill}%
\pgfsetlinewidth{0.602250pt}%
\definecolor{currentstroke}{rgb}{0.000000,0.000000,0.000000}%
\pgfsetstrokecolor{currentstroke}%
\pgfsetdash{}{0pt}%
\pgfsys@defobject{currentmarker}{\pgfqpoint{-0.027778in}{0.000000in}}{\pgfqpoint{0.000000in}{0.000000in}}{%
\pgfpathmoveto{\pgfqpoint{0.000000in}{0.000000in}}%
\pgfpathlineto{\pgfqpoint{-0.027778in}{0.000000in}}%
\pgfusepath{stroke,fill}%
}%
\begin{pgfscope}%
\pgfsys@transformshift{8.282041in}{3.909036in}%
\pgfsys@useobject{currentmarker}{}%
\end{pgfscope}%
\end{pgfscope}%
\begin{pgfscope}%
\pgfsetbuttcap%
\pgfsetroundjoin%
\definecolor{currentfill}{rgb}{0.000000,0.000000,0.000000}%
\pgfsetfillcolor{currentfill}%
\pgfsetlinewidth{0.602250pt}%
\definecolor{currentstroke}{rgb}{0.000000,0.000000,0.000000}%
\pgfsetstrokecolor{currentstroke}%
\pgfsetdash{}{0pt}%
\pgfsys@defobject{currentmarker}{\pgfqpoint{-0.027778in}{0.000000in}}{\pgfqpoint{0.000000in}{0.000000in}}{%
\pgfpathmoveto{\pgfqpoint{0.000000in}{0.000000in}}%
\pgfpathlineto{\pgfqpoint{-0.027778in}{0.000000in}}%
\pgfusepath{stroke,fill}%
}%
\begin{pgfscope}%
\pgfsys@transformshift{8.282041in}{3.926300in}%
\pgfsys@useobject{currentmarker}{}%
\end{pgfscope}%
\end{pgfscope}%
\begin{pgfscope}%
\pgfsetbuttcap%
\pgfsetroundjoin%
\definecolor{currentfill}{rgb}{0.000000,0.000000,0.000000}%
\pgfsetfillcolor{currentfill}%
\pgfsetlinewidth{0.602250pt}%
\definecolor{currentstroke}{rgb}{0.000000,0.000000,0.000000}%
\pgfsetstrokecolor{currentstroke}%
\pgfsetdash{}{0pt}%
\pgfsys@defobject{currentmarker}{\pgfqpoint{-0.027778in}{0.000000in}}{\pgfqpoint{0.000000in}{0.000000in}}{%
\pgfpathmoveto{\pgfqpoint{0.000000in}{0.000000in}}%
\pgfpathlineto{\pgfqpoint{-0.027778in}{0.000000in}}%
\pgfusepath{stroke,fill}%
}%
\begin{pgfscope}%
\pgfsys@transformshift{8.282041in}{3.941529in}%
\pgfsys@useobject{currentmarker}{}%
\end{pgfscope}%
\end{pgfscope}%
\begin{pgfscope}%
\pgfsetbuttcap%
\pgfsetroundjoin%
\definecolor{currentfill}{rgb}{0.000000,0.000000,0.000000}%
\pgfsetfillcolor{currentfill}%
\pgfsetlinewidth{0.602250pt}%
\definecolor{currentstroke}{rgb}{0.000000,0.000000,0.000000}%
\pgfsetstrokecolor{currentstroke}%
\pgfsetdash{}{0pt}%
\pgfsys@defobject{currentmarker}{\pgfqpoint{-0.027778in}{0.000000in}}{\pgfqpoint{0.000000in}{0.000000in}}{%
\pgfpathmoveto{\pgfqpoint{0.000000in}{0.000000in}}%
\pgfpathlineto{\pgfqpoint{-0.027778in}{0.000000in}}%
\pgfusepath{stroke,fill}%
}%
\begin{pgfscope}%
\pgfsys@transformshift{8.282041in}{4.044770in}%
\pgfsys@useobject{currentmarker}{}%
\end{pgfscope}%
\end{pgfscope}%
\begin{pgfscope}%
\pgfsetbuttcap%
\pgfsetroundjoin%
\definecolor{currentfill}{rgb}{0.000000,0.000000,0.000000}%
\pgfsetfillcolor{currentfill}%
\pgfsetlinewidth{0.602250pt}%
\definecolor{currentstroke}{rgb}{0.000000,0.000000,0.000000}%
\pgfsetstrokecolor{currentstroke}%
\pgfsetdash{}{0pt}%
\pgfsys@defobject{currentmarker}{\pgfqpoint{-0.027778in}{0.000000in}}{\pgfqpoint{0.000000in}{0.000000in}}{%
\pgfpathmoveto{\pgfqpoint{0.000000in}{0.000000in}}%
\pgfpathlineto{\pgfqpoint{-0.027778in}{0.000000in}}%
\pgfusepath{stroke,fill}%
}%
\begin{pgfscope}%
\pgfsys@transformshift{8.282041in}{4.097193in}%
\pgfsys@useobject{currentmarker}{}%
\end{pgfscope}%
\end{pgfscope}%
\begin{pgfscope}%
\pgfsetbuttcap%
\pgfsetroundjoin%
\definecolor{currentfill}{rgb}{0.000000,0.000000,0.000000}%
\pgfsetfillcolor{currentfill}%
\pgfsetlinewidth{0.602250pt}%
\definecolor{currentstroke}{rgb}{0.000000,0.000000,0.000000}%
\pgfsetstrokecolor{currentstroke}%
\pgfsetdash{}{0pt}%
\pgfsys@defobject{currentmarker}{\pgfqpoint{-0.027778in}{0.000000in}}{\pgfqpoint{0.000000in}{0.000000in}}{%
\pgfpathmoveto{\pgfqpoint{0.000000in}{0.000000in}}%
\pgfpathlineto{\pgfqpoint{-0.027778in}{0.000000in}}%
\pgfusepath{stroke,fill}%
}%
\begin{pgfscope}%
\pgfsys@transformshift{8.282041in}{4.134388in}%
\pgfsys@useobject{currentmarker}{}%
\end{pgfscope}%
\end{pgfscope}%
\begin{pgfscope}%
\pgfsetbuttcap%
\pgfsetroundjoin%
\definecolor{currentfill}{rgb}{0.000000,0.000000,0.000000}%
\pgfsetfillcolor{currentfill}%
\pgfsetlinewidth{0.602250pt}%
\definecolor{currentstroke}{rgb}{0.000000,0.000000,0.000000}%
\pgfsetstrokecolor{currentstroke}%
\pgfsetdash{}{0pt}%
\pgfsys@defobject{currentmarker}{\pgfqpoint{-0.027778in}{0.000000in}}{\pgfqpoint{0.000000in}{0.000000in}}{%
\pgfpathmoveto{\pgfqpoint{0.000000in}{0.000000in}}%
\pgfpathlineto{\pgfqpoint{-0.027778in}{0.000000in}}%
\pgfusepath{stroke,fill}%
}%
\begin{pgfscope}%
\pgfsys@transformshift{8.282041in}{4.163239in}%
\pgfsys@useobject{currentmarker}{}%
\end{pgfscope}%
\end{pgfscope}%
\begin{pgfscope}%
\pgfsetbuttcap%
\pgfsetroundjoin%
\definecolor{currentfill}{rgb}{0.000000,0.000000,0.000000}%
\pgfsetfillcolor{currentfill}%
\pgfsetlinewidth{0.602250pt}%
\definecolor{currentstroke}{rgb}{0.000000,0.000000,0.000000}%
\pgfsetstrokecolor{currentstroke}%
\pgfsetdash{}{0pt}%
\pgfsys@defobject{currentmarker}{\pgfqpoint{-0.027778in}{0.000000in}}{\pgfqpoint{0.000000in}{0.000000in}}{%
\pgfpathmoveto{\pgfqpoint{0.000000in}{0.000000in}}%
\pgfpathlineto{\pgfqpoint{-0.027778in}{0.000000in}}%
\pgfusepath{stroke,fill}%
}%
\begin{pgfscope}%
\pgfsys@transformshift{8.282041in}{4.186812in}%
\pgfsys@useobject{currentmarker}{}%
\end{pgfscope}%
\end{pgfscope}%
\begin{pgfscope}%
\pgfsetbuttcap%
\pgfsetroundjoin%
\definecolor{currentfill}{rgb}{0.000000,0.000000,0.000000}%
\pgfsetfillcolor{currentfill}%
\pgfsetlinewidth{0.602250pt}%
\definecolor{currentstroke}{rgb}{0.000000,0.000000,0.000000}%
\pgfsetstrokecolor{currentstroke}%
\pgfsetdash{}{0pt}%
\pgfsys@defobject{currentmarker}{\pgfqpoint{-0.027778in}{0.000000in}}{\pgfqpoint{0.000000in}{0.000000in}}{%
\pgfpathmoveto{\pgfqpoint{0.000000in}{0.000000in}}%
\pgfpathlineto{\pgfqpoint{-0.027778in}{0.000000in}}%
\pgfusepath{stroke,fill}%
}%
\begin{pgfscope}%
\pgfsys@transformshift{8.282041in}{4.206742in}%
\pgfsys@useobject{currentmarker}{}%
\end{pgfscope}%
\end{pgfscope}%
\begin{pgfscope}%
\pgfsetbuttcap%
\pgfsetroundjoin%
\definecolor{currentfill}{rgb}{0.000000,0.000000,0.000000}%
\pgfsetfillcolor{currentfill}%
\pgfsetlinewidth{0.602250pt}%
\definecolor{currentstroke}{rgb}{0.000000,0.000000,0.000000}%
\pgfsetstrokecolor{currentstroke}%
\pgfsetdash{}{0pt}%
\pgfsys@defobject{currentmarker}{\pgfqpoint{-0.027778in}{0.000000in}}{\pgfqpoint{0.000000in}{0.000000in}}{%
\pgfpathmoveto{\pgfqpoint{0.000000in}{0.000000in}}%
\pgfpathlineto{\pgfqpoint{-0.027778in}{0.000000in}}%
\pgfusepath{stroke,fill}%
}%
\begin{pgfscope}%
\pgfsys@transformshift{8.282041in}{4.224007in}%
\pgfsys@useobject{currentmarker}{}%
\end{pgfscope}%
\end{pgfscope}%
\begin{pgfscope}%
\pgfpathrectangle{\pgfqpoint{8.282041in}{2.849537in}}{\pgfqpoint{1.897959in}{1.372727in}} %
\pgfusepath{clip}%
\pgfsetbuttcap%
\pgfsetroundjoin%
\pgfsetlinewidth{1.505625pt}%
\definecolor{currentstroke}{rgb}{1.000000,0.000000,0.000000}%
\pgfsetstrokecolor{currentstroke}%
\pgfsetdash{{5.550000pt}{2.400000pt}}{0.000000pt}%
\pgfpathmoveto{\pgfqpoint{8.368312in}{4.078457in}}%
\pgfpathlineto{\pgfqpoint{8.382934in}{4.075878in}}%
\pgfpathlineto{\pgfqpoint{8.397556in}{4.073335in}}%
\pgfpathlineto{\pgfqpoint{8.412178in}{4.070827in}}%
\pgfpathlineto{\pgfqpoint{8.426800in}{4.068351in}}%
\pgfpathlineto{\pgfqpoint{8.441423in}{4.065908in}}%
\pgfpathlineto{\pgfqpoint{8.456045in}{4.063497in}}%
\pgfpathlineto{\pgfqpoint{8.470667in}{4.061118in}}%
\pgfpathlineto{\pgfqpoint{8.485289in}{4.058771in}}%
\pgfpathlineto{\pgfqpoint{8.499911in}{4.056455in}}%
\pgfpathlineto{\pgfqpoint{8.514534in}{4.054170in}}%
\pgfpathlineto{\pgfqpoint{8.529156in}{4.051917in}}%
\pgfpathlineto{\pgfqpoint{8.543778in}{4.049695in}}%
\pgfpathlineto{\pgfqpoint{8.558400in}{4.047504in}}%
\pgfpathlineto{\pgfqpoint{8.573022in}{4.045344in}}%
\pgfpathlineto{\pgfqpoint{8.587644in}{4.043216in}}%
\pgfpathlineto{\pgfqpoint{8.602267in}{4.041118in}}%
\pgfpathlineto{\pgfqpoint{8.616889in}{4.039051in}}%
\pgfpathlineto{\pgfqpoint{8.631511in}{4.037014in}}%
\pgfpathlineto{\pgfqpoint{8.646133in}{4.035008in}}%
\pgfpathlineto{\pgfqpoint{8.660755in}{4.033033in}}%
\pgfpathlineto{\pgfqpoint{8.675378in}{4.031088in}}%
\pgfpathlineto{\pgfqpoint{8.690000in}{4.029173in}}%
\pgfpathlineto{\pgfqpoint{8.704622in}{4.027287in}}%
\pgfpathlineto{\pgfqpoint{8.719244in}{4.025431in}}%
\pgfpathlineto{\pgfqpoint{8.733866in}{4.023605in}}%
\pgfpathlineto{\pgfqpoint{8.748488in}{4.021808in}}%
\pgfpathlineto{\pgfqpoint{8.763111in}{4.020039in}}%
\pgfpathlineto{\pgfqpoint{8.777733in}{4.018300in}}%
\pgfpathlineto{\pgfqpoint{8.792355in}{4.016588in}}%
\pgfpathlineto{\pgfqpoint{8.806977in}{4.014905in}}%
\pgfpathlineto{\pgfqpoint{8.821599in}{4.013250in}}%
\pgfpathlineto{\pgfqpoint{8.836222in}{4.011622in}}%
\pgfpathlineto{\pgfqpoint{8.850844in}{4.010022in}}%
\pgfpathlineto{\pgfqpoint{8.865466in}{4.008448in}}%
\pgfpathlineto{\pgfqpoint{8.880088in}{4.006902in}}%
\pgfpathlineto{\pgfqpoint{8.894710in}{4.005382in}}%
\pgfpathlineto{\pgfqpoint{8.909332in}{4.003888in}}%
\pgfpathlineto{\pgfqpoint{8.923955in}{4.002420in}}%
\pgfpathlineto{\pgfqpoint{8.938577in}{4.000977in}}%
\pgfpathlineto{\pgfqpoint{8.953199in}{3.999560in}}%
\pgfpathlineto{\pgfqpoint{8.967821in}{3.998168in}}%
\pgfpathlineto{\pgfqpoint{8.982443in}{3.996801in}}%
\pgfpathlineto{\pgfqpoint{8.997066in}{3.995458in}}%
\pgfpathlineto{\pgfqpoint{9.011688in}{3.994140in}}%
\pgfpathlineto{\pgfqpoint{9.026310in}{3.992845in}}%
\pgfpathlineto{\pgfqpoint{9.040932in}{3.991574in}}%
\pgfpathlineto{\pgfqpoint{9.055554in}{3.990327in}}%
\pgfpathlineto{\pgfqpoint{9.070176in}{3.989102in}}%
\pgfpathlineto{\pgfqpoint{9.084799in}{3.987901in}}%
\pgfpathlineto{\pgfqpoint{9.099421in}{3.986722in}}%
\pgfpathlineto{\pgfqpoint{9.114043in}{3.985566in}}%
\pgfpathlineto{\pgfqpoint{9.128665in}{3.984431in}}%
\pgfpathlineto{\pgfqpoint{9.143287in}{3.983319in}}%
\pgfpathlineto{\pgfqpoint{9.157909in}{3.982228in}}%
\pgfpathlineto{\pgfqpoint{9.172532in}{3.981158in}}%
\pgfpathlineto{\pgfqpoint{9.187154in}{3.980110in}}%
\pgfpathlineto{\pgfqpoint{9.201776in}{3.979083in}}%
\pgfpathlineto{\pgfqpoint{9.216398in}{3.978076in}}%
\pgfpathlineto{\pgfqpoint{9.231020in}{3.977090in}}%
\pgfpathlineto{\pgfqpoint{9.245643in}{3.976124in}}%
\pgfpathlineto{\pgfqpoint{9.260265in}{3.975178in}}%
\pgfpathlineto{\pgfqpoint{9.274887in}{3.974252in}}%
\pgfpathlineto{\pgfqpoint{9.289509in}{3.973345in}}%
\pgfpathlineto{\pgfqpoint{9.304131in}{3.972458in}}%
\pgfpathlineto{\pgfqpoint{9.318753in}{3.971591in}}%
\pgfpathlineto{\pgfqpoint{9.333376in}{3.970742in}}%
\pgfpathlineto{\pgfqpoint{9.347998in}{3.969912in}}%
\pgfpathlineto{\pgfqpoint{9.362620in}{3.969101in}}%
\pgfpathlineto{\pgfqpoint{9.377242in}{3.968308in}}%
\pgfpathlineto{\pgfqpoint{9.391864in}{3.967534in}}%
\pgfpathlineto{\pgfqpoint{9.406487in}{3.966777in}}%
\pgfpathlineto{\pgfqpoint{9.421109in}{3.966039in}}%
\pgfpathlineto{\pgfqpoint{9.435731in}{3.965319in}}%
\pgfpathlineto{\pgfqpoint{9.450353in}{3.964616in}}%
\pgfpathlineto{\pgfqpoint{9.464975in}{3.963931in}}%
\pgfpathlineto{\pgfqpoint{9.479597in}{3.963263in}}%
\pgfpathlineto{\pgfqpoint{9.494220in}{3.962612in}}%
\pgfpathlineto{\pgfqpoint{9.508842in}{3.961978in}}%
\pgfpathlineto{\pgfqpoint{9.523464in}{3.961362in}}%
\pgfpathlineto{\pgfqpoint{9.538086in}{3.960762in}}%
\pgfpathlineto{\pgfqpoint{9.552708in}{3.960179in}}%
\pgfpathlineto{\pgfqpoint{9.567331in}{3.959612in}}%
\pgfpathlineto{\pgfqpoint{9.581953in}{3.959062in}}%
\pgfpathlineto{\pgfqpoint{9.596575in}{3.958528in}}%
\pgfpathlineto{\pgfqpoint{9.611197in}{3.958010in}}%
\pgfpathlineto{\pgfqpoint{9.625819in}{3.957508in}}%
\pgfpathlineto{\pgfqpoint{9.640441in}{3.957023in}}%
\pgfpathlineto{\pgfqpoint{9.655064in}{3.956553in}}%
\pgfpathlineto{\pgfqpoint{9.669686in}{3.956099in}}%
\pgfpathlineto{\pgfqpoint{9.684308in}{3.955660in}}%
\pgfpathlineto{\pgfqpoint{9.698930in}{3.955237in}}%
\pgfpathlineto{\pgfqpoint{9.713552in}{3.954830in}}%
\pgfpathlineto{\pgfqpoint{9.728175in}{3.954438in}}%
\pgfpathlineto{\pgfqpoint{9.742797in}{3.954061in}}%
\pgfpathlineto{\pgfqpoint{9.757419in}{3.953700in}}%
\pgfpathlineto{\pgfqpoint{9.772041in}{3.953354in}}%
\pgfpathlineto{\pgfqpoint{9.786663in}{3.953023in}}%
\pgfpathlineto{\pgfqpoint{9.801285in}{3.952706in}}%
\pgfpathlineto{\pgfqpoint{9.815908in}{3.952405in}}%
\pgfpathlineto{\pgfqpoint{9.830530in}{3.952119in}}%
\pgfpathlineto{\pgfqpoint{9.845152in}{3.951847in}}%
\pgfpathlineto{\pgfqpoint{9.859774in}{3.951590in}}%
\pgfpathlineto{\pgfqpoint{9.874396in}{3.951348in}}%
\pgfpathlineto{\pgfqpoint{9.889019in}{3.951120in}}%
\pgfpathlineto{\pgfqpoint{9.903641in}{3.950907in}}%
\pgfpathlineto{\pgfqpoint{9.918263in}{3.950709in}}%
\pgfpathlineto{\pgfqpoint{9.932885in}{3.950525in}}%
\pgfpathlineto{\pgfqpoint{9.947507in}{3.950356in}}%
\pgfpathlineto{\pgfqpoint{9.962129in}{3.950201in}}%
\pgfpathlineto{\pgfqpoint{9.976752in}{3.950060in}}%
\pgfpathlineto{\pgfqpoint{9.991374in}{3.949934in}}%
\pgfpathlineto{\pgfqpoint{10.005996in}{3.949822in}}%
\pgfpathlineto{\pgfqpoint{10.020618in}{3.949724in}}%
\pgfpathlineto{\pgfqpoint{10.035240in}{3.949641in}}%
\pgfpathlineto{\pgfqpoint{10.049863in}{3.949572in}}%
\pgfpathlineto{\pgfqpoint{10.064485in}{3.949517in}}%
\pgfpathlineto{\pgfqpoint{10.079107in}{3.949476in}}%
\pgfpathlineto{\pgfqpoint{10.093729in}{3.949450in}}%
\pgfusepath{stroke}%
\end{pgfscope}%
\begin{pgfscope}%
\pgfpathrectangle{\pgfqpoint{8.282041in}{2.849537in}}{\pgfqpoint{1.897959in}{1.372727in}} %
\pgfusepath{clip}%
\pgfsetbuttcap%
\pgfsetmiterjoin%
\definecolor{currentfill}{rgb}{1.000000,0.000000,0.000000}%
\pgfsetfillcolor{currentfill}%
\pgfsetlinewidth{1.003750pt}%
\definecolor{currentstroke}{rgb}{1.000000,0.000000,0.000000}%
\pgfsetstrokecolor{currentstroke}%
\pgfsetdash{}{0pt}%
\pgfsys@defobject{currentmarker}{\pgfqpoint{-0.041667in}{-0.041667in}}{\pgfqpoint{0.041667in}{0.041667in}}{%
\pgfpathmoveto{\pgfqpoint{-0.041667in}{-0.041667in}}%
\pgfpathlineto{\pgfqpoint{0.041667in}{-0.041667in}}%
\pgfpathlineto{\pgfqpoint{0.041667in}{0.041667in}}%
\pgfpathlineto{\pgfqpoint{-0.041667in}{0.041667in}}%
\pgfpathclose%
\pgfusepath{stroke,fill}%
}%
\begin{pgfscope}%
\pgfsys@transformshift{8.368312in}{4.078457in}%
\pgfsys@useobject{currentmarker}{}%
\end{pgfscope}%
\begin{pgfscope}%
\pgfsys@transformshift{8.719244in}{4.025431in}%
\pgfsys@useobject{currentmarker}{}%
\end{pgfscope}%
\begin{pgfscope}%
\pgfsys@transformshift{9.070176in}{3.989102in}%
\pgfsys@useobject{currentmarker}{}%
\end{pgfscope}%
\begin{pgfscope}%
\pgfsys@transformshift{9.421109in}{3.966039in}%
\pgfsys@useobject{currentmarker}{}%
\end{pgfscope}%
\begin{pgfscope}%
\pgfsys@transformshift{9.772041in}{3.953354in}%
\pgfsys@useobject{currentmarker}{}%
\end{pgfscope}%
\end{pgfscope}%
\begin{pgfscope}%
\pgfpathrectangle{\pgfqpoint{8.282041in}{2.849537in}}{\pgfqpoint{1.897959in}{1.372727in}} %
\pgfusepath{clip}%
\pgfsetrectcap%
\pgfsetroundjoin%
\pgfsetlinewidth{1.505625pt}%
\definecolor{currentstroke}{rgb}{0.000000,0.000000,1.000000}%
\pgfsetstrokecolor{currentstroke}%
\pgfsetdash{}{0pt}%
\pgfpathmoveto{\pgfqpoint{8.368312in}{3.665392in}}%
\pgfpathlineto{\pgfqpoint{8.382934in}{3.660990in}}%
\pgfpathlineto{\pgfqpoint{8.397556in}{3.656839in}}%
\pgfpathlineto{\pgfqpoint{8.412178in}{3.652885in}}%
\pgfpathlineto{\pgfqpoint{8.426800in}{3.649090in}}%
\pgfpathlineto{\pgfqpoint{8.441423in}{3.645424in}}%
\pgfpathlineto{\pgfqpoint{8.456045in}{3.641868in}}%
\pgfpathlineto{\pgfqpoint{8.470667in}{3.638402in}}%
\pgfpathlineto{\pgfqpoint{8.485289in}{3.635015in}}%
\pgfpathlineto{\pgfqpoint{8.499911in}{3.631695in}}%
\pgfpathlineto{\pgfqpoint{8.514534in}{3.628434in}}%
\pgfpathlineto{\pgfqpoint{8.529156in}{3.625225in}}%
\pgfpathlineto{\pgfqpoint{8.543778in}{3.622061in}}%
\pgfpathlineto{\pgfqpoint{8.558400in}{3.618937in}}%
\pgfpathlineto{\pgfqpoint{8.573022in}{3.615848in}}%
\pgfpathlineto{\pgfqpoint{8.587644in}{3.612792in}}%
\pgfpathlineto{\pgfqpoint{8.602267in}{3.609764in}}%
\pgfpathlineto{\pgfqpoint{8.616889in}{3.606762in}}%
\pgfpathlineto{\pgfqpoint{8.631511in}{3.603782in}}%
\pgfpathlineto{\pgfqpoint{8.646133in}{3.600823in}}%
\pgfpathlineto{\pgfqpoint{8.660755in}{3.597882in}}%
\pgfpathlineto{\pgfqpoint{8.675378in}{3.594957in}}%
\pgfpathlineto{\pgfqpoint{8.690000in}{3.592047in}}%
\pgfpathlineto{\pgfqpoint{8.704622in}{3.589149in}}%
\pgfpathlineto{\pgfqpoint{8.719244in}{3.586263in}}%
\pgfpathlineto{\pgfqpoint{8.733866in}{3.583386in}}%
\pgfpathlineto{\pgfqpoint{8.748488in}{3.580517in}}%
\pgfpathlineto{\pgfqpoint{8.763111in}{3.577655in}}%
\pgfpathlineto{\pgfqpoint{8.777733in}{3.574799in}}%
\pgfpathlineto{\pgfqpoint{8.792355in}{3.571948in}}%
\pgfpathlineto{\pgfqpoint{8.806977in}{3.569099in}}%
\pgfpathlineto{\pgfqpoint{8.821599in}{3.566253in}}%
\pgfpathlineto{\pgfqpoint{8.836222in}{3.563408in}}%
\pgfpathlineto{\pgfqpoint{8.850844in}{3.560563in}}%
\pgfpathlineto{\pgfqpoint{8.865466in}{3.557716in}}%
\pgfpathlineto{\pgfqpoint{8.880088in}{3.554868in}}%
\pgfpathlineto{\pgfqpoint{8.894710in}{3.552016in}}%
\pgfpathlineto{\pgfqpoint{8.909332in}{3.549161in}}%
\pgfpathlineto{\pgfqpoint{8.923955in}{3.546300in}}%
\pgfpathlineto{\pgfqpoint{8.938577in}{3.543433in}}%
\pgfpathlineto{\pgfqpoint{8.953199in}{3.540559in}}%
\pgfpathlineto{\pgfqpoint{8.967821in}{3.537677in}}%
\pgfpathlineto{\pgfqpoint{8.982443in}{3.534786in}}%
\pgfpathlineto{\pgfqpoint{8.997066in}{3.531884in}}%
\pgfpathlineto{\pgfqpoint{9.011688in}{3.528972in}}%
\pgfpathlineto{\pgfqpoint{9.026310in}{3.526048in}}%
\pgfpathlineto{\pgfqpoint{9.040932in}{3.523110in}}%
\pgfpathlineto{\pgfqpoint{9.055554in}{3.520159in}}%
\pgfpathlineto{\pgfqpoint{9.070176in}{3.517192in}}%
\pgfpathlineto{\pgfqpoint{9.084799in}{3.514209in}}%
\pgfpathlineto{\pgfqpoint{9.099421in}{3.511208in}}%
\pgfpathlineto{\pgfqpoint{9.114043in}{3.508189in}}%
\pgfpathlineto{\pgfqpoint{9.128665in}{3.505151in}}%
\pgfpathlineto{\pgfqpoint{9.143287in}{3.502091in}}%
\pgfpathlineto{\pgfqpoint{9.157909in}{3.499010in}}%
\pgfpathlineto{\pgfqpoint{9.172532in}{3.495905in}}%
\pgfpathlineto{\pgfqpoint{9.187154in}{3.492775in}}%
\pgfpathlineto{\pgfqpoint{9.201776in}{3.489620in}}%
\pgfpathlineto{\pgfqpoint{9.216398in}{3.486437in}}%
\pgfpathlineto{\pgfqpoint{9.231020in}{3.483226in}}%
\pgfpathlineto{\pgfqpoint{9.245643in}{3.479984in}}%
\pgfpathlineto{\pgfqpoint{9.260265in}{3.476711in}}%
\pgfpathlineto{\pgfqpoint{9.274887in}{3.473404in}}%
\pgfpathlineto{\pgfqpoint{9.289509in}{3.470062in}}%
\pgfpathlineto{\pgfqpoint{9.304131in}{3.466683in}}%
\pgfpathlineto{\pgfqpoint{9.318753in}{3.463266in}}%
\pgfpathlineto{\pgfqpoint{9.333376in}{3.459808in}}%
\pgfpathlineto{\pgfqpoint{9.347998in}{3.456307in}}%
\pgfpathlineto{\pgfqpoint{9.362620in}{3.452762in}}%
\pgfpathlineto{\pgfqpoint{9.377242in}{3.449170in}}%
\pgfpathlineto{\pgfqpoint{9.391864in}{3.445529in}}%
\pgfpathlineto{\pgfqpoint{9.406487in}{3.441836in}}%
\pgfpathlineto{\pgfqpoint{9.421109in}{3.438089in}}%
\pgfpathlineto{\pgfqpoint{9.435731in}{3.434285in}}%
\pgfpathlineto{\pgfqpoint{9.450353in}{3.430421in}}%
\pgfpathlineto{\pgfqpoint{9.464975in}{3.426493in}}%
\pgfpathlineto{\pgfqpoint{9.479597in}{3.422500in}}%
\pgfpathlineto{\pgfqpoint{9.494220in}{3.418436in}}%
\pgfpathlineto{\pgfqpoint{9.508842in}{3.414299in}}%
\pgfpathlineto{\pgfqpoint{9.523464in}{3.410084in}}%
\pgfpathlineto{\pgfqpoint{9.538086in}{3.405787in}}%
\pgfpathlineto{\pgfqpoint{9.552708in}{3.401403in}}%
\pgfpathlineto{\pgfqpoint{9.567331in}{3.396928in}}%
\pgfpathlineto{\pgfqpoint{9.581953in}{3.392356in}}%
\pgfpathlineto{\pgfqpoint{9.596575in}{3.387681in}}%
\pgfpathlineto{\pgfqpoint{9.611197in}{3.382898in}}%
\pgfpathlineto{\pgfqpoint{9.625819in}{3.377998in}}%
\pgfpathlineto{\pgfqpoint{9.640441in}{3.372975in}}%
\pgfpathlineto{\pgfqpoint{9.655064in}{3.367821in}}%
\pgfpathlineto{\pgfqpoint{9.669686in}{3.362527in}}%
\pgfpathlineto{\pgfqpoint{9.684308in}{3.357083in}}%
\pgfpathlineto{\pgfqpoint{9.698930in}{3.351479in}}%
\pgfpathlineto{\pgfqpoint{9.713552in}{3.345701in}}%
\pgfpathlineto{\pgfqpoint{9.728175in}{3.339739in}}%
\pgfpathlineto{\pgfqpoint{9.742797in}{3.333576in}}%
\pgfpathlineto{\pgfqpoint{9.757419in}{3.327197in}}%
\pgfpathlineto{\pgfqpoint{9.772041in}{3.320583in}}%
\pgfpathlineto{\pgfqpoint{9.786663in}{3.313713in}}%
\pgfpathlineto{\pgfqpoint{9.801285in}{3.306564in}}%
\pgfpathlineto{\pgfqpoint{9.815908in}{3.299109in}}%
\pgfpathlineto{\pgfqpoint{9.830530in}{3.291317in}}%
\pgfpathlineto{\pgfqpoint{9.845152in}{3.283153in}}%
\pgfpathlineto{\pgfqpoint{9.859774in}{3.274574in}}%
\pgfpathlineto{\pgfqpoint{9.874396in}{3.265533in}}%
\pgfpathlineto{\pgfqpoint{9.889019in}{3.255969in}}%
\pgfpathlineto{\pgfqpoint{9.903641in}{3.245815in}}%
\pgfpathlineto{\pgfqpoint{9.918263in}{3.234986in}}%
\pgfpathlineto{\pgfqpoint{9.932885in}{3.223377in}}%
\pgfpathlineto{\pgfqpoint{9.947507in}{3.210857in}}%
\pgfpathlineto{\pgfqpoint{9.962129in}{3.197262in}}%
\pgfpathlineto{\pgfqpoint{9.976752in}{3.182375in}}%
\pgfpathlineto{\pgfqpoint{9.991374in}{3.165908in}}%
\pgfpathlineto{\pgfqpoint{10.005996in}{3.147462in}}%
\pgfpathlineto{\pgfqpoint{10.020618in}{3.126466in}}%
\pgfpathlineto{\pgfqpoint{10.035240in}{3.102057in}}%
\pgfpathlineto{\pgfqpoint{10.049863in}{3.072835in}}%
\pgfpathlineto{\pgfqpoint{10.064485in}{3.036310in}}%
\pgfpathlineto{\pgfqpoint{10.079107in}{2.987320in}}%
\pgfpathlineto{\pgfqpoint{10.093729in}{2.911933in}}%
\pgfusepath{stroke}%
\end{pgfscope}%
\begin{pgfscope}%
\pgfpathrectangle{\pgfqpoint{8.282041in}{2.849537in}}{\pgfqpoint{1.897959in}{1.372727in}} %
\pgfusepath{clip}%
\pgfsetbuttcap%
\pgfsetroundjoin%
\definecolor{currentfill}{rgb}{0.000000,0.000000,1.000000}%
\pgfsetfillcolor{currentfill}%
\pgfsetlinewidth{1.003750pt}%
\definecolor{currentstroke}{rgb}{0.000000,0.000000,1.000000}%
\pgfsetstrokecolor{currentstroke}%
\pgfsetdash{}{0pt}%
\pgfsys@defobject{currentmarker}{\pgfqpoint{-0.041667in}{-0.041667in}}{\pgfqpoint{0.041667in}{0.041667in}}{%
\pgfpathmoveto{\pgfqpoint{0.000000in}{-0.041667in}}%
\pgfpathcurveto{\pgfqpoint{0.011050in}{-0.041667in}}{\pgfqpoint{0.021649in}{-0.037276in}}{\pgfqpoint{0.029463in}{-0.029463in}}%
\pgfpathcurveto{\pgfqpoint{0.037276in}{-0.021649in}}{\pgfqpoint{0.041667in}{-0.011050in}}{\pgfqpoint{0.041667in}{0.000000in}}%
\pgfpathcurveto{\pgfqpoint{0.041667in}{0.011050in}}{\pgfqpoint{0.037276in}{0.021649in}}{\pgfqpoint{0.029463in}{0.029463in}}%
\pgfpathcurveto{\pgfqpoint{0.021649in}{0.037276in}}{\pgfqpoint{0.011050in}{0.041667in}}{\pgfqpoint{0.000000in}{0.041667in}}%
\pgfpathcurveto{\pgfqpoint{-0.011050in}{0.041667in}}{\pgfqpoint{-0.021649in}{0.037276in}}{\pgfqpoint{-0.029463in}{0.029463in}}%
\pgfpathcurveto{\pgfqpoint{-0.037276in}{0.021649in}}{\pgfqpoint{-0.041667in}{0.011050in}}{\pgfqpoint{-0.041667in}{0.000000in}}%
\pgfpathcurveto{\pgfqpoint{-0.041667in}{-0.011050in}}{\pgfqpoint{-0.037276in}{-0.021649in}}{\pgfqpoint{-0.029463in}{-0.029463in}}%
\pgfpathcurveto{\pgfqpoint{-0.021649in}{-0.037276in}}{\pgfqpoint{-0.011050in}{-0.041667in}}{\pgfqpoint{0.000000in}{-0.041667in}}%
\pgfpathclose%
\pgfusepath{stroke,fill}%
}%
\begin{pgfscope}%
\pgfsys@transformshift{8.368312in}{3.665392in}%
\pgfsys@useobject{currentmarker}{}%
\end{pgfscope}%
\begin{pgfscope}%
\pgfsys@transformshift{8.719244in}{3.586263in}%
\pgfsys@useobject{currentmarker}{}%
\end{pgfscope}%
\begin{pgfscope}%
\pgfsys@transformshift{9.070176in}{3.517192in}%
\pgfsys@useobject{currentmarker}{}%
\end{pgfscope}%
\begin{pgfscope}%
\pgfsys@transformshift{9.421109in}{3.438089in}%
\pgfsys@useobject{currentmarker}{}%
\end{pgfscope}%
\begin{pgfscope}%
\pgfsys@transformshift{9.772041in}{3.320583in}%
\pgfsys@useobject{currentmarker}{}%
\end{pgfscope}%
\end{pgfscope}%
\begin{pgfscope}%
\pgfpathrectangle{\pgfqpoint{8.282041in}{2.849537in}}{\pgfqpoint{1.897959in}{1.372727in}} %
\pgfusepath{clip}%
\pgfsetbuttcap%
\pgfsetroundjoin%
\pgfsetlinewidth{1.505625pt}%
\definecolor{currentstroke}{rgb}{0.000000,0.750000,0.750000}%
\pgfsetstrokecolor{currentstroke}%
\pgfsetdash{{9.600000pt}{2.400000pt}{1.500000pt}{2.400000pt}}{0.000000pt}%
\pgfpathmoveto{\pgfqpoint{8.368312in}{4.057155in}}%
\pgfpathlineto{\pgfqpoint{8.382934in}{4.034604in}}%
\pgfpathlineto{\pgfqpoint{8.397556in}{4.014328in}}%
\pgfpathlineto{\pgfqpoint{8.412178in}{3.995930in}}%
\pgfpathlineto{\pgfqpoint{8.426800in}{3.979108in}}%
\pgfpathlineto{\pgfqpoint{8.441423in}{3.963629in}}%
\pgfpathlineto{\pgfqpoint{8.456045in}{3.949307in}}%
\pgfpathlineto{\pgfqpoint{8.470667in}{3.935993in}}%
\pgfpathlineto{\pgfqpoint{8.485289in}{3.923566in}}%
\pgfpathlineto{\pgfqpoint{8.499911in}{3.911925in}}%
\pgfpathlineto{\pgfqpoint{8.514534in}{3.900985in}}%
\pgfpathlineto{\pgfqpoint{8.529156in}{3.890674in}}%
\pgfpathlineto{\pgfqpoint{8.543778in}{3.880932in}}%
\pgfpathlineto{\pgfqpoint{8.558400in}{3.871706in}}%
\pgfpathlineto{\pgfqpoint{8.573022in}{3.862949in}}%
\pgfpathlineto{\pgfqpoint{8.587644in}{3.854624in}}%
\pgfpathlineto{\pgfqpoint{8.602267in}{3.846693in}}%
\pgfpathlineto{\pgfqpoint{8.616889in}{3.839127in}}%
\pgfpathlineto{\pgfqpoint{8.631511in}{3.831899in}}%
\pgfpathlineto{\pgfqpoint{8.646133in}{3.824983in}}%
\pgfpathlineto{\pgfqpoint{8.660755in}{3.818358in}}%
\pgfpathlineto{\pgfqpoint{8.675378in}{3.812005in}}%
\pgfpathlineto{\pgfqpoint{8.690000in}{3.805905in}}%
\pgfpathlineto{\pgfqpoint{8.704622in}{3.800043in}}%
\pgfpathlineto{\pgfqpoint{8.719244in}{3.794404in}}%
\pgfpathlineto{\pgfqpoint{8.733866in}{3.788975in}}%
\pgfpathlineto{\pgfqpoint{8.748488in}{3.783744in}}%
\pgfpathlineto{\pgfqpoint{8.763111in}{3.778699in}}%
\pgfpathlineto{\pgfqpoint{8.777733in}{3.773832in}}%
\pgfpathlineto{\pgfqpoint{8.792355in}{3.769131in}}%
\pgfpathlineto{\pgfqpoint{8.806977in}{3.764589in}}%
\pgfpathlineto{\pgfqpoint{8.821599in}{3.760198in}}%
\pgfpathlineto{\pgfqpoint{8.836222in}{3.755951in}}%
\pgfpathlineto{\pgfqpoint{8.850844in}{3.751840in}}%
\pgfpathlineto{\pgfqpoint{8.865466in}{3.747859in}}%
\pgfpathlineto{\pgfqpoint{8.880088in}{3.744002in}}%
\pgfpathlineto{\pgfqpoint{8.894710in}{3.740265in}}%
\pgfpathlineto{\pgfqpoint{8.909332in}{3.736641in}}%
\pgfpathlineto{\pgfqpoint{8.923955in}{3.733126in}}%
\pgfpathlineto{\pgfqpoint{8.938577in}{3.729715in}}%
\pgfpathlineto{\pgfqpoint{8.953199in}{3.726405in}}%
\pgfpathlineto{\pgfqpoint{8.967821in}{3.723191in}}%
\pgfpathlineto{\pgfqpoint{8.982443in}{3.720069in}}%
\pgfpathlineto{\pgfqpoint{8.997066in}{3.717037in}}%
\pgfpathlineto{\pgfqpoint{9.011688in}{3.714090in}}%
\pgfpathlineto{\pgfqpoint{9.026310in}{3.711226in}}%
\pgfpathlineto{\pgfqpoint{9.040932in}{3.708442in}}%
\pgfpathlineto{\pgfqpoint{9.055554in}{3.705735in}}%
\pgfpathlineto{\pgfqpoint{9.070176in}{3.703102in}}%
\pgfpathlineto{\pgfqpoint{9.084799in}{3.700541in}}%
\pgfpathlineto{\pgfqpoint{9.099421in}{3.698050in}}%
\pgfpathlineto{\pgfqpoint{9.114043in}{3.695626in}}%
\pgfpathlineto{\pgfqpoint{9.128665in}{3.693268in}}%
\pgfpathlineto{\pgfqpoint{9.143287in}{3.690973in}}%
\pgfpathlineto{\pgfqpoint{9.157909in}{3.688739in}}%
\pgfpathlineto{\pgfqpoint{9.172532in}{3.686565in}}%
\pgfpathlineto{\pgfqpoint{9.187154in}{3.684449in}}%
\pgfpathlineto{\pgfqpoint{9.201776in}{3.682389in}}%
\pgfpathlineto{\pgfqpoint{9.216398in}{3.680384in}}%
\pgfpathlineto{\pgfqpoint{9.231020in}{3.678432in}}%
\pgfpathlineto{\pgfqpoint{9.245643in}{3.676532in}}%
\pgfpathlineto{\pgfqpoint{9.260265in}{3.674683in}}%
\pgfpathlineto{\pgfqpoint{9.274887in}{3.672883in}}%
\pgfpathlineto{\pgfqpoint{9.289509in}{3.671131in}}%
\pgfpathlineto{\pgfqpoint{9.304131in}{3.669426in}}%
\pgfpathlineto{\pgfqpoint{9.318753in}{3.667766in}}%
\pgfpathlineto{\pgfqpoint{9.333376in}{3.666151in}}%
\pgfpathlineto{\pgfqpoint{9.347998in}{3.664580in}}%
\pgfpathlineto{\pgfqpoint{9.362620in}{3.663052in}}%
\pgfpathlineto{\pgfqpoint{9.377242in}{3.661565in}}%
\pgfpathlineto{\pgfqpoint{9.391864in}{3.660120in}}%
\pgfpathlineto{\pgfqpoint{9.406487in}{3.658714in}}%
\pgfpathlineto{\pgfqpoint{9.421109in}{3.657347in}}%
\pgfpathlineto{\pgfqpoint{9.435731in}{3.656019in}}%
\pgfpathlineto{\pgfqpoint{9.450353in}{3.654729in}}%
\pgfpathlineto{\pgfqpoint{9.464975in}{3.653475in}}%
\pgfpathlineto{\pgfqpoint{9.479597in}{3.652258in}}%
\pgfpathlineto{\pgfqpoint{9.494220in}{3.651076in}}%
\pgfpathlineto{\pgfqpoint{9.508842in}{3.649930in}}%
\pgfpathlineto{\pgfqpoint{9.523464in}{3.648817in}}%
\pgfpathlineto{\pgfqpoint{9.538086in}{3.647739in}}%
\pgfpathlineto{\pgfqpoint{9.552708in}{3.646694in}}%
\pgfpathlineto{\pgfqpoint{9.567331in}{3.645681in}}%
\pgfpathlineto{\pgfqpoint{9.581953in}{3.644701in}}%
\pgfpathlineto{\pgfqpoint{9.596575in}{3.643752in}}%
\pgfpathlineto{\pgfqpoint{9.611197in}{3.642835in}}%
\pgfpathlineto{\pgfqpoint{9.625819in}{3.641949in}}%
\pgfpathlineto{\pgfqpoint{9.640441in}{3.641092in}}%
\pgfpathlineto{\pgfqpoint{9.655064in}{3.640266in}}%
\pgfpathlineto{\pgfqpoint{9.669686in}{3.639470in}}%
\pgfpathlineto{\pgfqpoint{9.684308in}{3.638702in}}%
\pgfpathlineto{\pgfqpoint{9.698930in}{3.637964in}}%
\pgfpathlineto{\pgfqpoint{9.713552in}{3.637254in}}%
\pgfpathlineto{\pgfqpoint{9.728175in}{3.636572in}}%
\pgfpathlineto{\pgfqpoint{9.742797in}{3.635918in}}%
\pgfpathlineto{\pgfqpoint{9.757419in}{3.635291in}}%
\pgfpathlineto{\pgfqpoint{9.772041in}{3.634692in}}%
\pgfpathlineto{\pgfqpoint{9.786663in}{3.634120in}}%
\pgfpathlineto{\pgfqpoint{9.801285in}{3.633575in}}%
\pgfpathlineto{\pgfqpoint{9.815908in}{3.633056in}}%
\pgfpathlineto{\pgfqpoint{9.830530in}{3.632564in}}%
\pgfpathlineto{\pgfqpoint{9.845152in}{3.632097in}}%
\pgfpathlineto{\pgfqpoint{9.859774in}{3.631657in}}%
\pgfpathlineto{\pgfqpoint{9.874396in}{3.631242in}}%
\pgfpathlineto{\pgfqpoint{9.889019in}{3.630853in}}%
\pgfpathlineto{\pgfqpoint{9.903641in}{3.630489in}}%
\pgfpathlineto{\pgfqpoint{9.918263in}{3.630150in}}%
\pgfpathlineto{\pgfqpoint{9.932885in}{3.629836in}}%
\pgfpathlineto{\pgfqpoint{9.947507in}{3.629547in}}%
\pgfpathlineto{\pgfqpoint{9.962129in}{3.629283in}}%
\pgfpathlineto{\pgfqpoint{9.976752in}{3.629044in}}%
\pgfpathlineto{\pgfqpoint{9.991374in}{3.628829in}}%
\pgfpathlineto{\pgfqpoint{10.005996in}{3.628639in}}%
\pgfpathlineto{\pgfqpoint{10.020618in}{3.628473in}}%
\pgfpathlineto{\pgfqpoint{10.035240in}{3.628331in}}%
\pgfpathlineto{\pgfqpoint{10.049863in}{3.628214in}}%
\pgfpathlineto{\pgfqpoint{10.064485in}{3.628121in}}%
\pgfpathlineto{\pgfqpoint{10.079107in}{3.628052in}}%
\pgfpathlineto{\pgfqpoint{10.093729in}{3.628007in}}%
\pgfusepath{stroke}%
\end{pgfscope}%
\begin{pgfscope}%
\pgfpathrectangle{\pgfqpoint{8.282041in}{2.849537in}}{\pgfqpoint{1.897959in}{1.372727in}} %
\pgfusepath{clip}%
\pgfsetbuttcap%
\pgfsetmiterjoin%
\definecolor{currentfill}{rgb}{0.000000,0.750000,0.750000}%
\pgfsetfillcolor{currentfill}%
\pgfsetlinewidth{1.003750pt}%
\definecolor{currentstroke}{rgb}{0.000000,0.750000,0.750000}%
\pgfsetstrokecolor{currentstroke}%
\pgfsetdash{}{0pt}%
\pgfsys@defobject{currentmarker}{\pgfqpoint{-0.041667in}{-0.041667in}}{\pgfqpoint{0.041667in}{0.041667in}}{%
\pgfpathmoveto{\pgfqpoint{-0.000000in}{-0.041667in}}%
\pgfpathlineto{\pgfqpoint{0.041667in}{0.041667in}}%
\pgfpathlineto{\pgfqpoint{-0.041667in}{0.041667in}}%
\pgfpathclose%
\pgfusepath{stroke,fill}%
}%
\begin{pgfscope}%
\pgfsys@transformshift{8.368312in}{4.057155in}%
\pgfsys@useobject{currentmarker}{}%
\end{pgfscope}%
\begin{pgfscope}%
\pgfsys@transformshift{8.719244in}{3.794404in}%
\pgfsys@useobject{currentmarker}{}%
\end{pgfscope}%
\begin{pgfscope}%
\pgfsys@transformshift{9.070176in}{3.703102in}%
\pgfsys@useobject{currentmarker}{}%
\end{pgfscope}%
\begin{pgfscope}%
\pgfsys@transformshift{9.421109in}{3.657347in}%
\pgfsys@useobject{currentmarker}{}%
\end{pgfscope}%
\begin{pgfscope}%
\pgfsys@transformshift{9.772041in}{3.634692in}%
\pgfsys@useobject{currentmarker}{}%
\end{pgfscope}%
\end{pgfscope}%
\begin{pgfscope}%
\pgfpathrectangle{\pgfqpoint{8.282041in}{2.849537in}}{\pgfqpoint{1.897959in}{1.372727in}} %
\pgfusepath{clip}%
\pgfsetbuttcap%
\pgfsetroundjoin%
\pgfsetlinewidth{1.505625pt}%
\definecolor{currentstroke}{rgb}{0.000000,0.000000,0.000000}%
\pgfsetstrokecolor{currentstroke}%
\pgfsetdash{{1.500000pt}{2.475000pt}}{0.000000pt}%
\pgfpathmoveto{\pgfqpoint{8.368312in}{4.159867in}}%
\pgfpathlineto{\pgfqpoint{8.382934in}{4.149899in}}%
\pgfpathlineto{\pgfqpoint{8.397556in}{4.140219in}}%
\pgfpathlineto{\pgfqpoint{8.412178in}{4.131817in}}%
\pgfpathlineto{\pgfqpoint{8.426800in}{4.124412in}}%
\pgfpathlineto{\pgfqpoint{8.441423in}{4.117799in}}%
\pgfpathlineto{\pgfqpoint{8.456045in}{4.111826in}}%
\pgfpathlineto{\pgfqpoint{8.470667in}{4.106376in}}%
\pgfpathlineto{\pgfqpoint{8.485289in}{4.101359in}}%
\pgfpathlineto{\pgfqpoint{8.499911in}{4.096706in}}%
\pgfpathlineto{\pgfqpoint{8.514534in}{4.092361in}}%
\pgfpathlineto{\pgfqpoint{8.529156in}{4.088282in}}%
\pgfpathlineto{\pgfqpoint{8.543778in}{4.084432in}}%
\pgfpathlineto{\pgfqpoint{8.558400in}{4.080781in}}%
\pgfpathlineto{\pgfqpoint{8.573022in}{4.077307in}}%
\pgfpathlineto{\pgfqpoint{8.587644in}{4.073990in}}%
\pgfpathlineto{\pgfqpoint{8.602267in}{4.070812in}}%
\pgfpathlineto{\pgfqpoint{8.616889in}{4.067761in}}%
\pgfpathlineto{\pgfqpoint{8.631511in}{4.064824in}}%
\pgfpathlineto{\pgfqpoint{8.646133in}{4.061991in}}%
\pgfpathlineto{\pgfqpoint{8.660755in}{4.059254in}}%
\pgfpathlineto{\pgfqpoint{8.675378in}{4.056605in}}%
\pgfpathlineto{\pgfqpoint{8.690000in}{4.054038in}}%
\pgfpathlineto{\pgfqpoint{8.704622in}{4.051547in}}%
\pgfpathlineto{\pgfqpoint{8.719244in}{4.049126in}}%
\pgfpathlineto{\pgfqpoint{8.733866in}{4.046773in}}%
\pgfpathlineto{\pgfqpoint{8.748488in}{4.044483in}}%
\pgfpathlineto{\pgfqpoint{8.763111in}{4.042252in}}%
\pgfpathlineto{\pgfqpoint{8.777733in}{4.040078in}}%
\pgfpathlineto{\pgfqpoint{8.792355in}{4.037958in}}%
\pgfpathlineto{\pgfqpoint{8.806977in}{4.035888in}}%
\pgfpathlineto{\pgfqpoint{8.821599in}{4.033868in}}%
\pgfpathlineto{\pgfqpoint{8.836222in}{4.031894in}}%
\pgfpathlineto{\pgfqpoint{8.850844in}{4.029965in}}%
\pgfpathlineto{\pgfqpoint{8.865466in}{4.028080in}}%
\pgfpathlineto{\pgfqpoint{8.880088in}{4.026236in}}%
\pgfpathlineto{\pgfqpoint{8.894710in}{4.024433in}}%
\pgfpathlineto{\pgfqpoint{8.909332in}{4.022669in}}%
\pgfpathlineto{\pgfqpoint{8.923955in}{4.020942in}}%
\pgfpathlineto{\pgfqpoint{8.938577in}{4.019252in}}%
\pgfpathlineto{\pgfqpoint{8.953199in}{4.017598in}}%
\pgfpathlineto{\pgfqpoint{8.967821in}{4.015978in}}%
\pgfpathlineto{\pgfqpoint{8.982443in}{4.014392in}}%
\pgfpathlineto{\pgfqpoint{8.997066in}{4.012839in}}%
\pgfpathlineto{\pgfqpoint{9.011688in}{4.011317in}}%
\pgfpathlineto{\pgfqpoint{9.026310in}{4.009827in}}%
\pgfpathlineto{\pgfqpoint{9.040932in}{4.008367in}}%
\pgfpathlineto{\pgfqpoint{9.055554in}{4.006937in}}%
\pgfpathlineto{\pgfqpoint{9.070176in}{4.005536in}}%
\pgfpathlineto{\pgfqpoint{9.084799in}{4.004163in}}%
\pgfpathlineto{\pgfqpoint{9.099421in}{4.002818in}}%
\pgfpathlineto{\pgfqpoint{9.114043in}{4.001501in}}%
\pgfpathlineto{\pgfqpoint{9.128665in}{4.000210in}}%
\pgfpathlineto{\pgfqpoint{9.143287in}{3.998946in}}%
\pgfpathlineto{\pgfqpoint{9.157909in}{3.997707in}}%
\pgfpathlineto{\pgfqpoint{9.172532in}{3.996494in}}%
\pgfpathlineto{\pgfqpoint{9.187154in}{3.995305in}}%
\pgfpathlineto{\pgfqpoint{9.201776in}{3.994142in}}%
\pgfpathlineto{\pgfqpoint{9.216398in}{3.993002in}}%
\pgfpathlineto{\pgfqpoint{9.231020in}{3.991886in}}%
\pgfpathlineto{\pgfqpoint{9.245643in}{3.990793in}}%
\pgfpathlineto{\pgfqpoint{9.260265in}{3.989724in}}%
\pgfpathlineto{\pgfqpoint{9.274887in}{3.988677in}}%
\pgfpathlineto{\pgfqpoint{9.289509in}{3.987652in}}%
\pgfpathlineto{\pgfqpoint{9.304131in}{3.986649in}}%
\pgfpathlineto{\pgfqpoint{9.318753in}{3.985668in}}%
\pgfpathlineto{\pgfqpoint{9.333376in}{3.984708in}}%
\pgfpathlineto{\pgfqpoint{9.347998in}{3.983770in}}%
\pgfpathlineto{\pgfqpoint{9.362620in}{3.982852in}}%
\pgfpathlineto{\pgfqpoint{9.377242in}{3.981955in}}%
\pgfpathlineto{\pgfqpoint{9.391864in}{3.981078in}}%
\pgfpathlineto{\pgfqpoint{9.406487in}{3.980222in}}%
\pgfpathlineto{\pgfqpoint{9.421109in}{3.979385in}}%
\pgfpathlineto{\pgfqpoint{9.435731in}{3.978568in}}%
\pgfpathlineto{\pgfqpoint{9.450353in}{3.977770in}}%
\pgfpathlineto{\pgfqpoint{9.464975in}{3.976992in}}%
\pgfpathlineto{\pgfqpoint{9.479597in}{3.976233in}}%
\pgfpathlineto{\pgfqpoint{9.494220in}{3.975492in}}%
\pgfpathlineto{\pgfqpoint{9.508842in}{3.974770in}}%
\pgfpathlineto{\pgfqpoint{9.523464in}{3.974067in}}%
\pgfpathlineto{\pgfqpoint{9.538086in}{3.973382in}}%
\pgfpathlineto{\pgfqpoint{9.552708in}{3.972715in}}%
\pgfpathlineto{\pgfqpoint{9.567331in}{3.972066in}}%
\pgfpathlineto{\pgfqpoint{9.581953in}{3.971435in}}%
\pgfpathlineto{\pgfqpoint{9.596575in}{3.970821in}}%
\pgfpathlineto{\pgfqpoint{9.611197in}{3.970226in}}%
\pgfpathlineto{\pgfqpoint{9.625819in}{3.969648in}}%
\pgfpathlineto{\pgfqpoint{9.640441in}{3.969087in}}%
\pgfpathlineto{\pgfqpoint{9.655064in}{3.968543in}}%
\pgfpathlineto{\pgfqpoint{9.669686in}{3.968016in}}%
\pgfpathlineto{\pgfqpoint{9.684308in}{3.967506in}}%
\pgfpathlineto{\pgfqpoint{9.698930in}{3.967013in}}%
\pgfpathlineto{\pgfqpoint{9.713552in}{3.966537in}}%
\pgfpathlineto{\pgfqpoint{9.728175in}{3.966078in}}%
\pgfpathlineto{\pgfqpoint{9.742797in}{3.965635in}}%
\pgfpathlineto{\pgfqpoint{9.757419in}{3.965208in}}%
\pgfpathlineto{\pgfqpoint{9.772041in}{3.964798in}}%
\pgfpathlineto{\pgfqpoint{9.786663in}{3.964405in}}%
\pgfpathlineto{\pgfqpoint{9.801285in}{3.964027in}}%
\pgfpathlineto{\pgfqpoint{9.815908in}{3.963666in}}%
\pgfpathlineto{\pgfqpoint{9.830530in}{3.963321in}}%
\pgfpathlineto{\pgfqpoint{9.845152in}{3.962992in}}%
\pgfpathlineto{\pgfqpoint{9.859774in}{3.962679in}}%
\pgfpathlineto{\pgfqpoint{9.874396in}{3.962382in}}%
\pgfpathlineto{\pgfqpoint{9.889019in}{3.962100in}}%
\pgfpathlineto{\pgfqpoint{9.903641in}{3.961835in}}%
\pgfpathlineto{\pgfqpoint{9.918263in}{3.961586in}}%
\pgfpathlineto{\pgfqpoint{9.932885in}{3.961352in}}%
\pgfpathlineto{\pgfqpoint{9.947507in}{3.961134in}}%
\pgfpathlineto{\pgfqpoint{9.962129in}{3.960932in}}%
\pgfpathlineto{\pgfqpoint{9.976752in}{3.960746in}}%
\pgfpathlineto{\pgfqpoint{9.991374in}{3.960576in}}%
\pgfpathlineto{\pgfqpoint{10.005996in}{3.960421in}}%
\pgfpathlineto{\pgfqpoint{10.020618in}{3.960283in}}%
\pgfpathlineto{\pgfqpoint{10.035240in}{3.960160in}}%
\pgfpathlineto{\pgfqpoint{10.049863in}{3.960053in}}%
\pgfpathlineto{\pgfqpoint{10.064485in}{3.959962in}}%
\pgfpathlineto{\pgfqpoint{10.079107in}{3.959887in}}%
\pgfpathlineto{\pgfqpoint{10.093729in}{3.959828in}}%
\pgfusepath{stroke}%
\end{pgfscope}%
\begin{pgfscope}%
\pgfpathrectangle{\pgfqpoint{8.282041in}{2.849537in}}{\pgfqpoint{1.897959in}{1.372727in}} %
\pgfusepath{clip}%
\pgfsetbuttcap%
\pgfsetroundjoin%
\definecolor{currentfill}{rgb}{0.000000,0.000000,0.000000}%
\pgfsetfillcolor{currentfill}%
\pgfsetlinewidth{1.003750pt}%
\definecolor{currentstroke}{rgb}{0.000000,0.000000,0.000000}%
\pgfsetstrokecolor{currentstroke}%
\pgfsetdash{}{0pt}%
\pgfsys@defobject{currentmarker}{\pgfqpoint{-0.041667in}{-0.041667in}}{\pgfqpoint{0.041667in}{0.041667in}}{%
\pgfpathmoveto{\pgfqpoint{-0.041667in}{0.000000in}}%
\pgfpathlineto{\pgfqpoint{0.041667in}{0.000000in}}%
\pgfpathmoveto{\pgfqpoint{0.000000in}{-0.041667in}}%
\pgfpathlineto{\pgfqpoint{0.000000in}{0.041667in}}%
\pgfusepath{stroke,fill}%
}%
\begin{pgfscope}%
\pgfsys@transformshift{8.368312in}{4.159867in}%
\pgfsys@useobject{currentmarker}{}%
\end{pgfscope}%
\begin{pgfscope}%
\pgfsys@transformshift{8.719244in}{4.049126in}%
\pgfsys@useobject{currentmarker}{}%
\end{pgfscope}%
\begin{pgfscope}%
\pgfsys@transformshift{9.070176in}{4.005536in}%
\pgfsys@useobject{currentmarker}{}%
\end{pgfscope}%
\begin{pgfscope}%
\pgfsys@transformshift{9.421109in}{3.979385in}%
\pgfsys@useobject{currentmarker}{}%
\end{pgfscope}%
\begin{pgfscope}%
\pgfsys@transformshift{9.772041in}{3.964798in}%
\pgfsys@useobject{currentmarker}{}%
\end{pgfscope}%
\end{pgfscope}%
\begin{pgfscope}%
\pgfsetrectcap%
\pgfsetmiterjoin%
\pgfsetlinewidth{0.803000pt}%
\definecolor{currentstroke}{rgb}{0.000000,0.000000,0.000000}%
\pgfsetstrokecolor{currentstroke}%
\pgfsetdash{}{0pt}%
\pgfpathmoveto{\pgfqpoint{8.282041in}{2.849537in}}%
\pgfpathlineto{\pgfqpoint{8.282041in}{4.222264in}}%
\pgfusepath{stroke}%
\end{pgfscope}%
\begin{pgfscope}%
\pgfsetrectcap%
\pgfsetmiterjoin%
\pgfsetlinewidth{0.803000pt}%
\definecolor{currentstroke}{rgb}{0.000000,0.000000,0.000000}%
\pgfsetstrokecolor{currentstroke}%
\pgfsetdash{}{0pt}%
\pgfpathmoveto{\pgfqpoint{10.180000in}{2.849537in}}%
\pgfpathlineto{\pgfqpoint{10.180000in}{4.222264in}}%
\pgfusepath{stroke}%
\end{pgfscope}%
\begin{pgfscope}%
\pgfsetrectcap%
\pgfsetmiterjoin%
\pgfsetlinewidth{0.803000pt}%
\definecolor{currentstroke}{rgb}{0.000000,0.000000,0.000000}%
\pgfsetstrokecolor{currentstroke}%
\pgfsetdash{}{0pt}%
\pgfpathmoveto{\pgfqpoint{8.282041in}{2.849537in}}%
\pgfpathlineto{\pgfqpoint{10.180000in}{2.849537in}}%
\pgfusepath{stroke}%
\end{pgfscope}%
\begin{pgfscope}%
\pgfsetrectcap%
\pgfsetmiterjoin%
\pgfsetlinewidth{0.803000pt}%
\definecolor{currentstroke}{rgb}{0.000000,0.000000,0.000000}%
\pgfsetstrokecolor{currentstroke}%
\pgfsetdash{}{0pt}%
\pgfpathmoveto{\pgfqpoint{8.282041in}{4.222264in}}%
\pgfpathlineto{\pgfqpoint{10.180000in}{4.222264in}}%
\pgfusepath{stroke}%
\end{pgfscope}%
\begin{pgfscope}%
\pgfsetbuttcap%
\pgfsetmiterjoin%
\definecolor{currentfill}{rgb}{1.000000,1.000000,1.000000}%
\pgfsetfillcolor{currentfill}%
\pgfsetlinewidth{0.000000pt}%
\definecolor{currentstroke}{rgb}{0.000000,0.000000,0.000000}%
\pgfsetstrokecolor{currentstroke}%
\pgfsetstrokeopacity{0.000000}%
\pgfsetdash{}{0pt}%
\pgfpathmoveto{\pgfqpoint{0.880000in}{0.790446in}}%
\pgfpathlineto{\pgfqpoint{2.777959in}{0.790446in}}%
\pgfpathlineto{\pgfqpoint{2.777959in}{2.163173in}}%
\pgfpathlineto{\pgfqpoint{0.880000in}{2.163173in}}%
\pgfpathclose%
\pgfusepath{fill}%
\end{pgfscope}%
\begin{pgfscope}%
\pgfsetbuttcap%
\pgfsetroundjoin%
\definecolor{currentfill}{rgb}{0.000000,0.000000,0.000000}%
\pgfsetfillcolor{currentfill}%
\pgfsetlinewidth{0.803000pt}%
\definecolor{currentstroke}{rgb}{0.000000,0.000000,0.000000}%
\pgfsetstrokecolor{currentstroke}%
\pgfsetdash{}{0pt}%
\pgfsys@defobject{currentmarker}{\pgfqpoint{0.000000in}{-0.048611in}}{\pgfqpoint{0.000000in}{0.000000in}}{%
\pgfpathmoveto{\pgfqpoint{0.000000in}{0.000000in}}%
\pgfpathlineto{\pgfqpoint{0.000000in}{-0.048611in}}%
\pgfusepath{stroke,fill}%
}%
\begin{pgfscope}%
\pgfsys@transformshift{0.896456in}{0.790446in}%
\pgfsys@useobject{currentmarker}{}%
\end{pgfscope}%
\end{pgfscope}%
\begin{pgfscope}%
\pgftext[x=0.896456in,y=0.693224in,,top]{\rmfamily\fontsize{10.000000}{12.000000}\selectfont \(\displaystyle 0.0\)}%
\end{pgfscope}%
\begin{pgfscope}%
\pgfsetbuttcap%
\pgfsetroundjoin%
\definecolor{currentfill}{rgb}{0.000000,0.000000,0.000000}%
\pgfsetfillcolor{currentfill}%
\pgfsetlinewidth{0.803000pt}%
\definecolor{currentstroke}{rgb}{0.000000,0.000000,0.000000}%
\pgfsetstrokecolor{currentstroke}%
\pgfsetdash{}{0pt}%
\pgfsys@defobject{currentmarker}{\pgfqpoint{0.000000in}{-0.048611in}}{\pgfqpoint{0.000000in}{0.000000in}}{%
\pgfpathmoveto{\pgfqpoint{0.000000in}{0.000000in}}%
\pgfpathlineto{\pgfqpoint{0.000000in}{-0.048611in}}%
\pgfusepath{stroke,fill}%
}%
\begin{pgfscope}%
\pgfsys@transformshift{1.494867in}{0.790446in}%
\pgfsys@useobject{currentmarker}{}%
\end{pgfscope}%
\end{pgfscope}%
\begin{pgfscope}%
\pgftext[x=1.494867in,y=0.693224in,,top]{\rmfamily\fontsize{10.000000}{12.000000}\selectfont \(\displaystyle 0.5\)}%
\end{pgfscope}%
\begin{pgfscope}%
\pgfsetbuttcap%
\pgfsetroundjoin%
\definecolor{currentfill}{rgb}{0.000000,0.000000,0.000000}%
\pgfsetfillcolor{currentfill}%
\pgfsetlinewidth{0.803000pt}%
\definecolor{currentstroke}{rgb}{0.000000,0.000000,0.000000}%
\pgfsetstrokecolor{currentstroke}%
\pgfsetdash{}{0pt}%
\pgfsys@defobject{currentmarker}{\pgfqpoint{0.000000in}{-0.048611in}}{\pgfqpoint{0.000000in}{0.000000in}}{%
\pgfpathmoveto{\pgfqpoint{0.000000in}{0.000000in}}%
\pgfpathlineto{\pgfqpoint{0.000000in}{-0.048611in}}%
\pgfusepath{stroke,fill}%
}%
\begin{pgfscope}%
\pgfsys@transformshift{2.093278in}{0.790446in}%
\pgfsys@useobject{currentmarker}{}%
\end{pgfscope}%
\end{pgfscope}%
\begin{pgfscope}%
\pgftext[x=2.093278in,y=0.693224in,,top]{\rmfamily\fontsize{10.000000}{12.000000}\selectfont \(\displaystyle 1.0\)}%
\end{pgfscope}%
\begin{pgfscope}%
\pgfsetbuttcap%
\pgfsetroundjoin%
\definecolor{currentfill}{rgb}{0.000000,0.000000,0.000000}%
\pgfsetfillcolor{currentfill}%
\pgfsetlinewidth{0.803000pt}%
\definecolor{currentstroke}{rgb}{0.000000,0.000000,0.000000}%
\pgfsetstrokecolor{currentstroke}%
\pgfsetdash{}{0pt}%
\pgfsys@defobject{currentmarker}{\pgfqpoint{0.000000in}{-0.048611in}}{\pgfqpoint{0.000000in}{0.000000in}}{%
\pgfpathmoveto{\pgfqpoint{0.000000in}{0.000000in}}%
\pgfpathlineto{\pgfqpoint{0.000000in}{-0.048611in}}%
\pgfusepath{stroke,fill}%
}%
\begin{pgfscope}%
\pgfsys@transformshift{2.691688in}{0.790446in}%
\pgfsys@useobject{currentmarker}{}%
\end{pgfscope}%
\end{pgfscope}%
\begin{pgfscope}%
\pgftext[x=2.691688in,y=0.693224in,,top]{\rmfamily\fontsize{10.000000}{12.000000}\selectfont \(\displaystyle 1.5\)}%
\end{pgfscope}%
\begin{pgfscope}%
\pgfsetbuttcap%
\pgfsetroundjoin%
\definecolor{currentfill}{rgb}{0.000000,0.000000,0.000000}%
\pgfsetfillcolor{currentfill}%
\pgfsetlinewidth{0.803000pt}%
\definecolor{currentstroke}{rgb}{0.000000,0.000000,0.000000}%
\pgfsetstrokecolor{currentstroke}%
\pgfsetdash{}{0pt}%
\pgfsys@defobject{currentmarker}{\pgfqpoint{-0.048611in}{0.000000in}}{\pgfqpoint{0.000000in}{0.000000in}}{%
\pgfpathmoveto{\pgfqpoint{0.000000in}{0.000000in}}%
\pgfpathlineto{\pgfqpoint{-0.048611in}{0.000000in}}%
\pgfusepath{stroke,fill}%
}%
\begin{pgfscope}%
\pgfsys@transformshift{0.880000in}{1.236240in}%
\pgfsys@useobject{currentmarker}{}%
\end{pgfscope}%
\end{pgfscope}%
\begin{pgfscope}%
\pgftext[x=0.494775in,y=1.183479in,left,base]{\rmfamily\fontsize{10.000000}{12.000000}\selectfont \(\displaystyle 10^{-8}\)}%
\end{pgfscope}%
\begin{pgfscope}%
\pgfsetbuttcap%
\pgfsetroundjoin%
\definecolor{currentfill}{rgb}{0.000000,0.000000,0.000000}%
\pgfsetfillcolor{currentfill}%
\pgfsetlinewidth{0.803000pt}%
\definecolor{currentstroke}{rgb}{0.000000,0.000000,0.000000}%
\pgfsetstrokecolor{currentstroke}%
\pgfsetdash{}{0pt}%
\pgfsys@defobject{currentmarker}{\pgfqpoint{-0.048611in}{0.000000in}}{\pgfqpoint{0.000000in}{0.000000in}}{%
\pgfpathmoveto{\pgfqpoint{0.000000in}{0.000000in}}%
\pgfpathlineto{\pgfqpoint{-0.048611in}{0.000000in}}%
\pgfusepath{stroke,fill}%
}%
\begin{pgfscope}%
\pgfsys@transformshift{0.880000in}{1.814659in}%
\pgfsys@useobject{currentmarker}{}%
\end{pgfscope}%
\end{pgfscope}%
\begin{pgfscope}%
\pgftext[x=0.494775in,y=1.761897in,left,base]{\rmfamily\fontsize{10.000000}{12.000000}\selectfont \(\displaystyle 10^{-6}\)}%
\end{pgfscope}%
\begin{pgfscope}%
\pgfpathrectangle{\pgfqpoint{0.880000in}{0.790446in}}{\pgfqpoint{1.897959in}{1.372727in}} %
\pgfusepath{clip}%
\pgfsetbuttcap%
\pgfsetroundjoin%
\pgfsetlinewidth{1.505625pt}%
\definecolor{currentstroke}{rgb}{1.000000,0.000000,0.000000}%
\pgfsetstrokecolor{currentstroke}%
\pgfsetdash{{5.550000pt}{2.400000pt}}{0.000000pt}%
\pgfpathmoveto{\pgfqpoint{0.966271in}{2.043665in}}%
\pgfpathlineto{\pgfqpoint{1.095927in}{2.017278in}}%
\pgfpathlineto{\pgfqpoint{1.205635in}{1.997116in}}%
\pgfpathlineto{\pgfqpoint{1.315344in}{1.979177in}}%
\pgfpathlineto{\pgfqpoint{1.425052in}{1.963437in}}%
\pgfpathlineto{\pgfqpoint{1.544735in}{1.948600in}}%
\pgfpathlineto{\pgfqpoint{1.664417in}{1.935959in}}%
\pgfpathlineto{\pgfqpoint{1.794072in}{1.924464in}}%
\pgfpathlineto{\pgfqpoint{1.933701in}{1.914344in}}%
\pgfpathlineto{\pgfqpoint{2.083304in}{1.905798in}}%
\pgfpathlineto{\pgfqpoint{2.242880in}{1.899013in}}%
\pgfpathlineto{\pgfqpoint{2.402456in}{1.894406in}}%
\pgfpathlineto{\pgfqpoint{2.572006in}{1.891722in}}%
\pgfpathlineto{\pgfqpoint{2.691688in}{1.891138in}}%
\pgfpathlineto{\pgfqpoint{2.691688in}{1.891138in}}%
\pgfusepath{stroke}%
\end{pgfscope}%
\begin{pgfscope}%
\pgfpathrectangle{\pgfqpoint{0.880000in}{0.790446in}}{\pgfqpoint{1.897959in}{1.372727in}} %
\pgfusepath{clip}%
\pgfsetbuttcap%
\pgfsetmiterjoin%
\definecolor{currentfill}{rgb}{1.000000,0.000000,0.000000}%
\pgfsetfillcolor{currentfill}%
\pgfsetlinewidth{1.003750pt}%
\definecolor{currentstroke}{rgb}{1.000000,0.000000,0.000000}%
\pgfsetstrokecolor{currentstroke}%
\pgfsetdash{}{0pt}%
\pgfsys@defobject{currentmarker}{\pgfqpoint{-0.041667in}{-0.041667in}}{\pgfqpoint{0.041667in}{0.041667in}}{%
\pgfpathmoveto{\pgfqpoint{-0.041667in}{-0.041667in}}%
\pgfpathlineto{\pgfqpoint{0.041667in}{-0.041667in}}%
\pgfpathlineto{\pgfqpoint{0.041667in}{0.041667in}}%
\pgfpathlineto{\pgfqpoint{-0.041667in}{0.041667in}}%
\pgfpathclose%
\pgfusepath{stroke,fill}%
}%
\begin{pgfscope}%
\pgfsys@transformshift{0.966271in}{2.043665in}%
\pgfsys@useobject{currentmarker}{}%
\end{pgfscope}%
\begin{pgfscope}%
\pgfsys@transformshift{1.315344in}{1.979177in}%
\pgfsys@useobject{currentmarker}{}%
\end{pgfscope}%
\begin{pgfscope}%
\pgfsys@transformshift{1.664417in}{1.935959in}%
\pgfsys@useobject{currentmarker}{}%
\end{pgfscope}%
\begin{pgfscope}%
\pgfsys@transformshift{2.013490in}{1.909508in}%
\pgfsys@useobject{currentmarker}{}%
\end{pgfscope}%
\begin{pgfscope}%
\pgfsys@transformshift{2.362562in}{1.895363in}%
\pgfsys@useobject{currentmarker}{}%
\end{pgfscope}%
\end{pgfscope}%
\begin{pgfscope}%
\pgfpathrectangle{\pgfqpoint{0.880000in}{0.790446in}}{\pgfqpoint{1.897959in}{1.372727in}} %
\pgfusepath{clip}%
\pgfsetrectcap%
\pgfsetroundjoin%
\pgfsetlinewidth{1.505625pt}%
\definecolor{currentstroke}{rgb}{0.000000,0.000000,1.000000}%
\pgfsetstrokecolor{currentstroke}%
\pgfsetdash{}{0pt}%
\pgfpathmoveto{\pgfqpoint{0.966271in}{1.734005in}}%
\pgfpathlineto{\pgfqpoint{1.016138in}{1.720666in}}%
\pgfpathlineto{\pgfqpoint{1.095927in}{1.702318in}}%
\pgfpathlineto{\pgfqpoint{1.215609in}{1.677315in}}%
\pgfpathlineto{\pgfqpoint{1.415079in}{1.638292in}}%
\pgfpathlineto{\pgfqpoint{1.734231in}{1.575956in}}%
\pgfpathlineto{\pgfqpoint{1.873860in}{1.546401in}}%
\pgfpathlineto{\pgfqpoint{1.983569in}{1.521106in}}%
\pgfpathlineto{\pgfqpoint{2.073331in}{1.498399in}}%
\pgfpathlineto{\pgfqpoint{2.153119in}{1.476099in}}%
\pgfpathlineto{\pgfqpoint{2.222933in}{1.454369in}}%
\pgfpathlineto{\pgfqpoint{2.282774in}{1.433536in}}%
\pgfpathlineto{\pgfqpoint{2.332642in}{1.414117in}}%
\pgfpathlineto{\pgfqpoint{2.382509in}{1.392219in}}%
\pgfpathlineto{\pgfqpoint{2.422404in}{1.372344in}}%
\pgfpathlineto{\pgfqpoint{2.462298in}{1.349657in}}%
\pgfpathlineto{\pgfqpoint{2.492218in}{1.330185in}}%
\pgfpathlineto{\pgfqpoint{2.522139in}{1.307870in}}%
\pgfpathlineto{\pgfqpoint{2.542086in}{1.290914in}}%
\pgfpathlineto{\pgfqpoint{2.562033in}{1.271765in}}%
\pgfpathlineto{\pgfqpoint{2.581980in}{1.249727in}}%
\pgfpathlineto{\pgfqpoint{2.601927in}{1.223709in}}%
\pgfpathlineto{\pgfqpoint{2.621874in}{1.191850in}}%
\pgfpathlineto{\pgfqpoint{2.631847in}{1.172730in}}%
\pgfpathlineto{\pgfqpoint{2.641821in}{1.150563in}}%
\pgfpathlineto{\pgfqpoint{2.651794in}{1.124145in}}%
\pgfpathlineto{\pgfqpoint{2.661768in}{1.091381in}}%
\pgfpathlineto{\pgfqpoint{2.671741in}{1.048081in}}%
\pgfpathlineto{\pgfqpoint{2.681715in}{0.983699in}}%
\pgfpathlineto{\pgfqpoint{2.691688in}{0.852843in}}%
\pgfpathlineto{\pgfqpoint{2.691688in}{0.852843in}}%
\pgfusepath{stroke}%
\end{pgfscope}%
\begin{pgfscope}%
\pgfpathrectangle{\pgfqpoint{0.880000in}{0.790446in}}{\pgfqpoint{1.897959in}{1.372727in}} %
\pgfusepath{clip}%
\pgfsetbuttcap%
\pgfsetroundjoin%
\definecolor{currentfill}{rgb}{0.000000,0.000000,1.000000}%
\pgfsetfillcolor{currentfill}%
\pgfsetlinewidth{1.003750pt}%
\definecolor{currentstroke}{rgb}{0.000000,0.000000,1.000000}%
\pgfsetstrokecolor{currentstroke}%
\pgfsetdash{}{0pt}%
\pgfsys@defobject{currentmarker}{\pgfqpoint{-0.041667in}{-0.041667in}}{\pgfqpoint{0.041667in}{0.041667in}}{%
\pgfpathmoveto{\pgfqpoint{0.000000in}{-0.041667in}}%
\pgfpathcurveto{\pgfqpoint{0.011050in}{-0.041667in}}{\pgfqpoint{0.021649in}{-0.037276in}}{\pgfqpoint{0.029463in}{-0.029463in}}%
\pgfpathcurveto{\pgfqpoint{0.037276in}{-0.021649in}}{\pgfqpoint{0.041667in}{-0.011050in}}{\pgfqpoint{0.041667in}{0.000000in}}%
\pgfpathcurveto{\pgfqpoint{0.041667in}{0.011050in}}{\pgfqpoint{0.037276in}{0.021649in}}{\pgfqpoint{0.029463in}{0.029463in}}%
\pgfpathcurveto{\pgfqpoint{0.021649in}{0.037276in}}{\pgfqpoint{0.011050in}{0.041667in}}{\pgfqpoint{0.000000in}{0.041667in}}%
\pgfpathcurveto{\pgfqpoint{-0.011050in}{0.041667in}}{\pgfqpoint{-0.021649in}{0.037276in}}{\pgfqpoint{-0.029463in}{0.029463in}}%
\pgfpathcurveto{\pgfqpoint{-0.037276in}{0.021649in}}{\pgfqpoint{-0.041667in}{0.011050in}}{\pgfqpoint{-0.041667in}{0.000000in}}%
\pgfpathcurveto{\pgfqpoint{-0.041667in}{-0.011050in}}{\pgfqpoint{-0.037276in}{-0.021649in}}{\pgfqpoint{-0.029463in}{-0.029463in}}%
\pgfpathcurveto{\pgfqpoint{-0.021649in}{-0.037276in}}{\pgfqpoint{-0.011050in}{-0.041667in}}{\pgfqpoint{0.000000in}{-0.041667in}}%
\pgfpathclose%
\pgfusepath{stroke,fill}%
}%
\begin{pgfscope}%
\pgfsys@transformshift{0.966271in}{1.734005in}%
\pgfsys@useobject{currentmarker}{}%
\end{pgfscope}%
\begin{pgfscope}%
\pgfsys@transformshift{1.315344in}{1.657568in}%
\pgfsys@useobject{currentmarker}{}%
\end{pgfscope}%
\begin{pgfscope}%
\pgfsys@transformshift{1.664417in}{1.589995in}%
\pgfsys@useobject{currentmarker}{}%
\end{pgfscope}%
\begin{pgfscope}%
\pgfsys@transformshift{2.013490in}{1.513772in}%
\pgfsys@useobject{currentmarker}{}%
\end{pgfscope}%
\begin{pgfscope}%
\pgfsys@transformshift{2.362562in}{1.401323in}%
\pgfsys@useobject{currentmarker}{}%
\end{pgfscope}%
\end{pgfscope}%
\begin{pgfscope}%
\pgfpathrectangle{\pgfqpoint{0.880000in}{0.790446in}}{\pgfqpoint{1.897959in}{1.372727in}} %
\pgfusepath{clip}%
\pgfsetbuttcap%
\pgfsetroundjoin%
\pgfsetlinewidth{1.505625pt}%
\definecolor{currentstroke}{rgb}{0.000000,0.750000,0.750000}%
\pgfsetstrokecolor{currentstroke}%
\pgfsetdash{{9.600000pt}{2.400000pt}{1.500000pt}{2.400000pt}}{0.000000pt}%
\pgfpathmoveto{\pgfqpoint{0.966271in}{1.956604in}}%
\pgfpathlineto{\pgfqpoint{0.986218in}{1.907845in}}%
\pgfpathlineto{\pgfqpoint{1.006165in}{1.868227in}}%
\pgfpathlineto{\pgfqpoint{1.026112in}{1.834967in}}%
\pgfpathlineto{\pgfqpoint{1.046059in}{1.806388in}}%
\pgfpathlineto{\pgfqpoint{1.066006in}{1.781402in}}%
\pgfpathlineto{\pgfqpoint{1.085953in}{1.759260in}}%
\pgfpathlineto{\pgfqpoint{1.115874in}{1.730245in}}%
\pgfpathlineto{\pgfqpoint{1.145794in}{1.705181in}}%
\pgfpathlineto{\pgfqpoint{1.175715in}{1.683209in}}%
\pgfpathlineto{\pgfqpoint{1.205635in}{1.663721in}}%
\pgfpathlineto{\pgfqpoint{1.245529in}{1.640850in}}%
\pgfpathlineto{\pgfqpoint{1.285423in}{1.620847in}}%
\pgfpathlineto{\pgfqpoint{1.325317in}{1.603165in}}%
\pgfpathlineto{\pgfqpoint{1.375185in}{1.583715in}}%
\pgfpathlineto{\pgfqpoint{1.425052in}{1.566678in}}%
\pgfpathlineto{\pgfqpoint{1.484893in}{1.548818in}}%
\pgfpathlineto{\pgfqpoint{1.544735in}{1.533271in}}%
\pgfpathlineto{\pgfqpoint{1.614549in}{1.517524in}}%
\pgfpathlineto{\pgfqpoint{1.694337in}{1.502114in}}%
\pgfpathlineto{\pgfqpoint{1.774125in}{1.488976in}}%
\pgfpathlineto{\pgfqpoint{1.863887in}{1.476445in}}%
\pgfpathlineto{\pgfqpoint{1.963622in}{1.464860in}}%
\pgfpathlineto{\pgfqpoint{2.073331in}{1.454509in}}%
\pgfpathlineto{\pgfqpoint{2.193013in}{1.445650in}}%
\pgfpathlineto{\pgfqpoint{2.312695in}{1.438994in}}%
\pgfpathlineto{\pgfqpoint{2.442351in}{1.433993in}}%
\pgfpathlineto{\pgfqpoint{2.581980in}{1.430956in}}%
\pgfpathlineto{\pgfqpoint{2.691688in}{1.430187in}}%
\pgfpathlineto{\pgfqpoint{2.691688in}{1.430187in}}%
\pgfusepath{stroke}%
\end{pgfscope}%
\begin{pgfscope}%
\pgfpathrectangle{\pgfqpoint{0.880000in}{0.790446in}}{\pgfqpoint{1.897959in}{1.372727in}} %
\pgfusepath{clip}%
\pgfsetbuttcap%
\pgfsetmiterjoin%
\definecolor{currentfill}{rgb}{0.000000,0.750000,0.750000}%
\pgfsetfillcolor{currentfill}%
\pgfsetlinewidth{1.003750pt}%
\definecolor{currentstroke}{rgb}{0.000000,0.750000,0.750000}%
\pgfsetstrokecolor{currentstroke}%
\pgfsetdash{}{0pt}%
\pgfsys@defobject{currentmarker}{\pgfqpoint{-0.041667in}{-0.041667in}}{\pgfqpoint{0.041667in}{0.041667in}}{%
\pgfpathmoveto{\pgfqpoint{-0.000000in}{-0.041667in}}%
\pgfpathlineto{\pgfqpoint{0.041667in}{0.041667in}}%
\pgfpathlineto{\pgfqpoint{-0.041667in}{0.041667in}}%
\pgfpathclose%
\pgfusepath{stroke,fill}%
}%
\begin{pgfscope}%
\pgfsys@transformshift{0.966271in}{1.956604in}%
\pgfsys@useobject{currentmarker}{}%
\end{pgfscope}%
\begin{pgfscope}%
\pgfsys@transformshift{1.315344in}{1.607392in}%
\pgfsys@useobject{currentmarker}{}%
\end{pgfscope}%
\begin{pgfscope}%
\pgfsys@transformshift{1.664417in}{1.507602in}%
\pgfsys@useobject{currentmarker}{}%
\end{pgfscope}%
\begin{pgfscope}%
\pgfsys@transformshift{2.013490in}{1.459868in}%
\pgfsys@useobject{currentmarker}{}%
\end{pgfscope}%
\begin{pgfscope}%
\pgfsys@transformshift{2.362562in}{1.436810in}%
\pgfsys@useobject{currentmarker}{}%
\end{pgfscope}%
\end{pgfscope}%
\begin{pgfscope}%
\pgfpathrectangle{\pgfqpoint{0.880000in}{0.790446in}}{\pgfqpoint{1.897959in}{1.372727in}} %
\pgfusepath{clip}%
\pgfsetbuttcap%
\pgfsetroundjoin%
\pgfsetlinewidth{1.505625pt}%
\definecolor{currentstroke}{rgb}{0.000000,0.000000,0.000000}%
\pgfsetstrokecolor{currentstroke}%
\pgfsetdash{{1.500000pt}{2.475000pt}}{0.000000pt}%
\pgfpathmoveto{\pgfqpoint{0.966271in}{2.100776in}}%
\pgfpathlineto{\pgfqpoint{0.986218in}{2.085041in}}%
\pgfpathlineto{\pgfqpoint{1.006165in}{2.072915in}}%
\pgfpathlineto{\pgfqpoint{1.026112in}{2.063490in}}%
\pgfpathlineto{\pgfqpoint{1.056032in}{2.052227in}}%
\pgfpathlineto{\pgfqpoint{1.095927in}{2.040172in}}%
\pgfpathlineto{\pgfqpoint{1.145794in}{2.027637in}}%
\pgfpathlineto{\pgfqpoint{1.215609in}{2.012678in}}%
\pgfpathlineto{\pgfqpoint{1.295397in}{1.997853in}}%
\pgfpathlineto{\pgfqpoint{1.395132in}{1.981701in}}%
\pgfpathlineto{\pgfqpoint{1.504840in}{1.966318in}}%
\pgfpathlineto{\pgfqpoint{1.624523in}{1.951885in}}%
\pgfpathlineto{\pgfqpoint{1.754178in}{1.938595in}}%
\pgfpathlineto{\pgfqpoint{1.893807in}{1.926634in}}%
\pgfpathlineto{\pgfqpoint{2.043410in}{1.916182in}}%
\pgfpathlineto{\pgfqpoint{2.193013in}{1.907915in}}%
\pgfpathlineto{\pgfqpoint{2.352589in}{1.901300in}}%
\pgfpathlineto{\pgfqpoint{2.512165in}{1.896831in}}%
\pgfpathlineto{\pgfqpoint{2.671741in}{1.894473in}}%
\pgfpathlineto{\pgfqpoint{2.691688in}{1.894345in}}%
\pgfpathlineto{\pgfqpoint{2.691688in}{1.894345in}}%
\pgfusepath{stroke}%
\end{pgfscope}%
\begin{pgfscope}%
\pgfpathrectangle{\pgfqpoint{0.880000in}{0.790446in}}{\pgfqpoint{1.897959in}{1.372727in}} %
\pgfusepath{clip}%
\pgfsetbuttcap%
\pgfsetroundjoin%
\definecolor{currentfill}{rgb}{0.000000,0.000000,0.000000}%
\pgfsetfillcolor{currentfill}%
\pgfsetlinewidth{1.003750pt}%
\definecolor{currentstroke}{rgb}{0.000000,0.000000,0.000000}%
\pgfsetstrokecolor{currentstroke}%
\pgfsetdash{}{0pt}%
\pgfsys@defobject{currentmarker}{\pgfqpoint{-0.041667in}{-0.041667in}}{\pgfqpoint{0.041667in}{0.041667in}}{%
\pgfpathmoveto{\pgfqpoint{-0.041667in}{0.000000in}}%
\pgfpathlineto{\pgfqpoint{0.041667in}{0.000000in}}%
\pgfpathmoveto{\pgfqpoint{0.000000in}{-0.041667in}}%
\pgfpathlineto{\pgfqpoint{0.000000in}{0.041667in}}%
\pgfusepath{stroke,fill}%
}%
\begin{pgfscope}%
\pgfsys@transformshift{0.966271in}{2.100776in}%
\pgfsys@useobject{currentmarker}{}%
\end{pgfscope}%
\begin{pgfscope}%
\pgfsys@transformshift{1.315344in}{1.994433in}%
\pgfsys@useobject{currentmarker}{}%
\end{pgfscope}%
\begin{pgfscope}%
\pgfsys@transformshift{1.664417in}{1.947553in}%
\pgfsys@useobject{currentmarker}{}%
\end{pgfscope}%
\begin{pgfscope}%
\pgfsys@transformshift{2.013490in}{1.918091in}%
\pgfsys@useobject{currentmarker}{}%
\end{pgfscope}%
\begin{pgfscope}%
\pgfsys@transformshift{2.362562in}{1.900959in}%
\pgfsys@useobject{currentmarker}{}%
\end{pgfscope}%
\end{pgfscope}%
\begin{pgfscope}%
\pgfsetrectcap%
\pgfsetmiterjoin%
\pgfsetlinewidth{0.803000pt}%
\definecolor{currentstroke}{rgb}{0.000000,0.000000,0.000000}%
\pgfsetstrokecolor{currentstroke}%
\pgfsetdash{}{0pt}%
\pgfpathmoveto{\pgfqpoint{0.880000in}{0.790446in}}%
\pgfpathlineto{\pgfqpoint{0.880000in}{2.163173in}}%
\pgfusepath{stroke}%
\end{pgfscope}%
\begin{pgfscope}%
\pgfsetrectcap%
\pgfsetmiterjoin%
\pgfsetlinewidth{0.803000pt}%
\definecolor{currentstroke}{rgb}{0.000000,0.000000,0.000000}%
\pgfsetstrokecolor{currentstroke}%
\pgfsetdash{}{0pt}%
\pgfpathmoveto{\pgfqpoint{2.777959in}{0.790446in}}%
\pgfpathlineto{\pgfqpoint{2.777959in}{2.163173in}}%
\pgfusepath{stroke}%
\end{pgfscope}%
\begin{pgfscope}%
\pgfsetrectcap%
\pgfsetmiterjoin%
\pgfsetlinewidth{0.803000pt}%
\definecolor{currentstroke}{rgb}{0.000000,0.000000,0.000000}%
\pgfsetstrokecolor{currentstroke}%
\pgfsetdash{}{0pt}%
\pgfpathmoveto{\pgfqpoint{0.880000in}{0.790446in}}%
\pgfpathlineto{\pgfqpoint{2.777959in}{0.790446in}}%
\pgfusepath{stroke}%
\end{pgfscope}%
\begin{pgfscope}%
\pgfsetrectcap%
\pgfsetmiterjoin%
\pgfsetlinewidth{0.803000pt}%
\definecolor{currentstroke}{rgb}{0.000000,0.000000,0.000000}%
\pgfsetstrokecolor{currentstroke}%
\pgfsetdash{}{0pt}%
\pgfpathmoveto{\pgfqpoint{0.880000in}{2.163173in}}%
\pgfpathlineto{\pgfqpoint{2.777959in}{2.163173in}}%
\pgfusepath{stroke}%
\end{pgfscope}%
\begin{pgfscope}%
\pgfsetbuttcap%
\pgfsetmiterjoin%
\definecolor{currentfill}{rgb}{1.000000,1.000000,1.000000}%
\pgfsetfillcolor{currentfill}%
\pgfsetlinewidth{0.000000pt}%
\definecolor{currentstroke}{rgb}{0.000000,0.000000,0.000000}%
\pgfsetstrokecolor{currentstroke}%
\pgfsetstrokeopacity{0.000000}%
\pgfsetdash{}{0pt}%
\pgfpathmoveto{\pgfqpoint{3.347347in}{0.790446in}}%
\pgfpathlineto{\pgfqpoint{5.245306in}{0.790446in}}%
\pgfpathlineto{\pgfqpoint{5.245306in}{2.163173in}}%
\pgfpathlineto{\pgfqpoint{3.347347in}{2.163173in}}%
\pgfpathclose%
\pgfusepath{fill}%
\end{pgfscope}%
\begin{pgfscope}%
\pgfsetbuttcap%
\pgfsetroundjoin%
\definecolor{currentfill}{rgb}{0.000000,0.000000,0.000000}%
\pgfsetfillcolor{currentfill}%
\pgfsetlinewidth{0.803000pt}%
\definecolor{currentstroke}{rgb}{0.000000,0.000000,0.000000}%
\pgfsetstrokecolor{currentstroke}%
\pgfsetdash{}{0pt}%
\pgfsys@defobject{currentmarker}{\pgfqpoint{0.000000in}{-0.048611in}}{\pgfqpoint{0.000000in}{0.000000in}}{%
\pgfpathmoveto{\pgfqpoint{0.000000in}{0.000000in}}%
\pgfpathlineto{\pgfqpoint{0.000000in}{-0.048611in}}%
\pgfusepath{stroke,fill}%
}%
\begin{pgfscope}%
\pgfsys@transformshift{3.346953in}{0.790446in}%
\pgfsys@useobject{currentmarker}{}%
\end{pgfscope}%
\end{pgfscope}%
\begin{pgfscope}%
\pgftext[x=3.346953in,y=0.693224in,,top]{\rmfamily\fontsize{10.000000}{12.000000}\selectfont \(\displaystyle 0\)}%
\end{pgfscope}%
\begin{pgfscope}%
\pgfsetbuttcap%
\pgfsetroundjoin%
\definecolor{currentfill}{rgb}{0.000000,0.000000,0.000000}%
\pgfsetfillcolor{currentfill}%
\pgfsetlinewidth{0.803000pt}%
\definecolor{currentstroke}{rgb}{0.000000,0.000000,0.000000}%
\pgfsetstrokecolor{currentstroke}%
\pgfsetdash{}{0pt}%
\pgfsys@defobject{currentmarker}{\pgfqpoint{0.000000in}{-0.048611in}}{\pgfqpoint{0.000000in}{0.000000in}}{%
\pgfpathmoveto{\pgfqpoint{0.000000in}{0.000000in}}%
\pgfpathlineto{\pgfqpoint{0.000000in}{-0.048611in}}%
\pgfusepath{stroke,fill}%
}%
\begin{pgfscope}%
\pgfsys@transformshift{4.292387in}{0.790446in}%
\pgfsys@useobject{currentmarker}{}%
\end{pgfscope}%
\end{pgfscope}%
\begin{pgfscope}%
\pgftext[x=4.292387in,y=0.693224in,,top]{\rmfamily\fontsize{10.000000}{12.000000}\selectfont \(\displaystyle 1\)}%
\end{pgfscope}%
\begin{pgfscope}%
\pgfsetbuttcap%
\pgfsetroundjoin%
\definecolor{currentfill}{rgb}{0.000000,0.000000,0.000000}%
\pgfsetfillcolor{currentfill}%
\pgfsetlinewidth{0.803000pt}%
\definecolor{currentstroke}{rgb}{0.000000,0.000000,0.000000}%
\pgfsetstrokecolor{currentstroke}%
\pgfsetdash{}{0pt}%
\pgfsys@defobject{currentmarker}{\pgfqpoint{0.000000in}{-0.048611in}}{\pgfqpoint{0.000000in}{0.000000in}}{%
\pgfpathmoveto{\pgfqpoint{0.000000in}{0.000000in}}%
\pgfpathlineto{\pgfqpoint{0.000000in}{-0.048611in}}%
\pgfusepath{stroke,fill}%
}%
\begin{pgfscope}%
\pgfsys@transformshift{5.237821in}{0.790446in}%
\pgfsys@useobject{currentmarker}{}%
\end{pgfscope}%
\end{pgfscope}%
\begin{pgfscope}%
\pgftext[x=5.237821in,y=0.693224in,,top]{\rmfamily\fontsize{10.000000}{12.000000}\selectfont \(\displaystyle 2\)}%
\end{pgfscope}%
\begin{pgfscope}%
\pgfsetbuttcap%
\pgfsetroundjoin%
\definecolor{currentfill}{rgb}{0.000000,0.000000,0.000000}%
\pgfsetfillcolor{currentfill}%
\pgfsetlinewidth{0.803000pt}%
\definecolor{currentstroke}{rgb}{0.000000,0.000000,0.000000}%
\pgfsetstrokecolor{currentstroke}%
\pgfsetdash{}{0pt}%
\pgfsys@defobject{currentmarker}{\pgfqpoint{-0.048611in}{0.000000in}}{\pgfqpoint{0.000000in}{0.000000in}}{%
\pgfpathmoveto{\pgfqpoint{0.000000in}{0.000000in}}%
\pgfpathlineto{\pgfqpoint{-0.048611in}{0.000000in}}%
\pgfusepath{stroke,fill}%
}%
\begin{pgfscope}%
\pgfsys@transformshift{3.347347in}{0.954399in}%
\pgfsys@useobject{currentmarker}{}%
\end{pgfscope}%
\end{pgfscope}%
\begin{pgfscope}%
\pgftext[x=2.962122in,y=0.901638in,left,base]{\rmfamily\fontsize{10.000000}{12.000000}\selectfont \(\displaystyle 10^{-7}\)}%
\end{pgfscope}%
\begin{pgfscope}%
\pgfsetbuttcap%
\pgfsetroundjoin%
\definecolor{currentfill}{rgb}{0.000000,0.000000,0.000000}%
\pgfsetfillcolor{currentfill}%
\pgfsetlinewidth{0.803000pt}%
\definecolor{currentstroke}{rgb}{0.000000,0.000000,0.000000}%
\pgfsetstrokecolor{currentstroke}%
\pgfsetdash{}{0pt}%
\pgfsys@defobject{currentmarker}{\pgfqpoint{-0.048611in}{0.000000in}}{\pgfqpoint{0.000000in}{0.000000in}}{%
\pgfpathmoveto{\pgfqpoint{0.000000in}{0.000000in}}%
\pgfpathlineto{\pgfqpoint{-0.048611in}{0.000000in}}%
\pgfusepath{stroke,fill}%
}%
\begin{pgfscope}%
\pgfsys@transformshift{3.347347in}{1.380173in}%
\pgfsys@useobject{currentmarker}{}%
\end{pgfscope}%
\end{pgfscope}%
\begin{pgfscope}%
\pgftext[x=2.962122in,y=1.327412in,left,base]{\rmfamily\fontsize{10.000000}{12.000000}\selectfont \(\displaystyle 10^{-6}\)}%
\end{pgfscope}%
\begin{pgfscope}%
\pgfsetbuttcap%
\pgfsetroundjoin%
\definecolor{currentfill}{rgb}{0.000000,0.000000,0.000000}%
\pgfsetfillcolor{currentfill}%
\pgfsetlinewidth{0.803000pt}%
\definecolor{currentstroke}{rgb}{0.000000,0.000000,0.000000}%
\pgfsetstrokecolor{currentstroke}%
\pgfsetdash{}{0pt}%
\pgfsys@defobject{currentmarker}{\pgfqpoint{-0.048611in}{0.000000in}}{\pgfqpoint{0.000000in}{0.000000in}}{%
\pgfpathmoveto{\pgfqpoint{0.000000in}{0.000000in}}%
\pgfpathlineto{\pgfqpoint{-0.048611in}{0.000000in}}%
\pgfusepath{stroke,fill}%
}%
\begin{pgfscope}%
\pgfsys@transformshift{3.347347in}{1.805947in}%
\pgfsys@useobject{currentmarker}{}%
\end{pgfscope}%
\end{pgfscope}%
\begin{pgfscope}%
\pgftext[x=2.962122in,y=1.753185in,left,base]{\rmfamily\fontsize{10.000000}{12.000000}\selectfont \(\displaystyle 10^{-5}\)}%
\end{pgfscope}%
\begin{pgfscope}%
\pgfsetbuttcap%
\pgfsetroundjoin%
\definecolor{currentfill}{rgb}{0.000000,0.000000,0.000000}%
\pgfsetfillcolor{currentfill}%
\pgfsetlinewidth{0.602250pt}%
\definecolor{currentstroke}{rgb}{0.000000,0.000000,0.000000}%
\pgfsetstrokecolor{currentstroke}%
\pgfsetdash{}{0pt}%
\pgfsys@defobject{currentmarker}{\pgfqpoint{-0.027778in}{0.000000in}}{\pgfqpoint{0.000000in}{0.000000in}}{%
\pgfpathmoveto{\pgfqpoint{0.000000in}{0.000000in}}%
\pgfpathlineto{\pgfqpoint{-0.027778in}{0.000000in}}%
\pgfusepath{stroke,fill}%
}%
\begin{pgfscope}%
\pgfsys@transformshift{3.347347in}{0.826229in}%
\pgfsys@useobject{currentmarker}{}%
\end{pgfscope}%
\end{pgfscope}%
\begin{pgfscope}%
\pgfsetbuttcap%
\pgfsetroundjoin%
\definecolor{currentfill}{rgb}{0.000000,0.000000,0.000000}%
\pgfsetfillcolor{currentfill}%
\pgfsetlinewidth{0.602250pt}%
\definecolor{currentstroke}{rgb}{0.000000,0.000000,0.000000}%
\pgfsetstrokecolor{currentstroke}%
\pgfsetdash{}{0pt}%
\pgfsys@defobject{currentmarker}{\pgfqpoint{-0.027778in}{0.000000in}}{\pgfqpoint{0.000000in}{0.000000in}}{%
\pgfpathmoveto{\pgfqpoint{0.000000in}{0.000000in}}%
\pgfpathlineto{\pgfqpoint{-0.027778in}{0.000000in}}%
\pgfusepath{stroke,fill}%
}%
\begin{pgfscope}%
\pgfsys@transformshift{3.347347in}{0.859942in}%
\pgfsys@useobject{currentmarker}{}%
\end{pgfscope}%
\end{pgfscope}%
\begin{pgfscope}%
\pgfsetbuttcap%
\pgfsetroundjoin%
\definecolor{currentfill}{rgb}{0.000000,0.000000,0.000000}%
\pgfsetfillcolor{currentfill}%
\pgfsetlinewidth{0.602250pt}%
\definecolor{currentstroke}{rgb}{0.000000,0.000000,0.000000}%
\pgfsetstrokecolor{currentstroke}%
\pgfsetdash{}{0pt}%
\pgfsys@defobject{currentmarker}{\pgfqpoint{-0.027778in}{0.000000in}}{\pgfqpoint{0.000000in}{0.000000in}}{%
\pgfpathmoveto{\pgfqpoint{0.000000in}{0.000000in}}%
\pgfpathlineto{\pgfqpoint{-0.027778in}{0.000000in}}%
\pgfusepath{stroke,fill}%
}%
\begin{pgfscope}%
\pgfsys@transformshift{3.347347in}{0.888446in}%
\pgfsys@useobject{currentmarker}{}%
\end{pgfscope}%
\end{pgfscope}%
\begin{pgfscope}%
\pgfsetbuttcap%
\pgfsetroundjoin%
\definecolor{currentfill}{rgb}{0.000000,0.000000,0.000000}%
\pgfsetfillcolor{currentfill}%
\pgfsetlinewidth{0.602250pt}%
\definecolor{currentstroke}{rgb}{0.000000,0.000000,0.000000}%
\pgfsetstrokecolor{currentstroke}%
\pgfsetdash{}{0pt}%
\pgfsys@defobject{currentmarker}{\pgfqpoint{-0.027778in}{0.000000in}}{\pgfqpoint{0.000000in}{0.000000in}}{%
\pgfpathmoveto{\pgfqpoint{0.000000in}{0.000000in}}%
\pgfpathlineto{\pgfqpoint{-0.027778in}{0.000000in}}%
\pgfusepath{stroke,fill}%
}%
\begin{pgfscope}%
\pgfsys@transformshift{3.347347in}{0.913138in}%
\pgfsys@useobject{currentmarker}{}%
\end{pgfscope}%
\end{pgfscope}%
\begin{pgfscope}%
\pgfsetbuttcap%
\pgfsetroundjoin%
\definecolor{currentfill}{rgb}{0.000000,0.000000,0.000000}%
\pgfsetfillcolor{currentfill}%
\pgfsetlinewidth{0.602250pt}%
\definecolor{currentstroke}{rgb}{0.000000,0.000000,0.000000}%
\pgfsetstrokecolor{currentstroke}%
\pgfsetdash{}{0pt}%
\pgfsys@defobject{currentmarker}{\pgfqpoint{-0.027778in}{0.000000in}}{\pgfqpoint{0.000000in}{0.000000in}}{%
\pgfpathmoveto{\pgfqpoint{0.000000in}{0.000000in}}%
\pgfpathlineto{\pgfqpoint{-0.027778in}{0.000000in}}%
\pgfusepath{stroke,fill}%
}%
\begin{pgfscope}%
\pgfsys@transformshift{3.347347in}{0.934917in}%
\pgfsys@useobject{currentmarker}{}%
\end{pgfscope}%
\end{pgfscope}%
\begin{pgfscope}%
\pgfsetbuttcap%
\pgfsetroundjoin%
\definecolor{currentfill}{rgb}{0.000000,0.000000,0.000000}%
\pgfsetfillcolor{currentfill}%
\pgfsetlinewidth{0.602250pt}%
\definecolor{currentstroke}{rgb}{0.000000,0.000000,0.000000}%
\pgfsetstrokecolor{currentstroke}%
\pgfsetdash{}{0pt}%
\pgfsys@defobject{currentmarker}{\pgfqpoint{-0.027778in}{0.000000in}}{\pgfqpoint{0.000000in}{0.000000in}}{%
\pgfpathmoveto{\pgfqpoint{0.000000in}{0.000000in}}%
\pgfpathlineto{\pgfqpoint{-0.027778in}{0.000000in}}%
\pgfusepath{stroke,fill}%
}%
\begin{pgfscope}%
\pgfsys@transformshift{3.347347in}{1.082570in}%
\pgfsys@useobject{currentmarker}{}%
\end{pgfscope}%
\end{pgfscope}%
\begin{pgfscope}%
\pgfsetbuttcap%
\pgfsetroundjoin%
\definecolor{currentfill}{rgb}{0.000000,0.000000,0.000000}%
\pgfsetfillcolor{currentfill}%
\pgfsetlinewidth{0.602250pt}%
\definecolor{currentstroke}{rgb}{0.000000,0.000000,0.000000}%
\pgfsetstrokecolor{currentstroke}%
\pgfsetdash{}{0pt}%
\pgfsys@defobject{currentmarker}{\pgfqpoint{-0.027778in}{0.000000in}}{\pgfqpoint{0.000000in}{0.000000in}}{%
\pgfpathmoveto{\pgfqpoint{0.000000in}{0.000000in}}%
\pgfpathlineto{\pgfqpoint{-0.027778in}{0.000000in}}%
\pgfusepath{stroke,fill}%
}%
\begin{pgfscope}%
\pgfsys@transformshift{3.347347in}{1.157545in}%
\pgfsys@useobject{currentmarker}{}%
\end{pgfscope}%
\end{pgfscope}%
\begin{pgfscope}%
\pgfsetbuttcap%
\pgfsetroundjoin%
\definecolor{currentfill}{rgb}{0.000000,0.000000,0.000000}%
\pgfsetfillcolor{currentfill}%
\pgfsetlinewidth{0.602250pt}%
\definecolor{currentstroke}{rgb}{0.000000,0.000000,0.000000}%
\pgfsetstrokecolor{currentstroke}%
\pgfsetdash{}{0pt}%
\pgfsys@defobject{currentmarker}{\pgfqpoint{-0.027778in}{0.000000in}}{\pgfqpoint{0.000000in}{0.000000in}}{%
\pgfpathmoveto{\pgfqpoint{0.000000in}{0.000000in}}%
\pgfpathlineto{\pgfqpoint{-0.027778in}{0.000000in}}%
\pgfusepath{stroke,fill}%
}%
\begin{pgfscope}%
\pgfsys@transformshift{3.347347in}{1.210741in}%
\pgfsys@useobject{currentmarker}{}%
\end{pgfscope}%
\end{pgfscope}%
\begin{pgfscope}%
\pgfsetbuttcap%
\pgfsetroundjoin%
\definecolor{currentfill}{rgb}{0.000000,0.000000,0.000000}%
\pgfsetfillcolor{currentfill}%
\pgfsetlinewidth{0.602250pt}%
\definecolor{currentstroke}{rgb}{0.000000,0.000000,0.000000}%
\pgfsetstrokecolor{currentstroke}%
\pgfsetdash{}{0pt}%
\pgfsys@defobject{currentmarker}{\pgfqpoint{-0.027778in}{0.000000in}}{\pgfqpoint{0.000000in}{0.000000in}}{%
\pgfpathmoveto{\pgfqpoint{0.000000in}{0.000000in}}%
\pgfpathlineto{\pgfqpoint{-0.027778in}{0.000000in}}%
\pgfusepath{stroke,fill}%
}%
\begin{pgfscope}%
\pgfsys@transformshift{3.347347in}{1.252002in}%
\pgfsys@useobject{currentmarker}{}%
\end{pgfscope}%
\end{pgfscope}%
\begin{pgfscope}%
\pgfsetbuttcap%
\pgfsetroundjoin%
\definecolor{currentfill}{rgb}{0.000000,0.000000,0.000000}%
\pgfsetfillcolor{currentfill}%
\pgfsetlinewidth{0.602250pt}%
\definecolor{currentstroke}{rgb}{0.000000,0.000000,0.000000}%
\pgfsetstrokecolor{currentstroke}%
\pgfsetdash{}{0pt}%
\pgfsys@defobject{currentmarker}{\pgfqpoint{-0.027778in}{0.000000in}}{\pgfqpoint{0.000000in}{0.000000in}}{%
\pgfpathmoveto{\pgfqpoint{0.000000in}{0.000000in}}%
\pgfpathlineto{\pgfqpoint{-0.027778in}{0.000000in}}%
\pgfusepath{stroke,fill}%
}%
\begin{pgfscope}%
\pgfsys@transformshift{3.347347in}{1.285716in}%
\pgfsys@useobject{currentmarker}{}%
\end{pgfscope}%
\end{pgfscope}%
\begin{pgfscope}%
\pgfsetbuttcap%
\pgfsetroundjoin%
\definecolor{currentfill}{rgb}{0.000000,0.000000,0.000000}%
\pgfsetfillcolor{currentfill}%
\pgfsetlinewidth{0.602250pt}%
\definecolor{currentstroke}{rgb}{0.000000,0.000000,0.000000}%
\pgfsetstrokecolor{currentstroke}%
\pgfsetdash{}{0pt}%
\pgfsys@defobject{currentmarker}{\pgfqpoint{-0.027778in}{0.000000in}}{\pgfqpoint{0.000000in}{0.000000in}}{%
\pgfpathmoveto{\pgfqpoint{0.000000in}{0.000000in}}%
\pgfpathlineto{\pgfqpoint{-0.027778in}{0.000000in}}%
\pgfusepath{stroke,fill}%
}%
\begin{pgfscope}%
\pgfsys@transformshift{3.347347in}{1.314220in}%
\pgfsys@useobject{currentmarker}{}%
\end{pgfscope}%
\end{pgfscope}%
\begin{pgfscope}%
\pgfsetbuttcap%
\pgfsetroundjoin%
\definecolor{currentfill}{rgb}{0.000000,0.000000,0.000000}%
\pgfsetfillcolor{currentfill}%
\pgfsetlinewidth{0.602250pt}%
\definecolor{currentstroke}{rgb}{0.000000,0.000000,0.000000}%
\pgfsetstrokecolor{currentstroke}%
\pgfsetdash{}{0pt}%
\pgfsys@defobject{currentmarker}{\pgfqpoint{-0.027778in}{0.000000in}}{\pgfqpoint{0.000000in}{0.000000in}}{%
\pgfpathmoveto{\pgfqpoint{0.000000in}{0.000000in}}%
\pgfpathlineto{\pgfqpoint{-0.027778in}{0.000000in}}%
\pgfusepath{stroke,fill}%
}%
\begin{pgfscope}%
\pgfsys@transformshift{3.347347in}{1.338911in}%
\pgfsys@useobject{currentmarker}{}%
\end{pgfscope}%
\end{pgfscope}%
\begin{pgfscope}%
\pgfsetbuttcap%
\pgfsetroundjoin%
\definecolor{currentfill}{rgb}{0.000000,0.000000,0.000000}%
\pgfsetfillcolor{currentfill}%
\pgfsetlinewidth{0.602250pt}%
\definecolor{currentstroke}{rgb}{0.000000,0.000000,0.000000}%
\pgfsetstrokecolor{currentstroke}%
\pgfsetdash{}{0pt}%
\pgfsys@defobject{currentmarker}{\pgfqpoint{-0.027778in}{0.000000in}}{\pgfqpoint{0.000000in}{0.000000in}}{%
\pgfpathmoveto{\pgfqpoint{0.000000in}{0.000000in}}%
\pgfpathlineto{\pgfqpoint{-0.027778in}{0.000000in}}%
\pgfusepath{stroke,fill}%
}%
\begin{pgfscope}%
\pgfsys@transformshift{3.347347in}{1.360691in}%
\pgfsys@useobject{currentmarker}{}%
\end{pgfscope}%
\end{pgfscope}%
\begin{pgfscope}%
\pgfsetbuttcap%
\pgfsetroundjoin%
\definecolor{currentfill}{rgb}{0.000000,0.000000,0.000000}%
\pgfsetfillcolor{currentfill}%
\pgfsetlinewidth{0.602250pt}%
\definecolor{currentstroke}{rgb}{0.000000,0.000000,0.000000}%
\pgfsetstrokecolor{currentstroke}%
\pgfsetdash{}{0pt}%
\pgfsys@defobject{currentmarker}{\pgfqpoint{-0.027778in}{0.000000in}}{\pgfqpoint{0.000000in}{0.000000in}}{%
\pgfpathmoveto{\pgfqpoint{0.000000in}{0.000000in}}%
\pgfpathlineto{\pgfqpoint{-0.027778in}{0.000000in}}%
\pgfusepath{stroke,fill}%
}%
\begin{pgfscope}%
\pgfsys@transformshift{3.347347in}{1.508344in}%
\pgfsys@useobject{currentmarker}{}%
\end{pgfscope}%
\end{pgfscope}%
\begin{pgfscope}%
\pgfsetbuttcap%
\pgfsetroundjoin%
\definecolor{currentfill}{rgb}{0.000000,0.000000,0.000000}%
\pgfsetfillcolor{currentfill}%
\pgfsetlinewidth{0.602250pt}%
\definecolor{currentstroke}{rgb}{0.000000,0.000000,0.000000}%
\pgfsetstrokecolor{currentstroke}%
\pgfsetdash{}{0pt}%
\pgfsys@defobject{currentmarker}{\pgfqpoint{-0.027778in}{0.000000in}}{\pgfqpoint{0.000000in}{0.000000in}}{%
\pgfpathmoveto{\pgfqpoint{0.000000in}{0.000000in}}%
\pgfpathlineto{\pgfqpoint{-0.027778in}{0.000000in}}%
\pgfusepath{stroke,fill}%
}%
\begin{pgfscope}%
\pgfsys@transformshift{3.347347in}{1.583319in}%
\pgfsys@useobject{currentmarker}{}%
\end{pgfscope}%
\end{pgfscope}%
\begin{pgfscope}%
\pgfsetbuttcap%
\pgfsetroundjoin%
\definecolor{currentfill}{rgb}{0.000000,0.000000,0.000000}%
\pgfsetfillcolor{currentfill}%
\pgfsetlinewidth{0.602250pt}%
\definecolor{currentstroke}{rgb}{0.000000,0.000000,0.000000}%
\pgfsetstrokecolor{currentstroke}%
\pgfsetdash{}{0pt}%
\pgfsys@defobject{currentmarker}{\pgfqpoint{-0.027778in}{0.000000in}}{\pgfqpoint{0.000000in}{0.000000in}}{%
\pgfpathmoveto{\pgfqpoint{0.000000in}{0.000000in}}%
\pgfpathlineto{\pgfqpoint{-0.027778in}{0.000000in}}%
\pgfusepath{stroke,fill}%
}%
\begin{pgfscope}%
\pgfsys@transformshift{3.347347in}{1.636514in}%
\pgfsys@useobject{currentmarker}{}%
\end{pgfscope}%
\end{pgfscope}%
\begin{pgfscope}%
\pgfsetbuttcap%
\pgfsetroundjoin%
\definecolor{currentfill}{rgb}{0.000000,0.000000,0.000000}%
\pgfsetfillcolor{currentfill}%
\pgfsetlinewidth{0.602250pt}%
\definecolor{currentstroke}{rgb}{0.000000,0.000000,0.000000}%
\pgfsetstrokecolor{currentstroke}%
\pgfsetdash{}{0pt}%
\pgfsys@defobject{currentmarker}{\pgfqpoint{-0.027778in}{0.000000in}}{\pgfqpoint{0.000000in}{0.000000in}}{%
\pgfpathmoveto{\pgfqpoint{0.000000in}{0.000000in}}%
\pgfpathlineto{\pgfqpoint{-0.027778in}{0.000000in}}%
\pgfusepath{stroke,fill}%
}%
\begin{pgfscope}%
\pgfsys@transformshift{3.347347in}{1.677776in}%
\pgfsys@useobject{currentmarker}{}%
\end{pgfscope}%
\end{pgfscope}%
\begin{pgfscope}%
\pgfsetbuttcap%
\pgfsetroundjoin%
\definecolor{currentfill}{rgb}{0.000000,0.000000,0.000000}%
\pgfsetfillcolor{currentfill}%
\pgfsetlinewidth{0.602250pt}%
\definecolor{currentstroke}{rgb}{0.000000,0.000000,0.000000}%
\pgfsetstrokecolor{currentstroke}%
\pgfsetdash{}{0pt}%
\pgfsys@defobject{currentmarker}{\pgfqpoint{-0.027778in}{0.000000in}}{\pgfqpoint{0.000000in}{0.000000in}}{%
\pgfpathmoveto{\pgfqpoint{0.000000in}{0.000000in}}%
\pgfpathlineto{\pgfqpoint{-0.027778in}{0.000000in}}%
\pgfusepath{stroke,fill}%
}%
\begin{pgfscope}%
\pgfsys@transformshift{3.347347in}{1.711490in}%
\pgfsys@useobject{currentmarker}{}%
\end{pgfscope}%
\end{pgfscope}%
\begin{pgfscope}%
\pgfsetbuttcap%
\pgfsetroundjoin%
\definecolor{currentfill}{rgb}{0.000000,0.000000,0.000000}%
\pgfsetfillcolor{currentfill}%
\pgfsetlinewidth{0.602250pt}%
\definecolor{currentstroke}{rgb}{0.000000,0.000000,0.000000}%
\pgfsetstrokecolor{currentstroke}%
\pgfsetdash{}{0pt}%
\pgfsys@defobject{currentmarker}{\pgfqpoint{-0.027778in}{0.000000in}}{\pgfqpoint{0.000000in}{0.000000in}}{%
\pgfpathmoveto{\pgfqpoint{0.000000in}{0.000000in}}%
\pgfpathlineto{\pgfqpoint{-0.027778in}{0.000000in}}%
\pgfusepath{stroke,fill}%
}%
\begin{pgfscope}%
\pgfsys@transformshift{3.347347in}{1.739994in}%
\pgfsys@useobject{currentmarker}{}%
\end{pgfscope}%
\end{pgfscope}%
\begin{pgfscope}%
\pgfsetbuttcap%
\pgfsetroundjoin%
\definecolor{currentfill}{rgb}{0.000000,0.000000,0.000000}%
\pgfsetfillcolor{currentfill}%
\pgfsetlinewidth{0.602250pt}%
\definecolor{currentstroke}{rgb}{0.000000,0.000000,0.000000}%
\pgfsetstrokecolor{currentstroke}%
\pgfsetdash{}{0pt}%
\pgfsys@defobject{currentmarker}{\pgfqpoint{-0.027778in}{0.000000in}}{\pgfqpoint{0.000000in}{0.000000in}}{%
\pgfpathmoveto{\pgfqpoint{0.000000in}{0.000000in}}%
\pgfpathlineto{\pgfqpoint{-0.027778in}{0.000000in}}%
\pgfusepath{stroke,fill}%
}%
\begin{pgfscope}%
\pgfsys@transformshift{3.347347in}{1.764685in}%
\pgfsys@useobject{currentmarker}{}%
\end{pgfscope}%
\end{pgfscope}%
\begin{pgfscope}%
\pgfsetbuttcap%
\pgfsetroundjoin%
\definecolor{currentfill}{rgb}{0.000000,0.000000,0.000000}%
\pgfsetfillcolor{currentfill}%
\pgfsetlinewidth{0.602250pt}%
\definecolor{currentstroke}{rgb}{0.000000,0.000000,0.000000}%
\pgfsetstrokecolor{currentstroke}%
\pgfsetdash{}{0pt}%
\pgfsys@defobject{currentmarker}{\pgfqpoint{-0.027778in}{0.000000in}}{\pgfqpoint{0.000000in}{0.000000in}}{%
\pgfpathmoveto{\pgfqpoint{0.000000in}{0.000000in}}%
\pgfpathlineto{\pgfqpoint{-0.027778in}{0.000000in}}%
\pgfusepath{stroke,fill}%
}%
\begin{pgfscope}%
\pgfsys@transformshift{3.347347in}{1.786465in}%
\pgfsys@useobject{currentmarker}{}%
\end{pgfscope}%
\end{pgfscope}%
\begin{pgfscope}%
\pgfsetbuttcap%
\pgfsetroundjoin%
\definecolor{currentfill}{rgb}{0.000000,0.000000,0.000000}%
\pgfsetfillcolor{currentfill}%
\pgfsetlinewidth{0.602250pt}%
\definecolor{currentstroke}{rgb}{0.000000,0.000000,0.000000}%
\pgfsetstrokecolor{currentstroke}%
\pgfsetdash{}{0pt}%
\pgfsys@defobject{currentmarker}{\pgfqpoint{-0.027778in}{0.000000in}}{\pgfqpoint{0.000000in}{0.000000in}}{%
\pgfpathmoveto{\pgfqpoint{0.000000in}{0.000000in}}%
\pgfpathlineto{\pgfqpoint{-0.027778in}{0.000000in}}%
\pgfusepath{stroke,fill}%
}%
\begin{pgfscope}%
\pgfsys@transformshift{3.347347in}{1.934118in}%
\pgfsys@useobject{currentmarker}{}%
\end{pgfscope}%
\end{pgfscope}%
\begin{pgfscope}%
\pgfsetbuttcap%
\pgfsetroundjoin%
\definecolor{currentfill}{rgb}{0.000000,0.000000,0.000000}%
\pgfsetfillcolor{currentfill}%
\pgfsetlinewidth{0.602250pt}%
\definecolor{currentstroke}{rgb}{0.000000,0.000000,0.000000}%
\pgfsetstrokecolor{currentstroke}%
\pgfsetdash{}{0pt}%
\pgfsys@defobject{currentmarker}{\pgfqpoint{-0.027778in}{0.000000in}}{\pgfqpoint{0.000000in}{0.000000in}}{%
\pgfpathmoveto{\pgfqpoint{0.000000in}{0.000000in}}%
\pgfpathlineto{\pgfqpoint{-0.027778in}{0.000000in}}%
\pgfusepath{stroke,fill}%
}%
\begin{pgfscope}%
\pgfsys@transformshift{3.347347in}{2.009093in}%
\pgfsys@useobject{currentmarker}{}%
\end{pgfscope}%
\end{pgfscope}%
\begin{pgfscope}%
\pgfsetbuttcap%
\pgfsetroundjoin%
\definecolor{currentfill}{rgb}{0.000000,0.000000,0.000000}%
\pgfsetfillcolor{currentfill}%
\pgfsetlinewidth{0.602250pt}%
\definecolor{currentstroke}{rgb}{0.000000,0.000000,0.000000}%
\pgfsetstrokecolor{currentstroke}%
\pgfsetdash{}{0pt}%
\pgfsys@defobject{currentmarker}{\pgfqpoint{-0.027778in}{0.000000in}}{\pgfqpoint{0.000000in}{0.000000in}}{%
\pgfpathmoveto{\pgfqpoint{0.000000in}{0.000000in}}%
\pgfpathlineto{\pgfqpoint{-0.027778in}{0.000000in}}%
\pgfusepath{stroke,fill}%
}%
\begin{pgfscope}%
\pgfsys@transformshift{3.347347in}{2.062288in}%
\pgfsys@useobject{currentmarker}{}%
\end{pgfscope}%
\end{pgfscope}%
\begin{pgfscope}%
\pgfsetbuttcap%
\pgfsetroundjoin%
\definecolor{currentfill}{rgb}{0.000000,0.000000,0.000000}%
\pgfsetfillcolor{currentfill}%
\pgfsetlinewidth{0.602250pt}%
\definecolor{currentstroke}{rgb}{0.000000,0.000000,0.000000}%
\pgfsetstrokecolor{currentstroke}%
\pgfsetdash{}{0pt}%
\pgfsys@defobject{currentmarker}{\pgfqpoint{-0.027778in}{0.000000in}}{\pgfqpoint{0.000000in}{0.000000in}}{%
\pgfpathmoveto{\pgfqpoint{0.000000in}{0.000000in}}%
\pgfpathlineto{\pgfqpoint{-0.027778in}{0.000000in}}%
\pgfusepath{stroke,fill}%
}%
\begin{pgfscope}%
\pgfsys@transformshift{3.347347in}{2.103550in}%
\pgfsys@useobject{currentmarker}{}%
\end{pgfscope}%
\end{pgfscope}%
\begin{pgfscope}%
\pgfsetbuttcap%
\pgfsetroundjoin%
\definecolor{currentfill}{rgb}{0.000000,0.000000,0.000000}%
\pgfsetfillcolor{currentfill}%
\pgfsetlinewidth{0.602250pt}%
\definecolor{currentstroke}{rgb}{0.000000,0.000000,0.000000}%
\pgfsetstrokecolor{currentstroke}%
\pgfsetdash{}{0pt}%
\pgfsys@defobject{currentmarker}{\pgfqpoint{-0.027778in}{0.000000in}}{\pgfqpoint{0.000000in}{0.000000in}}{%
\pgfpathmoveto{\pgfqpoint{0.000000in}{0.000000in}}%
\pgfpathlineto{\pgfqpoint{-0.027778in}{0.000000in}}%
\pgfusepath{stroke,fill}%
}%
\begin{pgfscope}%
\pgfsys@transformshift{3.347347in}{2.137263in}%
\pgfsys@useobject{currentmarker}{}%
\end{pgfscope}%
\end{pgfscope}%
\begin{pgfscope}%
\pgfsetbuttcap%
\pgfsetroundjoin%
\definecolor{currentfill}{rgb}{0.000000,0.000000,0.000000}%
\pgfsetfillcolor{currentfill}%
\pgfsetlinewidth{0.602250pt}%
\definecolor{currentstroke}{rgb}{0.000000,0.000000,0.000000}%
\pgfsetstrokecolor{currentstroke}%
\pgfsetdash{}{0pt}%
\pgfsys@defobject{currentmarker}{\pgfqpoint{-0.027778in}{0.000000in}}{\pgfqpoint{0.000000in}{0.000000in}}{%
\pgfpathmoveto{\pgfqpoint{0.000000in}{0.000000in}}%
\pgfpathlineto{\pgfqpoint{-0.027778in}{0.000000in}}%
\pgfusepath{stroke,fill}%
}%
\begin{pgfscope}%
\pgfsys@transformshift{3.347347in}{2.165768in}%
\pgfsys@useobject{currentmarker}{}%
\end{pgfscope}%
\end{pgfscope}%
\begin{pgfscope}%
\pgfpathrectangle{\pgfqpoint{3.347347in}{0.790446in}}{\pgfqpoint{1.897959in}{1.372727in}} %
\pgfusepath{clip}%
\pgfsetbuttcap%
\pgfsetroundjoin%
\pgfsetlinewidth{1.505625pt}%
\definecolor{currentstroke}{rgb}{1.000000,0.000000,0.000000}%
\pgfsetstrokecolor{currentstroke}%
\pgfsetdash{{5.550000pt}{2.400000pt}}{0.000000pt}%
\pgfpathmoveto{\pgfqpoint{3.433618in}{2.022449in}}%
\pgfpathlineto{\pgfqpoint{3.504525in}{1.989368in}}%
\pgfpathlineto{\pgfqpoint{3.575433in}{1.958899in}}%
\pgfpathlineto{\pgfqpoint{3.638462in}{1.934124in}}%
\pgfpathlineto{\pgfqpoint{3.709369in}{1.908785in}}%
\pgfpathlineto{\pgfqpoint{3.780277in}{1.885944in}}%
\pgfpathlineto{\pgfqpoint{3.851185in}{1.865367in}}%
\pgfpathlineto{\pgfqpoint{3.929971in}{1.844868in}}%
\pgfpathlineto{\pgfqpoint{4.008757in}{1.826566in}}%
\pgfpathlineto{\pgfqpoint{4.095422in}{1.808656in}}%
\pgfpathlineto{\pgfqpoint{4.189965in}{1.791434in}}%
\pgfpathlineto{\pgfqpoint{4.292387in}{1.775141in}}%
\pgfpathlineto{\pgfqpoint{4.402688in}{1.759980in}}%
\pgfpathlineto{\pgfqpoint{4.520867in}{1.746116in}}%
\pgfpathlineto{\pgfqpoint{4.646925in}{1.733695in}}%
\pgfpathlineto{\pgfqpoint{4.780862in}{1.722851in}}%
\pgfpathlineto{\pgfqpoint{4.922677in}{1.713718in}}%
\pgfpathlineto{\pgfqpoint{5.072370in}{1.706438in}}%
\pgfpathlineto{\pgfqpoint{5.159035in}{1.703252in}}%
\pgfpathlineto{\pgfqpoint{5.159035in}{1.703252in}}%
\pgfusepath{stroke}%
\end{pgfscope}%
\begin{pgfscope}%
\pgfpathrectangle{\pgfqpoint{3.347347in}{0.790446in}}{\pgfqpoint{1.897959in}{1.372727in}} %
\pgfusepath{clip}%
\pgfsetbuttcap%
\pgfsetmiterjoin%
\definecolor{currentfill}{rgb}{1.000000,0.000000,0.000000}%
\pgfsetfillcolor{currentfill}%
\pgfsetlinewidth{1.003750pt}%
\definecolor{currentstroke}{rgb}{1.000000,0.000000,0.000000}%
\pgfsetstrokecolor{currentstroke}%
\pgfsetdash{}{0pt}%
\pgfsys@defobject{currentmarker}{\pgfqpoint{-0.041667in}{-0.041667in}}{\pgfqpoint{0.041667in}{0.041667in}}{%
\pgfpathmoveto{\pgfqpoint{-0.041667in}{-0.041667in}}%
\pgfpathlineto{\pgfqpoint{0.041667in}{-0.041667in}}%
\pgfpathlineto{\pgfqpoint{0.041667in}{0.041667in}}%
\pgfpathlineto{\pgfqpoint{-0.041667in}{0.041667in}}%
\pgfpathclose%
\pgfusepath{stroke,fill}%
}%
\begin{pgfscope}%
\pgfsys@transformshift{3.433618in}{2.022449in}%
\pgfsys@useobject{currentmarker}{}%
\end{pgfscope}%
\begin{pgfscope}%
\pgfsys@transformshift{3.780277in}{1.885944in}%
\pgfsys@useobject{currentmarker}{}%
\end{pgfscope}%
\begin{pgfscope}%
\pgfsys@transformshift{4.126936in}{1.802663in}%
\pgfsys@useobject{currentmarker}{}%
\end{pgfscope}%
\begin{pgfscope}%
\pgfsys@transformshift{4.473595in}{1.751387in}%
\pgfsys@useobject{currentmarker}{}%
\end{pgfscope}%
\begin{pgfscope}%
\pgfsys@transformshift{4.820255in}{1.720083in}%
\pgfsys@useobject{currentmarker}{}%
\end{pgfscope}%
\end{pgfscope}%
\begin{pgfscope}%
\pgfpathrectangle{\pgfqpoint{3.347347in}{0.790446in}}{\pgfqpoint{1.897959in}{1.372727in}} %
\pgfusepath{clip}%
\pgfsetrectcap%
\pgfsetroundjoin%
\pgfsetlinewidth{1.505625pt}%
\definecolor{currentstroke}{rgb}{0.000000,0.000000,1.000000}%
\pgfsetstrokecolor{currentstroke}%
\pgfsetdash{}{0pt}%
\pgfpathmoveto{\pgfqpoint{3.433618in}{1.430064in}}%
\pgfpathlineto{\pgfqpoint{3.465132in}{1.414454in}}%
\pgfpathlineto{\pgfqpoint{3.512404in}{1.393935in}}%
\pgfpathlineto{\pgfqpoint{3.567554in}{1.372539in}}%
\pgfpathlineto{\pgfqpoint{3.638462in}{1.347570in}}%
\pgfpathlineto{\pgfqpoint{3.725127in}{1.319669in}}%
\pgfpathlineto{\pgfqpoint{3.827549in}{1.289210in}}%
\pgfpathlineto{\pgfqpoint{3.953607in}{1.254120in}}%
\pgfpathlineto{\pgfqpoint{4.150572in}{1.201950in}}%
\pgfpathlineto{\pgfqpoint{4.434202in}{1.126673in}}%
\pgfpathlineto{\pgfqpoint{4.568139in}{1.088819in}}%
\pgfpathlineto{\pgfqpoint{4.678440in}{1.055432in}}%
\pgfpathlineto{\pgfqpoint{4.772983in}{1.024523in}}%
\pgfpathlineto{\pgfqpoint{4.851769in}{0.996587in}}%
\pgfpathlineto{\pgfqpoint{4.922677in}{0.969233in}}%
\pgfpathlineto{\pgfqpoint{4.985706in}{0.942663in}}%
\pgfpathlineto{\pgfqpoint{5.040856in}{0.917205in}}%
\pgfpathlineto{\pgfqpoint{5.096006in}{0.889128in}}%
\pgfpathlineto{\pgfqpoint{5.143278in}{0.862409in}}%
\pgfpathlineto{\pgfqpoint{5.159035in}{0.852843in}}%
\pgfpathlineto{\pgfqpoint{5.159035in}{0.852843in}}%
\pgfusepath{stroke}%
\end{pgfscope}%
\begin{pgfscope}%
\pgfpathrectangle{\pgfqpoint{3.347347in}{0.790446in}}{\pgfqpoint{1.897959in}{1.372727in}} %
\pgfusepath{clip}%
\pgfsetbuttcap%
\pgfsetroundjoin%
\definecolor{currentfill}{rgb}{0.000000,0.000000,1.000000}%
\pgfsetfillcolor{currentfill}%
\pgfsetlinewidth{1.003750pt}%
\definecolor{currentstroke}{rgb}{0.000000,0.000000,1.000000}%
\pgfsetstrokecolor{currentstroke}%
\pgfsetdash{}{0pt}%
\pgfsys@defobject{currentmarker}{\pgfqpoint{-0.041667in}{-0.041667in}}{\pgfqpoint{0.041667in}{0.041667in}}{%
\pgfpathmoveto{\pgfqpoint{0.000000in}{-0.041667in}}%
\pgfpathcurveto{\pgfqpoint{0.011050in}{-0.041667in}}{\pgfqpoint{0.021649in}{-0.037276in}}{\pgfqpoint{0.029463in}{-0.029463in}}%
\pgfpathcurveto{\pgfqpoint{0.037276in}{-0.021649in}}{\pgfqpoint{0.041667in}{-0.011050in}}{\pgfqpoint{0.041667in}{0.000000in}}%
\pgfpathcurveto{\pgfqpoint{0.041667in}{0.011050in}}{\pgfqpoint{0.037276in}{0.021649in}}{\pgfqpoint{0.029463in}{0.029463in}}%
\pgfpathcurveto{\pgfqpoint{0.021649in}{0.037276in}}{\pgfqpoint{0.011050in}{0.041667in}}{\pgfqpoint{0.000000in}{0.041667in}}%
\pgfpathcurveto{\pgfqpoint{-0.011050in}{0.041667in}}{\pgfqpoint{-0.021649in}{0.037276in}}{\pgfqpoint{-0.029463in}{0.029463in}}%
\pgfpathcurveto{\pgfqpoint{-0.037276in}{0.021649in}}{\pgfqpoint{-0.041667in}{0.011050in}}{\pgfqpoint{-0.041667in}{0.000000in}}%
\pgfpathcurveto{\pgfqpoint{-0.041667in}{-0.011050in}}{\pgfqpoint{-0.037276in}{-0.021649in}}{\pgfqpoint{-0.029463in}{-0.029463in}}%
\pgfpathcurveto{\pgfqpoint{-0.021649in}{-0.037276in}}{\pgfqpoint{-0.011050in}{-0.041667in}}{\pgfqpoint{0.000000in}{-0.041667in}}%
\pgfpathclose%
\pgfusepath{stroke,fill}%
}%
\begin{pgfscope}%
\pgfsys@transformshift{3.433618in}{1.430064in}%
\pgfsys@useobject{currentmarker}{}%
\end{pgfscope}%
\begin{pgfscope}%
\pgfsys@transformshift{3.780277in}{1.302994in}%
\pgfsys@useobject{currentmarker}{}%
\end{pgfscope}%
\begin{pgfscope}%
\pgfsys@transformshift{4.126936in}{1.208134in}%
\pgfsys@useobject{currentmarker}{}%
\end{pgfscope}%
\begin{pgfscope}%
\pgfsys@transformshift{4.473595in}{1.115778in}%
\pgfsys@useobject{currentmarker}{}%
\end{pgfscope}%
\begin{pgfscope}%
\pgfsys@transformshift{4.820255in}{1.008037in}%
\pgfsys@useobject{currentmarker}{}%
\end{pgfscope}%
\end{pgfscope}%
\begin{pgfscope}%
\pgfpathrectangle{\pgfqpoint{3.347347in}{0.790446in}}{\pgfqpoint{1.897959in}{1.372727in}} %
\pgfusepath{clip}%
\pgfsetbuttcap%
\pgfsetroundjoin%
\pgfsetlinewidth{1.505625pt}%
\definecolor{currentstroke}{rgb}{0.000000,0.750000,0.750000}%
\pgfsetstrokecolor{currentstroke}%
\pgfsetdash{{9.600000pt}{2.400000pt}{1.500000pt}{2.400000pt}}{0.000000pt}%
\pgfpathmoveto{\pgfqpoint{3.433618in}{1.890956in}}%
\pgfpathlineto{\pgfqpoint{3.449375in}{1.843858in}}%
\pgfpathlineto{\pgfqpoint{3.465132in}{1.803246in}}%
\pgfpathlineto{\pgfqpoint{3.480890in}{1.767621in}}%
\pgfpathlineto{\pgfqpoint{3.496647in}{1.735952in}}%
\pgfpathlineto{\pgfqpoint{3.520283in}{1.694292in}}%
\pgfpathlineto{\pgfqpoint{3.543918in}{1.658140in}}%
\pgfpathlineto{\pgfqpoint{3.567554in}{1.626302in}}%
\pgfpathlineto{\pgfqpoint{3.591190in}{1.597931in}}%
\pgfpathlineto{\pgfqpoint{3.614826in}{1.572407in}}%
\pgfpathlineto{\pgfqpoint{3.646341in}{1.542007in}}%
\pgfpathlineto{\pgfqpoint{3.677855in}{1.515012in}}%
\pgfpathlineto{\pgfqpoint{3.709369in}{1.490810in}}%
\pgfpathlineto{\pgfqpoint{3.748763in}{1.463791in}}%
\pgfpathlineto{\pgfqpoint{3.788156in}{1.439741in}}%
\pgfpathlineto{\pgfqpoint{3.827549in}{1.418151in}}%
\pgfpathlineto{\pgfqpoint{3.874820in}{1.394948in}}%
\pgfpathlineto{\pgfqpoint{3.922092in}{1.374215in}}%
\pgfpathlineto{\pgfqpoint{3.977242in}{1.352629in}}%
\pgfpathlineto{\pgfqpoint{4.032393in}{1.333395in}}%
\pgfpathlineto{\pgfqpoint{4.095422in}{1.313828in}}%
\pgfpathlineto{\pgfqpoint{4.166329in}{1.294397in}}%
\pgfpathlineto{\pgfqpoint{4.237237in}{1.277265in}}%
\pgfpathlineto{\pgfqpoint{4.316023in}{1.260507in}}%
\pgfpathlineto{\pgfqpoint{4.402688in}{1.244421in}}%
\pgfpathlineto{\pgfqpoint{4.497231in}{1.229258in}}%
\pgfpathlineto{\pgfqpoint{4.599653in}{1.215228in}}%
\pgfpathlineto{\pgfqpoint{4.709954in}{1.202515in}}%
\pgfpathlineto{\pgfqpoint{4.828133in}{1.191283in}}%
\pgfpathlineto{\pgfqpoint{4.954191in}{1.181696in}}%
\pgfpathlineto{\pgfqpoint{5.080249in}{1.174314in}}%
\pgfpathlineto{\pgfqpoint{5.159035in}{1.170731in}}%
\pgfpathlineto{\pgfqpoint{5.159035in}{1.170731in}}%
\pgfusepath{stroke}%
\end{pgfscope}%
\begin{pgfscope}%
\pgfpathrectangle{\pgfqpoint{3.347347in}{0.790446in}}{\pgfqpoint{1.897959in}{1.372727in}} %
\pgfusepath{clip}%
\pgfsetbuttcap%
\pgfsetmiterjoin%
\definecolor{currentfill}{rgb}{0.000000,0.750000,0.750000}%
\pgfsetfillcolor{currentfill}%
\pgfsetlinewidth{1.003750pt}%
\definecolor{currentstroke}{rgb}{0.000000,0.750000,0.750000}%
\pgfsetstrokecolor{currentstroke}%
\pgfsetdash{}{0pt}%
\pgfsys@defobject{currentmarker}{\pgfqpoint{-0.041667in}{-0.041667in}}{\pgfqpoint{0.041667in}{0.041667in}}{%
\pgfpathmoveto{\pgfqpoint{-0.000000in}{-0.041667in}}%
\pgfpathlineto{\pgfqpoint{0.041667in}{0.041667in}}%
\pgfpathlineto{\pgfqpoint{-0.041667in}{0.041667in}}%
\pgfpathclose%
\pgfusepath{stroke,fill}%
}%
\begin{pgfscope}%
\pgfsys@transformshift{3.433618in}{1.890956in}%
\pgfsys@useobject{currentmarker}{}%
\end{pgfscope}%
\begin{pgfscope}%
\pgfsys@transformshift{3.780277in}{1.444340in}%
\pgfsys@useobject{currentmarker}{}%
\end{pgfscope}%
\begin{pgfscope}%
\pgfsys@transformshift{4.126936in}{1.304881in}%
\pgfsys@useobject{currentmarker}{}%
\end{pgfscope}%
\begin{pgfscope}%
\pgfsys@transformshift{4.473595in}{1.232836in}%
\pgfsys@useobject{currentmarker}{}%
\end{pgfscope}%
\begin{pgfscope}%
\pgfsys@transformshift{4.820255in}{1.191961in}%
\pgfsys@useobject{currentmarker}{}%
\end{pgfscope}%
\end{pgfscope}%
\begin{pgfscope}%
\pgfpathrectangle{\pgfqpoint{3.347347in}{0.790446in}}{\pgfqpoint{1.897959in}{1.372727in}} %
\pgfusepath{clip}%
\pgfsetbuttcap%
\pgfsetroundjoin%
\pgfsetlinewidth{1.505625pt}%
\definecolor{currentstroke}{rgb}{0.000000,0.000000,0.000000}%
\pgfsetstrokecolor{currentstroke}%
\pgfsetdash{{1.500000pt}{2.475000pt}}{0.000000pt}%
\pgfpathmoveto{\pgfqpoint{3.433618in}{2.100776in}}%
\pgfpathlineto{\pgfqpoint{3.457254in}{2.073759in}}%
\pgfpathlineto{\pgfqpoint{3.473011in}{2.058916in}}%
\pgfpathlineto{\pgfqpoint{3.496647in}{2.040049in}}%
\pgfpathlineto{\pgfqpoint{3.520283in}{2.023882in}}%
\pgfpathlineto{\pgfqpoint{3.551797in}{2.005056in}}%
\pgfpathlineto{\pgfqpoint{3.591190in}{1.984453in}}%
\pgfpathlineto{\pgfqpoint{3.638462in}{1.962651in}}%
\pgfpathlineto{\pgfqpoint{3.693612in}{1.940103in}}%
\pgfpathlineto{\pgfqpoint{3.748763in}{1.919934in}}%
\pgfpathlineto{\pgfqpoint{3.811791in}{1.899225in}}%
\pgfpathlineto{\pgfqpoint{3.882699in}{1.878395in}}%
\pgfpathlineto{\pgfqpoint{3.961485in}{1.857813in}}%
\pgfpathlineto{\pgfqpoint{4.040271in}{1.839509in}}%
\pgfpathlineto{\pgfqpoint{4.126936in}{1.821613in}}%
\pgfpathlineto{\pgfqpoint{4.221480in}{1.804381in}}%
\pgfpathlineto{\pgfqpoint{4.323902in}{1.788025in}}%
\pgfpathlineto{\pgfqpoint{4.434202in}{1.772726in}}%
\pgfpathlineto{\pgfqpoint{4.552382in}{1.758641in}}%
\pgfpathlineto{\pgfqpoint{4.678440in}{1.745917in}}%
\pgfpathlineto{\pgfqpoint{4.812376in}{1.734694in}}%
\pgfpathlineto{\pgfqpoint{4.954191in}{1.725126in}}%
\pgfpathlineto{\pgfqpoint{5.096006in}{1.717733in}}%
\pgfpathlineto{\pgfqpoint{5.159035in}{1.715106in}}%
\pgfpathlineto{\pgfqpoint{5.159035in}{1.715106in}}%
\pgfusepath{stroke}%
\end{pgfscope}%
\begin{pgfscope}%
\pgfpathrectangle{\pgfqpoint{3.347347in}{0.790446in}}{\pgfqpoint{1.897959in}{1.372727in}} %
\pgfusepath{clip}%
\pgfsetbuttcap%
\pgfsetroundjoin%
\definecolor{currentfill}{rgb}{0.000000,0.000000,0.000000}%
\pgfsetfillcolor{currentfill}%
\pgfsetlinewidth{1.003750pt}%
\definecolor{currentstroke}{rgb}{0.000000,0.000000,0.000000}%
\pgfsetstrokecolor{currentstroke}%
\pgfsetdash{}{0pt}%
\pgfsys@defobject{currentmarker}{\pgfqpoint{-0.041667in}{-0.041667in}}{\pgfqpoint{0.041667in}{0.041667in}}{%
\pgfpathmoveto{\pgfqpoint{-0.041667in}{0.000000in}}%
\pgfpathlineto{\pgfqpoint{0.041667in}{0.000000in}}%
\pgfpathmoveto{\pgfqpoint{0.000000in}{-0.041667in}}%
\pgfpathlineto{\pgfqpoint{0.000000in}{0.041667in}}%
\pgfusepath{stroke,fill}%
}%
\begin{pgfscope}%
\pgfsys@transformshift{3.433618in}{2.100776in}%
\pgfsys@useobject{currentmarker}{}%
\end{pgfscope}%
\begin{pgfscope}%
\pgfsys@transformshift{3.780277in}{1.909296in}%
\pgfsys@useobject{currentmarker}{}%
\end{pgfscope}%
\begin{pgfscope}%
\pgfsys@transformshift{4.126936in}{1.821613in}%
\pgfsys@useobject{currentmarker}{}%
\end{pgfscope}%
\begin{pgfscope}%
\pgfsys@transformshift{4.473595in}{1.767781in}%
\pgfsys@useobject{currentmarker}{}%
\end{pgfscope}%
\begin{pgfscope}%
\pgfsys@transformshift{4.820255in}{1.734103in}%
\pgfsys@useobject{currentmarker}{}%
\end{pgfscope}%
\end{pgfscope}%
\begin{pgfscope}%
\pgfsetrectcap%
\pgfsetmiterjoin%
\pgfsetlinewidth{0.803000pt}%
\definecolor{currentstroke}{rgb}{0.000000,0.000000,0.000000}%
\pgfsetstrokecolor{currentstroke}%
\pgfsetdash{}{0pt}%
\pgfpathmoveto{\pgfqpoint{3.347347in}{0.790446in}}%
\pgfpathlineto{\pgfqpoint{3.347347in}{2.163173in}}%
\pgfusepath{stroke}%
\end{pgfscope}%
\begin{pgfscope}%
\pgfsetrectcap%
\pgfsetmiterjoin%
\pgfsetlinewidth{0.803000pt}%
\definecolor{currentstroke}{rgb}{0.000000,0.000000,0.000000}%
\pgfsetstrokecolor{currentstroke}%
\pgfsetdash{}{0pt}%
\pgfpathmoveto{\pgfqpoint{5.245306in}{0.790446in}}%
\pgfpathlineto{\pgfqpoint{5.245306in}{2.163173in}}%
\pgfusepath{stroke}%
\end{pgfscope}%
\begin{pgfscope}%
\pgfsetrectcap%
\pgfsetmiterjoin%
\pgfsetlinewidth{0.803000pt}%
\definecolor{currentstroke}{rgb}{0.000000,0.000000,0.000000}%
\pgfsetstrokecolor{currentstroke}%
\pgfsetdash{}{0pt}%
\pgfpathmoveto{\pgfqpoint{3.347347in}{0.790446in}}%
\pgfpathlineto{\pgfqpoint{5.245306in}{0.790446in}}%
\pgfusepath{stroke}%
\end{pgfscope}%
\begin{pgfscope}%
\pgfsetrectcap%
\pgfsetmiterjoin%
\pgfsetlinewidth{0.803000pt}%
\definecolor{currentstroke}{rgb}{0.000000,0.000000,0.000000}%
\pgfsetstrokecolor{currentstroke}%
\pgfsetdash{}{0pt}%
\pgfpathmoveto{\pgfqpoint{3.347347in}{2.163173in}}%
\pgfpathlineto{\pgfqpoint{5.245306in}{2.163173in}}%
\pgfusepath{stroke}%
\end{pgfscope}%
\begin{pgfscope}%
\pgfsetbuttcap%
\pgfsetmiterjoin%
\definecolor{currentfill}{rgb}{1.000000,1.000000,1.000000}%
\pgfsetfillcolor{currentfill}%
\pgfsetlinewidth{0.000000pt}%
\definecolor{currentstroke}{rgb}{0.000000,0.000000,0.000000}%
\pgfsetstrokecolor{currentstroke}%
\pgfsetstrokeopacity{0.000000}%
\pgfsetdash{}{0pt}%
\pgfpathmoveto{\pgfqpoint{5.814694in}{0.790446in}}%
\pgfpathlineto{\pgfqpoint{7.712653in}{0.790446in}}%
\pgfpathlineto{\pgfqpoint{7.712653in}{2.163173in}}%
\pgfpathlineto{\pgfqpoint{5.814694in}{2.163173in}}%
\pgfpathclose%
\pgfusepath{fill}%
\end{pgfscope}%
\begin{pgfscope}%
\pgfsetbuttcap%
\pgfsetroundjoin%
\definecolor{currentfill}{rgb}{0.000000,0.000000,0.000000}%
\pgfsetfillcolor{currentfill}%
\pgfsetlinewidth{0.803000pt}%
\definecolor{currentstroke}{rgb}{0.000000,0.000000,0.000000}%
\pgfsetstrokecolor{currentstroke}%
\pgfsetdash{}{0pt}%
\pgfsys@defobject{currentmarker}{\pgfqpoint{0.000000in}{-0.048611in}}{\pgfqpoint{0.000000in}{0.000000in}}{%
\pgfpathmoveto{\pgfqpoint{0.000000in}{0.000000in}}%
\pgfpathlineto{\pgfqpoint{0.000000in}{-0.048611in}}%
\pgfusepath{stroke,fill}%
}%
\begin{pgfscope}%
\pgfsys@transformshift{5.824280in}{0.790446in}%
\pgfsys@useobject{currentmarker}{}%
\end{pgfscope}%
\end{pgfscope}%
\begin{pgfscope}%
\pgftext[x=5.824280in,y=0.693224in,,top]{\rmfamily\fontsize{10.000000}{12.000000}\selectfont \(\displaystyle 0.0\)}%
\end{pgfscope}%
\begin{pgfscope}%
\pgfsetbuttcap%
\pgfsetroundjoin%
\definecolor{currentfill}{rgb}{0.000000,0.000000,0.000000}%
\pgfsetfillcolor{currentfill}%
\pgfsetlinewidth{0.803000pt}%
\definecolor{currentstroke}{rgb}{0.000000,0.000000,0.000000}%
\pgfsetstrokecolor{currentstroke}%
\pgfsetdash{}{0pt}%
\pgfsys@defobject{currentmarker}{\pgfqpoint{0.000000in}{-0.048611in}}{\pgfqpoint{0.000000in}{0.000000in}}{%
\pgfpathmoveto{\pgfqpoint{0.000000in}{0.000000in}}%
\pgfpathlineto{\pgfqpoint{0.000000in}{-0.048611in}}%
\pgfusepath{stroke,fill}%
}%
\begin{pgfscope}%
\pgfsys@transformshift{6.591132in}{0.790446in}%
\pgfsys@useobject{currentmarker}{}%
\end{pgfscope}%
\end{pgfscope}%
\begin{pgfscope}%
\pgftext[x=6.591132in,y=0.693224in,,top]{\rmfamily\fontsize{10.000000}{12.000000}\selectfont \(\displaystyle 0.5\)}%
\end{pgfscope}%
\begin{pgfscope}%
\pgfsetbuttcap%
\pgfsetroundjoin%
\definecolor{currentfill}{rgb}{0.000000,0.000000,0.000000}%
\pgfsetfillcolor{currentfill}%
\pgfsetlinewidth{0.803000pt}%
\definecolor{currentstroke}{rgb}{0.000000,0.000000,0.000000}%
\pgfsetstrokecolor{currentstroke}%
\pgfsetdash{}{0pt}%
\pgfsys@defobject{currentmarker}{\pgfqpoint{0.000000in}{-0.048611in}}{\pgfqpoint{0.000000in}{0.000000in}}{%
\pgfpathmoveto{\pgfqpoint{0.000000in}{0.000000in}}%
\pgfpathlineto{\pgfqpoint{0.000000in}{-0.048611in}}%
\pgfusepath{stroke,fill}%
}%
\begin{pgfscope}%
\pgfsys@transformshift{7.357984in}{0.790446in}%
\pgfsys@useobject{currentmarker}{}%
\end{pgfscope}%
\end{pgfscope}%
\begin{pgfscope}%
\pgftext[x=7.357984in,y=0.693224in,,top]{\rmfamily\fontsize{10.000000}{12.000000}\selectfont \(\displaystyle 1.0\)}%
\end{pgfscope}%
\begin{pgfscope}%
\pgfsetbuttcap%
\pgfsetroundjoin%
\definecolor{currentfill}{rgb}{0.000000,0.000000,0.000000}%
\pgfsetfillcolor{currentfill}%
\pgfsetlinewidth{0.803000pt}%
\definecolor{currentstroke}{rgb}{0.000000,0.000000,0.000000}%
\pgfsetstrokecolor{currentstroke}%
\pgfsetdash{}{0pt}%
\pgfsys@defobject{currentmarker}{\pgfqpoint{-0.048611in}{0.000000in}}{\pgfqpoint{0.000000in}{0.000000in}}{%
\pgfpathmoveto{\pgfqpoint{0.000000in}{0.000000in}}%
\pgfpathlineto{\pgfqpoint{-0.048611in}{0.000000in}}%
\pgfusepath{stroke,fill}%
}%
\begin{pgfscope}%
\pgfsys@transformshift{5.814694in}{0.844602in}%
\pgfsys@useobject{currentmarker}{}%
\end{pgfscope}%
\end{pgfscope}%
\begin{pgfscope}%
\pgftext[x=5.429469in,y=0.791841in,left,base]{\rmfamily\fontsize{10.000000}{12.000000}\selectfont \(\displaystyle 10^{-7}\)}%
\end{pgfscope}%
\begin{pgfscope}%
\pgfsetbuttcap%
\pgfsetroundjoin%
\definecolor{currentfill}{rgb}{0.000000,0.000000,0.000000}%
\pgfsetfillcolor{currentfill}%
\pgfsetlinewidth{0.803000pt}%
\definecolor{currentstroke}{rgb}{0.000000,0.000000,0.000000}%
\pgfsetstrokecolor{currentstroke}%
\pgfsetdash{}{0pt}%
\pgfsys@defobject{currentmarker}{\pgfqpoint{-0.048611in}{0.000000in}}{\pgfqpoint{0.000000in}{0.000000in}}{%
\pgfpathmoveto{\pgfqpoint{0.000000in}{0.000000in}}%
\pgfpathlineto{\pgfqpoint{-0.048611in}{0.000000in}}%
\pgfusepath{stroke,fill}%
}%
\begin{pgfscope}%
\pgfsys@transformshift{5.814694in}{1.352175in}%
\pgfsys@useobject{currentmarker}{}%
\end{pgfscope}%
\end{pgfscope}%
\begin{pgfscope}%
\pgftext[x=5.429469in,y=1.299414in,left,base]{\rmfamily\fontsize{10.000000}{12.000000}\selectfont \(\displaystyle 10^{-6}\)}%
\end{pgfscope}%
\begin{pgfscope}%
\pgfsetbuttcap%
\pgfsetroundjoin%
\definecolor{currentfill}{rgb}{0.000000,0.000000,0.000000}%
\pgfsetfillcolor{currentfill}%
\pgfsetlinewidth{0.803000pt}%
\definecolor{currentstroke}{rgb}{0.000000,0.000000,0.000000}%
\pgfsetstrokecolor{currentstroke}%
\pgfsetdash{}{0pt}%
\pgfsys@defobject{currentmarker}{\pgfqpoint{-0.048611in}{0.000000in}}{\pgfqpoint{0.000000in}{0.000000in}}{%
\pgfpathmoveto{\pgfqpoint{0.000000in}{0.000000in}}%
\pgfpathlineto{\pgfqpoint{-0.048611in}{0.000000in}}%
\pgfusepath{stroke,fill}%
}%
\begin{pgfscope}%
\pgfsys@transformshift{5.814694in}{1.859748in}%
\pgfsys@useobject{currentmarker}{}%
\end{pgfscope}%
\end{pgfscope}%
\begin{pgfscope}%
\pgftext[x=5.429469in,y=1.806987in,left,base]{\rmfamily\fontsize{10.000000}{12.000000}\selectfont \(\displaystyle 10^{-5}\)}%
\end{pgfscope}%
\begin{pgfscope}%
\pgfsetbuttcap%
\pgfsetroundjoin%
\definecolor{currentfill}{rgb}{0.000000,0.000000,0.000000}%
\pgfsetfillcolor{currentfill}%
\pgfsetlinewidth{0.602250pt}%
\definecolor{currentstroke}{rgb}{0.000000,0.000000,0.000000}%
\pgfsetstrokecolor{currentstroke}%
\pgfsetdash{}{0pt}%
\pgfsys@defobject{currentmarker}{\pgfqpoint{-0.027778in}{0.000000in}}{\pgfqpoint{0.000000in}{0.000000in}}{%
\pgfpathmoveto{\pgfqpoint{0.000000in}{0.000000in}}%
\pgfpathlineto{\pgfqpoint{-0.027778in}{0.000000in}}%
\pgfusepath{stroke,fill}%
}%
\begin{pgfscope}%
\pgfsys@transformshift{5.814694in}{0.795413in}%
\pgfsys@useobject{currentmarker}{}%
\end{pgfscope}%
\end{pgfscope}%
\begin{pgfscope}%
\pgfsetbuttcap%
\pgfsetroundjoin%
\definecolor{currentfill}{rgb}{0.000000,0.000000,0.000000}%
\pgfsetfillcolor{currentfill}%
\pgfsetlinewidth{0.602250pt}%
\definecolor{currentstroke}{rgb}{0.000000,0.000000,0.000000}%
\pgfsetstrokecolor{currentstroke}%
\pgfsetdash{}{0pt}%
\pgfsys@defobject{currentmarker}{\pgfqpoint{-0.027778in}{0.000000in}}{\pgfqpoint{0.000000in}{0.000000in}}{%
\pgfpathmoveto{\pgfqpoint{0.000000in}{0.000000in}}%
\pgfpathlineto{\pgfqpoint{-0.027778in}{0.000000in}}%
\pgfusepath{stroke,fill}%
}%
\begin{pgfscope}%
\pgfsys@transformshift{5.814694in}{0.821377in}%
\pgfsys@useobject{currentmarker}{}%
\end{pgfscope}%
\end{pgfscope}%
\begin{pgfscope}%
\pgfsetbuttcap%
\pgfsetroundjoin%
\definecolor{currentfill}{rgb}{0.000000,0.000000,0.000000}%
\pgfsetfillcolor{currentfill}%
\pgfsetlinewidth{0.602250pt}%
\definecolor{currentstroke}{rgb}{0.000000,0.000000,0.000000}%
\pgfsetstrokecolor{currentstroke}%
\pgfsetdash{}{0pt}%
\pgfsys@defobject{currentmarker}{\pgfqpoint{-0.027778in}{0.000000in}}{\pgfqpoint{0.000000in}{0.000000in}}{%
\pgfpathmoveto{\pgfqpoint{0.000000in}{0.000000in}}%
\pgfpathlineto{\pgfqpoint{-0.027778in}{0.000000in}}%
\pgfusepath{stroke,fill}%
}%
\begin{pgfscope}%
\pgfsys@transformshift{5.814694in}{0.997397in}%
\pgfsys@useobject{currentmarker}{}%
\end{pgfscope}%
\end{pgfscope}%
\begin{pgfscope}%
\pgfsetbuttcap%
\pgfsetroundjoin%
\definecolor{currentfill}{rgb}{0.000000,0.000000,0.000000}%
\pgfsetfillcolor{currentfill}%
\pgfsetlinewidth{0.602250pt}%
\definecolor{currentstroke}{rgb}{0.000000,0.000000,0.000000}%
\pgfsetstrokecolor{currentstroke}%
\pgfsetdash{}{0pt}%
\pgfsys@defobject{currentmarker}{\pgfqpoint{-0.027778in}{0.000000in}}{\pgfqpoint{0.000000in}{0.000000in}}{%
\pgfpathmoveto{\pgfqpoint{0.000000in}{0.000000in}}%
\pgfpathlineto{\pgfqpoint{-0.027778in}{0.000000in}}%
\pgfusepath{stroke,fill}%
}%
\begin{pgfscope}%
\pgfsys@transformshift{5.814694in}{1.086776in}%
\pgfsys@useobject{currentmarker}{}%
\end{pgfscope}%
\end{pgfscope}%
\begin{pgfscope}%
\pgfsetbuttcap%
\pgfsetroundjoin%
\definecolor{currentfill}{rgb}{0.000000,0.000000,0.000000}%
\pgfsetfillcolor{currentfill}%
\pgfsetlinewidth{0.602250pt}%
\definecolor{currentstroke}{rgb}{0.000000,0.000000,0.000000}%
\pgfsetstrokecolor{currentstroke}%
\pgfsetdash{}{0pt}%
\pgfsys@defobject{currentmarker}{\pgfqpoint{-0.027778in}{0.000000in}}{\pgfqpoint{0.000000in}{0.000000in}}{%
\pgfpathmoveto{\pgfqpoint{0.000000in}{0.000000in}}%
\pgfpathlineto{\pgfqpoint{-0.027778in}{0.000000in}}%
\pgfusepath{stroke,fill}%
}%
\begin{pgfscope}%
\pgfsys@transformshift{5.814694in}{1.150192in}%
\pgfsys@useobject{currentmarker}{}%
\end{pgfscope}%
\end{pgfscope}%
\begin{pgfscope}%
\pgfsetbuttcap%
\pgfsetroundjoin%
\definecolor{currentfill}{rgb}{0.000000,0.000000,0.000000}%
\pgfsetfillcolor{currentfill}%
\pgfsetlinewidth{0.602250pt}%
\definecolor{currentstroke}{rgb}{0.000000,0.000000,0.000000}%
\pgfsetstrokecolor{currentstroke}%
\pgfsetdash{}{0pt}%
\pgfsys@defobject{currentmarker}{\pgfqpoint{-0.027778in}{0.000000in}}{\pgfqpoint{0.000000in}{0.000000in}}{%
\pgfpathmoveto{\pgfqpoint{0.000000in}{0.000000in}}%
\pgfpathlineto{\pgfqpoint{-0.027778in}{0.000000in}}%
\pgfusepath{stroke,fill}%
}%
\begin{pgfscope}%
\pgfsys@transformshift{5.814694in}{1.199381in}%
\pgfsys@useobject{currentmarker}{}%
\end{pgfscope}%
\end{pgfscope}%
\begin{pgfscope}%
\pgfsetbuttcap%
\pgfsetroundjoin%
\definecolor{currentfill}{rgb}{0.000000,0.000000,0.000000}%
\pgfsetfillcolor{currentfill}%
\pgfsetlinewidth{0.602250pt}%
\definecolor{currentstroke}{rgb}{0.000000,0.000000,0.000000}%
\pgfsetstrokecolor{currentstroke}%
\pgfsetdash{}{0pt}%
\pgfsys@defobject{currentmarker}{\pgfqpoint{-0.027778in}{0.000000in}}{\pgfqpoint{0.000000in}{0.000000in}}{%
\pgfpathmoveto{\pgfqpoint{0.000000in}{0.000000in}}%
\pgfpathlineto{\pgfqpoint{-0.027778in}{0.000000in}}%
\pgfusepath{stroke,fill}%
}%
\begin{pgfscope}%
\pgfsys@transformshift{5.814694in}{1.239571in}%
\pgfsys@useobject{currentmarker}{}%
\end{pgfscope}%
\end{pgfscope}%
\begin{pgfscope}%
\pgfsetbuttcap%
\pgfsetroundjoin%
\definecolor{currentfill}{rgb}{0.000000,0.000000,0.000000}%
\pgfsetfillcolor{currentfill}%
\pgfsetlinewidth{0.602250pt}%
\definecolor{currentstroke}{rgb}{0.000000,0.000000,0.000000}%
\pgfsetstrokecolor{currentstroke}%
\pgfsetdash{}{0pt}%
\pgfsys@defobject{currentmarker}{\pgfqpoint{-0.027778in}{0.000000in}}{\pgfqpoint{0.000000in}{0.000000in}}{%
\pgfpathmoveto{\pgfqpoint{0.000000in}{0.000000in}}%
\pgfpathlineto{\pgfqpoint{-0.027778in}{0.000000in}}%
\pgfusepath{stroke,fill}%
}%
\begin{pgfscope}%
\pgfsys@transformshift{5.814694in}{1.273551in}%
\pgfsys@useobject{currentmarker}{}%
\end{pgfscope}%
\end{pgfscope}%
\begin{pgfscope}%
\pgfsetbuttcap%
\pgfsetroundjoin%
\definecolor{currentfill}{rgb}{0.000000,0.000000,0.000000}%
\pgfsetfillcolor{currentfill}%
\pgfsetlinewidth{0.602250pt}%
\definecolor{currentstroke}{rgb}{0.000000,0.000000,0.000000}%
\pgfsetstrokecolor{currentstroke}%
\pgfsetdash{}{0pt}%
\pgfsys@defobject{currentmarker}{\pgfqpoint{-0.027778in}{0.000000in}}{\pgfqpoint{0.000000in}{0.000000in}}{%
\pgfpathmoveto{\pgfqpoint{0.000000in}{0.000000in}}%
\pgfpathlineto{\pgfqpoint{-0.027778in}{0.000000in}}%
\pgfusepath{stroke,fill}%
}%
\begin{pgfscope}%
\pgfsys@transformshift{5.814694in}{1.302986in}%
\pgfsys@useobject{currentmarker}{}%
\end{pgfscope}%
\end{pgfscope}%
\begin{pgfscope}%
\pgfsetbuttcap%
\pgfsetroundjoin%
\definecolor{currentfill}{rgb}{0.000000,0.000000,0.000000}%
\pgfsetfillcolor{currentfill}%
\pgfsetlinewidth{0.602250pt}%
\definecolor{currentstroke}{rgb}{0.000000,0.000000,0.000000}%
\pgfsetstrokecolor{currentstroke}%
\pgfsetdash{}{0pt}%
\pgfsys@defobject{currentmarker}{\pgfqpoint{-0.027778in}{0.000000in}}{\pgfqpoint{0.000000in}{0.000000in}}{%
\pgfpathmoveto{\pgfqpoint{0.000000in}{0.000000in}}%
\pgfpathlineto{\pgfqpoint{-0.027778in}{0.000000in}}%
\pgfusepath{stroke,fill}%
}%
\begin{pgfscope}%
\pgfsys@transformshift{5.814694in}{1.328950in}%
\pgfsys@useobject{currentmarker}{}%
\end{pgfscope}%
\end{pgfscope}%
\begin{pgfscope}%
\pgfsetbuttcap%
\pgfsetroundjoin%
\definecolor{currentfill}{rgb}{0.000000,0.000000,0.000000}%
\pgfsetfillcolor{currentfill}%
\pgfsetlinewidth{0.602250pt}%
\definecolor{currentstroke}{rgb}{0.000000,0.000000,0.000000}%
\pgfsetstrokecolor{currentstroke}%
\pgfsetdash{}{0pt}%
\pgfsys@defobject{currentmarker}{\pgfqpoint{-0.027778in}{0.000000in}}{\pgfqpoint{0.000000in}{0.000000in}}{%
\pgfpathmoveto{\pgfqpoint{0.000000in}{0.000000in}}%
\pgfpathlineto{\pgfqpoint{-0.027778in}{0.000000in}}%
\pgfusepath{stroke,fill}%
}%
\begin{pgfscope}%
\pgfsys@transformshift{5.814694in}{1.504970in}%
\pgfsys@useobject{currentmarker}{}%
\end{pgfscope}%
\end{pgfscope}%
\begin{pgfscope}%
\pgfsetbuttcap%
\pgfsetroundjoin%
\definecolor{currentfill}{rgb}{0.000000,0.000000,0.000000}%
\pgfsetfillcolor{currentfill}%
\pgfsetlinewidth{0.602250pt}%
\definecolor{currentstroke}{rgb}{0.000000,0.000000,0.000000}%
\pgfsetstrokecolor{currentstroke}%
\pgfsetdash{}{0pt}%
\pgfsys@defobject{currentmarker}{\pgfqpoint{-0.027778in}{0.000000in}}{\pgfqpoint{0.000000in}{0.000000in}}{%
\pgfpathmoveto{\pgfqpoint{0.000000in}{0.000000in}}%
\pgfpathlineto{\pgfqpoint{-0.027778in}{0.000000in}}%
\pgfusepath{stroke,fill}%
}%
\begin{pgfscope}%
\pgfsys@transformshift{5.814694in}{1.594349in}%
\pgfsys@useobject{currentmarker}{}%
\end{pgfscope}%
\end{pgfscope}%
\begin{pgfscope}%
\pgfsetbuttcap%
\pgfsetroundjoin%
\definecolor{currentfill}{rgb}{0.000000,0.000000,0.000000}%
\pgfsetfillcolor{currentfill}%
\pgfsetlinewidth{0.602250pt}%
\definecolor{currentstroke}{rgb}{0.000000,0.000000,0.000000}%
\pgfsetstrokecolor{currentstroke}%
\pgfsetdash{}{0pt}%
\pgfsys@defobject{currentmarker}{\pgfqpoint{-0.027778in}{0.000000in}}{\pgfqpoint{0.000000in}{0.000000in}}{%
\pgfpathmoveto{\pgfqpoint{0.000000in}{0.000000in}}%
\pgfpathlineto{\pgfqpoint{-0.027778in}{0.000000in}}%
\pgfusepath{stroke,fill}%
}%
\begin{pgfscope}%
\pgfsys@transformshift{5.814694in}{1.657765in}%
\pgfsys@useobject{currentmarker}{}%
\end{pgfscope}%
\end{pgfscope}%
\begin{pgfscope}%
\pgfsetbuttcap%
\pgfsetroundjoin%
\definecolor{currentfill}{rgb}{0.000000,0.000000,0.000000}%
\pgfsetfillcolor{currentfill}%
\pgfsetlinewidth{0.602250pt}%
\definecolor{currentstroke}{rgb}{0.000000,0.000000,0.000000}%
\pgfsetstrokecolor{currentstroke}%
\pgfsetdash{}{0pt}%
\pgfsys@defobject{currentmarker}{\pgfqpoint{-0.027778in}{0.000000in}}{\pgfqpoint{0.000000in}{0.000000in}}{%
\pgfpathmoveto{\pgfqpoint{0.000000in}{0.000000in}}%
\pgfpathlineto{\pgfqpoint{-0.027778in}{0.000000in}}%
\pgfusepath{stroke,fill}%
}%
\begin{pgfscope}%
\pgfsys@transformshift{5.814694in}{1.706954in}%
\pgfsys@useobject{currentmarker}{}%
\end{pgfscope}%
\end{pgfscope}%
\begin{pgfscope}%
\pgfsetbuttcap%
\pgfsetroundjoin%
\definecolor{currentfill}{rgb}{0.000000,0.000000,0.000000}%
\pgfsetfillcolor{currentfill}%
\pgfsetlinewidth{0.602250pt}%
\definecolor{currentstroke}{rgb}{0.000000,0.000000,0.000000}%
\pgfsetstrokecolor{currentstroke}%
\pgfsetdash{}{0pt}%
\pgfsys@defobject{currentmarker}{\pgfqpoint{-0.027778in}{0.000000in}}{\pgfqpoint{0.000000in}{0.000000in}}{%
\pgfpathmoveto{\pgfqpoint{0.000000in}{0.000000in}}%
\pgfpathlineto{\pgfqpoint{-0.027778in}{0.000000in}}%
\pgfusepath{stroke,fill}%
}%
\begin{pgfscope}%
\pgfsys@transformshift{5.814694in}{1.747144in}%
\pgfsys@useobject{currentmarker}{}%
\end{pgfscope}%
\end{pgfscope}%
\begin{pgfscope}%
\pgfsetbuttcap%
\pgfsetroundjoin%
\definecolor{currentfill}{rgb}{0.000000,0.000000,0.000000}%
\pgfsetfillcolor{currentfill}%
\pgfsetlinewidth{0.602250pt}%
\definecolor{currentstroke}{rgb}{0.000000,0.000000,0.000000}%
\pgfsetstrokecolor{currentstroke}%
\pgfsetdash{}{0pt}%
\pgfsys@defobject{currentmarker}{\pgfqpoint{-0.027778in}{0.000000in}}{\pgfqpoint{0.000000in}{0.000000in}}{%
\pgfpathmoveto{\pgfqpoint{0.000000in}{0.000000in}}%
\pgfpathlineto{\pgfqpoint{-0.027778in}{0.000000in}}%
\pgfusepath{stroke,fill}%
}%
\begin{pgfscope}%
\pgfsys@transformshift{5.814694in}{1.781124in}%
\pgfsys@useobject{currentmarker}{}%
\end{pgfscope}%
\end{pgfscope}%
\begin{pgfscope}%
\pgfsetbuttcap%
\pgfsetroundjoin%
\definecolor{currentfill}{rgb}{0.000000,0.000000,0.000000}%
\pgfsetfillcolor{currentfill}%
\pgfsetlinewidth{0.602250pt}%
\definecolor{currentstroke}{rgb}{0.000000,0.000000,0.000000}%
\pgfsetstrokecolor{currentstroke}%
\pgfsetdash{}{0pt}%
\pgfsys@defobject{currentmarker}{\pgfqpoint{-0.027778in}{0.000000in}}{\pgfqpoint{0.000000in}{0.000000in}}{%
\pgfpathmoveto{\pgfqpoint{0.000000in}{0.000000in}}%
\pgfpathlineto{\pgfqpoint{-0.027778in}{0.000000in}}%
\pgfusepath{stroke,fill}%
}%
\begin{pgfscope}%
\pgfsys@transformshift{5.814694in}{1.810559in}%
\pgfsys@useobject{currentmarker}{}%
\end{pgfscope}%
\end{pgfscope}%
\begin{pgfscope}%
\pgfsetbuttcap%
\pgfsetroundjoin%
\definecolor{currentfill}{rgb}{0.000000,0.000000,0.000000}%
\pgfsetfillcolor{currentfill}%
\pgfsetlinewidth{0.602250pt}%
\definecolor{currentstroke}{rgb}{0.000000,0.000000,0.000000}%
\pgfsetstrokecolor{currentstroke}%
\pgfsetdash{}{0pt}%
\pgfsys@defobject{currentmarker}{\pgfqpoint{-0.027778in}{0.000000in}}{\pgfqpoint{0.000000in}{0.000000in}}{%
\pgfpathmoveto{\pgfqpoint{0.000000in}{0.000000in}}%
\pgfpathlineto{\pgfqpoint{-0.027778in}{0.000000in}}%
\pgfusepath{stroke,fill}%
}%
\begin{pgfscope}%
\pgfsys@transformshift{5.814694in}{1.836523in}%
\pgfsys@useobject{currentmarker}{}%
\end{pgfscope}%
\end{pgfscope}%
\begin{pgfscope}%
\pgfsetbuttcap%
\pgfsetroundjoin%
\definecolor{currentfill}{rgb}{0.000000,0.000000,0.000000}%
\pgfsetfillcolor{currentfill}%
\pgfsetlinewidth{0.602250pt}%
\definecolor{currentstroke}{rgb}{0.000000,0.000000,0.000000}%
\pgfsetstrokecolor{currentstroke}%
\pgfsetdash{}{0pt}%
\pgfsys@defobject{currentmarker}{\pgfqpoint{-0.027778in}{0.000000in}}{\pgfqpoint{0.000000in}{0.000000in}}{%
\pgfpathmoveto{\pgfqpoint{0.000000in}{0.000000in}}%
\pgfpathlineto{\pgfqpoint{-0.027778in}{0.000000in}}%
\pgfusepath{stroke,fill}%
}%
\begin{pgfscope}%
\pgfsys@transformshift{5.814694in}{2.012543in}%
\pgfsys@useobject{currentmarker}{}%
\end{pgfscope}%
\end{pgfscope}%
\begin{pgfscope}%
\pgfsetbuttcap%
\pgfsetroundjoin%
\definecolor{currentfill}{rgb}{0.000000,0.000000,0.000000}%
\pgfsetfillcolor{currentfill}%
\pgfsetlinewidth{0.602250pt}%
\definecolor{currentstroke}{rgb}{0.000000,0.000000,0.000000}%
\pgfsetstrokecolor{currentstroke}%
\pgfsetdash{}{0pt}%
\pgfsys@defobject{currentmarker}{\pgfqpoint{-0.027778in}{0.000000in}}{\pgfqpoint{0.000000in}{0.000000in}}{%
\pgfpathmoveto{\pgfqpoint{0.000000in}{0.000000in}}%
\pgfpathlineto{\pgfqpoint{-0.027778in}{0.000000in}}%
\pgfusepath{stroke,fill}%
}%
\begin{pgfscope}%
\pgfsys@transformshift{5.814694in}{2.101922in}%
\pgfsys@useobject{currentmarker}{}%
\end{pgfscope}%
\end{pgfscope}%
\begin{pgfscope}%
\pgfsetbuttcap%
\pgfsetroundjoin%
\definecolor{currentfill}{rgb}{0.000000,0.000000,0.000000}%
\pgfsetfillcolor{currentfill}%
\pgfsetlinewidth{0.602250pt}%
\definecolor{currentstroke}{rgb}{0.000000,0.000000,0.000000}%
\pgfsetstrokecolor{currentstroke}%
\pgfsetdash{}{0pt}%
\pgfsys@defobject{currentmarker}{\pgfqpoint{-0.027778in}{0.000000in}}{\pgfqpoint{0.000000in}{0.000000in}}{%
\pgfpathmoveto{\pgfqpoint{0.000000in}{0.000000in}}%
\pgfpathlineto{\pgfqpoint{-0.027778in}{0.000000in}}%
\pgfusepath{stroke,fill}%
}%
\begin{pgfscope}%
\pgfsys@transformshift{5.814694in}{2.165338in}%
\pgfsys@useobject{currentmarker}{}%
\end{pgfscope}%
\end{pgfscope}%
\begin{pgfscope}%
\pgfpathrectangle{\pgfqpoint{5.814694in}{0.790446in}}{\pgfqpoint{1.897959in}{1.372727in}} %
\pgfusepath{clip}%
\pgfsetbuttcap%
\pgfsetroundjoin%
\pgfsetlinewidth{1.505625pt}%
\definecolor{currentstroke}{rgb}{1.000000,0.000000,0.000000}%
\pgfsetstrokecolor{currentstroke}%
\pgfsetdash{{5.550000pt}{2.400000pt}}{0.000000pt}%
\pgfpathmoveto{\pgfqpoint{5.900965in}{1.923560in}}%
\pgfpathlineto{\pgfqpoint{6.118240in}{1.842845in}}%
\pgfpathlineto{\pgfqpoint{6.246048in}{1.797788in}}%
\pgfpathlineto{\pgfqpoint{6.361076in}{1.759623in}}%
\pgfpathlineto{\pgfqpoint{6.476104in}{1.723905in}}%
\pgfpathlineto{\pgfqpoint{6.591132in}{1.690602in}}%
\pgfpathlineto{\pgfqpoint{6.706160in}{1.659579in}}%
\pgfpathlineto{\pgfqpoint{6.833968in}{1.627568in}}%
\pgfpathlineto{\pgfqpoint{6.961777in}{1.597895in}}%
\pgfpathlineto{\pgfqpoint{7.102367in}{1.567661in}}%
\pgfpathlineto{\pgfqpoint{7.242956in}{1.539660in}}%
\pgfpathlineto{\pgfqpoint{7.396327in}{1.511356in}}%
\pgfpathlineto{\pgfqpoint{7.562478in}{1.483004in}}%
\pgfpathlineto{\pgfqpoint{7.626382in}{1.472675in}}%
\pgfpathlineto{\pgfqpoint{7.626382in}{1.472675in}}%
\pgfusepath{stroke}%
\end{pgfscope}%
\begin{pgfscope}%
\pgfpathrectangle{\pgfqpoint{5.814694in}{0.790446in}}{\pgfqpoint{1.897959in}{1.372727in}} %
\pgfusepath{clip}%
\pgfsetbuttcap%
\pgfsetmiterjoin%
\definecolor{currentfill}{rgb}{1.000000,0.000000,0.000000}%
\pgfsetfillcolor{currentfill}%
\pgfsetlinewidth{1.003750pt}%
\definecolor{currentstroke}{rgb}{1.000000,0.000000,0.000000}%
\pgfsetstrokecolor{currentstroke}%
\pgfsetdash{}{0pt}%
\pgfsys@defobject{currentmarker}{\pgfqpoint{-0.041667in}{-0.041667in}}{\pgfqpoint{0.041667in}{0.041667in}}{%
\pgfpathmoveto{\pgfqpoint{-0.041667in}{-0.041667in}}%
\pgfpathlineto{\pgfqpoint{0.041667in}{-0.041667in}}%
\pgfpathlineto{\pgfqpoint{0.041667in}{0.041667in}}%
\pgfpathlineto{\pgfqpoint{-0.041667in}{0.041667in}}%
\pgfpathclose%
\pgfusepath{stroke,fill}%
}%
\begin{pgfscope}%
\pgfsys@transformshift{5.900965in}{1.923560in}%
\pgfsys@useobject{currentmarker}{}%
\end{pgfscope}%
\begin{pgfscope}%
\pgfsys@transformshift{6.246048in}{1.797788in}%
\pgfsys@useobject{currentmarker}{}%
\end{pgfscope}%
\begin{pgfscope}%
\pgfsys@transformshift{6.591132in}{1.690602in}%
\pgfsys@useobject{currentmarker}{}%
\end{pgfscope}%
\begin{pgfscope}%
\pgfsys@transformshift{6.936215in}{1.603654in}%
\pgfsys@useobject{currentmarker}{}%
\end{pgfscope}%
\begin{pgfscope}%
\pgfsys@transformshift{7.281299in}{1.532376in}%
\pgfsys@useobject{currentmarker}{}%
\end{pgfscope}%
\begin{pgfscope}%
\pgfsys@transformshift{7.626382in}{1.472675in}%
\pgfsys@useobject{currentmarker}{}%
\end{pgfscope}%
\end{pgfscope}%
\begin{pgfscope}%
\pgfpathrectangle{\pgfqpoint{5.814694in}{0.790446in}}{\pgfqpoint{1.897959in}{1.372727in}} %
\pgfusepath{clip}%
\pgfsetrectcap%
\pgfsetroundjoin%
\pgfsetlinewidth{1.505625pt}%
\definecolor{currentstroke}{rgb}{0.000000,0.000000,1.000000}%
\pgfsetstrokecolor{currentstroke}%
\pgfsetdash{}{0pt}%
\pgfpathmoveto{\pgfqpoint{5.900965in}{1.394954in}}%
\pgfpathlineto{\pgfqpoint{5.939307in}{1.383451in}}%
\pgfpathlineto{\pgfqpoint{5.990431in}{1.371355in}}%
\pgfpathlineto{\pgfqpoint{6.054335in}{1.358792in}}%
\pgfpathlineto{\pgfqpoint{6.143801in}{1.343871in}}%
\pgfpathlineto{\pgfqpoint{6.246048in}{1.329266in}}%
\pgfpathlineto{\pgfqpoint{6.361076in}{1.315050in}}%
\pgfpathlineto{\pgfqpoint{6.501666in}{1.300052in}}%
\pgfpathlineto{\pgfqpoint{6.655036in}{1.285954in}}%
\pgfpathlineto{\pgfqpoint{6.833968in}{1.271767in}}%
\pgfpathlineto{\pgfqpoint{7.038462in}{1.257796in}}%
\pgfpathlineto{\pgfqpoint{7.281299in}{1.243515in}}%
\pgfpathlineto{\pgfqpoint{7.562478in}{1.229281in}}%
\pgfpathlineto{\pgfqpoint{7.626382in}{1.226319in}}%
\pgfpathlineto{\pgfqpoint{7.626382in}{1.226319in}}%
\pgfusepath{stroke}%
\end{pgfscope}%
\begin{pgfscope}%
\pgfpathrectangle{\pgfqpoint{5.814694in}{0.790446in}}{\pgfqpoint{1.897959in}{1.372727in}} %
\pgfusepath{clip}%
\pgfsetbuttcap%
\pgfsetroundjoin%
\definecolor{currentfill}{rgb}{0.000000,0.000000,1.000000}%
\pgfsetfillcolor{currentfill}%
\pgfsetlinewidth{1.003750pt}%
\definecolor{currentstroke}{rgb}{0.000000,0.000000,1.000000}%
\pgfsetstrokecolor{currentstroke}%
\pgfsetdash{}{0pt}%
\pgfsys@defobject{currentmarker}{\pgfqpoint{-0.041667in}{-0.041667in}}{\pgfqpoint{0.041667in}{0.041667in}}{%
\pgfpathmoveto{\pgfqpoint{0.000000in}{-0.041667in}}%
\pgfpathcurveto{\pgfqpoint{0.011050in}{-0.041667in}}{\pgfqpoint{0.021649in}{-0.037276in}}{\pgfqpoint{0.029463in}{-0.029463in}}%
\pgfpathcurveto{\pgfqpoint{0.037276in}{-0.021649in}}{\pgfqpoint{0.041667in}{-0.011050in}}{\pgfqpoint{0.041667in}{0.000000in}}%
\pgfpathcurveto{\pgfqpoint{0.041667in}{0.011050in}}{\pgfqpoint{0.037276in}{0.021649in}}{\pgfqpoint{0.029463in}{0.029463in}}%
\pgfpathcurveto{\pgfqpoint{0.021649in}{0.037276in}}{\pgfqpoint{0.011050in}{0.041667in}}{\pgfqpoint{0.000000in}{0.041667in}}%
\pgfpathcurveto{\pgfqpoint{-0.011050in}{0.041667in}}{\pgfqpoint{-0.021649in}{0.037276in}}{\pgfqpoint{-0.029463in}{0.029463in}}%
\pgfpathcurveto{\pgfqpoint{-0.037276in}{0.021649in}}{\pgfqpoint{-0.041667in}{0.011050in}}{\pgfqpoint{-0.041667in}{0.000000in}}%
\pgfpathcurveto{\pgfqpoint{-0.041667in}{-0.011050in}}{\pgfqpoint{-0.037276in}{-0.021649in}}{\pgfqpoint{-0.029463in}{-0.029463in}}%
\pgfpathcurveto{\pgfqpoint{-0.021649in}{-0.037276in}}{\pgfqpoint{-0.011050in}{-0.041667in}}{\pgfqpoint{0.000000in}{-0.041667in}}%
\pgfpathclose%
\pgfusepath{stroke,fill}%
}%
\begin{pgfscope}%
\pgfsys@transformshift{5.900965in}{1.394954in}%
\pgfsys@useobject{currentmarker}{}%
\end{pgfscope}%
\begin{pgfscope}%
\pgfsys@transformshift{6.246048in}{1.329266in}%
\pgfsys@useobject{currentmarker}{}%
\end{pgfscope}%
\begin{pgfscope}%
\pgfsys@transformshift{6.591132in}{1.291580in}%
\pgfsys@useobject{currentmarker}{}%
\end{pgfscope}%
\begin{pgfscope}%
\pgfsys@transformshift{6.936215in}{1.264523in}%
\pgfsys@useobject{currentmarker}{}%
\end{pgfscope}%
\begin{pgfscope}%
\pgfsys@transformshift{7.281299in}{1.243515in}%
\pgfsys@useobject{currentmarker}{}%
\end{pgfscope}%
\begin{pgfscope}%
\pgfsys@transformshift{7.626382in}{1.226319in}%
\pgfsys@useobject{currentmarker}{}%
\end{pgfscope}%
\end{pgfscope}%
\begin{pgfscope}%
\pgfpathrectangle{\pgfqpoint{5.814694in}{0.790446in}}{\pgfqpoint{1.897959in}{1.372727in}} %
\pgfusepath{clip}%
\pgfsetbuttcap%
\pgfsetroundjoin%
\pgfsetlinewidth{1.505625pt}%
\definecolor{currentstroke}{rgb}{0.000000,0.750000,0.750000}%
\pgfsetstrokecolor{currentstroke}%
\pgfsetdash{{9.600000pt}{2.400000pt}{1.500000pt}{2.400000pt}}{0.000000pt}%
\pgfpathmoveto{\pgfqpoint{5.900965in}{1.957717in}}%
\pgfpathlineto{\pgfqpoint{5.913746in}{1.909026in}}%
\pgfpathlineto{\pgfqpoint{5.926526in}{1.865714in}}%
\pgfpathlineto{\pgfqpoint{5.952088in}{1.791269in}}%
\pgfpathlineto{\pgfqpoint{5.977650in}{1.728805in}}%
\pgfpathlineto{\pgfqpoint{6.003212in}{1.675049in}}%
\pgfpathlineto{\pgfqpoint{6.028773in}{1.627911in}}%
\pgfpathlineto{\pgfqpoint{6.054335in}{1.585973in}}%
\pgfpathlineto{\pgfqpoint{6.079897in}{1.548229in}}%
\pgfpathlineto{\pgfqpoint{6.105459in}{1.513939in}}%
\pgfpathlineto{\pgfqpoint{6.131020in}{1.482541in}}%
\pgfpathlineto{\pgfqpoint{6.169363in}{1.439942in}}%
\pgfpathlineto{\pgfqpoint{6.207706in}{1.401781in}}%
\pgfpathlineto{\pgfqpoint{6.246048in}{1.367249in}}%
\pgfpathlineto{\pgfqpoint{6.284391in}{1.335737in}}%
\pgfpathlineto{\pgfqpoint{6.322733in}{1.306777in}}%
\pgfpathlineto{\pgfqpoint{6.373857in}{1.271506in}}%
\pgfpathlineto{\pgfqpoint{6.424980in}{1.239443in}}%
\pgfpathlineto{\pgfqpoint{6.476104in}{1.210073in}}%
\pgfpathlineto{\pgfqpoint{6.527227in}{1.182991in}}%
\pgfpathlineto{\pgfqpoint{6.591132in}{1.151875in}}%
\pgfpathlineto{\pgfqpoint{6.655036in}{1.123346in}}%
\pgfpathlineto{\pgfqpoint{6.718940in}{1.097018in}}%
\pgfpathlineto{\pgfqpoint{6.795626in}{1.067903in}}%
\pgfpathlineto{\pgfqpoint{6.872311in}{1.041101in}}%
\pgfpathlineto{\pgfqpoint{6.961777in}{1.012316in}}%
\pgfpathlineto{\pgfqpoint{7.051243in}{0.985815in}}%
\pgfpathlineto{\pgfqpoint{7.153490in}{0.957907in}}%
\pgfpathlineto{\pgfqpoint{7.268518in}{0.929091in}}%
\pgfpathlineto{\pgfqpoint{7.383546in}{0.902595in}}%
\pgfpathlineto{\pgfqpoint{7.511354in}{0.875467in}}%
\pgfpathlineto{\pgfqpoint{7.626382in}{0.852843in}}%
\pgfpathlineto{\pgfqpoint{7.626382in}{0.852843in}}%
\pgfusepath{stroke}%
\end{pgfscope}%
\begin{pgfscope}%
\pgfpathrectangle{\pgfqpoint{5.814694in}{0.790446in}}{\pgfqpoint{1.897959in}{1.372727in}} %
\pgfusepath{clip}%
\pgfsetbuttcap%
\pgfsetmiterjoin%
\definecolor{currentfill}{rgb}{0.000000,0.750000,0.750000}%
\pgfsetfillcolor{currentfill}%
\pgfsetlinewidth{1.003750pt}%
\definecolor{currentstroke}{rgb}{0.000000,0.750000,0.750000}%
\pgfsetstrokecolor{currentstroke}%
\pgfsetdash{}{0pt}%
\pgfsys@defobject{currentmarker}{\pgfqpoint{-0.041667in}{-0.041667in}}{\pgfqpoint{0.041667in}{0.041667in}}{%
\pgfpathmoveto{\pgfqpoint{-0.000000in}{-0.041667in}}%
\pgfpathlineto{\pgfqpoint{0.041667in}{0.041667in}}%
\pgfpathlineto{\pgfqpoint{-0.041667in}{0.041667in}}%
\pgfpathclose%
\pgfusepath{stroke,fill}%
}%
\begin{pgfscope}%
\pgfsys@transformshift{5.900965in}{1.957717in}%
\pgfsys@useobject{currentmarker}{}%
\end{pgfscope}%
\begin{pgfscope}%
\pgfsys@transformshift{6.246048in}{1.367249in}%
\pgfsys@useobject{currentmarker}{}%
\end{pgfscope}%
\begin{pgfscope}%
\pgfsys@transformshift{6.591132in}{1.151875in}%
\pgfsys@useobject{currentmarker}{}%
\end{pgfscope}%
\begin{pgfscope}%
\pgfsys@transformshift{6.936215in}{1.020291in}%
\pgfsys@useobject{currentmarker}{}%
\end{pgfscope}%
\begin{pgfscope}%
\pgfsys@transformshift{7.281299in}{0.926039in}%
\pgfsys@useobject{currentmarker}{}%
\end{pgfscope}%
\begin{pgfscope}%
\pgfsys@transformshift{7.626382in}{0.852843in}%
\pgfsys@useobject{currentmarker}{}%
\end{pgfscope}%
\end{pgfscope}%
\begin{pgfscope}%
\pgfpathrectangle{\pgfqpoint{5.814694in}{0.790446in}}{\pgfqpoint{1.897959in}{1.372727in}} %
\pgfusepath{clip}%
\pgfsetbuttcap%
\pgfsetroundjoin%
\pgfsetlinewidth{1.505625pt}%
\definecolor{currentstroke}{rgb}{0.000000,0.000000,0.000000}%
\pgfsetstrokecolor{currentstroke}%
\pgfsetdash{{1.500000pt}{2.475000pt}}{0.000000pt}%
\pgfpathmoveto{\pgfqpoint{5.900965in}{2.100776in}}%
\pgfpathlineto{\pgfqpoint{5.926526in}{2.055718in}}%
\pgfpathlineto{\pgfqpoint{5.939307in}{2.036912in}}%
\pgfpathlineto{\pgfqpoint{5.952088in}{2.020698in}}%
\pgfpathlineto{\pgfqpoint{5.977650in}{1.993840in}}%
\pgfpathlineto{\pgfqpoint{6.003212in}{1.972049in}}%
\pgfpathlineto{\pgfqpoint{6.028773in}{1.953587in}}%
\pgfpathlineto{\pgfqpoint{6.054335in}{1.937414in}}%
\pgfpathlineto{\pgfqpoint{6.092678in}{1.916092in}}%
\pgfpathlineto{\pgfqpoint{6.131020in}{1.897190in}}%
\pgfpathlineto{\pgfqpoint{6.182144in}{1.874519in}}%
\pgfpathlineto{\pgfqpoint{6.246048in}{1.848977in}}%
\pgfpathlineto{\pgfqpoint{6.322733in}{1.821205in}}%
\pgfpathlineto{\pgfqpoint{6.399419in}{1.795782in}}%
\pgfpathlineto{\pgfqpoint{6.488885in}{1.768487in}}%
\pgfpathlineto{\pgfqpoint{6.591132in}{1.739885in}}%
\pgfpathlineto{\pgfqpoint{6.693379in}{1.713639in}}%
\pgfpathlineto{\pgfqpoint{6.808407in}{1.686542in}}%
\pgfpathlineto{\pgfqpoint{6.923434in}{1.661682in}}%
\pgfpathlineto{\pgfqpoint{7.051243in}{1.636344in}}%
\pgfpathlineto{\pgfqpoint{7.191833in}{1.610884in}}%
\pgfpathlineto{\pgfqpoint{7.345203in}{1.585596in}}%
\pgfpathlineto{\pgfqpoint{7.498573in}{1.562542in}}%
\pgfpathlineto{\pgfqpoint{7.626382in}{1.544818in}}%
\pgfpathlineto{\pgfqpoint{7.626382in}{1.544818in}}%
\pgfusepath{stroke}%
\end{pgfscope}%
\begin{pgfscope}%
\pgfpathrectangle{\pgfqpoint{5.814694in}{0.790446in}}{\pgfqpoint{1.897959in}{1.372727in}} %
\pgfusepath{clip}%
\pgfsetbuttcap%
\pgfsetroundjoin%
\definecolor{currentfill}{rgb}{0.000000,0.000000,0.000000}%
\pgfsetfillcolor{currentfill}%
\pgfsetlinewidth{1.003750pt}%
\definecolor{currentstroke}{rgb}{0.000000,0.000000,0.000000}%
\pgfsetstrokecolor{currentstroke}%
\pgfsetdash{}{0pt}%
\pgfsys@defobject{currentmarker}{\pgfqpoint{-0.041667in}{-0.041667in}}{\pgfqpoint{0.041667in}{0.041667in}}{%
\pgfpathmoveto{\pgfqpoint{-0.041667in}{0.000000in}}%
\pgfpathlineto{\pgfqpoint{0.041667in}{0.000000in}}%
\pgfpathmoveto{\pgfqpoint{0.000000in}{-0.041667in}}%
\pgfpathlineto{\pgfqpoint{0.000000in}{0.041667in}}%
\pgfusepath{stroke,fill}%
}%
\begin{pgfscope}%
\pgfsys@transformshift{5.900965in}{2.100776in}%
\pgfsys@useobject{currentmarker}{}%
\end{pgfscope}%
\begin{pgfscope}%
\pgfsys@transformshift{6.246048in}{1.848977in}%
\pgfsys@useobject{currentmarker}{}%
\end{pgfscope}%
\begin{pgfscope}%
\pgfsys@transformshift{6.591132in}{1.739885in}%
\pgfsys@useobject{currentmarker}{}%
\end{pgfscope}%
\begin{pgfscope}%
\pgfsys@transformshift{6.936215in}{1.659046in}%
\pgfsys@useobject{currentmarker}{}%
\end{pgfscope}%
\begin{pgfscope}%
\pgfsys@transformshift{7.281299in}{1.595842in}%
\pgfsys@useobject{currentmarker}{}%
\end{pgfscope}%
\begin{pgfscope}%
\pgfsys@transformshift{7.626382in}{1.544818in}%
\pgfsys@useobject{currentmarker}{}%
\end{pgfscope}%
\end{pgfscope}%
\begin{pgfscope}%
\pgfsetrectcap%
\pgfsetmiterjoin%
\pgfsetlinewidth{0.803000pt}%
\definecolor{currentstroke}{rgb}{0.000000,0.000000,0.000000}%
\pgfsetstrokecolor{currentstroke}%
\pgfsetdash{}{0pt}%
\pgfpathmoveto{\pgfqpoint{5.814694in}{0.790446in}}%
\pgfpathlineto{\pgfqpoint{5.814694in}{2.163173in}}%
\pgfusepath{stroke}%
\end{pgfscope}%
\begin{pgfscope}%
\pgfsetrectcap%
\pgfsetmiterjoin%
\pgfsetlinewidth{0.803000pt}%
\definecolor{currentstroke}{rgb}{0.000000,0.000000,0.000000}%
\pgfsetstrokecolor{currentstroke}%
\pgfsetdash{}{0pt}%
\pgfpathmoveto{\pgfqpoint{7.712653in}{0.790446in}}%
\pgfpathlineto{\pgfqpoint{7.712653in}{2.163173in}}%
\pgfusepath{stroke}%
\end{pgfscope}%
\begin{pgfscope}%
\pgfsetrectcap%
\pgfsetmiterjoin%
\pgfsetlinewidth{0.803000pt}%
\definecolor{currentstroke}{rgb}{0.000000,0.000000,0.000000}%
\pgfsetstrokecolor{currentstroke}%
\pgfsetdash{}{0pt}%
\pgfpathmoveto{\pgfqpoint{5.814694in}{0.790446in}}%
\pgfpathlineto{\pgfqpoint{7.712653in}{0.790446in}}%
\pgfusepath{stroke}%
\end{pgfscope}%
\begin{pgfscope}%
\pgfsetrectcap%
\pgfsetmiterjoin%
\pgfsetlinewidth{0.803000pt}%
\definecolor{currentstroke}{rgb}{0.000000,0.000000,0.000000}%
\pgfsetstrokecolor{currentstroke}%
\pgfsetdash{}{0pt}%
\pgfpathmoveto{\pgfqpoint{5.814694in}{2.163173in}}%
\pgfpathlineto{\pgfqpoint{7.712653in}{2.163173in}}%
\pgfusepath{stroke}%
\end{pgfscope}%
\begin{pgfscope}%
\pgfsetbuttcap%
\pgfsetmiterjoin%
\definecolor{currentfill}{rgb}{1.000000,1.000000,1.000000}%
\pgfsetfillcolor{currentfill}%
\pgfsetlinewidth{0.000000pt}%
\definecolor{currentstroke}{rgb}{0.000000,0.000000,0.000000}%
\pgfsetstrokecolor{currentstroke}%
\pgfsetstrokeopacity{0.000000}%
\pgfsetdash{}{0pt}%
\pgfpathmoveto{\pgfqpoint{8.282041in}{0.790446in}}%
\pgfpathlineto{\pgfqpoint{10.180000in}{0.790446in}}%
\pgfpathlineto{\pgfqpoint{10.180000in}{2.163173in}}%
\pgfpathlineto{\pgfqpoint{8.282041in}{2.163173in}}%
\pgfpathclose%
\pgfusepath{fill}%
\end{pgfscope}%
\begin{pgfscope}%
\pgfsetbuttcap%
\pgfsetroundjoin%
\definecolor{currentfill}{rgb}{0.000000,0.000000,0.000000}%
\pgfsetfillcolor{currentfill}%
\pgfsetlinewidth{0.803000pt}%
\definecolor{currentstroke}{rgb}{0.000000,0.000000,0.000000}%
\pgfsetstrokecolor{currentstroke}%
\pgfsetdash{}{0pt}%
\pgfsys@defobject{currentmarker}{\pgfqpoint{0.000000in}{-0.048611in}}{\pgfqpoint{0.000000in}{0.000000in}}{%
\pgfpathmoveto{\pgfqpoint{0.000000in}{0.000000in}}%
\pgfpathlineto{\pgfqpoint{0.000000in}{-0.048611in}}%
\pgfusepath{stroke,fill}%
}%
\begin{pgfscope}%
\pgfsys@transformshift{8.703810in}{0.790446in}%
\pgfsys@useobject{currentmarker}{}%
\end{pgfscope}%
\end{pgfscope}%
\begin{pgfscope}%
\pgftext[x=8.703810in,y=0.693224in,,top]{\rmfamily\fontsize{10.000000}{12.000000}\selectfont \(\displaystyle 0.25\)}%
\end{pgfscope}%
\begin{pgfscope}%
\pgfsetbuttcap%
\pgfsetroundjoin%
\definecolor{currentfill}{rgb}{0.000000,0.000000,0.000000}%
\pgfsetfillcolor{currentfill}%
\pgfsetlinewidth{0.803000pt}%
\definecolor{currentstroke}{rgb}{0.000000,0.000000,0.000000}%
\pgfsetstrokecolor{currentstroke}%
\pgfsetdash{}{0pt}%
\pgfsys@defobject{currentmarker}{\pgfqpoint{0.000000in}{-0.048611in}}{\pgfqpoint{0.000000in}{0.000000in}}{%
\pgfpathmoveto{\pgfqpoint{0.000000in}{0.000000in}}%
\pgfpathlineto{\pgfqpoint{0.000000in}{-0.048611in}}%
\pgfusepath{stroke,fill}%
}%
\begin{pgfscope}%
\pgfsys@transformshift{9.183092in}{0.790446in}%
\pgfsys@useobject{currentmarker}{}%
\end{pgfscope}%
\end{pgfscope}%
\begin{pgfscope}%
\pgftext[x=9.183092in,y=0.693224in,,top]{\rmfamily\fontsize{10.000000}{12.000000}\selectfont \(\displaystyle 0.50\)}%
\end{pgfscope}%
\begin{pgfscope}%
\pgfsetbuttcap%
\pgfsetroundjoin%
\definecolor{currentfill}{rgb}{0.000000,0.000000,0.000000}%
\pgfsetfillcolor{currentfill}%
\pgfsetlinewidth{0.803000pt}%
\definecolor{currentstroke}{rgb}{0.000000,0.000000,0.000000}%
\pgfsetstrokecolor{currentstroke}%
\pgfsetdash{}{0pt}%
\pgfsys@defobject{currentmarker}{\pgfqpoint{0.000000in}{-0.048611in}}{\pgfqpoint{0.000000in}{0.000000in}}{%
\pgfpathmoveto{\pgfqpoint{0.000000in}{0.000000in}}%
\pgfpathlineto{\pgfqpoint{0.000000in}{-0.048611in}}%
\pgfusepath{stroke,fill}%
}%
\begin{pgfscope}%
\pgfsys@transformshift{9.662375in}{0.790446in}%
\pgfsys@useobject{currentmarker}{}%
\end{pgfscope}%
\end{pgfscope}%
\begin{pgfscope}%
\pgftext[x=9.662375in,y=0.693224in,,top]{\rmfamily\fontsize{10.000000}{12.000000}\selectfont \(\displaystyle 0.75\)}%
\end{pgfscope}%
\begin{pgfscope}%
\pgfsetbuttcap%
\pgfsetroundjoin%
\definecolor{currentfill}{rgb}{0.000000,0.000000,0.000000}%
\pgfsetfillcolor{currentfill}%
\pgfsetlinewidth{0.803000pt}%
\definecolor{currentstroke}{rgb}{0.000000,0.000000,0.000000}%
\pgfsetstrokecolor{currentstroke}%
\pgfsetdash{}{0pt}%
\pgfsys@defobject{currentmarker}{\pgfqpoint{0.000000in}{-0.048611in}}{\pgfqpoint{0.000000in}{0.000000in}}{%
\pgfpathmoveto{\pgfqpoint{0.000000in}{0.000000in}}%
\pgfpathlineto{\pgfqpoint{0.000000in}{-0.048611in}}%
\pgfusepath{stroke,fill}%
}%
\begin{pgfscope}%
\pgfsys@transformshift{10.141657in}{0.790446in}%
\pgfsys@useobject{currentmarker}{}%
\end{pgfscope}%
\end{pgfscope}%
\begin{pgfscope}%
\pgftext[x=10.141657in,y=0.693224in,,top]{\rmfamily\fontsize{10.000000}{12.000000}\selectfont \(\displaystyle 1.00\)}%
\end{pgfscope}%
\begin{pgfscope}%
\pgfsetbuttcap%
\pgfsetroundjoin%
\definecolor{currentfill}{rgb}{0.000000,0.000000,0.000000}%
\pgfsetfillcolor{currentfill}%
\pgfsetlinewidth{0.803000pt}%
\definecolor{currentstroke}{rgb}{0.000000,0.000000,0.000000}%
\pgfsetstrokecolor{currentstroke}%
\pgfsetdash{}{0pt}%
\pgfsys@defobject{currentmarker}{\pgfqpoint{-0.048611in}{0.000000in}}{\pgfqpoint{0.000000in}{0.000000in}}{%
\pgfpathmoveto{\pgfqpoint{0.000000in}{0.000000in}}%
\pgfpathlineto{\pgfqpoint{-0.048611in}{0.000000in}}%
\pgfusepath{stroke,fill}%
}%
\begin{pgfscope}%
\pgfsys@transformshift{8.282041in}{0.944172in}%
\pgfsys@useobject{currentmarker}{}%
\end{pgfscope}%
\end{pgfscope}%
\begin{pgfscope}%
\pgftext[x=7.896816in,y=0.891410in,left,base]{\rmfamily\fontsize{10.000000}{12.000000}\selectfont \(\displaystyle 10^{-7}\)}%
\end{pgfscope}%
\begin{pgfscope}%
\pgfsetbuttcap%
\pgfsetroundjoin%
\definecolor{currentfill}{rgb}{0.000000,0.000000,0.000000}%
\pgfsetfillcolor{currentfill}%
\pgfsetlinewidth{0.803000pt}%
\definecolor{currentstroke}{rgb}{0.000000,0.000000,0.000000}%
\pgfsetstrokecolor{currentstroke}%
\pgfsetdash{}{0pt}%
\pgfsys@defobject{currentmarker}{\pgfqpoint{-0.048611in}{0.000000in}}{\pgfqpoint{0.000000in}{0.000000in}}{%
\pgfpathmoveto{\pgfqpoint{0.000000in}{0.000000in}}%
\pgfpathlineto{\pgfqpoint{-0.048611in}{0.000000in}}%
\pgfusepath{stroke,fill}%
}%
\begin{pgfscope}%
\pgfsys@transformshift{8.282041in}{1.470685in}%
\pgfsys@useobject{currentmarker}{}%
\end{pgfscope}%
\end{pgfscope}%
\begin{pgfscope}%
\pgftext[x=7.896816in,y=1.417923in,left,base]{\rmfamily\fontsize{10.000000}{12.000000}\selectfont \(\displaystyle 10^{-6}\)}%
\end{pgfscope}%
\begin{pgfscope}%
\pgfsetbuttcap%
\pgfsetroundjoin%
\definecolor{currentfill}{rgb}{0.000000,0.000000,0.000000}%
\pgfsetfillcolor{currentfill}%
\pgfsetlinewidth{0.803000pt}%
\definecolor{currentstroke}{rgb}{0.000000,0.000000,0.000000}%
\pgfsetstrokecolor{currentstroke}%
\pgfsetdash{}{0pt}%
\pgfsys@defobject{currentmarker}{\pgfqpoint{-0.048611in}{0.000000in}}{\pgfqpoint{0.000000in}{0.000000in}}{%
\pgfpathmoveto{\pgfqpoint{0.000000in}{0.000000in}}%
\pgfpathlineto{\pgfqpoint{-0.048611in}{0.000000in}}%
\pgfusepath{stroke,fill}%
}%
\begin{pgfscope}%
\pgfsys@transformshift{8.282041in}{1.997198in}%
\pgfsys@useobject{currentmarker}{}%
\end{pgfscope}%
\end{pgfscope}%
\begin{pgfscope}%
\pgftext[x=7.896816in,y=1.944436in,left,base]{\rmfamily\fontsize{10.000000}{12.000000}\selectfont \(\displaystyle 10^{-5}\)}%
\end{pgfscope}%
\begin{pgfscope}%
\pgfsetbuttcap%
\pgfsetroundjoin%
\definecolor{currentfill}{rgb}{0.000000,0.000000,0.000000}%
\pgfsetfillcolor{currentfill}%
\pgfsetlinewidth{0.602250pt}%
\definecolor{currentstroke}{rgb}{0.000000,0.000000,0.000000}%
\pgfsetstrokecolor{currentstroke}%
\pgfsetdash{}{0pt}%
\pgfsys@defobject{currentmarker}{\pgfqpoint{-0.027778in}{0.000000in}}{\pgfqpoint{0.000000in}{0.000000in}}{%
\pgfpathmoveto{\pgfqpoint{0.000000in}{0.000000in}}%
\pgfpathlineto{\pgfqpoint{-0.027778in}{0.000000in}}%
\pgfusepath{stroke,fill}%
}%
\begin{pgfscope}%
\pgfsys@transformshift{8.282041in}{0.785675in}%
\pgfsys@useobject{currentmarker}{}%
\end{pgfscope}%
\end{pgfscope}%
\begin{pgfscope}%
\pgfsetbuttcap%
\pgfsetroundjoin%
\definecolor{currentfill}{rgb}{0.000000,0.000000,0.000000}%
\pgfsetfillcolor{currentfill}%
\pgfsetlinewidth{0.602250pt}%
\definecolor{currentstroke}{rgb}{0.000000,0.000000,0.000000}%
\pgfsetstrokecolor{currentstroke}%
\pgfsetdash{}{0pt}%
\pgfsys@defobject{currentmarker}{\pgfqpoint{-0.027778in}{0.000000in}}{\pgfqpoint{0.000000in}{0.000000in}}{%
\pgfpathmoveto{\pgfqpoint{0.000000in}{0.000000in}}%
\pgfpathlineto{\pgfqpoint{-0.027778in}{0.000000in}}%
\pgfusepath{stroke,fill}%
}%
\begin{pgfscope}%
\pgfsys@transformshift{8.282041in}{0.827365in}%
\pgfsys@useobject{currentmarker}{}%
\end{pgfscope}%
\end{pgfscope}%
\begin{pgfscope}%
\pgfsetbuttcap%
\pgfsetroundjoin%
\definecolor{currentfill}{rgb}{0.000000,0.000000,0.000000}%
\pgfsetfillcolor{currentfill}%
\pgfsetlinewidth{0.602250pt}%
\definecolor{currentstroke}{rgb}{0.000000,0.000000,0.000000}%
\pgfsetstrokecolor{currentstroke}%
\pgfsetdash{}{0pt}%
\pgfsys@defobject{currentmarker}{\pgfqpoint{-0.027778in}{0.000000in}}{\pgfqpoint{0.000000in}{0.000000in}}{%
\pgfpathmoveto{\pgfqpoint{0.000000in}{0.000000in}}%
\pgfpathlineto{\pgfqpoint{-0.027778in}{0.000000in}}%
\pgfusepath{stroke,fill}%
}%
\begin{pgfscope}%
\pgfsys@transformshift{8.282041in}{0.862614in}%
\pgfsys@useobject{currentmarker}{}%
\end{pgfscope}%
\end{pgfscope}%
\begin{pgfscope}%
\pgfsetbuttcap%
\pgfsetroundjoin%
\definecolor{currentfill}{rgb}{0.000000,0.000000,0.000000}%
\pgfsetfillcolor{currentfill}%
\pgfsetlinewidth{0.602250pt}%
\definecolor{currentstroke}{rgb}{0.000000,0.000000,0.000000}%
\pgfsetstrokecolor{currentstroke}%
\pgfsetdash{}{0pt}%
\pgfsys@defobject{currentmarker}{\pgfqpoint{-0.027778in}{0.000000in}}{\pgfqpoint{0.000000in}{0.000000in}}{%
\pgfpathmoveto{\pgfqpoint{0.000000in}{0.000000in}}%
\pgfpathlineto{\pgfqpoint{-0.027778in}{0.000000in}}%
\pgfusepath{stroke,fill}%
}%
\begin{pgfscope}%
\pgfsys@transformshift{8.282041in}{0.893147in}%
\pgfsys@useobject{currentmarker}{}%
\end{pgfscope}%
\end{pgfscope}%
\begin{pgfscope}%
\pgfsetbuttcap%
\pgfsetroundjoin%
\definecolor{currentfill}{rgb}{0.000000,0.000000,0.000000}%
\pgfsetfillcolor{currentfill}%
\pgfsetlinewidth{0.602250pt}%
\definecolor{currentstroke}{rgb}{0.000000,0.000000,0.000000}%
\pgfsetstrokecolor{currentstroke}%
\pgfsetdash{}{0pt}%
\pgfsys@defobject{currentmarker}{\pgfqpoint{-0.027778in}{0.000000in}}{\pgfqpoint{0.000000in}{0.000000in}}{%
\pgfpathmoveto{\pgfqpoint{0.000000in}{0.000000in}}%
\pgfpathlineto{\pgfqpoint{-0.027778in}{0.000000in}}%
\pgfusepath{stroke,fill}%
}%
\begin{pgfscope}%
\pgfsys@transformshift{8.282041in}{0.920080in}%
\pgfsys@useobject{currentmarker}{}%
\end{pgfscope}%
\end{pgfscope}%
\begin{pgfscope}%
\pgfsetbuttcap%
\pgfsetroundjoin%
\definecolor{currentfill}{rgb}{0.000000,0.000000,0.000000}%
\pgfsetfillcolor{currentfill}%
\pgfsetlinewidth{0.602250pt}%
\definecolor{currentstroke}{rgb}{0.000000,0.000000,0.000000}%
\pgfsetstrokecolor{currentstroke}%
\pgfsetdash{}{0pt}%
\pgfsys@defobject{currentmarker}{\pgfqpoint{-0.027778in}{0.000000in}}{\pgfqpoint{0.000000in}{0.000000in}}{%
\pgfpathmoveto{\pgfqpoint{0.000000in}{0.000000in}}%
\pgfpathlineto{\pgfqpoint{-0.027778in}{0.000000in}}%
\pgfusepath{stroke,fill}%
}%
\begin{pgfscope}%
\pgfsys@transformshift{8.282041in}{1.102668in}%
\pgfsys@useobject{currentmarker}{}%
\end{pgfscope}%
\end{pgfscope}%
\begin{pgfscope}%
\pgfsetbuttcap%
\pgfsetroundjoin%
\definecolor{currentfill}{rgb}{0.000000,0.000000,0.000000}%
\pgfsetfillcolor{currentfill}%
\pgfsetlinewidth{0.602250pt}%
\definecolor{currentstroke}{rgb}{0.000000,0.000000,0.000000}%
\pgfsetstrokecolor{currentstroke}%
\pgfsetdash{}{0pt}%
\pgfsys@defobject{currentmarker}{\pgfqpoint{-0.027778in}{0.000000in}}{\pgfqpoint{0.000000in}{0.000000in}}{%
\pgfpathmoveto{\pgfqpoint{0.000000in}{0.000000in}}%
\pgfpathlineto{\pgfqpoint{-0.027778in}{0.000000in}}%
\pgfusepath{stroke,fill}%
}%
\begin{pgfscope}%
\pgfsys@transformshift{8.282041in}{1.195382in}%
\pgfsys@useobject{currentmarker}{}%
\end{pgfscope}%
\end{pgfscope}%
\begin{pgfscope}%
\pgfsetbuttcap%
\pgfsetroundjoin%
\definecolor{currentfill}{rgb}{0.000000,0.000000,0.000000}%
\pgfsetfillcolor{currentfill}%
\pgfsetlinewidth{0.602250pt}%
\definecolor{currentstroke}{rgb}{0.000000,0.000000,0.000000}%
\pgfsetstrokecolor{currentstroke}%
\pgfsetdash{}{0pt}%
\pgfsys@defobject{currentmarker}{\pgfqpoint{-0.027778in}{0.000000in}}{\pgfqpoint{0.000000in}{0.000000in}}{%
\pgfpathmoveto{\pgfqpoint{0.000000in}{0.000000in}}%
\pgfpathlineto{\pgfqpoint{-0.027778in}{0.000000in}}%
\pgfusepath{stroke,fill}%
}%
\begin{pgfscope}%
\pgfsys@transformshift{8.282041in}{1.261164in}%
\pgfsys@useobject{currentmarker}{}%
\end{pgfscope}%
\end{pgfscope}%
\begin{pgfscope}%
\pgfsetbuttcap%
\pgfsetroundjoin%
\definecolor{currentfill}{rgb}{0.000000,0.000000,0.000000}%
\pgfsetfillcolor{currentfill}%
\pgfsetlinewidth{0.602250pt}%
\definecolor{currentstroke}{rgb}{0.000000,0.000000,0.000000}%
\pgfsetstrokecolor{currentstroke}%
\pgfsetdash{}{0pt}%
\pgfsys@defobject{currentmarker}{\pgfqpoint{-0.027778in}{0.000000in}}{\pgfqpoint{0.000000in}{0.000000in}}{%
\pgfpathmoveto{\pgfqpoint{0.000000in}{0.000000in}}%
\pgfpathlineto{\pgfqpoint{-0.027778in}{0.000000in}}%
\pgfusepath{stroke,fill}%
}%
\begin{pgfscope}%
\pgfsys@transformshift{8.282041in}{1.312189in}%
\pgfsys@useobject{currentmarker}{}%
\end{pgfscope}%
\end{pgfscope}%
\begin{pgfscope}%
\pgfsetbuttcap%
\pgfsetroundjoin%
\definecolor{currentfill}{rgb}{0.000000,0.000000,0.000000}%
\pgfsetfillcolor{currentfill}%
\pgfsetlinewidth{0.602250pt}%
\definecolor{currentstroke}{rgb}{0.000000,0.000000,0.000000}%
\pgfsetstrokecolor{currentstroke}%
\pgfsetdash{}{0pt}%
\pgfsys@defobject{currentmarker}{\pgfqpoint{-0.027778in}{0.000000in}}{\pgfqpoint{0.000000in}{0.000000in}}{%
\pgfpathmoveto{\pgfqpoint{0.000000in}{0.000000in}}%
\pgfpathlineto{\pgfqpoint{-0.027778in}{0.000000in}}%
\pgfusepath{stroke,fill}%
}%
\begin{pgfscope}%
\pgfsys@transformshift{8.282041in}{1.353879in}%
\pgfsys@useobject{currentmarker}{}%
\end{pgfscope}%
\end{pgfscope}%
\begin{pgfscope}%
\pgfsetbuttcap%
\pgfsetroundjoin%
\definecolor{currentfill}{rgb}{0.000000,0.000000,0.000000}%
\pgfsetfillcolor{currentfill}%
\pgfsetlinewidth{0.602250pt}%
\definecolor{currentstroke}{rgb}{0.000000,0.000000,0.000000}%
\pgfsetstrokecolor{currentstroke}%
\pgfsetdash{}{0pt}%
\pgfsys@defobject{currentmarker}{\pgfqpoint{-0.027778in}{0.000000in}}{\pgfqpoint{0.000000in}{0.000000in}}{%
\pgfpathmoveto{\pgfqpoint{0.000000in}{0.000000in}}%
\pgfpathlineto{\pgfqpoint{-0.027778in}{0.000000in}}%
\pgfusepath{stroke,fill}%
}%
\begin{pgfscope}%
\pgfsys@transformshift{8.282041in}{1.389127in}%
\pgfsys@useobject{currentmarker}{}%
\end{pgfscope}%
\end{pgfscope}%
\begin{pgfscope}%
\pgfsetbuttcap%
\pgfsetroundjoin%
\definecolor{currentfill}{rgb}{0.000000,0.000000,0.000000}%
\pgfsetfillcolor{currentfill}%
\pgfsetlinewidth{0.602250pt}%
\definecolor{currentstroke}{rgb}{0.000000,0.000000,0.000000}%
\pgfsetstrokecolor{currentstroke}%
\pgfsetdash{}{0pt}%
\pgfsys@defobject{currentmarker}{\pgfqpoint{-0.027778in}{0.000000in}}{\pgfqpoint{0.000000in}{0.000000in}}{%
\pgfpathmoveto{\pgfqpoint{0.000000in}{0.000000in}}%
\pgfpathlineto{\pgfqpoint{-0.027778in}{0.000000in}}%
\pgfusepath{stroke,fill}%
}%
\begin{pgfscope}%
\pgfsys@transformshift{8.282041in}{1.419660in}%
\pgfsys@useobject{currentmarker}{}%
\end{pgfscope}%
\end{pgfscope}%
\begin{pgfscope}%
\pgfsetbuttcap%
\pgfsetroundjoin%
\definecolor{currentfill}{rgb}{0.000000,0.000000,0.000000}%
\pgfsetfillcolor{currentfill}%
\pgfsetlinewidth{0.602250pt}%
\definecolor{currentstroke}{rgb}{0.000000,0.000000,0.000000}%
\pgfsetstrokecolor{currentstroke}%
\pgfsetdash{}{0pt}%
\pgfsys@defobject{currentmarker}{\pgfqpoint{-0.027778in}{0.000000in}}{\pgfqpoint{0.000000in}{0.000000in}}{%
\pgfpathmoveto{\pgfqpoint{0.000000in}{0.000000in}}%
\pgfpathlineto{\pgfqpoint{-0.027778in}{0.000000in}}%
\pgfusepath{stroke,fill}%
}%
\begin{pgfscope}%
\pgfsys@transformshift{8.282041in}{1.446593in}%
\pgfsys@useobject{currentmarker}{}%
\end{pgfscope}%
\end{pgfscope}%
\begin{pgfscope}%
\pgfsetbuttcap%
\pgfsetroundjoin%
\definecolor{currentfill}{rgb}{0.000000,0.000000,0.000000}%
\pgfsetfillcolor{currentfill}%
\pgfsetlinewidth{0.602250pt}%
\definecolor{currentstroke}{rgb}{0.000000,0.000000,0.000000}%
\pgfsetstrokecolor{currentstroke}%
\pgfsetdash{}{0pt}%
\pgfsys@defobject{currentmarker}{\pgfqpoint{-0.027778in}{0.000000in}}{\pgfqpoint{0.000000in}{0.000000in}}{%
\pgfpathmoveto{\pgfqpoint{0.000000in}{0.000000in}}%
\pgfpathlineto{\pgfqpoint{-0.027778in}{0.000000in}}%
\pgfusepath{stroke,fill}%
}%
\begin{pgfscope}%
\pgfsys@transformshift{8.282041in}{1.629181in}%
\pgfsys@useobject{currentmarker}{}%
\end{pgfscope}%
\end{pgfscope}%
\begin{pgfscope}%
\pgfsetbuttcap%
\pgfsetroundjoin%
\definecolor{currentfill}{rgb}{0.000000,0.000000,0.000000}%
\pgfsetfillcolor{currentfill}%
\pgfsetlinewidth{0.602250pt}%
\definecolor{currentstroke}{rgb}{0.000000,0.000000,0.000000}%
\pgfsetstrokecolor{currentstroke}%
\pgfsetdash{}{0pt}%
\pgfsys@defobject{currentmarker}{\pgfqpoint{-0.027778in}{0.000000in}}{\pgfqpoint{0.000000in}{0.000000in}}{%
\pgfpathmoveto{\pgfqpoint{0.000000in}{0.000000in}}%
\pgfpathlineto{\pgfqpoint{-0.027778in}{0.000000in}}%
\pgfusepath{stroke,fill}%
}%
\begin{pgfscope}%
\pgfsys@transformshift{8.282041in}{1.721895in}%
\pgfsys@useobject{currentmarker}{}%
\end{pgfscope}%
\end{pgfscope}%
\begin{pgfscope}%
\pgfsetbuttcap%
\pgfsetroundjoin%
\definecolor{currentfill}{rgb}{0.000000,0.000000,0.000000}%
\pgfsetfillcolor{currentfill}%
\pgfsetlinewidth{0.602250pt}%
\definecolor{currentstroke}{rgb}{0.000000,0.000000,0.000000}%
\pgfsetstrokecolor{currentstroke}%
\pgfsetdash{}{0pt}%
\pgfsys@defobject{currentmarker}{\pgfqpoint{-0.027778in}{0.000000in}}{\pgfqpoint{0.000000in}{0.000000in}}{%
\pgfpathmoveto{\pgfqpoint{0.000000in}{0.000000in}}%
\pgfpathlineto{\pgfqpoint{-0.027778in}{0.000000in}}%
\pgfusepath{stroke,fill}%
}%
\begin{pgfscope}%
\pgfsys@transformshift{8.282041in}{1.787677in}%
\pgfsys@useobject{currentmarker}{}%
\end{pgfscope}%
\end{pgfscope}%
\begin{pgfscope}%
\pgfsetbuttcap%
\pgfsetroundjoin%
\definecolor{currentfill}{rgb}{0.000000,0.000000,0.000000}%
\pgfsetfillcolor{currentfill}%
\pgfsetlinewidth{0.602250pt}%
\definecolor{currentstroke}{rgb}{0.000000,0.000000,0.000000}%
\pgfsetstrokecolor{currentstroke}%
\pgfsetdash{}{0pt}%
\pgfsys@defobject{currentmarker}{\pgfqpoint{-0.027778in}{0.000000in}}{\pgfqpoint{0.000000in}{0.000000in}}{%
\pgfpathmoveto{\pgfqpoint{0.000000in}{0.000000in}}%
\pgfpathlineto{\pgfqpoint{-0.027778in}{0.000000in}}%
\pgfusepath{stroke,fill}%
}%
\begin{pgfscope}%
\pgfsys@transformshift{8.282041in}{1.838702in}%
\pgfsys@useobject{currentmarker}{}%
\end{pgfscope}%
\end{pgfscope}%
\begin{pgfscope}%
\pgfsetbuttcap%
\pgfsetroundjoin%
\definecolor{currentfill}{rgb}{0.000000,0.000000,0.000000}%
\pgfsetfillcolor{currentfill}%
\pgfsetlinewidth{0.602250pt}%
\definecolor{currentstroke}{rgb}{0.000000,0.000000,0.000000}%
\pgfsetstrokecolor{currentstroke}%
\pgfsetdash{}{0pt}%
\pgfsys@defobject{currentmarker}{\pgfqpoint{-0.027778in}{0.000000in}}{\pgfqpoint{0.000000in}{0.000000in}}{%
\pgfpathmoveto{\pgfqpoint{0.000000in}{0.000000in}}%
\pgfpathlineto{\pgfqpoint{-0.027778in}{0.000000in}}%
\pgfusepath{stroke,fill}%
}%
\begin{pgfscope}%
\pgfsys@transformshift{8.282041in}{1.880392in}%
\pgfsys@useobject{currentmarker}{}%
\end{pgfscope}%
\end{pgfscope}%
\begin{pgfscope}%
\pgfsetbuttcap%
\pgfsetroundjoin%
\definecolor{currentfill}{rgb}{0.000000,0.000000,0.000000}%
\pgfsetfillcolor{currentfill}%
\pgfsetlinewidth{0.602250pt}%
\definecolor{currentstroke}{rgb}{0.000000,0.000000,0.000000}%
\pgfsetstrokecolor{currentstroke}%
\pgfsetdash{}{0pt}%
\pgfsys@defobject{currentmarker}{\pgfqpoint{-0.027778in}{0.000000in}}{\pgfqpoint{0.000000in}{0.000000in}}{%
\pgfpathmoveto{\pgfqpoint{0.000000in}{0.000000in}}%
\pgfpathlineto{\pgfqpoint{-0.027778in}{0.000000in}}%
\pgfusepath{stroke,fill}%
}%
\begin{pgfscope}%
\pgfsys@transformshift{8.282041in}{1.915640in}%
\pgfsys@useobject{currentmarker}{}%
\end{pgfscope}%
\end{pgfscope}%
\begin{pgfscope}%
\pgfsetbuttcap%
\pgfsetroundjoin%
\definecolor{currentfill}{rgb}{0.000000,0.000000,0.000000}%
\pgfsetfillcolor{currentfill}%
\pgfsetlinewidth{0.602250pt}%
\definecolor{currentstroke}{rgb}{0.000000,0.000000,0.000000}%
\pgfsetstrokecolor{currentstroke}%
\pgfsetdash{}{0pt}%
\pgfsys@defobject{currentmarker}{\pgfqpoint{-0.027778in}{0.000000in}}{\pgfqpoint{0.000000in}{0.000000in}}{%
\pgfpathmoveto{\pgfqpoint{0.000000in}{0.000000in}}%
\pgfpathlineto{\pgfqpoint{-0.027778in}{0.000000in}}%
\pgfusepath{stroke,fill}%
}%
\begin{pgfscope}%
\pgfsys@transformshift{8.282041in}{1.946174in}%
\pgfsys@useobject{currentmarker}{}%
\end{pgfscope}%
\end{pgfscope}%
\begin{pgfscope}%
\pgfsetbuttcap%
\pgfsetroundjoin%
\definecolor{currentfill}{rgb}{0.000000,0.000000,0.000000}%
\pgfsetfillcolor{currentfill}%
\pgfsetlinewidth{0.602250pt}%
\definecolor{currentstroke}{rgb}{0.000000,0.000000,0.000000}%
\pgfsetstrokecolor{currentstroke}%
\pgfsetdash{}{0pt}%
\pgfsys@defobject{currentmarker}{\pgfqpoint{-0.027778in}{0.000000in}}{\pgfqpoint{0.000000in}{0.000000in}}{%
\pgfpathmoveto{\pgfqpoint{0.000000in}{0.000000in}}%
\pgfpathlineto{\pgfqpoint{-0.027778in}{0.000000in}}%
\pgfusepath{stroke,fill}%
}%
\begin{pgfscope}%
\pgfsys@transformshift{8.282041in}{1.973106in}%
\pgfsys@useobject{currentmarker}{}%
\end{pgfscope}%
\end{pgfscope}%
\begin{pgfscope}%
\pgfsetbuttcap%
\pgfsetroundjoin%
\definecolor{currentfill}{rgb}{0.000000,0.000000,0.000000}%
\pgfsetfillcolor{currentfill}%
\pgfsetlinewidth{0.602250pt}%
\definecolor{currentstroke}{rgb}{0.000000,0.000000,0.000000}%
\pgfsetstrokecolor{currentstroke}%
\pgfsetdash{}{0pt}%
\pgfsys@defobject{currentmarker}{\pgfqpoint{-0.027778in}{0.000000in}}{\pgfqpoint{0.000000in}{0.000000in}}{%
\pgfpathmoveto{\pgfqpoint{0.000000in}{0.000000in}}%
\pgfpathlineto{\pgfqpoint{-0.027778in}{0.000000in}}%
\pgfusepath{stroke,fill}%
}%
\begin{pgfscope}%
\pgfsys@transformshift{8.282041in}{2.155694in}%
\pgfsys@useobject{currentmarker}{}%
\end{pgfscope}%
\end{pgfscope}%
\begin{pgfscope}%
\pgfpathrectangle{\pgfqpoint{8.282041in}{0.790446in}}{\pgfqpoint{1.897959in}{1.372727in}} %
\pgfusepath{clip}%
\pgfsetbuttcap%
\pgfsetroundjoin%
\pgfsetlinewidth{1.505625pt}%
\definecolor{currentstroke}{rgb}{1.000000,0.000000,0.000000}%
\pgfsetstrokecolor{currentstroke}%
\pgfsetdash{{5.550000pt}{2.400000pt}}{0.000000pt}%
\pgfpathmoveto{\pgfqpoint{8.368312in}{2.007879in}}%
\pgfpathlineto{\pgfqpoint{8.384288in}{2.002324in}}%
\pgfpathlineto{\pgfqpoint{8.400264in}{1.996760in}}%
\pgfpathlineto{\pgfqpoint{8.416240in}{1.991191in}}%
\pgfpathlineto{\pgfqpoint{8.432216in}{1.985620in}}%
\pgfpathlineto{\pgfqpoint{8.448192in}{1.980048in}}%
\pgfpathlineto{\pgfqpoint{8.464168in}{1.974479in}}%
\pgfpathlineto{\pgfqpoint{8.480144in}{1.968915in}}%
\pgfpathlineto{\pgfqpoint{8.496120in}{1.963359in}}%
\pgfpathlineto{\pgfqpoint{8.512096in}{1.957814in}}%
\pgfpathlineto{\pgfqpoint{8.528073in}{1.952281in}}%
\pgfpathlineto{\pgfqpoint{8.544049in}{1.946763in}}%
\pgfpathlineto{\pgfqpoint{8.560025in}{1.941261in}}%
\pgfpathlineto{\pgfqpoint{8.576001in}{1.935778in}}%
\pgfpathlineto{\pgfqpoint{8.591977in}{1.930316in}}%
\pgfpathlineto{\pgfqpoint{8.607953in}{1.924876in}}%
\pgfpathlineto{\pgfqpoint{8.623929in}{1.919459in}}%
\pgfpathlineto{\pgfqpoint{8.639905in}{1.914068in}}%
\pgfpathlineto{\pgfqpoint{8.655881in}{1.908703in}}%
\pgfpathlineto{\pgfqpoint{8.671857in}{1.903366in}}%
\pgfpathlineto{\pgfqpoint{8.687833in}{1.898057in}}%
\pgfpathlineto{\pgfqpoint{8.703810in}{1.892779in}}%
\pgfpathlineto{\pgfqpoint{8.719786in}{1.887531in}}%
\pgfpathlineto{\pgfqpoint{8.735762in}{1.882315in}}%
\pgfpathlineto{\pgfqpoint{8.751738in}{1.877131in}}%
\pgfpathlineto{\pgfqpoint{8.767714in}{1.871980in}}%
\pgfpathlineto{\pgfqpoint{8.783690in}{1.866863in}}%
\pgfpathlineto{\pgfqpoint{8.799666in}{1.861780in}}%
\pgfpathlineto{\pgfqpoint{8.815642in}{1.856731in}}%
\pgfpathlineto{\pgfqpoint{8.831618in}{1.851718in}}%
\pgfpathlineto{\pgfqpoint{8.847594in}{1.846739in}}%
\pgfpathlineto{\pgfqpoint{8.863570in}{1.841797in}}%
\pgfpathlineto{\pgfqpoint{8.879546in}{1.836890in}}%
\pgfpathlineto{\pgfqpoint{8.895523in}{1.832019in}}%
\pgfpathlineto{\pgfqpoint{8.911499in}{1.827184in}}%
\pgfpathlineto{\pgfqpoint{8.927475in}{1.822386in}}%
\pgfpathlineto{\pgfqpoint{8.943451in}{1.817623in}}%
\pgfpathlineto{\pgfqpoint{8.959427in}{1.812897in}}%
\pgfpathlineto{\pgfqpoint{8.975403in}{1.808208in}}%
\pgfpathlineto{\pgfqpoint{8.991379in}{1.803554in}}%
\pgfpathlineto{\pgfqpoint{9.007355in}{1.798937in}}%
\pgfpathlineto{\pgfqpoint{9.023331in}{1.794356in}}%
\pgfpathlineto{\pgfqpoint{9.039307in}{1.789810in}}%
\pgfpathlineto{\pgfqpoint{9.055283in}{1.785301in}}%
\pgfpathlineto{\pgfqpoint{9.071260in}{1.780827in}}%
\pgfpathlineto{\pgfqpoint{9.087236in}{1.776388in}}%
\pgfpathlineto{\pgfqpoint{9.103212in}{1.771985in}}%
\pgfpathlineto{\pgfqpoint{9.119188in}{1.767617in}}%
\pgfpathlineto{\pgfqpoint{9.135164in}{1.763283in}}%
\pgfpathlineto{\pgfqpoint{9.151140in}{1.758984in}}%
\pgfpathlineto{\pgfqpoint{9.167116in}{1.754720in}}%
\pgfpathlineto{\pgfqpoint{9.183092in}{1.750489in}}%
\pgfpathlineto{\pgfqpoint{9.199068in}{1.746292in}}%
\pgfpathlineto{\pgfqpoint{9.215044in}{1.742128in}}%
\pgfpathlineto{\pgfqpoint{9.231020in}{1.737998in}}%
\pgfpathlineto{\pgfqpoint{9.246996in}{1.733901in}}%
\pgfpathlineto{\pgfqpoint{9.262973in}{1.729836in}}%
\pgfpathlineto{\pgfqpoint{9.278949in}{1.725803in}}%
\pgfpathlineto{\pgfqpoint{9.294925in}{1.721803in}}%
\pgfpathlineto{\pgfqpoint{9.310901in}{1.717834in}}%
\pgfpathlineto{\pgfqpoint{9.326877in}{1.713896in}}%
\pgfpathlineto{\pgfqpoint{9.342853in}{1.709990in}}%
\pgfpathlineto{\pgfqpoint{9.358829in}{1.706114in}}%
\pgfpathlineto{\pgfqpoint{9.374805in}{1.702268in}}%
\pgfpathlineto{\pgfqpoint{9.390781in}{1.698453in}}%
\pgfpathlineto{\pgfqpoint{9.406757in}{1.694667in}}%
\pgfpathlineto{\pgfqpoint{9.422733in}{1.690911in}}%
\pgfpathlineto{\pgfqpoint{9.438710in}{1.687184in}}%
\pgfpathlineto{\pgfqpoint{9.454686in}{1.683486in}}%
\pgfpathlineto{\pgfqpoint{9.470662in}{1.679816in}}%
\pgfpathlineto{\pgfqpoint{9.486638in}{1.676175in}}%
\pgfpathlineto{\pgfqpoint{9.502614in}{1.672561in}}%
\pgfpathlineto{\pgfqpoint{9.518590in}{1.668975in}}%
\pgfpathlineto{\pgfqpoint{9.534566in}{1.665416in}}%
\pgfpathlineto{\pgfqpoint{9.550542in}{1.661884in}}%
\pgfpathlineto{\pgfqpoint{9.566518in}{1.658379in}}%
\pgfpathlineto{\pgfqpoint{9.582494in}{1.654900in}}%
\pgfpathlineto{\pgfqpoint{9.598470in}{1.651447in}}%
\pgfpathlineto{\pgfqpoint{9.614447in}{1.648020in}}%
\pgfpathlineto{\pgfqpoint{9.630423in}{1.644618in}}%
\pgfpathlineto{\pgfqpoint{9.646399in}{1.641242in}}%
\pgfpathlineto{\pgfqpoint{9.662375in}{1.637890in}}%
\pgfpathlineto{\pgfqpoint{9.678351in}{1.634563in}}%
\pgfpathlineto{\pgfqpoint{9.694327in}{1.631260in}}%
\pgfpathlineto{\pgfqpoint{9.710303in}{1.627981in}}%
\pgfpathlineto{\pgfqpoint{9.726279in}{1.624726in}}%
\pgfpathlineto{\pgfqpoint{9.742255in}{1.621494in}}%
\pgfpathlineto{\pgfqpoint{9.758231in}{1.618285in}}%
\pgfpathlineto{\pgfqpoint{9.774207in}{1.615100in}}%
\pgfpathlineto{\pgfqpoint{9.790183in}{1.611937in}}%
\pgfpathlineto{\pgfqpoint{9.806160in}{1.608796in}}%
\pgfpathlineto{\pgfqpoint{9.822136in}{1.605677in}}%
\pgfpathlineto{\pgfqpoint{9.838112in}{1.602581in}}%
\pgfpathlineto{\pgfqpoint{9.854088in}{1.599506in}}%
\pgfpathlineto{\pgfqpoint{9.870064in}{1.596452in}}%
\pgfpathlineto{\pgfqpoint{9.886040in}{1.593419in}}%
\pgfpathlineto{\pgfqpoint{9.902016in}{1.590408in}}%
\pgfpathlineto{\pgfqpoint{9.917992in}{1.587417in}}%
\pgfpathlineto{\pgfqpoint{9.933968in}{1.584446in}}%
\pgfpathlineto{\pgfqpoint{9.949944in}{1.581496in}}%
\pgfpathlineto{\pgfqpoint{9.965920in}{1.578565in}}%
\pgfpathlineto{\pgfqpoint{9.981897in}{1.575654in}}%
\pgfpathlineto{\pgfqpoint{9.997873in}{1.572763in}}%
\pgfpathlineto{\pgfqpoint{10.013849in}{1.569891in}}%
\pgfpathlineto{\pgfqpoint{10.029825in}{1.567037in}}%
\pgfpathlineto{\pgfqpoint{10.045801in}{1.564203in}}%
\pgfpathlineto{\pgfqpoint{10.061777in}{1.561388in}}%
\pgfpathlineto{\pgfqpoint{10.077753in}{1.558590in}}%
\pgfpathlineto{\pgfqpoint{10.093729in}{1.555811in}}%
\pgfusepath{stroke}%
\end{pgfscope}%
\begin{pgfscope}%
\pgfpathrectangle{\pgfqpoint{8.282041in}{0.790446in}}{\pgfqpoint{1.897959in}{1.372727in}} %
\pgfusepath{clip}%
\pgfsetbuttcap%
\pgfsetmiterjoin%
\definecolor{currentfill}{rgb}{1.000000,0.000000,0.000000}%
\pgfsetfillcolor{currentfill}%
\pgfsetlinewidth{1.003750pt}%
\definecolor{currentstroke}{rgb}{1.000000,0.000000,0.000000}%
\pgfsetstrokecolor{currentstroke}%
\pgfsetdash{}{0pt}%
\pgfsys@defobject{currentmarker}{\pgfqpoint{-0.041667in}{-0.041667in}}{\pgfqpoint{0.041667in}{0.041667in}}{%
\pgfpathmoveto{\pgfqpoint{-0.041667in}{-0.041667in}}%
\pgfpathlineto{\pgfqpoint{0.041667in}{-0.041667in}}%
\pgfpathlineto{\pgfqpoint{0.041667in}{0.041667in}}%
\pgfpathlineto{\pgfqpoint{-0.041667in}{0.041667in}}%
\pgfpathclose%
\pgfusepath{stroke,fill}%
}%
\begin{pgfscope}%
\pgfsys@transformshift{8.368312in}{2.007879in}%
\pgfsys@useobject{currentmarker}{}%
\end{pgfscope}%
\begin{pgfscope}%
\pgfsys@transformshift{8.719786in}{1.887531in}%
\pgfsys@useobject{currentmarker}{}%
\end{pgfscope}%
\begin{pgfscope}%
\pgfsys@transformshift{9.071260in}{1.780827in}%
\pgfsys@useobject{currentmarker}{}%
\end{pgfscope}%
\begin{pgfscope}%
\pgfsys@transformshift{9.422733in}{1.690911in}%
\pgfsys@useobject{currentmarker}{}%
\end{pgfscope}%
\begin{pgfscope}%
\pgfsys@transformshift{9.774207in}{1.615100in}%
\pgfsys@useobject{currentmarker}{}%
\end{pgfscope}%
\end{pgfscope}%
\begin{pgfscope}%
\pgfpathrectangle{\pgfqpoint{8.282041in}{0.790446in}}{\pgfqpoint{1.897959in}{1.372727in}} %
\pgfusepath{clip}%
\pgfsetrectcap%
\pgfsetroundjoin%
\pgfsetlinewidth{1.505625pt}%
\definecolor{currentstroke}{rgb}{0.000000,0.000000,1.000000}%
\pgfsetstrokecolor{currentstroke}%
\pgfsetdash{}{0pt}%
\pgfpathmoveto{\pgfqpoint{8.368312in}{1.531881in}}%
\pgfpathlineto{\pgfqpoint{8.384288in}{1.529662in}}%
\pgfpathlineto{\pgfqpoint{8.400264in}{1.527538in}}%
\pgfpathlineto{\pgfqpoint{8.416240in}{1.525496in}}%
\pgfpathlineto{\pgfqpoint{8.432216in}{1.523526in}}%
\pgfpathlineto{\pgfqpoint{8.448192in}{1.521620in}}%
\pgfpathlineto{\pgfqpoint{8.464168in}{1.519774in}}%
\pgfpathlineto{\pgfqpoint{8.480144in}{1.517981in}}%
\pgfpathlineto{\pgfqpoint{8.496120in}{1.516237in}}%
\pgfpathlineto{\pgfqpoint{8.512096in}{1.514540in}}%
\pgfpathlineto{\pgfqpoint{8.528073in}{1.512886in}}%
\pgfpathlineto{\pgfqpoint{8.544049in}{1.511272in}}%
\pgfpathlineto{\pgfqpoint{8.560025in}{1.509697in}}%
\pgfpathlineto{\pgfqpoint{8.576001in}{1.508158in}}%
\pgfpathlineto{\pgfqpoint{8.591977in}{1.506654in}}%
\pgfpathlineto{\pgfqpoint{8.607953in}{1.505182in}}%
\pgfpathlineto{\pgfqpoint{8.623929in}{1.503742in}}%
\pgfpathlineto{\pgfqpoint{8.639905in}{1.502333in}}%
\pgfpathlineto{\pgfqpoint{8.655881in}{1.500952in}}%
\pgfpathlineto{\pgfqpoint{8.671857in}{1.499600in}}%
\pgfpathlineto{\pgfqpoint{8.687833in}{1.498274in}}%
\pgfpathlineto{\pgfqpoint{8.703810in}{1.496975in}}%
\pgfpathlineto{\pgfqpoint{8.719786in}{1.495700in}}%
\pgfpathlineto{\pgfqpoint{8.735762in}{1.494449in}}%
\pgfpathlineto{\pgfqpoint{8.751738in}{1.493222in}}%
\pgfpathlineto{\pgfqpoint{8.767714in}{1.492018in}}%
\pgfpathlineto{\pgfqpoint{8.783690in}{1.490835in}}%
\pgfpathlineto{\pgfqpoint{8.799666in}{1.489674in}}%
\pgfpathlineto{\pgfqpoint{8.815642in}{1.488533in}}%
\pgfpathlineto{\pgfqpoint{8.831618in}{1.487412in}}%
\pgfpathlineto{\pgfqpoint{8.847594in}{1.486310in}}%
\pgfpathlineto{\pgfqpoint{8.863570in}{1.485227in}}%
\pgfpathlineto{\pgfqpoint{8.879546in}{1.484163in}}%
\pgfpathlineto{\pgfqpoint{8.895523in}{1.483116in}}%
\pgfpathlineto{\pgfqpoint{8.911499in}{1.482086in}}%
\pgfpathlineto{\pgfqpoint{8.927475in}{1.481072in}}%
\pgfpathlineto{\pgfqpoint{8.943451in}{1.480076in}}%
\pgfpathlineto{\pgfqpoint{8.959427in}{1.479094in}}%
\pgfpathlineto{\pgfqpoint{8.975403in}{1.478129in}}%
\pgfpathlineto{\pgfqpoint{8.991379in}{1.477178in}}%
\pgfpathlineto{\pgfqpoint{9.007355in}{1.476241in}}%
\pgfpathlineto{\pgfqpoint{9.023331in}{1.475319in}}%
\pgfpathlineto{\pgfqpoint{9.039307in}{1.474410in}}%
\pgfpathlineto{\pgfqpoint{9.055283in}{1.473515in}}%
\pgfpathlineto{\pgfqpoint{9.071260in}{1.472633in}}%
\pgfpathlineto{\pgfqpoint{9.087236in}{1.471764in}}%
\pgfpathlineto{\pgfqpoint{9.103212in}{1.470907in}}%
\pgfpathlineto{\pgfqpoint{9.119188in}{1.470062in}}%
\pgfpathlineto{\pgfqpoint{9.135164in}{1.469229in}}%
\pgfpathlineto{\pgfqpoint{9.151140in}{1.468407in}}%
\pgfpathlineto{\pgfqpoint{9.167116in}{1.467596in}}%
\pgfpathlineto{\pgfqpoint{9.183092in}{1.466797in}}%
\pgfpathlineto{\pgfqpoint{9.199068in}{1.466007in}}%
\pgfpathlineto{\pgfqpoint{9.215044in}{1.465229in}}%
\pgfpathlineto{\pgfqpoint{9.231020in}{1.464460in}}%
\pgfpathlineto{\pgfqpoint{9.246996in}{1.463701in}}%
\pgfpathlineto{\pgfqpoint{9.262973in}{1.462952in}}%
\pgfpathlineto{\pgfqpoint{9.278949in}{1.462213in}}%
\pgfpathlineto{\pgfqpoint{9.294925in}{1.461482in}}%
\pgfpathlineto{\pgfqpoint{9.310901in}{1.460761in}}%
\pgfpathlineto{\pgfqpoint{9.326877in}{1.460048in}}%
\pgfpathlineto{\pgfqpoint{9.342853in}{1.459344in}}%
\pgfpathlineto{\pgfqpoint{9.358829in}{1.458648in}}%
\pgfpathlineto{\pgfqpoint{9.374805in}{1.457960in}}%
\pgfpathlineto{\pgfqpoint{9.390781in}{1.457280in}}%
\pgfpathlineto{\pgfqpoint{9.406757in}{1.456608in}}%
\pgfpathlineto{\pgfqpoint{9.422733in}{1.455944in}}%
\pgfpathlineto{\pgfqpoint{9.438710in}{1.455287in}}%
\pgfpathlineto{\pgfqpoint{9.454686in}{1.454637in}}%
\pgfpathlineto{\pgfqpoint{9.470662in}{1.453995in}}%
\pgfpathlineto{\pgfqpoint{9.486638in}{1.453359in}}%
\pgfpathlineto{\pgfqpoint{9.502614in}{1.452731in}}%
\pgfpathlineto{\pgfqpoint{9.518590in}{1.452109in}}%
\pgfpathlineto{\pgfqpoint{9.534566in}{1.451493in}}%
\pgfpathlineto{\pgfqpoint{9.550542in}{1.450884in}}%
\pgfpathlineto{\pgfqpoint{9.566518in}{1.450281in}}%
\pgfpathlineto{\pgfqpoint{9.582494in}{1.449684in}}%
\pgfpathlineto{\pgfqpoint{9.598470in}{1.449094in}}%
\pgfpathlineto{\pgfqpoint{9.614447in}{1.448509in}}%
\pgfpathlineto{\pgfqpoint{9.630423in}{1.447930in}}%
\pgfpathlineto{\pgfqpoint{9.646399in}{1.447356in}}%
\pgfpathlineto{\pgfqpoint{9.662375in}{1.446788in}}%
\pgfpathlineto{\pgfqpoint{9.678351in}{1.446226in}}%
\pgfpathlineto{\pgfqpoint{9.694327in}{1.445669in}}%
\pgfpathlineto{\pgfqpoint{9.710303in}{1.445117in}}%
\pgfpathlineto{\pgfqpoint{9.726279in}{1.444570in}}%
\pgfpathlineto{\pgfqpoint{9.742255in}{1.444028in}}%
\pgfpathlineto{\pgfqpoint{9.758231in}{1.443491in}}%
\pgfpathlineto{\pgfqpoint{9.774207in}{1.442958in}}%
\pgfpathlineto{\pgfqpoint{9.790183in}{1.442431in}}%
\pgfpathlineto{\pgfqpoint{9.806160in}{1.441908in}}%
\pgfpathlineto{\pgfqpoint{9.822136in}{1.441390in}}%
\pgfpathlineto{\pgfqpoint{9.838112in}{1.440876in}}%
\pgfpathlineto{\pgfqpoint{9.854088in}{1.440366in}}%
\pgfpathlineto{\pgfqpoint{9.870064in}{1.439861in}}%
\pgfpathlineto{\pgfqpoint{9.886040in}{1.439360in}}%
\pgfpathlineto{\pgfqpoint{9.902016in}{1.438863in}}%
\pgfpathlineto{\pgfqpoint{9.917992in}{1.438370in}}%
\pgfpathlineto{\pgfqpoint{9.933968in}{1.437881in}}%
\pgfpathlineto{\pgfqpoint{9.949944in}{1.437396in}}%
\pgfpathlineto{\pgfqpoint{9.965920in}{1.436914in}}%
\pgfpathlineto{\pgfqpoint{9.981897in}{1.436437in}}%
\pgfpathlineto{\pgfqpoint{9.997873in}{1.435963in}}%
\pgfpathlineto{\pgfqpoint{10.013849in}{1.435493in}}%
\pgfpathlineto{\pgfqpoint{10.029825in}{1.435026in}}%
\pgfpathlineto{\pgfqpoint{10.045801in}{1.434563in}}%
\pgfpathlineto{\pgfqpoint{10.061777in}{1.434103in}}%
\pgfpathlineto{\pgfqpoint{10.077753in}{1.433647in}}%
\pgfpathlineto{\pgfqpoint{10.093729in}{1.433194in}}%
\pgfusepath{stroke}%
\end{pgfscope}%
\begin{pgfscope}%
\pgfpathrectangle{\pgfqpoint{8.282041in}{0.790446in}}{\pgfqpoint{1.897959in}{1.372727in}} %
\pgfusepath{clip}%
\pgfsetbuttcap%
\pgfsetroundjoin%
\definecolor{currentfill}{rgb}{0.000000,0.000000,1.000000}%
\pgfsetfillcolor{currentfill}%
\pgfsetlinewidth{1.003750pt}%
\definecolor{currentstroke}{rgb}{0.000000,0.000000,1.000000}%
\pgfsetstrokecolor{currentstroke}%
\pgfsetdash{}{0pt}%
\pgfsys@defobject{currentmarker}{\pgfqpoint{-0.041667in}{-0.041667in}}{\pgfqpoint{0.041667in}{0.041667in}}{%
\pgfpathmoveto{\pgfqpoint{0.000000in}{-0.041667in}}%
\pgfpathcurveto{\pgfqpoint{0.011050in}{-0.041667in}}{\pgfqpoint{0.021649in}{-0.037276in}}{\pgfqpoint{0.029463in}{-0.029463in}}%
\pgfpathcurveto{\pgfqpoint{0.037276in}{-0.021649in}}{\pgfqpoint{0.041667in}{-0.011050in}}{\pgfqpoint{0.041667in}{0.000000in}}%
\pgfpathcurveto{\pgfqpoint{0.041667in}{0.011050in}}{\pgfqpoint{0.037276in}{0.021649in}}{\pgfqpoint{0.029463in}{0.029463in}}%
\pgfpathcurveto{\pgfqpoint{0.021649in}{0.037276in}}{\pgfqpoint{0.011050in}{0.041667in}}{\pgfqpoint{0.000000in}{0.041667in}}%
\pgfpathcurveto{\pgfqpoint{-0.011050in}{0.041667in}}{\pgfqpoint{-0.021649in}{0.037276in}}{\pgfqpoint{-0.029463in}{0.029463in}}%
\pgfpathcurveto{\pgfqpoint{-0.037276in}{0.021649in}}{\pgfqpoint{-0.041667in}{0.011050in}}{\pgfqpoint{-0.041667in}{0.000000in}}%
\pgfpathcurveto{\pgfqpoint{-0.041667in}{-0.011050in}}{\pgfqpoint{-0.037276in}{-0.021649in}}{\pgfqpoint{-0.029463in}{-0.029463in}}%
\pgfpathcurveto{\pgfqpoint{-0.021649in}{-0.037276in}}{\pgfqpoint{-0.011050in}{-0.041667in}}{\pgfqpoint{0.000000in}{-0.041667in}}%
\pgfpathclose%
\pgfusepath{stroke,fill}%
}%
\begin{pgfscope}%
\pgfsys@transformshift{8.368312in}{1.531881in}%
\pgfsys@useobject{currentmarker}{}%
\end{pgfscope}%
\begin{pgfscope}%
\pgfsys@transformshift{8.719786in}{1.495700in}%
\pgfsys@useobject{currentmarker}{}%
\end{pgfscope}%
\begin{pgfscope}%
\pgfsys@transformshift{9.071260in}{1.472633in}%
\pgfsys@useobject{currentmarker}{}%
\end{pgfscope}%
\begin{pgfscope}%
\pgfsys@transformshift{9.422733in}{1.455944in}%
\pgfsys@useobject{currentmarker}{}%
\end{pgfscope}%
\begin{pgfscope}%
\pgfsys@transformshift{9.774207in}{1.442958in}%
\pgfsys@useobject{currentmarker}{}%
\end{pgfscope}%
\end{pgfscope}%
\begin{pgfscope}%
\pgfpathrectangle{\pgfqpoint{8.282041in}{0.790446in}}{\pgfqpoint{1.897959in}{1.372727in}} %
\pgfusepath{clip}%
\pgfsetbuttcap%
\pgfsetroundjoin%
\pgfsetlinewidth{1.505625pt}%
\definecolor{currentstroke}{rgb}{0.000000,0.750000,0.750000}%
\pgfsetstrokecolor{currentstroke}%
\pgfsetdash{{9.600000pt}{2.400000pt}{1.500000pt}{2.400000pt}}{0.000000pt}%
\pgfpathmoveto{\pgfqpoint{8.368312in}{1.786740in}}%
\pgfpathlineto{\pgfqpoint{8.384288in}{1.753759in}}%
\pgfpathlineto{\pgfqpoint{8.400264in}{1.723196in}}%
\pgfpathlineto{\pgfqpoint{8.416240in}{1.694725in}}%
\pgfpathlineto{\pgfqpoint{8.432216in}{1.668085in}}%
\pgfpathlineto{\pgfqpoint{8.448192in}{1.643057in}}%
\pgfpathlineto{\pgfqpoint{8.464168in}{1.619461in}}%
\pgfpathlineto{\pgfqpoint{8.480144in}{1.597147in}}%
\pgfpathlineto{\pgfqpoint{8.496120in}{1.575984in}}%
\pgfpathlineto{\pgfqpoint{8.512096in}{1.555862in}}%
\pgfpathlineto{\pgfqpoint{8.528073in}{1.536686in}}%
\pgfpathlineto{\pgfqpoint{8.544049in}{1.518373in}}%
\pgfpathlineto{\pgfqpoint{8.560025in}{1.500850in}}%
\pgfpathlineto{\pgfqpoint{8.576001in}{1.484054in}}%
\pgfpathlineto{\pgfqpoint{8.591977in}{1.467928in}}%
\pgfpathlineto{\pgfqpoint{8.607953in}{1.452421in}}%
\pgfpathlineto{\pgfqpoint{8.623929in}{1.437490in}}%
\pgfpathlineto{\pgfqpoint{8.639905in}{1.423093in}}%
\pgfpathlineto{\pgfqpoint{8.655881in}{1.409196in}}%
\pgfpathlineto{\pgfqpoint{8.671857in}{1.395764in}}%
\pgfpathlineto{\pgfqpoint{8.687833in}{1.382770in}}%
\pgfpathlineto{\pgfqpoint{8.703810in}{1.370185in}}%
\pgfpathlineto{\pgfqpoint{8.719786in}{1.357985in}}%
\pgfpathlineto{\pgfqpoint{8.735762in}{1.346149in}}%
\pgfpathlineto{\pgfqpoint{8.751738in}{1.334654in}}%
\pgfpathlineto{\pgfqpoint{8.767714in}{1.323484in}}%
\pgfpathlineto{\pgfqpoint{8.783690in}{1.312620in}}%
\pgfpathlineto{\pgfqpoint{8.799666in}{1.302046in}}%
\pgfpathlineto{\pgfqpoint{8.815642in}{1.291748in}}%
\pgfpathlineto{\pgfqpoint{8.831618in}{1.281711in}}%
\pgfpathlineto{\pgfqpoint{8.847594in}{1.271924in}}%
\pgfpathlineto{\pgfqpoint{8.863570in}{1.262374in}}%
\pgfpathlineto{\pgfqpoint{8.879546in}{1.253051in}}%
\pgfpathlineto{\pgfqpoint{8.895523in}{1.243943in}}%
\pgfpathlineto{\pgfqpoint{8.911499in}{1.235042in}}%
\pgfpathlineto{\pgfqpoint{8.927475in}{1.226339in}}%
\pgfpathlineto{\pgfqpoint{8.943451in}{1.217824in}}%
\pgfpathlineto{\pgfqpoint{8.959427in}{1.209491in}}%
\pgfpathlineto{\pgfqpoint{8.975403in}{1.201331in}}%
\pgfpathlineto{\pgfqpoint{8.991379in}{1.193339in}}%
\pgfpathlineto{\pgfqpoint{9.007355in}{1.185506in}}%
\pgfpathlineto{\pgfqpoint{9.023331in}{1.177828in}}%
\pgfpathlineto{\pgfqpoint{9.039307in}{1.170298in}}%
\pgfpathlineto{\pgfqpoint{9.055283in}{1.162910in}}%
\pgfpathlineto{\pgfqpoint{9.071260in}{1.155660in}}%
\pgfpathlineto{\pgfqpoint{9.087236in}{1.148543in}}%
\pgfpathlineto{\pgfqpoint{9.103212in}{1.141553in}}%
\pgfpathlineto{\pgfqpoint{9.119188in}{1.134687in}}%
\pgfpathlineto{\pgfqpoint{9.135164in}{1.127940in}}%
\pgfpathlineto{\pgfqpoint{9.151140in}{1.121308in}}%
\pgfpathlineto{\pgfqpoint{9.167116in}{1.114787in}}%
\pgfpathlineto{\pgfqpoint{9.183092in}{1.108375in}}%
\pgfpathlineto{\pgfqpoint{9.199068in}{1.102066in}}%
\pgfpathlineto{\pgfqpoint{9.215044in}{1.095859in}}%
\pgfpathlineto{\pgfqpoint{9.231020in}{1.089749in}}%
\pgfpathlineto{\pgfqpoint{9.246996in}{1.083735in}}%
\pgfpathlineto{\pgfqpoint{9.262973in}{1.077812in}}%
\pgfpathlineto{\pgfqpoint{9.278949in}{1.071979in}}%
\pgfpathlineto{\pgfqpoint{9.294925in}{1.066232in}}%
\pgfpathlineto{\pgfqpoint{9.310901in}{1.060570in}}%
\pgfpathlineto{\pgfqpoint{9.326877in}{1.054989in}}%
\pgfpathlineto{\pgfqpoint{9.342853in}{1.049488in}}%
\pgfpathlineto{\pgfqpoint{9.358829in}{1.044064in}}%
\pgfpathlineto{\pgfqpoint{9.374805in}{1.038715in}}%
\pgfpathlineto{\pgfqpoint{9.390781in}{1.033439in}}%
\pgfpathlineto{\pgfqpoint{9.406757in}{1.028235in}}%
\pgfpathlineto{\pgfqpoint{9.422733in}{1.023099in}}%
\pgfpathlineto{\pgfqpoint{9.438710in}{1.018032in}}%
\pgfpathlineto{\pgfqpoint{9.454686in}{1.013030in}}%
\pgfpathlineto{\pgfqpoint{9.470662in}{1.008092in}}%
\pgfpathlineto{\pgfqpoint{9.486638in}{1.003217in}}%
\pgfpathlineto{\pgfqpoint{9.502614in}{0.998402in}}%
\pgfpathlineto{\pgfqpoint{9.518590in}{0.993647in}}%
\pgfpathlineto{\pgfqpoint{9.534566in}{0.988950in}}%
\pgfpathlineto{\pgfqpoint{9.550542in}{0.984310in}}%
\pgfpathlineto{\pgfqpoint{9.566518in}{0.979725in}}%
\pgfpathlineto{\pgfqpoint{9.582494in}{0.975195in}}%
\pgfpathlineto{\pgfqpoint{9.598470in}{0.970717in}}%
\pgfpathlineto{\pgfqpoint{9.614447in}{0.966290in}}%
\pgfpathlineto{\pgfqpoint{9.630423in}{0.961914in}}%
\pgfpathlineto{\pgfqpoint{9.646399in}{0.957587in}}%
\pgfpathlineto{\pgfqpoint{9.662375in}{0.953309in}}%
\pgfpathlineto{\pgfqpoint{9.678351in}{0.949078in}}%
\pgfpathlineto{\pgfqpoint{9.694327in}{0.944892in}}%
\pgfpathlineto{\pgfqpoint{9.710303in}{0.940752in}}%
\pgfpathlineto{\pgfqpoint{9.726279in}{0.936657in}}%
\pgfpathlineto{\pgfqpoint{9.742255in}{0.932604in}}%
\pgfpathlineto{\pgfqpoint{9.758231in}{0.928594in}}%
\pgfpathlineto{\pgfqpoint{9.774207in}{0.924626in}}%
\pgfpathlineto{\pgfqpoint{9.790183in}{0.920698in}}%
\pgfpathlineto{\pgfqpoint{9.806160in}{0.916810in}}%
\pgfpathlineto{\pgfqpoint{9.822136in}{0.912961in}}%
\pgfpathlineto{\pgfqpoint{9.838112in}{0.909150in}}%
\pgfpathlineto{\pgfqpoint{9.854088in}{0.905377in}}%
\pgfpathlineto{\pgfqpoint{9.870064in}{0.901641in}}%
\pgfpathlineto{\pgfqpoint{9.886040in}{0.897941in}}%
\pgfpathlineto{\pgfqpoint{9.902016in}{0.894277in}}%
\pgfpathlineto{\pgfqpoint{9.917992in}{0.890647in}}%
\pgfpathlineto{\pgfqpoint{9.933968in}{0.887051in}}%
\pgfpathlineto{\pgfqpoint{9.949944in}{0.883489in}}%
\pgfpathlineto{\pgfqpoint{9.965920in}{0.879960in}}%
\pgfpathlineto{\pgfqpoint{9.981897in}{0.876463in}}%
\pgfpathlineto{\pgfqpoint{9.997873in}{0.872998in}}%
\pgfpathlineto{\pgfqpoint{10.013849in}{0.869564in}}%
\pgfpathlineto{\pgfqpoint{10.029825in}{0.866160in}}%
\pgfpathlineto{\pgfqpoint{10.045801in}{0.862787in}}%
\pgfpathlineto{\pgfqpoint{10.061777in}{0.859443in}}%
\pgfpathlineto{\pgfqpoint{10.077753in}{0.856129in}}%
\pgfpathlineto{\pgfqpoint{10.093729in}{0.852843in}}%
\pgfusepath{stroke}%
\end{pgfscope}%
\begin{pgfscope}%
\pgfpathrectangle{\pgfqpoint{8.282041in}{0.790446in}}{\pgfqpoint{1.897959in}{1.372727in}} %
\pgfusepath{clip}%
\pgfsetbuttcap%
\pgfsetmiterjoin%
\definecolor{currentfill}{rgb}{0.000000,0.750000,0.750000}%
\pgfsetfillcolor{currentfill}%
\pgfsetlinewidth{1.003750pt}%
\definecolor{currentstroke}{rgb}{0.000000,0.750000,0.750000}%
\pgfsetstrokecolor{currentstroke}%
\pgfsetdash{}{0pt}%
\pgfsys@defobject{currentmarker}{\pgfqpoint{-0.041667in}{-0.041667in}}{\pgfqpoint{0.041667in}{0.041667in}}{%
\pgfpathmoveto{\pgfqpoint{-0.000000in}{-0.041667in}}%
\pgfpathlineto{\pgfqpoint{0.041667in}{0.041667in}}%
\pgfpathlineto{\pgfqpoint{-0.041667in}{0.041667in}}%
\pgfpathclose%
\pgfusepath{stroke,fill}%
}%
\begin{pgfscope}%
\pgfsys@transformshift{8.368312in}{1.786740in}%
\pgfsys@useobject{currentmarker}{}%
\end{pgfscope}%
\begin{pgfscope}%
\pgfsys@transformshift{8.719786in}{1.357985in}%
\pgfsys@useobject{currentmarker}{}%
\end{pgfscope}%
\begin{pgfscope}%
\pgfsys@transformshift{9.071260in}{1.155660in}%
\pgfsys@useobject{currentmarker}{}%
\end{pgfscope}%
\begin{pgfscope}%
\pgfsys@transformshift{9.422733in}{1.023099in}%
\pgfsys@useobject{currentmarker}{}%
\end{pgfscope}%
\begin{pgfscope}%
\pgfsys@transformshift{9.774207in}{0.924626in}%
\pgfsys@useobject{currentmarker}{}%
\end{pgfscope}%
\end{pgfscope}%
\begin{pgfscope}%
\pgfpathrectangle{\pgfqpoint{8.282041in}{0.790446in}}{\pgfqpoint{1.897959in}{1.372727in}} %
\pgfusepath{clip}%
\pgfsetbuttcap%
\pgfsetroundjoin%
\pgfsetlinewidth{1.505625pt}%
\definecolor{currentstroke}{rgb}{0.000000,0.000000,0.000000}%
\pgfsetstrokecolor{currentstroke}%
\pgfsetdash{{1.500000pt}{2.475000pt}}{0.000000pt}%
\pgfpathmoveto{\pgfqpoint{8.368312in}{2.100776in}}%
\pgfpathlineto{\pgfqpoint{8.384288in}{2.089725in}}%
\pgfpathlineto{\pgfqpoint{8.400264in}{2.078890in}}%
\pgfpathlineto{\pgfqpoint{8.416240in}{2.068959in}}%
\pgfpathlineto{\pgfqpoint{8.432216in}{2.059746in}}%
\pgfpathlineto{\pgfqpoint{8.448192in}{2.051119in}}%
\pgfpathlineto{\pgfqpoint{8.464168in}{2.042976in}}%
\pgfpathlineto{\pgfqpoint{8.480144in}{2.035239in}}%
\pgfpathlineto{\pgfqpoint{8.496120in}{2.027843in}}%
\pgfpathlineto{\pgfqpoint{8.512096in}{2.020739in}}%
\pgfpathlineto{\pgfqpoint{8.528073in}{2.013887in}}%
\pgfpathlineto{\pgfqpoint{8.544049in}{2.007256in}}%
\pgfpathlineto{\pgfqpoint{8.560025in}{2.000819in}}%
\pgfpathlineto{\pgfqpoint{8.576001in}{1.994555in}}%
\pgfpathlineto{\pgfqpoint{8.591977in}{1.988447in}}%
\pgfpathlineto{\pgfqpoint{8.607953in}{1.982479in}}%
\pgfpathlineto{\pgfqpoint{8.623929in}{1.976638in}}%
\pgfpathlineto{\pgfqpoint{8.639905in}{1.970917in}}%
\pgfpathlineto{\pgfqpoint{8.655881in}{1.965304in}}%
\pgfpathlineto{\pgfqpoint{8.671857in}{1.959794in}}%
\pgfpathlineto{\pgfqpoint{8.687833in}{1.954379in}}%
\pgfpathlineto{\pgfqpoint{8.703810in}{1.949055in}}%
\pgfpathlineto{\pgfqpoint{8.719786in}{1.943816in}}%
\pgfpathlineto{\pgfqpoint{8.735762in}{1.938660in}}%
\pgfpathlineto{\pgfqpoint{8.751738in}{1.933580in}}%
\pgfpathlineto{\pgfqpoint{8.767714in}{1.928575in}}%
\pgfpathlineto{\pgfqpoint{8.783690in}{1.923643in}}%
\pgfpathlineto{\pgfqpoint{8.799666in}{1.918779in}}%
\pgfpathlineto{\pgfqpoint{8.815642in}{1.913984in}}%
\pgfpathlineto{\pgfqpoint{8.831618in}{1.909253in}}%
\pgfpathlineto{\pgfqpoint{8.847594in}{1.904586in}}%
\pgfpathlineto{\pgfqpoint{8.863570in}{1.899980in}}%
\pgfpathlineto{\pgfqpoint{8.879546in}{1.895435in}}%
\pgfpathlineto{\pgfqpoint{8.895523in}{1.890949in}}%
\pgfpathlineto{\pgfqpoint{8.911499in}{1.886521in}}%
\pgfpathlineto{\pgfqpoint{8.927475in}{1.882149in}}%
\pgfpathlineto{\pgfqpoint{8.943451in}{1.877831in}}%
\pgfpathlineto{\pgfqpoint{8.959427in}{1.873567in}}%
\pgfpathlineto{\pgfqpoint{8.975403in}{1.869357in}}%
\pgfpathlineto{\pgfqpoint{8.991379in}{1.865198in}}%
\pgfpathlineto{\pgfqpoint{9.007355in}{1.861090in}}%
\pgfpathlineto{\pgfqpoint{9.023331in}{1.857033in}}%
\pgfpathlineto{\pgfqpoint{9.039307in}{1.853026in}}%
\pgfpathlineto{\pgfqpoint{9.055283in}{1.849066in}}%
\pgfpathlineto{\pgfqpoint{9.071260in}{1.845154in}}%
\pgfpathlineto{\pgfqpoint{9.087236in}{1.841290in}}%
\pgfpathlineto{\pgfqpoint{9.103212in}{1.837470in}}%
\pgfpathlineto{\pgfqpoint{9.119188in}{1.833697in}}%
\pgfpathlineto{\pgfqpoint{9.135164in}{1.829969in}}%
\pgfpathlineto{\pgfqpoint{9.151140in}{1.826285in}}%
\pgfpathlineto{\pgfqpoint{9.167116in}{1.822643in}}%
\pgfpathlineto{\pgfqpoint{9.183092in}{1.819045in}}%
\pgfpathlineto{\pgfqpoint{9.199068in}{1.815488in}}%
\pgfpathlineto{\pgfqpoint{9.215044in}{1.811973in}}%
\pgfpathlineto{\pgfqpoint{9.231020in}{1.808499in}}%
\pgfpathlineto{\pgfqpoint{9.246996in}{1.805064in}}%
\pgfpathlineto{\pgfqpoint{9.262973in}{1.801668in}}%
\pgfpathlineto{\pgfqpoint{9.278949in}{1.798313in}}%
\pgfpathlineto{\pgfqpoint{9.294925in}{1.794997in}}%
\pgfpathlineto{\pgfqpoint{9.310901in}{1.791719in}}%
\pgfpathlineto{\pgfqpoint{9.326877in}{1.788476in}}%
\pgfpathlineto{\pgfqpoint{9.342853in}{1.785271in}}%
\pgfpathlineto{\pgfqpoint{9.358829in}{1.782102in}}%
\pgfpathlineto{\pgfqpoint{9.374805in}{1.778969in}}%
\pgfpathlineto{\pgfqpoint{9.390781in}{1.775871in}}%
\pgfpathlineto{\pgfqpoint{9.406757in}{1.772807in}}%
\pgfpathlineto{\pgfqpoint{9.422733in}{1.769777in}}%
\pgfpathlineto{\pgfqpoint{9.438710in}{1.766781in}}%
\pgfpathlineto{\pgfqpoint{9.454686in}{1.763819in}}%
\pgfpathlineto{\pgfqpoint{9.470662in}{1.760888in}}%
\pgfpathlineto{\pgfqpoint{9.486638in}{1.757990in}}%
\pgfpathlineto{\pgfqpoint{9.502614in}{1.755124in}}%
\pgfpathlineto{\pgfqpoint{9.518590in}{1.752289in}}%
\pgfpathlineto{\pgfqpoint{9.534566in}{1.749485in}}%
\pgfpathlineto{\pgfqpoint{9.550542in}{1.746711in}}%
\pgfpathlineto{\pgfqpoint{9.566518in}{1.743967in}}%
\pgfpathlineto{\pgfqpoint{9.582494in}{1.741252in}}%
\pgfpathlineto{\pgfqpoint{9.598470in}{1.738566in}}%
\pgfpathlineto{\pgfqpoint{9.614447in}{1.735909in}}%
\pgfpathlineto{\pgfqpoint{9.630423in}{1.733280in}}%
\pgfpathlineto{\pgfqpoint{9.646399in}{1.730679in}}%
\pgfpathlineto{\pgfqpoint{9.662375in}{1.728106in}}%
\pgfpathlineto{\pgfqpoint{9.678351in}{1.725560in}}%
\pgfpathlineto{\pgfqpoint{9.694327in}{1.723041in}}%
\pgfpathlineto{\pgfqpoint{9.710303in}{1.720547in}}%
\pgfpathlineto{\pgfqpoint{9.726279in}{1.718079in}}%
\pgfpathlineto{\pgfqpoint{9.742255in}{1.715637in}}%
\pgfpathlineto{\pgfqpoint{9.758231in}{1.713220in}}%
\pgfpathlineto{\pgfqpoint{9.774207in}{1.710828in}}%
\pgfpathlineto{\pgfqpoint{9.790183in}{1.708461in}}%
\pgfpathlineto{\pgfqpoint{9.806160in}{1.706118in}}%
\pgfpathlineto{\pgfqpoint{9.822136in}{1.703798in}}%
\pgfpathlineto{\pgfqpoint{9.838112in}{1.701502in}}%
\pgfpathlineto{\pgfqpoint{9.854088in}{1.699228in}}%
\pgfpathlineto{\pgfqpoint{9.870064in}{1.696978in}}%
\pgfpathlineto{\pgfqpoint{9.886040in}{1.694750in}}%
\pgfpathlineto{\pgfqpoint{9.902016in}{1.692545in}}%
\pgfpathlineto{\pgfqpoint{9.917992in}{1.690362in}}%
\pgfpathlineto{\pgfqpoint{9.933968in}{1.688199in}}%
\pgfpathlineto{\pgfqpoint{9.949944in}{1.686058in}}%
\pgfpathlineto{\pgfqpoint{9.965920in}{1.683938in}}%
\pgfpathlineto{\pgfqpoint{9.981897in}{1.681839in}}%
\pgfpathlineto{\pgfqpoint{9.997873in}{1.679760in}}%
\pgfpathlineto{\pgfqpoint{10.013849in}{1.677701in}}%
\pgfpathlineto{\pgfqpoint{10.029825in}{1.675663in}}%
\pgfpathlineto{\pgfqpoint{10.045801in}{1.673643in}}%
\pgfpathlineto{\pgfqpoint{10.061777in}{1.671643in}}%
\pgfpathlineto{\pgfqpoint{10.077753in}{1.669662in}}%
\pgfpathlineto{\pgfqpoint{10.093729in}{1.667700in}}%
\pgfusepath{stroke}%
\end{pgfscope}%
\begin{pgfscope}%
\pgfpathrectangle{\pgfqpoint{8.282041in}{0.790446in}}{\pgfqpoint{1.897959in}{1.372727in}} %
\pgfusepath{clip}%
\pgfsetbuttcap%
\pgfsetroundjoin%
\definecolor{currentfill}{rgb}{0.000000,0.000000,0.000000}%
\pgfsetfillcolor{currentfill}%
\pgfsetlinewidth{1.003750pt}%
\definecolor{currentstroke}{rgb}{0.000000,0.000000,0.000000}%
\pgfsetstrokecolor{currentstroke}%
\pgfsetdash{}{0pt}%
\pgfsys@defobject{currentmarker}{\pgfqpoint{-0.041667in}{-0.041667in}}{\pgfqpoint{0.041667in}{0.041667in}}{%
\pgfpathmoveto{\pgfqpoint{-0.041667in}{0.000000in}}%
\pgfpathlineto{\pgfqpoint{0.041667in}{0.000000in}}%
\pgfpathmoveto{\pgfqpoint{0.000000in}{-0.041667in}}%
\pgfpathlineto{\pgfqpoint{0.000000in}{0.041667in}}%
\pgfusepath{stroke,fill}%
}%
\begin{pgfscope}%
\pgfsys@transformshift{8.368312in}{2.100776in}%
\pgfsys@useobject{currentmarker}{}%
\end{pgfscope}%
\begin{pgfscope}%
\pgfsys@transformshift{8.719786in}{1.943816in}%
\pgfsys@useobject{currentmarker}{}%
\end{pgfscope}%
\begin{pgfscope}%
\pgfsys@transformshift{9.071260in}{1.845154in}%
\pgfsys@useobject{currentmarker}{}%
\end{pgfscope}%
\begin{pgfscope}%
\pgfsys@transformshift{9.422733in}{1.769777in}%
\pgfsys@useobject{currentmarker}{}%
\end{pgfscope}%
\begin{pgfscope}%
\pgfsys@transformshift{9.774207in}{1.710828in}%
\pgfsys@useobject{currentmarker}{}%
\end{pgfscope}%
\end{pgfscope}%
\begin{pgfscope}%
\pgfsetrectcap%
\pgfsetmiterjoin%
\pgfsetlinewidth{0.803000pt}%
\definecolor{currentstroke}{rgb}{0.000000,0.000000,0.000000}%
\pgfsetstrokecolor{currentstroke}%
\pgfsetdash{}{0pt}%
\pgfpathmoveto{\pgfqpoint{8.282041in}{0.790446in}}%
\pgfpathlineto{\pgfqpoint{8.282041in}{2.163173in}}%
\pgfusepath{stroke}%
\end{pgfscope}%
\begin{pgfscope}%
\pgfsetrectcap%
\pgfsetmiterjoin%
\pgfsetlinewidth{0.803000pt}%
\definecolor{currentstroke}{rgb}{0.000000,0.000000,0.000000}%
\pgfsetstrokecolor{currentstroke}%
\pgfsetdash{}{0pt}%
\pgfpathmoveto{\pgfqpoint{10.180000in}{0.790446in}}%
\pgfpathlineto{\pgfqpoint{10.180000in}{2.163173in}}%
\pgfusepath{stroke}%
\end{pgfscope}%
\begin{pgfscope}%
\pgfsetrectcap%
\pgfsetmiterjoin%
\pgfsetlinewidth{0.803000pt}%
\definecolor{currentstroke}{rgb}{0.000000,0.000000,0.000000}%
\pgfsetstrokecolor{currentstroke}%
\pgfsetdash{}{0pt}%
\pgfpathmoveto{\pgfqpoint{8.282041in}{0.790446in}}%
\pgfpathlineto{\pgfqpoint{10.180000in}{0.790446in}}%
\pgfusepath{stroke}%
\end{pgfscope}%
\begin{pgfscope}%
\pgfsetrectcap%
\pgfsetmiterjoin%
\pgfsetlinewidth{0.803000pt}%
\definecolor{currentstroke}{rgb}{0.000000,0.000000,0.000000}%
\pgfsetstrokecolor{currentstroke}%
\pgfsetdash{}{0pt}%
\pgfpathmoveto{\pgfqpoint{8.282041in}{2.163173in}}%
\pgfpathlineto{\pgfqpoint{10.180000in}{2.163173in}}%
\pgfusepath{stroke}%
\end{pgfscope}%
\begin{pgfscope}%
\pgfsetbuttcap%
\pgfsetmiterjoin%
\definecolor{currentfill}{rgb}{1.000000,1.000000,1.000000}%
\pgfsetfillcolor{currentfill}%
\pgfsetfillopacity{0.800000}%
\pgfsetlinewidth{1.003750pt}%
\definecolor{currentstroke}{rgb}{0.800000,0.800000,0.800000}%
\pgfsetstrokecolor{currentstroke}%
\pgfsetstrokeopacity{0.800000}%
\pgfsetdash{}{0pt}%
\pgfpathmoveto{\pgfqpoint{10.598481in}{4.153628in}}%
\pgfpathlineto{\pgfqpoint{12.667643in}{4.153628in}}%
\pgfpathquadraticcurveto{\pgfqpoint{12.706532in}{4.153628in}}{\pgfqpoint{12.706532in}{4.192517in}}%
\pgfpathlineto{\pgfqpoint{12.706532in}{5.320180in}}%
\pgfpathquadraticcurveto{\pgfqpoint{12.706532in}{5.359069in}}{\pgfqpoint{12.667643in}{5.359069in}}%
\pgfpathlineto{\pgfqpoint{10.598481in}{5.359069in}}%
\pgfpathquadraticcurveto{\pgfqpoint{10.559592in}{5.359069in}}{\pgfqpoint{10.559592in}{5.320180in}}%
\pgfpathlineto{\pgfqpoint{10.559592in}{4.192517in}}%
\pgfpathquadraticcurveto{\pgfqpoint{10.559592in}{4.153628in}}{\pgfqpoint{10.598481in}{4.153628in}}%
\pgfpathclose%
\pgfusepath{stroke,fill}%
\end{pgfscope}%
\begin{pgfscope}%
\pgfsetbuttcap%
\pgfsetroundjoin%
\pgfsetlinewidth{1.505625pt}%
\definecolor{currentstroke}{rgb}{1.000000,0.000000,0.000000}%
\pgfsetstrokecolor{currentstroke}%
\pgfsetdash{{5.550000pt}{2.400000pt}}{0.000000pt}%
\pgfpathmoveto{\pgfqpoint{10.637370in}{5.201614in}}%
\pgfpathlineto{\pgfqpoint{11.026259in}{5.201614in}}%
\pgfusepath{stroke}%
\end{pgfscope}%
\begin{pgfscope}%
\pgfsetbuttcap%
\pgfsetmiterjoin%
\definecolor{currentfill}{rgb}{1.000000,0.000000,0.000000}%
\pgfsetfillcolor{currentfill}%
\pgfsetlinewidth{1.003750pt}%
\definecolor{currentstroke}{rgb}{1.000000,0.000000,0.000000}%
\pgfsetstrokecolor{currentstroke}%
\pgfsetdash{}{0pt}%
\pgfsys@defobject{currentmarker}{\pgfqpoint{-0.041667in}{-0.041667in}}{\pgfqpoint{0.041667in}{0.041667in}}{%
\pgfpathmoveto{\pgfqpoint{-0.041667in}{-0.041667in}}%
\pgfpathlineto{\pgfqpoint{0.041667in}{-0.041667in}}%
\pgfpathlineto{\pgfqpoint{0.041667in}{0.041667in}}%
\pgfpathlineto{\pgfqpoint{-0.041667in}{0.041667in}}%
\pgfpathclose%
\pgfusepath{stroke,fill}%
}%
\begin{pgfscope}%
\pgfsys@transformshift{10.831814in}{5.201614in}%
\pgfsys@useobject{currentmarker}{}%
\end{pgfscope}%
\end{pgfscope}%
\begin{pgfscope}%
\pgftext[x=11.181814in,y=5.133559in,left,base]{\rmfamily\fontsize{14.000000}{16.800000}\selectfont Coulomb force}%
\end{pgfscope}%
\begin{pgfscope}%
\pgfsetrectcap%
\pgfsetroundjoin%
\pgfsetlinewidth{1.505625pt}%
\definecolor{currentstroke}{rgb}{0.000000,0.000000,1.000000}%
\pgfsetstrokecolor{currentstroke}%
\pgfsetdash{}{0pt}%
\pgfpathmoveto{\pgfqpoint{10.637370in}{4.916214in}}%
\pgfpathlineto{\pgfqpoint{11.026259in}{4.916214in}}%
\pgfusepath{stroke}%
\end{pgfscope}%
\begin{pgfscope}%
\pgfsetbuttcap%
\pgfsetroundjoin%
\definecolor{currentfill}{rgb}{0.000000,0.000000,1.000000}%
\pgfsetfillcolor{currentfill}%
\pgfsetlinewidth{1.003750pt}%
\definecolor{currentstroke}{rgb}{0.000000,0.000000,1.000000}%
\pgfsetstrokecolor{currentstroke}%
\pgfsetdash{}{0pt}%
\pgfsys@defobject{currentmarker}{\pgfqpoint{-0.041667in}{-0.041667in}}{\pgfqpoint{0.041667in}{0.041667in}}{%
\pgfpathmoveto{\pgfqpoint{0.000000in}{-0.041667in}}%
\pgfpathcurveto{\pgfqpoint{0.011050in}{-0.041667in}}{\pgfqpoint{0.021649in}{-0.037276in}}{\pgfqpoint{0.029463in}{-0.029463in}}%
\pgfpathcurveto{\pgfqpoint{0.037276in}{-0.021649in}}{\pgfqpoint{0.041667in}{-0.011050in}}{\pgfqpoint{0.041667in}{0.000000in}}%
\pgfpathcurveto{\pgfqpoint{0.041667in}{0.011050in}}{\pgfqpoint{0.037276in}{0.021649in}}{\pgfqpoint{0.029463in}{0.029463in}}%
\pgfpathcurveto{\pgfqpoint{0.021649in}{0.037276in}}{\pgfqpoint{0.011050in}{0.041667in}}{\pgfqpoint{0.000000in}{0.041667in}}%
\pgfpathcurveto{\pgfqpoint{-0.011050in}{0.041667in}}{\pgfqpoint{-0.021649in}{0.037276in}}{\pgfqpoint{-0.029463in}{0.029463in}}%
\pgfpathcurveto{\pgfqpoint{-0.037276in}{0.021649in}}{\pgfqpoint{-0.041667in}{0.011050in}}{\pgfqpoint{-0.041667in}{0.000000in}}%
\pgfpathcurveto{\pgfqpoint{-0.041667in}{-0.011050in}}{\pgfqpoint{-0.037276in}{-0.021649in}}{\pgfqpoint{-0.029463in}{-0.029463in}}%
\pgfpathcurveto{\pgfqpoint{-0.021649in}{-0.037276in}}{\pgfqpoint{-0.011050in}{-0.041667in}}{\pgfqpoint{0.000000in}{-0.041667in}}%
\pgfpathclose%
\pgfusepath{stroke,fill}%
}%
\begin{pgfscope}%
\pgfsys@transformshift{10.831814in}{4.916214in}%
\pgfsys@useobject{currentmarker}{}%
\end{pgfscope}%
\end{pgfscope}%
\begin{pgfscope}%
\pgftext[x=11.181814in,y=4.848158in,left,base]{\rmfamily\fontsize{14.000000}{16.800000}\selectfont Drag force}%
\end{pgfscope}%
\begin{pgfscope}%
\pgfsetbuttcap%
\pgfsetroundjoin%
\pgfsetlinewidth{1.505625pt}%
\definecolor{currentstroke}{rgb}{0.000000,0.750000,0.750000}%
\pgfsetstrokecolor{currentstroke}%
\pgfsetdash{{9.600000pt}{2.400000pt}{1.500000pt}{2.400000pt}}{0.000000pt}%
\pgfpathmoveto{\pgfqpoint{10.637370in}{4.628060in}}%
\pgfpathlineto{\pgfqpoint{11.026259in}{4.628060in}}%
\pgfusepath{stroke}%
\end{pgfscope}%
\begin{pgfscope}%
\pgfsetbuttcap%
\pgfsetmiterjoin%
\definecolor{currentfill}{rgb}{0.000000,0.750000,0.750000}%
\pgfsetfillcolor{currentfill}%
\pgfsetlinewidth{1.003750pt}%
\definecolor{currentstroke}{rgb}{0.000000,0.750000,0.750000}%
\pgfsetstrokecolor{currentstroke}%
\pgfsetdash{}{0pt}%
\pgfsys@defobject{currentmarker}{\pgfqpoint{-0.041667in}{-0.041667in}}{\pgfqpoint{0.041667in}{0.041667in}}{%
\pgfpathmoveto{\pgfqpoint{-0.000000in}{-0.041667in}}%
\pgfpathlineto{\pgfqpoint{0.041667in}{0.041667in}}%
\pgfpathlineto{\pgfqpoint{-0.041667in}{0.041667in}}%
\pgfpathclose%
\pgfusepath{stroke,fill}%
}%
\begin{pgfscope}%
\pgfsys@transformshift{10.831814in}{4.628060in}%
\pgfsys@useobject{currentmarker}{}%
\end{pgfscope}%
\end{pgfscope}%
\begin{pgfscope}%
\pgftext[x=11.181814in,y=4.560005in,left,base]{\rmfamily\fontsize{14.000000}{16.800000}\selectfont Image force}%
\end{pgfscope}%
\begin{pgfscope}%
\pgfsetbuttcap%
\pgfsetroundjoin%
\pgfsetlinewidth{1.505625pt}%
\definecolor{currentstroke}{rgb}{0.000000,0.000000,0.000000}%
\pgfsetstrokecolor{currentstroke}%
\pgfsetdash{{1.500000pt}{2.475000pt}}{0.000000pt}%
\pgfpathmoveto{\pgfqpoint{10.637370in}{4.339907in}}%
\pgfpathlineto{\pgfqpoint{11.026259in}{4.339907in}}%
\pgfusepath{stroke}%
\end{pgfscope}%
\begin{pgfscope}%
\pgfsetbuttcap%
\pgfsetroundjoin%
\definecolor{currentfill}{rgb}{0.000000,0.000000,0.000000}%
\pgfsetfillcolor{currentfill}%
\pgfsetlinewidth{1.003750pt}%
\definecolor{currentstroke}{rgb}{0.000000,0.000000,0.000000}%
\pgfsetstrokecolor{currentstroke}%
\pgfsetdash{}{0pt}%
\pgfsys@defobject{currentmarker}{\pgfqpoint{-0.041667in}{-0.041667in}}{\pgfqpoint{0.041667in}{0.041667in}}{%
\pgfpathmoveto{\pgfqpoint{-0.041667in}{0.000000in}}%
\pgfpathlineto{\pgfqpoint{0.041667in}{0.000000in}}%
\pgfpathmoveto{\pgfqpoint{0.000000in}{-0.041667in}}%
\pgfpathlineto{\pgfqpoint{0.000000in}{0.041667in}}%
\pgfusepath{stroke,fill}%
}%
\begin{pgfscope}%
\pgfsys@transformshift{10.831814in}{4.339907in}%
\pgfsys@useobject{currentmarker}{}%
\end{pgfscope}%
\end{pgfscope}%
\begin{pgfscope}%
\pgftext[x=11.181814in,y=4.271851in,left,base]{\rmfamily\fontsize{14.000000}{16.800000}\selectfont Inertia}%
\end{pgfscope}%
\begin{pgfscope}%
\pgftext[x=5.380000in,y=0.140446in,,base]{\rmfamily\fontsize{14.000000}{16.800000}\selectfont \(\displaystyle t\) (s)}%
\end{pgfscope}%
\begin{pgfscope}%
\pgftext[x=0.247732in,y=4.095635in,left,base,rotate=90.000000]{\rmfamily\fontsize{14.000000}{16.800000}\selectfont force (N)}%
\end{pgfscope}%
\end{pgfpicture}%
\makeatother%
\endgroup%
}
    \caption{Simulated forces acting on the droplet. Experiments are shown by order of increasing apoapse.\label{fig:forces}}
\end{figure}

In the non-dimensional trajectories with short-time scaling shown in Figure \ref{fig:series_s_ds}, we see that the trajectory apoapses are consistently $\mathcal{O}(1)$, but most trajectories overshoot their characteristic time scale (which predicts returns at $\bar{t}  =2$ to the first order). We also observe that that $\mathbb{E}\mbox{u}$ is not typically a small number in this regime, imperiling our use of asymptotic estimates in this regime. We can perhaps gain some insight by comparing the asymptotic estimate for return times to the scaled experiemental return times. We see in Figure \ref{fig:times} that the long-time scaled non-dimensional time of first bounce in the experiment $t_b / t_c$, compares poorly to the asymptotic estimate for returns $t_f$ as $\mathbb{E}\mbox{u}_+$ becomes large. This is very much as we'd expect, but it is also unfortunate; for small $\mathbb{E}\mbox{u}$ the asymptotic length scale could be used to improve the characteristic length scale by $t_a = t_c t_f$. We also notice a two-tailed effect in the data; for the small $\mathbb{E}\mbox{u}$ droplets, the long-time scaling distorts the return times considerably.   
\begin{figure}[htb]
    \centering
    %% Creator: Matplotlib, PGF backend
%%
%% To include the figure in your LaTeX document, write
%%   \input{<filename>.pgf}
%%
%% Make sure the required packages are loaded in your preamble
%%   \usepackage{pgf}
%%
%% Figures using additional raster images can only be included by \input if
%% they are in the same directory as the main LaTeX file. For loading figures
%% from other directories you can use the `import` package
%%   \usepackage{import}
%% and then include the figures with
%%   \import{<path to file>}{<filename>.pgf}
%%
%% Matplotlib used the following preamble
%%   \usepackage{fontspec}
%%   \setmainfont{DejaVuSerif.ttf}[Path=/home/erin/anaconda3/lib/python3.6/site-packages/matplotlib/mpl-data/fonts/ttf/]
%%   \setsansfont{DejaVuSans.ttf}[Path=/home/erin/anaconda3/lib/python3.6/site-packages/matplotlib/mpl-data/fonts/ttf/]
%%   \setmonofont{DejaVuSansMono.ttf}[Path=/home/erin/anaconda3/lib/python3.6/site-packages/matplotlib/mpl-data/fonts/ttf/]
%%
\begingroup%
\makeatletter%
\begin{pgfpicture}%
\pgfpathrectangle{\pgfpointorigin}{\pgfqpoint{5.311276in}{3.690214in}}%
\pgfusepath{use as bounding box, clip}%
\begin{pgfscope}%
\pgfsetbuttcap%
\pgfsetmiterjoin%
\definecolor{currentfill}{rgb}{1.000000,1.000000,1.000000}%
\pgfsetfillcolor{currentfill}%
\pgfsetlinewidth{0.000000pt}%
\definecolor{currentstroke}{rgb}{1.000000,1.000000,1.000000}%
\pgfsetstrokecolor{currentstroke}%
\pgfsetdash{}{0pt}%
\pgfpathmoveto{\pgfqpoint{0.000000in}{0.000000in}}%
\pgfpathlineto{\pgfqpoint{5.311276in}{0.000000in}}%
\pgfpathlineto{\pgfqpoint{5.311276in}{3.690214in}}%
\pgfpathlineto{\pgfqpoint{0.000000in}{3.690214in}}%
\pgfpathclose%
\pgfusepath{fill}%
\end{pgfscope}%
\begin{pgfscope}%
\pgfsetbuttcap%
\pgfsetmiterjoin%
\definecolor{currentfill}{rgb}{1.000000,1.000000,1.000000}%
\pgfsetfillcolor{currentfill}%
\pgfsetlinewidth{0.000000pt}%
\definecolor{currentstroke}{rgb}{0.000000,0.000000,0.000000}%
\pgfsetstrokecolor{currentstroke}%
\pgfsetstrokeopacity{0.000000}%
\pgfsetdash{}{0pt}%
\pgfpathmoveto{\pgfqpoint{0.564660in}{0.521603in}}%
\pgfpathlineto{\pgfqpoint{4.284660in}{0.521603in}}%
\pgfpathlineto{\pgfqpoint{4.284660in}{3.541603in}}%
\pgfpathlineto{\pgfqpoint{0.564660in}{3.541603in}}%
\pgfpathclose%
\pgfusepath{fill}%
\end{pgfscope}%
\begin{pgfscope}%
\pgfsetbuttcap%
\pgfsetroundjoin%
\definecolor{currentfill}{rgb}{0.000000,0.000000,0.000000}%
\pgfsetfillcolor{currentfill}%
\pgfsetlinewidth{0.803000pt}%
\definecolor{currentstroke}{rgb}{0.000000,0.000000,0.000000}%
\pgfsetstrokecolor{currentstroke}%
\pgfsetdash{}{0pt}%
\pgfsys@defobject{currentmarker}{\pgfqpoint{0.000000in}{-0.048611in}}{\pgfqpoint{0.000000in}{0.000000in}}{%
\pgfpathmoveto{\pgfqpoint{0.000000in}{0.000000in}}%
\pgfpathlineto{\pgfqpoint{0.000000in}{-0.048611in}}%
\pgfusepath{stroke,fill}%
}%
\begin{pgfscope}%
\pgfsys@transformshift{0.564660in}{0.521603in}%
\pgfsys@useobject{currentmarker}{}%
\end{pgfscope}%
\end{pgfscope}%
\begin{pgfscope}%
\definecolor{textcolor}{rgb}{0.000000,0.000000,0.000000}%
\pgfsetstrokecolor{textcolor}%
\pgfsetfillcolor{textcolor}%
\pgftext[x=0.564660in,y=0.424381in,,top]{\color{textcolor}\rmfamily\fontsize{10.000000}{12.000000}\selectfont \(\displaystyle 0.0\)}%
\end{pgfscope}%
\begin{pgfscope}%
\pgfsetbuttcap%
\pgfsetroundjoin%
\definecolor{currentfill}{rgb}{0.000000,0.000000,0.000000}%
\pgfsetfillcolor{currentfill}%
\pgfsetlinewidth{0.803000pt}%
\definecolor{currentstroke}{rgb}{0.000000,0.000000,0.000000}%
\pgfsetstrokecolor{currentstroke}%
\pgfsetdash{}{0pt}%
\pgfsys@defobject{currentmarker}{\pgfqpoint{0.000000in}{-0.048611in}}{\pgfqpoint{0.000000in}{0.000000in}}{%
\pgfpathmoveto{\pgfqpoint{0.000000in}{0.000000in}}%
\pgfpathlineto{\pgfqpoint{0.000000in}{-0.048611in}}%
\pgfusepath{stroke,fill}%
}%
\begin{pgfscope}%
\pgfsys@transformshift{1.029660in}{0.521603in}%
\pgfsys@useobject{currentmarker}{}%
\end{pgfscope}%
\end{pgfscope}%
\begin{pgfscope}%
\definecolor{textcolor}{rgb}{0.000000,0.000000,0.000000}%
\pgfsetstrokecolor{textcolor}%
\pgfsetfillcolor{textcolor}%
\pgftext[x=1.029660in,y=0.424381in,,top]{\color{textcolor}\rmfamily\fontsize{10.000000}{12.000000}\selectfont \(\displaystyle 0.5\)}%
\end{pgfscope}%
\begin{pgfscope}%
\pgfsetbuttcap%
\pgfsetroundjoin%
\definecolor{currentfill}{rgb}{0.000000,0.000000,0.000000}%
\pgfsetfillcolor{currentfill}%
\pgfsetlinewidth{0.803000pt}%
\definecolor{currentstroke}{rgb}{0.000000,0.000000,0.000000}%
\pgfsetstrokecolor{currentstroke}%
\pgfsetdash{}{0pt}%
\pgfsys@defobject{currentmarker}{\pgfqpoint{0.000000in}{-0.048611in}}{\pgfqpoint{0.000000in}{0.000000in}}{%
\pgfpathmoveto{\pgfqpoint{0.000000in}{0.000000in}}%
\pgfpathlineto{\pgfqpoint{0.000000in}{-0.048611in}}%
\pgfusepath{stroke,fill}%
}%
\begin{pgfscope}%
\pgfsys@transformshift{1.494660in}{0.521603in}%
\pgfsys@useobject{currentmarker}{}%
\end{pgfscope}%
\end{pgfscope}%
\begin{pgfscope}%
\definecolor{textcolor}{rgb}{0.000000,0.000000,0.000000}%
\pgfsetstrokecolor{textcolor}%
\pgfsetfillcolor{textcolor}%
\pgftext[x=1.494660in,y=0.424381in,,top]{\color{textcolor}\rmfamily\fontsize{10.000000}{12.000000}\selectfont \(\displaystyle 1.0\)}%
\end{pgfscope}%
\begin{pgfscope}%
\pgfsetbuttcap%
\pgfsetroundjoin%
\definecolor{currentfill}{rgb}{0.000000,0.000000,0.000000}%
\pgfsetfillcolor{currentfill}%
\pgfsetlinewidth{0.803000pt}%
\definecolor{currentstroke}{rgb}{0.000000,0.000000,0.000000}%
\pgfsetstrokecolor{currentstroke}%
\pgfsetdash{}{0pt}%
\pgfsys@defobject{currentmarker}{\pgfqpoint{0.000000in}{-0.048611in}}{\pgfqpoint{0.000000in}{0.000000in}}{%
\pgfpathmoveto{\pgfqpoint{0.000000in}{0.000000in}}%
\pgfpathlineto{\pgfqpoint{0.000000in}{-0.048611in}}%
\pgfusepath{stroke,fill}%
}%
\begin{pgfscope}%
\pgfsys@transformshift{1.959660in}{0.521603in}%
\pgfsys@useobject{currentmarker}{}%
\end{pgfscope}%
\end{pgfscope}%
\begin{pgfscope}%
\definecolor{textcolor}{rgb}{0.000000,0.000000,0.000000}%
\pgfsetstrokecolor{textcolor}%
\pgfsetfillcolor{textcolor}%
\pgftext[x=1.959660in,y=0.424381in,,top]{\color{textcolor}\rmfamily\fontsize{10.000000}{12.000000}\selectfont \(\displaystyle 1.5\)}%
\end{pgfscope}%
\begin{pgfscope}%
\pgfsetbuttcap%
\pgfsetroundjoin%
\definecolor{currentfill}{rgb}{0.000000,0.000000,0.000000}%
\pgfsetfillcolor{currentfill}%
\pgfsetlinewidth{0.803000pt}%
\definecolor{currentstroke}{rgb}{0.000000,0.000000,0.000000}%
\pgfsetstrokecolor{currentstroke}%
\pgfsetdash{}{0pt}%
\pgfsys@defobject{currentmarker}{\pgfqpoint{0.000000in}{-0.048611in}}{\pgfqpoint{0.000000in}{0.000000in}}{%
\pgfpathmoveto{\pgfqpoint{0.000000in}{0.000000in}}%
\pgfpathlineto{\pgfqpoint{0.000000in}{-0.048611in}}%
\pgfusepath{stroke,fill}%
}%
\begin{pgfscope}%
\pgfsys@transformshift{2.424660in}{0.521603in}%
\pgfsys@useobject{currentmarker}{}%
\end{pgfscope}%
\end{pgfscope}%
\begin{pgfscope}%
\definecolor{textcolor}{rgb}{0.000000,0.000000,0.000000}%
\pgfsetstrokecolor{textcolor}%
\pgfsetfillcolor{textcolor}%
\pgftext[x=2.424660in,y=0.424381in,,top]{\color{textcolor}\rmfamily\fontsize{10.000000}{12.000000}\selectfont \(\displaystyle 2.0\)}%
\end{pgfscope}%
\begin{pgfscope}%
\pgfsetbuttcap%
\pgfsetroundjoin%
\definecolor{currentfill}{rgb}{0.000000,0.000000,0.000000}%
\pgfsetfillcolor{currentfill}%
\pgfsetlinewidth{0.803000pt}%
\definecolor{currentstroke}{rgb}{0.000000,0.000000,0.000000}%
\pgfsetstrokecolor{currentstroke}%
\pgfsetdash{}{0pt}%
\pgfsys@defobject{currentmarker}{\pgfqpoint{0.000000in}{-0.048611in}}{\pgfqpoint{0.000000in}{0.000000in}}{%
\pgfpathmoveto{\pgfqpoint{0.000000in}{0.000000in}}%
\pgfpathlineto{\pgfqpoint{0.000000in}{-0.048611in}}%
\pgfusepath{stroke,fill}%
}%
\begin{pgfscope}%
\pgfsys@transformshift{2.889660in}{0.521603in}%
\pgfsys@useobject{currentmarker}{}%
\end{pgfscope}%
\end{pgfscope}%
\begin{pgfscope}%
\definecolor{textcolor}{rgb}{0.000000,0.000000,0.000000}%
\pgfsetstrokecolor{textcolor}%
\pgfsetfillcolor{textcolor}%
\pgftext[x=2.889660in,y=0.424381in,,top]{\color{textcolor}\rmfamily\fontsize{10.000000}{12.000000}\selectfont \(\displaystyle 2.5\)}%
\end{pgfscope}%
\begin{pgfscope}%
\pgfsetbuttcap%
\pgfsetroundjoin%
\definecolor{currentfill}{rgb}{0.000000,0.000000,0.000000}%
\pgfsetfillcolor{currentfill}%
\pgfsetlinewidth{0.803000pt}%
\definecolor{currentstroke}{rgb}{0.000000,0.000000,0.000000}%
\pgfsetstrokecolor{currentstroke}%
\pgfsetdash{}{0pt}%
\pgfsys@defobject{currentmarker}{\pgfqpoint{0.000000in}{-0.048611in}}{\pgfqpoint{0.000000in}{0.000000in}}{%
\pgfpathmoveto{\pgfqpoint{0.000000in}{0.000000in}}%
\pgfpathlineto{\pgfqpoint{0.000000in}{-0.048611in}}%
\pgfusepath{stroke,fill}%
}%
\begin{pgfscope}%
\pgfsys@transformshift{3.354660in}{0.521603in}%
\pgfsys@useobject{currentmarker}{}%
\end{pgfscope}%
\end{pgfscope}%
\begin{pgfscope}%
\definecolor{textcolor}{rgb}{0.000000,0.000000,0.000000}%
\pgfsetstrokecolor{textcolor}%
\pgfsetfillcolor{textcolor}%
\pgftext[x=3.354660in,y=0.424381in,,top]{\color{textcolor}\rmfamily\fontsize{10.000000}{12.000000}\selectfont \(\displaystyle 3.0\)}%
\end{pgfscope}%
\begin{pgfscope}%
\pgfsetbuttcap%
\pgfsetroundjoin%
\definecolor{currentfill}{rgb}{0.000000,0.000000,0.000000}%
\pgfsetfillcolor{currentfill}%
\pgfsetlinewidth{0.803000pt}%
\definecolor{currentstroke}{rgb}{0.000000,0.000000,0.000000}%
\pgfsetstrokecolor{currentstroke}%
\pgfsetdash{}{0pt}%
\pgfsys@defobject{currentmarker}{\pgfqpoint{0.000000in}{-0.048611in}}{\pgfqpoint{0.000000in}{0.000000in}}{%
\pgfpathmoveto{\pgfqpoint{0.000000in}{0.000000in}}%
\pgfpathlineto{\pgfqpoint{0.000000in}{-0.048611in}}%
\pgfusepath{stroke,fill}%
}%
\begin{pgfscope}%
\pgfsys@transformshift{3.819660in}{0.521603in}%
\pgfsys@useobject{currentmarker}{}%
\end{pgfscope}%
\end{pgfscope}%
\begin{pgfscope}%
\definecolor{textcolor}{rgb}{0.000000,0.000000,0.000000}%
\pgfsetstrokecolor{textcolor}%
\pgfsetfillcolor{textcolor}%
\pgftext[x=3.819660in,y=0.424381in,,top]{\color{textcolor}\rmfamily\fontsize{10.000000}{12.000000}\selectfont \(\displaystyle 3.5\)}%
\end{pgfscope}%
\begin{pgfscope}%
\pgfsetbuttcap%
\pgfsetroundjoin%
\definecolor{currentfill}{rgb}{0.000000,0.000000,0.000000}%
\pgfsetfillcolor{currentfill}%
\pgfsetlinewidth{0.803000pt}%
\definecolor{currentstroke}{rgb}{0.000000,0.000000,0.000000}%
\pgfsetstrokecolor{currentstroke}%
\pgfsetdash{}{0pt}%
\pgfsys@defobject{currentmarker}{\pgfqpoint{0.000000in}{-0.048611in}}{\pgfqpoint{0.000000in}{0.000000in}}{%
\pgfpathmoveto{\pgfqpoint{0.000000in}{0.000000in}}%
\pgfpathlineto{\pgfqpoint{0.000000in}{-0.048611in}}%
\pgfusepath{stroke,fill}%
}%
\begin{pgfscope}%
\pgfsys@transformshift{4.284660in}{0.521603in}%
\pgfsys@useobject{currentmarker}{}%
\end{pgfscope}%
\end{pgfscope}%
\begin{pgfscope}%
\definecolor{textcolor}{rgb}{0.000000,0.000000,0.000000}%
\pgfsetstrokecolor{textcolor}%
\pgfsetfillcolor{textcolor}%
\pgftext[x=4.284660in,y=0.424381in,,top]{\color{textcolor}\rmfamily\fontsize{10.000000}{12.000000}\selectfont \(\displaystyle 4.0\)}%
\end{pgfscope}%
\begin{pgfscope}%
\definecolor{textcolor}{rgb}{0.000000,0.000000,0.000000}%
\pgfsetstrokecolor{textcolor}%
\pgfsetfillcolor{textcolor}%
\pgftext[x=2.424660in,y=0.234413in,,top]{\color{textcolor}\rmfamily\fontsize{10.000000}{12.000000}\selectfont \(\displaystyle t^*\)}%
\end{pgfscope}%
\begin{pgfscope}%
\pgfsetbuttcap%
\pgfsetroundjoin%
\definecolor{currentfill}{rgb}{0.000000,0.000000,0.000000}%
\pgfsetfillcolor{currentfill}%
\pgfsetlinewidth{0.803000pt}%
\definecolor{currentstroke}{rgb}{0.000000,0.000000,0.000000}%
\pgfsetstrokecolor{currentstroke}%
\pgfsetdash{}{0pt}%
\pgfsys@defobject{currentmarker}{\pgfqpoint{-0.048611in}{0.000000in}}{\pgfqpoint{0.000000in}{0.000000in}}{%
\pgfpathmoveto{\pgfqpoint{0.000000in}{0.000000in}}%
\pgfpathlineto{\pgfqpoint{-0.048611in}{0.000000in}}%
\pgfusepath{stroke,fill}%
}%
\begin{pgfscope}%
\pgfsys@transformshift{0.564660in}{0.671953in}%
\pgfsys@useobject{currentmarker}{}%
\end{pgfscope}%
\end{pgfscope}%
\begin{pgfscope}%
\definecolor{textcolor}{rgb}{0.000000,0.000000,0.000000}%
\pgfsetstrokecolor{textcolor}%
\pgfsetfillcolor{textcolor}%
\pgftext[x=0.289968in,y=0.619192in,left,base]{\color{textcolor}\rmfamily\fontsize{10.000000}{12.000000}\selectfont \(\displaystyle 0.0\)}%
\end{pgfscope}%
\begin{pgfscope}%
\pgfsetbuttcap%
\pgfsetroundjoin%
\definecolor{currentfill}{rgb}{0.000000,0.000000,0.000000}%
\pgfsetfillcolor{currentfill}%
\pgfsetlinewidth{0.803000pt}%
\definecolor{currentstroke}{rgb}{0.000000,0.000000,0.000000}%
\pgfsetstrokecolor{currentstroke}%
\pgfsetdash{}{0pt}%
\pgfsys@defobject{currentmarker}{\pgfqpoint{-0.048611in}{0.000000in}}{\pgfqpoint{0.000000in}{0.000000in}}{%
\pgfpathmoveto{\pgfqpoint{0.000000in}{0.000000in}}%
\pgfpathlineto{\pgfqpoint{-0.048611in}{0.000000in}}%
\pgfusepath{stroke,fill}%
}%
\begin{pgfscope}%
\pgfsys@transformshift{0.564660in}{1.124453in}%
\pgfsys@useobject{currentmarker}{}%
\end{pgfscope}%
\end{pgfscope}%
\begin{pgfscope}%
\definecolor{textcolor}{rgb}{0.000000,0.000000,0.000000}%
\pgfsetstrokecolor{textcolor}%
\pgfsetfillcolor{textcolor}%
\pgftext[x=0.289968in,y=1.071692in,left,base]{\color{textcolor}\rmfamily\fontsize{10.000000}{12.000000}\selectfont \(\displaystyle 0.2\)}%
\end{pgfscope}%
\begin{pgfscope}%
\pgfsetbuttcap%
\pgfsetroundjoin%
\definecolor{currentfill}{rgb}{0.000000,0.000000,0.000000}%
\pgfsetfillcolor{currentfill}%
\pgfsetlinewidth{0.803000pt}%
\definecolor{currentstroke}{rgb}{0.000000,0.000000,0.000000}%
\pgfsetstrokecolor{currentstroke}%
\pgfsetdash{}{0pt}%
\pgfsys@defobject{currentmarker}{\pgfqpoint{-0.048611in}{0.000000in}}{\pgfqpoint{0.000000in}{0.000000in}}{%
\pgfpathmoveto{\pgfqpoint{0.000000in}{0.000000in}}%
\pgfpathlineto{\pgfqpoint{-0.048611in}{0.000000in}}%
\pgfusepath{stroke,fill}%
}%
\begin{pgfscope}%
\pgfsys@transformshift{0.564660in}{1.576953in}%
\pgfsys@useobject{currentmarker}{}%
\end{pgfscope}%
\end{pgfscope}%
\begin{pgfscope}%
\definecolor{textcolor}{rgb}{0.000000,0.000000,0.000000}%
\pgfsetstrokecolor{textcolor}%
\pgfsetfillcolor{textcolor}%
\pgftext[x=0.289968in,y=1.524192in,left,base]{\color{textcolor}\rmfamily\fontsize{10.000000}{12.000000}\selectfont \(\displaystyle 0.4\)}%
\end{pgfscope}%
\begin{pgfscope}%
\pgfsetbuttcap%
\pgfsetroundjoin%
\definecolor{currentfill}{rgb}{0.000000,0.000000,0.000000}%
\pgfsetfillcolor{currentfill}%
\pgfsetlinewidth{0.803000pt}%
\definecolor{currentstroke}{rgb}{0.000000,0.000000,0.000000}%
\pgfsetstrokecolor{currentstroke}%
\pgfsetdash{}{0pt}%
\pgfsys@defobject{currentmarker}{\pgfqpoint{-0.048611in}{0.000000in}}{\pgfqpoint{0.000000in}{0.000000in}}{%
\pgfpathmoveto{\pgfqpoint{0.000000in}{0.000000in}}%
\pgfpathlineto{\pgfqpoint{-0.048611in}{0.000000in}}%
\pgfusepath{stroke,fill}%
}%
\begin{pgfscope}%
\pgfsys@transformshift{0.564660in}{2.029453in}%
\pgfsys@useobject{currentmarker}{}%
\end{pgfscope}%
\end{pgfscope}%
\begin{pgfscope}%
\definecolor{textcolor}{rgb}{0.000000,0.000000,0.000000}%
\pgfsetstrokecolor{textcolor}%
\pgfsetfillcolor{textcolor}%
\pgftext[x=0.289968in,y=1.976692in,left,base]{\color{textcolor}\rmfamily\fontsize{10.000000}{12.000000}\selectfont \(\displaystyle 0.6\)}%
\end{pgfscope}%
\begin{pgfscope}%
\pgfsetbuttcap%
\pgfsetroundjoin%
\definecolor{currentfill}{rgb}{0.000000,0.000000,0.000000}%
\pgfsetfillcolor{currentfill}%
\pgfsetlinewidth{0.803000pt}%
\definecolor{currentstroke}{rgb}{0.000000,0.000000,0.000000}%
\pgfsetstrokecolor{currentstroke}%
\pgfsetdash{}{0pt}%
\pgfsys@defobject{currentmarker}{\pgfqpoint{-0.048611in}{0.000000in}}{\pgfqpoint{0.000000in}{0.000000in}}{%
\pgfpathmoveto{\pgfqpoint{0.000000in}{0.000000in}}%
\pgfpathlineto{\pgfqpoint{-0.048611in}{0.000000in}}%
\pgfusepath{stroke,fill}%
}%
\begin{pgfscope}%
\pgfsys@transformshift{0.564660in}{2.481953in}%
\pgfsys@useobject{currentmarker}{}%
\end{pgfscope}%
\end{pgfscope}%
\begin{pgfscope}%
\definecolor{textcolor}{rgb}{0.000000,0.000000,0.000000}%
\pgfsetstrokecolor{textcolor}%
\pgfsetfillcolor{textcolor}%
\pgftext[x=0.289968in,y=2.429192in,left,base]{\color{textcolor}\rmfamily\fontsize{10.000000}{12.000000}\selectfont \(\displaystyle 0.8\)}%
\end{pgfscope}%
\begin{pgfscope}%
\pgfsetbuttcap%
\pgfsetroundjoin%
\definecolor{currentfill}{rgb}{0.000000,0.000000,0.000000}%
\pgfsetfillcolor{currentfill}%
\pgfsetlinewidth{0.803000pt}%
\definecolor{currentstroke}{rgb}{0.000000,0.000000,0.000000}%
\pgfsetstrokecolor{currentstroke}%
\pgfsetdash{}{0pt}%
\pgfsys@defobject{currentmarker}{\pgfqpoint{-0.048611in}{0.000000in}}{\pgfqpoint{0.000000in}{0.000000in}}{%
\pgfpathmoveto{\pgfqpoint{0.000000in}{0.000000in}}%
\pgfpathlineto{\pgfqpoint{-0.048611in}{0.000000in}}%
\pgfusepath{stroke,fill}%
}%
\begin{pgfscope}%
\pgfsys@transformshift{0.564660in}{2.934453in}%
\pgfsys@useobject{currentmarker}{}%
\end{pgfscope}%
\end{pgfscope}%
\begin{pgfscope}%
\definecolor{textcolor}{rgb}{0.000000,0.000000,0.000000}%
\pgfsetstrokecolor{textcolor}%
\pgfsetfillcolor{textcolor}%
\pgftext[x=0.289968in,y=2.881692in,left,base]{\color{textcolor}\rmfamily\fontsize{10.000000}{12.000000}\selectfont \(\displaystyle 1.0\)}%
\end{pgfscope}%
\begin{pgfscope}%
\pgfsetbuttcap%
\pgfsetroundjoin%
\definecolor{currentfill}{rgb}{0.000000,0.000000,0.000000}%
\pgfsetfillcolor{currentfill}%
\pgfsetlinewidth{0.803000pt}%
\definecolor{currentstroke}{rgb}{0.000000,0.000000,0.000000}%
\pgfsetstrokecolor{currentstroke}%
\pgfsetdash{}{0pt}%
\pgfsys@defobject{currentmarker}{\pgfqpoint{-0.048611in}{0.000000in}}{\pgfqpoint{0.000000in}{0.000000in}}{%
\pgfpathmoveto{\pgfqpoint{0.000000in}{0.000000in}}%
\pgfpathlineto{\pgfqpoint{-0.048611in}{0.000000in}}%
\pgfusepath{stroke,fill}%
}%
\begin{pgfscope}%
\pgfsys@transformshift{0.564660in}{3.386954in}%
\pgfsys@useobject{currentmarker}{}%
\end{pgfscope}%
\end{pgfscope}%
\begin{pgfscope}%
\definecolor{textcolor}{rgb}{0.000000,0.000000,0.000000}%
\pgfsetstrokecolor{textcolor}%
\pgfsetfillcolor{textcolor}%
\pgftext[x=0.289968in,y=3.334192in,left,base]{\color{textcolor}\rmfamily\fontsize{10.000000}{12.000000}\selectfont \(\displaystyle 1.2\)}%
\end{pgfscope}%
\begin{pgfscope}%
\definecolor{textcolor}{rgb}{0.000000,0.000000,0.000000}%
\pgfsetstrokecolor{textcolor}%
\pgfsetfillcolor{textcolor}%
\pgftext[x=0.234413in,y=2.031603in,,bottom,rotate=90.000000]{\color{textcolor}\rmfamily\fontsize{10.000000}{12.000000}\selectfont \(\displaystyle y^*\)}%
\end{pgfscope}%
\begin{pgfscope}%
\pgfpathrectangle{\pgfqpoint{0.564660in}{0.521603in}}{\pgfqpoint{3.720000in}{3.020000in}}%
\pgfusepath{clip}%
\pgfsetrectcap%
\pgfsetroundjoin%
\pgfsetlinewidth{1.505625pt}%
\definecolor{currentstroke}{rgb}{0.993248,0.906157,0.143936}%
\pgfsetstrokecolor{currentstroke}%
\pgfsetdash{}{0pt}%
\pgfpathmoveto{\pgfqpoint{1.409488in}{2.299358in}}%
\pgfpathlineto{\pgfqpoint{1.486291in}{2.475880in}}%
\pgfpathlineto{\pgfqpoint{1.563093in}{2.637799in}}%
\pgfpathlineto{\pgfqpoint{1.639896in}{2.784889in}}%
\pgfpathlineto{\pgfqpoint{1.716698in}{2.916927in}}%
\pgfpathlineto{\pgfqpoint{1.793501in}{3.033686in}}%
\pgfpathlineto{\pgfqpoint{1.870303in}{3.134943in}}%
\pgfpathlineto{\pgfqpoint{1.947106in}{3.220472in}}%
\pgfpathlineto{\pgfqpoint{2.023908in}{3.290048in}}%
\pgfpathlineto{\pgfqpoint{2.100711in}{3.343447in}}%
\pgfpathlineto{\pgfqpoint{2.177513in}{3.380444in}}%
\pgfpathlineto{\pgfqpoint{2.254316in}{3.400813in}}%
\pgfpathlineto{\pgfqpoint{2.331118in}{3.404331in}}%
\pgfpathlineto{\pgfqpoint{2.407921in}{3.390904in}}%
\pgfpathlineto{\pgfqpoint{2.484723in}{3.360261in}}%
\pgfpathlineto{\pgfqpoint{2.561526in}{3.312160in}}%
\pgfpathlineto{\pgfqpoint{2.638328in}{3.246345in}}%
\pgfpathlineto{\pgfqpoint{2.715131in}{3.162550in}}%
\pgfpathlineto{\pgfqpoint{2.791933in}{3.060539in}}%
\pgfpathlineto{\pgfqpoint{2.868736in}{2.939892in}}%
\pgfpathlineto{\pgfqpoint{2.945539in}{2.800327in}}%
\pgfpathlineto{\pgfqpoint{3.022341in}{2.641559in}}%
\pgfpathlineto{\pgfqpoint{3.099144in}{2.463303in}}%
\pgfpathlineto{\pgfqpoint{3.175946in}{2.265275in}}%
\pgfpathlineto{\pgfqpoint{3.252749in}{2.047191in}}%
\pgfpathlineto{\pgfqpoint{3.329551in}{1.808766in}}%
\pgfpathlineto{\pgfqpoint{3.406354in}{1.549716in}}%
\pgfpathlineto{\pgfqpoint{3.483156in}{1.269756in}}%
\pgfpathlineto{\pgfqpoint{3.559959in}{0.968602in}}%
\pgfusepath{stroke}%
\end{pgfscope}%
\begin{pgfscope}%
\pgfpathrectangle{\pgfqpoint{0.564660in}{0.521603in}}{\pgfqpoint{3.720000in}{3.020000in}}%
\pgfusepath{clip}%
\pgfsetrectcap%
\pgfsetroundjoin%
\pgfsetlinewidth{1.505625pt}%
\definecolor{currentstroke}{rgb}{0.762373,0.876424,0.137064}%
\pgfsetstrokecolor{currentstroke}%
\pgfsetdash{}{0pt}%
\pgfpathmoveto{\pgfqpoint{1.233611in}{0.756271in}}%
\pgfpathlineto{\pgfqpoint{1.300506in}{0.938723in}}%
\pgfpathlineto{\pgfqpoint{1.367401in}{1.107373in}}%
\pgfpathlineto{\pgfqpoint{1.434296in}{1.262342in}}%
\pgfpathlineto{\pgfqpoint{1.501191in}{1.403753in}}%
\pgfpathlineto{\pgfqpoint{1.568086in}{1.531729in}}%
\pgfpathlineto{\pgfqpoint{1.634981in}{1.646390in}}%
\pgfpathlineto{\pgfqpoint{1.701876in}{1.747859in}}%
\pgfpathlineto{\pgfqpoint{1.768772in}{1.836259in}}%
\pgfpathlineto{\pgfqpoint{1.835667in}{1.911710in}}%
\pgfpathlineto{\pgfqpoint{1.902562in}{1.974336in}}%
\pgfpathlineto{\pgfqpoint{1.969457in}{2.024259in}}%
\pgfpathlineto{\pgfqpoint{2.036352in}{2.061600in}}%
\pgfpathlineto{\pgfqpoint{2.103247in}{2.086685in}}%
\pgfpathlineto{\pgfqpoint{2.170142in}{2.099560in}}%
\pgfpathlineto{\pgfqpoint{2.237037in}{2.100321in}}%
\pgfpathlineto{\pgfqpoint{2.303932in}{2.089048in}}%
\pgfpathlineto{\pgfqpoint{2.370827in}{2.065802in}}%
\pgfpathlineto{\pgfqpoint{2.437722in}{2.030627in}}%
\pgfpathlineto{\pgfqpoint{2.504617in}{1.983622in}}%
\pgfpathlineto{\pgfqpoint{2.571512in}{1.924599in}}%
\pgfpathlineto{\pgfqpoint{2.638407in}{1.853579in}}%
\pgfpathlineto{\pgfqpoint{2.705302in}{1.770581in}}%
\pgfpathlineto{\pgfqpoint{2.772198in}{1.675623in}}%
\pgfpathlineto{\pgfqpoint{2.839093in}{1.568725in}}%
\pgfpathlineto{\pgfqpoint{2.905988in}{1.449905in}}%
\pgfpathlineto{\pgfqpoint{2.972883in}{1.319184in}}%
\pgfpathlineto{\pgfqpoint{3.039778in}{1.176579in}}%
\pgfpathlineto{\pgfqpoint{3.106673in}{1.022111in}}%
\pgfpathlineto{\pgfqpoint{3.173568in}{0.855798in}}%
\pgfpathlineto{\pgfqpoint{3.240463in}{0.677659in}}%
\pgfusepath{stroke}%
\end{pgfscope}%
\begin{pgfscope}%
\pgfpathrectangle{\pgfqpoint{0.564660in}{0.521603in}}{\pgfqpoint{3.720000in}{3.020000in}}%
\pgfusepath{clip}%
\pgfsetrectcap%
\pgfsetroundjoin%
\pgfsetlinewidth{1.505625pt}%
\definecolor{currentstroke}{rgb}{0.751884,0.874951,0.143228}%
\pgfsetstrokecolor{currentstroke}%
\pgfsetdash{}{0pt}%
\pgfpathmoveto{\pgfqpoint{1.178273in}{2.036187in}}%
\pgfpathlineto{\pgfqpoint{1.246452in}{2.193330in}}%
\pgfpathlineto{\pgfqpoint{1.314631in}{2.339382in}}%
\pgfpathlineto{\pgfqpoint{1.382811in}{2.474114in}}%
\pgfpathlineto{\pgfqpoint{1.450990in}{2.597300in}}%
\pgfpathlineto{\pgfqpoint{1.519169in}{2.708710in}}%
\pgfpathlineto{\pgfqpoint{1.587348in}{2.808117in}}%
\pgfpathlineto{\pgfqpoint{1.655527in}{2.895293in}}%
\pgfpathlineto{\pgfqpoint{1.723707in}{2.970010in}}%
\pgfpathlineto{\pgfqpoint{1.791886in}{3.032040in}}%
\pgfpathlineto{\pgfqpoint{1.860065in}{3.081155in}}%
\pgfpathlineto{\pgfqpoint{1.928244in}{3.117128in}}%
\pgfpathlineto{\pgfqpoint{1.996423in}{3.139729in}}%
\pgfpathlineto{\pgfqpoint{2.064603in}{3.148971in}}%
\pgfpathlineto{\pgfqpoint{2.132782in}{3.144533in}}%
\pgfpathlineto{\pgfqpoint{2.200961in}{3.126161in}}%
\pgfpathlineto{\pgfqpoint{2.269140in}{3.093578in}}%
\pgfpathlineto{\pgfqpoint{2.337319in}{3.046489in}}%
\pgfpathlineto{\pgfqpoint{2.405499in}{2.984576in}}%
\pgfpathlineto{\pgfqpoint{2.473678in}{2.907503in}}%
\pgfpathlineto{\pgfqpoint{2.541857in}{2.814997in}}%
\pgfpathlineto{\pgfqpoint{2.610036in}{2.706455in}}%
\pgfpathlineto{\pgfqpoint{2.678215in}{2.581512in}}%
\pgfpathlineto{\pgfqpoint{2.746395in}{2.439805in}}%
\pgfpathlineto{\pgfqpoint{2.814574in}{2.280968in}}%
\pgfpathlineto{\pgfqpoint{2.882753in}{2.104637in}}%
\pgfpathlineto{\pgfqpoint{2.950932in}{1.910447in}}%
\pgfpathlineto{\pgfqpoint{3.019111in}{1.698036in}}%
\pgfpathlineto{\pgfqpoint{3.087291in}{1.467037in}}%
\pgfpathlineto{\pgfqpoint{3.155470in}{1.217087in}}%
\pgfpathlineto{\pgfqpoint{3.223649in}{0.947822in}}%
\pgfpathlineto{\pgfqpoint{3.291828in}{0.658876in}}%
\pgfusepath{stroke}%
\end{pgfscope}%
\begin{pgfscope}%
\pgfpathrectangle{\pgfqpoint{0.564660in}{0.521603in}}{\pgfqpoint{3.720000in}{3.020000in}}%
\pgfusepath{clip}%
\pgfsetrectcap%
\pgfsetroundjoin%
\pgfsetlinewidth{1.505625pt}%
\definecolor{currentstroke}{rgb}{0.377779,0.791781,0.377939}%
\pgfsetstrokecolor{currentstroke}%
\pgfsetdash{}{0pt}%
\pgfpathmoveto{\pgfqpoint{1.114895in}{1.137753in}}%
\pgfpathlineto{\pgfqpoint{1.183674in}{1.296900in}}%
\pgfpathlineto{\pgfqpoint{1.252453in}{1.444075in}}%
\pgfpathlineto{\pgfqpoint{1.321233in}{1.579273in}}%
\pgfpathlineto{\pgfqpoint{1.390012in}{1.702487in}}%
\pgfpathlineto{\pgfqpoint{1.458791in}{1.813712in}}%
\pgfpathlineto{\pgfqpoint{1.527570in}{1.912942in}}%
\pgfpathlineto{\pgfqpoint{1.596350in}{2.000173in}}%
\pgfpathlineto{\pgfqpoint{1.665129in}{2.075397in}}%
\pgfpathlineto{\pgfqpoint{1.733908in}{2.138609in}}%
\pgfpathlineto{\pgfqpoint{1.802688in}{2.189804in}}%
\pgfpathlineto{\pgfqpoint{1.871467in}{2.228976in}}%
\pgfpathlineto{\pgfqpoint{1.940246in}{2.256119in}}%
\pgfpathlineto{\pgfqpoint{2.009025in}{2.271336in}}%
\pgfpathlineto{\pgfqpoint{2.077805in}{2.274580in}}%
\pgfpathlineto{\pgfqpoint{2.146584in}{2.265833in}}%
\pgfpathlineto{\pgfqpoint{2.215363in}{2.245065in}}%
\pgfpathlineto{\pgfqpoint{2.284143in}{2.212239in}}%
\pgfpathlineto{\pgfqpoint{2.352922in}{2.167308in}}%
\pgfpathlineto{\pgfqpoint{2.421701in}{2.110253in}}%
\pgfpathlineto{\pgfqpoint{2.490481in}{2.040905in}}%
\pgfpathlineto{\pgfqpoint{2.559260in}{1.959203in}}%
\pgfpathlineto{\pgfqpoint{2.628039in}{1.865086in}}%
\pgfpathlineto{\pgfqpoint{2.696818in}{1.758495in}}%
\pgfpathlineto{\pgfqpoint{2.765598in}{1.639368in}}%
\pgfpathlineto{\pgfqpoint{2.834377in}{1.507646in}}%
\pgfpathlineto{\pgfqpoint{2.903156in}{1.363268in}}%
\pgfpathlineto{\pgfqpoint{2.971936in}{1.206173in}}%
\pgfpathlineto{\pgfqpoint{3.040715in}{1.036301in}}%
\pgfpathlineto{\pgfqpoint{3.109494in}{0.853591in}}%
\pgfusepath{stroke}%
\end{pgfscope}%
\begin{pgfscope}%
\pgfpathrectangle{\pgfqpoint{0.564660in}{0.521603in}}{\pgfqpoint{3.720000in}{3.020000in}}%
\pgfusepath{clip}%
\pgfsetrectcap%
\pgfsetroundjoin%
\pgfsetlinewidth{1.505625pt}%
\definecolor{currentstroke}{rgb}{0.128729,0.563265,0.551229}%
\pgfsetstrokecolor{currentstroke}%
\pgfsetdash{}{0pt}%
\pgfpathmoveto{\pgfqpoint{1.002735in}{1.139100in}}%
\pgfpathlineto{\pgfqpoint{1.046542in}{1.242081in}}%
\pgfpathlineto{\pgfqpoint{1.090350in}{1.340628in}}%
\pgfpathlineto{\pgfqpoint{1.134157in}{1.434751in}}%
\pgfpathlineto{\pgfqpoint{1.177965in}{1.524460in}}%
\pgfpathlineto{\pgfqpoint{1.221772in}{1.609765in}}%
\pgfpathlineto{\pgfqpoint{1.265580in}{1.690674in}}%
\pgfpathlineto{\pgfqpoint{1.309387in}{1.767198in}}%
\pgfpathlineto{\pgfqpoint{1.353195in}{1.839346in}}%
\pgfpathlineto{\pgfqpoint{1.397002in}{1.907128in}}%
\pgfpathlineto{\pgfqpoint{1.440810in}{1.970554in}}%
\pgfpathlineto{\pgfqpoint{1.484617in}{2.029633in}}%
\pgfpathlineto{\pgfqpoint{1.528425in}{2.084375in}}%
\pgfpathlineto{\pgfqpoint{1.572232in}{2.134787in}}%
\pgfpathlineto{\pgfqpoint{1.616040in}{2.180881in}}%
\pgfpathlineto{\pgfqpoint{1.659847in}{2.222665in}}%
\pgfpathlineto{\pgfqpoint{1.703655in}{2.260153in}}%
\pgfpathlineto{\pgfqpoint{1.747462in}{2.293356in}}%
\pgfpathlineto{\pgfqpoint{1.791269in}{2.322287in}}%
\pgfpathlineto{\pgfqpoint{1.835077in}{2.346957in}}%
\pgfpathlineto{\pgfqpoint{1.878884in}{2.367380in}}%
\pgfpathlineto{\pgfqpoint{1.922692in}{2.383567in}}%
\pgfpathlineto{\pgfqpoint{1.966499in}{2.395529in}}%
\pgfpathlineto{\pgfqpoint{2.010307in}{2.403277in}}%
\pgfpathlineto{\pgfqpoint{2.054114in}{2.406820in}}%
\pgfpathlineto{\pgfqpoint{2.097922in}{2.406163in}}%
\pgfpathlineto{\pgfqpoint{2.141729in}{2.401310in}}%
\pgfpathlineto{\pgfqpoint{2.185537in}{2.392259in}}%
\pgfpathlineto{\pgfqpoint{2.229344in}{2.379005in}}%
\pgfpathlineto{\pgfqpoint{2.273152in}{2.361536in}}%
\pgfpathlineto{\pgfqpoint{2.316959in}{2.339837in}}%
\pgfpathlineto{\pgfqpoint{2.360767in}{2.313886in}}%
\pgfpathlineto{\pgfqpoint{2.404574in}{2.283652in}}%
\pgfpathlineto{\pgfqpoint{2.448382in}{2.249093in}}%
\pgfpathlineto{\pgfqpoint{2.492189in}{2.210162in}}%
\pgfpathlineto{\pgfqpoint{2.535997in}{2.166798in}}%
\pgfpathlineto{\pgfqpoint{2.579804in}{2.118935in}}%
\pgfpathlineto{\pgfqpoint{2.623611in}{2.066498in}}%
\pgfpathlineto{\pgfqpoint{2.667419in}{2.009403in}}%
\pgfpathlineto{\pgfqpoint{2.711226in}{1.947561in}}%
\pgfpathlineto{\pgfqpoint{2.755034in}{1.880877in}}%
\pgfpathlineto{\pgfqpoint{2.798841in}{1.809275in}}%
\pgfpathlineto{\pgfqpoint{2.842649in}{1.732590in}}%
\pgfpathlineto{\pgfqpoint{2.886456in}{1.650724in}}%
\pgfpathlineto{\pgfqpoint{2.930264in}{1.563579in}}%
\pgfpathlineto{\pgfqpoint{2.974071in}{1.471057in}}%
\pgfpathlineto{\pgfqpoint{3.017879in}{1.373060in}}%
\pgfpathlineto{\pgfqpoint{3.061686in}{1.269489in}}%
\pgfpathlineto{\pgfqpoint{3.105494in}{1.160248in}}%
\pgfpathlineto{\pgfqpoint{3.149301in}{1.045237in}}%
\pgfpathlineto{\pgfqpoint{3.193109in}{0.924360in}}%
\pgfpathlineto{\pgfqpoint{3.236916in}{0.797517in}}%
\pgfusepath{stroke}%
\end{pgfscope}%
\begin{pgfscope}%
\pgfpathrectangle{\pgfqpoint{0.564660in}{0.521603in}}{\pgfqpoint{3.720000in}{3.020000in}}%
\pgfusepath{clip}%
\pgfsetrectcap%
\pgfsetroundjoin%
\pgfsetlinewidth{1.505625pt}%
\definecolor{currentstroke}{rgb}{0.131172,0.555899,0.552459}%
\pgfsetstrokecolor{currentstroke}%
\pgfsetdash{}{0pt}%
\pgfpathmoveto{\pgfqpoint{0.946461in}{0.834509in}}%
\pgfpathlineto{\pgfqpoint{0.988883in}{0.934924in}}%
\pgfpathlineto{\pgfqpoint{1.031305in}{1.030942in}}%
\pgfpathlineto{\pgfqpoint{1.073728in}{1.122608in}}%
\pgfpathlineto{\pgfqpoint{1.116150in}{1.209968in}}%
\pgfpathlineto{\pgfqpoint{1.158572in}{1.293067in}}%
\pgfpathlineto{\pgfqpoint{1.200994in}{1.371953in}}%
\pgfpathlineto{\pgfqpoint{1.243417in}{1.446670in}}%
\pgfpathlineto{\pgfqpoint{1.285839in}{1.517264in}}%
\pgfpathlineto{\pgfqpoint{1.328261in}{1.583780in}}%
\pgfpathlineto{\pgfqpoint{1.370683in}{1.646266in}}%
\pgfpathlineto{\pgfqpoint{1.413106in}{1.704766in}}%
\pgfpathlineto{\pgfqpoint{1.455528in}{1.759326in}}%
\pgfpathlineto{\pgfqpoint{1.497950in}{1.810015in}}%
\pgfpathlineto{\pgfqpoint{1.540372in}{1.856870in}}%
\pgfpathlineto{\pgfqpoint{1.582795in}{1.899937in}}%
\pgfpathlineto{\pgfqpoint{1.625217in}{1.939262in}}%
\pgfpathlineto{\pgfqpoint{1.667639in}{1.974887in}}%
\pgfpathlineto{\pgfqpoint{1.710062in}{2.006852in}}%
\pgfpathlineto{\pgfqpoint{1.752484in}{2.035195in}}%
\pgfpathlineto{\pgfqpoint{1.794906in}{2.059948in}}%
\pgfpathlineto{\pgfqpoint{1.837328in}{2.081141in}}%
\pgfpathlineto{\pgfqpoint{1.879751in}{2.098802in}}%
\pgfpathlineto{\pgfqpoint{1.922173in}{2.112950in}}%
\pgfpathlineto{\pgfqpoint{1.964595in}{2.123602in}}%
\pgfpathlineto{\pgfqpoint{2.007017in}{2.130770in}}%
\pgfpathlineto{\pgfqpoint{2.049440in}{2.134462in}}%
\pgfpathlineto{\pgfqpoint{2.091862in}{2.134679in}}%
\pgfpathlineto{\pgfqpoint{2.134284in}{2.131418in}}%
\pgfpathlineto{\pgfqpoint{2.176706in}{2.124673in}}%
\pgfpathlineto{\pgfqpoint{2.219129in}{2.114430in}}%
\pgfpathlineto{\pgfqpoint{2.261551in}{2.100668in}}%
\pgfpathlineto{\pgfqpoint{2.303973in}{2.083364in}}%
\pgfpathlineto{\pgfqpoint{2.346396in}{2.062487in}}%
\pgfpathlineto{\pgfqpoint{2.388818in}{2.037999in}}%
\pgfpathlineto{\pgfqpoint{2.431240in}{2.009850in}}%
\pgfpathlineto{\pgfqpoint{2.473662in}{1.977985in}}%
\pgfpathlineto{\pgfqpoint{2.516085in}{1.942336in}}%
\pgfpathlineto{\pgfqpoint{2.558507in}{1.902831in}}%
\pgfpathlineto{\pgfqpoint{2.600929in}{1.859388in}}%
\pgfpathlineto{\pgfqpoint{2.643351in}{1.811918in}}%
\pgfpathlineto{\pgfqpoint{2.685774in}{1.760326in}}%
\pgfpathlineto{\pgfqpoint{2.728196in}{1.704511in}}%
\pgfpathlineto{\pgfqpoint{2.770618in}{1.644390in}}%
\pgfpathlineto{\pgfqpoint{2.813041in}{1.579793in}}%
\pgfpathlineto{\pgfqpoint{2.855463in}{1.510617in}}%
\pgfpathlineto{\pgfqpoint{2.897885in}{1.436756in}}%
\pgfpathlineto{\pgfqpoint{2.940307in}{1.358108in}}%
\pgfpathlineto{\pgfqpoint{2.982730in}{1.274568in}}%
\pgfpathlineto{\pgfqpoint{3.025152in}{1.186032in}}%
\pgfpathlineto{\pgfqpoint{3.067574in}{1.092395in}}%
\pgfpathlineto{\pgfqpoint{3.109996in}{0.993555in}}%
\pgfpathlineto{\pgfqpoint{3.152419in}{0.889407in}}%
\pgfpathlineto{\pgfqpoint{3.194841in}{0.779846in}}%
\pgfusepath{stroke}%
\end{pgfscope}%
\begin{pgfscope}%
\pgfpathrectangle{\pgfqpoint{0.564660in}{0.521603in}}{\pgfqpoint{3.720000in}{3.020000in}}%
\pgfusepath{clip}%
\pgfsetrectcap%
\pgfsetroundjoin%
\pgfsetlinewidth{1.505625pt}%
\definecolor{currentstroke}{rgb}{0.192357,0.403199,0.555836}%
\pgfsetstrokecolor{currentstroke}%
\pgfsetdash{}{0pt}%
\pgfpathmoveto{\pgfqpoint{0.880673in}{1.105364in}}%
\pgfpathlineto{\pgfqpoint{0.920175in}{1.199183in}}%
\pgfpathlineto{\pgfqpoint{0.959676in}{1.289171in}}%
\pgfpathlineto{\pgfqpoint{0.999178in}{1.375366in}}%
\pgfpathlineto{\pgfqpoint{1.038679in}{1.457803in}}%
\pgfpathlineto{\pgfqpoint{1.078181in}{1.536521in}}%
\pgfpathlineto{\pgfqpoint{1.117683in}{1.611554in}}%
\pgfpathlineto{\pgfqpoint{1.157184in}{1.682940in}}%
\pgfpathlineto{\pgfqpoint{1.196686in}{1.750715in}}%
\pgfpathlineto{\pgfqpoint{1.236187in}{1.814915in}}%
\pgfpathlineto{\pgfqpoint{1.275689in}{1.875578in}}%
\pgfpathlineto{\pgfqpoint{1.315191in}{1.932740in}}%
\pgfpathlineto{\pgfqpoint{1.354692in}{1.986437in}}%
\pgfpathlineto{\pgfqpoint{1.394194in}{2.036717in}}%
\pgfpathlineto{\pgfqpoint{1.433695in}{2.083612in}}%
\pgfpathlineto{\pgfqpoint{1.473197in}{2.127160in}}%
\pgfpathlineto{\pgfqpoint{1.512698in}{2.167399in}}%
\pgfpathlineto{\pgfqpoint{1.552200in}{2.204364in}}%
\pgfpathlineto{\pgfqpoint{1.591702in}{2.238089in}}%
\pgfpathlineto{\pgfqpoint{1.631203in}{2.268605in}}%
\pgfpathlineto{\pgfqpoint{1.670705in}{2.295944in}}%
\pgfpathlineto{\pgfqpoint{1.710206in}{2.320132in}}%
\pgfpathlineto{\pgfqpoint{1.749708in}{2.341194in}}%
\pgfpathlineto{\pgfqpoint{1.789210in}{2.359153in}}%
\pgfpathlineto{\pgfqpoint{1.828711in}{2.374028in}}%
\pgfpathlineto{\pgfqpoint{1.868213in}{2.385835in}}%
\pgfpathlineto{\pgfqpoint{1.907714in}{2.394588in}}%
\pgfpathlineto{\pgfqpoint{1.947216in}{2.400298in}}%
\pgfpathlineto{\pgfqpoint{1.986718in}{2.402974in}}%
\pgfpathlineto{\pgfqpoint{2.026219in}{2.402622in}}%
\pgfpathlineto{\pgfqpoint{2.065721in}{2.399247in}}%
\pgfpathlineto{\pgfqpoint{2.105222in}{2.392852in}}%
\pgfpathlineto{\pgfqpoint{2.144724in}{2.383435in}}%
\pgfpathlineto{\pgfqpoint{2.184226in}{2.370997in}}%
\pgfpathlineto{\pgfqpoint{2.223727in}{2.355533in}}%
\pgfpathlineto{\pgfqpoint{2.263229in}{2.337035in}}%
\pgfpathlineto{\pgfqpoint{2.302730in}{2.315494in}}%
\pgfpathlineto{\pgfqpoint{2.342232in}{2.290894in}}%
\pgfpathlineto{\pgfqpoint{2.381734in}{2.263219in}}%
\pgfpathlineto{\pgfqpoint{2.421235in}{2.232447in}}%
\pgfpathlineto{\pgfqpoint{2.460737in}{2.198548in}}%
\pgfpathlineto{\pgfqpoint{2.500238in}{2.161482in}}%
\pgfpathlineto{\pgfqpoint{2.539740in}{2.121202in}}%
\pgfpathlineto{\pgfqpoint{2.579241in}{2.077655in}}%
\pgfpathlineto{\pgfqpoint{2.618743in}{2.030778in}}%
\pgfpathlineto{\pgfqpoint{2.658245in}{1.980499in}}%
\pgfpathlineto{\pgfqpoint{2.697746in}{1.926742in}}%
\pgfpathlineto{\pgfqpoint{2.737248in}{1.869423in}}%
\pgfpathlineto{\pgfqpoint{2.776749in}{1.808450in}}%
\pgfpathlineto{\pgfqpoint{2.816251in}{1.743754in}}%
\pgfpathlineto{\pgfqpoint{2.855753in}{1.675176in}}%
\pgfpathlineto{\pgfqpoint{2.895254in}{1.602623in}}%
\pgfpathlineto{\pgfqpoint{2.934756in}{1.526002in}}%
\pgfpathlineto{\pgfqpoint{2.974257in}{1.445220in}}%
\pgfpathlineto{\pgfqpoint{3.013759in}{1.360184in}}%
\pgfpathlineto{\pgfqpoint{3.053261in}{1.270802in}}%
\pgfpathlineto{\pgfqpoint{3.092762in}{1.176979in}}%
\pgfpathlineto{\pgfqpoint{3.132264in}{1.078623in}}%
\pgfpathlineto{\pgfqpoint{3.171765in}{0.975641in}}%
\pgfpathlineto{\pgfqpoint{3.211267in}{0.867940in}}%
\pgfpathlineto{\pgfqpoint{3.250769in}{0.755427in}}%
\pgfusepath{stroke}%
\end{pgfscope}%
\begin{pgfscope}%
\pgfpathrectangle{\pgfqpoint{0.564660in}{0.521603in}}{\pgfqpoint{3.720000in}{3.020000in}}%
\pgfusepath{clip}%
\pgfsetrectcap%
\pgfsetroundjoin%
\pgfsetlinewidth{1.505625pt}%
\definecolor{currentstroke}{rgb}{0.204903,0.375746,0.553533}%
\pgfsetstrokecolor{currentstroke}%
\pgfsetdash{}{0pt}%
\pgfpathmoveto{\pgfqpoint{0.777183in}{0.903667in}}%
\pgfpathlineto{\pgfqpoint{0.824410in}{1.028543in}}%
\pgfpathlineto{\pgfqpoint{0.871637in}{1.144664in}}%
\pgfpathlineto{\pgfqpoint{0.918864in}{1.252614in}}%
\pgfpathlineto{\pgfqpoint{0.966091in}{1.352977in}}%
\pgfpathlineto{\pgfqpoint{1.013319in}{1.446337in}}%
\pgfpathlineto{\pgfqpoint{1.060546in}{1.533277in}}%
\pgfpathlineto{\pgfqpoint{1.107773in}{1.614526in}}%
\pgfpathlineto{\pgfqpoint{1.155000in}{1.690645in}}%
\pgfpathlineto{\pgfqpoint{1.202227in}{1.762123in}}%
\pgfpathlineto{\pgfqpoint{1.249455in}{1.829368in}}%
\pgfpathlineto{\pgfqpoint{1.296682in}{1.892688in}}%
\pgfpathlineto{\pgfqpoint{1.343909in}{1.952300in}}%
\pgfpathlineto{\pgfqpoint{1.391136in}{2.008353in}}%
\pgfpathlineto{\pgfqpoint{1.438363in}{2.060952in}}%
\pgfpathlineto{\pgfqpoint{1.485591in}{2.110180in}}%
\pgfpathlineto{\pgfqpoint{1.532818in}{2.156125in}}%
\pgfpathlineto{\pgfqpoint{1.580045in}{2.198865in}}%
\pgfpathlineto{\pgfqpoint{1.627272in}{2.238491in}}%
\pgfpathlineto{\pgfqpoint{1.674499in}{2.275099in}}%
\pgfpathlineto{\pgfqpoint{1.721727in}{2.308759in}}%
\pgfpathlineto{\pgfqpoint{1.768954in}{2.339537in}}%
\pgfpathlineto{\pgfqpoint{1.816181in}{2.367501in}}%
\pgfpathlineto{\pgfqpoint{1.863408in}{2.392699in}}%
\pgfpathlineto{\pgfqpoint{1.910635in}{2.415187in}}%
\pgfpathlineto{\pgfqpoint{1.957863in}{2.435022in}}%
\pgfpathlineto{\pgfqpoint{2.005090in}{2.452236in}}%
\pgfpathlineto{\pgfqpoint{2.052317in}{2.466867in}}%
\pgfpathlineto{\pgfqpoint{2.099544in}{2.478946in}}%
\pgfpathlineto{\pgfqpoint{2.146771in}{2.488493in}}%
\pgfpathlineto{\pgfqpoint{2.193998in}{2.495539in}}%
\pgfpathlineto{\pgfqpoint{2.241226in}{2.500111in}}%
\pgfpathlineto{\pgfqpoint{2.288453in}{2.502214in}}%
\pgfpathlineto{\pgfqpoint{2.335680in}{2.501856in}}%
\pgfpathlineto{\pgfqpoint{2.382907in}{2.499038in}}%
\pgfpathlineto{\pgfqpoint{2.430134in}{2.493748in}}%
\pgfpathlineto{\pgfqpoint{2.477362in}{2.485987in}}%
\pgfpathlineto{\pgfqpoint{2.524589in}{2.475751in}}%
\pgfpathlineto{\pgfqpoint{2.571816in}{2.463015in}}%
\pgfpathlineto{\pgfqpoint{2.619043in}{2.447756in}}%
\pgfpathlineto{\pgfqpoint{2.666270in}{2.429945in}}%
\pgfpathlineto{\pgfqpoint{2.713498in}{2.409538in}}%
\pgfpathlineto{\pgfqpoint{2.760725in}{2.386510in}}%
\pgfpathlineto{\pgfqpoint{2.807952in}{2.360822in}}%
\pgfpathlineto{\pgfqpoint{2.855179in}{2.332412in}}%
\pgfpathlineto{\pgfqpoint{2.902406in}{2.301223in}}%
\pgfpathlineto{\pgfqpoint{2.949634in}{2.267176in}}%
\pgfpathlineto{\pgfqpoint{2.996861in}{2.230185in}}%
\pgfpathlineto{\pgfqpoint{3.044088in}{2.190174in}}%
\pgfpathlineto{\pgfqpoint{3.091315in}{2.147045in}}%
\pgfpathlineto{\pgfqpoint{3.138542in}{2.100688in}}%
\pgfpathlineto{\pgfqpoint{3.185770in}{2.050993in}}%
\pgfpathlineto{\pgfqpoint{3.232997in}{1.997837in}}%
\pgfpathlineto{\pgfqpoint{3.280224in}{1.941090in}}%
\pgfpathlineto{\pgfqpoint{3.327451in}{1.880616in}}%
\pgfpathlineto{\pgfqpoint{3.374678in}{1.816238in}}%
\pgfpathlineto{\pgfqpoint{3.421906in}{1.747737in}}%
\pgfpathlineto{\pgfqpoint{3.469133in}{1.674840in}}%
\pgfpathlineto{\pgfqpoint{3.516360in}{1.597184in}}%
\pgfpathlineto{\pgfqpoint{3.563587in}{1.514330in}}%
\pgfpathlineto{\pgfqpoint{3.610814in}{1.425765in}}%
\pgfpathlineto{\pgfqpoint{3.658041in}{1.330913in}}%
\pgfpathlineto{\pgfqpoint{3.705269in}{1.229080in}}%
\pgfpathlineto{\pgfqpoint{3.752496in}{1.119610in}}%
\pgfpathlineto{\pgfqpoint{3.799723in}{1.001890in}}%
\pgfpathlineto{\pgfqpoint{3.846950in}{0.875303in}}%
\pgfpathlineto{\pgfqpoint{3.894177in}{0.739236in}}%
\pgfpathlineto{\pgfqpoint{3.894177in}{0.739236in}}%
\pgfusepath{stroke}%
\end{pgfscope}%
\begin{pgfscope}%
\pgfpathrectangle{\pgfqpoint{0.564660in}{0.521603in}}{\pgfqpoint{3.720000in}{3.020000in}}%
\pgfusepath{clip}%
\pgfsetrectcap%
\pgfsetroundjoin%
\pgfsetlinewidth{1.505625pt}%
\definecolor{currentstroke}{rgb}{0.206756,0.371758,0.553117}%
\pgfsetstrokecolor{currentstroke}%
\pgfsetdash{}{0pt}%
\pgfpathmoveto{\pgfqpoint{0.760917in}{0.842885in}}%
\pgfpathlineto{\pgfqpoint{0.788954in}{0.915650in}}%
\pgfpathlineto{\pgfqpoint{0.816990in}{0.986020in}}%
\pgfpathlineto{\pgfqpoint{0.845027in}{1.054055in}}%
\pgfpathlineto{\pgfqpoint{0.873064in}{1.119814in}}%
\pgfpathlineto{\pgfqpoint{0.901100in}{1.183358in}}%
\pgfpathlineto{\pgfqpoint{0.929137in}{1.244745in}}%
\pgfpathlineto{\pgfqpoint{0.957174in}{1.304035in}}%
\pgfpathlineto{\pgfqpoint{0.985210in}{1.361287in}}%
\pgfpathlineto{\pgfqpoint{1.013247in}{1.416561in}}%
\pgfpathlineto{\pgfqpoint{1.041283in}{1.469916in}}%
\pgfpathlineto{\pgfqpoint{1.069320in}{1.521412in}}%
\pgfpathlineto{\pgfqpoint{1.097357in}{1.571108in}}%
\pgfpathlineto{\pgfqpoint{1.125393in}{1.619107in}}%
\pgfpathlineto{\pgfqpoint{1.153430in}{1.665454in}}%
\pgfpathlineto{\pgfqpoint{1.181467in}{1.710206in}}%
\pgfpathlineto{\pgfqpoint{1.209503in}{1.753415in}}%
\pgfpathlineto{\pgfqpoint{1.237540in}{1.795130in}}%
\pgfpathlineto{\pgfqpoint{1.265577in}{1.835393in}}%
\pgfpathlineto{\pgfqpoint{1.293613in}{1.874245in}}%
\pgfpathlineto{\pgfqpoint{1.321650in}{1.911718in}}%
\pgfpathlineto{\pgfqpoint{1.349687in}{1.947842in}}%
\pgfpathlineto{\pgfqpoint{1.377723in}{1.982638in}}%
\pgfpathlineto{\pgfqpoint{1.405760in}{2.016128in}}%
\pgfpathlineto{\pgfqpoint{1.433797in}{2.048325in}}%
\pgfpathlineto{\pgfqpoint{1.461833in}{2.079241in}}%
\pgfpathlineto{\pgfqpoint{1.489870in}{2.108889in}}%
\pgfpathlineto{\pgfqpoint{1.517907in}{2.137277in}}%
\pgfpathlineto{\pgfqpoint{1.545943in}{2.164419in}}%
\pgfpathlineto{\pgfqpoint{1.573980in}{2.190323in}}%
\pgfpathlineto{\pgfqpoint{1.602017in}{2.215000in}}%
\pgfpathlineto{\pgfqpoint{1.630053in}{2.238457in}}%
\pgfpathlineto{\pgfqpoint{1.658090in}{2.260704in}}%
\pgfpathlineto{\pgfqpoint{1.686127in}{2.281750in}}%
\pgfpathlineto{\pgfqpoint{1.714163in}{2.301607in}}%
\pgfpathlineto{\pgfqpoint{1.742200in}{2.320281in}}%
\pgfpathlineto{\pgfqpoint{1.770237in}{2.337783in}}%
\pgfpathlineto{\pgfqpoint{1.798273in}{2.354122in}}%
\pgfpathlineto{\pgfqpoint{1.826310in}{2.369308in}}%
\pgfpathlineto{\pgfqpoint{1.854347in}{2.383351in}}%
\pgfpathlineto{\pgfqpoint{1.882383in}{2.396257in}}%
\pgfpathlineto{\pgfqpoint{1.910420in}{2.408035in}}%
\pgfpathlineto{\pgfqpoint{1.938457in}{2.418693in}}%
\pgfpathlineto{\pgfqpoint{1.966493in}{2.428238in}}%
\pgfpathlineto{\pgfqpoint{1.994530in}{2.436674in}}%
\pgfpathlineto{\pgfqpoint{2.022567in}{2.444005in}}%
\pgfpathlineto{\pgfqpoint{2.050603in}{2.450235in}}%
\pgfpathlineto{\pgfqpoint{2.078640in}{2.455370in}}%
\pgfpathlineto{\pgfqpoint{2.106677in}{2.459411in}}%
\pgfpathlineto{\pgfqpoint{2.134713in}{2.462360in}}%
\pgfpathlineto{\pgfqpoint{2.162750in}{2.464218in}}%
\pgfpathlineto{\pgfqpoint{2.190787in}{2.464987in}}%
\pgfpathlineto{\pgfqpoint{2.218823in}{2.464668in}}%
\pgfpathlineto{\pgfqpoint{2.246860in}{2.463261in}}%
\pgfpathlineto{\pgfqpoint{2.274897in}{2.460761in}}%
\pgfpathlineto{\pgfqpoint{2.302933in}{2.457169in}}%
\pgfpathlineto{\pgfqpoint{2.330970in}{2.452482in}}%
\pgfpathlineto{\pgfqpoint{2.359006in}{2.446698in}}%
\pgfpathlineto{\pgfqpoint{2.387043in}{2.439814in}}%
\pgfpathlineto{\pgfqpoint{2.415080in}{2.431826in}}%
\pgfpathlineto{\pgfqpoint{2.443116in}{2.422731in}}%
\pgfpathlineto{\pgfqpoint{2.471153in}{2.412527in}}%
\pgfpathlineto{\pgfqpoint{2.499190in}{2.401211in}}%
\pgfpathlineto{\pgfqpoint{2.527226in}{2.388777in}}%
\pgfpathlineto{\pgfqpoint{2.555263in}{2.375221in}}%
\pgfpathlineto{\pgfqpoint{2.583300in}{2.360538in}}%
\pgfpathlineto{\pgfqpoint{2.611336in}{2.344725in}}%
\pgfpathlineto{\pgfqpoint{2.639373in}{2.327775in}}%
\pgfpathlineto{\pgfqpoint{2.667410in}{2.309681in}}%
\pgfpathlineto{\pgfqpoint{2.695446in}{2.290436in}}%
\pgfpathlineto{\pgfqpoint{2.723483in}{2.270033in}}%
\pgfpathlineto{\pgfqpoint{2.751520in}{2.248463in}}%
\pgfpathlineto{\pgfqpoint{2.779556in}{2.225716in}}%
\pgfpathlineto{\pgfqpoint{2.807593in}{2.201781in}}%
\pgfpathlineto{\pgfqpoint{2.835630in}{2.176646in}}%
\pgfpathlineto{\pgfqpoint{2.863666in}{2.150300in}}%
\pgfpathlineto{\pgfqpoint{2.891703in}{2.122729in}}%
\pgfpathlineto{\pgfqpoint{2.919740in}{2.093919in}}%
\pgfpathlineto{\pgfqpoint{2.947776in}{2.063854in}}%
\pgfpathlineto{\pgfqpoint{2.975813in}{2.032517in}}%
\pgfpathlineto{\pgfqpoint{3.003850in}{1.999893in}}%
\pgfpathlineto{\pgfqpoint{3.031886in}{1.965961in}}%
\pgfpathlineto{\pgfqpoint{3.059923in}{1.930700in}}%
\pgfpathlineto{\pgfqpoint{3.087960in}{1.894087in}}%
\pgfpathlineto{\pgfqpoint{3.115996in}{1.856099in}}%
\pgfpathlineto{\pgfqpoint{3.144033in}{1.816708in}}%
\pgfpathlineto{\pgfqpoint{3.172070in}{1.775884in}}%
\pgfpathlineto{\pgfqpoint{3.200106in}{1.733591in}}%
\pgfpathlineto{\pgfqpoint{3.228143in}{1.689794in}}%
\pgfpathlineto{\pgfqpoint{3.256180in}{1.644451in}}%
\pgfpathlineto{\pgfqpoint{3.284216in}{1.597521in}}%
\pgfpathlineto{\pgfqpoint{3.312253in}{1.548956in}}%
\pgfpathlineto{\pgfqpoint{3.340290in}{1.498708in}}%
\pgfpathlineto{\pgfqpoint{3.368326in}{1.446725in}}%
\pgfpathlineto{\pgfqpoint{3.396363in}{1.392964in}}%
\pgfpathlineto{\pgfqpoint{3.424400in}{1.337346in}}%
\pgfpathlineto{\pgfqpoint{3.452436in}{1.279820in}}%
\pgfpathlineto{\pgfqpoint{3.480473in}{1.220332in}}%
\pgfpathlineto{\pgfqpoint{3.508510in}{1.158830in}}%
\pgfpathlineto{\pgfqpoint{3.536546in}{1.095262in}}%
\pgfpathlineto{\pgfqpoint{3.564583in}{1.029575in}}%
\pgfpathlineto{\pgfqpoint{3.592620in}{0.961718in}}%
\pgfpathlineto{\pgfqpoint{3.620656in}{0.891637in}}%
\pgfpathlineto{\pgfqpoint{3.648693in}{0.819280in}}%
\pgfpathlineto{\pgfqpoint{3.676729in}{0.744596in}}%
\pgfusepath{stroke}%
\end{pgfscope}%
\begin{pgfscope}%
\pgfpathrectangle{\pgfqpoint{0.564660in}{0.521603in}}{\pgfqpoint{3.720000in}{3.020000in}}%
\pgfusepath{clip}%
\pgfsetrectcap%
\pgfsetroundjoin%
\pgfsetlinewidth{1.505625pt}%
\definecolor{currentstroke}{rgb}{0.282656,0.100196,0.422160}%
\pgfsetstrokecolor{currentstroke}%
\pgfsetdash{}{0pt}%
\pgfpathmoveto{\pgfqpoint{0.712990in}{0.794843in}}%
\pgfpathlineto{\pgfqpoint{0.768614in}{0.924476in}}%
\pgfpathlineto{\pgfqpoint{0.824238in}{1.046999in}}%
\pgfpathlineto{\pgfqpoint{0.879862in}{1.162599in}}%
\pgfpathlineto{\pgfqpoint{0.935485in}{1.271466in}}%
\pgfpathlineto{\pgfqpoint{0.991109in}{1.373802in}}%
\pgfpathlineto{\pgfqpoint{1.046733in}{1.469785in}}%
\pgfpathlineto{\pgfqpoint{1.102357in}{1.559566in}}%
\pgfpathlineto{\pgfqpoint{1.157980in}{1.643285in}}%
\pgfpathlineto{\pgfqpoint{1.213604in}{1.721089in}}%
\pgfpathlineto{\pgfqpoint{1.269228in}{1.793159in}}%
\pgfpathlineto{\pgfqpoint{1.324852in}{1.859702in}}%
\pgfpathlineto{\pgfqpoint{1.380475in}{1.920924in}}%
\pgfpathlineto{\pgfqpoint{1.436099in}{1.977014in}}%
\pgfpathlineto{\pgfqpoint{1.491723in}{2.028122in}}%
\pgfpathlineto{\pgfqpoint{1.547347in}{2.074359in}}%
\pgfpathlineto{\pgfqpoint{1.602970in}{2.115802in}}%
\pgfpathlineto{\pgfqpoint{1.658594in}{2.152519in}}%
\pgfpathlineto{\pgfqpoint{1.714218in}{2.184578in}}%
\pgfpathlineto{\pgfqpoint{1.769842in}{2.212052in}}%
\pgfpathlineto{\pgfqpoint{1.825465in}{2.235017in}}%
\pgfpathlineto{\pgfqpoint{1.881089in}{2.253552in}}%
\pgfpathlineto{\pgfqpoint{1.936713in}{2.267724in}}%
\pgfpathlineto{\pgfqpoint{1.992337in}{2.277577in}}%
\pgfpathlineto{\pgfqpoint{2.047960in}{2.283134in}}%
\pgfpathlineto{\pgfqpoint{2.103584in}{2.284398in}}%
\pgfpathlineto{\pgfqpoint{2.159208in}{2.281353in}}%
\pgfpathlineto{\pgfqpoint{2.214832in}{2.273974in}}%
\pgfpathlineto{\pgfqpoint{2.270455in}{2.262231in}}%
\pgfpathlineto{\pgfqpoint{2.326079in}{2.246084in}}%
\pgfpathlineto{\pgfqpoint{2.381703in}{2.225476in}}%
\pgfpathlineto{\pgfqpoint{2.437327in}{2.200336in}}%
\pgfpathlineto{\pgfqpoint{2.492950in}{2.170586in}}%
\pgfpathlineto{\pgfqpoint{2.548574in}{2.136148in}}%
\pgfpathlineto{\pgfqpoint{2.604198in}{2.096952in}}%
\pgfpathlineto{\pgfqpoint{2.659821in}{2.052936in}}%
\pgfpathlineto{\pgfqpoint{2.715445in}{2.004040in}}%
\pgfpathlineto{\pgfqpoint{2.771069in}{1.950187in}}%
\pgfpathlineto{\pgfqpoint{2.826693in}{1.891268in}}%
\pgfpathlineto{\pgfqpoint{2.882316in}{1.827141in}}%
\pgfpathlineto{\pgfqpoint{2.937940in}{1.757638in}}%
\pgfpathlineto{\pgfqpoint{2.993564in}{1.682578in}}%
\pgfpathlineto{\pgfqpoint{3.049188in}{1.601776in}}%
\pgfpathlineto{\pgfqpoint{3.104811in}{1.515043in}}%
\pgfpathlineto{\pgfqpoint{3.160435in}{1.422172in}}%
\pgfpathlineto{\pgfqpoint{3.216059in}{1.322907in}}%
\pgfpathlineto{\pgfqpoint{3.271683in}{1.216926in}}%
\pgfpathlineto{\pgfqpoint{3.327306in}{1.103849in}}%
\pgfpathlineto{\pgfqpoint{3.382930in}{0.983232in}}%
\pgfpathlineto{\pgfqpoint{3.438554in}{0.854656in}}%
\pgfpathlineto{\pgfqpoint{3.494178in}{0.717710in}}%
\pgfpathlineto{\pgfqpoint{3.512719in}{0.670130in}}%
\pgfpathlineto{\pgfqpoint{3.512719in}{0.670130in}}%
\pgfusepath{stroke}%
\end{pgfscope}%
\begin{pgfscope}%
\pgfsetrectcap%
\pgfsetmiterjoin%
\pgfsetlinewidth{0.803000pt}%
\definecolor{currentstroke}{rgb}{0.501961,0.501961,0.501961}%
\pgfsetstrokecolor{currentstroke}%
\pgfsetdash{}{0pt}%
\pgfpathmoveto{\pgfqpoint{0.564660in}{0.521603in}}%
\pgfpathlineto{\pgfqpoint{0.564660in}{3.541603in}}%
\pgfusepath{stroke}%
\end{pgfscope}%
\begin{pgfscope}%
\pgfsetrectcap%
\pgfsetmiterjoin%
\pgfsetlinewidth{0.803000pt}%
\definecolor{currentstroke}{rgb}{0.501961,0.501961,0.501961}%
\pgfsetstrokecolor{currentstroke}%
\pgfsetdash{}{0pt}%
\pgfpathmoveto{\pgfqpoint{4.284660in}{0.521603in}}%
\pgfpathlineto{\pgfqpoint{4.284660in}{3.541603in}}%
\pgfusepath{stroke}%
\end{pgfscope}%
\begin{pgfscope}%
\pgfsetrectcap%
\pgfsetmiterjoin%
\pgfsetlinewidth{0.803000pt}%
\definecolor{currentstroke}{rgb}{0.501961,0.501961,0.501961}%
\pgfsetstrokecolor{currentstroke}%
\pgfsetdash{}{0pt}%
\pgfpathmoveto{\pgfqpoint{0.564660in}{0.521603in}}%
\pgfpathlineto{\pgfqpoint{4.284660in}{0.521603in}}%
\pgfusepath{stroke}%
\end{pgfscope}%
\begin{pgfscope}%
\pgfsetrectcap%
\pgfsetmiterjoin%
\pgfsetlinewidth{0.803000pt}%
\definecolor{currentstroke}{rgb}{0.501961,0.501961,0.501961}%
\pgfsetstrokecolor{currentstroke}%
\pgfsetdash{}{0pt}%
\pgfpathmoveto{\pgfqpoint{0.564660in}{3.541603in}}%
\pgfpathlineto{\pgfqpoint{4.284660in}{3.541603in}}%
\pgfusepath{stroke}%
\end{pgfscope}%
\begin{pgfscope}%
\pgfpathrectangle{\pgfqpoint{4.517160in}{0.521603in}}{\pgfqpoint{0.151000in}{3.020000in}}%
\pgfusepath{clip}%
\pgfsetbuttcap%
\pgfsetmiterjoin%
\definecolor{currentfill}{rgb}{1.000000,1.000000,1.000000}%
\pgfsetfillcolor{currentfill}%
\pgfsetlinewidth{0.010037pt}%
\definecolor{currentstroke}{rgb}{1.000000,1.000000,1.000000}%
\pgfsetstrokecolor{currentstroke}%
\pgfsetdash{}{0pt}%
\pgfpathmoveto{\pgfqpoint{4.517160in}{0.521603in}}%
\pgfpathlineto{\pgfqpoint{4.517160in}{0.533400in}}%
\pgfpathlineto{\pgfqpoint{4.517160in}{3.529806in}}%
\pgfpathlineto{\pgfqpoint{4.517160in}{3.541603in}}%
\pgfpathlineto{\pgfqpoint{4.668160in}{3.541603in}}%
\pgfpathlineto{\pgfqpoint{4.668160in}{3.529806in}}%
\pgfpathlineto{\pgfqpoint{4.668160in}{0.533400in}}%
\pgfpathlineto{\pgfqpoint{4.668160in}{0.521603in}}%
\pgfpathclose%
\pgfusepath{stroke,fill}%
\end{pgfscope}%
\begin{pgfscope}%
\pgfsys@transformshift{4.513889in}{0.537437in}%
\pgftext[left,bottom]{\pgfimage[interpolate=true,width=0.152778in,height=3.013889in]{series_s_ds-img0.png}}%
\end{pgfscope}%
\begin{pgfscope}%
\pgfsetbuttcap%
\pgfsetroundjoin%
\definecolor{currentfill}{rgb}{0.000000,0.000000,0.000000}%
\pgfsetfillcolor{currentfill}%
\pgfsetlinewidth{0.803000pt}%
\definecolor{currentstroke}{rgb}{0.000000,0.000000,0.000000}%
\pgfsetstrokecolor{currentstroke}%
\pgfsetdash{}{0pt}%
\pgfsys@defobject{currentmarker}{\pgfqpoint{0.000000in}{0.000000in}}{\pgfqpoint{0.048611in}{0.000000in}}{%
\pgfpathmoveto{\pgfqpoint{0.000000in}{0.000000in}}%
\pgfpathlineto{\pgfqpoint{0.048611in}{0.000000in}}%
\pgfusepath{stroke,fill}%
}%
\begin{pgfscope}%
\pgfsys@transformshift{4.668160in}{0.687294in}%
\pgfsys@useobject{currentmarker}{}%
\end{pgfscope}%
\end{pgfscope}%
\begin{pgfscope}%
\definecolor{textcolor}{rgb}{0.000000,0.000000,0.000000}%
\pgfsetstrokecolor{textcolor}%
\pgfsetfillcolor{textcolor}%
\pgftext[x=4.765383in,y=0.634533in,left,base]{\color{textcolor}\rmfamily\fontsize{10.000000}{12.000000}\selectfont \(\displaystyle 10^{0}\)}%
\end{pgfscope}%
\begin{pgfscope}%
\pgfsetbuttcap%
\pgfsetroundjoin%
\definecolor{currentfill}{rgb}{0.000000,0.000000,0.000000}%
\pgfsetfillcolor{currentfill}%
\pgfsetlinewidth{0.803000pt}%
\definecolor{currentstroke}{rgb}{0.000000,0.000000,0.000000}%
\pgfsetstrokecolor{currentstroke}%
\pgfsetdash{}{0pt}%
\pgfsys@defobject{currentmarker}{\pgfqpoint{0.000000in}{0.000000in}}{\pgfqpoint{0.048611in}{0.000000in}}{%
\pgfpathmoveto{\pgfqpoint{0.000000in}{0.000000in}}%
\pgfpathlineto{\pgfqpoint{0.048611in}{0.000000in}}%
\pgfusepath{stroke,fill}%
}%
\begin{pgfscope}%
\pgfsys@transformshift{4.668160in}{3.444394in}%
\pgfsys@useobject{currentmarker}{}%
\end{pgfscope}%
\end{pgfscope}%
\begin{pgfscope}%
\definecolor{textcolor}{rgb}{0.000000,0.000000,0.000000}%
\pgfsetstrokecolor{textcolor}%
\pgfsetfillcolor{textcolor}%
\pgftext[x=4.765383in,y=3.391633in,left,base]{\color{textcolor}\rmfamily\fontsize{10.000000}{12.000000}\selectfont \(\displaystyle 10^{1}\)}%
\end{pgfscope}%
\begin{pgfscope}%
\pgfsetbuttcap%
\pgfsetroundjoin%
\definecolor{currentfill}{rgb}{0.000000,0.000000,0.000000}%
\pgfsetfillcolor{currentfill}%
\pgfsetlinewidth{0.602250pt}%
\definecolor{currentstroke}{rgb}{0.000000,0.000000,0.000000}%
\pgfsetstrokecolor{currentstroke}%
\pgfsetdash{}{0pt}%
\pgfsys@defobject{currentmarker}{\pgfqpoint{0.000000in}{0.000000in}}{\pgfqpoint{0.027778in}{0.000000in}}{%
\pgfpathmoveto{\pgfqpoint{0.000000in}{0.000000in}}%
\pgfpathlineto{\pgfqpoint{0.027778in}{0.000000in}}%
\pgfusepath{stroke,fill}%
}%
\begin{pgfscope}%
\pgfsys@transformshift{4.668160in}{0.561136in}%
\pgfsys@useobject{currentmarker}{}%
\end{pgfscope}%
\end{pgfscope}%
\begin{pgfscope}%
\pgfsetbuttcap%
\pgfsetroundjoin%
\definecolor{currentfill}{rgb}{0.000000,0.000000,0.000000}%
\pgfsetfillcolor{currentfill}%
\pgfsetlinewidth{0.602250pt}%
\definecolor{currentstroke}{rgb}{0.000000,0.000000,0.000000}%
\pgfsetstrokecolor{currentstroke}%
\pgfsetdash{}{0pt}%
\pgfsys@defobject{currentmarker}{\pgfqpoint{0.000000in}{0.000000in}}{\pgfqpoint{0.027778in}{0.000000in}}{%
\pgfpathmoveto{\pgfqpoint{0.000000in}{0.000000in}}%
\pgfpathlineto{\pgfqpoint{0.027778in}{0.000000in}}%
\pgfusepath{stroke,fill}%
}%
\begin{pgfscope}%
\pgfsys@transformshift{4.668160in}{1.517264in}%
\pgfsys@useobject{currentmarker}{}%
\end{pgfscope}%
\end{pgfscope}%
\begin{pgfscope}%
\pgfsetbuttcap%
\pgfsetroundjoin%
\definecolor{currentfill}{rgb}{0.000000,0.000000,0.000000}%
\pgfsetfillcolor{currentfill}%
\pgfsetlinewidth{0.602250pt}%
\definecolor{currentstroke}{rgb}{0.000000,0.000000,0.000000}%
\pgfsetstrokecolor{currentstroke}%
\pgfsetdash{}{0pt}%
\pgfsys@defobject{currentmarker}{\pgfqpoint{0.000000in}{0.000000in}}{\pgfqpoint{0.027778in}{0.000000in}}{%
\pgfpathmoveto{\pgfqpoint{0.000000in}{0.000000in}}%
\pgfpathlineto{\pgfqpoint{0.027778in}{0.000000in}}%
\pgfusepath{stroke,fill}%
}%
\begin{pgfscope}%
\pgfsys@transformshift{4.668160in}{2.002765in}%
\pgfsys@useobject{currentmarker}{}%
\end{pgfscope}%
\end{pgfscope}%
\begin{pgfscope}%
\pgfsetbuttcap%
\pgfsetroundjoin%
\definecolor{currentfill}{rgb}{0.000000,0.000000,0.000000}%
\pgfsetfillcolor{currentfill}%
\pgfsetlinewidth{0.602250pt}%
\definecolor{currentstroke}{rgb}{0.000000,0.000000,0.000000}%
\pgfsetstrokecolor{currentstroke}%
\pgfsetdash{}{0pt}%
\pgfsys@defobject{currentmarker}{\pgfqpoint{0.000000in}{0.000000in}}{\pgfqpoint{0.027778in}{0.000000in}}{%
\pgfpathmoveto{\pgfqpoint{0.000000in}{0.000000in}}%
\pgfpathlineto{\pgfqpoint{0.027778in}{0.000000in}}%
\pgfusepath{stroke,fill}%
}%
\begin{pgfscope}%
\pgfsys@transformshift{4.668160in}{2.347234in}%
\pgfsys@useobject{currentmarker}{}%
\end{pgfscope}%
\end{pgfscope}%
\begin{pgfscope}%
\pgfsetbuttcap%
\pgfsetroundjoin%
\definecolor{currentfill}{rgb}{0.000000,0.000000,0.000000}%
\pgfsetfillcolor{currentfill}%
\pgfsetlinewidth{0.602250pt}%
\definecolor{currentstroke}{rgb}{0.000000,0.000000,0.000000}%
\pgfsetstrokecolor{currentstroke}%
\pgfsetdash{}{0pt}%
\pgfsys@defobject{currentmarker}{\pgfqpoint{0.000000in}{0.000000in}}{\pgfqpoint{0.027778in}{0.000000in}}{%
\pgfpathmoveto{\pgfqpoint{0.000000in}{0.000000in}}%
\pgfpathlineto{\pgfqpoint{0.027778in}{0.000000in}}%
\pgfusepath{stroke,fill}%
}%
\begin{pgfscope}%
\pgfsys@transformshift{4.668160in}{2.614424in}%
\pgfsys@useobject{currentmarker}{}%
\end{pgfscope}%
\end{pgfscope}%
\begin{pgfscope}%
\pgfsetbuttcap%
\pgfsetroundjoin%
\definecolor{currentfill}{rgb}{0.000000,0.000000,0.000000}%
\pgfsetfillcolor{currentfill}%
\pgfsetlinewidth{0.602250pt}%
\definecolor{currentstroke}{rgb}{0.000000,0.000000,0.000000}%
\pgfsetstrokecolor{currentstroke}%
\pgfsetdash{}{0pt}%
\pgfsys@defobject{currentmarker}{\pgfqpoint{0.000000in}{0.000000in}}{\pgfqpoint{0.027778in}{0.000000in}}{%
\pgfpathmoveto{\pgfqpoint{0.000000in}{0.000000in}}%
\pgfpathlineto{\pgfqpoint{0.027778in}{0.000000in}}%
\pgfusepath{stroke,fill}%
}%
\begin{pgfscope}%
\pgfsys@transformshift{4.668160in}{2.832735in}%
\pgfsys@useobject{currentmarker}{}%
\end{pgfscope}%
\end{pgfscope}%
\begin{pgfscope}%
\pgfsetbuttcap%
\pgfsetroundjoin%
\definecolor{currentfill}{rgb}{0.000000,0.000000,0.000000}%
\pgfsetfillcolor{currentfill}%
\pgfsetlinewidth{0.602250pt}%
\definecolor{currentstroke}{rgb}{0.000000,0.000000,0.000000}%
\pgfsetstrokecolor{currentstroke}%
\pgfsetdash{}{0pt}%
\pgfsys@defobject{currentmarker}{\pgfqpoint{0.000000in}{0.000000in}}{\pgfqpoint{0.027778in}{0.000000in}}{%
\pgfpathmoveto{\pgfqpoint{0.000000in}{0.000000in}}%
\pgfpathlineto{\pgfqpoint{0.027778in}{0.000000in}}%
\pgfusepath{stroke,fill}%
}%
\begin{pgfscope}%
\pgfsys@transformshift{4.668160in}{3.017314in}%
\pgfsys@useobject{currentmarker}{}%
\end{pgfscope}%
\end{pgfscope}%
\begin{pgfscope}%
\pgfsetbuttcap%
\pgfsetroundjoin%
\definecolor{currentfill}{rgb}{0.000000,0.000000,0.000000}%
\pgfsetfillcolor{currentfill}%
\pgfsetlinewidth{0.602250pt}%
\definecolor{currentstroke}{rgb}{0.000000,0.000000,0.000000}%
\pgfsetstrokecolor{currentstroke}%
\pgfsetdash{}{0pt}%
\pgfsys@defobject{currentmarker}{\pgfqpoint{0.000000in}{0.000000in}}{\pgfqpoint{0.027778in}{0.000000in}}{%
\pgfpathmoveto{\pgfqpoint{0.000000in}{0.000000in}}%
\pgfpathlineto{\pgfqpoint{0.027778in}{0.000000in}}%
\pgfusepath{stroke,fill}%
}%
\begin{pgfscope}%
\pgfsys@transformshift{4.668160in}{3.177204in}%
\pgfsys@useobject{currentmarker}{}%
\end{pgfscope}%
\end{pgfscope}%
\begin{pgfscope}%
\pgfsetbuttcap%
\pgfsetroundjoin%
\definecolor{currentfill}{rgb}{0.000000,0.000000,0.000000}%
\pgfsetfillcolor{currentfill}%
\pgfsetlinewidth{0.602250pt}%
\definecolor{currentstroke}{rgb}{0.000000,0.000000,0.000000}%
\pgfsetstrokecolor{currentstroke}%
\pgfsetdash{}{0pt}%
\pgfsys@defobject{currentmarker}{\pgfqpoint{0.000000in}{0.000000in}}{\pgfqpoint{0.027778in}{0.000000in}}{%
\pgfpathmoveto{\pgfqpoint{0.000000in}{0.000000in}}%
\pgfpathlineto{\pgfqpoint{0.027778in}{0.000000in}}%
\pgfusepath{stroke,fill}%
}%
\begin{pgfscope}%
\pgfsys@transformshift{4.668160in}{3.318236in}%
\pgfsys@useobject{currentmarker}{}%
\end{pgfscope}%
\end{pgfscope}%
\begin{pgfscope}%
\definecolor{textcolor}{rgb}{0.000000,0.000000,0.000000}%
\pgfsetstrokecolor{textcolor}%
\pgfsetfillcolor{textcolor}%
\pgftext[x=5.077690in,y=2.031603in,,top]{\color{textcolor}\rmfamily\fontsize{14.000000}{16.800000}\selectfont \(\displaystyle {\mathbf{E} \mbox{u}}\)}%
\end{pgfscope}%
\begin{pgfscope}%
\pgfsetbuttcap%
\pgfsetmiterjoin%
\pgfsetlinewidth{0.803000pt}%
\definecolor{currentstroke}{rgb}{0.501961,0.501961,0.501961}%
\pgfsetstrokecolor{currentstroke}%
\pgfsetdash{}{0pt}%
\pgfpathmoveto{\pgfqpoint{4.517160in}{0.521603in}}%
\pgfpathlineto{\pgfqpoint{4.517160in}{0.533400in}}%
\pgfpathlineto{\pgfqpoint{4.517160in}{3.529806in}}%
\pgfpathlineto{\pgfqpoint{4.517160in}{3.541603in}}%
\pgfpathlineto{\pgfqpoint{4.668160in}{3.541603in}}%
\pgfpathlineto{\pgfqpoint{4.668160in}{3.529806in}}%
\pgfpathlineto{\pgfqpoint{4.668160in}{0.533400in}}%
\pgfpathlineto{\pgfqpoint{4.668160in}{0.521603in}}%
\pgfpathclose%
\pgfusepath{stroke}%
\end{pgfscope}%
\end{pgfpicture}%
\makeatother%
\endgroup%

    \caption{Non-dimensional trajectories with the short-time scaling.\label{fig:series_s_ds}}
\end{figure}

\begin{figure}[htb]
    \centering
    %% Creator: Matplotlib, PGF backend
%%
%% To include the figure in your LaTeX document, write
%%   \input{<filename>.pgf}
%%
%% Make sure the required packages are loaded in your preamble
%%   \usepackage{pgf}
%%
%% Figures using additional raster images can only be included by \input if
%% they are in the same directory as the main LaTeX file. For loading figures
%% from other directories you can use the `import` package
%%   \usepackage{import}
%% and then include the figures with
%%   \import{<path to file>}{<filename>.pgf}
%%
%% Matplotlib used the following preamble
%%   \usepackage{fontspec}
%%   \setmainfont{DejaVu Serif}
%%   \setsansfont{DejaVu Sans}
%%   \setmonofont{DejaVu Sans Mono}
%%
\begingroup%
\makeatletter%
\begin{pgfpicture}%
\pgfpathrectangle{\pgfpointorigin}{\pgfqpoint{5.360508in}{3.676603in}}%
\pgfusepath{use as bounding box, clip}%
\begin{pgfscope}%
\pgfsetbuttcap%
\pgfsetmiterjoin%
\definecolor{currentfill}{rgb}{1.000000,1.000000,1.000000}%
\pgfsetfillcolor{currentfill}%
\pgfsetlinewidth{0.000000pt}%
\definecolor{currentstroke}{rgb}{1.000000,1.000000,1.000000}%
\pgfsetstrokecolor{currentstroke}%
\pgfsetdash{}{0pt}%
\pgfpathmoveto{\pgfqpoint{0.000000in}{0.000000in}}%
\pgfpathlineto{\pgfqpoint{5.360508in}{0.000000in}}%
\pgfpathlineto{\pgfqpoint{5.360508in}{3.676603in}}%
\pgfpathlineto{\pgfqpoint{0.000000in}{3.676603in}}%
\pgfpathclose%
\pgfusepath{fill}%
\end{pgfscope}%
\begin{pgfscope}%
\pgfsetbuttcap%
\pgfsetmiterjoin%
\definecolor{currentfill}{rgb}{1.000000,1.000000,1.000000}%
\pgfsetfillcolor{currentfill}%
\pgfsetlinewidth{0.000000pt}%
\definecolor{currentstroke}{rgb}{0.000000,0.000000,0.000000}%
\pgfsetstrokecolor{currentstroke}%
\pgfsetstrokeopacity{0.000000}%
\pgfsetdash{}{0pt}%
\pgfpathmoveto{\pgfqpoint{0.575508in}{0.521603in}}%
\pgfpathlineto{\pgfqpoint{5.225508in}{0.521603in}}%
\pgfpathlineto{\pgfqpoint{5.225508in}{3.541603in}}%
\pgfpathlineto{\pgfqpoint{0.575508in}{3.541603in}}%
\pgfpathclose%
\pgfusepath{fill}%
\end{pgfscope}%
\begin{pgfscope}%
\pgfpathrectangle{\pgfqpoint{0.575508in}{0.521603in}}{\pgfqpoint{4.650000in}{3.020000in}} %
\pgfusepath{clip}%
\pgfsetbuttcap%
\pgfsetroundjoin%
\definecolor{currentfill}{rgb}{1.000000,1.000000,1.000000}%
\pgfsetfillcolor{currentfill}%
\pgfsetlinewidth{1.003750pt}%
\definecolor{currentstroke}{rgb}{0.000000,0.000000,0.000000}%
\pgfsetstrokecolor{currentstroke}%
\pgfsetdash{}{0pt}%
\pgfpathmoveto{\pgfqpoint{0.876702in}{0.652871in}}%
\pgfpathcurveto{\pgfqpoint{0.887752in}{0.652871in}}{\pgfqpoint{0.898351in}{0.657261in}}{\pgfqpoint{0.906165in}{0.665075in}}%
\pgfpathcurveto{\pgfqpoint{0.913979in}{0.672888in}}{\pgfqpoint{0.918369in}{0.683487in}}{\pgfqpoint{0.918369in}{0.694537in}}%
\pgfpathcurveto{\pgfqpoint{0.918369in}{0.705588in}}{\pgfqpoint{0.913979in}{0.716187in}}{\pgfqpoint{0.906165in}{0.724000in}}%
\pgfpathcurveto{\pgfqpoint{0.898351in}{0.731814in}}{\pgfqpoint{0.887752in}{0.736204in}}{\pgfqpoint{0.876702in}{0.736204in}}%
\pgfpathcurveto{\pgfqpoint{0.865652in}{0.736204in}}{\pgfqpoint{0.855053in}{0.731814in}}{\pgfqpoint{0.847239in}{0.724000in}}%
\pgfpathcurveto{\pgfqpoint{0.839426in}{0.716187in}}{\pgfqpoint{0.835036in}{0.705588in}}{\pgfqpoint{0.835036in}{0.694537in}}%
\pgfpathcurveto{\pgfqpoint{0.835036in}{0.683487in}}{\pgfqpoint{0.839426in}{0.672888in}}{\pgfqpoint{0.847239in}{0.665075in}}%
\pgfpathcurveto{\pgfqpoint{0.855053in}{0.657261in}}{\pgfqpoint{0.865652in}{0.652871in}}{\pgfqpoint{0.876702in}{0.652871in}}%
\pgfpathclose%
\pgfusepath{stroke,fill}%
\end{pgfscope}%
\begin{pgfscope}%
\pgfpathrectangle{\pgfqpoint{0.575508in}{0.521603in}}{\pgfqpoint{4.650000in}{3.020000in}} %
\pgfusepath{clip}%
\pgfsetbuttcap%
\pgfsetroundjoin%
\definecolor{currentfill}{rgb}{1.000000,1.000000,1.000000}%
\pgfsetfillcolor{currentfill}%
\pgfsetlinewidth{1.003750pt}%
\definecolor{currentstroke}{rgb}{0.000000,0.000000,0.000000}%
\pgfsetstrokecolor{currentstroke}%
\pgfsetdash{}{0pt}%
\pgfpathmoveto{\pgfqpoint{0.876702in}{0.690201in}}%
\pgfpathcurveto{\pgfqpoint{0.887752in}{0.690201in}}{\pgfqpoint{0.898351in}{0.694591in}}{\pgfqpoint{0.906165in}{0.702405in}}%
\pgfpathcurveto{\pgfqpoint{0.913979in}{0.710218in}}{\pgfqpoint{0.918369in}{0.720818in}}{\pgfqpoint{0.918369in}{0.731868in}}%
\pgfpathcurveto{\pgfqpoint{0.918369in}{0.742918in}}{\pgfqpoint{0.913979in}{0.753517in}}{\pgfqpoint{0.906165in}{0.761330in}}%
\pgfpathcurveto{\pgfqpoint{0.898351in}{0.769144in}}{\pgfqpoint{0.887752in}{0.773534in}}{\pgfqpoint{0.876702in}{0.773534in}}%
\pgfpathcurveto{\pgfqpoint{0.865652in}{0.773534in}}{\pgfqpoint{0.855053in}{0.769144in}}{\pgfqpoint{0.847239in}{0.761330in}}%
\pgfpathcurveto{\pgfqpoint{0.839426in}{0.753517in}}{\pgfqpoint{0.835036in}{0.742918in}}{\pgfqpoint{0.835036in}{0.731868in}}%
\pgfpathcurveto{\pgfqpoint{0.835036in}{0.720818in}}{\pgfqpoint{0.839426in}{0.710218in}}{\pgfqpoint{0.847239in}{0.702405in}}%
\pgfpathcurveto{\pgfqpoint{0.855053in}{0.694591in}}{\pgfqpoint{0.865652in}{0.690201in}}{\pgfqpoint{0.876702in}{0.690201in}}%
\pgfpathclose%
\pgfusepath{stroke,fill}%
\end{pgfscope}%
\begin{pgfscope}%
\pgfpathrectangle{\pgfqpoint{0.575508in}{0.521603in}}{\pgfqpoint{4.650000in}{3.020000in}} %
\pgfusepath{clip}%
\pgfsetbuttcap%
\pgfsetroundjoin%
\definecolor{currentfill}{rgb}{1.000000,1.000000,1.000000}%
\pgfsetfillcolor{currentfill}%
\pgfsetlinewidth{1.003750pt}%
\definecolor{currentstroke}{rgb}{0.000000,0.000000,0.000000}%
\pgfsetstrokecolor{currentstroke}%
\pgfsetdash{}{0pt}%
\pgfpathmoveto{\pgfqpoint{0.811940in}{0.739534in}}%
\pgfpathcurveto{\pgfqpoint{0.822991in}{0.739534in}}{\pgfqpoint{0.833590in}{0.743925in}}{\pgfqpoint{0.841403in}{0.751738in}}%
\pgfpathcurveto{\pgfqpoint{0.849217in}{0.759552in}}{\pgfqpoint{0.853607in}{0.770151in}}{\pgfqpoint{0.853607in}{0.781201in}}%
\pgfpathcurveto{\pgfqpoint{0.853607in}{0.792251in}}{\pgfqpoint{0.849217in}{0.802850in}}{\pgfqpoint{0.841403in}{0.810664in}}%
\pgfpathcurveto{\pgfqpoint{0.833590in}{0.818478in}}{\pgfqpoint{0.822991in}{0.822868in}}{\pgfqpoint{0.811940in}{0.822868in}}%
\pgfpathcurveto{\pgfqpoint{0.800890in}{0.822868in}}{\pgfqpoint{0.790291in}{0.818478in}}{\pgfqpoint{0.782478in}{0.810664in}}%
\pgfpathcurveto{\pgfqpoint{0.774664in}{0.802850in}}{\pgfqpoint{0.770274in}{0.792251in}}{\pgfqpoint{0.770274in}{0.781201in}}%
\pgfpathcurveto{\pgfqpoint{0.770274in}{0.770151in}}{\pgfqpoint{0.774664in}{0.759552in}}{\pgfqpoint{0.782478in}{0.751738in}}%
\pgfpathcurveto{\pgfqpoint{0.790291in}{0.743925in}}{\pgfqpoint{0.800890in}{0.739534in}}{\pgfqpoint{0.811940in}{0.739534in}}%
\pgfpathclose%
\pgfusepath{stroke,fill}%
\end{pgfscope}%
\begin{pgfscope}%
\pgfpathrectangle{\pgfqpoint{0.575508in}{0.521603in}}{\pgfqpoint{4.650000in}{3.020000in}} %
\pgfusepath{clip}%
\pgfsetbuttcap%
\pgfsetroundjoin%
\definecolor{currentfill}{rgb}{1.000000,1.000000,1.000000}%
\pgfsetfillcolor{currentfill}%
\pgfsetlinewidth{1.003750pt}%
\definecolor{currentstroke}{rgb}{0.000000,0.000000,0.000000}%
\pgfsetstrokecolor{currentstroke}%
\pgfsetdash{}{0pt}%
\pgfpathmoveto{\pgfqpoint{0.876702in}{0.807081in}}%
\pgfpathcurveto{\pgfqpoint{0.887752in}{0.807081in}}{\pgfqpoint{0.898351in}{0.811471in}}{\pgfqpoint{0.906165in}{0.819285in}}%
\pgfpathcurveto{\pgfqpoint{0.913979in}{0.827098in}}{\pgfqpoint{0.918369in}{0.837697in}}{\pgfqpoint{0.918369in}{0.848748in}}%
\pgfpathcurveto{\pgfqpoint{0.918369in}{0.859798in}}{\pgfqpoint{0.913979in}{0.870397in}}{\pgfqpoint{0.906165in}{0.878210in}}%
\pgfpathcurveto{\pgfqpoint{0.898351in}{0.886024in}}{\pgfqpoint{0.887752in}{0.890414in}}{\pgfqpoint{0.876702in}{0.890414in}}%
\pgfpathcurveto{\pgfqpoint{0.865652in}{0.890414in}}{\pgfqpoint{0.855053in}{0.886024in}}{\pgfqpoint{0.847239in}{0.878210in}}%
\pgfpathcurveto{\pgfqpoint{0.839426in}{0.870397in}}{\pgfqpoint{0.835036in}{0.859798in}}{\pgfqpoint{0.835036in}{0.848748in}}%
\pgfpathcurveto{\pgfqpoint{0.835036in}{0.837697in}}{\pgfqpoint{0.839426in}{0.827098in}}{\pgfqpoint{0.847239in}{0.819285in}}%
\pgfpathcurveto{\pgfqpoint{0.855053in}{0.811471in}}{\pgfqpoint{0.865652in}{0.807081in}}{\pgfqpoint{0.876702in}{0.807081in}}%
\pgfpathclose%
\pgfusepath{stroke,fill}%
\end{pgfscope}%
\begin{pgfscope}%
\pgfpathrectangle{\pgfqpoint{0.575508in}{0.521603in}}{\pgfqpoint{4.650000in}{3.020000in}} %
\pgfusepath{clip}%
\pgfsetbuttcap%
\pgfsetroundjoin%
\definecolor{currentfill}{rgb}{1.000000,1.000000,1.000000}%
\pgfsetfillcolor{currentfill}%
\pgfsetlinewidth{1.003750pt}%
\definecolor{currentstroke}{rgb}{0.000000,0.000000,0.000000}%
\pgfsetstrokecolor{currentstroke}%
\pgfsetdash{}{0pt}%
\pgfpathmoveto{\pgfqpoint{1.589082in}{1.189637in}}%
\pgfpathcurveto{\pgfqpoint{1.600132in}{1.189637in}}{\pgfqpoint{1.610731in}{1.194027in}}{\pgfqpoint{1.618545in}{1.201841in}}%
\pgfpathcurveto{\pgfqpoint{1.626358in}{1.209655in}}{\pgfqpoint{1.630748in}{1.220254in}}{\pgfqpoint{1.630748in}{1.231304in}}%
\pgfpathcurveto{\pgfqpoint{1.630748in}{1.242354in}}{\pgfqpoint{1.626358in}{1.252953in}}{\pgfqpoint{1.618545in}{1.260767in}}%
\pgfpathcurveto{\pgfqpoint{1.610731in}{1.268580in}}{\pgfqpoint{1.600132in}{1.272970in}}{\pgfqpoint{1.589082in}{1.272970in}}%
\pgfpathcurveto{\pgfqpoint{1.578032in}{1.272970in}}{\pgfqpoint{1.567433in}{1.268580in}}{\pgfqpoint{1.559619in}{1.260767in}}%
\pgfpathcurveto{\pgfqpoint{1.551805in}{1.252953in}}{\pgfqpoint{1.547415in}{1.242354in}}{\pgfqpoint{1.547415in}{1.231304in}}%
\pgfpathcurveto{\pgfqpoint{1.547415in}{1.220254in}}{\pgfqpoint{1.551805in}{1.209655in}}{\pgfqpoint{1.559619in}{1.201841in}}%
\pgfpathcurveto{\pgfqpoint{1.567433in}{1.194027in}}{\pgfqpoint{1.578032in}{1.189637in}}{\pgfqpoint{1.589082in}{1.189637in}}%
\pgfpathclose%
\pgfusepath{stroke,fill}%
\end{pgfscope}%
\begin{pgfscope}%
\pgfpathrectangle{\pgfqpoint{0.575508in}{0.521603in}}{\pgfqpoint{4.650000in}{3.020000in}} %
\pgfusepath{clip}%
\pgfsetbuttcap%
\pgfsetroundjoin%
\definecolor{currentfill}{rgb}{1.000000,1.000000,1.000000}%
\pgfsetfillcolor{currentfill}%
\pgfsetlinewidth{1.003750pt}%
\definecolor{currentstroke}{rgb}{0.000000,0.000000,0.000000}%
\pgfsetstrokecolor{currentstroke}%
\pgfsetdash{}{0pt}%
\pgfpathmoveto{\pgfqpoint{1.783367in}{1.230970in}}%
\pgfpathcurveto{\pgfqpoint{1.794417in}{1.230970in}}{\pgfqpoint{1.805016in}{1.235361in}}{\pgfqpoint{1.812830in}{1.243174in}}%
\pgfpathcurveto{\pgfqpoint{1.820644in}{1.250988in}}{\pgfqpoint{1.825034in}{1.261587in}}{\pgfqpoint{1.825034in}{1.272637in}}%
\pgfpathcurveto{\pgfqpoint{1.825034in}{1.283687in}}{\pgfqpoint{1.820644in}{1.294286in}}{\pgfqpoint{1.812830in}{1.302100in}}%
\pgfpathcurveto{\pgfqpoint{1.805016in}{1.309913in}}{\pgfqpoint{1.794417in}{1.314304in}}{\pgfqpoint{1.783367in}{1.314304in}}%
\pgfpathcurveto{\pgfqpoint{1.772317in}{1.314304in}}{\pgfqpoint{1.761718in}{1.309913in}}{\pgfqpoint{1.753904in}{1.302100in}}%
\pgfpathcurveto{\pgfqpoint{1.746091in}{1.294286in}}{\pgfqpoint{1.741700in}{1.283687in}}{\pgfqpoint{1.741700in}{1.272637in}}%
\pgfpathcurveto{\pgfqpoint{1.741700in}{1.261587in}}{\pgfqpoint{1.746091in}{1.250988in}}{\pgfqpoint{1.753904in}{1.243174in}}%
\pgfpathcurveto{\pgfqpoint{1.761718in}{1.235361in}}{\pgfqpoint{1.772317in}{1.230970in}}{\pgfqpoint{1.783367in}{1.230970in}}%
\pgfpathclose%
\pgfusepath{stroke,fill}%
\end{pgfscope}%
\begin{pgfscope}%
\pgfpathrectangle{\pgfqpoint{0.575508in}{0.521603in}}{\pgfqpoint{4.650000in}{3.020000in}} %
\pgfusepath{clip}%
\pgfsetbuttcap%
\pgfsetroundjoin%
\definecolor{currentfill}{rgb}{1.000000,1.000000,1.000000}%
\pgfsetfillcolor{currentfill}%
\pgfsetlinewidth{1.003750pt}%
\definecolor{currentstroke}{rgb}{0.000000,0.000000,0.000000}%
\pgfsetstrokecolor{currentstroke}%
\pgfsetdash{}{0pt}%
\pgfpathmoveto{\pgfqpoint{1.621463in}{1.241729in}}%
\pgfpathcurveto{\pgfqpoint{1.632513in}{1.241729in}}{\pgfqpoint{1.643112in}{1.246119in}}{\pgfqpoint{1.650925in}{1.253933in}}%
\pgfpathcurveto{\pgfqpoint{1.658739in}{1.261746in}}{\pgfqpoint{1.663129in}{1.272345in}}{\pgfqpoint{1.663129in}{1.283395in}}%
\pgfpathcurveto{\pgfqpoint{1.663129in}{1.294445in}}{\pgfqpoint{1.658739in}{1.305044in}}{\pgfqpoint{1.650925in}{1.312858in}}%
\pgfpathcurveto{\pgfqpoint{1.643112in}{1.320672in}}{\pgfqpoint{1.632513in}{1.325062in}}{\pgfqpoint{1.621463in}{1.325062in}}%
\pgfpathcurveto{\pgfqpoint{1.610413in}{1.325062in}}{\pgfqpoint{1.599814in}{1.320672in}}{\pgfqpoint{1.592000in}{1.312858in}}%
\pgfpathcurveto{\pgfqpoint{1.584186in}{1.305044in}}{\pgfqpoint{1.579796in}{1.294445in}}{\pgfqpoint{1.579796in}{1.283395in}}%
\pgfpathcurveto{\pgfqpoint{1.579796in}{1.272345in}}{\pgfqpoint{1.584186in}{1.261746in}}{\pgfqpoint{1.592000in}{1.253933in}}%
\pgfpathcurveto{\pgfqpoint{1.599814in}{1.246119in}}{\pgfqpoint{1.610413in}{1.241729in}}{\pgfqpoint{1.621463in}{1.241729in}}%
\pgfpathclose%
\pgfusepath{stroke,fill}%
\end{pgfscope}%
\begin{pgfscope}%
\pgfpathrectangle{\pgfqpoint{0.575508in}{0.521603in}}{\pgfqpoint{4.650000in}{3.020000in}} %
\pgfusepath{clip}%
\pgfsetbuttcap%
\pgfsetroundjoin%
\definecolor{currentfill}{rgb}{1.000000,1.000000,1.000000}%
\pgfsetfillcolor{currentfill}%
\pgfsetlinewidth{1.003750pt}%
\definecolor{currentstroke}{rgb}{0.000000,0.000000,0.000000}%
\pgfsetstrokecolor{currentstroke}%
\pgfsetdash{}{0pt}%
\pgfpathmoveto{\pgfqpoint{4.989075in}{3.327003in}}%
\pgfpathcurveto{\pgfqpoint{5.000125in}{3.327003in}}{\pgfqpoint{5.010724in}{3.331393in}}{\pgfqpoint{5.018538in}{3.339206in}}%
\pgfpathcurveto{\pgfqpoint{5.026351in}{3.347020in}}{\pgfqpoint{5.030742in}{3.357619in}}{\pgfqpoint{5.030742in}{3.368669in}}%
\pgfpathcurveto{\pgfqpoint{5.030742in}{3.379719in}}{\pgfqpoint{5.026351in}{3.390318in}}{\pgfqpoint{5.018538in}{3.398132in}}%
\pgfpathcurveto{\pgfqpoint{5.010724in}{3.405946in}}{\pgfqpoint{5.000125in}{3.410336in}}{\pgfqpoint{4.989075in}{3.410336in}}%
\pgfpathcurveto{\pgfqpoint{4.978025in}{3.410336in}}{\pgfqpoint{4.967426in}{3.405946in}}{\pgfqpoint{4.959612in}{3.398132in}}%
\pgfpathcurveto{\pgfqpoint{4.951799in}{3.390318in}}{\pgfqpoint{4.947408in}{3.379719in}}{\pgfqpoint{4.947408in}{3.368669in}}%
\pgfpathcurveto{\pgfqpoint{4.947408in}{3.357619in}}{\pgfqpoint{4.951799in}{3.347020in}}{\pgfqpoint{4.959612in}{3.339206in}}%
\pgfpathcurveto{\pgfqpoint{4.967426in}{3.331393in}}{\pgfqpoint{4.978025in}{3.327003in}}{\pgfqpoint{4.989075in}{3.327003in}}%
\pgfpathclose%
\pgfusepath{stroke,fill}%
\end{pgfscope}%
\begin{pgfscope}%
\pgfpathrectangle{\pgfqpoint{0.575508in}{0.521603in}}{\pgfqpoint{4.650000in}{3.020000in}} %
\pgfusepath{clip}%
\pgfsetbuttcap%
\pgfsetroundjoin%
\definecolor{currentfill}{rgb}{1.000000,1.000000,1.000000}%
\pgfsetfillcolor{currentfill}%
\pgfsetlinewidth{1.003750pt}%
\definecolor{currentstroke}{rgb}{0.000000,0.000000,0.000000}%
\pgfsetstrokecolor{currentstroke}%
\pgfsetdash{}{0pt}%
\pgfpathmoveto{\pgfqpoint{4.730028in}{3.200774in}}%
\pgfpathcurveto{\pgfqpoint{4.741078in}{3.200774in}}{\pgfqpoint{4.751677in}{3.205165in}}{\pgfqpoint{4.759491in}{3.212978in}}%
\pgfpathcurveto{\pgfqpoint{4.767304in}{3.220792in}}{\pgfqpoint{4.771695in}{3.231391in}}{\pgfqpoint{4.771695in}{3.242441in}}%
\pgfpathcurveto{\pgfqpoint{4.771695in}{3.253491in}}{\pgfqpoint{4.767304in}{3.264090in}}{\pgfqpoint{4.759491in}{3.271904in}}%
\pgfpathcurveto{\pgfqpoint{4.751677in}{3.279718in}}{\pgfqpoint{4.741078in}{3.284108in}}{\pgfqpoint{4.730028in}{3.284108in}}%
\pgfpathcurveto{\pgfqpoint{4.718978in}{3.284108in}}{\pgfqpoint{4.708379in}{3.279718in}}{\pgfqpoint{4.700565in}{3.271904in}}%
\pgfpathcurveto{\pgfqpoint{4.692752in}{3.264090in}}{\pgfqpoint{4.688361in}{3.253491in}}{\pgfqpoint{4.688361in}{3.242441in}}%
\pgfpathcurveto{\pgfqpoint{4.688361in}{3.231391in}}{\pgfqpoint{4.692752in}{3.220792in}}{\pgfqpoint{4.700565in}{3.212978in}}%
\pgfpathcurveto{\pgfqpoint{4.708379in}{3.205165in}}{\pgfqpoint{4.718978in}{3.200774in}}{\pgfqpoint{4.730028in}{3.200774in}}%
\pgfpathclose%
\pgfusepath{stroke,fill}%
\end{pgfscope}%
\begin{pgfscope}%
\pgfpathrectangle{\pgfqpoint{0.575508in}{0.521603in}}{\pgfqpoint{4.650000in}{3.020000in}} %
\pgfusepath{clip}%
\pgfsetbuttcap%
\pgfsetroundjoin%
\definecolor{currentfill}{rgb}{1.000000,1.000000,1.000000}%
\pgfsetfillcolor{currentfill}%
\pgfsetlinewidth{1.003750pt}%
\definecolor{currentstroke}{rgb}{0.000000,0.000000,0.000000}%
\pgfsetstrokecolor{currentstroke}%
\pgfsetdash{}{0pt}%
\pgfpathmoveto{\pgfqpoint{3.208126in}{2.165670in}}%
\pgfpathcurveto{\pgfqpoint{3.219176in}{2.165670in}}{\pgfqpoint{3.229775in}{2.170060in}}{\pgfqpoint{3.237589in}{2.177874in}}%
\pgfpathcurveto{\pgfqpoint{3.245403in}{2.185687in}}{\pgfqpoint{3.249793in}{2.196286in}}{\pgfqpoint{3.249793in}{2.207337in}}%
\pgfpathcurveto{\pgfqpoint{3.249793in}{2.218387in}}{\pgfqpoint{3.245403in}{2.228986in}}{\pgfqpoint{3.237589in}{2.236799in}}%
\pgfpathcurveto{\pgfqpoint{3.229775in}{2.244613in}}{\pgfqpoint{3.219176in}{2.249003in}}{\pgfqpoint{3.208126in}{2.249003in}}%
\pgfpathcurveto{\pgfqpoint{3.197076in}{2.249003in}}{\pgfqpoint{3.186477in}{2.244613in}}{\pgfqpoint{3.178663in}{2.236799in}}%
\pgfpathcurveto{\pgfqpoint{3.170850in}{2.228986in}}{\pgfqpoint{3.166460in}{2.218387in}}{\pgfqpoint{3.166460in}{2.207337in}}%
\pgfpathcurveto{\pgfqpoint{3.166460in}{2.196286in}}{\pgfqpoint{3.170850in}{2.185687in}}{\pgfqpoint{3.178663in}{2.177874in}}%
\pgfpathcurveto{\pgfqpoint{3.186477in}{2.170060in}}{\pgfqpoint{3.197076in}{2.165670in}}{\pgfqpoint{3.208126in}{2.165670in}}%
\pgfpathclose%
\pgfusepath{stroke,fill}%
\end{pgfscope}%
\begin{pgfscope}%
\pgfsetbuttcap%
\pgfsetroundjoin%
\definecolor{currentfill}{rgb}{0.000000,0.000000,0.000000}%
\pgfsetfillcolor{currentfill}%
\pgfsetlinewidth{0.803000pt}%
\definecolor{currentstroke}{rgb}{0.000000,0.000000,0.000000}%
\pgfsetstrokecolor{currentstroke}%
\pgfsetdash{}{0pt}%
\pgfsys@defobject{currentmarker}{\pgfqpoint{0.000000in}{-0.048611in}}{\pgfqpoint{0.000000in}{0.000000in}}{%
\pgfpathmoveto{\pgfqpoint{0.000000in}{0.000000in}}%
\pgfpathlineto{\pgfqpoint{0.000000in}{-0.048611in}}%
\pgfusepath{stroke,fill}%
}%
\begin{pgfscope}%
\pgfsys@transformshift{1.135749in}{0.521603in}%
\pgfsys@useobject{currentmarker}{}%
\end{pgfscope}%
\end{pgfscope}%
\begin{pgfscope}%
\pgftext[x=1.135749in,y=0.424381in,,top]{\rmfamily\fontsize{10.000000}{12.000000}\selectfont \(\displaystyle 0.4\)}%
\end{pgfscope}%
\begin{pgfscope}%
\pgfsetbuttcap%
\pgfsetroundjoin%
\definecolor{currentfill}{rgb}{0.000000,0.000000,0.000000}%
\pgfsetfillcolor{currentfill}%
\pgfsetlinewidth{0.803000pt}%
\definecolor{currentstroke}{rgb}{0.000000,0.000000,0.000000}%
\pgfsetstrokecolor{currentstroke}%
\pgfsetdash{}{0pt}%
\pgfsys@defobject{currentmarker}{\pgfqpoint{0.000000in}{-0.048611in}}{\pgfqpoint{0.000000in}{0.000000in}}{%
\pgfpathmoveto{\pgfqpoint{0.000000in}{0.000000in}}%
\pgfpathlineto{\pgfqpoint{0.000000in}{-0.048611in}}%
\pgfusepath{stroke,fill}%
}%
\begin{pgfscope}%
\pgfsys@transformshift{1.912891in}{0.521603in}%
\pgfsys@useobject{currentmarker}{}%
\end{pgfscope}%
\end{pgfscope}%
\begin{pgfscope}%
\pgftext[x=1.912891in,y=0.424381in,,top]{\rmfamily\fontsize{10.000000}{12.000000}\selectfont \(\displaystyle 0.6\)}%
\end{pgfscope}%
\begin{pgfscope}%
\pgfsetbuttcap%
\pgfsetroundjoin%
\definecolor{currentfill}{rgb}{0.000000,0.000000,0.000000}%
\pgfsetfillcolor{currentfill}%
\pgfsetlinewidth{0.803000pt}%
\definecolor{currentstroke}{rgb}{0.000000,0.000000,0.000000}%
\pgfsetstrokecolor{currentstroke}%
\pgfsetdash{}{0pt}%
\pgfsys@defobject{currentmarker}{\pgfqpoint{0.000000in}{-0.048611in}}{\pgfqpoint{0.000000in}{0.000000in}}{%
\pgfpathmoveto{\pgfqpoint{0.000000in}{0.000000in}}%
\pgfpathlineto{\pgfqpoint{0.000000in}{-0.048611in}}%
\pgfusepath{stroke,fill}%
}%
\begin{pgfscope}%
\pgfsys@transformshift{2.690032in}{0.521603in}%
\pgfsys@useobject{currentmarker}{}%
\end{pgfscope}%
\end{pgfscope}%
\begin{pgfscope}%
\pgftext[x=2.690032in,y=0.424381in,,top]{\rmfamily\fontsize{10.000000}{12.000000}\selectfont \(\displaystyle 0.8\)}%
\end{pgfscope}%
\begin{pgfscope}%
\pgfsetbuttcap%
\pgfsetroundjoin%
\definecolor{currentfill}{rgb}{0.000000,0.000000,0.000000}%
\pgfsetfillcolor{currentfill}%
\pgfsetlinewidth{0.803000pt}%
\definecolor{currentstroke}{rgb}{0.000000,0.000000,0.000000}%
\pgfsetstrokecolor{currentstroke}%
\pgfsetdash{}{0pt}%
\pgfsys@defobject{currentmarker}{\pgfqpoint{0.000000in}{-0.048611in}}{\pgfqpoint{0.000000in}{0.000000in}}{%
\pgfpathmoveto{\pgfqpoint{0.000000in}{0.000000in}}%
\pgfpathlineto{\pgfqpoint{0.000000in}{-0.048611in}}%
\pgfusepath{stroke,fill}%
}%
\begin{pgfscope}%
\pgfsys@transformshift{3.467173in}{0.521603in}%
\pgfsys@useobject{currentmarker}{}%
\end{pgfscope}%
\end{pgfscope}%
\begin{pgfscope}%
\pgftext[x=3.467173in,y=0.424381in,,top]{\rmfamily\fontsize{10.000000}{12.000000}\selectfont \(\displaystyle 1.0\)}%
\end{pgfscope}%
\begin{pgfscope}%
\pgfsetbuttcap%
\pgfsetroundjoin%
\definecolor{currentfill}{rgb}{0.000000,0.000000,0.000000}%
\pgfsetfillcolor{currentfill}%
\pgfsetlinewidth{0.803000pt}%
\definecolor{currentstroke}{rgb}{0.000000,0.000000,0.000000}%
\pgfsetstrokecolor{currentstroke}%
\pgfsetdash{}{0pt}%
\pgfsys@defobject{currentmarker}{\pgfqpoint{0.000000in}{-0.048611in}}{\pgfqpoint{0.000000in}{0.000000in}}{%
\pgfpathmoveto{\pgfqpoint{0.000000in}{0.000000in}}%
\pgfpathlineto{\pgfqpoint{0.000000in}{-0.048611in}}%
\pgfusepath{stroke,fill}%
}%
\begin{pgfscope}%
\pgfsys@transformshift{4.244315in}{0.521603in}%
\pgfsys@useobject{currentmarker}{}%
\end{pgfscope}%
\end{pgfscope}%
\begin{pgfscope}%
\pgftext[x=4.244315in,y=0.424381in,,top]{\rmfamily\fontsize{10.000000}{12.000000}\selectfont \(\displaystyle 1.2\)}%
\end{pgfscope}%
\begin{pgfscope}%
\pgfsetbuttcap%
\pgfsetroundjoin%
\definecolor{currentfill}{rgb}{0.000000,0.000000,0.000000}%
\pgfsetfillcolor{currentfill}%
\pgfsetlinewidth{0.803000pt}%
\definecolor{currentstroke}{rgb}{0.000000,0.000000,0.000000}%
\pgfsetstrokecolor{currentstroke}%
\pgfsetdash{}{0pt}%
\pgfsys@defobject{currentmarker}{\pgfqpoint{0.000000in}{-0.048611in}}{\pgfqpoint{0.000000in}{0.000000in}}{%
\pgfpathmoveto{\pgfqpoint{0.000000in}{0.000000in}}%
\pgfpathlineto{\pgfqpoint{0.000000in}{-0.048611in}}%
\pgfusepath{stroke,fill}%
}%
\begin{pgfscope}%
\pgfsys@transformshift{5.021456in}{0.521603in}%
\pgfsys@useobject{currentmarker}{}%
\end{pgfscope}%
\end{pgfscope}%
\begin{pgfscope}%
\pgftext[x=5.021456in,y=0.424381in,,top]{\rmfamily\fontsize{10.000000}{12.000000}\selectfont \(\displaystyle 1.4\)}%
\end{pgfscope}%
\begin{pgfscope}%
\pgftext[x=2.900508in,y=0.234413in,,top]{\rmfamily\fontsize{10.000000}{12.000000}\selectfont \(\displaystyle {t_b}\) (s)}%
\end{pgfscope}%
\begin{pgfscope}%
\pgfsetbuttcap%
\pgfsetroundjoin%
\definecolor{currentfill}{rgb}{0.000000,0.000000,0.000000}%
\pgfsetfillcolor{currentfill}%
\pgfsetlinewidth{0.803000pt}%
\definecolor{currentstroke}{rgb}{0.000000,0.000000,0.000000}%
\pgfsetstrokecolor{currentstroke}%
\pgfsetdash{}{0pt}%
\pgfsys@defobject{currentmarker}{\pgfqpoint{-0.048611in}{0.000000in}}{\pgfqpoint{0.000000in}{0.000000in}}{%
\pgfpathmoveto{\pgfqpoint{0.000000in}{0.000000in}}%
\pgfpathlineto{\pgfqpoint{-0.048611in}{0.000000in}}%
\pgfusepath{stroke,fill}%
}%
\begin{pgfscope}%
\pgfsys@transformshift{0.575508in}{0.542723in}%
\pgfsys@useobject{currentmarker}{}%
\end{pgfscope}%
\end{pgfscope}%
\begin{pgfscope}%
\pgftext[x=0.300816in,y=0.489962in,left,base]{\rmfamily\fontsize{10.000000}{12.000000}\selectfont \(\displaystyle 0.2\)}%
\end{pgfscope}%
\begin{pgfscope}%
\pgfsetbuttcap%
\pgfsetroundjoin%
\definecolor{currentfill}{rgb}{0.000000,0.000000,0.000000}%
\pgfsetfillcolor{currentfill}%
\pgfsetlinewidth{0.803000pt}%
\definecolor{currentstroke}{rgb}{0.000000,0.000000,0.000000}%
\pgfsetstrokecolor{currentstroke}%
\pgfsetdash{}{0pt}%
\pgfsys@defobject{currentmarker}{\pgfqpoint{-0.048611in}{0.000000in}}{\pgfqpoint{0.000000in}{0.000000in}}{%
\pgfpathmoveto{\pgfqpoint{0.000000in}{0.000000in}}%
\pgfpathlineto{\pgfqpoint{-0.048611in}{0.000000in}}%
\pgfusepath{stroke,fill}%
}%
\begin{pgfscope}%
\pgfsys@transformshift{0.575508in}{0.901714in}%
\pgfsys@useobject{currentmarker}{}%
\end{pgfscope}%
\end{pgfscope}%
\begin{pgfscope}%
\pgftext[x=0.300816in,y=0.848953in,left,base]{\rmfamily\fontsize{10.000000}{12.000000}\selectfont \(\displaystyle 0.3\)}%
\end{pgfscope}%
\begin{pgfscope}%
\pgfsetbuttcap%
\pgfsetroundjoin%
\definecolor{currentfill}{rgb}{0.000000,0.000000,0.000000}%
\pgfsetfillcolor{currentfill}%
\pgfsetlinewidth{0.803000pt}%
\definecolor{currentstroke}{rgb}{0.000000,0.000000,0.000000}%
\pgfsetstrokecolor{currentstroke}%
\pgfsetdash{}{0pt}%
\pgfsys@defobject{currentmarker}{\pgfqpoint{-0.048611in}{0.000000in}}{\pgfqpoint{0.000000in}{0.000000in}}{%
\pgfpathmoveto{\pgfqpoint{0.000000in}{0.000000in}}%
\pgfpathlineto{\pgfqpoint{-0.048611in}{0.000000in}}%
\pgfusepath{stroke,fill}%
}%
\begin{pgfscope}%
\pgfsys@transformshift{0.575508in}{1.260705in}%
\pgfsys@useobject{currentmarker}{}%
\end{pgfscope}%
\end{pgfscope}%
\begin{pgfscope}%
\pgftext[x=0.300816in,y=1.207944in,left,base]{\rmfamily\fontsize{10.000000}{12.000000}\selectfont \(\displaystyle 0.4\)}%
\end{pgfscope}%
\begin{pgfscope}%
\pgfsetbuttcap%
\pgfsetroundjoin%
\definecolor{currentfill}{rgb}{0.000000,0.000000,0.000000}%
\pgfsetfillcolor{currentfill}%
\pgfsetlinewidth{0.803000pt}%
\definecolor{currentstroke}{rgb}{0.000000,0.000000,0.000000}%
\pgfsetstrokecolor{currentstroke}%
\pgfsetdash{}{0pt}%
\pgfsys@defobject{currentmarker}{\pgfqpoint{-0.048611in}{0.000000in}}{\pgfqpoint{0.000000in}{0.000000in}}{%
\pgfpathmoveto{\pgfqpoint{0.000000in}{0.000000in}}%
\pgfpathlineto{\pgfqpoint{-0.048611in}{0.000000in}}%
\pgfusepath{stroke,fill}%
}%
\begin{pgfscope}%
\pgfsys@transformshift{0.575508in}{1.619697in}%
\pgfsys@useobject{currentmarker}{}%
\end{pgfscope}%
\end{pgfscope}%
\begin{pgfscope}%
\pgftext[x=0.300816in,y=1.566935in,left,base]{\rmfamily\fontsize{10.000000}{12.000000}\selectfont \(\displaystyle 0.5\)}%
\end{pgfscope}%
\begin{pgfscope}%
\pgfsetbuttcap%
\pgfsetroundjoin%
\definecolor{currentfill}{rgb}{0.000000,0.000000,0.000000}%
\pgfsetfillcolor{currentfill}%
\pgfsetlinewidth{0.803000pt}%
\definecolor{currentstroke}{rgb}{0.000000,0.000000,0.000000}%
\pgfsetstrokecolor{currentstroke}%
\pgfsetdash{}{0pt}%
\pgfsys@defobject{currentmarker}{\pgfqpoint{-0.048611in}{0.000000in}}{\pgfqpoint{0.000000in}{0.000000in}}{%
\pgfpathmoveto{\pgfqpoint{0.000000in}{0.000000in}}%
\pgfpathlineto{\pgfqpoint{-0.048611in}{0.000000in}}%
\pgfusepath{stroke,fill}%
}%
\begin{pgfscope}%
\pgfsys@transformshift{0.575508in}{1.978688in}%
\pgfsys@useobject{currentmarker}{}%
\end{pgfscope}%
\end{pgfscope}%
\begin{pgfscope}%
\pgftext[x=0.300816in,y=1.925926in,left,base]{\rmfamily\fontsize{10.000000}{12.000000}\selectfont \(\displaystyle 0.6\)}%
\end{pgfscope}%
\begin{pgfscope}%
\pgfsetbuttcap%
\pgfsetroundjoin%
\definecolor{currentfill}{rgb}{0.000000,0.000000,0.000000}%
\pgfsetfillcolor{currentfill}%
\pgfsetlinewidth{0.803000pt}%
\definecolor{currentstroke}{rgb}{0.000000,0.000000,0.000000}%
\pgfsetstrokecolor{currentstroke}%
\pgfsetdash{}{0pt}%
\pgfsys@defobject{currentmarker}{\pgfqpoint{-0.048611in}{0.000000in}}{\pgfqpoint{0.000000in}{0.000000in}}{%
\pgfpathmoveto{\pgfqpoint{0.000000in}{0.000000in}}%
\pgfpathlineto{\pgfqpoint{-0.048611in}{0.000000in}}%
\pgfusepath{stroke,fill}%
}%
\begin{pgfscope}%
\pgfsys@transformshift{0.575508in}{2.337679in}%
\pgfsys@useobject{currentmarker}{}%
\end{pgfscope}%
\end{pgfscope}%
\begin{pgfscope}%
\pgftext[x=0.300816in,y=2.284918in,left,base]{\rmfamily\fontsize{10.000000}{12.000000}\selectfont \(\displaystyle 0.7\)}%
\end{pgfscope}%
\begin{pgfscope}%
\pgfsetbuttcap%
\pgfsetroundjoin%
\definecolor{currentfill}{rgb}{0.000000,0.000000,0.000000}%
\pgfsetfillcolor{currentfill}%
\pgfsetlinewidth{0.803000pt}%
\definecolor{currentstroke}{rgb}{0.000000,0.000000,0.000000}%
\pgfsetstrokecolor{currentstroke}%
\pgfsetdash{}{0pt}%
\pgfsys@defobject{currentmarker}{\pgfqpoint{-0.048611in}{0.000000in}}{\pgfqpoint{0.000000in}{0.000000in}}{%
\pgfpathmoveto{\pgfqpoint{0.000000in}{0.000000in}}%
\pgfpathlineto{\pgfqpoint{-0.048611in}{0.000000in}}%
\pgfusepath{stroke,fill}%
}%
\begin{pgfscope}%
\pgfsys@transformshift{0.575508in}{2.696670in}%
\pgfsys@useobject{currentmarker}{}%
\end{pgfscope}%
\end{pgfscope}%
\begin{pgfscope}%
\pgftext[x=0.300816in,y=2.643909in,left,base]{\rmfamily\fontsize{10.000000}{12.000000}\selectfont \(\displaystyle 0.8\)}%
\end{pgfscope}%
\begin{pgfscope}%
\pgfsetbuttcap%
\pgfsetroundjoin%
\definecolor{currentfill}{rgb}{0.000000,0.000000,0.000000}%
\pgfsetfillcolor{currentfill}%
\pgfsetlinewidth{0.803000pt}%
\definecolor{currentstroke}{rgb}{0.000000,0.000000,0.000000}%
\pgfsetstrokecolor{currentstroke}%
\pgfsetdash{}{0pt}%
\pgfsys@defobject{currentmarker}{\pgfqpoint{-0.048611in}{0.000000in}}{\pgfqpoint{0.000000in}{0.000000in}}{%
\pgfpathmoveto{\pgfqpoint{0.000000in}{0.000000in}}%
\pgfpathlineto{\pgfqpoint{-0.048611in}{0.000000in}}%
\pgfusepath{stroke,fill}%
}%
\begin{pgfscope}%
\pgfsys@transformshift{0.575508in}{3.055661in}%
\pgfsys@useobject{currentmarker}{}%
\end{pgfscope}%
\end{pgfscope}%
\begin{pgfscope}%
\pgftext[x=0.300816in,y=3.002900in,left,base]{\rmfamily\fontsize{10.000000}{12.000000}\selectfont \(\displaystyle 0.9\)}%
\end{pgfscope}%
\begin{pgfscope}%
\pgfsetbuttcap%
\pgfsetroundjoin%
\definecolor{currentfill}{rgb}{0.000000,0.000000,0.000000}%
\pgfsetfillcolor{currentfill}%
\pgfsetlinewidth{0.803000pt}%
\definecolor{currentstroke}{rgb}{0.000000,0.000000,0.000000}%
\pgfsetstrokecolor{currentstroke}%
\pgfsetdash{}{0pt}%
\pgfsys@defobject{currentmarker}{\pgfqpoint{-0.048611in}{0.000000in}}{\pgfqpoint{0.000000in}{0.000000in}}{%
\pgfpathmoveto{\pgfqpoint{0.000000in}{0.000000in}}%
\pgfpathlineto{\pgfqpoint{-0.048611in}{0.000000in}}%
\pgfusepath{stroke,fill}%
}%
\begin{pgfscope}%
\pgfsys@transformshift{0.575508in}{3.414653in}%
\pgfsys@useobject{currentmarker}{}%
\end{pgfscope}%
\end{pgfscope}%
\begin{pgfscope}%
\pgftext[x=0.300816in,y=3.361891in,left,base]{\rmfamily\fontsize{10.000000}{12.000000}\selectfont \(\displaystyle 1.0\)}%
\end{pgfscope}%
\begin{pgfscope}%
\pgftext[x=0.245260in,y=2.031603in,,bottom,rotate=90.000000]{\rmfamily\fontsize{10.000000}{12.000000}\selectfont \(\displaystyle t_c t_f\) (s)}%
\end{pgfscope}%
\begin{pgfscope}%
\pgfsetrectcap%
\pgfsetmiterjoin%
\pgfsetlinewidth{0.803000pt}%
\definecolor{currentstroke}{rgb}{0.000000,0.000000,0.000000}%
\pgfsetstrokecolor{currentstroke}%
\pgfsetdash{}{0pt}%
\pgfpathmoveto{\pgfqpoint{0.575508in}{0.521603in}}%
\pgfpathlineto{\pgfqpoint{0.575508in}{3.541603in}}%
\pgfusepath{stroke}%
\end{pgfscope}%
\begin{pgfscope}%
\pgfsetrectcap%
\pgfsetmiterjoin%
\pgfsetlinewidth{0.803000pt}%
\definecolor{currentstroke}{rgb}{0.000000,0.000000,0.000000}%
\pgfsetstrokecolor{currentstroke}%
\pgfsetdash{}{0pt}%
\pgfpathmoveto{\pgfqpoint{5.225508in}{0.521603in}}%
\pgfpathlineto{\pgfqpoint{5.225508in}{3.541603in}}%
\pgfusepath{stroke}%
\end{pgfscope}%
\begin{pgfscope}%
\pgfsetrectcap%
\pgfsetmiterjoin%
\pgfsetlinewidth{0.803000pt}%
\definecolor{currentstroke}{rgb}{0.000000,0.000000,0.000000}%
\pgfsetstrokecolor{currentstroke}%
\pgfsetdash{}{0pt}%
\pgfpathmoveto{\pgfqpoint{0.575508in}{0.521603in}}%
\pgfpathlineto{\pgfqpoint{5.225508in}{0.521603in}}%
\pgfusepath{stroke}%
\end{pgfscope}%
\begin{pgfscope}%
\pgfsetrectcap%
\pgfsetmiterjoin%
\pgfsetlinewidth{0.803000pt}%
\definecolor{currentstroke}{rgb}{0.000000,0.000000,0.000000}%
\pgfsetstrokecolor{currentstroke}%
\pgfsetdash{}{0pt}%
\pgfpathmoveto{\pgfqpoint{0.575508in}{3.541603in}}%
\pgfpathlineto{\pgfqpoint{5.225508in}{3.541603in}}%
\pgfusepath{stroke}%
\end{pgfscope}%
\end{pgfpicture}%
\makeatother%
\endgroup%

    \caption{.\label{fig:times}}
\end{figure}

The covariance of $\mathbb{I}\mbox{m}$ with $\mathbb{E}\mbox{u}$ is shown in Figure \ref{fig:dnumbs}. Predictably, there is quite strong correlation between the dimensionless groups. We also see that $\mathbb{I}\mbox{m} < 1$ for all drops. Using an OLS regression, we find the model $\mathbb{I}\mbox{m} \sim (0.012 \pm 0.003) \mathbb{E}\mbox{u} + (0.212 \pm 0.036) $ with $R^2 =0.59$.
\begin{figure}[htb]
    \centering
    %% Creator: Matplotlib, PGF backend
%%
%% To include the figure in your LaTeX document, write
%%   \input{<filename>.pgf}
%%
%% Make sure the required packages are loaded in your preamble
%%   \usepackage{pgf}
%%
%% Figures using additional raster images can only be included by \input if
%% they are in the same directory as the main LaTeX file. For loading figures
%% from other directories you can use the `import` package
%%   \usepackage{import}
%% and then include the figures with
%%   \import{<path to file>}{<filename>.pgf}
%%
%% Matplotlib used the following preamble
%%   \usepackage{fontspec}
%%   \setmainfont{DejaVuSerif.ttf}[Path=/home/erin/anaconda3/lib/python3.6/site-packages/matplotlib/mpl-data/fonts/ttf/]
%%   \setsansfont{DejaVuSans.ttf}[Path=/home/erin/anaconda3/lib/python3.6/site-packages/matplotlib/mpl-data/fonts/ttf/]
%%   \setmonofont{DejaVuSansMono.ttf}[Path=/home/erin/anaconda3/lib/python3.6/site-packages/matplotlib/mpl-data/fonts/ttf/]
%%
\begingroup%
\makeatletter%
\begin{pgfpicture}%
\pgfpathrectangle{\pgfpointorigin}{\pgfqpoint{5.314660in}{3.641603in}}%
\pgfusepath{use as bounding box, clip}%
\begin{pgfscope}%
\pgfsetbuttcap%
\pgfsetmiterjoin%
\definecolor{currentfill}{rgb}{1.000000,1.000000,1.000000}%
\pgfsetfillcolor{currentfill}%
\pgfsetlinewidth{0.000000pt}%
\definecolor{currentstroke}{rgb}{1.000000,1.000000,1.000000}%
\pgfsetstrokecolor{currentstroke}%
\pgfsetdash{}{0pt}%
\pgfpathmoveto{\pgfqpoint{0.000000in}{0.000000in}}%
\pgfpathlineto{\pgfqpoint{5.314660in}{0.000000in}}%
\pgfpathlineto{\pgfqpoint{5.314660in}{3.641603in}}%
\pgfpathlineto{\pgfqpoint{0.000000in}{3.641603in}}%
\pgfpathclose%
\pgfusepath{fill}%
\end{pgfscope}%
\begin{pgfscope}%
\pgfsetbuttcap%
\pgfsetmiterjoin%
\definecolor{currentfill}{rgb}{1.000000,1.000000,1.000000}%
\pgfsetfillcolor{currentfill}%
\pgfsetlinewidth{0.000000pt}%
\definecolor{currentstroke}{rgb}{0.000000,0.000000,0.000000}%
\pgfsetstrokecolor{currentstroke}%
\pgfsetstrokeopacity{0.000000}%
\pgfsetdash{}{0pt}%
\pgfpathmoveto{\pgfqpoint{0.564660in}{0.521603in}}%
\pgfpathlineto{\pgfqpoint{5.214660in}{0.521603in}}%
\pgfpathlineto{\pgfqpoint{5.214660in}{3.541603in}}%
\pgfpathlineto{\pgfqpoint{0.564660in}{3.541603in}}%
\pgfpathclose%
\pgfusepath{fill}%
\end{pgfscope}%
\begin{pgfscope}%
\pgfpathrectangle{\pgfqpoint{0.564660in}{0.521603in}}{\pgfqpoint{4.650000in}{3.020000in}}%
\pgfusepath{clip}%
\pgfsetbuttcap%
\pgfsetroundjoin%
\definecolor{currentfill}{rgb}{1.000000,1.000000,1.000000}%
\pgfsetfillcolor{currentfill}%
\pgfsetlinewidth{1.003750pt}%
\definecolor{currentstroke}{rgb}{0.000000,0.000000,0.000000}%
\pgfsetstrokecolor{currentstroke}%
\pgfsetdash{}{0pt}%
\pgfpathmoveto{\pgfqpoint{0.816771in}{2.246882in}}%
\pgfpathcurveto{\pgfqpoint{0.827822in}{2.246882in}}{\pgfqpoint{0.838421in}{2.251272in}}{\pgfqpoint{0.846234in}{2.259086in}}%
\pgfpathcurveto{\pgfqpoint{0.854048in}{2.266900in}}{\pgfqpoint{0.858438in}{2.277499in}}{\pgfqpoint{0.858438in}{2.288549in}}%
\pgfpathcurveto{\pgfqpoint{0.858438in}{2.299599in}}{\pgfqpoint{0.854048in}{2.310198in}}{\pgfqpoint{0.846234in}{2.318012in}}%
\pgfpathcurveto{\pgfqpoint{0.838421in}{2.325825in}}{\pgfqpoint{0.827822in}{2.330215in}}{\pgfqpoint{0.816771in}{2.330215in}}%
\pgfpathcurveto{\pgfqpoint{0.805721in}{2.330215in}}{\pgfqpoint{0.795122in}{2.325825in}}{\pgfqpoint{0.787309in}{2.318012in}}%
\pgfpathcurveto{\pgfqpoint{0.779495in}{2.310198in}}{\pgfqpoint{0.775105in}{2.299599in}}{\pgfqpoint{0.775105in}{2.288549in}}%
\pgfpathcurveto{\pgfqpoint{0.775105in}{2.277499in}}{\pgfqpoint{0.779495in}{2.266900in}}{\pgfqpoint{0.787309in}{2.259086in}}%
\pgfpathcurveto{\pgfqpoint{0.795122in}{2.251272in}}{\pgfqpoint{0.805721in}{2.246882in}}{\pgfqpoint{0.816771in}{2.246882in}}%
\pgfpathclose%
\pgfusepath{stroke,fill}%
\end{pgfscope}%
\begin{pgfscope}%
\pgfpathrectangle{\pgfqpoint{0.564660in}{0.521603in}}{\pgfqpoint{4.650000in}{3.020000in}}%
\pgfusepath{clip}%
\pgfsetbuttcap%
\pgfsetroundjoin%
\definecolor{currentfill}{rgb}{1.000000,1.000000,1.000000}%
\pgfsetfillcolor{currentfill}%
\pgfsetlinewidth{1.003750pt}%
\definecolor{currentstroke}{rgb}{0.000000,0.000000,0.000000}%
\pgfsetstrokecolor{currentstroke}%
\pgfsetdash{}{0pt}%
\pgfpathmoveto{\pgfqpoint{3.858085in}{1.799540in}}%
\pgfpathcurveto{\pgfqpoint{3.869135in}{1.799540in}}{\pgfqpoint{3.879734in}{1.803931in}}{\pgfqpoint{3.887547in}{1.811744in}}%
\pgfpathcurveto{\pgfqpoint{3.895361in}{1.819558in}}{\pgfqpoint{3.899751in}{1.830157in}}{\pgfqpoint{3.899751in}{1.841207in}}%
\pgfpathcurveto{\pgfqpoint{3.899751in}{1.852257in}}{\pgfqpoint{3.895361in}{1.862856in}}{\pgfqpoint{3.887547in}{1.870670in}}%
\pgfpathcurveto{\pgfqpoint{3.879734in}{1.878484in}}{\pgfqpoint{3.869135in}{1.882874in}}{\pgfqpoint{3.858085in}{1.882874in}}%
\pgfpathcurveto{\pgfqpoint{3.847034in}{1.882874in}}{\pgfqpoint{3.836435in}{1.878484in}}{\pgfqpoint{3.828622in}{1.870670in}}%
\pgfpathcurveto{\pgfqpoint{3.820808in}{1.862856in}}{\pgfqpoint{3.816418in}{1.852257in}}{\pgfqpoint{3.816418in}{1.841207in}}%
\pgfpathcurveto{\pgfqpoint{3.816418in}{1.830157in}}{\pgfqpoint{3.820808in}{1.819558in}}{\pgfqpoint{3.828622in}{1.811744in}}%
\pgfpathcurveto{\pgfqpoint{3.836435in}{1.803931in}}{\pgfqpoint{3.847034in}{1.799540in}}{\pgfqpoint{3.858085in}{1.799540in}}%
\pgfpathclose%
\pgfusepath{stroke,fill}%
\end{pgfscope}%
\begin{pgfscope}%
\pgfpathrectangle{\pgfqpoint{0.564660in}{0.521603in}}{\pgfqpoint{4.650000in}{3.020000in}}%
\pgfusepath{clip}%
\pgfsetbuttcap%
\pgfsetroundjoin%
\definecolor{currentfill}{rgb}{1.000000,1.000000,1.000000}%
\pgfsetfillcolor{currentfill}%
\pgfsetlinewidth{1.003750pt}%
\definecolor{currentstroke}{rgb}{0.000000,0.000000,0.000000}%
\pgfsetstrokecolor{currentstroke}%
\pgfsetdash{}{0pt}%
\pgfpathmoveto{\pgfqpoint{4.993344in}{2.156424in}}%
\pgfpathcurveto{\pgfqpoint{5.004394in}{2.156424in}}{\pgfqpoint{5.014993in}{2.160815in}}{\pgfqpoint{5.022807in}{2.168628in}}%
\pgfpathcurveto{\pgfqpoint{5.030620in}{2.176442in}}{\pgfqpoint{5.035010in}{2.187041in}}{\pgfqpoint{5.035010in}{2.198091in}}%
\pgfpathcurveto{\pgfqpoint{5.035010in}{2.209141in}}{\pgfqpoint{5.030620in}{2.219740in}}{\pgfqpoint{5.022807in}{2.227554in}}%
\pgfpathcurveto{\pgfqpoint{5.014993in}{2.235367in}}{\pgfqpoint{5.004394in}{2.239758in}}{\pgfqpoint{4.993344in}{2.239758in}}%
\pgfpathcurveto{\pgfqpoint{4.982294in}{2.239758in}}{\pgfqpoint{4.971695in}{2.235367in}}{\pgfqpoint{4.963881in}{2.227554in}}%
\pgfpathcurveto{\pgfqpoint{4.956067in}{2.219740in}}{\pgfqpoint{4.951677in}{2.209141in}}{\pgfqpoint{4.951677in}{2.198091in}}%
\pgfpathcurveto{\pgfqpoint{4.951677in}{2.187041in}}{\pgfqpoint{4.956067in}{2.176442in}}{\pgfqpoint{4.963881in}{2.168628in}}%
\pgfpathcurveto{\pgfqpoint{4.971695in}{2.160815in}}{\pgfqpoint{4.982294in}{2.156424in}}{\pgfqpoint{4.993344in}{2.156424in}}%
\pgfpathclose%
\pgfusepath{stroke,fill}%
\end{pgfscope}%
\begin{pgfscope}%
\pgfpathrectangle{\pgfqpoint{0.564660in}{0.521603in}}{\pgfqpoint{4.650000in}{3.020000in}}%
\pgfusepath{clip}%
\pgfsetbuttcap%
\pgfsetroundjoin%
\definecolor{currentfill}{rgb}{1.000000,1.000000,1.000000}%
\pgfsetfillcolor{currentfill}%
\pgfsetlinewidth{1.003750pt}%
\definecolor{currentstroke}{rgb}{0.000000,0.000000,0.000000}%
\pgfsetstrokecolor{currentstroke}%
\pgfsetdash{}{0pt}%
\pgfpathmoveto{\pgfqpoint{1.914828in}{1.195437in}}%
\pgfpathcurveto{\pgfqpoint{1.925878in}{1.195437in}}{\pgfqpoint{1.936477in}{1.199828in}}{\pgfqpoint{1.944290in}{1.207641in}}%
\pgfpathcurveto{\pgfqpoint{1.952104in}{1.215455in}}{\pgfqpoint{1.956494in}{1.226054in}}{\pgfqpoint{1.956494in}{1.237104in}}%
\pgfpathcurveto{\pgfqpoint{1.956494in}{1.248154in}}{\pgfqpoint{1.952104in}{1.258753in}}{\pgfqpoint{1.944290in}{1.266567in}}%
\pgfpathcurveto{\pgfqpoint{1.936477in}{1.274380in}}{\pgfqpoint{1.925878in}{1.278771in}}{\pgfqpoint{1.914828in}{1.278771in}}%
\pgfpathcurveto{\pgfqpoint{1.903777in}{1.278771in}}{\pgfqpoint{1.893178in}{1.274380in}}{\pgfqpoint{1.885365in}{1.266567in}}%
\pgfpathcurveto{\pgfqpoint{1.877551in}{1.258753in}}{\pgfqpoint{1.873161in}{1.248154in}}{\pgfqpoint{1.873161in}{1.237104in}}%
\pgfpathcurveto{\pgfqpoint{1.873161in}{1.226054in}}{\pgfqpoint{1.877551in}{1.215455in}}{\pgfqpoint{1.885365in}{1.207641in}}%
\pgfpathcurveto{\pgfqpoint{1.893178in}{1.199828in}}{\pgfqpoint{1.903777in}{1.195437in}}{\pgfqpoint{1.914828in}{1.195437in}}%
\pgfpathclose%
\pgfusepath{stroke,fill}%
\end{pgfscope}%
\begin{pgfscope}%
\pgfpathrectangle{\pgfqpoint{0.564660in}{0.521603in}}{\pgfqpoint{4.650000in}{3.020000in}}%
\pgfusepath{clip}%
\pgfsetbuttcap%
\pgfsetroundjoin%
\definecolor{currentfill}{rgb}{1.000000,1.000000,1.000000}%
\pgfsetfillcolor{currentfill}%
\pgfsetlinewidth{1.003750pt}%
\definecolor{currentstroke}{rgb}{0.000000,0.000000,0.000000}%
\pgfsetstrokecolor{currentstroke}%
\pgfsetdash{}{0pt}%
\pgfpathmoveto{\pgfqpoint{2.773519in}{3.358147in}}%
\pgfpathcurveto{\pgfqpoint{2.784569in}{3.358147in}}{\pgfqpoint{2.795168in}{3.362537in}}{\pgfqpoint{2.802981in}{3.370350in}}%
\pgfpathcurveto{\pgfqpoint{2.810795in}{3.378164in}}{\pgfqpoint{2.815185in}{3.388763in}}{\pgfqpoint{2.815185in}{3.399813in}}%
\pgfpathcurveto{\pgfqpoint{2.815185in}{3.410863in}}{\pgfqpoint{2.810795in}{3.421462in}}{\pgfqpoint{2.802981in}{3.429276in}}%
\pgfpathcurveto{\pgfqpoint{2.795168in}{3.437090in}}{\pgfqpoint{2.784569in}{3.441480in}}{\pgfqpoint{2.773519in}{3.441480in}}%
\pgfpathcurveto{\pgfqpoint{2.762469in}{3.441480in}}{\pgfqpoint{2.751870in}{3.437090in}}{\pgfqpoint{2.744056in}{3.429276in}}%
\pgfpathcurveto{\pgfqpoint{2.736242in}{3.421462in}}{\pgfqpoint{2.731852in}{3.410863in}}{\pgfqpoint{2.731852in}{3.399813in}}%
\pgfpathcurveto{\pgfqpoint{2.731852in}{3.388763in}}{\pgfqpoint{2.736242in}{3.378164in}}{\pgfqpoint{2.744056in}{3.370350in}}%
\pgfpathcurveto{\pgfqpoint{2.751870in}{3.362537in}}{\pgfqpoint{2.762469in}{3.358147in}}{\pgfqpoint{2.773519in}{3.358147in}}%
\pgfpathclose%
\pgfusepath{stroke,fill}%
\end{pgfscope}%
\begin{pgfscope}%
\pgfpathrectangle{\pgfqpoint{0.564660in}{0.521603in}}{\pgfqpoint{4.650000in}{3.020000in}}%
\pgfusepath{clip}%
\pgfsetbuttcap%
\pgfsetroundjoin%
\definecolor{currentfill}{rgb}{1.000000,1.000000,1.000000}%
\pgfsetfillcolor{currentfill}%
\pgfsetlinewidth{1.003750pt}%
\definecolor{currentstroke}{rgb}{0.000000,0.000000,0.000000}%
\pgfsetstrokecolor{currentstroke}%
\pgfsetdash{}{0pt}%
\pgfpathmoveto{\pgfqpoint{2.234616in}{0.983815in}}%
\pgfpathcurveto{\pgfqpoint{2.245666in}{0.983815in}}{\pgfqpoint{2.256265in}{0.988205in}}{\pgfqpoint{2.264079in}{0.996018in}}%
\pgfpathcurveto{\pgfqpoint{2.271892in}{1.003832in}}{\pgfqpoint{2.276283in}{1.014431in}}{\pgfqpoint{2.276283in}{1.025481in}}%
\pgfpathcurveto{\pgfqpoint{2.276283in}{1.036531in}}{\pgfqpoint{2.271892in}{1.047130in}}{\pgfqpoint{2.264079in}{1.054944in}}%
\pgfpathcurveto{\pgfqpoint{2.256265in}{1.062758in}}{\pgfqpoint{2.245666in}{1.067148in}}{\pgfqpoint{2.234616in}{1.067148in}}%
\pgfpathcurveto{\pgfqpoint{2.223566in}{1.067148in}}{\pgfqpoint{2.212967in}{1.062758in}}{\pgfqpoint{2.205153in}{1.054944in}}%
\pgfpathcurveto{\pgfqpoint{2.197340in}{1.047130in}}{\pgfqpoint{2.192949in}{1.036531in}}{\pgfqpoint{2.192949in}{1.025481in}}%
\pgfpathcurveto{\pgfqpoint{2.192949in}{1.014431in}}{\pgfqpoint{2.197340in}{1.003832in}}{\pgfqpoint{2.205153in}{0.996018in}}%
\pgfpathcurveto{\pgfqpoint{2.212967in}{0.988205in}}{\pgfqpoint{2.223566in}{0.983815in}}{\pgfqpoint{2.234616in}{0.983815in}}%
\pgfpathclose%
\pgfusepath{stroke,fill}%
\end{pgfscope}%
\begin{pgfscope}%
\pgfpathrectangle{\pgfqpoint{0.564660in}{0.521603in}}{\pgfqpoint{4.650000in}{3.020000in}}%
\pgfusepath{clip}%
\pgfsetbuttcap%
\pgfsetroundjoin%
\definecolor{currentfill}{rgb}{1.000000,1.000000,1.000000}%
\pgfsetfillcolor{currentfill}%
\pgfsetlinewidth{1.003750pt}%
\definecolor{currentstroke}{rgb}{0.000000,0.000000,0.000000}%
\pgfsetstrokecolor{currentstroke}%
\pgfsetdash{}{0pt}%
\pgfpathmoveto{\pgfqpoint{0.819079in}{1.480427in}}%
\pgfpathcurveto{\pgfqpoint{0.830129in}{1.480427in}}{\pgfqpoint{0.840728in}{1.484817in}}{\pgfqpoint{0.848542in}{1.492631in}}%
\pgfpathcurveto{\pgfqpoint{0.856355in}{1.500444in}}{\pgfqpoint{0.860746in}{1.511043in}}{\pgfqpoint{0.860746in}{1.522093in}}%
\pgfpathcurveto{\pgfqpoint{0.860746in}{1.533144in}}{\pgfqpoint{0.856355in}{1.543743in}}{\pgfqpoint{0.848542in}{1.551556in}}%
\pgfpathcurveto{\pgfqpoint{0.840728in}{1.559370in}}{\pgfqpoint{0.830129in}{1.563760in}}{\pgfqpoint{0.819079in}{1.563760in}}%
\pgfpathcurveto{\pgfqpoint{0.808029in}{1.563760in}}{\pgfqpoint{0.797430in}{1.559370in}}{\pgfqpoint{0.789616in}{1.551556in}}%
\pgfpathcurveto{\pgfqpoint{0.781803in}{1.543743in}}{\pgfqpoint{0.777412in}{1.533144in}}{\pgfqpoint{0.777412in}{1.522093in}}%
\pgfpathcurveto{\pgfqpoint{0.777412in}{1.511043in}}{\pgfqpoint{0.781803in}{1.500444in}}{\pgfqpoint{0.789616in}{1.492631in}}%
\pgfpathcurveto{\pgfqpoint{0.797430in}{1.484817in}}{\pgfqpoint{0.808029in}{1.480427in}}{\pgfqpoint{0.819079in}{1.480427in}}%
\pgfpathclose%
\pgfusepath{stroke,fill}%
\end{pgfscope}%
\begin{pgfscope}%
\pgfpathrectangle{\pgfqpoint{0.564660in}{0.521603in}}{\pgfqpoint{4.650000in}{3.020000in}}%
\pgfusepath{clip}%
\pgfsetbuttcap%
\pgfsetroundjoin%
\definecolor{currentfill}{rgb}{1.000000,1.000000,1.000000}%
\pgfsetfillcolor{currentfill}%
\pgfsetlinewidth{1.003750pt}%
\definecolor{currentstroke}{rgb}{0.000000,0.000000,0.000000}%
\pgfsetstrokecolor{currentstroke}%
\pgfsetdash{}{0pt}%
\pgfpathmoveto{\pgfqpoint{1.363796in}{0.692093in}}%
\pgfpathcurveto{\pgfqpoint{1.374846in}{0.692093in}}{\pgfqpoint{1.385445in}{0.696483in}}{\pgfqpoint{1.393259in}{0.704297in}}%
\pgfpathcurveto{\pgfqpoint{1.401072in}{0.712110in}}{\pgfqpoint{1.405462in}{0.722709in}}{\pgfqpoint{1.405462in}{0.733759in}}%
\pgfpathcurveto{\pgfqpoint{1.405462in}{0.744810in}}{\pgfqpoint{1.401072in}{0.755409in}}{\pgfqpoint{1.393259in}{0.763222in}}%
\pgfpathcurveto{\pgfqpoint{1.385445in}{0.771036in}}{\pgfqpoint{1.374846in}{0.775426in}}{\pgfqpoint{1.363796in}{0.775426in}}%
\pgfpathcurveto{\pgfqpoint{1.352746in}{0.775426in}}{\pgfqpoint{1.342147in}{0.771036in}}{\pgfqpoint{1.334333in}{0.763222in}}%
\pgfpathcurveto{\pgfqpoint{1.326519in}{0.755409in}}{\pgfqpoint{1.322129in}{0.744810in}}{\pgfqpoint{1.322129in}{0.733759in}}%
\pgfpathcurveto{\pgfqpoint{1.322129in}{0.722709in}}{\pgfqpoint{1.326519in}{0.712110in}}{\pgfqpoint{1.334333in}{0.704297in}}%
\pgfpathcurveto{\pgfqpoint{1.342147in}{0.696483in}}{\pgfqpoint{1.352746in}{0.692093in}}{\pgfqpoint{1.363796in}{0.692093in}}%
\pgfpathclose%
\pgfusepath{stroke,fill}%
\end{pgfscope}%
\begin{pgfscope}%
\pgfpathrectangle{\pgfqpoint{0.564660in}{0.521603in}}{\pgfqpoint{4.650000in}{3.020000in}}%
\pgfusepath{clip}%
\pgfsetbuttcap%
\pgfsetroundjoin%
\definecolor{currentfill}{rgb}{1.000000,1.000000,1.000000}%
\pgfsetfillcolor{currentfill}%
\pgfsetlinewidth{1.003750pt}%
\definecolor{currentstroke}{rgb}{0.000000,0.000000,0.000000}%
\pgfsetstrokecolor{currentstroke}%
\pgfsetdash{}{0pt}%
\pgfpathmoveto{\pgfqpoint{1.096147in}{0.621727in}}%
\pgfpathcurveto{\pgfqpoint{1.107197in}{0.621727in}}{\pgfqpoint{1.117796in}{0.626117in}}{\pgfqpoint{1.125609in}{0.633931in}}%
\pgfpathcurveto{\pgfqpoint{1.133423in}{0.641744in}}{\pgfqpoint{1.137813in}{0.652343in}}{\pgfqpoint{1.137813in}{0.663393in}}%
\pgfpathcurveto{\pgfqpoint{1.137813in}{0.674444in}}{\pgfqpoint{1.133423in}{0.685043in}}{\pgfqpoint{1.125609in}{0.692856in}}%
\pgfpathcurveto{\pgfqpoint{1.117796in}{0.700670in}}{\pgfqpoint{1.107197in}{0.705060in}}{\pgfqpoint{1.096147in}{0.705060in}}%
\pgfpathcurveto{\pgfqpoint{1.085096in}{0.705060in}}{\pgfqpoint{1.074497in}{0.700670in}}{\pgfqpoint{1.066684in}{0.692856in}}%
\pgfpathcurveto{\pgfqpoint{1.058870in}{0.685043in}}{\pgfqpoint{1.054480in}{0.674444in}}{\pgfqpoint{1.054480in}{0.663393in}}%
\pgfpathcurveto{\pgfqpoint{1.054480in}{0.652343in}}{\pgfqpoint{1.058870in}{0.641744in}}{\pgfqpoint{1.066684in}{0.633931in}}%
\pgfpathcurveto{\pgfqpoint{1.074497in}{0.626117in}}{\pgfqpoint{1.085096in}{0.621727in}}{\pgfqpoint{1.096147in}{0.621727in}}%
\pgfpathclose%
\pgfusepath{stroke,fill}%
\end{pgfscope}%
\begin{pgfscope}%
\pgfpathrectangle{\pgfqpoint{0.564660in}{0.521603in}}{\pgfqpoint{4.650000in}{3.020000in}}%
\pgfusepath{clip}%
\pgfsetbuttcap%
\pgfsetroundjoin%
\definecolor{currentfill}{rgb}{1.000000,1.000000,1.000000}%
\pgfsetfillcolor{currentfill}%
\pgfsetlinewidth{1.003750pt}%
\definecolor{currentstroke}{rgb}{0.000000,0.000000,0.000000}%
\pgfsetstrokecolor{currentstroke}%
\pgfsetdash{}{0pt}%
\pgfpathmoveto{\pgfqpoint{0.887904in}{1.305315in}}%
\pgfpathcurveto{\pgfqpoint{0.898954in}{1.305315in}}{\pgfqpoint{0.909553in}{1.309705in}}{\pgfqpoint{0.917367in}{1.317519in}}%
\pgfpathcurveto{\pgfqpoint{0.925180in}{1.325332in}}{\pgfqpoint{0.929570in}{1.335932in}}{\pgfqpoint{0.929570in}{1.346982in}}%
\pgfpathcurveto{\pgfqpoint{0.929570in}{1.358032in}}{\pgfqpoint{0.925180in}{1.368631in}}{\pgfqpoint{0.917367in}{1.376444in}}%
\pgfpathcurveto{\pgfqpoint{0.909553in}{1.384258in}}{\pgfqpoint{0.898954in}{1.388648in}}{\pgfqpoint{0.887904in}{1.388648in}}%
\pgfpathcurveto{\pgfqpoint{0.876854in}{1.388648in}}{\pgfqpoint{0.866255in}{1.384258in}}{\pgfqpoint{0.858441in}{1.376444in}}%
\pgfpathcurveto{\pgfqpoint{0.850627in}{1.368631in}}{\pgfqpoint{0.846237in}{1.358032in}}{\pgfqpoint{0.846237in}{1.346982in}}%
\pgfpathcurveto{\pgfqpoint{0.846237in}{1.335932in}}{\pgfqpoint{0.850627in}{1.325332in}}{\pgfqpoint{0.858441in}{1.317519in}}%
\pgfpathcurveto{\pgfqpoint{0.866255in}{1.309705in}}{\pgfqpoint{0.876854in}{1.305315in}}{\pgfqpoint{0.887904in}{1.305315in}}%
\pgfpathclose%
\pgfusepath{stroke,fill}%
\end{pgfscope}%
\begin{pgfscope}%
\pgfpathrectangle{\pgfqpoint{0.564660in}{0.521603in}}{\pgfqpoint{4.650000in}{3.020000in}}%
\pgfusepath{clip}%
\pgfsetbuttcap%
\pgfsetroundjoin%
\definecolor{currentfill}{rgb}{1.000000,1.000000,1.000000}%
\pgfsetfillcolor{currentfill}%
\pgfsetlinewidth{1.003750pt}%
\definecolor{currentstroke}{rgb}{0.000000,0.000000,0.000000}%
\pgfsetstrokecolor{currentstroke}%
\pgfsetdash{}{0pt}%
\pgfpathmoveto{\pgfqpoint{2.143771in}{2.158900in}}%
\pgfpathcurveto{\pgfqpoint{2.154821in}{2.158900in}}{\pgfqpoint{2.165420in}{2.163290in}}{\pgfqpoint{2.173234in}{2.171104in}}%
\pgfpathcurveto{\pgfqpoint{2.181047in}{2.178918in}}{\pgfqpoint{2.185438in}{2.189517in}}{\pgfqpoint{2.185438in}{2.200567in}}%
\pgfpathcurveto{\pgfqpoint{2.185438in}{2.211617in}}{\pgfqpoint{2.181047in}{2.222216in}}{\pgfqpoint{2.173234in}{2.230029in}}%
\pgfpathcurveto{\pgfqpoint{2.165420in}{2.237843in}}{\pgfqpoint{2.154821in}{2.242233in}}{\pgfqpoint{2.143771in}{2.242233in}}%
\pgfpathcurveto{\pgfqpoint{2.132721in}{2.242233in}}{\pgfqpoint{2.122122in}{2.237843in}}{\pgfqpoint{2.114308in}{2.230029in}}%
\pgfpathcurveto{\pgfqpoint{2.106494in}{2.222216in}}{\pgfqpoint{2.102104in}{2.211617in}}{\pgfqpoint{2.102104in}{2.200567in}}%
\pgfpathcurveto{\pgfqpoint{2.102104in}{2.189517in}}{\pgfqpoint{2.106494in}{2.178918in}}{\pgfqpoint{2.114308in}{2.171104in}}%
\pgfpathcurveto{\pgfqpoint{2.122122in}{2.163290in}}{\pgfqpoint{2.132721in}{2.158900in}}{\pgfqpoint{2.143771in}{2.158900in}}%
\pgfpathclose%
\pgfusepath{stroke,fill}%
\end{pgfscope}%
\begin{pgfscope}%
\pgfpathrectangle{\pgfqpoint{0.564660in}{0.521603in}}{\pgfqpoint{4.650000in}{3.020000in}}%
\pgfusepath{clip}%
\pgfsetbuttcap%
\pgfsetroundjoin%
\definecolor{currentfill}{rgb}{1.000000,1.000000,1.000000}%
\pgfsetfillcolor{currentfill}%
\pgfsetlinewidth{1.003750pt}%
\definecolor{currentstroke}{rgb}{0.000000,0.000000,0.000000}%
\pgfsetstrokecolor{currentstroke}%
\pgfsetdash{}{0pt}%
\pgfpathmoveto{\pgfqpoint{1.086864in}{0.741510in}}%
\pgfpathcurveto{\pgfqpoint{1.097914in}{0.741510in}}{\pgfqpoint{1.108513in}{0.745901in}}{\pgfqpoint{1.116327in}{0.753714in}}%
\pgfpathcurveto{\pgfqpoint{1.124141in}{0.761528in}}{\pgfqpoint{1.128531in}{0.772127in}}{\pgfqpoint{1.128531in}{0.783177in}}%
\pgfpathcurveto{\pgfqpoint{1.128531in}{0.794227in}}{\pgfqpoint{1.124141in}{0.804826in}}{\pgfqpoint{1.116327in}{0.812640in}}%
\pgfpathcurveto{\pgfqpoint{1.108513in}{0.820453in}}{\pgfqpoint{1.097914in}{0.824844in}}{\pgfqpoint{1.086864in}{0.824844in}}%
\pgfpathcurveto{\pgfqpoint{1.075814in}{0.824844in}}{\pgfqpoint{1.065215in}{0.820453in}}{\pgfqpoint{1.057401in}{0.812640in}}%
\pgfpathcurveto{\pgfqpoint{1.049588in}{0.804826in}}{\pgfqpoint{1.045197in}{0.794227in}}{\pgfqpoint{1.045197in}{0.783177in}}%
\pgfpathcurveto{\pgfqpoint{1.045197in}{0.772127in}}{\pgfqpoint{1.049588in}{0.761528in}}{\pgfqpoint{1.057401in}{0.753714in}}%
\pgfpathcurveto{\pgfqpoint{1.065215in}{0.745901in}}{\pgfqpoint{1.075814in}{0.741510in}}{\pgfqpoint{1.086864in}{0.741510in}}%
\pgfpathclose%
\pgfusepath{stroke,fill}%
\end{pgfscope}%
\begin{pgfscope}%
\pgfpathrectangle{\pgfqpoint{0.564660in}{0.521603in}}{\pgfqpoint{4.650000in}{3.020000in}}%
\pgfusepath{clip}%
\pgfsetbuttcap%
\pgfsetroundjoin%
\definecolor{currentfill}{rgb}{1.000000,1.000000,1.000000}%
\pgfsetfillcolor{currentfill}%
\pgfsetlinewidth{1.003750pt}%
\definecolor{currentstroke}{rgb}{0.000000,0.000000,0.000000}%
\pgfsetstrokecolor{currentstroke}%
\pgfsetdash{}{0pt}%
\pgfpathmoveto{\pgfqpoint{0.785977in}{1.118987in}}%
\pgfpathcurveto{\pgfqpoint{0.797027in}{1.118987in}}{\pgfqpoint{0.807626in}{1.123377in}}{\pgfqpoint{0.815440in}{1.131190in}}%
\pgfpathcurveto{\pgfqpoint{0.823253in}{1.139004in}}{\pgfqpoint{0.827643in}{1.149603in}}{\pgfqpoint{0.827643in}{1.160653in}}%
\pgfpathcurveto{\pgfqpoint{0.827643in}{1.171703in}}{\pgfqpoint{0.823253in}{1.182302in}}{\pgfqpoint{0.815440in}{1.190116in}}%
\pgfpathcurveto{\pgfqpoint{0.807626in}{1.197930in}}{\pgfqpoint{0.797027in}{1.202320in}}{\pgfqpoint{0.785977in}{1.202320in}}%
\pgfpathcurveto{\pgfqpoint{0.774927in}{1.202320in}}{\pgfqpoint{0.764328in}{1.197930in}}{\pgfqpoint{0.756514in}{1.190116in}}%
\pgfpathcurveto{\pgfqpoint{0.748700in}{1.182302in}}{\pgfqpoint{0.744310in}{1.171703in}}{\pgfqpoint{0.744310in}{1.160653in}}%
\pgfpathcurveto{\pgfqpoint{0.744310in}{1.149603in}}{\pgfqpoint{0.748700in}{1.139004in}}{\pgfqpoint{0.756514in}{1.131190in}}%
\pgfpathcurveto{\pgfqpoint{0.764328in}{1.123377in}}{\pgfqpoint{0.774927in}{1.118987in}}{\pgfqpoint{0.785977in}{1.118987in}}%
\pgfpathclose%
\pgfusepath{stroke,fill}%
\end{pgfscope}%
\begin{pgfscope}%
\pgfpathrectangle{\pgfqpoint{0.564660in}{0.521603in}}{\pgfqpoint{4.650000in}{3.020000in}}%
\pgfusepath{clip}%
\pgfsetbuttcap%
\pgfsetroundjoin%
\definecolor{currentfill}{rgb}{1.000000,1.000000,1.000000}%
\pgfsetfillcolor{currentfill}%
\pgfsetlinewidth{1.003750pt}%
\definecolor{currentstroke}{rgb}{0.000000,0.000000,0.000000}%
\pgfsetstrokecolor{currentstroke}%
\pgfsetdash{}{0pt}%
\pgfpathmoveto{\pgfqpoint{2.036724in}{2.941966in}}%
\pgfpathcurveto{\pgfqpoint{2.047774in}{2.941966in}}{\pgfqpoint{2.058373in}{2.946356in}}{\pgfqpoint{2.066187in}{2.954170in}}%
\pgfpathcurveto{\pgfqpoint{2.074000in}{2.961983in}}{\pgfqpoint{2.078390in}{2.972582in}}{\pgfqpoint{2.078390in}{2.983632in}}%
\pgfpathcurveto{\pgfqpoint{2.078390in}{2.994682in}}{\pgfqpoint{2.074000in}{3.005282in}}{\pgfqpoint{2.066187in}{3.013095in}}%
\pgfpathcurveto{\pgfqpoint{2.058373in}{3.020909in}}{\pgfqpoint{2.047774in}{3.025299in}}{\pgfqpoint{2.036724in}{3.025299in}}%
\pgfpathcurveto{\pgfqpoint{2.025674in}{3.025299in}}{\pgfqpoint{2.015075in}{3.020909in}}{\pgfqpoint{2.007261in}{3.013095in}}%
\pgfpathcurveto{\pgfqpoint{1.999447in}{3.005282in}}{\pgfqpoint{1.995057in}{2.994682in}}{\pgfqpoint{1.995057in}{2.983632in}}%
\pgfpathcurveto{\pgfqpoint{1.995057in}{2.972582in}}{\pgfqpoint{1.999447in}{2.961983in}}{\pgfqpoint{2.007261in}{2.954170in}}%
\pgfpathcurveto{\pgfqpoint{2.015075in}{2.946356in}}{\pgfqpoint{2.025674in}{2.941966in}}{\pgfqpoint{2.036724in}{2.941966in}}%
\pgfpathclose%
\pgfusepath{stroke,fill}%
\end{pgfscope}%
\begin{pgfscope}%
\pgfpathrectangle{\pgfqpoint{0.564660in}{0.521603in}}{\pgfqpoint{4.650000in}{3.020000in}}%
\pgfusepath{clip}%
\pgfsetbuttcap%
\pgfsetroundjoin%
\definecolor{currentfill}{rgb}{1.000000,1.000000,1.000000}%
\pgfsetfillcolor{currentfill}%
\pgfsetlinewidth{1.003750pt}%
\definecolor{currentstroke}{rgb}{0.000000,0.000000,0.000000}%
\pgfsetstrokecolor{currentstroke}%
\pgfsetdash{}{0pt}%
\pgfpathmoveto{\pgfqpoint{1.309364in}{1.315672in}}%
\pgfpathcurveto{\pgfqpoint{1.320414in}{1.315672in}}{\pgfqpoint{1.331013in}{1.320062in}}{\pgfqpoint{1.338826in}{1.327876in}}%
\pgfpathcurveto{\pgfqpoint{1.346640in}{1.335690in}}{\pgfqpoint{1.351030in}{1.346289in}}{\pgfqpoint{1.351030in}{1.357339in}}%
\pgfpathcurveto{\pgfqpoint{1.351030in}{1.368389in}}{\pgfqpoint{1.346640in}{1.378988in}}{\pgfqpoint{1.338826in}{1.386802in}}%
\pgfpathcurveto{\pgfqpoint{1.331013in}{1.394615in}}{\pgfqpoint{1.320414in}{1.399006in}}{\pgfqpoint{1.309364in}{1.399006in}}%
\pgfpathcurveto{\pgfqpoint{1.298314in}{1.399006in}}{\pgfqpoint{1.287715in}{1.394615in}}{\pgfqpoint{1.279901in}{1.386802in}}%
\pgfpathcurveto{\pgfqpoint{1.272087in}{1.378988in}}{\pgfqpoint{1.267697in}{1.368389in}}{\pgfqpoint{1.267697in}{1.357339in}}%
\pgfpathcurveto{\pgfqpoint{1.267697in}{1.346289in}}{\pgfqpoint{1.272087in}{1.335690in}}{\pgfqpoint{1.279901in}{1.327876in}}%
\pgfpathcurveto{\pgfqpoint{1.287715in}{1.320062in}}{\pgfqpoint{1.298314in}{1.315672in}}{\pgfqpoint{1.309364in}{1.315672in}}%
\pgfpathclose%
\pgfusepath{stroke,fill}%
\end{pgfscope}%
\begin{pgfscope}%
\pgfsetbuttcap%
\pgfsetroundjoin%
\definecolor{currentfill}{rgb}{0.000000,0.000000,0.000000}%
\pgfsetfillcolor{currentfill}%
\pgfsetlinewidth{0.803000pt}%
\definecolor{currentstroke}{rgb}{0.000000,0.000000,0.000000}%
\pgfsetstrokecolor{currentstroke}%
\pgfsetdash{}{0pt}%
\pgfsys@defobject{currentmarker}{\pgfqpoint{0.000000in}{-0.048611in}}{\pgfqpoint{0.000000in}{0.000000in}}{%
\pgfpathmoveto{\pgfqpoint{0.000000in}{0.000000in}}%
\pgfpathlineto{\pgfqpoint{0.000000in}{-0.048611in}}%
\pgfusepath{stroke,fill}%
}%
\begin{pgfscope}%
\pgfsys@transformshift{0.667447in}{0.521603in}%
\pgfsys@useobject{currentmarker}{}%
\end{pgfscope}%
\end{pgfscope}%
\begin{pgfscope}%
\definecolor{textcolor}{rgb}{0.000000,0.000000,0.000000}%
\pgfsetstrokecolor{textcolor}%
\pgfsetfillcolor{textcolor}%
\pgftext[x=0.667447in,y=0.424381in,,top]{\color{textcolor}\rmfamily\fontsize{10.000000}{12.000000}\selectfont \(\displaystyle 0\)}%
\end{pgfscope}%
\begin{pgfscope}%
\pgfsetbuttcap%
\pgfsetroundjoin%
\definecolor{currentfill}{rgb}{0.000000,0.000000,0.000000}%
\pgfsetfillcolor{currentfill}%
\pgfsetlinewidth{0.803000pt}%
\definecolor{currentstroke}{rgb}{0.000000,0.000000,0.000000}%
\pgfsetstrokecolor{currentstroke}%
\pgfsetdash{}{0pt}%
\pgfsys@defobject{currentmarker}{\pgfqpoint{0.000000in}{-0.048611in}}{\pgfqpoint{0.000000in}{0.000000in}}{%
\pgfpathmoveto{\pgfqpoint{0.000000in}{0.000000in}}%
\pgfpathlineto{\pgfqpoint{0.000000in}{-0.048611in}}%
\pgfusepath{stroke,fill}%
}%
\begin{pgfscope}%
\pgfsys@transformshift{1.348050in}{0.521603in}%
\pgfsys@useobject{currentmarker}{}%
\end{pgfscope}%
\end{pgfscope}%
\begin{pgfscope}%
\definecolor{textcolor}{rgb}{0.000000,0.000000,0.000000}%
\pgfsetstrokecolor{textcolor}%
\pgfsetfillcolor{textcolor}%
\pgftext[x=1.348050in,y=0.424381in,,top]{\color{textcolor}\rmfamily\fontsize{10.000000}{12.000000}\selectfont \(\displaystyle 5\)}%
\end{pgfscope}%
\begin{pgfscope}%
\pgfsetbuttcap%
\pgfsetroundjoin%
\definecolor{currentfill}{rgb}{0.000000,0.000000,0.000000}%
\pgfsetfillcolor{currentfill}%
\pgfsetlinewidth{0.803000pt}%
\definecolor{currentstroke}{rgb}{0.000000,0.000000,0.000000}%
\pgfsetstrokecolor{currentstroke}%
\pgfsetdash{}{0pt}%
\pgfsys@defobject{currentmarker}{\pgfqpoint{0.000000in}{-0.048611in}}{\pgfqpoint{0.000000in}{0.000000in}}{%
\pgfpathmoveto{\pgfqpoint{0.000000in}{0.000000in}}%
\pgfpathlineto{\pgfqpoint{0.000000in}{-0.048611in}}%
\pgfusepath{stroke,fill}%
}%
\begin{pgfscope}%
\pgfsys@transformshift{2.028653in}{0.521603in}%
\pgfsys@useobject{currentmarker}{}%
\end{pgfscope}%
\end{pgfscope}%
\begin{pgfscope}%
\definecolor{textcolor}{rgb}{0.000000,0.000000,0.000000}%
\pgfsetstrokecolor{textcolor}%
\pgfsetfillcolor{textcolor}%
\pgftext[x=2.028653in,y=0.424381in,,top]{\color{textcolor}\rmfamily\fontsize{10.000000}{12.000000}\selectfont \(\displaystyle 10\)}%
\end{pgfscope}%
\begin{pgfscope}%
\pgfsetbuttcap%
\pgfsetroundjoin%
\definecolor{currentfill}{rgb}{0.000000,0.000000,0.000000}%
\pgfsetfillcolor{currentfill}%
\pgfsetlinewidth{0.803000pt}%
\definecolor{currentstroke}{rgb}{0.000000,0.000000,0.000000}%
\pgfsetstrokecolor{currentstroke}%
\pgfsetdash{}{0pt}%
\pgfsys@defobject{currentmarker}{\pgfqpoint{0.000000in}{-0.048611in}}{\pgfqpoint{0.000000in}{0.000000in}}{%
\pgfpathmoveto{\pgfqpoint{0.000000in}{0.000000in}}%
\pgfpathlineto{\pgfqpoint{0.000000in}{-0.048611in}}%
\pgfusepath{stroke,fill}%
}%
\begin{pgfscope}%
\pgfsys@transformshift{2.709256in}{0.521603in}%
\pgfsys@useobject{currentmarker}{}%
\end{pgfscope}%
\end{pgfscope}%
\begin{pgfscope}%
\definecolor{textcolor}{rgb}{0.000000,0.000000,0.000000}%
\pgfsetstrokecolor{textcolor}%
\pgfsetfillcolor{textcolor}%
\pgftext[x=2.709256in,y=0.424381in,,top]{\color{textcolor}\rmfamily\fontsize{10.000000}{12.000000}\selectfont \(\displaystyle 15\)}%
\end{pgfscope}%
\begin{pgfscope}%
\pgfsetbuttcap%
\pgfsetroundjoin%
\definecolor{currentfill}{rgb}{0.000000,0.000000,0.000000}%
\pgfsetfillcolor{currentfill}%
\pgfsetlinewidth{0.803000pt}%
\definecolor{currentstroke}{rgb}{0.000000,0.000000,0.000000}%
\pgfsetstrokecolor{currentstroke}%
\pgfsetdash{}{0pt}%
\pgfsys@defobject{currentmarker}{\pgfqpoint{0.000000in}{-0.048611in}}{\pgfqpoint{0.000000in}{0.000000in}}{%
\pgfpathmoveto{\pgfqpoint{0.000000in}{0.000000in}}%
\pgfpathlineto{\pgfqpoint{0.000000in}{-0.048611in}}%
\pgfusepath{stroke,fill}%
}%
\begin{pgfscope}%
\pgfsys@transformshift{3.389859in}{0.521603in}%
\pgfsys@useobject{currentmarker}{}%
\end{pgfscope}%
\end{pgfscope}%
\begin{pgfscope}%
\definecolor{textcolor}{rgb}{0.000000,0.000000,0.000000}%
\pgfsetstrokecolor{textcolor}%
\pgfsetfillcolor{textcolor}%
\pgftext[x=3.389859in,y=0.424381in,,top]{\color{textcolor}\rmfamily\fontsize{10.000000}{12.000000}\selectfont \(\displaystyle 20\)}%
\end{pgfscope}%
\begin{pgfscope}%
\pgfsetbuttcap%
\pgfsetroundjoin%
\definecolor{currentfill}{rgb}{0.000000,0.000000,0.000000}%
\pgfsetfillcolor{currentfill}%
\pgfsetlinewidth{0.803000pt}%
\definecolor{currentstroke}{rgb}{0.000000,0.000000,0.000000}%
\pgfsetstrokecolor{currentstroke}%
\pgfsetdash{}{0pt}%
\pgfsys@defobject{currentmarker}{\pgfqpoint{0.000000in}{-0.048611in}}{\pgfqpoint{0.000000in}{0.000000in}}{%
\pgfpathmoveto{\pgfqpoint{0.000000in}{0.000000in}}%
\pgfpathlineto{\pgfqpoint{0.000000in}{-0.048611in}}%
\pgfusepath{stroke,fill}%
}%
\begin{pgfscope}%
\pgfsys@transformshift{4.070462in}{0.521603in}%
\pgfsys@useobject{currentmarker}{}%
\end{pgfscope}%
\end{pgfscope}%
\begin{pgfscope}%
\definecolor{textcolor}{rgb}{0.000000,0.000000,0.000000}%
\pgfsetstrokecolor{textcolor}%
\pgfsetfillcolor{textcolor}%
\pgftext[x=4.070462in,y=0.424381in,,top]{\color{textcolor}\rmfamily\fontsize{10.000000}{12.000000}\selectfont \(\displaystyle 25\)}%
\end{pgfscope}%
\begin{pgfscope}%
\pgfsetbuttcap%
\pgfsetroundjoin%
\definecolor{currentfill}{rgb}{0.000000,0.000000,0.000000}%
\pgfsetfillcolor{currentfill}%
\pgfsetlinewidth{0.803000pt}%
\definecolor{currentstroke}{rgb}{0.000000,0.000000,0.000000}%
\pgfsetstrokecolor{currentstroke}%
\pgfsetdash{}{0pt}%
\pgfsys@defobject{currentmarker}{\pgfqpoint{0.000000in}{-0.048611in}}{\pgfqpoint{0.000000in}{0.000000in}}{%
\pgfpathmoveto{\pgfqpoint{0.000000in}{0.000000in}}%
\pgfpathlineto{\pgfqpoint{0.000000in}{-0.048611in}}%
\pgfusepath{stroke,fill}%
}%
\begin{pgfscope}%
\pgfsys@transformshift{4.751065in}{0.521603in}%
\pgfsys@useobject{currentmarker}{}%
\end{pgfscope}%
\end{pgfscope}%
\begin{pgfscope}%
\definecolor{textcolor}{rgb}{0.000000,0.000000,0.000000}%
\pgfsetstrokecolor{textcolor}%
\pgfsetfillcolor{textcolor}%
\pgftext[x=4.751065in,y=0.424381in,,top]{\color{textcolor}\rmfamily\fontsize{10.000000}{12.000000}\selectfont \(\displaystyle 30\)}%
\end{pgfscope}%
\begin{pgfscope}%
\definecolor{textcolor}{rgb}{0.000000,0.000000,0.000000}%
\pgfsetstrokecolor{textcolor}%
\pgfsetfillcolor{textcolor}%
\pgftext[x=2.889660in,y=0.234413in,,top]{\color{textcolor}\rmfamily\fontsize{10.000000}{12.000000}\selectfont \(\displaystyle \mathbf{E}\mbox{u}\)}%
\end{pgfscope}%
\begin{pgfscope}%
\pgfsetbuttcap%
\pgfsetroundjoin%
\definecolor{currentfill}{rgb}{0.000000,0.000000,0.000000}%
\pgfsetfillcolor{currentfill}%
\pgfsetlinewidth{0.803000pt}%
\definecolor{currentstroke}{rgb}{0.000000,0.000000,0.000000}%
\pgfsetstrokecolor{currentstroke}%
\pgfsetdash{}{0pt}%
\pgfsys@defobject{currentmarker}{\pgfqpoint{-0.048611in}{0.000000in}}{\pgfqpoint{0.000000in}{0.000000in}}{%
\pgfpathmoveto{\pgfqpoint{0.000000in}{0.000000in}}%
\pgfpathlineto{\pgfqpoint{-0.048611in}{0.000000in}}%
\pgfusepath{stroke,fill}%
}%
\begin{pgfscope}%
\pgfsys@transformshift{0.564660in}{1.017529in}%
\pgfsys@useobject{currentmarker}{}%
\end{pgfscope}%
\end{pgfscope}%
\begin{pgfscope}%
\definecolor{textcolor}{rgb}{0.000000,0.000000,0.000000}%
\pgfsetstrokecolor{textcolor}%
\pgfsetfillcolor{textcolor}%
\pgftext[x=0.289968in,y=0.964768in,left,base]{\color{textcolor}\rmfamily\fontsize{10.000000}{12.000000}\selectfont \(\displaystyle 0.3\)}%
\end{pgfscope}%
\begin{pgfscope}%
\pgfsetbuttcap%
\pgfsetroundjoin%
\definecolor{currentfill}{rgb}{0.000000,0.000000,0.000000}%
\pgfsetfillcolor{currentfill}%
\pgfsetlinewidth{0.803000pt}%
\definecolor{currentstroke}{rgb}{0.000000,0.000000,0.000000}%
\pgfsetstrokecolor{currentstroke}%
\pgfsetdash{}{0pt}%
\pgfsys@defobject{currentmarker}{\pgfqpoint{-0.048611in}{0.000000in}}{\pgfqpoint{0.000000in}{0.000000in}}{%
\pgfpathmoveto{\pgfqpoint{0.000000in}{0.000000in}}%
\pgfpathlineto{\pgfqpoint{-0.048611in}{0.000000in}}%
\pgfusepath{stroke,fill}%
}%
\begin{pgfscope}%
\pgfsys@transformshift{0.564660in}{1.618847in}%
\pgfsys@useobject{currentmarker}{}%
\end{pgfscope}%
\end{pgfscope}%
\begin{pgfscope}%
\definecolor{textcolor}{rgb}{0.000000,0.000000,0.000000}%
\pgfsetstrokecolor{textcolor}%
\pgfsetfillcolor{textcolor}%
\pgftext[x=0.289968in,y=1.566085in,left,base]{\color{textcolor}\rmfamily\fontsize{10.000000}{12.000000}\selectfont \(\displaystyle 0.4\)}%
\end{pgfscope}%
\begin{pgfscope}%
\pgfsetbuttcap%
\pgfsetroundjoin%
\definecolor{currentfill}{rgb}{0.000000,0.000000,0.000000}%
\pgfsetfillcolor{currentfill}%
\pgfsetlinewidth{0.803000pt}%
\definecolor{currentstroke}{rgb}{0.000000,0.000000,0.000000}%
\pgfsetstrokecolor{currentstroke}%
\pgfsetdash{}{0pt}%
\pgfsys@defobject{currentmarker}{\pgfqpoint{-0.048611in}{0.000000in}}{\pgfqpoint{0.000000in}{0.000000in}}{%
\pgfpathmoveto{\pgfqpoint{0.000000in}{0.000000in}}%
\pgfpathlineto{\pgfqpoint{-0.048611in}{0.000000in}}%
\pgfusepath{stroke,fill}%
}%
\begin{pgfscope}%
\pgfsys@transformshift{0.564660in}{2.220164in}%
\pgfsys@useobject{currentmarker}{}%
\end{pgfscope}%
\end{pgfscope}%
\begin{pgfscope}%
\definecolor{textcolor}{rgb}{0.000000,0.000000,0.000000}%
\pgfsetstrokecolor{textcolor}%
\pgfsetfillcolor{textcolor}%
\pgftext[x=0.289968in,y=2.167403in,left,base]{\color{textcolor}\rmfamily\fontsize{10.000000}{12.000000}\selectfont \(\displaystyle 0.5\)}%
\end{pgfscope}%
\begin{pgfscope}%
\pgfsetbuttcap%
\pgfsetroundjoin%
\definecolor{currentfill}{rgb}{0.000000,0.000000,0.000000}%
\pgfsetfillcolor{currentfill}%
\pgfsetlinewidth{0.803000pt}%
\definecolor{currentstroke}{rgb}{0.000000,0.000000,0.000000}%
\pgfsetstrokecolor{currentstroke}%
\pgfsetdash{}{0pt}%
\pgfsys@defobject{currentmarker}{\pgfqpoint{-0.048611in}{0.000000in}}{\pgfqpoint{0.000000in}{0.000000in}}{%
\pgfpathmoveto{\pgfqpoint{0.000000in}{0.000000in}}%
\pgfpathlineto{\pgfqpoint{-0.048611in}{0.000000in}}%
\pgfusepath{stroke,fill}%
}%
\begin{pgfscope}%
\pgfsys@transformshift{0.564660in}{2.821481in}%
\pgfsys@useobject{currentmarker}{}%
\end{pgfscope}%
\end{pgfscope}%
\begin{pgfscope}%
\definecolor{textcolor}{rgb}{0.000000,0.000000,0.000000}%
\pgfsetstrokecolor{textcolor}%
\pgfsetfillcolor{textcolor}%
\pgftext[x=0.289968in,y=2.768720in,left,base]{\color{textcolor}\rmfamily\fontsize{10.000000}{12.000000}\selectfont \(\displaystyle 0.6\)}%
\end{pgfscope}%
\begin{pgfscope}%
\pgfsetbuttcap%
\pgfsetroundjoin%
\definecolor{currentfill}{rgb}{0.000000,0.000000,0.000000}%
\pgfsetfillcolor{currentfill}%
\pgfsetlinewidth{0.803000pt}%
\definecolor{currentstroke}{rgb}{0.000000,0.000000,0.000000}%
\pgfsetstrokecolor{currentstroke}%
\pgfsetdash{}{0pt}%
\pgfsys@defobject{currentmarker}{\pgfqpoint{-0.048611in}{0.000000in}}{\pgfqpoint{0.000000in}{0.000000in}}{%
\pgfpathmoveto{\pgfqpoint{0.000000in}{0.000000in}}%
\pgfpathlineto{\pgfqpoint{-0.048611in}{0.000000in}}%
\pgfusepath{stroke,fill}%
}%
\begin{pgfscope}%
\pgfsys@transformshift{0.564660in}{3.422799in}%
\pgfsys@useobject{currentmarker}{}%
\end{pgfscope}%
\end{pgfscope}%
\begin{pgfscope}%
\definecolor{textcolor}{rgb}{0.000000,0.000000,0.000000}%
\pgfsetstrokecolor{textcolor}%
\pgfsetfillcolor{textcolor}%
\pgftext[x=0.289968in,y=3.370037in,left,base]{\color{textcolor}\rmfamily\fontsize{10.000000}{12.000000}\selectfont \(\displaystyle 0.7\)}%
\end{pgfscope}%
\begin{pgfscope}%
\definecolor{textcolor}{rgb}{0.000000,0.000000,0.000000}%
\pgfsetstrokecolor{textcolor}%
\pgfsetfillcolor{textcolor}%
\pgftext[x=0.234413in,y=2.031603in,,bottom,rotate=90.000000]{\color{textcolor}\rmfamily\fontsize{10.000000}{12.000000}\selectfont \(\displaystyle \mathbf{I}\mbox{g}\)}%
\end{pgfscope}%
\begin{pgfscope}%
\pgfpathrectangle{\pgfqpoint{0.564660in}{0.521603in}}{\pgfqpoint{4.650000in}{3.020000in}}%
\pgfusepath{clip}%
\pgfsetrectcap%
\pgfsetroundjoin%
\pgfsetlinewidth{1.505625pt}%
\definecolor{currentstroke}{rgb}{0.000000,0.000000,0.000000}%
\pgfsetstrokecolor{currentstroke}%
\pgfsetstrokeopacity{0.300000}%
\pgfsetdash{}{0pt}%
\pgfpathmoveto{\pgfqpoint{0.816771in}{1.337601in}}%
\pgfpathlineto{\pgfqpoint{3.858085in}{2.234101in}}%
\pgfpathlineto{\pgfqpoint{4.993344in}{2.568746in}}%
\pgfpathlineto{\pgfqpoint{1.914828in}{1.661279in}}%
\pgfpathlineto{\pgfqpoint{2.773519in}{1.914399in}}%
\pgfpathlineto{\pgfqpoint{2.234616in}{1.755544in}}%
\pgfpathlineto{\pgfqpoint{0.819079in}{1.338281in}}%
\pgfpathlineto{\pgfqpoint{1.363796in}{1.498849in}}%
\pgfpathlineto{\pgfqpoint{1.096147in}{1.419953in}}%
\pgfpathlineto{\pgfqpoint{0.887904in}{1.358569in}}%
\pgfpathlineto{\pgfqpoint{2.143771in}{1.728766in}}%
\pgfpathlineto{\pgfqpoint{1.086864in}{1.417217in}}%
\pgfpathlineto{\pgfqpoint{0.785977in}{1.328523in}}%
\pgfpathlineto{\pgfqpoint{2.036724in}{1.697211in}}%
\pgfpathlineto{\pgfqpoint{1.309364in}{1.482804in}}%
\pgfusepath{stroke}%
\end{pgfscope}%
\begin{pgfscope}%
\pgfsetrectcap%
\pgfsetmiterjoin%
\pgfsetlinewidth{0.803000pt}%
\definecolor{currentstroke}{rgb}{0.501961,0.501961,0.501961}%
\pgfsetstrokecolor{currentstroke}%
\pgfsetdash{}{0pt}%
\pgfpathmoveto{\pgfqpoint{0.564660in}{0.521603in}}%
\pgfpathlineto{\pgfqpoint{0.564660in}{3.541603in}}%
\pgfusepath{stroke}%
\end{pgfscope}%
\begin{pgfscope}%
\pgfsetrectcap%
\pgfsetmiterjoin%
\pgfsetlinewidth{0.803000pt}%
\definecolor{currentstroke}{rgb}{0.501961,0.501961,0.501961}%
\pgfsetstrokecolor{currentstroke}%
\pgfsetdash{}{0pt}%
\pgfpathmoveto{\pgfqpoint{5.214660in}{0.521603in}}%
\pgfpathlineto{\pgfqpoint{5.214660in}{3.541603in}}%
\pgfusepath{stroke}%
\end{pgfscope}%
\begin{pgfscope}%
\pgfsetrectcap%
\pgfsetmiterjoin%
\pgfsetlinewidth{0.803000pt}%
\definecolor{currentstroke}{rgb}{0.501961,0.501961,0.501961}%
\pgfsetstrokecolor{currentstroke}%
\pgfsetdash{}{0pt}%
\pgfpathmoveto{\pgfqpoint{0.564660in}{0.521603in}}%
\pgfpathlineto{\pgfqpoint{5.214660in}{0.521603in}}%
\pgfusepath{stroke}%
\end{pgfscope}%
\begin{pgfscope}%
\pgfsetrectcap%
\pgfsetmiterjoin%
\pgfsetlinewidth{0.803000pt}%
\definecolor{currentstroke}{rgb}{0.501961,0.501961,0.501961}%
\pgfsetstrokecolor{currentstroke}%
\pgfsetdash{}{0pt}%
\pgfpathmoveto{\pgfqpoint{0.564660in}{3.541603in}}%
\pgfpathlineto{\pgfqpoint{5.214660in}{3.541603in}}%
\pgfusepath{stroke}%
\end{pgfscope}%
\begin{pgfscope}%
\pgfsetbuttcap%
\pgfsetmiterjoin%
\definecolor{currentfill}{rgb}{1.000000,1.000000,1.000000}%
\pgfsetfillcolor{currentfill}%
\pgfsetfillopacity{0.800000}%
\pgfsetlinewidth{1.003750pt}%
\definecolor{currentstroke}{rgb}{0.800000,0.800000,0.800000}%
\pgfsetstrokecolor{currentstroke}%
\pgfsetstrokeopacity{0.800000}%
\pgfsetdash{}{0pt}%
\pgfpathmoveto{\pgfqpoint{2.661499in}{0.591048in}}%
\pgfpathlineto{\pgfqpoint{5.117438in}{0.591048in}}%
\pgfpathquadraticcurveto{\pgfqpoint{5.145216in}{0.591048in}}{\pgfqpoint{5.145216in}{0.618826in}}%
\pgfpathlineto{\pgfqpoint{5.145216in}{0.825238in}}%
\pgfpathquadraticcurveto{\pgfqpoint{5.145216in}{0.853016in}}{\pgfqpoint{5.117438in}{0.853016in}}%
\pgfpathlineto{\pgfqpoint{2.661499in}{0.853016in}}%
\pgfpathquadraticcurveto{\pgfqpoint{2.633721in}{0.853016in}}{\pgfqpoint{2.633721in}{0.825238in}}%
\pgfpathlineto{\pgfqpoint{2.633721in}{0.618826in}}%
\pgfpathquadraticcurveto{\pgfqpoint{2.633721in}{0.591048in}}{\pgfqpoint{2.661499in}{0.591048in}}%
\pgfpathclose%
\pgfusepath{stroke,fill}%
\end{pgfscope}%
\begin{pgfscope}%
\pgfsetrectcap%
\pgfsetroundjoin%
\pgfsetlinewidth{1.505625pt}%
\definecolor{currentstroke}{rgb}{0.000000,0.000000,0.000000}%
\pgfsetstrokecolor{currentstroke}%
\pgfsetstrokeopacity{0.300000}%
\pgfsetdash{}{0pt}%
\pgfpathmoveto{\pgfqpoint{2.689277in}{0.726071in}}%
\pgfpathlineto{\pgfqpoint{2.967055in}{0.726071in}}%
\pgfusepath{stroke}%
\end{pgfscope}%
\begin{pgfscope}%
\definecolor{textcolor}{rgb}{0.501961,0.501961,0.501961}%
\pgfsetstrokecolor{textcolor}%
\pgfsetfillcolor{textcolor}%
\pgftext[x=3.078166in,y=0.677460in,left,base]{\color{textcolor}\rmfamily\fontsize{10.000000}{12.000000}\selectfont \(\displaystyle \mathbf{I}\mbox{g} \approx 0.013 \mathbf{E}\mbox{u} + 0.230\), \(\displaystyle R^2=0.57\)}%
\end{pgfscope}%
\end{pgfpicture}%
\makeatother%
\endgroup%

    \caption{.\label{fig:dnumbs}}
\end{figure}

Finally we show several trajectories in the long-time scaled regime in Figure \ref{fig:series_l_ds}.
\begin{figure}[htb]
    \centering
    %% Creator: Matplotlib, PGF backend
%%
%% To include the figure in your LaTeX document, write
%%   \input{<filename>.pgf}
%%
%% Make sure the required packages are loaded in your preamble
%%   \usepackage{pgf}
%%
%% Figures using additional raster images can only be included by \input if
%% they are in the same directory as the main LaTeX file. For loading figures
%% from other directories you can use the `import` package
%%   \usepackage{import}
%% and then include the figures with
%%   \import{<path to file>}{<filename>.pgf}
%%
%% Matplotlib used the following preamble
%%   \usepackage{fontspec}
%%   \setmainfont{DejaVu Serif}
%%   \setsansfont{DejaVu Sans}
%%   \setmonofont{DejaVu Sans Mono}
%%
\begingroup%
\makeatletter%
\begin{pgfpicture}%
\pgfpathrectangle{\pgfpointorigin}{\pgfqpoint{5.665633in}{3.676603in}}%
\pgfusepath{use as bounding box, clip}%
\begin{pgfscope}%
\pgfsetbuttcap%
\pgfsetmiterjoin%
\definecolor{currentfill}{rgb}{1.000000,1.000000,1.000000}%
\pgfsetfillcolor{currentfill}%
\pgfsetlinewidth{0.000000pt}%
\definecolor{currentstroke}{rgb}{1.000000,1.000000,1.000000}%
\pgfsetstrokecolor{currentstroke}%
\pgfsetdash{}{0pt}%
\pgfpathmoveto{\pgfqpoint{0.000000in}{0.000000in}}%
\pgfpathlineto{\pgfqpoint{5.665633in}{0.000000in}}%
\pgfpathlineto{\pgfqpoint{5.665633in}{3.676603in}}%
\pgfpathlineto{\pgfqpoint{0.000000in}{3.676603in}}%
\pgfpathclose%
\pgfusepath{fill}%
\end{pgfscope}%
\begin{pgfscope}%
\pgfsetbuttcap%
\pgfsetmiterjoin%
\definecolor{currentfill}{rgb}{1.000000,1.000000,1.000000}%
\pgfsetfillcolor{currentfill}%
\pgfsetlinewidth{0.000000pt}%
\definecolor{currentstroke}{rgb}{0.000000,0.000000,0.000000}%
\pgfsetstrokecolor{currentstroke}%
\pgfsetstrokeopacity{0.000000}%
\pgfsetdash{}{0pt}%
\pgfpathmoveto{\pgfqpoint{0.526080in}{0.521603in}}%
\pgfpathlineto{\pgfqpoint{4.246080in}{0.521603in}}%
\pgfpathlineto{\pgfqpoint{4.246080in}{3.541603in}}%
\pgfpathlineto{\pgfqpoint{0.526080in}{3.541603in}}%
\pgfpathclose%
\pgfusepath{fill}%
\end{pgfscope}%
\begin{pgfscope}%
\pgfsetbuttcap%
\pgfsetroundjoin%
\definecolor{currentfill}{rgb}{0.000000,0.000000,0.000000}%
\pgfsetfillcolor{currentfill}%
\pgfsetlinewidth{0.803000pt}%
\definecolor{currentstroke}{rgb}{0.000000,0.000000,0.000000}%
\pgfsetstrokecolor{currentstroke}%
\pgfsetdash{}{0pt}%
\pgfsys@defobject{currentmarker}{\pgfqpoint{0.000000in}{-0.048611in}}{\pgfqpoint{0.000000in}{0.000000in}}{%
\pgfpathmoveto{\pgfqpoint{0.000000in}{0.000000in}}%
\pgfpathlineto{\pgfqpoint{0.000000in}{-0.048611in}}%
\pgfusepath{stroke,fill}%
}%
\begin{pgfscope}%
\pgfsys@transformshift{0.642375in}{0.521603in}%
\pgfsys@useobject{currentmarker}{}%
\end{pgfscope}%
\end{pgfscope}%
\begin{pgfscope}%
\pgftext[x=0.642375in,y=0.424381in,,top]{\rmfamily\fontsize{10.000000}{12.000000}\selectfont \(\displaystyle 0.0\)}%
\end{pgfscope}%
\begin{pgfscope}%
\pgfsetbuttcap%
\pgfsetroundjoin%
\definecolor{currentfill}{rgb}{0.000000,0.000000,0.000000}%
\pgfsetfillcolor{currentfill}%
\pgfsetlinewidth{0.803000pt}%
\definecolor{currentstroke}{rgb}{0.000000,0.000000,0.000000}%
\pgfsetstrokecolor{currentstroke}%
\pgfsetdash{}{0pt}%
\pgfsys@defobject{currentmarker}{\pgfqpoint{0.000000in}{-0.048611in}}{\pgfqpoint{0.000000in}{0.000000in}}{%
\pgfpathmoveto{\pgfqpoint{0.000000in}{0.000000in}}%
\pgfpathlineto{\pgfqpoint{0.000000in}{-0.048611in}}%
\pgfusepath{stroke,fill}%
}%
\begin{pgfscope}%
\pgfsys@transformshift{1.189375in}{0.521603in}%
\pgfsys@useobject{currentmarker}{}%
\end{pgfscope}%
\end{pgfscope}%
\begin{pgfscope}%
\pgftext[x=1.189375in,y=0.424381in,,top]{\rmfamily\fontsize{10.000000}{12.000000}\selectfont \(\displaystyle 2.5\)}%
\end{pgfscope}%
\begin{pgfscope}%
\pgfsetbuttcap%
\pgfsetroundjoin%
\definecolor{currentfill}{rgb}{0.000000,0.000000,0.000000}%
\pgfsetfillcolor{currentfill}%
\pgfsetlinewidth{0.803000pt}%
\definecolor{currentstroke}{rgb}{0.000000,0.000000,0.000000}%
\pgfsetstrokecolor{currentstroke}%
\pgfsetdash{}{0pt}%
\pgfsys@defobject{currentmarker}{\pgfqpoint{0.000000in}{-0.048611in}}{\pgfqpoint{0.000000in}{0.000000in}}{%
\pgfpathmoveto{\pgfqpoint{0.000000in}{0.000000in}}%
\pgfpathlineto{\pgfqpoint{0.000000in}{-0.048611in}}%
\pgfusepath{stroke,fill}%
}%
\begin{pgfscope}%
\pgfsys@transformshift{1.736375in}{0.521603in}%
\pgfsys@useobject{currentmarker}{}%
\end{pgfscope}%
\end{pgfscope}%
\begin{pgfscope}%
\pgftext[x=1.736375in,y=0.424381in,,top]{\rmfamily\fontsize{10.000000}{12.000000}\selectfont \(\displaystyle 5.0\)}%
\end{pgfscope}%
\begin{pgfscope}%
\pgfsetbuttcap%
\pgfsetroundjoin%
\definecolor{currentfill}{rgb}{0.000000,0.000000,0.000000}%
\pgfsetfillcolor{currentfill}%
\pgfsetlinewidth{0.803000pt}%
\definecolor{currentstroke}{rgb}{0.000000,0.000000,0.000000}%
\pgfsetstrokecolor{currentstroke}%
\pgfsetdash{}{0pt}%
\pgfsys@defobject{currentmarker}{\pgfqpoint{0.000000in}{-0.048611in}}{\pgfqpoint{0.000000in}{0.000000in}}{%
\pgfpathmoveto{\pgfqpoint{0.000000in}{0.000000in}}%
\pgfpathlineto{\pgfqpoint{0.000000in}{-0.048611in}}%
\pgfusepath{stroke,fill}%
}%
\begin{pgfscope}%
\pgfsys@transformshift{2.283375in}{0.521603in}%
\pgfsys@useobject{currentmarker}{}%
\end{pgfscope}%
\end{pgfscope}%
\begin{pgfscope}%
\pgftext[x=2.283375in,y=0.424381in,,top]{\rmfamily\fontsize{10.000000}{12.000000}\selectfont \(\displaystyle 7.5\)}%
\end{pgfscope}%
\begin{pgfscope}%
\pgfsetbuttcap%
\pgfsetroundjoin%
\definecolor{currentfill}{rgb}{0.000000,0.000000,0.000000}%
\pgfsetfillcolor{currentfill}%
\pgfsetlinewidth{0.803000pt}%
\definecolor{currentstroke}{rgb}{0.000000,0.000000,0.000000}%
\pgfsetstrokecolor{currentstroke}%
\pgfsetdash{}{0pt}%
\pgfsys@defobject{currentmarker}{\pgfqpoint{0.000000in}{-0.048611in}}{\pgfqpoint{0.000000in}{0.000000in}}{%
\pgfpathmoveto{\pgfqpoint{0.000000in}{0.000000in}}%
\pgfpathlineto{\pgfqpoint{0.000000in}{-0.048611in}}%
\pgfusepath{stroke,fill}%
}%
\begin{pgfscope}%
\pgfsys@transformshift{2.830375in}{0.521603in}%
\pgfsys@useobject{currentmarker}{}%
\end{pgfscope}%
\end{pgfscope}%
\begin{pgfscope}%
\pgftext[x=2.830375in,y=0.424381in,,top]{\rmfamily\fontsize{10.000000}{12.000000}\selectfont \(\displaystyle 10.0\)}%
\end{pgfscope}%
\begin{pgfscope}%
\pgfsetbuttcap%
\pgfsetroundjoin%
\definecolor{currentfill}{rgb}{0.000000,0.000000,0.000000}%
\pgfsetfillcolor{currentfill}%
\pgfsetlinewidth{0.803000pt}%
\definecolor{currentstroke}{rgb}{0.000000,0.000000,0.000000}%
\pgfsetstrokecolor{currentstroke}%
\pgfsetdash{}{0pt}%
\pgfsys@defobject{currentmarker}{\pgfqpoint{0.000000in}{-0.048611in}}{\pgfqpoint{0.000000in}{0.000000in}}{%
\pgfpathmoveto{\pgfqpoint{0.000000in}{0.000000in}}%
\pgfpathlineto{\pgfqpoint{0.000000in}{-0.048611in}}%
\pgfusepath{stroke,fill}%
}%
\begin{pgfscope}%
\pgfsys@transformshift{3.377375in}{0.521603in}%
\pgfsys@useobject{currentmarker}{}%
\end{pgfscope}%
\end{pgfscope}%
\begin{pgfscope}%
\pgftext[x=3.377375in,y=0.424381in,,top]{\rmfamily\fontsize{10.000000}{12.000000}\selectfont \(\displaystyle 12.5\)}%
\end{pgfscope}%
\begin{pgfscope}%
\pgfsetbuttcap%
\pgfsetroundjoin%
\definecolor{currentfill}{rgb}{0.000000,0.000000,0.000000}%
\pgfsetfillcolor{currentfill}%
\pgfsetlinewidth{0.803000pt}%
\definecolor{currentstroke}{rgb}{0.000000,0.000000,0.000000}%
\pgfsetstrokecolor{currentstroke}%
\pgfsetdash{}{0pt}%
\pgfsys@defobject{currentmarker}{\pgfqpoint{0.000000in}{-0.048611in}}{\pgfqpoint{0.000000in}{0.000000in}}{%
\pgfpathmoveto{\pgfqpoint{0.000000in}{0.000000in}}%
\pgfpathlineto{\pgfqpoint{0.000000in}{-0.048611in}}%
\pgfusepath{stroke,fill}%
}%
\begin{pgfscope}%
\pgfsys@transformshift{3.924375in}{0.521603in}%
\pgfsys@useobject{currentmarker}{}%
\end{pgfscope}%
\end{pgfscope}%
\begin{pgfscope}%
\pgftext[x=3.924375in,y=0.424381in,,top]{\rmfamily\fontsize{10.000000}{12.000000}\selectfont \(\displaystyle 15.0\)}%
\end{pgfscope}%
\begin{pgfscope}%
\pgftext[x=2.386080in,y=0.234413in,,top]{\rmfamily\fontsize{10.000000}{12.000000}\selectfont \(\displaystyle \bar{t}\)}%
\end{pgfscope}%
\begin{pgfscope}%
\pgfsetbuttcap%
\pgfsetroundjoin%
\definecolor{currentfill}{rgb}{0.000000,0.000000,0.000000}%
\pgfsetfillcolor{currentfill}%
\pgfsetlinewidth{0.803000pt}%
\definecolor{currentstroke}{rgb}{0.000000,0.000000,0.000000}%
\pgfsetstrokecolor{currentstroke}%
\pgfsetdash{}{0pt}%
\pgfsys@defobject{currentmarker}{\pgfqpoint{-0.048611in}{0.000000in}}{\pgfqpoint{0.000000in}{0.000000in}}{%
\pgfpathmoveto{\pgfqpoint{0.000000in}{0.000000in}}%
\pgfpathlineto{\pgfqpoint{-0.048611in}{0.000000in}}%
\pgfusepath{stroke,fill}%
}%
\begin{pgfscope}%
\pgfsys@transformshift{0.526080in}{0.624160in}%
\pgfsys@useobject{currentmarker}{}%
\end{pgfscope}%
\end{pgfscope}%
\begin{pgfscope}%
\pgftext[x=0.359413in,y=0.571398in,left,base]{\rmfamily\fontsize{10.000000}{12.000000}\selectfont \(\displaystyle 0\)}%
\end{pgfscope}%
\begin{pgfscope}%
\pgfsetbuttcap%
\pgfsetroundjoin%
\definecolor{currentfill}{rgb}{0.000000,0.000000,0.000000}%
\pgfsetfillcolor{currentfill}%
\pgfsetlinewidth{0.803000pt}%
\definecolor{currentstroke}{rgb}{0.000000,0.000000,0.000000}%
\pgfsetstrokecolor{currentstroke}%
\pgfsetdash{}{0pt}%
\pgfsys@defobject{currentmarker}{\pgfqpoint{-0.048611in}{0.000000in}}{\pgfqpoint{0.000000in}{0.000000in}}{%
\pgfpathmoveto{\pgfqpoint{0.000000in}{0.000000in}}%
\pgfpathlineto{\pgfqpoint{-0.048611in}{0.000000in}}%
\pgfusepath{stroke,fill}%
}%
\begin{pgfscope}%
\pgfsys@transformshift{0.526080in}{1.050912in}%
\pgfsys@useobject{currentmarker}{}%
\end{pgfscope}%
\end{pgfscope}%
\begin{pgfscope}%
\pgftext[x=0.359413in,y=0.998151in,left,base]{\rmfamily\fontsize{10.000000}{12.000000}\selectfont \(\displaystyle 2\)}%
\end{pgfscope}%
\begin{pgfscope}%
\pgfsetbuttcap%
\pgfsetroundjoin%
\definecolor{currentfill}{rgb}{0.000000,0.000000,0.000000}%
\pgfsetfillcolor{currentfill}%
\pgfsetlinewidth{0.803000pt}%
\definecolor{currentstroke}{rgb}{0.000000,0.000000,0.000000}%
\pgfsetstrokecolor{currentstroke}%
\pgfsetdash{}{0pt}%
\pgfsys@defobject{currentmarker}{\pgfqpoint{-0.048611in}{0.000000in}}{\pgfqpoint{0.000000in}{0.000000in}}{%
\pgfpathmoveto{\pgfqpoint{0.000000in}{0.000000in}}%
\pgfpathlineto{\pgfqpoint{-0.048611in}{0.000000in}}%
\pgfusepath{stroke,fill}%
}%
\begin{pgfscope}%
\pgfsys@transformshift{0.526080in}{1.477665in}%
\pgfsys@useobject{currentmarker}{}%
\end{pgfscope}%
\end{pgfscope}%
\begin{pgfscope}%
\pgftext[x=0.359413in,y=1.424903in,left,base]{\rmfamily\fontsize{10.000000}{12.000000}\selectfont \(\displaystyle 4\)}%
\end{pgfscope}%
\begin{pgfscope}%
\pgfsetbuttcap%
\pgfsetroundjoin%
\definecolor{currentfill}{rgb}{0.000000,0.000000,0.000000}%
\pgfsetfillcolor{currentfill}%
\pgfsetlinewidth{0.803000pt}%
\definecolor{currentstroke}{rgb}{0.000000,0.000000,0.000000}%
\pgfsetstrokecolor{currentstroke}%
\pgfsetdash{}{0pt}%
\pgfsys@defobject{currentmarker}{\pgfqpoint{-0.048611in}{0.000000in}}{\pgfqpoint{0.000000in}{0.000000in}}{%
\pgfpathmoveto{\pgfqpoint{0.000000in}{0.000000in}}%
\pgfpathlineto{\pgfqpoint{-0.048611in}{0.000000in}}%
\pgfusepath{stroke,fill}%
}%
\begin{pgfscope}%
\pgfsys@transformshift{0.526080in}{1.904417in}%
\pgfsys@useobject{currentmarker}{}%
\end{pgfscope}%
\end{pgfscope}%
\begin{pgfscope}%
\pgftext[x=0.359413in,y=1.851656in,left,base]{\rmfamily\fontsize{10.000000}{12.000000}\selectfont \(\displaystyle 6\)}%
\end{pgfscope}%
\begin{pgfscope}%
\pgfsetbuttcap%
\pgfsetroundjoin%
\definecolor{currentfill}{rgb}{0.000000,0.000000,0.000000}%
\pgfsetfillcolor{currentfill}%
\pgfsetlinewidth{0.803000pt}%
\definecolor{currentstroke}{rgb}{0.000000,0.000000,0.000000}%
\pgfsetstrokecolor{currentstroke}%
\pgfsetdash{}{0pt}%
\pgfsys@defobject{currentmarker}{\pgfqpoint{-0.048611in}{0.000000in}}{\pgfqpoint{0.000000in}{0.000000in}}{%
\pgfpathmoveto{\pgfqpoint{0.000000in}{0.000000in}}%
\pgfpathlineto{\pgfqpoint{-0.048611in}{0.000000in}}%
\pgfusepath{stroke,fill}%
}%
\begin{pgfscope}%
\pgfsys@transformshift{0.526080in}{2.331170in}%
\pgfsys@useobject{currentmarker}{}%
\end{pgfscope}%
\end{pgfscope}%
\begin{pgfscope}%
\pgftext[x=0.359413in,y=2.278408in,left,base]{\rmfamily\fontsize{10.000000}{12.000000}\selectfont \(\displaystyle 8\)}%
\end{pgfscope}%
\begin{pgfscope}%
\pgfsetbuttcap%
\pgfsetroundjoin%
\definecolor{currentfill}{rgb}{0.000000,0.000000,0.000000}%
\pgfsetfillcolor{currentfill}%
\pgfsetlinewidth{0.803000pt}%
\definecolor{currentstroke}{rgb}{0.000000,0.000000,0.000000}%
\pgfsetstrokecolor{currentstroke}%
\pgfsetdash{}{0pt}%
\pgfsys@defobject{currentmarker}{\pgfqpoint{-0.048611in}{0.000000in}}{\pgfqpoint{0.000000in}{0.000000in}}{%
\pgfpathmoveto{\pgfqpoint{0.000000in}{0.000000in}}%
\pgfpathlineto{\pgfqpoint{-0.048611in}{0.000000in}}%
\pgfusepath{stroke,fill}%
}%
\begin{pgfscope}%
\pgfsys@transformshift{0.526080in}{2.757923in}%
\pgfsys@useobject{currentmarker}{}%
\end{pgfscope}%
\end{pgfscope}%
\begin{pgfscope}%
\pgftext[x=0.289968in,y=2.705161in,left,base]{\rmfamily\fontsize{10.000000}{12.000000}\selectfont \(\displaystyle 10\)}%
\end{pgfscope}%
\begin{pgfscope}%
\pgfsetbuttcap%
\pgfsetroundjoin%
\definecolor{currentfill}{rgb}{0.000000,0.000000,0.000000}%
\pgfsetfillcolor{currentfill}%
\pgfsetlinewidth{0.803000pt}%
\definecolor{currentstroke}{rgb}{0.000000,0.000000,0.000000}%
\pgfsetstrokecolor{currentstroke}%
\pgfsetdash{}{0pt}%
\pgfsys@defobject{currentmarker}{\pgfqpoint{-0.048611in}{0.000000in}}{\pgfqpoint{0.000000in}{0.000000in}}{%
\pgfpathmoveto{\pgfqpoint{0.000000in}{0.000000in}}%
\pgfpathlineto{\pgfqpoint{-0.048611in}{0.000000in}}%
\pgfusepath{stroke,fill}%
}%
\begin{pgfscope}%
\pgfsys@transformshift{0.526080in}{3.184675in}%
\pgfsys@useobject{currentmarker}{}%
\end{pgfscope}%
\end{pgfscope}%
\begin{pgfscope}%
\pgftext[x=0.289968in,y=3.131914in,left,base]{\rmfamily\fontsize{10.000000}{12.000000}\selectfont \(\displaystyle 12\)}%
\end{pgfscope}%
\begin{pgfscope}%
\pgftext[x=0.234413in,y=2.031603in,,bottom,rotate=90.000000]{\rmfamily\fontsize{10.000000}{12.000000}\selectfont \(\displaystyle \bar{y}\)}%
\end{pgfscope}%
\begin{pgfscope}%
\pgfpathrectangle{\pgfqpoint{0.526080in}{0.521603in}}{\pgfqpoint{3.720000in}{3.020000in}} %
\pgfusepath{clip}%
\pgfsetrectcap%
\pgfsetroundjoin%
\pgfsetlinewidth{1.505625pt}%
\definecolor{currentstroke}{rgb}{1.000000,0.682749,0.366979}%
\pgfsetstrokecolor{currentstroke}%
\pgfsetdash{}{0pt}%
\pgfpathmoveto{\pgfqpoint{0.706228in}{0.663592in}}%
\pgfpathlineto{\pgfqpoint{0.714209in}{0.667809in}}%
\pgfpathlineto{\pgfqpoint{0.722191in}{0.674171in}}%
\pgfpathlineto{\pgfqpoint{0.730173in}{0.686079in}}%
\pgfpathlineto{\pgfqpoint{0.738154in}{0.692849in}}%
\pgfpathlineto{\pgfqpoint{0.746136in}{0.695567in}}%
\pgfpathlineto{\pgfqpoint{0.762099in}{0.714094in}}%
\pgfpathlineto{\pgfqpoint{0.786044in}{0.731841in}}%
\pgfpathlineto{\pgfqpoint{0.794025in}{0.739472in}}%
\pgfpathlineto{\pgfqpoint{0.817970in}{0.756162in}}%
\pgfpathlineto{\pgfqpoint{0.825952in}{0.763304in}}%
\pgfpathlineto{\pgfqpoint{0.849896in}{0.779623in}}%
\pgfpathlineto{\pgfqpoint{0.865860in}{0.790296in}}%
\pgfpathlineto{\pgfqpoint{0.873841in}{0.794770in}}%
\pgfpathlineto{\pgfqpoint{0.889804in}{0.806381in}}%
\pgfpathlineto{\pgfqpoint{0.905768in}{0.815071in}}%
\pgfpathlineto{\pgfqpoint{0.921731in}{0.825749in}}%
\pgfpathlineto{\pgfqpoint{0.937694in}{0.833826in}}%
\pgfpathlineto{\pgfqpoint{0.953657in}{0.843844in}}%
\pgfpathlineto{\pgfqpoint{0.969620in}{0.851630in}}%
\pgfpathlineto{\pgfqpoint{0.985583in}{0.860649in}}%
\pgfpathlineto{\pgfqpoint{1.001547in}{0.868000in}}%
\pgfpathlineto{\pgfqpoint{1.009528in}{0.872760in}}%
\pgfpathlineto{\pgfqpoint{1.065399in}{0.897941in}}%
\pgfpathlineto{\pgfqpoint{1.081362in}{0.904522in}}%
\pgfpathlineto{\pgfqpoint{1.161178in}{0.935486in}}%
\pgfpathlineto{\pgfqpoint{1.225031in}{0.956108in}}%
\pgfpathlineto{\pgfqpoint{1.336773in}{0.984407in}}%
\pgfpathlineto{\pgfqpoint{1.360718in}{0.989358in}}%
\pgfpathlineto{\pgfqpoint{1.408608in}{0.998560in}}%
\pgfpathlineto{\pgfqpoint{1.504387in}{1.011695in}}%
\pgfpathlineto{\pgfqpoint{1.608147in}{1.019363in}}%
\pgfpathlineto{\pgfqpoint{1.640074in}{1.020569in}}%
\pgfpathlineto{\pgfqpoint{1.687963in}{1.021125in}}%
\pgfpathlineto{\pgfqpoint{1.767779in}{1.018826in}}%
\pgfpathlineto{\pgfqpoint{1.815669in}{1.015558in}}%
\pgfpathlineto{\pgfqpoint{1.879521in}{1.008911in}}%
\pgfpathlineto{\pgfqpoint{1.935392in}{1.001017in}}%
\pgfpathlineto{\pgfqpoint{1.999245in}{0.989079in}}%
\pgfpathlineto{\pgfqpoint{2.055116in}{0.976184in}}%
\pgfpathlineto{\pgfqpoint{2.118969in}{0.958124in}}%
\pgfpathlineto{\pgfqpoint{2.174840in}{0.939380in}}%
\pgfpathlineto{\pgfqpoint{2.222730in}{0.921008in}}%
\pgfpathlineto{\pgfqpoint{2.270619in}{0.900458in}}%
\pgfpathlineto{\pgfqpoint{2.318509in}{0.877668in}}%
\pgfpathlineto{\pgfqpoint{2.366398in}{0.852369in}}%
\pgfpathlineto{\pgfqpoint{2.414288in}{0.824504in}}%
\pgfpathlineto{\pgfqpoint{2.462177in}{0.793768in}}%
\pgfpathlineto{\pgfqpoint{2.502085in}{0.765700in}}%
\pgfpathlineto{\pgfqpoint{2.502085in}{0.765700in}}%
\pgfusepath{stroke}%
\end{pgfscope}%
\begin{pgfscope}%
\pgfpathrectangle{\pgfqpoint{0.526080in}{0.521603in}}{\pgfqpoint{3.720000in}{3.020000in}} %
\pgfusepath{clip}%
\pgfsetrectcap%
\pgfsetroundjoin%
\pgfsetlinewidth{1.505625pt}%
\definecolor{currentstroke}{rgb}{1.000000,0.000000,0.000000}%
\pgfsetstrokecolor{currentstroke}%
\pgfsetdash{}{0pt}%
\pgfpathmoveto{\pgfqpoint{0.695171in}{0.658876in}}%
\pgfpathlineto{\pgfqpoint{0.770593in}{0.726383in}}%
\pgfpathlineto{\pgfqpoint{0.793220in}{0.745428in}}%
\pgfpathlineto{\pgfqpoint{0.868642in}{0.802821in}}%
\pgfpathlineto{\pgfqpoint{0.981776in}{0.876306in}}%
\pgfpathlineto{\pgfqpoint{1.072282in}{0.926083in}}%
\pgfpathlineto{\pgfqpoint{1.162789in}{0.969019in}}%
\pgfpathlineto{\pgfqpoint{1.230669in}{0.997081in}}%
\pgfpathlineto{\pgfqpoint{1.275923in}{1.014143in}}%
\pgfpathlineto{\pgfqpoint{1.351345in}{1.039585in}}%
\pgfpathlineto{\pgfqpoint{1.434309in}{1.063616in}}%
\pgfpathlineto{\pgfqpoint{1.487105in}{1.077118in}}%
\pgfpathlineto{\pgfqpoint{1.562527in}{1.093884in}}%
\pgfpathlineto{\pgfqpoint{1.622865in}{1.105308in}}%
\pgfpathlineto{\pgfqpoint{1.683203in}{1.115109in}}%
\pgfpathlineto{\pgfqpoint{1.758625in}{1.125078in}}%
\pgfpathlineto{\pgfqpoint{1.841590in}{1.133269in}}%
\pgfpathlineto{\pgfqpoint{1.917012in}{1.138211in}}%
\pgfpathlineto{\pgfqpoint{2.022604in}{1.140972in}}%
\pgfpathlineto{\pgfqpoint{2.098026in}{1.140013in}}%
\pgfpathlineto{\pgfqpoint{2.158364in}{1.137537in}}%
\pgfpathlineto{\pgfqpoint{2.241328in}{1.131323in}}%
\pgfpathlineto{\pgfqpoint{2.309208in}{1.123881in}}%
\pgfpathlineto{\pgfqpoint{2.369546in}{1.115189in}}%
\pgfpathlineto{\pgfqpoint{2.407257in}{1.108724in}}%
\pgfpathlineto{\pgfqpoint{2.407257in}{1.108724in}}%
\pgfusepath{stroke}%
\end{pgfscope}%
\begin{pgfscope}%
\pgfpathrectangle{\pgfqpoint{0.526080in}{0.521603in}}{\pgfqpoint{3.720000in}{3.020000in}} %
\pgfusepath{clip}%
\pgfsetrectcap%
\pgfsetroundjoin%
\pgfsetlinewidth{1.505625pt}%
\definecolor{currentstroke}{rgb}{0.605882,0.986201,0.645928}%
\pgfsetstrokecolor{currentstroke}%
\pgfsetdash{}{0pt}%
\pgfpathmoveto{\pgfqpoint{0.747633in}{0.697697in}}%
\pgfpathlineto{\pgfqpoint{0.757202in}{0.700571in}}%
\pgfpathlineto{\pgfqpoint{0.766770in}{0.713762in}}%
\pgfpathlineto{\pgfqpoint{0.776339in}{0.724791in}}%
\pgfpathlineto{\pgfqpoint{0.785908in}{0.731170in}}%
\pgfpathlineto{\pgfqpoint{0.795477in}{0.739259in}}%
\pgfpathlineto{\pgfqpoint{0.824184in}{0.768428in}}%
\pgfpathlineto{\pgfqpoint{0.833753in}{0.775034in}}%
\pgfpathlineto{\pgfqpoint{0.843321in}{0.783949in}}%
\pgfpathlineto{\pgfqpoint{0.852890in}{0.794440in}}%
\pgfpathlineto{\pgfqpoint{0.872028in}{0.809080in}}%
\pgfpathlineto{\pgfqpoint{0.881597in}{0.816029in}}%
\pgfpathlineto{\pgfqpoint{0.910303in}{0.841804in}}%
\pgfpathlineto{\pgfqpoint{0.919872in}{0.847079in}}%
\pgfpathlineto{\pgfqpoint{0.948579in}{0.871614in}}%
\pgfpathlineto{\pgfqpoint{0.967717in}{0.883426in}}%
\pgfpathlineto{\pgfqpoint{0.996423in}{0.906291in}}%
\pgfpathlineto{\pgfqpoint{1.005992in}{0.911030in}}%
\pgfpathlineto{\pgfqpoint{1.034699in}{0.933130in}}%
\pgfpathlineto{\pgfqpoint{1.053836in}{0.943650in}}%
\pgfpathlineto{\pgfqpoint{1.082543in}{0.963712in}}%
\pgfpathlineto{\pgfqpoint{1.092112in}{0.968323in}}%
\pgfpathlineto{\pgfqpoint{1.120819in}{0.988103in}}%
\pgfpathlineto{\pgfqpoint{1.139956in}{0.997540in}}%
\pgfpathlineto{\pgfqpoint{1.168663in}{1.015805in}}%
\pgfpathlineto{\pgfqpoint{1.178232in}{1.019762in}}%
\pgfpathlineto{\pgfqpoint{1.206938in}{1.037936in}}%
\pgfpathlineto{\pgfqpoint{1.226076in}{1.046499in}}%
\pgfpathlineto{\pgfqpoint{1.254783in}{1.062878in}}%
\pgfpathlineto{\pgfqpoint{1.264352in}{1.066857in}}%
\pgfpathlineto{\pgfqpoint{1.293058in}{1.083342in}}%
\pgfpathlineto{\pgfqpoint{1.312196in}{1.091194in}}%
\pgfpathlineto{\pgfqpoint{1.340903in}{1.106045in}}%
\pgfpathlineto{\pgfqpoint{1.360040in}{1.114829in}}%
\pgfpathlineto{\pgfqpoint{1.379178in}{1.124997in}}%
\pgfpathlineto{\pgfqpoint{1.398316in}{1.132263in}}%
\pgfpathlineto{\pgfqpoint{1.417453in}{1.142654in}}%
\pgfpathlineto{\pgfqpoint{1.436591in}{1.149577in}}%
\pgfpathlineto{\pgfqpoint{1.465298in}{1.163527in}}%
\pgfpathlineto{\pgfqpoint{1.484436in}{1.170224in}}%
\pgfpathlineto{\pgfqpoint{1.503573in}{1.179814in}}%
\pgfpathlineto{\pgfqpoint{1.522711in}{1.186228in}}%
\pgfpathlineto{\pgfqpoint{1.551418in}{1.199089in}}%
\pgfpathlineto{\pgfqpoint{1.570555in}{1.205282in}}%
\pgfpathlineto{\pgfqpoint{1.589693in}{1.214081in}}%
\pgfpathlineto{\pgfqpoint{1.608831in}{1.220120in}}%
\pgfpathlineto{\pgfqpoint{1.637537in}{1.232013in}}%
\pgfpathlineto{\pgfqpoint{1.656675in}{1.237882in}}%
\pgfpathlineto{\pgfqpoint{1.675813in}{1.245835in}}%
\pgfpathlineto{\pgfqpoint{1.694951in}{1.251601in}}%
\pgfpathlineto{\pgfqpoint{1.723657in}{1.262487in}}%
\pgfpathlineto{\pgfqpoint{1.742795in}{1.268164in}}%
\pgfpathlineto{\pgfqpoint{1.761933in}{1.275367in}}%
\pgfpathlineto{\pgfqpoint{1.790639in}{1.284273in}}%
\pgfpathlineto{\pgfqpoint{1.809777in}{1.290793in}}%
\pgfpathlineto{\pgfqpoint{1.828915in}{1.296117in}}%
\pgfpathlineto{\pgfqpoint{1.848053in}{1.302888in}}%
\pgfpathlineto{\pgfqpoint{1.876759in}{1.311158in}}%
\pgfpathlineto{\pgfqpoint{1.895897in}{1.317257in}}%
\pgfpathlineto{\pgfqpoint{1.915035in}{1.322242in}}%
\pgfpathlineto{\pgfqpoint{1.934172in}{1.328602in}}%
\pgfpathlineto{\pgfqpoint{1.962879in}{1.336425in}}%
\pgfpathlineto{\pgfqpoint{1.982017in}{1.341991in}}%
\pgfpathlineto{\pgfqpoint{2.001154in}{1.346788in}}%
\pgfpathlineto{\pgfqpoint{2.020292in}{1.352746in}}%
\pgfpathlineto{\pgfqpoint{2.048999in}{1.360126in}}%
\pgfpathlineto{\pgfqpoint{2.068137in}{1.365082in}}%
\pgfpathlineto{\pgfqpoint{2.087274in}{1.369761in}}%
\pgfpathlineto{\pgfqpoint{2.106412in}{1.375228in}}%
\pgfpathlineto{\pgfqpoint{2.135119in}{1.382148in}}%
\pgfpathlineto{\pgfqpoint{2.154256in}{1.386705in}}%
\pgfpathlineto{\pgfqpoint{2.173394in}{1.391207in}}%
\pgfpathlineto{\pgfqpoint{2.192532in}{1.396172in}}%
\pgfpathlineto{\pgfqpoint{2.211670in}{1.400078in}}%
\pgfpathlineto{\pgfqpoint{2.240376in}{1.406880in}}%
\pgfpathlineto{\pgfqpoint{2.259514in}{1.411237in}}%
\pgfpathlineto{\pgfqpoint{2.278652in}{1.415806in}}%
\pgfpathlineto{\pgfqpoint{2.307358in}{1.421830in}}%
\pgfpathlineto{\pgfqpoint{2.355203in}{1.432315in}}%
\pgfpathlineto{\pgfqpoint{2.623131in}{1.481874in}}%
\pgfpathlineto{\pgfqpoint{2.670975in}{1.489390in}}%
\pgfpathlineto{\pgfqpoint{2.737957in}{1.500038in}}%
\pgfpathlineto{\pgfqpoint{2.881490in}{1.519969in}}%
\pgfpathlineto{\pgfqpoint{2.881490in}{1.519969in}}%
\pgfusepath{stroke}%
\end{pgfscope}%
\begin{pgfscope}%
\pgfpathrectangle{\pgfqpoint{0.526080in}{0.521603in}}{\pgfqpoint{3.720000in}{3.020000in}} %
\pgfusepath{clip}%
\pgfsetrectcap%
\pgfsetroundjoin%
\pgfsetlinewidth{1.505625pt}%
\definecolor{currentstroke}{rgb}{0.131373,0.547220,0.958381}%
\pgfsetstrokecolor{currentstroke}%
\pgfsetdash{}{0pt}%
\pgfpathmoveto{\pgfqpoint{0.765106in}{0.683992in}}%
\pgfpathlineto{\pgfqpoint{0.806017in}{0.726957in}}%
\pgfpathlineto{\pgfqpoint{0.846927in}{0.760770in}}%
\pgfpathlineto{\pgfqpoint{0.887837in}{0.804022in}}%
\pgfpathlineto{\pgfqpoint{0.949203in}{0.854810in}}%
\pgfpathlineto{\pgfqpoint{0.990113in}{0.896510in}}%
\pgfpathlineto{\pgfqpoint{1.010568in}{0.910665in}}%
\pgfpathlineto{\pgfqpoint{1.031024in}{0.927219in}}%
\pgfpathlineto{\pgfqpoint{1.071934in}{0.967690in}}%
\pgfpathlineto{\pgfqpoint{1.112844in}{0.997591in}}%
\pgfpathlineto{\pgfqpoint{1.133300in}{1.014972in}}%
\pgfpathlineto{\pgfqpoint{1.153755in}{1.035542in}}%
\pgfpathlineto{\pgfqpoint{1.174210in}{1.052607in}}%
\pgfpathlineto{\pgfqpoint{1.194665in}{1.066297in}}%
\pgfpathlineto{\pgfqpoint{1.215120in}{1.082500in}}%
\pgfpathlineto{\pgfqpoint{1.235576in}{1.102136in}}%
\pgfpathlineto{\pgfqpoint{1.256031in}{1.119579in}}%
\pgfpathlineto{\pgfqpoint{1.296941in}{1.148045in}}%
\pgfpathlineto{\pgfqpoint{1.358307in}{1.199268in}}%
\pgfpathlineto{\pgfqpoint{1.378762in}{1.212600in}}%
\pgfpathlineto{\pgfqpoint{1.399217in}{1.228878in}}%
\pgfpathlineto{\pgfqpoint{1.419672in}{1.246810in}}%
\pgfpathlineto{\pgfqpoint{1.440127in}{1.262625in}}%
\pgfpathlineto{\pgfqpoint{1.481038in}{1.290327in}}%
\pgfpathlineto{\pgfqpoint{1.521948in}{1.324108in}}%
\pgfpathlineto{\pgfqpoint{1.562858in}{1.351149in}}%
\pgfpathlineto{\pgfqpoint{1.624224in}{1.398735in}}%
\pgfpathlineto{\pgfqpoint{1.665134in}{1.426044in}}%
\pgfpathlineto{\pgfqpoint{1.706045in}{1.457655in}}%
\pgfpathlineto{\pgfqpoint{1.746955in}{1.484071in}}%
\pgfpathlineto{\pgfqpoint{1.787866in}{1.515653in}}%
\pgfpathlineto{\pgfqpoint{1.849231in}{1.556147in}}%
\pgfpathlineto{\pgfqpoint{1.869686in}{1.571973in}}%
\pgfpathlineto{\pgfqpoint{1.992417in}{1.654784in}}%
\pgfpathlineto{\pgfqpoint{2.012873in}{1.667715in}}%
\pgfpathlineto{\pgfqpoint{2.074238in}{1.709613in}}%
\pgfpathlineto{\pgfqpoint{2.094693in}{1.722158in}}%
\pgfpathlineto{\pgfqpoint{2.176514in}{1.776832in}}%
\pgfpathlineto{\pgfqpoint{2.217424in}{1.803976in}}%
\pgfpathlineto{\pgfqpoint{2.237880in}{1.817858in}}%
\pgfpathlineto{\pgfqpoint{2.278790in}{1.842834in}}%
\pgfpathlineto{\pgfqpoint{2.340156in}{1.883672in}}%
\pgfpathlineto{\pgfqpoint{2.381066in}{1.909010in}}%
\pgfpathlineto{\pgfqpoint{2.421976in}{1.935929in}}%
\pgfpathlineto{\pgfqpoint{2.462887in}{1.960817in}}%
\pgfpathlineto{\pgfqpoint{2.503797in}{1.988032in}}%
\pgfpathlineto{\pgfqpoint{2.565163in}{2.025834in}}%
\pgfpathlineto{\pgfqpoint{2.606073in}{2.051615in}}%
\pgfpathlineto{\pgfqpoint{2.646983in}{2.076576in}}%
\pgfpathlineto{\pgfqpoint{2.687894in}{2.102878in}}%
\pgfpathlineto{\pgfqpoint{2.728804in}{2.127389in}}%
\pgfpathlineto{\pgfqpoint{2.790170in}{2.165352in}}%
\pgfpathlineto{\pgfqpoint{2.831080in}{2.190259in}}%
\pgfpathlineto{\pgfqpoint{2.871990in}{2.215649in}}%
\pgfpathlineto{\pgfqpoint{2.994721in}{2.289342in}}%
\pgfpathlineto{\pgfqpoint{3.056087in}{2.326774in}}%
\pgfpathlineto{\pgfqpoint{3.587922in}{2.640101in}}%
\pgfpathlineto{\pgfqpoint{3.608377in}{2.651714in}}%
\pgfpathlineto{\pgfqpoint{3.608377in}{2.651714in}}%
\pgfusepath{stroke}%
\end{pgfscope}%
\begin{pgfscope}%
\pgfpathrectangle{\pgfqpoint{0.526080in}{0.521603in}}{\pgfqpoint{3.720000in}{3.020000in}} %
\pgfusepath{clip}%
\pgfsetrectcap%
\pgfsetroundjoin%
\pgfsetlinewidth{1.505625pt}%
\definecolor{currentstroke}{rgb}{0.500000,0.000000,1.000000}%
\pgfsetstrokecolor{currentstroke}%
\pgfsetdash{}{0pt}%
\pgfpathmoveto{\pgfqpoint{0.897842in}{0.851531in}}%
\pgfpathlineto{\pgfqpoint{0.926228in}{0.874721in}}%
\pgfpathlineto{\pgfqpoint{0.954613in}{0.897808in}}%
\pgfpathlineto{\pgfqpoint{0.982998in}{0.930043in}}%
\pgfpathlineto{\pgfqpoint{1.011383in}{0.959535in}}%
\pgfpathlineto{\pgfqpoint{1.039768in}{0.984673in}}%
\pgfpathlineto{\pgfqpoint{1.068154in}{1.010878in}}%
\pgfpathlineto{\pgfqpoint{1.096539in}{1.034443in}}%
\pgfpathlineto{\pgfqpoint{1.124924in}{1.062436in}}%
\pgfpathlineto{\pgfqpoint{1.153309in}{1.092331in}}%
\pgfpathlineto{\pgfqpoint{1.181695in}{1.115067in}}%
\pgfpathlineto{\pgfqpoint{1.210080in}{1.140018in}}%
\pgfpathlineto{\pgfqpoint{1.238465in}{1.170494in}}%
\pgfpathlineto{\pgfqpoint{1.266850in}{1.197819in}}%
\pgfpathlineto{\pgfqpoint{1.295236in}{1.221825in}}%
\pgfpathlineto{\pgfqpoint{1.323621in}{1.242591in}}%
\pgfpathlineto{\pgfqpoint{1.352006in}{1.269830in}}%
\pgfpathlineto{\pgfqpoint{1.380391in}{1.298104in}}%
\pgfpathlineto{\pgfqpoint{1.408777in}{1.324582in}}%
\pgfpathlineto{\pgfqpoint{1.437162in}{1.347233in}}%
\pgfpathlineto{\pgfqpoint{1.465547in}{1.372548in}}%
\pgfpathlineto{\pgfqpoint{1.493932in}{1.399924in}}%
\pgfpathlineto{\pgfqpoint{1.522318in}{1.424375in}}%
\pgfpathlineto{\pgfqpoint{1.550703in}{1.446256in}}%
\pgfpathlineto{\pgfqpoint{1.579088in}{1.470940in}}%
\pgfpathlineto{\pgfqpoint{1.607473in}{1.496776in}}%
\pgfpathlineto{\pgfqpoint{1.635858in}{1.522881in}}%
\pgfpathlineto{\pgfqpoint{1.664244in}{1.543645in}}%
\pgfpathlineto{\pgfqpoint{1.692629in}{1.568456in}}%
\pgfpathlineto{\pgfqpoint{1.721014in}{1.594362in}}%
\pgfpathlineto{\pgfqpoint{1.749399in}{1.618943in}}%
\pgfpathlineto{\pgfqpoint{1.777785in}{1.641122in}}%
\pgfpathlineto{\pgfqpoint{1.806170in}{1.663719in}}%
\pgfpathlineto{\pgfqpoint{1.834555in}{1.688229in}}%
\pgfpathlineto{\pgfqpoint{1.862940in}{1.713460in}}%
\pgfpathlineto{\pgfqpoint{1.891326in}{1.733690in}}%
\pgfpathlineto{\pgfqpoint{1.919711in}{1.758754in}}%
\pgfpathlineto{\pgfqpoint{1.948096in}{1.782359in}}%
\pgfpathlineto{\pgfqpoint{1.976481in}{1.806856in}}%
\pgfpathlineto{\pgfqpoint{2.004867in}{1.829283in}}%
\pgfpathlineto{\pgfqpoint{2.033252in}{1.851166in}}%
\pgfpathlineto{\pgfqpoint{2.061637in}{1.872011in}}%
\pgfpathlineto{\pgfqpoint{2.090022in}{1.897080in}}%
\pgfpathlineto{\pgfqpoint{2.118408in}{1.918923in}}%
\pgfpathlineto{\pgfqpoint{2.146793in}{1.943433in}}%
\pgfpathlineto{\pgfqpoint{2.175178in}{1.966741in}}%
\pgfpathlineto{\pgfqpoint{2.203563in}{1.990779in}}%
\pgfpathlineto{\pgfqpoint{2.231948in}{2.013570in}}%
\pgfpathlineto{\pgfqpoint{2.260334in}{2.033832in}}%
\pgfpathlineto{\pgfqpoint{2.288719in}{2.053813in}}%
\pgfpathlineto{\pgfqpoint{2.317104in}{2.077674in}}%
\pgfpathlineto{\pgfqpoint{2.345489in}{2.100351in}}%
\pgfpathlineto{\pgfqpoint{2.373875in}{2.123695in}}%
\pgfpathlineto{\pgfqpoint{2.402260in}{2.147246in}}%
\pgfpathlineto{\pgfqpoint{2.430645in}{2.170564in}}%
\pgfpathlineto{\pgfqpoint{2.459030in}{2.193500in}}%
\pgfpathlineto{\pgfqpoint{2.487416in}{2.213515in}}%
\pgfpathlineto{\pgfqpoint{2.515801in}{2.232974in}}%
\pgfpathlineto{\pgfqpoint{2.544186in}{2.254460in}}%
\pgfpathlineto{\pgfqpoint{2.572571in}{2.278011in}}%
\pgfpathlineto{\pgfqpoint{2.600957in}{2.301097in}}%
\pgfpathlineto{\pgfqpoint{2.629342in}{2.325568in}}%
\pgfpathlineto{\pgfqpoint{2.657727in}{2.348175in}}%
\pgfpathlineto{\pgfqpoint{2.686112in}{2.370803in}}%
\pgfpathlineto{\pgfqpoint{2.714498in}{2.390295in}}%
\pgfpathlineto{\pgfqpoint{2.742883in}{2.409761in}}%
\pgfpathlineto{\pgfqpoint{2.771268in}{2.430227in}}%
\pgfpathlineto{\pgfqpoint{2.799653in}{2.452762in}}%
\pgfpathlineto{\pgfqpoint{2.828038in}{2.476669in}}%
\pgfpathlineto{\pgfqpoint{2.856424in}{2.501393in}}%
\pgfpathlineto{\pgfqpoint{2.884809in}{2.523432in}}%
\pgfpathlineto{\pgfqpoint{2.913194in}{2.545381in}}%
\pgfpathlineto{\pgfqpoint{2.941579in}{2.564949in}}%
\pgfpathlineto{\pgfqpoint{2.969965in}{2.583321in}}%
\pgfpathlineto{\pgfqpoint{2.998350in}{2.603499in}}%
\pgfpathlineto{\pgfqpoint{3.026735in}{2.624811in}}%
\pgfpathlineto{\pgfqpoint{3.055120in}{2.649022in}}%
\pgfpathlineto{\pgfqpoint{3.083506in}{2.673144in}}%
\pgfpathlineto{\pgfqpoint{3.111891in}{2.695830in}}%
\pgfpathlineto{\pgfqpoint{3.140276in}{2.717813in}}%
\pgfpathlineto{\pgfqpoint{3.168661in}{2.736500in}}%
\pgfpathlineto{\pgfqpoint{3.197047in}{2.754509in}}%
\pgfpathlineto{\pgfqpoint{3.225432in}{2.774623in}}%
\pgfpathlineto{\pgfqpoint{3.253817in}{2.796039in}}%
\pgfpathlineto{\pgfqpoint{3.282202in}{2.819343in}}%
\pgfpathlineto{\pgfqpoint{3.310588in}{2.843921in}}%
\pgfpathlineto{\pgfqpoint{3.338973in}{2.865888in}}%
\pgfpathlineto{\pgfqpoint{3.367358in}{2.887151in}}%
\pgfpathlineto{\pgfqpoint{3.395743in}{2.906588in}}%
\pgfpathlineto{\pgfqpoint{3.424128in}{2.923607in}}%
\pgfpathlineto{\pgfqpoint{3.452514in}{2.943185in}}%
\pgfpathlineto{\pgfqpoint{3.480899in}{2.964021in}}%
\pgfpathlineto{\pgfqpoint{3.509284in}{2.987010in}}%
\pgfpathlineto{\pgfqpoint{3.537669in}{3.011940in}}%
\pgfpathlineto{\pgfqpoint{3.566055in}{3.033738in}}%
\pgfpathlineto{\pgfqpoint{3.594440in}{3.054424in}}%
\pgfpathlineto{\pgfqpoint{3.622825in}{3.074806in}}%
\pgfpathlineto{\pgfqpoint{3.651210in}{3.091367in}}%
\pgfpathlineto{\pgfqpoint{3.679596in}{3.110316in}}%
\pgfpathlineto{\pgfqpoint{3.707981in}{3.131102in}}%
\pgfpathlineto{\pgfqpoint{3.736366in}{3.154478in}}%
\pgfpathlineto{\pgfqpoint{3.764751in}{3.178561in}}%
\pgfpathlineto{\pgfqpoint{3.793137in}{3.199724in}}%
\pgfpathlineto{\pgfqpoint{3.821522in}{3.219976in}}%
\pgfpathlineto{\pgfqpoint{3.849907in}{3.240613in}}%
\pgfpathlineto{\pgfqpoint{3.878292in}{3.257596in}}%
\pgfpathlineto{\pgfqpoint{3.906678in}{3.275667in}}%
\pgfpathlineto{\pgfqpoint{3.935063in}{3.296679in}}%
\pgfpathlineto{\pgfqpoint{3.963448in}{3.320218in}}%
\pgfpathlineto{\pgfqpoint{3.991833in}{3.344027in}}%
\pgfpathlineto{\pgfqpoint{4.020218in}{3.363704in}}%
\pgfpathlineto{\pgfqpoint{4.048604in}{3.385783in}}%
\pgfpathlineto{\pgfqpoint{4.076989in}{3.404331in}}%
\pgfusepath{stroke}%
\end{pgfscope}%
\begin{pgfscope}%
\pgfsetrectcap%
\pgfsetmiterjoin%
\pgfsetlinewidth{0.803000pt}%
\definecolor{currentstroke}{rgb}{0.000000,0.000000,0.000000}%
\pgfsetstrokecolor{currentstroke}%
\pgfsetdash{}{0pt}%
\pgfpathmoveto{\pgfqpoint{0.526080in}{0.521603in}}%
\pgfpathlineto{\pgfqpoint{0.526080in}{3.541603in}}%
\pgfusepath{stroke}%
\end{pgfscope}%
\begin{pgfscope}%
\pgfsetrectcap%
\pgfsetmiterjoin%
\pgfsetlinewidth{0.803000pt}%
\definecolor{currentstroke}{rgb}{0.000000,0.000000,0.000000}%
\pgfsetstrokecolor{currentstroke}%
\pgfsetdash{}{0pt}%
\pgfpathmoveto{\pgfqpoint{4.246080in}{0.521603in}}%
\pgfpathlineto{\pgfqpoint{4.246080in}{3.541603in}}%
\pgfusepath{stroke}%
\end{pgfscope}%
\begin{pgfscope}%
\pgfsetrectcap%
\pgfsetmiterjoin%
\pgfsetlinewidth{0.803000pt}%
\definecolor{currentstroke}{rgb}{0.000000,0.000000,0.000000}%
\pgfsetstrokecolor{currentstroke}%
\pgfsetdash{}{0pt}%
\pgfpathmoveto{\pgfqpoint{0.526080in}{0.521603in}}%
\pgfpathlineto{\pgfqpoint{4.246080in}{0.521603in}}%
\pgfusepath{stroke}%
\end{pgfscope}%
\begin{pgfscope}%
\pgfsetrectcap%
\pgfsetmiterjoin%
\pgfsetlinewidth{0.803000pt}%
\definecolor{currentstroke}{rgb}{0.000000,0.000000,0.000000}%
\pgfsetstrokecolor{currentstroke}%
\pgfsetdash{}{0pt}%
\pgfpathmoveto{\pgfqpoint{0.526080in}{3.541603in}}%
\pgfpathlineto{\pgfqpoint{4.246080in}{3.541603in}}%
\pgfusepath{stroke}%
\end{pgfscope}%
\begin{pgfscope}%
\pgfpathrectangle{\pgfqpoint{4.478580in}{0.521603in}}{\pgfqpoint{0.151000in}{3.020000in}} %
\pgfusepath{clip}%
\pgfsetbuttcap%
\pgfsetmiterjoin%
\definecolor{currentfill}{rgb}{1.000000,1.000000,1.000000}%
\pgfsetfillcolor{currentfill}%
\pgfsetlinewidth{0.010037pt}%
\definecolor{currentstroke}{rgb}{1.000000,1.000000,1.000000}%
\pgfsetstrokecolor{currentstroke}%
\pgfsetdash{}{0pt}%
\pgfpathmoveto{\pgfqpoint{4.478580in}{0.521603in}}%
\pgfpathlineto{\pgfqpoint{4.478580in}{0.533400in}}%
\pgfpathlineto{\pgfqpoint{4.478580in}{3.529806in}}%
\pgfpathlineto{\pgfqpoint{4.478580in}{3.541603in}}%
\pgfpathlineto{\pgfqpoint{4.629580in}{3.541603in}}%
\pgfpathlineto{\pgfqpoint{4.629580in}{3.529806in}}%
\pgfpathlineto{\pgfqpoint{4.629580in}{0.533400in}}%
\pgfpathlineto{\pgfqpoint{4.629580in}{0.521603in}}%
\pgfpathclose%
\pgfusepath{stroke,fill}%
\end{pgfscope}%
\begin{pgfscope}%
\pgfsys@transformshift{4.480000in}{0.526603in}%
\pgftext[left,bottom]{\pgfimage[interpolate=true,width=0.150000in,height=3.020000in]{series_l_ds-img0.png}}%
\end{pgfscope}%
\begin{pgfscope}%
\pgfsetbuttcap%
\pgfsetroundjoin%
\definecolor{currentfill}{rgb}{0.000000,0.000000,0.000000}%
\pgfsetfillcolor{currentfill}%
\pgfsetlinewidth{0.803000pt}%
\definecolor{currentstroke}{rgb}{0.000000,0.000000,0.000000}%
\pgfsetstrokecolor{currentstroke}%
\pgfsetdash{}{0pt}%
\pgfsys@defobject{currentmarker}{\pgfqpoint{0.000000in}{0.000000in}}{\pgfqpoint{0.048611in}{0.000000in}}{%
\pgfpathmoveto{\pgfqpoint{0.000000in}{0.000000in}}%
\pgfpathlineto{\pgfqpoint{0.048611in}{0.000000in}}%
\pgfusepath{stroke,fill}%
}%
\begin{pgfscope}%
\pgfsys@transformshift{4.629580in}{1.060049in}%
\pgfsys@useobject{currentmarker}{}%
\end{pgfscope}%
\end{pgfscope}%
\begin{pgfscope}%
\pgftext[x=4.726802in,y=1.007287in,left,base]{\rmfamily\fontsize{10.000000}{12.000000}\selectfont \(\displaystyle 2\times10^{-1}\)}%
\end{pgfscope}%
\begin{pgfscope}%
\pgfsetbuttcap%
\pgfsetroundjoin%
\definecolor{currentfill}{rgb}{0.000000,0.000000,0.000000}%
\pgfsetfillcolor{currentfill}%
\pgfsetlinewidth{0.803000pt}%
\definecolor{currentstroke}{rgb}{0.000000,0.000000,0.000000}%
\pgfsetstrokecolor{currentstroke}%
\pgfsetdash{}{0pt}%
\pgfsys@defobject{currentmarker}{\pgfqpoint{0.000000in}{0.000000in}}{\pgfqpoint{0.048611in}{0.000000in}}{%
\pgfpathmoveto{\pgfqpoint{0.000000in}{0.000000in}}%
\pgfpathlineto{\pgfqpoint{0.048611in}{0.000000in}}%
\pgfusepath{stroke,fill}%
}%
\begin{pgfscope}%
\pgfsys@transformshift{4.629580in}{1.889642in}%
\pgfsys@useobject{currentmarker}{}%
\end{pgfscope}%
\end{pgfscope}%
\begin{pgfscope}%
\pgftext[x=4.726802in,y=1.836881in,left,base]{\rmfamily\fontsize{10.000000}{12.000000}\selectfont \(\displaystyle 3\times10^{-1}\)}%
\end{pgfscope}%
\begin{pgfscope}%
\pgfsetbuttcap%
\pgfsetroundjoin%
\definecolor{currentfill}{rgb}{0.000000,0.000000,0.000000}%
\pgfsetfillcolor{currentfill}%
\pgfsetlinewidth{0.803000pt}%
\definecolor{currentstroke}{rgb}{0.000000,0.000000,0.000000}%
\pgfsetstrokecolor{currentstroke}%
\pgfsetdash{}{0pt}%
\pgfsys@defobject{currentmarker}{\pgfqpoint{0.000000in}{0.000000in}}{\pgfqpoint{0.048611in}{0.000000in}}{%
\pgfpathmoveto{\pgfqpoint{0.000000in}{0.000000in}}%
\pgfpathlineto{\pgfqpoint{0.048611in}{0.000000in}}%
\pgfusepath{stroke,fill}%
}%
\begin{pgfscope}%
\pgfsys@transformshift{4.629580in}{2.478248in}%
\pgfsys@useobject{currentmarker}{}%
\end{pgfscope}%
\end{pgfscope}%
\begin{pgfscope}%
\pgftext[x=4.726802in,y=2.425487in,left,base]{\rmfamily\fontsize{10.000000}{12.000000}\selectfont \(\displaystyle 4\times10^{-1}\)}%
\end{pgfscope}%
\begin{pgfscope}%
\pgfsetbuttcap%
\pgfsetroundjoin%
\definecolor{currentfill}{rgb}{0.000000,0.000000,0.000000}%
\pgfsetfillcolor{currentfill}%
\pgfsetlinewidth{0.803000pt}%
\definecolor{currentstroke}{rgb}{0.000000,0.000000,0.000000}%
\pgfsetstrokecolor{currentstroke}%
\pgfsetdash{}{0pt}%
\pgfsys@defobject{currentmarker}{\pgfqpoint{0.000000in}{0.000000in}}{\pgfqpoint{0.048611in}{0.000000in}}{%
\pgfpathmoveto{\pgfqpoint{0.000000in}{0.000000in}}%
\pgfpathlineto{\pgfqpoint{0.048611in}{0.000000in}}%
\pgfusepath{stroke,fill}%
}%
\begin{pgfscope}%
\pgfsys@transformshift{4.629580in}{2.934806in}%
\pgfsys@useobject{currentmarker}{}%
\end{pgfscope}%
\end{pgfscope}%
\begin{pgfscope}%
\pgfsetbuttcap%
\pgfsetroundjoin%
\definecolor{currentfill}{rgb}{0.000000,0.000000,0.000000}%
\pgfsetfillcolor{currentfill}%
\pgfsetlinewidth{0.803000pt}%
\definecolor{currentstroke}{rgb}{0.000000,0.000000,0.000000}%
\pgfsetstrokecolor{currentstroke}%
\pgfsetdash{}{0pt}%
\pgfsys@defobject{currentmarker}{\pgfqpoint{0.000000in}{0.000000in}}{\pgfqpoint{0.048611in}{0.000000in}}{%
\pgfpathmoveto{\pgfqpoint{0.000000in}{0.000000in}}%
\pgfpathlineto{\pgfqpoint{0.048611in}{0.000000in}}%
\pgfusepath{stroke,fill}%
}%
\begin{pgfscope}%
\pgfsys@transformshift{4.629580in}{3.307841in}%
\pgfsys@useobject{currentmarker}{}%
\end{pgfscope}%
\end{pgfscope}%
\begin{pgfscope}%
\pgftext[x=4.726802in,y=3.255080in,left,base]{\rmfamily\fontsize{10.000000}{12.000000}\selectfont \(\displaystyle 6\times10^{-1}\)}%
\end{pgfscope}%
\begin{pgfscope}%
\pgftext[x=5.448446in,y=2.031603in,,top]{\rmfamily\fontsize{12.000000}{14.400000}\selectfont \(\displaystyle {\mathbf{E} \mbox{u}}\)}%
\end{pgfscope}%
\begin{pgfscope}%
\pgfsetbuttcap%
\pgfsetmiterjoin%
\pgfsetlinewidth{0.803000pt}%
\definecolor{currentstroke}{rgb}{0.000000,0.000000,0.000000}%
\pgfsetstrokecolor{currentstroke}%
\pgfsetdash{}{0pt}%
\pgfpathmoveto{\pgfqpoint{4.478580in}{0.521603in}}%
\pgfpathlineto{\pgfqpoint{4.478580in}{0.533400in}}%
\pgfpathlineto{\pgfqpoint{4.478580in}{3.529806in}}%
\pgfpathlineto{\pgfqpoint{4.478580in}{3.541603in}}%
\pgfpathlineto{\pgfqpoint{4.629580in}{3.541603in}}%
\pgfpathlineto{\pgfqpoint{4.629580in}{3.529806in}}%
\pgfpathlineto{\pgfqpoint{4.629580in}{0.533400in}}%
\pgfpathlineto{\pgfqpoint{4.629580in}{0.521603in}}%
\pgfpathclose%
\pgfusepath{stroke}%
\end{pgfscope}%
\end{pgfpicture}%
\makeatother%
\endgroup%

    \caption{Non-dimensional trajectories with the long-time scaling as a function of ${\mathbb{E}\mbox{u}}_+$.\label{fig:series_l_ds}}
\end{figure}

\begin{figure}[htb]
    \centering
    %% Creator: Matplotlib, PGF backend
%%
%% To include the figure in your LaTeX document, write
%%   \input{<filename>.pgf}
%%
%% Make sure the required packages are loaded in your preamble
%%   \usepackage{pgf}
%%
%% Figures using additional raster images can only be included by \input if
%% they are in the same directory as the main LaTeX file. For loading figures
%% from other directories you can use the `import` package
%%   \usepackage{import}
%% and then include the figures with
%%   \import{<path to file>}{<filename>.pgf}
%%
%% Matplotlib used the following preamble
%%   \usepackage{fontspec}
%%   \setmainfont{DejaVu Serif}
%%   \setsansfont{DejaVu Sans}
%%   \setmonofont{DejaVu Sans Mono}
%%
\begingroup%
\makeatletter%
\begin{pgfpicture}%
\pgfpathrectangle{\pgfpointorigin}{\pgfqpoint{5.534462in}{3.694365in}}%
\pgfusepath{use as bounding box, clip}%
\begin{pgfscope}%
\pgfsetbuttcap%
\pgfsetmiterjoin%
\definecolor{currentfill}{rgb}{1.000000,1.000000,1.000000}%
\pgfsetfillcolor{currentfill}%
\pgfsetlinewidth{0.000000pt}%
\definecolor{currentstroke}{rgb}{1.000000,1.000000,1.000000}%
\pgfsetstrokecolor{currentstroke}%
\pgfsetdash{}{0pt}%
\pgfpathmoveto{\pgfqpoint{0.000000in}{0.000000in}}%
\pgfpathlineto{\pgfqpoint{5.534462in}{0.000000in}}%
\pgfpathlineto{\pgfqpoint{5.534462in}{3.694365in}}%
\pgfpathlineto{\pgfqpoint{0.000000in}{3.694365in}}%
\pgfpathclose%
\pgfusepath{fill}%
\end{pgfscope}%
\begin{pgfscope}%
\pgfsetbuttcap%
\pgfsetmiterjoin%
\definecolor{currentfill}{rgb}{1.000000,1.000000,1.000000}%
\pgfsetfillcolor{currentfill}%
\pgfsetlinewidth{0.000000pt}%
\definecolor{currentstroke}{rgb}{0.000000,0.000000,0.000000}%
\pgfsetstrokecolor{currentstroke}%
\pgfsetstrokeopacity{0.000000}%
\pgfsetdash{}{0pt}%
\pgfpathmoveto{\pgfqpoint{0.634105in}{0.521603in}}%
\pgfpathlineto{\pgfqpoint{4.354105in}{0.521603in}}%
\pgfpathlineto{\pgfqpoint{4.354105in}{3.541603in}}%
\pgfpathlineto{\pgfqpoint{0.634105in}{3.541603in}}%
\pgfpathclose%
\pgfusepath{fill}%
\end{pgfscope}%
\begin{pgfscope}%
\pgfsetbuttcap%
\pgfsetroundjoin%
\definecolor{currentfill}{rgb}{0.000000,0.000000,0.000000}%
\pgfsetfillcolor{currentfill}%
\pgfsetlinewidth{0.803000pt}%
\definecolor{currentstroke}{rgb}{0.000000,0.000000,0.000000}%
\pgfsetstrokecolor{currentstroke}%
\pgfsetdash{}{0pt}%
\pgfsys@defobject{currentmarker}{\pgfqpoint{0.000000in}{-0.048611in}}{\pgfqpoint{0.000000in}{0.000000in}}{%
\pgfpathmoveto{\pgfqpoint{0.000000in}{0.000000in}}%
\pgfpathlineto{\pgfqpoint{0.000000in}{-0.048611in}}%
\pgfusepath{stroke,fill}%
}%
\begin{pgfscope}%
\pgfsys@transformshift{0.634105in}{0.521603in}%
\pgfsys@useobject{currentmarker}{}%
\end{pgfscope}%
\end{pgfscope}%
\begin{pgfscope}%
\pgftext[x=0.634105in,y=0.424381in,,top]{\rmfamily\fontsize{10.000000}{12.000000}\selectfont \(\displaystyle 0.0\)}%
\end{pgfscope}%
\begin{pgfscope}%
\pgfsetbuttcap%
\pgfsetroundjoin%
\definecolor{currentfill}{rgb}{0.000000,0.000000,0.000000}%
\pgfsetfillcolor{currentfill}%
\pgfsetlinewidth{0.803000pt}%
\definecolor{currentstroke}{rgb}{0.000000,0.000000,0.000000}%
\pgfsetstrokecolor{currentstroke}%
\pgfsetdash{}{0pt}%
\pgfsys@defobject{currentmarker}{\pgfqpoint{0.000000in}{-0.048611in}}{\pgfqpoint{0.000000in}{0.000000in}}{%
\pgfpathmoveto{\pgfqpoint{0.000000in}{0.000000in}}%
\pgfpathlineto{\pgfqpoint{0.000000in}{-0.048611in}}%
\pgfusepath{stroke,fill}%
}%
\begin{pgfscope}%
\pgfsys@transformshift{1.254105in}{0.521603in}%
\pgfsys@useobject{currentmarker}{}%
\end{pgfscope}%
\end{pgfscope}%
\begin{pgfscope}%
\pgftext[x=1.254105in,y=0.424381in,,top]{\rmfamily\fontsize{10.000000}{12.000000}\selectfont \(\displaystyle 0.5\)}%
\end{pgfscope}%
\begin{pgfscope}%
\pgfsetbuttcap%
\pgfsetroundjoin%
\definecolor{currentfill}{rgb}{0.000000,0.000000,0.000000}%
\pgfsetfillcolor{currentfill}%
\pgfsetlinewidth{0.803000pt}%
\definecolor{currentstroke}{rgb}{0.000000,0.000000,0.000000}%
\pgfsetstrokecolor{currentstroke}%
\pgfsetdash{}{0pt}%
\pgfsys@defobject{currentmarker}{\pgfqpoint{0.000000in}{-0.048611in}}{\pgfqpoint{0.000000in}{0.000000in}}{%
\pgfpathmoveto{\pgfqpoint{0.000000in}{0.000000in}}%
\pgfpathlineto{\pgfqpoint{0.000000in}{-0.048611in}}%
\pgfusepath{stroke,fill}%
}%
\begin{pgfscope}%
\pgfsys@transformshift{1.874105in}{0.521603in}%
\pgfsys@useobject{currentmarker}{}%
\end{pgfscope}%
\end{pgfscope}%
\begin{pgfscope}%
\pgftext[x=1.874105in,y=0.424381in,,top]{\rmfamily\fontsize{10.000000}{12.000000}\selectfont \(\displaystyle 1.0\)}%
\end{pgfscope}%
\begin{pgfscope}%
\pgfsetbuttcap%
\pgfsetroundjoin%
\definecolor{currentfill}{rgb}{0.000000,0.000000,0.000000}%
\pgfsetfillcolor{currentfill}%
\pgfsetlinewidth{0.803000pt}%
\definecolor{currentstroke}{rgb}{0.000000,0.000000,0.000000}%
\pgfsetstrokecolor{currentstroke}%
\pgfsetdash{}{0pt}%
\pgfsys@defobject{currentmarker}{\pgfqpoint{0.000000in}{-0.048611in}}{\pgfqpoint{0.000000in}{0.000000in}}{%
\pgfpathmoveto{\pgfqpoint{0.000000in}{0.000000in}}%
\pgfpathlineto{\pgfqpoint{0.000000in}{-0.048611in}}%
\pgfusepath{stroke,fill}%
}%
\begin{pgfscope}%
\pgfsys@transformshift{2.494105in}{0.521603in}%
\pgfsys@useobject{currentmarker}{}%
\end{pgfscope}%
\end{pgfscope}%
\begin{pgfscope}%
\pgftext[x=2.494105in,y=0.424381in,,top]{\rmfamily\fontsize{10.000000}{12.000000}\selectfont \(\displaystyle 1.5\)}%
\end{pgfscope}%
\begin{pgfscope}%
\pgfsetbuttcap%
\pgfsetroundjoin%
\definecolor{currentfill}{rgb}{0.000000,0.000000,0.000000}%
\pgfsetfillcolor{currentfill}%
\pgfsetlinewidth{0.803000pt}%
\definecolor{currentstroke}{rgb}{0.000000,0.000000,0.000000}%
\pgfsetstrokecolor{currentstroke}%
\pgfsetdash{}{0pt}%
\pgfsys@defobject{currentmarker}{\pgfqpoint{0.000000in}{-0.048611in}}{\pgfqpoint{0.000000in}{0.000000in}}{%
\pgfpathmoveto{\pgfqpoint{0.000000in}{0.000000in}}%
\pgfpathlineto{\pgfqpoint{0.000000in}{-0.048611in}}%
\pgfusepath{stroke,fill}%
}%
\begin{pgfscope}%
\pgfsys@transformshift{3.114105in}{0.521603in}%
\pgfsys@useobject{currentmarker}{}%
\end{pgfscope}%
\end{pgfscope}%
\begin{pgfscope}%
\pgftext[x=3.114105in,y=0.424381in,,top]{\rmfamily\fontsize{10.000000}{12.000000}\selectfont \(\displaystyle 2.0\)}%
\end{pgfscope}%
\begin{pgfscope}%
\pgfsetbuttcap%
\pgfsetroundjoin%
\definecolor{currentfill}{rgb}{0.000000,0.000000,0.000000}%
\pgfsetfillcolor{currentfill}%
\pgfsetlinewidth{0.803000pt}%
\definecolor{currentstroke}{rgb}{0.000000,0.000000,0.000000}%
\pgfsetstrokecolor{currentstroke}%
\pgfsetdash{}{0pt}%
\pgfsys@defobject{currentmarker}{\pgfqpoint{0.000000in}{-0.048611in}}{\pgfqpoint{0.000000in}{0.000000in}}{%
\pgfpathmoveto{\pgfqpoint{0.000000in}{0.000000in}}%
\pgfpathlineto{\pgfqpoint{0.000000in}{-0.048611in}}%
\pgfusepath{stroke,fill}%
}%
\begin{pgfscope}%
\pgfsys@transformshift{3.734105in}{0.521603in}%
\pgfsys@useobject{currentmarker}{}%
\end{pgfscope}%
\end{pgfscope}%
\begin{pgfscope}%
\pgftext[x=3.734105in,y=0.424381in,,top]{\rmfamily\fontsize{10.000000}{12.000000}\selectfont \(\displaystyle 2.5\)}%
\end{pgfscope}%
\begin{pgfscope}%
\pgfsetbuttcap%
\pgfsetroundjoin%
\definecolor{currentfill}{rgb}{0.000000,0.000000,0.000000}%
\pgfsetfillcolor{currentfill}%
\pgfsetlinewidth{0.803000pt}%
\definecolor{currentstroke}{rgb}{0.000000,0.000000,0.000000}%
\pgfsetstrokecolor{currentstroke}%
\pgfsetdash{}{0pt}%
\pgfsys@defobject{currentmarker}{\pgfqpoint{0.000000in}{-0.048611in}}{\pgfqpoint{0.000000in}{0.000000in}}{%
\pgfpathmoveto{\pgfqpoint{0.000000in}{0.000000in}}%
\pgfpathlineto{\pgfqpoint{0.000000in}{-0.048611in}}%
\pgfusepath{stroke,fill}%
}%
\begin{pgfscope}%
\pgfsys@transformshift{4.354105in}{0.521603in}%
\pgfsys@useobject{currentmarker}{}%
\end{pgfscope}%
\end{pgfscope}%
\begin{pgfscope}%
\pgftext[x=4.354105in,y=0.424381in,,top]{\rmfamily\fontsize{10.000000}{12.000000}\selectfont \(\displaystyle 3.0\)}%
\end{pgfscope}%
\begin{pgfscope}%
\pgftext[x=2.494105in,y=0.234413in,,top]{\rmfamily\fontsize{10.000000}{12.000000}\selectfont \(\displaystyle \bar{t}\)}%
\end{pgfscope}%
\begin{pgfscope}%
\pgfsetbuttcap%
\pgfsetroundjoin%
\definecolor{currentfill}{rgb}{0.000000,0.000000,0.000000}%
\pgfsetfillcolor{currentfill}%
\pgfsetlinewidth{0.803000pt}%
\definecolor{currentstroke}{rgb}{0.000000,0.000000,0.000000}%
\pgfsetstrokecolor{currentstroke}%
\pgfsetdash{}{0pt}%
\pgfsys@defobject{currentmarker}{\pgfqpoint{-0.048611in}{0.000000in}}{\pgfqpoint{0.000000in}{0.000000in}}{%
\pgfpathmoveto{\pgfqpoint{0.000000in}{0.000000in}}%
\pgfpathlineto{\pgfqpoint{-0.048611in}{0.000000in}}%
\pgfusepath{stroke,fill}%
}%
\begin{pgfscope}%
\pgfsys@transformshift{0.634105in}{0.521603in}%
\pgfsys@useobject{currentmarker}{}%
\end{pgfscope}%
\end{pgfscope}%
\begin{pgfscope}%
\pgftext[x=0.289968in,y=0.468842in,left,base]{\rmfamily\fontsize{10.000000}{12.000000}\selectfont \(\displaystyle 0.00\)}%
\end{pgfscope}%
\begin{pgfscope}%
\pgfsetbuttcap%
\pgfsetroundjoin%
\definecolor{currentfill}{rgb}{0.000000,0.000000,0.000000}%
\pgfsetfillcolor{currentfill}%
\pgfsetlinewidth{0.803000pt}%
\definecolor{currentstroke}{rgb}{0.000000,0.000000,0.000000}%
\pgfsetstrokecolor{currentstroke}%
\pgfsetdash{}{0pt}%
\pgfsys@defobject{currentmarker}{\pgfqpoint{-0.048611in}{0.000000in}}{\pgfqpoint{0.000000in}{0.000000in}}{%
\pgfpathmoveto{\pgfqpoint{0.000000in}{0.000000in}}%
\pgfpathlineto{\pgfqpoint{-0.048611in}{0.000000in}}%
\pgfusepath{stroke,fill}%
}%
\begin{pgfscope}%
\pgfsys@transformshift{0.634105in}{0.899103in}%
\pgfsys@useobject{currentmarker}{}%
\end{pgfscope}%
\end{pgfscope}%
\begin{pgfscope}%
\pgftext[x=0.289968in,y=0.846342in,left,base]{\rmfamily\fontsize{10.000000}{12.000000}\selectfont \(\displaystyle 0.25\)}%
\end{pgfscope}%
\begin{pgfscope}%
\pgfsetbuttcap%
\pgfsetroundjoin%
\definecolor{currentfill}{rgb}{0.000000,0.000000,0.000000}%
\pgfsetfillcolor{currentfill}%
\pgfsetlinewidth{0.803000pt}%
\definecolor{currentstroke}{rgb}{0.000000,0.000000,0.000000}%
\pgfsetstrokecolor{currentstroke}%
\pgfsetdash{}{0pt}%
\pgfsys@defobject{currentmarker}{\pgfqpoint{-0.048611in}{0.000000in}}{\pgfqpoint{0.000000in}{0.000000in}}{%
\pgfpathmoveto{\pgfqpoint{0.000000in}{0.000000in}}%
\pgfpathlineto{\pgfqpoint{-0.048611in}{0.000000in}}%
\pgfusepath{stroke,fill}%
}%
\begin{pgfscope}%
\pgfsys@transformshift{0.634105in}{1.276603in}%
\pgfsys@useobject{currentmarker}{}%
\end{pgfscope}%
\end{pgfscope}%
\begin{pgfscope}%
\pgftext[x=0.289968in,y=1.223842in,left,base]{\rmfamily\fontsize{10.000000}{12.000000}\selectfont \(\displaystyle 0.50\)}%
\end{pgfscope}%
\begin{pgfscope}%
\pgfsetbuttcap%
\pgfsetroundjoin%
\definecolor{currentfill}{rgb}{0.000000,0.000000,0.000000}%
\pgfsetfillcolor{currentfill}%
\pgfsetlinewidth{0.803000pt}%
\definecolor{currentstroke}{rgb}{0.000000,0.000000,0.000000}%
\pgfsetstrokecolor{currentstroke}%
\pgfsetdash{}{0pt}%
\pgfsys@defobject{currentmarker}{\pgfqpoint{-0.048611in}{0.000000in}}{\pgfqpoint{0.000000in}{0.000000in}}{%
\pgfpathmoveto{\pgfqpoint{0.000000in}{0.000000in}}%
\pgfpathlineto{\pgfqpoint{-0.048611in}{0.000000in}}%
\pgfusepath{stroke,fill}%
}%
\begin{pgfscope}%
\pgfsys@transformshift{0.634105in}{1.654103in}%
\pgfsys@useobject{currentmarker}{}%
\end{pgfscope}%
\end{pgfscope}%
\begin{pgfscope}%
\pgftext[x=0.289968in,y=1.601342in,left,base]{\rmfamily\fontsize{10.000000}{12.000000}\selectfont \(\displaystyle 0.75\)}%
\end{pgfscope}%
\begin{pgfscope}%
\pgfsetbuttcap%
\pgfsetroundjoin%
\definecolor{currentfill}{rgb}{0.000000,0.000000,0.000000}%
\pgfsetfillcolor{currentfill}%
\pgfsetlinewidth{0.803000pt}%
\definecolor{currentstroke}{rgb}{0.000000,0.000000,0.000000}%
\pgfsetstrokecolor{currentstroke}%
\pgfsetdash{}{0pt}%
\pgfsys@defobject{currentmarker}{\pgfqpoint{-0.048611in}{0.000000in}}{\pgfqpoint{0.000000in}{0.000000in}}{%
\pgfpathmoveto{\pgfqpoint{0.000000in}{0.000000in}}%
\pgfpathlineto{\pgfqpoint{-0.048611in}{0.000000in}}%
\pgfusepath{stroke,fill}%
}%
\begin{pgfscope}%
\pgfsys@transformshift{0.634105in}{2.031603in}%
\pgfsys@useobject{currentmarker}{}%
\end{pgfscope}%
\end{pgfscope}%
\begin{pgfscope}%
\pgftext[x=0.289968in,y=1.978842in,left,base]{\rmfamily\fontsize{10.000000}{12.000000}\selectfont \(\displaystyle 1.00\)}%
\end{pgfscope}%
\begin{pgfscope}%
\pgfsetbuttcap%
\pgfsetroundjoin%
\definecolor{currentfill}{rgb}{0.000000,0.000000,0.000000}%
\pgfsetfillcolor{currentfill}%
\pgfsetlinewidth{0.803000pt}%
\definecolor{currentstroke}{rgb}{0.000000,0.000000,0.000000}%
\pgfsetstrokecolor{currentstroke}%
\pgfsetdash{}{0pt}%
\pgfsys@defobject{currentmarker}{\pgfqpoint{-0.048611in}{0.000000in}}{\pgfqpoint{0.000000in}{0.000000in}}{%
\pgfpathmoveto{\pgfqpoint{0.000000in}{0.000000in}}%
\pgfpathlineto{\pgfqpoint{-0.048611in}{0.000000in}}%
\pgfusepath{stroke,fill}%
}%
\begin{pgfscope}%
\pgfsys@transformshift{0.634105in}{2.409103in}%
\pgfsys@useobject{currentmarker}{}%
\end{pgfscope}%
\end{pgfscope}%
\begin{pgfscope}%
\pgftext[x=0.289968in,y=2.356342in,left,base]{\rmfamily\fontsize{10.000000}{12.000000}\selectfont \(\displaystyle 1.25\)}%
\end{pgfscope}%
\begin{pgfscope}%
\pgfsetbuttcap%
\pgfsetroundjoin%
\definecolor{currentfill}{rgb}{0.000000,0.000000,0.000000}%
\pgfsetfillcolor{currentfill}%
\pgfsetlinewidth{0.803000pt}%
\definecolor{currentstroke}{rgb}{0.000000,0.000000,0.000000}%
\pgfsetstrokecolor{currentstroke}%
\pgfsetdash{}{0pt}%
\pgfsys@defobject{currentmarker}{\pgfqpoint{-0.048611in}{0.000000in}}{\pgfqpoint{0.000000in}{0.000000in}}{%
\pgfpathmoveto{\pgfqpoint{0.000000in}{0.000000in}}%
\pgfpathlineto{\pgfqpoint{-0.048611in}{0.000000in}}%
\pgfusepath{stroke,fill}%
}%
\begin{pgfscope}%
\pgfsys@transformshift{0.634105in}{2.786603in}%
\pgfsys@useobject{currentmarker}{}%
\end{pgfscope}%
\end{pgfscope}%
\begin{pgfscope}%
\pgftext[x=0.289968in,y=2.733842in,left,base]{\rmfamily\fontsize{10.000000}{12.000000}\selectfont \(\displaystyle 1.50\)}%
\end{pgfscope}%
\begin{pgfscope}%
\pgfsetbuttcap%
\pgfsetroundjoin%
\definecolor{currentfill}{rgb}{0.000000,0.000000,0.000000}%
\pgfsetfillcolor{currentfill}%
\pgfsetlinewidth{0.803000pt}%
\definecolor{currentstroke}{rgb}{0.000000,0.000000,0.000000}%
\pgfsetstrokecolor{currentstroke}%
\pgfsetdash{}{0pt}%
\pgfsys@defobject{currentmarker}{\pgfqpoint{-0.048611in}{0.000000in}}{\pgfqpoint{0.000000in}{0.000000in}}{%
\pgfpathmoveto{\pgfqpoint{0.000000in}{0.000000in}}%
\pgfpathlineto{\pgfqpoint{-0.048611in}{0.000000in}}%
\pgfusepath{stroke,fill}%
}%
\begin{pgfscope}%
\pgfsys@transformshift{0.634105in}{3.164103in}%
\pgfsys@useobject{currentmarker}{}%
\end{pgfscope}%
\end{pgfscope}%
\begin{pgfscope}%
\pgftext[x=0.289968in,y=3.111342in,left,base]{\rmfamily\fontsize{10.000000}{12.000000}\selectfont \(\displaystyle 1.75\)}%
\end{pgfscope}%
\begin{pgfscope}%
\pgfsetbuttcap%
\pgfsetroundjoin%
\definecolor{currentfill}{rgb}{0.000000,0.000000,0.000000}%
\pgfsetfillcolor{currentfill}%
\pgfsetlinewidth{0.803000pt}%
\definecolor{currentstroke}{rgb}{0.000000,0.000000,0.000000}%
\pgfsetstrokecolor{currentstroke}%
\pgfsetdash{}{0pt}%
\pgfsys@defobject{currentmarker}{\pgfqpoint{-0.048611in}{0.000000in}}{\pgfqpoint{0.000000in}{0.000000in}}{%
\pgfpathmoveto{\pgfqpoint{0.000000in}{0.000000in}}%
\pgfpathlineto{\pgfqpoint{-0.048611in}{0.000000in}}%
\pgfusepath{stroke,fill}%
}%
\begin{pgfscope}%
\pgfsys@transformshift{0.634105in}{3.541603in}%
\pgfsys@useobject{currentmarker}{}%
\end{pgfscope}%
\end{pgfscope}%
\begin{pgfscope}%
\pgftext[x=0.289968in,y=3.488842in,left,base]{\rmfamily\fontsize{10.000000}{12.000000}\selectfont \(\displaystyle 2.00\)}%
\end{pgfscope}%
\begin{pgfscope}%
\pgftext[x=0.234413in,y=2.031603in,,bottom,rotate=90.000000]{\rmfamily\fontsize{10.000000}{12.000000}\selectfont \(\displaystyle \bar{y}\)}%
\end{pgfscope}%
\begin{pgfscope}%
\pgfpathrectangle{\pgfqpoint{0.634105in}{0.521603in}}{\pgfqpoint{3.720000in}{3.020000in}} %
\pgfusepath{clip}%
\pgfsetrectcap%
\pgfsetroundjoin%
\pgfsetlinewidth{1.505625pt}%
\definecolor{currentstroke}{rgb}{1.000000,0.231948,0.116773}%
\pgfsetstrokecolor{currentstroke}%
\pgfsetdash{}{0pt}%
\pgfpathmoveto{\pgfqpoint{0.679135in}{0.567197in}}%
\pgfpathlineto{\pgfqpoint{0.683228in}{0.572448in}}%
\pgfpathlineto{\pgfqpoint{0.687322in}{0.578864in}}%
\pgfpathlineto{\pgfqpoint{0.691416in}{0.579875in}}%
\pgfpathlineto{\pgfqpoint{0.695509in}{0.582942in}}%
\pgfpathlineto{\pgfqpoint{0.699603in}{0.588149in}}%
\pgfpathlineto{\pgfqpoint{0.703697in}{0.589974in}}%
\pgfpathlineto{\pgfqpoint{0.707790in}{0.592367in}}%
\pgfpathlineto{\pgfqpoint{0.711884in}{0.594347in}}%
\pgfpathlineto{\pgfqpoint{0.715978in}{0.595994in}}%
\pgfpathlineto{\pgfqpoint{0.720071in}{0.596556in}}%
\pgfpathlineto{\pgfqpoint{0.724165in}{0.596814in}}%
\pgfpathlineto{\pgfqpoint{0.728258in}{0.597092in}}%
\pgfpathlineto{\pgfqpoint{0.732352in}{0.596210in}}%
\pgfpathlineto{\pgfqpoint{0.736446in}{0.595786in}}%
\pgfpathlineto{\pgfqpoint{0.740539in}{0.594200in}}%
\pgfpathlineto{\pgfqpoint{0.744633in}{0.592635in}}%
\pgfpathlineto{\pgfqpoint{0.748727in}{0.590182in}}%
\pgfpathlineto{\pgfqpoint{0.752820in}{0.587762in}}%
\pgfpathlineto{\pgfqpoint{0.756914in}{0.584396in}}%
\pgfpathlineto{\pgfqpoint{0.761007in}{0.580867in}}%
\pgfpathlineto{\pgfqpoint{0.765101in}{0.576889in}}%
\pgfpathlineto{\pgfqpoint{0.769195in}{0.572173in}}%
\pgfpathlineto{\pgfqpoint{0.773288in}{0.567236in}}%
\pgfpathlineto{\pgfqpoint{0.777382in}{0.561366in}}%
\pgfpathlineto{\pgfqpoint{0.781476in}{0.555609in}}%
\pgfpathlineto{\pgfqpoint{0.785569in}{0.548600in}}%
\pgfpathlineto{\pgfqpoint{0.789663in}{0.541110in}}%
\pgfpathlineto{\pgfqpoint{0.793757in}{0.532166in}}%
\pgfpathlineto{\pgfqpoint{0.797850in}{0.523177in}}%
\pgfpathlineto{\pgfqpoint{0.801944in}{0.511781in}}%
\pgfpathlineto{\pgfqpoint{0.802029in}{0.511603in}}%
\pgfpathmoveto{\pgfqpoint{0.822398in}{0.511603in}}%
\pgfpathlineto{\pgfqpoint{0.822412in}{0.511633in}}%
\pgfpathlineto{\pgfqpoint{0.826506in}{0.519933in}}%
\pgfpathlineto{\pgfqpoint{0.830599in}{0.526199in}}%
\pgfpathlineto{\pgfqpoint{0.834693in}{0.531074in}}%
\pgfpathlineto{\pgfqpoint{0.838786in}{0.534930in}}%
\pgfpathlineto{\pgfqpoint{0.842880in}{0.537227in}}%
\pgfpathlineto{\pgfqpoint{0.846974in}{0.539169in}}%
\pgfpathlineto{\pgfqpoint{0.851067in}{0.539132in}}%
\pgfpathlineto{\pgfqpoint{0.855161in}{0.538285in}}%
\pgfpathlineto{\pgfqpoint{0.859255in}{0.536016in}}%
\pgfpathlineto{\pgfqpoint{0.863348in}{0.532797in}}%
\pgfpathlineto{\pgfqpoint{0.867442in}{0.528126in}}%
\pgfpathlineto{\pgfqpoint{0.871536in}{0.522435in}}%
\pgfpathlineto{\pgfqpoint{0.875629in}{0.513845in}}%
\pgfpathlineto{\pgfqpoint{0.876641in}{0.511603in}}%
\pgfpathmoveto{\pgfqpoint{0.894648in}{0.511603in}}%
\pgfpathlineto{\pgfqpoint{0.896097in}{0.514528in}}%
\pgfpathlineto{\pgfqpoint{0.900191in}{0.522408in}}%
\pgfpathlineto{\pgfqpoint{0.904285in}{0.527497in}}%
\pgfpathlineto{\pgfqpoint{0.908378in}{0.531007in}}%
\pgfpathlineto{\pgfqpoint{0.912472in}{0.533081in}}%
\pgfpathlineto{\pgfqpoint{0.916565in}{0.534014in}}%
\pgfpathlineto{\pgfqpoint{0.920659in}{0.533780in}}%
\pgfpathlineto{\pgfqpoint{0.924753in}{0.532208in}}%
\pgfpathlineto{\pgfqpoint{0.928846in}{0.529697in}}%
\pgfpathlineto{\pgfqpoint{0.932940in}{0.526041in}}%
\pgfpathlineto{\pgfqpoint{0.937034in}{0.519917in}}%
\pgfpathlineto{\pgfqpoint{0.941127in}{0.511912in}}%
\pgfpathlineto{\pgfqpoint{0.941391in}{0.511603in}}%
\pgfusepath{stroke}%
\end{pgfscope}%
\begin{pgfscope}%
\pgfpathrectangle{\pgfqpoint{0.634105in}{0.521603in}}{\pgfqpoint{3.720000in}{3.020000in}} %
\pgfusepath{clip}%
\pgfsetrectcap%
\pgfsetroundjoin%
\pgfsetlinewidth{1.505625pt}%
\definecolor{currentstroke}{rgb}{0.966667,0.743145,0.406737}%
\pgfsetstrokecolor{currentstroke}%
\pgfsetdash{}{0pt}%
\pgfpathmoveto{\pgfqpoint{0.684711in}{0.511603in}}%
\pgfpathlineto{\pgfqpoint{0.693694in}{0.533551in}}%
\pgfpathlineto{\pgfqpoint{0.703625in}{0.550991in}}%
\pgfpathlineto{\pgfqpoint{0.723488in}{0.582094in}}%
\pgfpathlineto{\pgfqpoint{0.733420in}{0.597498in}}%
\pgfpathlineto{\pgfqpoint{0.743351in}{0.619939in}}%
\pgfpathlineto{\pgfqpoint{0.753282in}{0.635169in}}%
\pgfpathlineto{\pgfqpoint{0.773145in}{0.652145in}}%
\pgfpathlineto{\pgfqpoint{0.783077in}{0.658132in}}%
\pgfpathlineto{\pgfqpoint{0.793008in}{0.670496in}}%
\pgfpathlineto{\pgfqpoint{0.812871in}{0.687318in}}%
\pgfpathlineto{\pgfqpoint{0.822803in}{0.686385in}}%
\pgfpathlineto{\pgfqpoint{0.832734in}{0.687719in}}%
\pgfpathlineto{\pgfqpoint{0.862529in}{0.706007in}}%
\pgfpathlineto{\pgfqpoint{0.872460in}{0.704489in}}%
\pgfpathlineto{\pgfqpoint{0.882391in}{0.699912in}}%
\pgfpathlineto{\pgfqpoint{0.892323in}{0.698852in}}%
\pgfpathlineto{\pgfqpoint{0.912186in}{0.703031in}}%
\pgfpathlineto{\pgfqpoint{0.941980in}{0.680835in}}%
\pgfpathlineto{\pgfqpoint{0.951912in}{0.675676in}}%
\pgfpathlineto{\pgfqpoint{0.961843in}{0.675573in}}%
\pgfpathlineto{\pgfqpoint{0.971775in}{0.667871in}}%
\pgfpathlineto{\pgfqpoint{0.981706in}{0.655834in}}%
\pgfpathlineto{\pgfqpoint{0.991638in}{0.641305in}}%
\pgfpathlineto{\pgfqpoint{1.001569in}{0.634261in}}%
\pgfpathlineto{\pgfqpoint{1.011501in}{0.625601in}}%
\pgfpathlineto{\pgfqpoint{1.021432in}{0.613190in}}%
\pgfpathlineto{\pgfqpoint{1.031363in}{0.596585in}}%
\pgfpathlineto{\pgfqpoint{1.051226in}{0.556099in}}%
\pgfpathlineto{\pgfqpoint{1.061158in}{0.539079in}}%
\pgfpathlineto{\pgfqpoint{1.073548in}{0.511603in}}%
\pgfpathmoveto{\pgfqpoint{1.162302in}{0.511603in}}%
\pgfpathlineto{\pgfqpoint{1.170404in}{0.528482in}}%
\pgfpathlineto{\pgfqpoint{1.190267in}{0.561201in}}%
\pgfpathlineto{\pgfqpoint{1.210130in}{0.594399in}}%
\pgfpathlineto{\pgfqpoint{1.220061in}{0.608903in}}%
\pgfpathlineto{\pgfqpoint{1.229993in}{0.617567in}}%
\pgfpathlineto{\pgfqpoint{1.249856in}{0.630064in}}%
\pgfpathlineto{\pgfqpoint{1.259787in}{0.640353in}}%
\pgfpathlineto{\pgfqpoint{1.269719in}{0.647353in}}%
\pgfpathlineto{\pgfqpoint{1.279650in}{0.650631in}}%
\pgfpathlineto{\pgfqpoint{1.299513in}{0.651555in}}%
\pgfpathlineto{\pgfqpoint{1.319376in}{0.657982in}}%
\pgfpathlineto{\pgfqpoint{1.329307in}{0.658316in}}%
\pgfpathlineto{\pgfqpoint{1.339239in}{0.655264in}}%
\pgfpathlineto{\pgfqpoint{1.349170in}{0.648368in}}%
\pgfpathlineto{\pgfqpoint{1.359102in}{0.646835in}}%
\pgfpathlineto{\pgfqpoint{1.369033in}{0.643831in}}%
\pgfpathlineto{\pgfqpoint{1.378965in}{0.639500in}}%
\pgfpathlineto{\pgfqpoint{1.408759in}{0.609630in}}%
\pgfpathlineto{\pgfqpoint{1.428622in}{0.592100in}}%
\pgfpathlineto{\pgfqpoint{1.438553in}{0.577611in}}%
\pgfpathlineto{\pgfqpoint{1.458416in}{0.541999in}}%
\pgfpathlineto{\pgfqpoint{1.473646in}{0.511603in}}%
\pgfpathmoveto{\pgfqpoint{1.571674in}{0.511603in}}%
\pgfpathlineto{\pgfqpoint{1.577594in}{0.519187in}}%
\pgfpathlineto{\pgfqpoint{1.597457in}{0.550353in}}%
\pgfpathlineto{\pgfqpoint{1.617320in}{0.565899in}}%
\pgfpathlineto{\pgfqpoint{1.637183in}{0.569407in}}%
\pgfpathlineto{\pgfqpoint{1.647114in}{0.568388in}}%
\pgfpathlineto{\pgfqpoint{1.666977in}{0.571352in}}%
\pgfpathlineto{\pgfqpoint{1.686840in}{0.555446in}}%
\pgfpathlineto{\pgfqpoint{1.696772in}{0.541893in}}%
\pgfpathlineto{\pgfqpoint{1.706703in}{0.532829in}}%
\pgfpathlineto{\pgfqpoint{1.725427in}{0.511603in}}%
\pgfpathmoveto{\pgfqpoint{1.831808in}{0.511603in}}%
\pgfpathlineto{\pgfqpoint{1.835812in}{0.516648in}}%
\pgfpathlineto{\pgfqpoint{1.845744in}{0.525459in}}%
\pgfpathlineto{\pgfqpoint{1.865606in}{0.538455in}}%
\pgfpathlineto{\pgfqpoint{1.885469in}{0.545000in}}%
\pgfpathlineto{\pgfqpoint{1.905332in}{0.549671in}}%
\pgfpathlineto{\pgfqpoint{1.915264in}{0.546432in}}%
\pgfpathlineto{\pgfqpoint{1.925195in}{0.541396in}}%
\pgfpathlineto{\pgfqpoint{1.935127in}{0.528701in}}%
\pgfpathlineto{\pgfqpoint{1.945058in}{0.519179in}}%
\pgfpathlineto{\pgfqpoint{1.958048in}{0.511603in}}%
\pgfpathmoveto{\pgfqpoint{2.078334in}{0.511603in}}%
\pgfpathlineto{\pgfqpoint{2.103962in}{0.532231in}}%
\pgfpathlineto{\pgfqpoint{2.113893in}{0.532527in}}%
\pgfpathlineto{\pgfqpoint{2.123825in}{0.531260in}}%
\pgfpathlineto{\pgfqpoint{2.133756in}{0.533424in}}%
\pgfpathlineto{\pgfqpoint{2.143687in}{0.528623in}}%
\pgfpathlineto{\pgfqpoint{2.153619in}{0.522181in}}%
\pgfpathlineto{\pgfqpoint{2.166155in}{0.511603in}}%
\pgfpathmoveto{\pgfqpoint{2.295413in}{0.511603in}}%
\pgfpathlineto{\pgfqpoint{2.312522in}{0.524666in}}%
\pgfpathlineto{\pgfqpoint{2.322454in}{0.528499in}}%
\pgfpathlineto{\pgfqpoint{2.342317in}{0.528867in}}%
\pgfpathlineto{\pgfqpoint{2.352248in}{0.525934in}}%
\pgfpathlineto{\pgfqpoint{2.372111in}{0.512800in}}%
\pgfpathlineto{\pgfqpoint{2.373618in}{0.511603in}}%
\pgfpathmoveto{\pgfqpoint{2.509184in}{0.511603in}}%
\pgfpathlineto{\pgfqpoint{2.511152in}{0.513326in}}%
\pgfpathlineto{\pgfqpoint{2.521083in}{0.520296in}}%
\pgfpathlineto{\pgfqpoint{2.531015in}{0.524421in}}%
\pgfpathlineto{\pgfqpoint{2.540946in}{0.526322in}}%
\pgfpathlineto{\pgfqpoint{2.550877in}{0.525648in}}%
\pgfpathlineto{\pgfqpoint{2.560809in}{0.521972in}}%
\pgfpathlineto{\pgfqpoint{2.577772in}{0.511603in}}%
\pgfpathmoveto{\pgfqpoint{2.721172in}{0.511603in}}%
\pgfpathlineto{\pgfqpoint{2.729644in}{0.517486in}}%
\pgfpathlineto{\pgfqpoint{2.739575in}{0.522447in}}%
\pgfpathlineto{\pgfqpoint{2.759438in}{0.522684in}}%
\pgfpathlineto{\pgfqpoint{2.769370in}{0.518631in}}%
\pgfpathlineto{\pgfqpoint{2.778545in}{0.511603in}}%
\pgfpathlineto{\pgfqpoint{2.778545in}{0.511603in}}%
\pgfusepath{stroke}%
\end{pgfscope}%
\begin{pgfscope}%
\pgfpathrectangle{\pgfqpoint{0.634105in}{0.521603in}}{\pgfqpoint{3.720000in}{3.020000in}} %
\pgfusepath{clip}%
\pgfsetrectcap%
\pgfsetroundjoin%
\pgfsetlinewidth{1.505625pt}%
\definecolor{currentstroke}{rgb}{1.000000,0.000000,0.000000}%
\pgfsetstrokecolor{currentstroke}%
\pgfsetdash{}{0pt}%
\pgfpathmoveto{\pgfqpoint{0.674529in}{0.606569in}}%
\pgfpathlineto{\pgfqpoint{0.683512in}{0.616333in}}%
\pgfpathlineto{\pgfqpoint{0.688004in}{0.620013in}}%
\pgfpathlineto{\pgfqpoint{0.696987in}{0.629290in}}%
\pgfpathlineto{\pgfqpoint{0.701479in}{0.631299in}}%
\pgfpathlineto{\pgfqpoint{0.710462in}{0.639076in}}%
\pgfpathlineto{\pgfqpoint{0.714953in}{0.638660in}}%
\pgfpathlineto{\pgfqpoint{0.723936in}{0.642534in}}%
\pgfpathlineto{\pgfqpoint{0.728428in}{0.642124in}}%
\pgfpathlineto{\pgfqpoint{0.737411in}{0.642975in}}%
\pgfpathlineto{\pgfqpoint{0.755377in}{0.637288in}}%
\pgfpathlineto{\pgfqpoint{0.768852in}{0.628197in}}%
\pgfpathlineto{\pgfqpoint{0.782327in}{0.614703in}}%
\pgfpathlineto{\pgfqpoint{0.795802in}{0.596620in}}%
\pgfpathlineto{\pgfqpoint{0.804785in}{0.580174in}}%
\pgfpathlineto{\pgfqpoint{0.809276in}{0.571238in}}%
\pgfpathlineto{\pgfqpoint{0.822751in}{0.534086in}}%
\pgfpathlineto{\pgfqpoint{0.827243in}{0.527421in}}%
\pgfpathlineto{\pgfqpoint{0.831734in}{0.529998in}}%
\pgfpathlineto{\pgfqpoint{0.849701in}{0.567880in}}%
\pgfpathlineto{\pgfqpoint{0.863175in}{0.581526in}}%
\pgfpathlineto{\pgfqpoint{0.872159in}{0.584644in}}%
\pgfpathlineto{\pgfqpoint{0.876650in}{0.584366in}}%
\pgfpathlineto{\pgfqpoint{0.885633in}{0.580293in}}%
\pgfpathlineto{\pgfqpoint{0.894616in}{0.572453in}}%
\pgfpathlineto{\pgfqpoint{0.903600in}{0.558238in}}%
\pgfpathlineto{\pgfqpoint{0.917074in}{0.529006in}}%
\pgfpathlineto{\pgfqpoint{0.921566in}{0.526426in}}%
\pgfpathlineto{\pgfqpoint{0.926057in}{0.530750in}}%
\pgfpathlineto{\pgfqpoint{0.939532in}{0.555518in}}%
\pgfpathlineto{\pgfqpoint{0.948515in}{0.563181in}}%
\pgfpathlineto{\pgfqpoint{0.953007in}{0.566396in}}%
\pgfpathlineto{\pgfqpoint{0.961990in}{0.565590in}}%
\pgfpathlineto{\pgfqpoint{0.966482in}{0.564378in}}%
\pgfpathlineto{\pgfqpoint{0.975465in}{0.556141in}}%
\pgfpathlineto{\pgfqpoint{0.979956in}{0.549731in}}%
\pgfpathlineto{\pgfqpoint{0.988940in}{0.533023in}}%
\pgfpathlineto{\pgfqpoint{0.993431in}{0.526776in}}%
\pgfpathlineto{\pgfqpoint{0.997923in}{0.527013in}}%
\pgfpathlineto{\pgfqpoint{1.002414in}{0.531980in}}%
\pgfpathlineto{\pgfqpoint{1.006906in}{0.540400in}}%
\pgfpathlineto{\pgfqpoint{1.015889in}{0.550956in}}%
\pgfpathlineto{\pgfqpoint{1.020381in}{0.552893in}}%
\pgfpathlineto{\pgfqpoint{1.024872in}{0.552804in}}%
\pgfpathlineto{\pgfqpoint{1.029364in}{0.551129in}}%
\pgfpathlineto{\pgfqpoint{1.033855in}{0.547597in}}%
\pgfpathlineto{\pgfqpoint{1.042838in}{0.534568in}}%
\pgfpathlineto{\pgfqpoint{1.047330in}{0.527910in}}%
\pgfpathlineto{\pgfqpoint{1.051822in}{0.525628in}}%
\pgfpathlineto{\pgfqpoint{1.056313in}{0.528987in}}%
\pgfpathlineto{\pgfqpoint{1.065296in}{0.541915in}}%
\pgfpathlineto{\pgfqpoint{1.069788in}{0.545347in}}%
\pgfpathlineto{\pgfqpoint{1.074280in}{0.545576in}}%
\pgfpathlineto{\pgfqpoint{1.078771in}{0.543098in}}%
\pgfpathlineto{\pgfqpoint{1.092246in}{0.529687in}}%
\pgfpathlineto{\pgfqpoint{1.096737in}{0.528651in}}%
\pgfpathlineto{\pgfqpoint{1.101229in}{0.530392in}}%
\pgfpathlineto{\pgfqpoint{1.110212in}{0.539194in}}%
\pgfpathlineto{\pgfqpoint{1.114704in}{0.541671in}}%
\pgfpathlineto{\pgfqpoint{1.119195in}{0.541046in}}%
\pgfpathlineto{\pgfqpoint{1.132670in}{0.528635in}}%
\pgfpathlineto{\pgfqpoint{1.137162in}{0.528416in}}%
\pgfpathlineto{\pgfqpoint{1.141653in}{0.531284in}}%
\pgfpathlineto{\pgfqpoint{1.150636in}{0.538934in}}%
\pgfpathlineto{\pgfqpoint{1.155128in}{0.540202in}}%
\pgfpathlineto{\pgfqpoint{1.159620in}{0.538609in}}%
\pgfpathlineto{\pgfqpoint{1.173094in}{0.530291in}}%
\pgfpathlineto{\pgfqpoint{1.177586in}{0.530619in}}%
\pgfpathlineto{\pgfqpoint{1.195552in}{0.538402in}}%
\pgfpathlineto{\pgfqpoint{1.200044in}{0.534851in}}%
\pgfpathlineto{\pgfqpoint{1.213518in}{0.531767in}}%
\pgfpathlineto{\pgfqpoint{1.231485in}{0.537803in}}%
\pgfpathlineto{\pgfqpoint{1.240468in}{0.535595in}}%
\pgfpathlineto{\pgfqpoint{1.249451in}{0.532448in}}%
\pgfpathlineto{\pgfqpoint{1.258434in}{0.533323in}}%
\pgfpathlineto{\pgfqpoint{1.271909in}{0.537449in}}%
\pgfpathlineto{\pgfqpoint{1.276401in}{0.537362in}}%
\pgfpathlineto{\pgfqpoint{1.294367in}{0.533152in}}%
\pgfpathlineto{\pgfqpoint{1.316825in}{0.535892in}}%
\pgfpathlineto{\pgfqpoint{1.325808in}{0.533478in}}%
\pgfpathlineto{\pgfqpoint{1.334791in}{0.533423in}}%
\pgfpathlineto{\pgfqpoint{1.348266in}{0.536236in}}%
\pgfpathlineto{\pgfqpoint{1.357249in}{0.534576in}}%
\pgfpathlineto{\pgfqpoint{1.366232in}{0.532832in}}%
\pgfpathlineto{\pgfqpoint{1.375215in}{0.534040in}}%
\pgfpathlineto{\pgfqpoint{1.388690in}{0.536062in}}%
\pgfpathlineto{\pgfqpoint{1.411148in}{0.533477in}}%
\pgfpathlineto{\pgfqpoint{1.429114in}{0.535737in}}%
\pgfpathlineto{\pgfqpoint{1.447081in}{0.533045in}}%
\pgfpathlineto{\pgfqpoint{1.469538in}{0.534705in}}%
\pgfpathlineto{\pgfqpoint{1.483013in}{0.533351in}}%
\pgfpathlineto{\pgfqpoint{1.505471in}{0.533820in}}%
\pgfpathlineto{\pgfqpoint{1.518946in}{0.532966in}}%
\pgfpathlineto{\pgfqpoint{1.541404in}{0.533713in}}%
\pgfpathlineto{\pgfqpoint{1.559370in}{0.533396in}}%
\pgfpathlineto{\pgfqpoint{1.577336in}{0.533893in}}%
\pgfpathlineto{\pgfqpoint{1.595303in}{0.533399in}}%
\pgfpathlineto{\pgfqpoint{1.608777in}{0.534036in}}%
\pgfpathlineto{\pgfqpoint{1.626744in}{0.533194in}}%
\pgfpathlineto{\pgfqpoint{1.649202in}{0.534036in}}%
\pgfpathlineto{\pgfqpoint{1.667168in}{0.533542in}}%
\pgfpathlineto{\pgfqpoint{1.676151in}{0.534240in}}%
\pgfpathlineto{\pgfqpoint{1.676151in}{0.534240in}}%
\pgfusepath{stroke}%
\end{pgfscope}%
\begin{pgfscope}%
\pgfpathrectangle{\pgfqpoint{0.634105in}{0.521603in}}{\pgfqpoint{3.720000in}{3.020000in}} %
\pgfusepath{clip}%
\pgfsetrectcap%
\pgfsetroundjoin%
\pgfsetlinewidth{1.505625pt}%
\definecolor{currentstroke}{rgb}{0.966667,0.743145,0.406737}%
\pgfsetstrokecolor{currentstroke}%
\pgfsetdash{}{0pt}%
\pgfpathmoveto{\pgfqpoint{0.757314in}{0.625750in}}%
\pgfpathlineto{\pgfqpoint{0.772715in}{0.647214in}}%
\pgfpathlineto{\pgfqpoint{0.788116in}{0.664234in}}%
\pgfpathlineto{\pgfqpoint{0.803517in}{0.675281in}}%
\pgfpathlineto{\pgfqpoint{0.834319in}{0.705904in}}%
\pgfpathlineto{\pgfqpoint{0.849720in}{0.712987in}}%
\pgfpathlineto{\pgfqpoint{0.865122in}{0.725114in}}%
\pgfpathlineto{\pgfqpoint{0.880523in}{0.733527in}}%
\pgfpathlineto{\pgfqpoint{0.895924in}{0.737410in}}%
\pgfpathlineto{\pgfqpoint{0.911325in}{0.746429in}}%
\pgfpathlineto{\pgfqpoint{0.926726in}{0.749348in}}%
\pgfpathlineto{\pgfqpoint{0.942127in}{0.748664in}}%
\pgfpathlineto{\pgfqpoint{0.957528in}{0.755146in}}%
\pgfpathlineto{\pgfqpoint{0.972929in}{0.754094in}}%
\pgfpathlineto{\pgfqpoint{0.988330in}{0.750332in}}%
\pgfpathlineto{\pgfqpoint{1.003732in}{0.752171in}}%
\pgfpathlineto{\pgfqpoint{1.034534in}{0.740414in}}%
\pgfpathlineto{\pgfqpoint{1.049935in}{0.737340in}}%
\pgfpathlineto{\pgfqpoint{1.080737in}{0.717459in}}%
\pgfpathlineto{\pgfqpoint{1.096138in}{0.710640in}}%
\pgfpathlineto{\pgfqpoint{1.111539in}{0.696048in}}%
\pgfpathlineto{\pgfqpoint{1.142342in}{0.672116in}}%
\pgfpathlineto{\pgfqpoint{1.173144in}{0.635465in}}%
\pgfpathlineto{\pgfqpoint{1.188545in}{0.619236in}}%
\pgfpathlineto{\pgfqpoint{1.234748in}{0.547016in}}%
\pgfpathlineto{\pgfqpoint{1.250149in}{0.518873in}}%
\pgfpathlineto{\pgfqpoint{1.255943in}{0.511603in}}%
\pgfpathmoveto{\pgfqpoint{1.330286in}{0.511603in}}%
\pgfpathlineto{\pgfqpoint{1.342556in}{0.531415in}}%
\pgfpathlineto{\pgfqpoint{1.357957in}{0.552943in}}%
\pgfpathlineto{\pgfqpoint{1.373358in}{0.568471in}}%
\pgfpathlineto{\pgfqpoint{1.388759in}{0.585711in}}%
\pgfpathlineto{\pgfqpoint{1.404160in}{0.598984in}}%
\pgfpathlineto{\pgfqpoint{1.419561in}{0.602060in}}%
\pgfpathlineto{\pgfqpoint{1.434963in}{0.615726in}}%
\pgfpathlineto{\pgfqpoint{1.450364in}{0.622081in}}%
\pgfpathlineto{\pgfqpoint{1.465765in}{0.619922in}}%
\pgfpathlineto{\pgfqpoint{1.481166in}{0.625901in}}%
\pgfpathlineto{\pgfqpoint{1.496567in}{0.628487in}}%
\pgfpathlineto{\pgfqpoint{1.511968in}{0.621430in}}%
\pgfpathlineto{\pgfqpoint{1.527369in}{0.623101in}}%
\pgfpathlineto{\pgfqpoint{1.542770in}{0.616497in}}%
\pgfpathlineto{\pgfqpoint{1.558171in}{0.606210in}}%
\pgfpathlineto{\pgfqpoint{1.573573in}{0.601948in}}%
\pgfpathlineto{\pgfqpoint{1.588974in}{0.589059in}}%
\pgfpathlineto{\pgfqpoint{1.604375in}{0.572453in}}%
\pgfpathlineto{\pgfqpoint{1.619776in}{0.560373in}}%
\pgfpathlineto{\pgfqpoint{1.656677in}{0.511603in}}%
\pgfpathmoveto{\pgfqpoint{1.733800in}{0.511603in}}%
\pgfpathlineto{\pgfqpoint{1.742985in}{0.521711in}}%
\pgfpathlineto{\pgfqpoint{1.758386in}{0.541868in}}%
\pgfpathlineto{\pgfqpoint{1.773787in}{0.553572in}}%
\pgfpathlineto{\pgfqpoint{1.789188in}{0.568459in}}%
\pgfpathlineto{\pgfqpoint{1.804589in}{0.573963in}}%
\pgfpathlineto{\pgfqpoint{1.819990in}{0.574364in}}%
\pgfpathlineto{\pgfqpoint{1.835391in}{0.577997in}}%
\pgfpathlineto{\pgfqpoint{1.850793in}{0.575150in}}%
\pgfpathlineto{\pgfqpoint{1.881595in}{0.557096in}}%
\pgfpathlineto{\pgfqpoint{1.896996in}{0.543161in}}%
\pgfpathlineto{\pgfqpoint{1.912397in}{0.522643in}}%
\pgfpathlineto{\pgfqpoint{1.926008in}{0.511603in}}%
\pgfpathmoveto{\pgfqpoint{2.000997in}{0.511603in}}%
\pgfpathlineto{\pgfqpoint{2.004804in}{0.514640in}}%
\pgfpathlineto{\pgfqpoint{2.020205in}{0.534842in}}%
\pgfpathlineto{\pgfqpoint{2.035606in}{0.548195in}}%
\pgfpathlineto{\pgfqpoint{2.051007in}{0.556620in}}%
\pgfpathlineto{\pgfqpoint{2.066408in}{0.563050in}}%
\pgfpathlineto{\pgfqpoint{2.081809in}{0.564707in}}%
\pgfpathlineto{\pgfqpoint{2.112611in}{0.557227in}}%
\pgfpathlineto{\pgfqpoint{2.128013in}{0.549363in}}%
\pgfpathlineto{\pgfqpoint{2.143414in}{0.535244in}}%
\pgfpathlineto{\pgfqpoint{2.158815in}{0.525198in}}%
\pgfpathlineto{\pgfqpoint{2.170680in}{0.511603in}}%
\pgfpathmoveto{\pgfqpoint{2.241182in}{0.511603in}}%
\pgfpathlineto{\pgfqpoint{2.251221in}{0.521918in}}%
\pgfpathlineto{\pgfqpoint{2.282024in}{0.544526in}}%
\pgfpathlineto{\pgfqpoint{2.297425in}{0.551882in}}%
\pgfpathlineto{\pgfqpoint{2.312826in}{0.555132in}}%
\pgfpathlineto{\pgfqpoint{2.328227in}{0.553199in}}%
\pgfpathlineto{\pgfqpoint{2.343628in}{0.545932in}}%
\pgfpathlineto{\pgfqpoint{2.374430in}{0.522114in}}%
\pgfpathlineto{\pgfqpoint{2.386821in}{0.511603in}}%
\pgfpathmoveto{\pgfqpoint{2.451362in}{0.511603in}}%
\pgfpathlineto{\pgfqpoint{2.497639in}{0.546702in}}%
\pgfpathlineto{\pgfqpoint{2.513040in}{0.552660in}}%
\pgfpathlineto{\pgfqpoint{2.528441in}{0.554987in}}%
\pgfpathlineto{\pgfqpoint{2.543843in}{0.551510in}}%
\pgfpathlineto{\pgfqpoint{2.559244in}{0.543980in}}%
\pgfpathlineto{\pgfqpoint{2.590046in}{0.520550in}}%
\pgfpathlineto{\pgfqpoint{2.602114in}{0.511603in}}%
\pgfpathmoveto{\pgfqpoint{2.655833in}{0.511603in}}%
\pgfpathlineto{\pgfqpoint{2.697854in}{0.544167in}}%
\pgfpathlineto{\pgfqpoint{2.713255in}{0.551956in}}%
\pgfpathlineto{\pgfqpoint{2.728656in}{0.555492in}}%
\pgfpathlineto{\pgfqpoint{2.744057in}{0.554124in}}%
\pgfpathlineto{\pgfqpoint{2.759458in}{0.548100in}}%
\pgfpathlineto{\pgfqpoint{2.790260in}{0.527853in}}%
\pgfpathlineto{\pgfqpoint{2.814690in}{0.511603in}}%
\pgfpathmoveto{\pgfqpoint{2.853794in}{0.511603in}}%
\pgfpathlineto{\pgfqpoint{2.867266in}{0.519769in}}%
\pgfpathlineto{\pgfqpoint{2.898068in}{0.542514in}}%
\pgfpathlineto{\pgfqpoint{2.913469in}{0.551721in}}%
\pgfpathlineto{\pgfqpoint{2.928870in}{0.556047in}}%
\pgfpathlineto{\pgfqpoint{2.944271in}{0.557474in}}%
\pgfpathlineto{\pgfqpoint{2.959673in}{0.554045in}}%
\pgfpathlineto{\pgfqpoint{2.990475in}{0.537387in}}%
\pgfpathlineto{\pgfqpoint{3.021277in}{0.517253in}}%
\pgfpathlineto{\pgfqpoint{3.036124in}{0.511603in}}%
\pgfpathmoveto{\pgfqpoint{3.053396in}{0.511603in}}%
\pgfpathlineto{\pgfqpoint{3.067480in}{0.516610in}}%
\pgfpathlineto{\pgfqpoint{3.113684in}{0.546533in}}%
\pgfpathlineto{\pgfqpoint{3.129085in}{0.554096in}}%
\pgfpathlineto{\pgfqpoint{3.144486in}{0.557355in}}%
\pgfpathlineto{\pgfqpoint{3.159887in}{0.556835in}}%
\pgfpathlineto{\pgfqpoint{3.175288in}{0.552420in}}%
\pgfpathlineto{\pgfqpoint{3.206090in}{0.534916in}}%
\pgfpathlineto{\pgfqpoint{3.221491in}{0.524501in}}%
\pgfpathlineto{\pgfqpoint{3.236893in}{0.517757in}}%
\pgfpathlineto{\pgfqpoint{3.252294in}{0.515346in}}%
\pgfpathlineto{\pgfqpoint{3.267695in}{0.516993in}}%
\pgfpathlineto{\pgfqpoint{3.283096in}{0.524640in}}%
\pgfpathlineto{\pgfqpoint{3.313898in}{0.543088in}}%
\pgfpathlineto{\pgfqpoint{3.329299in}{0.552142in}}%
\pgfpathlineto{\pgfqpoint{3.344700in}{0.558552in}}%
\pgfpathlineto{\pgfqpoint{3.360101in}{0.559246in}}%
\pgfpathlineto{\pgfqpoint{3.375503in}{0.557739in}}%
\pgfpathlineto{\pgfqpoint{3.390904in}{0.551775in}}%
\pgfpathlineto{\pgfqpoint{3.452508in}{0.519923in}}%
\pgfpathlineto{\pgfqpoint{3.467909in}{0.518426in}}%
\pgfpathlineto{\pgfqpoint{3.483310in}{0.521241in}}%
\pgfpathlineto{\pgfqpoint{3.498711in}{0.527261in}}%
\pgfpathlineto{\pgfqpoint{3.529514in}{0.543756in}}%
\pgfpathlineto{\pgfqpoint{3.544915in}{0.550012in}}%
\pgfpathlineto{\pgfqpoint{3.560316in}{0.554648in}}%
\pgfpathlineto{\pgfqpoint{3.575717in}{0.555702in}}%
\pgfpathlineto{\pgfqpoint{3.591118in}{0.552240in}}%
\pgfpathlineto{\pgfqpoint{3.606519in}{0.546987in}}%
\pgfpathlineto{\pgfqpoint{3.652723in}{0.525740in}}%
\pgfpathlineto{\pgfqpoint{3.668124in}{0.521945in}}%
\pgfpathlineto{\pgfqpoint{3.683525in}{0.521625in}}%
\pgfpathlineto{\pgfqpoint{3.698926in}{0.524999in}}%
\pgfpathlineto{\pgfqpoint{3.714327in}{0.530328in}}%
\pgfpathlineto{\pgfqpoint{3.745129in}{0.544184in}}%
\pgfpathlineto{\pgfqpoint{3.760530in}{0.548434in}}%
\pgfpathlineto{\pgfqpoint{3.775931in}{0.550501in}}%
\pgfpathlineto{\pgfqpoint{3.791333in}{0.549883in}}%
\pgfpathlineto{\pgfqpoint{3.806734in}{0.546075in}}%
\pgfpathlineto{\pgfqpoint{3.852937in}{0.527944in}}%
\pgfpathlineto{\pgfqpoint{3.868338in}{0.524324in}}%
\pgfpathlineto{\pgfqpoint{3.883739in}{0.523413in}}%
\pgfpathlineto{\pgfqpoint{3.899140in}{0.525199in}}%
\pgfpathlineto{\pgfqpoint{3.914541in}{0.529873in}}%
\pgfpathlineto{\pgfqpoint{3.929943in}{0.536038in}}%
\pgfpathlineto{\pgfqpoint{3.960745in}{0.545210in}}%
\pgfpathlineto{\pgfqpoint{3.976146in}{0.547218in}}%
\pgfpathlineto{\pgfqpoint{3.991547in}{0.546335in}}%
\pgfpathlineto{\pgfqpoint{4.006948in}{0.543521in}}%
\pgfpathlineto{\pgfqpoint{4.037750in}{0.533744in}}%
\pgfpathlineto{\pgfqpoint{4.053151in}{0.528955in}}%
\pgfpathlineto{\pgfqpoint{4.068553in}{0.525890in}}%
\pgfpathlineto{\pgfqpoint{4.083954in}{0.524932in}}%
\pgfpathlineto{\pgfqpoint{4.099355in}{0.527068in}}%
\pgfpathlineto{\pgfqpoint{4.130157in}{0.534764in}}%
\pgfpathlineto{\pgfqpoint{4.145558in}{0.539952in}}%
\pgfpathlineto{\pgfqpoint{4.160959in}{0.542876in}}%
\pgfpathlineto{\pgfqpoint{4.176360in}{0.544092in}}%
\pgfpathlineto{\pgfqpoint{4.191761in}{0.543474in}}%
\pgfpathlineto{\pgfqpoint{4.237965in}{0.532836in}}%
\pgfpathlineto{\pgfqpoint{4.237965in}{0.532836in}}%
\pgfusepath{stroke}%
\end{pgfscope}%
\begin{pgfscope}%
\pgfpathrectangle{\pgfqpoint{0.634105in}{0.521603in}}{\pgfqpoint{3.720000in}{3.020000in}} %
\pgfusepath{clip}%
\pgfsetrectcap%
\pgfsetroundjoin%
\pgfsetlinewidth{1.505625pt}%
\definecolor{currentstroke}{rgb}{0.770588,0.911023,0.542053}%
\pgfsetstrokecolor{currentstroke}%
\pgfsetdash{}{0pt}%
\pgfpathmoveto{\pgfqpoint{0.890481in}{0.760480in}}%
\pgfpathlineto{\pgfqpoint{0.916119in}{0.795136in}}%
\pgfpathlineto{\pgfqpoint{0.941757in}{0.819646in}}%
\pgfpathlineto{\pgfqpoint{0.967394in}{0.844131in}}%
\pgfpathlineto{\pgfqpoint{0.993032in}{0.876460in}}%
\pgfpathlineto{\pgfqpoint{1.018670in}{0.898035in}}%
\pgfpathlineto{\pgfqpoint{1.044307in}{0.919042in}}%
\pgfpathlineto{\pgfqpoint{1.069945in}{0.947140in}}%
\pgfpathlineto{\pgfqpoint{1.095583in}{0.964598in}}%
\pgfpathlineto{\pgfqpoint{1.121220in}{0.984462in}}%
\pgfpathlineto{\pgfqpoint{1.146858in}{1.005407in}}%
\pgfpathlineto{\pgfqpoint{1.172496in}{1.018298in}}%
\pgfpathlineto{\pgfqpoint{1.198133in}{1.038036in}}%
\pgfpathlineto{\pgfqpoint{1.223771in}{1.051288in}}%
\pgfpathlineto{\pgfqpoint{1.249409in}{1.064227in}}%
\pgfpathlineto{\pgfqpoint{1.275046in}{1.078720in}}%
\pgfpathlineto{\pgfqpoint{1.300684in}{1.086058in}}%
\pgfpathlineto{\pgfqpoint{1.326322in}{1.098056in}}%
\pgfpathlineto{\pgfqpoint{1.351959in}{1.105221in}}%
\pgfpathlineto{\pgfqpoint{1.377597in}{1.111034in}}%
\pgfpathlineto{\pgfqpoint{1.403235in}{1.120639in}}%
\pgfpathlineto{\pgfqpoint{1.428872in}{1.121871in}}%
\pgfpathlineto{\pgfqpoint{1.454510in}{1.126734in}}%
\pgfpathlineto{\pgfqpoint{1.480148in}{1.129152in}}%
\pgfpathlineto{\pgfqpoint{1.505785in}{1.127957in}}%
\pgfpathlineto{\pgfqpoint{1.531423in}{1.131342in}}%
\pgfpathlineto{\pgfqpoint{1.557060in}{1.127528in}}%
\pgfpathlineto{\pgfqpoint{1.582698in}{1.125486in}}%
\pgfpathlineto{\pgfqpoint{1.608336in}{1.122135in}}%
\pgfpathlineto{\pgfqpoint{1.633973in}{1.115048in}}%
\pgfpathlineto{\pgfqpoint{1.659611in}{1.111994in}}%
\pgfpathlineto{\pgfqpoint{1.685249in}{1.102270in}}%
\pgfpathlineto{\pgfqpoint{1.710886in}{1.093267in}}%
\pgfpathlineto{\pgfqpoint{1.736524in}{1.084803in}}%
\pgfpathlineto{\pgfqpoint{1.762162in}{1.071012in}}%
\pgfpathlineto{\pgfqpoint{1.787799in}{1.060684in}}%
\pgfpathlineto{\pgfqpoint{1.813437in}{1.045596in}}%
\pgfpathlineto{\pgfqpoint{1.839075in}{1.029245in}}%
\pgfpathlineto{\pgfqpoint{1.864712in}{1.014724in}}%
\pgfpathlineto{\pgfqpoint{1.890350in}{0.994182in}}%
\pgfpathlineto{\pgfqpoint{1.915988in}{0.977307in}}%
\pgfpathlineto{\pgfqpoint{1.941625in}{0.955521in}}%
\pgfpathlineto{\pgfqpoint{1.967263in}{0.932141in}}%
\pgfpathlineto{\pgfqpoint{1.992901in}{0.910443in}}%
\pgfpathlineto{\pgfqpoint{2.018538in}{0.883661in}}%
\pgfpathlineto{\pgfqpoint{2.044176in}{0.859306in}}%
\pgfpathlineto{\pgfqpoint{2.069814in}{0.830902in}}%
\pgfpathlineto{\pgfqpoint{2.095451in}{0.799518in}}%
\pgfpathlineto{\pgfqpoint{2.121089in}{0.770167in}}%
\pgfpathlineto{\pgfqpoint{2.146727in}{0.735003in}}%
\pgfpathlineto{\pgfqpoint{2.172364in}{0.701410in}}%
\pgfpathlineto{\pgfqpoint{2.198002in}{0.663628in}}%
\pgfpathlineto{\pgfqpoint{2.223640in}{0.619877in}}%
\pgfpathlineto{\pgfqpoint{2.249277in}{0.575938in}}%
\pgfpathlineto{\pgfqpoint{2.274915in}{0.529379in}}%
\pgfpathlineto{\pgfqpoint{2.291418in}{0.511603in}}%
\pgfpathmoveto{\pgfqpoint{2.381235in}{0.511603in}}%
\pgfpathlineto{\pgfqpoint{2.403103in}{0.539999in}}%
\pgfpathlineto{\pgfqpoint{2.428741in}{0.580083in}}%
\pgfpathlineto{\pgfqpoint{2.454378in}{0.612797in}}%
\pgfpathlineto{\pgfqpoint{2.480016in}{0.636780in}}%
\pgfpathlineto{\pgfqpoint{2.505654in}{0.659763in}}%
\pgfpathlineto{\pgfqpoint{2.531291in}{0.674071in}}%
\pgfpathlineto{\pgfqpoint{2.556929in}{0.687603in}}%
\pgfpathlineto{\pgfqpoint{2.582567in}{0.702130in}}%
\pgfpathlineto{\pgfqpoint{2.608204in}{0.708262in}}%
\pgfpathlineto{\pgfqpoint{2.633842in}{0.715206in}}%
\pgfpathlineto{\pgfqpoint{2.659480in}{0.716509in}}%
\pgfpathlineto{\pgfqpoint{2.685117in}{0.713886in}}%
\pgfpathlineto{\pgfqpoint{2.710755in}{0.712081in}}%
\pgfpathlineto{\pgfqpoint{2.736393in}{0.703539in}}%
\pgfpathlineto{\pgfqpoint{2.762030in}{0.695553in}}%
\pgfpathlineto{\pgfqpoint{2.787668in}{0.684121in}}%
\pgfpathlineto{\pgfqpoint{2.813306in}{0.665633in}}%
\pgfpathlineto{\pgfqpoint{2.838943in}{0.646212in}}%
\pgfpathlineto{\pgfqpoint{2.864581in}{0.620268in}}%
\pgfpathlineto{\pgfqpoint{2.890219in}{0.589130in}}%
\pgfpathlineto{\pgfqpoint{2.915856in}{0.554304in}}%
\pgfpathlineto{\pgfqpoint{2.941494in}{0.523101in}}%
\pgfpathlineto{\pgfqpoint{2.957360in}{0.511603in}}%
\pgfusepath{stroke}%
\end{pgfscope}%
\begin{pgfscope}%
\pgfpathrectangle{\pgfqpoint{0.634105in}{0.521603in}}{\pgfqpoint{3.720000in}{3.020000in}} %
\pgfusepath{clip}%
\pgfsetrectcap%
\pgfsetroundjoin%
\pgfsetlinewidth{1.505625pt}%
\definecolor{currentstroke}{rgb}{0.590196,0.989980,0.655284}%
\pgfsetstrokecolor{currentstroke}%
\pgfsetdash{}{0pt}%
\pgfpathmoveto{\pgfqpoint{0.975079in}{0.778661in}}%
\pgfpathlineto{\pgfqpoint{1.012966in}{0.806786in}}%
\pgfpathlineto{\pgfqpoint{1.088738in}{0.904234in}}%
\pgfpathlineto{\pgfqpoint{1.126624in}{0.941899in}}%
\pgfpathlineto{\pgfqpoint{1.164510in}{0.967791in}}%
\pgfpathlineto{\pgfqpoint{1.202396in}{1.006816in}}%
\pgfpathlineto{\pgfqpoint{1.240282in}{1.049332in}}%
\pgfpathlineto{\pgfqpoint{1.278168in}{1.074616in}}%
\pgfpathlineto{\pgfqpoint{1.316054in}{1.098069in}}%
\pgfpathlineto{\pgfqpoint{1.353940in}{1.133763in}}%
\pgfpathlineto{\pgfqpoint{1.391826in}{1.163836in}}%
\pgfpathlineto{\pgfqpoint{1.429712in}{1.179205in}}%
\pgfpathlineto{\pgfqpoint{1.467598in}{1.197659in}}%
\pgfpathlineto{\pgfqpoint{1.505484in}{1.226789in}}%
\pgfpathlineto{\pgfqpoint{1.543371in}{1.246489in}}%
\pgfpathlineto{\pgfqpoint{1.581257in}{1.255160in}}%
\pgfpathlineto{\pgfqpoint{1.619143in}{1.274145in}}%
\pgfpathlineto{\pgfqpoint{1.657029in}{1.295431in}}%
\pgfpathlineto{\pgfqpoint{1.694915in}{1.304516in}}%
\pgfpathlineto{\pgfqpoint{1.732801in}{1.307142in}}%
\pgfpathlineto{\pgfqpoint{1.808573in}{1.335319in}}%
\pgfpathlineto{\pgfqpoint{1.846459in}{1.334818in}}%
\pgfpathlineto{\pgfqpoint{1.884345in}{1.337119in}}%
\pgfpathlineto{\pgfqpoint{1.922231in}{1.347478in}}%
\pgfpathlineto{\pgfqpoint{1.960117in}{1.351577in}}%
\pgfpathlineto{\pgfqpoint{1.998003in}{1.343886in}}%
\pgfpathlineto{\pgfqpoint{2.035889in}{1.341598in}}%
\pgfpathlineto{\pgfqpoint{2.073776in}{1.345870in}}%
\pgfpathlineto{\pgfqpoint{2.111662in}{1.339442in}}%
\pgfpathlineto{\pgfqpoint{2.149548in}{1.327251in}}%
\pgfpathlineto{\pgfqpoint{2.225320in}{1.320223in}}%
\pgfpathlineto{\pgfqpoint{2.301092in}{1.285892in}}%
\pgfpathlineto{\pgfqpoint{2.338978in}{1.277020in}}%
\pgfpathlineto{\pgfqpoint{2.376864in}{1.264347in}}%
\pgfpathlineto{\pgfqpoint{2.414750in}{1.239858in}}%
\pgfpathlineto{\pgfqpoint{2.452636in}{1.220689in}}%
\pgfpathlineto{\pgfqpoint{2.490522in}{1.206676in}}%
\pgfpathlineto{\pgfqpoint{2.528408in}{1.183144in}}%
\pgfpathlineto{\pgfqpoint{2.566294in}{1.151384in}}%
\pgfpathlineto{\pgfqpoint{2.642067in}{1.104041in}}%
\pgfpathlineto{\pgfqpoint{2.717839in}{1.034127in}}%
\pgfpathlineto{\pgfqpoint{2.755725in}{1.005051in}}%
\pgfpathlineto{\pgfqpoint{2.793611in}{0.971667in}}%
\pgfpathlineto{\pgfqpoint{2.831497in}{0.926766in}}%
\pgfpathlineto{\pgfqpoint{2.869383in}{0.883923in}}%
\pgfpathlineto{\pgfqpoint{2.907269in}{0.845583in}}%
\pgfpathlineto{\pgfqpoint{2.945155in}{0.799699in}}%
\pgfpathlineto{\pgfqpoint{3.058813in}{0.638659in}}%
\pgfpathlineto{\pgfqpoint{3.096699in}{0.567561in}}%
\pgfpathlineto{\pgfqpoint{3.131686in}{0.511603in}}%
\pgfpathmoveto{\pgfqpoint{3.349318in}{0.511603in}}%
\pgfpathlineto{\pgfqpoint{3.361902in}{0.526709in}}%
\pgfpathlineto{\pgfqpoint{3.399788in}{0.577689in}}%
\pgfpathlineto{\pgfqpoint{3.437674in}{0.639365in}}%
\pgfpathlineto{\pgfqpoint{3.475560in}{0.697399in}}%
\pgfpathlineto{\pgfqpoint{3.513446in}{0.743852in}}%
\pgfpathlineto{\pgfqpoint{3.551332in}{0.781241in}}%
\pgfpathlineto{\pgfqpoint{3.627104in}{0.846867in}}%
\pgfpathlineto{\pgfqpoint{3.664990in}{0.877137in}}%
\pgfpathlineto{\pgfqpoint{3.740763in}{0.924841in}}%
\pgfpathlineto{\pgfqpoint{3.778649in}{0.944716in}}%
\pgfpathlineto{\pgfqpoint{3.816535in}{0.962654in}}%
\pgfpathlineto{\pgfqpoint{3.854421in}{0.976250in}}%
\pgfpathlineto{\pgfqpoint{3.930193in}{1.000212in}}%
\pgfpathlineto{\pgfqpoint{3.968079in}{1.006295in}}%
\pgfpathlineto{\pgfqpoint{4.005965in}{1.010996in}}%
\pgfpathlineto{\pgfqpoint{4.081737in}{1.014252in}}%
\pgfpathlineto{\pgfqpoint{4.119623in}{1.011378in}}%
\pgfpathlineto{\pgfqpoint{4.157509in}{1.004805in}}%
\pgfpathlineto{\pgfqpoint{4.195395in}{0.999771in}}%
\pgfpathlineto{\pgfqpoint{4.233282in}{0.990097in}}%
\pgfpathlineto{\pgfqpoint{4.271168in}{0.977067in}}%
\pgfpathlineto{\pgfqpoint{4.309054in}{0.962290in}}%
\pgfpathlineto{\pgfqpoint{4.346940in}{0.945186in}}%
\pgfpathlineto{\pgfqpoint{4.364105in}{0.936063in}}%
\pgfpathlineto{\pgfqpoint{4.364105in}{0.936063in}}%
\pgfusepath{stroke}%
\end{pgfscope}%
\begin{pgfscope}%
\pgfpathrectangle{\pgfqpoint{0.634105in}{0.521603in}}{\pgfqpoint{3.720000in}{3.020000in}} %
\pgfusepath{clip}%
\pgfsetrectcap%
\pgfsetroundjoin%
\pgfsetlinewidth{1.505625pt}%
\definecolor{currentstroke}{rgb}{0.692157,0.954791,0.592758}%
\pgfsetstrokecolor{currentstroke}%
\pgfsetdash{}{0pt}%
\pgfpathmoveto{\pgfqpoint{1.005025in}{0.959802in}}%
\pgfpathlineto{\pgfqpoint{1.051390in}{1.014789in}}%
\pgfpathlineto{\pgfqpoint{1.097755in}{1.057254in}}%
\pgfpathlineto{\pgfqpoint{1.144120in}{1.117153in}}%
\pgfpathlineto{\pgfqpoint{1.236850in}{1.209672in}}%
\pgfpathlineto{\pgfqpoint{1.283215in}{1.250496in}}%
\pgfpathlineto{\pgfqpoint{1.329580in}{1.298437in}}%
\pgfpathlineto{\pgfqpoint{1.375945in}{1.333786in}}%
\pgfpathlineto{\pgfqpoint{1.422310in}{1.376241in}}%
\pgfpathlineto{\pgfqpoint{1.468675in}{1.407268in}}%
\pgfpathlineto{\pgfqpoint{1.515040in}{1.444106in}}%
\pgfpathlineto{\pgfqpoint{1.561405in}{1.473029in}}%
\pgfpathlineto{\pgfqpoint{1.607770in}{1.504183in}}%
\pgfpathlineto{\pgfqpoint{1.654135in}{1.528567in}}%
\pgfpathlineto{\pgfqpoint{1.700500in}{1.557282in}}%
\pgfpathlineto{\pgfqpoint{1.746865in}{1.576819in}}%
\pgfpathlineto{\pgfqpoint{1.793230in}{1.601599in}}%
\pgfpathlineto{\pgfqpoint{1.839595in}{1.618028in}}%
\pgfpathlineto{\pgfqpoint{1.885960in}{1.638798in}}%
\pgfpathlineto{\pgfqpoint{1.932325in}{1.652580in}}%
\pgfpathlineto{\pgfqpoint{1.978690in}{1.669354in}}%
\pgfpathlineto{\pgfqpoint{2.025055in}{1.679216in}}%
\pgfpathlineto{\pgfqpoint{2.071420in}{1.691996in}}%
\pgfpathlineto{\pgfqpoint{2.117785in}{1.698521in}}%
\pgfpathlineto{\pgfqpoint{2.164150in}{1.707783in}}%
\pgfpathlineto{\pgfqpoint{2.210515in}{1.710738in}}%
\pgfpathlineto{\pgfqpoint{2.256880in}{1.716546in}}%
\pgfpathlineto{\pgfqpoint{2.303245in}{1.716226in}}%
\pgfpathlineto{\pgfqpoint{2.349610in}{1.717357in}}%
\pgfpathlineto{\pgfqpoint{2.395975in}{1.713471in}}%
\pgfpathlineto{\pgfqpoint{2.442340in}{1.711822in}}%
\pgfpathlineto{\pgfqpoint{2.488705in}{1.704411in}}%
\pgfpathlineto{\pgfqpoint{2.535070in}{1.698883in}}%
\pgfpathlineto{\pgfqpoint{2.581436in}{1.688101in}}%
\pgfpathlineto{\pgfqpoint{2.627801in}{1.679124in}}%
\pgfpathlineto{\pgfqpoint{2.674166in}{1.664506in}}%
\pgfpathlineto{\pgfqpoint{2.720531in}{1.651507in}}%
\pgfpathlineto{\pgfqpoint{2.766896in}{1.633553in}}%
\pgfpathlineto{\pgfqpoint{2.813261in}{1.617120in}}%
\pgfpathlineto{\pgfqpoint{2.859626in}{1.595177in}}%
\pgfpathlineto{\pgfqpoint{2.905991in}{1.574835in}}%
\pgfpathlineto{\pgfqpoint{2.998721in}{1.525582in}}%
\pgfpathlineto{\pgfqpoint{3.091451in}{1.468008in}}%
\pgfpathlineto{\pgfqpoint{3.137816in}{1.436240in}}%
\pgfpathlineto{\pgfqpoint{3.184181in}{1.402984in}}%
\pgfpathlineto{\pgfqpoint{3.276911in}{1.329761in}}%
\pgfpathlineto{\pgfqpoint{3.323276in}{1.289761in}}%
\pgfpathlineto{\pgfqpoint{3.369641in}{1.247800in}}%
\pgfpathlineto{\pgfqpoint{3.462371in}{1.156274in}}%
\pgfpathlineto{\pgfqpoint{3.508736in}{1.107239in}}%
\pgfpathlineto{\pgfqpoint{3.555101in}{1.055962in}}%
\pgfpathlineto{\pgfqpoint{3.601466in}{1.001396in}}%
\pgfpathlineto{\pgfqpoint{3.647831in}{0.944813in}}%
\pgfpathlineto{\pgfqpoint{3.694196in}{0.884659in}}%
\pgfpathlineto{\pgfqpoint{3.740561in}{0.819386in}}%
\pgfpathlineto{\pgfqpoint{3.786926in}{0.746162in}}%
\pgfpathlineto{\pgfqpoint{3.879656in}{0.593296in}}%
\pgfpathlineto{\pgfqpoint{3.926021in}{0.538569in}}%
\pgfpathlineto{\pgfqpoint{3.972386in}{0.521533in}}%
\pgfpathlineto{\pgfqpoint{4.018751in}{0.556254in}}%
\pgfpathlineto{\pgfqpoint{4.065116in}{0.621215in}}%
\pgfpathlineto{\pgfqpoint{4.111481in}{0.694189in}}%
\pgfpathlineto{\pgfqpoint{4.204211in}{0.817229in}}%
\pgfpathlineto{\pgfqpoint{4.296941in}{0.915495in}}%
\pgfpathlineto{\pgfqpoint{4.364105in}{0.975922in}}%
\pgfpathlineto{\pgfqpoint{4.364105in}{0.975922in}}%
\pgfusepath{stroke}%
\end{pgfscope}%
\begin{pgfscope}%
\pgfpathrectangle{\pgfqpoint{0.634105in}{0.521603in}}{\pgfqpoint{3.720000in}{3.020000in}} %
\pgfusepath{clip}%
\pgfsetrectcap%
\pgfsetroundjoin%
\pgfsetlinewidth{1.505625pt}%
\definecolor{currentstroke}{rgb}{0.268627,0.934680,0.823253}%
\pgfsetstrokecolor{currentstroke}%
\pgfsetdash{}{0pt}%
\pgfpathmoveto{\pgfqpoint{0.886853in}{0.511603in}}%
\pgfpathlineto{\pgfqpoint{0.927236in}{0.604079in}}%
\pgfpathlineto{\pgfqpoint{1.000519in}{0.712064in}}%
\pgfpathlineto{\pgfqpoint{1.073802in}{0.782778in}}%
\pgfpathlineto{\pgfqpoint{1.147085in}{0.868779in}}%
\pgfpathlineto{\pgfqpoint{1.220367in}{0.999497in}}%
\pgfpathlineto{\pgfqpoint{1.293650in}{1.110070in}}%
\pgfpathlineto{\pgfqpoint{1.366933in}{1.182386in}}%
\pgfpathlineto{\pgfqpoint{1.440216in}{1.253138in}}%
\pgfpathlineto{\pgfqpoint{1.513499in}{1.351997in}}%
\pgfpathlineto{\pgfqpoint{1.586781in}{1.418811in}}%
\pgfpathlineto{\pgfqpoint{1.660064in}{1.507132in}}%
\pgfpathlineto{\pgfqpoint{1.733347in}{1.569710in}}%
\pgfpathlineto{\pgfqpoint{1.806630in}{1.619925in}}%
\pgfpathlineto{\pgfqpoint{1.879913in}{1.689815in}}%
\pgfpathlineto{\pgfqpoint{1.953195in}{1.778534in}}%
\pgfpathlineto{\pgfqpoint{2.026478in}{1.850467in}}%
\pgfpathlineto{\pgfqpoint{2.099761in}{1.897632in}}%
\pgfpathlineto{\pgfqpoint{2.173044in}{1.948583in}}%
\pgfpathlineto{\pgfqpoint{2.246327in}{2.017705in}}%
\pgfpathlineto{\pgfqpoint{2.319610in}{2.092402in}}%
\pgfpathlineto{\pgfqpoint{2.392892in}{2.138831in}}%
\pgfpathlineto{\pgfqpoint{2.466175in}{2.174814in}}%
\pgfpathlineto{\pgfqpoint{2.539458in}{2.216274in}}%
\pgfpathlineto{\pgfqpoint{2.612741in}{2.277625in}}%
\pgfpathlineto{\pgfqpoint{2.686024in}{2.342775in}}%
\pgfpathlineto{\pgfqpoint{2.759306in}{2.379438in}}%
\pgfpathlineto{\pgfqpoint{2.832589in}{2.405449in}}%
\pgfpathlineto{\pgfqpoint{2.905872in}{2.450696in}}%
\pgfpathlineto{\pgfqpoint{2.979155in}{2.507117in}}%
\pgfpathlineto{\pgfqpoint{3.052438in}{2.548978in}}%
\pgfpathlineto{\pgfqpoint{3.125720in}{2.573876in}}%
\pgfpathlineto{\pgfqpoint{3.199003in}{2.594518in}}%
\pgfpathlineto{\pgfqpoint{3.272286in}{2.634118in}}%
\pgfpathlineto{\pgfqpoint{3.345569in}{2.681968in}}%
\pgfpathlineto{\pgfqpoint{3.418852in}{2.715123in}}%
\pgfpathlineto{\pgfqpoint{3.492135in}{2.725654in}}%
\pgfpathlineto{\pgfqpoint{3.565417in}{2.750951in}}%
\pgfpathlineto{\pgfqpoint{3.638700in}{2.790621in}}%
\pgfpathlineto{\pgfqpoint{3.711983in}{2.822393in}}%
\pgfpathlineto{\pgfqpoint{3.785266in}{2.838584in}}%
\pgfpathlineto{\pgfqpoint{3.858549in}{2.846121in}}%
\pgfpathlineto{\pgfqpoint{3.931831in}{2.864853in}}%
\pgfpathlineto{\pgfqpoint{4.005114in}{2.894217in}}%
\pgfpathlineto{\pgfqpoint{4.078397in}{2.919467in}}%
\pgfpathlineto{\pgfqpoint{4.151680in}{2.921178in}}%
\pgfpathlineto{\pgfqpoint{4.224963in}{2.927706in}}%
\pgfpathlineto{\pgfqpoint{4.364105in}{2.968102in}}%
\pgfpathlineto{\pgfqpoint{4.364105in}{2.968102in}}%
\pgfusepath{stroke}%
\end{pgfscope}%
\begin{pgfscope}%
\pgfpathrectangle{\pgfqpoint{0.634105in}{0.521603in}}{\pgfqpoint{3.720000in}{3.020000in}} %
\pgfusepath{clip}%
\pgfsetrectcap%
\pgfsetroundjoin%
\pgfsetlinewidth{1.505625pt}%
\definecolor{currentstroke}{rgb}{0.190196,0.883910,0.856638}%
\pgfsetstrokecolor{currentstroke}%
\pgfsetdash{}{0pt}%
\pgfpathmoveto{\pgfqpoint{1.427698in}{1.095716in}}%
\pgfpathlineto{\pgfqpoint{1.526897in}{1.236730in}}%
\pgfpathlineto{\pgfqpoint{1.626097in}{1.349912in}}%
\pgfpathlineto{\pgfqpoint{1.725296in}{1.450581in}}%
\pgfpathlineto{\pgfqpoint{1.824495in}{1.562787in}}%
\pgfpathlineto{\pgfqpoint{1.923694in}{1.672742in}}%
\pgfpathlineto{\pgfqpoint{2.022893in}{1.769751in}}%
\pgfpathlineto{\pgfqpoint{2.122092in}{1.882064in}}%
\pgfpathlineto{\pgfqpoint{2.320491in}{2.076316in}}%
\pgfpathlineto{\pgfqpoint{2.419690in}{2.178363in}}%
\pgfpathlineto{\pgfqpoint{2.518889in}{2.269153in}}%
\pgfpathlineto{\pgfqpoint{2.618088in}{2.367452in}}%
\pgfpathlineto{\pgfqpoint{2.717287in}{2.463918in}}%
\pgfpathlineto{\pgfqpoint{2.816487in}{2.546404in}}%
\pgfpathlineto{\pgfqpoint{2.915686in}{2.633721in}}%
\pgfpathlineto{\pgfqpoint{3.014885in}{2.725968in}}%
\pgfpathlineto{\pgfqpoint{3.114084in}{2.803482in}}%
\pgfpathlineto{\pgfqpoint{3.213283in}{2.888482in}}%
\pgfpathlineto{\pgfqpoint{3.312482in}{2.976580in}}%
\pgfpathlineto{\pgfqpoint{3.411682in}{3.049839in}}%
\pgfpathlineto{\pgfqpoint{3.510881in}{3.126467in}}%
\pgfpathlineto{\pgfqpoint{3.610080in}{3.207450in}}%
\pgfpathlineto{\pgfqpoint{3.709279in}{3.274279in}}%
\pgfpathlineto{\pgfqpoint{3.808478in}{3.350224in}}%
\pgfpathlineto{\pgfqpoint{3.907678in}{3.419275in}}%
\pgfpathlineto{\pgfqpoint{4.006877in}{3.483556in}}%
\pgfpathlineto{\pgfqpoint{4.104526in}{3.551603in}}%
\pgfpathlineto{\pgfqpoint{4.104526in}{3.551603in}}%
\pgfusepath{stroke}%
\end{pgfscope}%
\begin{pgfscope}%
\pgfpathrectangle{\pgfqpoint{0.634105in}{0.521603in}}{\pgfqpoint{3.720000in}{3.020000in}} %
\pgfusepath{clip}%
\pgfsetrectcap%
\pgfsetroundjoin%
\pgfsetlinewidth{1.505625pt}%
\definecolor{currentstroke}{rgb}{0.052941,0.645928,0.938988}%
\pgfsetstrokecolor{currentstroke}%
\pgfsetdash{}{0pt}%
\pgfpathmoveto{\pgfqpoint{1.742430in}{1.376272in}}%
\pgfpathlineto{\pgfqpoint{1.880970in}{1.467682in}}%
\pgfpathlineto{\pgfqpoint{2.019511in}{1.605567in}}%
\pgfpathlineto{\pgfqpoint{2.158051in}{1.863657in}}%
\pgfpathlineto{\pgfqpoint{2.296592in}{2.010400in}}%
\pgfpathlineto{\pgfqpoint{2.435133in}{2.069316in}}%
\pgfpathlineto{\pgfqpoint{2.573673in}{2.274797in}}%
\pgfpathlineto{\pgfqpoint{2.712214in}{2.470861in}}%
\pgfpathlineto{\pgfqpoint{2.850754in}{2.591980in}}%
\pgfpathlineto{\pgfqpoint{2.989295in}{2.695523in}}%
\pgfpathlineto{\pgfqpoint{3.127836in}{2.855530in}}%
\pgfpathlineto{\pgfqpoint{3.266376in}{3.020905in}}%
\pgfpathlineto{\pgfqpoint{3.404917in}{3.138215in}}%
\pgfpathlineto{\pgfqpoint{3.543457in}{3.249992in}}%
\pgfpathlineto{\pgfqpoint{3.681998in}{3.382659in}}%
\pgfpathlineto{\pgfqpoint{3.820539in}{3.537448in}}%
\pgfpathlineto{\pgfqpoint{3.837893in}{3.551603in}}%
\pgfpathlineto{\pgfqpoint{3.837893in}{3.551603in}}%
\pgfusepath{stroke}%
\end{pgfscope}%
\begin{pgfscope}%
\pgfpathrectangle{\pgfqpoint{0.634105in}{0.521603in}}{\pgfqpoint{3.720000in}{3.020000in}} %
\pgfusepath{clip}%
\pgfsetrectcap%
\pgfsetroundjoin%
\pgfsetlinewidth{1.505625pt}%
\definecolor{currentstroke}{rgb}{0.162745,0.505325,0.965124}%
\pgfsetstrokecolor{currentstroke}%
\pgfsetdash{}{0pt}%
\pgfpathmoveto{\pgfqpoint{2.272311in}{1.950771in}}%
\pgfpathlineto{\pgfqpoint{2.421239in}{2.006614in}}%
\pgfpathlineto{\pgfqpoint{2.570167in}{2.262977in}}%
\pgfpathlineto{\pgfqpoint{2.719095in}{2.477309in}}%
\pgfpathlineto{\pgfqpoint{2.868023in}{2.601283in}}%
\pgfpathlineto{\pgfqpoint{3.016950in}{2.758502in}}%
\pgfpathlineto{\pgfqpoint{3.165878in}{2.951134in}}%
\pgfpathlineto{\pgfqpoint{3.314806in}{3.172912in}}%
\pgfpathlineto{\pgfqpoint{3.463734in}{3.325380in}}%
\pgfpathlineto{\pgfqpoint{3.612662in}{3.453764in}}%
\pgfpathlineto{\pgfqpoint{3.696764in}{3.551603in}}%
\pgfpathlineto{\pgfqpoint{3.696764in}{3.551603in}}%
\pgfusepath{stroke}%
\end{pgfscope}%
\begin{pgfscope}%
\pgfpathrectangle{\pgfqpoint{0.634105in}{0.521603in}}{\pgfqpoint{3.720000in}{3.020000in}} %
\pgfusepath{clip}%
\pgfsetrectcap%
\pgfsetroundjoin%
\pgfsetlinewidth{1.505625pt}%
\definecolor{currentstroke}{rgb}{0.205882,0.895163,0.850217}%
\pgfsetstrokecolor{currentstroke}%
\pgfsetdash{}{0pt}%
\pgfpathmoveto{\pgfqpoint{1.483883in}{1.248124in}}%
\pgfpathlineto{\pgfqpoint{1.605280in}{1.371664in}}%
\pgfpathlineto{\pgfqpoint{1.726677in}{1.510548in}}%
\pgfpathlineto{\pgfqpoint{1.848074in}{1.640228in}}%
\pgfpathlineto{\pgfqpoint{1.969471in}{1.800315in}}%
\pgfpathlineto{\pgfqpoint{2.090868in}{1.921024in}}%
\pgfpathlineto{\pgfqpoint{2.333662in}{2.172306in}}%
\pgfpathlineto{\pgfqpoint{2.455059in}{2.305984in}}%
\pgfpathlineto{\pgfqpoint{2.576456in}{2.433721in}}%
\pgfpathlineto{\pgfqpoint{2.697853in}{2.557386in}}%
\pgfpathlineto{\pgfqpoint{2.819250in}{2.671996in}}%
\pgfpathlineto{\pgfqpoint{2.940647in}{2.798322in}}%
\pgfpathlineto{\pgfqpoint{3.062044in}{2.910781in}}%
\pgfpathlineto{\pgfqpoint{3.183440in}{3.020746in}}%
\pgfpathlineto{\pgfqpoint{3.304837in}{3.139999in}}%
\pgfpathlineto{\pgfqpoint{3.426234in}{3.242866in}}%
\pgfpathlineto{\pgfqpoint{3.547631in}{3.354918in}}%
\pgfpathlineto{\pgfqpoint{3.669028in}{3.452036in}}%
\pgfpathlineto{\pgfqpoint{3.778976in}{3.551603in}}%
\pgfpathlineto{\pgfqpoint{3.778976in}{3.551603in}}%
\pgfusepath{stroke}%
\end{pgfscope}%
\begin{pgfscope}%
\pgfpathrectangle{\pgfqpoint{0.634105in}{0.521603in}}{\pgfqpoint{3.720000in}{3.020000in}} %
\pgfusepath{clip}%
\pgfsetrectcap%
\pgfsetroundjoin%
\pgfsetlinewidth{1.505625pt}%
\definecolor{currentstroke}{rgb}{0.076471,0.617278,0.945184}%
\pgfsetstrokecolor{currentstroke}%
\pgfsetdash{}{0pt}%
\pgfpathmoveto{\pgfqpoint{1.996700in}{1.640427in}}%
\pgfpathlineto{\pgfqpoint{2.386013in}{2.084118in}}%
\pgfpathlineto{\pgfqpoint{2.580669in}{2.331415in}}%
\pgfpathlineto{\pgfqpoint{2.775326in}{2.553187in}}%
\pgfpathlineto{\pgfqpoint{2.969982in}{2.763132in}}%
\pgfpathlineto{\pgfqpoint{3.164639in}{3.002485in}}%
\pgfpathlineto{\pgfqpoint{3.359295in}{3.187254in}}%
\pgfpathlineto{\pgfqpoint{3.553952in}{3.411432in}}%
\pgfpathlineto{\pgfqpoint{3.687544in}{3.551603in}}%
\pgfpathlineto{\pgfqpoint{3.687544in}{3.551603in}}%
\pgfusepath{stroke}%
\end{pgfscope}%
\begin{pgfscope}%
\pgfpathrectangle{\pgfqpoint{0.634105in}{0.521603in}}{\pgfqpoint{3.720000in}{3.020000in}} %
\pgfusepath{clip}%
\pgfsetrectcap%
\pgfsetroundjoin%
\pgfsetlinewidth{1.505625pt}%
\definecolor{currentstroke}{rgb}{0.500000,0.000000,1.000000}%
\pgfsetstrokecolor{currentstroke}%
\pgfsetdash{}{0pt}%
\pgfpathmoveto{\pgfqpoint{2.761471in}{1.816645in}}%
\pgfpathlineto{\pgfqpoint{3.470593in}{2.746592in}}%
\pgfpathlineto{\pgfqpoint{3.825154in}{3.120487in}}%
\pgfpathlineto{\pgfqpoint{4.179716in}{3.478450in}}%
\pgfpathlineto{\pgfqpoint{4.234934in}{3.551603in}}%
\pgfpathlineto{\pgfqpoint{4.234934in}{3.551603in}}%
\pgfusepath{stroke}%
\end{pgfscope}%
\begin{pgfscope}%
\pgfsetrectcap%
\pgfsetmiterjoin%
\pgfsetlinewidth{0.803000pt}%
\definecolor{currentstroke}{rgb}{0.000000,0.000000,0.000000}%
\pgfsetstrokecolor{currentstroke}%
\pgfsetdash{}{0pt}%
\pgfpathmoveto{\pgfqpoint{0.634105in}{0.521603in}}%
\pgfpathlineto{\pgfqpoint{0.634105in}{3.541603in}}%
\pgfusepath{stroke}%
\end{pgfscope}%
\begin{pgfscope}%
\pgfsetrectcap%
\pgfsetmiterjoin%
\pgfsetlinewidth{0.803000pt}%
\definecolor{currentstroke}{rgb}{0.000000,0.000000,0.000000}%
\pgfsetstrokecolor{currentstroke}%
\pgfsetdash{}{0pt}%
\pgfpathmoveto{\pgfqpoint{4.354105in}{0.521603in}}%
\pgfpathlineto{\pgfqpoint{4.354105in}{3.541603in}}%
\pgfusepath{stroke}%
\end{pgfscope}%
\begin{pgfscope}%
\pgfsetrectcap%
\pgfsetmiterjoin%
\pgfsetlinewidth{0.803000pt}%
\definecolor{currentstroke}{rgb}{0.000000,0.000000,0.000000}%
\pgfsetstrokecolor{currentstroke}%
\pgfsetdash{}{0pt}%
\pgfpathmoveto{\pgfqpoint{0.634105in}{0.521603in}}%
\pgfpathlineto{\pgfqpoint{4.354105in}{0.521603in}}%
\pgfusepath{stroke}%
\end{pgfscope}%
\begin{pgfscope}%
\pgfsetrectcap%
\pgfsetmiterjoin%
\pgfsetlinewidth{0.803000pt}%
\definecolor{currentstroke}{rgb}{0.000000,0.000000,0.000000}%
\pgfsetstrokecolor{currentstroke}%
\pgfsetdash{}{0pt}%
\pgfpathmoveto{\pgfqpoint{0.634105in}{3.541603in}}%
\pgfpathlineto{\pgfqpoint{4.354105in}{3.541603in}}%
\pgfusepath{stroke}%
\end{pgfscope}%
\begin{pgfscope}%
\pgfpathrectangle{\pgfqpoint{4.586605in}{0.521603in}}{\pgfqpoint{0.151000in}{3.020000in}} %
\pgfusepath{clip}%
\pgfsetbuttcap%
\pgfsetmiterjoin%
\definecolor{currentfill}{rgb}{1.000000,1.000000,1.000000}%
\pgfsetfillcolor{currentfill}%
\pgfsetlinewidth{0.010037pt}%
\definecolor{currentstroke}{rgb}{1.000000,1.000000,1.000000}%
\pgfsetstrokecolor{currentstroke}%
\pgfsetdash{}{0pt}%
\pgfpathmoveto{\pgfqpoint{4.586605in}{0.521603in}}%
\pgfpathlineto{\pgfqpoint{4.586605in}{0.533400in}}%
\pgfpathlineto{\pgfqpoint{4.586605in}{3.529806in}}%
\pgfpathlineto{\pgfqpoint{4.586605in}{3.541603in}}%
\pgfpathlineto{\pgfqpoint{4.737605in}{3.541603in}}%
\pgfpathlineto{\pgfqpoint{4.737605in}{3.529806in}}%
\pgfpathlineto{\pgfqpoint{4.737605in}{0.533400in}}%
\pgfpathlineto{\pgfqpoint{4.737605in}{0.521603in}}%
\pgfpathclose%
\pgfusepath{stroke,fill}%
\end{pgfscope}%
\begin{pgfscope}%
\pgfsys@transformshift{4.590000in}{0.524365in}%
\pgftext[left,bottom]{\pgfimage[interpolate=true,width=0.150000in,height=3.020000in]{series_m_ds-img0.png}}%
\end{pgfscope}%
\begin{pgfscope}%
\pgfsetbuttcap%
\pgfsetroundjoin%
\definecolor{currentfill}{rgb}{0.000000,0.000000,0.000000}%
\pgfsetfillcolor{currentfill}%
\pgfsetlinewidth{0.803000pt}%
\definecolor{currentstroke}{rgb}{0.000000,0.000000,0.000000}%
\pgfsetstrokecolor{currentstroke}%
\pgfsetdash{}{0pt}%
\pgfsys@defobject{currentmarker}{\pgfqpoint{0.000000in}{0.000000in}}{\pgfqpoint{0.048611in}{0.000000in}}{%
\pgfpathmoveto{\pgfqpoint{0.000000in}{0.000000in}}%
\pgfpathlineto{\pgfqpoint{0.048611in}{0.000000in}}%
\pgfusepath{stroke,fill}%
}%
\begin{pgfscope}%
\pgfsys@transformshift{4.737605in}{0.617964in}%
\pgfsys@useobject{currentmarker}{}%
\end{pgfscope}%
\end{pgfscope}%
\begin{pgfscope}%
\pgfsetbuttcap%
\pgfsetroundjoin%
\definecolor{currentfill}{rgb}{0.000000,0.000000,0.000000}%
\pgfsetfillcolor{currentfill}%
\pgfsetlinewidth{0.803000pt}%
\definecolor{currentstroke}{rgb}{0.000000,0.000000,0.000000}%
\pgfsetstrokecolor{currentstroke}%
\pgfsetdash{}{0pt}%
\pgfsys@defobject{currentmarker}{\pgfqpoint{0.000000in}{0.000000in}}{\pgfqpoint{0.048611in}{0.000000in}}{%
\pgfpathmoveto{\pgfqpoint{0.000000in}{0.000000in}}%
\pgfpathlineto{\pgfqpoint{0.048611in}{0.000000in}}%
\pgfusepath{stroke,fill}%
}%
\begin{pgfscope}%
\pgfsys@transformshift{4.737605in}{0.796036in}%
\pgfsys@useobject{currentmarker}{}%
\end{pgfscope}%
\end{pgfscope}%
\begin{pgfscope}%
\pgfsetbuttcap%
\pgfsetroundjoin%
\definecolor{currentfill}{rgb}{0.000000,0.000000,0.000000}%
\pgfsetfillcolor{currentfill}%
\pgfsetlinewidth{0.803000pt}%
\definecolor{currentstroke}{rgb}{0.000000,0.000000,0.000000}%
\pgfsetstrokecolor{currentstroke}%
\pgfsetdash{}{0pt}%
\pgfsys@defobject{currentmarker}{\pgfqpoint{0.000000in}{0.000000in}}{\pgfqpoint{0.048611in}{0.000000in}}{%
\pgfpathmoveto{\pgfqpoint{0.000000in}{0.000000in}}%
\pgfpathlineto{\pgfqpoint{0.048611in}{0.000000in}}%
\pgfusepath{stroke,fill}%
}%
\begin{pgfscope}%
\pgfsys@transformshift{4.737605in}{0.941531in}%
\pgfsys@useobject{currentmarker}{}%
\end{pgfscope}%
\end{pgfscope}%
\begin{pgfscope}%
\pgfsetbuttcap%
\pgfsetroundjoin%
\definecolor{currentfill}{rgb}{0.000000,0.000000,0.000000}%
\pgfsetfillcolor{currentfill}%
\pgfsetlinewidth{0.803000pt}%
\definecolor{currentstroke}{rgb}{0.000000,0.000000,0.000000}%
\pgfsetstrokecolor{currentstroke}%
\pgfsetdash{}{0pt}%
\pgfsys@defobject{currentmarker}{\pgfqpoint{0.000000in}{0.000000in}}{\pgfqpoint{0.048611in}{0.000000in}}{%
\pgfpathmoveto{\pgfqpoint{0.000000in}{0.000000in}}%
\pgfpathlineto{\pgfqpoint{0.048611in}{0.000000in}}%
\pgfusepath{stroke,fill}%
}%
\begin{pgfscope}%
\pgfsys@transformshift{4.737605in}{1.064545in}%
\pgfsys@useobject{currentmarker}{}%
\end{pgfscope}%
\end{pgfscope}%
\begin{pgfscope}%
\pgfsetbuttcap%
\pgfsetroundjoin%
\definecolor{currentfill}{rgb}{0.000000,0.000000,0.000000}%
\pgfsetfillcolor{currentfill}%
\pgfsetlinewidth{0.803000pt}%
\definecolor{currentstroke}{rgb}{0.000000,0.000000,0.000000}%
\pgfsetstrokecolor{currentstroke}%
\pgfsetdash{}{0pt}%
\pgfsys@defobject{currentmarker}{\pgfqpoint{0.000000in}{0.000000in}}{\pgfqpoint{0.048611in}{0.000000in}}{%
\pgfpathmoveto{\pgfqpoint{0.000000in}{0.000000in}}%
\pgfpathlineto{\pgfqpoint{0.048611in}{0.000000in}}%
\pgfusepath{stroke,fill}%
}%
\begin{pgfscope}%
\pgfsys@transformshift{4.737605in}{1.171104in}%
\pgfsys@useobject{currentmarker}{}%
\end{pgfscope}%
\end{pgfscope}%
\begin{pgfscope}%
\pgfsetbuttcap%
\pgfsetroundjoin%
\definecolor{currentfill}{rgb}{0.000000,0.000000,0.000000}%
\pgfsetfillcolor{currentfill}%
\pgfsetlinewidth{0.803000pt}%
\definecolor{currentstroke}{rgb}{0.000000,0.000000,0.000000}%
\pgfsetstrokecolor{currentstroke}%
\pgfsetdash{}{0pt}%
\pgfsys@defobject{currentmarker}{\pgfqpoint{0.000000in}{0.000000in}}{\pgfqpoint{0.048611in}{0.000000in}}{%
\pgfpathmoveto{\pgfqpoint{0.000000in}{0.000000in}}%
\pgfpathlineto{\pgfqpoint{0.048611in}{0.000000in}}%
\pgfusepath{stroke,fill}%
}%
\begin{pgfscope}%
\pgfsys@transformshift{4.737605in}{1.265097in}%
\pgfsys@useobject{currentmarker}{}%
\end{pgfscope}%
\end{pgfscope}%
\begin{pgfscope}%
\pgfsetbuttcap%
\pgfsetroundjoin%
\definecolor{currentfill}{rgb}{0.000000,0.000000,0.000000}%
\pgfsetfillcolor{currentfill}%
\pgfsetlinewidth{0.803000pt}%
\definecolor{currentstroke}{rgb}{0.000000,0.000000,0.000000}%
\pgfsetstrokecolor{currentstroke}%
\pgfsetdash{}{0pt}%
\pgfsys@defobject{currentmarker}{\pgfqpoint{0.000000in}{0.000000in}}{\pgfqpoint{0.048611in}{0.000000in}}{%
\pgfpathmoveto{\pgfqpoint{0.000000in}{0.000000in}}%
\pgfpathlineto{\pgfqpoint{0.048611in}{0.000000in}}%
\pgfusepath{stroke,fill}%
}%
\begin{pgfscope}%
\pgfsys@transformshift{4.737605in}{1.349176in}%
\pgfsys@useobject{currentmarker}{}%
\end{pgfscope}%
\end{pgfscope}%
\begin{pgfscope}%
\pgftext[x=4.834827in,y=1.296414in,left,base]{\rmfamily\fontsize{10.000000}{12.000000}\selectfont \(\displaystyle 10^{-1}\)}%
\end{pgfscope}%
\begin{pgfscope}%
\pgfsetbuttcap%
\pgfsetroundjoin%
\definecolor{currentfill}{rgb}{0.000000,0.000000,0.000000}%
\pgfsetfillcolor{currentfill}%
\pgfsetlinewidth{0.803000pt}%
\definecolor{currentstroke}{rgb}{0.000000,0.000000,0.000000}%
\pgfsetstrokecolor{currentstroke}%
\pgfsetdash{}{0pt}%
\pgfsys@defobject{currentmarker}{\pgfqpoint{0.000000in}{0.000000in}}{\pgfqpoint{0.048611in}{0.000000in}}{%
\pgfpathmoveto{\pgfqpoint{0.000000in}{0.000000in}}%
\pgfpathlineto{\pgfqpoint{0.048611in}{0.000000in}}%
\pgfusepath{stroke,fill}%
}%
\begin{pgfscope}%
\pgfsys@transformshift{4.737605in}{1.902316in}%
\pgfsys@useobject{currentmarker}{}%
\end{pgfscope}%
\end{pgfscope}%
\begin{pgfscope}%
\pgfsetbuttcap%
\pgfsetroundjoin%
\definecolor{currentfill}{rgb}{0.000000,0.000000,0.000000}%
\pgfsetfillcolor{currentfill}%
\pgfsetlinewidth{0.803000pt}%
\definecolor{currentstroke}{rgb}{0.000000,0.000000,0.000000}%
\pgfsetstrokecolor{currentstroke}%
\pgfsetdash{}{0pt}%
\pgfsys@defobject{currentmarker}{\pgfqpoint{0.000000in}{0.000000in}}{\pgfqpoint{0.048611in}{0.000000in}}{%
\pgfpathmoveto{\pgfqpoint{0.000000in}{0.000000in}}%
\pgfpathlineto{\pgfqpoint{0.048611in}{0.000000in}}%
\pgfusepath{stroke,fill}%
}%
\begin{pgfscope}%
\pgfsys@transformshift{4.737605in}{2.225882in}%
\pgfsys@useobject{currentmarker}{}%
\end{pgfscope}%
\end{pgfscope}%
\begin{pgfscope}%
\pgfsetbuttcap%
\pgfsetroundjoin%
\definecolor{currentfill}{rgb}{0.000000,0.000000,0.000000}%
\pgfsetfillcolor{currentfill}%
\pgfsetlinewidth{0.803000pt}%
\definecolor{currentstroke}{rgb}{0.000000,0.000000,0.000000}%
\pgfsetstrokecolor{currentstroke}%
\pgfsetdash{}{0pt}%
\pgfsys@defobject{currentmarker}{\pgfqpoint{0.000000in}{0.000000in}}{\pgfqpoint{0.048611in}{0.000000in}}{%
\pgfpathmoveto{\pgfqpoint{0.000000in}{0.000000in}}%
\pgfpathlineto{\pgfqpoint{0.048611in}{0.000000in}}%
\pgfusepath{stroke,fill}%
}%
\begin{pgfscope}%
\pgfsys@transformshift{4.737605in}{2.455456in}%
\pgfsys@useobject{currentmarker}{}%
\end{pgfscope}%
\end{pgfscope}%
\begin{pgfscope}%
\pgfsetbuttcap%
\pgfsetroundjoin%
\definecolor{currentfill}{rgb}{0.000000,0.000000,0.000000}%
\pgfsetfillcolor{currentfill}%
\pgfsetlinewidth{0.803000pt}%
\definecolor{currentstroke}{rgb}{0.000000,0.000000,0.000000}%
\pgfsetstrokecolor{currentstroke}%
\pgfsetdash{}{0pt}%
\pgfsys@defobject{currentmarker}{\pgfqpoint{0.000000in}{0.000000in}}{\pgfqpoint{0.048611in}{0.000000in}}{%
\pgfpathmoveto{\pgfqpoint{0.000000in}{0.000000in}}%
\pgfpathlineto{\pgfqpoint{0.048611in}{0.000000in}}%
\pgfusepath{stroke,fill}%
}%
\begin{pgfscope}%
\pgfsys@transformshift{4.737605in}{2.633527in}%
\pgfsys@useobject{currentmarker}{}%
\end{pgfscope}%
\end{pgfscope}%
\begin{pgfscope}%
\pgfsetbuttcap%
\pgfsetroundjoin%
\definecolor{currentfill}{rgb}{0.000000,0.000000,0.000000}%
\pgfsetfillcolor{currentfill}%
\pgfsetlinewidth{0.803000pt}%
\definecolor{currentstroke}{rgb}{0.000000,0.000000,0.000000}%
\pgfsetstrokecolor{currentstroke}%
\pgfsetdash{}{0pt}%
\pgfsys@defobject{currentmarker}{\pgfqpoint{0.000000in}{0.000000in}}{\pgfqpoint{0.048611in}{0.000000in}}{%
\pgfpathmoveto{\pgfqpoint{0.000000in}{0.000000in}}%
\pgfpathlineto{\pgfqpoint{0.048611in}{0.000000in}}%
\pgfusepath{stroke,fill}%
}%
\begin{pgfscope}%
\pgfsys@transformshift{4.737605in}{2.779022in}%
\pgfsys@useobject{currentmarker}{}%
\end{pgfscope}%
\end{pgfscope}%
\begin{pgfscope}%
\pgfsetbuttcap%
\pgfsetroundjoin%
\definecolor{currentfill}{rgb}{0.000000,0.000000,0.000000}%
\pgfsetfillcolor{currentfill}%
\pgfsetlinewidth{0.803000pt}%
\definecolor{currentstroke}{rgb}{0.000000,0.000000,0.000000}%
\pgfsetstrokecolor{currentstroke}%
\pgfsetdash{}{0pt}%
\pgfsys@defobject{currentmarker}{\pgfqpoint{0.000000in}{0.000000in}}{\pgfqpoint{0.048611in}{0.000000in}}{%
\pgfpathmoveto{\pgfqpoint{0.000000in}{0.000000in}}%
\pgfpathlineto{\pgfqpoint{0.048611in}{0.000000in}}%
\pgfusepath{stroke,fill}%
}%
\begin{pgfscope}%
\pgfsys@transformshift{4.737605in}{2.902036in}%
\pgfsys@useobject{currentmarker}{}%
\end{pgfscope}%
\end{pgfscope}%
\begin{pgfscope}%
\pgfsetbuttcap%
\pgfsetroundjoin%
\definecolor{currentfill}{rgb}{0.000000,0.000000,0.000000}%
\pgfsetfillcolor{currentfill}%
\pgfsetlinewidth{0.803000pt}%
\definecolor{currentstroke}{rgb}{0.000000,0.000000,0.000000}%
\pgfsetstrokecolor{currentstroke}%
\pgfsetdash{}{0pt}%
\pgfsys@defobject{currentmarker}{\pgfqpoint{0.000000in}{0.000000in}}{\pgfqpoint{0.048611in}{0.000000in}}{%
\pgfpathmoveto{\pgfqpoint{0.000000in}{0.000000in}}%
\pgfpathlineto{\pgfqpoint{0.048611in}{0.000000in}}%
\pgfusepath{stroke,fill}%
}%
\begin{pgfscope}%
\pgfsys@transformshift{4.737605in}{3.008596in}%
\pgfsys@useobject{currentmarker}{}%
\end{pgfscope}%
\end{pgfscope}%
\begin{pgfscope}%
\pgfsetbuttcap%
\pgfsetroundjoin%
\definecolor{currentfill}{rgb}{0.000000,0.000000,0.000000}%
\pgfsetfillcolor{currentfill}%
\pgfsetlinewidth{0.803000pt}%
\definecolor{currentstroke}{rgb}{0.000000,0.000000,0.000000}%
\pgfsetstrokecolor{currentstroke}%
\pgfsetdash{}{0pt}%
\pgfsys@defobject{currentmarker}{\pgfqpoint{0.000000in}{0.000000in}}{\pgfqpoint{0.048611in}{0.000000in}}{%
\pgfpathmoveto{\pgfqpoint{0.000000in}{0.000000in}}%
\pgfpathlineto{\pgfqpoint{0.048611in}{0.000000in}}%
\pgfusepath{stroke,fill}%
}%
\begin{pgfscope}%
\pgfsys@transformshift{4.737605in}{3.102588in}%
\pgfsys@useobject{currentmarker}{}%
\end{pgfscope}%
\end{pgfscope}%
\begin{pgfscope}%
\pgfsetbuttcap%
\pgfsetroundjoin%
\definecolor{currentfill}{rgb}{0.000000,0.000000,0.000000}%
\pgfsetfillcolor{currentfill}%
\pgfsetlinewidth{0.803000pt}%
\definecolor{currentstroke}{rgb}{0.000000,0.000000,0.000000}%
\pgfsetstrokecolor{currentstroke}%
\pgfsetdash{}{0pt}%
\pgfsys@defobject{currentmarker}{\pgfqpoint{0.000000in}{0.000000in}}{\pgfqpoint{0.048611in}{0.000000in}}{%
\pgfpathmoveto{\pgfqpoint{0.000000in}{0.000000in}}%
\pgfpathlineto{\pgfqpoint{0.048611in}{0.000000in}}%
\pgfusepath{stroke,fill}%
}%
\begin{pgfscope}%
\pgfsys@transformshift{4.737605in}{3.186667in}%
\pgfsys@useobject{currentmarker}{}%
\end{pgfscope}%
\end{pgfscope}%
\begin{pgfscope}%
\pgftext[x=4.834827in,y=3.133906in,left,base]{\rmfamily\fontsize{10.000000}{12.000000}\selectfont \(\displaystyle 10^{0}\)}%
\end{pgfscope}%
\begin{pgfscope}%
\pgftext[x=5.317274in,y=2.031603in,,top]{\rmfamily\fontsize{12.000000}{14.400000}\selectfont \(\displaystyle {\mathbf{E} \mbox{u}}\)}%
\end{pgfscope}%
\begin{pgfscope}%
\pgfsetbuttcap%
\pgfsetmiterjoin%
\pgfsetlinewidth{0.803000pt}%
\definecolor{currentstroke}{rgb}{0.000000,0.000000,0.000000}%
\pgfsetstrokecolor{currentstroke}%
\pgfsetdash{}{0pt}%
\pgfpathmoveto{\pgfqpoint{4.586605in}{0.521603in}}%
\pgfpathlineto{\pgfqpoint{4.586605in}{0.533400in}}%
\pgfpathlineto{\pgfqpoint{4.586605in}{3.529806in}}%
\pgfpathlineto{\pgfqpoint{4.586605in}{3.541603in}}%
\pgfpathlineto{\pgfqpoint{4.737605in}{3.541603in}}%
\pgfpathlineto{\pgfqpoint{4.737605in}{3.529806in}}%
\pgfpathlineto{\pgfqpoint{4.737605in}{0.533400in}}%
\pgfpathlineto{\pgfqpoint{4.737605in}{0.521603in}}%
\pgfpathclose%
\pgfusepath{stroke}%
\end{pgfscope}%
\end{pgfpicture}%
\makeatother%
\endgroup%

    \caption{.\label{fig:dnumbs}}
\end{figure}
\begin{figure}[htb]
    \centering
    %% Creator: Matplotlib, PGF backend
%%
%% To include the figure in your LaTeX document, write
%%   \input{<filename>.pgf}
%%
%% Make sure the required packages are loaded in your preamble
%%   \usepackage{pgf}
%%
%% Figures using additional raster images can only be included by \input if
%% they are in the same directory as the main LaTeX file. For loading figures
%% from other directories you can use the `import` package
%%   \usepackage{import}
%% and then include the figures with
%%   \import{<path to file>}{<filename>.pgf}
%%
%% Matplotlib used the following preamble
%%   \usepackage{fontspec}
%%   \setmainfont{DejaVu Serif}
%%   \setsansfont{DejaVu Sans}
%%   \setmonofont{DejaVu Sans Mono}
%%
\begingroup%
\makeatletter%
\begin{pgfpicture}%
\pgfpathrectangle{\pgfpointorigin}{\pgfqpoint{5.426437in}{3.676603in}}%
\pgfusepath{use as bounding box, clip}%
\begin{pgfscope}%
\pgfsetbuttcap%
\pgfsetmiterjoin%
\definecolor{currentfill}{rgb}{1.000000,1.000000,1.000000}%
\pgfsetfillcolor{currentfill}%
\pgfsetlinewidth{0.000000pt}%
\definecolor{currentstroke}{rgb}{1.000000,1.000000,1.000000}%
\pgfsetstrokecolor{currentstroke}%
\pgfsetdash{}{0pt}%
\pgfpathmoveto{\pgfqpoint{0.000000in}{0.000000in}}%
\pgfpathlineto{\pgfqpoint{5.426437in}{0.000000in}}%
\pgfpathlineto{\pgfqpoint{5.426437in}{3.676603in}}%
\pgfpathlineto{\pgfqpoint{0.000000in}{3.676603in}}%
\pgfpathclose%
\pgfusepath{fill}%
\end{pgfscope}%
\begin{pgfscope}%
\pgfsetbuttcap%
\pgfsetmiterjoin%
\definecolor{currentfill}{rgb}{1.000000,1.000000,1.000000}%
\pgfsetfillcolor{currentfill}%
\pgfsetlinewidth{0.000000pt}%
\definecolor{currentstroke}{rgb}{0.000000,0.000000,0.000000}%
\pgfsetstrokecolor{currentstroke}%
\pgfsetstrokeopacity{0.000000}%
\pgfsetdash{}{0pt}%
\pgfpathmoveto{\pgfqpoint{0.526080in}{0.521603in}}%
\pgfpathlineto{\pgfqpoint{4.246080in}{0.521603in}}%
\pgfpathlineto{\pgfqpoint{4.246080in}{3.541603in}}%
\pgfpathlineto{\pgfqpoint{0.526080in}{3.541603in}}%
\pgfpathclose%
\pgfusepath{fill}%
\end{pgfscope}%
\begin{pgfscope}%
\pgfsetbuttcap%
\pgfsetroundjoin%
\definecolor{currentfill}{rgb}{0.000000,0.000000,0.000000}%
\pgfsetfillcolor{currentfill}%
\pgfsetlinewidth{0.803000pt}%
\definecolor{currentstroke}{rgb}{0.000000,0.000000,0.000000}%
\pgfsetstrokecolor{currentstroke}%
\pgfsetdash{}{0pt}%
\pgfsys@defobject{currentmarker}{\pgfqpoint{0.000000in}{-0.048611in}}{\pgfqpoint{0.000000in}{0.000000in}}{%
\pgfpathmoveto{\pgfqpoint{0.000000in}{0.000000in}}%
\pgfpathlineto{\pgfqpoint{0.000000in}{-0.048611in}}%
\pgfusepath{stroke,fill}%
}%
\begin{pgfscope}%
\pgfsys@transformshift{0.692556in}{0.521603in}%
\pgfsys@useobject{currentmarker}{}%
\end{pgfscope}%
\end{pgfscope}%
\begin{pgfscope}%
\pgftext[x=0.692556in,y=0.424381in,,top]{\rmfamily\fontsize{10.000000}{12.000000}\selectfont \(\displaystyle 0\)}%
\end{pgfscope}%
\begin{pgfscope}%
\pgfsetbuttcap%
\pgfsetroundjoin%
\definecolor{currentfill}{rgb}{0.000000,0.000000,0.000000}%
\pgfsetfillcolor{currentfill}%
\pgfsetlinewidth{0.803000pt}%
\definecolor{currentstroke}{rgb}{0.000000,0.000000,0.000000}%
\pgfsetstrokecolor{currentstroke}%
\pgfsetdash{}{0pt}%
\pgfsys@defobject{currentmarker}{\pgfqpoint{0.000000in}{-0.048611in}}{\pgfqpoint{0.000000in}{0.000000in}}{%
\pgfpathmoveto{\pgfqpoint{0.000000in}{0.000000in}}%
\pgfpathlineto{\pgfqpoint{0.000000in}{-0.048611in}}%
\pgfusepath{stroke,fill}%
}%
\begin{pgfscope}%
\pgfsys@transformshift{1.508853in}{0.521603in}%
\pgfsys@useobject{currentmarker}{}%
\end{pgfscope}%
\end{pgfscope}%
\begin{pgfscope}%
\pgftext[x=1.508853in,y=0.424381in,,top]{\rmfamily\fontsize{10.000000}{12.000000}\selectfont \(\displaystyle 10\)}%
\end{pgfscope}%
\begin{pgfscope}%
\pgfsetbuttcap%
\pgfsetroundjoin%
\definecolor{currentfill}{rgb}{0.000000,0.000000,0.000000}%
\pgfsetfillcolor{currentfill}%
\pgfsetlinewidth{0.803000pt}%
\definecolor{currentstroke}{rgb}{0.000000,0.000000,0.000000}%
\pgfsetstrokecolor{currentstroke}%
\pgfsetdash{}{0pt}%
\pgfsys@defobject{currentmarker}{\pgfqpoint{0.000000in}{-0.048611in}}{\pgfqpoint{0.000000in}{0.000000in}}{%
\pgfpathmoveto{\pgfqpoint{0.000000in}{0.000000in}}%
\pgfpathlineto{\pgfqpoint{0.000000in}{-0.048611in}}%
\pgfusepath{stroke,fill}%
}%
\begin{pgfscope}%
\pgfsys@transformshift{2.325151in}{0.521603in}%
\pgfsys@useobject{currentmarker}{}%
\end{pgfscope}%
\end{pgfscope}%
\begin{pgfscope}%
\pgftext[x=2.325151in,y=0.424381in,,top]{\rmfamily\fontsize{10.000000}{12.000000}\selectfont \(\displaystyle 20\)}%
\end{pgfscope}%
\begin{pgfscope}%
\pgfsetbuttcap%
\pgfsetroundjoin%
\definecolor{currentfill}{rgb}{0.000000,0.000000,0.000000}%
\pgfsetfillcolor{currentfill}%
\pgfsetlinewidth{0.803000pt}%
\definecolor{currentstroke}{rgb}{0.000000,0.000000,0.000000}%
\pgfsetstrokecolor{currentstroke}%
\pgfsetdash{}{0pt}%
\pgfsys@defobject{currentmarker}{\pgfqpoint{0.000000in}{-0.048611in}}{\pgfqpoint{0.000000in}{0.000000in}}{%
\pgfpathmoveto{\pgfqpoint{0.000000in}{0.000000in}}%
\pgfpathlineto{\pgfqpoint{0.000000in}{-0.048611in}}%
\pgfusepath{stroke,fill}%
}%
\begin{pgfscope}%
\pgfsys@transformshift{3.141449in}{0.521603in}%
\pgfsys@useobject{currentmarker}{}%
\end{pgfscope}%
\end{pgfscope}%
\begin{pgfscope}%
\pgftext[x=3.141449in,y=0.424381in,,top]{\rmfamily\fontsize{10.000000}{12.000000}\selectfont \(\displaystyle 30\)}%
\end{pgfscope}%
\begin{pgfscope}%
\pgfsetbuttcap%
\pgfsetroundjoin%
\definecolor{currentfill}{rgb}{0.000000,0.000000,0.000000}%
\pgfsetfillcolor{currentfill}%
\pgfsetlinewidth{0.803000pt}%
\definecolor{currentstroke}{rgb}{0.000000,0.000000,0.000000}%
\pgfsetstrokecolor{currentstroke}%
\pgfsetdash{}{0pt}%
\pgfsys@defobject{currentmarker}{\pgfqpoint{0.000000in}{-0.048611in}}{\pgfqpoint{0.000000in}{0.000000in}}{%
\pgfpathmoveto{\pgfqpoint{0.000000in}{0.000000in}}%
\pgfpathlineto{\pgfqpoint{0.000000in}{-0.048611in}}%
\pgfusepath{stroke,fill}%
}%
\begin{pgfscope}%
\pgfsys@transformshift{3.957747in}{0.521603in}%
\pgfsys@useobject{currentmarker}{}%
\end{pgfscope}%
\end{pgfscope}%
\begin{pgfscope}%
\pgftext[x=3.957747in,y=0.424381in,,top]{\rmfamily\fontsize{10.000000}{12.000000}\selectfont \(\displaystyle 40\)}%
\end{pgfscope}%
\begin{pgfscope}%
\pgftext[x=2.386080in,y=0.234413in,,top]{\rmfamily\fontsize{10.000000}{12.000000}\selectfont \(\displaystyle \bar{t}\)}%
\end{pgfscope}%
\begin{pgfscope}%
\pgfsetbuttcap%
\pgfsetroundjoin%
\definecolor{currentfill}{rgb}{0.000000,0.000000,0.000000}%
\pgfsetfillcolor{currentfill}%
\pgfsetlinewidth{0.803000pt}%
\definecolor{currentstroke}{rgb}{0.000000,0.000000,0.000000}%
\pgfsetstrokecolor{currentstroke}%
\pgfsetdash{}{0pt}%
\pgfsys@defobject{currentmarker}{\pgfqpoint{-0.048611in}{0.000000in}}{\pgfqpoint{0.000000in}{0.000000in}}{%
\pgfpathmoveto{\pgfqpoint{0.000000in}{0.000000in}}%
\pgfpathlineto{\pgfqpoint{-0.048611in}{0.000000in}}%
\pgfusepath{stroke,fill}%
}%
\begin{pgfscope}%
\pgfsys@transformshift{0.526080in}{0.672547in}%
\pgfsys@useobject{currentmarker}{}%
\end{pgfscope}%
\end{pgfscope}%
\begin{pgfscope}%
\pgftext[x=0.359413in,y=0.619786in,left,base]{\rmfamily\fontsize{10.000000}{12.000000}\selectfont \(\displaystyle 0\)}%
\end{pgfscope}%
\begin{pgfscope}%
\pgfsetbuttcap%
\pgfsetroundjoin%
\definecolor{currentfill}{rgb}{0.000000,0.000000,0.000000}%
\pgfsetfillcolor{currentfill}%
\pgfsetlinewidth{0.803000pt}%
\definecolor{currentstroke}{rgb}{0.000000,0.000000,0.000000}%
\pgfsetstrokecolor{currentstroke}%
\pgfsetdash{}{0pt}%
\pgfsys@defobject{currentmarker}{\pgfqpoint{-0.048611in}{0.000000in}}{\pgfqpoint{0.000000in}{0.000000in}}{%
\pgfpathmoveto{\pgfqpoint{0.000000in}{0.000000in}}%
\pgfpathlineto{\pgfqpoint{-0.048611in}{0.000000in}}%
\pgfusepath{stroke,fill}%
}%
\begin{pgfscope}%
\pgfsys@transformshift{0.526080in}{1.142524in}%
\pgfsys@useobject{currentmarker}{}%
\end{pgfscope}%
\end{pgfscope}%
\begin{pgfscope}%
\pgftext[x=0.359413in,y=1.089763in,left,base]{\rmfamily\fontsize{10.000000}{12.000000}\selectfont \(\displaystyle 5\)}%
\end{pgfscope}%
\begin{pgfscope}%
\pgfsetbuttcap%
\pgfsetroundjoin%
\definecolor{currentfill}{rgb}{0.000000,0.000000,0.000000}%
\pgfsetfillcolor{currentfill}%
\pgfsetlinewidth{0.803000pt}%
\definecolor{currentstroke}{rgb}{0.000000,0.000000,0.000000}%
\pgfsetstrokecolor{currentstroke}%
\pgfsetdash{}{0pt}%
\pgfsys@defobject{currentmarker}{\pgfqpoint{-0.048611in}{0.000000in}}{\pgfqpoint{0.000000in}{0.000000in}}{%
\pgfpathmoveto{\pgfqpoint{0.000000in}{0.000000in}}%
\pgfpathlineto{\pgfqpoint{-0.048611in}{0.000000in}}%
\pgfusepath{stroke,fill}%
}%
\begin{pgfscope}%
\pgfsys@transformshift{0.526080in}{1.612501in}%
\pgfsys@useobject{currentmarker}{}%
\end{pgfscope}%
\end{pgfscope}%
\begin{pgfscope}%
\pgftext[x=0.289968in,y=1.559740in,left,base]{\rmfamily\fontsize{10.000000}{12.000000}\selectfont \(\displaystyle 10\)}%
\end{pgfscope}%
\begin{pgfscope}%
\pgfsetbuttcap%
\pgfsetroundjoin%
\definecolor{currentfill}{rgb}{0.000000,0.000000,0.000000}%
\pgfsetfillcolor{currentfill}%
\pgfsetlinewidth{0.803000pt}%
\definecolor{currentstroke}{rgb}{0.000000,0.000000,0.000000}%
\pgfsetstrokecolor{currentstroke}%
\pgfsetdash{}{0pt}%
\pgfsys@defobject{currentmarker}{\pgfqpoint{-0.048611in}{0.000000in}}{\pgfqpoint{0.000000in}{0.000000in}}{%
\pgfpathmoveto{\pgfqpoint{0.000000in}{0.000000in}}%
\pgfpathlineto{\pgfqpoint{-0.048611in}{0.000000in}}%
\pgfusepath{stroke,fill}%
}%
\begin{pgfscope}%
\pgfsys@transformshift{0.526080in}{2.082478in}%
\pgfsys@useobject{currentmarker}{}%
\end{pgfscope}%
\end{pgfscope}%
\begin{pgfscope}%
\pgftext[x=0.289968in,y=2.029717in,left,base]{\rmfamily\fontsize{10.000000}{12.000000}\selectfont \(\displaystyle 15\)}%
\end{pgfscope}%
\begin{pgfscope}%
\pgfsetbuttcap%
\pgfsetroundjoin%
\definecolor{currentfill}{rgb}{0.000000,0.000000,0.000000}%
\pgfsetfillcolor{currentfill}%
\pgfsetlinewidth{0.803000pt}%
\definecolor{currentstroke}{rgb}{0.000000,0.000000,0.000000}%
\pgfsetstrokecolor{currentstroke}%
\pgfsetdash{}{0pt}%
\pgfsys@defobject{currentmarker}{\pgfqpoint{-0.048611in}{0.000000in}}{\pgfqpoint{0.000000in}{0.000000in}}{%
\pgfpathmoveto{\pgfqpoint{0.000000in}{0.000000in}}%
\pgfpathlineto{\pgfqpoint{-0.048611in}{0.000000in}}%
\pgfusepath{stroke,fill}%
}%
\begin{pgfscope}%
\pgfsys@transformshift{0.526080in}{2.552455in}%
\pgfsys@useobject{currentmarker}{}%
\end{pgfscope}%
\end{pgfscope}%
\begin{pgfscope}%
\pgftext[x=0.289968in,y=2.499694in,left,base]{\rmfamily\fontsize{10.000000}{12.000000}\selectfont \(\displaystyle 20\)}%
\end{pgfscope}%
\begin{pgfscope}%
\pgfsetbuttcap%
\pgfsetroundjoin%
\definecolor{currentfill}{rgb}{0.000000,0.000000,0.000000}%
\pgfsetfillcolor{currentfill}%
\pgfsetlinewidth{0.803000pt}%
\definecolor{currentstroke}{rgb}{0.000000,0.000000,0.000000}%
\pgfsetstrokecolor{currentstroke}%
\pgfsetdash{}{0pt}%
\pgfsys@defobject{currentmarker}{\pgfqpoint{-0.048611in}{0.000000in}}{\pgfqpoint{0.000000in}{0.000000in}}{%
\pgfpathmoveto{\pgfqpoint{0.000000in}{0.000000in}}%
\pgfpathlineto{\pgfqpoint{-0.048611in}{0.000000in}}%
\pgfusepath{stroke,fill}%
}%
\begin{pgfscope}%
\pgfsys@transformshift{0.526080in}{3.022432in}%
\pgfsys@useobject{currentmarker}{}%
\end{pgfscope}%
\end{pgfscope}%
\begin{pgfscope}%
\pgftext[x=0.289968in,y=2.969671in,left,base]{\rmfamily\fontsize{10.000000}{12.000000}\selectfont \(\displaystyle 25\)}%
\end{pgfscope}%
\begin{pgfscope}%
\pgfsetbuttcap%
\pgfsetroundjoin%
\definecolor{currentfill}{rgb}{0.000000,0.000000,0.000000}%
\pgfsetfillcolor{currentfill}%
\pgfsetlinewidth{0.803000pt}%
\definecolor{currentstroke}{rgb}{0.000000,0.000000,0.000000}%
\pgfsetstrokecolor{currentstroke}%
\pgfsetdash{}{0pt}%
\pgfsys@defobject{currentmarker}{\pgfqpoint{-0.048611in}{0.000000in}}{\pgfqpoint{0.000000in}{0.000000in}}{%
\pgfpathmoveto{\pgfqpoint{0.000000in}{0.000000in}}%
\pgfpathlineto{\pgfqpoint{-0.048611in}{0.000000in}}%
\pgfusepath{stroke,fill}%
}%
\begin{pgfscope}%
\pgfsys@transformshift{0.526080in}{3.492409in}%
\pgfsys@useobject{currentmarker}{}%
\end{pgfscope}%
\end{pgfscope}%
\begin{pgfscope}%
\pgftext[x=0.289968in,y=3.439648in,left,base]{\rmfamily\fontsize{10.000000}{12.000000}\selectfont \(\displaystyle 30\)}%
\end{pgfscope}%
\begin{pgfscope}%
\pgftext[x=0.234413in,y=2.031603in,,bottom,rotate=90.000000]{\rmfamily\fontsize{10.000000}{12.000000}\selectfont \(\displaystyle \bar{y}\)}%
\end{pgfscope}%
\begin{pgfscope}%
\pgfpathrectangle{\pgfqpoint{0.526080in}{0.521603in}}{\pgfqpoint{3.720000in}{3.020000in}} %
\pgfusepath{clip}%
\pgfsetrectcap%
\pgfsetroundjoin%
\pgfsetlinewidth{1.505625pt}%
\definecolor{currentstroke}{rgb}{1.000000,0.231948,0.116773}%
\pgfsetstrokecolor{currentstroke}%
\pgfsetdash{}{0pt}%
\pgfpathmoveto{\pgfqpoint{0.695520in}{0.675385in}}%
\pgfpathlineto{\pgfqpoint{0.695789in}{0.675712in}}%
\pgfpathlineto{\pgfqpoint{0.696059in}{0.676112in}}%
\pgfpathlineto{\pgfqpoint{0.696328in}{0.676175in}}%
\pgfpathlineto{\pgfqpoint{0.696598in}{0.676365in}}%
\pgfpathlineto{\pgfqpoint{0.696867in}{0.676690in}}%
\pgfpathlineto{\pgfqpoint{0.697137in}{0.676803in}}%
\pgfpathlineto{\pgfqpoint{0.697406in}{0.676952in}}%
\pgfpathlineto{\pgfqpoint{0.697676in}{0.677075in}}%
\pgfpathlineto{\pgfqpoint{0.697945in}{0.677178in}}%
\pgfpathlineto{\pgfqpoint{0.698215in}{0.677213in}}%
\pgfpathlineto{\pgfqpoint{0.698484in}{0.677229in}}%
\pgfpathlineto{\pgfqpoint{0.698754in}{0.677246in}}%
\pgfpathlineto{\pgfqpoint{0.699023in}{0.677191in}}%
\pgfpathlineto{\pgfqpoint{0.699293in}{0.677165in}}%
\pgfpathlineto{\pgfqpoint{0.699562in}{0.677066in}}%
\pgfpathlineto{\pgfqpoint{0.699832in}{0.676969in}}%
\pgfpathlineto{\pgfqpoint{0.700101in}{0.676816in}}%
\pgfpathlineto{\pgfqpoint{0.700371in}{0.676665in}}%
\pgfpathlineto{\pgfqpoint{0.700640in}{0.676456in}}%
\pgfpathlineto{\pgfqpoint{0.700910in}{0.676236in}}%
\pgfpathlineto{\pgfqpoint{0.701179in}{0.675989in}}%
\pgfpathlineto{\pgfqpoint{0.701449in}{0.675695in}}%
\pgfpathlineto{\pgfqpoint{0.701718in}{0.675388in}}%
\pgfpathlineto{\pgfqpoint{0.701988in}{0.675022in}}%
\pgfpathlineto{\pgfqpoint{0.702257in}{0.674664in}}%
\pgfpathlineto{\pgfqpoint{0.702527in}{0.674228in}}%
\pgfpathlineto{\pgfqpoint{0.702796in}{0.673761in}}%
\pgfpathlineto{\pgfqpoint{0.703066in}{0.673205in}}%
\pgfpathlineto{\pgfqpoint{0.703335in}{0.672645in}}%
\pgfpathlineto{\pgfqpoint{0.703605in}{0.671936in}}%
\pgfpathlineto{\pgfqpoint{0.703874in}{0.671400in}}%
\pgfpathlineto{\pgfqpoint{0.704143in}{0.670970in}}%
\pgfpathlineto{\pgfqpoint{0.704413in}{0.671012in}}%
\pgfpathlineto{\pgfqpoint{0.704682in}{0.671393in}}%
\pgfpathlineto{\pgfqpoint{0.704952in}{0.671927in}}%
\pgfpathlineto{\pgfqpoint{0.705221in}{0.672443in}}%
\pgfpathlineto{\pgfqpoint{0.705491in}{0.672833in}}%
\pgfpathlineto{\pgfqpoint{0.705760in}{0.673137in}}%
\pgfpathlineto{\pgfqpoint{0.706030in}{0.673377in}}%
\pgfpathlineto{\pgfqpoint{0.706299in}{0.673520in}}%
\pgfpathlineto{\pgfqpoint{0.706569in}{0.673641in}}%
\pgfpathlineto{\pgfqpoint{0.706838in}{0.673638in}}%
\pgfpathlineto{\pgfqpoint{0.707108in}{0.673586in}}%
\pgfpathlineto{\pgfqpoint{0.707377in}{0.673444in}}%
\pgfpathlineto{\pgfqpoint{0.707647in}{0.673244in}}%
\pgfpathlineto{\pgfqpoint{0.707916in}{0.672953in}}%
\pgfpathlineto{\pgfqpoint{0.708186in}{0.672599in}}%
\pgfpathlineto{\pgfqpoint{0.708455in}{0.672064in}}%
\pgfpathlineto{\pgfqpoint{0.708725in}{0.671500in}}%
\pgfpathlineto{\pgfqpoint{0.708994in}{0.671177in}}%
\pgfpathlineto{\pgfqpoint{0.709264in}{0.671217in}}%
\pgfpathlineto{\pgfqpoint{0.709533in}{0.671592in}}%
\pgfpathlineto{\pgfqpoint{0.709803in}{0.672107in}}%
\pgfpathlineto{\pgfqpoint{0.710072in}{0.672597in}}%
\pgfpathlineto{\pgfqpoint{0.710342in}{0.672914in}}%
\pgfpathlineto{\pgfqpoint{0.710611in}{0.673133in}}%
\pgfpathlineto{\pgfqpoint{0.710881in}{0.673262in}}%
\pgfpathlineto{\pgfqpoint{0.711150in}{0.673320in}}%
\pgfpathlineto{\pgfqpoint{0.711420in}{0.673305in}}%
\pgfpathlineto{\pgfqpoint{0.711689in}{0.673207in}}%
\pgfpathlineto{\pgfqpoint{0.711959in}{0.673051in}}%
\pgfpathlineto{\pgfqpoint{0.712228in}{0.672823in}}%
\pgfpathlineto{\pgfqpoint{0.712498in}{0.672442in}}%
\pgfpathlineto{\pgfqpoint{0.712767in}{0.671944in}}%
\pgfpathlineto{\pgfqpoint{0.713037in}{0.671646in}}%
\pgfusepath{stroke}%
\end{pgfscope}%
\begin{pgfscope}%
\pgfpathrectangle{\pgfqpoint{0.526080in}{0.521603in}}{\pgfqpoint{3.720000in}{3.020000in}} %
\pgfusepath{clip}%
\pgfsetrectcap%
\pgfsetroundjoin%
\pgfsetlinewidth{1.505625pt}%
\definecolor{currentstroke}{rgb}{0.966667,0.743145,0.406737}%
\pgfsetstrokecolor{currentstroke}%
\pgfsetdash{}{0pt}%
\pgfpathmoveto{\pgfqpoint{0.695171in}{0.670014in}}%
\pgfpathlineto{\pgfqpoint{0.698440in}{0.676313in}}%
\pgfpathlineto{\pgfqpoint{0.703670in}{0.682365in}}%
\pgfpathlineto{\pgfqpoint{0.707593in}{0.684026in}}%
\pgfpathlineto{\pgfqpoint{0.714131in}{0.682132in}}%
\pgfpathlineto{\pgfqpoint{0.718054in}{0.678248in}}%
\pgfpathlineto{\pgfqpoint{0.724591in}{0.667864in}}%
\pgfpathlineto{\pgfqpoint{0.725245in}{0.668428in}}%
\pgfpathlineto{\pgfqpoint{0.735052in}{0.680579in}}%
\pgfpathlineto{\pgfqpoint{0.741590in}{0.679886in}}%
\pgfpathlineto{\pgfqpoint{0.747474in}{0.672565in}}%
\pgfpathlineto{\pgfqpoint{0.750743in}{0.668688in}}%
\pgfpathlineto{\pgfqpoint{0.752705in}{0.669644in}}%
\pgfpathlineto{\pgfqpoint{0.759242in}{0.675459in}}%
\pgfpathlineto{\pgfqpoint{0.761204in}{0.675175in}}%
\pgfpathlineto{\pgfqpoint{0.769049in}{0.669136in}}%
\pgfpathlineto{\pgfqpoint{0.774280in}{0.673821in}}%
\pgfpathlineto{\pgfqpoint{0.777549in}{0.673779in}}%
\pgfpathlineto{\pgfqpoint{0.782779in}{0.669093in}}%
\pgfpathlineto{\pgfqpoint{0.784740in}{0.669416in}}%
\pgfpathlineto{\pgfqpoint{0.786702in}{0.671170in}}%
\pgfpathlineto{\pgfqpoint{0.790625in}{0.673148in}}%
\pgfpathlineto{\pgfqpoint{0.792586in}{0.672583in}}%
\pgfpathlineto{\pgfqpoint{0.798470in}{0.668984in}}%
\pgfpathlineto{\pgfqpoint{0.805662in}{0.672817in}}%
\pgfpathlineto{\pgfqpoint{0.812853in}{0.669156in}}%
\pgfpathlineto{\pgfqpoint{0.818738in}{0.672799in}}%
\pgfpathlineto{\pgfqpoint{0.822007in}{0.670618in}}%
\pgfpathlineto{\pgfqpoint{0.824622in}{0.668621in}}%
\pgfpathlineto{\pgfqpoint{0.826583in}{0.669054in}}%
\pgfpathlineto{\pgfqpoint{0.831813in}{0.672616in}}%
\pgfpathlineto{\pgfqpoint{0.833121in}{0.672362in}}%
\pgfpathlineto{\pgfqpoint{0.839659in}{0.668575in}}%
\pgfpathlineto{\pgfqpoint{0.840313in}{0.668971in}}%
\pgfpathlineto{\pgfqpoint{0.840313in}{0.668971in}}%
\pgfusepath{stroke}%
\end{pgfscope}%
\begin{pgfscope}%
\pgfpathrectangle{\pgfqpoint{0.526080in}{0.521603in}}{\pgfqpoint{3.720000in}{3.020000in}} %
\pgfusepath{clip}%
\pgfsetrectcap%
\pgfsetroundjoin%
\pgfsetlinewidth{1.505625pt}%
\definecolor{currentstroke}{rgb}{1.000000,0.000000,0.000000}%
\pgfsetstrokecolor{currentstroke}%
\pgfsetdash{}{0pt}%
\pgfpathmoveto{\pgfqpoint{0.695217in}{0.677836in}}%
\pgfpathlineto{\pgfqpoint{0.698469in}{0.680075in}}%
\pgfpathlineto{\pgfqpoint{0.701722in}{0.678913in}}%
\pgfpathlineto{\pgfqpoint{0.704087in}{0.675637in}}%
\pgfpathlineto{\pgfqpoint{0.705270in}{0.672909in}}%
\pgfpathlineto{\pgfqpoint{0.705861in}{0.673675in}}%
\pgfpathlineto{\pgfqpoint{0.708227in}{0.676471in}}%
\pgfpathlineto{\pgfqpoint{0.710001in}{0.675278in}}%
\pgfpathlineto{\pgfqpoint{0.711479in}{0.672847in}}%
\pgfpathlineto{\pgfqpoint{0.712071in}{0.673682in}}%
\pgfpathlineto{\pgfqpoint{0.714140in}{0.675285in}}%
\pgfpathlineto{\pgfqpoint{0.715323in}{0.674298in}}%
\pgfpathlineto{\pgfqpoint{0.716506in}{0.672884in}}%
\pgfpathlineto{\pgfqpoint{0.716802in}{0.673193in}}%
\pgfpathlineto{\pgfqpoint{0.718576in}{0.674385in}}%
\pgfpathlineto{\pgfqpoint{0.720350in}{0.673007in}}%
\pgfpathlineto{\pgfqpoint{0.720941in}{0.673812in}}%
\pgfpathlineto{\pgfqpoint{0.722124in}{0.673602in}}%
\pgfpathlineto{\pgfqpoint{0.723307in}{0.673094in}}%
\pgfpathlineto{\pgfqpoint{0.723602in}{0.673408in}}%
\pgfpathlineto{\pgfqpoint{0.725081in}{0.673218in}}%
\pgfpathlineto{\pgfqpoint{0.726263in}{0.673412in}}%
\pgfpathlineto{\pgfqpoint{0.727742in}{0.673221in}}%
\pgfpathlineto{\pgfqpoint{0.730403in}{0.673154in}}%
\pgfpathlineto{\pgfqpoint{0.745187in}{0.673385in}}%
\pgfpathlineto{\pgfqpoint{0.751101in}{0.673275in}}%
\pgfpathlineto{\pgfqpoint{0.761154in}{0.673334in}}%
\pgfpathlineto{\pgfqpoint{0.761154in}{0.673334in}}%
\pgfusepath{stroke}%
\end{pgfscope}%
\begin{pgfscope}%
\pgfpathrectangle{\pgfqpoint{0.526080in}{0.521603in}}{\pgfqpoint{3.720000in}{3.020000in}} %
\pgfusepath{clip}%
\pgfsetrectcap%
\pgfsetroundjoin%
\pgfsetlinewidth{1.505625pt}%
\definecolor{currentstroke}{rgb}{0.966667,0.743145,0.406737}%
\pgfsetstrokecolor{currentstroke}%
\pgfsetdash{}{0pt}%
\pgfpathmoveto{\pgfqpoint{0.700666in}{0.679030in}}%
\pgfpathlineto{\pgfqpoint{0.705736in}{0.684020in}}%
\pgfpathlineto{\pgfqpoint{0.714861in}{0.687019in}}%
\pgfpathlineto{\pgfqpoint{0.722971in}{0.684314in}}%
\pgfpathlineto{\pgfqpoint{0.731082in}{0.675763in}}%
\pgfpathlineto{\pgfqpoint{0.735138in}{0.669745in}}%
\pgfpathlineto{\pgfqpoint{0.737166in}{0.670284in}}%
\pgfpathlineto{\pgfqpoint{0.743249in}{0.677364in}}%
\pgfpathlineto{\pgfqpoint{0.752373in}{0.678454in}}%
\pgfpathlineto{\pgfqpoint{0.758457in}{0.673667in}}%
\pgfpathlineto{\pgfqpoint{0.762512in}{0.670185in}}%
\pgfpathlineto{\pgfqpoint{0.764540in}{0.671499in}}%
\pgfpathlineto{\pgfqpoint{0.769609in}{0.675806in}}%
\pgfpathlineto{\pgfqpoint{0.773665in}{0.675281in}}%
\pgfpathlineto{\pgfqpoint{0.776706in}{0.672612in}}%
\pgfpathlineto{\pgfqpoint{0.780762in}{0.670544in}}%
\pgfpathlineto{\pgfqpoint{0.787859in}{0.675230in}}%
\pgfpathlineto{\pgfqpoint{0.791914in}{0.673396in}}%
\pgfpathlineto{\pgfqpoint{0.796983in}{0.671081in}}%
\pgfpathlineto{\pgfqpoint{0.805094in}{0.674062in}}%
\pgfpathlineto{\pgfqpoint{0.810164in}{0.670949in}}%
\pgfpathlineto{\pgfqpoint{0.813205in}{0.672689in}}%
\pgfpathlineto{\pgfqpoint{0.817261in}{0.674625in}}%
\pgfpathlineto{\pgfqpoint{0.820302in}{0.673247in}}%
\pgfpathlineto{\pgfqpoint{0.824358in}{0.671287in}}%
\pgfpathlineto{\pgfqpoint{0.832469in}{0.674197in}}%
\pgfpathlineto{\pgfqpoint{0.838552in}{0.671852in}}%
\pgfpathlineto{\pgfqpoint{0.844635in}{0.674780in}}%
\pgfpathlineto{\pgfqpoint{0.848690in}{0.672868in}}%
\pgfpathlineto{\pgfqpoint{0.851732in}{0.671896in}}%
\pgfpathlineto{\pgfqpoint{0.860857in}{0.673915in}}%
\pgfpathlineto{\pgfqpoint{0.865926in}{0.672260in}}%
\pgfpathlineto{\pgfqpoint{0.873023in}{0.674797in}}%
\pgfpathlineto{\pgfqpoint{0.881134in}{0.672899in}}%
\pgfpathlineto{\pgfqpoint{0.886203in}{0.674670in}}%
\pgfpathlineto{\pgfqpoint{0.896342in}{0.673557in}}%
\pgfpathlineto{\pgfqpoint{0.900397in}{0.674308in}}%
\pgfpathlineto{\pgfqpoint{0.908508in}{0.673062in}}%
\pgfpathlineto{\pgfqpoint{0.913578in}{0.674087in}}%
\pgfpathlineto{\pgfqpoint{0.922702in}{0.673366in}}%
\pgfpathlineto{\pgfqpoint{0.926758in}{0.673909in}}%
\pgfpathlineto{\pgfqpoint{0.929799in}{0.673246in}}%
\pgfpathlineto{\pgfqpoint{0.929799in}{0.673246in}}%
\pgfusepath{stroke}%
\end{pgfscope}%
\begin{pgfscope}%
\pgfpathrectangle{\pgfqpoint{0.526080in}{0.521603in}}{\pgfqpoint{3.720000in}{3.020000in}} %
\pgfusepath{clip}%
\pgfsetrectcap%
\pgfsetroundjoin%
\pgfsetlinewidth{1.505625pt}%
\definecolor{currentstroke}{rgb}{0.770588,0.911023,0.542053}%
\pgfsetstrokecolor{currentstroke}%
\pgfsetdash{}{0pt}%
\pgfpathmoveto{\pgfqpoint{0.709433in}{0.687417in}}%
\pgfpathlineto{\pgfqpoint{0.711121in}{0.689574in}}%
\pgfpathlineto{\pgfqpoint{0.712808in}{0.691100in}}%
\pgfpathlineto{\pgfqpoint{0.714496in}{0.692624in}}%
\pgfpathlineto{\pgfqpoint{0.716184in}{0.694636in}}%
\pgfpathlineto{\pgfqpoint{0.717872in}{0.695980in}}%
\pgfpathlineto{\pgfqpoint{0.719559in}{0.697287in}}%
\pgfpathlineto{\pgfqpoint{0.721247in}{0.699036in}}%
\pgfpathlineto{\pgfqpoint{0.722935in}{0.700123in}}%
\pgfpathlineto{\pgfqpoint{0.724623in}{0.701359in}}%
\pgfpathlineto{\pgfqpoint{0.726310in}{0.702663in}}%
\pgfpathlineto{\pgfqpoint{0.727998in}{0.703466in}}%
\pgfpathlineto{\pgfqpoint{0.729686in}{0.704694in}}%
\pgfpathlineto{\pgfqpoint{0.731374in}{0.705519in}}%
\pgfpathlineto{\pgfqpoint{0.733061in}{0.706325in}}%
\pgfpathlineto{\pgfqpoint{0.734749in}{0.707227in}}%
\pgfpathlineto{\pgfqpoint{0.736437in}{0.707684in}}%
\pgfpathlineto{\pgfqpoint{0.738125in}{0.708431in}}%
\pgfpathlineto{\pgfqpoint{0.739812in}{0.708877in}}%
\pgfpathlineto{\pgfqpoint{0.741500in}{0.709238in}}%
\pgfpathlineto{\pgfqpoint{0.743188in}{0.709836in}}%
\pgfpathlineto{\pgfqpoint{0.744876in}{0.709913in}}%
\pgfpathlineto{\pgfqpoint{0.746563in}{0.710216in}}%
\pgfpathlineto{\pgfqpoint{0.748251in}{0.710366in}}%
\pgfpathlineto{\pgfqpoint{0.749939in}{0.710292in}}%
\pgfpathlineto{\pgfqpoint{0.751626in}{0.710503in}}%
\pgfpathlineto{\pgfqpoint{0.753314in}{0.710265in}}%
\pgfpathlineto{\pgfqpoint{0.755002in}{0.710138in}}%
\pgfpathlineto{\pgfqpoint{0.756690in}{0.709929in}}%
\pgfpathlineto{\pgfqpoint{0.758377in}{0.709488in}}%
\pgfpathlineto{\pgfqpoint{0.760065in}{0.709298in}}%
\pgfpathlineto{\pgfqpoint{0.761753in}{0.708693in}}%
\pgfpathlineto{\pgfqpoint{0.763441in}{0.708132in}}%
\pgfpathlineto{\pgfqpoint{0.765128in}{0.707606in}}%
\pgfpathlineto{\pgfqpoint{0.766816in}{0.706747in}}%
\pgfpathlineto{\pgfqpoint{0.768504in}{0.706104in}}%
\pgfpathlineto{\pgfqpoint{0.770192in}{0.705165in}}%
\pgfpathlineto{\pgfqpoint{0.771879in}{0.704147in}}%
\pgfpathlineto{\pgfqpoint{0.773567in}{0.703243in}}%
\pgfpathlineto{\pgfqpoint{0.775255in}{0.701965in}}%
\pgfpathlineto{\pgfqpoint{0.776943in}{0.700914in}}%
\pgfpathlineto{\pgfqpoint{0.778630in}{0.699558in}}%
\pgfpathlineto{\pgfqpoint{0.780318in}{0.698103in}}%
\pgfpathlineto{\pgfqpoint{0.782006in}{0.696752in}}%
\pgfpathlineto{\pgfqpoint{0.783693in}{0.695085in}}%
\pgfpathlineto{\pgfqpoint{0.785381in}{0.693569in}}%
\pgfpathlineto{\pgfqpoint{0.787069in}{0.691801in}}%
\pgfpathlineto{\pgfqpoint{0.788757in}{0.689847in}}%
\pgfpathlineto{\pgfqpoint{0.790444in}{0.688020in}}%
\pgfpathlineto{\pgfqpoint{0.792132in}{0.685831in}}%
\pgfpathlineto{\pgfqpoint{0.793820in}{0.683740in}}%
\pgfpathlineto{\pgfqpoint{0.795508in}{0.681388in}}%
\pgfpathlineto{\pgfqpoint{0.797195in}{0.678665in}}%
\pgfpathlineto{\pgfqpoint{0.798883in}{0.675929in}}%
\pgfpathlineto{\pgfqpoint{0.800571in}{0.673031in}}%
\pgfpathlineto{\pgfqpoint{0.802259in}{0.671312in}}%
\pgfpathlineto{\pgfqpoint{0.803946in}{0.669546in}}%
\pgfpathlineto{\pgfqpoint{0.805634in}{0.670272in}}%
\pgfpathlineto{\pgfqpoint{0.807322in}{0.671620in}}%
\pgfpathlineto{\pgfqpoint{0.809010in}{0.673692in}}%
\pgfpathlineto{\pgfqpoint{0.810697in}{0.676187in}}%
\pgfpathlineto{\pgfqpoint{0.812385in}{0.678224in}}%
\pgfpathlineto{\pgfqpoint{0.814073in}{0.679717in}}%
\pgfpathlineto{\pgfqpoint{0.815761in}{0.681147in}}%
\pgfpathlineto{\pgfqpoint{0.817448in}{0.682038in}}%
\pgfpathlineto{\pgfqpoint{0.819136in}{0.682880in}}%
\pgfpathlineto{\pgfqpoint{0.820824in}{0.683785in}}%
\pgfpathlineto{\pgfqpoint{0.822511in}{0.684166in}}%
\pgfpathlineto{\pgfqpoint{0.824199in}{0.684599in}}%
\pgfpathlineto{\pgfqpoint{0.825887in}{0.684680in}}%
\pgfpathlineto{\pgfqpoint{0.827575in}{0.684517in}}%
\pgfpathlineto{\pgfqpoint{0.829262in}{0.684404in}}%
\pgfpathlineto{\pgfqpoint{0.830950in}{0.683872in}}%
\pgfpathlineto{\pgfqpoint{0.832638in}{0.683375in}}%
\pgfpathlineto{\pgfqpoint{0.834326in}{0.682664in}}%
\pgfpathlineto{\pgfqpoint{0.836013in}{0.681513in}}%
\pgfpathlineto{\pgfqpoint{0.837701in}{0.680304in}}%
\pgfpathlineto{\pgfqpoint{0.839389in}{0.678689in}}%
\pgfpathlineto{\pgfqpoint{0.841077in}{0.676751in}}%
\pgfpathlineto{\pgfqpoint{0.842764in}{0.674583in}}%
\pgfpathlineto{\pgfqpoint{0.844452in}{0.672640in}}%
\pgfpathlineto{\pgfqpoint{0.846140in}{0.671484in}}%
\pgfpathlineto{\pgfqpoint{0.847828in}{0.670600in}}%
\pgfusepath{stroke}%
\end{pgfscope}%
\begin{pgfscope}%
\pgfpathrectangle{\pgfqpoint{0.526080in}{0.521603in}}{\pgfqpoint{3.720000in}{3.020000in}} %
\pgfusepath{clip}%
\pgfsetrectcap%
\pgfsetroundjoin%
\pgfsetlinewidth{1.505625pt}%
\definecolor{currentstroke}{rgb}{0.590196,0.989980,0.655284}%
\pgfsetstrokecolor{currentstroke}%
\pgfsetdash{}{0pt}%
\pgfpathmoveto{\pgfqpoint{0.715002in}{0.688549in}}%
\pgfpathlineto{\pgfqpoint{0.719990in}{0.693281in}}%
\pgfpathlineto{\pgfqpoint{0.724978in}{0.698710in}}%
\pgfpathlineto{\pgfqpoint{0.729966in}{0.702751in}}%
\pgfpathlineto{\pgfqpoint{0.734955in}{0.706971in}}%
\pgfpathlineto{\pgfqpoint{0.759895in}{0.720717in}}%
\pgfpathlineto{\pgfqpoint{0.779848in}{0.724212in}}%
\pgfpathlineto{\pgfqpoint{0.787330in}{0.723857in}}%
\pgfpathlineto{\pgfqpoint{0.799800in}{0.721224in}}%
\pgfpathlineto{\pgfqpoint{0.822247in}{0.710248in}}%
\pgfpathlineto{\pgfqpoint{0.827235in}{0.706633in}}%
\pgfpathlineto{\pgfqpoint{0.842199in}{0.692714in}}%
\pgfpathlineto{\pgfqpoint{0.852175in}{0.679834in}}%
\pgfpathlineto{\pgfqpoint{0.859657in}{0.668527in}}%
\pgfpathlineto{\pgfqpoint{0.862152in}{0.665752in}}%
\pgfpathlineto{\pgfqpoint{0.867140in}{0.666773in}}%
\pgfpathlineto{\pgfqpoint{0.884598in}{0.688709in}}%
\pgfpathlineto{\pgfqpoint{0.897068in}{0.697648in}}%
\pgfpathlineto{\pgfqpoint{0.907045in}{0.701601in}}%
\pgfpathlineto{\pgfqpoint{0.914527in}{0.703011in}}%
\pgfpathlineto{\pgfqpoint{0.924503in}{0.702626in}}%
\pgfpathlineto{\pgfqpoint{0.931985in}{0.700899in}}%
\pgfpathlineto{\pgfqpoint{0.941961in}{0.696166in}}%
\pgfpathlineto{\pgfqpoint{0.949444in}{0.690939in}}%
\pgfpathlineto{\pgfqpoint{0.959420in}{0.680822in}}%
\pgfpathlineto{\pgfqpoint{0.966902in}{0.671127in}}%
\pgfpathlineto{\pgfqpoint{0.971890in}{0.667229in}}%
\pgfpathlineto{\pgfqpoint{0.976878in}{0.668553in}}%
\pgfpathlineto{\pgfqpoint{0.981866in}{0.673130in}}%
\pgfpathlineto{\pgfqpoint{0.986854in}{0.679104in}}%
\pgfpathlineto{\pgfqpoint{0.994337in}{0.685087in}}%
\pgfpathlineto{\pgfqpoint{0.999325in}{0.687313in}}%
\pgfpathlineto{\pgfqpoint{1.009301in}{0.689168in}}%
\pgfpathlineto{\pgfqpoint{1.019277in}{0.687731in}}%
\pgfpathlineto{\pgfqpoint{1.026759in}{0.684752in}}%
\pgfpathlineto{\pgfqpoint{1.041724in}{0.671314in}}%
\pgfpathlineto{\pgfqpoint{1.049206in}{0.668604in}}%
\pgfpathlineto{\pgfqpoint{1.054194in}{0.671225in}}%
\pgfpathlineto{\pgfqpoint{1.069158in}{0.681092in}}%
\pgfpathlineto{\pgfqpoint{1.074146in}{0.682909in}}%
\pgfpathlineto{\pgfqpoint{1.086617in}{0.679618in}}%
\pgfpathlineto{\pgfqpoint{1.094099in}{0.673731in}}%
\pgfpathlineto{\pgfqpoint{1.099087in}{0.670279in}}%
\pgfpathlineto{\pgfqpoint{1.101581in}{0.668768in}}%
\pgfpathlineto{\pgfqpoint{1.106569in}{0.669309in}}%
\pgfpathlineto{\pgfqpoint{1.114051in}{0.674443in}}%
\pgfpathlineto{\pgfqpoint{1.119039in}{0.679039in}}%
\pgfpathlineto{\pgfqpoint{1.129016in}{0.682741in}}%
\pgfpathlineto{\pgfqpoint{1.138992in}{0.681762in}}%
\pgfpathlineto{\pgfqpoint{1.146474in}{0.678799in}}%
\pgfpathlineto{\pgfqpoint{1.151462in}{0.674464in}}%
\pgfpathlineto{\pgfqpoint{1.161438in}{0.669136in}}%
\pgfpathlineto{\pgfqpoint{1.166427in}{0.670542in}}%
\pgfpathlineto{\pgfqpoint{1.183885in}{0.680085in}}%
\pgfpathlineto{\pgfqpoint{1.193861in}{0.679403in}}%
\pgfpathlineto{\pgfqpoint{1.211320in}{0.670036in}}%
\pgfpathlineto{\pgfqpoint{1.216308in}{0.670177in}}%
\pgfpathlineto{\pgfqpoint{1.238754in}{0.680295in}}%
\pgfpathlineto{\pgfqpoint{1.246236in}{0.679855in}}%
\pgfpathlineto{\pgfqpoint{1.251224in}{0.677655in}}%
\pgfpathlineto{\pgfqpoint{1.261201in}{0.672716in}}%
\pgfpathlineto{\pgfqpoint{1.266189in}{0.670818in}}%
\pgfpathlineto{\pgfqpoint{1.271177in}{0.671000in}}%
\pgfpathlineto{\pgfqpoint{1.273671in}{0.671925in}}%
\pgfpathlineto{\pgfqpoint{1.273671in}{0.671925in}}%
\pgfusepath{stroke}%
\end{pgfscope}%
\begin{pgfscope}%
\pgfpathrectangle{\pgfqpoint{0.526080in}{0.521603in}}{\pgfqpoint{3.720000in}{3.020000in}} %
\pgfusepath{clip}%
\pgfsetrectcap%
\pgfsetroundjoin%
\pgfsetlinewidth{1.505625pt}%
\definecolor{currentstroke}{rgb}{0.692157,0.954791,0.592758}%
\pgfsetstrokecolor{currentstroke}%
\pgfsetdash{}{0pt}%
\pgfpathmoveto{\pgfqpoint{0.716973in}{0.699824in}}%
\pgfpathlineto{\pgfqpoint{0.732235in}{0.715378in}}%
\pgfpathlineto{\pgfqpoint{0.753600in}{0.731772in}}%
\pgfpathlineto{\pgfqpoint{0.765809in}{0.738233in}}%
\pgfpathlineto{\pgfqpoint{0.774966in}{0.742091in}}%
\pgfpathlineto{\pgfqpoint{0.793279in}{0.746385in}}%
\pgfpathlineto{\pgfqpoint{0.811593in}{0.746637in}}%
\pgfpathlineto{\pgfqpoint{0.829906in}{0.742882in}}%
\pgfpathlineto{\pgfqpoint{0.848219in}{0.735043in}}%
\pgfpathlineto{\pgfqpoint{0.863481in}{0.725175in}}%
\pgfpathlineto{\pgfqpoint{0.875689in}{0.714932in}}%
\pgfpathlineto{\pgfqpoint{0.887898in}{0.702414in}}%
\pgfpathlineto{\pgfqpoint{0.897055in}{0.691084in}}%
\pgfpathlineto{\pgfqpoint{0.909264in}{0.673603in}}%
\pgfpathlineto{\pgfqpoint{0.912316in}{0.672543in}}%
\pgfpathlineto{\pgfqpoint{0.915368in}{0.674704in}}%
\pgfpathlineto{\pgfqpoint{0.930630in}{0.693995in}}%
\pgfpathlineto{\pgfqpoint{0.939786in}{0.702257in}}%
\pgfpathlineto{\pgfqpoint{0.955047in}{0.712310in}}%
\pgfpathlineto{\pgfqpoint{0.967256in}{0.717762in}}%
\pgfpathlineto{\pgfqpoint{0.979465in}{0.721061in}}%
\pgfpathlineto{\pgfqpoint{0.991674in}{0.722303in}}%
\pgfpathlineto{\pgfqpoint{1.003883in}{0.721588in}}%
\pgfpathlineto{\pgfqpoint{1.016092in}{0.718815in}}%
\pgfpathlineto{\pgfqpoint{1.028301in}{0.714087in}}%
\pgfpathlineto{\pgfqpoint{1.040510in}{0.707307in}}%
\pgfpathlineto{\pgfqpoint{1.052719in}{0.698323in}}%
\pgfpathlineto{\pgfqpoint{1.064928in}{0.686328in}}%
\pgfpathlineto{\pgfqpoint{1.074084in}{0.675189in}}%
\pgfpathlineto{\pgfqpoint{1.077137in}{0.673503in}}%
\pgfpathlineto{\pgfqpoint{1.080189in}{0.674905in}}%
\pgfpathlineto{\pgfqpoint{1.104607in}{0.696838in}}%
\pgfpathlineto{\pgfqpoint{1.113763in}{0.701555in}}%
\pgfpathlineto{\pgfqpoint{1.129025in}{0.705452in}}%
\pgfpathlineto{\pgfqpoint{1.138181in}{0.706064in}}%
\pgfpathlineto{\pgfqpoint{1.150390in}{0.704323in}}%
\pgfpathlineto{\pgfqpoint{1.162599in}{0.699719in}}%
\pgfpathlineto{\pgfqpoint{1.174808in}{0.691845in}}%
\pgfpathlineto{\pgfqpoint{1.193121in}{0.674720in}}%
\pgfpathlineto{\pgfqpoint{1.196174in}{0.675869in}}%
\pgfpathlineto{\pgfqpoint{1.214487in}{0.691912in}}%
\pgfpathlineto{\pgfqpoint{1.223644in}{0.695769in}}%
\pgfpathlineto{\pgfqpoint{1.232800in}{0.697581in}}%
\pgfpathlineto{\pgfqpoint{1.241957in}{0.697444in}}%
\pgfpathlineto{\pgfqpoint{1.251114in}{0.695393in}}%
\pgfpathlineto{\pgfqpoint{1.260270in}{0.691283in}}%
\pgfpathlineto{\pgfqpoint{1.269427in}{0.684504in}}%
\pgfpathlineto{\pgfqpoint{1.278584in}{0.676487in}}%
\pgfpathlineto{\pgfqpoint{1.281636in}{0.676075in}}%
\pgfpathlineto{\pgfqpoint{1.284688in}{0.677194in}}%
\pgfpathlineto{\pgfqpoint{1.296897in}{0.685819in}}%
\pgfpathlineto{\pgfqpoint{1.306054in}{0.688432in}}%
\pgfpathlineto{\pgfqpoint{1.312158in}{0.688789in}}%
\pgfpathlineto{\pgfqpoint{1.321315in}{0.687175in}}%
\pgfpathlineto{\pgfqpoint{1.330472in}{0.682254in}}%
\pgfpathlineto{\pgfqpoint{1.336576in}{0.677887in}}%
\pgfpathlineto{\pgfqpoint{1.339628in}{0.676797in}}%
\pgfpathlineto{\pgfqpoint{1.342681in}{0.677257in}}%
\pgfpathlineto{\pgfqpoint{1.360994in}{0.686617in}}%
\pgfpathlineto{\pgfqpoint{1.370151in}{0.687233in}}%
\pgfpathlineto{\pgfqpoint{1.379307in}{0.685045in}}%
\pgfpathlineto{\pgfqpoint{1.394569in}{0.677364in}}%
\pgfpathlineto{\pgfqpoint{1.400673in}{0.679038in}}%
\pgfpathlineto{\pgfqpoint{1.403725in}{0.680931in}}%
\pgfpathlineto{\pgfqpoint{1.403725in}{0.680931in}}%
\pgfusepath{stroke}%
\end{pgfscope}%
\begin{pgfscope}%
\pgfpathrectangle{\pgfqpoint{0.526080in}{0.521603in}}{\pgfqpoint{3.720000in}{3.020000in}} %
\pgfusepath{clip}%
\pgfsetrectcap%
\pgfsetroundjoin%
\pgfsetlinewidth{1.505625pt}%
\definecolor{currentstroke}{rgb}{0.268627,0.934680,0.823253}%
\pgfsetstrokecolor{currentstroke}%
\pgfsetdash{}{0pt}%
\pgfpathmoveto{\pgfqpoint{0.707028in}{0.667235in}}%
\pgfpathlineto{\pgfqpoint{0.711853in}{0.677681in}}%
\pgfpathlineto{\pgfqpoint{0.716677in}{0.684403in}}%
\pgfpathlineto{\pgfqpoint{0.726325in}{0.694158in}}%
\pgfpathlineto{\pgfqpoint{0.735974in}{0.709178in}}%
\pgfpathlineto{\pgfqpoint{0.745622in}{0.718084in}}%
\pgfpathlineto{\pgfqpoint{0.750446in}{0.724238in}}%
\pgfpathlineto{\pgfqpoint{0.774568in}{0.745267in}}%
\pgfpathlineto{\pgfqpoint{0.784216in}{0.755267in}}%
\pgfpathlineto{\pgfqpoint{0.793865in}{0.761375in}}%
\pgfpathlineto{\pgfqpoint{0.808337in}{0.773217in}}%
\pgfpathlineto{\pgfqpoint{0.817986in}{0.778038in}}%
\pgfpathlineto{\pgfqpoint{0.832459in}{0.788195in}}%
\pgfpathlineto{\pgfqpoint{0.837283in}{0.789814in}}%
\pgfpathlineto{\pgfqpoint{0.856580in}{0.800298in}}%
\pgfpathlineto{\pgfqpoint{0.861404in}{0.801583in}}%
\pgfpathlineto{\pgfqpoint{0.875877in}{0.809091in}}%
\pgfpathlineto{\pgfqpoint{0.885525in}{0.811321in}}%
\pgfpathlineto{\pgfqpoint{0.895174in}{0.815768in}}%
\pgfpathlineto{\pgfqpoint{0.943416in}{0.825482in}}%
\pgfpathlineto{\pgfqpoint{0.953065in}{0.825299in}}%
\pgfpathlineto{\pgfqpoint{0.967537in}{0.826670in}}%
\pgfpathlineto{\pgfqpoint{0.977186in}{0.826380in}}%
\pgfpathlineto{\pgfqpoint{0.986834in}{0.826885in}}%
\pgfpathlineto{\pgfqpoint{1.001307in}{0.824958in}}%
\pgfpathlineto{\pgfqpoint{1.010956in}{0.824271in}}%
\pgfpathlineto{\pgfqpoint{1.020604in}{0.822063in}}%
\pgfpathlineto{\pgfqpoint{1.030253in}{0.821054in}}%
\pgfpathlineto{\pgfqpoint{1.078495in}{0.805716in}}%
\pgfpathlineto{\pgfqpoint{1.092968in}{0.799235in}}%
\pgfpathlineto{\pgfqpoint{1.102616in}{0.794247in}}%
\pgfpathlineto{\pgfqpoint{1.136386in}{0.774427in}}%
\pgfpathlineto{\pgfqpoint{1.146034in}{0.767681in}}%
\pgfpathlineto{\pgfqpoint{1.179804in}{0.740186in}}%
\pgfpathlineto{\pgfqpoint{1.194277in}{0.726323in}}%
\pgfpathlineto{\pgfqpoint{1.213574in}{0.705652in}}%
\pgfpathlineto{\pgfqpoint{1.242519in}{0.670843in}}%
\pgfpathlineto{\pgfqpoint{1.252168in}{0.663192in}}%
\pgfpathlineto{\pgfqpoint{1.256992in}{0.661757in}}%
\pgfpathlineto{\pgfqpoint{1.266640in}{0.664882in}}%
\pgfpathlineto{\pgfqpoint{1.276289in}{0.673672in}}%
\pgfpathlineto{\pgfqpoint{1.281113in}{0.677953in}}%
\pgfpathlineto{\pgfqpoint{1.290762in}{0.689792in}}%
\pgfpathlineto{\pgfqpoint{1.300410in}{0.698817in}}%
\pgfpathlineto{\pgfqpoint{1.339004in}{0.725812in}}%
\pgfpathlineto{\pgfqpoint{1.363125in}{0.736575in}}%
\pgfpathlineto{\pgfqpoint{1.392071in}{0.744513in}}%
\pgfpathlineto{\pgfqpoint{1.421016in}{0.747401in}}%
\pgfpathlineto{\pgfqpoint{1.430665in}{0.747558in}}%
\pgfpathlineto{\pgfqpoint{1.464434in}{0.741917in}}%
\pgfpathlineto{\pgfqpoint{1.478907in}{0.737528in}}%
\pgfpathlineto{\pgfqpoint{1.503028in}{0.726753in}}%
\pgfpathlineto{\pgfqpoint{1.522325in}{0.715232in}}%
\pgfpathlineto{\pgfqpoint{1.541622in}{0.700214in}}%
\pgfpathlineto{\pgfqpoint{1.556095in}{0.686035in}}%
\pgfpathlineto{\pgfqpoint{1.570568in}{0.670264in}}%
\pgfpathlineto{\pgfqpoint{1.580216in}{0.663145in}}%
\pgfpathlineto{\pgfqpoint{1.585040in}{0.662963in}}%
\pgfpathlineto{\pgfqpoint{1.589865in}{0.664798in}}%
\pgfpathlineto{\pgfqpoint{1.599513in}{0.673300in}}%
\pgfpathlineto{\pgfqpoint{1.628458in}{0.700665in}}%
\pgfpathlineto{\pgfqpoint{1.642931in}{0.709609in}}%
\pgfpathlineto{\pgfqpoint{1.657404in}{0.716853in}}%
\pgfpathlineto{\pgfqpoint{1.676701in}{0.723891in}}%
\pgfpathlineto{\pgfqpoint{1.705646in}{0.728305in}}%
\pgfpathlineto{\pgfqpoint{1.720119in}{0.728722in}}%
\pgfpathlineto{\pgfqpoint{1.749065in}{0.723976in}}%
\pgfpathlineto{\pgfqpoint{1.768361in}{0.717538in}}%
\pgfpathlineto{\pgfqpoint{1.792483in}{0.704929in}}%
\pgfpathlineto{\pgfqpoint{1.806955in}{0.694803in}}%
\pgfpathlineto{\pgfqpoint{1.806955in}{0.694803in}}%
\pgfusepath{stroke}%
\end{pgfscope}%
\begin{pgfscope}%
\pgfpathrectangle{\pgfqpoint{0.526080in}{0.521603in}}{\pgfqpoint{3.720000in}{3.020000in}} %
\pgfusepath{clip}%
\pgfsetrectcap%
\pgfsetroundjoin%
\pgfsetlinewidth{1.505625pt}%
\definecolor{currentstroke}{rgb}{0.190196,0.883910,0.856638}%
\pgfsetstrokecolor{currentstroke}%
\pgfsetdash{}{0pt}%
\pgfpathmoveto{\pgfqpoint{0.744798in}{0.708285in}}%
\pgfpathlineto{\pgfqpoint{0.757859in}{0.724108in}}%
\pgfpathlineto{\pgfqpoint{0.783980in}{0.750243in}}%
\pgfpathlineto{\pgfqpoint{0.797041in}{0.763250in}}%
\pgfpathlineto{\pgfqpoint{0.836223in}{0.798588in}}%
\pgfpathlineto{\pgfqpoint{0.875405in}{0.829926in}}%
\pgfpathlineto{\pgfqpoint{0.908056in}{0.852923in}}%
\pgfpathlineto{\pgfqpoint{0.960299in}{0.884211in}}%
\pgfpathlineto{\pgfqpoint{0.992951in}{0.900192in}}%
\pgfpathlineto{\pgfqpoint{1.012542in}{0.909028in}}%
\pgfpathlineto{\pgfqpoint{1.058254in}{0.926004in}}%
\pgfpathlineto{\pgfqpoint{1.110497in}{0.940668in}}%
\pgfpathlineto{\pgfqpoint{1.175800in}{0.950883in}}%
\pgfpathlineto{\pgfqpoint{1.201921in}{0.953040in}}%
\pgfpathlineto{\pgfqpoint{1.234573in}{0.953763in}}%
\pgfpathlineto{\pgfqpoint{1.273755in}{0.952140in}}%
\pgfpathlineto{\pgfqpoint{1.325997in}{0.945325in}}%
\pgfpathlineto{\pgfqpoint{1.378240in}{0.933396in}}%
\pgfpathlineto{\pgfqpoint{1.410892in}{0.922993in}}%
\pgfpathlineto{\pgfqpoint{1.443543in}{0.910259in}}%
\pgfpathlineto{\pgfqpoint{1.469665in}{0.898947in}}%
\pgfpathlineto{\pgfqpoint{1.521907in}{0.871137in}}%
\pgfpathlineto{\pgfqpoint{1.548028in}{0.854907in}}%
\pgfpathlineto{\pgfqpoint{1.580680in}{0.832365in}}%
\pgfpathlineto{\pgfqpoint{1.619862in}{0.801398in}}%
\pgfpathlineto{\pgfqpoint{1.652514in}{0.772383in}}%
\pgfpathlineto{\pgfqpoint{1.685165in}{0.740136in}}%
\pgfpathlineto{\pgfqpoint{1.711287in}{0.711653in}}%
\pgfpathlineto{\pgfqpoint{1.730878in}{0.688597in}}%
\pgfpathlineto{\pgfqpoint{1.743938in}{0.671236in}}%
\pgfpathlineto{\pgfqpoint{1.756999in}{0.658961in}}%
\pgfpathlineto{\pgfqpoint{1.763529in}{0.658876in}}%
\pgfpathlineto{\pgfqpoint{1.770060in}{0.662803in}}%
\pgfpathlineto{\pgfqpoint{1.783120in}{0.675296in}}%
\pgfpathlineto{\pgfqpoint{1.789651in}{0.682779in}}%
\pgfpathlineto{\pgfqpoint{1.815772in}{0.704433in}}%
\pgfpathlineto{\pgfqpoint{1.828833in}{0.713466in}}%
\pgfpathlineto{\pgfqpoint{1.861484in}{0.734378in}}%
\pgfpathlineto{\pgfqpoint{1.900666in}{0.753385in}}%
\pgfpathlineto{\pgfqpoint{1.952909in}{0.770493in}}%
\pgfpathlineto{\pgfqpoint{1.998621in}{0.777497in}}%
\pgfpathlineto{\pgfqpoint{2.018212in}{0.778813in}}%
\pgfpathlineto{\pgfqpoint{2.044333in}{0.778277in}}%
\pgfpathlineto{\pgfqpoint{2.070455in}{0.775378in}}%
\pgfpathlineto{\pgfqpoint{2.103106in}{0.768883in}}%
\pgfpathlineto{\pgfqpoint{2.142288in}{0.755460in}}%
\pgfpathlineto{\pgfqpoint{2.174940in}{0.739864in}}%
\pgfpathlineto{\pgfqpoint{2.207592in}{0.719312in}}%
\pgfpathlineto{\pgfqpoint{2.214122in}{0.714661in}}%
\pgfpathlineto{\pgfqpoint{2.214122in}{0.714661in}}%
\pgfusepath{stroke}%
\end{pgfscope}%
\begin{pgfscope}%
\pgfpathrectangle{\pgfqpoint{0.526080in}{0.521603in}}{\pgfqpoint{3.720000in}{3.020000in}} %
\pgfusepath{clip}%
\pgfsetrectcap%
\pgfsetroundjoin%
\pgfsetlinewidth{1.505625pt}%
\definecolor{currentstroke}{rgb}{0.052941,0.645928,0.938988}%
\pgfsetstrokecolor{currentstroke}%
\pgfsetdash{}{0pt}%
\pgfpathmoveto{\pgfqpoint{0.765517in}{0.725749in}}%
\pgfpathlineto{\pgfqpoint{0.774637in}{0.731439in}}%
\pgfpathlineto{\pgfqpoint{0.783758in}{0.740022in}}%
\pgfpathlineto{\pgfqpoint{0.792878in}{0.756088in}}%
\pgfpathlineto{\pgfqpoint{0.801998in}{0.765223in}}%
\pgfpathlineto{\pgfqpoint{0.811118in}{0.768890in}}%
\pgfpathlineto{\pgfqpoint{0.829358in}{0.793886in}}%
\pgfpathlineto{\pgfqpoint{0.856719in}{0.817831in}}%
\pgfpathlineto{\pgfqpoint{0.865839in}{0.828125in}}%
\pgfpathlineto{\pgfqpoint{0.893200in}{0.850644in}}%
\pgfpathlineto{\pgfqpoint{0.902320in}{0.860279in}}%
\pgfpathlineto{\pgfqpoint{0.929681in}{0.882298in}}%
\pgfpathlineto{\pgfqpoint{0.938801in}{0.890317in}}%
\pgfpathlineto{\pgfqpoint{0.966161in}{0.910498in}}%
\pgfpathlineto{\pgfqpoint{0.975282in}{0.918399in}}%
\pgfpathlineto{\pgfqpoint{0.993522in}{0.930123in}}%
\pgfpathlineto{\pgfqpoint{1.011762in}{0.944530in}}%
\pgfpathlineto{\pgfqpoint{1.030003in}{0.955428in}}%
\pgfpathlineto{\pgfqpoint{1.048243in}{0.968944in}}%
\pgfpathlineto{\pgfqpoint{1.057363in}{0.973458in}}%
\pgfpathlineto{\pgfqpoint{1.093844in}{0.996167in}}%
\pgfpathlineto{\pgfqpoint{1.157685in}{1.032582in}}%
\pgfpathlineto{\pgfqpoint{1.175926in}{1.041931in}}%
\pgfpathlineto{\pgfqpoint{1.185046in}{1.047155in}}%
\pgfpathlineto{\pgfqpoint{1.203286in}{1.055123in}}%
\pgfpathlineto{\pgfqpoint{1.221527in}{1.064649in}}%
\pgfpathlineto{\pgfqpoint{1.239767in}{1.072101in}}%
\pgfpathlineto{\pgfqpoint{1.258007in}{1.081013in}}%
\pgfpathlineto{\pgfqpoint{1.276248in}{1.088185in}}%
\pgfpathlineto{\pgfqpoint{1.294488in}{1.096227in}}%
\pgfpathlineto{\pgfqpoint{1.312729in}{1.102985in}}%
\pgfpathlineto{\pgfqpoint{1.330969in}{1.110150in}}%
\pgfpathlineto{\pgfqpoint{1.349209in}{1.116795in}}%
\pgfpathlineto{\pgfqpoint{1.367450in}{1.123244in}}%
\pgfpathlineto{\pgfqpoint{1.385690in}{1.129485in}}%
\pgfpathlineto{\pgfqpoint{1.403931in}{1.135139in}}%
\pgfpathlineto{\pgfqpoint{1.504253in}{1.163458in}}%
\pgfpathlineto{\pgfqpoint{1.549854in}{1.173445in}}%
\pgfpathlineto{\pgfqpoint{1.577214in}{1.179267in}}%
\pgfpathlineto{\pgfqpoint{1.704897in}{1.198651in}}%
\pgfpathlineto{\pgfqpoint{1.759618in}{1.203483in}}%
\pgfpathlineto{\pgfqpoint{1.805219in}{1.206492in}}%
\pgfpathlineto{\pgfqpoint{1.850820in}{1.207907in}}%
\pgfpathlineto{\pgfqpoint{1.960262in}{1.206069in}}%
\pgfpathlineto{\pgfqpoint{2.042344in}{1.199675in}}%
\pgfpathlineto{\pgfqpoint{2.087945in}{1.194436in}}%
\pgfpathlineto{\pgfqpoint{2.151786in}{1.184417in}}%
\pgfpathlineto{\pgfqpoint{2.197387in}{1.175560in}}%
\pgfpathlineto{\pgfqpoint{2.270349in}{1.157915in}}%
\pgfpathlineto{\pgfqpoint{2.334190in}{1.138968in}}%
\pgfpathlineto{\pgfqpoint{2.398031in}{1.116352in}}%
\pgfpathlineto{\pgfqpoint{2.452753in}{1.093932in}}%
\pgfpathlineto{\pgfqpoint{2.498354in}{1.073052in}}%
\pgfpathlineto{\pgfqpoint{2.543954in}{1.050164in}}%
\pgfpathlineto{\pgfqpoint{2.598676in}{1.019913in}}%
\pgfpathlineto{\pgfqpoint{2.644277in}{0.992151in}}%
\pgfpathlineto{\pgfqpoint{2.689878in}{0.962114in}}%
\pgfpathlineto{\pgfqpoint{2.735479in}{0.929467in}}%
\pgfpathlineto{\pgfqpoint{2.781079in}{0.894083in}}%
\pgfpathlineto{\pgfqpoint{2.817560in}{0.863512in}}%
\pgfpathlineto{\pgfqpoint{2.817560in}{0.863512in}}%
\pgfusepath{stroke}%
\end{pgfscope}%
\begin{pgfscope}%
\pgfpathrectangle{\pgfqpoint{0.526080in}{0.521603in}}{\pgfqpoint{3.720000in}{3.020000in}} %
\pgfusepath{clip}%
\pgfsetrectcap%
\pgfsetroundjoin%
\pgfsetlinewidth{1.505625pt}%
\definecolor{currentstroke}{rgb}{0.162745,0.505325,0.965124}%
\pgfsetstrokecolor{currentstroke}%
\pgfsetdash{}{0pt}%
\pgfpathmoveto{\pgfqpoint{0.800399in}{0.761511in}}%
\pgfpathlineto{\pgfqpoint{0.810203in}{0.764987in}}%
\pgfpathlineto{\pgfqpoint{0.820007in}{0.780945in}}%
\pgfpathlineto{\pgfqpoint{0.829811in}{0.794287in}}%
\pgfpathlineto{\pgfqpoint{0.839615in}{0.802004in}}%
\pgfpathlineto{\pgfqpoint{0.849419in}{0.811791in}}%
\pgfpathlineto{\pgfqpoint{0.859223in}{0.823782in}}%
\pgfpathlineto{\pgfqpoint{0.869027in}{0.837587in}}%
\pgfpathlineto{\pgfqpoint{0.898439in}{0.865855in}}%
\pgfpathlineto{\pgfqpoint{0.908243in}{0.878546in}}%
\pgfpathlineto{\pgfqpoint{0.927851in}{0.896257in}}%
\pgfpathlineto{\pgfqpoint{0.937655in}{0.904665in}}%
\pgfpathlineto{\pgfqpoint{0.967067in}{0.935846in}}%
\pgfpathlineto{\pgfqpoint{0.976871in}{0.942228in}}%
\pgfpathlineto{\pgfqpoint{1.006283in}{0.971910in}}%
\pgfpathlineto{\pgfqpoint{1.025891in}{0.986199in}}%
\pgfpathlineto{\pgfqpoint{1.055303in}{1.013860in}}%
\pgfpathlineto{\pgfqpoint{1.065107in}{1.019593in}}%
\pgfpathlineto{\pgfqpoint{1.094519in}{1.046329in}}%
\pgfpathlineto{\pgfqpoint{1.114127in}{1.059056in}}%
\pgfpathlineto{\pgfqpoint{1.143539in}{1.083326in}}%
\pgfpathlineto{\pgfqpoint{1.153343in}{1.088905in}}%
\pgfpathlineto{\pgfqpoint{1.182755in}{1.112834in}}%
\pgfpathlineto{\pgfqpoint{1.202363in}{1.124250in}}%
\pgfpathlineto{\pgfqpoint{1.231775in}{1.146346in}}%
\pgfpathlineto{\pgfqpoint{1.241579in}{1.151134in}}%
\pgfpathlineto{\pgfqpoint{1.270991in}{1.173120in}}%
\pgfpathlineto{\pgfqpoint{1.290599in}{1.183479in}}%
\pgfpathlineto{\pgfqpoint{1.320011in}{1.203294in}}%
\pgfpathlineto{\pgfqpoint{1.329815in}{1.208108in}}%
\pgfpathlineto{\pgfqpoint{1.359227in}{1.228050in}}%
\pgfpathlineto{\pgfqpoint{1.378835in}{1.237550in}}%
\pgfpathlineto{\pgfqpoint{1.408247in}{1.255515in}}%
\pgfpathlineto{\pgfqpoint{1.418051in}{1.260192in}}%
\pgfpathlineto{\pgfqpoint{1.457267in}{1.282755in}}%
\pgfpathlineto{\pgfqpoint{1.467071in}{1.287233in}}%
\pgfpathlineto{\pgfqpoint{1.486679in}{1.299804in}}%
\pgfpathlineto{\pgfqpoint{1.506287in}{1.308180in}}%
\pgfpathlineto{\pgfqpoint{1.535699in}{1.325055in}}%
\pgfpathlineto{\pgfqpoint{1.555307in}{1.333158in}}%
\pgfpathlineto{\pgfqpoint{1.574915in}{1.344759in}}%
\pgfpathlineto{\pgfqpoint{1.594523in}{1.352519in}}%
\pgfpathlineto{\pgfqpoint{1.623935in}{1.368077in}}%
\pgfpathlineto{\pgfqpoint{1.643543in}{1.375569in}}%
\pgfpathlineto{\pgfqpoint{1.663151in}{1.386214in}}%
\pgfpathlineto{\pgfqpoint{1.682759in}{1.393519in}}%
\pgfpathlineto{\pgfqpoint{1.712170in}{1.407908in}}%
\pgfpathlineto{\pgfqpoint{1.731778in}{1.415007in}}%
\pgfpathlineto{\pgfqpoint{1.751386in}{1.424629in}}%
\pgfpathlineto{\pgfqpoint{1.770994in}{1.431604in}}%
\pgfpathlineto{\pgfqpoint{1.800406in}{1.444774in}}%
\pgfpathlineto{\pgfqpoint{1.820014in}{1.451641in}}%
\pgfpathlineto{\pgfqpoint{1.839622in}{1.460356in}}%
\pgfpathlineto{\pgfqpoint{1.859230in}{1.466939in}}%
\pgfpathlineto{\pgfqpoint{1.888642in}{1.479018in}}%
\pgfpathlineto{\pgfqpoint{1.908250in}{1.485458in}}%
\pgfpathlineto{\pgfqpoint{1.927858in}{1.493649in}}%
\pgfpathlineto{\pgfqpoint{1.957270in}{1.503654in}}%
\pgfpathlineto{\pgfqpoint{1.976878in}{1.511033in}}%
\pgfpathlineto{\pgfqpoint{1.996486in}{1.517063in}}%
\pgfpathlineto{\pgfqpoint{2.016094in}{1.524758in}}%
\pgfpathlineto{\pgfqpoint{2.045506in}{1.534222in}}%
\pgfpathlineto{\pgfqpoint{2.065114in}{1.540955in}}%
\pgfpathlineto{\pgfqpoint{2.084722in}{1.546758in}}%
\pgfpathlineto{\pgfqpoint{2.104330in}{1.553966in}}%
\pgfpathlineto{\pgfqpoint{2.123938in}{1.559327in}}%
\pgfpathlineto{\pgfqpoint{2.153350in}{1.568890in}}%
\pgfpathlineto{\pgfqpoint{2.172958in}{1.574550in}}%
\pgfpathlineto{\pgfqpoint{2.192566in}{1.581164in}}%
\pgfpathlineto{\pgfqpoint{2.221978in}{1.589536in}}%
\pgfpathlineto{\pgfqpoint{2.241586in}{1.595048in}}%
\pgfpathlineto{\pgfqpoint{2.261194in}{1.600495in}}%
\pgfpathlineto{\pgfqpoint{2.280802in}{1.606501in}}%
\pgfpathlineto{\pgfqpoint{2.300410in}{1.611227in}}%
\pgfpathlineto{\pgfqpoint{2.329822in}{1.619455in}}%
\pgfpathlineto{\pgfqpoint{2.349430in}{1.624726in}}%
\pgfpathlineto{\pgfqpoint{2.369038in}{1.630255in}}%
\pgfpathlineto{\pgfqpoint{2.388646in}{1.634755in}}%
\pgfpathlineto{\pgfqpoint{2.418058in}{1.642236in}}%
\pgfpathlineto{\pgfqpoint{2.437666in}{1.647482in}}%
\pgfpathlineto{\pgfqpoint{2.457274in}{1.652403in}}%
\pgfpathlineto{\pgfqpoint{2.486686in}{1.659319in}}%
\pgfpathlineto{\pgfqpoint{2.545510in}{1.673088in}}%
\pgfpathlineto{\pgfqpoint{2.574922in}{1.679546in}}%
\pgfpathlineto{\pgfqpoint{2.672961in}{1.700252in}}%
\pgfpathlineto{\pgfqpoint{2.692569in}{1.704043in}}%
\pgfpathlineto{\pgfqpoint{2.721981in}{1.710181in}}%
\pgfpathlineto{\pgfqpoint{2.751393in}{1.715924in}}%
\pgfpathlineto{\pgfqpoint{2.986689in}{1.756267in}}%
\pgfpathlineto{\pgfqpoint{2.986689in}{1.756267in}}%
\pgfusepath{stroke}%
\end{pgfscope}%
\begin{pgfscope}%
\pgfpathrectangle{\pgfqpoint{0.526080in}{0.521603in}}{\pgfqpoint{3.720000in}{3.020000in}} %
\pgfusepath{clip}%
\pgfsetrectcap%
\pgfsetroundjoin%
\pgfsetlinewidth{1.505625pt}%
\definecolor{currentstroke}{rgb}{0.205882,0.895163,0.850217}%
\pgfsetstrokecolor{currentstroke}%
\pgfsetdash{}{0pt}%
\pgfpathmoveto{\pgfqpoint{0.748497in}{0.717772in}}%
\pgfpathlineto{\pgfqpoint{0.772472in}{0.742180in}}%
\pgfpathlineto{\pgfqpoint{0.780463in}{0.752145in}}%
\pgfpathlineto{\pgfqpoint{0.812430in}{0.783622in}}%
\pgfpathlineto{\pgfqpoint{0.844396in}{0.814270in}}%
\pgfpathlineto{\pgfqpoint{0.900338in}{0.861806in}}%
\pgfpathlineto{\pgfqpoint{0.924312in}{0.879682in}}%
\pgfpathlineto{\pgfqpoint{0.948287in}{0.897131in}}%
\pgfpathlineto{\pgfqpoint{0.996237in}{0.928027in}}%
\pgfpathlineto{\pgfqpoint{1.020212in}{0.942210in}}%
\pgfpathlineto{\pgfqpoint{1.092136in}{0.977893in}}%
\pgfpathlineto{\pgfqpoint{1.124103in}{0.991031in}}%
\pgfpathlineto{\pgfqpoint{1.172053in}{1.007547in}}%
\pgfpathlineto{\pgfqpoint{1.212011in}{1.018585in}}%
\pgfpathlineto{\pgfqpoint{1.251969in}{1.027127in}}%
\pgfpathlineto{\pgfqpoint{1.283935in}{1.032191in}}%
\pgfpathlineto{\pgfqpoint{1.307910in}{1.035134in}}%
\pgfpathlineto{\pgfqpoint{1.355860in}{1.038303in}}%
\pgfpathlineto{\pgfqpoint{1.395818in}{1.038353in}}%
\pgfpathlineto{\pgfqpoint{1.451759in}{1.034412in}}%
\pgfpathlineto{\pgfqpoint{1.483726in}{1.030096in}}%
\pgfpathlineto{\pgfqpoint{1.523684in}{1.022490in}}%
\pgfpathlineto{\pgfqpoint{1.563642in}{1.012400in}}%
\pgfpathlineto{\pgfqpoint{1.595608in}{1.002565in}}%
\pgfpathlineto{\pgfqpoint{1.635566in}{0.987971in}}%
\pgfpathlineto{\pgfqpoint{1.675524in}{0.970793in}}%
\pgfpathlineto{\pgfqpoint{1.715483in}{0.950854in}}%
\pgfpathlineto{\pgfqpoint{1.747449in}{0.932888in}}%
\pgfpathlineto{\pgfqpoint{1.787407in}{0.907952in}}%
\pgfpathlineto{\pgfqpoint{1.835357in}{0.873977in}}%
\pgfpathlineto{\pgfqpoint{1.875315in}{0.842408in}}%
\pgfpathlineto{\pgfqpoint{1.915273in}{0.807654in}}%
\pgfpathlineto{\pgfqpoint{1.955231in}{0.769834in}}%
\pgfpathlineto{\pgfqpoint{1.995189in}{0.728908in}}%
\pgfpathlineto{\pgfqpoint{2.027156in}{0.693205in}}%
\pgfpathlineto{\pgfqpoint{2.043139in}{0.672822in}}%
\pgfpathlineto{\pgfqpoint{2.051130in}{0.666053in}}%
\pgfpathlineto{\pgfqpoint{2.059122in}{0.662302in}}%
\pgfpathlineto{\pgfqpoint{2.067114in}{0.665870in}}%
\pgfpathlineto{\pgfqpoint{2.083097in}{0.680636in}}%
\pgfpathlineto{\pgfqpoint{2.099080in}{0.696301in}}%
\pgfpathlineto{\pgfqpoint{2.163013in}{0.742319in}}%
\pgfpathlineto{\pgfqpoint{2.234938in}{0.784960in}}%
\pgfpathlineto{\pgfqpoint{2.282887in}{0.808107in}}%
\pgfpathlineto{\pgfqpoint{2.330837in}{0.826743in}}%
\pgfpathlineto{\pgfqpoint{2.370795in}{0.838878in}}%
\pgfpathlineto{\pgfqpoint{2.402762in}{0.846395in}}%
\pgfpathlineto{\pgfqpoint{2.442720in}{0.852612in}}%
\pgfpathlineto{\pgfqpoint{2.474686in}{0.855459in}}%
\pgfpathlineto{\pgfqpoint{2.506653in}{0.856081in}}%
\pgfpathlineto{\pgfqpoint{2.538619in}{0.854610in}}%
\pgfpathlineto{\pgfqpoint{2.546611in}{0.853769in}}%
\pgfpathlineto{\pgfqpoint{2.546611in}{0.853769in}}%
\pgfusepath{stroke}%
\end{pgfscope}%
\begin{pgfscope}%
\pgfpathrectangle{\pgfqpoint{0.526080in}{0.521603in}}{\pgfqpoint{3.720000in}{3.020000in}} %
\pgfusepath{clip}%
\pgfsetrectcap%
\pgfsetroundjoin%
\pgfsetlinewidth{1.505625pt}%
\definecolor{currentstroke}{rgb}{0.076471,0.617278,0.945184}%
\pgfsetstrokecolor{currentstroke}%
\pgfsetdash{}{0pt}%
\pgfpathmoveto{\pgfqpoint{0.782256in}{0.742192in}}%
\pgfpathlineto{\pgfqpoint{0.846327in}{0.812079in}}%
\pgfpathlineto{\pgfqpoint{0.859142in}{0.826979in}}%
\pgfpathlineto{\pgfqpoint{0.871956in}{0.838480in}}%
\pgfpathlineto{\pgfqpoint{0.897585in}{0.865149in}}%
\pgfpathlineto{\pgfqpoint{0.987285in}{0.951969in}}%
\pgfpathlineto{\pgfqpoint{1.000099in}{0.962915in}}%
\pgfpathlineto{\pgfqpoint{1.012914in}{0.975372in}}%
\pgfpathlineto{\pgfqpoint{1.038542in}{0.997564in}}%
\pgfpathlineto{\pgfqpoint{1.076985in}{1.030963in}}%
\pgfpathlineto{\pgfqpoint{1.102614in}{1.051987in}}%
\pgfpathlineto{\pgfqpoint{1.166686in}{1.103037in}}%
\pgfpathlineto{\pgfqpoint{1.217943in}{1.141306in}}%
\pgfpathlineto{\pgfqpoint{1.243572in}{1.160136in}}%
\pgfpathlineto{\pgfqpoint{1.333272in}{1.221676in}}%
\pgfpathlineto{\pgfqpoint{1.448601in}{1.293491in}}%
\pgfpathlineto{\pgfqpoint{1.525487in}{1.336940in}}%
\pgfpathlineto{\pgfqpoint{1.602373in}{1.377355in}}%
\pgfpathlineto{\pgfqpoint{1.653630in}{1.402617in}}%
\pgfpathlineto{\pgfqpoint{1.717702in}{1.432398in}}%
\pgfpathlineto{\pgfqpoint{1.781773in}{1.460222in}}%
\pgfpathlineto{\pgfqpoint{1.858659in}{1.491338in}}%
\pgfpathlineto{\pgfqpoint{1.922731in}{1.515266in}}%
\pgfpathlineto{\pgfqpoint{2.025245in}{1.550177in}}%
\pgfpathlineto{\pgfqpoint{2.140574in}{1.584798in}}%
\pgfpathlineto{\pgfqpoint{2.243089in}{1.611784in}}%
\pgfpathlineto{\pgfqpoint{2.332789in}{1.632405in}}%
\pgfpathlineto{\pgfqpoint{2.435304in}{1.652880in}}%
\pgfpathlineto{\pgfqpoint{2.499375in}{1.664048in}}%
\pgfpathlineto{\pgfqpoint{2.614704in}{1.680916in}}%
\pgfpathlineto{\pgfqpoint{2.742848in}{1.695089in}}%
\pgfpathlineto{\pgfqpoint{2.845362in}{1.702935in}}%
\pgfpathlineto{\pgfqpoint{2.909434in}{1.706336in}}%
\pgfpathlineto{\pgfqpoint{3.011948in}{1.709115in}}%
\pgfpathlineto{\pgfqpoint{3.101649in}{1.709061in}}%
\pgfpathlineto{\pgfqpoint{3.178535in}{1.707003in}}%
\pgfpathlineto{\pgfqpoint{3.293864in}{1.700678in}}%
\pgfpathlineto{\pgfqpoint{3.422007in}{1.688538in}}%
\pgfpathlineto{\pgfqpoint{3.511707in}{1.676963in}}%
\pgfpathlineto{\pgfqpoint{3.588593in}{1.664725in}}%
\pgfpathlineto{\pgfqpoint{3.678293in}{1.647368in}}%
\pgfpathlineto{\pgfqpoint{3.691108in}{1.644637in}}%
\pgfpathlineto{\pgfqpoint{3.691108in}{1.644637in}}%
\pgfusepath{stroke}%
\end{pgfscope}%
\begin{pgfscope}%
\pgfpathrectangle{\pgfqpoint{0.526080in}{0.521603in}}{\pgfqpoint{3.720000in}{3.020000in}} %
\pgfusepath{clip}%
\pgfsetrectcap%
\pgfsetroundjoin%
\pgfsetlinewidth{1.505625pt}%
\definecolor{currentstroke}{rgb}{0.500000,0.000000,1.000000}%
\pgfsetstrokecolor{currentstroke}%
\pgfsetdash{}{0pt}%
\pgfpathmoveto{\pgfqpoint{0.832601in}{0.753162in}}%
\pgfpathlineto{\pgfqpoint{0.879283in}{0.811050in}}%
\pgfpathlineto{\pgfqpoint{0.925965in}{0.856607in}}%
\pgfpathlineto{\pgfqpoint{0.972647in}{0.914881in}}%
\pgfpathlineto{\pgfqpoint{0.995988in}{0.937524in}}%
\pgfpathlineto{\pgfqpoint{1.019328in}{0.958567in}}%
\pgfpathlineto{\pgfqpoint{1.042669in}{0.983309in}}%
\pgfpathlineto{\pgfqpoint{1.066010in}{1.012450in}}%
\pgfpathlineto{\pgfqpoint{1.089351in}{1.039492in}}%
\pgfpathlineto{\pgfqpoint{1.112692in}{1.058564in}}%
\pgfpathlineto{\pgfqpoint{1.136033in}{1.080868in}}%
\pgfpathlineto{\pgfqpoint{1.159374in}{1.109064in}}%
\pgfpathlineto{\pgfqpoint{1.182715in}{1.135396in}}%
\pgfpathlineto{\pgfqpoint{1.229397in}{1.175682in}}%
\pgfpathlineto{\pgfqpoint{1.252738in}{1.199100in}}%
\pgfpathlineto{\pgfqpoint{1.276079in}{1.226815in}}%
\pgfpathlineto{\pgfqpoint{1.299420in}{1.249807in}}%
\pgfpathlineto{\pgfqpoint{1.322760in}{1.268251in}}%
\pgfpathlineto{\pgfqpoint{1.346101in}{1.290082in}}%
\pgfpathlineto{\pgfqpoint{1.369442in}{1.316539in}}%
\pgfpathlineto{\pgfqpoint{1.392783in}{1.340041in}}%
\pgfpathlineto{\pgfqpoint{1.439465in}{1.378393in}}%
\pgfpathlineto{\pgfqpoint{1.462806in}{1.401823in}}%
\pgfpathlineto{\pgfqpoint{1.486147in}{1.426943in}}%
\pgfpathlineto{\pgfqpoint{1.509488in}{1.447407in}}%
\pgfpathlineto{\pgfqpoint{1.532829in}{1.465370in}}%
\pgfpathlineto{\pgfqpoint{1.556170in}{1.487302in}}%
\pgfpathlineto{\pgfqpoint{1.579511in}{1.511462in}}%
\pgfpathlineto{\pgfqpoint{1.602851in}{1.532770in}}%
\pgfpathlineto{\pgfqpoint{1.649533in}{1.570094in}}%
\pgfpathlineto{\pgfqpoint{1.696215in}{1.615608in}}%
\pgfpathlineto{\pgfqpoint{1.742897in}{1.652041in}}%
\pgfpathlineto{\pgfqpoint{1.812920in}{1.716155in}}%
\pgfpathlineto{\pgfqpoint{1.836261in}{1.733677in}}%
\pgfpathlineto{\pgfqpoint{1.859602in}{1.752950in}}%
\pgfpathlineto{\pgfqpoint{1.906283in}{1.795540in}}%
\pgfpathlineto{\pgfqpoint{1.952965in}{1.831130in}}%
\pgfpathlineto{\pgfqpoint{1.999647in}{1.873682in}}%
\pgfpathlineto{\pgfqpoint{2.069670in}{1.928240in}}%
\pgfpathlineto{\pgfqpoint{2.093011in}{1.949564in}}%
\pgfpathlineto{\pgfqpoint{2.116352in}{1.968253in}}%
\pgfpathlineto{\pgfqpoint{2.139693in}{1.985243in}}%
\pgfpathlineto{\pgfqpoint{2.163034in}{2.003919in}}%
\pgfpathlineto{\pgfqpoint{2.209715in}{2.043878in}}%
\pgfpathlineto{\pgfqpoint{2.256397in}{2.078560in}}%
\pgfpathlineto{\pgfqpoint{2.326420in}{2.135011in}}%
\pgfpathlineto{\pgfqpoint{2.349761in}{2.151913in}}%
\pgfpathlineto{\pgfqpoint{2.419784in}{2.208791in}}%
\pgfpathlineto{\pgfqpoint{2.466465in}{2.242998in}}%
\pgfpathlineto{\pgfqpoint{2.513147in}{2.280852in}}%
\pgfpathlineto{\pgfqpoint{2.559829in}{2.314503in}}%
\pgfpathlineto{\pgfqpoint{2.629852in}{2.369525in}}%
\pgfpathlineto{\pgfqpoint{2.676534in}{2.403663in}}%
\pgfpathlineto{\pgfqpoint{2.723216in}{2.439932in}}%
\pgfpathlineto{\pgfqpoint{2.769897in}{2.473465in}}%
\pgfpathlineto{\pgfqpoint{2.816579in}{2.510131in}}%
\pgfpathlineto{\pgfqpoint{2.886602in}{2.561064in}}%
\pgfpathlineto{\pgfqpoint{2.933284in}{2.595799in}}%
\pgfpathlineto{\pgfqpoint{2.979966in}{2.629430in}}%
\pgfpathlineto{\pgfqpoint{3.026648in}{2.664868in}}%
\pgfpathlineto{\pgfqpoint{3.073329in}{2.697892in}}%
\pgfpathlineto{\pgfqpoint{3.143352in}{2.749040in}}%
\pgfpathlineto{\pgfqpoint{3.190034in}{2.782598in}}%
\pgfpathlineto{\pgfqpoint{3.236716in}{2.816807in}}%
\pgfpathlineto{\pgfqpoint{3.306739in}{2.866975in}}%
\pgfpathlineto{\pgfqpoint{3.353420in}{2.899876in}}%
\pgfpathlineto{\pgfqpoint{3.400102in}{2.933290in}}%
\pgfpathlineto{\pgfqpoint{3.470125in}{2.982817in}}%
\pgfpathlineto{\pgfqpoint{3.750216in}{3.179367in}}%
\pgfpathlineto{\pgfqpoint{3.866921in}{3.260256in}}%
\pgfpathlineto{\pgfqpoint{4.076989in}{3.404331in}}%
\pgfpathlineto{\pgfqpoint{4.076989in}{3.404331in}}%
\pgfusepath{stroke}%
\end{pgfscope}%
\begin{pgfscope}%
\pgfsetrectcap%
\pgfsetmiterjoin%
\pgfsetlinewidth{0.803000pt}%
\definecolor{currentstroke}{rgb}{0.000000,0.000000,0.000000}%
\pgfsetstrokecolor{currentstroke}%
\pgfsetdash{}{0pt}%
\pgfpathmoveto{\pgfqpoint{0.526080in}{0.521603in}}%
\pgfpathlineto{\pgfqpoint{0.526080in}{3.541603in}}%
\pgfusepath{stroke}%
\end{pgfscope}%
\begin{pgfscope}%
\pgfsetrectcap%
\pgfsetmiterjoin%
\pgfsetlinewidth{0.803000pt}%
\definecolor{currentstroke}{rgb}{0.000000,0.000000,0.000000}%
\pgfsetstrokecolor{currentstroke}%
\pgfsetdash{}{0pt}%
\pgfpathmoveto{\pgfqpoint{4.246080in}{0.521603in}}%
\pgfpathlineto{\pgfqpoint{4.246080in}{3.541603in}}%
\pgfusepath{stroke}%
\end{pgfscope}%
\begin{pgfscope}%
\pgfsetrectcap%
\pgfsetmiterjoin%
\pgfsetlinewidth{0.803000pt}%
\definecolor{currentstroke}{rgb}{0.000000,0.000000,0.000000}%
\pgfsetstrokecolor{currentstroke}%
\pgfsetdash{}{0pt}%
\pgfpathmoveto{\pgfqpoint{0.526080in}{0.521603in}}%
\pgfpathlineto{\pgfqpoint{4.246080in}{0.521603in}}%
\pgfusepath{stroke}%
\end{pgfscope}%
\begin{pgfscope}%
\pgfsetrectcap%
\pgfsetmiterjoin%
\pgfsetlinewidth{0.803000pt}%
\definecolor{currentstroke}{rgb}{0.000000,0.000000,0.000000}%
\pgfsetstrokecolor{currentstroke}%
\pgfsetdash{}{0pt}%
\pgfpathmoveto{\pgfqpoint{0.526080in}{3.541603in}}%
\pgfpathlineto{\pgfqpoint{4.246080in}{3.541603in}}%
\pgfusepath{stroke}%
\end{pgfscope}%
\begin{pgfscope}%
\pgfpathrectangle{\pgfqpoint{4.478580in}{0.521603in}}{\pgfqpoint{0.151000in}{3.020000in}} %
\pgfusepath{clip}%
\pgfsetbuttcap%
\pgfsetmiterjoin%
\definecolor{currentfill}{rgb}{1.000000,1.000000,1.000000}%
\pgfsetfillcolor{currentfill}%
\pgfsetlinewidth{0.010037pt}%
\definecolor{currentstroke}{rgb}{1.000000,1.000000,1.000000}%
\pgfsetstrokecolor{currentstroke}%
\pgfsetdash{}{0pt}%
\pgfpathmoveto{\pgfqpoint{4.478580in}{0.521603in}}%
\pgfpathlineto{\pgfqpoint{4.478580in}{0.533400in}}%
\pgfpathlineto{\pgfqpoint{4.478580in}{3.529806in}}%
\pgfpathlineto{\pgfqpoint{4.478580in}{3.541603in}}%
\pgfpathlineto{\pgfqpoint{4.629580in}{3.541603in}}%
\pgfpathlineto{\pgfqpoint{4.629580in}{3.529806in}}%
\pgfpathlineto{\pgfqpoint{4.629580in}{0.533400in}}%
\pgfpathlineto{\pgfqpoint{4.629580in}{0.521603in}}%
\pgfpathclose%
\pgfusepath{stroke,fill}%
\end{pgfscope}%
\begin{pgfscope}%
\pgfsys@transformshift{4.480000in}{0.526603in}%
\pgftext[left,bottom]{\pgfimage[interpolate=true,width=0.150000in,height=3.020000in]{series_m2_ds-img0.png}}%
\end{pgfscope}%
\begin{pgfscope}%
\pgfsetbuttcap%
\pgfsetroundjoin%
\definecolor{currentfill}{rgb}{0.000000,0.000000,0.000000}%
\pgfsetfillcolor{currentfill}%
\pgfsetlinewidth{0.803000pt}%
\definecolor{currentstroke}{rgb}{0.000000,0.000000,0.000000}%
\pgfsetstrokecolor{currentstroke}%
\pgfsetdash{}{0pt}%
\pgfsys@defobject{currentmarker}{\pgfqpoint{0.000000in}{0.000000in}}{\pgfqpoint{0.048611in}{0.000000in}}{%
\pgfpathmoveto{\pgfqpoint{0.000000in}{0.000000in}}%
\pgfpathlineto{\pgfqpoint{0.048611in}{0.000000in}}%
\pgfusepath{stroke,fill}%
}%
\begin{pgfscope}%
\pgfsys@transformshift{4.629580in}{0.617964in}%
\pgfsys@useobject{currentmarker}{}%
\end{pgfscope}%
\end{pgfscope}%
\begin{pgfscope}%
\pgfsetbuttcap%
\pgfsetroundjoin%
\definecolor{currentfill}{rgb}{0.000000,0.000000,0.000000}%
\pgfsetfillcolor{currentfill}%
\pgfsetlinewidth{0.803000pt}%
\definecolor{currentstroke}{rgb}{0.000000,0.000000,0.000000}%
\pgfsetstrokecolor{currentstroke}%
\pgfsetdash{}{0pt}%
\pgfsys@defobject{currentmarker}{\pgfqpoint{0.000000in}{0.000000in}}{\pgfqpoint{0.048611in}{0.000000in}}{%
\pgfpathmoveto{\pgfqpoint{0.000000in}{0.000000in}}%
\pgfpathlineto{\pgfqpoint{0.048611in}{0.000000in}}%
\pgfusepath{stroke,fill}%
}%
\begin{pgfscope}%
\pgfsys@transformshift{4.629580in}{0.796036in}%
\pgfsys@useobject{currentmarker}{}%
\end{pgfscope}%
\end{pgfscope}%
\begin{pgfscope}%
\pgfsetbuttcap%
\pgfsetroundjoin%
\definecolor{currentfill}{rgb}{0.000000,0.000000,0.000000}%
\pgfsetfillcolor{currentfill}%
\pgfsetlinewidth{0.803000pt}%
\definecolor{currentstroke}{rgb}{0.000000,0.000000,0.000000}%
\pgfsetstrokecolor{currentstroke}%
\pgfsetdash{}{0pt}%
\pgfsys@defobject{currentmarker}{\pgfqpoint{0.000000in}{0.000000in}}{\pgfqpoint{0.048611in}{0.000000in}}{%
\pgfpathmoveto{\pgfqpoint{0.000000in}{0.000000in}}%
\pgfpathlineto{\pgfqpoint{0.048611in}{0.000000in}}%
\pgfusepath{stroke,fill}%
}%
\begin{pgfscope}%
\pgfsys@transformshift{4.629580in}{0.941531in}%
\pgfsys@useobject{currentmarker}{}%
\end{pgfscope}%
\end{pgfscope}%
\begin{pgfscope}%
\pgfsetbuttcap%
\pgfsetroundjoin%
\definecolor{currentfill}{rgb}{0.000000,0.000000,0.000000}%
\pgfsetfillcolor{currentfill}%
\pgfsetlinewidth{0.803000pt}%
\definecolor{currentstroke}{rgb}{0.000000,0.000000,0.000000}%
\pgfsetstrokecolor{currentstroke}%
\pgfsetdash{}{0pt}%
\pgfsys@defobject{currentmarker}{\pgfqpoint{0.000000in}{0.000000in}}{\pgfqpoint{0.048611in}{0.000000in}}{%
\pgfpathmoveto{\pgfqpoint{0.000000in}{0.000000in}}%
\pgfpathlineto{\pgfqpoint{0.048611in}{0.000000in}}%
\pgfusepath{stroke,fill}%
}%
\begin{pgfscope}%
\pgfsys@transformshift{4.629580in}{1.064545in}%
\pgfsys@useobject{currentmarker}{}%
\end{pgfscope}%
\end{pgfscope}%
\begin{pgfscope}%
\pgfsetbuttcap%
\pgfsetroundjoin%
\definecolor{currentfill}{rgb}{0.000000,0.000000,0.000000}%
\pgfsetfillcolor{currentfill}%
\pgfsetlinewidth{0.803000pt}%
\definecolor{currentstroke}{rgb}{0.000000,0.000000,0.000000}%
\pgfsetstrokecolor{currentstroke}%
\pgfsetdash{}{0pt}%
\pgfsys@defobject{currentmarker}{\pgfqpoint{0.000000in}{0.000000in}}{\pgfqpoint{0.048611in}{0.000000in}}{%
\pgfpathmoveto{\pgfqpoint{0.000000in}{0.000000in}}%
\pgfpathlineto{\pgfqpoint{0.048611in}{0.000000in}}%
\pgfusepath{stroke,fill}%
}%
\begin{pgfscope}%
\pgfsys@transformshift{4.629580in}{1.171104in}%
\pgfsys@useobject{currentmarker}{}%
\end{pgfscope}%
\end{pgfscope}%
\begin{pgfscope}%
\pgfsetbuttcap%
\pgfsetroundjoin%
\definecolor{currentfill}{rgb}{0.000000,0.000000,0.000000}%
\pgfsetfillcolor{currentfill}%
\pgfsetlinewidth{0.803000pt}%
\definecolor{currentstroke}{rgb}{0.000000,0.000000,0.000000}%
\pgfsetstrokecolor{currentstroke}%
\pgfsetdash{}{0pt}%
\pgfsys@defobject{currentmarker}{\pgfqpoint{0.000000in}{0.000000in}}{\pgfqpoint{0.048611in}{0.000000in}}{%
\pgfpathmoveto{\pgfqpoint{0.000000in}{0.000000in}}%
\pgfpathlineto{\pgfqpoint{0.048611in}{0.000000in}}%
\pgfusepath{stroke,fill}%
}%
\begin{pgfscope}%
\pgfsys@transformshift{4.629580in}{1.265097in}%
\pgfsys@useobject{currentmarker}{}%
\end{pgfscope}%
\end{pgfscope}%
\begin{pgfscope}%
\pgfsetbuttcap%
\pgfsetroundjoin%
\definecolor{currentfill}{rgb}{0.000000,0.000000,0.000000}%
\pgfsetfillcolor{currentfill}%
\pgfsetlinewidth{0.803000pt}%
\definecolor{currentstroke}{rgb}{0.000000,0.000000,0.000000}%
\pgfsetstrokecolor{currentstroke}%
\pgfsetdash{}{0pt}%
\pgfsys@defobject{currentmarker}{\pgfqpoint{0.000000in}{0.000000in}}{\pgfqpoint{0.048611in}{0.000000in}}{%
\pgfpathmoveto{\pgfqpoint{0.000000in}{0.000000in}}%
\pgfpathlineto{\pgfqpoint{0.048611in}{0.000000in}}%
\pgfusepath{stroke,fill}%
}%
\begin{pgfscope}%
\pgfsys@transformshift{4.629580in}{1.349176in}%
\pgfsys@useobject{currentmarker}{}%
\end{pgfscope}%
\end{pgfscope}%
\begin{pgfscope}%
\pgftext[x=4.726802in,y=1.296414in,left,base]{\rmfamily\fontsize{10.000000}{12.000000}\selectfont \(\displaystyle 10^{-1}\)}%
\end{pgfscope}%
\begin{pgfscope}%
\pgfsetbuttcap%
\pgfsetroundjoin%
\definecolor{currentfill}{rgb}{0.000000,0.000000,0.000000}%
\pgfsetfillcolor{currentfill}%
\pgfsetlinewidth{0.803000pt}%
\definecolor{currentstroke}{rgb}{0.000000,0.000000,0.000000}%
\pgfsetstrokecolor{currentstroke}%
\pgfsetdash{}{0pt}%
\pgfsys@defobject{currentmarker}{\pgfqpoint{0.000000in}{0.000000in}}{\pgfqpoint{0.048611in}{0.000000in}}{%
\pgfpathmoveto{\pgfqpoint{0.000000in}{0.000000in}}%
\pgfpathlineto{\pgfqpoint{0.048611in}{0.000000in}}%
\pgfusepath{stroke,fill}%
}%
\begin{pgfscope}%
\pgfsys@transformshift{4.629580in}{1.902316in}%
\pgfsys@useobject{currentmarker}{}%
\end{pgfscope}%
\end{pgfscope}%
\begin{pgfscope}%
\pgfsetbuttcap%
\pgfsetroundjoin%
\definecolor{currentfill}{rgb}{0.000000,0.000000,0.000000}%
\pgfsetfillcolor{currentfill}%
\pgfsetlinewidth{0.803000pt}%
\definecolor{currentstroke}{rgb}{0.000000,0.000000,0.000000}%
\pgfsetstrokecolor{currentstroke}%
\pgfsetdash{}{0pt}%
\pgfsys@defobject{currentmarker}{\pgfqpoint{0.000000in}{0.000000in}}{\pgfqpoint{0.048611in}{0.000000in}}{%
\pgfpathmoveto{\pgfqpoint{0.000000in}{0.000000in}}%
\pgfpathlineto{\pgfqpoint{0.048611in}{0.000000in}}%
\pgfusepath{stroke,fill}%
}%
\begin{pgfscope}%
\pgfsys@transformshift{4.629580in}{2.225882in}%
\pgfsys@useobject{currentmarker}{}%
\end{pgfscope}%
\end{pgfscope}%
\begin{pgfscope}%
\pgfsetbuttcap%
\pgfsetroundjoin%
\definecolor{currentfill}{rgb}{0.000000,0.000000,0.000000}%
\pgfsetfillcolor{currentfill}%
\pgfsetlinewidth{0.803000pt}%
\definecolor{currentstroke}{rgb}{0.000000,0.000000,0.000000}%
\pgfsetstrokecolor{currentstroke}%
\pgfsetdash{}{0pt}%
\pgfsys@defobject{currentmarker}{\pgfqpoint{0.000000in}{0.000000in}}{\pgfqpoint{0.048611in}{0.000000in}}{%
\pgfpathmoveto{\pgfqpoint{0.000000in}{0.000000in}}%
\pgfpathlineto{\pgfqpoint{0.048611in}{0.000000in}}%
\pgfusepath{stroke,fill}%
}%
\begin{pgfscope}%
\pgfsys@transformshift{4.629580in}{2.455456in}%
\pgfsys@useobject{currentmarker}{}%
\end{pgfscope}%
\end{pgfscope}%
\begin{pgfscope}%
\pgfsetbuttcap%
\pgfsetroundjoin%
\definecolor{currentfill}{rgb}{0.000000,0.000000,0.000000}%
\pgfsetfillcolor{currentfill}%
\pgfsetlinewidth{0.803000pt}%
\definecolor{currentstroke}{rgb}{0.000000,0.000000,0.000000}%
\pgfsetstrokecolor{currentstroke}%
\pgfsetdash{}{0pt}%
\pgfsys@defobject{currentmarker}{\pgfqpoint{0.000000in}{0.000000in}}{\pgfqpoint{0.048611in}{0.000000in}}{%
\pgfpathmoveto{\pgfqpoint{0.000000in}{0.000000in}}%
\pgfpathlineto{\pgfqpoint{0.048611in}{0.000000in}}%
\pgfusepath{stroke,fill}%
}%
\begin{pgfscope}%
\pgfsys@transformshift{4.629580in}{2.633527in}%
\pgfsys@useobject{currentmarker}{}%
\end{pgfscope}%
\end{pgfscope}%
\begin{pgfscope}%
\pgfsetbuttcap%
\pgfsetroundjoin%
\definecolor{currentfill}{rgb}{0.000000,0.000000,0.000000}%
\pgfsetfillcolor{currentfill}%
\pgfsetlinewidth{0.803000pt}%
\definecolor{currentstroke}{rgb}{0.000000,0.000000,0.000000}%
\pgfsetstrokecolor{currentstroke}%
\pgfsetdash{}{0pt}%
\pgfsys@defobject{currentmarker}{\pgfqpoint{0.000000in}{0.000000in}}{\pgfqpoint{0.048611in}{0.000000in}}{%
\pgfpathmoveto{\pgfqpoint{0.000000in}{0.000000in}}%
\pgfpathlineto{\pgfqpoint{0.048611in}{0.000000in}}%
\pgfusepath{stroke,fill}%
}%
\begin{pgfscope}%
\pgfsys@transformshift{4.629580in}{2.779022in}%
\pgfsys@useobject{currentmarker}{}%
\end{pgfscope}%
\end{pgfscope}%
\begin{pgfscope}%
\pgfsetbuttcap%
\pgfsetroundjoin%
\definecolor{currentfill}{rgb}{0.000000,0.000000,0.000000}%
\pgfsetfillcolor{currentfill}%
\pgfsetlinewidth{0.803000pt}%
\definecolor{currentstroke}{rgb}{0.000000,0.000000,0.000000}%
\pgfsetstrokecolor{currentstroke}%
\pgfsetdash{}{0pt}%
\pgfsys@defobject{currentmarker}{\pgfqpoint{0.000000in}{0.000000in}}{\pgfqpoint{0.048611in}{0.000000in}}{%
\pgfpathmoveto{\pgfqpoint{0.000000in}{0.000000in}}%
\pgfpathlineto{\pgfqpoint{0.048611in}{0.000000in}}%
\pgfusepath{stroke,fill}%
}%
\begin{pgfscope}%
\pgfsys@transformshift{4.629580in}{2.902036in}%
\pgfsys@useobject{currentmarker}{}%
\end{pgfscope}%
\end{pgfscope}%
\begin{pgfscope}%
\pgfsetbuttcap%
\pgfsetroundjoin%
\definecolor{currentfill}{rgb}{0.000000,0.000000,0.000000}%
\pgfsetfillcolor{currentfill}%
\pgfsetlinewidth{0.803000pt}%
\definecolor{currentstroke}{rgb}{0.000000,0.000000,0.000000}%
\pgfsetstrokecolor{currentstroke}%
\pgfsetdash{}{0pt}%
\pgfsys@defobject{currentmarker}{\pgfqpoint{0.000000in}{0.000000in}}{\pgfqpoint{0.048611in}{0.000000in}}{%
\pgfpathmoveto{\pgfqpoint{0.000000in}{0.000000in}}%
\pgfpathlineto{\pgfqpoint{0.048611in}{0.000000in}}%
\pgfusepath{stroke,fill}%
}%
\begin{pgfscope}%
\pgfsys@transformshift{4.629580in}{3.008596in}%
\pgfsys@useobject{currentmarker}{}%
\end{pgfscope}%
\end{pgfscope}%
\begin{pgfscope}%
\pgfsetbuttcap%
\pgfsetroundjoin%
\definecolor{currentfill}{rgb}{0.000000,0.000000,0.000000}%
\pgfsetfillcolor{currentfill}%
\pgfsetlinewidth{0.803000pt}%
\definecolor{currentstroke}{rgb}{0.000000,0.000000,0.000000}%
\pgfsetstrokecolor{currentstroke}%
\pgfsetdash{}{0pt}%
\pgfsys@defobject{currentmarker}{\pgfqpoint{0.000000in}{0.000000in}}{\pgfqpoint{0.048611in}{0.000000in}}{%
\pgfpathmoveto{\pgfqpoint{0.000000in}{0.000000in}}%
\pgfpathlineto{\pgfqpoint{0.048611in}{0.000000in}}%
\pgfusepath{stroke,fill}%
}%
\begin{pgfscope}%
\pgfsys@transformshift{4.629580in}{3.102588in}%
\pgfsys@useobject{currentmarker}{}%
\end{pgfscope}%
\end{pgfscope}%
\begin{pgfscope}%
\pgfsetbuttcap%
\pgfsetroundjoin%
\definecolor{currentfill}{rgb}{0.000000,0.000000,0.000000}%
\pgfsetfillcolor{currentfill}%
\pgfsetlinewidth{0.803000pt}%
\definecolor{currentstroke}{rgb}{0.000000,0.000000,0.000000}%
\pgfsetstrokecolor{currentstroke}%
\pgfsetdash{}{0pt}%
\pgfsys@defobject{currentmarker}{\pgfqpoint{0.000000in}{0.000000in}}{\pgfqpoint{0.048611in}{0.000000in}}{%
\pgfpathmoveto{\pgfqpoint{0.000000in}{0.000000in}}%
\pgfpathlineto{\pgfqpoint{0.048611in}{0.000000in}}%
\pgfusepath{stroke,fill}%
}%
\begin{pgfscope}%
\pgfsys@transformshift{4.629580in}{3.186667in}%
\pgfsys@useobject{currentmarker}{}%
\end{pgfscope}%
\end{pgfscope}%
\begin{pgfscope}%
\pgftext[x=4.726802in,y=3.133906in,left,base]{\rmfamily\fontsize{10.000000}{12.000000}\selectfont \(\displaystyle 10^{0}\)}%
\end{pgfscope}%
\begin{pgfscope}%
\pgftext[x=5.209249in,y=2.031603in,,top]{\rmfamily\fontsize{12.000000}{14.400000}\selectfont \(\displaystyle {\mathbf{E} \mbox{u}}\)}%
\end{pgfscope}%
\begin{pgfscope}%
\pgfsetbuttcap%
\pgfsetmiterjoin%
\pgfsetlinewidth{0.803000pt}%
\definecolor{currentstroke}{rgb}{0.000000,0.000000,0.000000}%
\pgfsetstrokecolor{currentstroke}%
\pgfsetdash{}{0pt}%
\pgfpathmoveto{\pgfqpoint{4.478580in}{0.521603in}}%
\pgfpathlineto{\pgfqpoint{4.478580in}{0.533400in}}%
\pgfpathlineto{\pgfqpoint{4.478580in}{3.529806in}}%
\pgfpathlineto{\pgfqpoint{4.478580in}{3.541603in}}%
\pgfpathlineto{\pgfqpoint{4.629580in}{3.541603in}}%
\pgfpathlineto{\pgfqpoint{4.629580in}{3.529806in}}%
\pgfpathlineto{\pgfqpoint{4.629580in}{0.533400in}}%
\pgfpathlineto{\pgfqpoint{4.629580in}{0.521603in}}%
\pgfpathclose%
\pgfusepath{stroke}%
\end{pgfscope}%
\end{pgfpicture}%
\makeatother%
\endgroup%

    \caption{.\label{fig:dnumbs}}
\end{figure}
We should note that there are several kinds of systematic error that influence our data. We assume that droplet translate purely along the central axis of the electric field, but in practice, despite the improvement in surface charge density uniformity produced by corona charging, there are still local areas of especially high charge density. In principle, this kind of error should become small for droplets which are far enough away from the charge distribution, that the geometry of the charge distribution disappears, and the electric field looks like that due to a point charge. Another form of error is in the initial velocity as it appears in $\mathbb{E}\mbox{u}$. Because we usually lose the first few frames of video due to camera shake transients at the start of the low-gravity experiment, we will consistently underestimate $U_0$ because the droplet will already have decelerated significantly during that period of time. The primary sources of random error are the effect of contact line hysteresis on the droplet initial velocity, and of the variance in the MLE parameter estimates.
\newpage

\section{Impact Dynamics \hl{[placeholder section]}}
Using the unique capabilities of the low-gravity environment we obtained data on dimensionless contact time and coefficients of restitution at very low Ohnesorge numbers for a range of electric Bond numbers. Despite strong electric fields (20-30 $kV/cm$) we found little evidence for wetting transitions due to excession of a critical pressure (the ``Fakir impalement''). There is no obvious trend in dimensionless contact time or coefficient of restitution with electric Bond number.

Jump velocities are more strongly damped for relatively small droplet volumes in the presence of the electric fields than was shown by Attari \emph{et. al.}. This may be evidence for electrowetting paradoxically enhancing the effect of contact angle hysteresis pinning on sharp corners. (How does this tie into the coefficients of restitution problem?)

\begin{figure}[htb]
    \centering
    %% Creator: Matplotlib, PGF backend
%%
%% To include the figure in your LaTeX document, write
%%   \input{<filename>.pgf}
%%
%% Make sure the required packages are loaded in your preamble
%%   \usepackage{pgf}
%%
%% Figures using additional raster images can only be included by \input if
%% they are in the same directory as the main LaTeX file. For loading figures
%% from other directories you can use the `import` package
%%   \usepackage{import}
%% and then include the figures with
%%   \import{<path to file>}{<filename>.pgf}
%%
%% Matplotlib used the following preamble
%%   \usepackage{fontspec}
%%   \setmainfont{DejaVu Serif}
%%   \setsansfont{DejaVu Sans}
%%   \setmonofont{DejaVu Sans Mono}
%%
\begingroup%
\makeatletter%
\begin{pgfpicture}%
\pgfpathrectangle{\pgfpointorigin}{\pgfqpoint{5.329166in}{3.676603in}}%
\pgfusepath{use as bounding box, clip}%
\begin{pgfscope}%
\pgfsetbuttcap%
\pgfsetmiterjoin%
\definecolor{currentfill}{rgb}{1.000000,1.000000,1.000000}%
\pgfsetfillcolor{currentfill}%
\pgfsetlinewidth{0.000000pt}%
\definecolor{currentstroke}{rgb}{1.000000,1.000000,1.000000}%
\pgfsetstrokecolor{currentstroke}%
\pgfsetdash{}{0pt}%
\pgfpathmoveto{\pgfqpoint{-0.000000in}{0.000000in}}%
\pgfpathlineto{\pgfqpoint{5.329166in}{0.000000in}}%
\pgfpathlineto{\pgfqpoint{5.329166in}{3.676603in}}%
\pgfpathlineto{\pgfqpoint{-0.000000in}{3.676603in}}%
\pgfpathclose%
\pgfusepath{fill}%
\end{pgfscope}%
\begin{pgfscope}%
\pgfsetbuttcap%
\pgfsetmiterjoin%
\definecolor{currentfill}{rgb}{1.000000,1.000000,1.000000}%
\pgfsetfillcolor{currentfill}%
\pgfsetlinewidth{0.000000pt}%
\definecolor{currentstroke}{rgb}{0.000000,0.000000,0.000000}%
\pgfsetstrokecolor{currentstroke}%
\pgfsetstrokeopacity{0.000000}%
\pgfsetdash{}{0pt}%
\pgfpathmoveto{\pgfqpoint{0.466126in}{0.521603in}}%
\pgfpathlineto{\pgfqpoint{4.186126in}{0.521603in}}%
\pgfpathlineto{\pgfqpoint{4.186126in}{3.541603in}}%
\pgfpathlineto{\pgfqpoint{0.466126in}{3.541603in}}%
\pgfpathclose%
\pgfusepath{fill}%
\end{pgfscope}%
\begin{pgfscope}%
\pgfpathrectangle{\pgfqpoint{0.466126in}{0.521603in}}{\pgfqpoint{3.720000in}{3.020000in}} %
\pgfusepath{clip}%
\pgfsetbuttcap%
\pgfsetroundjoin%
\definecolor{currentfill}{rgb}{1.000000,0.255843,0.128999}%
\pgfsetfillcolor{currentfill}%
\pgfsetlinewidth{1.003750pt}%
\definecolor{currentstroke}{rgb}{1.000000,0.255843,0.128999}%
\pgfsetstrokecolor{currentstroke}%
\pgfsetdash{}{0pt}%
\pgfpathmoveto{\pgfqpoint{2.577334in}{1.175315in}}%
\pgfpathcurveto{\pgfqpoint{2.588384in}{1.175315in}}{\pgfqpoint{2.598983in}{1.179706in}}{\pgfqpoint{2.606797in}{1.187519in}}%
\pgfpathcurveto{\pgfqpoint{2.614611in}{1.195333in}}{\pgfqpoint{2.619001in}{1.205932in}}{\pgfqpoint{2.619001in}{1.216982in}}%
\pgfpathcurveto{\pgfqpoint{2.619001in}{1.228032in}}{\pgfqpoint{2.614611in}{1.238631in}}{\pgfqpoint{2.606797in}{1.246445in}}%
\pgfpathcurveto{\pgfqpoint{2.598983in}{1.254259in}}{\pgfqpoint{2.588384in}{1.258649in}}{\pgfqpoint{2.577334in}{1.258649in}}%
\pgfpathcurveto{\pgfqpoint{2.566284in}{1.258649in}}{\pgfqpoint{2.555685in}{1.254259in}}{\pgfqpoint{2.547872in}{1.246445in}}%
\pgfpathcurveto{\pgfqpoint{2.540058in}{1.238631in}}{\pgfqpoint{2.535668in}{1.228032in}}{\pgfqpoint{2.535668in}{1.216982in}}%
\pgfpathcurveto{\pgfqpoint{2.535668in}{1.205932in}}{\pgfqpoint{2.540058in}{1.195333in}}{\pgfqpoint{2.547872in}{1.187519in}}%
\pgfpathcurveto{\pgfqpoint{2.555685in}{1.179706in}}{\pgfqpoint{2.566284in}{1.175315in}}{\pgfqpoint{2.577334in}{1.175315in}}%
\pgfpathclose%
\pgfusepath{stroke,fill}%
\end{pgfscope}%
\begin{pgfscope}%
\pgfpathrectangle{\pgfqpoint{0.466126in}{0.521603in}}{\pgfqpoint{3.720000in}{3.020000in}} %
\pgfusepath{clip}%
\pgfsetbuttcap%
\pgfsetroundjoin%
\definecolor{currentfill}{rgb}{0.319608,0.279583,0.989980}%
\pgfsetfillcolor{currentfill}%
\pgfsetlinewidth{1.003750pt}%
\definecolor{currentstroke}{rgb}{0.319608,0.279583,0.989980}%
\pgfsetstrokecolor{currentstroke}%
\pgfsetdash{}{0pt}%
\pgfpathmoveto{\pgfqpoint{2.622359in}{1.378029in}}%
\pgfpathcurveto{\pgfqpoint{2.633409in}{1.378029in}}{\pgfqpoint{2.644008in}{1.382419in}}{\pgfqpoint{2.651822in}{1.390233in}}%
\pgfpathcurveto{\pgfqpoint{2.659636in}{1.398047in}}{\pgfqpoint{2.664026in}{1.408646in}}{\pgfqpoint{2.664026in}{1.419696in}}%
\pgfpathcurveto{\pgfqpoint{2.664026in}{1.430746in}}{\pgfqpoint{2.659636in}{1.441345in}}{\pgfqpoint{2.651822in}{1.449158in}}%
\pgfpathcurveto{\pgfqpoint{2.644008in}{1.456972in}}{\pgfqpoint{2.633409in}{1.461362in}}{\pgfqpoint{2.622359in}{1.461362in}}%
\pgfpathcurveto{\pgfqpoint{2.611309in}{1.461362in}}{\pgfqpoint{2.600710in}{1.456972in}}{\pgfqpoint{2.592896in}{1.449158in}}%
\pgfpathcurveto{\pgfqpoint{2.585083in}{1.441345in}}{\pgfqpoint{2.580692in}{1.430746in}}{\pgfqpoint{2.580692in}{1.419696in}}%
\pgfpathcurveto{\pgfqpoint{2.580692in}{1.408646in}}{\pgfqpoint{2.585083in}{1.398047in}}{\pgfqpoint{2.592896in}{1.390233in}}%
\pgfpathcurveto{\pgfqpoint{2.600710in}{1.382419in}}{\pgfqpoint{2.611309in}{1.378029in}}{\pgfqpoint{2.622359in}{1.378029in}}%
\pgfpathclose%
\pgfusepath{stroke,fill}%
\end{pgfscope}%
\begin{pgfscope}%
\pgfpathrectangle{\pgfqpoint{0.466126in}{0.521603in}}{\pgfqpoint{3.720000in}{3.020000in}} %
\pgfusepath{clip}%
\pgfsetbuttcap%
\pgfsetroundjoin%
\definecolor{currentfill}{rgb}{0.460784,0.061561,0.999526}%
\pgfsetfillcolor{currentfill}%
\pgfsetlinewidth{1.003750pt}%
\definecolor{currentstroke}{rgb}{0.460784,0.061561,0.999526}%
\pgfsetstrokecolor{currentstroke}%
\pgfsetdash{}{0pt}%
\pgfpathmoveto{\pgfqpoint{1.993095in}{1.931016in}}%
\pgfpathcurveto{\pgfqpoint{2.004145in}{1.931016in}}{\pgfqpoint{2.014744in}{1.935406in}}{\pgfqpoint{2.022557in}{1.943220in}}%
\pgfpathcurveto{\pgfqpoint{2.030371in}{1.951033in}}{\pgfqpoint{2.034761in}{1.961632in}}{\pgfqpoint{2.034761in}{1.972682in}}%
\pgfpathcurveto{\pgfqpoint{2.034761in}{1.983733in}}{\pgfqpoint{2.030371in}{1.994332in}}{\pgfqpoint{2.022557in}{2.002145in}}%
\pgfpathcurveto{\pgfqpoint{2.014744in}{2.009959in}}{\pgfqpoint{2.004145in}{2.014349in}}{\pgfqpoint{1.993095in}{2.014349in}}%
\pgfpathcurveto{\pgfqpoint{1.982045in}{2.014349in}}{\pgfqpoint{1.971446in}{2.009959in}}{\pgfqpoint{1.963632in}{2.002145in}}%
\pgfpathcurveto{\pgfqpoint{1.955818in}{1.994332in}}{\pgfqpoint{1.951428in}{1.983733in}}{\pgfqpoint{1.951428in}{1.972682in}}%
\pgfpathcurveto{\pgfqpoint{1.951428in}{1.961632in}}{\pgfqpoint{1.955818in}{1.951033in}}{\pgfqpoint{1.963632in}{1.943220in}}%
\pgfpathcurveto{\pgfqpoint{1.971446in}{1.935406in}}{\pgfqpoint{1.982045in}{1.931016in}}{\pgfqpoint{1.993095in}{1.931016in}}%
\pgfpathclose%
\pgfusepath{stroke,fill}%
\end{pgfscope}%
\begin{pgfscope}%
\pgfpathrectangle{\pgfqpoint{0.466126in}{0.521603in}}{\pgfqpoint{3.720000in}{3.020000in}} %
\pgfusepath{clip}%
\pgfsetbuttcap%
\pgfsetroundjoin%
\definecolor{currentfill}{rgb}{0.500000,0.000000,1.000000}%
\pgfsetfillcolor{currentfill}%
\pgfsetlinewidth{1.003750pt}%
\definecolor{currentstroke}{rgb}{0.500000,0.000000,1.000000}%
\pgfsetstrokecolor{currentstroke}%
\pgfsetdash{}{0pt}%
\pgfpathmoveto{\pgfqpoint{0.696445in}{1.941612in}}%
\pgfpathcurveto{\pgfqpoint{0.707495in}{1.941612in}}{\pgfqpoint{0.718094in}{1.946002in}}{\pgfqpoint{0.725908in}{1.953816in}}%
\pgfpathcurveto{\pgfqpoint{0.733721in}{1.961629in}}{\pgfqpoint{0.738112in}{1.972228in}}{\pgfqpoint{0.738112in}{1.983278in}}%
\pgfpathcurveto{\pgfqpoint{0.738112in}{1.994329in}}{\pgfqpoint{0.733721in}{2.004928in}}{\pgfqpoint{0.725908in}{2.012741in}}%
\pgfpathcurveto{\pgfqpoint{0.718094in}{2.020555in}}{\pgfqpoint{0.707495in}{2.024945in}}{\pgfqpoint{0.696445in}{2.024945in}}%
\pgfpathcurveto{\pgfqpoint{0.685395in}{2.024945in}}{\pgfqpoint{0.674796in}{2.020555in}}{\pgfqpoint{0.666982in}{2.012741in}}%
\pgfpathcurveto{\pgfqpoint{0.659168in}{2.004928in}}{\pgfqpoint{0.654778in}{1.994329in}}{\pgfqpoint{0.654778in}{1.983278in}}%
\pgfpathcurveto{\pgfqpoint{0.654778in}{1.972228in}}{\pgfqpoint{0.659168in}{1.961629in}}{\pgfqpoint{0.666982in}{1.953816in}}%
\pgfpathcurveto{\pgfqpoint{0.674796in}{1.946002in}}{\pgfqpoint{0.685395in}{1.941612in}}{\pgfqpoint{0.696445in}{1.941612in}}%
\pgfpathclose%
\pgfusepath{stroke,fill}%
\end{pgfscope}%
\begin{pgfscope}%
\pgfpathrectangle{\pgfqpoint{0.466126in}{0.521603in}}{\pgfqpoint{3.720000in}{3.020000in}} %
\pgfusepath{clip}%
\pgfsetbuttcap%
\pgfsetroundjoin%
\definecolor{currentfill}{rgb}{1.000000,0.000000,0.000000}%
\pgfsetfillcolor{currentfill}%
\pgfsetlinewidth{1.003750pt}%
\definecolor{currentstroke}{rgb}{1.000000,0.000000,0.000000}%
\pgfsetstrokecolor{currentstroke}%
\pgfsetdash{}{0pt}%
\pgfpathmoveto{\pgfqpoint{3.840259in}{0.755334in}}%
\pgfpathcurveto{\pgfqpoint{3.851309in}{0.755334in}}{\pgfqpoint{3.861909in}{0.759724in}}{\pgfqpoint{3.869722in}{0.767538in}}%
\pgfpathcurveto{\pgfqpoint{3.877536in}{0.775352in}}{\pgfqpoint{3.881926in}{0.785951in}}{\pgfqpoint{3.881926in}{0.797001in}}%
\pgfpathcurveto{\pgfqpoint{3.881926in}{0.808051in}}{\pgfqpoint{3.877536in}{0.818650in}}{\pgfqpoint{3.869722in}{0.826464in}}%
\pgfpathcurveto{\pgfqpoint{3.861909in}{0.834277in}}{\pgfqpoint{3.851309in}{0.838668in}}{\pgfqpoint{3.840259in}{0.838668in}}%
\pgfpathcurveto{\pgfqpoint{3.829209in}{0.838668in}}{\pgfqpoint{3.818610in}{0.834277in}}{\pgfqpoint{3.810797in}{0.826464in}}%
\pgfpathcurveto{\pgfqpoint{3.802983in}{0.818650in}}{\pgfqpoint{3.798593in}{0.808051in}}{\pgfqpoint{3.798593in}{0.797001in}}%
\pgfpathcurveto{\pgfqpoint{3.798593in}{0.785951in}}{\pgfqpoint{3.802983in}{0.775352in}}{\pgfqpoint{3.810797in}{0.767538in}}%
\pgfpathcurveto{\pgfqpoint{3.818610in}{0.759724in}}{\pgfqpoint{3.829209in}{0.755334in}}{\pgfqpoint{3.840259in}{0.755334in}}%
\pgfpathclose%
\pgfusepath{stroke,fill}%
\end{pgfscope}%
\begin{pgfscope}%
\pgfpathrectangle{\pgfqpoint{0.466126in}{0.521603in}}{\pgfqpoint{3.720000in}{3.020000in}} %
\pgfusepath{clip}%
\pgfsetbuttcap%
\pgfsetroundjoin%
\definecolor{currentfill}{rgb}{1.000000,0.000000,0.000000}%
\pgfsetfillcolor{currentfill}%
\pgfsetlinewidth{1.003750pt}%
\definecolor{currentstroke}{rgb}{1.000000,0.000000,0.000000}%
\pgfsetstrokecolor{currentstroke}%
\pgfsetdash{}{0pt}%
\pgfpathmoveto{\pgfqpoint{3.268783in}{0.755334in}}%
\pgfpathcurveto{\pgfqpoint{3.279833in}{0.755334in}}{\pgfqpoint{3.290432in}{0.759724in}}{\pgfqpoint{3.298246in}{0.767538in}}%
\pgfpathcurveto{\pgfqpoint{3.306059in}{0.775352in}}{\pgfqpoint{3.310450in}{0.785951in}}{\pgfqpoint{3.310450in}{0.797001in}}%
\pgfpathcurveto{\pgfqpoint{3.310450in}{0.808051in}}{\pgfqpoint{3.306059in}{0.818650in}}{\pgfqpoint{3.298246in}{0.826464in}}%
\pgfpathcurveto{\pgfqpoint{3.290432in}{0.834277in}}{\pgfqpoint{3.279833in}{0.838668in}}{\pgfqpoint{3.268783in}{0.838668in}}%
\pgfpathcurveto{\pgfqpoint{3.257733in}{0.838668in}}{\pgfqpoint{3.247134in}{0.834277in}}{\pgfqpoint{3.239320in}{0.826464in}}%
\pgfpathcurveto{\pgfqpoint{3.231507in}{0.818650in}}{\pgfqpoint{3.227116in}{0.808051in}}{\pgfqpoint{3.227116in}{0.797001in}}%
\pgfpathcurveto{\pgfqpoint{3.227116in}{0.785951in}}{\pgfqpoint{3.231507in}{0.775352in}}{\pgfqpoint{3.239320in}{0.767538in}}%
\pgfpathcurveto{\pgfqpoint{3.247134in}{0.759724in}}{\pgfqpoint{3.257733in}{0.755334in}}{\pgfqpoint{3.268783in}{0.755334in}}%
\pgfpathclose%
\pgfusepath{stroke,fill}%
\end{pgfscope}%
\begin{pgfscope}%
\pgfpathrectangle{\pgfqpoint{0.466126in}{0.521603in}}{\pgfqpoint{3.720000in}{3.020000in}} %
\pgfusepath{clip}%
\pgfsetbuttcap%
\pgfsetroundjoin%
\definecolor{currentfill}{rgb}{1.000000,0.171626,0.086133}%
\pgfsetfillcolor{currentfill}%
\pgfsetlinewidth{1.003750pt}%
\definecolor{currentstroke}{rgb}{1.000000,0.171626,0.086133}%
\pgfsetstrokecolor{currentstroke}%
\pgfsetdash{}{0pt}%
\pgfpathmoveto{\pgfqpoint{3.194710in}{1.070975in}}%
\pgfpathcurveto{\pgfqpoint{3.205760in}{1.070975in}}{\pgfqpoint{3.216359in}{1.075365in}}{\pgfqpoint{3.224173in}{1.083179in}}%
\pgfpathcurveto{\pgfqpoint{3.231986in}{1.090993in}}{\pgfqpoint{3.236377in}{1.101592in}}{\pgfqpoint{3.236377in}{1.112642in}}%
\pgfpathcurveto{\pgfqpoint{3.236377in}{1.123692in}}{\pgfqpoint{3.231986in}{1.134291in}}{\pgfqpoint{3.224173in}{1.142104in}}%
\pgfpathcurveto{\pgfqpoint{3.216359in}{1.149918in}}{\pgfqpoint{3.205760in}{1.154308in}}{\pgfqpoint{3.194710in}{1.154308in}}%
\pgfpathcurveto{\pgfqpoint{3.183660in}{1.154308in}}{\pgfqpoint{3.173061in}{1.149918in}}{\pgfqpoint{3.165247in}{1.142104in}}%
\pgfpathcurveto{\pgfqpoint{3.157434in}{1.134291in}}{\pgfqpoint{3.153043in}{1.123692in}}{\pgfqpoint{3.153043in}{1.112642in}}%
\pgfpathcurveto{\pgfqpoint{3.153043in}{1.101592in}}{\pgfqpoint{3.157434in}{1.090993in}}{\pgfqpoint{3.165247in}{1.083179in}}%
\pgfpathcurveto{\pgfqpoint{3.173061in}{1.075365in}}{\pgfqpoint{3.183660in}{1.070975in}}{\pgfqpoint{3.194710in}{1.070975in}}%
\pgfpathclose%
\pgfusepath{stroke,fill}%
\end{pgfscope}%
\begin{pgfscope}%
\pgfpathrectangle{\pgfqpoint{0.466126in}{0.521603in}}{\pgfqpoint{3.720000in}{3.020000in}} %
\pgfusepath{clip}%
\pgfsetbuttcap%
\pgfsetroundjoin%
\definecolor{currentfill}{rgb}{1.000000,0.171626,0.086133}%
\pgfsetfillcolor{currentfill}%
\pgfsetlinewidth{1.003750pt}%
\definecolor{currentstroke}{rgb}{1.000000,0.171626,0.086133}%
\pgfsetstrokecolor{currentstroke}%
\pgfsetdash{}{0pt}%
\pgfpathmoveto{\pgfqpoint{2.826142in}{1.357655in}}%
\pgfpathcurveto{\pgfqpoint{2.837193in}{1.357655in}}{\pgfqpoint{2.847792in}{1.362046in}}{\pgfqpoint{2.855605in}{1.369859in}}%
\pgfpathcurveto{\pgfqpoint{2.863419in}{1.377673in}}{\pgfqpoint{2.867809in}{1.388272in}}{\pgfqpoint{2.867809in}{1.399322in}}%
\pgfpathcurveto{\pgfqpoint{2.867809in}{1.410372in}}{\pgfqpoint{2.863419in}{1.420971in}}{\pgfqpoint{2.855605in}{1.428785in}}%
\pgfpathcurveto{\pgfqpoint{2.847792in}{1.436598in}}{\pgfqpoint{2.837193in}{1.440989in}}{\pgfqpoint{2.826142in}{1.440989in}}%
\pgfpathcurveto{\pgfqpoint{2.815092in}{1.440989in}}{\pgfqpoint{2.804493in}{1.436598in}}{\pgfqpoint{2.796680in}{1.428785in}}%
\pgfpathcurveto{\pgfqpoint{2.788866in}{1.420971in}}{\pgfqpoint{2.784476in}{1.410372in}}{\pgfqpoint{2.784476in}{1.399322in}}%
\pgfpathcurveto{\pgfqpoint{2.784476in}{1.388272in}}{\pgfqpoint{2.788866in}{1.377673in}}{\pgfqpoint{2.796680in}{1.369859in}}%
\pgfpathcurveto{\pgfqpoint{2.804493in}{1.362046in}}{\pgfqpoint{2.815092in}{1.357655in}}{\pgfqpoint{2.826142in}{1.357655in}}%
\pgfpathclose%
\pgfusepath{stroke,fill}%
\end{pgfscope}%
\begin{pgfscope}%
\pgfpathrectangle{\pgfqpoint{0.466126in}{0.521603in}}{\pgfqpoint{3.720000in}{3.020000in}} %
\pgfusepath{clip}%
\pgfsetbuttcap%
\pgfsetroundjoin%
\definecolor{currentfill}{rgb}{0.660784,0.968276,0.612420}%
\pgfsetfillcolor{currentfill}%
\pgfsetlinewidth{1.003750pt}%
\definecolor{currentstroke}{rgb}{0.660784,0.968276,0.612420}%
\pgfsetstrokecolor{currentstroke}%
\pgfsetdash{}{0pt}%
\pgfpathmoveto{\pgfqpoint{2.517392in}{1.171558in}}%
\pgfpathcurveto{\pgfqpoint{2.528442in}{1.171558in}}{\pgfqpoint{2.539041in}{1.175948in}}{\pgfqpoint{2.546854in}{1.183762in}}%
\pgfpathcurveto{\pgfqpoint{2.554668in}{1.191575in}}{\pgfqpoint{2.559058in}{1.202174in}}{\pgfqpoint{2.559058in}{1.213224in}}%
\pgfpathcurveto{\pgfqpoint{2.559058in}{1.224274in}}{\pgfqpoint{2.554668in}{1.234873in}}{\pgfqpoint{2.546854in}{1.242687in}}%
\pgfpathcurveto{\pgfqpoint{2.539041in}{1.250501in}}{\pgfqpoint{2.528442in}{1.254891in}}{\pgfqpoint{2.517392in}{1.254891in}}%
\pgfpathcurveto{\pgfqpoint{2.506341in}{1.254891in}}{\pgfqpoint{2.495742in}{1.250501in}}{\pgfqpoint{2.487929in}{1.242687in}}%
\pgfpathcurveto{\pgfqpoint{2.480115in}{1.234873in}}{\pgfqpoint{2.475725in}{1.224274in}}{\pgfqpoint{2.475725in}{1.213224in}}%
\pgfpathcurveto{\pgfqpoint{2.475725in}{1.202174in}}{\pgfqpoint{2.480115in}{1.191575in}}{\pgfqpoint{2.487929in}{1.183762in}}%
\pgfpathcurveto{\pgfqpoint{2.495742in}{1.175948in}}{\pgfqpoint{2.506341in}{1.171558in}}{\pgfqpoint{2.517392in}{1.171558in}}%
\pgfpathclose%
\pgfusepath{stroke,fill}%
\end{pgfscope}%
\begin{pgfscope}%
\pgfpathrectangle{\pgfqpoint{0.466126in}{0.521603in}}{\pgfqpoint{3.720000in}{3.020000in}} %
\pgfusepath{clip}%
\pgfsetbuttcap%
\pgfsetroundjoin%
\definecolor{currentfill}{rgb}{0.103922,0.812622,0.889604}%
\pgfsetfillcolor{currentfill}%
\pgfsetlinewidth{1.003750pt}%
\definecolor{currentstroke}{rgb}{0.103922,0.812622,0.889604}%
\pgfsetstrokecolor{currentstroke}%
\pgfsetdash{}{0pt}%
\pgfpathmoveto{\pgfqpoint{2.450118in}{1.378029in}}%
\pgfpathcurveto{\pgfqpoint{2.461168in}{1.378029in}}{\pgfqpoint{2.471768in}{1.382419in}}{\pgfqpoint{2.479581in}{1.390233in}}%
\pgfpathcurveto{\pgfqpoint{2.487395in}{1.398047in}}{\pgfqpoint{2.491785in}{1.408646in}}{\pgfqpoint{2.491785in}{1.419696in}}%
\pgfpathcurveto{\pgfqpoint{2.491785in}{1.430746in}}{\pgfqpoint{2.487395in}{1.441345in}}{\pgfqpoint{2.479581in}{1.449158in}}%
\pgfpathcurveto{\pgfqpoint{2.471768in}{1.456972in}}{\pgfqpoint{2.461168in}{1.461362in}}{\pgfqpoint{2.450118in}{1.461362in}}%
\pgfpathcurveto{\pgfqpoint{2.439068in}{1.461362in}}{\pgfqpoint{2.428469in}{1.456972in}}{\pgfqpoint{2.420656in}{1.449158in}}%
\pgfpathcurveto{\pgfqpoint{2.412842in}{1.441345in}}{\pgfqpoint{2.408452in}{1.430746in}}{\pgfqpoint{2.408452in}{1.419696in}}%
\pgfpathcurveto{\pgfqpoint{2.408452in}{1.408646in}}{\pgfqpoint{2.412842in}{1.398047in}}{\pgfqpoint{2.420656in}{1.390233in}}%
\pgfpathcurveto{\pgfqpoint{2.428469in}{1.382419in}}{\pgfqpoint{2.439068in}{1.378029in}}{\pgfqpoint{2.450118in}{1.378029in}}%
\pgfpathclose%
\pgfusepath{stroke,fill}%
\end{pgfscope}%
\begin{pgfscope}%
\pgfpathrectangle{\pgfqpoint{0.466126in}{0.521603in}}{\pgfqpoint{3.720000in}{3.020000in}} %
\pgfusepath{clip}%
\pgfsetbuttcap%
\pgfsetroundjoin%
\definecolor{currentfill}{rgb}{0.005882,0.700543,0.925638}%
\pgfsetfillcolor{currentfill}%
\pgfsetlinewidth{1.003750pt}%
\definecolor{currentstroke}{rgb}{0.005882,0.700543,0.925638}%
\pgfsetstrokecolor{currentstroke}%
\pgfsetdash{}{0pt}%
\pgfpathmoveto{\pgfqpoint{1.717732in}{0.972602in}}%
\pgfpathcurveto{\pgfqpoint{1.728782in}{0.972602in}}{\pgfqpoint{1.739381in}{0.976992in}}{\pgfqpoint{1.747195in}{0.984806in}}%
\pgfpathcurveto{\pgfqpoint{1.755008in}{0.992619in}}{\pgfqpoint{1.759398in}{1.003218in}}{\pgfqpoint{1.759398in}{1.014269in}}%
\pgfpathcurveto{\pgfqpoint{1.759398in}{1.025319in}}{\pgfqpoint{1.755008in}{1.035918in}}{\pgfqpoint{1.747195in}{1.043731in}}%
\pgfpathcurveto{\pgfqpoint{1.739381in}{1.051545in}}{\pgfqpoint{1.728782in}{1.055935in}}{\pgfqpoint{1.717732in}{1.055935in}}%
\pgfpathcurveto{\pgfqpoint{1.706682in}{1.055935in}}{\pgfqpoint{1.696083in}{1.051545in}}{\pgfqpoint{1.688269in}{1.043731in}}%
\pgfpathcurveto{\pgfqpoint{1.680455in}{1.035918in}}{\pgfqpoint{1.676065in}{1.025319in}}{\pgfqpoint{1.676065in}{1.014269in}}%
\pgfpathcurveto{\pgfqpoint{1.676065in}{1.003218in}}{\pgfqpoint{1.680455in}{0.992619in}}{\pgfqpoint{1.688269in}{0.984806in}}%
\pgfpathcurveto{\pgfqpoint{1.696083in}{0.976992in}}{\pgfqpoint{1.706682in}{0.972602in}}{\pgfqpoint{1.717732in}{0.972602in}}%
\pgfpathclose%
\pgfusepath{stroke,fill}%
\end{pgfscope}%
\begin{pgfscope}%
\pgfpathrectangle{\pgfqpoint{0.466126in}{0.521603in}}{\pgfqpoint{3.720000in}{3.020000in}} %
\pgfusepath{clip}%
\pgfsetbuttcap%
\pgfsetroundjoin%
\definecolor{currentfill}{rgb}{0.005882,0.700543,0.925638}%
\pgfsetfillcolor{currentfill}%
\pgfsetlinewidth{1.003750pt}%
\definecolor{currentstroke}{rgb}{0.005882,0.700543,0.925638}%
\pgfsetstrokecolor{currentstroke}%
\pgfsetdash{}{0pt}%
\pgfpathmoveto{\pgfqpoint{0.977083in}{0.769888in}}%
\pgfpathcurveto{\pgfqpoint{0.988133in}{0.769888in}}{\pgfqpoint{0.998732in}{0.774279in}}{\pgfqpoint{1.006545in}{0.782092in}}%
\pgfpathcurveto{\pgfqpoint{1.014359in}{0.789906in}}{\pgfqpoint{1.018749in}{0.800505in}}{\pgfqpoint{1.018749in}{0.811555in}}%
\pgfpathcurveto{\pgfqpoint{1.018749in}{0.822605in}}{\pgfqpoint{1.014359in}{0.833204in}}{\pgfqpoint{1.006545in}{0.841018in}}%
\pgfpathcurveto{\pgfqpoint{0.998732in}{0.848831in}}{\pgfqpoint{0.988133in}{0.853222in}}{\pgfqpoint{0.977083in}{0.853222in}}%
\pgfpathcurveto{\pgfqpoint{0.966032in}{0.853222in}}{\pgfqpoint{0.955433in}{0.848831in}}{\pgfqpoint{0.947620in}{0.841018in}}%
\pgfpathcurveto{\pgfqpoint{0.939806in}{0.833204in}}{\pgfqpoint{0.935416in}{0.822605in}}{\pgfqpoint{0.935416in}{0.811555in}}%
\pgfpathcurveto{\pgfqpoint{0.935416in}{0.800505in}}{\pgfqpoint{0.939806in}{0.789906in}}{\pgfqpoint{0.947620in}{0.782092in}}%
\pgfpathcurveto{\pgfqpoint{0.955433in}{0.774279in}}{\pgfqpoint{0.966032in}{0.769888in}}{\pgfqpoint{0.977083in}{0.769888in}}%
\pgfpathclose%
\pgfusepath{stroke,fill}%
\end{pgfscope}%
\begin{pgfscope}%
\pgfpathrectangle{\pgfqpoint{0.466126in}{0.521603in}}{\pgfqpoint{3.720000in}{3.020000in}} %
\pgfusepath{clip}%
\pgfsetbuttcap%
\pgfsetroundjoin%
\definecolor{currentfill}{rgb}{0.139216,0.536867,0.960122}%
\pgfsetfillcolor{currentfill}%
\pgfsetlinewidth{1.003750pt}%
\definecolor{currentstroke}{rgb}{0.139216,0.536867,0.960122}%
\pgfsetstrokecolor{currentstroke}%
\pgfsetdash{}{0pt}%
\pgfpathmoveto{\pgfqpoint{2.309822in}{1.357655in}}%
\pgfpathcurveto{\pgfqpoint{2.320872in}{1.357655in}}{\pgfqpoint{2.331471in}{1.362046in}}{\pgfqpoint{2.339284in}{1.369859in}}%
\pgfpathcurveto{\pgfqpoint{2.347098in}{1.377673in}}{\pgfqpoint{2.351488in}{1.388272in}}{\pgfqpoint{2.351488in}{1.399322in}}%
\pgfpathcurveto{\pgfqpoint{2.351488in}{1.410372in}}{\pgfqpoint{2.347098in}{1.420971in}}{\pgfqpoint{2.339284in}{1.428785in}}%
\pgfpathcurveto{\pgfqpoint{2.331471in}{1.436598in}}{\pgfqpoint{2.320872in}{1.440989in}}{\pgfqpoint{2.309822in}{1.440989in}}%
\pgfpathcurveto{\pgfqpoint{2.298771in}{1.440989in}}{\pgfqpoint{2.288172in}{1.436598in}}{\pgfqpoint{2.280359in}{1.428785in}}%
\pgfpathcurveto{\pgfqpoint{2.272545in}{1.420971in}}{\pgfqpoint{2.268155in}{1.410372in}}{\pgfqpoint{2.268155in}{1.399322in}}%
\pgfpathcurveto{\pgfqpoint{2.268155in}{1.388272in}}{\pgfqpoint{2.272545in}{1.377673in}}{\pgfqpoint{2.280359in}{1.369859in}}%
\pgfpathcurveto{\pgfqpoint{2.288172in}{1.362046in}}{\pgfqpoint{2.298771in}{1.357655in}}{\pgfqpoint{2.309822in}{1.357655in}}%
\pgfpathclose%
\pgfusepath{stroke,fill}%
\end{pgfscope}%
\begin{pgfscope}%
\pgfpathrectangle{\pgfqpoint{0.466126in}{0.521603in}}{\pgfqpoint{3.720000in}{3.020000in}} %
\pgfusepath{clip}%
\pgfsetbuttcap%
\pgfsetroundjoin%
\definecolor{currentfill}{rgb}{0.139216,0.536867,0.960122}%
\pgfsetfillcolor{currentfill}%
\pgfsetlinewidth{1.003750pt}%
\definecolor{currentstroke}{rgb}{0.139216,0.536867,0.960122}%
\pgfsetstrokecolor{currentstroke}%
\pgfsetdash{}{0pt}%
\pgfpathmoveto{\pgfqpoint{1.871755in}{1.357655in}}%
\pgfpathcurveto{\pgfqpoint{1.882806in}{1.357655in}}{\pgfqpoint{1.893405in}{1.362046in}}{\pgfqpoint{1.901218in}{1.369859in}}%
\pgfpathcurveto{\pgfqpoint{1.909032in}{1.377673in}}{\pgfqpoint{1.913422in}{1.388272in}}{\pgfqpoint{1.913422in}{1.399322in}}%
\pgfpathcurveto{\pgfqpoint{1.913422in}{1.410372in}}{\pgfqpoint{1.909032in}{1.420971in}}{\pgfqpoint{1.901218in}{1.428785in}}%
\pgfpathcurveto{\pgfqpoint{1.893405in}{1.436598in}}{\pgfqpoint{1.882806in}{1.440989in}}{\pgfqpoint{1.871755in}{1.440989in}}%
\pgfpathcurveto{\pgfqpoint{1.860705in}{1.440989in}}{\pgfqpoint{1.850106in}{1.436598in}}{\pgfqpoint{1.842293in}{1.428785in}}%
\pgfpathcurveto{\pgfqpoint{1.834479in}{1.420971in}}{\pgfqpoint{1.830089in}{1.410372in}}{\pgfqpoint{1.830089in}{1.399322in}}%
\pgfpathcurveto{\pgfqpoint{1.830089in}{1.388272in}}{\pgfqpoint{1.834479in}{1.377673in}}{\pgfqpoint{1.842293in}{1.369859in}}%
\pgfpathcurveto{\pgfqpoint{1.850106in}{1.362046in}}{\pgfqpoint{1.860705in}{1.357655in}}{\pgfqpoint{1.871755in}{1.357655in}}%
\pgfpathclose%
\pgfusepath{stroke,fill}%
\end{pgfscope}%
\begin{pgfscope}%
\pgfpathrectangle{\pgfqpoint{0.466126in}{0.521603in}}{\pgfqpoint{3.720000in}{3.020000in}} %
\pgfusepath{clip}%
\pgfsetbuttcap%
\pgfsetroundjoin%
\definecolor{currentfill}{rgb}{0.139216,0.536867,0.960122}%
\pgfsetfillcolor{currentfill}%
\pgfsetlinewidth{1.003750pt}%
\definecolor{currentstroke}{rgb}{0.139216,0.536867,0.960122}%
\pgfsetstrokecolor{currentstroke}%
\pgfsetdash{}{0pt}%
\pgfpathmoveto{\pgfqpoint{1.613588in}{1.357655in}}%
\pgfpathcurveto{\pgfqpoint{1.624638in}{1.357655in}}{\pgfqpoint{1.635237in}{1.362046in}}{\pgfqpoint{1.643051in}{1.369859in}}%
\pgfpathcurveto{\pgfqpoint{1.650865in}{1.377673in}}{\pgfqpoint{1.655255in}{1.388272in}}{\pgfqpoint{1.655255in}{1.399322in}}%
\pgfpathcurveto{\pgfqpoint{1.655255in}{1.410372in}}{\pgfqpoint{1.650865in}{1.420971in}}{\pgfqpoint{1.643051in}{1.428785in}}%
\pgfpathcurveto{\pgfqpoint{1.635237in}{1.436598in}}{\pgfqpoint{1.624638in}{1.440989in}}{\pgfqpoint{1.613588in}{1.440989in}}%
\pgfpathcurveto{\pgfqpoint{1.602538in}{1.440989in}}{\pgfqpoint{1.591939in}{1.436598in}}{\pgfqpoint{1.584125in}{1.428785in}}%
\pgfpathcurveto{\pgfqpoint{1.576312in}{1.420971in}}{\pgfqpoint{1.571922in}{1.410372in}}{\pgfqpoint{1.571922in}{1.399322in}}%
\pgfpathcurveto{\pgfqpoint{1.571922in}{1.388272in}}{\pgfqpoint{1.576312in}{1.377673in}}{\pgfqpoint{1.584125in}{1.369859in}}%
\pgfpathcurveto{\pgfqpoint{1.591939in}{1.362046in}}{\pgfqpoint{1.602538in}{1.357655in}}{\pgfqpoint{1.613588in}{1.357655in}}%
\pgfpathclose%
\pgfusepath{stroke,fill}%
\end{pgfscope}%
\begin{pgfscope}%
\pgfpathrectangle{\pgfqpoint{0.466126in}{0.521603in}}{\pgfqpoint{3.720000in}{3.020000in}} %
\pgfusepath{clip}%
\pgfsetbuttcap%
\pgfsetroundjoin%
\definecolor{currentfill}{rgb}{0.139216,0.536867,0.960122}%
\pgfsetfillcolor{currentfill}%
\pgfsetlinewidth{1.003750pt}%
\definecolor{currentstroke}{rgb}{0.139216,0.536867,0.960122}%
\pgfsetstrokecolor{currentstroke}%
\pgfsetdash{}{0pt}%
\pgfpathmoveto{\pgfqpoint{1.372829in}{1.357655in}}%
\pgfpathcurveto{\pgfqpoint{1.383879in}{1.357655in}}{\pgfqpoint{1.394478in}{1.362046in}}{\pgfqpoint{1.402291in}{1.369859in}}%
\pgfpathcurveto{\pgfqpoint{1.410105in}{1.377673in}}{\pgfqpoint{1.414495in}{1.388272in}}{\pgfqpoint{1.414495in}{1.399322in}}%
\pgfpathcurveto{\pgfqpoint{1.414495in}{1.410372in}}{\pgfqpoint{1.410105in}{1.420971in}}{\pgfqpoint{1.402291in}{1.428785in}}%
\pgfpathcurveto{\pgfqpoint{1.394478in}{1.436598in}}{\pgfqpoint{1.383879in}{1.440989in}}{\pgfqpoint{1.372829in}{1.440989in}}%
\pgfpathcurveto{\pgfqpoint{1.361778in}{1.440989in}}{\pgfqpoint{1.351179in}{1.436598in}}{\pgfqpoint{1.343366in}{1.428785in}}%
\pgfpathcurveto{\pgfqpoint{1.335552in}{1.420971in}}{\pgfqpoint{1.331162in}{1.410372in}}{\pgfqpoint{1.331162in}{1.399322in}}%
\pgfpathcurveto{\pgfqpoint{1.331162in}{1.388272in}}{\pgfqpoint{1.335552in}{1.377673in}}{\pgfqpoint{1.343366in}{1.369859in}}%
\pgfpathcurveto{\pgfqpoint{1.351179in}{1.362046in}}{\pgfqpoint{1.361778in}{1.357655in}}{\pgfqpoint{1.372829in}{1.357655in}}%
\pgfpathclose%
\pgfusepath{stroke,fill}%
\end{pgfscope}%
\begin{pgfscope}%
\pgfpathrectangle{\pgfqpoint{0.466126in}{0.521603in}}{\pgfqpoint{3.720000in}{3.020000in}} %
\pgfusepath{clip}%
\pgfsetbuttcap%
\pgfsetroundjoin%
\definecolor{currentfill}{rgb}{0.139216,0.536867,0.960122}%
\pgfsetfillcolor{currentfill}%
\pgfsetlinewidth{1.003750pt}%
\definecolor{currentstroke}{rgb}{0.139216,0.536867,0.960122}%
\pgfsetstrokecolor{currentstroke}%
\pgfsetdash{}{0pt}%
\pgfpathmoveto{\pgfqpoint{0.707998in}{1.070975in}}%
\pgfpathcurveto{\pgfqpoint{0.719048in}{1.070975in}}{\pgfqpoint{0.729647in}{1.075365in}}{\pgfqpoint{0.737460in}{1.083179in}}%
\pgfpathcurveto{\pgfqpoint{0.745274in}{1.090993in}}{\pgfqpoint{0.749664in}{1.101592in}}{\pgfqpoint{0.749664in}{1.112642in}}%
\pgfpathcurveto{\pgfqpoint{0.749664in}{1.123692in}}{\pgfqpoint{0.745274in}{1.134291in}}{\pgfqpoint{0.737460in}{1.142104in}}%
\pgfpathcurveto{\pgfqpoint{0.729647in}{1.149918in}}{\pgfqpoint{0.719048in}{1.154308in}}{\pgfqpoint{0.707998in}{1.154308in}}%
\pgfpathcurveto{\pgfqpoint{0.696947in}{1.154308in}}{\pgfqpoint{0.686348in}{1.149918in}}{\pgfqpoint{0.678535in}{1.142104in}}%
\pgfpathcurveto{\pgfqpoint{0.670721in}{1.134291in}}{\pgfqpoint{0.666331in}{1.123692in}}{\pgfqpoint{0.666331in}{1.112642in}}%
\pgfpathcurveto{\pgfqpoint{0.666331in}{1.101592in}}{\pgfqpoint{0.670721in}{1.090993in}}{\pgfqpoint{0.678535in}{1.083179in}}%
\pgfpathcurveto{\pgfqpoint{0.686348in}{1.075365in}}{\pgfqpoint{0.696947in}{1.070975in}}{\pgfqpoint{0.707998in}{1.070975in}}%
\pgfpathclose%
\pgfusepath{stroke,fill}%
\end{pgfscope}%
\begin{pgfscope}%
\pgfsetbuttcap%
\pgfsetroundjoin%
\definecolor{currentfill}{rgb}{0.000000,0.000000,0.000000}%
\pgfsetfillcolor{currentfill}%
\pgfsetlinewidth{0.803000pt}%
\definecolor{currentstroke}{rgb}{0.000000,0.000000,0.000000}%
\pgfsetstrokecolor{currentstroke}%
\pgfsetdash{}{0pt}%
\pgfsys@defobject{currentmarker}{\pgfqpoint{0.000000in}{-0.048611in}}{\pgfqpoint{0.000000in}{0.000000in}}{%
\pgfpathmoveto{\pgfqpoint{0.000000in}{0.000000in}}%
\pgfpathlineto{\pgfqpoint{0.000000in}{-0.048611in}}%
\pgfusepath{stroke,fill}%
}%
\begin{pgfscope}%
\pgfsys@transformshift{1.268041in}{0.521603in}%
\pgfsys@useobject{currentmarker}{}%
\end{pgfscope}%
\end{pgfscope}%
\begin{pgfscope}%
\pgftext[x=1.268041in,y=0.424381in,,top]{\rmfamily\fontsize{10.000000}{12.000000}\selectfont \(\displaystyle 10^{-1}\)}%
\end{pgfscope}%
\begin{pgfscope}%
\pgfsetbuttcap%
\pgfsetroundjoin%
\definecolor{currentfill}{rgb}{0.000000,0.000000,0.000000}%
\pgfsetfillcolor{currentfill}%
\pgfsetlinewidth{0.803000pt}%
\definecolor{currentstroke}{rgb}{0.000000,0.000000,0.000000}%
\pgfsetstrokecolor{currentstroke}%
\pgfsetdash{}{0pt}%
\pgfsys@defobject{currentmarker}{\pgfqpoint{0.000000in}{-0.048611in}}{\pgfqpoint{0.000000in}{0.000000in}}{%
\pgfpathmoveto{\pgfqpoint{0.000000in}{0.000000in}}%
\pgfpathlineto{\pgfqpoint{0.000000in}{-0.048611in}}%
\pgfusepath{stroke,fill}%
}%
\begin{pgfscope}%
\pgfsys@transformshift{3.931945in}{0.521603in}%
\pgfsys@useobject{currentmarker}{}%
\end{pgfscope}%
\end{pgfscope}%
\begin{pgfscope}%
\pgftext[x=3.931945in,y=0.424381in,,top]{\rmfamily\fontsize{10.000000}{12.000000}\selectfont \(\displaystyle 10^{0}\)}%
\end{pgfscope}%
\begin{pgfscope}%
\pgfsetbuttcap%
\pgfsetroundjoin%
\definecolor{currentfill}{rgb}{0.000000,0.000000,0.000000}%
\pgfsetfillcolor{currentfill}%
\pgfsetlinewidth{0.602250pt}%
\definecolor{currentstroke}{rgb}{0.000000,0.000000,0.000000}%
\pgfsetstrokecolor{currentstroke}%
\pgfsetdash{}{0pt}%
\pgfsys@defobject{currentmarker}{\pgfqpoint{0.000000in}{-0.027778in}}{\pgfqpoint{0.000000in}{0.000000in}}{%
\pgfpathmoveto{\pgfqpoint{0.000000in}{0.000000in}}%
\pgfpathlineto{\pgfqpoint{0.000000in}{-0.027778in}}%
\pgfusepath{stroke,fill}%
}%
\begin{pgfscope}%
\pgfsys@transformshift{0.466126in}{0.521603in}%
\pgfsys@useobject{currentmarker}{}%
\end{pgfscope}%
\end{pgfscope}%
\begin{pgfscope}%
\pgfsetbuttcap%
\pgfsetroundjoin%
\definecolor{currentfill}{rgb}{0.000000,0.000000,0.000000}%
\pgfsetfillcolor{currentfill}%
\pgfsetlinewidth{0.602250pt}%
\definecolor{currentstroke}{rgb}{0.000000,0.000000,0.000000}%
\pgfsetstrokecolor{currentstroke}%
\pgfsetdash{}{0pt}%
\pgfsys@defobject{currentmarker}{\pgfqpoint{0.000000in}{-0.027778in}}{\pgfqpoint{0.000000in}{0.000000in}}{%
\pgfpathmoveto{\pgfqpoint{0.000000in}{0.000000in}}%
\pgfpathlineto{\pgfqpoint{0.000000in}{-0.027778in}}%
\pgfusepath{stroke,fill}%
}%
\begin{pgfscope}%
\pgfsys@transformshift{0.677058in}{0.521603in}%
\pgfsys@useobject{currentmarker}{}%
\end{pgfscope}%
\end{pgfscope}%
\begin{pgfscope}%
\pgfsetbuttcap%
\pgfsetroundjoin%
\definecolor{currentfill}{rgb}{0.000000,0.000000,0.000000}%
\pgfsetfillcolor{currentfill}%
\pgfsetlinewidth{0.602250pt}%
\definecolor{currentstroke}{rgb}{0.000000,0.000000,0.000000}%
\pgfsetstrokecolor{currentstroke}%
\pgfsetdash{}{0pt}%
\pgfsys@defobject{currentmarker}{\pgfqpoint{0.000000in}{-0.027778in}}{\pgfqpoint{0.000000in}{0.000000in}}{%
\pgfpathmoveto{\pgfqpoint{0.000000in}{0.000000in}}%
\pgfpathlineto{\pgfqpoint{0.000000in}{-0.027778in}}%
\pgfusepath{stroke,fill}%
}%
\begin{pgfscope}%
\pgfsys@transformshift{0.855397in}{0.521603in}%
\pgfsys@useobject{currentmarker}{}%
\end{pgfscope}%
\end{pgfscope}%
\begin{pgfscope}%
\pgfsetbuttcap%
\pgfsetroundjoin%
\definecolor{currentfill}{rgb}{0.000000,0.000000,0.000000}%
\pgfsetfillcolor{currentfill}%
\pgfsetlinewidth{0.602250pt}%
\definecolor{currentstroke}{rgb}{0.000000,0.000000,0.000000}%
\pgfsetstrokecolor{currentstroke}%
\pgfsetdash{}{0pt}%
\pgfsys@defobject{currentmarker}{\pgfqpoint{0.000000in}{-0.027778in}}{\pgfqpoint{0.000000in}{0.000000in}}{%
\pgfpathmoveto{\pgfqpoint{0.000000in}{0.000000in}}%
\pgfpathlineto{\pgfqpoint{0.000000in}{-0.027778in}}%
\pgfusepath{stroke,fill}%
}%
\begin{pgfscope}%
\pgfsys@transformshift{1.009882in}{0.521603in}%
\pgfsys@useobject{currentmarker}{}%
\end{pgfscope}%
\end{pgfscope}%
\begin{pgfscope}%
\pgfsetbuttcap%
\pgfsetroundjoin%
\definecolor{currentfill}{rgb}{0.000000,0.000000,0.000000}%
\pgfsetfillcolor{currentfill}%
\pgfsetlinewidth{0.602250pt}%
\definecolor{currentstroke}{rgb}{0.000000,0.000000,0.000000}%
\pgfsetstrokecolor{currentstroke}%
\pgfsetdash{}{0pt}%
\pgfsys@defobject{currentmarker}{\pgfqpoint{0.000000in}{-0.027778in}}{\pgfqpoint{0.000000in}{0.000000in}}{%
\pgfpathmoveto{\pgfqpoint{0.000000in}{0.000000in}}%
\pgfpathlineto{\pgfqpoint{0.000000in}{-0.027778in}}%
\pgfusepath{stroke,fill}%
}%
\begin{pgfscope}%
\pgfsys@transformshift{1.146148in}{0.521603in}%
\pgfsys@useobject{currentmarker}{}%
\end{pgfscope}%
\end{pgfscope}%
\begin{pgfscope}%
\pgfsetbuttcap%
\pgfsetroundjoin%
\definecolor{currentfill}{rgb}{0.000000,0.000000,0.000000}%
\pgfsetfillcolor{currentfill}%
\pgfsetlinewidth{0.602250pt}%
\definecolor{currentstroke}{rgb}{0.000000,0.000000,0.000000}%
\pgfsetstrokecolor{currentstroke}%
\pgfsetdash{}{0pt}%
\pgfsys@defobject{currentmarker}{\pgfqpoint{0.000000in}{-0.027778in}}{\pgfqpoint{0.000000in}{0.000000in}}{%
\pgfpathmoveto{\pgfqpoint{0.000000in}{0.000000in}}%
\pgfpathlineto{\pgfqpoint{0.000000in}{-0.027778in}}%
\pgfusepath{stroke,fill}%
}%
\begin{pgfscope}%
\pgfsys@transformshift{2.069956in}{0.521603in}%
\pgfsys@useobject{currentmarker}{}%
\end{pgfscope}%
\end{pgfscope}%
\begin{pgfscope}%
\pgfsetbuttcap%
\pgfsetroundjoin%
\definecolor{currentfill}{rgb}{0.000000,0.000000,0.000000}%
\pgfsetfillcolor{currentfill}%
\pgfsetlinewidth{0.602250pt}%
\definecolor{currentstroke}{rgb}{0.000000,0.000000,0.000000}%
\pgfsetstrokecolor{currentstroke}%
\pgfsetdash{}{0pt}%
\pgfsys@defobject{currentmarker}{\pgfqpoint{0.000000in}{-0.027778in}}{\pgfqpoint{0.000000in}{0.000000in}}{%
\pgfpathmoveto{\pgfqpoint{0.000000in}{0.000000in}}%
\pgfpathlineto{\pgfqpoint{0.000000in}{-0.027778in}}%
\pgfusepath{stroke,fill}%
}%
\begin{pgfscope}%
\pgfsys@transformshift{2.539046in}{0.521603in}%
\pgfsys@useobject{currentmarker}{}%
\end{pgfscope}%
\end{pgfscope}%
\begin{pgfscope}%
\pgfsetbuttcap%
\pgfsetroundjoin%
\definecolor{currentfill}{rgb}{0.000000,0.000000,0.000000}%
\pgfsetfillcolor{currentfill}%
\pgfsetlinewidth{0.602250pt}%
\definecolor{currentstroke}{rgb}{0.000000,0.000000,0.000000}%
\pgfsetstrokecolor{currentstroke}%
\pgfsetdash{}{0pt}%
\pgfsys@defobject{currentmarker}{\pgfqpoint{0.000000in}{-0.027778in}}{\pgfqpoint{0.000000in}{0.000000in}}{%
\pgfpathmoveto{\pgfqpoint{0.000000in}{0.000000in}}%
\pgfpathlineto{\pgfqpoint{0.000000in}{-0.027778in}}%
\pgfusepath{stroke,fill}%
}%
\begin{pgfscope}%
\pgfsys@transformshift{2.871871in}{0.521603in}%
\pgfsys@useobject{currentmarker}{}%
\end{pgfscope}%
\end{pgfscope}%
\begin{pgfscope}%
\pgfsetbuttcap%
\pgfsetroundjoin%
\definecolor{currentfill}{rgb}{0.000000,0.000000,0.000000}%
\pgfsetfillcolor{currentfill}%
\pgfsetlinewidth{0.602250pt}%
\definecolor{currentstroke}{rgb}{0.000000,0.000000,0.000000}%
\pgfsetstrokecolor{currentstroke}%
\pgfsetdash{}{0pt}%
\pgfsys@defobject{currentmarker}{\pgfqpoint{0.000000in}{-0.027778in}}{\pgfqpoint{0.000000in}{0.000000in}}{%
\pgfpathmoveto{\pgfqpoint{0.000000in}{0.000000in}}%
\pgfpathlineto{\pgfqpoint{0.000000in}{-0.027778in}}%
\pgfusepath{stroke,fill}%
}%
\begin{pgfscope}%
\pgfsys@transformshift{3.130030in}{0.521603in}%
\pgfsys@useobject{currentmarker}{}%
\end{pgfscope}%
\end{pgfscope}%
\begin{pgfscope}%
\pgfsetbuttcap%
\pgfsetroundjoin%
\definecolor{currentfill}{rgb}{0.000000,0.000000,0.000000}%
\pgfsetfillcolor{currentfill}%
\pgfsetlinewidth{0.602250pt}%
\definecolor{currentstroke}{rgb}{0.000000,0.000000,0.000000}%
\pgfsetstrokecolor{currentstroke}%
\pgfsetdash{}{0pt}%
\pgfsys@defobject{currentmarker}{\pgfqpoint{0.000000in}{-0.027778in}}{\pgfqpoint{0.000000in}{0.000000in}}{%
\pgfpathmoveto{\pgfqpoint{0.000000in}{0.000000in}}%
\pgfpathlineto{\pgfqpoint{0.000000in}{-0.027778in}}%
\pgfusepath{stroke,fill}%
}%
\begin{pgfscope}%
\pgfsys@transformshift{3.340961in}{0.521603in}%
\pgfsys@useobject{currentmarker}{}%
\end{pgfscope}%
\end{pgfscope}%
\begin{pgfscope}%
\pgfsetbuttcap%
\pgfsetroundjoin%
\definecolor{currentfill}{rgb}{0.000000,0.000000,0.000000}%
\pgfsetfillcolor{currentfill}%
\pgfsetlinewidth{0.602250pt}%
\definecolor{currentstroke}{rgb}{0.000000,0.000000,0.000000}%
\pgfsetstrokecolor{currentstroke}%
\pgfsetdash{}{0pt}%
\pgfsys@defobject{currentmarker}{\pgfqpoint{0.000000in}{-0.027778in}}{\pgfqpoint{0.000000in}{0.000000in}}{%
\pgfpathmoveto{\pgfqpoint{0.000000in}{0.000000in}}%
\pgfpathlineto{\pgfqpoint{0.000000in}{-0.027778in}}%
\pgfusepath{stroke,fill}%
}%
\begin{pgfscope}%
\pgfsys@transformshift{3.519301in}{0.521603in}%
\pgfsys@useobject{currentmarker}{}%
\end{pgfscope}%
\end{pgfscope}%
\begin{pgfscope}%
\pgfsetbuttcap%
\pgfsetroundjoin%
\definecolor{currentfill}{rgb}{0.000000,0.000000,0.000000}%
\pgfsetfillcolor{currentfill}%
\pgfsetlinewidth{0.602250pt}%
\definecolor{currentstroke}{rgb}{0.000000,0.000000,0.000000}%
\pgfsetstrokecolor{currentstroke}%
\pgfsetdash{}{0pt}%
\pgfsys@defobject{currentmarker}{\pgfqpoint{0.000000in}{-0.027778in}}{\pgfqpoint{0.000000in}{0.000000in}}{%
\pgfpathmoveto{\pgfqpoint{0.000000in}{0.000000in}}%
\pgfpathlineto{\pgfqpoint{0.000000in}{-0.027778in}}%
\pgfusepath{stroke,fill}%
}%
\begin{pgfscope}%
\pgfsys@transformshift{3.673786in}{0.521603in}%
\pgfsys@useobject{currentmarker}{}%
\end{pgfscope}%
\end{pgfscope}%
\begin{pgfscope}%
\pgfsetbuttcap%
\pgfsetroundjoin%
\definecolor{currentfill}{rgb}{0.000000,0.000000,0.000000}%
\pgfsetfillcolor{currentfill}%
\pgfsetlinewidth{0.602250pt}%
\definecolor{currentstroke}{rgb}{0.000000,0.000000,0.000000}%
\pgfsetstrokecolor{currentstroke}%
\pgfsetdash{}{0pt}%
\pgfsys@defobject{currentmarker}{\pgfqpoint{0.000000in}{-0.027778in}}{\pgfqpoint{0.000000in}{0.000000in}}{%
\pgfpathmoveto{\pgfqpoint{0.000000in}{0.000000in}}%
\pgfpathlineto{\pgfqpoint{0.000000in}{-0.027778in}}%
\pgfusepath{stroke,fill}%
}%
\begin{pgfscope}%
\pgfsys@transformshift{3.810051in}{0.521603in}%
\pgfsys@useobject{currentmarker}{}%
\end{pgfscope}%
\end{pgfscope}%
\begin{pgfscope}%
\pgftext[x=2.326126in,y=0.234413in,,top]{\rmfamily\fontsize{10.000000}{12.000000}\selectfont \(\displaystyle \mathbf{W}\mbox{e}\)}%
\end{pgfscope}%
\begin{pgfscope}%
\pgfsetbuttcap%
\pgfsetroundjoin%
\definecolor{currentfill}{rgb}{0.000000,0.000000,0.000000}%
\pgfsetfillcolor{currentfill}%
\pgfsetlinewidth{0.803000pt}%
\definecolor{currentstroke}{rgb}{0.000000,0.000000,0.000000}%
\pgfsetstrokecolor{currentstroke}%
\pgfsetdash{}{0pt}%
\pgfsys@defobject{currentmarker}{\pgfqpoint{-0.048611in}{0.000000in}}{\pgfqpoint{0.000000in}{0.000000in}}{%
\pgfpathmoveto{\pgfqpoint{0.000000in}{0.000000in}}%
\pgfpathlineto{\pgfqpoint{-0.048611in}{0.000000in}}%
\pgfusepath{stroke,fill}%
}%
\begin{pgfscope}%
\pgfsys@transformshift{0.466126in}{0.521603in}%
\pgfsys@useobject{currentmarker}{}%
\end{pgfscope}%
\end{pgfscope}%
\begin{pgfscope}%
\pgftext[x=0.299459in,y=0.468842in,left,base]{\rmfamily\fontsize{10.000000}{12.000000}\selectfont \(\displaystyle 2\)}%
\end{pgfscope}%
\begin{pgfscope}%
\pgfsetbuttcap%
\pgfsetroundjoin%
\definecolor{currentfill}{rgb}{0.000000,0.000000,0.000000}%
\pgfsetfillcolor{currentfill}%
\pgfsetlinewidth{0.803000pt}%
\definecolor{currentstroke}{rgb}{0.000000,0.000000,0.000000}%
\pgfsetstrokecolor{currentstroke}%
\pgfsetdash{}{0pt}%
\pgfsys@defobject{currentmarker}{\pgfqpoint{-0.048611in}{0.000000in}}{\pgfqpoint{0.000000in}{0.000000in}}{%
\pgfpathmoveto{\pgfqpoint{0.000000in}{0.000000in}}%
\pgfpathlineto{\pgfqpoint{-0.048611in}{0.000000in}}%
\pgfusepath{stroke,fill}%
}%
\begin{pgfscope}%
\pgfsys@transformshift{0.466126in}{1.086125in}%
\pgfsys@useobject{currentmarker}{}%
\end{pgfscope}%
\end{pgfscope}%
\begin{pgfscope}%
\pgftext[x=0.299459in,y=1.033363in,left,base]{\rmfamily\fontsize{10.000000}{12.000000}\selectfont \(\displaystyle 3\)}%
\end{pgfscope}%
\begin{pgfscope}%
\pgfsetbuttcap%
\pgfsetroundjoin%
\definecolor{currentfill}{rgb}{0.000000,0.000000,0.000000}%
\pgfsetfillcolor{currentfill}%
\pgfsetlinewidth{0.803000pt}%
\definecolor{currentstroke}{rgb}{0.000000,0.000000,0.000000}%
\pgfsetstrokecolor{currentstroke}%
\pgfsetdash{}{0pt}%
\pgfsys@defobject{currentmarker}{\pgfqpoint{-0.048611in}{0.000000in}}{\pgfqpoint{0.000000in}{0.000000in}}{%
\pgfpathmoveto{\pgfqpoint{0.000000in}{0.000000in}}%
\pgfpathlineto{\pgfqpoint{-0.048611in}{0.000000in}}%
\pgfusepath{stroke,fill}%
}%
\begin{pgfscope}%
\pgfsys@transformshift{0.466126in}{1.650646in}%
\pgfsys@useobject{currentmarker}{}%
\end{pgfscope}%
\end{pgfscope}%
\begin{pgfscope}%
\pgftext[x=0.299459in,y=1.597885in,left,base]{\rmfamily\fontsize{10.000000}{12.000000}\selectfont \(\displaystyle 4\)}%
\end{pgfscope}%
\begin{pgfscope}%
\pgfsetbuttcap%
\pgfsetroundjoin%
\definecolor{currentfill}{rgb}{0.000000,0.000000,0.000000}%
\pgfsetfillcolor{currentfill}%
\pgfsetlinewidth{0.803000pt}%
\definecolor{currentstroke}{rgb}{0.000000,0.000000,0.000000}%
\pgfsetstrokecolor{currentstroke}%
\pgfsetdash{}{0pt}%
\pgfsys@defobject{currentmarker}{\pgfqpoint{-0.048611in}{0.000000in}}{\pgfqpoint{0.000000in}{0.000000in}}{%
\pgfpathmoveto{\pgfqpoint{0.000000in}{0.000000in}}%
\pgfpathlineto{\pgfqpoint{-0.048611in}{0.000000in}}%
\pgfusepath{stroke,fill}%
}%
\begin{pgfscope}%
\pgfsys@transformshift{0.466126in}{2.215168in}%
\pgfsys@useobject{currentmarker}{}%
\end{pgfscope}%
\end{pgfscope}%
\begin{pgfscope}%
\pgftext[x=0.299459in,y=2.162406in,left,base]{\rmfamily\fontsize{10.000000}{12.000000}\selectfont \(\displaystyle 5\)}%
\end{pgfscope}%
\begin{pgfscope}%
\pgfsetbuttcap%
\pgfsetroundjoin%
\definecolor{currentfill}{rgb}{0.000000,0.000000,0.000000}%
\pgfsetfillcolor{currentfill}%
\pgfsetlinewidth{0.803000pt}%
\definecolor{currentstroke}{rgb}{0.000000,0.000000,0.000000}%
\pgfsetstrokecolor{currentstroke}%
\pgfsetdash{}{0pt}%
\pgfsys@defobject{currentmarker}{\pgfqpoint{-0.048611in}{0.000000in}}{\pgfqpoint{0.000000in}{0.000000in}}{%
\pgfpathmoveto{\pgfqpoint{0.000000in}{0.000000in}}%
\pgfpathlineto{\pgfqpoint{-0.048611in}{0.000000in}}%
\pgfusepath{stroke,fill}%
}%
\begin{pgfscope}%
\pgfsys@transformshift{0.466126in}{2.779690in}%
\pgfsys@useobject{currentmarker}{}%
\end{pgfscope}%
\end{pgfscope}%
\begin{pgfscope}%
\pgftext[x=0.299459in,y=2.726928in,left,base]{\rmfamily\fontsize{10.000000}{12.000000}\selectfont \(\displaystyle 6\)}%
\end{pgfscope}%
\begin{pgfscope}%
\pgfsetbuttcap%
\pgfsetroundjoin%
\definecolor{currentfill}{rgb}{0.000000,0.000000,0.000000}%
\pgfsetfillcolor{currentfill}%
\pgfsetlinewidth{0.803000pt}%
\definecolor{currentstroke}{rgb}{0.000000,0.000000,0.000000}%
\pgfsetstrokecolor{currentstroke}%
\pgfsetdash{}{0pt}%
\pgfsys@defobject{currentmarker}{\pgfqpoint{-0.048611in}{0.000000in}}{\pgfqpoint{0.000000in}{0.000000in}}{%
\pgfpathmoveto{\pgfqpoint{0.000000in}{0.000000in}}%
\pgfpathlineto{\pgfqpoint{-0.048611in}{0.000000in}}%
\pgfusepath{stroke,fill}%
}%
\begin{pgfscope}%
\pgfsys@transformshift{0.466126in}{3.344211in}%
\pgfsys@useobject{currentmarker}{}%
\end{pgfscope}%
\end{pgfscope}%
\begin{pgfscope}%
\pgftext[x=0.299459in,y=3.291450in,left,base]{\rmfamily\fontsize{10.000000}{12.000000}\selectfont \(\displaystyle 7\)}%
\end{pgfscope}%
\begin{pgfscope}%
\pgftext[x=0.243904in,y=2.031603in,,bottom,rotate=90.000000]{\rmfamily\fontsize{10.000000}{12.000000}\selectfont \(\displaystyle t_j/ \tau\)}%
\end{pgfscope}%
\begin{pgfscope}%
\pgfpathrectangle{\pgfqpoint{0.466126in}{0.521603in}}{\pgfqpoint{3.720000in}{3.020000in}} %
\pgfusepath{clip}%
\pgfsetbuttcap%
\pgfsetroundjoin%
\pgfsetlinewidth{1.505625pt}%
\definecolor{currentstroke}{rgb}{0.000000,0.000000,0.000000}%
\pgfsetstrokecolor{currentstroke}%
\pgfsetdash{{5.550000pt}{2.400000pt}}{0.000000pt}%
\pgfpathmoveto{\pgfqpoint{0.456126in}{0.634508in}}%
\pgfpathlineto{\pgfqpoint{0.456740in}{0.634508in}}%
\pgfpathlineto{\pgfqpoint{0.667278in}{0.634508in}}%
\pgfpathlineto{\pgfqpoint{0.845337in}{0.634508in}}%
\pgfpathlineto{\pgfqpoint{0.999612in}{0.634508in}}%
\pgfpathlineto{\pgfqpoint{1.135713in}{0.634508in}}%
\pgfpathlineto{\pgfqpoint{1.257476in}{0.634508in}}%
\pgfpathlineto{\pgfqpoint{1.367635in}{0.634508in}}%
\pgfpathlineto{\pgfqpoint{1.468210in}{0.634508in}}%
\pgfpathlineto{\pgfqpoint{1.560738in}{0.634508in}}%
\pgfpathlineto{\pgfqpoint{1.646410in}{0.634508in}}%
\pgfpathlineto{\pgfqpoint{1.726173in}{0.634508in}}%
\pgfpathlineto{\pgfqpoint{1.800789in}{0.634508in}}%
\pgfpathlineto{\pgfqpoint{1.870884in}{0.634508in}}%
\pgfpathlineto{\pgfqpoint{1.936973in}{0.634508in}}%
\pgfpathlineto{\pgfqpoint{1.999490in}{0.634508in}}%
\pgfpathlineto{\pgfqpoint{2.058801in}{0.634508in}}%
\pgfpathlineto{\pgfqpoint{2.115219in}{0.634508in}}%
\pgfpathlineto{\pgfqpoint{2.169013in}{0.634508in}}%
\pgfpathlineto{\pgfqpoint{2.220417in}{0.634508in}}%
\pgfpathlineto{\pgfqpoint{2.269634in}{0.634508in}}%
\pgfpathlineto{\pgfqpoint{2.316842in}{0.634508in}}%
\pgfpathlineto{\pgfqpoint{2.362199in}{0.634508in}}%
\pgfpathlineto{\pgfqpoint{2.405844in}{0.634508in}}%
\pgfpathlineto{\pgfqpoint{2.447903in}{0.634508in}}%
\pgfpathlineto{\pgfqpoint{2.488486in}{0.634508in}}%
\pgfpathlineto{\pgfqpoint{2.527694in}{0.634508in}}%
\pgfpathlineto{\pgfqpoint{2.565617in}{0.634508in}}%
\pgfpathlineto{\pgfqpoint{2.602335in}{0.634508in}}%
\pgfpathlineto{\pgfqpoint{2.637925in}{0.634508in}}%
\pgfpathlineto{\pgfqpoint{2.672452in}{0.634508in}}%
\pgfpathlineto{\pgfqpoint{2.705978in}{0.634508in}}%
\pgfpathlineto{\pgfqpoint{2.738560in}{0.634508in}}%
\pgfpathlineto{\pgfqpoint{2.770249in}{0.634508in}}%
\pgfpathlineto{\pgfqpoint{2.801094in}{0.634508in}}%
\pgfpathlineto{\pgfqpoint{2.831138in}{0.634508in}}%
\pgfpathlineto{\pgfqpoint{2.860421in}{0.634508in}}%
\pgfpathlineto{\pgfqpoint{2.888981in}{0.634508in}}%
\pgfpathlineto{\pgfqpoint{2.916853in}{0.634508in}}%
\pgfpathlineto{\pgfqpoint{2.944069in}{0.634508in}}%
\pgfpathlineto{\pgfqpoint{2.970660in}{0.634508in}}%
\pgfpathlineto{\pgfqpoint{2.996653in}{0.634508in}}%
\pgfpathlineto{\pgfqpoint{3.022075in}{0.634508in}}%
\pgfpathlineto{\pgfqpoint{3.046951in}{0.634508in}}%
\pgfpathlineto{\pgfqpoint{3.071303in}{0.634508in}}%
\pgfpathlineto{\pgfqpoint{3.095152in}{0.634508in}}%
\pgfpathlineto{\pgfqpoint{3.118521in}{0.634508in}}%
\pgfpathlineto{\pgfqpoint{3.141426in}{0.634508in}}%
\pgfpathlineto{\pgfqpoint{3.163887in}{0.634508in}}%
\pgfpathlineto{\pgfqpoint{3.185920in}{0.634508in}}%
\pgfpathlineto{\pgfqpoint{3.207541in}{0.634508in}}%
\pgfpathlineto{\pgfqpoint{3.228765in}{0.634508in}}%
\pgfpathlineto{\pgfqpoint{3.249607in}{0.634508in}}%
\pgfpathlineto{\pgfqpoint{3.270081in}{0.634508in}}%
\pgfpathlineto{\pgfqpoint{3.290198in}{0.634508in}}%
\pgfpathlineto{\pgfqpoint{3.309971in}{0.634508in}}%
\pgfpathlineto{\pgfqpoint{3.329412in}{0.634508in}}%
\pgfpathlineto{\pgfqpoint{3.348532in}{0.634508in}}%
\pgfpathlineto{\pgfqpoint{3.367341in}{0.634508in}}%
\pgfpathlineto{\pgfqpoint{3.385849in}{0.634508in}}%
\pgfpathlineto{\pgfqpoint{3.404066in}{0.634508in}}%
\pgfpathlineto{\pgfqpoint{3.422000in}{0.634508in}}%
\pgfpathlineto{\pgfqpoint{3.439661in}{0.634508in}}%
\pgfpathlineto{\pgfqpoint{3.457056in}{0.634508in}}%
\pgfpathlineto{\pgfqpoint{3.474193in}{0.634508in}}%
\pgfpathlineto{\pgfqpoint{3.491080in}{0.634508in}}%
\pgfpathlineto{\pgfqpoint{3.507724in}{0.634508in}}%
\pgfpathlineto{\pgfqpoint{3.524132in}{0.634508in}}%
\pgfpathlineto{\pgfqpoint{3.540311in}{0.634508in}}%
\pgfpathlineto{\pgfqpoint{3.556266in}{0.634508in}}%
\pgfpathlineto{\pgfqpoint{3.572005in}{0.634508in}}%
\pgfpathlineto{\pgfqpoint{3.587532in}{0.634508in}}%
\pgfpathlineto{\pgfqpoint{3.602854in}{0.634508in}}%
\pgfpathlineto{\pgfqpoint{3.617975in}{0.634508in}}%
\pgfpathlineto{\pgfqpoint{3.632901in}{0.634508in}}%
\pgfpathlineto{\pgfqpoint{3.647637in}{0.634508in}}%
\pgfpathlineto{\pgfqpoint{3.662188in}{0.634508in}}%
\pgfpathlineto{\pgfqpoint{3.676558in}{0.634508in}}%
\pgfpathlineto{\pgfqpoint{3.690752in}{0.634508in}}%
\pgfpathlineto{\pgfqpoint{3.704773in}{0.634508in}}%
\pgfpathlineto{\pgfqpoint{3.718627in}{0.634508in}}%
\pgfpathlineto{\pgfqpoint{3.732317in}{0.634508in}}%
\pgfpathlineto{\pgfqpoint{3.745847in}{0.634508in}}%
\pgfpathlineto{\pgfqpoint{3.759220in}{0.634508in}}%
\pgfpathlineto{\pgfqpoint{3.772441in}{0.634508in}}%
\pgfpathlineto{\pgfqpoint{3.785512in}{0.634508in}}%
\pgfpathlineto{\pgfqpoint{3.798437in}{0.634508in}}%
\pgfpathlineto{\pgfqpoint{3.811219in}{0.634508in}}%
\pgfpathlineto{\pgfqpoint{3.823862in}{0.634508in}}%
\pgfpathlineto{\pgfqpoint{3.836368in}{0.634508in}}%
\pgfpathlineto{\pgfqpoint{3.848740in}{0.634508in}}%
\pgfpathlineto{\pgfqpoint{3.860981in}{0.634508in}}%
\pgfpathlineto{\pgfqpoint{3.873095in}{0.634508in}}%
\pgfpathlineto{\pgfqpoint{3.885082in}{0.634508in}}%
\pgfpathlineto{\pgfqpoint{3.896947in}{0.634508in}}%
\pgfpathlineto{\pgfqpoint{3.908691in}{0.634508in}}%
\pgfpathlineto{\pgfqpoint{3.920317in}{0.634508in}}%
\pgfusepath{stroke}%
\end{pgfscope}%
\begin{pgfscope}%
\pgfpathrectangle{\pgfqpoint{0.466126in}{0.521603in}}{\pgfqpoint{3.720000in}{3.020000in}} %
\pgfusepath{clip}%
\pgfsetbuttcap%
\pgfsetroundjoin%
\pgfsetlinewidth{1.505625pt}%
\definecolor{currentstroke}{rgb}{0.501961,0.501961,0.501961}%
\pgfsetstrokecolor{currentstroke}%
\pgfsetdash{{9.600000pt}{2.400000pt}{1.500000pt}{2.400000pt}}{0.000000pt}%
\pgfpathmoveto{\pgfqpoint{0.456126in}{2.392640in}}%
\pgfpathlineto{\pgfqpoint{0.456740in}{2.392341in}}%
\pgfpathlineto{\pgfqpoint{0.667278in}{2.289608in}}%
\pgfpathlineto{\pgfqpoint{0.845337in}{2.202724in}}%
\pgfpathlineto{\pgfqpoint{0.999612in}{2.127445in}}%
\pgfpathlineto{\pgfqpoint{1.135713in}{2.061034in}}%
\pgfpathlineto{\pgfqpoint{1.257476in}{2.001619in}}%
\pgfpathlineto{\pgfqpoint{1.367635in}{1.947867in}}%
\pgfpathlineto{\pgfqpoint{1.468210in}{1.898791in}}%
\pgfpathlineto{\pgfqpoint{1.560738in}{1.853642in}}%
\pgfpathlineto{\pgfqpoint{1.646410in}{1.811838in}}%
\pgfpathlineto{\pgfqpoint{1.726173in}{1.772917in}}%
\pgfpathlineto{\pgfqpoint{1.800789in}{1.736508in}}%
\pgfpathlineto{\pgfqpoint{1.870884in}{1.702305in}}%
\pgfpathlineto{\pgfqpoint{1.936973in}{1.670057in}}%
\pgfpathlineto{\pgfqpoint{1.999490in}{1.639552in}}%
\pgfpathlineto{\pgfqpoint{2.058801in}{1.610611in}}%
\pgfpathlineto{\pgfqpoint{2.115219in}{1.583081in}}%
\pgfpathlineto{\pgfqpoint{2.169013in}{1.556832in}}%
\pgfpathlineto{\pgfqpoint{2.220417in}{1.531750in}}%
\pgfpathlineto{\pgfqpoint{2.269634in}{1.507734in}}%
\pgfpathlineto{\pgfqpoint{2.316842in}{1.484699in}}%
\pgfpathlineto{\pgfqpoint{2.362199in}{1.462567in}}%
\pgfpathlineto{\pgfqpoint{2.405844in}{1.441270in}}%
\pgfpathlineto{\pgfqpoint{2.447903in}{1.420747in}}%
\pgfpathlineto{\pgfqpoint{2.488486in}{1.400944in}}%
\pgfpathlineto{\pgfqpoint{2.527694in}{1.381813in}}%
\pgfpathlineto{\pgfqpoint{2.565617in}{1.363308in}}%
\pgfpathlineto{\pgfqpoint{2.602335in}{1.345391in}}%
\pgfpathlineto{\pgfqpoint{2.637925in}{1.328026in}}%
\pgfpathlineto{\pgfqpoint{2.672452in}{1.311178in}}%
\pgfpathlineto{\pgfqpoint{2.705978in}{1.294819in}}%
\pgfpathlineto{\pgfqpoint{2.738560in}{1.278920in}}%
\pgfpathlineto{\pgfqpoint{2.770249in}{1.263457in}}%
\pgfpathlineto{\pgfqpoint{2.801094in}{1.248407in}}%
\pgfpathlineto{\pgfqpoint{2.831138in}{1.233747in}}%
\pgfpathlineto{\pgfqpoint{2.860421in}{1.219458in}}%
\pgfpathlineto{\pgfqpoint{2.888981in}{1.205522in}}%
\pgfpathlineto{\pgfqpoint{2.916853in}{1.191922in}}%
\pgfpathlineto{\pgfqpoint{2.944069in}{1.178642in}}%
\pgfpathlineto{\pgfqpoint{2.970660in}{1.165667in}}%
\pgfpathlineto{\pgfqpoint{2.996653in}{1.152983in}}%
\pgfpathlineto{\pgfqpoint{3.022075in}{1.140578in}}%
\pgfpathlineto{\pgfqpoint{3.046951in}{1.128440in}}%
\pgfpathlineto{\pgfqpoint{3.071303in}{1.116558in}}%
\pgfpathlineto{\pgfqpoint{3.095152in}{1.104920in}}%
\pgfpathlineto{\pgfqpoint{3.118521in}{1.093518in}}%
\pgfpathlineto{\pgfqpoint{3.141426in}{1.082341in}}%
\pgfpathlineto{\pgfqpoint{3.163887in}{1.071381in}}%
\pgfpathlineto{\pgfqpoint{3.185920in}{1.060630in}}%
\pgfpathlineto{\pgfqpoint{3.207541in}{1.050080in}}%
\pgfpathlineto{\pgfqpoint{3.228765in}{1.039723in}}%
\pgfpathlineto{\pgfqpoint{3.249607in}{1.029553in}}%
\pgfpathlineto{\pgfqpoint{3.270081in}{1.019563in}}%
\pgfpathlineto{\pgfqpoint{3.290198in}{1.009747in}}%
\pgfpathlineto{\pgfqpoint{3.309971in}{1.000099in}}%
\pgfpathlineto{\pgfqpoint{3.329412in}{0.990612in}}%
\pgfpathlineto{\pgfqpoint{3.348532in}{0.981283in}}%
\pgfpathlineto{\pgfqpoint{3.367341in}{0.972105in}}%
\pgfpathlineto{\pgfqpoint{3.385849in}{0.963074in}}%
\pgfpathlineto{\pgfqpoint{3.404066in}{0.954185in}}%
\pgfpathlineto{\pgfqpoint{3.422000in}{0.945434in}}%
\pgfpathlineto{\pgfqpoint{3.439661in}{0.936816in}}%
\pgfpathlineto{\pgfqpoint{3.457056in}{0.928328in}}%
\pgfpathlineto{\pgfqpoint{3.474193in}{0.919966in}}%
\pgfpathlineto{\pgfqpoint{3.491080in}{0.911726in}}%
\pgfpathlineto{\pgfqpoint{3.507724in}{0.903605in}}%
\pgfpathlineto{\pgfqpoint{3.524132in}{0.895598in}}%
\pgfpathlineto{\pgfqpoint{3.540311in}{0.887704in}}%
\pgfpathlineto{\pgfqpoint{3.556266in}{0.879918in}}%
\pgfpathlineto{\pgfqpoint{3.572005in}{0.872239in}}%
\pgfpathlineto{\pgfqpoint{3.587532in}{0.864662in}}%
\pgfpathlineto{\pgfqpoint{3.602854in}{0.857186in}}%
\pgfpathlineto{\pgfqpoint{3.617975in}{0.849807in}}%
\pgfpathlineto{\pgfqpoint{3.632901in}{0.842524in}}%
\pgfpathlineto{\pgfqpoint{3.647637in}{0.835334in}}%
\pgfpathlineto{\pgfqpoint{3.662188in}{0.828234in}}%
\pgfpathlineto{\pgfqpoint{3.676558in}{0.821222in}}%
\pgfpathlineto{\pgfqpoint{3.690752in}{0.814296in}}%
\pgfpathlineto{\pgfqpoint{3.704773in}{0.807454in}}%
\pgfpathlineto{\pgfqpoint{3.718627in}{0.800694in}}%
\pgfpathlineto{\pgfqpoint{3.732317in}{0.794014in}}%
\pgfpathlineto{\pgfqpoint{3.745847in}{0.787412in}}%
\pgfpathlineto{\pgfqpoint{3.759220in}{0.780886in}}%
\pgfpathlineto{\pgfqpoint{3.772441in}{0.774435in}}%
\pgfpathlineto{\pgfqpoint{3.785512in}{0.768057in}}%
\pgfpathlineto{\pgfqpoint{3.798437in}{0.761751in}}%
\pgfpathlineto{\pgfqpoint{3.811219in}{0.755513in}}%
\pgfpathlineto{\pgfqpoint{3.823862in}{0.749344in}}%
\pgfpathlineto{\pgfqpoint{3.836368in}{0.743242in}}%
\pgfpathlineto{\pgfqpoint{3.848740in}{0.737205in}}%
\pgfpathlineto{\pgfqpoint{3.860981in}{0.731232in}}%
\pgfpathlineto{\pgfqpoint{3.873095in}{0.725321in}}%
\pgfpathlineto{\pgfqpoint{3.885082in}{0.719472in}}%
\pgfpathlineto{\pgfqpoint{3.896947in}{0.713682in}}%
\pgfpathlineto{\pgfqpoint{3.908691in}{0.707952in}}%
\pgfpathlineto{\pgfqpoint{3.920317in}{0.702279in}}%
\pgfusepath{stroke}%
\end{pgfscope}%
\begin{pgfscope}%
\pgfpathrectangle{\pgfqpoint{0.466126in}{0.521603in}}{\pgfqpoint{3.720000in}{3.020000in}} %
\pgfusepath{clip}%
\pgfsetrectcap%
\pgfsetroundjoin%
\pgfsetlinewidth{1.505625pt}%
\definecolor{currentstroke}{rgb}{0.174510,0.872120,0.862929}%
\pgfsetstrokecolor{currentstroke}%
\pgfsetdash{}{0pt}%
\pgfpathmoveto{\pgfqpoint{0.456126in}{1.360815in}}%
\pgfpathlineto{\pgfqpoint{0.456740in}{1.360702in}}%
\pgfpathlineto{\pgfqpoint{0.667278in}{1.323228in}}%
\pgfpathlineto{\pgfqpoint{0.845337in}{1.292751in}}%
\pgfpathlineto{\pgfqpoint{0.999612in}{1.267322in}}%
\pgfpathlineto{\pgfqpoint{1.135713in}{1.245679in}}%
\pgfpathlineto{\pgfqpoint{1.257476in}{1.226964in}}%
\pgfpathlineto{\pgfqpoint{1.367635in}{1.210570in}}%
\pgfpathlineto{\pgfqpoint{1.468210in}{1.196051in}}%
\pgfpathlineto{\pgfqpoint{1.560738in}{1.183076in}}%
\pgfpathlineto{\pgfqpoint{1.646410in}{1.171387in}}%
\pgfpathlineto{\pgfqpoint{1.726173in}{1.160786in}}%
\pgfpathlineto{\pgfqpoint{1.800789in}{1.151113in}}%
\pgfpathlineto{\pgfqpoint{1.870884in}{1.142240in}}%
\pgfpathlineto{\pgfqpoint{1.936973in}{1.134062in}}%
\pgfpathlineto{\pgfqpoint{1.999490in}{1.126493in}}%
\pgfpathlineto{\pgfqpoint{2.058801in}{1.119462in}}%
\pgfpathlineto{\pgfqpoint{2.115219in}{1.112906in}}%
\pgfpathlineto{\pgfqpoint{2.169013in}{1.106775in}}%
\pgfpathlineto{\pgfqpoint{2.220417in}{1.101025in}}%
\pgfpathlineto{\pgfqpoint{2.269634in}{1.095617in}}%
\pgfpathlineto{\pgfqpoint{2.316842in}{1.090520in}}%
\pgfpathlineto{\pgfqpoint{2.362199in}{1.085704in}}%
\pgfpathlineto{\pgfqpoint{2.405844in}{1.081145in}}%
\pgfpathlineto{\pgfqpoint{2.447903in}{1.076820in}}%
\pgfpathlineto{\pgfqpoint{2.488486in}{1.072710in}}%
\pgfpathlineto{\pgfqpoint{2.527694in}{1.068798in}}%
\pgfpathlineto{\pgfqpoint{2.565617in}{1.065069in}}%
\pgfpathlineto{\pgfqpoint{2.602335in}{1.061508in}}%
\pgfpathlineto{\pgfqpoint{2.637925in}{1.058103in}}%
\pgfpathlineto{\pgfqpoint{2.672452in}{1.054844in}}%
\pgfpathlineto{\pgfqpoint{2.705978in}{1.051719in}}%
\pgfpathlineto{\pgfqpoint{2.738560in}{1.048721in}}%
\pgfpathlineto{\pgfqpoint{2.770249in}{1.045841in}}%
\pgfpathlineto{\pgfqpoint{2.801094in}{1.043071in}}%
\pgfpathlineto{\pgfqpoint{2.831138in}{1.040404in}}%
\pgfpathlineto{\pgfqpoint{2.860421in}{1.037835in}}%
\pgfpathlineto{\pgfqpoint{2.888981in}{1.035357in}}%
\pgfpathlineto{\pgfqpoint{2.916853in}{1.032965in}}%
\pgfpathlineto{\pgfqpoint{2.944069in}{1.030654in}}%
\pgfpathlineto{\pgfqpoint{2.970660in}{1.028420in}}%
\pgfpathlineto{\pgfqpoint{2.996653in}{1.026258in}}%
\pgfpathlineto{\pgfqpoint{3.022075in}{1.024165in}}%
\pgfpathlineto{\pgfqpoint{3.046951in}{1.022138in}}%
\pgfpathlineto{\pgfqpoint{3.071303in}{1.020172in}}%
\pgfpathlineto{\pgfqpoint{3.095152in}{1.018265in}}%
\pgfpathlineto{\pgfqpoint{3.118521in}{1.016414in}}%
\pgfpathlineto{\pgfqpoint{3.141426in}{1.014616in}}%
\pgfpathlineto{\pgfqpoint{3.163887in}{1.012869in}}%
\pgfpathlineto{\pgfqpoint{3.185920in}{1.011170in}}%
\pgfpathlineto{\pgfqpoint{3.207541in}{1.009517in}}%
\pgfpathlineto{\pgfqpoint{3.228765in}{1.007908in}}%
\pgfpathlineto{\pgfqpoint{3.249607in}{1.006342in}}%
\pgfpathlineto{\pgfqpoint{3.270081in}{1.004816in}}%
\pgfpathlineto{\pgfqpoint{3.290198in}{1.003329in}}%
\pgfpathlineto{\pgfqpoint{3.309971in}{1.001879in}}%
\pgfpathlineto{\pgfqpoint{3.329412in}{1.000464in}}%
\pgfpathlineto{\pgfqpoint{3.348532in}{0.999083in}}%
\pgfpathlineto{\pgfqpoint{3.367341in}{0.997736in}}%
\pgfpathlineto{\pgfqpoint{3.385849in}{0.996419in}}%
\pgfpathlineto{\pgfqpoint{3.404066in}{0.995133in}}%
\pgfpathlineto{\pgfqpoint{3.422000in}{0.993876in}}%
\pgfpathlineto{\pgfqpoint{3.439661in}{0.992648in}}%
\pgfpathlineto{\pgfqpoint{3.457056in}{0.991446in}}%
\pgfpathlineto{\pgfqpoint{3.474193in}{0.990270in}}%
\pgfpathlineto{\pgfqpoint{3.491080in}{0.989119in}}%
\pgfpathlineto{\pgfqpoint{3.507724in}{0.987993in}}%
\pgfpathlineto{\pgfqpoint{3.524132in}{0.986890in}}%
\pgfpathlineto{\pgfqpoint{3.540311in}{0.985810in}}%
\pgfpathlineto{\pgfqpoint{3.556266in}{0.984751in}}%
\pgfpathlineto{\pgfqpoint{3.572005in}{0.983714in}}%
\pgfpathlineto{\pgfqpoint{3.587532in}{0.982697in}}%
\pgfpathlineto{\pgfqpoint{3.602854in}{0.981700in}}%
\pgfpathlineto{\pgfqpoint{3.617975in}{0.980722in}}%
\pgfpathlineto{\pgfqpoint{3.632901in}{0.979762in}}%
\pgfpathlineto{\pgfqpoint{3.647637in}{0.978821in}}%
\pgfpathlineto{\pgfqpoint{3.662188in}{0.977896in}}%
\pgfpathlineto{\pgfqpoint{3.676558in}{0.976989in}}%
\pgfpathlineto{\pgfqpoint{3.690752in}{0.976098in}}%
\pgfpathlineto{\pgfqpoint{3.704773in}{0.975223in}}%
\pgfpathlineto{\pgfqpoint{3.718627in}{0.974363in}}%
\pgfpathlineto{\pgfqpoint{3.732317in}{0.973519in}}%
\pgfpathlineto{\pgfqpoint{3.745847in}{0.972688in}}%
\pgfpathlineto{\pgfqpoint{3.759220in}{0.971872in}}%
\pgfpathlineto{\pgfqpoint{3.772441in}{0.971070in}}%
\pgfpathlineto{\pgfqpoint{3.785512in}{0.970281in}}%
\pgfpathlineto{\pgfqpoint{3.798437in}{0.969505in}}%
\pgfpathlineto{\pgfqpoint{3.811219in}{0.968742in}}%
\pgfpathlineto{\pgfqpoint{3.823862in}{0.967991in}}%
\pgfpathlineto{\pgfqpoint{3.836368in}{0.967251in}}%
\pgfpathlineto{\pgfqpoint{3.848740in}{0.966524in}}%
\pgfpathlineto{\pgfqpoint{3.860981in}{0.965808in}}%
\pgfpathlineto{\pgfqpoint{3.873095in}{0.965103in}}%
\pgfpathlineto{\pgfqpoint{3.885082in}{0.964408in}}%
\pgfpathlineto{\pgfqpoint{3.896947in}{0.963724in}}%
\pgfpathlineto{\pgfqpoint{3.908691in}{0.963051in}}%
\pgfpathlineto{\pgfqpoint{3.920317in}{0.962387in}}%
\pgfusepath{stroke}%
\end{pgfscope}%
\begin{pgfscope}%
\pgfpathrectangle{\pgfqpoint{0.466126in}{0.521603in}}{\pgfqpoint{3.720000in}{3.020000in}} %
\pgfusepath{clip}%
\pgfsetrectcap%
\pgfsetroundjoin%
\pgfsetlinewidth{1.505625pt}%
\definecolor{currentstroke}{rgb}{0.174510,0.872120,0.862929}%
\pgfsetstrokecolor{currentstroke}%
\pgfsetdash{}{0pt}%
\pgfpathmoveto{\pgfqpoint{0.456126in}{1.298707in}}%
\pgfpathlineto{\pgfqpoint{0.456740in}{1.298599in}}%
\pgfpathlineto{\pgfqpoint{0.667278in}{1.261802in}}%
\pgfpathlineto{\pgfqpoint{0.845337in}{1.230924in}}%
\pgfpathlineto{\pgfqpoint{0.999612in}{1.204496in}}%
\pgfpathlineto{\pgfqpoint{1.135713in}{1.181522in}}%
\pgfpathlineto{\pgfqpoint{1.257476in}{1.161297in}}%
\pgfpathlineto{\pgfqpoint{1.367635in}{1.143307in}}%
\pgfpathlineto{\pgfqpoint{1.468210in}{1.127164in}}%
\pgfpathlineto{\pgfqpoint{1.560738in}{1.112568in}}%
\pgfpathlineto{\pgfqpoint{1.646410in}{1.099285in}}%
\pgfpathlineto{\pgfqpoint{1.726173in}{1.087128in}}%
\pgfpathlineto{\pgfqpoint{1.800789in}{1.075945in}}%
\pgfpathlineto{\pgfqpoint{1.870884in}{1.065612in}}%
\pgfpathlineto{\pgfqpoint{1.936973in}{1.056026in}}%
\pgfpathlineto{\pgfqpoint{1.999490in}{1.047100in}}%
\pgfpathlineto{\pgfqpoint{2.058801in}{1.038762in}}%
\pgfpathlineto{\pgfqpoint{2.115219in}{1.030949in}}%
\pgfpathlineto{\pgfqpoint{2.169013in}{1.023609in}}%
\pgfpathlineto{\pgfqpoint{2.220417in}{1.016695in}}%
\pgfpathlineto{\pgfqpoint{2.269634in}{1.010168in}}%
\pgfpathlineto{\pgfqpoint{2.316842in}{1.003993in}}%
\pgfpathlineto{\pgfqpoint{2.362199in}{0.998138in}}%
\pgfpathlineto{\pgfqpoint{2.405844in}{0.992578in}}%
\pgfpathlineto{\pgfqpoint{2.447903in}{0.987288in}}%
\pgfpathlineto{\pgfqpoint{2.488486in}{0.982247in}}%
\pgfpathlineto{\pgfqpoint{2.527694in}{0.977437in}}%
\pgfpathlineto{\pgfqpoint{2.565617in}{0.972839in}}%
\pgfpathlineto{\pgfqpoint{2.602335in}{0.968439in}}%
\pgfpathlineto{\pgfqpoint{2.637925in}{0.964222in}}%
\pgfpathlineto{\pgfqpoint{2.672452in}{0.960178in}}%
\pgfpathlineto{\pgfqpoint{2.705978in}{0.956293in}}%
\pgfpathlineto{\pgfqpoint{2.738560in}{0.952558in}}%
\pgfpathlineto{\pgfqpoint{2.770249in}{0.948964in}}%
\pgfpathlineto{\pgfqpoint{2.801094in}{0.945501in}}%
\pgfpathlineto{\pgfqpoint{2.831138in}{0.942162in}}%
\pgfpathlineto{\pgfqpoint{2.860421in}{0.938940in}}%
\pgfpathlineto{\pgfqpoint{2.888981in}{0.935828in}}%
\pgfpathlineto{\pgfqpoint{2.916853in}{0.932820in}}%
\pgfpathlineto{\pgfqpoint{2.944069in}{0.929910in}}%
\pgfpathlineto{\pgfqpoint{2.970660in}{0.927093in}}%
\pgfpathlineto{\pgfqpoint{2.996653in}{0.924365in}}%
\pgfpathlineto{\pgfqpoint{3.022075in}{0.921720in}}%
\pgfpathlineto{\pgfqpoint{3.046951in}{0.919154in}}%
\pgfpathlineto{\pgfqpoint{3.071303in}{0.916664in}}%
\pgfpathlineto{\pgfqpoint{3.095152in}{0.914246in}}%
\pgfpathlineto{\pgfqpoint{3.118521in}{0.911896in}}%
\pgfpathlineto{\pgfqpoint{3.141426in}{0.909612in}}%
\pgfpathlineto{\pgfqpoint{3.163887in}{0.907390in}}%
\pgfpathlineto{\pgfqpoint{3.185920in}{0.905228in}}%
\pgfpathlineto{\pgfqpoint{3.207541in}{0.903122in}}%
\pgfpathlineto{\pgfqpoint{3.228765in}{0.901072in}}%
\pgfpathlineto{\pgfqpoint{3.249607in}{0.899073in}}%
\pgfpathlineto{\pgfqpoint{3.270081in}{0.897124in}}%
\pgfpathlineto{\pgfqpoint{3.290198in}{0.895223in}}%
\pgfpathlineto{\pgfqpoint{3.309971in}{0.893369in}}%
\pgfpathlineto{\pgfqpoint{3.329412in}{0.891558in}}%
\pgfpathlineto{\pgfqpoint{3.348532in}{0.889790in}}%
\pgfpathlineto{\pgfqpoint{3.367341in}{0.888063in}}%
\pgfpathlineto{\pgfqpoint{3.385849in}{0.886375in}}%
\pgfpathlineto{\pgfqpoint{3.404066in}{0.884725in}}%
\pgfpathlineto{\pgfqpoint{3.422000in}{0.883111in}}%
\pgfpathlineto{\pgfqpoint{3.439661in}{0.881532in}}%
\pgfpathlineto{\pgfqpoint{3.457056in}{0.879987in}}%
\pgfpathlineto{\pgfqpoint{3.474193in}{0.878475in}}%
\pgfpathlineto{\pgfqpoint{3.491080in}{0.876994in}}%
\pgfpathlineto{\pgfqpoint{3.507724in}{0.875544in}}%
\pgfpathlineto{\pgfqpoint{3.524132in}{0.874123in}}%
\pgfpathlineto{\pgfqpoint{3.540311in}{0.872731in}}%
\pgfpathlineto{\pgfqpoint{3.556266in}{0.871366in}}%
\pgfpathlineto{\pgfqpoint{3.572005in}{0.870028in}}%
\pgfpathlineto{\pgfqpoint{3.587532in}{0.868715in}}%
\pgfpathlineto{\pgfqpoint{3.602854in}{0.867428in}}%
\pgfpathlineto{\pgfqpoint{3.617975in}{0.866164in}}%
\pgfpathlineto{\pgfqpoint{3.632901in}{0.864924in}}%
\pgfpathlineto{\pgfqpoint{3.647637in}{0.863707in}}%
\pgfpathlineto{\pgfqpoint{3.662188in}{0.862511in}}%
\pgfpathlineto{\pgfqpoint{3.676558in}{0.861337in}}%
\pgfpathlineto{\pgfqpoint{3.690752in}{0.860184in}}%
\pgfpathlineto{\pgfqpoint{3.704773in}{0.859051in}}%
\pgfpathlineto{\pgfqpoint{3.718627in}{0.857937in}}%
\pgfpathlineto{\pgfqpoint{3.732317in}{0.856842in}}%
\pgfpathlineto{\pgfqpoint{3.745847in}{0.855766in}}%
\pgfpathlineto{\pgfqpoint{3.759220in}{0.854708in}}%
\pgfpathlineto{\pgfqpoint{3.772441in}{0.853667in}}%
\pgfpathlineto{\pgfqpoint{3.785512in}{0.852643in}}%
\pgfpathlineto{\pgfqpoint{3.798437in}{0.851636in}}%
\pgfpathlineto{\pgfqpoint{3.811219in}{0.850645in}}%
\pgfpathlineto{\pgfqpoint{3.823862in}{0.849669in}}%
\pgfpathlineto{\pgfqpoint{3.836368in}{0.848709in}}%
\pgfpathlineto{\pgfqpoint{3.848740in}{0.847763in}}%
\pgfpathlineto{\pgfqpoint{3.860981in}{0.846832in}}%
\pgfpathlineto{\pgfqpoint{3.873095in}{0.845915in}}%
\pgfpathlineto{\pgfqpoint{3.885082in}{0.845012in}}%
\pgfpathlineto{\pgfqpoint{3.896947in}{0.844122in}}%
\pgfpathlineto{\pgfqpoint{3.908691in}{0.843246in}}%
\pgfpathlineto{\pgfqpoint{3.920317in}{0.842382in}}%
\pgfusepath{stroke}%
\end{pgfscope}%
\begin{pgfscope}%
\pgfpathrectangle{\pgfqpoint{0.466126in}{0.521603in}}{\pgfqpoint{3.720000in}{3.020000in}} %
\pgfusepath{clip}%
\pgfsetrectcap%
\pgfsetroundjoin%
\pgfsetlinewidth{1.505625pt}%
\definecolor{currentstroke}{rgb}{0.582353,0.991645,0.659925}%
\pgfsetstrokecolor{currentstroke}%
\pgfsetdash{}{0pt}%
\pgfpathmoveto{\pgfqpoint{0.456126in}{1.402729in}}%
\pgfpathlineto{\pgfqpoint{0.456740in}{1.402620in}}%
\pgfpathlineto{\pgfqpoint{0.667278in}{1.364742in}}%
\pgfpathlineto{\pgfqpoint{0.845337in}{1.332654in}}%
\pgfpathlineto{\pgfqpoint{0.999612in}{1.304970in}}%
\pgfpathlineto{\pgfqpoint{1.135713in}{1.280737in}}%
\pgfpathlineto{\pgfqpoint{1.257476in}{1.259278in}}%
\pgfpathlineto{\pgfqpoint{1.367635in}{1.240090in}}%
\pgfpathlineto{\pgfqpoint{1.468210in}{1.222792in}}%
\pgfpathlineto{\pgfqpoint{1.560738in}{1.207089in}}%
\pgfpathlineto{\pgfqpoint{1.646410in}{1.192746in}}%
\pgfpathlineto{\pgfqpoint{1.726173in}{1.179575in}}%
\pgfpathlineto{\pgfqpoint{1.800789in}{1.167424in}}%
\pgfpathlineto{\pgfqpoint{1.870884in}{1.156165in}}%
\pgfpathlineto{\pgfqpoint{1.936973in}{1.145695in}}%
\pgfpathlineto{\pgfqpoint{1.999490in}{1.135923in}}%
\pgfpathlineto{\pgfqpoint{2.058801in}{1.126776in}}%
\pgfpathlineto{\pgfqpoint{2.115219in}{1.118189in}}%
\pgfpathlineto{\pgfqpoint{2.169013in}{1.110107in}}%
\pgfpathlineto{\pgfqpoint{2.220417in}{1.102482in}}%
\pgfpathlineto{\pgfqpoint{2.269634in}{1.095272in}}%
\pgfpathlineto{\pgfqpoint{2.316842in}{1.088441in}}%
\pgfpathlineto{\pgfqpoint{2.362199in}{1.081956in}}%
\pgfpathlineto{\pgfqpoint{2.405844in}{1.075789in}}%
\pgfpathlineto{\pgfqpoint{2.447903in}{1.069916in}}%
\pgfpathlineto{\pgfqpoint{2.488486in}{1.064312in}}%
\pgfpathlineto{\pgfqpoint{2.527694in}{1.058959in}}%
\pgfpathlineto{\pgfqpoint{2.565617in}{1.053837in}}%
\pgfpathlineto{\pgfqpoint{2.602335in}{1.048931in}}%
\pgfpathlineto{\pgfqpoint{2.637925in}{1.044226in}}%
\pgfpathlineto{\pgfqpoint{2.672452in}{1.039709in}}%
\pgfpathlineto{\pgfqpoint{2.705978in}{1.035366in}}%
\pgfpathlineto{\pgfqpoint{2.738560in}{1.031188in}}%
\pgfpathlineto{\pgfqpoint{2.770249in}{1.027164in}}%
\pgfpathlineto{\pgfqpoint{2.801094in}{1.023285in}}%
\pgfpathlineto{\pgfqpoint{2.831138in}{1.019543in}}%
\pgfpathlineto{\pgfqpoint{2.860421in}{1.015929in}}%
\pgfpathlineto{\pgfqpoint{2.888981in}{1.012436in}}%
\pgfpathlineto{\pgfqpoint{2.916853in}{1.009058in}}%
\pgfpathlineto{\pgfqpoint{2.944069in}{1.005788in}}%
\pgfpathlineto{\pgfqpoint{2.970660in}{1.002621in}}%
\pgfpathlineto{\pgfqpoint{2.996653in}{0.999552in}}%
\pgfpathlineto{\pgfqpoint{3.022075in}{0.996575in}}%
\pgfpathlineto{\pgfqpoint{3.046951in}{0.993687in}}%
\pgfpathlineto{\pgfqpoint{3.071303in}{0.990882in}}%
\pgfpathlineto{\pgfqpoint{3.095152in}{0.988157in}}%
\pgfpathlineto{\pgfqpoint{3.118521in}{0.985508in}}%
\pgfpathlineto{\pgfqpoint{3.141426in}{0.982932in}}%
\pgfpathlineto{\pgfqpoint{3.163887in}{0.980425in}}%
\pgfpathlineto{\pgfqpoint{3.185920in}{0.977984in}}%
\pgfpathlineto{\pgfqpoint{3.207541in}{0.975607in}}%
\pgfpathlineto{\pgfqpoint{3.228765in}{0.973291in}}%
\pgfpathlineto{\pgfqpoint{3.249607in}{0.971032in}}%
\pgfpathlineto{\pgfqpoint{3.270081in}{0.968830in}}%
\pgfpathlineto{\pgfqpoint{3.290198in}{0.966681in}}%
\pgfpathlineto{\pgfqpoint{3.309971in}{0.964583in}}%
\pgfpathlineto{\pgfqpoint{3.329412in}{0.962535in}}%
\pgfpathlineto{\pgfqpoint{3.348532in}{0.960534in}}%
\pgfpathlineto{\pgfqpoint{3.367341in}{0.958579in}}%
\pgfpathlineto{\pgfqpoint{3.385849in}{0.956667in}}%
\pgfpathlineto{\pgfqpoint{3.404066in}{0.954798in}}%
\pgfpathlineto{\pgfqpoint{3.422000in}{0.952970in}}%
\pgfpathlineto{\pgfqpoint{3.439661in}{0.951181in}}%
\pgfpathlineto{\pgfqpoint{3.457056in}{0.949430in}}%
\pgfpathlineto{\pgfqpoint{3.474193in}{0.947716in}}%
\pgfpathlineto{\pgfqpoint{3.491080in}{0.946037in}}%
\pgfpathlineto{\pgfqpoint{3.507724in}{0.944392in}}%
\pgfpathlineto{\pgfqpoint{3.524132in}{0.942780in}}%
\pgfpathlineto{\pgfqpoint{3.540311in}{0.941200in}}%
\pgfpathlineto{\pgfqpoint{3.556266in}{0.939651in}}%
\pgfpathlineto{\pgfqpoint{3.572005in}{0.938132in}}%
\pgfpathlineto{\pgfqpoint{3.587532in}{0.936641in}}%
\pgfpathlineto{\pgfqpoint{3.602854in}{0.935179in}}%
\pgfpathlineto{\pgfqpoint{3.617975in}{0.933744in}}%
\pgfpathlineto{\pgfqpoint{3.632901in}{0.932335in}}%
\pgfpathlineto{\pgfqpoint{3.647637in}{0.930952in}}%
\pgfpathlineto{\pgfqpoint{3.662188in}{0.929593in}}%
\pgfpathlineto{\pgfqpoint{3.676558in}{0.928259in}}%
\pgfpathlineto{\pgfqpoint{3.690752in}{0.926948in}}%
\pgfpathlineto{\pgfqpoint{3.704773in}{0.925659in}}%
\pgfpathlineto{\pgfqpoint{3.718627in}{0.924393in}}%
\pgfpathlineto{\pgfqpoint{3.732317in}{0.923147in}}%
\pgfpathlineto{\pgfqpoint{3.745847in}{0.921923in}}%
\pgfpathlineto{\pgfqpoint{3.759220in}{0.920719in}}%
\pgfpathlineto{\pgfqpoint{3.772441in}{0.919535in}}%
\pgfpathlineto{\pgfqpoint{3.785512in}{0.918370in}}%
\pgfpathlineto{\pgfqpoint{3.798437in}{0.917223in}}%
\pgfpathlineto{\pgfqpoint{3.811219in}{0.916094in}}%
\pgfpathlineto{\pgfqpoint{3.823862in}{0.914984in}}%
\pgfpathlineto{\pgfqpoint{3.836368in}{0.913890in}}%
\pgfpathlineto{\pgfqpoint{3.848740in}{0.912813in}}%
\pgfpathlineto{\pgfqpoint{3.860981in}{0.911753in}}%
\pgfpathlineto{\pgfqpoint{3.873095in}{0.910708in}}%
\pgfpathlineto{\pgfqpoint{3.885082in}{0.909679in}}%
\pgfpathlineto{\pgfqpoint{3.896947in}{0.908666in}}%
\pgfpathlineto{\pgfqpoint{3.908691in}{0.907667in}}%
\pgfpathlineto{\pgfqpoint{3.920317in}{0.906682in}}%
\pgfusepath{stroke}%
\end{pgfscope}%
\begin{pgfscope}%
\pgfpathrectangle{\pgfqpoint{0.466126in}{0.521603in}}{\pgfqpoint{3.720000in}{3.020000in}} %
\pgfusepath{clip}%
\pgfsetrectcap%
\pgfsetroundjoin%
\pgfsetlinewidth{1.505625pt}%
\definecolor{currentstroke}{rgb}{0.990196,0.717912,0.389786}%
\pgfsetstrokecolor{currentstroke}%
\pgfsetdash{}{0pt}%
\pgfpathmoveto{\pgfqpoint{0.456126in}{1.953597in}}%
\pgfpathlineto{\pgfqpoint{0.456740in}{1.953452in}}%
\pgfpathlineto{\pgfqpoint{0.667278in}{1.903913in}}%
\pgfpathlineto{\pgfqpoint{0.845337in}{1.862388in}}%
\pgfpathlineto{\pgfqpoint{0.999612in}{1.826879in}}%
\pgfpathlineto{\pgfqpoint{1.135713in}{1.796035in}}%
\pgfpathlineto{\pgfqpoint{1.257476in}{1.768901in}}%
\pgfpathlineto{\pgfqpoint{1.367635in}{1.744779in}}%
\pgfpathlineto{\pgfqpoint{1.468210in}{1.723143in}}%
\pgfpathlineto{\pgfqpoint{1.560738in}{1.703591in}}%
\pgfpathlineto{\pgfqpoint{1.646410in}{1.685805in}}%
\pgfpathlineto{\pgfqpoint{1.726173in}{1.669532in}}%
\pgfpathlineto{\pgfqpoint{1.800789in}{1.654567in}}%
\pgfpathlineto{\pgfqpoint{1.870884in}{1.640744in}}%
\pgfpathlineto{\pgfqpoint{1.936973in}{1.627924in}}%
\pgfpathlineto{\pgfqpoint{1.999490in}{1.615989in}}%
\pgfpathlineto{\pgfqpoint{2.058801in}{1.604843in}}%
\pgfpathlineto{\pgfqpoint{2.115219in}{1.594402in}}%
\pgfpathlineto{\pgfqpoint{2.169013in}{1.584594in}}%
\pgfpathlineto{\pgfqpoint{2.220417in}{1.575357in}}%
\pgfpathlineto{\pgfqpoint{2.269634in}{1.566638in}}%
\pgfpathlineto{\pgfqpoint{2.316842in}{1.558390in}}%
\pgfpathlineto{\pgfqpoint{2.362199in}{1.550573in}}%
\pgfpathlineto{\pgfqpoint{2.405844in}{1.543149in}}%
\pgfpathlineto{\pgfqpoint{2.447903in}{1.536087in}}%
\pgfpathlineto{\pgfqpoint{2.488486in}{1.529358in}}%
\pgfpathlineto{\pgfqpoint{2.527694in}{1.522937in}}%
\pgfpathlineto{\pgfqpoint{2.565617in}{1.516801in}}%
\pgfpathlineto{\pgfqpoint{2.602335in}{1.510929in}}%
\pgfpathlineto{\pgfqpoint{2.637925in}{1.505303in}}%
\pgfpathlineto{\pgfqpoint{2.672452in}{1.499907in}}%
\pgfpathlineto{\pgfqpoint{2.705978in}{1.494724in}}%
\pgfpathlineto{\pgfqpoint{2.738560in}{1.489742in}}%
\pgfpathlineto{\pgfqpoint{2.770249in}{1.484947in}}%
\pgfpathlineto{\pgfqpoint{2.801094in}{1.480329in}}%
\pgfpathlineto{\pgfqpoint{2.831138in}{1.475876in}}%
\pgfpathlineto{\pgfqpoint{2.860421in}{1.471579in}}%
\pgfpathlineto{\pgfqpoint{2.888981in}{1.467429in}}%
\pgfpathlineto{\pgfqpoint{2.916853in}{1.463418in}}%
\pgfpathlineto{\pgfqpoint{2.944069in}{1.459538in}}%
\pgfpathlineto{\pgfqpoint{2.970660in}{1.455782in}}%
\pgfpathlineto{\pgfqpoint{2.996653in}{1.452144in}}%
\pgfpathlineto{\pgfqpoint{3.022075in}{1.448618in}}%
\pgfpathlineto{\pgfqpoint{3.046951in}{1.445198in}}%
\pgfpathlineto{\pgfqpoint{3.071303in}{1.441879in}}%
\pgfpathlineto{\pgfqpoint{3.095152in}{1.438655in}}%
\pgfpathlineto{\pgfqpoint{3.118521in}{1.435524in}}%
\pgfpathlineto{\pgfqpoint{3.141426in}{1.432479in}}%
\pgfpathlineto{\pgfqpoint{3.163887in}{1.429518in}}%
\pgfpathlineto{\pgfqpoint{3.185920in}{1.426636in}}%
\pgfpathlineto{\pgfqpoint{3.207541in}{1.423830in}}%
\pgfpathlineto{\pgfqpoint{3.228765in}{1.421097in}}%
\pgfpathlineto{\pgfqpoint{3.249607in}{1.418434in}}%
\pgfpathlineto{\pgfqpoint{3.270081in}{1.415837in}}%
\pgfpathlineto{\pgfqpoint{3.290198in}{1.413304in}}%
\pgfpathlineto{\pgfqpoint{3.309971in}{1.410833in}}%
\pgfpathlineto{\pgfqpoint{3.329412in}{1.408420in}}%
\pgfpathlineto{\pgfqpoint{3.348532in}{1.406064in}}%
\pgfpathlineto{\pgfqpoint{3.367341in}{1.403763in}}%
\pgfpathlineto{\pgfqpoint{3.385849in}{1.401514in}}%
\pgfpathlineto{\pgfqpoint{3.404066in}{1.399316in}}%
\pgfpathlineto{\pgfqpoint{3.422000in}{1.397166in}}%
\pgfpathlineto{\pgfqpoint{3.439661in}{1.395063in}}%
\pgfpathlineto{\pgfqpoint{3.457056in}{1.393005in}}%
\pgfpathlineto{\pgfqpoint{3.474193in}{1.390991in}}%
\pgfpathlineto{\pgfqpoint{3.491080in}{1.389018in}}%
\pgfpathlineto{\pgfqpoint{3.507724in}{1.387086in}}%
\pgfpathlineto{\pgfqpoint{3.524132in}{1.385194in}}%
\pgfpathlineto{\pgfqpoint{3.540311in}{1.383339in}}%
\pgfpathlineto{\pgfqpoint{3.556266in}{1.381521in}}%
\pgfpathlineto{\pgfqpoint{3.572005in}{1.379739in}}%
\pgfpathlineto{\pgfqpoint{3.587532in}{1.377990in}}%
\pgfpathlineto{\pgfqpoint{3.602854in}{1.376275in}}%
\pgfpathlineto{\pgfqpoint{3.617975in}{1.374593in}}%
\pgfpathlineto{\pgfqpoint{3.632901in}{1.372941in}}%
\pgfpathlineto{\pgfqpoint{3.647637in}{1.371320in}}%
\pgfpathlineto{\pgfqpoint{3.662188in}{1.369728in}}%
\pgfpathlineto{\pgfqpoint{3.676558in}{1.368164in}}%
\pgfpathlineto{\pgfqpoint{3.690752in}{1.366628in}}%
\pgfpathlineto{\pgfqpoint{3.704773in}{1.365119in}}%
\pgfpathlineto{\pgfqpoint{3.718627in}{1.363636in}}%
\pgfpathlineto{\pgfqpoint{3.732317in}{1.362178in}}%
\pgfpathlineto{\pgfqpoint{3.745847in}{1.360745in}}%
\pgfpathlineto{\pgfqpoint{3.759220in}{1.359336in}}%
\pgfpathlineto{\pgfqpoint{3.772441in}{1.357950in}}%
\pgfpathlineto{\pgfqpoint{3.785512in}{1.356587in}}%
\pgfpathlineto{\pgfqpoint{3.798437in}{1.355245in}}%
\pgfpathlineto{\pgfqpoint{3.811219in}{1.353925in}}%
\pgfpathlineto{\pgfqpoint{3.823862in}{1.352626in}}%
\pgfpathlineto{\pgfqpoint{3.836368in}{1.351348in}}%
\pgfpathlineto{\pgfqpoint{3.848740in}{1.350089in}}%
\pgfpathlineto{\pgfqpoint{3.860981in}{1.348849in}}%
\pgfpathlineto{\pgfqpoint{3.873095in}{1.347628in}}%
\pgfpathlineto{\pgfqpoint{3.885082in}{1.346426in}}%
\pgfpathlineto{\pgfqpoint{3.896947in}{1.345241in}}%
\pgfpathlineto{\pgfqpoint{3.908691in}{1.344074in}}%
\pgfpathlineto{\pgfqpoint{3.920317in}{1.342924in}}%
\pgfusepath{stroke}%
\end{pgfscope}%
\begin{pgfscope}%
\pgfpathrectangle{\pgfqpoint{0.466126in}{0.521603in}}{\pgfqpoint{3.720000in}{3.020000in}} %
\pgfusepath{clip}%
\pgfsetrectcap%
\pgfsetroundjoin%
\pgfsetlinewidth{1.505625pt}%
\definecolor{currentstroke}{rgb}{1.000000,0.462204,0.237935}%
\pgfsetstrokecolor{currentstroke}%
\pgfsetdash{}{0pt}%
\pgfpathmoveto{\pgfqpoint{0.456126in}{2.807218in}}%
\pgfpathlineto{\pgfqpoint{0.456740in}{2.807019in}}%
\pgfpathlineto{\pgfqpoint{0.667278in}{2.740540in}}%
\pgfpathlineto{\pgfqpoint{0.845337in}{2.685938in}}%
\pgfpathlineto{\pgfqpoint{0.999612in}{2.640020in}}%
\pgfpathlineto{\pgfqpoint{1.135713in}{2.600687in}}%
\pgfpathlineto{\pgfqpoint{1.257476in}{2.566493in}}%
\pgfpathlineto{\pgfqpoint{1.367635in}{2.536403in}}%
\pgfpathlineto{\pgfqpoint{1.468210in}{2.509652in}}%
\pgfpathlineto{\pgfqpoint{1.560738in}{2.485664in}}%
\pgfpathlineto{\pgfqpoint{1.646410in}{2.463992in}}%
\pgfpathlineto{\pgfqpoint{1.726173in}{2.444285in}}%
\pgfpathlineto{\pgfqpoint{1.800789in}{2.426261in}}%
\pgfpathlineto{\pgfqpoint{1.870884in}{2.409694in}}%
\pgfpathlineto{\pgfqpoint{1.936973in}{2.394397in}}%
\pgfpathlineto{\pgfqpoint{1.999490in}{2.380215in}}%
\pgfpathlineto{\pgfqpoint{2.058801in}{2.367019in}}%
\pgfpathlineto{\pgfqpoint{2.115219in}{2.354699in}}%
\pgfpathlineto{\pgfqpoint{2.169013in}{2.343163in}}%
\pgfpathlineto{\pgfqpoint{2.220417in}{2.332331in}}%
\pgfpathlineto{\pgfqpoint{2.269634in}{2.322133in}}%
\pgfpathlineto{\pgfqpoint{2.316842in}{2.312510in}}%
\pgfpathlineto{\pgfqpoint{2.362199in}{2.303411in}}%
\pgfpathlineto{\pgfqpoint{2.405844in}{2.294788in}}%
\pgfpathlineto{\pgfqpoint{2.447903in}{2.286603in}}%
\pgfpathlineto{\pgfqpoint{2.488486in}{2.278818in}}%
\pgfpathlineto{\pgfqpoint{2.527694in}{2.271403in}}%
\pgfpathlineto{\pgfqpoint{2.565617in}{2.264329in}}%
\pgfpathlineto{\pgfqpoint{2.602335in}{2.257571in}}%
\pgfpathlineto{\pgfqpoint{2.637925in}{2.251105in}}%
\pgfpathlineto{\pgfqpoint{2.672452in}{2.244912in}}%
\pgfpathlineto{\pgfqpoint{2.705978in}{2.238972in}}%
\pgfpathlineto{\pgfqpoint{2.738560in}{2.233270in}}%
\pgfpathlineto{\pgfqpoint{2.770249in}{2.227788in}}%
\pgfpathlineto{\pgfqpoint{2.801094in}{2.222515in}}%
\pgfpathlineto{\pgfqpoint{2.831138in}{2.217436in}}%
\pgfpathlineto{\pgfqpoint{2.860421in}{2.212539in}}%
\pgfpathlineto{\pgfqpoint{2.888981in}{2.207816in}}%
\pgfpathlineto{\pgfqpoint{2.916853in}{2.203254in}}%
\pgfpathlineto{\pgfqpoint{2.944069in}{2.198846in}}%
\pgfpathlineto{\pgfqpoint{2.970660in}{2.194583in}}%
\pgfpathlineto{\pgfqpoint{2.996653in}{2.190457in}}%
\pgfpathlineto{\pgfqpoint{3.022075in}{2.186461in}}%
\pgfpathlineto{\pgfqpoint{3.046951in}{2.182588in}}%
\pgfpathlineto{\pgfqpoint{3.071303in}{2.178832in}}%
\pgfpathlineto{\pgfqpoint{3.095152in}{2.175188in}}%
\pgfpathlineto{\pgfqpoint{3.118521in}{2.171649in}}%
\pgfpathlineto{\pgfqpoint{3.141426in}{2.168211in}}%
\pgfpathlineto{\pgfqpoint{3.163887in}{2.164869in}}%
\pgfpathlineto{\pgfqpoint{3.185920in}{2.161619in}}%
\pgfpathlineto{\pgfqpoint{3.207541in}{2.158457in}}%
\pgfpathlineto{\pgfqpoint{3.228765in}{2.155379in}}%
\pgfpathlineto{\pgfqpoint{3.249607in}{2.152380in}}%
\pgfpathlineto{\pgfqpoint{3.270081in}{2.149458in}}%
\pgfpathlineto{\pgfqpoint{3.290198in}{2.146610in}}%
\pgfpathlineto{\pgfqpoint{3.309971in}{2.143832in}}%
\pgfpathlineto{\pgfqpoint{3.329412in}{2.141122in}}%
\pgfpathlineto{\pgfqpoint{3.348532in}{2.138476in}}%
\pgfpathlineto{\pgfqpoint{3.367341in}{2.135893in}}%
\pgfpathlineto{\pgfqpoint{3.385849in}{2.133370in}}%
\pgfpathlineto{\pgfqpoint{3.404066in}{2.130905in}}%
\pgfpathlineto{\pgfqpoint{3.422000in}{2.128495in}}%
\pgfpathlineto{\pgfqpoint{3.439661in}{2.126138in}}%
\pgfpathlineto{\pgfqpoint{3.457056in}{2.123833in}}%
\pgfpathlineto{\pgfqpoint{3.474193in}{2.121578in}}%
\pgfpathlineto{\pgfqpoint{3.491080in}{2.119371in}}%
\pgfpathlineto{\pgfqpoint{3.507724in}{2.117209in}}%
\pgfpathlineto{\pgfqpoint{3.524132in}{2.115093in}}%
\pgfpathlineto{\pgfqpoint{3.540311in}{2.113019in}}%
\pgfpathlineto{\pgfqpoint{3.556266in}{2.110987in}}%
\pgfpathlineto{\pgfqpoint{3.572005in}{2.108996in}}%
\pgfpathlineto{\pgfqpoint{3.587532in}{2.107043in}}%
\pgfpathlineto{\pgfqpoint{3.602854in}{2.105128in}}%
\pgfpathlineto{\pgfqpoint{3.617975in}{2.103250in}}%
\pgfpathlineto{\pgfqpoint{3.632901in}{2.101407in}}%
\pgfpathlineto{\pgfqpoint{3.647637in}{2.099598in}}%
\pgfpathlineto{\pgfqpoint{3.662188in}{2.097823in}}%
\pgfpathlineto{\pgfqpoint{3.676558in}{2.096079in}}%
\pgfpathlineto{\pgfqpoint{3.690752in}{2.094367in}}%
\pgfpathlineto{\pgfqpoint{3.704773in}{2.092686in}}%
\pgfpathlineto{\pgfqpoint{3.718627in}{2.091033in}}%
\pgfpathlineto{\pgfqpoint{3.732317in}{2.089410in}}%
\pgfpathlineto{\pgfqpoint{3.745847in}{2.087814in}}%
\pgfpathlineto{\pgfqpoint{3.759220in}{2.086245in}}%
\pgfpathlineto{\pgfqpoint{3.772441in}{2.084702in}}%
\pgfpathlineto{\pgfqpoint{3.785512in}{2.083185in}}%
\pgfpathlineto{\pgfqpoint{3.798437in}{2.081693in}}%
\pgfpathlineto{\pgfqpoint{3.811219in}{2.080225in}}%
\pgfpathlineto{\pgfqpoint{3.823862in}{2.078781in}}%
\pgfpathlineto{\pgfqpoint{3.836368in}{2.077359in}}%
\pgfpathlineto{\pgfqpoint{3.848740in}{2.075959in}}%
\pgfpathlineto{\pgfqpoint{3.860981in}{2.074582in}}%
\pgfpathlineto{\pgfqpoint{3.873095in}{2.073225in}}%
\pgfpathlineto{\pgfqpoint{3.885082in}{2.071889in}}%
\pgfpathlineto{\pgfqpoint{3.896947in}{2.070574in}}%
\pgfpathlineto{\pgfqpoint{3.908691in}{2.069277in}}%
\pgfpathlineto{\pgfqpoint{3.920317in}{2.068000in}}%
\pgfusepath{stroke}%
\end{pgfscope}%
\begin{pgfscope}%
\pgfsetrectcap%
\pgfsetmiterjoin%
\pgfsetlinewidth{0.803000pt}%
\definecolor{currentstroke}{rgb}{0.000000,0.000000,0.000000}%
\pgfsetstrokecolor{currentstroke}%
\pgfsetdash{}{0pt}%
\pgfpathmoveto{\pgfqpoint{0.466126in}{0.521603in}}%
\pgfpathlineto{\pgfqpoint{0.466126in}{3.541603in}}%
\pgfusepath{stroke}%
\end{pgfscope}%
\begin{pgfscope}%
\pgfsetrectcap%
\pgfsetmiterjoin%
\pgfsetlinewidth{0.803000pt}%
\definecolor{currentstroke}{rgb}{0.000000,0.000000,0.000000}%
\pgfsetstrokecolor{currentstroke}%
\pgfsetdash{}{0pt}%
\pgfpathmoveto{\pgfqpoint{4.186126in}{0.521603in}}%
\pgfpathlineto{\pgfqpoint{4.186126in}{3.541603in}}%
\pgfusepath{stroke}%
\end{pgfscope}%
\begin{pgfscope}%
\pgfsetrectcap%
\pgfsetmiterjoin%
\pgfsetlinewidth{0.803000pt}%
\definecolor{currentstroke}{rgb}{0.000000,0.000000,0.000000}%
\pgfsetstrokecolor{currentstroke}%
\pgfsetdash{}{0pt}%
\pgfpathmoveto{\pgfqpoint{0.466126in}{0.521603in}}%
\pgfpathlineto{\pgfqpoint{4.186126in}{0.521603in}}%
\pgfusepath{stroke}%
\end{pgfscope}%
\begin{pgfscope}%
\pgfsetrectcap%
\pgfsetmiterjoin%
\pgfsetlinewidth{0.803000pt}%
\definecolor{currentstroke}{rgb}{0.000000,0.000000,0.000000}%
\pgfsetstrokecolor{currentstroke}%
\pgfsetdash{}{0pt}%
\pgfpathmoveto{\pgfqpoint{0.466126in}{3.541603in}}%
\pgfpathlineto{\pgfqpoint{4.186126in}{3.541603in}}%
\pgfusepath{stroke}%
\end{pgfscope}%
\begin{pgfscope}%
\pgfsetbuttcap%
\pgfsetmiterjoin%
\definecolor{currentfill}{rgb}{1.000000,1.000000,1.000000}%
\pgfsetfillcolor{currentfill}%
\pgfsetfillopacity{0.800000}%
\pgfsetlinewidth{1.003750pt}%
\definecolor{currentstroke}{rgb}{0.800000,0.800000,0.800000}%
\pgfsetstrokecolor{currentstroke}%
\pgfsetstrokeopacity{0.800000}%
\pgfsetdash{}{0pt}%
\pgfpathmoveto{\pgfqpoint{0.563349in}{2.003492in}}%
\pgfpathlineto{\pgfqpoint{3.101215in}{2.003492in}}%
\pgfpathquadraticcurveto{\pgfqpoint{3.128993in}{2.003492in}}{\pgfqpoint{3.128993in}{2.031269in}}%
\pgfpathlineto{\pgfqpoint{3.128993in}{3.444381in}}%
\pgfpathquadraticcurveto{\pgfqpoint{3.128993in}{3.472159in}}{\pgfqpoint{3.101215in}{3.472159in}}%
\pgfpathlineto{\pgfqpoint{0.563349in}{3.472159in}}%
\pgfpathquadraticcurveto{\pgfqpoint{0.535571in}{3.472159in}}{\pgfqpoint{0.535571in}{3.444381in}}%
\pgfpathlineto{\pgfqpoint{0.535571in}{2.031269in}}%
\pgfpathquadraticcurveto{\pgfqpoint{0.535571in}{2.003492in}}{\pgfqpoint{0.563349in}{2.003492in}}%
\pgfpathclose%
\pgfusepath{stroke,fill}%
\end{pgfscope}%
\begin{pgfscope}%
\pgfsetbuttcap%
\pgfsetroundjoin%
\pgfsetlinewidth{1.505625pt}%
\definecolor{currentstroke}{rgb}{0.000000,0.000000,0.000000}%
\pgfsetstrokecolor{currentstroke}%
\pgfsetdash{{5.550000pt}{2.400000pt}}{0.000000pt}%
\pgfpathmoveto{\pgfqpoint{0.591126in}{3.359691in}}%
\pgfpathlineto{\pgfqpoint{0.868904in}{3.359691in}}%
\pgfusepath{stroke}%
\end{pgfscope}%
\begin{pgfscope}%
\pgftext[x=0.980015in,y=3.311080in,left,base]{\rmfamily\fontsize{10.000000}{12.000000}\selectfont Richards, 2001}%
\end{pgfscope}%
\begin{pgfscope}%
\pgfsetbuttcap%
\pgfsetroundjoin%
\pgfsetlinewidth{1.505625pt}%
\definecolor{currentstroke}{rgb}{0.501961,0.501961,0.501961}%
\pgfsetstrokecolor{currentstroke}%
\pgfsetdash{{9.600000pt}{2.400000pt}{1.500000pt}{2.400000pt}}{0.000000pt}%
\pgfpathmoveto{\pgfqpoint{0.591126in}{3.155834in}}%
\pgfpathlineto{\pgfqpoint{0.868904in}{3.155834in}}%
\pgfusepath{stroke}%
\end{pgfscope}%
\begin{pgfscope}%
\pgftext[x=0.980015in,y=3.107223in,left,base]{\rmfamily\fontsize{10.000000}{12.000000}\selectfont Gopinath et al., 2002}%
\end{pgfscope}%
\begin{pgfscope}%
\pgfsetrectcap%
\pgfsetroundjoin%
\pgfsetlinewidth{1.505625pt}%
\definecolor{currentstroke}{rgb}{0.174510,0.872120,0.862929}%
\pgfsetstrokecolor{currentstroke}%
\pgfsetdash{}{0pt}%
\pgfpathmoveto{\pgfqpoint{0.591126in}{2.951977in}}%
\pgfpathlineto{\pgfqpoint{0.868904in}{2.951977in}}%
\pgfusepath{stroke}%
\end{pgfscope}%
\begin{pgfscope}%
\pgftext[x=0.980015in,y=2.903366in,left,base]{\rmfamily\fontsize{10.000000}{12.000000}\selectfont Molacek et al., 2012, \(\displaystyle \mathbf{B}\mbox{o} = \)0.2}%
\end{pgfscope}%
\begin{pgfscope}%
\pgfsetrectcap%
\pgfsetroundjoin%
\pgfsetlinewidth{1.505625pt}%
\definecolor{currentstroke}{rgb}{0.174510,0.872120,0.862929}%
\pgfsetstrokecolor{currentstroke}%
\pgfsetdash{}{0pt}%
\pgfpathmoveto{\pgfqpoint{0.591126in}{2.748120in}}%
\pgfpathlineto{\pgfqpoint{0.868904in}{2.748120in}}%
\pgfusepath{stroke}%
\end{pgfscope}%
\begin{pgfscope}%
\pgftext[x=0.980015in,y=2.699509in,left,base]{\rmfamily\fontsize{10.000000}{12.000000}\selectfont Molacek et al., 2012, \(\displaystyle \mathbf{B}\mbox{o} = \)0.4}%
\end{pgfscope}%
\begin{pgfscope}%
\pgfsetrectcap%
\pgfsetroundjoin%
\pgfsetlinewidth{1.505625pt}%
\definecolor{currentstroke}{rgb}{0.582353,0.991645,0.659925}%
\pgfsetstrokecolor{currentstroke}%
\pgfsetdash{}{0pt}%
\pgfpathmoveto{\pgfqpoint{0.591126in}{2.544262in}}%
\pgfpathlineto{\pgfqpoint{0.868904in}{2.544262in}}%
\pgfusepath{stroke}%
\end{pgfscope}%
\begin{pgfscope}%
\pgftext[x=0.980015in,y=2.495651in,left,base]{\rmfamily\fontsize{10.000000}{12.000000}\selectfont Molacek et al., 2012, \(\displaystyle \mathbf{B}\mbox{o} = \)0.6}%
\end{pgfscope}%
\begin{pgfscope}%
\pgfsetrectcap%
\pgfsetroundjoin%
\pgfsetlinewidth{1.505625pt}%
\definecolor{currentstroke}{rgb}{0.990196,0.717912,0.389786}%
\pgfsetstrokecolor{currentstroke}%
\pgfsetdash{}{0pt}%
\pgfpathmoveto{\pgfqpoint{0.591126in}{2.340405in}}%
\pgfpathlineto{\pgfqpoint{0.868904in}{2.340405in}}%
\pgfusepath{stroke}%
\end{pgfscope}%
\begin{pgfscope}%
\pgftext[x=0.980015in,y=2.291794in,left,base]{\rmfamily\fontsize{10.000000}{12.000000}\selectfont Molacek et al., 2012, \(\displaystyle \mathbf{B}\mbox{o} = \)0.8}%
\end{pgfscope}%
\begin{pgfscope}%
\pgfsetrectcap%
\pgfsetroundjoin%
\pgfsetlinewidth{1.505625pt}%
\definecolor{currentstroke}{rgb}{1.000000,0.462204,0.237935}%
\pgfsetstrokecolor{currentstroke}%
\pgfsetdash{}{0pt}%
\pgfpathmoveto{\pgfqpoint{0.591126in}{2.136548in}}%
\pgfpathlineto{\pgfqpoint{0.868904in}{2.136548in}}%
\pgfusepath{stroke}%
\end{pgfscope}%
\begin{pgfscope}%
\pgftext[x=0.980015in,y=2.087937in,left,base]{\rmfamily\fontsize{10.000000}{12.000000}\selectfont Molacek et al., 2012, \(\displaystyle \mathbf{B}\mbox{o} = \)0.9}%
\end{pgfscope}%
\begin{pgfscope}%
\pgfpathrectangle{\pgfqpoint{4.418626in}{0.521603in}}{\pgfqpoint{0.151000in}{3.020000in}} %
\pgfusepath{clip}%
\pgfsetbuttcap%
\pgfsetmiterjoin%
\definecolor{currentfill}{rgb}{1.000000,1.000000,1.000000}%
\pgfsetfillcolor{currentfill}%
\pgfsetlinewidth{0.010037pt}%
\definecolor{currentstroke}{rgb}{1.000000,1.000000,1.000000}%
\pgfsetstrokecolor{currentstroke}%
\pgfsetdash{}{0pt}%
\pgfpathmoveto{\pgfqpoint{4.418626in}{0.521603in}}%
\pgfpathlineto{\pgfqpoint{4.418626in}{0.533400in}}%
\pgfpathlineto{\pgfqpoint{4.418626in}{3.529806in}}%
\pgfpathlineto{\pgfqpoint{4.418626in}{3.541603in}}%
\pgfpathlineto{\pgfqpoint{4.569626in}{3.541603in}}%
\pgfpathlineto{\pgfqpoint{4.569626in}{3.529806in}}%
\pgfpathlineto{\pgfqpoint{4.569626in}{0.533400in}}%
\pgfpathlineto{\pgfqpoint{4.569626in}{0.521603in}}%
\pgfpathclose%
\pgfusepath{stroke,fill}%
\end{pgfscope}%
\begin{pgfscope}%
\pgfsys@transformshift{4.420000in}{0.526603in}%
\pgftext[left,bottom]{\pgfimage[interpolate=true,width=0.150000in,height=3.020000in]{contact-img0.png}}%
\end{pgfscope}%
\begin{pgfscope}%
\pgfsetbuttcap%
\pgfsetroundjoin%
\definecolor{currentfill}{rgb}{0.000000,0.000000,0.000000}%
\pgfsetfillcolor{currentfill}%
\pgfsetlinewidth{0.803000pt}%
\definecolor{currentstroke}{rgb}{0.000000,0.000000,0.000000}%
\pgfsetstrokecolor{currentstroke}%
\pgfsetdash{}{0pt}%
\pgfsys@defobject{currentmarker}{\pgfqpoint{0.000000in}{0.000000in}}{\pgfqpoint{0.048611in}{0.000000in}}{%
\pgfpathmoveto{\pgfqpoint{0.000000in}{0.000000in}}%
\pgfpathlineto{\pgfqpoint{0.048611in}{0.000000in}}%
\pgfusepath{stroke,fill}%
}%
\begin{pgfscope}%
\pgfsys@transformshift{4.569626in}{0.925593in}%
\pgfsys@useobject{currentmarker}{}%
\end{pgfscope}%
\end{pgfscope}%
\begin{pgfscope}%
\pgftext[x=4.666849in,y=0.872831in,left,base]{\rmfamily\fontsize{10.000000}{12.000000}\selectfont \(\displaystyle 0.2\)}%
\end{pgfscope}%
\begin{pgfscope}%
\pgfsetbuttcap%
\pgfsetroundjoin%
\definecolor{currentfill}{rgb}{0.000000,0.000000,0.000000}%
\pgfsetfillcolor{currentfill}%
\pgfsetlinewidth{0.803000pt}%
\definecolor{currentstroke}{rgb}{0.000000,0.000000,0.000000}%
\pgfsetstrokecolor{currentstroke}%
\pgfsetdash{}{0pt}%
\pgfsys@defobject{currentmarker}{\pgfqpoint{0.000000in}{0.000000in}}{\pgfqpoint{0.048611in}{0.000000in}}{%
\pgfpathmoveto{\pgfqpoint{0.000000in}{0.000000in}}%
\pgfpathlineto{\pgfqpoint{0.048611in}{0.000000in}}%
\pgfusepath{stroke,fill}%
}%
\begin{pgfscope}%
\pgfsys@transformshift{4.569626in}{1.540182in}%
\pgfsys@useobject{currentmarker}{}%
\end{pgfscope}%
\end{pgfscope}%
\begin{pgfscope}%
\pgftext[x=4.666849in,y=1.487421in,left,base]{\rmfamily\fontsize{10.000000}{12.000000}\selectfont \(\displaystyle 0.4\)}%
\end{pgfscope}%
\begin{pgfscope}%
\pgfsetbuttcap%
\pgfsetroundjoin%
\definecolor{currentfill}{rgb}{0.000000,0.000000,0.000000}%
\pgfsetfillcolor{currentfill}%
\pgfsetlinewidth{0.803000pt}%
\definecolor{currentstroke}{rgb}{0.000000,0.000000,0.000000}%
\pgfsetstrokecolor{currentstroke}%
\pgfsetdash{}{0pt}%
\pgfsys@defobject{currentmarker}{\pgfqpoint{0.000000in}{0.000000in}}{\pgfqpoint{0.048611in}{0.000000in}}{%
\pgfpathmoveto{\pgfqpoint{0.000000in}{0.000000in}}%
\pgfpathlineto{\pgfqpoint{0.048611in}{0.000000in}}%
\pgfusepath{stroke,fill}%
}%
\begin{pgfscope}%
\pgfsys@transformshift{4.569626in}{2.154772in}%
\pgfsys@useobject{currentmarker}{}%
\end{pgfscope}%
\end{pgfscope}%
\begin{pgfscope}%
\pgftext[x=4.666849in,y=2.102010in,left,base]{\rmfamily\fontsize{10.000000}{12.000000}\selectfont \(\displaystyle 0.6\)}%
\end{pgfscope}%
\begin{pgfscope}%
\pgfsetbuttcap%
\pgfsetroundjoin%
\definecolor{currentfill}{rgb}{0.000000,0.000000,0.000000}%
\pgfsetfillcolor{currentfill}%
\pgfsetlinewidth{0.803000pt}%
\definecolor{currentstroke}{rgb}{0.000000,0.000000,0.000000}%
\pgfsetstrokecolor{currentstroke}%
\pgfsetdash{}{0pt}%
\pgfsys@defobject{currentmarker}{\pgfqpoint{0.000000in}{0.000000in}}{\pgfqpoint{0.048611in}{0.000000in}}{%
\pgfpathmoveto{\pgfqpoint{0.000000in}{0.000000in}}%
\pgfpathlineto{\pgfqpoint{0.048611in}{0.000000in}}%
\pgfusepath{stroke,fill}%
}%
\begin{pgfscope}%
\pgfsys@transformshift{4.569626in}{2.769361in}%
\pgfsys@useobject{currentmarker}{}%
\end{pgfscope}%
\end{pgfscope}%
\begin{pgfscope}%
\pgftext[x=4.666849in,y=2.716599in,left,base]{\rmfamily\fontsize{10.000000}{12.000000}\selectfont \(\displaystyle 0.8\)}%
\end{pgfscope}%
\begin{pgfscope}%
\pgfsetbuttcap%
\pgfsetroundjoin%
\definecolor{currentfill}{rgb}{0.000000,0.000000,0.000000}%
\pgfsetfillcolor{currentfill}%
\pgfsetlinewidth{0.803000pt}%
\definecolor{currentstroke}{rgb}{0.000000,0.000000,0.000000}%
\pgfsetstrokecolor{currentstroke}%
\pgfsetdash{}{0pt}%
\pgfsys@defobject{currentmarker}{\pgfqpoint{0.000000in}{0.000000in}}{\pgfqpoint{0.048611in}{0.000000in}}{%
\pgfpathmoveto{\pgfqpoint{0.000000in}{0.000000in}}%
\pgfpathlineto{\pgfqpoint{0.048611in}{0.000000in}}%
\pgfusepath{stroke,fill}%
}%
\begin{pgfscope}%
\pgfsys@transformshift{4.569626in}{3.383950in}%
\pgfsys@useobject{currentmarker}{}%
\end{pgfscope}%
\end{pgfscope}%
\begin{pgfscope}%
\pgftext[x=4.666849in,y=3.331189in,left,base]{\rmfamily\fontsize{10.000000}{12.000000}\selectfont \(\displaystyle 1.0\)}%
\end{pgfscope}%
\begin{pgfscope}%
\pgftext[x=4.899874in,y=2.031603in,,top,rotate=90.000000]{\rmfamily\fontsize{10.000000}{12.000000}\selectfont \(\displaystyle \mathrm{\mathit{Bo_e}} \equiv \frac{\epsilon E_0^2 R_0}{\gamma}\)}%
\end{pgfscope}%
\begin{pgfscope}%
\pgfsetbuttcap%
\pgfsetmiterjoin%
\pgfsetlinewidth{0.803000pt}%
\definecolor{currentstroke}{rgb}{0.000000,0.000000,0.000000}%
\pgfsetstrokecolor{currentstroke}%
\pgfsetdash{}{0pt}%
\pgfpathmoveto{\pgfqpoint{4.418626in}{0.521603in}}%
\pgfpathlineto{\pgfqpoint{4.418626in}{0.533400in}}%
\pgfpathlineto{\pgfqpoint{4.418626in}{3.529806in}}%
\pgfpathlineto{\pgfqpoint{4.418626in}{3.541603in}}%
\pgfpathlineto{\pgfqpoint{4.569626in}{3.541603in}}%
\pgfpathlineto{\pgfqpoint{4.569626in}{3.529806in}}%
\pgfpathlineto{\pgfqpoint{4.569626in}{0.533400in}}%
\pgfpathlineto{\pgfqpoint{4.569626in}{0.521603in}}%
\pgfpathclose%
\pgfusepath{stroke}%
\end{pgfscope}%
\end{pgfpicture}%
\makeatother%
\endgroup%

    \caption{.\label{fig:contact}}
\end{figure}

\begin{figure}[htb]
    \centering
    %% Creator: Matplotlib, PGF backend
%%
%% To include the figure in your LaTeX document, write
%%   \input{<filename>.pgf}
%%
%% Make sure the required packages are loaded in your preamble
%%   \usepackage{pgf}
%%
%% Figures using additional raster images can only be included by \input if
%% they are in the same directory as the main LaTeX file. For loading figures
%% from other directories you can use the `import` package
%%   \usepackage{import}
%% and then include the figures with
%%   \import{<path to file>}{<filename>.pgf}
%%
%% Matplotlib used the following preamble
%%   \usepackage{fontspec}
%%   \setmainfont{DejaVu Serif}
%%   \setsansfont{DejaVu Sans}
%%   \setmonofont{DejaVu Sans Mono}
%%
\begingroup%
\makeatletter%
\begin{pgfpicture}%
\pgfpathrectangle{\pgfpointorigin}{\pgfqpoint{5.427700in}{3.676603in}}%
\pgfusepath{use as bounding box, clip}%
\begin{pgfscope}%
\pgfsetbuttcap%
\pgfsetmiterjoin%
\definecolor{currentfill}{rgb}{1.000000,1.000000,1.000000}%
\pgfsetfillcolor{currentfill}%
\pgfsetlinewidth{0.000000pt}%
\definecolor{currentstroke}{rgb}{1.000000,1.000000,1.000000}%
\pgfsetstrokecolor{currentstroke}%
\pgfsetdash{}{0pt}%
\pgfpathmoveto{\pgfqpoint{0.000000in}{0.000000in}}%
\pgfpathlineto{\pgfqpoint{5.427700in}{0.000000in}}%
\pgfpathlineto{\pgfqpoint{5.427700in}{3.676603in}}%
\pgfpathlineto{\pgfqpoint{0.000000in}{3.676603in}}%
\pgfpathclose%
\pgfusepath{fill}%
\end{pgfscope}%
\begin{pgfscope}%
\pgfsetbuttcap%
\pgfsetmiterjoin%
\definecolor{currentfill}{rgb}{1.000000,1.000000,1.000000}%
\pgfsetfillcolor{currentfill}%
\pgfsetlinewidth{0.000000pt}%
\definecolor{currentstroke}{rgb}{0.000000,0.000000,0.000000}%
\pgfsetstrokecolor{currentstroke}%
\pgfsetstrokeopacity{0.000000}%
\pgfsetdash{}{0pt}%
\pgfpathmoveto{\pgfqpoint{0.564660in}{0.521603in}}%
\pgfpathlineto{\pgfqpoint{4.284660in}{0.521603in}}%
\pgfpathlineto{\pgfqpoint{4.284660in}{3.541603in}}%
\pgfpathlineto{\pgfqpoint{0.564660in}{3.541603in}}%
\pgfpathclose%
\pgfusepath{fill}%
\end{pgfscope}%
\begin{pgfscope}%
\pgfpathrectangle{\pgfqpoint{0.564660in}{0.521603in}}{\pgfqpoint{3.720000in}{3.020000in}} %
\pgfusepath{clip}%
\pgfsetbuttcap%
\pgfsetroundjoin%
\definecolor{currentfill}{rgb}{1.000000,0.255843,0.128999}%
\pgfsetfillcolor{currentfill}%
\pgfsetlinewidth{1.003750pt}%
\definecolor{currentstroke}{rgb}{1.000000,0.255843,0.128999}%
\pgfsetstrokecolor{currentstroke}%
\pgfsetdash{}{0pt}%
\pgfpathmoveto{\pgfqpoint{3.297585in}{2.468980in}}%
\pgfpathcurveto{\pgfqpoint{3.308636in}{2.468980in}}{\pgfqpoint{3.319235in}{2.473370in}}{\pgfqpoint{3.327048in}{2.481184in}}%
\pgfpathcurveto{\pgfqpoint{3.334862in}{2.488998in}}{\pgfqpoint{3.339252in}{2.499597in}}{\pgfqpoint{3.339252in}{2.510647in}}%
\pgfpathcurveto{\pgfqpoint{3.339252in}{2.521697in}}{\pgfqpoint{3.334862in}{2.532296in}}{\pgfqpoint{3.327048in}{2.540110in}}%
\pgfpathcurveto{\pgfqpoint{3.319235in}{2.547923in}}{\pgfqpoint{3.308636in}{2.552314in}}{\pgfqpoint{3.297585in}{2.552314in}}%
\pgfpathcurveto{\pgfqpoint{3.286535in}{2.552314in}}{\pgfqpoint{3.275936in}{2.547923in}}{\pgfqpoint{3.268123in}{2.540110in}}%
\pgfpathcurveto{\pgfqpoint{3.260309in}{2.532296in}}{\pgfqpoint{3.255919in}{2.521697in}}{\pgfqpoint{3.255919in}{2.510647in}}%
\pgfpathcurveto{\pgfqpoint{3.255919in}{2.499597in}}{\pgfqpoint{3.260309in}{2.488998in}}{\pgfqpoint{3.268123in}{2.481184in}}%
\pgfpathcurveto{\pgfqpoint{3.275936in}{2.473370in}}{\pgfqpoint{3.286535in}{2.468980in}}{\pgfqpoint{3.297585in}{2.468980in}}%
\pgfpathclose%
\pgfusepath{stroke,fill}%
\end{pgfscope}%
\begin{pgfscope}%
\pgfpathrectangle{\pgfqpoint{0.564660in}{0.521603in}}{\pgfqpoint{3.720000in}{3.020000in}} %
\pgfusepath{clip}%
\pgfsetbuttcap%
\pgfsetroundjoin%
\definecolor{currentfill}{rgb}{0.319608,0.279583,0.989980}%
\pgfsetfillcolor{currentfill}%
\pgfsetlinewidth{1.003750pt}%
\definecolor{currentstroke}{rgb}{0.319608,0.279583,0.989980}%
\pgfsetstrokecolor{currentstroke}%
\pgfsetdash{}{0pt}%
\pgfpathmoveto{\pgfqpoint{3.328555in}{1.873234in}}%
\pgfpathcurveto{\pgfqpoint{3.339606in}{1.873234in}}{\pgfqpoint{3.350205in}{1.877624in}}{\pgfqpoint{3.358018in}{1.885438in}}%
\pgfpathcurveto{\pgfqpoint{3.365832in}{1.893251in}}{\pgfqpoint{3.370222in}{1.903850in}}{\pgfqpoint{3.370222in}{1.914900in}}%
\pgfpathcurveto{\pgfqpoint{3.370222in}{1.925951in}}{\pgfqpoint{3.365832in}{1.936550in}}{\pgfqpoint{3.358018in}{1.944363in}}%
\pgfpathcurveto{\pgfqpoint{3.350205in}{1.952177in}}{\pgfqpoint{3.339606in}{1.956567in}}{\pgfqpoint{3.328555in}{1.956567in}}%
\pgfpathcurveto{\pgfqpoint{3.317505in}{1.956567in}}{\pgfqpoint{3.306906in}{1.952177in}}{\pgfqpoint{3.299093in}{1.944363in}}%
\pgfpathcurveto{\pgfqpoint{3.291279in}{1.936550in}}{\pgfqpoint{3.286889in}{1.925951in}}{\pgfqpoint{3.286889in}{1.914900in}}%
\pgfpathcurveto{\pgfqpoint{3.286889in}{1.903850in}}{\pgfqpoint{3.291279in}{1.893251in}}{\pgfqpoint{3.299093in}{1.885438in}}%
\pgfpathcurveto{\pgfqpoint{3.306906in}{1.877624in}}{\pgfqpoint{3.317505in}{1.873234in}}{\pgfqpoint{3.328555in}{1.873234in}}%
\pgfpathclose%
\pgfusepath{stroke,fill}%
\end{pgfscope}%
\begin{pgfscope}%
\pgfpathrectangle{\pgfqpoint{0.564660in}{0.521603in}}{\pgfqpoint{3.720000in}{3.020000in}} %
\pgfusepath{clip}%
\pgfsetbuttcap%
\pgfsetroundjoin%
\definecolor{currentfill}{rgb}{0.460784,0.061561,0.999526}%
\pgfsetfillcolor{currentfill}%
\pgfsetlinewidth{1.003750pt}%
\definecolor{currentstroke}{rgb}{0.460784,0.061561,0.999526}%
\pgfsetstrokecolor{currentstroke}%
\pgfsetdash{}{0pt}%
\pgfpathmoveto{\pgfqpoint{2.895722in}{1.799203in}}%
\pgfpathcurveto{\pgfqpoint{2.906772in}{1.799203in}}{\pgfqpoint{2.917371in}{1.803593in}}{\pgfqpoint{2.925185in}{1.811407in}}%
\pgfpathcurveto{\pgfqpoint{2.932998in}{1.819220in}}{\pgfqpoint{2.937389in}{1.829819in}}{\pgfqpoint{2.937389in}{1.840870in}}%
\pgfpathcurveto{\pgfqpoint{2.937389in}{1.851920in}}{\pgfqpoint{2.932998in}{1.862519in}}{\pgfqpoint{2.925185in}{1.870332in}}%
\pgfpathcurveto{\pgfqpoint{2.917371in}{1.878146in}}{\pgfqpoint{2.906772in}{1.882536in}}{\pgfqpoint{2.895722in}{1.882536in}}%
\pgfpathcurveto{\pgfqpoint{2.884672in}{1.882536in}}{\pgfqpoint{2.874073in}{1.878146in}}{\pgfqpoint{2.866259in}{1.870332in}}%
\pgfpathcurveto{\pgfqpoint{2.858445in}{1.862519in}}{\pgfqpoint{2.854055in}{1.851920in}}{\pgfqpoint{2.854055in}{1.840870in}}%
\pgfpathcurveto{\pgfqpoint{2.854055in}{1.829819in}}{\pgfqpoint{2.858445in}{1.819220in}}{\pgfqpoint{2.866259in}{1.811407in}}%
\pgfpathcurveto{\pgfqpoint{2.874073in}{1.803593in}}{\pgfqpoint{2.884672in}{1.799203in}}{\pgfqpoint{2.895722in}{1.799203in}}%
\pgfpathclose%
\pgfusepath{stroke,fill}%
\end{pgfscope}%
\begin{pgfscope}%
\pgfpathrectangle{\pgfqpoint{0.564660in}{0.521603in}}{\pgfqpoint{3.720000in}{3.020000in}} %
\pgfusepath{clip}%
\pgfsetbuttcap%
\pgfsetroundjoin%
\definecolor{currentfill}{rgb}{0.500000,0.000000,1.000000}%
\pgfsetfillcolor{currentfill}%
\pgfsetlinewidth{1.003750pt}%
\definecolor{currentstroke}{rgb}{0.500000,0.000000,1.000000}%
\pgfsetstrokecolor{currentstroke}%
\pgfsetdash{}{0pt}%
\pgfpathmoveto{\pgfqpoint{2.003834in}{2.169242in}}%
\pgfpathcurveto{\pgfqpoint{2.014884in}{2.169242in}}{\pgfqpoint{2.025483in}{2.173632in}}{\pgfqpoint{2.033297in}{2.181446in}}%
\pgfpathcurveto{\pgfqpoint{2.041110in}{2.189259in}}{\pgfqpoint{2.045500in}{2.199858in}}{\pgfqpoint{2.045500in}{2.210908in}}%
\pgfpathcurveto{\pgfqpoint{2.045500in}{2.221959in}}{\pgfqpoint{2.041110in}{2.232558in}}{\pgfqpoint{2.033297in}{2.240371in}}%
\pgfpathcurveto{\pgfqpoint{2.025483in}{2.248185in}}{\pgfqpoint{2.014884in}{2.252575in}}{\pgfqpoint{2.003834in}{2.252575in}}%
\pgfpathcurveto{\pgfqpoint{1.992784in}{2.252575in}}{\pgfqpoint{1.982185in}{2.248185in}}{\pgfqpoint{1.974371in}{2.240371in}}%
\pgfpathcurveto{\pgfqpoint{1.966557in}{2.232558in}}{\pgfqpoint{1.962167in}{2.221959in}}{\pgfqpoint{1.962167in}{2.210908in}}%
\pgfpathcurveto{\pgfqpoint{1.962167in}{2.199858in}}{\pgfqpoint{1.966557in}{2.189259in}}{\pgfqpoint{1.974371in}{2.181446in}}%
\pgfpathcurveto{\pgfqpoint{1.982185in}{2.173632in}}{\pgfqpoint{1.992784in}{2.169242in}}{\pgfqpoint{2.003834in}{2.169242in}}%
\pgfpathclose%
\pgfusepath{stroke,fill}%
\end{pgfscope}%
\begin{pgfscope}%
\pgfpathrectangle{\pgfqpoint{0.564660in}{0.521603in}}{\pgfqpoint{3.720000in}{3.020000in}} %
\pgfusepath{clip}%
\pgfsetbuttcap%
\pgfsetroundjoin%
\definecolor{currentfill}{rgb}{1.000000,0.000000,0.000000}%
\pgfsetfillcolor{currentfill}%
\pgfsetlinewidth{1.003750pt}%
\definecolor{currentstroke}{rgb}{1.000000,0.000000,0.000000}%
\pgfsetstrokecolor{currentstroke}%
\pgfsetdash{}{0pt}%
\pgfpathmoveto{\pgfqpoint{4.166276in}{1.932500in}}%
\pgfpathcurveto{\pgfqpoint{4.177326in}{1.932500in}}{\pgfqpoint{4.187926in}{1.936890in}}{\pgfqpoint{4.195739in}{1.944704in}}%
\pgfpathcurveto{\pgfqpoint{4.203553in}{1.952518in}}{\pgfqpoint{4.207943in}{1.963117in}}{\pgfqpoint{4.207943in}{1.974167in}}%
\pgfpathcurveto{\pgfqpoint{4.207943in}{1.985217in}}{\pgfqpoint{4.203553in}{1.995816in}}{\pgfqpoint{4.195739in}{2.003630in}}%
\pgfpathcurveto{\pgfqpoint{4.187926in}{2.011443in}}{\pgfqpoint{4.177326in}{2.015834in}}{\pgfqpoint{4.166276in}{2.015834in}}%
\pgfpathcurveto{\pgfqpoint{4.155226in}{2.015834in}}{\pgfqpoint{4.144627in}{2.011443in}}{\pgfqpoint{4.136814in}{2.003630in}}%
\pgfpathcurveto{\pgfqpoint{4.129000in}{1.995816in}}{\pgfqpoint{4.124610in}{1.985217in}}{\pgfqpoint{4.124610in}{1.974167in}}%
\pgfpathcurveto{\pgfqpoint{4.124610in}{1.963117in}}{\pgfqpoint{4.129000in}{1.952518in}}{\pgfqpoint{4.136814in}{1.944704in}}%
\pgfpathcurveto{\pgfqpoint{4.144627in}{1.936890in}}{\pgfqpoint{4.155226in}{1.932500in}}{\pgfqpoint{4.166276in}{1.932500in}}%
\pgfpathclose%
\pgfusepath{stroke,fill}%
\end{pgfscope}%
\begin{pgfscope}%
\pgfpathrectangle{\pgfqpoint{0.564660in}{0.521603in}}{\pgfqpoint{3.720000in}{3.020000in}} %
\pgfusepath{clip}%
\pgfsetbuttcap%
\pgfsetroundjoin%
\definecolor{currentfill}{rgb}{1.000000,0.000000,0.000000}%
\pgfsetfillcolor{currentfill}%
\pgfsetlinewidth{1.003750pt}%
\definecolor{currentstroke}{rgb}{1.000000,0.000000,0.000000}%
\pgfsetstrokecolor{currentstroke}%
\pgfsetdash{}{0pt}%
\pgfpathmoveto{\pgfqpoint{3.773192in}{2.692753in}}%
\pgfpathcurveto{\pgfqpoint{3.784242in}{2.692753in}}{\pgfqpoint{3.794841in}{2.697143in}}{\pgfqpoint{3.802655in}{2.704957in}}%
\pgfpathcurveto{\pgfqpoint{3.810468in}{2.712770in}}{\pgfqpoint{3.814859in}{2.723369in}}{\pgfqpoint{3.814859in}{2.734420in}}%
\pgfpathcurveto{\pgfqpoint{3.814859in}{2.745470in}}{\pgfqpoint{3.810468in}{2.756069in}}{\pgfqpoint{3.802655in}{2.763882in}}%
\pgfpathcurveto{\pgfqpoint{3.794841in}{2.771696in}}{\pgfqpoint{3.784242in}{2.776086in}}{\pgfqpoint{3.773192in}{2.776086in}}%
\pgfpathcurveto{\pgfqpoint{3.762142in}{2.776086in}}{\pgfqpoint{3.751543in}{2.771696in}}{\pgfqpoint{3.743729in}{2.763882in}}%
\pgfpathcurveto{\pgfqpoint{3.735915in}{2.756069in}}{\pgfqpoint{3.731525in}{2.745470in}}{\pgfqpoint{3.731525in}{2.734420in}}%
\pgfpathcurveto{\pgfqpoint{3.731525in}{2.723369in}}{\pgfqpoint{3.735915in}{2.712770in}}{\pgfqpoint{3.743729in}{2.704957in}}%
\pgfpathcurveto{\pgfqpoint{3.751543in}{2.697143in}}{\pgfqpoint{3.762142in}{2.692753in}}{\pgfqpoint{3.773192in}{2.692753in}}%
\pgfpathclose%
\pgfusepath{stroke,fill}%
\end{pgfscope}%
\begin{pgfscope}%
\pgfpathrectangle{\pgfqpoint{0.564660in}{0.521603in}}{\pgfqpoint{3.720000in}{3.020000in}} %
\pgfusepath{clip}%
\pgfsetbuttcap%
\pgfsetroundjoin%
\definecolor{currentfill}{rgb}{1.000000,0.171626,0.086133}%
\pgfsetfillcolor{currentfill}%
\pgfsetlinewidth{1.003750pt}%
\definecolor{currentstroke}{rgb}{1.000000,0.171626,0.086133}%
\pgfsetstrokecolor{currentstroke}%
\pgfsetdash{}{0pt}%
\pgfpathmoveto{\pgfqpoint{3.722241in}{2.302709in}}%
\pgfpathcurveto{\pgfqpoint{3.733292in}{2.302709in}}{\pgfqpoint{3.743891in}{2.307099in}}{\pgfqpoint{3.751704in}{2.314913in}}%
\pgfpathcurveto{\pgfqpoint{3.759518in}{2.322726in}}{\pgfqpoint{3.763908in}{2.333325in}}{\pgfqpoint{3.763908in}{2.344375in}}%
\pgfpathcurveto{\pgfqpoint{3.763908in}{2.355425in}}{\pgfqpoint{3.759518in}{2.366025in}}{\pgfqpoint{3.751704in}{2.373838in}}%
\pgfpathcurveto{\pgfqpoint{3.743891in}{2.381652in}}{\pgfqpoint{3.733292in}{2.386042in}}{\pgfqpoint{3.722241in}{2.386042in}}%
\pgfpathcurveto{\pgfqpoint{3.711191in}{2.386042in}}{\pgfqpoint{3.700592in}{2.381652in}}{\pgfqpoint{3.692779in}{2.373838in}}%
\pgfpathcurveto{\pgfqpoint{3.684965in}{2.366025in}}{\pgfqpoint{3.680575in}{2.355425in}}{\pgfqpoint{3.680575in}{2.344375in}}%
\pgfpathcurveto{\pgfqpoint{3.680575in}{2.333325in}}{\pgfqpoint{3.684965in}{2.322726in}}{\pgfqpoint{3.692779in}{2.314913in}}%
\pgfpathcurveto{\pgfqpoint{3.700592in}{2.307099in}}{\pgfqpoint{3.711191in}{2.302709in}}{\pgfqpoint{3.722241in}{2.302709in}}%
\pgfpathclose%
\pgfusepath{stroke,fill}%
\end{pgfscope}%
\begin{pgfscope}%
\pgfpathrectangle{\pgfqpoint{0.564660in}{0.521603in}}{\pgfqpoint{3.720000in}{3.020000in}} %
\pgfusepath{clip}%
\pgfsetbuttcap%
\pgfsetroundjoin%
\definecolor{currentfill}{rgb}{1.000000,0.171626,0.086133}%
\pgfsetfillcolor{currentfill}%
\pgfsetlinewidth{1.003750pt}%
\definecolor{currentstroke}{rgb}{1.000000,0.171626,0.086133}%
\pgfsetstrokecolor{currentstroke}%
\pgfsetdash{}{0pt}%
\pgfpathmoveto{\pgfqpoint{3.468726in}{0.678635in}}%
\pgfpathcurveto{\pgfqpoint{3.479776in}{0.678635in}}{\pgfqpoint{3.490375in}{0.683025in}}{\pgfqpoint{3.498189in}{0.690839in}}%
\pgfpathcurveto{\pgfqpoint{3.506002in}{0.698653in}}{\pgfqpoint{3.510392in}{0.709252in}}{\pgfqpoint{3.510392in}{0.720302in}}%
\pgfpathcurveto{\pgfqpoint{3.510392in}{0.731352in}}{\pgfqpoint{3.506002in}{0.741951in}}{\pgfqpoint{3.498189in}{0.749765in}}%
\pgfpathcurveto{\pgfqpoint{3.490375in}{0.757578in}}{\pgfqpoint{3.479776in}{0.761968in}}{\pgfqpoint{3.468726in}{0.761968in}}%
\pgfpathcurveto{\pgfqpoint{3.457676in}{0.761968in}}{\pgfqpoint{3.447077in}{0.757578in}}{\pgfqpoint{3.439263in}{0.749765in}}%
\pgfpathcurveto{\pgfqpoint{3.431449in}{0.741951in}}{\pgfqpoint{3.427059in}{0.731352in}}{\pgfqpoint{3.427059in}{0.720302in}}%
\pgfpathcurveto{\pgfqpoint{3.427059in}{0.709252in}}{\pgfqpoint{3.431449in}{0.698653in}}{\pgfqpoint{3.439263in}{0.690839in}}%
\pgfpathcurveto{\pgfqpoint{3.447077in}{0.683025in}}{\pgfqpoint{3.457676in}{0.678635in}}{\pgfqpoint{3.468726in}{0.678635in}}%
\pgfpathclose%
\pgfusepath{stroke,fill}%
\end{pgfscope}%
\begin{pgfscope}%
\pgfpathrectangle{\pgfqpoint{0.564660in}{0.521603in}}{\pgfqpoint{3.720000in}{3.020000in}} %
\pgfusepath{clip}%
\pgfsetbuttcap%
\pgfsetroundjoin%
\definecolor{currentfill}{rgb}{0.660784,0.968276,0.612420}%
\pgfsetfillcolor{currentfill}%
\pgfsetlinewidth{1.003750pt}%
\definecolor{currentstroke}{rgb}{0.660784,0.968276,0.612420}%
\pgfsetstrokecolor{currentstroke}%
\pgfsetdash{}{0pt}%
\pgfpathmoveto{\pgfqpoint{3.256354in}{0.943869in}}%
\pgfpathcurveto{\pgfqpoint{3.267405in}{0.943869in}}{\pgfqpoint{3.278004in}{0.948259in}}{\pgfqpoint{3.285817in}{0.956072in}}%
\pgfpathcurveto{\pgfqpoint{3.293631in}{0.963886in}}{\pgfqpoint{3.298021in}{0.974485in}}{\pgfqpoint{3.298021in}{0.985535in}}%
\pgfpathcurveto{\pgfqpoint{3.298021in}{0.996585in}}{\pgfqpoint{3.293631in}{1.007184in}}{\pgfqpoint{3.285817in}{1.014998in}}%
\pgfpathcurveto{\pgfqpoint{3.278004in}{1.022812in}}{\pgfqpoint{3.267405in}{1.027202in}}{\pgfqpoint{3.256354in}{1.027202in}}%
\pgfpathcurveto{\pgfqpoint{3.245304in}{1.027202in}}{\pgfqpoint{3.234705in}{1.022812in}}{\pgfqpoint{3.226892in}{1.014998in}}%
\pgfpathcurveto{\pgfqpoint{3.219078in}{1.007184in}}{\pgfqpoint{3.214688in}{0.996585in}}{\pgfqpoint{3.214688in}{0.985535in}}%
\pgfpathcurveto{\pgfqpoint{3.214688in}{0.974485in}}{\pgfqpoint{3.219078in}{0.963886in}}{\pgfqpoint{3.226892in}{0.956072in}}%
\pgfpathcurveto{\pgfqpoint{3.234705in}{0.948259in}}{\pgfqpoint{3.245304in}{0.943869in}}{\pgfqpoint{3.256354in}{0.943869in}}%
\pgfpathclose%
\pgfusepath{stroke,fill}%
\end{pgfscope}%
\begin{pgfscope}%
\pgfpathrectangle{\pgfqpoint{0.564660in}{0.521603in}}{\pgfqpoint{3.720000in}{3.020000in}} %
\pgfusepath{clip}%
\pgfsetbuttcap%
\pgfsetroundjoin%
\definecolor{currentfill}{rgb}{0.103922,0.812622,0.889604}%
\pgfsetfillcolor{currentfill}%
\pgfsetlinewidth{1.003750pt}%
\definecolor{currentstroke}{rgb}{0.103922,0.812622,0.889604}%
\pgfsetstrokecolor{currentstroke}%
\pgfsetdash{}{0pt}%
\pgfpathmoveto{\pgfqpoint{3.210081in}{1.002082in}}%
\pgfpathcurveto{\pgfqpoint{3.221131in}{1.002082in}}{\pgfqpoint{3.231730in}{1.006472in}}{\pgfqpoint{3.239544in}{1.014286in}}%
\pgfpathcurveto{\pgfqpoint{3.247358in}{1.022099in}}{\pgfqpoint{3.251748in}{1.032698in}}{\pgfqpoint{3.251748in}{1.043748in}}%
\pgfpathcurveto{\pgfqpoint{3.251748in}{1.054799in}}{\pgfqpoint{3.247358in}{1.065398in}}{\pgfqpoint{3.239544in}{1.073211in}}%
\pgfpathcurveto{\pgfqpoint{3.231730in}{1.081025in}}{\pgfqpoint{3.221131in}{1.085415in}}{\pgfqpoint{3.210081in}{1.085415in}}%
\pgfpathcurveto{\pgfqpoint{3.199031in}{1.085415in}}{\pgfqpoint{3.188432in}{1.081025in}}{\pgfqpoint{3.180618in}{1.073211in}}%
\pgfpathcurveto{\pgfqpoint{3.172805in}{1.065398in}}{\pgfqpoint{3.168415in}{1.054799in}}{\pgfqpoint{3.168415in}{1.043748in}}%
\pgfpathcurveto{\pgfqpoint{3.168415in}{1.032698in}}{\pgfqpoint{3.172805in}{1.022099in}}{\pgfqpoint{3.180618in}{1.014286in}}%
\pgfpathcurveto{\pgfqpoint{3.188432in}{1.006472in}}{\pgfqpoint{3.199031in}{1.002082in}}{\pgfqpoint{3.210081in}{1.002082in}}%
\pgfpathclose%
\pgfusepath{stroke,fill}%
\end{pgfscope}%
\begin{pgfscope}%
\pgfpathrectangle{\pgfqpoint{0.564660in}{0.521603in}}{\pgfqpoint{3.720000in}{3.020000in}} %
\pgfusepath{clip}%
\pgfsetbuttcap%
\pgfsetroundjoin%
\definecolor{currentfill}{rgb}{0.005882,0.700543,0.925638}%
\pgfsetfillcolor{currentfill}%
\pgfsetlinewidth{1.003750pt}%
\definecolor{currentstroke}{rgb}{0.005882,0.700543,0.925638}%
\pgfsetstrokecolor{currentstroke}%
\pgfsetdash{}{0pt}%
\pgfpathmoveto{\pgfqpoint{2.706316in}{1.513387in}}%
\pgfpathcurveto{\pgfqpoint{2.717366in}{1.513387in}}{\pgfqpoint{2.727965in}{1.517777in}}{\pgfqpoint{2.735779in}{1.525591in}}%
\pgfpathcurveto{\pgfqpoint{2.743593in}{1.533405in}}{\pgfqpoint{2.747983in}{1.544004in}}{\pgfqpoint{2.747983in}{1.555054in}}%
\pgfpathcurveto{\pgfqpoint{2.747983in}{1.566104in}}{\pgfqpoint{2.743593in}{1.576703in}}{\pgfqpoint{2.735779in}{1.584517in}}%
\pgfpathcurveto{\pgfqpoint{2.727965in}{1.592330in}}{\pgfqpoint{2.717366in}{1.596720in}}{\pgfqpoint{2.706316in}{1.596720in}}%
\pgfpathcurveto{\pgfqpoint{2.695266in}{1.596720in}}{\pgfqpoint{2.684667in}{1.592330in}}{\pgfqpoint{2.676853in}{1.584517in}}%
\pgfpathcurveto{\pgfqpoint{2.669040in}{1.576703in}}{\pgfqpoint{2.664649in}{1.566104in}}{\pgfqpoint{2.664649in}{1.555054in}}%
\pgfpathcurveto{\pgfqpoint{2.664649in}{1.544004in}}{\pgfqpoint{2.669040in}{1.533405in}}{\pgfqpoint{2.676853in}{1.525591in}}%
\pgfpathcurveto{\pgfqpoint{2.684667in}{1.517777in}}{\pgfqpoint{2.695266in}{1.513387in}}{\pgfqpoint{2.706316in}{1.513387in}}%
\pgfpathclose%
\pgfusepath{stroke,fill}%
\end{pgfscope}%
\begin{pgfscope}%
\pgfpathrectangle{\pgfqpoint{0.564660in}{0.521603in}}{\pgfqpoint{3.720000in}{3.020000in}} %
\pgfusepath{clip}%
\pgfsetbuttcap%
\pgfsetroundjoin%
\definecolor{currentfill}{rgb}{0.005882,0.700543,0.925638}%
\pgfsetfillcolor{currentfill}%
\pgfsetlinewidth{1.003750pt}%
\definecolor{currentstroke}{rgb}{0.005882,0.700543,0.925638}%
\pgfsetstrokecolor{currentstroke}%
\pgfsetdash{}{0pt}%
\pgfpathmoveto{\pgfqpoint{2.196868in}{2.803645in}}%
\pgfpathcurveto{\pgfqpoint{2.207918in}{2.803645in}}{\pgfqpoint{2.218517in}{2.808035in}}{\pgfqpoint{2.226330in}{2.815849in}}%
\pgfpathcurveto{\pgfqpoint{2.234144in}{2.823663in}}{\pgfqpoint{2.238534in}{2.834262in}}{\pgfqpoint{2.238534in}{2.845312in}}%
\pgfpathcurveto{\pgfqpoint{2.238534in}{2.856362in}}{\pgfqpoint{2.234144in}{2.866961in}}{\pgfqpoint{2.226330in}{2.874775in}}%
\pgfpathcurveto{\pgfqpoint{2.218517in}{2.882588in}}{\pgfqpoint{2.207918in}{2.886978in}}{\pgfqpoint{2.196868in}{2.886978in}}%
\pgfpathcurveto{\pgfqpoint{2.185818in}{2.886978in}}{\pgfqpoint{2.175219in}{2.882588in}}{\pgfqpoint{2.167405in}{2.874775in}}%
\pgfpathcurveto{\pgfqpoint{2.159591in}{2.866961in}}{\pgfqpoint{2.155201in}{2.856362in}}{\pgfqpoint{2.155201in}{2.845312in}}%
\pgfpathcurveto{\pgfqpoint{2.155201in}{2.834262in}}{\pgfqpoint{2.159591in}{2.823663in}}{\pgfqpoint{2.167405in}{2.815849in}}%
\pgfpathcurveto{\pgfqpoint{2.175219in}{2.808035in}}{\pgfqpoint{2.185818in}{2.803645in}}{\pgfqpoint{2.196868in}{2.803645in}}%
\pgfpathclose%
\pgfusepath{stroke,fill}%
\end{pgfscope}%
\begin{pgfscope}%
\pgfpathrectangle{\pgfqpoint{0.564660in}{0.521603in}}{\pgfqpoint{3.720000in}{3.020000in}} %
\pgfusepath{clip}%
\pgfsetbuttcap%
\pgfsetroundjoin%
\definecolor{currentfill}{rgb}{0.139216,0.536867,0.960122}%
\pgfsetfillcolor{currentfill}%
\pgfsetlinewidth{1.003750pt}%
\definecolor{currentstroke}{rgb}{0.139216,0.536867,0.960122}%
\pgfsetstrokecolor{currentstroke}%
\pgfsetdash{}{0pt}%
\pgfpathmoveto{\pgfqpoint{3.113579in}{2.722935in}}%
\pgfpathcurveto{\pgfqpoint{3.124629in}{2.722935in}}{\pgfqpoint{3.135229in}{2.727325in}}{\pgfqpoint{3.143042in}{2.735139in}}%
\pgfpathcurveto{\pgfqpoint{3.150856in}{2.742952in}}{\pgfqpoint{3.155246in}{2.753551in}}{\pgfqpoint{3.155246in}{2.764601in}}%
\pgfpathcurveto{\pgfqpoint{3.155246in}{2.775651in}}{\pgfqpoint{3.150856in}{2.786250in}}{\pgfqpoint{3.143042in}{2.794064in}}%
\pgfpathcurveto{\pgfqpoint{3.135229in}{2.801878in}}{\pgfqpoint{3.124629in}{2.806268in}}{\pgfqpoint{3.113579in}{2.806268in}}%
\pgfpathcurveto{\pgfqpoint{3.102529in}{2.806268in}}{\pgfqpoint{3.091930in}{2.801878in}}{\pgfqpoint{3.084117in}{2.794064in}}%
\pgfpathcurveto{\pgfqpoint{3.076303in}{2.786250in}}{\pgfqpoint{3.071913in}{2.775651in}}{\pgfqpoint{3.071913in}{2.764601in}}%
\pgfpathcurveto{\pgfqpoint{3.071913in}{2.753551in}}{\pgfqpoint{3.076303in}{2.742952in}}{\pgfqpoint{3.084117in}{2.735139in}}%
\pgfpathcurveto{\pgfqpoint{3.091930in}{2.727325in}}{\pgfqpoint{3.102529in}{2.722935in}}{\pgfqpoint{3.113579in}{2.722935in}}%
\pgfpathclose%
\pgfusepath{stroke,fill}%
\end{pgfscope}%
\begin{pgfscope}%
\pgfpathrectangle{\pgfqpoint{0.564660in}{0.521603in}}{\pgfqpoint{3.720000in}{3.020000in}} %
\pgfusepath{clip}%
\pgfsetbuttcap%
\pgfsetroundjoin%
\definecolor{currentfill}{rgb}{0.139216,0.536867,0.960122}%
\pgfsetfillcolor{currentfill}%
\pgfsetlinewidth{1.003750pt}%
\definecolor{currentstroke}{rgb}{0.139216,0.536867,0.960122}%
\pgfsetstrokecolor{currentstroke}%
\pgfsetdash{}{0pt}%
\pgfpathmoveto{\pgfqpoint{2.812260in}{2.289241in}}%
\pgfpathcurveto{\pgfqpoint{2.823310in}{2.289241in}}{\pgfqpoint{2.833909in}{2.293632in}}{\pgfqpoint{2.841723in}{2.301445in}}%
\pgfpathcurveto{\pgfqpoint{2.849536in}{2.309259in}}{\pgfqpoint{2.853926in}{2.319858in}}{\pgfqpoint{2.853926in}{2.330908in}}%
\pgfpathcurveto{\pgfqpoint{2.853926in}{2.341958in}}{\pgfqpoint{2.849536in}{2.352557in}}{\pgfqpoint{2.841723in}{2.360371in}}%
\pgfpathcurveto{\pgfqpoint{2.833909in}{2.368185in}}{\pgfqpoint{2.823310in}{2.372575in}}{\pgfqpoint{2.812260in}{2.372575in}}%
\pgfpathcurveto{\pgfqpoint{2.801210in}{2.372575in}}{\pgfqpoint{2.790611in}{2.368185in}}{\pgfqpoint{2.782797in}{2.360371in}}%
\pgfpathcurveto{\pgfqpoint{2.774983in}{2.352557in}}{\pgfqpoint{2.770593in}{2.341958in}}{\pgfqpoint{2.770593in}{2.330908in}}%
\pgfpathcurveto{\pgfqpoint{2.770593in}{2.319858in}}{\pgfqpoint{2.774983in}{2.309259in}}{\pgfqpoint{2.782797in}{2.301445in}}%
\pgfpathcurveto{\pgfqpoint{2.790611in}{2.293632in}}{\pgfqpoint{2.801210in}{2.289241in}}{\pgfqpoint{2.812260in}{2.289241in}}%
\pgfpathclose%
\pgfusepath{stroke,fill}%
\end{pgfscope}%
\begin{pgfscope}%
\pgfpathrectangle{\pgfqpoint{0.564660in}{0.521603in}}{\pgfqpoint{3.720000in}{3.020000in}} %
\pgfusepath{clip}%
\pgfsetbuttcap%
\pgfsetroundjoin%
\definecolor{currentfill}{rgb}{0.139216,0.536867,0.960122}%
\pgfsetfillcolor{currentfill}%
\pgfsetlinewidth{1.003750pt}%
\definecolor{currentstroke}{rgb}{0.139216,0.536867,0.960122}%
\pgfsetstrokecolor{currentstroke}%
\pgfsetdash{}{0pt}%
\pgfpathmoveto{\pgfqpoint{2.634682in}{2.190627in}}%
\pgfpathcurveto{\pgfqpoint{2.645732in}{2.190627in}}{\pgfqpoint{2.656331in}{2.195017in}}{\pgfqpoint{2.664145in}{2.202831in}}%
\pgfpathcurveto{\pgfqpoint{2.671958in}{2.210644in}}{\pgfqpoint{2.676349in}{2.221243in}}{\pgfqpoint{2.676349in}{2.232294in}}%
\pgfpathcurveto{\pgfqpoint{2.676349in}{2.243344in}}{\pgfqpoint{2.671958in}{2.253943in}}{\pgfqpoint{2.664145in}{2.261756in}}%
\pgfpathcurveto{\pgfqpoint{2.656331in}{2.269570in}}{\pgfqpoint{2.645732in}{2.273960in}}{\pgfqpoint{2.634682in}{2.273960in}}%
\pgfpathcurveto{\pgfqpoint{2.623632in}{2.273960in}}{\pgfqpoint{2.613033in}{2.269570in}}{\pgfqpoint{2.605219in}{2.261756in}}%
\pgfpathcurveto{\pgfqpoint{2.597406in}{2.253943in}}{\pgfqpoint{2.593015in}{2.243344in}}{\pgfqpoint{2.593015in}{2.232294in}}%
\pgfpathcurveto{\pgfqpoint{2.593015in}{2.221243in}}{\pgfqpoint{2.597406in}{2.210644in}}{\pgfqpoint{2.605219in}{2.202831in}}%
\pgfpathcurveto{\pgfqpoint{2.613033in}{2.195017in}}{\pgfqpoint{2.623632in}{2.190627in}}{\pgfqpoint{2.634682in}{2.190627in}}%
\pgfpathclose%
\pgfusepath{stroke,fill}%
\end{pgfscope}%
\begin{pgfscope}%
\pgfpathrectangle{\pgfqpoint{0.564660in}{0.521603in}}{\pgfqpoint{3.720000in}{3.020000in}} %
\pgfusepath{clip}%
\pgfsetbuttcap%
\pgfsetroundjoin%
\definecolor{currentfill}{rgb}{0.139216,0.536867,0.960122}%
\pgfsetfillcolor{currentfill}%
\pgfsetlinewidth{1.003750pt}%
\definecolor{currentstroke}{rgb}{0.139216,0.536867,0.960122}%
\pgfsetstrokecolor{currentstroke}%
\pgfsetdash{}{0pt}%
\pgfpathmoveto{\pgfqpoint{2.469078in}{1.382766in}}%
\pgfpathcurveto{\pgfqpoint{2.480128in}{1.382766in}}{\pgfqpoint{2.490727in}{1.387157in}}{\pgfqpoint{2.498541in}{1.394970in}}%
\pgfpathcurveto{\pgfqpoint{2.506354in}{1.402784in}}{\pgfqpoint{2.510744in}{1.413383in}}{\pgfqpoint{2.510744in}{1.424433in}}%
\pgfpathcurveto{\pgfqpoint{2.510744in}{1.435483in}}{\pgfqpoint{2.506354in}{1.446082in}}{\pgfqpoint{2.498541in}{1.453896in}}%
\pgfpathcurveto{\pgfqpoint{2.490727in}{1.461710in}}{\pgfqpoint{2.480128in}{1.466100in}}{\pgfqpoint{2.469078in}{1.466100in}}%
\pgfpathcurveto{\pgfqpoint{2.458028in}{1.466100in}}{\pgfqpoint{2.447429in}{1.461710in}}{\pgfqpoint{2.439615in}{1.453896in}}%
\pgfpathcurveto{\pgfqpoint{2.431801in}{1.446082in}}{\pgfqpoint{2.427411in}{1.435483in}}{\pgfqpoint{2.427411in}{1.424433in}}%
\pgfpathcurveto{\pgfqpoint{2.427411in}{1.413383in}}{\pgfqpoint{2.431801in}{1.402784in}}{\pgfqpoint{2.439615in}{1.394970in}}%
\pgfpathcurveto{\pgfqpoint{2.447429in}{1.387157in}}{\pgfqpoint{2.458028in}{1.382766in}}{\pgfqpoint{2.469078in}{1.382766in}}%
\pgfpathclose%
\pgfusepath{stroke,fill}%
\end{pgfscope}%
\begin{pgfscope}%
\pgfpathrectangle{\pgfqpoint{0.564660in}{0.521603in}}{\pgfqpoint{3.720000in}{3.020000in}} %
\pgfusepath{clip}%
\pgfsetbuttcap%
\pgfsetroundjoin%
\definecolor{currentfill}{rgb}{0.139216,0.536867,0.960122}%
\pgfsetfillcolor{currentfill}%
\pgfsetlinewidth{1.003750pt}%
\definecolor{currentstroke}{rgb}{0.139216,0.536867,0.960122}%
\pgfsetstrokecolor{currentstroke}%
\pgfsetdash{}{0pt}%
\pgfpathmoveto{\pgfqpoint{2.011780in}{3.301238in}}%
\pgfpathcurveto{\pgfqpoint{2.022830in}{3.301238in}}{\pgfqpoint{2.033429in}{3.305628in}}{\pgfqpoint{2.041243in}{3.313442in}}%
\pgfpathcurveto{\pgfqpoint{2.049057in}{3.321256in}}{\pgfqpoint{2.053447in}{3.331855in}}{\pgfqpoint{2.053447in}{3.342905in}}%
\pgfpathcurveto{\pgfqpoint{2.053447in}{3.353955in}}{\pgfqpoint{2.049057in}{3.364554in}}{\pgfqpoint{2.041243in}{3.372368in}}%
\pgfpathcurveto{\pgfqpoint{2.033429in}{3.380181in}}{\pgfqpoint{2.022830in}{3.384572in}}{\pgfqpoint{2.011780in}{3.384572in}}%
\pgfpathcurveto{\pgfqpoint{2.000730in}{3.384572in}}{\pgfqpoint{1.990131in}{3.380181in}}{\pgfqpoint{1.982317in}{3.372368in}}%
\pgfpathcurveto{\pgfqpoint{1.974504in}{3.364554in}}{\pgfqpoint{1.970113in}{3.353955in}}{\pgfqpoint{1.970113in}{3.342905in}}%
\pgfpathcurveto{\pgfqpoint{1.970113in}{3.331855in}}{\pgfqpoint{1.974504in}{3.321256in}}{\pgfqpoint{1.982317in}{3.313442in}}%
\pgfpathcurveto{\pgfqpoint{1.990131in}{3.305628in}}{\pgfqpoint{2.000730in}{3.301238in}}{\pgfqpoint{2.011780in}{3.301238in}}%
\pgfpathclose%
\pgfusepath{stroke,fill}%
\end{pgfscope}%
\begin{pgfscope}%
\pgfsetbuttcap%
\pgfsetroundjoin%
\definecolor{currentfill}{rgb}{0.000000,0.000000,0.000000}%
\pgfsetfillcolor{currentfill}%
\pgfsetlinewidth{0.803000pt}%
\definecolor{currentstroke}{rgb}{0.000000,0.000000,0.000000}%
\pgfsetstrokecolor{currentstroke}%
\pgfsetdash{}{0pt}%
\pgfsys@defobject{currentmarker}{\pgfqpoint{0.000000in}{-0.048611in}}{\pgfqpoint{0.000000in}{0.000000in}}{%
\pgfpathmoveto{\pgfqpoint{0.000000in}{0.000000in}}%
\pgfpathlineto{\pgfqpoint{0.000000in}{-0.048611in}}%
\pgfusepath{stroke,fill}%
}%
\begin{pgfscope}%
\pgfsys@transformshift{0.564660in}{0.521603in}%
\pgfsys@useobject{currentmarker}{}%
\end{pgfscope}%
\end{pgfscope}%
\begin{pgfscope}%
\pgftext[x=0.564660in,y=0.424381in,,top]{\rmfamily\fontsize{10.000000}{12.000000}\selectfont \(\displaystyle 10^{-2}\)}%
\end{pgfscope}%
\begin{pgfscope}%
\pgfsetbuttcap%
\pgfsetroundjoin%
\definecolor{currentfill}{rgb}{0.000000,0.000000,0.000000}%
\pgfsetfillcolor{currentfill}%
\pgfsetlinewidth{0.803000pt}%
\definecolor{currentstroke}{rgb}{0.000000,0.000000,0.000000}%
\pgfsetstrokecolor{currentstroke}%
\pgfsetdash{}{0pt}%
\pgfsys@defobject{currentmarker}{\pgfqpoint{0.000000in}{-0.048611in}}{\pgfqpoint{0.000000in}{0.000000in}}{%
\pgfpathmoveto{\pgfqpoint{0.000000in}{0.000000in}}%
\pgfpathlineto{\pgfqpoint{0.000000in}{-0.048611in}}%
\pgfusepath{stroke,fill}%
}%
\begin{pgfscope}%
\pgfsys@transformshift{2.397001in}{0.521603in}%
\pgfsys@useobject{currentmarker}{}%
\end{pgfscope}%
\end{pgfscope}%
\begin{pgfscope}%
\pgftext[x=2.397001in,y=0.424381in,,top]{\rmfamily\fontsize{10.000000}{12.000000}\selectfont \(\displaystyle 10^{-1}\)}%
\end{pgfscope}%
\begin{pgfscope}%
\pgfsetbuttcap%
\pgfsetroundjoin%
\definecolor{currentfill}{rgb}{0.000000,0.000000,0.000000}%
\pgfsetfillcolor{currentfill}%
\pgfsetlinewidth{0.803000pt}%
\definecolor{currentstroke}{rgb}{0.000000,0.000000,0.000000}%
\pgfsetstrokecolor{currentstroke}%
\pgfsetdash{}{0pt}%
\pgfsys@defobject{currentmarker}{\pgfqpoint{0.000000in}{-0.048611in}}{\pgfqpoint{0.000000in}{0.000000in}}{%
\pgfpathmoveto{\pgfqpoint{0.000000in}{0.000000in}}%
\pgfpathlineto{\pgfqpoint{0.000000in}{-0.048611in}}%
\pgfusepath{stroke,fill}%
}%
\begin{pgfscope}%
\pgfsys@transformshift{4.229341in}{0.521603in}%
\pgfsys@useobject{currentmarker}{}%
\end{pgfscope}%
\end{pgfscope}%
\begin{pgfscope}%
\pgftext[x=4.229341in,y=0.424381in,,top]{\rmfamily\fontsize{10.000000}{12.000000}\selectfont \(\displaystyle 10^{0}\)}%
\end{pgfscope}%
\begin{pgfscope}%
\pgfsetbuttcap%
\pgfsetroundjoin%
\definecolor{currentfill}{rgb}{0.000000,0.000000,0.000000}%
\pgfsetfillcolor{currentfill}%
\pgfsetlinewidth{0.602250pt}%
\definecolor{currentstroke}{rgb}{0.000000,0.000000,0.000000}%
\pgfsetstrokecolor{currentstroke}%
\pgfsetdash{}{0pt}%
\pgfsys@defobject{currentmarker}{\pgfqpoint{0.000000in}{-0.027778in}}{\pgfqpoint{0.000000in}{0.000000in}}{%
\pgfpathmoveto{\pgfqpoint{0.000000in}{0.000000in}}%
\pgfpathlineto{\pgfqpoint{0.000000in}{-0.027778in}}%
\pgfusepath{stroke,fill}%
}%
\begin{pgfscope}%
\pgfsys@transformshift{1.116250in}{0.521603in}%
\pgfsys@useobject{currentmarker}{}%
\end{pgfscope}%
\end{pgfscope}%
\begin{pgfscope}%
\pgfsetbuttcap%
\pgfsetroundjoin%
\definecolor{currentfill}{rgb}{0.000000,0.000000,0.000000}%
\pgfsetfillcolor{currentfill}%
\pgfsetlinewidth{0.602250pt}%
\definecolor{currentstroke}{rgb}{0.000000,0.000000,0.000000}%
\pgfsetstrokecolor{currentstroke}%
\pgfsetdash{}{0pt}%
\pgfsys@defobject{currentmarker}{\pgfqpoint{0.000000in}{-0.027778in}}{\pgfqpoint{0.000000in}{0.000000in}}{%
\pgfpathmoveto{\pgfqpoint{0.000000in}{0.000000in}}%
\pgfpathlineto{\pgfqpoint{0.000000in}{-0.027778in}}%
\pgfusepath{stroke,fill}%
}%
\begin{pgfscope}%
\pgfsys@transformshift{1.438909in}{0.521603in}%
\pgfsys@useobject{currentmarker}{}%
\end{pgfscope}%
\end{pgfscope}%
\begin{pgfscope}%
\pgfsetbuttcap%
\pgfsetroundjoin%
\definecolor{currentfill}{rgb}{0.000000,0.000000,0.000000}%
\pgfsetfillcolor{currentfill}%
\pgfsetlinewidth{0.602250pt}%
\definecolor{currentstroke}{rgb}{0.000000,0.000000,0.000000}%
\pgfsetstrokecolor{currentstroke}%
\pgfsetdash{}{0pt}%
\pgfsys@defobject{currentmarker}{\pgfqpoint{0.000000in}{-0.027778in}}{\pgfqpoint{0.000000in}{0.000000in}}{%
\pgfpathmoveto{\pgfqpoint{0.000000in}{0.000000in}}%
\pgfpathlineto{\pgfqpoint{0.000000in}{-0.027778in}}%
\pgfusepath{stroke,fill}%
}%
\begin{pgfscope}%
\pgfsys@transformshift{1.667839in}{0.521603in}%
\pgfsys@useobject{currentmarker}{}%
\end{pgfscope}%
\end{pgfscope}%
\begin{pgfscope}%
\pgfsetbuttcap%
\pgfsetroundjoin%
\definecolor{currentfill}{rgb}{0.000000,0.000000,0.000000}%
\pgfsetfillcolor{currentfill}%
\pgfsetlinewidth{0.602250pt}%
\definecolor{currentstroke}{rgb}{0.000000,0.000000,0.000000}%
\pgfsetstrokecolor{currentstroke}%
\pgfsetdash{}{0pt}%
\pgfsys@defobject{currentmarker}{\pgfqpoint{0.000000in}{-0.027778in}}{\pgfqpoint{0.000000in}{0.000000in}}{%
\pgfpathmoveto{\pgfqpoint{0.000000in}{0.000000in}}%
\pgfpathlineto{\pgfqpoint{0.000000in}{-0.027778in}}%
\pgfusepath{stroke,fill}%
}%
\begin{pgfscope}%
\pgfsys@transformshift{1.845411in}{0.521603in}%
\pgfsys@useobject{currentmarker}{}%
\end{pgfscope}%
\end{pgfscope}%
\begin{pgfscope}%
\pgfsetbuttcap%
\pgfsetroundjoin%
\definecolor{currentfill}{rgb}{0.000000,0.000000,0.000000}%
\pgfsetfillcolor{currentfill}%
\pgfsetlinewidth{0.602250pt}%
\definecolor{currentstroke}{rgb}{0.000000,0.000000,0.000000}%
\pgfsetstrokecolor{currentstroke}%
\pgfsetdash{}{0pt}%
\pgfsys@defobject{currentmarker}{\pgfqpoint{0.000000in}{-0.027778in}}{\pgfqpoint{0.000000in}{0.000000in}}{%
\pgfpathmoveto{\pgfqpoint{0.000000in}{0.000000in}}%
\pgfpathlineto{\pgfqpoint{0.000000in}{-0.027778in}}%
\pgfusepath{stroke,fill}%
}%
\begin{pgfscope}%
\pgfsys@transformshift{1.990498in}{0.521603in}%
\pgfsys@useobject{currentmarker}{}%
\end{pgfscope}%
\end{pgfscope}%
\begin{pgfscope}%
\pgfsetbuttcap%
\pgfsetroundjoin%
\definecolor{currentfill}{rgb}{0.000000,0.000000,0.000000}%
\pgfsetfillcolor{currentfill}%
\pgfsetlinewidth{0.602250pt}%
\definecolor{currentstroke}{rgb}{0.000000,0.000000,0.000000}%
\pgfsetstrokecolor{currentstroke}%
\pgfsetdash{}{0pt}%
\pgfsys@defobject{currentmarker}{\pgfqpoint{0.000000in}{-0.027778in}}{\pgfqpoint{0.000000in}{0.000000in}}{%
\pgfpathmoveto{\pgfqpoint{0.000000in}{0.000000in}}%
\pgfpathlineto{\pgfqpoint{0.000000in}{-0.027778in}}%
\pgfusepath{stroke,fill}%
}%
\begin{pgfscope}%
\pgfsys@transformshift{2.113168in}{0.521603in}%
\pgfsys@useobject{currentmarker}{}%
\end{pgfscope}%
\end{pgfscope}%
\begin{pgfscope}%
\pgfsetbuttcap%
\pgfsetroundjoin%
\definecolor{currentfill}{rgb}{0.000000,0.000000,0.000000}%
\pgfsetfillcolor{currentfill}%
\pgfsetlinewidth{0.602250pt}%
\definecolor{currentstroke}{rgb}{0.000000,0.000000,0.000000}%
\pgfsetstrokecolor{currentstroke}%
\pgfsetdash{}{0pt}%
\pgfsys@defobject{currentmarker}{\pgfqpoint{0.000000in}{-0.027778in}}{\pgfqpoint{0.000000in}{0.000000in}}{%
\pgfpathmoveto{\pgfqpoint{0.000000in}{0.000000in}}%
\pgfpathlineto{\pgfqpoint{0.000000in}{-0.027778in}}%
\pgfusepath{stroke,fill}%
}%
\begin{pgfscope}%
\pgfsys@transformshift{2.219429in}{0.521603in}%
\pgfsys@useobject{currentmarker}{}%
\end{pgfscope}%
\end{pgfscope}%
\begin{pgfscope}%
\pgfsetbuttcap%
\pgfsetroundjoin%
\definecolor{currentfill}{rgb}{0.000000,0.000000,0.000000}%
\pgfsetfillcolor{currentfill}%
\pgfsetlinewidth{0.602250pt}%
\definecolor{currentstroke}{rgb}{0.000000,0.000000,0.000000}%
\pgfsetstrokecolor{currentstroke}%
\pgfsetdash{}{0pt}%
\pgfsys@defobject{currentmarker}{\pgfqpoint{0.000000in}{-0.027778in}}{\pgfqpoint{0.000000in}{0.000000in}}{%
\pgfpathmoveto{\pgfqpoint{0.000000in}{0.000000in}}%
\pgfpathlineto{\pgfqpoint{0.000000in}{-0.027778in}}%
\pgfusepath{stroke,fill}%
}%
\begin{pgfscope}%
\pgfsys@transformshift{2.313157in}{0.521603in}%
\pgfsys@useobject{currentmarker}{}%
\end{pgfscope}%
\end{pgfscope}%
\begin{pgfscope}%
\pgfsetbuttcap%
\pgfsetroundjoin%
\definecolor{currentfill}{rgb}{0.000000,0.000000,0.000000}%
\pgfsetfillcolor{currentfill}%
\pgfsetlinewidth{0.602250pt}%
\definecolor{currentstroke}{rgb}{0.000000,0.000000,0.000000}%
\pgfsetstrokecolor{currentstroke}%
\pgfsetdash{}{0pt}%
\pgfsys@defobject{currentmarker}{\pgfqpoint{0.000000in}{-0.027778in}}{\pgfqpoint{0.000000in}{0.000000in}}{%
\pgfpathmoveto{\pgfqpoint{0.000000in}{0.000000in}}%
\pgfpathlineto{\pgfqpoint{0.000000in}{-0.027778in}}%
\pgfusepath{stroke,fill}%
}%
\begin{pgfscope}%
\pgfsys@transformshift{2.948590in}{0.521603in}%
\pgfsys@useobject{currentmarker}{}%
\end{pgfscope}%
\end{pgfscope}%
\begin{pgfscope}%
\pgfsetbuttcap%
\pgfsetroundjoin%
\definecolor{currentfill}{rgb}{0.000000,0.000000,0.000000}%
\pgfsetfillcolor{currentfill}%
\pgfsetlinewidth{0.602250pt}%
\definecolor{currentstroke}{rgb}{0.000000,0.000000,0.000000}%
\pgfsetstrokecolor{currentstroke}%
\pgfsetdash{}{0pt}%
\pgfsys@defobject{currentmarker}{\pgfqpoint{0.000000in}{-0.027778in}}{\pgfqpoint{0.000000in}{0.000000in}}{%
\pgfpathmoveto{\pgfqpoint{0.000000in}{0.000000in}}%
\pgfpathlineto{\pgfqpoint{0.000000in}{-0.027778in}}%
\pgfusepath{stroke,fill}%
}%
\begin{pgfscope}%
\pgfsys@transformshift{3.271249in}{0.521603in}%
\pgfsys@useobject{currentmarker}{}%
\end{pgfscope}%
\end{pgfscope}%
\begin{pgfscope}%
\pgfsetbuttcap%
\pgfsetroundjoin%
\definecolor{currentfill}{rgb}{0.000000,0.000000,0.000000}%
\pgfsetfillcolor{currentfill}%
\pgfsetlinewidth{0.602250pt}%
\definecolor{currentstroke}{rgb}{0.000000,0.000000,0.000000}%
\pgfsetstrokecolor{currentstroke}%
\pgfsetdash{}{0pt}%
\pgfsys@defobject{currentmarker}{\pgfqpoint{0.000000in}{-0.027778in}}{\pgfqpoint{0.000000in}{0.000000in}}{%
\pgfpathmoveto{\pgfqpoint{0.000000in}{0.000000in}}%
\pgfpathlineto{\pgfqpoint{0.000000in}{-0.027778in}}%
\pgfusepath{stroke,fill}%
}%
\begin{pgfscope}%
\pgfsys@transformshift{3.500180in}{0.521603in}%
\pgfsys@useobject{currentmarker}{}%
\end{pgfscope}%
\end{pgfscope}%
\begin{pgfscope}%
\pgfsetbuttcap%
\pgfsetroundjoin%
\definecolor{currentfill}{rgb}{0.000000,0.000000,0.000000}%
\pgfsetfillcolor{currentfill}%
\pgfsetlinewidth{0.602250pt}%
\definecolor{currentstroke}{rgb}{0.000000,0.000000,0.000000}%
\pgfsetstrokecolor{currentstroke}%
\pgfsetdash{}{0pt}%
\pgfsys@defobject{currentmarker}{\pgfqpoint{0.000000in}{-0.027778in}}{\pgfqpoint{0.000000in}{0.000000in}}{%
\pgfpathmoveto{\pgfqpoint{0.000000in}{0.000000in}}%
\pgfpathlineto{\pgfqpoint{0.000000in}{-0.027778in}}%
\pgfusepath{stroke,fill}%
}%
\begin{pgfscope}%
\pgfsys@transformshift{3.677752in}{0.521603in}%
\pgfsys@useobject{currentmarker}{}%
\end{pgfscope}%
\end{pgfscope}%
\begin{pgfscope}%
\pgfsetbuttcap%
\pgfsetroundjoin%
\definecolor{currentfill}{rgb}{0.000000,0.000000,0.000000}%
\pgfsetfillcolor{currentfill}%
\pgfsetlinewidth{0.602250pt}%
\definecolor{currentstroke}{rgb}{0.000000,0.000000,0.000000}%
\pgfsetstrokecolor{currentstroke}%
\pgfsetdash{}{0pt}%
\pgfsys@defobject{currentmarker}{\pgfqpoint{0.000000in}{-0.027778in}}{\pgfqpoint{0.000000in}{0.000000in}}{%
\pgfpathmoveto{\pgfqpoint{0.000000in}{0.000000in}}%
\pgfpathlineto{\pgfqpoint{0.000000in}{-0.027778in}}%
\pgfusepath{stroke,fill}%
}%
\begin{pgfscope}%
\pgfsys@transformshift{3.822839in}{0.521603in}%
\pgfsys@useobject{currentmarker}{}%
\end{pgfscope}%
\end{pgfscope}%
\begin{pgfscope}%
\pgfsetbuttcap%
\pgfsetroundjoin%
\definecolor{currentfill}{rgb}{0.000000,0.000000,0.000000}%
\pgfsetfillcolor{currentfill}%
\pgfsetlinewidth{0.602250pt}%
\definecolor{currentstroke}{rgb}{0.000000,0.000000,0.000000}%
\pgfsetstrokecolor{currentstroke}%
\pgfsetdash{}{0pt}%
\pgfsys@defobject{currentmarker}{\pgfqpoint{0.000000in}{-0.027778in}}{\pgfqpoint{0.000000in}{0.000000in}}{%
\pgfpathmoveto{\pgfqpoint{0.000000in}{0.000000in}}%
\pgfpathlineto{\pgfqpoint{0.000000in}{-0.027778in}}%
\pgfusepath{stroke,fill}%
}%
\begin{pgfscope}%
\pgfsys@transformshift{3.945508in}{0.521603in}%
\pgfsys@useobject{currentmarker}{}%
\end{pgfscope}%
\end{pgfscope}%
\begin{pgfscope}%
\pgfsetbuttcap%
\pgfsetroundjoin%
\definecolor{currentfill}{rgb}{0.000000,0.000000,0.000000}%
\pgfsetfillcolor{currentfill}%
\pgfsetlinewidth{0.602250pt}%
\definecolor{currentstroke}{rgb}{0.000000,0.000000,0.000000}%
\pgfsetstrokecolor{currentstroke}%
\pgfsetdash{}{0pt}%
\pgfsys@defobject{currentmarker}{\pgfqpoint{0.000000in}{-0.027778in}}{\pgfqpoint{0.000000in}{0.000000in}}{%
\pgfpathmoveto{\pgfqpoint{0.000000in}{0.000000in}}%
\pgfpathlineto{\pgfqpoint{0.000000in}{-0.027778in}}%
\pgfusepath{stroke,fill}%
}%
\begin{pgfscope}%
\pgfsys@transformshift{4.051769in}{0.521603in}%
\pgfsys@useobject{currentmarker}{}%
\end{pgfscope}%
\end{pgfscope}%
\begin{pgfscope}%
\pgfsetbuttcap%
\pgfsetroundjoin%
\definecolor{currentfill}{rgb}{0.000000,0.000000,0.000000}%
\pgfsetfillcolor{currentfill}%
\pgfsetlinewidth{0.602250pt}%
\definecolor{currentstroke}{rgb}{0.000000,0.000000,0.000000}%
\pgfsetstrokecolor{currentstroke}%
\pgfsetdash{}{0pt}%
\pgfsys@defobject{currentmarker}{\pgfqpoint{0.000000in}{-0.027778in}}{\pgfqpoint{0.000000in}{0.000000in}}{%
\pgfpathmoveto{\pgfqpoint{0.000000in}{0.000000in}}%
\pgfpathlineto{\pgfqpoint{0.000000in}{-0.027778in}}%
\pgfusepath{stroke,fill}%
}%
\begin{pgfscope}%
\pgfsys@transformshift{4.145498in}{0.521603in}%
\pgfsys@useobject{currentmarker}{}%
\end{pgfscope}%
\end{pgfscope}%
\begin{pgfscope}%
\pgftext[x=2.424660in,y=0.234413in,,top]{\rmfamily\fontsize{10.000000}{12.000000}\selectfont \(\displaystyle \mathbf{W}\mbox{e}\)}%
\end{pgfscope}%
\begin{pgfscope}%
\pgfsetbuttcap%
\pgfsetroundjoin%
\definecolor{currentfill}{rgb}{0.000000,0.000000,0.000000}%
\pgfsetfillcolor{currentfill}%
\pgfsetlinewidth{0.803000pt}%
\definecolor{currentstroke}{rgb}{0.000000,0.000000,0.000000}%
\pgfsetstrokecolor{currentstroke}%
\pgfsetdash{}{0pt}%
\pgfsys@defobject{currentmarker}{\pgfqpoint{-0.048611in}{0.000000in}}{\pgfqpoint{0.000000in}{0.000000in}}{%
\pgfpathmoveto{\pgfqpoint{0.000000in}{0.000000in}}%
\pgfpathlineto{\pgfqpoint{-0.048611in}{0.000000in}}%
\pgfusepath{stroke,fill}%
}%
\begin{pgfscope}%
\pgfsys@transformshift{0.564660in}{1.024740in}%
\pgfsys@useobject{currentmarker}{}%
\end{pgfscope}%
\end{pgfscope}%
\begin{pgfscope}%
\pgftext[x=0.289968in,y=0.971978in,left,base]{\rmfamily\fontsize{10.000000}{12.000000}\selectfont \(\displaystyle 0.4\)}%
\end{pgfscope}%
\begin{pgfscope}%
\pgfsetbuttcap%
\pgfsetroundjoin%
\definecolor{currentfill}{rgb}{0.000000,0.000000,0.000000}%
\pgfsetfillcolor{currentfill}%
\pgfsetlinewidth{0.803000pt}%
\definecolor{currentstroke}{rgb}{0.000000,0.000000,0.000000}%
\pgfsetstrokecolor{currentstroke}%
\pgfsetdash{}{0pt}%
\pgfsys@defobject{currentmarker}{\pgfqpoint{-0.048611in}{0.000000in}}{\pgfqpoint{0.000000in}{0.000000in}}{%
\pgfpathmoveto{\pgfqpoint{0.000000in}{0.000000in}}%
\pgfpathlineto{\pgfqpoint{-0.048611in}{0.000000in}}%
\pgfusepath{stroke,fill}%
}%
\begin{pgfscope}%
\pgfsys@transformshift{0.564660in}{1.643093in}%
\pgfsys@useobject{currentmarker}{}%
\end{pgfscope}%
\end{pgfscope}%
\begin{pgfscope}%
\pgftext[x=0.289968in,y=1.590331in,left,base]{\rmfamily\fontsize{10.000000}{12.000000}\selectfont \(\displaystyle 0.5\)}%
\end{pgfscope}%
\begin{pgfscope}%
\pgfsetbuttcap%
\pgfsetroundjoin%
\definecolor{currentfill}{rgb}{0.000000,0.000000,0.000000}%
\pgfsetfillcolor{currentfill}%
\pgfsetlinewidth{0.803000pt}%
\definecolor{currentstroke}{rgb}{0.000000,0.000000,0.000000}%
\pgfsetstrokecolor{currentstroke}%
\pgfsetdash{}{0pt}%
\pgfsys@defobject{currentmarker}{\pgfqpoint{-0.048611in}{0.000000in}}{\pgfqpoint{0.000000in}{0.000000in}}{%
\pgfpathmoveto{\pgfqpoint{0.000000in}{0.000000in}}%
\pgfpathlineto{\pgfqpoint{-0.048611in}{0.000000in}}%
\pgfusepath{stroke,fill}%
}%
\begin{pgfscope}%
\pgfsys@transformshift{0.564660in}{2.261445in}%
\pgfsys@useobject{currentmarker}{}%
\end{pgfscope}%
\end{pgfscope}%
\begin{pgfscope}%
\pgftext[x=0.289968in,y=2.208684in,left,base]{\rmfamily\fontsize{10.000000}{12.000000}\selectfont \(\displaystyle 0.6\)}%
\end{pgfscope}%
\begin{pgfscope}%
\pgfsetbuttcap%
\pgfsetroundjoin%
\definecolor{currentfill}{rgb}{0.000000,0.000000,0.000000}%
\pgfsetfillcolor{currentfill}%
\pgfsetlinewidth{0.803000pt}%
\definecolor{currentstroke}{rgb}{0.000000,0.000000,0.000000}%
\pgfsetstrokecolor{currentstroke}%
\pgfsetdash{}{0pt}%
\pgfsys@defobject{currentmarker}{\pgfqpoint{-0.048611in}{0.000000in}}{\pgfqpoint{0.000000in}{0.000000in}}{%
\pgfpathmoveto{\pgfqpoint{0.000000in}{0.000000in}}%
\pgfpathlineto{\pgfqpoint{-0.048611in}{0.000000in}}%
\pgfusepath{stroke,fill}%
}%
\begin{pgfscope}%
\pgfsys@transformshift{0.564660in}{2.879798in}%
\pgfsys@useobject{currentmarker}{}%
\end{pgfscope}%
\end{pgfscope}%
\begin{pgfscope}%
\pgftext[x=0.289968in,y=2.827036in,left,base]{\rmfamily\fontsize{10.000000}{12.000000}\selectfont \(\displaystyle 0.7\)}%
\end{pgfscope}%
\begin{pgfscope}%
\pgfsetbuttcap%
\pgfsetroundjoin%
\definecolor{currentfill}{rgb}{0.000000,0.000000,0.000000}%
\pgfsetfillcolor{currentfill}%
\pgfsetlinewidth{0.803000pt}%
\definecolor{currentstroke}{rgb}{0.000000,0.000000,0.000000}%
\pgfsetstrokecolor{currentstroke}%
\pgfsetdash{}{0pt}%
\pgfsys@defobject{currentmarker}{\pgfqpoint{-0.048611in}{0.000000in}}{\pgfqpoint{0.000000in}{0.000000in}}{%
\pgfpathmoveto{\pgfqpoint{0.000000in}{0.000000in}}%
\pgfpathlineto{\pgfqpoint{-0.048611in}{0.000000in}}%
\pgfusepath{stroke,fill}%
}%
\begin{pgfscope}%
\pgfsys@transformshift{0.564660in}{3.498150in}%
\pgfsys@useobject{currentmarker}{}%
\end{pgfscope}%
\end{pgfscope}%
\begin{pgfscope}%
\pgftext[x=0.289968in,y=3.445389in,left,base]{\rmfamily\fontsize{10.000000}{12.000000}\selectfont \(\displaystyle 0.8\)}%
\end{pgfscope}%
\begin{pgfscope}%
\pgftext[x=0.234413in,y=2.031603in,,bottom,rotate=90.000000]{\rmfamily\fontsize{10.000000}{12.000000}\selectfont \(\displaystyle C_r\)}%
\end{pgfscope}%
\begin{pgfscope}%
\pgfsetrectcap%
\pgfsetmiterjoin%
\pgfsetlinewidth{0.803000pt}%
\definecolor{currentstroke}{rgb}{0.000000,0.000000,0.000000}%
\pgfsetstrokecolor{currentstroke}%
\pgfsetdash{}{0pt}%
\pgfpathmoveto{\pgfqpoint{0.564660in}{0.521603in}}%
\pgfpathlineto{\pgfqpoint{0.564660in}{3.541603in}}%
\pgfusepath{stroke}%
\end{pgfscope}%
\begin{pgfscope}%
\pgfsetrectcap%
\pgfsetmiterjoin%
\pgfsetlinewidth{0.803000pt}%
\definecolor{currentstroke}{rgb}{0.000000,0.000000,0.000000}%
\pgfsetstrokecolor{currentstroke}%
\pgfsetdash{}{0pt}%
\pgfpathmoveto{\pgfqpoint{4.284660in}{0.521603in}}%
\pgfpathlineto{\pgfqpoint{4.284660in}{3.541603in}}%
\pgfusepath{stroke}%
\end{pgfscope}%
\begin{pgfscope}%
\pgfsetrectcap%
\pgfsetmiterjoin%
\pgfsetlinewidth{0.803000pt}%
\definecolor{currentstroke}{rgb}{0.000000,0.000000,0.000000}%
\pgfsetstrokecolor{currentstroke}%
\pgfsetdash{}{0pt}%
\pgfpathmoveto{\pgfqpoint{0.564660in}{0.521603in}}%
\pgfpathlineto{\pgfqpoint{4.284660in}{0.521603in}}%
\pgfusepath{stroke}%
\end{pgfscope}%
\begin{pgfscope}%
\pgfsetrectcap%
\pgfsetmiterjoin%
\pgfsetlinewidth{0.803000pt}%
\definecolor{currentstroke}{rgb}{0.000000,0.000000,0.000000}%
\pgfsetstrokecolor{currentstroke}%
\pgfsetdash{}{0pt}%
\pgfpathmoveto{\pgfqpoint{0.564660in}{3.541603in}}%
\pgfpathlineto{\pgfqpoint{4.284660in}{3.541603in}}%
\pgfusepath{stroke}%
\end{pgfscope}%
\begin{pgfscope}%
\pgfsetbuttcap%
\pgfsetmiterjoin%
\definecolor{currentfill}{rgb}{1.000000,1.000000,1.000000}%
\pgfsetfillcolor{currentfill}%
\pgfsetfillopacity{0.700000}%
\pgfsetlinewidth{1.003750pt}%
\definecolor{currentstroke}{rgb}{0.500000,0.500000,0.500000}%
\pgfsetstrokecolor{currentstroke}%
\pgfsetstrokeopacity{0.700000}%
\pgfsetdash{}{0pt}%
\pgfpathmoveto{\pgfqpoint{0.887319in}{3.107570in}}%
\pgfpathlineto{\pgfqpoint{1.534885in}{3.107570in}}%
\pgfpathquadraticcurveto{\pgfqpoint{1.576552in}{3.107570in}}{\pgfqpoint{1.576552in}{3.149237in}}%
\pgfpathlineto{\pgfqpoint{1.576552in}{3.294497in}}%
\pgfpathquadraticcurveto{\pgfqpoint{1.576552in}{3.336164in}}{\pgfqpoint{1.534885in}{3.336164in}}%
\pgfpathlineto{\pgfqpoint{0.887319in}{3.336164in}}%
\pgfpathquadraticcurveto{\pgfqpoint{0.845653in}{3.336164in}}{\pgfqpoint{0.845653in}{3.294497in}}%
\pgfpathlineto{\pgfqpoint{0.845653in}{3.149237in}}%
\pgfpathquadraticcurveto{\pgfqpoint{0.845653in}{3.107570in}}{\pgfqpoint{0.887319in}{3.107570in}}%
\pgfpathclose%
\pgfusepath{stroke,fill}%
\end{pgfscope}%
\begin{pgfscope}%
\pgftext[x=0.887319in,y=3.188974in,left,base]{\rmfamily\fontsize{10.000000}{12.000000}\selectfont \(\displaystyle \mathbf{O}\mbox{h}_{\mu} = 2.2\)}%
\end{pgfscope}%
\begin{pgfscope}%
\pgfpathrectangle{\pgfqpoint{4.517160in}{0.521603in}}{\pgfqpoint{0.151000in}{3.020000in}} %
\pgfusepath{clip}%
\pgfsetbuttcap%
\pgfsetmiterjoin%
\definecolor{currentfill}{rgb}{1.000000,1.000000,1.000000}%
\pgfsetfillcolor{currentfill}%
\pgfsetlinewidth{0.010037pt}%
\definecolor{currentstroke}{rgb}{1.000000,1.000000,1.000000}%
\pgfsetstrokecolor{currentstroke}%
\pgfsetdash{}{0pt}%
\pgfpathmoveto{\pgfqpoint{4.517160in}{0.521603in}}%
\pgfpathlineto{\pgfqpoint{4.517160in}{0.533400in}}%
\pgfpathlineto{\pgfqpoint{4.517160in}{3.529806in}}%
\pgfpathlineto{\pgfqpoint{4.517160in}{3.541603in}}%
\pgfpathlineto{\pgfqpoint{4.668160in}{3.541603in}}%
\pgfpathlineto{\pgfqpoint{4.668160in}{3.529806in}}%
\pgfpathlineto{\pgfqpoint{4.668160in}{0.533400in}}%
\pgfpathlineto{\pgfqpoint{4.668160in}{0.521603in}}%
\pgfpathclose%
\pgfusepath{stroke,fill}%
\end{pgfscope}%
\begin{pgfscope}%
\pgfsys@transformshift{4.520000in}{0.526603in}%
\pgftext[left,bottom]{\pgfimage[interpolate=true,width=0.150000in,height=3.020000in]{restitution-img0.png}}%
\end{pgfscope}%
\begin{pgfscope}%
\pgfsetbuttcap%
\pgfsetroundjoin%
\definecolor{currentfill}{rgb}{0.000000,0.000000,0.000000}%
\pgfsetfillcolor{currentfill}%
\pgfsetlinewidth{0.803000pt}%
\definecolor{currentstroke}{rgb}{0.000000,0.000000,0.000000}%
\pgfsetstrokecolor{currentstroke}%
\pgfsetdash{}{0pt}%
\pgfsys@defobject{currentmarker}{\pgfqpoint{0.000000in}{0.000000in}}{\pgfqpoint{0.048611in}{0.000000in}}{%
\pgfpathmoveto{\pgfqpoint{0.000000in}{0.000000in}}%
\pgfpathlineto{\pgfqpoint{0.048611in}{0.000000in}}%
\pgfusepath{stroke,fill}%
}%
\begin{pgfscope}%
\pgfsys@transformshift{4.668160in}{0.925593in}%
\pgfsys@useobject{currentmarker}{}%
\end{pgfscope}%
\end{pgfscope}%
\begin{pgfscope}%
\pgftext[x=4.765383in,y=0.872831in,left,base]{\rmfamily\fontsize{10.000000}{12.000000}\selectfont \(\displaystyle 0.2\)}%
\end{pgfscope}%
\begin{pgfscope}%
\pgfsetbuttcap%
\pgfsetroundjoin%
\definecolor{currentfill}{rgb}{0.000000,0.000000,0.000000}%
\pgfsetfillcolor{currentfill}%
\pgfsetlinewidth{0.803000pt}%
\definecolor{currentstroke}{rgb}{0.000000,0.000000,0.000000}%
\pgfsetstrokecolor{currentstroke}%
\pgfsetdash{}{0pt}%
\pgfsys@defobject{currentmarker}{\pgfqpoint{0.000000in}{0.000000in}}{\pgfqpoint{0.048611in}{0.000000in}}{%
\pgfpathmoveto{\pgfqpoint{0.000000in}{0.000000in}}%
\pgfpathlineto{\pgfqpoint{0.048611in}{0.000000in}}%
\pgfusepath{stroke,fill}%
}%
\begin{pgfscope}%
\pgfsys@transformshift{4.668160in}{1.540182in}%
\pgfsys@useobject{currentmarker}{}%
\end{pgfscope}%
\end{pgfscope}%
\begin{pgfscope}%
\pgftext[x=4.765383in,y=1.487421in,left,base]{\rmfamily\fontsize{10.000000}{12.000000}\selectfont \(\displaystyle 0.4\)}%
\end{pgfscope}%
\begin{pgfscope}%
\pgfsetbuttcap%
\pgfsetroundjoin%
\definecolor{currentfill}{rgb}{0.000000,0.000000,0.000000}%
\pgfsetfillcolor{currentfill}%
\pgfsetlinewidth{0.803000pt}%
\definecolor{currentstroke}{rgb}{0.000000,0.000000,0.000000}%
\pgfsetstrokecolor{currentstroke}%
\pgfsetdash{}{0pt}%
\pgfsys@defobject{currentmarker}{\pgfqpoint{0.000000in}{0.000000in}}{\pgfqpoint{0.048611in}{0.000000in}}{%
\pgfpathmoveto{\pgfqpoint{0.000000in}{0.000000in}}%
\pgfpathlineto{\pgfqpoint{0.048611in}{0.000000in}}%
\pgfusepath{stroke,fill}%
}%
\begin{pgfscope}%
\pgfsys@transformshift{4.668160in}{2.154772in}%
\pgfsys@useobject{currentmarker}{}%
\end{pgfscope}%
\end{pgfscope}%
\begin{pgfscope}%
\pgftext[x=4.765383in,y=2.102010in,left,base]{\rmfamily\fontsize{10.000000}{12.000000}\selectfont \(\displaystyle 0.6\)}%
\end{pgfscope}%
\begin{pgfscope}%
\pgfsetbuttcap%
\pgfsetroundjoin%
\definecolor{currentfill}{rgb}{0.000000,0.000000,0.000000}%
\pgfsetfillcolor{currentfill}%
\pgfsetlinewidth{0.803000pt}%
\definecolor{currentstroke}{rgb}{0.000000,0.000000,0.000000}%
\pgfsetstrokecolor{currentstroke}%
\pgfsetdash{}{0pt}%
\pgfsys@defobject{currentmarker}{\pgfqpoint{0.000000in}{0.000000in}}{\pgfqpoint{0.048611in}{0.000000in}}{%
\pgfpathmoveto{\pgfqpoint{0.000000in}{0.000000in}}%
\pgfpathlineto{\pgfqpoint{0.048611in}{0.000000in}}%
\pgfusepath{stroke,fill}%
}%
\begin{pgfscope}%
\pgfsys@transformshift{4.668160in}{2.769361in}%
\pgfsys@useobject{currentmarker}{}%
\end{pgfscope}%
\end{pgfscope}%
\begin{pgfscope}%
\pgftext[x=4.765383in,y=2.716599in,left,base]{\rmfamily\fontsize{10.000000}{12.000000}\selectfont \(\displaystyle 0.8\)}%
\end{pgfscope}%
\begin{pgfscope}%
\pgfsetbuttcap%
\pgfsetroundjoin%
\definecolor{currentfill}{rgb}{0.000000,0.000000,0.000000}%
\pgfsetfillcolor{currentfill}%
\pgfsetlinewidth{0.803000pt}%
\definecolor{currentstroke}{rgb}{0.000000,0.000000,0.000000}%
\pgfsetstrokecolor{currentstroke}%
\pgfsetdash{}{0pt}%
\pgfsys@defobject{currentmarker}{\pgfqpoint{0.000000in}{0.000000in}}{\pgfqpoint{0.048611in}{0.000000in}}{%
\pgfpathmoveto{\pgfqpoint{0.000000in}{0.000000in}}%
\pgfpathlineto{\pgfqpoint{0.048611in}{0.000000in}}%
\pgfusepath{stroke,fill}%
}%
\begin{pgfscope}%
\pgfsys@transformshift{4.668160in}{3.383950in}%
\pgfsys@useobject{currentmarker}{}%
\end{pgfscope}%
\end{pgfscope}%
\begin{pgfscope}%
\pgftext[x=4.765383in,y=3.331189in,left,base]{\rmfamily\fontsize{10.000000}{12.000000}\selectfont \(\displaystyle 1.0\)}%
\end{pgfscope}%
\begin{pgfscope}%
\pgftext[x=4.998408in,y=2.031603in,,top,rotate=90.000000]{\rmfamily\fontsize{10.000000}{12.000000}\selectfont \(\displaystyle \mathrm{\mathit{Bo_e}} \equiv \frac{\epsilon E_0^2 R_0}{\gamma}\)}%
\end{pgfscope}%
\begin{pgfscope}%
\pgfsetbuttcap%
\pgfsetmiterjoin%
\pgfsetlinewidth{0.803000pt}%
\definecolor{currentstroke}{rgb}{0.000000,0.000000,0.000000}%
\pgfsetstrokecolor{currentstroke}%
\pgfsetdash{}{0pt}%
\pgfpathmoveto{\pgfqpoint{4.517160in}{0.521603in}}%
\pgfpathlineto{\pgfqpoint{4.517160in}{0.533400in}}%
\pgfpathlineto{\pgfqpoint{4.517160in}{3.529806in}}%
\pgfpathlineto{\pgfqpoint{4.517160in}{3.541603in}}%
\pgfpathlineto{\pgfqpoint{4.668160in}{3.541603in}}%
\pgfpathlineto{\pgfqpoint{4.668160in}{3.529806in}}%
\pgfpathlineto{\pgfqpoint{4.668160in}{0.533400in}}%
\pgfpathlineto{\pgfqpoint{4.668160in}{0.521603in}}%
\pgfpathclose%
\pgfusepath{stroke}%
\end{pgfscope}%
\end{pgfpicture}%
\makeatother%
\endgroup%

    \caption{.\label{fig:restitution}}
\end{figure}

\newpage
\appendix
\section{\\Droplet Charge} \label{sec.drop_charge}
\subsection*{Parallel Plate Method}
Since, by the earlier scaling, we presuppose the source of the droplet bouncing behavior to be primarily Coulombic in origin (as opposed to dielectrophoretic), the droplet must have some free charge in addition to the charge induced by the electric field. Whether this free charge arises due to contact charge or field induction To measure this charge concurrent methodologies were used. We determined the droplet free charge by observation of the deflection of the droplets in the region of a known uniform field in a fashion inspired by Millikan's famous experiment to determine the fundamental charge of the electron.

Droplets were jumped in free-fall from a superhydrophobic surface placed between the plates of a parallel plate capacitor of known uniform electric field. The surface was charge neutralized. Since the droplet initial velocity $U_0$ is parallel to the electric field, the droplets are inertial in the direction of the electric force, and neglecting the effect of image charges mirrored across the conductors, we can determine the magnitude of the droplet charge by a balance of Coulombic force and inertia given by the equation of motion

\[ y'(t) = q\mathbf{E}. \]

Since the drag is negligible in the inertial limit we can find the charge $q$ by fitting a second-order least squares polynomial to the measured droplet positions, equating the $t^2$ term to the constant acceleration, and dividing by the known, constant magnitude of the electric field.  

A 200-880 VAC source with a full wave bridge rectifier circuit was prototyped on perf-board for initial experiments to measure droplet charge. The circuit was analyzed on an laboratory oscilloscope to verify that the AC component of the signal was appropriately small (13 mV at 35 kHz). Current was determined to be a relatively low 80 $\mu$A. The high-voltage source terminals were led to two parallel polished 150x150 mm aluminum plate electrodes. The electrodes were mounted on an insulated 80/20 extruded aluminum rail for ease of adjustment. All droplet charge experiments were conducted with an electrode spacing of 28.30 mm. With this spacing the calibrated electric field between the plates was $\mathbf{E} \approx 35$kV/m. The electrodes were electrically isolated from the drop rig by two alternating layers of 4 mm thick PMMA sheet and Kapton tape. Potential across the plates was measured periodically with a load-impedance corrected multimeter to account for battery depletion. The typical capacitor rise time of the plates was measured to be 1.4 s, thus to make the most economical use of the brief window a low-gravity a weighted switch was set by hand prior to the drop to close the high-voltage circuit, but which passively safed the system at the resumption of 1-g conditions in the tower. From a survey of literature we suppose the droplet charge, if they are indeed charged by contact with PTFE, to be some function of the droplet volume and the residence time on the superhydrophobic surface. However, sweeping though droplet volumes over a series of drop tower experiments we find little correlation between droplet volume  and free droplet charge.

A brief screening experiment was conducted which alternated the polarity of the field by switching the positive and negative terminal leads between plates. Qualitative observations of droplet electrode preference seem to indicate that the assumption of small polarization stress was well founded. Following this a orthogonal array $3^2$ factorial design experiment with two replicates was conducted to test the effect of varying droplet volume and surface stay time on free charge at the time of jumping. It was hypothesized in accordance with previous studies [ref], that free charge would increase for levels of both factors. ANOVA analysis in \emph{R} of the linear multiple regression model for the data set indicates that neither droplet volume ($p=0.105$), nor surface stay time ($p=0.358$) is significant at the 95\% confidence level. The overall model F-statistics (2.177 in 2 and 13 degrees of freedom), and coefficients of determination ($r^2 = 0.2509$) indicate that the linear model neither fits the data particularly well, nor does it offer an improvement over the mean model. The mean charge was determined to be positive $2.3 \cdot 10^{-11}$ C, with a standard deviation of $1.8 \cdot 10^{-11}$ C.

\printbibliography
\end{document}
